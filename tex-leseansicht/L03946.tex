%% latex-leseansicht-vorspann.tex
%% Vorspann für die Leseansicht.
%% Lädt die gemeinsame Datei latex-vorspann.tex mit nicht gesetztem Schalter.

\newif\ifkorrekturansicht
\korrekturansichtfalse

\input{../tex-inputs/latex-vorspann}


\section[Arthur Schnitzler an Berta Zuckerkandl, 19. 3. 1923]{L03946 Arthur Schnitzler an Berta Zuckerkandl, 19. 3. 1923}
\nopagebreak\mylabel{L03946v}
\rehead{ }\normalsize\beginnumbering\briefempfaengerindex{Zuckerkandl, Berta@\textsc{Zuckerkandl, Berta}!zzzSchnitzler, Arthur@\emph{von Arthur Schnitzler}!1923-03-191@{19. 3. 1923}|(be}
\toendnotes[C]{\smallbreak\pagebreak[2]}
\correspDesc{Versand  durch Arthur Schnitzler am 19. 3. 1923 in Wien
\newline{}Erhalt  durch Berta Zuckerkandl im Zeitraum [20. 3. 1923
                  – 22. 3. 1923?] in Paris}\toendnotes[C]{\smallbreak}
\Standort{DLA, HS.1985.1.2282.}
\physDesc{Brief, Durchschlag, 1 Blatt, 2 Seiten, 2522 Zeichen
\newline{}Schreibmaschine
\newline{}Handschrift: roter Buntstift, lateinische Kurrent (\noindent{}beschriftet: »\uline{Zuckerkandl}«, zwölf Unterstreichungen)}\toendnotes[C]{\smallbreak}
\pstart
           \raggedleft{}{\pb}19. 3. 1923.\pend
           
\pstart{}Liebe und verehrte Frau Hofrätin.\pend\vspace{0.5em}
\pstart
           Um zuerst einmal Ihre Frage betreffs England\oindex{England@\textbf{England}, \emph{Land}|pw} zu
               beantworten: »\label{K_L03946-1v}\edtext{Liebelei\pwindex{Schnitzler, Arthur 15. 5. 1862 Wien – 21. 10. 1931 ebd.@\textsc{Schnitzler, Arthur} (15. 5. 1862 Wien – 21. 10. 1931 ebd.), \emph{Schriftsteller, Mediziner}!Liebelei. Schauspiel in drei Akten@\strich\emph{Liebelei. Schauspiel in drei Akten}|pw}« ist dort im Jahre 1907}{\lemma{\textnormal{\emph{Liebelei« … 1907}}}\Cendnote{\textnormal{Die Theaterpremiere von \emph{Light o’ Love}\pwindex{Schnitzler, Arthur 15. 5. 1862 Wien – 21. 10. 1931 ebd.@\textsc{Schnitzler, Arthur} (15. 5. 1862 Wien – 21. 10. 1931 ebd.), \emph{Schriftsteller, Mediziner}!Light o’ Love@\strich\emph{Light o’ Love}|pwk}\eventindex{Royal Court Theatre@\textbf{Royal Court Theatre}!Premiere von Light o’ Love, 14.5.1909@Premiere von Light o’ Love, 14.5.1909|pwk} fand nicht 1907, sondern am 14. 5. 1909 am \emph{Royal Court Theatre}\orgindex{Royal Court Theatre@Royal Court Theatre|pwk}
                  statt. }}}\label{K_L03946-1}{ }aufgeführt\eventindex{Royal Court Theatre@\textbf{Royal Court Theatre}!Premiere von Light o’ Love, 14.5.1909@Premiere von Light o’ Love, 14.5.1909|pwv} worden, der
                  »\label{K_L03946-2v}\edtext{Grüne Kakadu\pwindex{Schnitzler, Arthur 15. 5. 1862 Wien – 21. 10. 1931 ebd.@\textsc{Schnitzler, Arthur} (15. 5. 1862 Wien – 21. 10. 1931 ebd.), \emph{Schriftsteller, Mediziner}!grüne Kakadu. Groteske in einem Akt@\strich\emph{Der grüne Kakadu. Groteske in einem Akt}|pw}« 1909}{\lemma{\textnormal{\emph{Grüne Kakadu« 1909}}}\Cendnote{\textnormal{Zwar lässt sich eine Ankündigung
                  nachweisen, dass Cyril Maude\pwindex{Maude, Cyril 24.\,4.\,1862 London – 20.\,2.\,1951 Torquay@\textsc{Maude, Cyril} (24.\,4.\,1862 London – 20.\,2.\,1951 Torquay), \emph{Schauspieler, Theatermanager}|pwk} das Stück am
                     \emph{Haymarket Theatre}\orgindex{Theatre Royal Haymarket@Theatre Royal Haymarket|pwk} inszenieren wollte (\emph{The Daily Express}\textcolor{red}{\textsuperscript{XXXX indx2}}, Nr. 2990,
                        10. 11. 1909, S. 7.), aber keine tatsächliche Aufführung
                  vor 1913. }}}\label{K_L03946-2}, das »\label{K_L03946-3v}\edtext{Märchen\pwindex{Schnitzler, Arthur 15. 5. 1862 Wien – 21. 10. 1931 ebd.@\textsc{Schnitzler, Arthur} (15. 5. 1862 Wien – 21. 10. 1931 ebd.), \emph{Schriftsteller, Mediziner}!Märchen. English version@\strich\emph{Das Märchen. English version}|pw}\pwindex{Schnitzler, Arthur 15. 5. 1862 Wien – 21. 10. 1931 ebd.@\textsc{Schnitzler, Arthur} (15. 5. 1862 Wien – 21. 10. 1931 ebd.), \emph{Schriftsteller, Mediziner}!Märchen. Schauspiel in drei Aufzügen@\strich\emph{Das Märchen. Schauspiel in drei Aufzügen}|pw}« 1911}{\lemma{\textnormal{\emph{Märchen« 1911}}}\Cendnote{\textnormal{ Nur eine einmalige Vorführung von \emph{Das Märchen.
                        English version}\pwindex{Schnitzler, Arthur 15. 5. 1862 Wien – 21. 10. 1931 ebd.@\textsc{Schnitzler, Arthur} (15. 5. 1862 Wien – 21. 10. 1931 ebd.), \emph{Schriftsteller, Mediziner}!Märchen. English version@\strich\emph{Das Märchen. English version}|pwk}\eventindex{Premiere von Das Märchen. English version, 28.1.1912@Premiere von Das Märchen. English version, 28.1.1912|pwk} (Übersetzung von Harley
                     Granville-Barker\pwindex{Granville-Barker, Harley 25.\,11.\,1877 London – 31.\,8.\,1946 Paris@\textsc{Granville-Barker, Harley} (25.\,11.\,1877 London – 31.\,8.\,1946 Paris), \emph{Theaterleiter, Schauspieler, Übersetzer}|pwk} und Charles Edwin
                     Wheeler\pwindex{Wheeler, Charles Edwin 1868 Adelaide – 2.\,2.\,1947 London@\textsc{Wheeler, Charles Edwin} (1868 Adelaide – 2.\,2.\,1947 London), \emph{Mediziner, Theaterleiter}|pwk}) ist bislang belegt, diese fand am 28. 1. 1912 durch die \emph{Adelphi Play
                     Society}\orgindex{Adelphi Play Society@Adelphi Play Society|pwk} statt. }}}\label{K_L03946-3}, \label{K_L03946-4v}\edtext{der
                  »Anatol\pwindex{Schnitzler, Arthur 15. 5. 1862 Wien – 21. 10. 1931 ebd.@\textsc{Schnitzler, Arthur} (15. 5. 1862 Wien – 21. 10. 1931 ebd.), \emph{Schriftsteller, Mediziner}!Anatol@\strich\emph{Anatol}|pw}« 1911}{\lemma{\textnormal{\emph{der
                  »Anatol« 1911}}}\Cendnote{\textnormal{Mindestens fünf Einakter des \emph{Anatol-Zyklus\pwindex{Schnitzler, Arthur 15. 5. 1862 Wien – 21. 10. 1931 ebd.@\textsc{Schnitzler, Arthur} (15. 5. 1862 Wien – 21. 10. 1931 ebd.), \emph{Schriftsteller, Mediziner}!Anatol. A Sequence of Dialogues by Arthur Schnitzler. Paraphrased for the English Stage@\strich\emph{Anatol. A Sequence of Dialogues by Arthur Schnitzler. Paraphrased for the English Stage}|pwkv}}\pwindex{Schnitzler, Arthur 15. 5. 1862 Wien – 21. 10. 1931 ebd.@\textsc{Schnitzler, Arthur} (15. 5. 1862 Wien – 21. 10. 1931 ebd.), \emph{Schriftsteller, Mediziner}!Anatol@\strich\emph{Anatol}|pwk} wurden am 11. 3. 1911 im Little Theatre in
                     the Adelphi\oindex{Little Theatre in the Adelphi@\textbf{Little Theatre in the Adelphi}, \emph{Theater}|pwk} in London\oindex{London@\textbf{London}, \emph{Hauptstadt}|pwk}{ }erstaufgeführt\eventindex{Little Theatre in the Adelphi@\textbf{Little Theatre in the Adelphi}!Premiere von Anatol. A Sequence of Dialogues by Arthur Schnitzler@Premiere von Anatol. A Sequence of Dialogues by Arthur Schnitzler|pwkv}.
                     (Nicole Robertson: \emph{Arthur Schnitzler in Great Britain.
                        An Examniation of Power an Translation}, London:
                        \emph{Institute of Modern Languages Research}{ }2022, S. 111.)}}}\label{K_L03946-4}. Ja sogar eine \label{K_L03946-5v}\edtext{Anatol-Operette\pwindex{Schnitzler, Arthur 15. 5. 1862 Wien – 21. 10. 1931 ebd.@\textsc{Schnitzler, Arthur} (15. 5. 1862 Wien – 21. 10. 1931 ebd.), \emph{Schriftsteller, Mediziner}!Anatol. A Sequence of Dialogues by Arthur Schnitzler. Paraphrased for the English Stage@\strich\emph{Anatol. A Sequence of Dialogues by Arthur Schnitzler. Paraphrased for the English Stage}|pwv}}{\lemma{\textnormal{\emph{Anatol-Operette}}}\Cendnote{\textnormal{Harley Granville-Barker\pwindex{Granville-Barker, Harley 25.\,11.\,1877 London – 31.\,8.\,1946 Paris@\textsc{Granville-Barker, Harley} (25.\,11.\,1877 London – 31.\,8.\,1946 Paris), \emph{Theaterleiter, Schauspieler, Übersetzer}|pwk} fragte
                     1912 bei Schnitzler an, ob
                  er seine (stark bearbeitete) Übersetzung von \emph{Anatol}\pwindex{Schnitzler, Arthur 15. 5. 1862 Wien – 21. 10. 1931 ebd.@\textsc{Schnitzler, Arthur} (15. 5. 1862 Wien – 21. 10. 1931 ebd.), \emph{Schriftsteller, Mediziner}!Anatol@\strich\emph{Anatol}|pwk} für eine Operette verwenden könne. Schnitzler empfand das als Übergriff. Ob es trotzdem zu
                  einer Aufführung kam, ist ungewiss. (Nicole Robertson: \emph{Arthur Schnitzler in Great Britain. An Examination of Power and
                        Translation}. London: \emph{Institute of
                        Modern Languages Research}{ }2022, S. 112.) }}}\label{K_L03946-5}, über die mir nichts Näheres bekannt
               ist im Jahre 1912. Der »Anatol\pwindex{Schnitzler, Arthur 15. 5. 1862 Wien – 21. 10. 1931 ebd.@\textsc{Schnitzler, Arthur} (15. 5. 1862 Wien – 21. 10. 1931 ebd.), \emph{Schriftsteller, Mediziner}!Anatol@\strich\emph{Anatol}|pw}\pwindex{Schnitzler, Arthur 15. 5. 1862 Wien – 21. 10. 1931 ebd.@\textsc{Schnitzler, Arthur} (15. 5. 1862 Wien – 21. 10. 1931 ebd.), \emph{Schriftsteller, Mediziner}!Anatol. A Sequence of Dialogues by Arthur Schnitzler. Paraphrased for the English Stage@\strich\emph{Anatol. A Sequence of Dialogues by Arthur Schnitzler. Paraphrased for the English Stage}|pw}«-Zyklus ist in festen Händen, »Liebelei\pwindex{Schnitzler, Arthur 15. 5. 1862 Wien – 21. 10. 1931 ebd.@\textsc{Schnitzler, Arthur} (15. 5. 1862 Wien – 21. 10. 1931 ebd.), \emph{Schriftsteller, Mediziner}!Liebelei. Schauspiel in drei Akten@\strich\emph{Liebelei. Schauspiel in drei Akten}|pw}«, »Märchen\pwindex{Schnitzler, Arthur 15. 5. 1862 Wien – 21. 10. 1931 ebd.@\textsc{Schnitzler, Arthur} (15. 5. 1862 Wien – 21. 10. 1931 ebd.), \emph{Schriftsteller, Mediziner}!Märchen. Schauspiel in drei Aufzügen@\strich\emph{Das Märchen. Schauspiel in drei Aufzügen}|pw}« und »Der grüne Kakadu\pwindex{Schnitzler, Arthur 15. 5. 1862 Wien – 21. 10. 1931 ebd.@\textsc{Schnitzler, Arthur} (15. 5. 1862 Wien – 21. 10. 1931 ebd.), \emph{Schriftsteller, Mediziner}!grüne Kakadu. Groteske in einem Akt@\strich\emph{Der grüne Kakadu. Groteske in einem Akt}|pw}« sind aber meiner Auffassung
               nach wieder ganz zu meiner freien Verfügung, da ja die Autorisationen abgelaufen und
               jene alten Uebersetzungen – ich habe keine Ahnung, wer sie gemacht hat – längst
               nirgendmehr gespielt werden. Immerhin müsste man die Herren Brown {\kaufmannsund} Co.\orgindex{Little, Brown and Company@Little, Brown and Company|pw} auf die Tatsachen
               aufmerksam machen. Vor einigen Jahren wurde durch Fischer\pwindex{Fischer, Samuel 24.\,12.\,1859 Liptovský Mikuláš – 15.\,10.\,1934 Berlin@\textsc{Fischer, Samuel} (24.\,12.\,1859 Liptovský Mikuláš – 15.\,10.\,1934 Berlin), \emph{Verleger}|pw} über »Bernhardi\pwindex{Schnitzler, Arthur 15. 5. 1862 Wien – 21. 10. 1931 ebd.@\textsc{Schnitzler, Arthur} (15. 5. 1862 Wien – 21. 10. 1931 ebd.), \emph{Schriftsteller, Mediziner}!Professor Bernhardi. Komödie in fünf Akten@\strich\emph{Professor Bernhardi. Komödie in fünf Akten}|pw}« und auch über
                  »Liebelei\pwindex{Schnitzler, Arthur 15. 5. 1862 Wien – 21. 10. 1931 ebd.@\textsc{Schnitzler, Arthur} (15. 5. 1862 Wien – 21. 10. 1931 ebd.), \emph{Schriftsteller, Mediziner}!Liebelei. Schauspiel in drei Akten@\strich\emph{Liebelei. Schauspiel in drei Akten}|pw}« vor einigen Monaten auch von Karczag\pwindex{Karczag, Wilhelm 28.\,8.\,1857 Karcag – 11.\,10.\,1923 Baden bei Wien@\textsc{Karczag, Wilhelm} (28.\,8.\,1857 Karcag – 11.\,10.\,1923 Baden bei Wien), \emph{Schriftsteller, Theaterleiter, Musikverleger}|pw} über »Liebelei\pwindex{Schnitzler, Arthur 15. 5. 1862 Wien – 21. 10. 1931 ebd.@\textsc{Schnitzler, Arthur} (15. 5. 1862 Wien – 21. 10. 1931 ebd.), \emph{Schriftsteller, Mediziner}!Liebelei. Schauspiel in drei Akten@\strich\emph{Liebelei. Schauspiel in drei Akten}|pw}« verhandelt, ohne dass Resultate erzielt wurden. Unter diesen
               Umständen würde also ein Verleger in England\oindex{England@\textbf{England}, \emph{Land}|pw} ein
               reiches Feld finden. Was die Bedingungen anbelangt, so werden hier wohl Besnard\pwindex{Besnard, Lucien 19.\,1.\,1872 Nonancourt – 1955 Paris@\textsc{Besnard, Lucien} (19.\,1.\,1872 Nonancourt – 1955 Paris), \emph{Schriftsteller}|pw} und Geraldy\pwindex{Géraldy, Paul 6.\,3.\,1885 Paris – 9.\,3.\,1983 Neuilly-sur-Seine@\textsc{Géraldy, Paul} (6.\,3.\,1885 Paris – 9.\,3.\,1983 Neuilly-sur-Seine), \emph{Schriftsteller}|pw} am besten raten können. Wegen »Zwischenspiel\pwindex{Schnitzler, Arthur 15. 5. 1862 Wien – 21. 10. 1931 ebd.@\textsc{Schnitzler, Arthur} (15. 5. 1862 Wien – 21. 10. 1931 ebd.), \emph{Schriftsteller, Mediziner}!Zwischenspiel. Komödie in drei Akten@\strich\emph{Zwischenspiel. Komödie in drei Akten}|pw}« \label{K_L03946-6v}\edtext{bewirbt sich
               eben Spachner\pwindex{Spachner, Leopold @\textsc{Spachner, Leopold}, \emph{Theateragent}|pw}}{\lemma{\textnormal{\emph{bewirbt … Spachner}}}\Cendnote{\textnormal{Vom Beginn der Verhandlungen berichtete
                     Schnitzler an Leo Greiner\pwindex{Greiner, Leo 1.\,4.\,1876 Brünn – 22.\,8.\,1928 Berlin@\textsc{Greiner, Leo} (1.\,4.\,1876 Brünn – 22.\,8.\,1928 Berlin), \emph{Schriftsteller, Verlagslektor}|pwk} vom \emph{Fischerverlag}\orgindex{S. Fischer Verlag@S. Fischer Verlag|pwk} am 6. 10. 1922, vgl. Arthur Schnitzler: \emph{Mikrofilme}, \url{https://schnitzler\_mikrofilme.acdh.oeaw.ac.at/1416739\_0245}. Die später
                  konfliktreiche Korrespondenz dokumentieren die überlieferten Korrespondenzstücke
                     Schnitzlers an Spachner\pwindex{Spachner, Leopold @\textsc{Spachner, Leopold}, \emph{Theateragent}|pwk} aus dem Zeitraum
                     1922–1929, vgl. \emph{DLA Marbach}, HS.1985.1.1959,1-14.}}}\label{K_L03946-6}, \label{T_L03946-1v}\edtext{Frau Kalisch\pwindex{Kalich, Bertha 17.\,5.\,1874 Lviv – 18.\,4.\,1939 New York City@\textsc{Kalich, Bertha} (17.\,5.\,1874 Lviv – 18.\,4.\,1939 New York City), \emph{Schauspielerin}|pw}}{\lemma{\textnormal{\emph{Frau Kalisch}}}\Cendnote{\textnormal{Der Satz ist grammatikalisch nicht
                  aufzulösen. Leopold Spachner\pwindex{Spachner, Leopold @\textsc{Spachner, Leopold}, \emph{Theateragent}|pwk} und Bertha Kalich\pwindex{Spachner, Leopold @\textsc{Spachner, Leopold}, \emph{Theateragent}|pwk} waren verheiratet und bewarben
                  sich gemeinsam um die Rechte.}}}\label{T_L03946-1}, die vorläufig nur die Rechte für Amerika\oindex{Vereinigte Staaten von Amerika [USA]@\textbf{Vereinigte Staaten von Amerika [USA]}|pw} erworben hat. Ich würde das »Weite Land\pwindex{Schnitzler, Arthur 15. 5. 1862 Wien – 21. 10. 1931 ebd.@\textsc{Schnitzler, Arthur} (15. 5. 1862 Wien – 21. 10. 1931 ebd.), \emph{Schriftsteller, Mediziner}!weite Land. Tragikomödie in fünf Akten@\strich\emph{Das weite Land. Tragikomödie in fünf Akten}|pw}«, »Komödie der Worte\pwindex{Schnitzler, Arthur 15. 5. 1862 Wien – 21. 10. 1931 ebd.@\textsc{Schnitzler, Arthur} (15. 5. 1862 Wien – 21. 10. 1931 ebd.), \emph{Schriftsteller, Mediziner}!Komödie der Worte. Drei Einakter@\strich\emph{Komödie der Worte. Drei Einakter}|pw}«, »Bernhardi\pwindex{Schnitzler, Arthur 15. 5. 1862 Wien – 21. 10. 1931 ebd.@\textsc{Schnitzler, Arthur} (15. 5. 1862 Wien – 21. 10. 1931 ebd.), \emph{Schriftsteller, Mediziner}!Professor Bernhardi. Komödie in fünf Akten@\strich\emph{Professor Bernhardi. Komödie in fünf Akten}|pw}«, »Lebendige Stunden\pwindex{Schnitzler, Arthur 15. 5. 1862 Wien – 21. 10. 1931 ebd.@\textsc{Schnitzler, Arthur} (15. 5. 1862 Wien – 21. 10. 1931 ebd.), \emph{Schriftsteller, Mediziner}!Lebendige Stunden. Vier Einakter@\strich\emph{Lebendige Stunden. Vier Einakter}|pw}«, »Liebelei\pwindex{Schnitzler, Arthur 15. 5. 1862 Wien – 21. 10. 1931 ebd.@\textsc{Schnitzler, Arthur} (15. 5. 1862 Wien – 21. 10. 1931 ebd.), \emph{Schriftsteller, Mediziner}!Liebelei. Schauspiel in drei Akten@\strich\emph{Liebelei. Schauspiel in drei Akten}|pw}« zur Erwägung geben.\pend
           
\pstart
           Für Ihre bisherigen Bemühungen, verehrteste Frau Hofrätin, mit Mme. Cabir\pwindex{Cabire, Emma @\textsc{Cabire, Emma}, \emph{Übersetzerin, Redakteurin, Literaturagentin}|pw} und M. Hella\pwindex{Hella, Alzir 30.\,12.\,1881 Vieux Condé – 14.\,7.\,1953 Paris@\textsc{Hella, Alzir} (30.\,12.\,1881 Vieux Condé – 14.\,7.\,1953 Paris), \emph{Übersetzer}|pw}
               sage ich Ihnen vielen Dank. Ich freue mich, dass Sie bald wiederkommen werden. Sie
               haben hier allerlei mehr oder weniger interessante Thaterereignisse versäumt, über
               die Sie wohl schon Bericht haben werden. »Der Unbestechliche\pwindex{Hofmannsthal, Hugo von 1.\,2.\,1874 Wien – 15.\,7.\,1929 Rodaun@\textsc{Hofmannsthal, Hugo von} (1.\,2.\,1874 Wien – 15.\,7.\,1929 Rodaun), \emph{Schriftsteller}!Unbestechliche. Lustspiel in fünf Akten@\strich\emph{Der Unbestechliche. Lustspiel in fünf Akten}|pw}\eventindex{Raimund-Theater@\textbf{Raimund-Theater}!Premiere von Der Unbestechliche, 16.3.1923@Premiere von Der Unbestechliche, 16.3.1923|pwv}« war ein grosser Erfolg für Pallenberg\pwindex{Pallenberg, Max 18.\,12.\,1877 Wien – 26.\,6.\,1934 Karlsbad@\textsc{Pallenberg, Max} (18.\,12.\,1877 Wien – 26.\,6.\,1934 Karlsbad), \emph{Schauspieler}|pw}
               und ein ganz guter für Hugo\pwindex{Hofmannsthal, Hugo von 1.\,2.\,1874 Wien – 15.\,7.\,1929 Rodaun@\textsc{Hofmannsthal, Hugo von} (1.\,2.\,1874 Wien – 15.\,7.\,1929 Rodaun), \emph{Schriftsteller}|pw}. In meinen
               Angelegenheiten gibt {\pb}es nicht viel Neues, an die
               geschäftlichen Aergerlichkeiten gewöhnt man sich, wie an chronische Uebel. Ein paar
               neue, nicht uninteressante \label{K_L03946-7v}\edtext{Amerikaner\pwindex{Hofmann @\textsc{Hofmann}, \emph{Mediziner}|pwv}\pwindex{Viereck, George Sylvester 31.\,12.\,1884 München – 18.\,3.\,1962 Holyoke@\textsc{Viereck, George Sylvester} (31.\,12.\,1884 München – 18.\,3.\,1962 Holyoke), \emph{Schriftsteller, Journalist}|pwv} habe ich
               kennen gelernt}{\lemma{\textnormal{\emph{Amerikaner … gelernt}}}\Cendnote{\textnormal{Vgl. A. S.: \emph{Tagebuch}, 12. 3. 1923 und 14. 3. 1923.}}}\label{K_L03946-7},
               von denen ich Ihnen mündlich mehr erzähle, ein paar Mal war ich bei Alma\pwindex{Mahler-Werfel, Alma Maria 31.\,8.\,1879 Wien – 11.\,12.\,1964 New York City@\textsc{Mahler-Werfel, Alma Maria} (31.\,8.\,1879 Wien – 11.\,12.\,1964 New York City)|pw}, so \label{K_L03946-8v}\edtext{gestern Mittag}{\lemma{\textnormal{\emph{gestern Mittag}}}\Cendnote{\textnormal{Siehe A. S.: \emph{Tagebuch}, 18. 3. 1923.}}}\label{K_L03946-8} mit
                  Hofmannsthal\pwindex{Hofmannsthal, Hugo von 1.\,2.\,1874 Wien – 15.\,7.\,1929 Rodaun@\textsc{Hofmannsthal, Hugo von} (1.\,2.\,1874 Wien – 15.\,7.\,1929 Rodaun), \emph{Schriftsteller}|pw}, Saltens\pwindex{Salten, Felix 6.\,9.\,1869 Budapest – 8.\,10.\,1945 Zürich@\textsc{Salten, Felix} (6.\,9.\,1869 Budapest – 8.\,10.\,1945 Zürich), \emph{Schriftsteller, Journalist, Chefredakteur}|pw}\pwindex{Salten, Ottilie 7.\,3.\,1868 Prag – 22.\,6.\,1942 Zürich@\textsc{Salten, Ottilie} (7.\,3.\,1868 Prag – 22.\,6.\,1942 Zürich), \emph{Schauspielerin}|pw}, Molls\pwindex{Moll, Carl 23.\,4.\,1861 Wien – 13.\,4.\,1945 ebd.@\textsc{Moll, Carl} (23.\,4.\,1861 Wien – 13.\,4.\,1945 ebd.), \emph{Maler}|pw}\pwindex{Moll, Anna Sofie 20.\,11.\,1857 Hamburg – 29.\,11.\,1938 Wien@\textsc{Moll, Anna Sofie} (20.\,11.\,1857 Hamburg – 29.\,11.\,1938 Wien)|pw}, Pallenberg\pwindex{Pallenberg, Max 18.\,12.\,1877 Wien – 26.\,6.\,1934 Karlsbad@\textsc{Pallenberg, Max} (18.\,12.\,1877 Wien – 26.\,6.\,1934 Karlsbad), \emph{Schauspieler}|pw} mit der Massari\pwindex{Massary, Fritzi 1.\,3.\,1882 Wien – 30.\,1.\,1969 Beverly Hills@\textsc{Massary, Fritzi} (1.\,3.\,1882 Wien – 30.\,1.\,1969 Beverly Hills), \emph{Schauspielerin, Sängerin, Musikerin}|pw}, die beide ich erst bei dieser
               Gelegenheit kennen lernte. Es war sehr anregend. Nach Tisch aber kamen so viele
               Leute, dass das Zimmer überquoll. Ich war unter den Ueberquellenden.\pend
           
\pstart
           \label{K_L03946-9v}\edtext{Uebermorgen kommt Olga\pwindex{Schnitzler, Olga 17.\,1.\,1882 Wien – 13.\,1.\,1970 Lugano@\textsc{Schnitzler, Olga} (17.\,1.\,1882 Wien – 13.\,1.\,1970 Lugano), \emph{Schauspielerin, Sängerin}|pw}}{\lemma{\textnormal{\emph{Uebermorgen kommt Olga}}}\Cendnote{\textnormal{Vgl. A. S.: \emph{Tagebuch}, 21. 3. 1923.}}}\label{K_L03946-9} und
               wird bei Alma\pwindex{Mahler-Werfel, Alma Maria 31.\,8.\,1879 Wien – 11.\,12.\,1964 New York City@\textsc{Mahler-Werfel, Alma Maria} (31.\,8.\,1879 Wien – 11.\,12.\,1964 New York City)|pw} wohnen, wie Sie ja wissen. Ich
               denke wohl, dass sie über Ostern bleiben wird. – Meine \label{K_L03946-10v}\edtext{Nordlandsreise\oindex{Skandinavien@\textbf{Skandinavien}|pw}}{\lemma{\textnormal{\emph{Nordlandsreise}}}\Cendnote{\textnormal{Schnitzler brach am 7. 5. 1923 zu einer
                  Reise auf, die ihn nach Kopenhagen\oindex{Kopenhagen@\textbf{Kopenhagen}, \emph{Hauptstadt}|pwk} und Stockholm\oindex{Stockholm@\textbf{Stockholm}, \emph{Hauptstadt}|pwk} führte. Am 27. 5. 1923 kehrte er
                  nach Wien\oindex{Wien@\textbf{Wien}, \emph{Verwaltungsgebiet}|pwk} zurück.}}}\label{K_L03946-10} ist noch unsicher,
               vielleicht begnüge ich mich mit Dänemark\oindex{Dänemark@\textbf{Dänemark}|pw}
               allein.\pend
           
\pstart
           Darf ich Sie bitten mich Ihrer Frau Schwester\pwindex{Clemenceau, Sophie 25.\,5.\,1862 – 24.\,9.\,1937@\textsc{Clemenceau, Sophie} (25.\,5.\,1862 – 24.\,9.\,1937)|pwv} bestens zu empfehlen. Sie aber, liebe und
               verehrte Freundin, seien aufs Herzlichste gegrüsst von{\\[\baselineskip]} Ihrem dankbar
               ergebenen\pend
           \leftskip=0em{}\selectlanguage{ngerman}\endnumbering\briefempfaengerindex{Zuckerkandl, Berta@\textsc{Zuckerkandl, Berta}!zzzSchnitzler, Arthur@\emph{von Arthur Schnitzler}!1923-03-191@{19. 3. 1923}|)be}\mylabel{L03946h}
\begin{anhang}
\end{anhang}\newcommand{\dateiname}{L03946}\newcommand{\titel}{Arthur Schnitzler an Berta Zuckerkandl, 19. 3. 1923}\newcommand{\editorInnen}{Herausgegeben von Jahnke, SelmaMüller, Martin Anton}%% latex-leseansicht-abspann.tex
%% Abspann für die Leseansicht.
%% Der Schalter \ifkorrekturansicht ist bereits durch den Vorspann gesetzt.

%% latex-abspann.tex
%% Gemeinsamer Abspann für Korrekturansicht und Leseansicht.
%% Setzt den Schalter \ifkorrekturansicht voraus (gesetzt in den
%% einbindenden Dateien latex-korrekturansicht-abspann.tex bzw.
%% latex-leseansicht-abspann.tex).
%% ---------------------------------------------------------------

\normalsize

% Das esempio-Environment wird nur in der Leseansicht benötigt
\ifkorrekturansicht\else
\newenvironment{esempio}[3]%
{
    \vspace{1.5ex}
    \rlap{\underline{#1}}
    \par
    \setlength{\parindent}{0cm}
    \nopagebreak
    \leftskip=#2cm
    \rightskip=#3cm
}
{
    \par
}
\fi

\doendnotes{C}
\bigskip
\vfill

\clearpage

\footnotesize

\ifkorrekturansicht
  \lohead{\textsc{register}}
\fi

% theindex-Environment neu definieren ohne reledmac
\makeatletter
\renewenvironment{theindex}{%
  \ifkorrekturansicht
    \section*{\indexname}%
  \else
    \subsubsection*{Index der erwähnten Entitäten}%
  \fi
  \setlength{\parindent}{0pt}%
  \setlength{\parskip}{0pt plus 0.3pt}%
  \let\item\@idxitem
}{%
  \ifkorrekturansicht\clearpage\fi
}
\makeatother

\IfFileExists{\jobname-pw.ind}{\input{\jobname-pw.ind}}{}

% Quellenangabe nur in der Leseansicht
\ifkorrekturansicht\else
% Fallback-Definitionen, falls die .tex-Datei \titel etc. nicht gesetzt hat
\providecommand{\titel}{}
\providecommand{\editorInnen}{}
\providecommand{\dateiname}{\jobname}

\vspace{3cm}

\vfill

\footnotesize
\textsc{Quelle}: \titel. Herausgegeben von {\editorInnen}. In: \emph{Arthur Schnitzler: Briefwechsel mit Autorinnen und Autoren}.
 Digitale Edition, https://schnitzler-briefe.acdh.oeaw.ac.at/{\dateiname}.html (Stand \today)
\fi

\end{document}


