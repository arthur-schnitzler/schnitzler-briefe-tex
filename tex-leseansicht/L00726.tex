%% latex-korrekturansicht-vorspann.tex
%% Vorspann für die Korrekturansicht.
%% Lädt die gemeinsame Datei latex-vorspann.tex mit gesetztem Schalter.

\newif\ifkorrekturansicht
\korrekturansichttrue

\input{../tex-inputs/latex-vorspann}


\section[Arthur Schnitzler an Richard Beer-Hofmann, 2. 10. 1897]{L00726 Arthur Schnitzler an Richard Beer-Hofmann, 2. 10. 1897}
\nopagebreak\mylabel{L00726v}
\rehead{ }\normalsize\beginnumbering\briefempfaengerindex{Beer-Hofmann, Richard@\textsc{Beer-Hofmann, Richard}!zzzSchnitzler, Arthur@\emph{von Arthur Schnitzler}!1897-10-021@{2. 10. 1897}|(be}
\toendnotes[C]{\smallbreak\pagebreak[2]}\Standort{YCGL, MSS 31.}
\physDesc{Briefkarte, , Umschlag, 341 Zeichen
\newline{}Handschrift: Bleistift, deutsche Kurrent
\newline{}Versand: 1) Stempel: »\nobreak{}\oindex{IX., Alsergrund@\textbf{IX., Alsergrund}, \emph{A.ADM3}|pwk}Wien 9/3, 2. 10. {[}9{]}7, 7–\textcolor{gray}{8} N\nobreak{}«.   2) Stempel: »\nobreak{}\oindex{I., Innere Stadt@\textbf{I., Innere Stadt}, \emph{A.ADM3}|pwk}{\pb}Wien 1/1, \textcolor{gray}{3.} 10. 97, 10 ½ V., Bestellt\nobreak{}«. }\toendnotes[C]{\smallbreak}\pstart{}{\pb}Herrn \textsc{Dr. Richard
                     Beer-Hofmann}\pend{}\pstart{}Wien\oindex{Wien@\textbf{Wien}, \emph{A.ADM2}|pw}\pend{}\pstart{}\textsc{I. Wollzeile 15}\oindex{Wollzeile@\textbf{Wollzeile}, \emph{Straße (K.STR)}|pw}.\pend{}{\bigskip}\vspace{1em}
\pstart
           \noindent{}{\pb}Lieber Richard, Ich vergaſs, daſs in
               jenem \label{K_L00726-1v}\edtext{Brief}{\lemma{\textnormal{\emph{Brief}}}\Cendnote{\textnormal{nicht überliefert}}}\label{K_L00726-1} von Andrian\pwindex{Andrian-Werburg, Leopold von 09.05.1875 – 19.11.1951@\textsc{Andrian-Werburg, Leopold von} (09.05.1875 – 19.11.1951), \emph{Schriftsteller/Schriftstellerin, Diplomat/Diplomatin}|pw} auch ſteht, Sie mögen ihn \uline{jedenfalls}{ }\textsc{en route}\pwindex{En route@\emph{En route}|pw} von \textsc{Huysmans}\pwindex{Huysmans, Joris-Karl 05.02.1848 – 12.05.1907@\textsc{Huysmans, Joris-Karl} (05.02.1848 – 12.05.1907), \emph{Schriftsteller/Schriftstellerin}|pw} u. etwas über den Milton\pwindex{Milton, John 9.12.1608 – 8.11.1674@\textsc{Milton, John} (9.12.1608 – 8.11.1674), \emph{Philosoph/Philosophin, Dichter/Dichterin, Beamter/Beamte}|pw} (? unleſerlich)
               von \textsc{Stendhal}\pwindex{Stendhal 1783-01-23 – 1842-03-23@\textsc{Stendhal} (1783-01-23 – 1842-03-23), \emph{Schriftsteller/Schriftstellerin}|pw}{ }ſchicken.\pend
           
\pstart
           Seine Adreſſe iſt \textsc{Baden Baden}\oindex{Baden-Baden@\textbf{Baden-Baden}, \emph{P.PPL}|pw}, {\pb}\textsc{Sanatorium Frey}\oindex{Sanatorium Frey-Dengler@\textbf{Sanatorium Frey-Dengler}, \emph{Sanatorium (K.SAN)}|pw}. –\pend
           
\pstart
           Ich gehe vielleicht morgen (So{\geminationn}tag)
                  Abend ins \label{K_L00726-2v}\edtext{Carltheater\oindex{Carl-Theater@\textbf{Carl-Theater}, \emph{Theater (K.THE)}|pw}}{\lemma{\textnormal{\emph{Carltheater}}}\Cendnote{\textnormal{Schnitzler besuchte die Aufführung von \emph{Der Stellvertreter}\pwindex{Stellvertreter. Schwank in 3 Acten@\emph{Der Stellvertreter. Schwank in 3 Acten}|pwk} von William Busnach\pwindex{Busnach, William 07.03.1832 – 21.01.1907@\textsc{Busnach, William} (07.03.1832 – 21.01.1907), \emph{Schriftsteller/Schriftstellerin}|pwk} und Georges
                     Duval\pwindex{Duval, Georges 02.02.1847 – 24.09.1919@\textsc{Duval, Georges} (02.02.1847 – 24.09.1919), \emph{Schriftsteller/Schriftstellerin}|pwk}.}}}\label{K_L00726-2}.\pend
           \pstart Herzlich Ihr\spacefill\mbox{Arthur.}\pend{}\selectlanguage{ngerman}\endnumbering\briefempfaengerindex{Beer-Hofmann, Richard@\textsc{Beer-Hofmann, Richard}!zzzSchnitzler, Arthur@\emph{von Arthur Schnitzler}!1897-10-021@{2. 10. 1897}|)be}\mylabel{L00726h}  \normalsize

\doendnotes{C}
\bigskip
\vfill

\clearpage

\footnotesize

\lohead{\textsc{register}}

% Definiere theindex-Environment komplett neu ohne reledmac
\makeatletter
\renewenvironment{theindex}{%
  \section*{\indexname}%
  \setlength{\parindent}{0pt}%
  \setlength{\parskip}{0pt plus 0.3pt}%
  \let\item\@idxitem
}{%
  \clearpage
}
\makeatother

\IfFileExists{\jobname-pw.ind}{\input{\jobname-pw.ind}}{}

\end{document}

      