%% latex-leseansicht-vorspann.tex
%% Vorspann für die Leseansicht.
%% Lädt die gemeinsame Datei latex-vorspann.tex mit nicht gesetztem Schalter.

\newif\ifkorrekturansicht
\korrekturansichtfalse

\input{../tex-inputs/latex-vorspann}


         
         \renewcommand{\erwaehntePersonen}{Personen: Leopold von Andrian-Werburg, Richard Beer-Hofmann, William Busnach, Georges Duval, Joris-Karl Huysmans, John Milton,  Stendhal}
         \renewcommand{\erwaehnteOrte}{Orte: Baden-Baden, Carl-Theater, I., Innere Stadt, IX., Alsergrund, Sanatorium Frey-Dengler, Wien, Wollzeile}
         \renewcommand{\erwaehnteWerke}{Werke: Der Stellvertreter. Schwank in 3 Acten, En route}
               \section[Arthur Schnitzler an Richard Beer-Hofmann, 2. 10. 1897]{ Arthur Schnitzler an Richard Beer-Hofmann, 2. 10. 1897}\nopagebreak\mylabel{v}\rehead{ }\begin{ledgroupsized}[t]{13cm}\normalsize\beginnumbering \toendnotes[C]{\smallbreak\pagebreak[2]} \Standort{YCGL, MSS 31.}
\physDesc{Briefkarte, Umschlag
\newline{}Handschrift: Bleistift, deutsche Kurrent\newline{}Versand: 1) Stempel: »\nobreak{}\oindex{IX., Alsergrund@\textbf{IX., Alsergrund}|pwk}Wien 9/3, 2. 10. {[}9{]}7, 7–\textcolor{gray}{8} N\nobreak{}«.   2) Stempel: »\nobreak{}\oindex{I., Innere Stadt@\textbf{I., Innere Stadt}|pwk}{\pb}Wien 1/1, \textcolor{gray}{3.} 10. 97, 10 ½ V., Bestellt\nobreak{}«. }\toendnotes[C]{\smallbreak}\pstart{}{\pb}Herrn \textsc{Dr. Richard
                     Beer-Hofmann}\pend{}\pstart{}Wien\oindex{Wien@\textbf{Wien}|pw}\pend{}\pstart{}\textsc{I. Wollzeile 15}\oindex{Wollzeile@\textbf{Wollzeile}|pw}.\pend{}{\bigskip}\pstart
           \noindent{}{\pb}Lieber Richard, Ich vergaſs, daſs in
               jenem \label{K_L00726_1v}\edtext{Brief}{\lemma{\textnormal{\emph{Brief}}}\Cendnote{\textnormal{nicht überliefert}}}\label{K_L00726_1h} von Andrian\pwindex{Andrian-Werburg, Leopold von 09.05.1875 – 19.11.1951@\textsc{Andrian-Werburg, Leopold von} (09.05.1875 – 19.11.1951), \emph{Schriftsteller, Diplomat}|pw} auch ſteht, Sie mögen ihn \uline{jedenfalls}{ }\textsc{en route}\pwindex{Huysmans, Joris-Karl 05.02.1848 – 12.05.1907@\textsc{Huysmans, Joris-Karl} (05.02.1848 – 12.05.1907), \emph{Schriftsteller}!En route1895@\strich\emph{En route} {[}1895{]}|pw} von \textsc{Huysmans}\pwindex{Huysmans, Joris-Karl 05.02.1848 – 12.05.1907@\textsc{Huysmans, Joris-Karl} (05.02.1848 – 12.05.1907), \emph{Schriftsteller}|pw} u. etwas über den Milton\pwindex{Milton, John 9.12.1608 – 8.11.1674@\textsc{Milton, John} (9.12.1608 – 8.11.1674), \emph{Philosoph, Dichter, Beamter}|pw} (? unleſerlich)
               von \textsc{Stendhal}\pwindex{Stendhal 1783-01-23 – 1842-03-23@\textsc{Stendhal} (1783-01-23 – 1842-03-23), \emph{Schriftsteller}|pw}{ }ſchicken.\pend
           \pstart
           Seine Adreſſe iſt \textsc{Baden Baden}\oindex{Baden-Baden@\textbf{Baden-Baden}|pw}, {\pb}\textsc{Sanatorium Frey}\oindex{Sanatorium Frey-Dengler@\textbf{Sanatorium Frey-Dengler}|pw}. –\pend
           \pstart
           Ich gehe vielleicht morgen (So{\geminationn}tag)
                  Abend ins \label{K_L00726_2v}\edtext{Carltheater\oindex{Carl-Theater@\textbf{Carl-Theater}|pw}}{\lemma{\textnormal{\emph{Carltheater}}}\Cendnote{\textnormal{Er besuchte die Aufführung von \emph{Der Stellvertreter}\pwindex{Busnach, William 07.03.1832 – 21.01.1907@\textsc{Busnach, William} (07.03.1832 – 21.01.1907), \emph{Schriftsteller}!Stellvertreter. Schwank in 3 Acten1897@\strich\emph{Der Stellvertreter. Schwank in 3 Acten} {[}1897{]}|pwk}\pwindex{Duval, Georges 02.02.1847 – 24.09.1919@\textsc{Duval, Georges} (02.02.1847 – 24.09.1919), \emph{Schriftsteller}!Stellvertreter. Schwank in 3 Acten1897@\strich\emph{Der Stellvertreter. Schwank in 3 Acten} {[}1897{]}|pwk} von William Busnach\pwindex{Busnach, William 07.03.1832 – 21.01.1907@\textsc{Busnach, William} (07.03.1832 – 21.01.1907), \emph{Schriftsteller}|pwk} und Georges
                     Duval\pwindex{Duval, Georges 02.02.1847 – 24.09.1919@\textsc{Duval, Georges} (02.02.1847 – 24.09.1919), \emph{Schriftsteller}|pwk}.}}}\label{K_L00726_2h}.\pend
           \pstart Herzlich Ihr\spacefill\mbox{Arthur.}\pend{}
         
         \endnumbering\mylabel{h}\end{ledgroupsized}  \newcommand{\dateiname}{L00726}\newcommand{\titel}{Arthur Schnitzler an Richard Beer-Hofmann, 2. 10. 1897}\newcommand{\editorInnen}{Martin Anton Müller und Gerd-Hermann Susen}%% latex-leseansicht-abspann.tex
%% Abspann für die Leseansicht.
%% Der Schalter \ifkorrekturansicht ist bereits durch den Vorspann gesetzt.

%% latex-abspann.tex
%% Gemeinsamer Abspann für Korrekturansicht und Leseansicht.
%% Setzt den Schalter \ifkorrekturansicht voraus (gesetzt in den
%% einbindenden Dateien latex-korrekturansicht-abspann.tex bzw.
%% latex-leseansicht-abspann.tex).
%% ---------------------------------------------------------------

\normalsize

% Das esempio-Environment wird nur in der Leseansicht benötigt
\ifkorrekturansicht\else
\newenvironment{esempio}[3]%
{
    \vspace{1.5ex}
    \rlap{\underline{#1}}
    \par
    \setlength{\parindent}{0cm}
    \nopagebreak
    \leftskip=#2cm
    \rightskip=#3cm
}
{
    \par
}
\fi

\doendnotes{C}
\bigskip
\vfill

\clearpage

\footnotesize

\ifkorrekturansicht
  \lohead{\textsc{register}}
\fi

% theindex-Environment neu definieren ohne reledmac
\makeatletter
\renewenvironment{theindex}{%
  \ifkorrekturansicht
    \section*{\indexname}%
  \else
    \subsubsection*{Index der erwähnten Entitäten}%
  \fi
  \setlength{\parindent}{0pt}%
  \setlength{\parskip}{0pt plus 0.3pt}%
  \let\item\@idxitem
}{%
  \ifkorrekturansicht\clearpage\fi
}
\makeatother

\IfFileExists{\jobname-pw.ind}{\input{\jobname-pw.ind}}{}

% Quellenangabe nur in der Leseansicht
\ifkorrekturansicht\else
% Fallback-Definitionen, falls die .tex-Datei \titel etc. nicht gesetzt hat
\providecommand{\titel}{}
\providecommand{\editorInnen}{}
\providecommand{\dateiname}{\jobname}

\vspace{3cm}

\vfill

\footnotesize
\textsc{Quelle}: \titel. Herausgegeben von {\editorInnen}. In: \emph{Arthur Schnitzler: Briefwechsel mit Autorinnen und Autoren}.
 Digitale Edition, https://schnitzler-briefe.acdh.oeaw.ac.at/{\dateiname}.html (Stand \today)
\fi

\end{document}


      