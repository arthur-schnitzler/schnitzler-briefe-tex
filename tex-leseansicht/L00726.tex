%% latex-leseansicht-vorspann.tex
%% Vorspann für die Leseansicht.
%% Lädt die gemeinsame Datei latex-vorspann.tex mit nicht gesetztem Schalter.

\newif\ifkorrekturansicht
\korrekturansichtfalse

\input{../tex-inputs/latex-vorspann}


\section[Arthur Schnitzler an Richard Beer-Hofmann, 2. 10. 1897]{L00726 Arthur Schnitzler an Richard Beer-Hofmann, 2. 10. 1897}
\nopagebreak\mylabel{L00726v}
\rehead{ }\normalsize\beginnumbering\briefempfaengerindex{Beer-Hofmann, Richard@\textsc{Beer-Hofmann, Richard}!zzzSchnitzler, Arthur@\emph{von Arthur Schnitzler}!1897-10-021@{2. 10. 1897}|(be}
\toendnotes[C]{\smallbreak\pagebreak[2]}
\correspDesc{Versand  durch Arthur Schnitzler am 2. 10. 1897 in Wien
\newline{}Erhalt  durch Richard Beer-Hofmann am 3. 10. 1897 in Wien}\toendnotes[C]{\smallbreak}
\Standort{YCGL, MSS 31.}
\physDesc{Briefkarte, , Kuvert, 341 Zeichen
\newline{}Handschrift: Bleistift, deutsche Kurrent
\newline{}Versand: 1) Stempel: »\nobreak{}\oindex{IX., Alsergrund@\textbf{IX., Alsergrund}, \emph{Verwaltungsgebiet}|pwk}Wien 9/3, 2. 10. {[}9{]}7, 7–\textcolor{gray}{8} N\nobreak{}«.   2) Stempel: »\nobreak{}\oindex{I., Innere Stadt@\textbf{I., Innere Stadt}, \emph{Verwaltungsgebiet}|pwk}{\pb}Wien 1/1, \textcolor{gray}{3.} 10. 97, 10 ½ V., Bestellt\nobreak{}«. }\toendnotes[C]{\smallbreak}\pstart{}{\pb}Herrn \textsc{Dr. Richard
                     Beer-Hofmann}\pend{}\pstart{}Wien\oindex{Wien@\textbf{Wien}, \emph{Verwaltungsgebiet}|pw}\pend{}\pstart{}\textsc{I. Wollzeile 15}\oindex{Wien@\textbf{Wien}!I., Innere Stadt@\textbf{I., Innere Stadt}!Wollzeile 15 (»Berthahof«)@\textbf{Wollzeile 15 (»Berthahof«)}, \emph{Wohngebäude}|pw}.\pend{}{\bigskip}\vspace{1em}
\pstart
           \noindent{}{\pb}Lieber Richard, Ich vergaſs, daſs in
               jenem \label{K_L00726-1v}\edtext{Brief}{\lemma{\textnormal{\emph{Brief}}}\Cendnote{\textnormal{nicht überliefert}}}\label{K_L00726-1} von Andrian\pwindex{Andrian-Werburg, Leopold von 9.\,5.\,1875 Berlin – 19.\,11.\,1951 Fribourg@\textsc{Andrian-Werburg, Leopold von} (9.\,5.\,1875 Berlin – 19.\,11.\,1951 Fribourg), \emph{Schriftsteller, Diplomat}|pw} auch{ }ſteht, Sie mögen ihn \uline{jedenfalls}{ }\textsc{en route}\pwindex{Huysmans, Joris-Karl 5.\,2.\,1848 Paris – 12.\,5.\,1907 ebd.@\textsc{Huysmans, Joris-Karl} (5.\,2.\,1848 Paris – 12.\,5.\,1907 ebd.), \emph{Schriftsteller}!En route@\strich\emph{En route}|pw} von \textsc{Huysmans}\pwindex{Huysmans, Joris-Karl 5.\,2.\,1848 Paris – 12.\,5.\,1907 ebd.@\textsc{Huysmans, Joris-Karl} (5.\,2.\,1848 Paris – 12.\,5.\,1907 ebd.), \emph{Schriftsteller}|pw} u. etwas über den Milton\pwindex{Milton, John 9.\,12.\,1608 – 8.\,11.\,1674@\textsc{Milton, John} (9.\,12.\,1608 – 8.\,11.\,1674), \emph{Philosoph, Dichter, Beamter}|pw} (? unleſerlich)
               von \textsc{Stendhal}\pwindex{Stendhal 23.\,1.\,1783 Grenoble – 23.\,3.\,1842 Paris@\textsc{Stendhal} (23.\,1.\,1783 Grenoble – 23.\,3.\,1842 Paris), \emph{Schriftsteller}|pw}{ }ſchicken.\pend
           
\pstart
           Seine Adreſſe iſt \textsc{Baden Baden}\oindex{Baden-Baden@\textbf{Baden-Baden}|pw}, {\pb}\textsc{Sanatorium Frey}\oindex{Sanatorium Frey-Dengler@\textbf{Sanatorium Frey-Dengler}, \emph{Sanatorium}|pw}. –\pend
           
\pstart
           Ich gehe vielleicht morgen (So{\geminationn}tag)
                  Abend ins \label{K_L00726-2v}\edtext{Carltheater\oindex{Wien@\textbf{Wien}!II., Leopoldstadt@\textbf{II., Leopoldstadt}!Carl-Theater@\textbf{Carl-Theater}, \emph{Theater}|pw}}{\lemma{\textnormal{\emph{Carltheater}}}\Cendnote{\textnormal{Schnitzler besuchte die Aufführung von \emph{Der Stellvertreter}\pwindex{Busnach, William 7.\,3.\,1832 Paris – 21.\,1.\,1907 ebd.@\textsc{Busnach, William} (7.\,3.\,1832 Paris – 21.\,1.\,1907 ebd.), \emph{Schriftsteller}!Stellvertreter. Schwank in 3 Acten@\strich\emph{Der Stellvertreter. Schwank in 3 Acten}|pwk}\pwindex{Duval, Georges 2.\,2.\,1847 Paris – 24.\,9.\,1919 ebd.@\textsc{Duval, Georges} (2.\,2.\,1847 Paris – 24.\,9.\,1919 ebd.), \emph{Schriftsteller}!Stellvertreter. Schwank in 3 Acten@\strich\emph{Der Stellvertreter. Schwank in 3 Acten}|pwk} von William Busnach\pwindex{Busnach, William 7.\,3.\,1832 Paris – 21.\,1.\,1907 ebd.@\textsc{Busnach, William} (7.\,3.\,1832 Paris – 21.\,1.\,1907 ebd.), \emph{Schriftsteller}|pwk} und Georges
                     Duval\pwindex{Duval, Georges 2.\,2.\,1847 Paris – 24.\,9.\,1919 ebd.@\textsc{Duval, Georges} (2.\,2.\,1847 Paris – 24.\,9.\,1919 ebd.), \emph{Schriftsteller}|pwk}.}}}\label{K_L00726-2}.\pend
           \pstart Herzlich Ihr\spacefill\mbox{Arthur.}\pend{}\selectlanguage{ngerman}\endnumbering\briefempfaengerindex{Beer-Hofmann, Richard@\textsc{Beer-Hofmann, Richard}!zzzSchnitzler, Arthur@\emph{von Arthur Schnitzler}!1897-10-021@{2. 10. 1897}|)be}\mylabel{L00726h}  \newcommand{\dateiname}{L00726}\newcommand{\titel}{Arthur Schnitzler an Richard Beer-Hofmann, 2. 10. 1897}\newcommand{\editorInnen}{Martin Anton Müller und Gerd-Hermann Susen}%% latex-leseansicht-abspann.tex
%% Abspann für die Leseansicht.
%% Der Schalter \ifkorrekturansicht ist bereits durch den Vorspann gesetzt.

%% latex-abspann.tex
%% Gemeinsamer Abspann für Korrekturansicht und Leseansicht.
%% Setzt den Schalter \ifkorrekturansicht voraus (gesetzt in den
%% einbindenden Dateien latex-korrekturansicht-abspann.tex bzw.
%% latex-leseansicht-abspann.tex).
%% ---------------------------------------------------------------

\normalsize

% Das esempio-Environment wird nur in der Leseansicht benötigt
\ifkorrekturansicht\else
\newenvironment{esempio}[3]%
{
    \vspace{1.5ex}
    \rlap{\underline{#1}}
    \par
    \setlength{\parindent}{0cm}
    \nopagebreak
    \leftskip=#2cm
    \rightskip=#3cm
}
{
    \par
}
\fi

\doendnotes{C}
\bigskip
\vfill

\clearpage

\footnotesize

\ifkorrekturansicht
  \lohead{\textsc{register}}
\fi

% theindex-Environment neu definieren ohne reledmac
\makeatletter
\renewenvironment{theindex}{%
  \ifkorrekturansicht
    \section*{\indexname}%
  \else
    \subsubsection*{Index der erwähnten Entitäten}%
  \fi
  \setlength{\parindent}{0pt}%
  \setlength{\parskip}{0pt plus 0.3pt}%
  \let\item\@idxitem
}{%
  \ifkorrekturansicht\clearpage\fi
}
\makeatother

\IfFileExists{\jobname-pw.ind}{\input{\jobname-pw.ind}}{}

% Quellenangabe nur in der Leseansicht
\ifkorrekturansicht\else
% Fallback-Definitionen, falls die .tex-Datei \titel etc. nicht gesetzt hat
\providecommand{\titel}{}
\providecommand{\editorInnen}{}
\providecommand{\dateiname}{\jobname}

\vspace{3cm}

\vfill

\footnotesize
\textsc{Quelle}: \titel. Herausgegeben von {\editorInnen}. In: \emph{Arthur Schnitzler: Briefwechsel mit Autorinnen und Autoren}.
 Digitale Edition, https://schnitzler-briefe.acdh.oeaw.ac.at/{\dateiname}.html (Stand \today)
\fi

\end{document}


