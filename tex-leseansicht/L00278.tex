%% latex-leseansicht-vorspann.tex
%% Vorspann für die Leseansicht.
%% Lädt die gemeinsame Datei latex-vorspann.tex mit nicht gesetztem Schalter.

\newif\ifkorrekturansicht
\korrekturansichtfalse

\input{../tex-inputs/latex-vorspann}


\section[Hermann Bahr an Arthur Schnitzler, 3. 11. 1893]{L00278 Hermann Bahr an Arthur Schnitzler, 3. 11. 1893}
\nopagebreak\mylabel{L00278v}
\rehead{ }\normalsize\beginnumbering\briefempfaengerindex{Schnitzler, Arthur@\textsc{Schnitzler, Arthur}!zzzBahr, Hermann@\emph{von Hermann Bahr}!1893-11-031@{3. 11. 1893}|(be}
\toendnotes[C]{\smallbreak\pagebreak[2]}
\correspDesc{Versand  durch Hermann Bahr am 3. 11. 1893 in Wien
\newline{}Erhalt  durch Arthur Schnitzler im Zeitraum [3. 11. 1893
                  – 7. 11. 1893?] in Wien}\toendnotes[C]{\smallbreak}
\Standort{CUL, Schnitzler, B 5b.}
\physDesc{Brief, 1 Blatt, 1 Seite, 499 Zeichen
\newline{}Handschrift Hermann Bahr: schwarze Tinte, deutsche Kurrent (\noindent{}Unterschrift)
\newline{}Handschrift Schreibkraft: schwarze Tinte, deutsche Kurrent
\newline{}Ordnung: mit rotem Buntstift von unbekannter Hand und mit Bleistift
                                 jeweils nummeriert: »16« }
\buchAbdrucke{\weitereDrucke{Hermann Bahr, Arthur Schnitzler: \emph{Briefwechsel, Aufzeichnungen, Dokumente (1891–1931)}. Herausgegeben von Kurt Ifkovits und Martin Anton Müller. Göttingen: \emph{Wallstein} 2018, S. 46.} }\toendnotes[C]{\smallbreak}
\pstart
           {\pb}\textcolor{gray}{\textbf{Deutſche Zeitung\orgindex{Deutsche Zeitung@Deutsche Zeitung|pw}}}\hfill \uline{Wien\oindex{Wien@\textbf{Wien}, \emph{Verwaltungsgebiet}|pw}}, 3. Novbr. 1893.\pend
           
\pstart
           \textcolor{gray}{\textbf{Wien\oindex{Wien@\textbf{Wien}, \emph{Verwaltungsgebiet}|pw}}}\hfill III. Saleſianerg. 12\oindex{Wien@\textbf{Wien}!III., Landstraße@\textbf{III., Landstraße}!Salesianergasse 12@\textbf{Salesianergasse 12}, \emph{Wohngebäude}|pw}\pend
           
\pstart
           \textcolor{gray}{\textbf{IX., Pelikangaſſse 4\oindex{Wien@\textbf{Wien}!IX., Alsergrund@\textbf{IX., Alsergrund}!Pelikangasse@\textbf{Pelikangasse}, \emph{Straße}|pw}.}}\pend
           
\pstart{}Lieber Freund!\pend\vspace{0.5em}
\pstart
           Wenn Sie mir nichts anderes geben, will ich es verſuchen den \textsc{Artifex\pwindex{Schnitzler, Arthur 15.\,5.\,1862 Wien – 21.\,10.\,1931 ebd.@\textsc{Schnitzler, Arthur} (15.\,5.\,1862 Wien – 21.\,10.\,1931 ebd.), \emph{Schriftsteller, Mediziner}!Artifex@\strich\emph{Artifex}|pw}} durchzuſetzen. Doch wäre mir aufrichtig geſagt etwas anderes lieber. Aber das
               Wichtigſte bleibt, daſz Sie mir endlich etwas für den Wiener Spiegel\pwindex{Schnitzler, Arthur 15.\,5.\,1862 Wien – 21.\,10.\,1931 ebd.@\textsc{Schnitzler, Arthur} (15.\,5.\,1862 Wien – 21.\,10.\,1931 ebd.), \emph{Schriftsteller, Mediziner}!Spaziergang@\strich\emph{Spaziergang}|pwv}{ }ſenden – nun haben Sie einmal verſprochen, nun
               hilft Ihnen nichts mehr Sie müſſen in den{ }ſauren Apfel beiſzen und bitte vergeſzen
               Sie mir auch nicht das \label{K_L00278-1v}\edtext{Feuilleton}{\lemma{\textnormal{\emph{Feuilleton}}}\Cendnote{\textnormal{nicht erschienen}}}\label{K_L00278-1} über \textsc{Schönlein}\pwindex{Schönlein, Johann Lukas 30.\,11.\,1793 Bamberg – 23.\,1.\,1864 ebd.@\textsc{Schönlein, Johann Lukas} (30.\,11.\,1793 Bamberg – 23.\,1.\,1864 ebd.), \emph{Mediziner}|pw} zu beſorgen.\pend
           
\pstart
           Mit herzlichen Grüſzen Ihr treuer{\\[\baselineskip]}\spacefill\mbox{{[}hs. Bahr:{]} Hermann Bahr}\pend
           \leftskip=0em{}\selectlanguage{ngerman}\endnumbering\briefempfaengerindex{Schnitzler, Arthur@\textsc{Schnitzler, Arthur}!zzzBahr, Hermann@\emph{von Hermann Bahr}!1893-11-031@{3. 11. 1893}|)be}\mylabel{L00278h}  \newcommand{\dateiname}{L00278}\newcommand{\titel}{Hermann Bahr an Arthur Schnitzler, 3. 11. 1893}\newcommand{\editorInnen}{Herausgegeben von Martin Anton Müller}%% latex-leseansicht-abspann.tex
%% Abspann für die Leseansicht.
%% Der Schalter \ifkorrekturansicht ist bereits durch den Vorspann gesetzt.

%% latex-abspann.tex
%% Gemeinsamer Abspann für Korrekturansicht und Leseansicht.
%% Setzt den Schalter \ifkorrekturansicht voraus (gesetzt in den
%% einbindenden Dateien latex-korrekturansicht-abspann.tex bzw.
%% latex-leseansicht-abspann.tex).
%% ---------------------------------------------------------------

\normalsize

% Das esempio-Environment wird nur in der Leseansicht benötigt
\ifkorrekturansicht\else
\newenvironment{esempio}[3]%
{
    \vspace{1.5ex}
    \rlap{\underline{#1}}
    \par
    \setlength{\parindent}{0cm}
    \nopagebreak
    \leftskip=#2cm
    \rightskip=#3cm
}
{
    \par
}
\fi

\doendnotes{C}
\bigskip
\vfill

\clearpage

\footnotesize

\ifkorrekturansicht
  \lohead{\textsc{register}}
\fi

% theindex-Environment neu definieren ohne reledmac
\makeatletter
\renewenvironment{theindex}{%
  \ifkorrekturansicht
    \section*{\indexname}%
  \else
    \subsubsection*{Index der erwähnten Entitäten}%
  \fi
  \setlength{\parindent}{0pt}%
  \setlength{\parskip}{0pt plus 0.3pt}%
  \let\item\@idxitem
}{%
  \ifkorrekturansicht\clearpage\fi
}
\makeatother

\IfFileExists{\jobname-pw.ind}{\input{\jobname-pw.ind}}{}

% Quellenangabe nur in der Leseansicht
\ifkorrekturansicht\else
% Fallback-Definitionen, falls die .tex-Datei \titel etc. nicht gesetzt hat
\providecommand{\titel}{}
\providecommand{\editorInnen}{}
\providecommand{\dateiname}{\jobname}

\vspace{3cm}

\vfill

\footnotesize
\textsc{Quelle}: \titel. Herausgegeben von {\editorInnen}. In: \emph{Arthur Schnitzler: Briefwechsel mit Autorinnen und Autoren}.
 Digitale Edition, https://schnitzler-briefe.acdh.oeaw.ac.at/{\dateiname}.html (Stand \today)
\fi

\end{document}


