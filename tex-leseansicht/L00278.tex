%% latex-korrekturansicht-vorspann.tex
%% Vorspann für die Korrekturansicht.
%% Lädt die gemeinsame Datei latex-vorspann.tex mit gesetztem Schalter.

\newif\ifkorrekturansicht
\korrekturansichttrue

\input{../tex-inputs/latex-vorspann}


\section[Hermann Bahr an Arthur Schnitzler, 3. 11. 1893]{L00278 Hermann Bahr an Arthur Schnitzler, 3. 11. 1893}
\nopagebreak\mylabel{L00278v}
\rehead{ }\normalsize\beginnumbering\briefempfaengerindex{Schnitzler, Arthur@\textsc{Schnitzler, Arthur}!zzzBahr, Hermann@\emph{von Hermann Bahr}!1893-11-031@{3. 11. 1893}|(be}
\toendnotes[C]{\smallbreak\pagebreak[2]}\Standort{CUL, Schnitzler, B 5b.}
\physDesc{Brief, 1 Blatt, 1 Seite, 499 Zeichen
\newline{}Handschrift Hermann Bahr: schwarze Tinte, deutsche Kurrent (\noindent{}Unterschrift)
\newline{}Handschrift Schreibkraft: schwarze Tinte, deutsche Kurrent
\newline{}Ordnung: mit rotem Buntstift von unbekannter Hand und mit Bleistift
                                 jeweils nummeriert: »16« }
\buchAbdrucke{\weitereDrucke{Hermann Bahr, Arthur Schnitzler: \emph{Briefwechsel, Aufzeichnungen, Dokumente (1891–1931)}. Göttingen: \emph{Wallstein} 2018, S. 46.} }\toendnotes[C]{\smallbreak}
\pstart
           {\pb}\textcolor{gray}{\textbf{Deutſche Zeitung\orgindex{Deutsche Zeitung@Deutsche Zeitung|pw}}}\hfill \uline{Wien\oindex{Wien@\textbf{Wien}, \emph{A.ADM2}|pw}}, 3. Novbr. 1893.\pend
           
\pstart
           \textcolor{gray}{\textbf{Wien\oindex{Wien@\textbf{Wien}, \emph{A.ADM2}|pw}}}\hfill III. Saleſianerg. 12\oindex{Salesianergasse 12@\textbf{Salesianergasse 12}, \emph{Wohngebäude (K.WHS)}|pw}\pend
           
\pstart
           \textcolor{gray}{\textbf{IX., Pelikangaſſse 4\oindex{Pelikangasse@\textbf{Pelikangasse}, \emph{Straße (K.STR)}|pw}.}}\pend
           
\pstart{}Lieber Freund!\pend\vspace{0.5em}
\pstart
           Wenn Sie mir nichts anderes geben, will ich es verſuchen den \textsc{Artifex\pwindex{Artifex@\emph{Artifex}|pw}} durchzuſetzen. Doch wäre mir aufrichtig geſagt etwas anderes lieber. Aber das
               Wichtigſte bleibt, daſz Sie mir endlich etwas für den Wiener Spiegel\pwindex{Spaziergang@\emph{Spaziergang}|pwv}{ }ſenden – nun haben Sie einmal verſprochen, nun
               hilft Ihnen nichts mehr Sie müſſen in den ſauren Apfel beiſzen und bitte vergeſzen
               Sie mir auch nicht das \label{K_L00278-1v}\edtext{Feuilleton}{\lemma{\textnormal{\emph{Feuilleton}}}\Cendnote{\textnormal{nicht erschienen}}}\label{K_L00278-1} über \textsc{Schönlein}\pwindex{Schoenlein, Johann Lukas 30.11.1793 – 23.01.1864@\textsc{Schönlein, Johann Lukas} (30.11.1793 – 23.01.1864), \emph{Mediziner/Medizinerin}|pw} zu beſorgen.\pend
           
\pstart
           Mit herzlichen Grüſzen Ihr treuer{\\[\baselineskip]}\spacefill\mbox{{[}hs. :{]} Hermann Bahr}\pend
           \leftskip=0em{}\selectlanguage{ngerman}\endnumbering\briefempfaengerindex{Schnitzler, Arthur@\textsc{Schnitzler, Arthur}!zzzBahr, Hermann@\emph{von Hermann Bahr}!1893-11-031@{3. 11. 1893}|)be}\mylabel{L00278h}  \normalsize

\doendnotes{C}
\bigskip
\vfill

\clearpage

\footnotesize

\lohead{\textsc{register}}

% Definiere theindex-Environment komplett neu ohne reledmac
\makeatletter
\renewenvironment{theindex}{%
  \section*{\indexname}%
  \setlength{\parindent}{0pt}%
  \setlength{\parskip}{0pt plus 0.3pt}%
  \let\item\@idxitem
}{%
  \clearpage
}
\makeatother

\IfFileExists{\jobname-pw.ind}{\input{\jobname-pw.ind}}{}

\end{document}

      