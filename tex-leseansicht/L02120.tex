%% latex-leseansicht-vorspann.tex
%% Vorspann für die Leseansicht.
%% Lädt die gemeinsame Datei latex-vorspann.tex mit nicht gesetztem Schalter.

\newif\ifkorrekturansicht
\korrekturansichtfalse

\input{../tex-inputs/latex-vorspann}


         
         \renewcommand{\erwaehntePersonen}{Personen: Peter Altenberg, Hermann Bahr, Anna Bahr-Mildenburg, Georg Engländer, Hugo von Hofmannsthal}
         \renewcommand{\erwaehnteOrte}{Orte: Otto-Wagner-Spital, Seidlgasse, Semmering, Wien}
         \renewcommand{\erwaehnteWerke}{Werke: Der Weg ins Freie. Roman, »Semmering 1912«}
               \section[Peter Altenberg und Georg Engländer an Arthur Schnitzler, {[}Mitte April{]} 1913]{ Peter Altenberg und Georg Engländer an Arthur Schnitzler, {[}Mitte April{]}
               1913}\nopagebreak\mylabel{v}\rehead{ }\begin{ledgroupsized}[t]{13cm}\normalsize\beginnumbering\briefempfaengerindex{Schnitzler, Arthur@\textsc{Schnitzler, Arthur}!zzzEnglaender, Georg@\emph{von Georg Engländer}!1913-04-152@{{[}Mitte April{]} 1913}|(be}\briefempfaengerindex{Schnitzler, Arthur@\textsc{Schnitzler, Arthur}!zzzAltenberg, Peter@\emph{von Peter Altenberg}!1913-04-152@{{[}Mitte April{]} 1913}|(be} \toendnotes[C]{\smallbreak\pagebreak[2]} \Standort{CUL, Schnitzler, B 2.}
\physDesc{Brief, 1 Blatt, 4 Seiten, 1776 Zeichen
\newline{}Handschrift: schwarze Tinte, deutsche Kurrent
\newline{}Schnitzler: 1) mit Bleistift erstes Blatt beschriftet: »\textsc{Altenberg}« und datiert: »April 1913«  2) mit rotem Buntstift eine Unterstreichung
\newline{}Ordnung: mit Bleistift von unbekannter Hand nummeriert:
                                    »14« }\Standort{CUL, Schnitzler, B 2.}
\physDesc{Brief, 1 Blatt, 3 Seiten, 1776 Zeichen
\newline{}Handschrift: schwarze Tinte, deutsche Kurrent
\newline{}Schnitzler: mit Bleistift beschriftet: »\textsc{Engländer}« und datiert: »1914/1915« 
\newline{}Editorischer Hinweis: Die Hinzufügung dieses Blattes zum Korrespondenzstück erfolgt in
                                 Abgleich mit einem Brief Altenbergs und Engländers an Bahr (\emph{Briefwechsel} Bahr/Schnitzler,
                                    480–481), der offensichtlich zeitnah entstand. Zudem ist
                                 aus dem Inhalt erkenntlich, dass es sich nicht um ein
                                 eigenständiges Schreiben handelt. }\buchAbdrucke{\weitereDrucke{Kurt Bergel: \emph{Arthur Schnitzlers unveröffentlichte Tragikomödie Das Wort.} In: \emph{Studies in Arthur Schnitzler. Centennial Commemorative
                        Volume}. Hg. Herbert W. Reichert und Herman Salinger. Chapel Hill: \emph{University of North Carolina Press} 1963, S. 22 (UNC Studies in the Germanic Languages and Literatures, 42).} }\toendnotes[C]{\smallbreak}\pstart{}{\pb}Lieber lieber Herr \textsc{D\textsuperscript{r}} Arthur Schnitzler,\pend\pstart
           ein Verlorener, Zuſammengeſtürzter, unmittelbar nach einem paradieſiſchen Semmering\oindex{Semmering@\textbf{Semmering}|pw}-Jahr 1912, ein \uuline{\edtext{tiefſt}{\Cendnote{dreifach unterstrichen}}} Verzweifelter, wendet ſich an Sie als
               Menſchenfreundlichen und Dichter vor allem, dann als Kollegen und langjährigen
               litterariſchen Genoſſen – – – Hilfe, Rettung, Erbarmen, in einer ſo \uline{ſchauerlichen} Situation, die noch nie, noch nie, noch nie,
               ein Dichter, ein Künſtler-Menſch erlitten hat! {\pb}Der ſüßen unentbehrlichen Freiheit
               beraubt, verbringe ich meine Tage u. Nächte in unermeſslichen Qualen, eingefangen,
               kontrollirt wie ein \uline{böſes gefährliches giftiges
                  Reptil}!\pend
           \pstart
           \label{K_L02120-1v}\edtext{Hilfe, Errettung, Weg ins Freie\pwindex{Schnitzler, Arthur 15.05.1862 – 21.10.1931@\textsc{Schnitzler, Arthur} (15.05.1862 – 21.10.1931), \emph{Schriftsteller, Mediziner}!Weg ins Freie. Roman1.1.1908 – 1.6.1908@\strich\emph{Der Weg ins Freie. Roman} {[}1.1.1908 – 1.6.1908{]}|pwv}}{\lemma{\textnormal{\emph{Hilfe, … Freie}}}\Cendnote{\textnormal{Vermutlich Mitte April 1913{ }schrieb Altenberg\pwindex{Altenberg, Peter 09.03.1859 – 08.01.1919@\textsc{Altenberg, Peter} (09.03.1859 – 08.01.1919), \emph{Schriftsteller}|pwk} an Hermann Bahr\pwindex{Bahr, Hermann 19.07.1863 – 15.01.1934@\textsc{Bahr, Hermann} (19.07.1863 – 15.01.1934), \emph{Schriftsteller, Kritiker}|pwk} und,
                  separat, an dessen Gattin Anna
                     Bahr-Mildenburg\pwindex{Bahr-Mildenburg, Anna 29.11.1872 – 27.01.1947@\textsc{Bahr-Mildenburg, Anna} (29.11.1872 – 27.01.1947), \emph{Sängerin}|pwk} (\emph{Korrespondenz von Peter Altenberg an Hermann Bahr
                        (1895–1913)}. Hgg. Heinz Lunzer, Victoria Lunzer-Talos. In: Jeanne
                     Bennay, Alfred Pfabigan, Hgg.: \emph{Hermann Bahr – Für eine andere
                        Moderne}. Bern: \emph{Peter Lang}{ }2004, S. 249-262, hier S. 259–262.) In Folge dessen schrieb
                     Bahr\pwindex{Bahr, Hermann 19.07.1863 – 15.01.1934@\textsc{Bahr, Hermann} (19.07.1863 – 15.01.1934), \emph{Schriftsteller, Kritiker}|pwk} am 16. 4. 1913 an Schnitzler\pwindex{Schnitzler, Arthur 15.05.1862 – 21.10.1931@\textsc{Schnitzler, Arthur} (15.05.1862 – 21.10.1931), \emph{Schriftsteller, Mediziner}|pwk} über den »verworrenen
                     Brief«. Dieser antwortete zwei Tage später, er habe gleichfalls einen
                  Brief Altenberg\pwindex{Altenberg, Peter 09.03.1859 – 08.01.1919@\textsc{Altenberg, Peter} (09.03.1859 – 08.01.1919), \emph{Schriftsteller}|pwk}s erhalten. Die sprachliche
                  Entsprechung von Formulierungen, wie »Hilfe, Errettung,
                  Erbarmen!!!« an Bahr\pwindex{Bahr, Hermann 19.07.1863 – 15.01.1934@\textsc{Bahr, Hermann} (19.07.1863 – 15.01.1934), \emph{Schriftsteller, Kritiker}|pwk} legen die
                  zeitliche Unmittelbarkeit der beiden Korrespondenzstücke an Bahr\pwindex{Bahr, Hermann 19.07.1863 – 15.01.1934@\textsc{Bahr, Hermann} (19.07.1863 – 15.01.1934), \emph{Schriftsteller, Kritiker}|pwk} und Schnitzler\pwindex{Schnitzler, Arthur 15.05.1862 – 21.10.1931@\textsc{Schnitzler, Arthur} (15.05.1862 – 21.10.1931), \emph{Schriftsteller, Mediziner}|pwk}
                  nahe.}}}\label{K_L02120-1h}!!!\pend
           \pstart
           Auch geht es mir ökonomiſch ſchlecht, und bitte ich Sie und Hofmannsthal\pwindex{Hofmannsthal, Hugo von 1874-02-01 – 1929-07-15@\textsc{Hofmannsthal, Hugo von} (1874-02-01 – 1929-07-15), \emph{Schriftsteller}|pw} um die mir {\pb}zugeſagten \uline{20} Kr. monatlich ſeit \uline{November 1912}, da ich
               gerade damals zuſammenbrach und nicht mehr denken konnte!\pend
           \pstart
           \uuline{\edtext{Hilfe}{\Cendnote{dreifach unterstrichen}}}, um Gotteswillen, ehe ich ganz zerſtört
               bin!\pend
           \pstart
           Ich möchte auf dem Semmering\oindex{Semmering@\textbf{Semmering}|pw} ruhig vegetiren, in
               Freiheit und Frieden! Hilfe von \uuline{\edtext{Bruder}{\Cendnote{dreifach unterstrichen}}}-Seelen!
               Dichter, Künſtler, Menſchen, helft mir!!!\pend
           \pstart \spacefill\mbox{Peter Altenberg}\pend{}\pstart
           \noindent{}{\pb}\label{T_L02120-1v}\edtext{Adreſſe}{\lemma{\textnormal{\emph{Adreſſe}}}\Cendnote{\textnormal{Hier wechselt die Schreibrichtung und das Blatt ist entlang
                     des Mittelfalzes beschrieben.}}}\label{T_L02120-1h}: XIII/\textsubscript{12}{ }\label{K_L02120-2v}\edtext{\textsc{Villa Austria}}{\lemma{\textnormal{\emph{Villa Austria}}}\Cendnote{\textnormal{Pavillon der
                        Landesnervenheilanstalt Am
                        Steinhof\oindex{Otto-Wagner-Spital@\textbf{Otto-Wagner-Spital}|pwk}.}}}\label{K_L02120-2h}\oindex{Otto-Wagner-Spital@\textbf{Otto-Wagner-Spital}|pw}\pend
           \pstart
           Leſen Sie mein letztes Buch:\pend
           \pstart
           \centering{}»Semmering 1912\pwindex{Altenberg, Peter 09.03.1859 – 08.01.1919@\textsc{Altenberg, Peter} (09.03.1859 – 08.01.1919), \emph{Schriftsteller}!Semmering 1912«1913@\strich\emph{»Semmering 1912«} {[}1913{]}|pw}«\pend
           \pstart
           \noindent{}und denken Sie, wie dem Autor zumute iſt, der nun wie ein wildes Tier eingeſperrt
                  ſchmachtet, ſeit \uline{5} Monaten!!!\pend
           \pstart
           Ihr{\\}\spacefill\mbox{PA}\pend
           \pstart
           \noindent{}{\pb}{[}hs. Engländer:{]} \uline{Zur Aufklärung}. \textsc{(Diskret!)}\pend
           \pstart{}Sehr geehrter Herr.\pend\pstart
           Am 10 Dec. v. J. mußte ich meinen Bruder in einem erbarmungswürdigen \textsc{Nerven-Zustand} auf den \textsc{Steinhof}\oindex{Otto-Wagner-Spital@\textbf{Otto-Wagner-Spital}|pw} überführen.\pend
           \pstart
           Nun erſt ſeit 3 {\pb}Wochen ko{\geminationm}t er allmählich zum \textsc{Bewusstsein}{ }{\kaufmannsund} iſt empört über den Zwang den Ärzte{ }{\kaufmannsund} Pfleger auf ihn ausüben {\kaufmannsund}
               will durchaus entfliehen. Ärztliche {\pb}Freunde finden aber auch jetzt noch ſeinen Kopf {\kaufmannsund}{ }\textsc{Nervenzustand} ſo labil daſs ſie auch nur einige Tage
               Freiheit ſchon für ſeine Gesundheit als \textsc{katastrophal}
               befürchten.\pend
           \pstart
           Hochachtend{\\[\baselineskip]}\spacefill\mbox{G. Engländer}\pend
           \leftskip=0em{}\pstart
           \noindent{}III \textsc{Seidlgasse} 23\oindex{Seidlgasse@\textbf{Seidlgasse}|pw}.\pend
           \pstart
           P.S. Seine \textsc{Correſp}. wird mir von der \textsc{Anstalt}\oindex{Otto-Wagner-Spital@\textbf{Otto-Wagner-Spital}|pw} offen zugeſandt!!\pend
           
         
         \endnumbering\mylabel{h}\end{ledgroupsized}  \newcommand{\dateiname}{L02120}\newcommand{\titel}{Peter Altenberg und Georg Engländer an Arthur Schnitzler, [Mitte April] 1913}\newcommand{\editorInnen}{Martin Anton Müller und Gerd-Hermann Susen}%% latex-leseansicht-abspann.tex
%% Abspann für die Leseansicht.
%% Der Schalter \ifkorrekturansicht ist bereits durch den Vorspann gesetzt.

%% latex-abspann.tex
%% Gemeinsamer Abspann für Korrekturansicht und Leseansicht.
%% Setzt den Schalter \ifkorrekturansicht voraus (gesetzt in den
%% einbindenden Dateien latex-korrekturansicht-abspann.tex bzw.
%% latex-leseansicht-abspann.tex).
%% ---------------------------------------------------------------

\normalsize

% Das esempio-Environment wird nur in der Leseansicht benötigt
\ifkorrekturansicht\else
\newenvironment{esempio}[3]%
{
    \vspace{1.5ex}
    \rlap{\underline{#1}}
    \par
    \setlength{\parindent}{0cm}
    \nopagebreak
    \leftskip=#2cm
    \rightskip=#3cm
}
{
    \par
}
\fi

\doendnotes{C}
\bigskip
\vfill

\clearpage

\footnotesize

\ifkorrekturansicht
  \lohead{\textsc{register}}
\fi

% theindex-Environment neu definieren ohne reledmac
\makeatletter
\renewenvironment{theindex}{%
  \ifkorrekturansicht
    \section*{\indexname}%
  \else
    \subsubsection*{Index der erwähnten Entitäten}%
  \fi
  \setlength{\parindent}{0pt}%
  \setlength{\parskip}{0pt plus 0.3pt}%
  \let\item\@idxitem
}{%
  \ifkorrekturansicht\clearpage\fi
}
\makeatother

\IfFileExists{\jobname-pw.ind}{\input{\jobname-pw.ind}}{}

% Quellenangabe nur in der Leseansicht
\ifkorrekturansicht\else
% Fallback-Definitionen, falls die .tex-Datei \titel etc. nicht gesetzt hat
\providecommand{\titel}{}
\providecommand{\editorInnen}{}
\providecommand{\dateiname}{\jobname}

\vspace{3cm}

\vfill

\footnotesize
\textsc{Quelle}: \titel. Herausgegeben von {\editorInnen}. In: \emph{Arthur Schnitzler: Briefwechsel mit Autorinnen und Autoren}.
 Digitale Edition, https://schnitzler-briefe.acdh.oeaw.ac.at/{\dateiname}.html (Stand \today)
\fi

\end{document}


      