%% latex-leseansicht-vorspann.tex
%% Vorspann für die Leseansicht.
%% Lädt die gemeinsame Datei latex-vorspann.tex mit nicht gesetztem Schalter.

\newif\ifkorrekturansicht
\korrekturansichtfalse

\input{../tex-inputs/latex-vorspann}


\section[Theodor Herzl an Arthur Schnitzler, 29. 6. 1893]{L03831 Theodor Herzl an Arthur Schnitzler, 29. 6. 1893}
\nopagebreak\mylabel{L03831v}
\rehead{ }\normalsize\beginnumbering\briefempfaengerindex{Schnitzler, Arthur@\textsc{Schnitzler, Arthur}!zzzHerzl, Theodor@\emph{von Theodor Herzl}!1893-06-291@{29. 6. 1893}|(be}
\toendnotes[C]{\smallbreak\pagebreak[2]}
\correspDesc{Versand  durch Theodor Herzl am 29. 6. 1893 in Morschach
\newline{}Erhalt  durch Arthur Schnitzler im Zeitraum [30. 6. 1893
                  – 4. 7. 1893?] in Wien}\toendnotes[C]{\smallbreak}
\Standort{Jerusalem, Central Zionist Archives, H1:2540-4.}
\physDesc{Brief, Fotokopie, 2 Blätter, 2 Seiten, 828 Zeichen
\newline{}Handschrift: schwarze Tinte, lateinische Kurrent
\newline{}Ordnung: 1) mit Bleistift von unbekannter Hand nummeriert: »24« und »25«  2) mit Bleistift von unbekannter Hand datiert: »29. Juni 1893«
\newline{}Zusatz: Der Originalbrief findet sich Katalog 576 von
                                    \emph{J. A. Stargardt}, wo er als Lot. Nr. 1006 um 500 DM
                                    angeboten wurde. Die Auktion fand von 24. – 25. 5. 1966 in Marburg\oindex{XXXX Ortsangabe fehlt|pw}
                                    statt. Der gegenwärtige Verbleib ist nicht geklärt. }\Standort{CUL, Schnitzler, B 39.}
\physDesc{Brief, Fotokopie, 1 Blatt, 2 Seiten, 828 Zeichen
\newline{}Handschrift: schwarze Tinte, lateinische Kurrent
\newline{}Zusatz: In der Nachlassmappe B 39 hat Heinrich Schnitzler\pwindex{Schnitzler, Heinrich 9.\,8.\,1902 Hinterbrühl – 12.\,7.\,1982 Wien@\textsc{Schnitzler, Heinrich} (9.\,8.\,1902 Hinterbrühl – 12.\,7.\,1982 Wien), \emph{Regisseur, Schauspieler}|pw} vermerkt: »\noindent{}2 Briefe
                                       geschenkt ans Wolf-Museum Eisenstadt\orgindex{Landesmuseum Burgenland@Landesmuseum Burgenland|pw}{ }22. VIII. 1937.{ / }1 Brief entnommen{ / }1 Brief geschenkt an Paul Marx\pwindex{Marx, Paul 21.\,7.\,1879 Wien – 30.\,10.\,1956 ebd.@\textsc{Marx, Paul} (21.\,7.\,1879 Wien – 30.\,10.\,1956 ebd.), \emph{Regisseur, Schauspieler}|pw}{ }15. VIII. 1936.{ / }1 Brief gegeben an Mutter\pwindex{Schnitzler, Olga 17.\,1.\,1882 Wien – 13.\,1.\,1970 Lugano@\textsc{Schnitzler, Olga} (17.\,1.\,1882 Wien – 13.\,1.\,1970 Lugano), \emph{Schauspielerin, Sängerin}|pwv}, 15. VIII. 36.« Das entspricht
                                 der Anzahl von fünf Korrespondenzstücken von Herzl, die nicht im Original überliefert sind. Alle finden sich in einer Abschrift, die nach
                                 Arthur Schnitzlers Tod im Zeitraum 1932 bis 1936 entstanden sein dürfte. Beim vorliegenden
                              Korrespondenzstück dürfte es sich um das mit »1 Brief entnommen« bezeichnete handeln, da bislang weder für die an 
                                 Paul Marx\pwindex{Marx, Paul 21.\,7.\,1879 Wien – 30.\,10.\,1956 ebd.@\textsc{Marx, Paul} (21.\,7.\,1879 Wien – 30.\,10.\,1956 ebd.), \emph{Regisseur, Schauspieler}|pw} noch für die an das Wolf-Museum Eisenstadt\orgindex{Landesmuseum Burgenland@Landesmuseum Burgenland|pw} übergebenen Briefe 
                                 eine spätere Existenz belegt werden konnte. Auf der Kopie findet sich folgende Aufschrift: »Present location of
                                    original of this letter is unknown. A xerox copy is in the
                                    Central Zionist Archives, Jerusalem, ref. HN III 33. (Copy, from
                                    the xerox copy, presented by Dr J. Wachten, Martin Buber
                                    Institute, Köln).« }
\buchAbdrucke{\weitereDrucke{Theodor Herzl: \emph{Briefe und
                        autobiographische Notizen 1866–1895}. Bearbeitet von Johannes Wachten in Zusammenarbeit mit Chaya Harel, Daisy Tycho und Manfred Winkler. Berlin, Frankfurt am Main, Wien: \emph{Propyläen} 1983, S. 532 (Briefe und Tagebücher. Herausgegeben von Alex Bein, Hermann Greive, Moshe Schaerf, Julius H. Schoeps und Johannes Wachten, 1).} }\toendnotes[C]{\smallbreak}
\pstart
           {\pb}\textcolor{gray}{\textbf{HOTEL {\kaufmannsund}
                           PENSION FROHNALP\oindex{Hotel {\kaufmannsund} Pension Frohnalp@\textbf{Hotel {\kaufmannsund} Pension Frohnalp}, \emph{Hotel}|pw}}}\pend
           
\pstart
           \textcolor{gray}{\textbf{MORSCHACH\oindex{Morschach@\textbf{Morschach}, \emph{Verwaltungsgebiet}|pw}}}\pend
           
\pstart
           \textcolor{gray}{\textbf{(Vierwaldstättersee\oindex{Vierwaldstättersee@\textbf{Vierwaldstättersee}, \emph{See}|pw})}}\pend
           
\pstart
           \textcolor{gray}{\textbf{AMBROS EBERLE\pwindex{Eberle, Ambros 9.\,5.\,1820 Einsiedeln – 9.\,1.\,1883 Schwyz@\textsc{Eberle, Ambros} (9.\,5.\,1820 Einsiedeln – 9.\,1.\,1883 Schwyz), \emph{Hotelier, Politiker}|pw}}}\pend
           
\pstart
           \textcolor{gray}{\textbf{Miteigenthümer}}\pend
           
\pstart
           \textcolor{gray}{\textbf{von}}\pend
           
\pstart
           \textcolor{gray}{\textbf{Hotel Axenstein\oindex{Hotel Axenstein@\textbf{Hotel Axenstein}, \emph{Hotel}|pw}}}\pend
           
\pstart{}Lieber Freund!\pend\vspace{0.5em}
\pstart
           Ihren lieben Brief bekam ich einen Moment vor der Abreise. Wir sind jetzt für ein
               paar Tage auf dem Axenstein\oindex{Axenstein@\textbf{Axenstein}, \emph{Ausflugsziel}|pw}, dann gehts nach
                  Oestreich\oindex{Österreich-Ungarn@\textbf{Österreich-Ungarn}|pw}.\pend
           
\pstart
           Aber wie so vieles hatte ich mir auch diese Urlaubstage anders vorgestellt.
               Wenigstens der Anfang ist übel. Kaum waren wir hier angelangt, so legte sich meine
                  Frau\pwindex{Herzl, Julie 1.\,2.\,1868 Budapest – 10.\,11.\,1907 Wien@\textsc{Herzl, Julie} (1.\,2.\,1868 Budapest – 10.\,11.\,1907 Wien)|pwv} mit heftiger
                  Halsentzündung{[}.{]} Noch in der Nacht musste der Arzt\pwindex{?? [Arzt in Brunnen] @\textsc{?? [Arzt in Brunnen]}|pwv} – mehr Bader – von Brunnen\oindex{Brunnen@\textbf{Brunnen}|pw} heraufgeholt werden.\pend
           
\pstart
           Heute gehts ihr etwas besser {\pb}immer noch zwischen 38°–39° Temparatur.
               Hals sehr belegt. Die Kinder\pwindex{Neumann, Margarethe 20.\,5.\,1893 Paris – 15.\,3.\,1943 Konzentrationslager Theresienstadt@\textsc{Neumann, Margarethe} (20.\,5.\,1893 Paris – 15.\,3.\,1943 Konzentrationslager Theresienstadt)|pwv}\pwindex{Herzl, Hans 10.\,6.\,1891 Wien – 14.\,9.\,1930 Bordeaux@\textsc{Herzl, Hans} (10.\,6.\,1891 Wien – 14.\,9.\,1930 Bordeaux)|pwv}\pwindex{Hüft, Pauline 29.\,3.\,1890 – 8.\,9.\,1930@\textsc{Hüft, Pauline} (29.\,3.\,1890 – 8.\,9.\,1930)|pwv} werden separirt u. ich sitze da u. pinsle \label{K_L03831-1v}\edtext{Höllenstein}{\lemma{\textnormal{\emph{Höllenstein}}}\Cendnote{\textnormal{Lapis infernalis, Silbernitrat, wirkt als Lösung antiseptisch
                  und adstringierend}}}\label{K_L03831-1}. Statt Axenstein\oindex{Axenstein@\textbf{Axenstein}, \emph{Ausflugsziel}|pw}
               Höllenstein.\pend
           
\pstart
           Aber die Luft ist wie man sagt balsamisch. Wenn man schon krank sein muss soll man es
               hier sein!\pend
           
\pstart
           Sobald ich nach Wien\oindex{Wien@\textbf{Wien}, \emph{Verwaltungsgebiet}|pw} komme hören Sies natürlich
               von Ihrem Hausmeister wenn Sie nicht zu Hause gewesen sein sollten.\pend
           
\pstart
           Herzlich Ihr{\\[\baselineskip]}\spacefill\mbox{Th Herzl}\pend
           \leftskip=0em{}
\pstart
           29 Juni 893\pend
           \selectlanguage{ngerman}\endnumbering\briefempfaengerindex{Schnitzler, Arthur@\textsc{Schnitzler, Arthur}!zzzHerzl, Theodor@\emph{von Theodor Herzl}!1893-06-291@{29. 6. 1893}|)be}\mylabel{L03831h}
\begin{anhang}
\end{anhang}\newcommand{\dateiname}{L03831}\newcommand{\titel}{Theodor Herzl an Arthur Schnitzler, 29. 6. 1893}\newcommand{\editorInnen}{Selma Jahnke und Martin Anton Müller}%% latex-leseansicht-abspann.tex
%% Abspann für die Leseansicht.
%% Der Schalter \ifkorrekturansicht ist bereits durch den Vorspann gesetzt.

%% latex-abspann.tex
%% Gemeinsamer Abspann für Korrekturansicht und Leseansicht.
%% Setzt den Schalter \ifkorrekturansicht voraus (gesetzt in den
%% einbindenden Dateien latex-korrekturansicht-abspann.tex bzw.
%% latex-leseansicht-abspann.tex).
%% ---------------------------------------------------------------

\normalsize

% Das esempio-Environment wird nur in der Leseansicht benötigt
\ifkorrekturansicht\else
\newenvironment{esempio}[3]%
{
    \vspace{1.5ex}
    \rlap{\underline{#1}}
    \par
    \setlength{\parindent}{0cm}
    \nopagebreak
    \leftskip=#2cm
    \rightskip=#3cm
}
{
    \par
}
\fi

\doendnotes{C}
\bigskip
\vfill

\clearpage

\footnotesize

\ifkorrekturansicht
  \lohead{\textsc{register}}
\fi

% theindex-Environment neu definieren ohne reledmac
\makeatletter
\renewenvironment{theindex}{%
  \ifkorrekturansicht
    \section*{\indexname}%
  \else
    \subsubsection*{Index der erwähnten Entitäten}%
  \fi
  \setlength{\parindent}{0pt}%
  \setlength{\parskip}{0pt plus 0.3pt}%
  \let\item\@idxitem
}{%
  \ifkorrekturansicht\clearpage\fi
}
\makeatother

\IfFileExists{\jobname-pw.ind}{\input{\jobname-pw.ind}}{}

% Quellenangabe nur in der Leseansicht
\ifkorrekturansicht\else
% Fallback-Definitionen, falls die .tex-Datei \titel etc. nicht gesetzt hat
\providecommand{\titel}{}
\providecommand{\editorInnen}{}
\providecommand{\dateiname}{\jobname}

\vspace{3cm}

\vfill

\footnotesize
\textsc{Quelle}: \titel. Herausgegeben von {\editorInnen}. In: \emph{Arthur Schnitzler: Briefwechsel mit Autorinnen und Autoren}.
 Digitale Edition, https://schnitzler-briefe.acdh.oeaw.ac.at/{\dateiname}.html (Stand \today)
\fi

\end{document}


