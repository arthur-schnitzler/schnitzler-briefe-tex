%% latex-korrekturansicht-vorspann.tex
%% Vorspann für die Korrekturansicht.
%% Lädt die gemeinsame Datei latex-vorspann.tex mit gesetztem Schalter.

\newif\ifkorrekturansicht
\korrekturansichttrue

\input{../tex-inputs/latex-vorspann}


\section[Theodor Herzl an Arthur Schnitzler, 29. 6. 1893]{L03831 Theodor Herzl an Arthur Schnitzler, 29. 6. 1893}
\nopagebreak\mylabel{L03831v}
\rehead{ }\normalsize\beginnumbering\briefempfaengerindex{Schnitzler, Arthur@\textsc{Schnitzler, Arthur}!zzzHerzl, Theodor@\emph{von Theodor Herzl}!1893-06-291@{29. 6. 1893}|(be}
\toendnotes[C]{\smallbreak\pagebreak[2]}\Standort{CUL, Schnitzler, B 39.}
\physDesc{Brief, 1 Blatt, 2 Seiten, 828 Zeichen
\newline{}Handschrift: , lateinische Kurrent}\toendnotes[C]{\smallbreak}
\pstart
           {\pb}\textcolor{gray}{\textbf{HOTEL {\kaufmannsund} PENSION FROHNALP\oindex{Hotel {\kaufmannsund} Pension Frohnalp@\textbf{Hotel {\kaufmannsund} Pension Frohnalp}, \emph{Hotel (K.HTL)}|pw}}}\pend
           
\pstart
           \textcolor{gray}{\textbf{MORSCHACH\oindex{Morschach@\textbf{Morschach}, \emph{A.ADM3}|pw}}}\pend
           
\pstart
           \textcolor{gray}{\textbf{(Vierwaldstättersee\oindex{Vierwaldstaettersee@\textbf{Vierwaldstättersee}, \emph{See (N.SEE)}|pw})}}\pend
           
\pstart
           \textcolor{gray}{\textbf{
                     AMBROS EBERLE
                  \pwindex{Eberle, Ambros 1820-05-09 – 1883-01-09@\textsc{Eberle, Ambros} (1820-05-09 – 1883-01-09), \emph{Hotelier/Hotelière, Politiker/Politikerin}|pw}}}\pend
           
\pstart
           \textcolor{gray}{\textbf{Miteigenthümer}}\pend
           
\pstart
           \textcolor{gray}{\textbf{von}}\pend
           
\pstart
           \textcolor{gray}{\textbf{
                  Hotel Axenstein
                \oindex{Hotel Axenstein@\textbf{Hotel Axenstein}, \emph{Hotel (K.HTL)}|pw}}}\pend
           
\pstart{}Lieber Freund!\pend\vspace{0.5em}
\pstart
           Ihren lieben Brief bekam ich
      einen Moment vor der Abreise.
      Wir sind jetzt für ein paar
      Tage auf dem Axenstein\oindex{Axenstein@\textbf{Axenstein}, \emph{Ausflugsziel}|pw}, dann
               gehts nach Oestreich\oindex{Oesterreich-Ungarn@\textbf{Österreich-Ungarn}, \emph{Land (A.LND)}|pw}.
      \pend
           
\pstart
           Aber wie so vieles hatte ich
      mir auch diese Urlaubstage
      anders vorgestellt. Wenigstens
      der Anfang ist übel. Kaum
      waren wir hier angelangt,
      so legte sich meine Frau\pwindex{Herzl, Julie 01.02.1868 – 10.11.1907@\textsc{Herzl, Julie} (01.02.1868 – 10.11.1907)|pwv}
      mit heftiger Halsentzündung{[}.{]}
      Noch in der Nacht musste der
               Arzt\pwindex{?? [Arzt in Brunnen] @\textsc{?? [Arzt in Brunnen]}|pwv} – mehr Bader – von Brunnen\oindex{Brunnen@\textbf{Brunnen}, \emph{P.PPL}|pw}
      heraufgeholt werden.\pend
           
\pstart
           Heute gehts ihr etwas besser {\pb}immer noch zwischen 38°–39°
         Temparatur. Hals sehr belegt. Die Kinder\pwindex{Neumann, Margarethe 20.05.1893 – 15.03.1943@\textsc{Neumann, Margarethe} (20.05.1893 – 15.03.1943)|pwv}\pwindex{Herzl, Hans 10.06.1891 – 14.09.1930@\textsc{Herzl, Hans} (10.06.1891 – 14.09.1930)|pwv}\pwindex{Hueft, Pauline 1890-03-29 – 1930-09-08@\textsc{Hüft, Pauline} (1890-03-29 – 1930-09-08)|pwv} werden separirt
      u. ich sitze da u. pinsle
         \label{K_L03831-1v}\edtext{Höllenstein}{\lemma{\textnormal{\emph{Höllenstein}}}\Cendnote{\textnormal{Lapis infernalis, Silbernitrat, wirkt als Lösung antiseptisch und adstringierend}}}\label{K_L03831-1}. Statt Axenstein\oindex{Axenstein@\textbf{Axenstein}, \emph{Ausflugsziel}|pw}
      Höllenstein.\pend
           
\pstart
           Aber die Luft ist wie man
      sagt balsamisch. Wenn man
      schon krank sein muss soll man
      es hier sein!\pend
           
\pstart
           Sobald ich nach Wien\oindex{Wien@\textbf{Wien}, \emph{A.ADM2}|pw} komme
      hören Sies natürlich von
      Ihrem Hausmeister wenn Sie
      nicht zu Hause gewesen sein
      sollten.\pend
           
\pstart
           Herzlich Ihr{\\[\baselineskip]}\spacefill\mbox{Th Herzl}\pend
           \leftskip=0em{}
\pstart
           29 Juni 893\pend
           \selectlanguage{ngerman}\endnumbering\briefempfaengerindex{Schnitzler, Arthur@\textsc{Schnitzler, Arthur}!zzzHerzl, Theodor@\emph{von Theodor Herzl}!1893-06-291@{29. 6. 1893}|)be}\mylabel{L03831h}
\begin{anhang}
\end{anhang}\normalsize

\doendnotes{C}
\bigskip
\vfill

\clearpage

\footnotesize

\lohead{\textsc{register}}

% Definiere theindex-Environment komplett neu ohne reledmac
\makeatletter
\renewenvironment{theindex}{%
  \section*{\indexname}%
  \setlength{\parindent}{0pt}%
  \setlength{\parskip}{0pt plus 0.3pt}%
  \let\item\@idxitem
}{%
  \clearpage
}
\makeatother

\IfFileExists{\jobname-pw.ind}{\input{\jobname-pw.ind}}{}

\end{document}

      