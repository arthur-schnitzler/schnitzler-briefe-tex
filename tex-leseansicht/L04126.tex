%% latex-leseansicht-vorspann.tex
%% Vorspann für die Leseansicht.
%% Lädt die gemeinsame Datei latex-vorspann.tex mit nicht gesetztem Schalter.

\newif\ifkorrekturansicht
\korrekturansichtfalse

\input{../tex-inputs/latex-vorspann}


\section[Arthur Schnitzler an Gustav Schwarzkopf, 21. 7. 1898]{L04126 Arthur Schnitzler an Gustav Schwarzkopf, 21. 7. 1898}
\nopagebreak\mylabel{L04126v}
\rehead{ }\normalsize\beginnumbering\briefempfaengerindex{Schwarzkopf, Gustav@\textsc{Schwarzkopf, Gustav}!zzzSchnitzler, Arthur@\emph{von Arthur Schnitzler}!1898-07-211@{21. 7. 1898}|(be}
\toendnotes[C]{\smallbreak\pagebreak[2]}
\correspDesc{Versand  durch Arthur Schnitzler am 21. 7. 1898 in Bad Gastein
\newline{}Erhalt  durch Gustav Schwarzkopf am 22. 7. 1898 in Wien}\toendnotes[C]{\smallbreak}
\Standort{CUL, Schnitzler, B 96.}
\physDesc{Postkarte, 540 Zeichen
\newline{}Handschrift: Bleistift, deutsche Kurrent
\newline{}Versand: Stempel: »\nobreak{}\oindex{Bad Gastein@\textbf{Bad Gastein}, \emph{Hauptstadt}|pwk}Badgastein, 21/7 98, 5–A\nobreak{}«.  
\newline{}Zusatz: mit Bleistift auf der Adressseite von unbekannter Hand Vermerk: »\noindent{}9.20{ / }1.07{ / }1.08{ / }4.22« }\toendnotes[C]{\smallbreak}\pstart{}{\pb}Herrn \textsc{Gustav Schwarzkopf}\pend{}\pstart{}Wien\oindex{Wien@\textbf{Wien}, \emph{Verwaltungsgebiet}|pw}\pend{}\pstart{}\textsc{I. Tiefer Graben 23}\oindex{Wien@\textbf{Wien}!I., Innere Stadt@\textbf{I., Innere Stadt}!Tiefer Graben 23@\textbf{Tiefer Graben 23}, \emph{Wohngebäude}|pw}.\pend{}{\bigskip}\vspace{1em}
\pstart
           {\pb}\textsc{Gastein\oindex{Bad Gastein@\textbf{Bad Gastein}, \emph{Hauptstadt}|pw}}, 21. 7. 98\pend
           \vspace{0.5em}
\pstart
           Lieber Guſtav, wenn Sie am 27. nach Salzburg\oindex{Salzburg@\textbf{Salzburg}, \emph{Verwaltungsgebiet}|pw} ko{\geminationm}en,{ }ſo iſt das ſo
               vortrefflich als möglich – ich bleibe da{\geminationn} noch 2–3 Tage
               dort und ich verſpreche Ihnen, dſs Sie von meinem Rad nicht das geringſte zu leiden
               haben werden. Ich will bis Montag (25.) hier bleiben; (bitte noch um ein
               Wort hieher), fahre \textsc{per} Rad nach Salzb.\oindex{Salzburg@\textbf{Salzburg}, \emph{Verwaltungsgebiet}|pw}, wo ich am 26. Abd ſein werde. Hotel
               erfahren Sie noch – ich möchte das Electrizitätshotel\oindex{Hotel Bristol Salzburg@\textbf{Hotel Bristol Salzburg}, \emph{Hotel}|pw} verſuchen, das billig und angenehm ſein ſoll.\pend
           
\pstart
           Von Herzen{\\[\baselineskip]} Ihr \spacefill\mbox{Arth}\pend
           \leftskip=0em{}
\pstart
           \noindent{}\label{T_L04126-1v}\edtext{Beſte Grüße an Mama\pwindex{Schnitzler, Louise 8.\,7.\,1840 Kőszeg – 9.\,9.\,1911 Wien@\textsc{Schnitzler, Louise} (8.\,7.\,1840 Kőszeg – 9.\,9.\,1911 Wien)|pwv}.}{\lemma{\textnormal{\emph{Beste Grüße an Mama.}}}\Cendnote{\textnormal{Entlang der oberen Blattkante, verkehrt zum restlichen Text.}}}\label{T_L04126-1}\pend
           \selectlanguage{ngerman}\endnumbering\briefempfaengerindex{Schwarzkopf, Gustav@\textsc{Schwarzkopf, Gustav}!zzzSchnitzler, Arthur@\emph{von Arthur Schnitzler}!1898-07-211@{21. 7. 1898}|)be}\mylabel{L04126h}
\begin{anhang}
\end{anhang}\newcommand{\dateiname}{L04126}\newcommand{\titel}{Arthur Schnitzler an Gustav Schwarzkopf, 21. 7. 1898}\newcommand{\editorInnen}{Herausgegeben von Jahnke, SelmaMüller, Martin Anton}%% latex-leseansicht-abspann.tex
%% Abspann für die Leseansicht.
%% Der Schalter \ifkorrekturansicht ist bereits durch den Vorspann gesetzt.

%% latex-abspann.tex
%% Gemeinsamer Abspann für Korrekturansicht und Leseansicht.
%% Setzt den Schalter \ifkorrekturansicht voraus (gesetzt in den
%% einbindenden Dateien latex-korrekturansicht-abspann.tex bzw.
%% latex-leseansicht-abspann.tex).
%% ---------------------------------------------------------------

\normalsize

% Das esempio-Environment wird nur in der Leseansicht benötigt
\ifkorrekturansicht\else
\newenvironment{esempio}[3]%
{
    \vspace{1.5ex}
    \rlap{\underline{#1}}
    \par
    \setlength{\parindent}{0cm}
    \nopagebreak
    \leftskip=#2cm
    \rightskip=#3cm
}
{
    \par
}
\fi

\doendnotes{C}
\bigskip
\vfill

\clearpage

\footnotesize

\ifkorrekturansicht
  \lohead{\textsc{register}}
\fi

% theindex-Environment neu definieren ohne reledmac
\makeatletter
\renewenvironment{theindex}{%
  \ifkorrekturansicht
    \section*{\indexname}%
  \else
    \subsubsection*{Index der erwähnten Entitäten}%
  \fi
  \setlength{\parindent}{0pt}%
  \setlength{\parskip}{0pt plus 0.3pt}%
  \let\item\@idxitem
}{%
  \ifkorrekturansicht\clearpage\fi
}
\makeatother

\IfFileExists{\jobname-pw.ind}{\input{\jobname-pw.ind}}{}

% Quellenangabe nur in der Leseansicht
\ifkorrekturansicht\else
% Fallback-Definitionen, falls die .tex-Datei \titel etc. nicht gesetzt hat
\providecommand{\titel}{}
\providecommand{\editorInnen}{}
\providecommand{\dateiname}{\jobname}

\vspace{3cm}

\vfill

\footnotesize
\textsc{Quelle}: \titel. Herausgegeben von {\editorInnen}. In: \emph{Arthur Schnitzler: Briefwechsel mit Autorinnen und Autoren}.
 Digitale Edition, https://schnitzler-briefe.acdh.oeaw.ac.at/{\dateiname}.html (Stand \today)
\fi

\end{document}


