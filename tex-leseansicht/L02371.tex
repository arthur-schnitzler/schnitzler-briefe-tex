%% latex-korrekturansicht-vorspann.tex
%% Vorspann für die Korrekturansicht.
%% Lädt die gemeinsame Datei latex-vorspann.tex mit gesetztem Schalter.

\newif\ifkorrekturansicht
\korrekturansichttrue

\input{../tex-inputs/latex-vorspann}


\section[Olga Schnitzler an Hermann Bahr, 16. 12. 1921]{L02371 Olga Schnitzler an Hermann Bahr, 16. 12. 1921}
\nopagebreak\mylabel{L02371v}
\rehead{ }\normalsize\beginnumbering\briefempfaengerindex{Bahr, Hermann@\textsc{Bahr, Hermann}!zzzSchnitzler, Olga@\emph{von Olga Schnitzler}!1921-12-161@{16. 12. 1921}|(be}
\toendnotes[C]{\smallbreak\pagebreak[2]}\Standort{TMW, HS AM 69560 Ba.}
\physDesc{Brief, 1 Blatt, 2 Seiten, 743 Zeichen
\newline{}Handschrift: schwarze Tinte, lateinische Kurrent}
\buchAbdrucke{\weitereDrucke{1) Arthur Schnitzler: \emph{The Letters of Arthur Schnitzler to Hermann Bahr}. Chapel Hill: \emph{The University of North Carolina Press} 1978, S. 116.} \weitereDrucke{2) Hermann Bahr, Arthur Schnitzler: \emph{Briefwechsel, Aufzeichnungen, Dokumente (1891–1931)}. Göttingen: \emph{Wallstein} 2018, S. 545.} }\toendnotes[C]{\smallbreak}
\pstart{}{\pb}Sehr verehrter
                  lieber Herr Bahr,\pend\vspace{0.5em}
\pstart
           schon längst wollt ich mich wieder bei Ihnen melden. Aber ich hatte Besuch, – und nun
               seh ich Wien\oindex{Wien@\textbf{Wien}, \emph{A.ADM2}|pw}er Gesichter auftauchen und da denk
               ich, Sie werden keine ruhigen Tage haben, – und wage schon gar nichts für mich zu
               erbitten.\pend
           
\pstart
           Dem Arthur hab ich von den beiden Spaziergängen
               mit Ihnen berichtet, daraufhin schrieb er mir neulich eine Menge schöner Dinge über
               Sie und nun fragt er immer nach Ihnen, – ich wünschte so sehr – er würde Ihnen einmal
               in einer guten Stunde begegnen. Von allen Menschen, die ich kenne, glaub ich, sind
               Sie der Einzige, der befreiend auf ihn wirken könnte.\pend
           
\pstart
           Meine Kinder\pwindex{Cappellini, Lili 13.09.1909 – 26.07.1928@\textsc{Cappellini, Lili} (13.09.1909 – 26.07.1928)|pwv}\pwindex{Schnitzler, Heinrich 09.08.1902 – 12.07.1982@\textsc{Schnitzler, Heinrich} (09.08.1902 – 12.07.1982), \emph{Regisseur/Regisseurin, Schauspieler/Schauspielerin}|pwv} kommen
               zu Weihnachten hieher zu mir.\pend
           
\pstart
           {\pb}Ich wünsche Ihnen gute
               und frohe Tage! \pend
           
\pstart
           Von Herzen ergeben{\\[\baselineskip]}Ihre{\\[\baselineskip]}\spacefill\mbox{Olga Schnitzler.}\pend
           \leftskip=0em{}
\pstart
           16. Dec. 21.\pend
           \selectlanguage{ngerman}\endnumbering\briefempfaengerindex{Bahr, Hermann@\textsc{Bahr, Hermann}!zzzSchnitzler, Olga@\emph{von Olga Schnitzler}!1921-12-161@{16. 12. 1921}|)be}\mylabel{L02371h}  \normalsize

\doendnotes{C}
\bigskip
\vfill

\clearpage

\footnotesize

\lohead{\textsc{register}}

% Definiere theindex-Environment komplett neu ohne reledmac
\makeatletter
\renewenvironment{theindex}{%
  \section*{\indexname}%
  \setlength{\parindent}{0pt}%
  \setlength{\parskip}{0pt plus 0.3pt}%
  \let\item\@idxitem
}{%
  \clearpage
}
\makeatother

\IfFileExists{\jobname-pw.ind}{\input{\jobname-pw.ind}}{}

\end{document}

      