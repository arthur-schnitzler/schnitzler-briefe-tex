%% latex-korrekturansicht-vorspann.tex
%% Vorspann für die Korrekturansicht.
%% Lädt die gemeinsame Datei latex-vorspann.tex mit gesetztem Schalter.

\newif\ifkorrekturansicht
\korrekturansichttrue

\input{../tex-inputs/latex-vorspann}


\section[Arthur Schnitzler an Therese Rie-Andro, 12. 2. 1912]{L02574 Arthur Schnitzler an Therese Rie-Andro, 12. 2. 1912}
\nopagebreak\mylabel{L02574v}
\rehead{ }\normalsize\beginnumbering\briefempfaengerindex{Rie, Therese@\textsc{Rie, Therese}!zzzSchnitzler, Arthur@\emph{von Arthur Schnitzler}!1912-02-122@{12. 2. 1912}|(be}
\toendnotes[C]{\smallbreak\pagebreak[2]}\Standort{DLA, A:Schnitzler, HS1985.1.253.}
\physDesc{Brief, Durchschlag1 Blatt, 2 Seiten, 1490 Zeichen
\newline{}Schreibmaschine
\newline{}Handschrift Arthur Schnitzler: roter Buntstift, lateinische Kurrent (\noindent{}Beschriftung mit »Andro« in der linken, mit
                                    »Ri« in rechten oberen Ecke. Oberhalb von
                                    »musikalische Legende« der Name des Werks:
                                       »(Palestrina\pwindex{Palestrina. Musikalische Legende in drei Akten@\emph{Palestrina. Musikalische Legende in drei Akten}|pw})« und zwei Unterstreichungen)
\newline{}Handschrift Schreibkraft: roter Buntstift, lateinische Kurrent (\noindent{}in der rechten oberen Ecke Vermerk, dass es sich um einen
                                 Durchschlag (Kopie) handelt: »K«)}
\buchAbdrucke{\weitereDrucke{Arthur Schnitzler: \emph{Briefe 1875–1912}. Frankfurt am Main: \emph{S. Fischer} 1981, S. 690–691.} }\toendnotes[C]{\smallbreak}
\pstart
           \centering{}{\pb}12. \strikeout{1}2. 1912.\pend
           
\pstart\center{}Sehr verehrte Frau.\pend\vspace{0.5em}
\pstart
           Die musikalische Legende\pwindex{Palestrina. Musikalische Legende in drei Akten@\emph{Palestrina. Musikalische Legende in drei Akten}|pwv} von
                  Hans Pfitzner\pwindex{Pfitzner, Hans 05.05.1869 – 22.05.1949@\textsc{Pfitzner, Hans} (05.05.1869 – 22.05.1949), \emph{Komponist/Komponistin}|pw} habe ich mit grösstem
               Interesse gelesen; als Grundlage für musikalische Bearbeitung scheint mir das Buch\pwindex{Palestrina. Musikalische Legende in drei Akten@\emph{Palestrina. Musikalische Legende in drei Akten}|pwv} sehr glücklich entworfen,
               aber \strikeout{auch} dichterische und theatralische Qualitäten
               selbständiger Art würden für Einfall und Durchführung auch bei solchen Lesern
               Anteilnahme werben, die nicht, wie es mir begegnet ist, schon während der Lektüre
               immerfort Musik mitklingen hörten, leider noch nicht die von Pfitzner\pwindex{Pfitzner, Hans 05.05.1869 – 22.05.1949@\textsc{Pfitzner, Hans} (05.05.1869 – 22.05.1949), \emph{Komponist/Komponistin}|pw}, der ich mich diesmal ganz besonders entgegenfreue.
               Vielleicht gebricht es dem zweiten
                  Akt\pwindex{Palestrina. Musikalische Legende in drei Akten@\emph{Palestrina. Musikalische Legende in drei Akten}|pwv} ein wenig an innerer Klarheit, doch denke ich mir wird die Musik hier
               manches zu entwirren imstande sein, was die Knappheit des Textes allzu dicht
               verknotet hat. Eine Kleinigkeit noch. Im letzten Akt\pwindex{Palestrina. Musikalische Legende in drei Akten@\emph{Palestrina. Musikalische Legende in drei Akten}|pwv} sollten die Leute auf der Strasse nicht »\label{K_L02574-1v}\edtext{Eviva}{\lemma{\textnormal{\emph{Eviva}}}\Cendnote{\textnormal{Das monierte Detail wurde von Pfitzner\pwindex{Pfitzner, Hans 05.05.1869 – 22.05.1949@\textsc{Pfitzner, Hans} (05.05.1869 – 22.05.1949), \emph{Komponist/Komponistin}|pwk} nicht geändert.}}}\label{K_L02574-1}!« rufen; man muss ja
               annehmen, dass das Ganze aus dem Italienischen\oindex{Italien@\textbf{Italien}, \emph{A.PCLI}|pw}
               ins Deutsche über{\pb}tragen ist und so wirkt es etwas unlogisch,
               dass gerade dieses eine populäre Wort italienisch\oindex{Italien@\textbf{Italien}, \emph{A.PCLI}|pw} stehen geblieben ist.\pend
           
\pstart
           Bitte, verehrte Frau, Hans Pfitzner\pwindex{Pfitzner, Hans 05.05.1869 – 22.05.1949@\textsc{Pfitzner, Hans} (05.05.1869 – 22.05.1949), \emph{Komponist/Komponistin}|pw} in meinem
               Namen für sein Vertrauen aufs Herzlichste zu danken{[}.{]} Ich hoffe
               es bald persönlich tun zu können, da er ja im Frühjahr nach Wien\oindex{Wien@\textbf{Wien}, \emph{A.ADM2}|pw} kommen dürfte. Von Ihnen hoffe ich bald wieder etwas zu
               lesen; ich irre mich ja nicht, wenn ich Sie mit der Verfasserin eines Novellenbuches
               (hiess es nicht die »Augen des Hy\strikeout{e}ronimus\pwindex{Augen des Hieronymus@\emph{Die Augen des Hieronymus}|pw}«) identifiziere, das ich vor einer Reihe von Jahren mit
               Vergnügen kennen gelernt habe.\pend
           \pstart Mit verbildlichem Gruss\pend{}{\vspace{1\baselineskip}}
\pstart
           \noindent{}Frau L. Andro, Wien\oindex{Wien@\textbf{Wien}, \emph{A.ADM2}|pw}.\pend
           \selectlanguage{ngerman}\endnumbering\briefempfaengerindex{Rie, Therese@\textsc{Rie, Therese}!zzzSchnitzler, Arthur@\emph{von Arthur Schnitzler}!1912-02-122@{12. 2. 1912}|)be}\mylabel{L02574h}  \normalsize

\doendnotes{C}
\bigskip
\vfill

\clearpage

\footnotesize

\lohead{\textsc{register}}

% Definiere theindex-Environment komplett neu ohne reledmac
\makeatletter
\renewenvironment{theindex}{%
  \section*{\indexname}%
  \setlength{\parindent}{0pt}%
  \setlength{\parskip}{0pt plus 0.3pt}%
  \let\item\@idxitem
}{%
  \clearpage
}
\makeatother

\IfFileExists{\jobname-pw.ind}{\input{\jobname-pw.ind}}{}

\end{document}

      