%% latex-leseansicht-vorspann.tex
%% Vorspann für die Leseansicht.
%% Lädt die gemeinsame Datei latex-vorspann.tex mit nicht gesetztem Schalter.

\newif\ifkorrekturansicht
\korrekturansichtfalse

\input{../tex-inputs/latex-vorspann}


               \section[Arthur Schnitzler an Therese Rie-Andro, 12. 2. 1912]{ Arthur Schnitzler an Therese Rie-Andro, 12. 2. 1912}\nopagebreak\mylabel{v}\rehead{ }\begin{ledgroupsized}[t]{13cm}\normalsize\beginnumbering\briefempfaengerindex{Rie, Therese@\textsc{Rie, Therese}!zzzSchnitzler, Arthur@\emph{von Arthur Schnitzler}!1912-02-122@{12. 2. 1912}|(be} \toendnotes[C]{\smallbreak\pagebreak[2]} \Standort{DLA, A:Schnitzler, HS1985.1.253.}
\physDesc{Brief, 1 Blatt, 2 Seiten, maschineller Durchschlag
\newline{}Schreibmaschine
\newline{}Handschrift Arthur Schnitzler: roter Buntstift, lateinische Kurrent (\noindent{}Beschriftung mit »Andro« in der linken,
                                            mit »Ri« in rechten oberen Ecke. Oberhalb
                                            von »musikalische Legende« der Name des
                                            Werks: »(Palestrina\pwindex{Pfitzner, Hans 05.05.1869 – 22.05.1949@\textsc{Pfitzner, Hans} (05.05.1869 – 22.05.1949), \emph{Komponist}!Palestrina. Musikalische Legende in drei Akten1912@\strich\emph{Palestrina. Musikalische Legende in drei Akten} {[}1912{]}|pw})« und zwei
                                            Unterstreichungen)\newline{}Handschrift  : roter Buntstift, lateinische Kurrent (\noindent{}in der rechten oberen Ecke Vermerk, dass es sich um einen
                                            Durchschlag (Kopie) handelt: »K«)}\buchAbdrucke{\weitereDrucke{Arthur Schnitzler: \emph{Briefe 1875–1912}. Hg. Therese Nickl und Heinrich Schnitzler. Frankfurt am Main: \emph{S. Fischer} 1981, S. 690–691.} }\toendnotes[C]{\smallbreak}\pstart
           \centering{}{\pb}12. \strikeout{1}2. 1912.\pend
           \pstart\center{}Sehr verehrte Frau.\pend\pstart
           Die musikalische Legende\pwindex{Pfitzner, Hans 05.05.1869 – 22.05.1949@\textsc{Pfitzner, Hans} (05.05.1869 – 22.05.1949), \emph{Komponist}!Palestrina. Musikalische Legende in drei Akten1912@\strich\emph{Palestrina. Musikalische Legende in drei Akten} {[}1912{]}|pwv} von
                        Hans Pfitzner\pwindex{Pfitzner, Hans 05.05.1869 – 22.05.1949@\textsc{Pfitzner, Hans} (05.05.1869 – 22.05.1949), \emph{Komponist}|pw} habe ich mit grösstem
                    Interesse gelesen; als Grundlage für musikalische Bearbeitung scheint mir das
                        Buch\pwindex{Pfitzner, Hans 05.05.1869 – 22.05.1949@\textsc{Pfitzner, Hans} (05.05.1869 – 22.05.1949), \emph{Komponist}!Palestrina. Musikalische Legende in drei Akten1912@\strich\emph{Palestrina. Musikalische Legende in drei Akten} {[}1912{]}|pwv} sehr glücklich
                    entworfen, aber \strikeout{auch} dichterische und
                    theatralische Qualitäten selbständiger Art würden für Einfall und Durchführung
                    auch bei solchen Lesern Anteilnahme werben, die nicht, wie es mir begegnet ist,
                    schon während der Lektüre immerfort Musik mitklingen hörten, leider noch nicht
                    die von Pfitzner\pwindex{Pfitzner, Hans 05.05.1869 – 22.05.1949@\textsc{Pfitzner, Hans} (05.05.1869 – 22.05.1949), \emph{Komponist}|pw}, der ich mich diesmal ganz
                    besonders entgegenfreue. Vielleicht gebricht es dem zweiten Akt\pwindex{Pfitzner, Hans 05.05.1869 – 22.05.1949@\textsc{Pfitzner, Hans} (05.05.1869 – 22.05.1949), \emph{Komponist}!Palestrina. Musikalische Legende in drei Akten1912@\strich\emph{Palestrina. Musikalische Legende in drei Akten} {[}1912{]}|pwv} ein wenig an innerer Klarheit, doch denke
                    ich mir wird die Musik hier manches zu entwirren imstande sein, was die
                    Knappheit des Textes allzu dicht verknotet hat. Eine Kleinigkeit noch. Im
                    letzten Akt\pwindex{Pfitzner, Hans 05.05.1869 – 22.05.1949@\textsc{Pfitzner, Hans} (05.05.1869 – 22.05.1949), \emph{Komponist}!Palestrina. Musikalische Legende in drei Akten1912@\strich\emph{Palestrina. Musikalische Legende in drei Akten} {[}1912{]}|pwv} sollten die
                    Leute auf der Strasse nicht »\label{K_L02574-1v}\edtext{Eviva}{\lemma{\textnormal{\emph{Eviva}}}\Cendnote{\textnormal{Das monierte Detail
                        wurde von Pfitzner\pwindex{Pfitzner, Hans 05.05.1869 – 22.05.1949@\textsc{Pfitzner, Hans} (05.05.1869 – 22.05.1949), \emph{Komponist}|pwk} nicht
                    geändert.}}}\label{K_L02574-1h}!« rufen; man muss ja annehmen, dass das Ganze aus dem Italienischen\oindex{Italien@\textbf{Italien}|pw} ins Deutsche über{\pb}tragen ist und so wirkt es etwas unlogisch, dass gerade
                    dieses eine populäre Wort italienisch\oindex{Italien@\textbf{Italien}|pw} stehen
                    geblieben ist.\pend
           \pstart
           Bitte, verehrte Frau, Hans Pfitzner\pwindex{Pfitzner, Hans 05.05.1869 – 22.05.1949@\textsc{Pfitzner, Hans} (05.05.1869 – 22.05.1949), \emph{Komponist}|pw} in meinem
                    Namen für sein Vertrauen aufs Herzlichste zu danken{[}.{]} Ich
                    hoffe es bald persönlich tun zu können, da er ja im Frühjahr nach Wien\oindex{Wien@\textbf{Wien}|pw} kommen dürfte. Von Ihnen hoffe ich bald
                    wieder etwas zu lesen; ich irre mich ja nicht, wenn ich Sie mit der Verfasserin
                    eines Novellenbuches (hiess es nicht die »Augen des
                            Hy\strikeout{e}ronimus\pwindex{Augen des Hieronymus1905@\emph{Die Augen des Hieronymus} {[}1905{]}|pw}«) identifiziere, das ich vor
                    einer Reihe von Jahren mit Vergnügen kennen gelernt habe.\pend
           \pstart Mit verbildlichem Gruss\pend{}{\bigskip}\pstart
           \noindent{}Frau L. Andro, Wien\oindex{Wien@\textbf{Wien}|pw}.\pend
           \endnumbering\briefempfaengerindex{Rie, Therese@\textsc{Rie, Therese}!zzzSchnitzler, Arthur@\emph{von Arthur Schnitzler}!1912-02-122@{12. 2. 1912}|)be}\mylabel{h}\end{ledgroupsized}  \newcommand{\dateiname}{L02574}\newcommand{\titel}{Arthur Schnitzler an Therese Rie-Andro, 12. 2. 1912}\newcommand{\editorInnen}{Martin Anton Müller und Gerd-Hermann Susen}%% latex-leseansicht-abspann.tex
%% Abspann für die Leseansicht.
%% Der Schalter \ifkorrekturansicht ist bereits durch den Vorspann gesetzt.

%% latex-abspann.tex
%% Gemeinsamer Abspann für Korrekturansicht und Leseansicht.
%% Setzt den Schalter \ifkorrekturansicht voraus (gesetzt in den
%% einbindenden Dateien latex-korrekturansicht-abspann.tex bzw.
%% latex-leseansicht-abspann.tex).
%% ---------------------------------------------------------------

\normalsize

% Das esempio-Environment wird nur in der Leseansicht benötigt
\ifkorrekturansicht\else
\newenvironment{esempio}[3]%
{
    \vspace{1.5ex}
    \rlap{\underline{#1}}
    \par
    \setlength{\parindent}{0cm}
    \nopagebreak
    \leftskip=#2cm
    \rightskip=#3cm
}
{
    \par
}
\fi

\doendnotes{C}
\bigskip
\vfill

\clearpage

\footnotesize

\ifkorrekturansicht
  \lohead{\textsc{register}}
\fi

% theindex-Environment neu definieren ohne reledmac
\makeatletter
\renewenvironment{theindex}{%
  \ifkorrekturansicht
    \section*{\indexname}%
  \else
    \subsubsection*{Index der erwähnten Entitäten}%
  \fi
  \setlength{\parindent}{0pt}%
  \setlength{\parskip}{0pt plus 0.3pt}%
  \let\item\@idxitem
}{%
  \ifkorrekturansicht\clearpage\fi
}
\makeatother

\IfFileExists{\jobname-pw.ind}{\input{\jobname-pw.ind}}{}

% Quellenangabe nur in der Leseansicht
\ifkorrekturansicht\else
% Fallback-Definitionen, falls die .tex-Datei \titel etc. nicht gesetzt hat
\providecommand{\titel}{}
\providecommand{\editorInnen}{}
\providecommand{\dateiname}{\jobname}

\vspace{3cm}

\vfill

\footnotesize
\textsc{Quelle}: \titel. Herausgegeben von {\editorInnen}. In: \emph{Arthur Schnitzler: Briefwechsel mit Autorinnen und Autoren}.
 Digitale Edition, https://schnitzler-briefe.acdh.oeaw.ac.at/{\dateiname}.html (Stand \today)
\fi

\end{document}


      