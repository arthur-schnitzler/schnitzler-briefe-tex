%% latex-leseansicht-vorspann.tex
%% Vorspann für die Leseansicht.
%% Lädt die gemeinsame Datei latex-vorspann.tex mit nicht gesetztem Schalter.

\newif\ifkorrekturansicht
\korrekturansichtfalse

\input{../tex-inputs/latex-vorspann}


\section[Richard Beer-Hofmann an Arthur Schnitzler, 15. 9. 1896]{L00591 Richard Beer-Hofmann an Arthur Schnitzler, 15. 9. 1896}
\nopagebreak\mylabel{L00591v}
\rehead{ }\normalsize\beginnumbering\briefempfaengerindex{Schnitzler, Arthur@\textsc{Schnitzler, Arthur}!zzzBeer-Hofmann, Richard@\emph{von Richard Beer-Hofmann}!1896-09-151@{15. 9. 1896}|(be}
\toendnotes[C]{\smallbreak\pagebreak[2]}
\correspDesc{Versand  durch Richard Beer-Hofmann am 15. 9. 1896 in Baden bei Wien
\newline{}Erhalt  durch Arthur Schnitzler im Zeitraum [16. 9. 1896
                  – 20. 9. 1896?] in Wien}\toendnotes[C]{\smallbreak}
\Standort{CUL, Schnitzler, B 8.}
\physDesc{Brief, 1 Blatt, 4 Seiten, 716 Zeichen
\newline{}Handschrift: blauer Buntstift, lateinische Kurrent
\newline{}Ordnung: mit Bleistift von unbekannter Hand nummeriert:
                                    »85« }
\buchAbdrucke{\weitereDrucke{Arthur Schnitzler, Richard Beer-Hofmann: \emph{Briefwechsel 1891–1931}. Herausgegeben von Konstanze Fliedl. Wien, Zürich: \emph{Europaverlag} 1992, S. 97.} }
\pstart
           {\pb}Baden\oindex{Baden bei Wien@\textbf{Baden bei Wien}, \emph{Hauptstadt}|pw}{ }15/IX 96\pend
           \vspace{0.5em}
\pstart
           Lieber Arthur, das schreib ich Ihnen im Park der jetzt um 10 Uhr Vorm. sehr leer und sehr schön ist\pend
           
\pstart
           Ich bin wahrscheinlich Donnerstag auf einige Stunden in Wien\oindex{Wien@\textbf{Wien}, \emph{Verwaltungsgebiet}|pw}. Wie ist denn jetzt Ihre normale Stundeneintheilung? – ohne
                  {\pb}Bindung–. Wissen Sie wieviel
               Exempl. vom »Kind\pwindex{Beer-Hofmann, Richard 11.\,7.\,1866 Wien – 26.\,9.\,1945 New York City@\textsc{Beer-Hofmann, Richard} (11.\,7.\,1866 Wien – 26.\,9.\,1945 New York City), \emph{Schriftsteller}!Kind@\strich\emph{Das Kind}|pw}« \uline{verkauft} wurden – (Freiex an mich, Recensionsex. etc. \uline{nicht} eingerechnet)?\pend
           
\pstart
           \uline{944} – (neunhundertvierundvierzig!) Räthselhaft wie
               viel Menschen sich das kaufen–? Nicht? Trotzdem {\pb}fehlen dem p. t. Zuchthäusler – wie
                  Brandes\pwindex{Brandes, Georg 4.\,2.\,1842 Kopenhagen – 19.\,2.\,1927 ebd.@\textsc{Brandes, Georg} (4.\,2.\,1842 Kopenhagen – 19.\,2.\,1927 ebd.)|pw} diese Herren nennt, noch 14 Mark
               und einige Pfennige zur Deckung der Kosten. Verstehn Sie das?\pend
           
\pstart
           Natürlich haben Paula\pwindex{Beer-Hofmann, Paula 25.\,2.\,1879 Wien – 30.\,10.\,1939 Zürich@\textsc{Beer-Hofmann, Paula} (25.\,2.\,1879 Wien – 30.\,10.\,1939 Zürich)|pw} und ich uns wieder
               lieber als {\pb}je, – das ist doch
               natürlich – oder \strikeout{an} einmal mehr gedreht
               unnatürlich?\pend
           
\pstart
           Herzlichst{\\[\baselineskip]}Ihr{\\[\baselineskip]}\spacefill\mbox{Richard}\pend
           \leftskip=0em{}\selectlanguage{ngerman}\endnumbering\briefempfaengerindex{Schnitzler, Arthur@\textsc{Schnitzler, Arthur}!zzzBeer-Hofmann, Richard@\emph{von Richard Beer-Hofmann}!1896-09-151@{15. 9. 1896}|)be}\mylabel{L00591h}  \newcommand{\dateiname}{L00591}\newcommand{\titel}{Richard Beer-Hofmann an Arthur Schnitzler, 15. 9. 1896}\newcommand{\editorInnen}{Martin Anton Müller und Gerd-Hermann Susen}%% latex-leseansicht-abspann.tex
%% Abspann für die Leseansicht.
%% Der Schalter \ifkorrekturansicht ist bereits durch den Vorspann gesetzt.

%% latex-abspann.tex
%% Gemeinsamer Abspann für Korrekturansicht und Leseansicht.
%% Setzt den Schalter \ifkorrekturansicht voraus (gesetzt in den
%% einbindenden Dateien latex-korrekturansicht-abspann.tex bzw.
%% latex-leseansicht-abspann.tex).
%% ---------------------------------------------------------------

\normalsize

% Das esempio-Environment wird nur in der Leseansicht benötigt
\ifkorrekturansicht\else
\newenvironment{esempio}[3]%
{
    \vspace{1.5ex}
    \rlap{\underline{#1}}
    \par
    \setlength{\parindent}{0cm}
    \nopagebreak
    \leftskip=#2cm
    \rightskip=#3cm
}
{
    \par
}
\fi

\doendnotes{C}
\bigskip
\vfill

\clearpage

\footnotesize

\ifkorrekturansicht
  \lohead{\textsc{register}}
\fi

% theindex-Environment neu definieren ohne reledmac
\makeatletter
\renewenvironment{theindex}{%
  \ifkorrekturansicht
    \section*{\indexname}%
  \else
    \subsubsection*{Index der erwähnten Entitäten}%
  \fi
  \setlength{\parindent}{0pt}%
  \setlength{\parskip}{0pt plus 0.3pt}%
  \let\item\@idxitem
}{%
  \ifkorrekturansicht\clearpage\fi
}
\makeatother

\IfFileExists{\jobname-pw.ind}{\input{\jobname-pw.ind}}{}

% Quellenangabe nur in der Leseansicht
\ifkorrekturansicht\else
% Fallback-Definitionen, falls die .tex-Datei \titel etc. nicht gesetzt hat
\providecommand{\titel}{}
\providecommand{\editorInnen}{}
\providecommand{\dateiname}{\jobname}

\vspace{3cm}

\vfill

\footnotesize
\textsc{Quelle}: \titel. Herausgegeben von {\editorInnen}. In: \emph{Arthur Schnitzler: Briefwechsel mit Autorinnen und Autoren}.
 Digitale Edition, https://schnitzler-briefe.acdh.oeaw.ac.at/{\dateiname}.html (Stand \today)
\fi

\end{document}


