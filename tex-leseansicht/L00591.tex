%% latex-korrekturansicht-vorspann.tex
%% Vorspann für die Korrekturansicht.
%% Lädt die gemeinsame Datei latex-vorspann.tex mit gesetztem Schalter.

\newif\ifkorrekturansicht
\korrekturansichttrue

\input{../tex-inputs/latex-vorspann}


\section[Richard Beer-Hofmann an Arthur Schnitzler, 15. 9. 1896]{L00591 Richard Beer-Hofmann an Arthur Schnitzler, 15. 9. 1896}
\nopagebreak\mylabel{L00591v}
\rehead{ }\normalsize\beginnumbering\briefempfaengerindex{Schnitzler, Arthur@\textsc{Schnitzler, Arthur}!zzzBeer-Hofmann, Richard@\emph{von Richard Beer-Hofmann}!1896-09-151@{15. 9. 1896}|(be}
\toendnotes[C]{\smallbreak\pagebreak[2]}\Standort{CUL, Schnitzler, B 8.}
\physDesc{Brief, 1 Blatt, 4 Seiten, 716 Zeichen
\newline{}Handschrift: blauer Buntstift, lateinische Kurrent
\newline{}Ordnung: mit Bleistift von unbekannter Hand nummeriert:
                                    »85« }
\buchAbdrucke{\weitereDrucke{Arthur Schnitzler, Richard Beer-Hofmann: \emph{Briefwechsel 1891–1931}. Wien, Zürich: \emph{Europaverlag} 1992, S. 97.} }
\pstart
           {\pb}Baden\oindex{Baden bei Wien@\textbf{Baden bei Wien}, \emph{P.PPLA3}|pw}{ }15/IX 96\pend
           \vspace{0.5em}
\pstart
           Lieber Arthur, das schreib ich Ihnen im Park der jetzt um 10
                  Uhr Vorm. sehr leer und sehr schön ist\pend
           
\pstart
           Ich bin wahrscheinlich Donnerstag auf einige Stunden in Wien\oindex{Wien@\textbf{Wien}, \emph{A.ADM2}|pw}. Wie ist denn jetzt Ihre normale Stundeneintheilung? – ohne
                  {\pb}Bindung–. Wissen Sie wieviel
               Exempl. vom »Kind\pwindex{Kind@\emph{Das Kind}|pw}« \uline{verkauft} wurden – (Freiex an mich, Recensionsex. etc. \uline{nicht} eingerechnet)?\pend
           
\pstart
           \uline{944} – (neunhundertvierundvierzig!) Räthselhaft wie
               viel Menschen sich das kaufen–? Nicht? Trotzdem {\pb}fehlen dem p. t. Zuchthäusler – wie
                  Brandes\pwindex{Brandes, Georg 04.02.1842 – 19.02.1927@\textsc{Brandes, Georg} (04.02.1842 – 19.02.1927)|pw} diese Herren nennt, noch 14 Mark
               und einige Pfennige zur Deckung der Kosten. Verstehn Sie das?\pend
           
\pstart
           Natürlich haben Paula\pwindex{Beer-Hofmann, Paula 25.02.1879 – 30.10.1939@\textsc{Beer-Hofmann, Paula} (25.02.1879 – 30.10.1939)|pw} und ich uns wieder
               lieber als {\pb}je, – das ist doch
               natürlich – oder \strikeout{an} einmal mehr gedreht
               unnatürlich?\pend
           
\pstart
           Herzlichst{\\[\baselineskip]}Ihr{\\[\baselineskip]}\spacefill\mbox{Richard}\pend
           \leftskip=0em{}\selectlanguage{ngerman}\endnumbering\briefempfaengerindex{Schnitzler, Arthur@\textsc{Schnitzler, Arthur}!zzzBeer-Hofmann, Richard@\emph{von Richard Beer-Hofmann}!1896-09-151@{15. 9. 1896}|)be}\mylabel{L00591h}  \normalsize

\doendnotes{C}
\bigskip
\vfill

\clearpage

\footnotesize

\lohead{\textsc{register}}

% Definiere theindex-Environment komplett neu ohne reledmac
\makeatletter
\renewenvironment{theindex}{%
  \section*{\indexname}%
  \setlength{\parindent}{0pt}%
  \setlength{\parskip}{0pt plus 0.3pt}%
  \let\item\@idxitem
}{%
  \clearpage
}
\makeatother

\IfFileExists{\jobname-pw.ind}{\input{\jobname-pw.ind}}{}

\end{document}

      