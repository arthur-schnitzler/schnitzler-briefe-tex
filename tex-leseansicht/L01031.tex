%% latex-leseansicht-vorspann.tex
%% Vorspann für die Leseansicht.
%% Lädt die gemeinsame Datei latex-vorspann.tex mit nicht gesetztem Schalter.

\newif\ifkorrekturansicht
\korrekturansichtfalse

\input{../tex-inputs/latex-vorspann}


\section[Arthur Schnitzler an Hugo von Hofmannsthal, 9. 4. 1900]{L01031 Arthur Schnitzler an Hugo von Hofmannsthal, 9. 4. 1900}
\nopagebreak\mylabel{L01031v}
\rehead{ }\normalsize\beginnumbering\briefempfaengerindex{Hofmannsthal, Hugo von@\textsc{Hofmannsthal, Hugo von}!zzzSchnitzler, Arthur@\emph{von Arthur Schnitzler}!1900-04-091@{9. 4. 1900}|(be}
\toendnotes[C]{\smallbreak\pagebreak[2]}
\correspDesc{Versand  durch Arthur Schnitzler am 9. 4. 1900 in Wien
\newline{}Erhalt  durch Hugo von Hofmannsthal im Zeitraum [10. 4. 1900
                  – 14. 4. 1900?] in Paris}\toendnotes[C]{\smallbreak}
\Standort{FDH, Hs-30885,92.}
\physDesc{Brief, 2 Blätter, 6 Seiten, 2513 Zeichen
\newline{}Handschrift: schwarze Tinte, deutsche Kurrent
\newline{}Ordnung: mit Bleistift Beide Blätter von Schnitzler mutmaßlich bei der Durchsicht der
                                 Korrespondenz 1929 mit »9/4 900« datiert }
\buchAbdrucke{\weitereDrucke{Hugo von Hofmannsthal, Arthur Schnitzler: \emph{Briefwechsel}. Herausgegeben von Therese Nickl und Heinrich Schnitzler. Frankfurt am Main: \emph{S. Fischer} 1964, S. 137–138.} }\toendnotes[C]{\smallbreak}
\pstart
           \raggedleft{}{\pb}9/4 900.\pend
           \vspace{0.5em}
\pstart
           mein lieber Hugo, heute Vormittag habe ich Ihren Papa\pwindex{Hofmannsthal, Hugo August von 21.\,12.\,1841 Wien – 8.\,12.\,1915 ebd.@\textsc{Hofmannsthal, Hugo August von} (21.\,12.\,1841 Wien – 8.\,12.\,1915 ebd.), \emph{Bankdirektor}|pwv} geſprochen, und ihn zu meiner Freude{ }ſo vortrefflich ausſehend und bei{ }ſo guter Sti{\geminationm}ung
               getroffen, wie nur einer{ }ſein kann, der morgen wieder aufſteht. Ich war geſtern
                  früh gleich nach meiner Ankunft bei Ihrer Mama\pwindex{Hofmannsthal, Anna von 27.\,1.\,1849 Wien – 22.\,3.\,1904 Sanatorium Fürth@\textsc{Hofmannsthal, Anna von} (27.\,1.\,1849 Wien – 22.\,3.\,1904 Sanatorium Fürth)|pwv} und fand{ }ſie{ }ſchon vollkommen beruhigt
               und hauptſächlich froh über die viele Sympathie von allen Seiten, die bei dieſer
               Gelegenheit{ }ſich ausſprach. {\pb}Soweit ich (ohne
               Unterſuchung) das ganze beurtheilen kann,{ }ſcheint mir eine organiſche Erkrankung \introOben{}(des Herzens)\introOben{} vollkommen ausgeſchloſſen; ich weiſs nicht
               einmal, ob es richtig iſt, von »Anfällen von Herzſchwäche« zu{ }ſprechen; mir ko{\geminationm}t der \textsc{vagus} als der{ }ſchuldige
               vor, und als ich heute vor Ihrer Mama\pwindex{Hofmannsthal, Anna von 27.\,1.\,1849 Wien – 22.\,3.\,1904 Sanatorium Fürth@\textsc{Hofmannsthal, Anna von} (27.\,1.\,1849 Wien – 22.\,3.\,1904 Sanatorium Fürth)|pwv} von \textsc{vagus} Neuroſe{ }ſprach,{ }ſagte{ }ſie, Dr. \textsc{Schandlbauer}\pwindex{Schandlbauer, Hans 12.\,1.\,1844 Krumbach – 25.\,5.\,1910 ebd.@\textsc{Schandlbauer, Hans} (12.\,1.\,1844 Krumbach – 25.\,5.\,1910 ebd.), \emph{Mediziner}|pw} habe dieſelbe Vermuthung ausgeſprochen. Jedenfalls dürfen Sie{ }ſo vergnügt und
               unbeſorgt weiterleben als vorher. Allerdings ko{\geminationm}t’s {\pb}mir{ }ſehr fraglich vor, daſs Ihre Mama\pwindex{Hofmannsthal, Anna von 27.\,1.\,1849 Wien – 22.\,3.\,1904 Sanatorium Fürth@\textsc{Hofmannsthal, Anna von} (27.\,1.\,1849 Wien – 22.\,3.\,1904 Sanatorium Fürth)|pwv}{ }ſich entſchließen wird, Ihren Papa\pwindex{Hofmannsthal, Hugo August von 21.\,12.\,1841 Wien – 8.\,12.\,1915 ebd.@\textsc{Hofmannsthal, Hugo August von} (21.\,12.\,1841 Wien – 8.\,12.\,1915 ebd.), \emph{Bankdirektor}|pwv} zu Ihnen nach Paris\oindex{Paris@\textbf{Paris}, \emph{Hauptstadt}|pw} fahren zu laſſen; das iſt ganz begreiflich. Ich höre immer
               wieder, von Richard\pwindex{Beer-Hofmann, Richard 11.\,7.\,1866 Wien – 26.\,9.\,1945 New York City@\textsc{Beer-Hofmann, Richard} (11.\,7.\,1866 Wien – 26.\,9.\,1945 New York City), \emph{Schriftsteller}|pw} und von Ihrer Mama\pwindex{Hofmannsthal, Anna von 27.\,1.\,1849 Wien – 22.\,3.\,1904 Sanatorium Fürth@\textsc{Hofmannsthal, Anna von} (27.\,1.\,1849 Wien – 22.\,3.\,1904 Sanatorium Fürth)|pwv}, dſs Sie{ }ſich{ }ſo wohl
               fühlen und mit Luſt arbeiten, und{ }ſo freue ich mich nicht nur auf Sie{ }ſondern auch
               auf das, was Sie mitbringen werden. Ich war auf meiner Reiſe eigentlich nur in den
               Stunden ziemlich gut dran, in denen ich geſchrieben habe; – {\pb}das Wetter war{ }ſelten{ }ſchön, nur in \textsc{Ragusa}\oindex{Dubrovnik@\textbf{Dubrovnik}|pw} 3 klare Tage, aber da wars für \textsc{Ragusa}\oindex{Dubrovnik@\textbf{Dubrovnik}|pw} und für Anfang April doch zu kühl. In Abbazia\oindex{Opatija@\textbf{Opatija}, \emph{Hauptstadt}|pw} hat es ununterbrochen gegoſſen; dort war ich viel mit
                  Georg Hirſchfeld\pwindex{Hirschfeld, Georg 11.\,2.\,1873 Berlin – 17.\,1.\,1942 München@\textsc{Hirschfeld, Georg} (11.\,2.\,1873 Berlin – 17.\,1.\,1942 München), \emph{Schriftsteller}|pw} zuſammen, zu dem ich neue
               Sympathie gefaſſt habe. Elly\pwindex{Petersen, Elly 26.\,2.\,1874 Berlin – 29.\,12.\,1965 München@\textsc{Petersen, Elly} (26.\,2.\,1874 Berlin – 29.\,12.\,1965 München), \emph{Schriftstellerin}|pw} liebe ich aber
               noch immer nicht. Es war mir auffallend, wie \label{K_L01031-1v}\edtext{viel ich auf meiner Reiſe}{\lemma{\textnormal{\emph{viel … Reise}}}\Cendnote{\textnormal{Er erwähnt mehrere davon im \emph{Tagebuch}\pwindex{Schnitzler, Arthur 15.\,5.\,1862 Wien – 21.\,10.\,1931 ebd.@\textsc{Schnitzler, Arthur} (15.\,5.\,1862 Wien – 21.\,10.\,1931 ebd.), \emph{Schriftsteller, Mediziner}!Tagebuch@\strich\emph{Tagebuch}|pwk} (1. 4. 1900, 4. 4. 1900, 5. 4. 1900, 6. 4. 1900).}}}\label{K_L01031-1} geträumt habe;{ }ſo lebhaft und bewegt wie
               nie, und meine Todte\pwindex{Reinhard, Marie 13.\,3.\,1871 Wien – 18.\,3.\,1899 ebd.@\textsc{Reinhard, Marie} (13.\,3.\,1871 Wien – 18.\,3.\,1899 ebd.), \emph{Gesangspädagogin}|pwv} iſt mir
               vier oder fünf Mal erſchienen. {\pb}Der{ }ſonderbarſte von
               allen Träumen war der, dſs ich träumte, \label{K_L01031-2v}\edtext{ich hätte drei Träume}{\lemma{\textnormal{\emph{ich hätte drei Träume}}}\Cendnote{\textnormal{Siehe A. S.: \emph{Tagebuch}, 6. 4. 1900.
               }}}\label{K_L01031-2} gehabt, die mir den Tod vorhergeſagt und erzählte jemandem dieſe 3 Träume,
               nach dem Aufwachen erinnerte ich mich nur an einen davon deutlich. – Ich bin noch
               immer an der langen Novelle\pwindex{Schnitzler, Arthur 15.\,5.\,1862 Wien – 21.\,10.\,1931 ebd.@\textsc{Schnitzler, Arthur} (15.\,5.\,1862 Wien – 21.\,10.\,1931 ebd.), \emph{Schriftsteller, Mediziner}!Frau Bertha Garlan. Roman@\strich\emph{Frau Bertha Garlan. Roman}|pwv},
               vor Oſtern wird{ }ſie doch fertig, dann dictir ich{ }ſie; fange aber gleich
               was neues an, entweder eine kurze Geſchichte\pwindex{Schnitzler, Arthur 15.\,5.\,1862 Wien – 21.\,10.\,1931 ebd.@\textsc{Schnitzler, Arthur} (15.\,5.\,1862 Wien – 21.\,10.\,1931 ebd.), \emph{Schriftsteller, Mediziner}!Erfolg@\strich\emph{Ein Erfolg}|pwuv} oder dieſes Sommerſtück\pwindex{Schnitzler, Arthur 15.\,5.\,1862 Wien – 21.\,10.\,1931 ebd.@\textsc{Schnitzler, Arthur} (15.\,5.\,1862 Wien – 21.\,10.\,1931 ebd.), \emph{Schriftsteller, Mediziner}!Im Spiel der Sommerlüfte. In drei Aufzügen@\strich\emph{Im Spiel der Sommerlüfte. In drei Aufzügen}|pwv}; – eigentlich hab ich ein Gefühl von
                  Unerſchöpf{\pb}lichkeit wie nie zuvor, aber es iſt mehr
               theoretiſch, – macht mich nicht beſonders glücklich. Ich empfinde meinen Verluſt{ }ſchwerer und{ }ſichrer als je.\pend
           
\pstart
           Leben Sie wohl und{ }ſchreiben Sie mir bald ein Wort.\pend
           \pstart Von Herzen Ihr \spacefill\mbox{Arthur.}\pend{}
\pstart
           \noindent{}Ich hoffe Sie haben meinen Brief \introOben{}(aus Wien\oindex{Wien@\textbf{Wien}, \emph{Verwaltungsgebiet}|pw})\introOben{} und auch die Karten aus Dalmatien\oindex{Dalmatien@\textbf{Dalmatien}, \emph{Ehemalige Region}|pw} bekommen.\pend
           
\pstart
           Wien\oindex{Wien@\textbf{Wien}, \emph{Verwaltungsgebiet}|pw}, 9. 4. 900.\pend
           \selectlanguage{ngerman}\endnumbering\briefempfaengerindex{Hofmannsthal, Hugo von@\textsc{Hofmannsthal, Hugo von}!zzzSchnitzler, Arthur@\emph{von Arthur Schnitzler}!1900-04-091@{9. 4. 1900}|)be}\mylabel{L01031h}  \newcommand{\dateiname}{L01031}\newcommand{\titel}{Arthur Schnitzler an Hugo von Hofmannsthal, 9. 4. 1900}\newcommand{\editorInnen}{Martin Anton Müller und Gerd-Hermann Susen}%% latex-leseansicht-abspann.tex
%% Abspann für die Leseansicht.
%% Der Schalter \ifkorrekturansicht ist bereits durch den Vorspann gesetzt.

%% latex-abspann.tex
%% Gemeinsamer Abspann für Korrekturansicht und Leseansicht.
%% Setzt den Schalter \ifkorrekturansicht voraus (gesetzt in den
%% einbindenden Dateien latex-korrekturansicht-abspann.tex bzw.
%% latex-leseansicht-abspann.tex).
%% ---------------------------------------------------------------

\normalsize

% Das esempio-Environment wird nur in der Leseansicht benötigt
\ifkorrekturansicht\else
\newenvironment{esempio}[3]%
{
    \vspace{1.5ex}
    \rlap{\underline{#1}}
    \par
    \setlength{\parindent}{0cm}
    \nopagebreak
    \leftskip=#2cm
    \rightskip=#3cm
}
{
    \par
}
\fi

\doendnotes{C}
\bigskip
\vfill

\clearpage

\footnotesize

\ifkorrekturansicht
  \lohead{\textsc{register}}
\fi

% theindex-Environment neu definieren ohne reledmac
\makeatletter
\renewenvironment{theindex}{%
  \ifkorrekturansicht
    \section*{\indexname}%
  \else
    \subsubsection*{Index der erwähnten Entitäten}%
  \fi
  \setlength{\parindent}{0pt}%
  \setlength{\parskip}{0pt plus 0.3pt}%
  \let\item\@idxitem
}{%
  \ifkorrekturansicht\clearpage\fi
}
\makeatother

\IfFileExists{\jobname-pw.ind}{\input{\jobname-pw.ind}}{}

% Quellenangabe nur in der Leseansicht
\ifkorrekturansicht\else
% Fallback-Definitionen, falls die .tex-Datei \titel etc. nicht gesetzt hat
\providecommand{\titel}{}
\providecommand{\editorInnen}{}
\providecommand{\dateiname}{\jobname}

\vspace{3cm}

\vfill

\footnotesize
\textsc{Quelle}: \titel. Herausgegeben von {\editorInnen}. In: \emph{Arthur Schnitzler: Briefwechsel mit Autorinnen und Autoren}.
 Digitale Edition, https://schnitzler-briefe.acdh.oeaw.ac.at/{\dateiname}.html (Stand \today)
\fi

\end{document}


