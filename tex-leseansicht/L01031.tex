%% latex-korrekturansicht-vorspann.tex
%% Vorspann für die Korrekturansicht.
%% Lädt die gemeinsame Datei latex-vorspann.tex mit gesetztem Schalter.

\newif\ifkorrekturansicht
\korrekturansichttrue

\input{../tex-inputs/latex-vorspann}


\section[Arthur Schnitzler an Hugo von Hofmannsthal, 9. 4. 1900]{L01031 Arthur Schnitzler an Hugo von Hofmannsthal, 9. 4. 1900}
\nopagebreak\mylabel{L01031v}
\rehead{ }\normalsize\beginnumbering\briefempfaengerindex{Hofmannsthal, Hugo von@\textsc{Hofmannsthal, Hugo von}!zzzSchnitzler, Arthur@\emph{von Arthur Schnitzler}!1900-04-091@{9. 4. 1900}|(be}
\toendnotes[C]{\smallbreak\pagebreak[2]}\Standort{FDH, Hs-30885,92.}
\physDesc{Brief, 2 Blätter, 6 Seiten, 2513 Zeichen
\newline{}Handschrift: schwarze Tinte, deutsche Kurrent
\newline{}Ordnung: mit Bleistift Beide Blätter von Schnitzler mutmaßlich bei der Durchsicht der
                                 Korrespondenz 1929 mit »9/4 900« datiert }
\buchAbdrucke{\weitereDrucke{Hugo von Hofmannsthal, Arthur Schnitzler: \emph{Briefwechsel}. Frankfurt am Main: \emph{S. Fischer} 1964, S. 137–138.} }\toendnotes[C]{\smallbreak}
\pstart
           \raggedleft{}{\pb}9/4 900. \pend
           \vspace{0.5em}
\pstart
           mein lieber Hugo, heute Vormittag habe ich Ihren Papa\pwindex{Hofmannsthal, Hugo August von 21.12.1841 – 08.12.1915@\textsc{Hofmannsthal, Hugo August von} (21.12.1841 – 08.12.1915), \emph{Bankdirektor/Bankdirektorin}|pwv} geſprochen, und ihn zu meiner Freude
               ſo vortrefflich ausſehend und bei ſo guter Sti{\geminationm}ung
               getroffen, wie nur einer ſein kann, der morgen wieder aufſteht. Ich war geſtern
                  früh gleich nach meiner Ankunft bei Ihrer Mama\pwindex{Hofmannsthal, Anna von 27.01.1849 – 22.03.1904@\textsc{Hofmannsthal, Anna von} (27.01.1849 – 22.03.1904)|pwv} und fand ſie ſchon vollkommen beruhigt
               und hauptſächlich froh über die viele Sympathie von allen Seiten, die bei dieſer
               Gelegenheit ſich ausſprach. {\pb}Soweit ich (ohne
               Unterſuchung) das ganze beurtheilen kann, ſcheint mir eine organiſche Erkrankung \introOben{}(des Herzens)\introOben{} vollkommen ausgeſchloſſen; ich weiſs nicht
               einmal, ob es richtig iſt, von »Anfällen von Herzſchwäche« zu ſprechen; mir ko{\geminationm}t der \textsc{vagus} als der ſchuldige
               vor, und als ich heute vor Ihrer Mama\pwindex{Hofmannsthal, Anna von 27.01.1849 – 22.03.1904@\textsc{Hofmannsthal, Anna von} (27.01.1849 – 22.03.1904)|pwv} von \textsc{vagus} Neuroſe ſprach, ſagte ſie, Dr. \textsc{Schandlbauer}\pwindex{Schandlbauer, Hans 12.1.1844 – 1910-05-25@\textsc{Schandlbauer, Hans} (12.1.1844 – 1910-05-25), \emph{Mediziner/Medizinerin}|pw} habe dieſelbe Vermuthung ausgeſprochen. Jedenfalls dürfen Sie ſo vergnügt und
               unbeſorgt weiterleben als vorher. Allerdings ko{\geminationm}t’s {\pb}mir ſehr fraglich vor, daſs Ihre Mama\pwindex{Hofmannsthal, Anna von 27.01.1849 – 22.03.1904@\textsc{Hofmannsthal, Anna von} (27.01.1849 – 22.03.1904)|pwv}{ }ſich entſchließen wird, Ihren Papa\pwindex{Hofmannsthal, Hugo August von 21.12.1841 – 08.12.1915@\textsc{Hofmannsthal, Hugo August von} (21.12.1841 – 08.12.1915), \emph{Bankdirektor/Bankdirektorin}|pwv} zu Ihnen nach Paris\oindex{Paris@\textbf{Paris}, \emph{P.PPLC}|pw} fahren zu laſſen; das iſt ganz begreiflich. Ich höre immer
               wieder, von Richard\pwindex{Beer-Hofmann, Richard 1866-07-11 – 1945-09-26@\textsc{Beer-Hofmann, Richard} (1866-07-11 – 1945-09-26), \emph{Schriftsteller/Schriftstellerin}|pw} und von Ihrer Mama\pwindex{Hofmannsthal, Anna von 27.01.1849 – 22.03.1904@\textsc{Hofmannsthal, Anna von} (27.01.1849 – 22.03.1904)|pwv}, dſs Sie ſich ſo wohl
               fühlen und mit Luſt arbeiten, und ſo freue ich mich nicht nur auf Sie ſondern auch
               auf das, was Sie mitbringen werden. Ich war auf meiner Reiſe eigentlich nur in den
               Stunden ziemlich gut dran, in denen ich geſchrieben habe; – {\pb}das Wetter war ſelten ſchön, nur in \textsc{Ragusa}\oindex{Dubrovnik@\textbf{Dubrovnik}, \emph{P.PPLA}|pw} 3 klare Tage, aber da wars für \textsc{Ragusa}\oindex{Dubrovnik@\textbf{Dubrovnik}, \emph{P.PPLA}|pw} und für Anfang April doch zu kühl. In Abbazia\oindex{Opatija@\textbf{Opatija}, \emph{P.PPLA2}|pw} hat es ununterbrochen gegoſſen; dort war ich viel mit
                  Georg Hirſchfeld\pwindex{Hirschfeld, Georg 11.02.1873 – 17.01.1942@\textsc{Hirschfeld, Georg} (11.02.1873 – 17.01.1942), \emph{Schriftsteller/Schriftstellerin}|pw} zuſammen, zu dem ich neue
               Sympathie gefaſſt habe. Elly\pwindex{Petersen, Elly 26.02.1874 – 29.12.1965@\textsc{Petersen, Elly} (26.02.1874 – 29.12.1965), \emph{Schriftsteller/Schriftstellerin}|pw} liebe ich aber
               noch immer nicht. Es war mir auffallend, wie \label{K_L01031-1v}\edtext{viel ich auf meiner Reiſe}{\lemma{\textnormal{\emph{viel … Reiſe}}}\Cendnote{\textnormal{Er erwähnt mehrere davon im \emph{Tagebuch}\pwindex{Tagebuch@\emph{Tagebuch}|pwk} (1. 4. 1900, 4. 4. 1900, 5. 4. 1900, 6. 4. 1900).}}}\label{K_L01031-1} geträumt habe; ſo lebhaft und bewegt wie
               nie, und meine Todte\pwindex{Reinhard, Marie 1871-03-13 – 1899-03-18@\textsc{Reinhard, Marie} (1871-03-13 – 1899-03-18), \emph{Gesangspädagoge/Gesangspädagogin}|pwv} iſt mir
               vier oder fünf Mal erſchienen. {\pb}Der ſonderbarſte von
               allen Träumen war der, dſs ich träumte, \label{K_L01031-2v}\edtext{ich hätte drei Träume}{\lemma{\textnormal{\emph{ich hätte drei Träume}}}\Cendnote{\textnormal{Siehe A. S.: \emph{Tagebuch}, 6. 4. 1900.
               }}}\label{K_L01031-2} gehabt, die mir den Tod vorhergeſagt und erzählte jemandem dieſe 3 Träume,
               nach dem Aufwachen erinnerte ich mich nur an einen davon deutlich. – Ich bin noch
               immer an der langen Novelle\pwindex{Frau Bertha Garlan. Roman@\emph{Frau Bertha Garlan. Roman}|pwv},
               vor Oſtern wird ſie doch fertig, dann dictir ich ſie; fange aber gleich
               was neues an, entweder eine kurze Geſchichte\pwindex{Erfolg@\emph{Ein Erfolg}|pwuv} oder dieſes Sommerſtück\pwindex{Im Spiel der Sommerluefte. In drei Aufzuegen@\emph{Im Spiel der Sommerlüfte. In drei Aufzügen}|pwv}; – eigentlich hab ich ein Gefühl von
                  Unerſchöpf{\pb}lichkeit wie nie zuvor, aber es iſt mehr
               theoretiſch, – macht mich nicht beſonders glücklich. Ich empfinde meinen Verluſt
               ſchwerer und ſichrer als je.\pend
           
\pstart
           Leben Sie wohl und ſchreiben Sie mir bald ein Wort.\pend
           \pstart Von Herzen Ihr \spacefill\mbox{Arthur.}\pend{}
\pstart
           \noindent{}Ich hoffe Sie haben meinen Brief \introOben{}(aus Wien\oindex{Wien@\textbf{Wien}, \emph{A.ADM2}|pw})\introOben{} und auch die Karten aus Dalmatien\oindex{Dalmatien@\textbf{Dalmatien}, \emph{L.RGNH}|pw} bekommen.\pend
           
\pstart
           Wien\oindex{Wien@\textbf{Wien}, \emph{A.ADM2}|pw}, 9. 4. 900.\pend
           \selectlanguage{ngerman}\endnumbering\briefempfaengerindex{Hofmannsthal, Hugo von@\textsc{Hofmannsthal, Hugo von}!zzzSchnitzler, Arthur@\emph{von Arthur Schnitzler}!1900-04-091@{9. 4. 1900}|)be}\mylabel{L01031h}  \normalsize

\doendnotes{C}
\bigskip
\vfill

\clearpage

\footnotesize

\lohead{\textsc{register}}

% Definiere theindex-Environment komplett neu ohne reledmac
\makeatletter
\renewenvironment{theindex}{%
  \section*{\indexname}%
  \setlength{\parindent}{0pt}%
  \setlength{\parskip}{0pt plus 0.3pt}%
  \let\item\@idxitem
}{%
  \clearpage
}
\makeatother

\IfFileExists{\jobname-pw.ind}{\input{\jobname-pw.ind}}{}

\end{document}

      