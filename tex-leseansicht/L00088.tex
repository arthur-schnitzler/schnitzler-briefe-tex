\input{../tex-inputs/latex-pdf-vorspann}
\begin{center}
            \textcolor{red}{ENTWURF. ENTZIFFERUNG NOCH NICHT KORREKTURGELESEN}
                      \end{center}
            
               \section[Hugo von Hofmannsthal an Arthur Schnitzler, 26. 3. 1892]{ Hugo von Hofmannsthal an Arthur Schnitzler, 26. 3. 1892}\nopagebreak\mylabel{v}\rehead{ }\begin{ledgroupsized}[t]{13cm}\normalsize\beginnumbering\briefempfaengerindex{Schnitzler, Arthur@\textsc{Schnitzler, Arthur}!zzzHofmannsthal, Hugo von@\emph{von Hugo von Hofmannsthal}!1892-03-261@{26. 3. 1892}|(be} \toendnotes[C]{\smallbreak\pagebreak[2]} \Standort{CUL, Schnitzler, B 43.}
\physDesc{Postkarte
\newline{}Handschrift: schwarze Tinte, lateinische Kurrent\newline{}Versand: Stempel: »\nobreak{}Wien 1/1, 26. 3. 92, 10–1\textcolor{gray}{2}N\nobreak{}«.  
\newline{}Schnitzler: mit Bleistift datiert: »26/3 92« \newline{}Ordnung: von unbekannter Hand nummeriert:
                                    »21« }\buchAbdrucke{\weitereDrucke{Hugo von Hofmannsthal, Arthur Schnitzler: \emph{Briefwechsel}. Hg. Therese Nickl und Heinrich Schnitzler. Frankfurt am Main: \emph{S. Fischer} 1964, S. 18.} }\pstart{}{\pb}\textsc{Herrn D\textsuperscript{r} Arthur
                            Schnitzler}\pend{}\pstart{}\textsc{I. Wien\oindex{I., Innere Stadt@\textbf{I., Innere Stadt}|pw}}\pend{}\pstart{}\textsc{Kärntnerring 12\oindex{Kaerntnerring@\textbf{Kärntnerring}|pw}}\pend{}{\bigskip}\pstart{}{\pb}Lieber Freund,\pend\pstart
           Ich bin für morgen zu Tisch geladen. Es ist also wieder nichts. Herr Bölsche\pwindex{Boelsche, Wilhelm 02.01.1861 – 31.08.1939@\textsc{Bölsche, Wilhelm} (02.01.1861 – 31.08.1939), \emph{Schriftsteller, Publizist}|pw} hat mir das »Kind\pwindex{Hofmannsthal, Hugo von 01.02.1874 – 15.07.1929@\textsc{Hofmannsthal, Hugo von} (01.02.1874 – 15.07.1929), \emph{Schriftsteller}!Age of Innocence1930@\strich\emph{Age of Innocence} {[}1930{]}|pw}« zurückgeschickt; natürlich mit einem sehr artigen
                    Brief.\pend
           \pstart
           Auf Wiedersehen!{\\[\baselineskip]}\spacefill\mbox{Loris.}\pend
           \leftskip=0em{}\pstart
           Samstag.\pend
           \endnumbering\briefempfaengerindex{Schnitzler, Arthur@\textsc{Schnitzler, Arthur}!zzzHofmannsthal, Hugo von@\emph{von Hugo von Hofmannsthal}!1892-03-261@{26. 3. 1892}|)be}\mylabel{h}\end{ledgroupsized}  \newcommand{\dateiname}{L00088}\newcommand{\titel}{Hugo von Hofmannsthal an Arthur Schnitzler, 26. 3. 1892}\newcommand{\editorInnen}{Martin Anton Müller und Gerd-Hermann Susen}\input{../tex-inputs/latex-pdf-abspann}
      