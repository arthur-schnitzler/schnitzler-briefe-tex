%% latex-leseansicht-vorspann.tex
%% Vorspann für die Leseansicht.
%% Lädt die gemeinsame Datei latex-vorspann.tex mit nicht gesetztem Schalter.

\newif\ifkorrekturansicht
\korrekturansichtfalse

\input{../tex-inputs/latex-vorspann}

\begin{center}
            \textcolor{red}{ENTWURF, NICHT FERTIG KORRIGIERT}
                      \end{center}
            
         
         \newcommand{\erwaehntePersonen}{Personen: Felix Salten}
         \newcommand{\erwaehnteInstitutionen}{}
         \newcommand{\erwaehnteOrte}{Orte: Café Pucher, Wien}
         \newcommand{\erwaehnteWerke}{Werke: Burgtheater. (»Cyrano von Bergerac«, romantische Komödie in fünf Aufzügen von Edmond Rostand, deutsch von Ludwig Fulda. – Zum erstenmale aufgeführt am 11. October 1898.), Cyrano von Bergerac. Romantische Komödie in fünf Aufzügen, Deutsches Volkstheater. (»Mutter Erde«, Drama in fünf Acten von Max Halbe. – Zum erstenmale am 8. October 1898.), Im weißen Rößl. Schwank in drei Acten, Mutter Erde. Drama in fünf Acten, Wiener Allgemeine Zeitung}
               \section[Arthur Schnitzler an Felix Salten, {[}10. 11. 1898?{]}]{ Arthur Schnitzler an Felix Salten, {[}10. 11. 1898?{]}}\nopagebreak\mylabel{v}\rehead{ }\begin{ledgroupsized}[t]{13cm}\normalsize\beginnumbering \toendnotes[C]{\smallbreak\pagebreak[2]} \Standort{Wienbibliothek im Rathaus, ZPH 1681, 2.1.516.}
\physDesc{
\newline{}Handschrift: , deutsche Kurrent}\toendnotes[C]{\smallbreak}\pstart
           \noindent{}{\pb}Lieber Freund, vielleicht ſind Sie \label{K_L03033-111v}\edtext{morgen nach dem weißen Röſſl\pwindex{\textcolor{red}{\textsuperscript{XXXX1 indx}}!Im weissen Roessl. Schwank in drei Acten1897-12-30@\strich\emph{Im weißen Rößl. Schwank in drei Acten} {[}1897-12-30{]}|pw}\pwindex{\textcolor{red}{\textsuperscript{XXXX1 indx}}!Im weissen Roessl. Schwank in drei Acten1897-12-30@\strich\emph{Im weißen Rößl. Schwank in drei Acten} {[}1897-12-30{]}|pw}}{\lemma{\textnormal{\emph{morgen … Röſſl}}}\Cendnote{\textnormal{Das erlaubt die Datierung des Korrespondenzstücks, da die Premiere des Stücks am 
                     15. 10. 1898 stattgefunden hatte und damit nachweislich bereits vergangen war, als dieser Brief 
                     geschrieben wurde. Folglich kann sich Schnitzler\pwindex{Schnitzler, Arthur 15.05.1862 – 21.10.1931@\textsc{Schnitzler, Arthur} (15.05.1862 – 21.10.1931), \emph{Schriftsteller, Mediziner}|pwk} mit ziemlicher Wahrscheinlichkeit nur auf eine Aufführung beziehen, die
                     er selbst zu besuchen gedachte, nämlich jene vom 11. 11. 1898.}}}\label{K_L03033-111h} im Pucher\oindex{Cafe Pucher@\textbf{Café Pucher}|pw}? –
               Sehr ſchön haben Sie über die Mutter Erde\pwindex{\textcolor{red}{\textsuperscript{XXXX1 indx}}!Mutter Erde. Drama in fuenf Acten1897@\strich\emph{Mutter Erde. Drama in fünf Acten} {[}1897{]}|pw} u den
                  Cyrano\pwindex{\textcolor{red}{\textsuperscript{XXXX1 indx}}!Cyrano von Bergerac. Romantische Komoedie in fuenf Aufzuegen1898-09-14@\strich\emph{Cyrano von Bergerac. Romantische Komödie in fünf Aufzügen} {[}1898-09-14{]}|pw}{ }\label{K_L03033-11v}\edtext{geſchrieben\pwindex{Salten, Felix 06.09.1869 – 08.10.1945@\textsc{Salten, Felix} (06.09.1869 – 08.10.1945), \emph{Schriftsteller, Journalist}!Deutsches Volkstheater. (»Mutter Erde«, Drama in fuenf Acten von Max Halbe. – Zum erstenmale am 8. October 1898.)1898-10-11@\strich\emph{Deutsches Volkstheater. (»Mutter Erde«, Drama in fünf Acten von Max Halbe. – Zum erstenmale am 8. October 1898.)} {[}1898-10-11{]}|pwv}\pwindex{Salten, Felix 06.09.1869 – 08.10.1945@\textsc{Salten, Felix} (06.09.1869 – 08.10.1945), \emph{Schriftsteller, Journalist}!Burgtheater. (»Cyrano von Bergerac«, romantische Komoedie in fuenf Aufzuegen von Edmond Rostand, deutsch von Ludwig Fulda. – Zum erstenmale aufgefuehrt am 11. October 1898.)1898-10-13@\strich\emph{Burgtheater. (»Cyrano von Bergerac«, romantische Komödie in fünf Aufzügen von Edmond Rostand, deutsch von Ludwig Fulda. – Zum erstenmale aufgeführt am 11. October 1898.)} {[}1898-10-13{]}|pwv}}{\lemma{\textnormal{\emph{geſchrieben}}}\Cendnote{\textnormal{Felix Salten\pwindex{Salten, Felix 06.09.1869 – 08.10.1945@\textsc{Salten, Felix} (06.09.1869 – 08.10.1945), \emph{Schriftsteller, Journalist}|pwk}: \emph{Deutsches Volkstheater. (»Mutter Erde«,
                     Drama in fünf Acten von Max Halbe. – Zum erstenmale am 8. October 1898.)}\pwindex{Salten, Felix 06.09.1869 – 08.10.1945@\textsc{Salten, Felix} (06.09.1869 – 08.10.1945), \emph{Schriftsteller, Journalist}!Deutsches Volkstheater. (»Mutter Erde«, Drama in fuenf Acten von Max Halbe. – Zum erstenmale am 8. October 1898.)1898-10-11@\strich\emph{Deutsches Volkstheater. (»Mutter Erde«, Drama in fünf Acten von Max Halbe. – Zum erstenmale am 8. October 1898.)} {[}1898-10-11{]}|pwk} In:
                        \emph{Wiener Allgemeine Zeitung}\pwindex{?? Werk@Nicht ermittelte Verfasserinnen und Verfasser!Wiener Allgemeine Zeitung1.3.1880 – 11.2.1934@\emph{Wiener Allgemeine Zeitung} {[}1.3.1880 – 11.2.1934{]}|pwk}, Nr. 6.183, 11. 10. 1898, S. 2 und 
                  Felix Salten\pwindex{Salten, Felix 06.09.1869 – 08.10.1945@\textsc{Salten, Felix} (06.09.1869 – 08.10.1945), \emph{Schriftsteller, Journalist}|pwk}: \emph{Burgtheater. (»Cyrano von Bergerac«, romantische Komödie in
                        fünf Aufzügen von Edmond Rostand, deutsch von Ludwig Fulda. – Zum erstenmale
                        aufgeführt am 11. October 1898.)}\pwindex{Salten, Felix 06.09.1869 – 08.10.1945@\textsc{Salten, Felix} (06.09.1869 – 08.10.1945), \emph{Schriftsteller, Journalist}!Burgtheater. (»Cyrano von Bergerac«, romantische Komoedie in fuenf Aufzuegen von Edmond Rostand, deutsch von Ludwig Fulda. – Zum erstenmale aufgefuehrt am 11. October 1898.)1898-10-13@\strich\emph{Burgtheater. (»Cyrano von Bergerac«, romantische Komödie in fünf Aufzügen von Edmond Rostand, deutsch von Ludwig Fulda. – Zum erstenmale aufgeführt am 11. October 1898.)} {[}1898-10-13{]}|pwk} In: \emph{Wiener Allgemeine Zeitung}\pwindex{?? Werk@Nicht ermittelte Verfasserinnen und Verfasser!Wiener Allgemeine Zeitung1.3.1880 – 11.2.1934@\emph{Wiener Allgemeine Zeitung} {[}1.3.1880 – 11.2.1934{]}|pwk}, Nr. 6.185,
                        13. 10. 1898, S. 2–3.}}}\label{K_L03033-11h} – beide {\pb}Mal gleich dorthin gegriffen, wo die Dinge zu faſſen ſind. \pend
           \pstart
           Auf Wiederſehen {\\[\baselineskip]}Herzlichſt {\\[\baselineskip]}Ihr {\\[\baselineskip]}\spacefill\mbox{A. S.}\pend
           \leftskip=0em{}
         
         \endnumbering\mylabel{h}\end{ledgroupsized}\begin{anhang}\end{anhang}\newcommand{\dateiname}{L03033}\newcommand{\titel}{Arthur Schnitzler an Felix Salten, [10. 11. 1898?]}\newcommand{\editorInnen}{Martin Anton Müller und Laura Untner}%% latex-leseansicht-abspann.tex
%% Abspann für die Leseansicht.
%% Der Schalter \ifkorrekturansicht ist bereits durch den Vorspann gesetzt.

%% latex-abspann.tex
%% Gemeinsamer Abspann für Korrekturansicht und Leseansicht.
%% Setzt den Schalter \ifkorrekturansicht voraus (gesetzt in den
%% einbindenden Dateien latex-korrekturansicht-abspann.tex bzw.
%% latex-leseansicht-abspann.tex).
%% ---------------------------------------------------------------

\normalsize

% Das esempio-Environment wird nur in der Leseansicht benötigt
\ifkorrekturansicht\else
\newenvironment{esempio}[3]%
{
    \vspace{1.5ex}
    \rlap{\underline{#1}}
    \par
    \setlength{\parindent}{0cm}
    \nopagebreak
    \leftskip=#2cm
    \rightskip=#3cm
}
{
    \par
}
\fi

\doendnotes{C}
\bigskip
\vfill

\clearpage

\footnotesize

\ifkorrekturansicht
  \lohead{\textsc{register}}
\fi

% theindex-Environment neu definieren ohne reledmac
\makeatletter
\renewenvironment{theindex}{%
  \ifkorrekturansicht
    \section*{\indexname}%
  \else
    \subsubsection*{Index der erwähnten Entitäten}%
  \fi
  \setlength{\parindent}{0pt}%
  \setlength{\parskip}{0pt plus 0.3pt}%
  \let\item\@idxitem
}{%
  \ifkorrekturansicht\clearpage\fi
}
\makeatother

\IfFileExists{\jobname-pw.ind}{\input{\jobname-pw.ind}}{}

% Quellenangabe nur in der Leseansicht
\ifkorrekturansicht\else
% Fallback-Definitionen, falls die .tex-Datei \titel etc. nicht gesetzt hat
\providecommand{\titel}{}
\providecommand{\editorInnen}{}
\providecommand{\dateiname}{\jobname}

\vspace{3cm}

\vfill

\footnotesize
\textsc{Quelle}: \titel. Herausgegeben von {\editorInnen}. In: \emph{Arthur Schnitzler: Briefwechsel mit Autorinnen und Autoren}.
 Digitale Edition, https://schnitzler-briefe.acdh.oeaw.ac.at/{\dateiname}.html (Stand \today)
\fi

\end{document}


      