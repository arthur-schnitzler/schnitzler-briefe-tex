%% latex-korrekturansicht-vorspann.tex
%% Vorspann für die Korrekturansicht.
%% Lädt die gemeinsame Datei latex-vorspann.tex mit gesetztem Schalter.

\newif\ifkorrekturansicht
\korrekturansichttrue

\input{../tex-inputs/latex-vorspann}


\section[ Arthur Schnitzler an Felix Salten, {[}14. 10. 1898?{]}]{L03033 Arthur Schnitzler an Felix Salten, {[}14. 10. 1898?{]}}
\nopagebreak\mylabel{L03033v}
\rehead{ }\normalsize\beginnumbering\briefempfaengerindex{Salten, Felix@\textsc{Salten, Felix}!zzzSchnitzler, Arthur@\emph{von Arthur Schnitzler}!1898-10-143@{{[}14. 10. 1898?{]}}|(be}
\toendnotes[C]{\smallbreak\pagebreak[2]}\Standort{Wienbibliothek im Rathaus, ZPH 1681, 2.1.516.}
\physDesc{Karte, 233 Zeichen
\newline{}Handschrift: Bleistift, deutsche Kurrent
\newline{}Ordnung: mit Bleistift von unbekannter Hand nummeriert: »35« }\toendnotes[C]{\smallbreak}
\pstart
           \noindent{}{\pb}Lieber Freund, vielleicht ſind Sie \label{K_L03033-1v}\edtext{morgen nach dem weißen
                  Röſſl\pwindex{Im weissen Roessl. Schwank in drei Acten@\emph{Im weißen Rößl. Schwank in drei Acten}|pw}}{\lemma{\textnormal{\emph{morgen … Röſſl}}}\Cendnote{\textnormal{Das erlaubt die Datierung des
                  Korrespondenzstücks. Die Premiere von \emph{Im weißen
                     Rößl}\pwindex{Im weissen Roessl. Schwank in drei Acten@\emph{Im weißen Rößl. Schwank in drei Acten}|pwk} fand am 15. 10. 1898 statt. 
                  Schnitzler  besuchte die Aufführung nicht,
                  sondern erst jene am 11. 11. 1898. Trotzdem 
                  dürfte er von der Premiere und nicht vom eigenen Besuch sprechen, da er sich in Folge auf zwei wenige Tage
                  zuvor erschienene Artikel Saltens\pwindex{Salten, Felix 06.09.1869 – 08.10.1945@\textsc{Salten, Felix} (06.09.1869 – 08.10.1945), \emph{Schriftsteller/Schriftstellerin, Journalist/Journalistin, Chefredakteur/Chefredakteurin}|pwk} bezieht,
                  über die er einen Monat später längst mit Salten\pwindex{Salten, Felix 06.09.1869 – 08.10.1945@\textsc{Salten, Felix} (06.09.1869 – 08.10.1945), \emph{Schriftsteller/Schriftstellerin, Journalist/Journalistin, Chefredakteur/Chefredakteurin}|pwk}
                  gesprochen haben dürfte.}}}\label{K_L03033-1} im
                  \label{K_L03033-2v}\edtext{Pucher\oindex{Cafe Pucher@\textbf{Café Pucher}, \emph{Kaffeehaus (K.KAF)}|pw}}{\lemma{\textnormal{\emph{Pucher}}}\Cendnote{\textnormal{nicht belegt}}}\label{K_L03033-2}? – Sehr ſchön
               haben Sie über die \label{K_L03033-3v}\edtext{Mutter Erde\pwindex{Mutter Erde. Drama in fuenf Acten@\emph{Mutter Erde. Drama in fünf Acten}|pw}}{\lemma{\textnormal{\emph{Mutter Erde}}}\Cendnote{\textnormal{Felix Salten\pwindex{Salten, Felix 06.09.1869 – 08.10.1945@\textsc{Salten, Felix} (06.09.1869 – 08.10.1945), \emph{Schriftsteller/Schriftstellerin, Journalist/Journalistin, Chefredakteur/Chefredakteurin}|pwk}: \emph{Deutsches Volkstheater. (»Mutter Erde«, Drama in fünf Acten
                     von Max Halbe. – Zum erstenmale am 8. October 1898.)}\pwindex{Deutsches Volkstheater. (»Mutter Erde«, Drama in fuenf Acten von Max Halbe. – Zum erstenmale am 8. October 1898.)@\emph{Deutsches Volkstheater. (»Mutter Erde«, Drama in fünf Acten von Max Halbe. – Zum erstenmale am 8. October 1898.)}|pwk} In: \emph{Wiener Allgemeine Zeitung}\pwindex{Wiener Allgemeine Zeitung@\emph{Wiener Allgemeine Zeitung}|pwk}, Nr. 6183, 11. 10. 1898, S. 2.
               }}}\label{K_L03033-3} u den \label{K_L03033-4v}\edtext{Cyrano\pwindex{Cyrano von Bergerac. Romantische Komoedie in fuenf Aufzuegen@\emph{Cyrano von Bergerac. Romantische Komödie in fünf Aufzügen}|pw}}{\lemma{\textnormal{\emph{Cyrano}}}\Cendnote{\textnormal{Felix Salten\pwindex{Salten, Felix 06.09.1869 – 08.10.1945@\textsc{Salten, Felix} (06.09.1869 – 08.10.1945), \emph{Schriftsteller/Schriftstellerin, Journalist/Journalistin, Chefredakteur/Chefredakteurin}|pwk}: \emph{Burgtheater. (»Cyrano von Bergerac«, romantische Komödie in
                              fünf Aufzügen von Edmond Rostand, deutsch von Ludwig Fulda. – Zum erstenmale
                              aufgeführt am 11. October 1898.)}\pwindex{Burgtheater. (»Cyrano von Bergerac«, romantische Komoedie in fuenf Aufzuegen von Edmond Rostand, deutsch von Ludwig Fulda. – Zum erstenmale aufgefuehrt am 11. October 1898.)@\emph{Burgtheater. (»Cyrano von Bergerac«, romantische Komödie in fünf Aufzügen von Edmond Rostand, deutsch von Ludwig Fulda. – Zum erstenmale aufgeführt am 11. October 1898.)}|pwk} In: \emph{Wiener Allgemeine Zeitung}\pwindex{Wiener Allgemeine Zeitung@\emph{Wiener Allgemeine Zeitung}|pwk}, Nr. 6185, 13. 10. 1898, S. 2–3.}}}\label{K_L03033-4}{ }geſchrieben\pwindex{Deutsches Volkstheater. (»Mutter Erde«, Drama in fuenf Acten von Max Halbe. – Zum erstenmale am 8. October 1898.)@\emph{Deutsches Volkstheater. (»Mutter Erde«, Drama in fünf Acten von Max Halbe. – Zum erstenmale am 8. October 1898.)}|pwv}\pwindex{Burgtheater. (»Cyrano von Bergerac«, romantische Komoedie in fuenf Aufzuegen von Edmond Rostand, deutsch von Ludwig Fulda. – Zum erstenmale aufgefuehrt am 11. October 1898.)@\emph{Burgtheater. (»Cyrano von Bergerac«, romantische Komödie in fünf Aufzügen von Edmond Rostand, deutsch von Ludwig Fulda. – Zum erstenmale aufgeführt am 11. October 1898.)}|pwv}
                – beide {\pb}Mal gleich dorthin gegriffen, wo die Dinge zu
               faſſen ſind.\pend
           
\pstart
           Auf Wiederſehen {\\[\baselineskip]}Herzlichſt Ihr {\\[\baselineskip]}\spacefill\mbox{A. S.}\pend
           \leftskip=0em{}\selectlanguage{ngerman}\endnumbering\briefempfaengerindex{Salten, Felix@\textsc{Salten, Felix}!zzzSchnitzler, Arthur@\emph{von Arthur Schnitzler}!1898-10-143@{{[}14. 10. 1898?{]}}|)be}\mylabel{L03033h}  \normalsize

\doendnotes{C}
\bigskip
\vfill

\clearpage

\footnotesize

\lohead{\textsc{register}}

% Definiere theindex-Environment komplett neu ohne reledmac
\makeatletter
\renewenvironment{theindex}{%
  \section*{\indexname}%
  \setlength{\parindent}{0pt}%
  \setlength{\parskip}{0pt plus 0.3pt}%
  \let\item\@idxitem
}{%
  \clearpage
}
\makeatother

\IfFileExists{\jobname-pw.ind}{\input{\jobname-pw.ind}}{}

\end{document}

      