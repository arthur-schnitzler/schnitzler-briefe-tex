%% latex-leseansicht-vorspann.tex
%% Vorspann für die Leseansicht.
%% Lädt die gemeinsame Datei latex-vorspann.tex mit nicht gesetztem Schalter.

\newif\ifkorrekturansicht
\korrekturansichtfalse

\input{../tex-inputs/latex-vorspann}


         
         \renewcommand{\erwaehntePersonen}{Personen: Felix Salten}
         \renewcommand{\erwaehnteOrte}{Orte: Café Pucher, Wien}
         \renewcommand{\erwaehnteWerke}{Werke: Burgtheater. (»Cyrano von Bergerac«, romantische Komödie in fünf Aufzügen von Edmond Rostand, deutsch von Ludwig Fulda. – Zum erstenmale aufgeführt am 11. October 1898.), Cyrano von Bergerac. Romantische Komödie in fünf Aufzügen, Deutsches Volkstheater. (»Mutter Erde«, Drama in fünf Acten von Max Halbe. – Zum erstenmale am 8. October 1898.), Im weißen Rößl. Schwank in drei Acten, Mutter Erde. Drama in fünf Acten, Wiener Allgemeine Zeitung}
               \section[ Arthur Schnitzler an Felix Salten, {[}14. 10. 1898?{]}]{ Arthur Schnitzler an Felix Salten, {[}14. 10. 1898?{]}}\nopagebreak\mylabel{v}\rehead{ }\begin{ledgroupsized}[t]{13cm}\normalsize\beginnumbering\briefempfaengerindex{Salten, Felix@\textsc{Salten, Felix}!zzzSchnitzler, Arthur@\emph{von Arthur Schnitzler}!1898-10-143@{{[}14. 10. 1898?{]}}|(be} \toendnotes[C]{\smallbreak\pagebreak[2]} \Standort{Wienbibliothek im Rathaus, ZPH 1681, 2.1.516.}
\physDesc{Karte, 233 Zeichen
\newline{}Handschrift: Bleistift, deutsche Kurrent
\newline{}Ordnung: mit Bleistift von unbekannter Hand nummeriert: »35« }\toendnotes[C]{\smallbreak}\pstart
           \noindent{}{\pb}Lieber Freund, vielleicht ſind Sie \label{K_L03033-1v}\edtext{morgen nach dem weißen
                  Röſſl\pwindex{\textcolor{red}{\textsuperscript{XXXX1 indx}}!Im weissen Roessl. Schwank in drei Acten1897-12-30@\strich\emph{Im weißen Rößl. Schwank in drei Acten} {[}1897-12-30{]}|pw}\pwindex{\textcolor{red}{\textsuperscript{XXXX1 indx}}!Im weissen Roessl. Schwank in drei Acten1897-12-30@\strich\emph{Im weißen Rößl. Schwank in drei Acten} {[}1897-12-30{]}|pw}}{\lemma{\textnormal{\emph{morgen … Röſſl}}}\Cendnote{\textnormal{Das erlaubt die Datierung des
                  Korrespondenzstücks. Die Premiere von \emph{Im weißen
                     Rößl}\pwindex{\textcolor{red}{\textsuperscript{XXXX1 indx}}!Im weissen Roessl. Schwank in drei Acten1897-12-30@\strich\emph{Im weißen Rößl. Schwank in drei Acten} {[}1897-12-30{]}|pwk}\pwindex{\textcolor{red}{\textsuperscript{XXXX1 indx}}!Im weissen Roessl. Schwank in drei Acten1897-12-30@\strich\emph{Im weißen Rößl. Schwank in drei Acten} {[}1897-12-30{]}|pwk} fand am 15. 10. 1898 statt. 
                  Schnitzler\pwindex{Schnitzler, Arthur 15.05.1862 – 21.10.1931@\textsc{Schnitzler, Arthur} (15.05.1862 – 21.10.1931), \emph{Schriftsteller, Mediziner}|pwk}  besuchte die Aufführung nicht,
                  sondern erst jene am 11. 11. 1898. Trotzdem 
                  dürfte er von der Premiere und nicht vom eigenen Besuch sprechen, da er sich in Folge auf zwei wenige Tage
                  zuvor erschienene Artikel Salten\pwindex{Salten, Felix 06.09.1869 – 08.10.1945@\textsc{Salten, Felix} (06.09.1869 – 08.10.1945), \emph{Schriftsteller, Journalist}|pwk}s bezieht,
                  über die er einen Monat später längst mit Salten\pwindex{Salten, Felix 06.09.1869 – 08.10.1945@\textsc{Salten, Felix} (06.09.1869 – 08.10.1945), \emph{Schriftsteller, Journalist}|pwk}
                  gesprochen haben könnte.}}}\label{K_L03033-1h} im
                  \label{K_L03033-2v}\edtext{Pucher\oindex{Cafe Pucher@\textbf{Café Pucher}|pw}}{\lemma{\textnormal{\emph{Pucher}}}\Cendnote{\textnormal{nicht belegt}}}\label{K_L03033-2h}? – Sehr ſchön
               haben Sie über die \label{K_L03033-3v}\edtext{Mutter Erde\pwindex{\textcolor{red}{\textsuperscript{XXXX1 indx}}!Mutter Erde. Drama in fuenf Acten1897@\strich\emph{Mutter Erde. Drama in fünf Acten} {[}1897{]}|pw}}{\lemma{\textnormal{\emph{Mutter Erde}}}\Cendnote{\textnormal{Felix Salten\pwindex{Salten, Felix 06.09.1869 – 08.10.1945@\textsc{Salten, Felix} (06.09.1869 – 08.10.1945), \emph{Schriftsteller, Journalist}|pwk}: \emph{Deutsches Volkstheater. (»Mutter Erde«, Drama in fünf Acten
                     von Max Halbe. – Zum erstenmale am 8. October 1898.)}\pwindex{Salten, Felix 06.09.1869 – 08.10.1945@\textsc{Salten, Felix} (06.09.1869 – 08.10.1945), \emph{Schriftsteller, Journalist}!Deutsches Volkstheater. (»Mutter Erde«, Drama in fuenf Acten von Max Halbe. – Zum erstenmale am 8. October 1898.)1898-10-11@\strich\emph{Deutsches Volkstheater. (»Mutter Erde«, Drama in fünf Acten von Max Halbe. – Zum erstenmale am 8. October 1898.)} {[}1898-10-11{]}|pwk} In: \emph{Wiener Allgemeine Zeitung}\pwindex{?? Werk@Nicht ermittelte Verfasserinnen und Verfasser!Wiener Allgemeine Zeitung1.3.1880 – 11.2.1934@\emph{Wiener Allgemeine Zeitung} {[}1.3.1880 – 11.2.1934{]}|pwk}, Nr. 6.183, 11. 10. 1898, S. 2}}}\label{K_L03033-3h} u den \label{K_L03033-4v}\edtext{Cyrano\pwindex{\textcolor{red}{\textsuperscript{XXXX1 indx}}!Cyrano von Bergerac. Romantische Komoedie in fuenf Aufzuegen1898-09-14@\strich\emph{Cyrano von Bergerac. Romantische Komödie in fünf Aufzügen} {[}Übersetzung, 1898-09-14{]}|pw}}{\lemma{\textnormal{\emph{Cyrano}}}\Cendnote{\textnormal{Felix Salten\pwindex{Salten, Felix 06.09.1869 – 08.10.1945@\textsc{Salten, Felix} (06.09.1869 – 08.10.1945), \emph{Schriftsteller, Journalist}|pwk}: \emph{Burgtheater. (»Cyrano von Bergerac«, romantische Komödie in
                              fünf Aufzügen von Edmond Rostand, deutsch von Ludwig Fulda. – Zum erstenmale
                              aufgeführt am 11. October 1898.)}\pwindex{Salten, Felix 06.09.1869 – 08.10.1945@\textsc{Salten, Felix} (06.09.1869 – 08.10.1945), \emph{Schriftsteller, Journalist}!Burgtheater. (»Cyrano von Bergerac«, romantische Komoedie in fuenf Aufzuegen von Edmond Rostand, deutsch von Ludwig Fulda. – Zum erstenmale aufgefuehrt am 11. October 1898.)1898-10-13@\strich\emph{Burgtheater. (»Cyrano von Bergerac«, romantische Komödie in fünf Aufzügen von Edmond Rostand, deutsch von Ludwig Fulda. – Zum erstenmale aufgeführt am 11. October 1898.)} {[}1898-10-13{]}|pwk} In: \emph{Wiener Allgemeine Zeitung}\pwindex{?? Werk@Nicht ermittelte Verfasserinnen und Verfasser!Wiener Allgemeine Zeitung1.3.1880 – 11.2.1934@\emph{Wiener Allgemeine Zeitung} {[}1.3.1880 – 11.2.1934{]}|pwk}, Nr. 6.185, 13. 10. 1898, S. 2–3.}}}\label{K_L03033-4h}{ }geſchrieben\pwindex{Salten, Felix 06.09.1869 – 08.10.1945@\textsc{Salten, Felix} (06.09.1869 – 08.10.1945), \emph{Schriftsteller, Journalist}!Deutsches Volkstheater. (»Mutter Erde«, Drama in fuenf Acten von Max Halbe. – Zum erstenmale am 8. October 1898.)1898-10-11@\strich\emph{Deutsches Volkstheater. (»Mutter Erde«, Drama in fünf Acten von Max Halbe. – Zum erstenmale am 8. October 1898.)} {[}1898-10-11{]}|pwv}\pwindex{Salten, Felix 06.09.1869 – 08.10.1945@\textsc{Salten, Felix} (06.09.1869 – 08.10.1945), \emph{Schriftsteller, Journalist}!Burgtheater. (»Cyrano von Bergerac«, romantische Komoedie in fuenf Aufzuegen von Edmond Rostand, deutsch von Ludwig Fulda. – Zum erstenmale aufgefuehrt am 11. October 1898.)1898-10-13@\strich\emph{Burgtheater. (»Cyrano von Bergerac«, romantische Komödie in fünf Aufzügen von Edmond Rostand, deutsch von Ludwig Fulda. – Zum erstenmale aufgeführt am 11. October 1898.)} {[}1898-10-13{]}|pwv}
                – beide {\pb}Mal gleich dorthin gegriffen, wo die Dinge zu
               faſſen ſind.\pend
           \pstart
           Auf Wiederſehen {\\[\baselineskip]}Herzlichſt Ihr {\\[\baselineskip]}\spacefill\mbox{A. S.}\pend
           \leftskip=0em{}
         
         \endnumbering\mylabel{h}\end{ledgroupsized}  \newcommand{\dateiname}{L03033}\newcommand{\titel}{Arthur Schnitzler an Felix Salten, [14. 10. 1898?]}\newcommand{\editorInnen}{Martin Anton Müller und Laura Untner}%% latex-leseansicht-abspann.tex
%% Abspann für die Leseansicht.
%% Der Schalter \ifkorrekturansicht ist bereits durch den Vorspann gesetzt.

%% latex-abspann.tex
%% Gemeinsamer Abspann für Korrekturansicht und Leseansicht.
%% Setzt den Schalter \ifkorrekturansicht voraus (gesetzt in den
%% einbindenden Dateien latex-korrekturansicht-abspann.tex bzw.
%% latex-leseansicht-abspann.tex).
%% ---------------------------------------------------------------

\normalsize

% Das esempio-Environment wird nur in der Leseansicht benötigt
\ifkorrekturansicht\else
\newenvironment{esempio}[3]%
{
    \vspace{1.5ex}
    \rlap{\underline{#1}}
    \par
    \setlength{\parindent}{0cm}
    \nopagebreak
    \leftskip=#2cm
    \rightskip=#3cm
}
{
    \par
}
\fi

\doendnotes{C}
\bigskip
\vfill

\clearpage

\footnotesize

\ifkorrekturansicht
  \lohead{\textsc{register}}
\fi

% theindex-Environment neu definieren ohne reledmac
\makeatletter
\renewenvironment{theindex}{%
  \ifkorrekturansicht
    \section*{\indexname}%
  \else
    \subsubsection*{Index der erwähnten Entitäten}%
  \fi
  \setlength{\parindent}{0pt}%
  \setlength{\parskip}{0pt plus 0.3pt}%
  \let\item\@idxitem
}{%
  \ifkorrekturansicht\clearpage\fi
}
\makeatother

\IfFileExists{\jobname-pw.ind}{\input{\jobname-pw.ind}}{}

% Quellenangabe nur in der Leseansicht
\ifkorrekturansicht\else
% Fallback-Definitionen, falls die .tex-Datei \titel etc. nicht gesetzt hat
\providecommand{\titel}{}
\providecommand{\editorInnen}{}
\providecommand{\dateiname}{\jobname}

\vspace{3cm}

\vfill

\footnotesize
\textsc{Quelle}: \titel. Herausgegeben von {\editorInnen}. In: \emph{Arthur Schnitzler: Briefwechsel mit Autorinnen und Autoren}.
 Digitale Edition, https://schnitzler-briefe.acdh.oeaw.ac.at/{\dateiname}.html (Stand \today)
\fi

\end{document}


      