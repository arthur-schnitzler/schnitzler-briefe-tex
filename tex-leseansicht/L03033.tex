%% latex-leseansicht-vorspann.tex
%% Vorspann für die Leseansicht.
%% Lädt die gemeinsame Datei latex-vorspann.tex mit nicht gesetztem Schalter.

\newif\ifkorrekturansicht
\korrekturansichtfalse

\input{../tex-inputs/latex-vorspann}


\section[ Arthur Schnitzler an Felix Salten, [14. 10. 1898?]]{L03033 Arthur Schnitzler an Felix Salten,  [14. 10. 1898?]}
\nopagebreak\mylabel{L03033v}
\rehead{ }\normalsize\beginnumbering\briefempfaengerindex{Salten, Felix@\textsc{Salten, Felix}!zzzSchnitzler, Arthur@\emph{von Arthur Schnitzler}!1898-10-143@{{[}14. 10. 1898?{]}}|(be}
\toendnotes[C]{\smallbreak\pagebreak[2]}
\correspDesc{Versand  durch Arthur Schnitzler am [14. 10. 1898?] in Wien
\newline{}Erhalt  durch Felix Salten im Zeitraum [14. 10. 1898 – 17. 10. 1898?] in Wien}\toendnotes[C]{\smallbreak}
\Standort{Wienbibliothek im Rathaus, ZPH 1681, 2.1.516.}
\physDesc{Karte, 233 Zeichen
\newline{}Handschrift: Bleistift, deutsche Kurrent
\newline{}Ordnung: mit Bleistift von unbekannter Hand nummeriert: »35« }\toendnotes[C]{\smallbreak}
\pstart
           \noindent{}{\pb}Lieber Freund, vielleicht{ }ſind Sie \label{K_L03033-1v}\edtext{morgen nach dem weißen
                  Röſſl\pwindex{\textcolor{red}{\textsuperscript{XXXX indx1}}!Im weißen Rößl. Schwank in drei Acten@\strich\emph{Im weißen Rößl. Schwank in drei Acten}|pw}\pwindex{\textcolor{red}{\textsuperscript{XXXX indx1}}!Im weißen Rößl. Schwank in drei Acten@\strich\emph{Im weißen Rößl. Schwank in drei Acten}|pw}}{\lemma{\textnormal{\emph{morgen … Rössl}}}\Cendnote{\textnormal{Das erlaubt die Datierung des
                  Korrespondenzstücks. Die Premiere von \emph{Im weißen
                     Rößl}\pwindex{\textcolor{red}{\textsuperscript{XXXX indx1}}!Im weißen Rößl. Schwank in drei Acten@\strich\emph{Im weißen Rößl. Schwank in drei Acten}|pwk}\pwindex{\textcolor{red}{\textsuperscript{XXXX indx1}}!Im weißen Rößl. Schwank in drei Acten@\strich\emph{Im weißen Rößl. Schwank in drei Acten}|pwk} fand am 15. 10. 1898 statt. 
                  Schnitzler  besuchte die Aufführung nicht,
                  sondern erst jene am 11. 11. 1898. Trotzdem 
                  dürfte er von der Premiere und nicht vom eigenen Besuch sprechen, da er sich in Folge auf zwei wenige Tage
                  zuvor erschienene Artikel Saltens\pwindex{Salten, Felix 6.\,9.\,1869 Budapest – 8.\,10.\,1945 Zürich@\textsc{Salten, Felix} (6.\,9.\,1869 Budapest – 8.\,10.\,1945 Zürich), \emph{Schriftsteller, Journalist, Chefredakteur}|pwk} bezieht,
                  über die er einen Monat später längst mit Salten\pwindex{Salten, Felix 6.\,9.\,1869 Budapest – 8.\,10.\,1945 Zürich@\textsc{Salten, Felix} (6.\,9.\,1869 Budapest – 8.\,10.\,1945 Zürich), \emph{Schriftsteller, Journalist, Chefredakteur}|pwk}
                  gesprochen haben dürfte.}}}\label{K_L03033-1} im
                  \label{K_L03033-2v}\edtext{Pucher\oindex{Wien@\textbf{Wien}!I., Innere Stadt@\textbf{I., Innere Stadt}!Café Pucher@\textbf{Café Pucher}, \emph{Kaffeehaus}|pw}}{\lemma{\textnormal{\emph{Pucher}}}\Cendnote{\textnormal{nicht belegt}}}\label{K_L03033-2}? – Sehr{ }ſchön
               haben Sie über die \label{K_L03033-3v}\edtext{Mutter Erde\pwindex{\textcolor{red}{\textsuperscript{XXXX indx1}}!Mutter Erde. Drama in fünf Acten@\strich\emph{Mutter Erde. Drama in fünf Acten}|pw}}{\lemma{\textnormal{\emph{Mutter Erde}}}\Cendnote{\textnormal{Felix Salten\pwindex{Salten, Felix 6.\,9.\,1869 Budapest – 8.\,10.\,1945 Zürich@\textsc{Salten, Felix} (6.\,9.\,1869 Budapest – 8.\,10.\,1945 Zürich), \emph{Schriftsteller, Journalist, Chefredakteur}|pwk}: \emph{Deutsches Volkstheater. (»Mutter Erde«, Drama in fünf Acten
                     von Max Halbe. – Zum erstenmale am 8. October 1898.)}\pwindex{Salten, Felix 6.\,9.\,1869 Budapest – 8.\,10.\,1945 Zürich@\textsc{Salten, Felix} (6.\,9.\,1869 Budapest – 8.\,10.\,1945 Zürich), \emph{Schriftsteller, Journalist, Chefredakteur}!Deutsches Volkstheater. (»Mutter Erde«, Drama in fünf Acten von Max Halbe. – Zum erstenmale am 8. October 1898.)@\strich\emph{Deutsches Volkstheater. (»Mutter Erde«, Drama in fünf Acten von Max Halbe. – Zum erstenmale am 8. October 1898.)}|pwk} In: \emph{Wiener Allgemeine Zeitung}\pwindex{Wiener Allgemeine Zeitung@\emph{Wiener Allgemeine Zeitung}|pwk}, Nr. 6183, 11. 10. 1898, S. 2.
               }}}\label{K_L03033-3} u den \label{K_L03033-4v}\edtext{Cyrano\pwindex{\textcolor{red}{\textsuperscript{XXXX indx1}}!Cyrano von Bergerac. Romantische Komödie in fünf Aufzügen@\strich\emph{Cyrano von Bergerac. Romantische Komödie in fünf Aufzügen}|pw}}{\lemma{\textnormal{\emph{Cyrano}}}\Cendnote{\textnormal{Felix Salten\pwindex{Salten, Felix 6.\,9.\,1869 Budapest – 8.\,10.\,1945 Zürich@\textsc{Salten, Felix} (6.\,9.\,1869 Budapest – 8.\,10.\,1945 Zürich), \emph{Schriftsteller, Journalist, Chefredakteur}|pwk}: \emph{Burgtheater. (»Cyrano von Bergerac«, romantische Komödie in
                              fünf Aufzügen von Edmond Rostand, deutsch von Ludwig Fulda. – Zum erstenmale
                              aufgeführt am 11. October 1898.)}\pwindex{Salten, Felix 6.\,9.\,1869 Budapest – 8.\,10.\,1945 Zürich@\textsc{Salten, Felix} (6.\,9.\,1869 Budapest – 8.\,10.\,1945 Zürich), \emph{Schriftsteller, Journalist, Chefredakteur}!Burgtheater. (»Cyrano von Bergerac«, romantische Komödie in fünf Aufzügen von Edmond Rostand, deutsch von Ludwig Fulda. – Zum erstenmale aufgeführt am 11. October 1898.)@\strich\emph{Burgtheater. (»Cyrano von Bergerac«, romantische Komödie in fünf Aufzügen von Edmond Rostand, deutsch von Ludwig Fulda. – Zum erstenmale aufgeführt am 11. October 1898.)}|pwk} In: \emph{Wiener Allgemeine Zeitung}\pwindex{Wiener Allgemeine Zeitung@\emph{Wiener Allgemeine Zeitung}|pwk}, Nr. 6185, 13. 10. 1898, S. 2–3.}}}\label{K_L03033-4}{ }geſchrieben\pwindex{Salten, Felix 6.\,9.\,1869 Budapest – 8.\,10.\,1945 Zürich@\textsc{Salten, Felix} (6.\,9.\,1869 Budapest – 8.\,10.\,1945 Zürich), \emph{Schriftsteller, Journalist, Chefredakteur}!Deutsches Volkstheater. (»Mutter Erde«, Drama in fünf Acten von Max Halbe. – Zum erstenmale am 8. October 1898.)@\strich\emph{Deutsches Volkstheater. (»Mutter Erde«, Drama in fünf Acten von Max Halbe. – Zum erstenmale am 8. October 1898.)}|pwv}\pwindex{Salten, Felix 6.\,9.\,1869 Budapest – 8.\,10.\,1945 Zürich@\textsc{Salten, Felix} (6.\,9.\,1869 Budapest – 8.\,10.\,1945 Zürich), \emph{Schriftsteller, Journalist, Chefredakteur}!Burgtheater. (»Cyrano von Bergerac«, romantische Komödie in fünf Aufzügen von Edmond Rostand, deutsch von Ludwig Fulda. – Zum erstenmale aufgeführt am 11. October 1898.)@\strich\emph{Burgtheater. (»Cyrano von Bergerac«, romantische Komödie in fünf Aufzügen von Edmond Rostand, deutsch von Ludwig Fulda. – Zum erstenmale aufgeführt am 11. October 1898.)}|pwv}
                – beide {\pb}Mal gleich dorthin gegriffen, wo die Dinge zu
               faſſen{ }ſind.\pend
           
\pstart
           Auf Wiederſehen {\\[\baselineskip]}Herzlichſt Ihr {\\[\baselineskip]}\spacefill\mbox{A. S.}\pend
           \leftskip=0em{}\selectlanguage{ngerman}\endnumbering\briefempfaengerindex{Salten, Felix@\textsc{Salten, Felix}!zzzSchnitzler, Arthur@\emph{von Arthur Schnitzler}!1898-10-143@{{[}14. 10. 1898?{]}}|)be}\mylabel{L03033h}  \newcommand{\dateiname}{L03033}\newcommand{\titel}{Arthur Schnitzler an Felix Salten, [14. 10. 1898?]}\newcommand{\editorInnen}{Martin Anton Müller und Laura Untner}%% latex-leseansicht-abspann.tex
%% Abspann für die Leseansicht.
%% Der Schalter \ifkorrekturansicht ist bereits durch den Vorspann gesetzt.

%% latex-abspann.tex
%% Gemeinsamer Abspann für Korrekturansicht und Leseansicht.
%% Setzt den Schalter \ifkorrekturansicht voraus (gesetzt in den
%% einbindenden Dateien latex-korrekturansicht-abspann.tex bzw.
%% latex-leseansicht-abspann.tex).
%% ---------------------------------------------------------------

\normalsize

% Das esempio-Environment wird nur in der Leseansicht benötigt
\ifkorrekturansicht\else
\newenvironment{esempio}[3]%
{
    \vspace{1.5ex}
    \rlap{\underline{#1}}
    \par
    \setlength{\parindent}{0cm}
    \nopagebreak
    \leftskip=#2cm
    \rightskip=#3cm
}
{
    \par
}
\fi

\doendnotes{C}
\bigskip
\vfill

\clearpage

\footnotesize

\ifkorrekturansicht
  \lohead{\textsc{register}}
\fi

% theindex-Environment neu definieren ohne reledmac
\makeatletter
\renewenvironment{theindex}{%
  \ifkorrekturansicht
    \section*{\indexname}%
  \else
    \subsubsection*{Index der erwähnten Entitäten}%
  \fi
  \setlength{\parindent}{0pt}%
  \setlength{\parskip}{0pt plus 0.3pt}%
  \let\item\@idxitem
}{%
  \ifkorrekturansicht\clearpage\fi
}
\makeatother

\IfFileExists{\jobname-pw.ind}{\input{\jobname-pw.ind}}{}

% Quellenangabe nur in der Leseansicht
\ifkorrekturansicht\else
% Fallback-Definitionen, falls die .tex-Datei \titel etc. nicht gesetzt hat
\providecommand{\titel}{}
\providecommand{\editorInnen}{}
\providecommand{\dateiname}{\jobname}

\vspace{3cm}

\vfill

\footnotesize
\textsc{Quelle}: \titel. Herausgegeben von {\editorInnen}. In: \emph{Arthur Schnitzler: Briefwechsel mit Autorinnen und Autoren}.
 Digitale Edition, https://schnitzler-briefe.acdh.oeaw.ac.at/{\dateiname}.html (Stand \today)
\fi

\end{document}


