%% latex-leseansicht-vorspann.tex
%% Vorspann für die Leseansicht.
%% Lädt die gemeinsame Datei latex-vorspann.tex mit nicht gesetztem Schalter.

\newif\ifkorrekturansicht
\korrekturansichtfalse

\input{../tex-inputs/latex-vorspann}


\section[ Paul Goldmann an Arthur Schnitzler, 14. 7. {[}1896{]}]{L02781 Paul Goldmann an Arthur Schnitzler,  14. 7. [1896]}
\nopagebreak\mylabel{L02781v}
\rehead{ }\normalsize\beginnumbering\briefempfaengerindex{Schnitzler, Arthur@\textsc{Schnitzler, Arthur}!zzzGoldmann, Paul@\emph{von Paul Goldmann}!1896-07-141@{14. 7. [1896]}|(be}
\toendnotes[C]{\smallbreak\pagebreak[2]}
\correspDesc{Versand  durch Paul Goldmann am 14. 7. [1896] in Paris
\newline{}Erhalt  durch Arthur Schnitzler am [24.?] 7. 1896 in Oslo}\toendnotes[C]{\smallbreak}
\Standort{DLA, A:Schnitzler, HS.NZ85.1.3166.}
\physDesc{Brief, 2 Blätter, 7 Seiten, 2602 Zeichen
\newline{}Handschrift: blaue Tinte, deutsche Kurrent
\newline{}Schnitzler: 1) mit Bleistift das Jahr »96« vermerkt  2) mit rotem Buntstift zwei Unterstreichungen}\toendnotes[C]{\smallbreak}
\pstart
           {\pb}\textcolor{gray}{\textbf{\textbf{Frankfurter Zeitung\orgindex{Frankfurter Zeitung@Frankfurter Zeitung|pw}}}}\pend
           
\pstart
           \textcolor{gray}{\textbf{(\begin{otherlanguage}{french}Gazette de Francfort\end{otherlanguage}\orgindex{Frankfurter Zeitung@Frankfurter Zeitung|pw}).}}\pend
           
\pstart
           \textcolor{gray}{\textbf{\textbf{\begin{otherlanguage}{french}Fondateur M.\end{otherlanguage}{ }L. Sonnemann\pwindex{Sonnemann, Leopold 29.\,10.\,1831 Höchberg – 30.\,10.\,1909 Frankfurt am Main@\textsc{Sonnemann, Leopold} (29.\,10.\,1831 Höchberg – 30.\,10.\,1909 Frankfurt am Main), \emph{Journalist, Herausgeber}|pw}.}}}\pend
           
\pstart
           \begin{otherlanguage}{french}\textcolor{gray}{\textbf{Journal\pwindex{Frankfurter Zeitung@\emph{Frankfurter Zeitung}|pwv} politique,
                        financier,}}\end{otherlanguage}\pend
           
\pstart
           \begin{otherlanguage}{french}\textcolor{gray}{\textbf{commercial et littéraire.}}\end{otherlanguage}\pend
           
\pstart
           \begin{otherlanguage}{french}\textcolor{gray}{\textbf{\textbf{Paraissant trois fois par jour.}}}\end{otherlanguage}\pend
           
\pstart
           \begin{otherlanguage}{french}\textcolor{gray}{\textbf{\textbf{Bureau à Paris\oindex{Paris@\textbf{Paris}, \emph{Hauptstadt}|pw}}}}\end{otherlanguage}\hfill \textsc{Paris\oindex{Paris@\textbf{Paris}, \emph{Hauptstadt}|pw}}, 14. Juli.\pend
           
\pstart
           \begin{otherlanguage}{french}\textcolor{gray}{\textbf{\textbf{24. Rue Feydeau\oindex{rue Feydeau@\textbf{rue Feydeau}, \emph{Straße}|pw}.}}}\end{otherlanguage}\pend
           
\pstart\center{}Mein lieber Freund,\pend\vspace{0.5em}
\pstart
           Da Du mir{ }ſchreibſt, daß Norwegen\oindex{Norwegen@\textbf{Norwegen}|pw} wirklich
               exiſtirt, muß ichs wohl glauben und{ }ſchreibe Dir nach \label{K_L02781-1v}\edtext{\textsc{Christiania\oindex{Oslo@\textbf{Oslo}, \emph{Hauptstadt}|pwv}}}{\lemma{\textnormal{\emph{Christiania}}}\Cendnote{\textnormal{Schnitzler kam am 24. 7. 1896 in Christiania\oindex{Oslo@\textbf{Oslo}, \emph{Hauptstadt}|pwkv} (Oslo\oindex{Oslo@\textbf{Oslo}, \emph{Hauptstadt}|pwk}) an, las den Brief also vermutlich erst
                  rund zehn Tage später.}}}\label{K_L02781-1}, welches{ }ſich hoffentlich an Ort und Stelle auch
               wirklich als die Hauptſtadt\oindex{Oslo@\textbf{Oslo}, \emph{Hauptstadt}|pwv}
               dieſes unwahrſcheinlichen Land\oindex{Norwegen@\textbf{Norwegen}|pwv}es erweiſt.\pend
           
\pstart
           Ich danke Dir für Deine lieben Nachrichten. Deine Karten athmen frohe Reiſeſtimmung,
               und ich freue mich deſſen.\pend
           
\pstart
           {\pb}Nur möchte ich auch einmal etwas Genaueres über
               unſer \label{K_L02781-2v}\edtext{Zuſammentreffen}{\lemma{\textnormal{\emph{Zusammentreffen}}}\Cendnote{\textnormal{Siehe XXXX Auszeichnungsfehler: Dokument L02772 nicht gefunden.
               }}}\label{K_L02781-2} wiſſen. Werden wir uns{ }ſo zwiſchen erſtem und
                  fünftem Auguſt in Kopenhagen\oindex{Kopenhagen@\textbf{Kopenhagen}, \emph{Hauptstadt}|pw} treffen? Ich weiß zwar noch immer nicht, wann und ob ich von hier
               fortkomme (Geld, Geld, Geld!), – auch kann es in dieſem Lande\oindex{Frankreich@\textbf{Frankreich}|pwv} während vierzehn Tagen{ }ſtets \strikeout{ſ\textcolor{gray}{p}\textcolor{gray}{×}\-\textcolor{gray}{×}\-\textcolor{gray}{×}\textcolor{gray}{iren}} paſſiren, daß Herr \label{K_L02781-3v}\edtext{\textsc{Felix Faure\pwindex{Faure, Félix 30.\,1.\,1841 Paris – 16.\,2.\,1899 ebd.@\textsc{Faure, Félix} (30.\,1.\,1841 Paris – 16.\,2.\,1899 ebd.), \emph{Politiker, Präsident}|pw}}}{\lemma{\textnormal{\emph{Felix Faure}}}\Cendnote{\textnormal{fran\oindex{Frankreich@\textbf{Frankreich}|pwkv}zösischer Präsident\pwindex{Faure, Félix 30.\,1.\,1841 Paris – 16.\,2.\,1899 ebd.@\textsc{Faure, Félix} (30.\,1.\,1841 Paris – 16.\,2.\,1899 ebd.), \emph{Politiker, Präsident}|pwkv}
                     (1895–1899)}}}\label{K_L02781-3} den Sonnenſtich bekommt oder der
                  Herzog von \textsc{Orleans\oindex{Orléans@\textbf{Orléans}|pw}}\pwindex{Louis Philippe Robert d’Orléans, duc d’Orléans 6.\,2.\,1869 Twickenham – 28.\,3.\,1926 Palermo@\textsc{Louis Philippe Robert d’Orléans, duc d’Orléans} (6.\,2.\,1869 Twickenham – 28.\,3.\,1926 Palermo), \emph{Thronprätendent}|pwv} den Thron von Frankreich\oindex{Frankreich@\textbf{Frankreich}|pw}{ }{\pb}beſteigt – aber immerhin, wenn ich doch nach Dänemark\oindex{Dänemark@\textbf{Dänemark}|pw} käme, wäre es doch vielleicht nicht
               übel, \strikeout{f\textcolor{gray}{als}} falls wir uns dort treffen könnten, und zu dieſem Zweck müßte ich zunächſt
               einmal wiſſen, \label{K_L02781-4v}\edtext{wo Ihr\pwindex{Beer-Hofmann, Richard 11.\,7.\,1866 Wien – 26.\,9.\,1945 New York City@\textsc{Beer-Hofmann, Richard} (11.\,7.\,1866 Wien – 26.\,9.\,1945 New York City), \emph{Schriftsteller}|pwv}{ }ſeid}{\lemma{\textnormal{\emph{wo Ihr seid}}}\Cendnote{\textnormal{Zu diesem Zeitpunkt war Schnitzler noch auf dem Schiff unterwegs und besuchte diverse norweg\oindex{Norwegen@\textbf{Norwegen}|pwkv}ische Städte.}}}\label{K_L02781-4},
               was Ihr mir bisher mit anerkennenswerther Beharrlichkeit verſchwiegen habt.\pend
           
\pstart
           Kürzlich wollte ich noch \textsc{Thorel\pwindex{Thorel, Jean 11.\,9.\,1859 Éragny – 20.\,8.\,1916 Enghien-les-Bains@\textsc{Thorel, Jean} (11.\,9.\,1859 Éragny – 20.\,8.\,1916 Enghien-les-Bains), \emph{Übersetzer, Dramatiker}|pw}} – der gegenwärtig bei \label{K_L02781-5v}\edtext{\textsc{Pierre Loti\pwindex{Loti, Pierre 14.\,1.\,1850 Rochefort – 10.\,6.\,1923 Hendaye@\textsc{Loti, Pierre} (14.\,1.\,1850 Rochefort – 10.\,6.\,1923 Hendaye), \emph{Schriftsteller}|pw}} an der ſpani\oindex{Spanien@\textbf{Spanien}|pwv}ſchen
                  Grenze}{\lemma{\textnormal{\emph{Pierre … Grenze}}}\Cendnote{\textnormal{Loti\pwindex{Loti, Pierre 14.\,1.\,1850 Rochefort – 10.\,6.\,1923 Hendaye@\textsc{Loti, Pierre} (14.\,1.\,1850 Rochefort – 10.\,6.\,1923 Hendaye), \emph{Schriftsteller}|pwk} lebte seit 1892 in Hendaye\oindex{Hendaye@\textbf{Hendaye}|pwk}.}}}\label{K_L02781-5} iſt – zu
                  \textsc{Antoine\pwindex{Antoine, André 31.\,1.\,1858 Limoges – 23.\,10.\,1943 Le Pouliguen@\textsc{Antoine, André} (31.\,1.\,1858 Limoges – 23.\,10.\,1943 Le Pouliguen), \emph{Theaterleiter, Schauspieler}|pw}}{ }ſchicken. Aber er meinte, mit \textsc{Antoine\pwindex{Antoine, André 31.\,1.\,1858 Limoges – 23.\,10.\,1943 Le Pouliguen@\textsc{Antoine, André} (31.\,1.\,1858 Limoges – 23.\,10.\,1943 Le Pouliguen), \emph{Theaterleiter, Schauspieler}|pw}}{ }ſei fürs Erſte {\pb}nichts zu machen, derſelbe{ }ſei
               verrückter als je, habe keine Ahnung, was er wolle, und nehme als deutſche Stücke
               zunächſt nur \textsc{Wallenstein\pwindex{Schiller, Friedrich von 10.\,11.\,1759 Marbach am Neckar – 9.\,5.\,1805 Weimar@\textsc{Schiller, Friedrich von} (10.\,11.\,1759 Marbach am Neckar – 9.\,5.\,1805 Weimar), \emph{Schriftsteller, Historiker, Philosoph}!Wallenstein@\strich\emph{Wallenstein}|pw}} und \textsc{Don Carlos\pwindex{Schiller, Friedrich von 10.\,11.\,1759 Marbach am Neckar – 9.\,5.\,1805 Weimar@\textsc{Schiller, Friedrich von} (10.\,11.\,1759 Marbach am Neckar – 9.\,5.\,1805 Weimar), \emph{Schriftsteller, Historiker, Philosoph}!Don Karlos, Infant von Spanien@\strich\emph{Don Karlos, Infant von Spanien}|pw}} in Ausſicht. Wenn man ihm glauben machen könnte, daß die »Liebelei\pwindex{Schnitzler, Arthur 15.\,5.\,1862 Wien – 21.\,10.\,1931 ebd.@\textsc{Schnitzler, Arthur} (15.\,5.\,1862 Wien – 21.\,10.\,1931 ebd.), \emph{Schriftsteller, Mediziner}!Liebelei. Schauspiel in drei Akten@\strich\emph{Liebelei. Schauspiel in drei Akten}|pw}« von \textsc{Schiller\pwindex{Schiller, Friedrich von 10.\,11.\,1759 Marbach am Neckar – 9.\,5.\,1805 Weimar@\textsc{Schiller, Friedrich von} (10.\,11.\,1759 Marbach am Neckar – 9.\,5.\,1805 Weimar), \emph{Schriftsteller, Historiker, Philosoph}|pw}} wäre,{ }ſo wäre die Sache{ }ſofort erledigt; aber das wird{ }ſchwer halten. Kurzum,
               wir müſſen bis zur »\label{K_L02781-6v}\edtext{\begin{otherlanguage}{french}\textsc{rentrée}\end{otherlanguage}}{\lemma{\textnormal{\emph{rentrée}}}\Cendnote{\textnormal{französisch: Rückkehr (nach der
                  Sommerpause)}}}\label{K_L02781-6}« warten, und \textsc{Thorel\pwindex{Thorel, Jean 11.\,9.\,1859 Éragny – 20.\,8.\,1916 Enghien-les-Bains@\textsc{Thorel, Jean} (11.\,9.\,1859 Éragny – 20.\,8.\,1916 Enghien-les-Bains), \emph{Übersetzer, Dramatiker}|pw}} möchte inzwiſchen die Überſetzung\pwindex{Schnitzler, Arthur 15.\,5.\,1862 Wien – 21.\,10.\,1931 ebd.@\textsc{Schnitzler, Arthur} (15.\,5.\,1862 Wien – 21.\,10.\,1931 ebd.), \emph{Schriftsteller, Mediziner}!Amourette. Pièce en trois actes. Adaptée de Arthur Schnitzler@\strich\emph{Amourette. Pièce en trois actes. Adaptée de Arthur Schnitzler}|pwv} anfertigen (Preis 5-600 \textsc{Francs}, – du
               verſtehſt?). {\pb}Wir reden darüber bald mündlich,{ }ſo
               Gott will.\pend
           
\pstart
           \strikeout{Sonſt} Vielen Dank für \textsc{Altenberg\pwindex{Altenberg, Peter 9.\,3.\,1859 Wien – 8.\,1.\,1919 ebd.@\textsc{Altenberg, Peter} (9.\,3.\,1859 Wien – 8.\,1.\,1919 ebd.), \emph{Schriftsteller}!Wie ich es sehe@\strich\emph{Wie ich es sehe}|pwv}\pwindex{Altenberg, Peter 9.\,3.\,1859 Wien – 8.\,1.\,1919 ebd.@\textsc{Altenberg, Peter} (9.\,3.\,1859 Wien – 8.\,1.\,1919 ebd.), \emph{Schriftsteller}|pw}}! Ich habe die erſten Seiten\pwindex{Altenberg, Peter 9.\,3.\,1859 Wien – 8.\,1.\,1919 ebd.@\textsc{Altenberg, Peter} (9.\,3.\,1859 Wien – 8.\,1.\,1919 ebd.), \emph{Schriftsteller}!Wie ich es sehe@\strich\emph{Wie ich es sehe}|pwv} geleſen und weiß noch nicht recht, wo und wie? Manchmal \strikeout{\textcolor{gray}{mei}} meint man, es{ }ſei ein Dichter, manchmal meint man, es{ }ſei \textsc{Hermann Bahr\pwindex{Bahr, Hermann 19.\,7.\,1863 Linz – 15.\,1.\,1934 München@\textsc{Bahr, Hermann} (19.\,7.\,1863 Linz – 15.\,1.\,1934 München), \emph{Schriftsteller, Kritiker}|pwv}}. Aber jedenfalls leſe ich das Buch\pwindex{Altenberg, Peter 9.\,3.\,1859 Wien – 8.\,1.\,1919 ebd.@\textsc{Altenberg, Peter} (9.\,3.\,1859 Wien – 8.\,1.\,1919 ebd.), \emph{Schriftsteller}!Wie ich es sehe@\strich\emph{Wie ich es sehe}|pwv} zu Ende.\pend
           
\pstart
           Auf Deiner \label{K_L02781-7v}\edtext{Karte}{\lemma{\textnormal{\emph{Karte}}}\Cendnote{\textnormal{Es dürfte sich um das gleiche
                  Postkartenmotiv handeln, das Schnitzler am
                     XXXX Auszeichnungsfehler: Dokument L00561 nicht gefunden an Beer-Hofmann\pwindex{Beer-Hofmann, Richard 11.\,7.\,1866 Wien – 26.\,9.\,1945 New York City@\textsc{Beer-Hofmann, Richard} (11.\,7.\,1866 Wien – 26.\,9.\,1945 New York City), \emph{Schriftsteller}|pwk} gesandt hat (siehe XXXX Auszeichnungsfehler: Dokument L00561 nicht gefunden).}}}\label{K_L02781-7} fand ich ein
               roth angeſtrichenes {\pb}Schiff, über dem ein\strikeout{\textcolor{gray}{e}} blaues Geſtirn{ }ſchwebt, das in erklärender Unterſchrift den Beſchauer als
                  »\label{K_L02781-8v}\edtext{\begin{otherlanguage}{french}\textsc{soleil de minuit}\end{otherlanguage}}{\lemma{\textnormal{\emph{soleil de minuit}}}\Cendnote{\textnormal{französisch: Mitternachtssonne}}}\label{K_L02781-8}«
               vorgeſtellt wird. Das Schiff iſt vor \strikeout{de\textcolor{gray}{m}} der Mitternachtsſonne vorgefahren, wie ein Hotel-Omnibus vor der Hausthür des
               Gaſthofes. Nicht genug damit,{ }ſteht auch noch das \label{K_L02781-9v}\edtext{Nordcap\oindex{Nordkap@\textbf{Nordkap}, \emph{Kap}|pw}}{\lemma{\textnormal{\emph{Nordcap}}}\Cendnote{\textnormal{Schnitzler kam am 19. 7. 1896 an das Nordkap\oindex{Nordkap@\textbf{Nordkap}, \emph{Kap}|pwk}. }}}\label{K_L02781-9} dabei. Herrgott, biſt Du ein
               Protz! {\dotsfour}\pend
           
\pstart
           Blonde Kinder mit Märchenhaar! Das weckt {\pb}in meinem
               Herzen die Sehnſucht auf. Nur einmal{ }ſolch’ ein Mädchen in die Arme{ }ſchließen und
               hören, daß{ }ſie mich liebt! Einmal nur, – raſch noch in der letzten Viertelſunde
               dieſer{ }ſo ganz verlorenen Jugend! {\dotsfour}\pend
           
\pstart
           Grüß’ Dich Gott, mein theurer Freund, und reiſe froh und glücklich!\pend
           
\pstart
           Dein treuer {\\[\baselineskip]}\spacefill\mbox{Paul Goldmnn}\pend
           \leftskip=0em{}\selectlanguage{ngerman}\endnumbering\briefempfaengerindex{Schnitzler, Arthur@\textsc{Schnitzler, Arthur}!zzzGoldmann, Paul@\emph{von Paul Goldmann}!1896-07-141@{14. 7. [1896]}|)be}\mylabel{L02781h}  \newcommand{\dateiname}{L02781}\newcommand{\titel}{Paul Goldmann an Arthur Schnitzler, 14. 7. [1896]}\newcommand{\editorInnen}{Martin Anton Müller und Laura Untner}%% latex-leseansicht-abspann.tex
%% Abspann für die Leseansicht.
%% Der Schalter \ifkorrekturansicht ist bereits durch den Vorspann gesetzt.

%% latex-abspann.tex
%% Gemeinsamer Abspann für Korrekturansicht und Leseansicht.
%% Setzt den Schalter \ifkorrekturansicht voraus (gesetzt in den
%% einbindenden Dateien latex-korrekturansicht-abspann.tex bzw.
%% latex-leseansicht-abspann.tex).
%% ---------------------------------------------------------------

\normalsize

% Das esempio-Environment wird nur in der Leseansicht benötigt
\ifkorrekturansicht\else
\newenvironment{esempio}[3]%
{
    \vspace{1.5ex}
    \rlap{\underline{#1}}
    \par
    \setlength{\parindent}{0cm}
    \nopagebreak
    \leftskip=#2cm
    \rightskip=#3cm
}
{
    \par
}
\fi

\doendnotes{C}
\bigskip
\vfill

\clearpage

\footnotesize

\ifkorrekturansicht
  \lohead{\textsc{register}}
\fi

% theindex-Environment neu definieren ohne reledmac
\makeatletter
\renewenvironment{theindex}{%
  \ifkorrekturansicht
    \section*{\indexname}%
  \else
    \subsubsection*{Index der erwähnten Entitäten}%
  \fi
  \setlength{\parindent}{0pt}%
  \setlength{\parskip}{0pt plus 0.3pt}%
  \let\item\@idxitem
}{%
  \ifkorrekturansicht\clearpage\fi
}
\makeatother

\IfFileExists{\jobname-pw.ind}{\input{\jobname-pw.ind}}{}

% Quellenangabe nur in der Leseansicht
\ifkorrekturansicht\else
% Fallback-Definitionen, falls die .tex-Datei \titel etc. nicht gesetzt hat
\providecommand{\titel}{}
\providecommand{\editorInnen}{}
\providecommand{\dateiname}{\jobname}

\vspace{3cm}

\vfill

\footnotesize
\textsc{Quelle}: \titel. Herausgegeben von {\editorInnen}. In: \emph{Arthur Schnitzler: Briefwechsel mit Autorinnen und Autoren}.
 Digitale Edition, https://schnitzler-briefe.acdh.oeaw.ac.at/{\dateiname}.html (Stand \today)
\fi

\end{document}


