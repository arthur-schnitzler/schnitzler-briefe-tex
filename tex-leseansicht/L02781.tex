%% latex-korrekturansicht-vorspann.tex
%% Vorspann für die Korrekturansicht.
%% Lädt die gemeinsame Datei latex-vorspann.tex mit gesetztem Schalter.

\newif\ifkorrekturansicht
\korrekturansichttrue

\input{../tex-inputs/latex-vorspann}


\section[ Paul Goldmann an Arthur Schnitzler, 14. 7. {[}1896{]}]{L02781 Paul Goldmann an Arthur Schnitzler, 14. 7. {[}1896{]}}
\nopagebreak\mylabel{L02781v}
\rehead{ }\normalsize\beginnumbering\briefempfaengerindex{Schnitzler, Arthur@\textsc{Schnitzler, Arthur}!zzzGoldmann, Paul@\emph{von Paul Goldmann}!1896-07-141@{14. 7. {[}1896{]}}|(be}
\toendnotes[C]{\smallbreak\pagebreak[2]}\Standort{DLA, A:Schnitzler, HS.NZ85.1.3166.}
\physDesc{Brief, 2 Blätter, 7 Seiten, 2602 Zeichen
\newline{}Handschrift: blaue Tinte, deutsche Kurrent
\newline{}Schnitzler: 1) mit Bleistift das Jahr »96« vermerkt  2) mit rotem Buntstift zwei Unterstreichungen}\toendnotes[C]{\smallbreak}
\pstart
           {\pb}\textcolor{gray}{\textbf{\textbf{Frankfurter Zeitung\orgindex{Frankfurter Zeitung@Frankfurter Zeitung|pw}}}}\pend
           
\pstart
           \textcolor{gray}{\textbf{(\begin{otherlanguage}{french}Gazette de Francfort\end{otherlanguage}\orgindex{Frankfurter Zeitung@Frankfurter Zeitung|pw}).}}\pend
           
\pstart
           \textcolor{gray}{\textbf{\textbf{\begin{otherlanguage}{french}Fondateur M.\end{otherlanguage}{ }L. Sonnemann\pwindex{Sonnemann, Leopold 1831-10-29 – 1909-10-30@\textsc{Sonnemann, Leopold} (1831-10-29 – 1909-10-30), \emph{Journalist/Journalistin, Herausgeber/Herausgeberin}|pw}.}}}\pend
           
\pstart
           \begin{otherlanguage}{french}\textcolor{gray}{\textbf{Journal\pwindex{Frankfurter Zeitung@\emph{Frankfurter Zeitung}|pwv} politique,
                        financier,}}\end{otherlanguage}\pend
           
\pstart
           \begin{otherlanguage}{french}\textcolor{gray}{\textbf{commercial et littéraire.}}\end{otherlanguage}\pend
           
\pstart
           \begin{otherlanguage}{french}\textcolor{gray}{\textbf{\textbf{Paraissant trois fois par jour.}}}\end{otherlanguage}\pend
           
\pstart
           \begin{otherlanguage}{french}\textcolor{gray}{\textbf{\textbf{Bureau à Paris\oindex{Paris@\textbf{Paris}, \emph{P.PPLC}|pw}}}}\end{otherlanguage}\hfill \textsc{Paris\oindex{Paris@\textbf{Paris}, \emph{P.PPLC}|pw}}, 14. Juli.\pend
           
\pstart
           \begin{otherlanguage}{french}\textcolor{gray}{\textbf{\textbf{24. Rue Feydeau\oindex{rue Feydeau@\textbf{rue Feydeau}, \emph{Straße (K.STR)}|pw}.}}}\end{otherlanguage}\pend
           
\pstart\center{}Mein lieber Freund,\pend\vspace{0.5em}
\pstart
           Da Du mir ſchreibſt, daß Norwegen\oindex{Norwegen@\textbf{Norwegen}, \emph{A.PCLI}|pw} wirklich
               exiſtirt, muß ichs wohl glauben und ſchreibe Dir nach \label{K_L02781-1v}\edtext{\textsc{Christiania\oindex{Oslo@\textbf{Oslo}, \emph{P.PPLC}|pwv}}}{\lemma{\textnormal{\emph{Christiania}}}\Cendnote{\textnormal{Schnitzler kam am 24. 7. 1896 in Christiania\oindex{Oslo@\textbf{Oslo}, \emph{P.PPLC}|pwkv} (Oslo\oindex{Oslo@\textbf{Oslo}, \emph{P.PPLC}|pwk}) an, las den Brief also vermutlich erst
                  rund zehn Tage später.}}}\label{K_L02781-1}, welches ſich hoffentlich an Ort und Stelle auch
               wirklich als die Hauptſtadt\oindex{Oslo@\textbf{Oslo}, \emph{P.PPLC}|pwv}
               dieſes unwahrſcheinlichen Land\oindex{Norwegen@\textbf{Norwegen}, \emph{A.PCLI}|pwv}es erweiſt.\pend
           
\pstart
           Ich danke Dir für Deine lieben Nachrichten. Deine Karten athmen frohe Reiſeſtimmung,
               und ich freue mich deſſen.\pend
           
\pstart
           {\pb}Nur möchte ich auch einmal etwas Genaueres über
               unſer \label{K_L02781-2v}\edtext{Zuſammentreffen}{\lemma{\textnormal{\emph{Zuſammentreffen}}}\Cendnote{\textnormal{Siehe Paul Goldmann an Arthur Schnitzler, 29. 4. [1896].
               }}}\label{K_L02781-2} wiſſen. Werden wir uns ſo zwiſchen erſtem und
                  fünftem Auguſt in Kopenhagen\oindex{Kopenhagen@\textbf{Kopenhagen}, \emph{P.PPLC}|pw} treffen? Ich weiß zwar noch immer nicht, wann und ob ich von hier
               fortkomme (Geld, Geld, Geld!), – auch kann es in dieſem Lande\oindex{Frankreich@\textbf{Frankreich}, \emph{A.PCLI}|pwv} während vierzehn Tagen ſtets \strikeout{ ſ\textcolor{gray}{p}\textcolor{gray}{×}\-\textcolor{gray}{×}\-\textcolor{gray}{×}\textcolor{gray}{iren}} paſſiren, daß Herr \label{K_L02781-3v}\edtext{\textsc{Felix Faure\pwindex{Faure, Felix 1841-01-30 – 1899-02-16@\textsc{Faure, Félix} (1841-01-30 – 1899-02-16), \emph{Politiker/Politikerin, Präsident/Präsidentin}|pw}}}{\lemma{\textnormal{\emph{Felix Faure}}}\Cendnote{\textnormal{fran\oindex{Frankreich@\textbf{Frankreich}, \emph{A.PCLI}|pwkv}zösischer Präsident\pwindex{Faure, Felix 1841-01-30 – 1899-02-16@\textsc{Faure, Félix} (1841-01-30 – 1899-02-16), \emph{Politiker/Politikerin, Präsident/Präsidentin}|pwkv}
                     (1895–1899)}}}\label{K_L02781-3} den Sonnenſtich bekommt oder der
                  Herzog von \textsc{Orleans\oindex{Orleans@\textbf{Orléans}, \emph{P.PPLA}|pw}}\pwindex{Louis Philippe Robert DOrleans, duc DOrleans 1869-02-06 – 1926-03-28@\textsc{Louis Philippe Robert d’Orléans, duc d’Orléans} (1869-02-06 – 1926-03-28), \emph{Thronprätendent/Thronprätendentin}|pwv} den Thron von Frankreich\oindex{Frankreich@\textbf{Frankreich}, \emph{A.PCLI}|pw}{ }{\pb}beſteigt – aber immerhin, wenn ich doch nach Dänemark\oindex{Daenemark@\textbf{Dänemark}, \emph{A.PCLI}|pw} käme, wäre es doch vielleicht nicht
               übel, \strikeout{f\textcolor{gray}{als}} falls wir uns dort treffen könnten, und zu dieſem Zweck müßte ich zunächſt
               einmal wiſſen, \label{K_L02781-4v}\edtext{wo Ihr\pwindex{Beer-Hofmann, Richard 1866-07-11 – 1945-09-26@\textsc{Beer-Hofmann, Richard} (1866-07-11 – 1945-09-26), \emph{Schriftsteller/Schriftstellerin}|pwv} ſeid}{\lemma{\textnormal{\emph{wo Ihr ſeid}}}\Cendnote{\textnormal{Zu diesem Zeitpunkt war Schnitzler noch auf dem Schiff unterwegs und besuchte diverse norweg\oindex{Norwegen@\textbf{Norwegen}, \emph{A.PCLI}|pwkv}ische Städte.}}}\label{K_L02781-4},
               was Ihr mir bisher mit anerkennenswerther Beharrlichkeit verſchwiegen habt.\pend
           
\pstart
           Kürzlich wollte ich noch \textsc{Thorel\pwindex{Thorel, Jean 1859-09-11 – 1916-08-20@\textsc{Thorel, Jean} (1859-09-11 – 1916-08-20), \emph{Übersetzer/Übersetzerin, Dramatiker/Dramatikerin}|pw}} – der gegenwärtig bei \label{K_L02781-5v}\edtext{\textsc{Pierre Loti\pwindex{Loti, Pierre 14.01.1850 – 10.06.1923@\textsc{Loti, Pierre} (14.01.1850 – 10.06.1923), \emph{Schriftsteller/Schriftstellerin}|pw}} an der ſpani\oindex{Spanien@\textbf{Spanien}, \emph{A.PCLI}|pwv}ſchen
                  Grenze}{\lemma{\textnormal{\emph{Pierre … Grenze}}}\Cendnote{\textnormal{Loti\pwindex{Loti, Pierre 14.01.1850 – 10.06.1923@\textsc{Loti, Pierre} (14.01.1850 – 10.06.1923), \emph{Schriftsteller/Schriftstellerin}|pwk} lebte seit 1892 in Hendaye\oindex{Hendaye@\textbf{Hendaye}, \emph{P.PPL}|pwk}.}}}\label{K_L02781-5} iſt – zu
                  \textsc{Antoine\pwindex{Antoine, Andre 1858-01-31 – 1943-10-23@\textsc{Antoine, André} (1858-01-31 – 1943-10-23), \emph{Theaterleiter/Theaterleiterin, Schauspieler/Schauspielerin}|pw}} ſchicken. Aber er meinte, mit \textsc{Antoine\pwindex{Antoine, Andre 1858-01-31 – 1943-10-23@\textsc{Antoine, André} (1858-01-31 – 1943-10-23), \emph{Theaterleiter/Theaterleiterin, Schauspieler/Schauspielerin}|pw}} ſei fürs Erſte {\pb}nichts zu machen, derſelbe ſei
               verrückter als je, habe keine Ahnung, was er wolle, und nehme als deutſche Stücke
               zunächſt nur \textsc{Wallenstein\pwindex{Wallenstein@\emph{Wallenstein}|pw}} und \textsc{Don Carlos\pwindex{Don Karlos, Infant von Spanien@\emph{Don Karlos, Infant von Spanien}|pw}} in Ausſicht. Wenn man ihm glauben machen könnte, daß die »Liebelei\pwindex{Liebelei. Schauspiel in drei Akten@\emph{Liebelei. Schauspiel in drei Akten}|pw}« von \textsc{Schiller\pwindex{Schiller, Friedrich von 10.11.1759 – 09.05.1805@\textsc{Schiller, Friedrich von} (10.11.1759 – 09.05.1805), \emph{Schriftsteller/Schriftstellerin, Historiker/Historikerin, Philosoph/Philosophin}|pw}} wäre, ſo wäre die Sache ſofort erledigt; aber das wird ſchwer halten. Kurzum,
               wir müſſen bis zur »\label{K_L02781-6v}\edtext{\begin{otherlanguage}{french}\textsc{rentrée}\end{otherlanguage}}{\lemma{\textnormal{\emph{rentrée}}}\Cendnote{\textnormal{französisch: Rückkehr (nach der
                  Sommerpause)}}}\label{K_L02781-6}« warten, und \textsc{Thorel\pwindex{Thorel, Jean 1859-09-11 – 1916-08-20@\textsc{Thorel, Jean} (1859-09-11 – 1916-08-20), \emph{Übersetzer/Übersetzerin, Dramatiker/Dramatikerin}|pw}} möchte inzwiſchen die Überſetzung\pwindex{Amourette. Piece en trois actes. Adaptee de Arthur Schnitzler@\emph{Amourette. Pièce en trois actes. Adaptée de Arthur Schnitzler}|pwv} anfertigen (Preis 5-600 \textsc{Francs}, – du
               verſtehſt?). {\pb}Wir reden darüber bald mündlich, ſo
               Gott will.\pend
           
\pstart
           \strikeout{Sonſt} Vielen Dank für \textsc{Altenberg\pwindex{Wie ich es sehe@\emph{Wie ich es sehe}|pwv}\pwindex{Altenberg, Peter 09.03.1859 – 08.01.1919@\textsc{Altenberg, Peter} (09.03.1859 – 08.01.1919), \emph{Schriftsteller/Schriftstellerin}|pw}}! Ich habe die erſten Seiten\pwindex{Wie ich es sehe@\emph{Wie ich es sehe}|pwv} geleſen und weiß noch nicht recht, wo und wie? Manchmal \strikeout{\textcolor{gray}{mei}} meint man, es ſei ein Dichter, manchmal meint man, es ſei \textsc{Hermann Bahr\pwindex{Bahr, Hermann 19.07.1863 – 15.01.1934@\textsc{Bahr, Hermann} (19.07.1863 – 15.01.1934), \emph{Schriftsteller/Schriftstellerin, Kritiker/Kritikerin}|pwv}}. Aber jedenfalls leſe ich das Buch\pwindex{Wie ich es sehe@\emph{Wie ich es sehe}|pwv} zu Ende.\pend
           
\pstart
           Auf Deiner \label{K_L02781-7v}\edtext{Karte}{\lemma{\textnormal{\emph{Karte}}}\Cendnote{\textnormal{Es dürfte sich um das gleiche
                  Postkartenmotiv handeln, das Schnitzler am
                     9. 7. 1896 an Beer-Hofmann\pwindex{Beer-Hofmann, Richard 1866-07-11 – 1945-09-26@\textsc{Beer-Hofmann, Richard} (1866-07-11 – 1945-09-26), \emph{Schriftsteller/Schriftstellerin}|pwk} gesandt hat (siehe Arthur Schnitzler an Richard Beer-Hofmann, 9. 7. 1896).}}}\label{K_L02781-7} fand ich ein
               roth angeſtrichenes {\pb}Schiff, über dem ein\strikeout{\textcolor{gray}{e}} blaues Geſtirn ſchwebt, das in erklärender Unterſchrift den Beſchauer als
                  »\label{K_L02781-8v}\edtext{\begin{otherlanguage}{french}\textsc{soleil de minuit}\end{otherlanguage}}{\lemma{\textnormal{\emph{soleil de minuit}}}\Cendnote{\textnormal{französisch: Mitternachtssonne}}}\label{K_L02781-8}«
               vorgeſtellt wird. Das Schiff iſt vor \strikeout{de\textcolor{gray}{m}} der Mitternachtsſonne vorgefahren, wie ein Hotel-Omnibus vor der Hausthür des
               Gaſthofes. Nicht genug damit, ſteht auch noch das \label{K_L02781-9v}\edtext{Nordcap\oindex{Nordkap@\textbf{Nordkap}, \emph{Kap (N.KAP)}|pw}}{\lemma{\textnormal{\emph{Nordcap}}}\Cendnote{\textnormal{Schnitzler kam am 19. 7. 1896 an das Nordkap\oindex{Nordkap@\textbf{Nordkap}, \emph{Kap (N.KAP)}|pwk}. }}}\label{K_L02781-9} dabei. Herrgott, biſt Du ein
               Protz! {\dotsfour}\pend
           
\pstart
           Blonde Kinder mit Märchenhaar! Das weckt {\pb}in meinem
               Herzen die Sehnſucht auf. Nur einmal ſolch’ ein Mädchen in die Arme ſchließen und
               hören, daß ſie mich liebt! Einmal nur, – raſch noch in der letzten Viertelſunde
               dieſer ſo ganz verlorenen Jugend! {\dotsfour}\pend
           
\pstart
           Grüß’ Dich Gott, mein theurer Freund, und reiſe froh und glücklich!\pend
           
\pstart
           Dein treuer {\\[\baselineskip]}\spacefill\mbox{Paul Goldmnn}\pend
           \leftskip=0em{}\selectlanguage{ngerman}\endnumbering\briefempfaengerindex{Schnitzler, Arthur@\textsc{Schnitzler, Arthur}!zzzGoldmann, Paul@\emph{von Paul Goldmann}!1896-07-141@{14. 7. {[}1896{]}}|)be}\mylabel{L02781h}  \normalsize

\doendnotes{C}
\bigskip
\vfill

\clearpage

\footnotesize

\lohead{\textsc{register}}

% Definiere theindex-Environment komplett neu ohne reledmac
\makeatletter
\renewenvironment{theindex}{%
  \section*{\indexname}%
  \setlength{\parindent}{0pt}%
  \setlength{\parskip}{0pt plus 0.3pt}%
  \let\item\@idxitem
}{%
  \clearpage
}
\makeatother

\IfFileExists{\jobname-pw.ind}{\input{\jobname-pw.ind}}{}

\end{document}

      