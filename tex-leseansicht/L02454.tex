%% latex-leseansicht-vorspann.tex
%% Vorspann für die Leseansicht.
%% Lädt die gemeinsame Datei latex-vorspann.tex mit nicht gesetztem Schalter.

\newif\ifkorrekturansicht
\korrekturansichtfalse

\input{../tex-inputs/latex-vorspann}


         \newcommand{\erwaehnteOrte}{Orte: Bad Aussee, Berlin, Hotel Esplanade, Wien}
         \newcommand{\erwaehnteWerke}{Werke: Das Schicksal des Freiherrn von Leisenbohg. Novellette, Der Turm. Ein Trauerspiel, Der tapfere Cassian. Puppenspiel in einem Akt, Die Frau des Richters. Novelle}
               \section[Hugo Hofmannsthal an Arthur Schnitzler, 14. 11. 1925]{ Hugo Hofmannsthal an Arthur Schnitzler, 14. 11. 1925}\nopagebreak\mylabel{v}\rehead{ }\begin{ledgroupsized}[t]{13cm}\normalsize\beginnumbering \toendnotes[C]{\smallbreak\pagebreak[2]} \Standort{CUL, Schnitzler, B 43.}
\physDesc{Brief, 1 Blatt, 1 Seite
\newline{}Handschrift: schwarze Tinte, lateinische Kurrent
\newline{}Schnitzler: 1) mit Bleistift beschriftet: »\textsc{Hugo}«  2) mit rotem Buntstift mehrere Unterstreichungen\newline{}Ordnung: 1) mit Bleistift von unbekannter Hand nummeriert: »\strikeout{369}«  2) mit Bleistift von unbekannter Hand nummeriert: »378«}\buchAbdrucke{\weitereDrucke{Hugo von Hofmannsthal, Arthur Schnitzler: \emph{Briefwechsel}. Hg. Therese Nickl und Heinrich Schnitzler. Frankfurt am Main: \emph{S. Fischer} 1964, S. 302.} }\toendnotes[C]{\smallbreak}\pstart
           {\pb}Bad Aussee\oindex{Bad Aussee@\textbf{Bad Aussee}|pw}{ }14 XI 25. \pend
           \pstart{}lieber Arthur\pend\pstart
           eben ko{\geminationm}t ein kleines Buch\pwindex{Schnitzler, Arthur 15.05.1862 – 21.10.1931@\textsc{Schnitzler, Arthur} (15.05.1862 – 21.10.1931), \emph{Schriftsteller, Mediziner}!Frau des Richters. Novelle7.8.1925 – 15.8.1925@\strich\emph{Die Frau des Richters. Novelle} {[}7.8.1925 – 15.8.1925{]}|pwv}: eine Erzählung von Ihrer Hand, und ich freue mich
               äußerst darauf, sie abends zu lesen: ein Vorgefühl (genährt durch Hineinschauen) sagt
               mir, dass sie an meine besonderen Lieblinge: »Leisenbohg\pwindex{Schnitzler, Arthur 15.05.1862 – 21.10.1931@\textsc{Schnitzler, Arthur} (15.05.1862 – 21.10.1931), \emph{Schriftsteller, Mediziner}!Schicksal des Freiherrn von Leisenbohg. Novellette01. 07. 1904@\strich\emph{Das Schicksal des Freiherrn von Leisenbohg. Novellette} {[}01. 07. 1904{]}|pw}« und »Cassian\pwindex{Schnitzler, Arthur 15.05.1862 – 21.10.1931@\textsc{Schnitzler, Arthur} (15.05.1862 – 21.10.1931), \emph{Schriftsteller, Mediziner}!tapfere Cassian. Puppenspiel in einem Akt01. 02. 1904@\strich\emph{Der tapfere Cassian. Puppenspiel in einem Akt} {[}01. 02. 1904{]}|pw}«, \label{T_L02454_1v}\edtext{angrenzt}{\lemma{\textnormal{\emph{angrenzt}}}\Cendnote{\textnormal{Er schreibt »angränzt«.}}}\label{T_L02454_1h}.\pend
           \pstart
           Arthur, aber haben Sie in Berlin\oindex{Berlin@\textbf{Berlin}|pw} den »Turm\pwindex{Hofmannsthal, Hugo von 1874-02-01 – 1929-07-15@\textsc{Hofmannsthal, Hugo von} (1874-02-01 – 1929-07-15), \emph{Schriftsteller}!Turm. Ein Trauerspiel1925@\strich\emph{Der Turm. Ein Trauerspiel} {[}1925{]}|pw}« beko{\geminationm}en?\hspace*{1.5em}Fast ko{\geminationm}t mir der Gedanke,
               dass \uline{nicht}. Und diese Exemplare einer (vorläufigen)
               mehr nur Luxusausgabe sind wenige, es täte mir leid, we{\geminationn} eines verloren wäre.\hspace*{1.5em}Würden Sie eventuell ans Esplanade\oindex{Hotel Esplanade@\textbf{Hotel Esplanade}|pw} ein reclamierendes Wort schreiben? Mir liegt viel daran, diese
               Arbeit endlich in Ihren Händen zu wissen!\hspace*{1.5em}– Ich bin,
               in großer Stille, sehr anhaltend fleissig.\pend
           \pstart Ihr\spacefill\mbox{Hugo.}\pend{}
         
         \endnumbering\mylabel{h}\end{ledgroupsized}  \newcommand{\dateiname}{L02454}\newcommand{\titel}{Hugo Hofmannsthal an Arthur Schnitzler, 14. 11. 1925}\newcommand{\editorInnen}{Martin Anton Müller und Gerd-Hermann Susen}%% latex-leseansicht-abspann.tex
%% Abspann für die Leseansicht.
%% Der Schalter \ifkorrekturansicht ist bereits durch den Vorspann gesetzt.

%% latex-abspann.tex
%% Gemeinsamer Abspann für Korrekturansicht und Leseansicht.
%% Setzt den Schalter \ifkorrekturansicht voraus (gesetzt in den
%% einbindenden Dateien latex-korrekturansicht-abspann.tex bzw.
%% latex-leseansicht-abspann.tex).
%% ---------------------------------------------------------------

\normalsize

% Das esempio-Environment wird nur in der Leseansicht benötigt
\ifkorrekturansicht\else
\newenvironment{esempio}[3]%
{
    \vspace{1.5ex}
    \rlap{\underline{#1}}
    \par
    \setlength{\parindent}{0cm}
    \nopagebreak
    \leftskip=#2cm
    \rightskip=#3cm
}
{
    \par
}
\fi

\doendnotes{C}
\bigskip
\vfill

\clearpage

\footnotesize

\ifkorrekturansicht
  \lohead{\textsc{register}}
\fi

% theindex-Environment neu definieren ohne reledmac
\makeatletter
\renewenvironment{theindex}{%
  \ifkorrekturansicht
    \section*{\indexname}%
  \else
    \subsubsection*{Index der erwähnten Entitäten}%
  \fi
  \setlength{\parindent}{0pt}%
  \setlength{\parskip}{0pt plus 0.3pt}%
  \let\item\@idxitem
}{%
  \ifkorrekturansicht\clearpage\fi
}
\makeatother

\IfFileExists{\jobname-pw.ind}{\input{\jobname-pw.ind}}{}

% Quellenangabe nur in der Leseansicht
\ifkorrekturansicht\else
% Fallback-Definitionen, falls die .tex-Datei \titel etc. nicht gesetzt hat
\providecommand{\titel}{}
\providecommand{\editorInnen}{}
\providecommand{\dateiname}{\jobname}

\vspace{3cm}

\vfill

\footnotesize
\textsc{Quelle}: \titel. Herausgegeben von {\editorInnen}. In: \emph{Arthur Schnitzler: Briefwechsel mit Autorinnen und Autoren}.
 Digitale Edition, https://schnitzler-briefe.acdh.oeaw.ac.at/{\dateiname}.html (Stand \today)
\fi

\end{document}


      