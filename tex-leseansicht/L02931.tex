%% latex-leseansicht-vorspann.tex
%% Vorspann für die Leseansicht.
%% Lädt die gemeinsame Datei latex-vorspann.tex mit nicht gesetztem Schalter.

\newif\ifkorrekturansicht
\korrekturansichtfalse

\input{../tex-inputs/latex-vorspann}


\section[ Paul Goldmann an Arthur Schnitzler, 19. 9. {[}1900{]}]{L02931 Paul Goldmann an Arthur Schnitzler,  19. 9. [1900]}
\nopagebreak\mylabel{L02931v}
\rehead{ }\normalsize\beginnumbering\briefempfaengerindex{Schnitzler, Arthur@\textsc{Schnitzler, Arthur}!zzzGoldmann, Paul@\emph{von Paul Goldmann}!1900-09-191@{19. 9. [1900]}|(be}
\toendnotes[C]{\smallbreak\pagebreak[2]}
\correspDesc{Versand  durch Paul Goldmann am 19. 9. [1900] in Berlin
\newline{}Erhalt  durch Arthur Schnitzler im Zeitraum [20. 9. 1900
                  – 24. 9. 1900?] in Wien}\toendnotes[C]{\smallbreak}
\Standort{DLA, A:Schnitzler, HS.NZ85.1.3170.}
\physDesc{Brief, 1 Blatt, 4 Seiten, 1968 Zeichen
\newline{}Handschrift: blaue Tinte, deutsche Kurrent
\newline{}Schnitzler: 1) mit Bleistift das Jahr »900« sowie »\textcolor{gray}{I)}« vermerkt; letzteres womöglich ein Hinweis auf das
                                 Postskript auf der ersten Seite  2) mit rotem Buntstift drei Unterstreichungen und eine seitliche
                                 Markierung}\toendnotes[C]{\smallbreak}
\pstart
           \raggedleft{}{\pb}Berlin\oindex{Berlin@\textbf{Berlin}, \emph{Hauptstadt}|pw}, 19. September.\pend
           
\pstart\center{}Mein lieber \label{K_L02931-1v}\edtext{Onkel\pwindex{Mamroth, Fedor 21.\,2.\,1851 Breslau – 25.\,6.\,1907 Frankfurt am Main@\textsc{Mamroth, Fedor} (21.\,2.\,1851 Breslau – 25.\,6.\,1907 Frankfurt am Main), \emph{Journalist, Kritiker}|pwv}}{\lemma{\textnormal{\emph{Onkel}}}\Cendnote{\textnormal{Unachtsamkeit und Verwechslung der
                     Anrede mit jener für Fedor Mamroth\pwindex{Mamroth, Fedor 21.\,2.\,1851 Breslau – 25.\,6.\,1907 Frankfurt am Main@\textsc{Mamroth, Fedor} (21.\,2.\,1851 Breslau – 25.\,6.\,1907 Frankfurt am Main), \emph{Journalist, Kritiker}|pwk} oder,
                     innerhalb der Korrespondenz untypisch, Witz}}}\label{K_L02931-1},\pend\vspace{0.5em}
\pstart
           Den \label{K_L02931-2v}\edtext{Artikel\pwindex{Paul Schlenther und die Wiener Kritik@\emph{Paul Schlenther und die Wiener Kritik}|pwv} des »Berliner Tageblatt\pwindex{Berliner Tageblatt@\emph{Berliner Tageblatt}|pw}«}{\lemma{\textnormal{\emph{Artikel … Tageblatt«}}}\Cendnote{\textnormal{[O. V.]: \emph{Paul Schlenther und die Wiener
                        Kritik}\pwindex{Paul Schlenther und die Wiener Kritik@\emph{Paul Schlenther und die Wiener Kritik}|pwk}. In: \emph{Berliner Tageblatt}\pwindex{Berliner Tageblatt@\emph{Berliner Tageblatt}|pwk},
                     Jg. 29, Nr. 470, 15. 9. 1900, Abend-Ausgabe,
                     S. 1–2.}}}\label{K_L02931-2} hatte ich natürlich, unter Hervorhebung der Dir günſtigen
               Stellen, telegraphirt; die Redaktion\orgindex{Neue Freie Presse@Neue Freie Presse|pwv} hat mein Telegramm, wie ich heut{ }ſehe, nicht veröffentlicht (was ich Dir im Vertrauen mittheile).\pend
           
\pstart
           Im Übrigen iſt die \label{K_L02931-3v}\edtext{Affaire}{\lemma{\textnormal{\emph{Affaire}}}\Cendnote{\textnormal{Siehe XXXX Auszeichnungsfehler: Dokument L01073 nicht gefunden.
               }}}\label{K_L02931-3}{ }ſehr günſtig für Dich und{ }ſehr ungünſtig für Herrn \textsc{Schlenther\pwindex{Schlenther, Paul 20.\,8.\,1854 Chernyakhovsk – 30.\,4.\,1916 Berlin@\textsc{Schlenther, Paul} (20.\,8.\,1854 Chernyakhovsk – 30.\,4.\,1916 Berlin), \emph{Schriftsteller, Kritiker, Theaterleiter}|pw}}. Selbſt in Berlin\oindex{Berlin@\textbf{Berlin}, \emph{Hauptstadt}|pw} war man genöthigt, ihm
               harte Wahrheiten zu{ }ſagen. Und was auch die Leute darüber{ }ſagen, – und obwohl Du{ }ſelbſt {\pb}ganz gewiß nicht \uline{dieſen} Zweck im Auge gehabt haſt, – die Wirkung iſt: \strikeout{alle} alle Welt iſt auf Dein Stück\pwindex{Schnitzler, Arthur 15.\,5.\,1862 Wien – 21.\,10.\,1931 ebd.@\textsc{Schnitzler, Arthur} (15.\,5.\,1862 Wien – 21.\,10.\,1931 ebd.), \emph{Schriftsteller, Mediziner}!Schleier der Beatrice. Schauspiel in fünf Akten@\strich\emph{Der Schleier der Beatrice. Schauspiel in fünf Akten}|pwv} aufmerkſam geworden, und die Bühnen
               haben einen Grund mehr, Dich aufzuführen. Daß die Fernſtehenden durch die Affaire ein
               falſches Bild von \strikeout{d} Dir gewinnen könnten,{ }ſoll Dich
               nicht kümmern. Erſtens{ }ſehe ich nicht ein, aus welchem Grunde. Und zweitens,{ }ſelbſt
               wenn es{ }ſo{ }ſein{ }ſollte: glaubſt Du,{ }ſie haben vorher ein richtiges Bild von Dir
               gehabt? {\pb}Immerhin iſt zu conſtatiren, daß von den
                  Berlin\oindex{Berlin@\textbf{Berlin}, \emph{Hauptstadt}|pw}er Blättern, die Dir doch gewiß
               fernſtehen, keines{ }ſich in einer Weiſe über Dich geäußert hat, die Dich hätte
               verletzen können. Und wenn das Berliner
                  Tageblatt\pwindex{Berliner Tageblatt@\emph{Berliner Tageblatt}|pw} die \label{K_L02931-4v}\edtext{Preisgebung des \textsc{Schlenther\pwindex{Schlenther, Paul 20.\,8.\,1854 Chernyakhovsk – 30.\,4.\,1916 Berlin@\textsc{Schlenther, Paul} (20.\,8.\,1854 Chernyakhovsk – 30.\,4.\,1916 Berlin), \emph{Schriftsteller, Kritiker, Theaterleiter}|pw}}’ſchen Briefes\pwindex{Bahr, Hermann 19.\,7.\,1863 Linz – 15.\,1.\,1934 München@\textsc{Bahr, Hermann} (19.\,7.\,1863 Linz – 15.\,1.\,1934 München), \emph{Schriftsteller, Kritiker}!Erklärung [Schleier der Beatrice]@\strich\emph{Erklärung [Schleier der Beatrice]}|pwv}\pwindex{Salten, Felix 6.\,9.\,1869 Budapest – 8.\,10.\,1945 Zürich@\textsc{Salten, Felix} (6.\,9.\,1869 Budapest – 8.\,10.\,1945 Zürich), \emph{Schriftsteller, Journalist, Chefredakteur}!Erklärung [Schleier der Beatrice]@\strich\emph{Erklärung [Schleier der Beatrice]}|pwv}\pwindex{Bauer, Julius 15.\,10.\,1853 Szigetvár – 11.\,6.\,1941 Wien@\textsc{Bauer, Julius} (15.\,10.\,1853 Szigetvár – 11.\,6.\,1941 Wien), \emph{Schriftsteller, Journalist, Kritiker}!Erklärung [Schleier der Beatrice]@\strich\emph{Erklärung [Schleier der Beatrice]}|pwv}\pwindex{Hirschfeld, Robert 17.\,9.\,1857 Žďár nad Sázavou – 2.\,4.\,1914 Salzburg@\textsc{Hirschfeld, Robert} (17.\,9.\,1857 Žďár nad Sázavou – 2.\,4.\,1914 Salzburg), \emph{Journalist, Musikkritiker}!Erklärung [Schleier der Beatrice]@\strich\emph{Erklärung [Schleier der Beatrice]}|pwv}\pwindex{Speidel, Ludwig 11.\,4.\,1830 Ulm – 3.\,2.\,1906 Wien@\textsc{Speidel, Ludwig} (11.\,4.\,1830 Ulm – 3.\,2.\,1906 Wien), \emph{Journalist, Kritiker}!Erklärung [Schleier der Beatrice]@\strich\emph{Erklärung [Schleier der Beatrice]}|pwv}\pwindex{David, Jakob Julius 6.\,2.\,1859 Hranice – 20.\,11.\,1906 Wien@\textsc{David, Jakob Julius} (6.\,2.\,1859 Hranice – 20.\,11.\,1906 Wien), \emph{Schriftsteller, Journalist}!Erklärung [Schleier der Beatrice]@\strich\emph{Erklärung [Schleier der Beatrice]}|pwv}}{\lemma{\textnormal{\emph{Preisgebung … Briefes}}}\Cendnote{\textnormal{In die \emph{Erklärung}\pwindex{Bahr, Hermann 19.\,7.\,1863 Linz – 15.\,1.\,1934 München@\textsc{Bahr, Hermann} (19.\,7.\,1863 Linz – 15.\,1.\,1934 München), \emph{Schriftsteller, Kritiker}!Erklärung [Schleier der Beatrice]@\strich\emph{Erklärung [Schleier der Beatrice]}|pwk}\pwindex{Salten, Felix 6.\,9.\,1869 Budapest – 8.\,10.\,1945 Zürich@\textsc{Salten, Felix} (6.\,9.\,1869 Budapest – 8.\,10.\,1945 Zürich), \emph{Schriftsteller, Journalist, Chefredakteur}!Erklärung [Schleier der Beatrice]@\strich\emph{Erklärung [Schleier der Beatrice]}|pwk}\pwindex{Bauer, Julius 15.\,10.\,1853 Szigetvár – 11.\,6.\,1941 Wien@\textsc{Bauer, Julius} (15.\,10.\,1853 Szigetvár – 11.\,6.\,1941 Wien), \emph{Schriftsteller, Journalist, Kritiker}!Erklärung [Schleier der Beatrice]@\strich\emph{Erklärung [Schleier der Beatrice]}|pwk}\pwindex{Hirschfeld, Robert 17.\,9.\,1857 Žďár nad Sázavou – 2.\,4.\,1914 Salzburg@\textsc{Hirschfeld, Robert} (17.\,9.\,1857 Žďár nad Sázavou – 2.\,4.\,1914 Salzburg), \emph{Journalist, Musikkritiker}!Erklärung [Schleier der Beatrice]@\strich\emph{Erklärung [Schleier der Beatrice]}|pwk}\pwindex{Speidel, Ludwig 11.\,4.\,1830 Ulm – 3.\,2.\,1906 Wien@\textsc{Speidel, Ludwig} (11.\,4.\,1830 Ulm – 3.\,2.\,1906 Wien), \emph{Journalist, Kritiker}!Erklärung [Schleier der Beatrice]@\strich\emph{Erklärung [Schleier der Beatrice]}|pwk}\pwindex{David, Jakob Julius 6.\,2.\,1859 Hranice – 20.\,11.\,1906 Wien@\textsc{David, Jakob Julius} (6.\,2.\,1859 Hranice – 20.\,11.\,1906 Wien), \emph{Schriftsteller, Journalist}!Erklärung [Schleier der Beatrice]@\strich\emph{Erklärung [Schleier der Beatrice]}|pwk} war auch ein Brief Paul
                     Schlenthers\pwindex{Schlenther, Paul 20.\,8.\,1854 Chernyakhovsk – 30.\,4.\,1916 Berlin@\textsc{Schlenther, Paul} (20.\,8.\,1854 Chernyakhovsk – 30.\,4.\,1916 Berlin), \emph{Schriftsteller, Kritiker, Theaterleiter}|pwk} an Schnitzler
                  aufgenommen worden, dessen Publikation nicht autorisiert war. Vgl. Hermann Bahr\pwindex{Bahr, Hermann 19.\,7.\,1863 Linz – 15.\,1.\,1934 München@\textsc{Bahr, Hermann} (19.\,7.\,1863 Linz – 15.\,1.\,1934 München), \emph{Schriftsteller, Kritiker}|pwk}, Julius Bauer\pwindex{Bauer, Julius 15.\,10.\,1853 Szigetvár – 11.\,6.\,1941 Wien@\textsc{Bauer, Julius} (15.\,10.\,1853 Szigetvár – 11.\,6.\,1941 Wien), \emph{Schriftsteller, Journalist, Kritiker}|pwk}, J. J.
                        David\pwindex{David, Jakob Julius 6.\,2.\,1859 Hranice – 20.\,11.\,1906 Wien@\textsc{David, Jakob Julius} (6.\,2.\,1859 Hranice – 20.\,11.\,1906 Wien), \emph{Schriftsteller, Journalist}|pwk}, Robert Hirschfeld\pwindex{Hirschfeld, Robert 17.\,9.\,1857 Žďár nad Sázavou – 2.\,4.\,1914 Salzburg@\textsc{Hirschfeld, Robert} (17.\,9.\,1857 Žďár nad Sázavou – 2.\,4.\,1914 Salzburg), \emph{Journalist, Musikkritiker}|pwk}, Felix Salten\pwindex{Salten, Felix 6.\,9.\,1869 Budapest – 8.\,10.\,1945 Zürich@\textsc{Salten, Felix} (6.\,9.\,1869 Budapest – 8.\,10.\,1945 Zürich), \emph{Schriftsteller, Journalist, Chefredakteur}|pwk} und Ludwig Speidel\pwindex{Speidel, Ludwig 11.\,4.\,1830 Ulm – 3.\,2.\,1906 Wien@\textsc{Speidel, Ludwig} (11.\,4.\,1830 Ulm – 3.\,2.\,1906 Wien), \emph{Journalist, Kritiker}|pwk}: \emph{Erklärung}\pwindex{Bahr, Hermann 19.\,7.\,1863 Linz – 15.\,1.\,1934 München@\textsc{Bahr, Hermann} (19.\,7.\,1863 Linz – 15.\,1.\,1934 München), \emph{Schriftsteller, Kritiker}!Erklärung [Schleier der Beatrice]@\strich\emph{Erklärung [Schleier der Beatrice]}|pwk}\pwindex{Salten, Felix 6.\,9.\,1869 Budapest – 8.\,10.\,1945 Zürich@\textsc{Salten, Felix} (6.\,9.\,1869 Budapest – 8.\,10.\,1945 Zürich), \emph{Schriftsteller, Journalist, Chefredakteur}!Erklärung [Schleier der Beatrice]@\strich\emph{Erklärung [Schleier der Beatrice]}|pwk}\pwindex{Bauer, Julius 15.\,10.\,1853 Szigetvár – 11.\,6.\,1941 Wien@\textsc{Bauer, Julius} (15.\,10.\,1853 Szigetvár – 11.\,6.\,1941 Wien), \emph{Schriftsteller, Journalist, Kritiker}!Erklärung [Schleier der Beatrice]@\strich\emph{Erklärung [Schleier der Beatrice]}|pwk}\pwindex{Hirschfeld, Robert 17.\,9.\,1857 Žďár nad Sázavou – 2.\,4.\,1914 Salzburg@\textsc{Hirschfeld, Robert} (17.\,9.\,1857 Žďár nad Sázavou – 2.\,4.\,1914 Salzburg), \emph{Journalist, Musikkritiker}!Erklärung [Schleier der Beatrice]@\strich\emph{Erklärung [Schleier der Beatrice]}|pwk}\pwindex{Speidel, Ludwig 11.\,4.\,1830 Ulm – 3.\,2.\,1906 Wien@\textsc{Speidel, Ludwig} (11.\,4.\,1830 Ulm – 3.\,2.\,1906 Wien), \emph{Journalist, Kritiker}!Erklärung [Schleier der Beatrice]@\strich\emph{Erklärung [Schleier der Beatrice]}|pwk}\pwindex{David, Jakob Julius 6.\,2.\,1859 Hranice – 20.\,11.\,1906 Wien@\textsc{David, Jakob Julius} (6.\,2.\,1859 Hranice – 20.\,11.\,1906 Wien), \emph{Schriftsteller, Journalist}!Erklärung [Schleier der Beatrice]@\strich\emph{Erklärung [Schleier der Beatrice]}|pwk}. In: \emph{Neues Wiener
                        Tagblatt}\pwindex{Neues Wiener Tagblatt@\emph{Neues Wiener Tagblatt}|pwk} [u. a.], Jg. 34, Nr. 252, 14. 9. 1900, S. 9–10, hier: S. 9.}}}\label{K_L02931-4} als inkorrekt bezeichnet\pwindex{Paul Schlenther und die Wiener Kritik@\emph{Paul Schlenther und die Wiener Kritik}|pwv} hat,{ }ſo geſchieht
               dies wohl hauptſächlich darum, \strikeout{d\textcolor{gray}{a}} weil{ }ſich die Berlin\oindex{Berlin@\textbf{Berlin}, \emph{Hauptstadt}|pw}er über \label{K_L02931-5v}\edtext{den das »Deutſche Theater\orgindex{Deutsches Theater Berlin@Deutsches Theater Berlin|pw}« betreffenden Paſſus\pwindex{Paul Schlenther und die Wiener Kritik@\emph{Paul Schlenther und die Wiener Kritik}|pwv}}{\lemma{\textnormal{\emph{den … Passus}}}\Cendnote{\textnormal{In dem erwähnten, abgedruckten Brief\pwindex{Bahr, Hermann 19.\,7.\,1863 Linz – 15.\,1.\,1934 München@\textsc{Bahr, Hermann} (19.\,7.\,1863 Linz – 15.\,1.\,1934 München), \emph{Schriftsteller, Kritiker}!Erklärung [Schleier der Beatrice]@\strich\emph{Erklärung [Schleier der Beatrice]}|pwkv}\pwindex{Salten, Felix 6.\,9.\,1869 Budapest – 8.\,10.\,1945 Zürich@\textsc{Salten, Felix} (6.\,9.\,1869 Budapest – 8.\,10.\,1945 Zürich), \emph{Schriftsteller, Journalist, Chefredakteur}!Erklärung [Schleier der Beatrice]@\strich\emph{Erklärung [Schleier der Beatrice]}|pwkv}\pwindex{Bauer, Julius 15.\,10.\,1853 Szigetvár – 11.\,6.\,1941 Wien@\textsc{Bauer, Julius} (15.\,10.\,1853 Szigetvár – 11.\,6.\,1941 Wien), \emph{Schriftsteller, Journalist, Kritiker}!Erklärung [Schleier der Beatrice]@\strich\emph{Erklärung [Schleier der Beatrice]}|pwkv}\pwindex{Hirschfeld, Robert 17.\,9.\,1857 Žďár nad Sázavou – 2.\,4.\,1914 Salzburg@\textsc{Hirschfeld, Robert} (17.\,9.\,1857 Žďár nad Sázavou – 2.\,4.\,1914 Salzburg), \emph{Journalist, Musikkritiker}!Erklärung [Schleier der Beatrice]@\strich\emph{Erklärung [Schleier der Beatrice]}|pwkv}\pwindex{Speidel, Ludwig 11.\,4.\,1830 Ulm – 3.\,2.\,1906 Wien@\textsc{Speidel, Ludwig} (11.\,4.\,1830 Ulm – 3.\,2.\,1906 Wien), \emph{Journalist, Kritiker}!Erklärung [Schleier der Beatrice]@\strich\emph{Erklärung [Schleier der Beatrice]}|pwkv}\pwindex{David, Jakob Julius 6.\,2.\,1859 Hranice – 20.\,11.\,1906 Wien@\textsc{David, Jakob Julius} (6.\,2.\,1859 Hranice – 20.\,11.\,1906 Wien), \emph{Schriftsteller, Journalist}!Erklärung [Schleier der Beatrice]@\strich\emph{Erklärung [Schleier der Beatrice]}|pwkv}{ }Paul Schlenthers\pwindex{Schlenther, Paul 20.\,8.\,1854 Chernyakhovsk – 30.\,4.\,1916 Berlin@\textsc{Schlenther, Paul} (20.\,8.\,1854 Chernyakhovsk – 30.\,4.\,1916 Berlin), \emph{Schriftsteller, Kritiker, Theaterleiter}|pwk} warnt dieser Schnitzler vor dem \emph{Deutschen Theater}\orgindex{Deutsches Theater Berlin@Deutsches Theater Berlin|pwk}, da dieses der
                     »Riesenaufgabe« einer Aufführung von \emph{Der Schleier der Beatrice}\pwindex{Schnitzler, Arthur 15.\,5.\,1862 Wien – 21.\,10.\,1931 ebd.@\textsc{Schnitzler, Arthur} (15.\,5.\,1862 Wien – 21.\,10.\,1931 ebd.), \emph{Schriftsteller, Mediziner}!Schleier der Beatrice. Schauspiel in fünf Akten@\strich\emph{Der Schleier der Beatrice. Schauspiel in fünf Akten}|pwk} »nicht gewachsen«
                  sei.}}}\label{K_L02931-5} ärgern.\pend
           
\pstart
           Daß ich \label{K_L02931-6v}\edtext{\textsc{Richard\pwindex{Beer-Hofmann, Richard 11.\,7.\,1866 Wien – 26.\,9.\,1945 New York City@\textsc{Beer-Hofmann, Richard} (11.\,7.\,1866 Wien – 26.\,9.\,1945 New York City), \emph{Schriftsteller}|pw}} verfehlt}{\lemma{\textnormal{\emph{Richard verfehlt}}}\Cendnote{\textnormal{Siehe XXXX Auszeichnungsfehler: Dokument L01071 nicht gefunden.
               }}}\label{K_L02931-6} habe, thut mir unendlich leid. Anderſeits war ich ja über {\pb}eine Woche in Wien\oindex{Wien@\textbf{Wien}, \emph{Verwaltungsgebiet}|pw}; und wenn er wirklich das Bedürfniß gehabt hätte, mit mir zuſammen zu{ }ſein,{ }ſo hätte er auch etwas früher \label{K_L02931-7v}\edtext{zurückkommen}{\lemma{\textnormal{\emph{zurückkommen}}}\Cendnote{\textnormal{aus Altaussee\oindex{Altaussee@\textbf{Altaussee}, \emph{Verwaltungsgebiet}|pwk}, siehe XXXX Auszeichnungsfehler: Dokument L01073 nicht gefunden.}}}\label{K_L02931-7} können. Grüße ihn recht herzlich von mir und{ }ſage ihm, daß ich ihm eine der
               wenigen freundlichen \strikeout{Erin} Erinnerungen an \strikeout{\textcolor{gray}{u}} meine diesjährige Urlaubsreiſe danke. Und er{ }ſoll mir \textsc{Mirjams\pwindex{Beer-Hofmann, Mirjam 4.\,9.\,1897 Wien – 24.\,12.\,1984 New York City@\textsc{Beer-Hofmann, Mirjam} (4.\,9.\,1897 Wien – 24.\,12.\,1984 New York City)|pw}} Wiegenlied\pwindex{Beer-Hofmann, Richard 11.\,7.\,1866 Wien – 26.\,9.\,1945 New York City@\textsc{Beer-Hofmann, Richard} (11.\,7.\,1866 Wien – 26.\,9.\,1945 New York City), \emph{Schriftsteller}!Schlaflied für Mirjam@\strich\emph{Schlaflied für Mirjam}|pwv}{ }ſchicken.\pend
           
\pstart
           Ich leide,{ }ſeit ich zurück bin, an einem Tag und Nacht andauernden, wühlenden
               Kopfſchmerz, bin vollkommen arbeitsunfähig und fürchte unheimliche Dinge in meinem
               Gehirn. Viele Grüße! Dein {\\}\spacefill\mbox{P. G.}\pend
           
\pstart
           \noindent{}{\pb}\label{T_L02931-1v}\edtext{Viele Grüße an die beiden Fräulein\pwindex{Schnitzler, Olga 17.\,1.\,1882 Wien – 13.\,1.\,1970 Lugano@\textsc{Schnitzler, Olga} (17.\,1.\,1882 Wien – 13.\,1.\,1970 Lugano), \emph{Schauspielerin, Sängerin}|pwv}\pwindex{Steinrück, Elisabeth 19.\,11.\,1885 – 7.\,4.\,1920 Partenkirchen@\textsc{Steinrück, Elisabeth} (19.\,11.\,1885 – 7.\,4.\,1920 Partenkirchen)|pwv} aus der \label{K_L02931-8v}\edtext{Rothe-Stern-Gaſſe\oindex{Wien@\textbf{Wien}!II., Leopoldstadt@\textbf{II., Leopoldstadt}!Rotensterngasse@\textbf{Rotensterngasse}, \emph{Straße}|pw}}{\lemma{\textnormal{\emph{Rothe-Stern-Gasse}}}\Cendnote{\textnormal{Wohnadresse von Schnitzlers Partnerin und zukünftiger Ehefrau Olga Gussmann\pwindex{Schnitzler, Olga 17.\,1.\,1882 Wien – 13.\,1.\,1970 Lugano@\textsc{Schnitzler, Olga} (17.\,1.\,1882 Wien – 13.\,1.\,1970 Lugano), \emph{Schauspielerin, Sängerin}|pwk} und ihrer Schwester Elisabeth\pwindex{Steinrück, Elisabeth 19.\,11.\,1885 – 7.\,4.\,1920 Partenkirchen@\textsc{Steinrück, Elisabeth} (19.\,11.\,1885 – 7.\,4.\,1920 Partenkirchen)|pwk} (nachmalig Steinrück\pwindex{Steinrück, Elisabeth 19.\,11.\,1885 – 7.\,4.\,1920 Partenkirchen@\textsc{Steinrück, Elisabeth} (19.\,11.\,1885 – 7.\,4.\,1920 Partenkirchen)|pwkv}), vgl. A. S.: \emph{Tagebuch}, 21. 12. 1920.}}}\label{K_L02931-8}!}{\lemma{\textnormal{\emph{Viele … Rothe-Stern-Gasse!}}}\Cendnote{\textnormal{kopfüber am oberen Rand der
                     ersten Seite}}}\label{T_L02931-1}\pend
           \selectlanguage{ngerman}\endnumbering\briefempfaengerindex{Schnitzler, Arthur@\textsc{Schnitzler, Arthur}!zzzGoldmann, Paul@\emph{von Paul Goldmann}!1900-09-191@{19. 9. [1900]}|)be}\mylabel{L02931h}  \newcommand{\dateiname}{L02931}\newcommand{\titel}{Paul Goldmann an Arthur Schnitzler, 19. 9. [1900]}\newcommand{\editorInnen}{Martin Anton Müller und Laura Untner}%% latex-leseansicht-abspann.tex
%% Abspann für die Leseansicht.
%% Der Schalter \ifkorrekturansicht ist bereits durch den Vorspann gesetzt.

%% latex-abspann.tex
%% Gemeinsamer Abspann für Korrekturansicht und Leseansicht.
%% Setzt den Schalter \ifkorrekturansicht voraus (gesetzt in den
%% einbindenden Dateien latex-korrekturansicht-abspann.tex bzw.
%% latex-leseansicht-abspann.tex).
%% ---------------------------------------------------------------

\normalsize

% Das esempio-Environment wird nur in der Leseansicht benötigt
\ifkorrekturansicht\else
\newenvironment{esempio}[3]%
{
    \vspace{1.5ex}
    \rlap{\underline{#1}}
    \par
    \setlength{\parindent}{0cm}
    \nopagebreak
    \leftskip=#2cm
    \rightskip=#3cm
}
{
    \par
}
\fi

\doendnotes{C}
\bigskip
\vfill

\clearpage

\footnotesize

\ifkorrekturansicht
  \lohead{\textsc{register}}
\fi

% theindex-Environment neu definieren ohne reledmac
\makeatletter
\renewenvironment{theindex}{%
  \ifkorrekturansicht
    \section*{\indexname}%
  \else
    \subsubsection*{Index der erwähnten Entitäten}%
  \fi
  \setlength{\parindent}{0pt}%
  \setlength{\parskip}{0pt plus 0.3pt}%
  \let\item\@idxitem
}{%
  \ifkorrekturansicht\clearpage\fi
}
\makeatother

\IfFileExists{\jobname-pw.ind}{\input{\jobname-pw.ind}}{}

% Quellenangabe nur in der Leseansicht
\ifkorrekturansicht\else
% Fallback-Definitionen, falls die .tex-Datei \titel etc. nicht gesetzt hat
\providecommand{\titel}{}
\providecommand{\editorInnen}{}
\providecommand{\dateiname}{\jobname}

\vspace{3cm}

\vfill

\footnotesize
\textsc{Quelle}: \titel. Herausgegeben von {\editorInnen}. In: \emph{Arthur Schnitzler: Briefwechsel mit Autorinnen und Autoren}.
 Digitale Edition, https://schnitzler-briefe.acdh.oeaw.ac.at/{\dateiname}.html (Stand \today)
\fi

\end{document}


