%% latex-leseansicht-vorspann.tex
%% Vorspann für die Leseansicht.
%% Lädt die gemeinsame Datei latex-vorspann.tex mit nicht gesetztem Schalter.

\newif\ifkorrekturansicht
\korrekturansichtfalse

\input{../tex-inputs/latex-vorspann}


         
         \renewcommand{\erwaehntePersonen}{Personen: Hermann Bahr, Julius Bauer, Richard Beer-Hofmann, Mirjam Beer-Hofmann, Jakob Julius David, Robert Hirschfeld, Fedor Mamroth, Felix Salten, Paul Schlenther, Olga Schnitzler, Ludwig Speidel, Elisabeth Steinrück}
         \renewcommand{\erwaehnteInstitutionen}{Institutionen: Deutsches Theater Berlin, Neue Freie Presse}
         \renewcommand{\erwaehnteOrte}{Orte: Altaussee, Berlin, Rotensterngasse, Wien}
         \renewcommand{\erwaehnteWerke}{Werke: Berliner Tageblatt, Der Schleier der Beatrice. Schauspiel in fünf Akten, Erklärung [Schleier der Beatrice], Neues Wiener Tagblatt, Paul Schlenther und die Wiener Kritik, Schlaflied für Mirjam}
               \section[ Paul Goldmann an Arthur Schnitzler, 19. 9. {[}1900{]}]{ Paul Goldmann an Arthur Schnitzler, 19. 9. {[}1900{]}}\nopagebreak\mylabel{v}\rehead{ }\begin{ledgroupsized}[t]{13cm}\normalsize\beginnumbering \toendnotes[C]{\smallbreak\pagebreak[2]} \Standort{DLA, A:Schnitzler, HS.NZ85.1.3170.}
\physDesc{Brief, 1 Blatt, 4 Seiten, 1968 Zeichen
\newline{}Handschrift: blaue Tinte, deutsche Kurrent
\newline{}Schnitzler: 1) mit Bleistift das Jahr »{[}1{]}900« sowie »\textcolor{gray}{I)}« vermerkt; letzteres womöglich ein Hinweis auf das
                                 Postskript auf der ersten Seite  2) mit rotem Buntstift drei Unterstreichungen und eine seitliche
                                 Markierung}\toendnotes[C]{\smallbreak}\pstart
           \raggedleft{}{\pb}Berlin\oindex{Berlin@\textbf{Berlin}|pw}, 19. September.\pend
           \pstart\center{}Mein lieber \label{K_L02931-1v}\edtext{Onkel\pwindex{Mamroth, Fedor 21.02.1851 – 25.06.1907@\textsc{Mamroth, Fedor} (21.02.1851 – 25.06.1907), \emph{Journalist, Kritiker}|pwv}}{\lemma{\textnormal{\emph{Onkel}}}\Cendnote{\textnormal{Unachtsamkeit und Verwechslung der
                     Anrede mit jener für Fedor Mamroth\pwindex{Mamroth, Fedor 21.02.1851 – 25.06.1907@\textsc{Mamroth, Fedor} (21.02.1851 – 25.06.1907), \emph{Journalist, Kritiker}|pwk} oder,
                     innerhalb der Korrespondenz untypisch, Witz}}}\label{K_L02931-1h},\pend\pstart
           Den \label{K_L02931-11v}\edtext{Artikel\pwindex{?? Werk@Nicht ermittelte Verfasserinnen und Verfasser!Paul Schlenther und die Wiener Kritik1900-09-15@\emph{Paul Schlenther und die Wiener Kritik} {[}1900-09-15{]}|pwv} des »Berliner Tageblatt\pwindex{?? Werk@Nicht ermittelte Verfasserinnen und Verfasser!Berliner Tageblatt1872 – 1939@\emph{Berliner Tageblatt} {[}1872 – 1939{]}|pw}«}{\lemma{\textnormal{\emph{Artikel … Tageblatt«}}}\Cendnote{\textnormal{[O. V.]: \emph{Paul Schlenther und die Wiener
                        Kritik}\pwindex{?? Werk@Nicht ermittelte Verfasserinnen und Verfasser!Paul Schlenther und die Wiener Kritik1900-09-15@\emph{Paul Schlenther und die Wiener Kritik} {[}1900-09-15{]}|pwk}. In: \emph{Berliner Tageblatt}\pwindex{?? Werk@Nicht ermittelte Verfasserinnen und Verfasser!Berliner Tageblatt1872 – 1939@\emph{Berliner Tageblatt} {[}1872 – 1939{]}|pwk},
                     Jg. 29, Nr. 470, 15. 9. 1900, Abend-Ausgabe,
                     S. 1–2.}}}\label{K_L02931-11h} hatte ich natürlich, unter Hervorhebung der Dir günſtigen
               Stellen, telegraphirt; die Redaktion\orgindex{Neue Freie Presse@Neue Freie Presse|pwv} hat mein Telegramm, wie ich heut
               ſehe, nicht veröffentlicht (was ich Dir im Vertrauen mittheile).\pend
           \pstart
           Im Übrigen iſt die \label{K_L02931-2v}\edtext{Affaire}{\lemma{\textnormal{\emph{Affaire}}}\Cendnote{\textnormal{siehe Richard Beer-Hofmann an Arthur Schnitzler, 14. 9. 1900}}}\label{K_L02931-2h} ſehr günſtig für Dich und ſehr ungünſtig für Herrn \textsc{Schlenther\pwindex{Schlenther, Paul 20.08.1854 – 30.04.1916@\textsc{Schlenther, Paul} (20.08.1854 – 30.04.1916), \emph{Schriftsteller, Kritiker, Theaterleiter}|pw}}. Selbſt in Berlin\oindex{Berlin@\textbf{Berlin}|pw} war man genöthigt, ihm
               harte Wahrheiten zu ſagen. Und was auch die Leute darüber ſagen, – und obwohl Du
               ſelbſt {\pb}ganz gewiß nicht \uline{dieſen} Zweck im Auge gehabt haſt, – die Wirkung iſt: \strikeout{alle} alle Welt iſt auf Dein Stück\pwindex{Schnitzler, Arthur 15.05.1862 – 21.10.1931@\textsc{Schnitzler, Arthur} (15.05.1862 – 21.10.1931), \emph{Schriftsteller, Mediziner}!Schleier der Beatrice. Schauspiel in fuenf Akten1900-12-01@\strich\emph{Der Schleier der Beatrice. Schauspiel in fünf Akten} {[}1900-12-01{]}|pwv} aufmerkſam geworden, und die Bühnen
               haben einen Grund mehr, Dich aufzuführen. Daß die Fernſtehenden durch die Affaire ein
               falſches Bild von \strikeout{d} Dir gewinnen könnten, ſoll Dich
               nicht kümmern. Erſtens ſehe ich nicht ein, aus welchem Grunde. Und zweitens, ſelbſt
               wenn es ſo ſein ſollte: glaubſt Du, ſie haben vorher ein richtiges Bild von Dir
               gehabt? {\pb}Immerhin iſt zu conſtatiren, daß von den
                  Berlin\oindex{Berlin@\textbf{Berlin}|pw}er Blättern, die Dir doch gewiß
               fernſtehen, keines ſich in einer Weiſe über Dich geäußert hat, die Dich hätte
               verletzen können. Und wenn das Berliner
                  Tageblatt\pwindex{?? Werk@Nicht ermittelte Verfasserinnen und Verfasser!Berliner Tageblatt1872 – 1939@\emph{Berliner Tageblatt} {[}1872 – 1939{]}|pw} die \label{K_L02931-3v}\edtext{Preisgebung des \textsc{Schlenther\pwindex{Schlenther, Paul 20.08.1854 – 30.04.1916@\textsc{Schlenther, Paul} (20.08.1854 – 30.04.1916), \emph{Schriftsteller, Kritiker, Theaterleiter}|pw}}’ſchen Briefes\pwindex{Bahr, Hermann 19.07.1863 – 15.01.1934@\textsc{Bahr, Hermann} (19.07.1863 – 15.01.1934), \emph{Schriftsteller, Kritiker}!Erklaerung [Schleier der Beatrice]1900-09-14@\strich\emph{Erklärung [Schleier der Beatrice]} {[}1900-09-14{]}|pwv}\pwindex{Salten, Felix 06.09.1869 – 08.10.1945@\textsc{Salten, Felix} (06.09.1869 – 08.10.1945), \emph{Schriftsteller, Journalist}!Erklaerung [Schleier der Beatrice]1900-09-14@\strich\emph{Erklärung [Schleier der Beatrice]} {[}1900-09-14{]}|pwv}\pwindex{Bauer, Julius 15.10.1853 – 11.06.1941@\textsc{Bauer, Julius} (15.10.1853 – 11.06.1941), \emph{Schriftsteller, Journalist, Kritiker}!Erklaerung [Schleier der Beatrice]1900-09-14@\strich\emph{Erklärung [Schleier der Beatrice]} {[}1900-09-14{]}|pwv}\pwindex{Hirschfeld, Robert 17.09.1857 – 02.04.1914@\textsc{Hirschfeld, Robert} (17.09.1857 – 02.04.1914), \emph{Journalist, Kritiker}!Erklaerung [Schleier der Beatrice]1900-09-14@\strich\emph{Erklärung [Schleier der Beatrice]} {[}1900-09-14{]}|pwv}\pwindex{Speidel, Ludwig 1830-04-11 – 1906-02-03@\textsc{Speidel, Ludwig} (1830-04-11 – 1906-02-03), \emph{Journalist, Kritiker}!Erklaerung [Schleier der Beatrice]1900-09-14@\strich\emph{Erklärung [Schleier der Beatrice]} {[}1900-09-14{]}|pwv}\pwindex{David, Jakob Julius 1859-02-06 – 1906-11-20@\textsc{David, Jakob Julius} (1859-02-06 – 1906-11-20), \emph{Schriftsteller, Journalist}!Erklaerung [Schleier der Beatrice]1900-09-14@\strich\emph{Erklärung [Schleier der Beatrice]} {[}1900-09-14{]}|pwv}}{\lemma{\textnormal{\emph{Preisgebung … Briefes}}}\Cendnote{\textnormal{In der \emph{Erklärung}\pwindex{Bahr, Hermann 19.07.1863 – 15.01.1934@\textsc{Bahr, Hermann} (19.07.1863 – 15.01.1934), \emph{Schriftsteller, Kritiker}!Erklaerung [Schleier der Beatrice]1900-09-14@\strich\emph{Erklärung [Schleier der Beatrice]} {[}1900-09-14{]}|pwk}\pwindex{Salten, Felix 06.09.1869 – 08.10.1945@\textsc{Salten, Felix} (06.09.1869 – 08.10.1945), \emph{Schriftsteller, Journalist}!Erklaerung [Schleier der Beatrice]1900-09-14@\strich\emph{Erklärung [Schleier der Beatrice]} {[}1900-09-14{]}|pwk}\pwindex{Bauer, Julius 15.10.1853 – 11.06.1941@\textsc{Bauer, Julius} (15.10.1853 – 11.06.1941), \emph{Schriftsteller, Journalist, Kritiker}!Erklaerung [Schleier der Beatrice]1900-09-14@\strich\emph{Erklärung [Schleier der Beatrice]} {[}1900-09-14{]}|pwk}\pwindex{Hirschfeld, Robert 17.09.1857 – 02.04.1914@\textsc{Hirschfeld, Robert} (17.09.1857 – 02.04.1914), \emph{Journalist, Kritiker}!Erklaerung [Schleier der Beatrice]1900-09-14@\strich\emph{Erklärung [Schleier der Beatrice]} {[}1900-09-14{]}|pwk}\pwindex{Speidel, Ludwig 1830-04-11 – 1906-02-03@\textsc{Speidel, Ludwig} (1830-04-11 – 1906-02-03), \emph{Journalist, Kritiker}!Erklaerung [Schleier der Beatrice]1900-09-14@\strich\emph{Erklärung [Schleier der Beatrice]} {[}1900-09-14{]}|pwk}\pwindex{David, Jakob Julius 1859-02-06 – 1906-11-20@\textsc{David, Jakob Julius} (1859-02-06 – 1906-11-20), \emph{Schriftsteller, Journalist}!Erklaerung [Schleier der Beatrice]1900-09-14@\strich\emph{Erklärung [Schleier der Beatrice]} {[}1900-09-14{]}|pwk} war auch ein Brief Paul
                     Schlenther\pwindex{Schlenther, Paul 20.08.1854 – 30.04.1916@\textsc{Schlenther, Paul} (20.08.1854 – 30.04.1916), \emph{Schriftsteller, Kritiker, Theaterleiter}|pwk}s an Schnitzler\pwindex{Schnitzler, Arthur 15.05.1862 – 21.10.1931@\textsc{Schnitzler, Arthur} (15.05.1862 – 21.10.1931), \emph{Schriftsteller, Mediziner}|pwk}
                  aufgenommen, dessen Publikation nicht autorisiert war. Vgl. Hermann Bahr\pwindex{Bahr, Hermann 19.07.1863 – 15.01.1934@\textsc{Bahr, Hermann} (19.07.1863 – 15.01.1934), \emph{Schriftsteller, Kritiker}|pwk}, Julius Bauer\pwindex{Bauer, Julius 15.10.1853 – 11.06.1941@\textsc{Bauer, Julius} (15.10.1853 – 11.06.1941), \emph{Schriftsteller, Journalist, Kritiker}|pwk}, J. J.
                        David\pwindex{David, Jakob Julius 1859-02-06 – 1906-11-20@\textsc{David, Jakob Julius} (1859-02-06 – 1906-11-20), \emph{Schriftsteller, Journalist}|pwk}, Robert Hirschfeld\pwindex{Hirschfeld, Robert 17.09.1857 – 02.04.1914@\textsc{Hirschfeld, Robert} (17.09.1857 – 02.04.1914), \emph{Journalist, Kritiker}|pwk}, Felix Salten\pwindex{Salten, Felix 06.09.1869 – 08.10.1945@\textsc{Salten, Felix} (06.09.1869 – 08.10.1945), \emph{Schriftsteller, Journalist}|pwk} und Ludwig Speidel\pwindex{Speidel, Ludwig 1830-04-11 – 1906-02-03@\textsc{Speidel, Ludwig} (1830-04-11 – 1906-02-03), \emph{Journalist, Kritiker}|pwk}: \emph{Erklärung}\pwindex{Bahr, Hermann 19.07.1863 – 15.01.1934@\textsc{Bahr, Hermann} (19.07.1863 – 15.01.1934), \emph{Schriftsteller, Kritiker}!Erklaerung [Schleier der Beatrice]1900-09-14@\strich\emph{Erklärung [Schleier der Beatrice]} {[}1900-09-14{]}|pwk}\pwindex{Salten, Felix 06.09.1869 – 08.10.1945@\textsc{Salten, Felix} (06.09.1869 – 08.10.1945), \emph{Schriftsteller, Journalist}!Erklaerung [Schleier der Beatrice]1900-09-14@\strich\emph{Erklärung [Schleier der Beatrice]} {[}1900-09-14{]}|pwk}\pwindex{Bauer, Julius 15.10.1853 – 11.06.1941@\textsc{Bauer, Julius} (15.10.1853 – 11.06.1941), \emph{Schriftsteller, Journalist, Kritiker}!Erklaerung [Schleier der Beatrice]1900-09-14@\strich\emph{Erklärung [Schleier der Beatrice]} {[}1900-09-14{]}|pwk}\pwindex{Hirschfeld, Robert 17.09.1857 – 02.04.1914@\textsc{Hirschfeld, Robert} (17.09.1857 – 02.04.1914), \emph{Journalist, Kritiker}!Erklaerung [Schleier der Beatrice]1900-09-14@\strich\emph{Erklärung [Schleier der Beatrice]} {[}1900-09-14{]}|pwk}\pwindex{Speidel, Ludwig 1830-04-11 – 1906-02-03@\textsc{Speidel, Ludwig} (1830-04-11 – 1906-02-03), \emph{Journalist, Kritiker}!Erklaerung [Schleier der Beatrice]1900-09-14@\strich\emph{Erklärung [Schleier der Beatrice]} {[}1900-09-14{]}|pwk}\pwindex{David, Jakob Julius 1859-02-06 – 1906-11-20@\textsc{David, Jakob Julius} (1859-02-06 – 1906-11-20), \emph{Schriftsteller, Journalist}!Erklaerung [Schleier der Beatrice]1900-09-14@\strich\emph{Erklärung [Schleier der Beatrice]} {[}1900-09-14{]}|pwk}. In: \emph{Neues Wiener
                        Tagblatt}\pwindex{?? Werk@Nicht ermittelte Verfasserinnen und Verfasser!Neues Wiener Tagblatt1867 – 1945@\emph{Neues Wiener Tagblatt} {[}1867 – 1945{]}|pwk} [u. a.], Jg. 34, Nr. 252, 14. 9. 1900, S. 9–10, hier: S. 9.}}}\label{K_L02931-3h} als inkorrekt bezeichnet\pwindex{?? Werk@Nicht ermittelte Verfasserinnen und Verfasser!Paul Schlenther und die Wiener Kritik1900-09-15@\emph{Paul Schlenther und die Wiener Kritik} {[}1900-09-15{]}|pwv} hat, ſo geſchieht
               dies wohl hauptſächlich darum, \strikeout{d\textcolor{gray}{a}} weil ſich die Berlin\oindex{Berlin@\textbf{Berlin}|pw}er über \label{K_L02931-4v}\edtext{den das »Deutſche Theater\orgindex{Deutsches Theater Berlin@Deutsches Theater Berlin|pw}« betreffenden Paſſus\pwindex{?? Werk@Nicht ermittelte Verfasserinnen und Verfasser!Paul Schlenther und die Wiener Kritik1900-09-15@\emph{Paul Schlenther und die Wiener Kritik} {[}1900-09-15{]}|pwv}}{\lemma{\textnormal{\emph{den … Paſſus}}}\Cendnote{\textnormal{In dem erwähnten, abgedruckten Brief\pwindex{Bahr, Hermann 19.07.1863 – 15.01.1934@\textsc{Bahr, Hermann} (19.07.1863 – 15.01.1934), \emph{Schriftsteller, Kritiker}!Erklaerung [Schleier der Beatrice]1900-09-14@\strich\emph{Erklärung [Schleier der Beatrice]} {[}1900-09-14{]}|pwkv}\pwindex{Salten, Felix 06.09.1869 – 08.10.1945@\textsc{Salten, Felix} (06.09.1869 – 08.10.1945), \emph{Schriftsteller, Journalist}!Erklaerung [Schleier der Beatrice]1900-09-14@\strich\emph{Erklärung [Schleier der Beatrice]} {[}1900-09-14{]}|pwkv}\pwindex{Bauer, Julius 15.10.1853 – 11.06.1941@\textsc{Bauer, Julius} (15.10.1853 – 11.06.1941), \emph{Schriftsteller, Journalist, Kritiker}!Erklaerung [Schleier der Beatrice]1900-09-14@\strich\emph{Erklärung [Schleier der Beatrice]} {[}1900-09-14{]}|pwkv}\pwindex{Hirschfeld, Robert 17.09.1857 – 02.04.1914@\textsc{Hirschfeld, Robert} (17.09.1857 – 02.04.1914), \emph{Journalist, Kritiker}!Erklaerung [Schleier der Beatrice]1900-09-14@\strich\emph{Erklärung [Schleier der Beatrice]} {[}1900-09-14{]}|pwkv}\pwindex{Speidel, Ludwig 1830-04-11 – 1906-02-03@\textsc{Speidel, Ludwig} (1830-04-11 – 1906-02-03), \emph{Journalist, Kritiker}!Erklaerung [Schleier der Beatrice]1900-09-14@\strich\emph{Erklärung [Schleier der Beatrice]} {[}1900-09-14{]}|pwkv}\pwindex{David, Jakob Julius 1859-02-06 – 1906-11-20@\textsc{David, Jakob Julius} (1859-02-06 – 1906-11-20), \emph{Schriftsteller, Journalist}!Erklaerung [Schleier der Beatrice]1900-09-14@\strich\emph{Erklärung [Schleier der Beatrice]} {[}1900-09-14{]}|pwkv}{ }Paul Schlenther\pwindex{Schlenther, Paul 20.08.1854 – 30.04.1916@\textsc{Schlenther, Paul} (20.08.1854 – 30.04.1916), \emph{Schriftsteller, Kritiker, Theaterleiter}|pwk}s warnt dieser Schnitzler\pwindex{Schnitzler, Arthur 15.05.1862 – 21.10.1931@\textsc{Schnitzler, Arthur} (15.05.1862 – 21.10.1931), \emph{Schriftsteller, Mediziner}|pwk} vor dem \emph{Deutschen Theater}\orgindex{Deutsches Theater Berlin@Deutsches Theater Berlin|pwk}, da dieses der
                     »Riesenaufgabe« einer Aufführung von \emph{Der Schleier der Beatrice}\pwindex{Schnitzler, Arthur 15.05.1862 – 21.10.1931@\textsc{Schnitzler, Arthur} (15.05.1862 – 21.10.1931), \emph{Schriftsteller, Mediziner}!Schleier der Beatrice. Schauspiel in fuenf Akten1900-12-01@\strich\emph{Der Schleier der Beatrice. Schauspiel in fünf Akten} {[}1900-12-01{]}|pwk} »nicht gewachsen«
                  sei.}}}\label{K_L02931-4h} ärgern.\pend
           \pstart
           Daß ich \label{K_L02931-123v}\edtext{\textsc{Richard\pwindex{Beer-Hofmann, Richard 1866-07-11 – 1945-09-26@\textsc{Beer-Hofmann, Richard} (1866-07-11 – 1945-09-26), \emph{Schriftsteller}|pw}} verfehlt}{\lemma{\textnormal{\emph{Richard verfehlt}}}\Cendnote{\textnormal{siehe Richard Beer-Hofmann an Arthur Schnitzler, 6. 9. 1900}}}\label{K_L02931-123h} habe, thut mir unendlich leid. Anderſeits war ich ja über {\pb}eine Woche in Wien\oindex{Wien@\textbf{Wien}|pw}; und wenn er wirklich das Bedürfniß gehabt hätte, mit mir zuſammen zu
               ſein, ſo hätte er auch etwas früher \label{K_L02931-5v}\edtext{zurückkommen}{\lemma{\textnormal{\emph{zurückkommen}}}\Cendnote{\textnormal{aus Altaussee\oindex{Altaussee@\textbf{Altaussee}|pwk}, siehe Richard Beer-Hofmann an Arthur Schnitzler, 14. 9. 1900}}}\label{K_L02931-5h} können. Grüße ihn recht herzlich von mir und ſage ihm, daß ich ihm eine der
               wenigen freundlichen \strikeout{Erin} Erinnerungen an \strikeout{\textcolor{gray}{u}} meine diesjährige Urlaubsreiſe danke. Und er ſoll mir \textsc{Mirjam\pwindex{Beer-Hofmann, Mirjam 04.09.1897 – 24.12.1984@\textsc{Beer-Hofmann, Mirjam} (04.09.1897 – 24.12.1984)|pw}s} Wiegenlied\pwindex{Beer-Hofmann, Richard 1866-07-11 – 1945-09-26@\textsc{Beer-Hofmann, Richard} (1866-07-11 – 1945-09-26), \emph{Schriftsteller}!Schlaflied fuer Mirjam15. 11. 1898@\strich\emph{Schlaflied für Mirjam} {[}15. 11. 1898{]}|pwv}
               ſchicken.\pend
           \pstart
           Ich leide, ſeit ich zurück bin, an einem Tag und Nacht andauernden, wühlenden
               Kopfſchmerz, bin vollkommen arbeitsunfähig und fürchte unheimliche Dinge in meinem
               Gehirn. Viele Grüße! Dein {\\}\spacefill\mbox{P. G.}\pend
           \pstart
           \noindent{}{\pb}\label{T_L02931-1v}\edtext{Viele Grüße an die beiden Fräulein\pwindex{Schnitzler, Olga 17.01.1882 – 13.01.1970@\textsc{Schnitzler, Olga} (17.01.1882 – 13.01.1970), \emph{Schauspielerin, Sängerin}|pwv}\pwindex{Steinrueck, Elisabeth 19.11.1885 – 07.04.1920@\textsc{Steinrück, Elisabeth} (19.11.1885 – 07.04.1920)|pwv} aus der \label{K_L02931-7v}\edtext{Rothe-Stern-Gaſſe\oindex{Rotensterngasse@\textbf{Rotensterngasse}|pw}}{\lemma{\textnormal{\emph{Rothe-Stern-Gaſſe}}}\Cendnote{\textnormal{Wohnadresse von Schnitzler\pwindex{Schnitzler, Arthur 15.05.1862 – 21.10.1931@\textsc{Schnitzler, Arthur} (15.05.1862 – 21.10.1931), \emph{Schriftsteller, Mediziner}|pwk}s Partnerin und zukünftiger Ehefrau Olga Gussmann\pwindex{Schnitzler, Olga 17.01.1882 – 13.01.1970@\textsc{Schnitzler, Olga} (17.01.1882 – 13.01.1970), \emph{Schauspielerin, Sängerin}|pwk} und ihrer Schwester Elisabeth\pwindex{Steinrueck, Elisabeth 19.11.1885 – 07.04.1920@\textsc{Steinrück, Elisabeth} (19.11.1885 – 07.04.1920)|pwk} (nachmalig Steinrück\pwindex{Steinrueck, Elisabeth 19.11.1885 – 07.04.1920@\textsc{Steinrück, Elisabeth} (19.11.1885 – 07.04.1920)|pwkv}), vgl. A. S.: \emph{Tagebuch}, 21. 12. 1920}}}\label{K_L02931-7h}!}{\lemma{\textnormal{\emph{Viele … Rothe-Stern-Gaſſe!}}}\Cendnote{\textnormal{kopfüber am oberen Rand der
                     ersten Seite}}}\label{T_L02931-1h}\pend
           
         
         \endnumbering\mylabel{h}\end{ledgroupsized}  \newcommand{\dateiname}{L02931}\newcommand{\titel}{Paul Goldmann an Arthur Schnitzler, 19. 9. [1900]}\newcommand{\editorInnen}{Martin Anton Müller und Laura Untner}%% latex-leseansicht-abspann.tex
%% Abspann für die Leseansicht.
%% Der Schalter \ifkorrekturansicht ist bereits durch den Vorspann gesetzt.

%% latex-abspann.tex
%% Gemeinsamer Abspann für Korrekturansicht und Leseansicht.
%% Setzt den Schalter \ifkorrekturansicht voraus (gesetzt in den
%% einbindenden Dateien latex-korrekturansicht-abspann.tex bzw.
%% latex-leseansicht-abspann.tex).
%% ---------------------------------------------------------------

\normalsize

% Das esempio-Environment wird nur in der Leseansicht benötigt
\ifkorrekturansicht\else
\newenvironment{esempio}[3]%
{
    \vspace{1.5ex}
    \rlap{\underline{#1}}
    \par
    \setlength{\parindent}{0cm}
    \nopagebreak
    \leftskip=#2cm
    \rightskip=#3cm
}
{
    \par
}
\fi

\doendnotes{C}
\bigskip
\vfill

\clearpage

\footnotesize

\ifkorrekturansicht
  \lohead{\textsc{register}}
\fi

% theindex-Environment neu definieren ohne reledmac
\makeatletter
\renewenvironment{theindex}{%
  \ifkorrekturansicht
    \section*{\indexname}%
  \else
    \subsubsection*{Index der erwähnten Entitäten}%
  \fi
  \setlength{\parindent}{0pt}%
  \setlength{\parskip}{0pt plus 0.3pt}%
  \let\item\@idxitem
}{%
  \ifkorrekturansicht\clearpage\fi
}
\makeatother

\IfFileExists{\jobname-pw.ind}{\input{\jobname-pw.ind}}{}

% Quellenangabe nur in der Leseansicht
\ifkorrekturansicht\else
% Fallback-Definitionen, falls die .tex-Datei \titel etc. nicht gesetzt hat
\providecommand{\titel}{}
\providecommand{\editorInnen}{}
\providecommand{\dateiname}{\jobname}

\vspace{3cm}

\vfill

\footnotesize
\textsc{Quelle}: \titel. Herausgegeben von {\editorInnen}. In: \emph{Arthur Schnitzler: Briefwechsel mit Autorinnen und Autoren}.
 Digitale Edition, https://schnitzler-briefe.acdh.oeaw.ac.at/{\dateiname}.html (Stand \today)
\fi

\end{document}


      