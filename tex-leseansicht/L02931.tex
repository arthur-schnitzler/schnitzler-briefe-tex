%% latex-korrekturansicht-vorspann.tex
%% Vorspann für die Korrekturansicht.
%% Lädt die gemeinsame Datei latex-vorspann.tex mit gesetztem Schalter.

\newif\ifkorrekturansicht
\korrekturansichttrue

\input{../tex-inputs/latex-vorspann}


\section[ Paul Goldmann an Arthur Schnitzler, 19. 9. {[}1900{]}]{L02931 Paul Goldmann an Arthur Schnitzler, 19. 9. {[}1900{]}}
\nopagebreak\mylabel{L02931v}
\rehead{ }\normalsize\beginnumbering\briefempfaengerindex{Schnitzler, Arthur@\textsc{Schnitzler, Arthur}!zzzGoldmann, Paul@\emph{von Paul Goldmann}!1900-09-191@{19. 9. {[}1900{]}}|(be}
\toendnotes[C]{\smallbreak\pagebreak[2]}\Standort{DLA, A:Schnitzler, HS.NZ85.1.3170.}
\physDesc{Brief, 1 Blatt, 4 Seiten, 1968 Zeichen
\newline{}Handschrift: blaue Tinte, deutsche Kurrent
\newline{}Schnitzler: 1) mit Bleistift das Jahr »900« sowie »\textcolor{gray}{I)}« vermerkt; letzteres womöglich ein Hinweis auf das
                                 Postskript auf der ersten Seite  2) mit rotem Buntstift drei Unterstreichungen und eine seitliche
                                 Markierung}\toendnotes[C]{\smallbreak}
\pstart
           \raggedleft{}{\pb}Berlin\oindex{Berlin@\textbf{Berlin}, \emph{P.PPLC}|pw}, 19. September.\pend
           
\pstart\center{}Mein lieber \label{K_L02931-1v}\edtext{Onkel\pwindex{Mamroth, Fedor 21.02.1851 – 25.06.1907@\textsc{Mamroth, Fedor} (21.02.1851 – 25.06.1907), \emph{Journalist/Journalistin, Kritiker/Kritikerin}|pwv}}{\lemma{\textnormal{\emph{Onkel}}}\Cendnote{\textnormal{Unachtsamkeit und Verwechslung der
                     Anrede mit jener für Fedor Mamroth\pwindex{Mamroth, Fedor 21.02.1851 – 25.06.1907@\textsc{Mamroth, Fedor} (21.02.1851 – 25.06.1907), \emph{Journalist/Journalistin, Kritiker/Kritikerin}|pwk} oder,
                     innerhalb der Korrespondenz untypisch, Witz}}}\label{K_L02931-1},\pend\vspace{0.5em}
\pstart
           Den \label{K_L02931-2v}\edtext{Artikel\pwindex{Paul Schlenther und die Wiener Kritik@\emph{Paul Schlenther und die Wiener Kritik}|pwv} des »Berliner Tageblatt\pwindex{Berliner Tageblatt@\emph{Berliner Tageblatt}|pw}«}{\lemma{\textnormal{\emph{Artikel … Tageblatt«}}}\Cendnote{\textnormal{[O. V.]: \emph{Paul Schlenther und die Wiener
                        Kritik}\pwindex{Paul Schlenther und die Wiener Kritik@\emph{Paul Schlenther und die Wiener Kritik}|pwk}. In: \emph{Berliner Tageblatt}\pwindex{Berliner Tageblatt@\emph{Berliner Tageblatt}|pwk},
                     Jg. 29, Nr. 470, 15. 9. 1900, Abend-Ausgabe,
                     S. 1–2.}}}\label{K_L02931-2} hatte ich natürlich, unter Hervorhebung der Dir günſtigen
               Stellen, telegraphirt; die Redaktion\orgindex{Neue Freie Presse@Neue Freie Presse|pwv} hat mein Telegramm, wie ich heut
               ſehe, nicht veröffentlicht (was ich Dir im Vertrauen mittheile).\pend
           
\pstart
           Im Übrigen iſt die \label{K_L02931-3v}\edtext{Affaire}{\lemma{\textnormal{\emph{Affaire}}}\Cendnote{\textnormal{Siehe Richard Beer-Hofmann an Arthur Schnitzler, 14. 9. 1900.
               }}}\label{K_L02931-3} ſehr günſtig für Dich und ſehr ungünſtig für Herrn \textsc{Schlenther\pwindex{Schlenther, Paul 20.08.1854 – 30.04.1916@\textsc{Schlenther, Paul} (20.08.1854 – 30.04.1916), \emph{Schriftsteller/Schriftstellerin, Kritiker/Kritikerin, Theaterleiter/Theaterleiterin}|pw}}. Selbſt in Berlin\oindex{Berlin@\textbf{Berlin}, \emph{P.PPLC}|pw} war man genöthigt, ihm
               harte Wahrheiten zu ſagen. Und was auch die Leute darüber ſagen, – und obwohl Du
               ſelbſt {\pb}ganz gewiß nicht \uline{dieſen} Zweck im Auge gehabt haſt, – die Wirkung iſt: \strikeout{alle} alle Welt iſt auf Dein Stück\pwindex{Schleier der Beatrice. Schauspiel in fuenf Akten@\emph{Der Schleier der Beatrice. Schauspiel in fünf Akten}|pwv} aufmerkſam geworden, und die Bühnen
               haben einen Grund mehr, Dich aufzuführen. Daß die Fernſtehenden durch die Affaire ein
               falſches Bild von \strikeout{d} Dir gewinnen könnten, ſoll Dich
               nicht kümmern. Erſtens ſehe ich nicht ein, aus welchem Grunde. Und zweitens, ſelbſt
               wenn es ſo ſein ſollte: glaubſt Du, ſie haben vorher ein richtiges Bild von Dir
               gehabt? {\pb}Immerhin iſt zu conſtatiren, daß von den
                  Berlin\oindex{Berlin@\textbf{Berlin}, \emph{P.PPLC}|pw}er Blättern, die Dir doch gewiß
               fernſtehen, keines ſich in einer Weiſe über Dich geäußert hat, die Dich hätte
               verletzen können. Und wenn das Berliner
                  Tageblatt\pwindex{Berliner Tageblatt@\emph{Berliner Tageblatt}|pw} die \label{K_L02931-4v}\edtext{Preisgebung des \textsc{Schlenther\pwindex{Schlenther, Paul 20.08.1854 – 30.04.1916@\textsc{Schlenther, Paul} (20.08.1854 – 30.04.1916), \emph{Schriftsteller/Schriftstellerin, Kritiker/Kritikerin, Theaterleiter/Theaterleiterin}|pw}}’ſchen Briefes\pwindex{Erklaerung [Schleier der Beatrice]@\emph{Erklärung [Schleier der Beatrice]}|pwv}}{\lemma{\textnormal{\emph{Preisgebung … Briefes}}}\Cendnote{\textnormal{In die \emph{Erklärung}\pwindex{Erklaerung [Schleier der Beatrice]@\emph{Erklärung [Schleier der Beatrice]}|pwk} war auch ein Brief Paul
                     Schlenthers\pwindex{Schlenther, Paul 20.08.1854 – 30.04.1916@\textsc{Schlenther, Paul} (20.08.1854 – 30.04.1916), \emph{Schriftsteller/Schriftstellerin, Kritiker/Kritikerin, Theaterleiter/Theaterleiterin}|pwk} an Schnitzler
                  aufgenommen worden, dessen Publikation nicht autorisiert war. Vgl. Hermann Bahr\pwindex{Bahr, Hermann 19.07.1863 – 15.01.1934@\textsc{Bahr, Hermann} (19.07.1863 – 15.01.1934), \emph{Schriftsteller/Schriftstellerin, Kritiker/Kritikerin}|pwk}, Julius Bauer\pwindex{Bauer, Julius 15.10.1853 – 11.06.1941@\textsc{Bauer, Julius} (15.10.1853 – 11.06.1941), \emph{Schriftsteller/Schriftstellerin, Journalist/Journalistin, Kritiker/Kritikerin}|pwk}, J. J.
                        David\pwindex{David, Jakob Julius 1859-02-06 – 1906-11-20@\textsc{David, Jakob Julius} (1859-02-06 – 1906-11-20), \emph{Schriftsteller/Schriftstellerin, Journalist/Journalistin}|pwk}, Robert Hirschfeld\pwindex{Hirschfeld, Robert 17.09.1857 – 02.04.1914@\textsc{Hirschfeld, Robert} (17.09.1857 – 02.04.1914), \emph{Journalist/Journalistin, Musikkritiker/Musikkritikerin}|pwk}, Felix Salten\pwindex{Salten, Felix 06.09.1869 – 08.10.1945@\textsc{Salten, Felix} (06.09.1869 – 08.10.1945), \emph{Schriftsteller/Schriftstellerin, Journalist/Journalistin, Chefredakteur/Chefredakteurin}|pwk} und Ludwig Speidel\pwindex{Speidel, Ludwig 1830-04-11 – 1906-02-03@\textsc{Speidel, Ludwig} (1830-04-11 – 1906-02-03), \emph{Journalist/Journalistin, Kritiker/Kritikerin}|pwk}: \emph{Erklärung}\pwindex{Erklaerung [Schleier der Beatrice]@\emph{Erklärung [Schleier der Beatrice]}|pwk}. In: \emph{Neues Wiener
                        Tagblatt}\pwindex{Neues Wiener Tagblatt@\emph{Neues Wiener Tagblatt}|pwk} [u. a.], Jg. 34, Nr. 252, 14. 9. 1900, S. 9–10, hier: S. 9.}}}\label{K_L02931-4} als inkorrekt bezeichnet\pwindex{Paul Schlenther und die Wiener Kritik@\emph{Paul Schlenther und die Wiener Kritik}|pwv} hat, ſo geſchieht
               dies wohl hauptſächlich darum, \strikeout{d\textcolor{gray}{a}} weil ſich die Berlin\oindex{Berlin@\textbf{Berlin}, \emph{P.PPLC}|pw}er über \label{K_L02931-5v}\edtext{den das »Deutſche Theater\orgindex{Deutsches Theater Berlin@Deutsches Theater Berlin|pw}« betreffenden Paſſus\pwindex{Paul Schlenther und die Wiener Kritik@\emph{Paul Schlenther und die Wiener Kritik}|pwv}}{\lemma{\textnormal{\emph{den … Paſſus}}}\Cendnote{\textnormal{In dem erwähnten, abgedruckten Brief\pwindex{Erklaerung [Schleier der Beatrice]@\emph{Erklärung [Schleier der Beatrice]}|pwkv}{ }Paul Schlenthers\pwindex{Schlenther, Paul 20.08.1854 – 30.04.1916@\textsc{Schlenther, Paul} (20.08.1854 – 30.04.1916), \emph{Schriftsteller/Schriftstellerin, Kritiker/Kritikerin, Theaterleiter/Theaterleiterin}|pwk} warnt dieser Schnitzler vor dem \emph{Deutschen Theater}\orgindex{Deutsches Theater Berlin@Deutsches Theater Berlin|pwk}, da dieses der
                     »Riesenaufgabe« einer Aufführung von \emph{Der Schleier der Beatrice}\pwindex{Schleier der Beatrice. Schauspiel in fuenf Akten@\emph{Der Schleier der Beatrice. Schauspiel in fünf Akten}|pwk} »nicht gewachsen«
                  sei.}}}\label{K_L02931-5} ärgern.\pend
           
\pstart
           Daß ich \label{K_L02931-6v}\edtext{\textsc{Richard\pwindex{Beer-Hofmann, Richard 1866-07-11 – 1945-09-26@\textsc{Beer-Hofmann, Richard} (1866-07-11 – 1945-09-26), \emph{Schriftsteller/Schriftstellerin}|pw}} verfehlt}{\lemma{\textnormal{\emph{Richard verfehlt}}}\Cendnote{\textnormal{Siehe Richard Beer-Hofmann an Arthur Schnitzler, 6. 9. 1900.
               }}}\label{K_L02931-6} habe, thut mir unendlich leid. Anderſeits war ich ja über {\pb}eine Woche in Wien\oindex{Wien@\textbf{Wien}, \emph{A.ADM2}|pw}; und wenn er wirklich das Bedürfniß gehabt hätte, mit mir zuſammen zu
               ſein, ſo hätte er auch etwas früher \label{K_L02931-7v}\edtext{zurückkommen}{\lemma{\textnormal{\emph{zurückkommen}}}\Cendnote{\textnormal{aus Altaussee\oindex{Altaussee@\textbf{Altaussee}, \emph{A.ADM3}|pwk}, siehe Richard Beer-Hofmann an Arthur Schnitzler, 14. 9. 1900.}}}\label{K_L02931-7} können. Grüße ihn recht herzlich von mir und ſage ihm, daß ich ihm eine der
               wenigen freundlichen \strikeout{Erin} Erinnerungen an \strikeout{\textcolor{gray}{u}} meine diesjährige Urlaubsreiſe danke. Und er ſoll mir \textsc{Mirjams\pwindex{Beer-Hofmann, Mirjam 04.09.1897 – 24.12.1984@\textsc{Beer-Hofmann, Mirjam} (04.09.1897 – 24.12.1984)|pw}} Wiegenlied\pwindex{Schlaflied fuer Mirjam@\emph{Schlaflied für Mirjam}|pwv}
               ſchicken.\pend
           
\pstart
           Ich leide, ſeit ich zurück bin, an einem Tag und Nacht andauernden, wühlenden
               Kopfſchmerz, bin vollkommen arbeitsunfähig und fürchte unheimliche Dinge in meinem
               Gehirn. Viele Grüße! Dein {\\}\spacefill\mbox{P. G.}\pend
           
\pstart
           \noindent{}{\pb}\label{T_L02931-1v}\edtext{Viele Grüße an die beiden Fräulein\pwindex{Schnitzler, Olga 17.01.1882 – 13.01.1970@\textsc{Schnitzler, Olga} (17.01.1882 – 13.01.1970), \emph{Schauspieler/Schauspielerin, Sänger/Sängerin}|pwv}\pwindex{Steinrueck, Elisabeth 19.11.1885 – 07.04.1920@\textsc{Steinrück, Elisabeth} (19.11.1885 – 07.04.1920)|pwv} aus der \label{K_L02931-8v}\edtext{Rothe-Stern-Gaſſe\oindex{Rotensterngasse@\textbf{Rotensterngasse}, \emph{Straße (K.STR)}|pw}}{\lemma{\textnormal{\emph{Rothe-Stern-Gaſſe}}}\Cendnote{\textnormal{Wohnadresse von Schnitzlers Partnerin und zukünftiger Ehefrau Olga Gussmann\pwindex{Schnitzler, Olga 17.01.1882 – 13.01.1970@\textsc{Schnitzler, Olga} (17.01.1882 – 13.01.1970), \emph{Schauspieler/Schauspielerin, Sänger/Sängerin}|pwk} und ihrer Schwester Elisabeth\pwindex{Steinrueck, Elisabeth 19.11.1885 – 07.04.1920@\textsc{Steinrück, Elisabeth} (19.11.1885 – 07.04.1920)|pwk} (nachmalig Steinrück\pwindex{Steinrueck, Elisabeth 19.11.1885 – 07.04.1920@\textsc{Steinrück, Elisabeth} (19.11.1885 – 07.04.1920)|pwkv}), vgl. A. S.: \emph{Tagebuch}, 21. 12. 1920.}}}\label{K_L02931-8}!}{\lemma{\textnormal{\emph{Viele … Rothe-Stern-Gaſſe!}}}\Cendnote{\textnormal{kopfüber am oberen Rand der
                     ersten Seite}}}\label{T_L02931-1}\pend
           \selectlanguage{ngerman}\endnumbering\briefempfaengerindex{Schnitzler, Arthur@\textsc{Schnitzler, Arthur}!zzzGoldmann, Paul@\emph{von Paul Goldmann}!1900-09-191@{19. 9. {[}1900{]}}|)be}\mylabel{L02931h}  \normalsize

\doendnotes{C}
\bigskip
\vfill

\clearpage

\footnotesize

\lohead{\textsc{register}}

% Definiere theindex-Environment komplett neu ohne reledmac
\makeatletter
\renewenvironment{theindex}{%
  \section*{\indexname}%
  \setlength{\parindent}{0pt}%
  \setlength{\parskip}{0pt plus 0.3pt}%
  \let\item\@idxitem
}{%
  \clearpage
}
\makeatother

\IfFileExists{\jobname-pw.ind}{\input{\jobname-pw.ind}}{}

\end{document}

      