%% latex-leseansicht-vorspann.tex
%% Vorspann für die Leseansicht.
%% Lädt die gemeinsame Datei latex-vorspann.tex mit nicht gesetztem Schalter.

\newif\ifkorrekturansicht
\korrekturansichtfalse

\input{../tex-inputs/latex-vorspann}


         
         \renewcommand{\erwaehntePersonen}{Personen:  Florus, Otto Fürth, Titus Maccius Plautus, Gaius Valerius Flaccus}
         \renewcommand{\erwaehnteOrte}{Orte: England, Italien, Wien}
         \renewcommand{\erwaehnteWerke}{Werke: Die Schwestern oder Casanova in Spa. Lustspiel in Versen, Träume auf der Asphodelosinsel}
               \section[Robert Adam an Arthur Schnitzler, 13. 2. 1920]{ Robert Adam an Arthur Schnitzler, 13. 2. 1920}\nopagebreak\mylabel{v}\rehead{ }\begin{ledgroupsized}[t]{13cm}\normalsize\beginnumbering \toendnotes[C]{\smallbreak\pagebreak[2]} \Standort{CUL, Schnitzler, B 1.}
\physDesc{Brief, 1 Blatt, 4 Seiten, 3814 Zeichen
\newline{}Handschrift: blaue Tinte, deutsche Kurrent
\newline{}Schnitzler: 1) mit Bleistift beschriftet: »\textsc{Adam}«  2) mit rotem Buntstift vereinzelte Unterstreichungen
\newline{}Ordnung: mit Bleistift von unbekannter Hand nummeriert:
                                    »14« }\Standort{Wien, Österreichische Nationalbibliothek, Cod.ser. 52.268, 57 recto, 61.}
\physDesc{handschriftliche Abschrift, 2 Blätter, 2 Seiten
\newline{}Handschrift: schwarze Tinte, Gabelsberger Kurzschrift}\Standort{Wien, Österreichische Nationalbibliothek, Cod.ser. 52.268, 57 recto, 61.}
\physDesc{maschinenschriftliche Abschrift, 2 Blätter, 2 Seiten
\newline{}Schreibmaschine}\toendnotes[C]{\smallbreak}\pstart
           \raggedleft{}{\pb}Wien\oindex{Wien@\textbf{Wien}|pw}, am 13. Februar 1920\pend
           \pstart\center{}Hochverehrter Herr Doktor!\pend\pstart
           Für Ihre »Schweſtern\pwindex{Schnitzler, Arthur 15.05.1862 – 21.10.1931@\textsc{Schnitzler, Arthur} (15.05.1862 – 21.10.1931), \emph{Schriftsteller, Mediziner}!Schwestern oder Casanova in Spa. Lustspiel in Versen01. 10. 1919@\strich\emph{Die Schwestern oder Casanova in Spa. Lustspiel in Versen} {[}01. 10. 1919{]}|pw}«, die mir geſtern zukamen,
               meinen beſten Dank! Ich habe ſie ſofort geleſen, ſehr begierig, Sie wieder, nach
               langer Zeit, in Versen reden zu hören. Der Vers iſt mir, dem Mann der alten Schule,
               doch immer das Berufsgewand des Dichters, nicht ein Salonanzug, und mir will
               ſcheinen, daß man im Berufsgewand am freieſten und förderlichſten Arbeit leiſtet.
               Ihre Verſe fließen wundervoll und leihen Ihren Gedanken neuen Reiz, ohne ihnen die
               charakteriſtiſchen Eigentümlichkeiten Ihrer Proſa zu nehmen. Ich sage dies, obwohl
               ich den Blankvers, der nur der einſilbigen engliſchen\oindex{England@\textbf{England}|pw} oder noch der verſchleifenden italieniſchen\oindex{Italien@\textbf{Italien}|pw}{ }Sprache angemeſſen iſt, im Deutſchen ſonſt herzlich
               haſſe (was ich Ihnen ſchon geſagt habe); denn der deutſche Blankvers, mei{\pb}ſterhaft gehandhabt, das iſt gemeiſtert,
               das iſt oft gebrochen, gezerrt, gepreßt, iſt ein unerträgliches Geſchöpf, eine
               endloſe Melodie, ein ſtätiges Meeresrauſchen. (Mir iſt dies jetzt wieder klar
               geworden, da ich ein gerade erſchienenes Buch eines Vetters: »Träume auf der Aſphodelosinſel\pwindex{Fuerth, Otto 07.11.1894 – 15.12.1979@\textsc{Fürth, Otto} (07.11.1894 – 15.12.1979), \emph{Schriftsteller}!Traeume auf der Asphodelosinsel1920@\strich\emph{Träume auf der Asphodelosinsel} {[}1920{]}|pw}«, ein philoſophiſches
               Troſtbüchlein in Verſen von \textsc{Otto Fürth}\pwindex{Fuerth, Otto 07.11.1894 – 15.12.1979@\textsc{Fürth, Otto} (07.11.1894 – 15.12.1979), \emph{Schriftsteller}|pw}, leſe, ein ſehr klar und geiſtvoll geſchriebenes Buch, deſſen Blankvers blitz
               und blank iſt und deshalb endlos wogt und flutet: was ja im konkreten Falle
               vielleicht nicht übel iſt, da es zur Stärkung der Illuſion, man ſei auf einer Inſel,
               gewiß beiträgt. Der deutſche Vers \textsc{par excellence}{ }ſcheint mir doch der Knittelvers zu ſein.)\pend
           \pstart
           Ich bewundere Ihre großartige Charakteriſierung des Caſanova\pwindex{Schnitzler, Arthur 15.05.1862 – 21.10.1931@\textsc{Schnitzler, Arthur} (15.05.1862 – 21.10.1931), \emph{Schriftsteller, Mediziner}!Schwestern oder Casanova in Spa. Lustspiel in Versen01. 10. 1919@\strich\emph{Die Schwestern oder Casanova in Spa. Lustspiel in Versen} {[}01. 10. 1919{]}|pwv}-Milieus; jede der Geſtalten der Komödie iſt auf
                  Caſanova\pwindex{Schnitzler, Arthur 15.05.1862 – 21.10.1931@\textsc{Schnitzler, Arthur} (15.05.1862 – 21.10.1931), \emph{Schriftsteller, Mediziner}!Schwestern oder Casanova in Spa. Lustspiel in Versen01. 10. 1919@\strich\emph{Die Schwestern oder Casanova in Spa. Lustspiel in Versen} {[}01. 10. 1919{]}|pwv} abgeſtellt, dazu
               geboren, einmal mit ihm zuſammenzutreffen, ohne das Abenteuer Caſanova\pwindex{Schnitzler, Arthur 15.05.1862 – 21.10.1931@\textsc{Schnitzler, Arthur} (15.05.1862 – 21.10.1931), \emph{Schriftsteller, Mediziner}!Schwestern oder Casanova in Spa. Lustspiel in Versen01. 10. 1919@\strich\emph{Die Schwestern oder Casanova in Spa. Lustspiel in Versen} {[}01. 10. 1919{]}|pwv} nicht zu denken. Und dabei tragen
               die meiſten einen oder den andern Zug, den Caſanova\pwindex{Schnitzler, Arthur 15.05.1862 – 21.10.1931@\textsc{Schnitzler, Arthur} (15.05.1862 – 21.10.1931), \emph{Schriftsteller, Mediziner}!Schwestern oder Casanova in Spa. Lustspiel in Versen01. 10. 1919@\strich\emph{Die Schwestern oder Casanova in Spa. Lustspiel in Versen} {[}01. 10. 1919{]}|pwv} gezeigt hat oder dereinſt zeigen wird; wie \textsc{Gudar}\pwindex{Schnitzler, Arthur 15.05.1862 – 21.10.1931@\textsc{Schnitzler, Arthur} (15.05.1862 – 21.10.1931), \emph{Schriftsteller, Mediziner}!Schwestern oder Casanova in Spa. Lustspiel in Versen01. 10. 1919@\strich\emph{Die Schwestern oder Casanova in Spa. Lustspiel in Versen} {[}01. 10. 1919{]}|pwv} einmal etwas wie Caſanova\pwindex{Schnitzler, Arthur 15.05.1862 – 21.10.1931@\textsc{Schnitzler, Arthur} (15.05.1862 – 21.10.1931), \emph{Schriftsteller, Mediziner}!Schwestern oder Casanova in Spa. Lustspiel in Versen01. 10. 1919@\strich\emph{Die Schwestern oder Casanova in Spa. Lustspiel in Versen} {[}01. 10. 1919{]}|pwv}{ }{\pb}geweſen iſt, wird \textsc{Tito} wohl ſeinerzeit zu einem werden; und in \textsc{Santis}\pwindex{Schnitzler, Arthur 15.05.1862 – 21.10.1931@\textsc{Schnitzler, Arthur} (15.05.1862 – 21.10.1931), \emph{Schriftsteller, Mediziner}!Schwestern oder Casanova in Spa. Lustspiel in Versen01. 10. 1919@\strich\emph{Die Schwestern oder Casanova in Spa. Lustspiel in Versen} {[}01. 10. 1919{]}|pwv}{ }ſammeln ſich jene üblen Eigenſchaften, die der
               alternde Caſanova\pwindex{Schnitzler, Arthur 15.05.1862 – 21.10.1931@\textsc{Schnitzler, Arthur} (15.05.1862 – 21.10.1931), \emph{Schriftsteller, Mediziner}!Schwestern oder Casanova in Spa. Lustspiel in Versen01. 10. 1919@\strich\emph{Die Schwestern oder Casanova in Spa. Lustspiel in Versen} {[}01. 10. 1919{]}|pwv} in \strikeout{\textcolor{gray}{geeigne}} Panne-Situationen hervorkehrt, zu \strikeout{eigner} einer
               eigenen, aber gutmütig-ſchäbigen Geſtalt. Nur mit dem \textsc{Andrea}\pwindex{Schnitzler, Arthur 15.05.1862 – 21.10.1931@\textsc{Schnitzler, Arthur} (15.05.1862 – 21.10.1931), \emph{Schriftsteller, Mediziner}!Schwestern oder Casanova in Spa. Lustspiel in Versen01. 10. 1919@\strich\emph{Die Schwestern oder Casanova in Spa. Lustspiel in Versen} {[}01. 10. 1919{]}|pwv} bin ich, um aufrichtig zu ſein, nicht ganz einverſtanden; ich hätte ihn um ein
               gut Teil mehr \textsc{Bourgeois} gewünſcht; daß er das Mädel, mit
               dem er durchgeht, heiraten will, daß er nur einmal ſpielt und daß er darob trotz
               Gewinns Reue empfindet, macht dem Sohn ehrbarer Bürger alle Ehre; aber ich meine, er
               müßte die Dukaten noch mit viel ſchwererem Herzen hergeben und nicht 1050, ſondern
               ſagen wir 950. Auch im \textsc{Problema}-Streit iſt er mir zu
               freiſinnig, zu großzügig, zu amoraliſch; mag dies auch gewiß dem Zeitalter
               entſprechen, ſo entgeht doch, ſcheint mir, dem Drama dadurch ein ſcharfer Kontraſt.
               Hingegen sind die zwei, nein drei Caſanova\pwindex{Schnitzler, Arthur 15.05.1862 – 21.10.1931@\textsc{Schnitzler, Arthur} (15.05.1862 – 21.10.1931), \emph{Schriftsteller, Mediziner}!Schwestern oder Casanova in Spa. Lustspiel in Versen01. 10. 1919@\strich\emph{Die Schwestern oder Casanova in Spa. Lustspiel in Versen} {[}01. 10. 1919{]}|pwv}-Damen herrlich, \textsc{Flaminia}\pwindex{Schnitzler, Arthur 15.05.1862 – 21.10.1931@\textsc{Schnitzler, Arthur} (15.05.1862 – 21.10.1931), \emph{Schriftsteller, Mediziner}!Schwestern oder Casanova in Spa. Lustspiel in Versen01. 10. 1919@\strich\emph{Die Schwestern oder Casanova in Spa. Lustspiel in Versen} {[}01. 10. 1919{]}|pwv} wie \textsc{Anina}\pwindex{Schnitzler, Arthur 15.05.1862 – 21.10.1931@\textsc{Schnitzler, Arthur} (15.05.1862 – 21.10.1931), \emph{Schriftsteller, Mediziner}!Schwestern oder Casanova in Spa. Lustspiel in Versen01. 10. 1919@\strich\emph{Die Schwestern oder Casanova in Spa. Lustspiel in Versen} {[}01. 10. 1919{]}|pwv} und \textsc{Theresa}\pwindex{Schnitzler, Arthur 15.05.1862 – 21.10.1931@\textsc{Schnitzler, Arthur} (15.05.1862 – 21.10.1931), \emph{Schriftsteller, Mediziner}!Schwestern oder Casanova in Spa. Lustspiel in Versen01. 10. 1919@\strich\emph{Die Schwestern oder Casanova in Spa. Lustspiel in Versen} {[}01. 10. 1919{]}|pwv}. Daß die große Szene zwiſchen \textsc{Flaminia}\pwindex{Schnitzler, Arthur 15.05.1862 – 21.10.1931@\textsc{Schnitzler, Arthur} (15.05.1862 – 21.10.1931), \emph{Schriftsteller, Mediziner}!Schwestern oder Casanova in Spa. Lustspiel in Versen01. 10. 1919@\strich\emph{Die Schwestern oder Casanova in Spa. Lustspiel in Versen} {[}01. 10. 1919{]}|pwv} und \textsc{Anina}\pwindex{Schnitzler, Arthur 15.05.1862 – 21.10.1931@\textsc{Schnitzler, Arthur} (15.05.1862 – 21.10.1931), \emph{Schriftsteller, Mediziner}!Schwestern oder Casanova in Spa. Lustspiel in Versen01. 10. 1919@\strich\emph{Die Schwestern oder Casanova in Spa. Lustspiel in Versen} {[}01. 10. 1919{]}|pwv} im zweiten Akte bei der Aufführung etwas – für Moraliſch-Imprägnierte –
                  Bedenk{\pb}liches haben dürfte, kann ich
               nicht verkennen; zu fein geſpielt dürften die beiden Damen zu viel von ihrer
               Schweſterſchaft einbüßen, und eine Vergröberung aus der fein gedachten und geformten
               Szene eine ſehr unangenehme jenes Neides machen, für den der Wien\oindex{Wien@\textbf{Wien}|pw}er einen nicht wiederzugebenden Ausdruck hat. –\pend
           \pstart
           Daß ich mich nie mit etwas Gedrucktem revanchieren kann, betrübt mich tief. Aber die
               Zeiten wollen daran nichts ändern. Ich ſchreibe gar nichts und vertiefe mich, wenn
               ich nicht an Akten arbeite, in die alten Italiener\oindex{Italien@\textbf{Italien}|pw} und – das iſt meine letzte Leidenſchaft – Lateiner: \textsc{Vergil}\pwindex{Fuerth, Otto 07.11.1894 – 15.12.1979@\textsc{Fürth, Otto} (07.11.1894 – 15.12.1979), \emph{Schriftsteller}|pw} (den ich erſt jetzt auf's Höchſte verehren lernte), \textsc{Plautus}\pwindex{Plautus, Titus Maccius um 250 v. u. Z. – 184 v. u. Z.@\textsc{Plautus, Titus Maccius} (um 250 v. u. Z. – 184 v. u. Z.), \emph{Schriftsteller}|pw}, \textsc{Valerius Flaccus}\pwindex{Valerius Flaccus, Gaius †~vor 90@\textsc{Valerius Flaccus, Gaius} (†~vor 90), \emph{Schriftsteller}|pw}, \textsc{Florus}\pwindex{Florus 1./2. Jh.@\textsc{Florus} (1./2. Jh.), \emph{Geschichtsschreiber}|pw} und andere. Ich habe ſchon einen ganzen Stoß römiſcher Autoren zuſammengekauft;
               es iſt ein Lichtblick in ſchwarzen Tagen, daß die Valutaentwertung auf das klaſſiſche
               Altertum nur mit ungefähr 50 {\%} rückwirkt. –\pend
           \pstart
           Nochmals vielen Dank und die ergebenſten Grüße!\hspace*{3.5em}Ihr\pend
           \pstart \spacefill\mbox{D\textsuperscript{r}RAdam}\pend{}
         
         \endnumbering\mylabel{h}\end{ledgroupsized}  \newcommand{\dateiname}{L02335}\newcommand{\titel}{Robert Adam an Arthur Schnitzler, 13. 2. 1920}\newcommand{\editorInnen}{Martin Anton Müller und Gerd-Hermann Susen}%% latex-leseansicht-abspann.tex
%% Abspann für die Leseansicht.
%% Der Schalter \ifkorrekturansicht ist bereits durch den Vorspann gesetzt.

%% latex-abspann.tex
%% Gemeinsamer Abspann für Korrekturansicht und Leseansicht.
%% Setzt den Schalter \ifkorrekturansicht voraus (gesetzt in den
%% einbindenden Dateien latex-korrekturansicht-abspann.tex bzw.
%% latex-leseansicht-abspann.tex).
%% ---------------------------------------------------------------

\normalsize

% Das esempio-Environment wird nur in der Leseansicht benötigt
\ifkorrekturansicht\else
\newenvironment{esempio}[3]%
{
    \vspace{1.5ex}
    \rlap{\underline{#1}}
    \par
    \setlength{\parindent}{0cm}
    \nopagebreak
    \leftskip=#2cm
    \rightskip=#3cm
}
{
    \par
}
\fi

\doendnotes{C}
\bigskip
\vfill

\clearpage

\footnotesize

\ifkorrekturansicht
  \lohead{\textsc{register}}
\fi

% theindex-Environment neu definieren ohne reledmac
\makeatletter
\renewenvironment{theindex}{%
  \ifkorrekturansicht
    \section*{\indexname}%
  \else
    \subsubsection*{Index der erwähnten Entitäten}%
  \fi
  \setlength{\parindent}{0pt}%
  \setlength{\parskip}{0pt plus 0.3pt}%
  \let\item\@idxitem
}{%
  \ifkorrekturansicht\clearpage\fi
}
\makeatother

\IfFileExists{\jobname-pw.ind}{\input{\jobname-pw.ind}}{}

% Quellenangabe nur in der Leseansicht
\ifkorrekturansicht\else
% Fallback-Definitionen, falls die .tex-Datei \titel etc. nicht gesetzt hat
\providecommand{\titel}{}
\providecommand{\editorInnen}{}
\providecommand{\dateiname}{\jobname}

\vspace{3cm}

\vfill

\footnotesize
\textsc{Quelle}: \titel. Herausgegeben von {\editorInnen}. In: \emph{Arthur Schnitzler: Briefwechsel mit Autorinnen und Autoren}.
 Digitale Edition, https://schnitzler-briefe.acdh.oeaw.ac.at/{\dateiname}.html (Stand \today)
\fi

\end{document}


      