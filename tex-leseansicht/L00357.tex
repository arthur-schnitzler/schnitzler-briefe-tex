%% latex-korrekturansicht-vorspann.tex
%% Vorspann für die Korrekturansicht.
%% Lädt die gemeinsame Datei latex-vorspann.tex mit gesetztem Schalter.

\newif\ifkorrekturansicht
\korrekturansichttrue

\input{../tex-inputs/latex-vorspann}


\section[Hermann Bahr an Arthur Schnitzler, 20. 7. 1894]{L00357 Hermann Bahr an Arthur Schnitzler, 20. 7. 1894}
\nopagebreak\mylabel{L00357v}
\rehead{ }\normalsize\beginnumbering\briefempfaengerindex{Schnitzler, Arthur@\textsc{Schnitzler, Arthur}!zzzBahr, Hermann@\emph{von Hermann Bahr}!1894-07-201@{20. 7. 1894}|(be}
\toendnotes[C]{\smallbreak\pagebreak[2]}\Standort{CUL, Schnitzler, B 5b.}
\physDesc{Kartenbrief, 313 Zeichen
\newline{}Handschrift: 1) Bleistift, deutsche Kurrent\hspace{1em}2) Bleistift, lateinische Kurrent (\noindent{}Adresse)\hspace{1em}
\newline{}Versand: 1) Stempel: »\nobreak{}\oindex{Grundlsee [Gemeinde]@\textbf{Grundlsee [Gemeinde]}, \emph{Besiedelter Ort (A.BSO)}|pwk}Grundlsee, 20/7\nobreak{}«.   2) Stempel: »\nobreak{}\oindex{Bad Ischl@\textbf{Bad Ischl}, \emph{P.PPL}|pwk}Ischl, 20 7 94, 12M\nobreak{}«. 
\newline{}Schnitzler: mit Bleistift datiert: »20/7 94« 
\newline{}Ordnung: 1) mit rotem Buntstift von unbekannter Hand nummeriert:
                                    »24«  2) mit Bleistift von unbekannter Hand nummeriert:
                                    »24«}
\buchAbdrucke{\weitereDrucke{Hermann Bahr, Arthur Schnitzler: \emph{Briefwechsel, Aufzeichnungen, Dokumente (1891–1931)}. Göttingen: \emph{Wallstein} 2018, S. 76.} }\pstart{}{\pb}Herrn D\textsuperscript{r} Arthur
                  Schnitzler\pend{}\pstart{}Pension Leop. Petta\oindex{Hotel und Pension Rudolfshoehe (Leopold Petter)@\textbf{Hotel und Pension Rudolfshöhe (Leopold Petter)}, \emph{Hotel (K.HTL)}|pw}{ }Rudolfshöhe\oindex{Hotel und Pension Rudolfshoehe (Leopold Petter)@\textbf{Hotel und Pension Rudolfshöhe (Leopold Petter)}, \emph{Hotel (K.HTL)}|pw}\pend{}\pstart{}Ischl\oindex{Bad Ischl@\textbf{Bad Ischl}, \emph{P.PPL}|pw}\pend{}{\bigskip}\vspace{1em}
\pstart
           \noindent{}{\pb}Lieber Thuri! Ich komme Samſtag mit dem Zuge, der
                  9 Uhr 40 von \textsc{Aussee}\oindex{Bad Aussee@\textbf{Bad Aussee}, \emph{P.PPLA3}|pw} geht. Da{\geminationn}{ }ſchaue ich ins \textsc{Café Walter}\oindex{Cafe Walther@\textbf{Café Walther}, \emph{Kaffeehaus (K.KAF)}|pw} u. ſuche zunächſt einen \textsc{Masseur} oder \textsc{Masseuse}, da ich wahnſinnige rheumat. Kreuzſchmerzen habe.
               Dann bleibe ich bei euch bis 6 Uhr Abds.\pend
           
\pstart
           Herzlichst Dein{\\[\baselineskip]}\spacefill\mbox{Hermann}\pend
           \leftskip=0em{}\selectlanguage{ngerman}\endnumbering\briefempfaengerindex{Schnitzler, Arthur@\textsc{Schnitzler, Arthur}!zzzBahr, Hermann@\emph{von Hermann Bahr}!1894-07-201@{20. 7. 1894}|)be}\mylabel{L00357h}  \normalsize

\doendnotes{C}
\bigskip
\vfill

\clearpage

\footnotesize

\lohead{\textsc{register}}

% Definiere theindex-Environment komplett neu ohne reledmac
\makeatletter
\renewenvironment{theindex}{%
  \section*{\indexname}%
  \setlength{\parindent}{0pt}%
  \setlength{\parskip}{0pt plus 0.3pt}%
  \let\item\@idxitem
}{%
  \clearpage
}
\makeatother

\IfFileExists{\jobname-pw.ind}{\input{\jobname-pw.ind}}{}

\end{document}

      