%% latex-leseansicht-vorspann.tex
%% Vorspann für die Leseansicht.
%% Lädt die gemeinsame Datei latex-vorspann.tex mit nicht gesetztem Schalter.

\newif\ifkorrekturansicht
\korrekturansichtfalse

\input{../tex-inputs/latex-vorspann}


\section[Arthur Schnitzler an Gustav Schwarzkopf, {[}13.–14. 3. 1897?{]}]{L04194 Arthur Schnitzler an Gustav Schwarzkopf, {[}13.–14. 3. 1897?{]}}
\nopagebreak\mylabel{L04194v}
\rehead{ }\normalsize\beginnumbering\briefempfaengerindex{Schwarzkopf, Gustav@\textsc{Schwarzkopf, Gustav}!zzzSchnitzler, Arthur@\emph{von Arthur Schnitzler}!1897-03-141@{{[}13.–14. 3. 1897?{]}}|(be}
\toendnotes[C]{\smallbreak\pagebreak[2]}
\correspDesc{Versand  durch Arthur Schnitzler im Zeitraum [13.–14. 3. 1897?] in Wien
\newline{}Erhalt  durch Gustav Schwarzkopf im Zeitraum [13.–14. 3. 1897?] in Wien}\toendnotes[C]{\smallbreak}
\Standort{CUL, Schnitzler, B 96.}
\physDesc{Briefkarte, 194 Zeichen
\newline{}Handschrift: Bleistift, deutsche Kurrent}\toendnotes[C]{\smallbreak}
\pstart
           \noindent{}{\pb}Lieber Guſtav,{ }dieſen
                  \label{K_L04194-1v}\edtext{Mittwoch den 17.\eventindex{Frankgasse 1@\textbf{Frankgasse 1}!Private Lesung von Reigen, 17.3.1897@Private Lesung von Reigen, 17.3.1897|pwv}}{\lemma{\textnormal{\emph{Mittwoch den 17.}}}\Cendnote{\textnormal{Der Brief ist undatiert, aber bezieht sich durch die Datierung einer privaten
                     Lesung auf den 17., bei der Georg Hirschfeld\pwindex{Hirschfeld, Georg 11.\,2.\,1873 Berlin – 17.\,1.\,1942 München@\textsc{Hirschfeld, Georg} (11.\,2.\,1873 Berlin – 17.\,1.\,1942 München), \emph{Schriftsteller}|pwk} anwesend war,
                     auf die erste Lesung von \emph{Reigen}\pwindex{Schnitzler, Arthur 15. 5. 1862 Wien – 21. 10. 1931 ebd.@\textsc{Schnitzler, Arthur} (15. 5. 1862 Wien – 21. 10. 1931 ebd.), \emph{Schriftsteller, Mediziner}!Reigen. Zehn Dialoge@\strich\emph{Reigen. Zehn Dialoge}|pwk} am 17. 3. 1897. 
                     Da Schnitzler schon am XXXX Auszeichnungsfehler: Dokument L00651 nicht gefunden 
                  die Zusage von Schwarzkopf\pwindex{Schwarzkopf, Gustav 7.\,11.\,1853 Wien – 13.\,11.\,1939 ebd.@\textsc{Schwarzkopf, Gustav} (7.\,11.\,1853 Wien – 13.\,11.\,1939 ebd.), \emph{Schriftsteller}|pwk} besaß, muss das vorliegende Schreiben an diesem Tag oder
                     am Vortag verfasst sein, da am XXXX Auszeichnungsfehler: Dokument L00649 nicht gefunden noch keine
                     Rede von der privaten Lesung\eventindex{Frankgasse 1@\textbf{Frankgasse 1}!Private Lesung von Reigen, 17.3.1897@Private Lesung von Reigen, 17.3.1897|pwkv} ist.}}}\label{K_L04194-1}
               findet einer der \label{K_L04194-2v}\edtext{ſ. Z.}{\lemma{\textnormal{\emph{s. Z.}}}\Cendnote{\textnormal{seiner Zeit}}}\label{K_L04194-2} ſo beliebten Dinſtage
               bei mir ſtatt. Alſo bitte kommen Sie, Muſik und \label{K_L04194-3v}\edtext{Hirſchfeld\pwindex{Hirschfeld, Georg 11.\,2.\,1873 Berlin – 17.\,1.\,1942 München@\textsc{Hirschfeld, Georg} (11.\,2.\,1873 Berlin – 17.\,1.\,1942 München), \emph{Schriftsteller}|pw} der Unbe\textsc{greif}liche}{\lemma{\textnormal{\emph{Hirschfeld der Unbegreifliche}}}\Cendnote{\textnormal{Anspielung unklar}}}\label{K_L04194-3} garantirt. ½ 10.-\pend
           \pstart Herzlich Ihr\spacefill\mbox{Arth Schn}\pend{}\selectlanguage{ngerman}\endnumbering\briefempfaengerindex{Schwarzkopf, Gustav@\textsc{Schwarzkopf, Gustav}!zzzSchnitzler, Arthur@\emph{von Arthur Schnitzler}!1897-03-131@{{[}13.–14. 3. 1897?{]}}|)be}\mylabel{L04194h}
\begin{anhang}
\end{anhang}\newcommand{\dateiname}{L04194}\newcommand{\titel}{Arthur Schnitzler an Gustav Schwarzkopf, [13.–14. 3. 1897?]}\newcommand{\editorInnen}{Herausgegeben von Jahnke, SelmaMüller, Martin Anton}%% latex-leseansicht-abspann.tex
%% Abspann für die Leseansicht.
%% Der Schalter \ifkorrekturansicht ist bereits durch den Vorspann gesetzt.

%% latex-abspann.tex
%% Gemeinsamer Abspann für Korrekturansicht und Leseansicht.
%% Setzt den Schalter \ifkorrekturansicht voraus (gesetzt in den
%% einbindenden Dateien latex-korrekturansicht-abspann.tex bzw.
%% latex-leseansicht-abspann.tex).
%% ---------------------------------------------------------------

\normalsize

% Das esempio-Environment wird nur in der Leseansicht benötigt
\ifkorrekturansicht\else
\newenvironment{esempio}[3]%
{
    \vspace{1.5ex}
    \rlap{\underline{#1}}
    \par
    \setlength{\parindent}{0cm}
    \nopagebreak
    \leftskip=#2cm
    \rightskip=#3cm
}
{
    \par
}
\fi

\doendnotes{C}
\bigskip
\vfill

\clearpage

\footnotesize

\ifkorrekturansicht
  \lohead{\textsc{register}}
\fi

% theindex-Environment neu definieren ohne reledmac
\makeatletter
\renewenvironment{theindex}{%
  \ifkorrekturansicht
    \section*{\indexname}%
  \else
    \subsubsection*{Index der erwähnten Entitäten}%
  \fi
  \setlength{\parindent}{0pt}%
  \setlength{\parskip}{0pt plus 0.3pt}%
  \let\item\@idxitem
}{%
  \ifkorrekturansicht\clearpage\fi
}
\makeatother

\IfFileExists{\jobname-pw.ind}{\input{\jobname-pw.ind}}{}

% Quellenangabe nur in der Leseansicht
\ifkorrekturansicht\else
% Fallback-Definitionen, falls die .tex-Datei \titel etc. nicht gesetzt hat
\providecommand{\titel}{}
\providecommand{\editorInnen}{}
\providecommand{\dateiname}{\jobname}

\vspace{3cm}

\vfill

\footnotesize
\textsc{Quelle}: \titel. Herausgegeben von {\editorInnen}. In: \emph{Arthur Schnitzler: Briefwechsel mit Autorinnen und Autoren}.
 Digitale Edition, https://schnitzler-briefe.acdh.oeaw.ac.at/{\dateiname}.html (Stand \today)
\fi

\end{document}


