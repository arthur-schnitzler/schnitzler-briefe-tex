%% latex-leseansicht-vorspann.tex
%% Vorspann für die Leseansicht.
%% Lädt die gemeinsame Datei latex-vorspann.tex mit nicht gesetztem Schalter.

\newif\ifkorrekturansicht
\korrekturansichtfalse

\input{../tex-inputs/latex-vorspann}


               \section[Arthur Schnitzler an Richard Beer-Hofmann, 31. 7. 1899]{ Arthur Schnitzler an Richard Beer-Hofmann, 31. 7. 1899}\nopagebreak\mylabel{v}\rehead{ }\begin{ledgroupsized}[t]{13cm}\normalsize\beginnumbering\briefempfaengerindex{Beer-Hofmann, Richard@\textsc{Beer-Hofmann, Richard}!zzzSchnitzler, Arthur@\emph{von Arthur Schnitzler}!1899-07-311@{31. 7. 1899}|(be} \toendnotes[C]{\smallbreak\pagebreak[2]} \Standort{YCGL, MSS 31.}
\physDesc{Postkarte
\newline{}Handschrift: Bleistift, deutsche Kurrent\newline{}Versand: 1) Stempel: »\nobreak{}\oindex{Spittal an der Drau@\textbf{Spittal an der Drau}|pwk}Spittal an der Drau, 31/7 {[}1899{]}\nobreak{}«.  2) Stempel: »\nobreak{}\oindex{Seeboden@\textbf{Seeboden}|pwk}{[}Seebod{]}en, 31. 7. \textcolor{gray}{9}9\nobreak{}«. \newline{}Ordnung: mit Bleistift von unbekannter Hand
                                 datiert: »31. 7.« }\pstart{}{\pb}\textsc{Dr Richard Beer Hofmann}\pend{}\pstart{}\textsc{Villa} Platzer\oindex{Villa Platzer@\textbf{Villa Platzer}|pw}\pend{}\pstart{}\textsc{Seeboden}\oindex{Seeboden@\textbf{Seeboden}|pw}\pend{}\pstart{}\textsc{am Millstätter}ſee\oindex{Millstaetter See@\textbf{Millstätter See}|pw}\pend{}{\bigskip}\pstart
           \noindent{}{\pb}lieber; es iſt abſolut unſinnnig, am 1. Tag ſich ſo raſend zu
               ſtrapaziren, und beſonders we{\geminationn} der 2. Tag die
               ſchwierigſte Partie (Giau\oindex{Passo di Giau@\textbf{Passo di Giau}|pw}) enthält und \strikeout{die} wir doch nur möglichſt arbeitsfriſch betreten
               wollen. Wir werden daher die Tour I in 2 Tage zerlegen, dafür am 1. Tag den Pragſer See\oindex{Pragser Wildsee@\textbf{Pragser Wildsee}|pw} mitnehmen. Da{\geminationn} bleibt es auch gewahrt dſs alle Nachmittag frei
               ſind. – Ich ſchreibe Ihnen das gleich hier, um nicht nervös zu ſein. –\pend
           \pstart Herzliche Grüße Ihr \spacefill\mbox{A. S.}\pend{}\pstart
           Spital\oindex{Spittal an der Drau@\textbf{Spittal an der Drau}|pw}, 31. 7. 99, eben ſchlägt’s
                     7 Uhr früh.\pend
                     \endnumbering\briefempfaengerindex{Beer-Hofmann, Richard@\textsc{Beer-Hofmann, Richard}!zzzSchnitzler, Arthur@\emph{von Arthur Schnitzler}!1899-07-311@{31. 7. 1899}|)be}\mylabel{h}\end{ledgroupsized}  \newcommand{\dateiname}{L00953}\newcommand{\titel}{Arthur Schnitzler an Richard Beer-Hofmann, 31. 7. 1899}\newcommand{\editorInnen}{Martin Anton Müller und Gerd-Hermann Susen}%% latex-leseansicht-abspann.tex
%% Abspann für die Leseansicht.
%% Der Schalter \ifkorrekturansicht ist bereits durch den Vorspann gesetzt.

%% latex-abspann.tex
%% Gemeinsamer Abspann für Korrekturansicht und Leseansicht.
%% Setzt den Schalter \ifkorrekturansicht voraus (gesetzt in den
%% einbindenden Dateien latex-korrekturansicht-abspann.tex bzw.
%% latex-leseansicht-abspann.tex).
%% ---------------------------------------------------------------

\normalsize

% Das esempio-Environment wird nur in der Leseansicht benötigt
\ifkorrekturansicht\else
\newenvironment{esempio}[3]%
{
    \vspace{1.5ex}
    \rlap{\underline{#1}}
    \par
    \setlength{\parindent}{0cm}
    \nopagebreak
    \leftskip=#2cm
    \rightskip=#3cm
}
{
    \par
}
\fi

\doendnotes{C}
\bigskip
\vfill

\clearpage

\footnotesize

\ifkorrekturansicht
  \lohead{\textsc{register}}
\fi

% theindex-Environment neu definieren ohne reledmac
\makeatletter
\renewenvironment{theindex}{%
  \ifkorrekturansicht
    \section*{\indexname}%
  \else
    \subsubsection*{Index der erwähnten Entitäten}%
  \fi
  \setlength{\parindent}{0pt}%
  \setlength{\parskip}{0pt plus 0.3pt}%
  \let\item\@idxitem
}{%
  \ifkorrekturansicht\clearpage\fi
}
\makeatother

\IfFileExists{\jobname-pw.ind}{\input{\jobname-pw.ind}}{}

% Quellenangabe nur in der Leseansicht
\ifkorrekturansicht\else
% Fallback-Definitionen, falls die .tex-Datei \titel etc. nicht gesetzt hat
\providecommand{\titel}{}
\providecommand{\editorInnen}{}
\providecommand{\dateiname}{\jobname}

\vspace{3cm}

\vfill

\footnotesize
\textsc{Quelle}: \titel. Herausgegeben von {\editorInnen}. In: \emph{Arthur Schnitzler: Briefwechsel mit Autorinnen und Autoren}.
 Digitale Edition, https://schnitzler-briefe.acdh.oeaw.ac.at/{\dateiname}.html (Stand \today)
\fi

\end{document}


      