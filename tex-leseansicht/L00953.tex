%% latex-korrekturansicht-vorspann.tex
%% Vorspann für die Korrekturansicht.
%% Lädt die gemeinsame Datei latex-vorspann.tex mit gesetztem Schalter.

\newif\ifkorrekturansicht
\korrekturansichttrue

\input{../tex-inputs/latex-vorspann}


\section[Arthur Schnitzler an Richard Beer-Hofmann, 31. 7. 1899]{L00953 Arthur Schnitzler an Richard Beer-Hofmann, 31. 7. 1899}
\nopagebreak\mylabel{L00953v}
\rehead{ }\normalsize\beginnumbering\briefempfaengerindex{Beer-Hofmann, Richard@\textsc{Beer-Hofmann, Richard}!zzzSchnitzler, Arthur@\emph{von Arthur Schnitzler}!1899-07-311@{31. 7. 1899}|(be}
\toendnotes[C]{\smallbreak\pagebreak[2]}\Standort{YCGL, MSS 31.}
\physDesc{Postkarte, 538 Zeichen
\newline{}Handschrift: Bleistift, deutsche Kurrent
\newline{}Versand: 1) Stempel: »\nobreak{}\oindex{Spittal an der Drau@\textbf{Spittal an der Drau}, \emph{P.PPLA3}|pwk}Spittal an der Drau, 31/7 {[}1899{]}\nobreak{}«.   2) Stempel: »\nobreak{}\oindex{Seeboden@\textbf{Seeboden}, \emph{A.ADM3}|pwk}{[}Seebod{]}en, 31. 7. \textcolor{gray}{9}9\nobreak{}«. 
\newline{}Ordnung: mit Bleistift von unbekannter Hand datiert: »31. 7.« }\pstart{}{\pb}\textsc{Dr Richard Beer Hofmann}\pend{}\pstart{}\textsc{Villa} Platzer\oindex{Villa Platzer@\textbf{Villa Platzer}, \emph{Gebäude (K.GBD)}|pw}\pend{}\pstart{}\textsc{Seeboden}\oindex{Seeboden@\textbf{Seeboden}, \emph{A.ADM3}|pw}\pend{}\pstart{}\textsc{am Millstätter}ſee\oindex{Millstaetter See@\textbf{Millstätter See}, \emph{See (N.SEE)}|pw}\pend{}{\bigskip}\vspace{1em}
\pstart
           \noindent{}{\pb}lieber; es iſt abſolut unſinnnig, am 1. Tag ſich ſo raſend zu
               ſtrapaziren, und beſonders we{\geminationn} der 2. Tag die
               ſchwierigſte Partie (Giau\oindex{Passo di Giau@\textbf{Passo di Giau}, \emph{Pass (N.PAS)}|pw}) enthält und \strikeout{die} wir doch nur möglichſt arbeitsfriſch betreten
               wollen. Wir werden daher die Tour I in 2 Tage zerlegen, dafür am 1. Tag den Pragſer See\oindex{Pragser Wildsee@\textbf{Pragser Wildsee}, \emph{See (N.SEE)}|pw} mitnehmen. Da{\geminationn} bleibt es auch gewahrt dſs alle Nachmittag frei
               ſind. – Ich ſchreibe Ihnen das gleich hier, um nicht nervös zu ſein. –\pend
           \pstart Herzliche Grüße Ihr \spacefill\mbox{A. S.}\pend{}
\pstart
           Spital\oindex{Spittal an der Drau@\textbf{Spittal an der Drau}, \emph{P.PPLA3}|pw}, 31. 7. 99, eben
                  ſchlägt’s 7 Uhr früh.\pend
           \selectlanguage{ngerman}\endnumbering\briefempfaengerindex{Beer-Hofmann, Richard@\textsc{Beer-Hofmann, Richard}!zzzSchnitzler, Arthur@\emph{von Arthur Schnitzler}!1899-07-311@{31. 7. 1899}|)be}\mylabel{L00953h}  \normalsize

\doendnotes{C}
\bigskip
\vfill

\clearpage

\footnotesize

\lohead{\textsc{register}}

% Definiere theindex-Environment komplett neu ohne reledmac
\makeatletter
\renewenvironment{theindex}{%
  \section*{\indexname}%
  \setlength{\parindent}{0pt}%
  \setlength{\parskip}{0pt plus 0.3pt}%
  \let\item\@idxitem
}{%
  \clearpage
}
\makeatother

\IfFileExists{\jobname-pw.ind}{\input{\jobname-pw.ind}}{}

\end{document}

      