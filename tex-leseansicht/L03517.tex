%% latex-leseansicht-vorspann.tex
%% Vorspann für die Leseansicht.
%% Lädt die gemeinsame Datei latex-vorspann.tex mit nicht gesetztem Schalter.

\newif\ifkorrekturansicht
\korrekturansichtfalse

\input{../tex-inputs/latex-vorspann}


\section[ Paul Goldmann an Arthur Schnitzler, 19. 5. 1931]{L03517 Paul Goldmann an Arthur Schnitzler,  19. 5. 1931}
\nopagebreak\mylabel{L03517v}
\rehead{ }\normalsize\beginnumbering\briefempfaengerindex{Schnitzler, Arthur@\textsc{Schnitzler, Arthur}!zzzGoldmann, Paul@\emph{von Paul Goldmann}!1931-05-191@{19. 5. 1931}|(be}
\toendnotes[C]{\smallbreak\pagebreak[2]}
\correspDesc{Versand  durch Paul Goldmann am 19. 5. 1931 in Berlin
\newline{}Erhalt  durch Arthur Schnitzler im Zeitraum [20. 5. 1931
                  – 24. 5. 1931?] in Wien}\toendnotes[C]{\smallbreak}
\Standort{DLA, A:Schnitzler, HS.NZ85.1.3176.}
\physDesc{Postkarte, 603 Zeichen
\newline{}Schreibmaschine
\newline{}Handschrift: lila Tinte, lateinische Kurrent (\noindent{}ein Komma und Unterschrift)
\newline{}Versand: 1) Stempel: »\nobreak{}Luftpost. Befördert Briefe – Zeitungen –
                                       Pakete\nobreak{}«.   2) Stempel: »\nobreak{}\oindex{Berlin@\textbf{Berlin}, \emph{Hauptstadt}|pwk}Berlin SW 11, 19. 5. 31, 14–15 \textcolor{gray}{N}\nobreak{}«. 
\newline{}Schnitzler: mit rotem Buntstift drei Unterstreichungen }\toendnotes[C]{\smallbreak}\pstart{}\textcolor{gray}{\textbf{\textit{{\pb}Dr. PAUL GOLDMANN}}}\pend{}\pstart{}\textcolor{gray}{\textbf{\textit{BENDLERSTR. 36\oindex{Stauffenbergstraße@\textbf{Stauffenbergstraße}, \emph{Straße}|pw}}}}\pend{}\pstart{}\textcolor{gray}{\textbf{\textit{BERLIN W.\oindex{Berlin@\textbf{Berlin}, \emph{Hauptstadt}|pw}}}}\pend{}{\bigskip}\pstart{}Herrn\pend{}\pstart{}Dr. Arthur Schnitzler\pend{}\pstart{}\so{Wien}\oindex{Wien@\textbf{Wien}, \emph{Verwaltungsgebiet}|pw}\pend{}\pstart{}\label{T_L03517-1v}\edtext{XVIII.}{\lemma{\textnormal{\emph{XVIII.}}}\Cendnote{\textnormal{korrigiert aus »XV111.«}}}\label{T_L03517-1}
                     Sternwartstrasse 71\oindex{Wien@\textbf{Wien}!XVIII., Währing@\textbf{XVIII., Währing}!Sternwartestraße 71@\textbf{Sternwartestraße 71}, \emph{Wohngebäude}|pw}\pend{}{\bigskip}\vspace{1em}
\pstart
           \centering{}{\pb}Berlin\oindex{Berlin@\textbf{Berlin}, \emph{Hauptstadt}|pw}, den 19. Mai 1931\pend
           
\pstart{}Lieber Freund,\pend\vspace{0.5em}
\pstart
           Ich danke Dir herzlichst für die so überraschend schnelle Übersendung der beiden
                  \label{K_L03517-1v}\edtext{Bücher\pwindex{?? [Roman mit erotischen Schilderungen]@\emph{?? [Roman mit erotischen Schilderungen]}|pwv}\pwindex{Schnitzler, Arthur 15.\,5.\,1862 Wien – 21.\,10.\,1931 ebd.@\textsc{Schnitzler, Arthur} (15.\,5.\,1862 Wien – 21.\,10.\,1931 ebd.), \emph{Schriftsteller, Mediziner}!Im Spiel der Sommerlüfte. In drei Aufzügen@\strich\emph{Im Spiel der Sommerlüfte. In drei Aufzügen}|pwv}}{\lemma{\textnormal{\emph{Bücher}}}\Cendnote{\textnormal{Es handelt sich um einen nicht zu
                  identifizierenden Roman\pwindex{?? [Roman mit erotischen Schilderungen]@\emph{?? [Roman mit erotischen Schilderungen]}|pwkv}
                  und ein Schauspiel von Schnitzler. Bei
                  Letzterem könnte es sich um den Dreiakter \emph{Im Spiel
                     der Sommerlüfte}\pwindex{Schnitzler, Arthur 15.\,5.\,1862 Wien – 21.\,10.\,1931 ebd.@\textsc{Schnitzler, Arthur} (15.\,5.\,1862 Wien – 21.\,10.\,1931 ebd.), \emph{Schriftsteller, Mediziner}!Im Spiel der Sommerlüfte. In drei Aufzügen@\strich\emph{Im Spiel der Sommerlüfte. In drei Aufzügen}|pwk} handeln, der bereits am 21. 12. 1929 bei \emph{S. Fischer}\orgindex{S. Fischer Verlag@S. Fischer Verlag|pwk} in Berlin\oindex{Berlin@\textbf{Berlin}, \emph{Hauptstadt}|pwk} erschienen war.}}}\label{K_L03517-1}. Den Roman, den
               ich zurücksenden muss, werde ich so rasch als möglich lesen. Immerhin könnten einige
               Wochen vergehen\introOben{},\introOben{} und ich bitte Dich, trotzdem ganz sicher
               zu sein, dass D\substVorne{}\textsuperscript{i}\substDazwischen{}u\substHinten{} Dein Buch\pwindex{?? [Roman mit erotischen Schilderungen]@\emph{?? [Roman mit erotischen Schilderungen]}|pwv}
               zurückbekommst. Für die Widmung in dem Exemplar Deines Schauspiels\pwindex{Schnitzler, Arthur 15.\,5.\,1862 Wien – 21.\,10.\,1931 ebd.@\textsc{Schnitzler, Arthur} (15.\,5.\,1862 Wien – 21.\,10.\,1931 ebd.), \emph{Schriftsteller, Mediziner}!Im Spiel der Sommerlüfte. In drei Aufzügen@\strich\emph{Im Spiel der Sommerlüfte. In drei Aufzügen}|pwuv} danke ich Dir
               noch ganz besonders. Ich wünsche Dir angenehme Tage auf dem \label{K_L03517-2v}\edtext{Semmering\oindex{Semmering@\textbf{Semmering}, \emph{Verwaltungsgebiet}|pw}}{\lemma{\textnormal{\emph{Semmering}}}\Cendnote{\textnormal{Schnitzler verbrachte die Tage um seinen
                  69. Geburtstag am Semmering\oindex{Semmering@\textbf{Semmering}, \emph{Verwaltungsgebiet}|pwk}. Hier hielt er sich vom 13. 5. 1931 bis zum
                     16. 5. 1931 auf.}}}\label{K_L03517-2} und verbleibe mit herzlichen Grüssen {\\}Dein {\\}{[}hs.:{]} \spacefill\mbox{Paul Goldmann.}\pend
           \selectlanguage{ngerman}\endnumbering\briefempfaengerindex{Schnitzler, Arthur@\textsc{Schnitzler, Arthur}!zzzGoldmann, Paul@\emph{von Paul Goldmann}!1931-05-191@{19. 5. 1931}|)be}\mylabel{L03517h}  \newcommand{\dateiname}{L03517}\newcommand{\titel}{Paul Goldmann an Arthur Schnitzler, 19. 5. 1931}\newcommand{\editorInnen}{Martin Anton Müller und Laura Untner}%% latex-leseansicht-abspann.tex
%% Abspann für die Leseansicht.
%% Der Schalter \ifkorrekturansicht ist bereits durch den Vorspann gesetzt.

%% latex-abspann.tex
%% Gemeinsamer Abspann für Korrekturansicht und Leseansicht.
%% Setzt den Schalter \ifkorrekturansicht voraus (gesetzt in den
%% einbindenden Dateien latex-korrekturansicht-abspann.tex bzw.
%% latex-leseansicht-abspann.tex).
%% ---------------------------------------------------------------

\normalsize

% Das esempio-Environment wird nur in der Leseansicht benötigt
\ifkorrekturansicht\else
\newenvironment{esempio}[3]%
{
    \vspace{1.5ex}
    \rlap{\underline{#1}}
    \par
    \setlength{\parindent}{0cm}
    \nopagebreak
    \leftskip=#2cm
    \rightskip=#3cm
}
{
    \par
}
\fi

\doendnotes{C}
\bigskip
\vfill

\clearpage

\footnotesize

\ifkorrekturansicht
  \lohead{\textsc{register}}
\fi

% theindex-Environment neu definieren ohne reledmac
\makeatletter
\renewenvironment{theindex}{%
  \ifkorrekturansicht
    \section*{\indexname}%
  \else
    \subsubsection*{Index der erwähnten Entitäten}%
  \fi
  \setlength{\parindent}{0pt}%
  \setlength{\parskip}{0pt plus 0.3pt}%
  \let\item\@idxitem
}{%
  \ifkorrekturansicht\clearpage\fi
}
\makeatother

\IfFileExists{\jobname-pw.ind}{\input{\jobname-pw.ind}}{}

% Quellenangabe nur in der Leseansicht
\ifkorrekturansicht\else
% Fallback-Definitionen, falls die .tex-Datei \titel etc. nicht gesetzt hat
\providecommand{\titel}{}
\providecommand{\editorInnen}{}
\providecommand{\dateiname}{\jobname}

\vspace{3cm}

\vfill

\footnotesize
\textsc{Quelle}: \titel. Herausgegeben von {\editorInnen}. In: \emph{Arthur Schnitzler: Briefwechsel mit Autorinnen und Autoren}.
 Digitale Edition, https://schnitzler-briefe.acdh.oeaw.ac.at/{\dateiname}.html (Stand \today)
\fi

\end{document}


