%% latex-leseansicht-vorspann.tex
%% Vorspann für die Leseansicht.
%% Lädt die gemeinsame Datei latex-vorspann.tex mit nicht gesetztem Schalter.

\newif\ifkorrekturansicht
\korrekturansichtfalse

\input{../tex-inputs/latex-vorspann}

\begin{center}
            \textcolor{red}{ENTWURF, NICHT FERTIG KORRIGIERT}
                      \end{center}
            
         \renewcommand{\erwaehnteInstitutionen}{Institutionen: S. Fischer Verlag}
         \renewcommand{\erwaehnteOrte}{Orte: Bendlerstraße, Berlin, Semmering, Sternwartestraße, Wien}
         \renewcommand{\erwaehnteWerke}{Werke: Im Spiel der Sommerlüfte. In drei Aufzügen}
               \section[ Paul Goldmann an Arthur Schnitzler, 19. 5. 1931]{ Paul Goldmann an Arthur Schnitzler, 19. 5. 1931}\nopagebreak\mylabel{v}\rehead{ }\begin{ledgroupsized}[t]{13cm}\normalsize\beginnumbering \toendnotes[C]{\smallbreak\pagebreak[2]} \Standort{DLA, A:Schnitzler, HS.NZ85.1.3176.}
\physDesc{Postkarte, 606 Zeichen
\newline{}Schreibmaschine
\newline{}Handschrift: blaue Tinte, lateinische Kurrent (\noindent{}ein Komma und Unterschrift)
\newline{}Versand: Stempel: »\nobreak{}\oindex{Berlin@\textbf{Berlin}|pwk}Berlin SW 11, 19. 5. 31, 14—15\nobreak{}«.  
\newline{}Schnitzler: mit rotem Buntstift drei Unterstreichungen }\toendnotes[C]{\smallbreak}\pstart{}\textcolor{gray}{\textbf{\textit{{\pb}Dr. PAUL GOLDMANN}}}\pend{}\pstart{}\textcolor{gray}{\textbf{\textit{BENDLERSTR. 36\oindex{Bendlerstrasse@\textbf{Bendlerstraße}|pw}}}}\pend{}\pstart{}\textcolor{gray}{\textbf{\textit{Berlin W.\oindex{Berlin@\textbf{Berlin}|pw}}}}\pend{}{\bigskip}\pstart{}Herrn\pend{}\pstart{}Dr. Arthur Schnitzler\pend{}\pstart{}Wien\oindex{Wien@\textbf{Wien}|pw}\pend{}\pstart{}XVlll. Sternwartstrasse 71\oindex{Sternwartestrasse@\textbf{Sternwartestraße}|pw}\pend{}{\bigskip}\pstart
           \centering{}{\pb}Berlin\oindex{Berlin@\textbf{Berlin}|pw}, den 19. Mai 1931\pend
           \pstart\center{}Lieber Freund,\pend\pstart
           Ich danke Dir herzlichst für die so überraschend schnelle Übersendung der beiden
                  \label{K_L03517-1v}\edtext{Bücher\pwindex{Schnitzler, Arthur 15.05.1862 – 21.10.1931@\textsc{Schnitzler, Arthur} (15.05.1862 – 21.10.1931), \emph{Schriftsteller, Mediziner}!Im Spiel der Sommerluefte. In drei Aufzuegen1929-12-21@\strich\emph{Im Spiel der Sommerlüfte. In drei Aufzügen} {[}1929-12-21{]}|pwuv}}{\lemma{\textnormal{\emph{Bücher}}}\Cendnote{\textnormal{Das erwähnte Schauspiel war womöglich
                  der schon am 21. 12. 1929 bei \emph{S. Fischer}\orgindex{S. Fischer Verlag@S. Fischer Verlag|pwk} in Berlin\oindex{Berlin@\textbf{Berlin}|pwk}
                  erschienene Dreiakter \emph{Im Spiel der
                     Sommerlüfte}\pwindex{Schnitzler, Arthur 15.05.1862 – 21.10.1931@\textsc{Schnitzler, Arthur} (15.05.1862 – 21.10.1931), \emph{Schriftsteller, Mediziner}!Im Spiel der Sommerluefte. In drei Aufzuegen1929-12-21@\strich\emph{Im Spiel der Sommerlüfte. In drei Aufzügen} {[}1929-12-21{]}|pwk}. Der Roman konnte nicht ermittelt werden.}}}\label{K_L03517-1h}. Den Roman,
               den ich zurücksenden muss, werde ich so rasch als möglich lesen. Immerhin könnten
               einige Wochen vergehen\introOben{},\introOben{} und ich bitte Dich, trotzdem ganz
               sicher zu sein, dass D\substVorne{}\textsuperscript{i}\substDazwischen{}u\substHinten{} Dein Buch zurückbekommst. Für die Widmung in dem Exemplar Deines Schauspiel\pwindex{Schnitzler, Arthur 15.05.1862 – 21.10.1931@\textsc{Schnitzler, Arthur} (15.05.1862 – 21.10.1931), \emph{Schriftsteller, Mediziner}!Im Spiel der Sommerluefte. In drei Aufzuegen1929-12-21@\strich\emph{Im Spiel der Sommerlüfte. In drei Aufzügen} {[}1929-12-21{]}|pwuv}s danke
               ich Dir noch ganz besonders. Ich wünsche Dir angenehme Tage auf dem \label{K_L03517-2v}\edtext{Semmering\oindex{Semmering@\textbf{Semmering}|pw}}{\lemma{\textnormal{\emph{Semmering}}}\Cendnote{\textnormal{Schnitzler\pwindex{Schnitzler, Arthur 15.05.1862 – 21.10.1931@\textsc{Schnitzler, Arthur} (15.05.1862 – 21.10.1931), \emph{Schriftsteller, Mediziner}|pwk} war erst im Juli auf dem Semmering\oindex{Semmering@\textbf{Semmering}|pwk} (16. 7. 1931–28. 7. 1931).}}}\label{K_L03517-2h}
               und verbleibe mit herzlichen Grüssen {\\}Dein {\\}{[}hs. Goldmann:{]} \spacefill\mbox{Paul Goldmann.}\pend
           
         
         \endnumbering\mylabel{h}\end{ledgroupsized}\begin{anhang}\end{anhang}\newcommand{\dateiname}{L03517}\newcommand{\titel}{Paul Goldmann an Arthur Schnitzler, 19. 5. 1931}\newcommand{\editorInnen}{Martin Anton Müller und Laura Untner}%% latex-leseansicht-abspann.tex
%% Abspann für die Leseansicht.
%% Der Schalter \ifkorrekturansicht ist bereits durch den Vorspann gesetzt.

%% latex-abspann.tex
%% Gemeinsamer Abspann für Korrekturansicht und Leseansicht.
%% Setzt den Schalter \ifkorrekturansicht voraus (gesetzt in den
%% einbindenden Dateien latex-korrekturansicht-abspann.tex bzw.
%% latex-leseansicht-abspann.tex).
%% ---------------------------------------------------------------

\normalsize

% Das esempio-Environment wird nur in der Leseansicht benötigt
\ifkorrekturansicht\else
\newenvironment{esempio}[3]%
{
    \vspace{1.5ex}
    \rlap{\underline{#1}}
    \par
    \setlength{\parindent}{0cm}
    \nopagebreak
    \leftskip=#2cm
    \rightskip=#3cm
}
{
    \par
}
\fi

\doendnotes{C}
\bigskip
\vfill

\clearpage

\footnotesize

\ifkorrekturansicht
  \lohead{\textsc{register}}
\fi

% theindex-Environment neu definieren ohne reledmac
\makeatletter
\renewenvironment{theindex}{%
  \ifkorrekturansicht
    \section*{\indexname}%
  \else
    \subsubsection*{Index der erwähnten Entitäten}%
  \fi
  \setlength{\parindent}{0pt}%
  \setlength{\parskip}{0pt plus 0.3pt}%
  \let\item\@idxitem
}{%
  \ifkorrekturansicht\clearpage\fi
}
\makeatother

\IfFileExists{\jobname-pw.ind}{\input{\jobname-pw.ind}}{}

% Quellenangabe nur in der Leseansicht
\ifkorrekturansicht\else
% Fallback-Definitionen, falls die .tex-Datei \titel etc. nicht gesetzt hat
\providecommand{\titel}{}
\providecommand{\editorInnen}{}
\providecommand{\dateiname}{\jobname}

\vspace{3cm}

\vfill

\footnotesize
\textsc{Quelle}: \titel. Herausgegeben von {\editorInnen}. In: \emph{Arthur Schnitzler: Briefwechsel mit Autorinnen und Autoren}.
 Digitale Edition, https://schnitzler-briefe.acdh.oeaw.ac.at/{\dateiname}.html (Stand \today)
\fi

\end{document}


      