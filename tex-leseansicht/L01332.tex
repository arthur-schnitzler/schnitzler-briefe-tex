%% latex-korrekturansicht-vorspann.tex
%% Vorspann für die Korrekturansicht.
%% Lädt die gemeinsame Datei latex-vorspann.tex mit gesetztem Schalter.

\newif\ifkorrekturansicht
\korrekturansichttrue

\input{../tex-inputs/latex-vorspann}


\section[Hermann Bahr: Widmungsexemplar Dialog vom Tragischen für Arthur Schnitzler, 31. 10. 1903]{L01332 Hermann Bahr: Widmungsexemplar Dialog vom Tragischen für Arthur
               Schnitzler, 31. 10. 1903}
\nopagebreak\mylabel{L01332v}
\rehead{ }\normalsize\beginnumbering\briefempfaengerindex{Schnitzler, Arthur@\textsc{Schnitzler, Arthur}!zzzBahr, Hermann@\emph{von Hermann Bahr}!1903-10-311@{31. 10. 1903}|(be}
\toendnotes[C]{\smallbreak\pagebreak[2]}\Standort{DLA, G:Schnitzler, Arthur (Sammlung Heinrich Schnitzler).}
\physDesc{Widmung am Vorsatzblatt, 46 Zeichen
\newline{}Handschrift: schwarze Tinte, deutsche Kurrent
\newline{}Ordnung: bei der Enteignung des Exemplars 1938 von
                                 unbekannter Hand mit Bleistift ergänzte Informationen:
                                    »Dublette zu 430.288-A« }
\buchAbdrucke{\weitereDrucke{Hermann Bahr, Arthur Schnitzler: \emph{Briefwechsel, Aufzeichnungen, Dokumente (1891–1931)}. Göttingen: \emph{Wallstein} 2018, S. 273.} }
\pstart
           \noindent{}{\pb}Seinem lieben Arthur\pend
           
\pstart
           Herzlichſt{\\[\baselineskip]}\spacefill\mbox{B.}\pend
           \leftskip=0em{}
\pstart
           \noindent{}31. 10. 1903\pend
           \selectlanguage{ngerman}\vspace{1em}{\vspace{1\baselineskip}}
\pstart
           \centering{}{\pb}\textcolor{gray}{\textbf{Dialog{\\}vom Tragischen\pwindex{Dialog vom Tragischen@\emph{Dialog vom Tragischen}|pw}}}\pend
           
\pstart
           \centering{}\textcolor{gray}{\textbf{von}}\pend
           
\pstart
           \centering{}\textcolor{gray}{\textbf{Hermann Bahr}}\pend
           {\vspace{1\baselineskip}}
\pstart
           \centering{}\textcolor{gray}{\textbf{Berlin\oindex{Berlin@\textbf{Berlin}, \emph{P.PPLC}|pw}}}\pend
           
\pstart
           \centering{}\textcolor{gray}{\textbf{\so{S. Fischer}, Verlag\orgindex{S. Fischer Verlag@S. Fischer Verlag|pw}}}\pend
           
\pstart
           \centering{}\textcolor{gray}{\textbf{1904}}\pend
           \selectlanguage{ngerman}\endnumbering\briefempfaengerindex{Schnitzler, Arthur@\textsc{Schnitzler, Arthur}!zzzBahr, Hermann@\emph{von Hermann Bahr}!1903-10-311@{31. 10. 1903}|)be}\mylabel{L01332h}  \normalsize

\doendnotes{C}
\bigskip
\vfill

\clearpage

\footnotesize

\lohead{\textsc{register}}

% Definiere theindex-Environment komplett neu ohne reledmac
\makeatletter
\renewenvironment{theindex}{%
  \section*{\indexname}%
  \setlength{\parindent}{0pt}%
  \setlength{\parskip}{0pt plus 0.3pt}%
  \let\item\@idxitem
}{%
  \clearpage
}
\makeatother

\IfFileExists{\jobname-pw.ind}{\input{\jobname-pw.ind}}{}

\end{document}

      