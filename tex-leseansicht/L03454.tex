%% latex-leseansicht-vorspann.tex
%% Vorspann für die Leseansicht.
%% Lädt die gemeinsame Datei latex-vorspann.tex mit nicht gesetztem Schalter.

\newif\ifkorrekturansicht
\korrekturansichtfalse

\input{../tex-inputs/latex-vorspann}


\section[ Paul Goldmann an Arthur Schnitzler, 31. 8. 1904]{L03454 Paul Goldmann an Arthur Schnitzler,  31. 8. 1904}
\nopagebreak\mylabel{L03454v}
\rehead{ }\normalsize\beginnumbering\briefempfaengerindex{Schnitzler, Arthur@\textsc{Schnitzler, Arthur}!zzzGoldmann, Paul@\emph{von Paul Goldmann}!1904-08-311@{31. 8. 1904}|(be}
\toendnotes[C]{\smallbreak\pagebreak[2]}
\correspDesc{Versand  durch Paul Goldmann am 31. 8. 1904 in Madonna di Campiglio
\newline{}Erhalt  durch Arthur Schnitzler im Zeitraum [1. 9. 1904
                  – 5. 9. 1904?] in Wien}\toendnotes[C]{\smallbreak}
\Standort{DLA, A:Schnitzler, HS.NZ85.1.3174.}
\physDesc{Brief, 1 Blatt, 2 Seiten, 632 Zeichen
\newline{}Handschrift: schwarze Tinte, deutsche Kurrent
\newline{}Schnitzler: mit rotem Buntstift eine Unterstreichung }\toendnotes[C]{\smallbreak}
\pstart
           \raggedleft{}{\pb}\textcolor{gray}{\textbf{BRENTA-DOLOMITEN\oindex{Brenta (Gebirge)@\textbf{Brenta (Gebirge)}, \emph{Gebirge}|pw} (CIMA TOSA\oindex{Cima Tosa@\textbf{Cima Tosa}, \emph{Bergspitze}|pw} 3176 M) VOM M. SPINALE\oindex{Monte Spinale@\textbf{Monte Spinale}, \emph{Gebirge}|pw} 2021 M.}}\pend
           
\pstart
           \raggedleft{}\textcolor{gray}{\textbf{MADONNA DI CAMPIGLIO\oindex{Madonna di Campiglio@\textbf{Madonna di Campiglio}|pw} 1553 MTR.}}\pend
           
\pstart
           \textcolor{gray}{\textbf{Imperial Hôtel Trento\oindex{Imperial Hotel Trento@\textbf{Imperial Hotel Trento}, \emph{Hotel}|pw}}}\pend
           
\pstart
           \textcolor{gray}{\textbf{\begin{otherlanguage}{french}vis-à-vis de la Gare\oindex{Bahnhof Trient@\textbf{Bahnhof Trient}, \emph{Bahnhof}|pw}\end{otherlanguage}.}}\hfill \textcolor{gray}{\textbf{Saison Juni–September.}}\pend
           
\pstart
           \textcolor{gray}{\textbf{\textsc{Trient\oindex{Trient@\textbf{Trient}|pw} (Südtirol\oindex{Südtirol@\textbf{Südtirol}, \emph{Verwaltungsgebiet}|pw})}}}\pend
           
\pstart
           \textcolor{gray}{\textbf{MÊME MAISON:}}{\\}\textcolor{gray}{\textbf{\textsc{Grand Hôtel des Alpes\oindex{Grand Hotel des Alpes@\textbf{Grand Hotel des Alpes}, \emph{Hotel}|pw}}}}{\\}\textcolor{gray}{\textbf{\textsc{Madonna di Campiglio\oindex{Madonna di Campiglio@\textbf{Madonna di Campiglio}|pw}.}}}\pend
           
\pstart
           \textcolor{gray}{\textbf{\emph{F. J. Oesterreicher\pwindex{Österreicher, Franz Joseph 4.\,8.\,1846 Brand – 22.\,8.\,1909@\textsc{Österreicher, Franz Joseph} (4.\,8.\,1846 Brand – 22.\,8.\,1909), \emph{Hotelbesitzer}|pw}, Prop\textsuperscript{re}}}}\pend
           
\pstart
           \raggedleft{}\textcolor{gray}{\textbf{\textbf{Madonna di Campiglio\oindex{Madonna di Campiglio@\textbf{Madonna di Campiglio}|pw}},}}{\\}31. Auguſt \textcolor{gray}{\textbf{190}}4.\pend
           
\pstart\center{}Mein lieber Freund,\pend\vspace{0.5em}
\pstart
           Es regnet in \textsc{Campiglio\oindex{Madonna di Campiglio@\textbf{Madonna di Campiglio}|pw}}, und ich fahre morgen nach \textsc{Riva\oindex{Riva del Garda@\textbf{Riva del Garda}, \emph{Hauptstadt}|pw}}. Von da wahrſcheinlich weiter nach Oberitalien\oindex{Italien@\textbf{Italien}|pw}. Das Nähere hängt von meinem Onkel\pwindex{Mamroth, Fedor 21.\,2.\,1851 Breslau – 25.\,6.\,1907 Frankfurt am Main@\textsc{Mamroth, Fedor} (21.\,2.\,1851 Breslau – 25.\,6.\,1907 Frankfurt am Main), \emph{Journalist, Kritiker}|pwv} ab, der aus dem Engadin\oindex{Engadin@\textbf{Engadin}, \emph{Tal}|pw} an einen von ihm noch zu beſtimmenden Ort kommt. Im Gebirge\oindex{Dolomiten@\textbf{Dolomiten}, \emph{Gebirge}|pwv} werde ich Dich alſo wohl nicht{ }ſehen können, – rathe Dir auch {\pb}dringend ab, nach
                  Tirol\oindex{Tirol@\textbf{Tirol}, \emph{Land}|pw}\oindex{Südtirol@\textbf{Südtirol}, \emph{Verwaltungsgebiet}|pw} zu kommen, ehe das Wetter{ }ſich
               gebeſſert hat (wozu anſcheinend wenig Ausſicht.) Aber wenn ich über Wien\oindex{Wien@\textbf{Wien}, \emph{Verwaltungsgebiet}|pw} zurückkehre (es iſt allerdings auch möglich, daß ich \textsc{Gotthardt\oindex{Gotthardpass@\textbf{Gotthardpass}, \emph{Pass}|pw}}–Frankfurt\oindex{Frankfurt am Main@\textbf{Frankfurt am Main}, \emph{Hauptstadt}|pw} fahre) hoffe ich{ }ſehr, Dir dort
               die \label{K_L03454-1v}\edtext{Hand drücken}{\lemma{\textnormal{\emph{Hand drücken}}}\Cendnote{\textnormal{Goldmann\pwindex{Goldmann, Paul 31.\,1.\,1865 Breslau – 25.\,9.\,1935 Wien@\textsc{Goldmann, Paul} (31.\,1.\,1865 Breslau – 25.\,9.\,1935 Wien), \emph{Schriftsteller, Journalist}|pwk} kehrte über Wien\oindex{Wien@\textbf{Wien}, \emph{Verwaltungsgebiet}|pwk} zurück. Am 21. 9. 1904 besuchte er Schnitzler.}}}\label{K_L03454-1} zu können.\pend
           
\pstart
           Mit herzlichen Grüßen an Dich, Frau\pwindex{Schnitzler, Olga 17.\,1.\,1882 Wien – 13.\,1.\,1970 Lugano@\textsc{Schnitzler, Olga} (17.\,1.\,1882 Wien – 13.\,1.\,1970 Lugano), \emph{Schauspielerin, Sängerin}|pwv} und Kind\pwindex{Schnitzler, Heinrich 9.\,8.\,1902 Hinterbrühl – 12.\,7.\,1982 Wien@\textsc{Schnitzler, Heinrich} (9.\,8.\,1902 Hinterbrühl – 12.\,7.\,1982 Wien), \emph{Regisseur, Schauspieler}|pwv} bin ich {\\[\baselineskip]}Dein getreuer {\\[\baselineskip]}\spacefill\mbox{Paul Goldmann.}\pend
           \leftskip=0em{}\selectlanguage{ngerman}\endnumbering\briefempfaengerindex{Schnitzler, Arthur@\textsc{Schnitzler, Arthur}!zzzGoldmann, Paul@\emph{von Paul Goldmann}!1904-08-311@{31. 8. 1904}|)be}\mylabel{L03454h}  \newcommand{\dateiname}{L03454}\newcommand{\titel}{Paul Goldmann an Arthur Schnitzler, 31. 8. 1904}\newcommand{\editorInnen}{Martin Anton Müller und Laura Untner}%% latex-leseansicht-abspann.tex
%% Abspann für die Leseansicht.
%% Der Schalter \ifkorrekturansicht ist bereits durch den Vorspann gesetzt.

%% latex-abspann.tex
%% Gemeinsamer Abspann für Korrekturansicht und Leseansicht.
%% Setzt den Schalter \ifkorrekturansicht voraus (gesetzt in den
%% einbindenden Dateien latex-korrekturansicht-abspann.tex bzw.
%% latex-leseansicht-abspann.tex).
%% ---------------------------------------------------------------

\normalsize

% Das esempio-Environment wird nur in der Leseansicht benötigt
\ifkorrekturansicht\else
\newenvironment{esempio}[3]%
{
    \vspace{1.5ex}
    \rlap{\underline{#1}}
    \par
    \setlength{\parindent}{0cm}
    \nopagebreak
    \leftskip=#2cm
    \rightskip=#3cm
}
{
    \par
}
\fi

\doendnotes{C}
\bigskip
\vfill

\clearpage

\footnotesize

\ifkorrekturansicht
  \lohead{\textsc{register}}
\fi

% theindex-Environment neu definieren ohne reledmac
\makeatletter
\renewenvironment{theindex}{%
  \ifkorrekturansicht
    \section*{\indexname}%
  \else
    \subsubsection*{Index der erwähnten Entitäten}%
  \fi
  \setlength{\parindent}{0pt}%
  \setlength{\parskip}{0pt plus 0.3pt}%
  \let\item\@idxitem
}{%
  \ifkorrekturansicht\clearpage\fi
}
\makeatother

\IfFileExists{\jobname-pw.ind}{\input{\jobname-pw.ind}}{}

% Quellenangabe nur in der Leseansicht
\ifkorrekturansicht\else
% Fallback-Definitionen, falls die .tex-Datei \titel etc. nicht gesetzt hat
\providecommand{\titel}{}
\providecommand{\editorInnen}{}
\providecommand{\dateiname}{\jobname}

\vspace{3cm}

\vfill

\footnotesize
\textsc{Quelle}: \titel. Herausgegeben von {\editorInnen}. In: \emph{Arthur Schnitzler: Briefwechsel mit Autorinnen und Autoren}.
 Digitale Edition, https://schnitzler-briefe.acdh.oeaw.ac.at/{\dateiname}.html (Stand \today)
\fi

\end{document}


