%% latex-leseansicht-vorspann.tex
%% Vorspann für die Leseansicht.
%% Lädt die gemeinsame Datei latex-vorspann.tex mit nicht gesetztem Schalter.

\newif\ifkorrekturansicht
\korrekturansichtfalse

\input{../tex-inputs/latex-vorspann}


         
         \renewcommand{\erwaehntePersonen}{Personen: Paul Goldmann, Fedor Mamroth, Olga Schnitzler, Heinrich Schnitzler, Franz Joseph Österreicher}
         \renewcommand{\erwaehnteOrte}{Orte: Bahnhof Trient, Brenta (Gebirge), Cima Tosa, Dolomiten, Engadin, Frankfurt am Main, Gotthardpass, Grand Hotel des Alpes, Imperial Hotel Trento, Italien, Madonna di Campiglio, Monte Spinale, Riva del Garda, Südtirol, Tirol, Trient, Wien}
         \renewcommand{\erwaehnteWerke}{}
               \section[ Paul Goldmann an Arthur Schnitzler, 31. 8. 1904]{ Paul Goldmann an Arthur Schnitzler, 31. 8. 1904}\nopagebreak\mylabel{v}\rehead{ }\begin{ledgroupsized}[t]{13cm}\normalsize\beginnumbering\briefempfaengerindex{Schnitzler, Arthur@\textsc{Schnitzler, Arthur}!zzzGoldmann, Paul@\emph{von Paul Goldmann}!1904-08-311@{31. 8. 1904}|(be} \toendnotes[C]{\smallbreak\pagebreak[2]} \Standort{DLA, A:Schnitzler, HS.NZ85.1.3174.}
\physDesc{Brief, 1 Blatt, 2 Seiten, 632 Zeichen
\newline{}Handschrift: schwarze Tinte, deutsche Kurrent
\newline{}Schnitzler: mit rotem Buntstift eine Unterstreichung }\toendnotes[C]{\smallbreak}\pstart
           \noindent{}\raggedleft{}{\pb}\textcolor{gray}{\textbf{BRENTA-DOLOMITEN\oindex{Brenta (Gebirge)@\textbf{Brenta (Gebirge)}|pw} (CIMA TOSA\oindex{Cima Tosa@\textbf{Cima Tosa}|pw} 3176 M) VOM M. SPINALE\oindex{Monte Spinale@\textbf{Monte Spinale}|pw} 2021 M.}}\pend
           \pstart
           \noindent{}\raggedleft{}\textcolor{gray}{\textbf{MADONNA DI CAMPIGLIO\oindex{Madonna di Campiglio@\textbf{Madonna di Campiglio}|pw} 1553 MTR.}}\pend
           \pstart
           \noindent{}\textcolor{gray}{\textbf{Imperial Hôtel Trento\oindex{Imperial Hotel Trento@\textbf{Imperial Hotel Trento}|pw}}}\pend
           \pstart
           \textcolor{gray}{\textbf{\begin{otherlanguage}{french}vis-à-vis de la Gare\oindex{Bahnhof Trient@\textbf{Bahnhof Trient}|pw}\end{otherlanguage}.}}\hfill \textcolor{gray}{\textbf{Saison Juni–September.}}\pend
           \pstart
           \textcolor{gray}{\textbf{\textsc{Trient\oindex{Trient@\textbf{Trient}|pw} (Südtirol\oindex{Suedtirol@\textbf{Südtirol}|pw})}}}\pend
           \pstart
           \textcolor{gray}{\textbf{MÊME MAISON:}}{\\}\textcolor{gray}{\textbf{\textsc{Grand Hôtel des Alpes\oindex{Grand Hotel des Alpes@\textbf{Grand Hotel des Alpes}|pw}}}}{\\}\textcolor{gray}{\textbf{\textsc{Madonna di Campiglio\oindex{Madonna di Campiglio@\textbf{Madonna di Campiglio}|pw}.}}}\pend
           \pstart
           \textcolor{gray}{\textbf{\emph{F. J. Oesterreicher\pwindex{Oesterreicher, Franz Joseph 1846-08-04 – 1909-08-22@\textsc{Österreicher, Franz Joseph} (1846-08-04 – 1909-08-22), \emph{Hotelbesitzer}|pw}, Prop\textsuperscript{re}}}}\pend
           \pstart
           \raggedleft{}\textcolor{gray}{\textbf{\textbf{Madonna di Campiglio\oindex{Madonna di Campiglio@\textbf{Madonna di Campiglio}|pw}},}}{\\}31. Auguſt \textcolor{gray}{\textbf{190}}4.\pend
           \pstart\center{}Mein lieber Freund,\pend\pstart
           Es regnet in \textsc{Campiglio\oindex{Madonna di Campiglio@\textbf{Madonna di Campiglio}|pw}}, und ich fahre morgen nach \textsc{Riva\oindex{Riva del Garda@\textbf{Riva del Garda}|pw}}. Von da wahrſcheinlich weiter nach Oberitalien\oindex{Italien@\textbf{Italien}|pw}. Das Nähere hängt von meinem Onkel\pwindex{Mamroth, Fedor 21.02.1851 – 25.06.1907@\textsc{Mamroth, Fedor} (21.02.1851 – 25.06.1907), \emph{Journalist, Kritiker}|pwv} ab, der aus dem Engadin\oindex{Engadin@\textbf{Engadin}|pw} an einen von ihm noch zu beſtimmenden Ort kommt. Im Gebirge\oindex{Dolomiten@\textbf{Dolomiten}|pwv} werde ich Dich alſo wohl nicht
               ſehen können, – rathe Dir auch {\pb}dringend ab, nach
                  Tirol\oindex{Tirol@\textbf{Tirol}|pw}\oindex{Suedtirol@\textbf{Südtirol}|pw} zu kommen, ehe das Wetter ſich
               gebeſſert hat (wozu anſcheinend wenig Ausſicht.) Aber wenn ich über Wien\oindex{Wien@\textbf{Wien}|pw} zurückkehre (es iſt allerdings auch möglich, daß ich \textsc{Gotthardt\oindex{Gotthardpass@\textbf{Gotthardpass}|pw}}–Frankfurt\oindex{Frankfurt am Main@\textbf{Frankfurt am Main}|pw} fahre) hoffe ich ſehr, Dir dort
               die \label{K_L03454-1v}\edtext{Hand drücken}{\lemma{\textnormal{\emph{Hand drücken}}}\Cendnote{\textnormal{Goldmann\pwindex{Goldmann, Paul 31.01.1865 – 25.09.1935@\textsc{Goldmann, Paul} (31.01.1865 – 25.09.1935), \emph{Schriftsteller, Journalist}|pwk} kehrte über Wien\oindex{Wien@\textbf{Wien}|pwk} zurück. Am 21. 9. 1904 besuchte er Schnitzler\pwindex{Schnitzler, Arthur 15.05.1862 – 21.10.1931@\textsc{Schnitzler, Arthur} (15.05.1862 – 21.10.1931), \emph{Schriftsteller, Mediziner}|pwk}.}}}\label{K_L03454-1h} zu können.\pend
           \pstart
           Mit herzlichen Grüßen an Dich, Frau\pwindex{Schnitzler, Olga 17.01.1882 – 13.01.1970@\textsc{Schnitzler, Olga} (17.01.1882 – 13.01.1970), \emph{Schauspielerin, Sängerin}|pwv} und Kind\pwindex{Schnitzler, Heinrich 09.08.1902 – 12.07.1982@\textsc{Schnitzler, Heinrich} (09.08.1902 – 12.07.1982), \emph{Regisseur, Schauspieler}|pwv} bin ich {\\[\baselineskip]}Dein getreuer {\\[\baselineskip]}\spacefill\mbox{Paul Goldmann.}\pend
           \leftskip=0em{}
         
         \endnumbering\mylabel{h}\end{ledgroupsized}  \newcommand{\dateiname}{L03454}\newcommand{\titel}{Paul Goldmann an Arthur Schnitzler, 31. 8. 1904}\newcommand{\editorInnen}{Martin Anton Müller und Laura Untner}%% latex-leseansicht-abspann.tex
%% Abspann für die Leseansicht.
%% Der Schalter \ifkorrekturansicht ist bereits durch den Vorspann gesetzt.

%% latex-abspann.tex
%% Gemeinsamer Abspann für Korrekturansicht und Leseansicht.
%% Setzt den Schalter \ifkorrekturansicht voraus (gesetzt in den
%% einbindenden Dateien latex-korrekturansicht-abspann.tex bzw.
%% latex-leseansicht-abspann.tex).
%% ---------------------------------------------------------------

\normalsize

% Das esempio-Environment wird nur in der Leseansicht benötigt
\ifkorrekturansicht\else
\newenvironment{esempio}[3]%
{
    \vspace{1.5ex}
    \rlap{\underline{#1}}
    \par
    \setlength{\parindent}{0cm}
    \nopagebreak
    \leftskip=#2cm
    \rightskip=#3cm
}
{
    \par
}
\fi

\doendnotes{C}
\bigskip
\vfill

\clearpage

\footnotesize

\ifkorrekturansicht
  \lohead{\textsc{register}}
\fi

% theindex-Environment neu definieren ohne reledmac
\makeatletter
\renewenvironment{theindex}{%
  \ifkorrekturansicht
    \section*{\indexname}%
  \else
    \subsubsection*{Index der erwähnten Entitäten}%
  \fi
  \setlength{\parindent}{0pt}%
  \setlength{\parskip}{0pt plus 0.3pt}%
  \let\item\@idxitem
}{%
  \ifkorrekturansicht\clearpage\fi
}
\makeatother

\IfFileExists{\jobname-pw.ind}{\input{\jobname-pw.ind}}{}

% Quellenangabe nur in der Leseansicht
\ifkorrekturansicht\else
% Fallback-Definitionen, falls die .tex-Datei \titel etc. nicht gesetzt hat
\providecommand{\titel}{}
\providecommand{\editorInnen}{}
\providecommand{\dateiname}{\jobname}

\vspace{3cm}

\vfill

\footnotesize
\textsc{Quelle}: \titel. Herausgegeben von {\editorInnen}. In: \emph{Arthur Schnitzler: Briefwechsel mit Autorinnen und Autoren}.
 Digitale Edition, https://schnitzler-briefe.acdh.oeaw.ac.at/{\dateiname}.html (Stand \today)
\fi

\end{document}


      