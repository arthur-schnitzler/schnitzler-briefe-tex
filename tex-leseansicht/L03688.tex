%% latex-leseansicht-vorspann.tex
%% Vorspann für die Leseansicht.
%% Lädt die gemeinsame Datei latex-vorspann.tex mit nicht gesetztem Schalter.

\newif\ifkorrekturansicht
\korrekturansichtfalse

\input{../tex-inputs/latex-vorspann}


\section[Stefan Zweig an Arthur Schnitzler, 18. 1. 1928]{L03688 Stefan Zweig an Arthur Schnitzler, 18. 1. 1928}
\nopagebreak\mylabel{L03688v}
\rehead{ }\normalsize\beginnumbering\briefempfaengerindex{Schnitzler, Arthur@\textsc{Schnitzler, Arthur}!zzzZweig, Stefan@\emph{von Stefan Zweig}!1928-01-181@{18. 1. 1928}|(be}
\toendnotes[C]{\smallbreak\pagebreak[2]}
\correspDesc{Versand  durch Stefan Zweig am 18. 1. 1928 in Salzburg
\newline{}Erhalt  durch Arthur Schnitzler im Zeitraum [19. 1. 1928
                  – 23. 1. 1928?] in Wien}\toendnotes[C]{\smallbreak}
\Standort{CUL, Schnitzler, B 118.}
\physDesc{Brief, 1 Blatt, 2 Seiten, 2674 Zeichen
\newline{}Schreibmaschine
\newline{}Handschrift: lila Tinte, lateinische Kurrent (\noindent{}Unterschrift, einige Ergänzungen)
\newline{}Schnitzler: 1) mit Bleistift beschriftet: »\textsc{Zweig}«  2) mit rotem Buntstift acht Unterstreichungen}
\buchAbdrucke{\weitereDrucke{Stefan Zweig: \emph{Briefwechsel mit Hermann Bahr, Sigmund Freud, Rainer Maria
                        Rilke und Arthur Schnitzler}. Herausgegeben von Jeffrey B. Berlin, Hans-Ulrich Lindken und Donald A. Prater. Frankfurt am Main: \emph{S. Fischer} 1987, S. 436–437.} }\toendnotes[C]{\smallbreak}
\pstart
           {\pb}\textcolor{gray}{\textbf{SZ}}\hfill \textcolor{gray}{\textbf{KAPUZINERBERG 5\oindex{Paschinger Schlössl@\textbf{Paschinger Schlössl}, \emph{Wohngebäude}|pw}}}\pend
           
\pstart
           \raggedleft{}\textcolor{gray}{\textbf{SALZBURG\oindex{Salzburg@\textbf{Salzburg}, \emph{Verwaltungsgebiet}|pw}}}\pend
           
\pstart
           \raggedleft{}18. Jan 1928.\pend
           
\pstart{}Sehr verehrter, lieber Herr Doktor!\pend\vspace{0.5em}
\pstart
           Von einem Menschen, den man liebt, ausgelacht zu werden, tut nicht sehr weh. So
               gestehe ich Ihnen offen das kuriose Faktum ein, dass ich gar nicht weiss, was für
               Bedingungen ich damals für meine Erlaubnis der \label{K_L03688-1v}\edtext{russischen\oindex{Russland@\textbf{Russland}|pw}{ }Gesamtausgabe\pwindex{Zweig, Stefan 28.\,11.\,1881 Wien – 23.\,2.\,1942 Petrópolis@\textsc{Zweig, Stefan} (28.\,11.\,1881 Wien – 23.\,2.\,1942 Petrópolis), \emph{Schriftsteller}!Sobranie sočinenij@\strich\emph{Sobranie sočinenij}|pw}}{\lemma{\textnormal{\emph{russischen Gesamtausgabe}}}\Cendnote{\textnormal{Zu der russischen Werkausgabe\pwindex{Zweig, Stefan 28.\,11.\,1881 Wien – 23.\,2.\,1942 Petrópolis@\textsc{Zweig, Stefan} (28.\,11.\,1881 Wien – 23.\,2.\,1942 Petrópolis), \emph{Schriftsteller}!Sobranie sočinenij@\strich\emph{Sobranie sočinenij}|pwkv}{ }Stefan Zweigs\pwindex{Zweig, Stefan 28.\,11.\,1881 Wien – 23.\,2.\,1942 Petrópolis@\textsc{Zweig, Stefan} (28.\,11.\,1881 Wien – 23.\,2.\,1942 Petrópolis), \emph{Schriftsteller}|pwk} mit Vorwort Maxim Gorkis\pwindex{Gorkij, Maxim 28.\,3.\,1868 Nischni Nowgorod – 18.\,6.\,1936 Moskau@\textsc{Gorkij, Maxim} (28.\,3.\,1868 Nischni Nowgorod – 18.\,6.\,1936 Moskau), \emph{Schriftsteller}|pwk} und Einleitung\pwindex{Specht, Richard 7.\,12.\,1870 Wien – 18.\,3.\,1932 ebd.@\textsc{Specht, Richard} (7.\,12.\,1870 Wien – 18.\,3.\,1932 ebd.), \emph{Schriftsteller, Journalist, Kritiker}!Stefan Zweig. Versuch eines Bildnisses@\strich\emph{Stefan Zweig. Versuch eines Bildnisses}|pwkv} von Richard
                     Specht\pwindex{Specht, Richard 7.\,12.\,1870 Wien – 18.\,3.\,1932 ebd.@\textsc{Specht, Richard} (7.\,12.\,1870 Wien – 18.\,3.\,1932 ebd.), \emph{Schriftsteller, Journalist, Kritiker}|pwk}, die zwischen 1927 und 1932 in zwölf
                  Bänden beim Verlag \emph{Wremja}\orgindex{Wremja@Wremja|pwk} entstand, vgl.
                     Konstantin Asadowski: \emph{Stefan Zweig in der UdSSR}.
                     In: Ders.: \emph{Russisch-deutsche Verflechtungen. Ausgewählte
                        Beiträge zur Literatur- und Kulturgeschichte des 19. und
                        20. Jahrhunderts}. Herausgegeben von Fedor Poljakov und Natalia
                     Bakshi. In: \emph{Schriftenreihe des Instituts für russisch-deutsche
                        Literatur- {\kaufmannsund} Kulturbeziehungen an der RGGU
                        Moskau, Band 24}, Paderborn: \emph{Fink
                        Brill}{ }2022, S. 291–313, hier S. 298–299.}}}\label{K_L03688-1} gemacht habe. Ich
               bekam einmal 150 Dollar und einmal 50 Dollar, setzte für Specht\pwindex{Specht, Richard 7.\,12.\,1870 Wien – 18.\,3.\,1932 ebd.@\textsc{Specht, Richard} (7.\,12.\,1870 Wien – 18.\,3.\,1932 ebd.), \emph{Schriftsteller, Journalist, Kritiker}|pw} eine recht anständige Bezahlung\pwindex{Specht, Richard 7.\,12.\,1870 Wien – 18.\,3.\,1932 ebd.@\textsc{Specht, Richard} (7.\,12.\,1870 Wien – 18.\,3.\,1932 ebd.), \emph{Schriftsteller, Journalist, Kritiker}!Stefan Zweig. Versuch eines Bildnisses@\strich\emph{Stefan Zweig. Versuch eines Bildnisses}|pwv} durch. Seit der Zeit habe ich keine Abrechnung
               bekommen und sie auch gar nicht eingefordert, wie überhaupt meine ganzen materiellen
               Angelegenheiten in einer etwas fantastisch leichtfertigen Art von mir geführt werden.
               Ich glaub\strikeout{t}e, \introOben{}in diesem Bezug\introOben{}
               allerhand Rekorde zu schlagen, weil ich immer das annehme, was man mir schickt, und
               nie nachfrage und mich erkundige. Im allgemeinen stehe ich auf dem Standpunkt, dass
               aus dem Ausland mit Ausnahme von England\oindex{Vereinigtes Königreich@\textbf{Vereinigtes Königreich}|pw} und
                  Amerika\oindex{Vereinigte Staaten von Amerika [USA]@\textbf{Vereinigte Staaten von Amerika [USA]}|pw} nichts zu holen ist. Die russischen\oindex{Russland@\textbf{Russland}|pw} Bücher sind dermassen billig und
               unsere Rechte so dubios, die Zeit mit Korrespondenzen und Mahnungen \label{T_L03688-1v}\edtext{unsererseits}{\lemma{\textnormal{\emph{unsererseits}}}\Cendnote{\textnormal{Er schreibt: »ansererseits«.}}}\label{T_L03688-1} so
               kostbar, dass ich einfach da die Zügel rennen liess. Auch was ich von Norwegen\oindex{Norwegen@\textbf{Norwegen}|pw}, Schweden\oindex{Schweden@\textbf{Schweden}|pw}, Polen\oindex{Polen@\textbf{Polen}|pw}, Ungarn\oindex{Ungarn@\textbf{Ungarn}|pw} beziehe, ist gleich null; in diesen kleinen Ländem
               kommt man zu nichts. Mein Traum ist und bleibt, dass ein paar von uns sich
               zusammentun sollten und sich einen Agenten halten, der den ganzen Nachrdrucks- und
               Uebersetzungsbetrieb und – \label{K_L03688-2v}\edtext{\begin{otherlanguage}{english}last not least\end{otherlanguage}}{\lemma{\textnormal{\emph{last not least}}}\Cendnote{\textnormal{englisch: nicht zuletzt}}}\label{K_L03688-2} – unsere
               ganze Korrespondenz übemimmt. So machen es die Engländer\oindex{England@\textbf{England}, \emph{Land}|pw}: die schreiben auf jeden Brief mit Blaustift eine Zeile, der Agent
                  \strikeout{be}antwortet für ihn, kämpft für ihn, holt ihm
               trotz der 10{\%} die zehnfachen Honorare heraus und die innere
               Ruhe ist nicht verstört. Ihnen mag es vielleicht etwas besser gehen. Ich aber, der
               ich mich durch meine \substVorne{}\textsuperscript{N}\substDazwischen{}n\substHinten{}eugierige und kommunikative Natur und durch eine aktive Internationalität {\pb}in vielerlei eingelassen habe, stehe heute
               vor dem Problem, dass auf eine Seite Produktion 15 Seiten Briefe kommen. Und deshalb
               eben habe ich beschlossen, zumindest die materiellen kleinen Angelegenheiten wie
               Uebersetzungen selbst laufen zu lassen und Sie haben sich vielleicht an die
               allerschlechteste Auskunftei in diesen Dingen gewandt.\pend
           
\pstart
           Ich hoffe, bald mit meiner Essay-Arbeit\pwindex{Zweig, Stefan 28.\,11.\,1881 Wien – 23.\,2.\,1942 Petrópolis@\textsc{Zweig, Stefan} (28.\,11.\,1881 Wien – 23.\,2.\,1942 Petrópolis), \emph{Schriftsteller}!Drei Dichter ihres Lebens. Casanova – Stendhal – Tolstoi@\strich\emph{Drei Dichter ihres Lebens. Casanova – Stendhal – Tolstoi}|pwv} fertig zu sein. Die kleine Komödie\pwindex{Zweig, Stefan 28.\,11.\,1881 Wien – 23.\,2.\,1942 Petrópolis@\textsc{Zweig, Stefan} (28.\,11.\,1881 Wien – 23.\,2.\,1942 Petrópolis), \emph{Schriftsteller}!Quiproquo. Komödie in drei Akten@\strich\emph{Quiproquo. Komödie in drei Akten}|pwv}\pwindex{Lernet-Holenia, Alexander 21.\,10.\,1897 Wien – 3.\,7.\,1976 ebd.@\textsc{Lernet-Holenia, Alexander} (21.\,10.\,1897 Wien – 3.\,7.\,1976 ebd.), \emph{Schriftsteller}!Quiproquo. Komödie in drei Akten@\strich\emph{Quiproquo. Komödie in drei Akten}|pwv} war ich zu faul und zu dumm selbst zu schreiben und
               es macht\introOben{}e\introOben{} mir Spass, zum erstenmal im Leben mich mit einer
               Kompagnie zu versuchen. Mein Freund und Nachbar Lernet-Holenia\pwindex{Lernet-Holenia, Alexander 21.\,10.\,1897 Wien – 3.\,7.\,1976 ebd.@\textsc{Lernet-Holenia, Alexander} (21.\,10.\,1897 Wien – 3.\,7.\,1976 ebd.), \emph{Schriftsteller}|pw}\introOben{}, der gerade vorbeikam\introOben{}{[},{]} war von Thema und Linie sehr entzückt und nun amüsieren wir
               uns täglich drei Stunden ausgezeichnet, indem wir vergnüglich tun, was man sonst
               Arbeit zu nennen pflegt \substVorne{}\textsuperscript{.}\substDazwischen{}: ob das Kind\pwindex{Zweig, Stefan 28.\,11.\,1881 Wien – 23.\,2.\,1942 Petrópolis@\textsc{Zweig, Stefan} (28.\,11.\,1881 Wien – 23.\,2.\,1942 Petrópolis), \emph{Schriftsteller}!Quiproquo. Komödie in drei Akten@\strich\emph{Quiproquo. Komödie in drei Akten}|pwv}\pwindex{Lernet-Holenia, Alexander 21.\,10.\,1897 Wien – 3.\,7.\,1976 ebd.@\textsc{Lernet-Holenia, Alexander} (21.\,10.\,1897 Wien – 3.\,7.\,1976 ebd.), \emph{Schriftsteller}!Quiproquo. Komödie in drei Akten@\strich\emph{Quiproquo. Komödie in drei Akten}|pwv} lebendig bleibt{[},{]} weiss Gott,
                     jedesfalls macht es viel Spass, es zu schaukeln.\substHinten{}\pend
           
\pstart
           Mit vielen innigen und getreuen Grüssen{\\[\baselineskip]}Ihr{\\[\baselineskip]}\spacefill\mbox{{[}hs.:{]} Stefan Zweig}\pend
           \leftskip=0em{}\selectlanguage{ngerman}\endnumbering\briefempfaengerindex{Schnitzler, Arthur@\textsc{Schnitzler, Arthur}!zzzZweig, Stefan@\emph{von Stefan Zweig}!1928-01-181@{18. 1. 1928}|)be}\mylabel{L03688h}  \newcommand{\dateiname}{L03688}\newcommand{\titel}{Stefan Zweig an Arthur Schnitzler, 18. 1. 1928}\newcommand{\editorInnen}{Selma Jahnke und Martin Anton Müller}%% latex-leseansicht-abspann.tex
%% Abspann für die Leseansicht.
%% Der Schalter \ifkorrekturansicht ist bereits durch den Vorspann gesetzt.

%% latex-abspann.tex
%% Gemeinsamer Abspann für Korrekturansicht und Leseansicht.
%% Setzt den Schalter \ifkorrekturansicht voraus (gesetzt in den
%% einbindenden Dateien latex-korrekturansicht-abspann.tex bzw.
%% latex-leseansicht-abspann.tex).
%% ---------------------------------------------------------------

\normalsize

% Das esempio-Environment wird nur in der Leseansicht benötigt
\ifkorrekturansicht\else
\newenvironment{esempio}[3]%
{
    \vspace{1.5ex}
    \rlap{\underline{#1}}
    \par
    \setlength{\parindent}{0cm}
    \nopagebreak
    \leftskip=#2cm
    \rightskip=#3cm
}
{
    \par
}
\fi

\doendnotes{C}
\bigskip
\vfill

\clearpage

\footnotesize

\ifkorrekturansicht
  \lohead{\textsc{register}}
\fi

% theindex-Environment neu definieren ohne reledmac
\makeatletter
\renewenvironment{theindex}{%
  \ifkorrekturansicht
    \section*{\indexname}%
  \else
    \subsubsection*{Index der erwähnten Entitäten}%
  \fi
  \setlength{\parindent}{0pt}%
  \setlength{\parskip}{0pt plus 0.3pt}%
  \let\item\@idxitem
}{%
  \ifkorrekturansicht\clearpage\fi
}
\makeatother

\IfFileExists{\jobname-pw.ind}{\input{\jobname-pw.ind}}{}

% Quellenangabe nur in der Leseansicht
\ifkorrekturansicht\else
% Fallback-Definitionen, falls die .tex-Datei \titel etc. nicht gesetzt hat
\providecommand{\titel}{}
\providecommand{\editorInnen}{}
\providecommand{\dateiname}{\jobname}

\vspace{3cm}

\vfill

\footnotesize
\textsc{Quelle}: \titel. Herausgegeben von {\editorInnen}. In: \emph{Arthur Schnitzler: Briefwechsel mit Autorinnen und Autoren}.
 Digitale Edition, https://schnitzler-briefe.acdh.oeaw.ac.at/{\dateiname}.html (Stand \today)
\fi

\end{document}


