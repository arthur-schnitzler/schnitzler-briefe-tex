%% latex-leseansicht-vorspann.tex
%% Vorspann für die Leseansicht.
%% Lädt die gemeinsame Datei latex-vorspann.tex mit nicht gesetztem Schalter.

\newif\ifkorrekturansicht
\korrekturansichtfalse

\input{../tex-inputs/latex-vorspann}


         
         \renewcommand{\erwaehntePersonen}{Personen: Ferdinand Bronner, Ida Falk, Johann Wolfgang von Goethe, Ludwig Grann, Hugo von Hofmannsthal, Guy de Maupassant, Friedrich von Oppeln-Bronikowski, Felix Salten, Siegfried Trebitsch}
         \renewcommand{\erwaehnteInstitutionen}{Institutionen: Emil Goldschmidt Verlag}
         \renewcommand{\erwaehnteOrte}{Orte: Bad Ischl, Berlin, Frankfurt am Main, Hotel und Pension Rudolfshöhe (Leopold Petter), München, Nürnberg, Wien, Wiesbaden}
         \renewcommand{\erwaehnteWerke}{Werke: Aus dem Nachlasse von Maupassant. Eine Leidenschaft, Begräbnis, Das Bergwerk zu Falun, Das Manhard-Zimmer, Der Hinterbliebene, Der Hinterbliebene. Kurze Novellen, Der Schleier der Beatrice. Schauspiel in fünf Akten, Eine Goethe-Enquête, Familie Wawroch. Ein österreichisches Drama in vier Akten, Fernen, Flucht, Heldentod, Lebenszeit, Schöne Seelen. Komödie in einem Akt, Sedan, Vater Milon und andere Erzählungen. Neue Novellen aus dem litterarischen Nachlaß, Wiener Allgemeine Montags-Zeitung, Wiener Allgemeine Rundschau}
               \section[ Arthur Schnitzler an Felix Salten, 4. 9. 1899]{ Arthur Schnitzler an Felix Salten, 4. 9. 1899}\nopagebreak\mylabel{v}\rehead{ }\begin{ledgroupsized}[t]{13cm}\normalsize\beginnumbering\briefempfaengerindex{Salten, Felix@\textsc{Salten, Felix}!zzzSchnitzler, Arthur@\emph{von Arthur Schnitzler}!1899-09-041@{4. 9. 1899}|(be} \toendnotes[C]{\smallbreak\pagebreak[2]} \Standort{Wienbibliothek im Rathaus, ZPH 1681, 2.1.516.}
\physDesc{Brief, 2 Blätter, 8 Seiten, 1693 Zeichen
\newline{}Handschrift: Bleistift, deutsche Kurrent
\newline{}Ordnung: mit Bleistift von unbekannter Hand Nummerierung der Doppelseiten des
                                 Konvoluts: »69«–»72« }\buchAbdrucke{\weitereDrucke{Arthur Schnitzler: \emph{Briefe 1875–1912}. Hg. Therese Nickl und Heinrich Schnitzler. Frankfurt am Main: \emph{S. Fischer} 1981, S. 375–376.} }\toendnotes[C]{\smallbreak}\pstart
           \raggedleft{}{\pb}Ischl, Rudolfshöhe\oindex{Hotel und Pension Rudolfshoehe (Leopold Petter)@\textbf{Hotel und Pension Rudolfshöhe (Leopold Petter)}|pw}{ }4/9 99.\pend
           \pstart
           lieber Freund, ich will Ihnen vor allem ſagen, dſs mir nicht nur
                  »Flucht\pwindex{Salten, Felix 06.09.1869 – 08.10.1945@\textsc{Salten, Felix} (06.09.1869 – 08.10.1945), \emph{Schriftsteller, Journalist}!Flucht1899-07-31@\strich\emph{Flucht} {[}1899-07-31{]}|pw}«, ſondern auch das \textsc{Manhard}zi{\geminationm}er\pwindex{Salten, Felix 06.09.1869 – 08.10.1945@\textsc{Salten, Felix} (06.09.1869 – 08.10.1945), \emph{Schriftsteller, Journalist}!Manhard-Zimmer1899-08-21@\strich\emph{Das Manhard-Zimmer} {[}1899-08-21{]}|pw} noch
               beſſer \label{K_L02967-1v}\edtext{gefallen}{\lemma{\textnormal{\emph{gefallen}}}\Cendnote{\textnormal{siehe Felix Salten an Arthur Schnitzler, [29. 8. 1899]}}}\label{K_L02967-1h} haben, als nach dem erſten Leſen. Ich zweifle nicht, dſs Ihre Novelletten\pwindex{Salten, Felix 06.09.1869 – 08.10.1945@\textsc{Salten, Felix} (06.09.1869 – 08.10.1945), \emph{Schriftsteller, Journalist}!Flucht1899-07-31@\strich\emph{Flucht} {[}1899-07-31{]}|pwv}\pwindex{Salten, Felix 06.09.1869 – 08.10.1945@\textsc{Salten, Felix} (06.09.1869 – 08.10.1945), \emph{Schriftsteller, Journalist}!Fernen1897-12-25@\strich\emph{Fernen} {[}1897-12-25{]}|pwv}\pwindex{Salten, Felix 06.09.1869 – 08.10.1945@\textsc{Salten, Felix} (06.09.1869 – 08.10.1945), \emph{Schriftsteller, Journalist}!Sedan1899-07-03@\strich\emph{Sedan} {[}1899-07-03{]}|pwv}\pwindex{Salten, Felix 06.09.1869 – 08.10.1945@\textsc{Salten, Felix} (06.09.1869 – 08.10.1945), \emph{Schriftsteller, Journalist}!Lebenszeit1900@\strich\emph{Lebenszeit} {[}1900{]}|pwv}\pwindex{Salten, Felix 06.09.1869 – 08.10.1945@\textsc{Salten, Felix} (06.09.1869 – 08.10.1945), \emph{Schriftsteller, Journalist}!Hinterbliebene1899-03-04 – 1899-03-11@\strich\emph{Der Hinterbliebene} {[}1899-03-04 – 1899-03-11{]}|pwv}\pwindex{Salten, Felix 06.09.1869 – 08.10.1945@\textsc{Salten, Felix} (06.09.1869 – 08.10.1945), \emph{Schriftsteller, Journalist}!Manhard-Zimmer1899-08-21@\strich\emph{Das Manhard-Zimmer} {[}1899-08-21{]}|pwv}\pwindex{Salten, Felix 06.09.1869 – 08.10.1945@\textsc{Salten, Felix} (06.09.1869 – 08.10.1945), \emph{Schriftsteller, Journalist}!Begraebnis17. 7. 1893@\strich\emph{Begräbnis} {[}17. 7. 1893{]}|pwv}\pwindex{Salten, Felix 06.09.1869 – 08.10.1945@\textsc{Salten, Felix} (06.09.1869 – 08.10.1945), \emph{Schriftsteller, Journalist}!Heldentod1895-01-01@\strich\emph{Heldentod} {[}1895-01-01{]}|pwv} ein hübſches { }Buch\pwindex{Salten, Felix 06.09.1869 – 08.10.1945@\textsc{Salten, Felix} (06.09.1869 – 08.10.1945), \emph{Schriftsteller, Journalist}!Hinterbliebene. Kurze Novellen1900@\strich\emph{Der Hinterbliebene. Kurze Novellen} {[}1900{]}|pwv} gäben, möchte aber von
               einem entgiltigen {\pb}Urtheil über die Wirkung
               als ganzes, \uline{alle} Sachen auf einmal, womöglich in der
               von Ihnen gewählten Reihenfolge leſen. Herausgeben unbedingt, ſag ich ſchon heute,
               und womöglich zugleich mit dem \label{K_L02967-2v}\edtext{Stück\pwindex{Salten, Felix 06.09.1869 – 08.10.1945@\textsc{Salten, Felix} (06.09.1869 – 08.10.1945), \emph{Schriftsteller, Journalist}!Schoene Seelen. Komoedie in einem Akt1902@\strich\emph{Schöne Seelen. Komödie in einem Akt} {[}1902{]}|pwv}}{\lemma{\textnormal{\emph{Stück}}}\Cendnote{\textnormal{siehe Felix Salten an Arthur Schnitzler, 9. 10. 1899}}}\label{K_L02967-2h} herauskommen. – In der {\pb}\label{K_L02967-3v}\edtext{Zeitung\pwindex{Wiener Allgemeine Montags-Zeitung1899-07-03 – 1899-12-18@\emph{Wiener Allgemeine Montags-Zeitung} {[}1899-07-03 – 1899-12-18{]}|pwv}}{\lemma{\textnormal{\emph{Zeitung}}}\Cendnote{\textnormal{Von der ersten Ausgabe weg, die am 3. 7. 1899 erschienen war, betreute Salten\pwindex{Salten, Felix 06.09.1869 – 08.10.1945@\textsc{Salten, Felix} (06.09.1869 – 08.10.1945), \emph{Schriftsteller, Journalist}|pwk} die Rubrik »\emph{Wiener
                     Allgemeine Rundschau}\pwindex{Wiener Allgemeine Rundschau1899-07-03 – 1899-12-18@\emph{Wiener Allgemeine Rundschau} {[}1899-07-03 – 1899-12-18{]}|pwk}« der wöchentlich erscheinenenden \emph{Wiener Allgemeinen Montags-Zeitung}\pwindex{Wiener Allgemeine Montags-Zeitung1899-07-03 – 1899-12-18@\emph{Wiener Allgemeine Montags-Zeitung} {[}1899-07-03 – 1899-12-18{]}|pwk}. Das Blatt\pwindex{Wiener Allgemeine Montags-Zeitung1899-07-03 – 1899-12-18@\emph{Wiener Allgemeine Montags-Zeitung} {[}1899-07-03 – 1899-12-18{]}|pwkv} wurde Mitte Dezember 1899 eingestellt.}}}\label{K_L02967-3h} findet ſich viel leſenswerthes;
               natürlich iſt es Ihnen aus Gründen, die nicht in Ihnen liegen, unmöglich, das
               Anſtrebenswerthe daraus zu machen. Glänzend hab ich Ihre \label{K_L02967-4v}\edtext{Goethe\pwindex{Goethe, Johann Wolfgang von 1749-08-28 – 1832-03-22@\textsc{Goethe, Johann Wolfgang von} (1749-08-28 – 1832-03-22), \emph{Schriftsteller}|pw}späße\pwindex{Eine Goethe-Enquête1899-08-28@\emph{Eine Goethe-Enquête} {[}1899-08-28{]}|pwv}}{\lemma{\textnormal{\emph{Goethespäße}}}\Cendnote{\textnormal{[Felix Salten\pwindex{Salten, Felix 06.09.1869 – 08.10.1945@\textsc{Salten, Felix} (06.09.1869 – 08.10.1945), \emph{Schriftsteller, Journalist}|pwk}]: \emph{Eine Goethe-Enquête}\pwindex{Eine Goethe-Enquête1899-08-28@\emph{Eine Goethe-Enquête} {[}1899-08-28{]}|pwk}. In: \emph{Wiener Allgemeine Montags-Zeitung}\pwindex{Wiener Allgemeine Montags-Zeitung1899-07-03 – 1899-12-18@\emph{Wiener Allgemeine Montags-Zeitung} {[}1899-07-03 – 1899-12-18{]}|pwk}, 28. 8. 1899, S. 3. Die nicht gezeichnete Umfrage\pwindex{Eine Goethe-Enquête1899-08-28@\emph{Eine Goethe-Enquête} {[}1899-08-28{]}|pwkv} ist deutlich als
                  Satire erkennbar (»indem wir eine Anzahl hervorragender Persönlichkeiten im
                     Geiste um ihre Meinung befragt«) und bringt erfundene Aussagen
                  von 17 Prominenten zu Goethe\pwindex{Goethe, Johann Wolfgang von 1749-08-28 – 1832-03-22@\textsc{Goethe, Johann Wolfgang von} (1749-08-28 – 1832-03-22), \emph{Schriftsteller}|pwk}.}}}\label{K_L02967-4h}
               gefunden. Können Sie mir die \label{K_L02967-5v}\edtext{Familie{ }{\pb}\textsc{Wawroch}\pwindex{Familie Wawroch. Ein oesterreichisches Drama in vier Akten1899@\emph{Familie Wawroch. Ein österreichisches Drama in vier Akten} {[}1899{]}|pw} von Adamus\pwindex{Bronner, Ferdinand 15.10.1867 – 08.06.1948@\textsc{Bronner, Ferdinand} (15.10.1867 – 08.06.1948), \emph{Schriftsteller, Gymnasiallehrer}|pw}}{\lemma{\textnormal{\emph{Familie … Adamus}}}\Cendnote{\textnormal{Ferdinand Bronner\pwindex{Bronner, Ferdinand 15.10.1867 – 08.06.1948@\textsc{Bronner, Ferdinand} (15.10.1867 – 08.06.1948), \emph{Schriftsteller, Gymnasiallehrer}|pwk}s \emph{Familie Wawroch. Ein österreichisches Drama in vier Akten}\pwindex{Familie Wawroch. Ein oesterreichisches Drama in vier Akten1899@\emph{Familie Wawroch. Ein österreichisches Drama in vier Akten} {[}1899{]}|pwk}
                  war 1899 unter dem Pseudonym Franz Adamus\pwindex{Bronner, Ferdinand 15.10.1867 – 08.06.1948@\textsc{Bronner, Ferdinand} (15.10.1867 – 08.06.1948), \emph{Schriftsteller, Gymnasiallehrer}|pwk} erschienen. Eine Lektüre durch Schnitzler\pwindex{Schnitzler, Arthur 15.05.1862 – 21.10.1931@\textsc{Schnitzler, Arthur} (15.05.1862 – 21.10.1931), \emph{Schriftsteller, Mediziner}|pwk} ist nicht nachweisbar.}}}\label{K_L02967-5h}
               ſchicken? (Ich glaub mich zu erinnern dſs Sie ſie haben.) – Die Überſetzungen von \textsc{S. Tr.\pwindex{Trebitsch, Siegfried 22.12.1868 – 03.06.1956@\textsc{Trebitsch, Siegfried} (22.12.1868 – 03.06.1956), \emph{Schriftsteller, Übersetzer}|pw}} find ich ſchlecht. – Das \label{K_L02967-6v}\edtext{raſche
               Abdrucken des neuen \textsc{Maupassant}\pwindex{Oppeln-Bronikowski, Friedrich von 07.04.1873 – 09.10.1936@\textsc{Oppeln-Bronikowski, Friedrich von} (07.04.1873 – 09.10.1936), \emph{Schriftsteller, Übersetzer}!Aus dem Nachlasse von Maupassant. Eine Leidenschaft1899-08-28@\strich\emph{Aus dem Nachlasse von Maupassant. Eine Leidenschaft} {[}Übersetzung, 1899-08-28{]}|pwv}\pwindex{Maupassant, Guy de 05.08.1850 – 07.07.1893@\textsc{Maupassant, Guy de} (05.08.1850 – 07.07.1893), \emph{Schriftsteller}|pw}}{\lemma{\textnormal{\emph{raſche … Maupassant}}}\Cendnote{\textnormal{Guy de Maupassant\pwindex{Maupassant, Guy de 05.08.1850 – 07.07.1893@\textsc{Maupassant, Guy de} (05.08.1850 – 07.07.1893), \emph{Schriftsteller}|pwk}: \emph{Aus dem Nachlasse von Maupassant. Eine Leidenschaft}\pwindex{Oppeln-Bronikowski, Friedrich von 07.04.1873 – 09.10.1936@\textsc{Oppeln-Bronikowski, Friedrich von} (07.04.1873 – 09.10.1936), \emph{Schriftsteller, Übersetzer}!Aus dem Nachlasse von Maupassant. Eine Leidenschaft1899-08-28@\strich\emph{Aus dem Nachlasse von Maupassant. Eine Leidenschaft} {[}Übersetzung, 1899-08-28{]}|pwk}.
                     In: \emph{Wiener Allgemeine Montags-Zeitung}\pwindex{Wiener Allgemeine Montags-Zeitung1899-07-03 – 1899-12-18@\emph{Wiener Allgemeine Montags-Zeitung} {[}1899-07-03 – 1899-12-18{]}|pwk},
                        28. 8. 1899, S. 2–4. Der Text ist der
                     Ende Juli im Verlag \emph{Emil Goldschmidt}\orgindex{Emil Goldschmidt Verlag@Emil Goldschmidt Verlag|pwk} erschienenen Buchausgabe \emph{Vater Milon und andere Erzählungen. Neue Novellen aus dem
                     litterarischen Nachlaß}\pwindex{Oppeln-Bronikowski, Friedrich von 07.04.1873 – 09.10.1936@\textsc{Oppeln-Bronikowski, Friedrich von} (07.04.1873 – 09.10.1936), \emph{Schriftsteller, Übersetzer}!Vater Milon und andere Erzaehlungen. Neue Novellen aus dem litterarischen
                  Nachlass1899-07-31@\strich\emph{Vater Milon und andere Erzählungen. Neue Novellen aus dem litterarischen Nachlaß} {[}Übersetzung, 1899-07-31{]}|pwk} entnommen. Die in der Buchausgabe\pwindex{Oppeln-Bronikowski, Friedrich von 07.04.1873 – 09.10.1936@\textsc{Oppeln-Bronikowski, Friedrich von} (07.04.1873 – 09.10.1936), \emph{Schriftsteller, Übersetzer}!Vater Milon und andere Erzaehlungen. Neue Novellen aus dem litterarischen
                  Nachlass1899-07-31@\strich\emph{Vater Milon und andere Erzählungen. Neue Novellen aus dem litterarischen Nachlaß} {[}Übersetzung, 1899-07-31{]}|pwkv}, nicht aber im Abdruck\pwindex{Oppeln-Bronikowski, Friedrich von 07.04.1873 – 09.10.1936@\textsc{Oppeln-Bronikowski, Friedrich von} (07.04.1873 – 09.10.1936), \emph{Schriftsteller, Übersetzer}!Aus dem Nachlasse von Maupassant. Eine Leidenschaft1899-08-28@\strich\emph{Aus dem Nachlasse von Maupassant. Eine Leidenschaft} {[}Übersetzung, 1899-08-28{]}|pwkv} gezeichnete
                  Übersetzung stammt von Friedrich von
                     Oppeln-Bronikowski\pwindex{Oppeln-Bronikowski, Friedrich von 07.04.1873 – 09.10.1936@\textsc{Oppeln-Bronikowski, Friedrich von} (07.04.1873 – 09.10.1936), \emph{Schriftsteller, Übersetzer}|pwk}.}}}\label{K_L02967-6h} zeigt den rechten Weg auf dieſem Gebiet. –\pend
           \pstart
           Ich bleibe noch bis etwa 10. oder 9.{ }\label{K_L02967-7v}\edtext{hier\oindex{Bad Ischl@\textbf{Bad Ischl}|pwv}}{\lemma{\textnormal{\emph{hier}}}\Cendnote{\textnormal{Schnitzler\pwindex{Schnitzler, Arthur 15.05.1862 – 21.10.1931@\textsc{Schnitzler, Arthur} (15.05.1862 – 21.10.1931), \emph{Schriftsteller, Mediziner}|pwk} reiste am 12. 9. 1899 von Ischl\oindex{Bad Ischl@\textbf{Bad Ischl}|pwk} nach München\oindex{Muenchen@\textbf{München}|pwk} ab. Von dort reiste er am 16. 9. 1899 weiter nach Nürnberg\oindex{Nuernberg@\textbf{Nürnberg}|pwk}, dann am 19. 9. 1899 weiter nach Frankfurt am Main\oindex{Frankfurt am Main@\textbf{Frankfurt am Main}|pwk} und am 24. 9. 1899 nach Wiesbaden\oindex{Wiesbaden@\textbf{Wiesbaden}|pwk}. Zwischen 4. 10. 1899 und 11. 10. 1899 war er in Berlin\oindex{Berlin@\textbf{Berlin}|pwk}. Am
                     12. 10. 1899
                  kehrte er nach Wien\oindex{Wien@\textbf{Wien}|pwk} zurück.}}}\label{K_L02967-7h}. Dann {\pb}vorerſt München\oindex{Muenchen@\textbf{München}|pw}, dann? – 20, 22. werd ich in Berlin\oindex{Berlin@\textbf{Berlin}|pw} ſein.
                  Wahrſcheinli{[}ch{]} iſt mein \label{K_L02967-8v}\edtext{Stück\pwindex{Schnitzler, Arthur 15.05.1862 – 21.10.1931@\textsc{Schnitzler, Arthur} (15.05.1862 – 21.10.1931), \emph{Schriftsteller, Mediziner}!Schleier der Beatrice. Schauspiel in fuenf Akten1900-12-01@\strich\emph{Der Schleier der Beatrice. Schauspiel in fünf Akten} {[}1900-12-01{]}|pwv}}{\lemma{\textnormal{\emph{Stück}}}\Cendnote{\textnormal{Schnitzler\pwindex{Schnitzler, Arthur 15.05.1862 – 21.10.1931@\textsc{Schnitzler, Arthur} (15.05.1862 – 21.10.1931), \emph{Schriftsteller, Mediziner}|pwk} schloss \emph{Der Schleier der Beatrice}\pwindex{Schnitzler, Arthur 15.05.1862 – 21.10.1931@\textsc{Schnitzler, Arthur} (15.05.1862 – 21.10.1931), \emph{Schriftsteller, Mediziner}!Schleier der Beatrice. Schauspiel in fuenf Akten1900-12-01@\strich\emph{Der Schleier der Beatrice. Schauspiel in fünf Akten} {[}1900-12-01{]}|pwk} am 9. 9. 1899 vorläufig
                  ab.}}}\label{K_L02967-8h} bis dahin fertig. Die Führung und mancherlei ausgeſprochnes dürfte gut
               ſein; doch fühl ich oft, wie die Kraft des Ausdrucks {\pb}aus dem Gehirn (denn da ſcheint ſie mir zu
               ſein) nicht in den Bleiſtift will. –\pend
           \pstart
           Arbeiten bleibt endlich doch das einzige. Sonſt iſts im Weſentlichen i{\geminationm}er gleich traurig. – Auch Hugo\pwindex{Hofmannsthal, Hugo von 1874-02-01 – 1929-07-15@\textsc{Hofmannsthal, Hugo von} (1874-02-01 – 1929-07-15), \emph{Schriftsteller}|pw} arbeitet {\pb}hier\oindex{Bad Ischl@\textbf{Bad Ischl}|pwv} an einem \label{K_L02967-9v}\edtext{neuen Stück}{\lemma{\textnormal{\emph{neuen Stück}}}\Cendnote{\textnormal{Am 31. 8. 1899 hatte Hugo von
                     Hofmannsthal\pwindex{Hofmannsthal, Hugo von 1874-02-01 – 1929-07-15@\textsc{Hofmannsthal, Hugo von} (1874-02-01 – 1929-07-15), \emph{Schriftsteller}|pwk}{ }Schnitzler\pwindex{Schnitzler, Arthur 15.05.1862 – 21.10.1931@\textsc{Schnitzler, Arthur} (15.05.1862 – 21.10.1931), \emph{Schriftsteller, Mediziner}|pwk} bereits zwei Akte aus \emph{Das Bergwerk zu Falun}\pwindex{Hofmannsthal, Hugo von 1874-02-01 – 1929-07-15@\textsc{Hofmannsthal, Hugo von} (1874-02-01 – 1929-07-15), \emph{Schriftsteller}!Bergwerk zu Falun1900 – 1933@\strich\emph{Das Bergwerk zu Falun} {[}1900 – 1933{]}|pwk} vorgelesen.}}}\label{K_L02967-9h} (Bergwerk von \textsc{Falun}\pwindex{Hofmannsthal, Hugo von 1874-02-01 – 1929-07-15@\textsc{Hofmannsthal, Hugo von} (1874-02-01 – 1929-07-15), \emph{Schriftsteller}!Bergwerk zu Falun1900 – 1933@\strich\emph{Das Bergwerk zu Falun} {[}1900 – 1933{]}|pw} – Sie wiſſens ja ſchon.) Auch
               ihm hat Flucht\pwindex{Salten, Felix 06.09.1869 – 08.10.1945@\textsc{Salten, Felix} (06.09.1869 – 08.10.1945), \emph{Schriftsteller, Journalist}!Flucht1899-07-31@\strich\emph{Flucht} {[}1899-07-31{]}|pw} gut gefallen \strikeout{(} (das andre\pwindex{Salten, Felix 06.09.1869 – 08.10.1945@\textsc{Salten, Felix} (06.09.1869 – 08.10.1945), \emph{Schriftsteller, Journalist}!Manhard-Zimmer1899-08-21@\strich\emph{Das Manhard-Zimmer} {[}1899-08-21{]}|pwv}\pwindex{Salten, Felix 06.09.1869 – 08.10.1945@\textsc{Salten, Felix} (06.09.1869 – 08.10.1945), \emph{Schriftsteller, Journalist}!Sedan1899-07-03@\strich\emph{Sedan} {[}1899-07-03{]}|pwv} hat er noch nicht geleſen.) –\pend
           \pstart
           Heute traf ich Frau \textsc{Ida {\pb}F.\pwindex{Falk, Ida 1862? – 1926?@\textsc{Falk, Ida} (1862? – 1926?)|pw}} – \label{K_L02967-10v}\edtext{Verlobt}{\lemma{\textnormal{\emph{Verlobt}}}\Cendnote{\textnormal{Ida Falk\pwindex{Falk, Ida 1862? – 1926?@\textsc{Falk, Ida} (1862? – 1926?)|pwk}, ehemalige Geliebte sowohl von Schnitzler\pwindex{Schnitzler, Arthur 15.05.1862 – 21.10.1931@\textsc{Schnitzler, Arthur} (15.05.1862 – 21.10.1931), \emph{Schriftsteller, Mediziner}|pwk} als auch von Salten\pwindex{Salten, Felix 06.09.1869 – 08.10.1945@\textsc{Salten, Felix} (06.09.1869 – 08.10.1945), \emph{Schriftsteller, Journalist}|pwk}, hatte sich mit Ludwig Grann\pwindex{Grann, Ludwig 1860/1861 – 1935-12-23@\textsc{Grann, Ludwig} (1860/1861 – 1935-12-23), \emph{Anwalt}|pwk} verlobt, vgl. A. S.: \emph{Tagebuch}, 23. 10. 1899.}}}\label{K_L02967-10h}. \substVorne{}\textsuperscript{\textcolor{gray}{×}\-\textcolor{gray}{×}\-\textcolor{gray}{×}\-\textcolor{gray}{×}\-\textcolor{gray}{×}\-\textcolor{gray}{×}{ }\textcolor{gray}{×}\-\textcolor{gray}{×}\-\textcolor{gray}{×}\-\textcolor{gray}{×}}\substDazwischen{}Man ſoll nie Namen ſchreiben\substHinten{}. – Komiſcherweiſe iſt \introOben{}hier\introOben{} eine vorübergehende
               Verbindg zwiſchen mir und einer abſoluten Wiederholung jenes Typus eingetreten. –\pend
           \pstart
           Herzlichſt Ihr {\\[\baselineskip]}\spacefill\mbox{A. S.}\pend
           \leftskip=0em{}
         
         \endnumbering\mylabel{h}\end{ledgroupsized}  \newcommand{\dateiname}{L02967}\newcommand{\titel}{Arthur Schnitzler an Felix Salten, 4. 9. 1899}\newcommand{\editorInnen}{Martin Anton Müller und Laura Untner}%% latex-leseansicht-abspann.tex
%% Abspann für die Leseansicht.
%% Der Schalter \ifkorrekturansicht ist bereits durch den Vorspann gesetzt.

%% latex-abspann.tex
%% Gemeinsamer Abspann für Korrekturansicht und Leseansicht.
%% Setzt den Schalter \ifkorrekturansicht voraus (gesetzt in den
%% einbindenden Dateien latex-korrekturansicht-abspann.tex bzw.
%% latex-leseansicht-abspann.tex).
%% ---------------------------------------------------------------

\normalsize

% Das esempio-Environment wird nur in der Leseansicht benötigt
\ifkorrekturansicht\else
\newenvironment{esempio}[3]%
{
    \vspace{1.5ex}
    \rlap{\underline{#1}}
    \par
    \setlength{\parindent}{0cm}
    \nopagebreak
    \leftskip=#2cm
    \rightskip=#3cm
}
{
    \par
}
\fi

\doendnotes{C}
\bigskip
\vfill

\clearpage

\footnotesize

\ifkorrekturansicht
  \lohead{\textsc{register}}
\fi

% theindex-Environment neu definieren ohne reledmac
\makeatletter
\renewenvironment{theindex}{%
  \ifkorrekturansicht
    \section*{\indexname}%
  \else
    \subsubsection*{Index der erwähnten Entitäten}%
  \fi
  \setlength{\parindent}{0pt}%
  \setlength{\parskip}{0pt plus 0.3pt}%
  \let\item\@idxitem
}{%
  \ifkorrekturansicht\clearpage\fi
}
\makeatother

\IfFileExists{\jobname-pw.ind}{\input{\jobname-pw.ind}}{}

% Quellenangabe nur in der Leseansicht
\ifkorrekturansicht\else
% Fallback-Definitionen, falls die .tex-Datei \titel etc. nicht gesetzt hat
\providecommand{\titel}{}
\providecommand{\editorInnen}{}
\providecommand{\dateiname}{\jobname}

\vspace{3cm}

\vfill

\footnotesize
\textsc{Quelle}: \titel. Herausgegeben von {\editorInnen}. In: \emph{Arthur Schnitzler: Briefwechsel mit Autorinnen und Autoren}.
 Digitale Edition, https://schnitzler-briefe.acdh.oeaw.ac.at/{\dateiname}.html (Stand \today)
\fi

\end{document}


      