%% latex-leseansicht-vorspann.tex
%% Vorspann für die Leseansicht.
%% Lädt die gemeinsame Datei latex-vorspann.tex mit nicht gesetztem Schalter.

\newif\ifkorrekturansicht
\korrekturansichtfalse

\input{../tex-inputs/latex-vorspann}


\section[ Arthur Schnitzler an Felix Salten, 4. 9. 1899]{L02967 Arthur Schnitzler an Felix Salten,  4. 9. 1899}
\nopagebreak\mylabel{L02967v}
\rehead{ }\normalsize\beginnumbering\briefempfaengerindex{Salten, Felix@\textsc{Salten, Felix}!zzzSchnitzler, Arthur@\emph{von Arthur Schnitzler}!1899-09-041@{4. 9. 1899}|(be}
\toendnotes[C]{\smallbreak\pagebreak[2]}
\correspDesc{Versand  durch Arthur Schnitzler am 4. 9. 1899 in Bad Ischl
\newline{}Erhalt  durch Felix Salten im Zeitraum [5. 9. 1899
                  – 9. 9. 1899?] in Wien}\toendnotes[C]{\smallbreak}
\Standort{Wienbibliothek im Rathaus, ZPH 1681, 2.1.516.}
\physDesc{Brief, 2 Blätter, 8 Seiten, 1693 Zeichen
\newline{}Handschrift: Bleistift, deutsche Kurrent
\newline{}Ordnung: mit Bleistift von unbekannter Hand Nummerierung der Doppelseiten des
                                 Konvoluts: »69«–»72« }
\buchAbdrucke{\weitereDrucke{Arthur Schnitzler: \emph{Briefe 1875–1912}. Herausgegeben von Therese Nickl und Heinrich Schnitzler. Frankfurt am Main: \emph{S. Fischer} 1981, S. 375–376.} }\toendnotes[C]{\smallbreak}
\pstart
           \raggedleft{}{\pb}Ischl, Rudolfshöhe\oindex{Hotel und Pension Rudolfshöhe (Leopold Petter)@\textbf{Hotel und Pension Rudolfshöhe (Leopold Petter)}, \emph{Hotel}|pw}{ }4/9 99.\pend
           \vspace{0.5em}
\pstart
           lieber Freund, ich will Ihnen vor allem{ }ſagen, dſs mir nicht nur
                  »Flucht\pwindex{Salten, Felix 6.\,9.\,1869 Budapest – 8.\,10.\,1945 Zürich@\textsc{Salten, Felix} (6.\,9.\,1869 Budapest – 8.\,10.\,1945 Zürich), \emph{Schriftsteller, Journalist, Chefredakteur}!Flucht@\strich\emph{Flucht}|pw}«,{ }ſondern auch das \textsc{Manhard}zi{\geminationm}er\pwindex{Salten, Felix 6.\,9.\,1869 Budapest – 8.\,10.\,1945 Zürich@\textsc{Salten, Felix} (6.\,9.\,1869 Budapest – 8.\,10.\,1945 Zürich), \emph{Schriftsteller, Journalist, Chefredakteur}!Manhard-Zimmer@\strich\emph{Das Manhard-Zimmer}|pw} noch
               beſſer \label{K_L02967-1v}\edtext{gefallen}{\lemma{\textnormal{\emph{gefallen}}}\Cendnote{\textnormal{Siehe XXXX Auszeichnungsfehler: Dokument L03299 nicht gefunden.
               }}}\label{K_L02967-1} haben, als nach dem erſten Leſen. Ich zweifle nicht, dſs Ihre Novelletten\pwindex{Salten, Felix 6.\,9.\,1869 Budapest – 8.\,10.\,1945 Zürich@\textsc{Salten, Felix} (6.\,9.\,1869 Budapest – 8.\,10.\,1945 Zürich), \emph{Schriftsteller, Journalist, Chefredakteur}!Flucht@\strich\emph{Flucht}|pwv}\pwindex{Salten, Felix 6.\,9.\,1869 Budapest – 8.\,10.\,1945 Zürich@\textsc{Salten, Felix} (6.\,9.\,1869 Budapest – 8.\,10.\,1945 Zürich), \emph{Schriftsteller, Journalist, Chefredakteur}!Fernen@\strich\emph{Fernen}|pwv}\pwindex{Salten, Felix 6.\,9.\,1869 Budapest – 8.\,10.\,1945 Zürich@\textsc{Salten, Felix} (6.\,9.\,1869 Budapest – 8.\,10.\,1945 Zürich), \emph{Schriftsteller, Journalist, Chefredakteur}!Sedan@\strich\emph{Sedan}|pwv}\pwindex{Salten, Felix 6.\,9.\,1869 Budapest – 8.\,10.\,1945 Zürich@\textsc{Salten, Felix} (6.\,9.\,1869 Budapest – 8.\,10.\,1945 Zürich), \emph{Schriftsteller, Journalist, Chefredakteur}!Lebenszeit@\strich\emph{Lebenszeit}|pwv}\pwindex{Salten, Felix 6.\,9.\,1869 Budapest – 8.\,10.\,1945 Zürich@\textsc{Salten, Felix} (6.\,9.\,1869 Budapest – 8.\,10.\,1945 Zürich), \emph{Schriftsteller, Journalist, Chefredakteur}!Hinterbliebene@\strich\emph{Der Hinterbliebene}|pwv}\pwindex{Salten, Felix 6.\,9.\,1869 Budapest – 8.\,10.\,1945 Zürich@\textsc{Salten, Felix} (6.\,9.\,1869 Budapest – 8.\,10.\,1945 Zürich), \emph{Schriftsteller, Journalist, Chefredakteur}!Manhard-Zimmer@\strich\emph{Das Manhard-Zimmer}|pwv}\pwindex{Salten, Felix 6.\,9.\,1869 Budapest – 8.\,10.\,1945 Zürich@\textsc{Salten, Felix} (6.\,9.\,1869 Budapest – 8.\,10.\,1945 Zürich), \emph{Schriftsteller, Journalist, Chefredakteur}!Begräbnis@\strich\emph{Begräbnis}|pwv}\pwindex{Salten, Felix 6.\,9.\,1869 Budapest – 8.\,10.\,1945 Zürich@\textsc{Salten, Felix} (6.\,9.\,1869 Budapest – 8.\,10.\,1945 Zürich), \emph{Schriftsteller, Journalist, Chefredakteur}!Heldentod@\strich\emph{Heldentod}|pwv} ein hübſches { }Buch\pwindex{Salten, Felix 6.\,9.\,1869 Budapest – 8.\,10.\,1945 Zürich@\textsc{Salten, Felix} (6.\,9.\,1869 Budapest – 8.\,10.\,1945 Zürich), \emph{Schriftsteller, Journalist, Chefredakteur}!Hinterbliebene. Kurze Novellen@\strich\emph{Der Hinterbliebene. Kurze Novellen}|pwv} gäben, möchte aber von
               einem entgiltigen {\pb}Urtheil über die Wirkung
               als ganzes, \uline{alle} Sachen auf einmal, womöglich in der
               von Ihnen gewählten Reihenfolge leſen. Herausgeben unbedingt,{ }ſag ich{ }ſchon heute,
               und womöglich zugleich mit dem \label{K_L02967-2v}\edtext{Stück\pwindex{Salten, Felix 6.\,9.\,1869 Budapest – 8.\,10.\,1945 Zürich@\textsc{Salten, Felix} (6.\,9.\,1869 Budapest – 8.\,10.\,1945 Zürich), \emph{Schriftsteller, Journalist, Chefredakteur}!Schöne Seelen. Komödie in einem Akt@\strich\emph{Schöne Seelen. Komödie in einem Akt}|pwv}}{\lemma{\textnormal{\emph{Stück}}}\Cendnote{\textnormal{Siehe XXXX Auszeichnungsfehler: Dokument L03301 nicht gefunden.
               }}}\label{K_L02967-2} herauskommen. – In der {\pb}\label{K_L02967-3v}\edtext{Zeitung\pwindex{Wiener Allgemeine Montags-Zeitung@\emph{Wiener Allgemeine Montags-Zeitung}|pwv}}{\lemma{\textnormal{\emph{Zeitung}}}\Cendnote{\textnormal{Von der ersten Ausgabe an, die am 3. 7. 1899 erschienen war, betreute Salten\pwindex{Salten, Felix 6.\,9.\,1869 Budapest – 8.\,10.\,1945 Zürich@\textsc{Salten, Felix} (6.\,9.\,1869 Budapest – 8.\,10.\,1945 Zürich), \emph{Schriftsteller, Journalist, Chefredakteur}|pwk} die Rubrik »\emph{Wiener
                     Allgemeine Rundschau}\pwindex{Wiener Allgemeine Rundschau@\emph{Wiener Allgemeine Rundschau}|pwk}« der wöchentlich erscheinenenden \emph{Wiener Allgemeinen Montags-Zeitung}\pwindex{Wiener Allgemeine Montags-Zeitung@\emph{Wiener Allgemeine Montags-Zeitung}|pwk}. Das Blatt\pwindex{Wiener Allgemeine Montags-Zeitung@\emph{Wiener Allgemeine Montags-Zeitung}|pwkv} wurde Mitte Dezember 1899 eingestellt.}}}\label{K_L02967-3} findet{ }ſich viel leſenswerthes;
               natürlich iſt es Ihnen aus Gründen, die nicht in Ihnen liegen, unmöglich, das
               Anſtrebenswerthe daraus zu machen. Glänzend hab ich Ihre \label{K_L02967-4v}\edtext{Goethe\pwindex{Goethe, Johann Wolfgang von 28.\,8.\,1749 Frankfurt am Main – 22.\,3.\,1832 Weimar@\textsc{Goethe, Johann Wolfgang von} (28.\,8.\,1749 Frankfurt am Main – 22.\,3.\,1832 Weimar), \emph{Schriftsteller}|pw}späße\pwindex{Eine Goethe-Enquête@\emph{Eine Goethe-Enquête}|pwv}}{\lemma{\textnormal{\emph{Goethespäße}}}\Cendnote{\textnormal{[Felix Salten\pwindex{Salten, Felix 6.\,9.\,1869 Budapest – 8.\,10.\,1945 Zürich@\textsc{Salten, Felix} (6.\,9.\,1869 Budapest – 8.\,10.\,1945 Zürich), \emph{Schriftsteller, Journalist, Chefredakteur}|pwk}]: \emph{Eine Goethe-Enquête}\pwindex{Eine Goethe-Enquête@\emph{Eine Goethe-Enquête}|pwk}. In: \emph{Wiener Allgemeine Montags-Zeitung}\pwindex{Wiener Allgemeine Montags-Zeitung@\emph{Wiener Allgemeine Montags-Zeitung}|pwk}, 28. 8. 1899, S. 3. Die nicht gezeichnete Umfrage\pwindex{Eine Goethe-Enquête@\emph{Eine Goethe-Enquête}|pwkv} ist deutlich als
                  Satire erkennbar (»indem wir eine Anzahl hervorragender Persönlichkeiten im
                     Geiste um ihre Meinung befragt«) und bringt erfundene Aussagen
                  von 17 Prominenten zu Goethe\pwindex{Goethe, Johann Wolfgang von 28.\,8.\,1749 Frankfurt am Main – 22.\,3.\,1832 Weimar@\textsc{Goethe, Johann Wolfgang von} (28.\,8.\,1749 Frankfurt am Main – 22.\,3.\,1832 Weimar), \emph{Schriftsteller}|pwk}.}}}\label{K_L02967-4}
               gefunden. Können Sie mir die \label{K_L02967-5v}\edtext{Familie{ }{\pb}\textsc{Wawroch}\pwindex{Bronner, Ferdinand 15.\,10.\,1867 Oświęcim – 8.\,6.\,1948 Bad Ischl@\textsc{Bronner, Ferdinand} (15.\,10.\,1867 Oświęcim – 8.\,6.\,1948 Bad Ischl), \emph{Schriftsteller, Gymnasiallehrer}!Familie Wawroch. Ein österreichisches Drama in vier Akten@\strich\emph{Familie Wawroch. Ein österreichisches Drama in vier Akten}|pw} von Adamus\pwindex{Bronner, Ferdinand 15.\,10.\,1867 Oświęcim – 8.\,6.\,1948 Bad Ischl@\textsc{Bronner, Ferdinand} (15.\,10.\,1867 Oświęcim – 8.\,6.\,1948 Bad Ischl), \emph{Schriftsteller, Gymnasiallehrer}|pw}}{\lemma{\textnormal{\emph{Familie … Adamus}}}\Cendnote{\textnormal{Ferdinand Bronners\pwindex{Bronner, Ferdinand 15.\,10.\,1867 Oświęcim – 8.\,6.\,1948 Bad Ischl@\textsc{Bronner, Ferdinand} (15.\,10.\,1867 Oświęcim – 8.\,6.\,1948 Bad Ischl), \emph{Schriftsteller, Gymnasiallehrer}|pwk}{ }\emph{Familie Wawroch. Ein österreichisches Drama in vier Akten}\pwindex{Bronner, Ferdinand 15.\,10.\,1867 Oświęcim – 8.\,6.\,1948 Bad Ischl@\textsc{Bronner, Ferdinand} (15.\,10.\,1867 Oświęcim – 8.\,6.\,1948 Bad Ischl), \emph{Schriftsteller, Gymnasiallehrer}!Familie Wawroch. Ein österreichisches Drama in vier Akten@\strich\emph{Familie Wawroch. Ein österreichisches Drama in vier Akten}|pwk}
                  war 1899 unter dem Pseudonym Franz Adamus\pwindex{Bronner, Ferdinand 15.\,10.\,1867 Oświęcim – 8.\,6.\,1948 Bad Ischl@\textsc{Bronner, Ferdinand} (15.\,10.\,1867 Oświęcim – 8.\,6.\,1948 Bad Ischl), \emph{Schriftsteller, Gymnasiallehrer}|pwk} erschienen. Eine Lektüre durch Schnitzler ist nicht nachweisbar.}}}\label{K_L02967-5}{ }ſchicken? (Ich glaub mich zu erinnern dſs Sie{ }ſie haben.) – Die Überſetzungen von \textsc{S. Tr.\pwindex{Trebitsch, Siegfried 22.\,12.\,1868 Wien – 3.\,6.\,1956 Zürich@\textsc{Trebitsch, Siegfried} (22.\,12.\,1868 Wien – 3.\,6.\,1956 Zürich), \emph{Schriftsteller, Übersetzer}|pw}} find ich{ }ſchlecht. – Das \label{K_L02967-6v}\edtext{raſche
               Abdrucken des neuen \textsc{Maupassant}\pwindex{Maupassant, Guy de 5.\,8.\,1850 Tourville-sur-Arques – 7.\,7.\,1893 Paris@\textsc{Maupassant, Guy de} (5.\,8.\,1850 Tourville-sur-Arques – 7.\,7.\,1893 Paris), \emph{Schriftsteller}!Aus dem Nachlasse von Maupassant. Eine Leidenschaft@\strich\emph{Aus dem Nachlasse von Maupassant. Eine Leidenschaft}|pwv}\pwindex{Maupassant, Guy de 5.\,8.\,1850 Tourville-sur-Arques – 7.\,7.\,1893 Paris@\textsc{Maupassant, Guy de} (5.\,8.\,1850 Tourville-sur-Arques – 7.\,7.\,1893 Paris), \emph{Schriftsteller}|pw}}{\lemma{\textnormal{\emph{rasche … Maupassant}}}\Cendnote{\textnormal{Guy de Maupassant\pwindex{Maupassant, Guy de 5.\,8.\,1850 Tourville-sur-Arques – 7.\,7.\,1893 Paris@\textsc{Maupassant, Guy de} (5.\,8.\,1850 Tourville-sur-Arques – 7.\,7.\,1893 Paris), \emph{Schriftsteller}|pwk}: \emph{Aus dem Nachlasse von Maupassant. Eine Leidenschaft}\pwindex{Maupassant, Guy de 5.\,8.\,1850 Tourville-sur-Arques – 7.\,7.\,1893 Paris@\textsc{Maupassant, Guy de} (5.\,8.\,1850 Tourville-sur-Arques – 7.\,7.\,1893 Paris), \emph{Schriftsteller}!Aus dem Nachlasse von Maupassant. Eine Leidenschaft@\strich\emph{Aus dem Nachlasse von Maupassant. Eine Leidenschaft}|pwk}.
                     In: \emph{Wiener Allgemeine Montags-Zeitung}\pwindex{Wiener Allgemeine Montags-Zeitung@\emph{Wiener Allgemeine Montags-Zeitung}|pwk},
                        28. 8. 1899, S. 2–4. Der Text ist der
                     Ende Juli im Verlag \emph{Emil Goldschmidt}\orgindex{Emil Goldschmidt Verlag@Emil Goldschmidt Verlag|pwk} erschienenen Buchausgabe \emph{Vater Milon und andere Erzählungen. Neue Novellen aus dem
                     litterarischen Nachlaß}\pwindex{Maupassant, Guy de 5.\,8.\,1850 Tourville-sur-Arques – 7.\,7.\,1893 Paris@\textsc{Maupassant, Guy de} (5.\,8.\,1850 Tourville-sur-Arques – 7.\,7.\,1893 Paris), \emph{Schriftsteller}!Vater Milon und andere Erzählungen. Neue Novellen aus dem litterarischen Nachlaß@\strich\emph{Vater Milon und andere Erzählungen. Neue Novellen aus dem litterarischen Nachlaß}|pwk} entnommen. Die in der Buchausgabe\pwindex{Maupassant, Guy de 5.\,8.\,1850 Tourville-sur-Arques – 7.\,7.\,1893 Paris@\textsc{Maupassant, Guy de} (5.\,8.\,1850 Tourville-sur-Arques – 7.\,7.\,1893 Paris), \emph{Schriftsteller}!Vater Milon und andere Erzählungen. Neue Novellen aus dem litterarischen Nachlaß@\strich\emph{Vater Milon und andere Erzählungen. Neue Novellen aus dem litterarischen Nachlaß}|pwkv}, nicht aber im Abdruck\pwindex{Maupassant, Guy de 5.\,8.\,1850 Tourville-sur-Arques – 7.\,7.\,1893 Paris@\textsc{Maupassant, Guy de} (5.\,8.\,1850 Tourville-sur-Arques – 7.\,7.\,1893 Paris), \emph{Schriftsteller}!Aus dem Nachlasse von Maupassant. Eine Leidenschaft@\strich\emph{Aus dem Nachlasse von Maupassant. Eine Leidenschaft}|pwkv} gezeichnete
                  Übersetzung stammt von Friedrich von
                     Oppeln-Bronikowski\pwindex{Oppeln-Bronikowski, Friedrich von 7.\,4.\,1873 Kassel – 9.\,10.\,1936 Berlin@\textsc{Oppeln-Bronikowski, Friedrich von} (7.\,4.\,1873 Kassel – 9.\,10.\,1936 Berlin), \emph{Schriftsteller, Übersetzer}|pwk}.}}}\label{K_L02967-6} zeigt den rechten Weg auf dieſem Gebiet. –\pend
           
\pstart
           Ich bleibe noch bis etwa 10. oder 9.{ }\label{K_L02967-7v}\edtext{hier\oindex{Bad Ischl@\textbf{Bad Ischl}|pwv}}{\lemma{\textnormal{\emph{hier}}}\Cendnote{\textnormal{Schnitzler reiste am 12. 9. 1899 von Ischl\oindex{Bad Ischl@\textbf{Bad Ischl}|pwk} nach München\oindex{München@\textbf{München}|pwk} ab. Von dort reiste er am 16. 9. 1899 weiter nach Nürnberg\oindex{Nürnberg@\textbf{Nürnberg}|pwk}, dann am 19. 9. 1899 weiter nach Frankfurt am Main\oindex{Frankfurt am Main@\textbf{Frankfurt am Main}, \emph{Hauptstadt}|pwk} und am 24. 9. 1899 nach Wiesbaden\oindex{Wiesbaden@\textbf{Wiesbaden}|pwk}. Zwischen 4. 10. 1899 und 11. 10. 1899 war er in Berlin\oindex{Berlin@\textbf{Berlin}, \emph{Hauptstadt}|pwk}. Am
                     12. 10. 1899
                  kehrte er nach Wien\oindex{Wien@\textbf{Wien}, \emph{Verwaltungsgebiet}|pwk} zurück.}}}\label{K_L02967-7}. Dann {\pb}vorerſt München\oindex{München@\textbf{München}|pw}, dann? – 20, 22. werd ich in Berlin\oindex{Berlin@\textbf{Berlin}, \emph{Hauptstadt}|pw}{ }ſein.
                  Wahrſcheinli{[}ch{]} iſt mein \label{K_L02967-8v}\edtext{Stück\pwindex{Schnitzler, Arthur 15.\,5.\,1862 Wien – 21.\,10.\,1931 ebd.@\textsc{Schnitzler, Arthur} (15.\,5.\,1862 Wien – 21.\,10.\,1931 ebd.), \emph{Schriftsteller, Mediziner}!Schleier der Beatrice. Schauspiel in fünf Akten@\strich\emph{Der Schleier der Beatrice. Schauspiel in fünf Akten}|pwv}}{\lemma{\textnormal{\emph{Stück}}}\Cendnote{\textnormal{Schnitzler schloss \emph{Der Schleier der Beatrice}\pwindex{Schnitzler, Arthur 15.\,5.\,1862 Wien – 21.\,10.\,1931 ebd.@\textsc{Schnitzler, Arthur} (15.\,5.\,1862 Wien – 21.\,10.\,1931 ebd.), \emph{Schriftsteller, Mediziner}!Schleier der Beatrice. Schauspiel in fünf Akten@\strich\emph{Der Schleier der Beatrice. Schauspiel in fünf Akten}|pwk} am 9. 9. 1899 vorläufig
                  ab.}}}\label{K_L02967-8} bis dahin fertig. Die Führung und mancherlei ausgeſprochnes dürfte gut{ }ſein; doch fühl ich oft, wie die Kraft des Ausdrucks {\pb}aus dem Gehirn (denn da{ }ſcheint{ }ſie mir zu{ }ſein) nicht in den Bleiſtift will. –\pend
           
\pstart
           Arbeiten bleibt endlich doch das einzige. Sonſt iſts im Weſentlichen i{\geminationm}er gleich traurig. – Auch Hugo\pwindex{Hofmannsthal, Hugo von 1.\,2.\,1874 Wien – 15.\,7.\,1929 Rodaun@\textsc{Hofmannsthal, Hugo von} (1.\,2.\,1874 Wien – 15.\,7.\,1929 Rodaun), \emph{Schriftsteller}|pw} arbeitet {\pb}hier\oindex{Bad Ischl@\textbf{Bad Ischl}|pwv} an einem \label{K_L02967-9v}\edtext{neuen Stück}{\lemma{\textnormal{\emph{neuen Stück}}}\Cendnote{\textnormal{Am 31. 8. 1899 hatte Hugo von
                     Hofmannsthal\pwindex{Hofmannsthal, Hugo von 1.\,2.\,1874 Wien – 15.\,7.\,1929 Rodaun@\textsc{Hofmannsthal, Hugo von} (1.\,2.\,1874 Wien – 15.\,7.\,1929 Rodaun), \emph{Schriftsteller}|pwk}{ }Schnitzler bereits zwei Akte aus \emph{Das Bergwerk zu Falun}\pwindex{Hofmannsthal, Hugo von 1.\,2.\,1874 Wien – 15.\,7.\,1929 Rodaun@\textsc{Hofmannsthal, Hugo von} (1.\,2.\,1874 Wien – 15.\,7.\,1929 Rodaun), \emph{Schriftsteller}!Bergwerk zu Falun@\strich\emph{Das Bergwerk zu Falun}|pwk} vorgelesen.}}}\label{K_L02967-9} (Bergwerk von \textsc{Falun}\pwindex{Hofmannsthal, Hugo von 1.\,2.\,1874 Wien – 15.\,7.\,1929 Rodaun@\textsc{Hofmannsthal, Hugo von} (1.\,2.\,1874 Wien – 15.\,7.\,1929 Rodaun), \emph{Schriftsteller}!Bergwerk zu Falun@\strich\emph{Das Bergwerk zu Falun}|pw} – Sie wiſſens ja{ }ſchon.) Auch
               ihm hat Flucht\pwindex{Salten, Felix 6.\,9.\,1869 Budapest – 8.\,10.\,1945 Zürich@\textsc{Salten, Felix} (6.\,9.\,1869 Budapest – 8.\,10.\,1945 Zürich), \emph{Schriftsteller, Journalist, Chefredakteur}!Flucht@\strich\emph{Flucht}|pw} gut gefallen \strikeout{(} (das andre\pwindex{Salten, Felix 6.\,9.\,1869 Budapest – 8.\,10.\,1945 Zürich@\textsc{Salten, Felix} (6.\,9.\,1869 Budapest – 8.\,10.\,1945 Zürich), \emph{Schriftsteller, Journalist, Chefredakteur}!Manhard-Zimmer@\strich\emph{Das Manhard-Zimmer}|pwv}\pwindex{Salten, Felix 6.\,9.\,1869 Budapest – 8.\,10.\,1945 Zürich@\textsc{Salten, Felix} (6.\,9.\,1869 Budapest – 8.\,10.\,1945 Zürich), \emph{Schriftsteller, Journalist, Chefredakteur}!Sedan@\strich\emph{Sedan}|pwv} hat er noch nicht geleſen.) –\pend
           
\pstart
           Heute traf ich Frau \textsc{Ida {\pb}F.\pwindex{Falk, Ida 1862? – 1926?@\textsc{Falk, Ida} (1862? – 1926?)|pw}} – \label{K_L02967-10v}\edtext{Verlobt}{\lemma{\textnormal{\emph{Verlobt}}}\Cendnote{\textnormal{Ida Falk\pwindex{Falk, Ida 1862? – 1926?@\textsc{Falk, Ida} (1862? – 1926?)|pwk}, ehemalige Geliebte sowohl von Schnitzler als auch von Salten\pwindex{Salten, Felix 6.\,9.\,1869 Budapest – 8.\,10.\,1945 Zürich@\textsc{Salten, Felix} (6.\,9.\,1869 Budapest – 8.\,10.\,1945 Zürich), \emph{Schriftsteller, Journalist, Chefredakteur}|pwk}, hatte sich mit Ludwig Grann\pwindex{Grann, Ludwig 2.\,3.\,1860 Wien – 23.\,12.\,1935 ebd.@\textsc{Grann, Ludwig} (2.\,3.\,1860 Wien – 23.\,12.\,1935 ebd.), \emph{Anwalt}|pwk} verlobt, vgl. A. S.: \emph{Tagebuch}, 23. 10. 1899.}}}\label{K_L02967-10}. \substVorne{}\textsuperscript{\textcolor{gray}{×}\-\textcolor{gray}{×}\-\textcolor{gray}{×}\-\textcolor{gray}{×}\-\textcolor{gray}{×}\-\textcolor{gray}{×}{ }\textcolor{gray}{×}\-\textcolor{gray}{×}\-\textcolor{gray}{×}\-\textcolor{gray}{×}}\substDazwischen{}Man{ }ſoll nie Namen{ }ſchreiben\substHinten{}. – Komiſcherweiſe iſt \introOben{}hier\introOben{} eine vorübergehende
               Verbindg zwiſchen mir und einer abſoluten Wiederholung jenes Typus eingetreten. –\pend
           
\pstart
           Herzlichſt Ihr {\\[\baselineskip]}\spacefill\mbox{A. S.}\pend
           \leftskip=0em{}\selectlanguage{ngerman}\endnumbering\briefempfaengerindex{Salten, Felix@\textsc{Salten, Felix}!zzzSchnitzler, Arthur@\emph{von Arthur Schnitzler}!1899-09-041@{4. 9. 1899}|)be}\mylabel{L02967h}  \newcommand{\dateiname}{L02967}\newcommand{\titel}{Arthur Schnitzler an Felix Salten, 4. 9. 1899}\newcommand{\editorInnen}{Martin Anton Müller und Laura Untner}%% latex-leseansicht-abspann.tex
%% Abspann für die Leseansicht.
%% Der Schalter \ifkorrekturansicht ist bereits durch den Vorspann gesetzt.

%% latex-abspann.tex
%% Gemeinsamer Abspann für Korrekturansicht und Leseansicht.
%% Setzt den Schalter \ifkorrekturansicht voraus (gesetzt in den
%% einbindenden Dateien latex-korrekturansicht-abspann.tex bzw.
%% latex-leseansicht-abspann.tex).
%% ---------------------------------------------------------------

\normalsize

% Das esempio-Environment wird nur in der Leseansicht benötigt
\ifkorrekturansicht\else
\newenvironment{esempio}[3]%
{
    \vspace{1.5ex}
    \rlap{\underline{#1}}
    \par
    \setlength{\parindent}{0cm}
    \nopagebreak
    \leftskip=#2cm
    \rightskip=#3cm
}
{
    \par
}
\fi

\doendnotes{C}
\bigskip
\vfill

\clearpage

\footnotesize

\ifkorrekturansicht
  \lohead{\textsc{register}}
\fi

% theindex-Environment neu definieren ohne reledmac
\makeatletter
\renewenvironment{theindex}{%
  \ifkorrekturansicht
    \section*{\indexname}%
  \else
    \subsubsection*{Index der erwähnten Entitäten}%
  \fi
  \setlength{\parindent}{0pt}%
  \setlength{\parskip}{0pt plus 0.3pt}%
  \let\item\@idxitem
}{%
  \ifkorrekturansicht\clearpage\fi
}
\makeatother

\IfFileExists{\jobname-pw.ind}{\input{\jobname-pw.ind}}{}

% Quellenangabe nur in der Leseansicht
\ifkorrekturansicht\else
% Fallback-Definitionen, falls die .tex-Datei \titel etc. nicht gesetzt hat
\providecommand{\titel}{}
\providecommand{\editorInnen}{}
\providecommand{\dateiname}{\jobname}

\vspace{3cm}

\vfill

\footnotesize
\textsc{Quelle}: \titel. Herausgegeben von {\editorInnen}. In: \emph{Arthur Schnitzler: Briefwechsel mit Autorinnen und Autoren}.
 Digitale Edition, https://schnitzler-briefe.acdh.oeaw.ac.at/{\dateiname}.html (Stand \today)
\fi

\end{document}


