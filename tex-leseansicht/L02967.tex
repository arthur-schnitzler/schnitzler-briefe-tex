%% latex-leseansicht-vorspann.tex
%% Vorspann für die Leseansicht.
%% Lädt die gemeinsame Datei latex-vorspann.tex mit nicht gesetztem Schalter.

\newif\ifkorrekturansicht
\korrekturansichtfalse

\input{../tex-inputs/latex-vorspann}

\begin{center}
            \textcolor{red}{ENTWURF, NICHT FERTIG KORRIGIERT}
                      \end{center}
            
         
         \renewcommand{\erwaehntePersonen}{Personen: Ferdinand Bronner, Ida Falk, Johann Wolfgang von Goethe, Hugo von Hofmannsthal, Guy de Maupassant, Felix Salten, Siegfried Trebitsch}
         \renewcommand{\erwaehnteInstitutionen}{Institutionen: Wiener Allgemeine Zeitung}
         \renewcommand{\erwaehnteOrte}{Orte: Bad Ischl, Berlin, Hotel und Pension Rudolfshöhe (Leopold Petter), München, Wien}
         \renewcommand{\erwaehnteWerke}{Werke: ?? [Goethespäße], ?? [Maupassant-Übersetzung], Das Bergwerk zu Falun, Das Manhard-Zimmer, Der Schleier der Beatrice. Schauspiel in fünf Akten, Familie Wawroch. Ein österreichisches Drama in vier Akten., Flucht}
               \section[Arthur Schnitzler an Felix Salten, 4. 9. 1899]{ Arthur Schnitzler an Felix Salten, 4. 9. 1899}\nopagebreak\mylabel{v}\rehead{ }\begin{ledgroupsized}[t]{13cm}\normalsize\beginnumbering \toendnotes[C]{\smallbreak\pagebreak[2]} \Standort{Wienbibliothek im Rathaus, ZPH 1681, 2.1.516.}
\physDesc{Brief, 2 Blätter, 8 Seiten
\newline{}Handschrift: Bleistift, deutsche Kurrent\newline{}Ordnung: mit Bleistift von unbekannter Hand Nummerierung der ungeraden
                                 Seiten: »69«–»72« }\toendnotes[C]{\smallbreak}\pstart
           \raggedleft{}{\pb}Ischl, Rudolfshöhe\oindex{Hotel und Pension Rudolfshoehe (Leopold Petter)@\textbf{Hotel und Pension Rudolfshöhe (Leopold Petter)}|pw}{ }4/9 99\pend
           \pstart
           lieber Freund, ich will Ihnen vor allem ſagen, ds mir nicht nur
               »Flucht«, ſondern auch das Manhardzi{\geminationm}er\pwindex{Salten, Felix 06.09.1869 – 08.10.1945@\textsc{Salten, Felix} (06.09.1869 – 08.10.1945), \emph{Schriftsteller, Journalist}!Manhard-ZimmerNone@\strich\emph{Das Manhard-Zimmer} {[}None{]}|pw} noch beſſer gefallen haben, als nach dem
               erſten Leſen. Ich zweifle nicht, dſs Ihre Novelletten ein hübſches Buch gäben, möchte
               aber von einem entgiltigen {\pb}Urtheil über die
               Wirkung als ganzes, \uline{alle} Sachen auf einmal, womöglich
               in der von Ihnen gewählten Reihenfolge leſen. Herausgeben unbedingt, ſag ich ſchon
               heute, und womöglich zugleich mit dem Stück\textcolor{red}{\textsuperscript{\textbf{KEY}}} herauskommen.– In der {\pb}\label{K_L02967-11v}\edtext{Zeitung\orgindex{Wiener Allgemeine Zeitung@Wiener Allgemeine Zeitung|pwv}}{\lemma{\textnormal{\emph{Zeitung}}}\Cendnote{\textnormal{Von der ersten Ausgabe weg, die am
                     3. 7. 1899 erschien, betreute Salten\pwindex{Salten, Felix 06.09.1869 – 08.10.1945@\textsc{Salten, Felix} (06.09.1869 – 08.10.1945), \emph{Schriftsteller, Journalist}|pwk} die Rubrik »Wiener Allgemeine Rundschau« der wöchentlich
                  erscheinenenden \emph{Wiener Allgemeinen
                     Montags-Zeitung}\orgindex{Wiener Allgemeine Zeitung@Wiener Allgemeine Zeitung|pwk}. Das Blatt wurde vor Jahresende wieder eingestellt.}}}\label{K_L02967-11h}
               findet ſich viel leſenswerthes; natürlich iſt es Ihnen aus Gründen, die nicht in
               Ihnen liegen, unmöglich, das Anſtrebenswerthe daraus zu machen. Glänzend hab ich Ihre
                  Goethe\pwindex{Goethe, Johann Wolfgang von 1749-08-28 – 1832-03-22@\textsc{Goethe, Johann Wolfgang von} (1749-08-28 – 1832-03-22), \emph{Schriftsteller}|pw}späße\pwindex{Salten, Felix 06.09.1869 – 08.10.1945@\textsc{Salten, Felix} (06.09.1869 – 08.10.1945), \emph{Schriftsteller, Journalist}!?? [Goethespaesse]nach 2.7.1899@\strich\emph{?? [Goethespäße]} {[}nach 2.7.1899{]}|pwv} gefunden. Können Sie mir
               die Familie{ }{\pb}\textsc{Wawroch}\pwindex{Familie Wawroch. Ein oesterreichisches Drama in vier Akten.1899@\emph{Familie Wawroch. Ein österreichisches Drama in vier Akten.} {[}1899{]}|pw} von Adamus\pwindex{Bronner, Ferdinand 15.10.1867 – 08.06.1948@\textsc{Bronner, Ferdinand} (15.10.1867 – 08.06.1948), \emph{Schriftsteller, Lehrer}|pw} ſchicken? (Ich glaube mich zu
               erinnern dſs Sie ſie haben.) – Die Überſetzungen von\textsc{S. Tr.\pwindex{Trebitsch, Siegfried 22.12.1868 – 03.06.1956@\textsc{Trebitsch, Siegfried} (22.12.1868 – 03.06.1956), \emph{Schriftsteller, Übersetzer}|pw}} find ich ſchlecht.– Das raſche Abdrucken des neuen Maupassant\pwindex{Maupassant, Guy de 05.08.1850 – 07.07.1893@\textsc{Maupassant, Guy de} (05.08.1850 – 07.07.1893), \emph{Schriftsteller}!?? [Maupassant-Uebersetzung]nach 2.7.1899@\strich\emph{?? [Maupassant-Übersetzung]} {[}nach 2.7.1899{]}|pwv}\pwindex{Maupassant, Guy de 05.08.1850 – 07.07.1893@\textsc{Maupassant, Guy de} (05.08.1850 – 07.07.1893), \emph{Schriftsteller}|pw} zeigt den rechten Weg auf dieſem Gebiet.– \pend
           \pstart
           Ich bleibe noch bis etwa 10. oder 9. hier. Dann {\pb}vorerſt München\oindex{Muenchen@\textbf{München}|pw}, dann?– 20, 22. werd ich in Berlin\oindex{Berlin@\textbf{Berlin}|pw} ſein. Wahrſcheinlich iſt mein Stück\pwindex{Schnitzler, Arthur 15.05.1862 – 21.10.1931@\textsc{Schnitzler, Arthur} (15.05.1862 – 21.10.1931), \emph{Schriftsteller, Mediziner}!Schleier der Beatrice. Schauspiel in fuenf Akten1900-12-01@\strich\emph{Der Schleier der Beatrice. Schauspiel in fünf Akten} {[}1900-12-01{]}|pwv} bis dahin fertig. Die
               Führung und mancherlei ausgeſprochnes dürfte gut ſein; doch fühl ich oft, wie die
               Kraft des Ausdrucks {\pb}aus dem Gehirn (denn da
               ſcheint ſie mir zu ſein) nicht in den Bleiſtift will.– \pend
           \pstart
           Arbeiten bleibt endlich doch das einzige. Sonſt iſts im Weſentlichen i{\geminationm}er gleich traurig.– Auch Hugo\pwindex{Hofmannsthal, Hugo von 1874-02-01 – 1929-07-15@\textsc{Hofmannsthal, Hugo von} (1874-02-01 – 1929-07-15), \emph{Schriftsteller}|pw} arbeitet {\pb}hier
               an einem neuen Stück (Bergwerk von Falun\pwindex{Hofmannsthal, Hugo von 1874-02-01 – 1929-07-15@\textsc{Hofmannsthal, Hugo von} (1874-02-01 – 1929-07-15), \emph{Schriftsteller}!Bergwerk zu Falun1900 – 1933@\strich\emph{Das Bergwerk zu Falun} {[}1900 – 1933{]}|pw} – Sie
               wiſſens ja ſchon.) Auch ihm hat Flucht\pwindex{Salten, Felix 06.09.1869 – 08.10.1945@\textsc{Salten, Felix} (06.09.1869 – 08.10.1945), \emph{Schriftsteller, Journalist}!Flucht1900@\strich\emph{Flucht} {[}1900{]}|pw} gut
               gefallen \strikeout{(} (das andre hat er noch nicht geleſen.) – \pend
           \pstart
           Heute traf ich Frau \textsc{Ida {\pb}F.\pwindex{Falk, Ida 1862? – 1926?@\textsc{Falk, Ida} (1862? – 1926?)|pw}} – Verlobt \substVorne{}\textsuperscript{\textcolor{gray}{×}\-\textcolor{gray}{×}\-\textcolor{gray}{×}{ }\textcolor{gray}{×}\-\textcolor{gray}{×}\-\textcolor{gray}{×}}\substDazwischen{}Man ſoll nie Namen ſchreiben\substHinten{}.– Komiſcherweiſe hier iſt eine vorübergehende Verbindg zwiſchen
               mir und einer abſoluten Wiederholung jenes Typus eingetreten.–\pend
           \pstart
           Herzlichſt Ihr{\\[\baselineskip]}\spacefill\mbox{A. S.}\pend
           \leftskip=0em{}
         
         \endnumbering\mylabel{h}\end{ledgroupsized}\begin{anhang}\end{anhang}\newcommand{\dateiname}{L02967}\newcommand{\titel}{Arthur Schnitzler an Felix Salten, 4. 9. 1899}\newcommand{\editorInnen}{Martin Anton Müller und Laura Untner}%% latex-leseansicht-abspann.tex
%% Abspann für die Leseansicht.
%% Der Schalter \ifkorrekturansicht ist bereits durch den Vorspann gesetzt.

%% latex-abspann.tex
%% Gemeinsamer Abspann für Korrekturansicht und Leseansicht.
%% Setzt den Schalter \ifkorrekturansicht voraus (gesetzt in den
%% einbindenden Dateien latex-korrekturansicht-abspann.tex bzw.
%% latex-leseansicht-abspann.tex).
%% ---------------------------------------------------------------

\normalsize

% Das esempio-Environment wird nur in der Leseansicht benötigt
\ifkorrekturansicht\else
\newenvironment{esempio}[3]%
{
    \vspace{1.5ex}
    \rlap{\underline{#1}}
    \par
    \setlength{\parindent}{0cm}
    \nopagebreak
    \leftskip=#2cm
    \rightskip=#3cm
}
{
    \par
}
\fi

\doendnotes{C}
\bigskip
\vfill

\clearpage

\footnotesize

\ifkorrekturansicht
  \lohead{\textsc{register}}
\fi

% theindex-Environment neu definieren ohne reledmac
\makeatletter
\renewenvironment{theindex}{%
  \ifkorrekturansicht
    \section*{\indexname}%
  \else
    \subsubsection*{Index der erwähnten Entitäten}%
  \fi
  \setlength{\parindent}{0pt}%
  \setlength{\parskip}{0pt plus 0.3pt}%
  \let\item\@idxitem
}{%
  \ifkorrekturansicht\clearpage\fi
}
\makeatother

\IfFileExists{\jobname-pw.ind}{\input{\jobname-pw.ind}}{}

% Quellenangabe nur in der Leseansicht
\ifkorrekturansicht\else
% Fallback-Definitionen, falls die .tex-Datei \titel etc. nicht gesetzt hat
\providecommand{\titel}{}
\providecommand{\editorInnen}{}
\providecommand{\dateiname}{\jobname}

\vspace{3cm}

\vfill

\footnotesize
\textsc{Quelle}: \titel. Herausgegeben von {\editorInnen}. In: \emph{Arthur Schnitzler: Briefwechsel mit Autorinnen und Autoren}.
 Digitale Edition, https://schnitzler-briefe.acdh.oeaw.ac.at/{\dateiname}.html (Stand \today)
\fi

\end{document}


      