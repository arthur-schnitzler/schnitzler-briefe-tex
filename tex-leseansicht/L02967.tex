%% latex-korrekturansicht-vorspann.tex
%% Vorspann für die Korrekturansicht.
%% Lädt die gemeinsame Datei latex-vorspann.tex mit gesetztem Schalter.

\newif\ifkorrekturansicht
\korrekturansichttrue

\input{../tex-inputs/latex-vorspann}


\section[ Arthur Schnitzler an Felix Salten, 4. 9. 1899]{L02967 Arthur Schnitzler an Felix Salten, 4. 9. 1899}
\nopagebreak\mylabel{L02967v}
\rehead{ }\normalsize\beginnumbering\briefempfaengerindex{Salten, Felix@\textsc{Salten, Felix}!zzzSchnitzler, Arthur@\emph{von Arthur Schnitzler}!1899-09-041@{4. 9. 1899}|(be}
\toendnotes[C]{\smallbreak\pagebreak[2]}\Standort{Wienbibliothek im Rathaus, ZPH 1681, 2.1.516.}
\physDesc{Brief, 2 Blätter, 8 Seiten, 1693 Zeichen
\newline{}Handschrift: Bleistift, deutsche Kurrent
\newline{}Ordnung: mit Bleistift von unbekannter Hand Nummerierung der Doppelseiten des
                                 Konvoluts: »69«–»72« }
\buchAbdrucke{\weitereDrucke{Arthur Schnitzler: \emph{Briefe 1875–1912}. Frankfurt am Main: \emph{S. Fischer} 1981, S. 375–376.} }\toendnotes[C]{\smallbreak}
\pstart
           \raggedleft{}{\pb}Ischl, Rudolfshöhe\oindex{Hotel und Pension Rudolfshoehe (Leopold Petter)@\textbf{Hotel und Pension Rudolfshöhe (Leopold Petter)}, \emph{Hotel (K.HTL)}|pw}{ }4/9 99.\pend
           \vspace{0.5em}
\pstart
           lieber Freund, ich will Ihnen vor allem ſagen, dſs mir nicht nur
                  »Flucht\pwindex{Flucht@\emph{Flucht}|pw}«, ſondern auch das \textsc{Manhard}zi{\geminationm}er\pwindex{Manhard-Zimmer@\emph{Das Manhard-Zimmer}|pw} noch
               beſſer \label{K_L02967-1v}\edtext{gefallen}{\lemma{\textnormal{\emph{gefallen}}}\Cendnote{\textnormal{Siehe Felix Salten an Arthur Schnitzler, [29. 8. 1899].
               }}}\label{K_L02967-1} haben, als nach dem erſten Leſen. Ich zweifle nicht, dſs Ihre Novelletten\pwindex{Flucht@\emph{Flucht}|pwv}\pwindex{Fernen@\emph{Fernen}|pwv}\pwindex{Sedan@\emph{Sedan}|pwv}\pwindex{Lebenszeit@\emph{Lebenszeit}|pwv}\pwindex{Hinterbliebene@\emph{Der Hinterbliebene}|pwv}\pwindex{Manhard-Zimmer@\emph{Das Manhard-Zimmer}|pwv}\pwindex{Begraebnis@\emph{Begräbnis}|pwv}\pwindex{Heldentod@\emph{Heldentod}|pwv} ein hübſches { }Buch\pwindex{Hinterbliebene. Kurze Novellen@\emph{Der Hinterbliebene. Kurze Novellen}|pwv} gäben, möchte aber von
               einem entgiltigen {\pb}Urtheil über die Wirkung
               als ganzes, \uline{alle} Sachen auf einmal, womöglich in der
               von Ihnen gewählten Reihenfolge leſen. Herausgeben unbedingt, ſag ich ſchon heute,
               und womöglich zugleich mit dem \label{K_L02967-2v}\edtext{Stück\pwindex{Schoene Seelen. Komoedie in einem Akt@\emph{Schöne Seelen. Komödie in einem Akt}|pwv}}{\lemma{\textnormal{\emph{Stück}}}\Cendnote{\textnormal{Siehe Felix Salten an Arthur Schnitzler, 9. 10. 1899.
               }}}\label{K_L02967-2} herauskommen. – In der {\pb}\label{K_L02967-3v}\edtext{Zeitung\pwindex{Wiener Allgemeine Montags-Zeitung@\emph{Wiener Allgemeine Montags-Zeitung}|pwv}}{\lemma{\textnormal{\emph{Zeitung}}}\Cendnote{\textnormal{Von der ersten Ausgabe an, die am 3. 7. 1899 erschienen war, betreute Salten\pwindex{Salten, Felix 06.09.1869 – 08.10.1945@\textsc{Salten, Felix} (06.09.1869 – 08.10.1945), \emph{Schriftsteller/Schriftstellerin, Journalist/Journalistin, Chefredakteur/Chefredakteurin}|pwk} die Rubrik »\emph{Wiener
                     Allgemeine Rundschau}\pwindex{Wiener Allgemeine Rundschau@\emph{Wiener Allgemeine Rundschau}|pwk}« der wöchentlich erscheinenenden \emph{Wiener Allgemeinen Montags-Zeitung}\pwindex{Wiener Allgemeine Montags-Zeitung@\emph{Wiener Allgemeine Montags-Zeitung}|pwk}. Das Blatt\pwindex{Wiener Allgemeine Montags-Zeitung@\emph{Wiener Allgemeine Montags-Zeitung}|pwkv} wurde Mitte Dezember 1899 eingestellt.}}}\label{K_L02967-3} findet ſich viel leſenswerthes;
               natürlich iſt es Ihnen aus Gründen, die nicht in Ihnen liegen, unmöglich, das
               Anſtrebenswerthe daraus zu machen. Glänzend hab ich Ihre \label{K_L02967-4v}\edtext{Goethe\pwindex{Goethe, Johann Wolfgang von 1749-08-28 – 1832-03-22@\textsc{Goethe, Johann Wolfgang von} (1749-08-28 – 1832-03-22), \emph{Schriftsteller/Schriftstellerin}|pw}späße\pwindex{Eine Goethe-Enquête@\emph{Eine Goethe-Enquête}|pwv}}{\lemma{\textnormal{\emph{Goethespäße}}}\Cendnote{\textnormal{[Felix Salten\pwindex{Salten, Felix 06.09.1869 – 08.10.1945@\textsc{Salten, Felix} (06.09.1869 – 08.10.1945), \emph{Schriftsteller/Schriftstellerin, Journalist/Journalistin, Chefredakteur/Chefredakteurin}|pwk}]: \emph{Eine Goethe-Enquête}\pwindex{Eine Goethe-Enquête@\emph{Eine Goethe-Enquête}|pwk}. In: \emph{Wiener Allgemeine Montags-Zeitung}\pwindex{Wiener Allgemeine Montags-Zeitung@\emph{Wiener Allgemeine Montags-Zeitung}|pwk}, 28. 8. 1899, S. 3. Die nicht gezeichnete Umfrage\pwindex{Eine Goethe-Enquête@\emph{Eine Goethe-Enquête}|pwkv} ist deutlich als
                  Satire erkennbar (»indem wir eine Anzahl hervorragender Persönlichkeiten im
                     Geiste um ihre Meinung befragt«) und bringt erfundene Aussagen
                  von 17 Prominenten zu Goethe\pwindex{Goethe, Johann Wolfgang von 1749-08-28 – 1832-03-22@\textsc{Goethe, Johann Wolfgang von} (1749-08-28 – 1832-03-22), \emph{Schriftsteller/Schriftstellerin}|pwk}.}}}\label{K_L02967-4}
               gefunden. Können Sie mir die \label{K_L02967-5v}\edtext{Familie{ }{\pb}\textsc{Wawroch}\pwindex{Familie Wawroch. Ein oesterreichisches Drama in vier Akten@\emph{Familie Wawroch. Ein österreichisches Drama in vier Akten}|pw} von Adamus\pwindex{Bronner, Ferdinand 15.10.1867 – 08.06.1948@\textsc{Bronner, Ferdinand} (15.10.1867 – 08.06.1948), \emph{Schriftsteller/Schriftstellerin, Gymnasiallehrer/Gymnasiallehrerin}|pw}}{\lemma{\textnormal{\emph{Familie … Adamus}}}\Cendnote{\textnormal{Ferdinand Bronners\pwindex{Bronner, Ferdinand 15.10.1867 – 08.06.1948@\textsc{Bronner, Ferdinand} (15.10.1867 – 08.06.1948), \emph{Schriftsteller/Schriftstellerin, Gymnasiallehrer/Gymnasiallehrerin}|pwk}{ }\emph{Familie Wawroch. Ein österreichisches Drama in vier Akten}\pwindex{Familie Wawroch. Ein oesterreichisches Drama in vier Akten@\emph{Familie Wawroch. Ein österreichisches Drama in vier Akten}|pwk}
                  war 1899 unter dem Pseudonym Franz Adamus\pwindex{Bronner, Ferdinand 15.10.1867 – 08.06.1948@\textsc{Bronner, Ferdinand} (15.10.1867 – 08.06.1948), \emph{Schriftsteller/Schriftstellerin, Gymnasiallehrer/Gymnasiallehrerin}|pwk} erschienen. Eine Lektüre durch Schnitzler ist nicht nachweisbar.}}}\label{K_L02967-5}
               ſchicken? (Ich glaub mich zu erinnern dſs Sie ſie haben.) – Die Überſetzungen von \textsc{S. Tr.\pwindex{Trebitsch, Siegfried 22.12.1868 – 03.06.1956@\textsc{Trebitsch, Siegfried} (22.12.1868 – 03.06.1956), \emph{Schriftsteller/Schriftstellerin, Übersetzer/Übersetzerin}|pw}} find ich ſchlecht. – Das \label{K_L02967-6v}\edtext{raſche
               Abdrucken des neuen \textsc{Maupassant}\pwindex{Aus dem Nachlasse von Maupassant. Eine Leidenschaft@\emph{Aus dem Nachlasse von Maupassant. Eine Leidenschaft}|pwv}\pwindex{Maupassant, Guy de 05.08.1850 – 07.07.1893@\textsc{Maupassant, Guy de} (05.08.1850 – 07.07.1893), \emph{Schriftsteller/Schriftstellerin}|pw}}{\lemma{\textnormal{\emph{raſche … Maupassant}}}\Cendnote{\textnormal{Guy de Maupassant\pwindex{Maupassant, Guy de 05.08.1850 – 07.07.1893@\textsc{Maupassant, Guy de} (05.08.1850 – 07.07.1893), \emph{Schriftsteller/Schriftstellerin}|pwk}: \emph{Aus dem Nachlasse von Maupassant. Eine Leidenschaft}\pwindex{Aus dem Nachlasse von Maupassant. Eine Leidenschaft@\emph{Aus dem Nachlasse von Maupassant. Eine Leidenschaft}|pwk}.
                     In: \emph{Wiener Allgemeine Montags-Zeitung}\pwindex{Wiener Allgemeine Montags-Zeitung@\emph{Wiener Allgemeine Montags-Zeitung}|pwk},
                        28. 8. 1899, S. 2–4. Der Text ist der
                     Ende Juli im Verlag \emph{Emil Goldschmidt}\orgindex{Emil Goldschmidt Verlag@Emil Goldschmidt Verlag|pwk} erschienenen Buchausgabe \emph{Vater Milon und andere Erzählungen. Neue Novellen aus dem
                     litterarischen Nachlaß}\pwindex{Vater Milon und andere Erzaehlungen. Neue Novellen aus dem litterarischen Nachlass@\emph{Vater Milon und andere Erzählungen. Neue Novellen aus dem litterarischen Nachlaß}|pwk} entnommen. Die in der Buchausgabe\pwindex{Vater Milon und andere Erzaehlungen. Neue Novellen aus dem litterarischen Nachlass@\emph{Vater Milon und andere Erzählungen. Neue Novellen aus dem litterarischen Nachlaß}|pwkv}, nicht aber im Abdruck\pwindex{Aus dem Nachlasse von Maupassant. Eine Leidenschaft@\emph{Aus dem Nachlasse von Maupassant. Eine Leidenschaft}|pwkv} gezeichnete
                  Übersetzung stammt von Friedrich von
                     Oppeln-Bronikowski\pwindex{Oppeln-Bronikowski, Friedrich von 07.04.1873 – 09.10.1936@\textsc{Oppeln-Bronikowski, Friedrich von} (07.04.1873 – 09.10.1936), \emph{Schriftsteller/Schriftstellerin, Übersetzer/Übersetzerin}|pwk}.}}}\label{K_L02967-6} zeigt den rechten Weg auf dieſem Gebiet. –\pend
           
\pstart
           Ich bleibe noch bis etwa 10. oder 9.{ }\label{K_L02967-7v}\edtext{hier\oindex{Bad Ischl@\textbf{Bad Ischl}, \emph{P.PPL}|pwv}}{\lemma{\textnormal{\emph{hier}}}\Cendnote{\textnormal{Schnitzler reiste am 12. 9. 1899 von Ischl\oindex{Bad Ischl@\textbf{Bad Ischl}, \emph{P.PPL}|pwk} nach München\oindex{Muenchen@\textbf{München}, \emph{P.PPLA}|pwk} ab. Von dort reiste er am 16. 9. 1899 weiter nach Nürnberg\oindex{Nuernberg@\textbf{Nürnberg}, \emph{P.PPL}|pwk}, dann am 19. 9. 1899 weiter nach Frankfurt am Main\oindex{Frankfurt am Main@\textbf{Frankfurt am Main}, \emph{P.PPLA3}|pwk} und am 24. 9. 1899 nach Wiesbaden\oindex{Wiesbaden@\textbf{Wiesbaden}, \emph{P.PPLA}|pwk}. Zwischen 4. 10. 1899 und 11. 10. 1899 war er in Berlin\oindex{Berlin@\textbf{Berlin}, \emph{P.PPLC}|pwk}. Am
                     12. 10. 1899
                  kehrte er nach Wien\oindex{Wien@\textbf{Wien}, \emph{A.ADM2}|pwk} zurück.}}}\label{K_L02967-7}. Dann {\pb}vorerſt München\oindex{Muenchen@\textbf{München}, \emph{P.PPLA}|pw}, dann? – 20, 22. werd ich in Berlin\oindex{Berlin@\textbf{Berlin}, \emph{P.PPLC}|pw} ſein.
                  Wahrſcheinli{[}ch{]} iſt mein \label{K_L02967-8v}\edtext{Stück\pwindex{Schleier der Beatrice. Schauspiel in fuenf Akten@\emph{Der Schleier der Beatrice. Schauspiel in fünf Akten}|pwv}}{\lemma{\textnormal{\emph{Stück}}}\Cendnote{\textnormal{Schnitzler schloss \emph{Der Schleier der Beatrice}\pwindex{Schleier der Beatrice. Schauspiel in fuenf Akten@\emph{Der Schleier der Beatrice. Schauspiel in fünf Akten}|pwk} am 9. 9. 1899 vorläufig
                  ab.}}}\label{K_L02967-8} bis dahin fertig. Die Führung und mancherlei ausgeſprochnes dürfte gut
               ſein; doch fühl ich oft, wie die Kraft des Ausdrucks {\pb}aus dem Gehirn (denn da ſcheint ſie mir zu
               ſein) nicht in den Bleiſtift will. –\pend
           
\pstart
           Arbeiten bleibt endlich doch das einzige. Sonſt iſts im Weſentlichen i{\geminationm}er gleich traurig. – Auch Hugo\pwindex{Hofmannsthal, Hugo von 1874-02-01 – 1929-07-15@\textsc{Hofmannsthal, Hugo von} (1874-02-01 – 1929-07-15), \emph{Schriftsteller/Schriftstellerin}|pw} arbeitet {\pb}hier\oindex{Bad Ischl@\textbf{Bad Ischl}, \emph{P.PPL}|pwv} an einem \label{K_L02967-9v}\edtext{neuen Stück}{\lemma{\textnormal{\emph{neuen Stück}}}\Cendnote{\textnormal{Am 31. 8. 1899 hatte Hugo von
                     Hofmannsthal\pwindex{Hofmannsthal, Hugo von 1874-02-01 – 1929-07-15@\textsc{Hofmannsthal, Hugo von} (1874-02-01 – 1929-07-15), \emph{Schriftsteller/Schriftstellerin}|pwk}{ }Schnitzler bereits zwei Akte aus \emph{Das Bergwerk zu Falun}\pwindex{Bergwerk zu Falun@\emph{Das Bergwerk zu Falun}|pwk} vorgelesen.}}}\label{K_L02967-9} (Bergwerk von \textsc{Falun}\pwindex{Bergwerk zu Falun@\emph{Das Bergwerk zu Falun}|pw} – Sie wiſſens ja ſchon.) Auch
               ihm hat Flucht\pwindex{Flucht@\emph{Flucht}|pw} gut gefallen \strikeout{(} (das andre\pwindex{Manhard-Zimmer@\emph{Das Manhard-Zimmer}|pwv}\pwindex{Sedan@\emph{Sedan}|pwv} hat er noch nicht geleſen.) –\pend
           
\pstart
           Heute traf ich Frau \textsc{Ida {\pb}F.\pwindex{Falk, Ida 1862? – 1926?@\textsc{Falk, Ida} (1862? – 1926?)|pw}} – \label{K_L02967-10v}\edtext{Verlobt}{\lemma{\textnormal{\emph{Verlobt}}}\Cendnote{\textnormal{Ida Falk\pwindex{Falk, Ida 1862? – 1926?@\textsc{Falk, Ida} (1862? – 1926?)|pwk}, ehemalige Geliebte sowohl von Schnitzler als auch von Salten\pwindex{Salten, Felix 06.09.1869 – 08.10.1945@\textsc{Salten, Felix} (06.09.1869 – 08.10.1945), \emph{Schriftsteller/Schriftstellerin, Journalist/Journalistin, Chefredakteur/Chefredakteurin}|pwk}, hatte sich mit Ludwig Grann\pwindex{Grann, Ludwig 1860-03-02 – 1935-12-23@\textsc{Grann, Ludwig} (1860-03-02 – 1935-12-23), \emph{Anwalt/Anwältin}|pwk} verlobt, vgl. A. S.: \emph{Tagebuch}, 23. 10. 1899.}}}\label{K_L02967-10}. \substVorne{}\textsuperscript{\textcolor{gray}{×}\-\textcolor{gray}{×}\-\textcolor{gray}{×}\-\textcolor{gray}{×}\-\textcolor{gray}{×}\-\textcolor{gray}{×}{ }\textcolor{gray}{×}\-\textcolor{gray}{×}\-\textcolor{gray}{×}\-\textcolor{gray}{×}}\substDazwischen{}Man ſoll nie Namen ſchreiben\substHinten{}. – Komiſcherweiſe iſt \introOben{}hier\introOben{} eine vorübergehende
               Verbindg zwiſchen mir und einer abſoluten Wiederholung jenes Typus eingetreten. –\pend
           
\pstart
           Herzlichſt Ihr {\\[\baselineskip]}\spacefill\mbox{A. S.}\pend
           \leftskip=0em{}\selectlanguage{ngerman}\endnumbering\briefempfaengerindex{Salten, Felix@\textsc{Salten, Felix}!zzzSchnitzler, Arthur@\emph{von Arthur Schnitzler}!1899-09-041@{4. 9. 1899}|)be}\mylabel{L02967h}  \normalsize

\doendnotes{C}
\bigskip
\vfill

\clearpage

\footnotesize

\lohead{\textsc{register}}

% Definiere theindex-Environment komplett neu ohne reledmac
\makeatletter
\renewenvironment{theindex}{%
  \section*{\indexname}%
  \setlength{\parindent}{0pt}%
  \setlength{\parskip}{0pt plus 0.3pt}%
  \let\item\@idxitem
}{%
  \clearpage
}
\makeatother

\IfFileExists{\jobname-pw.ind}{\input{\jobname-pw.ind}}{}

\end{document}

      