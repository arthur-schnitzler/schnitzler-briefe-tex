%% latex-korrekturansicht-vorspann.tex
%% Vorspann für die Korrekturansicht.
%% Lädt die gemeinsame Datei latex-vorspann.tex mit gesetztem Schalter.

\newif\ifkorrekturansicht
\korrekturansichttrue

\input{../tex-inputs/latex-vorspann}


\section[Paul Goldmann an Arthur Schnitzler, 18. 9. 1899]{L02682 Paul Goldmann an Arthur Schnitzler, 18. 9. 1899}
\nopagebreak\mylabel{L02682v}
\rehead{ }\normalsize\beginnumbering\briefempfaengerindex{Schnitzler, Arthur@\textsc{Schnitzler, Arthur}!zzzGoldmann, Paul@\emph{von Paul Goldmann}!1899-09-182@{18. 9. 1899}|(be}
\toendnotes[C]{\smallbreak\pagebreak[2]}\Standort{DLA, A:Schnitzler, HS.NZ85.1.3169.}
\physDesc{Telegramm, 337 Zeichen
\newline{}maschinell
\newline{}Handschrift einer Schreibkraft: Bleistift, lateinische Kurrent (\noindent{}Vordruck für den Empfang)
\newline{}Versand: »\noindent{}\textcolor{gray}{\textbf{K. B. Telegraphenanſtalt
                                             Nürnberg\orgindex{Telegraphenanstalt Nuernberg@Telegraphenanstalt Nürnberg|pw}.}}{ / }\textcolor{gray}{\textbf{Aufgegeben in}}{ }Frankfurt\textsuperscript{M}\oindex{Frankfurt am Main@\textbf{Frankfurt am Main}, \emph{P.PPLA3}|pw}{ / }\textcolor{gray}{\textbf{Nr.}}{ }53 \textcolor{gray}{\textbf{W.
                                          den}}{ }18/9 \textcolor{gray}{\textbf{1899}}{ / }5 \textcolor{gray}{\textbf{Uhr}} 27 \textcolor{gray}{\textbf{Min. Nachm.}}{ / }\textcolor{gray}{\textbf{Abgefertigt den}}{ }18 9 \textcolor{gray}{\textbf{1899}}{ }6 \textcolor{gray}{\textbf{Uhr}} 40 \textcolor{gray}{\textbf{Min.}}« 
\newline{}Schnitzler: mit Bleistift datiert: »18/9 99« }\toendnotes[C]{\smallbreak}\pstart{}{\pb}arthur schnitzler\pend{}\pstart{}nuernberg kaiserhof\oindex{Hotel Deutscher Kaiser@\textbf{Hotel Deutscher Kaiser}, \emph{Hotel (K.HTL)}|pw}:=\pend{}{\bigskip}\vspace{1em}
\pstart
           \noindent{}{\pb}sandte dir gestern
               langes \label{K_L02682-1v}\edtext{telegramm}{\lemma{\textnormal{\emph{telegramm}}}\Cendnote{\textnormal{nicht überliefert}}}\label{K_L02682-1}{ }nuernberg\oindex{Nuernberg@\textbf{Nürnberg}, \emph{P.PPL}|pw} poste restante worin ich dir
               mittheilte dass ich samstag meinen urlaub antrete nach
                  florenz\oindex{Florenz@\textbf{Florenz}, \emph{P.PPLA}|pw} fahre und mich wenn du willst auch
               dort oder unterwegs mit dir \label{K_L02682-2v}\edtext{treffen}{\lemma{\textnormal{\emph{treffen}}}\Cendnote{\textnormal{Am Folgetag, einem Dienstag
                     (19. 9. 1899),
                  reiste Schnitzler nach Frankfurt am Main\oindex{Frankfurt am Main@\textbf{Frankfurt am Main}, \emph{P.PPLA3}|pwk}, wo er auf Goldmann\pwindex{Goldmann, Paul 31.01.1865 – 25.09.1935@\textsc{Goldmann, Paul} (31.01.1865 – 25.09.1935), \emph{Schriftsteller/Schriftstellerin, Journalist/Journalistin}|pwk} traf.}}}\label{K_L02682-2} koennte. natuerlich werde ich mich
               auch unendlich freuen dich in frankfurt\oindex{Frankfurt am Main@\textbf{Frankfurt am Main}, \emph{P.PPLA3}|pw} zu
               sehen.\pend
           \pstart gruss = \spacefill\mbox{goldmann. +}\pend{}\selectlanguage{ngerman}\endnumbering\briefempfaengerindex{Schnitzler, Arthur@\textsc{Schnitzler, Arthur}!zzzGoldmann, Paul@\emph{von Paul Goldmann}!1899-09-182@{18. 9. 1899}|)be}\mylabel{L02682h}  \normalsize

\doendnotes{C}
\bigskip
\vfill

\clearpage

\footnotesize

\lohead{\textsc{register}}

% Definiere theindex-Environment komplett neu ohne reledmac
\makeatletter
\renewenvironment{theindex}{%
  \section*{\indexname}%
  \setlength{\parindent}{0pt}%
  \setlength{\parskip}{0pt plus 0.3pt}%
  \let\item\@idxitem
}{%
  \clearpage
}
\makeatother

\IfFileExists{\jobname-pw.ind}{\input{\jobname-pw.ind}}{}

\end{document}

      