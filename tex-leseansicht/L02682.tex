\input{../tex-inputs/latex-pdf-vorspann}
\begin{center}
            \textcolor{red}{ENTWURF. ENTZIFFERUNG NOCH NICHT KORREKTURGELESEN}
                      \end{center}
            
               \section[Paul Goldmann an Arthur Schnitzler, 18. 9. 1899]{ Paul Goldmann an Arthur Schnitzler, 18. 9. 1899}\nopagebreak\mylabel{v}\rehead{ }\begin{ledgroupsized}[t]{13cm}\normalsize\beginnumbering\briefempfaengerindex{Schnitzler, Arthur@\textsc{Schnitzler, Arthur}!zzzGoldmann, Paul@\emph{von Paul Goldmann}!1899-09-181@{18. 9. 1899}|(be} \toendnotes[C]{\smallbreak\pagebreak[2]} \Standort{DLA, A:Schnitzler, HS.NZ85.1.3169.}
\physDesc{Telegramm, 1 Blatt, 2 Seiten
\newline{}maschinell
\newline{}Handschrift einer Schreibkraft: Bleistift, lateinische Kurrent (\noindent{}Vordruck für den Empfang)\newline{}Versand: »\noindent{}\textcolor{gray}{\textbf{K. B. Telegraphenanſtalt
                                             Nürnberg\orgindex{Telegraphenanstalt Nuernberg@Telegraphenanstalt Nürnberg|pw}.}}{ / }\textcolor{gray}{\textbf{Aufgegeben in}}{ }Frankfurt\textsuperscript{M}\oindex{Frankfurt am Main@\textbf{Frankfurt am Main}|pw}{ / }\textcolor{gray}{\textbf{Nr.}}{ }53 \textcolor{gray}{\textbf{W.
                                          den}}{ }18/9 \textcolor{gray}{\textbf{1899}}{ / }5 \textcolor{gray}{\textbf{Uhr}} 27 \textcolor{gray}{\textbf{Min. Nachm.}}{ / }\textcolor{gray}{\textbf{Abgefertigt den}}{ }18 9 \textcolor{gray}{\textbf{1899}}{ }6 \textcolor{gray}{\textbf{Uhr }}40 \textcolor{gray}{\textbf{Min.}}« 
\newline{}Schnitzler: mit Bleistift datiert: »18/9 99« }\toendnotes[C]{\smallbreak}\pstart{}{\pb}arthur schnitzler\pend{}\pstart{}nuernberg kaiserhof\oindex{Hotel Deutscher Kaiser@\textbf{Hotel Deutscher Kaiser}|pw}:=\pend{}{\bigskip}\pstart
           \noindent{}{\pb}sandte dir gestern
               langes telegramm nuernberg\oindex{Nuernberg@\textbf{Nürnberg}|pw} poste restante worin
               ich dir mittheilte dass ich samstag meinen urlaub
               antrete nach florenz\oindex{Florenz@\textbf{Florenz}|pw} fahre und mich wenn du
               willst auch dort oder unterwegs mit dir \label{K_L02682-1v}\edtext{treffen}{\lemma{\textnormal{\emph{treffen}}}\Cendnote{\textnormal{Am 19. 9. 1899 reiste Schnitzler\pwindex{Schnitzler, Arthur 15.05.1862 – 21.10.1931@\textsc{Schnitzler, Arthur} (15.05.1862 – 21.10.1931), \emph{Schriftsteller, Mediziner}|pwk} nach Frankfurt am Main\oindex{Frankfurt am Main@\textbf{Frankfurt am Main}|pwk}, wo er auch Goldmann\pwindex{Goldmann, Paul 31.01.1865 – 25.09.1935@\textsc{Goldmann, Paul} (31.01.1865 – 25.09.1935), \emph{Schriftsteller, Journalist}|pwk} wiedertraf.}}}\label{K_L02682-1h} koennte. natuerlich werde ich
               mich auch unendlich freuen dich in frankfurt\oindex{Frankfurt am Main@\textbf{Frankfurt am Main}|pw} zu
               sehen.\pend
           \pstart gruss = \spacefill\mbox{goldmann. +}\pend{}\endnumbering\briefempfaengerindex{Schnitzler, Arthur@\textsc{Schnitzler, Arthur}!zzzGoldmann, Paul@\emph{von Paul Goldmann}!1899-09-181@{18. 9. 1899}|)be}\mylabel{h}\end{ledgroupsized}\begin{anhang}\end{anhang}\newcommand{\dateiname}{L02682}\newcommand{\titel}{Paul Goldmann an Arthur Schnitzler, 18. 9. 1899}\newcommand{\editorInnen}{Martin Anton Müller und Laura Untner}\input{../tex-inputs/latex-pdf-abspann}
      