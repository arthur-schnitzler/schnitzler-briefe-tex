%% latex-leseansicht-vorspann.tex
%% Vorspann für die Leseansicht.
%% Lädt die gemeinsame Datei latex-vorspann.tex mit nicht gesetztem Schalter.

\newif\ifkorrekturansicht
\korrekturansichtfalse

\input{../tex-inputs/latex-vorspann}


\section[Elsa Plessner an Arthur Schnitzler, 26. 2. 1900]{L03725 Elsa Plessner an Arthur Schnitzler, 26. 2. 1900}
\nopagebreak\mylabel{L03725v}
\rehead{ }\normalsize\beginnumbering\briefempfaengerindex{Schnitzler, Arthur@\textsc{Schnitzler, Arthur}!zzzPlessner, Elsa@\emph{von Elsa Plessner}!1900-02-261@{26. 2. 1900}|(be}
\toendnotes[C]{\smallbreak\pagebreak[2]}
\correspDesc{Versand  durch Elsa Plessner am 26. 2. 1900 in Wien
\newline{}Erhalt  durch Arthur Schnitzler im Zeitraum [26. 2. 1900 –
            1. 3. 1900?] in Wien}\toendnotes[C]{\smallbreak}
\Standort{DLA, A:Schnitzler, HS.1985.1.419.}
\physDesc{Brief, 2 Blätter, 5 Seiten, 2129 Zeichen
\newline{}Handschrift: schwarze Tinte, lateinische Kurrent}\toendnotes[C]{\smallbreak}
\pstart
           \raggedleft{}{\pb}Wien I. Kärnthnerstraße N\textsuperscript{o} 10\oindex{Wien@\textbf{Wien}!I., Innere Stadt@\textbf{I., Innere Stadt}!Kärntner Straße 10@\textbf{Kärntner Straße 10}, \emph{Wohngebäude}|pw}\pend
           
\pstart
           \raggedleft{}den 26. Februar 1900\pend
           
\pstart{}Verehrter Herr Doctor!\pend\vspace{0.5em}
\pstart
           Ich hoffe, dass Sie nicht lachen werden, wenn Sie diesen Brief zu Ende gelesen haben. Sie
          werden wahrscheinlich lächeln – aber das macht nichts.\pend
           
\pstart
           Aus den \label{K_L03725-1v}\edtext{morgigen
            Blättern}{\lemma{\textnormal{\emph{morgigen
            Blättern}}}\Cendnote{\textnormal{Das \emph{Illustrierte Wiener Extrablatt}\pwindex{Illustrirtes Wiener Extrablatt@\emph{Illustrirtes Wiener Extrablatt}|pwk} gab am 28. 2. 1900 bekannt: »Die Direction des \emph{\so{Deutschen Volkstheaters}}\orgindex{Volkstheater@Volkstheater|pwk} hat das
            dreiactige Schauspiel ›\emph{Die Ehrlosen}\pwindex{Plessner, Elsa 22.\,8.\,1875 Wien – 7.\,5.\,1932 Alicante@\textsc{Plessner, Elsa} (22.\,8.\,1875 Wien – 7.\,5.\,1932 Alicante), \emph{Schriftstellerin}!Ehrlosen. Schauspiel in drei Acten@\strich\emph{Die Ehrlosen. Schauspiel in drei Acten}|pwk}‹ von Elsa Pleßner\pwindex{Plessner, Elsa 22.\,8.\,1875 Wien – 7.\,5.\,1932 Alicante@\textsc{Plessner, Elsa} (22.\,8.\,1875 Wien – 7.\,5.\,1932 Alicante), \emph{Schriftstellerin}|pwk}, einer jungen Wiener\oindex{Wien@\textbf{Wien}, \emph{Verwaltungsgebiet}|pwk} Schriftstellerin, zur Aufführung in der nächsten
              Spielzeit angenommen« (\emph{Illustrierte
                Wiener Extrablatt}\pwindex{Illustrirtes Wiener Extrablatt@\emph{Illustrirtes Wiener Extrablatt}|pwk}, Jg. 29, Nr. 57,
              28. 2. 1900, S. 12). Ähnliche Formulierungen
            brachten unter dem gleichen Datum \emph{Neues Wiener
              Journal}\pwindex{Neues Wiener Journal@\emph{Neues Wiener Journal}|pwk}, \emph{Wiener Zeitung}\pwindex{Wiener Zeitung@\emph{Wiener Zeitung}|pwk} und \emph{Das Vaterland}\pwindex{Vaterland@\emph{Das Vaterland}|pwk}.}}}\label{K_L03725-1} werden Sie entnehmen, dass ich meine bis
               heute sorgfältig gehütete Anonymität aufgegeben habe – weil \label{K_L03725-2v}\edtext{die Première\pwindex{Plessner, Elsa 22.\,8.\,1875 Wien – 7.\,5.\,1932 Alicante@\textsc{Plessner, Elsa} (22.\,8.\,1875 Wien – 7.\,5.\,1932 Alicante), \emph{Schriftstellerin}!Ehrlosen. Schauspiel in drei Acten@\strich\emph{Die Ehrlosen. Schauspiel in drei Acten}|pwv}\eventindex{Volkstheater@\textbf{Volkstheater}!Uraufführung von Die Ehrlosen, 16.3.1901@Uraufführung von Die Ehrlosen, 16.3.1901|pwv}}{\lemma{\textnormal{\emph{die Première}}}\Cendnote{\textnormal{
                  Die Theateruraufführung von \emph{Die Ehrlosen}\pwindex{Plessner, Elsa 22.\,8.\,1875 Wien – 7.\,5.\,1932 Alicante@\textsc{Plessner, Elsa} (22.\,8.\,1875 Wien – 7.\,5.\,1932 Alicante), \emph{Schriftstellerin}!Ehrlosen. Schauspiel in drei Acten@\strich\emph{Die Ehrlosen. Schauspiel in drei Acten}|pwk} von Elsa Plessner\pwindex{Plessner, Elsa 22.\,8.\,1875 Wien – 7.\,5.\,1932 Alicante@\textsc{Plessner, Elsa} (22.\,8.\,1875 Wien – 7.\,5.\,1932 Alicante), \emph{Schriftstellerin}|pwk}\eventindex{Volkstheater@\textbf{Volkstheater}!Uraufführung von Die Ehrlosen, 16.3.1901@Uraufführung von Die Ehrlosen, 16.3.1901|pwk} fand am 16. 3. 1901 am \emph{Volkstheater}\orgindex{Volkstheater@Volkstheater|pwk} statt.
               }}}\label{K_L03725-2} in die nächste Saison verschoben wurde und \label{K_L03725-3v}\edtext{Bloch\pwindex{Bloch, Richard 3.\,3.\,1856 Berlin – 1928 ebd.@\textsc{Bloch, Richard} (3.\,3.\,1856 Berlin – 1928 ebd.), \emph{Theaterverleger}|pwu}}{\lemma{\textnormal{\emph{Bloch}}}\Cendnote{\textnormal{Es dürfte 
               Richard Bloch\pwindex{Bloch, Richard 3.\,3.\,1856 Berlin – 1928 ebd.@\textsc{Bloch, Richard} (3.\,3.\,1856 Berlin – 1928 ebd.), \emph{Theaterverleger}|pwk} gemeint sein, der bei \emph{Felix Bloch}\orgindex{Felix Bloch Erben@Felix Bloch Erben|pwk} angestellt war. Demnach hatte sie ihr Stück\pwindex{Plessner, Elsa 22.\,8.\,1875 Wien – 7.\,5.\,1932 Alicante@\textsc{Plessner, Elsa} (22.\,8.\,1875 Wien – 7.\,5.\,1932 Alicante), \emph{Schriftstellerin}!Ehrlosen. Schauspiel in drei Acten@\strich\emph{Die Ehrlosen. Schauspiel in drei Acten}|pwkv} dem Verlag zum Vertrieb überlassen.}}}\label{K_L03725-3} mich
          gedrängt hat – {\pb}aber das ist Nebensache. –\pend
           
\pstart
           Hauptsache ist,
          dass Sie aus dem Titel »Die Ehrlosen\pwindex{Plessner, Elsa 22.\,8.\,1875 Wien – 7.\,5.\,1932 Alicante@\textsc{Plessner, Elsa} (22.\,8.\,1875 Wien – 7.\,5.\,1932 Alicante), \emph{Schriftstellerin}!Ehrlosen. Schauspiel in drei Acten@\strich\emph{Die Ehrlosen. Schauspiel in drei Acten}|pw}« gewiss
          errathen haben, dass das vom Volkstheater\orgindex{Volkstheater@Volkstheater|pw} angenommene
            Stück\pwindex{Plessner, Elsa 22.\,8.\,1875 Wien – 7.\,5.\,1932 Alicante@\textsc{Plessner, Elsa} (22.\,8.\,1875 Wien – 7.\,5.\,1932 Alicante), \emph{Schriftstellerin}!Ehrlosen. Schauspiel in drei Acten@\strich\emph{Die Ehrlosen. Schauspiel in drei Acten}|pwv} – – dasselbe ist,
          dasselbe, das Sie mir im vergangenen Jahr so furchtbar \label{K_L03725-4v}\edtext{verdonnert haben}{\lemma{\textnormal{\emph{verdonnert haben}}}\Cendnote{\textnormal{Schnitzlers Kritik ist nicht überliefert, aber die
            Erschütterung Plessners\pwindex{Plessner, Elsa 22.\,8.\,1875 Wien – 7.\,5.\,1932 Alicante@\textsc{Plessner, Elsa} (22.\,8.\,1875 Wien – 7.\,5.\,1932 Alicante), \emph{Schriftstellerin}|pwk} darüber, vgl. XXXX Auszeichnungsfehler: Dokument L03722 nicht gefunden.}}}\label{K_L03725-4}. Darum hab ich auch
          letzthin Angst gehabt – es Ihnen zu gestehen. Für heute fühle ich mich so
          gewissermaßen gedrängt, Ihnen zu versichern, dass ich auch heute, nachdem man
          sich hier und \label{K_L03725-5v}\edtext{in Berlin\oindex{Berlin@\textbf{Berlin}, \emph{Hauptstadt}|pw}}{\lemma{\textnormal{\emph{in Berlin}}}\Cendnote{\textnormal{In Berlin\oindex{Berlin@\textbf{Berlin}, \emph{Hauptstadt}|pwk}
            kam das Schauspiel\pwindex{Plessner, Elsa 22.\,8.\,1875 Wien – 7.\,5.\,1932 Alicante@\textsc{Plessner, Elsa} (22.\,8.\,1875 Wien – 7.\,5.\,1932 Alicante), \emph{Schriftstellerin}!Ehrlosen. Schauspiel in drei Acten@\strich\emph{Die Ehrlosen. Schauspiel in drei Acten}|pwkv} nicht zur
            Aufführung und es ist nicht bekannt, mit welchem Theater dort Plessner\pwindex{Plessner, Elsa 22.\,8.\,1875 Wien – 7.\,5.\,1932 Alicante@\textsc{Plessner, Elsa} (22.\,8.\,1875 Wien – 7.\,5.\,1932 Alicante), \emph{Schriftstellerin}|pwk} in Verhandlung stand.}}}\label{K_L03725-5} ziemlich viel von der Arbeit\pwindex{Plessner, Elsa 22.\,8.\,1875 Wien – 7.\,5.\,1932 Alicante@\textsc{Plessner, Elsa} (22.\,8.\,1875 Wien – 7.\,5.\,1932 Alicante), \emph{Schriftstellerin}!Ehrlosen. Schauspiel in drei Acten@\strich\emph{Die Ehrlosen. Schauspiel in drei Acten}|pwv}{ }{\pb}verspricht, ziemlich im Klaren bin über den wahren
          literarischen Wert des Stückes\pwindex{Plessner, Elsa 22.\,8.\,1875 Wien – 7.\,5.\,1932 Alicante@\textsc{Plessner, Elsa} (22.\,8.\,1875 Wien – 7.\,5.\,1932 Alicante), \emph{Schriftstellerin}!Ehrlosen. Schauspiel in drei Acten@\strich\emph{Die Ehrlosen. Schauspiel in drei Acten}|pwv} –
          d. h. dass meine Ansicht darüber nicht allzu sehr von der Ihren abweicht. Aber – Sie
          wissen beim Theater weiß man nie etwas – und hoffentlich wird nicht diese unsere wahre
          Meinung vom Publicum getheilt werden. Ich bitte Sie vielmals, das nicht für Arroganz oder
          Pose zu halten, dass ich Ihnen das sage – ich glaube, dass ich weder das Eine, noch das
          andere Ihnen gegenüber {\pb}nöthig habe. Dass ich Ihnen letzthin
          aus heiler Haut mein neues \label{K_L03725-6v}\edtext{Stück\pwindex{Plessner, Elsa 22.\,8.\,1875 Wien – 7.\,5.\,1932 Alicante@\textsc{Plessner, Elsa} (22.\,8.\,1875 Wien – 7.\,5.\,1932 Alicante), \emph{Schriftstellerin}!erste Kapitel. Schauspiel in drei Akten@\strich\emph{Das erste Kapitel. Schauspiel in drei Akten}|pwv} schickte}{\lemma{\textnormal{\emph{Stück schickte}}}\Cendnote{\textnormal{Vgl. XXXX Auszeichnungsfehler: Dokument L03723 nicht gefunden.}}}\label{K_L03725-6}, soll Sie
          überzeugen, dass mich selbst ein eventueller Erfolg der »Ehrlosen\pwindex{Plessner, Elsa 22.\,8.\,1875 Wien – 7.\,5.\,1932 Alicante@\textsc{Plessner, Elsa} (22.\,8.\,1875 Wien – 7.\,5.\,1932 Alicante), \emph{Schriftstellerin}!Ehrlosen. Schauspiel in drei Acten@\strich\emph{Die Ehrlosen. Schauspiel in drei Acten}|pw}« nicht auf den Holzweg locken soll, den ich damit eingeschlagen habe. Sie
          sehen – ich habe echte, aufrichtige\introOben{}n\introOben{}{ }\uline{literarischen} Ehrgeiz und wenn ich auch nicht den
          Heroismus besitze, mit einem, wenn auch minderwerthigen Stücke\pwindex{Plessner, Elsa 22.\,8.\,1875 Wien – 7.\,5.\,1932 Alicante@\textsc{Plessner, Elsa} (22.\,8.\,1875 Wien – 7.\,5.\,1932 Alicante), \emph{Schriftstellerin}!Ehrlosen. Schauspiel in drei Acten@\strich\emph{Die Ehrlosen. Schauspiel in drei Acten}|pwv} in dem doch ziemlich hochstehenden Theater\oindex{Wien@\textbf{Wien}!VII., Neubau@\textbf{VII., Neubau}!Volkstheater@\textbf{Volkstheater}, \emph{Theater}|pwv} aufgeführt zu werden – als eine teuflische
          Versuchung von mir zu {\pb}weisen, so weiß ich doch ganz gut, dass
          das äußerliche Emporkommen noch nichts bedeutet, wenn nicht – ja wenn nicht u. s. w. Was ich also Ihnen jetzt als Beichtgeheimnis anvertraue, soll mich nur in Ihren Augen
          reinwaschen und wenn Sie nicht schlecht von mir denken, so werden Sie sehr erfreuen Ihre
          Sie stets hochverehrende\pend
           \pstart \spacefill\mbox{Elsa Plessner}.\pend{}\selectlanguage{ngerman}\endnumbering\briefempfaengerindex{Schnitzler, Arthur@\textsc{Schnitzler, Arthur}!zzzPlessner, Elsa@\emph{von Elsa Plessner}!1900-02-261@{26. 2. 1900}|)be}\mylabel{L03725h}  \newcommand{\dateiname}{L03725}\newcommand{\titel}{Elsa Plessner an Arthur Schnitzler, 26. 2. 1900}\newcommand{\editorInnen}{Selma Jahnke und Martin Anton Müller}%% latex-leseansicht-abspann.tex
%% Abspann für die Leseansicht.
%% Der Schalter \ifkorrekturansicht ist bereits durch den Vorspann gesetzt.

%% latex-abspann.tex
%% Gemeinsamer Abspann für Korrekturansicht und Leseansicht.
%% Setzt den Schalter \ifkorrekturansicht voraus (gesetzt in den
%% einbindenden Dateien latex-korrekturansicht-abspann.tex bzw.
%% latex-leseansicht-abspann.tex).
%% ---------------------------------------------------------------

\normalsize

% Das esempio-Environment wird nur in der Leseansicht benötigt
\ifkorrekturansicht\else
\newenvironment{esempio}[3]%
{
    \vspace{1.5ex}
    \rlap{\underline{#1}}
    \par
    \setlength{\parindent}{0cm}
    \nopagebreak
    \leftskip=#2cm
    \rightskip=#3cm
}
{
    \par
}
\fi

\doendnotes{C}
\bigskip
\vfill

\clearpage

\footnotesize

\ifkorrekturansicht
  \lohead{\textsc{register}}
\fi

% theindex-Environment neu definieren ohne reledmac
\makeatletter
\renewenvironment{theindex}{%
  \ifkorrekturansicht
    \section*{\indexname}%
  \else
    \subsubsection*{Index der erwähnten Entitäten}%
  \fi
  \setlength{\parindent}{0pt}%
  \setlength{\parskip}{0pt plus 0.3pt}%
  \let\item\@idxitem
}{%
  \ifkorrekturansicht\clearpage\fi
}
\makeatother

\IfFileExists{\jobname-pw.ind}{\input{\jobname-pw.ind}}{}

% Quellenangabe nur in der Leseansicht
\ifkorrekturansicht\else
% Fallback-Definitionen, falls die .tex-Datei \titel etc. nicht gesetzt hat
\providecommand{\titel}{}
\providecommand{\editorInnen}{}
\providecommand{\dateiname}{\jobname}

\vspace{3cm}

\vfill

\footnotesize
\textsc{Quelle}: \titel. Herausgegeben von {\editorInnen}. In: \emph{Arthur Schnitzler: Briefwechsel mit Autorinnen und Autoren}.
 Digitale Edition, https://schnitzler-briefe.acdh.oeaw.ac.at/{\dateiname}.html (Stand \today)
\fi

\end{document}


