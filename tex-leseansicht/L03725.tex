%% latex-korrekturansicht-vorspann.tex
%% Vorspann für die Korrekturansicht.
%% Lädt die gemeinsame Datei latex-vorspann.tex mit gesetztem Schalter.

\newif\ifkorrekturansicht
\korrekturansichttrue

\input{../tex-inputs/latex-vorspann}


\section[Elsa Plessner an Arthur Schnitzler, 26. 2. 1900]{L03725 Elsa Plessner an Arthur Schnitzler, 26. 2. 1900}
\nopagebreak\mylabel{L03725v}
\rehead{ }\normalsize\beginnumbering\briefempfaengerindex{Schnitzler, Arthur@\textsc{Schnitzler, Arthur}!zzzPlessner, Elsa@\emph{von Elsa Plessner}!1900-02-261@{26. 2. 1900}|(be}
\toendnotes[C]{\smallbreak\pagebreak[2]}\Standort{DLA, A:Schnitzler, HS.1985.1.419.}
\physDesc{Brief,  Blätter, 5 Seiten, 2131 Zeichen
\newline{}Handschrift: , lateinische Kurrent}\toendnotes[C]{\smallbreak}
\pstart
           {\pb}Wien I. Kärnthnerstraße N\textsuperscript{o} 10\oindex{Kaerntner Strasse 10@\textbf{Kärntner Straße 10}, \emph{Wohngebäude (K.WHS)}|pw}\pend
           
\pstart
           \raggedleft{}den 26. Februar 1900\pend
           
\pstart{}Verehrter Herr Doctor!\pend\vspace{0.5em}
\pstart
           Ich hoffe, dass Sie nicht lachen werden, wenn Sie diesen Brief zu Ende gelesen haben.
               Sie werden wahrscheinlich lächeln – aber das macht nichts. \pend
           
\pstart
           Aus den morgigen Blättern werden Sie entnehmen, dass ich meine bis
                  heute sorgfältig gehütete Anonymität aufgegeben habe – weil die
                  Première\pwindex{Ehrlosen. Schauspiel in drei Acten@\emph{Die Ehrlosen. Schauspiel in drei Acten}|pwv} in die
                  nächste Saison verschoben wurde und Bloch\pwindex{Bloch, Richard 03.03.1856 – 1928@\textsc{Bloch, Richard} (03.03.1856 – 1928), \emph{Theaterverleger/Theaterverlegerin}|pw} mich gedrängt hat – {\pb}aber das ist Nebensache. –
               Hauptsache ist, dass Sie aus dem Titel »Die
                  Ehrlosen\pwindex{Ehrlosen. Schauspiel in drei Acten@\emph{Die Ehrlosen. Schauspiel in drei Acten}|pw}« gewiss errathen haben, dass das vom Volkstheater\orgindex{Volkstheater@Volkstheater|pw} angenommene Stück\pwindex{Ehrlosen. Schauspiel in drei Acten@\emph{Die Ehrlosen. Schauspiel in drei Acten}|pwv} – – dasselbe ist, dasselbe, das Sie mir im vergangenen
                  Jahr so furchtbar \label{K_L03725-1v}\edtext{verdonnert
                  haben}{\lemma{\textnormal{\emph{verdonnert
                  haben}}}\Cendnote{\textnormal{Schnitzlers Kritik ist nicht überliefert, aber die Erschütterung Plessners\pwindex{Plessner, Elsa 22.08.1875 – 01.05.1932@\textsc{Plessner, Elsa} (22.08.1875 – 01.05.1932), \emph{Schriftsteller/Schriftstellerin}|pwk} darüber, vgl. Elsa Plessner an Arthur Schnitzler, 26. 1. 1899.}}}\label{K_L03725-1}. Darum hab ich
               auch letzthin Angst gehabt – es Ihnen zu gestehen. Für heute fühle ich
               mich so gewissermaßen gedrängt, Ihnen zu versichern, dass ich auch heute,
               nachdem man sich hier und \label{K_L03725-2v}\edtext{in Berlin\oindex{Berlin@\textbf{Berlin}, \emph{P.PPLC}|pw}}{\lemma{\textnormal{\emph{in Berlin}}}\Cendnote{\textnormal{In Berlin\oindex{Berlin@\textbf{Berlin}, \emph{P.PPLC}|pwk} kam das Schauspiel\pwindex{Ehrlosen. Schauspiel in drei Acten@\emph{Die Ehrlosen. Schauspiel in drei Acten}|pwkv} nicht zur Aufführung und es ist nicht bekannt, mit welchem Theater dort Plessner\pwindex{Plessner, Elsa 22.08.1875 – 01.05.1932@\textsc{Plessner, Elsa} (22.08.1875 – 01.05.1932), \emph{Schriftsteller/Schriftstellerin}|pwk} in Verhandlung stand.}}}\label{K_L03725-2} ziemlich viel
               von der Arbeit\pwindex{Ehrlosen. Schauspiel in drei Acten@\emph{Die Ehrlosen. Schauspiel in drei Acten}|pwv}{ }{\pb}verspricht, ziemlich im Klaren bin über den wahren literarischen Wert
               des Stückes\pwindex{Ehrlosen. Schauspiel in drei Acten@\emph{Die Ehrlosen. Schauspiel in drei Acten}|pwv} – d. h.
               dass meine Ansicht darüber nicht allzu sehr von der Ihren abweicht. Aber – Sie wissen
               beim Theater weiß man nie etwas – und hoffentlich wird nicht diese unsere wahre
               Meinung vom Publicum getheilt werden. Ich bitte Sie vielmals, das nicht für Arroganz
               oder Pose zu halten, dass ich Ihnen das sage – ich glaube, dass ich weder das Eine,
               noch das andere Ihnen gegenüber {\pb}nöthig habe. Dass ich Ihnen letzthin aus
               heiler Haut mein neues \label{K_L03725-3v}\edtext{Stück\pwindex{erste Kapitel. Schauspiel in drei Akten@\emph{Das erste Kapitel. Schauspiel in drei Akten}|pwv} schickte}{\lemma{\textnormal{\emph{Stück schickte}}}\Cendnote{\textnormal{Vgl. Elsa Plessner an Arthur Schnitzler, 9. 1. 1900.}}}\label{K_L03725-3}, soll Sie überzeugen, dass mich selbst ein eventueller Erfolg der »Ehrlosen\pwindex{Ehrlosen. Schauspiel in drei Acten@\emph{Die Ehrlosen. Schauspiel in drei Acten}|pw}« nicht auf den Holzweg locken soll, den
               ich damit eingeschlagen habe. Sie sehen – ich habe echte, aufrichtige\introOben{}n\introOben{}{ }\uline{literarischen}
               Ehrgeiz und wenn ich auch nicht den Heroismus besitze, mit einem, wenn auch
               minderwertigen Stücke\pwindex{Ehrlosen. Schauspiel in drei Acten@\emph{Die Ehrlosen. Schauspiel in drei Acten}|pwv} in dem doch
               ziemlich hochstehenden Theater\oindex{Volkstheater@\textbf{Volkstheater}, \emph{Theater (K.THE)}|pwv} aufgeführt zu werden – als eine teuflische Versuchung
               von mir zu {\pb}weisen, so weiß ich doch ganz gut, dass das äußerliche
               Emporkommen noch nichts bedeutet, wenn nicht – ja wenn nicht u. s. w.\pend
           
\pstart
           Was ich also
               Ihnen jetzt als Beichtgeheimnis anvertraue, soll mich nur in Ihren Augen reinwaschen
               und wenn Sie nicht schlecht von mir denken, so werden Sie sehr erfreuen Ihre Sie stets
               hochverehrende\pend
           \pstart \spacefill\mbox{Elsa Plessner}.\pend{}\selectlanguage{ngerman}\endnumbering\briefempfaengerindex{Schnitzler, Arthur@\textsc{Schnitzler, Arthur}!zzzPlessner, Elsa@\emph{von Elsa Plessner}!1900-02-261@{26. 2. 1900}|)be}\mylabel{L03725h}
\begin{anhang}
\end{anhang}\normalsize

\doendnotes{C}
\bigskip
\vfill

\clearpage

\footnotesize

\lohead{\textsc{register}}

% Definiere theindex-Environment komplett neu ohne reledmac
\makeatletter
\renewenvironment{theindex}{%
  \section*{\indexname}%
  \setlength{\parindent}{0pt}%
  \setlength{\parskip}{0pt plus 0.3pt}%
  \let\item\@idxitem
}{%
  \clearpage
}
\makeatother

\IfFileExists{\jobname-pw.ind}{\input{\jobname-pw.ind}}{}

\end{document}

      