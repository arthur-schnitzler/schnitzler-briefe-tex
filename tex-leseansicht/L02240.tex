%% latex-leseansicht-vorspann.tex
%% Vorspann für die Leseansicht.
%% Lädt die gemeinsame Datei latex-vorspann.tex mit nicht gesetztem Schalter.

\newif\ifkorrekturansicht
\korrekturansichtfalse

\input{../tex-inputs/latex-vorspann}


         
         \newcommand{\erwaehntePersonen}{Personen: Frieda Pollak}
         \newcommand{\erwaehnteInstitutionen}{}
         \newcommand{\erwaehnteOrte}{Orte: Altaussee, Bad Aussee}
         \newcommand{\erwaehnteWerke}{Werke: Der Sohn des Geisterkönigs}
               \section[Hugo von Hofmannsthal an Arthur Schnitzler, {[}25. 8. 1916{]}]{ Hugo von Hofmannsthal an Arthur Schnitzler, {[}25. 8. 1916{]}}\nopagebreak\mylabel{v}\rehead{ }\begin{ledgroupsized}[t]{13cm}\normalsize\beginnumbering \toendnotes[C]{\smallbreak\pagebreak[2]} \Standort{CUL, Schnitzler, B 43.}
\physDesc{Briefkarte
\newline{}Handschrift: schwarze Tinte, deutsche Kurrent
\newline{}Schnitzler: 1) mit Bleistift datiert: »25/8 16« und beschriftet: »\textsc{Aussee\oindex{Bad Aussee@\textbf{Bad Aussee}|pw}}« und »Hugo«  2) mit rotem Buntstift eine Unterstreichung\newline{}Ordnung: 1) mit Bleistift von Frieda Pollak\pwindex{Pollak, Frieda 08.12.1881 – 13.07.1937@\textsc{Pollak, Frieda} (08.12.1881 – 13.07.1937), \emph{Sekretärin}|pw} (?) mit dem Buchstaben »A« (Abgeschrieben/Abschrift) gekennzeichnet  2) mit Bleistift von unbekannter Hand nummeriert: »\strikeout{344}« 3) mit Bleistift von unbekannter Hand nummeriert:
                                    »354«}\buchAbdrucke{\weitereDrucke{Hugo von Hofmannsthal, Arthur Schnitzler: \emph{Briefwechsel}. Hg. Therese Nickl und Heinrich Schnitzler. Frankfurt am Main: \emph{S. Fischer} 1964, S. 280.} }\toendnotes[C]{\smallbreak}\pstart
           \raggedleft{}{\pb}Freitag.\pend
           \pstart{}mein guter Arthur\pend\pstart
           ich will Sie nicht bedrängen u. beläſtigen aber ich fühle wie woltätig mir – ſo oder
               ſo – die Möglichkeit Ihnen dieſe problematiſchen Fragmente\pwindex{Hofmannsthal, Hugo von 1874-02-01 – 1929-07-15@\textsc{Hofmannsthal, Hugo von} (1874-02-01 – 1929-07-15), \emph{Schriftsteller}!Sohn des Geisterkoenigs1917@\strich\emph{Der Sohn des Geisterkönigs} {[}1917{]}|pwv} vorzuleſen ſein wird. Ich werde dieſe vielleicht
               allzu gewagte Arbeit nachher entweder {\pb}weglegen oder mit größerer
               Zuverſicht wieder anpacken.\pend
           \pstart
           Wäre es zu denken daſs Sie dieſe 1½ Stunden in den allernächſten Tagen mir ſchenken
               könnten – in der Früh – am ſpäten Vormittag{[},{]} am Abend oder wann
                  i{\geminationm}er? \pend
           \pstart Herzlich Ihr\spacefill\mbox{Hugo.}\pend{}
         
         \endnumbering\mylabel{h}\end{ledgroupsized}  \newcommand{\dateiname}{L02240}\newcommand{\titel}{Hugo von Hofmannsthal an Arthur Schnitzler, [25. 8. 1916]}\newcommand{\editorInnen}{Martin Anton Müller und Gerd-Hermann Susen}%% latex-leseansicht-abspann.tex
%% Abspann für die Leseansicht.
%% Der Schalter \ifkorrekturansicht ist bereits durch den Vorspann gesetzt.

%% latex-abspann.tex
%% Gemeinsamer Abspann für Korrekturansicht und Leseansicht.
%% Setzt den Schalter \ifkorrekturansicht voraus (gesetzt in den
%% einbindenden Dateien latex-korrekturansicht-abspann.tex bzw.
%% latex-leseansicht-abspann.tex).
%% ---------------------------------------------------------------

\normalsize

% Das esempio-Environment wird nur in der Leseansicht benötigt
\ifkorrekturansicht\else
\newenvironment{esempio}[3]%
{
    \vspace{1.5ex}
    \rlap{\underline{#1}}
    \par
    \setlength{\parindent}{0cm}
    \nopagebreak
    \leftskip=#2cm
    \rightskip=#3cm
}
{
    \par
}
\fi

\doendnotes{C}
\bigskip
\vfill

\clearpage

\footnotesize

\ifkorrekturansicht
  \lohead{\textsc{register}}
\fi

% theindex-Environment neu definieren ohne reledmac
\makeatletter
\renewenvironment{theindex}{%
  \ifkorrekturansicht
    \section*{\indexname}%
  \else
    \subsubsection*{Index der erwähnten Entitäten}%
  \fi
  \setlength{\parindent}{0pt}%
  \setlength{\parskip}{0pt plus 0.3pt}%
  \let\item\@idxitem
}{%
  \ifkorrekturansicht\clearpage\fi
}
\makeatother

\IfFileExists{\jobname-pw.ind}{\input{\jobname-pw.ind}}{}

% Quellenangabe nur in der Leseansicht
\ifkorrekturansicht\else
% Fallback-Definitionen, falls die .tex-Datei \titel etc. nicht gesetzt hat
\providecommand{\titel}{}
\providecommand{\editorInnen}{}
\providecommand{\dateiname}{\jobname}

\vspace{3cm}

\vfill

\footnotesize
\textsc{Quelle}: \titel. Herausgegeben von {\editorInnen}. In: \emph{Arthur Schnitzler: Briefwechsel mit Autorinnen und Autoren}.
 Digitale Edition, https://schnitzler-briefe.acdh.oeaw.ac.at/{\dateiname}.html (Stand \today)
\fi

\end{document}


      