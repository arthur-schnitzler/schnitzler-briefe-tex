%% latex-korrekturansicht-vorspann.tex
%% Vorspann für die Korrekturansicht.
%% Lädt die gemeinsame Datei latex-vorspann.tex mit gesetztem Schalter.

\newif\ifkorrekturansicht
\korrekturansichttrue

\input{../tex-inputs/latex-vorspann}


\section[Hugo von Hofmannsthal an Arthur Schnitzler, {[}13. 8. 1899{]}]{L00958 Hugo von Hofmannsthal an Arthur Schnitzler, {[}13. 8. 1899{]}}
\nopagebreak\mylabel{L00958v}
\rehead{ }\normalsize\beginnumbering\briefempfaengerindex{Schnitzler, Arthur@\textsc{Schnitzler, Arthur}!zzzHofmannsthal, Hugo von@\emph{von Hugo von Hofmannsthal}!1899-08-131@{{[}13. 8. 1899{]}}|(be}
\toendnotes[C]{\smallbreak\pagebreak[2]}\Standort{CUL, Schnitzler, B 43.}
\physDesc{Brief, 1 Blatt, 2 Seiten, 374 Zeichen
\newline{}Handschrift: schwarze Tinte, deutsche Kurrent
\newline{}Schnitzler: mit Bleistift datiert: »13/8 99« 
\newline{}Ordnung: mit Bleistift von unbekannter Hand nummeriert:
                                    »155« }
\buchAbdrucke{\weitereDrucke{Hugo von Hofmannsthal, Arthur Schnitzler: \emph{Briefwechsel}. Frankfurt am Main: \emph{S. Fischer} 1964, S. 129.} }
\pstart
           \centering{}{\pb}Sonntg\pend
           \vspace{0.5em}
\pstart
           Fand Euer geſtriges Telegra{\geminationm} erſt abends, konnte erſt
               heute Bozen\oindex{Bozen@\textbf{Bozen}, \emph{P.PPLA2}|pw} telegrafieren, erhielt dann Eure
               zweite Depeſche. Möchte Mittwoch erſter Zug Iſchl\oindex{Bad Ischl@\textbf{Bad Ischl}, \emph{P.PPL}|pw}
                  anko{\geminationm}en, Tag mit Ihnen verbringen, Rad mitnehmen,
               abends Auſſee\oindex{Bad Aussee@\textbf{Bad Aussee}, \emph{P.PPLA3}|pw} zurück, da ja Richard\pwindex{Beer-Hofmann, Richard 1866-07-11 – 1945-09-26@\textsc{Beer-Hofmann, Richard} (1866-07-11 – 1945-09-26), \emph{Schriftsteller/Schriftstellerin}|pw}{ }{\pb}Do{\geminationn}erstag Auſſee\oindex{Bad Aussee@\textbf{Bad Aussee}, \emph{P.PPLA3}|pw} ko{\geminationm}t.\pend
           
\pstart
           Vielleicht fahren wir zuſa{\geminationm}en Hallſtadt\oindex{Hallstatt@\textbf{Hallstatt}, \emph{P.PPL}|pw}? oder Sie ko{\geminationm}en ſchon
               Mittwoch Auſſee\oindex{Bad Aussee@\textbf{Bad Aussee}, \emph{P.PPLA3}|pw}? Aber was bei ſchlechtem
               Wetter?\pend
           
\pstart
           Herzlich{\\[\baselineskip]}\spacefill\mbox{Hugo.}\pend
           \leftskip=0em{}\selectlanguage{ngerman}\endnumbering\briefempfaengerindex{Schnitzler, Arthur@\textsc{Schnitzler, Arthur}!zzzHofmannsthal, Hugo von@\emph{von Hugo von Hofmannsthal}!1899-08-131@{{[}13. 8. 1899{]}}|)be}\mylabel{L00958h}  \normalsize

\doendnotes{C}
\bigskip
\vfill

\clearpage

\footnotesize

\lohead{\textsc{register}}

% Definiere theindex-Environment komplett neu ohne reledmac
\makeatletter
\renewenvironment{theindex}{%
  \section*{\indexname}%
  \setlength{\parindent}{0pt}%
  \setlength{\parskip}{0pt plus 0.3pt}%
  \let\item\@idxitem
}{%
  \clearpage
}
\makeatother

\IfFileExists{\jobname-pw.ind}{\input{\jobname-pw.ind}}{}

\end{document}

      