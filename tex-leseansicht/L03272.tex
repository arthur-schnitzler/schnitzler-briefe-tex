%% latex-korrekturansicht-vorspann.tex
%% Vorspann für die Korrekturansicht.
%% Lädt die gemeinsame Datei latex-vorspann.tex mit gesetztem Schalter.

\newif\ifkorrekturansicht
\korrekturansichttrue

\input{../tex-inputs/latex-vorspann}


\section[ Felix Salten an Arthur Schnitzler, 21. 8. 1897]{L03272 Felix Salten an Arthur Schnitzler, 21. 8. 1897}
\nopagebreak\mylabel{L03272v}
\rehead{ }\normalsize\beginnumbering\briefempfaengerindex{Schnitzler, Arthur@\textsc{Schnitzler, Arthur}!zzzSalten, Felix@\emph{von Felix Salten}!1897-08-211@{21. 8. 1897}|(be}
\toendnotes[C]{\smallbreak\pagebreak[2]}\Standort{CUL, Schnitzler, B 89, A 2.}
\physDesc{Postkarte, 339 Zeichen
\newline{}Handschrift: Bleistift, lateinische Kurrent
\newline{}Versand: Stempel: »\nobreak{}\oindex{Salzburg@\textbf{Salzburg}, \emph{A.ADM2}|pwk}Salzburg-Stadt, 21/8 \textcolor{gray}{9}7\nobreak{}«. Stempel: »\nobreak{}\oindex{Bad Ischl@\textbf{Bad Ischl}, \emph{P.PPL}|pwk}Ischl, 21. 8. 97, 10–11 \textcolor{gray}{N}\nobreak{}«.  
\newline{}Ordnung: mit Bleistift von unbekannter Hand nummeriert: »95« }\toendnotes[C]{\smallbreak}\pstart{}{\pb}Herrn D\textsuperscript{r} Arthur Schnitzler \pend{}\pstart{}Ischl\oindex{Bad Ischl@\textbf{Bad Ischl}, \emph{P.PPL}|pw}\pend{}\pstart{}Pension Rudolfshöhe\oindex{Hotel und Pension Rudolfshoehe (Leopold Petter)@\textbf{Hotel und Pension Rudolfshöhe (Leopold Petter)}, \emph{Hotel (K.HTL)}|pw}\pend{}{\bigskip}\vspace{1em}
\pstart
           \noindent{}{\pb}lieber Arthur, bin Mittwoch mit Van Jung\pwindex{Van-Jung, Leo 15.10.1866 – 02.07.1939@\textsc{Van-Jung, Leo} (15.10.1866 – 02.07.1939), \emph{Gesangspädagoge/Gesangspädagogin, Mathematiker/Mathematikerin}|pw} leider zu spät \label{K_L03272-1v}\edtext{hereingekommen}{\lemma{\textnormal{\emph{hereingekommen}}}\Cendnote{\textnormal{Vermutlich kamen die beiden aus Pressbaum\oindex{Pressbaum@\textbf{Pressbaum}, \emph{P.PPLA3}|pwk} nach Wien\oindex{Wien@\textbf{Wien}, \emph{A.ADM2}|pwk} zurück (vgl. Felix Salten an Arthur Schnitzler, 13. 7. 1897). Am Folgetag, dem 18. 8. 1897, war
                     Schnitzler nach Ischl\oindex{Bad Ischl@\textbf{Bad Ischl}, \emph{P.PPL}|pwk} gereist, sodass sie sich verpasst hatten.}}}\label{K_L03272-1} und habe
               sehr bedauert, Sie nicht mehr sehen zu können. Bin seit heute{ }früh{ }\label{K_L03272-2v}\edtext{hier\oindex{Salzburg@\textbf{Salzburg}, \emph{A.ADM2}|pwv}}{\lemma{\textnormal{\emph{hier}}}\Cendnote{\textnormal{Salzburg\oindex{Salzburg@\textbf{Salzburg}, \emph{A.ADM2}|pwk}}}}\label{K_L03272-2}, Linzerstraße 74\oindex{Linzer Gasse@\textbf{Linzer Gasse}, \emph{Straße (K.STR)}|pw} bei Frau Sandholzer\pwindex{Sandholzer, Maria @\textsc{Sandholzer, Maria}, \emph{männliche Hebamme/Hebamme}|pw}.\pend
           
\pstart
           Vielleicht \label{K_L03272-3v}\edtext{kommen Sie einmal her, oder
               ich nach Ischl\oindex{Bad Ischl@\textbf{Bad Ischl}, \emph{P.PPL}|pw}}{\lemma{\textnormal{\emph{kommen … Ischl}}}\Cendnote{\textnormal{Dazu kam es nicht,  vgl. Felix Salten an Arthur Schnitzler, 31. 8. 1897.
               }}}\label{K_L03272-3}. Jedenfalls verständigen wir uns vorher davon.\pend
           
\pstart
           Herzlich {\\[\baselineskip]}\spacefill\mbox{Salten}\pend
           \leftskip=0em{}\selectlanguage{ngerman}\endnumbering\briefempfaengerindex{Schnitzler, Arthur@\textsc{Schnitzler, Arthur}!zzzSalten, Felix@\emph{von Felix Salten}!1897-08-211@{21. 8. 1897}|)be}\mylabel{L03272h}  \normalsize

\doendnotes{C}
\bigskip
\vfill

\clearpage

\footnotesize

\lohead{\textsc{register}}

% Definiere theindex-Environment komplett neu ohne reledmac
\makeatletter
\renewenvironment{theindex}{%
  \section*{\indexname}%
  \setlength{\parindent}{0pt}%
  \setlength{\parskip}{0pt plus 0.3pt}%
  \let\item\@idxitem
}{%
  \clearpage
}
\makeatother

\IfFileExists{\jobname-pw.ind}{\input{\jobname-pw.ind}}{}

\end{document}

      