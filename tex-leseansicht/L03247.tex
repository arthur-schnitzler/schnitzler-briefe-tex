%% latex-korrekturansicht-vorspann.tex
%% Vorspann für die Korrekturansicht.
%% Lädt die gemeinsame Datei latex-vorspann.tex mit gesetztem Schalter.

\newif\ifkorrekturansicht
\korrekturansichttrue

\input{../tex-inputs/latex-vorspann}


\section[ Paul Goldmann an Arthur Schnitzler, 10. 7. 1906]{L03247 Paul Goldmann an Arthur Schnitzler, 10. 7. 1906}
\nopagebreak\mylabel{L03247v}
\rehead{ }\normalsize\beginnumbering\briefempfaengerindex{Schnitzler, Arthur@\textsc{Schnitzler, Arthur}!zzzGoldmann, Paul@\emph{von Paul Goldmann}!1906-07-101@{10. 7. 1906}|(be}
\toendnotes[C]{\smallbreak\pagebreak[2]}\Standort{DLA, A:Schnitzler, HS.NZ85.1.3175.}
\physDesc{Postkarte, 529 Zeichen
\newline{}Handschrift: 1) blaue Tinte, deutsche Kurrent\hspace{1em}2) blaue Tinte, lateinische Kurrent (\noindent{}Adresse)\hspace{1em}
\newline{}Versand: 1) Stempel: »\nobreak{}\oindex{Berlin@\textbf{Berlin}, \emph{P.PPLC}|pwk}Berlin, S.W. 11, 10. 7. 06, 12–1N.\nobreak{}«.   2) Stempel: »\nobreak{}\oindex{Helsingør@\textbf{Helsingør}, \emph{P.PPLA2}|pwk}Helsingør, 11. 7. 06, 9–10F\nobreak{}«. 
\newline{}Schnitzler: mit Bleistift das Jahr »906« vermerkt }\toendnotes[C]{\smallbreak}\pstart{}{\pb}Welt-\textcolor{gray}{\textbf{Poſtkarte}}\pend{}\pstart{}Herrn\pend{}\pstart{}Dr. Arthur Schnitzler \introOben{}(aus Wien\oindex{Wien@\textbf{Wien}, \emph{A.ADM2}|pw})\introOben{}\pend{}\pstart{}Marienlyst\oindex{Marienlyst@\textbf{Marienlyst}, \emph{S.EST}|pw}\pend{}\pstart{}Dänemark\oindex{Daenemark@\textbf{Dänemark}, \emph{A.PCLI}|pw}\pend{}{\bigskip}\vspace{1em}
\pstart
           \noindent{}{\pb}Berlin\oindex{Berlin@\textbf{Berlin}, \emph{P.PPLC}|pw}, 10. Juli.
               Herzlichen Dank, mein lieber Freund, für Deine Karte!
               Ich freue mich, daß es Euch\pwindex{Schnitzler, Olga 17.01.1882 – 13.01.1970@\textsc{Schnitzler, Olga} (17.01.1882 – 13.01.1970), \emph{Schauspieler/Schauspielerin, Sänger/Sängerin}|pwv}
               gut geht u. daß es Euch in \label{K_L03247-1v}\edtext{Dänemark\oindex{Daenemark@\textbf{Dänemark}, \emph{A.PCLI}|pw}}{\lemma{\textnormal{\emph{Dänemark}}}\Cendnote{\textnormal{Schnitzler war vom 28. 6. 1906 bis zum 11. 8. 1906 in Dänemark\oindex{Daenemark@\textbf{Dänemark}, \emph{A.PCLI}|pwk}, hauptsächlich in Marienlyst\oindex{Marienlyst@\textbf{Marienlyst}, \emph{S.EST}|pwk} und ein paar Tage in Kopenhagen\oindex{Kopenhagen@\textbf{Kopenhagen}, \emph{P.PPLC}|pwk}.}}}\label{K_L03247-1} wieder ſo gut gefällt. Gern hätte ich
               Dich auf der \label{K_L03247-2v}\edtext{Durchreiſe}{\lemma{\textnormal{\emph{Durchreiſe}}}\Cendnote{\textnormal{Schnitzler hatte sich am 26. 6. 1906 und am 27. 6. 1906 in Berlin\oindex{Berlin@\textbf{Berlin}, \emph{P.PPLC}|pwk} aufgehalten.}}}\label{K_L03247-2} in Berlin\oindex{Berlin@\textbf{Berlin}, \emph{P.PPLC}|pw} geſehen; aber ich begreife, daß die Zeit dazu zu knapp
               war, u. hoffe auf ein baldiges \label{K_L03247-3v}\edtext{Wiederſehen}{\lemma{\textnormal{\emph{Wiederſehen}}}\Cendnote{\textnormal{Schnitzler und Goldmann\pwindex{Goldmann, Paul 31.01.1865 – 25.09.1935@\textsc{Goldmann, Paul} (31.01.1865 – 25.09.1935), \emph{Schriftsteller/Schriftstellerin, Journalist/Journalistin}|pwk} trafen sich erst am 24. 5. 1907 in Wien\oindex{Wien@\textbf{Wien}, \emph{A.ADM2}|pwk} wieder.}}}\label{K_L03247-3} bei günſtigerer Gelegenheit.
               Ich wünſche Dir u. Deiner Frau\pwindex{Schnitzler, Olga 17.01.1882 – 13.01.1970@\textsc{Schnitzler, Olga} (17.01.1882 – 13.01.1970), \emph{Schauspieler/Schauspielerin, Sänger/Sängerin}|pwv} auch weiterhin einen frohen u. behaglichen Verlauf der Sommerreiſe u.
               bin mit vielen herzlichen Grüßen Dein {\\}\spacefill\mbox{Paul Goldmnn}\pend
           \selectlanguage{ngerman}\endnumbering\briefempfaengerindex{Schnitzler, Arthur@\textsc{Schnitzler, Arthur}!zzzGoldmann, Paul@\emph{von Paul Goldmann}!1906-07-101@{10. 7. 1906}|)be}\mylabel{L03247h}  \normalsize

\doendnotes{C}
\bigskip
\vfill

\clearpage

\footnotesize

\lohead{\textsc{register}}

% Definiere theindex-Environment komplett neu ohne reledmac
\makeatletter
\renewenvironment{theindex}{%
  \section*{\indexname}%
  \setlength{\parindent}{0pt}%
  \setlength{\parskip}{0pt plus 0.3pt}%
  \let\item\@idxitem
}{%
  \clearpage
}
\makeatother

\IfFileExists{\jobname-pw.ind}{\input{\jobname-pw.ind}}{}

\end{document}

      