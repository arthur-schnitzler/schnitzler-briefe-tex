%% latex-leseansicht-vorspann.tex
%% Vorspann für die Leseansicht.
%% Lädt die gemeinsame Datei latex-vorspann.tex mit nicht gesetztem Schalter.

\newif\ifkorrekturansicht
\korrekturansichtfalse

\input{../tex-inputs/latex-vorspann}


\section[Albert Ehrenstein an Arthur Schnitzler, 6. 11. 1909]{L01884 Albert Ehrenstein an Arthur Schnitzler, 6. 11. 1909}
\nopagebreak\mylabel{L01884v}
\rehead{ }\normalsize\beginnumbering\briefempfaengerindex{Schnitzler, Arthur@\textsc{Schnitzler, Arthur}!zzzEhrenstein, Albert@\emph{von Albert Ehrenstein}!1909-11-062@{06. 11. 1909}|(be}
\toendnotes[C]{\smallbreak\pagebreak[2]}
\correspDesc{Versand  durch Albert Ehrenstein am 06. 11. 1909 in Wien
\newline{}Erhalt  durch Arthur Schnitzler im Zeitraum [6. 11. 1909
                  – 10. 11. 1909?] in Wien}\toendnotes[C]{\smallbreak}
\Standort{CUL, Schnitzler, B 30.}
\physDesc{Brief, 1 Blatt, 3 Seiten, 1692 Zeichen
\newline{}Handschrift: schwarze Tinte, deutsche Kurrent}
\buchAbdrucke{\weitereDrucke{Albert Ehrenstein: \emph{Briefe}. Herausgegeben von Hanni Mittelmann. München: \emph{Boer} 1989, S. 34–35 (Werke, 1).} }
\pstart
           
\pstart
           {\pb}\textsc{Albert Ehrenstein}\pend
           
\pstart
           \raggedleft{}6. XI. 09.\pend
           \pend
           
\pstart
           \textsc{XVI. Ottakringerstr 114\oindex{Wien@\textbf{Wien}!XVI., Ottakring@\textbf{XVI., Ottakring}!Ottakringer Straße@\textbf{Ottakringer Straße}, \emph{Straße}|pw}\oindex{Wien@\textbf{Wien}!XVII., Hernals@\textbf{XVII., Hernals}!Ottakringer Straße@\textbf{Ottakringer Straße}, \emph{Straße}|pw}.}\pend
           
\pstart{}\textsc{Sehr geehrter Herr Doktor,}\pend\vspace{0.5em}
\pstart
           nun habe ich auf meiner Tournee durch die Schattenſeiten des Metiers zu meiner nicht
               ganz gelinden Verzweiflung auch noch die kennen gelernt, welche{ }ſich in
               Maſchinenfräuleins und deren Schreibfehlern verkörpert. Von den Arbeiten, die{ }ſich in
               dieſer Neugeſtaltung bei Ihnen,{ }ſehr geehrter Herr Doktor, einfinden,{ }ſind Ihnen nur
                  »Mitgefühl\pwindex{Ehrenstein, Albert 23.\,12.\,1886 Wien – 8.\,4.\,1950 New York City@\textsc{Ehrenstein, Albert} (23.\,12.\,1886 Wien – 8.\,4.\,1950 New York City), \emph{Schriftsteller}!Mitgefühl@\strich\emph{Mitgefühl}|pw}« und »Saccumum\pwindex{Ehrenstein, Albert 23.\,12.\,1886 Wien – 8.\,4.\,1950 New York City@\textsc{Ehrenstein, Albert} (23.\,12.\,1886 Wien – 8.\,4.\,1950 New York City), \emph{Schriftsteller}!Saccumum@\strich\emph{Saccumum}|pw}« unbekannt.\pend
           
\pstart
           Da ich keine Ahnung habe, was für Sachen einem Verlegerherzen goldhaltig{ }ſcheinen
               können, habe ich keine beſondere Auswahl {\pb}unter meinen Produkten getroffen – wahrſcheinlich iſt{ }ſo etwas wie eine Sichtung
               auch kaum durchführbar. Ich wenigſtens habe nicht herausfinden können, welches die
               langweiligſten{ }ſind – es tut einem wirklich die Wahl weh. Gäben die Götter, daß der
               Herr Ko{\geminationm}erzienrat Fiſcher\pwindex{Fischer, Samuel 24.\,12.\,1859 Liptovský Mikuláš – 15.\,10.\,1934 Berlin@\textsc{Fischer, Samuel} (24.\,12.\,1859 Liptovský Mikuláš – 15.\,10.\,1934 Berlin), \emph{Verleger}|pw} dieſen angeblichen Novellenzyklus akzeptiert oder – was ihn ja
               nichts koſten würde – irgendetwas in der Rundſchau\pwindex{neue Rundschau@\emph{Die neue Rundschau}|pw} bringt. Es wäre das für mich eine kleine Verſicherung gegen
               gewiſſe Stupiditäten der Außenwelt, die{ }ſich demnächſt in zudringlichen Fragen
               hiſtoriſchen Charakters manifeſtieren dürften.\pend
           
\pstart
           {\pb}Und ein etwaiger Mißerfolg wäre im
               Vorhinein kompenſiert.\pend
           
\pstart
           Sollte eine Art von grauſamem, aber vielleicht logiſchem und gerechtem Parallelismus
               mich auf beiden Seiten zuſchanden werden laſſen, meinen Erfahrungen gemäß nicht bloß
               auf Ihren Empfehlungen,{ }ſondern auch auf meinen Leiſtungen{ }ſo etwas wie ein Fluch
               liegen, bleibe ich Ihnen,{ }ſehr geehrter Herr Doktor, noch immer äußerſt dankbar für{ }ſo manches frühere, namentlich für Ihre harten Worte über mein Übelwollen – denn auch
               eine derartige Frottierung hatte äußerſt nötig Ihr ergebenſter\pend
           \pstart \spacefill\mbox{Albert Ehrenstein.}\pend{}\selectlanguage{ngerman}\endnumbering\briefempfaengerindex{Schnitzler, Arthur@\textsc{Schnitzler, Arthur}!zzzEhrenstein, Albert@\emph{von Albert Ehrenstein}!1909-11-062@{06. 11. 1909}|)be}\mylabel{L01884h}  \newcommand{\dateiname}{L01884}\newcommand{\titel}{Albert Ehrenstein an Arthur Schnitzler, 6. 11. 1909}\newcommand{\editorInnen}{Martin Anton Müller und Gerd-Hermann Susen}%% latex-leseansicht-abspann.tex
%% Abspann für die Leseansicht.
%% Der Schalter \ifkorrekturansicht ist bereits durch den Vorspann gesetzt.

%% latex-abspann.tex
%% Gemeinsamer Abspann für Korrekturansicht und Leseansicht.
%% Setzt den Schalter \ifkorrekturansicht voraus (gesetzt in den
%% einbindenden Dateien latex-korrekturansicht-abspann.tex bzw.
%% latex-leseansicht-abspann.tex).
%% ---------------------------------------------------------------

\normalsize

% Das esempio-Environment wird nur in der Leseansicht benötigt
\ifkorrekturansicht\else
\newenvironment{esempio}[3]%
{
    \vspace{1.5ex}
    \rlap{\underline{#1}}
    \par
    \setlength{\parindent}{0cm}
    \nopagebreak
    \leftskip=#2cm
    \rightskip=#3cm
}
{
    \par
}
\fi

\doendnotes{C}
\bigskip
\vfill

\clearpage

\footnotesize

\ifkorrekturansicht
  \lohead{\textsc{register}}
\fi

% theindex-Environment neu definieren ohne reledmac
\makeatletter
\renewenvironment{theindex}{%
  \ifkorrekturansicht
    \section*{\indexname}%
  \else
    \subsubsection*{Index der erwähnten Entitäten}%
  \fi
  \setlength{\parindent}{0pt}%
  \setlength{\parskip}{0pt plus 0.3pt}%
  \let\item\@idxitem
}{%
  \ifkorrekturansicht\clearpage\fi
}
\makeatother

\IfFileExists{\jobname-pw.ind}{\input{\jobname-pw.ind}}{}

% Quellenangabe nur in der Leseansicht
\ifkorrekturansicht\else
% Fallback-Definitionen, falls die .tex-Datei \titel etc. nicht gesetzt hat
\providecommand{\titel}{}
\providecommand{\editorInnen}{}
\providecommand{\dateiname}{\jobname}

\vspace{3cm}

\vfill

\footnotesize
\textsc{Quelle}: \titel. Herausgegeben von {\editorInnen}. In: \emph{Arthur Schnitzler: Briefwechsel mit Autorinnen und Autoren}.
 Digitale Edition, https://schnitzler-briefe.acdh.oeaw.ac.at/{\dateiname}.html (Stand \today)
\fi

\end{document}


