%% latex-korrekturansicht-vorspann.tex
%% Vorspann für die Korrekturansicht.
%% Lädt die gemeinsame Datei latex-vorspann.tex mit gesetztem Schalter.

\newif\ifkorrekturansicht
\korrekturansichttrue

\input{../tex-inputs/latex-vorspann}


\section[ Paul Goldmann an Olga und Elisabeth Gussmann, 10. 12. {[}1901{]}]{L03536 Paul Goldmann an Olga und Elisabeth Gussmann, 10. 12. {[}1901{]}}
\nopagebreak\mylabel{L03536v}
\rehead{ }\normalsize\beginnumbering\briefempfaengerindex{Steinrueck, Elisabeth@\textsc{Steinrück, Elisabeth}!zzzGoldmann, Paul@\emph{von Paul Goldmann}!1901-12-101@{10. 12. {[}1901{]}}|(be}\briefempfaengerindex{Schnitzler, Olga@\textsc{Schnitzler, Olga}!zzzGoldmann, Paul@\emph{von Paul Goldmann}!1901-12-101@{10. 12. {[}1901{]}}|(be}
\toendnotes[C]{\smallbreak\pagebreak[2]}\Standort{DLA, A:Schnitzler, HS.NZ85.1.5247.}
\physDesc{Brief, 1 Blatt, 4 Seiten, 1609 Zeichen
\newline{}Handschrift: blaue Tinte, deutsche Kurrent}\toendnotes[C]{\smallbreak}
\pstart
           \raggedleft{}{\pb}\textcolor{gray}{\textbf{DESSAUERSTRASSE 19\oindex{Dessauer Strasse@\textbf{Dessauer Straße}, \emph{Straße (K.STR)}|pw}}}\pend
           
\pstart
           Berlin\oindex{Berlin@\textbf{Berlin}, \emph{P.PPLC}|pw}, 10. Dezember.\pend
           
\pstart\center{}Liebes Fräulein \textsc{Olga},\pend\vspace{0.5em}
\pstart
           Haben Sie vielen Dank für Ihren lieben Brief! Antworten kann ich Ihnen noch nicht. Es
               iſt nicht mit Worten zu beſchreiben, was ich zu thun habe! Ich will Ihnen nur ſagen,
               wie ſehr mich Ihre Zeilen gefreut \strikeout{h\textcolor{gray}{a}} haben, in denen Sie als das liebe Wien\oindex{Wien@\textbf{Wien}, \emph{A.ADM2}|pw}er
               Mädel erſcheinen, als das ich Sie kenne. Warum man weinen muß, wenn \textsc{Hauptmann\pwindex{Hauptmann, Gerhart 15.11.1862 – 06.06.1946@\textsc{Hauptmann, Gerhart} (15.11.1862 – 06.06.1946), \emph{Schriftsteller/Schriftstellerin}|pw}} ein \label{K_L03536-1v}\edtext{ſchlechtes Stück\pwindex{rothe Hahn. Tragikomoedie in vier Akten@\emph{Der rothe Hahn. Tragikomödie in vier Akten}|pwuv}}{\lemma{\textnormal{\emph{ſchlechtes Stück}}}\Cendnote{\textnormal{\emph{Der rothe Hahn}\pwindex{rothe Hahn. Tragikomoedie in vier Akten@\emph{Der rothe Hahn. Tragikomödie in vier Akten}|pwk}, siehe Paul Goldmann an Arthur Schnitzler, 29. 11. [1901].}}}\label{K_L03536-1} ſchreibt, iſt mir {\pb}zwar unklar, aber \label{K_L03536-2v}\edtext{über \textsc{Hauptmann\pwindex{Hauptmann, Gerhart 15.11.1862 – 06.06.1946@\textsc{Hauptmann, Gerhart} (15.11.1862 – 06.06.1946), \emph{Schriftsteller/Schriftstellerin}|pw}} wollen wir nicht mehr miteinander reden}{\lemma{\textnormal{\emph{über … reden}}}\Cendnote{\textnormal{Vgl. Paul Goldmann an Olga Gussmann, 15. 11. [1901].
               }}}\label{K_L03536-2}. Bezüglich des dritten Akt\pwindex{Hoffmanns Erzaehlungen@\emph{Hoffmanns Erzählungen}|pwv}es von \label{K_L03536-3v}\edtext{\textsc{Hoffmanns} Erzählungen\pwindex{Hoffmanns Erzaehlungen@\emph{Hoffmanns Erzählungen}|pw}}{\lemma{\textnormal{\emph{Hoffmanns Erzählungen}}}\Cendnote{\textnormal{Das deutet darauf hin, dass 
                  Olga Gussmann\pwindex{Schnitzler, Olga 17.01.1882 – 13.01.1970@\textsc{Schnitzler, Olga} (17.01.1882 – 13.01.1970), \emph{Schauspieler/Schauspielerin, Sänger/Sängerin}|pwk} die Oper\pwindex{Hoffmanns Erzaehlungen@\emph{Hoffmanns Erzählungen}|pwkv} am 29. 11. 1901 gemeinsam
                  mit Schnitzler besucht hat.}}}\label{K_L03536-3} bin ich ganz
               Ihrer Anſicht. Ich habe ihn immer für das ſchönſte gehalten, wenn auch die \textsc{Barcarole\pwindex{Barcarole@\emph{Barcarole}|pw}} mein Lieblingsſtück bleibt. Nur \textsc{Arthur} hat, wie Sie ſich erinnern werden, die ganze Oper\pwindex{Hoffmanns Erzaehlungen@\emph{Hoffmanns Erzählungen}|pwv} als talentloſes Machwerk \label{K_L03536-4v}\edtext{bezeichnet}{\lemma{\textnormal{\emph{bezeichnet}}}\Cendnote{\textnormal{Am 28. 11. 1900 hatten 
                  Schnitzler und Goldmann\pwindex{Goldmann, Paul 31.01.1865 – 25.09.1935@\textsc{Goldmann, Paul} (31.01.1865 – 25.09.1935), \emph{Schriftsteller/Schriftstellerin, Journalist/Journalistin}|pwk} die Oper\pwindex{Hoffmanns Erzaehlungen@\emph{Hoffmanns Erzählungen}|pwkv} gemeinsam besucht und danach
                  noch gemeinsam gegessen. Vermutlich war auch Olga Gussmann\pwindex{Schnitzler, Olga 17.01.1882 – 13.01.1970@\textsc{Schnitzler, Olga} (17.01.1882 – 13.01.1970), \emph{Schauspieler/Schauspielerin, Sänger/Sängerin}|pwk} dabei.}}}\label{K_L03536-4}
               und hat dadurch wieder bewieſen, daß er vom Theater nichts verſteht.\pend
           
\pstart
           \textsc{Alfred Gold\pwindex{Gold, Alfred 28.06.1874 – 24.10.1958@\textsc{Gold, Alfred} (28.06.1874 – 24.10.1958), \emph{Schriftsteller/Schriftstellerin, Journalist/Journalistin, Kunsthändler/Kunsthändlerin}|pw}}, der verworrene {\pb}und alberne Literatur-Lausbub\pwindex{Gold, Alfred 28.06.1874 – 24.10.1958@\textsc{Gold, Alfred} (28.06.1874 – 24.10.1958), \emph{Schriftsteller/Schriftstellerin, Journalist/Journalistin, Kunsthändler/Kunsthändlerin}|pwv}, ein \textsc{\begin{otherlanguage}{french}Protégé\end{otherlanguage}} der Frau\pwindex{Mamroth, Johanna 1872-05-19 – 1910-09-12@\textsc{Mamroth, Johanna} (1872-05-19 – 1910-09-12)|pwv} meines Onkels\pwindex{Mamroth, Fedor 21.02.1851 – 25.06.1907@\textsc{Mamroth, Fedor} (21.02.1851 – 25.06.1907), \emph{Journalist/Journalistin, Kritiker/Kritikerin}|pwv}, iſt von meinem Onkel\pwindex{Mamroth, Fedor 21.02.1851 – 25.06.1907@\textsc{Mamroth, Fedor} (21.02.1851 – 25.06.1907), \emph{Journalist/Journalistin, Kritiker/Kritikerin}|pwv} als Berlin\oindex{Berlin@\textbf{Berlin}, \emph{P.PPLC}|pw}er Feuilleton-Correſpondent der Frankfurter Zeitung\orgindex{Frankfurter Zeitung@Frankfurter Zeitung|pw} engagirt worden!!!\pend
           
\pstart
           Laſſen Sie es ſich gut gehen in Ihrer \label{K_L03536-5v}\edtext{neuen Penſion\oindex{?? [Wohnung von Olga und Elisabeth Gussmann, 1901/1902]@\textbf{?? [Wohnung von Olga und Elisabeth Gussmann, 1901/1902]}, \emph{Wohngebäude (K.WHS)}|pwv}}{\lemma{\textnormal{\emph{neuen Penſion}}}\Cendnote{\textnormal{Im 
               Frühling 1901 waren Olga\pwindex{Schnitzler, Olga 17.01.1882 – 13.01.1970@\textsc{Schnitzler, Olga} (17.01.1882 – 13.01.1970), \emph{Schauspieler/Schauspielerin, Sänger/Sängerin}|pwk} und Elisabeth Gussmann\pwindex{Steinrueck, Elisabeth 19.11.1885 – 07.04.1920@\textsc{Steinrück, Elisabeth} (19.11.1885 – 07.04.1920)|pwk}
               in die Grünentorgasse\oindex{Gruenentorgasse@\textbf{Grünentorgasse}, \emph{Straße (K.STR)}|pwk} gezogen. Vermutlich seit November 1901 war  Olga\pwindex{Schnitzler, Olga 17.01.1882 – 13.01.1970@\textsc{Schnitzler, Olga} (17.01.1882 – 13.01.1970), \emph{Schauspieler/Schauspielerin, Sänger/Sängerin}|pwk}
               schwanger (vgl. A. S.: \emph{Tagebuch}, 10. 11. 1901). Die vorliegende Stelle deutet auf eine neue Unterkunft, die sie bis zur Übersiedelung in die Hauptstraße 56 in Hinterbrühl\oindex{Hauptstrasse 56@\textbf{Hauptstraße 56}, \emph{Wohngebäude (K.WHS)}|pwk} am 21. 3. 1902
               bewohnte.}}}\label{K_L03536-5} mit den \label{K_L03536-6v}\edtext{\textsc{\begin{otherlanguage}{english}new style\end{otherlanguage}}-Möbeln}{\lemma{\textnormal{\emph{new style-Möbeln}}}\Cendnote{\textnormal{›\begin{otherlanguage}{english}New Style\end{otherlanguage}‹ ist synonym mit
                  \begin{otherlanguage}{french}l’art nouveau\end{otherlanguage}/Jugendstil.}}}\label{K_L03536-6} und ſeien Sie (bis ich Ihnen ausführlich ſchreibe) einſtweilen
                  herzlich\uuline{ſt} (nicht herzl\uuline{ich}, wie Sie ſchreiben) gegrüßt von Ihrem getreuen{\\[\baselineskip]}\spacefill\mbox{Paul Goldmann.}\pend
           \leftskip=0em{}\selectlanguage{ngerman}\vspace{1em}{\vspace{1\baselineskip}}
\pstart
           \noindent{}{\pb}Liebes Fräulein \textsc{Liesl}, der unglaublich
               blöde Brief, den Sie mir geſchrieben haben, hat mich ſehr gefreut. Seien Sie brav und
               lernen Sie was! Zur Belohnung dürfen Sie dann auch \strikeout{wieder}{ }\label{K_L03536-7v}\edtext{nach Berlin\oindex{Berlin@\textbf{Berlin}, \emph{P.PPLC}|pw} kommen}{\lemma{\textnormal{\emph{nach Berlin kommen}}}\Cendnote{\textnormal{Elisabeth Gussmann\pwindex{Steinrueck, Elisabeth 19.11.1885 – 07.04.1920@\textsc{Steinrück, Elisabeth} (19.11.1885 – 07.04.1920)|pwk} war jedenfalls Ende Januar 1902 in Berlin\oindex{Berlin@\textbf{Berlin}, \emph{P.PPLC}|pwk}, vgl. die Korrespondenz zwischen Goldmann\pwindex{Goldmann, Paul 31.01.1865 – 25.09.1935@\textsc{Goldmann, Paul} (31.01.1865 – 25.09.1935), \emph{Schriftsteller/Schriftstellerin, Journalist/Journalistin}|pwk} und Elisabeth
                        Gussmann\pwindex{Steinrueck, Elisabeth 19.11.1885 – 07.04.1920@\textsc{Steinrück, Elisabeth} (19.11.1885 – 07.04.1920)|pwk}: \emph{DLA}, HS.1985.1.5246.}}}\label{K_L03536-7} und
               wieder einmal in meinem Umgang ſich fortbilden. \label{K_L03536-8v}\edtext{\textsc{Kohrl\pwindex{Kohrl @\textsc{Kohrl}|pw}}}{\lemma{\textnormal{\emph{Kohrl}}}\Cendnote{\textnormal{nicht ermittelt}}}\label{K_L03536-8} verlebt in Tirol\oindex{Tirol@\textbf{Tirol}, \emph{A.ADM1}|pw}\oindex{Suedtirol@\textbf{Südtirol}, \emph{A.ADM2}|pw} gewiß glückliche Tage, ſeit er Sie los iſt. Grüßen Sie Herrn \textsc{Paul\pwindex{Marx, Paul 04.06.1861 – 27.11.1919@\textsc{Marx, Paul} (04.06.1861 – 27.11.1919), \emph{Journalist/Journalistin, Kritiker/Kritikerin}|pw}} und ſeien Sie ſelbſt herzlichſt gegrüßt von Ihrem getreuen {\\}\spacefill\mbox{Paul Goldmann}\pend
           \selectlanguage{ngerman}\endnumbering\briefempfaengerindex{Steinrueck, Elisabeth@\textsc{Steinrück, Elisabeth}!zzzGoldmann, Paul@\emph{von Paul Goldmann}!1901-12-101@{10. 12. {[}1901{]}}|)be}\briefempfaengerindex{Schnitzler, Olga@\textsc{Schnitzler, Olga}!zzzGoldmann, Paul@\emph{von Paul Goldmann}!1901-12-101@{10. 12. {[}1901{]}}|)be}\mylabel{L03536h}  \normalsize

\doendnotes{C}
\bigskip
\vfill

\clearpage

\footnotesize

\lohead{\textsc{register}}

% Definiere theindex-Environment komplett neu ohne reledmac
\makeatletter
\renewenvironment{theindex}{%
  \section*{\indexname}%
  \setlength{\parindent}{0pt}%
  \setlength{\parskip}{0pt plus 0.3pt}%
  \let\item\@idxitem
}{%
  \clearpage
}
\makeatother

\IfFileExists{\jobname-pw.ind}{\input{\jobname-pw.ind}}{}

\end{document}

      