%% latex-leseansicht-vorspann.tex
%% Vorspann für die Leseansicht.
%% Lädt die gemeinsame Datei latex-vorspann.tex mit nicht gesetztem Schalter.

\newif\ifkorrekturansicht
\korrekturansichtfalse

\input{../tex-inputs/latex-vorspann}


\section[ Paul Goldmann an Olga und Elisabeth Gussmann, 10. 12. {[}1901{]}]{L03536 Paul Goldmann an Olga und Elisabeth Gussmann,  10. 12. [1901]}
\nopagebreak\mylabel{L03536v}
\rehead{ }\normalsize\beginnumbering\briefempfaengerindex{Steinrück, Elisabeth@\textsc{Steinrück, Elisabeth}!zzzGoldmann, Paul@\emph{von Paul Goldmann}!1901-12-101@{10. 12. [1901]}|(be}\briefempfaengerindex{Schnitzler, Olga@\textsc{Schnitzler, Olga}!zzzGoldmann, Paul@\emph{von Paul Goldmann}!1901-12-101@{10. 12. [1901]}|(be}
\toendnotes[C]{\smallbreak\pagebreak[2]}
\correspDesc{Versand  durch Paul Goldmann am 10. 12. [1901] in Berlin
\newline{}Erhalt  durch Olga Gussmann, Elisabeth Gussmann im Zeitraum [11. 12. 1901 – 15. 12. 1901?] in Wien?}\toendnotes[C]{\smallbreak}
\Standort{DLA, A:Schnitzler, HS.NZ85.1.5247.}
\physDesc{Brief, 1 Blatt, 4 Seiten, 1609 Zeichen
\newline{}Handschrift: blaue Tinte, deutsche Kurrent}\toendnotes[C]{\smallbreak}
\pstart
           \raggedleft{}{\pb}\textcolor{gray}{\textbf{DESSAUERSTRASSE 19\oindex{Dessauer Straße@\textbf{Dessauer Straße}, \emph{Straße}|pw}}}\pend
           
\pstart
           Berlin\oindex{Berlin@\textbf{Berlin}, \emph{Hauptstadt}|pw}, 10. Dezember.\pend
           
\pstart\center{}Liebes Fräulein \textsc{Olga},\pend\vspace{0.5em}
\pstart
           Haben Sie vielen Dank für Ihren lieben Brief! Antworten kann ich Ihnen noch nicht. Es
               iſt nicht mit Worten zu beſchreiben, was ich zu thun habe! Ich will Ihnen nur{ }ſagen,
               wie{ }ſehr mich Ihre Zeilen gefreut \strikeout{h\textcolor{gray}{a}} haben, in denen Sie als das liebe Wien\oindex{Wien@\textbf{Wien}, \emph{Verwaltungsgebiet}|pw}er
               Mädel erſcheinen, als das ich Sie kenne. Warum man weinen muß, wenn \textsc{Hauptmann\pwindex{Hauptmann, Gerhart 15.\,11.\,1862 Szczawno-Zdrój – 6.\,6.\,1946 Jagniątków@\textsc{Hauptmann, Gerhart} (15.\,11.\,1862 Szczawno-Zdrój – 6.\,6.\,1946 Jagniątków), \emph{Schriftsteller}|pw}} ein \label{K_L03536-1v}\edtext{ſchlechtes Stück\pwindex{Hauptmann, Gerhart 15.\,11.\,1862 Szczawno-Zdrój – 6.\,6.\,1946 Jagniątków@\textsc{Hauptmann, Gerhart} (15.\,11.\,1862 Szczawno-Zdrój – 6.\,6.\,1946 Jagniątków), \emph{Schriftsteller}!rothe Hahn. Tragikomödie in vier Akten@\strich\emph{Der rothe Hahn. Tragikomödie in vier Akten}|pwuv}}{\lemma{\textnormal{\emph{schlechtes Stück}}}\Cendnote{\textnormal{\emph{Der rothe Hahn}\pwindex{Hauptmann, Gerhart 15.\,11.\,1862 Szczawno-Zdrój – 6.\,6.\,1946 Jagniątków@\textsc{Hauptmann, Gerhart} (15.\,11.\,1862 Szczawno-Zdrój – 6.\,6.\,1946 Jagniątków), \emph{Schriftsteller}!rothe Hahn. Tragikomödie in vier Akten@\strich\emph{Der rothe Hahn. Tragikomödie in vier Akten}|pwk}, siehe XXXX Auszeichnungsfehler: Dokument L03092 nicht gefunden.}}}\label{K_L03536-1}{ }ſchreibt, iſt
               mir {\pb}zwar unklar, aber \label{K_L03536-2v}\edtext{über \textsc{Hauptmann\pwindex{Hauptmann, Gerhart 15.\,11.\,1862 Szczawno-Zdrój – 6.\,6.\,1946 Jagniątków@\textsc{Hauptmann, Gerhart} (15.\,11.\,1862 Szczawno-Zdrój – 6.\,6.\,1946 Jagniątków), \emph{Schriftsteller}|pw}} wollen wir nicht mehr miteinander reden}{\lemma{\textnormal{\emph{über … reden}}}\Cendnote{\textnormal{Vgl. XXXX Auszeichnungsfehler: Dokument L03535 nicht gefunden. }}}\label{K_L03536-2}. Bezüglich
               des dritten Akt\pwindex{\textcolor{red}{\textsuperscript{XXXX indx1}}!Hoffmanns Erzählungen@\strich\emph{Hoffmanns Erzählungen}|pwv}es von \label{K_L03536-3v}\edtext{\textsc{Hoffmanns} Erzählungen\pwindex{\textcolor{red}{\textsuperscript{XXXX indx1}}!Hoffmanns Erzählungen@\strich\emph{Hoffmanns Erzählungen}|pw}}{\lemma{\textnormal{\emph{Hoffmanns Erzählungen}}}\Cendnote{\textnormal{Das deutet darauf hin, dass Olga Gussmann\pwindex{Schnitzler, Olga 17.\,1.\,1882 Wien – 13.\,1.\,1970 Lugano@\textsc{Schnitzler, Olga} (17.\,1.\,1882 Wien – 13.\,1.\,1970 Lugano), \emph{Schauspielerin, Sängerin}|pwk} die Oper\pwindex{\textcolor{red}{\textsuperscript{XXXX indx1}}!Hoffmanns Erzählungen@\strich\emph{Hoffmanns Erzählungen}|pwkv} am 29. 11. 1901 gemeinsam mit Schnitzler besucht hat.}}}\label{K_L03536-3} bin ich ganz Ihrer Anſicht.
               Ich habe ihn immer für das{ }ſchönſte gehalten, wenn auch die \textsc{Barcarole\pwindex{\textcolor{red}{\textsuperscript{XXXX indx1}}!Barcarole@\strich\emph{Barcarole}|pw}} mein Lieblingsſtück bleibt. Nur \textsc{Arthur} hat, wie Sie{ }ſich erinnern werden, die ganze Oper\pwindex{\textcolor{red}{\textsuperscript{XXXX indx1}}!Hoffmanns Erzählungen@\strich\emph{Hoffmanns Erzählungen}|pwv} als talentloſes Machwerk \label{K_L03536-4v}\edtext{bezeichnet}{\lemma{\textnormal{\emph{bezeichnet}}}\Cendnote{\textnormal{Am
                     28. 11. 1900
                  hatten Schnitzler und Goldmann\pwindex{Goldmann, Paul 31.\,1.\,1865 Breslau – 25.\,9.\,1935 Wien@\textsc{Goldmann, Paul} (31.\,1.\,1865 Breslau – 25.\,9.\,1935 Wien), \emph{Schriftsteller, Journalist}|pwk} die Oper\pwindex{\textcolor{red}{\textsuperscript{XXXX indx1}}!Hoffmanns Erzählungen@\strich\emph{Hoffmanns Erzählungen}|pwkv} gemeinsam besucht und danach noch gemeinsam gegessen.
                  Vermutlich war auch Olga Gussmann\pwindex{Schnitzler, Olga 17.\,1.\,1882 Wien – 13.\,1.\,1970 Lugano@\textsc{Schnitzler, Olga} (17.\,1.\,1882 Wien – 13.\,1.\,1970 Lugano), \emph{Schauspielerin, Sängerin}|pwk}
                  dabei.}}}\label{K_L03536-4} und hat dadurch wieder bewieſen, daß er vom Theater nichts
               verſteht.\pend
           
\pstart
           \textsc{Alfred Gold\pwindex{Gold, Alfred 28.\,6.\,1874 Wien – 24.\,10.\,1958 New York City@\textsc{Gold, Alfred} (28.\,6.\,1874 Wien – 24.\,10.\,1958 New York City), \emph{Schriftsteller, Journalist, Kunsthändler}|pw}}, der verworrene {\pb}und alberne Literatur-Lausbub\pwindex{Gold, Alfred 28.\,6.\,1874 Wien – 24.\,10.\,1958 New York City@\textsc{Gold, Alfred} (28.\,6.\,1874 Wien – 24.\,10.\,1958 New York City), \emph{Schriftsteller, Journalist, Kunsthändler}|pwv}, ein \textsc{\begin{otherlanguage}{french}Protégé\end{otherlanguage}} der Frau\pwindex{Mamroth, Johanna 19.\,5.\,1872 Frankfurt am Main – 12.\,9.\,1910@\textsc{Mamroth, Johanna} (19.\,5.\,1872 Frankfurt am Main – 12.\,9.\,1910)|pwv} meines Onkels\pwindex{Mamroth, Fedor 21.\,2.\,1851 Breslau – 25.\,6.\,1907 Frankfurt am Main@\textsc{Mamroth, Fedor} (21.\,2.\,1851 Breslau – 25.\,6.\,1907 Frankfurt am Main), \emph{Journalist, Kritiker}|pwv}, iſt von meinem Onkel\pwindex{Mamroth, Fedor 21.\,2.\,1851 Breslau – 25.\,6.\,1907 Frankfurt am Main@\textsc{Mamroth, Fedor} (21.\,2.\,1851 Breslau – 25.\,6.\,1907 Frankfurt am Main), \emph{Journalist, Kritiker}|pwv} als Berlin\oindex{Berlin@\textbf{Berlin}, \emph{Hauptstadt}|pw}er Feuilleton-Correſpondent der Frankfurter Zeitung\orgindex{Frankfurter Zeitung@Frankfurter Zeitung|pw} engagirt worden!!!\pend
           
\pstart
           Laſſen Sie es{ }ſich gut gehen in Ihrer \label{K_L03536-5v}\edtext{neuen Penſion\oindex{?? [Wohnung von Olga und Elisabeth Gussmann, 1901/1902]@\textbf{?? [Wohnung von Olga und Elisabeth Gussmann, 1901/1902]}, \emph{Wohngebäude}|pwv}}{\lemma{\textnormal{\emph{neuen Pension}}}\Cendnote{\textnormal{Im Frühling 1901 waren Olga\pwindex{Schnitzler, Olga 17.\,1.\,1882 Wien – 13.\,1.\,1970 Lugano@\textsc{Schnitzler, Olga} (17.\,1.\,1882 Wien – 13.\,1.\,1970 Lugano), \emph{Schauspielerin, Sängerin}|pwk} und Elisabeth Gussmann\pwindex{Steinrück, Elisabeth 19.\,11.\,1885 – 7.\,4.\,1920 Partenkirchen@\textsc{Steinrück, Elisabeth} (19.\,11.\,1885 – 7.\,4.\,1920 Partenkirchen)|pwk} in die Grünentorgasse\oindex{Wien@\textbf{Wien}!IX., Alsergrund@\textbf{IX., Alsergrund}!Grünentorgasse@\textbf{Grünentorgasse}, \emph{Straße}|pwk} gezogen. Vermutlich seit November 1901 war
                     Olga\pwindex{Schnitzler, Olga 17.\,1.\,1882 Wien – 13.\,1.\,1970 Lugano@\textsc{Schnitzler, Olga} (17.\,1.\,1882 Wien – 13.\,1.\,1970 Lugano), \emph{Schauspielerin, Sängerin}|pwk} schwanger (vgl. A. S.: \emph{Tagebuch}, 10. 11. 1901). Die vorliegende
                  Stelle deutet auf eine neue Unterkunft, die sie bis zur Übersiedelung in die Hauptstraße 56 in Hinterbrühl\oindex{Hauptstraße 56@\textbf{Hauptstraße 56}, \emph{Wohngebäude}|pwk} am 21. 3. 1902 bewohnte.
                  Aus einem Schreiben Schnitzlers an Schwarzkopf\pwindex{\textcolor{red}{\textsuperscript{XXXX indx1}}|pwk} vom XXXX Auszeichnungsfehler: Dokument L04179 nicht gefunden scheint als
                  Adresse die Lage in der Garnisonsgasse\oindex{XXXX Ortsangabe fehlt|pwk}, um die
                  Ecke von Schnitzler, wahrscheinlich.}}}\label{K_L03536-5}
               mit den \label{K_L03536-6v}\edtext{\textsc{\begin{otherlanguage}{english}new style\end{otherlanguage}}-Möbeln}{\lemma{\textnormal{\emph{new style-Möbeln}}}\Cendnote{\textnormal{›\begin{otherlanguage}{english}New
                     Style\end{otherlanguage}‹ ist synonym mit \begin{otherlanguage}{french}l’art
                     nouveau\end{otherlanguage}/Jugendstil.}}}\label{K_L03536-6} und{ }ſeien Sie (bis ich Ihnen ausführlich{ }ſchreibe) einſtweilen herzlich\uuline{ſt} (nicht herzl\uuline{ich}, wie Sie{ }ſchreiben) gegrüßt von Ihrem getreuen{\\[\baselineskip]}\spacefill\mbox{Paul Goldmann.}\pend
           \leftskip=0em{}\selectlanguage{ngerman}\vspace{1em}{\vspace{1\baselineskip}}
\pstart
           \noindent{}{\pb}Liebes Fräulein \textsc{Liesl}, der unglaublich
               blöde Brief, den Sie mir geſchrieben haben, hat mich{ }ſehr gefreut. Seien Sie brav und
               lernen Sie was! Zur Belohnung dürfen Sie dann auch \strikeout{wieder}{ }\label{K_L03536-7v}\edtext{nach Berlin\oindex{Berlin@\textbf{Berlin}, \emph{Hauptstadt}|pw} kommen}{\lemma{\textnormal{\emph{nach Berlin kommen}}}\Cendnote{\textnormal{Elisabeth Gussmann\pwindex{Steinrück, Elisabeth 19.\,11.\,1885 – 7.\,4.\,1920 Partenkirchen@\textsc{Steinrück, Elisabeth} (19.\,11.\,1885 – 7.\,4.\,1920 Partenkirchen)|pwk} war jedenfalls Ende Januar 1902 in Berlin\oindex{Berlin@\textbf{Berlin}, \emph{Hauptstadt}|pwk}, vgl. die Korrespondenz zwischen Goldmann\pwindex{Goldmann, Paul 31.\,1.\,1865 Breslau – 25.\,9.\,1935 Wien@\textsc{Goldmann, Paul} (31.\,1.\,1865 Breslau – 25.\,9.\,1935 Wien), \emph{Schriftsteller, Journalist}|pwk} und Elisabeth
                        Gussmann\pwindex{Steinrück, Elisabeth 19.\,11.\,1885 – 7.\,4.\,1920 Partenkirchen@\textsc{Steinrück, Elisabeth} (19.\,11.\,1885 – 7.\,4.\,1920 Partenkirchen)|pwk}: \emph{DLA}, HS.1985.1.5246.}}}\label{K_L03536-7} und
               wieder einmal in meinem Umgang{ }ſich fortbilden. \label{K_L03536-8v}\edtext{\textsc{Kohrl\pwindex{Kohrl @\textsc{Kohrl}|pw}}}{\lemma{\textnormal{\emph{Kohrl}}}\Cendnote{\textnormal{nicht ermittelt}}}\label{K_L03536-8} verlebt in Tirol\oindex{Tirol@\textbf{Tirol}, \emph{Land}|pw}\oindex{Südtirol@\textbf{Südtirol}, \emph{Verwaltungsgebiet}|pw} gewiß glückliche Tage,{ }ſeit er Sie
               los iſt. Grüßen Sie Herrn \textsc{Paul\pwindex{Marx, Paul 4.\,6.\,1861 Breslau – 27.\,11.\,1919 Berlin@\textsc{Marx, Paul} (4.\,6.\,1861 Breslau – 27.\,11.\,1919 Berlin), \emph{Journalist, Kritiker}|pw}} und{ }ſeien Sie{ }ſelbſt herzlichſt gegrüßt von Ihrem getreuen {\\}\spacefill\mbox{Paul Goldmann}\pend
           \selectlanguage{ngerman}\endnumbering\briefempfaengerindex{Steinrück, Elisabeth@\textsc{Steinrück, Elisabeth}!zzzGoldmann, Paul@\emph{von Paul Goldmann}!1901-12-101@{10. 12. [1901]}|)be}\briefempfaengerindex{Schnitzler, Olga@\textsc{Schnitzler, Olga}!zzzGoldmann, Paul@\emph{von Paul Goldmann}!1901-12-101@{10. 12. [1901]}|)be}\mylabel{L03536h}  \newcommand{\dateiname}{L03536}\newcommand{\titel}{Paul Goldmann an Olga und Elisabeth Gussmann, 10. 12. [1901]}\newcommand{\editorInnen}{Martin Anton Müller und Laura Untner}%% latex-leseansicht-abspann.tex
%% Abspann für die Leseansicht.
%% Der Schalter \ifkorrekturansicht ist bereits durch den Vorspann gesetzt.

%% latex-abspann.tex
%% Gemeinsamer Abspann für Korrekturansicht und Leseansicht.
%% Setzt den Schalter \ifkorrekturansicht voraus (gesetzt in den
%% einbindenden Dateien latex-korrekturansicht-abspann.tex bzw.
%% latex-leseansicht-abspann.tex).
%% ---------------------------------------------------------------

\normalsize

% Das esempio-Environment wird nur in der Leseansicht benötigt
\ifkorrekturansicht\else
\newenvironment{esempio}[3]%
{
    \vspace{1.5ex}
    \rlap{\underline{#1}}
    \par
    \setlength{\parindent}{0cm}
    \nopagebreak
    \leftskip=#2cm
    \rightskip=#3cm
}
{
    \par
}
\fi

\doendnotes{C}
\bigskip
\vfill

\clearpage

\footnotesize

\ifkorrekturansicht
  \lohead{\textsc{register}}
\fi

% theindex-Environment neu definieren ohne reledmac
\makeatletter
\renewenvironment{theindex}{%
  \ifkorrekturansicht
    \section*{\indexname}%
  \else
    \subsubsection*{Index der erwähnten Entitäten}%
  \fi
  \setlength{\parindent}{0pt}%
  \setlength{\parskip}{0pt plus 0.3pt}%
  \let\item\@idxitem
}{%
  \ifkorrekturansicht\clearpage\fi
}
\makeatother

\IfFileExists{\jobname-pw.ind}{\input{\jobname-pw.ind}}{}

% Quellenangabe nur in der Leseansicht
\ifkorrekturansicht\else
% Fallback-Definitionen, falls die .tex-Datei \titel etc. nicht gesetzt hat
\providecommand{\titel}{}
\providecommand{\editorInnen}{}
\providecommand{\dateiname}{\jobname}

\vspace{3cm}

\vfill

\footnotesize
\textsc{Quelle}: \titel. Herausgegeben von {\editorInnen}. In: \emph{Arthur Schnitzler: Briefwechsel mit Autorinnen und Autoren}.
 Digitale Edition, https://schnitzler-briefe.acdh.oeaw.ac.at/{\dateiname}.html (Stand \today)
\fi

\end{document}


