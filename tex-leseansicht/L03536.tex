%% latex-leseansicht-vorspann.tex
%% Vorspann für die Leseansicht.
%% Lädt die gemeinsame Datei latex-vorspann.tex mit nicht gesetztem Schalter.

\newif\ifkorrekturansicht
\korrekturansichtfalse

\input{../tex-inputs/latex-vorspann}

\begin{center}
            \textcolor{red}{ENTWURF, NICHT FERTIG KORRIGIERT}
                      \end{center}
            
         
         \renewcommand{\erwaehntePersonen}{Personen: Gerhart Hauptmann, Olga Schnitzler}
         \renewcommand{\erwaehnteOrte}{Orte: Berlin, Wien}
         \renewcommand{\erwaehnteWerke}{}
               \section[ Paul Goldmann an Olga XXXX Gussmann/Schnitzler, 10. 12. {[}XXXX{]}]{ Paul Goldmann an Olga XXXX Gussmann/Schnitzler, 10. 12. {[}XXXX{]}}\nopagebreak\mylabel{v}\rehead{ }\begin{ledgroupsized}[t]{13cm}\normalsize\beginnumbering \toendnotes[C]{\smallbreak\pagebreak[2]} \Standort{DLA, A:Schnitzler, HS.1985.1.5247.}
\physDesc{,  Blätter,  Seiten
\newline{}Handschrift: , deutsche Kurrent}\toendnotes[C]{\smallbreak}\pstart
           \noindent{}{\pb}\pend
           \textcolor{gray}{\textbf{DESSAUERSTRASSE 19\oindex{XXXX Ortsangabe fehlt|pw}}}\textcolor{red}{\textsuperscript{\textbf{KEY}}}\pstart
           Berlin\oindex{Berlin@\textbf{Berlin}|pw}, 10. Dezember.\pend
           \pstart{}Liebes Fräulein \textsc{Olga},\pend\pstart
           Haben Sie vielen Dank für Ihren lieben Brief! Antworten kann ich Ihnen noch nicht. Es
               iſt nicht mit Worten zu beſchreiben, was ich zu thun habe! Ich will Ihnen nur ſagen,
               wie ſehr mich Ihre Zeilen gefreut \strikeout{h\textcolor{gray}{a}} haben, in denen Sie als das liebe Wien\oindex{Wien@\textbf{Wien}|pw}er
               Mädel erſcheinen, als das ich Sie kenne. Warum man weinen muß, wenn \textsc{Hauptmann\pwindex{Hauptmann, Gerhart 15.11.1862 – 06.06.1946@\textsc{Hauptmann, Gerhart} (15.11.1862 – 06.06.1946), \emph{Schriftsteller}|pw}} ein ſchlechtes Stück ſchreibt, iſt mir {\pb} zwar unklar, aber über \textsc{Hauptmann\pwindex{Hauptmann, Gerhart 15.11.1862 – 06.06.1946@\textsc{Hauptmann, Gerhart} (15.11.1862 – 06.06.1946), \emph{Schriftsteller}|pw}} wollen wir nicht mehr miteinander reden. Bezüglich des dritten Akt\textcolor{red}{\textsuperscript{\textbf{KEY}}}es von \label{XXXXv}\edtext{\textsc{Hoffmanns} Erzählungen\textcolor{red}{\textsuperscript{\textbf{KEY}}}[Kommentar:
                  Offenbach]}{\lemma{\textnormal{\emph{XXXX Lemmafehler}}}\Cendnote{\textnormal{}}}\label{XXXX} bin ich ganz Ihrer Anſicht. Ich habe ihn
               immer ſie das ſchönſte gehalten, wenn auch die \textsc{Barcarole\textcolor{red}{\textsuperscript{\textbf{KEY}}}} mein Lieblingsſtück bleibt. Nur \textsc{Arthur\pwindex{Schnitzler, Arthur 15.05.1862 – 21.10.1931@\textsc{Schnitzler, Arthur} (15.05.1862 – 21.10.1931), \emph{Schriftsteller, Mediziner}|pw}} hat, wie Sie ſich erinnern werden, die ganze Oper\textcolor{red}{\textsuperscript{\textbf{KEY}}}
               als talentloſes Machwerk bezeichnet und hat dadurch wieder bewieſen, daß er vom
               Theater nichts verſteht.\pend
           \pstart
           \textsc{Alfred Gold\textcolor{red}{\textsuperscript{\textbf{KEY}}}}, der verworrene {\pb} und
               alberne Literatur-Lausbub\textcolor{red}{\textsuperscript{\textbf{KEY}}}, ein \textsc{Protégé \textcolor{red}{\textsuperscript{\textbf{KEY}}}} der der Frau\textcolor{red}{\textsuperscript{\textbf{KEY}}} meines Onkel\textcolor{red}{\textsuperscript{\textbf{KEY}}}s, iſt von meinem Onkel\textcolor{red}{\textsuperscript{\textbf{KEY}}} als Berlin\oindex{Berlin@\textbf{Berlin}|pw}er
               Feuilleton-Correſpondent der Frankfurter Zeitung\textcolor{red}{\textsuperscript{\textbf{KEY}}} engagirt
               worden!!! {\\[\baselineskip]}Laſſen Sie es ſich gut gehen\pend
           \leftskip=0em{}\pstart
           {\\[\baselineskip]}in Ihrer neuen Penſion\textcolor{red}{\textsuperscript{\textbf{KEY}}} mit\pend
           \leftskip=0em{}\pstart
           {\\[\baselineskip]}den \textsc{new style}-Möbeln\pend
           \leftskip=0em{}\pstart
           {\\[\baselineskip]}und ſeien Sie (bis ich Ihnen\pend
           \leftskip=0em{}\pstart
           {\\[\baselineskip]}ausführlich ſchreibe) einſtweilen\pend
           \leftskip=0em{}\pstart
           {\\[\baselineskip]}herzlich\strikeout{ſt} (nicht herzl\strikeout{ich}, wie\pend
           \leftskip=0em{}\pstart
           {\\[\baselineskip]}Sie ſchreiben) gegrüßt von\pend
           \leftskip=0em{}\pstart
           {\\[\baselineskip]}Ihrem getreuen\pend
           \leftskip=0em{}\pstart
           {\\[\baselineskip]}\spacefill\mbox{Paul Goldmann.}\pend
           \leftskip=0em{}\pstart
           {\pb}\pend
           \pstart
           Liebes Fräulein \textsc{Liesl}, der unglaublich
               blöde Brief, den Sie mir geſchrieben haben, hat mich ſehr gefreut. Seien Sie brav und
               lernen Sie was! Zur Belohnung dürfen Sie dann auch \strikeout{wieder} nach Berlin\oindex{Berlin@\textbf{Berlin}|pw} kommen und wieder
               einmal in meinem Umgang ſich fortbilden. \textsc{Kohrl\textcolor{red}{\textsuperscript{\textbf{KEY}}}} verlebt in Tirol\textcolor{red}{\textsuperscript{\textbf{KEY}}} gewiß glückliche Tage, ſeit er
               Sie los iſt. Grüßen Sie Herrn \textsc{Paul\textcolor{red}{\textsuperscript{\textbf{KEY}}}} und ſeien Sie ſelbſt herzlichſt gegrüßt von Ihrem getreuen \spacefill\mbox{Paul
                  Goldmann}\pend
           
         
         \endnumbering\mylabel{h}\end{ledgroupsized}\begin{anhang}\end{anhang}\newcommand{\dateiname}{L03536}\newcommand{\titel}{Paul Goldmann an Olga XXXX Gussmann/Schnitzler, 10. 12. [XXXX]}\newcommand{\editorInnen}{Martin Anton Müller und Laura Untner}%% latex-leseansicht-abspann.tex
%% Abspann für die Leseansicht.
%% Der Schalter \ifkorrekturansicht ist bereits durch den Vorspann gesetzt.

%% latex-abspann.tex
%% Gemeinsamer Abspann für Korrekturansicht und Leseansicht.
%% Setzt den Schalter \ifkorrekturansicht voraus (gesetzt in den
%% einbindenden Dateien latex-korrekturansicht-abspann.tex bzw.
%% latex-leseansicht-abspann.tex).
%% ---------------------------------------------------------------

\normalsize

% Das esempio-Environment wird nur in der Leseansicht benötigt
\ifkorrekturansicht\else
\newenvironment{esempio}[3]%
{
    \vspace{1.5ex}
    \rlap{\underline{#1}}
    \par
    \setlength{\parindent}{0cm}
    \nopagebreak
    \leftskip=#2cm
    \rightskip=#3cm
}
{
    \par
}
\fi

\doendnotes{C}
\bigskip
\vfill

\clearpage

\footnotesize

\ifkorrekturansicht
  \lohead{\textsc{register}}
\fi

% theindex-Environment neu definieren ohne reledmac
\makeatletter
\renewenvironment{theindex}{%
  \ifkorrekturansicht
    \section*{\indexname}%
  \else
    \subsubsection*{Index der erwähnten Entitäten}%
  \fi
  \setlength{\parindent}{0pt}%
  \setlength{\parskip}{0pt plus 0.3pt}%
  \let\item\@idxitem
}{%
  \ifkorrekturansicht\clearpage\fi
}
\makeatother

\IfFileExists{\jobname-pw.ind}{\input{\jobname-pw.ind}}{}

% Quellenangabe nur in der Leseansicht
\ifkorrekturansicht\else
% Fallback-Definitionen, falls die .tex-Datei \titel etc. nicht gesetzt hat
\providecommand{\titel}{}
\providecommand{\editorInnen}{}
\providecommand{\dateiname}{\jobname}

\vspace{3cm}

\vfill

\footnotesize
\textsc{Quelle}: \titel. Herausgegeben von {\editorInnen}. In: \emph{Arthur Schnitzler: Briefwechsel mit Autorinnen und Autoren}.
 Digitale Edition, https://schnitzler-briefe.acdh.oeaw.ac.at/{\dateiname}.html (Stand \today)
\fi

\end{document}


      