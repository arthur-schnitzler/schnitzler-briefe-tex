%% latex-korrekturansicht-vorspann.tex
%% Vorspann für die Korrekturansicht.
%% Lädt die gemeinsame Datei latex-vorspann.tex mit gesetztem Schalter.

\newif\ifkorrekturansicht
\korrekturansichttrue

\input{../tex-inputs/latex-vorspann}


\section[Hugo von Hofmannsthal an Arthur Schnitzler, {[}24./25.?{]} 7. 1904]{L01418 Hugo von Hofmannsthal an Arthur Schnitzler, {[}24./25.?{]} 7. 1904}
\nopagebreak\mylabel{L01418v}
\rehead{ }\normalsize\beginnumbering\briefempfaengerindex{Schnitzler, Arthur@\textsc{Schnitzler, Arthur}!zzzHofmannsthal, Hugo von@\emph{von Hugo von Hofmannsthal}!1904-07-251@{{[}24./25.?{]} 7. 1904}|(be}
\toendnotes[C]{\smallbreak\pagebreak[2]}\Standort{CUL, Schnitzler, B 43.}
\physDesc{Brief, 1 Blatt, 4 Seiten, 1284 Zeichen
\newline{}Handschrift: schwarze Tinte, deutsche Kurrent
\newline{}Schnitzler: mit Bleistift Monat und Jahreszahl ergänzt: »7. 904.« 
\newline{}Ordnung: 1) mit Bleistift von unbekannter Hand nummeriert: »\strikeout{77}«  2) mit Bleistift von unbekannter Hand nummeriert: »230«}
\buchAbdrucke{\weitereDrucke{Hugo von Hofmannsthal, Arthur Schnitzler: \emph{Briefwechsel}. Frankfurt am Main: \emph{S. Fischer} 1964, S. 191.} }\toendnotes[C]{\smallbreak}
\pstart
           \raggedleft{}{\pb}Bad Fuſch\oindex{Bad Fusch@\textbf{Bad Fusch}, \emph{A.ADM3}|pw}{ }2\textcolor{gray}{×}\textsc{ten}\pend
           
\pstart{}lieber, \pend\vspace{0.5em}
\pstart
           hier bin ich wirklich wie unter dem erſten Anhauch der Luft geſund geworden, und von
               einem innern Reichthum, daſs ich manchmal, gegen Abend, auf eine ſteile Berglehne hin
               aufklettern muſs, nur um das Blut vom Kopf abzuleiten und den unaufhörlichen {\pb}Zudrang von Gedanken, Bildern,
               Situationen, abzuleiten.\pend
           
\pstart
           Es iſt mir ſchwerer, in ſolchen Zeiten ein Buch zu leſen. Ich möchte alles, was mir
               in die Hände fällt, dramatiſieren, ſelbſt den Goethe\pwindex{Goethe, Johann Wolfgang von 1749-08-28 – 1832-03-22@\textsc{Goethe, Johann Wolfgang von} (1749-08-28 – 1832-03-22), \emph{Schriftsteller/Schriftstellerin}|pw}–Schiller\pwindex{Schiller, Friedrich von 10.11.1759 – 09.05.1805@\textsc{Schiller, Friedrich von} (10.11.1759 – 09.05.1805), \emph{Schriftsteller/Schriftstellerin, Historiker/Historikerin, Philosoph/Philosophin}|pw}’ſchen Briefwechſel\pwindex{Briefwechsel zwischen Schiller und Goethe@\emph{Briefwechsel zwischen Schiller und Goethe}|pw}, oder die Linzer Tages-poſt\pwindex{Linzer Tages-Post@\emph{Linzer Tages-Post}|pw}.\pend
           
\pstart
           Das »\label{T_L01418-1v}\edtext{gerettete Venedig\pwindex{gerettete Venedig. Trauerspiel in fuenf Aufzuegen@\emph{Das gerettete Venedig. Trauerspiel in fünf Aufzügen}|pw}}{\lemma{\textnormal{\emph{gerettete Venedig}}}\Cendnote{\textnormal{wohl von Schnitzler mit Bleistift
                  unterstrichen}}}\label{T_L01418-1}« hab ich \label{K_L01418-1v}\edtext{heute
                  abgeschloſſen}{\lemma{\textnormal{\emph{heute
                  abgeschloſſen}}}\Cendnote{\textnormal{Das erlaubt die
                  annähernde Datierung: Am 24. 7. 1904 schrieb Hofmannsthal\pwindex{Hofmannsthal, Hugo von 1874-02-01 – 1929-07-15@\textsc{Hofmannsthal, Hugo von} (1874-02-01 – 1929-07-15), \emph{Schriftsteller/Schriftstellerin}|pwk} dem Vater\pwindex{Hofmannsthal, Hugo August von 21.12.1841 – 08.12.1915@\textsc{Hofmannsthal, Hugo August von} (21.12.1841 – 08.12.1915), \emph{Bankdirektor/Bankdirektorin}|pwkv}, das Stück beendet zu haben (Hugo von Hofmannsthal\pwindex{Hofmannsthal, Hugo von 1874-02-01 – 1929-07-15@\textsc{Hofmannsthal, Hugo von} (1874-02-01 – 1929-07-15), \emph{Schriftsteller/Schriftstellerin}|pwk}: \emph{Aufzeichnungen}. Herausgegeben von Rudolf Hirsch † und Ellen Ritter
                     † in Zusammenarbeit mit Konrad Heumann und Peter Michael Braunwarth. Frankfurt
                     am Main: \emph{S. Fischer}\orgindex{S. Fischer Verlag@S. Fischer Verlag|pwk}{ }2013, Erläuterungen, S. 789 (\emph{Sämtliche
                        Werke}, XXXIX)). Am Folgetag, dem 25. 7. 1904,
                  hielt er zudem den Abschluss in einer persönlichen Aufzeichnung fest
                     (S. 482).}}}\label{K_L01418-1}. Was noch {\pb}daran zu thun iſt, das wenige
               läſst ſich unter dem Abſchreiben thun.\hspace*{1.5em}Indeſſen ſind
               aber, wie leuchtende Wolkeninſeln hinter den Bergen hervor andere Stoffe geſtiegen,
               zum Theil aus dem geheimnisvollen Abgrund des niemals ſchlafenden, umbildenden
               Gedächtniſſes: das »Leben ein Traum\pwindex{Turm. Ein Trauerspiel@\emph{Der Turm. Ein Trauerspiel}|pw}« dieſer faſt
               zu große Stoff, hat ſeinen tiefen {\pb}dem Calderon\pwindex{Calderón de la Barca, Pedro 17.01.1600 – 25.05.1681@\textsc{Calderón de la Barca, Pedro} (17.01.1600 – 25.05.1681), \emph{Schriftsteller/Schriftstellerin}|pw} faſt entgegen geſetzten Schluſs
               gefunden, »\textsc{Pentheus}\pwindex{Pentheus. Trauerspiel in zwei Aufzuegen@\emph{Pentheus. Trauerspiel in zwei Aufzügen}|pw}« im Stoff den \textsc{Bacchen}\pwindex{Bakchen@\emph{Die Bakchen}|pw} des \textsc{Euripides}\pwindex{Euripides 485? – 406? v. u. Z.@\textsc{Euripides} (485? – 406? v. u. Z.), \emph{Schriftsteller/Schriftstellerin}|pw} nahe, aber viel reicher und ſchöner, hat ſich zum Scenarium gegliedert,
               zweiactig; »\textsc{Orest in Delphi}\pwindex{Orest in Delphi@\emph{Orest in Delphi}|pw}« der \textsc{Elektra}\pwindex{Elektra. Tragoedie in einem Aufzug@\emph{Elektra. Tragödie in einem Aufzug}|pw} 2\textsuperscript{ter} Theil zeigt ſeine Geſtalten unheimlich
               deutlich – mit dieſer Fracht gehe ich den 31\textsuperscript{ten} nach \textsc{Markt-Aussee}, Rammgut\oindex{Ramgut@\textbf{Ramgut}, \emph{Schloss (K.SLS)}|pw}.\pend
           \pstart Laſſen Sie mich hier oder dort nicht ohne Nachricht. Ihr\spacefill\mbox{Hugo.}\pend{}\selectlanguage{ngerman}\endnumbering\briefempfaengerindex{Schnitzler, Arthur@\textsc{Schnitzler, Arthur}!zzzHofmannsthal, Hugo von@\emph{von Hugo von Hofmannsthal}!1904-07-241@{{[}24./25.?{]} 7. 1904}|)be}\mylabel{L01418h}  \normalsize

\doendnotes{C}
\bigskip
\vfill

\clearpage

\footnotesize

\lohead{\textsc{register}}

% Definiere theindex-Environment komplett neu ohne reledmac
\makeatletter
\renewenvironment{theindex}{%
  \section*{\indexname}%
  \setlength{\parindent}{0pt}%
  \setlength{\parskip}{0pt plus 0.3pt}%
  \let\item\@idxitem
}{%
  \clearpage
}
\makeatother

\IfFileExists{\jobname-pw.ind}{\input{\jobname-pw.ind}}{}

\end{document}

      