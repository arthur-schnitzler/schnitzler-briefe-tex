%% latex-leseansicht-vorspann.tex
%% Vorspann für die Leseansicht.
%% Lädt die gemeinsame Datei latex-vorspann.tex mit nicht gesetztem Schalter.

\newif\ifkorrekturansicht
\korrekturansichtfalse

\input{../tex-inputs/latex-vorspann}


         
         \renewcommand{\erwaehntePersonen}{Personen: Pedro Calderón de la Barca,  Euripides, Johann Wolfgang von Goethe, Hugo von Hofmannsthal, Hugo August von Hofmannsthal, Friedrich von Schiller}
         \renewcommand{\erwaehnteInstitutionen}{Institutionen: S. Fischer Verlag}
         \renewcommand{\erwaehnteOrte}{Orte: Bad Fusch, Ramgut, Wien}
         \renewcommand{\erwaehnteWerke}{Werke: Briefwechsel zwischen Schiller und Goethe, Das gerettete Venedig. Trauerspiel in fünf Aufzügen, Der Turm. Ein Trauerspiel, Die Bakchen, Elektra. Tragödie in einem Aufzug, Linzer Tages-Post, Orest in Delphi, Pentheus. Trauerspiel in zwei Aufzügen}
               \section[Hugo von Hofmannsthal an Arthur Schnitzler, {[}24./25.?{]} 7. 1904]{ Hugo von Hofmannsthal an Arthur Schnitzler, {[}24./25.?{]} 7. 1904}\nopagebreak\mylabel{v}\rehead{ }\begin{ledgroupsized}[t]{13cm}\normalsize\beginnumbering\briefempfaengerindex{Schnitzler, Arthur@\textsc{Schnitzler, Arthur}!zzzHofmannsthal, Hugo von@\emph{von Hugo von Hofmannsthal}!1904-07-251@{{[}24./25.?{]} 7. 1904}|(be} \toendnotes[C]{\smallbreak\pagebreak[2]} \Standort{CUL, Schnitzler, B 43.}
\physDesc{Brief, 1 Blatt, 4 Seiten, 1284 Zeichen
\newline{}Handschrift: schwarze Tinte, deutsche Kurrent
\newline{}Schnitzler: mit Bleistift Monat und Jahreszahl ergänzt: »7. 904.« 
\newline{}Ordnung: 1) mit Bleistift von unbekannter Hand nummeriert: »\strikeout{77}«  2) mit Bleistift von unbekannter Hand nummeriert:
                                    »230«}\buchAbdrucke{\weitereDrucke{Hugo von Hofmannsthal, Arthur Schnitzler: \emph{Briefwechsel}. Hg. Therese Nickl und Heinrich Schnitzler. Frankfurt am Main: \emph{S. Fischer} 1964, S. 191.} }\toendnotes[C]{\smallbreak}\pstart
           \raggedleft{}{\pb}Bad Fuſch\oindex{Bad Fusch@\textbf{Bad Fusch}|pw}{ }2\textcolor{gray}{×}\textsc{ten}\pend
           \pstart{}lieber, \pend\pstart
           hier bin ich wirklich wie unter dem erſten Anhauch der Luft geſund geworden, und von
               einem innern Reichthum, daſs ich manchmal, gegen Abend, auf eine ſteile Berglehne hin
               aufklettern muſs, nur um das Blut vom Kopf abzuleiten und den unaufhörlichen {\pb}Zudrang von Gedanken, Bildern,
               Situationen, abzuleiten.\pend
           \pstart
           Es iſt mir ſchwerer, in ſolchen Zeiten ein Buch zu leſen. Ich möchte alles, was mir
               in die Hände fällt, dramatiſieren, ſelbſt den Goethe\pwindex{Goethe, Johann Wolfgang von 1749-08-28 – 1832-03-22@\textsc{Goethe, Johann Wolfgang von} (1749-08-28 – 1832-03-22), \emph{Schriftsteller}|pw}–Schiller\pwindex{Schiller, Friedrich von 10.11.1759 – 09.05.1805@\textsc{Schiller, Friedrich von} (10.11.1759 – 09.05.1805), \emph{Schriftsteller, Historiker, Philosoph}|pw}’ſchen Briefwechſel\pwindex{Schiller, Friedrich von 10.11.1759 – 09.05.1805@\textsc{Schiller, Friedrich von} (10.11.1759 – 09.05.1805), \emph{Schriftsteller, Historiker, Philosoph}!Briefwechsel zwischen Schiller und Goethe1791 – 1805@\strich\emph{Briefwechsel zwischen Schiller und Goethe} {[}1791 – 1805{]}|pw}\pwindex{Goethe, Johann Wolfgang von 1749-08-28 – 1832-03-22@\textsc{Goethe, Johann Wolfgang von} (1749-08-28 – 1832-03-22), \emph{Schriftsteller}!Briefwechsel zwischen Schiller und Goethe1791 – 1805@\strich\emph{Briefwechsel zwischen Schiller und Goethe} {[}1791 – 1805{]}|pw}, oder die Linzer Tages-poſt\pwindex{?? Werk@Nicht ermittelte Verfasserinnen und Verfasser!Linzer Tages-Post1.1.1865 – 1945@\emph{Linzer Tages-Post} {[}1.1.1865 – 1945{]}|pw}.\pend
           \pstart
           Das »\label{T_L01418-1v}\edtext{gerettete Venedig\pwindex{Hofmannsthal, Hugo von 1874-02-01 – 1929-07-15@\textsc{Hofmannsthal, Hugo von} (1874-02-01 – 1929-07-15), \emph{Schriftsteller}!gerettete Venedig. Trauerspiel in fuenf Aufzuegen1905@\strich\emph{Das gerettete Venedig. Trauerspiel in fünf Aufzügen} {[}1905{]}|pw}}{\lemma{\textnormal{\emph{gerettete Venedig}}}\Cendnote{\textnormal{wohl von Schnitzler mit Bleistift
                  unterstrichen}}}\label{T_L01418-1h}« hab ich \label{K_L01418-1v}\edtext{heute
                  abgeschloſſen}{\lemma{\textnormal{\emph{heute
                  abgeschloſſen}}}\Cendnote{\textnormal{Das erlaubt die
                  annähernde Datierung: Am 24. 7. 1904 schreibt Hofmannsthal\pwindex{Hofmannsthal, Hugo von 1874-02-01 – 1929-07-15@\textsc{Hofmannsthal, Hugo von} (1874-02-01 – 1929-07-15), \emph{Schriftsteller}|pwk} dem Vater\pwindex{Hofmannsthal, Hugo August von 21.12.1841 – 08.12.1915@\textsc{Hofmannsthal, Hugo August von} (21.12.1841 – 08.12.1915), \emph{Bankdirektor}|pwkv}, das Stück beendet zu haben. (Hugo von Hofmannsthal\pwindex{Hofmannsthal, Hugo von 1874-02-01 – 1929-07-15@\textsc{Hofmannsthal, Hugo von} (1874-02-01 – 1929-07-15), \emph{Schriftsteller}|pwk}: \emph{Aufzeichnungen}. Hg. Rudolf Hirsch † und Ellen Ritter † in
                     Zusammenarbeit mit Konrad Heumann und Peter Michael Braunwarth. Frankfurt am
                     Main: \emph{S. Fischer}\orgindex{S. Fischer Verlag@S. Fischer Verlag|pwk}{ }2013, Erläuterungen, S. 789 (\emph{Sämtliche
                        Werke}, XXXIX)) Am Folgetag, dem 25. 7. 1904,
                  hält er zudem den Abschluss in einer persönlichen Aufzeichnung fest.
                     (S. 482.)}}}\label{K_L01418-1h}. Was noch {\pb}daran zu thun iſt, das wenige
               läſst ſich unter dem Abſchreiben thun.\hspace*{1.5em}Indeſſen ſind
               aber, wie leuchtende Wolkeninſeln hinter den Bergen hervor andere Stoffe geſtiegen,
               zum Theil aus dem geheimnisvollen Abgrund des niemals ſchlafenden, umbildenden
               Gedächtniſſes: das »Leben ein Traum\pwindex{Hofmannsthal, Hugo von 1874-02-01 – 1929-07-15@\textsc{Hofmannsthal, Hugo von} (1874-02-01 – 1929-07-15), \emph{Schriftsteller}!Turm. Ein Trauerspiel1925@\strich\emph{Der Turm. Ein Trauerspiel} {[}1925{]}|pw}« dieſer faſt
               zu große Stoff, hat ſeinen tiefen {\pb}dem Calderon\pwindex{Calderón de la Barca, Pedro 17.01.1600 – 25.05.1681@\textsc{Calderón de la Barca, Pedro} (17.01.1600 – 25.05.1681), \emph{Schriftsteller}|pw} faſt entgegen geſetzten Schluſs
               gefunden, »\textsc{Pentheus}\pwindex{Hofmannsthal, Hugo von 1874-02-01 – 1929-07-15@\textsc{Hofmannsthal, Hugo von} (1874-02-01 – 1929-07-15), \emph{Schriftsteller}!Pentheus. Trauerspiel in zwei Aufzuegen1936@\strich\emph{Pentheus. Trauerspiel in zwei Aufzügen} {[}1936{]}|pw}« im Stoff den \textsc{Bacchen}\pwindex{Euripides 485? – 406? v. u. Z.@\textsc{Euripides} (485? – 406? v. u. Z.), \emph{Schriftsteller}!Bakchen.0405@\strich\emph{Die Bakchen} {[}.0405{]}|pw} des \textsc{Euripides}\pwindex{Euripides 485? – 406? v. u. Z.@\textsc{Euripides} (485? – 406? v. u. Z.), \emph{Schriftsteller}|pw} nahe, aber viel reicher und ſchöner, hat ſich zum Scenarium gegliedert,
               zweiactig; »\textsc{Orest in Delphi}\pwindex{Hofmannsthal, Hugo von 1874-02-01 – 1929-07-15@\textsc{Hofmannsthal, Hugo von} (1874-02-01 – 1929-07-15), \emph{Schriftsteller}!Orest in Delphi@\strich\emph{Orest in Delphi}|pw}« der \textsc{Elektra}\pwindex{Hofmannsthal, Hugo von 1874-02-01 – 1929-07-15@\textsc{Hofmannsthal, Hugo von} (1874-02-01 – 1929-07-15), \emph{Schriftsteller}!Elektra. Tragoedie in einem Aufzug1903@\strich\emph{Elektra. Tragödie in einem Aufzug} {[}1903{]}|pw} 2\textsuperscript{ter} Theil zeigt ſeine Geſtalten unheimlich
               deutlich – mit dieſer Fracht gehe ich den 31\textsuperscript{ten} nach \textsc{Markt-Aussee}, Rammgut\oindex{Ramgut@\textbf{Ramgut}|pw}.\pend
           \pstart Laſſen Sie mich hier oder dort nicht ohne Nachricht. Ihr\spacefill\mbox{Hugo.}\pend{}
         
         \endnumbering\mylabel{h}\end{ledgroupsized}  \newcommand{\dateiname}{L01418}\newcommand{\titel}{Hugo von Hofmannsthal an Arthur Schnitzler, [24./25.?] 7. 1904}\newcommand{\editorInnen}{Martin Anton Müller und Gerd-Hermann Susen}%% latex-leseansicht-abspann.tex
%% Abspann für die Leseansicht.
%% Der Schalter \ifkorrekturansicht ist bereits durch den Vorspann gesetzt.

%% latex-abspann.tex
%% Gemeinsamer Abspann für Korrekturansicht und Leseansicht.
%% Setzt den Schalter \ifkorrekturansicht voraus (gesetzt in den
%% einbindenden Dateien latex-korrekturansicht-abspann.tex bzw.
%% latex-leseansicht-abspann.tex).
%% ---------------------------------------------------------------

\normalsize

% Das esempio-Environment wird nur in der Leseansicht benötigt
\ifkorrekturansicht\else
\newenvironment{esempio}[3]%
{
    \vspace{1.5ex}
    \rlap{\underline{#1}}
    \par
    \setlength{\parindent}{0cm}
    \nopagebreak
    \leftskip=#2cm
    \rightskip=#3cm
}
{
    \par
}
\fi

\doendnotes{C}
\bigskip
\vfill

\clearpage

\footnotesize

\ifkorrekturansicht
  \lohead{\textsc{register}}
\fi

% theindex-Environment neu definieren ohne reledmac
\makeatletter
\renewenvironment{theindex}{%
  \ifkorrekturansicht
    \section*{\indexname}%
  \else
    \subsubsection*{Index der erwähnten Entitäten}%
  \fi
  \setlength{\parindent}{0pt}%
  \setlength{\parskip}{0pt plus 0.3pt}%
  \let\item\@idxitem
}{%
  \ifkorrekturansicht\clearpage\fi
}
\makeatother

\IfFileExists{\jobname-pw.ind}{\input{\jobname-pw.ind}}{}

% Quellenangabe nur in der Leseansicht
\ifkorrekturansicht\else
% Fallback-Definitionen, falls die .tex-Datei \titel etc. nicht gesetzt hat
\providecommand{\titel}{}
\providecommand{\editorInnen}{}
\providecommand{\dateiname}{\jobname}

\vspace{3cm}

\vfill

\footnotesize
\textsc{Quelle}: \titel. Herausgegeben von {\editorInnen}. In: \emph{Arthur Schnitzler: Briefwechsel mit Autorinnen und Autoren}.
 Digitale Edition, https://schnitzler-briefe.acdh.oeaw.ac.at/{\dateiname}.html (Stand \today)
\fi

\end{document}


      