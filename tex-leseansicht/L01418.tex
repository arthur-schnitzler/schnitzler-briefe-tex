%% latex-leseansicht-vorspann.tex
%% Vorspann für die Leseansicht.
%% Lädt die gemeinsame Datei latex-vorspann.tex mit nicht gesetztem Schalter.

\newif\ifkorrekturansicht
\korrekturansichtfalse

\input{../tex-inputs/latex-vorspann}


\section[Hugo von Hofmannsthal an Arthur Schnitzler, {[}24./25.?{]} 7. 1904]{L01418 Hugo von Hofmannsthal an Arthur Schnitzler, [24./25.?] 7. 1904}
\nopagebreak\mylabel{L01418v}
\rehead{ }\normalsize\beginnumbering\briefempfaengerindex{Schnitzler, Arthur@\textsc{Schnitzler, Arthur}!zzzHofmannsthal, Hugo von@\emph{von Hugo von Hofmannsthal}!1904-07-251@{[24./25.?] 7. 1904}|(be}
\toendnotes[C]{\smallbreak\pagebreak[2]}
\correspDesc{Versand  durch Hugo von Hofmannsthal im Zeitraum [24./25.?] 7. 1904 in Bad Fusch
\newline{}Erhalt  durch Arthur Schnitzler in Wien}\toendnotes[C]{\smallbreak}
\Standort{CUL, Schnitzler, B 43.}
\physDesc{Brief, 1 Blatt, 4 Seiten, 1284 Zeichen
\newline{}Handschrift: schwarze Tinte, deutsche Kurrent
\newline{}Schnitzler: mit Bleistift Monat und Jahreszahl ergänzt: »7. 904.« 
\newline{}Ordnung: 1) mit Bleistift von unbekannter Hand nummeriert: »\strikeout{77}«  2) mit Bleistift von unbekannter Hand nummeriert: »230«}
\buchAbdrucke{\weitereDrucke{Hugo von Hofmannsthal, Arthur Schnitzler: \emph{Briefwechsel}. Herausgegeben von Therese Nickl und Heinrich Schnitzler. Frankfurt am Main: \emph{S. Fischer} 1964, S. 191.} }\toendnotes[C]{\smallbreak}
\pstart
           \raggedleft{}{\pb}Bad Fuſch\oindex{Bad Fusch@\textbf{Bad Fusch}|pw}{ }2\textcolor{gray}{×}\textsc{ten}\pend
           
\pstart{}lieber,\pend\vspace{0.5em}
\pstart
           hier bin ich wirklich wie unter dem erſten Anhauch der Luft geſund geworden, und von
               einem innern Reichthum, daſs ich manchmal, gegen Abend, auf eine{ }ſteile Berglehne hin
               aufklettern muſs, nur um das Blut vom Kopf abzuleiten und den unaufhörlichen {\pb}Zudrang von Gedanken, Bildern,
               Situationen, abzuleiten.\pend
           
\pstart
           Es iſt mir{ }ſchwerer, in{ }ſolchen Zeiten ein Buch zu leſen. Ich möchte alles, was mir
               in die Hände fällt, dramatiſieren,{ }ſelbſt den Goethe\pwindex{Goethe, Johann Wolfgang von 28.\,8.\,1749 Frankfurt am Main – 22.\,3.\,1832 Weimar@\textsc{Goethe, Johann Wolfgang von} (28.\,8.\,1749 Frankfurt am Main – 22.\,3.\,1832 Weimar), \emph{Schriftsteller}|pw}–Schiller\pwindex{Schiller, Friedrich von 10.\,11.\,1759 Marbach am Neckar – 9.\,5.\,1805 Weimar@\textsc{Schiller, Friedrich von} (10.\,11.\,1759 Marbach am Neckar – 9.\,5.\,1805 Weimar), \emph{Schriftsteller, Historiker, Philosoph}|pw}’ſchen Briefwechſel\pwindex{Schiller, Friedrich von 10.\,11.\,1759 Marbach am Neckar – 9.\,5.\,1805 Weimar@\textsc{Schiller, Friedrich von} (10.\,11.\,1759 Marbach am Neckar – 9.\,5.\,1805 Weimar), \emph{Schriftsteller, Historiker, Philosoph}!Briefwechsel zwischen Schiller und Goethe@\strich\emph{Briefwechsel zwischen Schiller und Goethe}|pw}\pwindex{Goethe, Johann Wolfgang von 28.\,8.\,1749 Frankfurt am Main – 22.\,3.\,1832 Weimar@\textsc{Goethe, Johann Wolfgang von} (28.\,8.\,1749 Frankfurt am Main – 22.\,3.\,1832 Weimar), \emph{Schriftsteller}!Briefwechsel zwischen Schiller und Goethe@\strich\emph{Briefwechsel zwischen Schiller und Goethe}|pw}, oder die Linzer Tages-poſt\pwindex{Linzer Tages-Post@\emph{Linzer Tages-Post}|pw}.\pend
           
\pstart
           Das »\label{T_L01418-1v}\edtext{gerettete Venedig\pwindex{Hofmannsthal, Hugo von 1.\,2.\,1874 Wien – 15.\,7.\,1929 Rodaun@\textsc{Hofmannsthal, Hugo von} (1.\,2.\,1874 Wien – 15.\,7.\,1929 Rodaun), \emph{Schriftsteller}!gerettete Venedig. Trauerspiel in fünf Aufzügen@\strich\emph{Das gerettete Venedig. Trauerspiel in fünf Aufzügen}|pw}}{\lemma{\textnormal{\emph{gerettete Venedig}}}\Cendnote{\textnormal{wohl von Schnitzler mit Bleistift
                  unterstrichen}}}\label{T_L01418-1}« hab ich \label{K_L01418-1v}\edtext{heute
                  abgeschloſſen}{\lemma{\textnormal{\emph{heute
                  abgeschlossen}}}\Cendnote{\textnormal{Das erlaubt die
                  annähernde Datierung: Am 24. 7. 1904 schrieb Hofmannsthal\pwindex{Hofmannsthal, Hugo von 1.\,2.\,1874 Wien – 15.\,7.\,1929 Rodaun@\textsc{Hofmannsthal, Hugo von} (1.\,2.\,1874 Wien – 15.\,7.\,1929 Rodaun), \emph{Schriftsteller}|pwk} dem Vater\pwindex{Hofmannsthal, Hugo August von 21.\,12.\,1841 Wien – 8.\,12.\,1915 ebd.@\textsc{Hofmannsthal, Hugo August von} (21.\,12.\,1841 Wien – 8.\,12.\,1915 ebd.), \emph{Bankdirektor}|pwkv}, das Stück beendet zu haben (Hugo von Hofmannsthal\pwindex{Hofmannsthal, Hugo von 1.\,2.\,1874 Wien – 15.\,7.\,1929 Rodaun@\textsc{Hofmannsthal, Hugo von} (1.\,2.\,1874 Wien – 15.\,7.\,1929 Rodaun), \emph{Schriftsteller}|pwk}: \emph{Aufzeichnungen}. Herausgegeben von Rudolf Hirsch † und Ellen Ritter
                     † in Zusammenarbeit mit Konrad Heumann und Peter Michael Braunwarth. Frankfurt
                     am Main: \emph{S. Fischer}\orgindex{S. Fischer Verlag@S. Fischer Verlag|pwk}{ }2013, Erläuterungen, S. 789 (\emph{Sämtliche
                        Werke}, XXXIX)). Am Folgetag, dem 25. 7. 1904,
                  hielt er zudem den Abschluss in einer persönlichen Aufzeichnung fest
                     (S. 482).}}}\label{K_L01418-1}. Was noch {\pb}daran zu thun iſt, das wenige
               läſst{ }ſich unter dem Abſchreiben thun.\hspace*{1.5em}Indeſſen{ }ſind
               aber, wie leuchtende Wolkeninſeln hinter den Bergen hervor andere Stoffe geſtiegen,
               zum Theil aus dem geheimnisvollen Abgrund des niemals{ }ſchlafenden, umbildenden
               Gedächtniſſes: das »Leben ein Traum\pwindex{Hofmannsthal, Hugo von 1.\,2.\,1874 Wien – 15.\,7.\,1929 Rodaun@\textsc{Hofmannsthal, Hugo von} (1.\,2.\,1874 Wien – 15.\,7.\,1929 Rodaun), \emph{Schriftsteller}!Turm. Ein Trauerspiel@\strich\emph{Der Turm. Ein Trauerspiel}|pw}« dieſer faſt
               zu große Stoff, hat{ }ſeinen tiefen {\pb}dem Calderon\pwindex{Calderón de la Barca, Pedro 17.\,1.\,1600 Madrid – 25.\,5.\,1681 ebd.@\textsc{Calderón de la Barca, Pedro} (17.\,1.\,1600 Madrid – 25.\,5.\,1681 ebd.), \emph{Schriftsteller}|pw} faſt entgegen geſetzten Schluſs
               gefunden, »\textsc{Pentheus}\pwindex{Hofmannsthal, Hugo von 1.\,2.\,1874 Wien – 15.\,7.\,1929 Rodaun@\textsc{Hofmannsthal, Hugo von} (1.\,2.\,1874 Wien – 15.\,7.\,1929 Rodaun), \emph{Schriftsteller}!Pentheus. Trauerspiel in zwei Aufzügen@\strich\emph{Pentheus. Trauerspiel in zwei Aufzügen}|pw}« im Stoff den \textsc{Bacchen}\pwindex{Euripides 485? Salamina – 406? v.\,u.\,Z. Pella@\textsc{Euripides} (485? Salamina – 406? v.\,u.\,Z. Pella), \emph{Schriftsteller}!Bakchen@\strich\emph{Die Bakchen}|pw} des \textsc{Euripides}\pwindex{Euripides 485? Salamina – 406? v.\,u.\,Z. Pella@\textsc{Euripides} (485? Salamina – 406? v.\,u.\,Z. Pella), \emph{Schriftsteller}|pw} nahe, aber viel reicher und{ }ſchöner, hat{ }ſich zum Scenarium gegliedert,
               zweiactig; »\textsc{Orest in Delphi}\pwindex{Hofmannsthal, Hugo von 1.\,2.\,1874 Wien – 15.\,7.\,1929 Rodaun@\textsc{Hofmannsthal, Hugo von} (1.\,2.\,1874 Wien – 15.\,7.\,1929 Rodaun), \emph{Schriftsteller}!Orest in Delphi@\strich\emph{Orest in Delphi}|pw}« der \textsc{Elektra}\pwindex{Hofmannsthal, Hugo von 1.\,2.\,1874 Wien – 15.\,7.\,1929 Rodaun@\textsc{Hofmannsthal, Hugo von} (1.\,2.\,1874 Wien – 15.\,7.\,1929 Rodaun), \emph{Schriftsteller}!Elektra. Tragödie in einem Aufzug@\strich\emph{Elektra. Tragödie in einem Aufzug}|pw} 2\textsuperscript{ter} Theil zeigt{ }ſeine Geſtalten unheimlich
               deutlich – mit dieſer Fracht gehe ich den 31\textsuperscript{ten} nach \textsc{Markt-Aussee}, Rammgut\oindex{Ramgut@\textbf{Ramgut}, \emph{Schloss}|pw}.\pend
           \pstart Laſſen Sie mich hier oder dort nicht ohne Nachricht. Ihr\spacefill\mbox{Hugo.}\pend{}\selectlanguage{ngerman}\endnumbering\briefempfaengerindex{Schnitzler, Arthur@\textsc{Schnitzler, Arthur}!zzzHofmannsthal, Hugo von@\emph{von Hugo von Hofmannsthal}!1904-07-241@{[24./25.?] 7. 1904}|)be}\mylabel{L01418h}  \newcommand{\dateiname}{L01418}\newcommand{\titel}{Hugo von Hofmannsthal an Arthur Schnitzler, [24./25.?] 7. 1904}\newcommand{\editorInnen}{Martin Anton Müller und Gerd-Hermann Susen}%% latex-leseansicht-abspann.tex
%% Abspann für die Leseansicht.
%% Der Schalter \ifkorrekturansicht ist bereits durch den Vorspann gesetzt.

%% latex-abspann.tex
%% Gemeinsamer Abspann für Korrekturansicht und Leseansicht.
%% Setzt den Schalter \ifkorrekturansicht voraus (gesetzt in den
%% einbindenden Dateien latex-korrekturansicht-abspann.tex bzw.
%% latex-leseansicht-abspann.tex).
%% ---------------------------------------------------------------

\normalsize

% Das esempio-Environment wird nur in der Leseansicht benötigt
\ifkorrekturansicht\else
\newenvironment{esempio}[3]%
{
    \vspace{1.5ex}
    \rlap{\underline{#1}}
    \par
    \setlength{\parindent}{0cm}
    \nopagebreak
    \leftskip=#2cm
    \rightskip=#3cm
}
{
    \par
}
\fi

\doendnotes{C}
\bigskip
\vfill

\clearpage

\footnotesize

\ifkorrekturansicht
  \lohead{\textsc{register}}
\fi

% theindex-Environment neu definieren ohne reledmac
\makeatletter
\renewenvironment{theindex}{%
  \ifkorrekturansicht
    \section*{\indexname}%
  \else
    \subsubsection*{Index der erwähnten Entitäten}%
  \fi
  \setlength{\parindent}{0pt}%
  \setlength{\parskip}{0pt plus 0.3pt}%
  \let\item\@idxitem
}{%
  \ifkorrekturansicht\clearpage\fi
}
\makeatother

\IfFileExists{\jobname-pw.ind}{\input{\jobname-pw.ind}}{}

% Quellenangabe nur in der Leseansicht
\ifkorrekturansicht\else
% Fallback-Definitionen, falls die .tex-Datei \titel etc. nicht gesetzt hat
\providecommand{\titel}{}
\providecommand{\editorInnen}{}
\providecommand{\dateiname}{\jobname}

\vspace{3cm}

\vfill

\footnotesize
\textsc{Quelle}: \titel. Herausgegeben von {\editorInnen}. In: \emph{Arthur Schnitzler: Briefwechsel mit Autorinnen und Autoren}.
 Digitale Edition, https://schnitzler-briefe.acdh.oeaw.ac.at/{\dateiname}.html (Stand \today)
\fi

\end{document}


