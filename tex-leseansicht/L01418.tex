%% latex-leseansicht-vorspann.tex
%% Vorspann für die Leseansicht.
%% Lädt die gemeinsame Datei latex-vorspann.tex mit nicht gesetztem Schalter.

\newif\ifkorrekturansicht
\korrekturansichtfalse

\input{../tex-inputs/latex-vorspann}


         \renewcommand{\erwaehnteWerke}{}
               \section[Hugo von Hofmannsthal an Arthur Schnitzler, {[}24./25.?{]} 7. 1904]{ Hugo von Hofmannsthal an Arthur Schnitzler, {[}24./25.?{]} 7. 1904}\nopagebreak\mylabel{v}\rehead{ }\begin{ledgroupsized}[t]{13cm}\normalsize\beginnumbering \toendnotes[C]{\smallbreak\pagebreak[2]} \Standort{CUL, Schnitzler, B 43.}
\physDesc{Brief, 1 Blatt, 4 Seiten
\newline{}Handschrift: schwarze Tinte, deutsche Kurrent
\newline{}Schnitzler: mit Bleistift Monat und Jahreszahl ergänzt: »7. 904.« \newline{}Ordnung: 1) mit Bleistift von unbekannter Hand nummeriert: »\strikeout{77}«  2) mit Bleistift von unbekannter Hand nummeriert:
                                    »230«}\buchAbdrucke{\weitereDrucke{Hugo von Hofmannsthal, Arthur Schnitzler: \emph{Briefwechsel}. Hg. Therese Nickl und Heinrich Schnitzler. Frankfurt am Main: \emph{S. Fischer} 1964, S. 191.} }\toendnotes[C]{\smallbreak}\pstart
           \raggedleft{}{\pb}Bad Fuſch\oindex{XXXX Ortsangabe fehlt|pw}{ }2\textcolor{gray}{×}\textsc{ten}\pend
           \pstart{}lieber, \pend\pstart
           hier bin ich wirklich wie unter dem erſten Anhauch der Luft geſund geworden, und von
               einem innern Reichthum, daſs ich manchmal, gegen Abend, auf eine ſteile Berglehne hin
               aufklettern muſs, nur um das Blut vom Kopf abzuleiten und den unaufhörlichen {\pb}Zudrang von Gedanken, Bildern,
               Situationen, abzuleiten.\pend
           \pstart
           Es iſt mir ſchwerer, in ſolchen Zeiten ein Buch zu leſen. Ich möchte alles, was mir
               in die Hände fällt, dramatiſieren, ſelbſt den Goethe\pwindex{\textcolor{red}{\textsuperscript{XXXX1 indx}}|pw}–Schiller\pwindex{\textcolor{red}{\textsuperscript{XXXX1 indx}}|pw}’ſchen Briefwechſel\textcolor{red}{\textsuperscript{XXXX indx}}, oder die Linzer Tages-poſt\textcolor{red}{\textsuperscript{XXXX indx}}.\pend
           \pstart
           Das »\label{T_L01418_1v}\edtext{gerettete Venedig\textcolor{red}{\textsuperscript{XXXX indx}}}{\lemma{\textnormal{\emph{gerettete Venedig}}}\Cendnote{\textnormal{wohl von Schnitzler mit Bleistift
                  unterstrichen}}}\label{T_L01418_1h}« hab ich \label{K_L01418_1v}\edtext{heute abgeschloſſen}{\lemma{\textnormal{\emph{heute abgeschloſſen}}}\Cendnote{\textnormal{Das erlaubt die
                  annähernde Datierung: Am 24. 7. 1904 schreibt Hofmannsthal\pwindex{Hofmannsthal, Hugo von 1874-02-01 – 1929-07-15@\textsc{Hofmannsthal, Hugo von} (1874-02-01 – 1929-07-15), \emph{Schriftsteller}|pwk} dem Vater\pwindex{\textcolor{red}{\textsuperscript{XXXX1 indx}}|pwkv}, das Stück beendet zu haben. (Hugo von Hofmannsthal\pwindex{Hofmannsthal, Hugo von 1874-02-01 – 1929-07-15@\textsc{Hofmannsthal, Hugo von} (1874-02-01 – 1929-07-15), \emph{Schriftsteller}|pwk}: \emph{Aufzeichnungen}. Hg. Rudolf Hirsch † und Ellen Ritter † in
                     Zusammenarbeit mit Konrad Heumann und Peter Michael Braunwarth. Frankfurt am
                     Main: \emph{S. Fischer}XXXX ORGangabe fehlt{ }2013, Erläuterungen, S. 789 (\emph{Sämtliche
                        Werke}, XXXIX)) Am Folgetag, dem 25. 7. 1904, hält er zudem den
                  Abschluss in einer persönlichen Aufzeichnung fest.
                  (S. 482.)}}}\label{K_L01418_1h}. Was noch {\pb}daran zu thun iſt, das wenige
               läſst ſich unter dem Abſchreiben thun.\hspace*{1.5em}Indeſſen ſind
               aber, wie leuchtende Wolkeninſeln hinter den Bergen hervor andere Stoffe geſtiegen,
               zum Theil aus dem geheimnisvollen Abgrund des niemals ſchlafenden, umbildenden
               Gedächtniſſes: das »Leben ein Traum\textcolor{red}{\textsuperscript{XXXX indx}}« dieſer faſt zu
               große Stoff, hat ſeinen tiefen {\pb}dem Calderon\pwindex{\textcolor{red}{\textsuperscript{XXXX1 indx}}|pw} faſt entgegen geſetzten Schluſs
               gefunden, »\textsc{Pentheus}\textcolor{red}{\textsuperscript{XXXX indx}}« im Stoff den \textsc{Bacchen}\textcolor{red}{\textsuperscript{XXXX indx}} des \textsc{Euripides}\pwindex{\textcolor{red}{\textsuperscript{XXXX1 indx}}|pw} nahe, aber viel reicher und ſchöner, hat ſich zum Scenarium gegliedert,
               zweiactig; »\textsc{Orest in Delphi}\textcolor{red}{\textsuperscript{XXXX indx}}« der \textsc{Elektra}\textcolor{red}{\textsuperscript{XXXX indx}} 2\textsuperscript{ter} Theil zeigt ſeine Geſtalten unheimlich
               deutlich – mit dieſer Fracht gehe ich den 31\textsuperscript{ten} nach \textsc{Markt-Aussee}, Rammgut\oindex{XXXX Ortsangabe fehlt|pw}.\pend
           \pstart Laſſen Sie mich hier oder dort nicht ohne Nachricht. Ihr\spacefill\mbox{Hugo.}\pend{}
         
         \endnumbering\mylabel{h}\end{ledgroupsized}  \newcommand{\dateiname}{L01418}\newcommand{\titel}{Hugo von Hofmannsthal an Arthur Schnitzler, [24./25.?] 7. 1904}\newcommand{\editorInnen}{Martin Anton Müller und Gerd-Hermann Susen}%% latex-leseansicht-abspann.tex
%% Abspann für die Leseansicht.
%% Der Schalter \ifkorrekturansicht ist bereits durch den Vorspann gesetzt.

%% latex-abspann.tex
%% Gemeinsamer Abspann für Korrekturansicht und Leseansicht.
%% Setzt den Schalter \ifkorrekturansicht voraus (gesetzt in den
%% einbindenden Dateien latex-korrekturansicht-abspann.tex bzw.
%% latex-leseansicht-abspann.tex).
%% ---------------------------------------------------------------

\normalsize

% Das esempio-Environment wird nur in der Leseansicht benötigt
\ifkorrekturansicht\else
\newenvironment{esempio}[3]%
{
    \vspace{1.5ex}
    \rlap{\underline{#1}}
    \par
    \setlength{\parindent}{0cm}
    \nopagebreak
    \leftskip=#2cm
    \rightskip=#3cm
}
{
    \par
}
\fi

\doendnotes{C}
\bigskip
\vfill

\clearpage

\footnotesize

\ifkorrekturansicht
  \lohead{\textsc{register}}
\fi

% theindex-Environment neu definieren ohne reledmac
\makeatletter
\renewenvironment{theindex}{%
  \ifkorrekturansicht
    \section*{\indexname}%
  \else
    \subsubsection*{Index der erwähnten Entitäten}%
  \fi
  \setlength{\parindent}{0pt}%
  \setlength{\parskip}{0pt plus 0.3pt}%
  \let\item\@idxitem
}{%
  \ifkorrekturansicht\clearpage\fi
}
\makeatother

\IfFileExists{\jobname-pw.ind}{\input{\jobname-pw.ind}}{}

% Quellenangabe nur in der Leseansicht
\ifkorrekturansicht\else
% Fallback-Definitionen, falls die .tex-Datei \titel etc. nicht gesetzt hat
\providecommand{\titel}{}
\providecommand{\editorInnen}{}
\providecommand{\dateiname}{\jobname}

\vspace{3cm}

\vfill

\footnotesize
\textsc{Quelle}: \titel. Herausgegeben von {\editorInnen}. In: \emph{Arthur Schnitzler: Briefwechsel mit Autorinnen und Autoren}.
 Digitale Edition, https://schnitzler-briefe.acdh.oeaw.ac.at/{\dateiname}.html (Stand \today)
\fi

\end{document}


      