%% latex-korrekturansicht-vorspann.tex
%% Vorspann für die Korrekturansicht.
%% Lädt die gemeinsame Datei latex-vorspann.tex mit gesetztem Schalter.

\newif\ifkorrekturansicht
\korrekturansichttrue

\input{../tex-inputs/latex-vorspann}


\section[Hugo von Hofmannsthal an Arthur Schnitzler, {[}12.? 7. 1897{]}]{L00700 Hugo von Hofmannsthal an Arthur Schnitzler, {[}12.? 7. 1897{]}}
\nopagebreak\mylabel{L00700v}
\rehead{ }\normalsize\beginnumbering\briefempfaengerindex{Schnitzler, Arthur@\textsc{Schnitzler, Arthur}!zzzHofmannsthal, Hugo von@\emph{von Hugo von Hofmannsthal}!1897-07-121@{{[}12.? 7. 1897{]}}|(be}
\toendnotes[C]{\smallbreak\pagebreak[2]}\Standort{CUL, Schnitzler, B 43.}
\physDesc{Brief, 1 Blatt, 2 Seiten, 412 Zeichen
\newline{}Handschrift: schwarze Tinte, deutsche Kurrent
\newline{}Schnitzler: mit Bleistift datiert: »Mitte Juli 97« 
\newline{}Ordnung: 1) mit Bleistift von unbekannter Hand nummeriert: »\strikeout{96}«  2) mit Bleistift von unbekannter Hand nummeriert:
                                    »95«}
\buchAbdrucke{\weitereDrucke{Hugo von Hofmannsthal, Arthur Schnitzler: \emph{Briefwechsel}. Frankfurt am Main: \emph{S. Fischer} 1964, S. 91.} }
\pstart{}{\pb}mein lieber Arthur\pend\vspace{0.5em}
\pstart
           herzlichen Dank für den Brief. \textsc{Poldy}\pwindex{Andrian-Werburg, Leopold von 09.05.1875 – 19.11.1951@\textsc{Andrian-Werburg, Leopold von} (09.05.1875 – 19.11.1951), \emph{Schriftsteller/Schriftstellerin, Diplomat/Diplomatin}|pw}, dem es fortgeſetzt ſehr ſchlecht geht, kommt auf Widerhofers\pwindex{Widerhofer, Hermann 24.03.1832 – 28.07.1901@\textsc{Widerhofer, Hermann} (24.03.1832 – 28.07.1901), \emph{Mediziner/Medizinerin}|pw} dringenden Rat hieher zu mir. Ich muſs daher
               natürlich, um ihm Zeit zur Erholung zu geben, meinen Aufenthalt hier um mindeſt
               10–12 Tage verlängern. Bitte gleich Antwort ob \introOben{}für\introOben{} Sie und
                  Richard\pwindex{Beer-Hofmann, Richard 1866-07-11 – 1945-09-26@\textsc{Beer-Hofmann, Richard} (1866-07-11 – 1945-09-26), \emph{Schriftsteller/Schriftstellerin}|pw} das Salzburg\oindex{Salzburg@\textbf{Salzburg}, \emph{A.ADM2}|pw}er {\pb}\textsc{Rendezvous} in den erſten Tagen des Auguſt{ }ſich eintheilen läſst.\pend
           
\pstart
           Herzlich Ihr{\\[\baselineskip]}\spacefill\mbox{Hugo.}\pend
           \leftskip=0em{}
\pstart
           Bad Fusch\oindex{Bad Fusch@\textbf{Bad Fusch}, \emph{A.ADM3}|pw}, Montag.\pend
           \selectlanguage{ngerman}\endnumbering\briefempfaengerindex{Schnitzler, Arthur@\textsc{Schnitzler, Arthur}!zzzHofmannsthal, Hugo von@\emph{von Hugo von Hofmannsthal}!1897-07-121@{{[}12.? 7. 1897{]}}|)be}\mylabel{L00700h}  \normalsize

\doendnotes{C}
\bigskip
\vfill

\clearpage

\footnotesize

\lohead{\textsc{register}}

% Definiere theindex-Environment komplett neu ohne reledmac
\makeatletter
\renewenvironment{theindex}{%
  \section*{\indexname}%
  \setlength{\parindent}{0pt}%
  \setlength{\parskip}{0pt plus 0.3pt}%
  \let\item\@idxitem
}{%
  \clearpage
}
\makeatother

\IfFileExists{\jobname-pw.ind}{\input{\jobname-pw.ind}}{}

\end{document}

      