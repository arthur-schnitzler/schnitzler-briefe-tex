%% latex-leseansicht-vorspann.tex
%% Vorspann für die Leseansicht.
%% Lädt die gemeinsame Datei latex-vorspann.tex mit nicht gesetztem Schalter.

\newif\ifkorrekturansicht
\korrekturansichtfalse

\input{../tex-inputs/latex-vorspann}


\section[Arthur Schnitzler an Gustav Schwarzkopf, 24. 8. 1898]{L04127 Arthur Schnitzler an Gustav Schwarzkopf, 24. 8. 1898}
\nopagebreak\mylabel{L04127v}
\rehead{ }\normalsize\beginnumbering\briefempfaengerindex{Schwarzkopf, Gustav@\textsc{Schwarzkopf, Gustav}!zzzSchnitzler, Arthur@\emph{von Arthur Schnitzler}!1898-08-241@{24. 8. 1898}|(be}
\toendnotes[C]{\smallbreak\pagebreak[2]}
\correspDesc{Versand  durch Arthur Schnitzler am 24. 8. 1898 in Luzern
\newline{}Erhalt  durch Gustav Schwarzkopf am 28. 8. 1898 in Wien}\toendnotes[C]{\smallbreak}
\Standort{CUL, Schnitzler, B 96.}
\physDesc{Postkarte, 502 Zeichen
\newline{}Handschrift: Bleistift, deutsche Kurrent
\newline{}Versand: 1) Stempel: »\nobreak{}\oindex{Luzern@\textbf{Luzern}|pwk}Luzern Brf. Aufg., 24 VIII 98, XII\nobreak{}«.   2) Stempel: »\nobreak{}\oindex{I., Innere Stadt@\textbf{I., Innere Stadt}, \emph{Verwaltungsgebiet}|pwk}Wien 1/1 1, 28 8 98, Bestellt\nobreak{}«. }\toendnotes[C]{\smallbreak}\pstart{}{\pb}\textsc{Autriche}\oindex{Österreich@\textbf{Österreich}|pw}\pend{}\pstart{}Herrn \textsc{Gustav Schwarzkopf}\pend{}\pstart{}\textsc{Wien}\oindex{Wien@\textbf{Wien}, \emph{Verwaltungsgebiet}|pw}\pend{}\pstart{}\textsc{I. Tiefer Graben 23}\oindex{Wien@\textbf{Wien}!I., Innere Stadt@\textbf{I., Innere Stadt}!Tiefer Graben 23@\textbf{Tiefer Graben 23}, \emph{Wohngebäude}|pw}\pend{}{\bigskip}\vspace{1em}
\pstart
           \noindent{}{\pb}Lieber Guſtav, ſeit \label{K_L04127-1v}\edtext{So{\geminationn}tag Abends}{\lemma{\textnormal{\emph{Sonntag Abends}}}\Cendnote{\textnormal{Vgl. A. S.: \emph{Wiener Schnitzler}, 21. 8. 1898.
               }}}\label{K_L04127-1} bei ich in \textsc{Luzern\oindex{Luzern@\textbf{Luzern}|pw}}, nach einer
      wunderſchönen Reise von \textsc{Interlaken\oindex{Interlaken@\textbf{Interlaken}, \emph{Hauptstadt}|pw}} über \textsc{Mürren\oindex{Mürren@\textbf{Mürren}|pw}}, dann \textsc{Meiringen\oindex{Meiringen@\textbf{Meiringen}|pw}}, Grimsel\oindex{Grimselpass@\textbf{Grimselpass}, \emph{Pass}|pw} und Furka\oindex{Furka Pass@\textbf{Furka Pass}, \emph{Pass}|pw}, u. ſ. w. Ich möchte nicht ungern über
            einige nordital.\oindex{Italien@\textbf{Italien}|pw} Städte nach Hauſe fahren; nur die
      kleinern ko{\geminationm}en in Betracht – \textsc{Venedig\oindex{Venedig@\textbf{Venedig}|pw}}{ }\textsc{Florenz\oindex{Florenz@\textbf{Florenz}|pw}}{ }\textsc{Mailand\oindex{Mailand@\textbf{Mailand}|pw}}
      verſpar ich auf beſſere Zeiten. – Wie wenig ich arbeite, das
      können Sie ſich nicht träumen laſſen. In den erſten
      Sept. Tagen hoff ich Sie wiederzuſehen.\pend
           
\pstart
           Herzlichſte Grüße{\\[\baselineskip]}Ihr \spacefill\mbox{A. S.}\pend
           \leftskip=0em{}\selectlanguage{ngerman}\endnumbering\briefempfaengerindex{Schwarzkopf, Gustav@\textsc{Schwarzkopf, Gustav}!zzzSchnitzler, Arthur@\emph{von Arthur Schnitzler}!1898-08-241@{24. 8. 1898}|)be}\mylabel{L04127h}
\begin{anhang}
\end{anhang}\newcommand{\dateiname}{L04127}\newcommand{\titel}{Arthur Schnitzler an Gustav Schwarzkopf, 24. 8. 1898}\newcommand{\editorInnen}{Herausgegeben von Jahnke, SelmaMüller, Martin Anton}%% latex-leseansicht-abspann.tex
%% Abspann für die Leseansicht.
%% Der Schalter \ifkorrekturansicht ist bereits durch den Vorspann gesetzt.

%% latex-abspann.tex
%% Gemeinsamer Abspann für Korrekturansicht und Leseansicht.
%% Setzt den Schalter \ifkorrekturansicht voraus (gesetzt in den
%% einbindenden Dateien latex-korrekturansicht-abspann.tex bzw.
%% latex-leseansicht-abspann.tex).
%% ---------------------------------------------------------------

\normalsize

% Das esempio-Environment wird nur in der Leseansicht benötigt
\ifkorrekturansicht\else
\newenvironment{esempio}[3]%
{
    \vspace{1.5ex}
    \rlap{\underline{#1}}
    \par
    \setlength{\parindent}{0cm}
    \nopagebreak
    \leftskip=#2cm
    \rightskip=#3cm
}
{
    \par
}
\fi

\doendnotes{C}
\bigskip
\vfill

\clearpage

\footnotesize

\ifkorrekturansicht
  \lohead{\textsc{register}}
\fi

% theindex-Environment neu definieren ohne reledmac
\makeatletter
\renewenvironment{theindex}{%
  \ifkorrekturansicht
    \section*{\indexname}%
  \else
    \subsubsection*{Index der erwähnten Entitäten}%
  \fi
  \setlength{\parindent}{0pt}%
  \setlength{\parskip}{0pt plus 0.3pt}%
  \let\item\@idxitem
}{%
  \ifkorrekturansicht\clearpage\fi
}
\makeatother

\IfFileExists{\jobname-pw.ind}{\input{\jobname-pw.ind}}{}

% Quellenangabe nur in der Leseansicht
\ifkorrekturansicht\else
% Fallback-Definitionen, falls die .tex-Datei \titel etc. nicht gesetzt hat
\providecommand{\titel}{}
\providecommand{\editorInnen}{}
\providecommand{\dateiname}{\jobname}

\vspace{3cm}

\vfill

\footnotesize
\textsc{Quelle}: \titel. Herausgegeben von {\editorInnen}. In: \emph{Arthur Schnitzler: Briefwechsel mit Autorinnen und Autoren}.
 Digitale Edition, https://schnitzler-briefe.acdh.oeaw.ac.at/{\dateiname}.html (Stand \today)
\fi

\end{document}


