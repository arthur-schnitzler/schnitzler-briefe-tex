%% latex-korrekturansicht-vorspann.tex
%% Vorspann für die Korrekturansicht.
%% Lädt die gemeinsame Datei latex-vorspann.tex mit gesetztem Schalter.

\newif\ifkorrekturansicht
\korrekturansichttrue

\input{../tex-inputs/latex-vorspann}


\section[ Felix Salten an Arthur Schnitzler, 15. 7. 1907]{L03488 Felix Salten an Arthur Schnitzler, 15. 7. 1907}
\nopagebreak\mylabel{L03488v}
\rehead{ }\normalsize\beginnumbering\briefempfaengerindex{Schnitzler, Arthur@\textsc{Schnitzler, Arthur}!zzzSalten, Felix@\emph{von Felix Salten}!1907-07-152@{15. 7. 1907}|(be}
\toendnotes[C]{\smallbreak\pagebreak[2]}\Standort{CUL, Schnitzler, B 89, B 1.}
\physDesc{Postkarte, 978 Zeichen
\newline{}Handschrift: schwarze Tinte, lateinische Kurrent
\newline{}Versand: 1) Stempel: »\nobreak{}\oindex{XIX., Doebling@\textbf{XIX., Döbling}, \emph{A.ADM3}|pwk}19/\textsubscript{2} Wien 119, 15. VII. 07, 6\nobreak{}«.   2) Stempel: »\nobreak{}\oindex{Welsberg-Taisten@\textbf{Welsberg-Taisten}, \emph{A.ADM3}|pwk}Welsberg, 16\textcolor{gray}{.} 7. 07\nobreak{}«. 
\newline{}Ordnung: mit Bleistift von unbekannter Hand nummeriert: »231« }\toendnotes[C]{\smallbreak}\pstart{}{\pb}Herrn D\textsuperscript{r} Arthur Schnitzler\pend{}\pstart{}Wildbad Waldbrunn\oindex{Wildbad Waldbrunn@\textbf{Wildbad Waldbrunn}, \emph{S.SPA}|pw}{ }\textsuperscript{b}/Welsberg i
                     Pustertal\oindex{Welsberg-Taisten@\textbf{Welsberg-Taisten}, \emph{A.ADM3}|pw}\pend{}\pstart{}Tirol\oindex{Suedtirol@\textbf{Südtirol}, \emph{A.ADM2}|pw}\pend{}{\bigskip}\vspace{1em}
\pstart
           \noindent{}{\pb}Lieber, für die \label{K_L03488-1v}\edtext{Wochein\oindex{Wocheiner See@\textbf{Wocheiner See}, \emph{See (N.SEE)}|pwv}er Pläne ist Waldbrunn\oindex{Wildbad Waldbrunn@\textbf{Wildbad Waldbrunn}, \emph{S.SPA}|pw}}{\lemma{\textnormal{\emph{Wocheiner … Waldbrunn}}}\Cendnote{\textnormal{Arthur und Olga Schnitzler\pwindex{Schnitzler, Olga 17.01.1882 – 13.01.1970@\textsc{Schnitzler, Olga} (17.01.1882 – 13.01.1970), \emph{Schauspieler/Schauspielerin, Sänger/Sängerin}|pwk} waren zwischen 28. 6. 1907 und 30. 6. 1907 am Wocheiner See\oindex{Wocheiner See@\textbf{Wocheiner See}, \emph{See (N.SEE)}|pwk} gewesen. Seit 4. 7. 1907 und bis zum
                  26. 8. 1907 hielten sie sich im Wildbad Waldbrunn\oindex{Wildbad Waldbrunn@\textbf{Wildbad Waldbrunn}, \emph{S.SPA}|pwk}
                  in Welsberg\oindex{Welsberg-Taisten@\textbf{Welsberg-Taisten}, \emph{A.ADM3}|pwk} auf.}}}\label{K_L03488-1} immerhin ein
               überraschendes Resultat. Aber Welsberg\oindex{Welsberg-Taisten@\textbf{Welsberg-Taisten}, \emph{A.ADM3}|pw} ist sehr
               schön. – Was haben Sie denn für Wetter dort? Bei uns geht man im Winterrock, was die
                  Neue freie Presse\orgindex{Neue Freie Presse@Neue Freie Presse|pw} veranlaßt, ihre
               Sonntagsfeuilletonisten über Hitzschläge plaudern zu laßen. – Gestern wurde \label{K_L03488-2v}\edtext{Beer-Hofmanns\pwindex{Beer-Hofmann, Richard 1866-07-11 – 1945-09-26@\textsc{Beer-Hofmann, Richard} (1866-07-11 – 1945-09-26), \emph{Schriftsteller/Schriftstellerin}|pw}{ }Vater\pwindex{Hofmann, Alois 30.3.1830 – 11.7.1907@\textsc{Hofmann, Alois} (30.3.1830 – 11.7.1907), \emph{Industrieller/Industrielle}|pwv} begraben}{\lemma{\textnormal{\emph{Beer-Hofmanns Vater begraben}}}\Cendnote{\textnormal{Alois Hofmann\pwindex{Hofmann, Alois 30.3.1830 – 11.7.1907@\textsc{Hofmann, Alois} (30.3.1830 – 11.7.1907), \emph{Industrieller/Industrielle}|pwk} war am 11. 7. 1907 gestorben.}}}\label{K_L03488-2}, der furchtbar gelitten
               haben soll. \label{K_L03488-3v}\edtext{Mahler\pwindex{Mahler, Gustav 07.07.1860 – 18.05.1911@\textsc{Mahler, Gustav} (07.07.1860 – 18.05.1911), \emph{Theaterleiter/Theaterleiterin, Komponist/Komponistin, Dirigent/Dirigentin}|pw}’s Kind\pwindex{Mahler, Maria Anna 1902-11-03 – 1907-07-12@\textsc{Mahler, Maria Anna} (1902-11-03 – 1907-07-12)|pwv}}{\lemma{\textnormal{\emph{Mahler’s Kind}}}\Cendnote{\textnormal{Maria Anna Mahler\pwindex{Mahler, Maria Anna 1902-11-03 – 1907-07-12@\textsc{Mahler, Maria Anna} (1902-11-03 – 1907-07-12)|pwk} war am 12. 7. 1907 gestorben.}}}\label{K_L03488-3} – hat mich so
               ergriffen, dass ich garnicht zur Ruhe kommen konnte. – Erinnern Sie sich, dass ich
               seine Kindertotenlieder\pwindex{Kindertotenlieder@\emph{Kindertotenlieder}|pw} nicht hören konnte? –
               Überhaupt ist es ein lieblicher Sommer: mit meinem \label{K_L03488-4v}\edtext{Bruder Emil\pwindex{Salzmann, Michael Emil 1858-01-19 – 1908-06-26@\textsc{Salzmann, Michael Emil} (1858-01-19 – 1908-06-26), \emph{Versicherungsbeamter/Versicherungsbeamtin}|pw}}{\lemma{\textnormal{\emph{Bruder Emil}}}\Cendnote{\textnormal{Siehe Arthur Schnitzler an Felix Salten, 29. 6. 1908 und Felix Salten an Arthur Schnitzler, 5. 7. 1908.
               }}}\label{K_L03488-4} hatte ich noch manchen Schrecken auszustehen. Doch geht’s ihm jetzt in Edlach\oindex{Edlach@\textbf{Edlach}, \emph{P.PPL}|pw} besser. Otti\pwindex{Salten, Ottilie 07.03.1868 – 22.06.1942@\textsc{Salten, Ottilie} (07.03.1868 – 22.06.1942), \emph{Schauspieler/Schauspielerin}|pw} ist dauernd leidend und muß dieser Tage eine Operation überstehen.
               Lauter angenehme Dinge. Ob wir dann noch fortreisen, weiss ich nicht. Sehr weit
               schwerlich. Laßen Sie bald wieder was hören und seien Sie alle von uns herzlichst
               gegrüßt\pend
           \pstart Ihr \spacefill\mbox{Salten}\pend{}
\pstart
           15. 7. 07.\pend
           \selectlanguage{ngerman}\endnumbering\briefempfaengerindex{Schnitzler, Arthur@\textsc{Schnitzler, Arthur}!zzzSalten, Felix@\emph{von Felix Salten}!1907-07-152@{15. 7. 1907}|)be}\mylabel{L03488h}  \normalsize

\doendnotes{C}
\bigskip
\vfill

\clearpage

\footnotesize

\lohead{\textsc{register}}

% Definiere theindex-Environment komplett neu ohne reledmac
\makeatletter
\renewenvironment{theindex}{%
  \section*{\indexname}%
  \setlength{\parindent}{0pt}%
  \setlength{\parskip}{0pt plus 0.3pt}%
  \let\item\@idxitem
}{%
  \clearpage
}
\makeatother

\IfFileExists{\jobname-pw.ind}{\input{\jobname-pw.ind}}{}

\end{document}

      