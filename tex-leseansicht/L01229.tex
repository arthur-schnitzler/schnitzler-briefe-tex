%% latex-korrekturansicht-vorspann.tex
%% Vorspann für die Korrekturansicht.
%% Lädt die gemeinsame Datei latex-vorspann.tex mit gesetztem Schalter.

\newif\ifkorrekturansicht
\korrekturansichttrue

\input{../tex-inputs/latex-vorspann}


\section[Arthur Schnitzler an Hermann Bahr, {[}9. 7. 1902{]}]{L01229 Arthur Schnitzler an Hermann Bahr, {[}9. 7. 1902{]}}
\nopagebreak\mylabel{L01229v}
\rehead{ }\normalsize\beginnumbering\briefempfaengerindex{Bahr, Hermann@\textsc{Bahr, Hermann}!zzzSchnitzler, Arthur@\emph{von Arthur Schnitzler}!1902-07-091@{9. 7. 1902}|(be}
\toendnotes[C]{\smallbreak\pagebreak[2]}\Standort{TMW, HS AM 23386 Ba.}
\physDesc{Brief, 1 Blatt, 2 Seiten, 596 Zeichen
\newline{}Handschrift: schwarze Tinte, deutsche Kurrent
\newline{}Ordnung: Lochung }
\buchAbdrucke{\weitereDrucke{1) Arthur Schnitzler: \emph{The Letters of Arthur Schnitzler to Hermann Bahr}. Chapel Hill: \emph{The University of North Carolina Press} 1978, S. 98.} \weitereDrucke{2) Hermann Bahr, Arthur Schnitzler: \emph{Briefwechsel, Aufzeichnungen, Dokumente (1891–1931)}. Göttingen: \emph{Wallstein} 2018, S. 240.} }\toendnotes[C]{\smallbreak}
\pstart
           \raggedleft{}{\pb}9/7 \label{T_L01229-1v}\edtext{902}{\lemma{\textnormal{\emph{902}}}\Cendnote{\textnormal{Die nachgezogene Ziffer »2« von
                        unbekannter Hand fälschlich durch »7« überschrieben.}}}\label{T_L01229-1}\pend
           \vspace{0.5em}
\pstart
           lieber Hermann,{ }\label{K_L01229-1v}\edtext{beifolgenden Wiſch}{\lemma{\textnormal{\emph{beifolgenden Wiſch}}}\Cendnote{\textnormal{Ein Schreiben von Leopold Hipp\pwindex{Hipp, Leopold *~1853@\textsc{Hipp, Leopold} (*~1853), \emph{Steuerbeamter/Steuerbeamtin}|pwk} mit Aufforderung zur Angabe von Informationen
                  über Bahrs\pwindex{Bahr, Hermann 19.07.1863 – 15.01.1934@\textsc{Bahr, Hermann} (19.07.1863 – 15.01.1934), \emph{Schriftsteller/Schriftstellerin, Kritiker/Kritikerin}|pwk} finanzielle Situation, das sich
                  heute in der \emph{Cambridge University Library} befindet. Bahr\pwindex{Bahr, Hermann 19.07.1863 – 15.01.1934@\textsc{Bahr, Hermann} (19.07.1863 – 15.01.1934), \emph{Schriftsteller/Schriftstellerin, Kritiker/Kritikerin}|pwk} retournierte es wohl
                  mit seinem Antwortschreiben. Siehe Hermann Bahr, Arthur Schnitzler: \emph{Briefwechsel, Aufzeichnungen, Dokumente (1891–1931)}, Leopold Hipp an Arthur Schnitzler, 28. 6. 1902.}}}\label{K_L01229-1} erhielt ich nachgeſandt. Ich beabſichtigte nicht zu
               antworten, aber man ſagt mir, daſs unerhörter Weiſe eine \uline{Verpflichtung} dazu beſteht. Ich würde ſagen, dſs ich keine Ahnung habe. Aber
               vielleicht wünſchest du ſelbst irgend eine andre\substVorne{}\textsuperscript{.}\substDazwischen{}{ }Antwort.\substHinten{} Bitte theile mir mit, was {\pb}du für recht \strikeout{h\textcolor{gray}{ie}lt\textcolor{gray}{e}ſt} hältſt,
               und ſchicke mir das Formular zurück.\pend
           
\pstart
           Ich wollte dich ſelbſt beſuchen, komme aber in den allernächſten Tagen nicht dazu;
               daher iſt leider briefliche Erledigung nothwendig.\pend
           
\pstart
           Die Tour war ſehr ſchön; \textsc{Hugo\pwindex{Hofmannsthal, Hugo von 1874-02-01 – 1929-07-15@\textsc{Hofmannsthal, Hugo von} (1874-02-01 – 1929-07-15), \emph{Schriftsteller/Schriftstellerin}|pw}} iſt noch ein paar Tage in \textsc{Welsberg\oindex{Welsberg-Taisten@\textbf{Welsberg-Taisten}, \emph{A.ADM3}|pw}} geblieben,\pend
           
\pstart
           Von Herzen{\\[\baselineskip]}dein \spacefill\mbox{Arthur}\pend
           \leftskip=0em{}\selectlanguage{ngerman}\endnumbering\briefempfaengerindex{Bahr, Hermann@\textsc{Bahr, Hermann}!zzzSchnitzler, Arthur@\emph{von Arthur Schnitzler}!1902-07-091@{9. 7. 1902}|)be}\mylabel{L01229h}  \normalsize

\doendnotes{C}
\bigskip
\vfill

\clearpage

\footnotesize

\lohead{\textsc{register}}

% Definiere theindex-Environment komplett neu ohne reledmac
\makeatletter
\renewenvironment{theindex}{%
  \section*{\indexname}%
  \setlength{\parindent}{0pt}%
  \setlength{\parskip}{0pt plus 0.3pt}%
  \let\item\@idxitem
}{%
  \clearpage
}
\makeatother

\IfFileExists{\jobname-pw.ind}{\input{\jobname-pw.ind}}{}

\end{document}

      