%% latex-korrekturansicht-vorspann.tex
%% Vorspann für die Korrekturansicht.
%% Lädt die gemeinsame Datei latex-vorspann.tex mit gesetztem Schalter.

\newif\ifkorrekturansicht
\korrekturansichttrue

\input{../tex-inputs/latex-vorspann}


\section[Arthur Schnitzler an Georg Brandes, 8. 10. 1896]{L00602 Arthur Schnitzler an Georg Brandes, 8. 10. 1896}
\nopagebreak\mylabel{L00602v}
\rehead{ }\normalsize\beginnumbering\briefempfaengerindex{Brandes, Georg@\textsc{Brandes, Georg}!zzzSchnitzler, Arthur@\emph{von Arthur Schnitzler}!1896-10-081@{8. 10. 1896}|(be}
\toendnotes[C]{\smallbreak\pagebreak[2]}\Standort{Kopenhagen, Det Kongelige Bibliotek, Georg Brandes Arkiv, box 125.}
\physDesc{Brief, 1 Blatt, 3 Seiten, 1465 Zeichen
\newline{}Handschrift: schwarze Tinte, deutsche Kurrent
\newline{}Ordnung: mit Bleistift von unbekannter Hand auf der ersten Seite:
                                    »Schnitzler« vermerkt und nummeriert:
                                    »5« }
\buchAbdrucke{\weitereDrucke{Georg Brandes, Arthur Schnitzler: \emph{Ein Briefwechsel}. Bern: \emph{Francke} 1956, S. 58.} }\toendnotes[C]{\smallbreak}
\pstart
           \raggedleft{}{\pb}8. X\damage{.} 96. Wien\oindex{Wien@\textbf{Wien}, \emph{A.ADM2}|pw}.\pend
           
\pstart\center{}Verehrteſter Herr Brandes,\pend\vspace{0.5em}
\pstart
           der vollſtändige Titel des Buches lautet:\pend
           
\pstart
           Georg Brandes, Aus dem Reiche des
                     Abſolutismus{[}.{]} Charakterbilder aus \strikeout{dem} Leben, Politik, Sitten, Kunſt, Literatur
                  Rußlands\pwindex{Eindruecke aus Russland@\emph{Eindrücke aus Russland}|pw}. Überſetzt von \textsc{Alfred Forster}\pwindex{Forster, Alfred @\textsc{Forster, Alfred}, \emph{Übersetzer/Übersetzerin}|pw}.\pend
           
\pstart
           \textsc{Leipzig}, bei \textsc{Siegismund u Volkening}\orgindex{Siegismund u Volkening@Siegismund u Volkening|pw}.\pend
           
\pstart
           Was den Artikel über die Cenſur in
                     Polen\oindex{Polen@\textbf{Polen}, \emph{A.PCLI}|pw}\pwindex{Censur in Polen@\emph{Censur in Polen}|pwv} anbelangt, ſo werden freilich wenige auf die Vermuthung ko{\geminationm}en, daſs er aus einem {\pb}zehn Jahre alten Buch herausgeſchrieben iſt, – und
               ich möchte annehmen, daſs das auch der Redaction der Zeit\orgindex{Zeit. Wiener Wochenschrift@Die Zeit. Wiener Wochenschrift|pw} nicht bekannt war, von der Sie übrigens \label{K_L00602-1v}\edtext{perſönlich Aufklärung}{\lemma{\textnormal{\emph{perſönlich Aufklärung}}}\Cendnote{\textnormal{Vgl. Hermann Bahr, Arthur Schnitzler: \emph{Briefwechsel, Aufzeichnungen, Dokumente (1891–1931)}, Hermann Bahr an Georg Brandes, 8. 10. 1896.
               }}}\label{K_L00602-1} beko{\geminationm}en
               ſollen. Ich ſagte Ihnen ſchon im Sommer, daſs man bei uns u. wohl auch in Deutſchland\oindex{Deutschland@\textbf{Deutschland}, \emph{A.PCLI}|pw} keine rechte Vorſtellung davon hat, in
               welcher Art Überſetzungen Ihrer Werke verfertigt und in welcher Art ſie ausgenutzt
               werden. Vielfach iſt ſogar die Anſicht verbreitet, daſs Sie selbſt auch deutſche
               Artikel ſchreiben und manche Ihrer Sachen ſelbſt aus dem däniſchen\oindex{Daenemark@\textbf{Dänemark}, \emph{A.PCLI}|pw} ins deutſche übertragen.\pend
           
\pstart
           {\pb}All dies ſcheint Ihnen zuweilen doch ärgerlich zu
               ſein; aber ich erinnere mich nicht, daſs Sie ſich irgend einmal dagegen öffentlich
               verwahrt haben.\pend
           
\pstart
           Wäre es nicht doch ſchön und gut, wenn Sie das gelegentlich einmal thäten – nicht um
               Ihretwillen – aber um der allgemeinen Bedeutung willen, welche Fragen des
               literariſchen Rechts und des literariſchen Anſtands zukommt. –\pend
           
\pstart
           Verfügen Sie jederzeit über mich und ſeien Sie versichert, daſs ich dem Künſtler und
               dem Menſchen gleich herzlich ergeben bin.\pend
           \pstart Der Ihre mit vielen Grüßen \spacefill\mbox{ArtSchnitzler}\pend{}\selectlanguage{ngerman}\endnumbering\briefempfaengerindex{Brandes, Georg@\textsc{Brandes, Georg}!zzzSchnitzler, Arthur@\emph{von Arthur Schnitzler}!1896-10-081@{8. 10. 1896}|)be}\mylabel{L00602h}  \normalsize

\doendnotes{C}
\bigskip
\vfill

\clearpage

\footnotesize

\lohead{\textsc{register}}

% Definiere theindex-Environment komplett neu ohne reledmac
\makeatletter
\renewenvironment{theindex}{%
  \section*{\indexname}%
  \setlength{\parindent}{0pt}%
  \setlength{\parskip}{0pt plus 0.3pt}%
  \let\item\@idxitem
}{%
  \clearpage
}
\makeatother

\IfFileExists{\jobname-pw.ind}{\input{\jobname-pw.ind}}{}

\end{document}

      