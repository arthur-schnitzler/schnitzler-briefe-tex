%% latex-leseansicht-vorspann.tex
%% Vorspann für die Leseansicht.
%% Lädt die gemeinsame Datei latex-vorspann.tex mit nicht gesetztem Schalter.

\newif\ifkorrekturansicht
\korrekturansichtfalse

\input{../tex-inputs/latex-vorspann}


         
         \renewcommand{\erwaehntePersonen}{Personen: Hermann Bahr, Georg Brandes, Alfred Forster}
         \renewcommand{\erwaehnteInstitutionen}{Institutionen: Die Zeit. Wiener Wochenschrift, Siegismund u Volkening}
         \renewcommand{\erwaehnteOrte}{Orte: Deutschland, Dänemark, Polen, Wien}
         \renewcommand{\erwaehnteWerke}{Werke: Censur in Polen, Eindrücke aus Russland}
               \section[Arthur Schnitzler an Georg Brandes, 8. 10. 1896]{ Arthur Schnitzler an Georg Brandes, 8. 10. 1896}\nopagebreak\mylabel{v}\rehead{ }\begin{ledgroupsized}[t]{13cm}\normalsize\beginnumbering \toendnotes[C]{\smallbreak\pagebreak[2]} \Standort{Kopenhagen, Det Kongelige Bibliotek, Georg Brandes Arkiv, box 125.}
\physDesc{Brief, 1 Blatt, 3 Seiten
\newline{}Handschrift: schwarze Tinte, deutsche Kurrent\newline{}Ordnung: mit Bleistift von unbekannter Hand auf der ersten Seite: »Schnitzler« vermerkt und nummeriert: »5« }\buchAbdrucke{\weitereDrucke{Georg Brandes, Arthur Schnitzler: \emph{Ein Briefwechsel}. Hg. Kurt Bergel. Bern: \emph{Francke} 1956, S. 58.} }\toendnotes[C]{\smallbreak}\pstart
           \raggedleft{}{\pb}8. X\damage{.} 96. Wien\oindex{Wien@\textbf{Wien}|pw}.\pend
           \pstart\center{}Verehrteſter Herr Brandes,\pend\pstart
           der vollſtändige Titel des Buches lautet:\pend
           \pstart
           Georg Brandes, Aus dem Reiche des
                            Abſolutismus{[}.{]} Charakterbilder aus \strikeout{dem} Leben, Politik, Sitten, Kunſt, Literatur
                        Rußlands\pwindex{Brandes, Georg 04.02.1842 – 19.02.1927@\textsc{Brandes, Georg} (04.02.1842 – 19.02.1927)!Eindruecke aus Russland1888@\strich\emph{Eindrücke aus Russland} {[}1888{]}|pw}. Überſetzt von \textsc{Alfred Forster}\pwindex{Forster, Alfred @\textsc{Forster, Alfred}, \emph{Übersetzer}|pw}.\pend
           \pstart
           \textsc{Leipzig}, bei \textsc{Siegismund u Volkening}\orgindex{Siegismund u Volkening@Siegismund u Volkening|pw}.\pend
           \pstart
           Was den Artikel über die Cenſur in
                            Polen\oindex{Polen@\textbf{Polen}|pw}\pwindex{Brandes, Georg 04.02.1842 – 19.02.1927@\textsc{Brandes, Georg} (04.02.1842 – 19.02.1927)!Censur in Polen03. 10. 1896@\strich\emph{Censur in Polen} {[}03. 10. 1896{]}|pwv} anbelangt, ſo werden
                    freilich wenige auf die Vermuthung ko{\geminationm}en, daſs er
                    aus einem {\pb}zehn Jahre alten Buch
                    herausgeſchrieben iſt, – und ich möchte annehmen, daſs das auch der Redaction
                    der Zeit\orgindex{Zeit. Wiener Wochenschrift@Die Zeit. Wiener Wochenschrift|pw} nicht bekannt war, von der Sie
                    übrigens \label{K_L00602_1v}\edtext{perſönlich Aufklärung}{\lemma{\textnormal{\emph{perſönlich Aufklärung}}}\Cendnote{\textnormal{Der Brief Hermann Bahr\pwindex{Bahr, Hermann 19.07.1863 – 15.01.1934@\textsc{Bahr, Hermann} (19.07.1863 – 15.01.1934), \emph{Schriftsteller, Kritiker}|pwk}s an Brandes\pwindex{Brandes, Georg 04.02.1842 – 19.02.1927@\textsc{Brandes, Georg} (04.02.1842 – 19.02.1927)|pwk} ist
                        abgedruckt in Hermann Bahr\pwindex{Bahr, Hermann 19.07.1863 – 15.01.1934@\textsc{Bahr, Hermann} (19.07.1863 – 15.01.1934), \emph{Schriftsteller, Kritiker}|pwk}, Arthur Schnitzler\pwindex{Schnitzler, Arthur 15.05.1862 – 21.10.1931@\textsc{Schnitzler, Arthur} (15.05.1862 – 21.10.1931), \emph{Schriftsteller, Mediziner}|pwk}: \emph{Briefwechsel,
                                Aufzeichnungen, Dokumente}. Hg. Kurt Ifkovits und Martin
                            Anton Müller. Göttingen: \emph{Wallstein}{ }2018, S. 127.}}}\label{K_L00602_1h} beko{\geminationm}en ſollen. Ich ſagte Ihnen ſchon im Sommer, daſs
                    man bei uns u. wohl auch in Deutſchland\oindex{Deutschland@\textbf{Deutschland}|pw} keine
                    rechte Vorſtellung davon hat, in welcher Art Überſetzungen Ihrer Werke
                    verfertigt und in welcher Art ſie ausgenutzt werden. Vielfach iſt ſogar die
                    Anſicht verbreitet, daſs Sie selbſt auch deutſche Artikel ſchreiben und manche
                    Ihrer Sachen ſelbſt aus dem däniſchen\oindex{Daenemark@\textbf{Dänemark}|pw} ins
                    deutſche übertragen.\pend
           \pstart
           {\pb}All dies ſcheint Ihnen zuweilen doch
                    ärgerlich zu ſein; aber ich erinnere mich nicht, daſs Sie ſich irgend einmal
                    dagegen öffentlich verwahrt haben.\pend
           \pstart
           Wäre es nicht doch ſchön und gut, wenn Sie das gelegentlich einmal thäten – nicht
                    um Ihretwillen – aber um der allgemeinen Bedeutung willen, welche Fragen des
                    literariſchen Rechts und des literariſchen Anſtands zukommt. –\pend
           \pstart
           Verfügen Sie jederzeit über mich und ſeien Sie versichert, daſs ich dem Künſtler
                    und dem Menſchen gleich herzlich ergeben bin.\pend
           \pstart Der Ihre mit vielen Grüßen \spacefill\mbox{ArtSchnitzler}\pend{}
         
         \endnumbering\mylabel{h}\end{ledgroupsized}  \newcommand{\dateiname}{L00602}\newcommand{\titel}{Arthur Schnitzler an Georg Brandes, 8. 10. 1896}\newcommand{\editorInnen}{Martin Anton Müller und Gerd-Hermann Susen}%% latex-leseansicht-abspann.tex
%% Abspann für die Leseansicht.
%% Der Schalter \ifkorrekturansicht ist bereits durch den Vorspann gesetzt.

%% latex-abspann.tex
%% Gemeinsamer Abspann für Korrekturansicht und Leseansicht.
%% Setzt den Schalter \ifkorrekturansicht voraus (gesetzt in den
%% einbindenden Dateien latex-korrekturansicht-abspann.tex bzw.
%% latex-leseansicht-abspann.tex).
%% ---------------------------------------------------------------

\normalsize

% Das esempio-Environment wird nur in der Leseansicht benötigt
\ifkorrekturansicht\else
\newenvironment{esempio}[3]%
{
    \vspace{1.5ex}
    \rlap{\underline{#1}}
    \par
    \setlength{\parindent}{0cm}
    \nopagebreak
    \leftskip=#2cm
    \rightskip=#3cm
}
{
    \par
}
\fi

\doendnotes{C}
\bigskip
\vfill

\clearpage

\footnotesize

\ifkorrekturansicht
  \lohead{\textsc{register}}
\fi

% theindex-Environment neu definieren ohne reledmac
\makeatletter
\renewenvironment{theindex}{%
  \ifkorrekturansicht
    \section*{\indexname}%
  \else
    \subsubsection*{Index der erwähnten Entitäten}%
  \fi
  \setlength{\parindent}{0pt}%
  \setlength{\parskip}{0pt plus 0.3pt}%
  \let\item\@idxitem
}{%
  \ifkorrekturansicht\clearpage\fi
}
\makeatother

\IfFileExists{\jobname-pw.ind}{\input{\jobname-pw.ind}}{}

% Quellenangabe nur in der Leseansicht
\ifkorrekturansicht\else
% Fallback-Definitionen, falls die .tex-Datei \titel etc. nicht gesetzt hat
\providecommand{\titel}{}
\providecommand{\editorInnen}{}
\providecommand{\dateiname}{\jobname}

\vspace{3cm}

\vfill

\footnotesize
\textsc{Quelle}: \titel. Herausgegeben von {\editorInnen}. In: \emph{Arthur Schnitzler: Briefwechsel mit Autorinnen und Autoren}.
 Digitale Edition, https://schnitzler-briefe.acdh.oeaw.ac.at/{\dateiname}.html (Stand \today)
\fi

\end{document}


      