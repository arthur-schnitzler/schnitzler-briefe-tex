%% latex-korrekturansicht-vorspann.tex
%% Vorspann für die Korrekturansicht.
%% Lädt die gemeinsame Datei latex-vorspann.tex mit gesetztem Schalter.

\newif\ifkorrekturansicht
\korrekturansichttrue

\input{../tex-inputs/latex-vorspann}


\section[Hugo von Hofmannsthal an Arthur Schnitzler, 14. 1. 1898]{L00762 Hugo von Hofmannsthal an Arthur Schnitzler, 14. 1. 1898}
\nopagebreak\mylabel{L00762v}
\rehead{ }\normalsize\beginnumbering\briefempfaengerindex{Schnitzler, Arthur@\textsc{Schnitzler, Arthur}!zzzHofmannsthal, Hugo von@\emph{von Hugo von Hofmannsthal}!1898-01-141@{14. 1. 1898}|(be}
\toendnotes[C]{\smallbreak\pagebreak[2]}\Standort{CUL, Schnitzler, B 43.}
\physDesc{Kartenbrief, 259 Zeichen
\newline{}Handschrift: schwarze Tinte, deutsche Kurrent
\newline{}Versand: 1) Stempel: »\nobreak{}Wien 3/3, 14. 1. 98, 12 1 N\nobreak{}«.   2) Stempel: »\nobreak{}\oindex{IX., Alsergrund@\textbf{IX., Alsergrund}, \emph{A.ADM3}|pwk}Wien 9/3, 14. 1. 98, 5.N\nobreak{}«. 
\newline{}Schnitzler: mit Bleistift datiert: »14/1 98« 
\newline{}Ordnung: 1) mit Bleistift von unbekannter Hand nummeriert: »\strikeout{106}«  2) mit Bleistift von unbekannter Hand nummeriert: »105«}
\buchAbdrucke{\weitereDrucke{Hugo von Hofmannsthal, Arthur Schnitzler: \emph{Briefwechsel}. Frankfurt am Main: \emph{S. Fischer} 1964, S. 98.} }\toendnotes[C]{\smallbreak}\pstart{}{\pb}\textsc{Herrn D\textsuperscript{r} Arthur Schnitzler}\pend{}\pstart{}\textsc{Wien}\oindex{Wien@\textbf{Wien}, \emph{A.ADM2}|pw}\pend{}\pstart{}\textsc{IX Franckgasse} 1\oindex{Frankgasse 1@\textbf{Frankgasse 1}, \emph{Wohngebäude (K.WHS)}|pw}\pend{}{\bigskip}\vspace{1em}
\pstart{}{\pb}mein lieber Arthur\pend\vspace{0.5em}
\pstart
           wenn Sie zufällig ein oder gar 2 \textsc{entrées} für \label{K_L00762-1v}\edtext{Sonntag}{\lemma{\textnormal{\emph{Sonntag}}}\Cendnote{\textnormal{Am 16. 1. 1898 fand in den Sofiensälen\oindex{Sofiensaele@\textbf{Sofiensäle}, \emph{Veranstaltungsgebäude (K.VSB)}|pwk} in Wien\oindex{Wien@\textbf{Wien}, \emph{A.ADM2}|pwk} eine
                  Wohltätigkeitsveranstaltung zugunsten des Vereins \emph{Ferienheim}\orgindex{Ferienheim@Ferienheim|pwk} statt, der Landaufenthalte von Kindern förderte und
                  organisierte. Von Schnitzler wurden \emph{Weihnachts-Einkäufe}\pwindex{Weihnachts-Einkaeufe@\emph{Weihnachts-Einkäufe}|pwk} und \emph{Abschiedssouper}\pwindex{Abschiedssouper@\emph{Abschiedssouper}|pwk} gegeben.}}}\label{K_L00762-1} übrig hätten und dem \textsc{Poldy}\pwindex{Andrian-Werburg, Leopold von 09.05.1875 – 19.11.1951@\textsc{Andrian-Werburg, Leopold von} (09.05.1875 – 19.11.1951), \emph{Schriftsteller/Schriftstellerin, Diplomat/Diplomatin}|pw}{ }ſchicken wollten (d. h. nur wenn Sie ſie nicht
               anders verwenden wollen) würde es ihm ſehr viel Vergnügen machen.\pend
           
\pstart
           Ihr{\\[\baselineskip]}\spacefill\mbox{Hugo.}\pend
           \leftskip=0em{}\selectlanguage{ngerman}\endnumbering\briefempfaengerindex{Schnitzler, Arthur@\textsc{Schnitzler, Arthur}!zzzHofmannsthal, Hugo von@\emph{von Hugo von Hofmannsthal}!1898-01-141@{14. 1. 1898}|)be}\mylabel{L00762h}  \normalsize

\doendnotes{C}
\bigskip
\vfill

\clearpage

\footnotesize

\lohead{\textsc{register}}

% Definiere theindex-Environment komplett neu ohne reledmac
\makeatletter
\renewenvironment{theindex}{%
  \section*{\indexname}%
  \setlength{\parindent}{0pt}%
  \setlength{\parskip}{0pt plus 0.3pt}%
  \let\item\@idxitem
}{%
  \clearpage
}
\makeatother

\IfFileExists{\jobname-pw.ind}{\input{\jobname-pw.ind}}{}

\end{document}

      