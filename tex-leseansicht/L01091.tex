%% latex-korrekturansicht-vorspann.tex
%% Vorspann für die Korrekturansicht.
%% Lädt die gemeinsame Datei latex-vorspann.tex mit gesetztem Schalter.

\newif\ifkorrekturansicht
\korrekturansichttrue

\input{../tex-inputs/latex-vorspann}


\section[Hugo von Hofmannsthal an Arthur Schnitzler, 1{[}6?{]} 1. 1901]{L01091 Hugo von Hofmannsthal an Arthur Schnitzler, 1{[}6?{]} 1. 1901}
\nopagebreak\mylabel{L01091v}
\rehead{ }\normalsize\beginnumbering\briefempfaengerindex{Schnitzler, Arthur@\textsc{Schnitzler, Arthur}!zzzHofmannsthal, Hugo von@\emph{von Hugo von Hofmannsthal}!1901-01-161@{1{[}6?{]} 1. 1901}|(be}
\toendnotes[C]{\smallbreak\pagebreak[2]}\Standort{CUL, Schnitzler, B 43.}
\physDesc{Brief, 1 Blatt, 3 Seiten, 766 Zeichen
\newline{}Handschrift: schwarze Tinte, deutsche Kurrent
\newline{}Schnitzler: mit schwarzer Tinte datiert: »Januar 901« 
\newline{}Ordnung: mit Bleistift von unbekannter Hand nummeriert:
                                    »171« und frühere Nummerierungen unkenntlich
                                 gemacht }
\buchAbdrucke{\weitereDrucke{Hugo von Hofmannsthal, Arthur Schnitzler: \emph{Briefwechsel}. Frankfurt am Main: \emph{S. Fischer} 1964, S. 145–146.} }\toendnotes[C]{\smallbreak}
\pstart{}{\pb}lieber, \pend\vspace{0.5em}
\pstart
           hier iſt das Bild für die Schauſpielerinnen. Habe aus Neugierde den \label{K_L01091-1v}\edtext{erſten Theil}{\lemma{\textnormal{\emph{erſten Theil}}}\Cendnote{\textnormal{Die Datierung dieses Korrespondenzstücks gelingt durch implizite
                  Faktoren: Die \emph{Neue Deutsche Rundschau}\pwindex{Neue Deutsche Rundschau@\emph{Neue Deutsche Rundschau}|pwk}
                  erschien üblicherweise zur Monatsmitte, was die früheste Möglichkeit der Lektüre
                  von \emph{Frau Bertha Garlan}\pwindex{Frau Bertha Garlan. Roman@\emph{Frau Bertha Garlan. Roman}|pwk} ergibt. Da  Brief vom 17. 1. 1901
                  bereits auf die stattgefundene Lektüre verweist, ist das vorliegende Korrespondenzstück zeitlich davor
                  anzusetzen.}}}\label{K_L01091-1} von »Frau Bertha \textsc{Garlan}\pwindex{Frau Bertha Garlan. Roman@\emph{Frau Bertha Garlan. Roman}|pw}« geleſen und finde es wunderſchön, ſo reif, reich und leicht, voll Ruhe und
               Fülle, in zarten Farben, voll Luft, \uuline{ſehr}{ }ſchön. {\pb}Trotzdem bleibt der Schluſs des
                  »blinden Geronimo\pwindex{blinde Geronimo und sein Bruder@\emph{Der blinde Geronimo und sein Bruder}|pw}« in der gegenwärtigen Form
               mangelhaft, enttäuſchend. Es muſs aber ſehr leicht zu ändern ſein. Aber ich irre mich
               nicht, denn ich habs wieder \substVorne{}\textsuperscript{geſehen}\substDazwischen{}geleſen\substHinten{}.\pend
           
\pstart
           Ich hätte eine große Bitte: Daſs am \label{K_L01091-2v}\edtext{Sonntag}{\lemma{\textnormal{\emph{Sonntag}}}\Cendnote{\textnormal{Vgl. A. S.: \emph{Tagebuch}, 20. 1. 1901.
               }}}\label{K_L01091-2} mit dem Leſen ſchon um ½ 5 begonnen {\pb}wird. Ich freue mich ſeit langem
               mit der Gerty\pwindex{Hofmannsthal, Gertrude von 16.03.1880 – 09.11.1959@\textsc{Hofmannsthal, Gertrude von} (16.03.1880 – 09.11.1959)|pw}, die nie ein Stück\pwindex{Henry IV, Part 1@\emph{Henry IV, Part 1}|pwv} von \textsc{Shakespeare}\pwindex{Shakespeare, William 23.4.1564? – 03.05.1616@\textsc{Shakespeare, William} (23.4.1564? – 03.05.1616), \emph{Schauspieler/Schauspielerin, Dramatiker/Dramatikerin}|pw} geſehen hat, in eines zu gehen und ſo haben wir für Sonntag eine
               Loge für \textsc{Heinrich IV.}\pwindex{Henry IV, Part 1@\emph{Henry IV, Part 1}|pw} beſtellt.\pend
           
\pstart
           Ich hoffe, es läſst ſich durchführen und werde {\pb}pünktlich ½ 5 bei
               Ihnen ſein.\pend
           
\pstart
           Herzlich{\\}\spacefill\mbox{Hugo.}\pend
           \selectlanguage{ngerman}\endnumbering\briefempfaengerindex{Schnitzler, Arthur@\textsc{Schnitzler, Arthur}!zzzHofmannsthal, Hugo von@\emph{von Hugo von Hofmannsthal}!1901-01-161@{1{[}6?{]} 1. 1901}|)be}\mylabel{L01091h}  \normalsize

\doendnotes{C}
\bigskip
\vfill

\clearpage

\footnotesize

\lohead{\textsc{register}}

% Definiere theindex-Environment komplett neu ohne reledmac
\makeatletter
\renewenvironment{theindex}{%
  \section*{\indexname}%
  \setlength{\parindent}{0pt}%
  \setlength{\parskip}{0pt plus 0.3pt}%
  \let\item\@idxitem
}{%
  \clearpage
}
\makeatother

\IfFileExists{\jobname-pw.ind}{\input{\jobname-pw.ind}}{}

\end{document}

      