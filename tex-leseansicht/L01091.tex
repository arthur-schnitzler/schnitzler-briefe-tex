%% latex-leseansicht-vorspann.tex
%% Vorspann für die Leseansicht.
%% Lädt die gemeinsame Datei latex-vorspann.tex mit nicht gesetztem Schalter.

\newif\ifkorrekturansicht
\korrekturansichtfalse

\input{../tex-inputs/latex-vorspann}


\section[Hugo von Hofmannsthal an Arthur Schnitzler, 1{[}6?{]} 1. 1901]{L01091 Hugo von Hofmannsthal an Arthur Schnitzler, 1[6?] 1. 1901}
\nopagebreak\mylabel{L01091v}
\rehead{ }\normalsize\beginnumbering\briefempfaengerindex{Schnitzler, Arthur@\textsc{Schnitzler, Arthur}!zzzHofmannsthal, Hugo von@\emph{von Hugo von Hofmannsthal}!1901-01-161@{1[6?] 1. 1901}|(be}
\toendnotes[C]{\smallbreak\pagebreak[2]}
\correspDesc{Versand  durch Hugo von Hofmannsthal am 1[6?] 1. 1901 \textbf{Ort fehlend} 
\newline{}Erhalt  durch Arthur Schnitzler im Zeitraum [16. 1. 1901
                  – 20. 1. 1901?] in Wien}\toendnotes[C]{\smallbreak}
\Standort{CUL, Schnitzler, B 43.}
\physDesc{Brief, 1 Blatt, 3 Seiten, 766 Zeichen
\newline{}Handschrift: schwarze Tinte, deutsche Kurrent
\newline{}Schnitzler: mit schwarzer Tinte datiert: »Januar 901« 
\newline{}Ordnung: mit Bleistift von unbekannter Hand nummeriert:
                                    »171« und frühere Nummerierungen unkenntlich
                                 gemacht }
\buchAbdrucke{\weitereDrucke{Hugo von Hofmannsthal, Arthur Schnitzler: \emph{Briefwechsel}. Herausgegeben von Therese Nickl und Heinrich Schnitzler. Frankfurt am Main: \emph{S. Fischer} 1964, S. 145–146.} }\toendnotes[C]{\smallbreak}
\pstart{}{\pb}lieber,\pend\vspace{0.5em}
\pstart
           hier iſt das Bild für die Schauſpielerinnen. Habe aus Neugierde den \label{K_L01091-1v}\edtext{erſten Theil}{\lemma{\textnormal{\emph{ersten Theil}}}\Cendnote{\textnormal{Die Datierung dieses Korrespondenzstücks gelingt durch implizite
                  Faktoren: Die \emph{Neue Deutsche Rundschau}\pwindex{Neue Deutsche Rundschau@\emph{Neue Deutsche Rundschau}|pwk}
                  erschien üblicherweise zur Monatsmitte, was die früheste Möglichkeit der Lektüre
                  von \emph{Frau Bertha Garlan}\pwindex{Schnitzler, Arthur 15.\,5.\,1862 Wien – 21.\,10.\,1931 ebd.@\textsc{Schnitzler, Arthur} (15.\,5.\,1862 Wien – 21.\,10.\,1931 ebd.), \emph{Schriftsteller, Mediziner}!Frau Bertha Garlan. Roman@\strich\emph{Frau Bertha Garlan. Roman}|pwk} ergibt. Da  Brief vom XXXX Auszeichnungsfehler: Dokument L01092 nicht gefunden
                  bereits auf die stattgefundene Lektüre verweist, ist das vorliegende Korrespondenzstück zeitlich davor
                  anzusetzen.}}}\label{K_L01091-1} von »Frau Bertha \textsc{Garlan}\pwindex{Schnitzler, Arthur 15.\,5.\,1862 Wien – 21.\,10.\,1931 ebd.@\textsc{Schnitzler, Arthur} (15.\,5.\,1862 Wien – 21.\,10.\,1931 ebd.), \emph{Schriftsteller, Mediziner}!Frau Bertha Garlan. Roman@\strich\emph{Frau Bertha Garlan. Roman}|pw}« geleſen und finde es wunderſchön,{ }ſo reif, reich und leicht, voll Ruhe und
               Fülle, in zarten Farben, voll Luft, \uuline{ſehr}{ }ſchön. {\pb}Trotzdem bleibt der Schluſs des
                  »blinden Geronimo\pwindex{Schnitzler, Arthur 15.\,5.\,1862 Wien – 21.\,10.\,1931 ebd.@\textsc{Schnitzler, Arthur} (15.\,5.\,1862 Wien – 21.\,10.\,1931 ebd.), \emph{Schriftsteller, Mediziner}!blinde Geronimo und sein Bruder@\strich\emph{Der blinde Geronimo und sein Bruder}|pw}« in der gegenwärtigen Form
               mangelhaft, enttäuſchend. Es muſs aber{ }ſehr leicht zu ändern{ }ſein. Aber ich irre mich
               nicht, denn ich habs wieder \substVorne{}\textsuperscript{geſehen}\substDazwischen{}geleſen\substHinten{}.\pend
           
\pstart
           Ich hätte eine große Bitte: Daſs am \label{K_L01091-2v}\edtext{Sonntag}{\lemma{\textnormal{\emph{Sonntag}}}\Cendnote{\textnormal{Vgl. A. S.: \emph{Tagebuch}, 20. 1. 1901.
               }}}\label{K_L01091-2} mit dem Leſen{ }ſchon um ½ 5 begonnen {\pb}wird. Ich freue mich{ }ſeit langem
               mit der Gerty\pwindex{Hofmannsthal, Gertrude von 16.\,3.\,1880 Wien – 9.\,11.\,1959 Paddington@\textsc{Hofmannsthal, Gertrude von} (16.\,3.\,1880 Wien – 9.\,11.\,1959 Paddington)|pw}, die nie ein Stück\pwindex{Shakespeare, William 23.\,4.\,1564? Stratford-upon-Avon – 3.\,5.\,1616 ebd.@\textsc{Shakespeare, William} (23.\,4.\,1564? Stratford-upon-Avon – 3.\,5.\,1616 ebd.), \emph{Schauspieler, Dramatiker}!Henry IV, Part 1@\strich\emph{Henry IV, Part 1}|pwv} von \textsc{Shakespeare}\pwindex{Shakespeare, William 23.\,4.\,1564? Stratford-upon-Avon – 3.\,5.\,1616 ebd.@\textsc{Shakespeare, William} (23.\,4.\,1564? Stratford-upon-Avon – 3.\,5.\,1616 ebd.), \emph{Schauspieler, Dramatiker}|pw} geſehen hat, in eines zu gehen und{ }ſo haben wir für Sonntag eine
               Loge für \textsc{Heinrich IV.}\pwindex{Shakespeare, William 23.\,4.\,1564? Stratford-upon-Avon – 3.\,5.\,1616 ebd.@\textsc{Shakespeare, William} (23.\,4.\,1564? Stratford-upon-Avon – 3.\,5.\,1616 ebd.), \emph{Schauspieler, Dramatiker}!Henry IV, Part 1@\strich\emph{Henry IV, Part 1}|pw} beſtellt.\pend
           
\pstart
           Ich hoffe, es läſst{ }ſich durchführen und werde {\pb}pünktlich ½ 5 bei
               Ihnen{ }ſein.\pend
           
\pstart
           Herzlich{\\}\spacefill\mbox{Hugo.}\pend
           \selectlanguage{ngerman}\endnumbering\briefempfaengerindex{Schnitzler, Arthur@\textsc{Schnitzler, Arthur}!zzzHofmannsthal, Hugo von@\emph{von Hugo von Hofmannsthal}!1901-01-161@{1[6?] 1. 1901}|)be}\mylabel{L01091h}  \newcommand{\dateiname}{L01091}\newcommand{\titel}{Hugo von Hofmannsthal an Arthur Schnitzler, 1[6?] 1. 1901}\newcommand{\editorInnen}{Martin Anton Müller und Gerd-Hermann Susen}%% latex-leseansicht-abspann.tex
%% Abspann für die Leseansicht.
%% Der Schalter \ifkorrekturansicht ist bereits durch den Vorspann gesetzt.

%% latex-abspann.tex
%% Gemeinsamer Abspann für Korrekturansicht und Leseansicht.
%% Setzt den Schalter \ifkorrekturansicht voraus (gesetzt in den
%% einbindenden Dateien latex-korrekturansicht-abspann.tex bzw.
%% latex-leseansicht-abspann.tex).
%% ---------------------------------------------------------------

\normalsize

% Das esempio-Environment wird nur in der Leseansicht benötigt
\ifkorrekturansicht\else
\newenvironment{esempio}[3]%
{
    \vspace{1.5ex}
    \rlap{\underline{#1}}
    \par
    \setlength{\parindent}{0cm}
    \nopagebreak
    \leftskip=#2cm
    \rightskip=#3cm
}
{
    \par
}
\fi

\doendnotes{C}
\bigskip
\vfill

\clearpage

\footnotesize

\ifkorrekturansicht
  \lohead{\textsc{register}}
\fi

% theindex-Environment neu definieren ohne reledmac
\makeatletter
\renewenvironment{theindex}{%
  \ifkorrekturansicht
    \section*{\indexname}%
  \else
    \subsubsection*{Index der erwähnten Entitäten}%
  \fi
  \setlength{\parindent}{0pt}%
  \setlength{\parskip}{0pt plus 0.3pt}%
  \let\item\@idxitem
}{%
  \ifkorrekturansicht\clearpage\fi
}
\makeatother

\IfFileExists{\jobname-pw.ind}{\input{\jobname-pw.ind}}{}

% Quellenangabe nur in der Leseansicht
\ifkorrekturansicht\else
% Fallback-Definitionen, falls die .tex-Datei \titel etc. nicht gesetzt hat
\providecommand{\titel}{}
\providecommand{\editorInnen}{}
\providecommand{\dateiname}{\jobname}

\vspace{3cm}

\vfill

\footnotesize
\textsc{Quelle}: \titel. Herausgegeben von {\editorInnen}. In: \emph{Arthur Schnitzler: Briefwechsel mit Autorinnen und Autoren}.
 Digitale Edition, https://schnitzler-briefe.acdh.oeaw.ac.at/{\dateiname}.html (Stand \today)
\fi

\end{document}


