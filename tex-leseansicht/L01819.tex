%% latex-leseansicht-vorspann.tex
%% Vorspann für die Leseansicht.
%% Lädt die gemeinsame Datei latex-vorspann.tex mit nicht gesetztem Schalter.

\newif\ifkorrekturansicht
\korrekturansichtfalse

\input{../tex-inputs/latex-vorspann}


\section[Olga Schnitzler an Richard Beer-Hofmann, {[}27.? 12. 1908{]}]{L01819 Olga Schnitzler an Richard Beer-Hofmann, {[}27.? 12. 1908{]}}
\nopagebreak\mylabel{L01819v}
\rehead{ }\normalsize\beginnumbering\briefempfaengerindex{Beer-Hofmann, Richard@\textsc{Beer-Hofmann, Richard}!zzzSchnitzler, Olga@\emph{von Olga Schnitzler}!1908-12-272@{{[}27.? 12. 1908{]}}|(be}
\toendnotes[C]{\smallbreak\pagebreak[2]}
\correspDesc{Versand  durch Olga Schnitzler am [27.? 12. 1908] in Wien
\newline{}Erhalt  durch Richard Beer-Hofmann am [27.? 12. 1908] in Wien}\toendnotes[C]{\smallbreak}
\Standort{YCGL, MSS 31.}
\physDesc{Brief, 1 Blatt, 2 Seiten, Kuvert, 527 Zeichen
\newline{}Handschrift: schwarze Tinte, lateinische Kurrent
\newline{}Versand: ohne postalischen Übermittlungsvermerk }\toendnotes[C]{\smallbreak}\pstart{}{\pb}\textcolor{gray}{\textbf{O. S.}}\pend{}{\bigskip}\pstart{}{\pb}Herrn D\textsuperscript{r} Richard
                  Beer-Hofmann\pend{}\pstart{}Wien XVIII\oindex{XVIII., Währing@\textbf{XVIII., Währing}, \emph{Verwaltungsgebiet}|pw}\pend{}\pstart{}Hasenauerstr. 59\oindex{Wien@\textbf{Wien}!XVIII., Währing@\textbf{XVIII., Währing}!Hasenauerstraße 59@\textbf{Hasenauerstraße 59}, \emph{Wohngebäude}|pw}.\pend{}{\bigskip}\vspace{1em}
\pstart
           {\pb}\textcolor{gray}{\textbf{O. S.}}\pend
           \vspace{0.5em}
\pstart
           Lieber Herr Doctor, unser gewohnter \label{K_L01819-1v}\edtext{Sylvester-Familienabend}{\lemma{\textnormal{\emph{Sylvester-Familienabend}}}\Cendnote{\textnormal{Siehe A. S.: \emph{Tagebuch}, 31. 12. 1908.
               }}}\label{K_L01819-1} bei Mama\pwindex{Schnitzler, Louise 8.\,7.\,1840 Kőszeg – 9.\,9.\,1911 Wien@\textsc{Schnitzler, Louise} (8.\,7.\,1840 Kőszeg – 9.\,9.\,1911 Wien)|pwv} findet
               diesmal nicht statt, weil Mama\pwindex{Schnitzler, Louise 8.\,7.\,1840 Kőszeg – 9.\,9.\,1911 Wien@\textsc{Schnitzler, Louise} (8.\,7.\,1840 Kőszeg – 9.\,9.\,1911 Wien)|pwv} verkühlt ist und nicht aufbleiben darf, wir wollen also diesmal den
               Abend bei uns feiern und würden uns sehr sehr freuen, wenn Sie und Frau Paula\pwindex{Beer-Hofmann, Paula 25.\,2.\,1879 Wien – 30.\,10.\,1939 Zürich@\textsc{Beer-Hofmann, Paula} (25.\,2.\,1879 Wien – 30.\,10.\,1939 Zürich)|pw} kommen wollten.\hspace*{1.5em}Ich denke noch {\pb}D\textsuperscript{r}{ }Kaufmann\pwindex{Kaufmann, Arthur 4.\,4.\,1872 Iași – 25.\,7.\,1938 Wien@\textsc{Kaufmann, Arthur} (4.\,4.\,1872 Iași – 25.\,7.\,1938 Wien), \emph{Rechtswissenschaftler, Privatgelehrte, Privatier}|pw}, Van-Jung\pwindex{Van-Jung, Leo 15.\,10.\,1866 Odessa – 2.\,7.\,1939 Riga@\textsc{Van-Jung, Leo} (15.\,10.\,1866 Odessa – 2.\,7.\,1939 Riga), \emph{Gesangspädagoge, Mathematiker}|pw}, Wassermanns\pwindex{Wassermann, Jakob 10.\,3.\,1873 Fürth – 1.\,1.\,1934 Altaussee@\textsc{Wassermann, Jakob} (10.\,3.\,1873 Fürth – 1.\,1.\,1934 Altaussee), \emph{Schriftsteller}|pw}\pwindex{Wassermann, Julie 5.\,12.\,1876 Wien – April 1963 Zürich@\textsc{Wassermann, Julie} (5.\,12.\,1876 Wien – April 1963 Zürich), \emph{Schriftstellerin}|pw} und
               die Agnes\pwindex{Ulmann, Agnes 23.\,12.\,1875 Wien – 1.\,4.\,1942 New York City@\textsc{Ulmann, Agnes} (23.\,12.\,1875 Wien – 1.\,4.\,1942 New York City), \emph{Malerin, Bildhauerin}|pw} zu laden.\pend
           
\pstart
           Bitte lassen Sie mir ein bejahendes Wort sagen.\pend
           
\pstart
           Mit herzlichen Grüssen an Sie Beide\pwindex{Beer-Hofmann, Paula 25.\,2.\,1879 Wien – 30.\,10.\,1939 Zürich@\textsc{Beer-Hofmann, Paula} (25.\,2.\,1879 Wien – 30.\,10.\,1939 Zürich)|pwv} und die Kinder\pwindex{Beer-Hofmann, Naëmah 20.\,12.\,1898 Wien – 10.\,11.\,1971 New York City@\textsc{Beer-Hofmann, Naëmah} (20.\,12.\,1898 Wien – 10.\,11.\,1971 New York City)|pwv}\pwindex{Beer-Hofmann, Mirjam 4.\,9.\,1897 Wien – 24.\,12.\,1984 New York City@\textsc{Beer-Hofmann, Mirjam} (4.\,9.\,1897 Wien – 24.\,12.\,1984 New York City)|pwv}\pwindex{Beer-Hofmann, Gabriel 9.\,1.\,1901 Wien – 24.\,3.\,1971 St Albans@\textsc{Beer-Hofmann, Gabriel} (9.\,1.\,1901 Wien – 24.\,3.\,1971 St Albans), \emph{Schriftsteller, Filmagent}|pwv}\pend
           \pstart \spacefill\mbox{Olga Schnitzler.}\pend{}
\pstart
           \noindent{}Der \label{K_L01819-2v}\edtext{Abend mit T.\pwindex{Trebitsch, Siegfried 22.\,12.\,1868 Wien – 3.\,6.\,1956 Zürich@\textsc{Trebitsch, Siegfried} (22.\,12.\,1868 Wien – 3.\,6.\,1956 Zürich), \emph{Schriftsteller, Übersetzer}|pw}}{\lemma{\textnormal{\emph{Abend mit T.}}}\Cendnote{\textnormal{Vgl. A. S.: \emph{Tagebuch}, 26. 12. 1908.
                  }}}\label{K_L01819-2} war gar nicht so arg.\pend
           \selectlanguage{ngerman}\endnumbering\briefempfaengerindex{Beer-Hofmann, Richard@\textsc{Beer-Hofmann, Richard}!zzzSchnitzler, Olga@\emph{von Olga Schnitzler}!1908-12-272@{{[}27.? 12. 1908{]}}|)be}\mylabel{L01819h}  \newcommand{\dateiname}{L01819}\newcommand{\titel}{Olga Schnitzler an Richard Beer-Hofmann, [27.? 12. 1908]}\newcommand{\editorInnen}{Martin Anton Müller und Gerd-Hermann Susen}%% latex-leseansicht-abspann.tex
%% Abspann für die Leseansicht.
%% Der Schalter \ifkorrekturansicht ist bereits durch den Vorspann gesetzt.

%% latex-abspann.tex
%% Gemeinsamer Abspann für Korrekturansicht und Leseansicht.
%% Setzt den Schalter \ifkorrekturansicht voraus (gesetzt in den
%% einbindenden Dateien latex-korrekturansicht-abspann.tex bzw.
%% latex-leseansicht-abspann.tex).
%% ---------------------------------------------------------------

\normalsize

% Das esempio-Environment wird nur in der Leseansicht benötigt
\ifkorrekturansicht\else
\newenvironment{esempio}[3]%
{
    \vspace{1.5ex}
    \rlap{\underline{#1}}
    \par
    \setlength{\parindent}{0cm}
    \nopagebreak
    \leftskip=#2cm
    \rightskip=#3cm
}
{
    \par
}
\fi

\doendnotes{C}
\bigskip
\vfill

\clearpage

\footnotesize

\ifkorrekturansicht
  \lohead{\textsc{register}}
\fi

% theindex-Environment neu definieren ohne reledmac
\makeatletter
\renewenvironment{theindex}{%
  \ifkorrekturansicht
    \section*{\indexname}%
  \else
    \subsubsection*{Index der erwähnten Entitäten}%
  \fi
  \setlength{\parindent}{0pt}%
  \setlength{\parskip}{0pt plus 0.3pt}%
  \let\item\@idxitem
}{%
  \ifkorrekturansicht\clearpage\fi
}
\makeatother

\IfFileExists{\jobname-pw.ind}{\input{\jobname-pw.ind}}{}

% Quellenangabe nur in der Leseansicht
\ifkorrekturansicht\else
% Fallback-Definitionen, falls die .tex-Datei \titel etc. nicht gesetzt hat
\providecommand{\titel}{}
\providecommand{\editorInnen}{}
\providecommand{\dateiname}{\jobname}

\vspace{3cm}

\vfill

\footnotesize
\textsc{Quelle}: \titel. Herausgegeben von {\editorInnen}. In: \emph{Arthur Schnitzler: Briefwechsel mit Autorinnen und Autoren}.
 Digitale Edition, https://schnitzler-briefe.acdh.oeaw.ac.at/{\dateiname}.html (Stand \today)
\fi

\end{document}


