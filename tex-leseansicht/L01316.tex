%% latex-korrekturansicht-vorspann.tex
%% Vorspann für die Korrekturansicht.
%% Lädt die gemeinsame Datei latex-vorspann.tex mit gesetztem Schalter.

\newif\ifkorrekturansicht
\korrekturansichttrue

\input{../tex-inputs/latex-vorspann}


\section[Richard Beer-Hofmann an Olga Schnitzler, {[}nach dem 25. 8. 1903?{]}]{L01316 Richard Beer-Hofmann an Olga Schnitzler, {[}nach dem
               25. 8. 1903?{]}}
\nopagebreak\mylabel{L01316v}
\rehead{ }\normalsize\beginnumbering\briefempfaengerindex{Schnitzler, Olga@\textsc{Schnitzler, Olga}!zzzBeer-Hofmann, Richard@\emph{von Richard Beer-Hofmann}!1903-12-312@{{[}nach dem
                  25. 8. 1903?{]}}|(be}
\toendnotes[C]{\smallbreak\pagebreak[2]}\Standort{CUL, Schnitzler, B 8.}
\physDesc{Kartenbrief, 136 Zeichen
\newline{}Handschrift: schwarze Tinte, lateinische Kurrent
\newline{}Versand: ohne postalischen Übermittlungsvermerk 
\newline{}Ordnung: 1) von Schnitzler mit Bleistift beschriftet: »\textsc{Beerhofma{\geminationn}}«  2) mit Bleistift von unbekannter Hand nummeriert:
                                    »278b«}\toendnotes[C]{\smallbreak}\pstart{}{\pb}S. H.\pend{}\pstart{}Frau\pend{}\pstart{}Olga Schnitzler\pwindex{Schnitzler, Olga 17.01.1882 – 13.01.1970@\textsc{Schnitzler, Olga} (17.01.1882 – 13.01.1970), \emph{Schauspieler/Schauspielerin, Sänger/Sängerin}|pw}\pend{}{\bigskip}\vspace{1em}
\pstart
           \noindent{}{\pb}{[}Zeichnung, wie eine Braut (Olga Schnitzler) vom Bräutigam (Arthur Schnitzler) in die{]}{ }\label{K_L01316-1v}\edtext{Spöttelgasse 7\oindex{Edmund-Weiss-Gasse 7@\textbf{Edmund-Weiß-Gasse 7}, \emph{Wohngebäude (K.WHS)}|pw}}{\lemma{\textnormal{\emph{Spöttelgasse 7}}}\Cendnote{\textnormal{Die Zeichnung ist undatiert, dürfte
                  aber in zeitlicher Nähe zur Hochzeit und dem darauffolgenden Einzug in die neue
                  Wohnung entstanden sein.}}}\label{K_L01316-1}{ }{[}geführt wird{]}\pend
           \selectlanguage{ngerman}\endnumbering\briefempfaengerindex{Schnitzler, Olga@\textsc{Schnitzler, Olga}!zzzBeer-Hofmann, Richard@\emph{von Richard Beer-Hofmann}!1903-08-262@{{[}nach dem
                  25. 8. 1903?{]}}|)be}\mylabel{L01316h}  \normalsize

\doendnotes{C}
\bigskip
\vfill

\clearpage

\footnotesize

\lohead{\textsc{register}}

% Definiere theindex-Environment komplett neu ohne reledmac
\makeatletter
\renewenvironment{theindex}{%
  \section*{\indexname}%
  \setlength{\parindent}{0pt}%
  \setlength{\parskip}{0pt plus 0.3pt}%
  \let\item\@idxitem
}{%
  \clearpage
}
\makeatother

\IfFileExists{\jobname-pw.ind}{\input{\jobname-pw.ind}}{}

\end{document}

      