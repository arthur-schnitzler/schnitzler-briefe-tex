%% latex-korrekturansicht-vorspann.tex
%% Vorspann für die Korrekturansicht.
%% Lädt die gemeinsame Datei latex-vorspann.tex mit gesetztem Schalter.

\newif\ifkorrekturansicht
\korrekturansichttrue

\input{../tex-inputs/latex-vorspann}


\section[Arthur Schnitzler an Hugo von Hofmannsthal, 22. 7. 1897]{L00710 Arthur Schnitzler an Hugo von Hofmannsthal, 22. 7. 1897}
\nopagebreak\mylabel{L00710v}
\rehead{ }\normalsize\beginnumbering\briefempfaengerindex{Hofmannsthal, Hugo von@\textsc{Hofmannsthal, Hugo von}!zzzSchnitzler, Arthur@\emph{von Arthur Schnitzler}!1897-07-221@{22. 7. 1897}|(be}
\toendnotes[C]{\smallbreak\pagebreak[2]}\Standort{FDH, Hs-30885,63.}
\physDesc{Briefkarte, 581 Zeichen
\newline{}Handschrift: schwarze Tinte, deutsche Kurrent
\newline{}Ordnung: mit Bleistift von Schnitzler – wohl im Zuge der Durchsicht 1929 –
                                 die Jahreszahl ergänzt: »1898?« }
\buchAbdrucke{\weitereDrucke{Hugo von Hofmannsthal, Arthur Schnitzler: \emph{Briefwechsel}. Frankfurt am Main: \emph{S. Fischer} 1964, S. 94.} }\toendnotes[C]{\smallbreak}
\pstart
           \noindent{}{\pb}Mein lieber Hugo. Mit den Aerzten ſieht’s
               hier ſchlecht aus; am liebſten empfehle ich Ihnen Doctor Herſchmann\pwindex{Herschmann @\textsc{Herschmann}, \emph{Mediziner/Medizinerin}|pw}, der wohl der geſcheidteſte iſt, ſelbſt einmal mit
               ſeiner Lunge zu thun hatte u. jetzt ganz geſund iſt. – Es tut mir leid, dſs ich Poldy Andrian\pwindex{Andrian-Werburg, Leopold von 09.05.1875 – 19.11.1951@\textsc{Andrian-Werburg, Leopold von} (09.05.1875 – 19.11.1951), \emph{Schriftsteller/Schriftstellerin, Diplomat/Diplomatin}|pw} nicht in der nächſten Zeit ſehen
               kann; ich denke doch, dſs ihm manches {\pb}auszureden
               wäre. –\pend
           
\pstart
           \label{K_L00710-1v}\edtext{Heute}{\lemma{\textnormal{\emph{Heute}}}\Cendnote{\textnormal{Das erlaubt die Datierung des Korrespondenzstücks, da die
                  angesprochene Aufführung am Saison-Theater\oindex{Stadttheater [Gmunden]@\textbf{Stadttheater [Gmunden]}, \emph{Theater (K.THE)}|pwk}
                  in Gmunden\oindex{Gmunden@\textbf{Gmunden}, \emph{P.PPL}|pwk} am 22. 7. 1897
                  stattfand. Schnitzler und Beer-Hofmann\pwindex{Beer-Hofmann, Richard 1866-07-11 – 1945-09-26@\textsc{Beer-Hofmann, Richard} (1866-07-11 – 1945-09-26), \emph{Schriftsteller/Schriftstellerin}|pwk} nahmen teil.}}}\label{K_L00710-1} fahre ich
               vielleicht mit Richard\pwindex{Beer-Hofmann, Richard 1866-07-11 – 1945-09-26@\textsc{Beer-Hofmann, Richard} (1866-07-11 – 1945-09-26), \emph{Schriftsteller/Schriftstellerin}|pw} nach Gmund\textcolor{gray}{en}\oindex{Gmunden@\textbf{Gmunden}, \emph{P.PPL}|pw}, wo Freiwild\pwindex{Freiwild. Schauspiel in 3 Akten@\emph{Freiwild. Schauspiel in 3 Akten}|pw} iſt; morgen nach Salzburg\oindex{Salzburg@\textbf{Salzburg}, \emph{A.ADM2}|pw}; übermorgen Früh beginnt die bereits
               angedeutete Radtour. Zwei kleine \label{K_L00710-2v}\edtext{Schwäger\pwindex{Reinhard, Carl 01.03.1868 – 1904-09-29@\textsc{Reinhard, Carl} (01.03.1868 – 1904-09-29), \emph{Kapellmeister/Kapellmeisterin}|pwuv}\pwindex{Reinhard, Franz 28.05.1874 – 15.09.1939@\textsc{Reinhard, Franz} (28.05.1874 – 15.09.1939), \emph{Versicherungsbeamter/Versicherungsbeamtin}|pwuv}}{\lemma{\textnormal{\emph{Schwäger}}}\Cendnote{\textnormal{Die Radtour fand nicht statt. Die
                  Edition von Heinrich Schnitzler/Nickl gibt im Kommentar an, dass mit
                  dem »kleinen Schwager« des Briefes vom 21. 7. 1897 ein Bruder\pwindex{Reinhard, Carl 01.03.1868 – 1904-09-29@\textsc{Reinhard, Carl} (01.03.1868 – 1904-09-29), \emph{Kapellmeister/Kapellmeisterin}|pwkv}\pwindex{Reinhard, Franz 28.05.1874 – 15.09.1939@\textsc{Reinhard, Franz} (28.05.1874 – 15.09.1939), \emph{Versicherungsbeamter/Versicherungsbeamtin}|pwkv} von Marie Reinhard\pwindex{Reinhard, Marie 1871-03-13 – 1899-03-18@\textsc{Reinhard, Marie} (1871-03-13 – 1899-03-18), \emph{Gesangspädagoge/Gesangspädagogin}|pwk} gemeint sei. Entsprechend
                  könnten es sich hier um die beiden Brüder Karl\pwindex{Reinhard, Carl 01.03.1868 – 1904-09-29@\textsc{Reinhard, Carl} (01.03.1868 – 1904-09-29), \emph{Kapellmeister/Kapellmeisterin}|pwk} und Franz\pwindex{Reinhard, Franz 28.05.1874 – 15.09.1939@\textsc{Reinhard, Franz} (28.05.1874 – 15.09.1939), \emph{Versicherungsbeamter/Versicherungsbeamtin}|pwk} handeln. Zu der
                  Radreise kam es aber nicht, da Schnitzler
                  nach Wien\oindex{Wien@\textbf{Wien}, \emph{A.ADM2}|pwk} zurückkehrte, um ein Haus für eine
                  versteckte Geburt des gemeinsamen Kindes mit Marie Reinhard\pwindex{Reinhard, Marie 1871-03-13 – 1899-03-18@\textsc{Reinhard, Marie} (1871-03-13 – 1899-03-18), \emph{Gesangspädagoge/Gesangspädagogin}|pwk} zu suchen.}}}\label{K_L00710-2} und wahrſcheinlich Wolzogen\pwindex{Wolzogen, Ernst von 23.04.1855 – 30.07.1934@\textsc{Wolzogen, Ernst von} (23.04.1855 – 30.07.1934), \emph{Schriftsteller/Schriftstellerin}|pw} (Lumpengeſindel\pwindex{Lumpengesindel@\emph{Das Lumpengesindel}|pw})
               ſind mit mir.\pend
           
\pstart
           Herzlichen Gruß,{\\[\baselineskip]} Ihr \spacefill\mbox{Arthur}\pend
           \leftskip=0em{}\selectlanguage{ngerman}\endnumbering\briefempfaengerindex{Hofmannsthal, Hugo von@\textsc{Hofmannsthal, Hugo von}!zzzSchnitzler, Arthur@\emph{von Arthur Schnitzler}!1897-07-221@{22. 7. 1897}|)be}\mylabel{L00710h}  \normalsize

\doendnotes{C}
\bigskip
\vfill

\clearpage

\footnotesize

\lohead{\textsc{register}}

% Definiere theindex-Environment komplett neu ohne reledmac
\makeatletter
\renewenvironment{theindex}{%
  \section*{\indexname}%
  \setlength{\parindent}{0pt}%
  \setlength{\parskip}{0pt plus 0.3pt}%
  \let\item\@idxitem
}{%
  \clearpage
}
\makeatother

\IfFileExists{\jobname-pw.ind}{\input{\jobname-pw.ind}}{}

\end{document}

      