%% latex-leseansicht-vorspann.tex
%% Vorspann für die Leseansicht.
%% Lädt die gemeinsame Datei latex-vorspann.tex mit nicht gesetztem Schalter.

\newif\ifkorrekturansicht
\korrekturansichtfalse

\input{../tex-inputs/latex-vorspann}


\section[Arthur Schnitzler an Hugo von Hofmannsthal, 22. 7. 1897]{L00710 Arthur Schnitzler an Hugo von Hofmannsthal, 22. 7. 1897}
\nopagebreak\mylabel{L00710v}
\rehead{ }\normalsize\beginnumbering\briefempfaengerindex{Hofmannsthal, Hugo von@\textsc{Hofmannsthal, Hugo von}!zzzSchnitzler, Arthur@\emph{von Arthur Schnitzler}!1897-07-221@{22. 7. 1897}|(be}
\toendnotes[C]{\smallbreak\pagebreak[2]}
\correspDesc{Versand  durch Arthur Schnitzler am 22. 7. 1897 in Bad Ischl
\newline{}Erhalt  durch Hugo von Hofmannsthal im Zeitraum [23. 7. 1897
                  – 27. 7. 1897?] in Wien}\toendnotes[C]{\smallbreak}
\Standort{FDH, Hs-30885,63.}
\physDesc{Briefkarte, 581 Zeichen
\newline{}Handschrift: schwarze Tinte, deutsche Kurrent
\newline{}Ordnung: mit Bleistift von Schnitzler – wohl im Zuge der Durchsicht 1929 –
                                 die Jahreszahl ergänzt: »1898?« }
\buchAbdrucke{\weitereDrucke{Hugo von Hofmannsthal, Arthur Schnitzler: \emph{Briefwechsel}. Herausgegeben von Therese Nickl und Heinrich Schnitzler. Frankfurt am Main: \emph{S. Fischer} 1964, S. 94.} }\toendnotes[C]{\smallbreak}
\pstart
           \noindent{}{\pb}Mein lieber Hugo. Mit den Aerzten{ }ſieht’s
               hier{ }ſchlecht aus; am liebſten empfehle ich Ihnen Doctor Herſchmann\pwindex{Herschmann @\textsc{Herschmann}, \emph{Mediziner}|pw}, der wohl der geſcheidteſte iſt,{ }ſelbſt einmal mit{ }ſeiner Lunge zu thun hatte u. jetzt ganz geſund iſt. – Es tut mir leid, dſs ich Poldy Andrian\pwindex{Andrian-Werburg, Leopold von 9.\,5.\,1875 Berlin – 19.\,11.\,1951 Fribourg@\textsc{Andrian-Werburg, Leopold von} (9.\,5.\,1875 Berlin – 19.\,11.\,1951 Fribourg), \emph{Schriftsteller, Diplomat}|pw} nicht in der nächſten Zeit{ }ſehen
               kann; ich denke doch, dſs ihm manches {\pb}auszureden
               wäre. –\pend
           
\pstart
           \label{K_L00710-1v}\edtext{Heute}{\lemma{\textnormal{\emph{Heute}}}\Cendnote{\textnormal{Das erlaubt die Datierung des Korrespondenzstücks, da die
                  angesprochene Aufführung am Saison-Theater\oindex{Stadttheater [Gmunden]@\textbf{Stadttheater [Gmunden]}, \emph{Theater}|pwk}
                  in Gmunden\oindex{Gmunden@\textbf{Gmunden}|pwk} am 22. 7. 1897
                  stattfand. Schnitzler und Beer-Hofmann\pwindex{Beer-Hofmann, Richard 11.\,7.\,1866 Wien – 26.\,9.\,1945 New York City@\textsc{Beer-Hofmann, Richard} (11.\,7.\,1866 Wien – 26.\,9.\,1945 New York City), \emph{Schriftsteller}|pwk} nahmen teil.}}}\label{K_L00710-1} fahre ich
               vielleicht mit Richard\pwindex{Beer-Hofmann, Richard 11.\,7.\,1866 Wien – 26.\,9.\,1945 New York City@\textsc{Beer-Hofmann, Richard} (11.\,7.\,1866 Wien – 26.\,9.\,1945 New York City), \emph{Schriftsteller}|pw} nach Gmund\textcolor{gray}{en}\oindex{Gmunden@\textbf{Gmunden}|pw}, wo Freiwild\pwindex{Schnitzler, Arthur 15.\,5.\,1862 Wien – 21.\,10.\,1931 ebd.@\textsc{Schnitzler, Arthur} (15.\,5.\,1862 Wien – 21.\,10.\,1931 ebd.), \emph{Schriftsteller, Mediziner}!Freiwild. Schauspiel in 3 Akten@\strich\emph{Freiwild. Schauspiel in 3 Akten}|pw} iſt; morgen nach Salzburg\oindex{Salzburg@\textbf{Salzburg}, \emph{Verwaltungsgebiet}|pw}; übermorgen Früh beginnt die bereits
               angedeutete Radtour. Zwei kleine \label{K_L00710-2v}\edtext{Schwäger\pwindex{Reinhard, Carl 1.\,3.\,1868 Wien – 29.\,9.\,1904 ebd.@\textsc{Reinhard, Carl} (1.\,3.\,1868 Wien – 29.\,9.\,1904 ebd.), \emph{Kapellmeister}|pwuv}\pwindex{Reinhard, Franz 28.\,5.\,1874 Maria Enzersdorf – 15.\,9.\,1939 Wien@\textsc{Reinhard, Franz} (28.\,5.\,1874 Maria Enzersdorf – 15.\,9.\,1939 Wien), \emph{Versicherungsbeamter}|pwuv}}{\lemma{\textnormal{\emph{Schwäger}}}\Cendnote{\textnormal{Die Radtour fand nicht statt. Die
                  Edition von Heinrich Schnitzler/Nickl gibt im Kommentar an, dass mit
                  dem »kleinen Schwager« des Briefes vom 21. 7. 1897 ein Bruder\pwindex{Reinhard, Carl 1.\,3.\,1868 Wien – 29.\,9.\,1904 ebd.@\textsc{Reinhard, Carl} (1.\,3.\,1868 Wien – 29.\,9.\,1904 ebd.), \emph{Kapellmeister}|pwkv}\pwindex{Reinhard, Franz 28.\,5.\,1874 Maria Enzersdorf – 15.\,9.\,1939 Wien@\textsc{Reinhard, Franz} (28.\,5.\,1874 Maria Enzersdorf – 15.\,9.\,1939 Wien), \emph{Versicherungsbeamter}|pwkv} von Marie Reinhard\pwindex{Reinhard, Marie 13.\,3.\,1871 Wien – 18.\,3.\,1899 ebd.@\textsc{Reinhard, Marie} (13.\,3.\,1871 Wien – 18.\,3.\,1899 ebd.), \emph{Gesangspädagogin}|pwk} gemeint sei. Entsprechend
                  könnten es sich hier um die beiden Brüder Karl\pwindex{Reinhard, Carl 1.\,3.\,1868 Wien – 29.\,9.\,1904 ebd.@\textsc{Reinhard, Carl} (1.\,3.\,1868 Wien – 29.\,9.\,1904 ebd.), \emph{Kapellmeister}|pwk} und Franz\pwindex{Reinhard, Franz 28.\,5.\,1874 Maria Enzersdorf – 15.\,9.\,1939 Wien@\textsc{Reinhard, Franz} (28.\,5.\,1874 Maria Enzersdorf – 15.\,9.\,1939 Wien), \emph{Versicherungsbeamter}|pwk} handeln. Zu der
                  Radreise kam es aber nicht, da Schnitzler
                  nach Wien\oindex{Wien@\textbf{Wien}, \emph{Verwaltungsgebiet}|pwk} zurückkehrte, um ein Haus für eine
                  versteckte Geburt des gemeinsamen Kindes mit Marie Reinhard\pwindex{Reinhard, Marie 13.\,3.\,1871 Wien – 18.\,3.\,1899 ebd.@\textsc{Reinhard, Marie} (13.\,3.\,1871 Wien – 18.\,3.\,1899 ebd.), \emph{Gesangspädagogin}|pwk} zu suchen.}}}\label{K_L00710-2} und wahrſcheinlich Wolzogen\pwindex{Wolzogen, Ernst von 23.\,4.\,1855 Breslau – 30.\,7.\,1934 Puppling@\textsc{Wolzogen, Ernst von} (23.\,4.\,1855 Breslau – 30.\,7.\,1934 Puppling), \emph{Schriftsteller}|pw} (Lumpengeſindel\pwindex{Wolzogen, Ernst von 23.\,4.\,1855 Breslau – 30.\,7.\,1934 Puppling@\textsc{Wolzogen, Ernst von} (23.\,4.\,1855 Breslau – 30.\,7.\,1934 Puppling), \emph{Schriftsteller}!Lumpengesindel@\strich\emph{Das Lumpengesindel}|pw}){ }ſind mit mir.\pend
           
\pstart
           Herzlichen Gruß,{\\[\baselineskip]} Ihr \spacefill\mbox{Arthur}\pend
           \leftskip=0em{}\selectlanguage{ngerman}\endnumbering\briefempfaengerindex{Hofmannsthal, Hugo von@\textsc{Hofmannsthal, Hugo von}!zzzSchnitzler, Arthur@\emph{von Arthur Schnitzler}!1897-07-221@{22. 7. 1897}|)be}\mylabel{L00710h}  \newcommand{\dateiname}{L00710}\newcommand{\titel}{Arthur Schnitzler an Hugo von Hofmannsthal, 22. 7. 1897}\newcommand{\editorInnen}{Martin Anton Müller und Gerd-Hermann Susen}%% latex-leseansicht-abspann.tex
%% Abspann für die Leseansicht.
%% Der Schalter \ifkorrekturansicht ist bereits durch den Vorspann gesetzt.

%% latex-abspann.tex
%% Gemeinsamer Abspann für Korrekturansicht und Leseansicht.
%% Setzt den Schalter \ifkorrekturansicht voraus (gesetzt in den
%% einbindenden Dateien latex-korrekturansicht-abspann.tex bzw.
%% latex-leseansicht-abspann.tex).
%% ---------------------------------------------------------------

\normalsize

% Das esempio-Environment wird nur in der Leseansicht benötigt
\ifkorrekturansicht\else
\newenvironment{esempio}[3]%
{
    \vspace{1.5ex}
    \rlap{\underline{#1}}
    \par
    \setlength{\parindent}{0cm}
    \nopagebreak
    \leftskip=#2cm
    \rightskip=#3cm
}
{
    \par
}
\fi

\doendnotes{C}
\bigskip
\vfill

\clearpage

\footnotesize

\ifkorrekturansicht
  \lohead{\textsc{register}}
\fi

% theindex-Environment neu definieren ohne reledmac
\makeatletter
\renewenvironment{theindex}{%
  \ifkorrekturansicht
    \section*{\indexname}%
  \else
    \subsubsection*{Index der erwähnten Entitäten}%
  \fi
  \setlength{\parindent}{0pt}%
  \setlength{\parskip}{0pt plus 0.3pt}%
  \let\item\@idxitem
}{%
  \ifkorrekturansicht\clearpage\fi
}
\makeatother

\IfFileExists{\jobname-pw.ind}{\input{\jobname-pw.ind}}{}

% Quellenangabe nur in der Leseansicht
\ifkorrekturansicht\else
% Fallback-Definitionen, falls die .tex-Datei \titel etc. nicht gesetzt hat
\providecommand{\titel}{}
\providecommand{\editorInnen}{}
\providecommand{\dateiname}{\jobname}

\vspace{3cm}

\vfill

\footnotesize
\textsc{Quelle}: \titel. Herausgegeben von {\editorInnen}. In: \emph{Arthur Schnitzler: Briefwechsel mit Autorinnen und Autoren}.
 Digitale Edition, https://schnitzler-briefe.acdh.oeaw.ac.at/{\dateiname}.html (Stand \today)
\fi

\end{document}


