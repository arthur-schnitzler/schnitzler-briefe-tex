%% latex-leseansicht-vorspann.tex
%% Vorspann für die Leseansicht.
%% Lädt die gemeinsame Datei latex-vorspann.tex mit nicht gesetztem Schalter.

\newif\ifkorrekturansicht
\korrekturansichtfalse

\input{../tex-inputs/latex-vorspann}


         
         \renewcommand{\erwaehntePersonen}{Personen: Leopold von Andrian-Werburg, Richard Beer-Hofmann,  Herschmann, Hugo von Hofmannsthal, Carl Reinhard, Franz Reinhard, Marie Reinhard, Ernst von Wolzogen}
         \renewcommand{\erwaehnteOrte}{Orte: Bad Ischl, Gmunden, Salzburg, Stadttheater Gmunden, Wien}
         \renewcommand{\erwaehnteWerke}{Werke: Das Lumpengesindel, Freiwild. Schauspiel in 3 Akten}
               \section[Arthur Schnitzler an Hugo von Hofmannsthal, 22. 7. 1897]{ Arthur Schnitzler an Hugo von Hofmannsthal, 22. 7. 1897}\nopagebreak\mylabel{v}\rehead{ }\begin{ledgroupsized}[t]{13cm}\normalsize\beginnumbering \toendnotes[C]{\smallbreak\pagebreak[2]} \Standort{FDH, Hs-30885,63.}
\physDesc{Briefkarte
\newline{}Handschrift: schwarze Tinte, deutsche Kurrent\newline{}Ordnung: von Schnitzler – wohl im Zuge der
                                    Durchsicht 1929 – die Jahreszahl ergänzt: »1898?« }\buchAbdrucke{\weitereDrucke{Hugo von Hofmannsthal, Arthur Schnitzler: \emph{Briefwechsel}. Hg. Therese Nickl und Heinrich Schnitzler. Frankfurt am Main: \emph{S. Fischer} 1964, S. 94.} }\toendnotes[C]{\smallbreak}\pstart
           \noindent{}{\pb}Mein lieber Hugo. Mit den Aerzten
                    ſieht’s hier ſchlecht aus; am liebſten empfehle ich Ihnen Doctor Herſchmann\pwindex{Herschmann @\textsc{Herschmann}, \emph{Mediziner}|pw}, der wohl der geſcheidteſte iſt,
                    ſelbſt einmal mit ſeiner Lunge zu thun hatte u. jetzt ganz geſund iſt. – Es tut
                    mir leid, dſs ich Poldy Andrian\pwindex{Andrian-Werburg, Leopold von 09.05.1875 – 19.11.1951@\textsc{Andrian-Werburg, Leopold von} (09.05.1875 – 19.11.1951), \emph{Schriftsteller, Diplomat}|pw} nicht in
                    der nächſten Zeit ſehen kann; ich denke doch, dſs ihm manches {\pb}auszureden wäre. –\pend
           \pstart
           \label{K_L00710_1v}\edtext{Heute}{\lemma{\textnormal{\emph{Heute}}}\Cendnote{\textnormal{Das erlaubt die Datierung des Korrespondenzstücks, da die
                        angesprochene Aufführung am Saison-Theater\oindex{Stadttheater Gmunden@\textbf{Stadttheater Gmunden}|pwk} in Gmunden\oindex{Gmunden@\textbf{Gmunden}|pwk} am
                            22. 7. 1897 stattfand. Schnitzler\pwindex{Schnitzler, Arthur 15.05.1862 – 21.10.1931@\textsc{Schnitzler, Arthur} (15.05.1862 – 21.10.1931), \emph{Schriftsteller, Mediziner}|pwk} und Beer-Hofmann\pwindex{Beer-Hofmann, Richard 1866-07-11 – 1945-09-26@\textsc{Beer-Hofmann, Richard} (1866-07-11 – 1945-09-26), \emph{Schriftsteller}|pwk}
                        nahmen teil.}}}\label{K_L00710_1h} fahre ich vielleicht mit Richard\pwindex{Beer-Hofmann, Richard 1866-07-11 – 1945-09-26@\textsc{Beer-Hofmann, Richard} (1866-07-11 – 1945-09-26), \emph{Schriftsteller}|pw} nach Gmund\textcolor{gray}{en}\oindex{Gmunden@\textbf{Gmunden}|pw}, wo Freiwild\pwindex{Schnitzler, Arthur 15.05.1862 – 21.10.1931@\textsc{Schnitzler, Arthur} (15.05.1862 – 21.10.1931), \emph{Schriftsteller, Mediziner}!Freiwild. Schauspiel in 3 Akten1896@\strich\emph{Freiwild. Schauspiel in 3 Akten} {[}1896{]}|pw} iſt; morgen nach Salzburg\oindex{Salzburg@\textbf{Salzburg}|pw}; übermorgen Früh beginnt die bereits angedeutete
                    Radtour. Zwei kleine \label{K_L00710_2v}\edtext{Schwäger\pwindex{Reinhard, Carl 01.03.1868 – 1904-09-29@\textsc{Reinhard, Carl} (01.03.1868 – 1904-09-29), \emph{Kapellmeister}|pwuv}\pwindex{Reinhard, Franz 28.05.1874 – 15.09.1939@\textsc{Reinhard, Franz} (28.05.1874 – 15.09.1939), \emph{Versicherungsbeamter}|pwuv}}{\lemma{\textnormal{\emph{Schwäger}}}\Cendnote{\textnormal{Die Radtour fand nicht statt. Die
                        Edition von Heinrich Schnitzler/Nickl gibt im Kommentar an,
                        dass mit dem »kleinen Schwager« des Briefes vom 21. 7. 1897 ein
                            Bruder\pwindex{Reinhard, Carl 01.03.1868 – 1904-09-29@\textsc{Reinhard, Carl} (01.03.1868 – 1904-09-29), \emph{Kapellmeister}|pwkv}\pwindex{Reinhard, Franz 28.05.1874 – 15.09.1939@\textsc{Reinhard, Franz} (28.05.1874 – 15.09.1939), \emph{Versicherungsbeamter}|pwkv} von
                            Marie Reinhard\pwindex{Reinhard, Marie 1871-03-13 – 1899-03-18@\textsc{Reinhard, Marie} (1871-03-13 – 1899-03-18), \emph{Gesangspädagogin}|pwk} gemeint sei.
                        Entsprechend könnten es sich hier um die beiden Brüder Karl\pwindex{Reinhard, Carl 01.03.1868 – 1904-09-29@\textsc{Reinhard, Carl} (01.03.1868 – 1904-09-29), \emph{Kapellmeister}|pwk} und Franz\pwindex{Reinhard, Franz 28.05.1874 – 15.09.1939@\textsc{Reinhard, Franz} (28.05.1874 – 15.09.1939), \emph{Versicherungsbeamter}|pwk}
                        handeln. Zu der Radreise kam es aber nicht, da Schnitzler\pwindex{Schnitzler, Arthur 15.05.1862 – 21.10.1931@\textsc{Schnitzler, Arthur} (15.05.1862 – 21.10.1931), \emph{Schriftsteller, Mediziner}|pwk} nach Wien\oindex{Wien@\textbf{Wien}|pwk} zurückkehrte,
                        um ein Haus für eine versteckte Geburt des gemeinsamen Kindes mit Marie Reinhard\pwindex{Reinhard, Marie 1871-03-13 – 1899-03-18@\textsc{Reinhard, Marie} (1871-03-13 – 1899-03-18), \emph{Gesangspädagogin}|pwk} zu suchen.}}}\label{K_L00710_2h} und
                    wahrſcheinlich Wolzogen\pwindex{Wolzogen, Ernst von 23.04.1855 – 30.07.1934@\textsc{Wolzogen, Ernst von} (23.04.1855 – 30.07.1934), \emph{Schriftsteller}|pw} (Lumpengeſindel\pwindex{Wolzogen, Ernst von 23.04.1855 – 30.07.1934@\textsc{Wolzogen, Ernst von} (23.04.1855 – 30.07.1934), \emph{Schriftsteller}!Lumpengesindel1891-10-07 – 1891-10-30@\strich\emph{Das Lumpengesindel} {[}1891-10-07 – 1891-10-30{]}|pw}) ſind mit mir.\pend
           \pstart
           Herzlichen Gruß,{\\[\baselineskip]} Ihr \spacefill\mbox{Arthur}\pend
           \leftskip=0em{}
         
         \endnumbering\mylabel{h}\end{ledgroupsized}  \newcommand{\dateiname}{L00710}\newcommand{\titel}{Arthur Schnitzler an Hugo von Hofmannsthal, 22. 7. 1897}\newcommand{\editorInnen}{Martin Anton Müller und Gerd-Hermann Susen}%% latex-leseansicht-abspann.tex
%% Abspann für die Leseansicht.
%% Der Schalter \ifkorrekturansicht ist bereits durch den Vorspann gesetzt.

%% latex-abspann.tex
%% Gemeinsamer Abspann für Korrekturansicht und Leseansicht.
%% Setzt den Schalter \ifkorrekturansicht voraus (gesetzt in den
%% einbindenden Dateien latex-korrekturansicht-abspann.tex bzw.
%% latex-leseansicht-abspann.tex).
%% ---------------------------------------------------------------

\normalsize

% Das esempio-Environment wird nur in der Leseansicht benötigt
\ifkorrekturansicht\else
\newenvironment{esempio}[3]%
{
    \vspace{1.5ex}
    \rlap{\underline{#1}}
    \par
    \setlength{\parindent}{0cm}
    \nopagebreak
    \leftskip=#2cm
    \rightskip=#3cm
}
{
    \par
}
\fi

\doendnotes{C}
\bigskip
\vfill

\clearpage

\footnotesize

\ifkorrekturansicht
  \lohead{\textsc{register}}
\fi

% theindex-Environment neu definieren ohne reledmac
\makeatletter
\renewenvironment{theindex}{%
  \ifkorrekturansicht
    \section*{\indexname}%
  \else
    \subsubsection*{Index der erwähnten Entitäten}%
  \fi
  \setlength{\parindent}{0pt}%
  \setlength{\parskip}{0pt plus 0.3pt}%
  \let\item\@idxitem
}{%
  \ifkorrekturansicht\clearpage\fi
}
\makeatother

\IfFileExists{\jobname-pw.ind}{\input{\jobname-pw.ind}}{}

% Quellenangabe nur in der Leseansicht
\ifkorrekturansicht\else
% Fallback-Definitionen, falls die .tex-Datei \titel etc. nicht gesetzt hat
\providecommand{\titel}{}
\providecommand{\editorInnen}{}
\providecommand{\dateiname}{\jobname}

\vspace{3cm}

\vfill

\footnotesize
\textsc{Quelle}: \titel. Herausgegeben von {\editorInnen}. In: \emph{Arthur Schnitzler: Briefwechsel mit Autorinnen und Autoren}.
 Digitale Edition, https://schnitzler-briefe.acdh.oeaw.ac.at/{\dateiname}.html (Stand \today)
\fi

\end{document}


      