%% latex-leseansicht-vorspann.tex
%% Vorspann für die Leseansicht.
%% Lädt die gemeinsame Datei latex-vorspann.tex mit nicht gesetztem Schalter.

\newif\ifkorrekturansicht
\korrekturansichtfalse

\input{../tex-inputs/latex-vorspann}


         
         \renewcommand{\erwaehntePersonen}{Personen: Paul Goldmann, Hermann Nissen, Leopold Sonnemann}
         \renewcommand{\erwaehnteInstitutionen}{Institutionen: Frankfurter Zeitung}
         \renewcommand{\erwaehnteOrte}{Orte: Berlin, Deutsches Theater Berlin, Paris, Wien, rue Feydeau}
         \renewcommand{\erwaehnteWerke}{Werke: Frankfurter Zeitung, Freiwild. Schauspiel in 3 Akten, Liebelei. Schauspiel in drei Akten}
               \section[ Paul Goldmann an Arthur Schnitzler, 1. 11. {[}1896{]}]{ Paul Goldmann an Arthur Schnitzler, 1. 11. {[}1896{]}}\nopagebreak\mylabel{v}\rehead{ }\begin{ledgroupsized}[t]{13cm}\normalsize\beginnumbering\briefempfaengerindex{Schnitzler, Arthur@\textsc{Schnitzler, Arthur}!zzzGoldmann, Paul@\emph{von Paul Goldmann}!1896-11-011@{1. 11. {[}1896{]}}|(be} \toendnotes[C]{\smallbreak\pagebreak[2]} \Standort{DLA, A:Schnitzler, HS.NZ85.1.3166.}
\physDesc{Brief, 1 Blatt, 4 Seiten, 2001 Zeichen
\newline{}Handschrift: blaue Tinte, deutsche Kurrent
\newline{}Schnitzler: mit Bleistift das Jahr »96« vermerkt }\toendnotes[C]{\smallbreak}\pstart
           \noindent{}{\pb}\textcolor{gray}{\textbf{\textbf{Frankfurter Zeitung\orgindex{Frankfurter Zeitung@Frankfurter Zeitung|pw}}}}\pend
           \pstart
           \textcolor{gray}{\textbf{(\begin{otherlanguage}{french}Gazette de Francfort\end{otherlanguage}\orgindex{Frankfurter Zeitung@Frankfurter Zeitung|pw}).}}\pend
           \pstart
           \textcolor{gray}{\textbf{\textbf{\begin{otherlanguage}{french}Fondateur M.\end{otherlanguage}{ }L. Sonnemann\pwindex{Sonnemann, Leopold 1831-10-29 – 1909-10-30@\textsc{Sonnemann, Leopold} (1831-10-29 – 1909-10-30), \emph{Journalist, Herausgeber}|pw}.}}}\pend
           \pstart
           \begin{otherlanguage}{french}\textcolor{gray}{\textbf{Journal\pwindex{?? Werk@Nicht ermittelte Verfasserinnen und Verfasser!Frankfurter Zeitung1856 – 1943@\emph{Frankfurter Zeitung} {[}1856 – 1943{]}|pwv} politique,
                        financier,}}\end{otherlanguage}\pend
           \pstart
           \begin{otherlanguage}{french}\textcolor{gray}{\textbf{commercial et littéraire.}}\end{otherlanguage}\pend
           \pstart
           \begin{otherlanguage}{french}\textcolor{gray}{\textbf{\textbf{Paraissant trois fois par jour.}}}\end{otherlanguage}\hfill \textsc{Paris\oindex{Paris@\textbf{Paris}|pw}}, 1. November.\pend
           \pstart
           \begin{otherlanguage}{french}\textcolor{gray}{\textbf{\textbf{Bureau à Paris\oindex{Paris@\textbf{Paris}|pw}}}}\end{otherlanguage}\pend
           \pstart
           \begin{otherlanguage}{french}\textcolor{gray}{\textbf{\textbf{24. Rue Feydeau\oindex{rue Feydeau@\textbf{rue Feydeau}|pw}.}}}\end{otherlanguage}\pend
           \pstart{}Mein lieber Freund,\pend\pstart
           Es iſt ſehr lieb von Dir, daß Du inmitten all’ Deiner Obliegenheiten in Berlin\oindex{Berlin@\textbf{Berlin}|pw} noch Zeit gefunden, mir zu ſchreiben. Ich
               danke Dir und \label{K_L02789-1v}\edtext{ſende Dir dieſe
                  Zeilen}{\lemma{\textnormal{\emph{ſende Dir dieſe
                  Zeilen}}}\Cendnote{\textnormal{Goldmann\pwindex{Goldmann, Paul 31.01.1865 – 25.09.1935@\textsc{Goldmann, Paul} (31.01.1865 – 25.09.1935), \emph{Schriftsteller, Journalist}|pwk} ging also davon aus, dass ein am
                     Sonntag in Paris\oindex{Paris@\textbf{Paris}|pwk} abgeschickter Brief am Morgen des Dienstags in Berlin\oindex{Berlin@\textbf{Berlin}|pwk} vorliegen konnte.}}}\label{K_L02789-1h} nur, damit Du am Morgen des
               entſcheidenden \label{K_L02789-2v}\edtext{Tag}{\lemma{\textnormal{\emph{Tag}}}\Cendnote{\textnormal{die Uraufführung des \emph{Freiwild}\pwindex{Schnitzler, Arthur 15.05.1862 – 21.10.1931@\textsc{Schnitzler, Arthur} (15.05.1862 – 21.10.1931), \emph{Schriftsteller, Mediziner}!Freiwild. Schauspiel in 3 Akten1896@\strich\emph{Freiwild. Schauspiel in 3 Akten} {[}1896{]}|pwk}s am 3. 11. 1896 am Deutschen Theater\oindex{Deutsches Theater Berlin@\textbf{Deutsches Theater Berlin}|pwk} in Berlin\oindex{Berlin@\textbf{Berlin}|pwk}}}}\label{K_L02789-2h}es einen Gruß von mir bekommſt. Das heißt: entſcheiden wird der Tag gar nichts. Alles Weſentliche iſt entſchieden. Wir
               wiſſen Alle, wer Du biſt; und Dein neues Stück\pwindex{Schnitzler, Arthur 15.05.1862 – 21.10.1931@\textsc{Schnitzler, Arthur} (15.05.1862 – 21.10.1931), \emph{Schriftsteller, Mediziner}!Freiwild. Schauspiel in 3 Akten1896@\strich\emph{Freiwild. Schauspiel in 3 Akten} {[}1896{]}|pwv}, \label{K_L02789-3v}\edtext{wenn es
               Erfolg hat}{\lemma{\textnormal{\emph{wenn es
               Erfolg hat}}}\Cendnote{\textnormal{\emph{Freiwild}\pwindex{Schnitzler, Arthur 15.05.1862 – 21.10.1931@\textsc{Schnitzler, Arthur} (15.05.1862 – 21.10.1931), \emph{Schriftsteller, Mediziner}!Freiwild. Schauspiel in 3 Akten1896@\strich\emph{Freiwild. Schauspiel in 3 Akten} {[}1896{]}|pwk} war nicht ansatzweise so erfolgreich
                  wie die \emph{Liebelei}\pwindex{Schnitzler, Arthur 15.05.1862 – 21.10.1931@\textsc{Schnitzler, Arthur} (15.05.1862 – 21.10.1931), \emph{Schriftsteller, Mediziner}!Liebelei. Schauspiel in drei Akten1895-10-09@\strich\emph{Liebelei. Schauspiel in drei Akten} {[}1895-10-09{]}|pwk}.}}}\label{K_L02789-3h}, kann uns {\pb}nichts Neues lehren, – wenn \strikeout{\textcolor{gray}{ſ}} ſein Erfolg beſtritten wird, kann es an der bereits beſtehenden Thatſache
               nichts ändern, daß \textsc{Arthur Schnitzler} in der gegenwärtigen
               deutſchen dramatiſchen Bewegung eine der wenigen bemerkenswerthen Erſcheinungen iſt.
               Ich ſehe alſo dem 3. November lange nicht mit
               derſelben Spannung entgegen, wie dem Tage der \textsc{Première} der »Liebelei\pwindex{Schnitzler, Arthur 15.05.1862 – 21.10.1931@\textsc{Schnitzler, Arthur} (15.05.1862 – 21.10.1931), \emph{Schriftsteller, Mediziner}!Liebelei. Schauspiel in drei Akten1895-10-09@\strich\emph{Liebelei. Schauspiel in drei Akten} {[}1895-10-09{]}|pw}«. Ein neuer Erfolg wäre ſehr ſchön, aber nöthig iſt er gerade nicht.
               Die »Liebelei\pwindex{Schnitzler, Arthur 15.05.1862 – 21.10.1931@\textsc{Schnitzler, Arthur} (15.05.1862 – 21.10.1931), \emph{Schriftsteller, Mediziner}!Liebelei. Schauspiel in drei Akten1895-10-09@\strich\emph{Liebelei. Schauspiel in drei Akten} {[}1895-10-09{]}|pw}« \uline{mußte} Erfolg haben; denn {\pb}darin \strikeout{lag l\textcolor{gray}{a}g} lag Deine ganze Art, und es
               war die große, ein für alle Mal entſcheidende Frage: \strikeout{ob} ob das Publicum »Ja« oder »Nein« dazu ſagen würde. Was das Berlin\oindex{Berlin@\textbf{Berlin}|pw}er Publicum zu »Freiwild\pwindex{Schnitzler, Arthur 15.05.1862 – 21.10.1931@\textsc{Schnitzler, Arthur} (15.05.1862 – 21.10.1931), \emph{Schriftsteller, Mediziner}!Freiwild. Schauspiel in 3 Akten1896@\strich\emph{Freiwild. Schauspiel in 3 Akten} {[}1896{]}|pw}« ſagt, iſt \strikeout{\textcolor{gray}{wig}} wichtig mit Rückſicht auf die materiellen Conſequenzen – für das \uline{Weſentliche} aber iſt es ganz gleichgiltig. Daß ich Dir
               trotzdem für ein Telegramm am Mittwoch{ }Vormittag von Herzen dankbar ſein werde, verſteht ſich von ſelbſt.\pend
           \pstart
           {\pb}Schade, daß Du das \label{K_L02789-4v}\edtext{»befreiende« Wort}{\lemma{\textnormal{\emph{»befreiende« Wort}}}\Cendnote{\textnormal{siehe Paul Goldmann an Arthur Schnitzler, 27. 10. [1896]}}}\label{K_L02789-4h} nicht findeſt. \strikeout{Laß} Eigentlich iſt es \strikeout{eigentlich} ſchon enthalten in dem Ausſpruch: »Solche Leute haben im Frieden gar
                  keine Exiſtenz-Berechtigung\pwindex{Schnitzler, Arthur 15.05.1862 – 21.10.1931@\textsc{Schnitzler, Arthur} (15.05.1862 – 21.10.1931), \emph{Schriftsteller, Mediziner}!Freiwild. Schauspiel in 3 Akten1896@\strich\emph{Freiwild. Schauspiel in 3 Akten} {[}1896{]}|pwv}«. Laß’ den Schauſpieler\pwindex{Nissen, Hermann 17.05.1853 – 15.02.1914@\textsc{Nissen, Hermann} (17.05.1853 – 15.02.1914), \emph{Schauspieler}|pwv} das nur recht kräftig und deutungsvoll
               ſagen!\pend
           \pstart
           Ich hab’ einen Augenblick mit der Idee geliebäugelt, hier auf drei Tage durchzugehen
               und zur \textsc{Première} zu kommen. Aber, wie gewöhnlich, fehlte
               das Geld; auch bin ich doch nicht mehr jung genug für ſolche Huſarenſtücklein. Ich
               muß alſo wieder aus der Ferne zuſchauen. Statt meiner kommen meine Wünſche; ſie
               ſollen Dir alle{[}s{]} Liebe, Gute, Frohe für Dienſtag{ }Abend bringen. Ich umarme Dich von Herzen.\pend
           \pstart
           Dein treuer {\\[\baselineskip]}\spacefill\mbox{Paul Goldm}\pend
           \leftskip=0em{}\pstart
           \noindent{}\label{T_L02789-1v}\edtext{Du ſchreibſt mir wohl noch ein Wort
                  aus Berlin\oindex{Berlin@\textbf{Berlin}|pw}?}{\lemma{\textnormal{\emph{Du … Berlin?}}}\Cendnote{\textnormal{entlang des Seitenrands der letzten Seite, quer zum
                     Text}}}\label{T_L02789-1h}\pend
           
         
         \endnumbering\mylabel{h}\end{ledgroupsized}  \newcommand{\dateiname}{L02789}\newcommand{\titel}{Paul Goldmann an Arthur Schnitzler, 1. 11. [1896]}\newcommand{\editorInnen}{Martin Anton Müller und Laura Untner}%% latex-leseansicht-abspann.tex
%% Abspann für die Leseansicht.
%% Der Schalter \ifkorrekturansicht ist bereits durch den Vorspann gesetzt.

%% latex-abspann.tex
%% Gemeinsamer Abspann für Korrekturansicht und Leseansicht.
%% Setzt den Schalter \ifkorrekturansicht voraus (gesetzt in den
%% einbindenden Dateien latex-korrekturansicht-abspann.tex bzw.
%% latex-leseansicht-abspann.tex).
%% ---------------------------------------------------------------

\normalsize

% Das esempio-Environment wird nur in der Leseansicht benötigt
\ifkorrekturansicht\else
\newenvironment{esempio}[3]%
{
    \vspace{1.5ex}
    \rlap{\underline{#1}}
    \par
    \setlength{\parindent}{0cm}
    \nopagebreak
    \leftskip=#2cm
    \rightskip=#3cm
}
{
    \par
}
\fi

\doendnotes{C}
\bigskip
\vfill

\clearpage

\footnotesize

\ifkorrekturansicht
  \lohead{\textsc{register}}
\fi

% theindex-Environment neu definieren ohne reledmac
\makeatletter
\renewenvironment{theindex}{%
  \ifkorrekturansicht
    \section*{\indexname}%
  \else
    \subsubsection*{Index der erwähnten Entitäten}%
  \fi
  \setlength{\parindent}{0pt}%
  \setlength{\parskip}{0pt plus 0.3pt}%
  \let\item\@idxitem
}{%
  \ifkorrekturansicht\clearpage\fi
}
\makeatother

\IfFileExists{\jobname-pw.ind}{\input{\jobname-pw.ind}}{}

% Quellenangabe nur in der Leseansicht
\ifkorrekturansicht\else
% Fallback-Definitionen, falls die .tex-Datei \titel etc. nicht gesetzt hat
\providecommand{\titel}{}
\providecommand{\editorInnen}{}
\providecommand{\dateiname}{\jobname}

\vspace{3cm}

\vfill

\footnotesize
\textsc{Quelle}: \titel. Herausgegeben von {\editorInnen}. In: \emph{Arthur Schnitzler: Briefwechsel mit Autorinnen und Autoren}.
 Digitale Edition, https://schnitzler-briefe.acdh.oeaw.ac.at/{\dateiname}.html (Stand \today)
\fi

\end{document}


      