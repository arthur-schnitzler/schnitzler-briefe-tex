%% latex-leseansicht-vorspann.tex
%% Vorspann für die Leseansicht.
%% Lädt die gemeinsame Datei latex-vorspann.tex mit nicht gesetztem Schalter.

\newif\ifkorrekturansicht
\korrekturansichtfalse

\input{../tex-inputs/latex-vorspann}


               \section[Hugo von Hofmannsthal an Arthur Schnitzler, 21. 8. {[}1904{]}]{ Hugo von Hofmannsthal an Arthur Schnitzler, 21. 8. {[}1904{]}}\nopagebreak\mylabel{v}\rehead{ }\begin{ledgroupsized}[t]{13cm}\normalsize\beginnumbering\briefempfaengerindex{Schnitzler, Arthur@\textsc{Schnitzler, Arthur}!zzzHofmannsthal, Hugo von@\emph{von Hugo von Hofmannsthal}!1904-08-211@{21. 8. 1904}|(be} \toendnotes[C]{\smallbreak\pagebreak[2]} \Standort{CUL, Schnitzler, B 43.}
\physDesc{Brief, 1 Blatt, 4 Seiten
\newline{}Handschrift: schwarze Tinte, deutsche Kurrent
\newline{}Schnitzler: mit Bleistift die Jahreszahl ergänzt: »904« \newline{}Ordnung: mit Bleistift von unbekannter Hand nummeriert:
                              »234« }\buchAbdrucke{\weitereDrucke{Hugo von Hofmannsthal, Arthur Schnitzler: \emph{Briefwechsel}. Hg. Therese Nickl und Heinrich Schnitzler. Frankfurt am Main: \emph{S. Fischer} 1964, S. 199.} }\toendnotes[C]{\smallbreak}\pstart
           \raggedleft{}{\pb}Ramgut\oindex{Ramgut@\textbf{Ramgut}|pw}{ }21 VIII.\pend
           \pstart{}lieber, \pend\pstart
           das ſcheint ſich ja ſehr ſchön zu treffen. Gerty\pwindex{Hofmannsthal, Gertrude von 16.03.1880 – 09.11.1959@\textsc{Hofmannsthal, Gertrude von} (16.03.1880 – 09.11.1959)|pw}
               iſt auf jeden Fall ſehr froh mit Ihnen zu fahren und würde dafür eventuell bis zum
                     5\textsuperscript{ten} warten. Viel lieber wäre es ihr freilich, den 2\textsuperscript{ten} oder 3\textsuperscript{ten} zu fahren, was auch wohl möglich ſein wird, da mir Idchen Grünwald\pwindex{Gruenwald, Ida 28.06.1873 – Mai 1908@\textsc{Grünwald, Ida} (28.06.1873 – Mai 1908), \emph{Stenotypistin}|pw}{ }{\pb}heute aus \textsc{Haarlem}\oindex{Haarlem@\textbf{Haarlem}|pw} anzeigt daſs ſie pünktlich den 26\textsuperscript{ten} zurück ſein wird.\hspace*{1.5em}So werden wir dann
               hoffentlich eine ſchöne Woche zuſammen haben. Nur dürfte ich mich kaum in Iſchl\oindex{Bad Ischl@\textbf{Bad Ischl}|pw}{ }ſelber niederlaſſen, wo ich mit Sicherheit \textsc{Migraine} bekomme, ſondern nahe davon, etwa am Wolfgangſee\oindex{Wolfgangsee@\textbf{Wolfgangsee}|pw}. Wie ſchön {\pb}aber wenn wir doch ein paar Tage
               im gleichen Hôtel wären. Nur Iſchl\oindex{Bad Ischl@\textbf{Bad Ischl}|pw} iſt mir abſolut
               unerträglich, wegen des Klimas und wegen der Geſichter der Leute die ich immer
               weniger vertrage.\pend
           \pstart
           Mein Aufenthalt iſt nicht durch die Rückkehr nach Rodaun\oindex{Rodaun@\textbf{Rodaun}|pw} begrenzt, ſondern durch den Wunſch, ungefähr {\pb}15\textsuperscript{ten} oder 16\textsuperscript{ten} September für einen
               ruhigen mehrwöchentlichen Aufenthalt in Venedig\oindex{Venedig@\textbf{Venedig}|pw}
               einzutreffen. Denn das iſt die Stadt meiner arbeitſamſten Arbeit, meiner
               concentrierteſten Concentration und meiner einfältigſten Einfälle, und ſo hoffe ich
               denn dort wieder ein nicht ganz ſterbliches Drama\pwindex{Hofmannsthal, Hugo von 01.02.1874 – 15.07.1929@\textsc{Hofmannsthal, Hugo von} (01.02.1874 – 15.07.1929), \emph{Schriftsteller}!Oedipus und die Sphinx. Tragoedie in drei Aufzuegen1906@\strich\emph{Oedipus und die Sphinx. Tragödie in drei Aufzügen} {[}1906{]}|pwv} aufs erbleichende Papier zu ſchleudern. Wir nehmen den Weg
               dorthin etwa über \textsc{Trient}\oindex{Trient@\textbf{Trient}|pw} und durchs \textsc{val sugana}\oindex{Val Sugana@\textbf{Val Sugana}|pw}, und ſo iſt man etwa bis Bozen\oindex{Bozen@\textbf{Bozen}|pw} zuſa{\geminationm}en. Ei, niedlich!\pend
           \pstart Ihr\spacefill\mbox{Hugo}\pend{}          \endnumbering\briefempfaengerindex{Schnitzler, Arthur@\textsc{Schnitzler, Arthur}!zzzHofmannsthal, Hugo von@\emph{von Hugo von Hofmannsthal}!1904-08-211@{21. 8. 1904}|)be}\mylabel{h}\end{ledgroupsized}  \newcommand{\dateiname}{L01431}\newcommand{\titel}{Hugo von Hofmannsthal an Arthur Schnitzler, 21. 8. [1904]}\newcommand{\editorInnen}{Martin Anton Müller und Gerd-Hermann Susen}%% latex-leseansicht-abspann.tex
%% Abspann für die Leseansicht.
%% Der Schalter \ifkorrekturansicht ist bereits durch den Vorspann gesetzt.

%% latex-abspann.tex
%% Gemeinsamer Abspann für Korrekturansicht und Leseansicht.
%% Setzt den Schalter \ifkorrekturansicht voraus (gesetzt in den
%% einbindenden Dateien latex-korrekturansicht-abspann.tex bzw.
%% latex-leseansicht-abspann.tex).
%% ---------------------------------------------------------------

\normalsize

% Das esempio-Environment wird nur in der Leseansicht benötigt
\ifkorrekturansicht\else
\newenvironment{esempio}[3]%
{
    \vspace{1.5ex}
    \rlap{\underline{#1}}
    \par
    \setlength{\parindent}{0cm}
    \nopagebreak
    \leftskip=#2cm
    \rightskip=#3cm
}
{
    \par
}
\fi

\doendnotes{C}
\bigskip
\vfill

\clearpage

\footnotesize

\ifkorrekturansicht
  \lohead{\textsc{register}}
\fi

% theindex-Environment neu definieren ohne reledmac
\makeatletter
\renewenvironment{theindex}{%
  \ifkorrekturansicht
    \section*{\indexname}%
  \else
    \subsubsection*{Index der erwähnten Entitäten}%
  \fi
  \setlength{\parindent}{0pt}%
  \setlength{\parskip}{0pt plus 0.3pt}%
  \let\item\@idxitem
}{%
  \ifkorrekturansicht\clearpage\fi
}
\makeatother

\IfFileExists{\jobname-pw.ind}{\input{\jobname-pw.ind}}{}

% Quellenangabe nur in der Leseansicht
\ifkorrekturansicht\else
% Fallback-Definitionen, falls die .tex-Datei \titel etc. nicht gesetzt hat
\providecommand{\titel}{}
\providecommand{\editorInnen}{}
\providecommand{\dateiname}{\jobname}

\vspace{3cm}

\vfill

\footnotesize
\textsc{Quelle}: \titel. Herausgegeben von {\editorInnen}. In: \emph{Arthur Schnitzler: Briefwechsel mit Autorinnen und Autoren}.
 Digitale Edition, https://schnitzler-briefe.acdh.oeaw.ac.at/{\dateiname}.html (Stand \today)
\fi

\end{document}


      