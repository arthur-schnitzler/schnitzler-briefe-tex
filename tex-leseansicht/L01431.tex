%% latex-korrekturansicht-vorspann.tex
%% Vorspann für die Korrekturansicht.
%% Lädt die gemeinsame Datei latex-vorspann.tex mit gesetztem Schalter.

\newif\ifkorrekturansicht
\korrekturansichttrue

\input{../tex-inputs/latex-vorspann}


\section[Hugo von Hofmannsthal an Arthur Schnitzler, 21. 8. {[}1904{]}]{L01431 Hugo von Hofmannsthal an Arthur Schnitzler, 21. 8. {[}1904{]}}
\nopagebreak\mylabel{L01431v}
\rehead{ }\normalsize\beginnumbering\briefempfaengerindex{Schnitzler, Arthur@\textsc{Schnitzler, Arthur}!zzzHofmannsthal, Hugo von@\emph{von Hugo von Hofmannsthal}!1904-08-211@{21. 8. 1904}|(be}
\toendnotes[C]{\smallbreak\pagebreak[2]}\Standort{CUL, Schnitzler, B 43.}
\physDesc{Brief, 1 Blatt, 4 Seiten, 1275 Zeichen
\newline{}Handschrift: schwarze Tinte, deutsche Kurrent
\newline{}Schnitzler: mit Bleistift die Jahreszahl ergänzt: »904« 
\newline{}Ordnung: mit Bleistift von unbekannter Hand nummeriert:
                                    »234« }
\buchAbdrucke{\weitereDrucke{Hugo von Hofmannsthal, Arthur Schnitzler: \emph{Briefwechsel}. Frankfurt am Main: \emph{S. Fischer} 1964, S. 199.} }\toendnotes[C]{\smallbreak}
\pstart
           \raggedleft{}{\pb}Ramgut\oindex{Ramgut@\textbf{Ramgut}, \emph{Schloss (K.SLS)}|pw}{ }21 VIII.\pend
           
\pstart{}lieber, \pend\vspace{0.5em}
\pstart
           das ſcheint ſich ja ſehr ſchön zu treffen. Gerty\pwindex{Hofmannsthal, Gertrude von 16.03.1880 – 09.11.1959@\textsc{Hofmannsthal, Gertrude von} (16.03.1880 – 09.11.1959)|pw} iſt auf jeden Fall ſehr froh mit Ihnen zu fahren und würde dafür
               eventuell bis zum 5\textsuperscript{ten} warten. Viel lieber wäre es ihr freilich, den 2\textsuperscript{ten} oder 3\textsuperscript{ten} zu fahren, was auch wohl möglich ſein wird, da mir Idchen Grünwald\pwindex{Gruenwald, Ida 28.06.1873 – Mai 1908@\textsc{Grünwald, Ida} (28.06.1873 – Mai 1908), \emph{Stenotypistin/Stenotypistin}|pw}{ }{\pb}heute aus \textsc{Haarlem}\oindex{Haarlem@\textbf{Haarlem}, \emph{P.PPLA}|pw} anzeigt daſs ſie pünktlich den 26\textsuperscript{ten} zurück ſein wird.\hspace*{1.5em}So werden wir dann
               hoffentlich eine ſchöne Woche zuſammen haben. Nur dürfte ich mich kaum in Iſchl\oindex{Bad Ischl@\textbf{Bad Ischl}, \emph{P.PPL}|pw}{ }ſelber niederlaſſen, wo ich mit Sicherheit \textsc{Migraine} bekomme, ſondern nahe davon, etwa am Wolfgangſee\oindex{Wolfgangsee@\textbf{Wolfgangsee}, \emph{See (N.SEE)}|pw}. Wie ſchön {\pb}aber wenn wir doch ein paar Tage
               im gleichen Hôtel wären. Nur Iſchl\oindex{Bad Ischl@\textbf{Bad Ischl}, \emph{P.PPL}|pw} iſt mir
               abſolut unerträglich, wegen des Klimas und wegen der Geſichter der Leute die ich
               immer weniger vertrage.\pend
           
\pstart
           Mein Aufenthalt iſt nicht durch die Rückkehr nach Rodaun\oindex{Rodaun@\textbf{Rodaun}, \emph{A.ADM4}|pw} begrenzt, ſondern durch den Wunſch, ungefähr {\pb}15\textsuperscript{ten} oder 16\textsuperscript{ten} September für einen
               ruhigen mehrwöchentlichen Aufenthalt in Venedig\oindex{Venedig@\textbf{Venedig}, \emph{P.PPLA}|pw}
               einzutreffen. Denn das iſt die Stadt meiner arbeitſamſten Arbeit, meiner
               concentrierteſten Concentration und meiner einfältigſten Einfälle, und ſo hoffe ich
               denn dort wieder ein nicht ganz ſterbliches Drama\pwindex{Oedipus und die Sphinx. Tragoedie in drei Aufzuegen@\emph{Oedipus und die Sphinx. Tragödie in drei Aufzügen}|pwv} aufs erbleichende Papier zu ſchleudern. Wir nehmen den
               Weg dorthin etwa über \textsc{Trient}\oindex{Trient@\textbf{Trient}, \emph{P.PPLA}|pw} und durchs \textsc{val sugana}\oindex{Val Sugana@\textbf{Val Sugana}, \emph{T.VAL}|pw}, und ſo iſt man etwa bis Bozen\oindex{Bozen@\textbf{Bozen}, \emph{P.PPLA2}|pw} zuſa{\geminationm}en. Ei, niedlich!\pend
           \pstart Ihr\spacefill\mbox{Hugo}\pend{}\selectlanguage{ngerman}\endnumbering\briefempfaengerindex{Schnitzler, Arthur@\textsc{Schnitzler, Arthur}!zzzHofmannsthal, Hugo von@\emph{von Hugo von Hofmannsthal}!1904-08-211@{21. 8. 1904}|)be}\mylabel{L01431h}  \normalsize

\doendnotes{C}
\bigskip
\vfill

\clearpage

\footnotesize

\lohead{\textsc{register}}

% Definiere theindex-Environment komplett neu ohne reledmac
\makeatletter
\renewenvironment{theindex}{%
  \section*{\indexname}%
  \setlength{\parindent}{0pt}%
  \setlength{\parskip}{0pt plus 0.3pt}%
  \let\item\@idxitem
}{%
  \clearpage
}
\makeatother

\IfFileExists{\jobname-pw.ind}{\input{\jobname-pw.ind}}{}

\end{document}

      