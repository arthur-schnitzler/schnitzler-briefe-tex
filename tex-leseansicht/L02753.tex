%% latex-leseansicht-vorspann.tex
%% Vorspann für die Leseansicht.
%% Lädt die gemeinsame Datei latex-vorspann.tex mit nicht gesetztem Schalter.

\newif\ifkorrekturansicht
\korrekturansichtfalse

\input{../tex-inputs/latex-vorspann}


         
         \renewcommand{\erwaehntePersonen}{Personen: Hermann Bahr, Richard Beer-Hofmann, Jakob Julius David, Emil Granichstaedten, Theodor Herzl, Hugo von Hofmannsthal, Max Nordau, Leopold Sonnemann, Ludwig Speidel}
         \renewcommand{\erwaehnteInstitutionen}{Institutionen: Frankfurter Zeitung, Neue Freie Presse}
         \renewcommand{\erwaehnteOrte}{Orte: München, Paris, Wien, rue Feydeau}
         \renewcommand{\erwaehnteWerke}{Werke: Burgtheater. (»Liebelei«, Schauspiel in drei Aufzügen von Arthur Schnitzler. – »Rechte der Seele«, Schauspiel in einem Act von Giuseppe Giacosa, deutsch von Otto Eisenschitz.), Das Märchen. Schauspiel in drei Aufzügen, Deutsches Volkstheater. (»Ein Regentag«, Charakterbild von J. J. David.), Die Presse, Ein Regentag. Charakterbild, Freiwild. Schauspiel in 3 Akten, Liebelei. Schauspiel in drei Akten, Neue Freie Presse, Tagebuch, Theater- und Kunstnachrichten. [Deutsches Volkstheater.] [Ein Regentag]}
               \section[Paul Goldmann an Arthur Schnitzler, 15. 10. {[}1895{]}]{ Paul Goldmann an Arthur Schnitzler, 15. 10. {[}1895{]}}\nopagebreak\mylabel{v}\rehead{ }\begin{ledgroupsized}[t]{13cm}\normalsize\beginnumbering \toendnotes[C]{\smallbreak\pagebreak[2]} \Standort{DLA, A:Schnitzler, HS.NZ85.1.3165.}
\physDesc{Brief, 4 Blätter, 16 Seiten
\newline{}Handschrift: blaue Tinte, deutsche Kurrent
\newline{}Schnitzler: 1) mit schwarzer Tinte das Jahr » 95« vermerkt  2) mit rotem Buntstift sechs Unterstreichungen}\toendnotes[C]{\smallbreak}\pstart
           \noindent{}{\pb}\textcolor{gray}{\textbf{\textbf{Frankfurter Zeitung\orgindex{Frankfurter Zeitung@Frankfurter Zeitung|pw}}}}\pend
           \pstart
           \textcolor{gray}{\textbf{(\begin{otherlanguage}{french}Gazette de Francfort\end{otherlanguage}\orgindex{Frankfurter Zeitung@Frankfurter Zeitung|pw})}}.\pend
           \pstart
           \textcolor{gray}{\textbf{\textbf{\begin{otherlanguage}{french}Fondateur M. L.
                                 Sonnemann\pwindex{Sonnemann, Leopold 1831-10-29 – 1909-10-30@\textsc{Sonnemann, Leopold} (1831-10-29 – 1909-10-30), \emph{Journalist, Herausgeber}|pw}\end{otherlanguage}.}}}\hfill \textsc{Paris\oindex{Paris@\textbf{Paris}|pw}}, 1\substVorne{}\textsuperscript{\textcolor{gray}{4}}\substDazwischen{}5\substHinten{}. October.\pend
           \pstart
           \begin{otherlanguage}{french}\textcolor{gray}{\textbf{Journal politique, financier,}}\end{otherlanguage}\pend
           \pstart
           \begin{otherlanguage}{french}\textcolor{gray}{\textbf{commercial et littéraire.}}\end{otherlanguage}\pend
           \pstart
           \begin{otherlanguage}{french}\textcolor{gray}{\textbf{\textbf{Paraissant trois fois par jour.}}}\end{otherlanguage}\pend
           \pstart
           \begin{otherlanguage}{french}\textcolor{gray}{\textbf{\textbf{Bureau à Paris\oindex{Paris@\textbf{Paris}|pw}}}}\end{otherlanguage}\pend
           \pstart
           \begin{otherlanguage}{french}\textcolor{gray}{\textbf{\textbf{24. Rue Feydeau\oindex{rue Feydeau@\textbf{rue Feydeau}|pw}.}}}\end{otherlanguage}\pend
           \pstart\center{}Mein lieber Freund,\pend\pstart
           \textsc{Speidel\pwindex{Speidel, Ludwig 1830-04-11 – 1906-02-03@\textsc{Speidel, Ludwig} (1830-04-11 – 1906-02-03), \emph{Journalist, Kritiker}|pw}s}{ }\label{K_L02753-1v}\edtext{Feuilleton\pwindex{Burgtheater. (»Liebelei«, Schauspiel in drei Aufzuegen von Arthur Schnitzler. – »Rechte der Seele«, Schauspiel in einem Act von Giuseppe Giacosa, deutsch von Otto Eisenschitz.)1895-10-13@\emph{Burgtheater. (»Liebelei«, Schauspiel in drei Aufzügen von Arthur Schnitzler. – »Rechte der Seele«, Schauspiel in einem Act von Giuseppe Giacosa, deutsch von Otto Eisenschitz.)} {[}1895-10-13{]}|pwv}}{\lemma{\textnormal{\emph{Feuilleton}}}\Cendnote{\textnormal{L. Sp.\pwindex{Speidel, Ludwig 1830-04-11 – 1906-02-03@\textsc{Speidel, Ludwig} (1830-04-11 – 1906-02-03), \emph{Journalist, Kritiker}|pwkv} [=Ludwig Speidel\pwindex{Speidel, Ludwig 1830-04-11 – 1906-02-03@\textsc{Speidel, Ludwig} (1830-04-11 – 1906-02-03), \emph{Journalist, Kritiker}|pwk}]: \emph{Burgtheater. (»Liebelei«, Schauspiel in drei Aufzügen von
                        Arthur Schnitzler. – »Rechte der Seele«, Schauspiel in einem Act von
                        Giuseppe Giacosa, deutsch von Otto Eisenschitz.)}\pwindex{Burgtheater. (»Liebelei«, Schauspiel in drei Aufzuegen von Arthur Schnitzler. – »Rechte der Seele«, Schauspiel in einem Act von Giuseppe Giacosa, deutsch von Otto Eisenschitz.)1895-10-13@\emph{Burgtheater. (»Liebelei«, Schauspiel in drei Aufzügen von Arthur Schnitzler. – »Rechte der Seele«, Schauspiel in einem Act von Giuseppe Giacosa, deutsch von Otto Eisenschitz.)} {[}1895-10-13{]}|pwk}. In: \emph{Neue Freie Presse}\pwindex{Neue Freie Presse1864 – 1939@\emph{Neue Freie Presse} {[}1864 – 1939{]}|pwk}, Nr. 11.184, 13. 10. 1895, Morgenblatt, S. 1–3.}}}\label{K_L02753-1h} habe ich geſtern geleſen, und es hat mich entzückt. Es iſt ſchön
               und einfach geſchrieben, und vor Allem freut es mich, daß er Deinem \uline{Character} ſo gerecht wird, daß er ſo wohl verſteht,
               wie der Werth Deiner Production \strikeout{\textcolor{gray}{ne}b} neben allem Talent auch im Moraliſchen liegt, i\substVorne{}\textsuperscript{m}\substDazwischen{}n\substHinten{} dem Muthe, in dem starken Streben, ganz einfach das Wahre zu ſagen, {\pb}unbekümmert um \strikeout{die}
               das Treiben und Reden der Anderen. Er iſt doch ein großer Kritiker\pwindex{Speidel, Ludwig 1830-04-11 – 1906-02-03@\textsc{Speidel, Ludwig} (1830-04-11 – 1906-02-03), \emph{Journalist, Kritiker}|pwv}\textcolor{gray}{,} und z. B. \textsc{Herzl\pwindex{Herzl, Theodor 1860-05-02 – 1904-07-03@\textsc{Herzl, Theodor} (1860-05-02 – 1904-07-03), \emph{Schriftsteller, Journalist}|pw}} in ſeiner geſuchten und manierirten Art hätte das nie gefunden. Ob er Dich
               überſchätzt? Gewiß, er hätte Einiges tadeln können. Ich verſtehe vollſtändig, was Du
               meinſt. Ich begreife, daß es Dich in Verlegenheit ſetzt, ſo rückhaltslos gelobt zu
               werden. Vor Enttäuſchungen fürchte \uline{ich} mich zwar
               nicht. Aber ich kann es nachfühlen, daß Du, als ehrlich ſtrebender Menſch, Dich
               fortwährend unfertig {\pb}fühlſt und daß es Dir daher
               peinlich iſt, wenn man Dich als einen \strikeout{\textcolor{gray}{S}} Vollendeten hinſtellt. Ein \textsc{Herzl\pwindex{Herzl, Theodor 1860-05-02 – 1904-07-03@\textsc{Herzl, Theodor} (1860-05-02 – 1904-07-03), \emph{Schriftsteller, Journalist}|pw}}, \textsc{David\pwindex{David, Jakob Julius 1859-02-06 – 1906-11-20@\textsc{David, Jakob Julius} (1859-02-06 – 1906-11-20), \emph{Schriftsteller, Journalist}|pw}} oder \textsc{Nordau\pwindex{Nordau, Max 29.07.1849 – 22.01.1923@\textsc{Nordau, Max} (29.07.1849 – 22.01.1923), \emph{Schriftsteller, Mediziner}|pw}} hätte \textsc{Speidel\pwindex{Speidel, Ludwig 1830-04-11 – 1906-02-03@\textsc{Speidel, Ludwig} (1830-04-11 – 1906-02-03), \emph{Journalist, Kritiker}|pw}s}{ }Feuilleton\pwindex{Burgtheater. (»Liebelei«, Schauspiel in drei Aufzuegen von Arthur Schnitzler. – »Rechte der Seele«, Schauspiel in einem Act von Giuseppe Giacosa, deutsch von Otto Eisenschitz.)1895-10-13@\emph{Burgtheater. (»Liebelei«, Schauspiel in drei Aufzügen von Arthur Schnitzler. – »Rechte der Seele«, Schauspiel in einem Act von Giuseppe Giacosa, deutsch von Otto Eisenschitz.)} {[}1895-10-13{]}|pwv} einfach als den ihm
               gebührenden Tribut hingenommen. Du, in Deiner Beſcheidenheit und Grundehrlichkeit,
               mußteſt davon in Verlegenheit gebracht werden. Das ſtimmt Alles. Wenn aber Du ſagen
               mußt, \textsc{Speidel\pwindex{Speidel, Ludwig 1830-04-11 – 1906-02-03@\textsc{Speidel, Ludwig} (1830-04-11 – 1906-02-03), \emph{Journalist, Kritiker}|pw}} habe \strikeout{ich} Dich überſchätzft, ſo darf \uline{ich} ſagen: Nein, er überſchätzt Dich \uline{nicht}. \strikeout{\textcolor{gray}{Ve}}{ }\strikeout{Verg\textcolor{gray}{e}} Er ſagt von Dir gerade das,
               was Dir gebührt. Vergiß’ auch {\pb}nicht, mein lieber
               Freund, daß \textsc{Speidel\pwindex{Speidel, Ludwig 1830-04-11 – 1906-02-03@\textsc{Speidel, Ludwig} (1830-04-11 – 1906-02-03), \emph{Journalist, Kritiker}|pw}} Dich in Deiner ganzen Art neu entdeckt – daß Deine ganze Perſönlichkeit ihm
               eine neue Erſcheinung iſt, \strikeout{\textcolor{gray}{×}} während wir dieſelbe längſt kennen – und daß er ſich mit dieſer bedeutenden
               Perſönlichkeit (entſchuldige die ſtarken Ausdrücke, aber ſie laſſen ſich nicht
               vermeiden) \strikeout{a\textcolor{gray}{b}} im Ganzen abzufinden hat, nicht blos bei deren letztem Ausfluß, der »Liebelei\pwindex{Schnitzler, Arthur 15.05.1862 – 21.10.1931@\textsc{Schnitzler, Arthur} (15.05.1862 – 21.10.1931), \emph{Schriftsteller, Mediziner}!Liebelei. Schauspiel in drei Akten1895-10-09@\strich\emph{Liebelei. Schauspiel in drei Akten} {[}1895-10-09{]}|pw}«, deren kleine Mängel {\pb}er darum nicht ſieht, weil er das Geſammtbild in
               ſeinen großen Linien vor Augen hat. Das Feuilleton\pwindex{Burgtheater. (»Liebelei«, Schauspiel in drei Aufzuegen von Arthur Schnitzler. – »Rechte der Seele«, Schauspiel in einem Act von Giuseppe Giacosa, deutsch von Otto Eisenschitz.)1895-10-13@\emph{Burgtheater. (»Liebelei«, Schauspiel in drei Aufzügen von Arthur Schnitzler. – »Rechte der Seele«, Schauspiel in einem Act von Giuseppe Giacosa, deutsch von Otto Eisenschitz.)} {[}1895-10-13{]}|pwv} gilt auch mehr dem allgemeinen \textsc{Arthur Schnitzler}, als dem beſonderen Drama\pwindex{Schnitzler, Arthur 15.05.1862 – 21.10.1931@\textsc{Schnitzler, Arthur} (15.05.1862 – 21.10.1931), \emph{Schriftsteller, Mediziner}!Liebelei. Schauspiel in drei Akten1895-10-09@\strich\emph{Liebelei. Schauspiel in drei Akten} {[}1895-10-09{]}|pwv}.\pend
           \pstart
           Daß der materielle Erfolg ſich nun auch einſtellt, habe ich gleichfalls
               vorausgeſehen. Ganz Wien\oindex{Wien@\textbf{Wien}|pw}{ }\strikeout{iſt} wird hineinlaufen, um dieſes \strikeout{ech} echt Wien\oindex{Wien@\textbf{Wien}|pw}er Stück\pwindex{Schnitzler, Arthur 15.05.1862 – 21.10.1931@\textsc{Schnitzler, Arthur} (15.05.1862 – 21.10.1931), \emph{Schriftsteller, Mediziner}!Liebelei. Schauspiel in drei Akten1895-10-09@\strich\emph{Liebelei. Schauspiel in drei Akten} {[}1895-10-09{]}|pwv} zu \strikeout{ſehen} ſehen. {\pb}Ich bin
               wahrhaft glücklich, daß es ſo gut geht. Du ahnſt gar nicht, welch’ große \uline{materielle} Wirkung \textsc{Speidel\pwindex{Speidel, Ludwig 1830-04-11 – 1906-02-03@\textsc{Speidel, Ludwig} (1830-04-11 – 1906-02-03), \emph{Journalist, Kritiker}|pw}s}{ }Feuilleton\pwindex{Burgtheater. (»Liebelei«, Schauspiel in drei Aufzuegen von Arthur Schnitzler. – »Rechte der Seele«, Schauspiel in einem Act von Giuseppe Giacosa, deutsch von Otto Eisenschitz.)1895-10-13@\emph{Burgtheater. (»Liebelei«, Schauspiel in drei Aufzügen von Arthur Schnitzler. – »Rechte der Seele«, Schauspiel in einem Act von Giuseppe Giacosa, deutsch von Otto Eisenschitz.)} {[}1895-10-13{]}|pwv} für Dich haben wird\substVorne{}\textsuperscript{;}\substDazwischen{}.\substHinten{} In jeder Beziehung biſt Du nun lancirt, – biſt aus der Menge der im Dunkeln
               Strebenden herausgehoben und ſtehſt auf der Höhe mit den Wenigen.\pend
           \pstart
           Um Dich dort zu erhalten, wirſt Du weiter thätig ſein, wie bisher. Und zwar muß ſich
               – das wird {\pb}ſich auch naturgemäß als
               Entwickelungs-Reſultat ergeben – Deine Kunſt erweitern und vertiefen. Sie muß, ſtatt
               wie bisher nur eine Seite des Lebens, allmälig das \uline{ganze}{ }\uline{Leben} umfaſſen\textcolor{gray}{.} Concret \strikeout{\textcolor{gray}{l}eſ\textcolor{gray}{×}} geſprochen: Du darfſt \label{K_L02753-55v}\edtext{höchſtens noch ein Süßes-\substVorne{}\textsuperscript{Mädel-}{\allowbreak}\substDazwischen{}Mädel-\substHinten{}Stück}{\lemma{\textnormal{\emph{höchſtens … Süßes-Mädel-Mädel-Stück}}}\Cendnote{\textnormal{vgl. Paul Goldmann an Arthur Schnitzler, 31. 12. [1894]}}}\label{K_L02753-55h} ſchreiben. Dann mußt Du hinaus ins große Ganze – immer weiter von Deines
               Herzens beſonderen Erlebniſſen weg – mußt aus dem Vollen {\pb}nehmen und geſtalten. In »Märchen\pwindex{Schnitzler, Arthur 15.05.1862 – 21.10.1931@\textsc{Schnitzler, Arthur} (15.05.1862 – 21.10.1931), \emph{Schriftsteller, Mediziner}!Maerchen. Schauspiel in drei Aufzuegen1893-12-01@\strich\emph{Das Märchen. Schauspiel in drei Aufzügen} {[}1893-12-01{]}|pw}« und »Liebelei\pwindex{Schnitzler, Arthur 15.05.1862 – 21.10.1931@\textsc{Schnitzler, Arthur} (15.05.1862 – 21.10.1931), \emph{Schriftsteller, Mediziner}!Liebelei. Schauspiel in drei Akten1895-10-09@\strich\emph{Liebelei. Schauspiel in drei Akten} {[}1895-10-09{]}|pw}«
               haſt Du Deine eigene Jugend poetiſch ausgeſtaltet; vielleicht wirſt Du das auch in
                  »Freiwild\pwindex{Schnitzler, Arthur 15.05.1862 – 21.10.1931@\textsc{Schnitzler, Arthur} (15.05.1862 – 21.10.1931), \emph{Schriftsteller, Mediziner}!Freiwild. Schauspiel in 3 Akten1896@\strich\emph{Freiwild. Schauspiel in 3 Akten} {[}1896{]}|pw}« thun; das macht nichts. Dann aber
               mußt Du zeigen, daß Du nicht nur Dein Leben, ſondern auch das Leben \strikeout{And} der Anderen zu geſtalten weißt\substVorne{}\textsuperscript{.}\substDazwischen{},\substHinten{} – das eigentliche, das große Leben. Wenn Du das kannſt, wirſt Du ein großer
               Dichter ſein\substVorne{}\textsuperscript{;}\substDazwischen{}.\substHinten{} Und ich bin überzeugt – \strikeout{auch} nach all’ dem
               Schönen, was dieſe {\pb}Tage gebracht haben, werden wir
               auch das noch \strikeout{er} erleben. Alle Zeichen deuten darauf
               hin.\pend
           \pstart
           Was Deine Umänderungs-Pläne betrifft, ſo halte ich Dein Gefühl für durchaus richtig.
               Gewiß, der alte \textsc{Weiring\pwindex{Schnitzler, Arthur 15.05.1862 – 21.10.1931@\textsc{Schnitzler, Arthur} (15.05.1862 – 21.10.1931), \emph{Schriftsteller, Mediziner}!Liebelei. Schauspiel in drei Akten1895-10-09@\strich\emph{Liebelei. Schauspiel in drei Akten} {[}1895-10-09{]}|pwv}} müßte mehr hervortreten, müßte dramatiſcher werden. Die Art, wie Du ſeine
               dramatiſche \strikeout{B\textcolor{gray}{e}} Belebung Dir
               denkſt, finde ich {\pb}durchaus \strikeout{\textcolor{gray}{bi}ll} billigenswerth. Wenn Du Luſt und Stimmung dazu haſt,
               verſuchs immerhin. Der \label{K_L02753-2v}\edtext{zweite Akt\pwindex{Schnitzler, Arthur 15.05.1862 – 21.10.1931@\textsc{Schnitzler, Arthur} (15.05.1862 – 21.10.1931), \emph{Schriftsteller, Mediziner}!Liebelei. Schauspiel in drei Akten1895-10-09@\strich\emph{Liebelei. Schauspiel in drei Akten} {[}1895-10-09{]}|pwv}}{\lemma{\textnormal{\emph{zweite Akt}}}\Cendnote{\textnormal{Am 11. 10. 1895 notierte Schnitzler\pwindex{Schnitzler, Arthur 15.05.1862 – 21.10.1931@\textsc{Schnitzler, Arthur} (15.05.1862 – 21.10.1931), \emph{Schriftsteller, Mediziner}|pwk} im \emph{Tagebuch}\pwindex{Schnitzler, Arthur 15.05.1862 – 21.10.1931@\textsc{Schnitzler, Arthur} (15.05.1862 – 21.10.1931), \emph{Schriftsteller, Mediziner}!Tagebuch1981 – 2000@\strich\emph{Tagebuch} {[}1981 – 2000{]}|pwk}
                  die »Idee, die Schwester des alten Weiring\pwindex{Schnitzler, Arthur 15.05.1862 – 21.10.1931@\textsc{Schnitzler, Arthur} (15.05.1862 – 21.10.1931), \emph{Schriftsteller, Mediziner}!Liebelei. Schauspiel in drei Akten1895-10-09@\strich\emph{Liebelei. Schauspiel in drei Akten} {[}1895-10-09{]}|pwv} in den 2. Akt\pwindex{Schnitzler, Arthur 15.05.1862 – 21.10.1931@\textsc{Schnitzler, Arthur} (15.05.1862 – 21.10.1931), \emph{Schriftsteller, Mediziner}!Liebelei. Schauspiel in drei Akten1895-10-09@\strich\emph{Liebelei. Schauspiel in drei Akten} {[}1895-10-09{]}|pwv} zu bringen als Lebende«. Herzl\pwindex{Herzl, Theodor 1860-05-02 – 1904-07-03@\textsc{Herzl, Theodor} (1860-05-02 – 1904-07-03), \emph{Schriftsteller, Journalist}|pwk} habe außerdem die Idee gehabt, »Weir.\pwindex{Schnitzler, Arthur 15.05.1862 – 21.10.1931@\textsc{Schnitzler, Arthur} (15.05.1862 – 21.10.1931), \emph{Schriftsteller, Mediziner}!Liebelei. Schauspiel in drei Akten1895-10-09@\strich\emph{Liebelei. Schauspiel in drei Akten} {[}1895-10-09{]}|pwv} soll betonen, er
                     habe kein Recht, Christine\pwindex{Schnitzler, Arthur 15.05.1862 – 21.10.1931@\textsc{Schnitzler, Arthur} (15.05.1862 – 21.10.1931), \emph{Schriftsteller, Mediziner}!Liebelei. Schauspiel in drei Akten1895-10-09@\strich\emph{Liebelei. Schauspiel in drei Akten} {[}1895-10-09{]}|pwv} zu halten, da er sein Leben verträumt etc.«. Ab dem
                     17. 10. 1895
                  arbeitete Schnitzler\pwindex{Schnitzler, Arthur 15.05.1862 – 21.10.1931@\textsc{Schnitzler, Arthur} (15.05.1862 – 21.10.1931), \emph{Schriftsteller, Mediziner}|pwk} den zweiten Akt\pwindex{Schnitzler, Arthur 15.05.1862 – 21.10.1931@\textsc{Schnitzler, Arthur} (15.05.1862 – 21.10.1931), \emph{Schriftsteller, Mediziner}!Liebelei. Schauspiel in drei Akten1895-10-09@\strich\emph{Liebelei. Schauspiel in drei Akten} {[}1895-10-09{]}|pwkv} um, jedoch ohne je eine
                  neue Fassung fertigzustellen.}}}\label{K_L02753-2h} kann durch eine kräftige \strikeout{Sc\textcolor{gray}{e}} Scene dieſer Art nur gewinnen. Anderſeits möchte ich Dir aber zu bedenke\substVorne{}\textsuperscript{m}\substDazwischen{}n\substHinten{} geben, daß es immerhin gewagt iſt, ein fertiges Werk\pwindex{Schnitzler, Arthur 15.05.1862 – 21.10.1931@\textsc{Schnitzler, Arthur} (15.05.1862 – 21.10.1931), \emph{Schriftsteller, Mediziner}!Liebelei. Schauspiel in drei Akten1895-10-09@\strich\emph{Liebelei. Schauspiel in drei Akten} {[}1895-10-09{]}|pwv}, \uline{das auch
                  bereits vor dem Publicum ſeine Probe beſtanden hat}, nachträglich zu ändern.
               Werden die nachträglich {\pb}eingeſchobenen Scenen\pwindex{Schnitzler, Arthur 15.05.1862 – 21.10.1931@\textsc{Schnitzler, Arthur} (15.05.1862 – 21.10.1931), \emph{Schriftsteller, Mediziner}!Liebelei. Schauspiel in drei Akten1895-10-09@\strich\emph{Liebelei. Schauspiel in drei Akten} {[}1895-10-09{]}|pwv} nicht einen anderen Ton
               anſchlagen und ſo den Geſammt-Ton des Stück\pwindex{Schnitzler, Arthur 15.05.1862 – 21.10.1931@\textsc{Schnitzler, Arthur} (15.05.1862 – 21.10.1931), \emph{Schriftsteller, Mediziner}!Liebelei. Schauspiel in drei Akten1895-10-09@\strich\emph{Liebelei. Schauspiel in drei Akten} {[}1895-10-09{]}|pwv}es ſtören? Liegt nicht überhaupt die Gefahr \strikeout{f\textcolor{gray}{o}} vor, daß durch die nachträgliche Einſchiebung die ganze \strikeout{\textsc{Ökono\textcolor{gray}{m}}}{ }\textsc{Ökonomie} des Stück\pwindex{Schnitzler, Arthur 15.05.1862 – 21.10.1931@\textsc{Schnitzler, Arthur} (15.05.1862 – 21.10.1931), \emph{Schriftsteller, Mediziner}!Liebelei. Schauspiel in drei Akten1895-10-09@\strich\emph{Liebelei. Schauspiel in drei Akten} {[}1895-10-09{]}|pwv}es \strikeout{geſ\textcolor{gray}{c}} geſchädigt wird? Das ſind Fragen, die nur Du allein beantworten kannſt. Im
               Allgemeinen bin ich, nach Erwägung aller Gründe und Gegengründe, eher {\pb}für die Änderung als dagegen. Du hältſt ſie für
               nöthig und haſt Luſt und Kraft dazu. Das iſt entſcheidend.\pend
           \pstart
           \textsc{Herzl\pwindex{Herzl, Theodor 1860-05-02 – 1904-07-03@\textsc{Herzl, Theodor} (1860-05-02 – 1904-07-03), \emph{Schriftsteller, Journalist}|pw}s} Vorſchlag gibt mir nur einen
               neuen Beweis von der Urtheilsloſigkeit des Mann\pwindex{Herzl, Theodor 1860-05-02 – 1904-07-03@\textsc{Herzl, Theodor} (1860-05-02 – 1904-07-03), \emph{Schriftsteller, Journalist}|pwv}es\substVorne{}\textsuperscript{.}\substDazwischen{},\substHinten{} und ich verſtehe nicht, wie Du ſeinen Rath als »klug« bezeichnen kannſt. Er
               will die Exiſtenzfrage hineinmiſchen. Aber, Du lieber Gott, das bringt ja ein {\pb}ganz neues und ganz fremdes Element in das Stück\pwindex{Schnitzler, Arthur 15.05.1862 – 21.10.1931@\textsc{Schnitzler, Arthur} (15.05.1862 – 21.10.1931), \emph{Schriftsteller, Mediziner}!Liebelei. Schauspiel in drei Akten1895-10-09@\strich\emph{Liebelei. Schauspiel in drei Akten} {[}1895-10-09{]}|pwv} – das \uline{ſociale} Element, das Du, bewußt oder unbewußt, mit
               Feingefühl vermieden haſt! {\dotsfour}\pend
           \pstart
           \textsc{David\pwindex{David, Jakob Julius 1859-02-06 – 1906-11-20@\textsc{David, Jakob Julius} (1859-02-06 – 1906-11-20), \emph{Schriftsteller, Journalist}|pw}s} »Regentag\pwindex{David, Jakob Julius 1859-02-06 – 1906-11-20@\textsc{David, Jakob Julius} (1859-02-06 – 1906-11-20), \emph{Schriftsteller, Journalist}!Regentag. Charakterbild12. 10. 1895@\strich\emph{Ein Regentag. Charakterbild} {[}12. 10. 1895{]}|pw}« muß ein ſchöner Dreck ſein! Entzückend iſt die \label{K_L02753-3v}\edtext{»Neue Fr. Pr.\pwindex{Neue Freie Presse1864 – 1939@\emph{Neue Freie Presse} {[}1864 – 1939{]}|pwv}\orgindex{Neue Freie Presse@Neue Freie Presse|pw}«}{\lemma{\textnormal{\emph{»Neue Fr. Pr.«}}}\Cendnote{\textnormal{O. V.: \emph{Theater- und Kunstnachrichten.
                        [Deutsches Volkstheater.]}\pwindex{?? Werk@Nicht ermittelte Verfasserinnen und Verfasser!Theater- und Kunstnachrichten. [Deutsches Volkstheater.] [Ein Regentag]1895-10-13@\emph{Theater- und Kunstnachrichten. [Deutsches Volkstheater.] [Ein Regentag]} {[}1895-10-13{]}|pwk}. In: \emph{Neue
                        Freie Presse}\pwindex{Neue Freie Presse1864 – 1939@\emph{Neue Freie Presse} {[}1864 – 1939{]}|pwk}, Nr. 1184, 13. 10. 1895,
                     Morgenblatt, S. 7.}}}\label{K_L02753-3h}, die dieſen Anlaß braucht, {\pb}um darzuthun, was für ein bedeutender Mann \textsc{David\pwindex{David, Jakob Julius 1859-02-06 – 1906-11-20@\textsc{David, Jakob Julius} (1859-02-06 – 1906-11-20), \emph{Schriftsteller, Journalist}|pw}} iſt.\pend
           \pstart
           Über \textsc{Bahr\pwindex{Bahr, Hermann 19.07.1863 – 15.01.1934@\textsc{Bahr, Hermann} (19.07.1863 – 15.01.1934), \emph{Schriftsteller, Kritiker}|pw}} ſchrieb ich Dir bereits. Nochmals: ich erwarte von \textsc{Richard}\pwindex{Beer-Hofmann, Richard 1866-07-11 – 1945-09-26@\textsc{Beer-Hofmann, Richard} (1866-07-11 – 1945-09-26), \emph{Schriftsteller}|pw} oder \textsc{Loris\pwindex{Hofmannsthal, Hugo von 1874-02-01 – 1929-07-15@\textsc{Hofmannsthal, Hugo von} (1874-02-01 – 1929-07-15), \emph{Schriftsteller}|pw}} auf das Beſtimmteſte, daß ſie dem Burſchen\pwindex{Bahr, Hermann 19.07.1863 – 15.01.1934@\textsc{Bahr, Hermann} (19.07.1863 – 15.01.1934), \emph{Schriftsteller, Kritiker}|pwv} jene Zurechtweiſung zutheil werden laſſen, die
               infolge ſeiner perſönlichen Gemeinheiten unumgänglich nöthig geworden iſt, die Du ihm
               nicht ertheilen darfſt, und {\pb}die ich ihm leider,
                  \strikeout{nicht} fern von Wien\oindex{Wien@\textbf{Wien}|pw}, nicht ertheilen kann. Übrigens behalte ich mir doch noch ein
               Einſchreiten vor, falls die Wien\oindex{Wien@\textbf{Wien}|pw}er Freunde\pwindex{Hofmannsthal, Hugo von 1874-02-01 – 1929-07-15@\textsc{Hofmannsthal, Hugo von} (1874-02-01 – 1929-07-15), \emph{Schriftsteller}|pwv}\pwindex{Beer-Hofmann, Richard 1866-07-11 – 1945-09-26@\textsc{Beer-Hofmann, Richard} (1866-07-11 – 1945-09-26), \emph{Schriftsteller}|pwv} verſagen
               ſollten.\pend
           \pstart
           \label{K_L02753-4v}\edtext{\textsc{Granichstaedten\pwindex{Granichstaedten, Emil 1847-07-08 – 1904-07-02@\textsc{Granichstaedten, Emil} (1847-07-08 – 1904-07-02), \emph{Journalist, Rechtswissenschaftler}|pw}}}{\lemma{\textnormal{\emph{Granichstaedten}}}\Cendnote{\textnormal{Siehe Emil Granichstaedten\pwindex{Granichstaedten, Emil 1847-07-08 – 1904-07-02@\textsc{Granichstaedten, Emil} (1847-07-08 – 1904-07-02), \emph{Journalist, Rechtswissenschaftler}|pwk}: \emph{Deutsches Volkstheater. (»Ein Regentag«, Charakterbild von
                        J. J. David.)}\pwindex{Granichstaedten, Emil 1847-07-08 – 1904-07-02@\textsc{Granichstaedten, Emil} (1847-07-08 – 1904-07-02), \emph{Journalist, Rechtswissenschaftler}!Deutsches Volkstheater. (»Ein Regentag«, Charakterbild von J. J. David.)1895-10-15@\strich\emph{Deutsches Volkstheater. (»Ein Regentag«, Charakterbild von J. J. David.)} {[}1895-10-15{]}|pwk}. In: \emph{Die Presse}\pwindex{?? Werk@Nicht ermittelte Verfasserinnen und Verfasser!Presse1848-07-03@\emph{Die Presse} {[}1848-07-03{]}|pwk},
                     Jg. 48, Nr. 283, 15. 10. 1895, S. 1–2, hier:
                     S. 2. Siehe auch A. S.: \emph{Tagebuch}, 15. 10. 1895.}}}\label{K_L02753-4h}? Einen Dienſtmann engagiren, um ihm ins Geſicht zu \strikeout{ſpu\textcolor{gray}{ck}} ſpucken. Es lohnt nicht der Mühe, das ſelber zu thun. Aber \label{K_L02753-676v}\edtext{im Sommer}{\lemma{\textnormal{\emph{im Sommer}}}\Cendnote{\textnormal{Ab dem 31. 8. 1895 waren Schnitzler\pwindex{Schnitzler, Arthur 15.05.1862 – 21.10.1931@\textsc{Schnitzler, Arthur} (15.05.1862 – 21.10.1931), \emph{Schriftsteller, Mediziner}|pwk},
                     Goldmann\pwindex{Goldmann, Paul 31.01.1865 – 25.09.1935@\textsc{Goldmann, Paul} (31.01.1865 – 25.09.1935), \emph{Schriftsteller, Journalist}|pwk} und Beer-Hofmann\pwindex{Beer-Hofmann, Richard 1866-07-11 – 1945-09-26@\textsc{Beer-Hofmann, Richard} (1866-07-11 – 1945-09-26), \emph{Schriftsteller}|pwk} ein paar Tage gemeinsam in München\oindex{Muenchen@\textbf{München}|pwk}.}}}\label{K_L02753-676h} wart Ihr Beide\pwindex{Beer-Hofmann, Richard 1866-07-11 – 1945-09-26@\textsc{Beer-Hofmann, Richard} (1866-07-11 – 1945-09-26), \emph{Schriftsteller}|pwv} ja ſehr verſöhnlich geſtimmt gegen
               den Herrn\pwindex{Granichstaedten, Emil 1847-07-08 – 1904-07-02@\textsc{Granichstaedten, Emil} (1847-07-08 – 1904-07-02), \emph{Journalist, Rechtswissenschaftler}|pwv}! {\dotssix}\pend
           \pstart
           {\pb}Stolz werden? Nein, nein, ich \strikeout{weiß} weiß! So meinte ich es auch nie. Ich dachte an
               etwas Anderes, das kommen wird, zwiſchen Dir und mir oder zwiſchen mir und Dir\substVorne{}\textsuperscript{.}\substDazwischen{},\substHinten{} – langſam, langſam, aber ich fürchte, es kommt. In dieſer Beziehung ſiehſt
               Du, glaube ich, \strikeout{nicht\textcolor{gray}{,}} nicht ſo
               klar, wie ſonſt in allen Dingen.\pend
           \pstart
           Viele treue Grüße, mein lieber, lieber Freund! Wie bin ich froh, Dich ſoweit zu
               haben!\pend
           \pstart Dein \spacefill\mbox{Paul Goldmnn}\pend{}
         
         \endnumbering\mylabel{h}\end{ledgroupsized}  \newcommand{\dateiname}{L02753}\newcommand{\titel}{Paul Goldmann an Arthur Schnitzler, 15. 10. [1895]}\newcommand{\editorInnen}{Martin Anton Müller und Laura Untner}%% latex-leseansicht-abspann.tex
%% Abspann für die Leseansicht.
%% Der Schalter \ifkorrekturansicht ist bereits durch den Vorspann gesetzt.

%% latex-abspann.tex
%% Gemeinsamer Abspann für Korrekturansicht und Leseansicht.
%% Setzt den Schalter \ifkorrekturansicht voraus (gesetzt in den
%% einbindenden Dateien latex-korrekturansicht-abspann.tex bzw.
%% latex-leseansicht-abspann.tex).
%% ---------------------------------------------------------------

\normalsize

% Das esempio-Environment wird nur in der Leseansicht benötigt
\ifkorrekturansicht\else
\newenvironment{esempio}[3]%
{
    \vspace{1.5ex}
    \rlap{\underline{#1}}
    \par
    \setlength{\parindent}{0cm}
    \nopagebreak
    \leftskip=#2cm
    \rightskip=#3cm
}
{
    \par
}
\fi

\doendnotes{C}
\bigskip
\vfill

\clearpage

\footnotesize

\ifkorrekturansicht
  \lohead{\textsc{register}}
\fi

% theindex-Environment neu definieren ohne reledmac
\makeatletter
\renewenvironment{theindex}{%
  \ifkorrekturansicht
    \section*{\indexname}%
  \else
    \subsubsection*{Index der erwähnten Entitäten}%
  \fi
  \setlength{\parindent}{0pt}%
  \setlength{\parskip}{0pt plus 0.3pt}%
  \let\item\@idxitem
}{%
  \ifkorrekturansicht\clearpage\fi
}
\makeatother

\IfFileExists{\jobname-pw.ind}{\input{\jobname-pw.ind}}{}

% Quellenangabe nur in der Leseansicht
\ifkorrekturansicht\else
% Fallback-Definitionen, falls die .tex-Datei \titel etc. nicht gesetzt hat
\providecommand{\titel}{}
\providecommand{\editorInnen}{}
\providecommand{\dateiname}{\jobname}

\vspace{3cm}

\vfill

\footnotesize
\textsc{Quelle}: \titel. Herausgegeben von {\editorInnen}. In: \emph{Arthur Schnitzler: Briefwechsel mit Autorinnen und Autoren}.
 Digitale Edition, https://schnitzler-briefe.acdh.oeaw.ac.at/{\dateiname}.html (Stand \today)
\fi

\end{document}


      