%% latex-leseansicht-vorspann.tex
%% Vorspann für die Leseansicht.
%% Lädt die gemeinsame Datei latex-vorspann.tex mit nicht gesetztem Schalter.

\newif\ifkorrekturansicht
\korrekturansichtfalse

\input{../tex-inputs/latex-vorspann}


\section[Arthur Schnitzler an Richard Beer-Hofmann, 24. 8. 1895]{L00477 Arthur Schnitzler an Richard Beer-Hofmann, 24. 8. 1895}
\nopagebreak\mylabel{L00477v}
\rehead{ }\normalsize\beginnumbering\briefempfaengerindex{Beer-Hofmann, Richard@\textsc{Beer-Hofmann, Richard}!zzzSchnitzler, Arthur@\emph{von Arthur Schnitzler}!1895-08-241@{24. 8. 1895}|(be}
\toendnotes[C]{\smallbreak\pagebreak[2]}
\correspDesc{Versand  durch Arthur Schnitzler am 24. 8. 1895 in St. Johann in Tirol
\newline{}Erhalt  durch Richard Beer-Hofmann im Zeitraum [25. 8. 1895
                  – 29. 8. 1895?] in Bad Ischl}\toendnotes[C]{\smallbreak}
\Standort{YCGL, MSS 31.}
\physDesc{Brief, 1 Blatt, 2 Seiten, 627 Zeichen
\newline{}Handschrift: Bleistift, deutsche Kurrent}
\buchAbdrucke{\weitereDrucke{Arthur Schnitzler, Richard Beer-Hofmann: \emph{Briefwechsel 1891–1931}. Herausgegeben von Konstanze Fliedl. Wien, Zürich: \emph{Europaverlag} 1992, S. 78–79.} }\toendnotes[C]{\smallbreak}
\pstart
           \raggedleft{}{\pb}\textsc{St Johann in Tirol}\oindex{St. Johann in Tirol@\textbf{St. Johann in Tirol}, \emph{Verwaltungsgebiet}|pw}{\\}24. 8. 95\pend
           
\pstart{}Lieber Richard.\pend\vspace{0.5em}
\pstart
           Genau auf der \uline{Grenze} von \textsc{Baiern}\oindex{Bayern@\textbf{Bayern}, \emph{Land}|pw} u \textsc{Tirol}\oindex{Tirol@\textbf{Tirol}, \emph{Land}|pw}{ }ſauſte uns ein unheimlich gekleideter \textsc{Bicyclist} mit einem Dolch, Lederhoſen, Zugſchuhen, nackten
               Knieen, weißem Flanellhemd, keiner Cravate, Lodenhut entgegen, und war der Burckhard\pwindex{Burckhard, Max Eugen 14.\,7.\,1854 Korneuburg – 16.\,3.\,1912 Wien@\textsc{Burckhard, Max Eugen} (14.\,7.\,1854 Korneuburg – 16.\,3.\,1912 Wien), \emph{Schriftsteller, Rechtswissenschaftler, Theaterleiter}|pw}. –\pend
           
\pstart
           Jetzt hat es angefangen zu gießen, zu blitzen, zu donnern. Vielleicht{ }ſchlägt es ein;
                  da{\geminationn}{ }ſind wir extra von Salzburg\oindex{Salzburg@\textbf{Salzburg}, \emph{Verwaltungsgebiet}|pw} nach {\pb}Johann in Tirol\oindex{St. Johann in Tirol@\textbf{St. Johann in Tirol}, \emph{Verwaltungsgebiet}|pw} gefahren u. ſ. w. (Siehe Märchen\pwindex{Hofmannsthal, Hugo von 1.\,2.\,1874 Wien – 15.\,7.\,1929 Rodaun@\textsc{Hofmannsthal, Hugo von} (1.\,2.\,1874 Wien – 15.\,7.\,1929 Rodaun), \emph{Schriftsteller}!Märchen der 672. Nacht@\strich\emph{Das Märchen der 672. Nacht}|pw} von \textsc{Loris}\pwindex{Hofmannsthal, Hugo von 1.\,2.\,1874 Wien – 15.\,7.\,1929 Rodaun@\textsc{Hofmannsthal, Hugo von} (1.\,2.\,1874 Wien – 15.\,7.\,1929 Rodaun), \emph{Schriftsteller}|pw}.)\pend
           
\pstart
           Wir warten auf einen Zug. Die Partie war wunderbar. \label{K_L00477-1v}\edtext{\textsc{Le canif} das Federmeſſer}{\lemma{\textnormal{\emph{Le canif das Federmesser}}}\Cendnote{\textnormal{Die französische Vokabel ›canif‹ ist mit ›Federmesser‹ richtig übersetzt, die Anspielung bleibt unklar.}}}\label{K_L00477-1}.\pend
           
\pstart
           Herzliche Grüße{\\[\baselineskip]}Ihr \spacefill\mbox{Arthur}\pend
           \leftskip=0em{}
\pstart
           \noindent{}Wenn Sie jenes kleine Weſen\pwindex{Polak, Irma 11.\,6.\,1876 Ljubljana – 30.\,11.\,1931 Zagreb@\textsc{Polak, Irma} (11.\,6.\,1876 Ljubljana – 30.\,11.\,1931 Zagreb), \emph{Schauspielerin, Sängerin}|pwv}{ }ſehen, dem \label{K_L00477-2v}\edtext{Wehmut und Verachtung bevorſteht}{\lemma{\textnormal{\emph{Wehmut … bevorsteht}}}\Cendnote{\textnormal{Vgl. A. S.: \emph{Tagebuch}, 9. 8. 1895.
                  }}}\label{K_L00477-2}, grüßen Sie{ }ſie von mir.\pend
           \selectlanguage{ngerman}\endnumbering\briefempfaengerindex{Beer-Hofmann, Richard@\textsc{Beer-Hofmann, Richard}!zzzSchnitzler, Arthur@\emph{von Arthur Schnitzler}!1895-08-241@{24. 8. 1895}|)be}\mylabel{L00477h}  \newcommand{\dateiname}{L00477}\newcommand{\titel}{Arthur Schnitzler an Richard Beer-Hofmann, 24. 8. 1895}\newcommand{\editorInnen}{Martin Anton Müller und Gerd-Hermann Susen}%% latex-leseansicht-abspann.tex
%% Abspann für die Leseansicht.
%% Der Schalter \ifkorrekturansicht ist bereits durch den Vorspann gesetzt.

%% latex-abspann.tex
%% Gemeinsamer Abspann für Korrekturansicht und Leseansicht.
%% Setzt den Schalter \ifkorrekturansicht voraus (gesetzt in den
%% einbindenden Dateien latex-korrekturansicht-abspann.tex bzw.
%% latex-leseansicht-abspann.tex).
%% ---------------------------------------------------------------

\normalsize

% Das esempio-Environment wird nur in der Leseansicht benötigt
\ifkorrekturansicht\else
\newenvironment{esempio}[3]%
{
    \vspace{1.5ex}
    \rlap{\underline{#1}}
    \par
    \setlength{\parindent}{0cm}
    \nopagebreak
    \leftskip=#2cm
    \rightskip=#3cm
}
{
    \par
}
\fi

\doendnotes{C}
\bigskip
\vfill

\clearpage

\footnotesize

\ifkorrekturansicht
  \lohead{\textsc{register}}
\fi

% theindex-Environment neu definieren ohne reledmac
\makeatletter
\renewenvironment{theindex}{%
  \ifkorrekturansicht
    \section*{\indexname}%
  \else
    \subsubsection*{Index der erwähnten Entitäten}%
  \fi
  \setlength{\parindent}{0pt}%
  \setlength{\parskip}{0pt plus 0.3pt}%
  \let\item\@idxitem
}{%
  \ifkorrekturansicht\clearpage\fi
}
\makeatother

\IfFileExists{\jobname-pw.ind}{\input{\jobname-pw.ind}}{}

% Quellenangabe nur in der Leseansicht
\ifkorrekturansicht\else
% Fallback-Definitionen, falls die .tex-Datei \titel etc. nicht gesetzt hat
\providecommand{\titel}{}
\providecommand{\editorInnen}{}
\providecommand{\dateiname}{\jobname}

\vspace{3cm}

\vfill

\footnotesize
\textsc{Quelle}: \titel. Herausgegeben von {\editorInnen}. In: \emph{Arthur Schnitzler: Briefwechsel mit Autorinnen und Autoren}.
 Digitale Edition, https://schnitzler-briefe.acdh.oeaw.ac.at/{\dateiname}.html (Stand \today)
\fi

\end{document}


