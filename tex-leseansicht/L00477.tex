%% latex-korrekturansicht-vorspann.tex
%% Vorspann für die Korrekturansicht.
%% Lädt die gemeinsame Datei latex-vorspann.tex mit gesetztem Schalter.

\newif\ifkorrekturansicht
\korrekturansichttrue

\input{../tex-inputs/latex-vorspann}


\section[Arthur Schnitzler an Richard Beer-Hofmann, 24. 8. 1895]{L00477 Arthur Schnitzler an Richard Beer-Hofmann, 24. 8. 1895}
\nopagebreak\mylabel{L00477v}
\rehead{ }\normalsize\beginnumbering\briefempfaengerindex{Beer-Hofmann, Richard@\textsc{Beer-Hofmann, Richard}!zzzSchnitzler, Arthur@\emph{von Arthur Schnitzler}!1895-08-241@{24. 8. 1895}|(be}
\toendnotes[C]{\smallbreak\pagebreak[2]}\Standort{YCGL, MSS 31.}
\physDesc{Brief, 1 Blatt, 2 Seiten, 627 Zeichen
\newline{}Handschrift: Bleistift, deutsche Kurrent}
\buchAbdrucke{\weitereDrucke{Arthur Schnitzler, Richard Beer-Hofmann: \emph{Briefwechsel 1891–1931}. Wien, Zürich: \emph{Europaverlag} 1992, S. 78–79.} }\toendnotes[C]{\smallbreak}
\pstart
           \raggedleft{}{\pb}\textsc{St Johann in Tirol}\oindex{St. Johann in Tirol@\textbf{St. Johann in Tirol}, \emph{A.ADM3}|pw}{\\}24. 8. 95\pend
           
\pstart{}Lieber Richard.\pend\vspace{0.5em}
\pstart
           Genau auf der \uline{Grenze} von \textsc{Baiern}\oindex{Bayern@\textbf{Bayern}, \emph{A.ADM1}|pw} u \textsc{Tirol}\oindex{Tirol@\textbf{Tirol}, \emph{A.ADM1}|pw}{ }ſauſte uns ein unheimlich gekleideter \textsc{Bicyclist} mit einem Dolch, Lederhoſen, Zugſchuhen, nackten
               Knieen, weißem Flanellhemd, keiner Cravate, Lodenhut entgegen, und war der Burckhard\pwindex{Burckhard, Max Eugen 14.07.1854 – 16.03.1912@\textsc{Burckhard, Max Eugen} (14.07.1854 – 16.03.1912), \emph{Schriftsteller/Schriftstellerin, Rechtswissenschaftler/Rechtswissenschaftlerin, Theaterleiter/Theaterleiterin}|pw}. –\pend
           
\pstart
           Jetzt hat es angefangen zu gießen, zu blitzen, zu donnern. Vielleicht ſchlägt es ein;
                  da{\geminationn}{ }ſind wir extra von Salzburg\oindex{Salzburg@\textbf{Salzburg}, \emph{A.ADM2}|pw} nach {\pb}Johann in Tirol\oindex{St. Johann in Tirol@\textbf{St. Johann in Tirol}, \emph{A.ADM3}|pw} gefahren u. ſ. w. (Siehe Märchen\pwindex{Maerchen der 672. Nacht@\emph{Das Märchen der 672. Nacht}|pw} von \textsc{Loris}\pwindex{Hofmannsthal, Hugo von 1874-02-01 – 1929-07-15@\textsc{Hofmannsthal, Hugo von} (1874-02-01 – 1929-07-15), \emph{Schriftsteller/Schriftstellerin}|pw}.)\pend
           
\pstart
           Wir warten auf einen Zug. Die Partie war wunderbar. \label{K_L00477-1v}\edtext{\textsc{Le canif} das Federmeſſer}{\lemma{\textnormal{\emph{Le canif das Federmeſſer}}}\Cendnote{\textnormal{Die französische Vokabel ›canif‹ ist mit ›Federmesser‹ richtig übersetzt, die Anspielung bleibt unklar.}}}\label{K_L00477-1}.\pend
           
\pstart
           Herzliche Grüße{\\[\baselineskip]}Ihr \spacefill\mbox{Arthur}\pend
           \leftskip=0em{}
\pstart
           \noindent{}Wenn Sie jenes kleine Weſen\pwindex{Polak, Irma 11.06.1876 – 30.11.1931@\textsc{Polak, Irma} (11.06.1876 – 30.11.1931), \emph{Schauspieler/Schauspielerin, Sänger/Sängerin}|pwv}{ }ſehen, dem \label{K_L00477-2v}\edtext{Wehmut und Verachtung bevorſteht}{\lemma{\textnormal{\emph{Wehmut … bevorſteht}}}\Cendnote{\textnormal{Vgl. A. S.: \emph{Tagebuch}, 9. 8. 1895.
                  }}}\label{K_L00477-2}, grüßen Sie ſie von mir.\pend
           \selectlanguage{ngerman}\endnumbering\briefempfaengerindex{Beer-Hofmann, Richard@\textsc{Beer-Hofmann, Richard}!zzzSchnitzler, Arthur@\emph{von Arthur Schnitzler}!1895-08-241@{24. 8. 1895}|)be}\mylabel{L00477h}  \normalsize

\doendnotes{C}
\bigskip
\vfill

\clearpage

\footnotesize

\lohead{\textsc{register}}

% Definiere theindex-Environment komplett neu ohne reledmac
\makeatletter
\renewenvironment{theindex}{%
  \section*{\indexname}%
  \setlength{\parindent}{0pt}%
  \setlength{\parskip}{0pt plus 0.3pt}%
  \let\item\@idxitem
}{%
  \clearpage
}
\makeatother

\IfFileExists{\jobname-pw.ind}{\input{\jobname-pw.ind}}{}

\end{document}

      