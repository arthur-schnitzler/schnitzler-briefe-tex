%% latex-leseansicht-vorspann.tex
%% Vorspann für die Leseansicht.
%% Lädt die gemeinsame Datei latex-vorspann.tex mit nicht gesetztem Schalter.

\newif\ifkorrekturansicht
\korrekturansichtfalse

\input{../tex-inputs/latex-vorspann}

\begin{center}
            \textcolor{red}{ENTWURF. ENTZIFFERUNG NOCH NICHT KORREKTURGELESEN}
                      \end{center}
            
               \section[Arthur Schnitzler an Richard Beer-Hofmann, 24. 8. 1895]{ Arthur Schnitzler an Richard Beer-Hofmann, 24. 8. 1895}\nopagebreak\mylabel{v}\rehead{ }\begin{ledgroupsized}[t]{13cm}\normalsize\beginnumbering\briefempfaengerindex{Beer-Hofmann, Richard@\textsc{Beer-Hofmann, Richard}!zzzSchnitzler, Arthur@\emph{von Arthur Schnitzler}!1895-08-241@{24. 8. 1895}|(be} \toendnotes[C]{\smallbreak\pagebreak[2]} \Standort{YCGL, MSS 31.}
\physDesc{Brief, 1 Blatt, 2 Seiten
\newline{}Handschrift: Bleistift, deutsche Kurrent}\buchAbdrucke{\weitereDrucke{Arthur Schnitzler, Richard Beer-Hofmann: \emph{Briefwechsel 1891–1931}. Hg. Konstanze Fliedl. Wien, Zürich: \emph{Europaverlag} 1992, S. 78–79.} }\toendnotes[C]{\smallbreak}\pstart
           \raggedleft{}{\pb}\textsc{St Johann in Tirol}\oindex{St. Johann in Tirol@\textbf{St. Johann in Tirol}|pw}{\\}24. 8. 95\pend
           \pstart{}Lieber Richard.\pend\pstart
           Genau auf der \uline{Grenze} von \textsc{Baiern}\oindex{Bayern@\textbf{Bayern}|pw} u \textsc{Tirol}\oindex{Tirol@\textbf{Tirol}|pw}{ }ſauſte uns ein unheimlich gekleideter \textsc{Bicyclist} mit einem Dolch, Lederhoſen, Zugſchuhen, nackten
               Knieen, weißem Flanellhemd, keiner Cravate, Lodenhut entgegen, und war der Burckhard\pwindex{Burckhard, Max Eugen 14.07.1854 – 16.03.1912@\textsc{Burckhard, Max Eugen} (14.07.1854 – 16.03.1912), \emph{Schriftsteller, Rechtswissenschaftler, Theaterleiter}|pw}. –\pend
           \pstart
           Jetzt hat es angefangen zu gießen, zu blitzen, zu donnern. Vielleicht ſchlägt es ein;
                  da{\geminationn}{ }ſind wir extra von Salzburg\oindex{Salzburg@\textbf{Salzburg}|pw} nach {\pb}Johann in Tirol\oindex{St. Johann in Tirol@\textbf{St. Johann in Tirol}|pw} gefahren u. ſ. w. (Siehe Märchen\pwindex{Hofmannsthal, Hugo von 01.02.1874 – 15.07.1929@\textsc{Hofmannsthal, Hugo von} (01.02.1874 – 15.07.1929), \emph{Schriftsteller}!Maerchen der 672. Nacht2.11.1895 – 16.11.1895@\strich\emph{Das Märchen der 672. Nacht} {[}2.11.1895 – 16.11.1895{]}|pw} von \textsc{Loris}\pwindex{Hofmannsthal, Hugo von 01.02.1874 – 15.07.1929@\textsc{Hofmannsthal, Hugo von} (01.02.1874 – 15.07.1929), \emph{Schriftsteller}|pw}.)\pend
           \pstart
           Wir warten auf einen Zug. Die Partie war wunderbar. \label{K_L00477_1v}\edtext{\textsc{Le canif} das Federmeſſer}{\lemma{\textnormal{\emph{Le canif das Federmeſſer}}}\Cendnote{\textnormal{Die französische Vokabel »canif« richtig übersetzt, unklare
                  Anspielung.}}}\label{K_L00477_1h}.\pend
           \pstart
           Herzliche Grüße{\\[\baselineskip]}Ihr \spacefill\mbox{Arthur}\pend
           \leftskip=0em{}\pstart
           \noindent{}Wenn Sie jenes kleine Weſen\pwindex{Fabiani, Irma 11.06.1876 – 30.11.1931@\textsc{Fabiani, Irma} (11.06.1876 – 30.11.1931), \emph{Schauspielerin}|pwv}{ }ſehen, dem \label{K_L00477_2v}\edtext{Wehmut und Verachtung bevorſteht}{\lemma{\textnormal{\emph{Wehmut … bevorſteht}}}\Cendnote{\textnormal{vgl. A. S.: \emph{Tagebuch}, 9. 8. 1895}}}\label{K_L00477_2h}, grüßen
                  Sie ſie von mir.\pend
           \endnumbering\briefempfaengerindex{Beer-Hofmann, Richard@\textsc{Beer-Hofmann, Richard}!zzzSchnitzler, Arthur@\emph{von Arthur Schnitzler}!1895-08-241@{24. 8. 1895}|)be}\mylabel{h}\end{ledgroupsized}  \newcommand{\dateiname}{L00477}\newcommand{\titel}{Arthur Schnitzler an Richard Beer-Hofmann, 24. 8. 1895}\newcommand{\editorInnen}{Martin Anton Müller und Gerd-Hermann Susen}%% latex-leseansicht-abspann.tex
%% Abspann für die Leseansicht.
%% Der Schalter \ifkorrekturansicht ist bereits durch den Vorspann gesetzt.

%% latex-abspann.tex
%% Gemeinsamer Abspann für Korrekturansicht und Leseansicht.
%% Setzt den Schalter \ifkorrekturansicht voraus (gesetzt in den
%% einbindenden Dateien latex-korrekturansicht-abspann.tex bzw.
%% latex-leseansicht-abspann.tex).
%% ---------------------------------------------------------------

\normalsize

% Das esempio-Environment wird nur in der Leseansicht benötigt
\ifkorrekturansicht\else
\newenvironment{esempio}[3]%
{
    \vspace{1.5ex}
    \rlap{\underline{#1}}
    \par
    \setlength{\parindent}{0cm}
    \nopagebreak
    \leftskip=#2cm
    \rightskip=#3cm
}
{
    \par
}
\fi

\doendnotes{C}
\bigskip
\vfill

\clearpage

\footnotesize

\ifkorrekturansicht
  \lohead{\textsc{register}}
\fi

% theindex-Environment neu definieren ohne reledmac
\makeatletter
\renewenvironment{theindex}{%
  \ifkorrekturansicht
    \section*{\indexname}%
  \else
    \subsubsection*{Index der erwähnten Entitäten}%
  \fi
  \setlength{\parindent}{0pt}%
  \setlength{\parskip}{0pt plus 0.3pt}%
  \let\item\@idxitem
}{%
  \ifkorrekturansicht\clearpage\fi
}
\makeatother

\IfFileExists{\jobname-pw.ind}{\input{\jobname-pw.ind}}{}

% Quellenangabe nur in der Leseansicht
\ifkorrekturansicht\else
% Fallback-Definitionen, falls die .tex-Datei \titel etc. nicht gesetzt hat
\providecommand{\titel}{}
\providecommand{\editorInnen}{}
\providecommand{\dateiname}{\jobname}

\vspace{3cm}

\vfill

\footnotesize
\textsc{Quelle}: \titel. Herausgegeben von {\editorInnen}. In: \emph{Arthur Schnitzler: Briefwechsel mit Autorinnen und Autoren}.
 Digitale Edition, https://schnitzler-briefe.acdh.oeaw.ac.at/{\dateiname}.html (Stand \today)
\fi

\end{document}


      