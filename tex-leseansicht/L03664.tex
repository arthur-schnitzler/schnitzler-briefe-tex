%% latex-korrekturansicht-vorspann.tex
%% Vorspann für die Korrekturansicht.
%% Lädt die gemeinsame Datei latex-vorspann.tex mit gesetztem Schalter.

\newif\ifkorrekturansicht
\korrekturansichttrue

\input{../tex-inputs/latex-vorspann}


\section[Stefan Zweig an Arthur Schnitzler, 14. 4. 1919]{L03664 Stefan Zweig an Arthur Schnitzler, 14. 4. 1919}
\nopagebreak\mylabel{L03664v}
\rehead{ }\normalsize\beginnumbering\briefempfaengerindex{Schnitzler, Arthur@\textsc{Schnitzler, Arthur}!zzzZweig, Stefan@\emph{von Stefan Zweig}!1919-04-141@{14. 4. 1919}|(be}
\toendnotes[C]{\smallbreak\pagebreak[2]}\Standort{CUL, Schnitzler, B 118.}
\physDesc{Briefkarte, 1 Blatt, 1 Seite, 408 Zeichen
\newline{}Handschrift: lila Tinte, lateinische Kurrent}
\buchAbdrucke{\weitereDrucke{1) Stefan Zweig: \emph{Briefwechsel mit Hermann Bahr, Sigmund Freud, Rainer Maria
                        Rilke und Arthur Schnitzler}. Frankfurt am Main: \emph{S. Fischer} 1987, S. 410.} \weitereDrucke{2) Stefan Zweig: \emph{Briefe. Bd. II: 1914–1919}. Frankfurt am Main: \emph{S. Fischer} 1998, S. 278.} }\toendnotes[C]{\smallbreak}
\pstart
           {\pb}\textcolor{gray}{\textbf{SZ}}\hfill 14. April 1919\pend
           
\pstart
           \raggedleft{}\textcolor{gray}{\textbf{VIII. KOCHGASSE 8\oindex{Kochgasse 8@\textbf{Kochgasse 8}, \emph{Wohngebäude (K.WHS)}|pw}.}}\pend
           
\pstart
           \raggedleft{}Tel. 36 404\pend
           \vspace{0.5em}
\pstart
           Lieber verehrter Herr Doktor, meines Bleibens in Wien\oindex{Wien@\textbf{Wien}, \emph{A.ADM2}|pw} wird nicht lange sein: in etwa 18 Tagen gehe ich, und
               diesmal \label{K_L03664-1v}\edtext{wohl für immer}{\lemma{\textnormal{\emph{wohl für immer}}}\Cendnote{\textnormal{Am 29. 4. 1919
                  verlegte Zweig\pwindex{Zweig, Stefan 28.11.1881 – 23.02.1942@\textsc{Zweig, Stefan} (28.11.1881 – 23.02.1942), \emph{Schriftsteller/Schriftstellerin}|pwk} seinen Wohnsitz dauerhaft in
                  das Paschinger Schlössl\oindex{Paschinger Schloessl@\textbf{Paschinger Schlössl}, \emph{Wohngebäude (K.WHS)}|pwk} in Salzburg\oindex{Salzburg@\textbf{Salzburg}, \emph{A.ADM2}|pwk}.}}}\label{K_L03664-1}, fort. Gerne hätte ich gerade Sie, den
               wandellos Verehrten, zuvor noch \label{K_L03664-2v}\edtext{gesehen}{\lemma{\textnormal{\emph{gesehen}}}\Cendnote{\textnormal{Das gewünschte Treffen fand
                  am 22. 4. 1919
                  statt.}}}\label{K_L03664-2} und bitte Sie um Wort und Erlaubnis, wann ich zu Ihnen kommen darf.
               Mit vielen Empfehlungen Ihrer verehrten Frau Gemahlin\pwindex{Schnitzler, Olga 17.01.1882 – 13.01.1970@\textsc{Schnitzler, Olga} (17.01.1882 – 13.01.1970), \emph{Schauspieler/Schauspielerin, Sänger/Sängerin}|pwv} und den herzlichsten Grüssen Ihr Treu ergebener \pend
           \pstart \spacefill\mbox{Stefan Zweig}\pend{}\selectlanguage{ngerman}\endnumbering\briefempfaengerindex{Schnitzler, Arthur@\textsc{Schnitzler, Arthur}!zzzZweig, Stefan@\emph{von Stefan Zweig}!1919-04-141@{14. 4. 1919}|)be}\mylabel{L03664h}
\begin{anhang}
\end{anhang}\normalsize

\doendnotes{C}
\bigskip
\vfill

\clearpage

\footnotesize

\lohead{\textsc{register}}

% Definiere theindex-Environment komplett neu ohne reledmac
\makeatletter
\renewenvironment{theindex}{%
  \section*{\indexname}%
  \setlength{\parindent}{0pt}%
  \setlength{\parskip}{0pt plus 0.3pt}%
  \let\item\@idxitem
}{%
  \clearpage
}
\makeatother

\IfFileExists{\jobname-pw.ind}{\input{\jobname-pw.ind}}{}

\end{document}

      