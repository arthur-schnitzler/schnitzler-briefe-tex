%% latex-korrekturansicht-vorspann.tex
%% Vorspann für die Korrekturansicht.
%% Lädt die gemeinsame Datei latex-vorspann.tex mit gesetztem Schalter.

\newif\ifkorrekturansicht
\korrekturansichttrue

\input{../tex-inputs/latex-vorspann}


\section[Paul Goldmann an Arthur Schnitzler, 9. 8. {[}1895{]}]{L02743 Paul Goldmann an Arthur Schnitzler, 9. 8. {[}1895{]}}
\nopagebreak\mylabel{L02743v}
\rehead{ }\normalsize\beginnumbering\briefempfaengerindex{Schnitzler, Arthur@\textsc{Schnitzler, Arthur}!zzzGoldmann, Paul@\emph{von Paul Goldmann}!1895-08-093@{9. 8. {[}1895{]}}|(be}
\toendnotes[C]{\smallbreak\pagebreak[2]}\Standort{DLA, A:Schnitzler, HS.NZ85.1.3165.}
\physDesc{Brief, 2 Blätter, 8 Seiten, 2427 Zeichen
\newline{}Handschrift: schwarze Tinte, deutsche Kurrent
\newline{}Schnitzler: 1) mit Bleistift das Jahr »95« vermerkt  2) mit rotem Buntstift drei Unterstreichungen}\toendnotes[C]{\smallbreak}
\pstart
           \raggedleft{}{\pb}\textsc{Toelz\oindex{Bad Toelz@\textbf{Bad Tölz}, \emph{P.PPLA3}|pw}}, 9. Auguſt.\pend
           
\pstart\center{}Mein lieber Freund,\pend\vspace{0.5em}
\pstart
           Ich bin erſt Donnerſtag von \textsc{Paris\oindex{Paris@\textbf{Paris}, \emph{P.PPLC}|pw}} abgefahren u. bin ſpäter nach \textsc{Muenchen\oindex{Muenchen@\textbf{München}, \emph{P.PPLA}|pw}} gekommen, als ich gedacht. Denn ich habe mich in Straßburg\oindex{Strassburg@\textbf{Straßburg}, \emph{P.PPLA}|pw} u. im Schwarzwald\oindex{Schwarzwald@\textbf{Schwarzwald}, \emph{Gebirge (N.GBR)}|pw}
               aufgehalten zuſammen mit \textsc{Henri Albert\pwindex{Albert, Henri 1869-11-16 – 1921-08-03@\textsc{Albert, Henri} (1869-11-16 – 1921-08-03), \emph{Journalist/Journalistin, Kritiker/Kritikerin, Übersetzer/Übersetzerin}|pw}} u. \textsc{Charles Simon\pwindex{Simon, Charles 1850-07-08 – 1910-05-31@\textsc{Simon, Charles} (1850-07-08 – 1910-05-31), \emph{Schriftsteller/Schriftstellerin}|pw}}, einem neuen Bekannten\pwindex{Simon, Charles 1850-07-08 – 1910-05-31@\textsc{Simon, Charles} (1850-07-08 – 1910-05-31), \emph{Schriftsteller/Schriftstellerin}|pwv}, einem Menſchen\pwindex{Simon, Charles 1850-07-08 – 1910-05-31@\textsc{Simon, Charles} (1850-07-08 – 1910-05-31), \emph{Schriftsteller/Schriftstellerin}|pwv} von {\pb}Werth u. Eigenart, von dem
               ich Dir mündlich erzählen werde.\pend
           
\pstart
           In \textsc{Muenchen\oindex{Muenchen@\textbf{München}, \emph{P.PPLA}|pw}} fand ich Deine lieben Briefe vor, die mich innig erfreut haben. Ich wollte ſie
               gleich beantworten, kam aber nicht dazu. Denn meine Zeit wurde ausgefüllt von \textsc{Albert Langen\pwindex{Langen, Albert 1869-07-08 – 1909-04-30@\textsc{Langen, Albert} (1869-07-08 – 1909-04-30), \emph{Verleger/Verlegerin}|pw}}, dem Verleger\pwindex{Langen, Albert 1869-07-08 – 1909-04-30@\textsc{Langen, Albert} (1869-07-08 – 1909-04-30), \emph{Verleger/Verlegerin}|pwv} u. Lausbuben\pwindex{Langen, Albert 1869-07-08 – 1909-04-30@\textsc{Langen, Albert} (1869-07-08 – 1909-04-30), \emph{Verleger/Verlegerin}|pwv}, mit dem ich ein
               ſchweres Ärgerniß hatte, und von einem \label{K_L02743-1v}\edtext{Kindheits-{\pb}Freunde\pwindex{?? [Kindheitsfreund von Paul Goldmann] @\textsc{?? [Kindheitsfreund von Paul Goldmann]}|pwv}}{\lemma{\textnormal{\emph{Kindheits-Freunde}}}\Cendnote{\textnormal{nicht identifiziert}}}\label{K_L02743-1}, den ich
               zufällig dort traf. Seit geſtern bin ich in \textsc{Toelz\oindex{Bad Toelz@\textbf{Bad Tölz}, \emph{P.PPLA3}|pw}} u. die erſte freie Minute benütze ich, um Dir zu ſchreiben.\pend
           
\pstart
           Vielen, vielen Dank für Deine lieben Briefe. Es war ſoviel Tröſtliches u.
               Ermuthigendes darin! Das hat mich tief bewegt! {\dotsfive}\pend
           
\pstart
           Mir iſt es leid, daß ich auf unſere Zuſammenkunft noch ſo lange warten {\pb}ſoll. Aber es geht ja leider nicht anders wegen
               dieſer verdammten Kur\substVorne{}\textsuperscript{,}\substDazwischen{} (\substHinten{}die auch nicht nützen wird, wie alle früheren). Hier muß ich
               mindeſtens 3 Wochen bleiben, kann alſo vor 30.{ }\strikeout{Se}{ }Auguſt nicht fort. So muß ich Dich denn bitten: entweder
               tritt Deine \textsc{Bicycle}-Tour fünf Tage ſpäter an {\pb}oder komme auf ein paar Tage hierher. Letzteres wäre
               freilich eine \strikeout{Z} Zumuthung für Dich. Denn \label{K_L02743-2v}\edtext{\textsc{Toelz\oindex{Bad Toelz@\textbf{Bad Tölz}, \emph{P.PPLA3}|pw}}}{\lemma{\textnormal{\emph{Toelz}}}\Cendnote{\textnormal{Auch Schnitzler war von Bad Tölz\oindex{Bad Toelz@\textbf{Bad Tölz}, \emph{P.PPLA3}|pwk} nicht
                  angetan. Am 26. 8. 1895 notierte er im \emph{Tagebuch}\pwindex{Tagebuch@\emph{Tagebuch}|pwk}: »Schlechter Eindruck von Tölz\oindex{Bad Toelz@\textbf{Bad Tölz}, \emph{P.PPLA3}|pw}, verstimmend.«}}}\label{K_L02743-2} iſt das ſtumpfſinnigſte Neſt\oindex{Bad Toelz@\textbf{Bad Tölz}, \emph{P.PPLA3}|pwv}, das ich kenne, u. \strikeout{bat} bietet gar nichts. Auch landſchaftlich iſt es recht
               mäßig. Jedenfalls werde ich nicht mit Dir nach dem Norden reiſen können. Zwiſchen
                  10. u. 15. September{ }{\pb}muß ich wieder in \textsc{Paris\oindex{Paris@\textbf{Paris}, \emph{P.PPLC}|pw}} ſein. Auch habe ich kein Geld. Die Kur koſtet Unſummen.\pend
           
\pstart
           Was den Brief der \textsc{Candiani\pwindex{Candiani, Regina @\textsc{Candiani, Regina}, \emph{Schriftsteller/Schriftstellerin, Übersetzer/Übersetzerin}|pw}} betrifft, ſo kann ich Dir von hier aus nicht rathen. Ich hielt ſchon ſeinerzeit
               Umfrage, fand aber Niemanden, der die Dame\pwindex{Candiani, Regina @\textsc{Candiani, Regina}, \emph{Schriftsteller/Schriftstellerin, Übersetzer/Übersetzerin}|pw}
               kannte. Das Geſcheiteſte wäre, Du ſchriebeſt ihr: Herr \textsc{Goldmann}, der Mitte September{ }{\pb}nach \textsc{Paris\oindex{Paris@\textbf{Paris}, \emph{P.PPLC}|pw}} kommt, wird ſich mit Ihnen in Verbindung ſetzen. Ich würde dann hingehen u.
               verſuchen, mir \label{K_L02743-3v}\edtext{\begin{otherlanguage}{french}\textsc{de vive}\end{otherlanguage}}{\lemma{\textnormal{\emph{de vive}}}\Cendnote{\textnormal{französisch: aus dem Leben}}}\label{K_L02743-3} ein
               Urtheil zu bilden. Die »\textsc{Revue des jeunes filles\pwindex{Revue pour les jeunes filles@\emph{Revue pour les jeunes filles}|pwv}}«, von der ſie ſchreibt, iſt ein literariſch werthloſes, wenn ich nicht irre neu
               begründetes Blatt\strikeout{\textcolor{gray}{e}}\pwindex{Revue pour les jeunes filles@\emph{Revue pour les jeunes filles}|pwv} für höhere Töchter. Anbei der \label{K_L02743-4v}\edtext{Brief}{\lemma{\textnormal{\emph{Brief}}}\Cendnote{\textnormal{Beilage nicht erhalten}}}\label{K_L02743-4}.
                  {\pb}Daß Du den \label{K_L02743-5v}\edtext{erſten Akt von »Freiwild\pwindex{Freiwild. Schauspiel in 3 Akten@\emph{Freiwild. Schauspiel in 3 Akten}|pw}« beendet}{\lemma{\textnormal{\emph{erſten … beendet}}}\Cendnote{\textnormal{Siehe A. S.: \emph{Tagebuch}, 2. 8. 1895.
               }}}\label{K_L02743-5} haſt, iſt hoch erfreulich. Hoffentlich bringſt Du was zum Vorleſen mit.\pend
           
\pstart
           Die \label{K_L02743-6v}\edtext{Tinte}{\lemma{\textnormal{\emph{Tinte}}}\Cendnote{\textnormal{Die Tinte ist auf beiden Blättern häufig verschmiert.}}}\label{K_L02743-6}, mit
               der ich ſchreibe, gibt Dir einen Begriff von \textsc{Toelz\oindex{Bad Toelz@\textbf{Bad Tölz}, \emph{P.PPLA3}|pw}er} Comfort. Es ist die beſte im
               Ort.\pend
           
\pstart
           Schreib’ mir, bitte, eine Zeile: \textsc{Toelz\oindex{Bad Toelz@\textbf{Bad Tölz}, \emph{P.PPLA3}|pw}}, \textsc{Baiern\oindex{Bayern@\textbf{Bayern}, \emph{A.ADM1}|pw}}, \textsc{Poste restante}.\pend
           \pstart \label{T_L02743-1v}\edtext{Viele treue Grüße! Dein \spacefill\mbox{Paul
                  Goldmann}}{\lemma{\textnormal{\emph{Viele … Goldmann}}}\Cendnote{\textnormal{von oben nach unten entlang des linken
                  Randes, normal zum Text}}}\label{T_L02743-1}\pend{}
\pstart
           \noindent{}\label{T_L02743-2v}\edtext{Die herzlichſten Grüße an \textsc{Richard\pwindex{Beer-Hofmann, Richard 1866-07-11 – 1945-09-26@\textsc{Beer-Hofmann, Richard} (1866-07-11 – 1945-09-26), \emph{Schriftsteller/Schriftstellerin}|pw}}!}{\lemma{\textnormal{\emph{Die … Richard!}}}\Cendnote{\textnormal{entlang des Mittelfalzes von
                     unten nach oben, normal zum Text}}}\label{T_L02743-2}\pend
           \selectlanguage{ngerman}\endnumbering\briefempfaengerindex{Schnitzler, Arthur@\textsc{Schnitzler, Arthur}!zzzGoldmann, Paul@\emph{von Paul Goldmann}!1895-08-093@{9. 8. {[}1895{]}}|)be}\mylabel{L02743h}  \normalsize

\doendnotes{C}
\bigskip
\vfill

\clearpage

\footnotesize

\lohead{\textsc{register}}

% Definiere theindex-Environment komplett neu ohne reledmac
\makeatletter
\renewenvironment{theindex}{%
  \section*{\indexname}%
  \setlength{\parindent}{0pt}%
  \setlength{\parskip}{0pt plus 0.3pt}%
  \let\item\@idxitem
}{%
  \clearpage
}
\makeatother

\IfFileExists{\jobname-pw.ind}{\input{\jobname-pw.ind}}{}

\end{document}

      