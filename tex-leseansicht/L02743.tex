%% latex-leseansicht-vorspann.tex
%% Vorspann für die Leseansicht.
%% Lädt die gemeinsame Datei latex-vorspann.tex mit nicht gesetztem Schalter.

\newif\ifkorrekturansicht
\korrekturansichtfalse

\input{../tex-inputs/latex-vorspann}


\section[Paul Goldmann an Arthur Schnitzler, 9. 8. {[}1895{]}]{L02743 Paul Goldmann an Arthur Schnitzler, 9. 8. [1895]}
\nopagebreak\mylabel{L02743v}
\rehead{ }\normalsize\beginnumbering\briefempfaengerindex{Schnitzler, Arthur@\textsc{Schnitzler, Arthur}!zzzGoldmann, Paul@\emph{von Paul Goldmann}!1895-08-093@{9. 8. [1895]}|(be}
\toendnotes[C]{\smallbreak\pagebreak[2]}
\correspDesc{Versand  durch Paul Goldmann am 9. 8. [1895] in Bad Tölz
\newline{}Erhalt  durch Arthur Schnitzler im Zeitraum [10. 8. 1895
                  – 14. 8. 1895?] in Wien}\toendnotes[C]{\smallbreak}
\Standort{DLA, A:Schnitzler, HS.NZ85.1.3165.}
\physDesc{Brief, 2 Blätter, 8 Seiten, 2427 Zeichen
\newline{}Handschrift: schwarze Tinte, deutsche Kurrent
\newline{}Schnitzler: 1) mit Bleistift das Jahr »95« vermerkt  2) mit rotem Buntstift drei Unterstreichungen}\toendnotes[C]{\smallbreak}
\pstart
           \raggedleft{}{\pb}\textsc{Toelz\oindex{Bad Tölz@\textbf{Bad Tölz}, \emph{Hauptstadt}|pw}}, 9. Auguſt.\pend
           
\pstart\center{}Mein lieber Freund,\pend\vspace{0.5em}
\pstart
           Ich bin erſt Donnerſtag von \textsc{Paris\oindex{Paris@\textbf{Paris}, \emph{Hauptstadt}|pw}} abgefahren u. bin{ }ſpäter nach \textsc{Muenchen\oindex{München@\textbf{München}|pw}} gekommen, als ich gedacht. Denn ich habe mich in Straßburg\oindex{Straßburg@\textbf{Straßburg}|pw} u. im Schwarzwald\oindex{Schwarzwald@\textbf{Schwarzwald}, \emph{Gebirge}|pw}
               aufgehalten zuſammen mit \textsc{Henri Albert\pwindex{Albert, Henri 16.\,11.\,1869 Niederbronn-les-Bains – 3.\,8.\,1921 Straßburg@\textsc{Albert, Henri} (16.\,11.\,1869 Niederbronn-les-Bains – 3.\,8.\,1921 Straßburg), \emph{Journalist, Kritiker, Übersetzer}|pw}} u. \textsc{Charles Simon\pwindex{Simon, Charles 8.\,7.\,1850 Paris – 31.\,5.\,1910 ebd.@\textsc{Simon, Charles} (8.\,7.\,1850 Paris – 31.\,5.\,1910 ebd.), \emph{Schriftsteller}|pw}}, einem neuen Bekannten\pwindex{Simon, Charles 8.\,7.\,1850 Paris – 31.\,5.\,1910 ebd.@\textsc{Simon, Charles} (8.\,7.\,1850 Paris – 31.\,5.\,1910 ebd.), \emph{Schriftsteller}|pwv}, einem Menſchen\pwindex{Simon, Charles 8.\,7.\,1850 Paris – 31.\,5.\,1910 ebd.@\textsc{Simon, Charles} (8.\,7.\,1850 Paris – 31.\,5.\,1910 ebd.), \emph{Schriftsteller}|pwv} von {\pb}Werth u. Eigenart, von dem
               ich Dir mündlich erzählen werde.\pend
           
\pstart
           In \textsc{Muenchen\oindex{München@\textbf{München}|pw}} fand ich Deine lieben Briefe vor, die mich innig erfreut haben. Ich wollte{ }ſie
               gleich beantworten, kam aber nicht dazu. Denn meine Zeit wurde ausgefüllt von \textsc{Albert Langen\pwindex{Langen, Albert 8.\,7.\,1869 Antwerpen – 30.\,4.\,1909 München@\textsc{Langen, Albert} (8.\,7.\,1869 Antwerpen – 30.\,4.\,1909 München), \emph{Verleger}|pw}}, dem Verleger\pwindex{Langen, Albert 8.\,7.\,1869 Antwerpen – 30.\,4.\,1909 München@\textsc{Langen, Albert} (8.\,7.\,1869 Antwerpen – 30.\,4.\,1909 München), \emph{Verleger}|pwv} u. Lausbuben\pwindex{Langen, Albert 8.\,7.\,1869 Antwerpen – 30.\,4.\,1909 München@\textsc{Langen, Albert} (8.\,7.\,1869 Antwerpen – 30.\,4.\,1909 München), \emph{Verleger}|pwv}, mit dem ich ein{ }ſchweres Ärgerniß hatte, und von einem \label{K_L02743-1v}\edtext{Kindheits-{\pb}Freunde\pwindex{?? [Kindheitsfreund von Paul Goldmann] @\textsc{?? [Kindheitsfreund von Paul Goldmann]}|pwv}}{\lemma{\textnormal{\emph{Kindheits-Freunde}}}\Cendnote{\textnormal{nicht identifiziert}}}\label{K_L02743-1}, den ich
               zufällig dort traf. Seit geſtern bin ich in \textsc{Toelz\oindex{Bad Tölz@\textbf{Bad Tölz}, \emph{Hauptstadt}|pw}} u. die erſte freie Minute benütze ich, um Dir zu{ }ſchreiben.\pend
           
\pstart
           Vielen, vielen Dank für Deine lieben Briefe. Es war{ }ſoviel Tröſtliches u.
               Ermuthigendes darin! Das hat mich tief bewegt! {\dotsfive}\pend
           
\pstart
           Mir iſt es leid, daß ich auf unſere Zuſammenkunft noch{ }ſo lange warten {\pb}ſoll. Aber es geht ja leider nicht anders wegen
               dieſer verdammten Kur\substVorne{}\textsuperscript{,}\substDazwischen{} (\substHinten{}die auch nicht nützen wird, wie alle früheren). Hier muß ich
               mindeſtens 3 Wochen bleiben, kann alſo vor 30.{ }\strikeout{Se}{ }Auguſt nicht fort. So muß ich Dich denn bitten: entweder
               tritt Deine \textsc{Bicycle}-Tour fünf Tage{ }ſpäter an {\pb}oder komme auf ein paar Tage hierher. Letzteres wäre
               freilich eine \strikeout{Z} Zumuthung für Dich. Denn \label{K_L02743-2v}\edtext{\textsc{Toelz\oindex{Bad Tölz@\textbf{Bad Tölz}, \emph{Hauptstadt}|pw}}}{\lemma{\textnormal{\emph{Toelz}}}\Cendnote{\textnormal{Auch Schnitzler war von Bad Tölz\oindex{Bad Tölz@\textbf{Bad Tölz}, \emph{Hauptstadt}|pwk} nicht
                  angetan. Am 26. 8. 1895 notierte er im \emph{Tagebuch}\pwindex{Schnitzler, Arthur 15.\,5.\,1862 Wien – 21.\,10.\,1931 ebd.@\textsc{Schnitzler, Arthur} (15.\,5.\,1862 Wien – 21.\,10.\,1931 ebd.), \emph{Schriftsteller, Mediziner}!Tagebuch@\strich\emph{Tagebuch}|pwk}: »Schlechter Eindruck von Tölz\oindex{Bad Tölz@\textbf{Bad Tölz}, \emph{Hauptstadt}|pw}, verstimmend.«}}}\label{K_L02743-2} iſt das{ }ſtumpfſinnigſte Neſt\oindex{Bad Tölz@\textbf{Bad Tölz}, \emph{Hauptstadt}|pwv}, das ich kenne, u. \strikeout{bat} bietet gar nichts. Auch landſchaftlich iſt es recht
               mäßig. Jedenfalls werde ich nicht mit Dir nach dem Norden reiſen können. Zwiſchen
                  10. u. 15. September{ }{\pb}muß ich wieder in \textsc{Paris\oindex{Paris@\textbf{Paris}, \emph{Hauptstadt}|pw}}{ }ſein. Auch habe ich kein Geld. Die Kur koſtet Unſummen.\pend
           
\pstart
           Was den Brief der \textsc{Candiani\pwindex{Candiani, Regina @\textsc{Candiani, Regina}, \emph{Schriftstellerin, Übersetzerin}|pw}} betrifft,{ }ſo kann ich Dir von hier aus nicht rathen. Ich hielt{ }ſchon{ }ſeinerzeit
               Umfrage, fand aber Niemanden, der die Dame\pwindex{Candiani, Regina @\textsc{Candiani, Regina}, \emph{Schriftstellerin, Übersetzerin}|pw}
               kannte. Das Geſcheiteſte wäre, Du{ }ſchriebeſt ihr: Herr \textsc{Goldmann}, der Mitte September{ }{\pb}nach \textsc{Paris\oindex{Paris@\textbf{Paris}, \emph{Hauptstadt}|pw}} kommt, wird{ }ſich mit Ihnen in Verbindung{ }ſetzen. Ich würde dann hingehen u.
               verſuchen, mir \label{K_L02743-3v}\edtext{\begin{otherlanguage}{french}\textsc{de vive}\end{otherlanguage}}{\lemma{\textnormal{\emph{de vive}}}\Cendnote{\textnormal{französisch: aus dem Leben}}}\label{K_L02743-3} ein
               Urtheil zu bilden. Die »\textsc{Revue des jeunes filles\pwindex{Revue pour les jeunes filles@\emph{Revue pour les jeunes filles}|pwv}}«, von der{ }ſie{ }ſchreibt, iſt ein literariſch werthloſes, wenn ich nicht irre neu
               begründetes Blatt\strikeout{\textcolor{gray}{e}}\pwindex{Revue pour les jeunes filles@\emph{Revue pour les jeunes filles}|pwv} für höhere Töchter. Anbei der \label{K_L02743-4v}\edtext{Brief}{\lemma{\textnormal{\emph{Brief}}}\Cendnote{\textnormal{Beilage nicht erhalten}}}\label{K_L02743-4}.
                  {\pb}Daß Du den \label{K_L02743-5v}\edtext{erſten Akt von »Freiwild\pwindex{Schnitzler, Arthur 15.\,5.\,1862 Wien – 21.\,10.\,1931 ebd.@\textsc{Schnitzler, Arthur} (15.\,5.\,1862 Wien – 21.\,10.\,1931 ebd.), \emph{Schriftsteller, Mediziner}!Freiwild. Schauspiel in 3 Akten@\strich\emph{Freiwild. Schauspiel in 3 Akten}|pw}« beendet}{\lemma{\textnormal{\emph{ersten … beendet}}}\Cendnote{\textnormal{Siehe A. S.: \emph{Tagebuch}, 2. 8. 1895.
               }}}\label{K_L02743-5} haſt, iſt hoch erfreulich. Hoffentlich bringſt Du was zum Vorleſen mit.\pend
           
\pstart
           Die \label{K_L02743-6v}\edtext{Tinte}{\lemma{\textnormal{\emph{Tinte}}}\Cendnote{\textnormal{Die Tinte ist auf beiden Blättern häufig verschmiert.}}}\label{K_L02743-6}, mit
               der ich{ }ſchreibe, gibt Dir einen Begriff von \textsc{Toelz\oindex{Bad Tölz@\textbf{Bad Tölz}, \emph{Hauptstadt}|pw}er} Comfort. Es ist die beſte im
               Ort.\pend
           
\pstart
           Schreib’ mir, bitte, eine Zeile: \textsc{Toelz\oindex{Bad Tölz@\textbf{Bad Tölz}, \emph{Hauptstadt}|pw}}, \textsc{Baiern\oindex{Bayern@\textbf{Bayern}, \emph{Land}|pw}}, \textsc{Poste restante}.\pend
           \pstart \label{T_L02743-1v}\edtext{Viele treue Grüße! Dein \spacefill\mbox{Paul
                  Goldmann}}{\lemma{\textnormal{\emph{Viele … Goldmann}}}\Cendnote{\textnormal{von oben nach unten entlang des linken
                  Randes, normal zum Text}}}\label{T_L02743-1}\pend{}
\pstart
           \noindent{}\label{T_L02743-2v}\edtext{Die herzlichſten Grüße an \textsc{Richard\pwindex{Beer-Hofmann, Richard 11.\,7.\,1866 Wien – 26.\,9.\,1945 New York City@\textsc{Beer-Hofmann, Richard} (11.\,7.\,1866 Wien – 26.\,9.\,1945 New York City), \emph{Schriftsteller}|pw}}!}{\lemma{\textnormal{\emph{Die … Richard!}}}\Cendnote{\textnormal{entlang des Mittelfalzes von
                     unten nach oben, normal zum Text}}}\label{T_L02743-2}\pend
           \selectlanguage{ngerman}\endnumbering\briefempfaengerindex{Schnitzler, Arthur@\textsc{Schnitzler, Arthur}!zzzGoldmann, Paul@\emph{von Paul Goldmann}!1895-08-093@{9. 8. [1895]}|)be}\mylabel{L02743h}  \newcommand{\dateiname}{L02743}\newcommand{\titel}{Paul Goldmann an Arthur Schnitzler, 9. 8. [1895]}\newcommand{\editorInnen}{Martin Anton Müller und Laura Untner}%% latex-leseansicht-abspann.tex
%% Abspann für die Leseansicht.
%% Der Schalter \ifkorrekturansicht ist bereits durch den Vorspann gesetzt.

%% latex-abspann.tex
%% Gemeinsamer Abspann für Korrekturansicht und Leseansicht.
%% Setzt den Schalter \ifkorrekturansicht voraus (gesetzt in den
%% einbindenden Dateien latex-korrekturansicht-abspann.tex bzw.
%% latex-leseansicht-abspann.tex).
%% ---------------------------------------------------------------

\normalsize

% Das esempio-Environment wird nur in der Leseansicht benötigt
\ifkorrekturansicht\else
\newenvironment{esempio}[3]%
{
    \vspace{1.5ex}
    \rlap{\underline{#1}}
    \par
    \setlength{\parindent}{0cm}
    \nopagebreak
    \leftskip=#2cm
    \rightskip=#3cm
}
{
    \par
}
\fi

\doendnotes{C}
\bigskip
\vfill

\clearpage

\footnotesize

\ifkorrekturansicht
  \lohead{\textsc{register}}
\fi

% theindex-Environment neu definieren ohne reledmac
\makeatletter
\renewenvironment{theindex}{%
  \ifkorrekturansicht
    \section*{\indexname}%
  \else
    \subsubsection*{Index der erwähnten Entitäten}%
  \fi
  \setlength{\parindent}{0pt}%
  \setlength{\parskip}{0pt plus 0.3pt}%
  \let\item\@idxitem
}{%
  \ifkorrekturansicht\clearpage\fi
}
\makeatother

\IfFileExists{\jobname-pw.ind}{\input{\jobname-pw.ind}}{}

% Quellenangabe nur in der Leseansicht
\ifkorrekturansicht\else
% Fallback-Definitionen, falls die .tex-Datei \titel etc. nicht gesetzt hat
\providecommand{\titel}{}
\providecommand{\editorInnen}{}
\providecommand{\dateiname}{\jobname}

\vspace{3cm}

\vfill

\footnotesize
\textsc{Quelle}: \titel. Herausgegeben von {\editorInnen}. In: \emph{Arthur Schnitzler: Briefwechsel mit Autorinnen und Autoren}.
 Digitale Edition, https://schnitzler-briefe.acdh.oeaw.ac.at/{\dateiname}.html (Stand \today)
\fi

\end{document}


