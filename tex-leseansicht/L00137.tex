%% latex-leseansicht-vorspann.tex
%% Vorspann für die Leseansicht.
%% Lädt die gemeinsame Datei latex-vorspann.tex mit nicht gesetztem Schalter.

\newif\ifkorrekturansicht
\korrekturansichtfalse

\input{../tex-inputs/latex-vorspann}


\section[Karl Kraus an Arthur Schnitzler, 22. 11. 1892]{L00137 Karl Kraus an Arthur Schnitzler, 22. 11. 1892}
\nopagebreak\mylabel{L00137v}
\rehead{ }\normalsize\beginnumbering\briefempfaengerindex{Schnitzler, Arthur@\textsc{Schnitzler, Arthur}!zzzKraus, Karl@\emph{von Karl Kraus}!1892-11-221@{22. 11. 1892}|(be}
\toendnotes[C]{\smallbreak\pagebreak[2]}
\correspDesc{Versand  durch Karl Kraus am 22. 11. 1892 in Wien
\newline{}Erhalt  durch Arthur Schnitzler im Zeitraum [22. 11. 1892 – 26. 11. 1892?] in Wien}\toendnotes[C]{\smallbreak}
\Standort{CUL, Schnitzler, B 55.}
\physDesc{Postkarte, 623 Zeichen
\newline{}Handschrift: schwarze Tinte, deutsche Kurrent
\newline{}Versand: Stempel: »\nobreak{}\oindex{I., Innere Stadt@\textbf{I., Innere Stadt}, \emph{Verwaltungsgebiet}|pwk}Wien 1/1, 22. 11. 92, 4–5{[}N{]}\nobreak{}«.  }
\buchAbdrucke{\weitereDrucke{\emph{Karl Kraus und Arthur Schnitzler. Eine Dokumentation.}Herausgegeben von Reinhard Urbach In: \emph{Literatur und Kritik}, Bd. 49, Oktober 1970, S. 513.} }\toendnotes[C]{\smallbreak}\pstart{}{\pb}Herrn D\textsuperscript{r.}
                  Arthur Schnitzler\pend{}\pstart{}Schriftsteller\pend{}\pstart{}Wien I\oindex{I., Innere Stadt@\textbf{I., Innere Stadt}, \emph{Verwaltungsgebiet}|pw}\pend{}\pstart{}Grillparzerſtraße, 7\oindex{Wien@\textbf{Wien}!I., Innere Stadt@\textbf{I., Innere Stadt}!Grillparzerstraße@\textbf{Grillparzerstraße}, \emph{Straße}|pw}\pend{}{\bigskip}\vspace{1em}
\pstart
           \raggedleft{}{\pb}Postamt,
                  4 Uhr.\pend
           
\pstart{}Sehr verehrter Herr D\textsuperscript{r}\footnote{\noindent{}Bitte, das kann \uline{D}octo\uline{r}{ }\uline{und}{ }\uline{D}ichte\uline{r}
                        heißen!}!\pend\vspace{0.5em}
\pstart
           Heute nemlich habe ich von der »Allgemeinen\orgindex{Wiener Allgemeine Zeitung@Wiener Allgemeine Zeitung|pw}« das
                  Manuscript\pwindex{Kraus, Karl 28.\,4.\,1874 Jičín – 12.\,6.\,1936 Wien@\textsc{Kraus, Karl} (28.\,4.\,1874 Jičín – 12.\,6.\,1936 Wien), \emph{Schriftsteller, Publizist, Schriftsteller}!Wiener Lyriker@\strich\emph{Wiener Lyriker}|pwv}\pwindex{Kraus, Karl 28.\,4.\,1874 Jičín – 12.\,6.\,1936 Wien@\textsc{Kraus, Karl} (28.\,4.\,1874 Jičín – 12.\,6.\,1936 Wien), \emph{Schriftsteller, Publizist, Schriftsteller}!Arthur Schnitzler, Anatol@\strich\emph{Arthur Schnitzler, Anatol}|pwv}
               wiedererhalten. Die beiden andern Autoren\pwindex{Dörmann, Felix 29.\,5.\,1870 Wien – 26.\,10.\,1928 ebd.@\textsc{Dörmann, Felix} (29.\,5.\,1870 Wien – 26.\,10.\,1928 ebd.), \emph{Schriftsteller}|pwv}\pwindex{Specht, Richard 7.\,12.\,1870 Wien – 18.\,3.\,1932 ebd.@\textsc{Specht, Richard} (7.\,12.\,1870 Wien – 18.\,3.\,1932 ebd.), \emph{Schriftsteller, Journalist, Kritiker}|pwv}{ }ſind ihnen nicht wichtig genug und über \uline{Anatol\pwindex{Schnitzler, Arthur 15.\,5.\,1862 Wien – 21.\,10.\,1931 ebd.@\textsc{Schnitzler, Arthur} (15.\,5.\,1862 Wien – 21.\,10.\,1931 ebd.), \emph{Schriftsteller, Mediziner}!Anatol@\strich\emph{Anatol}|pw}} haben{ }ſie bereits \label{K_L00137-1v}\edtext{acceptiert}{\lemma{\textnormal{\emph{acceptiert}}}\Cendnote{\textnormal{Eine Rezension in der \emph{Wiener Allgemeinen Zeitung}\orgindex{Wiener Allgemeine Zeitung@Wiener Allgemeine Zeitung|pwk} 
                  ist nicht
                  nachgewiesen.}}}\label{K_L00137-1}.\pend
           
\pstart
           Faſt \uline{4 Wochen} wurde ich{ }ſo \uline{hingehalten}! Noch heute{ }ſende ich Anatol\pwindex{Schnitzler, Arthur 15.\,5.\,1862 Wien – 21.\,10.\,1931 ebd.@\textsc{Schnitzler, Arthur} (15.\,5.\,1862 Wien – 21.\,10.\,1931 ebd.), \emph{Schriftsteller, Mediziner}!Anatol@\strich\emph{Anatol}|pwv}\pwindex{Kraus, Karl 28.\,4.\,1874 Jičín – 12.\,6.\,1936 Wien@\textsc{Kraus, Karl} (28.\,4.\,1874 Jičín – 12.\,6.\,1936 Wien), \emph{Schriftsteller, Publizist, Schriftsteller}!Arthur Schnitzler, Anatol@\strich\emph{Arthur Schnitzler, Anatol}|pw}{ }\uline{allein}{ }\introOben{}D.\pwindex{Dörmann, Felix 29.\,5.\,1870 Wien – 26.\,10.\,1928 ebd.@\textsc{Dörmann, Felix} (29.\,5.\,1870 Wien – 26.\,10.\,1928 ebd.), \emph{Schriftsteller}|pw}{ }S.\pwindex{Specht, Richard 7.\,12.\,1870 Wien – 18.\,3.\,1932 ebd.@\textsc{Specht, Richard} (7.\,12.\,1870 Wien – 18.\,3.\,1932 ebd.), \emph{Schriftsteller, Journalist, Kritiker}|pw}\pwindex{Kraus, Karl 28.\,4.\,1874 Jičín – 12.\,6.\,1936 Wien@\textsc{Kraus, Karl} (28.\,4.\,1874 Jičín – 12.\,6.\,1936 Wien), \emph{Schriftsteller, Publizist, Schriftsteller}!Wiener Lyriker@\strich\emph{Wiener Lyriker}|pwv} extra\introOben{} an die »Gesellſch«.\pwindex{Gesellschaft. Monatsschrift für Litteratur, Kunst und Sozialpolitik@\emph{Die Gesellschaft. Monatsschrift für Litteratur, Kunst und Sozialpolitik}|pw}\pend
           
\pstart
           Freilich ist es{ }ſchon zu{ }ſpät für Dezemberheft. Werde jedenfalls \uline{meinen ganzen Einfluſs geltend}{ }\uline{machen}, daſs es noch ins Decemb.heft kommt. Wenn
               nicht iſt der Herr \uline{Osten}\pwindex{Osten, Heinrich 16.\,8.\,1855 Brody [Ukraine] – 1.\,8.\,1931 Wien@\textsc{Osten, Heinrich} (16.\,8.\,1855 Brody [Ukraine] – 1.\,8.\,1931 Wien), \emph{Schriftsteller, Journalist}|pw}, nicht \uline{ich} daran{ }ſchuld.\pend
           
\pstart
           Herzlichſten Gruß Ihr ergeb.{\\[\baselineskip]}\spacefill\mbox{Karl Kraus,}{ }Maximilianstr. 13\oindex{Wien@\textbf{Wien}!I., Innere Stadt@\textbf{I., Innere Stadt}!Mahlerstraße@\textbf{Mahlerstraße}, \emph{Straße}|pw}.\pend
           \leftskip=0em{}\selectlanguage{ngerman}\endnumbering\briefempfaengerindex{Schnitzler, Arthur@\textsc{Schnitzler, Arthur}!zzzKraus, Karl@\emph{von Karl Kraus}!1892-11-221@{22. 11. 1892}|)be}\mylabel{L00137h}  \newcommand{\dateiname}{L00137}\newcommand{\titel}{Karl Kraus an Arthur Schnitzler, 22. 11. 1892}\newcommand{\editorInnen}{Martin Anton Müller und Gerd-Hermann Susen}%% latex-leseansicht-abspann.tex
%% Abspann für die Leseansicht.
%% Der Schalter \ifkorrekturansicht ist bereits durch den Vorspann gesetzt.

%% latex-abspann.tex
%% Gemeinsamer Abspann für Korrekturansicht und Leseansicht.
%% Setzt den Schalter \ifkorrekturansicht voraus (gesetzt in den
%% einbindenden Dateien latex-korrekturansicht-abspann.tex bzw.
%% latex-leseansicht-abspann.tex).
%% ---------------------------------------------------------------

\normalsize

% Das esempio-Environment wird nur in der Leseansicht benötigt
\ifkorrekturansicht\else
\newenvironment{esempio}[3]%
{
    \vspace{1.5ex}
    \rlap{\underline{#1}}
    \par
    \setlength{\parindent}{0cm}
    \nopagebreak
    \leftskip=#2cm
    \rightskip=#3cm
}
{
    \par
}
\fi

\doendnotes{C}
\bigskip
\vfill

\clearpage

\footnotesize

\ifkorrekturansicht
  \lohead{\textsc{register}}
\fi

% theindex-Environment neu definieren ohne reledmac
\makeatletter
\renewenvironment{theindex}{%
  \ifkorrekturansicht
    \section*{\indexname}%
  \else
    \subsubsection*{Index der erwähnten Entitäten}%
  \fi
  \setlength{\parindent}{0pt}%
  \setlength{\parskip}{0pt plus 0.3pt}%
  \let\item\@idxitem
}{%
  \ifkorrekturansicht\clearpage\fi
}
\makeatother

\IfFileExists{\jobname-pw.ind}{\input{\jobname-pw.ind}}{}

% Quellenangabe nur in der Leseansicht
\ifkorrekturansicht\else
% Fallback-Definitionen, falls die .tex-Datei \titel etc. nicht gesetzt hat
\providecommand{\titel}{}
\providecommand{\editorInnen}{}
\providecommand{\dateiname}{\jobname}

\vspace{3cm}

\vfill

\footnotesize
\textsc{Quelle}: \titel. Herausgegeben von {\editorInnen}. In: \emph{Arthur Schnitzler: Briefwechsel mit Autorinnen und Autoren}.
 Digitale Edition, https://schnitzler-briefe.acdh.oeaw.ac.at/{\dateiname}.html (Stand \today)
\fi

\end{document}


