%% latex-korrekturansicht-vorspann.tex
%% Vorspann für die Korrekturansicht.
%% Lädt die gemeinsame Datei latex-vorspann.tex mit gesetztem Schalter.

\newif\ifkorrekturansicht
\korrekturansichttrue

\input{../tex-inputs/latex-vorspann}


\section[Karl Kraus an Arthur Schnitzler, 22. 11. 1892]{L00137 Karl Kraus an Arthur Schnitzler, 22. 11. 1892}
\nopagebreak\mylabel{L00137v}
\rehead{ }\normalsize\beginnumbering\briefempfaengerindex{Schnitzler, Arthur@\textsc{Schnitzler, Arthur}!zzzKraus, Karl@\emph{von Karl Kraus}!1892-11-221@{22. 11. 1892}|(be}
\toendnotes[C]{\smallbreak\pagebreak[2]}\Standort{CUL, Schnitzler, B 55.}
\physDesc{Postkarte, 623 Zeichen
\newline{}Handschrift: schwarze Tinte, deutsche Kurrent
\newline{}Versand: Stempel: »\nobreak{}\oindex{I., Innere Stadt@\textbf{I., Innere Stadt}, \emph{A.ADM3}|pwk}Wien 1/1, 22. 11. 92, 4–5{[}N{]}\nobreak{}«.  }
\buchAbdrucke{\weitereDrucke{\emph{Literatur und Kritik}, Bd. 49, Oktober 1970, S. 513.} }\toendnotes[C]{\smallbreak}\pstart{}{\pb}Herrn D\textsuperscript{r.}
                  Arthur Schnitzler\pend{}\pstart{}Schriftsteller\pend{}\pstart{}Wien I\oindex{I., Innere Stadt@\textbf{I., Innere Stadt}, \emph{A.ADM3}|pw}\pend{}\pstart{}Grillparzerſtraße, 7\oindex{Grillparzerstrasse@\textbf{Grillparzerstraße}, \emph{R.ST}|pw}\pend{}{\bigskip}\vspace{1em}
\pstart
           \raggedleft{}{\pb}Postamt,
                  4 Uhr.\pend
           
\pstart{}Sehr verehrter Herr D\textsuperscript{r}\noindent{}Bitte, das kann \uline{D}octo\uline{r}{ }\uline{und}{ }\uline{D}ichte\uline{r}
                        heißen!!\pend\vspace{0.5em}
\pstart
           Heute nemlich habe ich von der »Allgemeinen\orgindex{Wiener Allgemeine Zeitung@Wiener Allgemeine Zeitung|pw}« das
                  Manuscript\pwindex{Wiener Lyriker@\emph{Wiener Lyriker}|pwv}\pwindex{Arthur Schnitzler, Anatol@\emph{Arthur Schnitzler, Anatol}|pwv}
               wiedererhalten. Die beiden andern Autoren\pwindex{Doermann, Felix 29.05.1870 – 26.10.1928@\textsc{Dörmann, Felix} (29.05.1870 – 26.10.1928), \emph{Schriftsteller/Schriftstellerin}|pwv}\pwindex{Specht, Richard 07.12.1870 – 18.03.1932@\textsc{Specht, Richard} (07.12.1870 – 18.03.1932), \emph{Schriftsteller/Schriftstellerin, Journalist/Journalistin, Kritiker/Kritikerin}|pwv}{ }ſind ihnen nicht wichtig genug und über \uline{Anatol\pwindex{Anatol@\emph{Anatol}|pw}} haben ſie bereits \label{K_L00137-1v}\edtext{acceptiert}{\lemma{\textnormal{\emph{acceptiert}}}\Cendnote{\textnormal{Eine Rezension in der \emph{Wiener Allgemeinen Zeitung}\orgindex{Wiener Allgemeine Zeitung@Wiener Allgemeine Zeitung|pwk} 
                  ist nicht
                  nachgewiesen.}}}\label{K_L00137-1}.\pend
           
\pstart
           Faſt \uline{4 Wochen} wurde ich ſo \uline{hingehalten}! Noch heute ſende ich Anatol\pwindex{Anatol@\emph{Anatol}|pwv}\pwindex{Arthur Schnitzler, Anatol@\emph{Arthur Schnitzler, Anatol}|pw}{ }\uline{allein}{ }\introOben{}D.\pwindex{Doermann, Felix 29.05.1870 – 26.10.1928@\textsc{Dörmann, Felix} (29.05.1870 – 26.10.1928), \emph{Schriftsteller/Schriftstellerin}|pw}{ }S.\pwindex{Specht, Richard 07.12.1870 – 18.03.1932@\textsc{Specht, Richard} (07.12.1870 – 18.03.1932), \emph{Schriftsteller/Schriftstellerin, Journalist/Journalistin, Kritiker/Kritikerin}|pw}\pwindex{Wiener Lyriker@\emph{Wiener Lyriker}|pwv} extra\introOben{} an die »Gesellſch«.\pwindex{Gesellschaft. Monatsschrift fuer Litteratur, Kunst und Sozialpolitik@\emph{Die Gesellschaft. Monatsschrift für Litteratur, Kunst und Sozialpolitik}|pw}\pend
           
\pstart
           Freilich ist es ſchon zu ſpät für Dezemberheft. Werde jedenfalls \uline{meinen ganzen Einfluſs geltend}{ }\uline{machen}, daſs es noch ins Decemb.heft kommt. Wenn
               nicht iſt der Herr \uline{Osten}\pwindex{Osten, Heinrich 16.08.1855 – 01.08.1931@\textsc{Osten, Heinrich} (16.08.1855 – 01.08.1931), \emph{Schriftsteller/Schriftstellerin, Journalist/Journalistin}|pw}, nicht \uline{ich} daran ſchuld.\pend
           
\pstart
           Herzlichſten Gruß Ihr ergeb.{\\[\baselineskip]}\spacefill\mbox{Karl Kraus,}{ }Maximilianstr. 13\oindex{Mahlerstrasse@\textbf{Mahlerstraße}, \emph{Straße (K.STR)}|pw}. \pend
           \leftskip=0em{}\selectlanguage{ngerman}\endnumbering\briefempfaengerindex{Schnitzler, Arthur@\textsc{Schnitzler, Arthur}!zzzKraus, Karl@\emph{von Karl Kraus}!1892-11-221@{22. 11. 1892}|)be}\mylabel{L00137h}  \normalsize

\doendnotes{C}
\bigskip
\vfill

\clearpage

\footnotesize

\lohead{\textsc{register}}

% Definiere theindex-Environment komplett neu ohne reledmac
\makeatletter
\renewenvironment{theindex}{%
  \section*{\indexname}%
  \setlength{\parindent}{0pt}%
  \setlength{\parskip}{0pt plus 0.3pt}%
  \let\item\@idxitem
}{%
  \clearpage
}
\makeatother

\IfFileExists{\jobname-pw.ind}{\input{\jobname-pw.ind}}{}

\end{document}

      