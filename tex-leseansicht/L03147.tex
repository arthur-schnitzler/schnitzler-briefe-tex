%% latex-korrekturansicht-vorspann.tex
%% Vorspann für die Korrekturansicht.
%% Lädt die gemeinsame Datei latex-vorspann.tex mit gesetztem Schalter.

\newif\ifkorrekturansicht
\korrekturansichttrue

\input{../tex-inputs/latex-vorspann}


\section[ Felix Salten an Arthur Schnitzler, 15. 9. 189{[}4?{]}]{L03147 Felix Salten an Arthur Schnitzler, 15. 9. 189{[}4?{]}}
\nopagebreak\mylabel{L03147v}
\rehead{ }\normalsize\beginnumbering\briefempfaengerindex{Schnitzler, Arthur@\textsc{Schnitzler, Arthur}!zzzSalten, Felix@\emph{von Felix Salten}!1894-09-152@{15. 9. 189{[}4?{]}}|(be}
\toendnotes[C]{\smallbreak\pagebreak[2]}\Standort{CUL, Schnitzler, B 89, A 1.}
\physDesc{Visitenkarte, 298 Zeichen
\newline{}Handschrift: schwarze Tinte, lateinische Kurrent
\newline{}Schnitzler: 1) mit Bleistift beschriftet: »\noindent{}\textsc{Herr}{ }\textsc{M. J. Mayer}\pwindex{Mayer, M. J. @\textsc{Mayer, M. J.}, \emph{Medizinstudent/Medizinstudentin}|pw}.{ / }\textsc{Währ. Sechsschg. 4\oindex{Sechsschimmelgasse@\textbf{Sechsschimmelgasse}, \emph{Straße (K.STR)}|pw}\hspace*{1em}3. St.
                                             Th. 1\textcolor{gray}{4}}«  2) mit Bleistift datiert: »15/9 9\textcolor{gray}{4}«
\newline{}Ordnung: mit Bleistift von unbekannter Hand nummeriert: »48« }
\buchAbdrucke{\weitereDrucke{Hermann Bahr, Arthur Schnitzler: \emph{Briefwechsel, Aufzeichnungen, Dokumente (1891–1931)}. Göttingen: \emph{Wallstein} 2018, S. 81.} }\toendnotes[C]{\smallbreak}
\pstart
           \noindent{}\centering{}{\pb}\textcolor{gray}{\textbf{FELIX SALTEN}}\pend
           
\pstart
           
\pstart
           \textcolor{gray}{\textbf{WIEN\oindex{Wien@\textbf{Wien}, \emph{A.ADM2}|pw},}}\pend
           
\pstart
           \raggedleft{}\textcolor{gray}{\textbf{»Berliner Neueste
                        Nachrichten\orgindex{Berliner Neueste Nachrichten@Berliner Neueste Nachrichten|pw}.«}}\pend
           \pend
           
\pstart
           
\pstart
           \textcolor{gray}{\textbf{IX., Hörlgasse 16\oindex{Hoerlgasse 16@\textbf{Hörlgasse 16}, \emph{Wohngebäude (K.WHS)}|pw}.}}\pend
           
\pstart
           \raggedleft{}\textcolor{gray}{\textbf{»Münchener
                        General-Anzeiger\orgindex{Muenchener General-Anzeiger@Münchener General-Anzeiger|pw}.«}}\pend
           \pend
           
\pstart
           {\pb}Lieber Freund, wenn Sie dem Überbringer dieses irgend
               eine Abschreibearbeit geben können, so tun Sie's, bitte, wenn nicht, schicken Sie ihn
               vielleicht zu Bahr\pwindex{Bahr, Hermann 19.07.1863 – 15.01.1934@\textsc{Bahr, Hermann} (19.07.1863 – 15.01.1934), \emph{Schriftsteller/Schriftstellerin, Kritiker/Kritikerin}|pw}, der ja jetzt manches haben
               dürfte.\pend
           
\pstart
           Er ist \label{K_L03147-1v}\edtext{Mediziner im letzten
                  Jahrgang}{\lemma{\textnormal{\emph{Mediziner … Jahrgang}}}\Cendnote{\textnormal{Obwohl naheliegend, dürfte  nicht M. J. Mayer\pwindex{Mayer, M. J. @\textsc{Mayer, M. J.}, \emph{Medizinstudent/Medizinstudentin}|pwk} gemeibt sein –
                  zumindest hat niemand mit diesem Namen zu der Zeit in Wien\oindex{Wien@\textbf{Wien}, \emph{A.ADM2}|pwk} Medizin studiert.}}}\label{K_L03147-1} und es geht ihm sehr
               schlecht.\pend
           
\pstart
           Herzlichst {\\[\baselineskip]}\spacefill\mbox{Salten.}\pend
           \leftskip=0em{}
\pstart
           \noindent{}Vielleicht \label{K_L03147-2v}\edtext{Abends im Cafe}{\lemma{\textnormal{\emph{Abends im Cafe}}}\Cendnote{\textnormal{nicht
                     nachweisbar}}}\label{K_L03147-2}?\pend
           \selectlanguage{ngerman}\endnumbering\briefempfaengerindex{Schnitzler, Arthur@\textsc{Schnitzler, Arthur}!zzzSalten, Felix@\emph{von Felix Salten}!1894-09-152@{15. 9. 189{[}4?{]}}|)be}\mylabel{L03147h}  \normalsize

\doendnotes{C}
\bigskip
\vfill

\clearpage

\footnotesize

\lohead{\textsc{register}}

% Definiere theindex-Environment komplett neu ohne reledmac
\makeatletter
\renewenvironment{theindex}{%
  \section*{\indexname}%
  \setlength{\parindent}{0pt}%
  \setlength{\parskip}{0pt plus 0.3pt}%
  \let\item\@idxitem
}{%
  \clearpage
}
\makeatother

\IfFileExists{\jobname-pw.ind}{\input{\jobname-pw.ind}}{}

\end{document}

      