%% latex-leseansicht-vorspann.tex
%% Vorspann für die Leseansicht.
%% Lädt die gemeinsame Datei latex-vorspann.tex mit nicht gesetztem Schalter.

\newif\ifkorrekturansicht
\korrekturansichtfalse

\input{../tex-inputs/latex-vorspann}

\begin{center}
            \textcolor{red}{ENTWURF, NICHT FERTIG KORRIGIERT}
                      \end{center}
            
         
         \renewcommand{\erwaehntePersonen}{Personen: Hermann Bahr, M. J. Mayer}
         \renewcommand{\erwaehnteInstitutionen}{Institutionen: Berliner Neueste Nachrichten, Münchener General-Anzeiger}
         \renewcommand{\erwaehnteOrte}{Orte: Hörlgasse, Sechsschimmelgasse, Wien}
         \renewcommand{\erwaehnteWerke}{}
               \section[Felix Salten an Arthur Schnitzler, 15. 9. 189{[}4?{]}]{ Felix Salten an Arthur Schnitzler, 15. 9. 189{[}4?{]}}\nopagebreak\mylabel{v}\rehead{ }\begin{ledgroupsized}[t]{13cm}\normalsize\beginnumbering \toendnotes[C]{\smallbreak\pagebreak[2]} \Standort{CUL, Schnitzler, B 89, A 1.}
\physDesc{Visitenkarte
\newline{}Handschrift: schwarze Tinte, lateinische Kurrent
\newline{}Schnitzler: 1) mit Bleistift beschriftet: »\noindent{}\textsc{Herr}{ }\textsc{M. J. Mayer}\pwindex{Mayer, M. J. @\textsc{Mayer, M. J.}, \emph{Medizinstudent}|pw}.{ / }\textsc{Währ.
                                             Sechsschg. 4\oindex{Sechsschimmelgasse@\textbf{Sechsschimmelgasse}|pw}\hspace*{1em} 3. St.
                                             Th. 1\textcolor{gray}{4}}«  2) mit Bleistift datiert: »15/9
                                       9\textcolor{gray}{4}«}\buchAbdrucke{\weitereDrucke{Hermann Bahr, Arthur Schnitzler: \emph{Briefwechsel, Aufzeichnungen, Dokumente (1891–1931)}. Hg. Kurt Ifkovits und Martin Anton Müller. Göttingen: \emph{Wallstein} 2018, S. 81.} }\toendnotes[C]{\smallbreak}\pstart
           \noindent{}\centering{}{\pb}\textcolor{gray}{\textbf{FELIX SALTEN}}\pend
           \pstart
           \noindent{}\textcolor{gray}{\textbf{WIEN\oindex{Wien@\textbf{Wien}|pw},}}\hfill \textcolor{gray}{\textbf{»Berliner Neueste
                        Nachrichten\orgindex{Berliner Neueste Nachrichten@Berliner Neueste Nachrichten|pw}.«}}\pend
           \pstart
           \textcolor{gray}{\textbf{IX., Hörlgasse 16\oindex{Hoerlgasse@\textbf{Hörlgasse}|pw}.}}\hfill \textcolor{gray}{\textbf{»Münchener
                        General-Anzeiger\orgindex{Muenchener General-Anzeiger@Münchener General-Anzeiger|pw}.«}}\pend
           \pstart
           {\pb}Lieber Freund, wenn Sie dem Überbringer dieses irgend eine
               Abschreibearbeit geben können, so tun Sie's, bitte, wenn nicht, schicken Sie ihn
               vielleicht zu Bahr\pwindex{Bahr, Hermann 19.07.1863 – 15.01.1934@\textsc{Bahr, Hermann} (19.07.1863 – 15.01.1934), \emph{Schriftsteller, Kritiker}|pw}, der ja jetzt manches haben
               dürfte. \pend
           \pstart
           Er ist \label{K_L03147-1v}\edtext{Mediziner im letzten
                  Jahrgang}{\lemma{\textnormal{\emph{Mediziner … Jahrgang}}}\Cendnote{\textnormal{Obwohl naheliegend, dürfte es
                  sich nicht um M. J. Mayer\pwindex{Mayer, M. J. @\textsc{Mayer, M. J.}, \emph{Medizinstudent}|pwk} handeln,
                  zumindest hat niemand mit diesem Namen zu der Zeit in Wien\oindex{Wien@\textbf{Wien}|pwk} Medizin studiert.}}}\label{K_L03147-1h} und es geht ihm sehr schlecht. \pend
           \pstart
           {\\[\baselineskip]}Herzlichst {\\[\baselineskip]}\spacefill\mbox{Salten.}\pend
           \leftskip=0em{}\pstart
           \noindent{}Vielleicht Abends im Cafe?\pend
           
         
         \endnumbering\mylabel{h}\end{ledgroupsized}\begin{anhang}\end{anhang}\newcommand{\dateiname}{L03147}\newcommand{\titel}{Felix Salten an Arthur Schnitzler, 15. 9. 189[4?]}\newcommand{\editorInnen}{Martin Anton Müller und Laura Untner}%% latex-leseansicht-abspann.tex
%% Abspann für die Leseansicht.
%% Der Schalter \ifkorrekturansicht ist bereits durch den Vorspann gesetzt.

%% latex-abspann.tex
%% Gemeinsamer Abspann für Korrekturansicht und Leseansicht.
%% Setzt den Schalter \ifkorrekturansicht voraus (gesetzt in den
%% einbindenden Dateien latex-korrekturansicht-abspann.tex bzw.
%% latex-leseansicht-abspann.tex).
%% ---------------------------------------------------------------

\normalsize

% Das esempio-Environment wird nur in der Leseansicht benötigt
\ifkorrekturansicht\else
\newenvironment{esempio}[3]%
{
    \vspace{1.5ex}
    \rlap{\underline{#1}}
    \par
    \setlength{\parindent}{0cm}
    \nopagebreak
    \leftskip=#2cm
    \rightskip=#3cm
}
{
    \par
}
\fi

\doendnotes{C}
\bigskip
\vfill

\clearpage

\footnotesize

\ifkorrekturansicht
  \lohead{\textsc{register}}
\fi

% theindex-Environment neu definieren ohne reledmac
\makeatletter
\renewenvironment{theindex}{%
  \ifkorrekturansicht
    \section*{\indexname}%
  \else
    \subsubsection*{Index der erwähnten Entitäten}%
  \fi
  \setlength{\parindent}{0pt}%
  \setlength{\parskip}{0pt plus 0.3pt}%
  \let\item\@idxitem
}{%
  \ifkorrekturansicht\clearpage\fi
}
\makeatother

\IfFileExists{\jobname-pw.ind}{\input{\jobname-pw.ind}}{}

% Quellenangabe nur in der Leseansicht
\ifkorrekturansicht\else
% Fallback-Definitionen, falls die .tex-Datei \titel etc. nicht gesetzt hat
\providecommand{\titel}{}
\providecommand{\editorInnen}{}
\providecommand{\dateiname}{\jobname}

\vspace{3cm}

\vfill

\footnotesize
\textsc{Quelle}: \titel. Herausgegeben von {\editorInnen}. In: \emph{Arthur Schnitzler: Briefwechsel mit Autorinnen und Autoren}.
 Digitale Edition, https://schnitzler-briefe.acdh.oeaw.ac.at/{\dateiname}.html (Stand \today)
\fi

\end{document}


      