%% latex-leseansicht-vorspann.tex
%% Vorspann für die Leseansicht.
%% Lädt die gemeinsame Datei latex-vorspann.tex mit nicht gesetztem Schalter.

\newif\ifkorrekturansicht
\korrekturansichtfalse

\input{../tex-inputs/latex-vorspann}


\section[Arthur Schnitzler an Karin Michaëlis, 8. 11. 1916]{L02603 Arthur Schnitzler an Karin Michaëlis, 8. 11. 1916}
\nopagebreak\mylabel{L02603v}
\rehead{ }\normalsize\beginnumbering\briefempfaengerindex{Michaëlis, Karin@\textsc{Michaëlis, Karin}!zzzSchnitzler, Arthur@\emph{von Arthur Schnitzler}!1916-11-081@{8. 11. 1916}|(be}
\toendnotes[C]{\smallbreak\pagebreak[2]}
\correspDesc{Versand  durch Arthur Schnitzler am 8. 11. 1916 in Wien
\newline{}Weiterleitung  am 14. 11. 1916 in Svendborg
\newline{}Umleitung  in Thurø
\newline{}Erhalt  durch Karin Michaëlis im Zeitraum [15. 11. 1916 – 18. 11. 1916?] in Kopenhagen}\toendnotes[C]{\smallbreak}
\Standort{Kopenhagen, Det Kongelige Bibliotek, Palsbo Ac.}
\physDesc{Postkarte, 712 Zeichen
\newline{}Handschrift: schwarze Tinte, lateinische Kurrent
\newline{}Versand: 1) Stempel: »\nobreak{}\oindex{Wien@\textbf{Wien}, \emph{Verwaltungsgebiet}|pwk}Wien, 8. 11. 16, 4\nobreak{}«.   2) Stempel: »\nobreak{}Zensuriert K. u. k.
                                          Zensurstelle\orgindex{K. u. k. Zensurstelle@K. u. k. Zensurstelle|pw}\nobreak{}«.  3) Stempel: »\nobreak{}\oindex{Svendborg@\textbf{Svendborg}, \emph{Hauptstadt}|pwk}Svendborg, 14. 11. 16, 7–9F\nobreak{}«.  4) Stempel: »\nobreak{}\oindex{Thurø@\textbf{Thurø}, \emph{Insel}|pwk}Thurø\nobreak{}«.  5) ursprüngliche Adressierung überklebt und von unbekannter Hand
                                 mit schwarzer Tinte neue Empfangsadresse vermerkt: »\noindent{}adr{ / }Fru Herdis Bergstrøm\pwindex{Bergstrøm, Herdis 17.\,11.\,1874 Østermarie – 19.\,2.\,1962 Bagsværd@\textsc{Bergstrøm, Herdis} (17.\,11.\,1874 Østermarie – 19.\,2.\,1962 Bagsværd), \emph{Dramatikerin}|pw}{ / }Dosseringen 304 {\\}Kobenhamn\oindex{Sortedam Dossering@\textbf{Sortedam Dossering}, \emph{Straße}|pwu}« }
\buchAbdrucke{\weitereDrucke{Arthur Schnitzler: \emph{Arthur Schnitzlers Briefe nach Dänemark}. Herausgegeben von Ernst-Ullrich Pinkert. Roskilde: \emph{Center for Østrigsk-Nordiske Kulturstudier} 2006, S. 18.} }\toendnotes[C]{\smallbreak}\pstart{}{\pb}\textcolor{gray}{\textbf{Dr. Arthur Schnitzler}}\pend{}\pstart{}\textcolor{gray}{\textbf{Wien XVIII. Sternwartestrasse 71\oindex{Wien@\textbf{Wien}!XVIII., Währing@\textbf{XVIII., Währing}!Sternwartestraße 71@\textbf{Sternwartestraße 71}, \emph{Wohngebäude}|pw}}}\pend{}{\bigskip}\pstart{}Frau Karin Michaelis\pend{}\pstart{}{[}\label{K_L02603-1v}\edtext{Thurø}{\lemma{\textnormal{\emph{Thurø}}}\Cendnote{\textnormal{Bei dieser Adresszeile handelt es sich um eine
                           Rekonstruktion, da der betreffende Teil auf der Karte abgeklebt ist. Da
                           sich aber der Stempel von Thurø\oindex{Thurø@\textbf{Thurø}, \emph{Insel}|pwk} auf
                           der Karte findet und hier Karin Michaëlis einen Wohnsitz hatte, kann die
                           ursprüngliche Adressangabe, zumindest soweit es um die Ortsangabe geht,
                           erschlossen werden.}}}\label{K_L02603-1}\oindex{Thurø@\textbf{Thurø}, \emph{Insel}|pw}{]}\pend{}\pstart{}\damage{\textcolor{gray}{Dänemark\oindex{Dänemark@\textbf{Dänemark}|pw}}}\pend{}{\bigskip}\vspace{1em}
\pstart
           \raggedleft{}{\pb}8. 11. 916\pend
           \vspace{0.5em}
\pstart
           verehrte Frau Karin Michaelis – es freut mich sehr,
               daß Ihnen die Beate\pwindex{Schnitzler, Arthur 15.\,5.\,1862 Wien – 21.\,10.\,1931 ebd.@\textsc{Schnitzler, Arthur} (15.\,5.\,1862 Wien – 21.\,10.\,1931 ebd.), \emph{Schriftsteller, Mediziner}!Frau Beate und ihr Sohn. Novelle@\strich\emph{Frau Beate und ihr Sohn. Novelle}|pw} gefallen hat, eins meiner
               Werke, das vielfach und mit besondrer Vorliebe \label{K_L02603-2v}\edtext{misverstanden}{\lemma{\textnormal{\emph{misverstanden}}}\Cendnote{\textnormal{Kritisch begutachtet wurden vor allem die erotischen Inhalte, ganz besonders die
                  inzestuös deutbaren Momente in der Novelle \emph{Frau
                     Beate und ihr Sohn}\pwindex{Schnitzler, Arthur 15.\,5.\,1862 Wien – 21.\,10.\,1931 ebd.@\textsc{Schnitzler, Arthur} (15.\,5.\,1862 Wien – 21.\,10.\,1931 ebd.), \emph{Schriftsteller, Mediziner}!Frau Beate und ihr Sohn. Novelle@\strich\emph{Frau Beate und ihr Sohn. Novelle}|pwk} bzw. die »Unsittlichkeit« (vgl. A. S.: \emph{Tagebuch}, 14. 9. 1913) der
                  Protagonistin Beate.}}}\label{K_L02603-2} wird. Der Schluss scheint ja (offenbar aus
               künstlerischen – nicht dramatischen – Gründen) – wie mir der Zweifel auch \label{K_L02603-3v}\edtext{Wohlwollender}{\lemma{\textnormal{\emph{Wohlwollender}}}\Cendnote{\textnormal{Am 24. 2. 1913 las Schnitzler{ }\emph{Frau Beate und ihr Sohn}\pwindex{Schnitzler, Arthur 15.\,5.\,1862 Wien – 21.\,10.\,1931 ebd.@\textsc{Schnitzler, Arthur} (15.\,5.\,1862 Wien – 21.\,10.\,1931 ebd.), \emph{Schriftsteller, Mediziner}!Frau Beate und ihr Sohn. Novelle@\strich\emph{Frau Beate und ihr Sohn. Novelle}|pwk}{ }Richard Beer-Hofmann\pwindex{Beer-Hofmann, Richard 11.\,7.\,1866 Wien – 26.\,9.\,1945 New York City@\textsc{Beer-Hofmann, Richard} (11.\,7.\,1866 Wien – 26.\,9.\,1945 New York City), \emph{Schriftsteller}|pwk}, Hugo von Hofmannsthal\pwindex{Hofmannsthal, Hugo von 1.\,2.\,1874 Wien – 15.\,7.\,1929 Rodaun@\textsc{Hofmannsthal, Hugo von} (1.\,2.\,1874 Wien – 15.\,7.\,1929 Rodaun), \emph{Schriftsteller}|pwk}, Leo Van-Jung\pwindex{Van-Jung, Leo 15.\,10.\,1866 Odessa – 2.\,7.\,1939 Riga@\textsc{Van-Jung, Leo} (15.\,10.\,1866 Odessa – 2.\,7.\,1939 Riga), \emph{Gesangspädagoge, Mathematiker}|pwk}, Felix Salten\pwindex{Salten, Felix 6.\,9.\,1869 Budapest – 8.\,10.\,1945 Zürich@\textsc{Salten, Felix} (6.\,9.\,1869 Budapest – 8.\,10.\,1945 Zürich), \emph{Schriftsteller, Journalist, Chefredakteur}|pwk}, Jakob Wassermann\pwindex{Wassermann, Jakob 10.\,3.\,1873 Fürth – 1.\,1.\,1934 Altaussee@\textsc{Wassermann, Jakob} (10.\,3.\,1873 Fürth – 1.\,1.\,1934 Altaussee), \emph{Schriftsteller}|pwk}, Gustav Schwarzkopf\pwindex{Schwarzkopf, Gustav 7.\,11.\,1853 Wien – 13.\,11.\,1939 ebd.@\textsc{Schwarzkopf, Gustav} (7.\,11.\,1853 Wien – 13.\,11.\,1939 ebd.), \emph{Schriftsteller}|pwk} und seiner Frau Olga\pwindex{Schnitzler, Olga 17.\,1.\,1882 Wien – 13.\,1.\,1970 Lugano@\textsc{Schnitzler, Olga} (17.\,1.\,1882 Wien – 13.\,1.\,1970 Lugano), \emph{Schauspielerin, Sängerin}|pwk} vor. Vor allem für den Schluss wurde er kritisiert. Vgl. A. S.: \emph{Tagebuch}, 23. 2. 1913.
               }}}\label{K_L02603-3} zu bedenken gibt – nicht durchaus überzeugend zu sein. – Ich schreibe Ihnen
               meinen Dank und Gruß auf einer Karte – die nach meiner Erfahrung \label{K_L02603-4v}\edtext{sichrer ins neutrale Ausland\oindex{Dänemark@\textbf{Dänemark}|pwv} gelangt}{\lemma{\textnormal{\emph{sichrer … gelangt}}}\Cendnote{\textnormal{Postalisch versandte Korrespondenzstücke wurden von der \emph{K. u. k. Zensurstelle}\orgindex{K. u. k. Zensurstelle@K. u. k. Zensurstelle|pwk} gelesen, egal ob Brief
                  oder Postkarte. Bei Letzterer wurde aber, da sie offen versandt wurde, eher davon
                  ausgegangen, dass auf ihr keine Geheimnisse stehen konnten.}}}\label{K_L02603-4} als Briefe –
                  {\pb}auf die Gefahr
               hin, daß Sie mich für minder correct (aber gerade zu langweilig) halten wie früher.\pend
           
\pstart
           {\pb}Auf Wiedersehen
               hoffentlich, und sch\textcolor{gray}{oene} Grüße, auch von meiner Frau\pwindex{Schnitzler, Olga 17.\,1.\,1882 Wien – 13.\,1.\,1970 Lugano@\textsc{Schnitzler, Olga} (17.\,1.\,1882 Wien – 13.\,1.\,1970 Lugano), \emph{Schauspielerin, Sängerin}|pwv}. Ihr sehr ergebner{\\[\baselineskip]}\spacefill\mbox{Arthur Schnitzler}\pend
           \leftskip=0em{}\selectlanguage{ngerman}\endnumbering\briefempfaengerindex{Michaëlis, Karin@\textsc{Michaëlis, Karin}!zzzSchnitzler, Arthur@\emph{von Arthur Schnitzler}!1916-11-081@{8. 11. 1916}|)be}\mylabel{L02603h}  \newcommand{\dateiname}{L02603}\newcommand{\titel}{Arthur Schnitzler an Karin Michaëlis, 8. 11. 1916}\newcommand{\editorInnen}{Martin Anton Müller und Laura Untner}%% latex-leseansicht-abspann.tex
%% Abspann für die Leseansicht.
%% Der Schalter \ifkorrekturansicht ist bereits durch den Vorspann gesetzt.

%% latex-abspann.tex
%% Gemeinsamer Abspann für Korrekturansicht und Leseansicht.
%% Setzt den Schalter \ifkorrekturansicht voraus (gesetzt in den
%% einbindenden Dateien latex-korrekturansicht-abspann.tex bzw.
%% latex-leseansicht-abspann.tex).
%% ---------------------------------------------------------------

\normalsize

% Das esempio-Environment wird nur in der Leseansicht benötigt
\ifkorrekturansicht\else
\newenvironment{esempio}[3]%
{
    \vspace{1.5ex}
    \rlap{\underline{#1}}
    \par
    \setlength{\parindent}{0cm}
    \nopagebreak
    \leftskip=#2cm
    \rightskip=#3cm
}
{
    \par
}
\fi

\doendnotes{C}
\bigskip
\vfill

\clearpage

\footnotesize

\ifkorrekturansicht
  \lohead{\textsc{register}}
\fi

% theindex-Environment neu definieren ohne reledmac
\makeatletter
\renewenvironment{theindex}{%
  \ifkorrekturansicht
    \section*{\indexname}%
  \else
    \subsubsection*{Index der erwähnten Entitäten}%
  \fi
  \setlength{\parindent}{0pt}%
  \setlength{\parskip}{0pt plus 0.3pt}%
  \let\item\@idxitem
}{%
  \ifkorrekturansicht\clearpage\fi
}
\makeatother

\IfFileExists{\jobname-pw.ind}{\input{\jobname-pw.ind}}{}

% Quellenangabe nur in der Leseansicht
\ifkorrekturansicht\else
% Fallback-Definitionen, falls die .tex-Datei \titel etc. nicht gesetzt hat
\providecommand{\titel}{}
\providecommand{\editorInnen}{}
\providecommand{\dateiname}{\jobname}

\vspace{3cm}

\vfill

\footnotesize
\textsc{Quelle}: \titel. Herausgegeben von {\editorInnen}. In: \emph{Arthur Schnitzler: Briefwechsel mit Autorinnen und Autoren}.
 Digitale Edition, https://schnitzler-briefe.acdh.oeaw.ac.at/{\dateiname}.html (Stand \today)
\fi

\end{document}


