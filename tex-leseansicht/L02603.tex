\input{../tex-inputs/latex-pdf-vorspann}
\begin{center}
            \textcolor{red}{ENTWURF. ENTZIFFERUNG NOCH NICHT KORREKTURGELESEN}
                      \end{center}
            
               \section[Arthur Schnitzler an Karin Michaëlis, 8. 11. 1916]{ Arthur Schnitzler an Karin Michaëlis, 8. 11. 1916}\nopagebreak\mylabel{v}\rehead{ }\begin{ledgroupsized}[t]{13cm}\normalsize\beginnumbering\briefempfaengerindex{Michaelis, Karin@\textsc{Michaëlis, Karin}!zzzSchnitzler, Arthur@\emph{von Arthur Schnitzler}!1912-10-211@{8. 11. 1916}|(be} \toendnotes[C]{\smallbreak\pagebreak[2]} \Standort{Kopenhagen, Det Kongelige Bibliotek, Palsbo Ac.}
\physDesc{Postkarte
\newline{}Handschrift: schwarze Tinte, lateinische Kurrent\newline{}Versand: 1) Stempel: »\nobreak{}Wien, 8. 11. 16, 4\nobreak{}«.  2) Stempel: »\nobreak{}Zensuriert K. u. k.
                                          Zensurstelle\orgindex{K. u. k. Zensurstelle@K. u. k. Zensurstelle|pw}\nobreak{}«. 3) Stempel: »\nobreak{}\oindex{Svendborg@\textbf{Svendborg}|pwk}Svendborg, 14. 11. 16, 7–9F\nobreak{}«. 4) Stempel: »\nobreak{}\oindex{Thurø@\textbf{Thurø}|pwk}Thurø\nobreak{}«. 5) ursprüngliche Adressierung überklebt und von unbekannter Hand
                                 mit schwarzer Tinte neue Empfangsadresse vermerkt: »\noindent{}adr{ / }Fru Herdis Bergstrøm\pwindex{Bergstrøm, Herdis 1874-11-17 – 1962-02-19@\textsc{Bergstrøm, Herdis} (1874-11-17 – 1962-02-19), \emph{Dramatikerin}|pw}{ / }Dosseringen 304 {\\}Kobenhamn\oindex{Sortedam Dossering@\textbf{Sortedam Dossering}|pwu}« }\buchAbdrucke{\weitereDrucke{Arthur Schnitzler: \emph{Arthur Schnitzlers Briefe nach Dänemark}. Hg. Ernst-Ulrich Pinkert. Roskilde: \emph{Center for Østrigsk-Nordiske Kulturstudier} 2006, S. 18.} }\toendnotes[C]{\smallbreak}\pstart{}{\pb}\textcolor{gray}{\textbf{Dr. Arthur Schnitzler}}\pend{}\pstart{}\textcolor{gray}{\textbf{Wien XVIII. Sternwartestrasse 71\oindex{Sternwartestrasse@\textbf{Sternwartestraße}|pw}}}\pend{}{\bigskip}\pstart{}Frau Karin Michaelis\pend{}\pstart{}{[}\label{K_L02603-11v}\edtext{Thurø}{\lemma{\textnormal{\emph{Thurø}}}\Cendnote{\textnormal{Bei dieser Adresszeile handelt es sich um eine
                           Rekonstruktion, da der betreffende Teil auf der Karte abgeklebt ist. Da
                           sich aber der Stempel von Thurø\oindex{Thurø@\textbf{Thurø}|pwk} auf
                           der Karte findet und hier Karin Michaëlis einen Wohnsitz hatte, kann die
                           ursprüngliche Adressangabe, zumindest soweit es um die Ortsangabe geht,
                           erschlossen werden.}}}\label{K_L02603-11h}\oindex{Thurø@\textbf{Thurø}|pw}{]}\pend{}\pstart{}\damage{\textcolor{gray}{Dänemark\oindex{Daenemark@\textbf{Dänemark}|pw}}}\pend{}{\bigskip}\pstart
           \raggedleft{}{\pb}8. 11. 916\pend
           \pstart
           verehrte Frau Karin Michaelis – es freut mich sehr,
               daß Ihnen die Beate\pwindex{Schnitzler, Arthur 15.05.1862 – 21.10.1931@\textsc{Schnitzler, Arthur} (15.05.1862 – 21.10.1931), \emph{Schriftsteller, Mediziner}!Frau Beate und ihr Sohn. Novelle1.2.1913 – 1.4.1913@\strich\emph{Frau Beate und ihr Sohn. Novelle} {[}1.2.1913 – 1.4.1913{]}|pw} gefallen hat, eins meiner
               Werke, das vielfach und mit besondrer Vorliebe \label{K_L02603-1v}\edtext{misverstanden}{\lemma{\textnormal{\emph{misverstanden}}}\Cendnote{\textnormal{Kritisch begutachtet wurden vor allem die erotischen Inhalte, ganz besonders die
                  inzestuös deutbaren Momente in der Novelle \emph{Frau
                     Beate und ihr Sohn}\pwindex{Schnitzler, Arthur 15.05.1862 – 21.10.1931@\textsc{Schnitzler, Arthur} (15.05.1862 – 21.10.1931), \emph{Schriftsteller, Mediziner}!Frau Beate und ihr Sohn. Novelle1.2.1913 – 1.4.1913@\strich\emph{Frau Beate und ihr Sohn. Novelle} {[}1.2.1913 – 1.4.1913{]}|pwk} bzw. die »Unsittlichkeit« (vgl. A. S.: \emph{Tagebuch}, 14. 9. 1913) der Protagonistin
                  Beate.}}}\label{K_L02603-1h} wird. Der Schluss scheint ja (offenbar aus künstlerischen – nicht
               dramatischen – Gründen) – wie mir der Zweifel auch \label{K_L02603-2v}\edtext{Wohlwollender}{\lemma{\textnormal{\emph{Wohlwollender}}}\Cendnote{\textnormal{Am
                     24. 2. 1913 las Schnitzler\pwindex{Schnitzler, Arthur 15.05.1862 – 21.10.1931@\textsc{Schnitzler, Arthur} (15.05.1862 – 21.10.1931), \emph{Schriftsteller, Mediziner}|pwk}{ }\emph{Frau Beate und ihr Sohn}\pwindex{Schnitzler, Arthur 15.05.1862 – 21.10.1931@\textsc{Schnitzler, Arthur} (15.05.1862 – 21.10.1931), \emph{Schriftsteller, Mediziner}!Frau Beate und ihr Sohn. Novelle1.2.1913 – 1.4.1913@\strich\emph{Frau Beate und ihr Sohn. Novelle} {[}1.2.1913 – 1.4.1913{]}|pwk}{ }Richard Beer-Hofmann\pwindex{Beer-Hofmann, Richard 11.07.1866 – 26.09.1945@\textsc{Beer-Hofmann, Richard} (11.07.1866 – 26.09.1945), \emph{Schriftsteller}|pwk}, Hugo von Hofmannsthal\pwindex{Hofmannsthal, Hugo von 01.02.1874 – 15.07.1929@\textsc{Hofmannsthal, Hugo von} (01.02.1874 – 15.07.1929), \emph{Schriftsteller}|pwk}, Leo Van-Jung\pwindex{Van-Jung, Leo 15.10.1866 – 02.07.1939@\textsc{Van-Jung, Leo} (15.10.1866 – 02.07.1939), \emph{Gesangspädagoge, Mathematiker}|pwk}, Felix Salten\pwindex{Salten, Felix 06.09.1869 – 08.10.1945@\textsc{Salten, Felix} (06.09.1869 – 08.10.1945), \emph{Schriftsteller, Journalist}|pwk}, Jakob Wassermann\pwindex{Wassermann, Jakob 10.03.1873 – 01.01.1934@\textsc{Wassermann, Jakob} (10.03.1873 – 01.01.1934), \emph{Schriftsteller}|pwk}, Gustav Schwarzkopf\pwindex{Schwarzkopf, Gustav 07.11.1853 – 13.11.1939@\textsc{Schwarzkopf, Gustav} (07.11.1853 – 13.11.1939), \emph{Schriftsteller}|pwk} und seiner Frau Olga\pwindex{Schnitzler, Olga 17.01.1882 – 13.01.1970@\textsc{Schnitzler, Olga} (17.01.1882 – 13.01.1970), \emph{Schauspielerin, Sängerin}|pwk} vor und erntete vor allem für den Schluss Kritik. vgl. A. S.: \emph{Tagebuch}, 23. 2. 1913}}}\label{K_L02603-2h} zu bedenken gibt – nicht durchaus überzeugend zu sein. – Ich schreibe Ihnen
               meinen Dank und Gruß auf einer Karte – die nach meiner Erfahrung \label{K_L02603-3v}\edtext{sichrer ins neutrale Ausland\oindex{Daenemark@\textbf{Dänemark}|pwv} gelangt}{\lemma{\textnormal{\emph{sichrer … gelangt}}}\Cendnote{\textnormal{Postalisch versandte Korrespondenzstücke wurden von der \emph{K. u. k. Zensurstelle}\orgindex{K. u. k. Zensurstelle@K. u. k. Zensurstelle|pwk} gelesen, egal ob Brief
                  oder Postkarte. Bei letzterer wurde aber, da sie offen versandt wurde, eher davon
                  ausgegangen, dass auf ihr keine Geheimnisse stehen konnten.}}}\label{K_L02603-3h} als Briefe –
                  {\pb}auf die Gefahr
               hin, daß Sie mich für minder correct (aber gerade zu langweilig) halten wie früher. \pend
           \pstart
           {\pb}Auf Wiedersehen
               hoffentlich, und sch\textcolor{gray}{oene} Grüße, auch von meiner Frau\pwindex{Schnitzler, Olga 17.01.1882 – 13.01.1970@\textsc{Schnitzler, Olga} (17.01.1882 – 13.01.1970), \emph{Schauspielerin, Sängerin}|pwv}. Ihr sehr ergebner{\\[\baselineskip]}\spacefill\mbox{Arthur Schnitzler}\pend
           \leftskip=0em{}\endnumbering\briefempfaengerindex{Michaelis, Karin@\textsc{Michaëlis, Karin}!zzzSchnitzler, Arthur@\emph{von Arthur Schnitzler}!1912-10-211@{8. 11. 1916}|)be}\mylabel{h}\end{ledgroupsized}  \newcommand{\dateiname}{L02603}\newcommand{\titel}{Arthur Schnitzler an Karin Michaëlis, 8. 11. 1916}\newcommand{\editorInnen}{Martin Anton Müller und Laura Untner}\input{../tex-inputs/latex-pdf-abspann}
      