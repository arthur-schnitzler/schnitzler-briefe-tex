%% latex-korrekturansicht-vorspann.tex
%% Vorspann für die Korrekturansicht.
%% Lädt die gemeinsame Datei latex-vorspann.tex mit gesetztem Schalter.

\newif\ifkorrekturansicht
\korrekturansichttrue

\input{../tex-inputs/latex-vorspann}


\section[Arthur Schnitzler an Karin Michaëlis, 8. 11. 1916]{L02603 Arthur Schnitzler an Karin Michaëlis, 8. 11. 1916}
\nopagebreak\mylabel{L02603v}
\rehead{ }\normalsize\beginnumbering\briefempfaengerindex{Michaelis, Karin@\textsc{Michaëlis, Karin}!zzzSchnitzler, Arthur@\emph{von Arthur Schnitzler}!1916-11-081@{8. 11. 1916}|(be}
\toendnotes[C]{\smallbreak\pagebreak[2]}\Standort{Kopenhagen, Det Kongelige Bibliotek, Palsbo Ac.}
\physDesc{Postkarte, 712 Zeichen
\newline{}Handschrift: schwarze Tinte, lateinische Kurrent
\newline{}Versand: 1) Stempel: »\nobreak{}Wien, 8. 11. 16, 4\nobreak{}«.   2) Stempel: »\nobreak{}Zensuriert K. u. k.
                                          Zensurstelle\orgindex{K. u. k. Zensurstelle@K. u. k. Zensurstelle|pw}\nobreak{}«.  3) Stempel: »\nobreak{}\oindex{Svendborg@\textbf{Svendborg}, \emph{P.PPLA2}|pwk}Svendborg, 14. 11. 16, 7–9F\nobreak{}«.  4) Stempel: »\nobreak{}\oindex{Thurø@\textbf{Thurø}, \emph{Insel (N.INS)}|pwk}Thurø\nobreak{}«.  5) ursprüngliche Adressierung überklebt und von unbekannter Hand
                                 mit schwarzer Tinte neue Empfangsadresse vermerkt: »\noindent{}adr{ / }Fru Herdis Bergstrøm\pwindex{Bergstrøm, Herdis 1874-11-17 – 1962-02-19@\textsc{Bergstrøm, Herdis} (1874-11-17 – 1962-02-19), \emph{Dramatiker/Dramatikerin}|pw}{ / }Dosseringen 304 {\\}Kobenhamn\oindex{Sortedam Dossering@\textbf{Sortedam Dossering}, \emph{Straße (K.STR)}|pwu}« }
\buchAbdrucke{\weitereDrucke{Arthur Schnitzler: \emph{Arthur Schnitzlers Briefe nach Dänemark}. Roskilde: \emph{Center for Østrigsk-Nordiske Kulturstudier} 2006, S. 18.} }\toendnotes[C]{\smallbreak}\pstart{}{\pb}\textcolor{gray}{\textbf{Dr. Arthur Schnitzler}}\pend{}\pstart{}\textcolor{gray}{\textbf{Wien XVIII. Sternwartestrasse 71\oindex{Sternwartestrasse 71@\textbf{Sternwartestraße 71}, \emph{Wohngebäude (K.WHS)}|pw}}}\pend{}{\bigskip}\pstart{}Frau Karin Michaelis\pend{}\pstart{}{[}\label{K_L02603-1v}\edtext{Thurø}{\lemma{\textnormal{\emph{Thurø}}}\Cendnote{\textnormal{Bei dieser Adresszeile handelt es sich um eine
                           Rekonstruktion, da der betreffende Teil auf der Karte abgeklebt ist. Da
                           sich aber der Stempel von Thurø\oindex{Thurø@\textbf{Thurø}, \emph{Insel (N.INS)}|pwk} auf
                           der Karte findet und hier Karin Michaëlis einen Wohnsitz hatte, kann die
                           ursprüngliche Adressangabe, zumindest soweit es um die Ortsangabe geht,
                           erschlossen werden.}}}\label{K_L02603-1}\oindex{Thurø@\textbf{Thurø}, \emph{Insel (N.INS)}|pw}{]}\pend{}\pstart{}\damage{\textcolor{gray}{Dänemark\oindex{Daenemark@\textbf{Dänemark}, \emph{A.PCLI}|pw}}}\pend{}{\bigskip}\vspace{1em}
\pstart
           \raggedleft{}{\pb}8. 11. 916\pend
           \vspace{0.5em}
\pstart
           verehrte Frau Karin Michaelis – es freut mich sehr,
               daß Ihnen die Beate\pwindex{Frau Beate und ihr Sohn. Novelle@\emph{Frau Beate und ihr Sohn. Novelle}|pw} gefallen hat, eins meiner
               Werke, das vielfach und mit besondrer Vorliebe \label{K_L02603-2v}\edtext{misverstanden}{\lemma{\textnormal{\emph{misverstanden}}}\Cendnote{\textnormal{Kritisch begutachtet wurden vor allem die erotischen Inhalte, ganz besonders die
                  inzestuös deutbaren Momente in der Novelle \emph{Frau
                     Beate und ihr Sohn}\pwindex{Frau Beate und ihr Sohn. Novelle@\emph{Frau Beate und ihr Sohn. Novelle}|pwk} bzw. die »Unsittlichkeit« (vgl. A. S.: \emph{Tagebuch}, 14. 9. 1913) der
                  Protagonistin Beate.}}}\label{K_L02603-2} wird. Der Schluss scheint ja (offenbar aus
               künstlerischen – nicht dramatischen – Gründen) – wie mir der Zweifel auch \label{K_L02603-3v}\edtext{Wohlwollender}{\lemma{\textnormal{\emph{Wohlwollender}}}\Cendnote{\textnormal{Am 24. 2. 1913 las Schnitzler{ }\emph{Frau Beate und ihr Sohn}\pwindex{Frau Beate und ihr Sohn. Novelle@\emph{Frau Beate und ihr Sohn. Novelle}|pwk}{ }Richard Beer-Hofmann\pwindex{Beer-Hofmann, Richard 1866-07-11 – 1945-09-26@\textsc{Beer-Hofmann, Richard} (1866-07-11 – 1945-09-26), \emph{Schriftsteller/Schriftstellerin}|pwk}, Hugo von Hofmannsthal\pwindex{Hofmannsthal, Hugo von 1874-02-01 – 1929-07-15@\textsc{Hofmannsthal, Hugo von} (1874-02-01 – 1929-07-15), \emph{Schriftsteller/Schriftstellerin}|pwk}, Leo Van-Jung\pwindex{Van-Jung, Leo 15.10.1866 – 02.07.1939@\textsc{Van-Jung, Leo} (15.10.1866 – 02.07.1939), \emph{Gesangspädagoge/Gesangspädagogin, Mathematiker/Mathematikerin}|pwk}, Felix Salten\pwindex{Salten, Felix 06.09.1869 – 08.10.1945@\textsc{Salten, Felix} (06.09.1869 – 08.10.1945), \emph{Schriftsteller/Schriftstellerin, Journalist/Journalistin, Chefredakteur/Chefredakteurin}|pwk}, Jakob Wassermann\pwindex{Wassermann, Jakob 10.03.1873 – 01.01.1934@\textsc{Wassermann, Jakob} (10.03.1873 – 01.01.1934), \emph{Schriftsteller/Schriftstellerin}|pwk}, Gustav Schwarzkopf\pwindex{Schwarzkopf, Gustav 07.11.1853 – 13.11.1939@\textsc{Schwarzkopf, Gustav} (07.11.1853 – 13.11.1939), \emph{Schriftsteller/Schriftstellerin}|pwk} und seiner Frau Olga\pwindex{Schnitzler, Olga 17.01.1882 – 13.01.1970@\textsc{Schnitzler, Olga} (17.01.1882 – 13.01.1970), \emph{Schauspieler/Schauspielerin, Sänger/Sängerin}|pwk} vor. Vor allem für den Schluss wurde er kritisiert. Vgl. A. S.: \emph{Tagebuch}, 23. 2. 1913.
               }}}\label{K_L02603-3} zu bedenken gibt – nicht durchaus überzeugend zu sein. – Ich schreibe Ihnen
               meinen Dank und Gruß auf einer Karte – die nach meiner Erfahrung \label{K_L02603-4v}\edtext{sichrer ins neutrale Ausland\oindex{Daenemark@\textbf{Dänemark}, \emph{A.PCLI}|pwv} gelangt}{\lemma{\textnormal{\emph{sichrer … gelangt}}}\Cendnote{\textnormal{Postalisch versandte Korrespondenzstücke wurden von der \emph{K. u. k. Zensurstelle}\orgindex{K. u. k. Zensurstelle@K. u. k. Zensurstelle|pwk} gelesen, egal ob Brief
                  oder Postkarte. Bei Letzterer wurde aber, da sie offen versandt wurde, eher davon
                  ausgegangen, dass auf ihr keine Geheimnisse stehen konnten.}}}\label{K_L02603-4} als Briefe –
                  {\pb}auf die Gefahr
               hin, daß Sie mich für minder correct (aber gerade zu langweilig) halten wie früher. \pend
           
\pstart
           {\pb}Auf Wiedersehen
               hoffentlich, und sch\textcolor{gray}{oene} Grüße, auch von meiner Frau\pwindex{Schnitzler, Olga 17.01.1882 – 13.01.1970@\textsc{Schnitzler, Olga} (17.01.1882 – 13.01.1970), \emph{Schauspieler/Schauspielerin, Sänger/Sängerin}|pwv}. Ihr sehr ergebner{\\[\baselineskip]}\spacefill\mbox{Arthur Schnitzler}\pend
           \leftskip=0em{}\selectlanguage{ngerman}\endnumbering\briefempfaengerindex{Michaelis, Karin@\textsc{Michaëlis, Karin}!zzzSchnitzler, Arthur@\emph{von Arthur Schnitzler}!1916-11-081@{8. 11. 1916}|)be}\mylabel{L02603h}  \normalsize

\doendnotes{C}
\bigskip
\vfill

\clearpage

\footnotesize

\lohead{\textsc{register}}

% Definiere theindex-Environment komplett neu ohne reledmac
\makeatletter
\renewenvironment{theindex}{%
  \section*{\indexname}%
  \setlength{\parindent}{0pt}%
  \setlength{\parskip}{0pt plus 0.3pt}%
  \let\item\@idxitem
}{%
  \clearpage
}
\makeatother

\IfFileExists{\jobname-pw.ind}{\input{\jobname-pw.ind}}{}

\end{document}

      