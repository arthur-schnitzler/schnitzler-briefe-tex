%% latex-korrekturansicht-vorspann.tex
%% Vorspann für die Korrekturansicht.
%% Lädt die gemeinsame Datei latex-vorspann.tex mit gesetztem Schalter.

\newif\ifkorrekturansicht
\korrekturansichttrue

\input{../tex-inputs/latex-vorspann}


\section[Richard Beer-Hofmann an Arthur Schnitzler, {[}15.?{]} 5. 1906]{L01597 Richard Beer-Hofmann an Arthur Schnitzler, {[}15.?{]} 5. 1906}
\nopagebreak\mylabel{L01597v}
\rehead{ }\normalsize\beginnumbering\briefempfaengerindex{Schnitzler, Arthur@\textsc{Schnitzler, Arthur}!zzzBeer-Hofmann, Richard@\emph{von Richard Beer-Hofmann}!1906-05-151@{{[}15.?{]} 5. 1906}|(be}
\toendnotes[C]{\smallbreak\pagebreak[2]}\Standort{CUL, Schnitzler, B 8.}
\physDesc{Brief, 1 Blatt, 1 Seite, 669 Zeichen
\newline{}Handschrift: schwarze Tinte, lateinische Kurrent
\newline{}Ordnung: mit Bleistift von unbekannter Hand nummeriert: »205a« }
\buchAbdrucke{\weitereDrucke{Arthur Schnitzler, Richard Beer-Hofmann: \emph{Briefwechsel 1891–1931}. Wien, Zürich: \emph{Europaverlag} 1992, S. 178.} }\toendnotes[C]{\smallbreak}
\pstart
           \noindent{}\centering{}{\pb}\uline{»Der
                     einsame Weg\pwindex{einsame Weg@\emph{Der einsame Weg}|pw}\pwindex{einsame Weg. Schauspiel in fuenf Akten@\emph{Der einsame Weg. Schauspiel in fünf Akten}|pwv}«}\pend
           
\pstart
           \raggedleft{}\uline{An Arthur Schnitzler}\pend
           \stanza{}Alle Wege die wir tretenMünden in die Einsamkeit,Nimmermüde Stunden jätenAus, was wuchs, an Lust und Leid.Alles Glück, und alles ElendBlasst zu fernem Wi\strikeout{e}derschein,Was beseeligend, was quälend,Geht – lässt uns, mit uns allein.Schritt ich eben nicht im Reigen\pwindex{Reigen. Zehn Dialoge@\emph{Reigen. Zehn Dialoge}|pwv}?Und was traf, das traf gemeinsam!Bietet keine Hand sich? – SchweigenSieht mich an – der Weg wird einsam.Ob ich stieg von Glückesthronen,Ob ich klomm aus Leidensgründen –Dort, wohin ich geh zu wohnen,Wird sich keines zu mir finden!Ein Erkennen nur, mit klaarenAugen, will mich hingeleiten:Dass, auch vorher, um mich waren,– Unerkannt – nur Einsamkeiten!\stanzaend{}\pstart \spacefill\mbox{R. B-H.}\pend{}
\pstart
           Rodaun\oindex{Rodaun@\textbf{Rodaun}, \emph{A.ADM4}|pw}, \label{K_L01597-1v}\edtext{Mai 1906}{\lemma{\textnormal{\emph{Mai 1906}}}\Cendnote{\textnormal{Am 15. 5. 1906 feierte
                        Schnitzler seinen 44. Geburtstag. Am
                     selben Tag wurde \emph{Der einsame
                        Weg}\pwindex{einsame Weg. Schauspiel in fuenf Akten@\emph{Der einsame Weg. Schauspiel in fünf Akten}|pwk} im Zuge eines Gastspiels\pwindex{einsame Weg. Schauspiel in fuenf Akten@\emph{Der einsame Weg. Schauspiel in fünf Akten}|pwkv} des \emph{Lessing-Theaters}\orgindex{Lessing-Theater@Lessing-Theater|pwk} am
                     Theater an der Wien\oindex{Theater an der Wien@\textbf{Theater an der Wien}, \emph{Theater (K.THE)}|pwk} gegeben.}}}\label{K_L01597-1}\pend
           \selectlanguage{ngerman}\endnumbering\briefempfaengerindex{Schnitzler, Arthur@\textsc{Schnitzler, Arthur}!zzzBeer-Hofmann, Richard@\emph{von Richard Beer-Hofmann}!1906-05-151@{{[}15.?{]} 5. 1906}|)be}\mylabel{L01597h}  \normalsize

\doendnotes{C}
\bigskip
\vfill

\clearpage

\footnotesize

\lohead{\textsc{register}}

% Definiere theindex-Environment komplett neu ohne reledmac
\makeatletter
\renewenvironment{theindex}{%
  \section*{\indexname}%
  \setlength{\parindent}{0pt}%
  \setlength{\parskip}{0pt plus 0.3pt}%
  \let\item\@idxitem
}{%
  \clearpage
}
\makeatother

\IfFileExists{\jobname-pw.ind}{\input{\jobname-pw.ind}}{}

\end{document}

      