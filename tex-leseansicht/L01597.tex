%% latex-leseansicht-vorspann.tex
%% Vorspann für die Leseansicht.
%% Lädt die gemeinsame Datei latex-vorspann.tex mit nicht gesetztem Schalter.

\newif\ifkorrekturansicht
\korrekturansichtfalse

\input{../tex-inputs/latex-vorspann}


         \renewcommand{\erwaehnteOrte}{Orte: Rodaun, Wien}
         \renewcommand{\erwaehnteWerke}{Werke: Der einsame Weg. Schauspiel in fünf Akten, Reigen. Zehn Dialoge}
               \section[Richard Beer-Hofmann an Arthur Schnitzler, {[}zum 15.?{]} 5. 1906]{ Richard Beer-Hofmann an Arthur Schnitzler, {[}zum 15.?{]} 5. 1906}\nopagebreak\mylabel{v}\rehead{ }\begin{ledgroupsized}[t]{13cm}\normalsize\beginnumbering \toendnotes[C]{\smallbreak\pagebreak[2]} \Standort{CUL, Schnitzler, B 8.}
\physDesc{Brief, 1 Blatt, 1 Seite, 669 Zeichen
\newline{}Handschrift: schwarze Tinte, lateinische Kurrent
\newline{}Ordnung: mit Bleistift von unbekannter Hand nummeriert:
                                    »205a« }\buchAbdrucke{\weitereDrucke{Arthur Schnitzler, Richard Beer-Hofmann: \emph{Briefwechsel 1891–1931}. Hg. Konstanze Fliedl. Wien, Zürich: \emph{Europaverlag} 1992, S. 178.} }\toendnotes[C]{\smallbreak}\pstart
           \noindent{}\centering{}{\pb}\uline{»Der
                     einsame Weg\pwindex{Schnitzler, Arthur 15.05.1862 – 21.10.1931@\textsc{Schnitzler, Arthur} (15.05.1862 – 21.10.1931), \emph{Schriftsteller, Mediziner}!einsame Weg. Schauspiel in fuenf Akten1904@\strich\emph{Der einsame Weg. Schauspiel in fünf Akten} {[}1904{]}|pwv}«}\pend
           \pstart
           \noindent{}\raggedleft{}\uline{An Arthur Schnitzler}\pend
           \stanza{}Alle Wege die wir treten\newverse{}Münden in die Einsamkeit,\newverse{}Nimmermüde Stunden jäten\newverse{}Aus, was wuchs, an Lust und Leid.\newverse{}\newverse{}Alles Glück, und alles Elend\newverse{}Blasst zu fernem Wi\strikeout{e}derschein,\newverse{}Was beseeligend, was quälend,\newverse{}Geht – lässt uns, mit uns allein.\newverse{}\newverse{}Schritt ich eben nicht im Reigen\pwindex{Schnitzler, Arthur 15.05.1862 – 21.10.1931@\textsc{Schnitzler, Arthur} (15.05.1862 – 21.10.1931), \emph{Schriftsteller, Mediziner}!Reigen. Zehn Dialoge1900@\strich\emph{Reigen. Zehn Dialoge} {[}1900{]}|pwv}?\newverse{}Und was traf, das traf gemeinsam!\newverse{}Bietet keine Hand sich? – Schweigen\newverse{}Sieht mich an – der Weg wird einsam.\newverse{}\newverse{}Ob ich stieg von Glückesthronen,\newverse{}Ob ich klomm aus Leidensgründen –\newverse{}Dort, wohin ich geh zu wohnen,\newverse{}Wird sich keines zu mir finden!\newverse{}\newverse{}Ein Erkennen nur, mit klaaren\newverse{}Augen, will mich hingeleiten:\newverse{}Dass, auch vorher, um mich waren,\newverse{}– Unerkannt – nur Einsamkeiten!\stanzaend{}\pstart \spacefill\mbox{R. B-H.}\pend{}\pstart
           Rodaun\oindex{Rodaun@\textbf{Rodaun}|pw}, \label{K_L01597_1v}\edtext{Mai 1906}{\lemma{\textnormal{\emph{Mai 1906}}}\Cendnote{\textnormal{Am 15. 5. 1906 feierte
                        Schnitzler\pwindex{Schnitzler, Arthur 15.05.1862 – 21.10.1931@\textsc{Schnitzler, Arthur} (15.05.1862 – 21.10.1931), \emph{Schriftsteller, Mediziner}|pwk} seinen
                     44. Geburtstag.}}}\label{K_L01597_1h}\pend
           
         
         \endnumbering\mylabel{h}\end{ledgroupsized}  \newcommand{\dateiname}{L01597}\newcommand{\titel}{Richard Beer-Hofmann an Arthur Schnitzler, [zum 15.?] 5. 1906}\newcommand{\editorInnen}{Martin Anton Müller und Gerd-Hermann Susen}%% latex-leseansicht-abspann.tex
%% Abspann für die Leseansicht.
%% Der Schalter \ifkorrekturansicht ist bereits durch den Vorspann gesetzt.

%% latex-abspann.tex
%% Gemeinsamer Abspann für Korrekturansicht und Leseansicht.
%% Setzt den Schalter \ifkorrekturansicht voraus (gesetzt in den
%% einbindenden Dateien latex-korrekturansicht-abspann.tex bzw.
%% latex-leseansicht-abspann.tex).
%% ---------------------------------------------------------------

\normalsize

% Das esempio-Environment wird nur in der Leseansicht benötigt
\ifkorrekturansicht\else
\newenvironment{esempio}[3]%
{
    \vspace{1.5ex}
    \rlap{\underline{#1}}
    \par
    \setlength{\parindent}{0cm}
    \nopagebreak
    \leftskip=#2cm
    \rightskip=#3cm
}
{
    \par
}
\fi

\doendnotes{C}
\bigskip
\vfill

\clearpage

\footnotesize

\ifkorrekturansicht
  \lohead{\textsc{register}}
\fi

% theindex-Environment neu definieren ohne reledmac
\makeatletter
\renewenvironment{theindex}{%
  \ifkorrekturansicht
    \section*{\indexname}%
  \else
    \subsubsection*{Index der erwähnten Entitäten}%
  \fi
  \setlength{\parindent}{0pt}%
  \setlength{\parskip}{0pt plus 0.3pt}%
  \let\item\@idxitem
}{%
  \ifkorrekturansicht\clearpage\fi
}
\makeatother

\IfFileExists{\jobname-pw.ind}{\input{\jobname-pw.ind}}{}

% Quellenangabe nur in der Leseansicht
\ifkorrekturansicht\else
% Fallback-Definitionen, falls die .tex-Datei \titel etc. nicht gesetzt hat
\providecommand{\titel}{}
\providecommand{\editorInnen}{}
\providecommand{\dateiname}{\jobname}

\vspace{3cm}

\vfill

\footnotesize
\textsc{Quelle}: \titel. Herausgegeben von {\editorInnen}. In: \emph{Arthur Schnitzler: Briefwechsel mit Autorinnen und Autoren}.
 Digitale Edition, https://schnitzler-briefe.acdh.oeaw.ac.at/{\dateiname}.html (Stand \today)
\fi

\end{document}


      