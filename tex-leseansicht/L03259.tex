%% latex-leseansicht-vorspann.tex
%% Vorspann für die Leseansicht.
%% Lädt die gemeinsame Datei latex-vorspann.tex mit nicht gesetztem Schalter.

\newif\ifkorrekturansicht
\korrekturansichtfalse

\input{../tex-inputs/latex-vorspann}


\section[ Paul Goldmann an Arthur Schnitzler, 19. 9. 1907]{L03259 Paul Goldmann an Arthur Schnitzler,  19. 9. 1907}
\nopagebreak\mylabel{L03259v}
\rehead{ }\normalsize\beginnumbering\briefempfaengerindex{Schnitzler, Arthur@\textsc{Schnitzler, Arthur}!zzzGoldmann, Paul@\emph{von Paul Goldmann}!1907-09-191@{19. 9. 1907}|(be}
\toendnotes[C]{\smallbreak\pagebreak[2]}
\correspDesc{Versand  durch Paul Goldmann am 19. 9. 1907 in Venedig
\newline{}Erhalt  durch Arthur Schnitzler im Zeitraum [20. 9. 1907
                  – 24. 9. 1907?] in Wien}\toendnotes[C]{\smallbreak}
\Standort{DLA, A:Schnitzler, HS.NZ85.1.3175.}
\physDesc{Bildpostkarte, 230 Zeichen
\newline{}Handschrift: schwarze Tinte, deutsche Kurrent
\newline{}Versand: Stempel: »\nobreak{}\oindex{Venedig@\textbf{Venedig}|pwk}Venezia, 19 {[}9 07{]}, 2S\nobreak{}«.  }\toendnotes[C]{\smallbreak}\pstart{}\textsc{{\pb}Herrn}\pend{}\pstart{}\textsc{Dr. Arthur Schnitzler}\pend{}\pstart{}\textsc{Wien\oindex{Wien@\textbf{Wien}, \emph{Verwaltungsgebiet}|pw}}\pend{}\pstart{}\textsc{XVIII. Spöttelgaſse 7\oindex{Wien@\textbf{Wien}!XVIII., Währing@\textbf{XVIII., Währing}!Edmund-Weiß-Gasse 7@\textbf{Edmund-Weiß-Gasse 7}, \emph{Wohngebäude}|pw}}\pend{}\pstart{}\textsc{\begin{otherlanguage}{italian}Austria\oindex{Österreich-Ungarn@\textbf{Österreich-Ungarn}|pw}\end{otherlanguage}.}\pend{}{\bigskip}
\pstart
           \noindent{}\centering{}{\pb}{[}Venezianische\oindex{Venedig@\textbf{Venedig}|pw} Flagge{]}\pend
           \vspace{1em}
\pstart
           19. 9. 07.\pend
           \vspace{0.5em}
\pstart
           {\pb}Herzlichſte Grüße Dir u. Deiner Frau\pwindex{Schnitzler, Olga 17.\,1.\,1882 Wien – 13.\,1.\,1970 Lugano@\textsc{Schnitzler, Olga} (17.\,1.\,1882 Wien – 13.\,1.\,1970 Lugano), \emph{Schauspielerin, Sängerin}|pwv}. Auf dem Rückweg komme ich leider nicht über Wien\oindex{Wien@\textbf{Wien}, \emph{Verwaltungsgebiet}|pw}. Hoffentlich \label{K_L03259-1v}\edtext{ſehen wir uns bald}{\lemma{\textnormal{\emph{sehen wir uns bald}}}\Cendnote{\textnormal{Schnitzler und Goldmann\pwindex{Goldmann, Paul 31.\,1.\,1865 Breslau – 25.\,9.\,1935 Wien@\textsc{Goldmann, Paul} (31.\,1.\,1865 Breslau – 25.\,9.\,1935 Wien), \emph{Schriftsteller, Journalist}|pwk} sahen sich am 8. 10. 1907 in Wien\oindex{Wien@\textbf{Wien}, \emph{Verwaltungsgebiet}|pwk} wieder.}}}\label{K_L03259-1} in Berlin\oindex{Berlin@\textbf{Berlin}, \emph{Hauptstadt}|pw}.\pend
           
\pstart
           Dein{\\[\baselineskip]}\spacefill\mbox{Paul Goldmann.}\pend
           \leftskip=0em{}\selectlanguage{ngerman}\endnumbering\briefempfaengerindex{Schnitzler, Arthur@\textsc{Schnitzler, Arthur}!zzzGoldmann, Paul@\emph{von Paul Goldmann}!1907-09-191@{19. 9. 1907}|)be}\mylabel{L03259h}  \newcommand{\dateiname}{L03259}\newcommand{\titel}{Paul Goldmann an Arthur Schnitzler, 19. 9. 1907}\newcommand{\editorInnen}{Martin Anton Müller und Laura Untner}%% latex-leseansicht-abspann.tex
%% Abspann für die Leseansicht.
%% Der Schalter \ifkorrekturansicht ist bereits durch den Vorspann gesetzt.

%% latex-abspann.tex
%% Gemeinsamer Abspann für Korrekturansicht und Leseansicht.
%% Setzt den Schalter \ifkorrekturansicht voraus (gesetzt in den
%% einbindenden Dateien latex-korrekturansicht-abspann.tex bzw.
%% latex-leseansicht-abspann.tex).
%% ---------------------------------------------------------------

\normalsize

% Das esempio-Environment wird nur in der Leseansicht benötigt
\ifkorrekturansicht\else
\newenvironment{esempio}[3]%
{
    \vspace{1.5ex}
    \rlap{\underline{#1}}
    \par
    \setlength{\parindent}{0cm}
    \nopagebreak
    \leftskip=#2cm
    \rightskip=#3cm
}
{
    \par
}
\fi

\doendnotes{C}
\bigskip
\vfill

\clearpage

\footnotesize

\ifkorrekturansicht
  \lohead{\textsc{register}}
\fi

% theindex-Environment neu definieren ohne reledmac
\makeatletter
\renewenvironment{theindex}{%
  \ifkorrekturansicht
    \section*{\indexname}%
  \else
    \subsubsection*{Index der erwähnten Entitäten}%
  \fi
  \setlength{\parindent}{0pt}%
  \setlength{\parskip}{0pt plus 0.3pt}%
  \let\item\@idxitem
}{%
  \ifkorrekturansicht\clearpage\fi
}
\makeatother

\IfFileExists{\jobname-pw.ind}{\input{\jobname-pw.ind}}{}

% Quellenangabe nur in der Leseansicht
\ifkorrekturansicht\else
% Fallback-Definitionen, falls die .tex-Datei \titel etc. nicht gesetzt hat
\providecommand{\titel}{}
\providecommand{\editorInnen}{}
\providecommand{\dateiname}{\jobname}

\vspace{3cm}

\vfill

\footnotesize
\textsc{Quelle}: \titel. Herausgegeben von {\editorInnen}. In: \emph{Arthur Schnitzler: Briefwechsel mit Autorinnen und Autoren}.
 Digitale Edition, https://schnitzler-briefe.acdh.oeaw.ac.at/{\dateiname}.html (Stand \today)
\fi

\end{document}


