%% latex-leseansicht-vorspann.tex
%% Vorspann für die Leseansicht.
%% Lädt die gemeinsame Datei latex-vorspann.tex mit nicht gesetztem Schalter.

\newif\ifkorrekturansicht
\korrekturansichtfalse

\input{../tex-inputs/latex-vorspann}


\section[Stefan Zweig an Arthur Schnitzler, 16. 1. 1915]{L03650 Stefan Zweig an Arthur Schnitzler, 16. 1. 1915}
\nopagebreak\mylabel{L03650v}
\rehead{ }\normalsize\beginnumbering\briefempfaengerindex{Schnitzler, Arthur@\textsc{Schnitzler, Arthur}!zzzZweig, Stefan@\emph{von Stefan Zweig}!1915-01-161@{16. 1. 1915}|(be}
\toendnotes[C]{\smallbreak\pagebreak[2]}
\correspDesc{Versand  durch Stefan Zweig am 16. 1. 1915 in Wien
\newline{}Erhalt  durch Arthur Schnitzler im Zeitraum [16. 1. 1915
                  – 19. 1. 1915?] in Wien}\toendnotes[C]{\smallbreak}
\Standort{CUL, Schnitzler, B 118.}
\physDesc{Brief, 1 Blatt, 1 Seite, 557 Zeichen
\newline{}Handschrift: schwarze Tinte, lateinische Kurrent
\newline{}Schnitzler: mit rotem Buntstift eine Unterstreichung }
\buchAbdrucke{\weitereDrucke{1) Stefan Zweig: \emph{Briefwechsel mit Hermann Bahr, Sigmund Freud, Rainer Maria
                        Rilke und Arthur Schnitzler}. Herausgegeben von Jeffrey B. Berlin, Hans-Ulrich Lindken und Donald A. Prater. Frankfurt am Main: \emph{S. Fischer} 1987, S. 390.} \weitereDrucke{2) Stefan Zweig: \emph{Briefe. Bd. II: 1914–1919}. Herausgegeben von Knut Beck, Jeffrey B. Berlin und Natascha Weschenbach-Feggeler. Frankfurt am Main: \emph{S. Fischer} 1998, S. 50.} }\toendnotes[C]{\smallbreak}
\pstart
           {\pb}Wien\oindex{Wien@\textbf{Wien}, \emph{Verwaltungsgebiet}|pw}{ }16. Januar 1915\pend
           \vspace{0.5em}
\pstart
           Lieber verehrter Herr Doktor, den Ausschnitt\pwindex{Schnitzler, Arthur 15.\,5.\,1862 Wien – 21.\,10.\,1931 ebd.@\textsc{Schnitzler, Arthur} (15.\,5.\,1862 Wien – 21.\,10.\,1931 ebd.), \emph{Schriftsteller, Mediziner}!Une protestation d’Arthur Schnitzler@\strich\emph{Une protestation d’Arthur Schnitzler}|pwv} aus dem »Journal
                  de Genève\pwindex{Journal de Genève@\emph{Journal de Genève}|pw}« sandte ich Ihnen schon vor paar Tagen \label{K_L03650-1v}\edtext{durch Stringa\pwindex{Stringa, Alberto 12.\,1.\,1880 Caprino Veronese – 9.\,11.\,1931 ebd.@\textsc{Stringa, Alberto} (12.\,1.\,1880 Caprino Veronese – 9.\,11.\,1931 ebd.), \emph{Maler}|pw}}{\lemma{\textnormal{\emph{durch Stringa}}}\Cendnote{\textnormal{Alberto Stringa\pwindex{Stringa, Alberto 12.\,1.\,1880 Caprino Veronese – 9.\,11.\,1931 ebd.@\textsc{Stringa, Alberto} (12.\,1.\,1880 Caprino Veronese – 9.\,11.\,1931 ebd.), \emph{Maler}|pwk}
                     überbrachte den Ausschnitt\pwindex{Schnitzler, Arthur 15.\,5.\,1862 Wien – 21.\,10.\,1931 ebd.@\textsc{Schnitzler, Arthur} (15.\,5.\,1862 Wien – 21.\,10.\,1931 ebd.), \emph{Schriftsteller, Mediziner}!Une protestation d’Arthur Schnitzler@\strich\emph{Une protestation d’Arthur Schnitzler}|pwkv} erst am 17. 1. 1915.}}}\label{K_L03650-1}. Von Romain
                  Rolland\pwindex{Rolland, Romain 29.\,1.\,1866 Clamecy – 30.\,12.\,1944 Vézelay@\textsc{Rolland, Romain} (29.\,1.\,1866 Clamecy – 30.\,12.\,1944 Vézelay), \emph{Schriftsteller}|pw} habe ich plötzlich \label{K_L03650-2v}\edtext{keine Briefe mehr}{\lemma{\textnormal{\emph{keine Briefe mehr}}}\Cendnote{\textnormal{In
                     der Briefedition Rolland\pwindex{Rolland, Romain 29.\,1.\,1866 Clamecy – 30.\,12.\,1944 Vézelay@\textsc{Rolland, Romain} (29.\,1.\,1866 Clamecy – 30.\,12.\,1944 Vézelay), \emph{Schriftsteller}|pwk}–Zweig\pwindex{Zweig, Stefan 28.\,11.\,1881 Wien – 23.\,2.\,1942 Petrópolis@\textsc{Zweig, Stefan} (28.\,11.\,1881 Wien – 23.\,2.\,1942 Petrópolis), \emph{Schriftsteller}|pwk} sind folgende Briefe von Rolland\pwindex{Rolland, Romain 29.\,1.\,1866 Clamecy – 30.\,12.\,1944 Vézelay@\textsc{Rolland, Romain} (29.\,1.\,1866 Clamecy – 30.\,12.\,1944 Vézelay), \emph{Schriftsteller}|pwk}
                    aus dem Zeitraum abgedruckt: 22. 12. 1914, 11. 1. 1915, 5. 2. 1915.
                     Am 11. 1. 1915 schrieb er: »In den letzten vierzehn Tagen haben wir Ihnen drei Briefe geschrieben: sie kamen zu uns zurück.« (Romain Rolland\pwindex{Rolland, Romain 29.\,1.\,1866 Clamecy – 30.\,12.\,1944 Vézelay@\textsc{Rolland, Romain} (29.\,1.\,1866 Clamecy – 30.\,12.\,1944 Vézelay), \emph{Schriftsteller}|pwk}, Stefan Zweig\pwindex{Zweig, Stefan 28.\,11.\,1881 Wien – 23.\,2.\,1942 Petrópolis@\textsc{Zweig, Stefan} (28.\,11.\,1881 Wien – 23.\,2.\,1942 Petrópolis), \emph{Schriftsteller}|pwk}: \emph{Von Welt zu Welt. Briefe
                           einer Freundschaft 1914–1918}. Mit einem Begleitwort von Peter
                        Handke. Aus dem Französischen von Eva und Gerhard Schwewe (Briefe Rollands) und
                        Christel Gersch (Briefe Zweigs). Berlin: \emph{Aufbau
                           Verlag}{ }2014.) Zweig\pwindex{Zweig, Stefan 28.\,11.\,1881 Wien – 23.\,2.\,1942 Petrópolis@\textsc{Zweig, Stefan} (28.\,11.\,1881 Wien – 23.\,2.\,1942 Petrópolis), \emph{Schriftsteller}|pwk} sandte seinen nächsten Brief, datiert mit 17. 1. 1915, indem er ihn dem erwähnten
                     Stringa\pwindex{Stringa, Alberto 12.\,1.\,1880 Caprino Veronese – 9.\,11.\,1931 ebd.@\textsc{Stringa, Alberto} (12.\,1.\,1880 Caprino Veronese – 9.\,11.\,1931 ebd.), \emph{Maler}|pwk} nach  Italien\oindex{Italien@\textbf{Italien}|pwk} mitgab, um 
                     so die Briefzensur zu umgehen.}}}\label{K_L03650-2}, die Censur hat anscheinend
               unsere – doch zweifellos staatsgefährliche und an den Fundamenten Österreichs\oindex{Österreich@\textbf{Österreich}|pw} rüttelnde — Correspondenz unterbunden und
               abgedrosselt. Ich schreibe ihm über Italien\oindex{Italien@\textbf{Italien}|pw} und
               wende mich übrigens heute noch an die Briefcensur direct, um ihr den Begriff Romain Rolland\pwindex{Rolland, Romain 29.\,1.\,1866 Clamecy – 30.\,12.\,1944 Vézelay@\textsc{Rolland, Romain} (29.\,1.\,1866 Clamecy – 30.\,12.\,1944 Vézelay), \emph{Schriftsteller}|pw} aufzuklären. Hoffentlich
               ge­lingts! Viele viele Grüsse Ihres getreuen\pend
           \pstart \spacefill\mbox{Stefan Zweig}\pend{}\selectlanguage{ngerman}\endnumbering\briefempfaengerindex{Schnitzler, Arthur@\textsc{Schnitzler, Arthur}!zzzZweig, Stefan@\emph{von Stefan Zweig}!1915-01-161@{16. 1. 1915}|)be}\mylabel{L03650h}  \newcommand{\dateiname}{L03650}\newcommand{\titel}{Stefan Zweig an Arthur Schnitzler, 16. 1. 1915}\newcommand{\editorInnen}{Selma Jahnke und Martin Anton Müller}%% latex-leseansicht-abspann.tex
%% Abspann für die Leseansicht.
%% Der Schalter \ifkorrekturansicht ist bereits durch den Vorspann gesetzt.

%% latex-abspann.tex
%% Gemeinsamer Abspann für Korrekturansicht und Leseansicht.
%% Setzt den Schalter \ifkorrekturansicht voraus (gesetzt in den
%% einbindenden Dateien latex-korrekturansicht-abspann.tex bzw.
%% latex-leseansicht-abspann.tex).
%% ---------------------------------------------------------------

\normalsize

% Das esempio-Environment wird nur in der Leseansicht benötigt
\ifkorrekturansicht\else
\newenvironment{esempio}[3]%
{
    \vspace{1.5ex}
    \rlap{\underline{#1}}
    \par
    \setlength{\parindent}{0cm}
    \nopagebreak
    \leftskip=#2cm
    \rightskip=#3cm
}
{
    \par
}
\fi

\doendnotes{C}
\bigskip
\vfill

\clearpage

\footnotesize

\ifkorrekturansicht
  \lohead{\textsc{register}}
\fi

% theindex-Environment neu definieren ohne reledmac
\makeatletter
\renewenvironment{theindex}{%
  \ifkorrekturansicht
    \section*{\indexname}%
  \else
    \subsubsection*{Index der erwähnten Entitäten}%
  \fi
  \setlength{\parindent}{0pt}%
  \setlength{\parskip}{0pt plus 0.3pt}%
  \let\item\@idxitem
}{%
  \ifkorrekturansicht\clearpage\fi
}
\makeatother

\IfFileExists{\jobname-pw.ind}{\input{\jobname-pw.ind}}{}

% Quellenangabe nur in der Leseansicht
\ifkorrekturansicht\else
% Fallback-Definitionen, falls die .tex-Datei \titel etc. nicht gesetzt hat
\providecommand{\titel}{}
\providecommand{\editorInnen}{}
\providecommand{\dateiname}{\jobname}

\vspace{3cm}

\vfill

\footnotesize
\textsc{Quelle}: \titel. Herausgegeben von {\editorInnen}. In: \emph{Arthur Schnitzler: Briefwechsel mit Autorinnen und Autoren}.
 Digitale Edition, https://schnitzler-briefe.acdh.oeaw.ac.at/{\dateiname}.html (Stand \today)
\fi

\end{document}


