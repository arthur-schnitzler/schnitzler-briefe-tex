%% latex-korrekturansicht-vorspann.tex
%% Vorspann für die Korrekturansicht.
%% Lädt die gemeinsame Datei latex-vorspann.tex mit gesetztem Schalter.

\newif\ifkorrekturansicht
\korrekturansichttrue

\input{../tex-inputs/latex-vorspann}


\section[Stefan Zweig an Arthur Schnitzler, 16. 1. 1915]{L03650 Stefan Zweig an Arthur Schnitzler, 16. 1. 1915}
\nopagebreak\mylabel{L03650v}
\rehead{ }\normalsize\beginnumbering\briefempfaengerindex{Schnitzler, Arthur@\textsc{Schnitzler, Arthur}!zzzZweig, Stefan@\emph{von Stefan Zweig}!1915-01-161@{16. 1. 1915}|(be}
\toendnotes[C]{\smallbreak\pagebreak[2]}\Standort{CUL, Schnitzler, B 118.}
\physDesc{Brief, 1 Blatt, 1 Seite, 559 Zeichen
\newline{}Handschrift: schwarze Tinte, lateinische Kurrent
\newline{}Schnitzler: mit rotem Buntstift eine Unterstreichung }
\buchAbdrucke{\weitereDrucke{1) Stefan Zweig: \emph{Briefwechsel mit Hermann Bahr, Sigmund Freud, Rainer Maria
                        Rilke und Arthur Schnitzler}. Frankfurt am Main: \emph{S. Fischer} 1987, S. 390.} \weitereDrucke{2) Stefan Zweig: \emph{Briefe. Bd. II: 1914–1919}. Frankfurt am Main: \emph{S. Fischer} 1998, S. 50.} }\toendnotes[C]{\smallbreak}
\pstart
           {\pb}Wien\oindex{Wien@\textbf{Wien}, \emph{A.ADM2}|pw}{ }16. Januar 1915\pend
           \vspace{0.5em}
\pstart
           Lieber verehrter Herr Doktor, den Ausschnitt\pwindex{Une protestation DArthur Schnitzler@\emph{Une protestation d’Arthur Schnitzler}|pwv} aus dem »Journal
                  de Genève\pwindex{Journal de Geneve@\emph{Journal de Genève}|pw}« sandte ich Ihnen schon vor paar Tagen durch Stringa\pwindex{Stringa, Alberto 12.01.1880 – 09.11.1931@\textsc{Stringa, Alberto} (12.01.1880 – 09.11.1931), \emph{Maler/Malerin}|pw}. Von Romain
                  Rolland\pwindex{Rolland, Romain 29.01.1866 – 30.12.1944@\textsc{Rolland, Romain} (29.01.1866 – 30.12.1944), \emph{Schriftsteller/Schriftstellerin}|pw} habe ich plötzlich keine Briefe mehr, die Censur hat anscheinend
               unsere – doch zweifellos staatsgefährliche und an den Fundamenten Österreichs\oindex{Oesterreich@\textbf{Österreich}, \emph{A.PCLI}|pw} rüttelnde — Correspondenz unterbunden und
               abgedrosselt. Ich schreibe ihm über Italien\oindex{Italien@\textbf{Italien}, \emph{A.PCLI}|pw} und
               wende mich übrigens heute noch an die Briefcensur direct, um ihr den Begriff Romain Rolland\pwindex{Rolland, Romain 29.01.1866 – 30.12.1944@\textsc{Rolland, Romain} (29.01.1866 – 30.12.1944), \emph{Schriftsteller/Schriftstellerin}|pw} aufzuklären. Hoffentlich
               ge­lingts! Viele viele Grüsse Ihres getreuen\pend
           \pstart \spacefill\mbox{Stefan Zweig}\pend{}\selectlanguage{ngerman}\endnumbering\briefempfaengerindex{Schnitzler, Arthur@\textsc{Schnitzler, Arthur}!zzzZweig, Stefan@\emph{von Stefan Zweig}!1915-01-161@{16. 1. 1915}|)be}\mylabel{L03650h}
\begin{anhang}
\end{anhang}\normalsize

\doendnotes{C}
\bigskip
\vfill

\clearpage

\footnotesize

\lohead{\textsc{register}}

% Definiere theindex-Environment komplett neu ohne reledmac
\makeatletter
\renewenvironment{theindex}{%
  \section*{\indexname}%
  \setlength{\parindent}{0pt}%
  \setlength{\parskip}{0pt plus 0.3pt}%
  \let\item\@idxitem
}{%
  \clearpage
}
\makeatother

\IfFileExists{\jobname-pw.ind}{\input{\jobname-pw.ind}}{}

\end{document}

      