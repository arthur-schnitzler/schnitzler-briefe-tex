%% latex-leseansicht-vorspann.tex
%% Vorspann für die Leseansicht.
%% Lädt die gemeinsame Datei latex-vorspann.tex mit nicht gesetztem Schalter.

\newif\ifkorrekturansicht
\korrekturansichtfalse

\input{../tex-inputs/latex-vorspann}

\begin{center}
            \textcolor{red}{ENTWURF, NICHT FERTIG KORRIGIERT}
                      \end{center}
            
         
         \renewcommand{\erwaehntePersonen}{Personen: G. G.}
         \renewcommand{\erwaehnteOrte}{Orte: Brühl, Wien}
         \renewcommand{\erwaehnteWerke}{}
               \section[Felix Salten an Arthur Schnitzler, {[}10?. 3. 1902{]}]{ Felix Salten an Arthur Schnitzler, {[}10?. 3. 1902{]}}\nopagebreak\mylabel{v}\rehead{ }\begin{ledgroupsized}[t]{13cm}\normalsize\beginnumbering \toendnotes[C]{\smallbreak\pagebreak[2]} \Standort{CUL, Schnitzler, B 89, A 2.}
\physDesc{Brief, 1 Blatt, 1 Seite, 345 Zeichen
\newline{}Handschrift: schwarze Tinte, lateinische Kurrent
\newline{}Schnitzler: mit Bleistift datiert: »1\substVorne{}\textsuperscript{\textcolor{gray}{5}}\substDazwischen{}\textcolor{gray}{0}\substHinten{}. 3. 902« 
\newline{}Ordnung: mit Bleistift von unbekannter Hand nummeriert:
                                    »148« }\toendnotes[C]{\smallbreak}\pstart
           \raggedleft{}{\pb}\label{K_L03324-1v}\edtext{Montag}{\lemma{\textnormal{\emph{Montag}}}\Cendnote{\textnormal{Die Datierung Schnitzler\pwindex{Schnitzler, Arthur 15.05.1862 – 21.10.1931@\textsc{Schnitzler, Arthur} (15.05.1862 – 21.10.1931), \emph{Schriftsteller, Mediziner}|pwk}s ist bei der zweiten Ziffer des Kalendertags nicht mit
                        Sicherheit zu entziffern. Unter der Annahme, dass der Wochentag hier richtig
                        wiedergegeben ist, sind nur der 10. und 17.
                        mögliche Daten, wobei eine »7« bei Schnitzler\pwindex{Schnitzler, Arthur 15.05.1862 – 21.10.1931@\textsc{Schnitzler, Arthur} (15.05.1862 – 21.10.1931), \emph{Schriftsteller, Mediziner}|pwk} nicht zu erkennen ist. Weiters scheint er als Folge
                        dieses Briefs am Folgetag einen Krankenbesuch zu machen, vgl. Felix Salten an Arthur Schnitzler, [11. 3. 1902].}}}\label{K_L03324-1h}.\pend
           \pstart
           Lieber, bin seit acht Tagen recht krank und zu Bett. \label{K_L03324-19v}\edtext{Geschichte mit G. G.\pwindex{G., G. @\textsc{G., G.}|pw}}{\lemma{\textnormal{\emph{Geschichte mit G. G.}}}\Cendnote{\textnormal{unklar}}}\label{K_L03324-19h} hat sich nur auf N\textsuperscript{r} 10 bezogen, die »Conservatoristin« wurde dazu erfunden.
               So wird man manchmal beunruhigt. Warum sind Sie noch auf der Suche? Sagten Sie mir
               nicht, Sie hätten in der Brühl\oindex{Bruehl@\textbf{Brühl}|pw} schon fix
               gemietet? \pend
           \pstart
           Hoffentlich bin ich in 8 Tagen wieder wol.\pend
           \pstart
           Herzlichst Ihr {\\[\baselineskip]}\spacefill\mbox{Salten}\pend
           \leftskip=0em{}
         
         \endnumbering\mylabel{h}\end{ledgroupsized}\begin{anhang}\end{anhang}\newcommand{\dateiname}{L03324}\newcommand{\titel}{Felix Salten an Arthur Schnitzler, [10?. 3. 1902]}\newcommand{\editorInnen}{Martin Anton Müller und Laura Untner}%% latex-leseansicht-abspann.tex
%% Abspann für die Leseansicht.
%% Der Schalter \ifkorrekturansicht ist bereits durch den Vorspann gesetzt.

%% latex-abspann.tex
%% Gemeinsamer Abspann für Korrekturansicht und Leseansicht.
%% Setzt den Schalter \ifkorrekturansicht voraus (gesetzt in den
%% einbindenden Dateien latex-korrekturansicht-abspann.tex bzw.
%% latex-leseansicht-abspann.tex).
%% ---------------------------------------------------------------

\normalsize

% Das esempio-Environment wird nur in der Leseansicht benötigt
\ifkorrekturansicht\else
\newenvironment{esempio}[3]%
{
    \vspace{1.5ex}
    \rlap{\underline{#1}}
    \par
    \setlength{\parindent}{0cm}
    \nopagebreak
    \leftskip=#2cm
    \rightskip=#3cm
}
{
    \par
}
\fi

\doendnotes{C}
\bigskip
\vfill

\clearpage

\footnotesize

\ifkorrekturansicht
  \lohead{\textsc{register}}
\fi

% theindex-Environment neu definieren ohne reledmac
\makeatletter
\renewenvironment{theindex}{%
  \ifkorrekturansicht
    \section*{\indexname}%
  \else
    \subsubsection*{Index der erwähnten Entitäten}%
  \fi
  \setlength{\parindent}{0pt}%
  \setlength{\parskip}{0pt plus 0.3pt}%
  \let\item\@idxitem
}{%
  \ifkorrekturansicht\clearpage\fi
}
\makeatother

\IfFileExists{\jobname-pw.ind}{\input{\jobname-pw.ind}}{}

% Quellenangabe nur in der Leseansicht
\ifkorrekturansicht\else
% Fallback-Definitionen, falls die .tex-Datei \titel etc. nicht gesetzt hat
\providecommand{\titel}{}
\providecommand{\editorInnen}{}
\providecommand{\dateiname}{\jobname}

\vspace{3cm}

\vfill

\footnotesize
\textsc{Quelle}: \titel. Herausgegeben von {\editorInnen}. In: \emph{Arthur Schnitzler: Briefwechsel mit Autorinnen und Autoren}.
 Digitale Edition, https://schnitzler-briefe.acdh.oeaw.ac.at/{\dateiname}.html (Stand \today)
\fi

\end{document}


      