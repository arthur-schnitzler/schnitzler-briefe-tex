%% latex-leseansicht-vorspann.tex
%% Vorspann für die Leseansicht.
%% Lädt die gemeinsame Datei latex-vorspann.tex mit nicht gesetztem Schalter.

\newif\ifkorrekturansicht
\korrekturansichtfalse

\input{../tex-inputs/latex-vorspann}


\section[ Felix Salten an Arthur Schnitzler, [10?. 3. 1902]]{L03324 Felix Salten an Arthur Schnitzler,  [10?. 3. 1902]}
\nopagebreak\mylabel{L03324v}
\rehead{ }\normalsize\beginnumbering\briefempfaengerindex{Schnitzler, Arthur@\textsc{Schnitzler, Arthur}!zzzSalten, Felix@\emph{von Felix Salten}!1902-03-101@{{[}10?. 3. 1902{]}}|(be}
\toendnotes[C]{\smallbreak\pagebreak[2]}
\correspDesc{Versand  durch Felix Salten am [10?. 3. 1902] in Wien
\newline{}Erhalt  durch Arthur Schnitzler im Zeitraum [10. 3. 1902
                  – 13. 3. 1902?] in Wien}\toendnotes[C]{\smallbreak}
\Standort{CUL, Schnitzler, B 89, A 2.}
\physDesc{Brief, 1 Blatt, 1 Seite, 346 Zeichen
\newline{}Handschrift: schwarze Tinte, lateinische Kurrent
\newline{}Schnitzler: mit Bleistift datiert: »1\substVorne{}\textsuperscript{\textcolor{gray}{3}}\substDazwischen{}\textcolor{gray}{0}\substHinten{}. 3. 902« 
\newline{}Ordnung: mit Bleistift von unbekannter Hand nummeriert: »148« }\toendnotes[C]{\smallbreak}
\pstart
           \raggedleft{}{\pb}\label{K_L03324-1v}\edtext{Montag}{\lemma{\textnormal{\emph{Montag}}}\Cendnote{\textnormal{Die zweite Ziffer des Kalendertags von Schnitzlers Datierung ist nicht mit
                        Sicherheit zu entziffern. Saltens\pwindex{Salten, Felix 6.\,9.\,1869 Budapest – 8.\,10.\,1945 Zürich@\textsc{Salten, Felix} (6.\,9.\,1869 Budapest – 8.\,10.\,1945 Zürich), \emph{Schriftsteller, Journalist, Chefredakteur}|pwk} Angabe »Montag« erlaubt nur den 10. und den 17. 3. 1902 als
                        mögliche Daten. Eine ›7‹ ist nicht zu erkennen. Dazu kommt, dass Schnitzler wohl in Folge dieses Briefs
                        am Folgetag{ }Salten\pwindex{Salten, Felix 6.\,9.\,1869 Budapest – 8.\,10.\,1945 Zürich@\textsc{Salten, Felix} (6.\,9.\,1869 Budapest – 8.\,10.\,1945 Zürich), \emph{Schriftsteller, Journalist, Chefredakteur}|pwk} einen Krankenbesuch abstattete
                           (vgl. XXXX Auszeichnungsfehler: Dokument L03325 nicht gefunden).}}}\label{K_L03324-1}.\pend
           \vspace{0.5em}
\pstart
           Lieber, bin seit acht Tagen recht krank und zu Bett. \label{K_L03324-2v}\edtext{Geschichte mit G. G.\pwindex{G., G. @\textsc{G., G.}|pw}}{\lemma{\textnormal{\emph{Geschichte mit G. G.}}}\Cendnote{\textnormal{Bezug unklar. Der Hinweis auf Nr. 10 könnte
               aber ein Indiz sein, dass es sich um etwas in einer wöchentlich erscheinenden Publikation handelt, da seit 
               Jahresbeginn 10 Wochen vergangen waren.}}}\label{K_L03324-2} hat sich nur auf
                  N\textsuperscript{r} 10 bezogen, die »Conservatoristin« wurde dazu
               erfunden. So wird man manchmal beunruhigt. Warum sind Sie noch auf der \label{K_L03324-3v}\edtext{Suche}{\lemma{\textnormal{\emph{Suche}}}\Cendnote{\textnormal{Siehe XXXX Auszeichnungsfehler: Dokument L03192 nicht gefunden.
               }}}\label{K_L03324-3}? Sagten Sie mir nicht, Sie hätten in der Brühl\oindex{Brühl@\textbf{Brühl}, \emph{Tal}|pw} schon fix gemiethet?\pend
           
\pstart
           Hoffentlich bin ich in 8 Tagen wieder wol.\pend
           
\pstart
           Herzlichst Ihr{\\[\baselineskip]}\spacefill\mbox{Salten}\pend
           \leftskip=0em{}\selectlanguage{ngerman}\endnumbering\briefempfaengerindex{Schnitzler, Arthur@\textsc{Schnitzler, Arthur}!zzzSalten, Felix@\emph{von Felix Salten}!1902-03-101@{{[}10?. 3. 1902{]}}|)be}\mylabel{L03324h}  \newcommand{\dateiname}{L03324}\newcommand{\titel}{Felix Salten an Arthur Schnitzler, [10?. 3. 1902]}\newcommand{\editorInnen}{Martin Anton Müller und Laura Untner}%% latex-leseansicht-abspann.tex
%% Abspann für die Leseansicht.
%% Der Schalter \ifkorrekturansicht ist bereits durch den Vorspann gesetzt.

%% latex-abspann.tex
%% Gemeinsamer Abspann für Korrekturansicht und Leseansicht.
%% Setzt den Schalter \ifkorrekturansicht voraus (gesetzt in den
%% einbindenden Dateien latex-korrekturansicht-abspann.tex bzw.
%% latex-leseansicht-abspann.tex).
%% ---------------------------------------------------------------

\normalsize

% Das esempio-Environment wird nur in der Leseansicht benötigt
\ifkorrekturansicht\else
\newenvironment{esempio}[3]%
{
    \vspace{1.5ex}
    \rlap{\underline{#1}}
    \par
    \setlength{\parindent}{0cm}
    \nopagebreak
    \leftskip=#2cm
    \rightskip=#3cm
}
{
    \par
}
\fi

\doendnotes{C}
\bigskip
\vfill

\clearpage

\footnotesize

\ifkorrekturansicht
  \lohead{\textsc{register}}
\fi

% theindex-Environment neu definieren ohne reledmac
\makeatletter
\renewenvironment{theindex}{%
  \ifkorrekturansicht
    \section*{\indexname}%
  \else
    \subsubsection*{Index der erwähnten Entitäten}%
  \fi
  \setlength{\parindent}{0pt}%
  \setlength{\parskip}{0pt plus 0.3pt}%
  \let\item\@idxitem
}{%
  \ifkorrekturansicht\clearpage\fi
}
\makeatother

\IfFileExists{\jobname-pw.ind}{\input{\jobname-pw.ind}}{}

% Quellenangabe nur in der Leseansicht
\ifkorrekturansicht\else
% Fallback-Definitionen, falls die .tex-Datei \titel etc. nicht gesetzt hat
\providecommand{\titel}{}
\providecommand{\editorInnen}{}
\providecommand{\dateiname}{\jobname}

\vspace{3cm}

\vfill

\footnotesize
\textsc{Quelle}: \titel. Herausgegeben von {\editorInnen}. In: \emph{Arthur Schnitzler: Briefwechsel mit Autorinnen und Autoren}.
 Digitale Edition, https://schnitzler-briefe.acdh.oeaw.ac.at/{\dateiname}.html (Stand \today)
\fi

\end{document}


