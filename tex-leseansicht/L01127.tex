%% latex-korrekturansicht-vorspann.tex
%% Vorspann für die Korrekturansicht.
%% Lädt die gemeinsame Datei latex-vorspann.tex mit gesetztem Schalter.

\newif\ifkorrekturansicht
\korrekturansichttrue

\input{../tex-inputs/latex-vorspann}


\section[Arthur Schnitzler an Georg Brandes, 11. 6. 1901]{L01127 Arthur Schnitzler an Georg Brandes, 11. 6. 1901}
\nopagebreak\mylabel{L01127v}
\rehead{ }\normalsize\beginnumbering\briefempfaengerindex{Brandes, Georg@\textsc{Brandes, Georg}!zzzSchnitzler, Arthur@\emph{von Arthur Schnitzler}!1901-06-111@{11. 6. 1901}|(be}
\toendnotes[C]{\smallbreak\pagebreak[2]}\Standort{Kopenhagen, Det Kongelige Bibliotek, Georg Brandes Arkiv, box 125.}
\physDesc{Brief, 2 Blätter, 6 Seiten, 1718 Zeichen
\newline{}Handschrift: schwarze Tinte, deutsche Kurrent
\newline{}Ordnung: mit Bleistift von unbekannter Hand nummeriert:
                                    »24.« und zweimal mit Bleistift datiert: »11. 6. 01.«, »11/6 01«, das zweite Blatt auf einer leeren Seite mit Bleistift
                                 mit »Schnitzler« beschriftet }
\buchAbdrucke{\weitereDrucke{Georg Brandes, Arthur Schnitzler: \emph{Ein Briefwechsel}. Bern: \emph{Francke} 1956, S. 87–88.} }\toendnotes[C]{\smallbreak}
\pstart{}{\pb}Lieber und verehrter Herr Brandes,\pend\vspace{0.5em}
\pstart
           ehe ich wieder einmal auf Reiſen gehe – das geſchieht heute Abend und wahrſcheinlich
               für einige Monate, will ich Sie noch herzlich grüßen und Ihnen für Ihre Nachrichten
               aus Abazia\oindex{Hotel Guarnero@\textbf{Hotel Guarnero}, \emph{Hotel (K.HTL)}|pw} danken, das Sie übrigens raſcher
               verlaſſen haben, ſcheint mir, als Ihre Abſicht war. Daſs ich Sie {\pb}nicht wenigſtens auf ein paar Minuten zu ſehn und
               zu ſprechen bekam, auf der Rückreiſe, thut mir leid. Sie entſchuldigen ſich, dſs Sie
               mir die Zeit geraubt haben – als wenn Sie nicht wüßten, daſs ich Ihnen von ganzem
               Herzen für die Stunden danke, die Sie mir widmen. Muſs ich das wirklich erſt ſagen?–
               Daſs das Geld pünktlich angeko{\geminationm}en iſt, erſehen Sie
               daraus {\pb}daſs Sie weder Mahnbriefe noch einen
               Pfändungsauftrag bekommen haben. Richard \textsc{Beer H.}\pwindex{Beer-Hofmann, Richard 1866-07-11 – 1945-09-26@\textsc{Beer-Hofmann, Richard} (1866-07-11 – 1945-09-26), \emph{Schriftsteller/Schriftstellerin}|pw} iſt am Wörtherſee\oindex{Woerthersee@\textbf{Wörthersee}, \emph{H.LK}|pw}, in Pörtſchach, Villa Arnſtein\oindex{Villa Arnstein@\textbf{Villa Arnstein}, \emph{Wohngebäude (K.WHS)}|pw}, u. wird wohl den ganzen Sommer dort
               bleiben. Ich fahre vor allem nach Salzburg\oindex{Salzburg@\textbf{Salzburg}, \emph{A.ADM2}|pw} und
               weiſs kaum, was ich weiter unternehmen werde. Ich bin ſehr erfüllt von einem ſchönen
               Stoff, einem in heutiger Zeit ſpielenden {\pb}Trauerſpiel\pwindex{einsame Weg. Schauspiel in fuenf Akten@\emph{Der einsame Weg. Schauspiel in fünf Akten}|pwv} – und möchte das
               Stück gern irgendwo im grünen und ſtillen beginnen und zu Ende führen. Ich freue
               mich, dſs Sie die Novelle vom Lieutenant Guſtl\pwindex{Lieutenant Gustl. Novelle@\emph{Lieutenant Gustl. Novelle}|pw}
               amüſirt hat. Eine Novelle von \textsc{Dostojewski}\pwindex{Dostojevskij, Fjodor Mihajlovic 11.11.1821 – 09.02.1881@\textsc{Dostojevskij, Fjodor Mihajlovič} (11.11.1821 – 09.02.1881), \emph{Schriftsteller/Schriftstellerin}|pw}, \textsc{Krotkaja}\pwindex{Sanfte@\emph{Die Sanfte}|pw}, die ich nicht kenne, ſoll die gleiche Technik des Gedankenmonologs aufweiſen.
               Mir aber wurde der erſte Anlaſs zu der \uline{Form} durch
               eine Geſchichte {\pb}von \textsc{Dujardin}\pwindex{Dujardin, Edouard 10.10.1861 – 31.10.1949@\textsc{Dujardin, Édouard} (10.10.1861 – 31.10.1949), \emph{Schriftsteller/Schriftstellerin}|pw} gegeben, betitelt \textsc{les lauriers sont coupé}\substVorne{}\textsuperscript{\textsc{es}}\substDazwischen{}\textsc{s}\substHinten{}\pwindex{lauriers sont coupes@\emph{Les lauriers sont coupés}|pw}. Nur daſs dieſer Autor für ſeine Form nicht den rechten Stoff zu finden
               wußte. –\pend
           
\pstart
           Verbringen Sie einen angenehmen Sommer und laſſen Sie we{\geminationn}{ }Sie gelaunt ſind, einmal eine Zeile an mich
               gelangen. Ich will Ihnen bald ſchreiben, wo ich zur Ruhe geko{\geminationm}en {\pb}bin. Leben Sie
               wohl. Von Herzen\pend
           
\pstart
           Ihr{\\[\baselineskip]}\spacefill\mbox{ArthurSchnitzler}\pend
           \leftskip=0em{}
\pstart
           Wien\oindex{Wien@\textbf{Wien}, \emph{A.ADM2}|pw}, 11. 6. 901.\pend
           \selectlanguage{ngerman}\endnumbering\briefempfaengerindex{Brandes, Georg@\textsc{Brandes, Georg}!zzzSchnitzler, Arthur@\emph{von Arthur Schnitzler}!1901-06-111@{11. 6. 1901}|)be}\mylabel{L01127h}  \normalsize

\doendnotes{C}
\bigskip
\vfill

\clearpage

\footnotesize

\lohead{\textsc{register}}

% Definiere theindex-Environment komplett neu ohne reledmac
\makeatletter
\renewenvironment{theindex}{%
  \section*{\indexname}%
  \setlength{\parindent}{0pt}%
  \setlength{\parskip}{0pt plus 0.3pt}%
  \let\item\@idxitem
}{%
  \clearpage
}
\makeatother

\IfFileExists{\jobname-pw.ind}{\input{\jobname-pw.ind}}{}

\end{document}

      