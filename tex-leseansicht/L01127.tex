%% latex-leseansicht-vorspann.tex
%% Vorspann für die Leseansicht.
%% Lädt die gemeinsame Datei latex-vorspann.tex mit nicht gesetztem Schalter.

\newif\ifkorrekturansicht
\korrekturansichtfalse

\input{../tex-inputs/latex-vorspann}


         
         \renewcommand{\erwaehntePersonen}{Personen: Richard Beer-Hofmann, Georg Brandes, Fjodor Mihajlovič Dostojevskij, Édouard Dujardin}
         \renewcommand{\erwaehnteOrte}{Orte: Hotel Guarnero, Salzburg, Villa Arnstein, Wien, Wörthersee}
         \renewcommand{\erwaehnteWerke}{Werke: Der einsame Weg. Schauspiel in fünf Akten, Die Sanfte, Les lauriers sont coupés, Lieutenant Gustl. Novelle}
               \section[Arthur Schnitzler an Georg Brandes, 11. 6. 1901]{ Arthur Schnitzler an Georg Brandes, 11. 6. 1901}\nopagebreak\mylabel{v}\rehead{ }\begin{ledgroupsized}[t]{13cm}\normalsize\beginnumbering \toendnotes[C]{\smallbreak\pagebreak[2]} \Standort{Kopenhagen, Det Kongelige Bibliotek, Georg Brandes Arkiv, box 125.}
\physDesc{Brief, 2 Blätter, 6 Seiten
\newline{}Handschrift: schwarze Tinte, deutsche Kurrent\newline{}Ordnung: mit Bleistift von unbekannter Hand nummeriert: »24.« und zweimal
                                    mit Bleistift datiert: »11. 6. 01.«, »11/6 01«, das zweite Blatt auf einer leeren Seite mit Bleistift
                                    mit »Schnitzler« beschriftet }\buchAbdrucke{\weitereDrucke{Georg Brandes, Arthur Schnitzler: \emph{Ein Briefwechsel}. Hg. Kurt Bergel. Bern: \emph{Francke} 1956, S. 87–88.} }\toendnotes[C]{\smallbreak}\pstart{}{\pb}Lieber und verehrter Herr
                        Brandes,\pend\pstart
           ehe ich wieder einmal auf Reiſen gehe – das geſchieht heute Abend und
                    wahrſcheinlich für einige Monate, will ich Sie noch herzlich grüßen und Ihnen
                    für Ihre Nachrichten aus Abazia\oindex{Hotel Guarnero@\textbf{Hotel Guarnero}|pw} danken, das Sie
                    übrigens raſcher verlaſſen haben, ſcheint mir, als Ihre Abſicht war. Daſs ich
                    Sie {\pb}nicht wenigſtens auf ein paar
                    Minuten zu ſehn und zu ſprechen bekam, auf der Rückreiſe, thut mir leid. Sie
                    entſchuldigen ſich, dſs Sie mir die Zeit geraubt haben – als wenn Sie nicht
                    wüßten, daſs ich Ihnen von ganzem Herzen für die Stunden danke, die Sie mir
                    widmen. Muſs ich das wirklich erſt ſagen?– Daſs das Geld pünktlich angeko{\geminationm}en iſt, erſehen Sie daraus {\pb}daſs Sie weder Mahnbriefe noch einen
                    Pfändungsauftrag bekommen haben. Richard \textsc{Beer H.}\pwindex{Beer-Hofmann, Richard 1866-07-11 – 1945-09-26@\textsc{Beer-Hofmann, Richard} (1866-07-11 – 1945-09-26), \emph{Schriftsteller}|pw} iſt am Wörtherſee\oindex{Woerthersee@\textbf{Wörthersee}|pw}, in Pörtſchach, Villa Arnſtein\oindex{Villa Arnstein@\textbf{Villa Arnstein}|pw}, u. wird wohl den ganzen Sommer
                    dort bleiben. Ich fahre vor allem nach Salzburg\oindex{Salzburg@\textbf{Salzburg}|pw}
                    und weiſs kaum, was ich weiter unternehmen werde. Ich bin ſehr erfüllt von einem
                    ſchönen Stoff, einem in heutiger Zeit ſpielenden {\pb}Trauerſpiel\pwindex{Schnitzler, Arthur 15.05.1862 – 21.10.1931@\textsc{Schnitzler, Arthur} (15.05.1862 – 21.10.1931), \emph{Schriftsteller, Mediziner}!einsame Weg. Schauspiel in fuenf Akten1904@\strich\emph{Der einsame Weg. Schauspiel in fünf Akten} {[}1904{]}|pwv} – und möchte das
                    Stück gern irgendwo im grünen und ſtillen beginnen und zu Ende führen. Ich freue
                    mich, dſs Sie die Novelle vom Lieutenant Guſtl\pwindex{Schnitzler, Arthur 15.05.1862 – 21.10.1931@\textsc{Schnitzler, Arthur} (15.05.1862 – 21.10.1931), \emph{Schriftsteller, Mediziner}!Lieutenant Gustl. Novelle1900-12-25@\strich\emph{Lieutenant Gustl. Novelle} {[}1900-12-25{]}|pw}
                    amüſirt hat. Eine Novelle von \textsc{Dostojewski}\pwindex{Dostojevskij, Fjodor Mihajlovic 11.11.1821 – 09.02.1881@\textsc{Dostojevskij, Fjodor Mihajlovič} (11.11.1821 – 09.02.1881), \emph{Schriftsteller}|pw}, \textsc{Krotkaja}\pwindex{Dostojevskij, Fjodor Mihajlovic 11.11.1821 – 09.02.1881@\textsc{Dostojevskij, Fjodor Mihajlovič} (11.11.1821 – 09.02.1881), \emph{Schriftsteller}!Sanfte1876@\strich\emph{Die Sanfte} {[}1876{]}|pw}, die ich nicht kenne, ſoll die gleiche Technik des Gedankenmonologs
                    aufweiſen. Mir aber wurde der erſte Anlaſs zu der \uline{Form} durch eine Geſchichte {\pb}von
                        \textsc{Dujardin}\pwindex{Dujardin, Edouard 10.10.1861 – 31.10.1949@\textsc{Dujardin, Édouard} (10.10.1861 – 31.10.1949), \emph{Schriftsteller}|pw} gegeben, betitelt \textsc{les lauriers sont coupé}\substVorne{}\textsuperscript{\textsc{es}}\substDazwischen{}\textsc{s}\substHinten{}\pwindex{Dujardin, Edouard 10.10.1861 – 31.10.1949@\textsc{Dujardin, Édouard} (10.10.1861 – 31.10.1949), \emph{Schriftsteller}!lauriers sont coupes1887@\strich\emph{Les lauriers sont coupés} {[}1887{]}|pw}. Nur daſs dieſer Autor für ſeine Form nicht den rechten Stoff zu finden
                    wußte. –\pend
           \pstart
           Verbringen Sie einen angenehmen Sommer und laſſen Sie we{\geminationn}{ }Sie gelaunt ſind, einmal eine Zeile an mich
                    gelangen. Ich will Ihnen bald ſchreiben, wo ich zur Ruhe geko{\geminationm}en {\pb}bin.
                    Leben Sie wohl. Von Herzen\pend
           \pstart
           Ihr{\\[\baselineskip]}\spacefill\mbox{ArthurSchnitzler}\pend
           \leftskip=0em{}\pstart
           Wien\oindex{Wien@\textbf{Wien}|pw},
                        11. 6. 901.\pend
           
         
         \endnumbering\mylabel{h}\end{ledgroupsized}  \newcommand{\dateiname}{L01127}\newcommand{\titel}{Arthur Schnitzler an Georg Brandes, 11. 6. 1901}\newcommand{\editorInnen}{Martin Anton Müller und Gerd-Hermann Susen}%% latex-leseansicht-abspann.tex
%% Abspann für die Leseansicht.
%% Der Schalter \ifkorrekturansicht ist bereits durch den Vorspann gesetzt.

%% latex-abspann.tex
%% Gemeinsamer Abspann für Korrekturansicht und Leseansicht.
%% Setzt den Schalter \ifkorrekturansicht voraus (gesetzt in den
%% einbindenden Dateien latex-korrekturansicht-abspann.tex bzw.
%% latex-leseansicht-abspann.tex).
%% ---------------------------------------------------------------

\normalsize

% Das esempio-Environment wird nur in der Leseansicht benötigt
\ifkorrekturansicht\else
\newenvironment{esempio}[3]%
{
    \vspace{1.5ex}
    \rlap{\underline{#1}}
    \par
    \setlength{\parindent}{0cm}
    \nopagebreak
    \leftskip=#2cm
    \rightskip=#3cm
}
{
    \par
}
\fi

\doendnotes{C}
\bigskip
\vfill

\clearpage

\footnotesize

\ifkorrekturansicht
  \lohead{\textsc{register}}
\fi

% theindex-Environment neu definieren ohne reledmac
\makeatletter
\renewenvironment{theindex}{%
  \ifkorrekturansicht
    \section*{\indexname}%
  \else
    \subsubsection*{Index der erwähnten Entitäten}%
  \fi
  \setlength{\parindent}{0pt}%
  \setlength{\parskip}{0pt plus 0.3pt}%
  \let\item\@idxitem
}{%
  \ifkorrekturansicht\clearpage\fi
}
\makeatother

\IfFileExists{\jobname-pw.ind}{\input{\jobname-pw.ind}}{}

% Quellenangabe nur in der Leseansicht
\ifkorrekturansicht\else
% Fallback-Definitionen, falls die .tex-Datei \titel etc. nicht gesetzt hat
\providecommand{\titel}{}
\providecommand{\editorInnen}{}
\providecommand{\dateiname}{\jobname}

\vspace{3cm}

\vfill

\footnotesize
\textsc{Quelle}: \titel. Herausgegeben von {\editorInnen}. In: \emph{Arthur Schnitzler: Briefwechsel mit Autorinnen und Autoren}.
 Digitale Edition, https://schnitzler-briefe.acdh.oeaw.ac.at/{\dateiname}.html (Stand \today)
\fi

\end{document}


      