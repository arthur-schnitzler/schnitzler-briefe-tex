%% latex-leseansicht-vorspann.tex
%% Vorspann für die Leseansicht.
%% Lädt die gemeinsame Datei latex-vorspann.tex mit nicht gesetztem Schalter.

\newif\ifkorrekturansicht
\korrekturansichtfalse

\input{../tex-inputs/latex-vorspann}


\section[Arthur Schnitzler an Georg Brandes, 11. 6. 1901]{L01127 Arthur Schnitzler an Georg Brandes, 11. 6. 1901}
\nopagebreak\mylabel{L01127v}
\rehead{ }\normalsize\beginnumbering\briefempfaengerindex{Brandes, Georg@\textsc{Brandes, Georg}!zzzSchnitzler, Arthur@\emph{von Arthur Schnitzler}!1901-06-111@{11. 6. 1901}|(be}
\toendnotes[C]{\smallbreak\pagebreak[2]}
\correspDesc{Versand  durch Arthur Schnitzler am 11. 6. 1901 in Wien
\newline{}Erhalt  durch Georg Brandes im Zeitraum [11. 6. 1901
                  – 15. 6. 1901?] \textbf{Ort fehlend} }\toendnotes[C]{\smallbreak}
\Standort{Kopenhagen, Det Kongelige Bibliotek, Georg Brandes Arkiv, box 125.}
\physDesc{Brief, 2 Blätter, 6 Seiten, 1718 Zeichen
\newline{}Handschrift: schwarze Tinte, deutsche Kurrent
\newline{}Ordnung: mit Bleistift von unbekannter Hand nummeriert:
                                    »24.« und zweimal mit Bleistift datiert: »11. 6. 01.«, »11/6 01«, das zweite Blatt auf einer leeren Seite mit Bleistift
                                 mit »Schnitzler« beschriftet }
\buchAbdrucke{\weitereDrucke{Georg Brandes, Arthur Schnitzler: \emph{Ein Briefwechsel}. Herausgegeben von Kurt Bergel. Bern: \emph{Francke} 1956, S. 87–88.} }\toendnotes[C]{\smallbreak}
\pstart{}{\pb}Lieber und verehrter Herr Brandes,\pend\vspace{0.5em}
\pstart
           ehe ich wieder einmal auf Reiſen gehe – das geſchieht heute Abend und wahrſcheinlich
               für einige Monate, will ich Sie noch herzlich grüßen und Ihnen für Ihre Nachrichten
               aus Abazia\oindex{Hotel Guarnero@\textbf{Hotel Guarnero}, \emph{Hotel}|pw} danken, das Sie übrigens raſcher
               verlaſſen haben,{ }ſcheint mir, als Ihre Abſicht war. Daſs ich Sie {\pb}nicht wenigſtens auf ein paar Minuten zu{ }ſehn und
               zu{ }ſprechen bekam, auf der Rückreiſe, thut mir leid. Sie entſchuldigen{ }ſich, dſs Sie
               mir die Zeit geraubt haben – als wenn Sie nicht wüßten, daſs ich Ihnen von ganzem
               Herzen für die Stunden danke, die Sie mir widmen. Muſs ich das wirklich erſt{ }ſagen?–
               Daſs das Geld pünktlich angeko{\geminationm}en iſt, erſehen Sie
               daraus {\pb}daſs Sie weder Mahnbriefe noch einen
               Pfändungsauftrag bekommen haben. Richard \textsc{Beer H.}\pwindex{Beer-Hofmann, Richard 11.\,7.\,1866 Wien – 26.\,9.\,1945 New York City@\textsc{Beer-Hofmann, Richard} (11.\,7.\,1866 Wien – 26.\,9.\,1945 New York City), \emph{Schriftsteller}|pw} iſt am Wörtherſee\oindex{Wörthersee@\textbf{Wörthersee}, \emph{See}|pw}, in Pörtſchach, Villa Arnſtein\oindex{Villa Arnstein@\textbf{Villa Arnstein}, \emph{Wohngebäude}|pw}, u. wird wohl den ganzen Sommer dort
               bleiben. Ich fahre vor allem nach Salzburg\oindex{Salzburg@\textbf{Salzburg}, \emph{Verwaltungsgebiet}|pw} und
               weiſs kaum, was ich weiter unternehmen werde. Ich bin{ }ſehr erfüllt von einem{ }ſchönen
               Stoff, einem in heutiger Zeit{ }ſpielenden {\pb}Trauerſpiel\pwindex{Schnitzler, Arthur 15.\,5.\,1862 Wien – 21.\,10.\,1931 ebd.@\textsc{Schnitzler, Arthur} (15.\,5.\,1862 Wien – 21.\,10.\,1931 ebd.), \emph{Schriftsteller, Mediziner}!einsame Weg. Schauspiel in fünf Akten@\strich\emph{Der einsame Weg. Schauspiel in fünf Akten}|pwv} – und möchte das
               Stück gern irgendwo im grünen und{ }ſtillen beginnen und zu Ende führen. Ich freue
               mich, dſs Sie die Novelle vom Lieutenant Guſtl\pwindex{Schnitzler, Arthur 15.\,5.\,1862 Wien – 21.\,10.\,1931 ebd.@\textsc{Schnitzler, Arthur} (15.\,5.\,1862 Wien – 21.\,10.\,1931 ebd.), \emph{Schriftsteller, Mediziner}!Lieutenant Gustl. Novelle@\strich\emph{Lieutenant Gustl. Novelle}|pw}
               amüſirt hat. Eine Novelle von \textsc{Dostojewski}\pwindex{Dostojevskij, Fjodor Mihajlovič 11.\,11.\,1821 Moskau – 9.\,2.\,1881 Sankt Petersburg@\textsc{Dostojevskij, Fjodor Mihajlovič} (11.\,11.\,1821 Moskau – 9.\,2.\,1881 Sankt Petersburg), \emph{Schriftsteller}|pw}, \textsc{Krotkaja}\pwindex{Dostojevskij, Fjodor Mihajlovič 11.\,11.\,1821 Moskau – 9.\,2.\,1881 Sankt Petersburg@\textsc{Dostojevskij, Fjodor Mihajlovič} (11.\,11.\,1821 Moskau – 9.\,2.\,1881 Sankt Petersburg), \emph{Schriftsteller}!Sanfte@\strich\emph{Die Sanfte}|pw}, die ich nicht kenne,{ }ſoll die gleiche Technik des Gedankenmonologs aufweiſen.
               Mir aber wurde der erſte Anlaſs zu der \uline{Form} durch
               eine Geſchichte {\pb}von \textsc{Dujardin}\pwindex{Dujardin, Édouard 10.\,10.\,1861 Saint-Gervais-la-Forêt – 31.\,10.\,1949 Paris@\textsc{Dujardin, Édouard} (10.\,10.\,1861 Saint-Gervais-la-Forêt – 31.\,10.\,1949 Paris), \emph{Schriftsteller}|pw} gegeben, betitelt \textsc{les lauriers sont coupé}\substVorne{}\textsuperscript{\textsc{es}}\substDazwischen{}\textsc{s}\substHinten{}\pwindex{Dujardin, Édouard 10.\,10.\,1861 Saint-Gervais-la-Forêt – 31.\,10.\,1949 Paris@\textsc{Dujardin, Édouard} (10.\,10.\,1861 Saint-Gervais-la-Forêt – 31.\,10.\,1949 Paris), \emph{Schriftsteller}!lauriers sont coupés@\strich\emph{Les lauriers sont coupés}|pw}. Nur daſs dieſer Autor für{ }ſeine Form nicht den rechten Stoff zu finden
               wußte. –\pend
           
\pstart
           Verbringen Sie einen angenehmen Sommer und laſſen Sie we{\geminationn}{ }Sie gelaunt{ }ſind, einmal eine Zeile an mich
               gelangen. Ich will Ihnen bald{ }ſchreiben, wo ich zur Ruhe geko{\geminationm}en {\pb}bin. Leben Sie
               wohl. Von Herzen\pend
           
\pstart
           Ihr{\\[\baselineskip]}\spacefill\mbox{ArthurSchnitzler}\pend
           \leftskip=0em{}
\pstart
           Wien\oindex{Wien@\textbf{Wien}, \emph{Verwaltungsgebiet}|pw}, 11. 6. 901.\pend
           \selectlanguage{ngerman}\endnumbering\briefempfaengerindex{Brandes, Georg@\textsc{Brandes, Georg}!zzzSchnitzler, Arthur@\emph{von Arthur Schnitzler}!1901-06-111@{11. 6. 1901}|)be}\mylabel{L01127h}  \newcommand{\dateiname}{L01127}\newcommand{\titel}{Arthur Schnitzler an Georg Brandes, 11. 6. 1901}\newcommand{\editorInnen}{Martin Anton Müller und Gerd-Hermann Susen}%% latex-leseansicht-abspann.tex
%% Abspann für die Leseansicht.
%% Der Schalter \ifkorrekturansicht ist bereits durch den Vorspann gesetzt.

%% latex-abspann.tex
%% Gemeinsamer Abspann für Korrekturansicht und Leseansicht.
%% Setzt den Schalter \ifkorrekturansicht voraus (gesetzt in den
%% einbindenden Dateien latex-korrekturansicht-abspann.tex bzw.
%% latex-leseansicht-abspann.tex).
%% ---------------------------------------------------------------

\normalsize

% Das esempio-Environment wird nur in der Leseansicht benötigt
\ifkorrekturansicht\else
\newenvironment{esempio}[3]%
{
    \vspace{1.5ex}
    \rlap{\underline{#1}}
    \par
    \setlength{\parindent}{0cm}
    \nopagebreak
    \leftskip=#2cm
    \rightskip=#3cm
}
{
    \par
}
\fi

\doendnotes{C}
\bigskip
\vfill

\clearpage

\footnotesize

\ifkorrekturansicht
  \lohead{\textsc{register}}
\fi

% theindex-Environment neu definieren ohne reledmac
\makeatletter
\renewenvironment{theindex}{%
  \ifkorrekturansicht
    \section*{\indexname}%
  \else
    \subsubsection*{Index der erwähnten Entitäten}%
  \fi
  \setlength{\parindent}{0pt}%
  \setlength{\parskip}{0pt plus 0.3pt}%
  \let\item\@idxitem
}{%
  \ifkorrekturansicht\clearpage\fi
}
\makeatother

\IfFileExists{\jobname-pw.ind}{\input{\jobname-pw.ind}}{}

% Quellenangabe nur in der Leseansicht
\ifkorrekturansicht\else
% Fallback-Definitionen, falls die .tex-Datei \titel etc. nicht gesetzt hat
\providecommand{\titel}{}
\providecommand{\editorInnen}{}
\providecommand{\dateiname}{\jobname}

\vspace{3cm}

\vfill

\footnotesize
\textsc{Quelle}: \titel. Herausgegeben von {\editorInnen}. In: \emph{Arthur Schnitzler: Briefwechsel mit Autorinnen und Autoren}.
 Digitale Edition, https://schnitzler-briefe.acdh.oeaw.ac.at/{\dateiname}.html (Stand \today)
\fi

\end{document}


