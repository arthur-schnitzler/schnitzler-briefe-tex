%% latex-leseansicht-vorspann.tex
%% Vorspann für die Leseansicht.
%% Lädt die gemeinsame Datei latex-vorspann.tex mit nicht gesetztem Schalter.

\newif\ifkorrekturansicht
\korrekturansichtfalse

\input{../tex-inputs/latex-vorspann}


\section[Olga und Arthur Schnitzler an Richard und Paula Beer-Hofmann, 24. 5. 1914]{L02555 Olga und Arthur Schnitzler an Richard und Paula Beer-Hofmann, 24. 5. 1914}
\nopagebreak\mylabel{L02555v}
\rehead{ }\normalsize\beginnumbering\briefempfaengerindex{Beer-Hofmann, Paula@\textsc{Beer-Hofmann, Paula}!zzzSchnitzler, Arthur@\emph{von Arthur Schnitzler}!1914-05-241@{24. 5. 1914}|(be}\briefempfaengerindex{Beer-Hofmann, Paula@\textsc{Beer-Hofmann, Paula}!zzzSchnitzler, Olga@\emph{von Olga Schnitzler}!1914-05-241@{24. 5. 1914}|(be}\briefempfaengerindex{Beer-Hofmann, Richard@\textsc{Beer-Hofmann, Richard}!zzzSchnitzler, Arthur@\emph{von Arthur Schnitzler}!1914-05-241@{24. 5. 1914}|(be}\briefempfaengerindex{Beer-Hofmann, Richard@\textsc{Beer-Hofmann, Richard}!zzzSchnitzler, Olga@\emph{von Olga Schnitzler}!1914-05-241@{24. 5. 1914}|(be}
\toendnotes[C]{\smallbreak\pagebreak[2]}
\correspDesc{Versand  durch Olga Schnitzler, Arthur Schnitzler am 24. 5. 1914 in Amsterdam
\newline{}Übermittlung  am 25. 5. 1914 in Amsterdam
\newline{}Erhalt  durch Richard Beer-Hofmann, Paula Beer-Hofmann im Zeitraum [25. 5. 1914
                  – 29. 5. 1914?] in Wien}\toendnotes[C]{\smallbreak}
\Standort{YCGL, MSS 31.}
\physDesc{Bildpostkarte, 136 Zeichen
\newline{}Handschrift Arthur Schnitzler: Bleistift, deutsche Kurrent
\newline{}Handschrift Olga Schnitzler: Bleistift, lateinische Kurrent
\newline{}Versand: Stempel: »\nobreak{}\oindex{Amsterdam@\textbf{Amsterdam}, \emph{Hauptstadt}|pwk}Amsterdam, 25 V 1914, 11–12V\nobreak{}«.  }\toendnotes[C]{\smallbreak}\pstart{}{\pb}Herrn u. Frau\pend{}\pstart{}D\textsuperscript{r} Richard Beer-Hofmann \pend{}\pstart{}Wien XVIII\oindex{XVIII., Währing@\textbf{XVIII., Währing}, \emph{Verwaltungsgebiet}|pw}\pend{}\pstart{}Hasenauerstr. 59\oindex{Wien@\textbf{Wien}!XVIII., Währing@\textbf{XVIII., Währing}!Hasenauerstraße 59@\textbf{Hasenauerstraße 59}, \emph{Wohngebäude}|pw}.\pend{}{\bigskip}
\pstart
           \noindent{}\centering{}{\pb}\textcolor{gray}{\textbf{\textsc{Rembrandt van Rijn\pwindex{Rembrandt van Rijn 15.\,7.\,1606 Leiden – 4.\,10.\,1669 Amsterdam@\textsc{Rembrandt van Rijn} (15.\,7.\,1606 Leiden – 4.\,10.\,1669 Amsterdam), \emph{Maler}|pw}}.\hspace*{1.5em}\label{K_L02555-1v}\edtext{Elisabeth Jacobsd. Bas\pwindex{Bol, Ferdinand 24.\,6.\,1616 Dordrecht – 24.\,8.\,1680 Amsterdam@\textsc{Bol, Ferdinand} (24.\,6.\,1616 Dordrecht – 24.\,8.\,1680 Amsterdam), \emph{Maler, Grafiker, Künstler}!Elisabeth Jacobs@\strich\emph{Elisabeth Jacobs}|pw}}{\lemma{\textnormal{\emph{Elisabeth Jacobsd. Bas}}}\Cendnote{\textnormal{Das Werk wird heute Ferdinand Bol\pwindex{Bol, Ferdinand 24.\,6.\,1616 Dordrecht – 24.\,8.\,1680 Amsterdam@\textsc{Bol, Ferdinand} (24.\,6.\,1616 Dordrecht – 24.\,8.\,1680 Amsterdam), \emph{Maler, Grafiker, Künstler}|pwk}
                     zugeschrieben.}}}\label{K_L02555-1}.}}\pend
           \vspace{1em}
\pstart
           {\pb}Sonntag 24. Mai 1914.\pend
           \vspace{0.5em}
\pstart
           Herzliche Grüsse!{\\[\baselineskip]}Ihre\spacefill\mbox{Olga Schnitzler.}\pend
           \leftskip=0em{}
\pstart
           {[}hs. Schnitzler:{]} Herzlichſt{\\}\spacefill\mbox{Arthur}\pend
           \selectlanguage{ngerman}\endnumbering\briefempfaengerindex{Beer-Hofmann, Paula@\textsc{Beer-Hofmann, Paula}!zzzSchnitzler, Arthur@\emph{von Arthur Schnitzler}!1914-05-241@{24. 5. 1914}|)be}\briefempfaengerindex{Beer-Hofmann, Paula@\textsc{Beer-Hofmann, Paula}!zzzSchnitzler, Olga@\emph{von Olga Schnitzler}!1914-05-241@{24. 5. 1914}|)be}\briefempfaengerindex{Beer-Hofmann, Richard@\textsc{Beer-Hofmann, Richard}!zzzSchnitzler, Arthur@\emph{von Arthur Schnitzler}!1914-05-241@{24. 5. 1914}|)be}\briefempfaengerindex{Beer-Hofmann, Richard@\textsc{Beer-Hofmann, Richard}!zzzSchnitzler, Olga@\emph{von Olga Schnitzler}!1914-05-241@{24. 5. 1914}|)be}\mylabel{L02555h}  \newcommand{\dateiname}{L02555}\newcommand{\titel}{Olga und Arthur Schnitzler an Richard und Paula Beer-Hofmann, 24. 5. 1914}\newcommand{\editorInnen}{Martin Anton Müller und Gerd-Hermann Susen}%% latex-leseansicht-abspann.tex
%% Abspann für die Leseansicht.
%% Der Schalter \ifkorrekturansicht ist bereits durch den Vorspann gesetzt.

%% latex-abspann.tex
%% Gemeinsamer Abspann für Korrekturansicht und Leseansicht.
%% Setzt den Schalter \ifkorrekturansicht voraus (gesetzt in den
%% einbindenden Dateien latex-korrekturansicht-abspann.tex bzw.
%% latex-leseansicht-abspann.tex).
%% ---------------------------------------------------------------

\normalsize

% Das esempio-Environment wird nur in der Leseansicht benötigt
\ifkorrekturansicht\else
\newenvironment{esempio}[3]%
{
    \vspace{1.5ex}
    \rlap{\underline{#1}}
    \par
    \setlength{\parindent}{0cm}
    \nopagebreak
    \leftskip=#2cm
    \rightskip=#3cm
}
{
    \par
}
\fi

\doendnotes{C}
\bigskip
\vfill

\clearpage

\footnotesize

\ifkorrekturansicht
  \lohead{\textsc{register}}
\fi

% theindex-Environment neu definieren ohne reledmac
\makeatletter
\renewenvironment{theindex}{%
  \ifkorrekturansicht
    \section*{\indexname}%
  \else
    \subsubsection*{Index der erwähnten Entitäten}%
  \fi
  \setlength{\parindent}{0pt}%
  \setlength{\parskip}{0pt plus 0.3pt}%
  \let\item\@idxitem
}{%
  \ifkorrekturansicht\clearpage\fi
}
\makeatother

\IfFileExists{\jobname-pw.ind}{\input{\jobname-pw.ind}}{}

% Quellenangabe nur in der Leseansicht
\ifkorrekturansicht\else
% Fallback-Definitionen, falls die .tex-Datei \titel etc. nicht gesetzt hat
\providecommand{\titel}{}
\providecommand{\editorInnen}{}
\providecommand{\dateiname}{\jobname}

\vspace{3cm}

\vfill

\footnotesize
\textsc{Quelle}: \titel. Herausgegeben von {\editorInnen}. In: \emph{Arthur Schnitzler: Briefwechsel mit Autorinnen und Autoren}.
 Digitale Edition, https://schnitzler-briefe.acdh.oeaw.ac.at/{\dateiname}.html (Stand \today)
\fi

\end{document}


