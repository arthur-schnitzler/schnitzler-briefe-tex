%% latex-leseansicht-vorspann.tex
%% Vorspann für die Leseansicht.
%% Lädt die gemeinsame Datei latex-vorspann.tex mit nicht gesetztem Schalter.

\newif\ifkorrekturansicht
\korrekturansichtfalse

\input{../tex-inputs/latex-vorspann}

\begin{center}
            \textcolor{red}{ENTWURF, NICHT FERTIG KORRIGIERT}
                      \end{center}
            
         
         \renewcommand{\erwaehntePersonen}{Personen: Peter Cornelius, Paul Marx, Olga Schnitzler, Elisabeth Steinrück}
         \renewcommand{\erwaehnteOrte}{Orte: Berlin, Dessauer Straße, Wien}
         \renewcommand{\erwaehnteWerke}{Werke: Brautlieder}
               \section[ Paul Goldmann an Olga Gussmann, 1. 4. {[}1901{]}]{ Paul Goldmann an Olga Gussmann, 1. 4. {[}1901{]}}\nopagebreak\mylabel{v}\rehead{ }\begin{ledgroupsized}[t]{13cm}\normalsize\beginnumbering \toendnotes[C]{\smallbreak\pagebreak[2]} \Standort{DLA, A:Schnitzler, HS.NZ85.1.5247.}
\physDesc{Brief, 1 Blatt, 4 Seiten
\newline{}Handschrift: blaue Tinte, deutsche Kurrent\newline{}Ordnung: mit Bleistift von Arthur
                                    Schnitzler\pwindex{Schnitzler, Arthur 15.05.1862 – 21.10.1931@\textsc{Schnitzler, Arthur} (15.05.1862 – 21.10.1931), \emph{Schriftsteller, Mediziner}|pw} das Jahr »1901« vermerkt }\toendnotes[C]{\smallbreak}\pstart
           \noindent{}\raggedleft{}{\pb}\textcolor{gray}{\textbf{DESSAUERSTRASSE 19\oindex{Dessauer Strasse@\textbf{Dessauer Straße}|pw}}}\pend
           \pstart
           Berlin\oindex{Berlin@\textbf{Berlin}|pw}, 1. April.\pend
           \pstart\center{}Liebes Fräulein \textsc{Olga},\pend\pstart
           Endlich einmal eine freie Stunde, nach arbeitsſchweren Tagen. Heut will ich erſt Ihren lieben Brief beantworten. Das Schweſterchen\pwindex{Steinrueck, Elisabeth 19.11.1885 – 07.04.1920@\textsc{Steinrück, Elisabeth} (19.11.1885 – 07.04.1920)|pwv} kommt nächſtens an die
               Reihe.\pend
           \pstart
           Dieſer Ihr Brief war alſo ſehr ſchön. Ich ſage ihnen, es thut wohl, ein wenig
               geſtreichelt zu werden, – namentlich wenn man in dieſer Beziehung gar nicht, aber
               auch ſchon gar nicht verwöhnt iſt. Und doch, er kam ein wenig zu ſpät, dieſer Brief.
               Ich merke gar zu deutlich, daß mein lieber Freund \textsc{Arthur\pwindex{Schnitzler, Arthur 15.05.1862 – 21.10.1931@\textsc{Schnitzler, Arthur} (15.05.1862 – 21.10.1931), \emph{Schriftsteller, Mediziner}|pw}} hinter den Couliſſen die \label{K_L03524-1v}\edtext{Regie}{\lemma{\textnormal{\emph{Regie}}}\Cendnote{\textnormal{Goldmann\pwindex{Goldmann, Paul 31.01.1865 – 25.09.1935@\textsc{Goldmann, Paul} (31.01.1865 – 25.09.1935), \emph{Schriftsteller, Journalist}|pwk} hatte sich jedenfalls von 6. 9. 1900 bis 16. 9. [1900] in Wien\oindex{Wien@\textbf{Wien}|pwk} aufgehalten. Am 6. 9. 1900, 8. 9. 1900, 9. 9. 1900 und 13. 9. 1900 hatten Goldmann\pwindex{Goldmann, Paul 31.01.1865 – 25.09.1935@\textsc{Goldmann, Paul} (31.01.1865 – 25.09.1935), \emph{Schriftsteller, Journalist}|pwk}, Schnitzler\pwindex{Schnitzler, Arthur 15.05.1862 – 21.10.1931@\textsc{Schnitzler, Arthur} (15.05.1862 – 21.10.1931), \emph{Schriftsteller, Mediziner}|pwk} und Olga Gussmann\pwindex{Schnitzler, Olga 17.01.1882 – 13.01.1970@\textsc{Schnitzler, Olga} (17.01.1882 – 13.01.1970), \emph{Schauspielerin, Sängerin}|pwk}
                  gemeinsam Zeit verbracht.}}}\label{K_L03524-1h} führt. Ich habe ſchon aus \label{T_L03524-1v}\edtext{Wien\oindex{Wien@\textbf{Wien}|pw}}{\lemma{\textnormal{\emph{Wien}}}\Cendnote{\textnormal{korrigiert aus »den Wien\oindex{Wien@\textbf{Wien}|pw}«}}}\label{T_L03524-1h}{ }{\pb}den Eindruck mitgebracht, \strikeout{\textcolor{gray}{×}\-\textcolor{gray}{×}\-\textcolor{gray}{×}} daß Sie auf mich nur aufmerkſam geworden ſind, weil ich Ihnen an der Seite
               dieſes meines lieben Freund\pwindex{Schnitzler, Arthur 15.05.1862 – 21.10.1931@\textsc{Schnitzler, Arthur} (15.05.1862 – 21.10.1931), \emph{Schriftsteller, Mediziner}|pwv}es
               erſchienen bin. Sonſt wären Sie wahrſcheinlich an mir vorübergegangen, ohne mich zu
               ſehen. Ihre Briefe haben mir die Wahrnehmung beſtätigt. Natürlich werden Sie jetzt
               proteſtiren. Aber, glauben Sie mir, ich kenne ſo gut den Ton, den Diejenigen
               annehmen, die Einen verkennen. Ich höre ihn mit ſcharfem Ohr ſelbſt aus der
               Freundſchaft heraus. Ich bin ein Fachmann im Verkanntwerden.\pend
           \pstart
           Und da ich müde bin, immer wieder das ſelbe zu erleben, ſelbſt bei den ganz Klugen
               (es gibt kluge Frauen, die doch {\pb}nur Denjenigen richtig beurtheilen, den ſie lieben), ſo habe ich Ihnen vielleicht
               nicht ſo oft geſchrieben, als ich es hätte thun ſollen. Das iſt aber keine Behandlung
                  »\label{K_L03524-2v}\edtext{\textsc{\begin{otherlanguage}{french}en canaille\end{otherlanguage}}}{\lemma{\textnormal{\emph{en canaille}}}\Cendnote{\textnormal{französisch: verachtend}}}\label{K_L03524-2h}«,
               wahrhaftig nicht. Mit der Freundſchaſt hat das gar nichts zu thun. Ich will mit der
               Freundſchaft keine Geſchäfte machen, und es iſt mir ein ſehr feines und ein wenig
               weiſches Vergnügen, mehr geben zu können, als ich bekomme.\pend
           \pstart
           Vielleicht hätte ich das, was ich Ihnen, liebes Fräulein \textsc{Olga}, da erzählt habe, gar nicht geſpürt, wenn ich \strikeout{\textcolor{gray}{×}} nur einen einzigen Menſchen hätte (ſtatt Menſch iſt natürlich »Frau« zu
               leſen), der ſich für mich intereſſirt und der mich lieb hat. Aber ich habe {\pb}Niemand. So ſitze ich in der
               Einſamkeit und fange Grillen. Dieſer Brief iſt nichts als eine große gefangene
               Grille. Sie werden ihn als ſolche{[}n{]} behandeln und darüber lachen.
               Aber jetzt, wo mir die Oſterſonne zum Fenſter
               hereinſcheint, wird es \uline{gar} ſchlimm. Die dumme Frage
               regt ſich wieder, warum es für die ganze Welt Frühling wird und warum ich allein
               davon ausgenommen ſein ſoll? Es iſt ſchwer, ſeinen Gleichmuth zu bewahren, wenn einem
               ſo eine Frage im Kopfe rumort.\pend
           \pstart Liebes Fräulein ich möchte wiſſen, wie es Ihnen und dem lieben Schweſterchen\pwindex{Steinrueck, Elisabeth 19.11.1885 – 07.04.1920@\textsc{Steinrück, Elisabeth} (19.11.1885 – 07.04.1920)|pwv} geht. Und die \label{K_L03524-3v}\edtext{Brautlieder\pwindex{Cornelius, Peter 24.12.1824 – 26.10.1874@\textsc{Cornelius, Peter} (24.12.1824 – 26.10.1874), \emph{Komponist}!Brautlieder1856@\strich\emph{Brautlieder} {[}1856{]}|pwv} von \textsc{Cornelius\pwindex{Cornelius, Peter 24.12.1824 – 26.10.1874@\textsc{Cornelius, Peter} (24.12.1824 – 26.10.1874), \emph{Komponist}|pw}}}{\lemma{\textnormal{\emph{Brautlieder von Cornelius}}}\Cendnote{\textnormal{eine Komposition\pwindex{Cornelius, Peter 24.12.1824 – 26.10.1874@\textsc{Cornelius, Peter} (24.12.1824 – 26.10.1874), \emph{Komponist}!Brautlieder1856@\strich\emph{Brautlieder} {[}1856{]}|pwkv} aus dem Jahr 1856,
                  die Olga Gussmann\pwindex{Schnitzler, Olga 17.01.1882 – 13.01.1970@\textsc{Schnitzler, Olga} (17.01.1882 – 13.01.1970), \emph{Schauspielerin, Sängerin}|pwk} vermutlich gesanglich
                  einübte}}}\label{K_L03524-3h} möchte ich auch wohl einmal hören. Schreiben Sie mir bald wieder!
               Und fröhliche Oſtern! Ihr \spacefill\mbox{Dr. Paul
                  Goldmann.}\pend{}\pstart
           \noindent{}{\pb}\label{T_L03524-3v}\edtext{Herzliche Grüße an Sie Beide\pwindex{Schnitzler, Arthur 15.05.1862 – 21.10.1931@\textsc{Schnitzler, Arthur} (15.05.1862 – 21.10.1931), \emph{Schriftsteller, Mediziner}|pw} und an Herrn \label{K_L03524-4v}\edtext{\textsc{Paul\pwindex{Marx, Paul 21.07.1879 – 1956-10-30@\textsc{Marx, Paul} (21.07.1879 – 1956-10-30), \emph{Regisseur, Schauspieler}|pw}}}{\lemma{\textnormal{\emph{Paul}}}\Cendnote{\textnormal{Paul Mosé\pwindex{Marx, Paul 21.07.1879 – 1956-10-30@\textsc{Marx, Paul} (21.07.1879 – 1956-10-30), \emph{Regisseur, Schauspieler}|pwk} war zwischen 1900 und 1903 der Partner von Olga\pwindex{Schnitzler, Olga 17.01.1882 – 13.01.1970@\textsc{Schnitzler, Olga} (17.01.1882 – 13.01.1970), \emph{Schauspielerin, Sängerin}|pwk}s Schwester Elisabeth Gussmann\pwindex{Steinrueck, Elisabeth 19.11.1885 – 07.04.1920@\textsc{Steinrück, Elisabeth} (19.11.1885 – 07.04.1920)|pwk}.}}}\label{K_L03524-4h}!}{\lemma{\textnormal{\emph{Herzliche … Paul!}}}\Cendnote{\textnormal{am oberen Rand der ersten Seite, verkehrt zum Text}}}\label{T_L03524-3h}\pend
           
         
         \endnumbering\mylabel{h}\end{ledgroupsized}\begin{anhang}\end{anhang}\newcommand{\dateiname}{L03524}\newcommand{\titel}{Paul Goldmann an Olga Gussmann, 1. 4. [1901]}\newcommand{\editorInnen}{Martin Anton Müller und Laura Untner}%% latex-leseansicht-abspann.tex
%% Abspann für die Leseansicht.
%% Der Schalter \ifkorrekturansicht ist bereits durch den Vorspann gesetzt.

%% latex-abspann.tex
%% Gemeinsamer Abspann für Korrekturansicht und Leseansicht.
%% Setzt den Schalter \ifkorrekturansicht voraus (gesetzt in den
%% einbindenden Dateien latex-korrekturansicht-abspann.tex bzw.
%% latex-leseansicht-abspann.tex).
%% ---------------------------------------------------------------

\normalsize

% Das esempio-Environment wird nur in der Leseansicht benötigt
\ifkorrekturansicht\else
\newenvironment{esempio}[3]%
{
    \vspace{1.5ex}
    \rlap{\underline{#1}}
    \par
    \setlength{\parindent}{0cm}
    \nopagebreak
    \leftskip=#2cm
    \rightskip=#3cm
}
{
    \par
}
\fi

\doendnotes{C}
\bigskip
\vfill

\clearpage

\footnotesize

\ifkorrekturansicht
  \lohead{\textsc{register}}
\fi

% theindex-Environment neu definieren ohne reledmac
\makeatletter
\renewenvironment{theindex}{%
  \ifkorrekturansicht
    \section*{\indexname}%
  \else
    \subsubsection*{Index der erwähnten Entitäten}%
  \fi
  \setlength{\parindent}{0pt}%
  \setlength{\parskip}{0pt plus 0.3pt}%
  \let\item\@idxitem
}{%
  \ifkorrekturansicht\clearpage\fi
}
\makeatother

\IfFileExists{\jobname-pw.ind}{\input{\jobname-pw.ind}}{}

% Quellenangabe nur in der Leseansicht
\ifkorrekturansicht\else
% Fallback-Definitionen, falls die .tex-Datei \titel etc. nicht gesetzt hat
\providecommand{\titel}{}
\providecommand{\editorInnen}{}
\providecommand{\dateiname}{\jobname}

\vspace{3cm}

\vfill

\footnotesize
\textsc{Quelle}: \titel. Herausgegeben von {\editorInnen}. In: \emph{Arthur Schnitzler: Briefwechsel mit Autorinnen und Autoren}.
 Digitale Edition, https://schnitzler-briefe.acdh.oeaw.ac.at/{\dateiname}.html (Stand \today)
\fi

\end{document}


      