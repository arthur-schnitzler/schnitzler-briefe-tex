%% latex-leseansicht-vorspann.tex
%% Vorspann für die Leseansicht.
%% Lädt die gemeinsame Datei latex-vorspann.tex mit nicht gesetztem Schalter.

\newif\ifkorrekturansicht
\korrekturansichtfalse

\input{../tex-inputs/latex-vorspann}


\section[ Paul Goldmann an Olga Gussmann, 1. 4. [1901]]{L03524 Paul Goldmann an Olga Gussmann,  1. 4. [1901]}
\nopagebreak\mylabel{L03524v}
\rehead{ }\normalsize\beginnumbering\briefempfaengerindex{Schnitzler, Olga@\textsc{Schnitzler, Olga}!zzzGoldmann, Paul@\emph{von Paul Goldmann}!1901-04-012@{1. 4. [1901]}|(be}
\toendnotes[C]{\smallbreak\pagebreak[2]}
\correspDesc{Versand  durch Paul Goldmann am 1. 4. [1901] in Berlin
\newline{}Erhalt  durch Olga Gussmann am [2. 4. 1901?] in Wien?}\toendnotes[C]{\smallbreak}
\Standort{DLA, A:Schnitzler, HS.NZ85.1.5247.}
\physDesc{Brief, 1 Blatt, 4 Seiten, 2579 Zeichen
\newline{}Handschrift: blaue Tinte, deutsche Kurrent
\newline{}Ordnung: mit Bleistift von Arthur Schnitzler das
                                 Jahr »1901« vermerkt }\toendnotes[C]{\smallbreak}
\pstart
           \raggedleft{}{\pb}\textcolor{gray}{\textbf{DESSAUERSTRASSE 19\oindex{Dessauer Straße@\textbf{Dessauer Straße}, \emph{Straße}|pw}}}\pend
           
\pstart
           Berlin\oindex{Berlin@\textbf{Berlin}, \emph{Hauptstadt}|pw}, 1. April.\pend
           
\pstart\center{}Liebes Fräulein \textsc{Olga},\pend\vspace{0.5em}
\pstart
           Endlich einmal eine freie Stunde, nach arbeitsſchweren Tagen. Heut will ich erſt Ihren lieben Brief beantworten. Das Schweſterchen\pwindex{Steinrück, Elisabeth 19.\,11.\,1885 – 7.\,4.\,1920 Partenkirchen@\textsc{Steinrück, Elisabeth} (19.\,11.\,1885 – 7.\,4.\,1920 Partenkirchen)|pwv} kommt nächſtens an die
               Reihe.\pend
           
\pstart
           Dieſer Ihr Brief war alſo{ }ſehr{ }ſchön. Ich{ }ſage Ihnen, es thut wohl, ein wenig
               geſtreichelt zu werden, – namentlich wenn man in dieſer Beziehung gar nicht, aber
               auch{ }ſchon gar nicht verwöhnt iſt. Und doch, er kam ein wenig zu{ }ſpät, dieſer Brief.
               Ich merke gar zu deutlich, daß mein lieber Freund \textsc{Arthur} hinter den Couliſſen die Regie führt. Ich habe{ }ſchon aus den 
               Wien\oindex{Wien@\textbf{Wien}, \emph{Verwaltungsgebiet}|pw}er {[}Begegnungen{]}{ }{\pb}den Eindruck mitgebracht, \strikeout{\textcolor{gray}{×}\-\textcolor{gray}{×}\-\textcolor{gray}{×}} daß Sie auf mich nur aufmerkſam geworden{ }ſind, weil ich Ihnen an der Seite
               dieſes meines lieben Freundes
               erſchienen bin. Sonſt wären Sie wahrſcheinlich an mir vorübergegangen, ohne mich zu{ }ſehen. Ihre Briefe haben mir die Wahrnehmung beſtätigt. Natürlich werden Sie jetzt
               proteſtiren. Aber, glauben Sie mir, ich kenne{ }ſo gut den Ton, den Diejenigen
               annehmen, die Einen verkennen. Ich höre ihn mit{ }ſcharfem Ohr{ }ſelbſt aus der
               Freundſchaft heraus. Ich bin ein Fachmann im Verkanntwerden.\pend
           
\pstart
           Und da ich müde bin, immer wieder dasſelbe zu erleben,{ }ſelbſt bei den ganz Klugen (es
               gibt kluge Frauen, die doch {\pb}nur Denjenigen richtig
               beurtheilen, den{ }ſie lieben),{ }ſo habe ich Ihnen vielleicht nicht{ }ſo oft geſchrieben,
               als ich es hätte thun{ }ſollen. Das iſt aber keine Behandlung »\label{K_L03524-1v}\edtext{\textsc{\begin{otherlanguage}{french}en canaille\end{otherlanguage}}}{\lemma{\textnormal{\emph{en canaille}}}\Cendnote{\textnormal{französisch: verachtend}}}\label{K_L03524-1}«,
               wahrhaftig nicht. Mit der Freundſchaft hat das gar nichts zu thun. Ich will mit der
               Freundſchaft keine Geſchäfte machen; und es iſt mir ein{ }ſehr feines und ein wenig
               ironiſches Vergnügen, mehr geben zu können, als ich bekomme.\pend
           
\pstart
           Vielleicht hätte ich das, was ich Ihnen, liebes Fräulein \textsc{Olga}, da erzählt habe, gar nicht geſpürt, wenn ich \strikeout{\textcolor{gray}{×}} nur einen einzigen Menſchen hätte (ſtatt Menſch iſt natürlich »Frau« zu
               leſen), der{ }ſich für mich intereſſirt und der mich lieb hat. Aber ich habe {\pb}Niemand. So{ }ſitze ich in der Einſamkeit und fange
               Grillen. Dieſer Brief iſt nichts als eine große\textcolor{gray}{,} gefangene Grille.
               Sie werden ihn als{ }ſolche behandeln und darüber lachen. Aber jetzt, wo mir die Oſterſonne zum Fenſter hereinſcheint, wird es \uline{gar}{ }ſchlimm. Die dumme Frage regt{ }ſich wieder, warum
               es für die ganze Welt Frühling wird und warum ich allein davon ausgenommen{ }ſein{ }ſoll?
               Es iſt{ }ſchwer,{ }ſeinen Gleichmuth zu bewahren, wenn einem{ }ſo eine Frage im Kopfe
               rumort.\pend
           
\pstart
           Liebes Fräulein, ich möchte wiſſen, wie es Ihnen und dem lieben Schweſterchen\pwindex{Steinrück, Elisabeth 19.\,11.\,1885 – 7.\,4.\,1920 Partenkirchen@\textsc{Steinrück, Elisabeth} (19.\,11.\,1885 – 7.\,4.\,1920 Partenkirchen)|pwv} geht. Und die \label{K_L03524-2v}\edtext{Brautlieder\pwindex{Cornelius, Peter 24.\,12.\,1824 Mainz – 26.\,10.\,1874 ebd.@\textsc{Cornelius, Peter} (24.\,12.\,1824 Mainz – 26.\,10.\,1874 ebd.), \emph{Komponist}!Brautlieder@\strich\emph{Brautlieder}|pwv} von \textsc{Cornelius\pwindex{Cornelius, Peter 24.\,12.\,1824 Mainz – 26.\,10.\,1874 ebd.@\textsc{Cornelius, Peter} (24.\,12.\,1824 Mainz – 26.\,10.\,1874 ebd.), \emph{Komponist}|pw}}}{\lemma{\textnormal{\emph{Brautlieder von Cornelius}}}\Cendnote{\textnormal{eine Komposition\pwindex{Cornelius, Peter 24.\,12.\,1824 Mainz – 26.\,10.\,1874 ebd.@\textsc{Cornelius, Peter} (24.\,12.\,1824 Mainz – 26.\,10.\,1874 ebd.), \emph{Komponist}!Brautlieder@\strich\emph{Brautlieder}|pwkv} aus dem Jahr 1856,
                  die Olga Gussmann\pwindex{Schnitzler, Olga 17.\,1.\,1882 Wien – 13.\,1.\,1970 Lugano@\textsc{Schnitzler, Olga} (17.\,1.\,1882 Wien – 13.\,1.\,1970 Lugano), \emph{Schauspielerin, Sängerin}|pwk} vermutlich gesanglich
                  einstudierte}}}\label{K_L03524-2} möchte ich auch wohl einmal hören. Schreiben Sie mir bald wieder!
               Und fröhliche Oſtern!\pend
           \pstart Ihr \spacefill\mbox{Dr. Paul Goldmann}\pend{}
\pstart
           \noindent{}{\pb}\label{T_L03524-1v}\edtext{Herzliche Grüße an Sie Beide\pwindex{Steinrück, Elisabeth 19.\,11.\,1885 – 7.\,4.\,1920 Partenkirchen@\textsc{Steinrück, Elisabeth} (19.\,11.\,1885 – 7.\,4.\,1920 Partenkirchen)|pwv} und an Herrn \textsc{Paul\pwindex{Marx, Paul 21.\,7.\,1879 Wien – 30.\,10.\,1956 ebd.@\textsc{Marx, Paul} (21.\,7.\,1879 Wien – 30.\,10.\,1956 ebd.), \emph{Regisseur, Schauspieler}|pw}}!}{\lemma{\textnormal{\emph{Herzliche … Paul!}}}\Cendnote{\textnormal{am Kopf der ersten Seite,
                     verkehrt zum Text}}}\label{T_L03524-1}\pend
           \selectlanguage{ngerman}\endnumbering\briefempfaengerindex{Schnitzler, Olga@\textsc{Schnitzler, Olga}!zzzGoldmann, Paul@\emph{von Paul Goldmann}!1901-04-012@{1. 4. [1901]}|)be}\mylabel{L03524h}  \newcommand{\dateiname}{L03524}\newcommand{\titel}{Paul Goldmann an Olga Gussmann, 1. 4. [1901]}\newcommand{\editorInnen}{Martin Anton Müller und Laura Untner}%% latex-leseansicht-abspann.tex
%% Abspann für die Leseansicht.
%% Der Schalter \ifkorrekturansicht ist bereits durch den Vorspann gesetzt.

%% latex-abspann.tex
%% Gemeinsamer Abspann für Korrekturansicht und Leseansicht.
%% Setzt den Schalter \ifkorrekturansicht voraus (gesetzt in den
%% einbindenden Dateien latex-korrekturansicht-abspann.tex bzw.
%% latex-leseansicht-abspann.tex).
%% ---------------------------------------------------------------

\normalsize

% Das esempio-Environment wird nur in der Leseansicht benötigt
\ifkorrekturansicht\else
\newenvironment{esempio}[3]%
{
    \vspace{1.5ex}
    \rlap{\underline{#1}}
    \par
    \setlength{\parindent}{0cm}
    \nopagebreak
    \leftskip=#2cm
    \rightskip=#3cm
}
{
    \par
}
\fi

\doendnotes{C}
\bigskip
\vfill

\clearpage

\footnotesize

\ifkorrekturansicht
  \lohead{\textsc{register}}
\fi

% theindex-Environment neu definieren ohne reledmac
\makeatletter
\renewenvironment{theindex}{%
  \ifkorrekturansicht
    \section*{\indexname}%
  \else
    \subsubsection*{Index der erwähnten Entitäten}%
  \fi
  \setlength{\parindent}{0pt}%
  \setlength{\parskip}{0pt plus 0.3pt}%
  \let\item\@idxitem
}{%
  \ifkorrekturansicht\clearpage\fi
}
\makeatother

\IfFileExists{\jobname-pw.ind}{\input{\jobname-pw.ind}}{}

% Quellenangabe nur in der Leseansicht
\ifkorrekturansicht\else
% Fallback-Definitionen, falls die .tex-Datei \titel etc. nicht gesetzt hat
\providecommand{\titel}{}
\providecommand{\editorInnen}{}
\providecommand{\dateiname}{\jobname}

\vspace{3cm}

\vfill

\footnotesize
\textsc{Quelle}: \titel. Herausgegeben von {\editorInnen}. In: \emph{Arthur Schnitzler: Briefwechsel mit Autorinnen und Autoren}.
 Digitale Edition, https://schnitzler-briefe.acdh.oeaw.ac.at/{\dateiname}.html (Stand \today)
\fi

\end{document}


