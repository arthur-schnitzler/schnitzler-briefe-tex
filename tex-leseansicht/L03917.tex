%% latex-leseansicht-vorspann.tex
%% Vorspann für die Leseansicht.
%% Lädt die gemeinsame Datei latex-vorspann.tex mit nicht gesetztem Schalter.

\newif\ifkorrekturansicht
\korrekturansichtfalse

\input{../tex-inputs/latex-vorspann}


\section[Arthur Schnitzler an Theodor Herzl, {[}23. oder 24. 4. 1889?{]}]{L03917 Arthur Schnitzler an Theodor Herzl, {[}23. oder 24. 4. 1889?{]}}
\nopagebreak\mylabel{L03917v}
\rehead{ }\normalsize\beginnumbering\briefempfaengerindex{Herzl, Theodor@\textsc{Herzl, Theodor}!zzzSchnitzler, Arthur@\emph{von Arthur Schnitzler}!1889-04-241@{{[}23. oder 24. 4. 1889?{]}}|(be}
\toendnotes[C]{\smallbreak\pagebreak[2]}
\correspDesc{Versand  durch Arthur Schnitzler im Zeitraum [23. oder 24. 4. 1889?] in Wien
\newline{}Erhalt  durch Theodor Herzl im Zeitraum [23. oder 24. 4. 1889?] in Wien}\toendnotes[C]{\smallbreak}
\Standort{Jerusalem, Central Zionist Archives, H1:1926-7.}
\physDesc{Briefkarte, 665 Zeichen
\newline{}Handschrift: schwarze Tinte, deutsche Kurrent
\newline{}Ordnung: mit Bleistift von unbekannter Hand nummeriert: »(44) 1« }\toendnotes[C]{\smallbreak}
\pstart
           \noindent{}{\pb}Verehrteſter Herr Doktor! Ich erlaube mir heute eine Bitte an Sie!
               Sind Sie in der Lage, \label{K_L03917-1v}\edtext{Karten zur Judic\pwindex{Judic, Anna 19.\,7.\,1849 Semur-en-Auxois – 15.\,4.\,1911 Golfe-Juan@\textsc{Judic, Anna} (19.\,7.\,1849 Semur-en-Auxois – 15.\,4.\,1911 Golfe-Juan), \emph{Schauspielerin, Sängerin}|pw}}{\lemma{\textnormal{\emph{Karten zur Judic}}}\Cendnote{\textnormal{Das
               Korrespondenzstück ist undatiert. Die förmliche Anrede und Schlussformel verorten es eindeutig in der frühen Zeit der Bekanntschaft. In
               dieser Phase ist Anna Judic\pwindex{Judic, Anna 19.\,7.\,1849 Semur-en-Auxois – 15.\,4.\,1911 Golfe-Juan@\textsc{Judic, Anna} (19.\,7.\,1849 Semur-en-Auxois – 15.\,4.\,1911 Golfe-Juan), \emph{Schauspielerin, Sängerin}|pwk} nur zwischen 21. 4. 1889 und 
                  Sonntag, 5. 5. 1889 zu einem Gastspiel in Wien\oindex{Wien@\textbf{Wien}, \emph{Verwaltungsgebiet}|pwk}. Schnitzler
                  selbst besuchte am Mittwoch, den 1. 5. 1889 eine ihrer Aufführungen von 
                  \emph{Fiacre 117}\pwindex{\textcolor{red}{\textsuperscript{XXXX indx1}}!Le Fiacre 117. Comédie en trois actes@\strich\emph{Le Fiacre 117. Comédie en trois actes}|pwk}\pwindex{\textcolor{red}{\textsuperscript{XXXX indx1}}!Le Fiacre 117. Comédie en trois actes@\strich\emph{Le Fiacre 117. Comédie en trois actes}|pwk} und \emph{Les Charbonniers}\pwindex{\textcolor{red}{\textsuperscript{XXXX indx1}}!Charbonniers. Opérette en un acte@\strich\emph{Les Charbonniers. Opérette en un acte}|pwk}. Das \emph{Tagebuch}\pwindex{Schnitzler, Arthur 15.\,5.\,1862 Wien – 21.\,10.\,1931 ebd.@\textsc{Schnitzler, Arthur} (15.\,5.\,1862 Wien – 21.\,10.\,1931 ebd.), \emph{Schriftsteller, Mediziner}!Tagebuch@\strich\emph{Tagebuch}|pwk}
                  enthält für diese Tage keine Einträge. Am Samstag, dem 4. 5. 1889
                  wurde am \emph{Burgtheater}\orgindex{Burgtheater@Burgtheater|pwk} (unter anderem) \emph{Der Flüchtling}\pwindex{Herzl, Theodor 2.\,5.\,1860 Budapest – 3.\,7.\,1904 Edlach@\textsc{Herzl, Theodor} (2.\,5.\,1860 Budapest – 3.\,7.\,1904 Edlach), \emph{Schriftsteller, Journalist}!Flüchtling. Lustspiel in einem Aufzug@\strich\emph{Der Flüchtling. Lustspiel in einem Aufzug}|pwk}
                  von Herzl\pwindex{Herzl, Theodor 2.\,5.\,1860 Budapest – 3.\,7.\,1904 Edlach@\textsc{Herzl, Theodor} (2.\,5.\,1860 Budapest – 3.\,7.\,1904 Edlach), \emph{Schriftsteller, Journalist}|pwk} uraufgeführt. Schnitzler besuchte die 
                  Vorstellung und es scheint schwer vorstellbar, dass er sich von Herzl\pwindex{Herzl, Theodor 2.\,5.\,1860 Budapest – 3.\,7.\,1904 Edlach@\textsc{Herzl, Theodor} (2.\,5.\,1860 Budapest – 3.\,7.\,1904 Edlach), \emph{Schriftsteller, Journalist}|pwk} just für den Tag
                  eine Theaterkarte erbitten würde, für den Herzls\pwindex{Herzl, Theodor 2.\,5.\,1860 Budapest – 3.\,7.\,1904 Edlach@\textsc{Herzl, Theodor} (2.\,5.\,1860 Budapest – 3.\,7.\,1904 Edlach), \emph{Schriftsteller, Journalist}|pwk}{ }Stück\pwindex{Herzl, Theodor 2.\,5.\,1860 Budapest – 3.\,7.\,1904 Edlach@\textsc{Herzl, Theodor} (2.\,5.\,1860 Budapest – 3.\,7.\,1904 Edlach), \emph{Schriftsteller, Journalist}!Flüchtling. Lustspiel in einem Aufzug@\strich\emph{Der Flüchtling. Lustspiel in einem Aufzug}|pwkv} angesetzt war. Folglich dürfte Schnitzler 
                  um Karten für das vorletzte Wochenende (Donnerstag, 25. bis Sonntag, 28.) bitten. Da wiederum die ersten Rezensionen
                  des Gastspiels bereits erschienen waren, dürfte das Korrespondenzstück auf den 23. 4. 1889 oder
                  24. 4. 1889 zu datieren sein.}}}\label{K_L03917-1} zu
               verſchenken? Ich habe den Muth zu dieſer Frage, indem ich leſe, daſs es
               leer ſein ſoll. Im Falle Sie alſo eine gewiſſe Verfügungsmöglichkeit haben, verbinden
                  {\pb}Sie mich außerordentlich, we{\geminationn} Sie mir für einen oder
               den anderen Abend ein oder zwei \introOben{}\textsc{Parquet-}\introOben{}Sitze verſchaffen können. Beſonders ſympathiſch wäre mir \uline{Do{\geminationn}erstag}{ }\uline{Samstag} oder \uline{So{\geminationn}tag}. Nicht wahr Sie ſagen mir gefälligſt telephoniſch zwiſchen 2 und 3 Uhr \introOben{}–
                  (\textsc{event} bis 5 od 6)\introOben{} oder durch eine Karte Antwort? Und ſind nicht ungehalten
               über mein Erſuchen?\pend
           \pstart Mit herzlichem Gruße und Dank Ihr ergebner
               \spacefill\mbox{Dr. Arthur Schnitzler}\pend{}\selectlanguage{ngerman}\endnumbering\briefempfaengerindex{Herzl, Theodor@\textsc{Herzl, Theodor}!zzzSchnitzler, Arthur@\emph{von Arthur Schnitzler}!1889-04-231@{{[}23. oder 24. 4. 1889?{]}}|)be}\mylabel{L03917h}
\begin{anhang}
\end{anhang}\newcommand{\dateiname}{L03917}\newcommand{\titel}{Arthur Schnitzler an Theodor Herzl, [23. oder 24. 4. 1889?]}\newcommand{\editorInnen}{Herausgegeben von Jahnke, SelmaMüller, Martin Anton}%% latex-leseansicht-abspann.tex
%% Abspann für die Leseansicht.
%% Der Schalter \ifkorrekturansicht ist bereits durch den Vorspann gesetzt.

%% latex-abspann.tex
%% Gemeinsamer Abspann für Korrekturansicht und Leseansicht.
%% Setzt den Schalter \ifkorrekturansicht voraus (gesetzt in den
%% einbindenden Dateien latex-korrekturansicht-abspann.tex bzw.
%% latex-leseansicht-abspann.tex).
%% ---------------------------------------------------------------

\normalsize

% Das esempio-Environment wird nur in der Leseansicht benötigt
\ifkorrekturansicht\else
\newenvironment{esempio}[3]%
{
    \vspace{1.5ex}
    \rlap{\underline{#1}}
    \par
    \setlength{\parindent}{0cm}
    \nopagebreak
    \leftskip=#2cm
    \rightskip=#3cm
}
{
    \par
}
\fi

\doendnotes{C}
\bigskip
\vfill

\clearpage

\footnotesize

\ifkorrekturansicht
  \lohead{\textsc{register}}
\fi

% theindex-Environment neu definieren ohne reledmac
\makeatletter
\renewenvironment{theindex}{%
  \ifkorrekturansicht
    \section*{\indexname}%
  \else
    \subsubsection*{Index der erwähnten Entitäten}%
  \fi
  \setlength{\parindent}{0pt}%
  \setlength{\parskip}{0pt plus 0.3pt}%
  \let\item\@idxitem
}{%
  \ifkorrekturansicht\clearpage\fi
}
\makeatother

\IfFileExists{\jobname-pw.ind}{\input{\jobname-pw.ind}}{}

% Quellenangabe nur in der Leseansicht
\ifkorrekturansicht\else
% Fallback-Definitionen, falls die .tex-Datei \titel etc. nicht gesetzt hat
\providecommand{\titel}{}
\providecommand{\editorInnen}{}
\providecommand{\dateiname}{\jobname}

\vspace{3cm}

\vfill

\footnotesize
\textsc{Quelle}: \titel. Herausgegeben von {\editorInnen}. In: \emph{Arthur Schnitzler: Briefwechsel mit Autorinnen und Autoren}.
 Digitale Edition, https://schnitzler-briefe.acdh.oeaw.ac.at/{\dateiname}.html (Stand \today)
\fi

\end{document}


