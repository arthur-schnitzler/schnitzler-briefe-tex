%% latex-leseansicht-vorspann.tex
%% Vorspann für die Leseansicht.
%% Lädt die gemeinsame Datei latex-vorspann.tex mit nicht gesetztem Schalter.

\newif\ifkorrekturansicht
\korrekturansichtfalse

\input{../tex-inputs/latex-vorspann}

\begin{center}
            \textcolor{red}{ENTWURF. ENTZIFFERUNG NOCH NICHT KORREKTURGELESEN}
                      \end{center}
            
               \section[Joseph Victor Widmann an Arthur Schnitzler, {[}22.?{]} 2. 1902]{ Joseph Victor Widmann an Arthur Schnitzler, {[}22.?{]} 2. 1902}\nopagebreak\mylabel{v}\rehead{ }\begin{ledgroupsized}[t]{13cm}\normalsize\beginnumbering\briefempfaengerindex{Schnitzler, Arthur@\textsc{Schnitzler, Arthur}!zzzWidmann, Joseph Victor@\emph{von Joseph Victor Widmann}!1902-02-221@{{[}22.?{]} 2. 1902}|(be} \toendnotes[C]{\smallbreak\pagebreak[2]} \Standort{CUL, Schnitzler, B 113.}
\physDesc{Brief, 1 Blatt, 2 Seiten
\newline{}Gedruckte Danksagung
\newline{}Handschrift: schwarze Tinte (\noindent{}Unterschrift)
\newline{}Schnitzler: mit Bleistift beschriftet: »\textsc{Widmann}« \newline{}Ordnung: mit Bleistift von unbekannter Hand datiert: »1902« }\toendnotes[C]{\smallbreak}\stanza{}{\pb}\textcolor{gray}{\textbf{Vor’s Portal für Jubelgreise}}\newverse{}\textcolor{gray}{\textbf{Gängelt Ihr mich lobesam,}}\newverse{}\textcolor{gray}{\textbf{Da nun meine Lebensreise}}\newverse{}\textcolor{gray}{\textbf{An die Sechz’ger-Ecke kam.}}\stanzaend{}\stanza{}\textcolor{gray}{\textbf{Am Portal giebt’s lust’gen Thorschnack}}\newverse{}\textcolor{gray}{\textbf{Zeitungsflaggenwimpelei,}}\newverse{}\textcolor{gray}{\textbf{Künft’ger Nekrologe Vorschmack}}\newverse{}\textcolor{gray}{\textbf{Und wie lieb ich vielen sei.}}\stanzaend{}\stanza{}\textcolor{gray}{\textbf{Aber diese Zeitungsflaggen,}}\newverse{}\textcolor{gray}{\textbf{Die mir heute freundlich wehn,}}\newverse{}\textcolor{gray}{\textbf{Haben doch den Schalk im Nacken}}\newverse{}\textcolor{gray}{\textbf{Und ich kann sie gut verstehn.}}\stanzaend{}\stanza{}\textcolor{gray}{\textbf{Was mir manchmal schon als Ahnung}}\newverse{}\textcolor{gray}{\textbf{Leise durch die Seele glitt,}}\newverse{}\textcolor{gray}{\textbf{Wird zur öffentlichen Mahnung:}}\newverse{}\textcolor{gray}{\textbf{\emph{»Du bist alt! Thu nicht mehr mit!}}}\stanzaend{}\stanza{}\textcolor{gray}{\textbf{\emph{»Wie’s mit Winterstrahlenschrägheit}}}\newverse{}\textcolor{gray}{\textbf{\emph{»Jetzt die Alterssonne meint,}}}\newverse{}\textcolor{gray}{\textbf{\emph{»Fass’ es klug: Erlaubt ist Trägheit,}}}\newverse{}\textcolor{gray}{\textbf{\emph{»Die von nun an Würde scheint.«}}}\stanzaend{}\stanza{}{\pb}\textcolor{gray}{\textbf{Hm! Das laß ich mir gefallen,}}\newverse{}\textcolor{gray}{\textbf{Wenn Ihr’s nicht zu wörtlich nehmt.}}\newverse{}\textcolor{gray}{\textbf{Und ich sage Dank Euch allen,}}\newverse{}\textcolor{gray}{\textbf{Die mich heut’ bediademt}}\stanzaend{}\stanza{}\textcolor{gray}{\textbf{Oder doch bediaduselt}}\newverse{}\textcolor{gray}{\textbf{Mit so manchem art’gen Wort.}}\newverse{}\textcolor{gray}{\textbf{Musen! Jetzt ist ausgemuselt!}}\newverse{}\textcolor{gray}{\textbf{Alle neune schick’ ich fort.}}\stanzaend{}\stanza{}\textcolor{gray}{\textbf{Aber dass aus ihren Haaren}}\newverse{}\textcolor{gray}{\textbf{bleibt ein holder Duft zurück,}}\newverse{}\textcolor{gray}{\textbf{Der in neue Schreibgefahren}}\newverse{}\textcolor{gray}{\textbf{Lockt, in neuer Träume Glück, –}}\stanzaend{}\stanza{}\textcolor{gray}{\textbf{Dieses gänzlich zu verhüten,}}\newverse{}\textcolor{gray}{\textbf{Steht nur schwer in meiner Macht;}}\newverse{}\textcolor{gray}{\textbf{Sieht man doch auch späte Blüten,}}\newverse{}\textcolor{gray}{\textbf{Wenn vom Frost der Wald schon kracht.}}\stanzaend{}\stanza{}\textcolor{gray}{\textbf{Nehmt sie, wenn sie sprießen sollten,}}\newverse{}\textcolor{gray}{\textbf{Dann als Dank für Eure Huld.}}\newverse{}\textcolor{gray}{\textbf{\so{Denn, je mehr ein Mann gegolten,}}}\newverse{}\textcolor{gray}{\textbf{\so{Um so mehr steht er in Schuld.}}}\stanzaend{}\pstart
           \textcolor{gray}{\textbf{\emph{Bern\oindex{Bern@\textbf{Bern}|pw}}, am \label{K_L01205_1v}\edtext{20. Februar 1902}{\lemma{\textnormal{\emph{20. Februar 1902}}}\Cendnote{\textnormal{Widmanns sechzigster
                            Geburtstag.}}}\label{K_L01205_1h}.}}{\\[\baselineskip]}\spacefill\mbox{{[}hs.:{]} J. V. Widmann}\pend
           \leftskip=0em{}\endnumbering\briefempfaengerindex{Schnitzler, Arthur@\textsc{Schnitzler, Arthur}!zzzWidmann, Joseph Victor@\emph{von Joseph Victor Widmann}!1902-02-221@{{[}22.?{]} 2. 1902}|)be}\mylabel{h}\end{ledgroupsized}  \newcommand{\dateiname}{L01205}\newcommand{\titel}{Joseph Victor Widmann an Arthur Schnitzler, [22.?] 2. 1902}\newcommand{\editorInnen}{Martin Anton Müller und Gerd-Hermann Susen}%% latex-leseansicht-abspann.tex
%% Abspann für die Leseansicht.
%% Der Schalter \ifkorrekturansicht ist bereits durch den Vorspann gesetzt.

%% latex-abspann.tex
%% Gemeinsamer Abspann für Korrekturansicht und Leseansicht.
%% Setzt den Schalter \ifkorrekturansicht voraus (gesetzt in den
%% einbindenden Dateien latex-korrekturansicht-abspann.tex bzw.
%% latex-leseansicht-abspann.tex).
%% ---------------------------------------------------------------

\normalsize

% Das esempio-Environment wird nur in der Leseansicht benötigt
\ifkorrekturansicht\else
\newenvironment{esempio}[3]%
{
    \vspace{1.5ex}
    \rlap{\underline{#1}}
    \par
    \setlength{\parindent}{0cm}
    \nopagebreak
    \leftskip=#2cm
    \rightskip=#3cm
}
{
    \par
}
\fi

\doendnotes{C}
\bigskip
\vfill

\clearpage

\footnotesize

\ifkorrekturansicht
  \lohead{\textsc{register}}
\fi

% theindex-Environment neu definieren ohne reledmac
\makeatletter
\renewenvironment{theindex}{%
  \ifkorrekturansicht
    \section*{\indexname}%
  \else
    \subsubsection*{Index der erwähnten Entitäten}%
  \fi
  \setlength{\parindent}{0pt}%
  \setlength{\parskip}{0pt plus 0.3pt}%
  \let\item\@idxitem
}{%
  \ifkorrekturansicht\clearpage\fi
}
\makeatother

\IfFileExists{\jobname-pw.ind}{\input{\jobname-pw.ind}}{}

% Quellenangabe nur in der Leseansicht
\ifkorrekturansicht\else
% Fallback-Definitionen, falls die .tex-Datei \titel etc. nicht gesetzt hat
\providecommand{\titel}{}
\providecommand{\editorInnen}{}
\providecommand{\dateiname}{\jobname}

\vspace{3cm}

\vfill

\footnotesize
\textsc{Quelle}: \titel. Herausgegeben von {\editorInnen}. In: \emph{Arthur Schnitzler: Briefwechsel mit Autorinnen und Autoren}.
 Digitale Edition, https://schnitzler-briefe.acdh.oeaw.ac.at/{\dateiname}.html (Stand \today)
\fi

\end{document}


      