%% latex-korrekturansicht-vorspann.tex
%% Vorspann für die Korrekturansicht.
%% Lädt die gemeinsame Datei latex-vorspann.tex mit gesetztem Schalter.

\newif\ifkorrekturansicht
\korrekturansichttrue

\input{../tex-inputs/latex-vorspann}


\section[Joseph Victor Widmann an Arthur Schnitzler, {[}22.?{]} 2. 1902]{L01205 Joseph Victor Widmann an Arthur Schnitzler, {[}22.?{]} 2. 1902}
\nopagebreak\mylabel{L01205v}
\rehead{ }\normalsize\beginnumbering\briefempfaengerindex{Schnitzler, Arthur@\textsc{Schnitzler, Arthur}!zzzWidmann, Joseph Victor@\emph{von Joseph Victor Widmann}!1902-02-221@{{[}22.?{]} 2. 1902}|(be}
\toendnotes[C]{\smallbreak\pagebreak[2]}\Standort{CUL, Schnitzler, B 113.}
\physDesc{Brief, 1 Blatt, 2 Seiten, 13 Zeichen
\newline{}\noindent{}Gedruckte Danksagung\noindent{}Gedruckte Danksagung
\newline{}Handschrift: schwarze Tinte (\noindent{}Unterschrift)
\newline{}Schnitzler: mit Bleistift beschriftet: »\textsc{Widmann}« 
\newline{}Ordnung: mit Bleistift von unbekannter Hand datiert: »1902« }\toendnotes[C]{\smallbreak}\stanza{}{\pb}\textcolor{gray}{\textbf{Vor’s Portal für Jubelgreise}}\textcolor{gray}{\textbf{Gängelt Ihr mich lobesam,}}\textcolor{gray}{\textbf{Da nun meine Lebensreise}}\textcolor{gray}{\textbf{An die Sechz’ger-Ecke kam.}}\stanzaend{}\stanza{}\textcolor{gray}{\textbf{Am Portal giebt’s lust’gen Thorschnack}}\textcolor{gray}{\textbf{Zeitungsflaggenwimpelei,}}\textcolor{gray}{\textbf{Künft’ger Nekrologe Vorschmack}}\textcolor{gray}{\textbf{Und wie lieb ich vielen sei.}}\stanzaend{}\stanza{}\textcolor{gray}{\textbf{Aber diese Zeitungsflaggen,}}\textcolor{gray}{\textbf{Die mir heute freundlich wehn,}}\textcolor{gray}{\textbf{Haben doch den Schalk im Nacken}}\textcolor{gray}{\textbf{Und ich kann sie gut verstehn.}}\stanzaend{}\stanza{}\textcolor{gray}{\textbf{Was mir manchmal schon als Ahnung}}\textcolor{gray}{\textbf{Leise durch die Seele glitt,}}\textcolor{gray}{\textbf{Wird zur öffentlichen Mahnung:}}\textcolor{gray}{\textbf{\emph{»Du bist alt! Thu nicht mehr mit!}}}\stanzaend{}\stanza{}\textcolor{gray}{\textbf{\emph{»Wie’s mit Winterstrahlenschrägheit}}}\textcolor{gray}{\textbf{\emph{»Jetzt die Alterssonne meint,}}}\textcolor{gray}{\textbf{\emph{»Fass’ es klug: Erlaubt ist Trägheit,}}}\textcolor{gray}{\textbf{\emph{»Die von nun an Würde scheint.«}}}\stanzaend{}\stanza{}{\pb}\textcolor{gray}{\textbf{Hm! Das laß ich mir gefallen,}}\textcolor{gray}{\textbf{Wenn Ihr’s nicht zu wörtlich nehmt.}}\textcolor{gray}{\textbf{Und ich sage Dank Euch allen,}}\textcolor{gray}{\textbf{Die mich heut’ bediademt}}\stanzaend{}\stanza{}\textcolor{gray}{\textbf{Oder doch bediaduselt}}\textcolor{gray}{\textbf{Mit so manchem art’gen Wort.}}\textcolor{gray}{\textbf{Musen! Jetzt ist ausgemuselt!}}\textcolor{gray}{\textbf{Alle neune schick’ ich fort.}}\stanzaend{}\stanza{}\textcolor{gray}{\textbf{Aber dass aus ihren Haaren}}\textcolor{gray}{\textbf{bleibt ein holder Duft zurück,}}\textcolor{gray}{\textbf{Der in neue Schreibgefahren}}\textcolor{gray}{\textbf{Lockt, in neuer Träume Glück, –}}\stanzaend{}\stanza{}\textcolor{gray}{\textbf{Dieses gänzlich zu verhüten,}}\textcolor{gray}{\textbf{Steht nur schwer in meiner Macht;}}\textcolor{gray}{\textbf{Sieht man doch auch späte Blüten,}}\textcolor{gray}{\textbf{Wenn vom Frost der Wald schon kracht.}}\stanzaend{}\stanza{}\textcolor{gray}{\textbf{Nehmt sie, wenn sie sprießen sollten,}}\textcolor{gray}{\textbf{Dann als Dank für Eure Huld.}}\textcolor{gray}{\textbf{\so{Denn, je mehr ein Mann gegolten,}}}\textcolor{gray}{\textbf{\so{Um so mehr steht er in Schuld.}}}\stanzaend{}
\pstart
           \textcolor{gray}{\textbf{\emph{Bern\oindex{Bern@\textbf{Bern}, \emph{P.PPLC}|pw}}, am \label{K_L01205-1v}\edtext{20. Februar 1902}{\lemma{\textnormal{\emph{20. Februar 1902}}}\Cendnote{\textnormal{Widmanns sechzigster
                     Geburtstag}}}\label{K_L01205-1}.}}{\\[\baselineskip]}\spacefill\mbox{{[}hs.:{]} J. V. Widmann}\pend
           \leftskip=0em{}\selectlanguage{ngerman}\endnumbering\briefempfaengerindex{Schnitzler, Arthur@\textsc{Schnitzler, Arthur}!zzzWidmann, Joseph Victor@\emph{von Joseph Victor Widmann}!1902-02-221@{{[}22.?{]} 2. 1902}|)be}\mylabel{L01205h}  \normalsize

\doendnotes{C}
\bigskip
\vfill

\clearpage

\footnotesize

\lohead{\textsc{register}}

% Definiere theindex-Environment komplett neu ohne reledmac
\makeatletter
\renewenvironment{theindex}{%
  \section*{\indexname}%
  \setlength{\parindent}{0pt}%
  \setlength{\parskip}{0pt plus 0.3pt}%
  \let\item\@idxitem
}{%
  \clearpage
}
\makeatother

\IfFileExists{\jobname-pw.ind}{\input{\jobname-pw.ind}}{}

\end{document}

      