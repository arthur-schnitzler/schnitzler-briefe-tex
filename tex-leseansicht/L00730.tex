%% latex-korrekturansicht-vorspann.tex
%% Vorspann für die Korrekturansicht.
%% Lädt die gemeinsame Datei latex-vorspann.tex mit gesetztem Schalter.

\newif\ifkorrekturansicht
\korrekturansichttrue

\input{../tex-inputs/latex-vorspann}


\section[Arthur Schnitzler an Hugo von Hofmannsthal, 4. 10. 1897]{L00730 Arthur Schnitzler an Hugo von Hofmannsthal, 4. 10. 1897}
\nopagebreak\mylabel{L00730v}
\rehead{ }\normalsize\beginnumbering\briefempfaengerindex{Hofmannsthal, Hugo von@\textsc{Hofmannsthal, Hugo von}!zzzSchnitzler, Arthur@\emph{von Arthur Schnitzler}!1897-10-042@{4. 10. 1897}|(be}
\toendnotes[C]{\smallbreak\pagebreak[2]}\Standort{FDH, Hs-30885,64.}
\physDesc{Brief, 1 Blatt, 3 Seiten, 852 Zeichen
\newline{}Handschrift: schwarze Tinte, deutsche Kurrent
\newline{}Ordnung: mit Bleistift von Schnitzler mutmaßlich bei der Durchsicht der Korrespondenz
                                    1929 datiert: »4/10 97« }
\buchAbdrucke{\weitereDrucke{Hugo von Hofmannsthal, Arthur Schnitzler: \emph{Briefwechsel}. Frankfurt am Main: \emph{S. Fischer} 1964, S. 96.} }\toendnotes[C]{\smallbreak}
\pstart
           \noindent{}{\pb}Mein lieber Hugo, ich danke Ihnen ſehr; Sie wiſſen ja, dſs es i{\geminationm}er ſehr wohlthuend auf mich wirkt, we{\geminationn} mich irgendwas die Herzlichkeit unſres Verhältniſſes
               lebhaft empfinden läßt. – Es iſt ſehr ſchrecklich geweſen; im Anfang ſo ſchrecklich,
                  {\pb}dſs ich es garnicht begriffen habe. In den letzten
               Tagen hat es ſich raſch gemildert; beſonders ſeit dem Augenblick wo ich erfahren, dſs
               auch Sie\pwindex{Reinhard, Marie 1871-03-13 – 1899-03-18@\textsc{Reinhard, Marie} (1871-03-13 – 1899-03-18), \emph{Gesangspädagoge/Gesangspädagogin}|pwv}
                zwiſchen Tod und Leben war. –\pend
           
\pstart
           Ich habe auch zu arbeiten angefangen; d. h. ich leſe mein neues Stück\pwindex{Vermaechtnis. Schauspiel in drei Akten@\emph{Das Vermächtnis. Schauspiel in drei Akten}|pwv} durch und bin noch nicht drauf
               gekommen, wo der Hauptfehler ſteckt. –\pend
           
\pstart
           {\pb}Das neue\pwindex{Frau im Fenster@\emph{Die Frau im Fenster}|pwv}\pwindex{Hochzeit der Sobeide@\emph{Die Hochzeit der Sobeide}|pwv} was Sie geſchrieben haben möcht ich natürlich ſehr
               bald hören. Nicht wahr, ich weiſs es gleich, wenn Sie in Wien\oindex{Wien@\textbf{Wien}, \emph{A.ADM2}|pw} angeko{\geminationm}en ſind? Wie lange hab
               ich ſchon nicht mit Ihnen geſprochen!\pend
           
\pstart
           Das was Sie über die Rede von \textsc{D’Annunzio}\pwindex{Rede Gabriele DAnnunzios. Notizen von einer Reise im oberen Italien@\emph{Die Rede Gabriele d’Annunzios. Notizen von einer Reise im oberen Italien}|pw} geſagt haben, iſt ſehr ſchön. –\pend
           
\pstart
           Leben Sie wohl.\pend
           \pstart Von Herzen Ihr \spacefill\mbox{Arthur}\pend{}
\pstart
           Wien\oindex{Wien@\textbf{Wien}, \emph{A.ADM2}|pw}{ }4. 10. 97.\pend
           \selectlanguage{ngerman}\endnumbering\briefempfaengerindex{Hofmannsthal, Hugo von@\textsc{Hofmannsthal, Hugo von}!zzzSchnitzler, Arthur@\emph{von Arthur Schnitzler}!1897-10-042@{4. 10. 1897}|)be}\mylabel{L00730h}  \normalsize

\doendnotes{C}
\bigskip
\vfill

\clearpage

\footnotesize

\lohead{\textsc{register}}

% Definiere theindex-Environment komplett neu ohne reledmac
\makeatletter
\renewenvironment{theindex}{%
  \section*{\indexname}%
  \setlength{\parindent}{0pt}%
  \setlength{\parskip}{0pt plus 0.3pt}%
  \let\item\@idxitem
}{%
  \clearpage
}
\makeatother

\IfFileExists{\jobname-pw.ind}{\input{\jobname-pw.ind}}{}

\end{document}

      