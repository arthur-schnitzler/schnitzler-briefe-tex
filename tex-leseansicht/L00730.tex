%% latex-leseansicht-vorspann.tex
%% Vorspann für die Leseansicht.
%% Lädt die gemeinsame Datei latex-vorspann.tex mit nicht gesetztem Schalter.

\newif\ifkorrekturansicht
\korrekturansichtfalse

\input{../tex-inputs/latex-vorspann}


         
         \newcommand{\erwaehntePersonen}{Personen: }
         \newcommand{\erwaehnteInstitutionen}{}
         \newcommand{\erwaehnteOrte}{}
         \newcommand{\erwaehnteWerke}{
               \section[Arthur Schnitzler an Hugo von Hofmannsthal, 4. 10. 1897]{ Arthur Schnitzler an Hugo von Hofmannsthal, 4. 10. 1897}\nopagebreak\mylabel{v}\rehead{ }\begin{ledgroupsized}[t]{13cm}\normalsize\beginnumbering \toendnotes[C]{\smallbreak\pagebreak[2]} \Standort{FDH, Hs-30885,64.}
\physDesc{Brief, 1 Blatt, 3 Seiten
\newline{}Handschrift: schwarze Tinte, deutsche Kurrent\newline{}Ordnung: von Schnitzler mutmaßlich bei der Durchsicht der Korrespondenz 1929 mit
                                    Bleistift datiert: »4/10 97« }\buchAbdrucke{\weitereDrucke{Hugo von Hofmannsthal, Arthur Schnitzler: \emph{Briefwechsel}. Hg. Therese Nickl und Heinrich Schnitzler. Frankfurt am Main: \emph{S. Fischer} 1964, S. 96.} }\toendnotes[C]{\smallbreak}\pstart
           \noindent{}{\pb}Mein lieber Hugo, ich danke Ihnen ſehr; Sie wiſſen ja, dſs es
                        i{\geminationm}er ſehr wohlthuend auf mich wirkt, we{\geminationn} mich irgendwas die Herzlichkeit unſres
                    Verhältniſſes lebhaft empfinden läßt. – Es iſt ſehr ſchrecklich geweſen; im
                    Anfang ſo ſchrecklich, {\pb}dſs ich es garnicht
                    begriffen habe. In den letzten Tagen hat es ſich raſch gemildert; beſonders ſeit
                    dem Augenblick wo ich erfahren, dſs auch Sie zwiſchen Tod und Leben
                    war. –\pend
           \pstart
           Ich habe auch zu arbeiten angefangen; d. h. ich leſe mein neues Stück\textcolor{red}{\textsuperscript{XXXX indx}} durch und bin noch nicht drauf gekommen, wo der
                    Hauptfehler ſteckt. –\pend
           \pstart
           {\pb}Das neue\textcolor{red}{\textsuperscript{XXXX indx}}\textcolor{red}{\textsuperscript{XXXX indx}} was Sie geſchrieben haben möcht ich natürlich ſehr bald hören.
                    Nicht wahr, ich weiſs es gleich, wenn Sie in Wien\oindex{XXXX Ortsangabe fehlt|pw} angeko{\geminationm}en ſind? Wie lange hab
                    ich ſchon nicht mit Ihnen geſprochen!\pend
           \pstart
           Das was Sie über die Rede von \textsc{D’Annunzio}\textcolor{red}{\textsuperscript{XXXX indx}} geſagt haben, iſt ſehr ſchön. –\pend
           \pstart
           Leben Sie wohl.\pend
           \pstart Von Herzen Ihr \spacefill\mbox{Arthur}\pend{}\pstart
           Wien\oindex{XXXX Ortsangabe fehlt|pw}{ }4. 10. 97.\pend
           
         
         \endnumbering\mylabel{h}\end{ledgroupsized}  \newcommand{\dateiname}{L00730}\newcommand{\titel}{Arthur Schnitzler an Hugo von Hofmannsthal, 4. 10. 1897}\newcommand{\editorInnen}{Martin Anton Müller und Gerd-Hermann Susen}%% latex-leseansicht-abspann.tex
%% Abspann für die Leseansicht.
%% Der Schalter \ifkorrekturansicht ist bereits durch den Vorspann gesetzt.

%% latex-abspann.tex
%% Gemeinsamer Abspann für Korrekturansicht und Leseansicht.
%% Setzt den Schalter \ifkorrekturansicht voraus (gesetzt in den
%% einbindenden Dateien latex-korrekturansicht-abspann.tex bzw.
%% latex-leseansicht-abspann.tex).
%% ---------------------------------------------------------------

\normalsize

% Das esempio-Environment wird nur in der Leseansicht benötigt
\ifkorrekturansicht\else
\newenvironment{esempio}[3]%
{
    \vspace{1.5ex}
    \rlap{\underline{#1}}
    \par
    \setlength{\parindent}{0cm}
    \nopagebreak
    \leftskip=#2cm
    \rightskip=#3cm
}
{
    \par
}
\fi

\doendnotes{C}
\bigskip
\vfill

\clearpage

\footnotesize

\ifkorrekturansicht
  \lohead{\textsc{register}}
\fi

% theindex-Environment neu definieren ohne reledmac
\makeatletter
\renewenvironment{theindex}{%
  \ifkorrekturansicht
    \section*{\indexname}%
  \else
    \subsubsection*{Index der erwähnten Entitäten}%
  \fi
  \setlength{\parindent}{0pt}%
  \setlength{\parskip}{0pt plus 0.3pt}%
  \let\item\@idxitem
}{%
  \ifkorrekturansicht\clearpage\fi
}
\makeatother

\IfFileExists{\jobname-pw.ind}{\input{\jobname-pw.ind}}{}

% Quellenangabe nur in der Leseansicht
\ifkorrekturansicht\else
% Fallback-Definitionen, falls die .tex-Datei \titel etc. nicht gesetzt hat
\providecommand{\titel}{}
\providecommand{\editorInnen}{}
\providecommand{\dateiname}{\jobname}

\vspace{3cm}

\vfill

\footnotesize
\textsc{Quelle}: \titel. Herausgegeben von {\editorInnen}. In: \emph{Arthur Schnitzler: Briefwechsel mit Autorinnen und Autoren}.
 Digitale Edition, https://schnitzler-briefe.acdh.oeaw.ac.at/{\dateiname}.html (Stand \today)
\fi

\end{document}


      