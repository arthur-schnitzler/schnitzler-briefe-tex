%% latex-korrekturansicht-vorspann.tex
%% Vorspann für die Korrekturansicht.
%% Lädt die gemeinsame Datei latex-vorspann.tex mit gesetztem Schalter.

\newif\ifkorrekturansicht
\korrekturansichttrue

\input{../tex-inputs/latex-vorspann}


\section[Arthur Schnitzler an Hugo von Hofmannsthal, 28. 4. 1897]{L00672 Arthur Schnitzler an Hugo von Hofmannsthal, 28. 4. 1897}
\nopagebreak\mylabel{L00672v}
\rehead{ }\normalsize\beginnumbering\briefempfaengerindex{Hofmannsthal, Hugo von@\textsc{Hofmannsthal, Hugo von}!zzzSchnitzler, Arthur@\emph{von Arthur Schnitzler}!1897-04-281@{28. 4. 1897}|(be}
\toendnotes[C]{\smallbreak\pagebreak[2]}\Standort{FDH, Hs-30885,57.}
\physDesc{Brief, 1 Blatt, 4 Seiten, 1088 Zeichen
\newline{}Handschrift: schwarze Tinte, deutsche Kurrent}
\buchAbdrucke{\weitereDrucke{Hugo von Hofmannsthal, Arthur Schnitzler: \emph{Briefwechsel}. Frankfurt am Main: \emph{S. Fischer} 1964, S. 82–83.} }\toendnotes[C]{\smallbreak}
\pstart
           \raggedleft{}{\pb}5  \textsc{rue de Maubeuge}\oindex{rue de Maubeuge@\textbf{rue de Maubeuge}, \emph{Straße (K.STR)}|pw}{\\}\textsc{Paris}\oindex{Paris@\textbf{Paris}, \emph{P.PPLC}|pw}{ }28. 4. 97\pend
           
\pstart{}Lieber Hugo, \pend\vspace{0.5em}
\pstart
           an Fiſcher\pwindex{Fischer, Samuel 24.12.1859 – 15.10.1934@\textsc{Fischer, Samuel} (24.12.1859 – 15.10.1934), \emph{Verleger/Verlegerin}|pw} hab ich geſchrieben, ich zweifle
               nicht, dſs er ohne weiters einverſtanden iſt. Warum aber glauben Sie, daſs alle dieſe
               Sachen ſich nur von Paris\oindex{Paris@\textbf{Paris}, \emph{P.PPLC}|pw} aus komiſch anhören.
               Sie ſind übrigens mehr ekelhaft als komiſch. We{\geminationn}{ }{\pb}ſich Clara\pwindex{Pollaczek, Clara Katharina 15.01.1875 – 22.07.1951@\textsc{Pollaczek, Clara Katharina} (15.01.1875 – 22.07.1951), \emph{Schriftsteller/Schriftstellerin}|pw} nur
               nicht viel draus macht und ſich nicht gar zu viel \label{K_L00672-1v}\edtext{ſekiren}{\lemma{\textnormal{\emph{ſekiren}}}\Cendnote{\textnormal{österreichisch sekkieren: ärgern}}}\label{K_L00672-1} laſſen muſs. Grüßen Sie ſie u Anna\pwindex{Epstein, Anna 6.3.1877 – 16.3.1943@\textsc{Epstein, Anna} (6.3.1877 – 16.3.1943)|pw} von mir herzlich.\pend
           
\pstart
           – Iſt es möglich, dſs Minnie\pwindex{Schaffgotsch, Hermine von 25.11.1871 – 25.11.1928@\textsc{Schaffgotsch, Hermine von} (25.11.1871 – 25.11.1928)|pw} an dem Tratſch
               zum Theil ſchuld iſt? (Da wird ſie mir ja auch was ähnliches anrichten!) Sonderbarer
               Weiſe das einzige literariſche, worüber ich hier ein biſſel nachgedacht, iſt das Stück\pwindex{Weg ins Freie. Roman@\emph{Der Weg ins Freie. Roman}|pwv}, wo \strikeout{ſich} ſie mich {\pb}rettet. Aber
               ſie ändert ſich mir im Kopf, ſie ist ſchon beinah blond.\pend
           
\pstart
           Meinen Brief von geſtern oder vorgeſtern haben Sie doch? –\pend
           
\pstart
           Arbeiten Sie was?\pend
           
\pstart
           Eben komme ich von \textsc{Versailles}\oindex{Versailles@\textbf{Versailles}, \emph{P.PPLA2}|pw} zurück und habe eine unbeſchreibliche Luſt nach Grün und Luft und Stille
                  heimge{\pb}bracht; eine ſo heftige Ungeduld, daſs ich
               gleich wieder aus Paris wegmöchte, we{\geminationn}’s ſo ohne
               weiteres ginge.\pend
           
\pstart
           Das gibt ſich wieder.\pend
           
\pstart
           Seien Sie herzlich gegrüßt.{\\[\baselineskip]}Ihr\spacefill\mbox{Arthur.}\pend
           \leftskip=0em{}
\pstart
           \noindent{}Statt gemiſchten Hausbrodes eſſe ich gemiſchtes Hausbrod. –\pend
           \selectlanguage{ngerman}\endnumbering\briefempfaengerindex{Hofmannsthal, Hugo von@\textsc{Hofmannsthal, Hugo von}!zzzSchnitzler, Arthur@\emph{von Arthur Schnitzler}!1897-04-281@{28. 4. 1897}|)be}\mylabel{L00672h}  \normalsize

\doendnotes{C}
\bigskip
\vfill

\clearpage

\footnotesize

\lohead{\textsc{register}}

% Definiere theindex-Environment komplett neu ohne reledmac
\makeatletter
\renewenvironment{theindex}{%
  \section*{\indexname}%
  \setlength{\parindent}{0pt}%
  \setlength{\parskip}{0pt plus 0.3pt}%
  \let\item\@idxitem
}{%
  \clearpage
}
\makeatother

\IfFileExists{\jobname-pw.ind}{\input{\jobname-pw.ind}}{}

\end{document}

      