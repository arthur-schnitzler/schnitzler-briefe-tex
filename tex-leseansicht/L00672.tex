%% latex-leseansicht-vorspann.tex
%% Vorspann für die Leseansicht.
%% Lädt die gemeinsame Datei latex-vorspann.tex mit nicht gesetztem Schalter.

\newif\ifkorrekturansicht
\korrekturansichtfalse

\input{../tex-inputs/latex-vorspann}


         
         \newcommand{\erwaehntePersonen}{Personen: Anna Epstein, Samuel Fischer, Hugo von Hofmannsthal, Clara Katharina Pollaczek, Hermine von Schaffgotsch}
         \newcommand{\erwaehnteOrte}{Orte: Paris, Versailles, Wien, rue de Maubeuge}
         \newcommand{\erwaehnteWerke}{Werke: Der Weg ins Freie. Roman}
               \section[Arthur Schnitzler an Hugo von Hofmannsthal, 28. 4. 1897]{ Arthur Schnitzler an Hugo von Hofmannsthal, 28. 4. 1897}\nopagebreak\mylabel{v}\rehead{ }\begin{ledgroupsized}[t]{13cm}\normalsize\beginnumbering \toendnotes[C]{\smallbreak\pagebreak[2]} \Standort{FDH, Hs-30885,57.}
\physDesc{Brief, 1 Blatt, 4 Seiten
\newline{}Handschrift: schwarze Tinte, deutsche Kurrent}\buchAbdrucke{\weitereDrucke{Hugo von Hofmannsthal, Arthur Schnitzler: \emph{Briefwechsel}. Hg. Therese Nickl und Heinrich Schnitzler. Frankfurt am Main: \emph{S. Fischer} 1964, S. 82–83.} }\toendnotes[C]{\smallbreak}\pstart
           {\pb}5  \textsc{rue de Maubeuge}\oindex{rue de Maubeuge@\textbf{rue de Maubeuge}|pw}{\\}\textsc{Paris}\oindex{Paris@\textbf{Paris}|pw}{ }28. 4. 97\pend
           \pstart{}Lieber Hugo, \pend\pstart
           an Fiſcher\pwindex{Fischer, Samuel 24.12.1859 – 15.10.1934@\textsc{Fischer, Samuel} (24.12.1859 – 15.10.1934), \emph{Verleger}|pw} hab ich geſchrieben, ich zweifle
                    nicht, dſs er ohne weiters einverſtanden iſt. Warum aber glauben Sie, daſs alle
                    dieſe Sachen ſich nur von Paris\oindex{Paris@\textbf{Paris}|pw} aus komiſch
                    anhören. Sie ſind übrigens mehr ekelhaft als komiſch. We{\geminationn}{ }{\pb}ſich Clara\pwindex{Pollaczek, Clara Katharina 15.01.1875 – 22.07.1951@\textsc{Pollaczek, Clara Katharina} (15.01.1875 – 22.07.1951), \emph{Schriftstellerin}|pw} nur
                    nicht viel draus macht und ſich nicht gar zu viel \label{K_L00672_1v}\edtext{ſekiren}{\lemma{\textnormal{\emph{ſekiren}}}\Cendnote{\textnormal{österreichisch sekkieren: ärgern}}}\label{K_L00672_1h}
                    laſſen muſs. Grüßen Sie ſie u Anna\pwindex{Epstein, Anna 6.3.1877 – 16.3.1943@\textsc{Epstein, Anna} (6.3.1877 – 16.3.1943)|pw} von mir
                    herzlich.\pend
           \pstart
           – Iſt es möglich, dſs Minnie\pwindex{Schaffgotsch, Hermine von 25.11.1871 – 25.11.1928@\textsc{Schaffgotsch, Hermine von} (25.11.1871 – 25.11.1928)|pw} an dem Tratſch
                    zum Theil ſchuld iſt? (Da wird ſie mir ja auch was ähnliches anrichten!)
                    Sonderbarer Weiſe das einzige literariſche, worüber ich hier ein biſſel
                    nachgedacht, iſt das Stück\pwindex{Schnitzler, Arthur 15.05.1862 – 21.10.1931@\textsc{Schnitzler, Arthur} (15.05.1862 – 21.10.1931), \emph{Schriftsteller, Mediziner}!Weg ins Freie. Roman1.1.1908 – 1.6.1908@\strich\emph{Der Weg ins Freie. Roman} {[}1.1.1908 – 1.6.1908{]}|pwv},
                    wo \strikeout{ſich} ſie mich {\pb}rettet. Aber ſie ändert ſich mir im Kopf, ſie ist ſchon beinah blond.\pend
           \pstart
           Meinen Brief von geſtern oder vorgeſtern haben Sie doch? –\pend
           \pstart
           Arbeiten Sie was?\pend
           \pstart
           Eben komme ich von \textsc{Versailles}\oindex{Versailles@\textbf{Versailles}|pw} zurück und habe eine unbeſchreibliche Luſt nach Grün und Luft und Stille
                        heimge{\pb}bracht; eine ſo heftige Ungeduld, daſs ich
                    gleich wieder aus Paris wegmöchte, we{\geminationn}’s ſo ohne
                    weiteres ginge.\pend
           \pstart
           Das gibt ſich wieder.\pend
           \pstart
           Seien Sie herzlich gegrüßt.{\\[\baselineskip]}Ihr\spacefill\mbox{Arthur.}\pend
           \leftskip=0em{}\pstart
           \noindent{}Statt gemiſchten Hausbrodes eſſe ich gemiſchtes Hausbrod. –\pend
           
         
         \endnumbering\mylabel{h}\end{ledgroupsized}  \newcommand{\dateiname}{L00672}\newcommand{\titel}{Arthur Schnitzler an Hugo von Hofmannsthal, 28. 4. 1897}\newcommand{\editorInnen}{Martin Anton Müller und Gerd-Hermann Susen}%% latex-leseansicht-abspann.tex
%% Abspann für die Leseansicht.
%% Der Schalter \ifkorrekturansicht ist bereits durch den Vorspann gesetzt.

%% latex-abspann.tex
%% Gemeinsamer Abspann für Korrekturansicht und Leseansicht.
%% Setzt den Schalter \ifkorrekturansicht voraus (gesetzt in den
%% einbindenden Dateien latex-korrekturansicht-abspann.tex bzw.
%% latex-leseansicht-abspann.tex).
%% ---------------------------------------------------------------

\normalsize

% Das esempio-Environment wird nur in der Leseansicht benötigt
\ifkorrekturansicht\else
\newenvironment{esempio}[3]%
{
    \vspace{1.5ex}
    \rlap{\underline{#1}}
    \par
    \setlength{\parindent}{0cm}
    \nopagebreak
    \leftskip=#2cm
    \rightskip=#3cm
}
{
    \par
}
\fi

\doendnotes{C}
\bigskip
\vfill

\clearpage

\footnotesize

\ifkorrekturansicht
  \lohead{\textsc{register}}
\fi

% theindex-Environment neu definieren ohne reledmac
\makeatletter
\renewenvironment{theindex}{%
  \ifkorrekturansicht
    \section*{\indexname}%
  \else
    \subsubsection*{Index der erwähnten Entitäten}%
  \fi
  \setlength{\parindent}{0pt}%
  \setlength{\parskip}{0pt plus 0.3pt}%
  \let\item\@idxitem
}{%
  \ifkorrekturansicht\clearpage\fi
}
\makeatother

\IfFileExists{\jobname-pw.ind}{\input{\jobname-pw.ind}}{}

% Quellenangabe nur in der Leseansicht
\ifkorrekturansicht\else
% Fallback-Definitionen, falls die .tex-Datei \titel etc. nicht gesetzt hat
\providecommand{\titel}{}
\providecommand{\editorInnen}{}
\providecommand{\dateiname}{\jobname}

\vspace{3cm}

\vfill

\footnotesize
\textsc{Quelle}: \titel. Herausgegeben von {\editorInnen}. In: \emph{Arthur Schnitzler: Briefwechsel mit Autorinnen und Autoren}.
 Digitale Edition, https://schnitzler-briefe.acdh.oeaw.ac.at/{\dateiname}.html (Stand \today)
\fi

\end{document}


      