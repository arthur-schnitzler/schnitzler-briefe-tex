%% latex-korrekturansicht-vorspann.tex
%% Vorspann für die Korrekturansicht.
%% Lädt die gemeinsame Datei latex-vorspann.tex mit gesetztem Schalter.

\newif\ifkorrekturansicht
\korrekturansichttrue

\input{../tex-inputs/latex-vorspann}


\section[Paul Goldmann an Arthur Schnitzler, 29. 5. {[}1894{]}]{L02621 Paul Goldmann an Arthur Schnitzler, 29. 5. {[}1894{]}}
\nopagebreak\mylabel{L02621v}
\rehead{ }\normalsize\beginnumbering\briefempfaengerindex{Schnitzler, Arthur@\textsc{Schnitzler, Arthur}!zzzGoldmann, Paul@\emph{von Paul Goldmann}!1894-05-292@{29. 5. {[}1894{]}}|(be}
\toendnotes[C]{\smallbreak\pagebreak[2]}\Standort{DLA, A:Schnitzler, HS.NZ85.1.3164.}
\physDesc{Brief, 3 Blätter, 12 Seiten, 5941 Zeichen
\newline{}Handschrift: schwarze Tinte, deutsche Kurrent
\newline{}Schnitzler: 1) mit Bleistift auf dem ersten Blatt die Jahreszahl »94« vermerkt  2) mit rotem Buntstift fünf Unterstreichungen}\toendnotes[C]{\smallbreak}
\pstart
           {\pb}\textcolor{gray}{\textbf{Frankfurter Zeitung\orgindex{Frankfurter Zeitung@Frankfurter Zeitung|pw}}}\pend
           
\pstart
           \textcolor{gray}{\textbf{(Gazette de
                     Francfort\orgindex{Frankfurter Zeitung@Frankfurter Zeitung|pw}).}}\pend
           
\pstart
           \textcolor{gray}{\textbf{\begin{otherlanguage}{french}Fondateur \textbf{M. L. Sonnemann\pwindex{Sonnemann, Leopold 1831-10-29 – 1909-10-30@\textsc{Sonnemann, Leopold} (1831-10-29 – 1909-10-30), \emph{Journalist/Journalistin, Herausgeber/Herausgeberin}|pw}}.\end{otherlanguage}}}\hfill \textsc{Paris\oindex{Paris@\textbf{Paris}, \emph{P.PPLC}|pw}}, 29. Mai.\pend
           
\pstart
           \textcolor{gray}{\textbf{\begin{otherlanguage}{french}Journal politique, financier,\end{otherlanguage}}}\pend
           
\pstart
           \textcolor{gray}{\textbf{\begin{otherlanguage}{french}commercial et littéraire.\end{otherlanguage}}}\pend
           
\pstart
           \textcolor{gray}{\textbf{\begin{otherlanguage}{french}\textbf{Paraissant trois fois par jour}\end{otherlanguage}}}.\pend
           
\pstart
           \textcolor{gray}{\textbf{\begin{otherlanguage}{french}\textbf{Bureaux à Paris\oindex{Paris@\textbf{Paris}, \emph{P.PPLC}|pw}:}\end{otherlanguage}}}\pend
           
\pstart
           \textcolor{gray}{\textbf{\begin{otherlanguage}{french}\textbf{24. Rue Feydeau}\oindex{rue Feydeau@\textbf{rue Feydeau}, \emph{Straße (K.STR)}|pw}.\end{otherlanguage}}}\pend
           
\pstart\center{}Mein lieber Freund,\pend\vspace{0.5em}
\pstart
           Ich war acht Tage in Frankfurt\oindex{Frankfurt am Main@\textbf{Frankfurt am Main}, \emph{P.PPLA3}|pw}; Krankheit meines
                  Onkels\pwindex{Mamroth, Fedor 21.02.1851 – 25.06.1907@\textsc{Mamroth, Fedor} (21.02.1851 – 25.06.1907), \emph{Journalist/Journalistin, Kritiker/Kritikerin}|pwv} und
               meiner Mutter\pwindex{Goldmann, Clementine 1842-05-15 – 1924-02-24@\textsc{Goldmann, Clementine} (1842-05-15 – 1924-02-24)|pwv}. Bei meiner
               Rückkehr fand ich Deine Briefe. \label{K_L02621-1v}\edtext{Miniſterſturz und Miniſter-Kriſis}{\lemma{\textnormal{\emph{Miniſterſturz und Miniſter-Kriſis}}}\Cendnote{\textnormal{Gemeint war der am 22. 5. 1894 vollzogene (erzwungene) Rücktritt des
                  Kabinetts von Jean Casimir-Perier\pwindex{Casimir-Perier, Jean 1847-11-08 – 1907-03-11@\textsc{Casimir-Perier, Jean} (1847-11-08 – 1907-03-11), \emph{Politiker/Politikerin, Präsident/Präsidentin}|pwk}.}}}\label{K_L02621-1}
               geben Tauſenderlei zu thun. So komme ich erſt heut
               dazu, Dir zu antworten.\pend
           
\pstart
           Ich habe das \label{K_L02621-2v}\edtext{Geld}{\lemma{\textnormal{\emph{Geld}}}\Cendnote{\textnormal{Siehe Paul Goldmann an Arthur Schnitzler, 1. 5. [1894].
               }}}\label{K_L02621-2} ſofort an \textsc{Albert\pwindex{Albert, Henri 1869-11-16 – 1921-08-03@\textsc{Albert, Henri} (1869-11-16 – 1921-08-03), \emph{Journalist/Journalistin, Kritiker/Kritikerin, Übersetzer/Übersetzerin}|pw}} übergeben. Es iſt blödſinnig: aber ich kam mir vor, als wenn ich einen Raub an
               Dir beginge. Trotzdem geht Alles ehrlich zu. Aber das iſt mein Wahn, und noch heut{ }{\pb}iſt es mir unangenehm, davon zu ſprechen. \textsc{Albert\pwindex{Albert, Henri 1869-11-16 – 1921-08-03@\textsc{Albert, Henri} (1869-11-16 – 1921-08-03), \emph{Journalist/Journalistin, Kritiker/Kritikerin, Übersetzer/Übersetzerin}|pw}} bewährt ſich ſehr als mein Freund, folglich auch als Deiner. Gutes, feines,
               anſchmiegendes, liebes Naturell! Wir machen große Schlachtpläne für Dich. Ich glaube,
               er hat Dir \label{K_L02621-3v}\edtext{darüber geſchrieben}{\lemma{\textnormal{\emph{darüber geſchrieben}}}\Cendnote{\textnormal{Alberts\pwindex{Albert, Henri 1869-11-16 – 1921-08-03@\textsc{Albert, Henri} (1869-11-16 – 1921-08-03), \emph{Journalist/Journalistin, Kritiker/Kritikerin, Übersetzer/Übersetzerin}|pwk} Brief vom 23. 5. 1894
                  enthält neben dem Vorhaben, das ›\emph{Abschiedsouper}\pwindex{Abschiedssouper@\emph{Abschiedssouper}|pwk}‹ bei einer Freien Bühne aufführen zu lassen, mehrere
                  Textvorhaben: \emph{Denksteine}\pwindex{Denksteine@\emph{Denksteine}|pwk} und von ihm noch
                  nicht gelesene Textmanuskripte (\emph{Die
                     überspannte Person}\pwindex{ueberspannte Person@\emph{Die überspannte Person}|pwk} und \emph{Halb Zwei}\pwindex{Halb Zwei@\emph{Halb Zwei}|pwk}, vgl. Paul Goldmann an Arthur Schnitzler, 29. 5. [1894]) möchte er gegen Ende des Sommers im \emph{Mercure de France}\pwindex{Mercure de France@\emph{Mercure de France}|pwk} gedruckt sehen. Zusätzlich zu seiner
                  bevorstehenden Rezension\pwindex{Lettres allemandes. Drames Nouveaux@\emph{Les Lettres allemandes. Drames Nouveaux}|pwkv}
                  von \emph{Das Märchen}\pwindex{Maerchen. Schauspiel in drei Aufzuegen@\emph{Das Märchen. Schauspiel in drei Aufzügen}|pwk} in der \emph{Revue Blanche}\pwindex{Revue blanche@\emph{La Revue blanche}|pwk} plante er, in derselben Zeitschrift über die
                     »Jungen Wien\oindex{Wien@\textbf{Wien}, \emph{A.ADM2}|pw}er« zu
                  schreiben.}}}\label{K_L02621-3}. Vielleicht gelingt es gar, Dich \label{K_L02621-4v}\edtext{aufführen}{\lemma{\textnormal{\emph{aufführen}}}\Cendnote{\textnormal{Aus
                  dieser Zeit sind keine Aufführungen in Paris\oindex{Paris@\textbf{Paris}, \emph{P.PPLC}|pwk}
                  bekannt.}}}\label{K_L02621-4} zu laſſen. Ich denke, im nächſten Heft des »\textsc{Mercure\pwindex{Mercure de France@\emph{Mercure de France}|pw}}« wird \label{K_L02621-5v}\edtext{\textsc{Albert\pwindex{Albert, Henri 1869-11-16 – 1921-08-03@\textsc{Albert, Henri} (1869-11-16 – 1921-08-03), \emph{Journalist/Journalistin, Kritiker/Kritikerin, Übersetzer/Übersetzerin}|pw}} Dein »Märchen\pwindex{Maerchen. Schauspiel in drei Aufzuegen@\emph{Das Märchen. Schauspiel in drei Aufzügen}|pw}« beſprechen}{\lemma{\textnormal{\emph{Albert … beſprechen}}}\Cendnote{\textnormal{Alberts\pwindex{Albert, Henri 1869-11-16 – 1921-08-03@\textsc{Albert, Henri} (1869-11-16 – 1921-08-03), \emph{Journalist/Journalistin, Kritiker/Kritikerin, Übersetzer/Übersetzerin}|pwk}{ }Rezension\pwindex{Lettres allemandes. Drames Nouveaux@\emph{Les Lettres allemandes. Drames Nouveaux}|pwkv} erschien nicht im \emph{Mercure de France}\pwindex{Mercure de France@\emph{Mercure de France}|pwk}, sondern in der \emph{Revue Blanche}\pwindex{Revue blanche@\emph{La Revue blanche}|pwk}: Henri Albert\pwindex{Albert, Henri 1869-11-16 – 1921-08-03@\textsc{Albert, Henri} (1869-11-16 – 1921-08-03), \emph{Journalist/Journalistin, Kritiker/Kritikerin, Übersetzer/Übersetzerin}|pwk}: \emph{Les Lettres allemandes. Drames Nouveaux}\pwindex{Lettres allemandes. Drames Nouveaux@\emph{Les Lettres allemandes. Drames Nouveaux}|pwk}. In: \emph{La Revue Blanche}\pwindex{Revue blanche@\emph{La Revue blanche}|pwk}, Jg. 6, Nr. 32,
                        Juni 1894, S. 556–560, hier: S. 560.}}}\label{K_L02621-5}. Von den
               zwei Manuſkripten\pwindex{ueberspannte Person@\emph{Die überspannte Person}|pwv}\pwindex{Halb Zwei@\emph{Halb Zwei}|pwv},
               insbeſondere von der »Überſpannten
                  Perſon\pwindex{ueberspannte Person@\emph{Die überspannte Person}|pwv}« ſind wir Alle hoch entzückt. Unterſchied zwiſchen Dir und \textsc{Lavedan\pwindex{Lavedan, Henri Leon 09.04.1859 – 4.9.1940@\textsc{Lavedan, Henri Léon} (09.04.1859 – 4.9.1940), \emph{Schriftsteller/Schriftstellerin, Journalist/Journalistin}|pw}} und den \textsc{Lavedanisirenden}{ }Franz\oindex{Frankreich@\textbf{Frankreich}, \emph{A.PCLI}|pwv}oſen: In Frankreich\oindex{Frankreich@\textbf{Frankreich}, \emph{A.PCLI}|pw} Geiſt, Oberflächlichkeit, Dekadenz-Koketterie. Bei
               Dir: Na{\pb}türlichkeit, Tiefe, \uline{Sittlichkeit und Geſundheit} (Thut Dir wahrſcheinlich ſehr weh?). \strikeout{Geiſt} Geiſt natürlich auch. Das Rin\textcolor{gray}{dv}ieh\pwindex{Stauf von der March, Ottokar 29.08.1868 – 12.03.1941@\textsc{Stauf von der March, Ottokar} (29.08.1868 – 12.03.1941), \emph{Schriftsteller/Schriftstellerin, Journalist/Journalistin}|pwv}, das Dich in
               der Geſellſchaft zum Dekadenten-Häuptling macht, hat uns eine vergnügte Viertelſtunde
               bereitet.\pend
           
\pstart
           Kennſt Du Frau \textsc{Andreas-\strikeout{Sa\textcolor{gray}{l}} Salome\pwindex{Andreas-Salome, Lou 12.02.1861 – 05.02.1937@\textsc{Andreas-Salomé, Lou} (12.02.1861 – 05.02.1937), \emph{Schriftsteller/Schriftstellerin}|pw}}? Seltſame Frau\pwindex{Andreas-Salome, Lou 12.02.1861 – 05.02.1937@\textsc{Andreas-Salomé, Lou} (12.02.1861 – 05.02.1937), \emph{Schriftsteller/Schriftstellerin}|pwv}. Nicht
               ſchön, ich weiß nicht einmal, ob ſympathiſch, aber derzeit unſere gute Freundin\pwindex{Andreas-Salome, Lou 12.02.1861 – 05.02.1937@\textsc{Andreas-Salomé, Lou} (12.02.1861 – 05.02.1937), \emph{Schriftsteller/Schriftstellerin}|pwv}. Intime Freundin\pwindex{Andreas-Salome, Lou 12.02.1861 – 05.02.1937@\textsc{Andreas-Salomé, Lou} (12.02.1861 – 05.02.1937), \emph{Schriftsteller/Schriftstellerin}|pwv} von \textsc{Nietzsche\pwindex{Nietzsche, Friedrich 15.10.1844 – 25.08.1900@\textsc{Nietzsche, Friedrich} (15.10.1844 – 25.08.1900), \emph{Schriftsteller/Schriftstellerin, Philosoph/Philosophin}|pw}}. \label{K_L02621-6v}\edtext{Geſchlechtsloſe
                  Freundſchaft}{\lemma{\textnormal{\emph{Geſchlechtsloſe Freundſchaft}}}\Cendnote{\textnormal{Rein freundschaftlich war
                  die Beziehung zwischen Nietzsche\pwindex{Nietzsche, Friedrich 15.10.1844 – 25.08.1900@\textsc{Nietzsche, Friedrich} (15.10.1844 – 25.08.1900), \emph{Schriftsteller/Schriftstellerin, Philosoph/Philosophin}|pwk} und Andreas-Salomé\pwindex{Andreas-Salome, Lou 12.02.1861 – 05.02.1937@\textsc{Andreas-Salomé, Lou} (12.02.1861 – 05.02.1937), \emph{Schriftsteller/Schriftstellerin}|pwk} wahrscheinlich nicht. Wie Andreas-Salomés\pwindex{Andreas-Salome, Lou 12.02.1861 – 05.02.1937@\textsc{Andreas-Salomé, Lou} (12.02.1861 – 05.02.1937), \emph{Schriftsteller/Schriftstellerin}|pwk}{ }\emph{Lebensrückblick}\pwindex{Lebensrueckblick. Grundriss einiger Lebenserinnerungen. Aus dem Nachlass herausgegeben@\emph{Lebensrückblick. Grundriß einiger Lebenserinnerungen. Aus dem Nachlass herausgegeben}|pwk} zu entnehmen ist, soll ihr Nietzsche\pwindex{Nietzsche, Friedrich 15.10.1844 – 25.08.1900@\textsc{Nietzsche, Friedrich} (15.10.1844 – 25.08.1900), \emph{Schriftsteller/Schriftstellerin, Philosoph/Philosophin}|pwk}{ }1892 vergeblich einen Heiratsantrag gemacht haben. Es ist umstritten,
                  ob dieser Bericht wahr ist.}}}\label{K_L02621-6}, wie ich glaube. Hat vier Jahre lang mit ihm
               gelebt und gearbeitet. Ungeheures Wiſſen, Philo{\pb}ſophin\pwindex{Andreas-Salome, Lou 12.02.1861 – 05.02.1937@\textsc{Andreas-Salomé, Lou} (12.02.1861 – 05.02.1937), \emph{Schriftsteller/Schriftstellerin}|pwv} vom Fach. Hat
               ein merkwürdiges Buch\pwindex{Friedrich Nietzsche in seinen Werken@\emph{Friedrich Nietzsche in seinen Werken}|pwv} über \textsc{Nietzsche\pwindex{Nietzsche, Friedrich 15.10.1844 – 25.08.1900@\textsc{Nietzsche, Friedrich} (15.10.1844 – 25.08.1900), \emph{Schriftsteller/Schriftstellerin, Philosoph/Philosophin}|pw}} veröffentlicht. Specialität: Religions-Philosophie. Nun gut: Sie weilt ſeit
               einigen Wochen in \textsc{Paris\oindex{Paris@\textbf{Paris}, \emph{P.PPLC}|pw}}, und ſie ſchickt Dir dieſen \label{K_L02621-7v}\edtext{Brief}{\lemma{\textnormal{\emph{Brief}}}\Cendnote{\textnormal{Womöglich handelt es sich um
                  den Brief Andreas-Salomés\pwindex{Andreas-Salome, Lou 12.02.1861 – 05.02.1937@\textsc{Andreas-Salomé, Lou} (12.02.1861 – 05.02.1937), \emph{Schriftsteller/Schriftstellerin}|pwk} an Schnitzler vom 15. 5. 1894.}}}\label{K_L02621-7}. Willſt Du ihr
               antworten, ſo thus durch mich.\pend
           
\pstart
           Alſo es \strikeout{\textcolor{gray}{wa}} wird in Wien\oindex{Wien@\textbf{Wien}, \emph{A.ADM2}|pw} dieſe neue Revüe\pwindex{Zeit. Wiener Wochenschrift@\emph{Die Zeit. Wiener Wochenschrift}|pwv} begründet. Bitte ſchreib’ mir, was Du
               davon weißt und glaubſt (Zukunft). Ich habe die Empfindung, daß man ſich bei dieſer
               Gründung infam gegen mich benimmt. \textsc{Kanner\pwindex{Kanner, Heinrich 09.11.1864 – 15.02.1930@\textsc{Kanner, Heinrich} (09.11.1864 – 15.02.1930), \emph{Herausgeber/Herausgeberin, Publizist/Publizistin}|pw}} – Du weißt, wie hoch ich ſein Talent ſchätze, in welchem {\pb}wahrhaft geniale Züge ſind – iſt der intime Freund\pwindex{Kanner, Heinrich 09.11.1864 – 15.02.1930@\textsc{Kanner, Heinrich} (09.11.1864 – 15.02.1930), \emph{Herausgeber/Herausgeberin, Publizist/Publizistin}|pwv} meines Onkels\pwindex{Mamroth, Fedor 21.02.1851 – 25.06.1907@\textsc{Mamroth, Fedor} (21.02.1851 – 25.06.1907), \emph{Journalist/Journalistin, Kritiker/Kritikerin}|pwv} und meiner Familie. Mit
               mir ſteht er ſchlecht. Dieſer überlegen geſcheite Menſch\pwindex{Kanner, Heinrich 09.11.1864 – 15.02.1930@\textsc{Kanner, Heinrich} (09.11.1864 – 15.02.1930), \emph{Herausgeber/Herausgeberin, Publizist/Publizistin}|pwv} begeht die Dummheit, mir die Jahre hindurch
               nachzutragen, daß ich mich einmal in einem Geſpräch \strikeout{über} ihm gegenüber ironiſch-neckend über einige ſeiner Artikel ausgedrückt,
               die ich ſtets ehrlich bewundert habe. Und nun: Iſt es Haß? Iſt es Neid? Iſt es
               Verachtung? – bei dieſer Neugründung ignorirt er mich vollſtändig. Es hätte {\pb}ſich unbedingt gehört, daß man mich aufforderte, von
                  \textsc{Paris\oindex{Paris@\textbf{Paris}, \emph{P.PPLC}|pw}} aus für das Blatt\pwindex{Zeit. Wiener Wochenschrift@\emph{Die Zeit. Wiener Wochenschrift}|pwv} thätig
               zu ſein. Ich hätte es kaum je annehmen können, aber eine Einladung hätte erfolgen
               müſſen. Statt deſſen iſt \textsc{Bahr\pwindex{Bahr, Hermann 19.07.1863 – 15.01.1934@\textsc{Bahr, Hermann} (19.07.1863 – 15.01.1934), \emph{Schriftsteller/Schriftstellerin, Kritiker/Kritikerin}|pw}} ſeit geſtern in \textsc{Paris\oindex{Paris@\textbf{Paris}, \emph{P.PPLC}|pw}}, um \textsc{Albert\pwindex{Albert, Henri 1869-11-16 – 1921-08-03@\textsc{Albert, Henri} (1869-11-16 – 1921-08-03), \emph{Journalist/Journalistin, Kritiker/Kritikerin, Übersetzer/Übersetzerin}|pw}} die Pariſ\oindex{Paris@\textbf{Paris}, \emph{P.PPLC}|pw}er Vertretung zu übertragen. Ich
               habe ſelbſtverſtändlich \textsc{Albert\pwindex{Albert, Henri 1869-11-16 – 1921-08-03@\textsc{Albert, Henri} (1869-11-16 – 1921-08-03), \emph{Journalist/Journalistin, Kritiker/Kritikerin, Übersetzer/Übersetzerin}|pw}} zur Annahme gedrängt, da das in ſeinem Intereſſe iſt\textcolor{gray}{.} Aber
               die Kränkung iſt nichtsdeſtoweniger ſehr bitter. Da ſiehſt Du einmal in einem
               praktiſchen Falle, wie falſch Deine freundſchaftlichen Anſichten über meine Geltung
               ſind.\pend
           
\pstart
           {\pb}Ich habe gethan, was ich thun konnte, um eine
               Beſprechung des »\textsc{Anatol\pwindex{Anatol@\emph{Anatol}|pw}}« in der Frkf. Ztg.\pwindex{Frankfurter Zeitung@\emph{Frankfurter Zeitung}|pw} durchzuſetzen.
                  V\textcolor{gray}{o}rgebens der wahre Grund ſind gewiſſe \strikeout{inne} innere Vorgänge zwiſchen meinem Onkel\pwindex{Mamroth, Fedor 21.02.1851 – 25.06.1907@\textsc{Mamroth, Fedor} (21.02.1851 – 25.06.1907), \emph{Journalist/Journalistin, Kritiker/Kritikerin}|pwv} und mir, die ich Dir einmal mündlich
               erklären werde. Hingegen habe ich eine Beſprechung\pwindex{Neue Romane und Novellen@\emph{Neue Romane und Novellen}|pwv} für \textsc{Richard\pwindex{Beer-Hofmann, Richard 1866-07-11 – 1945-09-26@\textsc{Beer-Hofmann, Richard} (1866-07-11 – 1945-09-26), \emph{Schriftsteller/Schriftstellerin}|pw}} erwirkt. Nun haben aber die Referenten das Recht ungehindert ſeiner
               Meisungs-Äußerung bei uns, und das dumme Frauenzimmer\pwindex{Meyerhof, Leonie 02.03.1858 – 15.08.1933@\textsc{Meyerhof, Leonie} (02.03.1858 – 15.08.1933), \emph{Schriftsteller/Schriftstellerin, Journalist/Journalistin}|pwv}, das bei uns die deutſche Literatur voranleitet,
               hat \textsc{Richards\pwindex{Beer-Hofmann, Richard 1866-07-11 – 1945-09-26@\textsc{Beer-Hofmann, Richard} (1866-07-11 – 1945-09-26), \emph{Schriftsteller/Schriftstellerin}|pw}}{ }\strikeout{\textsc{B}}{ }Buch\pwindex{Novellen@\emph{Novellen}|pwv} abſolut nicht \label{K_L02621-8v}\edtext{ver{\pb}ſtanden\pwindex{Neue Romane und Novellen@\emph{Neue Romane und Novellen}|pwv}}{\lemma{\textnormal{\emph{verſtanden}}}\Cendnote{\textnormal{Leo Hildeck\pwindex{Meyerhof, Leonie 02.03.1858 – 15.08.1933@\textsc{Meyerhof, Leonie} (02.03.1858 – 15.08.1933), \emph{Schriftsteller/Schriftstellerin, Journalist/Journalistin}|pwk} [ = Leonie Meyerhof\pwindex{Meyerhof, Leonie 02.03.1858 – 15.08.1933@\textsc{Meyerhof, Leonie} (02.03.1858 – 15.08.1933), \emph{Schriftsteller/Schriftstellerin, Journalist/Journalistin}|pwk}]: \emph{Neue
                        Romane und Novellen}\pwindex{Neue Romane und Novellen@\emph{Neue Romane und Novellen}|pwk}. In: \emph{Frankfurter
                        Zeitung}\pwindex{Frankfurter Zeitung@\emph{Frankfurter Zeitung}|pwk}, Jg. 38, Nr. 142, 24. 5. 1894, Erstes Morgenblatt,
                     S. 1–2.}}}\label{K_L02621-8}. Dafür kann ich nichts, und ich kann es nur bedauern. Ich
               habe das Ehrenwort meines Onkels\pwindex{Mamroth, Fedor 21.02.1851 – 25.06.1907@\textsc{Mamroth, Fedor} (21.02.1851 – 25.06.1907), \emph{Journalist/Journalistin, Kritiker/Kritikerin}|pwv}, daß Dein neuer \label{K_L02621-9v}\edtext{Roman}{\lemma{\textnormal{\emph{Roman}}}\Cendnote{\textnormal{Nicht identifiziert.
                  Möglicherweise ging es um Schnitzlers Erzählung \emph{Blumen}\pwindex{Blumen@\emph{Blumen}|pwk}, deren Abdruck in der \emph{Frankfurter
                     Zeitung}\pwindex{Frankfurter Zeitung@\emph{Frankfurter Zeitung}|pwk}{ }Mamroth\pwindex{Mamroth, Fedor 21.02.1851 – 25.06.1907@\textsc{Mamroth, Fedor} (21.02.1851 – 25.06.1907), \emph{Journalist/Journalistin, Kritiker/Kritikerin}|pwk} jedenfalls am 4. 4. 1894 freundlich abgelehnt hattee.}}}\label{K_L02621-9}
               beſprochen wird, ſobald er in Buchform erſchienen iſt.\pend
           
\pstart
           Wenn ich keinen ſchweren Krankheitsanfall bekomme, will ich von meinem
               vierwöchentlichen Urlaub drei auf eine Reiſe verwenden. Ich habe keinen höheren
               Wunſch, als dieſe drei Wochen mit Dir zu verbringen. Aber das muß im \label{K_L02621-10v}\edtext{Auguſt}{\lemma{\textnormal{\emph{Auguſt}}}\Cendnote{\textnormal{Vom 23. 8. 1894 bis 3. 9. 1894 verbrachten Schnitzler und Goldmann\pwindex{Goldmann, Paul 31.01.1865 – 25.09.1935@\textsc{Goldmann, Paul} (31.01.1865 – 25.09.1935), \emph{Schriftsteller/Schriftstellerin, Journalist/Journalistin}|pwk} einige Zeit gemeinsam in Bad
                     Ischl\oindex{Bad Ischl@\textbf{Bad Ischl}, \emph{P.PPL}|pwk} und Bad Aussee\oindex{Bad Aussee@\textbf{Bad Aussee}, \emph{P.PPLA3}|pwk}.}}}\label{K_L02621-10} ſein.
               Kannſt du fort? Und wohin? Bitte, ſchreib’ mir bald darüber.\pend
           
\pstart
           {\pb}Oh dieſe Hypochondrie in Deinem letzten Briefe!
               Gewiß, es iſt wünſchenswerth frei zu ſein. Aber ich habe oft über die Freiheit
               nachgedacht, und ich fürchte beinahe, daß ſie doch nicht das Gut iſt, \substVorne{}\textsuperscript{daß}\substDazwischen{}das\substHinten{} wir glauben. Man würde glücklich auf allen Seiten Wege vor ſich ſehen. Und
               ich wenigſtens gehöre nicht zu den Leuten, die raſch entſchloſſen einen von den
               hundert Wegen einſchlagen, ſondern zu denen, die all’ ihr Leben lang damit vertändeln
               würden, davor zu ſtehen {\pb}und zu überlegen: ſoll ich
               dahin gehen oder dorthin? Und würde ich einen Weg wählen, welchen immer, ſo würde
               mich bis an meinen Tod die Reue verfolgen, daß ich nicht den andern eingeſchlagen.
               Biſt Du nicht auch ein wenig ſo? Gewiß, der Zwang iſt drückend. Aber es hat auch ſein
               gutes: es erſpart einem die Mühe der Wahl und die Verantwortung dafür. Der Zwang,
                  \label{K_L02621-11v}\edtext{\textsc{\begin{otherlanguage}{french}c’est une destinée toute faite\end{otherlanguage}}}{\lemma{\textnormal{\emph{c’est … faite}}}\Cendnote{\textnormal{französisch, etwa: das Schicksal ist
                  vorbestimmt}}}\label{K_L02621-11}. Und wenn er, wie bei Dir, nicht mit Infamie verbunden iſt (wie
               bei mir), ſo ſollte man ihn {\pb}ruhig tragen, zumal
               wenn man dabei auch noch graduieren kann. Wer weiß, ob nicht gerade in Deiner Abſcheu
               davor, ein ärztlicher \strikeout{ban} Banauſe zu werden, ein
               gutes Theil Deiner Productionskraft liegt. Und wer weiß, ob dieſe, die vielleicht zum
               großen Theil eine Reaktionserſcheinung iſt, nicht ſehr abnehmen würde, wenn auf der
               andern Seite die Aktion des Zwanges aufhörte. Dabei fällt mir ein, daß es im Obigen
               nicht Productions-Kraft heißen darf, ſondern »Wille zur Produktion«. Auch ſonſt habe
               ich es mir ganz {\pb}anders gedacht, als es da
               ausgedrückt iſt. Das macht aber nichts.\pend
           
\pstart
           Die von Dir erwähnte \label{K_L02621-12v}\edtext{Erwiderung\pwindex{Dekadent@\emph{Der Dekadent}|pwv} von \textsc{Christensen\pwindex{Christensen, Hjalmar 1869-05-05 – 1925-12-29@\textsc{Christensen, Hjalmar} (1869-05-05 – 1925-12-29), \emph{Kritiker/Kritikerin, Kunstschriftsteller/Kunstschriftstellerin}|pw}\pwindex{Dekadent@\emph{Der Dekadent}|pwv}}}{\lemma{\textnormal{\emph{Erwiderung von Christensen}}}\Cendnote{\textnormal{Hjalmar Christensen\pwindex{Christensen, Hjalmar 1869-05-05 – 1925-12-29@\textsc{Christensen, Hjalmar} (1869-05-05 – 1925-12-29), \emph{Kritiker/Kritikerin, Kunstschriftsteller/Kunstschriftstellerin}|pwk}: \emph{Der Dekadent}\pwindex{Dekadent@\emph{Der Dekadent}|pwk}. In: \emph{Frankfurter Zeitung}\pwindex{Frankfurter Zeitung@\emph{Frankfurter Zeitung}|pwk}, Jg. 38, Nr. 103, 14. 4. 1894, Erstes
                     Morgenblatt, S. 1–2. Eine unmittelbare Reaktion auf diesen Text lässt
                  sich nicht nachweisen, sehr wohl aber eine wohlwollende Erwähnung\pwindex{Decadence@\emph{Décadence}|pwkv} in der \emph{Neuen deutschen Rundschau}\pwindex{Neue Deutsche Rundschau@\emph{Neue Deutsche Rundschau}|pwk} vom Mai 1894
                     (Jg. 5, Nr. 5, S. 522–523). In der \emph{Neuen deutschen Rundschau}\pwindex{Neue Deutsche Rundschau@\emph{Neue Deutsche Rundschau}|pwk} findet sich auch ein Hinweis auf eine kritische
                  Einordnung von jüngeren Wien\oindex{Wien@\textbf{Wien}, \emph{A.ADM2}|pwk}er Autoren –
                  darunter Schnitzler, Hofmannstal\pwindex{Hofmannsthal, Hugo von 1874-02-01 – 1929-07-15@\textsc{Hofmannsthal, Hugo von} (1874-02-01 – 1929-07-15), \emph{Schriftsteller/Schriftstellerin}|pwk} und Bahr\pwindex{Bahr, Hermann 19.07.1863 – 15.01.1934@\textsc{Bahr, Hermann} (19.07.1863 – 15.01.1934), \emph{Schriftsteller/Schriftstellerin, Kritiker/Kritikerin}|pwk} – durch Stauf von der March\pwindex{Stauf von der March, Ottokar 29.08.1868 – 12.03.1941@\textsc{Stauf von der March, Ottokar} (29.08.1868 – 12.03.1941), \emph{Schriftsteller/Schriftstellerin, Journalist/Journalistin}|pwk} (Ottokar Stauf von der March\pwindex{Stauf von der March, Ottokar 29.08.1868 – 12.03.1941@\textsc{Stauf von der March, Ottokar} (29.08.1868 – 12.03.1941), \emph{Schriftsteller/Schriftstellerin, Journalist/Journalistin}|pwk}: \emph{Décadence. Randglossen}\pwindex{Decadence. Randglossen@\emph{Décadence. Randglossen}|pwk}. In: \emph{Die Gesellschaft}\pwindex{Gesellschaft. Monatsschrift fuer Litteratur, Kunst und Sozialpolitik@\emph{Die Gesellschaft. Monatsschrift für Litteratur, Kunst und Sozialpolitik}|pwk}, Jg. 10, H. 4, April 1894, S.                      526–533). Über Schnitzler steht
                  darin: »Der hervorragendste aller Dekadenten ist der schon öfter erwähnte
                        Wien\oindex{Wien@\textbf{Wien}, \emph{A.ADM2}|pw}er Arthur Schnitzler. Obgleich seine Dichtungen, vornehmlich:
                     Scenenbilder (›Anatol\pwindex{Anatol@\emph{Anatol}|pw}‹), vom denkbar
                     stärksten Décadence-Kolorit durchsättigt sind und darum den Leser in die
                     unbehaglichste Stimmung von der Welt versetzen, erscheinen sie doch durch ihre
                     Aufrichtigkeit und Selbsterkenntnis geadelt. Mit peinlicher Akkuratesse seziert
                     der Dichter seine
                     Probleme und erklärt dem staunenden Leser resigniert-lächelnd die angefaulten
                     Körperstellen. An Geist vermag sich mit ihm kein einziger Dekadent zu messen.
                        Schnitzlers Werke sprühen förmlich von
                     genialen Gedanken und Sentenzen. Er ist gewissermaßen der Klassiker der
                     Décadence, aber darum nicht minder krank, als die übrigen.«
                     (S. 531.)}}}\label{K_L02621-12} habe ich nirgends entdecken können. Könnteſt Du
               mir nicht die Nummer oder nur die ungefähre Erſcheinungs-Zeit angeben?\pend
           
\pstart
           Und \textsc{Richard\pwindex{Beer-Hofmann, Richard 1866-07-11 – 1945-09-26@\textsc{Beer-Hofmann, Richard} (1866-07-11 – 1945-09-26), \emph{Schriftsteller/Schriftstellerin}|pw}}? Und \textsc{Loris\pwindex{Hofmannsthal, Hugo von 1874-02-01 – 1929-07-15@\textsc{Hofmannsthal, Hugo von} (1874-02-01 – 1929-07-15), \emph{Schriftsteller/Schriftstellerin}|pw}}?\pend
           
\pstart
           Bitte, lies: \textsc{Bernard Lazare\pwindex{Lazare, Bernard 1865-06-15 – 1903-09-01@\textsc{Lazare, Bernard} (1865-06-15 – 1903-09-01), \emph{Journalist/Journalistin, Anarchist/Anarchistin}|pw}}: \textsc{L’Antisémitisme\pwindex{antisemitisme. Son histoire et ses causes@\emph{L’antisémitisme. Son histoire et ses causes}|pw}}. Soeben erſchienen bei \textsc{Léon Challey\pwindex{Chailley, Leon @\textsc{Chailley, Léon}, \emph{Verleger/Verlegerin}|pw}}, \textsc{8. Rue Saint-Joseph\oindex{Rue Saint-Joseph@\textbf{Rue Saint-Joseph}, \emph{Straße (K.STR)}|pw}}. Der Verfaſſer\pwindex{Lazare, Bernard 1865-06-15 – 1903-09-01@\textsc{Lazare, Bernard} (1865-06-15 – 1903-09-01), \emph{Journalist/Journalistin, Anarchist/Anarchistin}|pwv}, in
               unſerem Alter, iſt ſelbst Jude.\pend
           
\pstart
           Mein Schwager\pwindex{Rosengart, Josef 1860-02-08 – 1927-08-04@\textsc{Rosengart, Josef} (1860-02-08 – 1927-08-04), \emph{Arzt/Ärztin}|pwv} iſt
               hochbeglückt mit Deiner \label{K_L02621-13v}\edtext{Zeitſchrift\pwindex{Internationale klinische Rundschau@\emph{Internationale klinische Rundschau}|pwv}}{\lemma{\textnormal{\emph{Zeitſchrift}}}\Cendnote{\textnormal{Siehe Paul Goldmann an Arthur Schnitzler, 4. 11. [1893].
               }}}\label{K_L02621-13} und dankt Dir noch vielmals.\pend
           
\pstart
           Viele treue Grüße! {\\[\baselineskip]}Dein {\\[\baselineskip]}\spacefill\mbox{Paul Goldmann}\pend
           \leftskip=0em{}
\pstart
           \noindent{}Schreib’ bald!!\pend
           \selectlanguage{ngerman}\endnumbering\briefempfaengerindex{Schnitzler, Arthur@\textsc{Schnitzler, Arthur}!zzzGoldmann, Paul@\emph{von Paul Goldmann}!1894-05-292@{29. 5. {[}1894{]}}|)be}\mylabel{L02621h}  \normalsize

\doendnotes{C}
\bigskip
\vfill

\clearpage

\footnotesize

\lohead{\textsc{register}}

% Definiere theindex-Environment komplett neu ohne reledmac
\makeatletter
\renewenvironment{theindex}{%
  \section*{\indexname}%
  \setlength{\parindent}{0pt}%
  \setlength{\parskip}{0pt plus 0.3pt}%
  \let\item\@idxitem
}{%
  \clearpage
}
\makeatother

\IfFileExists{\jobname-pw.ind}{\input{\jobname-pw.ind}}{}

\end{document}

      