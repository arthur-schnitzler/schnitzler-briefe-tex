%% latex-leseansicht-vorspann.tex
%% Vorspann für die Leseansicht.
%% Lädt die gemeinsame Datei latex-vorspann.tex mit nicht gesetztem Schalter.

\newif\ifkorrekturansicht
\korrekturansichtfalse

\input{../tex-inputs/latex-vorspann}


\section[Paul Goldmann an Arthur Schnitzler, 29. 5. {[}1894{]}]{L02621 Paul Goldmann an Arthur Schnitzler, 29. 5. [1894]}
\nopagebreak\mylabel{L02621v}
\rehead{ }\normalsize\beginnumbering\briefempfaengerindex{Schnitzler, Arthur@\textsc{Schnitzler, Arthur}!zzzGoldmann, Paul@\emph{von Paul Goldmann}!1894-05-292@{29. 5. [1894]}|(be}
\toendnotes[C]{\smallbreak\pagebreak[2]}
\correspDesc{Versand  durch Paul Goldmann am 29. 5. [1894] in Paris
\newline{}Erhalt  durch Arthur Schnitzler im Zeitraum [30. 5. 1894
                  – 3. 6. 1894?] in Wien}\toendnotes[C]{\smallbreak}
\Standort{DLA, A:Schnitzler, HS.NZ85.1.3164.}
\physDesc{Brief, 3 Blätter, 12 Seiten, 5941 Zeichen
\newline{}Handschrift: schwarze Tinte, deutsche Kurrent
\newline{}Schnitzler: 1) mit Bleistift auf dem ersten Blatt die Jahreszahl »94« vermerkt  2) mit rotem Buntstift fünf Unterstreichungen}\toendnotes[C]{\smallbreak}
\pstart
           {\pb}\textcolor{gray}{\textbf{Frankfurter Zeitung\orgindex{Frankfurter Zeitung@Frankfurter Zeitung|pw}}}\pend
           
\pstart
           \textcolor{gray}{\textbf{(Gazette de
                     Francfort\orgindex{Frankfurter Zeitung@Frankfurter Zeitung|pw}).}}\pend
           
\pstart
           \textcolor{gray}{\textbf{\begin{otherlanguage}{french}Fondateur \textbf{M. L. Sonnemann\pwindex{Sonnemann, Leopold 29.\,10.\,1831 Höchberg – 30.\,10.\,1909 Frankfurt am Main@\textsc{Sonnemann, Leopold} (29.\,10.\,1831 Höchberg – 30.\,10.\,1909 Frankfurt am Main), \emph{Journalist, Herausgeber}|pw}}.\end{otherlanguage}}}\hfill \textsc{Paris\oindex{Paris@\textbf{Paris}, \emph{Hauptstadt}|pw}}, 29. Mai.\pend
           
\pstart
           \textcolor{gray}{\textbf{\begin{otherlanguage}{french}Journal politique, financier,\end{otherlanguage}}}\pend
           
\pstart
           \textcolor{gray}{\textbf{\begin{otherlanguage}{french}commercial et littéraire.\end{otherlanguage}}}\pend
           
\pstart
           \textcolor{gray}{\textbf{\begin{otherlanguage}{french}\textbf{Paraissant trois fois par jour}\end{otherlanguage}}}.\pend
           
\pstart
           \textcolor{gray}{\textbf{\begin{otherlanguage}{french}\textbf{Bureaux à Paris\oindex{Paris@\textbf{Paris}, \emph{Hauptstadt}|pw}:}\end{otherlanguage}}}\pend
           
\pstart
           \textcolor{gray}{\textbf{\begin{otherlanguage}{french}\textbf{24. Rue Feydeau}\oindex{rue Feydeau@\textbf{rue Feydeau}, \emph{Straße}|pw}.\end{otherlanguage}}}\pend
           
\pstart\center{}Mein lieber Freund,\pend\vspace{0.5em}
\pstart
           Ich war acht Tage in Frankfurt\oindex{Frankfurt am Main@\textbf{Frankfurt am Main}, \emph{Hauptstadt}|pw}; Krankheit meines
                  Onkels\pwindex{Mamroth, Fedor 21.\,2.\,1851 Breslau – 25.\,6.\,1907 Frankfurt am Main@\textsc{Mamroth, Fedor} (21.\,2.\,1851 Breslau – 25.\,6.\,1907 Frankfurt am Main), \emph{Journalist, Kritiker}|pwv} und
               meiner Mutter\pwindex{Goldmann, Clementine 15.\,5.\,1842 Breslau – 24.\,2.\,1924 Frankfurt am Main@\textsc{Goldmann, Clementine} (15.\,5.\,1842 Breslau – 24.\,2.\,1924 Frankfurt am Main)|pwv}. Bei meiner
               Rückkehr fand ich Deine Briefe. \label{K_L02621-1v}\edtext{Miniſterſturz und Miniſter-Kriſis}{\lemma{\textnormal{\emph{Ministersturz und Minister-Krisis}}}\Cendnote{\textnormal{Gemeint war der am 22. 5. 1894 vollzogene (erzwungene) Rücktritt des
                  Kabinetts von Jean Casimir-Perier\pwindex{Casimir-Perier, Jean 8.\,11.\,1847 Paris – 11.\,3.\,1907 ebd.@\textsc{Casimir-Perier, Jean} (8.\,11.\,1847 Paris – 11.\,3.\,1907 ebd.), \emph{Politiker, Präsident}|pwk}.}}}\label{K_L02621-1}
               geben Tauſenderlei zu thun. So komme ich erſt heut
               dazu, Dir zu antworten.\pend
           
\pstart
           Ich habe das \label{K_L02621-2v}\edtext{Geld}{\lemma{\textnormal{\emph{Geld}}}\Cendnote{\textnormal{Siehe XXXX Auszeichnungsfehler: Dokument L02619 nicht gefunden.
               }}}\label{K_L02621-2}{ }ſofort an \textsc{Albert\pwindex{Albert, Henri 16.\,11.\,1869 Niederbronn-les-Bains – 3.\,8.\,1921 Straßburg@\textsc{Albert, Henri} (16.\,11.\,1869 Niederbronn-les-Bains – 3.\,8.\,1921 Straßburg), \emph{Journalist, Kritiker, Übersetzer}|pw}} übergeben. Es iſt blödſinnig: aber ich kam mir vor, als wenn ich einen Raub an
               Dir beginge. Trotzdem geht Alles ehrlich zu. Aber das iſt mein Wahn, und noch heut{ }{\pb}iſt es mir unangenehm, davon zu{ }ſprechen. \textsc{Albert\pwindex{Albert, Henri 16.\,11.\,1869 Niederbronn-les-Bains – 3.\,8.\,1921 Straßburg@\textsc{Albert, Henri} (16.\,11.\,1869 Niederbronn-les-Bains – 3.\,8.\,1921 Straßburg), \emph{Journalist, Kritiker, Übersetzer}|pw}} bewährt{ }ſich{ }ſehr als mein Freund, folglich auch als Deiner. Gutes, feines,
               anſchmiegendes, liebes Naturell! Wir machen große Schlachtpläne für Dich. Ich glaube,
               er hat Dir \label{K_L02621-3v}\edtext{darüber geſchrieben}{\lemma{\textnormal{\emph{darüber geschrieben}}}\Cendnote{\textnormal{Alberts\pwindex{Albert, Henri 16.\,11.\,1869 Niederbronn-les-Bains – 3.\,8.\,1921 Straßburg@\textsc{Albert, Henri} (16.\,11.\,1869 Niederbronn-les-Bains – 3.\,8.\,1921 Straßburg), \emph{Journalist, Kritiker, Übersetzer}|pwk} Brief vom 23. 5. 1894
                  enthält neben dem Vorhaben, das ›\emph{Abschiedsouper}\pwindex{Schnitzler, Arthur 15.\,5.\,1862 Wien – 21.\,10.\,1931 ebd.@\textsc{Schnitzler, Arthur} (15.\,5.\,1862 Wien – 21.\,10.\,1931 ebd.), \emph{Schriftsteller, Mediziner}!Abschiedssouper@\strich\emph{Abschiedssouper}|pwk}‹ bei einer Freien Bühne aufführen zu lassen, mehrere
                  Textvorhaben: \emph{Denksteine}\pwindex{Schnitzler, Arthur 15.\,5.\,1862 Wien – 21.\,10.\,1931 ebd.@\textsc{Schnitzler, Arthur} (15.\,5.\,1862 Wien – 21.\,10.\,1931 ebd.), \emph{Schriftsteller, Mediziner}!Denksteine@\strich\emph{Denksteine}|pwk} und von ihm noch
                  nicht gelesene Textmanuskripte (\emph{Die
                     überspannte Person}\pwindex{Schnitzler, Arthur 15.\,5.\,1862 Wien – 21.\,10.\,1931 ebd.@\textsc{Schnitzler, Arthur} (15.\,5.\,1862 Wien – 21.\,10.\,1931 ebd.), \emph{Schriftsteller, Mediziner}!überspannte Person@\strich\emph{Die überspannte Person}|pwk} und \emph{Halb Zwei}\pwindex{Schnitzler, Arthur 15.\,5.\,1862 Wien – 21.\,10.\,1931 ebd.@\textsc{Schnitzler, Arthur} (15.\,5.\,1862 Wien – 21.\,10.\,1931 ebd.), \emph{Schriftsteller, Mediziner}!Halb Zwei@\strich\emph{Halb Zwei}|pwk}, vgl. XXXX Auszeichnungsfehler: Dokument L02621 nicht gefunden) möchte er gegen Ende des Sommers im \emph{Mercure de France}\pwindex{Mercure de France@\emph{Mercure de France}|pwk} gedruckt sehen. Zusätzlich zu seiner
                  bevorstehenden Rezension\pwindex{Albert, Henri 16.\,11.\,1869 Niederbronn-les-Bains – 3.\,8.\,1921 Straßburg@\textsc{Albert, Henri} (16.\,11.\,1869 Niederbronn-les-Bains – 3.\,8.\,1921 Straßburg), \emph{Journalist, Kritiker, Übersetzer}!Lettres allemandes. Drames Nouveaux@\strich\emph{Les Lettres allemandes. Drames Nouveaux}|pwkv}
                  von \emph{Das Märchen}\pwindex{Schnitzler, Arthur 15.\,5.\,1862 Wien – 21.\,10.\,1931 ebd.@\textsc{Schnitzler, Arthur} (15.\,5.\,1862 Wien – 21.\,10.\,1931 ebd.), \emph{Schriftsteller, Mediziner}!Märchen. Schauspiel in drei Aufzügen@\strich\emph{Das Märchen. Schauspiel in drei Aufzügen}|pwk} in der \emph{Revue Blanche}\pwindex{Revue blanche@\emph{La Revue blanche}|pwk} plante er, in derselben Zeitschrift über die
                     »Jungen Wien\oindex{Wien@\textbf{Wien}, \emph{Verwaltungsgebiet}|pw}er« zu
                  schreiben.}}}\label{K_L02621-3}. Vielleicht gelingt es gar, Dich \label{K_L02621-4v}\edtext{aufführen}{\lemma{\textnormal{\emph{aufführen}}}\Cendnote{\textnormal{Aus
                  dieser Zeit sind keine Aufführungen in Paris\oindex{Paris@\textbf{Paris}, \emph{Hauptstadt}|pwk}
                  bekannt.}}}\label{K_L02621-4} zu laſſen. Ich denke, im nächſten Heft des »\textsc{Mercure\pwindex{Mercure de France@\emph{Mercure de France}|pw}}« wird \label{K_L02621-5v}\edtext{\textsc{Albert\pwindex{Albert, Henri 16.\,11.\,1869 Niederbronn-les-Bains – 3.\,8.\,1921 Straßburg@\textsc{Albert, Henri} (16.\,11.\,1869 Niederbronn-les-Bains – 3.\,8.\,1921 Straßburg), \emph{Journalist, Kritiker, Übersetzer}|pw}} Dein »Märchen\pwindex{Schnitzler, Arthur 15.\,5.\,1862 Wien – 21.\,10.\,1931 ebd.@\textsc{Schnitzler, Arthur} (15.\,5.\,1862 Wien – 21.\,10.\,1931 ebd.), \emph{Schriftsteller, Mediziner}!Märchen. Schauspiel in drei Aufzügen@\strich\emph{Das Märchen. Schauspiel in drei Aufzügen}|pw}« beſprechen}{\lemma{\textnormal{\emph{Albert … besprechen}}}\Cendnote{\textnormal{Alberts\pwindex{Albert, Henri 16.\,11.\,1869 Niederbronn-les-Bains – 3.\,8.\,1921 Straßburg@\textsc{Albert, Henri} (16.\,11.\,1869 Niederbronn-les-Bains – 3.\,8.\,1921 Straßburg), \emph{Journalist, Kritiker, Übersetzer}|pwk}{ }Rezension\pwindex{Albert, Henri 16.\,11.\,1869 Niederbronn-les-Bains – 3.\,8.\,1921 Straßburg@\textsc{Albert, Henri} (16.\,11.\,1869 Niederbronn-les-Bains – 3.\,8.\,1921 Straßburg), \emph{Journalist, Kritiker, Übersetzer}!Lettres allemandes. Drames Nouveaux@\strich\emph{Les Lettres allemandes. Drames Nouveaux}|pwkv} erschien nicht im \emph{Mercure de France}\pwindex{Mercure de France@\emph{Mercure de France}|pwk}, sondern in der \emph{Revue Blanche}\pwindex{Revue blanche@\emph{La Revue blanche}|pwk}: Henri Albert\pwindex{Albert, Henri 16.\,11.\,1869 Niederbronn-les-Bains – 3.\,8.\,1921 Straßburg@\textsc{Albert, Henri} (16.\,11.\,1869 Niederbronn-les-Bains – 3.\,8.\,1921 Straßburg), \emph{Journalist, Kritiker, Übersetzer}|pwk}: \emph{Les Lettres allemandes. Drames Nouveaux}\pwindex{Albert, Henri 16.\,11.\,1869 Niederbronn-les-Bains – 3.\,8.\,1921 Straßburg@\textsc{Albert, Henri} (16.\,11.\,1869 Niederbronn-les-Bains – 3.\,8.\,1921 Straßburg), \emph{Journalist, Kritiker, Übersetzer}!Lettres allemandes. Drames Nouveaux@\strich\emph{Les Lettres allemandes. Drames Nouveaux}|pwk}. In: \emph{La Revue Blanche}\pwindex{Revue blanche@\emph{La Revue blanche}|pwk}, Jg. 6, Nr. 32,
                        Juni 1894, S. 556–560, hier: S. 560.}}}\label{K_L02621-5}. Von den
               zwei Manuſkripten\pwindex{Schnitzler, Arthur 15.\,5.\,1862 Wien – 21.\,10.\,1931 ebd.@\textsc{Schnitzler, Arthur} (15.\,5.\,1862 Wien – 21.\,10.\,1931 ebd.), \emph{Schriftsteller, Mediziner}!überspannte Person@\strich\emph{Die überspannte Person}|pwv}\pwindex{Schnitzler, Arthur 15.\,5.\,1862 Wien – 21.\,10.\,1931 ebd.@\textsc{Schnitzler, Arthur} (15.\,5.\,1862 Wien – 21.\,10.\,1931 ebd.), \emph{Schriftsteller, Mediziner}!Halb Zwei@\strich\emph{Halb Zwei}|pwv},
               insbeſondere von der »Überſpannten
                  Perſon\pwindex{Schnitzler, Arthur 15.\,5.\,1862 Wien – 21.\,10.\,1931 ebd.@\textsc{Schnitzler, Arthur} (15.\,5.\,1862 Wien – 21.\,10.\,1931 ebd.), \emph{Schriftsteller, Mediziner}!überspannte Person@\strich\emph{Die überspannte Person}|pwv}«{ }ſind wir Alle hoch entzückt. Unterſchied zwiſchen Dir und \textsc{Lavedan\pwindex{Lavedan, Henri Léon 9.\,4.\,1859 Orléans – 4.\,9.\,1940 Paris@\textsc{Lavedan, Henri Léon} (9.\,4.\,1859 Orléans – 4.\,9.\,1940 Paris), \emph{Schriftsteller, Journalist}|pw}} und den \textsc{Lavedanisirenden}{ }Franz\oindex{Frankreich@\textbf{Frankreich}|pwv}oſen: In Frankreich\oindex{Frankreich@\textbf{Frankreich}|pw} Geiſt, Oberflächlichkeit, Dekadenz-Koketterie. Bei
               Dir: Na{\pb}türlichkeit, Tiefe, \uline{Sittlichkeit und Geſundheit} (Thut Dir wahrſcheinlich{ }ſehr weh?). \strikeout{Geiſt} Geiſt natürlich auch. Das Rin\textcolor{gray}{dv}ieh\pwindex{Stauf von der March, Ottokar 29.\,8.\,1868 Olomouc – 12.\,3.\,1941 Wien@\textsc{Stauf von der March, Ottokar} (29.\,8.\,1868 Olomouc – 12.\,3.\,1941 Wien), \emph{Schriftsteller, Journalist}|pwv}, das Dich in
               der Geſellſchaft zum Dekadenten-Häuptling macht, hat uns eine vergnügte Viertelſtunde
               bereitet.\pend
           
\pstart
           Kennſt Du Frau \textsc{Andreas-\strikeout{Sa\textcolor{gray}{l}} Salome\pwindex{Andreas-Salomé, Lou 12.\,2.\,1861 Sankt Petersburg – 5.\,2.\,1937 Göttingen@\textsc{Andreas-Salomé, Lou} (12.\,2.\,1861 Sankt Petersburg – 5.\,2.\,1937 Göttingen), \emph{Schriftstellerin}|pw}}? Seltſame Frau\pwindex{Andreas-Salomé, Lou 12.\,2.\,1861 Sankt Petersburg – 5.\,2.\,1937 Göttingen@\textsc{Andreas-Salomé, Lou} (12.\,2.\,1861 Sankt Petersburg – 5.\,2.\,1937 Göttingen), \emph{Schriftstellerin}|pwv}. Nicht{ }ſchön, ich weiß nicht einmal, ob{ }ſympathiſch, aber derzeit unſere gute Freundin\pwindex{Andreas-Salomé, Lou 12.\,2.\,1861 Sankt Petersburg – 5.\,2.\,1937 Göttingen@\textsc{Andreas-Salomé, Lou} (12.\,2.\,1861 Sankt Petersburg – 5.\,2.\,1937 Göttingen), \emph{Schriftstellerin}|pwv}. Intime Freundin\pwindex{Andreas-Salomé, Lou 12.\,2.\,1861 Sankt Petersburg – 5.\,2.\,1937 Göttingen@\textsc{Andreas-Salomé, Lou} (12.\,2.\,1861 Sankt Petersburg – 5.\,2.\,1937 Göttingen), \emph{Schriftstellerin}|pwv} von \textsc{Nietzsche\pwindex{Nietzsche, Friedrich 15.\,10.\,1844 Röcken – 25.\,8.\,1900 Weimar@\textsc{Nietzsche, Friedrich} (15.\,10.\,1844 Röcken – 25.\,8.\,1900 Weimar), \emph{Schriftsteller, Philosoph}|pw}}. \label{K_L02621-6v}\edtext{Geſchlechtsloſe
                  Freundſchaft}{\lemma{\textnormal{\emph{Geschlechtslose Freundschaft}}}\Cendnote{\textnormal{Rein freundschaftlich war
                  die Beziehung zwischen Nietzsche\pwindex{Nietzsche, Friedrich 15.\,10.\,1844 Röcken – 25.\,8.\,1900 Weimar@\textsc{Nietzsche, Friedrich} (15.\,10.\,1844 Röcken – 25.\,8.\,1900 Weimar), \emph{Schriftsteller, Philosoph}|pwk} und Andreas-Salomé\pwindex{Andreas-Salomé, Lou 12.\,2.\,1861 Sankt Petersburg – 5.\,2.\,1937 Göttingen@\textsc{Andreas-Salomé, Lou} (12.\,2.\,1861 Sankt Petersburg – 5.\,2.\,1937 Göttingen), \emph{Schriftstellerin}|pwk} wahrscheinlich nicht. Wie Andreas-Salomés\pwindex{Andreas-Salomé, Lou 12.\,2.\,1861 Sankt Petersburg – 5.\,2.\,1937 Göttingen@\textsc{Andreas-Salomé, Lou} (12.\,2.\,1861 Sankt Petersburg – 5.\,2.\,1937 Göttingen), \emph{Schriftstellerin}|pwk}{ }\emph{Lebensrückblick}\pwindex{Andreas-Salomé, Lou 12.\,2.\,1861 Sankt Petersburg – 5.\,2.\,1937 Göttingen@\textsc{Andreas-Salomé, Lou} (12.\,2.\,1861 Sankt Petersburg – 5.\,2.\,1937 Göttingen), \emph{Schriftstellerin}!Lebensrückblick. Grundriß einiger Lebenserinnerungen. Aus dem Nachlass herausgegeben@\strich\emph{Lebensrückblick. Grundriß einiger Lebenserinnerungen. Aus dem Nachlass herausgegeben}|pwk} zu entnehmen ist, soll ihr Nietzsche\pwindex{Nietzsche, Friedrich 15.\,10.\,1844 Röcken – 25.\,8.\,1900 Weimar@\textsc{Nietzsche, Friedrich} (15.\,10.\,1844 Röcken – 25.\,8.\,1900 Weimar), \emph{Schriftsteller, Philosoph}|pwk}{ }1892 vergeblich einen Heiratsantrag gemacht haben. Es ist umstritten,
                  ob dieser Bericht wahr ist.}}}\label{K_L02621-6}, wie ich glaube. Hat vier Jahre lang mit ihm
               gelebt und gearbeitet. Ungeheures Wiſſen, Philo{\pb}ſophin\pwindex{Andreas-Salomé, Lou 12.\,2.\,1861 Sankt Petersburg – 5.\,2.\,1937 Göttingen@\textsc{Andreas-Salomé, Lou} (12.\,2.\,1861 Sankt Petersburg – 5.\,2.\,1937 Göttingen), \emph{Schriftstellerin}|pwv} vom Fach. Hat
               ein merkwürdiges Buch\pwindex{Andreas-Salomé, Lou 12.\,2.\,1861 Sankt Petersburg – 5.\,2.\,1937 Göttingen@\textsc{Andreas-Salomé, Lou} (12.\,2.\,1861 Sankt Petersburg – 5.\,2.\,1937 Göttingen), \emph{Schriftstellerin}!Friedrich Nietzsche in seinen Werken@\strich\emph{Friedrich Nietzsche in seinen Werken}|pwv} über \textsc{Nietzsche\pwindex{Nietzsche, Friedrich 15.\,10.\,1844 Röcken – 25.\,8.\,1900 Weimar@\textsc{Nietzsche, Friedrich} (15.\,10.\,1844 Röcken – 25.\,8.\,1900 Weimar), \emph{Schriftsteller, Philosoph}|pw}} veröffentlicht. Specialität: Religions-Philosophie. Nun gut: Sie weilt{ }ſeit
               einigen Wochen in \textsc{Paris\oindex{Paris@\textbf{Paris}, \emph{Hauptstadt}|pw}}, und{ }ſie{ }ſchickt Dir dieſen \label{K_L02621-7v}\edtext{Brief}{\lemma{\textnormal{\emph{Brief}}}\Cendnote{\textnormal{Womöglich handelt es sich um
                  den Brief Andreas-Salomés\pwindex{Andreas-Salomé, Lou 12.\,2.\,1861 Sankt Petersburg – 5.\,2.\,1937 Göttingen@\textsc{Andreas-Salomé, Lou} (12.\,2.\,1861 Sankt Petersburg – 5.\,2.\,1937 Göttingen), \emph{Schriftstellerin}|pwk} an Schnitzler vom XXXX Auszeichnungsfehler: Dokument L00325 nicht gefunden.}}}\label{K_L02621-7}. Willſt Du ihr
               antworten,{ }ſo thus durch mich.\pend
           
\pstart
           Alſo es \strikeout{\textcolor{gray}{wa}} wird in Wien\oindex{Wien@\textbf{Wien}, \emph{Verwaltungsgebiet}|pw} dieſe neue Revüe\pwindex{Zeit. Wiener Wochenschrift@\emph{Die Zeit. Wiener Wochenschrift}|pwv} begründet. Bitte{ }ſchreib’ mir, was Du
               davon weißt und glaubſt (Zukunft). Ich habe die Empfindung, daß man{ }ſich bei dieſer
               Gründung infam gegen mich benimmt. \textsc{Kanner\pwindex{Kanner, Heinrich 9.\,11.\,1864 Galați – 15.\,2.\,1930 Wien@\textsc{Kanner, Heinrich} (9.\,11.\,1864 Galați – 15.\,2.\,1930 Wien), \emph{Herausgeber, Publizist}|pw}} – Du weißt, wie hoch ich{ }ſein Talent{ }ſchätze, in welchem {\pb}wahrhaft geniale Züge{ }ſind – iſt der intime Freund\pwindex{Kanner, Heinrich 9.\,11.\,1864 Galați – 15.\,2.\,1930 Wien@\textsc{Kanner, Heinrich} (9.\,11.\,1864 Galați – 15.\,2.\,1930 Wien), \emph{Herausgeber, Publizist}|pwv} meines Onkels\pwindex{Mamroth, Fedor 21.\,2.\,1851 Breslau – 25.\,6.\,1907 Frankfurt am Main@\textsc{Mamroth, Fedor} (21.\,2.\,1851 Breslau – 25.\,6.\,1907 Frankfurt am Main), \emph{Journalist, Kritiker}|pwv} und meiner Familie. Mit
               mir{ }ſteht er{ }ſchlecht. Dieſer überlegen geſcheite Menſch\pwindex{Kanner, Heinrich 9.\,11.\,1864 Galați – 15.\,2.\,1930 Wien@\textsc{Kanner, Heinrich} (9.\,11.\,1864 Galați – 15.\,2.\,1930 Wien), \emph{Herausgeber, Publizist}|pwv} begeht die Dummheit, mir die Jahre hindurch
               nachzutragen, daß ich mich einmal in einem Geſpräch \strikeout{über} ihm gegenüber ironiſch-neckend über einige{ }ſeiner Artikel ausgedrückt,
               die ich{ }ſtets ehrlich bewundert habe. Und nun: Iſt es Haß? Iſt es Neid? Iſt es
               Verachtung? – bei dieſer Neugründung ignorirt er mich vollſtändig. Es hätte {\pb}ſich unbedingt gehört, daß man mich aufforderte, von
                  \textsc{Paris\oindex{Paris@\textbf{Paris}, \emph{Hauptstadt}|pw}} aus für das Blatt\pwindex{Zeit. Wiener Wochenschrift@\emph{Die Zeit. Wiener Wochenschrift}|pwv} thätig
               zu{ }ſein. Ich hätte es kaum je annehmen können, aber eine Einladung hätte erfolgen
               müſſen. Statt deſſen iſt \textsc{Bahr\pwindex{Bahr, Hermann 19.\,7.\,1863 Linz – 15.\,1.\,1934 München@\textsc{Bahr, Hermann} (19.\,7.\,1863 Linz – 15.\,1.\,1934 München), \emph{Schriftsteller, Kritiker}|pw}}{ }ſeit geſtern in \textsc{Paris\oindex{Paris@\textbf{Paris}, \emph{Hauptstadt}|pw}}, um \textsc{Albert\pwindex{Albert, Henri 16.\,11.\,1869 Niederbronn-les-Bains – 3.\,8.\,1921 Straßburg@\textsc{Albert, Henri} (16.\,11.\,1869 Niederbronn-les-Bains – 3.\,8.\,1921 Straßburg), \emph{Journalist, Kritiker, Übersetzer}|pw}} die Pariſ\oindex{Paris@\textbf{Paris}, \emph{Hauptstadt}|pw}er Vertretung zu übertragen. Ich
               habe{ }ſelbſtverſtändlich \textsc{Albert\pwindex{Albert, Henri 16.\,11.\,1869 Niederbronn-les-Bains – 3.\,8.\,1921 Straßburg@\textsc{Albert, Henri} (16.\,11.\,1869 Niederbronn-les-Bains – 3.\,8.\,1921 Straßburg), \emph{Journalist, Kritiker, Übersetzer}|pw}} zur Annahme gedrängt, da das in{ }ſeinem Intereſſe iſt\textcolor{gray}{.} Aber
               die Kränkung iſt nichtsdeſtoweniger{ }ſehr bitter. Da{ }ſiehſt Du einmal in einem
               praktiſchen Falle, wie falſch Deine freundſchaftlichen Anſichten über meine Geltung{ }ſind.\pend
           
\pstart
           {\pb}Ich habe gethan, was ich thun konnte, um eine
               Beſprechung des »\textsc{Anatol\pwindex{Schnitzler, Arthur 15.\,5.\,1862 Wien – 21.\,10.\,1931 ebd.@\textsc{Schnitzler, Arthur} (15.\,5.\,1862 Wien – 21.\,10.\,1931 ebd.), \emph{Schriftsteller, Mediziner}!Anatol@\strich\emph{Anatol}|pw}}« in der Frkf. Ztg.\pwindex{Frankfurter Zeitung@\emph{Frankfurter Zeitung}|pw} durchzuſetzen.
                  V\textcolor{gray}{o}rgebens der wahre Grund{ }ſind gewiſſe \strikeout{inne} innere Vorgänge zwiſchen meinem Onkel\pwindex{Mamroth, Fedor 21.\,2.\,1851 Breslau – 25.\,6.\,1907 Frankfurt am Main@\textsc{Mamroth, Fedor} (21.\,2.\,1851 Breslau – 25.\,6.\,1907 Frankfurt am Main), \emph{Journalist, Kritiker}|pwv} und mir, die ich Dir einmal mündlich
               erklären werde. Hingegen habe ich eine Beſprechung\pwindex{Meyerhof, Leonie 2.\,3.\,1858 Hildesheim – 15.\,8.\,1933 Frankfurt am Main@\textsc{Meyerhof, Leonie} (2.\,3.\,1858 Hildesheim – 15.\,8.\,1933 Frankfurt am Main), \emph{Schriftstellerin, Journalistin}!Neue Romane und Novellen@\strich\emph{Neue Romane und Novellen}|pwv} für \textsc{Richard\pwindex{Beer-Hofmann, Richard 11.\,7.\,1866 Wien – 26.\,9.\,1945 New York City@\textsc{Beer-Hofmann, Richard} (11.\,7.\,1866 Wien – 26.\,9.\,1945 New York City), \emph{Schriftsteller}|pw}} erwirkt. Nun haben aber die Referenten das Recht ungehindert{ }ſeiner
               Meisungs-Äußerung bei uns, und das dumme Frauenzimmer\pwindex{Meyerhof, Leonie 2.\,3.\,1858 Hildesheim – 15.\,8.\,1933 Frankfurt am Main@\textsc{Meyerhof, Leonie} (2.\,3.\,1858 Hildesheim – 15.\,8.\,1933 Frankfurt am Main), \emph{Schriftstellerin, Journalistin}|pwv}, das bei uns die deutſche Literatur voranleitet,
               hat \textsc{Richards\pwindex{Beer-Hofmann, Richard 11.\,7.\,1866 Wien – 26.\,9.\,1945 New York City@\textsc{Beer-Hofmann, Richard} (11.\,7.\,1866 Wien – 26.\,9.\,1945 New York City), \emph{Schriftsteller}|pw}}{ }\strikeout{\textsc{B}}{ }Buch\pwindex{Beer-Hofmann, Richard 11.\,7.\,1866 Wien – 26.\,9.\,1945 New York City@\textsc{Beer-Hofmann, Richard} (11.\,7.\,1866 Wien – 26.\,9.\,1945 New York City), \emph{Schriftsteller}!Novellen@\strich\emph{Novellen}|pwv} abſolut nicht \label{K_L02621-8v}\edtext{ver{\pb}ſtanden\pwindex{Meyerhof, Leonie 2.\,3.\,1858 Hildesheim – 15.\,8.\,1933 Frankfurt am Main@\textsc{Meyerhof, Leonie} (2.\,3.\,1858 Hildesheim – 15.\,8.\,1933 Frankfurt am Main), \emph{Schriftstellerin, Journalistin}!Neue Romane und Novellen@\strich\emph{Neue Romane und Novellen}|pwv}}{\lemma{\textnormal{\emph{verstanden}}}\Cendnote{\textnormal{Leo Hildeck\pwindex{Meyerhof, Leonie 2.\,3.\,1858 Hildesheim – 15.\,8.\,1933 Frankfurt am Main@\textsc{Meyerhof, Leonie} (2.\,3.\,1858 Hildesheim – 15.\,8.\,1933 Frankfurt am Main), \emph{Schriftstellerin, Journalistin}|pwk} [ = Leonie Meyerhof\pwindex{Meyerhof, Leonie 2.\,3.\,1858 Hildesheim – 15.\,8.\,1933 Frankfurt am Main@\textsc{Meyerhof, Leonie} (2.\,3.\,1858 Hildesheim – 15.\,8.\,1933 Frankfurt am Main), \emph{Schriftstellerin, Journalistin}|pwk}]: \emph{Neue
                        Romane und Novellen}\pwindex{Meyerhof, Leonie 2.\,3.\,1858 Hildesheim – 15.\,8.\,1933 Frankfurt am Main@\textsc{Meyerhof, Leonie} (2.\,3.\,1858 Hildesheim – 15.\,8.\,1933 Frankfurt am Main), \emph{Schriftstellerin, Journalistin}!Neue Romane und Novellen@\strich\emph{Neue Romane und Novellen}|pwk}. In: \emph{Frankfurter
                        Zeitung}\pwindex{Frankfurter Zeitung@\emph{Frankfurter Zeitung}|pwk}, Jg. 38, Nr. 142, 24. 5. 1894, Erstes Morgenblatt,
                     S. 1–2.}}}\label{K_L02621-8}. Dafür kann ich nichts, und ich kann es nur bedauern. Ich
               habe das Ehrenwort meines Onkels\pwindex{Mamroth, Fedor 21.\,2.\,1851 Breslau – 25.\,6.\,1907 Frankfurt am Main@\textsc{Mamroth, Fedor} (21.\,2.\,1851 Breslau – 25.\,6.\,1907 Frankfurt am Main), \emph{Journalist, Kritiker}|pwv}, daß Dein neuer \label{K_L02621-9v}\edtext{Roman}{\lemma{\textnormal{\emph{Roman}}}\Cendnote{\textnormal{Nicht identifiziert.
                  Möglicherweise ging es um Schnitzlers Erzählung \emph{Blumen}\pwindex{Schnitzler, Arthur 15.\,5.\,1862 Wien – 21.\,10.\,1931 ebd.@\textsc{Schnitzler, Arthur} (15.\,5.\,1862 Wien – 21.\,10.\,1931 ebd.), \emph{Schriftsteller, Mediziner}!Blumen@\strich\emph{Blumen}|pwk}, deren Abdruck in der \emph{Frankfurter
                     Zeitung}\pwindex{Frankfurter Zeitung@\emph{Frankfurter Zeitung}|pwk}{ }Mamroth\pwindex{Mamroth, Fedor 21.\,2.\,1851 Breslau – 25.\,6.\,1907 Frankfurt am Main@\textsc{Mamroth, Fedor} (21.\,2.\,1851 Breslau – 25.\,6.\,1907 Frankfurt am Main), \emph{Journalist, Kritiker}|pwk} jedenfalls am XXXX Auszeichnungsfehler: Dokument L00311 nicht gefunden freundlich abgelehnt hattee.}}}\label{K_L02621-9}
               beſprochen wird,{ }ſobald er in Buchform erſchienen iſt.\pend
           
\pstart
           Wenn ich keinen{ }ſchweren Krankheitsanfall bekomme, will ich von meinem
               vierwöchentlichen Urlaub drei auf eine Reiſe verwenden. Ich habe keinen höheren
               Wunſch, als dieſe drei Wochen mit Dir zu verbringen. Aber das muß im \label{K_L02621-10v}\edtext{Auguſt}{\lemma{\textnormal{\emph{August}}}\Cendnote{\textnormal{Vom 23. 8. 1894 bis 3. 9. 1894 verbrachten Schnitzler und Goldmann\pwindex{Goldmann, Paul 31.\,1.\,1865 Breslau – 25.\,9.\,1935 Wien@\textsc{Goldmann, Paul} (31.\,1.\,1865 Breslau – 25.\,9.\,1935 Wien), \emph{Schriftsteller, Journalist}|pwk} einige Zeit gemeinsam in Bad
                     Ischl\oindex{Bad Ischl@\textbf{Bad Ischl}|pwk} und Bad Aussee\oindex{Bad Aussee@\textbf{Bad Aussee}, \emph{Hauptstadt}|pwk}.}}}\label{K_L02621-10}{ }ſein.
               Kannſt du fort? Und wohin? Bitte,{ }ſchreib’ mir bald darüber.\pend
           
\pstart
           {\pb}Oh dieſe Hypochondrie in Deinem letzten Briefe!
               Gewiß, es iſt wünſchenswerth frei zu{ }ſein. Aber ich habe oft über die Freiheit
               nachgedacht, und ich fürchte beinahe, daß{ }ſie doch nicht das Gut iſt, \substVorne{}\textsuperscript{daß}\substDazwischen{}das\substHinten{} wir glauben. Man würde glücklich auf allen Seiten Wege vor{ }ſich{ }ſehen. Und
               ich wenigſtens gehöre nicht zu den Leuten, die raſch entſchloſſen einen von den
               hundert Wegen einſchlagen,{ }ſondern zu denen, die all’ ihr Leben lang damit vertändeln
               würden, davor zu{ }ſtehen {\pb}und zu überlegen:{ }ſoll ich
               dahin gehen oder dorthin? Und würde ich einen Weg wählen, welchen immer,{ }ſo würde
               mich bis an meinen Tod die Reue verfolgen, daß ich nicht den andern eingeſchlagen.
               Biſt Du nicht auch ein wenig{ }ſo? Gewiß, der Zwang iſt drückend. Aber es hat auch{ }ſein
               gutes: es erſpart einem die Mühe der Wahl und die Verantwortung dafür. Der Zwang,
                  \label{K_L02621-11v}\edtext{\textsc{\begin{otherlanguage}{french}c’est une destinée toute faite\end{otherlanguage}}}{\lemma{\textnormal{\emph{c’est … faite}}}\Cendnote{\textnormal{französisch, etwa: das Schicksal ist
                  vorbestimmt}}}\label{K_L02621-11}. Und wenn er, wie bei Dir, nicht mit Infamie verbunden iſt (wie
               bei mir),{ }ſo{ }ſollte man ihn {\pb}ruhig tragen, zumal
               wenn man dabei auch noch graduieren kann. Wer weiß, ob nicht gerade in Deiner Abſcheu
               davor, ein ärztlicher \strikeout{ban} Banauſe zu werden, ein
               gutes Theil Deiner Productionskraft liegt. Und wer weiß, ob dieſe, die vielleicht zum
               großen Theil eine Reaktionserſcheinung iſt, nicht{ }ſehr abnehmen würde, wenn auf der
               andern Seite die Aktion des Zwanges aufhörte. Dabei fällt mir ein, daß es im Obigen
               nicht Productions-Kraft heißen darf,{ }ſondern »Wille zur Produktion«. Auch{ }ſonſt habe
               ich es mir ganz {\pb}anders gedacht, als es da
               ausgedrückt iſt. Das macht aber nichts.\pend
           
\pstart
           Die von Dir erwähnte \label{K_L02621-12v}\edtext{Erwiderung\pwindex{Christensen, Hjalmar 5.\,5.\,1869 – 29.\,12.\,1925@\textsc{Christensen, Hjalmar} (5.\,5.\,1869 – 29.\,12.\,1925), \emph{Kritiker, Kunstschriftsteller}!Dekadent@\strich\emph{Der Dekadent}|pwv} von \textsc{Christensen\pwindex{Christensen, Hjalmar 5.\,5.\,1869 – 29.\,12.\,1925@\textsc{Christensen, Hjalmar} (5.\,5.\,1869 – 29.\,12.\,1925), \emph{Kritiker, Kunstschriftsteller}|pw}\pwindex{Christensen, Hjalmar 5.\,5.\,1869 – 29.\,12.\,1925@\textsc{Christensen, Hjalmar} (5.\,5.\,1869 – 29.\,12.\,1925), \emph{Kritiker, Kunstschriftsteller}!Dekadent@\strich\emph{Der Dekadent}|pwv}}}{\lemma{\textnormal{\emph{Erwiderung von Christensen}}}\Cendnote{\textnormal{Hjalmar Christensen\pwindex{Christensen, Hjalmar 5.\,5.\,1869 – 29.\,12.\,1925@\textsc{Christensen, Hjalmar} (5.\,5.\,1869 – 29.\,12.\,1925), \emph{Kritiker, Kunstschriftsteller}|pwk}: \emph{Der Dekadent}\pwindex{Christensen, Hjalmar 5.\,5.\,1869 – 29.\,12.\,1925@\textsc{Christensen, Hjalmar} (5.\,5.\,1869 – 29.\,12.\,1925), \emph{Kritiker, Kunstschriftsteller}!Dekadent@\strich\emph{Der Dekadent}|pwk}. In: \emph{Frankfurter Zeitung}\pwindex{Frankfurter Zeitung@\emph{Frankfurter Zeitung}|pwk}, Jg. 38, Nr. 103, 14. 4. 1894, Erstes
                     Morgenblatt, S. 1–2. Eine unmittelbare Reaktion auf diesen Text lässt
                  sich nicht nachweisen, sehr wohl aber eine wohlwollende Erwähnung\pwindex{Décadence@\emph{Décadence}|pwkv} in der \emph{Neuen deutschen Rundschau}\pwindex{Neue Deutsche Rundschau@\emph{Neue Deutsche Rundschau}|pwk} vom Mai 1894
                     (Jg. 5, Nr. 5, S. 522–523). In der \emph{Neuen deutschen Rundschau}\pwindex{Neue Deutsche Rundschau@\emph{Neue Deutsche Rundschau}|pwk} findet sich auch ein Hinweis auf eine kritische
                  Einordnung von jüngeren Wien\oindex{Wien@\textbf{Wien}, \emph{Verwaltungsgebiet}|pwk}er Autoren –
                  darunter Schnitzler, Hofmannstal\pwindex{Hofmannsthal, Hugo von 1.\,2.\,1874 Wien – 15.\,7.\,1929 Rodaun@\textsc{Hofmannsthal, Hugo von} (1.\,2.\,1874 Wien – 15.\,7.\,1929 Rodaun), \emph{Schriftsteller}|pwk} und Bahr\pwindex{Bahr, Hermann 19.\,7.\,1863 Linz – 15.\,1.\,1934 München@\textsc{Bahr, Hermann} (19.\,7.\,1863 Linz – 15.\,1.\,1934 München), \emph{Schriftsteller, Kritiker}|pwk} – durch Stauf von der March\pwindex{Stauf von der March, Ottokar 29.\,8.\,1868 Olomouc – 12.\,3.\,1941 Wien@\textsc{Stauf von der March, Ottokar} (29.\,8.\,1868 Olomouc – 12.\,3.\,1941 Wien), \emph{Schriftsteller, Journalist}|pwk} (Ottokar Stauf von der March\pwindex{Stauf von der March, Ottokar 29.\,8.\,1868 Olomouc – 12.\,3.\,1941 Wien@\textsc{Stauf von der March, Ottokar} (29.\,8.\,1868 Olomouc – 12.\,3.\,1941 Wien), \emph{Schriftsteller, Journalist}|pwk}: \emph{Décadence. Randglossen}\pwindex{Stauf von der March, Ottokar 29.\,8.\,1868 Olomouc – 12.\,3.\,1941 Wien@\textsc{Stauf von der March, Ottokar} (29.\,8.\,1868 Olomouc – 12.\,3.\,1941 Wien), \emph{Schriftsteller, Journalist}!Décadence. Randglossen@\strich\emph{Décadence. Randglossen}|pwk}. In: \emph{Die Gesellschaft}\pwindex{Gesellschaft. Monatsschrift für Litteratur, Kunst und Sozialpolitik@\emph{Die Gesellschaft. Monatsschrift für Litteratur, Kunst und Sozialpolitik}|pwk}, Jg. 10, H. 4, April 1894, S.                      526–533). Über Schnitzler steht
                  darin: »Der hervorragendste aller Dekadenten ist der schon öfter erwähnte
                        Wien\oindex{Wien@\textbf{Wien}, \emph{Verwaltungsgebiet}|pw}er Arthur Schnitzler. Obgleich seine Dichtungen, vornehmlich:
                     Scenenbilder (›Anatol\pwindex{Schnitzler, Arthur 15.\,5.\,1862 Wien – 21.\,10.\,1931 ebd.@\textsc{Schnitzler, Arthur} (15.\,5.\,1862 Wien – 21.\,10.\,1931 ebd.), \emph{Schriftsteller, Mediziner}!Anatol@\strich\emph{Anatol}|pw}‹), vom denkbar
                     stärksten Décadence-Kolorit durchsättigt sind und darum den Leser in die
                     unbehaglichste Stimmung von der Welt versetzen, erscheinen sie doch durch ihre
                     Aufrichtigkeit und Selbsterkenntnis geadelt. Mit peinlicher Akkuratesse seziert
                     der Dichter seine
                     Probleme und erklärt dem staunenden Leser resigniert-lächelnd die angefaulten
                     Körperstellen. An Geist vermag sich mit ihm kein einziger Dekadent zu messen.
                        Schnitzlers Werke sprühen förmlich von
                     genialen Gedanken und Sentenzen. Er ist gewissermaßen der Klassiker der
                     Décadence, aber darum nicht minder krank, als die übrigen.«
                     (S. 531.)}}}\label{K_L02621-12} habe ich nirgends entdecken können. Könnteſt Du
               mir nicht die Nummer oder nur die ungefähre Erſcheinungs-Zeit angeben?\pend
           
\pstart
           Und \textsc{Richard\pwindex{Beer-Hofmann, Richard 11.\,7.\,1866 Wien – 26.\,9.\,1945 New York City@\textsc{Beer-Hofmann, Richard} (11.\,7.\,1866 Wien – 26.\,9.\,1945 New York City), \emph{Schriftsteller}|pw}}? Und \textsc{Loris\pwindex{Hofmannsthal, Hugo von 1.\,2.\,1874 Wien – 15.\,7.\,1929 Rodaun@\textsc{Hofmannsthal, Hugo von} (1.\,2.\,1874 Wien – 15.\,7.\,1929 Rodaun), \emph{Schriftsteller}|pw}}?\pend
           
\pstart
           Bitte, lies: \textsc{Bernard Lazare\pwindex{Lazare, Bernard 15.\,6.\,1865 Nîmes – 1.\,9.\,1903 Paris@\textsc{Lazare, Bernard} (15.\,6.\,1865 Nîmes – 1.\,9.\,1903 Paris), \emph{Journalist, Anarchist}|pw}}: \textsc{L’Antisémitisme\pwindex{Lazare, Bernard 15.\,6.\,1865 Nîmes – 1.\,9.\,1903 Paris@\textsc{Lazare, Bernard} (15.\,6.\,1865 Nîmes – 1.\,9.\,1903 Paris), \emph{Journalist, Anarchist}!antisémitisme. Son histoire et ses causes@\strich\emph{L’antisémitisme. Son histoire et ses causes}|pw}}. Soeben erſchienen bei \textsc{Léon Challey\pwindex{Chailley, Léon @\textsc{Chailley, Léon}, \emph{Verleger}|pw}}, \textsc{8. Rue Saint-Joseph\oindex{Rue Saint-Joseph@\textbf{Rue Saint-Joseph}, \emph{Straße}|pw}}. Der Verfaſſer\pwindex{Lazare, Bernard 15.\,6.\,1865 Nîmes – 1.\,9.\,1903 Paris@\textsc{Lazare, Bernard} (15.\,6.\,1865 Nîmes – 1.\,9.\,1903 Paris), \emph{Journalist, Anarchist}|pwv}, in
               unſerem Alter, iſt{ }ſelbst Jude.\pend
           
\pstart
           Mein Schwager\pwindex{Rosengart, Josef 8.\,2.\,1860 Laupheim – 4.\,8.\,1927 Frankfurt am Main@\textsc{Rosengart, Josef} (8.\,2.\,1860 Laupheim – 4.\,8.\,1927 Frankfurt am Main), \emph{Arzt}|pwv} iſt
               hochbeglückt mit Deiner \label{K_L02621-13v}\edtext{Zeitſchrift\pwindex{Internationale klinische Rundschau@\emph{Internationale klinische Rundschau}|pwv}}{\lemma{\textnormal{\emph{Zeitschrift}}}\Cendnote{\textnormal{Siehe XXXX Auszeichnungsfehler: Dokument L02719 nicht gefunden.
               }}}\label{K_L02621-13} und dankt Dir noch vielmals.\pend
           
\pstart
           Viele treue Grüße! {\\[\baselineskip]}Dein {\\[\baselineskip]}\spacefill\mbox{Paul Goldmann}\pend
           \leftskip=0em{}
\pstart
           \noindent{}Schreib’ bald!!\pend
           \selectlanguage{ngerman}\endnumbering\briefempfaengerindex{Schnitzler, Arthur@\textsc{Schnitzler, Arthur}!zzzGoldmann, Paul@\emph{von Paul Goldmann}!1894-05-292@{29. 5. [1894]}|)be}\mylabel{L02621h}  \newcommand{\dateiname}{L02621}\newcommand{\titel}{Paul Goldmann an Arthur Schnitzler, 29. 5. [1894]}\newcommand{\editorInnen}{Martin Anton Müller und Laura Untner}%% latex-leseansicht-abspann.tex
%% Abspann für die Leseansicht.
%% Der Schalter \ifkorrekturansicht ist bereits durch den Vorspann gesetzt.

%% latex-abspann.tex
%% Gemeinsamer Abspann für Korrekturansicht und Leseansicht.
%% Setzt den Schalter \ifkorrekturansicht voraus (gesetzt in den
%% einbindenden Dateien latex-korrekturansicht-abspann.tex bzw.
%% latex-leseansicht-abspann.tex).
%% ---------------------------------------------------------------

\normalsize

% Das esempio-Environment wird nur in der Leseansicht benötigt
\ifkorrekturansicht\else
\newenvironment{esempio}[3]%
{
    \vspace{1.5ex}
    \rlap{\underline{#1}}
    \par
    \setlength{\parindent}{0cm}
    \nopagebreak
    \leftskip=#2cm
    \rightskip=#3cm
}
{
    \par
}
\fi

\doendnotes{C}
\bigskip
\vfill

\clearpage

\footnotesize

\ifkorrekturansicht
  \lohead{\textsc{register}}
\fi

% theindex-Environment neu definieren ohne reledmac
\makeatletter
\renewenvironment{theindex}{%
  \ifkorrekturansicht
    \section*{\indexname}%
  \else
    \subsubsection*{Index der erwähnten Entitäten}%
  \fi
  \setlength{\parindent}{0pt}%
  \setlength{\parskip}{0pt plus 0.3pt}%
  \let\item\@idxitem
}{%
  \ifkorrekturansicht\clearpage\fi
}
\makeatother

\IfFileExists{\jobname-pw.ind}{\input{\jobname-pw.ind}}{}

% Quellenangabe nur in der Leseansicht
\ifkorrekturansicht\else
% Fallback-Definitionen, falls die .tex-Datei \titel etc. nicht gesetzt hat
\providecommand{\titel}{}
\providecommand{\editorInnen}{}
\providecommand{\dateiname}{\jobname}

\vspace{3cm}

\vfill

\footnotesize
\textsc{Quelle}: \titel. Herausgegeben von {\editorInnen}. In: \emph{Arthur Schnitzler: Briefwechsel mit Autorinnen und Autoren}.
 Digitale Edition, https://schnitzler-briefe.acdh.oeaw.ac.at/{\dateiname}.html (Stand \today)
\fi

\end{document}


