%% latex-leseansicht-vorspann.tex
%% Vorspann für die Leseansicht.
%% Lädt die gemeinsame Datei latex-vorspann.tex mit nicht gesetztem Schalter.

\newif\ifkorrekturansicht
\korrekturansichtfalse

\input{../tex-inputs/latex-vorspann}


\section[Arthur Schnitzler an Theodor Herzl, 9. 3. 1895]{L03925 Arthur Schnitzler an Theodor Herzl, 9. 3. 1895}
\nopagebreak\mylabel{L03925v}
\rehead{ }\normalsize\beginnumbering\briefempfaengerindex{Herzl, Theodor@\textsc{Herzl, Theodor}!zzzSchnitzler, Arthur@\emph{von Arthur Schnitzler}!1895-03-092@{9. 3. 1895}|(be}
\toendnotes[C]{\smallbreak\pagebreak[2]}
\correspDesc{Versand  durch Arthur Schnitzler am 9. 3. 1895 in Wien
\newline{}Erhalt  durch Theodor Herzl in Wien}\toendnotes[C]{\smallbreak}
\Standort{Jerusalem, Central Zionist Archives, H1:1925-10.}
\physDesc{,  Blätter,  Seiten
\newline{}Handschrift: , deutsche Kurrent}
\buchAbdrucke{\weitereDrucke{Arthur Schnitzler: \emph{Briefe 1875–1912}. Herausgegeben von Therese Nickl und Heinrich Schnitzler. Frankfurt am Main: \emph{S. Fischer} 1981, S. 253–254.} }\toendnotes[C]{\smallbreak}
\pstart
           {\pb}\textcolor{gray}{\textbf{AS}}\pend
           
\pstart{}Mein lieber Freund.\pend\vspace{0.5em}
\pstart
           der Tritt iſt gegeben. Sagen Sie mir für alle Fälle, was ich nach einem »Nein« zu
               thun habe, um jeden überflüſſigen Aufſchub zu vermeiden.\pend
           
\pstart
           In ein paar Tagen ſchreib ich Ihnen dann ausführlich. Ich halt’ es für ſehr möglich,
               daſs Sie und ich am ſelben Abend in der Burg\oindex{Wien@\textbf{Wien}!I., Innere Stadt@\textbf{I., Innere Stadt}!Burgtheater@\textbf{Burgtheater}, \emph{Theater}|pw}{ }{\pb}drankommen\eventindex{Burgtheater@\textbf{Burgtheater}!Uraufführung von Liebelei, Premiere von Rechte der Seele, 9.10.1895@Uraufführung von Liebelei, Premiere von Rechte der Seele, 9.10.1895|pwv}\eventindex{Burgtheater@\textbf{Burgtheater}!Premiere von Tabarin und Verbotene Früchte, 2.5.1895@Premiere von Tabarin und Verbotene Früchte, 2.5.1895|pwv}. Denn es ſcheint, die nächſte
               Novität ſind die Verbotenen Früchte\pwindex{\textcolor{red}{\textsuperscript{XXXX indx1}}!Verbotene Früchte. Lustspiel nach einem Zwischenspiel von Cervantes@\strich\emph{Verbotene Früchte. Lustspiel nach einem Zwischenspiel von Cervantes}|pw} von \textsc{Cervantes\pwindex{Cervantes Saavedra, Miguel de 29.\,9.\,1547 Alcalá de Henares – 22.\,4.\,1616 Madrid@\textsc{Cervantes Saavedra, Miguel de} (29.\,9.\,1547 Alcalá de Henares – 22.\,4.\,1616 Madrid), \emph{Schriftsteller}|pw}} – und gerade wir zwei blieben noch übrig. \label{K_L03925-2v}\edtext{\textsc{Burckhard\pwindex{Burckhard, Max Eugen 14.\,7.\,1854 Korneuburg – 16.\,3.\,1912 Wien@\textsc{Burckhard, Max Eugen} (14.\,7.\,1854 Korneuburg – 16.\,3.\,1912 Wien), \emph{Schriftsteller, Rechtswissenschaftler, Theaterleiter}|pw}} zu ſprechen}{\lemma{\textnormal{\emph{Burckhard zu sprechen}}}\Cendnote{\textnormal{Schnitzler wohnte zu dieser Zeit im selben Haus wie Burckhard\pwindex{Burckhard, Max Eugen 14.\,7.\,1854 Korneuburg – 16.\,3.\,1912 Wien@\textsc{Burckhard, Max Eugen} (14.\,7.\,1854 Korneuburg – 16.\,3.\,1912 Wien), \emph{Schriftsteller, Rechtswissenschaftler, Theaterleiter}|pwk}. }}}\label{K_L03925-2}, vermeide ich ſeit Wochen
               – mir ko{\geminationm}t vor, jeder freundliche Gruſs müſſte ihm wie eine Frage
               vorko{\geminationm}en: »Na alſo was iſt mit mir?«\pend
           
\pstart
           Gearbeitet hab ich die letzten Wochen ja {\pb}Monaten rein
               nichts. Innere Verdrießlichkeiten – an ſich nicht mehr bedeutend als Zahnweh – aber
               können Sie mit Zahnweh arbeiten? Aber ich hoffe auf den Frühling und mancherlei
               andres.\pend
           \pstart Leben Sie wohl, ſeien Sie vielmals herzlich gegrüßt und ſchreiben Sie bald Ihrem
                  \spacefill\mbox{ArthSchn}\pend{}
\pstart
           9. 3. 95.\pend
           \selectlanguage{ngerman}\endnumbering\briefempfaengerindex{Herzl, Theodor@\textsc{Herzl, Theodor}!zzzSchnitzler, Arthur@\emph{von Arthur Schnitzler}!1895-03-092@{9. 3. 1895}|)be}\mylabel{L03925h}
\begin{anhang}
\end{anhang}\newcommand{\dateiname}{L03925}\newcommand{\titel}{Arthur Schnitzler an Theodor Herzl, 9. 3. 1895}\newcommand{\editorInnen}{Herausgegeben von Jahnke, SelmaMüller, Martin Anton}%% latex-leseansicht-abspann.tex
%% Abspann für die Leseansicht.
%% Der Schalter \ifkorrekturansicht ist bereits durch den Vorspann gesetzt.

%% latex-abspann.tex
%% Gemeinsamer Abspann für Korrekturansicht und Leseansicht.
%% Setzt den Schalter \ifkorrekturansicht voraus (gesetzt in den
%% einbindenden Dateien latex-korrekturansicht-abspann.tex bzw.
%% latex-leseansicht-abspann.tex).
%% ---------------------------------------------------------------

\normalsize

% Das esempio-Environment wird nur in der Leseansicht benötigt
\ifkorrekturansicht\else
\newenvironment{esempio}[3]%
{
    \vspace{1.5ex}
    \rlap{\underline{#1}}
    \par
    \setlength{\parindent}{0cm}
    \nopagebreak
    \leftskip=#2cm
    \rightskip=#3cm
}
{
    \par
}
\fi

\doendnotes{C}
\bigskip
\vfill

\clearpage

\footnotesize

\ifkorrekturansicht
  \lohead{\textsc{register}}
\fi

% theindex-Environment neu definieren ohne reledmac
\makeatletter
\renewenvironment{theindex}{%
  \ifkorrekturansicht
    \section*{\indexname}%
  \else
    \subsubsection*{Index der erwähnten Entitäten}%
  \fi
  \setlength{\parindent}{0pt}%
  \setlength{\parskip}{0pt plus 0.3pt}%
  \let\item\@idxitem
}{%
  \ifkorrekturansicht\clearpage\fi
}
\makeatother

\IfFileExists{\jobname-pw.ind}{\input{\jobname-pw.ind}}{}

% Quellenangabe nur in der Leseansicht
\ifkorrekturansicht\else
% Fallback-Definitionen, falls die .tex-Datei \titel etc. nicht gesetzt hat
\providecommand{\titel}{}
\providecommand{\editorInnen}{}
\providecommand{\dateiname}{\jobname}

\vspace{3cm}

\vfill

\footnotesize
\textsc{Quelle}: \titel. Herausgegeben von {\editorInnen}. In: \emph{Arthur Schnitzler: Briefwechsel mit Autorinnen und Autoren}.
 Digitale Edition, https://schnitzler-briefe.acdh.oeaw.ac.at/{\dateiname}.html (Stand \today)
\fi

\end{document}


