\input{../tex-inputs/latex-pdf-vorspann}
\begin{center}
            \textcolor{red}{ENTWURF. ENTZIFFERUNG NOCH NICHT KORREKTURGELESEN}
                      \end{center}
            
               \section[Arthur Schnitzler an Gerhart Hauptmann, 30. 11. 1902]{ Arthur Schnitzler an Gerhart Hauptmann, 30. 11. 1902}\nopagebreak\mylabel{v}\rehead{ }\begin{ledgroupsized}[t]{13cm}\normalsize\beginnumbering\briefempfaengerindex{Hauptmann, Gerhart@\textsc{Hauptmann, Gerhart}!zzzSchnitzler, Arthur@\emph{von Arthur Schnitzler}!1902-11-301@{30. 11. 1903}|(be} \toendnotes[C]{\smallbreak\pagebreak[2]} \Standort{Staatsbibliothek Berlin – Preußischer Kulturbesitz, GHBrBl A:Schnitzler (13).}
\physDesc{Brief, 1 Blatt, 2 Seiten
\newline{}Handschrift: schwarze Tinte, deutsche Kurrent\newline{}Ordnung: Lochung }\toendnotes[C]{\smallbreak}\pstart
           \noindent{}{\pb}von ganzem Herzen, lieber Herr
                        Hauptmann beglückwünſche ich Sie zu Ihrem wahrhaft großen \label{K_L01254_1v}\edtext{Erfolg}{\lemma{\textnormal{\emph{Erfolg}}}\Cendnote{\textnormal{Uraufführung von \emph{Der Arme Heinrich}\pwindex{Hauptmann, Gerhart 15.11.1862 – 06.06.1946@\textsc{Hauptmann, Gerhart} (15.11.1862 – 06.06.1946), \emph{Schriftsteller}!arme Heinrich – Eine deutsche Sage29.11.1902 – 29.11.1902@\strich\emph{Der arme Heinrich – Eine deutsche Sage} {[}29.11.1902 – 29.11.1902{]}|pwk} im Burgtheater\oindex{Burgtheater@\textbf{Burgtheater}|pwk} am 29. 11. 1902.
                            Schnitzler\pwindex{Schnitzler, Arthur 15.05.1862 – 21.10.1931@\textsc{Schnitzler, Arthur} (15.05.1862 – 21.10.1931), \emph{Schriftsteller, Mediziner}|pwk} war anwesend.}}}\label{K_L01254_1h}; – i\substVorne{}\textsuperscript{m}\substDazwischen{}ns\substHinten{} innerſte bewegt von der ſchönen Dichtung\pwindex{Hauptmann, Gerhart 15.11.1862 – 06.06.1946@\textsc{Hauptmann, Gerhart} (15.11.1862 – 06.06.1946), \emph{Schriftsteller}!arme Heinrich – Eine deutsche Sage29.11.1902 – 29.11.1902@\strich\emph{Der arme Heinrich – Eine deutsche Sage} {[}29.11.1902 – 29.11.1902{]}|pwv}, der nicht an Fülle reifſten vielleicht, der
                    aber, die schwellend von verhaltener Kraft, leuchtend in Reinheit, und in
                    reinlichſter {\pb}Einfachheit dahinfließend,
                    Ihren schönſten Werken ſich anschließt und in noch lichtere Höhen deutet.\pend
           \pstart
           In Bewunderung und Freundſchaft drück ich Ihnen die Hand{\\[\baselineskip]}Ihr{\\[\baselineskip]}\spacefill\mbox{Arthur Schnitzler}\pend
           \leftskip=0em{}\pstart
           30. 11. 902\pend
           \endnumbering\briefempfaengerindex{Hauptmann, Gerhart@\textsc{Hauptmann, Gerhart}!zzzSchnitzler, Arthur@\emph{von Arthur Schnitzler}!1902-11-301@{30. 11. 1903}|)be}\mylabel{h}\end{ledgroupsized}  \newcommand{\dateiname}{L01254}\newcommand{\titel}{Arthur Schnitzler an Gerhart Hauptmann, 30. 11. 1902}\newcommand{\editorInnen}{ Martin Anton Müller und Gerd-Hermann Susen}\input{../tex-inputs/latex-pdf-abspann}
      