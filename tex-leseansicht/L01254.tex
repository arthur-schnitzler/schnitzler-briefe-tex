%% latex-leseansicht-vorspann.tex
%% Vorspann für die Leseansicht.
%% Lädt die gemeinsame Datei latex-vorspann.tex mit nicht gesetztem Schalter.

\newif\ifkorrekturansicht
\korrekturansichtfalse

\input{../tex-inputs/latex-vorspann}


\section[Arthur Schnitzler an Gerhart Hauptmann, 30. 11. 1902]{L01254 Arthur Schnitzler an Gerhart Hauptmann, 30. 11. 1902}
\nopagebreak\mylabel{L01254v}
\rehead{ }\normalsize\beginnumbering\briefempfaengerindex{Hauptmann, Gerhart@\textsc{Hauptmann, Gerhart}!zzzSchnitzler, Arthur@\emph{von Arthur Schnitzler}!1902-11-301@{30. 11. 1903}|(be}
\toendnotes[C]{\smallbreak\pagebreak[2]}
\correspDesc{Versand  durch Arthur Schnitzler am 30. 11. 1903 in Wien
\newline{}Erhalt  durch Gerhart Hauptmann im Zeitraum [30. 11. 1902 – 4. 12. 1902?] \textbf{Ort fehlend} }\toendnotes[C]{\smallbreak}
\Standort{Staatsbibliothek Berlin – Preußischer Kulturbesitz, GHBrBl A:Schnitzler (13).}
\physDesc{Brief, 1 Blatt, 2 Seiten, 458 Zeichen
\newline{}Handschrift: schwarze Tinte, deutsche Kurrent
\newline{}Ordnung: Lochung }\toendnotes[C]{\smallbreak}
\pstart
           \noindent{}{\pb}von ganzem Herzen, lieber Herr Hauptmann beglückwünſche ich Sie zu Ihrem wahrhaft großen \label{K_L01254-1v}\edtext{Erfolg\eventindex{Burgtheater@\textbf{Burgtheater}!Uraufführung Der arme Heinrich, 29.11.1902@Uraufführung Der arme Heinrich, 29.11.1902|pwv}}{\lemma{\textnormal{\emph{Erfolg}}}\Cendnote{\textnormal{Uraufführung\eventindex{Burgtheater@\textbf{Burgtheater}!Uraufführung Der arme Heinrich, 29.11.1902@Uraufführung Der arme Heinrich, 29.11.1902|pwkv} von \emph{Der Arme
                     Heinrich}\pwindex{Hauptmann, Gerhart 15.\,11.\,1862 Szczawno-Zdrój – 6.\,6.\,1946 Jagniątków@\textsc{Hauptmann, Gerhart} (15.\,11.\,1862 Szczawno-Zdrój – 6.\,6.\,1946 Jagniątków), \emph{Schriftsteller}!arme Heinrich – Eine deutsche Sage@\strich\emph{Der arme Heinrich – Eine deutsche Sage}|pwk} im Burgtheater\oindex{Wien@\textbf{Wien}!I., Innere Stadt@\textbf{I., Innere Stadt}!Burgtheater@\textbf{Burgtheater}, \emph{Theater}|pwk} am
                     29. 11. 1902. Schnitzler war
                  anwesend.}}}\label{K_L01254-1}; – i\substVorne{}\textsuperscript{m}\substDazwischen{}ns\substHinten{} innerſte bewegt von der{ }ſchönen Dichtung\pwindex{Hauptmann, Gerhart 15.\,11.\,1862 Szczawno-Zdrój – 6.\,6.\,1946 Jagniątków@\textsc{Hauptmann, Gerhart} (15.\,11.\,1862 Szczawno-Zdrój – 6.\,6.\,1946 Jagniątków), \emph{Schriftsteller}!arme Heinrich – Eine deutsche Sage@\strich\emph{Der arme Heinrich – Eine deutsche Sage}|pwv}, der nicht an Fülle reifſten vielleicht, der aber,
               die schwellend von verhaltener Kraft, leuchtend in Reinheit, und in reinlichſter {\pb}Einfachheit dahinfließend, Ihren schönſten
               Werken{ }ſich anschließt und in noch lichtere Höhen deutet.\pend
           
\pstart
           In Bewunderung und Freundſchaft drück ich Ihnen die Hand{\\[\baselineskip]}Ihr{\\[\baselineskip]}\spacefill\mbox{Arthur Schnitzler}\pend
           \leftskip=0em{}
\pstart
           30. 11. 902\pend
           \selectlanguage{ngerman}\endnumbering\briefempfaengerindex{Hauptmann, Gerhart@\textsc{Hauptmann, Gerhart}!zzzSchnitzler, Arthur@\emph{von Arthur Schnitzler}!1902-11-301@{30. 11. 1903}|)be}\mylabel{L01254h}  \newcommand{\dateiname}{L01254}\newcommand{\titel}{Arthur Schnitzler an Gerhart Hauptmann, 30. 11. 1902}\newcommand{\editorInnen}{Martin Anton Müller und Gerd-Hermann Susen}%% latex-leseansicht-abspann.tex
%% Abspann für die Leseansicht.
%% Der Schalter \ifkorrekturansicht ist bereits durch den Vorspann gesetzt.

%% latex-abspann.tex
%% Gemeinsamer Abspann für Korrekturansicht und Leseansicht.
%% Setzt den Schalter \ifkorrekturansicht voraus (gesetzt in den
%% einbindenden Dateien latex-korrekturansicht-abspann.tex bzw.
%% latex-leseansicht-abspann.tex).
%% ---------------------------------------------------------------

\normalsize

% Das esempio-Environment wird nur in der Leseansicht benötigt
\ifkorrekturansicht\else
\newenvironment{esempio}[3]%
{
    \vspace{1.5ex}
    \rlap{\underline{#1}}
    \par
    \setlength{\parindent}{0cm}
    \nopagebreak
    \leftskip=#2cm
    \rightskip=#3cm
}
{
    \par
}
\fi

\doendnotes{C}
\bigskip
\vfill

\clearpage

\footnotesize

\ifkorrekturansicht
  \lohead{\textsc{register}}
\fi

% theindex-Environment neu definieren ohne reledmac
\makeatletter
\renewenvironment{theindex}{%
  \ifkorrekturansicht
    \section*{\indexname}%
  \else
    \subsubsection*{Index der erwähnten Entitäten}%
  \fi
  \setlength{\parindent}{0pt}%
  \setlength{\parskip}{0pt plus 0.3pt}%
  \let\item\@idxitem
}{%
  \ifkorrekturansicht\clearpage\fi
}
\makeatother

\IfFileExists{\jobname-pw.ind}{\input{\jobname-pw.ind}}{}

% Quellenangabe nur in der Leseansicht
\ifkorrekturansicht\else
% Fallback-Definitionen, falls die .tex-Datei \titel etc. nicht gesetzt hat
\providecommand{\titel}{}
\providecommand{\editorInnen}{}
\providecommand{\dateiname}{\jobname}

\vspace{3cm}

\vfill

\footnotesize
\textsc{Quelle}: \titel. Herausgegeben von {\editorInnen}. In: \emph{Arthur Schnitzler: Briefwechsel mit Autorinnen und Autoren}.
 Digitale Edition, https://schnitzler-briefe.acdh.oeaw.ac.at/{\dateiname}.html (Stand \today)
\fi

\end{document}


