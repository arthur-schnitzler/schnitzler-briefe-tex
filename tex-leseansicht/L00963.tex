%% latex-korrekturansicht-vorspann.tex
%% Vorspann für die Korrekturansicht.
%% Lädt die gemeinsame Datei latex-vorspann.tex mit gesetztem Schalter.

\newif\ifkorrekturansicht
\korrekturansichttrue

\input{../tex-inputs/latex-vorspann}


\section[Arthur Schnitzler an Richard Beer-Hofmann, 23. 8. 1899]{L00963 Arthur Schnitzler an Richard Beer-Hofmann, 23. 8. 1899}
\nopagebreak\mylabel{L00963v}
\rehead{ }\normalsize\beginnumbering\briefempfaengerindex{Beer-Hofmann, Richard@\textsc{Beer-Hofmann, Richard}!zzzSchnitzler, Arthur@\emph{von Arthur Schnitzler}!1899-08-231@{23. 8. 1899}|(be}
\toendnotes[C]{\smallbreak\pagebreak[2]}\Standort{YCGL, MSS 31.}
\physDesc{Telegramm, 171 Zeichen
\newline{}Handschrift einer Schreibkraft: schwarze Tinte, deutsche Kurrent
\newline{}Versand: »\noindent{}\textcolor{gray}{\textbf{Von}}{ }Kaltenbach Ischl\oindex{Kaltenbach@\textbf{Kaltenbach}, \emph{Teil eines besiedelten Ortes (A.BSOX)}|pw}{ / }\textcolor{gray}{\textbf{Aufgabe-Nr.}} 544 \textcolor{gray}{\textbf{mit}}{ }21 \textcolor{gray}{\textbf{Taxworten ({\dotsfive}Worten {\dotsfive}Chiffern)}}{ / }\textcolor{gray}{\textbf{Aufgegeben am}}{ }23/8 \textcolor{gray}{\textbf{18}}99{ }\textcolor{gray}{\textbf{um}}{ }9 \textcolor{gray}{\textbf{Uhr}}{ }45 \textcolor{gray}{\textbf{Min.}}{ }0 \textcolor{gray}{\textbf{Mittag}}{ / }\textcolor{gray}{\textbf{Eingelangt von}}{ }Kl{ }\textcolor{gray}{\textbf{auf Leitung Nr.}}{ }181{ }\textcolor{gray}{\textbf{am}}{ }23/8 \textcolor{gray}{\textbf{189}}9{ }\textcolor{gray}{\textbf{um}}{ }10 \textcolor{gray}{\textbf{Uhr}} 45 \textcolor{gray}{\textbf{Min.}}{ }\textcolor{gray}{v}\textcolor{gray}{\textbf{Mittag}}« }
\buchAbdrucke{\weitereDrucke{Arthur Schnitzler, Richard Beer-Hofmann: \emph{Briefwechsel 1891–1931}. Wien, Zürich: \emph{Europaverlag} 1992, S. 133.} }\toendnotes[C]{\smallbreak}\pstart{}{\pb}\textsc{Richard Beer Hofmann}{ }Villa \textsc{Platzer}\oindex{Villa Platzer@\textbf{Villa Platzer}, \emph{Gebäude (K.GBD)}|pw}{ }\textsc{Seeboden}{\\}Millſtätterſee\oindex{Seeboden@\textbf{Seeboden}, \emph{A.ADM3}|pw}\pend{}{\bigskip}\vspace{1em}
\pstart
           \noindent{}{\pb}Sprach \textsc{Singer}\pwindex{Singer, Isidor 16.01.1857 – 08.12.1927@\textsc{Singer, Isidor} (16.01.1857 – 08.12.1927), \emph{Journalist/Journalistin, Herausgeber/Herausgeberin, Soziologe/Soziologin}|pw} einsendung ihrer \textsc{Novelle}\pwindex{Tod Georgs. Fragment@\emph{Der Tod Georgs. Fragment}|pwv} möglichſt beſchleunigen we{\geminationn} Terminwünſche erfüllen
               werden ſollen herzlichſt\pend
           \pstart \spacefill\mbox{Arthur}\pend{}\selectlanguage{ngerman}\endnumbering\briefempfaengerindex{Beer-Hofmann, Richard@\textsc{Beer-Hofmann, Richard}!zzzSchnitzler, Arthur@\emph{von Arthur Schnitzler}!1899-08-231@{23. 8. 1899}|)be}\mylabel{L00963h}  \normalsize

\doendnotes{C}
\bigskip
\vfill

\clearpage

\footnotesize

\lohead{\textsc{register}}

% Definiere theindex-Environment komplett neu ohne reledmac
\makeatletter
\renewenvironment{theindex}{%
  \section*{\indexname}%
  \setlength{\parindent}{0pt}%
  \setlength{\parskip}{0pt plus 0.3pt}%
  \let\item\@idxitem
}{%
  \clearpage
}
\makeatother

\IfFileExists{\jobname-pw.ind}{\input{\jobname-pw.ind}}{}

\end{document}

      