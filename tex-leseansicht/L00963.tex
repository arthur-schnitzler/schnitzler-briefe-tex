%% latex-leseansicht-vorspann.tex
%% Vorspann für die Leseansicht.
%% Lädt die gemeinsame Datei latex-vorspann.tex mit nicht gesetztem Schalter.

\newif\ifkorrekturansicht
\korrekturansichtfalse

\input{../tex-inputs/latex-vorspann}


\section[Arthur Schnitzler an Richard Beer-Hofmann, 23. 8. 1899]{L00963 Arthur Schnitzler an Richard Beer-Hofmann, 23. 8. 1899}
\nopagebreak\mylabel{L00963v}
\rehead{ }\normalsize\beginnumbering\briefempfaengerindex{Beer-Hofmann, Richard@\textsc{Beer-Hofmann, Richard}!zzzSchnitzler, Arthur@\emph{von Arthur Schnitzler}!1899-08-231@{23. 8. 1899}|(be}
\toendnotes[C]{\smallbreak\pagebreak[2]}
\correspDesc{Versand  durch Arthur Schnitzler am 23. 8. 1899 in Bad Ischl
\newline{}Erhalt  durch Richard Beer-Hofmann am 23. 8. 1899 in Seeboden}\toendnotes[C]{\smallbreak}
\Standort{YCGL, MSS 31.}
\physDesc{Telegramm, 171 Zeichen
\newline{}HandschriftX2 einer Schreibkraft: schwarze Tinte, deutsche Kurrent
\newline{}Versand: »\noindent{}\textcolor{gray}{\textbf{Von}}{ }Kaltenbach Ischl\oindex{Kaltenbach@\textbf{Kaltenbach}, \emph{Teil eines besiedelten Ortes}|pw}{ / }\textcolor{gray}{\textbf{Aufgabe-Nr.}} 544 \textcolor{gray}{\textbf{mit}}{ }21 \textcolor{gray}{\textbf{Taxworten ({\dotsfive}Worten {\dotsfive}Chiffern)}}{ / }\textcolor{gray}{\textbf{Aufgegeben am}}{ }23/8 \textcolor{gray}{\textbf{18}}99{ }\textcolor{gray}{\textbf{um}}{ }9 \textcolor{gray}{\textbf{Uhr}}{ }45 \textcolor{gray}{\textbf{Min.}}{ }0 \textcolor{gray}{\textbf{Mittag}}{ / }\textcolor{gray}{\textbf{Eingelangt von}}{ }Kl{ }\textcolor{gray}{\textbf{auf Leitung Nr.}}{ }181{ }\textcolor{gray}{\textbf{am}}{ }23/8 \textcolor{gray}{\textbf{189}}9{ }\textcolor{gray}{\textbf{um}}{ }10 \textcolor{gray}{\textbf{Uhr}} 45 \textcolor{gray}{\textbf{Min.}}{ }\textcolor{gray}{v}\textcolor{gray}{\textbf{Mittag}}« }
\buchAbdrucke{\weitereDrucke{Arthur Schnitzler, Richard Beer-Hofmann: \emph{Briefwechsel 1891–1931}. Herausgegeben von Konstanze Fliedl. Wien, Zürich: \emph{Europaverlag} 1992, S. 133.} }\toendnotes[C]{\smallbreak}\pstart{}{\pb}\textsc{Richard Beer Hofmann}{ }Villa \textsc{Platzer}\oindex{Villa Platzer@\textbf{Villa Platzer}, \emph{Gebäude}|pw}{ }\textsc{Seeboden}{\\}Millſtätterſee\oindex{Seeboden am Millstättersee@\textbf{Seeboden am Millstättersee}|pw}\pend{}{\bigskip}\vspace{1em}
\pstart
           \noindent{}{\pb}Sprach \textsc{Singer}\pwindex{Singer, Isidor 16.\,1.\,1857 Budapest – 8.\,12.\,1927 Wien@\textsc{Singer, Isidor} (16.\,1.\,1857 Budapest – 8.\,12.\,1927 Wien), \emph{Journalist, Herausgeber, Soziologe}|pw} einsendung ihrer \textsc{Novelle}\pwindex{Beer-Hofmann, Richard 11.\,7.\,1866 Wien – 26.\,9.\,1945 New York City@\textsc{Beer-Hofmann, Richard} (11.\,7.\,1866 Wien – 26.\,9.\,1945 New York City), \emph{Schriftsteller}!Tod Georgs. Fragment@\strich\emph{Der Tod Georgs. Fragment}|pwv} möglichſt beſchleunigen we{\geminationn} Terminwünſche erfüllen
               werden{ }ſollen herzlichſt\pend
           \pstart \spacefill\mbox{Arthur}\pend{}\selectlanguage{ngerman}\endnumbering\briefempfaengerindex{Beer-Hofmann, Richard@\textsc{Beer-Hofmann, Richard}!zzzSchnitzler, Arthur@\emph{von Arthur Schnitzler}!1899-08-231@{23. 8. 1899}|)be}\mylabel{L00963h}  \newcommand{\dateiname}{L00963}\newcommand{\titel}{Arthur Schnitzler an Richard Beer-Hofmann, 23. 8. 1899}\newcommand{\editorInnen}{Martin Anton Müller und Gerd-Hermann Susen}%% latex-leseansicht-abspann.tex
%% Abspann für die Leseansicht.
%% Der Schalter \ifkorrekturansicht ist bereits durch den Vorspann gesetzt.

%% latex-abspann.tex
%% Gemeinsamer Abspann für Korrekturansicht und Leseansicht.
%% Setzt den Schalter \ifkorrekturansicht voraus (gesetzt in den
%% einbindenden Dateien latex-korrekturansicht-abspann.tex bzw.
%% latex-leseansicht-abspann.tex).
%% ---------------------------------------------------------------

\normalsize

% Das esempio-Environment wird nur in der Leseansicht benötigt
\ifkorrekturansicht\else
\newenvironment{esempio}[3]%
{
    \vspace{1.5ex}
    \rlap{\underline{#1}}
    \par
    \setlength{\parindent}{0cm}
    \nopagebreak
    \leftskip=#2cm
    \rightskip=#3cm
}
{
    \par
}
\fi

\doendnotes{C}
\bigskip
\vfill

\clearpage

\footnotesize

\ifkorrekturansicht
  \lohead{\textsc{register}}
\fi

% theindex-Environment neu definieren ohne reledmac
\makeatletter
\renewenvironment{theindex}{%
  \ifkorrekturansicht
    \section*{\indexname}%
  \else
    \subsubsection*{Index der erwähnten Entitäten}%
  \fi
  \setlength{\parindent}{0pt}%
  \setlength{\parskip}{0pt plus 0.3pt}%
  \let\item\@idxitem
}{%
  \ifkorrekturansicht\clearpage\fi
}
\makeatother

\IfFileExists{\jobname-pw.ind}{\input{\jobname-pw.ind}}{}

% Quellenangabe nur in der Leseansicht
\ifkorrekturansicht\else
% Fallback-Definitionen, falls die .tex-Datei \titel etc. nicht gesetzt hat
\providecommand{\titel}{}
\providecommand{\editorInnen}{}
\providecommand{\dateiname}{\jobname}

\vspace{3cm}

\vfill

\footnotesize
\textsc{Quelle}: \titel. Herausgegeben von {\editorInnen}. In: \emph{Arthur Schnitzler: Briefwechsel mit Autorinnen und Autoren}.
 Digitale Edition, https://schnitzler-briefe.acdh.oeaw.ac.at/{\dateiname}.html (Stand \today)
\fi

\end{document}


