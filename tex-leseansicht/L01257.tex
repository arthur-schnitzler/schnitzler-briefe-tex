\input{../tex-inputs/latex-pdf-vorspann}
\begin{center}
            \textcolor{red}{ENTWURF. ENTZIFFERUNG NOCH NICHT KORREKTURGELESEN}
                      \end{center}
            
               \section[Hugo von Hofmannsthal an Arthur Schnitzler, 18. 12. 1902]{ Hugo von Hofmannsthal an Arthur Schnitzler, 18. 12. 1902}\nopagebreak\mylabel{v}\rehead{ }\begin{ledgroupsized}[t]{13cm}\normalsize\beginnumbering\briefempfaengerindex{Schnitzler, Arthur@\textsc{Schnitzler, Arthur}!zzzHofmannsthal, Hugo von@\emph{von Hugo von Hofmannsthal}!1902-12-181@{18. 12. 1902}|(be} \toendnotes[C]{\smallbreak\pagebreak[2]} \Standort{CUL, Schnitzler, B 43.}
\physDesc{Postkarte
\newline{}Handschrift: schwarze Tinte, deutsche Kurrent\newline{}Versand: 1) Stempel: »\nobreak{}\oindex{Rodaun@\textbf{Rodaun}|pwk}Rodaun, 18 12 02\nobreak{}«.  2) Stempel: »\nobreak{}\oindex{IX., Alsergrund@\textbf{IX., Alsergrund}|pwk}Wien 9/3, 19. 12. 02, 8.V., Bestellt\nobreak{}«. 
\newline{}Schnitzler: mit Bleistift datiert: »18/12 902« \newline{}Ordnung: 1) mit Bleistift von unbekannter Hand nummeriert: »\strikeout{207}« 2) mit Bleistift von unbekannter Hand nummeriert: »189«}\buchAbdrucke{\weitereDrucke{Hugo von Hofmannsthal, Arthur Schnitzler: \emph{Briefwechsel}. Hg. Therese Nickl und Heinrich Schnitzler. Frankfurt am Main: \emph{S. Fischer} 1964, S. 164–165.} }\toendnotes[C]{\smallbreak}\pstart{}{\pb}\textsc{Herrn D\textsuperscript{r} Arthur Schnitzler}\pend{}\pstart{}\textsc{Wien}\oindex{Wien@\textbf{Wien}|pw}\pend{}\pstart{}\textsc{IX. Franckgasse 1}\oindex{Frankgasse@\textbf{Frankgasse}|pw}.\pend{}{\bigskip}\pstart
           \noindent{}{\pb}lieber, ſehe
               keine andere Möglichkeit Sie auf längere Zeit hinaus zu ſehen als wenn es geſtattet
               iſt \uline{\label{K_L01257_1v}\edtext{Samstag}{\lemma{\textnormal{\emph{Samstag}}}\Cendnote{\textnormal{siehe A. S.: \emph{Tagebuch}, 20. 12. 1902}}}\label{K_L01257_1h}} um
                  ½ 2 bei Ihrer Mama\pwindex{Schnitzler, Louise 08.07.1840 – 09.09.1911@\textsc{Schnitzler, Louise} (08.07.1840 – 09.09.1911)|pwv} mit Ihnen zu eſſen. Ich käme ſchon um 1\textsuperscript{h} zu Ihnen, um vorher ein biſſerl zu plaudern, weil um 3\textsuperscript{h} wieder weg müſste.\pend
           \pstart
           Hoffe es paſst Ihnen, dann \uuline{keine} Antwort nöthig,
               andernfalls bitte ſogleich telephonieren.\pend
           \pstart Von Herzen \spacefill\mbox{Hugo.}\pend{}\endnumbering\briefempfaengerindex{Schnitzler, Arthur@\textsc{Schnitzler, Arthur}!zzzHofmannsthal, Hugo von@\emph{von Hugo von Hofmannsthal}!1902-12-181@{18. 12. 1902}|)be}\mylabel{h}\end{ledgroupsized}  \newcommand{\dateiname}{L01257}\newcommand{\titel}{Hugo von Hofmannsthal an Arthur Schnitzler, 18. 12. 1902}\newcommand{\editorInnen}{Martin Anton Müller und Gerd-Hermann Susen}\input{../tex-inputs/latex-pdf-abspann}
      