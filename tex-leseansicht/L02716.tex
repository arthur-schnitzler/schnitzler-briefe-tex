%% latex-leseansicht-vorspann.tex
%% Vorspann für die Leseansicht.
%% Lädt die gemeinsame Datei latex-vorspann.tex mit nicht gesetztem Schalter.

\newif\ifkorrekturansicht
\korrekturansichtfalse

\input{../tex-inputs/latex-vorspann}


\section[Paul Goldmann an Arthur Schnitzler, 14. 9. [1893]]{L02716 Paul Goldmann an Arthur Schnitzler, 14. 9. [1893]}
\nopagebreak\mylabel{L02716v}
\rehead{ }\normalsize\beginnumbering\briefempfaengerindex{Schnitzler, Arthur@\textsc{Schnitzler, Arthur}!zzzGoldmann, Paul@\emph{von Paul Goldmann}!1893-09-141@{14. 9. [1893]}|(be}
\toendnotes[C]{\smallbreak\pagebreak[2]}
\correspDesc{Versand  durch Paul Goldmann am 14. 9. [1893] in Salzburg
\newline{}Erhalt  durch Arthur Schnitzler im Zeitraum [15. 9. 1893
                  – 19. 9. 1893?] in Wien}\toendnotes[C]{\smallbreak}
\Standort{DLA, A:Schnitzler, HS.NZ85.1.3163.}
\physDesc{Brief, 1 Blatt, 2 Seiten, 911 Zeichen
\newline{}Handschrift: schwarze Tinte, deutsche Kurrent
\newline{}Schnitzler: mit Bleistift das Jahr »93« vermerkt }\toendnotes[C]{\smallbreak}
\pstart
           {\pb}\textcolor{gray}{\textbf{\textbf{Frankfurter Zeitung\orgindex{Frankfurter Zeitung@Frankfurter Zeitung|pw}.}}}\pend
           
\pstart
           \textcolor{gray}{\textbf{\textbf{(\begin{otherlanguage}{french}Gazette de Francfort\end{otherlanguage}\orgindex{Frankfurter Zeitung@Frankfurter Zeitung|pw}.)}}}\pend
           
\pstart
           \textcolor{gray}{\textbf{\begin{otherlanguage}{french}Directeur\end{otherlanguage}{ }\textbf{M. L. Sonnemann\pwindex{Sonnemann, Leopold 29.\,10.\,1831 Höchberg – 30.\,10.\,1909 Frankfurt am Main@\textsc{Sonnemann, Leopold} (29.\,10.\,1831 Höchberg – 30.\,10.\,1909 Frankfurt am Main), \emph{Journalist, Herausgeber}|pw}.}}}\hfill \textsc{Salzburg\oindex{Salzburg@\textbf{Salzburg}, \emph{Verwaltungsgebiet}|pw}}, 14. September.\pend
           
\pstart
           \begin{otherlanguage}{french}\textcolor{gray}{\textbf{Journal politique, financier,}}\end{otherlanguage}\pend
           
\pstart
           \begin{otherlanguage}{french}\textcolor{gray}{\textbf{commercial et litteraire.}}\end{otherlanguage}\pend
           
\pstart
           \begin{otherlanguage}{french}\textcolor{gray}{\textbf{\textbf{Paraissant trois fois par jour}}}\end{otherlanguage}\pend
           
\pstart
           \begin{otherlanguage}{french}\textcolor{gray}{\textbf{\textbf{Bureaux à Paris\oindex{Paris@\textbf{Paris}, \emph{Hauptstadt}|pw}:}}}\end{otherlanguage}\pend
           
\pstart
           \begin{otherlanguage}{french}\textcolor{gray}{\textbf{\textbf{rue Richelieu 75\oindex{rue Richelieu@\textbf{rue Richelieu}, \emph{Straße}|pw}.}}}\end{otherlanguage}\pend
           
\pstart\center{}Mein lieber Arthur!\pend\vspace{0.5em}
\pstart
           Ich würdige das Opfer, das Du mir bringſt, in{ }ſeinem vollen Werth und danke es Dir \substVorne{}\textsuperscript{\textcolor{gray}{×}}\substDazwischen{}v\substHinten{}on Herzen. Die zwei Tage bis zu Deiner Ankunft werden recht lang werden. Aber
               noch ein letztes Mal: geringe Erwartung, bitte, in Bezug auf mich. Ich bin{ }ſo \label{K_L02716-1v}\edtext{\textsc{\begin{otherlanguage}{french}par terre\end{otherlanguage}}}{\lemma{\textnormal{\emph{par terre}}}\Cendnote{\textnormal{französisch: am Boden}}}\label{K_L02716-1} durch all’
               das Unheil.\pend
           
\pstart
           Mein Onkel\pwindex{Mamroth, Fedor 21.\,2.\,1851 Breslau – 25.\,6.\,1907 Frankfurt am Main@\textsc{Mamroth, Fedor} (21.\,2.\,1851 Breslau – 25.\,6.\,1907 Frankfurt am Main), \emph{Journalist, Kritiker}|pwv} iſt hier. Ob er
               noch \label{K_L02716-2v}\edtext{zur Zeit Deiner Ankunft hier}{\lemma{\textnormal{\emph{zur … hier}}}\Cendnote{\textnormal{Fedor Mamroth\pwindex{Mamroth, Fedor 21.\,2.\,1851 Breslau – 25.\,6.\,1907 Frankfurt am Main@\textsc{Mamroth, Fedor} (21.\,2.\,1851 Breslau – 25.\,6.\,1907 Frankfurt am Main), \emph{Journalist, Kritiker}|pwkv} war noch in
                     Salzburg\oindex{Salzburg@\textbf{Salzburg}, \emph{Verwaltungsgebiet}|pwk}. Am 17. 9. 1893 besuchte er gemeinsam mit Goldmann\pwindex{Goldmann, Paul 31.\,1.\,1865 Breslau – 25.\,9.\,1935 Wien@\textsc{Goldmann, Paul} (31.\,1.\,1865 Breslau – 25.\,9.\,1935 Wien), \emph{Schriftsteller, Journalist}|pwk} und Schnitzler{ }Hellbrunn\oindex{Hellbrunn@\textbf{Hellbrunn}|pwk}.}}}\label{K_L02716-2}{ }ſein wird, iſt nicht{ }ſicher,
               aber wahrſcheinlich. Ob das {\pb}Hotel\oindex{Hotel Goldenes Horn@\textbf{Hotel Goldenes Horn}, \emph{Hotel}|pwv} düſter iſt oder nicht,
               weiß ich eigentlich nicht recht zu{ }ſagen. Aber billige Wohnung, gute Koſt, angenehme
               Bedienung. Bitte, telegraphire noch Samſtag: Abgereiſt \substVorne{}\textsuperscript{.}\substDazwischen{}–\substHinten{} ein Wort. Dann beſtelle ich Dir ein Zimmer.\pend
           
\pstart
           \label{K_L02716-3v}\edtext{Volkstheater\orgindex{Volkstheater@Volkstheater|pw}}{\lemma{\textnormal{\emph{Volkstheater}}}\Cendnote{\textnormal{\emph{Das Märchen}\pwindex{Schnitzler, Arthur 15.\,5.\,1862 Wien – 21.\,10.\,1931 ebd.@\textsc{Schnitzler, Arthur} (15.\,5.\,1862 Wien – 21.\,10.\,1931 ebd.), \emph{Schriftsteller, Mediziner}!Märchen. Schauspiel in drei Aufzügen@\strich\emph{Das Märchen. Schauspiel in drei Aufzügen}|pwk} wurde am 1. 9. 1893 von Emerich von Bukovics\pwindex{Bukovics, Emerich von 28.\,2.\,1844 Wien – 4.\,7.\,1905 ebd.@\textsc{Bukovics, Emerich von} (28.\,2.\,1844 Wien – 4.\,7.\,1905 ebd.), \emph{Journalist, Theaterleiter}|pwk}, dem Leiter\pwindex{Bukovics, Emerich von 28.\,2.\,1844 Wien – 4.\,7.\,1905 ebd.@\textsc{Bukovics, Emerich von} (28.\,2.\,1844 Wien – 4.\,7.\,1905 ebd.), \emph{Journalist, Theaterleiter}|pwkv} des \emph{Volkstheaters}\orgindex{Volkstheater@Volkstheater|pwk}, angenommen. Am 1. 12. 1893 kam es dort zur
                  Uraufführung\eventindex{Volkstheater@\textbf{Volkstheater}!Uraufführung von Das Märchen, 1.12.1893@Uraufführung von Das Märchen, 1.12.1893|pwkv}.}}}\label{K_L02716-3}: Ich bin \label{K_L02716-4v}\edtext{nicht einverſtanden}{\lemma{\textnormal{\emph{nicht einverstanden}}}\Cendnote{\textnormal{Vgl. XXXX Auszeichnungsfehler: Dokument L02712 nicht gefunden.
               }}}\label{K_L02716-4}, wünſche aber natürlich, daß es zum Guten{ }ſein möge. Nun, wir reden ja
               darüber. Reden! Es iſt{ }ſo{ }ſchön, daß ich feſt überzeugt bin, es wird nichts
               daraus.\pend
           
\pstart
           Grüß’ Dich Gott, Lieber und Treuer! {\\[\baselineskip]}Dein {\\[\baselineskip]}\spacefill\mbox{Paul Goldmann.}\pend
           \leftskip=0em{}
\pstart
           \noindent{}\textsc{Getreidegasse\oindex{Getreidegasse@\textbf{Getreidegasse}, \emph{Weg}|pw}}, nicht -\textsc{markt}.\pend
           \selectlanguage{ngerman}\endnumbering\briefempfaengerindex{Schnitzler, Arthur@\textsc{Schnitzler, Arthur}!zzzGoldmann, Paul@\emph{von Paul Goldmann}!1893-09-141@{14. 9. [1893]}|)be}\mylabel{L02716h}  \newcommand{\dateiname}{L02716}\newcommand{\titel}{Paul Goldmann an Arthur Schnitzler, 14. 9. [1893]}\newcommand{\editorInnen}{Martin Anton Müller und Laura Untner}%% latex-leseansicht-abspann.tex
%% Abspann für die Leseansicht.
%% Der Schalter \ifkorrekturansicht ist bereits durch den Vorspann gesetzt.

%% latex-abspann.tex
%% Gemeinsamer Abspann für Korrekturansicht und Leseansicht.
%% Setzt den Schalter \ifkorrekturansicht voraus (gesetzt in den
%% einbindenden Dateien latex-korrekturansicht-abspann.tex bzw.
%% latex-leseansicht-abspann.tex).
%% ---------------------------------------------------------------

\normalsize

% Das esempio-Environment wird nur in der Leseansicht benötigt
\ifkorrekturansicht\else
\newenvironment{esempio}[3]%
{
    \vspace{1.5ex}
    \rlap{\underline{#1}}
    \par
    \setlength{\parindent}{0cm}
    \nopagebreak
    \leftskip=#2cm
    \rightskip=#3cm
}
{
    \par
}
\fi

\doendnotes{C}
\bigskip
\vfill

\clearpage

\footnotesize

\ifkorrekturansicht
  \lohead{\textsc{register}}
\fi

% theindex-Environment neu definieren ohne reledmac
\makeatletter
\renewenvironment{theindex}{%
  \ifkorrekturansicht
    \section*{\indexname}%
  \else
    \subsubsection*{Index der erwähnten Entitäten}%
  \fi
  \setlength{\parindent}{0pt}%
  \setlength{\parskip}{0pt plus 0.3pt}%
  \let\item\@idxitem
}{%
  \ifkorrekturansicht\clearpage\fi
}
\makeatother

\IfFileExists{\jobname-pw.ind}{\input{\jobname-pw.ind}}{}

% Quellenangabe nur in der Leseansicht
\ifkorrekturansicht\else
% Fallback-Definitionen, falls die .tex-Datei \titel etc. nicht gesetzt hat
\providecommand{\titel}{}
\providecommand{\editorInnen}{}
\providecommand{\dateiname}{\jobname}

\vspace{3cm}

\vfill

\footnotesize
\textsc{Quelle}: \titel. Herausgegeben von {\editorInnen}. In: \emph{Arthur Schnitzler: Briefwechsel mit Autorinnen und Autoren}.
 Digitale Edition, https://schnitzler-briefe.acdh.oeaw.ac.at/{\dateiname}.html (Stand \today)
\fi

\end{document}


