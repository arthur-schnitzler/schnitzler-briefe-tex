\input{../tex-inputs/latex-pdf-vorspann}
\begin{center}
            \textcolor{red}{ENTWURF. ENTZIFFERUNG NOCH NICHT KORREKTURGELESEN}
                      \end{center}
            
               \section[Paul Goldmann an Arthur Schnitzler, 14. 9. {[}1893{]}]{ Paul Goldmann an Arthur Schnitzler, 14. 9. {[}1893{]}}\nopagebreak\mylabel{v}\rehead{ }\begin{ledgroupsized}[t]{13cm}\normalsize\beginnumbering\briefempfaengerindex{Schnitzler, Arthur@\textsc{Schnitzler, Arthur}!zzzGoldmann, Paul@\emph{von Paul Goldmann}!1893-09-141@{14. 9. {[}1893{]}}|(be} \toendnotes[C]{\smallbreak\pagebreak[2]} \Standort{DLA, A:Schnitzler, HS.NZ85.1.3163.}
\physDesc{Brief, 1 Blatt, 2 Seiten
\newline{}Handschrift: schwarze Tinte, deutsche Kurrent
\newline{}Schnitzler: mit Bleistift das Jahr »93« vermerkt }\toendnotes[C]{\smallbreak}\pstart
           \noindent{}{\pb}\textcolor{gray}{\textbf{\textbf{Frankfurter Zeitung\orgindex{Frankfurter Zeitung@Frankfurter Zeitung|pw}.}}}\pend
           \pstart
           \textcolor{gray}{\textbf{\textbf{(\begin{otherlanguage}{french}Gazette de Francfort\end{otherlanguage}\orgindex{Frankfurter Zeitung@Frankfurter Zeitung|pw}.)}}}\pend
           \pstart
           \textcolor{gray}{\textbf{\begin{otherlanguage}{french}Directeur\pwindex{Sonnemann, Leopold 1831-10-29 – 1909-10-30@\textsc{Sonnemann, Leopold} (1831-10-29 – 1909-10-30), \emph{Journalist, Herausgeber}|pwv}\end{otherlanguage}{ }\textbf{M. L. Sonnemann\pwindex{Sonnemann, Leopold 1831-10-29 – 1909-10-30@\textsc{Sonnemann, Leopold} (1831-10-29 – 1909-10-30), \emph{Journalist, Herausgeber}|pw}.}}}\hfill \textsc{Salzburg\oindex{Salzburg@\textbf{Salzburg}|pw}}, 14. September.\pend
           \pstart
           \begin{otherlanguage}{french}\textcolor{gray}{\textbf{Journal\pwindex{Frankfurter Zeitung1856 – 1943@\emph{Frankfurter Zeitung}|pw} politique, financier,}}\end{otherlanguage}\pend
           \pstart
           \begin{otherlanguage}{french}\textcolor{gray}{\textbf{commercial et litteraire.}}\end{otherlanguage}\pend
           \pstart
           \begin{otherlanguage}{french}\textcolor{gray}{\textbf{\textbf{Paraissant trois fois par jour}}}\end{otherlanguage}\pend
           \pstart
           \begin{otherlanguage}{french}\textcolor{gray}{\textbf{\textbf{Bureaux à Paris\oindex{Paris@\textbf{Paris}|pw}:}}}\end{otherlanguage}\pend
           \pstart
           \begin{otherlanguage}{french}\textcolor{gray}{\textbf{\textbf{rue Richelieu 75\oindex{rue Richelieu@\textbf{rue Richelieu}|pw}.}}}\end{otherlanguage}\pend
           \pstart
           Mein lieber Arthur!\pend
           \pstart
           Ich würdige das Opfer, das Du mir bringſt, in ſeinem vollen Werth und danke es Dir
               von Herzen. Die zwei Tage bis zu Deiner Ankunft werden recht lang werden. Aber noch
               ein letztes Mal: geringe Erwartung, bitte, in Bezug auf mich. Ich bin ſo \label{K_L02716-1v}\edtext{\textsc{\begin{otherlanguage}{french}par terre\end{otherlanguage}}}{\lemma{\textnormal{\emph{par terre}}}\Cendnote{\textnormal{französisch: am Boden}}}\label{K_L02716-1h} durch all’
               das Unheil.\pend
           \pstart
           Mein Onkel\pwindex{Mamroth, Fedor 21.02.1851 – 25.06.1907@\textsc{Mamroth, Fedor} (21.02.1851 – 25.06.1907), \emph{Journalist, Kritiker}|pwv} iſt hier. Ob er
               noch \label{K_L02716-2v}\edtext{zur Zeit Deiner Ankunft hier}{\lemma{\textnormal{\emph{zur … hier}}}\Cendnote{\textnormal{Fedor Mamroth\pwindex{Mamroth, Fedor 21.02.1851 – 25.06.1907@\textsc{Mamroth, Fedor} (21.02.1851 – 25.06.1907), \emph{Journalist, Kritiker}|pwkv} war noch in
                     Salzburg\oindex{Salzburg@\textbf{Salzburg}|pwk}. Am 17. 9. 1893 besuchte er gemeinsam mit Goldmann\pwindex{Goldmann, Paul 31.01.1865 – 25.09.1935@\textsc{Goldmann, Paul} (31.01.1865 – 25.09.1935), \emph{Schriftsteller, Journalist}|pwk} und Schnitzler\pwindex{Schnitzler, Arthur 15.05.1862 – 21.10.1931@\textsc{Schnitzler, Arthur} (15.05.1862 – 21.10.1931), \emph{Schriftsteller, Mediziner}|pwk}{ }Hellbrunn\oindex{Hellbrunn@\textbf{Hellbrunn}|pwk}.}}}\label{K_L02716-2h} ſein wird, iſt nicht ſicher,
               aber wahrſcheinlich. Ob das {\pb}Hotel\oindex{XXXX Ortsangabe fehlt|pwv} düſter iſt oder nicht,
               weiß ich eigentlich nicht recht zu ſagen. Aber billige Wohnung, gute Koſt, angenehme
               Bedienung. Bitte, telegraphire noch Samſtag: Abgereiſt
                  \strikeout{.}– ein Wort. Dann beſtelle ich Dir ein Zimmer.\pend
           \pstart
           \label{K_L02716-3v}\edtext{Volkstheater\orgindex{Volkstheater@Volkstheater|pw}}{\lemma{\textnormal{\emph{Volkstheater}}}\Cendnote{\textnormal{\emph{Das Märchen}\pwindex{Schnitzler, Arthur 15.05.1862 – 21.10.1931@\textsc{Schnitzler, Arthur} (15.05.1862 – 21.10.1931), \emph{Schriftsteller, Mediziner}!Maerchen. Schauspiel in drei Aufzuegen1891 – 1891@\strich\emph{Das Märchen. Schauspiel in drei Aufzügen} {[}1891 – 1891{]}|pwk} wurde am 1. 9. 1893 von Emerich von Bukovics\pwindex{Bukovics, Emerich von 28.02.1844 – 04.07.1905@\textsc{Bukovics, Emerich von} (28.02.1844 – 04.07.1905), \emph{Journalist, Theaterleiter}|pwk}, dem Leiter\pwindex{Bukovics, Emerich von 28.02.1844 – 04.07.1905@\textsc{Bukovics, Emerich von} (28.02.1844 – 04.07.1905), \emph{Journalist, Theaterleiter}|pwkv} des \emph{Volkstheater}\orgindex{Volkstheater@Volkstheater|pwk}s, angenommen. Am 1. 12. 1893 kam es dort zur
                  Uraufführung.}}}\label{K_L02716-3h}: Ich bin nicht einverſtanden, wünſche aber natürlich, daß es
               zum Guten ſein möge. Nun, wir reden ja darüber. Reden! Es iſt ſo ſchön, daß ich feſt
               überzeugt bin, es wird nichts daraus.\pend
           \pstart
           Grüß’ Dich Gott, Lieber und Treuer! {\\[\baselineskip]}Dein {\\[\baselineskip]}\spacefill\mbox{Paul Goldmann.}\pend
           \leftskip=0em{}\pstart
           \noindent{}\textsc{Getreidegasse\oindex{Getreidegasse@\textbf{Getreidegasse}|pw}}, nicht -\textsc{markt}.\pend
           \endnumbering\briefempfaengerindex{Schnitzler, Arthur@\textsc{Schnitzler, Arthur}!zzzGoldmann, Paul@\emph{von Paul Goldmann}!1893-09-141@{14. 9. {[}1893{]}}|)be}\mylabel{h}\end{ledgroupsized}\begin{anhang}\end{anhang}\newcommand{\dateiname}{L02716}\newcommand{\titel}{Paul Goldmann an Arthur Schnitzler, 14. 9. [1893]}\newcommand{\editorInnen}{Martin Anton Müller und Laura Untner}\input{../tex-inputs/latex-pdf-abspann}
      