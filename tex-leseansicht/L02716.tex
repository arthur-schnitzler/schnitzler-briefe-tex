%% latex-korrekturansicht-vorspann.tex
%% Vorspann für die Korrekturansicht.
%% Lädt die gemeinsame Datei latex-vorspann.tex mit gesetztem Schalter.

\newif\ifkorrekturansicht
\korrekturansichttrue

\input{../tex-inputs/latex-vorspann}


\section[Paul Goldmann an Arthur Schnitzler, 14. 9. {[}1893{]}]{L02716 Paul Goldmann an Arthur Schnitzler, 14. 9. {[}1893{]}}
\nopagebreak\mylabel{L02716v}
\rehead{ }\normalsize\beginnumbering\briefempfaengerindex{Schnitzler, Arthur@\textsc{Schnitzler, Arthur}!zzzGoldmann, Paul@\emph{von Paul Goldmann}!1893-09-141@{14. 9. {[}1893{]}}|(be}
\toendnotes[C]{\smallbreak\pagebreak[2]}\Standort{DLA, A:Schnitzler, HS.NZ85.1.3163.}
\physDesc{Brief, 1 Blatt, 2 Seiten, 911 Zeichen
\newline{}Handschrift: schwarze Tinte, deutsche Kurrent
\newline{}Schnitzler: mit Bleistift das Jahr »93« vermerkt }\toendnotes[C]{\smallbreak}
\pstart
           {\pb}\textcolor{gray}{\textbf{\textbf{Frankfurter Zeitung\orgindex{Frankfurter Zeitung@Frankfurter Zeitung|pw}.}}}\pend
           
\pstart
           \textcolor{gray}{\textbf{\textbf{(\begin{otherlanguage}{french}Gazette de Francfort\end{otherlanguage}\orgindex{Frankfurter Zeitung@Frankfurter Zeitung|pw}.)}}}\pend
           
\pstart
           \textcolor{gray}{\textbf{\begin{otherlanguage}{french}Directeur\end{otherlanguage}{ }\textbf{M. L. Sonnemann\pwindex{Sonnemann, Leopold 1831-10-29 – 1909-10-30@\textsc{Sonnemann, Leopold} (1831-10-29 – 1909-10-30), \emph{Journalist/Journalistin, Herausgeber/Herausgeberin}|pw}.}}}\hfill \textsc{Salzburg\oindex{Salzburg@\textbf{Salzburg}, \emph{A.ADM2}|pw}}, 14. September.\pend
           
\pstart
           \begin{otherlanguage}{french}\textcolor{gray}{\textbf{Journal politique, financier,}}\end{otherlanguage}\pend
           
\pstart
           \begin{otherlanguage}{french}\textcolor{gray}{\textbf{commercial et litteraire.}}\end{otherlanguage}\pend
           
\pstart
           \begin{otherlanguage}{french}\textcolor{gray}{\textbf{\textbf{Paraissant trois fois par jour}}}\end{otherlanguage}\pend
           
\pstart
           \begin{otherlanguage}{french}\textcolor{gray}{\textbf{\textbf{Bureaux à Paris\oindex{Paris@\textbf{Paris}, \emph{P.PPLC}|pw}:}}}\end{otherlanguage}\pend
           
\pstart
           \begin{otherlanguage}{french}\textcolor{gray}{\textbf{\textbf{rue Richelieu 75\oindex{rue Richelieu@\textbf{rue Richelieu}, \emph{Straße (K.STR)}|pw}.}}}\end{otherlanguage}\pend
           
\pstart\center{}Mein lieber Arthur!\pend\vspace{0.5em}
\pstart
           Ich würdige das Opfer, das Du mir bringſt, in ſeinem vollen Werth und danke es Dir \substVorne{}\textsuperscript{\textcolor{gray}{×}}\substDazwischen{}v\substHinten{}on Herzen. Die zwei Tage bis zu Deiner Ankunft werden recht lang werden. Aber
               noch ein letztes Mal: geringe Erwartung, bitte, in Bezug auf mich. Ich bin ſo \label{K_L02716-1v}\edtext{\textsc{\begin{otherlanguage}{french}par terre\end{otherlanguage}}}{\lemma{\textnormal{\emph{par terre}}}\Cendnote{\textnormal{französisch: am Boden}}}\label{K_L02716-1} durch all’
               das Unheil.\pend
           
\pstart
           Mein Onkel\pwindex{Mamroth, Fedor 21.02.1851 – 25.06.1907@\textsc{Mamroth, Fedor} (21.02.1851 – 25.06.1907), \emph{Journalist/Journalistin, Kritiker/Kritikerin}|pwv} iſt hier. Ob er
               noch \label{K_L02716-2v}\edtext{zur Zeit Deiner Ankunft hier}{\lemma{\textnormal{\emph{zur … hier}}}\Cendnote{\textnormal{Fedor Mamroth\pwindex{Mamroth, Fedor 21.02.1851 – 25.06.1907@\textsc{Mamroth, Fedor} (21.02.1851 – 25.06.1907), \emph{Journalist/Journalistin, Kritiker/Kritikerin}|pwkv} war noch in
                     Salzburg\oindex{Salzburg@\textbf{Salzburg}, \emph{A.ADM2}|pwk}. Am 17. 9. 1893 besuchte er gemeinsam mit Goldmann\pwindex{Goldmann, Paul 31.01.1865 – 25.09.1935@\textsc{Goldmann, Paul} (31.01.1865 – 25.09.1935), \emph{Schriftsteller/Schriftstellerin, Journalist/Journalistin}|pwk} und Schnitzler{ }Hellbrunn\oindex{Hellbrunn@\textbf{Hellbrunn}, \emph{P.PPL}|pwk}.}}}\label{K_L02716-2} ſein wird, iſt nicht ſicher,
               aber wahrſcheinlich. Ob das {\pb}Hotel\oindex{Hotel Goldenes Horn@\textbf{Hotel Goldenes Horn}, \emph{Hotel (K.HTL)}|pwv} düſter iſt oder nicht,
               weiß ich eigentlich nicht recht zu ſagen. Aber billige Wohnung, gute Koſt, angenehme
               Bedienung. Bitte, telegraphire noch Samſtag: Abgereiſt \substVorne{}\textsuperscript{.}\substDazwischen{}–\substHinten{} ein Wort. Dann beſtelle ich Dir ein Zimmer.\pend
           
\pstart
           \label{K_L02716-3v}\edtext{Volkstheater\orgindex{Volkstheater@Volkstheater|pw}}{\lemma{\textnormal{\emph{Volkstheater}}}\Cendnote{\textnormal{\emph{Das Märchen}\pwindex{Maerchen. Schauspiel in drei Aufzuegen@\emph{Das Märchen. Schauspiel in drei Aufzügen}|pwk} wurde am 1. 9. 1893 von Emerich von Bukovics\pwindex{Bukovics, Emerich von 28.02.1844 – 04.07.1905@\textsc{Bukovics, Emerich von} (28.02.1844 – 04.07.1905), \emph{Journalist/Journalistin, Theaterleiter/Theaterleiterin}|pwk}, dem Leiter\pwindex{Bukovics, Emerich von 28.02.1844 – 04.07.1905@\textsc{Bukovics, Emerich von} (28.02.1844 – 04.07.1905), \emph{Journalist/Journalistin, Theaterleiter/Theaterleiterin}|pwkv} des \emph{Volkstheaters}\orgindex{Volkstheater@Volkstheater|pwk}, angenommen. Am 1. 12. 1893 kam es dort zur
                  Uraufführung.}}}\label{K_L02716-3}: Ich bin \label{K_L02716-4v}\edtext{nicht einverſtanden}{\lemma{\textnormal{\emph{nicht einverſtanden}}}\Cendnote{\textnormal{Vgl. Paul Goldmann an Arthur Schnitzler, 18. 8. [1893].
               }}}\label{K_L02716-4}, wünſche aber natürlich, daß es zum Guten ſein möge. Nun, wir reden ja
               darüber. Reden! Es iſt ſo ſchön, daß ich feſt überzeugt bin, es wird nichts
               daraus.\pend
           
\pstart
           Grüß’ Dich Gott, Lieber und Treuer! {\\[\baselineskip]}Dein {\\[\baselineskip]}\spacefill\mbox{Paul Goldmann.}\pend
           \leftskip=0em{}
\pstart
           \noindent{}\textsc{Getreidegasse\oindex{Getreidegasse@\textbf{Getreidegasse}, \emph{R.RD}|pw}}, nicht -\textsc{markt}.\pend
           \selectlanguage{ngerman}\endnumbering\briefempfaengerindex{Schnitzler, Arthur@\textsc{Schnitzler, Arthur}!zzzGoldmann, Paul@\emph{von Paul Goldmann}!1893-09-141@{14. 9. {[}1893{]}}|)be}\mylabel{L02716h}  \normalsize

\doendnotes{C}
\bigskip
\vfill

\clearpage

\footnotesize

\lohead{\textsc{register}}

% Definiere theindex-Environment komplett neu ohne reledmac
\makeatletter
\renewenvironment{theindex}{%
  \section*{\indexname}%
  \setlength{\parindent}{0pt}%
  \setlength{\parskip}{0pt plus 0.3pt}%
  \let\item\@idxitem
}{%
  \clearpage
}
\makeatother

\IfFileExists{\jobname-pw.ind}{\input{\jobname-pw.ind}}{}

\end{document}

      