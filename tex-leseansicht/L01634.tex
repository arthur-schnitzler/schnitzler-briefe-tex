%% latex-leseansicht-vorspann.tex
%% Vorspann für die Leseansicht.
%% Lädt die gemeinsame Datei latex-vorspann.tex mit nicht gesetztem Schalter.

\newif\ifkorrekturansicht
\korrekturansichtfalse

\input{../tex-inputs/latex-vorspann}


         
         \newcommand{\erwaehntePersonen}{Personen: }
         \newcommand{\erwaehnteInstitutionen}{}
         \newcommand{\erwaehnteOrte}{}
         \newcommand{\erwaehnteWerke}{
               \section[Arthur Schnitzler an Hermann Bahr, 18. 10. 1906]{ Arthur Schnitzler an Hermann Bahr, 18. 10. 1906 }\nopagebreak\mylabel{v}\rehead{ }\begin{ledgroupsized}[t]{13cm}\normalsize\beginnumbering \toendnotes[C]{\smallbreak\pagebreak[2]} \Standort{TMW, HS AM 23383 Ba.}
\physDesc{Brief, 1 Blatt, 2 Seiten
\newline{}Handschrift: schwarze Tinte, deutsche Kurrent\newline{}Ordnung: Lochung }\buchAbdrucke{\weitereDrucke{1) \emph{18. 10. 1906.} In: Arthur Schnitzler: \emph{The Letters of Arthur Schnitzler to Hermann Bahr}. Edited, annotated, and with an introduction, by Donald G.
                        Daviau. Chapel Hill: \emph{The University of North Carolina Press} 1978, S. 95–96 (University of North Carolina studies in the Germanic languages
                        and literatures, 89).} \weitereDrucke{2) Hermann Bahr, Arthur Schnitzler: \emph{Briefwechsel, Aufzeichnungen, Dokumente (1891–1931)}. Hg. Kurt Ifkovits und Martin Anton Müller. Göttingen: \emph{Wallstein} 2018, S. 383–384.} }\toendnotes[C]{\smallbreak}\pstart
           \raggedleft{}{\pb}Wien\oindex{XXXX Ortsangabe fehlt|pw}, 18. X. 906\pend
           \pstart{}lieber Hermann, \pend\pstart
           eine Aehnlichkeit zwiſchen deinem
                  Akt\textcolor{red}{\textsuperscript{XXXX indx}} und dem Abſchiedſouper\textcolor{red}{\textsuperscript{XXXX indx}} wäre höchſtens
               irgendwo im äußerlich ſtofflichen zu finden, im innerlich stofflichen ſchon nicht
               mehr, und gewiſs nicht im \strikeout{eigentlich} »ſeeliſch
               geſtaltlichen« – \introOben{}(\introOben{}um zu i{\geminationm}er
               grauenhafteren Worten auf- oder niederzuſteigen). Dein Problem ist viel verzwickter,
               der Fortgang der Handlung gedrehter, ſpiraliger, jüdiſcher gegenüber der naiv \textsc{gauloisen}\oindex{XXXX Ortsangabe fehlt|pw} Fabel des braven alten Anatolſtückls\textcolor{red}{\textsuperscript{XXXX indx}}, außerdem wird bei mir ſoupirt und bei dir doch eigentlich nur
                  \label{K_L01634_1v}\edtext{gejauſnet}{\lemma{\textnormal{\emph{gejauſnet}}}\Cendnote{\textnormal{österreichisch Jause: Zwischenmahlzeit}}}\label{K_L01634_1h}. Die Atmosphäre
               deines Stücks ist dünner, ſchärfer; das ganze brutaler (für {\pb}meinen Geſchmack im
               Beginn beſonders bis zum Abſtoßenden brutal) angepackt. Wenn du mir, oder dem guten
                  Anatol\textcolor{red}{\textsuperscript{XXXX indx}}, dieſen intereſſanten Einakter\textcolor{red}{\textsuperscript{XXXX indx}} widmen willſt, ſo nehm
               ich s natürlich mit Dank u Rührung an, nur mußt du mir erlauben, deine Erinnerung
               nicht als Anregungsqui\damage{tt}irung und Ausdruck einer Gewiſſensſchuld ſondern als ein neues und daher mir
                  willko{\geminationm}enes Zeichen unſerer guten Zuſa{\geminationm}engehörigkeit zu empfinden u zu empfangen.\pend
           \pstart
           Hoffentlich fügt es ſich, dſs wir einander vor deiner Abreiſe noch einmal ſehen.
               (Gern möcht ich auch etwas, \label{K_L01634_2v}\edtext{\textsc{Reinhardt}\pwindex{\textcolor{red}{\textsuperscript{XXXX1 indx}}|pw} betreffendes}{\lemma{\textnormal{\emph{Reinhardt betreffendes}}}\Cendnote{\textnormal{eine Aufführung von
                     \emph{Der Schleier der Beatrice}\textcolor{red}{\textsuperscript{XXXX indx}}, vgl. A. S.: \emph{Tagebuch}, 29. 10. 1906 und vgl. den
                     Brief von Schnitzler\pwindex{\textcolor{red}{\textsuperscript{XXXX1 indx}}|pwk} an Max Reinhardt\pwindex{\textcolor{red}{\textsuperscript{XXXX1 indx}}|pwk}, 24. 12. 1909 in A. S. \emph{Briefe} I,613–621.}}}\label{K_L01634_2h}, aber hauptſächlich in
                  \damage{mei}nem Intereſſe liegendes\strikeout{)} mit dir
               beſprechen.)\pend
           \pstart
           Herzlichſt, mit Grüßen von{\\[\baselineskip]}meiner Frau\pwindex{\textcolor{red}{\textsuperscript{XXXX1 indx}}|pwv} u mir{\\[\baselineskip]}dein{\\[\baselineskip]}\spacefill\mbox{Arthur}\pend
           \leftskip=0em{}
         
         \endnumbering\mylabel{h}\end{ledgroupsized}  \newcommand{\dateiname}{L01634}\newcommand{\titel}{Arthur Schnitzler an Hermann Bahr, 18. 10. 1906}\newcommand{\editorInnen}{ Kurt Ifkovits,  Martin Anton Müller}%% latex-leseansicht-abspann.tex
%% Abspann für die Leseansicht.
%% Der Schalter \ifkorrekturansicht ist bereits durch den Vorspann gesetzt.

%% latex-abspann.tex
%% Gemeinsamer Abspann für Korrekturansicht und Leseansicht.
%% Setzt den Schalter \ifkorrekturansicht voraus (gesetzt in den
%% einbindenden Dateien latex-korrekturansicht-abspann.tex bzw.
%% latex-leseansicht-abspann.tex).
%% ---------------------------------------------------------------

\normalsize

% Das esempio-Environment wird nur in der Leseansicht benötigt
\ifkorrekturansicht\else
\newenvironment{esempio}[3]%
{
    \vspace{1.5ex}
    \rlap{\underline{#1}}
    \par
    \setlength{\parindent}{0cm}
    \nopagebreak
    \leftskip=#2cm
    \rightskip=#3cm
}
{
    \par
}
\fi

\doendnotes{C}
\bigskip
\vfill

\clearpage

\footnotesize

\ifkorrekturansicht
  \lohead{\textsc{register}}
\fi

% theindex-Environment neu definieren ohne reledmac
\makeatletter
\renewenvironment{theindex}{%
  \ifkorrekturansicht
    \section*{\indexname}%
  \else
    \subsubsection*{Index der erwähnten Entitäten}%
  \fi
  \setlength{\parindent}{0pt}%
  \setlength{\parskip}{0pt plus 0.3pt}%
  \let\item\@idxitem
}{%
  \ifkorrekturansicht\clearpage\fi
}
\makeatother

\IfFileExists{\jobname-pw.ind}{\input{\jobname-pw.ind}}{}

% Quellenangabe nur in der Leseansicht
\ifkorrekturansicht\else
% Fallback-Definitionen, falls die .tex-Datei \titel etc. nicht gesetzt hat
\providecommand{\titel}{}
\providecommand{\editorInnen}{}
\providecommand{\dateiname}{\jobname}

\vspace{3cm}

\vfill

\footnotesize
\textsc{Quelle}: \titel. Herausgegeben von {\editorInnen}. In: \emph{Arthur Schnitzler: Briefwechsel mit Autorinnen und Autoren}.
 Digitale Edition, https://schnitzler-briefe.acdh.oeaw.ac.at/{\dateiname}.html (Stand \today)
\fi

\end{document}


      