%% latex-korrekturansicht-vorspann.tex
%% Vorspann für die Korrekturansicht.
%% Lädt die gemeinsame Datei latex-vorspann.tex mit gesetztem Schalter.

\newif\ifkorrekturansicht
\korrekturansichttrue

\input{../tex-inputs/latex-vorspann}


\section[Arthur Schnitzler an Hermann Bahr, 18. 10. 1906]{L01634 Arthur Schnitzler an Hermann Bahr, 18. 10. 1906 }
\nopagebreak\mylabel{L01634v}
\rehead{ }\normalsize\beginnumbering\briefempfaengerindex{Bahr, Hermann@\textsc{Bahr, Hermann}!zzzSchnitzler, Arthur@\emph{von Arthur Schnitzler}!1906-10-181@{18. 10. 1906}|(be}
\toendnotes[C]{\smallbreak\pagebreak[2]}\Standort{TMW, HS AM 23383 Ba.}
\physDesc{Brief, 1 Blatt, 2 Seiten, 1287 Zeichen
\newline{}Handschrift: schwarze Tinte, deutsche Kurrent
\newline{}Ordnung: Lochung }
\buchAbdrucke{\weitereDrucke{1) Arthur Schnitzler: \emph{The Letters of Arthur Schnitzler to Hermann Bahr}. Chapel Hill: \emph{The University of North Carolina Press} 1978, S. 95–96.} \weitereDrucke{2) Hermann Bahr, Arthur Schnitzler: \emph{Briefwechsel, Aufzeichnungen, Dokumente (1891–1931)}. Göttingen: \emph{Wallstein} 2018, S. 383–384.} }\toendnotes[C]{\smallbreak}
\pstart
           \raggedleft{}{\pb}Wien\oindex{Wien@\textbf{Wien}, \emph{A.ADM2}|pw}, 18. X. 906\pend
           
\pstart{}lieber Hermann, \pend\vspace{0.5em}
\pstart
           eine Aehnlichkeit zwiſchen deinem
                  Akt\pwindex{tiefe Natur. Ein Akt@\emph{Die tiefe Natur. Ein Akt}|pwv} und dem Abſchiedſouper\pwindex{Abschiedssouper@\emph{Abschiedssouper}|pw} wäre
               höchſtens irgendwo im äußerlich ſtofflichen zu finden, im innerlich stofflichen ſchon
               nicht mehr, und gewiſs nicht im \strikeout{eigentlich} »ſeeliſch
               geſtaltlichen« – \introOben{}(\introOben{}um zu i{\geminationm}er
               grauenhafteren Worten auf- oder niederzuſteigen). Dein Problem ist viel verzwickter,
               der Fortgang der Handlung gedrehter, ſpiraliger, jüdiſcher gegenüber der naiv \textsc{gauloisen}\oindex{Frankreich@\textbf{Frankreich}, \emph{A.PCLI}|pw} Fabel des braven alten Anatolſtückls\pwindex{Anatol@\emph{Anatol}|pwv}, außerdem wird bei mir ſoupirt und bei dir doch eigentlich nur
               \label{K_L01634-1v}\edtext{gejauſnet}{\lemma{\textnormal{\emph{gejauſnet}}}\Cendnote{\textnormal{Jause, österreichisch: Zwischenmahlzeit}}}\label{K_L01634-1}. Die Atmosphäre
               deines Stücks ist dünner, ſchärfer; das ganze brutaler (für {\pb}meinen Geſchmack im
               Beginn beſonders bis zum Abſtoßenden brutal) angepackt. Wenn du mir, oder dem guten
                  Anatol\pwindex{Anatol@\emph{Anatol}|pwv}, dieſen intereſſanten Einakter\pwindex{tiefe Natur. Ein Akt@\emph{Die tiefe Natur. Ein Akt}|pwv} widmen
               willſt, ſo nehm ich s natürlich mit Dank u Rührung an, nur mußt du mir erlauben,
               deine Erinnerung nicht als Anregungsqui\damage{tt}irung und Ausdruck einer Gewiſſensſchuld ſondern als ein neues und daher mir
                  willko{\geminationm}enes Zeichen unſerer guten Zuſa{\geminationm}engehörigkeit zu empfinden u zu empfangen.\pend
           
\pstart
           Hoffentlich fügt es ſich, dſs wir einander vor deiner Abreiſe noch einmal ſehen.
               (Gern möcht ich auch etwas, \label{K_L01634-2v}\edtext{\textsc{Reinhardt}\pwindex{Reinhardt, Max 09.09.1873 – 30.10.1943@\textsc{Reinhardt, Max} (09.09.1873 – 30.10.1943), \emph{Theaterleiter/Theaterleiterin, Regisseur/Regisseurin, Schauspieler/Schauspielerin}|pw} betreffendes}{\lemma{\textnormal{\emph{Reinhardt betreffendes}}}\Cendnote{\textnormal{Eine Aufführung von
                     \emph{Der Schleier der Beatrice}\pwindex{Schleier der Beatrice. Schauspiel in fuenf Akten@\emph{Der Schleier der Beatrice. Schauspiel in fünf Akten}|pwk}, vgl. A. S.: \emph{Tagebuch}, 29. 10. 1906 und vgl.
                     den Brief von Schnitzler an Max Reinhardt\pwindex{Reinhardt, Max 09.09.1873 – 30.10.1943@\textsc{Reinhardt, Max} (09.09.1873 – 30.10.1943), \emph{Theaterleiter/Theaterleiterin, Regisseur/Regisseurin, Schauspieler/Schauspielerin}|pwk}, 24. 12. 1909
                        in A. S. \emph{Briefe 1875–1912}, S. 613–621.}}}\label{K_L01634-2}, aber
               hauptſächlich in \damage{mei}nem Intereſſe liegendes\strikeout{)} mit dir
               beſprechen.)\pend
           
\pstart
           Herzlichſt, mit Grüßen von{\\[\baselineskip]}meiner Frau\pwindex{Schnitzler, Olga 17.01.1882 – 13.01.1970@\textsc{Schnitzler, Olga} (17.01.1882 – 13.01.1970), \emph{Schauspieler/Schauspielerin, Sänger/Sängerin}|pwv} u mir{\\[\baselineskip]}dein{\\[\baselineskip]}\spacefill\mbox{Arthur}\pend
           \leftskip=0em{}\selectlanguage{ngerman}\endnumbering\briefempfaengerindex{Bahr, Hermann@\textsc{Bahr, Hermann}!zzzSchnitzler, Arthur@\emph{von Arthur Schnitzler}!1906-10-181@{18. 10. 1906}|)be}\mylabel{L01634h}  \normalsize

\doendnotes{C}
\bigskip
\vfill

\clearpage

\footnotesize

\lohead{\textsc{register}}

% Definiere theindex-Environment komplett neu ohne reledmac
\makeatletter
\renewenvironment{theindex}{%
  \section*{\indexname}%
  \setlength{\parindent}{0pt}%
  \setlength{\parskip}{0pt plus 0.3pt}%
  \let\item\@idxitem
}{%
  \clearpage
}
\makeatother

\IfFileExists{\jobname-pw.ind}{\input{\jobname-pw.ind}}{}

\end{document}

      