%% latex-leseansicht-vorspann.tex
%% Vorspann für die Leseansicht.
%% Lädt die gemeinsame Datei latex-vorspann.tex mit nicht gesetztem Schalter.

\newif\ifkorrekturansicht
\korrekturansichtfalse

\input{../tex-inputs/latex-vorspann}


         
         \renewcommand{\erwaehntePersonen}{Personen: Otto Brahm}
         \renewcommand{\erwaehnteInstitutionen}{Institutionen: Überetscher Bahn}
         \renewcommand{\erwaehnteOrte}{Orte: Bozen, Caldaro sulla strada del vino, Eppaner Hof, Mendelpass, Sankt Michael, Vahrn}
         \renewcommand{\erwaehnteWerke}{Werke: Der Graf von Charolais. Ein Trauerspiel}
               \section[Richard Beer-Hofmann an Arthur Schnitzler, 3. 10. 1899]{ Richard Beer-Hofmann an Arthur Schnitzler, 3. 10. 1899}\nopagebreak\mylabel{v}\rehead{ }\begin{ledgroupsized}[t]{13cm}\normalsize\beginnumbering \toendnotes[C]{\smallbreak\pagebreak[2]} \Standort{CUL, Schnitzler, B 8.}
\physDesc{Brief, 1 Blatt, 2 Seiten, 1157 Zeichen
\newline{}Handschrift: schwarze Tinte, lateinische Kurrent
\newline{}Ordnung: mit Bleistift von unbekannter Hand nummeriert:
                                    »143« }\buchAbdrucke{\weitereDrucke{Arthur Schnitzler, Richard Beer-Hofmann: \emph{Briefwechsel 1891–1931}. Hg. Konstanze Fliedl. Wien, Zürich: \emph{Europaverlag} 1992, S. 138–139.} }\toendnotes[C]{\smallbreak}\pstart
           \raggedleft{}{\pb}St. Michael in Eppan\oindex{Sankt Michael@\textbf{Sankt Michael}|pw}{ }3 X 1899\pend
           \pstart
           Lieber Arthur 1.) Von Vahrn\oindex{Vahrn@\textbf{Vahrn}|pw} bin
               ich fort weil es in dieser Höhe circa 670\textsuperscript{m} schon zu kühl
               ist.\pend
           \pstart
           2.) Dieses St. Michael\oindex{Sankt Michael@\textbf{Sankt Michael}|pw} liegt an der heuer
               eröffneten Überetscher Bahn\orgindex{Ueberetscher Bahn@Überetscher Bahn|pw} – Bozen\oindex{Bozen@\textbf{Bozen}|pw} – Kaltern\oindex{Caldaro sulla strada del vino@\textbf{Caldaro sulla strada del vino}|pw} –, nur
               eine Wagenstunde von Bozen\oindex{Bozen@\textbf{Bozen}|pw}. Meistens ko{\geminationm}en hier nur die Leute die auf die Mendel\oindex{Mendelpass@\textbf{Mendelpass}|pw} fahren durch; ständig wohnen hier wenig Fremde. In
               unserem »Hôtel\oindex{Eppaner Hof@\textbf{Eppaner Hof}|pw}« außer uns Niemand. 3.) Auf die
               Idee hieherzuko{\geminationm}en hat mich ein Eisenbahnplakat
               gebracht. 4.) Ich dürfte nicht länger als 2 Wochen noch hierbleiben. \substVorne{}\textsuperscript{4}\substDazwischen{}5\substHinten{}.) Ich bin im I Akt\pwindex{Beer-Hofmann, Richard 1866-07-11 – 1945-09-26@\textsc{Beer-Hofmann, Richard} (1866-07-11 – 1945-09-26), \emph{Schriftsteller}!Graf von Charolais. Ein Trauerspiel1904-12-23@\strich\emph{Der Graf von Charolais. Ein Trauerspiel} {[}1904-12-23{]}|pwv}
               (der drei Abtheilungen hat) in der ersten Abtheilung im 5ten Versehundert. 433 Verse\pwindex{Beer-Hofmann, Richard 1866-07-11 – 1945-09-26@\textsc{Beer-Hofmann, Richard} (1866-07-11 – 1945-09-26), \emph{Schriftsteller}!Graf von Charolais. Ein Trauerspiel1904-12-23@\strich\emph{Der Graf von Charolais. Ein Trauerspiel} {[}1904-12-23{]}|pwv} hats gebraucht bis
               ich den Helden auf die Bühne gelassen habe. \substVorne{}\textsuperscript{5}\substDazwischen{}6\substHinten{}.) Meine Laune wäre besser {\pb}wenn ich mehr schlafen würde. Im übrigen hängt sie von der Arbeit ab. Viele Verse –
               gute Laune; wenig Verse – schlechte Laune. O Gott! Was wird mir nicht Alles
               gestrichen werden. »Die Brillanten werden sie mer stehn lassen«! Antworten sie
               höflich: »Also Alles«!.\pend
           \pstart
            Ich grüße Sie herzlich{\\[\baselineskip]}Ihr \spacefill\mbox{Richard}\pend
           \leftskip=0em{}\pstart
           \noindent{}Grüßen Sie Brahm\pwindex{Brahm, Otto 05.02.1856 – 28.11.1912@\textsc{Brahm, Otto} (05.02.1856 – 28.11.1912), \emph{Theaterleiter, Regisseur}|pw} und Kerr. Dem Brahm\pwindex{Brahm, Otto 05.02.1856 – 28.11.1912@\textsc{Brahm, Otto} (05.02.1856 – 28.11.1912), \emph{Theaterleiter, Regisseur}|pw} bringen Sie um Gotteswillen keine
                  bessere Meinung von mir bei! Bis auf Weiteres laßen Sie mich für ihn »Ein Herr mit
                  einem Monocle« sein.\pend
           
         
         \endnumbering\mylabel{h}\end{ledgroupsized}  \newcommand{\dateiname}{L00988}\newcommand{\titel}{Richard Beer-Hofmann an Arthur Schnitzler, 3. 10. 1899}\newcommand{\editorInnen}{Martin Anton Müller und Gerd-Hermann Susen}%% latex-leseansicht-abspann.tex
%% Abspann für die Leseansicht.
%% Der Schalter \ifkorrekturansicht ist bereits durch den Vorspann gesetzt.

%% latex-abspann.tex
%% Gemeinsamer Abspann für Korrekturansicht und Leseansicht.
%% Setzt den Schalter \ifkorrekturansicht voraus (gesetzt in den
%% einbindenden Dateien latex-korrekturansicht-abspann.tex bzw.
%% latex-leseansicht-abspann.tex).
%% ---------------------------------------------------------------

\normalsize

% Das esempio-Environment wird nur in der Leseansicht benötigt
\ifkorrekturansicht\else
\newenvironment{esempio}[3]%
{
    \vspace{1.5ex}
    \rlap{\underline{#1}}
    \par
    \setlength{\parindent}{0cm}
    \nopagebreak
    \leftskip=#2cm
    \rightskip=#3cm
}
{
    \par
}
\fi

\doendnotes{C}
\bigskip
\vfill

\clearpage

\footnotesize

\ifkorrekturansicht
  \lohead{\textsc{register}}
\fi

% theindex-Environment neu definieren ohne reledmac
\makeatletter
\renewenvironment{theindex}{%
  \ifkorrekturansicht
    \section*{\indexname}%
  \else
    \subsubsection*{Index der erwähnten Entitäten}%
  \fi
  \setlength{\parindent}{0pt}%
  \setlength{\parskip}{0pt plus 0.3pt}%
  \let\item\@idxitem
}{%
  \ifkorrekturansicht\clearpage\fi
}
\makeatother

\IfFileExists{\jobname-pw.ind}{\input{\jobname-pw.ind}}{}

% Quellenangabe nur in der Leseansicht
\ifkorrekturansicht\else
% Fallback-Definitionen, falls die .tex-Datei \titel etc. nicht gesetzt hat
\providecommand{\titel}{}
\providecommand{\editorInnen}{}
\providecommand{\dateiname}{\jobname}

\vspace{3cm}

\vfill

\footnotesize
\textsc{Quelle}: \titel. Herausgegeben von {\editorInnen}. In: \emph{Arthur Schnitzler: Briefwechsel mit Autorinnen und Autoren}.
 Digitale Edition, https://schnitzler-briefe.acdh.oeaw.ac.at/{\dateiname}.html (Stand \today)
\fi

\end{document}


      