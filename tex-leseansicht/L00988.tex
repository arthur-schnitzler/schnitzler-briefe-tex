%% latex-korrekturansicht-vorspann.tex
%% Vorspann für die Korrekturansicht.
%% Lädt die gemeinsame Datei latex-vorspann.tex mit gesetztem Schalter.

\newif\ifkorrekturansicht
\korrekturansichttrue

\input{../tex-inputs/latex-vorspann}


\section[Richard Beer-Hofmann an Arthur Schnitzler, 3. 10. 1899]{L00988 Richard Beer-Hofmann an Arthur Schnitzler, 3. 10. 1899}
\nopagebreak\mylabel{L00988v}
\rehead{ }\normalsize\beginnumbering\briefempfaengerindex{Schnitzler, Arthur@\textsc{Schnitzler, Arthur}!zzzBeer-Hofmann, Richard@\emph{von Richard Beer-Hofmann}!1899-10-032@{3. 10. 1899}|(be}
\toendnotes[C]{\smallbreak\pagebreak[2]}\Standort{CUL, Schnitzler, B 8.}
\physDesc{Brief, 1 Blatt, 2 Seiten, 1157 Zeichen
\newline{}Handschrift: schwarze Tinte, lateinische Kurrent
\newline{}Ordnung: mit Bleistift von unbekannter Hand nummeriert:
                                    »143« }
\buchAbdrucke{\weitereDrucke{Arthur Schnitzler, Richard Beer-Hofmann: \emph{Briefwechsel 1891–1931}. Wien, Zürich: \emph{Europaverlag} 1992, S. 138–139.} }\toendnotes[C]{\smallbreak}
\pstart
           \raggedleft{}{\pb}St. Michael in Eppan\oindex{Sankt Michael@\textbf{Sankt Michael}, \emph{Bezirk (A.BZK)}|pw}{ }3 X 1899\pend
           \vspace{0.5em}
\pstart
           Lieber Arthur 1.) Von Vahrn\oindex{Vahrn@\textbf{Vahrn}, \emph{P.PPLA3}|pw} bin
               ich fort weil es in dieser Höhe circa 670\textsuperscript{m} schon zu kühl
               ist.\pend
           
\pstart
           2.) Dieses St. Michael\oindex{Sankt Michael@\textbf{Sankt Michael}, \emph{Bezirk (A.BZK)}|pw} liegt an der heuer
               eröffneten Überetscher Bahn\orgindex{Ueberetscher Bahn@Überetscher Bahn|pw} – Bozen\oindex{Bozen@\textbf{Bozen}, \emph{P.PPLA2}|pw} – Kaltern\oindex{Caldaro sulla strada del vino@\textbf{Caldaro sulla strada del vino}, \emph{A.ADM3}|pw} –, nur
               eine Wagenstunde von Bozen\oindex{Bozen@\textbf{Bozen}, \emph{P.PPLA2}|pw}. Meistens ko{\geminationm}en hier nur die Leute die auf die Mendel\oindex{Mendelpass@\textbf{Mendelpass}, \emph{Pass (N.PAS)}|pw} fahren durch; ständig wohnen hier wenig Fremde. In
               unserem »Hôtel\oindex{Eppaner Hof@\textbf{Eppaner Hof}, \emph{Gastgewerbegebäude (K.GGW)}|pw}« außer uns Niemand. 3.) Auf die
               Idee hieherzuko{\geminationm}en hat mich ein Eisenbahnplakat
               gebracht. 4.) Ich dürfte nicht länger als 2 Wochen noch hierbleiben. \substVorne{}\textsuperscript{4}\substDazwischen{}5\substHinten{}.) Ich bin im I Akt\pwindex{Graf von Charolais. Ein Trauerspiel@\emph{Der Graf von Charolais. Ein Trauerspiel}|pwv}
               (der drei Abtheilungen hat) in der ersten Abtheilung im 5ten Versehundert. 433 Verse\pwindex{Graf von Charolais. Ein Trauerspiel@\emph{Der Graf von Charolais. Ein Trauerspiel}|pwv} hats gebraucht bis
               ich den Helden auf die Bühne gelassen habe. \substVorne{}\textsuperscript{5}\substDazwischen{}6\substHinten{}.) Meine Laune wäre besser {\pb}wenn ich mehr schlafen würde. Im übrigen hängt sie von der Arbeit ab. Viele Verse –
               gute Laune; wenig Verse – schlechte Laune. O Gott! Was wird mir nicht Alles
               gestrichen werden. »Die Brillanten werden sie mer stehn lassen«! Antworten sie
               höflich: »Also Alles«!.\pend
           
\pstart
            Ich grüße Sie herzlich{\\[\baselineskip]}Ihr \spacefill\mbox{Richard}\pend
           \leftskip=0em{}
\pstart
           \noindent{}Grüßen Sie Brahm\pwindex{Brahm, Otto 05.02.1856 – 28.11.1912@\textsc{Brahm, Otto} (05.02.1856 – 28.11.1912), \emph{Theaterleiter/Theaterleiterin, Regisseur/Regisseurin}|pw} und Kerr. Dem Brahm\pwindex{Brahm, Otto 05.02.1856 – 28.11.1912@\textsc{Brahm, Otto} (05.02.1856 – 28.11.1912), \emph{Theaterleiter/Theaterleiterin, Regisseur/Regisseurin}|pw} bringen Sie um Gotteswillen keine
                  bessere Meinung von mir bei! Bis auf Weiteres laßen Sie mich für ihn »Ein Herr mit
                  einem Monocle« sein.\pend
           \selectlanguage{ngerman}\endnumbering\briefempfaengerindex{Schnitzler, Arthur@\textsc{Schnitzler, Arthur}!zzzBeer-Hofmann, Richard@\emph{von Richard Beer-Hofmann}!1899-10-032@{3. 10. 1899}|)be}\mylabel{L00988h}  \normalsize

\doendnotes{C}
\bigskip
\vfill

\clearpage

\footnotesize

\lohead{\textsc{register}}

% Definiere theindex-Environment komplett neu ohne reledmac
\makeatletter
\renewenvironment{theindex}{%
  \section*{\indexname}%
  \setlength{\parindent}{0pt}%
  \setlength{\parskip}{0pt plus 0.3pt}%
  \let\item\@idxitem
}{%
  \clearpage
}
\makeatother

\IfFileExists{\jobname-pw.ind}{\input{\jobname-pw.ind}}{}

\end{document}

      