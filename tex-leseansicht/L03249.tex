%% latex-korrekturansicht-vorspann.tex
%% Vorspann für die Korrekturansicht.
%% Lädt die gemeinsame Datei latex-vorspann.tex mit gesetztem Schalter.

\newif\ifkorrekturansicht
\korrekturansichttrue

\input{../tex-inputs/latex-vorspann}


\section[ Paul Goldmann an Arthur Schnitzler, 3. 8. 1906]{L03249 Paul Goldmann an Arthur Schnitzler, 3. 8. 1906}
\nopagebreak\mylabel{L03249v}
\rehead{ }\normalsize\beginnumbering\briefempfaengerindex{Schnitzler, Arthur@\textsc{Schnitzler, Arthur}!zzzGoldmann, Paul@\emph{von Paul Goldmann}!1906-08-033@{3. 8. 1906}|(be}
\toendnotes[C]{\smallbreak\pagebreak[2]}\Standort{DLA, A:Schnitzler, HS.NZ85.1.3175.}
\physDesc{Postkarte, 455 Zeichen
\newline{}Handschrift: 1) blaue Tinte, deutsche Kurrent\hspace{1em}2) blaue Tinte, lateinische Kurrent (\noindent{}Adresse)\hspace{1em}
\newline{}Versand: 1) Stempel: »\nobreak{}\oindex{Berlin@\textbf{Berlin}, \emph{P.PPLC}|pwk}Berlin, S.W. 11, 3. 8. 06, 6–7N.\nobreak{}«.   2) Stempel: »\nobreak{}\oindex{Helsingør@\textbf{Helsingør}, \emph{P.PPLA2}|pwk}Helsingør, 4. 8. 06, 11–12F\nobreak{}«. 
\newline{}Schnitzler: mit Bleistift das Jahr »906« vermerkt }\toendnotes[C]{\smallbreak}\pstart{}{\pb}Welt-\textcolor{gray}{\textbf{Poſtkarte}}\pend{}\pstart{}Herrn\pend{}\pstart{}Dr. Arthur Schnitzler\pend{}\pstart{}Marienlyst\oindex{Marienlyst@\textbf{Marienlyst}, \emph{S.EST}|pw} (Dänemark\oindex{Daenemark@\textbf{Dänemark}, \emph{A.PCLI}|pw})\pend{}{\bigskip}\vspace{1em}
\pstart
           \noindent{}{\pb}3. Auguſt. Mein lieber
                  Freund, Auch mich hat der Anblick des \textsc{Skodsborger Hotels\oindex{Badehotellet@\textbf{Badehotellet}, \emph{Hotel (K.HTL)}|pwv}} auf Deiner lieben \label{K_L03249-1v}\edtext{Karte}{\lemma{\textnormal{\emph{Karte}}}\Cendnote{\textnormal{Am 1. 8. 1906 machte Schnitzler einen Ausflug nach Skodsborg\oindex{Skodsborg@\textbf{Skodsborg}, \emph{P.PPL}|pwk}, wo er sich im August 1896 gemeinsam mit Goldmann\pwindex{Goldmann, Paul 31.01.1865 – 25.09.1935@\textsc{Goldmann, Paul} (31.01.1865 – 25.09.1935), \emph{Schriftsteller/Schriftstellerin, Journalist/Journalistin}|pwk}, Richard Beer-Hofmann\pwindex{Beer-Hofmann, Richard 1866-07-11 – 1945-09-26@\textsc{Beer-Hofmann, Richard} (1866-07-11 – 1945-09-26), \emph{Schriftsteller/Schriftstellerin}|pwk} und
                  dessen Partnerin Paula Lissy\pwindex{Beer-Hofmann, Paula 25.02.1879 – 30.10.1939@\textsc{Beer-Hofmann, Paula} (25.02.1879 – 30.10.1939)|pwk} im Badehotel\oindex{Badehotellet@\textbf{Badehotellet}, \emph{Hotel (K.HTL)}|pwk} aufgehalten hatte. Die nicht
                  erhaltene Karte an Goldmann\pwindex{Goldmann, Paul 31.01.1865 – 25.09.1935@\textsc{Goldmann, Paul} (31.01.1865 – 25.09.1935), \emph{Schriftsteller/Schriftstellerin, Journalist/Journalistin}|pwk} könnte das
                  gleiche Motiv gehabt haben wie die am 1. 8. 1906 an Beer-Hofmann\pwindex{Beer-Hofmann, Richard 1866-07-11 – 1945-09-26@\textsc{Beer-Hofmann, Richard} (1866-07-11 – 1945-09-26), \emph{Schriftsteller/Schriftstellerin}|pwk}
                  gesandte.}}}\label{K_L03249-1} recht wehmütig geſtimmt. Ja, ja, die zehn Jahre ſind fort, unwiederbringlich!
               Was aber das Talent anlangt, – glaubſt Du nicht, daß Du noch ein wenig übrig haſt?
               Oder ſollte ich Dich überſchätzen?\pend
           
\pstart
           Herzliche Grüße Dir u. Deiner Frau\pwindex{Schnitzler, Olga 17.01.1882 – 13.01.1970@\textsc{Schnitzler, Olga} (17.01.1882 – 13.01.1970), \emph{Schauspieler/Schauspielerin, Sänger/Sängerin}|pwv} u. herzlichen Dank für Dein freundliches Gedenken! {\\[\baselineskip]}Dein getreuer {\\[\baselineskip]}\spacefill\mbox{Paul Goldmann.}\pend
           \leftskip=0em{}\selectlanguage{ngerman}\endnumbering\briefempfaengerindex{Schnitzler, Arthur@\textsc{Schnitzler, Arthur}!zzzGoldmann, Paul@\emph{von Paul Goldmann}!1906-08-033@{3. 8. 1906}|)be}\mylabel{L03249h}  \normalsize

\doendnotes{C}
\bigskip
\vfill

\clearpage

\footnotesize

\lohead{\textsc{register}}

% Definiere theindex-Environment komplett neu ohne reledmac
\makeatletter
\renewenvironment{theindex}{%
  \section*{\indexname}%
  \setlength{\parindent}{0pt}%
  \setlength{\parskip}{0pt plus 0.3pt}%
  \let\item\@idxitem
}{%
  \clearpage
}
\makeatother

\IfFileExists{\jobname-pw.ind}{\input{\jobname-pw.ind}}{}

\end{document}

      