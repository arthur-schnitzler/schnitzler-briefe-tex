%% latex-leseansicht-vorspann.tex
%% Vorspann für die Leseansicht.
%% Lädt die gemeinsame Datei latex-vorspann.tex mit nicht gesetztem Schalter.

\newif\ifkorrekturansicht
\korrekturansichtfalse

\input{../tex-inputs/latex-vorspann}

\begin{center}
            \textcolor{red}{ENTWURF, NICHT FERTIG KORRIGIERT}
                      \end{center}
            
         
         \renewcommand{\erwaehntePersonen}{Personen: Paula Beer-Hofmann, Richard Beer-Hofmann, Olga Schnitzler}
         \renewcommand{\erwaehnteOrte}{Orte: Badehotel, Berlin, Dänemark, Helsingør, Marienlyst, Skodsborg}
         \renewcommand{\erwaehnteWerke}{}
               \section[Paul Goldmann an Arthur Schnitzler, 3. 8. 1906]{ Paul Goldmann an Arthur Schnitzler, 3. 8. 1906}\nopagebreak\mylabel{v}\rehead{ }\begin{ledgroupsized}[t]{13cm}\normalsize\beginnumbering \toendnotes[C]{\smallbreak\pagebreak[2]} \Standort{DLA, A:Schnitzler, HS.NZ85.1.3175.}
\physDesc{Postkarte
\newline{}Handschrift: 1) blaue Tinte, deutsche Kurrent\hspace{1em}2) blaue Tinte, lateinische Kurrent (\noindent{}Adresse)\hspace{1em}\newline{}Versand: 1) Stempel: »\nobreak{}\oindex{Berlin@\textbf{Berlin}|pwk}Berlin S.W., 3. 8. 06, 6–7 N\nobreak{}«.   2) Stempel: »\nobreak{}\oindex{Helsingør@\textbf{Helsingør}|pwk}Helsingør, 4. 8. 06, 11–12 F\nobreak{}«. 
\newline{}Schnitzler: mit Bleistift das Jahr »{[}1{]}906« vermerkt }\toendnotes[C]{\smallbreak}\pstart{}{\pb}Welt-\textcolor{gray}{\textbf{Poſtkarte}}\pend{}\pstart{}Herrn\pend{}\pstart{}Dr. Arthur Schnitzler \pend{}\pstart{}Marienlyst\oindex{Marienlyst@\textbf{Marienlyst}|pw} (Dänemark\oindex{Daenemark@\textbf{Dänemark}|pw}) \pend{}{\bigskip}\pstart
           \noindent{}{\pb}3. August. Mein lieber
                  Freund, Auch mich hat der Anblick des \textsc{Skodsborger Hotel\oindex{Badehotel@\textbf{Badehotel}|pw}s} auf Deiner lieben
                  \label{K-L03249-1v}\edtext{Karte}{\lemma{\textnormal{\emph{Karte}}}\Cendnote{\textnormal{Siehe A. S.: \emph{Tagebuch}, 1. 8. 1906}}}\label{K-L03249-1h} recht wehmütig geſtimmt. Ja, ja, die \label{K-L03249-2v}\edtext{zehn Jahre}{\lemma{\textnormal{\emph{zehn Jahre}}}\Cendnote{\textnormal{Im August 1896 hatten sich
                     Schnitzler\pwindex{Schnitzler, Arthur 15.05.1862 – 21.10.1931@\textsc{Schnitzler, Arthur} (15.05.1862 – 21.10.1931), \emph{Schriftsteller, Mediziner}|pwk} und Goldmann\pwindex{Goldmann, Paul 31.01.1865 – 25.09.1935@\textsc{Goldmann, Paul} (31.01.1865 – 25.09.1935), \emph{Schriftsteller, Journalist}|pwk}, gemeinsam mit Paula\pwindex{Beer-Hofmann, Paula 25.02.1879 – 30.10.1939@\textsc{Beer-Hofmann, Paula} (25.02.1879 – 30.10.1939)|pwk} und Richard Beer-Hofmann\pwindex{Beer-Hofmann, Richard 1866-07-11 – 1945-09-26@\textsc{Beer-Hofmann, Richard} (1866-07-11 – 1945-09-26), \emph{Schriftsteller}|pwk}, im
                     Badehotel\oindex{Badehotel@\textbf{Badehotel}|pwk} in Skodsborg\oindex{Skodsborg@\textbf{Skodsborg}|pwk} aufgehalten. Schnitzler\pwindex{Schnitzler, Arthur 15.05.1862 – 21.10.1931@\textsc{Schnitzler, Arthur} (15.05.1862 – 21.10.1931), \emph{Schriftsteller, Mediziner}|pwk} schrieb auch an Richard
                     Beer-Hofmann\pwindex{Beer-Hofmann, Richard 1866-07-11 – 1945-09-26@\textsc{Beer-Hofmann, Richard} (1866-07-11 – 1945-09-26), \emph{Schriftsteller}|pwk} eine Karte, in der er an den früheren gemeinsamen Aufenthalt
                  erinnert, siehe Arthur Schnitzler an Richard Beer-Hofmann, 1. 8. 1906.}}}\label{K-L03249-2h}
               ſind fort, unwiederbringlich! Was aber das Talent anlangt, – glaubſt Du nicht, daß Du
               noch ein wenig übrig haſt? Oder ſollte ich Dich überſchätzen?\pend
           \pstart
           Herzliche Grüße Dir u. Deiner Frau\pwindex{Schnitzler, Olga 17.01.1882 – 13.01.1970@\textsc{Schnitzler, Olga} (17.01.1882 – 13.01.1970), \emph{Schauspielerin, Sängerin}|pwv} u. herzlichen Dank für Dein freundliches Gedenken! {\\[\baselineskip]}Dein getreuer
                  {\\[\baselineskip]}\spacefill\mbox{Paul Goldmann. }\pend
           \leftskip=0em{}
         
         \endnumbering\mylabel{h}\end{ledgroupsized}\begin{anhang}\end{anhang}\newcommand{\dateiname}{L03249}\newcommand{\titel}{Paul Goldmann an Arthur Schnitzler, 3. 8. 1906}\newcommand{\editorInnen}{Martin Anton Müller und Laura Untner}%% latex-leseansicht-abspann.tex
%% Abspann für die Leseansicht.
%% Der Schalter \ifkorrekturansicht ist bereits durch den Vorspann gesetzt.

%% latex-abspann.tex
%% Gemeinsamer Abspann für Korrekturansicht und Leseansicht.
%% Setzt den Schalter \ifkorrekturansicht voraus (gesetzt in den
%% einbindenden Dateien latex-korrekturansicht-abspann.tex bzw.
%% latex-leseansicht-abspann.tex).
%% ---------------------------------------------------------------

\normalsize

% Das esempio-Environment wird nur in der Leseansicht benötigt
\ifkorrekturansicht\else
\newenvironment{esempio}[3]%
{
    \vspace{1.5ex}
    \rlap{\underline{#1}}
    \par
    \setlength{\parindent}{0cm}
    \nopagebreak
    \leftskip=#2cm
    \rightskip=#3cm
}
{
    \par
}
\fi

\doendnotes{C}
\bigskip
\vfill

\clearpage

\footnotesize

\ifkorrekturansicht
  \lohead{\textsc{register}}
\fi

% theindex-Environment neu definieren ohne reledmac
\makeatletter
\renewenvironment{theindex}{%
  \ifkorrekturansicht
    \section*{\indexname}%
  \else
    \subsubsection*{Index der erwähnten Entitäten}%
  \fi
  \setlength{\parindent}{0pt}%
  \setlength{\parskip}{0pt plus 0.3pt}%
  \let\item\@idxitem
}{%
  \ifkorrekturansicht\clearpage\fi
}
\makeatother

\IfFileExists{\jobname-pw.ind}{\input{\jobname-pw.ind}}{}

% Quellenangabe nur in der Leseansicht
\ifkorrekturansicht\else
% Fallback-Definitionen, falls die .tex-Datei \titel etc. nicht gesetzt hat
\providecommand{\titel}{}
\providecommand{\editorInnen}{}
\providecommand{\dateiname}{\jobname}

\vspace{3cm}

\vfill

\footnotesize
\textsc{Quelle}: \titel. Herausgegeben von {\editorInnen}. In: \emph{Arthur Schnitzler: Briefwechsel mit Autorinnen und Autoren}.
 Digitale Edition, https://schnitzler-briefe.acdh.oeaw.ac.at/{\dateiname}.html (Stand \today)
\fi

\end{document}


      