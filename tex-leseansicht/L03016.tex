%% latex-leseansicht-vorspann.tex
%% Vorspann für die Leseansicht.
%% Lädt die gemeinsame Datei latex-vorspann.tex mit nicht gesetztem Schalter.

\newif\ifkorrekturansicht
\korrekturansichtfalse

\input{../tex-inputs/latex-vorspann}

\begin{center}
            \textcolor{red}{ENTWURF, NICHT FERTIG KORRIGIERT}
                      \end{center}
            
         
         \renewcommand{\erwaehntePersonen}{Personen: Felix Salten}
         \renewcommand{\erwaehnteOrte}{Orte: Attersee, Berghof, Semmering, Unterach am Attersee, Wien}
         \renewcommand{\erwaehnteWerke}{Werke: Die Zeit, Künstler sollen reden}
               \section[Arthur Schnitzler an Felix Salten, 27. 6. 1910]{ Arthur Schnitzler an Felix Salten, 27. 6. 1910}\nopagebreak\mylabel{v}\rehead{ }\begin{ledgroupsized}[t]{13cm}\normalsize\beginnumbering \toendnotes[C]{\smallbreak\pagebreak[2]} \Standort{Wienbibliothek im Rathaus, ZPH 1681, 2.1.516.}
\physDesc{
\newline{}Handschrift: , deutsche Kurrent}\toendnotes[C]{\smallbreak}\pstart{}{\pb}Hrn Felix Salten\pend{}\pstart{}Unterach\oindex{Unterach am Attersee@\textbf{Unterach am Attersee}|pw}\pend{}\pstart{}am Attersee\oindex{Attersee@\textbf{Attersee}|pw}\pend{}\pstart{}Berghof\oindex{Berghof@\textbf{Berghof}|pw}\pend{}{\bigskip}\pstart
           \noindent{}{\pb}lieber, ich glaube nicht, dſs wir vor Ende
                  Juli werden \textcolor{gray}{ü}berſiedeln kö{\geminationn}en, Anfang Juli gehn wir für ein paar Tage auf den Se{\geminationm}ering\oindex{Semmering@\textbf{Semmering}|pw}.– \pend
           \pstart
           Ich \label{K_L03016-1v}\edtext{geſtriges \textsc{Feu{[}i{]}lleton\pwindex{Salten, Felix 06.09.1869 – 08.10.1945@\textsc{Salten, Felix} (06.09.1869 – 08.10.1945), \emph{Schriftsteller, Journalist}!Kuenstler sollen reden1910-06-26@\strich\emph{Künstler sollen reden} {[}1910-06-26{]}|pw}}{\lemma{\textnormal{\emph{geſtriges Feuilleton}}}\Cendnote{\textnormal{Felix Salten\pwindex{Salten, Felix 06.09.1869 – 08.10.1945@\textsc{Salten, Felix} (06.09.1869 – 08.10.1945), \emph{Schriftsteller, Journalist}|pwk}: \emph{Künstler sollen
                           reden}\pwindex{Salten, Felix 06.09.1869 – 08.10.1945@\textsc{Salten, Felix} (06.09.1869 – 08.10.1945), \emph{Schriftsteller, Journalist}!Kuenstler sollen reden1910-06-26@\strich\emph{Künstler sollen reden} {[}1910-06-26{]}|pwk}. In: \emph{Die Zeit}\pwindex{Zeit1902 – 1919@\emph{Die Zeit} {[}1902 – 1919{]}|pwk},
                        Jg. 9, Nr. 2.784, 26. 6. 1910, Morgenblatt,
                        S. 1–2.}}}\label{K_L03016-1h}} – köſtlich!– Eins von denen, aus deren Tiefe es noch ſchöner glitzerte als auf
               der Fläche oben. Die wahrhaftig auch nicht ohne iſt. \pend
           \pstart
           Viele Grüße von uns zu Ihnen. {\\[\baselineskip]}Herzlichſt Ihr {\\[\baselineskip]}\spacefill\mbox{A.}\pend
           \leftskip=0em{}\pstart
           \raggedleft{}27. 6. 10\pend
           
         
         \endnumbering\mylabel{h}\end{ledgroupsized}\begin{anhang}\end{anhang}\newcommand{\dateiname}{L03016}\newcommand{\titel}{Arthur Schnitzler an Felix Salten, 27. 6. 1910}\newcommand{\editorInnen}{Martin Anton Müller und Laura Untner}%% latex-leseansicht-abspann.tex
%% Abspann für die Leseansicht.
%% Der Schalter \ifkorrekturansicht ist bereits durch den Vorspann gesetzt.

%% latex-abspann.tex
%% Gemeinsamer Abspann für Korrekturansicht und Leseansicht.
%% Setzt den Schalter \ifkorrekturansicht voraus (gesetzt in den
%% einbindenden Dateien latex-korrekturansicht-abspann.tex bzw.
%% latex-leseansicht-abspann.tex).
%% ---------------------------------------------------------------

\normalsize

% Das esempio-Environment wird nur in der Leseansicht benötigt
\ifkorrekturansicht\else
\newenvironment{esempio}[3]%
{
    \vspace{1.5ex}
    \rlap{\underline{#1}}
    \par
    \setlength{\parindent}{0cm}
    \nopagebreak
    \leftskip=#2cm
    \rightskip=#3cm
}
{
    \par
}
\fi

\doendnotes{C}
\bigskip
\vfill

\clearpage

\footnotesize

\ifkorrekturansicht
  \lohead{\textsc{register}}
\fi

% theindex-Environment neu definieren ohne reledmac
\makeatletter
\renewenvironment{theindex}{%
  \ifkorrekturansicht
    \section*{\indexname}%
  \else
    \subsubsection*{Index der erwähnten Entitäten}%
  \fi
  \setlength{\parindent}{0pt}%
  \setlength{\parskip}{0pt plus 0.3pt}%
  \let\item\@idxitem
}{%
  \ifkorrekturansicht\clearpage\fi
}
\makeatother

\IfFileExists{\jobname-pw.ind}{\input{\jobname-pw.ind}}{}

% Quellenangabe nur in der Leseansicht
\ifkorrekturansicht\else
% Fallback-Definitionen, falls die .tex-Datei \titel etc. nicht gesetzt hat
\providecommand{\titel}{}
\providecommand{\editorInnen}{}
\providecommand{\dateiname}{\jobname}

\vspace{3cm}

\vfill

\footnotesize
\textsc{Quelle}: \titel. Herausgegeben von {\editorInnen}. In: \emph{Arthur Schnitzler: Briefwechsel mit Autorinnen und Autoren}.
 Digitale Edition, https://schnitzler-briefe.acdh.oeaw.ac.at/{\dateiname}.html (Stand \today)
\fi

\end{document}


      