%% latex-korrekturansicht-vorspann.tex
%% Vorspann für die Korrekturansicht.
%% Lädt die gemeinsame Datei latex-vorspann.tex mit gesetztem Schalter.

\newif\ifkorrekturansicht
\korrekturansichttrue

\input{../tex-inputs/latex-vorspann}


\section[ Arthur Schnitzler an Felix Salten, 27. 6. 1910]{L03016 Arthur Schnitzler an Felix Salten, 27. 6. 1910}
\nopagebreak\mylabel{L03016v}
\rehead{ }\normalsize\beginnumbering\briefempfaengerindex{Salten, Felix@\textsc{Salten, Felix}!zzzSchnitzler, Arthur@\emph{von Arthur Schnitzler}!1910-06-271@{27. 6. 1910}|(be}
\toendnotes[C]{\smallbreak\pagebreak[2]}\Standort{Wienbibliothek im Rathaus, ZPH 1681, 2.1.516.}
\physDesc{Postkarte, 374 Zeichen
\newline{}Handschrift: 1) Bleistift, deutsche Kurrent\hspace{1em}2) Bleistift, lateinische Kurrent (\noindent{}Adresse)\hspace{1em}
\newline{}Versand: Stempel: »\nobreak{}\oindex{XVIII., Waehring@\textbf{XVIII., Währing}, \emph{A.ADM3}|pwk}1\textcolor{gray}{8}/\textsubscript{1} Wien 110, 27. VI. 10, 9 \textcolor{gray}{V}\nobreak{}«.  
\newline{}Ordnung: mit Bleistift von unbekannter Hand nummeriert: »3« }\toendnotes[C]{\smallbreak}\pstart{}{\pb}Hrn Felix Salten\pend{}\pstart{}Unterach\oindex{Unterach am Attersee@\textbf{Unterach am Attersee}, \emph{P.PPL}|pw}\pend{}\pstart{}am Attersee\oindex{Attersee@\textbf{Attersee}, \emph{H.LK}|pw}\pend{}\pstart{}Berghof\oindex{Berghof@\textbf{Berghof}, \emph{Wohngebäude (K.WHS)}|pw}.\pend{}{\bigskip}\vspace{1em}
\pstart
           \noindent{}{\pb}lieber, ich glaube nicht, dſs wir vor Ende
                  Juli werden \label{K_L03016-1v}\edtext{\textcolor{gray}{ü}berſiedeln}{\lemma{\textnormal{\emph{überſiedeln}}}\Cendnote{\textnormal{Der
                  Umzug in die Sternwartestraße 71\oindex{Sternwartestrasse 71@\textbf{Sternwartestraße 71}, \emph{Wohngebäude (K.WHS)}|pwk} begann am
                     13. 7. 1910.}}}\label{K_L03016-1} kö{\geminationn}en, \label{K_L03016-2v}\edtext{Anfang Juli gehn wir für ein paar Tage auf den Se{\geminationm}ering\oindex{Semmering@\textbf{Semmering}, \emph{A.ADM3}|pw}}{\lemma{\textnormal{\emph{Anfang … Semmering}}}\Cendnote{\textnormal{Schnitzler hielt sich zwischen 6. 7. 1910 und 10. 7. 1910 am Semmering\oindex{Semmering@\textbf{Semmering}, \emph{A.ADM3}|pwk} auf.}}}\label{K_L03016-2}. – \pend
           
\pstart
           Ich \label{K_L03016-3v}\edtext{geſtriges \textsc{Feu{[}i{]}lleton\pwindex{Kuenstler sollen reden@\emph{Künstler sollen reden}|pw}}}{\lemma{\textnormal{\emph{geſtriges Feuilleton}}}\Cendnote{\textnormal{Felix Salten\pwindex{Salten, Felix 06.09.1869 – 08.10.1945@\textsc{Salten, Felix} (06.09.1869 – 08.10.1945), \emph{Schriftsteller/Schriftstellerin, Journalist/Journalistin, Chefredakteur/Chefredakteurin}|pwk}: \emph{Künstler sollen reden}\pwindex{Kuenstler sollen reden@\emph{Künstler sollen reden}|pwk}. In: \emph{Die Zeit}\pwindex{Zeit@\emph{Die Zeit}|pwk}, Jg. 9, Nr. 2784, 26. 6. 1910, Morgenblatt, S. 1–2.}}}\label{K_L03016-3} – köſtlich! – Eins
               von denen, aus deren Tiefe es noch ſchöner glitzerte als auf der Fläche oben, die
               wahrhaftig auch nicht ohne iſt.\pend
           
\pstart
           Viele Grüße von uns zu Ihnen. {\\[\baselineskip]}Herzlichſt Ihr {\\[\baselineskip]}\spacefill\mbox{A.}\pend
           \leftskip=0em{}
\pstart
           27. 6. 10\pend
           \selectlanguage{ngerman}\endnumbering\briefempfaengerindex{Salten, Felix@\textsc{Salten, Felix}!zzzSchnitzler, Arthur@\emph{von Arthur Schnitzler}!1910-06-271@{27. 6. 1910}|)be}\mylabel{L03016h}  \normalsize

\doendnotes{C}
\bigskip
\vfill

\clearpage

\footnotesize

\lohead{\textsc{register}}

% Definiere theindex-Environment komplett neu ohne reledmac
\makeatletter
\renewenvironment{theindex}{%
  \section*{\indexname}%
  \setlength{\parindent}{0pt}%
  \setlength{\parskip}{0pt plus 0.3pt}%
  \let\item\@idxitem
}{%
  \clearpage
}
\makeatother

\IfFileExists{\jobname-pw.ind}{\input{\jobname-pw.ind}}{}

\end{document}

      