%% latex-korrekturansicht-vorspann.tex
%% Vorspann für die Korrekturansicht.
%% Lädt die gemeinsame Datei latex-vorspann.tex mit gesetztem Schalter.

\newif\ifkorrekturansicht
\korrekturansichttrue

\input{../tex-inputs/latex-vorspann}


\section[ Felix Salten an Arthur Schnitzler, {[}12. 1. 1902{]}]{L03322 Felix Salten an Arthur Schnitzler, {[}12. 1. 1902{]}}
\nopagebreak\mylabel{L03322v}
\rehead{ }\normalsize\beginnumbering\briefempfaengerindex{Schnitzler, Arthur@\textsc{Schnitzler, Arthur}!zzzSalten, Felix@\emph{von Felix Salten}!1902-01-121@{{[}12. 1. 1902{]}}|(be}
\toendnotes[C]{\smallbreak\pagebreak[2]}\Standort{CUL, Schnitzler, B 89, A 2.}
\physDesc{Brief, 1 Blatt, 1 Seite, 280 Zeichen
\newline{}Handschrift: Bleistift, lateinische Kurrent
\newline{}Schnitzler: mit Bleistift datiert: »12/1 902« 
\newline{}Ordnung: mit Bleistift von unbekannter Hand nummeriert: »146« }\toendnotes[C]{\smallbreak}
\pstart
           \raggedleft{}{\pb}Sonntag\pend
           \vspace{0.5em}
\pstart
           Lieber, danke herzlich für die \label{K_L03322-1v}\edtext{»lebendigen
                  Stunden\pwindex{Lebendige Stunden. Vier Einakter@\emph{Lebendige Stunden. Vier Einakter}|pw}«}{\lemma{\textnormal{\emph{»lebendigen
                  Stunden«}}}\Cendnote{\textnormal{Siehe Arthur Schnitzler: Widmungsexemplar Lebendige Stunden für Felix
               Salten, [11.?] 1. 1902.
               }}}\label{K_L03322-1}, die ich eben bekam. Hörte von Trebitsch\pwindex{Trebitsch, Siegfried 22.12.1868 – 03.06.1956@\textsc{Trebitsch, Siegfried} (22.12.1868 – 03.06.1956), \emph{Schriftsteller/Schriftstellerin, Übersetzer/Übersetzerin}|pw}, dass Sie wieder in Wien\oindex{Wien@\textbf{Wien}, \emph{A.ADM2}|pw} sind.
               Ich habe mich sehr über den großen \label{K_L03322-2v}\edtext{Erfolg}{\lemma{\textnormal{\emph{Erfolg}}}\Cendnote{\textnormal{Die Uraufführung des
                  Einakterzyklus’ \emph{Lebendige Stunden}\pwindex{Lebendige Stunden. Vier Einakter@\emph{Lebendige Stunden. Vier Einakter}|pwk} am 4. 1. 1902 am \emph{Deutschen Theater Berlin}\orgindex{Deutsches Theater Berlin@Deutsches Theater Berlin|pwk} wurde in der Presse sehr lobend besprochen.}}}\label{K_L03322-2}
               gefreut, besonders darüber, dass die \label{K_L03322-3v}\edtext{»Frau mit dem Dolch\pwindex{Frau mit dem Dolche@\emph{Die Frau mit dem Dolche}|pw}« uns Recht gegeben}{\lemma{\textnormal{\emph{»Frau … gegeben}}}\Cendnote{\textnormal{Vor der Premiere hatten gegenüber 
                     Schnitzler mehrere Personen die 
                     Schwierigkeit der Verwandlung der Szene von der Gegenwart 
                     in ein Renaissance-Atelier herausgestrichen. Wäre sie schlecht gemacht, hätte der Einakter
                     scheitern können.}}}\label{K_L03322-3}.
               Hoffentlich \label{K_L03322-4v}\edtext{sehe ich Sie bald}{\lemma{\textnormal{\emph{sehe ich Sie bald}}}\Cendnote{\textnormal{Nachweislich sahen sich Salten\pwindex{Salten, Felix 06.09.1869 – 08.10.1945@\textsc{Salten, Felix} (06.09.1869 – 08.10.1945), \emph{Schriftsteller/Schriftstellerin, Journalist/Journalistin, Chefredakteur/Chefredakteurin}|pwk} und Schnitzler
                  am 26. 1. 1902
                  wieder.}}}\label{K_L03322-4}.\pend
           
\pstart
           Ihr {\\[\baselineskip]}\spacefill\mbox{Salten}\pend
           \leftskip=0em{}\selectlanguage{ngerman}\endnumbering\briefempfaengerindex{Schnitzler, Arthur@\textsc{Schnitzler, Arthur}!zzzSalten, Felix@\emph{von Felix Salten}!1902-01-121@{{[}12. 1. 1902{]}}|)be}\mylabel{L03322h}  \normalsize

\doendnotes{C}
\bigskip
\vfill

\clearpage

\footnotesize

\lohead{\textsc{register}}

% Definiere theindex-Environment komplett neu ohne reledmac
\makeatletter
\renewenvironment{theindex}{%
  \section*{\indexname}%
  \setlength{\parindent}{0pt}%
  \setlength{\parskip}{0pt plus 0.3pt}%
  \let\item\@idxitem
}{%
  \clearpage
}
\makeatother

\IfFileExists{\jobname-pw.ind}{\input{\jobname-pw.ind}}{}

\end{document}

      