%% latex-leseansicht-vorspann.tex
%% Vorspann für die Leseansicht.
%% Lädt die gemeinsame Datei latex-vorspann.tex mit nicht gesetztem Schalter.

\newif\ifkorrekturansicht
\korrekturansichtfalse

\input{../tex-inputs/latex-vorspann}


\section[Felix Salten an Arthur Schnitzler, {{[}}zwischen 27. und 31. 10. 1903{{]}}]{L03349 Felix Salten an Arthur Schnitzler, {[}zwischen 27. und 31. 10. 1903{]}}
\nopagebreak\mylabel{L03349v}
\rehead{ }\normalsize\beginnumbering\briefempfaengerindex{Schnitzler, Arthur@\textsc{Schnitzler, Arthur}!zzzSalten, Felix@\emph{von Felix Salten}!1903-10-311@{{[}zwischen 27. und 31. 10. 1903{]}}|(be}
\toendnotes[C]{\smallbreak\pagebreak[2]}
\correspDesc{Versand  durch Felix Salten im Zeitraum [zwischen 27. und 31. 10. 1903] in Wien
\newline{}Erhalt  durch Arthur Schnitzler im Zeitraum [zwischen 27. und 31. 10. 1903?] in Wien}\toendnotes[C]{\smallbreak}
\Standort{CUL, Schnitzler, B 89, A 2.}
\physDesc{Brief, 1 Blatt, 1 Seite, 111 Zeichen
\newline{}Handschrift: Bleistift, lateinische Kurrent
\newline{}Schnitzler: mit Bleistift datiert: »Oct. 90\textcolor{gray}{3}.« 
\newline{}Ordnung: mit Bleistift von unbekannter Hand nummeriert: »175« }\toendnotes[C]{\smallbreak}
\pstart
           \noindent{}{\pb}Lieber,{ }\label{K_L03349-1v}\edtext{Trebitsch\pwindex{Trebitsch, Siegfried 22.\,12.\,1868 Wien – 3.\,6.\,1956 Zürich@\textsc{Trebitsch, Siegfried} (22.\,12.\,1868 Wien – 3.\,6.\,1956 Zürich), \emph{Schriftsteller, Übersetzer}|pw} ist mir natürlich recht}{\lemma{\textnormal{\emph{Trebitsch … recht}}}\Cendnote{\textnormal{Das Korrespondenzstück ist undatiert. Schnitzler datiert es auf den Zeitraum »Oct. 90\textcolor{gray}{3}.« Es dürfte sich um das gemeinsame Treffen mit Trebitsch\pwindex{Trebitsch, Siegfried 22.\,12.\,1868 Wien – 3.\,6.\,1956 Zürich@\textsc{Trebitsch, Siegfried} (22.\,12.\,1868 Wien – 3.\,6.\,1956 Zürich), \emph{Schriftsteller, Übersetzer}|pwk} am 1. 11. 1903 handeln. Damit wäre der Brief in der
                  vorangehenden Woche verfasst worden. Der ebenfalls undatierte Brief aus dieser Zeit XXXX Auszeichnungsfehler: Dokument L03348 nicht gefunden dürfte sich
                  ebenfalls auf dieses Treffen beziehen und muss vorher gelaufen sein, weil eine
                  dort fehlende Auskunft über die Teilnahme der Tochter Caroline\pwindex{Kotter, Caroline 7.\,7.\,1893 Wien – 1.\,7.\,1964 ebd.@\textsc{Kotter, Caroline} (7.\,7.\,1893 Wien – 1.\,7.\,1964 ebd.)|pwk} nachgereicht wird. Damit lässt sich das
                  Zeitfenster noch etwas verkleinern.}}}\label{K_L03349-1}. Lintscherl\pwindex{Kotter, Caroline 7.\,7.\,1893 Wien – 1.\,7.\,1964 ebd.@\textsc{Kotter, Caroline} (7.\,7.\,1893 Wien – 1.\,7.\,1964 ebd.)|pw} bleibt zu Hause, denn sie muß schlafen gehen.\pend
           
\pstart
           Herzlichst {\\[\baselineskip]}Ihr {\\[\baselineskip]}\spacefill\mbox{S.}\pend
           \leftskip=0em{}\selectlanguage{ngerman}\endnumbering\briefempfaengerindex{Schnitzler, Arthur@\textsc{Schnitzler, Arthur}!zzzSalten, Felix@\emph{von Felix Salten}!1903-10-271@{{[}zwischen 27. und 31. 10. 1903{]}}|)be}\mylabel{L03349h}  \newcommand{\dateiname}{L03349}\newcommand{\titel}{Felix Salten an Arthur Schnitzler, [zwischen 27. und 31. 10. 1903]}\newcommand{\editorInnen}{Martin Anton Müller und Laura Untner}%% latex-leseansicht-abspann.tex
%% Abspann für die Leseansicht.
%% Der Schalter \ifkorrekturansicht ist bereits durch den Vorspann gesetzt.

%% latex-abspann.tex
%% Gemeinsamer Abspann für Korrekturansicht und Leseansicht.
%% Setzt den Schalter \ifkorrekturansicht voraus (gesetzt in den
%% einbindenden Dateien latex-korrekturansicht-abspann.tex bzw.
%% latex-leseansicht-abspann.tex).
%% ---------------------------------------------------------------

\normalsize

% Das esempio-Environment wird nur in der Leseansicht benötigt
\ifkorrekturansicht\else
\newenvironment{esempio}[3]%
{
    \vspace{1.5ex}
    \rlap{\underline{#1}}
    \par
    \setlength{\parindent}{0cm}
    \nopagebreak
    \leftskip=#2cm
    \rightskip=#3cm
}
{
    \par
}
\fi

\doendnotes{C}
\bigskip
\vfill

\clearpage

\footnotesize

\ifkorrekturansicht
  \lohead{\textsc{register}}
\fi

% theindex-Environment neu definieren ohne reledmac
\makeatletter
\renewenvironment{theindex}{%
  \ifkorrekturansicht
    \section*{\indexname}%
  \else
    \subsubsection*{Index der erwähnten Entitäten}%
  \fi
  \setlength{\parindent}{0pt}%
  \setlength{\parskip}{0pt plus 0.3pt}%
  \let\item\@idxitem
}{%
  \ifkorrekturansicht\clearpage\fi
}
\makeatother

\IfFileExists{\jobname-pw.ind}{\input{\jobname-pw.ind}}{}

% Quellenangabe nur in der Leseansicht
\ifkorrekturansicht\else
% Fallback-Definitionen, falls die .tex-Datei \titel etc. nicht gesetzt hat
\providecommand{\titel}{}
\providecommand{\editorInnen}{}
\providecommand{\dateiname}{\jobname}

\vspace{3cm}

\vfill

\footnotesize
\textsc{Quelle}: \titel. Herausgegeben von {\editorInnen}. In: \emph{Arthur Schnitzler: Briefwechsel mit Autorinnen und Autoren}.
 Digitale Edition, https://schnitzler-briefe.acdh.oeaw.ac.at/{\dateiname}.html (Stand \today)
\fi

\end{document}


