%% latex-korrekturansicht-vorspann.tex
%% Vorspann für die Korrekturansicht.
%% Lädt die gemeinsame Datei latex-vorspann.tex mit gesetztem Schalter.

\newif\ifkorrekturansicht
\korrekturansichttrue

\input{../tex-inputs/latex-vorspann}


\section[Felix Salten an Arthur Schnitzler, {[}zwischen 27. und 31. 10. 1903{]}]{L03349 Felix Salten an Arthur Schnitzler,
               {[}zwischen 27. und 31. 10. 1903{]}}
\nopagebreak\mylabel{L03349v}
\rehead{ }\normalsize\beginnumbering\briefempfaengerindex{Schnitzler, Arthur@\textsc{Schnitzler, Arthur}!zzzSalten, Felix@\emph{von Felix Salten}!1903-10-311@{{[}zwischen 27. und 31. 10. 1903{]}}|(be}
\toendnotes[C]{\smallbreak\pagebreak[2]}\Standort{CUL, Schnitzler, B 89, A 2.}
\physDesc{Brief, 1 Blatt, 1 Seite, 111 Zeichen
\newline{}Handschrift: Bleistift, lateinische Kurrent
\newline{}Schnitzler: mit Bleistift datiert: »Oct. 90\textcolor{gray}{3}.« 
\newline{}Ordnung: mit Bleistift von unbekannter Hand nummeriert: »175« }\toendnotes[C]{\smallbreak}
\pstart
           \noindent{}{\pb}Lieber,{ }\label{K_L03349-1v}\edtext{Trebitsch\pwindex{Trebitsch, Siegfried 22.12.1868 – 03.06.1956@\textsc{Trebitsch, Siegfried} (22.12.1868 – 03.06.1956), \emph{Schriftsteller/Schriftstellerin, Übersetzer/Übersetzerin}|pw} ist mir natürlich recht}{\lemma{\textnormal{\emph{Trebitsch … recht}}}\Cendnote{\textnormal{Das Korrespondenzstück ist undatiert. Schnitzler datiert es auf den Zeitraum »Oct. 90\textcolor{gray}{3}.« Es dürfte sich um das gemeinsame Treffen mit Trebitsch\pwindex{Trebitsch, Siegfried 22.12.1868 – 03.06.1956@\textsc{Trebitsch, Siegfried} (22.12.1868 – 03.06.1956), \emph{Schriftsteller/Schriftstellerin, Übersetzer/Übersetzerin}|pwk} am 1. 11. 1903 handeln. Damit wäre der Brief in der
                  vorangehenden Woche verfasst worden. Der ebenfalls undatierte Brief aus dieser Zeit Felix Salten an Arthur
               Schnitzler, [zwischen 26. und 30. 10. 1903] dürfte sich
                  ebenfalls auf dieses Treffen beziehen und muss vorher gelaufen sein, weil eine
                  dort fehlende Auskunft über die Teilnahme der Tochter Caroline\pwindex{Kotter, Caroline 1893-07-07 – 1964-07-01@\textsc{Kotter, Caroline} (1893-07-07 – 1964-07-01)|pwk} nachgereicht wird. Damit lässt sich das
                  Zeitfenster noch etwas verkleinern.}}}\label{K_L03349-1}. Lintscherl\pwindex{Kotter, Caroline 1893-07-07 – 1964-07-01@\textsc{Kotter, Caroline} (1893-07-07 – 1964-07-01)|pw} bleibt zu Hause, denn sie muß schlafen gehen.\pend
           
\pstart
           Herzlichst {\\[\baselineskip]}Ihr {\\[\baselineskip]}\spacefill\mbox{S.}\pend
           \leftskip=0em{}\selectlanguage{ngerman}\endnumbering\briefempfaengerindex{Schnitzler, Arthur@\textsc{Schnitzler, Arthur}!zzzSalten, Felix@\emph{von Felix Salten}!1903-10-271@{{[}zwischen 27. und 31. 10. 1903{]}}|)be}\mylabel{L03349h}  \normalsize

\doendnotes{C}
\bigskip
\vfill

\clearpage

\footnotesize

\lohead{\textsc{register}}

% Definiere theindex-Environment komplett neu ohne reledmac
\makeatletter
\renewenvironment{theindex}{%
  \section*{\indexname}%
  \setlength{\parindent}{0pt}%
  \setlength{\parskip}{0pt plus 0.3pt}%
  \let\item\@idxitem
}{%
  \clearpage
}
\makeatother

\IfFileExists{\jobname-pw.ind}{\input{\jobname-pw.ind}}{}

\end{document}

      