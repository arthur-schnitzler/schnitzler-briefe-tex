%% latex-korrekturansicht-vorspann.tex
%% Vorspann für die Korrekturansicht.
%% Lädt die gemeinsame Datei latex-vorspann.tex mit gesetztem Schalter.

\newif\ifkorrekturansicht
\korrekturansichttrue

\input{../tex-inputs/latex-vorspann}


\section[ Arthur Schnitzler an Felix Salten, 26. 5. 1906]{L03006 Arthur Schnitzler an Felix Salten, 26. 5. 1906}
\nopagebreak\mylabel{L03006v}
\rehead{ }\normalsize\beginnumbering\briefempfaengerindex{Salten, Felix@\textsc{Salten, Felix}!zzzSchnitzler, Arthur@\emph{von Arthur Schnitzler}!1906-05-261@{26. 5. 1906}|(be}
\toendnotes[C]{\smallbreak\pagebreak[2]}\Standort{Wienbibliothek im Rathaus, ZPH 1681, 2.1.516.}
\physDesc{Karte, 194 Zeichen
\newline{}Handschrift: schwarze Tinte, deutsche Kurrent
\newline{}Ordnung: mit Bleistift von unbekannter Hand nummeriert: »11« }\toendnotes[C]{\smallbreak}
\pstart
           {\pb}\textcolor{gray}{\textbf{Dr. Arthur Schnitzler}}\hfill 26. Mai 906\pend
           
\pstart
           \textcolor{gray}{\textbf{Wien, XVIII. Spoettelgasse 7\oindex{Edmund-Weiss-Gasse 7@\textbf{Edmund-Weiß-Gasse 7}, \emph{Wohngebäude (K.WHS)}|pw}.}}\pend
           \vspace{0.5em}
\pstart
           lieber, wundervoll, ergreifend iſt Ihre \label{K_L03006-1v}\edtext{Todtenrede auf Ibſen\pwindex{Ibsen, Henrik 20.03.1828 – 23.05.1906@\textsc{Ibsen, Henrik} (20.03.1828 – 23.05.1906), \emph{Schriftsteller/Schriftstellerin}|pw}\pwindex{Totenrede auf Henrik Ibsen@\emph{Totenrede auf Henrik Ibsen}|pwv}}{\lemma{\textnormal{\emph{Todtenrede auf Ibſen}}}\Cendnote{\textnormal{Felix Salten\pwindex{Salten, Felix 06.09.1869 – 08.10.1945@\textsc{Salten, Felix} (06.09.1869 – 08.10.1945), \emph{Schriftsteller/Schriftstellerin, Journalist/Journalistin, Chefredakteur/Chefredakteurin}|pwk}: \emph{Totenrede auf Henrik Ibsen}\pwindex{Totenrede auf Henrik Ibsen@\emph{Totenrede auf Henrik Ibsen}|pwk}. In: \emph{B. Z. am Mittag}\pwindex{B.Z. am Mittag@\emph{B.Z. am Mittag}|pwk}, Jg. 30, Nr. 121, 25. 5. 1906, Zweites Beiblatt, S. 1.}}}\label{K_L03006-1}.
               Man möchte ſie von Mitterwurzer\pwindex{Mitterwurzer, Friedrich 16.10.1844 – 13.02.1897@\textsc{Mitterwurzer, Friedrich} (16.10.1844 – 13.02.1897), \emph{Schauspieler/Schauspielerin}|pw} geleſen
               hören.\pend
           
\pstart
           {\pb}Iſt es wahr, dſs Sie \label{K_L03006-2v}\edtext{zu Pfingſten
               nach Wien\oindex{Wien@\textbf{Wien}, \emph{A.ADM2}|pw} kommen? Und am 23. wieder}{\lemma{\textnormal{\emph{zu … wieder}}}\Cendnote{\textnormal{Das dürfte nicht
                  passiert sein, siehe die Folgebriefe.}}}\label{K_L03006-2}–?\pend
           
\pstart
           Von Herzen Ihr {\\[\baselineskip]}\spacefill\mbox{A.}\pend
           \leftskip=0em{}\selectlanguage{ngerman}\endnumbering\briefempfaengerindex{Salten, Felix@\textsc{Salten, Felix}!zzzSchnitzler, Arthur@\emph{von Arthur Schnitzler}!1906-05-261@{26. 5. 1906}|)be}\mylabel{L03006h}  \normalsize

\doendnotes{C}
\bigskip
\vfill

\clearpage

\footnotesize

\lohead{\textsc{register}}

% Definiere theindex-Environment komplett neu ohne reledmac
\makeatletter
\renewenvironment{theindex}{%
  \section*{\indexname}%
  \setlength{\parindent}{0pt}%
  \setlength{\parskip}{0pt plus 0.3pt}%
  \let\item\@idxitem
}{%
  \clearpage
}
\makeatother

\IfFileExists{\jobname-pw.ind}{\input{\jobname-pw.ind}}{}

\end{document}

      