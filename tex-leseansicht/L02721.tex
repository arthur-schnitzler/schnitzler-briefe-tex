%% latex-leseansicht-vorspann.tex
%% Vorspann für die Leseansicht.
%% Lädt die gemeinsame Datei latex-vorspann.tex mit nicht gesetztem Schalter.

\newif\ifkorrekturansicht
\korrekturansichtfalse

\input{../tex-inputs/latex-vorspann}

\begin{center}
            \textcolor{red}{ENTWURF, NICHT FERTIG KORRIGIERT}
                      \end{center}
            
               \section[Paul Goldmann an Arthur Schnitzler, 5. 12. {[}1893{]}]{ Paul Goldmann an Arthur Schnitzler, 5. 12. {[}1893{]}}\nopagebreak\mylabel{v}\rehead{ }\begin{ledgroupsized}[t]{13cm}\normalsize\beginnumbering\briefempfaengerindex{Schnitzler, Arthur@\textsc{Schnitzler, Arthur}!zzzGoldmann, Paul@\emph{von Paul Goldmann}!1893-12-051@{5. 12. {[}1893{]}}|(be} \toendnotes[C]{\smallbreak\pagebreak[2]} \Standort{DLA, A:Schnitzler, HS.NZ85.1.3163.}
\physDesc{Brief, 1 Blatt, 4 Seiten
\newline{}Handschrift: schwarze Tinte, deutsche Kurrent
\newline{}Schnitzler: 1) mit Bleistift das Jahr »93« vermerkt 2) mit rotem Buntstift vier Unterstreichungen}\toendnotes[C]{\smallbreak}\pstart
           \noindent{}{\pb}\textcolor{gray}{\textbf{\textbf{Frankfurter Zeitung\orgindex{Frankfurter Zeitung@Frankfurter Zeitung|pw}.}}}\pend
           \pstart
           \textcolor{gray}{\textbf{\textbf{(\begin{otherlanguage}{french}Gazette de Francfort\end{otherlanguage}\orgindex{Frankfurter Zeitung@Frankfurter Zeitung|pw}.)}}}\pend
           \pstart
           \textcolor{gray}{\textbf{\begin{otherlanguage}{french}Directeur\pwindex{Sonnemann, Leopold 1831-10-29 – 1909-10-30@\textsc{Sonnemann, Leopold} (1831-10-29 – 1909-10-30), \emph{Journalist, Herausgeber}|pwv}\end{otherlanguage}{ }\textbf{M. L. Sonnemann\pwindex{Sonnemann, Leopold 1831-10-29 – 1909-10-30@\textsc{Sonnemann, Leopold} (1831-10-29 – 1909-10-30), \emph{Journalist, Herausgeber}|pw}.}}}\hfill \textsc{Paris\oindex{Paris@\textbf{Paris}|pw}}, 5. December.\pend
           \pstart
           \begin{otherlanguage}{french}\textcolor{gray}{\textbf{Journal\pwindex{Frankfurter Zeitung1856 – 1943@\emph{Frankfurter Zeitung}|pw} politique, financier,}}\end{otherlanguage}\pend
           \pstart
           \begin{otherlanguage}{french}\textcolor{gray}{\textbf{commercial et litteraire.}}\end{otherlanguage}\pend
           \pstart
           \begin{otherlanguage}{french}\textcolor{gray}{\textbf{\textbf{Paraissant trois fois par jour}}}\end{otherlanguage}\pend
           \pstart
           \begin{otherlanguage}{french}\textcolor{gray}{\textbf{\textbf{Bureaux à Paris\oindex{Paris@\textbf{Paris}|pw}:}}}\end{otherlanguage}\pend
           \pstart
           \begin{otherlanguage}{french}\textcolor{gray}{\textbf{\textbf{rue Richelieu 75\oindex{rue Richelieu@\textbf{rue Richelieu}|pw}.}}}\end{otherlanguage}\pend
           \pstart
           Mein lieber Freund!\pend
           \pstart
           Nachdem ich bisher vergeblich auf die verſprochenen Kritiken oder wenigſtens auf eine
               briefliche Mittheilung über die \textsc{Première\pwindex{Schnitzler, Arthur 15.05.1862 – 21.10.1931@\textsc{Schnitzler, Arthur} (15.05.1862 – 21.10.1931), \emph{Schriftsteller, Mediziner}!Maerchen. Schauspiel in drei Aufzuegen1891 – 1891@\strich\emph{Das Märchen. Schauspiel in drei Aufzügen} {[}1891 – 1891{]}|pwv}n}-Eindrücke
               gewartet, habe ich mir das Nöthige von Frankfurt\oindex{Frankfurt am Main@\textbf{Frankfurt am Main}|pw}
               kommen laſſen und bitte Dich, Dich nun nicht mehr zu bemühen.\pend
           \pstart
           Wenn ich aus der Sammlung der Kritiken, die mir vor liegt, die \label{K_L02721-1v}\edtext{dummen Jungen\pwindex{?? Werk@Nicht ermittelte Verfasserinnen und Verfasser!Theater- und Kunstnachrichten [Maerchen-Urauffuehrung]1893-12-02 – 1893-12-02@\emph{Theater- und Kunstnachrichten [Märchen-Uraufführung]} {[}1893-12-02 – 1893-12-02{]}|pwv}\pwindex{Theater, Kunst und Literatur [Maerchen-Urauffuehrung]1893-12-02 – 1893-12-02@\emph{Theater, Kunst und Literatur [Märchen-Uraufführung]} {[}1893-12-02 – 1893-12-02{]}|pwv}\pwindex{Theater, Kunst und Literatur [Maerchen-Urauffuehrung]1893-12-02 – 1893-12-02@\emph{Theater, Kunst und Literatur [Märchen-Uraufführung]} {[}1893-12-02 – 1893-12-02{]}|pwv}\pwindex{Theater und Kunst [Maerchen-Urauffuehrung]1893-12-01 – 1893-12-01@\emph{Theater und Kunst [Märchen-Uraufführung]} {[}1893-12-01 – 1893-12-01{]}|pwv}}{\lemma{\textnormal{\emph{dummen Jungen}}}\Cendnote{\textnormal{Mit großer Wahrscheinlichkeit handelte
                  es sich um folgende Kritiken\pwindex{?? Werk@Nicht ermittelte Verfasserinnen und Verfasser!Theater- und Kunstnachrichten [Maerchen-Urauffuehrung]1893-12-02 – 1893-12-02@\emph{Theater- und Kunstnachrichten [Märchen-Uraufführung]} {[}1893-12-02 – 1893-12-02{]}|pwkv}\pwindex{Theater, Kunst und Literatur [Maerchen-Urauffuehrung]1893-12-02 – 1893-12-02@\emph{Theater, Kunst und Literatur [Märchen-Uraufführung]} {[}1893-12-02 – 1893-12-02{]}|pwkv}\pwindex{Theater, Kunst und Literatur [Maerchen-Urauffuehrung]1893-12-02 – 1893-12-02@\emph{Theater, Kunst und Literatur [Märchen-Uraufführung]} {[}1893-12-02 – 1893-12-02{]}|pwkv}\pwindex{Theater und Kunst [Maerchen-Urauffuehrung]1893-12-01 – 1893-12-01@\emph{Theater und Kunst [Märchen-Uraufführung]} {[}1893-12-01 – 1893-12-01{]}|pwkv}: N. N.: \emph{Theater- und Kunstnachrichten}\pwindex{?? Werk@Nicht ermittelte Verfasserinnen und Verfasser!Theater- und Kunstnachrichten [Maerchen-Urauffuehrung]1893-12-02 – 1893-12-02@\emph{Theater- und Kunstnachrichten [Märchen-Uraufführung]} {[}1893-12-02 – 1893-12-02{]}|pwk}. In: \emph{Neue Freie Presse}\pwindex{Neue Freie Presse1864 – 1939@\emph{Neue Freie Presse}|pwk}, Jg. 30, Nr. 10518, 2. 12. 1893, S. 7, l. b.\pwindex{b., l. @\textsc{b., l.}, \emph{Journalist/Journalistin}|pwk}: \emph{Theater, Kunst und Literatur}\pwindex{Theater, Kunst und Literatur [Maerchen-Urauffuehrung]1893-12-02 – 1893-12-02@\emph{Theater, Kunst und Literatur [Märchen-Uraufführung]} {[}1893-12-02 – 1893-12-02{]}|pwk}. In: \emph{Neues Wiener Tagblatt}\pwindex{Neues Wiener Tagblatt1867 – 1945@\emph{Neues Wiener Tagblatt}|pwk}, Jg. 27, Nr. 333, 2. 12. 1893S. 8, H. P.\pwindex{P., H. @\textsc{P., H.}, \emph{Journalist/Journalistin}|pwk}: \emph{Theater, Kunst und Literatur}\pwindex{Theater, Kunst und Literatur [Maerchen-Urauffuehrung]1893-12-02 – 1893-12-02@\emph{Theater, Kunst und Literatur [Märchen-Uraufführung]} {[}1893-12-02 – 1893-12-02{]}|pwk}. In: \emph{Deutsches Volksblatt}\pwindex{Deutsches Volksblatt1889 – 1922@\emph{Deutsches Volksblatt}|pwk}, Jg. 5, Nr. 1768, 2. 12. 1893, S.  6–7, –r–\pwindex{–r– @\textsc{–r–}, \emph{Theaterkritiker/Theaterkritikerin}|pwk}: \emph{Theater und Kunst}\pwindex{Theater und Kunst [Maerchen-Urauffuehrung]1893-12-01 – 1893-12-01@\emph{Theater und Kunst [Märchen-Uraufführung]} {[}1893-12-01 – 1893-12-01{]}|pwk}. In: \emph{Das
                        Vaterland}\pwindex{Vaterland1.12.1860 – 31.12.1911@\emph{Das Vaterland}|pwk}, Jg. 34, Nr. 333, 2. 12. 1893,
                     S. 7.}}}\label{K_L02721-1h} weglaſſe – \introOben{}Neue Freie Preſſe\pwindex{?? Werk@Nicht ermittelte Verfasserinnen und Verfasser!Theater- und Kunstnachrichten [Maerchen-Urauffuehrung]1893-12-02 – 1893-12-02@\emph{Theater- und Kunstnachrichten [Märchen-Uraufführung]} {[}1893-12-02 – 1893-12-02{]}|pwv}\introOben{}{ }Neues Wiener Tagblatt\pwindex{Theater, Kunst und Literatur [Maerchen-Urauffuehrung]1893-12-02 – 1893-12-02@\emph{Theater, Kunst und Literatur [Märchen-Uraufführung]} {[}1893-12-02 – 1893-12-02{]}|pwv} , Volksblatt\pwindex{Theater, Kunst und Literatur [Maerchen-Urauffuehrung]1893-12-02 – 1893-12-02@\emph{Theater, Kunst und Literatur [Märchen-Uraufführung]} {[}1893-12-02 – 1893-12-02{]}|pwv} , Vaterland\pwindex{Theater und Kunst [Maerchen-Urauffuehrung]1893-12-01 – 1893-12-01@\emph{Theater und Kunst [Märchen-Uraufführung]} {[}1893-12-01 – 1893-12-01{]}|pwv}{ }\textsc{etc}. – und mich nur an {\pb}die \label{K_L02721-2v}\edtext{Zurechnungsfähigen\pwindex{\textcolor{red}{\textsuperscript{XXXX1 indx}}|pwv}\pwindex{\textcolor{red}{\textsuperscript{XXXX1 indx}}|pwv}}{\lemma{\textnormal{\emph{Zurechnungsfähigen}}}\Cendnote{\textnormal{Höchstwahrscheinliche meinte Goldmann\pwindex{Goldmann, Paul 31.01.1865 – 25.09.1935@\textsc{Goldmann, Paul} (31.01.1865 – 25.09.1935), \emph{Schriftsteller, Journalist}|pwk} folgende Kritiken\pwindex{Feuilleton. Theater [Maerchen-Urauffuehrung]1893-12-02 – 1893-12-02@\emph{Feuilleton. Theater [Märchen-Uraufführung]} {[}1893-12-02 – 1893-12-02{]}|pwkv}\pwindex{Bahr, Hermann 19.07.1863 – 15.01.1934@\textsc{Bahr, Hermann} (19.07.1863 – 15.01.1934), \emph{Schriftsteller, Kritiker}!Maerchen (Schauspiel in drei Aufzuegen von Arthur Schnitzler)1893-12-02 – 1893-12-02@\strich\emph{Das Märchen (Schauspiel in drei Aufzügen von Arthur Schnitzler)} {[}1893-12-02 – 1893-12-02{]}|pwkv}: [Friedrich Uhl\pwindex{Uhl, Friedrich 14.05.1825 – 20.01.1906@\textsc{Uhl, Friedrich} (14.05.1825 – 20.01.1906), \emph{Journalist}|pwk}]: \emph{Feuilleton. Theater}\pwindex{Feuilleton. Theater [Maerchen-Urauffuehrung]1893-12-02 – 1893-12-02@\emph{Feuilleton. Theater [Märchen-Uraufführung]} {[}1893-12-02 – 1893-12-02{]}|pwk}. In: \emph{Wiener
                        Abendpost. Beilage zur Wiener Zeitung}\pwindex{Wiener Abendpost1.7.1863 – 31.12.1921@\emph{Wiener Abendpost}|pwk}, Jg. 190, Nr. 276, 2. 12. 1893, S. 1–2, Hermann Bahr\pwindex{Bahr, Hermann 19.07.1863 – 15.01.1934@\textsc{Bahr, Hermann} (19.07.1863 – 15.01.1934), \emph{Schriftsteller, Kritiker}|pwk}: \emph{Das Märchen (Schauspiel in drei Aufzügen von Arthur
                        Schnitzler. Zum ersten Male aufgeführt am Deutschen Volkstheater den 1.
                        December)}\pwindex{Bahr, Hermann 19.07.1863 – 15.01.1934@\textsc{Bahr, Hermann} (19.07.1863 – 15.01.1934), \emph{Schriftsteller, Kritiker}!Maerchen (Schauspiel in drei Aufzuegen von Arthur Schnitzler)1893-12-02 – 1893-12-02@\strich\emph{Das Märchen (Schauspiel in drei Aufzügen von Arthur Schnitzler)} {[}1893-12-02 – 1893-12-02{]}|pwk}. In: \emph{Deutsche Zeitung}\pwindex{Deutsche Zeitung1871 – 1907@\emph{Deutsche Zeitung}|pwk},
                     Jg. 23, Nr. 7879, 2. 12. 1893, Morgen-Ausgabe\pwindex{Deutsche Zeitung1871 – 1907@\emph{Deutsche Zeitung}|pwkv}, S. 1–3,
                     XXXX Brociner fehlt hier noch (nicht im Neuen Wiener Tagblatt, nicht in
                     Wiener Literatur-Zeitung)}}}\label{K_L02721-2h} halte, wie \textsc{Uhl\pwindex{Uhl, Friedrich 14.05.1825 – 20.01.1906@\textsc{Uhl, Friedrich} (14.05.1825 – 20.01.1906), \emph{Journalist}|pw}\pwindex{Feuilleton. Theater [Maerchen-Urauffuehrung]1893-12-02 – 1893-12-02@\emph{Feuilleton. Theater [Märchen-Uraufführung]} {[}1893-12-02 – 1893-12-02{]}|pwv}}, \textsc{Bahr\pwindex{Bahr, Hermann 19.07.1863 – 15.01.1934@\textsc{Bahr, Hermann} (19.07.1863 – 15.01.1934), \emph{Schriftsteller, Kritiker}|pw}\pwindex{Bahr, Hermann 19.07.1863 – 15.01.1934@\textsc{Bahr, Hermann} (19.07.1863 – 15.01.1934), \emph{Schriftsteller, Kritiker}!Maerchen (Schauspiel in drei Aufzuegen von Arthur Schnitzler)1893-12-02 – 1893-12-02@\strich\emph{Das Märchen (Schauspiel in drei Aufzügen von Arthur Schnitzler)} {[}1893-12-02 – 1893-12-02{]}|pwv}} und \textsc{Brociner\pwindex{Brociner, Marco 20.10.1852 – 12.04.1942@\textsc{Brociner, Marco} (20.10.1852 – 12.04.1942), \emph{Schriftsteller, Journalist, Kritiker}|pw}\textcolor{red}{\textsuperscript{\textbf{KEY}}}}, ſo finde ich, daß man Dich hier auch mehrfach mißverſteht, daß man Dir aber
               auch vielerlei Richtiges und Beherzigenswerthes ſagt. Beſonders \textsc{Uhl\pwindex{Uhl, Friedrich 14.05.1825 – 20.01.1906@\textsc{Uhl, Friedrich} (14.05.1825 – 20.01.1906), \emph{Journalist}|pw}\pwindex{Feuilleton. Theater [Maerchen-Urauffuehrung]1893-12-02 – 1893-12-02@\emph{Feuilleton. Theater [Märchen-Uraufführung]} {[}1893-12-02 – 1893-12-02{]}|pwv}} halte ich für im Weſentlichen richtig urtheilend. Du erinnerſt Dich, wir haben
               oft im Streit gelegen, Du und ich, und ich meine noch heute, heute erſt recht, daß
               Deinem glänzenden Talent beim Produciren die Disciplin fehlt. Auch beim Produciren
               denkſt Du ein wenig zu ſehr an Dich und zu wenig an das Andere, an die Forderungen
               der Kunſtform. Du ſchreibſt Deinem Herzeleid zuliebe und nicht {\pb}dem Drama zulieb\textcolor{gray}{e}. Das iſt
               falſch. Ich komme immer mehr dahinter, daß das Produciren ein Streben nach
               möglichſter Objectivirung ſein muß, am allermeiſten aber das dramatiſche Produciren.
               Ich habe das in \textsc{Paris\oindex{Paris@\textbf{Paris}|pw}} noch mehr gelernt, habe daraufhin das »Märchen\pwindex{Schnitzler, Arthur 15.05.1862 – 21.10.1931@\textsc{Schnitzler, Arthur} (15.05.1862 – 21.10.1931), \emph{Schriftsteller, Mediziner}!Maerchen. Schauspiel in drei Aufzuegen1891 – 1891@\strich\emph{Das Märchen. Schauspiel in drei Aufzügen} {[}1891 – 1891{]}|pw}« nochmals geleſen und meine Ausſtellungen von früher noch mehr
               beſtätigt gefunden. Erinnere Dich auch, was ich Dir ſtets über den dritten Act\pwindex{Schnitzler, Arthur 15.05.1862 – 21.10.1931@\textsc{Schnitzler, Arthur} (15.05.1862 – 21.10.1931), \emph{Schriftsteller, Mediziner}!Maerchen. Schauspiel in drei Aufzuegen1891 – 1891@\strich\emph{Das Märchen. Schauspiel in drei Aufzügen} {[}1891 – 1891{]}|pwv} geſagt! Im Allgemeinen aber
               denke ich, daß Du mit Deinem Debüt\pwindex{Schnitzler, Arthur 15.05.1862 – 21.10.1931@\textsc{Schnitzler, Arthur} (15.05.1862 – 21.10.1931), \emph{Schriftsteller, Mediziner}!Maerchen. Schauspiel in drei Aufzuegen1891 – 1891@\strich\emph{Das Märchen. Schauspiel in drei Aufzügen} {[}1891 – 1891{]}|pwv} nicht unzufrieden ſein darfſt. Du biſt den Kennern ſignaliſirt; alle
               Leute, die es verſtehen, haben Dein großes {\pb}Talent
               erkannt; die dummen Bande Publicum wirſt Du jetzt raſch gewinnen. Aber jetzt ſofort
               weiter ſchreiben! Vieles lernen aus den drei zurechnungsfähigen Kritiken\pwindex{Feuilleton. Theater [Maerchen-Urauffuehrung]1893-12-02 – 1893-12-02@\emph{Feuilleton. Theater [Märchen-Uraufführung]} {[}1893-12-02 – 1893-12-02{]}|pwv}\pwindex{Bahr, Hermann 19.07.1863 – 15.01.1934@\textsc{Bahr, Hermann} (19.07.1863 – 15.01.1934), \emph{Schriftsteller, Kritiker}!Maerchen (Schauspiel in drei Aufzuegen von Arthur Schnitzler)1893-12-02 – 1893-12-02@\strich\emph{Das Märchen (Schauspiel in drei Aufzügen von Arthur Schnitzler)} {[}1893-12-02 – 1893-12-02{]}|pwv}. Und ein Drama machen, keine Beichte, kein
               Tagebuch! Das koſtet mir eine Willensanſtrengung. Denn Du biſt, ich weiß es genau,
               ein Dramatiker allererſten Ranges. Mach’ auch einen \label{K_L02721-3v}\edtext{neuen Verſuch mit dem \textsc{Alkandi\pwindex{Schnitzler, Arthur 15.05.1862 – 21.10.1931@\textsc{Schnitzler, Arthur} (15.05.1862 – 21.10.1931), \emph{Schriftsteller, Mediziner}!Alkandi s Lied15.8.1890 – 1.9.1890@\strich\emph{Alkandi’s Lied} {[}15.8.1890 – 1.9.1890{]}|pw}}}{\lemma{\textnormal{\emph{neuen … Alkandi}}}\Cendnote{\textnormal{Siehe dazu etwa Siehe Ferdinand von Saar an Arthur Schnitzler, 5. 2. 1894 und Siehe A. S.: \emph{Tagebuch}, 8. 3. 1894.}}}\label{K_L02721-3h}, nachdem Du vorher den Schluß
                  verſtärk\strikeout{t}end umgearbeitet haſt. An \textsc{Uhl\pwindex{Uhl, Friedrich 14.05.1825 – 20.01.1906@\textsc{Uhl, Friedrich} (14.05.1825 – 20.01.1906), \emph{Journalist}|pw}} hatte ich geſchrieben, damit er Dich nicht \label{K_L02721-4v}\edtext{in der Frkf. Ztg.\pwindex{Frankfurter Zeitung1856 – 1943@\emph{Frankfurter Zeitung}|pw}\pwindex{Wiener Brief1893-12 – 1893-12@\emph{Wiener Brief} {[}1893-12 – 1893-12{]}|pwv}}{\lemma{\textnormal{\emph{in der Frkf. Ztg.}}}\Cendnote{\textnormal{Vermutlich handelte es sich hierbei um
                        [Friedrich Uhl\pwindex{Uhl, Friedrich 14.05.1825 – 20.01.1906@\textsc{Uhl, Friedrich} (14.05.1825 – 20.01.1906), \emph{Journalist}|pwk}]: \emph{Wiener Brief}\pwindex{Wiener Brief1893-12 – 1893-12@\emph{Wiener Brief} {[}1893-12 – 1893-12{]}|pwk}. In: \emph{Frankfurter Zeitung}\pwindex{Frankfurter Zeitung1856 – 1943@\emph{Frankfurter Zeitung}|pwk}, Jg. 37, Nr. XXXX, DatumXXXX,
                     S. XXXX.}}}\label{K_L02721-4h} etwas ſchlecht behandle. Ich glaube, er wer ganz
               anſtändig?\pend
           \pstart
           Treue Grüße! Dein \spacefill\mbox{P. G.}\pend
           \endnumbering\briefempfaengerindex{Schnitzler, Arthur@\textsc{Schnitzler, Arthur}!zzzGoldmann, Paul@\emph{von Paul Goldmann}!1893-12-051@{5. 12. {[}1893{]}}|)be}\mylabel{h}\end{ledgroupsized}\begin{anhang}\end{anhang}\newcommand{\dateiname}{L02721}\newcommand{\titel}{Paul Goldmann an Arthur Schnitzler, 5. 12. [1893]}\newcommand{\editorInnen}{Martin Anton Müller und Laura Untner}
            \footnotesize
\begin{ledgroupsized}[t]{11.5cm}
\doendnotes{C}
\end{ledgroupsized}
         %% latex-leseansicht-abspann.tex
%% Abspann für die Leseansicht.
%% Der Schalter \ifkorrekturansicht ist bereits durch den Vorspann gesetzt.

%% latex-abspann.tex
%% Gemeinsamer Abspann für Korrekturansicht und Leseansicht.
%% Setzt den Schalter \ifkorrekturansicht voraus (gesetzt in den
%% einbindenden Dateien latex-korrekturansicht-abspann.tex bzw.
%% latex-leseansicht-abspann.tex).
%% ---------------------------------------------------------------

\normalsize

% Das esempio-Environment wird nur in der Leseansicht benötigt
\ifkorrekturansicht\else
\newenvironment{esempio}[3]%
{
    \vspace{1.5ex}
    \rlap{\underline{#1}}
    \par
    \setlength{\parindent}{0cm}
    \nopagebreak
    \leftskip=#2cm
    \rightskip=#3cm
}
{
    \par
}
\fi

\doendnotes{C}
\bigskip
\vfill

\clearpage

\footnotesize

\ifkorrekturansicht
  \lohead{\textsc{register}}
\fi

% theindex-Environment neu definieren ohne reledmac
\makeatletter
\renewenvironment{theindex}{%
  \ifkorrekturansicht
    \section*{\indexname}%
  \else
    \subsubsection*{Index der erwähnten Entitäten}%
  \fi
  \setlength{\parindent}{0pt}%
  \setlength{\parskip}{0pt plus 0.3pt}%
  \let\item\@idxitem
}{%
  \ifkorrekturansicht\clearpage\fi
}
\makeatother

\IfFileExists{\jobname-pw.ind}{\input{\jobname-pw.ind}}{}

% Quellenangabe nur in der Leseansicht
\ifkorrekturansicht\else
% Fallback-Definitionen, falls die .tex-Datei \titel etc. nicht gesetzt hat
\providecommand{\titel}{}
\providecommand{\editorInnen}{}
\providecommand{\dateiname}{\jobname}

\vspace{3cm}

\vfill

\footnotesize
\textsc{Quelle}: \titel. Herausgegeben von {\editorInnen}. In: \emph{Arthur Schnitzler: Briefwechsel mit Autorinnen und Autoren}.
 Digitale Edition, https://schnitzler-briefe.acdh.oeaw.ac.at/{\dateiname}.html (Stand \today)
\fi

\end{document}


      