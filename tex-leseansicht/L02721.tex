%% latex-korrekturansicht-vorspann.tex
%% Vorspann für die Korrekturansicht.
%% Lädt die gemeinsame Datei latex-vorspann.tex mit gesetztem Schalter.

\newif\ifkorrekturansicht
\korrekturansichttrue

\input{../tex-inputs/latex-vorspann}


\section[Paul Goldmann an Arthur Schnitzler, 5. 12. {[}1893{]}]{L02721 Paul Goldmann an Arthur Schnitzler, 5. 12. {[}1893{]}}
\nopagebreak\mylabel{L02721v}
\rehead{ }\normalsize\beginnumbering\briefempfaengerindex{Schnitzler, Arthur@\textsc{Schnitzler, Arthur}!zzzGoldmann, Paul@\emph{von Paul Goldmann}!1893-12-051@{5. 12. {[}1893{]}}|(be}
\toendnotes[C]{\smallbreak\pagebreak[2]}\Standort{DLA, A:Schnitzler, HS.NZ85.1.3163.}
\physDesc{Brief, 1 Blatt, 4 Seiten, 2175 Zeichen
\newline{}Handschrift: schwarze Tinte, deutsche Kurrent
\newline{}Schnitzler: 1) mit Bleistift das Jahr »93« vermerkt  2) mit rotem Buntstift vier Unterstreichungen}\toendnotes[C]{\smallbreak}
\pstart
           {\pb}\textcolor{gray}{\textbf{\textbf{Frankfurter Zeitung\orgindex{Frankfurter Zeitung@Frankfurter Zeitung|pw}.}}}\pend
           
\pstart
           \textcolor{gray}{\textbf{\textbf{(\begin{otherlanguage}{french}Gazette de Francfort\end{otherlanguage}\orgindex{Frankfurter Zeitung@Frankfurter Zeitung|pw}.)}}}\pend
           
\pstart
           \textcolor{gray}{\textbf{\begin{otherlanguage}{french}Directeur\end{otherlanguage}{ }\textbf{M. L. Sonnemann\pwindex{Sonnemann, Leopold 1831-10-29 – 1909-10-30@\textsc{Sonnemann, Leopold} (1831-10-29 – 1909-10-30), \emph{Journalist/Journalistin, Herausgeber/Herausgeberin}|pw}.}}}\pend
           
\pstart
           \begin{otherlanguage}{french}\textcolor{gray}{\textbf{Journal politique, financier,}}\end{otherlanguage}\hfill \textsc{Paris\oindex{Paris@\textbf{Paris}, \emph{P.PPLC}|pw}}, 5. December.\pend
           
\pstart
           \begin{otherlanguage}{french}\textcolor{gray}{\textbf{commercial et litteraire.}}\end{otherlanguage}\pend
           
\pstart
           \begin{otherlanguage}{french}\textcolor{gray}{\textbf{\textbf{Paraissant trois fois par jour}}}\end{otherlanguage}\pend
           
\pstart
           \begin{otherlanguage}{french}\textcolor{gray}{\textbf{\textbf{Bureaux à Paris\oindex{Paris@\textbf{Paris}, \emph{P.PPLC}|pw}:}}}\end{otherlanguage}\pend
           
\pstart
           \begin{otherlanguage}{french}\textcolor{gray}{\textbf{\textbf{rue Richelieu 75\oindex{rue Richelieu@\textbf{rue Richelieu}, \emph{Straße (K.STR)}|pw}.}}}\end{otherlanguage}\pend
           
\pstart\center{}Mein lieber Freund!\pend\vspace{0.5em}
\pstart
           Nachdem ich bisher vergeblich auf die verſprochenen Kritiken oder wenigſtens auf eine
               briefliche Mittheilung über die \textsc{Première\pwindex{Maerchen. Schauspiel in drei Aufzuegen@\emph{Das Märchen. Schauspiel in drei Aufzügen}|pwv}n}-Eindrücke
               gewartet, habe ich mir das Nöthige von Frankfurt\oindex{Frankfurt am Main@\textbf{Frankfurt am Main}, \emph{P.PPLA3}|pw}
               kommen laſſen und bitte Dich, Dich nun nicht mehr zu bemühen.\pend
           
\pstart
           Wenn ich aus der Sammlung der Kritiken, die mir vorliegt, die dummen Jungen weglaſſe
               – \label{K_L02721-1v}\edtext{\introOben{}Neue Freie Preſſe\pwindex{Theater- und Kunstnachrichten [Urauffuehrung Das Maerchen]@\emph{Theater- und Kunstnachrichten [Uraufführung Das Märchen]}|pwv},\introOben{}}{\lemma{\textnormal{\emph{Neue Freie Preſſe,}}}\Cendnote{\textnormal{[Friedrich Schütz\pwindex{Schuetz, Friedrich 24.04.1844 – 22.12.1908@\textsc{Schütz, Friedrich} (24.04.1844 – 22.12.1908), \emph{Schriftsteller/Schriftstellerin, Journalist/Journalistin}|pwk}]: \emph{Theater- und Kunstnachrichten}\pwindex{Theater- und Kunstnachrichten [Urauffuehrung Das Maerchen]@\emph{Theater- und Kunstnachrichten [Uraufführung Das Märchen]}|pwk}. In: \emph{Neue Freie Presse}\pwindex{Neue Freie Presse@\emph{Neue Freie Presse}|pwk}, Jg. 30, Nr. 10.518, 2. 12. 1893, S. 7.}}}\label{K_L02721-1}{ }\label{K_L02721-2v}\edtext{Neues Wiener Tagblatt\pwindex{Theater, Kunst und Literatur [Urauffuehrung Das Maerchen]@\emph{Theater, Kunst und Literatur [Uraufführung Das Märchen]}|pwv}}{\lemma{\textnormal{\emph{Neues Wiener Tagblatt}}}\Cendnote{\textnormal{l. h. [ = Ludwig Held]\pwindex{Held, Ludwig 1837-04-14 – 1900-03-02@\textsc{Held, Ludwig} (1837-04-14 – 1900-03-02), \emph{Theaterkritiker/Theaterkritikerin, Librettist/Librettistin}|pwk}: \emph{Theater, Kunst und Literatur}\pwindex{Theater, Kunst und Literatur [Urauffuehrung Das Maerchen]@\emph{Theater, Kunst und Literatur [Uraufführung Das Märchen]}|pwk}. In: \emph{Neues Wiener Tagblatt}\pwindex{Neues Wiener Tagblatt@\emph{Neues Wiener Tagblatt}|pwk}, Jg. 27, Nr. 333, 2. 12. 1893, S. 8.}}}\label{K_L02721-2}, \label{K_L02721-3v}\edtext{Volksblatt\pwindex{Theater, Kunst und Literatur [Urauffuehrung Das Maerchen]@\emph{Theater, Kunst und Literatur [Uraufführung Das Märchen]}|pwv}}{\lemma{\textnormal{\emph{Volksblatt}}}\Cendnote{\textnormal{H. P.\pwindex{P., H. @\textsc{P., H.}, \emph{Journalist/Journalistin}|pwk}: \emph{Theater, Kunst und Literatur}\pwindex{Theater, Kunst und Literatur [Urauffuehrung Das Maerchen]@\emph{Theater, Kunst und Literatur [Uraufführung Das Märchen]}|pwk}. In: \emph{Deutsches Volksblatt}\pwindex{Deutsches Volksblatt@\emph{Deutsches Volksblatt}|pwk}, Jg. 5, Nr. 1768, 2. 12. 1893, S. 6–7.}}}\label{K_L02721-3}, \label{K_L02721-4v}\edtext{Vaterland\pwindex{Theater und Kunst [Urauffuehrung Das Maerchen]@\emph{Theater und Kunst [Uraufführung Das Märchen]}|pwv}}{\lemma{\textnormal{\emph{Vaterland}}}\Cendnote{\textnormal{–r–\pwindex{–r– @\textsc{–r–}, \emph{Theaterkritiker/Theaterkritikerin}|pwk}: \emph{Theater und Kunst}\pwindex{Theater und Kunst [Urauffuehrung Das Maerchen]@\emph{Theater und Kunst [Uraufführung Das Märchen]}|pwk}. In: \emph{Das
                        Vaterland}\pwindex{Vaterland@\emph{Das Vaterland}|pwk}, Jg. 34, Nr. 333, 2. 12. 1893,
                     S. 7.}}}\label{K_L02721-4}{ }\textsc{etc.} – und mich nur an {\pb}die Zurechnungsfähigen halte, wie \textsc{\label{K_L02721-5v}\edtext{Uhl\pwindex{Uhl, Friedrich 14.05.1825 – 20.01.1906@\textsc{Uhl, Friedrich} (14.05.1825 – 20.01.1906), \emph{Journalist/Journalistin}|pw}\pwindex{Feuilleton. Theater [Urauffuehrung Das Maerchen]@\emph{Feuilleton. Theater [Uraufführung Das Märchen]}|pwv}}{\lemma{\textnormal{\emph{Uhl}}}\Cendnote{\textnormal{[Friedrich Uhl\pwindex{Uhl, Friedrich 14.05.1825 – 20.01.1906@\textsc{Uhl, Friedrich} (14.05.1825 – 20.01.1906), \emph{Journalist/Journalistin}|pwk}]: \emph{Feuilleton. Theater}\pwindex{Feuilleton. Theater [Urauffuehrung Das Maerchen]@\emph{Feuilleton. Theater [Uraufführung Das Märchen]}|pwk}. In: \emph{Wiener Abendpost. Beilage zur Wiener Zeitung}\pwindex{Wiener Abendpost@\emph{Wiener Abendpost}|pwk},
                        Jg. 190, Nr. 276, 2. 12. 1893,
                     S. 1–2.}}}\label{K_L02721-5}}, \textsc{\label{K_L02721-6v}\edtext{Bahr\pwindex{Bahr, Hermann 19.07.1863 – 15.01.1934@\textsc{Bahr, Hermann} (19.07.1863 – 15.01.1934), \emph{Schriftsteller/Schriftstellerin, Kritiker/Kritikerin}|pw}\pwindex{Maerchen (Schauspiel in drei Aufzuegen von Arthur Schnitzler)@\emph{Das Märchen (Schauspiel in drei Aufzügen von Arthur Schnitzler)}|pwv}}{\lemma{\textnormal{\emph{Bahr}}}\Cendnote{\textnormal{Hermann Bahr\pwindex{Bahr, Hermann 19.07.1863 – 15.01.1934@\textsc{Bahr, Hermann} (19.07.1863 – 15.01.1934), \emph{Schriftsteller/Schriftstellerin, Kritiker/Kritikerin}|pwk}: \emph{Das Märchen (Schauspiel in drei Aufzügen von Arthur
                           Schnitzler. Zum ersten Male aufgeführt am Deutschen Volkstheater den 1.
                           December)}\pwindex{Maerchen (Schauspiel in drei Aufzuegen von Arthur Schnitzler)@\emph{Das Märchen (Schauspiel in drei Aufzügen von Arthur Schnitzler)}|pwk}. In: \emph{Deutsche
                        Zeitung}\pwindex{Deutsche Zeitung@\emph{Deutsche Zeitung}|pwk}, Jg. 23, Nr. 7879, 2. 12. 1893,
                        Morgen-Ausgabe, S. 1–3.}}}\label{K_L02721-6}} und \label{K_L02721-7v}\edtext{\textsc{Brociner\pwindex{Brociner, Marco 20.10.1852 – 12.04.1942@\textsc{Brociner, Marco} (20.10.1852 – 12.04.1942), \emph{Schriftsteller/Schriftstellerin, Journalist/Journalistin, Kritiker/Kritikerin}|pw}\pwindex{Maerchen.« (Schauspiel in 3 Aufzuegen von Arthur Schnitzler. Zum erstenmale im Deutschen Volkstheater aufgefuehrt am 1. Dezember.)@\emph{»Das Märchen.« (Schauspiel in 3 Aufzügen von Arthur Schnitzler. Zum erstenmale im Deutschen Volkstheater aufgeführt am 1. Dezember.)}|pwv}}}{\lemma{\textnormal{\emph{Brociner}}}\Cendnote{\textnormal{Marco Brociner\pwindex{Brociner, Marco 20.10.1852 – 12.04.1942@\textsc{Brociner, Marco} (20.10.1852 – 12.04.1942), \emph{Schriftsteller/Schriftstellerin, Journalist/Journalistin, Kritiker/Kritikerin}|pwk}: \emph{»Das Märchen.« (Schauspiel in 3 Aufzügen von Arthur
                        Schnitzler. Zum erstenmale im Deutschen Volkstheater aufgeführt am 1.
                        Dezember.)}\pwindex{Maerchen.« (Schauspiel in 3 Aufzuegen von Arthur Schnitzler. Zum erstenmale im Deutschen Volkstheater aufgefuehrt am 1. Dezember.)@\emph{»Das Märchen.« (Schauspiel in 3 Aufzügen von Arthur Schnitzler. Zum erstenmale im Deutschen Volkstheater aufgeführt am 1. Dezember.)}|pwk} In: \emph{Wiener Tagblatt}\pwindex{Wiener Tagblatt@\emph{Wiener Tagblatt}|pwk},
                     Jg. 43, Nr. 333, 2. 12. 1893,
                  S. 1–2.}}}\label{K_L02721-7}, ſo finde ich, daß man Dich hier auch mehrfach mißverſteht,
               daß man Dir aber auch vielerlei Richtiges und Beherzigenswerthes ſagt. Beſonders \textsc{Uhl\pwindex{Uhl, Friedrich 14.05.1825 – 20.01.1906@\textsc{Uhl, Friedrich} (14.05.1825 – 20.01.1906), \emph{Journalist/Journalistin}|pw}\pwindex{Feuilleton. Theater [Urauffuehrung Das Maerchen]@\emph{Feuilleton. Theater [Uraufführung Das Märchen]}|pwv}} halte ich für im Weſentlichen richtig urtheilend. Du erinnerſt Dich, wir haben
               oft im Streit gelegen, Du und ich, und ich meine noch heute, heute erſt recht, daß
               Deinem glänzenden Talent beim Produciren die Disciplin fehlt. Auch beim Produciren
               denkſt Du ein wenig zu ſehr an Dich und zu wenig an das Andere, an die Forderungen
               der Kunſtform. Du ſchreibſt Deinem Herzeleid zuliebe und nicht {\pb}dem Drama zuliebe. Das iſt falſch. Ich komme immer
               mehr dahinter, daß das Produciren ein Streben nach möglichſter Objectivirung ſein
               muß, am allermeiſten aber das dramatiſche Produciren. Ich habe das in \textsc{Paris\oindex{Paris@\textbf{Paris}, \emph{P.PPLC}|pw}}{ }\textcolor{gray}{no}ch mehr gelernt, habe daraufhin das »Märchen\pwindex{Maerchen. Schauspiel in drei Aufzuegen@\emph{Das Märchen. Schauspiel in drei Aufzügen}|pw}« nochmals geleſen und meine Ausſtellungen von früher
               noch mehr beſtätigt gefunden. Erinnere Dich auch, was ich Dir ſtets über den \label{K_L02721-8v}\edtext{dritten Act\pwindex{Maerchen. Schauspiel in drei Aufzuegen@\emph{Das Märchen. Schauspiel in drei Aufzügen}|pwv}}{\lemma{\textnormal{\emph{dritten Act}}}\Cendnote{\textnormal{Vgl. Paul Goldmann an Arthur Schnitzler, 12. 12. [1891] und Paul Goldmann an Arthur Schnitzler, 18. 12. [1891].
               }}}\label{K_L02721-8} geſagt! Im Allgemeinen aber denke ich, daß Du mit Deinem Debüt\pwindex{Maerchen. Schauspiel in drei Aufzuegen@\emph{Das Märchen. Schauspiel in drei Aufzügen}|pwv} nicht unzufrieden ſein darfſt. Du
               biſt den Kennern ſignaliſirt; alle Leute, die es verſtehen, haben Dein großes {\pb}Talent erkannt; die dumme Bande Publicum wirſt Du
               jetzt raſch gewinnen. Aber jetzt ſofort weiter ſchreiben! Vieles lernen aus den drei
               zurechnungsfähigen Kritiken\pwindex{Feuilleton. Theater [Urauffuehrung Das Maerchen]@\emph{Feuilleton. Theater [Uraufführung Das Märchen]}|pwv}\pwindex{Maerchen (Schauspiel in drei Aufzuegen von Arthur Schnitzler)@\emph{Das Märchen (Schauspiel in drei Aufzügen von Arthur Schnitzler)}|pwv}\pwindex{Maerchen.« (Schauspiel in 3 Aufzuegen von Arthur Schnitzler. Zum erstenmale im Deutschen Volkstheater aufgefuehrt am 1. Dezember.)@\emph{»Das Märchen.« (Schauspiel in 3 Aufzügen von Arthur Schnitzler. Zum erstenmale im Deutschen Volkstheater aufgeführt am 1. Dezember.)}|pwv}! Und ein Drama machen, keine
               Beichte, kein Tagebuch! Das koſtet nur eine Willensanſtrengung. Denn Du biſt, ich
               weiß es genau, ein Dramatiker allererſten Ranges. Mach’ auch einen \label{K_L02721-9v}\edtext{neuen Verſuch mit dem \textsc{Alkandi\pwindex{Alkandi s Lied@\emph{Alkandi’s Lied}|pw}}}{\lemma{\textnormal{\emph{neuen … Alkandi}}}\Cendnote{\textnormal{Vgl. Ferdinand von Saar an Arthur Schnitzler, 5. 2. 1894 und A. S.: \emph{Tagebuch}, 8. 3. 1894.
               }}}\label{K_L02721-9}, nachdem Du vorher den Schluß verſtärk\substVorne{}\textsuperscript{t}\substDazwischen{}end\substHinten{} umgearbeitet haſt. An \textsc{Uhl\pwindex{Uhl, Friedrich 14.05.1825 – 20.01.1906@\textsc{Uhl, Friedrich} (14.05.1825 – 20.01.1906), \emph{Journalist/Journalistin}|pw}} hatte ich geſchrieben, damit er Dich nicht \label{K_L02721-10v}\edtext{in der Frkf. Ztg.\pwindex{Frankfurter Zeitung@\emph{Frankfurter Zeitung}|pw}\pwindex{Wiener Brief@\emph{Wiener Brief}|pwv}}{\lemma{\textnormal{\emph{in der Frkf. Ztg.}}}\Cendnote{\textnormal{[Friedrich Uhl\pwindex{Uhl, Friedrich 14.05.1825 – 20.01.1906@\textsc{Uhl, Friedrich} (14.05.1825 – 20.01.1906), \emph{Journalist/Journalistin}|pwk}]: \emph{Wiener Brief}\pwindex{Wiener Brief@\emph{Wiener Brief}|pwk}. In: \emph{Frankfurter Zeitung}\pwindex{Frankfurter Zeitung@\emph{Frankfurter Zeitung}|pwk}, Jg. 38, Nr. 336, 4. 12. 1893,
                     Abendblatt, S. 1. Uhl\pwindex{Uhl, Friedrich 14.05.1825 – 20.01.1906@\textsc{Uhl, Friedrich} (14.05.1825 – 20.01.1906), \emph{Journalist/Journalistin}|pwk} lobt das
                     Stück\pwindex{Maerchen. Schauspiel in drei Aufzuegen@\emph{Das Märchen. Schauspiel in drei Aufzügen}|pwkv} als
                  Habilitationsschrift, kritisiert aber den dritten Akt\pwindex{Maerchen. Schauspiel in drei Aufzuegen@\emph{Das Märchen. Schauspiel in drei Aufzügen}|pwkv}, der nicht gefallen habe.}}}\label{K_L02721-10}
               etwa ſchlecht behandle. Ich glaube, er wer ganz anſtändig?\pend
           
\pstart
           Treue Grüße! Dein \spacefill\mbox{P. G.}\pend
           \selectlanguage{ngerman}\endnumbering\briefempfaengerindex{Schnitzler, Arthur@\textsc{Schnitzler, Arthur}!zzzGoldmann, Paul@\emph{von Paul Goldmann}!1893-12-051@{5. 12. {[}1893{]}}|)be}\mylabel{L02721h}  \normalsize

\doendnotes{C}
\bigskip
\vfill

\clearpage

\footnotesize

\lohead{\textsc{register}}

% Definiere theindex-Environment komplett neu ohne reledmac
\makeatletter
\renewenvironment{theindex}{%
  \section*{\indexname}%
  \setlength{\parindent}{0pt}%
  \setlength{\parskip}{0pt plus 0.3pt}%
  \let\item\@idxitem
}{%
  \clearpage
}
\makeatother

\IfFileExists{\jobname-pw.ind}{\input{\jobname-pw.ind}}{}

\end{document}

      