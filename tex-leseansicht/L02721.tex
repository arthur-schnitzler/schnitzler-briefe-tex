%% latex-leseansicht-vorspann.tex
%% Vorspann für die Leseansicht.
%% Lädt die gemeinsame Datei latex-vorspann.tex mit nicht gesetztem Schalter.

\newif\ifkorrekturansicht
\korrekturansichtfalse

\input{../tex-inputs/latex-vorspann}


         
         \renewcommand{\erwaehntePersonen}{Personen: Hermann Bahr, Marco Brociner, Paul Goldmann, Ludwig Held, H. P., Friedrich Schütz, Leopold Sonnemann, Friedrich Uhl,  –r–}
         \renewcommand{\erwaehnteInstitutionen}{Institutionen: Frankfurter Zeitung}
         \renewcommand{\erwaehnteOrte}{Orte: Frankfurt am Main, Paris, Wien, rue Richelieu}
         \renewcommand{\erwaehnteWerke}{Werke: Alkandi’s Lied, Das Märchen (Schauspiel in drei Aufzügen von Arthur Schnitzler), Das Märchen. Schauspiel in drei Aufzügen, Das Vaterland, Deutsche Zeitung, Deutsches Volksblatt, Feuilleton. Theater [Uraufführung Das Märchen], Frankfurter Zeitung, Neue Freie Presse, Neues Wiener Tagblatt, Theater und Kunst [Uraufführung Das Märchen], Theater, Kunst und Literatur [Uraufführung Das Märchen], Theater, Kunst und Literatur [Uraufführung Das Märchen], Theater- und Kunstnachrichten [Uraufführung Das Märchen], Wiener Abendpost, Wiener Brief, Wiener Tagblatt, »Das Märchen.« (Schauspiel in 3 Aufzügen von Arthur Schnitzler. Zum erstenmale im Deutschen Volkstheater aufgeführt am 1. Dezember.)}
               \section[Paul Goldmann an Arthur Schnitzler, 5. 12. {[}1893{]}]{ Paul Goldmann an Arthur Schnitzler, 5. 12. {[}1893{]}}\nopagebreak\mylabel{v}\rehead{ }\begin{ledgroupsized}[t]{13cm}\normalsize\beginnumbering \toendnotes[C]{\smallbreak\pagebreak[2]} \Standort{DLA, A:Schnitzler, HS.NZ85.1.3163.}
\physDesc{Brief, 1 Blatt, 4 Seiten, 2175 Zeichen
\newline{}Handschrift: schwarze Tinte, deutsche Kurrent
\newline{}Schnitzler: 1) mit Bleistift das Jahr »93« vermerkt  2) mit rotem Buntstift vier Unterstreichungen}\toendnotes[C]{\smallbreak}\pstart
           \noindent{}{\pb}\textcolor{gray}{\textbf{\textbf{Frankfurter Zeitung\orgindex{Frankfurter Zeitung@Frankfurter Zeitung|pw}.}}}\pend
           \pstart
           \textcolor{gray}{\textbf{\textbf{(\begin{otherlanguage}{french}Gazette de Francfort\end{otherlanguage}\orgindex{Frankfurter Zeitung@Frankfurter Zeitung|pw}.)}}}\pend
           \pstart
           \textcolor{gray}{\textbf{\begin{otherlanguage}{french}Directeur\end{otherlanguage}{ }\textbf{M. L. Sonnemann\pwindex{Sonnemann, Leopold 1831-10-29 – 1909-10-30@\textsc{Sonnemann, Leopold} (1831-10-29 – 1909-10-30), \emph{Journalist, Herausgeber}|pw}.}}}\pend
           \pstart
           \begin{otherlanguage}{french}\textcolor{gray}{\textbf{Journal politique, financier,}}\end{otherlanguage}\hfill \textsc{Paris\oindex{Paris@\textbf{Paris}|pw}}, 5. December.\pend
           \pstart
           \begin{otherlanguage}{french}\textcolor{gray}{\textbf{commercial et litteraire.}}\end{otherlanguage}\pend
           \pstart
           \begin{otherlanguage}{french}\textcolor{gray}{\textbf{\textbf{Paraissant trois fois par jour}}}\end{otherlanguage}\pend
           \pstart
           \begin{otherlanguage}{french}\textcolor{gray}{\textbf{\textbf{Bureaux à Paris\oindex{Paris@\textbf{Paris}|pw}:}}}\end{otherlanguage}\pend
           \pstart
           \begin{otherlanguage}{french}\textcolor{gray}{\textbf{\textbf{rue Richelieu 75\oindex{rue Richelieu@\textbf{rue Richelieu}|pw}.}}}\end{otherlanguage}\pend
           \pstart\center{}Mein lieber Freund!\pend\pstart
           Nachdem ich bisher vergeblich auf die verſprochenen Kritiken oder wenigſtens auf eine
               briefliche Mittheilung über die \textsc{Première\pwindex{Schnitzler, Arthur 15.05.1862 – 21.10.1931@\textsc{Schnitzler, Arthur} (15.05.1862 – 21.10.1931), \emph{Schriftsteller, Mediziner}!Maerchen. Schauspiel in drei Aufzuegen1893-12-01@\strich\emph{Das Märchen. Schauspiel in drei Aufzügen} {[}1893-12-01{]}|pwv}n}-Eindrücke
               gewartet, habe ich mir das Nöthige von Frankfurt\oindex{Frankfurt am Main@\textbf{Frankfurt am Main}|pw}
               kommen laſſen und bitte Dich, Dich nun nicht mehr zu bemühen.\pend
           \pstart
           Wenn ich aus der Sammlung der Kritiken, die mir vorliegt, die dummen Jungen weglaſſe
               – \label{K_L02721-1v}\edtext{\introOben{}Neue Freie Preſſe\pwindex{Theater- und Kunstnachrichten [Urauffuehrung Das Maerchen]1893-12-02@\emph{Theater- und Kunstnachrichten [Uraufführung Das Märchen]} {[}1893-12-02{]}|pwv},\introOben{}}{\lemma{\textnormal{\emph{Neue Freie Preſſe,}}}\Cendnote{\textnormal{[Friedrich Schütz\pwindex{Schuetz, Friedrich 24.04.1844 – 22.12.1908@\textsc{Schütz, Friedrich} (24.04.1844 – 22.12.1908), \emph{Schriftsteller, Journalist}|pwk}]: \emph{Theater- und Kunstnachrichten}\pwindex{Theater- und Kunstnachrichten [Urauffuehrung Das Maerchen]1893-12-02@\emph{Theater- und Kunstnachrichten [Uraufführung Das Märchen]} {[}1893-12-02{]}|pwk}. In: \emph{Neue Freie Presse}\pwindex{Neue Freie Presse1864 – 1939@\emph{Neue Freie Presse} {[}1864 – 1939{]}|pwk}, Jg. 30, Nr. 10.518, 2. 12. 1893, S. 7.}}}\label{K_L02721-1h}{ }\label{K_L02721-2v}\edtext{Neues Wiener Tagblatt\pwindex{Theater, Kunst und Literatur [Urauffuehrung Das Maerchen]1893-12-02@\emph{Theater, Kunst und Literatur [Uraufführung Das Märchen]} {[}1893-12-02{]}|pwv}}{\lemma{\textnormal{\emph{Neues Wiener Tagblatt}}}\Cendnote{\textnormal{l. h. [ = Ludwig Held]\pwindex{Held, Ludwig 1837-04-14 – 1900-03-02@\textsc{Held, Ludwig} (1837-04-14 – 1900-03-02), \emph{Theaterkritiker, Librettist}|pwk}: \emph{Theater, Kunst und Literatur}\pwindex{Theater, Kunst und Literatur [Urauffuehrung Das Maerchen]1893-12-02@\emph{Theater, Kunst und Literatur [Uraufführung Das Märchen]} {[}1893-12-02{]}|pwk}. In: \emph{Neues Wiener Tagblatt}\pwindex{?? Werk@Nicht ermittelte Verfasserinnen und Verfasser!Neues Wiener Tagblatt1867 – 1945@\emph{Neues Wiener Tagblatt} {[}1867 – 1945{]}|pwk}, Jg. 27, Nr. 333, 2. 12. 1893, S. 8.}}}\label{K_L02721-2h}, \label{K_L02721-3v}\edtext{Volksblatt\pwindex{Theater, Kunst und Literatur [Urauffuehrung Das Maerchen]1893-12-02@\emph{Theater, Kunst und Literatur [Uraufführung Das Märchen]} {[}1893-12-02{]}|pwv}}{\lemma{\textnormal{\emph{Volksblatt}}}\Cendnote{\textnormal{H. P.\pwindex{P., H. @\textsc{P., H.}, \emph{Journalist/Journalistin}|pwk}: \emph{Theater, Kunst und Literatur}\pwindex{Theater, Kunst und Literatur [Urauffuehrung Das Maerchen]1893-12-02@\emph{Theater, Kunst und Literatur [Uraufführung Das Märchen]} {[}1893-12-02{]}|pwk}. In: \emph{Deutsches Volksblatt}\pwindex{?? Werk@Nicht ermittelte Verfasserinnen und Verfasser!Deutsches Volksblatt1889 – 1922@\emph{Deutsches Volksblatt} {[}1889 – 1922{]}|pwk}, Jg. 5, Nr. 1.768, 2. 12. 1893, S. 6–7.}}}\label{K_L02721-3h}, \label{K_L02721-4v}\edtext{Vaterland\pwindex{Theater und Kunst [Urauffuehrung Das Maerchen]1893-12-01@\emph{Theater und Kunst [Uraufführung Das Märchen]} {[}1893-12-01{]}|pwv}}{\lemma{\textnormal{\emph{Vaterland}}}\Cendnote{\textnormal{–r–\pwindex{–r– @\textsc{–r–}, \emph{Theaterkritiker/Theaterkritikerin}|pwk}: \emph{Theater und Kunst}\pwindex{Theater und Kunst [Urauffuehrung Das Maerchen]1893-12-01@\emph{Theater und Kunst [Uraufführung Das Märchen]} {[}1893-12-01{]}|pwk}. In: \emph{Das
                        Vaterland}\pwindex{?? Werk@Nicht ermittelte Verfasserinnen und Verfasser!Vaterland1.12.1860 – 31.12.1911@\emph{Das Vaterland} {[}1.12.1860 – 31.12.1911{]}|pwk}, Jg. 34, Nr. 333, 2. 12. 1893,
                     S. 7.}}}\label{K_L02721-4h}{ }\textsc{etc.} – und mich nur an {\pb}die Zurechnungsfähigen halte, wie \textsc{\label{K_L02721-5v}\edtext{Uhl\pwindex{Uhl, Friedrich 14.05.1825 – 20.01.1906@\textsc{Uhl, Friedrich} (14.05.1825 – 20.01.1906), \emph{Journalist}|pw}\pwindex{Feuilleton. Theater [Urauffuehrung Das Maerchen]1893-12-02@\emph{Feuilleton. Theater [Uraufführung Das Märchen]} {[}1893-12-02{]}|pwv}}{\lemma{\textnormal{\emph{Uhl}}}\Cendnote{\textnormal{[Friedrich Uhl\pwindex{Uhl, Friedrich 14.05.1825 – 20.01.1906@\textsc{Uhl, Friedrich} (14.05.1825 – 20.01.1906), \emph{Journalist}|pwk}]: \emph{Feuilleton. Theater}\pwindex{Feuilleton. Theater [Urauffuehrung Das Maerchen]1893-12-02@\emph{Feuilleton. Theater [Uraufführung Das Märchen]} {[}1893-12-02{]}|pwk}. In: \emph{Wiener Abendpost. Beilage zur Wiener Zeitung}\pwindex{?? Werk@Nicht ermittelte Verfasserinnen und Verfasser!Wiener Abendpost1.7.1863 – 31.12.1921@\emph{Wiener Abendpost} {[}1.7.1863 – 31.12.1921{]}|pwk},
                        Jg. 190, Nr. 276, 2. 12. 1893,
                     S. 1–2.}}}\label{K_L02721-5h}}, \textsc{\label{K_L02721-6v}\edtext{Bahr\pwindex{Bahr, Hermann 19.07.1863 – 15.01.1934@\textsc{Bahr, Hermann} (19.07.1863 – 15.01.1934), \emph{Schriftsteller, Kritiker}|pw}\pwindex{Bahr, Hermann 19.07.1863 – 15.01.1934@\textsc{Bahr, Hermann} (19.07.1863 – 15.01.1934), \emph{Schriftsteller, Kritiker}!Maerchen (Schauspiel in drei Aufzuegen von Arthur Schnitzler)1893-12-02@\strich\emph{Das Märchen (Schauspiel in drei Aufzügen von Arthur Schnitzler)} {[}1893-12-02{]}|pwv}}{\lemma{\textnormal{\emph{Bahr}}}\Cendnote{\textnormal{Hermann Bahr\pwindex{Bahr, Hermann 19.07.1863 – 15.01.1934@\textsc{Bahr, Hermann} (19.07.1863 – 15.01.1934), \emph{Schriftsteller, Kritiker}|pwk}: \emph{Das Märchen (Schauspiel in drei Aufzügen von Arthur
                           Schnitzler. Zum ersten Male aufgeführt am Deutschen Volkstheater den 1.
                           December)}\pwindex{Bahr, Hermann 19.07.1863 – 15.01.1934@\textsc{Bahr, Hermann} (19.07.1863 – 15.01.1934), \emph{Schriftsteller, Kritiker}!Maerchen (Schauspiel in drei Aufzuegen von Arthur Schnitzler)1893-12-02@\strich\emph{Das Märchen (Schauspiel in drei Aufzügen von Arthur Schnitzler)} {[}1893-12-02{]}|pwk}. In: \emph{Deutsche
                        Zeitung}\pwindex{?? Werk@Nicht ermittelte Verfasserinnen und Verfasser!Deutsche Zeitung1871 – 1907@\emph{Deutsche Zeitung} {[}1871 – 1907{]}|pwk}, Jg. 23, Nr. 7.879, 2. 12. 1893,
                        Morgen-Ausgabe, S. 1–3.}}}\label{K_L02721-6h}} und \label{K_L02721-7v}\edtext{\textsc{Brociner\pwindex{Brociner, Marco 20.10.1852 – 12.04.1942@\textsc{Brociner, Marco} (20.10.1852 – 12.04.1942), \emph{Schriftsteller, Journalist, Kritiker}|pw}\pwindex{Brociner, Marco 20.10.1852 – 12.04.1942@\textsc{Brociner, Marco} (20.10.1852 – 12.04.1942), \emph{Schriftsteller, Journalist, Kritiker}!Maerchen.« (Schauspiel in 3 Aufzuegen von Arthur Schnitzler. Zum erstenmale
                  im Deutschen Volkstheater aufgefuehrt am 1. Dezember.)1893-12-02@\strich\emph{»Das Märchen.« (Schauspiel in 3 Aufzügen von Arthur Schnitzler. Zum erstenmale im Deutschen Volkstheater aufgeführt am 1. Dezember.)} {[}1893-12-02{]}|pwv}}}{\lemma{\textnormal{\emph{Brociner}}}\Cendnote{\textnormal{Marco Brociner\pwindex{Brociner, Marco 20.10.1852 – 12.04.1942@\textsc{Brociner, Marco} (20.10.1852 – 12.04.1942), \emph{Schriftsteller, Journalist, Kritiker}|pwk}: \emph{»Das Märchen.« (Schauspiel in 3 Aufzügen von Arthur
                        Schnitzler. Zum erstenmale im Deutschen Volkstheater aufgeführt am 1.
                        Dezember.)}\pwindex{Brociner, Marco 20.10.1852 – 12.04.1942@\textsc{Brociner, Marco} (20.10.1852 – 12.04.1942), \emph{Schriftsteller, Journalist, Kritiker}!Maerchen.« (Schauspiel in 3 Aufzuegen von Arthur Schnitzler. Zum erstenmale
                  im Deutschen Volkstheater aufgefuehrt am 1. Dezember.)1893-12-02@\strich\emph{»Das Märchen.« (Schauspiel in 3 Aufzügen von Arthur Schnitzler. Zum erstenmale im Deutschen Volkstheater aufgeführt am 1. Dezember.)} {[}1893-12-02{]}|pwk} In: \emph{Wiener Tagblatt}\pwindex{Wiener Tagblatt1886 – 1901@\emph{Wiener Tagblatt} {[}1886 – 1901{]}|pwk},
                     Jg. 43, Nr. 333, 2. 12. 1893,
                  S. 1–2.}}}\label{K_L02721-7h}, ſo finde ich, daß man Dich hier auch mehrfach mißverſteht,
               daß man Dir aber auch vielerlei Richtiges und Beherzigenswerthes ſagt. Beſonders \textsc{Uhl\pwindex{Uhl, Friedrich 14.05.1825 – 20.01.1906@\textsc{Uhl, Friedrich} (14.05.1825 – 20.01.1906), \emph{Journalist}|pw}\pwindex{Feuilleton. Theater [Urauffuehrung Das Maerchen]1893-12-02@\emph{Feuilleton. Theater [Uraufführung Das Märchen]} {[}1893-12-02{]}|pwv}} halte ich für im Weſentlichen richtig urtheilend. Du erinnerſt Dich, wir haben
               oft im Streit gelegen, Du und ich, und ich meine noch heute, heute erſt recht, daß
               Deinem glänzenden Talent beim Produciren die Disciplin fehlt. Auch beim Produciren
               denkſt Du ein wenig zu ſehr an Dich und zu wenig an das Andere, an die Forderungen
               der Kunſtform. Du ſchreibſt Deinem Herzeleid zuliebe und nicht {\pb}dem Drama zuliebe. Das iſt falſch. Ich komme immer
               mehr dahinter, daß das Produciren ein Streben nach möglichſter Objectivirung ſein
               muß, am allermeiſten aber das dramatiſche Produciren. Ich habe das in \textsc{Paris\oindex{Paris@\textbf{Paris}|pw}}{ }\textcolor{gray}{no}ch mehr gelernt, habe daraufhin das »Märchen\pwindex{Schnitzler, Arthur 15.05.1862 – 21.10.1931@\textsc{Schnitzler, Arthur} (15.05.1862 – 21.10.1931), \emph{Schriftsteller, Mediziner}!Maerchen. Schauspiel in drei Aufzuegen1893-12-01@\strich\emph{Das Märchen. Schauspiel in drei Aufzügen} {[}1893-12-01{]}|pw}« nochmals geleſen und meine Ausſtellungen von früher
               noch mehr beſtätigt gefunden. Erinnere Dich auch, was ich Dir ſtets über den \label{K_L02721-8v}\edtext{dritten Act\pwindex{Schnitzler, Arthur 15.05.1862 – 21.10.1931@\textsc{Schnitzler, Arthur} (15.05.1862 – 21.10.1931), \emph{Schriftsteller, Mediziner}!Maerchen. Schauspiel in drei Aufzuegen1893-12-01@\strich\emph{Das Märchen. Schauspiel in drei Aufzügen} {[}1893-12-01{]}|pwv}}{\lemma{\textnormal{\emph{dritten Act}}}\Cendnote{\textnormal{vgl. Paul Goldmann an Arthur Schnitzler, 12. 12. [1891] und Paul Goldmann an Arthur Schnitzler, 18. 12. [1891]}}}\label{K_L02721-8h} geſagt! Im Allgemeinen aber denke ich, daß Du mit Deinem Debüt\pwindex{Schnitzler, Arthur 15.05.1862 – 21.10.1931@\textsc{Schnitzler, Arthur} (15.05.1862 – 21.10.1931), \emph{Schriftsteller, Mediziner}!Maerchen. Schauspiel in drei Aufzuegen1893-12-01@\strich\emph{Das Märchen. Schauspiel in drei Aufzügen} {[}1893-12-01{]}|pwv} nicht unzufrieden ſein darfſt. Du
               biſt den Kennern ſignaliſirt; alle Leute, die es verſtehen, haben Dein großes {\pb}Talent erkannt; die dumme Bande Publicum wirſt Du
               jetzt raſch gewinnen. Aber jetzt ſofort weiter ſchreiben! Vieles lernen aus den drei
               zurechnungsfähigen Kritiken\pwindex{Feuilleton. Theater [Urauffuehrung Das Maerchen]1893-12-02@\emph{Feuilleton. Theater [Uraufführung Das Märchen]} {[}1893-12-02{]}|pwv}\pwindex{Bahr, Hermann 19.07.1863 – 15.01.1934@\textsc{Bahr, Hermann} (19.07.1863 – 15.01.1934), \emph{Schriftsteller, Kritiker}!Maerchen (Schauspiel in drei Aufzuegen von Arthur Schnitzler)1893-12-02@\strich\emph{Das Märchen (Schauspiel in drei Aufzügen von Arthur Schnitzler)} {[}1893-12-02{]}|pwv}\pwindex{Brociner, Marco 20.10.1852 – 12.04.1942@\textsc{Brociner, Marco} (20.10.1852 – 12.04.1942), \emph{Schriftsteller, Journalist, Kritiker}!Maerchen.« (Schauspiel in 3 Aufzuegen von Arthur Schnitzler. Zum erstenmale
                  im Deutschen Volkstheater aufgefuehrt am 1. Dezember.)1893-12-02@\strich\emph{»Das Märchen.« (Schauspiel in 3 Aufzügen von Arthur Schnitzler. Zum erstenmale im Deutschen Volkstheater aufgeführt am 1. Dezember.)} {[}1893-12-02{]}|pwv}! Und ein Drama machen, keine
               Beichte, kein Tagebuch! Das koſtet nur eine Willensanſtrengung. Denn Du biſt, ich
               weiß es genau, ein Dramatiker allererſten Ranges. Mach’ auch einen \label{K_L02721-9v}\edtext{neuen Verſuch mit dem \textsc{Alkandi\pwindex{Schnitzler, Arthur 15.05.1862 – 21.10.1931@\textsc{Schnitzler, Arthur} (15.05.1862 – 21.10.1931), \emph{Schriftsteller, Mediziner}!Alkandi s Lied15.8.1890 – 1.9.1890@\strich\emph{Alkandi’s Lied} {[}15.8.1890 – 1.9.1890{]}|pw}}}{\lemma{\textnormal{\emph{neuen … Alkandi}}}\Cendnote{\textnormal{vgl. Ferdinand von Saar an Arthur Schnitzler, 5. 2. 1894 und A. S.: \emph{Tagebuch}, 8. 3. 1894}}}\label{K_L02721-9h}, nachdem Du vorher den Schluß verſtärk\substVorne{}\textsuperscript{t}\substDazwischen{}end\substHinten{} umgearbeitet haſt. An \textsc{Uhl\pwindex{Uhl, Friedrich 14.05.1825 – 20.01.1906@\textsc{Uhl, Friedrich} (14.05.1825 – 20.01.1906), \emph{Journalist}|pw}} hatte ich geſchrieben, damit er Dich nicht \label{K_L02721-10v}\edtext{in der Frkf. Ztg.\pwindex{?? Werk@Nicht ermittelte Verfasserinnen und Verfasser!Frankfurter Zeitung1856 – 1943@\emph{Frankfurter Zeitung} {[}1856 – 1943{]}|pw}\pwindex{Wiener Brief1893-12-04@\emph{Wiener Brief} {[}1893-12-04{]}|pwv}}{\lemma{\textnormal{\emph{in der Frkf. Ztg.}}}\Cendnote{\textnormal{[Friedrich Uhl\pwindex{Uhl, Friedrich 14.05.1825 – 20.01.1906@\textsc{Uhl, Friedrich} (14.05.1825 – 20.01.1906), \emph{Journalist}|pwk}]: \emph{Wiener Brief}\pwindex{Wiener Brief1893-12-04@\emph{Wiener Brief} {[}1893-12-04{]}|pwk}. In: \emph{Frankfurter Zeitung}\pwindex{?? Werk@Nicht ermittelte Verfasserinnen und Verfasser!Frankfurter Zeitung1856 – 1943@\emph{Frankfurter Zeitung} {[}1856 – 1943{]}|pwk}, Jg. 38, Nr. 336, 4. 12. 1893,
                     Abendblatt, S. 1. Uhl\pwindex{Uhl, Friedrich 14.05.1825 – 20.01.1906@\textsc{Uhl, Friedrich} (14.05.1825 – 20.01.1906), \emph{Journalist}|pwk} lobt das
                     Stück\pwindex{Schnitzler, Arthur 15.05.1862 – 21.10.1931@\textsc{Schnitzler, Arthur} (15.05.1862 – 21.10.1931), \emph{Schriftsteller, Mediziner}!Maerchen. Schauspiel in drei Aufzuegen1893-12-01@\strich\emph{Das Märchen. Schauspiel in drei Aufzügen} {[}1893-12-01{]}|pwkv} als
                  Habilitationsschrift, kritisiert aber den dritten Akt\pwindex{Schnitzler, Arthur 15.05.1862 – 21.10.1931@\textsc{Schnitzler, Arthur} (15.05.1862 – 21.10.1931), \emph{Schriftsteller, Mediziner}!Maerchen. Schauspiel in drei Aufzuegen1893-12-01@\strich\emph{Das Märchen. Schauspiel in drei Aufzügen} {[}1893-12-01{]}|pwkv}, der nicht gefallen habe.}}}\label{K_L02721-10h}
               etwa ſchlecht behandle. Ich glaube, er wer ganz anſtändig?\pend
           \pstart
           Treue Grüße! Dein \spacefill\mbox{P. G.}\pend
           
         
         \endnumbering\mylabel{h}\end{ledgroupsized}  \newcommand{\dateiname}{L02721}\newcommand{\titel}{Paul Goldmann an Arthur Schnitzler, 5. 12. [1893]}\newcommand{\editorInnen}{Martin Anton Müller und Laura Untner}%% latex-leseansicht-abspann.tex
%% Abspann für die Leseansicht.
%% Der Schalter \ifkorrekturansicht ist bereits durch den Vorspann gesetzt.

%% latex-abspann.tex
%% Gemeinsamer Abspann für Korrekturansicht und Leseansicht.
%% Setzt den Schalter \ifkorrekturansicht voraus (gesetzt in den
%% einbindenden Dateien latex-korrekturansicht-abspann.tex bzw.
%% latex-leseansicht-abspann.tex).
%% ---------------------------------------------------------------

\normalsize

% Das esempio-Environment wird nur in der Leseansicht benötigt
\ifkorrekturansicht\else
\newenvironment{esempio}[3]%
{
    \vspace{1.5ex}
    \rlap{\underline{#1}}
    \par
    \setlength{\parindent}{0cm}
    \nopagebreak
    \leftskip=#2cm
    \rightskip=#3cm
}
{
    \par
}
\fi

\doendnotes{C}
\bigskip
\vfill

\clearpage

\footnotesize

\ifkorrekturansicht
  \lohead{\textsc{register}}
\fi

% theindex-Environment neu definieren ohne reledmac
\makeatletter
\renewenvironment{theindex}{%
  \ifkorrekturansicht
    \section*{\indexname}%
  \else
    \subsubsection*{Index der erwähnten Entitäten}%
  \fi
  \setlength{\parindent}{0pt}%
  \setlength{\parskip}{0pt plus 0.3pt}%
  \let\item\@idxitem
}{%
  \ifkorrekturansicht\clearpage\fi
}
\makeatother

\IfFileExists{\jobname-pw.ind}{\input{\jobname-pw.ind}}{}

% Quellenangabe nur in der Leseansicht
\ifkorrekturansicht\else
% Fallback-Definitionen, falls die .tex-Datei \titel etc. nicht gesetzt hat
\providecommand{\titel}{}
\providecommand{\editorInnen}{}
\providecommand{\dateiname}{\jobname}

\vspace{3cm}

\vfill

\footnotesize
\textsc{Quelle}: \titel. Herausgegeben von {\editorInnen}. In: \emph{Arthur Schnitzler: Briefwechsel mit Autorinnen und Autoren}.
 Digitale Edition, https://schnitzler-briefe.acdh.oeaw.ac.at/{\dateiname}.html (Stand \today)
\fi

\end{document}


      