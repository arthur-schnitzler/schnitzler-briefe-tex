%% latex-leseansicht-vorspann.tex
%% Vorspann für die Leseansicht.
%% Lädt die gemeinsame Datei latex-vorspann.tex mit nicht gesetztem Schalter.

\newif\ifkorrekturansicht
\korrekturansichtfalse

\input{../tex-inputs/latex-vorspann}


\section[Paul Goldmann an Arthur Schnitzler, 5. 12. {[}1893{]}]{L02721 Paul Goldmann an Arthur Schnitzler, 5. 12. [1893]}
\nopagebreak\mylabel{L02721v}
\rehead{ }\normalsize\beginnumbering\briefempfaengerindex{Schnitzler, Arthur@\textsc{Schnitzler, Arthur}!zzzGoldmann, Paul@\emph{von Paul Goldmann}!1893-12-051@{5. 12. [1893]}|(be}
\toendnotes[C]{\smallbreak\pagebreak[2]}
\correspDesc{Versand  durch Paul Goldmann am 5. 12. [1893] in Paris
\newline{}Erhalt  durch Arthur Schnitzler im Zeitraum [6. 12. 1893
                  – 10. 12. 1893?] in Wien}\toendnotes[C]{\smallbreak}
\Standort{DLA, A:Schnitzler, HS.NZ85.1.3163.}
\physDesc{Brief, 1 Blatt, 4 Seiten, 2175 Zeichen
\newline{}Handschrift: schwarze Tinte, deutsche Kurrent
\newline{}Schnitzler: 1) mit Bleistift das Jahr »93« vermerkt  2) mit rotem Buntstift vier Unterstreichungen}\toendnotes[C]{\smallbreak}
\pstart
           {\pb}\textcolor{gray}{\textbf{\textbf{Frankfurter Zeitung\orgindex{Frankfurter Zeitung@Frankfurter Zeitung|pw}.}}}\pend
           
\pstart
           \textcolor{gray}{\textbf{\textbf{(\begin{otherlanguage}{french}Gazette de Francfort\end{otherlanguage}\orgindex{Frankfurter Zeitung@Frankfurter Zeitung|pw}.)}}}\pend
           
\pstart
           \textcolor{gray}{\textbf{\begin{otherlanguage}{french}Directeur\end{otherlanguage}{ }\textbf{M. L. Sonnemann\pwindex{Sonnemann, Leopold 29.\,10.\,1831 Höchberg – 30.\,10.\,1909 Frankfurt am Main@\textsc{Sonnemann, Leopold} (29.\,10.\,1831 Höchberg – 30.\,10.\,1909 Frankfurt am Main), \emph{Journalist, Herausgeber}|pw}.}}}\pend
           
\pstart
           \begin{otherlanguage}{french}\textcolor{gray}{\textbf{Journal politique, financier,}}\end{otherlanguage}\hfill \textsc{Paris\oindex{Paris@\textbf{Paris}, \emph{Hauptstadt}|pw}}, 5. December.\pend
           
\pstart
           \begin{otherlanguage}{french}\textcolor{gray}{\textbf{commercial et litteraire.}}\end{otherlanguage}\pend
           
\pstart
           \begin{otherlanguage}{french}\textcolor{gray}{\textbf{\textbf{Paraissant trois fois par jour}}}\end{otherlanguage}\pend
           
\pstart
           \begin{otherlanguage}{french}\textcolor{gray}{\textbf{\textbf{Bureaux à Paris\oindex{Paris@\textbf{Paris}, \emph{Hauptstadt}|pw}:}}}\end{otherlanguage}\pend
           
\pstart
           \begin{otherlanguage}{french}\textcolor{gray}{\textbf{\textbf{rue Richelieu 75\oindex{rue Richelieu@\textbf{rue Richelieu}, \emph{Straße}|pw}.}}}\end{otherlanguage}\pend
           
\pstart\center{}Mein lieber Freund!\pend\vspace{0.5em}
\pstart
           Nachdem ich bisher vergeblich auf die verſprochenen Kritiken oder wenigſtens auf eine
               briefliche Mittheilung über die \textsc{Première\pwindex{Schnitzler, Arthur 15.\,5.\,1862 Wien – 21.\,10.\,1931 ebd.@\textsc{Schnitzler, Arthur} (15.\,5.\,1862 Wien – 21.\,10.\,1931 ebd.), \emph{Schriftsteller, Mediziner}!Märchen. Schauspiel in drei Aufzügen@\strich\emph{Das Märchen. Schauspiel in drei Aufzügen}|pwv}n}-Eindrücke
               gewartet, habe ich mir das Nöthige von Frankfurt\oindex{Frankfurt am Main@\textbf{Frankfurt am Main}, \emph{Hauptstadt}|pw}
               kommen laſſen und bitte Dich, Dich nun nicht mehr zu bemühen.\pend
           
\pstart
           Wenn ich aus der Sammlung der Kritiken, die mir vorliegt, die dummen Jungen weglaſſe
               – \label{K_L02721-1v}\edtext{\introOben{}Neue Freie Preſſe\pwindex{Theater- und Kunstnachrichten [Uraufführung Das Märchen]@\emph{Theater- und Kunstnachrichten [Uraufführung Das Märchen]}|pwv},\introOben{}}{\lemma{\textnormal{\emph{Neue Freie Presse,}}}\Cendnote{\textnormal{[Friedrich Schütz\pwindex{Schütz, Friedrich 24.\,4.\,1844 Prag – 22.\,12.\,1908 Wien@\textsc{Schütz, Friedrich} (24.\,4.\,1844 Prag – 22.\,12.\,1908 Wien), \emph{Schriftsteller, Journalist}|pwk}]: \emph{Theater- und Kunstnachrichten}\pwindex{Theater- und Kunstnachrichten [Uraufführung Das Märchen]@\emph{Theater- und Kunstnachrichten [Uraufführung Das Märchen]}|pwk}. In: \emph{Neue Freie Presse}\pwindex{Neue Freie Presse@\emph{Neue Freie Presse}|pwk}, Jg. 30, Nr. 10.518, 2. 12. 1893, S. 7.}}}\label{K_L02721-1}{ }\label{K_L02721-2v}\edtext{Neues Wiener Tagblatt\pwindex{Held, Ludwig 14.\,4.\,1837 Regensburg – 2.\,3.\,1900 Wien@\textsc{Held, Ludwig} (14.\,4.\,1837 Regensburg – 2.\,3.\,1900 Wien), \emph{Theaterkritiker, Librettist}!Theater, Kunst und Literatur [Uraufführung Das Märchen]@\strich\emph{Theater, Kunst und Literatur [Uraufführung Das Märchen]}|pwv}}{\lemma{\textnormal{\emph{Neues Wiener Tagblatt}}}\Cendnote{\textnormal{l. h. [ = Ludwig Held]\pwindex{Held, Ludwig 14.\,4.\,1837 Regensburg – 2.\,3.\,1900 Wien@\textsc{Held, Ludwig} (14.\,4.\,1837 Regensburg – 2.\,3.\,1900 Wien), \emph{Theaterkritiker, Librettist}|pwk}: \emph{Theater, Kunst und Literatur}\pwindex{Held, Ludwig 14.\,4.\,1837 Regensburg – 2.\,3.\,1900 Wien@\textsc{Held, Ludwig} (14.\,4.\,1837 Regensburg – 2.\,3.\,1900 Wien), \emph{Theaterkritiker, Librettist}!Theater, Kunst und Literatur [Uraufführung Das Märchen]@\strich\emph{Theater, Kunst und Literatur [Uraufführung Das Märchen]}|pwk}. In: \emph{Neues Wiener Tagblatt}\pwindex{Neues Wiener Tagblatt@\emph{Neues Wiener Tagblatt}|pwk}, Jg. 27, Nr. 333, 2. 12. 1893, S. 8.}}}\label{K_L02721-2}, \label{K_L02721-3v}\edtext{Volksblatt\pwindex{P., H. @\textsc{P., H.}, \emph{Journalist/Journalistin}!Theater, Kunst und Literatur [Uraufführung Das Märchen]@\strich\emph{Theater, Kunst und Literatur [Uraufführung Das Märchen]}|pwv}}{\lemma{\textnormal{\emph{Volksblatt}}}\Cendnote{\textnormal{H. P.\pwindex{P., H. @\textsc{P., H.}, \emph{Journalist/Journalistin}|pwk}: \emph{Theater, Kunst und Literatur}\pwindex{P., H. @\textsc{P., H.}, \emph{Journalist/Journalistin}!Theater, Kunst und Literatur [Uraufführung Das Märchen]@\strich\emph{Theater, Kunst und Literatur [Uraufführung Das Märchen]}|pwk}. In: \emph{Deutsches Volksblatt}\pwindex{Deutsches Volksblatt@\emph{Deutsches Volksblatt}|pwk}, Jg. 5, Nr. 1768, 2. 12. 1893, S. 6–7.}}}\label{K_L02721-3}, \label{K_L02721-4v}\edtext{Vaterland\pwindex{–r– @\textsc{–r–}, \emph{Theaterkritiker/Theaterkritikerin}!Theater und Kunst [Uraufführung Das Märchen]@\strich\emph{Theater und Kunst [Uraufführung Das Märchen]}|pwv}}{\lemma{\textnormal{\emph{Vaterland}}}\Cendnote{\textnormal{–r–\pwindex{–r– @\textsc{–r–}, \emph{Theaterkritiker/Theaterkritikerin}|pwk}: \emph{Theater und Kunst}\pwindex{–r– @\textsc{–r–}, \emph{Theaterkritiker/Theaterkritikerin}!Theater und Kunst [Uraufführung Das Märchen]@\strich\emph{Theater und Kunst [Uraufführung Das Märchen]}|pwk}. In: \emph{Das
                        Vaterland}\pwindex{Vaterland@\emph{Das Vaterland}|pwk}, Jg. 34, Nr. 333, 2. 12. 1893,
                     S. 7.}}}\label{K_L02721-4}{ }\textsc{etc.} – und mich nur an {\pb}die Zurechnungsfähigen halte, wie \textsc{\label{K_L02721-5v}\edtext{Uhl\pwindex{Uhl, Friedrich 14.\,5.\,1825 Cieszyn – 20.\,1.\,1906 Mondsee@\textsc{Uhl, Friedrich} (14.\,5.\,1825 Cieszyn – 20.\,1.\,1906 Mondsee), \emph{Journalist}|pw}\pwindex{Feuilleton. Theater [Uraufführung Das Märchen]@\emph{Feuilleton. Theater [Uraufführung Das Märchen]}|pwv}}{\lemma{\textnormal{\emph{Uhl}}}\Cendnote{\textnormal{[Friedrich Uhl\pwindex{Uhl, Friedrich 14.\,5.\,1825 Cieszyn – 20.\,1.\,1906 Mondsee@\textsc{Uhl, Friedrich} (14.\,5.\,1825 Cieszyn – 20.\,1.\,1906 Mondsee), \emph{Journalist}|pwk}]: \emph{Feuilleton. Theater}\pwindex{Feuilleton. Theater [Uraufführung Das Märchen]@\emph{Feuilleton. Theater [Uraufführung Das Märchen]}|pwk}. In: \emph{Wiener Abendpost. Beilage zur Wiener Zeitung}\pwindex{Wiener Abendpost@\emph{Wiener Abendpost}|pwk},
                        Jg. 190, Nr. 276, 2. 12. 1893,
                     S. 1–2.}}}\label{K_L02721-5}}, \textsc{\label{K_L02721-6v}\edtext{Bahr\pwindex{Bahr, Hermann 19.\,7.\,1863 Linz – 15.\,1.\,1934 München@\textsc{Bahr, Hermann} (19.\,7.\,1863 Linz – 15.\,1.\,1934 München), \emph{Schriftsteller, Kritiker}|pw}\pwindex{Bahr, Hermann 19.\,7.\,1863 Linz – 15.\,1.\,1934 München@\textsc{Bahr, Hermann} (19.\,7.\,1863 Linz – 15.\,1.\,1934 München), \emph{Schriftsteller, Kritiker}!Märchen (Schauspiel in drei Aufzügen von Arthur Schnitzler)@\strich\emph{Das Märchen (Schauspiel in drei Aufzügen von Arthur Schnitzler)}|pwv}}{\lemma{\textnormal{\emph{Bahr}}}\Cendnote{\textnormal{Hermann Bahr\pwindex{Bahr, Hermann 19.\,7.\,1863 Linz – 15.\,1.\,1934 München@\textsc{Bahr, Hermann} (19.\,7.\,1863 Linz – 15.\,1.\,1934 München), \emph{Schriftsteller, Kritiker}|pwk}: \emph{Das Märchen (Schauspiel in drei Aufzügen von Arthur
                           Schnitzler. Zum ersten Male aufgeführt am Deutschen Volkstheater den 1.
                           December)}\pwindex{Bahr, Hermann 19.\,7.\,1863 Linz – 15.\,1.\,1934 München@\textsc{Bahr, Hermann} (19.\,7.\,1863 Linz – 15.\,1.\,1934 München), \emph{Schriftsteller, Kritiker}!Märchen (Schauspiel in drei Aufzügen von Arthur Schnitzler)@\strich\emph{Das Märchen (Schauspiel in drei Aufzügen von Arthur Schnitzler)}|pwk}. In: \emph{Deutsche
                        Zeitung}\pwindex{Deutsche Zeitung@\emph{Deutsche Zeitung}|pwk}, Jg. 23, Nr. 7879, 2. 12. 1893,
                        Morgen-Ausgabe, S. 1–3.}}}\label{K_L02721-6}} und \label{K_L02721-7v}\edtext{\textsc{Brociner\pwindex{Brociner, Marco 20.\,10.\,1852 Iași – 12.\,4.\,1942 Wien@\textsc{Brociner, Marco} (20.\,10.\,1852 Iași – 12.\,4.\,1942 Wien), \emph{Schriftsteller, Journalist, Kritiker}|pw}\pwindex{Brociner, Marco 20.\,10.\,1852 Iași – 12.\,4.\,1942 Wien@\textsc{Brociner, Marco} (20.\,10.\,1852 Iași – 12.\,4.\,1942 Wien), \emph{Schriftsteller, Journalist, Kritiker}!Märchen.« (Schauspiel in 3 Aufzügen von Arthur Schnitzler. Zum erstenmale im Deutschen Volkstheater aufgeführt am 1. Dezember.)@\strich\emph{»Das Märchen.« (Schauspiel in 3 Aufzügen von Arthur Schnitzler. Zum erstenmale im Deutschen Volkstheater aufgeführt am 1. Dezember.)}|pwv}}}{\lemma{\textnormal{\emph{Brociner}}}\Cendnote{\textnormal{Marco Brociner\pwindex{Brociner, Marco 20.\,10.\,1852 Iași – 12.\,4.\,1942 Wien@\textsc{Brociner, Marco} (20.\,10.\,1852 Iași – 12.\,4.\,1942 Wien), \emph{Schriftsteller, Journalist, Kritiker}|pwk}: \emph{»Das Märchen.« (Schauspiel in 3 Aufzügen von Arthur
                        Schnitzler. Zum erstenmale im Deutschen Volkstheater aufgeführt am 1.
                        Dezember.)}\pwindex{Brociner, Marco 20.\,10.\,1852 Iași – 12.\,4.\,1942 Wien@\textsc{Brociner, Marco} (20.\,10.\,1852 Iași – 12.\,4.\,1942 Wien), \emph{Schriftsteller, Journalist, Kritiker}!Märchen.« (Schauspiel in 3 Aufzügen von Arthur Schnitzler. Zum erstenmale im Deutschen Volkstheater aufgeführt am 1. Dezember.)@\strich\emph{»Das Märchen.« (Schauspiel in 3 Aufzügen von Arthur Schnitzler. Zum erstenmale im Deutschen Volkstheater aufgeführt am 1. Dezember.)}|pwk} In: \emph{Wiener Tagblatt}\pwindex{Wiener Tagblatt@\emph{Wiener Tagblatt}|pwk},
                     Jg. 43, Nr. 333, 2. 12. 1893,
                  S. 1–2.}}}\label{K_L02721-7},{ }ſo finde ich, daß man Dich hier auch mehrfach mißverſteht,
               daß man Dir aber auch vielerlei Richtiges und Beherzigenswerthes{ }ſagt. Beſonders \textsc{Uhl\pwindex{Uhl, Friedrich 14.\,5.\,1825 Cieszyn – 20.\,1.\,1906 Mondsee@\textsc{Uhl, Friedrich} (14.\,5.\,1825 Cieszyn – 20.\,1.\,1906 Mondsee), \emph{Journalist}|pw}\pwindex{Feuilleton. Theater [Uraufführung Das Märchen]@\emph{Feuilleton. Theater [Uraufführung Das Märchen]}|pwv}} halte ich für im Weſentlichen richtig urtheilend. Du erinnerſt Dich, wir haben
               oft im Streit gelegen, Du und ich, und ich meine noch heute, heute erſt recht, daß
               Deinem glänzenden Talent beim Produciren die Disciplin fehlt. Auch beim Produciren
               denkſt Du ein wenig zu{ }ſehr an Dich und zu wenig an das Andere, an die Forderungen
               der Kunſtform. Du{ }ſchreibſt Deinem Herzeleid zuliebe und nicht {\pb}dem Drama zuliebe. Das iſt falſch. Ich komme immer
               mehr dahinter, daß das Produciren ein Streben nach möglichſter Objectivirung{ }ſein
               muß, am allermeiſten aber das dramatiſche Produciren. Ich habe das in \textsc{Paris\oindex{Paris@\textbf{Paris}, \emph{Hauptstadt}|pw}}{ }\textcolor{gray}{no}ch mehr gelernt, habe daraufhin das »Märchen\pwindex{Schnitzler, Arthur 15.\,5.\,1862 Wien – 21.\,10.\,1931 ebd.@\textsc{Schnitzler, Arthur} (15.\,5.\,1862 Wien – 21.\,10.\,1931 ebd.), \emph{Schriftsteller, Mediziner}!Märchen. Schauspiel in drei Aufzügen@\strich\emph{Das Märchen. Schauspiel in drei Aufzügen}|pw}« nochmals geleſen und meine Ausſtellungen von früher
               noch mehr beſtätigt gefunden. Erinnere Dich auch, was ich Dir{ }ſtets über den \label{K_L02721-8v}\edtext{dritten Act\pwindex{Schnitzler, Arthur 15.\,5.\,1862 Wien – 21.\,10.\,1931 ebd.@\textsc{Schnitzler, Arthur} (15.\,5.\,1862 Wien – 21.\,10.\,1931 ebd.), \emph{Schriftsteller, Mediziner}!Märchen. Schauspiel in drei Aufzügen@\strich\emph{Das Märchen. Schauspiel in drei Aufzügen}|pwv}}{\lemma{\textnormal{\emph{dritten Act}}}\Cendnote{\textnormal{Vgl. XXXX Auszeichnungsfehler: Dokument L02674 nicht gefunden und XXXX Auszeichnungsfehler: Dokument L02675 nicht gefunden.
               }}}\label{K_L02721-8} geſagt! Im Allgemeinen aber denke ich, daß Du mit Deinem Debüt\pwindex{Schnitzler, Arthur 15.\,5.\,1862 Wien – 21.\,10.\,1931 ebd.@\textsc{Schnitzler, Arthur} (15.\,5.\,1862 Wien – 21.\,10.\,1931 ebd.), \emph{Schriftsteller, Mediziner}!Märchen. Schauspiel in drei Aufzügen@\strich\emph{Das Märchen. Schauspiel in drei Aufzügen}|pwv} nicht unzufrieden{ }ſein darfſt. Du
               biſt den Kennern{ }ſignaliſirt; alle Leute, die es verſtehen, haben Dein großes {\pb}Talent erkannt; die dumme Bande Publicum wirſt Du
               jetzt raſch gewinnen. Aber jetzt{ }ſofort weiter{ }ſchreiben! Vieles lernen aus den drei
               zurechnungsfähigen Kritiken\pwindex{Feuilleton. Theater [Uraufführung Das Märchen]@\emph{Feuilleton. Theater [Uraufführung Das Märchen]}|pwv}\pwindex{Bahr, Hermann 19.\,7.\,1863 Linz – 15.\,1.\,1934 München@\textsc{Bahr, Hermann} (19.\,7.\,1863 Linz – 15.\,1.\,1934 München), \emph{Schriftsteller, Kritiker}!Märchen (Schauspiel in drei Aufzügen von Arthur Schnitzler)@\strich\emph{Das Märchen (Schauspiel in drei Aufzügen von Arthur Schnitzler)}|pwv}\pwindex{Brociner, Marco 20.\,10.\,1852 Iași – 12.\,4.\,1942 Wien@\textsc{Brociner, Marco} (20.\,10.\,1852 Iași – 12.\,4.\,1942 Wien), \emph{Schriftsteller, Journalist, Kritiker}!Märchen.« (Schauspiel in 3 Aufzügen von Arthur Schnitzler. Zum erstenmale im Deutschen Volkstheater aufgeführt am 1. Dezember.)@\strich\emph{»Das Märchen.« (Schauspiel in 3 Aufzügen von Arthur Schnitzler. Zum erstenmale im Deutschen Volkstheater aufgeführt am 1. Dezember.)}|pwv}! Und ein Drama machen, keine
               Beichte, kein Tagebuch! Das koſtet nur eine Willensanſtrengung. Denn Du biſt, ich
               weiß es genau, ein Dramatiker allererſten Ranges. Mach’ auch einen \label{K_L02721-9v}\edtext{neuen Verſuch mit dem \textsc{Alkandi\pwindex{Schnitzler, Arthur 15.\,5.\,1862 Wien – 21.\,10.\,1931 ebd.@\textsc{Schnitzler, Arthur} (15.\,5.\,1862 Wien – 21.\,10.\,1931 ebd.), \emph{Schriftsteller, Mediziner}!Alkandi’s Lied@\strich\emph{Alkandi’s Lied}|pw}}}{\lemma{\textnormal{\emph{neuen … Alkandi}}}\Cendnote{\textnormal{Vgl. XXXX Auszeichnungsfehler: Dokument L00296 nicht gefunden und A. S.: \emph{Tagebuch}, 8. 3. 1894.
               }}}\label{K_L02721-9}, nachdem Du vorher den Schluß verſtärk\substVorne{}\textsuperscript{t}\substDazwischen{}end\substHinten{} umgearbeitet haſt. An \textsc{Uhl\pwindex{Uhl, Friedrich 14.\,5.\,1825 Cieszyn – 20.\,1.\,1906 Mondsee@\textsc{Uhl, Friedrich} (14.\,5.\,1825 Cieszyn – 20.\,1.\,1906 Mondsee), \emph{Journalist}|pw}} hatte ich geſchrieben, damit er Dich nicht \label{K_L02721-10v}\edtext{in der Frkf. Ztg.\pwindex{Frankfurter Zeitung@\emph{Frankfurter Zeitung}|pw}\pwindex{Wiener Brief@\emph{Wiener Brief}|pwv}}{\lemma{\textnormal{\emph{in der Frkf. Ztg.}}}\Cendnote{\textnormal{[Friedrich Uhl\pwindex{Uhl, Friedrich 14.\,5.\,1825 Cieszyn – 20.\,1.\,1906 Mondsee@\textsc{Uhl, Friedrich} (14.\,5.\,1825 Cieszyn – 20.\,1.\,1906 Mondsee), \emph{Journalist}|pwk}]: \emph{Wiener Brief}\pwindex{Wiener Brief@\emph{Wiener Brief}|pwk}. In: \emph{Frankfurter Zeitung}\pwindex{Frankfurter Zeitung@\emph{Frankfurter Zeitung}|pwk}, Jg. 38, Nr. 336, 4. 12. 1893,
                     Abendblatt, S. 1. Uhl\pwindex{Uhl, Friedrich 14.\,5.\,1825 Cieszyn – 20.\,1.\,1906 Mondsee@\textsc{Uhl, Friedrich} (14.\,5.\,1825 Cieszyn – 20.\,1.\,1906 Mondsee), \emph{Journalist}|pwk} lobt das
                     Stück\pwindex{Schnitzler, Arthur 15.\,5.\,1862 Wien – 21.\,10.\,1931 ebd.@\textsc{Schnitzler, Arthur} (15.\,5.\,1862 Wien – 21.\,10.\,1931 ebd.), \emph{Schriftsteller, Mediziner}!Märchen. Schauspiel in drei Aufzügen@\strich\emph{Das Märchen. Schauspiel in drei Aufzügen}|pwkv} als
                  Habilitationsschrift, kritisiert aber den dritten Akt\pwindex{Schnitzler, Arthur 15.\,5.\,1862 Wien – 21.\,10.\,1931 ebd.@\textsc{Schnitzler, Arthur} (15.\,5.\,1862 Wien – 21.\,10.\,1931 ebd.), \emph{Schriftsteller, Mediziner}!Märchen. Schauspiel in drei Aufzügen@\strich\emph{Das Märchen. Schauspiel in drei Aufzügen}|pwkv}, der nicht gefallen habe.}}}\label{K_L02721-10}
               etwa{ }ſchlecht behandle. Ich glaube, er wer ganz anſtändig?\pend
           
\pstart
           Treue Grüße! Dein \spacefill\mbox{P. G.}\pend
           \selectlanguage{ngerman}\endnumbering\briefempfaengerindex{Schnitzler, Arthur@\textsc{Schnitzler, Arthur}!zzzGoldmann, Paul@\emph{von Paul Goldmann}!1893-12-051@{5. 12. [1893]}|)be}\mylabel{L02721h}  \newcommand{\dateiname}{L02721}\newcommand{\titel}{Paul Goldmann an Arthur Schnitzler, 5. 12. [1893]}\newcommand{\editorInnen}{Martin Anton Müller und Laura Untner}%% latex-leseansicht-abspann.tex
%% Abspann für die Leseansicht.
%% Der Schalter \ifkorrekturansicht ist bereits durch den Vorspann gesetzt.

%% latex-abspann.tex
%% Gemeinsamer Abspann für Korrekturansicht und Leseansicht.
%% Setzt den Schalter \ifkorrekturansicht voraus (gesetzt in den
%% einbindenden Dateien latex-korrekturansicht-abspann.tex bzw.
%% latex-leseansicht-abspann.tex).
%% ---------------------------------------------------------------

\normalsize

% Das esempio-Environment wird nur in der Leseansicht benötigt
\ifkorrekturansicht\else
\newenvironment{esempio}[3]%
{
    \vspace{1.5ex}
    \rlap{\underline{#1}}
    \par
    \setlength{\parindent}{0cm}
    \nopagebreak
    \leftskip=#2cm
    \rightskip=#3cm
}
{
    \par
}
\fi

\doendnotes{C}
\bigskip
\vfill

\clearpage

\footnotesize

\ifkorrekturansicht
  \lohead{\textsc{register}}
\fi

% theindex-Environment neu definieren ohne reledmac
\makeatletter
\renewenvironment{theindex}{%
  \ifkorrekturansicht
    \section*{\indexname}%
  \else
    \subsubsection*{Index der erwähnten Entitäten}%
  \fi
  \setlength{\parindent}{0pt}%
  \setlength{\parskip}{0pt plus 0.3pt}%
  \let\item\@idxitem
}{%
  \ifkorrekturansicht\clearpage\fi
}
\makeatother

\IfFileExists{\jobname-pw.ind}{\input{\jobname-pw.ind}}{}

% Quellenangabe nur in der Leseansicht
\ifkorrekturansicht\else
% Fallback-Definitionen, falls die .tex-Datei \titel etc. nicht gesetzt hat
\providecommand{\titel}{}
\providecommand{\editorInnen}{}
\providecommand{\dateiname}{\jobname}

\vspace{3cm}

\vfill

\footnotesize
\textsc{Quelle}: \titel. Herausgegeben von {\editorInnen}. In: \emph{Arthur Schnitzler: Briefwechsel mit Autorinnen und Autoren}.
 Digitale Edition, https://schnitzler-briefe.acdh.oeaw.ac.at/{\dateiname}.html (Stand \today)
\fi

\end{document}


