%% latex-korrekturansicht-vorspann.tex
%% Vorspann für die Korrekturansicht.
%% Lädt die gemeinsame Datei latex-vorspann.tex mit gesetztem Schalter.

\newif\ifkorrekturansicht
\korrekturansichttrue

\input{../tex-inputs/latex-vorspann}


\section[Arthur Schnitzler an Stefan Zweig, 7. 11. 1926]{L03811 Arthur Schnitzler an Stefan Zweig, 7. 11. 1926}
\nopagebreak\mylabel{L03811v}
\rehead{ }\normalsize\beginnumbering\briefempfaengerindex{Zweig, Stefan@\textsc{Zweig, Stefan}!zzzSchnitzler, Arthur@\emph{von Arthur Schnitzler}!1926-11-071@{7. 11. 1926}|(be}
\toendnotes[C]{\smallbreak\pagebreak[2]}\Standort{Jerusalem, National Library of Israel, ARC. Ms. Var. 305 1 58 Stefan Zweig Collection.}
\physDesc{Brief, 1 Blatt, 2 Seiten, 825 Zeichen
\newline{}Handschrift: schwarze Tinte, deutsche Kurrent}\toendnotes[C]{\smallbreak}
\pstart
           {\pb}Wien\oindex{Wien@\textbf{Wien}, \emph{A.ADM2}|pw}, 7. 11. 926\pend
           
\pstart{}lieber Herr Doctor,\pend\vspace{0.5em}
\pstart
           ich dachte Sie doch wenigstens bei der \label{K_L03811-1v}\edtext{Generalprobe\eventindex{Burgtheater@\textbf{Burgtheater}!Generalprobe von Volpone oder Der Fuchs, 5.11.1926@Generalprobe von Volpone oder Der Fuchs, 5.11.1926|pwv}}{\lemma{\textnormal{\emph{Generalprobe}}}\Cendnote{\textnormal{Schnitzler besuchte am 5. 11. 1926 die \emph{Generalprobe von \emph{Ben Jonsons »Volpone«}\pwindex{Ben Jonsons »Volpone« Eine lieblose Komoedie in drei Akten@\emph{Ben Jonsons »Volpone« Eine lieblose Komödie in drei Akten}|pwk}}\eventindex{Burgtheater@\textbf{Burgtheater}!Generalprobe von Volpone oder Der Fuchs, 5.11.1926@Generalprobe von Volpone oder Der Fuchs, 5.11.1926|pwk} im Burgtheater\oindex{Burgtheater@\textbf{Burgtheater}, \emph{S.THTR}|pwk}.}}}\label{K_L03811-1} zu sehn; nun muss
               ich, da Sie wohl wieder abgereist sind, Ihnen Glückwunsch u Dank nach Salzburg\oindex{Salzburg@\textbf{Salzburg}, \emph{A.ADM2}|pw} senden. Ich finde Sie haben das Stück\pwindex{Volpone, or, the foxe@\emph{Volpone, or, the foxe}|pwv} von Ben Jonson\pwindex{Jonson, Ben 11.06.1572 – 06.08.1637@\textsc{Jonson, Ben} (11.06.1572 – 06.08.1637), \emph{Schriftsteller/Schriftstellerin}|pw} in jedem Sinne höher gebracht als der
               Original-Autor gethan, – Sie haben es nicht nur für das Theater, sondern auch als
               Dichtung (für meinen Geschmack) erst lebendig gemacht. Ich las (gewissenhafter Weise)
               den \label{K_L03811-2v}\edtext{Ben Jonson\pwindex{Volpone, or, the foxe@\emph{Volpone, or, the foxe}|pwv}\pwindex{Jonson, Ben 11.06.1572 – 06.08.1637@\textsc{Jonson, Ben} (11.06.1572 – 06.08.1637), \emph{Schriftsteller/Schriftstellerin}|pw} (deutsch\pwindex{Herr von Fuchs. Ein Lustspiel in drei Aufzuegen@\emph{Herr von Fuchs. Ein Lustspiel in drei Aufzügen}|pwv}\pwindex{Volpone, oder, Der Fuchs@\emph{Volpone, oder, Der Fuchs}|pwv})}{\lemma{\textnormal{\emph{Ben Jonson (deutsch)}}}\Cendnote{\textnormal{\emph{Volpone, or, the foxe}\pwindex{Volpone, or, the foxe@\emph{Volpone, or, the foxe}|pwk} wurde bis dahin zweimal
                  auf deutsch übersetzt. Die erste Übersetzung (\emph{Herr von Fuchs. Ein Lustspiel in drei Aufzügen}\pwindex{Herr von Fuchs. Ein Lustspiel in drei Aufzuegen@\emph{Herr von Fuchs. Ein Lustspiel in drei Aufzügen}|pwk}, 1793)
                  stammte von Ludwig Tieck\pwindex{Tieck, Ludwig 1773-05-31 – 1853-04-28@\textsc{Tieck, Ludwig} (1773-05-31 – 1853-04-28), \emph{Schriftsteller/Schriftstellerin}|pwk}. Die bis dato
                  neueste (\emph{Volpone, oder, Der Fuchs}\pwindex{Volpone, oder, Der Fuchs@\emph{Volpone, oder, Der Fuchs}|pwk},
                     1912) von Margarethe
                     Mauthner\pwindex{Mauthner, Margarete 1863-07-07 – 1947-04-24@\textsc{Mauthner, Margarete} (1863-07-07 – 1947-04-24), \emph{Übersetzer/Übersetzerin, Kunstsammler/Kunstsammlerin}|pwk}. Welche von beiden Übersetzungen er las, dürfte aus einem Fehler
                  ersichtlich werden, der sich in seiner \emph{Leseliste}\pwindex{Notizen zu Lektuere und Theaterbesuchen (1879-1927)@\emph{Notizen zu Lektüre und Theaterbesuchen (1879-1927)}|pwk} befindet (A. S.: \emph{Lektüren}, England).
                  Hier findet sich \emph{Volpone}\pwindex{Volpone, or, the foxe@\emph{Volpone, or, the foxe}|pwk} unter den Titeln von
                     Philip Massinger\pwindex{Massinger, Philip 1583-01-01 – 1640-03-18@\textsc{Massinger, Philip} (1583-01-01 – 1640-03-18), \emph{Schriftsteller/Schriftstellerin}|pwk}. Hier wäre alphabetisch
                  die Übersetzerin Margarethe Mauthner\pwindex{Mauthner, Margarete 1863-07-07 – 1947-04-24@\textsc{Mauthner, Margarete} (1863-07-07 – 1947-04-24), \emph{Übersetzer/Übersetzerin, Kunstsammler/Kunstsammlerin}|pwk}
                  einzuordnen.}}}\label{K_L03811-2}, eh ich zur General{\pb}probe\eventindex{Burgtheater@\textbf{Burgtheater}!Generalprobe von Volpone oder Der Fuchs, 5.11.1926@Generalprobe von Volpone oder Der Fuchs, 5.11.1926|pwv} ging; – ich war
               von der Wirkung und dem Geist der Bearbeitung aufs angenehmste überrascht.
               Insbesondre den (– Ihren) dritten Akt\pwindex{Ben Jonsons »Volpone« Eine lieblose Komoedie in drei Akten@\emph{Ben Jonsons »Volpone« Eine lieblose Komödie in drei Akten}|pwv} fand ich glänzend.\pend
           
\pstart
           Und nun will ich Ihnen noch herzlich für die lieben Worte danken, die Sie mir in das
                  Buch\pwindex{Ben Jonsons »Volpone« Eine lieblose Komoedie in drei Akten@\emph{Ben Jonsons »Volpone« Eine lieblose Komödie in drei Akten}|pwv} geschrieben haben.
               Ich freue mich Ihrer Sympathie und erwidre sie von Herzen.\pend
           
\pstart
           Schönste Grüße, Ihr{\\[\baselineskip]}\spacefill\mbox{ArthurSchnitzler}\pend
           \leftskip=0em{}\selectlanguage{ngerman}\endnumbering\briefempfaengerindex{Zweig, Stefan@\textsc{Zweig, Stefan}!zzzSchnitzler, Arthur@\emph{von Arthur Schnitzler}!1926-11-071@{7. 11. 1926}|)be}\mylabel{L03811h}
\begin{anhang}
\end{anhang}\normalsize

\doendnotes{C}
\bigskip
\vfill

\clearpage

\footnotesize

\lohead{\textsc{register}}

% Definiere theindex-Environment komplett neu ohne reledmac
\makeatletter
\renewenvironment{theindex}{%
  \section*{\indexname}%
  \setlength{\parindent}{0pt}%
  \setlength{\parskip}{0pt plus 0.3pt}%
  \let\item\@idxitem
}{%
  \clearpage
}
\makeatother

\IfFileExists{\jobname-pw.ind}{\input{\jobname-pw.ind}}{}

\end{document}

      