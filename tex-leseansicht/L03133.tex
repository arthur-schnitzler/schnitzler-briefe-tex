%% latex-korrekturansicht-vorspann.tex
%% Vorspann für die Korrekturansicht.
%% Lädt die gemeinsame Datei latex-vorspann.tex mit gesetztem Schalter.

\newif\ifkorrekturansicht
\korrekturansichttrue

\input{../tex-inputs/latex-vorspann}


\section[Felix Salten an Arthur Schnitzler, {[}7.? 5. 1894{]}]{L03133 Felix Salten an Arthur Schnitzler, {[}7.? 5. 1894{]}}
\nopagebreak\mylabel{L03133v}
\rehead{ }\normalsize\beginnumbering\briefempfaengerindex{Schnitzler, Arthur@\textsc{Schnitzler, Arthur}!zzzSalten, Felix@\emph{von Felix Salten}!1894-05-072@{{[}7.? 5. 1894{]}}|(be}
\toendnotes[C]{\smallbreak\pagebreak[2]}\Standort{CUL, Schnitzler, B 89, A 1.}
\physDesc{Brief, 1 Blatt, 1 Seite, 367 Zeichen
\newline{}Handschrift: Bleistift, lateinische Kurrent
\newline{}Schnitzler: mit Bleistift datiert: »Mai 94« 
\newline{}Ordnung: mit Bleistift von unbekannter Hand nummeriert: »35« }\toendnotes[C]{\smallbreak}
\pstart
           \noindent{}{\pb}Lieber Frd, ich bekomme \label{K_L03133-1v}\edtext{\uline{keine} N\textsuperscript{o}}{\lemma{\textnormal{\emph{keine N\textsuperscript{o}}}}\Cendnote{\textnormal{In Wien\oindex{Wien@\textbf{Wien}, \emph{A.ADM2}|pwk} war das Fahrradfahren auf der Straße nur nach Absolvierung einer
                  Fahrprüfung erlaubt, die durch eine Nummer bestätigt wurde, welche wiederum sichtbar am Rad
                  montiert sein musste. Da Salten\pwindex{Salten, Felix 06.09.1869 – 08.10.1945@\textsc{Salten, Felix} (06.09.1869 – 08.10.1945), \emph{Schriftsteller/Schriftstellerin, Journalist/Journalistin, Chefredakteur/Chefredakteurin}|pwk} diese nicht
                  hatte, musste er, wie er weiter unten projektiert, sein Rad an die Stadt\oindex{Wien@\textbf{Wien}, \emph{A.ADM2}|pwkv}grenze transportieren
                  lassen und Ausflüge außerhalb machen.}}}\label{K_L03133-1}, Specht\pwindex{Specht, Richard 07.12.1870 – 18.03.1932@\textsc{Specht, Richard} (07.12.1870 – 18.03.1932), \emph{Schriftsteller/Schriftstellerin, Journalist/Journalistin, Kritiker/Kritikerin}|pw} will nicht, u. zureden kann ich auch nicht, ich \strikeout{werde} denke, es ist vielleicht das beste, wenn wir die
               Tour abändern, u. mit der \label{K_L03133-2v}\edtext{Franzjosefsbahn\orgindex{Kaiser Franz Josephs-Bahn@Kaiser Franz Josephs-Bahn|pw} fahren}{\lemma{\textnormal{\emph{Franzjosefsbahn fahren}}}\Cendnote{\textnormal{Von den gemeinsamen Ausflügen, die Salten\pwindex{Salten, Felix 06.09.1869 – 08.10.1945@\textsc{Salten, Felix} (06.09.1869 – 08.10.1945), \emph{Schriftsteller/Schriftstellerin, Journalist/Journalistin, Chefredakteur/Chefredakteurin}|pwk} und Schnitzler
                  im Mai 1894 unternahmen, deuten die Angabe des Startort\oindex{Dornbach@\textbf{Dornbach}, \emph{eingemeindeter Ort (A.VOO)}|pwkv}es und der benutzten
                     Bahnlinie\orgindex{Kaiser Franz Josephs-Bahn@Kaiser Franz Josephs-Bahn|pwkv} auf den
                  Ausflug nach Tulln\oindex{Tulln an der Donau@\textbf{Tulln an der Donau}, \emph{A.ADM3}|pwk} am 7. 5. 1894 hin. Da
                  das Korrespondenzstück keine zeitliche Verortung zum Ausflug enthält, könnte es
                  auch in den Tagen vor der Tour verfasst worden sein.}}}\label{K_L03133-2}, oder, sonst irgend
               wie. Ich frage jedenfalls auch einen Einspänner, was es kostet, wenn er mich bis Dornbach\oindex{Dornbach@\textbf{Dornbach}, \emph{eingemeindeter Ort (A.VOO)}|pw} führt.\pend
           
\pstart
           Bitte, theilen Sie mir jetzt gleich mit, was geschehen soll.\pend
           
\pstart
           Ihr {\\[\baselineskip]}\spacefill\mbox{Salten}\pend
           \leftskip=0em{}\selectlanguage{ngerman}\endnumbering\briefempfaengerindex{Schnitzler, Arthur@\textsc{Schnitzler, Arthur}!zzzSalten, Felix@\emph{von Felix Salten}!1894-05-072@{{[}7.? 5. 1894{]}}|)be}\mylabel{L03133h}  \normalsize

\doendnotes{C}
\bigskip
\vfill

\clearpage

\footnotesize

\lohead{\textsc{register}}

% Definiere theindex-Environment komplett neu ohne reledmac
\makeatletter
\renewenvironment{theindex}{%
  \section*{\indexname}%
  \setlength{\parindent}{0pt}%
  \setlength{\parskip}{0pt plus 0.3pt}%
  \let\item\@idxitem
}{%
  \clearpage
}
\makeatother

\IfFileExists{\jobname-pw.ind}{\input{\jobname-pw.ind}}{}

\end{document}

      