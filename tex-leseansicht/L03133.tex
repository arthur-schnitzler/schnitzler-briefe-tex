%% latex-leseansicht-vorspann.tex
%% Vorspann für die Leseansicht.
%% Lädt die gemeinsame Datei latex-vorspann.tex mit nicht gesetztem Schalter.

\newif\ifkorrekturansicht
\korrekturansichtfalse

\input{../tex-inputs/latex-vorspann}


\section[Felix Salten an Arthur Schnitzler, {{[}}7.? 5. 1894{{]}}]{L03133 Felix Salten an Arthur Schnitzler, {[}7.? 5. 1894{]}}
\nopagebreak\mylabel{L03133v}
\rehead{ }\normalsize\beginnumbering\briefempfaengerindex{Schnitzler, Arthur@\textsc{Schnitzler, Arthur}!zzzSalten, Felix@\emph{von Felix Salten}!1894-05-072@{{[}7.? 5. 1894{]}}|(be}
\toendnotes[C]{\smallbreak\pagebreak[2]}
\correspDesc{Versand  durch Felix Salten am [7.? 5. 1894] in Wien
\newline{}Erhalt  durch Arthur Schnitzler am [7.? 5. 1894] in Wien}\toendnotes[C]{\smallbreak}
\Standort{CUL, Schnitzler, B 89, A 1.}
\physDesc{Brief, 1 Blatt, 1 Seite, 367 Zeichen
\newline{}Handschrift: Bleistift, lateinische Kurrent
\newline{}Schnitzler: mit Bleistift datiert: »Mai 94« 
\newline{}Ordnung: mit Bleistift von unbekannter Hand nummeriert: »35« }\toendnotes[C]{\smallbreak}
\pstart
           \noindent{}{\pb}Lieber Frd, ich bekomme \label{K_L03133-1v}\edtext{\uline{keine} N\textsuperscript{o}}{\lemma{\textnormal{\emph{keine N\textsuperscript{o}}}}\Cendnote{\textnormal{In Wien\oindex{Wien@\textbf{Wien}, \emph{Verwaltungsgebiet}|pwk} war das Fahrradfahren auf der Straße nur nach Absolvierung einer
                  Fahrprüfung erlaubt, die durch eine Nummer bestätigt wurde, welche wiederum sichtbar am Rad
                  montiert sein musste. Da Salten\pwindex{Salten, Felix 6.\,9.\,1869 Budapest – 8.\,10.\,1945 Zürich@\textsc{Salten, Felix} (6.\,9.\,1869 Budapest – 8.\,10.\,1945 Zürich), \emph{Schriftsteller, Journalist, Chefredakteur}|pwk} diese nicht
                  hatte, musste er, wie er weiter unten projektiert, sein Rad an die Stadt\oindex{Wien@\textbf{Wien}, \emph{Verwaltungsgebiet}|pwkv}grenze transportieren
                  lassen und Ausflüge außerhalb machen.}}}\label{K_L03133-1}, Specht\pwindex{Specht, Richard 7.\,12.\,1870 Wien – 18.\,3.\,1932 ebd.@\textsc{Specht, Richard} (7.\,12.\,1870 Wien – 18.\,3.\,1932 ebd.), \emph{Schriftsteller, Journalist, Kritiker}|pw} will nicht, u. zureden kann ich auch nicht, ich \strikeout{werde} denke, es ist vielleicht das beste, wenn wir die
               Tour abändern, u. mit der \label{K_L03133-2v}\edtext{Franzjosefsbahn\orgindex{Kaiser Franz Josephs-Bahn@Kaiser Franz Josephs-Bahn|pw} fahren}{\lemma{\textnormal{\emph{Franzjosefsbahn fahren}}}\Cendnote{\textnormal{Von den gemeinsamen Ausflügen, die Salten\pwindex{Salten, Felix 6.\,9.\,1869 Budapest – 8.\,10.\,1945 Zürich@\textsc{Salten, Felix} (6.\,9.\,1869 Budapest – 8.\,10.\,1945 Zürich), \emph{Schriftsteller, Journalist, Chefredakteur}|pwk} und Schnitzler
                  im Mai 1894 unternahmen, deuten die Angabe des Startort\oindex{Wien@\textbf{Wien}!XVII., Hernals@\textbf{XVII., Hernals}!Dornbach@\textbf{Dornbach}|pwkv}es und der benutzten
                     Bahnlinie\orgindex{Kaiser Franz Josephs-Bahn@Kaiser Franz Josephs-Bahn|pwkv} auf den
                  Ausflug nach Tulln\oindex{Tulln an der Donau@\textbf{Tulln an der Donau}, \emph{Verwaltungsgebiet}|pwk} am 7. 5. 1894 hin. Da
                  das Korrespondenzstück keine zeitliche Verortung zum Ausflug enthält, könnte es
                  auch in den Tagen vor der Tour verfasst worden sein.}}}\label{K_L03133-2}, oder, sonst irgend
               wie. Ich frage jedenfalls auch einen Einspänner, was es kostet, wenn er mich bis Dornbach\oindex{Wien@\textbf{Wien}!XVII., Hernals@\textbf{XVII., Hernals}!Dornbach@\textbf{Dornbach}|pw} führt.\pend
           
\pstart
           Bitte, theilen Sie mir jetzt gleich mit, was geschehen soll.\pend
           
\pstart
           Ihr {\\[\baselineskip]}\spacefill\mbox{Salten}\pend
           \leftskip=0em{}\selectlanguage{ngerman}\endnumbering\briefempfaengerindex{Schnitzler, Arthur@\textsc{Schnitzler, Arthur}!zzzSalten, Felix@\emph{von Felix Salten}!1894-05-072@{{[}7.? 5. 1894{]}}|)be}\mylabel{L03133h}  \newcommand{\dateiname}{L03133}\newcommand{\titel}{Felix Salten an Arthur Schnitzler, [7.? 5. 1894]}\newcommand{\editorInnen}{Martin Anton Müller und Laura Untner}%% latex-leseansicht-abspann.tex
%% Abspann für die Leseansicht.
%% Der Schalter \ifkorrekturansicht ist bereits durch den Vorspann gesetzt.

%% latex-abspann.tex
%% Gemeinsamer Abspann für Korrekturansicht und Leseansicht.
%% Setzt den Schalter \ifkorrekturansicht voraus (gesetzt in den
%% einbindenden Dateien latex-korrekturansicht-abspann.tex bzw.
%% latex-leseansicht-abspann.tex).
%% ---------------------------------------------------------------

\normalsize

% Das esempio-Environment wird nur in der Leseansicht benötigt
\ifkorrekturansicht\else
\newenvironment{esempio}[3]%
{
    \vspace{1.5ex}
    \rlap{\underline{#1}}
    \par
    \setlength{\parindent}{0cm}
    \nopagebreak
    \leftskip=#2cm
    \rightskip=#3cm
}
{
    \par
}
\fi

\doendnotes{C}
\bigskip
\vfill

\clearpage

\footnotesize

\ifkorrekturansicht
  \lohead{\textsc{register}}
\fi

% theindex-Environment neu definieren ohne reledmac
\makeatletter
\renewenvironment{theindex}{%
  \ifkorrekturansicht
    \section*{\indexname}%
  \else
    \subsubsection*{Index der erwähnten Entitäten}%
  \fi
  \setlength{\parindent}{0pt}%
  \setlength{\parskip}{0pt plus 0.3pt}%
  \let\item\@idxitem
}{%
  \ifkorrekturansicht\clearpage\fi
}
\makeatother

\IfFileExists{\jobname-pw.ind}{\input{\jobname-pw.ind}}{}

% Quellenangabe nur in der Leseansicht
\ifkorrekturansicht\else
% Fallback-Definitionen, falls die .tex-Datei \titel etc. nicht gesetzt hat
\providecommand{\titel}{}
\providecommand{\editorInnen}{}
\providecommand{\dateiname}{\jobname}

\vspace{3cm}

\vfill

\footnotesize
\textsc{Quelle}: \titel. Herausgegeben von {\editorInnen}. In: \emph{Arthur Schnitzler: Briefwechsel mit Autorinnen und Autoren}.
 Digitale Edition, https://schnitzler-briefe.acdh.oeaw.ac.at/{\dateiname}.html (Stand \today)
\fi

\end{document}


