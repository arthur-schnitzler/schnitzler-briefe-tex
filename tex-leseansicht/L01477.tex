\input{../tex-inputs/latex-pdf-vorspann}
\begin{center}
            \textcolor{red}{ENTWURF. ENTZIFFERUNG NOCH NICHT KORREKTURGELESEN}
                      \end{center}
            
               \section[Arthur Schnitzler an Hermann Bahr, 14. 12. 1904]{ Arthur Schnitzler an Hermann Bahr, 14. 12. 1904}\nopagebreak\mylabel{v}\rehead{ }\begin{ledgroupsized}[t]{13cm}\normalsize\beginnumbering\briefempfaengerindex{Bahr, Hermann@\textsc{Bahr, Hermann}!zzzSchnitzler, Arthur@\emph{von Arthur Schnitzler}!1904-12-141@{14. 12. 1904}|(be} \toendnotes[C]{\smallbreak\pagebreak[2]} \Standort{TMW, HS AM 23368 Ba.}
\physDesc{Brief, 2 Blätter, 7 Seiten
\newline{}Handschrift: schwarze Tinte, deutsche Kurrent\newline{}Ordnung: Lochung }\buchAbdrucke{\weitereDrucke{1) Arthur Schnitzler: \emph{Briefe 1875–1912}. Hg. Therese Nickl und Heinrich Schnitzler. Frankfurt am Main: \emph{S. Fischer} 1981, S. 504–506.} \weitereDrucke{2) \emph{14. 12. 1904.} In: Arthur Schnitzler: \emph{The Letters of Arthur Schnitzler to Hermann Bahr}. Edited, annotated, and with an introduction, by Donald G.
                        Daviau. Chapel Hill: \emph{The University of North Carolina Press} 1978, S. 86–87 (University of North Carolina studies in the Germanic languages
                        and literatures, 89).} \weitereDrucke{3) Hermann Bahr, Arthur Schnitzler: \emph{Briefwechsel, Aufzeichnungen, Dokumente (1891–1931)}. Hg. Kurt Ifkovits und Martin Anton Müller. Göttingen: \emph{Wallstein} 2018, S. 333–334.} }\toendnotes[C]{\smallbreak}\pstart
           \raggedleft{}{\pb}Wien\oindex{Wien@\textbf{Wien}|pw}{ }14. 12. 904\pend
           \pstart
           mein lieber Hermann, es beſchämt mich faſt, daſs du über ein im
               Ganzen doch ziemlich unbeträchtliches Ding wie es der Puppenſpieler\pwindex{Schnitzler, Arthur 15.05.1862 – 21.10.1931@\textsc{Schnitzler, Arthur} (15.05.1862 – 21.10.1931), \emph{Schriftsteller, Mediziner}!Puppenspieler31. 05. 1903@\strich\emph{Der Puppenspieler} {[}31. 05. 1903{]}|pw} iſt (er gehörte \label{K_L01477_1v}\edtext{in den Cyclus Lebendg Stunden\pwindex{Schnitzler, Arthur 15.05.1862 – 21.10.1931@\textsc{Schnitzler, Arthur} (15.05.1862 – 21.10.1931), \emph{Schriftsteller, Mediziner}!Lebendige Stunden. Vier Einakter1901-12-23@\strich\emph{Lebendige Stunden. Vier Einakter} {[}1901-12-23{]}|pw}}{\lemma{\textnormal{\emph{in … Stunden}}}\Cendnote{\textnormal{vgl. Arthur Schnitzler an Hermann Bahr, 18. 10. 1901}}}\label{K_L01477_1h}, aber wegen zu großer Länge des Abends mußte er
                  zurückge{[}ſe{]}tzt werden) – ſo ſchöne Worte ſagſt. Vielleicht
               drücke ich mich beſſer aus, we{\geminationn} ich ſage: \uline{anläßlich} des Puppenſpielers\pwindex{Schnitzler, Arthur 15.05.1862 – 21.10.1931@\textsc{Schnitzler, Arthur} (15.05.1862 – 21.10.1931), \emph{Schriftsteller, Mediziner}!Puppenspieler31. 05. 1903@\strich\emph{Der Puppenspieler} {[}31. 05. 1903{]}|pw}. Denn deiner Auffaſſung des kleinen Stücks muſs ich
               widerſprechen. Vielleicht hab ich nicht das Recht dazu, denn es werden ja doch
               wahrſcheinlich künſt{\pb}leriſche Mängel der Sache ſchuld daran ſein, daſs du eine Lebensanſchauung darin
               findeſt, die ich nicht hineinlegen wollte und die mir perſönlich fremd iſt. Ebenſo
               verhält es ſich mit dem Einſ. Weg\pwindex{Schnitzler, Arthur 15.05.1862 – 21.10.1931@\textsc{Schnitzler, Arthur} (15.05.1862 – 21.10.1931), \emph{Schriftsteller, Mediziner}!einsame Weg. Schauspiel in fuenf Akten1904@\strich\emph{Der einsame Weg. Schauspiel in fünf Akten} {[}1904{]}|pw}. Ich ſtehe ſo
               wenig auf Seite des Oboëſpielers, als \introOben{}ich\introOben{} auf Seiten des Profeſſor Wegrath\pwindex{Schnitzler, Arthur 15.05.1862 – 21.10.1931@\textsc{Schnitzler, Arthur} (15.05.1862 – 21.10.1931), \emph{Schriftsteller, Mediziner}!einsame Weg. Schauspiel in fuenf Akten1904@\strich\emph{Der einsame Weg. Schauspiel in fünf Akten} {[}1904{]}|pwv} geſtanden habe –
               freilich auch nicht auf der des Julian und des Puppenſpielers. Aber warum? Weil ſie
               eben nicht ganze Kerle ſind, \introOben{}keine Leute\introOben{} die – nach der dir
               bekannten Anekdote von der alten Streitmann\pwindex{Streitmann, Katharina 1830/1831 – 1898-10-19@\textsc{Streitmann, Katharina} (1830/1831 – 1898-10-19)|pw} –
                  »\label{K_L01477_2v}\edtext{brav genug}{\lemma{\textnormal{\emph{brav genug}}}\Cendnote{\textnormal{\emph{Berliner Tageblatt}\orgindex{Berliner Tageblatt@Berliner Tageblatt|pwk}, Jg. 54, Nr. 227, 14. 5. 1925, Abend-Blatt, S. 2: »Arthur Schnitzler\pwindex{Schnitzler, Arthur 15.05.1862 – 21.10.1931@\textsc{Schnitzler, Arthur} (15.05.1862 – 21.10.1931), \emph{Schriftsteller, Mediziner}|pw} unterhält sich mit einem
                     Freund über Leutnants Bilse\pwindex{Bilse, Fritz Oswald 1878-03-31 – 1951-08-30@\textsc{Bilse, Fritz Oswald} (1878-03-31 – 1951-08-30), \emph{Schriftsteller, Militär}|pw}s Schlüsselroman
                        ›Aus einer kleinen Garnison\pwindex{Bilse, Fritz Oswald 1878-03-31 – 1951-08-30@\textsc{Bilse, Fritz Oswald} (1878-03-31 – 1951-08-30), \emph{Schriftsteller, Militär}!Aus einer kleinen Garnison1903@\strich\emph{Aus einer kleinen Garnison} {[}1903{]}|pw}‹, und es
                     entsteht die Frage, inwieweit ein Autor ein Recht habe, wirkliche Vorkommnisse
                     und Namen in ein Werk aufzunehmen, ›Die Frage‹, sagt Schnitzler\pwindex{Schnitzler, Arthur 15.05.1862 – 21.10.1931@\textsc{Schnitzler, Arthur} (15.05.1862 – 21.10.1931), \emph{Schriftsteller, Mediziner}|pw}, ›erinnert mich an eine reizende Episode ans
                     dem Leben des Tenors Streitmann\pwindex{Streitmann, Karl 1858-05-08 – 1937-10-29@\textsc{Streitmann, Karl} (1858-05-08 – 1937-10-29), \emph{Schauspieler, Sänger}|pw}; der war
                     nämlich schon ein berühmter Operettenheld, ohne daß ihn seine auf dem Land
                     lebende Mutter je auf den Brettern gesehen hatte. Eines Tages fährt sie nach
                        Wien\oindex{Wien@\textbf{Wien}|pw}, begibt sich – auf dem Zettel steht die
                        ›Fledermaus\pwindex{\textcolor{red}{\textsuperscript{XXXX1 indx}}!Fledermaus1874@\strich\emph{Die Fledermaus} {[}1874{]}|pw}‹ – ins Theater, wo ihr Sohn
                     auftritt. ›Nun?‹ fragt am Ende der Vorstellung Streitmann\pwindex{Streitmann, Karl 1858-05-08 – 1937-10-29@\textsc{Streitmann, Karl} (1858-05-08 – 1937-10-29), \emph{Schauspieler, Sänger}|pw}{ }seine Mutter\pwindex{Streitmann, Katharina 1830/1831 – 1898-10-19@\textsc{Streitmann, Katharina} (1830/1831 – 1898-10-19)|pwv}, ›wie habe ich dir gefallen?‹ – ›Sehr gut,
                     sehr brav, mein Kind – aber‹, und sie wird bedrückt, ›warum hast du nicht das
                     schöne Lied gesungen: ›Ach, ich
                        hab’ sie ja nur auf die Schulter geküßt\pwindex{\textcolor{red}{\textsuperscript{XXXX1 indx}}!Bettelstudent1882@\strich\emph{Der Bettelstudent} {[}1882{]}|pwv}\pwindex{\textcolor{red}{\textsuperscript{XXXX1 indx}}!Bettelstudent1882@\strich\emph{Der Bettelstudent} {[}1882{]}|pwv}?‹ – ›Aber Mama,‹ sagte der
                     Tenor, ›das kommt ja gar nicht in dieser Operette vor.‹ – ›Schön, kommt nicht
                     vor {\dots} aber warum hast du’s nicht doch gesungen?‹ –
                     ›Aber Mama, verstehst du nicht – ich hätt’ es ja gar nicht singen dürfen.‹
                     Darauf ein langer, mißtrauischer Blick der Mutter\pwindex{Streitmann, Katharina 1830/1831 – 1898-10-19@\textsc{Streitmann, Katharina} (1830/1831 – 1898-10-19)|pwv}: ›Wenn man brav ist, mein Kind, darf man
                     alles.‹ ›Das ist‹, fügt Schnitzler\pwindex{Schnitzler, Arthur 15.05.1862 – 21.10.1931@\textsc{Schnitzler, Arthur} (15.05.1862 – 21.10.1931), \emph{Schriftsteller, Mediziner}|pw} hinzu,
                     ›auch meine Meinung über den Schlüsselroman.‹«}}}\label{K_L01477_2h}« ſind – um alles
               zu dürfen. Wäre der Puppen{\pb}ſpieler\pwindex{Schnitzler, Arthur 15.05.1862 – 21.10.1931@\textsc{Schnitzler, Arthur} (15.05.1862 – 21.10.1931), \emph{Schriftsteller, Mediziner}!Puppenspieler31. 05. 1903@\strich\emph{Der Puppenspieler} {[}31. 05. 1903{]}|pwv}
               wirklich ein »Großer«, ſo bräuchte er ſich nicht in Lügen einzuſpinnen, um der
               größere zu bleiben – wäre Julian\pwindex{Schnitzler, Arthur 15.05.1862 – 21.10.1931@\textsc{Schnitzler, Arthur} (15.05.1862 – 21.10.1931), \emph{Schriftsteller, Mediziner}!einsame Weg. Schauspiel in fuenf Akten1904@\strich\emph{Der einsame Weg. Schauspiel in fünf Akten} {[}1904{]}|pwv}
               wirklich ein Großer – ſo würde das beſte ſeines Weſens nicht mit seiner Jugend
               auslöſchen. Gegen die Herzöge und gegen die \textsc{Sala\pwindex{Schnitzler, Arthur 15.05.1862 – 21.10.1931@\textsc{Schnitzler, Arthur} (15.05.1862 – 21.10.1931), \emph{Schriftsteller, Mediziner}!einsame Weg. Schauspiel in fuenf Akten1904@\strich\emph{Der einsame Weg. Schauspiel in fünf Akten} {[}1904{]}|pwv}}’s hab ich nichts – und vor den »Großen Räubern« ſalutir ich, gleich dir, in
               Ehrfurcht. Du haſt ganz recht: »Entſagung iſt nicht immer Reife.« – – nur ſetze ich
               hinzu: nicht bei allen. Wenn Individuen wie \uline{Wegrath}\pwindex{Schnitzler, Arthur 15.05.1862 – 21.10.1931@\textsc{Schnitzler, Arthur} (15.05.1862 – 21.10.1931), \emph{Schriftsteller, Mediziner}!einsame Weg. Schauspiel in fuenf Akten1904@\strich\emph{Der einsame Weg. Schauspiel in fünf Akten} {[}1904{]}|pwv} in irgend einem Moment ihrer Exiſtenz die Grenzen ihrer Begabung erkennen, –
                  {\pb}so iſt \uline{dieſe} Entſagung, wie jede \uline{Erke{\geminationn}tnis} innere Reife, oder wenigſtens ein
               Symptom innerer Reife. Ebenſo iſt für den Oboëſpieler wirklich der »Innere Friede und
               die ſchuldbefreite Bruſt« das einzig erreichbare Glück. Und daſs ein Menſch wie der
                  »Puppenſpieler\pwindex{Schnitzler, Arthur 15.05.1862 – 21.10.1931@\textsc{Schnitzler, Arthur} (15.05.1862 – 21.10.1931), \emph{Schriftsteller, Mediziner}!Puppenspieler31. 05. 1903@\strich\emph{Der Puppenspieler} {[}31. 05. 1903{]}|pwv}« nicht, wie es
               eben den Beſchränkungen ſeines Weſens angemeſſen wäre, \introOben{}zu\introOben{}
               entſagen im Stande iſt, ſich \introOben{}vielmehr\introOben{} dieſer Entſagung
                  \label{T_L01477_1v}\edtext{ſchämten}{\lemma{\textnormal{\emph{ſchämten}}}\Cendnote{\textnormal{Schreibfehler, das Wort ist deutlich zu entziffern}}}\label{T_L01477_1h} würde
               und daher den andern u ſich ein \introOben{}falſches\introOben{} Eigenſchickſal
               vorſpielt – iſt ein Zeichen, daſs er innere Reife nicht erlangte, welche eben nur in
               Selbſterkenntnis beſtehen kann. \strikeout{Daher} Es iſt {\pb}alſo nur natürlich,
               daſs bei manchen Menſchen, insbeſondre bei klugen, von mäßigem Talente und ſtillem
               Temperamente das was ihnen an innerer Reife überhaupt beſchieden iſt, in einer Art
               von »Entſagung« den entſprechenden Ausdruck findet.\pend
           \pstart
           Wohl denen, die’s nicht nöthig haben, – wohl uns, die wir wie mir ſcheint zu dieſen
               gehören – und hoffentlich nicht allein wegen Mangels an Klugheit. So ſpricht alſo
               nichts dagegen, mein lieber Hermann, daſs wir beide uns an die Arbeit machen, die du
               in meine {\pb}Hände legſt:
                  »\label{LL075-1v}Das Werk von der letzten Nacht einer alten
                  Zeit\label{LL075-1h}« – Und ſchließlich können es auch andre Werke ſein.\pend
           \pstart
           Zu »\label{K_L01477_3v}\edtext{Mahler\pwindex{Mahler, Gustav 07.07.1860 – 18.05.1911@\textsc{Mahler, Gustav} (07.07.1860 – 18.05.1911), \emph{Theaterleiter, Komponist, Dirigent}!Symphonie Nr. 3 D-Moll1902@\strich\emph{Symphonie Nr. 3 D-Moll} {[}1902{]}|pwv}« haben wir noch Sitze}{\lemma{\textnormal{\emph{Mahler« … Sitze}}}\Cendnote{\textnormal{Mahler\pwindex{Mahler, Gustav 07.07.1860 – 18.05.1911@\textsc{Mahler, Gustav} (07.07.1860 – 18.05.1911), \emph{Theaterleiter, Komponist, Dirigent}|pwk} dirigierte seine \emph{3. Symphonie}\pwindex{Mahler, Gustav 07.07.1860 – 18.05.1911@\textsc{Mahler, Gustav} (07.07.1860 – 18.05.1911), \emph{Theaterleiter, Komponist, Dirigent}!Symphonie Nr. 3 D-Moll1902@\strich\emph{Symphonie Nr. 3 D-Moll} {[}1902{]}|pwk} im Musikvereinssaal\oindex{Musikverein@\textbf{Musikverein}|pwk}.}}}\label{K_L01477_3h}{ }\damage{be}kommen, ſo ſeh ich dich hoffentlich auch heute Abend.\pend
           \pstart
           Jedenfalls aber sage oder schreibe mir pneumatiſch, ob du vielleicht Lust hätteſt, am
                  \uline{Samſtag} bei uns zu nachtmahlen.\pend
           \pstart
           Herzlichst der deine{\\[\baselineskip]}\spacefill\mbox{Arthur{\pb}}\pend
           \leftskip=0em{}\pstart
           \noindent{}Olga\pwindex{Schnitzler, Olga 17.01.1882 – 13.01.1970@\textsc{Schnitzler, Olga} (17.01.1882 – 13.01.1970), \emph{Schauspielerin, Sängerin}|pw} grüßt dich herzlich und ſagt dir, daſs sie \substVorne{}\textsuperscript{das}\substDazwischen{}von dem\substHinten{} was du anläßlich de\textcolor{gray}{s \textsc{P}.}
                  geſchrieben haſt, erſchüttert war.\pend
           \endnumbering\briefempfaengerindex{Bahr, Hermann@\textsc{Bahr, Hermann}!zzzSchnitzler, Arthur@\emph{von Arthur Schnitzler}!1904-12-141@{14. 12. 1904}|)be}\mylabel{h}\end{ledgroupsized}  \newcommand{\dateiname}{L01477}\newcommand{\titel}{Arthur Schnitzler an Hermann Bahr, 14. 12. 1904}\newcommand{\editorInnen}{ Kurt Ifkovits,  Martin Anton Müller}\input{../tex-inputs/latex-pdf-abspann}
      