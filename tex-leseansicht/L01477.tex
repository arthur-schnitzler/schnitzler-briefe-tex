%% latex-leseansicht-vorspann.tex
%% Vorspann für die Leseansicht.
%% Lädt die gemeinsame Datei latex-vorspann.tex mit nicht gesetztem Schalter.

\newif\ifkorrekturansicht
\korrekturansichtfalse

\input{../tex-inputs/latex-vorspann}


\section[Arthur Schnitzler an Hermann Bahr, 14. 12. 1904]{L01477 Arthur Schnitzler an Hermann Bahr, 14. 12. 1904}
\nopagebreak\mylabel{L01477v}
\rehead{ }\normalsize\beginnumbering\briefempfaengerindex{Bahr, Hermann@\textsc{Bahr, Hermann}!zzzSchnitzler, Arthur@\emph{von Arthur Schnitzler}!1904-12-141@{14. 12. 1904}|(be}
\toendnotes[C]{\smallbreak\pagebreak[2]}
\correspDesc{Versand  durch Arthur Schnitzler am 14. 12. 1904 in Wien
\newline{}Erhalt  durch Hermann Bahr im Zeitraum [14. 12. 1904 – 18. 12. 1904?] in Wien}\toendnotes[C]{\smallbreak}
\Standort{TMW, HS AM 23368 Ba.}
\physDesc{Brief, 2 Blätter, 7 Seiten, 3106 Zeichen
\newline{}Handschrift: schwarze Tinte, deutsche Kurrent
\newline{}Ordnung: Lochung }
\buchAbdrucke{\weitereDrucke{1) Arthur Schnitzler: \emph{Briefe 1875–1912}. Herausgegeben von Therese Nickl und Heinrich Schnitzler. Frankfurt am Main: \emph{S. Fischer} 1981, S. 504–506.} \weitereDrucke{2) \emph{14. 12. 1904.} In: Arthur Schnitzler: \emph{The Letters of Arthur Schnitzler to Hermann Bahr}. Edited, annotated, and with an introduction, by Donald G. Daviau. Chapel Hill: \emph{The University of North Carolina Press} 1978, S. 86–87 (University of North Carolina studies in the Germanic languages
                        and literatures, 89).} \weitereDrucke{3) Hermann Bahr, Arthur Schnitzler: \emph{Briefwechsel, Aufzeichnungen, Dokumente (1891–1931)}. Herausgegeben von Kurt Ifkovits und Martin Anton Müller. Göttingen: \emph{Wallstein} 2018, S. 333–334.} }\toendnotes[C]{\smallbreak}
\pstart
           \raggedleft{}{\pb}Wien\oindex{Wien@\textbf{Wien}, \emph{Verwaltungsgebiet}|pw}{ }14. 12. 904\pend
           \vspace{0.5em}
\pstart
           mein lieber Hermann, es beſchämt mich faſt, daſs du über ein im
               Ganzen doch ziemlich unbeträchtliches Ding wie es der Puppenſpieler\pwindex{Schnitzler, Arthur 15.\,5.\,1862 Wien – 21.\,10.\,1931 ebd.@\textsc{Schnitzler, Arthur} (15.\,5.\,1862 Wien – 21.\,10.\,1931 ebd.), \emph{Schriftsteller, Mediziner}!Puppenspieler. Studie in einem Aufzuge@\strich\emph{Der Puppenspieler. Studie in einem Aufzuge}|pw} iſt (er gehörte \label{K_L01477-1v}\edtext{in den Cyclus Lebendg Stunden\pwindex{Schnitzler, Arthur 15.\,5.\,1862 Wien – 21.\,10.\,1931 ebd.@\textsc{Schnitzler, Arthur} (15.\,5.\,1862 Wien – 21.\,10.\,1931 ebd.), \emph{Schriftsteller, Mediziner}!Lebendige Stunden. Vier Einakter@\strich\emph{Lebendige Stunden. Vier Einakter}|pw}}{\lemma{\textnormal{\emph{in … Stunden}}}\Cendnote{\textnormal{Vgl. XXXX Auszeichnungsfehler: Dokument L01181 nicht gefunden.
               }}}\label{K_L01477-1}, aber wegen zu großer Länge des Abends mußte er
               zurückge{[}ſe{]}tzt werden) –{ }ſo{ }ſchöne \label{K_L01477-2v}\edtext{Worte\pwindex{Bahr, Hermann 19.\,7.\,1863 Linz – 15.\,1.\,1934 München@\textsc{Bahr, Hermann} (19.\,7.\,1863 Linz – 15.\,1.\,1934 München), \emph{Schriftsteller, Kritiker}!Puppenspieler. (Studie in einem Aufzuge von Arthur Schnitzler. Zum ersten Mal aufgeführt im Carl-Theater am 12. Dezember 1904)@\strich\emph{Der Puppenspieler. (Studie in einem Aufzuge von Arthur Schnitzler. Zum ersten Mal aufgeführt im Carl-Theater am 12. Dezember 1904)}|pwv}}{\lemma{\textnormal{\emph{Worte}}}\Cendnote{\textnormal{Siehe Hermann Bahr, Arthur Schnitzler: \emph{Briefwechsel, Aufzeichnungen, Dokumente (1891–1931)}, Hermann Bahr: Der Puppenspieler, 13. 12. 1904.
               }}}\label{K_L01477-2}{ }ſagſt. Vielleicht
               drücke ich mich beſſer aus, we{\geminationn} ich{ }ſage: \uline{anläßlich} des Puppenſpielers\pwindex{Schnitzler, Arthur 15.\,5.\,1862 Wien – 21.\,10.\,1931 ebd.@\textsc{Schnitzler, Arthur} (15.\,5.\,1862 Wien – 21.\,10.\,1931 ebd.), \emph{Schriftsteller, Mediziner}!Puppenspieler. Studie in einem Aufzuge@\strich\emph{Der Puppenspieler. Studie in einem Aufzuge}|pw}. Denn deiner Auffaſſung des kleinen Stücks muſs ich
               widerſprechen. Vielleicht hab ich nicht das Recht dazu, denn es werden ja doch
               wahrſcheinlich künſt{\pb}leriſche Mängel der Sache{ }ſchuld daran{ }ſein, daſs du eine Lebensanſchauung darin
               findeſt, die ich nicht hineinlegen wollte und die mir perſönlich fremd iſt. Ebenſo
               verhält es{ }ſich mit dem Einſ. Weg\pwindex{Schnitzler, Arthur 15.\,5.\,1862 Wien – 21.\,10.\,1931 ebd.@\textsc{Schnitzler, Arthur} (15.\,5.\,1862 Wien – 21.\,10.\,1931 ebd.), \emph{Schriftsteller, Mediziner}!einsame Weg. Schauspiel in fünf Akten@\strich\emph{Der einsame Weg. Schauspiel in fünf Akten}|pw}. Ich{ }ſtehe{ }ſo
               wenig auf Seite des Oboëſpielers, als \introOben{}ich\introOben{} auf Seiten des Profeſſor Wegrath\pwindex{Schnitzler, Arthur 15.\,5.\,1862 Wien – 21.\,10.\,1931 ebd.@\textsc{Schnitzler, Arthur} (15.\,5.\,1862 Wien – 21.\,10.\,1931 ebd.), \emph{Schriftsteller, Mediziner}!einsame Weg. Schauspiel in fünf Akten@\strich\emph{Der einsame Weg. Schauspiel in fünf Akten}|pwv} geſtanden
               habe – freilich auch nicht auf der des Julian und des Puppenſpielers. Aber warum?
               Weil{ }ſie eben nicht ganze Kerle{ }ſind, \introOben{}keine Leute\introOben{} die – nach
               der dir bekannten Anekdote von der alten Streitmann\pwindex{Streitmann, Katharina 1830/1831 – 19.\,10.\,1898 Wien@\textsc{Streitmann, Katharina} (1830/1831 – 19.\,10.\,1898 Wien)|pw} – »\label{K_L01477-3v}\edtext{brav genug}{\lemma{\textnormal{\emph{brav genug}}}\Cendnote{\textnormal{\emph{Berliner Tageblatt}\orgindex{Berliner Tageblatt@Berliner Tageblatt|pwk}, Jg. 54, Nr. 227, 14. 5. 1925, Abend-Blatt, S. 2: »Arthur Schnitzler unterhält sich mit
                     einem Freund über Leutnants Bilses\pwindex{Bilse, Fritz Oswald 31.\,3.\,1878 Kirn – 30.\,8.\,1951 Eberswalde@\textsc{Bilse, Fritz Oswald} (31.\,3.\,1878 Kirn – 30.\,8.\,1951 Eberswalde), \emph{Schriftsteller, Militär}|pw}
                     Schlüsselroman ›Aus einer kleinen
                     Garnison\pwindex{Bilse, Fritz Oswald 31.\,3.\,1878 Kirn – 30.\,8.\,1951 Eberswalde@\textsc{Bilse, Fritz Oswald} (31.\,3.\,1878 Kirn – 30.\,8.\,1951 Eberswalde), \emph{Schriftsteller, Militär}!Aus einer kleinen Garnison@\strich\emph{Aus einer kleinen Garnison}|pw}‹, und es entsteht die Frage, inwieweit ein Autor ein Recht habe,
                     wirkliche Vorkommnisse und Namen in ein Werk aufzunehmen, ›Die Frage‹, sagt Schnitzler, ›erinnert mich an eine
                     reizende Episode ans dem Leben des Tenors Streitmann\pwindex{Streitmann, Karl 8.\,5.\,1858 Wien – 29.\,10.\,1937 ebd.@\textsc{Streitmann, Karl} (8.\,5.\,1858 Wien – 29.\,10.\,1937 ebd.), \emph{Schauspieler, Sänger}|pw}; der war nämlich schon ein berühmter Operettenheld, ohne
                     daß ihn seine auf dem Land lebende Mutter je auf den Brettern gesehen hatte.
                     Eines Tages fährt sie nach Wien\oindex{Wien@\textbf{Wien}, \emph{Verwaltungsgebiet}|pw}, begibt sich
                     – auf dem Zettel steht die ›Fledermaus\pwindex{\textcolor{red}{\textsuperscript{XXXX indx1}}!Fledermaus. Komische Oper in drei Acten@\strich\emph{Die Fledermaus. Komische Oper in drei Acten}|pw}\pwindex{\textcolor{red}{\textsuperscript{XXXX indx1}}!Fledermaus. Komische Oper in drei Acten@\strich\emph{Die Fledermaus. Komische Oper in drei Acten}|pw}‹ –
                     ins Theater, wo ihr Sohn auftritt. ›Nun?‹ fragt am Ende der Vorstellung Streitmann\pwindex{Streitmann, Karl 8.\,5.\,1858 Wien – 29.\,10.\,1937 ebd.@\textsc{Streitmann, Karl} (8.\,5.\,1858 Wien – 29.\,10.\,1937 ebd.), \emph{Schauspieler, Sänger}|pw}{ }seine Mutter\pwindex{Streitmann, Katharina 1830/1831 – 19.\,10.\,1898 Wien@\textsc{Streitmann, Katharina} (1830/1831 – 19.\,10.\,1898 Wien)|pwv}, ›wie habe ich dir gefallen?‹ – ›Sehr gut,
                     sehr brav, mein Kind – aber‹, und sie wird bedrückt, ›warum hast du nicht das
                     schöne Lied gesungen: ›Ach,
                        ich hab’ sie ja nur auf die Schulter geküßt\pwindex{\textcolor{red}{\textsuperscript{XXXX indx1}}!Bettelstudent@\strich\emph{Der Bettelstudent}|pwv}\pwindex{\textcolor{red}{\textsuperscript{XXXX indx1}}!Bettelstudent@\strich\emph{Der Bettelstudent}|pwv}\pwindex{\textcolor{red}{\textsuperscript{XXXX indx1}}!Bettelstudent@\strich\emph{Der Bettelstudent}|pwv}?‹ – ›Aber Mama,‹ sagte der
                     Tenor, ›das kommt ja gar nicht in dieser Operette vor.‹ – ›Schön, kommt nicht
                     vor {\dots} aber warum hast du’s nicht doch gesungen?‹ –
                     ›Aber Mama, verstehst du nicht – ich hätt’ es ja gar nicht singen dürfen.‹
                     Darauf ein langer, mißtrauischer Blick der Mutter\pwindex{Streitmann, Katharina 1830/1831 – 19.\,10.\,1898 Wien@\textsc{Streitmann, Katharina} (1830/1831 – 19.\,10.\,1898 Wien)|pwv}: ›Wenn man brav ist, mein Kind, darf man
                     alles.‹ ›Das ist‹, fügt Schnitzler hinzu,
                     ›auch meine Meinung über den Schlüsselroman.‹«}}}\label{K_L01477-3}«{ }ſind – um alles
               zu dürfen. Wäre der Puppen{\pb}ſpieler\pwindex{Schnitzler, Arthur 15.\,5.\,1862 Wien – 21.\,10.\,1931 ebd.@\textsc{Schnitzler, Arthur} (15.\,5.\,1862 Wien – 21.\,10.\,1931 ebd.), \emph{Schriftsteller, Mediziner}!Puppenspieler. Studie in einem Aufzuge@\strich\emph{Der Puppenspieler. Studie in einem Aufzuge}|pwv}
               wirklich ein »Großer«,{ }ſo bräuchte er{ }ſich nicht in Lügen einzuſpinnen, um der
               größere zu bleiben – wäre Julian\pwindex{Schnitzler, Arthur 15.\,5.\,1862 Wien – 21.\,10.\,1931 ebd.@\textsc{Schnitzler, Arthur} (15.\,5.\,1862 Wien – 21.\,10.\,1931 ebd.), \emph{Schriftsteller, Mediziner}!einsame Weg. Schauspiel in fünf Akten@\strich\emph{Der einsame Weg. Schauspiel in fünf Akten}|pwv} wirklich ein Großer –{ }ſo würde das beſte{ }ſeines Weſens nicht mit
               seiner Jugend auslöſchen. Gegen die Herzöge und gegen die \textsc{Sala\pwindex{Schnitzler, Arthur 15.\,5.\,1862 Wien – 21.\,10.\,1931 ebd.@\textsc{Schnitzler, Arthur} (15.\,5.\,1862 Wien – 21.\,10.\,1931 ebd.), \emph{Schriftsteller, Mediziner}!einsame Weg. Schauspiel in fünf Akten@\strich\emph{Der einsame Weg. Schauspiel in fünf Akten}|pwv}}’s hab ich nichts – und vor den »Großen Räubern«{ }ſalutir ich, gleich dir, in
               Ehrfurcht. Du haſt ganz recht: »Entſagung iſt nicht immer Reife.« – – nur{ }ſetze ich
               hinzu: nicht bei allen. Wenn Individuen wie \uline{Wegrath}\pwindex{Schnitzler, Arthur 15.\,5.\,1862 Wien – 21.\,10.\,1931 ebd.@\textsc{Schnitzler, Arthur} (15.\,5.\,1862 Wien – 21.\,10.\,1931 ebd.), \emph{Schriftsteller, Mediziner}!einsame Weg. Schauspiel in fünf Akten@\strich\emph{Der einsame Weg. Schauspiel in fünf Akten}|pwv} in irgend einem Moment ihrer Exiſtenz die Grenzen ihrer Begabung erkennen, –
               {\pb}so iſt \uline{dieſe} Entſagung, wie jede \uline{Erke{\geminationn}tnis} innere Reife, oder wenigſtens ein
               Symptom innerer Reife. Ebenſo iſt für den Oboëſpieler wirklich der »Innere Friede und
               die{ }ſchuldbefreite Bruſt« das einzig erreichbare Glück. Und daſs ein Menſch wie der
                  »Puppenſpieler\pwindex{Schnitzler, Arthur 15.\,5.\,1862 Wien – 21.\,10.\,1931 ebd.@\textsc{Schnitzler, Arthur} (15.\,5.\,1862 Wien – 21.\,10.\,1931 ebd.), \emph{Schriftsteller, Mediziner}!Puppenspieler. Studie in einem Aufzuge@\strich\emph{Der Puppenspieler. Studie in einem Aufzuge}|pwv}« nicht, wie
               es eben den Beſchränkungen{ }ſeines Weſens angemeſſen wäre, \introOben{}zu\introOben{}
               entſagen im Stande iſt,{ }ſich \introOben{}vielmehr\introOben{} dieſer Entſagung
               und daher den andern u{ }ſich ein \introOben{}falſches\introOben{} Eigenſchickſal
               vorſpielt – iſt ein Zeichen, daſs er innere Reife nicht erlangte, welche eben nur in
               Selbſterkenntnis beſtehen kann. \strikeout{Daher} Es iſt {\pb}alſo nur natürlich,
               daſs bei manchen Menſchen, insbeſondre bei klugen, von mäßigem Talente und{ }ſtillem
               Temperamente das was ihnen an innerer Reife überhaupt beſchieden iſt, in einer Art
               von »Entſagung« den entſprechenden Ausdruck findet.\pend
           
\pstart
           Wohl denen, die’s nicht nöthig haben, – wohl uns, die wir wie mir{ }ſcheint zu dieſen
               gehören – und hoffentlich nicht allein wegen Mangels an Klugheit. So{ }ſpricht alſo
               nichts dagegen, mein lieber Hermann, daſs wir beide uns an die Arbeit machen, die du
               in meine {\pb}Hände legſt:
                  »\label{LL075-1v}Das Werk von der letzten Nacht einer alten
                  Zeit\label{LL075-1h}« – Und{ }ſchließlich können es auch andre Werke{ }ſein.\pend
           
\pstart
           Zu »\label{K_L01477-4v}\edtext{Mahler\pwindex{Mahler, Gustav 7.\,7.\,1860 Kaliště – 18.\,5.\,1911 Wien@\textsc{Mahler, Gustav} (7.\,7.\,1860 Kaliště – 18.\,5.\,1911 Wien), \emph{Theaterleiter, Komponist, Dirigent}!3. Sinfonie in d-Moll@\strich\emph{3. Sinfonie in d-Moll}|pwv}« haben wir noch
                  Sitze}{\lemma{\textnormal{\emph{Mahler« … Sitze}}}\Cendnote{\textnormal{Mahler\pwindex{Mahler, Gustav 7.\,7.\,1860 Kaliště – 18.\,5.\,1911 Wien@\textsc{Mahler, Gustav} (7.\,7.\,1860 Kaliště – 18.\,5.\,1911 Wien), \emph{Theaterleiter, Komponist, Dirigent}|pwk} dirigierte seine \emph{3. Symphonie}\pwindex{Mahler, Gustav 7.\,7.\,1860 Kaliště – 18.\,5.\,1911 Wien@\textsc{Mahler, Gustav} (7.\,7.\,1860 Kaliště – 18.\,5.\,1911 Wien), \emph{Theaterleiter, Komponist, Dirigent}!3. Sinfonie in d-Moll@\strich\emph{3. Sinfonie in d-Moll}|pwk} im Musikvereinssaal\oindex{Wien@\textbf{Wien}!I., Innere Stadt@\textbf{I., Innere Stadt}!Musikverein@\textbf{Musikverein}, \emph{Konzertsaal}|pwk}.}}}\label{K_L01477-4}{ }\damage{be}kommen,{ }ſo{ }ſeh ich dich hoffentlich auch heute Abend.\pend
           
\pstart
           Jedenfalls aber sage oder schreibe mir pneumatiſch, ob du vielleicht Lust hätteſt, am
                  \uline{Samſtag} bei uns zu nachtmahlen.\pend
           
\pstart
           Herzlichst der deine{\\[\baselineskip]}\spacefill\mbox{Arthur}\pend
           \leftskip=0em{}
\pstart
           \noindent{}{\pb}Olga\pwindex{Schnitzler, Olga 17.\,1.\,1882 Wien – 13.\,1.\,1970 Lugano@\textsc{Schnitzler, Olga} (17.\,1.\,1882 Wien – 13.\,1.\,1970 Lugano), \emph{Schauspielerin, Sängerin}|pw} grüßt dich herzlich und{ }ſagt dir, daſs
                  sie \substVorne{}\textsuperscript{das}\substDazwischen{}von dem\substHinten{} was du anläßlich de\textcolor{gray}{s}{ }\textcolor{gray}{\textsc{P}\pwindex{Schnitzler, Arthur 15.\,5.\,1862 Wien – 21.\,10.\,1931 ebd.@\textsc{Schnitzler, Arthur} (15.\,5.\,1862 Wien – 21.\,10.\,1931 ebd.), \emph{Schriftsteller, Mediziner}!Puppenspieler. Studie in einem Aufzuge@\strich\emph{Der Puppenspieler. Studie in einem Aufzuge}|pwu}.}
                  geſchrieben haſt, erſchüttert war.\pend
           \selectlanguage{ngerman}\endnumbering\briefempfaengerindex{Bahr, Hermann@\textsc{Bahr, Hermann}!zzzSchnitzler, Arthur@\emph{von Arthur Schnitzler}!1904-12-141@{14. 12. 1904}|)be}\mylabel{L01477h}  \newcommand{\dateiname}{L01477}\newcommand{\titel}{Arthur Schnitzler an Hermann Bahr, 14. 12. 1904}\newcommand{\editorInnen}{Herausgegeben von Martin Anton Müller}%% latex-leseansicht-abspann.tex
%% Abspann für die Leseansicht.
%% Der Schalter \ifkorrekturansicht ist bereits durch den Vorspann gesetzt.

%% latex-abspann.tex
%% Gemeinsamer Abspann für Korrekturansicht und Leseansicht.
%% Setzt den Schalter \ifkorrekturansicht voraus (gesetzt in den
%% einbindenden Dateien latex-korrekturansicht-abspann.tex bzw.
%% latex-leseansicht-abspann.tex).
%% ---------------------------------------------------------------

\normalsize

% Das esempio-Environment wird nur in der Leseansicht benötigt
\ifkorrekturansicht\else
\newenvironment{esempio}[3]%
{
    \vspace{1.5ex}
    \rlap{\underline{#1}}
    \par
    \setlength{\parindent}{0cm}
    \nopagebreak
    \leftskip=#2cm
    \rightskip=#3cm
}
{
    \par
}
\fi

\doendnotes{C}
\bigskip
\vfill

\clearpage

\footnotesize

\ifkorrekturansicht
  \lohead{\textsc{register}}
\fi

% theindex-Environment neu definieren ohne reledmac
\makeatletter
\renewenvironment{theindex}{%
  \ifkorrekturansicht
    \section*{\indexname}%
  \else
    \subsubsection*{Index der erwähnten Entitäten}%
  \fi
  \setlength{\parindent}{0pt}%
  \setlength{\parskip}{0pt plus 0.3pt}%
  \let\item\@idxitem
}{%
  \ifkorrekturansicht\clearpage\fi
}
\makeatother

\IfFileExists{\jobname-pw.ind}{\input{\jobname-pw.ind}}{}

% Quellenangabe nur in der Leseansicht
\ifkorrekturansicht\else
% Fallback-Definitionen, falls die .tex-Datei \titel etc. nicht gesetzt hat
\providecommand{\titel}{}
\providecommand{\editorInnen}{}
\providecommand{\dateiname}{\jobname}

\vspace{3cm}

\vfill

\footnotesize
\textsc{Quelle}: \titel. Herausgegeben von {\editorInnen}. In: \emph{Arthur Schnitzler: Briefwechsel mit Autorinnen und Autoren}.
 Digitale Edition, https://schnitzler-briefe.acdh.oeaw.ac.at/{\dateiname}.html (Stand \today)
\fi

\end{document}


