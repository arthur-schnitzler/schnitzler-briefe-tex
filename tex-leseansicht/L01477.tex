%% latex-korrekturansicht-vorspann.tex
%% Vorspann für die Korrekturansicht.
%% Lädt die gemeinsame Datei latex-vorspann.tex mit gesetztem Schalter.

\newif\ifkorrekturansicht
\korrekturansichttrue

\input{../tex-inputs/latex-vorspann}


\section[Arthur Schnitzler an Hermann Bahr, 14. 12. 1904]{L01477 Arthur Schnitzler an Hermann Bahr, 14. 12. 1904}
\nopagebreak\mylabel{L01477v}
\rehead{ }\normalsize\beginnumbering\briefempfaengerindex{Bahr, Hermann@\textsc{Bahr, Hermann}!zzzSchnitzler, Arthur@\emph{von Arthur Schnitzler}!1904-12-141@{14. 12. 1904}|(be}
\toendnotes[C]{\smallbreak\pagebreak[2]}\Standort{TMW, HS AM 23368 Ba.}
\physDesc{Brief, 2 Blätter, 7 Seiten, 3106 Zeichen
\newline{}Handschrift: schwarze Tinte, deutsche Kurrent
\newline{}Ordnung: Lochung }
\buchAbdrucke{\weitereDrucke{1) Arthur Schnitzler: \emph{Briefe 1875–1912}. Frankfurt am Main: \emph{S. Fischer} 1981, S. 504–506.} \weitereDrucke{2) Arthur Schnitzler: \emph{The Letters of Arthur Schnitzler to Hermann Bahr}. Chapel Hill: \emph{The University of North Carolina Press} 1978, S. 86–87.} \weitereDrucke{3) Hermann Bahr, Arthur Schnitzler: \emph{Briefwechsel, Aufzeichnungen, Dokumente (1891–1931)}. Göttingen: \emph{Wallstein} 2018, S. 333–334.} }\toendnotes[C]{\smallbreak}
\pstart
           \raggedleft{}{\pb}Wien\oindex{Wien@\textbf{Wien}, \emph{A.ADM2}|pw}{ }14. 12. 904\pend
           \vspace{0.5em}
\pstart
           mein lieber Hermann, es beſchämt mich faſt, daſs du über ein im
               Ganzen doch ziemlich unbeträchtliches Ding wie es der Puppenſpieler\pwindex{Puppenspieler. Studie in einem Aufzuge@\emph{Der Puppenspieler. Studie in einem Aufzuge}|pw} iſt (er gehörte \label{K_L01477-1v}\edtext{in den Cyclus Lebendg Stunden\pwindex{Lebendige Stunden. Vier Einakter@\emph{Lebendige Stunden. Vier Einakter}|pw}}{\lemma{\textnormal{\emph{in … Stunden}}}\Cendnote{\textnormal{Vgl. Arthur Schnitzler an Hermann Bahr, 18. 10. 1901.
               }}}\label{K_L01477-1}, aber wegen zu großer Länge des Abends mußte er
               zurückge{[}ſe{]}tzt werden) – ſo ſchöne \label{K_L01477-2v}\edtext{Worte\pwindex{Puppenspieler. (Studie in einem Aufzuge von Arthur Schnitzler. Zum ersten Mal aufgefuehrt im Carl-Theater am 12. Dezember 1904)@\emph{Der Puppenspieler. (Studie in einem Aufzuge von Arthur Schnitzler. Zum ersten Mal aufgeführt im Carl-Theater am 12. Dezember 1904)}|pwv}}{\lemma{\textnormal{\emph{Worte}}}\Cendnote{\textnormal{Siehe Hermann Bahr, Arthur Schnitzler: \emph{Briefwechsel, Aufzeichnungen, Dokumente (1891–1931)}, Hermann Bahr: Der Puppenspieler, 13. 12. 1904.
               }}}\label{K_L01477-2} ſagſt. Vielleicht
               drücke ich mich beſſer aus, we{\geminationn} ich ſage: \uline{anläßlich} des Puppenſpielers\pwindex{Puppenspieler. Studie in einem Aufzuge@\emph{Der Puppenspieler. Studie in einem Aufzuge}|pw}. Denn deiner Auffaſſung des kleinen Stücks muſs ich
               widerſprechen. Vielleicht hab ich nicht das Recht dazu, denn es werden ja doch
               wahrſcheinlich künſt{\pb}leriſche Mängel der Sache ſchuld daran ſein, daſs du eine Lebensanſchauung darin
               findeſt, die ich nicht hineinlegen wollte und die mir perſönlich fremd iſt. Ebenſo
               verhält es ſich mit dem Einſ. Weg\pwindex{einsame Weg. Schauspiel in fuenf Akten@\emph{Der einsame Weg. Schauspiel in fünf Akten}|pw}. Ich ſtehe ſo
               wenig auf Seite des Oboëſpielers, als \introOben{}ich\introOben{} auf Seiten des Profeſſor Wegrath\pwindex{einsame Weg. Schauspiel in fuenf Akten@\emph{Der einsame Weg. Schauspiel in fünf Akten}|pwv} geſtanden
               habe – freilich auch nicht auf der des Julian und des Puppenſpielers. Aber warum?
               Weil ſie eben nicht ganze Kerle ſind, \introOben{}keine Leute\introOben{} die – nach
               der dir bekannten Anekdote von der alten Streitmann\pwindex{Streitmann, Katharina 1830/1831 – 1898-10-19@\textsc{Streitmann, Katharina} (1830/1831 – 1898-10-19)|pw} – »\label{K_L01477-3v}\edtext{brav genug}{\lemma{\textnormal{\emph{brav genug}}}\Cendnote{\textnormal{\emph{Berliner Tageblatt}\orgindex{Berliner Tageblatt@Berliner Tageblatt|pwk}, Jg. 54, Nr. 227, 14. 5. 1925, Abend-Blatt, S. 2: »Arthur Schnitzler unterhält sich mit
                     einem Freund über Leutnants Bilses\pwindex{Bilse, Fritz Oswald 1878-03-31 – 1951-08-30@\textsc{Bilse, Fritz Oswald} (1878-03-31 – 1951-08-30), \emph{Schriftsteller/Schriftstellerin, Militär/Militärin}|pw}
                     Schlüsselroman ›Aus einer kleinen
                     Garnison\pwindex{Aus einer kleinen Garnison@\emph{Aus einer kleinen Garnison}|pw}‹, und es entsteht die Frage, inwieweit ein Autor ein Recht habe,
                     wirkliche Vorkommnisse und Namen in ein Werk aufzunehmen, ›Die Frage‹, sagt Schnitzler, ›erinnert mich an eine
                     reizende Episode ans dem Leben des Tenors Streitmann\pwindex{Streitmann, Karl 1858-05-08 – 1937-10-29@\textsc{Streitmann, Karl} (1858-05-08 – 1937-10-29), \emph{Schauspieler/Schauspielerin, Sänger/Sängerin}|pw}; der war nämlich schon ein berühmter Operettenheld, ohne
                     daß ihn seine auf dem Land lebende Mutter je auf den Brettern gesehen hatte.
                     Eines Tages fährt sie nach Wien\oindex{Wien@\textbf{Wien}, \emph{A.ADM2}|pw}, begibt sich
                     – auf dem Zettel steht die ›Fledermaus\pwindex{Fledermaus. Komische Oper in drei Acten@\emph{Die Fledermaus. Komische Oper in drei Acten}|pw}‹ –
                     ins Theater, wo ihr Sohn auftritt. ›Nun?‹ fragt am Ende der Vorstellung Streitmann\pwindex{Streitmann, Karl 1858-05-08 – 1937-10-29@\textsc{Streitmann, Karl} (1858-05-08 – 1937-10-29), \emph{Schauspieler/Schauspielerin, Sänger/Sängerin}|pw}{ }seine Mutter\pwindex{Streitmann, Katharina 1830/1831 – 1898-10-19@\textsc{Streitmann, Katharina} (1830/1831 – 1898-10-19)|pwv}, ›wie habe ich dir gefallen?‹ – ›Sehr gut,
                     sehr brav, mein Kind – aber‹, und sie wird bedrückt, ›warum hast du nicht das
                     schöne Lied gesungen: ›Ach,
                        ich hab’ sie ja nur auf die Schulter geküßt\pwindex{Bettelstudent@\emph{Der Bettelstudent}|pwv}?‹ – ›Aber Mama,‹ sagte der
                     Tenor, ›das kommt ja gar nicht in dieser Operette vor.‹ – ›Schön, kommt nicht
                     vor {\dots} aber warum hast du’s nicht doch gesungen?‹ –
                     ›Aber Mama, verstehst du nicht – ich hätt’ es ja gar nicht singen dürfen.‹
                     Darauf ein langer, mißtrauischer Blick der Mutter\pwindex{Streitmann, Katharina 1830/1831 – 1898-10-19@\textsc{Streitmann, Katharina} (1830/1831 – 1898-10-19)|pwv}: ›Wenn man brav ist, mein Kind, darf man
                     alles.‹ ›Das ist‹, fügt Schnitzler hinzu,
                     ›auch meine Meinung über den Schlüsselroman.‹«}}}\label{K_L01477-3}« ſind – um alles
               zu dürfen. Wäre der Puppen{\pb}ſpieler\pwindex{Puppenspieler. Studie in einem Aufzuge@\emph{Der Puppenspieler. Studie in einem Aufzuge}|pwv}
               wirklich ein »Großer«, ſo bräuchte er ſich nicht in Lügen einzuſpinnen, um der
               größere zu bleiben – wäre Julian\pwindex{einsame Weg. Schauspiel in fuenf Akten@\emph{Der einsame Weg. Schauspiel in fünf Akten}|pwv} wirklich ein Großer – ſo würde das beſte ſeines Weſens nicht mit
               seiner Jugend auslöſchen. Gegen die Herzöge und gegen die \textsc{Sala\pwindex{einsame Weg. Schauspiel in fuenf Akten@\emph{Der einsame Weg. Schauspiel in fünf Akten}|pwv}}’s hab ich nichts – und vor den »Großen Räubern« ſalutir ich, gleich dir, in
               Ehrfurcht. Du haſt ganz recht: »Entſagung iſt nicht immer Reife.« – – nur ſetze ich
               hinzu: nicht bei allen. Wenn Individuen wie \uline{Wegrath}\pwindex{einsame Weg. Schauspiel in fuenf Akten@\emph{Der einsame Weg. Schauspiel in fünf Akten}|pwv} in irgend einem Moment ihrer Exiſtenz die Grenzen ihrer Begabung erkennen, –
                  {\pb}so iſt \uline{dieſe} Entſagung, wie jede \uline{Erke{\geminationn}tnis} innere Reife, oder wenigſtens ein
               Symptom innerer Reife. Ebenſo iſt für den Oboëſpieler wirklich der »Innere Friede und
               die ſchuldbefreite Bruſt« das einzig erreichbare Glück. Und daſs ein Menſch wie der
                  »Puppenſpieler\pwindex{Puppenspieler. Studie in einem Aufzuge@\emph{Der Puppenspieler. Studie in einem Aufzuge}|pwv}« nicht, wie
               es eben den Beſchränkungen ſeines Weſens angemeſſen wäre, \introOben{}zu\introOben{}
               entſagen im Stande iſt, ſich \introOben{}vielmehr\introOben{} dieſer Entſagung
               und daher den andern u ſich ein \introOben{}falſches\introOben{} Eigenſchickſal
               vorſpielt – iſt ein Zeichen, daſs er innere Reife nicht erlangte, welche eben nur in
               Selbſterkenntnis beſtehen kann. \strikeout{Daher} Es iſt {\pb}alſo nur natürlich,
               daſs bei manchen Menſchen, insbeſondre bei klugen, von mäßigem Talente und ſtillem
               Temperamente das was ihnen an innerer Reife überhaupt beſchieden iſt, in einer Art
               von »Entſagung« den entſprechenden Ausdruck findet.\pend
           
\pstart
           Wohl denen, die’s nicht nöthig haben, – wohl uns, die wir wie mir ſcheint zu dieſen
               gehören – und hoffentlich nicht allein wegen Mangels an Klugheit. So ſpricht alſo
               nichts dagegen, mein lieber Hermann, daſs wir beide uns an die Arbeit machen, die du
               in meine {\pb}Hände legſt:
                  »\label{LL075-1v}Das Werk von der letzten Nacht einer alten
                  Zeit\label{LL075-1h}« – Und ſchließlich können es auch andre Werke ſein.\pend
           
\pstart
           Zu »\label{K_L01477-4v}\edtext{Mahler\pwindex{3. Sinfonie in d-Moll@\emph{3. Sinfonie in d-Moll}|pwv}« haben wir noch
                  Sitze}{\lemma{\textnormal{\emph{Mahler« … Sitze}}}\Cendnote{\textnormal{Mahler\pwindex{Mahler, Gustav 07.07.1860 – 18.05.1911@\textsc{Mahler, Gustav} (07.07.1860 – 18.05.1911), \emph{Theaterleiter/Theaterleiterin, Komponist/Komponistin, Dirigent/Dirigentin}|pwk} dirigierte seine \emph{3. Symphonie}\pwindex{3. Sinfonie in d-Moll@\emph{3. Sinfonie in d-Moll}|pwk} im Musikvereinssaal\oindex{Musikverein@\textbf{Musikverein}, \emph{Konzertsaal (K.KNZ)}|pwk}.}}}\label{K_L01477-4}{ }\damage{be}kommen, ſo ſeh ich dich hoffentlich auch heute Abend.\pend
           
\pstart
           Jedenfalls aber sage oder schreibe mir pneumatiſch, ob du vielleicht Lust hätteſt, am
                  \uline{Samſtag} bei uns zu nachtmahlen.\pend
           
\pstart
           Herzlichst der deine{\\[\baselineskip]}\spacefill\mbox{Arthur{\pb}}\pend
           \leftskip=0em{}
\pstart
           \noindent{}Olga\pwindex{Schnitzler, Olga 17.01.1882 – 13.01.1970@\textsc{Schnitzler, Olga} (17.01.1882 – 13.01.1970), \emph{Schauspieler/Schauspielerin, Sänger/Sängerin}|pw} grüßt dich herzlich und ſagt dir, daſs
                  sie \substVorne{}\textsuperscript{das}\substDazwischen{}von dem\substHinten{} was du anläßlich de\textcolor{gray}{s}{ }\textcolor{gray}{\textsc{P}\pwindex{Puppenspieler. Studie in einem Aufzuge@\emph{Der Puppenspieler. Studie in einem Aufzuge}|pwu}.}
                  geſchrieben haſt, erſchüttert war.\pend
           \selectlanguage{ngerman}\endnumbering\briefempfaengerindex{Bahr, Hermann@\textsc{Bahr, Hermann}!zzzSchnitzler, Arthur@\emph{von Arthur Schnitzler}!1904-12-141@{14. 12. 1904}|)be}\mylabel{L01477h}  \normalsize

\doendnotes{C}
\bigskip
\vfill

\clearpage

\footnotesize

\lohead{\textsc{register}}

% Definiere theindex-Environment komplett neu ohne reledmac
\makeatletter
\renewenvironment{theindex}{%
  \section*{\indexname}%
  \setlength{\parindent}{0pt}%
  \setlength{\parskip}{0pt plus 0.3pt}%
  \let\item\@idxitem
}{%
  \clearpage
}
\makeatother

\IfFileExists{\jobname-pw.ind}{\input{\jobname-pw.ind}}{}

\end{document}

      