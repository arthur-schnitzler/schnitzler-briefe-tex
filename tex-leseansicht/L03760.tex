%% latex-leseansicht-vorspann.tex
%% Vorspann für die Leseansicht.
%% Lädt die gemeinsame Datei latex-vorspann.tex mit nicht gesetztem Schalter.

\newif\ifkorrekturansicht
\korrekturansichtfalse

\input{../tex-inputs/latex-vorspann}


\section[Arthur Schnitzler an Stefan Zweig, 10. 4. 1919]{L03760 Arthur Schnitzler an Stefan Zweig, 10. 4. 1919}
\nopagebreak\mylabel{L03760v}
\rehead{ }\normalsize\beginnumbering\briefempfaengerindex{Zweig, Stefan@\textsc{Zweig, Stefan}!zzzSchnitzler, Arthur@\emph{von Arthur Schnitzler}!1919-04-101@{10. 4. 1919}|(be}
\toendnotes[C]{\smallbreak\pagebreak[2]}
\correspDesc{Versand  durch Arthur Schnitzler am 10. 4. 1919 in Wien
\newline{}Erhalt  durch Stefan Zweig im Zeitraum [10. 4. 1919 – 13. 4. 1919?] in Wien}\toendnotes[C]{\smallbreak}
\Standort{Jerusalem, National Library of Israel, ARC. Ms. Var. 305 1 58 Stefan Zweig Collection.}
\physDesc{Postkarte, 322 Zeichen
\newline{}Handschrift: schwarze Tinte, deutsche Kurrent
\newline{}Versand: Stempel: »\nobreak{}10. IV. 19, XII\nobreak{}«.  }\toendnotes[C]{\smallbreak}\pstart{}{\pb}Wien XVIII\oindex{XVIII., Währing@\textbf{XVIII., Währing}, \emph{Verwaltungsgebiet}|pw}\pend{}\pstart{}\textsc{Sternwartestr 71}\oindex{Wien@\textbf{Wien}!XVIII., Währing@\textbf{XVIII., Währing}!Sternwartestraße 71@\textbf{Sternwartestraße 71}, \emph{Wohngebäude}|pw}.\pend{}{\bigskip}\pstart{}Hrn Dr \textsc{Stefan Zweig}\pend{}\pstart{}Wien VIII\oindex{VIII., Josefstadt@\textbf{VIII., Josefstadt}, \emph{Verwaltungsgebiet}|pw}\pend{}\pstart{}\textsc{Kochgasse 8\oindex{Wien@\textbf{Wien}!VIII., Josefstadt@\textbf{VIII., Josefstadt}!Kochgasse 8@\textbf{Kochgasse 8}, \emph{Wohngebäude}|pw}}.\pend{}{\bigskip}\vspace{1em}
\pstart
           \raggedleft{}{\pb}10. 4. 19\pend
           \vspace{0.5em}
\pstart
           lieber und verehrter Herr Doktor, Ihr{ }ſchönes Verhaeren\pwindex{Verhaeren, Émile 21.\,5.\,1855 Sint-Amands – 27.\,11.\,1916 Rouen@\textsc{Verhaeren, Émile} (21.\,5.\,1855 Sint-Amands – 27.\,11.\,1916 Rouen), \emph{Schriftsteller, Schriftsteller, Krimiautor}|pw} Buch\pwindex{Zweig, Stefan 28.\,11.\,1881 Wien – 23.\,2.\,1942 Petrópolis@\textsc{Zweig, Stefan} (28.\,11.\,1881 Wien – 23.\,2.\,1942 Petrópolis), \emph{Schriftsteller}!Erinnerungen an Émile Verhaeren@\strich\emph{Erinnerungen an Émile Verhaeren}|pwv} hab ich mit
               Ergriffenheit \label{K_L03760-1v}\edtext{geleſen}{\lemma{\textnormal{\emph{gelesen}}}\Cendnote{\textnormal{Vgl. A. S.: \emph{Tagebuch}, 9. 4. 1919. \emph{Erinnerungen an Émile Verhaeren}\pwindex{Zweig, Stefan 28.\,11.\,1881 Wien – 23.\,2.\,1942 Petrópolis@\textsc{Zweig, Stefan} (28.\,11.\,1881 Wien – 23.\,2.\,1942 Petrópolis), \emph{Schriftsteller}!Erinnerungen an Émile Verhaeren@\strich\emph{Erinnerungen an Émile Verhaeren}|pwk} erschien
               als Privatdruck.}}}\label{K_L03760-1} und hoffe Ihnen bald \label{K_L03760-2v}\edtext{perſönlich}{\lemma{\textnormal{\emph{persönlich}}}\Cendnote{\textnormal{Das nächste persönliche Zusammentreffen fand 
                  am 22. 4. 1919 statt.}}}\label{K_L03760-2} danken u bei dieser Gelegenheit
               den \label{K_L03760-3v}\edtext{Heimgekehrten}{\lemma{\textnormal{\emph{Heimgekehrten}}}\Cendnote{\textnormal{Am 24. 3. 1919 kehrte
                  Zweig\pwindex{Zweig, Stefan 28.\,11.\,1881 Wien – 23.\,2.\,1942 Petrópolis@\textsc{Zweig, Stefan} (28.\,11.\,1881 Wien – 23.\,2.\,1942 Petrópolis), \emph{Schriftsteller}|pwk} nach einem Aufenthalt von fast 14 Monaten aus der Schweiz\oindex{Schweiz@\textbf{Schweiz}|pwk} zurück. In der Zwischenzeit
                  dürfte es, mit Ausnahme eines nicht überlieferten Telegramms (vgl. A. S.: \emph{Tagebuch}, 26. 4. 1918) keinen Kontakt gegeben haben.}}}\label{K_L03760-3} herzlich willko{\geminationm}en heißen zu dürfen. Wir grüßen Sie {\pb}vielmals.\pend
           
\pstart
           Ihr{\\[\baselineskip]}\spacefill\mbox{Arthur Schnitzler}\pend
           \leftskip=0em{}\selectlanguage{ngerman}\endnumbering\briefempfaengerindex{Zweig, Stefan@\textsc{Zweig, Stefan}!zzzSchnitzler, Arthur@\emph{von Arthur Schnitzler}!1919-04-101@{10. 4. 1919}|)be}\mylabel{L03760h}  \newcommand{\dateiname}{L03760}\newcommand{\titel}{Arthur Schnitzler an Stefan Zweig, 10. 4. 1919}\newcommand{\editorInnen}{Selma Jahnke und Martin Anton Müller}%% latex-leseansicht-abspann.tex
%% Abspann für die Leseansicht.
%% Der Schalter \ifkorrekturansicht ist bereits durch den Vorspann gesetzt.

%% latex-abspann.tex
%% Gemeinsamer Abspann für Korrekturansicht und Leseansicht.
%% Setzt den Schalter \ifkorrekturansicht voraus (gesetzt in den
%% einbindenden Dateien latex-korrekturansicht-abspann.tex bzw.
%% latex-leseansicht-abspann.tex).
%% ---------------------------------------------------------------

\normalsize

% Das esempio-Environment wird nur in der Leseansicht benötigt
\ifkorrekturansicht\else
\newenvironment{esempio}[3]%
{
    \vspace{1.5ex}
    \rlap{\underline{#1}}
    \par
    \setlength{\parindent}{0cm}
    \nopagebreak
    \leftskip=#2cm
    \rightskip=#3cm
}
{
    \par
}
\fi

\doendnotes{C}
\bigskip
\vfill

\clearpage

\footnotesize

\ifkorrekturansicht
  \lohead{\textsc{register}}
\fi

% theindex-Environment neu definieren ohne reledmac
\makeatletter
\renewenvironment{theindex}{%
  \ifkorrekturansicht
    \section*{\indexname}%
  \else
    \subsubsection*{Index der erwähnten Entitäten}%
  \fi
  \setlength{\parindent}{0pt}%
  \setlength{\parskip}{0pt plus 0.3pt}%
  \let\item\@idxitem
}{%
  \ifkorrekturansicht\clearpage\fi
}
\makeatother

\IfFileExists{\jobname-pw.ind}{\input{\jobname-pw.ind}}{}

% Quellenangabe nur in der Leseansicht
\ifkorrekturansicht\else
% Fallback-Definitionen, falls die .tex-Datei \titel etc. nicht gesetzt hat
\providecommand{\titel}{}
\providecommand{\editorInnen}{}
\providecommand{\dateiname}{\jobname}

\vspace{3cm}

\vfill

\footnotesize
\textsc{Quelle}: \titel. Herausgegeben von {\editorInnen}. In: \emph{Arthur Schnitzler: Briefwechsel mit Autorinnen und Autoren}.
 Digitale Edition, https://schnitzler-briefe.acdh.oeaw.ac.at/{\dateiname}.html (Stand \today)
\fi

\end{document}


