%% latex-leseansicht-vorspann.tex
%% Vorspann für die Leseansicht.
%% Lädt die gemeinsame Datei latex-vorspann.tex mit nicht gesetztem Schalter.

\newif\ifkorrekturansicht
\korrekturansichtfalse

\input{../tex-inputs/latex-vorspann}


               \section[Arthur Schnitzler an Georg Brandes, 30. 11. 1918]{ Arthur Schnitzler an Georg Brandes, 30. 11. 1918}\nopagebreak\mylabel{v}\rehead{ }\begin{ledgroupsized}[t]{13cm}\normalsize\beginnumbering\briefempfaengerindex{Brandes, Georg@\textsc{Brandes, Georg}!zzzSchnitzler, Arthur@\emph{von Arthur Schnitzler}!1918-11-301@{30. 11. 1918}|(be} \toendnotes[C]{\smallbreak\pagebreak[2]} \Standort{Kopenhagen, Det Kongelige Bibliotek, Georg Brandes Arkiv, box 125.}
\physDesc{Brief, 1 Blatt, 1 Seite
\newline{}Handschrift: schwarze Tinte, lateinische Kurrent\newline{}Ordnung: mit Bleistift von unbekannter Hand beschriftet:
                                        »Schnitzler« und nummeriert:
                                        »41.« }\buchAbdrucke{\weitereDrucke{Georg Brandes, Arthur Schnitzler: \emph{Ein Briefwechsel}. Hg. Kurt Bergel. Bern: \emph{Francke} 1956, S. 125–126.} }\toendnotes[C]{\smallbreak}\pstart
           \raggedleft{}{\pb}Wien\oindex{Wien@\textbf{Wien}|pw}, 30. 11. 918\pend
           \pstart{}Lieber und verehrter Herr Brandes\pend\pstart
           Darf ich Sie bitten, Herrn Sonne\pwindex{Sonne, Abraham 13.09.1883 – 29.03.1950@\textsc{Sonne, Abraham} (13.09.1883 – 29.03.1950), \emph{Schriftsteller, Zionist}|pw}, der Ihnen
                    die herzlichsten Grüße überbringt, freundlich aufzunehmen? Er reist in
                    national-jüdischen Angelegenheiten nach Kopenhagen\oindex{Kopenhagen@\textbf{Kopenhagen}|pw}, und von dort weiter, und wird Ihnen, wenn Sie es gestatten
                    allerlei berichten, was Sie sehr interessiren wird. Jedenfalls werden Sie in ihm
                    einen sehr klugen, höchst unterrichteten und in bestem Sinne thätigen Mann
                    kennen lernen.\pend
           \pstart
           Lassen Sie mich Ihnen heute nur flüchtig für Ihren letzten Brief danken – in den
                    nächsten Tagen soll es ausführlicher geschehn – und hoffentlich läßt sich bald
                    schöneres erzählen als es heute möglich wäre. Die Meinen sind alle wohl; – und
                    ich arbeite so gut es geht; – aber es geht nicht gut. Immerhin erhalten Sie eine
                    neue \label{T_L02312_1v}\edtext{Novelle\pwindex{Schnitzler, Arthur 15.05.1862 – 21.10.1931@\textsc{Schnitzler, Arthur} (15.05.1862 – 21.10.1931), \emph{Schriftsteller, Mediziner}!Casanovas Heimfahrt1.7.1918 – 1.9.1918@\strich\emph{Casanovas Heimfahrt} {[}1.7.1918 – 1.9.1918{]}|pwv}}{\lemma{\textnormal{\emph{Novelle}}}\Cendnote{\textnormal{ab hier weiter am linken Rand}}}\label{T_L02312_1h}
                    von mir zugeschickt! Von Herzen\pend
           \pstart Ihr \spacefill\mbox{Arthur Schnitzler}\pend{}\endnumbering\briefempfaengerindex{Brandes, Georg@\textsc{Brandes, Georg}!zzzSchnitzler, Arthur@\emph{von Arthur Schnitzler}!1918-11-301@{30. 11. 1918}|)be}\mylabel{h}\end{ledgroupsized}  \newcommand{\dateiname}{L02312}\newcommand{\titel}{Arthur Schnitzler an Georg Brandes, 30. 11. 1918}\newcommand{\editorInnen}{Martin Anton Müller und Gerd-Hermann Susen}
            \footnotesize
\begin{ledgroupsized}[t]{11.5cm}
\doendnotes{C}
\end{ledgroupsized}
         %% latex-leseansicht-abspann.tex
%% Abspann für die Leseansicht.
%% Der Schalter \ifkorrekturansicht ist bereits durch den Vorspann gesetzt.

%% latex-abspann.tex
%% Gemeinsamer Abspann für Korrekturansicht und Leseansicht.
%% Setzt den Schalter \ifkorrekturansicht voraus (gesetzt in den
%% einbindenden Dateien latex-korrekturansicht-abspann.tex bzw.
%% latex-leseansicht-abspann.tex).
%% ---------------------------------------------------------------

\normalsize

% Das esempio-Environment wird nur in der Leseansicht benötigt
\ifkorrekturansicht\else
\newenvironment{esempio}[3]%
{
    \vspace{1.5ex}
    \rlap{\underline{#1}}
    \par
    \setlength{\parindent}{0cm}
    \nopagebreak
    \leftskip=#2cm
    \rightskip=#3cm
}
{
    \par
}
\fi

\doendnotes{C}
\bigskip
\vfill

\clearpage

\footnotesize

\ifkorrekturansicht
  \lohead{\textsc{register}}
\fi

% theindex-Environment neu definieren ohne reledmac
\makeatletter
\renewenvironment{theindex}{%
  \ifkorrekturansicht
    \section*{\indexname}%
  \else
    \subsubsection*{Index der erwähnten Entitäten}%
  \fi
  \setlength{\parindent}{0pt}%
  \setlength{\parskip}{0pt plus 0.3pt}%
  \let\item\@idxitem
}{%
  \ifkorrekturansicht\clearpage\fi
}
\makeatother

\IfFileExists{\jobname-pw.ind}{\input{\jobname-pw.ind}}{}

% Quellenangabe nur in der Leseansicht
\ifkorrekturansicht\else
% Fallback-Definitionen, falls die .tex-Datei \titel etc. nicht gesetzt hat
\providecommand{\titel}{}
\providecommand{\editorInnen}{}
\providecommand{\dateiname}{\jobname}

\vspace{3cm}

\vfill

\footnotesize
\textsc{Quelle}: \titel. Herausgegeben von {\editorInnen}. In: \emph{Arthur Schnitzler: Briefwechsel mit Autorinnen und Autoren}.
 Digitale Edition, https://schnitzler-briefe.acdh.oeaw.ac.at/{\dateiname}.html (Stand \today)
\fi

\end{document}


      