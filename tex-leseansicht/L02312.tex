%% latex-korrekturansicht-vorspann.tex
%% Vorspann für die Korrekturansicht.
%% Lädt die gemeinsame Datei latex-vorspann.tex mit gesetztem Schalter.

\newif\ifkorrekturansicht
\korrekturansichttrue

\input{../tex-inputs/latex-vorspann}


\section[Arthur Schnitzler an Georg Brandes, 30. 11. 1918]{L02312 Arthur Schnitzler an Georg Brandes, 30. 11. 1918}
\nopagebreak\mylabel{L02312v}
\rehead{ }\normalsize\beginnumbering\briefempfaengerindex{Brandes, Georg@\textsc{Brandes, Georg}!zzzSchnitzler, Arthur@\emph{von Arthur Schnitzler}!1918-11-301@{30. 11. 1918}|(be}
\toendnotes[C]{\smallbreak\pagebreak[2]}\Standort{Kopenhagen, Det Kongelige Bibliotek, Georg Brandes Arkiv, box 125.}
\physDesc{Brief, 1 Blatt, 1 Seite, 825 Zeichen
\newline{}Handschrift: schwarze Tinte, lateinische Kurrent
\newline{}Ordnung: mit Bleistift von unbekannter Hand beschriftet:
                                    »Schnitzler« und nummeriert:
                                 »41.« }
\buchAbdrucke{\weitereDrucke{Georg Brandes, Arthur Schnitzler: \emph{Ein Briefwechsel}. Bern: \emph{Francke} 1956, S. 125–126.} }\toendnotes[C]{\smallbreak}
\pstart
           \raggedleft{}{\pb}Wien\oindex{Wien@\textbf{Wien}, \emph{A.ADM2}|pw}, 30. 11. 918\pend
           
\pstart{}Lieber und verehrter Herr Brandes\pend\vspace{0.5em}
\pstart
           Darf ich Sie bitten, Herrn Sonne\pwindex{Sonne, Abraham 13.09.1883 – 29.03.1950@\textsc{Sonne, Abraham} (13.09.1883 – 29.03.1950), \emph{Schriftsteller/Schriftstellerin, Zionist/Zionistin}|pw}, der Ihnen die
               herzlichsten Grüße überbringt, freundlich aufzunehmen? Er reist in national-jüdischen
               Angelegenheiten nach Kopenhagen\oindex{Kopenhagen@\textbf{Kopenhagen}, \emph{P.PPLC}|pw}, und von dort
               weiter, und wird Ihnen, wenn Sie es gestatten allerlei berichten, was Sie sehr
               interessiren wird. Jedenfalls werden Sie in ihm einen sehr klugen, höchst
               unterrichteten und in bestem Sinne thätigen Mann kennen lernen.\pend
           
\pstart
           Lassen Sie mich Ihnen heute nur flüchtig für Ihren letzten Brief danken – in den
               nächsten Tagen soll es ausführlicher geschehn – und hoffentlich läßt sich bald
               schöneres erzählen als es heute möglich wäre. Die Meinen sind alle wohl; – und ich
               arbeite so gut es geht; – aber es geht nicht gut. Immerhin erhalten Sie eine neue
                  \label{T_L02312-1v}\edtext{Novelle\pwindex{Casanovas Heimfahrt@\emph{Casanovas Heimfahrt}|pwv}}{\lemma{\textnormal{\emph{Novelle}}}\Cendnote{\textnormal{ab hier weiter am linken Rand}}}\label{T_L02312-1} von
               mir zugeschickt! Von Herzen\pend
           \pstart Ihr \spacefill\mbox{Arthur Schnitzler}\pend{}\selectlanguage{ngerman}\endnumbering\briefempfaengerindex{Brandes, Georg@\textsc{Brandes, Georg}!zzzSchnitzler, Arthur@\emph{von Arthur Schnitzler}!1918-11-301@{30. 11. 1918}|)be}\mylabel{L02312h}  \normalsize

\doendnotes{C}
\bigskip
\vfill

\clearpage

\footnotesize

\lohead{\textsc{register}}

% Definiere theindex-Environment komplett neu ohne reledmac
\makeatletter
\renewenvironment{theindex}{%
  \section*{\indexname}%
  \setlength{\parindent}{0pt}%
  \setlength{\parskip}{0pt plus 0.3pt}%
  \let\item\@idxitem
}{%
  \clearpage
}
\makeatother

\IfFileExists{\jobname-pw.ind}{\input{\jobname-pw.ind}}{}

\end{document}

      