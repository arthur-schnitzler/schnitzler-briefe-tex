%% latex-leseansicht-vorspann.tex
%% Vorspann für die Leseansicht.
%% Lädt die gemeinsame Datei latex-vorspann.tex mit nicht gesetztem Schalter.

\newif\ifkorrekturansicht
\korrekturansichtfalse

\input{../tex-inputs/latex-vorspann}


\section[Arthur Schnitzler an Georg Brandes, 30. 11. 1918]{L02312 Arthur Schnitzler an Georg Brandes, 30. 11. 1918}
\nopagebreak\mylabel{L02312v}
\rehead{ }\normalsize\beginnumbering\briefempfaengerindex{Brandes, Georg@\textsc{Brandes, Georg}!zzzSchnitzler, Arthur@\emph{von Arthur Schnitzler}!1918-11-301@{30. 11. 1918}|(be}
\toendnotes[C]{\smallbreak\pagebreak[2]}
\correspDesc{Versand  durch Arthur Schnitzler am 30. 11. 1918 in Wien
\newline{}Erhalt  durch Georg Brandes im Zeitraum [1. 12. 1918
                  – 5. 12. 1918?] in Kopenhagen}\toendnotes[C]{\smallbreak}
\Standort{Kopenhagen, Det Kongelige Bibliotek, Georg Brandes Arkiv, box 125.}
\physDesc{Brief, 1 Blatt, 1 Seite, 825 Zeichen
\newline{}Handschrift: schwarze Tinte, lateinische Kurrent
\newline{}Ordnung: mit Bleistift von unbekannter Hand beschriftet:
                                    »Schnitzler« und nummeriert:
                                 »41.« }
\buchAbdrucke{\weitereDrucke{Georg Brandes, Arthur Schnitzler: \emph{Ein Briefwechsel}. Herausgegeben von Kurt Bergel. Bern: \emph{Francke} 1956, S. 125–126.} }\toendnotes[C]{\smallbreak}
\pstart
           \raggedleft{}{\pb}Wien\oindex{Wien@\textbf{Wien}, \emph{Verwaltungsgebiet}|pw}, 30. 11. 918\pend
           
\pstart{}Lieber und verehrter Herr Brandes\pend\vspace{0.5em}
\pstart
           Darf ich Sie bitten, Herrn Sonne\pwindex{Sonne, Abraham 13.\,9.\,1883 Przemyśl – 29.\,3.\,1950 Hod HaSharon@\textsc{Sonne, Abraham} (13.\,9.\,1883 Przemyśl – 29.\,3.\,1950 Hod HaSharon), \emph{Schriftsteller, Zionist}|pw}, der Ihnen die
               herzlichsten Grüße überbringt, freundlich aufzunehmen? Er reist in national-jüdischen
               Angelegenheiten nach Kopenhagen\oindex{Kopenhagen@\textbf{Kopenhagen}, \emph{Hauptstadt}|pw}, und von dort
               weiter, und wird Ihnen, wenn Sie es gestatten allerlei berichten, was Sie sehr
               interessiren wird. Jedenfalls werden Sie in ihm einen sehr klugen, höchst
               unterrichteten und in bestem Sinne thätigen Mann kennen lernen.\pend
           
\pstart
           Lassen Sie mich Ihnen heute nur flüchtig für Ihren letzten Brief danken – in den
               nächsten Tagen soll es ausführlicher geschehn – und hoffentlich läßt sich bald
               schöneres erzählen als es heute möglich wäre. Die Meinen sind alle wohl; – und ich
               arbeite so gut es geht; – aber es geht nicht gut. Immerhin erhalten Sie eine neue
                  \label{T_L02312-1v}\edtext{Novelle\pwindex{Schnitzler, Arthur 15.\,5.\,1862 Wien – 21.\,10.\,1931 ebd.@\textsc{Schnitzler, Arthur} (15.\,5.\,1862 Wien – 21.\,10.\,1931 ebd.), \emph{Schriftsteller, Mediziner}!Casanovas Heimfahrt@\strich\emph{Casanovas Heimfahrt}|pwv}}{\lemma{\textnormal{\emph{Novelle}}}\Cendnote{\textnormal{ab hier weiter am linken Rand}}}\label{T_L02312-1} von
               mir zugeschickt! Von Herzen\pend
           \pstart Ihr \spacefill\mbox{Arthur Schnitzler}\pend{}\selectlanguage{ngerman}\endnumbering\briefempfaengerindex{Brandes, Georg@\textsc{Brandes, Georg}!zzzSchnitzler, Arthur@\emph{von Arthur Schnitzler}!1918-11-301@{30. 11. 1918}|)be}\mylabel{L02312h}  \newcommand{\dateiname}{L02312}\newcommand{\titel}{Arthur Schnitzler an Georg Brandes, 30. 11. 1918}\newcommand{\editorInnen}{Martin Anton Müller und Gerd-Hermann Susen}%% latex-leseansicht-abspann.tex
%% Abspann für die Leseansicht.
%% Der Schalter \ifkorrekturansicht ist bereits durch den Vorspann gesetzt.

%% latex-abspann.tex
%% Gemeinsamer Abspann für Korrekturansicht und Leseansicht.
%% Setzt den Schalter \ifkorrekturansicht voraus (gesetzt in den
%% einbindenden Dateien latex-korrekturansicht-abspann.tex bzw.
%% latex-leseansicht-abspann.tex).
%% ---------------------------------------------------------------

\normalsize

% Das esempio-Environment wird nur in der Leseansicht benötigt
\ifkorrekturansicht\else
\newenvironment{esempio}[3]%
{
    \vspace{1.5ex}
    \rlap{\underline{#1}}
    \par
    \setlength{\parindent}{0cm}
    \nopagebreak
    \leftskip=#2cm
    \rightskip=#3cm
}
{
    \par
}
\fi

\doendnotes{C}
\bigskip
\vfill

\clearpage

\footnotesize

\ifkorrekturansicht
  \lohead{\textsc{register}}
\fi

% theindex-Environment neu definieren ohne reledmac
\makeatletter
\renewenvironment{theindex}{%
  \ifkorrekturansicht
    \section*{\indexname}%
  \else
    \subsubsection*{Index der erwähnten Entitäten}%
  \fi
  \setlength{\parindent}{0pt}%
  \setlength{\parskip}{0pt plus 0.3pt}%
  \let\item\@idxitem
}{%
  \ifkorrekturansicht\clearpage\fi
}
\makeatother

\IfFileExists{\jobname-pw.ind}{\input{\jobname-pw.ind}}{}

% Quellenangabe nur in der Leseansicht
\ifkorrekturansicht\else
% Fallback-Definitionen, falls die .tex-Datei \titel etc. nicht gesetzt hat
\providecommand{\titel}{}
\providecommand{\editorInnen}{}
\providecommand{\dateiname}{\jobname}

\vspace{3cm}

\vfill

\footnotesize
\textsc{Quelle}: \titel. Herausgegeben von {\editorInnen}. In: \emph{Arthur Schnitzler: Briefwechsel mit Autorinnen und Autoren}.
 Digitale Edition, https://schnitzler-briefe.acdh.oeaw.ac.at/{\dateiname}.html (Stand \today)
\fi

\end{document}


