%% latex-leseansicht-vorspann.tex
%% Vorspann für die Leseansicht.
%% Lädt die gemeinsame Datei latex-vorspann.tex mit nicht gesetztem Schalter.

\newif\ifkorrekturansicht
\korrekturansichtfalse

\input{../tex-inputs/latex-vorspann}


         
         \renewcommand{\erwaehntePersonen}{Personen: Richard Beer-Hofmann, Paula Beer-Hofmann, Max Eugen Burckhard, Jakob Julius David, Julius von Gans-Ludassy, Stefan Großmann, Maximilian Harden, Hugo von Hofmannsthal, Heinrich Kanner, Louise Metzl, Wolfgang Amadeus Mozart, Josef Redlich, Anna Katharina Rehmann, Felix Salten, Ottilie Salten, Paul Salten, Olga Schnitzler, Wilhelm Singer, Leopold Ferdinand Salvator Wölfling}
         \renewcommand{\erwaehnteInstitutionen}{Institutionen: Arbeiter-Zeitung, B.Z. am Mittag, Burgtheater, Concordia, Die Zeit, Neues Wiener Tagblatt, Wiener Allgemeine Zeitung, Wiener Beamtenverein}
         \renewcommand{\erwaehnteOrte}{Orte: Berlin, Hotel Saxonia, Kantstraße, Kochstraße, Russland, Volkstheater, Wien, Zoologischer Garten Berlin}
         \renewcommand{\erwaehnteWerke}{Werke: ?? [Ludassy will von Salten erpresst worden sein], B.Z. am Mittag, Der Ruf des Lebens. Schauspiel in drei Akten, Der goldene Boden. Volksstück in vier Aufzügen, Der letzte Knopf. Volksstück in drei Aufzügen, Deutsches Volkstheater. (»Der goldene Boden«, Volksstück in vier Aufzügen von Julius v. Gans-Ludassy. 26. März), Deutsches Volkstheater. (»Der letzte Knopf.« Volksstück in drei Aufzügen von J. v. Gans-Ludassy), Die Zeit, Die Zukunft, Erinnerungen, Extrapost. Unparteiische Montags-Zeitung, Frankfurter Zeitung, Gedenkrede auf Wolfgang Amade Mozart, Kritik der Kritik. (Erbauliches aus der »Concordia«.), Oedipus und die Sphinx. Tragödie in drei Aufzügen, Russisches Theater. I, Russisches Theater. II, Theater, Wiener Allgemeine Zeitung}
               \section[ Felix Salten an Arthur Schnitzler, 9. 3. 1906]{ Felix Salten an Arthur Schnitzler, 9. 3. 1906}\nopagebreak\mylabel{v}\rehead{ }\begin{ledgroupsized}[t]{13cm}\normalsize\beginnumbering\briefempfaengerindex{Schnitzler, Arthur@\textsc{Schnitzler, Arthur}!zzzSalten, Felix@\emph{von Felix Salten}!1906-03-091@{9. 3. 1906}|(be} \toendnotes[C]{\smallbreak\pagebreak[2]} \Standort{CUL, Schnitzler, B 89, B 1.}
\physDesc{Brief, 1 Blatt, 3 Seiten, 5504 Zeichen
\newline{}Handschrift: schwarze Tinte, lateinische Kurrent
\newline{}Ordnung: mit Bleistift von unbekannter Hand nummeriert: »206« }\toendnotes[C]{\smallbreak}\pstart
           \noindent{}{\pb}\textcolor{gray}{\textbf{\emph{B. Z. am Mittag}}}\orgindex{B.Z. am Mittag@B.Z. am Mittag|pw}\hfill \textcolor{gray}{\textbf{\emph{BERLIN SW\oindex{Berlin@\textbf{Berlin}|pw},}}}{ }9. III. 06.\pend
           \pstart
           \textcolor{gray}{\textbf{\emph{Chefredaktion}}}\hfill \textcolor{gray}{\textbf{\emph{Kochstr. 23–25\oindex{Kochstrasse@\textbf{Kochstraße}|pw}}}}\pend
           \pstart
           Lieber, hier sende ich Ihnen das \label{K_L03415-1v}\edtext{Feuilleton\pwindex{Salten, Felix 06.09.1869 – 08.10.1945@\textsc{Salten, Felix} (06.09.1869 – 08.10.1945), \emph{Schriftsteller, Journalist, Chefredakteur}!Russisches Theater. I1906-03-06@\strich\emph{Russisches Theater. I} {[}1906-03-06{]}|pwv}}{\lemma{\textnormal{\emph{Feuilleton}}}\Cendnote{\textnormal{Felix Salten\pwindex{Salten, Felix 06.09.1869 – 08.10.1945@\textsc{Salten, Felix} (06.09.1869 – 08.10.1945), \emph{Schriftsteller, Journalist, Chefredakteur}|pwk}: \emph{Russisches Theater. I}\pwindex{Salten, Felix 06.09.1869 – 08.10.1945@\textsc{Salten, Felix} (06.09.1869 – 08.10.1945), \emph{Schriftsteller, Journalist, Chefredakteur}!Russisches Theater. I1906-03-06@\strich\emph{Russisches Theater. I} {[}1906-03-06{]}|pwk}. In: \emph{B. Z. am Mittag}\pwindex{?? Werk@Nicht ermittelte Verfasserinnen und Verfasser!B.Z. am Mittag1904-10-22 – 1943@\emph{B.Z. am Mittag} {[}1904-10-22 – 1943{]}|pwk}, Jg. 30, Nr. 55, 6. 3. 1906, S. 2.}}}\label{K_L03415-1h} – das einzige, das
               bisher kam – aus der »B. Z.\pwindex{?? Werk@Nicht ermittelte Verfasserinnen und Verfasser!B.Z. am Mittag1904-10-22 – 1943@\emph{B.Z. am Mittag} {[}1904-10-22 – 1943{]}|pw}« Montag will ich nochmals \label{K_L03415-2v}\edtext{über die Russ\oindex{Russland@\textbf{Russland}|pwv}en schreiben\pwindex{Salten, Felix 06.09.1869 – 08.10.1945@\textsc{Salten, Felix} (06.09.1869 – 08.10.1945), \emph{Schriftsteller, Journalist, Chefredakteur}!Russisches Theater. I1906-03-06@\strich\emph{Russisches Theater. I} {[}1906-03-06{]}|pwv}}{\lemma{\textnormal{\emph{über … schreiben}}}\Cendnote{\textnormal{Felix Salten\pwindex{Salten, Felix 06.09.1869 – 08.10.1945@\textsc{Salten, Felix} (06.09.1869 – 08.10.1945), \emph{Schriftsteller, Journalist, Chefredakteur}|pwk}: \emph{Russisches Theater. II}\pwindex{Salten, Felix 06.09.1869 – 08.10.1945@\textsc{Salten, Felix} (06.09.1869 – 08.10.1945), \emph{Schriftsteller, Journalist, Chefredakteur}!Russisches Theater. II23. 03. 1906@\strich\emph{Russisches Theater. II} {[}23. 03. 1906{]}|pwk}. In: \emph{B. Z. am Mittag}\pwindex{?? Werk@Nicht ermittelte Verfasserinnen und Verfasser!B.Z. am Mittag1904-10-22 – 1943@\emph{B.Z. am Mittag} {[}1904-10-22 – 1943{]}|pwk}, Jg. 30, Nr. 70, 23. 3. 1906, S. 2–3.}}}\label{K_L03415-2h}, und schicke es
               Ihnen dann gleich zu. Dass Sie so \label{K_L03415-3v}\edtext{verstimmt von hier weggingen}{\lemma{\textnormal{\emph{verstimmt … weggingen}}}\Cendnote{\textnormal{Schnitzler\pwindex{Schnitzler, Arthur 15.05.1862 – 21.10.1931@\textsc{Schnitzler, Arthur} (15.05.1862 – 21.10.1931), \emph{Schriftsteller, Mediziner}|pwk} war anlässlich der Uraufführung
                  von \emph{Der Ruf des Lebens}\pwindex{Schnitzler, Arthur 15.05.1862 – 21.10.1931@\textsc{Schnitzler, Arthur} (15.05.1862 – 21.10.1931), \emph{Schriftsteller, Mediziner}!Ruf des Lebens. Schauspiel in drei Akten1906-02-20@\strich\emph{Der Ruf des Lebens. Schauspiel in drei Akten} {[}1906-02-20{]}|pwk} in Berlin\oindex{Berlin@\textbf{Berlin}|pwk} gewesen und am 27. 2. 1906 heimgekehrt. Zu diesem Zeitpunkt
                  waren bereits einige negative Kritiken erschienen.}}}\label{K_L03415-3h}, hat auch auf mich
               deprimirend gewirkt. Dieser »Ruf des Lebens\pwindex{Schnitzler, Arthur 15.05.1862 – 21.10.1931@\textsc{Schnitzler, Arthur} (15.05.1862 – 21.10.1931), \emph{Schriftsteller, Mediziner}!Ruf des Lebens. Schauspiel in drei Akten1906-02-20@\strich\emph{Der Ruf des Lebens. Schauspiel in drei Akten} {[}1906-02-20{]}|pw}«
               schien mir so unbezweifelbar, und ist es mir noch, dass seine Aufnahme für mich eine
               symptomatische Bedeutung annahm.\pend
           \pstart
           Es ist ein Glück, dass Sie stark genug sind, um sich kommende Produktion durch
               solche, an sich keineswegs wichtige Zwischenfälle, stören zu laßen. Darauf rechne ich
               sehr, und hoffe, bald von Ihnen zu hören, dass Sie arbeiten. Schlimm wäre es ja nur,
               wenn Sie, – mehr aus künstlerischer Hypochondrie als aus Selbstkritik – anfangen
               würden, in Ihrer Abschätzung dieses Stück\pwindex{Schnitzler, Arthur 15.05.1862 – 21.10.1931@\textsc{Schnitzler, Arthur} (15.05.1862 – 21.10.1931), \emph{Schriftsteller, Mediziner}!Ruf des Lebens. Schauspiel in drei Akten1906-02-20@\strich\emph{Der Ruf des Lebens. Schauspiel in drei Akten} {[}1906-02-20{]}|pwv}es wankend zu werden. Da kann man freilich für eine Weile den Boden
               unter sich schwinden fühlen. Aber es wäre, besonders in diesem Falle, das Falscheste!
               Sie müssen unbedingt dabei bleiben, dass Ihr Stück\pwindex{Schnitzler, Arthur 15.05.1862 – 21.10.1931@\textsc{Schnitzler, Arthur} (15.05.1862 – 21.10.1931), \emph{Schriftsteller, Mediziner}!Ruf des Lebens. Schauspiel in drei Akten1906-02-20@\strich\emph{Der Ruf des Lebens. Schauspiel in drei Akten} {[}1906-02-20{]}|pwv} im Recht ist, und dass die Zufälligkeit eines Abends nichts
               beweist. Dass \label{K_L03415-4v}\edtext{Harden\pwindex{Harden, Maximilian 20.10.1861 – 30.10.1927@\textsc{Harden, Maximilian} (20.10.1861 – 30.10.1927), \emph{Schriftsteller, Publizist}|pw}{ }\uline{so}{ }geschrieben\pwindex{Theater03. 03. 1906@\emph{Theater} {[}03. 03. 1906{]}|pwv}}{\lemma{\textnormal{\emph{Harden so geschrieben}}}\Cendnote{\textnormal{Bezug auf eine gemeinsame Besprechung
                  der Aufführungen von Hofmannsthals\pwindex{Hofmannsthal, Hugo von 1874-02-01 – 1929-07-15@\textsc{Hofmannsthal, Hugo von} (1874-02-01 – 1929-07-15), \emph{Schriftsteller}|pwk}{ }\emph{Oedipus und die Sphinx}\pwindex{Hofmannsthal, Hugo von 1874-02-01 – 1929-07-15@\textsc{Hofmannsthal, Hugo von} (1874-02-01 – 1929-07-15), \emph{Schriftsteller}!Oedipus und die Sphinx. Tragoedie in drei Aufzuegen1906@\strich\emph{Oedipus und die Sphinx. Tragödie in drei Aufzügen} {[}1906{]}|pwk} und Schnitzlers\pwindex{Schnitzler, Arthur 15.05.1862 – 21.10.1931@\textsc{Schnitzler, Arthur} (15.05.1862 – 21.10.1931), \emph{Schriftsteller, Mediziner}|pwk}{ }\emph{Der Ruf des Lebens}\pwindex{Schnitzler, Arthur 15.05.1862 – 21.10.1931@\textsc{Schnitzler, Arthur} (15.05.1862 – 21.10.1931), \emph{Schriftsteller, Mediziner}!Ruf des Lebens. Schauspiel in drei Akten1906-02-20@\strich\emph{Der Ruf des Lebens. Schauspiel in drei Akten} {[}1906-02-20{]}|pwk}: M. H.\pwindex{Harden, Maximilian 20.10.1861 – 30.10.1927@\textsc{Harden, Maximilian} (20.10.1861 – 30.10.1927), \emph{Schriftsteller, Publizist}|pwk} [ = Maximilian Harden\pwindex{Harden, Maximilian 20.10.1861 – 30.10.1927@\textsc{Harden, Maximilian} (20.10.1861 – 30.10.1927), \emph{Schriftsteller, Publizist}|pwk}]: \emph{Theater}\pwindex{Theater03. 03. 1906@\emph{Theater} {[}03. 03. 1906{]}|pwk}. In: \emph{Die Zukunft}\pwindex{Zukunft1892 – 1922@\emph{Die Zukunft} {[}1892 – 1922{]}|pwk}, Bd. 54,
                     H. 9, 3. 3. 1906, S. 346–356.}}}\label{K_L03415-4h} hat,
               ist im ersten Moment für Ihr Empfinden vielleicht sehr verletzend gewesen; tut aber
               wirklich nichts. Hätte er die Sache\pwindex{Schnitzler, Arthur 15.05.1862 – 21.10.1931@\textsc{Schnitzler, Arthur} (15.05.1862 – 21.10.1931), \emph{Schriftsteller, Mediziner}!Ruf des Lebens. Schauspiel in drei Akten1906-02-20@\strich\emph{Der Ruf des Lebens. Schauspiel in drei Akten} {[}1906-02-20{]}|pwv} ausführlich und mit der ganzen Kraft seiner Dialektik zerrupft und
               zergliedert, dann wäre es schlimmer gewesen, denn es hätte \uline{gewirkt}. So aber hat hier\oindex{Berlin@\textbf{Berlin}|pwv}, – und wol überall – jeder nur die Achsel gezuckt und gesagt: Das
               glaubt Harden\pwindex{Harden, Maximilian 20.10.1861 – 30.10.1927@\textsc{Harden, Maximilian} (20.10.1861 – 30.10.1927), \emph{Schriftsteller, Publizist}|pw} selber nicht. Die Politik war
               gar zu sichtbar, als dass ein kritischer Einfluß erfolgen könnte.\pend
           \pstart
           Nach und nach kommt meine \label{K_L03415-5v}\edtext{Wohnung\oindex{Kantstrasse@\textbf{Kantstraße}|pwv}}{\lemma{\textnormal{\emph{Wohnung}}}\Cendnote{\textnormal{Siehe Felix Salten an Arthur Schnitzler, 29. 1. 1906. }}}\label{K_L03415-5h} in Ordnung,
               und ich kann eine menschliche Existenz beginnen. Könnte ich jetzt wieder von hier\oindex{Berlin@\textbf{Berlin}|pwv} auswandern, dann wäre ich
               schon imstande, ein nettes Buch über Berlin\oindex{Berlin@\textbf{Berlin}|pw} zu
               schreiben. Aber, ich hoffe, dass ich hier nicht sterben muß, und doch einmal werde
               reden können. Nach Wien\oindex{Wien@\textbf{Wien}|pw} sehne ich mich aber auch
               nicht. Dazu liegt mir die Schweinerei der letzten Affären noch zu sehr im Magen.
               Haben Sie die letzte \label{K_L03415-6v}\edtext{Schurkerei des
               dramatischen Dichters Ludassy\pwindex{Gans-Ludassy, Julius von 13.04.1858 – 30.09.1922@\textsc{Gans-Ludassy, Julius von} (13.04.1858 – 30.09.1922), \emph{Schriftsteller, Journalist, Herausgeber}|pw}}{\lemma{\textnormal{\emph{Schurkerei … Ludassy}}}\Cendnote{\textnormal{Die Auseinandersetzung Ludassy\pwindex{Gans-Ludassy, Julius von 13.04.1858 – 30.09.1922@\textsc{Gans-Ludassy, Julius von} (13.04.1858 – 30.09.1922), \emph{Schriftsteller, Journalist, Herausgeber}|pwk}–Salten\pwindex{Salten, Felix 06.09.1869 – 08.10.1945@\textsc{Salten, Felix} (06.09.1869 – 08.10.1945), \emph{Schriftsteller, Journalist, Chefredakteur}|pwk} ist
                  komplex, da sie sowohl durch öffentliche Gerichtsverhandlungen als auch durch
                  Prozesse innerhalb der Journalismusorganisation \emph{Concordia}\orgindex{Concordia@Concordia|pwk} geführt wurde und jeweils unterschiedliche Aspekte verhandelt
                  wurden. In seinen \emph{Erinnerungen}\pwindex{Salten, Felix 06.09.1869 – 08.10.1945@\textsc{Salten, Felix} (06.09.1869 – 08.10.1945), \emph{Schriftsteller, Journalist, Chefredakteur}!Erinnerungen@\strich\emph{Erinnerungen}|pwk} schildert Salten\pwindex{Salten, Felix 06.09.1869 – 08.10.1945@\textsc{Salten, Felix} (06.09.1869 – 08.10.1945), \emph{Schriftsteller, Journalist, Chefredakteur}|pwk} die Sache wie folgt: »Jetzt
                     muss ich doch noch die Affaire Ludassy\pwindex{Gans-Ludassy, Julius von 13.04.1858 – 30.09.1922@\textsc{Gans-Ludassy, Julius von} (13.04.1858 – 30.09.1922), \emph{Schriftsteller, Journalist, Herausgeber}|pw}
                     erzählen, unter der ich kindischerweise länger als ein Jahr schmerzhaft
                     gelitten habe. Ludassy\pwindex{Gans-Ludassy, Julius von 13.04.1858 – 30.09.1922@\textsc{Gans-Ludassy, Julius von} (13.04.1858 – 30.09.1922), \emph{Schriftsteller, Journalist, Herausgeber}|pw} war mein erster
                     Chefredakteur und er war gewalttätig, wie man sich aus dem Aufbrechen des
                     Schreibtisches von J. J. David\pwindex{David, Jakob Julius 1859-02-06 – 1906-11-20@\textsc{David, Jakob Julius} (1859-02-06 – 1906-11-20), \emph{Schriftsteller, Journalist}|pw} erinnern
                     wird. Jetzt war er nicht mehr Chefredakteur und ich bei der ›Zeit\orgindex{Zeit@Die Zeit|pw}‹. Nun veranstaltete die Concordia\orgindex{Concordia@Concordia|pw} eine Protestversammlung gegen den
                     Chefredakteur der Zeit\orgindex{Zeit@Die Zeit|pw}, Dr. Kanner\pwindex{Kanner, Heinrich 09.11.1864 – 15.02.1930@\textsc{Kanner, Heinrich} (09.11.1864 – 15.02.1930), \emph{Herausgeber, Publizist}|pw}, mit der Anklage, er brülle die
                     Redakteure an. Ich erschien zu seiner Verteidigung und sagte unter anderem, das
                     Schreien bedeutet gar nichts. {[}›{]}Hier neben mir sitzt mein
                     Freund Ludassy\pwindex{Gans-Ludassy, Julius von 13.04.1858 – 30.09.1922@\textsc{Gans-Ludassy, Julius von} (13.04.1858 – 30.09.1922), \emph{Schriftsteller, Journalist, Herausgeber}|pw}, der mein erster
                     Chefredakteur gewesen ist und der auch geschrien hat. Deswegen sind wir doch
                     befreundet!‹ Ludassy\pwindex{Gans-Ludassy, Julius von 13.04.1858 – 30.09.1922@\textsc{Gans-Ludassy, Julius von} (13.04.1858 – 30.09.1922), \emph{Schriftsteller, Journalist, Herausgeber}|pw} rückte von mir ab,
                     murrte: ›Ihr Freund bin ich gewesen und diese Worte werden Sie
                        bereu{[}e{]}n!‹ Ich musste diese Worte, so harmlos sie auch
                     gemeint waren{[},{]} länger als ein Jahr bitterlich bereuen.
                     Denn Herr Dr. Ludassy\pwindex{Gans-Ludassy, Julius von 13.04.1858 – 30.09.1922@\textsc{Gans-Ludassy, Julius von} (13.04.1858 – 30.09.1922), \emph{Schriftsteller, Journalist, Herausgeber}|pw}, der mit einem
                     Theaterstück: der letzte Knopf\pwindex{Gans-Ludassy, Julius von 13.04.1858 – 30.09.1922@\textsc{Gans-Ludassy, Julius von} (13.04.1858 – 30.09.1922), \emph{Schriftsteller, Journalist, Herausgeber}!letzte Knopf. Volksstueck in drei Aufzuegen1900-04-08@\strich\emph{Der letzte Knopf. Volksstück in drei Aufzügen} {[}1900-04-08{]}|pw} im Deutschen Volkstheater\oindex{Volkstheater@\textbf{Volkstheater}|pw} zur Aufführung
                     gelangte, schrieb ungefähr ein Jahr nach der Aufführung, es sei vor seiner
                     Premiere, der Mann mit dem Revolver vor ihm gestanden.« (\emph{Wienbibliothek im Rathaus}, Nachlass Salten\pwindex{Salten, Felix 06.09.1869 – 08.10.1945@\textsc{Salten, Felix} (06.09.1869 – 08.10.1945), \emph{Schriftsteller, Journalist, Chefredakteur}|pwk}, ZPH 1681/1 1.1.1.9.1, [S. 61].) In seinem
                  Brief vom [18.? 10. 1906]
                  schildert Salten\pwindex{Salten, Felix 06.09.1869 – 08.10.1945@\textsc{Salten, Felix} (06.09.1869 – 08.10.1945), \emph{Schriftsteller, Journalist, Chefredakteur}|pwk} den ersten Teil etwas
                  genauer. Die Stelle, die Salten\pwindex{Salten, Felix 06.09.1869 – 08.10.1945@\textsc{Salten, Felix} (06.09.1869 – 08.10.1945), \emph{Schriftsteller, Journalist, Chefredakteur}|pwk} meint, wird
                  von einer Wien\oindex{Wien@\textbf{Wien}|pwk}er Zeitung so zitiert: »Herr Dr. von Ludaſſy\pwindex{Gans-Ludassy, Julius von 13.04.1858 – 30.09.1922@\textsc{Gans-Ludassy, Julius von} (13.04.1858 – 30.09.1922), \emph{Schriftsteller, Journalist, Herausgeber}|pw} ließ ſich über
                        einen Wien\oindex{Wien@\textbf{Wien}|pw}er Kritiker in einer Berliner\oindex{Berlin@\textbf{Berlin}|pw} Wochenſchrift wie folgt aus:{ / }›Je weniger die handfeſten Burſchen verſtehen, deſto hochmütiger,
                        abſprechender und unflätiger ſchreiben ſie. Es gibt deren auch, die vor der
                        Aufführung den Autor um höhere Beträge anzupumpen verſuchen (Ich habe ſelbſt
                        eine ſolche Kreatur als Zeitungsherausgeber zum Kritiker gemacht; als ich
                        auf der Bühne als Autor mein Heil verſuchte, ſtand dann dasſelbe Individuum
                        mit dem Revolver vor mir.) Antikritik hilft gegen ſolche Schelme nicht. Denn
                        niemand, dem ein unehrlicher oder übermütiger Kritiker Leides zugefügt hat,
                        will die Beſtie reizen.‹« (y.: \emph{Kritik der Kritik.
                        (Erbauliches aus der »Concordia«)}\pwindex{?? Werk@Nicht ermittelte Verfasserinnen und Verfasser!Kritik der Kritik. (Erbauliches aus der »Concordia«.)1905-12-18@\emph{Kritik der Kritik. (Erbauliches aus der »Concordia«.)} {[}1905-12-18{]}|pwk}. In: \emph{Wiener Montags-Journal}\pwindex{?? Werk@Nicht ermittelte Verfasserinnen und Verfasser!Extrapost. Unparteiische Montags-Zeitung1882 – 1905@\emph{Extrapost. Unparteiische Montags-Zeitung} {[}1882 – 1905{]}|pwk}, Jg. 24, Nr. 1245, 18. 12. 1905,
                     S. 3–4.) Saltens\pwindex{Salten, Felix 06.09.1869 – 08.10.1945@\textsc{Salten, Felix} (06.09.1869 – 08.10.1945), \emph{Schriftsteller, Journalist, Chefredakteur}|pwk} Narration ist in
                  der Chronologie unzuverlässig. Die zeitliche Einordnung der Ereignisse lässt sich
                  mit der Uraufführung von \emph{Der letzte Knopf}\pwindex{Gans-Ludassy, Julius von 13.04.1858 – 30.09.1922@\textsc{Gans-Ludassy, Julius von} (13.04.1858 – 30.09.1922), \emph{Schriftsteller, Journalist, Herausgeber}!letzte Knopf. Volksstueck in drei Aufzuegen1900-04-08@\strich\emph{Der letzte Knopf. Volksstück in drei Aufzügen} {[}1900-04-08{]}|pwk} am
                     8. 4. 1900 nur
                  scheinbar vornehmen. Saltens\pwindex{Salten, Felix 06.09.1869 – 08.10.1945@\textsc{Salten, Felix} (06.09.1869 – 08.10.1945), \emph{Schriftsteller, Journalist, Chefredakteur}|pwk}{ }Feuilleton\pwindex{Deutsches Volkstheater. (»Der letzte Knopf.« Volksstueck in drei Aufzuegen von
                  J. v. Gans-Ludassy)1900-04-10@\emph{Deutsches Volkstheater. (»Der letzte Knopf.« Volksstück in drei Aufzügen von J. v. Gans-Ludassy)} {[}1900-04-10{]}|pwkv} erschien zwei
                  Tage später: f. s.\pwindex{Salten, Felix 06.09.1869 – 08.10.1945@\textsc{Salten, Felix} (06.09.1869 – 08.10.1945), \emph{Schriftsteller, Journalist, Chefredakteur}|pwk}: \emph{Deutsches Volkstheater. (»Der letzte Knopf.« Volksstück in drei Aufzügen
                        von J. v. Gans-Ludassy)}\pwindex{Deutsches Volkstheater. (»Der letzte Knopf.« Volksstueck in drei Aufzuegen von
                  J. v. Gans-Ludassy)1900-04-10@\emph{Deutsches Volkstheater. (»Der letzte Knopf.« Volksstück in drei Aufzügen von J. v. Gans-Ludassy)} {[}1900-04-10{]}|pwk}. In: \emph{Wiener
                        Allgemeinen Zeitung. 6 Uhr-Blatt}\pwindex{Wiener Allgemeine Zeitung1.3.1880 – 11.2.1934@\emph{Wiener Allgemeine Zeitung} {[}1.3.1880 – 11.2.1934{]}|pwk}, Nr. 6628, 10. 4. 1900, S. 2–3. Damals gab es aber die Tageszeitung \emph{Die Zeit}\orgindex{Zeit@Die Zeit|pwk} noch nicht, sodass er das Stück
                  verwechselt haben dürfte. Zeitlich passender ist die Uraufführung von Ludassys\pwindex{Gans-Ludassy, Julius von 13.04.1858 – 30.09.1922@\textsc{Gans-Ludassy, Julius von} (13.04.1858 – 30.09.1922), \emph{Schriftsteller, Journalist, Herausgeber}|pwk}{ }\emph{Der goldene Boden. Volksstück in vier
                     Aufzügen}\pwindex{Gans-Ludassy, Julius von 13.04.1858 – 30.09.1922@\textsc{Gans-Ludassy, Julius von} (13.04.1858 – 30.09.1922), \emph{Schriftsteller, Journalist, Herausgeber}!goldene Boden. Volksstueck in vier Aufzuegen1904-03-26@\strich\emph{Der goldene Boden. Volksstück in vier Aufzügen} {[}1904-03-26{]}|pwk}, die am 26. 3. 1904 (in Anwesenheit Schnitzlers\pwindex{Schnitzler, Arthur 15.05.1862 – 21.10.1931@\textsc{Schnitzler, Arthur} (15.05.1862 – 21.10.1931), \emph{Schriftsteller, Mediziner}|pwk}) am \emph{Deutschen
                     Volkstheater}XXXX ORGangabe fehlt stattfand. Eine Besprechung Saltens\pwindex{Salten, Felix 06.09.1869 – 08.10.1945@\textsc{Salten, Felix} (06.09.1869 – 08.10.1945), \emph{Schriftsteller, Journalist, Chefredakteur}|pwk} lässt sich nicht nachweisen, doch dürfte das daran liegen, dass
                  unmittelbar davor ein längeres Feuilleton von Salten\pwindex{Salten, Felix 06.09.1869 – 08.10.1945@\textsc{Salten, Felix} (06.09.1869 – 08.10.1945), \emph{Schriftsteller, Journalist, Chefredakteur}|pwk} abgedruckt worden war und der Eindruck vermieden werden sollte,
                  dass das Blatt\orgindex{Wiener Allgemeine Zeitung@Wiener Allgemeine Zeitung|pwkv} zu wenige
                  Beiträgerinnen und Beiträger habe. Dementsprechend wäre das Kürzel »mm« Salten\pwindex{Salten, Felix 06.09.1869 – 08.10.1945@\textsc{Salten, Felix} (06.09.1869 – 08.10.1945), \emph{Schriftsteller, Journalist, Chefredakteur}|pwk} zuzuordnen: mm\pwindex{Salten, Felix 06.09.1869 – 08.10.1945@\textsc{Salten, Felix} (06.09.1869 – 08.10.1945), \emph{Schriftsteller, Journalist, Chefredakteur}|pwk} [ = Felix Salten\pwindex{Salten, Felix 06.09.1869 – 08.10.1945@\textsc{Salten, Felix} (06.09.1869 – 08.10.1945), \emph{Schriftsteller, Journalist, Chefredakteur}|pwk}?]: \emph{Deutsches
                        Volkstheater. (»Der goldene Boden«, Volksstück in vier Aufzügen von Julius
                        v. Gans-Ludassy. 26. März)}\pwindex{Deutsches Volkstheater. (»Der goldene Boden«, Volksstueck in vier Aufzuegen von
                  Julius v. Gans-Ludassy. 26. Maerz)1904-03-27@\emph{Deutsches Volkstheater. (»Der goldene Boden«, Volksstück in vier Aufzügen von Julius v. Gans-Ludassy. 26. März)} {[}1904-03-27{]}|pwk}. In: \emph{Die
                        Zeit}\pwindex{Zeit1902-09-27 – 1919@\emph{Die Zeit} {[}1902-09-27 – 1919{]}|pwk}, Jg. 3, Nr. 538, 27. 3. 1904,
                     S. 3. Was genau mit dem »Revolver« in Saltens\pwindex{Salten, Felix 06.09.1869 – 08.10.1945@\textsc{Salten, Felix} (06.09.1869 – 08.10.1945), \emph{Schriftsteller, Journalist, Chefredakteur}|pwk}{ }\emph{Erinnerungen}\pwindex{Salten, Felix 06.09.1869 – 08.10.1945@\textsc{Salten, Felix} (06.09.1869 – 08.10.1945), \emph{Schriftsteller, Journalist, Chefredakteur}!Erinnerungen@\strich\emph{Erinnerungen}|pwk} gemeint war, klärte er an einer
                  anderen Stelle: »Mein ehemaliger Chef Ludassy\pwindex{Gans-Ludassy, Julius von 13.04.1858 – 30.09.1922@\textsc{Gans-Ludassy, Julius von} (13.04.1858 – 30.09.1922), \emph{Schriftsteller, Journalist, Herausgeber}|pw} verleumdete mich, ich hätte vor seiner Premiere von
                     ihm 3000 Kronen erpressen wollen. Es war mir leicht ihn zu widerlegen. Der
                     damalige Erzherzog Leopold Ferdinand\pwindex{Woelfling, Leopold Ferdinand Salvator 1868-12-02 – 1935-07-04@\textsc{Wölfling, Leopold Ferdinand Salvator} (1868-12-02 – 1935-07-04), \emph{Erzherzog}|pw}
                     suchte einen Kredit in dieser Höhe und ich fragte Ludassy\pwindex{Gans-Ludassy, Julius von 13.04.1858 – 30.09.1922@\textsc{Gans-Ludassy, Julius von} (13.04.1858 – 30.09.1922), \emph{Schriftsteller, Journalist, Herausgeber}|pw} um Rat.« (ZPH 1681/1 1.1.1.9.1,
                     [S. 4]) Die Behauptung\pwindex{Gans-Ludassy, Julius von 13.04.1858 – 30.09.1922@\textsc{Gans-Ludassy, Julius von} (13.04.1858 – 30.09.1922), \emph{Schriftsteller, Journalist, Herausgeber}!?? [Ludassy will von Salten erpresst worden sein]Ende 1905@\strich\emph{?? [Ludassy will von Salten erpresst worden sein]} {[}Ende 1905{]}|pwkv}{ }Ludassys\pwindex{Gans-Ludassy, Julius von 13.04.1858 – 30.09.1922@\textsc{Gans-Ludassy, Julius von} (13.04.1858 – 30.09.1922), \emph{Schriftsteller, Journalist, Herausgeber}|pwk}, es wäre vor der Premiere versucht
                  worden, ihn zu erpressen, entwickelte sich in der Darstellung Saltens\pwindex{Salten, Felix 06.09.1869 – 08.10.1945@\textsc{Salten, Felix} (06.09.1869 – 08.10.1945), \emph{Schriftsteller, Journalist, Chefredakteur}|pwk} auf folgende Weise weiter:
                     »Verabredetermaßen fragte Stephan
                        Grossmann\pwindex{Grossmann, Stefan 19.05.1875 – 03.01.1935@\textsc{Großmann, Stefan} (19.05.1875 – 03.01.1935), \emph{Schriftsteller, Journalist}|pw} in der Arbeiterzeitung\orgindex{Arbeiter-Zeitung@Arbeiter-Zeitung|pw}
                     nach dem Namen des Revolvermanns. Dr Ludassy\pwindex{Gans-Ludassy, Julius von 13.04.1858 – 30.09.1922@\textsc{Gans-Ludassy, Julius von} (13.04.1858 – 30.09.1922), \emph{Schriftsteller, Journalist, Herausgeber}|pw} nannte mich. Worauf mich Stephan Grossmann\pwindex{Grossmann, Stefan 19.05.1875 – 03.01.1935@\textsc{Großmann, Stefan} (19.05.1875 – 03.01.1935), \emph{Schriftsteller, Journalist}|pw} mit einem Kübel Unrat überschüttete. Ich kam mir in
                     meiner persönlichen und wegen meiner publizistischen Ehre schwer verletzt vor
                     und rief ein Ehrengericht gegen mich an. Bei dieser Ehrengerichtlichen
                     Verhandlung legte ich folgende Beweise vor:  1. Ich hatte Ludassy\pwindex{Gans-Ludassy, Julius von 13.04.1858 – 30.09.1922@\textsc{Gans-Ludassy, Julius von} (13.04.1858 – 30.09.1922), \emph{Schriftsteller, Journalist, Herausgeber}|pw} nur im Namen des Erzherzog Leopolds\pwindex{Woelfling, Leopold Ferdinand Salvator 1868-12-02 – 1935-07-04@\textsc{Wölfling, Leopold Ferdinand Salvator} (1868-12-02 – 1935-07-04), \emph{Erzherzog}|pw} gefragt, wo man einen Kredit von 8000 Kronen
                     für den Erzherzog aufnehmen könnte{[}.{]} (Dieser Kredit wurde
                     ihm wenig später vom Beamtenverein\orgindex{Wiener Beamtenverein@Wiener Beamtenverein|pw}
                        erteilt{[}.{]}) 2. Ich legte mein Feuilleton\pwindex{Deutsches Volkstheater. (»Der goldene Boden«, Volksstueck in vier Aufzuegen von
                  Julius v. Gans-Ludassy. 26. Maerz)1904-03-27@\emph{Deutsches Volkstheater. (»Der goldene Boden«, Volksstück in vier Aufzügen von Julius v. Gans-Ludassy. 26. März)} {[}1904-03-27{]}|pwv} über Ludassys\pwindex{Gans-Ludassy, Julius von 13.04.1858 – 30.09.1922@\textsc{Gans-Ludassy, Julius von} (13.04.1858 – 30.09.1922), \emph{Schriftsteller, Journalist, Herausgeber}|pw}{ }Stück\pwindex{Gans-Ludassy, Julius von 13.04.1858 – 30.09.1922@\textsc{Gans-Ludassy, Julius von} (13.04.1858 – 30.09.1922), \emph{Schriftsteller, Journalist, Herausgeber}!goldene Boden. Volksstueck in vier Aufzuegen1904-03-26@\strich\emph{Der goldene Boden. Volksstück in vier Aufzügen} {[}1904-03-26{]}|pwv} vor, das eine
                     Lobeshymne darstellte. 3. Ich legte eine Reihe von Briefen und Eilpostkarten
                        Ludassys\pwindex{Gans-Ludassy, Julius von 13.04.1858 – 30.09.1922@\textsc{Gans-Ludassy, Julius von} (13.04.1858 – 30.09.1922), \emph{Schriftsteller, Journalist, Herausgeber}|pw} vor, in denen er teils für
                     meine Kritik\pwindex{Deutsches Volkstheater. (»Der goldene Boden«, Volksstueck in vier Aufzuegen von
                  Julius v. Gans-Ludassy. 26. Maerz)1904-03-27@\emph{Deutsches Volkstheater. (»Der goldene Boden«, Volksstück in vier Aufzügen von Julius v. Gans-Ludassy. 26. März)} {[}1904-03-27{]}|pwv} heissen
                     Dank aussprach, teils noch lange nach der Premiere und nach meiner Kritik\pwindex{Deutsches Volkstheater. (»Der goldene Boden«, Volksstueck in vier Aufzuegen von
                  Julius v. Gans-Ludassy. 26. Maerz)1904-03-27@\emph{Deutsches Volkstheater. (»Der goldene Boden«, Volksstück in vier Aufzügen von Julius v. Gans-Ludassy. 26. März)} {[}1904-03-27{]}|pwv} mir Briefe und
                     Eilkarten schrieb{[},{]} in denen er verlangte mich zu sehen, in
                     denen er meine Freundschaft pries, und die seinige beteuerte. Der Präsident
                     dieses Ehrenrates, Chefredakteur des Neuen
                        Wiener Tagblatt\orgindex{Neues Wiener Tagblatt@Neues Wiener Tagblatt|pw}es{[},{]}{ }Wilhelm Singer\pwindex{Singer, Wilhelm 26.11.1847 – 10.10.1917@\textsc{Singer, Wilhelm} (26.11.1847 – 10.10.1917), \emph{Journalist, Chefredakteur}|pw}, richtete mitten in der
                     Verhandlung an Ludassy\pwindex{Gans-Ludassy, Julius von 13.04.1858 – 30.09.1922@\textsc{Gans-Ludassy, Julius von} (13.04.1858 – 30.09.1922), \emph{Schriftsteller, Journalist, Herausgeber}|pw} die Frage: ›Sagen
                     Sie Herr Dr. schämen Sie sich denn gar nicht?!‹ Ludassy\pwindex{Gans-Ludassy, Julius von 13.04.1858 – 30.09.1922@\textsc{Gans-Ludassy, Julius von} (13.04.1858 – 30.09.1922), \emph{Schriftsteller, Journalist, Herausgeber}|pw} wurde zu einer schweren Rüge von der Concordia\orgindex{Concordia@Concordia|pw} verurteilt und zur Unfähigkeit
                     zwei Jahre lang ein Ehrenamt in der Concordia\orgindex{Concordia@Concordia|pw} zu bekleiden.« (ebd., [S. 61–62]). Das
                  Ehrengericht des Journalistenverbands\orgindex{Concordia@Concordia|pwkv} entschied am 12. 5. 1907 zugunsten Saltens\pwindex{Salten, Felix 06.09.1869 – 08.10.1945@\textsc{Salten, Felix} (06.09.1869 – 08.10.1945), \emph{Schriftsteller, Journalist, Chefredakteur}|pwk}. Die Rüge für Ludassy\pwindex{Gans-Ludassy, Julius von 13.04.1858 – 30.09.1922@\textsc{Gans-Ludassy, Julius von} (13.04.1858 – 30.09.1922), \emph{Schriftsteller, Journalist, Herausgeber}|pwk} lässt sich belegen, doch wurde er nur für ein Jahr von jeglichen
                  Ehrenämtern der \emph{Concordia}\orgindex{Concordia@Concordia|pwk} ausgeschlossen
                     (vgl. \emph{Wienbibliothek im Rathaus}, Nachlass Salten\pwindex{Salten, Felix 06.09.1869 – 08.10.1945@\textsc{Salten, Felix} (06.09.1869 – 08.10.1945), \emph{Schriftsteller, Journalist, Chefredakteur}|pwk}, ZPH 1681, 3.7.4). Salten\pwindex{Salten, Felix 06.09.1869 – 08.10.1945@\textsc{Salten, Felix} (06.09.1869 – 08.10.1945), \emph{Schriftsteller, Journalist, Chefredakteur}|pwk} schrieb in seinen \emph{Erinnerungen}\pwindex{Salten, Felix 06.09.1869 – 08.10.1945@\textsc{Salten, Felix} (06.09.1869 – 08.10.1945), \emph{Schriftsteller, Journalist, Chefredakteur}!Erinnerungen@\strich\emph{Erinnerungen}|pwk} weiter: »Damit beruhigte ich mich
                     aber nicht, rief ein zweites Ehrengericht an, das aus Prof. Dr. Joseph Redlich\pwindex{Redlich, Josef 18.06.1869 – 11.11.1936@\textsc{Redlich, Josef} (18.06.1869 – 11.11.1936), \emph{Politiker, Rechtswissenschaftler}|pw}, aus dem früheren Direktor
                     des Burgtheaters\orgindex{Burgtheater@Burgtheater|pw} Dr. Max Burkhard\pwindex{Burckhard, Max Eugen 14.07.1854 – 16.03.1912@\textsc{Burckhard, Max Eugen} (14.07.1854 – 16.03.1912), \emph{Schriftsteller, Rechtswissenschaftler, Theaterleiter}|pw} und aus zwei anderen hohen Richtern
                     bestand, die zwar keine Strafverfügung treffen konnten, deren Urteil aber mir
                     volle Genugtuung bot. Es hatten sich einige meiner Feinde zwar gemeldet, die
                     ich nur zum Teil persönlich kannte, und deren Zeugnis glatt abgewiesen
                     wurde.« (ZPH 1681/1 1.1.1.9.1, [S. 62]). Siehe auch A. S.: \emph{Tagebuch}, 30. 12. 1905, 14. 1. 1906 und 12. 5. 1907.}}}\label{K_L03415-6h}
               Jemandem erzählt? Wenn nicht, dann tun Sie’s doch, bitte. Es ist das Empörendste,
               dass so ein niederes {\pb}durch und
               durch verseuchtes Luder einen monatelang zwischen seinen Fingern halten darf; Na, Sie
               haben mich einmal einen »guten Hasser« genannt, – nicht ganz mit Recht, denn ich habe
               mich bisher noch nie an Jemandem gerächt. Aber diesmal will ich mir den Titel
               verdienen. So oder so. Und wenn nur der Prozess endlich anberaumt wird – ich hab
               mir’s genau überlegt – ich tue nichts, um ihn hinauszuschieben, dann will ich dafür
               sorgen, dass diesmal der Angeklagte wirklich Angeklagter sein soll.\pend
           \pstart
           Übrigens, laßen wir das. Es gibt, gottseidank, bessere Menschen. Z. B. Beer-Hofmann\pwindex{Beer-Hofmann, Richard 1866-07-11 – 1945-09-26@\textsc{Beer-Hofmann, Richard} (1866-07-11 – 1945-09-26), \emph{Schriftsteller}|pw}, nicht wahr? Wie finden Sie es,
               dass er mir bis heute noch keine Zeile schrieb, keine
               Karte, nichts! Dabei bin ich doch nicht einfach nur verreist, bin in einer
               Lebensepoche, in der es nicht ganz unwichtig ist, die Festigkeit gewisser Beziehungen
               zu spüren, bin in einer Situation, in der es \uline{vielleicht} sogar tröstlich, \uline{jedenfalls} aber
               animirend sein kann, von Freunden was zu hören. Dabei hab \uline{ich}, mitten im Übersiedlungsrummel, im Fieber der neuen Stellung\orgindex{B.Z. am Mittag@B.Z. am Mittag|pwv}, in der Unrast des Hotel\oindex{Hotel Saxonia@\textbf{Hotel Saxonia}|pwv}wohnens an B-H.\pwindex{Beer-Hofmann, Richard 1866-07-11 – 1945-09-26@\textsc{Beer-Hofmann, Richard} (1866-07-11 – 1945-09-26), \emph{Schriftsteller}|pw} geschrieben, als ich sein \label{K_L03415-7v}\edtext{Mozart\pwindex{Mozart, Wolfgang Amadeus 27.01.1756 – 05.12.1791@\textsc{Mozart, Wolfgang Amadeus} (27.01.1756 – 05.12.1791), \emph{Komponist}|pw} Feuilleton\pwindex{Beer-Hofmann, Richard 1866-07-11 – 1945-09-26@\textsc{Beer-Hofmann, Richard} (1866-07-11 – 1945-09-26), \emph{Schriftsteller}!Gedenkrede auf Wolfgang Amade Mozart1906-01-28@\strich\emph{Gedenkrede auf Wolfgang Amade Mozart} {[}1906-01-28{]}|pw}}{\lemma{\textnormal{\emph{Mozart Feuilleton}}}\Cendnote{\textnormal{Richard Beer-Hofmann\pwindex{Beer-Hofmann, Richard 1866-07-11 – 1945-09-26@\textsc{Beer-Hofmann, Richard} (1866-07-11 – 1945-09-26), \emph{Schriftsteller}|pwk}: \emph{Gedenkrede auf Wolfgang Amadé Mozart}\pwindex{Beer-Hofmann, Richard 1866-07-11 – 1945-09-26@\textsc{Beer-Hofmann, Richard} (1866-07-11 – 1945-09-26), \emph{Schriftsteller}!Gedenkrede auf Wolfgang Amade Mozart1906-01-28@\strich\emph{Gedenkrede auf Wolfgang Amade Mozart} {[}1906-01-28{]}|pwk}. In: \emph{Frankfurter Zeitung}\pwindex{?? Werk@Nicht ermittelte Verfasserinnen und Verfasser!Frankfurter Zeitung1856 – 1943@\emph{Frankfurter Zeitung} {[}1856 – 1943{]}|pwk}, Jg. 50, Nr. 27, 28. 1. 1906, Erstes Morgenblatt, S. 1–2. Mozart\pwindex{Mozart, Wolfgang Amadeus 27.01.1756 – 05.12.1791@\textsc{Mozart, Wolfgang Amadeus} (27.01.1756 – 05.12.1791), \emph{Komponist}|pwk} hätte am 27. 1. 1906 seinen 150. Geburtstag gefeiert.}}}\label{K_L03415-7h} las (auch dazu
               hatte ich Zeit gefunden){[},{]} dabei hatte ich noch ein \uline{zweitesmal} an ihn eine Karte
                  geschickt\textcolor{gray}{.} Dabei hat Otti\pwindex{Salten, Ottilie 07.03.1868 – 22.06.1942@\textsc{Salten, Ottilie} (07.03.1868 – 22.06.1942), \emph{Schauspielerin}|pw} an Frau Beer-Hofmann\pwindex{Beer-Hofmann, Paula 25.02.1879 – 30.10.1939@\textsc{Beer-Hofmann, Paula} (25.02.1879 – 30.10.1939)|pw}
               geschrieben. Und nichts. Nett, nicht wahr?, wenn dann die »besseren Menschen« \uline{so} aussehen. Ich hoffe, dass Sie mich so sehr arg
               nicht missverstehen, und für Empfindlichkeit oder gar für Beleidigtsein nehmen, was
               nur ein ganz klares Abrechnen ist. Bei diesem Abrechnen sind \uline{alle} mildernden Umstände, \uline{alle}
               psychologischen Möglich\substVorne{}\textsuperscript{\textcolor{gray}{g}\textcolor{gray}{×}}\substDazwischen{}k\substHinten{}eiten nachfühlenden Begreifens schon in Anschlag gebracht, mit dem Resultat:
               man kann \uline{immer} eine \uline{Karte} schreiben! \uline{eine} Zeile! Ich meine,
               dieses ist jenseits von Empfindlichkeit und Beleidigtsein. Es ist ganz, ganz was
               anderes! Das alles unter uns und im Vertrauen. Ich muß mich über diese Sache
               aussprechen, hab es gestern an Hofmannsthal\pwindex{Hofmannsthal, Hugo von 1874-02-01 – 1929-07-15@\textsc{Hofmannsthal, Hugo von} (1874-02-01 – 1929-07-15), \emph{Schriftsteller}|pw} gethan, und that es heute an Sie. \substVorne{}\textsuperscript{\textcolor{gray}{W}}\substDazwischen{}De\substHinten{}nn so ganz einfach und wortlos mochte ich diese neueste Erfahrung nicht »zu
               den übrigen legen.« Will aber keine Diskussion mit B\textcolor{gray}{.}-H.\pwindex{Beer-Hofmann, Richard 1866-07-11 – 1945-09-26@\textsc{Beer-Hofmann, Richard} (1866-07-11 – 1945-09-26), \emph{Schriftsteller}|pw}, weil die Sache absolut nicht diskutirbar und
               für mich erledigt ist. Will auch nicht, dass dritte Personen drum wissen, weil {\dots} weil ich mich schäme!\pend
           \pstart
           Wenn die Kur, die ich gebrauche (Kohlensäure Bäder und Vibrations-Massage) vorbei
               ist, wenn es wirklich Frühling geworden, fange ich gleich mit einer Arbeit an. Das
               ist so gut an Berlin\oindex{Berlin@\textbf{Berlin}|pw}, dass man hier nur am
               Arbeiten Freude hat, an nichts anderem. Nicht am Spazierengehen, nicht an
               Landparthien, nicht an gemütlichem Schwatz und nicht an irgend welchen anderen
               freundlichen aber zeitraubenden Dingen. Man muß immer arbeiten, den ganzen Tag
               arbeiten, wenn man sich wol fühlen will.\pend
           \pstart
           {\pb}Eines ist mir sehr erfreulich
                  hier\oindex{Berlin@\textbf{Berlin}|pwv}, wenns nur so bleibt:
               dass die Kinder\pwindex{Rehmann, Anna Katharina 18.08.1904 – 27.03.1977@\textsc{Rehmann, Anna Katharina} (18.08.1904 – 27.03.1977), \emph{Schauspielerin, Übersetzerin}|pwv}\pwindex{Salten, Paul 11.08.1903 – 08.05.1937@\textsc{Salten, Paul} (11.08.1903 – 08.05.1937), \emph{Filmcutter}|pwv}
               sich so wol fühlen, und so brav essen. Annerl\pwindex{Rehmann, Anna Katharina 18.08.1904 – 27.03.1977@\textsc{Rehmann, Anna Katharina} (18.08.1904 – 27.03.1977), \emph{Schauspielerin, Übersetzerin}|pw}
               spricht jetzt schon so viel wie der Paul\pwindex{Salten, Paul 11.08.1903 – 08.05.1937@\textsc{Salten, Paul} (11.08.1903 – 08.05.1937), \emph{Filmcutter}|pw}, und
               ist so lieb, dass sich’s kaum sagen läßt. Neulich waren wir zum ersten Mal im Zool.\oindex{Zoologischer Garten Berlin@\textbf{Zoologischer Garten Berlin}|pw} Und im Nilpferdhaus waren beide Kinder\pwindex{Rehmann, Anna Katharina 18.08.1904 – 27.03.1977@\textsc{Rehmann, Anna Katharina} (18.08.1904 – 27.03.1977), \emph{Schauspielerin, Übersetzerin}|pwv}\pwindex{Salten, Paul 11.08.1903 – 08.05.1937@\textsc{Salten, Paul} (11.08.1903 – 08.05.1937), \emph{Filmcutter}|pwv} sprachlos vor
               Staunen. Da fing das eine Nilpferd laut zu schnauben und zu wiehern an, und Paul\pwindex{Salten, Paul 11.08.1903 – 08.05.1937@\textsc{Salten, Paul} (11.08.1903 – 08.05.1937), \emph{Filmcutter}|pw} war darüber so entsetzt, dass er in Thränen
               ausbrach, Annerl\pwindex{Rehmann, Anna Katharina 18.08.1904 – 27.03.1977@\textsc{Rehmann, Anna Katharina} (18.08.1904 – 27.03.1977), \emph{Schauspielerin, Übersetzerin}|pw} aber rief dem Nilpferd zu:
               »Sei still, Nilpferd, sonst muß Pauli\pwindex{Salten, Paul 11.08.1903 – 08.05.1937@\textsc{Salten, Paul} (11.08.1903 – 08.05.1937), \emph{Filmcutter}|pw} weinen!«
               Und Pauli\pwindex{Salten, Paul 11.08.1903 – 08.05.1937@\textsc{Salten, Paul} (11.08.1903 – 08.05.1937), \emph{Filmcutter}|pw} erzählte zu Hause der Grossmama\pwindex{Metzl, Louise 1832-08-06 – 1909-09-10@\textsc{Metzl, Louise} (1832-08-06 – 1909-09-10)|pwuv}, das
               Nilpferd habe »mit dem Mund ein Gewitter gemacht!« Daran ließe sich etwa ein
               verallgemeinerndes Aphorisma knüpfen, was ich aber unterlaße.\pend
           \pstart
           Viele herzliche Grüße von uns\pwindex{Salten, Ottilie 07.03.1868 – 22.06.1942@\textsc{Salten, Ottilie} (07.03.1868 – 22.06.1942), \emph{Schauspielerin}|pwv} zu Ihnen\pwindex{Schnitzler, Olga 17.01.1882 – 13.01.1970@\textsc{Schnitzler, Olga} (17.01.1882 – 13.01.1970), \emph{Schauspielerin, Sängerin}|pwv}.
               {\\[\baselineskip]}Ihr {\\[\baselineskip]}\spacefill\mbox{Salten}\pend
           \leftskip=0em{}
         
         \endnumbering\mylabel{h}\end{ledgroupsized}  \newcommand{\dateiname}{L03415}\newcommand{\titel}{Felix Salten an Arthur Schnitzler, 9. 3. 1906}\newcommand{\editorInnen}{Martin Anton Müller und Laura Untner}%% latex-leseansicht-abspann.tex
%% Abspann für die Leseansicht.
%% Der Schalter \ifkorrekturansicht ist bereits durch den Vorspann gesetzt.

%% latex-abspann.tex
%% Gemeinsamer Abspann für Korrekturansicht und Leseansicht.
%% Setzt den Schalter \ifkorrekturansicht voraus (gesetzt in den
%% einbindenden Dateien latex-korrekturansicht-abspann.tex bzw.
%% latex-leseansicht-abspann.tex).
%% ---------------------------------------------------------------

\normalsize

% Das esempio-Environment wird nur in der Leseansicht benötigt
\ifkorrekturansicht\else
\newenvironment{esempio}[3]%
{
    \vspace{1.5ex}
    \rlap{\underline{#1}}
    \par
    \setlength{\parindent}{0cm}
    \nopagebreak
    \leftskip=#2cm
    \rightskip=#3cm
}
{
    \par
}
\fi

\doendnotes{C}
\bigskip
\vfill

\clearpage

\footnotesize

\ifkorrekturansicht
  \lohead{\textsc{register}}
\fi

% theindex-Environment neu definieren ohne reledmac
\makeatletter
\renewenvironment{theindex}{%
  \ifkorrekturansicht
    \section*{\indexname}%
  \else
    \subsubsection*{Index der erwähnten Entitäten}%
  \fi
  \setlength{\parindent}{0pt}%
  \setlength{\parskip}{0pt plus 0.3pt}%
  \let\item\@idxitem
}{%
  \ifkorrekturansicht\clearpage\fi
}
\makeatother

\IfFileExists{\jobname-pw.ind}{\input{\jobname-pw.ind}}{}

% Quellenangabe nur in der Leseansicht
\ifkorrekturansicht\else
% Fallback-Definitionen, falls die .tex-Datei \titel etc. nicht gesetzt hat
\providecommand{\titel}{}
\providecommand{\editorInnen}{}
\providecommand{\dateiname}{\jobname}

\vspace{3cm}

\vfill

\footnotesize
\textsc{Quelle}: \titel. Herausgegeben von {\editorInnen}. In: \emph{Arthur Schnitzler: Briefwechsel mit Autorinnen und Autoren}.
 Digitale Edition, https://schnitzler-briefe.acdh.oeaw.ac.at/{\dateiname}.html (Stand \today)
\fi

\end{document}


      