%% latex-leseansicht-vorspann.tex
%% Vorspann für die Leseansicht.
%% Lädt die gemeinsame Datei latex-vorspann.tex mit nicht gesetztem Schalter.

\newif\ifkorrekturansicht
\korrekturansichtfalse

\input{../tex-inputs/latex-vorspann}

\begin{center}
            \textcolor{red}{ENTWURF, NICHT FERTIG KORRIGIERT}
                      \end{center}
            
         
         \renewcommand{\erwaehntePersonen}{Personen: Richard Beer-Hofmann, Paula Beer-Hofmann, Julius von Gans-Ludassy, Maximilian Harden, Hugo von Hofmannsthal, Louise Metzl, Wolfgang Amadeus Mozart, Anna Katharina Rehmann, Ottilie Salten, Paul Salten}
         \renewcommand{\erwaehnteInstitutionen}{Institutionen: B.Z. am Mittag, Concordia}
         \renewcommand{\erwaehnteOrte}{Orte: Berlin, Kantstraße, Kochstraße, Russland, Wien, Zoologischer Garten Berlin}
         \renewcommand{\erwaehnteWerke}{Werke: ?? [Artikel über Salten, gegen den dieser klagt], ?? [Feuilleton über Lange], ?? [Russisches Theater I], B.Z. am Mittag, Der Ruf des Lebens. Schauspiel in drei Akten, Die Zukunft, Frankfurter Zeitung, Gedenkrede auf Wolfgang Amade Mozart, Oedipus und die Sphinx. Tragödie in drei Aufzügen, Theater}
               \section[Felix Salten an Arthur Schnitzler, 9. 3. 1906]{ Felix Salten an Arthur Schnitzler, 9. 3. 1906}\nopagebreak\mylabel{v}\rehead{ }\begin{ledgroupsized}[t]{13cm}\normalsize\beginnumbering \toendnotes[C]{\smallbreak\pagebreak[2]} \Standort{CUL, Schnitzler, B 89, B 1.}
\physDesc{Brief, 1 Blatt, 3 Seiten
\newline{}Handschrift: schwarze Tinte, lateinische Kurrent\newline{}Ordnung: mit Bleistift von unbekannter Hand nummeriert:
                                    »206« }\toendnotes[C]{\smallbreak}\pstart
           \noindent{}{\pb}\textcolor{gray}{\textbf{B. Z. am Mittag}}\orgindex{B.Z. am Mittag@B.Z. am Mittag|pw}\hfill \textcolor{gray}{\textbf{BERLIN SW\oindex{Berlin@\textbf{Berlin}|pw},}}{ }9. III. 06.\pend
           \pstart
           \textcolor{gray}{\textbf{Chefredaktion}}\hfill \textcolor{gray}{\textbf{Kochstr. 23–25}}\oindex{Kochstrasse@\textbf{Kochstraße}|pw}\pend
           \pstart
           Lieber, hier sende ich Ihnen das Feuilleton\pwindex{Salten, Felix 06.09.1869 – 08.10.1945@\textsc{Salten, Felix} (06.09.1869 – 08.10.1945), \emph{Schriftsteller, Journalist}!?? [Feuilleton ueber Lange]vor dem 9.3.1906@\strich\emph{?? [Feuilleton über Lange]} {[}vor dem 9.3.1906{]}|pwv} – das einzige, das bisher kam – aus der »B. Z.\pwindex{?? Werk@Nicht ermittelte Verfasserinnen und Verfasser!B.Z. am Mittag1904-10-22 – 1943@\emph{B.Z. am Mittag} {[}1904-10-22 – 1943{]}|pw}« Montag will ich nochmals \label{K_L03415-11v}\edtext{über die Russen\oindex{Russland@\textbf{Russland}|pw}}{\lemma{\textnormal{\emph{über die Russen}}}\Cendnote{\textnormal{Vermutlich Bezug auf Moskauer
                     Theatertruppe, die in Berlin gastierte. Felix Salten\pwindex{Salten, Felix 06.09.1869 – 08.10.1945@\textsc{Salten, Felix} (06.09.1869 – 08.10.1945), \emph{Schriftsteller, Journalist}|pwk}: \emph{XXXX}\pwindex{Salten, Felix 06.09.1869 – 08.10.1945@\textsc{Salten, Felix} (06.09.1869 – 08.10.1945), \emph{Schriftsteller, Journalist}!?? [Russisches Theater I]1906-03-12@\strich\emph{?? [Russisches Theater I]} {[}1906-03-12{]}|pwk}. In: \emph{B. Z. am
                           Mittag}\pwindex{?? Werk@Nicht ermittelte Verfasserinnen und Verfasser!B.Z. am Mittag1904-10-22 – 1943@\emph{B.Z. am Mittag} {[}1904-10-22 – 1943{]}|pwk}, Jg. YY, Nr. YYY, 12. 3. 1906,
                     S. YY.}}}\label{K_L03415-11h} schreiben\pwindex{Salten, Felix 06.09.1869 – 08.10.1945@\textsc{Salten, Felix} (06.09.1869 – 08.10.1945), \emph{Schriftsteller, Journalist}!?? [Russisches Theater I]1906-03-12@\strich\emph{?? [Russisches Theater I]} {[}1906-03-12{]}|pwv}, und schicke es Ihnen dann gleich zu. Dass
               Sie so \label{K_L03415-2v}\edtext{verstimmt von hier
                  weggingen}{\lemma{\textnormal{\emph{verstimmt … weggingen}}}\Cendnote{\textnormal{Schnitzler\pwindex{Schnitzler, Arthur 15.05.1862 – 21.10.1931@\textsc{Schnitzler, Arthur} (15.05.1862 – 21.10.1931), \emph{Schriftsteller, Mediziner}|pwk} war anlässlich der Uraufführung
                  von \emph{Ruf des Lebens}\pwindex{Schnitzler, Arthur 15.05.1862 – 21.10.1931@\textsc{Schnitzler, Arthur} (15.05.1862 – 21.10.1931), \emph{Schriftsteller, Mediziner}!Ruf des Lebens. Schauspiel in drei Akten1906-02-20@\strich\emph{Der Ruf des Lebens. Schauspiel in drei Akten} {[}1906-02-20{]}|pwk} in Berlin\oindex{Berlin@\textbf{Berlin}|pwk} gewesen und am 27. 2. 1906 heimgekehrt. Zu diesem Zeitpunkt
                  waren bereits einige negative Kritiken erschienen.}}}\label{K_L03415-2h}, hat auch auf mich
               deprimirend gewirkt. Dieser »Ruf des Lebens\pwindex{Schnitzler, Arthur 15.05.1862 – 21.10.1931@\textsc{Schnitzler, Arthur} (15.05.1862 – 21.10.1931), \emph{Schriftsteller, Mediziner}!Ruf des Lebens. Schauspiel in drei Akten1906-02-20@\strich\emph{Der Ruf des Lebens. Schauspiel in drei Akten} {[}1906-02-20{]}|pw}«
               schien mir so unbezweifelbar, und ist es mir noch, dass seine Aufnahme für mich eine
               symptomatische Bedeutung annahm. \pend
           \pstart
           Es ist ein Glück, dass Sie stark genug sind, um sich kommende Produktion durch
               solche, an sich keineswegs wichtige Zwischenfälle, stören zu laßen. Darauf rechne ich
               sehr, und hoffe, bald von Ihnen zu hören, dass Sie arbeiten. Schlimm wäre es ja nur,
               wenn Sie, – mehr aus künstlerischer Hypochondrie als aus Selbstkritik – anfangen
               würden, in Ihrer Abschätzung dieses Stückes\pwindex{Schnitzler, Arthur 15.05.1862 – 21.10.1931@\textsc{Schnitzler, Arthur} (15.05.1862 – 21.10.1931), \emph{Schriftsteller, Mediziner}!Ruf des Lebens. Schauspiel in drei Akten1906-02-20@\strich\emph{Der Ruf des Lebens. Schauspiel in drei Akten} {[}1906-02-20{]}|pwv} wankend zu werden. Da kann man freilich für eine Weile den Boden
               unter sich schwinden fühlen. Aber es wäre, besonders in diesem Falle, das Falscheste!
               Sie müssen unbedingt dabei bleiben, dass Ihr Stück\pwindex{Schnitzler, Arthur 15.05.1862 – 21.10.1931@\textsc{Schnitzler, Arthur} (15.05.1862 – 21.10.1931), \emph{Schriftsteller, Mediziner}!Ruf des Lebens. Schauspiel in drei Akten1906-02-20@\strich\emph{Der Ruf des Lebens. Schauspiel in drei Akten} {[}1906-02-20{]}|pwv} im Recht ist, und dass die Zufälligkeit eines Abends nichts
               beweist. Dass \label{K_L03415-55v}\edtext{Harden\pwindex{Harden, Maximilian 20.10.1861 – 30.10.1927@\textsc{Harden, Maximilian} (20.10.1861 – 30.10.1927), \emph{Schriftsteller, Publizist}|pw}{ }\uline{so}{ }geschrieben\pwindex{Theater03. 03. 1906@\emph{Theater} {[}03. 03. 1906{]}|pwv}}{\lemma{\textnormal{\emph{Harden so geschrieben}}}\Cendnote{\textnormal{Es handelt sich um eine gemeinsame
                  Besprechung der Aufführungen von Hofmannsthal\pwindex{Hofmannsthal, Hugo von 1874-02-01 – 1929-07-15@\textsc{Hofmannsthal, Hugo von} (1874-02-01 – 1929-07-15), \emph{Schriftsteller}|pwk}s \emph{Oedipus und die Sphinx}\pwindex{Hofmannsthal, Hugo von 1874-02-01 – 1929-07-15@\textsc{Hofmannsthal, Hugo von} (1874-02-01 – 1929-07-15), \emph{Schriftsteller}!Oedipus und die Sphinx. Tragoedie in drei Aufzuegen1906@\strich\emph{Oedipus und die Sphinx. Tragödie in drei Aufzügen} {[}1906{]}|pwk}
                  und Schnitzler\pwindex{Schnitzler, Arthur 15.05.1862 – 21.10.1931@\textsc{Schnitzler, Arthur} (15.05.1862 – 21.10.1931), \emph{Schriftsteller, Mediziner}|pwk}s \emph{Der Ruf des Lebens}\pwindex{Schnitzler, Arthur 15.05.1862 – 21.10.1931@\textsc{Schnitzler, Arthur} (15.05.1862 – 21.10.1931), \emph{Schriftsteller, Mediziner}!Ruf des Lebens. Schauspiel in drei Akten1906-02-20@\strich\emph{Der Ruf des Lebens. Schauspiel in drei Akten} {[}1906-02-20{]}|pwk}: M. H.\pwindex{Harden, Maximilian 20.10.1861 – 30.10.1927@\textsc{Harden, Maximilian} (20.10.1861 – 30.10.1927), \emph{Schriftsteller, Publizist}|pwk} [ = Maximilian Harden\pwindex{Harden, Maximilian 20.10.1861 – 30.10.1927@\textsc{Harden, Maximilian} (20.10.1861 – 30.10.1927), \emph{Schriftsteller, Publizist}|pwk}]: \emph{Theater}\pwindex{Theater03. 03. 1906@\emph{Theater} {[}03. 03. 1906{]}|pwk}. In: \emph{Die Zukunft}\pwindex{Zukunft1892 – 1922@\emph{Die Zukunft} {[}1892 – 1922{]}|pwk}, Bd. 54,
                     H. 9, 3. 3. 1906, S. 346–356.}}}\label{K_L03415-55h} hat, ist im ersten
               Moment für Ihr Empfinden vielleicht sehr verletzend gewesen; tut aber wirklich
               nichts. Hätte er die Sache ausführlich und mit der ganzen Kraft seiner Dialektik
               zerrupft und zergliedert, dann wäre es schlimmer gewesen, denn es hätte \uline{gewirkt}. So aber hat hier, – und wol überall – jeder
               nur die Achsel gezuckt und gesagt: das glaubt Harden\pwindex{Harden, Maximilian 20.10.1861 – 30.10.1927@\textsc{Harden, Maximilian} (20.10.1861 – 30.10.1927), \emph{Schriftsteller, Publizist}|pw} selber nicht. Die Politik war gar zu sichtbar, als dass ein
               kritischer Einfluß erfolgen könnte. \pend
           \pstart
           Nach und nach kommt meine Wohnung\oindex{Kantstrasse@\textbf{Kantstraße}|pwv} in Ordnung, und ich kann eine menschliche Existenz beginnen. Könnte
               ich jetzt wieder von hier auswandern, dann wäre ich schon imstande, ein nettes Buch
                  überBerlin\oindex{Berlin@\textbf{Berlin}|pw} zu schreiben. Aber, ich hoffe, dass
               ich hier nicht sterben muß, und doch einmal werde reden können. Nach Wien\oindex{Wien@\textbf{Wien}|pw} sehne ich mich aber auch nicht. Dazu liegt mir die
               Schweinerei der letzten Affairen noch zu sehr im Magen. Haben Sie die letzte \label{K_L03415-5v}\edtext{Schurkerei des dramatischen Dichters Ludassy\pwindex{Gans-Ludassy, Julius von 13.04.1858 – 30.09.1922@\textsc{Gans-Ludassy, Julius von} (13.04.1858 – 30.09.1922), \emph{Schriftsteller, Journalist, Herausgeber}|pw}}{\lemma{\textnormal{\emph{Schurkerei … Ludassy}}}\Cendnote{\textnormal{vgl. A. S.: \emph{Tagebuch}, 30. 12. 1905, 14. 1. 1906 und vgl. A. S.: \emph{Tagebuch}, 12. 5. 1907. Offenbar hatte
                  es auch einen Artikel\pwindex{Gans-Ludassy, Julius von 13.04.1858 – 30.09.1922@\textsc{Gans-Ludassy, Julius von} (13.04.1858 – 30.09.1922), \emph{Schriftsteller, Journalist, Herausgeber}!?? [Artikel ueber Salten, gegen den dieser klagt]Ende 1905/Anfang 1906{[}25.12.1905{]}@\strich\emph{?? [Artikel über Salten, gegen den dieser klagt]} {[}Ende 1905/Anfang 1906{[}25.12.1905{]}{]}|pwkv}{ }Ludassy\pwindex{Gans-Ludassy, Julius von 13.04.1858 – 30.09.1922@\textsc{Gans-Ludassy, Julius von} (13.04.1858 – 30.09.1922), \emph{Schriftsteller, Journalist, Herausgeber}|pwk}s gegeben, an dem sich die Klage Salten\pwindex{Salten, Felix 06.09.1869 – 08.10.1945@\textsc{Salten, Felix} (06.09.1869 – 08.10.1945), \emph{Schriftsteller, Journalist}|pwk}s festmachte. (vgl. Felix Salten an Arthur Schnitzler, 17. 5. 1906) Das Ehrengericht des
                  Journalistenverbands \emph{Concordia}\orgindex{Concordia@Concordia|pwk} entschied am
                     17. 5. 1907
                  zugunsten Salten\pwindex{Salten, Felix 06.09.1869 – 08.10.1945@\textsc{Salten, Felix} (06.09.1869 – 08.10.1945), \emph{Schriftsteller, Journalist}|pwk}s. (Vgl. Gerhard
                     Hubmann: \emph{»Menschen, die einmal beinahe Freunde waren«. Felix
                        Salten und Arthur Schnitzler}. In: \emph{Im Schatten von
                        Bambi. Felix Salten entdeckt die Wiener Moderne. Leben und Werk}. Hg.
                     Marcel Atze unter Mitarbeit von Tanja Gausterer. Salzburg,
                     Wien: \emph{Residenz Verlag}{ }2020, S. 195.) }}}\label{K_L03415-5h} Jemandem erzählt? Wenn nicht, dann
               tun Sie’s doch, bitte. Es ist das Empörendste, dass so ein niederes {\pb}durch und durch verseuchtes
               Luder einen monatelang zwischen seinen Fingern halten darf; Na, Sie haben mich einmal
               einen »guten Hasser« genannt, – nicht ganz mit Recht, denn ich habe mich bisher noch
               nie an Jemandem gerächt. Aber diesmal will ich mir den Titel verdienen. So oder so.
               Und wenn nur der Prozess endlich anberaumt wird – ich hab mir’s genau überlegt – ich
               tue nichts, um ihn hinauszuschieben, dann will ich dafür sorgen, dass diesmal der
               angeklagte wirklich Angeklagter sein soll. \pend
           \pstart
           Übrigens, laßen wir das. Es gibt, gottseidank, bessere Menschen. Z. B. Beer-Hofmann\pwindex{Beer-Hofmann, Richard 1866-07-11 – 1945-09-26@\textsc{Beer-Hofmann, Richard} (1866-07-11 – 1945-09-26), \emph{Schriftsteller}|pw}, nicht wahr? Wie finden Sie es,
               dass er mir bis heute noch keine Zeile schrieb, keine Karte, nichts! Dabei bin ich
               doch nicht einfach nur verreist, bin in einer Lebensepoche, in der es nicht ganz
               unwichtig ist, die Festigkeit gewisser Beziehungen zu spüren, bin in einer Situation,
               in der es \uline{vielleicht} sogar tröstlich, \uline{jedenfalls} aber animirend sein kann, von Freunden was
               zu hören. Dabei hab \uline{ich}, mitten im
               Übersiedlungsrummel, im Fieber der neuen Stellung, in der Unrast des Hotelwohnens an
                  B-H.\pwindex{Beer-Hofmann, Richard 1866-07-11 – 1945-09-26@\textsc{Beer-Hofmann, Richard} (1866-07-11 – 1945-09-26), \emph{Schriftsteller}|pw} geschrieben, als ich sein \label{K_L03415-88v}\edtext{Mozart Feuilleton\pwindex{Beer-Hofmann, Richard 1866-07-11 – 1945-09-26@\textsc{Beer-Hofmann, Richard} (1866-07-11 – 1945-09-26), \emph{Schriftsteller}!Gedenkrede auf Wolfgang Amade Mozart1906-01-28@\strich\emph{Gedenkrede auf Wolfgang Amade Mozart} {[}1906-01-28{]}|pw}}{\lemma{\textnormal{\emph{Mozart Feuilleton}}}\Cendnote{\textnormal{Richard Beer-Hofmann\pwindex{Beer-Hofmann, Richard 1866-07-11 – 1945-09-26@\textsc{Beer-Hofmann, Richard} (1866-07-11 – 1945-09-26), \emph{Schriftsteller}|pwk}: \emph{Gedenkrede auf
                        Wolfgang Amadé Mozart}\pwindex{Beer-Hofmann, Richard 1866-07-11 – 1945-09-26@\textsc{Beer-Hofmann, Richard} (1866-07-11 – 1945-09-26), \emph{Schriftsteller}!Gedenkrede auf Wolfgang Amade Mozart1906-01-28@\strich\emph{Gedenkrede auf Wolfgang Amade Mozart} {[}1906-01-28{]}|pwk}. In: \emph{Frankfurter
                        Zeitung}\pwindex{?? Werk@Nicht ermittelte Verfasserinnen und Verfasser!Frankfurter Zeitung1856 – 1943@\emph{Frankfurter Zeitung} {[}1856 – 1943{]}|pwk}, Jg. 50, Nr. 27, 28. 1. 1906,
                     Erstes Morgenblatt, S. 1–2. Mozart\pwindex{Mozart, Wolfgang Amadeus 27.01.1756 – 05.12.1791@\textsc{Mozart, Wolfgang Amadeus} (27.01.1756 – 05.12.1791), \emph{Komponist}|pwk} hätte am 27. 1. 1756 seinen 150. Geburtstag
                  gefeiert.}}}\label{K_L03415-88h} las (auch dazu hatte ich Zeit gefunden){[},{]}
               dabei hatte ich noch ein \uline{zweitesmal} an ihn eine Karte
               geschickt. Dabei hat Otti\pwindex{Salten, Ottilie 07.03.1868 – 22.06.1942@\textsc{Salten, Ottilie} (07.03.1868 – 22.06.1942), \emph{Schauspielerin}|pw} an Frau Beer-Hofmann\pwindex{Beer-Hofmann, Paula 25.02.1879 – 30.10.1939@\textsc{Beer-Hofmann, Paula} (25.02.1879 – 30.10.1939)|pw} geschrieben. Und nichts. Nett, nicht wahr?,
               wenn dann die »besseren Menschen« \uline{so} aussehen. Ich
               hoffe, dass Sie mich so sehr arg nicht missverstehen, und für Empfindlichkeit oder
               gar für Beleidigtsein nehmen, was nur ein ganz klares Abrechnen ist. Bei diesem
               Abrechnen sind \uline{alle} mildernden Umstände, \uline{alle} psychologischen Möglichkeiten nachfühlenden
               Begreifens schon in Anschlag gebracht, mit dem Resultat: man kann \uline{immer} eine \uline{Karte}
               schreiben! \uline{eine} Zeile! Ich meine, dieses ist jenseits
               von Empfindlichkeit und Beleidigtsein. Es ist ganz, ganz was anderes! Das alles unter
               uns und im Vertrauen. Ich muß mich über diese Sache aussprechen, hab es gestern an
                  Hofmannsthal\pwindex{Hofmannsthal, Hugo von 1874-02-01 – 1929-07-15@\textsc{Hofmannsthal, Hugo von} (1874-02-01 – 1929-07-15), \emph{Schriftsteller}|pw} gethan, und that es heute an
               Sie. Denn so ganz einfach und wortlos möchte ich diese neueste Erfahrung nicht »zu
               den übrigen legen.« Will aber keine Diskussion mit B-H.\pwindex{Beer-Hofmann, Richard 1866-07-11 – 1945-09-26@\textsc{Beer-Hofmann, Richard} (1866-07-11 – 1945-09-26), \emph{Schriftsteller}|pw}, weil die Sache absolut nicht diskutirbar und für mich erledigt ist.
               Will auch nicht, dass dritte Personen drum wissen, weil {\dots}
               weil ich mich schäme! \pend
           \pstart
           Wenn die Kur, die ich gebrauche (Kohlensäure Bäder und Vibrations-Massage) vorbei
               ist, wenn es wirklich Frühling geworden, fange ich gleich mit einer Arbeit an. Das
               ist so gut an Berlin\oindex{Berlin@\textbf{Berlin}|pw}, dass man hier nur am
               Arbeiten Freude hat, an nichts anderem. Nicht am Spazierengehen, nicht an
               Landparthien, nicht an gemütlichem Schwatz und nicht an irgend welchen anderen
               freundlichen aber zeitraubenden Dingen. Man muß immer arbeiten, den ganzen Tag
               arbeiten, wenn man sich wol fühlen will. {\pb}Eines ist mir sehr erfreulich
               hier, wenns nur so bleibt: dass die Kinder\pwindex{Rehmann, Anna Katharina 18.08.1904 – 27.03.1977@\textsc{Rehmann, Anna Katharina} (18.08.1904 – 27.03.1977), \emph{Schauspielerin}|pwv}\pwindex{Salten, Paul 11.08.1903 – 08.05.1937@\textsc{Salten, Paul} (11.08.1903 – 08.05.1937), \emph{Filmcutter}|pwv} sich so wol fühlen, und so brav essen. Annerl\pwindex{Rehmann, Anna Katharina 18.08.1904 – 27.03.1977@\textsc{Rehmann, Anna Katharina} (18.08.1904 – 27.03.1977), \emph{Schauspielerin}|pw} spricht jetzt schon so viel wie der Paul\pwindex{Salten, Paul 11.08.1903 – 08.05.1937@\textsc{Salten, Paul} (11.08.1903 – 08.05.1937), \emph{Filmcutter}|pw}, und ist so lieb, dass sich’s kaum sagen
               läßt. Neulich waren wir zum ersten Mal im Zoo\oindex{Zoologischer Garten Berlin@\textbf{Zoologischer Garten Berlin}|pw}.
               Und im Nilpferdhaus waren beide Kinder\pwindex{Rehmann, Anna Katharina 18.08.1904 – 27.03.1977@\textsc{Rehmann, Anna Katharina} (18.08.1904 – 27.03.1977), \emph{Schauspielerin}|pwv}\pwindex{Salten, Paul 11.08.1903 – 08.05.1937@\textsc{Salten, Paul} (11.08.1903 – 08.05.1937), \emph{Filmcutter}|pwv} sprachlos vor Staunen. Da fing das eine Nilpferd
               laut zu schnauben und zu wiehern an, und Paul\pwindex{Salten, Paul 11.08.1903 – 08.05.1937@\textsc{Salten, Paul} (11.08.1903 – 08.05.1937), \emph{Filmcutter}|pw}
               war darüber so entsetzt, dass er in Thränen ausbrach, Annerl\pwindex{Rehmann, Anna Katharina 18.08.1904 – 27.03.1977@\textsc{Rehmann, Anna Katharina} (18.08.1904 – 27.03.1977), \emph{Schauspielerin}|pw} aber rief dem Nilpferd zu: »Sei still, Nilpferd, sonst
               muß Pauli\pwindex{Salten, Paul 11.08.1903 – 08.05.1937@\textsc{Salten, Paul} (11.08.1903 – 08.05.1937), \emph{Filmcutter}|pw} weinen!« Und Pauli\pwindex{Salten, Paul 11.08.1903 – 08.05.1937@\textsc{Salten, Paul} (11.08.1903 – 08.05.1937), \emph{Filmcutter}|pw} erzählte zu Hause der Grossmama\pwindex{Metzl, Louise 1832-08-06 – 1909-09-10@\textsc{Metzl, Louise} (1832-08-06 – 1909-09-10)|pwuv}, das Nilpferd habe »mit dem Mund
               ein Gewitter gemacht!« Daran ließe sich etwa ein verallgemeinerndes Aphorisma
               knüpfen, was ich aber unterlaße. \pend
           \pstart
           Viele herzliche Grüße von uns zu Ihnen. {\\[\baselineskip]}Ihr {\\[\baselineskip]}\spacefill\mbox{Salten}\pend
           \leftskip=0em{}
         
         \endnumbering\mylabel{h}\end{ledgroupsized}\begin{anhang}\end{anhang}\newcommand{\dateiname}{L03415}\newcommand{\titel}{Felix Salten an Arthur Schnitzler, 9. 3. 1906}\newcommand{\editorInnen}{Martin Anton Müller und Laura Untner}%% latex-leseansicht-abspann.tex
%% Abspann für die Leseansicht.
%% Der Schalter \ifkorrekturansicht ist bereits durch den Vorspann gesetzt.

%% latex-abspann.tex
%% Gemeinsamer Abspann für Korrekturansicht und Leseansicht.
%% Setzt den Schalter \ifkorrekturansicht voraus (gesetzt in den
%% einbindenden Dateien latex-korrekturansicht-abspann.tex bzw.
%% latex-leseansicht-abspann.tex).
%% ---------------------------------------------------------------

\normalsize

% Das esempio-Environment wird nur in der Leseansicht benötigt
\ifkorrekturansicht\else
\newenvironment{esempio}[3]%
{
    \vspace{1.5ex}
    \rlap{\underline{#1}}
    \par
    \setlength{\parindent}{0cm}
    \nopagebreak
    \leftskip=#2cm
    \rightskip=#3cm
}
{
    \par
}
\fi

\doendnotes{C}
\bigskip
\vfill

\clearpage

\footnotesize

\ifkorrekturansicht
  \lohead{\textsc{register}}
\fi

% theindex-Environment neu definieren ohne reledmac
\makeatletter
\renewenvironment{theindex}{%
  \ifkorrekturansicht
    \section*{\indexname}%
  \else
    \subsubsection*{Index der erwähnten Entitäten}%
  \fi
  \setlength{\parindent}{0pt}%
  \setlength{\parskip}{0pt plus 0.3pt}%
  \let\item\@idxitem
}{%
  \ifkorrekturansicht\clearpage\fi
}
\makeatother

\IfFileExists{\jobname-pw.ind}{\input{\jobname-pw.ind}}{}

% Quellenangabe nur in der Leseansicht
\ifkorrekturansicht\else
% Fallback-Definitionen, falls die .tex-Datei \titel etc. nicht gesetzt hat
\providecommand{\titel}{}
\providecommand{\editorInnen}{}
\providecommand{\dateiname}{\jobname}

\vspace{3cm}

\vfill

\footnotesize
\textsc{Quelle}: \titel. Herausgegeben von {\editorInnen}. In: \emph{Arthur Schnitzler: Briefwechsel mit Autorinnen und Autoren}.
 Digitale Edition, https://schnitzler-briefe.acdh.oeaw.ac.at/{\dateiname}.html (Stand \today)
\fi

\end{document}


      