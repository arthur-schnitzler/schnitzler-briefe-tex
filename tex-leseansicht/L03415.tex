%% latex-leseansicht-vorspann.tex
%% Vorspann für die Leseansicht.
%% Lädt die gemeinsame Datei latex-vorspann.tex mit nicht gesetztem Schalter.

\newif\ifkorrekturansicht
\korrekturansichtfalse

\input{../tex-inputs/latex-vorspann}


\section[ Felix Salten an Arthur Schnitzler, 9. 3. 1906]{L03415 Felix Salten an Arthur Schnitzler,  9. 3. 1906}
\nopagebreak\mylabel{L03415v}
\rehead{ }\normalsize\beginnumbering\briefempfaengerindex{Schnitzler, Arthur@\textsc{Schnitzler, Arthur}!zzzSalten, Felix@\emph{von Felix Salten}!1906-03-091@{9. 3. 1906}|(be}
\toendnotes[C]{\smallbreak\pagebreak[2]}
\correspDesc{Versand  durch Felix Salten am 9. 3. 1906 in Berlin
\newline{}Erhalt  durch Arthur Schnitzler im Zeitraum [10. 3. 1906
                  – 14. 3. 1906?] in Wien}\toendnotes[C]{\smallbreak}
\Standort{CUL, Schnitzler, B 89, B 1.}
\physDesc{Brief, 1 Blatt, 3 Seiten, 5504 Zeichen
\newline{}Handschrift: schwarze Tinte, lateinische Kurrent
\newline{}Ordnung: mit Bleistift von unbekannter Hand nummeriert: »206« }\toendnotes[C]{\smallbreak}
\pstart
           {\pb}\textcolor{gray}{\textbf{\emph{B. Z. am Mittag}}}\orgindex{B.Z. am Mittag@B.Z. am Mittag|pw}\hfill \textcolor{gray}{\textbf{\emph{BERLIN SW\oindex{Berlin@\textbf{Berlin}, \emph{Hauptstadt}|pw},}}}{ }9. III. 06.\pend
           
\pstart
           \textcolor{gray}{\textbf{\emph{Chefredaktion}}}\hfill \textcolor{gray}{\textbf{\emph{Kochstr. 23–25\oindex{Kochstraße@\textbf{Kochstraße}, \emph{Straße}|pw}}}}\pend
           \vspace{0.5em}
\pstart
           Lieber, hier sende ich Ihnen das \label{K_L03415-1v}\edtext{Feuilleton\pwindex{Salten, Felix 6.\,9.\,1869 Budapest – 8.\,10.\,1945 Zürich@\textsc{Salten, Felix} (6.\,9.\,1869 Budapest – 8.\,10.\,1945 Zürich), \emph{Schriftsteller, Journalist, Chefredakteur}!Russisches Theater. I@\strich\emph{Russisches Theater. I}|pwv}}{\lemma{\textnormal{\emph{Feuilleton}}}\Cendnote{\textnormal{Felix Salten\pwindex{Salten, Felix 6.\,9.\,1869 Budapest – 8.\,10.\,1945 Zürich@\textsc{Salten, Felix} (6.\,9.\,1869 Budapest – 8.\,10.\,1945 Zürich), \emph{Schriftsteller, Journalist, Chefredakteur}|pwk}: \emph{Russisches Theater. I}\pwindex{Salten, Felix 6.\,9.\,1869 Budapest – 8.\,10.\,1945 Zürich@\textsc{Salten, Felix} (6.\,9.\,1869 Budapest – 8.\,10.\,1945 Zürich), \emph{Schriftsteller, Journalist, Chefredakteur}!Russisches Theater. I@\strich\emph{Russisches Theater. I}|pwk}. In: \emph{B. Z. am Mittag}\pwindex{B.Z. am Mittag@\emph{B.Z. am Mittag}|pwk}, Jg. 30, Nr. 55, 6. 3. 1906, S. 2.}}}\label{K_L03415-1} – das einzige, das
               bisher kam – aus der »B. Z.\pwindex{B.Z. am Mittag@\emph{B.Z. am Mittag}|pw}« Montag will ich nochmals \label{K_L03415-2v}\edtext{über die Russ\oindex{Russland@\textbf{Russland}|pwv}en schreiben\pwindex{Salten, Felix 6.\,9.\,1869 Budapest – 8.\,10.\,1945 Zürich@\textsc{Salten, Felix} (6.\,9.\,1869 Budapest – 8.\,10.\,1945 Zürich), \emph{Schriftsteller, Journalist, Chefredakteur}!Russisches Theater. I@\strich\emph{Russisches Theater. I}|pwv}}{\lemma{\textnormal{\emph{über … schreiben}}}\Cendnote{\textnormal{Felix Salten\pwindex{Salten, Felix 6.\,9.\,1869 Budapest – 8.\,10.\,1945 Zürich@\textsc{Salten, Felix} (6.\,9.\,1869 Budapest – 8.\,10.\,1945 Zürich), \emph{Schriftsteller, Journalist, Chefredakteur}|pwk}: \emph{Russisches Theater. II}\pwindex{Salten, Felix 6.\,9.\,1869 Budapest – 8.\,10.\,1945 Zürich@\textsc{Salten, Felix} (6.\,9.\,1869 Budapest – 8.\,10.\,1945 Zürich), \emph{Schriftsteller, Journalist, Chefredakteur}!Russisches Theater. II@\strich\emph{Russisches Theater. II}|pwk}. In: \emph{B. Z. am Mittag}\pwindex{B.Z. am Mittag@\emph{B.Z. am Mittag}|pwk}, Jg. 30, Nr. 70, 23. 3. 1906, S. 2–3.}}}\label{K_L03415-2}, und schicke es
               Ihnen dann gleich zu. Dass Sie so \label{K_L03415-3v}\edtext{verstimmt von hier weggingen}{\lemma{\textnormal{\emph{verstimmt … weggingen}}}\Cendnote{\textnormal{Schnitzler war anlässlich der Uraufführung
                  von \emph{Der Ruf des Lebens}\pwindex{Schnitzler, Arthur 15.\,5.\,1862 Wien – 21.\,10.\,1931 ebd.@\textsc{Schnitzler, Arthur} (15.\,5.\,1862 Wien – 21.\,10.\,1931 ebd.), \emph{Schriftsteller, Mediziner}!Ruf des Lebens. Schauspiel in drei Akten@\strich\emph{Der Ruf des Lebens. Schauspiel in drei Akten}|pwk} in Berlin\oindex{Berlin@\textbf{Berlin}, \emph{Hauptstadt}|pwk} gewesen und am 27. 2. 1906 heimgekehrt. Zu diesem Zeitpunkt
                  waren bereits einige negative Kritiken erschienen.}}}\label{K_L03415-3}, hat auch auf mich
               deprimirend gewirkt. Dieser »Ruf des Lebens\pwindex{Schnitzler, Arthur 15.\,5.\,1862 Wien – 21.\,10.\,1931 ebd.@\textsc{Schnitzler, Arthur} (15.\,5.\,1862 Wien – 21.\,10.\,1931 ebd.), \emph{Schriftsteller, Mediziner}!Ruf des Lebens. Schauspiel in drei Akten@\strich\emph{Der Ruf des Lebens. Schauspiel in drei Akten}|pw}«
               schien mir so unbezweifelbar, und ist es mir noch, dass seine Aufnahme für mich eine
               symptomatische Bedeutung annahm.\pend
           
\pstart
           Es ist ein Glück, dass Sie stark genug sind, um sich kommende Produktion durch
               solche, an sich keineswegs wichtige Zwischenfälle, stören zu laßen. Darauf rechne ich
               sehr, und hoffe, bald von Ihnen zu hören, dass Sie arbeiten. Schlimm wäre es ja nur,
               wenn Sie, – mehr aus künstlerischer Hypochondrie als aus Selbstkritik – anfangen
               würden, in Ihrer Abschätzung dieses Stück\pwindex{Schnitzler, Arthur 15.\,5.\,1862 Wien – 21.\,10.\,1931 ebd.@\textsc{Schnitzler, Arthur} (15.\,5.\,1862 Wien – 21.\,10.\,1931 ebd.), \emph{Schriftsteller, Mediziner}!Ruf des Lebens. Schauspiel in drei Akten@\strich\emph{Der Ruf des Lebens. Schauspiel in drei Akten}|pwv}es wankend zu werden. Da kann man freilich für eine Weile den Boden
               unter sich schwinden fühlen. Aber es wäre, besonders in diesem Falle, das Falscheste!
               Sie müssen unbedingt dabei bleiben, dass Ihr Stück\pwindex{Schnitzler, Arthur 15.\,5.\,1862 Wien – 21.\,10.\,1931 ebd.@\textsc{Schnitzler, Arthur} (15.\,5.\,1862 Wien – 21.\,10.\,1931 ebd.), \emph{Schriftsteller, Mediziner}!Ruf des Lebens. Schauspiel in drei Akten@\strich\emph{Der Ruf des Lebens. Schauspiel in drei Akten}|pwv} im Recht ist, und dass die Zufälligkeit eines Abends nichts
               beweist. Dass \label{K_L03415-4v}\edtext{Harden\pwindex{Harden, Maximilian 20.\,10.\,1861 Berlin – 30.\,10.\,1927 Montana@\textsc{Harden, Maximilian} (20.\,10.\,1861 Berlin – 30.\,10.\,1927 Montana), \emph{Schriftsteller, Publizist}|pw}{ }\uline{so}{ }geschrieben\pwindex{Harden, Maximilian 20.\,10.\,1861 Berlin – 30.\,10.\,1927 Montana@\textsc{Harden, Maximilian} (20.\,10.\,1861 Berlin – 30.\,10.\,1927 Montana), \emph{Schriftsteller, Publizist}!Theater@\strich\emph{Theater}|pwv}}{\lemma{\textnormal{\emph{Harden so geschrieben}}}\Cendnote{\textnormal{Bezug auf eine gemeinsame Besprechung
                  der Aufführungen von Hofmannsthals\pwindex{Hofmannsthal, Hugo von 1.\,2.\,1874 Wien – 15.\,7.\,1929 Rodaun@\textsc{Hofmannsthal, Hugo von} (1.\,2.\,1874 Wien – 15.\,7.\,1929 Rodaun), \emph{Schriftsteller}|pwk}{ }\emph{Oedipus und die Sphinx}\pwindex{Hofmannsthal, Hugo von 1.\,2.\,1874 Wien – 15.\,7.\,1929 Rodaun@\textsc{Hofmannsthal, Hugo von} (1.\,2.\,1874 Wien – 15.\,7.\,1929 Rodaun), \emph{Schriftsteller}!Oedipus und die Sphinx. Tragödie in drei Aufzügen@\strich\emph{Oedipus und die Sphinx. Tragödie in drei Aufzügen}|pwk} und Schnitzlers{ }\emph{Der Ruf des Lebens}\pwindex{Schnitzler, Arthur 15.\,5.\,1862 Wien – 21.\,10.\,1931 ebd.@\textsc{Schnitzler, Arthur} (15.\,5.\,1862 Wien – 21.\,10.\,1931 ebd.), \emph{Schriftsteller, Mediziner}!Ruf des Lebens. Schauspiel in drei Akten@\strich\emph{Der Ruf des Lebens. Schauspiel in drei Akten}|pwk}: M. H.\pwindex{Harden, Maximilian 20.\,10.\,1861 Berlin – 30.\,10.\,1927 Montana@\textsc{Harden, Maximilian} (20.\,10.\,1861 Berlin – 30.\,10.\,1927 Montana), \emph{Schriftsteller, Publizist}|pwk} [ = Maximilian Harden\pwindex{Harden, Maximilian 20.\,10.\,1861 Berlin – 30.\,10.\,1927 Montana@\textsc{Harden, Maximilian} (20.\,10.\,1861 Berlin – 30.\,10.\,1927 Montana), \emph{Schriftsteller, Publizist}|pwk}]: \emph{Theater}\pwindex{Harden, Maximilian 20.\,10.\,1861 Berlin – 30.\,10.\,1927 Montana@\textsc{Harden, Maximilian} (20.\,10.\,1861 Berlin – 30.\,10.\,1927 Montana), \emph{Schriftsteller, Publizist}!Theater@\strich\emph{Theater}|pwk}. In: \emph{Die Zukunft}\pwindex{Zukunft@\emph{Die Zukunft}|pwk}, Bd. 54,
                     H. 9, 3. 3. 1906, S. 346–356.}}}\label{K_L03415-4} hat,
               ist im ersten Moment für Ihr Empfinden vielleicht sehr verletzend gewesen; tut aber
               wirklich nichts. Hätte er die Sache\pwindex{Schnitzler, Arthur 15.\,5.\,1862 Wien – 21.\,10.\,1931 ebd.@\textsc{Schnitzler, Arthur} (15.\,5.\,1862 Wien – 21.\,10.\,1931 ebd.), \emph{Schriftsteller, Mediziner}!Ruf des Lebens. Schauspiel in drei Akten@\strich\emph{Der Ruf des Lebens. Schauspiel in drei Akten}|pwv} ausführlich und mit der ganzen Kraft seiner Dialektik zerrupft und
               zergliedert, dann wäre es schlimmer gewesen, denn es hätte \uline{gewirkt}. So aber hat hier\oindex{Berlin@\textbf{Berlin}, \emph{Hauptstadt}|pwv}, – und wol überall – jeder nur die Achsel gezuckt und gesagt: Das
               glaubt Harden\pwindex{Harden, Maximilian 20.\,10.\,1861 Berlin – 30.\,10.\,1927 Montana@\textsc{Harden, Maximilian} (20.\,10.\,1861 Berlin – 30.\,10.\,1927 Montana), \emph{Schriftsteller, Publizist}|pw} selber nicht. Die Politik war
               gar zu sichtbar, als dass ein kritischer Einfluß erfolgen könnte.\pend
           
\pstart
           Nach und nach kommt meine \label{K_L03415-5v}\edtext{Wohnung\oindex{Kantstraße@\textbf{Kantstraße}, \emph{Straße}|pwv}}{\lemma{\textnormal{\emph{Wohnung}}}\Cendnote{\textnormal{Siehe XXXX Auszeichnungsfehler: Dokument L03413 nicht gefunden. }}}\label{K_L03415-5} in Ordnung,
               und ich kann eine menschliche Existenz beginnen. Könnte ich jetzt wieder von hier\oindex{Berlin@\textbf{Berlin}, \emph{Hauptstadt}|pwv} auswandern, dann wäre ich
               schon imstande, ein nettes Buch über Berlin\oindex{Berlin@\textbf{Berlin}, \emph{Hauptstadt}|pw} zu
               schreiben. Aber, ich hoffe, dass ich hier nicht sterben muß, und doch einmal werde
               reden können. Nach Wien\oindex{Wien@\textbf{Wien}, \emph{Verwaltungsgebiet}|pw} sehne ich mich aber auch
               nicht. Dazu liegt mir die Schweinerei der letzten Affären noch zu sehr im Magen.
               Haben Sie die letzte \label{K_L03415-6v}\edtext{Schurkerei des
               dramatischen Dichters Ludassy\pwindex{Gans-Ludassy, Julius von 13.\,4.\,1858 Wien – 30.\,9.\,1922 ebd.@\textsc{Gans-Ludassy, Julius von} (13.\,4.\,1858 Wien – 30.\,9.\,1922 ebd.), \emph{Schriftsteller, Journalist, Herausgeber}|pw}}{\lemma{\textnormal{\emph{Schurkerei … Ludassy}}}\Cendnote{\textnormal{Die Auseinandersetzung Ludassy\pwindex{Gans-Ludassy, Julius von 13.\,4.\,1858 Wien – 30.\,9.\,1922 ebd.@\textsc{Gans-Ludassy, Julius von} (13.\,4.\,1858 Wien – 30.\,9.\,1922 ebd.), \emph{Schriftsteller, Journalist, Herausgeber}|pwk}–Salten\pwindex{Salten, Felix 6.\,9.\,1869 Budapest – 8.\,10.\,1945 Zürich@\textsc{Salten, Felix} (6.\,9.\,1869 Budapest – 8.\,10.\,1945 Zürich), \emph{Schriftsteller, Journalist, Chefredakteur}|pwk} ist
                  komplex, da sie sowohl durch öffentliche Gerichtsverhandlungen als auch durch
                  Prozesse innerhalb der Journalismusorganisation \emph{Concordia}\orgindex{Concordia. Journalisten- und Schriftstellerverein@Concordia. Journalisten- und Schriftstellerverein|pwk} geführt wurde und jeweils unterschiedliche Aspekte verhandelt
                  wurden. In seinen \emph{Erinnerungen}\pwindex{Salten, Felix 6.\,9.\,1869 Budapest – 8.\,10.\,1945 Zürich@\textsc{Salten, Felix} (6.\,9.\,1869 Budapest – 8.\,10.\,1945 Zürich), \emph{Schriftsteller, Journalist, Chefredakteur}!Erinnerungen@\strich\emph{Erinnerungen}|pwk} schildert Salten\pwindex{Salten, Felix 6.\,9.\,1869 Budapest – 8.\,10.\,1945 Zürich@\textsc{Salten, Felix} (6.\,9.\,1869 Budapest – 8.\,10.\,1945 Zürich), \emph{Schriftsteller, Journalist, Chefredakteur}|pwk} die Sache wie folgt: »Jetzt
                     muss ich doch noch die Affaire Ludassy\pwindex{Gans-Ludassy, Julius von 13.\,4.\,1858 Wien – 30.\,9.\,1922 ebd.@\textsc{Gans-Ludassy, Julius von} (13.\,4.\,1858 Wien – 30.\,9.\,1922 ebd.), \emph{Schriftsteller, Journalist, Herausgeber}|pw}
                     erzählen, unter der ich kindischerweise länger als ein Jahr schmerzhaft
                     gelitten habe. Ludassy\pwindex{Gans-Ludassy, Julius von 13.\,4.\,1858 Wien – 30.\,9.\,1922 ebd.@\textsc{Gans-Ludassy, Julius von} (13.\,4.\,1858 Wien – 30.\,9.\,1922 ebd.), \emph{Schriftsteller, Journalist, Herausgeber}|pw} war mein erster
                     Chefredakteur und er war gewalttätig, wie man sich aus dem Aufbrechen des
                     Schreibtisches von J. J. David\pwindex{David, Jakob Julius 6.\,2.\,1859 Hranice – 20.\,11.\,1906 Wien@\textsc{David, Jakob Julius} (6.\,2.\,1859 Hranice – 20.\,11.\,1906 Wien), \emph{Schriftsteller, Journalist}|pw} erinnern
                     wird. Jetzt war er nicht mehr Chefredakteur und ich bei der ›Zeit\orgindex{Zeit@Die Zeit|pw}‹. Nun veranstaltete die Concordia\orgindex{Concordia. Journalisten- und Schriftstellerverein@Concordia. Journalisten- und Schriftstellerverein|pw} eine Protestversammlung gegen den
                     Chefredakteur der Zeit\orgindex{Zeit@Die Zeit|pw}, Dr. Kanner\pwindex{Kanner, Heinrich 9.\,11.\,1864 Galați – 15.\,2.\,1930 Wien@\textsc{Kanner, Heinrich} (9.\,11.\,1864 Galați – 15.\,2.\,1930 Wien), \emph{Herausgeber, Publizist}|pw}, mit der Anklage, er brülle die
                     Redakteure an. Ich erschien zu seiner Verteidigung und sagte unter anderem, das
                     Schreien bedeutet gar nichts. {[}›{]}Hier neben mir sitzt mein
                     Freund Ludassy\pwindex{Gans-Ludassy, Julius von 13.\,4.\,1858 Wien – 30.\,9.\,1922 ebd.@\textsc{Gans-Ludassy, Julius von} (13.\,4.\,1858 Wien – 30.\,9.\,1922 ebd.), \emph{Schriftsteller, Journalist, Herausgeber}|pw}, der mein erster
                     Chefredakteur gewesen ist und der auch geschrien hat. Deswegen sind wir doch
                     befreundet!‹ Ludassy\pwindex{Gans-Ludassy, Julius von 13.\,4.\,1858 Wien – 30.\,9.\,1922 ebd.@\textsc{Gans-Ludassy, Julius von} (13.\,4.\,1858 Wien – 30.\,9.\,1922 ebd.), \emph{Schriftsteller, Journalist, Herausgeber}|pw} rückte von mir ab,
                     murrte: ›Ihr Freund bin ich gewesen und diese Worte werden Sie
                        bereu{[}e{]}n!‹ Ich musste diese Worte, so harmlos sie auch
                     gemeint waren{[},{]} länger als ein Jahr bitterlich bereuen.
                     Denn Herr Dr. Ludassy\pwindex{Gans-Ludassy, Julius von 13.\,4.\,1858 Wien – 30.\,9.\,1922 ebd.@\textsc{Gans-Ludassy, Julius von} (13.\,4.\,1858 Wien – 30.\,9.\,1922 ebd.), \emph{Schriftsteller, Journalist, Herausgeber}|pw}, der mit einem
                     Theaterstück: der letzte Knopf\pwindex{Gans-Ludassy, Julius von 13.\,4.\,1858 Wien – 30.\,9.\,1922 ebd.@\textsc{Gans-Ludassy, Julius von} (13.\,4.\,1858 Wien – 30.\,9.\,1922 ebd.), \emph{Schriftsteller, Journalist, Herausgeber}!letzte Knopf. Volksstück in drei Aufzügen@\strich\emph{Der letzte Knopf. Volksstück in drei Aufzügen}|pw} im Deutschen Volkstheater\oindex{Wien@\textbf{Wien}!VII., Neubau@\textbf{VII., Neubau}!Volkstheater@\textbf{Volkstheater}, \emph{Theater}|pw} zur Aufführung
                     gelangte, schrieb ungefähr ein Jahr nach der Aufführung, es sei vor seiner
                     Premiere, der Mann mit dem Revolver vor ihm gestanden.« (\emph{Wienbibliothek im Rathaus}, Nachlass Salten\pwindex{Salten, Felix 6.\,9.\,1869 Budapest – 8.\,10.\,1945 Zürich@\textsc{Salten, Felix} (6.\,9.\,1869 Budapest – 8.\,10.\,1945 Zürich), \emph{Schriftsteller, Journalist, Chefredakteur}|pwk}, ZPH 1681/1 1.1.1.9.1, [S. 61].) In seinem
                  Brief vom XXXX Auszeichnungsfehler: Dokument L03434 nicht gefunden
                  schildert Salten\pwindex{Salten, Felix 6.\,9.\,1869 Budapest – 8.\,10.\,1945 Zürich@\textsc{Salten, Felix} (6.\,9.\,1869 Budapest – 8.\,10.\,1945 Zürich), \emph{Schriftsteller, Journalist, Chefredakteur}|pwk} den ersten Teil etwas
                  genauer. Die Stelle, die Salten\pwindex{Salten, Felix 6.\,9.\,1869 Budapest – 8.\,10.\,1945 Zürich@\textsc{Salten, Felix} (6.\,9.\,1869 Budapest – 8.\,10.\,1945 Zürich), \emph{Schriftsteller, Journalist, Chefredakteur}|pwk} meint, wird
                  von einer Wien\oindex{Wien@\textbf{Wien}, \emph{Verwaltungsgebiet}|pwk}er Zeitung so zitiert: »Herr Dr. von Ludaſſy\pwindex{Gans-Ludassy, Julius von 13.\,4.\,1858 Wien – 30.\,9.\,1922 ebd.@\textsc{Gans-Ludassy, Julius von} (13.\,4.\,1858 Wien – 30.\,9.\,1922 ebd.), \emph{Schriftsteller, Journalist, Herausgeber}|pw} ließ{ }ſich über
                        einen Wien\oindex{Wien@\textbf{Wien}, \emph{Verwaltungsgebiet}|pw}er Kritiker in einer Berliner\oindex{Berlin@\textbf{Berlin}, \emph{Hauptstadt}|pw} Wochenſchrift wie folgt aus:{ / }›Je weniger die handfeſten Burſchen verſtehen, deſto hochmütiger,
                        abſprechender und unflätiger{ }ſchreiben{ }ſie. Es gibt deren auch, die vor der
                        Aufführung den Autor um höhere Beträge anzupumpen verſuchen (Ich habe{ }ſelbſt
                        eine{ }ſolche Kreatur als Zeitungsherausgeber zum Kritiker gemacht; als ich
                        auf der Bühne als Autor mein Heil verſuchte,{ }ſtand dann dasſelbe Individuum
                        mit dem Revolver vor mir.) Antikritik hilft gegen{ }ſolche Schelme nicht. Denn
                        niemand, dem ein unehrlicher oder übermütiger Kritiker Leides zugefügt hat,
                        will die Beſtie reizen.‹« (y.: \emph{Kritik der Kritik.
                        (Erbauliches aus der »Concordia«)}\pwindex{Kritik der Kritik. (Erbauliches aus der »Concordia«.)@\emph{Kritik der Kritik. (Erbauliches aus der »Concordia«.)}|pwk}. In: \emph{Wiener Montags-Journal}\pwindex{Extrapost. Unparteiische Montags-Zeitung@\emph{Extrapost. Unparteiische Montags-Zeitung}|pwk}, Jg. 24, Nr. 1245, 18. 12. 1905,
                     S. 3–4.) Saltens\pwindex{Salten, Felix 6.\,9.\,1869 Budapest – 8.\,10.\,1945 Zürich@\textsc{Salten, Felix} (6.\,9.\,1869 Budapest – 8.\,10.\,1945 Zürich), \emph{Schriftsteller, Journalist, Chefredakteur}|pwk} Narration ist in
                  der Chronologie unzuverlässig. Die zeitliche Einordnung der Ereignisse lässt sich
                  mit der Uraufführung von \emph{Der letzte Knopf}\pwindex{Gans-Ludassy, Julius von 13.\,4.\,1858 Wien – 30.\,9.\,1922 ebd.@\textsc{Gans-Ludassy, Julius von} (13.\,4.\,1858 Wien – 30.\,9.\,1922 ebd.), \emph{Schriftsteller, Journalist, Herausgeber}!letzte Knopf. Volksstück in drei Aufzügen@\strich\emph{Der letzte Knopf. Volksstück in drei Aufzügen}|pwk} am
                     8. 4. 1900 nur
                  scheinbar vornehmen. Saltens\pwindex{Salten, Felix 6.\,9.\,1869 Budapest – 8.\,10.\,1945 Zürich@\textsc{Salten, Felix} (6.\,9.\,1869 Budapest – 8.\,10.\,1945 Zürich), \emph{Schriftsteller, Journalist, Chefredakteur}|pwk}{ }Feuilleton\pwindex{Salten, Felix 6.\,9.\,1869 Budapest – 8.\,10.\,1945 Zürich@\textsc{Salten, Felix} (6.\,9.\,1869 Budapest – 8.\,10.\,1945 Zürich), \emph{Schriftsteller, Journalist, Chefredakteur}!Deutsches Volkstheater. (»Der letzte Knopf.« Volksstück in drei Aufzügen von J. v. Gans-Ludassy)@\strich\emph{Deutsches Volkstheater. (»Der letzte Knopf.« Volksstück in drei Aufzügen von J. v. Gans-Ludassy)}|pwkv} erschien zwei
                  Tage später: f. s.\pwindex{Salten, Felix 6.\,9.\,1869 Budapest – 8.\,10.\,1945 Zürich@\textsc{Salten, Felix} (6.\,9.\,1869 Budapest – 8.\,10.\,1945 Zürich), \emph{Schriftsteller, Journalist, Chefredakteur}|pwk}: \emph{Deutsches Volkstheater. (»Der letzte Knopf.« Volksstück in drei Aufzügen
                        von J. v. Gans-Ludassy)}\pwindex{Salten, Felix 6.\,9.\,1869 Budapest – 8.\,10.\,1945 Zürich@\textsc{Salten, Felix} (6.\,9.\,1869 Budapest – 8.\,10.\,1945 Zürich), \emph{Schriftsteller, Journalist, Chefredakteur}!Deutsches Volkstheater. (»Der letzte Knopf.« Volksstück in drei Aufzügen von J. v. Gans-Ludassy)@\strich\emph{Deutsches Volkstheater. (»Der letzte Knopf.« Volksstück in drei Aufzügen von J. v. Gans-Ludassy)}|pwk}. In: \emph{Wiener
                        Allgemeinen Zeitung. 6 Uhr-Blatt}\pwindex{Wiener Allgemeine Zeitung@\emph{Wiener Allgemeine Zeitung}|pwk}, Nr. 6628, 10. 4. 1900, S. 2–3. Damals gab es aber die Tageszeitung \emph{Die Zeit}\orgindex{Zeit@Die Zeit|pwk} noch nicht, sodass er das Stück
                  verwechselt haben dürfte. Zeitlich passender ist die Uraufführung von Ludassys\pwindex{Gans-Ludassy, Julius von 13.\,4.\,1858 Wien – 30.\,9.\,1922 ebd.@\textsc{Gans-Ludassy, Julius von} (13.\,4.\,1858 Wien – 30.\,9.\,1922 ebd.), \emph{Schriftsteller, Journalist, Herausgeber}|pwk}{ }\emph{Der goldene Boden. Volksstück in vier
                     Aufzügen}\pwindex{Gans-Ludassy, Julius von 13.\,4.\,1858 Wien – 30.\,9.\,1922 ebd.@\textsc{Gans-Ludassy, Julius von} (13.\,4.\,1858 Wien – 30.\,9.\,1922 ebd.), \emph{Schriftsteller, Journalist, Herausgeber}!goldene Boden. Volksstück in vier Aufzügen@\strich\emph{Der goldene Boden. Volksstück in vier Aufzügen}|pwk}, die am 26. 3. 1904 (in Anwesenheit Schnitzlers) am \emph{Deutschen
                     Volkstheater}\orgindex{Volkstheater@Volkstheater|pwk} stattfand. Eine Besprechung Saltens\pwindex{Salten, Felix 6.\,9.\,1869 Budapest – 8.\,10.\,1945 Zürich@\textsc{Salten, Felix} (6.\,9.\,1869 Budapest – 8.\,10.\,1945 Zürich), \emph{Schriftsteller, Journalist, Chefredakteur}|pwk} lässt sich nicht nachweisen, doch dürfte das daran liegen, dass
                  unmittelbar davor ein längeres Feuilleton von Salten\pwindex{Salten, Felix 6.\,9.\,1869 Budapest – 8.\,10.\,1945 Zürich@\textsc{Salten, Felix} (6.\,9.\,1869 Budapest – 8.\,10.\,1945 Zürich), \emph{Schriftsteller, Journalist, Chefredakteur}|pwk} abgedruckt worden war und der Eindruck vermieden werden sollte,
                  dass das Blatt\orgindex{Wiener Allgemeine Zeitung@Wiener Allgemeine Zeitung|pwkv} zu wenige
                  Beiträgerinnen und Beiträger habe. Dementsprechend wäre das Kürzel »mm« Salten\pwindex{Salten, Felix 6.\,9.\,1869 Budapest – 8.\,10.\,1945 Zürich@\textsc{Salten, Felix} (6.\,9.\,1869 Budapest – 8.\,10.\,1945 Zürich), \emph{Schriftsteller, Journalist, Chefredakteur}|pwk} zuzuordnen: mm\pwindex{Salten, Felix 6.\,9.\,1869 Budapest – 8.\,10.\,1945 Zürich@\textsc{Salten, Felix} (6.\,9.\,1869 Budapest – 8.\,10.\,1945 Zürich), \emph{Schriftsteller, Journalist, Chefredakteur}|pwk} [ = Felix Salten\pwindex{Salten, Felix 6.\,9.\,1869 Budapest – 8.\,10.\,1945 Zürich@\textsc{Salten, Felix} (6.\,9.\,1869 Budapest – 8.\,10.\,1945 Zürich), \emph{Schriftsteller, Journalist, Chefredakteur}|pwk}?]: \emph{Deutsches
                        Volkstheater. (»Der goldene Boden«, Volksstück in vier Aufzügen von Julius
                        v. Gans-Ludassy. 26. März)}\pwindex{Salten, Felix 6.\,9.\,1869 Budapest – 8.\,10.\,1945 Zürich@\textsc{Salten, Felix} (6.\,9.\,1869 Budapest – 8.\,10.\,1945 Zürich), \emph{Schriftsteller, Journalist, Chefredakteur}!Deutsches Volkstheater. (»Der goldene Boden«, Volksstück in vier Aufzügen von Julius v. Gans-Ludassy. 26. März)@\strich\emph{Deutsches Volkstheater. (»Der goldene Boden«, Volksstück in vier Aufzügen von Julius v. Gans-Ludassy. 26. März)}|pwk}. In: \emph{Die
                        Zeit}\pwindex{Zeit@\emph{Die Zeit}|pwk}, Jg. 3, Nr. 538, 27. 3. 1904,
                     S. 3. Was genau mit dem »Revolver« in Saltens\pwindex{Salten, Felix 6.\,9.\,1869 Budapest – 8.\,10.\,1945 Zürich@\textsc{Salten, Felix} (6.\,9.\,1869 Budapest – 8.\,10.\,1945 Zürich), \emph{Schriftsteller, Journalist, Chefredakteur}|pwk}{ }\emph{Erinnerungen}\pwindex{Salten, Felix 6.\,9.\,1869 Budapest – 8.\,10.\,1945 Zürich@\textsc{Salten, Felix} (6.\,9.\,1869 Budapest – 8.\,10.\,1945 Zürich), \emph{Schriftsteller, Journalist, Chefredakteur}!Erinnerungen@\strich\emph{Erinnerungen}|pwk} gemeint war, klärte er an einer
                  anderen Stelle: »Mein ehemaliger Chef Ludassy\pwindex{Gans-Ludassy, Julius von 13.\,4.\,1858 Wien – 30.\,9.\,1922 ebd.@\textsc{Gans-Ludassy, Julius von} (13.\,4.\,1858 Wien – 30.\,9.\,1922 ebd.), \emph{Schriftsteller, Journalist, Herausgeber}|pw} verleumdete mich, ich hätte vor seiner Premiere von
                     ihm 3000 Kronen erpressen wollen. Es war mir leicht ihn zu widerlegen. Der
                     damalige Erzherzog Leopold Ferdinand\pwindex{Wölfling, Leopold Ferdinand Salvator 2.\,12.\,1868 Salzburg – 4.\,7.\,1935 Berlin@\textsc{Wölfling, Leopold Ferdinand Salvator} (2.\,12.\,1868 Salzburg – 4.\,7.\,1935 Berlin), \emph{Erzherzog}|pw}
                     suchte einen Kredit in dieser Höhe und ich fragte Ludassy\pwindex{Gans-Ludassy, Julius von 13.\,4.\,1858 Wien – 30.\,9.\,1922 ebd.@\textsc{Gans-Ludassy, Julius von} (13.\,4.\,1858 Wien – 30.\,9.\,1922 ebd.), \emph{Schriftsteller, Journalist, Herausgeber}|pw} um Rat.« (ZPH 1681/1 1.1.1.9.1,
                     [S. 4]) Die Behauptung\pwindex{Gans-Ludassy, Julius von 13.\,4.\,1858 Wien – 30.\,9.\,1922 ebd.@\textsc{Gans-Ludassy, Julius von} (13.\,4.\,1858 Wien – 30.\,9.\,1922 ebd.), \emph{Schriftsteller, Journalist, Herausgeber}!?? [Ludassy will von Salten erpresst worden sein]@\strich\emph{?? [Ludassy will von Salten erpresst worden sein]}|pwkv}{ }Ludassys\pwindex{Gans-Ludassy, Julius von 13.\,4.\,1858 Wien – 30.\,9.\,1922 ebd.@\textsc{Gans-Ludassy, Julius von} (13.\,4.\,1858 Wien – 30.\,9.\,1922 ebd.), \emph{Schriftsteller, Journalist, Herausgeber}|pwk}, es wäre vor der Premiere versucht
                  worden, ihn zu erpressen, entwickelte sich in der Darstellung Saltens\pwindex{Salten, Felix 6.\,9.\,1869 Budapest – 8.\,10.\,1945 Zürich@\textsc{Salten, Felix} (6.\,9.\,1869 Budapest – 8.\,10.\,1945 Zürich), \emph{Schriftsteller, Journalist, Chefredakteur}|pwk} auf folgende Weise weiter:
                     »Verabredetermaßen fragte Stephan
                        Grossmann\pwindex{Großmann, Stefan 19.\,5.\,1875 Wien – 3.\,1.\,1935 ebd.@\textsc{Großmann, Stefan} (19.\,5.\,1875 Wien – 3.\,1.\,1935 ebd.), \emph{Schriftsteller, Journalist}|pw} in der Arbeiterzeitung\orgindex{Arbeiter-Zeitung@Arbeiter-Zeitung|pw}
                     nach dem Namen des Revolvermanns. Dr Ludassy\pwindex{Gans-Ludassy, Julius von 13.\,4.\,1858 Wien – 30.\,9.\,1922 ebd.@\textsc{Gans-Ludassy, Julius von} (13.\,4.\,1858 Wien – 30.\,9.\,1922 ebd.), \emph{Schriftsteller, Journalist, Herausgeber}|pw} nannte mich. Worauf mich Stephan Grossmann\pwindex{Großmann, Stefan 19.\,5.\,1875 Wien – 3.\,1.\,1935 ebd.@\textsc{Großmann, Stefan} (19.\,5.\,1875 Wien – 3.\,1.\,1935 ebd.), \emph{Schriftsteller, Journalist}|pw} mit einem Kübel Unrat überschüttete. Ich kam mir in
                     meiner persönlichen und wegen meiner publizistischen Ehre schwer verletzt vor
                     und rief ein Ehrengericht gegen mich an. Bei dieser Ehrengerichtlichen
                     Verhandlung legte ich folgende Beweise vor:  1. Ich hatte Ludassy\pwindex{Gans-Ludassy, Julius von 13.\,4.\,1858 Wien – 30.\,9.\,1922 ebd.@\textsc{Gans-Ludassy, Julius von} (13.\,4.\,1858 Wien – 30.\,9.\,1922 ebd.), \emph{Schriftsteller, Journalist, Herausgeber}|pw} nur im Namen des Erzherzog Leopolds\pwindex{Wölfling, Leopold Ferdinand Salvator 2.\,12.\,1868 Salzburg – 4.\,7.\,1935 Berlin@\textsc{Wölfling, Leopold Ferdinand Salvator} (2.\,12.\,1868 Salzburg – 4.\,7.\,1935 Berlin), \emph{Erzherzog}|pw} gefragt, wo man einen Kredit von 8000 Kronen
                     für den Erzherzog aufnehmen könnte{[}.{]} (Dieser Kredit wurde
                     ihm wenig später vom Beamtenverein\orgindex{Wiener Beamtenverein@Wiener Beamtenverein|pw}
                        erteilt{[}.{]}) 2. Ich legte mein Feuilleton\pwindex{Salten, Felix 6.\,9.\,1869 Budapest – 8.\,10.\,1945 Zürich@\textsc{Salten, Felix} (6.\,9.\,1869 Budapest – 8.\,10.\,1945 Zürich), \emph{Schriftsteller, Journalist, Chefredakteur}!Deutsches Volkstheater. (»Der goldene Boden«, Volksstück in vier Aufzügen von Julius v. Gans-Ludassy. 26. März)@\strich\emph{Deutsches Volkstheater. (»Der goldene Boden«, Volksstück in vier Aufzügen von Julius v. Gans-Ludassy. 26. März)}|pwv} über Ludassys\pwindex{Gans-Ludassy, Julius von 13.\,4.\,1858 Wien – 30.\,9.\,1922 ebd.@\textsc{Gans-Ludassy, Julius von} (13.\,4.\,1858 Wien – 30.\,9.\,1922 ebd.), \emph{Schriftsteller, Journalist, Herausgeber}|pw}{ }Stück\pwindex{Gans-Ludassy, Julius von 13.\,4.\,1858 Wien – 30.\,9.\,1922 ebd.@\textsc{Gans-Ludassy, Julius von} (13.\,4.\,1858 Wien – 30.\,9.\,1922 ebd.), \emph{Schriftsteller, Journalist, Herausgeber}!goldene Boden. Volksstück in vier Aufzügen@\strich\emph{Der goldene Boden. Volksstück in vier Aufzügen}|pwv} vor, das eine
                     Lobeshymne darstellte. 3. Ich legte eine Reihe von Briefen und Eilpostkarten
                        Ludassys\pwindex{Gans-Ludassy, Julius von 13.\,4.\,1858 Wien – 30.\,9.\,1922 ebd.@\textsc{Gans-Ludassy, Julius von} (13.\,4.\,1858 Wien – 30.\,9.\,1922 ebd.), \emph{Schriftsteller, Journalist, Herausgeber}|pw} vor, in denen er teils für
                     meine Kritik\pwindex{Salten, Felix 6.\,9.\,1869 Budapest – 8.\,10.\,1945 Zürich@\textsc{Salten, Felix} (6.\,9.\,1869 Budapest – 8.\,10.\,1945 Zürich), \emph{Schriftsteller, Journalist, Chefredakteur}!Deutsches Volkstheater. (»Der goldene Boden«, Volksstück in vier Aufzügen von Julius v. Gans-Ludassy. 26. März)@\strich\emph{Deutsches Volkstheater. (»Der goldene Boden«, Volksstück in vier Aufzügen von Julius v. Gans-Ludassy. 26. März)}|pwv} heissen
                     Dank aussprach, teils noch lange nach der Premiere und nach meiner Kritik\pwindex{Salten, Felix 6.\,9.\,1869 Budapest – 8.\,10.\,1945 Zürich@\textsc{Salten, Felix} (6.\,9.\,1869 Budapest – 8.\,10.\,1945 Zürich), \emph{Schriftsteller, Journalist, Chefredakteur}!Deutsches Volkstheater. (»Der goldene Boden«, Volksstück in vier Aufzügen von Julius v. Gans-Ludassy. 26. März)@\strich\emph{Deutsches Volkstheater. (»Der goldene Boden«, Volksstück in vier Aufzügen von Julius v. Gans-Ludassy. 26. März)}|pwv} mir Briefe und
                     Eilkarten schrieb{[},{]} in denen er verlangte mich zu sehen, in
                     denen er meine Freundschaft pries, und die seinige beteuerte. Der Präsident
                     dieses Ehrenrates, Chefredakteur des Neuen
                        Wiener Tagblatt\orgindex{Neues Wiener Tagblatt@Neues Wiener Tagblatt|pw}es{[},{]}{ }Wilhelm Singer\pwindex{Singer, Wilhelm 26.\,11.\,1847 Bzenec – 10.\,10.\,1917 Wien@\textsc{Singer, Wilhelm} (26.\,11.\,1847 Bzenec – 10.\,10.\,1917 Wien), \emph{Journalist, Chefredakteur}|pw}, richtete mitten in der
                     Verhandlung an Ludassy\pwindex{Gans-Ludassy, Julius von 13.\,4.\,1858 Wien – 30.\,9.\,1922 ebd.@\textsc{Gans-Ludassy, Julius von} (13.\,4.\,1858 Wien – 30.\,9.\,1922 ebd.), \emph{Schriftsteller, Journalist, Herausgeber}|pw} die Frage: ›Sagen
                     Sie Herr Dr. schämen Sie sich denn gar nicht?!‹ Ludassy\pwindex{Gans-Ludassy, Julius von 13.\,4.\,1858 Wien – 30.\,9.\,1922 ebd.@\textsc{Gans-Ludassy, Julius von} (13.\,4.\,1858 Wien – 30.\,9.\,1922 ebd.), \emph{Schriftsteller, Journalist, Herausgeber}|pw} wurde zu einer schweren Rüge von der Concordia\orgindex{Concordia. Journalisten- und Schriftstellerverein@Concordia. Journalisten- und Schriftstellerverein|pw} verurteilt und zur Unfähigkeit
                     zwei Jahre lang ein Ehrenamt in der Concordia\orgindex{Concordia. Journalisten- und Schriftstellerverein@Concordia. Journalisten- und Schriftstellerverein|pw} zu bekleiden.« (ebd., [S. 61–62]). Das
                  Ehrengericht des Journalistenverbands\orgindex{Concordia. Journalisten- und Schriftstellerverein@Concordia. Journalisten- und Schriftstellerverein|pwkv} entschied am 12. 5. 1907 zugunsten Saltens\pwindex{Salten, Felix 6.\,9.\,1869 Budapest – 8.\,10.\,1945 Zürich@\textsc{Salten, Felix} (6.\,9.\,1869 Budapest – 8.\,10.\,1945 Zürich), \emph{Schriftsteller, Journalist, Chefredakteur}|pwk}. Die Rüge für Ludassy\pwindex{Gans-Ludassy, Julius von 13.\,4.\,1858 Wien – 30.\,9.\,1922 ebd.@\textsc{Gans-Ludassy, Julius von} (13.\,4.\,1858 Wien – 30.\,9.\,1922 ebd.), \emph{Schriftsteller, Journalist, Herausgeber}|pwk} lässt sich belegen, doch wurde er nur für ein Jahr von jeglichen
                  Ehrenämtern der \emph{Concordia}\orgindex{Concordia. Journalisten- und Schriftstellerverein@Concordia. Journalisten- und Schriftstellerverein|pwk} ausgeschlossen
                     (vgl. \emph{Wienbibliothek im Rathaus}, Nachlass Salten\pwindex{Salten, Felix 6.\,9.\,1869 Budapest – 8.\,10.\,1945 Zürich@\textsc{Salten, Felix} (6.\,9.\,1869 Budapest – 8.\,10.\,1945 Zürich), \emph{Schriftsteller, Journalist, Chefredakteur}|pwk}, ZPH 1681, 3.7.4). Salten\pwindex{Salten, Felix 6.\,9.\,1869 Budapest – 8.\,10.\,1945 Zürich@\textsc{Salten, Felix} (6.\,9.\,1869 Budapest – 8.\,10.\,1945 Zürich), \emph{Schriftsteller, Journalist, Chefredakteur}|pwk} schrieb in seinen \emph{Erinnerungen}\pwindex{Salten, Felix 6.\,9.\,1869 Budapest – 8.\,10.\,1945 Zürich@\textsc{Salten, Felix} (6.\,9.\,1869 Budapest – 8.\,10.\,1945 Zürich), \emph{Schriftsteller, Journalist, Chefredakteur}!Erinnerungen@\strich\emph{Erinnerungen}|pwk} weiter: »Damit beruhigte ich mich
                     aber nicht, rief ein zweites Ehrengericht an, das aus Prof. Dr. Joseph Redlich\pwindex{Redlich, Josef 18.\,6.\,1869 Hodonín – 11.\,11.\,1936 Wien@\textsc{Redlich, Josef} (18.\,6.\,1869 Hodonín – 11.\,11.\,1936 Wien), \emph{Politiker, Rechtswissenschaftler}|pw}, aus dem früheren Direktor
                     des Burgtheaters\orgindex{Burgtheater@Burgtheater|pw} Dr. Max Burkhard\pwindex{Burckhard, Max Eugen 14.\,7.\,1854 Korneuburg – 16.\,3.\,1912 Wien@\textsc{Burckhard, Max Eugen} (14.\,7.\,1854 Korneuburg – 16.\,3.\,1912 Wien), \emph{Schriftsteller, Rechtswissenschaftler, Theaterleiter}|pw} und aus zwei anderen hohen Richtern
                     bestand, die zwar keine Strafverfügung treffen konnten, deren Urteil aber mir
                     volle Genugtuung bot. Es hatten sich einige meiner Feinde zwar gemeldet, die
                     ich nur zum Teil persönlich kannte, und deren Zeugnis glatt abgewiesen
                     wurde.« (ZPH 1681/1 1.1.1.9.1, [S. 62]). Siehe auch A. S.: \emph{Tagebuch}, 30. 12. 1905, 14. 1. 1906 und 12. 5. 1907.}}}\label{K_L03415-6}
               Jemandem erzählt? Wenn nicht, dann tun Sie’s doch, bitte. Es ist das Empörendste,
               dass so ein niederes {\pb}durch und
               durch verseuchtes Luder einen monatelang zwischen seinen Fingern halten darf; Na, Sie
               haben mich einmal einen »guten Hasser« genannt, – nicht ganz mit Recht, denn ich habe
               mich bisher noch nie an Jemandem gerächt. Aber diesmal will ich mir den Titel
               verdienen. So oder so. Und wenn nur der Prozess endlich anberaumt wird – ich hab
               mir’s genau überlegt – ich tue nichts, um ihn hinauszuschieben, dann will ich dafür
               sorgen, dass diesmal der Angeklagte wirklich Angeklagter sein soll.\pend
           
\pstart
           Übrigens, laßen wir das. Es gibt, gottseidank, bessere Menschen. Z. B. Beer-Hofmann\pwindex{Beer-Hofmann, Richard 11.\,7.\,1866 Wien – 26.\,9.\,1945 New York City@\textsc{Beer-Hofmann, Richard} (11.\,7.\,1866 Wien – 26.\,9.\,1945 New York City), \emph{Schriftsteller}|pw}, nicht wahr? Wie finden Sie es,
               dass er mir bis heute noch keine Zeile schrieb, keine
               Karte, nichts! Dabei bin ich doch nicht einfach nur verreist, bin in einer
               Lebensepoche, in der es nicht ganz unwichtig ist, die Festigkeit gewisser Beziehungen
               zu spüren, bin in einer Situation, in der es \uline{vielleicht} sogar tröstlich, \uline{jedenfalls} aber
               animirend sein kann, von Freunden was zu hören. Dabei hab \uline{ich}, mitten im Übersiedlungsrummel, im Fieber der neuen Stellung\orgindex{B.Z. am Mittag@B.Z. am Mittag|pwv}, in der Unrast des Hotel\oindex{Hotel Saxonia@\textbf{Hotel Saxonia}, \emph{Hotel}|pwv}wohnens an B-H.\pwindex{Beer-Hofmann, Richard 11.\,7.\,1866 Wien – 26.\,9.\,1945 New York City@\textsc{Beer-Hofmann, Richard} (11.\,7.\,1866 Wien – 26.\,9.\,1945 New York City), \emph{Schriftsteller}|pw} geschrieben, als ich sein \label{K_L03415-7v}\edtext{Mozart\pwindex{Mozart, Wolfgang Amadeus 27.\,1.\,1756 Salzburg – 5.\,12.\,1791 Wien@\textsc{Mozart, Wolfgang Amadeus} (27.\,1.\,1756 Salzburg – 5.\,12.\,1791 Wien), \emph{Komponist}|pw} Feuilleton\pwindex{Beer-Hofmann, Richard 11.\,7.\,1866 Wien – 26.\,9.\,1945 New York City@\textsc{Beer-Hofmann, Richard} (11.\,7.\,1866 Wien – 26.\,9.\,1945 New York City), \emph{Schriftsteller}!Gedenkrede auf Wolfgang Amade Mozart@\strich\emph{Gedenkrede auf Wolfgang Amade Mozart}|pw}}{\lemma{\textnormal{\emph{Mozart Feuilleton}}}\Cendnote{\textnormal{Richard Beer-Hofmann\pwindex{Beer-Hofmann, Richard 11.\,7.\,1866 Wien – 26.\,9.\,1945 New York City@\textsc{Beer-Hofmann, Richard} (11.\,7.\,1866 Wien – 26.\,9.\,1945 New York City), \emph{Schriftsteller}|pwk}: \emph{Gedenkrede auf Wolfgang Amadé Mozart}\pwindex{Beer-Hofmann, Richard 11.\,7.\,1866 Wien – 26.\,9.\,1945 New York City@\textsc{Beer-Hofmann, Richard} (11.\,7.\,1866 Wien – 26.\,9.\,1945 New York City), \emph{Schriftsteller}!Gedenkrede auf Wolfgang Amade Mozart@\strich\emph{Gedenkrede auf Wolfgang Amade Mozart}|pwk}. In: \emph{Frankfurter Zeitung}\pwindex{Frankfurter Zeitung@\emph{Frankfurter Zeitung}|pwk}, Jg. 50, Nr. 27, 28. 1. 1906, Erstes Morgenblatt, S. 1–2. Mozart\pwindex{Mozart, Wolfgang Amadeus 27.\,1.\,1756 Salzburg – 5.\,12.\,1791 Wien@\textsc{Mozart, Wolfgang Amadeus} (27.\,1.\,1756 Salzburg – 5.\,12.\,1791 Wien), \emph{Komponist}|pwk} hätte am 27. 1. 1906 seinen 150. Geburtstag gefeiert.}}}\label{K_L03415-7} las (auch dazu
               hatte ich Zeit gefunden){[},{]} dabei hatte ich noch ein \uline{zweitesmal} an ihn eine Karte
                  geschickt\textcolor{gray}{.} Dabei hat Otti\pwindex{Salten, Ottilie 7.\,3.\,1868 Prag – 22.\,6.\,1942 Zürich@\textsc{Salten, Ottilie} (7.\,3.\,1868 Prag – 22.\,6.\,1942 Zürich), \emph{Schauspielerin}|pw} an Frau Beer-Hofmann\pwindex{Beer-Hofmann, Paula 25.\,2.\,1879 Wien – 30.\,10.\,1939 Zürich@\textsc{Beer-Hofmann, Paula} (25.\,2.\,1879 Wien – 30.\,10.\,1939 Zürich)|pw}
               geschrieben. Und nichts. Nett, nicht wahr?, wenn dann die »besseren Menschen« \uline{so} aussehen. Ich hoffe, dass Sie mich so sehr arg
               nicht missverstehen, und für Empfindlichkeit oder gar für Beleidigtsein nehmen, was
               nur ein ganz klares Abrechnen ist. Bei diesem Abrechnen sind \uline{alle} mildernden Umstände, \uline{alle}
               psychologischen Möglich\substVorne{}\textsuperscript{\textcolor{gray}{g}\textcolor{gray}{×}}\substDazwischen{}k\substHinten{}eiten nachfühlenden Begreifens schon in Anschlag gebracht, mit dem Resultat:
               man kann \uline{immer} eine \uline{Karte} schreiben! \uline{eine} Zeile! Ich meine,
               dieses ist jenseits von Empfindlichkeit und Beleidigtsein. Es ist ganz, ganz was
               anderes! Das alles unter uns und im Vertrauen. Ich muß mich über diese Sache
               aussprechen, hab es gestern an Hofmannsthal\pwindex{Hofmannsthal, Hugo von 1.\,2.\,1874 Wien – 15.\,7.\,1929 Rodaun@\textsc{Hofmannsthal, Hugo von} (1.\,2.\,1874 Wien – 15.\,7.\,1929 Rodaun), \emph{Schriftsteller}|pw} gethan, und that es heute an Sie. \substVorne{}\textsuperscript{\textcolor{gray}{W}}\substDazwischen{}De\substHinten{}nn so ganz einfach und wortlos mochte ich diese neueste Erfahrung nicht »zu
               den übrigen legen.« Will aber keine Diskussion mit B\textcolor{gray}{.}-H.\pwindex{Beer-Hofmann, Richard 11.\,7.\,1866 Wien – 26.\,9.\,1945 New York City@\textsc{Beer-Hofmann, Richard} (11.\,7.\,1866 Wien – 26.\,9.\,1945 New York City), \emph{Schriftsteller}|pw}, weil die Sache absolut nicht diskutirbar und
               für mich erledigt ist. Will auch nicht, dass dritte Personen drum wissen, weil {\dots} weil ich mich schäme!\pend
           
\pstart
           Wenn die Kur, die ich gebrauche (Kohlensäure Bäder und Vibrations-Massage) vorbei
               ist, wenn es wirklich Frühling geworden, fange ich gleich mit einer Arbeit an. Das
               ist so gut an Berlin\oindex{Berlin@\textbf{Berlin}, \emph{Hauptstadt}|pw}, dass man hier nur am
               Arbeiten Freude hat, an nichts anderem. Nicht am Spazierengehen, nicht an
               Landparthien, nicht an gemütlichem Schwatz und nicht an irgend welchen anderen
               freundlichen aber zeitraubenden Dingen. Man muß immer arbeiten, den ganzen Tag
               arbeiten, wenn man sich wol fühlen will.\pend
           
\pstart
           {\pb}Eines ist mir sehr erfreulich
                  hier\oindex{Berlin@\textbf{Berlin}, \emph{Hauptstadt}|pwv}, wenns nur so bleibt:
               dass die Kinder\pwindex{Rehmann, Anna Katharina 18.\,8.\,1904 Wien – 27.\,3.\,1977 Zürich@\textsc{Rehmann, Anna Katharina} (18.\,8.\,1904 Wien – 27.\,3.\,1977 Zürich), \emph{Schauspielerin, Übersetzerin}|pwv}\pwindex{Salten, Paul 11.\,8.\,1903 Wien – 8.\,5.\,1937 ebd.@\textsc{Salten, Paul} (11.\,8.\,1903 Wien – 8.\,5.\,1937 ebd.), \emph{Filmcutter}|pwv}
               sich so wol fühlen, und so brav essen. Annerl\pwindex{Rehmann, Anna Katharina 18.\,8.\,1904 Wien – 27.\,3.\,1977 Zürich@\textsc{Rehmann, Anna Katharina} (18.\,8.\,1904 Wien – 27.\,3.\,1977 Zürich), \emph{Schauspielerin, Übersetzerin}|pw}
               spricht jetzt schon so viel wie der Paul\pwindex{Salten, Paul 11.\,8.\,1903 Wien – 8.\,5.\,1937 ebd.@\textsc{Salten, Paul} (11.\,8.\,1903 Wien – 8.\,5.\,1937 ebd.), \emph{Filmcutter}|pw}, und
               ist so lieb, dass sich’s kaum sagen läßt. Neulich waren wir zum ersten Mal im Zool.\oindex{Zoologischer Garten Berlin@\textbf{Zoologischer Garten Berlin}, \emph{Zoo}|pw} Und im Nilpferdhaus waren beide Kinder\pwindex{Rehmann, Anna Katharina 18.\,8.\,1904 Wien – 27.\,3.\,1977 Zürich@\textsc{Rehmann, Anna Katharina} (18.\,8.\,1904 Wien – 27.\,3.\,1977 Zürich), \emph{Schauspielerin, Übersetzerin}|pwv}\pwindex{Salten, Paul 11.\,8.\,1903 Wien – 8.\,5.\,1937 ebd.@\textsc{Salten, Paul} (11.\,8.\,1903 Wien – 8.\,5.\,1937 ebd.), \emph{Filmcutter}|pwv} sprachlos vor
               Staunen. Da fing das eine Nilpferd laut zu schnauben und zu wiehern an, und Paul\pwindex{Salten, Paul 11.\,8.\,1903 Wien – 8.\,5.\,1937 ebd.@\textsc{Salten, Paul} (11.\,8.\,1903 Wien – 8.\,5.\,1937 ebd.), \emph{Filmcutter}|pw} war darüber so entsetzt, dass er in Thränen
               ausbrach, Annerl\pwindex{Rehmann, Anna Katharina 18.\,8.\,1904 Wien – 27.\,3.\,1977 Zürich@\textsc{Rehmann, Anna Katharina} (18.\,8.\,1904 Wien – 27.\,3.\,1977 Zürich), \emph{Schauspielerin, Übersetzerin}|pw} aber rief dem Nilpferd zu:
               »Sei still, Nilpferd, sonst muß Pauli\pwindex{Salten, Paul 11.\,8.\,1903 Wien – 8.\,5.\,1937 ebd.@\textsc{Salten, Paul} (11.\,8.\,1903 Wien – 8.\,5.\,1937 ebd.), \emph{Filmcutter}|pw} weinen!«
               Und Pauli\pwindex{Salten, Paul 11.\,8.\,1903 Wien – 8.\,5.\,1937 ebd.@\textsc{Salten, Paul} (11.\,8.\,1903 Wien – 8.\,5.\,1937 ebd.), \emph{Filmcutter}|pw} erzählte zu Hause der Grossmama\pwindex{Metzl, Louise 6.\,8.\,1832 Údlice – 10.\,9.\,1909 Salzburg@\textsc{Metzl, Louise} (6.\,8.\,1832 Údlice – 10.\,9.\,1909 Salzburg)|pwuv}, das
               Nilpferd habe »mit dem Mund ein Gewitter gemacht!« Daran ließe sich etwa ein
               verallgemeinerndes Aphorisma knüpfen, was ich aber unterlaße.\pend
           
\pstart
           Viele herzliche Grüße von uns\pwindex{Salten, Ottilie 7.\,3.\,1868 Prag – 22.\,6.\,1942 Zürich@\textsc{Salten, Ottilie} (7.\,3.\,1868 Prag – 22.\,6.\,1942 Zürich), \emph{Schauspielerin}|pwv} zu Ihnen\pwindex{Schnitzler, Olga 17.\,1.\,1882 Wien – 13.\,1.\,1970 Lugano@\textsc{Schnitzler, Olga} (17.\,1.\,1882 Wien – 13.\,1.\,1970 Lugano), \emph{Schauspielerin, Sängerin}|pwv}.
               {\\[\baselineskip]}Ihr {\\[\baselineskip]}\spacefill\mbox{Salten}\pend
           \leftskip=0em{}\selectlanguage{ngerman}\endnumbering\briefempfaengerindex{Schnitzler, Arthur@\textsc{Schnitzler, Arthur}!zzzSalten, Felix@\emph{von Felix Salten}!1906-03-091@{9. 3. 1906}|)be}\mylabel{L03415h}  \newcommand{\dateiname}{L03415}\newcommand{\titel}{Felix Salten an Arthur Schnitzler, 9. 3. 1906}\newcommand{\editorInnen}{Martin Anton Müller und Laura Untner}%% latex-leseansicht-abspann.tex
%% Abspann für die Leseansicht.
%% Der Schalter \ifkorrekturansicht ist bereits durch den Vorspann gesetzt.

%% latex-abspann.tex
%% Gemeinsamer Abspann für Korrekturansicht und Leseansicht.
%% Setzt den Schalter \ifkorrekturansicht voraus (gesetzt in den
%% einbindenden Dateien latex-korrekturansicht-abspann.tex bzw.
%% latex-leseansicht-abspann.tex).
%% ---------------------------------------------------------------

\normalsize

% Das esempio-Environment wird nur in der Leseansicht benötigt
\ifkorrekturansicht\else
\newenvironment{esempio}[3]%
{
    \vspace{1.5ex}
    \rlap{\underline{#1}}
    \par
    \setlength{\parindent}{0cm}
    \nopagebreak
    \leftskip=#2cm
    \rightskip=#3cm
}
{
    \par
}
\fi

\doendnotes{C}
\bigskip
\vfill

\clearpage

\footnotesize

\ifkorrekturansicht
  \lohead{\textsc{register}}
\fi

% theindex-Environment neu definieren ohne reledmac
\makeatletter
\renewenvironment{theindex}{%
  \ifkorrekturansicht
    \section*{\indexname}%
  \else
    \subsubsection*{Index der erwähnten Entitäten}%
  \fi
  \setlength{\parindent}{0pt}%
  \setlength{\parskip}{0pt plus 0.3pt}%
  \let\item\@idxitem
}{%
  \ifkorrekturansicht\clearpage\fi
}
\makeatother

\IfFileExists{\jobname-pw.ind}{\input{\jobname-pw.ind}}{}

% Quellenangabe nur in der Leseansicht
\ifkorrekturansicht\else
% Fallback-Definitionen, falls die .tex-Datei \titel etc. nicht gesetzt hat
\providecommand{\titel}{}
\providecommand{\editorInnen}{}
\providecommand{\dateiname}{\jobname}

\vspace{3cm}

\vfill

\footnotesize
\textsc{Quelle}: \titel. Herausgegeben von {\editorInnen}. In: \emph{Arthur Schnitzler: Briefwechsel mit Autorinnen und Autoren}.
 Digitale Edition, https://schnitzler-briefe.acdh.oeaw.ac.at/{\dateiname}.html (Stand \today)
\fi

\end{document}


