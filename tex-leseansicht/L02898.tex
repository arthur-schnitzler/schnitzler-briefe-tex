%% latex-leseansicht-vorspann.tex
%% Vorspann für die Leseansicht.
%% Lädt die gemeinsame Datei latex-vorspann.tex mit nicht gesetztem Schalter.

\newif\ifkorrekturansicht
\korrekturansichtfalse

\input{../tex-inputs/latex-vorspann}


\section[ Paul Goldmann an Arthur Schnitzler, 11. 12. [1899]]{L02898 Paul Goldmann an Arthur Schnitzler,  11. 12. [1899]}
\nopagebreak\mylabel{L02898v}
\rehead{ }\normalsize\beginnumbering\briefempfaengerindex{Schnitzler, Arthur@\textsc{Schnitzler, Arthur}!zzzGoldmann, Paul@\emph{von Paul Goldmann}!1899-12-112@{11. 12. [1899]}|(be}
\toendnotes[C]{\smallbreak\pagebreak[2]}
\correspDesc{Versand  durch Paul Goldmann am 11. 12. [1899] in Frankfurt am Main
\newline{}Erhalt  durch Arthur Schnitzler im Zeitraum [12. 12. 1899 – 16. 12. 1899?] in Wien}\toendnotes[C]{\smallbreak}
\Standort{DLA, A:Schnitzler, HS.NZ85.1.3169.}
\physDesc{Brief, 1 Blatt, 4 Seiten, 2328 Zeichen
\newline{}Handschrift: blaue Tinte, deutsche Kurrent
\newline{}Schnitzler: 1) mit Bleistift das Jahr »99« vermerkt  2) mit rotem Buntstift sieben Unterstreichungen}\toendnotes[C]{\smallbreak}
\pstart
           \raggedleft{}{\pb}Frankfurt\oindex{Frankfurt am Main@\textbf{Frankfurt am Main}, \emph{Hauptstadt}|pw}, 11. Dezember.\pend
           
\pstart\center{}Mein lieber Freund,\pend\vspace{0.5em}
\pstart
           Vielen Dank für Deine intereſſanten Mittheilungen! Daß \label{K_L02898-1v}\edtext{\textsc{Bahr\pwindex{Bahr, Hermann 19.\,7.\,1863 Linz – 15.\,1.\,1934 München@\textsc{Bahr, Hermann} (19.\,7.\,1863 Linz – 15.\,1.\,1934 München), \emph{Schriftsteller, Kritiker}|pw}} gegen Dein Stück\pwindex{Schnitzler, Arthur 15.\,5.\,1862 Wien – 21.\,10.\,1931 ebd.@\textsc{Schnitzler, Arthur} (15.\,5.\,1862 Wien – 21.\,10.\,1931 ebd.), \emph{Schriftsteller, Mediziner}!grüne Kakadu. Groteske in einem Akt@\strich\emph{Der grüne Kakadu. Groteske in einem Akt}|pwv}
                  intriguirt}{\lemma{\textnormal{\emph{Bahr … intriguirt}}}\Cendnote{\textnormal{Vermutlich ist diese Stelle
                  so zu lesen, dass Bahr\pwindex{Bahr, Hermann 19.\,7.\,1863 Linz – 15.\,1.\,1934 München@\textsc{Bahr, Hermann} (19.\,7.\,1863 Linz – 15.\,1.\,1934 München), \emph{Schriftsteller, Kritiker}|pwk} sich dagegen
                  gewendet hatte, dass \emph{Der grüne Kakadu}\pwindex{Schnitzler, Arthur 15.\,5.\,1862 Wien – 21.\,10.\,1931 ebd.@\textsc{Schnitzler, Arthur} (15.\,5.\,1862 Wien – 21.\,10.\,1931 ebd.), \emph{Schriftsteller, Mediziner}!grüne Kakadu. Groteske in einem Akt@\strich\emph{Der grüne Kakadu. Groteske in einem Akt}|pwk} wieder
                  auf den Spielplan des \emph{Burgtheaters}\orgindex{Burgtheater@Burgtheater|pwk} gesetzt
                  wurde (siehe XXXX Auszeichnungsfehler: Dokument L02893 nicht gefunden).}}}\label{K_L02898-1}, iſt
               ein Zug, der ganz zum Charakterbilde dieſes Burſchen\pwindex{Bahr, Hermann 19.\,7.\,1863 Linz – 15.\,1.\,1934 München@\textsc{Bahr, Hermann} (19.\,7.\,1863 Linz – 15.\,1.\,1934 München), \emph{Schriftsteller, Kritiker}|pwv} paßt. Wenn \textsc{Schlenther\pwindex{Schlenther, Paul 20.\,8.\,1854 Chernyakhovsk – 30.\,4.\,1916 Berlin@\textsc{Schlenther, Paul} (20.\,8.\,1854 Chernyakhovsk – 30.\,4.\,1916 Berlin), \emph{Schriftsteller, Kritiker, Theaterleiter}|pw}} Dich auf die \label{K_L02898-2v}\edtext{Aufführung Deiner
               zwei Einakter\pwindex{Schnitzler, Arthur 15.\,5.\,1862 Wien – 21.\,10.\,1931 ebd.@\textsc{Schnitzler, Arthur} (15.\,5.\,1862 Wien – 21.\,10.\,1931 ebd.), \emph{Schriftsteller, Mediziner}!Paracelsus. Versspiel in einem Akt@\strich\emph{Paracelsus. Versspiel in einem Akt}|pwv}\pwindex{Schnitzler, Arthur 15.\,5.\,1862 Wien – 21.\,10.\,1931 ebd.@\textsc{Schnitzler, Arthur} (15.\,5.\,1862 Wien – 21.\,10.\,1931 ebd.), \emph{Schriftsteller, Mediziner}!Gefährtin. Schauspiel in einem Akt@\strich\emph{Die Gefährtin. Schauspiel in einem Akt}|pwv}}{\lemma{\textnormal{\emph{Aufführung … Einakter}}}\Cendnote{\textnormal{Während die Aufführung von \emph{Der grüne Kakadu}\pwindex{Schnitzler, Arthur 15.\,5.\,1862 Wien – 21.\,10.\,1931 ebd.@\textsc{Schnitzler, Arthur} (15.\,5.\,1862 Wien – 21.\,10.\,1931 ebd.), \emph{Schriftsteller, Mediziner}!grüne Kakadu. Groteske in einem Akt@\strich\emph{Der grüne Kakadu. Groteske in einem Akt}|pwk} verboten blieb, wurden die
                  zwei anderen Einakter\pwindex{Schnitzler, Arthur 15.\,5.\,1862 Wien – 21.\,10.\,1931 ebd.@\textsc{Schnitzler, Arthur} (15.\,5.\,1862 Wien – 21.\,10.\,1931 ebd.), \emph{Schriftsteller, Mediziner}!Paracelsus. Versspiel in einem Akt@\strich\emph{Paracelsus. Versspiel in einem Akt}|pwkv}\pwindex{Schnitzler, Arthur 15.\,5.\,1862 Wien – 21.\,10.\,1931 ebd.@\textsc{Schnitzler, Arthur} (15.\,5.\,1862 Wien – 21.\,10.\,1931 ebd.), \emph{Schriftsteller, Mediziner}!Gefährtin. Schauspiel in einem Akt@\strich\emph{Die Gefährtin. Schauspiel in einem Akt}|pwkv} des Zyklus\pwindex{Schnitzler, Arthur 15.\,5.\,1862 Wien – 21.\,10.\,1931 ebd.@\textsc{Schnitzler, Arthur} (15.\,5.\,1862 Wien – 21.\,10.\,1931 ebd.), \emph{Schriftsteller, Mediziner}!grüne Kakadu – Paracelsus – Die Gefährtin. Drei Einakter@\strich\emph{Der grüne Kakadu – Paracelsus – Die Gefährtin. Drei Einakter}|pwkv} – \emph{Paracelsus}\pwindex{Schnitzler, Arthur 15.\,5.\,1862 Wien – 21.\,10.\,1931 ebd.@\textsc{Schnitzler, Arthur} (15.\,5.\,1862 Wien – 21.\,10.\,1931 ebd.), \emph{Schriftsteller, Mediziner}!Paracelsus. Versspiel in einem Akt@\strich\emph{Paracelsus. Versspiel in einem Akt}|pwk} und \emph{Die Gefährtin}\pwindex{Schnitzler, Arthur 15.\,5.\,1862 Wien – 21.\,10.\,1931 ebd.@\textsc{Schnitzler, Arthur} (15.\,5.\,1862 Wien – 21.\,10.\,1931 ebd.), \emph{Schriftsteller, Mediziner}!Gefährtin. Schauspiel in einem Akt@\strich\emph{Die Gefährtin. Schauspiel in einem Akt}|pwk} – auch weiterhin gegeben.}}}\label{K_L02898-2}
               warten läßt,{ }ſo rächt er{ }ſich, nach Art gemeiner Naturen, für die \label{K_L02898-3v}\edtext{Demüthigung}{\lemma{\textnormal{\emph{Demüthigung}}}\Cendnote{\textnormal{Womöglich Bezug auf die Kommentare der Presse hinsichtlich
                  der Absetzung von \emph{Der grüne Kakadu}\pwindex{Schnitzler, Arthur 15.\,5.\,1862 Wien – 21.\,10.\,1931 ebd.@\textsc{Schnitzler, Arthur} (15.\,5.\,1862 Wien – 21.\,10.\,1931 ebd.), \emph{Schriftsteller, Mediziner}!grüne Kakadu. Groteske in einem Akt@\strich\emph{Der grüne Kakadu. Groteske in einem Akt}|pwk}, beispielsweise\pwindex{Burgtheater. [Die Absetzung von Der grüne Kakadu]@\emph{Burgtheater. [Die Absetzung von Der grüne Kakadu]}|pwkv} am 21. 12. 1899: »\so{Schnitzler}’s ›\so{Grüner Kakadu}\pwindex{Schnitzler, Arthur 15.\,5.\,1862 Wien – 21.\,10.\,1931 ebd.@\textsc{Schnitzler, Arthur} (15.\,5.\,1862 Wien – 21.\,10.\,1931 ebd.), \emph{Schriftsteller, Mediziner}!grüne Kakadu. Groteske in einem Akt@\strich\emph{Der grüne Kakadu. Groteske in einem Akt}|pw}‹, der sonst immer nach ›Paracelsus\pwindex{Schnitzler, Arthur 15.\,5.\,1862 Wien – 21.\,10.\,1931 ebd.@\textsc{Schnitzler, Arthur} (15.\,5.\,1862 Wien – 21.\,10.\,1931 ebd.), \emph{Schriftsteller, Mediziner}!Paracelsus. Versspiel in einem Akt@\strich\emph{Paracelsus. Versspiel in einem Akt}|pw}‹
                     und der ›Gefährtin\pwindex{Schnitzler, Arthur 15.\,5.\,1862 Wien – 21.\,10.\,1931 ebd.@\textsc{Schnitzler, Arthur} (15.\,5.\,1862 Wien – 21.\,10.\,1931 ebd.), \emph{Schriftsteller, Mediziner}!Gefährtin. Schauspiel in einem Akt@\strich\emph{Die Gefährtin. Schauspiel in einem Akt}|pw}‹ folgte, ist, wie man
                     hört, aus dem Spielplan des Burgtheater\orgindex{Burgtheater@Burgtheater|pw}
                     gestrichen. Allerlei Einflüsse allerlei höfischer Kreise sollen dies bewirkt
                     haben. Schade, daß Herr Direktor \so{Schlenther}\pwindex{Schlenther, Paul 20.\,8.\,1854 Chernyakhovsk – 30.\,4.\,1916 Berlin@\textsc{Schlenther, Paul} (20.\,8.\,1854 Chernyakhovsk – 30.\,4.\,1916 Berlin), \emph{Schriftsteller, Kritiker, Theaterleiter}|pw} das nun einmal von der Zensur\orgindex{K. u. k. Zensurstelle@K. u. k. Zensurstelle|pwv} der Hofbühnen genehmigte Stück\pwindex{Schnitzler, Arthur 15.\,5.\,1862 Wien – 21.\,10.\,1931 ebd.@\textsc{Schnitzler, Arthur} (15.\,5.\,1862 Wien – 21.\,10.\,1931 ebd.), \emph{Schriftsteller, Mediziner}!grüne Kakadu. Groteske in einem Akt@\strich\emph{Der grüne Kakadu. Groteske in einem Akt}|pwv} trotz aller Einflüsse nicht doch gegeben hat. Wir
                     können diese allzu große Nachgiebigkeit gegen gewisse Strömungen nicht
                     billigen. Ist es aber einmal entschieden, daß der ›Grüne Kakadu\pwindex{Schnitzler, Arthur 15.\,5.\,1862 Wien – 21.\,10.\,1931 ebd.@\textsc{Schnitzler, Arthur} (15.\,5.\,1862 Wien – 21.\,10.\,1931 ebd.), \emph{Schriftsteller, Mediziner}!grüne Kakadu. Groteske in einem Akt@\strich\emph{Der grüne Kakadu. Groteske in einem Akt}|pw}‹ nicht mehr auf dem Burgtheater\orgindex{Burgtheater@Burgtheater|pw} erscheinen soll, dann ist zu wünschen, daß
                     wir ihm bald auf einer anderen Bühne (etwa dem Deutschen Volkstheater\orgindex{Volkstheater@Volkstheater|pw}) wieder begegnen.« (\emph{Arbeiter-Zeitung}\pwindex{Arbeiter-Zeitung@\emph{Arbeiter-Zeitung}|pwk}, Jg. 11, Nr. 351,
                        21. 12. 1899, Morgenblatt, S. 8.)}}}\label{K_L02898-3}, die er im
               Streit mit Dir über den »Kakadu\pwindex{Schnitzler, Arthur 15.\,5.\,1862 Wien – 21.\,10.\,1931 ebd.@\textsc{Schnitzler, Arthur} (15.\,5.\,1862 Wien – 21.\,10.\,1931 ebd.), \emph{Schriftsteller, Mediziner}!grüne Kakadu. Groteske in einem Akt@\strich\emph{Der grüne Kakadu. Groteske in einem Akt}|pw}« erlitten.\pend
           
\pstart
           Im \label{K_L02898-4v}\edtext{Falle \textsc{Wassermann\pwindex{Wassermann, Jakob 10.\,3.\,1873 Fürth – 1.\,1.\,1934 Altaussee@\textsc{Wassermann, Jakob} (10.\,3.\,1873 Fürth – 1.\,1.\,1934 Altaussee), \emph{Schriftsteller}|pw}}}{\lemma{\textnormal{\emph{Falle Wassermann}}}\Cendnote{\textnormal{Siehe XXXX Auszeichnungsfehler: Dokument L02892 nicht gefunden, XXXX Auszeichnungsfehler: Dokument L02897 nicht gefunden und XXXX Auszeichnungsfehler: Dokument L02900 nicht gefunden. Die im
                  folgenden skizzierte Kritik Wassermanns\pwindex{Wassermann, Jakob 10.\,3.\,1873 Fürth – 1.\,1.\,1934 Altaussee@\textsc{Wassermann, Jakob} (10.\,3.\,1873 Fürth – 1.\,1.\,1934 Altaussee), \emph{Schriftsteller}|pwk}
                  über Eugen D’Albert\pwindex{d’Albert, Eugen 10.\,4.\,1864 Glasgow – 3.\,3.\,1932 Riga@\textsc{d’Albert, Eugen} (10.\,4.\,1864 Glasgow – 3.\,3.\,1932 Riga), \emph{Komponist}|pwk} und Clemens Frankenstein\pwindex{Franckenstein, Clemens von 14.\,7.\,1875 Wiesentheid – 19.\,8.\,1942 Hechendorf am Pilsensee@\textsc{Franckenstein, Clemens von} (14.\,7.\,1875 Wiesentheid – 19.\,8.\,1942 Hechendorf am Pilsensee), \emph{Theaterleiter, Komponist, Dirigent}|pwk} konnte nicht nachgewiesen werden und
                  dürfte nie gedruckt worden sein.}}}\label{K_L02898-4}, in welchem, wie Du{ }ſagſt, die »Frankfurter Zeitung\orgindex{Frankfurter Zeitung@Frankfurter Zeitung|pw}« durchaus im Unrecht iſt, iſt
               die »Frankfurter Zeitung\orgindex{Frankfurter Zeitung@Frankfurter Zeitung|pw}« durchaus im Recht. \textsc{D’Albert\pwindex{d’Albert, Eugen 10.\,4.\,1864 Glasgow – 3.\,3.\,1932 Riga@\textsc{d’Albert, Eugen} (10.\,4.\,1864 Glasgow – 3.\,3.\,1932 Riga), \emph{Komponist}|pw}’s} Compoſitionen{ }ſind
               mittelmäßige Leiſtungen. Das wiſſen wir hier und das hat \strikeout{\textcolor{gray}{e}b\textcolor{gray}{en}ſ} Niemand beſtritten. \textsc{Frankensteins\pwindex{Franckenstein, Clemens von 14.\,7.\,1875 Wiesentheid – 19.\,8.\,1942 Hechendorf am Pilsensee@\textsc{Franckenstein, Clemens von} (14.\,7.\,1875 Wiesentheid – 19.\,8.\,1942 Hechendorf am Pilsensee), \emph{Theaterleiter, Komponist, Dirigent}|pw}} Compoſitionen{ }ſind {\pb}ebenfalls mittelmäßige Leiſtungen, die{ }ſich
               vielleicht auf demſelben Niveau, eher{ }ſogar ein wenig tiefer halten. Es geht aber
               abſolut nicht an, in derſelben Kritik \textsc{d’Albert\pwindex{d’Albert, Eugen 10.\,4.\,1864 Glasgow – 3.\,3.\,1932 Riga@\textsc{d’Albert, Eugen} (10.\,4.\,1864 Glasgow – 3.\,3.\,1932 Riga), \emph{Komponist}|pw}} ganz zu verwerfen, \textsc{Frankenstein\pwindex{Franckenstein, Clemens von 14.\,7.\,1875 Wiesentheid – 19.\,8.\,1942 Hechendorf am Pilsensee@\textsc{Franckenstein, Clemens von} (14.\,7.\,1875 Wiesentheid – 19.\,8.\,1942 Hechendorf am Pilsensee), \emph{Theaterleiter, Komponist, Dirigent}|pw}} hingegen ihm gegenüber zu loben, mag das Lob noch{ }ſo eingeſchränkt{ }ſein.
               Namentlich in dieſer Zuſammenſtellung liegt die Fälſchung des Urtheils. Und wenn
               dieſe Kritik noch dazu von einem Mitarbeiter\pwindex{Wassermann, Jakob 10.\,3.\,1873 Fürth – 1.\,1.\,1934 Altaussee@\textsc{Wassermann, Jakob} (10.\,3.\,1873 Fürth – 1.\,1.\,1934 Altaussee), \emph{Schriftsteller}|pwv} eingeſandt wird, der{ }ſeine Berichterſtattung bisher{ }ſtets in
               einer ans Gewiſſenloſe grenzenden Weiſe vernachläſſigt hat, – wenn derſelbe Berichterſtatter\pwindex{Wassermann, Jakob 10.\,3.\,1873 Fürth – 1.\,1.\,1934 Altaussee@\textsc{Wassermann, Jakob} (10.\,3.\,1873 Fürth – 1.\,1.\,1934 Altaussee), \emph{Schriftsteller}|pwv}, der die
               Aufführungen der \textsc{Duse\pwindex{Duse, Eleonora 3.\,10.\,1858 Vigevano – 21.\,4.\,1924 Pittsburgh@\textsc{Duse, Eleonora} (3.\,10.\,1858 Vigevano – 21.\,4.\,1924 Pittsburgh), \emph{Schauspielerin}|pw}} mit vier \strikeout{Z\textcolor{gray}{ei}l}{ } Zeilen{ }\label{K_L02898-5v}\edtext{abthut\pwindex{Wassermann, Jakob 10.\,3.\,1873 Fürth – 1.\,1.\,1934 Altaussee@\textsc{Wassermann, Jakob} (10.\,3.\,1873 Fürth – 1.\,1.\,1934 Altaussee), \emph{Schriftsteller}!D’Annunzio’s »Gioconda«@\strich\emph{D’Annunzio’s »Gioconda«}|pwv}}{\lemma{\textnormal{\emph{abthut}}}\Cendnote{\textnormal{Eleonora Duse\pwindex{Duse, Eleonora 3.\,10.\,1858 Vigevano – 21.\,4.\,1924 Pittsburgh@\textsc{Duse, Eleonora} (3.\,10.\,1858 Vigevano – 21.\,4.\,1924 Pittsburgh), \emph{Schauspielerin}|pwk} war zwischen 10. und 20. 11. 1899 im
                  Zuge eines Gastspiels am \emph{Raimund-Theater}\orgindex{Raimund-Theater@Raimund-Theater|pwk}
                  aufgetreten. Eine Rezension von Wassermann\pwindex{Wassermann, Jakob 10.\,3.\,1873 Fürth – 1.\,1.\,1934 Altaussee@\textsc{Wassermann, Jakob} (10.\,3.\,1873 Fürth – 1.\,1.\,1934 Altaussee), \emph{Schriftsteller}|pwk}
                  erschien am 14. 11. 1899, aber sie war deutlich länger als vier Zeilen: rm.\pwindex{Wassermann, Jakob 10.\,3.\,1873 Fürth – 1.\,1.\,1934 Altaussee@\textsc{Wassermann, Jakob} (10.\,3.\,1873 Fürth – 1.\,1.\,1934 Altaussee), \emph{Schriftsteller}|pwk} [ = Jakob Wassermann\pwindex{Wassermann, Jakob 10.\,3.\,1873 Fürth – 1.\,1.\,1934 Altaussee@\textsc{Wassermann, Jakob} (10.\,3.\,1873 Fürth – 1.\,1.\,1934 Altaussee), \emph{Schriftsteller}|pwk}]: \emph{D’Annunzio’s »Gioconda«}\pwindex{Wassermann, Jakob 10.\,3.\,1873 Fürth – 1.\,1.\,1934 Altaussee@\textsc{Wassermann, Jakob} (10.\,3.\,1873 Fürth – 1.\,1.\,1934 Altaussee), \emph{Schriftsteller}!D’Annunzio’s »Gioconda«@\strich\emph{D’Annunzio’s »Gioconda«}|pwk}. In: \emph{Frankfurter Zeitung}\pwindex{Frankfurter Zeitung@\emph{Frankfurter Zeitung}|pwk}, Jg. 44, Nr. 316, 14. 11. 1899,
                     Abendblatt, S. 1–2.}}}\label{K_L02898-5}, dem \textsc{Frankenstein\pwindex{Franckenstein, Clemens von 14.\,7.\,1875 Wiesentheid – 19.\,8.\,1942 Hechendorf am Pilsensee@\textsc{Franckenstein, Clemens von} (14.\,7.\,1875 Wiesentheid – 19.\,8.\,1942 Hechendorf am Pilsensee), \emph{Theaterleiter, Komponist, Dirigent}|pw}}-Conzert, \strikeout{über} deſſen Bedeutungsloſigkeit in der
                  Wien\oindex{Wien@\textbf{Wien}, \emph{Verwaltungsgebiet}|pw}er Conzertfluth klar genug iſt, einen {\pb}ganzen \label{K_L02898-6v}\edtext{Bericht}{\lemma{\textnormal{\emph{Bericht}}}\Cendnote{\textnormal{Nicht erschienen. Am 21. 11. 1899 hatte im
                     Kleinen Musikvereinssaal\oindex{Wien@\textbf{Wien}!I., Innere Stadt@\textbf{I., Innere Stadt}!Musikverein@\textbf{Musikverein}, \emph{Konzertsaal}|pwk} ein
                     »Compositions-Concert mit Orchester Clemens Franckenstein\pwindex{Franckenstein, Clemens von 14.\,7.\,1875 Wiesentheid – 19.\,8.\,1942 Hechendorf am Pilsensee@\textsc{Franckenstein, Clemens von} (14.\,7.\,1875 Wiesentheid – 19.\,8.\,1942 Hechendorf am Pilsensee), \emph{Theaterleiter, Komponist, Dirigent}|pw}« stattgefunden. Schnitzler hatte
                  daran teilgenommen.}}}\label{K_L02898-6}{ }\strikeout{\textcolor{gray}{wid}} widmet,{ }ſo liegt ohne jeden Zweifel das Beſtreben einer perſönlichen
               Dienſtleiſtung vor, und keine anſtändige Zeitung wird es{ }ſich von einem Herrn \textsc{Wassermann\pwindex{Wassermann, Jakob 10.\,3.\,1873 Fürth – 1.\,1.\,1934 Altaussee@\textsc{Wassermann, Jakob} (10.\,3.\,1873 Fürth – 1.\,1.\,1934 Altaussee), \emph{Schriftsteller}|pw}} gefallen laſſen, daß er, der{ }ſonſt{ }ſo{ }ſäumig in{ }ſeinen dienſtlichen
               Obliegenheiten{ }ſich zeigt, gleich mit der Feder bei der Hand iſt, wenn es gilt, einem
                  Bekannten\pwindex{Franckenstein, Clemens von 14.\,7.\,1875 Wiesentheid – 19.\,8.\,1942 Hechendorf am Pilsensee@\textsc{Franckenstein, Clemens von} (14.\,7.\,1875 Wiesentheid – 19.\,8.\,1942 Hechendorf am Pilsensee), \emph{Theaterleiter, Komponist, Dirigent}|pwv} eine Reklame zu
               machen.\pend
           
\pstart
           An \textsc{Schwartzkopf\pwindex{Schwarzkopf, Gustav 7.\,11.\,1853 Wien – 13.\,11.\,1939 ebd.@\textsc{Schwarzkopf, Gustav} (7.\,11.\,1853 Wien – 13.\,11.\,1939 ebd.), \emph{Schriftsteller}|pw}} werde ich keinen liebenswürdigen Brief{ }ſchreiben. Ich{ }ſchätze und verehre ihn,
               wie Du weißt. Aber \textsc{Hirschfeld\pwindex{Hirschfeld, Robert 17.\,9.\,1857 Žďár nad Sázavou – 2.\,4.\,1914 Salzburg@\textsc{Hirschfeld, Robert} (17.\,9.\,1857 Žďár nad Sázavou – 2.\,4.\,1914 Salzburg), \emph{Journalist, Musikkritiker}|pw}}{ }ſteht mir näher und iſt auch ohne jeden Zweifel in{ }ſeiner ganzen Art
               geeigneter, die Berichterſtattung für die »Frankfurter
                  Zeitung\orgindex{Frankfurter Zeitung@Frankfurter Zeitung|pw}« zu übernehmen, obwohl \textsc{Schwarzkopf\pwindex{Schwarzkopf, Gustav 7.\,11.\,1853 Wien – 13.\,11.\,1939 ebd.@\textsc{Schwarzkopf, Gustav} (7.\,11.\,1853 Wien – 13.\,11.\,1939 ebd.), \emph{Schriftsteller}|pw}}{ }ſicherlich{ }ſeine Sache auch{ }ſehr gut machen würde. Immerhin {\pb}habe ich für \textsc{Schwarzkopf\pwindex{Schwarzkopf, Gustav 7.\,11.\,1853 Wien – 13.\,11.\,1939 ebd.@\textsc{Schwarzkopf, Gustav} (7.\,11.\,1853 Wien – 13.\,11.\,1939 ebd.), \emph{Schriftsteller}|pw}} gewirkt, weil ich meinte, damit etwas \label{K_L02898-7v}\edtext{Dir zu Liebe}{\lemma{\textnormal{\emph{Dir zu Liebe}}}\Cendnote{\textnormal{Siehe XXXX Auszeichnungsfehler: Dokument L02897 nicht gefunden.
               }}}\label{K_L02898-7} zu thun. Im Augenblick wo Du das ablehnſt, verliert die Angelegenheit alles
               Intereſſe für mich, und ich werde mich fortan jeder Einwirkung enthalten.\pend
           
\pstart
           Viele treue Grüße! {\\[\baselineskip]}Dein {\\[\baselineskip]}\spacefill\mbox{Paul Goldmann.}\pend
           \leftskip=0em{}\selectlanguage{ngerman}\endnumbering\briefempfaengerindex{Schnitzler, Arthur@\textsc{Schnitzler, Arthur}!zzzGoldmann, Paul@\emph{von Paul Goldmann}!1899-12-112@{11. 12. [1899]}|)be}\mylabel{L02898h}  \newcommand{\dateiname}{L02898}\newcommand{\titel}{Paul Goldmann an Arthur Schnitzler, 11. 12. [1899]}\newcommand{\editorInnen}{Martin Anton Müller und Laura Untner}%% latex-leseansicht-abspann.tex
%% Abspann für die Leseansicht.
%% Der Schalter \ifkorrekturansicht ist bereits durch den Vorspann gesetzt.

%% latex-abspann.tex
%% Gemeinsamer Abspann für Korrekturansicht und Leseansicht.
%% Setzt den Schalter \ifkorrekturansicht voraus (gesetzt in den
%% einbindenden Dateien latex-korrekturansicht-abspann.tex bzw.
%% latex-leseansicht-abspann.tex).
%% ---------------------------------------------------------------

\normalsize

% Das esempio-Environment wird nur in der Leseansicht benötigt
\ifkorrekturansicht\else
\newenvironment{esempio}[3]%
{
    \vspace{1.5ex}
    \rlap{\underline{#1}}
    \par
    \setlength{\parindent}{0cm}
    \nopagebreak
    \leftskip=#2cm
    \rightskip=#3cm
}
{
    \par
}
\fi

\doendnotes{C}
\bigskip
\vfill

\clearpage

\footnotesize

\ifkorrekturansicht
  \lohead{\textsc{register}}
\fi

% theindex-Environment neu definieren ohne reledmac
\makeatletter
\renewenvironment{theindex}{%
  \ifkorrekturansicht
    \section*{\indexname}%
  \else
    \subsubsection*{Index der erwähnten Entitäten}%
  \fi
  \setlength{\parindent}{0pt}%
  \setlength{\parskip}{0pt plus 0.3pt}%
  \let\item\@idxitem
}{%
  \ifkorrekturansicht\clearpage\fi
}
\makeatother

\IfFileExists{\jobname-pw.ind}{\input{\jobname-pw.ind}}{}

% Quellenangabe nur in der Leseansicht
\ifkorrekturansicht\else
% Fallback-Definitionen, falls die .tex-Datei \titel etc. nicht gesetzt hat
\providecommand{\titel}{}
\providecommand{\editorInnen}{}
\providecommand{\dateiname}{\jobname}

\vspace{3cm}

\vfill

\footnotesize
\textsc{Quelle}: \titel. Herausgegeben von {\editorInnen}. In: \emph{Arthur Schnitzler: Briefwechsel mit Autorinnen und Autoren}.
 Digitale Edition, https://schnitzler-briefe.acdh.oeaw.ac.at/{\dateiname}.html (Stand \today)
\fi

\end{document}


