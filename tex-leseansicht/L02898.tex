%% latex-leseansicht-vorspann.tex
%% Vorspann für die Leseansicht.
%% Lädt die gemeinsame Datei latex-vorspann.tex mit nicht gesetztem Schalter.

\newif\ifkorrekturansicht
\korrekturansichtfalse

\input{../tex-inputs/latex-vorspann}

\begin{center}
            \textcolor{red}{ENTWURF, NICHT FERTIG KORRIGIERT}
                      \end{center}
            
         
         \renewcommand{\erwaehntePersonen}{Personen: Hermann Bahr, Eleonora Duse, Clemens von Franckenstein, Robert Hirschfeld, Paul Schlenther, Gustav Schwarzkopf, Jakob Wassermann, Eugen d’Albert}
         \renewcommand{\erwaehnteInstitutionen}{Institutionen: Burgtheater, Frankfurter Zeitung, K. u. k. Zensurstelle, Raimund-Theater, Volkstheater}
         \renewcommand{\erwaehnteOrte}{Orte: Frankfurt am Main, Musikverein, Wien}
         \renewcommand{\erwaehnteWerke}{Werke: Arbeiter-Zeitung, Burgtheater. [Die Absetzung von Der grüne Kakadu], Der grüne Kakadu – Paracelsus – Die Gefährtin. Drei Einakter, Der grüne Kakadu. Groteske in einem Akt, Die Gefährtin. Schauspiel in einem Akt, D’Annunzio’s »Gioconda«, Frankfurter Zeitung, Paracelsus. Versspiel in einem Akt}
               \section[ Paul Goldmann an Arthur Schnitzler, 11. 12. {[}1899{]}]{ Paul Goldmann an Arthur Schnitzler, 11. 12. {[}1899{]}}\nopagebreak\mylabel{v}\rehead{ }\begin{ledgroupsized}[t]{13cm}\normalsize\beginnumbering \toendnotes[C]{\smallbreak\pagebreak[2]} \Standort{DLA, A:Schnitzler, HS.NZ85.1.3169.}
\physDesc{Brief, 1 Blatt, 4 Seiten
\newline{}Handschrift: blaue Tinte, deutsche Kurrent
\newline{}Schnitzler: 1) mit Bleistift das Jahr »99« vermerkt  2) mit rotem Buntstift sieben Unterstreichungen}\toendnotes[C]{\smallbreak}\pstart
           \raggedleft{}{\pb}Frankfurt\oindex{Frankfurt am Main@\textbf{Frankfurt am Main}|pw}, 11. Dezember.\pend
           \pstart\center{}Mein lieber Freund,\pend\pstart
           Vielen Dank für Deine intereſſanten Mittheilungen! Daß \label{K_L02898-1v}\edtext{\textsc{Bahr\pwindex{Bahr, Hermann 19.07.1863 – 15.01.1934@\textsc{Bahr, Hermann} (19.07.1863 – 15.01.1934), \emph{Schriftsteller, Kritiker}|pw}} gegen Dein Stück\pwindex{Schnitzler, Arthur 15.05.1862 – 21.10.1931@\textsc{Schnitzler, Arthur} (15.05.1862 – 21.10.1931), \emph{Schriftsteller, Mediziner}!gruene Kakadu. Groteske in einem Akt1. 3. 1899@\strich\emph{Der grüne Kakadu. Groteske in einem Akt} {[}1. 3. 1899{]}|pwv}
                  intriguirt}{\lemma{\textnormal{\emph{Bahr … intriguirt}}}\Cendnote{\textnormal{Vermutlich ist diese Stelle
                  so zu lesen, dass Bahr\pwindex{Bahr, Hermann 19.07.1863 – 15.01.1934@\textsc{Bahr, Hermann} (19.07.1863 – 15.01.1934), \emph{Schriftsteller, Kritiker}|pwk} sich dagegen
                  gewendet hatte, dass \emph{Der grünen Kakadu}\pwindex{Schnitzler, Arthur 15.05.1862 – 21.10.1931@\textsc{Schnitzler, Arthur} (15.05.1862 – 21.10.1931), \emph{Schriftsteller, Mediziner}!gruene Kakadu. Groteske in einem Akt1. 3. 1899@\strich\emph{Der grüne Kakadu. Groteske in einem Akt} {[}1. 3. 1899{]}|pwk} wieder
                  auf den Spielplan des \emph{Burgtheater}\orgindex{Burgtheater@Burgtheater|pwk}s gesetzt
                  werde (siehe Paul Goldmann an Arthur Schnitzler, 12. 11. [1899]).}}}\label{K_L02898-1h}, iſt
               ein Zug, der ganz zum Charakterbilde dieſes Burſchen\pwindex{Bahr, Hermann 19.07.1863 – 15.01.1934@\textsc{Bahr, Hermann} (19.07.1863 – 15.01.1934), \emph{Schriftsteller, Kritiker}|pwv} paßt. Wenn \textsc{Schlenther\pwindex{Schlenther, Paul 20.08.1854 – 30.04.1916@\textsc{Schlenther, Paul} (20.08.1854 – 30.04.1916), \emph{Schriftsteller, Kritiker, Theaterleiter}|pw}} Dich auf die \label{K_L02898-2v}\edtext{Aufführung Deiner
               zwei Einakter\pwindex{Schnitzler, Arthur 15.05.1862 – 21.10.1931@\textsc{Schnitzler, Arthur} (15.05.1862 – 21.10.1931), \emph{Schriftsteller, Mediziner}!Paracelsus. Versspiel in einem Akt01. 11. 1898@\strich\emph{Paracelsus. Versspiel in einem Akt} {[}01. 11. 1898{]}|pwv}\pwindex{Schnitzler, Arthur 15.05.1862 – 21.10.1931@\textsc{Schnitzler, Arthur} (15.05.1862 – 21.10.1931), \emph{Schriftsteller, Mediziner}!Gefaehrtin. Schauspiel in einem Akt1899-03-01@\strich\emph{Die Gefährtin. Schauspiel in einem Akt} {[}1899-03-01{]}|pwv}}{\lemma{\textnormal{\emph{Aufführung … Einakter}}}\Cendnote{\textnormal{Während die Aufführung von \emph{Der grüne Kakadu}\pwindex{Schnitzler, Arthur 15.05.1862 – 21.10.1931@\textsc{Schnitzler, Arthur} (15.05.1862 – 21.10.1931), \emph{Schriftsteller, Mediziner}!gruene Kakadu. Groteske in einem Akt1. 3. 1899@\strich\emph{Der grüne Kakadu. Groteske in einem Akt} {[}1. 3. 1899{]}|pwk} verboten blieb, wurden die
                  zwei anderen Einakter\pwindex{Schnitzler, Arthur 15.05.1862 – 21.10.1931@\textsc{Schnitzler, Arthur} (15.05.1862 – 21.10.1931), \emph{Schriftsteller, Mediziner}!Paracelsus. Versspiel in einem Akt01. 11. 1898@\strich\emph{Paracelsus. Versspiel in einem Akt} {[}01. 11. 1898{]}|pwkv}\pwindex{Schnitzler, Arthur 15.05.1862 – 21.10.1931@\textsc{Schnitzler, Arthur} (15.05.1862 – 21.10.1931), \emph{Schriftsteller, Mediziner}!Gefaehrtin. Schauspiel in einem Akt1899-03-01@\strich\emph{Die Gefährtin. Schauspiel in einem Akt} {[}1899-03-01{]}|pwkv} des Zyklus\pwindex{Schnitzler, Arthur 15.05.1862 – 21.10.1931@\textsc{Schnitzler, Arthur} (15.05.1862 – 21.10.1931), \emph{Schriftsteller, Mediziner}!gruene Kakadu – Paracelsus – Die Gefaehrtin. Drei Einakter1898 – 1899@\strich\emph{Der grüne Kakadu – Paracelsus – Die Gefährtin. Drei Einakter} {[}1898 – 1899{]}|pwkv}’ – \emph{Paracelsus}\pwindex{Schnitzler, Arthur 15.05.1862 – 21.10.1931@\textsc{Schnitzler, Arthur} (15.05.1862 – 21.10.1931), \emph{Schriftsteller, Mediziner}!Paracelsus. Versspiel in einem Akt01. 11. 1898@\strich\emph{Paracelsus. Versspiel in einem Akt} {[}01. 11. 1898{]}|pwk} und \emph{Die Gefährtin}\pwindex{Schnitzler, Arthur 15.05.1862 – 21.10.1931@\textsc{Schnitzler, Arthur} (15.05.1862 – 21.10.1931), \emph{Schriftsteller, Mediziner}!Gefaehrtin. Schauspiel in einem Akt1899-03-01@\strich\emph{Die Gefährtin. Schauspiel in einem Akt} {[}1899-03-01{]}|pwk} – auch weiterhin gegeben.}}}\label{K_L02898-2h}
               warten läßt, ſo rächt er ſich, nach Art gemeiner Naturen, für die \label{K_L02898-56v}\edtext{Demüthigung}{\lemma{\textnormal{\emph{Demüthigung}}}\Cendnote{\textnormal{Womöglich Bezug auf die Kommentare der Presse hinsichtlich
                  der Absetzung von \emph{Der grüne Kakadu}\pwindex{Schnitzler, Arthur 15.05.1862 – 21.10.1931@\textsc{Schnitzler, Arthur} (15.05.1862 – 21.10.1931), \emph{Schriftsteller, Mediziner}!gruene Kakadu. Groteske in einem Akt1. 3. 1899@\strich\emph{Der grüne Kakadu. Groteske in einem Akt} {[}1. 3. 1899{]}|pwk}, beispielsweise\pwindex{?? Werk@Nicht ermittelte Verfasserinnen und Verfasser!Burgtheater. [Die Absetzung von Der gruene Kakadu]1899-12-21@\emph{Burgtheater. [Die Absetzung von Der grüne Kakadu]} {[}1899-12-21{]}|pwkv} am 21. 12. 1899: »\so{Schnitzler}\pwindex{Schnitzler, Arthur 15.05.1862 – 21.10.1931@\textsc{Schnitzler, Arthur} (15.05.1862 – 21.10.1931), \emph{Schriftsteller, Mediziner}|pw}’s ›\so{Grüner Kakadu}\pwindex{Schnitzler, Arthur 15.05.1862 – 21.10.1931@\textsc{Schnitzler, Arthur} (15.05.1862 – 21.10.1931), \emph{Schriftsteller, Mediziner}!gruene Kakadu. Groteske in einem Akt1. 3. 1899@\strich\emph{Der grüne Kakadu. Groteske in einem Akt} {[}1. 3. 1899{]}|pw}‹, der sonst immer nach ›Paracelsus\pwindex{Schnitzler, Arthur 15.05.1862 – 21.10.1931@\textsc{Schnitzler, Arthur} (15.05.1862 – 21.10.1931), \emph{Schriftsteller, Mediziner}!Paracelsus. Versspiel in einem Akt01. 11. 1898@\strich\emph{Paracelsus. Versspiel in einem Akt} {[}01. 11. 1898{]}|pw}‹
                     und der ›Gefährtin\pwindex{Schnitzler, Arthur 15.05.1862 – 21.10.1931@\textsc{Schnitzler, Arthur} (15.05.1862 – 21.10.1931), \emph{Schriftsteller, Mediziner}!Gefaehrtin. Schauspiel in einem Akt1899-03-01@\strich\emph{Die Gefährtin. Schauspiel in einem Akt} {[}1899-03-01{]}|pw}‹ folgte, ist, wie man
                     hört, aus dem Spielplan des Burgtheater\orgindex{Burgtheater@Burgtheater|pw}
                     gestrichen. Allerlei Einflüsse allerlei höfischer Kreise sollen dies bewirkt
                     haben. Schade, daß Herr Direktor \so{Schlenther}\pwindex{Schlenther, Paul 20.08.1854 – 30.04.1916@\textsc{Schlenther, Paul} (20.08.1854 – 30.04.1916), \emph{Schriftsteller, Kritiker, Theaterleiter}|pw} das nun einmal von der Zensur\orgindex{K. u. k. Zensurstelle@K. u. k. Zensurstelle|pwv} der Hofbühnen genehmigte Stück\pwindex{Schnitzler, Arthur 15.05.1862 – 21.10.1931@\textsc{Schnitzler, Arthur} (15.05.1862 – 21.10.1931), \emph{Schriftsteller, Mediziner}!gruene Kakadu. Groteske in einem Akt1. 3. 1899@\strich\emph{Der grüne Kakadu. Groteske in einem Akt} {[}1. 3. 1899{]}|pwv} trotz aller Einflüsse nicht doch gegeben hat. Wir
                     können diese allzu große Nachgiebigkeit gegen gewisse Strömungen nicht
                     billigen. Ist es aber einmal entschieden, daß der ›Grüne Kakadu\pwindex{Schnitzler, Arthur 15.05.1862 – 21.10.1931@\textsc{Schnitzler, Arthur} (15.05.1862 – 21.10.1931), \emph{Schriftsteller, Mediziner}!gruene Kakadu. Groteske in einem Akt1. 3. 1899@\strich\emph{Der grüne Kakadu. Groteske in einem Akt} {[}1. 3. 1899{]}|pw}‹ nicht mehr auf dem Burgtheater\orgindex{Burgtheater@Burgtheater|pw} erscheinen soll, dann ist zu wünschen, daß
                     wir ihm bald auf einer anderen Bühne (etwa dem Deutschen Volkstheater\orgindex{Volkstheater@Volkstheater|pw}) wieder begegnen.« (\emph{Arbeiter-Zeitung}\pwindex{Arbeiter-Zeitung12.7.1881 – 31.10.1991@\emph{Arbeiter-Zeitung} {[}12.7.1881 – 31.10.1991{]}|pwk}, Jg. 11, Nr. 351,
                        21. 12. 1899, Morgenblatt, S. 8)}}}\label{K_L02898-56h}, die er im
               Streit mit Dir über den »Kakadu\pwindex{Schnitzler, Arthur 15.05.1862 – 21.10.1931@\textsc{Schnitzler, Arthur} (15.05.1862 – 21.10.1931), \emph{Schriftsteller, Mediziner}!gruene Kakadu. Groteske in einem Akt1. 3. 1899@\strich\emph{Der grüne Kakadu. Groteske in einem Akt} {[}1. 3. 1899{]}|pw}« erlitten.\pend
           \pstart
           Im \label{K_L02898-3v}\edtext{Falle \textsc{Wassermann\pwindex{Wassermann, Jakob 10.03.1873 – 01.01.1934@\textsc{Wassermann, Jakob} (10.03.1873 – 01.01.1934), \emph{Schriftsteller}|pw}}}{\lemma{\textnormal{\emph{Falle Wassermann}}}\Cendnote{\textnormal{siehe Paul Goldmann an Arthur Schnitzler, 26. 10. 1899, 6. 12. [1899] und 23. 12. [1899]. Die im
                  folgenden skizzierte Kritik Wassermann\pwindex{Wassermann, Jakob 10.03.1873 – 01.01.1934@\textsc{Wassermann, Jakob} (10.03.1873 – 01.01.1934), \emph{Schriftsteller}|pwk}s
                  über Eugen D’Albert\pwindex{DAlbert, Eugen 10.04.1864 – 03.03.1932@\textsc{d’Albert, Eugen} (10.04.1864 – 03.03.1932), \emph{Komponist}|pwk} und Clemens Frankenstein\pwindex{Franckenstein, Clemens von 14.07.1875 – 19.08.1942@\textsc{Franckenstein, Clemens von} (14.07.1875 – 19.08.1942), \emph{Theaterleiter, Komponist, Dirigent}|pwk} konnte nicht nachgewiesen werden und
                  dürfte nie gedruckt worden sein.}}}\label{K_L02898-3h}, in welchem, wie Du ſagſt, die »Frankfurter Zeitung\orgindex{Frankfurter Zeitung@Frankfurter Zeitung|pw}« durchaus im Unrecht iſt, iſt
               die »Frankfurter Zeitung\orgindex{Frankfurter Zeitung@Frankfurter Zeitung|pw}« durchaus im Recht. \textsc{D’Albert\pwindex{DAlbert, Eugen 10.04.1864 – 03.03.1932@\textsc{d’Albert, Eugen} (10.04.1864 – 03.03.1932), \emph{Komponist}|pw}’s} Compoſitionen ſind
               mittelmäßige Leiſtungen. Das wiſſen wir hier und das hat \strikeout{\textcolor{gray}{e}b\textcolor{gray}{en}ſ} Niemand beſtritten. \textsc{Frankenstein\pwindex{Franckenstein, Clemens von 14.07.1875 – 19.08.1942@\textsc{Franckenstein, Clemens von} (14.07.1875 – 19.08.1942), \emph{Theaterleiter, Komponist, Dirigent}|pw}s} Compoſitionen ſind {\pb}ebenfalls mittelmäßige Leiſtungen, die ſich
               vielleicht auf demſelben Niveau, eher ſogar ein wenig tiefer halten. Es geht aber
               abſolut nicht an, in derſelben Kritik \textsc{d’Albert\pwindex{DAlbert, Eugen 10.04.1864 – 03.03.1932@\textsc{d’Albert, Eugen} (10.04.1864 – 03.03.1932), \emph{Komponist}|pw}} ganz zu verwerfen, \textsc{Frankenstein\pwindex{Franckenstein, Clemens von 14.07.1875 – 19.08.1942@\textsc{Franckenstein, Clemens von} (14.07.1875 – 19.08.1942), \emph{Theaterleiter, Komponist, Dirigent}|pw}} hingegen ihm gegenüber zu loben, mag das Lob noch ſo eingeſchränkt ſein.
               Namentlich in dieſer Zuſammenſtellung liegt die Fälſchung des Urtheils. Und wenn
               dieſe Kritik noch dazu von einem Mitarbeiter\pwindex{Wassermann, Jakob 10.03.1873 – 01.01.1934@\textsc{Wassermann, Jakob} (10.03.1873 – 01.01.1934), \emph{Schriftsteller}|pwv} eingeſandt wird, der ſeine Berichterſtattung bisher ſtets in
               einer ans Gewiſſenloſe grenzenden Weiſe vernachläſſigt hat, – wenn derſelbe Berichterſtatter\pwindex{Wassermann, Jakob 10.03.1873 – 01.01.1934@\textsc{Wassermann, Jakob} (10.03.1873 – 01.01.1934), \emph{Schriftsteller}|pwv}, der die
               Aufführungen der \textsc{Duse\pwindex{Duse, Eleonora 03.10.1858 – 21.04.1924@\textsc{Duse, Eleonora} (03.10.1858 – 21.04.1924), \emph{Schauspielerin}|pw}} mit vier \strikeout{Z\textcolor{gray}{ei}l}{ } Zeilen{ }\label{K_L02898-4v}\edtext{abthut\pwindex{DAnnunzio s »Gioconda«1899-11-14@\emph{D’Annunzio’s »Gioconda«} {[}1899-11-14{]}|pwv}}{\lemma{\textnormal{\emph{abthut}}}\Cendnote{\textnormal{Eleonora Duse\pwindex{Duse, Eleonora 03.10.1858 – 21.04.1924@\textsc{Duse, Eleonora} (03.10.1858 – 21.04.1924), \emph{Schauspielerin}|pwk} war zwischen 10. und 20. 11. 1899 im
                  Zuge eines Gastspiels am \emph{Raimund-Theater}\orgindex{Raimund-Theater@Raimund-Theater|pwk}
                  aufgetreten. Eine Rezension von Wassermann\pwindex{Wassermann, Jakob 10.03.1873 – 01.01.1934@\textsc{Wassermann, Jakob} (10.03.1873 – 01.01.1934), \emph{Schriftsteller}|pwk}
                  erschien am 14. 11. 1899, aber deutlich länger als vier
                  Zeilen: rm.\pwindex{Wassermann, Jakob 10.03.1873 – 01.01.1934@\textsc{Wassermann, Jakob} (10.03.1873 – 01.01.1934), \emph{Schriftsteller}|pwk} [= Jakob Wassermann\pwindex{Wassermann, Jakob 10.03.1873 – 01.01.1934@\textsc{Wassermann, Jakob} (10.03.1873 – 01.01.1934), \emph{Schriftsteller}|pwk}]: \emph{D’Annunzio’s »Gioconda«}\pwindex{DAnnunzio s »Gioconda«1899-11-14@\emph{D’Annunzio’s »Gioconda«} {[}1899-11-14{]}|pwk}. In: \emph{Frankfurter Zeitung}\pwindex{?? Werk@Nicht ermittelte Verfasserinnen und Verfasser!Frankfurter Zeitung1856 – 1943@\emph{Frankfurter Zeitung} {[}1856 – 1943{]}|pwk}, Jg. 44, Nr. 316,
                        14. 11. 1899, Abendblatt, S. 1–2.}}}\label{K_L02898-4h},
               dem \textsc{Frankenstein\pwindex{Franckenstein, Clemens von 14.07.1875 – 19.08.1942@\textsc{Franckenstein, Clemens von} (14.07.1875 – 19.08.1942), \emph{Theaterleiter, Komponist, Dirigent}|pw}}-Conzert, \strikeout{über} deſſen Bedeutungsloſigkeit in der
                  Wien\oindex{Wien@\textbf{Wien}|pw}er Conzertfluth klar genug iſt, einen {\pb}ganzen \label{K_L02898-5v}\edtext{Bericht}{\lemma{\textnormal{\emph{Bericht}}}\Cendnote{\textnormal{nicht erschienen. Am 21. 11. 1899 hatte im
                     Kleinen Musikvereinssaal\oindex{Musikverein@\textbf{Musikverein}|pwk} ein
                     »Compositions-Concert mit Orchester Clemens Franckenstein\pwindex{Franckenstein, Clemens von 14.07.1875 – 19.08.1942@\textsc{Franckenstein, Clemens von} (14.07.1875 – 19.08.1942), \emph{Theaterleiter, Komponist, Dirigent}|pw}« stattgefunden. Schnitzler\pwindex{Schnitzler, Arthur 15.05.1862 – 21.10.1931@\textsc{Schnitzler, Arthur} (15.05.1862 – 21.10.1931), \emph{Schriftsteller, Mediziner}|pwk} hatte
                  daran teilgenommen.}}}\label{K_L02898-5h}{ }\strikeout{\textcolor{gray}{wid}} widmet, ſo liegt ohne jeden Zweifel das Beſtreben einer perſönlichen
               Dienſtleiſtung vor, und keine anſtändige Zeitung wird es ſich von einem Herrn \textsc{Wassermann\pwindex{Wassermann, Jakob 10.03.1873 – 01.01.1934@\textsc{Wassermann, Jakob} (10.03.1873 – 01.01.1934), \emph{Schriftsteller}|pw}} gefallen laſſen, daß er, der ſonſt ſo ſäumig in ſeinen dienſtlichen
               Obliegenheiten ſich zeigt, gleich mit der Feder bei der Hand iſt, wenn es gilt, einem
                  Bekannten\pwindex{Franckenstein, Clemens von 14.07.1875 – 19.08.1942@\textsc{Franckenstein, Clemens von} (14.07.1875 – 19.08.1942), \emph{Theaterleiter, Komponist, Dirigent}|pwv} eine Reklame zu
               machen.\pend
           \pstart
           An \textsc{Schwartzkopf\pwindex{Schwarzkopf, Gustav 07.11.1853 – 13.11.1939@\textsc{Schwarzkopf, Gustav} (07.11.1853 – 13.11.1939), \emph{Schriftsteller}|pw}} werde ich keinen liebenswürdigen Brief ſchreiben. Ich ſchätze und verehre ihn,
               wie Du weißt. Aber \textsc{Hirschfeld\pwindex{Hirschfeld, Robert 17.09.1857 – 02.04.1914@\textsc{Hirschfeld, Robert} (17.09.1857 – 02.04.1914), \emph{Journalist, Musikkritiker}|pw}} ſteht mir näher und iſt auch ohne jeden Zweifel in ſeiner ganzen Art
               geeigneter, die Berichterſtattung für die »Frankfurter
                  Zeitung\orgindex{Frankfurter Zeitung@Frankfurter Zeitung|pw}« zu übernehmen, obwohl \textsc{Schwarzkopf\pwindex{Schwarzkopf, Gustav 07.11.1853 – 13.11.1939@\textsc{Schwarzkopf, Gustav} (07.11.1853 – 13.11.1939), \emph{Schriftsteller}|pw}} ſicherlich ſeine Sache auch ſehr gut machen würde. Immerhin {\pb}habe ich für \textsc{Schwarzkopf\pwindex{Schwarzkopf, Gustav 07.11.1853 – 13.11.1939@\textsc{Schwarzkopf, Gustav} (07.11.1853 – 13.11.1939), \emph{Schriftsteller}|pw}} gewirkt, weil ich meinte, damit etwas \label{K_L02898-11v}\edtext{Dir zu Liebe}{\lemma{\textnormal{\emph{Dir zu Liebe}}}\Cendnote{\textnormal{siehe Paul Goldmann an Arthur Schnitzler, 6. 12. [1899]}}}\label{K_L02898-11h} zu thun. Im Augenblick wo Du das ablehnſt, verliert die Angelegenheit alles
               Intereſſe für mich, und ich werde mich fortan jeder Einwirkung enthalten.\pend
           \pstart
           Viele treue Grüße! {\\[\baselineskip]}Dein {\\[\baselineskip]}\spacefill\mbox{Paul Goldmann.}\pend
           \leftskip=0em{}
         
         \endnumbering\mylabel{h}\end{ledgroupsized}  \newcommand{\dateiname}{L02898}\newcommand{\titel}{Paul Goldmann an Arthur Schnitzler, 11. 12. [1899]}\newcommand{\editorInnen}{Martin Anton Müller und Laura Untner}%% latex-leseansicht-abspann.tex
%% Abspann für die Leseansicht.
%% Der Schalter \ifkorrekturansicht ist bereits durch den Vorspann gesetzt.

%% latex-abspann.tex
%% Gemeinsamer Abspann für Korrekturansicht und Leseansicht.
%% Setzt den Schalter \ifkorrekturansicht voraus (gesetzt in den
%% einbindenden Dateien latex-korrekturansicht-abspann.tex bzw.
%% latex-leseansicht-abspann.tex).
%% ---------------------------------------------------------------

\normalsize

% Das esempio-Environment wird nur in der Leseansicht benötigt
\ifkorrekturansicht\else
\newenvironment{esempio}[3]%
{
    \vspace{1.5ex}
    \rlap{\underline{#1}}
    \par
    \setlength{\parindent}{0cm}
    \nopagebreak
    \leftskip=#2cm
    \rightskip=#3cm
}
{
    \par
}
\fi

\doendnotes{C}
\bigskip
\vfill

\clearpage

\footnotesize

\ifkorrekturansicht
  \lohead{\textsc{register}}
\fi

% theindex-Environment neu definieren ohne reledmac
\makeatletter
\renewenvironment{theindex}{%
  \ifkorrekturansicht
    \section*{\indexname}%
  \else
    \subsubsection*{Index der erwähnten Entitäten}%
  \fi
  \setlength{\parindent}{0pt}%
  \setlength{\parskip}{0pt plus 0.3pt}%
  \let\item\@idxitem
}{%
  \ifkorrekturansicht\clearpage\fi
}
\makeatother

\IfFileExists{\jobname-pw.ind}{\input{\jobname-pw.ind}}{}

% Quellenangabe nur in der Leseansicht
\ifkorrekturansicht\else
% Fallback-Definitionen, falls die .tex-Datei \titel etc. nicht gesetzt hat
\providecommand{\titel}{}
\providecommand{\editorInnen}{}
\providecommand{\dateiname}{\jobname}

\vspace{3cm}

\vfill

\footnotesize
\textsc{Quelle}: \titel. Herausgegeben von {\editorInnen}. In: \emph{Arthur Schnitzler: Briefwechsel mit Autorinnen und Autoren}.
 Digitale Edition, https://schnitzler-briefe.acdh.oeaw.ac.at/{\dateiname}.html (Stand \today)
\fi

\end{document}


      