%% latex-korrekturansicht-vorspann.tex
%% Vorspann für die Korrekturansicht.
%% Lädt die gemeinsame Datei latex-vorspann.tex mit gesetztem Schalter.

\newif\ifkorrekturansicht
\korrekturansichttrue

\input{../tex-inputs/latex-vorspann}


\section[ Paul Goldmann an Arthur Schnitzler, 11. 12. {[}1899{]}]{L02898 Paul Goldmann an Arthur Schnitzler, 11. 12. {[}1899{]}}
\nopagebreak\mylabel{L02898v}
\rehead{ }\normalsize\beginnumbering\briefempfaengerindex{Schnitzler, Arthur@\textsc{Schnitzler, Arthur}!zzzGoldmann, Paul@\emph{von Paul Goldmann}!1899-12-111@{11. 12. {[}1899{]}}|(be}
\toendnotes[C]{\smallbreak\pagebreak[2]}\Standort{DLA, A:Schnitzler, HS.NZ85.1.3169.}
\physDesc{Brief, 1 Blatt, 4 Seiten, 2328 Zeichen
\newline{}Handschrift: blaue Tinte, deutsche Kurrent
\newline{}Schnitzler: 1) mit Bleistift das Jahr »99« vermerkt  2) mit rotem Buntstift sieben Unterstreichungen}\toendnotes[C]{\smallbreak}
\pstart
           \raggedleft{}{\pb}Frankfurt\oindex{Frankfurt am Main@\textbf{Frankfurt am Main}, \emph{P.PPLA3}|pw}, 11. Dezember.\pend
           
\pstart\center{}Mein lieber Freund,\pend\vspace{0.5em}
\pstart
           Vielen Dank für Deine intereſſanten Mittheilungen! Daß \label{K_L02898-1v}\edtext{\textsc{Bahr\pwindex{Bahr, Hermann 19.07.1863 – 15.01.1934@\textsc{Bahr, Hermann} (19.07.1863 – 15.01.1934), \emph{Schriftsteller/Schriftstellerin, Kritiker/Kritikerin}|pw}} gegen Dein Stück\pwindex{gruene Kakadu. Groteske in einem Akt@\emph{Der grüne Kakadu. Groteske in einem Akt}|pwv}
                  intriguirt}{\lemma{\textnormal{\emph{Bahr … intriguirt}}}\Cendnote{\textnormal{Vermutlich ist diese Stelle
                  so zu lesen, dass Bahr\pwindex{Bahr, Hermann 19.07.1863 – 15.01.1934@\textsc{Bahr, Hermann} (19.07.1863 – 15.01.1934), \emph{Schriftsteller/Schriftstellerin, Kritiker/Kritikerin}|pwk} sich dagegen
                  gewendet hatte, dass \emph{Der grüne Kakadu}\pwindex{gruene Kakadu. Groteske in einem Akt@\emph{Der grüne Kakadu. Groteske in einem Akt}|pwk} wieder
                  auf den Spielplan des \emph{Burgtheaters}\orgindex{Burgtheater@Burgtheater|pwk} gesetzt
                  wurde (siehe Paul Goldmann an Arthur Schnitzler, 12. 11. [1899]).}}}\label{K_L02898-1}, iſt
               ein Zug, der ganz zum Charakterbilde dieſes Burſchen\pwindex{Bahr, Hermann 19.07.1863 – 15.01.1934@\textsc{Bahr, Hermann} (19.07.1863 – 15.01.1934), \emph{Schriftsteller/Schriftstellerin, Kritiker/Kritikerin}|pwv} paßt. Wenn \textsc{Schlenther\pwindex{Schlenther, Paul 20.08.1854 – 30.04.1916@\textsc{Schlenther, Paul} (20.08.1854 – 30.04.1916), \emph{Schriftsteller/Schriftstellerin, Kritiker/Kritikerin, Theaterleiter/Theaterleiterin}|pw}} Dich auf die \label{K_L02898-2v}\edtext{Aufführung Deiner
               zwei Einakter\pwindex{Paracelsus. Versspiel in einem Akt@\emph{Paracelsus. Versspiel in einem Akt}|pwv}\pwindex{Gefaehrtin. Schauspiel in einem Akt@\emph{Die Gefährtin. Schauspiel in einem Akt}|pwv}}{\lemma{\textnormal{\emph{Aufführung … Einakter}}}\Cendnote{\textnormal{Während die Aufführung von \emph{Der grüne Kakadu}\pwindex{gruene Kakadu. Groteske in einem Akt@\emph{Der grüne Kakadu. Groteske in einem Akt}|pwk} verboten blieb, wurden die
                  zwei anderen Einakter\pwindex{Paracelsus. Versspiel in einem Akt@\emph{Paracelsus. Versspiel in einem Akt}|pwkv}\pwindex{Gefaehrtin. Schauspiel in einem Akt@\emph{Die Gefährtin. Schauspiel in einem Akt}|pwkv} des Zyklus\pwindex{gruene Kakadu – Paracelsus – Die Gefaehrtin. Drei Einakter@\emph{Der grüne Kakadu – Paracelsus – Die Gefährtin. Drei Einakter}|pwkv} – \emph{Paracelsus}\pwindex{Paracelsus. Versspiel in einem Akt@\emph{Paracelsus. Versspiel in einem Akt}|pwk} und \emph{Die Gefährtin}\pwindex{Gefaehrtin. Schauspiel in einem Akt@\emph{Die Gefährtin. Schauspiel in einem Akt}|pwk} – auch weiterhin gegeben.}}}\label{K_L02898-2}
               warten läßt, ſo rächt er ſich, nach Art gemeiner Naturen, für die \label{K_L02898-3v}\edtext{Demüthigung}{\lemma{\textnormal{\emph{Demüthigung}}}\Cendnote{\textnormal{Womöglich Bezug auf die Kommentare der Presse hinsichtlich
                  der Absetzung von \emph{Der grüne Kakadu}\pwindex{gruene Kakadu. Groteske in einem Akt@\emph{Der grüne Kakadu. Groteske in einem Akt}|pwk}, beispielsweise\pwindex{Burgtheater. [Die Absetzung von Der gruene Kakadu]@\emph{Burgtheater. [Die Absetzung von Der grüne Kakadu]}|pwkv} am 21. 12. 1899: »\so{Schnitzler}’s ›\so{Grüner Kakadu}\pwindex{gruene Kakadu. Groteske in einem Akt@\emph{Der grüne Kakadu. Groteske in einem Akt}|pw}‹, der sonst immer nach ›Paracelsus\pwindex{Paracelsus. Versspiel in einem Akt@\emph{Paracelsus. Versspiel in einem Akt}|pw}‹
                     und der ›Gefährtin\pwindex{Gefaehrtin. Schauspiel in einem Akt@\emph{Die Gefährtin. Schauspiel in einem Akt}|pw}‹ folgte, ist, wie man
                     hört, aus dem Spielplan des Burgtheater\orgindex{Burgtheater@Burgtheater|pw}
                     gestrichen. Allerlei Einflüsse allerlei höfischer Kreise sollen dies bewirkt
                     haben. Schade, daß Herr Direktor \so{Schlenther}\pwindex{Schlenther, Paul 20.08.1854 – 30.04.1916@\textsc{Schlenther, Paul} (20.08.1854 – 30.04.1916), \emph{Schriftsteller/Schriftstellerin, Kritiker/Kritikerin, Theaterleiter/Theaterleiterin}|pw} das nun einmal von der Zensur\orgindex{K. u. k. Zensurstelle@K. u. k. Zensurstelle|pwv} der Hofbühnen genehmigte Stück\pwindex{gruene Kakadu. Groteske in einem Akt@\emph{Der grüne Kakadu. Groteske in einem Akt}|pwv} trotz aller Einflüsse nicht doch gegeben hat. Wir
                     können diese allzu große Nachgiebigkeit gegen gewisse Strömungen nicht
                     billigen. Ist es aber einmal entschieden, daß der ›Grüne Kakadu\pwindex{gruene Kakadu. Groteske in einem Akt@\emph{Der grüne Kakadu. Groteske in einem Akt}|pw}‹ nicht mehr auf dem Burgtheater\orgindex{Burgtheater@Burgtheater|pw} erscheinen soll, dann ist zu wünschen, daß
                     wir ihm bald auf einer anderen Bühne (etwa dem Deutschen Volkstheater\orgindex{Volkstheater@Volkstheater|pw}) wieder begegnen.« (\emph{Arbeiter-Zeitung}\pwindex{Arbeiter-Zeitung@\emph{Arbeiter-Zeitung}|pwk}, Jg. 11, Nr. 351,
                        21. 12. 1899, Morgenblatt, S. 8.)}}}\label{K_L02898-3}, die er im
               Streit mit Dir über den »Kakadu\pwindex{gruene Kakadu. Groteske in einem Akt@\emph{Der grüne Kakadu. Groteske in einem Akt}|pw}« erlitten.\pend
           
\pstart
           Im \label{K_L02898-4v}\edtext{Falle \textsc{Wassermann\pwindex{Wassermann, Jakob 10.03.1873 – 01.01.1934@\textsc{Wassermann, Jakob} (10.03.1873 – 01.01.1934), \emph{Schriftsteller/Schriftstellerin}|pw}}}{\lemma{\textnormal{\emph{Falle Wassermann}}}\Cendnote{\textnormal{Siehe Paul Goldmann an Arthur Schnitzler, 26. 10. 1899, 6. 12. [1899] und 23. 12. [1899]. Die im
                  folgenden skizzierte Kritik Wassermanns\pwindex{Wassermann, Jakob 10.03.1873 – 01.01.1934@\textsc{Wassermann, Jakob} (10.03.1873 – 01.01.1934), \emph{Schriftsteller/Schriftstellerin}|pwk}
                  über Eugen D’Albert\pwindex{DAlbert, Eugen 10.04.1864 – 03.03.1932@\textsc{d’Albert, Eugen} (10.04.1864 – 03.03.1932), \emph{Komponist/Komponistin}|pwk} und Clemens Frankenstein\pwindex{Franckenstein, Clemens von 14.07.1875 – 19.08.1942@\textsc{Franckenstein, Clemens von} (14.07.1875 – 19.08.1942), \emph{Theaterleiter/Theaterleiterin, Komponist/Komponistin, Dirigent/Dirigentin}|pwk} konnte nicht nachgewiesen werden und
                  dürfte nie gedruckt worden sein.}}}\label{K_L02898-4}, in welchem, wie Du ſagſt, die »Frankfurter Zeitung\orgindex{Frankfurter Zeitung@Frankfurter Zeitung|pw}« durchaus im Unrecht iſt, iſt
               die »Frankfurter Zeitung\orgindex{Frankfurter Zeitung@Frankfurter Zeitung|pw}« durchaus im Recht. \textsc{D’Albert\pwindex{DAlbert, Eugen 10.04.1864 – 03.03.1932@\textsc{d’Albert, Eugen} (10.04.1864 – 03.03.1932), \emph{Komponist/Komponistin}|pw}’s} Compoſitionen ſind
               mittelmäßige Leiſtungen. Das wiſſen wir hier und das hat \strikeout{\textcolor{gray}{e}b\textcolor{gray}{en}ſ} Niemand beſtritten. \textsc{Frankensteins\pwindex{Franckenstein, Clemens von 14.07.1875 – 19.08.1942@\textsc{Franckenstein, Clemens von} (14.07.1875 – 19.08.1942), \emph{Theaterleiter/Theaterleiterin, Komponist/Komponistin, Dirigent/Dirigentin}|pw}} Compoſitionen ſind {\pb}ebenfalls mittelmäßige Leiſtungen, die ſich
               vielleicht auf demſelben Niveau, eher ſogar ein wenig tiefer halten. Es geht aber
               abſolut nicht an, in derſelben Kritik \textsc{d’Albert\pwindex{DAlbert, Eugen 10.04.1864 – 03.03.1932@\textsc{d’Albert, Eugen} (10.04.1864 – 03.03.1932), \emph{Komponist/Komponistin}|pw}} ganz zu verwerfen, \textsc{Frankenstein\pwindex{Franckenstein, Clemens von 14.07.1875 – 19.08.1942@\textsc{Franckenstein, Clemens von} (14.07.1875 – 19.08.1942), \emph{Theaterleiter/Theaterleiterin, Komponist/Komponistin, Dirigent/Dirigentin}|pw}} hingegen ihm gegenüber zu loben, mag das Lob noch ſo eingeſchränkt ſein.
               Namentlich in dieſer Zuſammenſtellung liegt die Fälſchung des Urtheils. Und wenn
               dieſe Kritik noch dazu von einem Mitarbeiter\pwindex{Wassermann, Jakob 10.03.1873 – 01.01.1934@\textsc{Wassermann, Jakob} (10.03.1873 – 01.01.1934), \emph{Schriftsteller/Schriftstellerin}|pwv} eingeſandt wird, der ſeine Berichterſtattung bisher ſtets in
               einer ans Gewiſſenloſe grenzenden Weiſe vernachläſſigt hat, – wenn derſelbe Berichterſtatter\pwindex{Wassermann, Jakob 10.03.1873 – 01.01.1934@\textsc{Wassermann, Jakob} (10.03.1873 – 01.01.1934), \emph{Schriftsteller/Schriftstellerin}|pwv}, der die
               Aufführungen der \textsc{Duse\pwindex{Duse, Eleonora 03.10.1858 – 21.04.1924@\textsc{Duse, Eleonora} (03.10.1858 – 21.04.1924), \emph{Schauspieler/Schauspielerin}|pw}} mit vier \strikeout{Z\textcolor{gray}{ei}l}{ } Zeilen{ }\label{K_L02898-5v}\edtext{abthut\pwindex{DAnnunzio s »Gioconda«@\emph{D’Annunzio’s »Gioconda«}|pwv}}{\lemma{\textnormal{\emph{abthut}}}\Cendnote{\textnormal{Eleonora Duse\pwindex{Duse, Eleonora 03.10.1858 – 21.04.1924@\textsc{Duse, Eleonora} (03.10.1858 – 21.04.1924), \emph{Schauspieler/Schauspielerin}|pwk} war zwischen 10. und 20. 11. 1899 im
                  Zuge eines Gastspiels am \emph{Raimund-Theater}\orgindex{Raimund-Theater@Raimund-Theater|pwk}
                  aufgetreten. Eine Rezension von Wassermann\pwindex{Wassermann, Jakob 10.03.1873 – 01.01.1934@\textsc{Wassermann, Jakob} (10.03.1873 – 01.01.1934), \emph{Schriftsteller/Schriftstellerin}|pwk}
                  erschien am 14. 11. 1899, aber sie war deutlich länger als vier Zeilen: rm.\pwindex{Wassermann, Jakob 10.03.1873 – 01.01.1934@\textsc{Wassermann, Jakob} (10.03.1873 – 01.01.1934), \emph{Schriftsteller/Schriftstellerin}|pwk} [ = Jakob Wassermann\pwindex{Wassermann, Jakob 10.03.1873 – 01.01.1934@\textsc{Wassermann, Jakob} (10.03.1873 – 01.01.1934), \emph{Schriftsteller/Schriftstellerin}|pwk}]: \emph{D’Annunzio’s »Gioconda«}\pwindex{DAnnunzio s »Gioconda«@\emph{D’Annunzio’s »Gioconda«}|pwk}. In: \emph{Frankfurter Zeitung}\pwindex{Frankfurter Zeitung@\emph{Frankfurter Zeitung}|pwk}, Jg. 44, Nr. 316, 14. 11. 1899,
                     Abendblatt, S. 1–2.}}}\label{K_L02898-5}, dem \textsc{Frankenstein\pwindex{Franckenstein, Clemens von 14.07.1875 – 19.08.1942@\textsc{Franckenstein, Clemens von} (14.07.1875 – 19.08.1942), \emph{Theaterleiter/Theaterleiterin, Komponist/Komponistin, Dirigent/Dirigentin}|pw}}-Conzert, \strikeout{über} deſſen Bedeutungsloſigkeit in der
                  Wien\oindex{Wien@\textbf{Wien}, \emph{A.ADM2}|pw}er Conzertfluth klar genug iſt, einen {\pb}ganzen \label{K_L02898-6v}\edtext{Bericht}{\lemma{\textnormal{\emph{Bericht}}}\Cendnote{\textnormal{Nicht erschienen. Am 21. 11. 1899 hatte im
                     Kleinen Musikvereinssaal\oindex{Musikverein@\textbf{Musikverein}, \emph{Konzertsaal (K.KNZ)}|pwk} ein
                     »Compositions-Concert mit Orchester Clemens Franckenstein\pwindex{Franckenstein, Clemens von 14.07.1875 – 19.08.1942@\textsc{Franckenstein, Clemens von} (14.07.1875 – 19.08.1942), \emph{Theaterleiter/Theaterleiterin, Komponist/Komponistin, Dirigent/Dirigentin}|pw}« stattgefunden. Schnitzler hatte
                  daran teilgenommen.}}}\label{K_L02898-6}{ }\strikeout{\textcolor{gray}{wid}} widmet, ſo liegt ohne jeden Zweifel das Beſtreben einer perſönlichen
               Dienſtleiſtung vor, und keine anſtändige Zeitung wird es ſich von einem Herrn \textsc{Wassermann\pwindex{Wassermann, Jakob 10.03.1873 – 01.01.1934@\textsc{Wassermann, Jakob} (10.03.1873 – 01.01.1934), \emph{Schriftsteller/Schriftstellerin}|pw}} gefallen laſſen, daß er, der ſonſt ſo ſäumig in ſeinen dienſtlichen
               Obliegenheiten ſich zeigt, gleich mit der Feder bei der Hand iſt, wenn es gilt, einem
                  Bekannten\pwindex{Franckenstein, Clemens von 14.07.1875 – 19.08.1942@\textsc{Franckenstein, Clemens von} (14.07.1875 – 19.08.1942), \emph{Theaterleiter/Theaterleiterin, Komponist/Komponistin, Dirigent/Dirigentin}|pwv} eine Reklame zu
               machen.\pend
           
\pstart
           An \textsc{Schwartzkopf\pwindex{Schwarzkopf, Gustav 07.11.1853 – 13.11.1939@\textsc{Schwarzkopf, Gustav} (07.11.1853 – 13.11.1939), \emph{Schriftsteller/Schriftstellerin}|pw}} werde ich keinen liebenswürdigen Brief ſchreiben. Ich ſchätze und verehre ihn,
               wie Du weißt. Aber \textsc{Hirschfeld\pwindex{Hirschfeld, Robert 17.09.1857 – 02.04.1914@\textsc{Hirschfeld, Robert} (17.09.1857 – 02.04.1914), \emph{Journalist/Journalistin, Musikkritiker/Musikkritikerin}|pw}} ſteht mir näher und iſt auch ohne jeden Zweifel in ſeiner ganzen Art
               geeigneter, die Berichterſtattung für die »Frankfurter
                  Zeitung\orgindex{Frankfurter Zeitung@Frankfurter Zeitung|pw}« zu übernehmen, obwohl \textsc{Schwarzkopf\pwindex{Schwarzkopf, Gustav 07.11.1853 – 13.11.1939@\textsc{Schwarzkopf, Gustav} (07.11.1853 – 13.11.1939), \emph{Schriftsteller/Schriftstellerin}|pw}} ſicherlich ſeine Sache auch ſehr gut machen würde. Immerhin {\pb}habe ich für \textsc{Schwarzkopf\pwindex{Schwarzkopf, Gustav 07.11.1853 – 13.11.1939@\textsc{Schwarzkopf, Gustav} (07.11.1853 – 13.11.1939), \emph{Schriftsteller/Schriftstellerin}|pw}} gewirkt, weil ich meinte, damit etwas \label{K_L02898-7v}\edtext{Dir zu Liebe}{\lemma{\textnormal{\emph{Dir zu Liebe}}}\Cendnote{\textnormal{Siehe Paul Goldmann an Arthur Schnitzler, 6. 12. [1899].
               }}}\label{K_L02898-7} zu thun. Im Augenblick wo Du das ablehnſt, verliert die Angelegenheit alles
               Intereſſe für mich, und ich werde mich fortan jeder Einwirkung enthalten.\pend
           
\pstart
           Viele treue Grüße! {\\[\baselineskip]}Dein {\\[\baselineskip]}\spacefill\mbox{Paul Goldmann.}\pend
           \leftskip=0em{}\selectlanguage{ngerman}\endnumbering\briefempfaengerindex{Schnitzler, Arthur@\textsc{Schnitzler, Arthur}!zzzGoldmann, Paul@\emph{von Paul Goldmann}!1899-12-111@{11. 12. {[}1899{]}}|)be}\mylabel{L02898h}  \normalsize

\doendnotes{C}
\bigskip
\vfill

\clearpage

\footnotesize

\lohead{\textsc{register}}

% Definiere theindex-Environment komplett neu ohne reledmac
\makeatletter
\renewenvironment{theindex}{%
  \section*{\indexname}%
  \setlength{\parindent}{0pt}%
  \setlength{\parskip}{0pt plus 0.3pt}%
  \let\item\@idxitem
}{%
  \clearpage
}
\makeatother

\IfFileExists{\jobname-pw.ind}{\input{\jobname-pw.ind}}{}

\end{document}

      