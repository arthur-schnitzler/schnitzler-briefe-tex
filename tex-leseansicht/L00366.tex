%% latex-korrekturansicht-vorspann.tex
%% Vorspann für die Korrekturansicht.
%% Lädt die gemeinsame Datei latex-vorspann.tex mit gesetztem Schalter.

\newif\ifkorrekturansicht
\korrekturansichttrue

\input{../tex-inputs/latex-vorspann}


\section[Richard Beer-Hofmann an Arthur Schnitzler, {[}5./6.? 9. 1894{]}]{L00366 Richard Beer-Hofmann an Arthur Schnitzler, {[}5./6.? 9. 1894{]}}
\nopagebreak\mylabel{L00366v}
\rehead{ }\normalsize\beginnumbering\briefempfaengerindex{Schnitzler, Arthur@\textsc{Schnitzler, Arthur}!zzzBeer-Hofmann, Richard@\emph{von Richard Beer-Hofmann}!1894-09-061@{{[}5./6.? 9. 1894{]}}|(be}
\toendnotes[C]{\smallbreak\pagebreak[2]}\Standort{CUL, Schnitzler, B 8.}
\physDesc{Brief, 1 Blatt, 2 Seiten, 241 Zeichen
\newline{}Handschrift: blauer Buntstift, lateinische Kurrent
\newline{}Schnitzler: mit Bleistift datiert: »July Sept. 94.« und nummeriert:
                                    »37« }
\buchAbdrucke{\weitereDrucke{Arthur Schnitzler, Richard Beer-Hofmann: \emph{Briefwechsel 1891–1931}. Wien, Zürich: \emph{Europaverlag} 1992, S. 59.} }\toendnotes[C]{\smallbreak}
\pstart
           \noindent{}{\pb}Lieber Arthur! Ich bin
               nicht hier in Wien\oindex{Wien@\textbf{Wien}, \emph{A.ADM2}|pw} – nur Ihr Stock ist hier – ich
               bin hoffentlich auf der \label{K_L00366-1v}\edtext{Route}{\lemma{\textnormal{\emph{Route}}}\Cendnote{\textnormal{Mehrere Hinweise erlauben das Einordnen
                  dieses Korrespondenzstücks. Am 4. 9. 1894 reiste Schnitzler von Bad Ischl\oindex{Bad Ischl@\textbf{Bad Ischl}, \emph{P.PPL}|pwk}
                  nach Wien\oindex{Wien@\textbf{Wien}, \emph{A.ADM2}|pwk}. Zwei Tage zuvor hatten er Beer-Hofmann\pwindex{Beer-Hofmann, Richard 1866-07-11 – 1945-09-26@\textsc{Beer-Hofmann, Richard} (1866-07-11 – 1945-09-26), \emph{Schriftsteller/Schriftstellerin}|pwk} zuletzt gesehen. Offenbar
                  nahm Schnitzler an, jener wäre ebenfalls in Wien\oindex{Wien@\textbf{Wien}, \emph{A.ADM2}|pwk}. Da
                     Beer-Hofmann\pwindex{Beer-Hofmann, Richard 1866-07-11 – 1945-09-26@\textsc{Beer-Hofmann, Richard} (1866-07-11 – 1945-09-26), \emph{Schriftsteller/Schriftstellerin}|pwk} aber am
                     7. 9. 1894 bereits wieder aus Bad
                     Ischl\oindex{Bad Ischl@\textbf{Bad Ischl}, \emph{P.PPL}|pwk} schreibt – was den Inhalt dieses Briefes obsolet werden ließe –
                  dürfte das Schreiben frühestens am 5. und spätestens am
                     6. 9. 1895 verfasst worden sein.}}}\label{K_L00366-1} nach Italien\oindex{Italien@\textbf{Italien}, \emph{A.PCLI}|pw}, momentan – {\pb}da ich dies schreibe, – friere ich
               in Ischl\oindex{Bad Ischl@\textbf{Bad Ischl}, \emph{P.PPL}|pw}, – hier. Dieser Brief ist unanständig
               wegen der vielen »hier«.\pend
           \pstart Herzlichst Ihr \spacefill\mbox{R}\pend{}\selectlanguage{ngerman}\endnumbering\briefempfaengerindex{Schnitzler, Arthur@\textsc{Schnitzler, Arthur}!zzzBeer-Hofmann, Richard@\emph{von Richard Beer-Hofmann}!1894-09-051@{{[}5./6.? 9. 1894{]}}|)be}\mylabel{L00366h}  \normalsize

\doendnotes{C}
\bigskip
\vfill

\clearpage

\footnotesize

\lohead{\textsc{register}}

% Definiere theindex-Environment komplett neu ohne reledmac
\makeatletter
\renewenvironment{theindex}{%
  \section*{\indexname}%
  \setlength{\parindent}{0pt}%
  \setlength{\parskip}{0pt plus 0.3pt}%
  \let\item\@idxitem
}{%
  \clearpage
}
\makeatother

\IfFileExists{\jobname-pw.ind}{\input{\jobname-pw.ind}}{}

\end{document}

      