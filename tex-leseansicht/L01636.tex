%% latex-leseansicht-vorspann.tex
%% Vorspann für die Leseansicht.
%% Lädt die gemeinsame Datei latex-vorspann.tex mit nicht gesetztem Schalter.

\newif\ifkorrekturansicht
\korrekturansichtfalse

\input{../tex-inputs/latex-vorspann}


\section[Max Mell an Arthur Schnitzler, 7. 11. 1906]{L01636 Max Mell an Arthur Schnitzler, 7. 11. 1906}
\nopagebreak\mylabel{L01636v}
\rehead{ }\normalsize\beginnumbering\briefempfaengerindex{Schnitzler, Arthur@\textsc{Schnitzler, Arthur}!zzzMell, Max@\emph{von Max Mell}!1906-11-071@{7. 11. 1906}|(be}
\toendnotes[C]{\smallbreak\pagebreak[2]}
\correspDesc{Versand  durch Max Mell am 7. 11. 1906 in Wien
\newline{}Erhalt  durch Arthur Schnitzler im Zeitraum [7. 11. 1906
                  – 11. 11. 1906?] in Wien}\toendnotes[C]{\smallbreak}
\Standort{TMW, HS Schn 3/74.}
\physDesc{Brief, 1 Blatt, 1 Seite, 682 Zeichen
\newline{}Handschrift: schwarze Tinte, deutsche Kurrent
\newline{}Schnitzler: mit rotem Buntstift zwei Unterstreichungen }\toendnotes[C]{\smallbreak}
\pstart
           \raggedleft{}{\pb}7. November 1906.\pend
           
\pstart{}Sehr verehrter Herr Doktor,\pend\vspace{0.5em}
\pstart
           Ihre Anſicht über mein Stück\pwindex{Mell, Max 10.\,11.\,1882 Maribor – 13.\,12.\,1971 Wien@\textsc{Mell, Max} (10.\,11.\,1882 Maribor – 13.\,12.\,1971 Wien), \emph{Schriftsteller}!Komödianten@\strich\emph{Die Komödianten}|pwv}
               iſt mir in jeder Hinſicht teuer und ich danke Ihnen dafür, daß Sie mir{ }ſie{ }ſagen. Ich
               kann alle Schritte für eine Aufführung aber durchaus mit innerer Ruhe tun, weil ich{ }ſelbſt jene Diſtanz zu dem Stück\pwindex{Mell, Max 10.\,11.\,1882 Maribor – 13.\,12.\,1971 Wien@\textsc{Mell, Max} (10.\,11.\,1882 Maribor – 13.\,12.\,1971 Wien), \emph{Schriftsteller}!Komödianten@\strich\emph{Die Komödianten}|pwv} noch nicht habe, die mir erforderlich{ }ſcheint, Ihrer Wertung in allem
               beizuſtimmen. Nach dem, was ich an mir erfuhr, geht aber wahrſcheinlich mein Weg
               dorthin, und es iſt möglich, daß ich Ihre Worte zu den meinen machen werde,{ }ſobald
               ich ein neues Stück geſchrieben habe oder die »Komödianten\pwindex{Mell, Max 10.\,11.\,1882 Maribor – 13.\,12.\,1971 Wien@\textsc{Mell, Max} (10.\,11.\,1882 Maribor – 13.\,12.\,1971 Wien), \emph{Schriftsteller}!Komödianten@\strich\emph{Die Komödianten}|pw}« geſpielt{ }ſehe. Der Weg über das neue Stück wäre mir lieber.\pend
           
\pstart
           Ich bin, in aufrichtiger Verehrung,{\\[\baselineskip]}Ihr ergebener{\\[\baselineskip]}\spacefill\mbox{Max Mell.}\pend
           \leftskip=0em{}\selectlanguage{ngerman}\endnumbering\briefempfaengerindex{Schnitzler, Arthur@\textsc{Schnitzler, Arthur}!zzzMell, Max@\emph{von Max Mell}!1906-11-071@{7. 11. 1906}|)be}\mylabel{L01636h}  \newcommand{\dateiname}{L01636}\newcommand{\titel}{Max Mell an Arthur Schnitzler, 7. 11. 1906}\newcommand{\editorInnen}{Martin Anton Müller und Gerd-Hermann Susen}%% latex-leseansicht-abspann.tex
%% Abspann für die Leseansicht.
%% Der Schalter \ifkorrekturansicht ist bereits durch den Vorspann gesetzt.

%% latex-abspann.tex
%% Gemeinsamer Abspann für Korrekturansicht und Leseansicht.
%% Setzt den Schalter \ifkorrekturansicht voraus (gesetzt in den
%% einbindenden Dateien latex-korrekturansicht-abspann.tex bzw.
%% latex-leseansicht-abspann.tex).
%% ---------------------------------------------------------------

\normalsize

% Das esempio-Environment wird nur in der Leseansicht benötigt
\ifkorrekturansicht\else
\newenvironment{esempio}[3]%
{
    \vspace{1.5ex}
    \rlap{\underline{#1}}
    \par
    \setlength{\parindent}{0cm}
    \nopagebreak
    \leftskip=#2cm
    \rightskip=#3cm
}
{
    \par
}
\fi

\doendnotes{C}
\bigskip
\vfill

\clearpage

\footnotesize

\ifkorrekturansicht
  \lohead{\textsc{register}}
\fi

% theindex-Environment neu definieren ohne reledmac
\makeatletter
\renewenvironment{theindex}{%
  \ifkorrekturansicht
    \section*{\indexname}%
  \else
    \subsubsection*{Index der erwähnten Entitäten}%
  \fi
  \setlength{\parindent}{0pt}%
  \setlength{\parskip}{0pt plus 0.3pt}%
  \let\item\@idxitem
}{%
  \ifkorrekturansicht\clearpage\fi
}
\makeatother

\IfFileExists{\jobname-pw.ind}{\input{\jobname-pw.ind}}{}

% Quellenangabe nur in der Leseansicht
\ifkorrekturansicht\else
% Fallback-Definitionen, falls die .tex-Datei \titel etc. nicht gesetzt hat
\providecommand{\titel}{}
\providecommand{\editorInnen}{}
\providecommand{\dateiname}{\jobname}

\vspace{3cm}

\vfill

\footnotesize
\textsc{Quelle}: \titel. Herausgegeben von {\editorInnen}. In: \emph{Arthur Schnitzler: Briefwechsel mit Autorinnen und Autoren}.
 Digitale Edition, https://schnitzler-briefe.acdh.oeaw.ac.at/{\dateiname}.html (Stand \today)
\fi

\end{document}


