%% latex-leseansicht-vorspann.tex
%% Vorspann für die Leseansicht.
%% Lädt die gemeinsame Datei latex-vorspann.tex mit nicht gesetztem Schalter.

\newif\ifkorrekturansicht
\korrekturansichtfalse

\input{../tex-inputs/latex-vorspann}


         \renewcommand{\erwaehnteOrte}{Orte: Wien}
         \renewcommand{\erwaehnteWerke}{Werke: Die Komödianten}
               \section[Max Mell an Arthur Schnitzler, 7. 11. 1906]{ Max Mell an Arthur Schnitzler, 7. 11. 1906}\nopagebreak\mylabel{v}\rehead{ }\begin{ledgroupsized}[t]{13cm}\normalsize\beginnumbering \toendnotes[C]{\smallbreak\pagebreak[2]} \Standort{TMW, HS Schn 3/74.}
\physDesc{Brief, 1 Blatt, 1 Seite
\newline{}Handschrift: schwarze Tinte, deutsche Kurrent
\newline{}Schnitzler: mit rotem Buntstift zwei Unterstreichungen }\toendnotes[C]{\smallbreak}\pstart
           \raggedleft{}{\pb}7. November 1906.\pend
           \pstart{}Sehr verehrter Herr Doktor,\pend\pstart
           Ihre Anſicht über mein Stück\pwindex{Mell, Max 10.11.1882 – 13.12.1971@\textsc{Mell, Max} (10.11.1882 – 13.12.1971), \emph{Schriftsteller}!KomoediantenNone@\strich\emph{Die Komödianten} {[}None{]}|pwv}
                    iſt mir in jeder Hinſicht teuer und ich danke Ihnen dafür, daß Sie mir ſie
                    ſagen. Ich kann alle Schritte für eine Aufführung aber durchaus mit innerer Ruhe
                    tun, weil ich ſelbſt jene Diſtanz zu dem Stück\pwindex{Mell, Max 10.11.1882 – 13.12.1971@\textsc{Mell, Max} (10.11.1882 – 13.12.1971), \emph{Schriftsteller}!KomoediantenNone@\strich\emph{Die Komödianten} {[}None{]}|pwv} noch nicht habe, die mir erforderlich ſcheint,
                    Ihrer Wertung in allem beizuſtimmen. Nach dem, was ich an mir erfuhr, geht aber
                    wahrſcheinlich mein Weg dorthin, und es iſt möglich, daß ich Ihre Worte zu den
                    meinen machen werde, ſobald ich ein neues Stück geſchrieben habe oder die »Komödianten\pwindex{Mell, Max 10.11.1882 – 13.12.1971@\textsc{Mell, Max} (10.11.1882 – 13.12.1971), \emph{Schriftsteller}!KomoediantenNone@\strich\emph{Die Komödianten} {[}None{]}|pw}« geſpielt ſehe. Der Weg über das
                    neue Stück wäre mir lieber.\pend
           \pstart
           Ich bin, in aufrichtiger Verehrung,{\\[\baselineskip]}Ihr ergebener{\\[\baselineskip]}\spacefill\mbox{Max Mell.}\pend
           \leftskip=0em{}
         
         \endnumbering\mylabel{h}\end{ledgroupsized}  \newcommand{\dateiname}{L01636}\newcommand{\titel}{Max Mell an Arthur Schnitzler, 7. 11. 1906}\newcommand{\editorInnen}{Martin Anton Müller und Gerd-Hermann Susen}%% latex-leseansicht-abspann.tex
%% Abspann für die Leseansicht.
%% Der Schalter \ifkorrekturansicht ist bereits durch den Vorspann gesetzt.

%% latex-abspann.tex
%% Gemeinsamer Abspann für Korrekturansicht und Leseansicht.
%% Setzt den Schalter \ifkorrekturansicht voraus (gesetzt in den
%% einbindenden Dateien latex-korrekturansicht-abspann.tex bzw.
%% latex-leseansicht-abspann.tex).
%% ---------------------------------------------------------------

\normalsize

% Das esempio-Environment wird nur in der Leseansicht benötigt
\ifkorrekturansicht\else
\newenvironment{esempio}[3]%
{
    \vspace{1.5ex}
    \rlap{\underline{#1}}
    \par
    \setlength{\parindent}{0cm}
    \nopagebreak
    \leftskip=#2cm
    \rightskip=#3cm
}
{
    \par
}
\fi

\doendnotes{C}
\bigskip
\vfill

\clearpage

\footnotesize

\ifkorrekturansicht
  \lohead{\textsc{register}}
\fi

% theindex-Environment neu definieren ohne reledmac
\makeatletter
\renewenvironment{theindex}{%
  \ifkorrekturansicht
    \section*{\indexname}%
  \else
    \subsubsection*{Index der erwähnten Entitäten}%
  \fi
  \setlength{\parindent}{0pt}%
  \setlength{\parskip}{0pt plus 0.3pt}%
  \let\item\@idxitem
}{%
  \ifkorrekturansicht\clearpage\fi
}
\makeatother

\IfFileExists{\jobname-pw.ind}{\input{\jobname-pw.ind}}{}

% Quellenangabe nur in der Leseansicht
\ifkorrekturansicht\else
% Fallback-Definitionen, falls die .tex-Datei \titel etc. nicht gesetzt hat
\providecommand{\titel}{}
\providecommand{\editorInnen}{}
\providecommand{\dateiname}{\jobname}

\vspace{3cm}

\vfill

\footnotesize
\textsc{Quelle}: \titel. Herausgegeben von {\editorInnen}. In: \emph{Arthur Schnitzler: Briefwechsel mit Autorinnen und Autoren}.
 Digitale Edition, https://schnitzler-briefe.acdh.oeaw.ac.at/{\dateiname}.html (Stand \today)
\fi

\end{document}


      