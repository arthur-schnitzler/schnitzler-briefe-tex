%% latex-leseansicht-vorspann.tex
%% Vorspann für die Leseansicht.
%% Lädt die gemeinsame Datei latex-vorspann.tex mit nicht gesetztem Schalter.

\newif\ifkorrekturansicht
\korrekturansichtfalse

\input{../tex-inputs/latex-vorspann}


\section[ Paul Goldmann an Arthur Schnitzler, 11. 1. {[}1901{]}]{L03054 Paul Goldmann an Arthur Schnitzler,  11. 1. [1901]}
\nopagebreak\mylabel{L03054v}
\rehead{ }\normalsize\beginnumbering\briefempfaengerindex{Schnitzler, Arthur@\textsc{Schnitzler, Arthur}!zzzGoldmann, Paul@\emph{von Paul Goldmann}!1901-01-111@{11. 1. [1901]}|(be}
\toendnotes[C]{\smallbreak\pagebreak[2]}
\correspDesc{Versand  durch Paul Goldmann am 11. 1. [1901] in Berlin
\newline{}Erhalt  durch Arthur Schnitzler im Zeitraum [11. 1. 1901
                  – 15. 1. 1901?] in Wien}\toendnotes[C]{\smallbreak}
\Standort{DLA, A:Schnitzler, HS.NZ85.1.3171.}
\physDesc{Brief, 1 Blatt, 2 Seiten, 456 Zeichen
\newline{}Handschrift: blaue Tinte, deutsche Kurrent
\newline{}Schnitzler: mit Bleistift das Jahr »901« vermerkt }\toendnotes[C]{\smallbreak}
\pstart
           \raggedleft{}{\pb}\textcolor{gray}{\textbf{DESSAUERSTRASSE 19}}\oindex{Dessauer Straße@\textbf{Dessauer Straße}, \emph{Straße}|pw}\pend
           
\pstart
           Berlin\oindex{Berlin@\textbf{Berlin}, \emph{Hauptstadt}|pw}, 11. Januar.\pend
           
\pstart\center{}Mein lieber Freund,\pend\vspace{0.5em}
\pstart
           Im »Börſencourier\pwindex{Berliner Börsen-Courier@\emph{Berliner Börsen-Courier}|pw}« finde ich ein \label{K_L03054-1v}\edtext{Telegramm\pwindex{Telegramm unseres Wiener Correspondenten]@\emph{[Ein Telegramm unseres Wiener Correspondenten]}|pwv}}{\lemma{\textnormal{\emph{Telegramm}}}\Cendnote{\textnormal{[O. V.]: \emph{[Ein Telegramm unseres Wiener
                        Correspondenten]}\pwindex{Telegramm unseres Wiener Correspondenten]@\emph{[Ein Telegramm unseres Wiener Correspondenten]}|pwk}. In: \emph{Berliner
                        Börsen-Courier}\pwindex{Berliner Börsen-Courier@\emph{Berliner Börsen-Courier}|pwk}, Jg. 34, Nr. 17, 11. 1. 1901, Morgen-Ausgabe, 1. Beilage, S. [1].}}}\label{K_L03054-1} über
                  \label{K_L03054-2v}\edtext{Maßregelungen}{\lemma{\textnormal{\emph{Maßregelungen}}}\Cendnote{\textnormal{\emph{Lieutenant Gustl}\pwindex{Schnitzler, Arthur 15.\,5.\,1862 Wien – 21.\,10.\,1931 ebd.@\textsc{Schnitzler, Arthur} (15.\,5.\,1862 Wien – 21.\,10.\,1931 ebd.), \emph{Schriftsteller, Mediziner}!Lieutenant Gustl. Novelle@\strich\emph{Lieutenant Gustl. Novelle}|pwk}, erschienen in der
                  Weihnachtsnummer der \emph{Neuen Freien Presse}\pwindex{Neue Freie Presse@\emph{Neue Freie Presse}|pwk},
                  wurde von Teilen der Armee als Verspottung des Offiziersstandes empfunden und
                  löste schnell die Einsetzung eines Militärtribunals aus, was im Juni 1901 zur Aberkennung von Schnitzlers Offizierspatent führte.}}}\label{K_L03054-2}, die Dir die Militärbehörde\orgindex{k. u. k. Kriegsministerium@k. u. k. Kriegsministerium|pwv} wegen des »Lieutenant Guſtl\pwindex{Schnitzler, Arthur 15.\,5.\,1862 Wien – 21.\,10.\,1931 ebd.@\textsc{Schnitzler, Arthur} (15.\,5.\,1862 Wien – 21.\,10.\,1931 ebd.), \emph{Schriftsteller, Mediziner}!Lieutenant Gustl. Novelle@\strich\emph{Lieutenant Gustl. Novelle}|pw}« angedroht habe. Ich bin lebhaft
               beunruhigt und bitte, mir umgehend mitzutheilen, was vorgeht. Wäre es Dir möglich,
               mir ein complet{[}t{]}es Exemplar der Erzählung\pwindex{Schnitzler, Arthur 15.\,5.\,1862 Wien – 21.\,10.\,1931 ebd.@\textsc{Schnitzler, Arthur} (15.\,5.\,1862 Wien – 21.\,10.\,1931 ebd.), \emph{Schriftsteller, Mediziner}!Lieutenant Gustl. Novelle@\strich\emph{Lieutenant Gustl. Novelle}|pwv} zu überſenden? {\pb}Ich habe ſie\pwindex{Schnitzler, Arthur 15.\,5.\,1862 Wien – 21.\,10.\,1931 ebd.@\textsc{Schnitzler, Arthur} (15.\,5.\,1862 Wien – 21.\,10.\,1931 ebd.), \emph{Schriftsteller, Mediziner}!Lieutenant Gustl. Novelle@\strich\emph{Lieutenant Gustl. Novelle}|pwv} bisher nicht geleſen, weil in der \label{K_L03054-3v}\edtext{Nummer der N. Fr. Pr.\pwindex{Neue Freie Presse@\emph{Neue Freie Presse}|pw}}{\lemma{\textnormal{\emph{Nummer der N. Fr. Pr.}}}\Cendnote{\textnormal{Arthur Schnitzler: \emph{Lieutenant Gustl}\pwindex{Schnitzler, Arthur 15.\,5.\,1862 Wien – 21.\,10.\,1931 ebd.@\textsc{Schnitzler, Arthur} (15.\,5.\,1862 Wien – 21.\,10.\,1931 ebd.), \emph{Schriftsteller, Mediziner}!Lieutenant Gustl. Novelle@\strich\emph{Lieutenant Gustl. Novelle}|pwk}. In: \emph{Neue Freie Presse}\pwindex{Neue Freie Presse@\emph{Neue Freie Presse}|pwk}, Nr. 13.053, 25. 12. 1900, Morgenblatt, S. 34–41.}}}\label{K_L03054-3}, die mir
               zugegangen iſt, der Schluß fehlt.\pend
           
\pstart
           Viele Grüße! {\\[\baselineskip]}Dein {\\[\baselineskip]}\spacefill\mbox{Paul Goldmann}\pend
           \leftskip=0em{}\selectlanguage{ngerman}\endnumbering\briefempfaengerindex{Schnitzler, Arthur@\textsc{Schnitzler, Arthur}!zzzGoldmann, Paul@\emph{von Paul Goldmann}!1901-01-111@{11. 1. [1901]}|)be}\mylabel{L03054h}  \newcommand{\dateiname}{L03054}\newcommand{\titel}{Paul Goldmann an Arthur Schnitzler, 11. 1. [1901]}\newcommand{\editorInnen}{Martin Anton Müller und Laura Untner}%% latex-leseansicht-abspann.tex
%% Abspann für die Leseansicht.
%% Der Schalter \ifkorrekturansicht ist bereits durch den Vorspann gesetzt.

%% latex-abspann.tex
%% Gemeinsamer Abspann für Korrekturansicht und Leseansicht.
%% Setzt den Schalter \ifkorrekturansicht voraus (gesetzt in den
%% einbindenden Dateien latex-korrekturansicht-abspann.tex bzw.
%% latex-leseansicht-abspann.tex).
%% ---------------------------------------------------------------

\normalsize

% Das esempio-Environment wird nur in der Leseansicht benötigt
\ifkorrekturansicht\else
\newenvironment{esempio}[3]%
{
    \vspace{1.5ex}
    \rlap{\underline{#1}}
    \par
    \setlength{\parindent}{0cm}
    \nopagebreak
    \leftskip=#2cm
    \rightskip=#3cm
}
{
    \par
}
\fi

\doendnotes{C}
\bigskip
\vfill

\clearpage

\footnotesize

\ifkorrekturansicht
  \lohead{\textsc{register}}
\fi

% theindex-Environment neu definieren ohne reledmac
\makeatletter
\renewenvironment{theindex}{%
  \ifkorrekturansicht
    \section*{\indexname}%
  \else
    \subsubsection*{Index der erwähnten Entitäten}%
  \fi
  \setlength{\parindent}{0pt}%
  \setlength{\parskip}{0pt plus 0.3pt}%
  \let\item\@idxitem
}{%
  \ifkorrekturansicht\clearpage\fi
}
\makeatother

\IfFileExists{\jobname-pw.ind}{\input{\jobname-pw.ind}}{}

% Quellenangabe nur in der Leseansicht
\ifkorrekturansicht\else
% Fallback-Definitionen, falls die .tex-Datei \titel etc. nicht gesetzt hat
\providecommand{\titel}{}
\providecommand{\editorInnen}{}
\providecommand{\dateiname}{\jobname}

\vspace{3cm}

\vfill

\footnotesize
\textsc{Quelle}: \titel. Herausgegeben von {\editorInnen}. In: \emph{Arthur Schnitzler: Briefwechsel mit Autorinnen und Autoren}.
 Digitale Edition, https://schnitzler-briefe.acdh.oeaw.ac.at/{\dateiname}.html (Stand \today)
\fi

\end{document}


