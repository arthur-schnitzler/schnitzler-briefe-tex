%% latex-korrekturansicht-vorspann.tex
%% Vorspann für die Korrekturansicht.
%% Lädt die gemeinsame Datei latex-vorspann.tex mit gesetztem Schalter.

\newif\ifkorrekturansicht
\korrekturansichttrue

\input{../tex-inputs/latex-vorspann}


\section[ Felix Salten an Arthur Schnitzler, {[}10. 12. 1907{]}]{L03494 Felix Salten an Arthur Schnitzler, {[}10. 12. 1907{]}}
\nopagebreak\mylabel{L03494v}
\rehead{ }\normalsize\beginnumbering\briefempfaengerindex{Schnitzler, Arthur@\textsc{Schnitzler, Arthur}!zzzSalten, Felix@\emph{von Felix Salten}!1907-12-102@{{[}10. 12. 1907{]}}|(be}
\toendnotes[C]{\smallbreak\pagebreak[2]}\Standort{CUL, Schnitzler, B 89, B 1.}
\physDesc{Brief, 1 Blatt, 1 Seite, 807 Zeichen
\newline{}Handschrift: schwarze Tinte, lateinische Kurrent
\newline{}Schnitzler: mit Bleistift datiert: »10/12 907« 
\newline{}Ordnung: mit Bleistift von unbekannter Hand nummeriert: »238« }\toendnotes[C]{\smallbreak}
\pstart
           \noindent{}{\pb}Lieber, wir haben gestern durch Julie Wassermann\pwindex{Wassermann, Julie 05.12.1876 – April 1963@\textsc{Wassermann, Julie} (05.12.1876 – April 1963), \emph{Schriftsteller/Schriftstellerin}|pw} von \label{K_L03494-1v}\edtext{Olga\pwindex{Schnitzler, Olga 17.01.1882 – 13.01.1970@\textsc{Schnitzler, Olga} (17.01.1882 – 13.01.1970), \emph{Schauspieler/Schauspielerin, Sänger/Sängerin}|pw}’s Erkrankung}{\lemma{\textnormal{\emph{Olga’s Erkrankung}}}\Cendnote{\textnormal{Sie hatte Scharlach, vgl. A. S.: \emph{Tagebuch}, 31. 12. 1907.}}}\label{K_L03494-1} gehört und sind tief bestürzt
               darüber. Bei Ihrem Herrn Bruder\pwindex{Schnitzler, Julius 13.07.1865 – 29.06.1939@\textsc{Schnitzler, Julius} (13.07.1865 – 29.06.1939), \emph{Chirurg/Chirurgin}|pwv}, bei dem ich telefonisch anfragte, bekam ich eine relative Auskunft.
               Wir denken unausgesetzt an Sie Beide\pwindex{Schnitzler, Olga 17.01.1882 – 13.01.1970@\textsc{Schnitzler, Olga} (17.01.1882 – 13.01.1970), \emph{Schauspieler/Schauspielerin, Sänger/Sängerin}|pwv}, wenn ich Ihnen jetzt, wo Sie so abgeschloßen sind, irgend etwas
               bringen, erledigen oder sonstwas helfen kann, würde ich es so sehr gerne thun. Und
               wir hoffen aufs \uline{Innigste}, dass Sie sehr, sehr bald
               von aller Besorgnis um Olga\pwindex{Schnitzler, Olga 17.01.1882 – 13.01.1970@\textsc{Schnitzler, Olga} (17.01.1882 – 13.01.1970), \emph{Schauspieler/Schauspielerin, Sänger/Sängerin}|pw} aufs Beste befreit
               werden, dass alles gut abläuft, dass wir Sie alle recht bald gesund wiedersehen.
               Inzwischen werde ich mir erlauben, bei Ihrer Mutter\pwindex{Schnitzler, Louise 1840-07-08 – 1911-09-09@\textsc{Schnitzler, Louise} (1840-07-08 – 1911-09-09)|pwv} u. bei Ihrem Bruder\pwindex{Schnitzler, Julius 13.07.1865 – 29.06.1939@\textsc{Schnitzler, Julius} (13.07.1865 – 29.06.1939), \emph{Chirurg/Chirurgin}|pwv} telefonisch anzufragen, denn wir möchten täglich
               wißen, wie es Olga\pwindex{Schnitzler, Olga 17.01.1882 – 13.01.1970@\textsc{Schnitzler, Olga} (17.01.1882 – 13.01.1970), \emph{Schauspieler/Schauspielerin, Sänger/Sängerin}|pw} geht und was Sie Beide
               machen.\pend
           
\pstart
           Mit tausend herzlichsten Wünschen {\\[\baselineskip]}guten Gedanken und
               Grüßen an Olga\pwindex{Schnitzler, Olga 17.01.1882 – 13.01.1970@\textsc{Schnitzler, Olga} (17.01.1882 – 13.01.1970), \emph{Schauspieler/Schauspielerin, Sänger/Sängerin}|pw} u. Sie {\\[\baselineskip]}Ihr {\\[\baselineskip]}\spacefill\mbox{Salten}\pend
           \leftskip=0em{}
\pstart
           Dienstag.\pend
           \selectlanguage{ngerman}\endnumbering\briefempfaengerindex{Schnitzler, Arthur@\textsc{Schnitzler, Arthur}!zzzSalten, Felix@\emph{von Felix Salten}!1907-12-102@{{[}10. 12. 1907{]}}|)be}\mylabel{L03494h}  \normalsize

\doendnotes{C}
\bigskip
\vfill

\clearpage

\footnotesize

\lohead{\textsc{register}}

% Definiere theindex-Environment komplett neu ohne reledmac
\makeatletter
\renewenvironment{theindex}{%
  \section*{\indexname}%
  \setlength{\parindent}{0pt}%
  \setlength{\parskip}{0pt plus 0.3pt}%
  \let\item\@idxitem
}{%
  \clearpage
}
\makeatother

\IfFileExists{\jobname-pw.ind}{\input{\jobname-pw.ind}}{}

\end{document}

      