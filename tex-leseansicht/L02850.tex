%% latex-leseansicht-vorspann.tex
%% Vorspann für die Leseansicht.
%% Lädt die gemeinsame Datei latex-vorspann.tex mit nicht gesetztem Schalter.

\newif\ifkorrekturansicht
\korrekturansichtfalse

\input{../tex-inputs/latex-vorspann}


\section[ Paul Goldmann an Arthur Schnitzler, 10. 4. 1898]{L02850 Paul Goldmann an Arthur Schnitzler,  10. 4. 1898}
\nopagebreak\mylabel{L02850v}
\rehead{ }\normalsize\beginnumbering\briefempfaengerindex{Schnitzler, Arthur@\textsc{Schnitzler, Arthur}!zzzGoldmann, Paul@\emph{von Paul Goldmann}!1898-04-101@{10. 4. 1898}|(be}
\toendnotes[C]{\smallbreak\pagebreak[2]}
\correspDesc{Versand  durch Paul Goldmann am 10. 4. 1898 in Port Said
\newline{}Erhalt  durch Arthur Schnitzler am 17. 4. 1898 in Wien}\toendnotes[C]{\smallbreak}
\Standort{DLA, A:Schnitzler, HS.NZ85.1.3168.}
\physDesc{Postkarte, 303 Zeichen
\newline{}Handschrift: Bleistift, deutsche Kurrent
\newline{}Versand: 1) Stempel: »\nobreak{}\oindex{Port Said@\textbf{Port Said}|pwk}Port-Said, 10 IV {[}9{]}8, 1\textcolor{gray}{1}\nobreak{}«.   2) Stempel: »\nobreak{}\oindex{IX., Alsergrund@\textbf{IX., Alsergrund}, \emph{Verwaltungsgebiet}|pwk}Wien
                                       \textcolor{gray}{9}/3 72, 17. 4. {[}9{]}8, \textcolor{gray}{9}. V, Bestellt\nobreak{}«. 
\newline{}Schnitzler: mit Bleistift das Datum »Apr 98« vermerkt }\toendnotes[C]{\smallbreak}\pstart{}{\pb}\textcolor{gray}{\textbf{\begin{otherlanguage}{french}A\end{otherlanguage}}}\pend{}\pstart{}\textsc{\begin{otherlanguage}{french}M. le Dr.\end{otherlanguage}}\pend{}\pstart{}\textsc{Arthur Schnitzler}\pend{}\pstart{}\textsc{IX. Frankgaſse 1\oindex{Wien@\textbf{Wien}!IX., Alsergrund@\textbf{IX., Alsergrund}!Frankgasse 1@\textbf{Frankgasse 1}, \emph{Wohngebäude}|pw}}\pend{}\pstart{}\textsc{Wien\oindex{Wien@\textbf{Wien}, \emph{Verwaltungsgebiet}|pw}}\pend{}\pstart{}\textsc{\begin{otherlanguage}{french}Autriche\oindex{Österreich@\textbf{Österreich}|pw}\end{otherlanguage}}\pend{}{\bigskip}\vspace{1em}
\pstart
           {\pb}Oſtermorgen in \textsc{Port-Said\oindex{Port Said@\textbf{Port Said}|pw}}\pend
           \vspace{0.5em}
\pstart
           Eine engl\oindex{England@\textbf{England}, \emph{Land}|pwv}iſche Muſikkapelle{ }ſpielt auf dem \textsc{Lesseps}-Platz\oindex{Ferdinand-de-Lesseps-Platz@\textbf{Ferdinand-de-Lesseps-Platz}, \emph{Platz}|pw}, und in der Araberſtadt\oindex{Araberstadt@\textbf{Araberstadt}, \emph{Teil eines besiedelten Ortes}|pw} wird Hochzeit gefeiert, und die Muſiker{ }ſitzen
               auf dem Pflaſter mit Pauken u. Flöten. Ich bin{ }ſehr{ }ſeekrank. Viele treue Grüße Dir
               u. \textsc{Richard\pwindex{Beer-Hofmann, Richard 11.\,7.\,1866 Wien – 26.\,9.\,1945 New York City@\textsc{Beer-Hofmann, Richard} (11.\,7.\,1866 Wien – 26.\,9.\,1945 New York City), \emph{Schriftsteller}|pw}}.\pend
           \pstart Dein \spacefill\mbox{P. G.}\pend{}\selectlanguage{ngerman}\endnumbering\briefempfaengerindex{Schnitzler, Arthur@\textsc{Schnitzler, Arthur}!zzzGoldmann, Paul@\emph{von Paul Goldmann}!1898-04-101@{10. 4. 1898}|)be}\mylabel{L02850h}  \newcommand{\dateiname}{L02850}\newcommand{\titel}{Paul Goldmann an Arthur Schnitzler, 10. 4. 1898}\newcommand{\editorInnen}{Martin Anton Müller und Laura Untner}%% latex-leseansicht-abspann.tex
%% Abspann für die Leseansicht.
%% Der Schalter \ifkorrekturansicht ist bereits durch den Vorspann gesetzt.

%% latex-abspann.tex
%% Gemeinsamer Abspann für Korrekturansicht und Leseansicht.
%% Setzt den Schalter \ifkorrekturansicht voraus (gesetzt in den
%% einbindenden Dateien latex-korrekturansicht-abspann.tex bzw.
%% latex-leseansicht-abspann.tex).
%% ---------------------------------------------------------------

\normalsize

% Das esempio-Environment wird nur in der Leseansicht benötigt
\ifkorrekturansicht\else
\newenvironment{esempio}[3]%
{
    \vspace{1.5ex}
    \rlap{\underline{#1}}
    \par
    \setlength{\parindent}{0cm}
    \nopagebreak
    \leftskip=#2cm
    \rightskip=#3cm
}
{
    \par
}
\fi

\doendnotes{C}
\bigskip
\vfill

\clearpage

\footnotesize

\ifkorrekturansicht
  \lohead{\textsc{register}}
\fi

% theindex-Environment neu definieren ohne reledmac
\makeatletter
\renewenvironment{theindex}{%
  \ifkorrekturansicht
    \section*{\indexname}%
  \else
    \subsubsection*{Index der erwähnten Entitäten}%
  \fi
  \setlength{\parindent}{0pt}%
  \setlength{\parskip}{0pt plus 0.3pt}%
  \let\item\@idxitem
}{%
  \ifkorrekturansicht\clearpage\fi
}
\makeatother

\IfFileExists{\jobname-pw.ind}{\input{\jobname-pw.ind}}{}

% Quellenangabe nur in der Leseansicht
\ifkorrekturansicht\else
% Fallback-Definitionen, falls die .tex-Datei \titel etc. nicht gesetzt hat
\providecommand{\titel}{}
\providecommand{\editorInnen}{}
\providecommand{\dateiname}{\jobname}

\vspace{3cm}

\vfill

\footnotesize
\textsc{Quelle}: \titel. Herausgegeben von {\editorInnen}. In: \emph{Arthur Schnitzler: Briefwechsel mit Autorinnen und Autoren}.
 Digitale Edition, https://schnitzler-briefe.acdh.oeaw.ac.at/{\dateiname}.html (Stand \today)
\fi

\end{document}


