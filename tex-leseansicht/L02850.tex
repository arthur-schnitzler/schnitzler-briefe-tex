%% latex-korrekturansicht-vorspann.tex
%% Vorspann für die Korrekturansicht.
%% Lädt die gemeinsame Datei latex-vorspann.tex mit gesetztem Schalter.

\newif\ifkorrekturansicht
\korrekturansichttrue

\input{../tex-inputs/latex-vorspann}


\section[ Paul Goldmann an Arthur Schnitzler, 10. 4. 1898]{L02850 Paul Goldmann an Arthur Schnitzler, 10. 4. 1898}
\nopagebreak\mylabel{L02850v}
\rehead{ }\normalsize\beginnumbering\briefempfaengerindex{Schnitzler, Arthur@\textsc{Schnitzler, Arthur}!zzzGoldmann, Paul@\emph{von Paul Goldmann}!1898-04-101@{10. 4. 1898}|(be}
\toendnotes[C]{\smallbreak\pagebreak[2]}\Standort{DLA, A:Schnitzler, HS.NZ85.1.3168.}
\physDesc{Postkarte, 303 Zeichen
\newline{}Handschrift: Bleistift, deutsche Kurrent
\newline{}Versand: 1) Stempel: »\nobreak{}\oindex{Port Said@\textbf{Port Said}, \emph{Besiedelter Ort (A.BSO)}|pwk}Port-Said, 10 IV {[}9{]}8, 1\textcolor{gray}{1}\nobreak{}«.   2) Stempel: »\nobreak{}\oindex{IX., Alsergrund@\textbf{IX., Alsergrund}, \emph{A.ADM3}|pwk}Wien
                                       \textcolor{gray}{9}/3 72, 17. 4. {[}9{]}8, \textcolor{gray}{9}. V, Bestellt\nobreak{}«. 
\newline{}Schnitzler: mit Bleistift das Datum »Apr 98« vermerkt }\toendnotes[C]{\smallbreak}\pstart{}{\pb}\textcolor{gray}{\textbf{\begin{otherlanguage}{french}A\end{otherlanguage}}}\pend{}\pstart{}\textsc{\begin{otherlanguage}{french}M. le Dr.\end{otherlanguage}}\pend{}\pstart{}\textsc{Arthur Schnitzler}\pend{}\pstart{}\textsc{IX. Frankgaſse 1\oindex{Frankgasse 1@\textbf{Frankgasse 1}, \emph{Wohngebäude (K.WHS)}|pw}}\pend{}\pstart{}\textsc{Wien\oindex{Wien@\textbf{Wien}, \emph{A.ADM2}|pw}}\pend{}\pstart{}\textsc{\begin{otherlanguage}{french}Autriche\oindex{Oesterreich@\textbf{Österreich}, \emph{A.PCLI}|pw}\end{otherlanguage}}\pend{}{\bigskip}\vspace{1em}
\pstart
           {\pb}Oſtermorgen in \textsc{Port-Said\oindex{Port Said@\textbf{Port Said}, \emph{Besiedelter Ort (A.BSO)}|pw}}\pend
           \vspace{0.5em}
\pstart
           Eine engl\oindex{England@\textbf{England}, \emph{A.ADM1}|pwv}iſche Muſikkapelle
               ſpielt auf dem \textsc{Lesseps}-Platz\oindex{Ferdinand-de-Lesseps-Platz@\textbf{Ferdinand-de-Lesseps-Platz}, \emph{Platz (K.PLT)}|pw}, und in der Araberſtadt\oindex{Araberstadt@\textbf{Araberstadt}, \emph{Teil eines besiedelten Ortes (A.BSOX)}|pw} wird Hochzeit gefeiert, und die Muſiker ſitzen
               auf dem Pflaſter mit Pauken u. Flöten. Ich bin ſehr ſeekrank. Viele treue Grüße Dir
               u. \textsc{Richard\pwindex{Beer-Hofmann, Richard 1866-07-11 – 1945-09-26@\textsc{Beer-Hofmann, Richard} (1866-07-11 – 1945-09-26), \emph{Schriftsteller/Schriftstellerin}|pw}}.\pend
           \pstart Dein \spacefill\mbox{P. G.}\pend{}\selectlanguage{ngerman}\endnumbering\briefempfaengerindex{Schnitzler, Arthur@\textsc{Schnitzler, Arthur}!zzzGoldmann, Paul@\emph{von Paul Goldmann}!1898-04-101@{10. 4. 1898}|)be}\mylabel{L02850h}  \normalsize

\doendnotes{C}
\bigskip
\vfill

\clearpage

\footnotesize

\lohead{\textsc{register}}

% Definiere theindex-Environment komplett neu ohne reledmac
\makeatletter
\renewenvironment{theindex}{%
  \section*{\indexname}%
  \setlength{\parindent}{0pt}%
  \setlength{\parskip}{0pt plus 0.3pt}%
  \let\item\@idxitem
}{%
  \clearpage
}
\makeatother

\IfFileExists{\jobname-pw.ind}{\input{\jobname-pw.ind}}{}

\end{document}

      