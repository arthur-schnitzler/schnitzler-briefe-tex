%% latex-leseansicht-vorspann.tex
%% Vorspann für die Leseansicht.
%% Lädt die gemeinsame Datei latex-vorspann.tex mit nicht gesetztem Schalter.

\newif\ifkorrekturansicht
\korrekturansichtfalse

\input{../tex-inputs/latex-vorspann}


         
         \renewcommand{\erwaehntePersonen}{Personen: Richard Beer-Hofmann, Paul Goldmann}
         \renewcommand{\erwaehnteOrte}{Orte: Klobenstein, Pörtschach, Vahrn, Villa Arnstein}
         \renewcommand{\erwaehnteWerke}{
               \section[Arthur Schnitzler an Richard Beer-Hofmann, 1. 8. 1901]{ Arthur Schnitzler an Richard Beer-Hofmann, 1. 8. 1901}\nopagebreak\mylabel{v}\rehead{ }\begin{ledgroupsized}[t]{13cm}\normalsize\beginnumbering \toendnotes[C]{\smallbreak\pagebreak[2]} \Standort{YCGL, MSS 31.}
\physDesc{Telegramm
\newline{}Handschrift einer Schreibkraft: schwarze Tinte, lateinische Kurrent\newline{}Versand: »\noindent{}\textcolor{gray}{\textbf{Von}}{ }Vahren\oindex{Vahrn@\textbf{Vahrn}|pw}{ / }\textcolor{gray}{\textbf{Aufgabe-Nr}} 4 \textcolor{gray}{\textbf{mit {\dots}
                                                  Taxworten (}}21 \textcolor{gray}{\textbf{Worten {\dots} Chiffern)}}{ / }\textcolor{gray}{\textbf{Eingelangt von {\dots} auf Leitung Nr. {\dots} am}}{ }1/\textcolor{gray}{8} \textcolor{gray}{\textbf{190}}{\dots}{ }\textcolor{gray}{\textbf{um}}{ }1 \textcolor{gray}{\textbf{Uhr}} 10 \textcolor{gray}{\textbf{Min.}}{ }n\textcolor{gray}{\textbf{Mittag}}{ / }\textcolor{gray}{\textbf{Aufgegeben am}}{ }1/8 \textcolor{gray}{\textbf{190}}1{ }\textcolor{gray}{\textbf{um}}{ }9 \textcolor{gray}{\textbf{Uhr {\dots} Min.}} n\textcolor{gray}{\textbf{Mittag}}« \newline{}Ordnung: mit Bleistift von unbekannter Hand datiert »1./8« }\toendnotes[C]{\smallbreak}\pstart{}{\pb}Richard Beer Hoffmann\pend{}\pstart{}Villa Arnstein\oindex{Villa Arnstein@\textbf{Villa Arnstein}|pw}\pend{}\pstart{}\textcolor{gray}{\textbf{\textit{PÖRTSCHACH AM SEE\oindex{Poertschach@\textbf{Pörtschach}|pw}}}}\pend{}{\bigskip}\pstart
           \noindent{}{\pb}Bitte Paul\pwindex{Goldmann, Paul 31.01.1865 – 25.09.1935@\textsc{Goldmann, Paul} (31.01.1865 – 25.09.1935), \emph{Schriftsteller, Journalist}|pw}
                    nachtelegrafieren er möge doch auf einen Tag \label{K_L01156_1v}\edtext{fahren\oindex{Vahrn@\textbf{Vahrn}|pw}}{\lemma{\textnormal{\emph{fahren}}}\Cendnote{\textnormal{Gemeint dürfte Vahrn\oindex{Vahrn@\textbf{Vahrn}|pwk} sein.}}}\label{K_L01156_1h} ko{\geminationm}en
                    behufs Rücksprache habe gewünschten Höhenort in Petto Klobenstein\oindex{Klobenstein@\textbf{Klobenstein}|pw} Herzlichst\pend
           \pstart \spacefill\mbox{Arthur}\pend{}
         
         \endnumbering\mylabel{h}\end{ledgroupsized}  \newcommand{\dateiname}{L01156}\newcommand{\titel}{Arthur Schnitzler an Richard Beer-Hofmann, 1. 8. 1901}\newcommand{\editorInnen}{Martin Anton Müller und Gerd-Hermann Susen}%% latex-leseansicht-abspann.tex
%% Abspann für die Leseansicht.
%% Der Schalter \ifkorrekturansicht ist bereits durch den Vorspann gesetzt.

%% latex-abspann.tex
%% Gemeinsamer Abspann für Korrekturansicht und Leseansicht.
%% Setzt den Schalter \ifkorrekturansicht voraus (gesetzt in den
%% einbindenden Dateien latex-korrekturansicht-abspann.tex bzw.
%% latex-leseansicht-abspann.tex).
%% ---------------------------------------------------------------

\normalsize

% Das esempio-Environment wird nur in der Leseansicht benötigt
\ifkorrekturansicht\else
\newenvironment{esempio}[3]%
{
    \vspace{1.5ex}
    \rlap{\underline{#1}}
    \par
    \setlength{\parindent}{0cm}
    \nopagebreak
    \leftskip=#2cm
    \rightskip=#3cm
}
{
    \par
}
\fi

\doendnotes{C}
\bigskip
\vfill

\clearpage

\footnotesize

\ifkorrekturansicht
  \lohead{\textsc{register}}
\fi

% theindex-Environment neu definieren ohne reledmac
\makeatletter
\renewenvironment{theindex}{%
  \ifkorrekturansicht
    \section*{\indexname}%
  \else
    \subsubsection*{Index der erwähnten Entitäten}%
  \fi
  \setlength{\parindent}{0pt}%
  \setlength{\parskip}{0pt plus 0.3pt}%
  \let\item\@idxitem
}{%
  \ifkorrekturansicht\clearpage\fi
}
\makeatother

\IfFileExists{\jobname-pw.ind}{\input{\jobname-pw.ind}}{}

% Quellenangabe nur in der Leseansicht
\ifkorrekturansicht\else
% Fallback-Definitionen, falls die .tex-Datei \titel etc. nicht gesetzt hat
\providecommand{\titel}{}
\providecommand{\editorInnen}{}
\providecommand{\dateiname}{\jobname}

\vspace{3cm}

\vfill

\footnotesize
\textsc{Quelle}: \titel. Herausgegeben von {\editorInnen}. In: \emph{Arthur Schnitzler: Briefwechsel mit Autorinnen und Autoren}.
 Digitale Edition, https://schnitzler-briefe.acdh.oeaw.ac.at/{\dateiname}.html (Stand \today)
\fi

\end{document}


      