%% latex-korrekturansicht-vorspann.tex
%% Vorspann für die Korrekturansicht.
%% Lädt die gemeinsame Datei latex-vorspann.tex mit gesetztem Schalter.

\newif\ifkorrekturansicht
\korrekturansichttrue

\input{../tex-inputs/latex-vorspann}


\section[Arthur Schnitzler an Richard Beer-Hofmann, 1. 8. 1901]{L01156 Arthur Schnitzler an Richard Beer-Hofmann, 1. 8. 1901}
\nopagebreak\mylabel{L01156v}
\rehead{ }\normalsize\beginnumbering\briefempfaengerindex{Beer-Hofmann, Richard@\textsc{Beer-Hofmann, Richard}!zzzSchnitzler, Arthur@\emph{von Arthur Schnitzler}!1901-08-011@{1. 8. 1901}|(be}
\toendnotes[C]{\smallbreak\pagebreak[2]}\Standort{YCGL, MSS 31.}
\physDesc{Telegramm, 199 Zeichen
\newline{}Handschrift einer Schreibkraft: schwarze Tinte, lateinische Kurrent
\newline{}Versand: »\noindent{}\textcolor{gray}{\textbf{Von}}{ }Vahren\oindex{Vahrn@\textbf{Vahrn}, \emph{P.PPLA3}|pw}{ / }\textcolor{gray}{\textbf{Aufgabe-Nr}} 4 \textcolor{gray}{\textbf{mit {\dots} Taxworten (}}21 \textcolor{gray}{\textbf{Worten {\dots}
                                          Chiffern)}}{ / }\textcolor{gray}{\textbf{Eingelangt von {\dots} auf Leitung Nr. {\dots}
                                          am}}{ }1/\textcolor{gray}{8}\textcolor{gray}{\textbf{190}}{\dots}{ }\textcolor{gray}{\textbf{um}}{ }1 \textcolor{gray}{\textbf{Uhr}} 10 \textcolor{gray}{\textbf{Min.}}{ }n\textcolor{gray}{\textbf{Mittag}}{ / }\textcolor{gray}{\textbf{Aufgegeben am}}{ }1/8 \textcolor{gray}{\textbf{190}}1{ }\textcolor{gray}{\textbf{um}}{ }9 \textcolor{gray}{\textbf{Uhr {\dots}
                                             Min.}} n\textcolor{gray}{\textbf{Mittag}}« 
\newline{}Ordnung: mit Bleistift von unbekannter Hand datiert »1./8« }\toendnotes[C]{\smallbreak}\pstart{}{\pb}Richard Beer Hoffmann\pend{}\pstart{}Villa Arnstein\oindex{Villa Arnstein@\textbf{Villa Arnstein}, \emph{Wohngebäude (K.WHS)}|pw}\pend{}\pstart{}\textcolor{gray}{\textbf{\textit{PÖRTSCHACH AM SEE\oindex{Poertschach am Woerthersee@\textbf{Pörtschach am Wörthersee}, \emph{P.PPL}|pw}}}}\pend{}{\bigskip}\vspace{1em}
\pstart
           \noindent{}{\pb}Bitte \label{K_L01156-1v}\edtext{Paul\pwindex{Goldmann, Paul 31.01.1865 – 25.09.1935@\textsc{Goldmann, Paul} (31.01.1865 – 25.09.1935), \emph{Schriftsteller/Schriftstellerin, Journalist/Journalistin}|pw} nachtelegrafieren}{\lemma{\textnormal{\emph{Paul nachtelegrafieren}}}\Cendnote{\textnormal{Am 5. 8. [1901] schrieb Goldmann\pwindex{Goldmann, Paul 31.01.1865 – 25.09.1935@\textsc{Goldmann, Paul} (31.01.1865 – 25.09.1935), \emph{Schriftsteller/Schriftstellerin, Journalist/Journalistin}|pwk},
                  er hätte das Telegramm zu spät erhalten. Da Beer-Hofmann\pwindex{Beer-Hofmann, Richard 1866-07-11 – 1945-09-26@\textsc{Beer-Hofmann, Richard} (1866-07-11 – 1945-09-26), \emph{Schriftsteller/Schriftstellerin}|pwk} und Goldmann\pwindex{Goldmann, Paul 31.01.1865 – 25.09.1935@\textsc{Goldmann, Paul} (31.01.1865 – 25.09.1935), \emph{Schriftsteller/Schriftstellerin, Journalist/Journalistin}|pwk} sich am
                     2. 8. 1901 sprachen,
                  dürfte das vorliegende Telegramm erst nach dem Zusammentreffen an Beer-Hofmann\pwindex{Beer-Hofmann, Richard 1866-07-11 – 1945-09-26@\textsc{Beer-Hofmann, Richard} (1866-07-11 – 1945-09-26), \emph{Schriftsteller/Schriftstellerin}|pwk} gelangt sein.}}}\label{K_L01156-1} er möge
               doch auf einen Tag \label{K_L01156-2v}\edtext{fahren\oindex{Vahrn@\textbf{Vahrn}, \emph{P.PPLA3}|pw}}{\lemma{\textnormal{\emph{fahren}}}\Cendnote{\textnormal{Gemeint dürfte Vahrn\oindex{Vahrn@\textbf{Vahrn}, \emph{P.PPLA3}|pwk} sein.}}}\label{K_L01156-2} ko{\geminationm}en behufs
               Rücksprache habe gewünschten Höhenort in Petto Klobenstein\oindex{Klobenstein@\textbf{Klobenstein}, \emph{P.PPL}|pw} Herzlichst\pend
           \pstart \spacefill\mbox{Arthur}\pend{}\selectlanguage{ngerman}\endnumbering\briefempfaengerindex{Beer-Hofmann, Richard@\textsc{Beer-Hofmann, Richard}!zzzSchnitzler, Arthur@\emph{von Arthur Schnitzler}!1901-08-011@{1. 8. 1901}|)be}\mylabel{L01156h}  \normalsize

\doendnotes{C}
\bigskip
\vfill

\clearpage

\footnotesize

\lohead{\textsc{register}}

% Definiere theindex-Environment komplett neu ohne reledmac
\makeatletter
\renewenvironment{theindex}{%
  \section*{\indexname}%
  \setlength{\parindent}{0pt}%
  \setlength{\parskip}{0pt plus 0.3pt}%
  \let\item\@idxitem
}{%
  \clearpage
}
\makeatother

\IfFileExists{\jobname-pw.ind}{\input{\jobname-pw.ind}}{}

\end{document}

      