%% latex-leseansicht-vorspann.tex
%% Vorspann für die Leseansicht.
%% Lädt die gemeinsame Datei latex-vorspann.tex mit nicht gesetztem Schalter.

\newif\ifkorrekturansicht
\korrekturansichtfalse

\input{../tex-inputs/latex-vorspann}


         
         \renewcommand{\erwaehntePersonen}{Personen: Peter Altenberg, Egon Friedell, Anna Holitscher}
         \renewcommand{\erwaehnteWerke}{Werke: Briefe von und über Peter Altenberg, Neues Wiener Journal}
               \section[Peter Altenberg an Arthur Schnitzler, {[}12. 7. 1894?{]}]{ Peter Altenberg an Arthur Schnitzler, {[}12. 7. 1894?{]}}\nopagebreak\mylabel{v}\rehead{ }\begin{ledgroupsized}[t]{13cm}\normalsize\beginnumbering\briefempfaengerindex{Schnitzler, Arthur@\textsc{Schnitzler, Arthur}!zzzAltenberg, Peter@\emph{von Peter Altenberg}!1894-07-122@{{[}12. 7. 1894{]}}|(be} \toendnotes[C]{\smallbreak\pagebreak[2]} \Standort{München, Bayerische Staatsbibliothek, DE-611-HS-86373.}
\physDesc{Brief, 1 Blatt, 4 Seiten, 2073 Zeichen
\newline{}Handschrift: schwarze Tinte, deutsche Kurrent
\newline{}Schnitzler: mit rotem Buntstift eine Unterstreichung 
\newline{}Zusatz: zum Brief vgl. Schnitzler an Egon Friedell\pwindex{Friedell, Egon 21.01.1878 – 16.03.1938@\textsc{Friedell, Egon} (21.01.1878 – 16.03.1938), \emph{Schriftsteller, Journalist, Kulturphilosoph}|pw}, 17. 5. 1920 (Egon Friedell\pwindex{Friedell, Egon 21.01.1878 – 16.03.1938@\textsc{Friedell, Egon} (21.01.1878 – 16.03.1938), \emph{Schriftsteller, Journalist, Kulturphilosoph}|pwk}: \emph{Briefe}. Ausgewählt und herausgegeben von Walther
                                    Schneider. Wien, Stuttgart: \emph{Georg
                                       Prachner}{ }{[}1954{]}, S. 39) }\buchAbdrucke{\weitereDrucke{1) \emph{Briefe von und über Peter Altenberg.} In: \emph{Die Wage}, Jg. N.F. 1, Nr. 8, 20. 11. 1920, S. 100–104, hier: S. 103–104.} \weitereDrucke{2) \pwindex{Altenberg, Peter 09.03.1859 – 08.01.1919@\textsc{Altenberg, Peter} (09.03.1859 – 08.01.1919), \emph{Schriftsteller}!Briefe von und ueber Peter Altenberg1920@\strich\emph{Briefe von und über Peter Altenberg} {[}1920{]}|pwk}\pwindex{Neues Wiener Journal1893 – 1939@\emph{Neues Wiener Journal} {[}1893 – 1939{]}|pwk}\emph{Letzte Briefe von Peter Altenberg.} In: \emph{Neues Wiener Journal}, Jg. 28, Nr. 9714, 21. 11. 1920, S. 8.} \weitereDrucke{3) \emph{Das Altenberg-Buch}. Friedell, Egon. Wien, Leipzig: \emph{Wiener Graphische Werkstätte} 1922, S. 77–81.} \weitereDrucke{4) Olga Schnitzler: \emph{Spiegelbild der Freundschaft}. Salzburg: \emph{Residenz Verlag} 1962, S. 35–36.} \weitereDrucke{5) \emph{Peter Altenberg. Leben und Werk in Texten und Bildern}. Kosler, Hans Christian. München: \emph{Matthes  Seitz} 1981.} \weitereDrucke{6) Gottfried Wunberg: \emph{Die Wiener Moderne}. Ditzingen: \emph{Reclam} 1981.} \weitereDrucke{7) Hans-Ulrich Lindken: \emph{Arthur Schnitzler. Aspekte und Akzente. Materialien zu Leben
                        und Werk}. Frankfurt am Main, Bern, Göttingen: \emph{Peter Lang} 1984, S. 174–175 (Europäische Hochschulschriften, Reihe 1, Deutsche Sprache und
                        Literatur, 754).} \weitereDrucke{8) Andrew Barker, Leo A. Lensing: \emph{Peter Altenberg: Rezept die Welt zu sehen}. Wien: \emph{Braumüller} 1995, S. 46 (Untersuchungen zur österreichischen Literatur des 20.
                        Jahrhunderts, 11).} \weitereDrucke{9) Peter Altenberg: \emph{Die Selbsterfindung eines Dichters. Briefe und Dokumente
                        1892–1896}. Hg. und mit einem Nachwort von Leo A. Lensing. Göttingen: \emph{Wallstein} 2009, S. 23–24.} }\toendnotes[C]{\smallbreak}\pstart{}{\pb}Lieber \textsc{D\textsuperscript{r.}} Arthur Schnitzler.\pend\pstart
           Ihr wunderſchöner \label{K_L00351-1v}\edtext{Brief}{\lemma{\textnormal{\emph{Brief}}}\Cendnote{\textnormal{Altenberg\pwindex{Altenberg, Peter 09.03.1859 – 08.01.1919@\textsc{Altenberg, Peter} (09.03.1859 – 08.01.1919), \emph{Schriftsteller}|pwk} erwähnt das Korrespondenzstück in
                  einem Brief vom 12. 7. 1894 an Annie
                     Holitscher\pwindex{Holitscher, Anna 06.05.1864 – 1942@\textsc{Holitscher, Anna} (06.05.1864 – 1942)|pwk} (\emph{Die Selbsterfindung eines Dichters}, S. 138). An
                  dieser Stelle erwähnt er auch, dass er Schnitzler\pwindex{Schnitzler, Arthur 15.05.1862 – 21.10.1931@\textsc{Schnitzler, Arthur} (15.05.1862 – 21.10.1931), \emph{Schriftsteller, Mediziner}|pwk} an eben diesem Tag geantwortet habe.}}}\label{K_L00351-1h} hat mich wirklich
               außerordentlich gefreut. Wie schreibe ich denn?!\pend
           \pstart
           Ganz frei, ganz ohne Bedenken. Nie weiß ich mein Thema vorher, nie denke ich nach.
               Ich nehme Papier und ſchreibe. Sogar den Titel ſchreibe ich ſo hin und hoffe, es wird
               ſich ſchon etwas machen, was mit dem Titel in Zuſa{\geminationm}enhang ſteht.\pend
           \pstart
           Man muß ſich auf ſich verlaſſen, ſich nicht Gewalt anthun, ſich entſetzlich frei
               ausleben laſſen, hinfliegen –. {\pb}Was dabei herausko{\geminationm}t, iſt ſicher das was wirklich u. tief in mir war. Ko{\geminationm}t nichts heraus, ſo war eben nichts wirklich und tief
               darin und das macht dann auch nichts.\pend
           \pstart
           Ich betrachte ſchreiben als eine natürliche organiſche Entlaſtung eines vollen, eines
               übervollen Menſchen.\pend
           \pstart
           Daher alle \strikeout{\textcolor{gray}{meine}} Fehler, Bläſſen. Ich haſſe die \textsc{Retouche}. Schmeiſſ’
               es hin und gut–! O\substVorne{}\textsuperscript{\textcolor{gray}{b}\textcolor{gray}{×}}\substDazwischen{}de\substHinten{}r ſchlecht! Was macht das?! Wenn nur du es biſt, Du und kein Anderer, dein
               heiliges Du! Ihr Wort »Selbſtſucher« iſt wirklich {\pb}außerordentlich. Wann werden Sie aber ſchreiben »Selbſtfinder«?!\pend
           \pstart
           \strikeout{Freiheit und} Meine Sachen haben das \textsc{Malheur}, daß ſie i{\geminationm}er für
               kleine Proben betrachtet werden, während ſie leider bereits das ſind, was ich
               überhaupt zu leiſten im Stand bin. Aber was macht es?! Ob ich ſchreibe oder nicht,
               iſt mir gleichgiltig.\pend
           \pstart
           Wichtiger iſt, daß ich in einem Kreiſe von feinen gebildeten jungen Leuten zeige, daß
                  \strikeout{f\textcolor{gray}{×}\-\textcolor{gray}{×}\-\textcolor{gray}{×}\-\textcolor{gray}{×}\-\textcolor{gray}{×}} in mir das Fünkchen glimmt. Sonſt kommt man sich ſo gedrückt vor, ſo
               zudringlich, ſo ſchief angeblinzelt. Ich bin ſo ſchon genug »\textsc{Invalide} des Lebens«.\pend
           \pstart
           Ihr Brief hat mich ſehr, ſehr gefreut! {\pb}\strikeout{Ich \textcolor{gray}{zeigs ohne}} Sie ſind überhaupt Alle ſo liebenswürdig gegen mich. Jeder iſt wolwollend. \uline{Sie} haben mir aber wirklich wundervolle Sachen geſagt.
               Beſonders das Wort »Selbſtſucher« eben.\pend
           \pstart
           Ich bitte Sie, man hat keinen Beruf, kein Geld, keine \textsc{Position} u. ſchon ſehr wenig Haare, da iſt ſo eine feine Anerkennung von
               einem »Wiſſenden« ſehr, ſehr angenehm.\pend
           \pstart
           Deshalb bin u. bleibe ich doch nur ein Schreiber von »Muſtern ohne Werth« u. die
               Waare ko{\geminationm}t alleweil nicht. Ich bin ſo ein kleiner
               Handſpiegel, \textsc{Toilette}ſpiegel, kein \strikeout{Weltſpiegel} Welten-Spiegel.\pend
           \pstart
           Ihr{\\[\baselineskip]}\spacefill\mbox{Richard Engländer.}\pend
           \leftskip=0em{}
         
         \endnumbering\mylabel{h}\end{ledgroupsized}  \newcommand{\dateiname}{L00351}\newcommand{\titel}{Peter Altenberg an Arthur Schnitzler, [12. 7. 1894?]}\newcommand{\editorInnen}{Martin Anton Müller und Gerd-Hermann Susen}%% latex-leseansicht-abspann.tex
%% Abspann für die Leseansicht.
%% Der Schalter \ifkorrekturansicht ist bereits durch den Vorspann gesetzt.

%% latex-abspann.tex
%% Gemeinsamer Abspann für Korrekturansicht und Leseansicht.
%% Setzt den Schalter \ifkorrekturansicht voraus (gesetzt in den
%% einbindenden Dateien latex-korrekturansicht-abspann.tex bzw.
%% latex-leseansicht-abspann.tex).
%% ---------------------------------------------------------------

\normalsize

% Das esempio-Environment wird nur in der Leseansicht benötigt
\ifkorrekturansicht\else
\newenvironment{esempio}[3]%
{
    \vspace{1.5ex}
    \rlap{\underline{#1}}
    \par
    \setlength{\parindent}{0cm}
    \nopagebreak
    \leftskip=#2cm
    \rightskip=#3cm
}
{
    \par
}
\fi

\doendnotes{C}
\bigskip
\vfill

\clearpage

\footnotesize

\ifkorrekturansicht
  \lohead{\textsc{register}}
\fi

% theindex-Environment neu definieren ohne reledmac
\makeatletter
\renewenvironment{theindex}{%
  \ifkorrekturansicht
    \section*{\indexname}%
  \else
    \subsubsection*{Index der erwähnten Entitäten}%
  \fi
  \setlength{\parindent}{0pt}%
  \setlength{\parskip}{0pt plus 0.3pt}%
  \let\item\@idxitem
}{%
  \ifkorrekturansicht\clearpage\fi
}
\makeatother

\IfFileExists{\jobname-pw.ind}{\input{\jobname-pw.ind}}{}

% Quellenangabe nur in der Leseansicht
\ifkorrekturansicht\else
% Fallback-Definitionen, falls die .tex-Datei \titel etc. nicht gesetzt hat
\providecommand{\titel}{}
\providecommand{\editorInnen}{}
\providecommand{\dateiname}{\jobname}

\vspace{3cm}

\vfill

\footnotesize
\textsc{Quelle}: \titel. Herausgegeben von {\editorInnen}. In: \emph{Arthur Schnitzler: Briefwechsel mit Autorinnen und Autoren}.
 Digitale Edition, https://schnitzler-briefe.acdh.oeaw.ac.at/{\dateiname}.html (Stand \today)
\fi

\end{document}


      