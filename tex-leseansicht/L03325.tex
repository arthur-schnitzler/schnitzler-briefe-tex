%% latex-korrekturansicht-vorspann.tex
%% Vorspann für die Korrekturansicht.
%% Lädt die gemeinsame Datei latex-vorspann.tex mit gesetztem Schalter.

\newif\ifkorrekturansicht
\korrekturansichttrue

\input{../tex-inputs/latex-vorspann}


\section[ Felix Salten an Arthur Schnitzler, {[}11. 3. 1902{]}]{L03325 Felix Salten an Arthur Schnitzler, {[}11. 3. 1902{]}}
\nopagebreak\mylabel{L03325v}
\rehead{ }\normalsize\beginnumbering\briefempfaengerindex{Schnitzler, Arthur@\textsc{Schnitzler, Arthur}!zzzSalten, Felix@\emph{von Felix Salten}!1902-03-111@{{[}11. 3. 1902{]}}|(be}
\toendnotes[C]{\smallbreak\pagebreak[2]}\Standort{CUL, Schnitzler, B 89, A 2.}
\physDesc{Brief, 1 Blatt, 1 Seite, 281 Zeichen
\newline{}Handschrift: Bleistift, lateinische Kurrent
\newline{}Schnitzler: mit Bleistift datiert: »11/3 902« 
\newline{}Ordnung: mit Bleistift von unbekannter Hand nummeriert: »149« }\toendnotes[C]{\smallbreak}
\pstart
           \noindent{}{\pb}Lieber,{ }Otti\pwindex{Salten, Ottilie 07.03.1868 – 22.06.1942@\textsc{Salten, Ottilie} (07.03.1868 – 22.06.1942), \emph{Schauspieler/Schauspielerin}|pw} ist ausgegangen und dem Mädchen\pwindex{?? [Haushaltshilfe von Felix Salten in der Kochgasse 1902] @\textsc{?? [Haushaltshilfe von Felix Salten in der Kochgasse 1902]}|pwv} wurde gesagt, es solle nicht »alle
               Leute« zu mir laßen. Diese Gans hat keine bessere \label{K_L03325-1v}\edtext{Ausrede}{\lemma{\textnormal{\emph{Ausrede}}}\Cendnote{\textnormal{Schnitzler dürfte nach dem Schreiben vom
                  Vortag (Felix Salten an Arthur Schnitzler, [10?. 3. 1902]) einen Krankenbesuch
                  versucht haben und abgewiesen worden sein. Am 14. 3. 1902, als es Salten\pwindex{Salten, Felix 06.09.1869 – 08.10.1945@\textsc{Salten, Felix} (06.09.1869 – 08.10.1945), \emph{Schriftsteller/Schriftstellerin, Journalist/Journalistin, Chefredakteur/Chefredakteurin}|pwk} bereits besser ging, machte er neuerlich einen Besuch.}}}\label{K_L03325-1} gewußt,
               als mich spazieren zu schicken. Ich bin \uline{natürlich}
               sehr zu Hause, d. h. im Bette, und hätte mich sehr gefreut Sie zu sehen.\pend
           
\pstart
           Herzlichst Ihr {\\[\baselineskip]}\spacefill\mbox{Salten}\pend
           \leftskip=0em{}\selectlanguage{ngerman}\endnumbering\briefempfaengerindex{Schnitzler, Arthur@\textsc{Schnitzler, Arthur}!zzzSalten, Felix@\emph{von Felix Salten}!1902-03-111@{{[}11. 3. 1902{]}}|)be}\mylabel{L03325h}  \normalsize

\doendnotes{C}
\bigskip
\vfill

\clearpage

\footnotesize

\lohead{\textsc{register}}

% Definiere theindex-Environment komplett neu ohne reledmac
\makeatletter
\renewenvironment{theindex}{%
  \section*{\indexname}%
  \setlength{\parindent}{0pt}%
  \setlength{\parskip}{0pt plus 0.3pt}%
  \let\item\@idxitem
}{%
  \clearpage
}
\makeatother

\IfFileExists{\jobname-pw.ind}{\input{\jobname-pw.ind}}{}

\end{document}

      