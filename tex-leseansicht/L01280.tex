%% latex-leseansicht-vorspann.tex
%% Vorspann für die Leseansicht.
%% Lädt die gemeinsame Datei latex-vorspann.tex mit nicht gesetztem Schalter.

\newif\ifkorrekturansicht
\korrekturansichtfalse

\input{../tex-inputs/latex-vorspann}


\section[Arthur Schnitzler an Richard Beer-Hofmann, 27. 3. 1903]{L01280 Arthur Schnitzler an Richard Beer-Hofmann, 27. 3. 1903}
\nopagebreak\mylabel{L01280v}
\rehead{ }\normalsize\beginnumbering\briefempfaengerindex{Beer-Hofmann, Richard@\textsc{Beer-Hofmann, Richard}!zzzSchnitzler, Arthur@\emph{von Arthur Schnitzler}!1903-03-271@{27. 3. 1903}|(be}
\toendnotes[C]{\smallbreak\pagebreak[2]}
\correspDesc{Versand  durch Arthur Schnitzler am 27. 3. 1903 in Wien
\newline{}Erhalt  durch Richard Beer-Hofmann am 27. 3. 1903 in Rodaun}\toendnotes[C]{\smallbreak}
\Standort{YCGL, MSS 31.}
\physDesc{Brief, 1 Blatt, 3 Seiten, Kuvert, 642 Zeichen
\newline{}Handschrift: Bleistift, deutsche Kurrent
\newline{}Versand: 1) Stempel: »\nobreak{}\oindex{IX., Alsergrund@\textbf{IX., Alsergrund}, \emph{Verwaltungsgebiet}|pwk}9/3 Wien, 27. 3. \textcolor{gray}{03}, 11–12V\nobreak{}«.   2) Stempel: »\nobreak{}\oindex{Wien@\textbf{Wien}!XXIII., Liesing@\textbf{XXIII., Liesing}!Rodaun@\textbf{Rodaun}, \emph{Region}|pwk}{\pb}Rodaun, 27. 3. 03, 11–12V\nobreak{}«. }
\buchAbdrucke{\weitereDrucke{Arthur Schnitzler, Richard Beer-Hofmann: \emph{Briefwechsel 1891–1931}. Herausgegeben von Konstanze Fliedl. Wien, Zürich: \emph{Europaverlag} 1992, S. 162.} }\toendnotes[C]{\smallbreak}\pstart{}{\pb}Herrn\pend{}\pstart{}\textsc{Dr. Richard Beer-Hofmann}\pend{}\pstart{}Rodaun\oindex{Wien@\textbf{Wien}!XXIII., Liesing@\textbf{XXIII., Liesing}!Rodaun@\textbf{Rodaun}, \emph{Region}|pw}\pend{}\pstart{}bei Lieſing\oindex{XXIII., Liesing@\textbf{XXIII., Liesing}, \emph{Verwaltungsgebiet}|pw}\pend{}\pstart{}Lieſinger Straße 2\oindex{Liesingerstraße@\textbf{Liesingerstraße}, \emph{Straße}|pw}. \pend{}{\bigskip}\vspace{1em}
\pstart
           \raggedleft{}{\pb}27./3 903.\pend
           
\pstart{}mein lieber Richard,\pend\vspace{0.5em}
\pstart
           Lear\pwindex{Shakespeare, William 23.\,4.\,1564? Stratford-upon-Avon – 3.\,5.\,1616 ebd.@\textsc{Shakespeare, William} (23.\,4.\,1564? Stratford-upon-Avon – 3.\,5.\,1616 ebd.), \emph{Schauspieler, Dramatiker}!König Lear. Trauerspiel in fünf Aufzügen@\strich\emph{König Lear. Trauerspiel in fünf Aufzügen}|pw} hab ich \label{K_L01280-1v}\edtext{heuer}{\lemma{\textnormal{\emph{heuer}}}\Cendnote{\textnormal{Gemeint ist
                  die Theatersaison, vgl. A. S.: \emph{Tagebuch}, 28. 9. 1902.}}}\label{K_L01280-1}{ }ſchon einmal geſehen; übrigens{ }ſind fünf in einer Loge zu viel,
                  un\textcolor{gray}{d} man hätte weder was von \textsc{Shakespeare}\pwindex{Shakespeare, William 23.\,4.\,1564? Stratford-upon-Avon – 3.\,5.\,1616 ebd.@\textsc{Shakespeare, William} (23.\,4.\,1564? Stratford-upon-Avon – 3.\,5.\,1616 ebd.), \emph{Schauspieler, Dramatiker}|pw} noch von einander\pend
           
\pstart
           Man könnte{ }ſich{ }ſchon viel öfter{ }ſehen, we{\geminationn} man nicht{ }ſo{ }ſchwerfällig wäre, was nicht {\pb}nur auf Sie,{ }ſondern
               eigentlich viel mehr auf mich geht. Übrigens hab ich von Tag zu Tag irgend was
               telephonisches von Ihnen erwartet. Auch denk ich im Laufe der nächſten Woche einmal,
                  Vormittags, vielleicht mit Olga\pwindex{Schnitzler, Olga 17.\,1.\,1882 Wien – 13.\,1.\,1970 Lugano@\textsc{Schnitzler, Olga} (17.\,1.\,1882 Wien – 13.\,1.\,1970 Lugano), \emph{Schauspielerin, Sängerin}|pw}, in Rodaun\oindex{Wien@\textbf{Wien}!XXIII., Liesing@\textbf{XXIII., Liesing}!Rodaun@\textbf{Rodaun}, \emph{Region}|pw} aufzutauchen.\pend
           
\pstart
           Grüß Sie Gott und verſichern {\pb}Sie Hugo\pwindex{Hofmannsthal, Hugo von 1.\,2.\,1874 Wien – 15.\,7.\,1929 Rodaun@\textsc{Hofmannsthal, Hugo von} (1.\,2.\,1874 Wien – 15.\,7.\,1929 Rodaun), \emph{Schriftsteller}|pw}, dem begabten \label{K_L01280-2v}\edtext{Adreſſenschreiber}{\lemma{\textnormal{\emph{Adressenschreiber}}}\Cendnote{\textnormal{Die Adressierung des Briefes vom XXXX Auszeichnungsfehler: Dokument L01279 nicht gefunden stammte von Hofmannsthal\pwindex{Hofmannsthal, Hugo von 1.\,2.\,1874 Wien – 15.\,7.\,1929 Rodaun@\textsc{Hofmannsthal, Hugo von} (1.\,2.\,1874 Wien – 15.\,7.\,1929 Rodaun), \emph{Schriftsteller}|pwk}.}}}\label{K_L01280-2}, das gleiche.\pend
           
\pstart
           Der Ihrige,{\\[\baselineskip]}\spacefill\mbox{A.}\pend
           \leftskip=0em{}\selectlanguage{ngerman}\endnumbering\briefempfaengerindex{Beer-Hofmann, Richard@\textsc{Beer-Hofmann, Richard}!zzzSchnitzler, Arthur@\emph{von Arthur Schnitzler}!1903-03-271@{27. 3. 1903}|)be}\mylabel{L01280h}  \newcommand{\dateiname}{L01280}\newcommand{\titel}{Arthur Schnitzler an Richard Beer-Hofmann, 27. 3. 1903}\newcommand{\editorInnen}{Martin Anton Müller und Gerd-Hermann Susen}%% latex-leseansicht-abspann.tex
%% Abspann für die Leseansicht.
%% Der Schalter \ifkorrekturansicht ist bereits durch den Vorspann gesetzt.

%% latex-abspann.tex
%% Gemeinsamer Abspann für Korrekturansicht und Leseansicht.
%% Setzt den Schalter \ifkorrekturansicht voraus (gesetzt in den
%% einbindenden Dateien latex-korrekturansicht-abspann.tex bzw.
%% latex-leseansicht-abspann.tex).
%% ---------------------------------------------------------------

\normalsize

% Das esempio-Environment wird nur in der Leseansicht benötigt
\ifkorrekturansicht\else
\newenvironment{esempio}[3]%
{
    \vspace{1.5ex}
    \rlap{\underline{#1}}
    \par
    \setlength{\parindent}{0cm}
    \nopagebreak
    \leftskip=#2cm
    \rightskip=#3cm
}
{
    \par
}
\fi

\doendnotes{C}
\bigskip
\vfill

\clearpage

\footnotesize

\ifkorrekturansicht
  \lohead{\textsc{register}}
\fi

% theindex-Environment neu definieren ohne reledmac
\makeatletter
\renewenvironment{theindex}{%
  \ifkorrekturansicht
    \section*{\indexname}%
  \else
    \subsubsection*{Index der erwähnten Entitäten}%
  \fi
  \setlength{\parindent}{0pt}%
  \setlength{\parskip}{0pt plus 0.3pt}%
  \let\item\@idxitem
}{%
  \ifkorrekturansicht\clearpage\fi
}
\makeatother

\IfFileExists{\jobname-pw.ind}{\input{\jobname-pw.ind}}{}

% Quellenangabe nur in der Leseansicht
\ifkorrekturansicht\else
% Fallback-Definitionen, falls die .tex-Datei \titel etc. nicht gesetzt hat
\providecommand{\titel}{}
\providecommand{\editorInnen}{}
\providecommand{\dateiname}{\jobname}

\vspace{3cm}

\vfill

\footnotesize
\textsc{Quelle}: \titel. Herausgegeben von {\editorInnen}. In: \emph{Arthur Schnitzler: Briefwechsel mit Autorinnen und Autoren}.
 Digitale Edition, https://schnitzler-briefe.acdh.oeaw.ac.at/{\dateiname}.html (Stand \today)
\fi

\end{document}


