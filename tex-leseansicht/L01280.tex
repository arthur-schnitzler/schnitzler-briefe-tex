%% latex-korrekturansicht-vorspann.tex
%% Vorspann für die Korrekturansicht.
%% Lädt die gemeinsame Datei latex-vorspann.tex mit gesetztem Schalter.

\newif\ifkorrekturansicht
\korrekturansichttrue

\input{../tex-inputs/latex-vorspann}


\section[Arthur Schnitzler an Richard Beer-Hofmann, 27. 3. 1903]{L01280 Arthur Schnitzler an Richard Beer-Hofmann, 27. 3. 1903}
\nopagebreak\mylabel{L01280v}
\rehead{ }\normalsize\beginnumbering\briefempfaengerindex{Beer-Hofmann, Richard@\textsc{Beer-Hofmann, Richard}!zzzSchnitzler, Arthur@\emph{von Arthur Schnitzler}!1903-03-271@{27. 3. 1903}|(be}
\toendnotes[C]{\smallbreak\pagebreak[2]}\Standort{YCGL, MSS 31.}
\physDesc{Brief, 1 Blatt, 3 Seiten, Umschlag, 642 Zeichen
\newline{}Handschrift: Bleistift, deutsche Kurrent
\newline{}Versand: 1) Stempel: »\nobreak{}\oindex{IX., Alsergrund@\textbf{IX., Alsergrund}, \emph{A.ADM3}|pwk}9/3 Wien, 27. 3. \textcolor{gray}{03}, 11–12V\nobreak{}«.   2) Stempel: »\nobreak{}\oindex{Rodaun@\textbf{Rodaun}, \emph{A.ADM4}|pwk}{\pb}Rodaun, 27. 3. 03, 11–12V\nobreak{}«. }
\buchAbdrucke{\weitereDrucke{Arthur Schnitzler, Richard Beer-Hofmann: \emph{Briefwechsel 1891–1931}. Wien, Zürich: \emph{Europaverlag} 1992, S. 162.} }\toendnotes[C]{\smallbreak}\pstart{}{\pb}Herrn\pend{}\pstart{}\textsc{Dr. Richard Beer-Hofmann}\pend{}\pstart{}Rodaun\oindex{Rodaun@\textbf{Rodaun}, \emph{A.ADM4}|pw}\pend{}\pstart{}bei Lieſing\oindex{XXIII., Liesing@\textbf{XXIII., Liesing}, \emph{A.ADM3}|pw}\pend{}\pstart{}Lieſinger Straße 2\oindex{Liesingerstrasse@\textbf{Liesingerstraße}, \emph{Straße (K.STR)}|pw}. \pend{}{\bigskip}\vspace{1em}
\pstart
           \raggedleft{}{\pb}27./3 903.\pend
           
\pstart{}mein lieber Richard,\pend\vspace{0.5em}
\pstart
           Lear\pwindex{Koenig Lear. Trauerspiel in fuenf Aufzuegen@\emph{König Lear. Trauerspiel in fünf Aufzügen}|pw} hab ich \label{K_L01280-1v}\edtext{heuer}{\lemma{\textnormal{\emph{heuer}}}\Cendnote{\textnormal{Gemeint ist
                  die Theatersaison, vgl. A. S.: \emph{Tagebuch}, 28. 9. 1902.}}}\label{K_L01280-1} ſchon einmal geſehen; übrigens ſind fünf in einer Loge zu viel,
                  un\textcolor{gray}{d} man hätte weder was von \textsc{Shakespeare}\pwindex{Shakespeare, William 23.4.1564? – 03.05.1616@\textsc{Shakespeare, William} (23.4.1564? – 03.05.1616), \emph{Schauspieler/Schauspielerin, Dramatiker/Dramatikerin}|pw} noch von einander\pend
           
\pstart
           Man könnte ſich ſchon viel öfter ſehen, we{\geminationn} man nicht ſo
               ſchwerfällig wäre, was nicht {\pb}nur auf Sie, ſondern
               eigentlich viel mehr auf mich geht. Übrigens hab ich von Tag zu Tag irgend was
               telephonisches von Ihnen erwartet. Auch denk ich im Laufe der nächſten Woche einmal,
                  Vormittags, vielleicht mit Olga\pwindex{Schnitzler, Olga 17.01.1882 – 13.01.1970@\textsc{Schnitzler, Olga} (17.01.1882 – 13.01.1970), \emph{Schauspieler/Schauspielerin, Sänger/Sängerin}|pw}, in Rodaun\oindex{Rodaun@\textbf{Rodaun}, \emph{A.ADM4}|pw} aufzutauchen.\pend
           
\pstart
           Grüß Sie Gott und verſichern {\pb}Sie Hugo\pwindex{Hofmannsthal, Hugo von 1874-02-01 – 1929-07-15@\textsc{Hofmannsthal, Hugo von} (1874-02-01 – 1929-07-15), \emph{Schriftsteller/Schriftstellerin}|pw}, dem begabten \label{K_L01280-2v}\edtext{Adreſſenschreiber}{\lemma{\textnormal{\emph{Adreſſenschreiber}}}\Cendnote{\textnormal{Die Adressierung des Briefes vom 26. 3. 1903 stammte von Hofmannsthal\pwindex{Hofmannsthal, Hugo von 1874-02-01 – 1929-07-15@\textsc{Hofmannsthal, Hugo von} (1874-02-01 – 1929-07-15), \emph{Schriftsteller/Schriftstellerin}|pwk}.}}}\label{K_L01280-2}, das gleiche.\pend
           
\pstart
           Der Ihrige,{\\[\baselineskip]}\spacefill\mbox{A.}\pend
           \leftskip=0em{}\selectlanguage{ngerman}\endnumbering\briefempfaengerindex{Beer-Hofmann, Richard@\textsc{Beer-Hofmann, Richard}!zzzSchnitzler, Arthur@\emph{von Arthur Schnitzler}!1903-03-271@{27. 3. 1903}|)be}\mylabel{L01280h}  \normalsize

\doendnotes{C}
\bigskip
\vfill

\clearpage

\footnotesize

\lohead{\textsc{register}}

% Definiere theindex-Environment komplett neu ohne reledmac
\makeatletter
\renewenvironment{theindex}{%
  \section*{\indexname}%
  \setlength{\parindent}{0pt}%
  \setlength{\parskip}{0pt plus 0.3pt}%
  \let\item\@idxitem
}{%
  \clearpage
}
\makeatother

\IfFileExists{\jobname-pw.ind}{\input{\jobname-pw.ind}}{}

\end{document}

      