%% latex-korrekturansicht-vorspann.tex
%% Vorspann für die Korrekturansicht.
%% Lädt die gemeinsame Datei latex-vorspann.tex mit gesetztem Schalter.

\newif\ifkorrekturansicht
\korrekturansichttrue

\input{../tex-inputs/latex-vorspann}


\section[ Paul Goldmann an Arthur Schnitzler, 16. 4. {[}1906{]}]{L03243 Paul Goldmann an Arthur Schnitzler, 16. 4. {[}1906{]}}
\nopagebreak\mylabel{L03243v}
\rehead{ }\normalsize\beginnumbering\briefempfaengerindex{Schnitzler, Arthur@\textsc{Schnitzler, Arthur}!zzzGoldmann, Paul@\emph{von Paul Goldmann}!1906-04-161@{16. 4. {[}1906{]}}|(be}
\toendnotes[C]{\smallbreak\pagebreak[2]}\Standort{DLA, A:Schnitzler, HS.NZ85.1.3175.}
\physDesc{Brief, 1 Blatt, 4 Seiten, 2185 Zeichen
\newline{}Handschrift: 1) schwarze Tinte, deutsche Kurrent (\noindent{}erster Absatz)\hspace{1em}2) blaue Tinte, deutsche Kurrent\hspace{1em}
\newline{}Schnitzler: 1) mit Bleistift »Mitte April
                                          906« vermerkt  2) mit rotem Buntstift drei Unterstreichungen}\toendnotes[C]{\smallbreak}
\pstart
           \raggedleft{}{\pb}\textcolor{gray}{\textbf{FRANKFURT a. M. \oindex{Frankfurt am Main@\textbf{Frankfurt am Main}, \emph{P.PPLA3}|pw}, Reuterweg 59\oindex{Reuterweg@\textbf{Reuterweg}, \emph{Straße (K.STR)}|pw}}}\pend
           
\pstart
           16. April.\pend
           \vspace{0.5em}
\pstart
           Lieber Freund, Ich danke Dir für Deinen lieben Brief,
               den ich kurz vor meiner \label{K_L03243-1v}\edtext{Abreiſe aus
                  Berlin\oindex{Berlin@\textbf{Berlin}, \emph{P.PPLC}|pw}}{\lemma{\textnormal{\emph{Abreiſe aus
                  Berlin}}}\Cendnote{\textnormal{vermutlich zwischen 10. und 15. 4. 1906}}}\label{K_L03243-1} erhielt, und komme mit einer großen Bitte.\pend
           
\pstart
           Hier\oindex{Frankfurt am Main@\textbf{Frankfurt am Main}, \emph{P.PPLA3}|pwv} habe ich eine ganz
               verzweifelte Situation vorgefunden. Die \label{K_L03243-2v}\edtext{Operation}{\lemma{\textnormal{\emph{Operation}}}\Cendnote{\textnormal{Siehe Paul Goldmann an Arthur Schnitzler, 9. 4. [1906].
               }}}\label{K_L03243-2} iſt verſucht worden. Man hat aber nach der Bauchöffnung konſtatirt, daß der
                  \textsc{tumor} an einer Stelle des Darms ſitzt, an die man nicht
               herankomme, weder von oben, noch von unten. Die Ärzte haben ſich alſo entſchloſſen,
               wieder zuzunähen, ohne etwas gemacht zu haben. Der Patient\pwindex{Mamroth, Fedor 21.02.1851 – 25.06.1907@\textsc{Mamroth, Fedor} (21.02.1851 – 25.06.1907), \emph{Journalist/Journalistin, Kritiker/Kritikerin}|pwv} ahnt das nicht und glaubt, er ſei
               mit Erfolg von einem gutartigen \textsc{tumor} operirt worden. Nur
               die Ärzte und ich wiſſen, daß er verloren iſt. {\pb}Mein
                  Schwager\pwindex{Rosengart, Josef 1860-02-08 – 1927-08-04@\textsc{Rosengart, Josef} (1860-02-08 – 1927-08-04), \emph{Arzt/Ärztin}|pwv}, der ein ebenſo
               bedeutender als bedachtſamer Arzt iſt, hat alle Eventualitäten in Betracht gezogen.
               Es gibt eine Operation, die \textsc{Kraske\pwindex{Kraske, Paul 1851-06-02 – 1930-06-15@\textsc{Kraske, Paul} (1851-06-02 – 1930-06-15), \emph{Chirurg/Chirurgin, Hochschullehrer/Hochschullehrerin, Offizier/Offizierin}|pw}} in Freiburg\oindex{Freiburg im Breisgau@\textbf{Freiburg im Breisgau}, \emph{P.PPLA2}|pw} macht und die an Geſchwüre, die
               an dieſer Stelle ſitzen, von hinten auf dem Wege der Durchmeißelung eines Knochens
               herankommt. Da aber der Erfolg dieſer Operation ſehr fraglich iſt und ſie zumeiſt zur
               Bildung einer Darmfiſtel führt, hat mein Schwager\pwindex{Rosengart, Josef 1860-02-08 – 1927-08-04@\textsc{Rosengart, Josef} (1860-02-08 – 1927-08-04), \emph{Arzt/Ärztin}|pwv}, um den Patient\pwindex{Mamroth, Fedor 21.02.1851 – 25.06.1907@\textsc{Mamroth, Fedor} (21.02.1851 – 25.06.1907), \emph{Journalist/Journalistin, Kritiker/Kritikerin}|pwv}en in ſeiner letzten Lebenszeit nicht unnötigen Qualen
               auszuſetzen, ſich entſchloſſen, auf dieſe Operation zu verzichten und will einfach
               das Unvermeidliche geſchehen laſſen.\pend
           
\pstart
           In dieſe Reſignation des Arzt\pwindex{Rosengart, Josef 1860-02-08 – 1927-08-04@\textsc{Rosengart, Josef} (1860-02-08 – 1927-08-04), \emph{Arzt/Ärztin}|pwv}es {\pb}mich hineinzufinden, iſt für mich
               unendlich ſchwer, – die Idee, daß da ein Menſch\pwindex{Mamroth, Fedor 21.02.1851 – 25.06.1907@\textsc{Mamroth, Fedor} (21.02.1851 – 25.06.1907), \emph{Journalist/Journalistin, Kritiker/Kritikerin}|pwv} liegt, den man liebt, und man \strikeout{oh\textcolor{gray}{n}} ohnmächtig zuſehen ſoll, wie er zu Grunde geht, vermag ich nicht zu faſſen. Im
               Grübeln über Rettungs-Möglichkeiten iſt mir Dein Bruder\pwindex{Schnitzler, Julius 13.07.1865 – 29.06.1939@\textsc{Schnitzler, Julius} (13.07.1865 – 29.06.1939), \emph{Chirurg/Chirurgin}|pwv} eingefallen, der ja ein ſo bedeutender Chirurg iſt,
               und ich bitte Dich nun recht ſehr, ihm \strikeout{d\textcolor{gray}{e}} den Fall zu erzählen und ihn zu fragen, wie er darüber denkt, was er thun
               würde und ob er nicht irgend einen \label{K_L03243-3v}\edtext{Rat}{\lemma{\textnormal{\emph{Rat}}}\Cendnote{\textnormal{Siehe Paul Goldmann an Arthur Schnitzler, 20. 4. [1906].
               }}}\label{K_L03243-3} weiß? Grüße ihn von mir und danke ihm in meinem Namen für Alles, was er ſagen
               und thun könnte. Und ſei auch Du vielmals und {\pb}herzlichſt im Voraus bedankt! Nur bitte ich Dich, daß Du mir umgehend antworteſt
               (an die Adreſſe meines Schwagers\pwindex{Rosengart, Josef 1860-02-08 – 1927-08-04@\textsc{Rosengart, Josef} (1860-02-08 – 1927-08-04), \emph{Arzt/Ärztin}|pwv}, \textsc{Dr. Goldmann}, bei \textsc{Dr. \strikeout{Rosenga}}{ }\textsc{Rosengart\pwindex{Rosengart, Josef 1860-02-08 – 1927-08-04@\textsc{Rosengart, Josef} (1860-02-08 – 1927-08-04), \emph{Arzt/Ärztin}|pw}}, das Weitere ſteht am Kopf des Briefes), da ich nur noch wenige Tage hier\oindex{Frankfurt am Main@\textbf{Frankfurt am Main}, \emph{P.PPLA3}|pwv} bleiben kann.\pend
           
\pstart
           Daß ich Dir das Alles nur im ſtrengſten Vertrauen mitteile, brauche ich ja nicht erſt
               zu ſagen.\pend
           
\pstart
           Viele treue Grüße! {\\[\baselineskip]}Dein \spacefill\mbox{Paul Goldmann.}\pend
           \leftskip=0em{}\selectlanguage{ngerman}\endnumbering\briefempfaengerindex{Schnitzler, Arthur@\textsc{Schnitzler, Arthur}!zzzGoldmann, Paul@\emph{von Paul Goldmann}!1906-04-161@{16. 4. {[}1906{]}}|)be}\mylabel{L03243h}  \normalsize

\doendnotes{C}
\bigskip
\vfill

\clearpage

\footnotesize

\lohead{\textsc{register}}

% Definiere theindex-Environment komplett neu ohne reledmac
\makeatletter
\renewenvironment{theindex}{%
  \section*{\indexname}%
  \setlength{\parindent}{0pt}%
  \setlength{\parskip}{0pt plus 0.3pt}%
  \let\item\@idxitem
}{%
  \clearpage
}
\makeatother

\IfFileExists{\jobname-pw.ind}{\input{\jobname-pw.ind}}{}

\end{document}

      