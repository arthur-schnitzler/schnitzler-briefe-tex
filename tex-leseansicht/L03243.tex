%% latex-leseansicht-vorspann.tex
%% Vorspann für die Leseansicht.
%% Lädt die gemeinsame Datei latex-vorspann.tex mit nicht gesetztem Schalter.

\newif\ifkorrekturansicht
\korrekturansichtfalse

\input{../tex-inputs/latex-vorspann}


\section[ Paul Goldmann an Arthur Schnitzler, 16. 4. [1906]]{L03243 Paul Goldmann an Arthur Schnitzler,  16. 4. [1906]}
\nopagebreak\mylabel{L03243v}
\rehead{ }\normalsize\beginnumbering\briefempfaengerindex{Schnitzler, Arthur@\textsc{Schnitzler, Arthur}!zzzGoldmann, Paul@\emph{von Paul Goldmann}!1906-04-161@{16. 4. [1906]}|(be}
\toendnotes[C]{\smallbreak\pagebreak[2]}
\correspDesc{Versand  durch Paul Goldmann am 16. 4. [1906] in Frankfurt am Main
\newline{}Erhalt  durch Arthur Schnitzler im Zeitraum [17. 4. 1906
                  – 21. 4. 1906?] in Wien}\toendnotes[C]{\smallbreak}
\Standort{DLA, A:Schnitzler, HS.NZ85.1.3175.}
\physDesc{Brief, 1 Blatt, 4 Seiten, 2185 Zeichen
\newline{}Handschrift: 1) schwarze Tinte, deutsche Kurrent (\noindent{}erster Absatz)\hspace{1em}2) blaue Tinte, deutsche Kurrent\hspace{1em}
\newline{}Schnitzler: 1) mit Bleistift »Mitte April 906« vermerkt  2) mit rotem Buntstift drei Unterstreichungen}\toendnotes[C]{\smallbreak}
\pstart
           \raggedleft{}{\pb}\textcolor{gray}{\textbf{FRANKFURT a. M.\oindex{Frankfurt am Main@\textbf{Frankfurt am Main}, \emph{Hauptstadt}|pw}, Reuterweg 59\oindex{Reuterweg@\textbf{Reuterweg}, \emph{Straße}|pw}}}\pend
           
\pstart
           16. April.\pend
           \vspace{0.5em}
\pstart
           Lieber Freund, Ich danke Dir für Deinen lieben Brief,
               den ich kurz vor meiner \label{K_L03243-1v}\edtext{Abreiſe aus
                  Berlin\oindex{Berlin@\textbf{Berlin}, \emph{Hauptstadt}|pw}}{\lemma{\textnormal{\emph{Abreise aus
                  Berlin}}}\Cendnote{\textnormal{vermutlich zwischen 10. und 15. 4. 1906}}}\label{K_L03243-1} erhielt, und komme mit einer großen Bitte.\pend
           
\pstart
           Hier\oindex{Frankfurt am Main@\textbf{Frankfurt am Main}, \emph{Hauptstadt}|pwv} habe ich eine ganz
               verzweifelte Situation vorgefunden. Die \label{K_L03243-2v}\edtext{Operation}{\lemma{\textnormal{\emph{Operation}}}\Cendnote{\textnormal{Siehe XXXX Auszeichnungsfehler: Dokument L03242 nicht gefunden.
               }}}\label{K_L03243-2} iſt verſucht worden. Man hat aber nach der Bauchöffnung konſtatirt, daß der
                  \textsc{tumor} an einer Stelle des Darms{ }ſitzt, an die man nicht
               herankomme, weder von oben, noch von unten. Die Ärzte haben{ }ſich alſo entſchloſſen,
               wieder zuzunähen, ohne etwas gemacht zu haben. Der Patient\pwindex{Mamroth, Fedor 21.\,2.\,1851 Breslau – 25.\,6.\,1907 Frankfurt am Main@\textsc{Mamroth, Fedor} (21.\,2.\,1851 Breslau – 25.\,6.\,1907 Frankfurt am Main), \emph{Journalist, Kritiker}|pwv} ahnt das nicht und glaubt, er{ }ſei
               mit Erfolg von einem gutartigen \textsc{tumor} operirt worden. Nur
               die Ärzte und ich wiſſen, daß er verloren iſt. {\pb}Mein
                  Schwager\pwindex{Rosengart, Josef 8.\,2.\,1860 Laupheim – 4.\,8.\,1927 Frankfurt am Main@\textsc{Rosengart, Josef} (8.\,2.\,1860 Laupheim – 4.\,8.\,1927 Frankfurt am Main), \emph{Arzt}|pwv}, der ein ebenſo
               bedeutender als bedachtſamer Arzt iſt, hat alle Eventualitäten in Betracht gezogen.
               Es gibt eine Operation, die \textsc{Kraske\pwindex{Kraske, Paul 2.\,6.\,1851 Berg (Bad Muskau) – 15.\,6.\,1930 Freiburg im Breisgau@\textsc{Kraske, Paul} (2.\,6.\,1851 Berg (Bad Muskau) – 15.\,6.\,1930 Freiburg im Breisgau), \emph{Chirurg, Hochschullehrer, Offizier}|pw}} in Freiburg\oindex{Freiburg im Breisgau@\textbf{Freiburg im Breisgau}, \emph{Hauptstadt}|pw} macht und die an Geſchwüre, die
               an dieſer Stelle{ }ſitzen, von hinten auf dem Wege der Durchmeißelung eines Knochens
               herankommt. Da aber der Erfolg dieſer Operation{ }ſehr fraglich iſt und{ }ſie zumeiſt zur
               Bildung einer Darmfiſtel führt, hat mein Schwager\pwindex{Rosengart, Josef 8.\,2.\,1860 Laupheim – 4.\,8.\,1927 Frankfurt am Main@\textsc{Rosengart, Josef} (8.\,2.\,1860 Laupheim – 4.\,8.\,1927 Frankfurt am Main), \emph{Arzt}|pwv}, um den Patient\pwindex{Mamroth, Fedor 21.\,2.\,1851 Breslau – 25.\,6.\,1907 Frankfurt am Main@\textsc{Mamroth, Fedor} (21.\,2.\,1851 Breslau – 25.\,6.\,1907 Frankfurt am Main), \emph{Journalist, Kritiker}|pwv}en in{ }ſeiner letzten Lebenszeit nicht unnötigen Qualen
               auszuſetzen,{ }ſich entſchloſſen, auf dieſe Operation zu verzichten und will einfach
               das Unvermeidliche geſchehen laſſen.\pend
           
\pstart
           In dieſe Reſignation des Arzt\pwindex{Rosengart, Josef 8.\,2.\,1860 Laupheim – 4.\,8.\,1927 Frankfurt am Main@\textsc{Rosengart, Josef} (8.\,2.\,1860 Laupheim – 4.\,8.\,1927 Frankfurt am Main), \emph{Arzt}|pwv}es {\pb}mich hineinzufinden, iſt für mich
               unendlich{ }ſchwer, – die Idee, daß da ein Menſch\pwindex{Mamroth, Fedor 21.\,2.\,1851 Breslau – 25.\,6.\,1907 Frankfurt am Main@\textsc{Mamroth, Fedor} (21.\,2.\,1851 Breslau – 25.\,6.\,1907 Frankfurt am Main), \emph{Journalist, Kritiker}|pwv} liegt, den man liebt, und man \strikeout{oh\textcolor{gray}{n}} ohnmächtig zuſehen{ }ſoll, wie er zu Grunde geht, vermag ich nicht zu faſſen. Im
               Grübeln über Rettungs-Möglichkeiten iſt mir Dein Bruder\pwindex{Schnitzler, Julius 13.\,7.\,1865 Wien – 29.\,6.\,1939 ebd.@\textsc{Schnitzler, Julius} (13.\,7.\,1865 Wien – 29.\,6.\,1939 ebd.), \emph{Chirurg}|pwv} eingefallen, der ja ein{ }ſo bedeutender Chirurg iſt,
               und ich bitte Dich nun recht{ }ſehr, ihm \strikeout{d\textcolor{gray}{e}} den Fall zu erzählen und ihn zu fragen, wie er darüber denkt, was er thun
               würde und ob er nicht irgend einen \label{K_L03243-3v}\edtext{Rat}{\lemma{\textnormal{\emph{Rat}}}\Cendnote{\textnormal{Siehe XXXX Auszeichnungsfehler: Dokument L03244 nicht gefunden.
               }}}\label{K_L03243-3} weiß? Grüße ihn von mir und danke ihm in meinem Namen für Alles, was er{ }ſagen
               und thun könnte. Und{ }ſei auch Du vielmals und {\pb}herzlichſt im Voraus bedankt! Nur bitte ich Dich, daß Du mir umgehend antworteſt
               (an die Adreſſe meines Schwagers\pwindex{Rosengart, Josef 8.\,2.\,1860 Laupheim – 4.\,8.\,1927 Frankfurt am Main@\textsc{Rosengart, Josef} (8.\,2.\,1860 Laupheim – 4.\,8.\,1927 Frankfurt am Main), \emph{Arzt}|pwv}, \textsc{Dr. Goldmann}, bei \textsc{Dr. \strikeout{Rosenga}}{ }\textsc{Rosengart\pwindex{Rosengart, Josef 8.\,2.\,1860 Laupheim – 4.\,8.\,1927 Frankfurt am Main@\textsc{Rosengart, Josef} (8.\,2.\,1860 Laupheim – 4.\,8.\,1927 Frankfurt am Main), \emph{Arzt}|pw}}, das Weitere{ }ſteht am Kopf des Briefes), da ich nur noch wenige Tage hier\oindex{Frankfurt am Main@\textbf{Frankfurt am Main}, \emph{Hauptstadt}|pwv} bleiben kann.\pend
           
\pstart
           Daß ich Dir das Alles nur im{ }ſtrengſten Vertrauen mitteile, brauche ich ja nicht erſt
               zu{ }ſagen.\pend
           
\pstart
           Viele treue Grüße! {\\[\baselineskip]}Dein \spacefill\mbox{Paul Goldmann.}\pend
           \leftskip=0em{}\selectlanguage{ngerman}\endnumbering\briefempfaengerindex{Schnitzler, Arthur@\textsc{Schnitzler, Arthur}!zzzGoldmann, Paul@\emph{von Paul Goldmann}!1906-04-161@{16. 4. [1906]}|)be}\mylabel{L03243h}  \newcommand{\dateiname}{L03243}\newcommand{\titel}{Paul Goldmann an Arthur Schnitzler, 16. 4. [1906]}\newcommand{\editorInnen}{Martin Anton Müller und Laura Untner}%% latex-leseansicht-abspann.tex
%% Abspann für die Leseansicht.
%% Der Schalter \ifkorrekturansicht ist bereits durch den Vorspann gesetzt.

%% latex-abspann.tex
%% Gemeinsamer Abspann für Korrekturansicht und Leseansicht.
%% Setzt den Schalter \ifkorrekturansicht voraus (gesetzt in den
%% einbindenden Dateien latex-korrekturansicht-abspann.tex bzw.
%% latex-leseansicht-abspann.tex).
%% ---------------------------------------------------------------

\normalsize

% Das esempio-Environment wird nur in der Leseansicht benötigt
\ifkorrekturansicht\else
\newenvironment{esempio}[3]%
{
    \vspace{1.5ex}
    \rlap{\underline{#1}}
    \par
    \setlength{\parindent}{0cm}
    \nopagebreak
    \leftskip=#2cm
    \rightskip=#3cm
}
{
    \par
}
\fi

\doendnotes{C}
\bigskip
\vfill

\clearpage

\footnotesize

\ifkorrekturansicht
  \lohead{\textsc{register}}
\fi

% theindex-Environment neu definieren ohne reledmac
\makeatletter
\renewenvironment{theindex}{%
  \ifkorrekturansicht
    \section*{\indexname}%
  \else
    \subsubsection*{Index der erwähnten Entitäten}%
  \fi
  \setlength{\parindent}{0pt}%
  \setlength{\parskip}{0pt plus 0.3pt}%
  \let\item\@idxitem
}{%
  \ifkorrekturansicht\clearpage\fi
}
\makeatother

\IfFileExists{\jobname-pw.ind}{\input{\jobname-pw.ind}}{}

% Quellenangabe nur in der Leseansicht
\ifkorrekturansicht\else
% Fallback-Definitionen, falls die .tex-Datei \titel etc. nicht gesetzt hat
\providecommand{\titel}{}
\providecommand{\editorInnen}{}
\providecommand{\dateiname}{\jobname}

\vspace{3cm}

\vfill

\footnotesize
\textsc{Quelle}: \titel. Herausgegeben von {\editorInnen}. In: \emph{Arthur Schnitzler: Briefwechsel mit Autorinnen und Autoren}.
 Digitale Edition, https://schnitzler-briefe.acdh.oeaw.ac.at/{\dateiname}.html (Stand \today)
\fi

\end{document}


