%% latex-korrekturansicht-vorspann.tex
%% Vorspann für die Korrekturansicht.
%% Lädt die gemeinsame Datei latex-vorspann.tex mit gesetztem Schalter.

\newif\ifkorrekturansicht
\korrekturansichttrue

\input{../tex-inputs/latex-vorspann}


\section[Arthur Schnitzler an Georg Brandes, 21. 8. 1920]{L02355 Arthur Schnitzler an Georg Brandes, 21. 8. 1920}
\nopagebreak\mylabel{L02355v}
\rehead{ }\normalsize\beginnumbering\briefempfaengerindex{Brandes, Georg@\textsc{Brandes, Georg}!zzzSchnitzler, Arthur@\emph{von Arthur Schnitzler}!1920-08-211@{21. 8. 1920}|(be}
\toendnotes[C]{\smallbreak\pagebreak[2]}\Standort{Kopenhagen, Det Kongelige Bibliotek, Georg Brandes Arkiv, box 125.}
\physDesc{Postkarte, 862 Zeichen
\newline{}Handschrift: Bleistift, lateinische Kurrent
\newline{}Versand: Stempel: »\nobreak{}\oindex{XVIII., Waehring@\textbf{XVIII., Währing}, \emph{A.ADM3}|pwk}18/\textcolor{gray}{×} Wien, 21. VIII. 20, 4\nobreak{}«.  
\newline{}Ordnung: 1) mit Bleistift von unbekannter Hand links der Briefmarke nummeriert:
                                    »43«  2) mit Bleistift von unbekannter Hand auf der Textseite zusätzlich die Datierung
                                 wiederholt: »21/8 20«}
\buchAbdrucke{\weitereDrucke{Georg Brandes, Arthur Schnitzler: \emph{Ein Briefwechsel}. Bern: \emph{Francke} 1956, S. 131.} }\pstart{}{\pb}Wien XVIII. Sternwartestr 71.\oindex{Sternwartestrasse 71@\textbf{Sternwartestraße 71}, \emph{Wohngebäude (K.WHS)}|pw},
                  A S\pend{}{\bigskip}\pstart{}Hrn Georg Brandes\pend{}\pstart{}Kopenhagen\oindex{Kopenhagen@\textbf{Kopenhagen}, \emph{P.PPLC}|pw}\pend{}\pstart{}Daenemark\oindex{Daenemark@\textbf{Dänemark}, \emph{A.PCLI}|pw}\pend{}{\bigskip}\vspace{1em}
\pstart
           \raggedleft{}{\pb}21. 8. 20\pend
           \vspace{0.5em}
\pstart
           lieber und verehrter Freund, eben trifft Ihre Karte vom
                  17. 8 ein. Ihr Brief vom 13. 6 ist angelangt; vor etwa
               4, 5 Tagen schrieb ich Ihnen einen sehr langen Brief\substVorne{}\textsuperscript{–}\substDazwischen{},\substHinten{} und wünschte mir sehr eine Bestätigung zu erhalten, daß Sie ihn in Händen
               haben, mir fällt ein, dſs ich Ihnen von gemeinsamen Beka{\geminationn}ten kaum etwas geschrieben habe. Richard Beer
                  Hofm\pwindex{Beer-Hofmann, Richard 1866-07-11 – 1945-09-26@\textsc{Beer-Hofmann, Richard} (1866-07-11 – 1945-09-26), \emph{Schriftsteller/Schriftstellerin}|pw} mit den Seinigen befindet sich wohl, und ich treffe nächster Tage mit
               ihm in Aussee\oindex{Bad Aussee@\textbf{Bad Aussee}, \emph{P.PPLA3}|pw} zusa{\geminationm}en. In der gleichen Gegend Hofma{\geminationn}sthal\pwindex{Hofmannsthal, Hugo von 1874-02-01 – 1929-07-15@\textsc{Hofmannsthal, Hugo von} (1874-02-01 – 1929-07-15), \emph{Schriftsteller/Schriftstellerin}|pw}, Salten\pwindex{Salten, Felix 06.09.1869 – 08.10.1945@\textsc{Salten, Felix} (06.09.1869 – 08.10.1945), \emph{Schriftsteller/Schriftstellerin, Journalist/Journalistin, Chefredakteur/Chefredakteurin}|pw} nicht weit davon am Attersee\oindex{Attersee@\textbf{Attersee}, \emph{H.LK}|pw};–
               wir alle sind eigentlich, we{\geminationn} mans recht bedenkt –
               bisher – über die Unbilden dieser Zeit ganz leidlich weggeko{\geminationm}en;– was fingen wir Menschen ohne {\pb}unsre bewunderungswürdige und etwas beschämende
               Accomodationsfähigkeit an.\pend
           
\pstart
           Ich bin wie immer von ganzem Herzen{\\[\baselineskip]}Ihr getreuer{\\[\baselineskip]}\spacefill\mbox{Arthur Schnitzler}\pend
           \leftskip=0em{}\selectlanguage{ngerman}\endnumbering\briefempfaengerindex{Brandes, Georg@\textsc{Brandes, Georg}!zzzSchnitzler, Arthur@\emph{von Arthur Schnitzler}!1920-08-211@{21. 8. 1920}|)be}\mylabel{L02355h}  \normalsize

\doendnotes{C}
\bigskip
\vfill

\clearpage

\footnotesize

\lohead{\textsc{register}}

% Definiere theindex-Environment komplett neu ohne reledmac
\makeatletter
\renewenvironment{theindex}{%
  \section*{\indexname}%
  \setlength{\parindent}{0pt}%
  \setlength{\parskip}{0pt plus 0.3pt}%
  \let\item\@idxitem
}{%
  \clearpage
}
\makeatother

\IfFileExists{\jobname-pw.ind}{\input{\jobname-pw.ind}}{}

\end{document}

      