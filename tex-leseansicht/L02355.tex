%% latex-leseansicht-vorspann.tex
%% Vorspann für die Leseansicht.
%% Lädt die gemeinsame Datei latex-vorspann.tex mit nicht gesetztem Schalter.

\newif\ifkorrekturansicht
\korrekturansichtfalse

\input{../tex-inputs/latex-vorspann}


         
         \renewcommand{\erwaehntePersonen}{Personen: Richard Beer-Hofmann, Georg Brandes, Hugo von Hofmannsthal, Felix Salten}
         \renewcommand{\erwaehnteOrte}{Orte: Bad Aussee, Dänemark, Kopenhagen, Sternwartestraße, Wien, XVIII., Währing}
         \renewcommand{\erwaehnteWerke}{
               \section[Arthur Schnitzler an Georg Brandes, 21. 8. 1920]{ Arthur Schnitzler an Georg Brandes, 21. 8. 1920}\nopagebreak\mylabel{v}\rehead{ }\begin{ledgroupsized}[t]{13cm}\normalsize\beginnumbering \toendnotes[C]{\smallbreak\pagebreak[2]} \Standort{Kopenhagen, Det Kongelige Bibliotek, Georg Brandes Arkiv, box 125.}
\physDesc{Postkarte
\newline{}Handschrift: Bleistift, lateinische Kurrent\newline{}Versand: Stempel: »\nobreak{}\oindex{XVIII., Waehring@\textbf{XVIII., Währing}|pwk}18/\textcolor{gray}{×} Wien, 21. VIII. 20, 4\nobreak{}«.  \newline{}Ordnung: 1) mit Bleistift von unbekannter Hand links der Briefmarke nummeriert: »43«  2) mit Bleistift von unbekannter Hand auf der Textseite zusätzlich
                                 die Datierung wiederholt: »21/8 20«}\buchAbdrucke{\weitereDrucke{Georg Brandes, Arthur Schnitzler: \emph{Ein Briefwechsel}. Hg. Kurt Bergel. Bern: \emph{Francke} 1956, S. 131.} }\pstart{}{\pb}Wien XVIII. Sternwartestr 71.\oindex{Sternwartestrasse@\textbf{Sternwartestraße}|pw},
                        A S\pend{}{\bigskip}\pstart{}Hrn Georg Brandes\pend{}\pstart{}Kopenhagen\oindex{Kopenhagen@\textbf{Kopenhagen}|pw}\pend{}\pstart{}Daenemark\oindex{Daenemark@\textbf{Dänemark}|pw}\pend{}{\bigskip}\pstart
           \raggedleft{}{\pb}21. 8. 20\pend
           \pstart
           lieber und verehrter Freund, eben trifft Ihre Karte vom
                        17. 8 ein. Ihr Brief vom 13. 6 ist angelangt; vor
                    etwa 4, 5 Tagen schrieb ich Ihnen einen sehr langen Brief\substVorne{}\textsuperscript{–}\substDazwischen{},\substHinten{} und wünschte mir sehr eine Bestätigung zu erhalten, daß Sie ihn in
                    Händen haben, mir fällt ein, dſs ich Ihnen von gemeinsamen Beka{\geminationn}ten kaum etwas geschrieben habe. Richard Beer Hofm\pwindex{Beer-Hofmann, Richard 1866-07-11 – 1945-09-26@\textsc{Beer-Hofmann, Richard} (1866-07-11 – 1945-09-26), \emph{Schriftsteller}|pw} mit den Seinigen befindet sich wohl,
                    und ich treffe nächster Tage mit ihm in Aussee\oindex{Bad Aussee@\textbf{Bad Aussee}|pw}
                        zusa{\geminationm}en. In der gleichen Gegend Hofma{\geminationn}sthal\pwindex{Hofmannsthal, Hugo von 1874-02-01 – 1929-07-15@\textsc{Hofmannsthal, Hugo von} (1874-02-01 – 1929-07-15), \emph{Schriftsteller}|pw},
                        Salten\pwindex{Salten, Felix 06.09.1869 – 08.10.1945@\textsc{Salten, Felix} (06.09.1869 – 08.10.1945), \emph{Schriftsteller, Journalist}|pw} nicht weit davon am Attersee;– wir alle sind eigentlich, we{\geminationn} mans recht bedenkt – bisher – über die Unbilden
                    dieser Zeit ganz leidlich weggeko{\geminationm}en;– was fingen
                    wir Menschen ohne {\pb}unsre bewunderungswürdige
                    und etwas beschämende Accomodationsfähigkeit an.\pend
           \pstart
           Ich bin wie immer von ganzem Herzen{\\[\baselineskip]}Ihr getreuer{\\[\baselineskip]}\spacefill\mbox{Arthur Schnitzler}\pend
           \leftskip=0em{}
         
         \endnumbering\mylabel{h}\end{ledgroupsized}  \newcommand{\dateiname}{L02355}\newcommand{\titel}{Arthur Schnitzler an Georg Brandes, 21. 8. 1920}\newcommand{\editorInnen}{Martin Anton Müller und Gerd-Hermann Susen}%% latex-leseansicht-abspann.tex
%% Abspann für die Leseansicht.
%% Der Schalter \ifkorrekturansicht ist bereits durch den Vorspann gesetzt.

%% latex-abspann.tex
%% Gemeinsamer Abspann für Korrekturansicht und Leseansicht.
%% Setzt den Schalter \ifkorrekturansicht voraus (gesetzt in den
%% einbindenden Dateien latex-korrekturansicht-abspann.tex bzw.
%% latex-leseansicht-abspann.tex).
%% ---------------------------------------------------------------

\normalsize

% Das esempio-Environment wird nur in der Leseansicht benötigt
\ifkorrekturansicht\else
\newenvironment{esempio}[3]%
{
    \vspace{1.5ex}
    \rlap{\underline{#1}}
    \par
    \setlength{\parindent}{0cm}
    \nopagebreak
    \leftskip=#2cm
    \rightskip=#3cm
}
{
    \par
}
\fi

\doendnotes{C}
\bigskip
\vfill

\clearpage

\footnotesize

\ifkorrekturansicht
  \lohead{\textsc{register}}
\fi

% theindex-Environment neu definieren ohne reledmac
\makeatletter
\renewenvironment{theindex}{%
  \ifkorrekturansicht
    \section*{\indexname}%
  \else
    \subsubsection*{Index der erwähnten Entitäten}%
  \fi
  \setlength{\parindent}{0pt}%
  \setlength{\parskip}{0pt plus 0.3pt}%
  \let\item\@idxitem
}{%
  \ifkorrekturansicht\clearpage\fi
}
\makeatother

\IfFileExists{\jobname-pw.ind}{\input{\jobname-pw.ind}}{}

% Quellenangabe nur in der Leseansicht
\ifkorrekturansicht\else
% Fallback-Definitionen, falls die .tex-Datei \titel etc. nicht gesetzt hat
\providecommand{\titel}{}
\providecommand{\editorInnen}{}
\providecommand{\dateiname}{\jobname}

\vspace{3cm}

\vfill

\footnotesize
\textsc{Quelle}: \titel. Herausgegeben von {\editorInnen}. In: \emph{Arthur Schnitzler: Briefwechsel mit Autorinnen und Autoren}.
 Digitale Edition, https://schnitzler-briefe.acdh.oeaw.ac.at/{\dateiname}.html (Stand \today)
\fi

\end{document}


      