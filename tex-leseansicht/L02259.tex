%% latex-leseansicht-vorspann.tex
%% Vorspann für die Leseansicht.
%% Lädt die gemeinsame Datei latex-vorspann.tex mit nicht gesetztem Schalter.

\newif\ifkorrekturansicht
\korrekturansichtfalse

\input{../tex-inputs/latex-vorspann}


\section[Hugo von Hofmannsthal an Arthur Schnitzler, 30. 4. {[}1917{]}]{L02259 Hugo von Hofmannsthal an Arthur Schnitzler, 30. 4. [1917]}
\nopagebreak\mylabel{L02259v}
\rehead{ }\normalsize\beginnumbering\briefempfaengerindex{Schnitzler, Arthur@\textsc{Schnitzler, Arthur}!zzzHofmannsthal, Hugo von@\emph{von Hugo von Hofmannsthal}!1917-04-301@{30. 4. [1917]}|(be}
\toendnotes[C]{\smallbreak\pagebreak[2]}
\correspDesc{Versand  durch Hugo von Hofmannsthal am 30. 4. [1917] in Rodaun
\newline{}Erhalt  durch Arthur Schnitzler im Zeitraum [1. 5. 1917
                  – 5. 5. 1917?] in Wien}\toendnotes[C]{\smallbreak}
\Standort{CUL, Schnitzler, B 43.}
\physDesc{Briefkarte, 757 Zeichen
\newline{}Handschrift: schwarze Tinte, deutsche Kurrent
\newline{}Schnitzler: mit Bleistift die Jahreszahl ergänzt: »17« und beschriftet: »Hugo« 
\newline{}Ordnung: 1) mit Bleistift von Frieda
                                    Pollak\pwindex{Pollak, Frieda 8.\,12.\,1881 Wien – 13.\,7.\,1937 ebd.@\textsc{Pollak, Frieda} (8.\,12.\,1881 Wien – 13.\,7.\,1937 ebd.), \emph{Sekretärin}|pw} (?) mit dem Buchstaben »A«
                                 (Abgeschrieben/Abschrift) gekennzeichnet  2) mit Bleistift von unbekannter Hand nummeriert: »\strikeout{347}« 3) mit Bleistift von unbekannter Hand nummeriert:
                                    »358«}
\buchAbdrucke{\weitereDrucke{Hugo von Hofmannsthal, Arthur Schnitzler: \emph{Briefwechsel}. Herausgegeben von Therese Nickl und Heinrich Schnitzler. Frankfurt am Main: \emph{S. Fischer} 1964, S. 281.} }\toendnotes[C]{\smallbreak}
\pstart
           \raggedleft{}{\pb}R.\oindex{Wien@\textbf{Wien}!XXIII., Liesing@\textbf{XXIII., Liesing}!Rodaun@\textbf{Rodaun}, \emph{Region}|pw}{ }30 IV.\pend
           
\pstart{}mein lieber Arthur\pend\vspace{0.5em}
\pstart
           ich weiß nicht, ob Sie nicht vielleicht ohnedies die Abſicht haben, zu der \label{K_L02259-1v}\edtext{\introOben{}Concordia-\orgindex{Concordia. Journalisten- und Schriftstellerverein@Concordia. Journalisten- und Schriftstellerverein|pw}\introOben{}Veranſtaltung}{\lemma{\textnormal{\emph{Concordia-Veranstaltung}}}\Cendnote{\textnormal{Vgl. A. S.: \emph{Tagebuch}, 3. 5. 1917.
               }}}\label{K_L02259-1} für die Schweiz\oindex{Schweiz@\textbf{Schweiz}|pw}er zuzuſagen u. zu ko{\geminationm}en – jedenfalls finde ich es – abgeſehen von meiner
               perſönlichen Freude, Sie dann dort zu{ }ſehen und in einem gewiſſen Sinn, nicht \uline{allein} zu{ }ſein –{ }ſo überaus nützlich und \uline{richtig} wenn Sie {\pb}kämen, denn es handelt{ }ſich ja
               nicht{ }ſo{ }ſehr um den mehr minder trivialen Abend, den wir da verbringen werden,{ }ſondern um die Rückwirkung nach der Schweiz\oindex{Schweiz@\textbf{Schweiz}|pw}
               hin, und es iſt doch nur natürlich, wenn da Ihre Gegenwart{ }ſehr ins Gewicht fällt,
               mehr als jede andere, da Sie ja eigentlich von allen deutſch{ }ſchreibenden
               Bühnendichtern der einzige \introOben{}im Ausland\introOben{} nicht nur bekannte,{ }ſondern wirklich populäre{ }ſind.\pend
           
\pstart
           Herzlich Ihr{\\[\baselineskip]}\spacefill\mbox{Hugo.}\pend
           \leftskip=0em{}\selectlanguage{ngerman}\endnumbering\briefempfaengerindex{Schnitzler, Arthur@\textsc{Schnitzler, Arthur}!zzzHofmannsthal, Hugo von@\emph{von Hugo von Hofmannsthal}!1917-04-301@{30. 4. [1917]}|)be}\mylabel{L02259h}  \newcommand{\dateiname}{L02259}\newcommand{\titel}{Hugo von Hofmannsthal an Arthur Schnitzler, 30. 4. [1917]}\newcommand{\editorInnen}{Martin Anton Müller und Gerd-Hermann Susen}%% latex-leseansicht-abspann.tex
%% Abspann für die Leseansicht.
%% Der Schalter \ifkorrekturansicht ist bereits durch den Vorspann gesetzt.

%% latex-abspann.tex
%% Gemeinsamer Abspann für Korrekturansicht und Leseansicht.
%% Setzt den Schalter \ifkorrekturansicht voraus (gesetzt in den
%% einbindenden Dateien latex-korrekturansicht-abspann.tex bzw.
%% latex-leseansicht-abspann.tex).
%% ---------------------------------------------------------------

\normalsize

% Das esempio-Environment wird nur in der Leseansicht benötigt
\ifkorrekturansicht\else
\newenvironment{esempio}[3]%
{
    \vspace{1.5ex}
    \rlap{\underline{#1}}
    \par
    \setlength{\parindent}{0cm}
    \nopagebreak
    \leftskip=#2cm
    \rightskip=#3cm
}
{
    \par
}
\fi

\doendnotes{C}
\bigskip
\vfill

\clearpage

\footnotesize

\ifkorrekturansicht
  \lohead{\textsc{register}}
\fi

% theindex-Environment neu definieren ohne reledmac
\makeatletter
\renewenvironment{theindex}{%
  \ifkorrekturansicht
    \section*{\indexname}%
  \else
    \subsubsection*{Index der erwähnten Entitäten}%
  \fi
  \setlength{\parindent}{0pt}%
  \setlength{\parskip}{0pt plus 0.3pt}%
  \let\item\@idxitem
}{%
  \ifkorrekturansicht\clearpage\fi
}
\makeatother

\IfFileExists{\jobname-pw.ind}{\input{\jobname-pw.ind}}{}

% Quellenangabe nur in der Leseansicht
\ifkorrekturansicht\else
% Fallback-Definitionen, falls die .tex-Datei \titel etc. nicht gesetzt hat
\providecommand{\titel}{}
\providecommand{\editorInnen}{}
\providecommand{\dateiname}{\jobname}

\vspace{3cm}

\vfill

\footnotesize
\textsc{Quelle}: \titel. Herausgegeben von {\editorInnen}. In: \emph{Arthur Schnitzler: Briefwechsel mit Autorinnen und Autoren}.
 Digitale Edition, https://schnitzler-briefe.acdh.oeaw.ac.at/{\dateiname}.html (Stand \today)
\fi

\end{document}


