%% latex-leseansicht-vorspann.tex
%% Vorspann für die Leseansicht.
%% Lädt die gemeinsame Datei latex-vorspann.tex mit nicht gesetztem Schalter.

\newif\ifkorrekturansicht
\korrekturansichtfalse

\input{../tex-inputs/latex-vorspann}


         
         \renewcommand{\erwaehntePersonen}{Personen: Frieda Pollak}
         \renewcommand{\erwaehnteInstitutionen}{Institutionen: Concordia}
         \renewcommand{\erwaehnteOrte}{Orte: Rodaun, Schweiz, Wien}
         \renewcommand{\erwaehnteWerke}{}
               \section[Hugo von Hofmannsthal an Arthur Schnitzler, 30. 4. {[}1917{]}]{ Hugo von Hofmannsthal an Arthur Schnitzler, 30. 4. {[}1917{]}}\nopagebreak\mylabel{v}\rehead{ }\begin{ledgroupsized}[t]{13cm}\normalsize\beginnumbering \toendnotes[C]{\smallbreak\pagebreak[2]} \Standort{CUL, Schnitzler, B 43.}
\physDesc{Briefkarte
\newline{}Handschrift: schwarze Tinte, deutsche Kurrent
\newline{}Schnitzler: mit Bleistift die Jahreszahl ergänzt: »17« und beschriftet: »Hugo« \newline{}Ordnung: 1) mit Bleistift von Frieda Pollak\pwindex{Pollak, Frieda 08.12.1881 – 13.07.1937@\textsc{Pollak, Frieda} (08.12.1881 – 13.07.1937), \emph{Sekretärin}|pw} (?) mit dem Buchstaben »A« (Abgeschrieben/Abschrift) gekennzeichnet  2) mit Bleistift von unbekannter Hand nummeriert: »\strikeout{347}« 3) mit Bleistift von unbekannter Hand nummeriert: »358«}\buchAbdrucke{\weitereDrucke{Hugo von Hofmannsthal, Arthur Schnitzler: \emph{Briefwechsel}. Hg. Therese Nickl und Heinrich Schnitzler. Frankfurt am Main: \emph{S. Fischer} 1964, S. 281.} }\toendnotes[C]{\smallbreak}\pstart
           \raggedleft{}{\pb}R.\oindex{Rodaun@\textbf{Rodaun}|pw}{ }30 IV.\pend
           \pstart{}mein lieber Arthur \pend\pstart
           ich weiß nicht, ob Sie nicht vielleicht ohnedies die Abſicht haben, zu der \label{K_L02259_1v}\edtext{\introOben{}Concordia-\orgindex{Concordia@Concordia|pw}\introOben{}Veranſtaltung}{\lemma{\textnormal{\emph{Concordia-Veranſtaltung}}}\Cendnote{\textnormal{vgl. A. S.: \emph{Tagebuch}, 3. 5. 1917}}}\label{K_L02259_1h} für die Schweiz\oindex{Schweiz@\textbf{Schweiz}|pw}er zuzuſagen u. zu ko{\geminationm}en – jedenfalls
               finde ich es – abgeſehen von meiner perſönlichen Freude, Sie dann dort zu ſehen und
               in einem gewiſſen Sinn, nicht \uline{allein} zu ſein – ſo
               überaus nützlich und \uline{richtig} wenn Sie {\pb}kämen, denn es handelt ſich ja
               nicht ſo ſehr um den mehr minder trivialen Abend, den wir da verbringen werden,
               ſondern um die Rückwirkung nach der Schweiz\oindex{Schweiz@\textbf{Schweiz}|pw} hin,
               und es iſt doch nur natürlich, wenn da Ihre Gegenwart ſehr ins Gewicht fällt, mehr
               als jede andere, da Sie ja eigentlich von allen deutſch ſchreibenden Bühnendichtern
               der einzige \introOben{}im Ausland\introOben{} nicht nur bekannte, ſondern wirklich
               populäre ſind.\pend
           \pstart
           Herzlich Ihr{\\[\baselineskip]}\spacefill\mbox{Hugo.}\pend
           \leftskip=0em{}
         
         \endnumbering\mylabel{h}\end{ledgroupsized}  \newcommand{\dateiname}{L02259}\newcommand{\titel}{Hugo von Hofmannsthal an Arthur Schnitzler, 30. 4. [1917]}\newcommand{\editorInnen}{Martin Anton Müller und Gerd-Hermann Susen}%% latex-leseansicht-abspann.tex
%% Abspann für die Leseansicht.
%% Der Schalter \ifkorrekturansicht ist bereits durch den Vorspann gesetzt.

%% latex-abspann.tex
%% Gemeinsamer Abspann für Korrekturansicht und Leseansicht.
%% Setzt den Schalter \ifkorrekturansicht voraus (gesetzt in den
%% einbindenden Dateien latex-korrekturansicht-abspann.tex bzw.
%% latex-leseansicht-abspann.tex).
%% ---------------------------------------------------------------

\normalsize

% Das esempio-Environment wird nur in der Leseansicht benötigt
\ifkorrekturansicht\else
\newenvironment{esempio}[3]%
{
    \vspace{1.5ex}
    \rlap{\underline{#1}}
    \par
    \setlength{\parindent}{0cm}
    \nopagebreak
    \leftskip=#2cm
    \rightskip=#3cm
}
{
    \par
}
\fi

\doendnotes{C}
\bigskip
\vfill

\clearpage

\footnotesize

\ifkorrekturansicht
  \lohead{\textsc{register}}
\fi

% theindex-Environment neu definieren ohne reledmac
\makeatletter
\renewenvironment{theindex}{%
  \ifkorrekturansicht
    \section*{\indexname}%
  \else
    \subsubsection*{Index der erwähnten Entitäten}%
  \fi
  \setlength{\parindent}{0pt}%
  \setlength{\parskip}{0pt plus 0.3pt}%
  \let\item\@idxitem
}{%
  \ifkorrekturansicht\clearpage\fi
}
\makeatother

\IfFileExists{\jobname-pw.ind}{\input{\jobname-pw.ind}}{}

% Quellenangabe nur in der Leseansicht
\ifkorrekturansicht\else
% Fallback-Definitionen, falls die .tex-Datei \titel etc. nicht gesetzt hat
\providecommand{\titel}{}
\providecommand{\editorInnen}{}
\providecommand{\dateiname}{\jobname}

\vspace{3cm}

\vfill

\footnotesize
\textsc{Quelle}: \titel. Herausgegeben von {\editorInnen}. In: \emph{Arthur Schnitzler: Briefwechsel mit Autorinnen und Autoren}.
 Digitale Edition, https://schnitzler-briefe.acdh.oeaw.ac.at/{\dateiname}.html (Stand \today)
\fi

\end{document}


      