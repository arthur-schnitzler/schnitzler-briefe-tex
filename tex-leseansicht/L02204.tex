%% latex-leseansicht-vorspann.tex
%% Vorspann für die Leseansicht.
%% Lädt die gemeinsame Datei latex-vorspann.tex mit nicht gesetztem Schalter.

\newif\ifkorrekturansicht
\korrekturansichtfalse

\input{../tex-inputs/latex-vorspann}


               \section[Hermann Bahr an Arthur Schnitzler, 10. 2. 1915]{ Hermann Bahr an Arthur Schnitzler, 10. 2. 1915}\nopagebreak\mylabel{v}\rehead{ }\begin{ledgroupsized}[t]{13cm}\normalsize\beginnumbering\briefempfaengerindex{Schnitzler, Arthur@\textsc{Schnitzler, Arthur}!zzzBahr, Hermann@\emph{von Hermann Bahr}!1915-02-101@{10. 2. 1915}|(be} \toendnotes[C]{\smallbreak\pagebreak[2]} \Standort{CUL, Schnitzler, B 5b.}
\physDesc{Brief, 1 Blatt, 1 Seite
\newline{}Handschrift: schwarze Tinte, deutsche Kurrent
\newline{}Schnitzler: 1) mit Bleistift ergänzt »Bahr« 2) mit rotem Buntstift eine Unterstreichung\newline{}Ordnung: mit Bleistift von unbekannter Hand
                           nummeriert: »181« }\buchAbdrucke{\weitereDrucke{Hermann Bahr, Arthur Schnitzler: \emph{Briefwechsel, Aufzeichnungen, Dokumente (1891–1931)}. Hg. Kurt Ifkovits und Martin Anton Müller. Göttingen: \emph{Wallstein} 2018, S. 497–498.} }\toendnotes[C]{\smallbreak}\pstart
           \raggedleft{}{\pb}10. 2. 15\pend
           \pstart\center{}Lieber Arthur!\pend\pstart
           Herzlichen Dank für den lieben Brief, der uns Beiden\pwindex{Bahr-Mildenburg, Anna 29.11.1872 – 27.01.1947@\textsc{Bahr-Mildenburg, Anna} (29.11.1872 – 27.01.1947), \emph{Sängerin}|pwv} eine große Freude gemacht hat! Meine Frau\pwindex{Bahr-Mildenburg, Anna 29.11.1872 – 27.01.1947@\textsc{Bahr-Mildenburg, Anna} (29.11.1872 – 27.01.1947), \emph{Sängerin}|pwv} möchte ſehr gern einmal in
                  Wien\oindex{Wien@\textbf{Wien}|pw} Lieder ſingen, Schubert\pwindex{Schubert, Franz Peter 31.01.1797 – 19.11.1828@\textsc{Schubert, Franz Peter} (31.01.1797 – 19.11.1828), \emph{Komponist}|pw}, Hugo Wolf\pwindex{Wolf, Hugo 13.03.1860 – 22.02.1903@\textsc{Wolf, Hugo} (13.03.1860 – 22.02.1903), \emph{Komponist}|pw} und die Weſendoncklieder\pwindex{\textcolor{red}{\textsuperscript{XXXX1 indx}}!Fuenf Gedichte von Mathilde Wesendonk fuer eine Frauenstimme und Klavier1857 – 1858@\strich\emph{Fünf Gedichte von Mathilde Wesendonk für eine Frauenstimme und Klavier} {[}1857 – 1858{]}|pw} am liebſten. Jetzt aber geht das
               nicht, sie kann hier nicht abkommen von ihrem \label{K_L02204_1v}\edtext{Spital\oindex{Krankenhaus der Barmherzigen Brueder@\textbf{Krankenhaus der Barmherzigen Brüder}|pwv}}{\lemma{\textnormal{\emph{Spital}}}\Cendnote{\textnormal{Sie arbeitete als freiwillige
                  Pflegehelferin im Salzburger Truppenspital
                     Nonntal\oindex{Krankenhaus der Barmherzigen Brueder@\textbf{Krankenhaus der Barmherzigen Brüder}|pwk}.}}}\label{K_L02204_1h} (ich ſchrieb das Heller\pwindex{Heller, Hugo 08.05.1870 – 29.11.1923@\textsc{Heller, Hugo} (08.05.1870 – 29.11.1923), \emph{Verleger, Buchhändler}|pw} geſtern ſchon). Auch bin ich der Meinung, daß es beſſer iſt, dazu
               eine ſtillere, für Kunſt empfänglichere Zeit abzuwarten. Willſt Du aber nicht ſo
               lange warten, ſo komm doch her, Du kannſt es bei uns viel ſchöner haben als je in
               einem Konzert, was doch von vorneherein die ſcheußlichſte Kunstwidrigkeit iſt! Wir
               würden uns herzlich freuen und ich hätte ja ſo viel mit Dir zu reden, Tage lang!\pend
           \pstart
           Grüße Frau Olga\pwindex{Schnitzler, Olga 17.01.1882 – 13.01.1970@\textsc{Schnitzler, Olga} (17.01.1882 – 13.01.1970), \emph{Schauspielerin, Sängerin}|pw} in alter herzlicher
                  \textcolor{gray}{Ve}rehrung ſchönſtens von mir und kommt wirklich bald einmal!
               (Aber mit Nachricht ein paar Tage früher, damit ich nicht gerade weg bin, in München\oindex{Muenchen@\textbf{München}|pw} oder in den Bergen!)\pend
           \pstart
           Herzlichſt{\\[\baselineskip]}Dein alter{\\[\baselineskip]}\spacefill\mbox{H}\pend
           \leftskip=0em{}\endnumbering\briefempfaengerindex{Schnitzler, Arthur@\textsc{Schnitzler, Arthur}!zzzBahr, Hermann@\emph{von Hermann Bahr}!1915-02-101@{10. 2. 1915}|)be}\mylabel{h}\end{ledgroupsized}  \newcommand{\dateiname}{L02204}\newcommand{\titel}{Hermann Bahr an Arthur Schnitzler, 10. 2. 1915}\newcommand{\editorInnen}{ Kurt Ifkovits,  Martin Anton Müller}%% latex-leseansicht-abspann.tex
%% Abspann für die Leseansicht.
%% Der Schalter \ifkorrekturansicht ist bereits durch den Vorspann gesetzt.

%% latex-abspann.tex
%% Gemeinsamer Abspann für Korrekturansicht und Leseansicht.
%% Setzt den Schalter \ifkorrekturansicht voraus (gesetzt in den
%% einbindenden Dateien latex-korrekturansicht-abspann.tex bzw.
%% latex-leseansicht-abspann.tex).
%% ---------------------------------------------------------------

\normalsize

% Das esempio-Environment wird nur in der Leseansicht benötigt
\ifkorrekturansicht\else
\newenvironment{esempio}[3]%
{
    \vspace{1.5ex}
    \rlap{\underline{#1}}
    \par
    \setlength{\parindent}{0cm}
    \nopagebreak
    \leftskip=#2cm
    \rightskip=#3cm
}
{
    \par
}
\fi

\doendnotes{C}
\bigskip
\vfill

\clearpage

\footnotesize

\ifkorrekturansicht
  \lohead{\textsc{register}}
\fi

% theindex-Environment neu definieren ohne reledmac
\makeatletter
\renewenvironment{theindex}{%
  \ifkorrekturansicht
    \section*{\indexname}%
  \else
    \subsubsection*{Index der erwähnten Entitäten}%
  \fi
  \setlength{\parindent}{0pt}%
  \setlength{\parskip}{0pt plus 0.3pt}%
  \let\item\@idxitem
}{%
  \ifkorrekturansicht\clearpage\fi
}
\makeatother

\IfFileExists{\jobname-pw.ind}{\input{\jobname-pw.ind}}{}

% Quellenangabe nur in der Leseansicht
\ifkorrekturansicht\else
% Fallback-Definitionen, falls die .tex-Datei \titel etc. nicht gesetzt hat
\providecommand{\titel}{}
\providecommand{\editorInnen}{}
\providecommand{\dateiname}{\jobname}

\vspace{3cm}

\vfill

\footnotesize
\textsc{Quelle}: \titel. Herausgegeben von {\editorInnen}. In: \emph{Arthur Schnitzler: Briefwechsel mit Autorinnen und Autoren}.
 Digitale Edition, https://schnitzler-briefe.acdh.oeaw.ac.at/{\dateiname}.html (Stand \today)
\fi

\end{document}


      