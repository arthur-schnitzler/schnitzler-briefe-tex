%% latex-korrekturansicht-vorspann.tex
%% Vorspann für die Korrekturansicht.
%% Lädt die gemeinsame Datei latex-vorspann.tex mit gesetztem Schalter.

\newif\ifkorrekturansicht
\korrekturansichttrue

\input{../tex-inputs/latex-vorspann}


\section[ Felix Salten an Arthur Schnitzler, 17. 7. 1906]{L03431 Felix Salten an Arthur Schnitzler, 17. 7. 1906}
\nopagebreak\mylabel{L03431v}
\rehead{ }\normalsize\beginnumbering\briefempfaengerindex{Schnitzler, Arthur@\textsc{Schnitzler, Arthur}!zzzSalten, Felix@\emph{von Felix Salten}!1906-07-172@{17. 7. 1906}|(be}
\toendnotes[C]{\smallbreak\pagebreak[2]}\Standort{CUL, Schnitzler, B 89, B 1.}
\physDesc{Brief, 1 Blatt, 1 Seite, 820 Zeichen
\newline{}Handschrift: schwarze Tinte, lateinische Kurrent
\newline{}Ordnung: mit Bleistift von unbekannter Hand nummeriert: »222« }\toendnotes[C]{\smallbreak}
\pstart
           \raggedleft{}{\pb}Bansin\oindex{Bansin@\textbf{Bansin}, \emph{P.PPL}|pw}, 17. 7. 06.\pend
           \vspace{0.5em}
\pstart
           Lieber, wir wollen schon bald – vielleicht schon diesen Freitag – nach Kopenhagen\oindex{Kopenhagen@\textbf{Kopenhagen}, \emph{P.PPLC}|pw} fahren, und dann \label{K_L03431-1v}\edtext{zu Ihnen nach Marienlyst\oindex{Marienlyst@\textbf{Marienlyst}, \emph{S.EST}|pw} kommen}{\lemma{\textnormal{\emph{zu … kommen}}}\Cendnote{\textnormal{Siehe A. S.: \emph{Tagebuch}, 2. 8. 1906.
               }}}\label{K_L03431-1}. Aber wol nicht länger als auf einen oder zwei Tage. Denn bis die Millionen,
               deren freilich nur Sie allein so sicher gewärtig sind, bis also die Millionen kommen,
               muß ich mich noch mit Kleinigkeiten abgeben und Verhandlungen führen, kann also nicht
               so lange fortbleiben. Ferner ist das Programm, dass ich nach Wien\oindex{Wien@\textbf{Wien}, \emph{A.ADM2}|pw} gehe. Von dort eventuell über Ischl\oindex{Bad Ischl@\textbf{Bad Ischl}, \emph{P.PPL}|pw}, Lueg\oindex{Lueg@\textbf{Lueg}, \emph{Teil eines besiedelten Ortes (A.BSOX)}|pw}, Gilgen\oindex{St. Gilgen@\textbf{St. Gilgen}, \emph{A.ADM3}|pw}{ }Salzburg\oindex{Salzburg@\textbf{Salzburg}, \emph{A.ADM2}|pw}{ }München\oindex{Muenchen@\textbf{München}, \emph{P.PPLA}|pw}{ }hierher\oindex{Bansin@\textbf{Bansin}, \emph{P.PPL}|pwv} zurück. Und endlich
               ist es meine Absicht, nach Weimar\oindex{Weimar@\textbf{Weimar}, \emph{A.ADM3}|pw} zu gehen, weil
               ich es Otti\pwindex{Salten, Ottilie 07.03.1868 – 22.06.1942@\textsc{Salten, Ottilie} (07.03.1868 – 22.06.1942), \emph{Schauspieler/Schauspielerin}|pw} unbedingt zeigen möchte, ehe wir
               das Deutsche Reich\oindex{Deutschland@\textbf{Deutschland}, \emph{A.PCLI}|pw} verlaßen. Wenn wir uns also
               nach Kopenhagen\oindex{Kopenhagen@\textbf{Kopenhagen}, \emph{P.PPLC}|pw} in Bewegung setzen, zeige ich
               es Ihnen telegrafisch an. Inzwischen viele herzliche Grüße von Otti\pwindex{Salten, Ottilie 07.03.1868 – 22.06.1942@\textsc{Salten, Ottilie} (07.03.1868 – 22.06.1942), \emph{Schauspieler/Schauspielerin}|pw} und mir an Sie Beide\pwindex{Schnitzler, Olga 17.01.1882 – 13.01.1970@\textsc{Schnitzler, Olga} (17.01.1882 – 13.01.1970), \emph{Schauspieler/Schauspielerin, Sänger/Sängerin}|pwv}.\pend
           \pstart Ihr \spacefill\mbox{FSalten}\pend{}\selectlanguage{ngerman}\endnumbering\briefempfaengerindex{Schnitzler, Arthur@\textsc{Schnitzler, Arthur}!zzzSalten, Felix@\emph{von Felix Salten}!1906-07-172@{17. 7. 1906}|)be}\mylabel{L03431h}  \normalsize

\doendnotes{C}
\bigskip
\vfill

\clearpage

\footnotesize

\lohead{\textsc{register}}

% Definiere theindex-Environment komplett neu ohne reledmac
\makeatletter
\renewenvironment{theindex}{%
  \section*{\indexname}%
  \setlength{\parindent}{0pt}%
  \setlength{\parskip}{0pt plus 0.3pt}%
  \let\item\@idxitem
}{%
  \clearpage
}
\makeatother

\IfFileExists{\jobname-pw.ind}{\input{\jobname-pw.ind}}{}

\end{document}

      