%% latex-leseansicht-vorspann.tex
%% Vorspann für die Leseansicht.
%% Lädt die gemeinsame Datei latex-vorspann.tex mit nicht gesetztem Schalter.

\newif\ifkorrekturansicht
\korrekturansichtfalse

\input{../tex-inputs/latex-vorspann}

\begin{center}
            \textcolor{red}{ENTWURF, NICHT FERTIG KORRIGIERT}
                      \end{center}
            
         
         \renewcommand{\erwaehntePersonen}{Personen: Ottilie Salten, Olga Schnitzler}
         \renewcommand{\erwaehnteOrte}{Orte: Bad Ischl, Bansin, Deutschland, Kopenhagen, Lueg am Wolfgangsee, Marienlyst, München, Salzburg, St. Gilgen, Weimar, Wien}
         \renewcommand{\erwaehnteWerke}{}
               \section[ Felix Salten an Arthur Schnitzler, 17. 7. 1906]{ Felix Salten an Arthur Schnitzler, 17. 7. 1906}\nopagebreak\mylabel{v}\rehead{ }\begin{ledgroupsized}[t]{13cm}\normalsize\beginnumbering \toendnotes[C]{\smallbreak\pagebreak[2]} \Standort{CUL, Schnitzler, B 89, B 1.}
\physDesc{Brief, 1 Blatt, 1 Seite, 820 Zeichen
\newline{}Handschrift: schwarze Tinte, lateinische Kurrent
\newline{}Ordnung: mit Bleistift von unbekannter Hand nummeriert: »222« }\toendnotes[C]{\smallbreak}\pstart
           \raggedleft{}{\pb}Bansin\oindex{Bansin@\textbf{Bansin}|pw}, 17. 7. 06.\pend
           \pstart
           Lieber, wir wollen schon bald – vielleicht schon diesen Freitag – nach Kopenhagen\oindex{Kopenhagen@\textbf{Kopenhagen}|pw} fahren, und dann \label{K_L03431-1v}\edtext{zu Ihnen nach Marienlyst\oindex{Marienlyst@\textbf{Marienlyst}|pw} kommen}{\lemma{\textnormal{\emph{zu … kommen}}}\Cendnote{\textnormal{siehe A. S.: \emph{Tagebuch}, 2. 8. 1906}}}\label{K_L03431-1h}. Aber wol nicht länger als auf einen oder zwei Tage. Denn bis die Millionen,
               deren freilich nur Sie allein so sicher gewärtig sind, bis also die Millionen kommen,
               muß ich mich noch mit Kleinigkeiten abgeben und Verhandlungen führen, kann also nicht
               so lange fortbleiben. Ferner ist das Programm, dass ich nach Wien\oindex{Wien@\textbf{Wien}|pw} gehe. Von dort eventuell über Ischl\oindex{Bad Ischl@\textbf{Bad Ischl}|pw}, Lueg\oindex{Lueg am Wolfgangsee@\textbf{Lueg am Wolfgangsee}|pw}, Gilgen\oindex{St. Gilgen@\textbf{St. Gilgen}|pw}{ }Salzburg\oindex{Salzburg@\textbf{Salzburg}|pw}{ }München\oindex{Muenchen@\textbf{München}|pw}{ }hierher\oindex{Bansin@\textbf{Bansin}|pwv} zurück. Und endlich
               ist es meine Absicht, nach Weimar\oindex{Weimar@\textbf{Weimar}|pw} zu gehen, weil
               ich es Otti\pwindex{Salten, Ottilie 07.03.1868 – 22.06.1942@\textsc{Salten, Ottilie} (07.03.1868 – 22.06.1942), \emph{Schauspielerin}|pw} unbedingt zeigen möchte, ehe wir
               das Deutsche Reich\oindex{Deutschland@\textbf{Deutschland}|pw} verlaßen. Wenn wir uns also
               nach Kopenhagen\oindex{Kopenhagen@\textbf{Kopenhagen}|pw} in Bewegung setzen, zeige ich
               es Ihnen telegrafisch an. Inzwischen viele herzliche Grüße von Otti\pwindex{Salten, Ottilie 07.03.1868 – 22.06.1942@\textsc{Salten, Ottilie} (07.03.1868 – 22.06.1942), \emph{Schauspielerin}|pw} und mir an Sie Beide\pwindex{Schnitzler, Olga 17.01.1882 – 13.01.1970@\textsc{Schnitzler, Olga} (17.01.1882 – 13.01.1970), \emph{Schauspielerin, Sängerin}|pwv}.\pend
           \pstart Ihr \spacefill\mbox{FSalten}\pend{}
         
         \endnumbering\mylabel{h}\end{ledgroupsized}  \newcommand{\dateiname}{L03431}\newcommand{\titel}{Felix Salten an Arthur Schnitzler, 17. 7. 1906}\newcommand{\editorInnen}{Martin Anton Müller und Laura Untner}%% latex-leseansicht-abspann.tex
%% Abspann für die Leseansicht.
%% Der Schalter \ifkorrekturansicht ist bereits durch den Vorspann gesetzt.

%% latex-abspann.tex
%% Gemeinsamer Abspann für Korrekturansicht und Leseansicht.
%% Setzt den Schalter \ifkorrekturansicht voraus (gesetzt in den
%% einbindenden Dateien latex-korrekturansicht-abspann.tex bzw.
%% latex-leseansicht-abspann.tex).
%% ---------------------------------------------------------------

\normalsize

% Das esempio-Environment wird nur in der Leseansicht benötigt
\ifkorrekturansicht\else
\newenvironment{esempio}[3]%
{
    \vspace{1.5ex}
    \rlap{\underline{#1}}
    \par
    \setlength{\parindent}{0cm}
    \nopagebreak
    \leftskip=#2cm
    \rightskip=#3cm
}
{
    \par
}
\fi

\doendnotes{C}
\bigskip
\vfill

\clearpage

\footnotesize

\ifkorrekturansicht
  \lohead{\textsc{register}}
\fi

% theindex-Environment neu definieren ohne reledmac
\makeatletter
\renewenvironment{theindex}{%
  \ifkorrekturansicht
    \section*{\indexname}%
  \else
    \subsubsection*{Index der erwähnten Entitäten}%
  \fi
  \setlength{\parindent}{0pt}%
  \setlength{\parskip}{0pt plus 0.3pt}%
  \let\item\@idxitem
}{%
  \ifkorrekturansicht\clearpage\fi
}
\makeatother

\IfFileExists{\jobname-pw.ind}{\input{\jobname-pw.ind}}{}

% Quellenangabe nur in der Leseansicht
\ifkorrekturansicht\else
% Fallback-Definitionen, falls die .tex-Datei \titel etc. nicht gesetzt hat
\providecommand{\titel}{}
\providecommand{\editorInnen}{}
\providecommand{\dateiname}{\jobname}

\vspace{3cm}

\vfill

\footnotesize
\textsc{Quelle}: \titel. Herausgegeben von {\editorInnen}. In: \emph{Arthur Schnitzler: Briefwechsel mit Autorinnen und Autoren}.
 Digitale Edition, https://schnitzler-briefe.acdh.oeaw.ac.at/{\dateiname}.html (Stand \today)
\fi

\end{document}


      