%% latex-korrekturansicht-vorspann.tex
%% Vorspann für die Korrekturansicht.
%% Lädt die gemeinsame Datei latex-vorspann.tex mit gesetztem Schalter.

\newif\ifkorrekturansicht
\korrekturansichttrue

\input{../tex-inputs/latex-vorspann}


\section[Hermann Bahr an Arthur Schnitzler, 2. 12. 1909]{L01891 Hermann Bahr an Arthur Schnitzler, 2. 12. 1909}
\nopagebreak\mylabel{L01891v}
\rehead{ }\normalsize\beginnumbering\briefempfaengerindex{Schnitzler, Arthur@\textsc{Schnitzler, Arthur}!zzzBahr, Hermann@\emph{von Hermann Bahr}!1909-12-021@{2. 12. 1909}|(be}
\toendnotes[C]{\smallbreak\pagebreak[2]}\Standort{CUL, Schnitzler, B 5b.}
\physDesc{Bildpostkarte, 245 Zeichen
\newline{}Handschrift: 1) schwarze Tinte, deutsche Kurrent\hspace{1em}2) schwarze Tinte, lateinische Kurrent (\noindent{}Adresse)\hspace{1em}
\newline{}Versand: Stempel: »\nobreak{}\oindex{Tschernyschewskoje@\textbf{Tschernyschewskoje}, \emph{P.PPL}|pwk}Dirschau Eydtkuhnen
                                       Bahnpost, 2. 12 09\nobreak{}«.  
\newline{}Schnitzler: mit Bleistift ergänzt »Bahr« 
\newline{}Ordnung: mit Bleistift von unbekannter Hand nummeriert:
                                    »162« }
\buchAbdrucke{\weitereDrucke{Hermann Bahr, Arthur Schnitzler: \emph{Briefwechsel, Aufzeichnungen, Dokumente (1891–1931)}. Göttingen: \emph{Wallstein} 2018, S. 426.} }\toendnotes[C]{\smallbreak}\pstart{}{\pb}D\textsuperscript{r} Arthur
                  Schnitzler\pend{}\pstart{}Spöttelgasse 7\oindex{Edmund-Weiss-Gasse 7@\textbf{Edmund-Weiß-Gasse 7}, \emph{Wohngebäude (K.WHS)}|pw}\pend{}\pstart{}Wien XVIII\oindex{XVIII., Waehring@\textbf{XVIII., Währing}, \emph{A.ADM3}|pw}\pend{}{\bigskip}
\pstart
           \noindent{}\centering{}{\pb}\textcolor{gray}{\textbf{Königsberg\oindex{Kaliningrad@\textbf{Kaliningrad}, \emph{P.PPLA}|pw}}}\pend
           
\pstart
           \centering{}\textcolor{gray}{\textbf{Partie am Pregel\oindex{Pregolya@\textbf{Pregolya}, \emph{H.STM}|pw} und Blick
                  auf den Dom\oindex{Dom@\textbf{Dom}, \emph{Kirche (K.KRC)}|pw}}}\pend
           \vspace{1em}
\pstart
           {\pb}2. Dez. 09\pend
           
\pstart{}Lieber Arthur!\pend\vspace{0.5em}
\pstart
           Heute hier im Goethebund\orgindex{Goethebund@Goethebund|pw}:{\\}Schnitzlerabend von
               HermannBahr.\pend
           
\pstart
           \label{LL323-1v}So bin ich unermüdlich um Deinen Ruhm in Nord
                  u. Süd beſorgt.\label{LL323-1h}\pend
           
\pstart
           Herzlichſt, mit ſchönen Grüßen an Frau\pwindex{Schnitzler, Olga 17.01.1882 – 13.01.1970@\textsc{Schnitzler, Olga} (17.01.1882 – 13.01.1970), \emph{Schauspieler/Schauspielerin, Sänger/Sängerin}|pwv}, Sohn\pwindex{Schnitzler, Heinrich 09.08.1902 – 12.07.1982@\textsc{Schnitzler, Heinrich} (09.08.1902 – 12.07.1982), \emph{Regisseur/Regisseurin, Schauspieler/Schauspielerin}|pwv} und Tochter\pwindex{Cappellini, Lili 13.09.1909 – 26.07.1928@\textsc{Cappellini, Lili} (13.09.1909 – 26.07.1928)|pwv},{\\[\baselineskip]}Dein{\\[\baselineskip]}\spacefill\mbox{H\textcolor{gray}{m}B.}\pend
           \leftskip=0em{}\selectlanguage{ngerman}\endnumbering\briefempfaengerindex{Schnitzler, Arthur@\textsc{Schnitzler, Arthur}!zzzBahr, Hermann@\emph{von Hermann Bahr}!1909-12-021@{2. 12. 1909}|)be}\mylabel{L01891h}  \normalsize

\doendnotes{C}
\bigskip
\vfill

\clearpage

\footnotesize

\lohead{\textsc{register}}

% Definiere theindex-Environment komplett neu ohne reledmac
\makeatletter
\renewenvironment{theindex}{%
  \section*{\indexname}%
  \setlength{\parindent}{0pt}%
  \setlength{\parskip}{0pt plus 0.3pt}%
  \let\item\@idxitem
}{%
  \clearpage
}
\makeatother

\IfFileExists{\jobname-pw.ind}{\input{\jobname-pw.ind}}{}

\end{document}

      