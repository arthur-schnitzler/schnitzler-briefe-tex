%% latex-korrekturansicht-vorspann.tex
%% Vorspann für die Korrekturansicht.
%% Lädt die gemeinsame Datei latex-vorspann.tex mit gesetztem Schalter.

\newif\ifkorrekturansicht
\korrekturansichttrue

\input{../tex-inputs/latex-vorspann}


\section[Stefan Großmann an Arthur Schnitzler, 2. 4. 1910]{L01921 Stefan Großmann an Arthur Schnitzler, 2. 4. 1910}
\nopagebreak\mylabel{L01921v}
\rehead{ }\normalsize\beginnumbering\briefempfaengerindex{Schnitzler, Arthur@\textsc{Schnitzler, Arthur}!zzzGrossmann, Stefan@\emph{von Stefan Großmann}!1910-04-021@{2. 4. 1910}|(be}
\toendnotes[C]{\smallbreak\pagebreak[2]}\Standort{CUL, Schnitzler, B 34.}
\physDesc{Brief, 1 Blatt, 4 Seiten, 1424 Zeichen
\newline{}Handschrift: schwarze Tinte, lateinische Kurrent
\newline{}Schnitzler: 1) mit Bleistift beschriftet: »Großmann«  2) mit rotem Buntstift zwei Unterstreichungen
\newline{}Ordnung: mit Bleistift von unbekannter Hand nummeriert:
                                 »8« }\toendnotes[C]{\smallbreak}
\pstart
           {\pb}\textcolor{gray}{\textbf{ARBEITER-ZEITUNG\orgindex{Arbeiter-Zeitung@Arbeiter-Zeitung|pw}}}\pend
           
\pstart
           \textcolor{gray}{\textbf{Wien\oindex{Wien@\textbf{Wien}, \emph{A.ADM2}|pw}, VI/1, Mariahilferstrasse 89\oindex{Mariahilfer Strasse@\textbf{Mariahilfer Straße}, \emph{Straße (K.STR)}|pw}}}\hfill \textcolor{gray}{\textbf{Wien\oindex{Wien@\textbf{Wien}, \emph{A.ADM2}|pw}, am}} 2. IV \textcolor{gray}{\textbf{19}}10\pend
           
\pstart
           \textcolor{gray}{\textbf{Telephon 880, 900}}\pend
           
\pstart
           \textcolor{gray}{\textbf{Postsparkassen-Scheck-Konto Nr. 19.210}}\pend
           
\pstart\center{}Verehrter Herr\pend\vspace{0.5em}
\pstart
           Verzeihen Sie \strikeout{Einem} mir, dass ich Ihren Brief erst
               heute beantworte.\pend
           
\pstart
           Die Schauspieler baten mich, Sie erst zur Première\pwindex{Literatur@\emph{Literatur}|pwv}\pwindex{letzten Masken@\emph{Die letzten Masken}|pwv}\pwindex{Frage an das Schicksal@\emph{Die Frage an das Schicksal}|pwv} zu laden, heute wurde noch irrsinnig
               gearbeitet. Sie wollten nicht im Rohzustande vor Sie hintreten.\pend
           
\pstart
           Die letzte Probe fand heute nachmittag statt und endete um ¼ 7
               abends.\pend
           
\pstart
           Leider wird Sie »Literatur\pwindex{Literatur@\emph{Literatur}|pw}« nicht voll erfreun.
               Ich war krank vor Ärger, weil die Leiter des Theaters das willigste \strikeout{erf} freudigste Publikum der Freien Volksbühne\orgindex{Wiener Freie Volksbuehne@Wiener Freie Volksbühne|pw} kennen und, seine Milde missbrauchend, sagen: Da
               brauchen wir uns nicht anzustrengen.\pend
           
\pstart
           {\pb}Ich war gestern im Ärger des Tags schon
               willig Sie zu bitten, lieber zu einer späteren Aufführung zu kommen. Jedenfalls wird
               die Qualität unserer Vorstellungen durch den »halben
                  Held\pwindex{halber Held. Tragoedie in fuenf Aufzuegen@\emph{Ein halber Held. Tragödie in fünf Aufzügen}|pw}« besser repräsentirt.\pend
           
\pstart
           Ich sage das zornknirschend, aber ich will Sie lieber nicht irreführen. Wenn ich
               unser Theater selbst leiten werde, werde ich jene {\pb}Commandogewalt über die Schauspieler haben,
               die unerlässlich ist.\pend
           
\pstart
           Um Ihnen nach diesen verdriesslichen Mittheilungen zu zeigen, wie sehr mir (der
               einmal als junger Esel sehr dumm vor Ihnen stand) an Ihrem Ja und Nein gelegen ist,
               müssen Sie mir gestatten, Ihnen meine Besprechung\pwindex{Arthur Schnitzler: Der Ruf des Lebens. Zur ersten Auffuehrung im Deutschen Volkstheater@\emph{Arthur Schnitzler: Der Ruf des Lebens. Zur ersten Aufführung im Deutschen Volkstheater}|pwv} des »Ruf des
                  Lebens\pwindex{Ruf des Lebens. Schauspiel in drei Akten@\emph{Der Ruf des Lebens. Schauspiel in drei Akten}|pw}« vorzulegen. Ihnen liegt selbstverständlich nichts an {\pb}meiner Huldigung. Ich will Ihnen nur zeigen,
               einen wie \uline{andächtigen} Abend ich Ihnen verdankte.\pend
           
\pstart
           S. Fischer\pwindex{Fischer, Samuel 24.12.1859 – 15.10.1934@\textsc{Fischer, Samuel} (24.12.1859 – 15.10.1934), \emph{Verleger/Verlegerin}|pw} wurde verständigt. Seine Zustimmung
               ist zweifellos.\pend
           
\pstart
           \uuline{Dank} und ergebensten Gruß:{\\[\baselineskip]}\spacefill\mbox{Stefan Großmann}\pend
           \leftskip=0em{}\selectlanguage{ngerman}\endnumbering\briefempfaengerindex{Schnitzler, Arthur@\textsc{Schnitzler, Arthur}!zzzGrossmann, Stefan@\emph{von Stefan Großmann}!1910-04-021@{2. 4. 1910}|)be}\mylabel{L01921h}  \normalsize

\doendnotes{C}
\bigskip
\vfill

\clearpage

\footnotesize

\lohead{\textsc{register}}

% Definiere theindex-Environment komplett neu ohne reledmac
\makeatletter
\renewenvironment{theindex}{%
  \section*{\indexname}%
  \setlength{\parindent}{0pt}%
  \setlength{\parskip}{0pt plus 0.3pt}%
  \let\item\@idxitem
}{%
  \clearpage
}
\makeatother

\IfFileExists{\jobname-pw.ind}{\input{\jobname-pw.ind}}{}

\end{document}

      