%% latex-leseansicht-vorspann.tex
%% Vorspann für die Leseansicht.
%% Lädt die gemeinsame Datei latex-vorspann.tex mit nicht gesetztem Schalter.

\newif\ifkorrekturansicht
\korrekturansichtfalse

\input{../tex-inputs/latex-vorspann}


         
         \renewcommand{\erwaehntePersonen}{Personen: Samuel Fischer, Stefan Großmann}
         \renewcommand{\erwaehnteInstitutionen}{Institutionen: Arbeiter-Zeitung, Wiener Freie Volksbühne}
         \renewcommand{\erwaehnteOrte}{Orte: Mariahilferstraße, Wien}
         \renewcommand{\erwaehnteWerke}{Werke: Arthur Schnitzler: Der Ruf des Lebens. Zur ersten Aufführung im Deutschen Volkstheater, Der Ruf des Lebens. Schauspiel in drei Akten, Die Frage an das Schicksal, Die letzten Masken, Ein halber Held. Tragödie, Literatur}
               \section[Stefan Großmann an Arthur Schnitzler, 2. 4. 1910]{ Stefan Großmann an Arthur Schnitzler, 2. 4. 1910}\nopagebreak\mylabel{v}\rehead{ }\begin{ledgroupsized}[t]{13cm}\normalsize\beginnumbering\briefempfaengerindex{Schnitzler, Arthur@\textsc{Schnitzler, Arthur}!zzzGrossmann, Stefan@\emph{von Stefan Großmann}!1910-04-021@{2. 4. 1910}|(be} \toendnotes[C]{\smallbreak\pagebreak[2]} \Standort{CUL, Schnitzler, B 34.}
\physDesc{Brief, 1 Blatt, 4 Seiten, 1424 Zeichen
\newline{}Handschrift: schwarze Tinte, lateinische Kurrent
\newline{}Schnitzler: 1) mit Bleistift beschriftet: »Großmann«  2) mit rotem Buntstift zwei Unterstreichungen
\newline{}Ordnung: mit Bleistift von unbekannter Hand nummeriert:
                                 »8« }\toendnotes[C]{\smallbreak}\pstart
           \noindent{}{\pb}\textcolor{gray}{\textbf{ARBEITER-ZEITUNG\orgindex{Arbeiter-Zeitung@Arbeiter-Zeitung|pw}}}\pend
           \pstart
           \textcolor{gray}{\textbf{Wien\oindex{Wien@\textbf{Wien}|pw}, VI/1, Mariahilferstrasse 89\oindex{Mariahilferstrasse@\textbf{Mariahilferstraße}|pw}}}\hfill \textcolor{gray}{\textbf{Wien\oindex{Wien@\textbf{Wien}|pw}, am}} 2. IV \textcolor{gray}{\textbf{19}}10\pend
           \pstart
           \textcolor{gray}{\textbf{Telephon 880, 900}}\pend
           \pstart
           \textcolor{gray}{\textbf{Postsparkassen-Scheck-Konto Nr. 19.210}}\pend
           \pstart\center{}Verehrter Herr\pend\pstart
           Verzeihen Sie \strikeout{Einem} mir, dass ich Ihren Brief erst
               heute beantworte.\pend
           \pstart
           Die Schauspieler baten mich, Sie erst zur Première\pwindex{Schnitzler, Arthur 15.05.1862 – 21.10.1931@\textsc{Schnitzler, Arthur} (15.05.1862 – 21.10.1931), \emph{Schriftsteller, Mediziner}!Literatur1901@\strich\emph{Literatur} {[}1901{]}|pwv}\pwindex{Schnitzler, Arthur 15.05.1862 – 21.10.1931@\textsc{Schnitzler, Arthur} (15.05.1862 – 21.10.1931), \emph{Schriftsteller, Mediziner}!letzten Masken1901@\strich\emph{Die letzten Masken} {[}1901{]}|pwv}\pwindex{Schnitzler, Arthur 15.05.1862 – 21.10.1931@\textsc{Schnitzler, Arthur} (15.05.1862 – 21.10.1931), \emph{Schriftsteller, Mediziner}!Frage an das Schicksal01. 05. 1890@\strich\emph{Die Frage an das Schicksal} {[}01. 05. 1890{]}|pwv} zu laden, heute wurde noch irrsinnig
               gearbeitet. Sie wollten nicht im Rohzustande vor Sie hintreten.\pend
           \pstart
           Die letzte Probe fand heute nachmittag statt und endete um ¼ 7
               abends.\pend
           \pstart
           Leider wird Sie »Literatur\pwindex{Schnitzler, Arthur 15.05.1862 – 21.10.1931@\textsc{Schnitzler, Arthur} (15.05.1862 – 21.10.1931), \emph{Schriftsteller, Mediziner}!Literatur1901@\strich\emph{Literatur} {[}1901{]}|pw}« nicht voll erfreun.
               Ich war krank vor Ärger, weil die Leiter des Theaters das willigste \strikeout{erf} freudigste Publikum der Freien Volksbühne\orgindex{Wiener Freie Volksbuehne@Wiener Freie Volksbühne|pw} kennen und, seine Milde missbrauchend, sagen: Da
               brauchen wir uns nicht anzustrengen.\pend
           \pstart
           {\pb}Ich war gestern im Ärger des Tags schon
               willig Sie zu bitten, lieber zu einer späteren Aufführung zu kommen. Jedenfalls wird
               die Qualität unserer Vorstellungen durch den »halben
                  Held\pwindex{\textcolor{red}{\textsuperscript{XXXX1 indx}}!halber Held. Tragoedie1904@\strich\emph{Ein halber Held. Tragödie} {[}1904{]}|pw}« besser repräsentirt.\pend
           \pstart
           Ich sage das zornknirschend, aber ich will Sie lieber nicht irreführen. Wenn ich
               unser Theater selbst leiten werde, werde ich jene {\pb}Commandogewalt über die Schauspieler haben,
               die unerlässlich ist.\pend
           \pstart
           Um Ihnen nach diesen verdriesslichen Mittheilungen zu zeigen, wie sehr mir (der
               einmal als junger Esel sehr dumm vor Ihnen stand) an Ihrem Ja und Nein gelegen ist,
               müssen Sie mir gestatten, Ihnen meine Besprechung\pwindex{Arthur Schnitzler: Der Ruf des Lebens. Zur ersten Auffuehrung im Deutschen Volkstheater1909-12-17@\emph{Arthur Schnitzler: Der Ruf des Lebens. Zur ersten Aufführung im Deutschen Volkstheater} {[}1909-12-17{]}|pwv} des »Ruf des
                  Lebens\pwindex{Schnitzler, Arthur 15.05.1862 – 21.10.1931@\textsc{Schnitzler, Arthur} (15.05.1862 – 21.10.1931), \emph{Schriftsteller, Mediziner}!Ruf des Lebens. Schauspiel in drei Akten1906-02-20@\strich\emph{Der Ruf des Lebens. Schauspiel in drei Akten} {[}1906-02-20{]}|pw}« vorzulegen. Ihnen liegt selbstverständlich nichts an {\pb}meiner Huldigung. Ich will Ihnen nur zeigen,
               einen wie \uline{andächtigen} Abend ich Ihnen verdankte.\pend
           \pstart
           S. Fischer\pwindex{Fischer, Samuel 24.12.1859 – 15.10.1934@\textsc{Fischer, Samuel} (24.12.1859 – 15.10.1934), \emph{Verleger}|pw} wurde verständigt. Seine Zustimmung
               ist zweifellos.\pend
           \pstart
           \uuline{Dank} und ergebensten Gruß:{\\[\baselineskip]}\spacefill\mbox{Stefan Großmann}\pend
           \leftskip=0em{}
         
         \endnumbering\mylabel{h}\end{ledgroupsized}  \newcommand{\dateiname}{L01921}\newcommand{\titel}{Stefan Großmann an Arthur Schnitzler, 2. 4. 1910}\newcommand{\editorInnen}{ Martin Anton Müller und Gerd-Hermann Susen}%% latex-leseansicht-abspann.tex
%% Abspann für die Leseansicht.
%% Der Schalter \ifkorrekturansicht ist bereits durch den Vorspann gesetzt.

%% latex-abspann.tex
%% Gemeinsamer Abspann für Korrekturansicht und Leseansicht.
%% Setzt den Schalter \ifkorrekturansicht voraus (gesetzt in den
%% einbindenden Dateien latex-korrekturansicht-abspann.tex bzw.
%% latex-leseansicht-abspann.tex).
%% ---------------------------------------------------------------

\normalsize

% Das esempio-Environment wird nur in der Leseansicht benötigt
\ifkorrekturansicht\else
\newenvironment{esempio}[3]%
{
    \vspace{1.5ex}
    \rlap{\underline{#1}}
    \par
    \setlength{\parindent}{0cm}
    \nopagebreak
    \leftskip=#2cm
    \rightskip=#3cm
}
{
    \par
}
\fi

\doendnotes{C}
\bigskip
\vfill

\clearpage

\footnotesize

\ifkorrekturansicht
  \lohead{\textsc{register}}
\fi

% theindex-Environment neu definieren ohne reledmac
\makeatletter
\renewenvironment{theindex}{%
  \ifkorrekturansicht
    \section*{\indexname}%
  \else
    \subsubsection*{Index der erwähnten Entitäten}%
  \fi
  \setlength{\parindent}{0pt}%
  \setlength{\parskip}{0pt plus 0.3pt}%
  \let\item\@idxitem
}{%
  \ifkorrekturansicht\clearpage\fi
}
\makeatother

\IfFileExists{\jobname-pw.ind}{\input{\jobname-pw.ind}}{}

% Quellenangabe nur in der Leseansicht
\ifkorrekturansicht\else
% Fallback-Definitionen, falls die .tex-Datei \titel etc. nicht gesetzt hat
\providecommand{\titel}{}
\providecommand{\editorInnen}{}
\providecommand{\dateiname}{\jobname}

\vspace{3cm}

\vfill

\footnotesize
\textsc{Quelle}: \titel. Herausgegeben von {\editorInnen}. In: \emph{Arthur Schnitzler: Briefwechsel mit Autorinnen und Autoren}.
 Digitale Edition, https://schnitzler-briefe.acdh.oeaw.ac.at/{\dateiname}.html (Stand \today)
\fi

\end{document}


      