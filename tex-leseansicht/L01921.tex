%% latex-leseansicht-vorspann.tex
%% Vorspann für die Leseansicht.
%% Lädt die gemeinsame Datei latex-vorspann.tex mit nicht gesetztem Schalter.

\newif\ifkorrekturansicht
\korrekturansichtfalse

\input{../tex-inputs/latex-vorspann}


\section[Stefan Großmann an Arthur Schnitzler, 2. 4. 1910]{L01921 Stefan Großmann an Arthur Schnitzler, 2. 4. 1910}
\nopagebreak\mylabel{L01921v}
\rehead{ }\normalsize\beginnumbering\briefempfaengerindex{Schnitzler, Arthur@\textsc{Schnitzler, Arthur}!zzzGroßmann, Stefan@\emph{von Stefan Großmann}!1910-04-021@{2. 4. 1910}|(be}
\toendnotes[C]{\smallbreak\pagebreak[2]}
\correspDesc{Versand  durch Stefan Großmann am 2. 4. 1910 in Wien
\newline{}Erhalt  durch Arthur Schnitzler im Zeitraum [2. 4. 1910
                  – 6. 4. 1910?] in Wien}\toendnotes[C]{\smallbreak}
\Standort{CUL, Schnitzler, B 34.}
\physDesc{Brief, 1 Blatt, 4 Seiten, 1424 Zeichen
\newline{}Handschrift: schwarze Tinte, lateinische Kurrent
\newline{}Schnitzler: 1) mit Bleistift beschriftet: »Großmann«  2) mit rotem Buntstift zwei Unterstreichungen
\newline{}Ordnung: mit Bleistift von unbekannter Hand nummeriert:
                                 »8« }\toendnotes[C]{\smallbreak}
\pstart
           {\pb}\textcolor{gray}{\textbf{ARBEITER-ZEITUNG\orgindex{Arbeiter-Zeitung@Arbeiter-Zeitung|pw}}}\pend
           
\pstart
           \textcolor{gray}{\textbf{Wien\oindex{Wien@\textbf{Wien}, \emph{Verwaltungsgebiet}|pw}, VI/1, Mariahilferstrasse 89\oindex{Wien@\textbf{Wien}!VI., Mariahilf@\textbf{VI., Mariahilf}!Mariahilfer Straße@\textbf{Mariahilfer Straße}, \emph{Straße}|pw}}}\hfill \textcolor{gray}{\textbf{Wien\oindex{Wien@\textbf{Wien}, \emph{Verwaltungsgebiet}|pw}, am}} 2. IV \textcolor{gray}{\textbf{19}}10\pend
           
\pstart
           \textcolor{gray}{\textbf{Telephon 880, 900}}\pend
           
\pstart
           \textcolor{gray}{\textbf{Postsparkassen-Scheck-Konto Nr. 19.210}}\pend
           
\pstart\center{}Verehrter Herr\pend\vspace{0.5em}
\pstart
           Verzeihen Sie \strikeout{Einem} mir, dass ich Ihren Brief erst
               heute beantworte.\pend
           
\pstart
           Die Schauspieler baten mich, Sie erst zur Première\pwindex{Schnitzler, Arthur 15.\,5.\,1862 Wien – 21.\,10.\,1931 ebd.@\textsc{Schnitzler, Arthur} (15.\,5.\,1862 Wien – 21.\,10.\,1931 ebd.), \emph{Schriftsteller, Mediziner}!Literatur@\strich\emph{Literatur}|pwv}\pwindex{Schnitzler, Arthur 15.\,5.\,1862 Wien – 21.\,10.\,1931 ebd.@\textsc{Schnitzler, Arthur} (15.\,5.\,1862 Wien – 21.\,10.\,1931 ebd.), \emph{Schriftsteller, Mediziner}!letzten Masken@\strich\emph{Die letzten Masken}|pwv}\pwindex{Schnitzler, Arthur 15.\,5.\,1862 Wien – 21.\,10.\,1931 ebd.@\textsc{Schnitzler, Arthur} (15.\,5.\,1862 Wien – 21.\,10.\,1931 ebd.), \emph{Schriftsteller, Mediziner}!Frage an das Schicksal@\strich\emph{Die Frage an das Schicksal}|pwv} zu laden, heute wurde noch irrsinnig
               gearbeitet. Sie wollten nicht im Rohzustande vor Sie hintreten.\pend
           
\pstart
           Die letzte Probe fand heute nachmittag statt und endete um ¼ 7
               abends.\pend
           
\pstart
           Leider wird Sie »Literatur\pwindex{Schnitzler, Arthur 15.\,5.\,1862 Wien – 21.\,10.\,1931 ebd.@\textsc{Schnitzler, Arthur} (15.\,5.\,1862 Wien – 21.\,10.\,1931 ebd.), \emph{Schriftsteller, Mediziner}!Literatur@\strich\emph{Literatur}|pw}« nicht voll erfreun.
               Ich war krank vor Ärger, weil die Leiter des Theaters das willigste \strikeout{erf} freudigste Publikum der Freien Volksbühne\orgindex{Wiener Freie Volksbühne@Wiener Freie Volksbühne|pw} kennen und, seine Milde missbrauchend, sagen: Da
               brauchen wir uns nicht anzustrengen.\pend
           
\pstart
           {\pb}Ich war gestern im Ärger des Tags schon
               willig Sie zu bitten, lieber zu einer späteren Aufführung zu kommen. Jedenfalls wird
               die Qualität unserer Vorstellungen durch den »halben
                  Held\pwindex{\textcolor{red}{\textsuperscript{XXXX indx1}}!halber Held. Tragödie in fünf Aufzügen@\strich\emph{Ein halber Held. Tragödie in fünf Aufzügen}|pw}« besser repräsentirt.\pend
           
\pstart
           Ich sage das zornknirschend, aber ich will Sie lieber nicht irreführen. Wenn ich
               unser Theater selbst leiten werde, werde ich jene {\pb}Commandogewalt über die Schauspieler haben,
               die unerlässlich ist.\pend
           
\pstart
           Um Ihnen nach diesen verdriesslichen Mittheilungen zu zeigen, wie sehr mir (der
               einmal als junger Esel sehr dumm vor Ihnen stand) an Ihrem Ja und Nein gelegen ist,
               müssen Sie mir gestatten, Ihnen meine Besprechung\pwindex{Großmann, Stefan 19.\,5.\,1875 Wien – 3.\,1.\,1935 ebd.@\textsc{Großmann, Stefan} (19.\,5.\,1875 Wien – 3.\,1.\,1935 ebd.), \emph{Schriftsteller, Journalist}!Arthur Schnitzler: Der Ruf des Lebens. Zur ersten Aufführung im Deutschen Volkstheater@\strich\emph{Arthur Schnitzler: Der Ruf des Lebens. Zur ersten Aufführung im Deutschen Volkstheater}|pwv} des »Ruf des
                  Lebens\pwindex{Schnitzler, Arthur 15.\,5.\,1862 Wien – 21.\,10.\,1931 ebd.@\textsc{Schnitzler, Arthur} (15.\,5.\,1862 Wien – 21.\,10.\,1931 ebd.), \emph{Schriftsteller, Mediziner}!Ruf des Lebens. Schauspiel in drei Akten@\strich\emph{Der Ruf des Lebens. Schauspiel in drei Akten}|pw}« vorzulegen. Ihnen liegt selbstverständlich nichts an {\pb}meiner Huldigung. Ich will Ihnen nur zeigen,
               einen wie \uline{andächtigen} Abend ich Ihnen verdankte.\pend
           
\pstart
           S. Fischer\pwindex{Fischer, Samuel 24.\,12.\,1859 Liptovský Mikuláš – 15.\,10.\,1934 Berlin@\textsc{Fischer, Samuel} (24.\,12.\,1859 Liptovský Mikuláš – 15.\,10.\,1934 Berlin), \emph{Verleger}|pw} wurde verständigt. Seine Zustimmung
               ist zweifellos.\pend
           
\pstart
           \uuline{Dank} und ergebensten Gruß:{\\[\baselineskip]}\spacefill\mbox{Stefan Großmann}\pend
           \leftskip=0em{}\selectlanguage{ngerman}\endnumbering\briefempfaengerindex{Schnitzler, Arthur@\textsc{Schnitzler, Arthur}!zzzGroßmann, Stefan@\emph{von Stefan Großmann}!1910-04-021@{2. 4. 1910}|)be}\mylabel{L01921h}  \newcommand{\dateiname}{L01921}\newcommand{\titel}{Stefan Großmann an Arthur Schnitzler, 2. 4. 1910}\newcommand{\editorInnen}{Herausgegeben von Martin Anton Müller}%% latex-leseansicht-abspann.tex
%% Abspann für die Leseansicht.
%% Der Schalter \ifkorrekturansicht ist bereits durch den Vorspann gesetzt.

%% latex-abspann.tex
%% Gemeinsamer Abspann für Korrekturansicht und Leseansicht.
%% Setzt den Schalter \ifkorrekturansicht voraus (gesetzt in den
%% einbindenden Dateien latex-korrekturansicht-abspann.tex bzw.
%% latex-leseansicht-abspann.tex).
%% ---------------------------------------------------------------

\normalsize

% Das esempio-Environment wird nur in der Leseansicht benötigt
\ifkorrekturansicht\else
\newenvironment{esempio}[3]%
{
    \vspace{1.5ex}
    \rlap{\underline{#1}}
    \par
    \setlength{\parindent}{0cm}
    \nopagebreak
    \leftskip=#2cm
    \rightskip=#3cm
}
{
    \par
}
\fi

\doendnotes{C}
\bigskip
\vfill

\clearpage

\footnotesize

\ifkorrekturansicht
  \lohead{\textsc{register}}
\fi

% theindex-Environment neu definieren ohne reledmac
\makeatletter
\renewenvironment{theindex}{%
  \ifkorrekturansicht
    \section*{\indexname}%
  \else
    \subsubsection*{Index der erwähnten Entitäten}%
  \fi
  \setlength{\parindent}{0pt}%
  \setlength{\parskip}{0pt plus 0.3pt}%
  \let\item\@idxitem
}{%
  \ifkorrekturansicht\clearpage\fi
}
\makeatother

\IfFileExists{\jobname-pw.ind}{\input{\jobname-pw.ind}}{}

% Quellenangabe nur in der Leseansicht
\ifkorrekturansicht\else
% Fallback-Definitionen, falls die .tex-Datei \titel etc. nicht gesetzt hat
\providecommand{\titel}{}
\providecommand{\editorInnen}{}
\providecommand{\dateiname}{\jobname}

\vspace{3cm}

\vfill

\footnotesize
\textsc{Quelle}: \titel. Herausgegeben von {\editorInnen}. In: \emph{Arthur Schnitzler: Briefwechsel mit Autorinnen und Autoren}.
 Digitale Edition, https://schnitzler-briefe.acdh.oeaw.ac.at/{\dateiname}.html (Stand \today)
\fi

\end{document}


