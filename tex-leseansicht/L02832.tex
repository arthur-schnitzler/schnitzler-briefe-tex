%% latex-leseansicht-vorspann.tex
%% Vorspann für die Leseansicht.
%% Lädt die gemeinsame Datei latex-vorspann.tex mit nicht gesetztem Schalter.

\newif\ifkorrekturansicht
\korrekturansichtfalse

\input{../tex-inputs/latex-vorspann}


         
         \renewcommand{\erwaehntePersonen}{Personen: Alfred Dreyfus, Paul Goldmann, Maximilian von Schwartzkoppen, Leopold Sonnemann, Ferdinand Walsin-Esterházy, Alice Ziegler}
         \renewcommand{\erwaehnteInstitutionen}{Institutionen: Frankfurter Zeitung}
         \renewcommand{\erwaehnteOrte}{Orte: Deutsches Haus, Paris, Prag, rue de la Bourse}
         \renewcommand{\erwaehnteWerke}{Werke: Das Vermächtnis. Schauspiel in drei Akten, Freiwild. Schauspiel in 3 Akten}
               \section[ Paul Goldmann an Arthur Schnitzler, 24. 11. {[}1897{]}]{ Paul Goldmann an Arthur Schnitzler, 24. 11. {[}1897{]}}\nopagebreak\mylabel{v}\rehead{ }\begin{ledgroupsized}[t]{13cm}\normalsize\beginnumbering \toendnotes[C]{\smallbreak\pagebreak[2]} \Standort{DLA, A:Schnitzler, HS.NZ85.1.3167.}
\physDesc{Brief, 1 Blatt, 3 Seiten, 1413 Zeichen
\newline{}Handschrift: blaue Tinte, deutsche Kurrent
\newline{}Schnitzler: 1) mit Bleistift das Jahr »97« vermerkt  2) mit rotem Buntstift eine Unterstreichung}\toendnotes[C]{\smallbreak}\pstart
           \noindent{}{\pb}\textcolor{gray}{\textbf{\textbf{Frankfurter Zeitung\orgindex{Frankfurter Zeitung@Frankfurter Zeitung|pw}}}}\pend
           \pstart
           \textcolor{gray}{\textbf{(\begin{otherlanguage}{french}Gazette de Francfort\end{otherlanguage}\orgindex{Frankfurter Zeitung@Frankfurter Zeitung|pw}).}}\pend
           \pstart
           \textcolor{gray}{\textbf{\textbf{\begin{otherlanguage}{french}Fondateur M.\end{otherlanguage}{ }L. Sonnemann\pwindex{Sonnemann, Leopold 1831-10-29 – 1909-10-30@\textsc{Sonnemann, Leopold} (1831-10-29 – 1909-10-30), \emph{Journalist, Herausgeber}|pw}.}}}\pend
           \pstart
           \begin{otherlanguage}{french}\textcolor{gray}{\textbf{Journal politique, financier,}}\end{otherlanguage}\pend
           \pstart
           \begin{otherlanguage}{french}\textcolor{gray}{\textbf{commercial et littéraire.}}\end{otherlanguage}\pend
           \pstart
           \begin{otherlanguage}{french}\textcolor{gray}{\textbf{\textbf{Paraissant trois fois par jour.}}}\end{otherlanguage}\hfill \textsc{Paris\oindex{Paris@\textbf{Paris}|pw}}, 24. November.\pend
           \pstart
           \begin{otherlanguage}{french}\textcolor{gray}{\textbf{\textbf{Bureau à Paris\oindex{Paris@\textbf{Paris}|pw}}}}\end{otherlanguage}\pend
           \pstart
           \begin{otherlanguage}{french}\textcolor{gray}{\textbf{\textbf{10 \so{Rue de la Bourse}\oindex{rue de la Bourse@\textbf{rue de la Bourse}|pw}.}}}\end{otherlanguage}\pend
           \pstart\center{}Mein lieber Freund,\pend\pstart
           Ich hoffe, die kleine \label{K_L02832-1v}\edtext{Reiſe}{\lemma{\textnormal{\emph{Reiſe}}}\Cendnote{\textnormal{Schnitzler\pwindex{Schnitzler, Arthur 15.05.1862 – 21.10.1931@\textsc{Schnitzler, Arthur} (15.05.1862 – 21.10.1931), \emph{Schriftsteller, Mediziner}|pwk} hielt sich von 24. 11. 1897 bis 28. 11. 1897 in Prag\oindex{Prag@\textbf{Prag}|pwk} auf. Am 25. 11. 1897 las er im gut besuchten Deutschen Haus\oindex{Deutsches Haus@\textbf{Deutsches Haus}|pwkv} und am 27. 11. 1897 fand die
                  Premiere von \emph{Freiwild}\pwindex{Schnitzler, Arthur 15.05.1862 – 21.10.1931@\textsc{Schnitzler, Arthur} (15.05.1862 – 21.10.1931), \emph{Schriftsteller, Mediziner}!Freiwild. Schauspiel in 3 Akten1896@\strich\emph{Freiwild. Schauspiel in 3 Akten} {[}1896{]}|pwk} statt – ein
                     »Erfolg; anfangs sehr stark, gegen Schluss sich schwächend.«
                     (A. S.: \emph{Tagebuch}, 27. 11. 1897)}}}\label{K_L02832-1h} wird Dir gut
               anſchlagen und Dich aus Deinen Hypochondrien herausreißen. Auch gibt es hoffentlich
               in \textsc{Prag\oindex{Prag@\textbf{Prag}|pw}} neue Erfolge. Wenigſtens wünſche ich das von Herzen.\pend
           \pstart
           Als ich heut Deinen Brief erhielt, bekam ich eine \strikeout{Se} ſolche Sehnſucht nach Heimath und Freunden und Ruhe!
               Und ich hatte eine ſolche Luſt, all’ dieſe undankbare Arbeit hier hinzuwerfen, die
               mir meine Geſundheit zerrüttet und mich um mein Leben beſtiehlt!\pend
           \pstart
           {\pb}Was bin ich doch für ein armer Sklave! Und wie biſt
               Du glücklich gegen mich, ſelbſt mit \label{K_L02832-2v}\edtext{Ohrenklingen}{\lemma{\textnormal{\emph{Ohrenklingen}}}\Cendnote{\textnormal{Bezug auf Schnitzler\pwindex{Schnitzler, Arthur 15.05.1862 – 21.10.1931@\textsc{Schnitzler, Arthur} (15.05.1862 – 21.10.1931), \emph{Schriftsteller, Mediziner}|pwk}s Otosklerose – eine Verknöcherung
                  des Innenohrs mit zunehmender Schwerhörigkeit –, an der er seit
                     Herbst 1896 litt}}}\label{K_L02832-2h}. Ich wünſchte, mir kl\substVorne{}\textsuperscript{\textcolor{gray}{i}}\substDazwischen{}ä\substHinten{}ngen die Ohren ſo wie Dir!\pend
           \pstart
           Dein Stück\pwindex{Schnitzler, Arthur 15.05.1862 – 21.10.1931@\textsc{Schnitzler, Arthur} (15.05.1862 – 21.10.1931), \emph{Schriftsteller, Mediziner}!Vermaechtnis. Schauspiel in drei Akten1898@\strich\emph{Das Vermächtnis. Schauspiel in drei Akten} {[}1898{]}|pwv} wird ſich ſchon aus
               dem Unklaren herausarbeiten. Kein Wunder, daß es nicht gleich \label{K_L02832-3v}\edtext{auf den erſten Wurf gelungen}{\lemma{\textnormal{\emph{auf … gelungen}}}\Cendnote{\textnormal{siehe A. S.: \emph{Tagebuch}, 21. 11. 1897}}}\label{K_L02832-3h} iſt, bei all’ den Aufregungen, welche Du haſt durchmachen müſſen. Auch haſt
               Du ja ſtets Deine Stücke mehrmals geſchrieben. Und wenn \strikeout{es} gar ſo kein Talent dazu gehörte, einen {\pb}guten erſten Akt zu ſchreiben, ſo gäbe es mehr gute erſte Akte, als es gibt.\pend
           \pstart
           Warum Du von Deiner trüben Zukunft ſprichſt, begreife ich auch nicht. Ich finde das
               genaue Gegentheil.\pend
           \pstart
           Alſo erhole Dich recht und genieße die Prag\oindex{Prag@\textbf{Prag}|pwv}er Tage!\pend
           \pstart
           Und \label{K_L02832-4v}\edtext{ſieh’ Dir das liebe Geſicht des
               kleinen Mädchen\pwindex{Ziegler, Alice 1880-01-05 – Dezember 1943@\textsc{Ziegler, Alice} (1880-01-05 – Dezember 1943)|pwv}s an}{\lemma{\textnormal{\emph{ſieh’ … an}}}\Cendnote{\textnormal{siehe Paul Goldmann an Arthur Schnitzler, 19. 11. [1897]}}}\label{K_L02832-4h} und ſage mir, was darin ſteht.\pend
           \pstart
           Berichte mir \strikeout{ba} bald und viel!\pend
           \pstart
           Von Herzen {\\[\baselineskip]}Dein {\\[\baselineskip]}\spacefill\mbox{Paul Goldmnn}\pend
           \leftskip=0em{}\pstart
           \noindent{}\label{T_L02832-1v}\edtext{{\pb}Ich hoffe, es kommt zur \label{K_L02832-5v}\edtext{Reviſion des Prozeſſes \textsc{Dreyfus\pwindex{Dreyfus, Alfred 1859-10-09 – 1935-07-12@\textsc{Dreyfus, Alfred} (1859-10-09 – 1935-07-12), \emph{Militär}|pw}}}{\lemma{\textnormal{\emph{Reviſion … Dreyfus}}}\Cendnote{\textnormal{Zu einem weiteren Gerichtsprozess in
                     der Dreyfus\pwindex{Dreyfus, Alfred 1859-10-09 – 1935-07-12@\textsc{Dreyfus, Alfred} (1859-10-09 – 1935-07-12), \emph{Militär}|pwk}-Affäre kam es erst am 10. und 11. 1. 1898.
                        Ferdinand Walsin-Esterházy\pwindex{Walsin-Esterházy, Ferdinand 1847-12-16 – 1923-05-21@\textsc{Walsin-Esterházy, Ferdinand} (1847-12-16 – 1923-05-21), \emph{Offizier, Spion}|pwk}, der das
                     Gerichtsverfahren gegen sich selbst beantragt hatte, wurde dort freigesprochen.
                     Eigentlich war aber er – und nicht Alfred
                        Dreyfus\pwindex{Dreyfus, Alfred 1859-10-09 – 1935-07-12@\textsc{Dreyfus, Alfred} (1859-10-09 – 1935-07-12), \emph{Militär}|pwk} – schuldig. Er hatte Maximilian von Schwartzkoppen\pwindex{Schwartzkoppen, Maximilian von 1850-02-24 – 1917-01-08@\textsc{Schwartzkoppen, Maximilian von} (1850-02-24 – 1917-01-08), \emph{Militärattaché, General}|pwk} (gegen Geld) die geheimen militärischen
                     Dokumente gegeben, die die Dreyfus\pwindex{Dreyfus, Alfred 1859-10-09 – 1935-07-12@\textsc{Dreyfus, Alfred} (1859-10-09 – 1935-07-12), \emph{Militär}|pwk}-Affäre
                     ausgelöst hatten.}}}\label{K_L02832-5h}. Der \textsc{Esterhazy\pwindex{Walsin-Esterházy, Ferdinand 1847-12-16 – 1923-05-21@\textsc{Walsin-Esterházy, Ferdinand} (1847-12-16 – 1923-05-21), \emph{Offizier, Spion}|pw}} iſt wohl ſchuldig. Aber weſſen? Des Verraths? Der Fälſchung? Dunkel,
                     dunkel!}{\lemma{\textnormal{\emph{Ich … dunkel!}}}\Cendnote{\textnormal{kopfüber am oberen Rand der
                     ersten Seite}}}\label{T_L02832-1h}\pend
           
         
         \endnumbering\mylabel{h}\end{ledgroupsized}  \newcommand{\dateiname}{L02832}\newcommand{\titel}{Paul Goldmann an Arthur Schnitzler, 24. 11. [1897]}\newcommand{\editorInnen}{Martin Anton Müller und Laura Untner}%% latex-leseansicht-abspann.tex
%% Abspann für die Leseansicht.
%% Der Schalter \ifkorrekturansicht ist bereits durch den Vorspann gesetzt.

%% latex-abspann.tex
%% Gemeinsamer Abspann für Korrekturansicht und Leseansicht.
%% Setzt den Schalter \ifkorrekturansicht voraus (gesetzt in den
%% einbindenden Dateien latex-korrekturansicht-abspann.tex bzw.
%% latex-leseansicht-abspann.tex).
%% ---------------------------------------------------------------

\normalsize

% Das esempio-Environment wird nur in der Leseansicht benötigt
\ifkorrekturansicht\else
\newenvironment{esempio}[3]%
{
    \vspace{1.5ex}
    \rlap{\underline{#1}}
    \par
    \setlength{\parindent}{0cm}
    \nopagebreak
    \leftskip=#2cm
    \rightskip=#3cm
}
{
    \par
}
\fi

\doendnotes{C}
\bigskip
\vfill

\clearpage

\footnotesize

\ifkorrekturansicht
  \lohead{\textsc{register}}
\fi

% theindex-Environment neu definieren ohne reledmac
\makeatletter
\renewenvironment{theindex}{%
  \ifkorrekturansicht
    \section*{\indexname}%
  \else
    \subsubsection*{Index der erwähnten Entitäten}%
  \fi
  \setlength{\parindent}{0pt}%
  \setlength{\parskip}{0pt plus 0.3pt}%
  \let\item\@idxitem
}{%
  \ifkorrekturansicht\clearpage\fi
}
\makeatother

\IfFileExists{\jobname-pw.ind}{\input{\jobname-pw.ind}}{}

% Quellenangabe nur in der Leseansicht
\ifkorrekturansicht\else
% Fallback-Definitionen, falls die .tex-Datei \titel etc. nicht gesetzt hat
\providecommand{\titel}{}
\providecommand{\editorInnen}{}
\providecommand{\dateiname}{\jobname}

\vspace{3cm}

\vfill

\footnotesize
\textsc{Quelle}: \titel. Herausgegeben von {\editorInnen}. In: \emph{Arthur Schnitzler: Briefwechsel mit Autorinnen und Autoren}.
 Digitale Edition, https://schnitzler-briefe.acdh.oeaw.ac.at/{\dateiname}.html (Stand \today)
\fi

\end{document}


      