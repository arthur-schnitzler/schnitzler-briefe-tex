%% latex-leseansicht-vorspann.tex
%% Vorspann für die Leseansicht.
%% Lädt die gemeinsame Datei latex-vorspann.tex mit nicht gesetztem Schalter.

\newif\ifkorrekturansicht
\korrekturansichtfalse

\input{../tex-inputs/latex-vorspann}


\section[ Paul Goldmann an Arthur Schnitzler, 24. 11. {[}1897{]}]{L02832 Paul Goldmann an Arthur Schnitzler,  24. 11. [1897]}
\nopagebreak\mylabel{L02832v}
\rehead{ }\normalsize\beginnumbering\briefempfaengerindex{Schnitzler, Arthur@\textsc{Schnitzler, Arthur}!zzzGoldmann, Paul@\emph{von Paul Goldmann}!1897-11-241@{24. 11. [1897]}|(be}
\toendnotes[C]{\smallbreak\pagebreak[2]}
\correspDesc{Versand  durch Paul Goldmann am 24. 11. [1897] in Paris
\newline{}Erhalt  durch Arthur Schnitzler im Zeitraum [25. 11. 1897 – 29. 11. 1897?] in Prag}\toendnotes[C]{\smallbreak}
\Standort{DLA, A:Schnitzler, HS.NZ85.1.3167.}
\physDesc{Brief, 1 Blatt, 3 Seiten, 1413 Zeichen
\newline{}Handschrift: blaue Tinte, deutsche Kurrent
\newline{}Schnitzler: 1) mit Bleistift das Jahr »97« vermerkt  2) mit rotem Buntstift eine Unterstreichung}\toendnotes[C]{\smallbreak}
\pstart
           {\pb}\textcolor{gray}{\textbf{\textbf{Frankfurter Zeitung\orgindex{Frankfurter Zeitung@Frankfurter Zeitung|pw}}}}\pend
           
\pstart
           \textcolor{gray}{\textbf{(\begin{otherlanguage}{french}Gazette de Francfort\end{otherlanguage}\orgindex{Frankfurter Zeitung@Frankfurter Zeitung|pw}).}}\pend
           
\pstart
           \textcolor{gray}{\textbf{\textbf{\begin{otherlanguage}{french}Fondateur M.\end{otherlanguage}{ }L. Sonnemann\pwindex{Sonnemann, Leopold 29.\,10.\,1831 Höchberg – 30.\,10.\,1909 Frankfurt am Main@\textsc{Sonnemann, Leopold} (29.\,10.\,1831 Höchberg – 30.\,10.\,1909 Frankfurt am Main), \emph{Journalist, Herausgeber}|pw}.}}}\pend
           
\pstart
           \begin{otherlanguage}{french}\textcolor{gray}{\textbf{Journal politique, financier,}}\end{otherlanguage}\pend
           
\pstart
           \begin{otherlanguage}{french}\textcolor{gray}{\textbf{commercial et littéraire.}}\end{otherlanguage}\pend
           
\pstart
           \begin{otherlanguage}{french}\textcolor{gray}{\textbf{\textbf{Paraissant trois fois par jour.}}}\end{otherlanguage}\hfill \textsc{Paris\oindex{Paris@\textbf{Paris}, \emph{Hauptstadt}|pw}}, 24. November.\pend
           
\pstart
           \begin{otherlanguage}{french}\textcolor{gray}{\textbf{\textbf{Bureau à Paris\oindex{Paris@\textbf{Paris}, \emph{Hauptstadt}|pw}}}}\end{otherlanguage}\pend
           
\pstart
           \begin{otherlanguage}{french}\textcolor{gray}{\textbf{\textbf{10 \so{Rue de la Bourse}\oindex{rue de la Bourse@\textbf{rue de la Bourse}, \emph{Straße}|pw}.}}}\end{otherlanguage}\pend
           
\pstart\center{}Mein lieber Freund,\pend\vspace{0.5em}
\pstart
           Ich hoffe, die kleine \label{K_L02832-1v}\edtext{Reiſe}{\lemma{\textnormal{\emph{Reise}}}\Cendnote{\textnormal{Schnitzler hielt sich vom 24. 11. 1897 bis zum 28. 11. 1897 in Prag\oindex{Prag@\textbf{Prag}, \emph{Land}|pwk} auf. Am 25. 11. 1897 las er im gut besuchten Deutschen Haus\oindex{Deutsches Haus [Prag]@\textbf{Deutsches Haus [Prag]}, \emph{Gebäude}|pwkv} und am 27. 11. 1897 fand die
                  Premiere von \emph{Freiwild}\pwindex{Schnitzler, Arthur 15.\,5.\,1862 Wien – 21.\,10.\,1931 ebd.@\textsc{Schnitzler, Arthur} (15.\,5.\,1862 Wien – 21.\,10.\,1931 ebd.), \emph{Schriftsteller, Mediziner}!Freiwild. Schauspiel in 3 Akten@\strich\emph{Freiwild. Schauspiel in 3 Akten}|pwk} statt – ein
                     »Erfolg; anfangs sehr stark, gegen Schluss sich schwächend.«
                     (A. S.: \emph{Tagebuch}, 27. 11. 1897.)}}}\label{K_L02832-1} wird Dir gut
               anſchlagen und Dich aus Deinen Hypochondrien herausreißen. Auch gibt es hoffentlich
               in \textsc{Prag\oindex{Prag@\textbf{Prag}, \emph{Land}|pw}} neue Erfolge. Wenigſtens wünſche ich das von Herzen.\pend
           
\pstart
           Als ich heut Deinen Brief erhielt, bekam ich eine \strikeout{Se}{ }ſolche Sehnſucht nach Heimath und Freunden und Ruhe!
               Und ich hatte eine{ }ſolche Luſt, all’ dieſe undankbare Arbeit hier hinzuwerfen, die
               mir meine Geſundheit zerrüttet und mich um mein Leben beſtiehlt!\pend
           
\pstart
           {\pb}Was bin ich doch für ein armer Sklave! Und wie biſt
               Du glücklich gegen mich,{ }ſelbſt mit \label{K_L02832-2v}\edtext{Ohrenklingen}{\lemma{\textnormal{\emph{Ohrenklingen}}}\Cendnote{\textnormal{Bezug auf Schnitzlers Otosklerose – eine Verknöcherung
                  des Innenohrs mit zunehmender Schwerhörigkeit –, an der er seit
                     Herbst 1896 litt}}}\label{K_L02832-2}. Ich wünſchte, mir kl\substVorne{}\textsuperscript{\textcolor{gray}{i}}\substDazwischen{}ä\substHinten{}ngen die Ohren{ }ſo wie Dir!\pend
           
\pstart
           Dein Stück\pwindex{Schnitzler, Arthur 15.\,5.\,1862 Wien – 21.\,10.\,1931 ebd.@\textsc{Schnitzler, Arthur} (15.\,5.\,1862 Wien – 21.\,10.\,1931 ebd.), \emph{Schriftsteller, Mediziner}!Vermächtnis. Schauspiel in drei Akten@\strich\emph{Das Vermächtnis. Schauspiel in drei Akten}|pwv} wird{ }ſich{ }ſchon aus
               dem Unklaren herausarbeiten. Kein Wunder, daß es nicht gleich \label{K_L02832-3v}\edtext{auf den erſten Wurf gelungen}{\lemma{\textnormal{\emph{auf … gelungen}}}\Cendnote{\textnormal{Siehe A. S.: \emph{Tagebuch}, 21. 11. 1897.
               }}}\label{K_L02832-3} iſt, bei all’ den Aufregungen, welche Du haſt durchmachen müſſen. Auch haſt
               Du ja{ }ſtets Deine Stücke mehrmals geſchrieben. Und wenn \strikeout{es} gar{ }ſo kein Talent dazu gehörte, einen {\pb}guten erſten Akt zu{ }ſchreiben,{ }ſo gäbe es mehr gute erſte Akte, als es gibt.\pend
           
\pstart
           Warum Du von Deiner trüben Zukunft{ }ſprichſt, begreife ich auch nicht. Ich finde das
               genaue Gegentheil.\pend
           
\pstart
           Alſo erhole Dich recht und genieße die Prag\oindex{Prag@\textbf{Prag}, \emph{Land}|pwv}er Tage!\pend
           
\pstart
           Und \label{K_L02832-4v}\edtext{ſieh’ Dir das liebe Geſicht des
               kleinen Mädchens\pwindex{Ziegler, Alice 5.\,1.\,1880 Prag – Dezember 1943 Konzentrationslager Auschwitz-Birkenau@\textsc{Ziegler, Alice} (5.\,1.\,1880 Prag – Dezember 1943 Konzentrationslager Auschwitz-Birkenau)|pwv} an}{\lemma{\textnormal{\emph{sieh’ … an}}}\Cendnote{\textnormal{Siehe XXXX Auszeichnungsfehler: Dokument L02831 nicht gefunden.
               }}}\label{K_L02832-4} und{ }ſage mir, was darin{ }ſteht.\pend
           
\pstart
           Berichte mir \strikeout{ba} bald und viel!\pend
           
\pstart
           Von Herzen {\\[\baselineskip]}Dein {\\[\baselineskip]}\spacefill\mbox{Paul Goldmnn}\pend
           \leftskip=0em{}
\pstart
           \noindent{}\label{T_L02832-1v}\edtext{{\pb}Ich hoffe, es kommt zur \label{K_L02832-5v}\edtext{Reviſion des Prozeſſes \textsc{Dreyfus\pwindex{Dreyfus, Alfred 9.\,10.\,1859 Mulhouse – 12.\,7.\,1935 Paris@\textsc{Dreyfus, Alfred} (9.\,10.\,1859 Mulhouse – 12.\,7.\,1935 Paris), \emph{Militär}|pw}}}{\lemma{\textnormal{\emph{Revision … Dreyfus}}}\Cendnote{\textnormal{Zu einem weiteren Gerichtsprozess in
                     der Dreyfus\pwindex{Dreyfus, Alfred 9.\,10.\,1859 Mulhouse – 12.\,7.\,1935 Paris@\textsc{Dreyfus, Alfred} (9.\,10.\,1859 Mulhouse – 12.\,7.\,1935 Paris), \emph{Militär}|pwk}-Affäre kam es erst am 10. 1. 1898 und 11. 1. 1898.
                        Ferdinand Walsin-Esterházy\pwindex{Walsin-Esterházy, Ferdinand 16.\,12.\,1847 Paris – 21.\,5.\,1923 Harpenden@\textsc{Walsin-Esterházy, Ferdinand} (16.\,12.\,1847 Paris – 21.\,5.\,1923 Harpenden), \emph{Offizier, Spion}|pwk}, der das
                     Gerichtsverfahren gegen sich selbst beantragt hatte, wurde dort freigesprochen.
                     Eigentlich war aber er – und nicht Alfred
                        Dreyfus\pwindex{Dreyfus, Alfred 9.\,10.\,1859 Mulhouse – 12.\,7.\,1935 Paris@\textsc{Dreyfus, Alfred} (9.\,10.\,1859 Mulhouse – 12.\,7.\,1935 Paris), \emph{Militär}|pwk} – schuldig. Er hatte Maximilian von Schwartzkoppen\pwindex{Schwartzkoppen, Maximilian von 24.\,2.\,1850 Potsdam – 8.\,1.\,1917 Berlin@\textsc{Schwartzkoppen, Maximilian von} (24.\,2.\,1850 Potsdam – 8.\,1.\,1917 Berlin), \emph{Militärattaché, General}|pwk} (gegen Geld) die geheimen militärischen
                     Dokumente gegeben, die die Dreyfus\pwindex{Dreyfus, Alfred 9.\,10.\,1859 Mulhouse – 12.\,7.\,1935 Paris@\textsc{Dreyfus, Alfred} (9.\,10.\,1859 Mulhouse – 12.\,7.\,1935 Paris), \emph{Militär}|pwk}-Affäre
                     auslösten.}}}\label{K_L02832-5}. Der \textsc{Esterhazy\pwindex{Walsin-Esterházy, Ferdinand 16.\,12.\,1847 Paris – 21.\,5.\,1923 Harpenden@\textsc{Walsin-Esterházy, Ferdinand} (16.\,12.\,1847 Paris – 21.\,5.\,1923 Harpenden), \emph{Offizier, Spion}|pw}} iſt wohl{ }ſchuldig. Aber weſſen? Des Verraths? Der Fälſchung? Dunkel,
                     dunkel!}{\lemma{\textnormal{\emph{Ich … dunkel!}}}\Cendnote{\textnormal{kopfüber am oberen Rand der
                     ersten Seite}}}\label{T_L02832-1}\pend
           \selectlanguage{ngerman}\endnumbering\briefempfaengerindex{Schnitzler, Arthur@\textsc{Schnitzler, Arthur}!zzzGoldmann, Paul@\emph{von Paul Goldmann}!1897-11-241@{24. 11. [1897]}|)be}\mylabel{L02832h}  \newcommand{\dateiname}{L02832}\newcommand{\titel}{Paul Goldmann an Arthur Schnitzler, 24. 11. [1897]}\newcommand{\editorInnen}{Martin Anton Müller und Laura Untner}%% latex-leseansicht-abspann.tex
%% Abspann für die Leseansicht.
%% Der Schalter \ifkorrekturansicht ist bereits durch den Vorspann gesetzt.

%% latex-abspann.tex
%% Gemeinsamer Abspann für Korrekturansicht und Leseansicht.
%% Setzt den Schalter \ifkorrekturansicht voraus (gesetzt in den
%% einbindenden Dateien latex-korrekturansicht-abspann.tex bzw.
%% latex-leseansicht-abspann.tex).
%% ---------------------------------------------------------------

\normalsize

% Das esempio-Environment wird nur in der Leseansicht benötigt
\ifkorrekturansicht\else
\newenvironment{esempio}[3]%
{
    \vspace{1.5ex}
    \rlap{\underline{#1}}
    \par
    \setlength{\parindent}{0cm}
    \nopagebreak
    \leftskip=#2cm
    \rightskip=#3cm
}
{
    \par
}
\fi

\doendnotes{C}
\bigskip
\vfill

\clearpage

\footnotesize

\ifkorrekturansicht
  \lohead{\textsc{register}}
\fi

% theindex-Environment neu definieren ohne reledmac
\makeatletter
\renewenvironment{theindex}{%
  \ifkorrekturansicht
    \section*{\indexname}%
  \else
    \subsubsection*{Index der erwähnten Entitäten}%
  \fi
  \setlength{\parindent}{0pt}%
  \setlength{\parskip}{0pt plus 0.3pt}%
  \let\item\@idxitem
}{%
  \ifkorrekturansicht\clearpage\fi
}
\makeatother

\IfFileExists{\jobname-pw.ind}{\input{\jobname-pw.ind}}{}

% Quellenangabe nur in der Leseansicht
\ifkorrekturansicht\else
% Fallback-Definitionen, falls die .tex-Datei \titel etc. nicht gesetzt hat
\providecommand{\titel}{}
\providecommand{\editorInnen}{}
\providecommand{\dateiname}{\jobname}

\vspace{3cm}

\vfill

\footnotesize
\textsc{Quelle}: \titel. Herausgegeben von {\editorInnen}. In: \emph{Arthur Schnitzler: Briefwechsel mit Autorinnen und Autoren}.
 Digitale Edition, https://schnitzler-briefe.acdh.oeaw.ac.at/{\dateiname}.html (Stand \today)
\fi

\end{document}


