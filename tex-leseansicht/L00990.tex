%% latex-leseansicht-vorspann.tex
%% Vorspann für die Leseansicht.
%% Lädt die gemeinsame Datei latex-vorspann.tex mit nicht gesetztem Schalter.

\newif\ifkorrekturansicht
\korrekturansichtfalse

\input{../tex-inputs/latex-vorspann}


         
         \renewcommand{\erwaehntePersonen}{Personen: Hermann Bahr, Otto Brahm, Paul Goldmann, Hugo von Hofmannsthal}
         \renewcommand{\erwaehnteInstitutionen}{Institutionen: Neues Wiener Tagblatt}
         \renewcommand{\erwaehnteOrte}{Orte: Berlin, Deutsches Theater Berlin, Venedig, Wien}
         \renewcommand{\erwaehnteWerke}{Werke: Das Bergwerk zu Falun, Der Schleier der Beatrice. Schauspiel in fünf Akten, Die Entdeckung der Provinz}
               \section[Arthur Schnitzler an Hugo von Hofmannsthal, 8. 10. 1899]{ Arthur Schnitzler an Hugo von Hofmannsthal, 8. 10. 1899}\nopagebreak\mylabel{v}\rehead{ }\begin{ledgroupsized}[t]{13cm}\normalsize\beginnumbering \toendnotes[C]{\smallbreak\pagebreak[2]} \Standort{FDH, Hs-30885,88.}
\physDesc{Brief, 1 Blatt, 4 Seiten
\newline{}Handschrift: schwarze Tinte, deutsche Kurrent}\buchAbdrucke{\weitereDrucke{1) Hugo von Hofmannsthal, Arthur Schnitzler: \emph{Briefwechsel}. Hg. Therese Nickl und Heinrich Schnitzler. Frankfurt am Main: \emph{S. Fischer} 1964, S. 132–133.} \weitereDrucke{2) Hermann Bahr, Arthur Schnitzler: \emph{Briefwechsel, Aufzeichnungen, Dokumente
                                (1891–1931)}. Hg. Kurt Ifkovits und Martin Anton Müller. Göttingen: \emph{Wallstein} 2018, S. 172.} }\toendnotes[C]{\smallbreak}\pstart
           \raggedleft{}{\pb}\textsc{Berlin\oindex{Berlin@\textbf{Berlin}|pw}}, 8. 10. 99.\pend
           \pstart
           mein lieber Hugo, geſtern Abend hab ich die \textsc{Beatrice\pwindex{Schnitzler, Arthur 15.05.1862 – 21.10.1931@\textsc{Schnitzler, Arthur} (15.05.1862 – 21.10.1931), \emph{Schriftsteller, Mediziner}!Schleier der Beatrice. Schauspiel in fuenf Akten1900-12-01@\strich\emph{Der Schleier der Beatrice. Schauspiel in fünf Akten} {[}1900-12-01{]}|pw}} dem Brahm\pwindex{Brahm, Otto 05.02.1856 – 28.11.1912@\textsc{Brahm, Otto} (05.02.1856 – 28.11.1912), \emph{Theaterleiter, Regisseur}|pw} vorgeleſen; mir ſcheint, ſie
                    hat auf ihn gewirkt, eigentlich hatte er keine Einwendungen, und jedenfalls kam
                    ihm die Sache fertiger vor als mir, der ich ſie keinesfalls vorläufig aus der
                    Hand gebe. Ich weiſs ſehr genau was noch daran zu machen iſt; und einiges wird
                    auch gelingen. Die entſchiedenſte {\pb}Einwendg von Brahm\pwindex{Brahm, Otto 05.02.1856 – 28.11.1912@\textsc{Brahm, Otto} (05.02.1856 – 28.11.1912), \emph{Theaterleiter, Regisseur}|pw} war eigentlich der Monolog oder beſſer
                    die Anrede des \textsc{Andrea} – das einzige Stückl, das Sie
                    kennen, – das er ganz hinaus haben möchte. Ich las, mit einer Souper
                    Unterbrechung von 7–12; ſo lang würde die Sache ungeſtrichen mindeſtens
                    ſpielen!\pend
           \pstart
           Ich werde wahrſcheinlich Donnerſtag in Wien\oindex{Wien@\textbf{Wien}|pw}{ }ſein;
                        Paul Goldmann\pwindex{Goldmann, Paul 31.01.1865 – 25.09.1935@\textsc{Goldmann, Paul} (31.01.1865 – 25.09.1935), \emph{Schriftsteller, Journalist}|pw} ko{\geminationm}t auch und wird etwa acht {\pb}Tage bei mir wohnen. Wann ſind Sie wieder in Wien\oindex{Wien@\textbf{Wien}|pw}? Es wäre ſchön, wenn G.\pwindex{Goldmann, Paul 31.01.1865 – 25.09.1935@\textsc{Goldmann, Paul} (31.01.1865 – 25.09.1935), \emph{Schriftsteller, Journalist}|pw}{ }Sie noch zu ſehen bekäme. –\pend
           \pstart
           Über das äußere Leben hier lieber mündlich. –\pend
           \pstart
           Ich weiſs nicht, ob Sie dieſes \label{K_L00990_1v}\edtext{Anfangsfeuilleton\pwindex{Bahr, Hermann 19.07.1863 – 15.01.1934@\textsc{Bahr, Hermann} (19.07.1863 – 15.01.1934), \emph{Schriftsteller, Kritiker}!Entdeckung der Provinz01. 10. 1899@\strich\emph{Die Entdeckung der Provinz} {[}01. 10. 1899{]}|pwv}}{\lemma{\textnormal{\emph{Anfangsfeuilleton}}}\Cendnote{\textnormal{\emph{Die Entdeckung der Provinz}\pwindex{Bahr, Hermann 19.07.1863 – 15.01.1934@\textsc{Bahr, Hermann} (19.07.1863 – 15.01.1934), \emph{Schriftsteller, Kritiker}!Entdeckung der Provinz01. 10. 1899@\strich\emph{Die Entdeckung der Provinz} {[}01. 10. 1899{]}|pwk} ist Bahrs
                        erstes Feuilleton für das \emph{Neue Wiener
                            Tagblatt}\orgindex{Neues Wiener Tagblatt@Neues Wiener Tagblatt|pwk}.}}}\label{K_L00990_1h} von Bahr\pwindex{Bahr, Hermann 19.07.1863 – 15.01.1934@\textsc{Bahr, Hermann} (19.07.1863 – 15.01.1934), \emph{Schriftsteller, Kritiker}|pw}
                    geleſen haben. Ich schicks Ihnen hier. \label{LL439-1v}Er
                        iſt gewiſs nicht nur ein Aff, ſondern auch ein boshafter Aff. –\label{LL439-1h}\pend
           \pstart
           Wie geht’s Ihnen? Fließt die Arbeit {\pb}munter fort? –
                    Daſs Ihnen das Stück\pwindex{Hofmannsthal, Hugo von 1874-02-01 – 1929-07-15@\textsc{Hofmannsthal, Hugo von} (1874-02-01 – 1929-07-15), \emph{Schriftsteller}!Bergwerk zu Falun1900 – 1933@\strich\emph{Das Bergwerk zu Falun} {[}1900 – 1933{]}|pwv}{ }ſich
                    verſagen könnte, iſt ganz unmöglich; es geht in ſo reiner Linie vorwärts, daſs
                    es nur mehr auf die rechte Sti{\geminationm}ung ankommt. Am Ende
                    bringen Sie’s ſchon vollendet nach Wien\oindex{Wien@\textbf{Wien}|pw}? –\pend
           \pstart
           Das Deutſche Theater\oindex{Deutsches Theater Berlin@\textbf{Deutsches Theater Berlin}|pw} braucht ungeheuer notwendig
                    ein oder mehrere Stücke. Br.\pwindex{Brahm, Otto 05.02.1856 – 28.11.1912@\textsc{Brahm, Otto} (05.02.1856 – 28.11.1912), \emph{Theaterleiter, Regisseur}|pw} hat ſo gut wie
                    gar nichts. Meines will ich in jedem Fall zuerſt in Wien\oindex{Wien@\textbf{Wien}|pw}{ }ſpielen laſſen; aber es eilt nicht. Ich habe viel vor und möchte
                    wohler, möchte ganz geſund ſein.\pend
           \pstart Von Herzen Ihr \spacefill\mbox{Arthur}\pend{}
         
         \endnumbering\mylabel{h}\end{ledgroupsized}  \newcommand{\dateiname}{L00990}\newcommand{\titel}{Arthur Schnitzler an Hugo von Hofmannsthal, 8. 10. 1899}\newcommand{\editorInnen}{ Martin Anton Müller und Gerd-Hermann Susen}%% latex-leseansicht-abspann.tex
%% Abspann für die Leseansicht.
%% Der Schalter \ifkorrekturansicht ist bereits durch den Vorspann gesetzt.

%% latex-abspann.tex
%% Gemeinsamer Abspann für Korrekturansicht und Leseansicht.
%% Setzt den Schalter \ifkorrekturansicht voraus (gesetzt in den
%% einbindenden Dateien latex-korrekturansicht-abspann.tex bzw.
%% latex-leseansicht-abspann.tex).
%% ---------------------------------------------------------------

\normalsize

% Das esempio-Environment wird nur in der Leseansicht benötigt
\ifkorrekturansicht\else
\newenvironment{esempio}[3]%
{
    \vspace{1.5ex}
    \rlap{\underline{#1}}
    \par
    \setlength{\parindent}{0cm}
    \nopagebreak
    \leftskip=#2cm
    \rightskip=#3cm
}
{
    \par
}
\fi

\doendnotes{C}
\bigskip
\vfill

\clearpage

\footnotesize

\ifkorrekturansicht
  \lohead{\textsc{register}}
\fi

% theindex-Environment neu definieren ohne reledmac
\makeatletter
\renewenvironment{theindex}{%
  \ifkorrekturansicht
    \section*{\indexname}%
  \else
    \subsubsection*{Index der erwähnten Entitäten}%
  \fi
  \setlength{\parindent}{0pt}%
  \setlength{\parskip}{0pt plus 0.3pt}%
  \let\item\@idxitem
}{%
  \ifkorrekturansicht\clearpage\fi
}
\makeatother

\IfFileExists{\jobname-pw.ind}{\input{\jobname-pw.ind}}{}

% Quellenangabe nur in der Leseansicht
\ifkorrekturansicht\else
% Fallback-Definitionen, falls die .tex-Datei \titel etc. nicht gesetzt hat
\providecommand{\titel}{}
\providecommand{\editorInnen}{}
\providecommand{\dateiname}{\jobname}

\vspace{3cm}

\vfill

\footnotesize
\textsc{Quelle}: \titel. Herausgegeben von {\editorInnen}. In: \emph{Arthur Schnitzler: Briefwechsel mit Autorinnen und Autoren}.
 Digitale Edition, https://schnitzler-briefe.acdh.oeaw.ac.at/{\dateiname}.html (Stand \today)
\fi

\end{document}


      