%% latex-korrekturansicht-vorspann.tex
%% Vorspann für die Korrekturansicht.
%% Lädt die gemeinsame Datei latex-vorspann.tex mit gesetztem Schalter.

\newif\ifkorrekturansicht
\korrekturansichttrue

\input{../tex-inputs/latex-vorspann}


\section[Arthur Schnitzler an Hugo von Hofmannsthal, 8. 10. 1899]{L00990 Arthur Schnitzler an Hugo von Hofmannsthal, 8. 10. 1899}
\nopagebreak\mylabel{L00990v}
\rehead{ }\normalsize\beginnumbering\briefempfaengerindex{Hofmannsthal, Hugo von@\textsc{Hofmannsthal, Hugo von}!zzzSchnitzler, Arthur@\emph{von Arthur Schnitzler}!1899-10-082@{8. 10. 1899}|(be}
\toendnotes[C]{\smallbreak\pagebreak[2]}\Standort{FDH, Hs-30885,88.}
\physDesc{Brief, 1 Blatt, 4 Seiten, 1529 Zeichen
\newline{}Handschrift: schwarze Tinte, deutsche Kurrent}
\buchAbdrucke{\weitereDrucke{1) Hugo von Hofmannsthal, Arthur Schnitzler: \emph{Briefwechsel}. Frankfurt am Main: \emph{S. Fischer} 1964, S. 132–133.} \weitereDrucke{2) Hermann Bahr, Arthur Schnitzler: \emph{Briefwechsel, Aufzeichnungen, Dokumente (1891–1931)}. Göttingen: \emph{Wallstein} 2018, S. 172.} }\toendnotes[C]{\smallbreak}
\pstart
           \raggedleft{}{\pb}\textsc{Berlin\oindex{Berlin@\textbf{Berlin}, \emph{P.PPLC}|pw}}, 8. 10. 99.\pend
           \vspace{0.5em}
\pstart
           mein lieber Hugo, geſtern Abend hab ich die \textsc{Beatrice\pwindex{Schleier der Beatrice. Schauspiel in fuenf Akten@\emph{Der Schleier der Beatrice. Schauspiel in fünf Akten}|pw}} dem Brahm\pwindex{Brahm, Otto 05.02.1856 – 28.11.1912@\textsc{Brahm, Otto} (05.02.1856 – 28.11.1912), \emph{Theaterleiter/Theaterleiterin, Regisseur/Regisseurin}|pw} vorgeleſen; mir ſcheint, ſie
               hat auf ihn gewirkt, eigentlich hatte er keine Einwendungen, und jedenfalls kam ihm
               die Sache fertiger vor als mir, der ich ſie keinesfalls vorläufig aus der Hand gebe.
               Ich weiſs ſehr genau was noch daran zu machen iſt; und einiges wird auch gelingen.
               Die entſchiedenſte {\pb}Einwendg von Brahm\pwindex{Brahm, Otto 05.02.1856 – 28.11.1912@\textsc{Brahm, Otto} (05.02.1856 – 28.11.1912), \emph{Theaterleiter/Theaterleiterin, Regisseur/Regisseurin}|pw} war eigentlich der Monolog oder beſſer die Anrede des \textsc{Andrea} – das einzige Stückl, das Sie kennen, – das er ganz
               hinaus haben möchte. Ich las, mit einer Souper Unterbrechung von 7–12; ſo lang würde
               die Sache ungeſtrichen mindeſtens ſpielen!\pend
           
\pstart
           Ich werde wahrſcheinlich Donnerſtag in Wien\oindex{Wien@\textbf{Wien}, \emph{A.ADM2}|pw}{ }ſein; Paul
                  Goldmann\pwindex{Goldmann, Paul 31.01.1865 – 25.09.1935@\textsc{Goldmann, Paul} (31.01.1865 – 25.09.1935), \emph{Schriftsteller/Schriftstellerin, Journalist/Journalistin}|pw} ko{\geminationm}t auch und wird etwa acht {\pb}Tage bei mir wohnen. Wann ſind Sie wieder in Wien\oindex{Wien@\textbf{Wien}, \emph{A.ADM2}|pw}? Es wäre ſchön, wenn G.\pwindex{Goldmann, Paul 31.01.1865 – 25.09.1935@\textsc{Goldmann, Paul} (31.01.1865 – 25.09.1935), \emph{Schriftsteller/Schriftstellerin, Journalist/Journalistin}|pw}{ }Sie noch zu ſehen bekäme. –\pend
           
\pstart
           Über das äußere Leben hier lieber mündlich. –\pend
           
\pstart
           Ich weiſs nicht, ob Sie dieſes \label{K_L00990-1v}\edtext{Anfangsfeuilleton\pwindex{Entdeckung der Provinz@\emph{Die Entdeckung der Provinz}|pwv}}{\lemma{\textnormal{\emph{Anfangsfeuilleton}}}\Cendnote{\textnormal{\emph{Die Entdeckung der Provinz}\pwindex{Entdeckung der Provinz@\emph{Die Entdeckung der Provinz}|pwk} ist Bahrs erstes
                  Feuilleton für das \emph{Neue Wiener
                  Tagblatt}\orgindex{Neues Wiener Tagblatt@Neues Wiener Tagblatt|pwk}.}}}\label{K_L00990-1} von Bahr\pwindex{Bahr, Hermann 19.07.1863 – 15.01.1934@\textsc{Bahr, Hermann} (19.07.1863 – 15.01.1934), \emph{Schriftsteller/Schriftstellerin, Kritiker/Kritikerin}|pw} geleſen
               haben. Ich schicks Ihnen hier. \label{LL439-1v}Er iſt gewiſs
                  nicht nur ein Aff, ſondern auch ein boshafter Aff. –\label{LL439-1h}\pend
           
\pstart
           Wie geht’s Ihnen? Fließt die Arbeit {\pb}munter fort? – Daſs
               Ihnen das Stück\pwindex{Bergwerk zu Falun@\emph{Das Bergwerk zu Falun}|pwv}{ }ſich verſagen könnte, iſt ganz unmöglich; es geht
               in ſo reiner Linie vorwärts, daſs es nur mehr auf die rechte Sti{\geminationm}ung ankommt. Am Ende bringen Sie’s ſchon vollendet nach
                  Wien\oindex{Wien@\textbf{Wien}, \emph{A.ADM2}|pw}? –\pend
           
\pstart
           Das Deutſche Theater\oindex{Deutsches Theater Berlin@\textbf{Deutsches Theater Berlin}, \emph{Theater (K.THE)}|pw} braucht ungeheuer notwendig
               ein oder mehrere Stücke. Br.\pwindex{Brahm, Otto 05.02.1856 – 28.11.1912@\textsc{Brahm, Otto} (05.02.1856 – 28.11.1912), \emph{Theaterleiter/Theaterleiterin, Regisseur/Regisseurin}|pw} hat ſo gut wie
               gar nichts. Meines will ich in jedem Fall zuerſt in Wien\oindex{Wien@\textbf{Wien}, \emph{A.ADM2}|pw}{ }ſpielen laſſen; aber es eilt nicht. Ich habe viel
               vor und möchte wohler, möchte ganz geſund ſein.\pend
           \pstart Von Herzen Ihr \spacefill\mbox{Arthur}\pend{}\selectlanguage{ngerman}\endnumbering\briefempfaengerindex{Hofmannsthal, Hugo von@\textsc{Hofmannsthal, Hugo von}!zzzSchnitzler, Arthur@\emph{von Arthur Schnitzler}!1899-10-082@{8. 10. 1899}|)be}\mylabel{L00990h}  \normalsize

\doendnotes{C}
\bigskip
\vfill

\clearpage

\footnotesize

\lohead{\textsc{register}}

% Definiere theindex-Environment komplett neu ohne reledmac
\makeatletter
\renewenvironment{theindex}{%
  \section*{\indexname}%
  \setlength{\parindent}{0pt}%
  \setlength{\parskip}{0pt plus 0.3pt}%
  \let\item\@idxitem
}{%
  \clearpage
}
\makeatother

\IfFileExists{\jobname-pw.ind}{\input{\jobname-pw.ind}}{}

\end{document}

      