%% latex-leseansicht-vorspann.tex
%% Vorspann für die Leseansicht.
%% Lädt die gemeinsame Datei latex-vorspann.tex mit nicht gesetztem Schalter.

\newif\ifkorrekturansicht
\korrekturansichtfalse

\input{../tex-inputs/latex-vorspann}


\section[Arthur Schnitzler an Hugo von Hofmannsthal, 8. 10. 1899]{L00990 Arthur Schnitzler an Hugo von Hofmannsthal, 8. 10. 1899}
\nopagebreak\mylabel{L00990v}
\rehead{ }\normalsize\beginnumbering\briefempfaengerindex{Hofmannsthal, Hugo von@\textsc{Hofmannsthal, Hugo von}!zzzSchnitzler, Arthur@\emph{von Arthur Schnitzler}!1899-10-082@{8. 10. 1899}|(be}
\toendnotes[C]{\smallbreak\pagebreak[2]}
\correspDesc{Versand  durch Arthur Schnitzler am 8. 10. 1899 in Berlin
\newline{}Erhalt  durch Hugo von Hofmannsthal im Zeitraum [9. 10. 1899
                  – 13. 10. 1899?] in Venedig}\toendnotes[C]{\smallbreak}
\Standort{FDH, Hs-30885,88.}
\physDesc{Brief, 1 Blatt, 4 Seiten, 1529 Zeichen
\newline{}Handschrift: schwarze Tinte, deutsche Kurrent}
\buchAbdrucke{\weitereDrucke{1) Hugo von Hofmannsthal, Arthur Schnitzler: \emph{Briefwechsel}. Herausgegeben von Therese Nickl und Heinrich Schnitzler. Frankfurt am Main: \emph{S. Fischer} 1964, S. 132–133.} \weitereDrucke{2) Hermann Bahr, Arthur Schnitzler: \emph{Briefwechsel, Aufzeichnungen, Dokumente (1891–1931)}. Herausgegeben von Kurt Ifkovits und Martin Anton Müller. Göttingen: \emph{Wallstein} 2018, S. 172.} }\toendnotes[C]{\smallbreak}
\pstart
           \raggedleft{}{\pb}\textsc{Berlin\oindex{Berlin@\textbf{Berlin}, \emph{Hauptstadt}|pw}}, 8. 10. 99.\pend
           \vspace{0.5em}
\pstart
           mein lieber Hugo, geſtern Abend hab ich die \textsc{Beatrice\pwindex{Schnitzler, Arthur 15.\,5.\,1862 Wien – 21.\,10.\,1931 ebd.@\textsc{Schnitzler, Arthur} (15.\,5.\,1862 Wien – 21.\,10.\,1931 ebd.), \emph{Schriftsteller, Mediziner}!Schleier der Beatrice. Schauspiel in fünf Akten@\strich\emph{Der Schleier der Beatrice. Schauspiel in fünf Akten}|pw}} dem Brahm\pwindex{Brahm, Otto 5.\,2.\,1856 Hamburg – 28.\,11.\,1912 Berlin@\textsc{Brahm, Otto} (5.\,2.\,1856 Hamburg – 28.\,11.\,1912 Berlin), \emph{Theaterleiter, Regisseur}|pw} vorgeleſen; mir{ }ſcheint,{ }ſie
               hat auf ihn gewirkt, eigentlich hatte er keine Einwendungen, und jedenfalls kam ihm
               die Sache fertiger vor als mir, der ich{ }ſie keinesfalls vorläufig aus der Hand gebe.
               Ich weiſs{ }ſehr genau was noch daran zu machen iſt; und einiges wird auch gelingen.
               Die entſchiedenſte {\pb}Einwendg von Brahm\pwindex{Brahm, Otto 5.\,2.\,1856 Hamburg – 28.\,11.\,1912 Berlin@\textsc{Brahm, Otto} (5.\,2.\,1856 Hamburg – 28.\,11.\,1912 Berlin), \emph{Theaterleiter, Regisseur}|pw} war eigentlich der Monolog oder beſſer die Anrede des \textsc{Andrea} – das einzige Stückl, das Sie kennen, – das er ganz
               hinaus haben möchte. Ich las, mit einer Souper Unterbrechung von 7–12;{ }ſo lang würde
               die Sache ungeſtrichen mindeſtens{ }ſpielen!\pend
           
\pstart
           Ich werde wahrſcheinlich Donnerſtag in Wien\oindex{Wien@\textbf{Wien}, \emph{Verwaltungsgebiet}|pw}{ }ſein; Paul
                  Goldmann\pwindex{Goldmann, Paul 31.\,1.\,1865 Breslau – 25.\,9.\,1935 Wien@\textsc{Goldmann, Paul} (31.\,1.\,1865 Breslau – 25.\,9.\,1935 Wien), \emph{Schriftsteller, Journalist}|pw} ko{\geminationm}t auch und wird etwa acht {\pb}Tage bei mir wohnen. Wann{ }ſind Sie wieder in Wien\oindex{Wien@\textbf{Wien}, \emph{Verwaltungsgebiet}|pw}? Es wäre{ }ſchön, wenn G.\pwindex{Goldmann, Paul 31.\,1.\,1865 Breslau – 25.\,9.\,1935 Wien@\textsc{Goldmann, Paul} (31.\,1.\,1865 Breslau – 25.\,9.\,1935 Wien), \emph{Schriftsteller, Journalist}|pw}{ }Sie noch zu{ }ſehen bekäme. –\pend
           
\pstart
           Über das äußere Leben hier lieber mündlich. –\pend
           
\pstart
           Ich weiſs nicht, ob Sie dieſes \label{K_L00990-1v}\edtext{Anfangsfeuilleton\pwindex{Bahr, Hermann 19.\,7.\,1863 Linz – 15.\,1.\,1934 München@\textsc{Bahr, Hermann} (19.\,7.\,1863 Linz – 15.\,1.\,1934 München), \emph{Schriftsteller, Kritiker}!Entdeckung der Provinz@\strich\emph{Die Entdeckung der Provinz}|pwv}}{\lemma{\textnormal{\emph{Anfangsfeuilleton}}}\Cendnote{\textnormal{\emph{Die Entdeckung der Provinz}\pwindex{Bahr, Hermann 19.\,7.\,1863 Linz – 15.\,1.\,1934 München@\textsc{Bahr, Hermann} (19.\,7.\,1863 Linz – 15.\,1.\,1934 München), \emph{Schriftsteller, Kritiker}!Entdeckung der Provinz@\strich\emph{Die Entdeckung der Provinz}|pwk} ist Bahrs erstes
                  Feuilleton für das \emph{Neue Wiener
                  Tagblatt}\orgindex{Neues Wiener Tagblatt@Neues Wiener Tagblatt|pwk}.}}}\label{K_L00990-1} von Bahr\pwindex{Bahr, Hermann 19.\,7.\,1863 Linz – 15.\,1.\,1934 München@\textsc{Bahr, Hermann} (19.\,7.\,1863 Linz – 15.\,1.\,1934 München), \emph{Schriftsteller, Kritiker}|pw} geleſen
               haben. Ich schicks Ihnen hier. \label{LL439-1v}Er iſt gewiſs
                  nicht nur ein Aff,{ }ſondern auch ein boshafter Aff. –\label{LL439-1h}\pend
           
\pstart
           Wie geht’s Ihnen? Fließt die Arbeit {\pb}munter fort? – Daſs
               Ihnen das Stück\pwindex{Hofmannsthal, Hugo von 1.\,2.\,1874 Wien – 15.\,7.\,1929 Rodaun@\textsc{Hofmannsthal, Hugo von} (1.\,2.\,1874 Wien – 15.\,7.\,1929 Rodaun), \emph{Schriftsteller}!Bergwerk zu Falun@\strich\emph{Das Bergwerk zu Falun}|pwv}{ }ſich verſagen könnte, iſt ganz unmöglich; es geht
               in{ }ſo reiner Linie vorwärts, daſs es nur mehr auf die rechte Sti{\geminationm}ung ankommt. Am Ende bringen Sie’s{ }ſchon vollendet nach
                  Wien\oindex{Wien@\textbf{Wien}, \emph{Verwaltungsgebiet}|pw}? –\pend
           
\pstart
           Das Deutſche Theater\oindex{Deutsches Theater Berlin@\textbf{Deutsches Theater Berlin}, \emph{Theater}|pw} braucht ungeheuer notwendig
               ein oder mehrere Stücke. Br.\pwindex{Brahm, Otto 5.\,2.\,1856 Hamburg – 28.\,11.\,1912 Berlin@\textsc{Brahm, Otto} (5.\,2.\,1856 Hamburg – 28.\,11.\,1912 Berlin), \emph{Theaterleiter, Regisseur}|pw} hat{ }ſo gut wie
               gar nichts. Meines will ich in jedem Fall zuerſt in Wien\oindex{Wien@\textbf{Wien}, \emph{Verwaltungsgebiet}|pw}{ }ſpielen laſſen; aber es eilt nicht. Ich habe viel
               vor und möchte wohler, möchte ganz geſund{ }ſein.\pend
           \pstart Von Herzen Ihr \spacefill\mbox{Arthur}\pend{}\selectlanguage{ngerman}\endnumbering\briefempfaengerindex{Hofmannsthal, Hugo von@\textsc{Hofmannsthal, Hugo von}!zzzSchnitzler, Arthur@\emph{von Arthur Schnitzler}!1899-10-082@{8. 10. 1899}|)be}\mylabel{L00990h}  \newcommand{\dateiname}{L00990}\newcommand{\titel}{Arthur Schnitzler an Hugo von Hofmannsthal, 8. 10. 1899}\newcommand{\editorInnen}{Herausgegeben von Martin Anton Müller}%% latex-leseansicht-abspann.tex
%% Abspann für die Leseansicht.
%% Der Schalter \ifkorrekturansicht ist bereits durch den Vorspann gesetzt.

%% latex-abspann.tex
%% Gemeinsamer Abspann für Korrekturansicht und Leseansicht.
%% Setzt den Schalter \ifkorrekturansicht voraus (gesetzt in den
%% einbindenden Dateien latex-korrekturansicht-abspann.tex bzw.
%% latex-leseansicht-abspann.tex).
%% ---------------------------------------------------------------

\normalsize

% Das esempio-Environment wird nur in der Leseansicht benötigt
\ifkorrekturansicht\else
\newenvironment{esempio}[3]%
{
    \vspace{1.5ex}
    \rlap{\underline{#1}}
    \par
    \setlength{\parindent}{0cm}
    \nopagebreak
    \leftskip=#2cm
    \rightskip=#3cm
}
{
    \par
}
\fi

\doendnotes{C}
\bigskip
\vfill

\clearpage

\footnotesize

\ifkorrekturansicht
  \lohead{\textsc{register}}
\fi

% theindex-Environment neu definieren ohne reledmac
\makeatletter
\renewenvironment{theindex}{%
  \ifkorrekturansicht
    \section*{\indexname}%
  \else
    \subsubsection*{Index der erwähnten Entitäten}%
  \fi
  \setlength{\parindent}{0pt}%
  \setlength{\parskip}{0pt plus 0.3pt}%
  \let\item\@idxitem
}{%
  \ifkorrekturansicht\clearpage\fi
}
\makeatother

\IfFileExists{\jobname-pw.ind}{\input{\jobname-pw.ind}}{}

% Quellenangabe nur in der Leseansicht
\ifkorrekturansicht\else
% Fallback-Definitionen, falls die .tex-Datei \titel etc. nicht gesetzt hat
\providecommand{\titel}{}
\providecommand{\editorInnen}{}
\providecommand{\dateiname}{\jobname}

\vspace{3cm}

\vfill

\footnotesize
\textsc{Quelle}: \titel. Herausgegeben von {\editorInnen}. In: \emph{Arthur Schnitzler: Briefwechsel mit Autorinnen und Autoren}.
 Digitale Edition, https://schnitzler-briefe.acdh.oeaw.ac.at/{\dateiname}.html (Stand \today)
\fi

\end{document}


