%% latex-korrekturansicht-vorspann.tex
%% Vorspann für die Korrekturansicht.
%% Lädt die gemeinsame Datei latex-vorspann.tex mit gesetztem Schalter.

\newif\ifkorrekturansicht
\korrekturansichttrue

\input{../tex-inputs/latex-vorspann}


\section[Felix Salten: Widmungsexemplar Der Schrei der Liebe für Arthur Schnitzler, Juli 1928]{L03044 Felix Salten: Widmungsexemplar Der Schrei der Liebe für Arthur
               Schnitzler, Juli 1928}
\nopagebreak\mylabel{L03044v}
\rehead{ }\normalsize\beginnumbering\briefempfaengerindex{Schnitzler, Arthur@\textsc{Schnitzler, Arthur}!zzzSalten, Felix@\emph{von Felix Salten}!1928-07-311@{Juli 1928}|(be}
\toendnotes[C]{\smallbreak\pagebreak[2]}\Standort{DLA, G:Schnitzler, Arthur (Sammlung Heinrich Schnitzler).}
\physDesc{Widmung am Schmutztitel, 51 Zeichen
\newline{}Handschrift: schwarze Tinte, lateinische Kurrent}
\pstart
           \noindent{}\centering{}{\pb}\textcolor{gray}{\textbf{\textsc{\so{FELIX SALTEN}}}}\pend
           
\pstart
           \centering{}\textcolor{gray}{\textbf{Geſammelte Werke}}\pend
           
\pstart
           \centering{}\textcolor{gray}{\textbf{in Einzelausgaben}}\pend
           
\pstart
           Arthur Schnitzler {\\}herzlich {\\}\spacefill\mbox{Felix Salten}\pend
           
\pstart
           Wien\oindex{Wien@\textbf{Wien}, \emph{A.ADM2}|pw}, Juli 28\pend
           {\vspace{1\baselineskip}}
\pstart
           \centering{}{\pb}\textcolor{gray}{\textbf{\textsc{\so{FELIX SALTEN}}}}\pend
           
\pstart
           \centering{}\textcolor{gray}{\textbf{Der Schrei der Liebe\pwindex{Schrei der Liebe. Novellen@\emph{Der Schrei der Liebe. Novellen}|pw}}}\pend
           
\pstart
           \centering{}\textcolor{gray}{\textbf{\textsc{\so{NOVELLEN}}}}\pend
           {\vspace{1\baselineskip}}
\pstart
           \centering{}\textcolor{gray}{\textbf{\textsc{1928}}}\pend
           
\pstart
           \centering{}\textcolor{gray}{\textbf{\textsc{\so{PAUL ZSOLNAY VERLAG}\orgindex{Paul Zsolnay Verlag@Paul Zsolnay Verlag|pw}}}}\pend
           
\pstart
           \centering{}\textcolor{gray}{\textbf{\textsc{BERLIN\oindex{Berlin@\textbf{Berlin}, \emph{P.PPLC}|pw} / WIEN\oindex{Wien@\textbf{Wien}, \emph{A.ADM2}|pw} / LEIPZIG\oindex{Leipzig@\textbf{Leipzig}, \emph{P.PPLA3}|pw}}}}\pend
           \selectlanguage{ngerman}\endnumbering\briefempfaengerindex{Schnitzler, Arthur@\textsc{Schnitzler, Arthur}!zzzSalten, Felix@\emph{von Felix Salten}!1928-07-011@{Juli 1928}|)be}\mylabel{L03044h}  \normalsize

\doendnotes{C}
\bigskip
\vfill

\clearpage

\footnotesize

\lohead{\textsc{register}}

% Definiere theindex-Environment komplett neu ohne reledmac
\makeatletter
\renewenvironment{theindex}{%
  \section*{\indexname}%
  \setlength{\parindent}{0pt}%
  \setlength{\parskip}{0pt plus 0.3pt}%
  \let\item\@idxitem
}{%
  \clearpage
}
\makeatother

\IfFileExists{\jobname-pw.ind}{\input{\jobname-pw.ind}}{}

\end{document}

      