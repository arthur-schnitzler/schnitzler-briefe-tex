%% latex-korrekturansicht-vorspann.tex
%% Vorspann für die Korrekturansicht.
%% Lädt die gemeinsame Datei latex-vorspann.tex mit gesetztem Schalter.

\newif\ifkorrekturansicht
\korrekturansichttrue

\input{../tex-inputs/latex-vorspann}


\section[Hugo von Hofmannsthal an Arthur Schnitzler, 16. 6. 1894]{L00339 Hugo von Hofmannsthal an Arthur Schnitzler, 16. 6. 1894}
\nopagebreak\mylabel{L00339v}
\rehead{ }\normalsize\beginnumbering\briefempfaengerindex{Schnitzler, Arthur@\textsc{Schnitzler, Arthur}!zzzHofmannsthal, Hugo von@\emph{von Hugo von Hofmannsthal}!1894-06-161@{16. 6. 1894}|(be}
\toendnotes[C]{\smallbreak\pagebreak[2]}\Standort{CUL, Schnitzler, B 43b/1.}
\physDesc{Kartenbrief, 284 Zeichen
\newline{}Handschrift: 1) Bleistift, deutsche Kurrent\hspace{1em}2) Bleistift, lateinische Kurrent (\noindent{}Adresse)\hspace{1em}
\newline{}Versand: 1) Stempel: »\nobreak{}\oindex{III., Landstrasse@\textbf{III., Landstraße}, \emph{A.ADM3}|pwk}Wien 3/3, 16. 6. 94, 5–6 N\nobreak{}«.   2) Stempel: »\nobreak{}Bestellt, \oindex{IX., Alsergrund@\textbf{IX., Alsergrund}, \emph{A.ADM3}|pwk}Wien 9/3, 17. 6. 94, 8. V\nobreak{}«. 
\newline{}Schnitzler: mit Bleistift das Datum ergänzt: »16/6 94« 
\newline{}Ordnung: mit Bleistift von unbekannter Hand nummeriert:
                                    »66« }
\buchAbdrucke{\weitereDrucke{1) Hugo von Hofmannsthal, Arthur Schnitzler: \emph{Briefwechsel}. Frankfurt am Main: \emph{S. Fischer} 1964, S. 52.} \weitereDrucke{2) Hermann Bahr, Arthur Schnitzler: \emph{Briefwechsel, Aufzeichnungen, Dokumente (1891–1931)}. Göttingen: \emph{Wallstein} 2018, S. 73.} }\toendnotes[C]{\smallbreak}\pstart{}{\pb}Herrn D\textsuperscript{r} Arthur Schnitzler\pend{}\pstart{}IX\oindex{IX., Alsergrund@\textbf{IX., Alsergrund}, \emph{A.ADM3}|pw}\pend{}\pstart{}Frankgasse 1\oindex{Frankgasse 1@\textbf{Frankgasse 1}, \emph{Wohngebäude (K.WHS)}|pw}\pend{}{\bigskip}\vspace{1em}
\pstart
           \noindent{}{\pb}lieber, ich werde dem
                  Bahr\pwindex{Bahr, Hermann 19.07.1863 – 15.01.1934@\textsc{Bahr, Hermann} (19.07.1863 – 15.01.1934), \emph{Schriftsteller/Schriftstellerin, Kritiker/Kritikerin}|pw}{ }\label{K_L00339-1v}\edtext{das Mitgehen}{\lemma{\textnormal{\emph{das Mitgehen}}}\Cendnote{\textnormal{Sie wollten nach Mödling\oindex{Moedling@\textbf{Mödling}, \emph{P.PPLA3}|pwk}, um Christine Schönberger\pwindex{Schoenberger, Christine 1875-11-17 – 1971-02-03@\textsc{Schönberger, Christine} (1875-11-17 – 1971-02-03), \emph{Gastwirt/Gastwirtin}|pwk}, die Wirtstochter des
                  Goldenen Sterns\oindex{Zum goldenen Stern@\textbf{Zum goldenen Stern}, \emph{Lokal (K.LKL)}|pwk} zu besuchen. Diese dürfte in \emph{Liebelei}\pwindex{Liebelei. Schauspiel in drei Akten@\emph{Liebelei. Schauspiel in drei Akten}|pwk} porträtiert sein, vgl. Hermann Bahr, Arthur Schnitzler: \emph{Briefwechsel, Aufzeichnungen, Dokumente (1891–1931)}, Hermann Bahr an Gerty Schlesinger, 30. 6. 1898 und Valerie Reichert-Heidt: \emph{Das Urbild der
                        Christine}. In: \emph{Neues Österreich}, Jg. 11,
                     Nr. 3208, 13. 11. 1955, S. 17–18.}}}\label{K_L00339-1}
               ausreden.\pend
           
\pstart
           Wenn es \uline{unzweifelhaft} hübſch iſt, weder drohend noch
               regneriſch, erwart ich Sie um Punkt \label{K_L00339-2v}\edtext{¼
                  4}{\lemma{\textnormal{\emph{¼
                  4}}}\Cendnote{\textnormal{15 Uhr 45}}}\label{K_L00339-2} unter den Arkaden der Oper\oindex{Oper@\textbf{Oper}, \emph{Oper (K.OPR)}|pw}, wo die Guttmann’ſche \label{K_L00339-3v}\edtext{Kalienhandlung}{\lemma{\textnormal{\emph{Kalienhandlung}}}\Cendnote{\textnormal{gemeint: Musikalienhandlung}}}\label{K_L00339-3}\oindex{Musikalienhandlung Albert J. Gutmann@\textbf{Musikalienhandlung Albert J. Gutmann}, \emph{Geschäft (K.GES)}|pw} iſt. Recht? Dadurch erſparen wir ½ Stunde.\pend
           
\pstart
           Ihr{\\[\baselineskip]}\spacefill\mbox{Hugo.}\pend
           \leftskip=0em{}\selectlanguage{ngerman}\endnumbering\briefempfaengerindex{Schnitzler, Arthur@\textsc{Schnitzler, Arthur}!zzzHofmannsthal, Hugo von@\emph{von Hugo von Hofmannsthal}!1894-06-161@{16. 6. 1894}|)be}\mylabel{L00339h}  \normalsize

\doendnotes{C}
\bigskip
\vfill

\clearpage

\footnotesize

\lohead{\textsc{register}}

% Definiere theindex-Environment komplett neu ohne reledmac
\makeatletter
\renewenvironment{theindex}{%
  \section*{\indexname}%
  \setlength{\parindent}{0pt}%
  \setlength{\parskip}{0pt plus 0.3pt}%
  \let\item\@idxitem
}{%
  \clearpage
}
\makeatother

\IfFileExists{\jobname-pw.ind}{\input{\jobname-pw.ind}}{}

\end{document}

      