%% latex-leseansicht-vorspann.tex
%% Vorspann für die Leseansicht.
%% Lädt die gemeinsame Datei latex-vorspann.tex mit nicht gesetztem Schalter.

\newif\ifkorrekturansicht
\korrekturansichtfalse

\input{../tex-inputs/latex-vorspann}


\section[Hugo von Hofmannsthal an Arthur Schnitzler, 27. 7. 1900]{L01061 Hugo von Hofmannsthal an Arthur Schnitzler, 27. 7. 1900}
\nopagebreak\mylabel{L01061v}
\rehead{ }\normalsize\beginnumbering\briefempfaengerindex{Schnitzler, Arthur@\textsc{Schnitzler, Arthur}!zzzHofmannsthal, Hugo von@\emph{von Hugo von Hofmannsthal}!1900-07-272@{27. 7. 1900}|(be}
\toendnotes[C]{\smallbreak\pagebreak[2]}
\correspDesc{Versand  durch Hugo von Hofmannsthal am 27. 7. 1900 in Bad Fusch
\newline{}Erhalt  durch Arthur Schnitzler im Zeitraum [28. 7. 1900
                  – 1. 8. 1900?] in Wien}\toendnotes[C]{\smallbreak}
\Standort{CUL, Schnitzler, B 43.}
\physDesc{Brief, 2 Blätter, 7 Seiten, 2314 Zeichen
\newline{}Handschrift: schwarze Tinte, deutsche Kurrent
\newline{}Ordnung: mit Bleistift von unbekannter Hand nummeriert:
                                    »164.1« beziehungsweise
                                 »164.2« }
\buchAbdrucke{\weitereDrucke{Hugo von Hofmannsthal, Arthur Schnitzler: \emph{Briefwechsel}. Herausgegeben von Therese Nickl und Heinrich Schnitzler. Frankfurt am Main: \emph{S. Fischer} 1964, S. 143.} }\toendnotes[C]{\smallbreak}
\pstart
           \raggedleft{}{\pb}Fuſch\oindex{Bad Fusch@\textbf{Bad Fusch}|pw}{ }27 VII.\pend
           
\pstart{}mein lieber Arthur\pend\vspace{0.5em}
\pstart
           es iſt{ }ſehr angenehm, durch die kleine Dora\pwindex{Michaelis, Dora 23.\,5.\,1881 Wien – 22.\,1.\,1946 New York City@\textsc{Michaelis, Dora} (23.\,5.\,1881 Wien – 22.\,1.\,1946 New York City)|pw},
               welche wirklich ein überaus nettes und angenehmes Geſchöpf iſt, von Zeit zu Zeit ein
               Wort über Sie zu hören.\pend
           
\pstart
           Die Tage in Salzburg\oindex{Salzburg@\textbf{Salzburg}, \emph{Verwaltungsgebiet}|pw} mit Richard\pwindex{Beer-Hofmann, Richard 11.\,7.\,1866 Wien – 26.\,9.\,1945 New York City@\textsc{Beer-Hofmann, Richard} (11.\,7.\,1866 Wien – 26.\,9.\,1945 New York City), \emph{Schriftsteller}|pw} waren mir doppelt wohlthuend, da ich gerade im Verkehr
               mit ihm immer das Gefühl zu{ }ſeltenen Zusammenseins, ungeſtillten {\pb}Hungers habe.\hspace*{1.5em}Gerade an dem Tag, wo Ihr Eure \label{K_L01061-1v}\edtext{Fußreiſe}{\lemma{\textnormal{\emph{Fußreise}}}\Cendnote{\textnormal{Vgl. XXXX Auszeichnungsfehler: Dokument L02920 nicht gefunden.
               }}}\label{K_L01061-1} antretet, dürfte ich zur Waffenübung einrücken. Nachher werd ich, um die
                  Mitte September, wahrſcheinlich an den \textsc{Gardasee}\oindex{Lago di Garda@\textbf{Lago di Garda}, \emph{See}|pw} gehen.\pend
           
\pstart
           Nun aber, die nächſten Tage, etwa vom letzten July an, bin ich in Salzburg\oindex{Salzburg@\textbf{Salzburg}, \emph{Verwaltungsgebiet}|pw}, im oeſterreich. Hof\oindex{Österreichischer Hof@\textbf{Österreichischer Hof}, \emph{Hotel}|pw}. Auch meine Eltern\pwindex{Hofmannsthal, Anna von 27.\,1.\,1849 Wien – 22.\,3.\,1904 Sanatorium Fürth@\textsc{Hofmannsthal, Anna von} (27.\,1.\,1849 Wien – 22.\,3.\,1904 Sanatorium Fürth)|pwv}\pwindex{Hofmannsthal, Hugo August von 21.\,12.\,1841 Wien – 8.\,12.\,1915 ebd.@\textsc{Hofmannsthal, Hugo August von} (21.\,12.\,1841 Wien – 8.\,12.\,1915 ebd.), \emph{Bankdirektor}|pwv} werden zur{ }ſelben Zeit dort{ }ſein, und einen Theil
               der Zeit auch die {\pb}Gerty\pwindex{Hofmannsthal, Gertrude von 16.\,3.\,1880 Wien – 9.\,11.\,1959 Paddington@\textsc{Hofmannsthal, Gertrude von} (16.\,3.\,1880 Wien – 9.\,11.\,1959 Paddington)|pw} mit ihrer Mutter\pwindex{Schlesinger, Franziska 17.\,8.\,1851 Wien – 11.\,8.\,1932 ebd.@\textsc{Schlesinger, Franziska} (17.\,8.\,1851 Wien – 11.\,8.\,1932 ebd.)|pwv}.\pend
           
\pstart
           Hier{ }ſcheint mir, indem ich{ }ſchreibe, in dem Nicht-erwähnen einer beſtehenden
               Situation zum erſten Mal eine wirkliche Unwahrheit zu liegen, und{ }ſo will ich denn,
               wie vor einigen Tagen dem Richard\pwindex{Beer-Hofmann, Richard 11.\,7.\,1866 Wien – 26.\,9.\,1945 New York City@\textsc{Beer-Hofmann, Richard} (11.\,7.\,1866 Wien – 26.\,9.\,1945 New York City), \emph{Schriftsteller}|pw}, auch Ihnen
               gern{ }ſagen, daß ich die Gerty\pwindex{Hofmannsthal, Gertrude von 16.\,3.\,1880 Wien – 9.\,11.\,1959 Paddington@\textsc{Hofmannsthal, Gertrude von} (16.\,3.\,1880 Wien – 9.\,11.\,1959 Paddington)|pw} im Lauf des
               nächſten Frühjahrs heirathen werde. Ich bitte Sie, davon zu niemandem als etwa {\pb}zu Richard\pwindex{Beer-Hofmann, Richard 11.\,7.\,1866 Wien – 26.\,9.\,1945 New York City@\textsc{Beer-Hofmann, Richard} (11.\,7.\,1866 Wien – 26.\,9.\,1945 New York City), \emph{Schriftsteller}|pw} zu{ }ſprechen. Freilich weiß ich daſs ein{ }ſolches Gerücht
               und die Überzeugung maſſenhafter Menſchen von dieſer Sache{ }ſeit langem, ja mir{ }ſcheint{ }ſchon{ }ſeit mehreren Jahren beſteht. Aber das war, bevor in den beiden, um die
               es{ }ſich handelt, irgend ein Gedanke, ja{ }ſogar bevor der Wunsch nach einer{ }ſolchen
               Verwirklichung beſtanden hatte. Und{ }ſo hatte das Gerede damals, und hat auch jetzt
                  {\pb}mit der Sache{ }ſelbſt
               eigentlich nichts zu thun, und{ }ſoll auch davon getrennt bleiben. Denn wenn man auch
               dazu geführt wird, etwas zu thun, was die Leute vorausgeſagt haben,{ }ſo iſt es doch,
               indem man’s thut durch ganze Abgründe von dem, was die andern in ihren Köpfen haben
               getrennt. – Ich bin alſo bis {\pb}halben Auguſt in Salzburg\oindex{Salzburg@\textbf{Salzburg}, \emph{Verwaltungsgebiet}|pw}. Ich
               hoffe beſtimmt, daſs wir uns da{ }ſehen. Sie können mich natürlich allein haben,{ }ſoviel
               wir uns das verlangen. Was{ }ſollte{ }ſich darin ändern oder künftig ändern müſſen? Und
               übrigens ergibt ja das Rad eine nette Form des Zuſa{\geminationm}enſeins.\pend
           
\pstart
           Gearbeitet hab ich recht {\pb}wenig,
               will{ }ſolche Zeiten aber von je\introOben{}t\introOben{}zt an ohne dieſe übermäßige
               Ungeduld ertragen. Auf Ihre phantaſtiſche Novelle\pwindex{Schnitzler, Arthur 15.\,5.\,1862 Wien – 21.\,10.\,1931 ebd.@\textsc{Schnitzler, Arthur} (15.\,5.\,1862 Wien – 21.\,10.\,1931 ebd.), \emph{Schriftsteller, Mediziner}!Lieutenant Gustl. Novelle@\strich\emph{Lieutenant Gustl. Novelle}|pwv} freu ich mich{ }ſehr. Wenn ich{ }ſie bald hören könnte,
               oder die lange\pwindex{Schnitzler, Arthur 15.\,5.\,1862 Wien – 21.\,10.\,1931 ebd.@\textsc{Schnitzler, Arthur} (15.\,5.\,1862 Wien – 21.\,10.\,1931 ebd.), \emph{Schriftsteller, Mediziner}!Frau Bertha Garlan. Roman@\strich\emph{Frau Bertha Garlan. Roman}|pwv}? Das iſt immer
               eine Freude, der nachher das Leſen nicht mehr gleichkommt.\pend
           \pstart Alſo hoffentlich{ }ſehn wir uns bald. Von Herzen Ihr \spacefill\mbox{Hugo.}\pend{}\selectlanguage{ngerman}\endnumbering\briefempfaengerindex{Schnitzler, Arthur@\textsc{Schnitzler, Arthur}!zzzHofmannsthal, Hugo von@\emph{von Hugo von Hofmannsthal}!1900-07-272@{27. 7. 1900}|)be}\mylabel{L01061h}  \newcommand{\dateiname}{L01061}\newcommand{\titel}{Hugo von Hofmannsthal an Arthur Schnitzler, 27. 7. 1900}\newcommand{\editorInnen}{Martin Anton Müller und Gerd-Hermann Susen}%% latex-leseansicht-abspann.tex
%% Abspann für die Leseansicht.
%% Der Schalter \ifkorrekturansicht ist bereits durch den Vorspann gesetzt.

%% latex-abspann.tex
%% Gemeinsamer Abspann für Korrekturansicht und Leseansicht.
%% Setzt den Schalter \ifkorrekturansicht voraus (gesetzt in den
%% einbindenden Dateien latex-korrekturansicht-abspann.tex bzw.
%% latex-leseansicht-abspann.tex).
%% ---------------------------------------------------------------

\normalsize

% Das esempio-Environment wird nur in der Leseansicht benötigt
\ifkorrekturansicht\else
\newenvironment{esempio}[3]%
{
    \vspace{1.5ex}
    \rlap{\underline{#1}}
    \par
    \setlength{\parindent}{0cm}
    \nopagebreak
    \leftskip=#2cm
    \rightskip=#3cm
}
{
    \par
}
\fi

\doendnotes{C}
\bigskip
\vfill

\clearpage

\footnotesize

\ifkorrekturansicht
  \lohead{\textsc{register}}
\fi

% theindex-Environment neu definieren ohne reledmac
\makeatletter
\renewenvironment{theindex}{%
  \ifkorrekturansicht
    \section*{\indexname}%
  \else
    \subsubsection*{Index der erwähnten Entitäten}%
  \fi
  \setlength{\parindent}{0pt}%
  \setlength{\parskip}{0pt plus 0.3pt}%
  \let\item\@idxitem
}{%
  \ifkorrekturansicht\clearpage\fi
}
\makeatother

\IfFileExists{\jobname-pw.ind}{\input{\jobname-pw.ind}}{}

% Quellenangabe nur in der Leseansicht
\ifkorrekturansicht\else
% Fallback-Definitionen, falls die .tex-Datei \titel etc. nicht gesetzt hat
\providecommand{\titel}{}
\providecommand{\editorInnen}{}
\providecommand{\dateiname}{\jobname}

\vspace{3cm}

\vfill

\footnotesize
\textsc{Quelle}: \titel. Herausgegeben von {\editorInnen}. In: \emph{Arthur Schnitzler: Briefwechsel mit Autorinnen und Autoren}.
 Digitale Edition, https://schnitzler-briefe.acdh.oeaw.ac.at/{\dateiname}.html (Stand \today)
\fi

\end{document}


