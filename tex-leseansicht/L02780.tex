%% latex-korrekturansicht-vorspann.tex
%% Vorspann für die Korrekturansicht.
%% Lädt die gemeinsame Datei latex-vorspann.tex mit gesetztem Schalter.

\newif\ifkorrekturansicht
\korrekturansichttrue

\input{../tex-inputs/latex-vorspann}


\section[ Paul Goldmann an Arthur Schnitzler, 4. 7. {[}1896{]}]{L02780 Paul Goldmann an Arthur Schnitzler, 4. 7. {[}1896{]}}
\nopagebreak\mylabel{L02780v}
\rehead{ }\normalsize\beginnumbering\briefempfaengerindex{Schnitzler, Arthur@\textsc{Schnitzler, Arthur}!zzzGoldmann, Paul@\emph{von Paul Goldmann}!1896-07-041@{4. 7. {[}1896{]}}|(be}
\toendnotes[C]{\smallbreak\pagebreak[2]}\Standort{DLA, A:Schnitzler, HS.NZ85.1.3166.}
\physDesc{Brief, 1 Blatt, 3 Seiten, 893 Zeichen
\newline{}Handschrift: blaue Tinte, deutsche Kurrent
\newline{}Schnitzler: mit Bleistift das Jahr »96« vermerkt }\toendnotes[C]{\smallbreak}
\pstart
           {\pb}\textcolor{gray}{\textbf{\textbf{Frankfurter Zeitung\orgindex{Frankfurter Zeitung@Frankfurter Zeitung|pw}}}}\pend
           
\pstart
           \textcolor{gray}{\textbf{(\begin{otherlanguage}{french}Gazette de Francfort\end{otherlanguage}\orgindex{Frankfurter Zeitung@Frankfurter Zeitung|pw}).}}\pend
           
\pstart
           \textcolor{gray}{\textbf{\textbf{\begin{otherlanguage}{french}Fondateur M.\end{otherlanguage}{ }L. Sonnemann\pwindex{Sonnemann, Leopold 1831-10-29 – 1909-10-30@\textsc{Sonnemann, Leopold} (1831-10-29 – 1909-10-30), \emph{Journalist/Journalistin, Herausgeber/Herausgeberin}|pw}.}}}\pend
           
\pstart
           \begin{otherlanguage}{french}\textcolor{gray}{\textbf{Journal\pwindex{Frankfurter Zeitung@\emph{Frankfurter Zeitung}|pwv} politique,
                        financier,}}\end{otherlanguage}\pend
           
\pstart
           \begin{otherlanguage}{french}\textcolor{gray}{\textbf{commercial et littéraire.}}\end{otherlanguage}\pend
           
\pstart
           \begin{otherlanguage}{french}\textcolor{gray}{\textbf{\textbf{Paraissant trois fois par jour.}}}\end{otherlanguage}\hfill \textsc{Paris\oindex{Paris@\textbf{Paris}, \emph{P.PPLC}|pw}}, 4. Juli.\pend
           
\pstart
           \begin{otherlanguage}{french}\textcolor{gray}{\textbf{\textbf{Bureau à Paris\oindex{Paris@\textbf{Paris}, \emph{P.PPLC}|pw}}}}\end{otherlanguage}\pend
           
\pstart
           \begin{otherlanguage}{french}\textcolor{gray}{\textbf{\textbf{24. Rue Feydeau\oindex{rue Feydeau@\textbf{rue Feydeau}, \emph{Straße (K.STR)}|pw}.}}}\end{otherlanguage}\pend
           
\pstart\center{}Mein lieber Freund,\pend\vspace{0.5em}
\pstart
           Alſo ſchön willkommen in \label{K_L02780-1v}\edtext{Hamburg\oindex{Hamburg@\textbf{Hamburg}, \emph{P.PPLA}|pw}}{\lemma{\textnormal{\emph{Hamburg}}}\Cendnote{\textnormal{Schnitzler hielt sich vom 4. 7. 1896 bis zum 7. 7. 1896 in Hamburg\oindex{Hamburg@\textbf{Hamburg}, \emph{P.PPLA}|pwk} auf, bevor er das Schiff nach Norwegen\oindex{Norwegen@\textbf{Norwegen}, \emph{A.PCLI}|pwk} bestieg.}}}\label{K_L02780-1} und von Herzen frohe
               Fahrt!\pend
           
\pstart
           Dieſer Brief ſoll Dir nur einen Gruß von mir \strikeout{\textcolor{gray}{×}\-\textcolor{gray}{×}\-\textcolor{gray}{×}} bringen.\pend
           
\pstart
           Neues weiß ich nicht.\pend
           
\pstart
           Auch hab’ ich keine Ahnung, wann ich von hier fortkomme. Die verfluchten Schwätzer im
                  {\pb}\label{K_L02780-2v}\edtext{\textsc{Palais Bourbon\oindex{Palais Bourbon@\textbf{Palais Bourbon}, \emph{Regierungsgebäude (K.RGB)}|pw}}}{\lemma{\textnormal{\emph{Palais Bourbon}}}\Cendnote{\textnormal{Sitz der \emph{französischen Nationalversammlung}\orgindex{Franzoesische Nationalversammlung@Französische Nationalversammlung|pwk}}}}\label{K_L02780-2} machen keiner\textcolor{gray}{l}lei Anſtalten, in die Ferien zu gehen. Auch
               ſonſt erſcheint mir meine Reiſe im dunkelſten Nebel.\pend
           
\pstart
           Ich ſchreibe Dir \strikeout{\textcolor{gray}{nac}h} nach Hamburg\oindex{Hamburg@\textbf{Hamburg}, \emph{P.PPLA}|pw},
               weil das noch im Bereich der Vorſtellungs-Möglichkeit liegt. Aber kannſt Du Dir,
               ehrlich geſagt, ein \textsc{Poste restante}-Büreau in \textsc{Trondjhem\oindex{Trondheim@\textbf{Trondheim}, \emph{P.PPLA2}|pwv}} vorſtellen? Ich nicht.\pend
           
\pstart
           Wie alle Jahre habe ich natürlich Furcht, Dich {\pb}wiederzuſehen, – diesmal aber mehr als je.\pend
           
\pstart
           Gott befohlen, mein lieber Freund, und möge Dir der ſchwed\oindex{Schweden@\textbf{Schweden}, \emph{A.PCLI}|pwv}iſche Himmel hold ſein (wenn es
               überhaupt in dieſem Lande\oindex{Schweden@\textbf{Schweden}, \emph{A.PCLI}|pwv},
               das ſeit \label{K_L02780-3v}\edtext{Guſtav Adolph\pwindex{Gustav II. Adolf von Schweden 19.12.1594 – 16.11.1632@\textsc{Gustav II. Adolf von Schweden} (19.12.1594 – 16.11.1632), \emph{König/Königin, Regent/Regentin}|pw}}{\lemma{\textnormal{\emph{Guſtav Adolph}}}\Cendnote{\textnormal{schwed\oindex{Schweden@\textbf{Schweden}, \emph{A.PCLI}|pwkv}ischer König\pwindex{Gustav II. Adolf von Schweden 19.12.1594 – 16.11.1632@\textsc{Gustav II. Adolf von Schweden} (19.12.1594 – 16.11.1632), \emph{König/Königin, Regent/Regentin}|pwkv} zwischen 1611 und 1632}}}\label{K_L02780-3} jede Exiſtenzberechtigung verloren hat, ſo etwas gibt, wie einen Himmel).\pend
           
\pstart
           Viele treue Grüße! {\\[\baselineskip]}Dein {\\[\baselineskip]}\spacefill\mbox{Paul Goldmann}\pend
           \leftskip=0em{}\selectlanguage{ngerman}\endnumbering\briefempfaengerindex{Schnitzler, Arthur@\textsc{Schnitzler, Arthur}!zzzGoldmann, Paul@\emph{von Paul Goldmann}!1896-07-041@{4. 7. {[}1896{]}}|)be}\mylabel{L02780h}  \normalsize

\doendnotes{C}
\bigskip
\vfill

\clearpage

\footnotesize

\lohead{\textsc{register}}

% Definiere theindex-Environment komplett neu ohne reledmac
\makeatletter
\renewenvironment{theindex}{%
  \section*{\indexname}%
  \setlength{\parindent}{0pt}%
  \setlength{\parskip}{0pt plus 0.3pt}%
  \let\item\@idxitem
}{%
  \clearpage
}
\makeatother

\IfFileExists{\jobname-pw.ind}{\input{\jobname-pw.ind}}{}

\end{document}

      