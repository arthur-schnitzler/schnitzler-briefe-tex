%% latex-leseansicht-vorspann.tex
%% Vorspann für die Leseansicht.
%% Lädt die gemeinsame Datei latex-vorspann.tex mit nicht gesetztem Schalter.

\newif\ifkorrekturansicht
\korrekturansichtfalse

\input{../tex-inputs/latex-vorspann}


         
         \renewcommand{\erwaehntePersonen}{Personen: Paul Goldmann,  Gustav II. Adolf von Schweden, Leopold Sonnemann}
         \renewcommand{\erwaehnteInstitutionen}{Institutionen: Frankfurter Zeitung, Französische Nationalversammlung}
         \renewcommand{\erwaehnteOrte}{Orte: Hamburg, Norwegen, Palais Bourbon, Paris, Schweden, Trondheim, rue Feydeau}
         \renewcommand{\erwaehnteWerke}{Werke: Frankfurter Zeitung}
               \section[ Paul Goldmann an Arthur Schnitzler, 4. 7. {[}1896{]}]{ Paul Goldmann an Arthur Schnitzler, 4. 7. {[}1896{]}}\nopagebreak\mylabel{v}\rehead{ }\begin{ledgroupsized}[t]{13cm}\normalsize\beginnumbering\briefempfaengerindex{Schnitzler, Arthur@\textsc{Schnitzler, Arthur}!zzzGoldmann, Paul@\emph{von Paul Goldmann}!1896-07-041@{4. 7. {[}1896{]}}|(be} \toendnotes[C]{\smallbreak\pagebreak[2]} \Standort{DLA, A:Schnitzler, HS.NZ85.1.3166.}
\physDesc{Brief, 1 Blatt, 3 Seiten, 893 Zeichen
\newline{}Handschrift: blaue Tinte, deutsche Kurrent
\newline{}Schnitzler: mit Bleistift das Jahr »96« vermerkt }\toendnotes[C]{\smallbreak}\pstart
           \noindent{}{\pb}\textcolor{gray}{\textbf{\textbf{Frankfurter Zeitung\orgindex{Frankfurter Zeitung@Frankfurter Zeitung|pw}}}}\pend
           \pstart
           \textcolor{gray}{\textbf{(\begin{otherlanguage}{french}Gazette de Francfort\end{otherlanguage}\orgindex{Frankfurter Zeitung@Frankfurter Zeitung|pw}).}}\pend
           \pstart
           \textcolor{gray}{\textbf{\textbf{\begin{otherlanguage}{french}Fondateur M.\end{otherlanguage}{ }L. Sonnemann\pwindex{Sonnemann, Leopold 1831-10-29 – 1909-10-30@\textsc{Sonnemann, Leopold} (1831-10-29 – 1909-10-30), \emph{Journalist, Herausgeber}|pw}.}}}\pend
           \pstart
           \begin{otherlanguage}{french}\textcolor{gray}{\textbf{Journal\pwindex{?? Werk@Nicht ermittelte Verfasserinnen und Verfasser!Frankfurter Zeitung1856 – 1943@\emph{Frankfurter Zeitung} {[}1856 – 1943{]}|pwv} politique,
                        financier,}}\end{otherlanguage}\pend
           \pstart
           \begin{otherlanguage}{french}\textcolor{gray}{\textbf{commercial et littéraire.}}\end{otherlanguage}\pend
           \pstart
           \begin{otherlanguage}{french}\textcolor{gray}{\textbf{\textbf{Paraissant trois fois par jour.}}}\end{otherlanguage}\hfill \textsc{Paris\oindex{Paris@\textbf{Paris}|pw}}, 4. Juli.\pend
           \pstart
           \begin{otherlanguage}{french}\textcolor{gray}{\textbf{\textbf{Bureau à Paris\oindex{Paris@\textbf{Paris}|pw}}}}\end{otherlanguage}\pend
           \pstart
           \begin{otherlanguage}{french}\textcolor{gray}{\textbf{\textbf{24. Rue Feydeau\oindex{rue Feydeau@\textbf{rue Feydeau}|pw}.}}}\end{otherlanguage}\pend
           \pstart\center{}Mein lieber Freund,\pend\pstart
           Alſo ſchön willkommen in \label{K_L02780-1v}\edtext{Hamburg\oindex{Hamburg@\textbf{Hamburg}|pw}}{\lemma{\textnormal{\emph{Hamburg}}}\Cendnote{\textnormal{Schnitzler\pwindex{Schnitzler, Arthur 15.05.1862 – 21.10.1931@\textsc{Schnitzler, Arthur} (15.05.1862 – 21.10.1931), \emph{Schriftsteller, Mediziner}|pwk} hielt sich vom 4. 7. 1896 bis zum 7. 7. 1896 in Hamburg\oindex{Hamburg@\textbf{Hamburg}|pwk} auf, bevor er das Schiff nach Norwegen\oindex{Norwegen@\textbf{Norwegen}|pwk} bestieg.}}}\label{K_L02780-1h} und von Herzen frohe
               Fahrt!\pend
           \pstart
           Dieſer Brief ſoll Dir nur einen Gruß von mir \strikeout{\textcolor{gray}{×}\-\textcolor{gray}{×}\-\textcolor{gray}{×}} bringen.\pend
           \pstart
           Neues weiß ich nicht.\pend
           \pstart
           Auch hab’ ich keine Ahnung, wann ich von hier fortkomme. Die verfluchten Schwätzer im
                  {\pb}\label{K_L02780-2v}\edtext{\textsc{Palais Bourbon\oindex{Palais Bourbon@\textbf{Palais Bourbon}|pw}}}{\lemma{\textnormal{\emph{Palais Bourbon}}}\Cendnote{\textnormal{Sitz der \emph{französischen Nationalversammlung}\orgindex{Franzoesische Nationalversammlung@Französische Nationalversammlung|pwk}}}}\label{K_L02780-2h} machen keiner\textcolor{gray}{l}lei Anſtalten, in die Ferien zu gehen. Auch
               ſonſt erſcheint mir meine Reiſe im dunkelſten Nebel.\pend
           \pstart
           Ich ſchreibe Dir \strikeout{\textcolor{gray}{nac}h} nach Hamburg\oindex{Hamburg@\textbf{Hamburg}|pw},
               weil das noch im Bereich der Vorſtellungs-Möglichkeit liegt. Aber kannſt Du Dir,
               ehrlich geſagt, ein \textsc{Poste restante}-Büreau in \textsc{Trondjhem\oindex{Trondheim@\textbf{Trondheim}|pwv}} vorſtellen? Ich nicht.\pend
           \pstart
           Wie alle Jahre habe ich natürlich Furcht, Dich {\pb}wiederzuſehen, – diesmal aber mehr als je.\pend
           \pstart
           Gott befohlen, mein lieber Freund, und möge Dir der ſchwed\oindex{Schweden@\textbf{Schweden}|pwv}iſche Himmel hold ſein (wenn es
               überhaupt in dieſem Lande\oindex{Schweden@\textbf{Schweden}|pwv},
               das ſeit \label{K_L02780-3v}\edtext{Guſtav Adolph\pwindex{Gustav II. Adolf von Schweden 19.12.1594 – 16.11.1632@\textsc{Gustav II. Adolf von Schweden} (19.12.1594 – 16.11.1632), \emph{König, Regent}|pw}}{\lemma{\textnormal{\emph{Guſtav Adolph}}}\Cendnote{\textnormal{schwed\oindex{Schweden@\textbf{Schweden}|pwkv}ischer König\pwindex{Gustav II. Adolf von Schweden 19.12.1594 – 16.11.1632@\textsc{Gustav II. Adolf von Schweden} (19.12.1594 – 16.11.1632), \emph{König, Regent}|pwkv} zwischen 1611 und 1632}}}\label{K_L02780-3h} jede Exiſtenzberechtigung verloren hat, ſo etwas gibt, wie einen Himmel).\pend
           \pstart
           Viele treue Grüße! {\\[\baselineskip]}Dein {\\[\baselineskip]}\spacefill\mbox{Paul Goldmann}\pend
           \leftskip=0em{}
         
         \endnumbering\mylabel{h}\end{ledgroupsized}  \newcommand{\dateiname}{L02780}\newcommand{\titel}{Paul Goldmann an Arthur Schnitzler, 4. 7. [1896]}\newcommand{\editorInnen}{Martin Anton Müller und Laura Untner}%% latex-leseansicht-abspann.tex
%% Abspann für die Leseansicht.
%% Der Schalter \ifkorrekturansicht ist bereits durch den Vorspann gesetzt.

%% latex-abspann.tex
%% Gemeinsamer Abspann für Korrekturansicht und Leseansicht.
%% Setzt den Schalter \ifkorrekturansicht voraus (gesetzt in den
%% einbindenden Dateien latex-korrekturansicht-abspann.tex bzw.
%% latex-leseansicht-abspann.tex).
%% ---------------------------------------------------------------

\normalsize

% Das esempio-Environment wird nur in der Leseansicht benötigt
\ifkorrekturansicht\else
\newenvironment{esempio}[3]%
{
    \vspace{1.5ex}
    \rlap{\underline{#1}}
    \par
    \setlength{\parindent}{0cm}
    \nopagebreak
    \leftskip=#2cm
    \rightskip=#3cm
}
{
    \par
}
\fi

\doendnotes{C}
\bigskip
\vfill

\clearpage

\footnotesize

\ifkorrekturansicht
  \lohead{\textsc{register}}
\fi

% theindex-Environment neu definieren ohne reledmac
\makeatletter
\renewenvironment{theindex}{%
  \ifkorrekturansicht
    \section*{\indexname}%
  \else
    \subsubsection*{Index der erwähnten Entitäten}%
  \fi
  \setlength{\parindent}{0pt}%
  \setlength{\parskip}{0pt plus 0.3pt}%
  \let\item\@idxitem
}{%
  \ifkorrekturansicht\clearpage\fi
}
\makeatother

\IfFileExists{\jobname-pw.ind}{\input{\jobname-pw.ind}}{}

% Quellenangabe nur in der Leseansicht
\ifkorrekturansicht\else
% Fallback-Definitionen, falls die .tex-Datei \titel etc. nicht gesetzt hat
\providecommand{\titel}{}
\providecommand{\editorInnen}{}
\providecommand{\dateiname}{\jobname}

\vspace{3cm}

\vfill

\footnotesize
\textsc{Quelle}: \titel. Herausgegeben von {\editorInnen}. In: \emph{Arthur Schnitzler: Briefwechsel mit Autorinnen und Autoren}.
 Digitale Edition, https://schnitzler-briefe.acdh.oeaw.ac.at/{\dateiname}.html (Stand \today)
\fi

\end{document}


      