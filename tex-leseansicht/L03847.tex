%% latex-leseansicht-vorspann.tex
%% Vorspann für die Leseansicht.
%% Lädt die gemeinsame Datei latex-vorspann.tex mit nicht gesetztem Schalter.

\newif\ifkorrekturansicht
\korrekturansichtfalse

\input{../tex-inputs/latex-vorspann}


\section[Theodor Herzl an Arthur Schnitzler, 1. 2. 1895]{L03847 Theodor Herzl an Arthur Schnitzler, 1. 2. 1895}
\nopagebreak\mylabel{L03847v}
\rehead{ }\normalsize\beginnumbering\briefempfaengerindex{Schnitzler, Arthur@\textsc{Schnitzler, Arthur}!zzzHerzl, Theodor@\emph{von Theodor Herzl}!1895-02-011@{1. 2. 1895}|(be}
\toendnotes[C]{\smallbreak\pagebreak[2]}
\correspDesc{Versand  durch Theodor Herzl am 1. 2. 1895 in Paris
\newline{}Erhalt  durch Arthur Schnitzler im Zeitraum [2. 2. 1895 – 6. 2. 1895?] in Wien}\toendnotes[C]{\smallbreak}
\Standort{CUL, Schnitzler, B 39.}
\physDesc{Brief, 1 Blatt, 3 Seiten, 1267 Zeichen
\newline{}Handschrift: schwarze Tinte, lateinische Kurrent
\newline{}Ordnung: mit Bleistift von unbekannter Hand nummeriert: »26« }
\buchAbdrucke{\weitereDrucke{Theodor Herzl: \emph{Briefe und
                        autobiographische Notizen 1866–1895}. Bearbeitet von Johannes Wachten in Zusammenarbeit mit Chaya Harel, Daisy Tycho und Manfred Winkler. Berlin, Frankfurt am Main, Wien: \emph{Propyläen} 1983, S. 571 (Briefe und Tagebücher. Herausgegeben von Alex Bein, Hermann Greive, Moshe Schaerf, Julius H. Schoeps und Johannes Wachten, 1).} }\toendnotes[C]{\smallbreak}
\pstart
           \raggedleft{}{\pb}Palais Bourbon\oindex{Palais Bourbon@\textbf{Palais Bourbon}, \emph{Regierungsgebäude}|pw}\pend
           
\pstart
           \raggedleft{}1. Febr. 895\pend
           
\pstart{}Lieber Freund!\pend\vspace{0.5em}
\pstart
           \label{K_L03847-1v}\edtext{»Das alles mein liebes Kindchen{\\} Ist
               mir schon einmal passirt...«}{\lemma{\textnormal{\emph{»Das … passirt...«}}}\Cendnote{\textnormal{Frei zitiert
                  nach der letzten Strophe in Heinrich Heines\pwindex{Heine, Heinrich 13.\,12.\,1797 Düsseldorf – 17.\,2.\,1856 Paris@\textsc{Heine, Heinrich} (13.\,12.\,1797 Düsseldorf – 17.\,2.\,1856 Paris), \emph{Schriftsteller}|pwk}
                  Gedicht Nr. 55 im \emph{Buch der Lieder}\pwindex{Heine, Heinrich 13.\,12.\,1797 Düsseldorf – 17.\,2.\,1856 Paris@\textsc{Heine, Heinrich} (13.\,12.\,1797 Düsseldorf – 17.\,2.\,1856 Paris), \emph{Schriftsteller}!Buch der Lieder@\strich\emph{Buch der Lieder}|pwk}:
                     »Glaub’ nicht, daß ich mich erschieße,| Wie schlimm auch die Sachen
                     stehn!| Das Alles, meine Süße,| Ist mir schon einmal geschehn.«}}}\label{K_L03847-1}\pend
           
\pstart
           Einmal? Wie oft! Man hat davon freilich doch immer wieder einen bittern Geschmack im
               Munde. Es hilft nicht viel, sich zu sagen, dass dieser Direktor\pwindex{Brahm, Otto 5.\,2.\,1856 Hamburg – 28.\,11.\,1912 Berlin@\textsc{Brahm, Otto} (5.\,2.\,1856 Hamburg – 28.\,11.\,1912 Berlin), \emph{Theaterleiter, Regisseur}|pwv} ein Strolch, ein Trottel od. dgl. ist.\pend
           
\pstart
           Ginge ich mir nach, ich würfe den Bettel jetzt schon in die Mistkiste. Aber ich habe
               nun schon angefangen und will standhaft bleiben. Aber abkürzen will ich nur den Ekel.
               Ich bitte Sie, von {\pb}der bekannten Hand\pwindex{?? [Schreibkraft für Arthur Schnitzler] @\textsc{?? [Schreibkraft für Arthur Schnitzler]}|pwuv} folgenden Brief
                  an Blumenthal\pwindex{Blumenthal, Oskar 13.\,3.\,1852 Berlin – 24.\,4.\,1917 ebd.@\textsc{Blumenthal, Oskar} (13.\,3.\,1852 Berlin – 24.\,4.\,1917 ebd.), \emph{Schriftsteller, Journalist, Theaterleiter}|pw} schreiben zu lassen.\pend
           
\pstart
           »Geehrter Herr! Mein vom Deutschen Theater\orgindex{Deutsches Theater Berlin@Deutsches Theater Berlin|pw} abgelehntes Schauspiel D. G.....\pwindex{Herzl, Theodor 2.\,5.\,1860 Budapest – 3.\,7.\,1904 Edlach@\textsc{Herzl, Theodor} (2.\,5.\,1860 Budapest – 3.\,7.\,1904 Edlach), \emph{Schriftsteller, Journalist}!neue Ghetto. Schauspiel in vier Acten@\strich\emph{Das neue Ghetto. Schauspiel in vier Acten}|pwv} ist Ihnen \strikeout{auf meinen \textcolor{gray}{×}\-\textcolor{gray}{×}\-\textcolor{gray}{×}\-\textcolor{gray}{×}\-\textcolor{gray}{×}\-\textcolor{gray}{×}\-\textcolor{gray}{×}}
                  zugegangen. Als ich bei der Einreichung diese Direction \strikeout{darum} ersuchte, Ihnen im Fall der Ablehnung \introOben{}das Manuskript\pwindex{Herzl, Theodor 2.\,5.\,1860 Budapest – 3.\,7.\,1904 Edlach@\textsc{Herzl, Theodor} (2.\,5.\,1860 Budapest – 3.\,7.\,1904 Edlach), \emph{Schriftsteller, Journalist}!neue Ghetto. Schauspiel in vier Acten@\strich\emph{Das neue Ghetto. Schauspiel in vier Acten}|pwv}\introOben{}
               zuzuschicken, wusste ich nicht, dass Sie die beiden anderen Theater\orgindex{Lessing-Theater@Lessing-Theater|pw}\orgindex{Berliner Theater@Berliner Theater|pw} unter Ihrer
                  Leitung vereinigen. Zur Entschliessung genügen dennoch drei Wochen. Diese Frist
                  läuft bis (flicken Sie das mir unbekannte Datum hinein, lieber Schnitzler). \strikeout{Dann} Dann ist das Manuscript\pwindex{Herzl, Theodor 2.\,5.\,1860 Budapest – 3.\,7.\,1904 Edlach@\textsc{Herzl, Theodor} (2.\,5.\,1860 Budapest – 3.\,7.\,1904 Edlach), \emph{Schriftsteller, Journalist}!neue Ghetto. Schauspiel in vier Acten@\strich\emph{Das neue Ghetto. Schauspiel in vier Acten}|pwv} an Herrn F. Schick\pwindex{Schik, Friedrich *~6.\,9.\,1857 Wien@\textsc{Schik, Friedrich} (*~6.\,9.\,1857 Wien), \emph{Notar, Journalist, Dramaturg}|pw}{ }Wien III Reisnerstrasse\oindex{Wien@\textbf{Wien}!III., Landstraße@\textbf{III., Landstraße}!Reisnerstraße 35@\textbf{Reisnerstraße 35}, \emph{Wohngebäude}|pw} – zurückzuschicken.\pend
           
\pstart
           \centering{}Achtungsvoll\pend
           
\pstart
           \raggedleft{}D\textsuperscript{r} A. S...l\pend
           
\pstart
           {\pb}Ich glaube ich werd’s dann dem Raimundtheater\orgindex{Raimund-Theater@Raimund-Theater|pw} geben. Und nach Raimund\orgindex{Raimund-Theater@Raimund-Theater|pw} geht’s schlafen.\pend
           
\pstart
           Leben Sie wohl mein sehr lieber Freund, und haben Sie mehr Theaterglück als Ihr {\\[\baselineskip]}herzlich ergebener {\\[\baselineskip]}\spacefill\mbox{Th H}\pend
           \leftskip=0em{}\selectlanguage{ngerman}\endnumbering\briefempfaengerindex{Schnitzler, Arthur@\textsc{Schnitzler, Arthur}!zzzHerzl, Theodor@\emph{von Theodor Herzl}!1895-02-011@{1. 2. 1895}|)be}\mylabel{L03847h}
\begin{anhang}
\end{anhang}\newcommand{\dateiname}{L03847}\newcommand{\titel}{Theodor Herzl an Arthur Schnitzler, 1. 2. 1895}\newcommand{\editorInnen}{Selma Jahnke und Martin Anton Müller}%% latex-leseansicht-abspann.tex
%% Abspann für die Leseansicht.
%% Der Schalter \ifkorrekturansicht ist bereits durch den Vorspann gesetzt.

%% latex-abspann.tex
%% Gemeinsamer Abspann für Korrekturansicht und Leseansicht.
%% Setzt den Schalter \ifkorrekturansicht voraus (gesetzt in den
%% einbindenden Dateien latex-korrekturansicht-abspann.tex bzw.
%% latex-leseansicht-abspann.tex).
%% ---------------------------------------------------------------

\normalsize

% Das esempio-Environment wird nur in der Leseansicht benötigt
\ifkorrekturansicht\else
\newenvironment{esempio}[3]%
{
    \vspace{1.5ex}
    \rlap{\underline{#1}}
    \par
    \setlength{\parindent}{0cm}
    \nopagebreak
    \leftskip=#2cm
    \rightskip=#3cm
}
{
    \par
}
\fi

\doendnotes{C}
\bigskip
\vfill

\clearpage

\footnotesize

\ifkorrekturansicht
  \lohead{\textsc{register}}
\fi

% theindex-Environment neu definieren ohne reledmac
\makeatletter
\renewenvironment{theindex}{%
  \ifkorrekturansicht
    \section*{\indexname}%
  \else
    \subsubsection*{Index der erwähnten Entitäten}%
  \fi
  \setlength{\parindent}{0pt}%
  \setlength{\parskip}{0pt plus 0.3pt}%
  \let\item\@idxitem
}{%
  \ifkorrekturansicht\clearpage\fi
}
\makeatother

\IfFileExists{\jobname-pw.ind}{\input{\jobname-pw.ind}}{}

% Quellenangabe nur in der Leseansicht
\ifkorrekturansicht\else
% Fallback-Definitionen, falls die .tex-Datei \titel etc. nicht gesetzt hat
\providecommand{\titel}{}
\providecommand{\editorInnen}{}
\providecommand{\dateiname}{\jobname}

\vspace{3cm}

\vfill

\footnotesize
\textsc{Quelle}: \titel. Herausgegeben von {\editorInnen}. In: \emph{Arthur Schnitzler: Briefwechsel mit Autorinnen und Autoren}.
 Digitale Edition, https://schnitzler-briefe.acdh.oeaw.ac.at/{\dateiname}.html (Stand \today)
\fi

\end{document}


