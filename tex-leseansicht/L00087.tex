%% latex-korrekturansicht-vorspann.tex
%% Vorspann für die Korrekturansicht.
%% Lädt die gemeinsame Datei latex-vorspann.tex mit gesetztem Schalter.

\newif\ifkorrekturansicht
\korrekturansichttrue

\input{../tex-inputs/latex-vorspann}


\section[Wilhelm Bölsche an Arthur Schnitzler, 24. 3. 1892]{L00087 Wilhelm Bölsche an Arthur Schnitzler, 24. 3. 1892}
\nopagebreak\mylabel{L00087v}
\rehead{ }\normalsize\beginnumbering\briefempfaengerindex{Schnitzler, Arthur@\textsc{Schnitzler, Arthur}!zzzBoelsche, Wilhelm@\emph{von Wilhelm Bölsche}!1892-03-241@{24. 3. 1892}|(be}
\toendnotes[C]{\smallbreak\pagebreak[2]}\Standort{DLA, A:Schnitzler, HS.NZ85.1.2577,4.}
\physDesc{Brief, 1 Blatt, 2 Seiten, 805 Zeichen
\newline{}Handschrift: schwarze Tinte, deutsche Kurrent
\newline{}Ordnung: mit rotem Buntstift von unbekannter Hand nummeriert:
                                    »5« }
\buchAbdrucke{\weitereDrucke{Wilhelm Bölsche: \emph{Briefwechsel. Mit Autoren der Freien Bühne}. Berlin: \emph{Weidler} 2010, S. 677.} }\toendnotes[C]{\smallbreak}
\pstart
           \raggedleft{}{\pb}Friedrichshagen\oindex{Friedrichshagen@\textbf{Friedrichshagen}, \emph{P.PPLX}|pw}{\\}24. III. 92. \pend
           
\pstart\center{}Hochgeehrter Herr Doktor!\pend\vspace{0.5em}
\pstart
           Verzeihen Sie, daß ich noch nicht geantwortet. Aber die Arbeitslaſt iſt für mich
               enorm in dieſen Momenten des \label{K_L00087-1v}\edtext{Neubaus!}{\lemma{\textnormal{\emph{Neubaus!}}}\Cendnote{\textnormal{Seit 1892 erschien die \emph{Freie Bühne}\pwindex{Freie Buehne fuer den Entwickelungskampf der Zeit@\emph{Freie Bühne für den Entwickelungskampf der Zeit}|pwk} nicht mehr als Wochen-, sondern
                  als Monatsschrift.}}}\label{K_L00087-1}\pend
           
\pstart
           Ihre »Elixire\pwindex{drei Elixire@\emph{Die drei Elixire}|pw}« bringe ich, ſobald es ſich machen
               läßt. Offen geſtanden, ſind ſie mir nicht ſo lieb wie die erſte Novelle\pwindex{Sohn. Aus den Papieren eines Arztes@\emph{Der Sohn. Aus den Papieren eines Arztes}|pwv}, ſie ſind lange nicht ſo aktuell.
               Aber ſie kommen doch!\pend
           
\pstart
           Mit den Gedichten iſt’s eine böſe Sache. Ich habe jetzt ein Lilienkron\pwindex{Liliencron, Detlev von 03.06.1844 – 22.07.1909@\textsc{Liliencron, Detlev von} (03.06.1844 – 22.07.1909), \emph{Schriftsteller/Schriftstellerin, Dichter/Dichterin, Dramatiker/Dramatikerin}|pw}’ſches\pwindex{Kartaeusermoench@\emph{Der Kartäusermönch}|pwv} probeweiſe einmal
               in’s nächſte Heft geſtreut {\pb}aber ich denke mir, es
               wird doch nur ſelten ſich auch nach dieſer Seite hin grade die »Freie Bühne\pwindex{Freie Buehne fuer den Entwickelungskampf der Zeit@\emph{Freie Bühne für den Entwickelungskampf der Zeit}|pw}« ausbauen laſſen. Lyriſche Zeitſchriften gibt’s ja
               genug, unſer Schwerpunkt muß unbedingt anderswo liegen. Wollen Sie’s indeſſen wagen,
               ſo ſenden Sie mir etwas, das Obige ſoll keine prinzipielle Ablehnung ſein!\pend
           
\pstart
           Mit beſtem Gruß{\\[\baselineskip]}Ihr{\\[\baselineskip]}\spacefill\mbox{Wilhelm Bölsche}\pend
           \leftskip=0em{}\selectlanguage{ngerman}\endnumbering\briefempfaengerindex{Schnitzler, Arthur@\textsc{Schnitzler, Arthur}!zzzBoelsche, Wilhelm@\emph{von Wilhelm Bölsche}!1892-03-241@{24. 3. 1892}|)be}\mylabel{L00087h}  \normalsize

\doendnotes{C}
\bigskip
\vfill

\clearpage

\footnotesize

\lohead{\textsc{register}}

% Definiere theindex-Environment komplett neu ohne reledmac
\makeatletter
\renewenvironment{theindex}{%
  \section*{\indexname}%
  \setlength{\parindent}{0pt}%
  \setlength{\parskip}{0pt plus 0.3pt}%
  \let\item\@idxitem
}{%
  \clearpage
}
\makeatother

\IfFileExists{\jobname-pw.ind}{\input{\jobname-pw.ind}}{}

\end{document}

      