\input{../tex-inputs/latex-pdf-vorspann}
\begin{center}
            \textcolor{red}{ENTWURF. ENTZIFFERUNG NOCH NICHT KORREKTURGELESEN}
                      \end{center}
            
               \section[Wilhelm Bölsche an Arthur Schnitzler, 24. 3. 1892]{ Wilhelm Bölsche an Arthur Schnitzler, 24. 3. 1892}\nopagebreak\mylabel{v}\rehead{ }\begin{ledgroupsized}[t]{13cm}\normalsize\beginnumbering\briefempfaengerindex{Schnitzler, Arthur@\textsc{Schnitzler, Arthur}!zzzBoelsche, Wilhelm@\emph{von Wilhelm Bölsche}!1892-03-241@{24. 3. 1892}|(be} \toendnotes[C]{\smallbreak\pagebreak[2]} \Standort{DLA, A:Schnitzler, HS.NZ85.1.2577,4.}
\physDesc{Brief, 1 Blatt, 2 Seiten
\newline{}Handschrift: schwarze Tinte, deutsche Kurrent\newline{}Ordnung: mit rotem Buntstift von unbekannter Hand nummeriert: »5« }\buchAbdrucke{\weitereDrucke{Wilhelm Bölsche: \emph{Briefwechsel. Mit Autoren der Freien Bühne}. Hg. Gerd-Hermann Susen. Berlin: \emph{Weidler} 2010, S. 677 (Werke und Briefe. Wissenschaftliche Ausgabe, Briefe I).} }\toendnotes[C]{\smallbreak}\pstart
           \raggedleft{}{\pb}Friedrichshagen\oindex{Friedrichshagen@\textbf{Friedrichshagen}|pw}{\\}24. III. 92. \pend
           \pstart\center{}Hochgeehrter Herr Doktor!\pend\pstart
           Verzeihen Sie, daß ich noch nicht geantwortet. Aber die Arbeitslaſt iſt für mich
                    enorm in dieſen Momenten des \label{K_L00087_1v}\edtext{Neubaus!}{\lemma{\textnormal{\emph{Neubaus!}}}\Cendnote{\textnormal{Seit 1892 erschien die \emph{Freie Bühne}\pwindex{Freie Buehne fuer den Entwickelungskampf der Zeit1892 – 1893@\emph{Freie Bühne für den Entwickelungskampf der Zeit}|pwk} nicht mehr als Wochen-,
                        sondern als Monatsschrift.}}}\label{K_L00087_1h}\pend
           \pstart
           Ihre »Elixire\pwindex{Schnitzler, Arthur 15.05.1862 – 21.10.1931@\textsc{Schnitzler, Arthur} (15.05.1862 – 21.10.1931), \emph{Schriftsteller, Mediziner}!drei Elixire1893@\strich\emph{Die drei Elixire} {[}1893{]}|pw}« bringe ich, ſobald es ſich
                    machen läßt. Offen geſtanden, ſind ſie mir nicht ſo lieb wie die erſte Novelle\pwindex{Hermann Bahrs Querulant17. 11. 1915@\emph{Hermann Bahrs Querulant} {[}17. 11. 1915{]}|pwv}, ſie ſind lange
                    nicht ſo aktuell. Aber ſie kommen doch!\pend
           \pstart
           Mit den Gedichten iſt’s eine böſe Sache. Ich habe jetzt ein Lilienkron\pwindex{Liliencron, Detlev von 03.06.1844 – 22.07.1909@\textsc{Liliencron, Detlev von} (03.06.1844 – 22.07.1909)|pw}’ſches\pwindex{Liliencron, Detlev von 03.06.1844 – 22.07.1909@\textsc{Liliencron, Detlev von} (03.06.1844 – 22.07.1909)!Kartaeusermoench01. 04. 1892@\strich\emph{Der Kartäusermönch} {[}01. 04. 1892{]}|pwv} probeweiſe einmal
                    in’s nächſte Heft geſtreut {\pb}aber ich denke mir, es
                    wird doch nur ſelten ſich auch nach dieſer Seite hin grade die »Freie Bühne\pwindex{Freie Buehne fuer den Entwickelungskampf der Zeit1892 – 1893@\emph{Freie Bühne für den Entwickelungskampf der Zeit}|pw}« ausbauen laſſen. Lyriſche Zeitſchriften gibt’s
                    ja genug, unſer Schwerpunkt muß unbedingt anderswo liegen. Wollen Sie’s indeſſen
                    wagen, ſo ſenden Sie mir etwas, das Obige ſoll keine prinzipielle Ablehnung
                    ſein!\pend
           \pstart
           Mit beſtem Gruß{\\[\baselineskip]}Ihr{\\[\baselineskip]}\spacefill\mbox{Wilhelm Bölsche}\pend
           \leftskip=0em{}\endnumbering\briefempfaengerindex{Schnitzler, Arthur@\textsc{Schnitzler, Arthur}!zzzBoelsche, Wilhelm@\emph{von Wilhelm Bölsche}!1892-03-241@{24. 3. 1892}|)be}\mylabel{h}\end{ledgroupsized}  \newcommand{\dateiname}{L00087}\newcommand{\titel}{Wilhelm Bölsche an Arthur Schnitzler, 24. 3. 1892}\newcommand{\editorInnen}{Martin Anton Müller und Gerd-Hermann Susen}\input{../tex-inputs/latex-pdf-abspann}
      