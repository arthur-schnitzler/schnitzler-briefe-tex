%% latex-korrekturansicht-vorspann.tex
%% Vorspann für die Korrekturansicht.
%% Lädt die gemeinsame Datei latex-vorspann.tex mit gesetztem Schalter.

\newif\ifkorrekturansicht
\korrekturansichttrue

\input{../tex-inputs/latex-vorspann}


\section[Peter Altenberg an Arthur Schnitzler, {[}zwischen 9. und 15. 11. 1912?{]}]{L02095 Peter Altenberg an Arthur Schnitzler, {[}zwischen 9. und
               15. 11. 1912?{]}}
\nopagebreak\mylabel{L02095v}
\rehead{ }\normalsize\beginnumbering\briefempfaengerindex{Schnitzler, Arthur@\textsc{Schnitzler, Arthur}!zzzAltenberg, Peter@\emph{von Peter Altenberg}!1912-11-151@{{[}zwischen 9. und
                  15. 11. 1912?{]}}|(be}
\toendnotes[C]{\smallbreak\pagebreak[2]}\Standort{DLA, A:Schnitzler, HS.NZ85.1.2342, S. 10–11.}
\physDesc{Brief, maschinenschriftliche Abschrift2 Blätter, 2 Seiten, 431 Zeichen
\newline{}Schreibmaschine}\toendnotes[C]{\smallbreak}
\pstart
           \raggedleft{}{\pb}Semmering\oindex{Semmering@\textbf{Semmering}, \emph{A.ADM3}|pw}, 1912.\pend
           \vspace{0.5em}
\pstart
           Lieber Herr Dr. Arthur Schnitzler,{ }\label{K_L02095-1v}\edtext{non réponse – – – c’est \uline{mille réponses}}{\lemma{\textnormal{\emph{non … réponses}}}\Cendnote{\textnormal{französisch, wörtlich: keine Antwort
                  kommt tausenden Antworten gleich. Die Nachfrage verrät, dass Schnitzler sich bis dahin nicht gemeldet hat, und sie
                  erklärt, warum er erst am 16. 11. 1912 in dieser Sache Bahr\pwindex{Bahr, Hermann 19.07.1863 – 15.01.1934@\textsc{Bahr, Hermann} (19.07.1863 – 15.01.1934), \emph{Schriftsteller/Schriftstellerin, Kritiker/Kritikerin}|pwk}
                  kontaktiert.}}}\label{K_L02095-1}! Aber weshalb? Das könnte Niemand verstehen – – – Meine
               körperlichen Qualen, (\uline{unertragbar} ohne Morphium),
               meine \uline{seelische} Verzweiflung, treiben mir das
               Schamgefühl aus. Ich {\pb}habe die Empfindung, dass ich
               doch \uline{irgendetwas} wert war, doch \uline{irgend}\uline{einen Anspruch} hätte an Mitleid vom
               Gleich-Kultivierten! Aber es scheint \uline{doch nicht so zu
                  sein.}{ }Schade!\pend
           \pstart Ihr\spacefill\mbox{P. A.}\pend{}\selectlanguage{ngerman}\endnumbering\briefempfaengerindex{Schnitzler, Arthur@\textsc{Schnitzler, Arthur}!zzzAltenberg, Peter@\emph{von Peter Altenberg}!1912-11-091@{{[}zwischen 9. und
                  15. 11. 1912?{]}}|)be}\mylabel{L02095h}  \normalsize

\doendnotes{C}
\bigskip
\vfill

\clearpage

\footnotesize

\lohead{\textsc{register}}

% Definiere theindex-Environment komplett neu ohne reledmac
\makeatletter
\renewenvironment{theindex}{%
  \section*{\indexname}%
  \setlength{\parindent}{0pt}%
  \setlength{\parskip}{0pt plus 0.3pt}%
  \let\item\@idxitem
}{%
  \clearpage
}
\makeatother

\IfFileExists{\jobname-pw.ind}{\input{\jobname-pw.ind}}{}

\end{document}

      