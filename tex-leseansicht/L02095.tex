%% latex-leseansicht-vorspann.tex
%% Vorspann für die Leseansicht.
%% Lädt die gemeinsame Datei latex-vorspann.tex mit nicht gesetztem Schalter.

\newif\ifkorrekturansicht
\korrekturansichtfalse

\input{../tex-inputs/latex-vorspann}


               \section[Peter Altenberg an Arthur Schnitzler, {[}zwischen 9. und 15. 11. 1912?{]}]{ Peter Altenberg an Arthur Schnitzler, {[}zwischen 9. und
               15. 11. 1912?{]}}\nopagebreak\mylabel{v}\rehead{ }\begin{ledgroupsized}[t]{13cm}\normalsize\beginnumbering\briefempfaengerindex{Schnitzler, Arthur@\textsc{Schnitzler, Arthur}!zzzAltenberg, Peter@\emph{von Peter Altenberg}!1912-11-091@{{[}zwischen 9. und
                  15. 11. 1912?{]}}|(be} \toendnotes[C]{\smallbreak\pagebreak[2]} \Standort{DLA, A:Schnitzler, HS.NZ85.1.2342, S. 10–11.}
\physDesc{maschinelle Abschrift\newline{}Zusatz: Original nicht nachweisbar }\toendnotes[C]{\smallbreak}\pstart
           \raggedleft{}{\pb}Semmering\oindex{Semmering@\textbf{Semmering}|pw}, 1912.\pend
           \pstart
           Lieber Herr Dr. Arthur Schnitzler, \label{K_L02095_1v}\edtext{non réponse – – – c’est \uline{mille réponses}}{\lemma{\textnormal{\emph{non … réponses}}}\Cendnote{\textnormal{französisch wörtlich: keine Antwort
                  kommt tausenden Antworten gleich. Die Nachfrage verrät, dass Schnitzler\pwindex{Schnitzler, Arthur 15.05.1862 – 21.10.1931@\textsc{Schnitzler, Arthur} (15.05.1862 – 21.10.1931), \emph{Schriftsteller, Mediziner}|pwk} sich bis dahin nicht gemeldet hat und sie
                  erklärt, warum er erst am 16. 11. 1912 in dieser Sache Bahr\pwindex{Bahr, Hermann 19.07.1863 – 15.01.1934@\textsc{Bahr, Hermann} (19.07.1863 – 15.01.1934), \emph{Schriftsteller, Kritiker}|pwk} kontaktiert.}}}\label{K_L02095_1h}!
               Aber weshalb? Das könnte Niemand verstehen – – – Meine körperlichen Qualen, (\uline{unertragbar} ohne Morphium), meine \uline{seelische} Verzweiflung, treiben mir das Schamgefühl aus. Ich
                  {\pb}habe die Empfindung, dass ich doch \uline{irgendetwas} wert war, doch \uline{irgend}\uline{einen Anspruch} hätte an Mitleid vom
               Gleich-Kultivierten! Aber es scheint \uline{doch nicht so zu
                  sein.}{ }Schade!\pend
           \pstart Ihr\spacefill\mbox{P. A.}\pend{}\endnumbering\briefempfaengerindex{Schnitzler, Arthur@\textsc{Schnitzler, Arthur}!zzzAltenberg, Peter@\emph{von Peter Altenberg}!1912-11-091@{{[}zwischen 9. und
                  15. 11. 1912?{]}}|)be}\mylabel{h}\end{ledgroupsized}  \newcommand{\dateiname}{L02095}\newcommand{\titel}{Peter Altenberg an Arthur Schnitzler, [zwischen 9. und 15. 11. 1912?]}\newcommand{\editorInnen}{Martin Anton Müller und Gerd-Hermann Susen}
            \footnotesize
\begin{ledgroupsized}[t]{11.5cm}
\doendnotes{C}
\end{ledgroupsized}
         %% latex-leseansicht-abspann.tex
%% Abspann für die Leseansicht.
%% Der Schalter \ifkorrekturansicht ist bereits durch den Vorspann gesetzt.

%% latex-abspann.tex
%% Gemeinsamer Abspann für Korrekturansicht und Leseansicht.
%% Setzt den Schalter \ifkorrekturansicht voraus (gesetzt in den
%% einbindenden Dateien latex-korrekturansicht-abspann.tex bzw.
%% latex-leseansicht-abspann.tex).
%% ---------------------------------------------------------------

\normalsize

% Das esempio-Environment wird nur in der Leseansicht benötigt
\ifkorrekturansicht\else
\newenvironment{esempio}[3]%
{
    \vspace{1.5ex}
    \rlap{\underline{#1}}
    \par
    \setlength{\parindent}{0cm}
    \nopagebreak
    \leftskip=#2cm
    \rightskip=#3cm
}
{
    \par
}
\fi

\doendnotes{C}
\bigskip
\vfill

\clearpage

\footnotesize

\ifkorrekturansicht
  \lohead{\textsc{register}}
\fi

% theindex-Environment neu definieren ohne reledmac
\makeatletter
\renewenvironment{theindex}{%
  \ifkorrekturansicht
    \section*{\indexname}%
  \else
    \subsubsection*{Index der erwähnten Entitäten}%
  \fi
  \setlength{\parindent}{0pt}%
  \setlength{\parskip}{0pt plus 0.3pt}%
  \let\item\@idxitem
}{%
  \ifkorrekturansicht\clearpage\fi
}
\makeatother

\IfFileExists{\jobname-pw.ind}{\input{\jobname-pw.ind}}{}

% Quellenangabe nur in der Leseansicht
\ifkorrekturansicht\else
% Fallback-Definitionen, falls die .tex-Datei \titel etc. nicht gesetzt hat
\providecommand{\titel}{}
\providecommand{\editorInnen}{}
\providecommand{\dateiname}{\jobname}

\vspace{3cm}

\vfill

\footnotesize
\textsc{Quelle}: \titel. Herausgegeben von {\editorInnen}. In: \emph{Arthur Schnitzler: Briefwechsel mit Autorinnen und Autoren}.
 Digitale Edition, https://schnitzler-briefe.acdh.oeaw.ac.at/{\dateiname}.html (Stand \today)
\fi

\end{document}


      