%% latex-leseansicht-vorspann.tex
%% Vorspann für die Leseansicht.
%% Lädt die gemeinsame Datei latex-vorspann.tex mit nicht gesetztem Schalter.

\newif\ifkorrekturansicht
\korrekturansichtfalse

\input{../tex-inputs/latex-vorspann}


\section[ Paul Goldmann an Arthur Schnitzler, 26. 5. [1904]]{L03444 Paul Goldmann an Arthur Schnitzler,  26. 5. [1904]}
\nopagebreak\mylabel{L03444v}
\rehead{ }\normalsize\beginnumbering\briefempfaengerindex{Schnitzler, Arthur@\textsc{Schnitzler, Arthur}!zzzGoldmann, Paul@\emph{von Paul Goldmann}!1904-05-261@{26. 5. [1904]}|(be}
\toendnotes[C]{\smallbreak\pagebreak[2]}
\correspDesc{Versand  durch Paul Goldmann am 26. 5. [1904] in Berlin
\newline{}Zustellung  im Zeitraum [27. 5. 1904
                  – 31. 5. 1904?] in Wien
\newline{}Erhalt  durch Arthur Schnitzler im Zeitraum [30. 5. 1904
                  – 31. 5. 1904?] in Wien}\toendnotes[C]{\smallbreak}
\Standort{DLA, A:Schnitzler, HS.NZ85.1.3174.}
\physDesc{Brief, 1 Blatt, 2 Seiten, 549 Zeichen
\newline{}Handschrift: blaue Tinte, deutsche Kurrent
\newline{}Schnitzler: mit Bleistift das Jahr »904« vermerkt }\toendnotes[C]{\smallbreak}
\pstart
           \raggedleft{}{\pb}\textcolor{gray}{\textbf{DESSAUERSTRASSE 19\oindex{Dessauer Straße@\textbf{Dessauer Straße}, \emph{Straße}|pw}}}\pend
           
\pstart
           Berlin\oindex{Berlin@\textbf{Berlin}, \emph{Hauptstadt}|pw}, 26. Mai.\pend
           
\pstart{}Mein lieber Freund,\pend\vspace{0.5em}
\pstart
           Deine \label{K_L03444-1v}\edtext{Karten}{\lemma{\textnormal{\emph{Karten}}}\Cendnote{\textnormal{Nachdem Goldmann\pwindex{Goldmann, Paul 31.\,1.\,1865 Breslau – 25.\,9.\,1935 Wien@\textsc{Goldmann, Paul} (31.\,1.\,1865 Breslau – 25.\,9.\,1935 Wien), \emph{Schriftsteller, Journalist}|pwk}
                  zuletzt aus Rom\oindex{Rom@\textbf{Rom}, \emph{Hauptstadt}|pwk} eine Karte bekommen hatte
                     (vgl. XXXX Auszeichnungsfehler: Dokument L03443 nicht gefunden), dürfte sich der
                  Dank nun auf eine Karte oder mehrere Karten aus Neapel\oindex{Neapel@\textbf{Neapel}|pwk} oder Sizilien\oindex{Sizilien@\textbf{Sizilien}, \emph{Land}|pwk} bezogen
                  haben.}}}\label{K_L03444-1} werden immer{ }ſchöner\substVorne{}\textsuperscript{, und}\substDazwischen{};\substHinten{} es muß eine herrliche Reiſe{ }ſein. Ich danke Dir vielmals, daß Du unterwegs
               meiner gedenkſt, und bedaure nur, daß ich Deine Adreſſe nicht weiß. Hoffentlich
               erreichen Dich meine nach Wien\oindex{Wien@\textbf{Wien}, \emph{Verwaltungsgebiet}|pw} gerichteten
               Briefe.\pend
           
\pstart
           Von mir iſt nichts Neues zu berichten. Es geht alles{ }ſeinen alten Gang.\pend
           
\pstart
           Nach Telegrammen aus \textsc{Kopenhagen\oindex{Kopenhagen@\textbf{Kopenhagen}, \emph{Hauptstadt}|pw}}, die ich in Berlin\oindex{Berlin@\textbf{Berlin}, \emph{Hauptstadt}|pw}er Blättern las,{ }ſind die
                  »Lebendigen Stunden\pwindex{Schnitzler, Arthur 15.\,5.\,1862 Wien – 21.\,10.\,1931 ebd.@\textsc{Schnitzler, Arthur} (15.\,5.\,1862 Wien – 21.\,10.\,1931 ebd.), \emph{Schriftsteller, Mediziner}!Lebendige Stunden. Vier Einakter@\strich\emph{Lebendige Stunden. Vier Einakter}|pw}« {\pb}dort\oindex{Kopenhagen@\textbf{Kopenhagen}, \emph{Hauptstadt}|pwv} mit großem Erfolg
                  \label{K_L03444-2v}\edtext{aufgeführt}{\lemma{\textnormal{\emph{aufgeführt}}}\Cendnote{\textnormal{\emph{Levende Timer. Skuespil i 1 akt}\pwindex{Schnitzler, Arthur 15.\,5.\,1862 Wien – 21.\,10.\,1931 ebd.@\textsc{Schnitzler, Arthur} (15.\,5.\,1862 Wien – 21.\,10.\,1931 ebd.), \emph{Schriftsteller, Mediziner}!Levende Timer. Skuespil i 1 akt@\strich\emph{Levende Timer. Skuespil i 1 akt}|pwk}
                  (Übersetzung: Johannes Nielsen\pwindex{Nielsen, Johannes 14.\,9.\,1870 Hindsholm – 20.\,12.\,1935@\textsc{Nielsen, Johannes} (14.\,9.\,1870 Hindsholm – 20.\,12.\,1935), \emph{Regisseur, Schauspieler, Übersetzer}|pwk}) hatte am
                     19. 5. 1904 am Kopenhagen\oindex{Kopenhagen@\textbf{Kopenhagen}, \emph{Hauptstadt}|pwk}er Kongelige Teater\oindex{Det Kongelige Teater@\textbf{Det Kongelige Teater}, \emph{Theater}|pwk}
                  Premiere.}}}\label{K_L03444-2} worden.\pend
           
\pstart
           Viele herzliche Grüße Dir und Deiner Frau\pwindex{Schnitzler, Olga 17.\,1.\,1882 Wien – 13.\,1.\,1970 Lugano@\textsc{Schnitzler, Olga} (17.\,1.\,1882 Wien – 13.\,1.\,1970 Lugano), \emph{Schauspielerin, Sängerin}|pwv} von {\\[\baselineskip]}Deinem getreuen {\\[\baselineskip]}\spacefill\mbox{Paul Goldmann.}\pend
           \leftskip=0em{}\selectlanguage{ngerman}\endnumbering\briefempfaengerindex{Schnitzler, Arthur@\textsc{Schnitzler, Arthur}!zzzGoldmann, Paul@\emph{von Paul Goldmann}!1904-05-261@{26. 5. [1904]}|)be}\mylabel{L03444h}  \newcommand{\dateiname}{L03444}\newcommand{\titel}{Paul Goldmann an Arthur Schnitzler, 26. 5. [1904]}\newcommand{\editorInnen}{Martin Anton Müller und Laura Untner}%% latex-leseansicht-abspann.tex
%% Abspann für die Leseansicht.
%% Der Schalter \ifkorrekturansicht ist bereits durch den Vorspann gesetzt.

%% latex-abspann.tex
%% Gemeinsamer Abspann für Korrekturansicht und Leseansicht.
%% Setzt den Schalter \ifkorrekturansicht voraus (gesetzt in den
%% einbindenden Dateien latex-korrekturansicht-abspann.tex bzw.
%% latex-leseansicht-abspann.tex).
%% ---------------------------------------------------------------

\normalsize

% Das esempio-Environment wird nur in der Leseansicht benötigt
\ifkorrekturansicht\else
\newenvironment{esempio}[3]%
{
    \vspace{1.5ex}
    \rlap{\underline{#1}}
    \par
    \setlength{\parindent}{0cm}
    \nopagebreak
    \leftskip=#2cm
    \rightskip=#3cm
}
{
    \par
}
\fi

\doendnotes{C}
\bigskip
\vfill

\clearpage

\footnotesize

\ifkorrekturansicht
  \lohead{\textsc{register}}
\fi

% theindex-Environment neu definieren ohne reledmac
\makeatletter
\renewenvironment{theindex}{%
  \ifkorrekturansicht
    \section*{\indexname}%
  \else
    \subsubsection*{Index der erwähnten Entitäten}%
  \fi
  \setlength{\parindent}{0pt}%
  \setlength{\parskip}{0pt plus 0.3pt}%
  \let\item\@idxitem
}{%
  \ifkorrekturansicht\clearpage\fi
}
\makeatother

\IfFileExists{\jobname-pw.ind}{\input{\jobname-pw.ind}}{}

% Quellenangabe nur in der Leseansicht
\ifkorrekturansicht\else
% Fallback-Definitionen, falls die .tex-Datei \titel etc. nicht gesetzt hat
\providecommand{\titel}{}
\providecommand{\editorInnen}{}
\providecommand{\dateiname}{\jobname}

\vspace{3cm}

\vfill

\footnotesize
\textsc{Quelle}: \titel. Herausgegeben von {\editorInnen}. In: \emph{Arthur Schnitzler: Briefwechsel mit Autorinnen und Autoren}.
 Digitale Edition, https://schnitzler-briefe.acdh.oeaw.ac.at/{\dateiname}.html (Stand \today)
\fi

\end{document}


