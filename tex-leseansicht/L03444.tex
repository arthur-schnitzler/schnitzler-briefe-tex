%% latex-leseansicht-vorspann.tex
%% Vorspann für die Leseansicht.
%% Lädt die gemeinsame Datei latex-vorspann.tex mit nicht gesetztem Schalter.

\newif\ifkorrekturansicht
\korrekturansichtfalse

\input{../tex-inputs/latex-vorspann}


         
         \renewcommand{\erwaehntePersonen}{Personen: Johannes Nielsen, Olga Schnitzler}
         \renewcommand{\erwaehnteOrte}{Orte: Berlin, Dessauer Straße, Det Kongelige Teater, Kopenhagen, Neapel, Rom, Sizilien, Wien}
         \renewcommand{\erwaehnteWerke}{Werke: Lebendige Stunden. Vier Einakter, Levende Timer. Skuespil i 1 akt}
               \section[ Paul Goldmann an Arthur Schnitzler, 26. 5. {[}1904{]}]{ Paul Goldmann an Arthur Schnitzler, 26. 5. {[}1904{]}}\nopagebreak\mylabel{v}\rehead{ }\begin{ledgroupsized}[t]{13cm}\normalsize\beginnumbering \toendnotes[C]{\smallbreak\pagebreak[2]} \Standort{DLA, A:Schnitzler, HS.NZ85.1.3174.}
\physDesc{Brief, 1 Blatt, 2 Seiten, 549 Zeichen
\newline{}Handschrift: blaue Tinte, deutsche Kurrent
\newline{}Schnitzler: mit Bleistift das Jahr »904« vermerkt }\toendnotes[C]{\smallbreak}\pstart
           \noindent{}\raggedleft{}{\pb}\textcolor{gray}{\textbf{DESSAUERSTRASSE 19\oindex{Dessauer Strasse@\textbf{Dessauer Straße}|pw}}}\pend
           \pstart
           Berlin\oindex{Berlin@\textbf{Berlin}|pw}, 26. Mai.\pend
           \pstart{}Mein lieber Freund,\pend\pstart
           Deine \label{K_L03444-1v}\edtext{Karten}{\lemma{\textnormal{\emph{Karten}}}\Cendnote{\textnormal{Nachdem Goldmann\pwindex{Goldmann, Paul 31.01.1865 – 25.09.1935@\textsc{Goldmann, Paul} (31.01.1865 – 25.09.1935), \emph{Schriftsteller, Journalist}|pwk}
                  zuletzt aus Rom\oindex{Rom@\textbf{Rom}|pwk} eine Karte bekommen hatte
                     (vgl. Paul Goldmann an Arthur Schnitzler, 1[7?]. 5. [1904]), dürfte sich der
                  Dank nun auf eine Karte oder mehrere Karten aus Neapel\oindex{Neapel@\textbf{Neapel}|pwk} oder Sizilien\oindex{Sizilien@\textbf{Sizilien}|pwk} bezogen
                  haben.}}}\label{K_L03444-1h} werden immer ſchöner\substVorne{}\textsuperscript{, und}\substDazwischen{};\substHinten{} es muß eine herrliche Reiſe ſein. Ich danke Dir vielmals, daß Du unterwegs
               meiner gedenkſt, und bedaure nur, daß ich Deine Adreſſe nicht weiß. Hoffentlich
               erreichen Dich meine nach Wien\oindex{Wien@\textbf{Wien}|pw} gerichteten
               Briefe.\pend
           \pstart
           Von mir iſt nichts Neues zu berichten. Es geht alles ſeinen alten Gang.\pend
           \pstart
           Nach Telegrammen aus \textsc{Kopenhagen\oindex{Kopenhagen@\textbf{Kopenhagen}|pw}}, die ich in Berlin\oindex{Berlin@\textbf{Berlin}|pw}er Blättern las, ſind die
                  »Lebendigen Stunden\pwindex{Schnitzler, Arthur 15.05.1862 – 21.10.1931@\textsc{Schnitzler, Arthur} (15.05.1862 – 21.10.1931), \emph{Schriftsteller, Mediziner}!Lebendige Stunden. Vier Einakter1901-12-23@\strich\emph{Lebendige Stunden. Vier Einakter} {[}1901-12-23{]}|pw}« {\pb}dort\oindex{Kopenhagen@\textbf{Kopenhagen}|pwv} mit großem Erfolg
                  \label{K_L03444-2v}\edtext{aufgeführt}{\lemma{\textnormal{\emph{aufgeführt}}}\Cendnote{\textnormal{\emph{Levende Timer. Skuespil i 1 akt}\pwindex{Nielsen, Johannes 1870-09-14 – 1935-12-20@\textsc{Nielsen, Johannes} (1870-09-14 – 1935-12-20), \emph{Regisseur, Schauspieler, Übersetzer}!Levende Timer. Skuespil i 1 akt1904-05-19@\strich\emph{Levende Timer. Skuespil i 1 akt} {[}Übersetzung, 1904-05-19{]}|pwk}
                  (Übersetzung: Johannes Nielsen\pwindex{Nielsen, Johannes 1870-09-14 – 1935-12-20@\textsc{Nielsen, Johannes} (1870-09-14 – 1935-12-20), \emph{Regisseur, Schauspieler, Übersetzer}|pwk}) hatte am
                     19. 5. 1904 am Kopenhagen\oindex{Kopenhagen@\textbf{Kopenhagen}|pwk}er Kongelige Teater\oindex{Det Kongelige Teater@\textbf{Det Kongelige Teater}|pwk}
                  Premiere.}}}\label{K_L03444-2h} worden.\pend
           \pstart
           Viele herzliche Grüße Dir und Deiner Frau\pwindex{Schnitzler, Olga 17.01.1882 – 13.01.1970@\textsc{Schnitzler, Olga} (17.01.1882 – 13.01.1970), \emph{Schauspielerin, Sängerin}|pwv} von {\\[\baselineskip]}Deinem getreuen {\\[\baselineskip]}\spacefill\mbox{Paul Goldmann.}\pend
           \leftskip=0em{}
         
         \endnumbering\mylabel{h}\end{ledgroupsized}  \newcommand{\dateiname}{L03444}\newcommand{\titel}{Paul Goldmann an Arthur Schnitzler, 26. 5. [1904]}\newcommand{\editorInnen}{Martin Anton Müller und Laura Untner}%% latex-leseansicht-abspann.tex
%% Abspann für die Leseansicht.
%% Der Schalter \ifkorrekturansicht ist bereits durch den Vorspann gesetzt.

%% latex-abspann.tex
%% Gemeinsamer Abspann für Korrekturansicht und Leseansicht.
%% Setzt den Schalter \ifkorrekturansicht voraus (gesetzt in den
%% einbindenden Dateien latex-korrekturansicht-abspann.tex bzw.
%% latex-leseansicht-abspann.tex).
%% ---------------------------------------------------------------

\normalsize

% Das esempio-Environment wird nur in der Leseansicht benötigt
\ifkorrekturansicht\else
\newenvironment{esempio}[3]%
{
    \vspace{1.5ex}
    \rlap{\underline{#1}}
    \par
    \setlength{\parindent}{0cm}
    \nopagebreak
    \leftskip=#2cm
    \rightskip=#3cm
}
{
    \par
}
\fi

\doendnotes{C}
\bigskip
\vfill

\clearpage

\footnotesize

\ifkorrekturansicht
  \lohead{\textsc{register}}
\fi

% theindex-Environment neu definieren ohne reledmac
\makeatletter
\renewenvironment{theindex}{%
  \ifkorrekturansicht
    \section*{\indexname}%
  \else
    \subsubsection*{Index der erwähnten Entitäten}%
  \fi
  \setlength{\parindent}{0pt}%
  \setlength{\parskip}{0pt plus 0.3pt}%
  \let\item\@idxitem
}{%
  \ifkorrekturansicht\clearpage\fi
}
\makeatother

\IfFileExists{\jobname-pw.ind}{\input{\jobname-pw.ind}}{}

% Quellenangabe nur in der Leseansicht
\ifkorrekturansicht\else
% Fallback-Definitionen, falls die .tex-Datei \titel etc. nicht gesetzt hat
\providecommand{\titel}{}
\providecommand{\editorInnen}{}
\providecommand{\dateiname}{\jobname}

\vspace{3cm}

\vfill

\footnotesize
\textsc{Quelle}: \titel. Herausgegeben von {\editorInnen}. In: \emph{Arthur Schnitzler: Briefwechsel mit Autorinnen und Autoren}.
 Digitale Edition, https://schnitzler-briefe.acdh.oeaw.ac.at/{\dateiname}.html (Stand \today)
\fi

\end{document}


      