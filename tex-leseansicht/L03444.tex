%% latex-korrekturansicht-vorspann.tex
%% Vorspann für die Korrekturansicht.
%% Lädt die gemeinsame Datei latex-vorspann.tex mit gesetztem Schalter.

\newif\ifkorrekturansicht
\korrekturansichttrue

\input{../tex-inputs/latex-vorspann}


\section[ Paul Goldmann an Arthur Schnitzler, 26. 5. {[}1904{]}]{L03444 Paul Goldmann an Arthur Schnitzler, 26. 5. {[}1904{]}}
\nopagebreak\mylabel{L03444v}
\rehead{ }\normalsize\beginnumbering\briefempfaengerindex{Schnitzler, Arthur@\textsc{Schnitzler, Arthur}!zzzGoldmann, Paul@\emph{von Paul Goldmann}!1904-05-261@{26. 5. {[}1904{]}}|(be}
\toendnotes[C]{\smallbreak\pagebreak[2]}\Standort{DLA, A:Schnitzler, HS.NZ85.1.3174.}
\physDesc{Brief, 1 Blatt, 2 Seiten, 549 Zeichen
\newline{}Handschrift: blaue Tinte, deutsche Kurrent
\newline{}Schnitzler: mit Bleistift das Jahr »904« vermerkt }\toendnotes[C]{\smallbreak}
\pstart
           \raggedleft{}{\pb}\textcolor{gray}{\textbf{DESSAUERSTRASSE 19\oindex{Dessauer Strasse@\textbf{Dessauer Straße}, \emph{Straße (K.STR)}|pw}}}\pend
           
\pstart
           Berlin\oindex{Berlin@\textbf{Berlin}, \emph{P.PPLC}|pw}, 26. Mai.\pend
           
\pstart{}Mein lieber Freund,\pend\vspace{0.5em}
\pstart
           Deine \label{K_L03444-1v}\edtext{Karten}{\lemma{\textnormal{\emph{Karten}}}\Cendnote{\textnormal{Nachdem Goldmann\pwindex{Goldmann, Paul 31.01.1865 – 25.09.1935@\textsc{Goldmann, Paul} (31.01.1865 – 25.09.1935), \emph{Schriftsteller/Schriftstellerin, Journalist/Journalistin}|pwk}
                  zuletzt aus Rom\oindex{Rom@\textbf{Rom}, \emph{P.PPLC}|pwk} eine Karte bekommen hatte
                     (vgl. Paul Goldmann an Arthur Schnitzler, 1[7?]. 5. [1904]), dürfte sich der
                  Dank nun auf eine Karte oder mehrere Karten aus Neapel\oindex{Neapel@\textbf{Neapel}, \emph{P.PPLA}|pwk} oder Sizilien\oindex{Sizilien@\textbf{Sizilien}, \emph{A.ADM1}|pwk} bezogen
                  haben.}}}\label{K_L03444-1} werden immer ſchöner\substVorne{}\textsuperscript{, und}\substDazwischen{};\substHinten{} es muß eine herrliche Reiſe ſein. Ich danke Dir vielmals, daß Du unterwegs
               meiner gedenkſt, und bedaure nur, daß ich Deine Adreſſe nicht weiß. Hoffentlich
               erreichen Dich meine nach Wien\oindex{Wien@\textbf{Wien}, \emph{A.ADM2}|pw} gerichteten
               Briefe.\pend
           
\pstart
           Von mir iſt nichts Neues zu berichten. Es geht alles ſeinen alten Gang.\pend
           
\pstart
           Nach Telegrammen aus \textsc{Kopenhagen\oindex{Kopenhagen@\textbf{Kopenhagen}, \emph{P.PPLC}|pw}}, die ich in Berlin\oindex{Berlin@\textbf{Berlin}, \emph{P.PPLC}|pw}er Blättern las, ſind die
                  »Lebendigen Stunden\pwindex{Lebendige Stunden. Vier Einakter@\emph{Lebendige Stunden. Vier Einakter}|pw}« {\pb}dort\oindex{Kopenhagen@\textbf{Kopenhagen}, \emph{P.PPLC}|pwv} mit großem Erfolg
                  \label{K_L03444-2v}\edtext{aufgeführt}{\lemma{\textnormal{\emph{aufgeführt}}}\Cendnote{\textnormal{\emph{Levende Timer. Skuespil i 1 akt}\pwindex{Levende Timer. Skuespil i 1 akt@\emph{Levende Timer. Skuespil i 1 akt}|pwk}
                  (Übersetzung: Johannes Nielsen\pwindex{Nielsen, Johannes 1870-09-14 – 1935-12-20@\textsc{Nielsen, Johannes} (1870-09-14 – 1935-12-20), \emph{Regisseur/Regisseurin, Schauspieler/Schauspielerin, Übersetzer/Übersetzerin}|pwk}) hatte am
                     19. 5. 1904 am Kopenhagen\oindex{Kopenhagen@\textbf{Kopenhagen}, \emph{P.PPLC}|pwk}er Kongelige Teater\oindex{Det Kongelige Teater@\textbf{Det Kongelige Teater}, \emph{Theater (K.THE)}|pwk}
                  Premiere.}}}\label{K_L03444-2} worden.\pend
           
\pstart
           Viele herzliche Grüße Dir und Deiner Frau\pwindex{Schnitzler, Olga 17.01.1882 – 13.01.1970@\textsc{Schnitzler, Olga} (17.01.1882 – 13.01.1970), \emph{Schauspieler/Schauspielerin, Sänger/Sängerin}|pwv} von {\\[\baselineskip]}Deinem getreuen {\\[\baselineskip]}\spacefill\mbox{Paul Goldmann.}\pend
           \leftskip=0em{}\selectlanguage{ngerman}\endnumbering\briefempfaengerindex{Schnitzler, Arthur@\textsc{Schnitzler, Arthur}!zzzGoldmann, Paul@\emph{von Paul Goldmann}!1904-05-261@{26. 5. {[}1904{]}}|)be}\mylabel{L03444h}  \normalsize

\doendnotes{C}
\bigskip
\vfill

\clearpage

\footnotesize

\lohead{\textsc{register}}

% Definiere theindex-Environment komplett neu ohne reledmac
\makeatletter
\renewenvironment{theindex}{%
  \section*{\indexname}%
  \setlength{\parindent}{0pt}%
  \setlength{\parskip}{0pt plus 0.3pt}%
  \let\item\@idxitem
}{%
  \clearpage
}
\makeatother

\IfFileExists{\jobname-pw.ind}{\input{\jobname-pw.ind}}{}

\end{document}

      