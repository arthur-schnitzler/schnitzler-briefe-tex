%% latex-korrekturansicht-vorspann.tex
%% Vorspann für die Korrekturansicht.
%% Lädt die gemeinsame Datei latex-vorspann.tex mit gesetztem Schalter.

\newif\ifkorrekturansicht
\korrekturansichttrue

\input{../tex-inputs/latex-vorspann}


\section[Paul Goldmann an Arthur Schnitzler, {[}Mitte? August 1894{]}]{L02604 Paul Goldmann an Arthur Schnitzler, {[}Mitte? August 1894{]}}
\nopagebreak\mylabel{L02604v}
\rehead{ }\normalsize\beginnumbering\briefempfaengerindex{Schnitzler, Arthur@\textsc{Schnitzler, Arthur}!zzzGoldmann, Paul@\emph{von Paul Goldmann}!1894-08-161@{{[}Mitte? August 1894{]}}|(be}
\toendnotes[C]{\smallbreak\pagebreak[2]}\Standort{DLA, A:Schnitzler, HS.NZ85.1.3164.}
\physDesc{Telegramm, 85 Zeichen
\newline{}maschinell
\newline{}Schnitzler: mit Bleistift Vermerk »\textsc{August 94}« 
\newline{}Ordnung: beschnitten }\toendnotes[C]{\smallbreak}
\pstart
           \noindent{}{\pb}ich \label{K_L02604-1v}\edtext{komme ischl\oindex{Bad Ischl@\textbf{Bad Ischl}, \emph{P.PPL}|pw}}{\lemma{\textnormal{\emph{komme ischl}}}\Cendnote{\textnormal{Der Brief Schnitzlers an Beer-Hofmann\pwindex{Beer-Hofmann, Richard 1866-07-11 – 1945-09-26@\textsc{Beer-Hofmann, Richard} (1866-07-11 – 1945-09-26), \emph{Schriftsteller/Schriftstellerin}|pwk} vom [18. 8. 1894] dürfte in unmittelbarer zeitlicher Nähe zu diesem
                  Telegramm verfasst worden sein, weil die Antwort auf das Telegramm skizziert wird.}}}\label{K_L02604-1}
               erbitte letztes einverstaendniss telegra\damage{m}{ }genf\oindex{Genf@\textbf{Genf}, \emph{P.PPLA}|pw} poste restante\pend
           \pstart \spacefill\mbox{= goldmann +}\pend{}\selectlanguage{ngerman}\endnumbering\briefempfaengerindex{Schnitzler, Arthur@\textsc{Schnitzler, Arthur}!zzzGoldmann, Paul@\emph{von Paul Goldmann}!1894-08-161@{{[}Mitte? August 1894{]}}|)be}\mylabel{L02604h}  \normalsize

\doendnotes{C}
\bigskip
\vfill

\clearpage

\footnotesize

\lohead{\textsc{register}}

% Definiere theindex-Environment komplett neu ohne reledmac
\makeatletter
\renewenvironment{theindex}{%
  \section*{\indexname}%
  \setlength{\parindent}{0pt}%
  \setlength{\parskip}{0pt plus 0.3pt}%
  \let\item\@idxitem
}{%
  \clearpage
}
\makeatother

\IfFileExists{\jobname-pw.ind}{\input{\jobname-pw.ind}}{}

\end{document}

      