%% latex-korrekturansicht-vorspann.tex
%% Vorspann für die Korrekturansicht.
%% Lädt die gemeinsame Datei latex-vorspann.tex mit gesetztem Schalter.

\newif\ifkorrekturansicht
\korrekturansichttrue

\input{../tex-inputs/latex-vorspann}


\section[Arthur Schnitzler an Felix Salten, {[}27. 6. 1891?{]}]{L02953 Arthur Schnitzler an Felix Salten, {[}27. 6. 1891?{]}}
\nopagebreak\mylabel{L02953v}
\rehead{ }\normalsize\beginnumbering\briefempfaengerindex{Salten, Felix@\textsc{Salten, Felix}!zzzSchnitzler, Arthur@\emph{von Arthur Schnitzler}!1891-06-271@{{[}27. 6. 1891?{]}}|(be}
\toendnotes[C]{\smallbreak\pagebreak[2]}\Standort{Wienbibliothek im Rathaus, ZPH 1681, 2.1.516.}
\physDesc{Brief, 1 Blatt, 2 Seiten, 345 Zeichen
\newline{}Handschrift: Bleistift, deutsche Kurrent
\newline{}Ordnung: mit Bleistift von unbekannter Hand Nummerierung der Seiten des Konvoluts:
                                    »17«–»18« }\toendnotes[C]{\smallbreak}
\pstart{}{\pb}Lieber Freund,\pend\vspace{0.5em}
\pstart
           Loris\pwindex{Hofmannsthal, Hugo von 1874-02-01 – 1929-07-15@\textsc{Hofmannsthal, Hugo von} (1874-02-01 – 1929-07-15), \emph{Schriftsteller/Schriftstellerin}|pw} war ſehr ärgerlich, als ich ihm ſagte,
               dß Sie morgen möglicherweiſe nicht ko{\geminationm}en; behauptet, er
               habe ſich extra Ihretwegen frei{\pb}gemacht;
               ſchwört, er ſagt Ihnen nicht Adieu wenn Sie wegfahren – was aus alldem folgt, iſt nur
               die längſt beka{\geminationn}te Thatſache, daſs Sie \label{K_L02953-1v}\edtext{morgen So{\geminationn}tag}{\lemma{\textnormal{\emph{morgen Sonntag}}}\Cendnote{\textnormal{Das Korrespondenzstück ist undatiert und kann
                  nur sehr unzuverlässig in die Korrespondenz eingeordnet werden.
                  Die Hinweise, die sich dem Text entnehmen lassen, besagen, dass der Brief an einem Samstag
                  verfasst wurde, sich Schnitzler und Hofmannsthal\pwindex{Hofmannsthal, Hugo von 1874-02-01 – 1929-07-15@\textsc{Hofmannsthal, Hugo von} (1874-02-01 – 1929-07-15), \emph{Schriftsteller/Schriftstellerin}|pwk} am Sonntag nachmittag treffen
                  wollten und möglicherweise eine Abreise Saltens\pwindex{Salten, Felix 06.09.1869 – 08.10.1945@\textsc{Salten, Felix} (06.09.1869 – 08.10.1945), \emph{Schriftsteller/Schriftstellerin, Journalist/Journalistin, Chefredakteur/Chefredakteurin}|pwk} bevorstand. Durch die Verwendung von »Loris\pwindex{Hofmannsthal, Hugo von 1874-02-01 – 1929-07-15@\textsc{Hofmannsthal, Hugo von} (1874-02-01 – 1929-07-15), \emph{Schriftsteller/Schriftstellerin}|pw}« als Name ist es vor 1893 einzuordnen. Ein
                  offensichtlicher Sonntag, an dem es zu einem Zusammentreffen aller drei an einem
                  Nachmittag kam, bietet sich im \emph{Tagebuch}\pwindex{Tagebuch@\emph{Tagebuch}|pwk}{ }Schnitzlers nicht an. Für Sonntag, den 21. 6. 1891 ist ein
                  besonderes Zusammentreffen zwischen Hofmannsthal\pwindex{Hofmannsthal, Hugo von 1874-02-01 – 1929-07-15@\textsc{Hofmannsthal, Hugo von} (1874-02-01 – 1929-07-15), \emph{Schriftsteller/Schriftstellerin}|pwk} und Salten\pwindex{Salten, Felix 06.09.1869 – 08.10.1945@\textsc{Salten, Felix} (06.09.1869 – 08.10.1945), \emph{Schriftsteller/Schriftstellerin, Journalist/Journalistin, Chefredakteur/Chefredakteurin}|pwk}
                  dokumentiert, durch das es nachvollziehbar scheint, dass Hofmannsthal\pwindex{Hofmannsthal, Hugo von 1874-02-01 – 1929-07-15@\textsc{Hofmannsthal, Hugo von} (1874-02-01 – 1929-07-15), \emph{Schriftsteller/Schriftstellerin}|pwk} an eine Fortsetzung des Gespräches lebhaftes
                  Interesse hatte: »Vorm. Loris\pwindex{Hofmannsthal, Hugo von 1874-02-01 – 1929-07-15@\textsc{Hofmannsthal, Hugo von} (1874-02-01 – 1929-07-15), \emph{Schriftsteller/Schriftstellerin}|pw} und
                        Salten\pwindex{Salten, Felix 06.09.1869 – 08.10.1945@\textsc{Salten, Felix} (06.09.1869 – 08.10.1945), \emph{Schriftsteller/Schriftstellerin, Journalist/Journalistin, Chefredakteur/Chefredakteurin}|pw} bei mir (letztrer hatte bei mir
                     geschlafen). Wir ›sprühten‹. Loris\pwindex{Hofmannsthal, Hugo von 1874-02-01 – 1929-07-15@\textsc{Hofmannsthal, Hugo von} (1874-02-01 – 1929-07-15), \emph{Schriftsteller/Schriftstellerin}|pw} ist
                     einfach stupend! –« In Saltens\pwindex{Salten, Felix 06.09.1869 – 08.10.1945@\textsc{Salten, Felix} (06.09.1869 – 08.10.1945), \emph{Schriftsteller/Schriftstellerin, Journalist/Journalistin, Chefredakteur/Chefredakteurin}|pwk}
                  Nachlass ist ein ›Protokoll‹ der geführten Gespräche überliefert (\emph{Wienbibliothek}, Nachlass Salten, ZPH 1681, Schachtel 5,
                     1.2.10). In den folgenden Tagen begegneten sich Schnitzler und Salten\pwindex{Salten, Felix 06.09.1869 – 08.10.1945@\textsc{Salten, Felix} (06.09.1869 – 08.10.1945), \emph{Schriftsteller/Schriftstellerin, Journalist/Journalistin, Chefredakteur/Chefredakteurin}|pwk}
                  mehrfach, vermutlich aber nicht am Samstag, dem 26. 6. 1891 für den
                     Schnitzler keinen Eintrag anlegte. Am
                  Folgetag, dem Sonntag, kam es am Abend zu einem gemeinsamen Essen von Schnitzler, Hofmannsthal\pwindex{Hofmannsthal, Hugo von 1874-02-01 – 1929-07-15@\textsc{Hofmannsthal, Hugo von} (1874-02-01 – 1929-07-15), \emph{Schriftsteller/Schriftstellerin}|pwk} und Beer-Hofmann\pwindex{Beer-Hofmann, Richard 1866-07-11 – 1945-09-26@\textsc{Beer-Hofmann, Richard} (1866-07-11 – 1945-09-26), \emph{Schriftsteller/Schriftstellerin}|pwk}, sodass es naheliegend scheint, dass dazu auch Salten\pwindex{Salten, Felix 06.09.1869 – 08.10.1945@\textsc{Salten, Felix} (06.09.1869 – 08.10.1945), \emph{Schriftsteller/Schriftstellerin, Journalist/Journalistin, Chefredakteur/Chefredakteurin}|pwk} geladen gewesen wäre.}}}\label{K_L02953-1}{ }5 Uhr ſicher von mir erwartet werden \pend
           
\pstart
           Herzlich Ihr {\\[\baselineskip]}\spacefill\mbox{Arthur}\pend
           \leftskip=0em{}\selectlanguage{ngerman}\endnumbering\briefempfaengerindex{Salten, Felix@\textsc{Salten, Felix}!zzzSchnitzler, Arthur@\emph{von Arthur Schnitzler}!1891-06-271@{{[}27. 6. 1891?{]}}|)be}\mylabel{L02953h}  \normalsize

\doendnotes{C}
\bigskip
\vfill

\clearpage

\footnotesize

\lohead{\textsc{register}}

% Definiere theindex-Environment komplett neu ohne reledmac
\makeatletter
\renewenvironment{theindex}{%
  \section*{\indexname}%
  \setlength{\parindent}{0pt}%
  \setlength{\parskip}{0pt plus 0.3pt}%
  \let\item\@idxitem
}{%
  \clearpage
}
\makeatother

\IfFileExists{\jobname-pw.ind}{\input{\jobname-pw.ind}}{}

\end{document}

      