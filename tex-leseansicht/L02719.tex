%% latex-leseansicht-vorspann.tex
%% Vorspann für die Leseansicht.
%% Lädt die gemeinsame Datei latex-vorspann.tex mit nicht gesetztem Schalter.

\newif\ifkorrekturansicht
\korrekturansichtfalse

\input{../tex-inputs/latex-vorspann}

\begin{center}
            \textcolor{red}{ENTWURF. ENTZIFFERUNG NOCH NICHT KORREKTURGELESEN}
                      \end{center}
            
               \section[Paul Goldmann an Arthur Schnitzler, 4. 11. {[}1893{]}]{ Paul Goldmann an Arthur Schnitzler, 4. 11. {[}1893{]}}\nopagebreak\mylabel{v}\rehead{ }\begin{ledgroupsized}[t]{13cm}\normalsize\beginnumbering\briefempfaengerindex{Schnitzler, Arthur@\textsc{Schnitzler, Arthur}!zzzGoldmann, Paul@\emph{von Paul Goldmann}!1893-11-041@{4. 11. {[}1893{]}}|(be} \toendnotes[C]{\smallbreak\pagebreak[2]} \Standort{DLA, A:Schnitzler, HS.NZ85.1.3163.}
\physDesc{Brief, 4 Blätter, 13 Seiten
\newline{}Handschrift: schwarze Tinte, deutsche Kurrent
\newline{}Schnitzler: 1) mit schwarzer Tinte das Jahr »93« vermerkt 2) mit rotem Buntstift zwei Unterstreichungen und fünf vertikale
                                 Markierungen}\toendnotes[C]{\smallbreak}\pstart
           \raggedleft{}{\pb}\textsc{Paris\oindex{Paris@\textbf{Paris}|pw}}, 4. November.\pend
           \pstart\center{}Mein lieber Freund,\pend\pstart
           Du mußt mir nicht böſe ſein: Ich habe hier wenig Beziehungen zur ärztlichen Welt und
               da ich außerdem mit tauſend Dingen die Hände voll zu thun hatte, habe ich eine Woche
               gebraucht, ehe ich Dir das Gewünſchte verſchaffen gekonnt. Ich ſende Dir anbei das
                  \label{K_L02719-1v}\edtext{»\textsc{Agenda médical\pwindex{Agenda medical@\emph{Agenda médical}|pw}}«}{\lemma{\textnormal{\emph{»Agenda médical«}}}\Cendnote{\textnormal{Die \emph{Agenda médical}\pwindex{Agenda medical@\emph{Agenda médical}|pwk} erschien jährlich und listete unter anderem
                  französische Mediziner. Goldmann\pwindex{Goldmann, Paul 31.01.1865 – 25.09.1935@\textsc{Goldmann, Paul} (31.01.1865 – 25.09.1935), \emph{Schriftsteller, Journalist}|pwk} sandte Schnitzler\pwindex{Schnitzler, Arthur 15.05.1862 – 21.10.1931@\textsc{Schnitzler, Arthur} (15.05.1862 – 21.10.1931), \emph{Schriftsteller, Mediziner}|pwk} vermutlich die neueste Ausgabe für
                  das Jahr 1894. Es ist unklar, wofür Schnitzler\pwindex{Schnitzler, Arthur 15.05.1862 – 21.10.1931@\textsc{Schnitzler, Arthur} (15.05.1862 – 21.10.1931), \emph{Schriftsteller, Mediziner}|pwk} die Namen der Professoren brauchte.}}}\label{K_L02719-1h}. Auf
               S. 381 findeſt Du die Namen derjenigen Profeſſoren unterſtrichen, die mir als die
               bedeutendſten bezeichnet worden; ihre Adreſſen ſind in dem S. 299 beginnenden {\pb}Verzeichniß enthalten. Wenn Du nun Weiteres brauchſt,
               für dieſe ſowie für alle zukünftigen Angelegenheiten – wenn Gänge zu machen oder
               Briefe auszutragen ſind etc. – ſo ſchreibe mir ſtets. Insbeſondern den mechaniſchen
               Theil eventueller journaliſtiſcher Maßnahmen kann ich Dir leicht beſtreiten helfen,
               da ich hier einen Büreaudiener habe. Aber auch ſonſt betrachte mich als Deinen
                  \label{K_L02719-2v}\edtext{\textsc{\begin{otherlanguage}{french}Ministre plénipotentiaire\end{otherlanguage}}}{\lemma{\textnormal{\emph{Ministre plénipotentiaire}}}\Cendnote{\textnormal{französisch: Gesandter}}}\label{K_L02719-2h} und gib’
               mir etwas zu {\pb}arbeiten. Freilich verlange ich einen
               Gegendienſt. Das iſt gemein, aber ich kann nicht anders. Schon während unſeres
                  \label{K_L02719-3v}\edtext{letzten Beiſammenſeins}{\lemma{\textnormal{\emph{letzten Beiſammenſeins}}}\Cendnote{\textnormal{am 18. 9. 1893 in Salzburg\oindex{Salzburg@\textbf{Salzburg}|pwk}}}}\label{K_L02719-3h} hatte ich die Bitte auf der Zunge, aber es erſchien mir doch gar zu
               erbärmlich, Dir damit zu kommen. Alſo ſchriftlich: Wäre Dir möglich, wenigſtens ein
               Paar Monate lang, meinem Schwager\pwindex{Rosengart, Josef 1860-02-08 – 1927-08-04@\textsc{Rosengart, Josef} (1860-02-08 – 1927-08-04), \emph{Arzt}|pwv} ein \label{K_L02719-4v}\edtext{Freiexemplar\pwindex{Internationale klinische Rundschau1887-01-01 – 1922@\emph{Internationale klinische Rundschau}|pwv}}{\lemma{\textnormal{\emph{Freiexemplar}}}\Cendnote{\textnormal{der \emph{Internationalen Klinischen Rundschau}\pwindex{Internationale klinische Rundschau1887-01-01 – 1922@\emph{Internationale klinische Rundschau}|pwk}, die bis September 1894 von Schnitzler\pwindex{Schnitzler, Arthur 15.05.1862 – 21.10.1931@\textsc{Schnitzler, Arthur} (15.05.1862 – 21.10.1931), \emph{Schriftsteller, Mediziner}|pwk}
                  herausgegeben wurde}}}\label{K_L02719-4h} zu bewilligen. Seine Praxi\strikeout{ſ}s geht noch nicht {\pb}gut genug, ihm ein
               Abonnement zu erlauben. Anderſeits möchte er gar zu gern\strikeout{,} das Blatt\pwindex{Internationale klinische Rundschau1887-01-01 – 1922@\emph{Internationale klinische Rundschau}|pwv} leſen.
               Und da durch einen glücklichen Zufall {\dotsfour} Ich bitte Dich
               alſo um Gewährung meiner Bitte, indem ich zugleich gegen die von mir begangene
               ſchamloſe Ausbeutung proteſtire. Adreſſe: \textsc{Dr. Josef Rosengart\pwindex{Rosengart, Josef 1860-02-08 – 1927-08-04@\textsc{Rosengart, Josef} (1860-02-08 – 1927-08-04), \emph{Arzt}|pw}}, \textsc{Frankfurt \textsuperscript{a}/\textsubscript{M}, \strikeout{Rossmark\oindex{Rossmarkt@\textbf{Roßmarkt}|pwv}}\oindex{Frankfurt am Main@\textbf{Frankfurt am Main}|pw}{ }Rossmarkt 20\oindex{Rossmarkt@\textbf{Roßmarkt}|pw}}.\pend
           \pstart
           Es iſt viel Erfreuliches in Deinem lieben Briefe. Vor allen Dingen bin ich {\pb}von Herzen froh, daß es endlich mit der Aufführung\pwindex{Schnitzler, Arthur 15.05.1862 – 21.10.1931@\textsc{Schnitzler, Arthur} (15.05.1862 – 21.10.1931), \emph{Schriftsteller, Mediziner}!Maerchen. Schauspiel in drei Aufzuegen1891 – 1891@\strich\emph{Das Märchen. Schauspiel in drei Aufzügen} {[}1891 – 1891{]}|pwv} ernſt wird. Da ich
               ſo gar nichts hörte, glaubte ich, es ſei wieder eine Verſchiebung eingetreten.
               Nochmals: ſobald die Aufführung feſtgeſetzt iſt, theile mir das \uline{umgehend} mit. Und reg’ Dich nicht auf wenn die Komödiantenbande\orgindex{Volkstheater@Volkstheater|pwv}, der Gewohnheit gemäß,
               Dich kränken ſollte. Ich hätte ſo gern genaue Details über die Proben gewußt, ich bin
               auch überzeugt, daß Du bei unſerem \label{K_L02719-6v}\edtext{nächſten {\pb}Beiſammerſein}{\lemma{\textnormal{\emph{nächſten Beiſammerſein}}}\Cendnote{\textnormal{am 23. 8. 1894 in Bad Ischl\oindex{Bad Ischl@\textbf{Bad Ischl}|pwk}}}}\label{K_L02719-6h} behaupten wirſt, ſie mir geſchrieben zu haben. Damit werde ich mich wohl
               begnügen müſſen. \strikeout{Sehr} Laß’ mich wenigſtens bald etwas
               über den Fortgang der Affaire wiſſen, – ja? Und ſtärkt Dir das nicht richtig die
               Productionsluſt, dieſe endliche Verwirklichung des ſo lange Erhofften?\pend
           \pstart
           Ich habe den »\textsc{Anatol\pwindex{Schnitzler, Arthur 15.05.1862 – 21.10.1931@\textsc{Schnitzler, Arthur} (15.05.1862 – 21.10.1931), \emph{Schriftsteller, Mediziner}!Anatol1892-10-29 – 1892-10-29@\strich\emph{Anatol} {[}1892-10-29 – 1892-10-29{]}|pw}}« und das »Märchen\pwindex{Schnitzler, Arthur 15.05.1862 – 21.10.1931@\textsc{Schnitzler, Arthur} (15.05.1862 – 21.10.1931), \emph{Schriftsteller, Mediziner}!Maerchen. Schauspiel in drei Aufzuegen1891 – 1891@\strich\emph{Das Märchen. Schauspiel in drei Aufzügen} {[}1891 – 1891{]}|pw}« hier dem neu
               begründeten Freien Theater für
                  ausländiſche Kunſt\orgindex{Theâtre de l'Œuvre@Théâtre de l'Œuvre|pwv}, dem »\textsc{Oeuvre\orgindex{Theâtre de l'Œuvre@Théâtre de l'Œuvre|pw}}« eingereicht. {\pb}Die \label{K_L02719-8v}\edtext{Herren\pwindex{Lugne-Poe, Aurelien-Marie 1869-12-27 – 1940-06-19@\textsc{Lugné-Poe, Aurélien-Marie} (1869-12-27 – 1940-06-19), \emph{Theaterleiter, Regisseur, Schauspieler}|pwv}}{\lemma{\textnormal{\emph{Herren}}}\Cendnote{\textnormal{Es ist nicht letztgültig zu klären, wen
                     Goldmann\pwindex{Goldmann, Paul 31.01.1865 – 25.09.1935@\textsc{Goldmann, Paul} (31.01.1865 – 25.09.1935), \emph{Schriftsteller, Journalist}|pwk} hiermit meinte. Geleitet wurde
                  das \emph{Théâtre de l'Œuvre}\orgindex{Theâtre de l'Œuvre@Théâtre de l'Œuvre|pwk} zu dieser Zeit
                  jedenfalls von Aurélien-Marie Lugné-Poe\pwindex{Lugne-Poe, Aurelien-Marie 1869-12-27 – 1940-06-19@\textsc{Lugné-Poe, Aurélien-Marie} (1869-12-27 – 1940-06-19), \emph{Theaterleiter, Regisseur, Schauspieler}|pwk}.
                  Auch in späteren Jahren spielte das \emph{Théâtre de
                     l'Œuvre}\orgindex{Theâtre de l'Œuvre@Théâtre de l'Œuvre|pwk} für Schnitzler\pwindex{Schnitzler, Arthur 15.05.1862 – 21.10.1931@\textsc{Schnitzler, Arthur} (15.05.1862 – 21.10.1931), \emph{Schriftsteller, Mediziner}|pwk} eine Rolle.
                  So empfahl etwa Marcel Schulz\pwindex{Schulz, Marcel *~1887?@\textsc{Schulz, Marcel} (*~1887?), \emph{Schriftsteller/Schriftstellerin}|pwk}{ }Lugné-Poe\pwindex{Lugne-Poe, Aurelien-Marie 1869-12-27 – 1940-06-19@\textsc{Lugné-Poe, Aurélien-Marie} (1869-12-27 – 1940-06-19), \emph{Theaterleiter, Regisseur, Schauspieler}|pwk} den \emph{Schleier der Beatrice}\pwindex{Schnitzler, Arthur 15.05.1862 – 21.10.1931@\textsc{Schnitzler, Arthur} (15.05.1862 – 21.10.1931), \emph{Schriftsteller, Mediziner}!Schleier der Beatrice. Schauspiel in fuenf Akten1900-12-01 – 1900-12-01@\strich\emph{Der Schleier der Beatrice. Schauspiel in fünf Akten} {[}1900-12-01 – 1900-12-01{]}|pwk} (Vgl. A. S.: \emph{Tagebuch}, 29. 1. 1907) und auch Paul Zifferer\pwindex{Zifferer, Paul 09.03.1879 – 14.02.1929@\textsc{Zifferer, Paul} (09.03.1879 – 14.02.1929), \emph{Schriftsteller, Journalist}|pwk} legte Schnitzler\pwindex{Schnitzler, Arthur 15.05.1862 – 21.10.1931@\textsc{Schnitzler, Arthur} (15.05.1862 – 21.10.1931), \emph{Schriftsteller, Mediziner}|pwk} das
                     \emph{Théâtre de l'Œuvre}\orgindex{Theâtre de l'Œuvre@Théâtre de l'Œuvre|pwk} »wegen [s]einer
                     Stücke für Paris\oindex{Paris@\textbf{Paris}|pw}« nahe (Vgl. A. S.: \emph{Tagebuch}, 6. 5. 1927).
                     1912 und 1922 inszenierte das
                     \emph{Théâtre de l'Œuvre}\orgindex{Theâtre de l'Œuvre@Théâtre de l'Œuvre|pwk} den \emph{Einakter}\pwindex{Schnitzler, Arthur 15.05.1862 – 21.10.1931@\textsc{Schnitzler, Arthur} (15.05.1862 – 21.10.1931), \emph{Schriftsteller, Mediziner}!letzten Masken1901@\strich\emph{Die letzten Masken} {[}1901{]}|pwk}{ }\emph{Die letzten Masken}\pwindex{Schnitzler, Arthur 15.05.1862 – 21.10.1931@\textsc{Schnitzler, Arthur} (15.05.1862 – 21.10.1931), \emph{Schriftsteller, Mediziner}!letzten Masken1901@\strich\emph{Die letzten Masken} {[}1901{]}|pwk} (Les Derniers masques\pwindex{Schnitzler, Arthur 15.05.1862 – 21.10.1931@\textsc{Schnitzler, Arthur} (15.05.1862 – 21.10.1931), \emph{Schriftsteller, Mediziner}!letzten Masken1901@\strich\emph{Die letzten Masken} {[}1901{]}|pwkv}).}}}\label{K_L02719-8h} waren ſehr
               vergnügt über mein ihnen gewidmetes \label{K_L02719-7v}\edtext{Feuilleton\pwindex{Goldmann, Paul 31.01.1865 – 25.09.1935@\textsc{Goldmann, Paul} (31.01.1865 – 25.09.1935), \emph{Schriftsteller, Journalist}!Dem Theâtre de l'Œuvre gewidmetes Feuilleton]1893 – 1893@\strich\emph{[Dem Théâtre de l'Œuvre gewidmetes Feuilleton]} {[}1893 – 1893{]}|pw}}{\lemma{\textnormal{\emph{Feuilleton}}}\Cendnote{\textnormal{XXXX bibl}}}\label{K_L02719-7h}, und da ich nicht gern
                  \strikeout{auf} die Gelegenheit zum Verlangen von
               Gegendienſten vorübergehen laſſe (ſiehe oben), ſo bat ich ſie, Deine Stücke\pwindex{Schnitzler, Arthur 15.05.1862 – 21.10.1931@\textsc{Schnitzler, Arthur} (15.05.1862 – 21.10.1931), \emph{Schriftsteller, Mediziner}!Anatol1892-10-29 – 1892-10-29@\strich\emph{Anatol} {[}1892-10-29 – 1892-10-29{]}|pwv}\pwindex{Schnitzler, Arthur 15.05.1862 – 21.10.1931@\textsc{Schnitzler, Arthur} (15.05.1862 – 21.10.1931), \emph{Schriftsteller, Mediziner}!Maerchen. Schauspiel in drei Aufzuegen1891 – 1891@\strich\emph{Das Märchen. Schauspiel in drei Aufzügen} {[}1891 – 1891{]}|pwv} zu leſen. Es ſind nämlich
               Leute darin, die deutſch können. Mach’ Dir aber keine allzu großen Hoffnungen. \strikeout{D}{ }\strikeout{\textcolor{gray}{St}ie} Sie frugen mich nämlich, ob die Stücke\pwindex{Schnitzler, Arthur 15.05.1862 – 21.10.1931@\textsc{Schnitzler, Arthur} (15.05.1862 – 21.10.1931), \emph{Schriftsteller, Mediziner}!Anatol1892-10-29 – 1892-10-29@\strich\emph{Anatol} {[}1892-10-29 – 1892-10-29{]}|pwv}\pwindex{Schnitzler, Arthur 15.05.1862 – 21.10.1931@\textsc{Schnitzler, Arthur} (15.05.1862 – 21.10.1931), \emph{Schriftsteller, Mediziner}!Maerchen. Schauspiel in drei Aufzuegen1891 – 1891@\strich\emph{Das Märchen. Schauspiel in drei Aufzügen} {[}1891 – 1891{]}|pwv} »myſtiſch« ſeien? Ich
               wußte nicht recht, was {\pb}ich ſagen ſollte: Bitte,
               ſind ſie myſtiſch?\pend
           \pstart
           Übrigens habe ich noch andere Eiſen für Dich hier im Feuer. Doch davon ſpäter.\pend
           \pstart
           Das Blühen in der lieben Wien\oindex{Wien@\textbf{Wien}|pw}er Künſtler-Laube –
               oh verdammt, welch’ ein Gleichniß! – beobachte ich mit wehmüthiger Freude. Gewiß, ich
               weiß, daß Eure\pwindex{Beer-Hofmann, Richard 11.07.1866 – 26.09.1945@\textsc{Beer-Hofmann, Richard} (11.07.1866 – 26.09.1945), \emph{Schriftsteller}|pwv}\pwindex{Hofmannsthal, Hugo von 01.02.1874 – 15.07.1929@\textsc{Hofmannsthal, Hugo von} (01.02.1874 – 15.07.1929), \emph{Schriftsteller}|pwv} drei Namen weit klingen werden, und in nicht langer Zeit. Ich ſehe, wie
               Ihr formt und ſchafft, und wünſche allen Segen {\pb}auf
               dieſes Schaffen herab. Und dann kehre ich in mich ein und habe das traurige Gefühl
               des Mannes, der einſam und ſchwach auf einem Stein ſitzen geblieben iſt und nur noch
               die fernen Stimmen der Begleiter hört, die durch den Wald hallen: aber ſie ſind weit
               und er wird ihnen nimmer nachkommen. Meine Arbeiten? Gewiß weiß ichs nicht, wenn ich
               etwas Gutes ſchreibe. Und wenn ich es wüßte: Hat das einen Werth, was ich thue? Geh’,
               das mußt {\pb}Du mir ſelbſt zugeben, daß ich in unſerem
               Kreiſe bereits immer deutlicher die bitterloſe Rolle über nehme »das Mannes, aus dem
               etwas hätte werden können«.\pend
           \pstart
           Ich bitte Dich inſtändig: veranlaſſe \textsc{Loris\pwindex{Hofmannsthal, Hugo von 01.02.1874 – 15.07.1929@\textsc{Hofmannsthal, Hugo von} (01.02.1874 – 15.07.1929), \emph{Schriftsteller}|pw}} und \textsc{Richard\pwindex{Beer-Hofmann, Richard 11.07.1866 – 26.09.1945@\textsc{Beer-Hofmann, Richard} (11.07.1866 – 26.09.1945), \emph{Schriftsteller}|pw}}, daß ſie mir die erſchienen oder zu erſcheinenden \label{K_L02719-10v}\edtext{Sachen\pwindex{Beer-Hofmann, Richard 11.07.1866 – 26.09.1945@\textsc{Beer-Hofmann, Richard} (11.07.1866 – 26.09.1945), \emph{Schriftsteller}!Novellen1. 12. 1893@\strich\emph{Novellen} {[}1. 12. 1893{]}|pwv}}{\lemma{\textnormal{\emph{Sachen}}}\Cendnote{\textnormal{Selbstständige Veröffentlichungen (Goldmann\pwindex{Goldmann, Paul 31.01.1865 – 25.09.1935@\textsc{Goldmann, Paul} (31.01.1865 – 25.09.1935), \emph{Schriftsteller, Journalist}|pwk} bezog sich auf »Bücher\pwindex{Beer-Hofmann, Richard 11.07.1866 – 26.09.1945@\textsc{Beer-Hofmann, Richard} (11.07.1866 – 26.09.1945), \emph{Schriftsteller}!Novellen1. 12. 1893@\strich\emph{Novellen} {[}1. 12. 1893{]}|pw}«) aus ungefähr dieser Zeit gibt es nur in Form von Richard Beer-Hofmann\pwindex{Beer-Hofmann, Richard 11.07.1866 – 26.09.1945@\textsc{Beer-Hofmann, Richard} (11.07.1866 – 26.09.1945), \emph{Schriftsteller}|pwk}s Novellensammlung\pwindex{Beer-Hofmann, Richard 11.07.1866 – 26.09.1945@\textsc{Beer-Hofmann, Richard} (11.07.1866 – 26.09.1945), \emph{Schriftsteller}!Novellen1. 12. 1893@\strich\emph{Novellen} {[}1. 12. 1893{]}|pwkv} mit dem bezeichnenden Titel \emph{Novellen}\pwindex{Beer-Hofmann, Richard 11.07.1866 – 26.09.1945@\textsc{Beer-Hofmann, Richard} (11.07.1866 – 26.09.1945), \emph{Schriftsteller}!Novellen1. 12. 1893@\strich\emph{Novellen} {[}1. 12. 1893{]}|pwk}. Von Hugo von Hofmannsthal\pwindex{Hofmannsthal, Hugo von 01.02.1874 – 15.07.1929@\textsc{Hofmannsthal, Hugo von} (01.02.1874 – 15.07.1929), \emph{Schriftsteller}|pwk} erschien in dieser Zeit nicht
                  nichts, jedoch handelte es sich stets um Veröffentlichungen in Zeitungen u.
                  Ä.}}}\label{K_L02719-10h} ſchicken. Ohne Briefe: ich weiß, daß die Briefe nach ſo langer Zeit
               ſchwer zu ſchreiben ſind. Die gewiſſe Furcht vor der Einleitung. Ich {\pb}möchte deßwegen aber nicht um die Bücher\pwindex{Beer-Hofmann, Richard 11.07.1866 – 26.09.1945@\textsc{Beer-Hofmann, Richard} (11.07.1866 – 26.09.1945), \emph{Schriftsteller}!Novellen1. 12. 1893@\strich\emph{Novellen} {[}1. 12. 1893{]}|pwv} kommen.\pend
           \pstart
           Wenn Du kannſt, ſo ſchick’ mir, bitte, gelegentlich noch einen »\textsc{Anatol\pwindex{Schnitzler, Arthur 15.05.1862 – 21.10.1931@\textsc{Schnitzler, Arthur} (15.05.1862 – 21.10.1931), \emph{Schriftsteller, Mediziner}!Anatol1892-10-29 – 1892-10-29@\strich\emph{Anatol} {[}1892-10-29 – 1892-10-29{]}|pw}}« – zu Propaganda-Zwecken.\pend
           \pstart
           \textsc{Bahr\pwindex{Bahr, Hermann 19.07.1863 – 15.01.1934@\textsc{Bahr, Hermann} (19.07.1863 – 15.01.1934), \emph{Schriftsteller, Kritiker}|pw}}: Du haſt eine ſo merkwürdige Art, gegen Leute gerecht ſein zu wollen, die ſich
               ſchurkiſch gegen Dich benehmen. Nein, – der Mann\pwindex{Bahr, Hermann 19.07.1863 – 15.01.1934@\textsc{Bahr, Hermann} (19.07.1863 – 15.01.1934), \emph{Schriftsteller, Kritiker}|pwv} iſt für mich kein großes Talent\pwindex{Bahr, Hermann 19.07.1863 – 15.01.1934@\textsc{Bahr, Hermann} (19.07.1863 – 15.01.1934), \emph{Schriftsteller, Kritiker}|pwv}, ſelbſt wenn er es ſein ſollte.
               Ungerechte {\pb}Beurheilung iſt bereits eine halbe
               Befriedigung des Haſſes. Und ſeit der \label{T_L02719-1v}\edtext{hundsföttischen}{\lemma{\textnormal{\emph{hundsföttischen}}}\Cendnote{\textnormal{Goldmann\pwindex{Goldmann, Paul 31.01.1865 – 25.09.1935@\textsc{Goldmann, Paul} (31.01.1865 – 25.09.1935), \emph{Schriftsteller, Journalist}|pwk} schrieb
                     »hundsfröttisch«}}}\label{T_L02719-1h}{ }Kritik\pwindex{Bahr, Hermann 19.07.1863 – 15.01.1934@\textsc{Bahr, Hermann} (19.07.1863 – 15.01.1934), \emph{Schriftsteller, Kritiker}!junge Oesterreich1893-09-20 – 1893-10-07@\strich\emph{Das junge Österreich} {[}1893-09-20 – 1893-10-07{]}|pwv} über Dich haſſe ich den
                  Kerl\pwindex{Bahr, Hermann 19.07.1863 – 15.01.1934@\textsc{Bahr, Hermann} (19.07.1863 – 15.01.1934), \emph{Schriftsteller, Kritiker}|pwv} mehr als je.\pend
           \pstart
           Der \label{K_L02719-12v}\edtext{Briefkaſten-Diebſtahl}{\lemma{\textnormal{\emph{Briefkaſten-Diebſtahl}}}\Cendnote{\textnormal{Das Buch\pwindex{Sosnosky, Theodor von 04.01.1866 – 03.02.1943@\textsc{Sosnosky, Theodor von} (04.01.1866 – 03.02.1943), \emph{Schriftsteller}!Ridicula1893 – 1893@\strich\emph{Ridicula} {[}1893 – 1893{]}|pwkv}, um das es hier ging, war \emph{Ridicula}\pwindex{Sosnosky, Theodor von 04.01.1866 – 03.02.1943@\textsc{Sosnosky, Theodor von} (04.01.1866 – 03.02.1943), \emph{Schriftsteller}!Ridicula1893 – 1893@\strich\emph{Ridicula} {[}1893 – 1893{]}|pwk} von \emph{Theodor von
                     Sosnosky}\textcolor{red}{\textsuperscript{XXXX indx}}, eine Sammlung vermeintlicher »literarische[r]
                     Lächerlichkeiten«, wie in der \emph{Schönen
                     blauen Donau}\pwindex{der schoenen blauen Donau1886 – 1896@\emph{An der schönen blauen Donau}|pwk} zusammengefasst wurde. Darin aufgenommen wurden – ohne
                  Erlaubnis – auch 50 Seiten aus dem Briefkasten\pwindex{der schoenen blauen Donau1886 – 1896@\emph{An der schönen blauen Donau}|pwkv} der \emph{Schönen
                     blauen Donau}\pwindex{der schoenen blauen Donau1886 – 1896@\emph{An der schönen blauen Donau}|pwk}. Siehe h. k.\pwindex{K., H. @\textsc{K., H.}, \emph{Journalist/Journalistin}|pwk}: \emph{Neue Bücher}\pwindex{Neue Buecher1893-12-01 – 1893-12-01@\emph{Neue Bücher} {[}1893-12-01 – 1893-12-01{]}|pwk}. In: \emph{An der schönen blauen
                        Donau}\pwindex{der schoenen blauen Donau1886 – 1896@\emph{An der schönen blauen Donau}|pwk}, Jg. 8, Nr. 23, 1. 12. 1893,
                     S. 552.}}}\label{K_L02719-12h} des \textsc{Sosnosky\pwindex{Sosnosky, Theodor von 04.01.1866 – 03.02.1943@\textsc{Sosnosky, Theodor von} (04.01.1866 – 03.02.1943), \emph{Schriftsteller}|pw}} iſt ſcheußlich. Ich habe mit meinem Onkel\pwindex{Mamroth, Fedor 21.02.1851 – 25.06.1907@\textsc{Mamroth, Fedor} (21.02.1851 – 25.06.1907), \emph{Journalist, Kritiker}|pwv} berathen, aber glaube, wir können nichts machen,
               geſetzlich. Höchſtens eine \textcolor{gray}{Zü}chtigung im Blatte\pwindex{der schoenen blauen Donau1886 – 1896@\emph{An der schönen blauen Donau}|pwv}, die aber auch eine Reklame für das
                  Buch\pwindex{Sosnosky, Theodor von 04.01.1866 – 03.02.1943@\textsc{Sosnosky, Theodor von} (04.01.1866 – 03.02.1943), \emph{Schriftsteller}!Ridicula1893 – 1893@\strich\emph{Ridicula} {[}1893 – 1893{]}|pwv} des Gauner\pwindex{Sosnosky, Theodor von 04.01.1866 – 03.02.1943@\textsc{Sosnosky, Theodor von} (04.01.1866 – 03.02.1943), \emph{Schriftsteller}|pwv}s wäre.\pend
           \pstart
           {\pb}\textsc{Herzl\pwindex{Herzl, Theodor 02.05.1860 – 03.07.1904@\textsc{Herzl, Theodor} (02.05.1860 – 03.07.1904), \emph{Schriftsteller, Journalist}|pw}} iſt ſeit einigen Wochen ſehr \label{K_L02719-13v}\edtext{krank}{\lemma{\textnormal{\emph{krank}}}\Cendnote{\textnormal{Von seiner Malariainfektion
                  berichtete Theodor Herzl\pwindex{Herzl, Theodor 02.05.1860 – 03.07.1904@\textsc{Herzl, Theodor} (02.05.1860 – 03.07.1904), \emph{Schriftsteller, Journalist}|pwk}{ }Schnitzler\pwindex{Schnitzler, Arthur 15.05.1862 – 21.10.1931@\textsc{Schnitzler, Arthur} (15.05.1862 – 21.10.1931), \emph{Schriftsteller, Mediziner}|pwk} am 8. 12. 1893 auch brieflich. Siehe Theodor Herzl\pwindex{Herzl, Theodor 02.05.1860 – 03.07.1904@\textsc{Herzl, Theodor} (02.05.1860 – 03.07.1904), \emph{Schriftsteller, Journalist}|pwk}: \emph{Briefe und
                        Tagebücher}. Hg. Alex Bein, Hermann Greive, Moshe Schaerf und Julius
                     H. Schoeps. Bd. 1.: \emph{Briefe und autobiographische Notizen.
                        1866–1895}. Bearbeitet von Johannes Wachten. In Zusammenarbeit mit
                     Chaya Harel, Daisy Tycho und Manfred Winkler. Berlin,
                     Frankfurt am Main, Wien: \emph{Ullstein}\orgindex{Ullstein Verlag@Ullstein Verlag|pwk}/\emph{Propyläen}\orgindex{Propylaeen Verlag@Propyläen Verlag|pwk}{ }1983, S. 545.}}}\label{K_L02719-13h}: \textsc{Malaria} oder ſo etwas.\pend
           \pstart
           Was Neues in Wien\oindex{Wien@\textbf{Wien}|pw}? Bitte ſchreibe bald.\pend
           \pstart
           Auch ein perſönliches Wort: Geſundheit, Production, materielle Fragen.\pend
           \pstart
           Mir geht es ſchlecht, oh ſo ſchlecht!\pend
           \pstart
           Viele treue Grüße!\pend
           \pstart
           Dein {\\[\baselineskip]}\spacefill\mbox{Paul Goldm}\pend
           \leftskip=0em{}\endnumbering\briefempfaengerindex{Schnitzler, Arthur@\textsc{Schnitzler, Arthur}!zzzGoldmann, Paul@\emph{von Paul Goldmann}!1893-11-041@{4. 11. {[}1893{]}}|)be}\mylabel{h}\end{ledgroupsized}\begin{anhang}\end{anhang}\newcommand{\dateiname}{L02719}\newcommand{\titel}{Paul Goldmann an Arthur Schnitzler, 4. 11. [1893]}\newcommand{\editorInnen}{Martin Anton Müller und Laura Untner}%% latex-leseansicht-abspann.tex
%% Abspann für die Leseansicht.
%% Der Schalter \ifkorrekturansicht ist bereits durch den Vorspann gesetzt.

%% latex-abspann.tex
%% Gemeinsamer Abspann für Korrekturansicht und Leseansicht.
%% Setzt den Schalter \ifkorrekturansicht voraus (gesetzt in den
%% einbindenden Dateien latex-korrekturansicht-abspann.tex bzw.
%% latex-leseansicht-abspann.tex).
%% ---------------------------------------------------------------

\normalsize

% Das esempio-Environment wird nur in der Leseansicht benötigt
\ifkorrekturansicht\else
\newenvironment{esempio}[3]%
{
    \vspace{1.5ex}
    \rlap{\underline{#1}}
    \par
    \setlength{\parindent}{0cm}
    \nopagebreak
    \leftskip=#2cm
    \rightskip=#3cm
}
{
    \par
}
\fi

\doendnotes{C}
\bigskip
\vfill

\clearpage

\footnotesize

\ifkorrekturansicht
  \lohead{\textsc{register}}
\fi

% theindex-Environment neu definieren ohne reledmac
\makeatletter
\renewenvironment{theindex}{%
  \ifkorrekturansicht
    \section*{\indexname}%
  \else
    \subsubsection*{Index der erwähnten Entitäten}%
  \fi
  \setlength{\parindent}{0pt}%
  \setlength{\parskip}{0pt plus 0.3pt}%
  \let\item\@idxitem
}{%
  \ifkorrekturansicht\clearpage\fi
}
\makeatother

\IfFileExists{\jobname-pw.ind}{\input{\jobname-pw.ind}}{}

% Quellenangabe nur in der Leseansicht
\ifkorrekturansicht\else
% Fallback-Definitionen, falls die .tex-Datei \titel etc. nicht gesetzt hat
\providecommand{\titel}{}
\providecommand{\editorInnen}{}
\providecommand{\dateiname}{\jobname}

\vspace{3cm}

\vfill

\footnotesize
\textsc{Quelle}: \titel. Herausgegeben von {\editorInnen}. In: \emph{Arthur Schnitzler: Briefwechsel mit Autorinnen und Autoren}.
 Digitale Edition, https://schnitzler-briefe.acdh.oeaw.ac.at/{\dateiname}.html (Stand \today)
\fi

\end{document}


      