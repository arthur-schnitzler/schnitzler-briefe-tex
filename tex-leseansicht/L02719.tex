%% latex-leseansicht-vorspann.tex
%% Vorspann für die Leseansicht.
%% Lädt die gemeinsame Datei latex-vorspann.tex mit nicht gesetztem Schalter.

\newif\ifkorrekturansicht
\korrekturansichtfalse

\input{../tex-inputs/latex-vorspann}


         
         \renewcommand{\erwaehntePersonen}{Personen: Hermann Bahr, Richard Beer-Hofmann, Theodor Herzl, Hugo von Hofmannsthal, Aurélien-Marie Lugné-Poe, Fedor Mamroth, Josef Rosengart, Marcel Schulz, Theodor von Sosnosky, Paul Zifferer, h. k.}
         \renewcommand{\erwaehnteInstitutionen}{Institutionen: Eduard Trewendt, Propyläen Verlag, Théâtre de l’Œuvre, Ullstein Verlag, Volkstheater}
         \renewcommand{\erwaehnteOrte}{Orte: Frankfurt am Main, Frankreich, Paris, Roßmarkt, Salzburg, Wien}
         \renewcommand{\erwaehnteWerke}{Werke: Agenda médical, An der schönen blauen Donau, Anatol, Das Märchen. Schauspiel in drei Aufzügen, Das junge Österreich, Der Schleier der Beatrice. Schauspiel in fünf Akten, Die letzten Masken, Frankfurter Zeitung, Internationale klinische Rundschau, Neue Bücher, Novellen, Pariser Theater, Ridicula}
               \section[Paul Goldmann an Arthur Schnitzler, 4. 11. {[}1893{]}]{ Paul Goldmann an Arthur Schnitzler, 4. 11. {[}1893{]}}\nopagebreak\mylabel{v}\rehead{ }\begin{ledgroupsized}[t]{13cm}\normalsize\beginnumbering \toendnotes[C]{\smallbreak\pagebreak[2]} \Standort{DLA, A:Schnitzler, HS.NZ85.1.3163.}
\physDesc{Brief, 5 Blätter, 13 Seiten, 5058 Zeichen
\newline{}Handschrift: schwarze Tinte, deutsche Kurrent
\newline{}Schnitzler: 1) mit schwarzer Tinte das Jahr »93« vermerkt  2) mit rotem Buntstift zwei Unterstreichungen und fünf vertikale
                                 Markierungen}\toendnotes[C]{\smallbreak}\pstart
           \raggedleft{}{\pb}\textsc{Paris\oindex{Paris@\textbf{Paris}|pw}}, 4. November.\pend
           \pstart{}Mein lieber Freund,\pend\pstart
           Du mußt mir nicht böſe ſein: Ich habe hier wenig Beziehungen zur ärztlichen Welt und
               da ich außerdem mit tauſend Dingen die Hände voll zu thun hatte, habe ich eine Woche
               gebraucht, ehe ich Dir das Gewünſchte\pwindex{?? Werk@Nicht ermittelte Verfasserinnen und Verfasser!Agenda medicalNone@\emph{Agenda médical} {[}None{]}|pwv} verſchaffen gekonnt. Ich ſende Dir anbei das \label{K_L02719-1v}\edtext{»\textsc{Agenda médical\pwindex{?? Werk@Nicht ermittelte Verfasserinnen und Verfasser!Agenda medicalNone@\emph{Agenda médical} {[}None{]}|pw}}«}{\lemma{\textnormal{\emph{»Agenda médical«}}}\Cendnote{\textnormal{Die \emph{Agenda médical}\pwindex{?? Werk@Nicht ermittelte Verfasserinnen und Verfasser!Agenda medicalNone@\emph{Agenda médical} {[}None{]}|pwk} erschien jährlich und listete unter anderem
                     franz\oindex{Frankreich@\textbf{Frankreich}|pwkv}ösische Mediziner.
                     Goldmann\pwindex{Goldmann, Paul 31.01.1865 – 25.09.1935@\textsc{Goldmann, Paul} (31.01.1865 – 25.09.1935), \emph{Schriftsteller, Journalist}|pwk} sandte Schnitzler\pwindex{Schnitzler, Arthur 15.05.1862 – 21.10.1931@\textsc{Schnitzler, Arthur} (15.05.1862 – 21.10.1931), \emph{Schriftsteller, Mediziner}|pwk} vermutlich die neueste Ausgabe\pwindex{?? Werk@Nicht ermittelte Verfasserinnen und Verfasser!Agenda medicalNone@\emph{Agenda médical} {[}None{]}|pwkv} für das Jahr 1894. Es ist unklar, wofür Schnitzler\pwindex{Schnitzler, Arthur 15.05.1862 – 21.10.1931@\textsc{Schnitzler, Arthur} (15.05.1862 – 21.10.1931), \emph{Schriftsteller, Mediziner}|pwk} die Namen der Professoren brauchte.}}}\label{K_L02719-1h}. Auf S. 381 findeſt
               Du die Namen derjenigen Profeſſoren unterſtrichen, die mir als die bedeutendſten
               bezeichnet worden; ihre Adreſſen ſind in dem S. 299 beginnenden {\pb}Verzeichniß\pwindex{?? Werk@Nicht ermittelte Verfasserinnen und Verfasser!Agenda medicalNone@\emph{Agenda médical} {[}None{]}|pwv} enthalten. Wenn Du
               nun Weiteres brauchſt, für dieſe ſowie für alle zukünftigen Angelegenheiten – wenn
               Gänge zu machen oder Briefe auszutragen ſind \textsc{etc.} – ſo
               ſchreibe mir ſtets. Insbeſondere den mechaniſchen Theil eventueller journaliſtiſcher
               Maßnahmen kann ich Dir leicht beſtreiten helfen, da ich hier einen Büreaudiener habe.
               Aber auch ſonſt betrachte mich als Deinen \label{K_L02719-2v}\edtext{\textsc{\begin{otherlanguage}{french}ministre plénipotentiaire\end{otherlanguage}}}{\lemma{\textnormal{\emph{ministre plénipotentiaire}}}\Cendnote{\textnormal{französisch: Gesandter}}}\label{K_L02719-2h} und gib’
               mir etwas zu {\pb}arbeiten. Freilich verlange ich einen
               Gegendienſt. Das iſt gemein, aber ich kann nicht anders. Schon während unſeres
                  \label{K_L02719-3v}\edtext{letzten Beiſammenſeins}{\lemma{\textnormal{\emph{letzten Beiſammenſeins}}}\Cendnote{\textnormal{am 18. 9. 1893 in Salzburg\oindex{Salzburg@\textbf{Salzburg}|pwk}}}}\label{K_L02719-3h} hatte ich die Bitte auf der Zunge, aber es erſchien mir doch gar zu
               erbärmlich, Dir damit zu kommen. Alſo ſchriftlich: Wäre Dir möglich, wenigſtens ein
               paar Monate lang, meinem Schwager\pwindex{Rosengart, Josef 1860-02-08 – 1927-08-04@\textsc{Rosengart, Josef} (1860-02-08 – 1927-08-04), \emph{Arzt}|pwv} ein \label{K_L02719-4v}\edtext{Freiexemplar\pwindex{Internationale klinische Rundschau1887-01-01 – 1922@\emph{Internationale klinische Rundschau} {[}1887-01-01 – 1922{]}|pwv}}{\lemma{\textnormal{\emph{Freiexemplar}}}\Cendnote{\textnormal{der \emph{Internationalen Klinischen Rundschau}\pwindex{Internationale klinische Rundschau1887-01-01 – 1922@\emph{Internationale klinische Rundschau} {[}1887-01-01 – 1922{]}|pwk}, die bis September 1894 von Schnitzler\pwindex{Schnitzler, Arthur 15.05.1862 – 21.10.1931@\textsc{Schnitzler, Arthur} (15.05.1862 – 21.10.1931), \emph{Schriftsteller, Mediziner}|pwk}
                  herausgegeben wurde (vgl. Paul Goldmann an Arthur Schnitzler, 29. 5. [1894])
               }}}\label{K_L02719-4h} zu bewilligen. Seine Praxi\substVorne{}\textsuperscript{ſ}\substDazwischen{}s\substHinten{} geht noch nicht {\pb}gut genug, ihm ein Abonnement\pwindex{Internationale klinische Rundschau1887-01-01 – 1922@\emph{Internationale klinische Rundschau} {[}1887-01-01 – 1922{]}|pwv} zu erlauben.
               Anderſeits möchte\strikeout{\textcolor{gray}{n}} er gar zu gern\strikeout{,} das Blatt\pwindex{Internationale klinische Rundschau1887-01-01 – 1922@\emph{Internationale klinische Rundschau} {[}1887-01-01 – 1922{]}|pwv} leſen. Und da durch einen glücklichen
                  Zufall {\dotsfour} Ich bitte Dich alſo um Gewährung meiner Bitte,
               indem ich zugleich gegen die von mir begangene ſchamloſe Ausbeutung proteſtire.
               Adreſſe: \textsc{Dr. Josef Rosengart\pwindex{Rosengart, Josef 1860-02-08 – 1927-08-04@\textsc{Rosengart, Josef} (1860-02-08 – 1927-08-04), \emph{Arzt}|pw}}, \textsc{Frankfurt \textsuperscript{a}/M, \strikeout{Rossmark\oindex{Rossmarkt@\textbf{Roßmarkt}|pwv}}\oindex{Frankfurt am Main@\textbf{Frankfurt am Main}|pw}{ }Rossmarkt 20\oindex{Rossmarkt@\textbf{Roßmarkt}|pw}}.\pend
           \pstart
           Es iſt viel Erfreuliches in Deinem lieben Briefe. Vor allen Dingen bin ich {\pb}von Herzen froh, daß es endlich mit der Aufführung\pwindex{Schnitzler, Arthur 15.05.1862 – 21.10.1931@\textsc{Schnitzler, Arthur} (15.05.1862 – 21.10.1931), \emph{Schriftsteller, Mediziner}!Maerchen. Schauspiel in drei Aufzuegen1893-12-01@\strich\emph{Das Märchen. Schauspiel in drei Aufzügen} {[}1893-12-01{]}|pwv} ernſt wird. Da ich
               ſo gar nichts hörte, glaubte ich, es ſei wieder eine Verſchiebung eingetreten.
               Nochmals: ſobald die Aufführung\pwindex{Schnitzler, Arthur 15.05.1862 – 21.10.1931@\textsc{Schnitzler, Arthur} (15.05.1862 – 21.10.1931), \emph{Schriftsteller, Mediziner}!Maerchen. Schauspiel in drei Aufzuegen1893-12-01@\strich\emph{Das Märchen. Schauspiel in drei Aufzügen} {[}1893-12-01{]}|pwv} feſtgeſetzt iſt, theile mir das \uline{umgehend} mit. Und reg’ Dich nicht auf wenn die Komödiantenbande\orgindex{Volkstheater@Volkstheater|pwv}, der Gewohnheit gemäß,
               Dich kränken ſollte. Ich hätte ſo gern genaue Details über die Proben\pwindex{Schnitzler, Arthur 15.05.1862 – 21.10.1931@\textsc{Schnitzler, Arthur} (15.05.1862 – 21.10.1931), \emph{Schriftsteller, Mediziner}!Maerchen. Schauspiel in drei Aufzuegen1893-12-01@\strich\emph{Das Märchen. Schauspiel in drei Aufzügen} {[}1893-12-01{]}|pwv} gewußt, ich bin auch überzeugt, daß
               Du bei unſerem nächſten {\pb}Beiſammenſein behaupten
               wirſt, ſie mir geſchrieben zu haben. Damit werde ich mich wohl begnügen müſſen. \strikeout{Sehr} Laß’ mich wenigſtens bald etwas über den Fortgang
               der Affaire\pwindex{Schnitzler, Arthur 15.05.1862 – 21.10.1931@\textsc{Schnitzler, Arthur} (15.05.1862 – 21.10.1931), \emph{Schriftsteller, Mediziner}!Maerchen. Schauspiel in drei Aufzuegen1893-12-01@\strich\emph{Das Märchen. Schauspiel in drei Aufzügen} {[}1893-12-01{]}|pwv} wiſſen, – ja? Und
               ſtärkt \substVorne{}\textsuperscript{d}\substDazwischen{}D\substHinten{}ir das nicht richtig die Productionsluſt, dieſe endliche Verwirklichung des
               ſo lange Erhofften?\pend
           \pstart
           Ich habe den »\textsc{Anatol\pwindex{Schnitzler, Arthur 15.05.1862 – 21.10.1931@\textsc{Schnitzler, Arthur} (15.05.1862 – 21.10.1931), \emph{Schriftsteller, Mediziner}!Anatol1892-10-29@\strich\emph{Anatol} {[}1892-10-29{]}|pw}}« und das »Märchen\pwindex{Schnitzler, Arthur 15.05.1862 – 21.10.1931@\textsc{Schnitzler, Arthur} (15.05.1862 – 21.10.1931), \emph{Schriftsteller, Mediziner}!Maerchen. Schauspiel in drei Aufzuegen1893-12-01@\strich\emph{Das Märchen. Schauspiel in drei Aufzügen} {[}1893-12-01{]}|pw}« hier dem neubegründeten
                  Freien Theater für ausländiſche
                  Kunſt\orgindex{Theâtre de l Œuvre@Théâtre de l’Œuvre|pwv}, dem »\textsc{Oeuvre\orgindex{Theâtre de l Œuvre@Théâtre de l’Œuvre|pw}}« eingereicht. {\pb}Die \label{K_L02719-8v}\edtext{Herren\pwindex{Lugne-Poe, Aurelien-Marie 1869-12-27 – 1940-06-19@\textsc{Lugné-Poe, Aurélien-Marie} (1869-12-27 – 1940-06-19), \emph{Theaterleiter, Regisseur, Schauspieler}|pwv}}{\lemma{\textnormal{\emph{Herren}}}\Cendnote{\textnormal{Es ist nicht letztgültig zu klären, wen
                     Goldmann\pwindex{Goldmann, Paul 31.01.1865 – 25.09.1935@\textsc{Goldmann, Paul} (31.01.1865 – 25.09.1935), \emph{Schriftsteller, Journalist}|pwk} hiermit meinte. Geleitet wurde
                  das \emph{Théâtre de l’Œuvre}\orgindex{Theâtre de l Œuvre@Théâtre de l’Œuvre|pwk} zu dieser Zeit
                  jedenfalls von Aurélien-Marie Lugné-Poe\pwindex{Lugne-Poe, Aurelien-Marie 1869-12-27 – 1940-06-19@\textsc{Lugné-Poe, Aurélien-Marie} (1869-12-27 – 1940-06-19), \emph{Theaterleiter, Regisseur, Schauspieler}|pwk}.
                  Auch in späteren Jahren spielte das \emph{Théâtre de
                     l’Œuvre}\orgindex{Theâtre de l Œuvre@Théâtre de l’Œuvre|pwk} für Schnitzler\pwindex{Schnitzler, Arthur 15.05.1862 – 21.10.1931@\textsc{Schnitzler, Arthur} (15.05.1862 – 21.10.1931), \emph{Schriftsteller, Mediziner}|pwk} eine Rolle.
                  So empfahl etwa Marcel Schulz\pwindex{Schulz, Marcel *~1887?@\textsc{Schulz, Marcel} (*~1887?), \emph{Schriftsteller}|pwk}{ }Lugné-Poe\pwindex{Lugne-Poe, Aurelien-Marie 1869-12-27 – 1940-06-19@\textsc{Lugné-Poe, Aurélien-Marie} (1869-12-27 – 1940-06-19), \emph{Theaterleiter, Regisseur, Schauspieler}|pwk} den \emph{Schleier der Beatrice}\pwindex{Schnitzler, Arthur 15.05.1862 – 21.10.1931@\textsc{Schnitzler, Arthur} (15.05.1862 – 21.10.1931), \emph{Schriftsteller, Mediziner}!Schleier der Beatrice. Schauspiel in fuenf Akten1900-12-01@\strich\emph{Der Schleier der Beatrice. Schauspiel in fünf Akten} {[}1900-12-01{]}|pwk} (vgl. A. S.: \emph{Tagebuch}, 29. 1. 1907) und auch Paul Zifferer\pwindex{Zifferer, Paul 09.03.1879 – 14.02.1929@\textsc{Zifferer, Paul} (09.03.1879 – 14.02.1929), \emph{Schriftsteller, Journalist}|pwk} legte Schnitzler\pwindex{Schnitzler, Arthur 15.05.1862 – 21.10.1931@\textsc{Schnitzler, Arthur} (15.05.1862 – 21.10.1931), \emph{Schriftsteller, Mediziner}|pwk} das
                     \emph{Théâtre de l’Œuvre}\orgindex{Theâtre de l Œuvre@Théâtre de l’Œuvre|pwk} »wegen [s]einer
                     Stücke für Paris\oindex{Paris@\textbf{Paris}|pw}« nahe (A. S.: \emph{Tagebuch}, 6. 5. 1927). 1912 und 1922 inszenierte das \emph{Théâtre de l’Œuvre}\orgindex{Theâtre de l Œuvre@Théâtre de l’Œuvre|pwk} den Einakter \emph{Die letzten Masken}\pwindex{Schnitzler, Arthur 15.05.1862 – 21.10.1931@\textsc{Schnitzler, Arthur} (15.05.1862 – 21.10.1931), \emph{Schriftsteller, Mediziner}!letzten Masken1901@\strich\emph{Die letzten Masken} {[}1901{]}|pwk} (Les Derniers masques\pwindex{Schnitzler, Arthur 15.05.1862 – 21.10.1931@\textsc{Schnitzler, Arthur} (15.05.1862 – 21.10.1931), \emph{Schriftsteller, Mediziner}!letzten Masken1901@\strich\emph{Die letzten Masken} {[}1901{]}|pwkv}).}}}\label{K_L02719-8h} waren ſehr vergnügt über
               mein ihnen gewidmetes \label{K_L02719-7v}\edtext{Feuilleton\pwindex{Goldmann, Paul 31.01.1865 – 25.09.1935@\textsc{Goldmann, Paul} (31.01.1865 – 25.09.1935), \emph{Schriftsteller, Journalist}!Pariser Theater1893-10-11@\strich\emph{Pariser Theater} {[}1893-10-11{]}|pw}}{\lemma{\textnormal{\emph{Feuilleton}}}\Cendnote{\textnormal{Paul Goldmann\pwindex{Goldmann, Paul 31.01.1865 – 25.09.1935@\textsc{Goldmann, Paul} (31.01.1865 – 25.09.1935), \emph{Schriftsteller, Journalist}|pwk}: \emph{Pariser Theater}\pwindex{Goldmann, Paul 31.01.1865 – 25.09.1935@\textsc{Goldmann, Paul} (31.01.1865 – 25.09.1935), \emph{Schriftsteller, Journalist}!Pariser Theater1893-10-11@\strich\emph{Pariser Theater} {[}1893-10-11{]}|pwk}. In: \emph{Frankfurter Zeitung}\pwindex{?? Werk@Nicht ermittelte Verfasserinnen und Verfasser!Frankfurter Zeitung1856 – 1943@\emph{Frankfurter Zeitung} {[}1856 – 1943{]}|pwk}, Jg. 38, Nr. 282, 11. 10. 1893, Erstes Morgenblatt, S. 1–2.}}}\label{K_L02719-7h}, und da ich
               nicht gern \strikeout{auf} die Gelegenheit zum Verlangen von
               Gegendienſten vorübergehen laſſe (ſiehe oben), ſo bat ich ſie, Deine Stücke\pwindex{Schnitzler, Arthur 15.05.1862 – 21.10.1931@\textsc{Schnitzler, Arthur} (15.05.1862 – 21.10.1931), \emph{Schriftsteller, Mediziner}!Anatol1892-10-29@\strich\emph{Anatol} {[}1892-10-29{]}|pwv}\pwindex{Schnitzler, Arthur 15.05.1862 – 21.10.1931@\textsc{Schnitzler, Arthur} (15.05.1862 – 21.10.1931), \emph{Schriftsteller, Mediziner}!Maerchen. Schauspiel in drei Aufzuegen1893-12-01@\strich\emph{Das Märchen. Schauspiel in drei Aufzügen} {[}1893-12-01{]}|pwv} zu leſen. Es ſind nämlich
               Leute darin, die deutſch können. Mach’ Dir aber keine allzu großen Hoffnungen. \strikeout{D Sie} Sie frugen mich nämlich, ob die Stücke\pwindex{Schnitzler, Arthur 15.05.1862 – 21.10.1931@\textsc{Schnitzler, Arthur} (15.05.1862 – 21.10.1931), \emph{Schriftsteller, Mediziner}!Anatol1892-10-29@\strich\emph{Anatol} {[}1892-10-29{]}|pwv}\pwindex{Schnitzler, Arthur 15.05.1862 – 21.10.1931@\textsc{Schnitzler, Arthur} (15.05.1862 – 21.10.1931), \emph{Schriftsteller, Mediziner}!Maerchen. Schauspiel in drei Aufzuegen1893-12-01@\strich\emph{Das Märchen. Schauspiel in drei Aufzügen} {[}1893-12-01{]}|pwv} »myſtiſch« ſeien? Ich
               wußte nicht recht, was {\pb}ich ſagen ſollte: Bitte,
               ſind ſie myſtiſch?\pend
           \pstart
           Übrigens habe ich noch andere Eiſen für \substVorne{}\textsuperscript{d}\substDazwischen{}D\substHinten{}ich hier im Feuer. Doch davon ſpäter.\pend
           \pstart
           Das Blühen in der lieben Wien\oindex{Wien@\textbf{Wien}|pw}er Künſtler-Laube –
               oh verdammt, welch’ ein Gleichniß! – beobachte ich mit wehmüthiger Freude. Gewiß, ich
               weiß, daß Eure\pwindex{Beer-Hofmann, Richard 1866-07-11 – 1945-09-26@\textsc{Beer-Hofmann, Richard} (1866-07-11 – 1945-09-26), \emph{Schriftsteller}|pwv}\pwindex{Hofmannsthal, Hugo von 1874-02-01 – 1929-07-15@\textsc{Hofmannsthal, Hugo von} (1874-02-01 – 1929-07-15), \emph{Schriftsteller}|pwv} drei Namen weit klingen werden, und in nicht langer Zeit. Ich ſehe, wie
                  Ihr\pwindex{Beer-Hofmann, Richard 1866-07-11 – 1945-09-26@\textsc{Beer-Hofmann, Richard} (1866-07-11 – 1945-09-26), \emph{Schriftsteller}|pwv}\pwindex{Hofmannsthal, Hugo von 1874-02-01 – 1929-07-15@\textsc{Hofmannsthal, Hugo von} (1874-02-01 – 1929-07-15), \emph{Schriftsteller}|pwv} formt und ſchafft, und wünſche allen Segen {\pb}auf dieſes Schaffen herab. Und dann kehre ich in
               mich ein und habe das traurige Gefühl des Mannes, der einſam und ſchwach auf einem
               Stein ſitzen geblieben iſt und nur noch die fernen Stimmen der Begleiter hört, die
               durch den Wald hallen: aber ſie ſind weit und er wird ihnen nimmer nachkommen. Meine
               Arbeiten? Gewiß weiß ichs nicht, wenn ich etwas Gutes ſchreibe. Und wenn ich es
               wüßte: Hat das einen Werth, was ich thue? Geh’, das mußt {\pb}Du mir ſelbſt zugeben, daß ich in unſerem Kreiſe
               bereits immer deutlicher die bitterböſe Rolle übernehme »des Mannes, aus dem etwas
               hätte werden können«.\pend
           \pstart
           Ich bitte Dich inſtändig: veranlaſſe \textsc{Loris\pwindex{Hofmannsthal, Hugo von 1874-02-01 – 1929-07-15@\textsc{Hofmannsthal, Hugo von} (1874-02-01 – 1929-07-15), \emph{Schriftsteller}|pw}} und \textsc{Richard\pwindex{Beer-Hofmann, Richard 1866-07-11 – 1945-09-26@\textsc{Beer-Hofmann, Richard} (1866-07-11 – 1945-09-26), \emph{Schriftsteller}|pw}}, daß ſie mir die erſchienen{[}en{]} oder zu erſcheinenden
                  \label{K_L02719-10v}\edtext{Sachen\pwindex{Beer-Hofmann, Richard 1866-07-11 – 1945-09-26@\textsc{Beer-Hofmann, Richard} (1866-07-11 – 1945-09-26), \emph{Schriftsteller}!Novellen1. 12. 1893@\strich\emph{Novellen} {[}1. 12. 1893{]}|pwv}}{\lemma{\textnormal{\emph{Sachen}}}\Cendnote{\textnormal{Die einzige selbstständige
                  Veröffentlichung – Goldmann\pwindex{Goldmann, Paul 31.01.1865 – 25.09.1935@\textsc{Goldmann, Paul} (31.01.1865 – 25.09.1935), \emph{Schriftsteller, Journalist}|pwk} bezieht sich
                  auf »Bücher\pwindex{Beer-Hofmann, Richard 1866-07-11 – 1945-09-26@\textsc{Beer-Hofmann, Richard} (1866-07-11 – 1945-09-26), \emph{Schriftsteller}!Novellen1. 12. 1893@\strich\emph{Novellen} {[}1. 12. 1893{]}|pw}« – aus dieser Zeit stellt eine Novellensammlung\pwindex{Beer-Hofmann, Richard 1866-07-11 – 1945-09-26@\textsc{Beer-Hofmann, Richard} (1866-07-11 – 1945-09-26), \emph{Schriftsteller}!Novellen1. 12. 1893@\strich\emph{Novellen} {[}1. 12. 1893{]}|pwkv}{ }Richard Beer-Hofmann\pwindex{Beer-Hofmann, Richard 1866-07-11 – 1945-09-26@\textsc{Beer-Hofmann, Richard} (1866-07-11 – 1945-09-26), \emph{Schriftsteller}|pwk}s dar, doch erschien
                  diese erst im Dezember 1893. Richard Beer-Hofmann\pwindex{Beer-Hofmann, Richard 1866-07-11 – 1945-09-26@\textsc{Beer-Hofmann, Richard} (1866-07-11 – 1945-09-26), \emph{Schriftsteller}|pwk}: \emph{Novellen}\pwindex{Beer-Hofmann, Richard 1866-07-11 – 1945-09-26@\textsc{Beer-Hofmann, Richard} (1866-07-11 – 1945-09-26), \emph{Schriftsteller}!Novellen1. 12. 1893@\strich\emph{Novellen} {[}1. 12. 1893{]}|pwk}. Berlin: \emph{Freund {\kaufmannsund} Jeckel}{ }1893.}}}\label{K_L02719-10h} ſchicken. Ohne Briefe: ich weiß, daß die Briefe nach ſo langer
               Zeit ſchwer zu ſchreiben ſind. Die gewiſſe Furcht vor der Einleitung. Ich {\pb}möchte deßwegen aber nicht um die Bücher\pwindex{Beer-Hofmann, Richard 1866-07-11 – 1945-09-26@\textsc{Beer-Hofmann, Richard} (1866-07-11 – 1945-09-26), \emph{Schriftsteller}!Novellen1. 12. 1893@\strich\emph{Novellen} {[}1. 12. 1893{]}|pwv} kommen.\pend
           \pstart
           Wenn Du kannſt, ſo ſchick’ mir, bitte, gelegentlich noch einen »\textsc{Anatol\pwindex{Schnitzler, Arthur 15.05.1862 – 21.10.1931@\textsc{Schnitzler, Arthur} (15.05.1862 – 21.10.1931), \emph{Schriftsteller, Mediziner}!Anatol1892-10-29@\strich\emph{Anatol} {[}1892-10-29{]}|pw}}« – zu Propaganda-Zwecken.\pend
           \pstart
           \textsc{Bahr\pwindex{Bahr, Hermann 19.07.1863 – 15.01.1934@\textsc{Bahr, Hermann} (19.07.1863 – 15.01.1934), \emph{Schriftsteller, Kritiker}|pw}}: Du haſt eine ſo merkwürdige Art, gegen Leute gerecht ſein zu wollen, die ſich
               ſchurkiſch gegen Dich benehmen. Nein, – der Mann\pwindex{Bahr, Hermann 19.07.1863 – 15.01.1934@\textsc{Bahr, Hermann} (19.07.1863 – 15.01.1934), \emph{Schriftsteller, Kritiker}|pwv} iſt für mich kein großes Talent, ſelbſt wenn er es ſein
               ſollte. Ungerechte {\pb}Beurtheilung iſt bereits eine
               halbe Befriedigung des Haſſes. Und ſeit der hundsföttischen Kritik\pwindex{Bahr, Hermann 19.07.1863 – 15.01.1934@\textsc{Bahr, Hermann} (19.07.1863 – 15.01.1934), \emph{Schriftsteller, Kritiker}!junge Oesterreich1893-09-20 – 1893-10-07@\strich\emph{Das junge Österreich} {[}1893-09-20 – 1893-10-07{]}|pwv} über Dich haſſe ich den Kerl\pwindex{Bahr, Hermann 19.07.1863 – 15.01.1934@\textsc{Bahr, Hermann} (19.07.1863 – 15.01.1934), \emph{Schriftsteller, Kritiker}|pwv} mehr als je.\pend
           \pstart
           Der \label{K_L02719-12v}\edtext{Briefkaſten-Diebſtahl}{\lemma{\textnormal{\emph{Briefkaſten-Diebſtahl}}}\Cendnote{\textnormal{In \emph{Ridicula}\pwindex{Sosnosky, Theodor von 04.01.1866 – 03.02.1943@\textsc{Sosnosky, Theodor von} (04.01.1866 – 03.02.1943), \emph{Schriftsteller}!Ridicula1893@\strich\emph{Ridicula} {[}1893{]}|pwk} versammelte Theodor von
                     Sosnosky\pwindex{Sosnosky, Theodor von 04.01.1866 – 03.02.1943@\textsc{Sosnosky, Theodor von} (04.01.1866 – 03.02.1943), \emph{Schriftsteller}|pwk} vermeintliche »literarische Lächerlichkeiten« (Breslau:{ }\emph{Trewendt}\orgindex{Eduard Trewendt@Eduard Trewendt|pwk}{ }1894 [von 1893 vordatiert]). Im Kapitel
                  »Briefkastenpoesie« wurden – ohne Erlaubnis – 50 Seiten aus dem Briefkasten\pwindex{?? Werk@Nicht ermittelte Verfasserinnen und Verfasser!der schoenen blauen Donau1886 – 1896@\emph{An der schönen blauen Donau} {[}1886 – 1896{]}|pwkv} der \emph{Schönen blauen Donau}\pwindex{?? Werk@Nicht ermittelte Verfasserinnen und Verfasser!der schoenen blauen Donau1886 – 1896@\emph{An der schönen blauen Donau} {[}1886 – 1896{]}|pwk} aufgenommen. Vgl. h. k.\pwindex{k., h. @\textsc{k., h.}, \emph{Journalist/Journalistin}|pwk}: \emph{Neue Bücher}\pwindex{Neue Buecher1893-12-01@\emph{Neue Bücher} {[}1893-12-01{]}|pwk}. In: \emph{An der schönen blauen
                        Donau}\pwindex{?? Werk@Nicht ermittelte Verfasserinnen und Verfasser!der schoenen blauen Donau1886 – 1896@\emph{An der schönen blauen Donau} {[}1886 – 1896{]}|pwk}, Jg. 8, Nr. 23, 1. 12. 1893,
                     S. 552.}}}\label{K_L02719-12h} des \textsc{Sosnosky\pwindex{Sosnosky, Theodor von 04.01.1866 – 03.02.1943@\textsc{Sosnosky, Theodor von} (04.01.1866 – 03.02.1943), \emph{Schriftsteller}|pw}} iſt ſcheußlich. Ich habe mit meinem Onkel\pwindex{Mamroth, Fedor 21.02.1851 – 25.06.1907@\textsc{Mamroth, Fedor} (21.02.1851 – 25.06.1907), \emph{Journalist, Kritiker}|pwv} berathen, aber ich glaube, wir können nichts machen\substVorne{}\textsuperscript{.}\substDazwischen{},\substHinten{} geſetzlich. Höchſtens eine Züchtigung im Blatte\pwindex{?? Werk@Nicht ermittelte Verfasserinnen und Verfasser!der schoenen blauen Donau1886 – 1896@\emph{An der schönen blauen Donau} {[}1886 – 1896{]}|pwv}, die aber auch eine Reklame für das Buch\pwindex{Sosnosky, Theodor von 04.01.1866 – 03.02.1943@\textsc{Sosnosky, Theodor von} (04.01.1866 – 03.02.1943), \emph{Schriftsteller}!Ridicula1893@\strich\emph{Ridicula} {[}1893{]}|pwv} des Gauner\pwindex{Sosnosky, Theodor von 04.01.1866 – 03.02.1943@\textsc{Sosnosky, Theodor von} (04.01.1866 – 03.02.1943), \emph{Schriftsteller}|pwv}s wäre.\pend
           \pstart
           {\pb}\textsc{Herzl\pwindex{Herzl, Theodor 1860-05-02 – 1904-07-03@\textsc{Herzl, Theodor} (1860-05-02 – 1904-07-03), \emph{Schriftsteller, Journalist}|pw}} iſt ſeit einigen Wochen ſehr \label{K_L02719-13v}\edtext{krank}{\lemma{\textnormal{\emph{krank}}}\Cendnote{\textnormal{Von seiner Malariainfektion
                  berichtete Theodor Herzl\pwindex{Herzl, Theodor 1860-05-02 – 1904-07-03@\textsc{Herzl, Theodor} (1860-05-02 – 1904-07-03), \emph{Schriftsteller, Journalist}|pwk} am 8. 12. 1893 in einem Brief an Schnitzler\pwindex{Schnitzler, Arthur 15.05.1862 – 21.10.1931@\textsc{Schnitzler, Arthur} (15.05.1862 – 21.10.1931), \emph{Schriftsteller, Mediziner}|pwk}. Vgl. Theodor Herzl\pwindex{Herzl, Theodor 1860-05-02 – 1904-07-03@\textsc{Herzl, Theodor} (1860-05-02 – 1904-07-03), \emph{Schriftsteller, Journalist}|pwk}: \emph{Briefe und Tagebücher}. Hg.
                     Alex Bein, Hermann Greive, Moshe Schaerf und Julius H. Schoeps. Bd. 1: \emph{Briefe und autobiographische Notizen. 1866–1895}.
                     Bearbeitet von Johannes Wachten. In Zusammenarbeit mit Chaya Harel, Daisy Tycho
                     und Manfred Winkler. Berlin, Frankfurt am
                        Main, Wien: \emph{Ullstein}\orgindex{Ullstein Verlag@Ullstein Verlag|pwk}/\emph{Propyläen}\orgindex{Propylaeen Verlag@Propyläen Verlag|pwk}{ }1983, S. 545.}}}\label{K_L02719-13h}: \textsc{Malaria} oder ſo etwas.\pend
           \pstart
           Was Neues in Wien\oindex{Wien@\textbf{Wien}|pw}? Bitte ſchreibe bald.\pend
           \pstart
           Auch ein perſönliches Wort: Geſundheit, Production, materielle Fragen.\pend
           \pstart
           Mir geht es ſchlecht, oh ſo ſchlecht!\pend
           \pstart
           Viele treue Grüße!\pend
           \pstart
           Dein {\\[\baselineskip]}\spacefill\mbox{Paul Goldmnn}\pend
           \leftskip=0em{}
         
         \endnumbering\mylabel{h}\end{ledgroupsized}  \newcommand{\dateiname}{L02719}\newcommand{\titel}{Paul Goldmann an Arthur Schnitzler, 4. 11. [1893]}\newcommand{\editorInnen}{Martin Anton Müller und Laura Untner}%% latex-leseansicht-abspann.tex
%% Abspann für die Leseansicht.
%% Der Schalter \ifkorrekturansicht ist bereits durch den Vorspann gesetzt.

%% latex-abspann.tex
%% Gemeinsamer Abspann für Korrekturansicht und Leseansicht.
%% Setzt den Schalter \ifkorrekturansicht voraus (gesetzt in den
%% einbindenden Dateien latex-korrekturansicht-abspann.tex bzw.
%% latex-leseansicht-abspann.tex).
%% ---------------------------------------------------------------

\normalsize

% Das esempio-Environment wird nur in der Leseansicht benötigt
\ifkorrekturansicht\else
\newenvironment{esempio}[3]%
{
    \vspace{1.5ex}
    \rlap{\underline{#1}}
    \par
    \setlength{\parindent}{0cm}
    \nopagebreak
    \leftskip=#2cm
    \rightskip=#3cm
}
{
    \par
}
\fi

\doendnotes{C}
\bigskip
\vfill

\clearpage

\footnotesize

\ifkorrekturansicht
  \lohead{\textsc{register}}
\fi

% theindex-Environment neu definieren ohne reledmac
\makeatletter
\renewenvironment{theindex}{%
  \ifkorrekturansicht
    \section*{\indexname}%
  \else
    \subsubsection*{Index der erwähnten Entitäten}%
  \fi
  \setlength{\parindent}{0pt}%
  \setlength{\parskip}{0pt plus 0.3pt}%
  \let\item\@idxitem
}{%
  \ifkorrekturansicht\clearpage\fi
}
\makeatother

\IfFileExists{\jobname-pw.ind}{\input{\jobname-pw.ind}}{}

% Quellenangabe nur in der Leseansicht
\ifkorrekturansicht\else
% Fallback-Definitionen, falls die .tex-Datei \titel etc. nicht gesetzt hat
\providecommand{\titel}{}
\providecommand{\editorInnen}{}
\providecommand{\dateiname}{\jobname}

\vspace{3cm}

\vfill

\footnotesize
\textsc{Quelle}: \titel. Herausgegeben von {\editorInnen}. In: \emph{Arthur Schnitzler: Briefwechsel mit Autorinnen und Autoren}.
 Digitale Edition, https://schnitzler-briefe.acdh.oeaw.ac.at/{\dateiname}.html (Stand \today)
\fi

\end{document}


      