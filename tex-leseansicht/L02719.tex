%% latex-leseansicht-vorspann.tex
%% Vorspann für die Leseansicht.
%% Lädt die gemeinsame Datei latex-vorspann.tex mit nicht gesetztem Schalter.

\newif\ifkorrekturansicht
\korrekturansichtfalse

\input{../tex-inputs/latex-vorspann}


\section[Paul Goldmann an Arthur Schnitzler, 4. 11. {[}1893{]}]{L02719 Paul Goldmann an Arthur Schnitzler, 4. 11. [1893]}
\nopagebreak\mylabel{L02719v}
\rehead{ }\normalsize\beginnumbering\briefempfaengerindex{Schnitzler, Arthur@\textsc{Schnitzler, Arthur}!zzzGoldmann, Paul@\emph{von Paul Goldmann}!1893-11-041@{4. 11. [1893]}|(be}
\toendnotes[C]{\smallbreak\pagebreak[2]}
\correspDesc{Versand  durch Paul Goldmann am 4. 11. [1893] in Paris
\newline{}Erhalt  durch Arthur Schnitzler im Zeitraum [5. 11. 1893
                  – 9. 11. 1893?] in Wien}\toendnotes[C]{\smallbreak}
\Standort{DLA, A:Schnitzler, HS.NZ85.1.3163.}
\physDesc{Brief, 5 Blätter, 13 Seiten, 5058 Zeichen
\newline{}Handschrift: schwarze Tinte, deutsche Kurrent
\newline{}Schnitzler: 1) mit schwarzer Tinte das Jahr »93« vermerkt  2) mit rotem Buntstift zwei Unterstreichungen und fünf vertikale
                                 Markierungen}\toendnotes[C]{\smallbreak}
\pstart
           \raggedleft{}{\pb}\textsc{Paris\oindex{Paris@\textbf{Paris}, \emph{Hauptstadt}|pw}}, 4. November.\pend
           
\pstart{}Mein lieber Freund,\pend\vspace{0.5em}
\pstart
           Du mußt mir nicht böſe{ }ſein: Ich habe hier wenig Beziehungen zur ärztlichen Welt und
               da ich außerdem mit tauſend Dingen die Hände voll zu thun hatte, habe ich eine Woche
               gebraucht, ehe ich Dir das Gewünſchte\pwindex{Agenda médical@\emph{Agenda médical}|pwv} verſchaffen gekonnt. Ich{ }ſende Dir anbei das \label{K_L02719-1v}\edtext{»\textsc{Agenda médical\pwindex{Agenda médical@\emph{Agenda médical}|pw}}«}{\lemma{\textnormal{\emph{»Agenda médical«}}}\Cendnote{\textnormal{Die \emph{Agenda médical}\pwindex{Agenda médical@\emph{Agenda médical}|pwk} erschien jährlich und listete unter anderem
                     franz\oindex{Frankreich@\textbf{Frankreich}|pwkv}ösische Mediziner.
                     Goldmann\pwindex{Goldmann, Paul 31.\,1.\,1865 Breslau – 25.\,9.\,1935 Wien@\textsc{Goldmann, Paul} (31.\,1.\,1865 Breslau – 25.\,9.\,1935 Wien), \emph{Schriftsteller, Journalist}|pwk} sandte Schnitzler vermutlich die neueste Ausgabe\pwindex{Agenda médical@\emph{Agenda médical}|pwkv} für das Jahr 1894. Es ist unklar, wofür Schnitzler die Namen der Professoren brauchte.}}}\label{K_L02719-1}. Auf S. 381 findeſt
               Du die Namen derjenigen Profeſſoren unterſtrichen, die mir als die bedeutendſten
               bezeichnet worden; ihre Adreſſen{ }ſind in dem S. 299 beginnenden {\pb}Verzeichniß\pwindex{Agenda médical@\emph{Agenda médical}|pwv} enthalten. Wenn Du
               nun Weiteres brauchſt, für dieſe{ }ſowie für alle zukünftigen Angelegenheiten – wenn
               Gänge zu machen oder Briefe auszutragen{ }ſind \textsc{etc.} –{ }ſo{ }ſchreibe mir{ }ſtets. Insbeſondere den mechaniſchen Theil eventueller journaliſtiſcher
               Maßnahmen kann ich Dir leicht beſtreiten helfen, da ich hier einen Büreaudiener habe.
               Aber auch{ }ſonſt betrachte mich als Deinen \label{K_L02719-2v}\edtext{\textsc{\begin{otherlanguage}{french}ministre plénipotentiaire\end{otherlanguage}}}{\lemma{\textnormal{\emph{ministre plénipotentiaire}}}\Cendnote{\textnormal{französisch: Gesandter}}}\label{K_L02719-2} und gib’
               mir etwas zu {\pb}arbeiten. Freilich verlange ich einen
               Gegendienſt. Das iſt gemein, aber ich kann nicht anders. Schon während unſeres
                  \label{K_L02719-3v}\edtext{letzten Beiſammenſeins}{\lemma{\textnormal{\emph{letzten Beisammenseins}}}\Cendnote{\textnormal{am 18. 9. 1893 in Salzburg\oindex{Salzburg@\textbf{Salzburg}, \emph{Verwaltungsgebiet}|pwk}}}}\label{K_L02719-3} hatte ich die Bitte auf der Zunge, aber es erſchien mir doch gar zu
               erbärmlich, Dir damit zu kommen. Alſo{ }ſchriftlich: Wäre Dir möglich, wenigſtens ein
               paar Monate lang, meinem Schwager\pwindex{Rosengart, Josef 8.\,2.\,1860 Laupheim – 4.\,8.\,1927 Frankfurt am Main@\textsc{Rosengart, Josef} (8.\,2.\,1860 Laupheim – 4.\,8.\,1927 Frankfurt am Main), \emph{Arzt}|pwv} ein \label{K_L02719-4v}\edtext{Freiexemplar\pwindex{Internationale klinische Rundschau@\emph{Internationale klinische Rundschau}|pwv}}{\lemma{\textnormal{\emph{Freiexemplar}}}\Cendnote{\textnormal{Schnitzler war  bis September 1894 Herausgeber
                  der \emph{Internationalen Klinischen Rundschau}\pwindex{Internationale klinische Rundschau@\emph{Internationale klinische Rundschau}|pwk} (vgl. XXXX Auszeichnungsfehler: Dokument L02621 nicht gefunden).
               }}}\label{K_L02719-4} zu bewilligen. Seine Praxi\substVorne{}\textsuperscript{ſ}\substDazwischen{}s\substHinten{} geht noch nicht {\pb}gut genug, ihm ein Abonnement\pwindex{Internationale klinische Rundschau@\emph{Internationale klinische Rundschau}|pwv} zu erlauben.
               Anderſeits möchte\strikeout{\textcolor{gray}{n}} er gar zu gern\strikeout{,} das Blatt\pwindex{Internationale klinische Rundschau@\emph{Internationale klinische Rundschau}|pwv} leſen. Und da durch einen glücklichen
                  Zufall {\dotsfour} Ich bitte Dich alſo um Gewährung meiner Bitte,
               indem ich zugleich gegen die von mir begangene{ }ſchamloſe Ausbeutung proteſtire.
               Adreſſe: \textsc{Dr. Josef Rosengart\pwindex{Rosengart, Josef 8.\,2.\,1860 Laupheim – 4.\,8.\,1927 Frankfurt am Main@\textsc{Rosengart, Josef} (8.\,2.\,1860 Laupheim – 4.\,8.\,1927 Frankfurt am Main), \emph{Arzt}|pw}}, \textsc{Frankfurt \textsuperscript{a}/M, \strikeout{Rossmark\oindex{Roßmarkt@\textbf{Roßmarkt}, \emph{Straße}|pwv}}\oindex{Frankfurt am Main@\textbf{Frankfurt am Main}, \emph{Hauptstadt}|pw}{ }Rossmarkt 20\oindex{Roßmarkt@\textbf{Roßmarkt}, \emph{Straße}|pw}}.\pend
           
\pstart
           Es iſt viel Erfreuliches in Deinem lieben Briefe. Vor allen Dingen bin ich {\pb}von Herzen froh, daß es endlich mit der Aufführung\pwindex{Schnitzler, Arthur 15.\,5.\,1862 Wien – 21.\,10.\,1931 ebd.@\textsc{Schnitzler, Arthur} (15.\,5.\,1862 Wien – 21.\,10.\,1931 ebd.), \emph{Schriftsteller, Mediziner}!Märchen. Schauspiel in drei Aufzügen@\strich\emph{Das Märchen. Schauspiel in drei Aufzügen}|pwv} ernſt wird. Da ich{ }ſo gar nichts hörte, glaubte ich, es{ }ſei wieder eine Verſchiebung eingetreten.
               Nochmals:{ }ſobald die Aufführung\pwindex{Schnitzler, Arthur 15.\,5.\,1862 Wien – 21.\,10.\,1931 ebd.@\textsc{Schnitzler, Arthur} (15.\,5.\,1862 Wien – 21.\,10.\,1931 ebd.), \emph{Schriftsteller, Mediziner}!Märchen. Schauspiel in drei Aufzügen@\strich\emph{Das Märchen. Schauspiel in drei Aufzügen}|pwv} feſtgeſetzt iſt, theile mir das \uline{umgehend} mit. Und reg’ Dich nicht auf wenn die Komödiantenbande\orgindex{Volkstheater@Volkstheater|pwv}, der Gewohnheit gemäß,
               Dich kränken{ }ſollte. Ich hätte{ }ſo gern genaue Details über die Proben\pwindex{Schnitzler, Arthur 15.\,5.\,1862 Wien – 21.\,10.\,1931 ebd.@\textsc{Schnitzler, Arthur} (15.\,5.\,1862 Wien – 21.\,10.\,1931 ebd.), \emph{Schriftsteller, Mediziner}!Märchen. Schauspiel in drei Aufzügen@\strich\emph{Das Märchen. Schauspiel in drei Aufzügen}|pwv} gewußt, ich bin auch überzeugt, daß
               Du bei unſerem nächſten {\pb}Beiſammenſein behaupten
               wirſt,{ }ſie mir geſchrieben zu haben. Damit werde ich mich wohl begnügen müſſen. \strikeout{Sehr} Laß’ mich wenigſtens bald etwas über den Fortgang
               der Affaire\pwindex{Schnitzler, Arthur 15.\,5.\,1862 Wien – 21.\,10.\,1931 ebd.@\textsc{Schnitzler, Arthur} (15.\,5.\,1862 Wien – 21.\,10.\,1931 ebd.), \emph{Schriftsteller, Mediziner}!Märchen. Schauspiel in drei Aufzügen@\strich\emph{Das Märchen. Schauspiel in drei Aufzügen}|pwv} wiſſen, – ja? Und{ }ſtärkt \substVorne{}\textsuperscript{d}\substDazwischen{}D\substHinten{}ir das nicht richtig die Productionsluſt, dieſe endliche Verwirklichung des{ }ſo lange Erhofften?\pend
           
\pstart
           Ich habe den »\textsc{Anatol\pwindex{Schnitzler, Arthur 15.\,5.\,1862 Wien – 21.\,10.\,1931 ebd.@\textsc{Schnitzler, Arthur} (15.\,5.\,1862 Wien – 21.\,10.\,1931 ebd.), \emph{Schriftsteller, Mediziner}!Anatol@\strich\emph{Anatol}|pw}}« und das »Märchen\pwindex{Schnitzler, Arthur 15.\,5.\,1862 Wien – 21.\,10.\,1931 ebd.@\textsc{Schnitzler, Arthur} (15.\,5.\,1862 Wien – 21.\,10.\,1931 ebd.), \emph{Schriftsteller, Mediziner}!Märchen. Schauspiel in drei Aufzügen@\strich\emph{Das Märchen. Schauspiel in drei Aufzügen}|pw}« hier dem neubegründeten
                  Freien Theater für ausländiſche
                  Kunſt\orgindex{Théâtre de l’Œuvre@Théâtre de l’Œuvre|pwv}, dem »\textsc{Oeuvre\orgindex{Théâtre de l’Œuvre@Théâtre de l’Œuvre|pw}}« eingereicht. {\pb}Die \label{K_L02719-5v}\edtext{Herren\pwindex{Lugné-Poe, Aurélien-Marie 27.\,12.\,1869 Paris – 19.\,6.\,1940 Villeneuve-les-Avignon@\textsc{Lugné-Poe, Aurélien-Marie} (27.\,12.\,1869 Paris – 19.\,6.\,1940 Villeneuve-les-Avignon), \emph{Theaterleiter, Regisseur, Schauspieler}|pwv}}{\lemma{\textnormal{\emph{Herren}}}\Cendnote{\textnormal{Es ist nicht letztgültig zu klären, wen
                     Goldmann\pwindex{Goldmann, Paul 31.\,1.\,1865 Breslau – 25.\,9.\,1935 Wien@\textsc{Goldmann, Paul} (31.\,1.\,1865 Breslau – 25.\,9.\,1935 Wien), \emph{Schriftsteller, Journalist}|pwk} hiermit meinte. Geleitet wurde
                  das \emph{Théâtre de l’Œuvre}\orgindex{Théâtre de l’Œuvre@Théâtre de l’Œuvre|pwk} zu dieser Zeit
                  jedenfalls von Aurélien-Marie Lugné-Poe\pwindex{Lugné-Poe, Aurélien-Marie 27.\,12.\,1869 Paris – 19.\,6.\,1940 Villeneuve-les-Avignon@\textsc{Lugné-Poe, Aurélien-Marie} (27.\,12.\,1869 Paris – 19.\,6.\,1940 Villeneuve-les-Avignon), \emph{Theaterleiter, Regisseur, Schauspieler}|pwk}.
                  Auch in späteren Jahren spielte das \emph{Théâtre de
                     l’Œuvre}\orgindex{Théâtre de l’Œuvre@Théâtre de l’Œuvre|pwk} für Schnitzler eine Rolle.
                  So empfahl etwa Marcel Schulz\pwindex{Schulz, Marcel *~1887?@\textsc{Schulz, Marcel} (*~1887?), \emph{Schriftsteller}|pwk}{ }Lugné-Poe\pwindex{Lugné-Poe, Aurélien-Marie 27.\,12.\,1869 Paris – 19.\,6.\,1940 Villeneuve-les-Avignon@\textsc{Lugné-Poe, Aurélien-Marie} (27.\,12.\,1869 Paris – 19.\,6.\,1940 Villeneuve-les-Avignon), \emph{Theaterleiter, Regisseur, Schauspieler}|pwk} den \emph{Schleier der Beatrice}\pwindex{Schnitzler, Arthur 15.\,5.\,1862 Wien – 21.\,10.\,1931 ebd.@\textsc{Schnitzler, Arthur} (15.\,5.\,1862 Wien – 21.\,10.\,1931 ebd.), \emph{Schriftsteller, Mediziner}!Schleier der Beatrice. Schauspiel in fünf Akten@\strich\emph{Der Schleier der Beatrice. Schauspiel in fünf Akten}|pwk} (vgl. A. S.: \emph{Tagebuch}, 29. 1. 1907) und auch Paul Zifferer\pwindex{Zifferer, Paul 9.\,3.\,1879 Bystřice pod Hostýnem – 14.\,2.\,1929 Wien@\textsc{Zifferer, Paul} (9.\,3.\,1879 Bystřice pod Hostýnem – 14.\,2.\,1929 Wien), \emph{Schriftsteller, Journalist}|pwk} legte Schnitzler das
                     \emph{Théâtre de l’Œuvre}\orgindex{Théâtre de l’Œuvre@Théâtre de l’Œuvre|pwk} »wegen [s]einer
                     Stücke für Paris\oindex{Paris@\textbf{Paris}, \emph{Hauptstadt}|pw}« nahe (A. S.: \emph{Tagebuch}, 6. 5. 1927). 1912 und 1922 inszenierte das \emph{Théâtre de l’Œuvre}\orgindex{Théâtre de l’Œuvre@Théâtre de l’Œuvre|pwk} den Einakter \emph{Die letzten Masken}\pwindex{Schnitzler, Arthur 15.\,5.\,1862 Wien – 21.\,10.\,1931 ebd.@\textsc{Schnitzler, Arthur} (15.\,5.\,1862 Wien – 21.\,10.\,1931 ebd.), \emph{Schriftsteller, Mediziner}!letzten Masken@\strich\emph{Die letzten Masken}|pwk} (Les Derniers masques\pwindex{Schnitzler, Arthur 15.\,5.\,1862 Wien – 21.\,10.\,1931 ebd.@\textsc{Schnitzler, Arthur} (15.\,5.\,1862 Wien – 21.\,10.\,1931 ebd.), \emph{Schriftsteller, Mediziner}!letzten Masken@\strich\emph{Die letzten Masken}|pwkv}).}}}\label{K_L02719-5} waren{ }ſehr vergnügt über
               mein ihnen gewidmetes \label{K_L02719-6v}\edtext{Feuilleton\pwindex{Goldmann, Paul 31.\,1.\,1865 Breslau – 25.\,9.\,1935 Wien@\textsc{Goldmann, Paul} (31.\,1.\,1865 Breslau – 25.\,9.\,1935 Wien), \emph{Schriftsteller, Journalist}!Pariser Theater@\strich\emph{Pariser Theater}|pw}}{\lemma{\textnormal{\emph{Feuilleton}}}\Cendnote{\textnormal{Paul Goldmann\pwindex{Goldmann, Paul 31.\,1.\,1865 Breslau – 25.\,9.\,1935 Wien@\textsc{Goldmann, Paul} (31.\,1.\,1865 Breslau – 25.\,9.\,1935 Wien), \emph{Schriftsteller, Journalist}|pwk}: \emph{Pariser Theater}\pwindex{Goldmann, Paul 31.\,1.\,1865 Breslau – 25.\,9.\,1935 Wien@\textsc{Goldmann, Paul} (31.\,1.\,1865 Breslau – 25.\,9.\,1935 Wien), \emph{Schriftsteller, Journalist}!Pariser Theater@\strich\emph{Pariser Theater}|pwk}. In: \emph{Frankfurter Zeitung}\pwindex{Frankfurter Zeitung@\emph{Frankfurter Zeitung}|pwk}, Jg. 38, Nr. 282, 11. 10. 1893, Erstes Morgenblatt, S. 1–2.}}}\label{K_L02719-6}, und da ich
               nicht gern \strikeout{auf} die Gelegenheit zum Verlangen von
               Gegendienſten vorübergehen laſſe (ſiehe oben),{ }ſo bat ich{ }ſie, Deine Stücke\pwindex{Schnitzler, Arthur 15.\,5.\,1862 Wien – 21.\,10.\,1931 ebd.@\textsc{Schnitzler, Arthur} (15.\,5.\,1862 Wien – 21.\,10.\,1931 ebd.), \emph{Schriftsteller, Mediziner}!Anatol@\strich\emph{Anatol}|pwv}\pwindex{Schnitzler, Arthur 15.\,5.\,1862 Wien – 21.\,10.\,1931 ebd.@\textsc{Schnitzler, Arthur} (15.\,5.\,1862 Wien – 21.\,10.\,1931 ebd.), \emph{Schriftsteller, Mediziner}!Märchen. Schauspiel in drei Aufzügen@\strich\emph{Das Märchen. Schauspiel in drei Aufzügen}|pwv} zu leſen. Es{ }ſind nämlich
               Leute darin, die deutſch können. Mach’ Dir aber keine allzu großen Hoffnungen. \strikeout{D Sie} Sie frugen mich nämlich, ob die Stücke\pwindex{Schnitzler, Arthur 15.\,5.\,1862 Wien – 21.\,10.\,1931 ebd.@\textsc{Schnitzler, Arthur} (15.\,5.\,1862 Wien – 21.\,10.\,1931 ebd.), \emph{Schriftsteller, Mediziner}!Anatol@\strich\emph{Anatol}|pwv}\pwindex{Schnitzler, Arthur 15.\,5.\,1862 Wien – 21.\,10.\,1931 ebd.@\textsc{Schnitzler, Arthur} (15.\,5.\,1862 Wien – 21.\,10.\,1931 ebd.), \emph{Schriftsteller, Mediziner}!Märchen. Schauspiel in drei Aufzügen@\strich\emph{Das Märchen. Schauspiel in drei Aufzügen}|pwv} »myſtiſch«{ }ſeien? Ich
               wußte nicht recht, was {\pb}ich{ }ſagen{ }ſollte: Bitte,{ }ſind{ }ſie myſtiſch?\pend
           
\pstart
           Übrigens habe ich noch andere Eiſen für \substVorne{}\textsuperscript{d}\substDazwischen{}D\substHinten{}ich hier im Feuer. Doch davon{ }ſpäter.\pend
           
\pstart
           Das Blühen in der lieben Wien\oindex{Wien@\textbf{Wien}, \emph{Verwaltungsgebiet}|pw}er Künſtler-Laube –
               oh verdammt, welch’ ein Gleichniß! – beobachte ich mit wehmüthiger Freude. Gewiß, ich
               weiß, daß Eure\pwindex{Beer-Hofmann, Richard 11.\,7.\,1866 Wien – 26.\,9.\,1945 New York City@\textsc{Beer-Hofmann, Richard} (11.\,7.\,1866 Wien – 26.\,9.\,1945 New York City), \emph{Schriftsteller}|pwv}\pwindex{Hofmannsthal, Hugo von 1.\,2.\,1874 Wien – 15.\,7.\,1929 Rodaun@\textsc{Hofmannsthal, Hugo von} (1.\,2.\,1874 Wien – 15.\,7.\,1929 Rodaun), \emph{Schriftsteller}|pwv} drei Namen weit klingen werden, und in nicht langer Zeit. Ich{ }ſehe, wie
                  Ihr\pwindex{Beer-Hofmann, Richard 11.\,7.\,1866 Wien – 26.\,9.\,1945 New York City@\textsc{Beer-Hofmann, Richard} (11.\,7.\,1866 Wien – 26.\,9.\,1945 New York City), \emph{Schriftsteller}|pwv}\pwindex{Hofmannsthal, Hugo von 1.\,2.\,1874 Wien – 15.\,7.\,1929 Rodaun@\textsc{Hofmannsthal, Hugo von} (1.\,2.\,1874 Wien – 15.\,7.\,1929 Rodaun), \emph{Schriftsteller}|pwv} formt und{ }ſchafft, und wünſche allen Segen {\pb}auf dieſes Schaffen herab. Und dann kehre ich in
               mich ein und habe das traurige Gefühl des Mannes, der einſam und{ }ſchwach auf einem
               Stein{ }ſitzen geblieben iſt und nur noch die fernen Stimmen der Begleiter hört, die
               durch den Wald hallen: aber{ }ſie{ }ſind weit und er wird ihnen nimmer nachkommen. Meine
               Arbeiten? Gewiß weiß ichs nicht, wenn ich etwas Gutes{ }ſchreibe. Und wenn ich es
               wüßte: Hat das einen Werth, was ich thue? Geh’, das mußt {\pb}Du mir{ }ſelbſt zugeben, daß ich in unſerem Kreiſe
               bereits immer deutlicher die bitterböſe Rolle übernehme »des Mannes, aus dem etwas
               hätte werden können«.\pend
           
\pstart
           Ich bitte Dich inſtändig: veranlaſſe \textsc{Loris\pwindex{Hofmannsthal, Hugo von 1.\,2.\,1874 Wien – 15.\,7.\,1929 Rodaun@\textsc{Hofmannsthal, Hugo von} (1.\,2.\,1874 Wien – 15.\,7.\,1929 Rodaun), \emph{Schriftsteller}|pw}} und \textsc{Richard\pwindex{Beer-Hofmann, Richard 11.\,7.\,1866 Wien – 26.\,9.\,1945 New York City@\textsc{Beer-Hofmann, Richard} (11.\,7.\,1866 Wien – 26.\,9.\,1945 New York City), \emph{Schriftsteller}|pw}}, daß{ }ſie mir die erſchienen{[}en{]} oder zu erſcheinenden
                  \label{K_L02719-7v}\edtext{Sachen\pwindex{Beer-Hofmann, Richard 11.\,7.\,1866 Wien – 26.\,9.\,1945 New York City@\textsc{Beer-Hofmann, Richard} (11.\,7.\,1866 Wien – 26.\,9.\,1945 New York City), \emph{Schriftsteller}!Novellen@\strich\emph{Novellen}|pwv}}{\lemma{\textnormal{\emph{Sachen}}}\Cendnote{\textnormal{Die einzige selbstständige
                  Veröffentlichung – Goldmann\pwindex{Goldmann, Paul 31.\,1.\,1865 Breslau – 25.\,9.\,1935 Wien@\textsc{Goldmann, Paul} (31.\,1.\,1865 Breslau – 25.\,9.\,1935 Wien), \emph{Schriftsteller, Journalist}|pwk} bezieht sich
                  auf »Bücher\pwindex{Beer-Hofmann, Richard 11.\,7.\,1866 Wien – 26.\,9.\,1945 New York City@\textsc{Beer-Hofmann, Richard} (11.\,7.\,1866 Wien – 26.\,9.\,1945 New York City), \emph{Schriftsteller}!Novellen@\strich\emph{Novellen}|pw}« – aus dieser Zeit stellt eine Novellensammlung\pwindex{Beer-Hofmann, Richard 11.\,7.\,1866 Wien – 26.\,9.\,1945 New York City@\textsc{Beer-Hofmann, Richard} (11.\,7.\,1866 Wien – 26.\,9.\,1945 New York City), \emph{Schriftsteller}!Novellen@\strich\emph{Novellen}|pwkv}{ }Richard Beer-Hofmanns\pwindex{Beer-Hofmann, Richard 11.\,7.\,1866 Wien – 26.\,9.\,1945 New York City@\textsc{Beer-Hofmann, Richard} (11.\,7.\,1866 Wien – 26.\,9.\,1945 New York City), \emph{Schriftsteller}|pwk} dar, doch erschien
                  diese erst im Dezember 1893. Richard Beer-Hofmann\pwindex{Beer-Hofmann, Richard 11.\,7.\,1866 Wien – 26.\,9.\,1945 New York City@\textsc{Beer-Hofmann, Richard} (11.\,7.\,1866 Wien – 26.\,9.\,1945 New York City), \emph{Schriftsteller}|pwk}: \emph{Novellen}\pwindex{Beer-Hofmann, Richard 11.\,7.\,1866 Wien – 26.\,9.\,1945 New York City@\textsc{Beer-Hofmann, Richard} (11.\,7.\,1866 Wien – 26.\,9.\,1945 New York City), \emph{Schriftsteller}!Novellen@\strich\emph{Novellen}|pwk}. Berlin: \emph{Freund {\kaufmannsund} Jeckel}{ }1893.}}}\label{K_L02719-7}{ }ſchicken. Ohne Briefe: ich weiß, daß die Briefe nach{ }ſo langer
               Zeit{ }ſchwer zu{ }ſchreiben{ }ſind. Die gewiſſe Furcht vor der Einleitung. Ich {\pb}möchte deßwegen aber nicht um die Bücher\pwindex{Beer-Hofmann, Richard 11.\,7.\,1866 Wien – 26.\,9.\,1945 New York City@\textsc{Beer-Hofmann, Richard} (11.\,7.\,1866 Wien – 26.\,9.\,1945 New York City), \emph{Schriftsteller}!Novellen@\strich\emph{Novellen}|pwv} kommen.\pend
           
\pstart
           Wenn Du kannſt,{ }ſo{ }ſchick’ mir, bitte, gelegentlich noch einen »\textsc{Anatol\pwindex{Schnitzler, Arthur 15.\,5.\,1862 Wien – 21.\,10.\,1931 ebd.@\textsc{Schnitzler, Arthur} (15.\,5.\,1862 Wien – 21.\,10.\,1931 ebd.), \emph{Schriftsteller, Mediziner}!Anatol@\strich\emph{Anatol}|pw}}« – zu Propaganda-Zwecken.\pend
           
\pstart
           \textsc{Bahr\pwindex{Bahr, Hermann 19.\,7.\,1863 Linz – 15.\,1.\,1934 München@\textsc{Bahr, Hermann} (19.\,7.\,1863 Linz – 15.\,1.\,1934 München), \emph{Schriftsteller, Kritiker}|pw}}: Du haſt eine{ }ſo merkwürdige Art, gegen Leute gerecht{ }ſein zu wollen, die{ }ſich{ }ſchurkiſch gegen Dich benehmen. Nein, – der Mann\pwindex{Bahr, Hermann 19.\,7.\,1863 Linz – 15.\,1.\,1934 München@\textsc{Bahr, Hermann} (19.\,7.\,1863 Linz – 15.\,1.\,1934 München), \emph{Schriftsteller, Kritiker}|pwv} iſt für mich kein großes Talent,{ }ſelbſt wenn er es{ }ſein{ }ſollte. Ungerechte {\pb}Beurtheilung iſt bereits eine
               halbe Befriedigung des Haſſes. Und{ }ſeit der hundsföttischen Kritik\pwindex{Bahr, Hermann 19.\,7.\,1863 Linz – 15.\,1.\,1934 München@\textsc{Bahr, Hermann} (19.\,7.\,1863 Linz – 15.\,1.\,1934 München), \emph{Schriftsteller, Kritiker}!junge Österreich@\strich\emph{Das junge Österreich}|pwv} über Dich haſſe ich den Kerl\pwindex{Bahr, Hermann 19.\,7.\,1863 Linz – 15.\,1.\,1934 München@\textsc{Bahr, Hermann} (19.\,7.\,1863 Linz – 15.\,1.\,1934 München), \emph{Schriftsteller, Kritiker}|pwv} mehr als je.\pend
           
\pstart
           Der \label{K_L02719-8v}\edtext{Briefkaſten-Diebſtahl}{\lemma{\textnormal{\emph{Briefkasten-Diebstahl}}}\Cendnote{\textnormal{In \emph{Ridicula}\pwindex{Sosnosky, Theodor von 4.\,1.\,1866 Budapest – 3.\,2.\,1943 Wien@\textsc{Sosnosky, Theodor von} (4.\,1.\,1866 Budapest – 3.\,2.\,1943 Wien), \emph{Schriftsteller}!Ridicula@\strich\emph{Ridicula}|pwk} versammelte Theodor von
                     Sosnosky\pwindex{Sosnosky, Theodor von 4.\,1.\,1866 Budapest – 3.\,2.\,1943 Wien@\textsc{Sosnosky, Theodor von} (4.\,1.\,1866 Budapest – 3.\,2.\,1943 Wien), \emph{Schriftsteller}|pwk} vermeintliche »literarische Lächerlichkeiten« (Breslau:{ }\emph{Trewendt}\orgindex{Eduard Trewendt@Eduard Trewendt|pwk}{ }1894 [von 1893 vordatiert]). Im Kapitel
                  »Briefkastenpoesie« wurden – ohne Erlaubnis – 50 Seiten aus dem Briefkasten\pwindex{der schönen blauen Donau@\emph{An der schönen blauen Donau}|pwkv} der \emph{Schönen blauen Donau}\pwindex{der schönen blauen Donau@\emph{An der schönen blauen Donau}|pwk} aufgenommen. (Vgl. h. k.\pwindex{k., h. @\textsc{k., h.}, \emph{Journalist/Journalistin}|pwk}: \emph{Neue Bücher}\pwindex{k., h. @\textsc{k., h.}, \emph{Journalist/Journalistin}!Neue Bücher@\strich\emph{Neue Bücher}|pwk}. In: \emph{An der schönen blauen
                        Donau}\pwindex{der schönen blauen Donau@\emph{An der schönen blauen Donau}|pwk}, Jg. 8, Nr. 23, 1. 12. 1893,
                     S. 552.)}}}\label{K_L02719-8} des \textsc{Sosnosky\pwindex{Sosnosky, Theodor von 4.\,1.\,1866 Budapest – 3.\,2.\,1943 Wien@\textsc{Sosnosky, Theodor von} (4.\,1.\,1866 Budapest – 3.\,2.\,1943 Wien), \emph{Schriftsteller}|pw}} iſt{ }ſcheußlich. Ich habe mit meinem Onkel\pwindex{Mamroth, Fedor 21.\,2.\,1851 Breslau – 25.\,6.\,1907 Frankfurt am Main@\textsc{Mamroth, Fedor} (21.\,2.\,1851 Breslau – 25.\,6.\,1907 Frankfurt am Main), \emph{Journalist, Kritiker}|pwv} berathen, aber ich glaube, wir können nichts machen\substVorne{}\textsuperscript{.}\substDazwischen{},\substHinten{} geſetzlich. Höchſtens eine Züchtigung im Blatte\pwindex{der schönen blauen Donau@\emph{An der schönen blauen Donau}|pwv}, die aber auch eine Reklame für das Buch\pwindex{Sosnosky, Theodor von 4.\,1.\,1866 Budapest – 3.\,2.\,1943 Wien@\textsc{Sosnosky, Theodor von} (4.\,1.\,1866 Budapest – 3.\,2.\,1943 Wien), \emph{Schriftsteller}!Ridicula@\strich\emph{Ridicula}|pwv} des Gauners\pwindex{Sosnosky, Theodor von 4.\,1.\,1866 Budapest – 3.\,2.\,1943 Wien@\textsc{Sosnosky, Theodor von} (4.\,1.\,1866 Budapest – 3.\,2.\,1943 Wien), \emph{Schriftsteller}|pwv} wäre.\pend
           
\pstart
           {\pb}\textsc{Herzl\pwindex{Herzl, Theodor 2.\,5.\,1860 Budapest – 3.\,7.\,1904 Edlach@\textsc{Herzl, Theodor} (2.\,5.\,1860 Budapest – 3.\,7.\,1904 Edlach), \emph{Schriftsteller, Journalist}|pw}} iſt{ }ſeit einigen Wochen{ }ſehr \label{K_L02719-9v}\edtext{krank}{\lemma{\textnormal{\emph{krank}}}\Cendnote{\textnormal{Von seiner Malariainfektion
                  berichtete Theodor Herzl\pwindex{Herzl, Theodor 2.\,5.\,1860 Budapest – 3.\,7.\,1904 Edlach@\textsc{Herzl, Theodor} (2.\,5.\,1860 Budapest – 3.\,7.\,1904 Edlach), \emph{Schriftsteller, Journalist}|pwk} am 8. 12. 1893 in einem Brief an Schnitzler. Vgl. Theodor Herzl\pwindex{Herzl, Theodor 2.\,5.\,1860 Budapest – 3.\,7.\,1904 Edlach@\textsc{Herzl, Theodor} (2.\,5.\,1860 Budapest – 3.\,7.\,1904 Edlach), \emph{Schriftsteller, Journalist}|pwk}: \emph{Briefe und Tagebücher}. Herausgegeben von 
                     Alex Bein, Hermann Greive, Moshe Schaerf und Julius H. Schoeps. Bd. 1: \emph{Briefe und autobiographische Notizen. 1866–1895}.
                     Bearbeitet von Johannes Wachten. In Zusammenarbeit mit Chaya Harel, Daisy Tycho
                     und Manfred Winkler. Berlin, Frankfurt am
                        Main, Wien: \emph{Ullstein}\orgindex{Ullstein Verlag@Ullstein Verlag|pwk}/\emph{Propyläen}\orgindex{Propyläen Verlag@Propyläen Verlag|pwk}{ }1983, S. 545.
               }}}\label{K_L02719-9}: \textsc{Malaria} oder{ }ſo etwas.\pend
           
\pstart
           Was Neues in Wien\oindex{Wien@\textbf{Wien}, \emph{Verwaltungsgebiet}|pw}? Bitte{ }ſchreibe bald.\pend
           
\pstart
           Auch ein perſönliches Wort: Geſundheit, Production, materielle Fragen.\pend
           
\pstart
           Mir geht es{ }ſchlecht, oh{ }ſo{ }ſchlecht!\pend
           
\pstart
           Viele treue Grüße!\pend
           
\pstart
           Dein {\\[\baselineskip]}\spacefill\mbox{Paul Goldmnn}\pend
           \leftskip=0em{}\selectlanguage{ngerman}\endnumbering\briefempfaengerindex{Schnitzler, Arthur@\textsc{Schnitzler, Arthur}!zzzGoldmann, Paul@\emph{von Paul Goldmann}!1893-11-041@{4. 11. [1893]}|)be}\mylabel{L02719h}  \newcommand{\dateiname}{L02719}\newcommand{\titel}{Paul Goldmann an Arthur Schnitzler, 4. 11. [1893]}\newcommand{\editorInnen}{Martin Anton Müller und Laura Untner}%% latex-leseansicht-abspann.tex
%% Abspann für die Leseansicht.
%% Der Schalter \ifkorrekturansicht ist bereits durch den Vorspann gesetzt.

%% latex-abspann.tex
%% Gemeinsamer Abspann für Korrekturansicht und Leseansicht.
%% Setzt den Schalter \ifkorrekturansicht voraus (gesetzt in den
%% einbindenden Dateien latex-korrekturansicht-abspann.tex bzw.
%% latex-leseansicht-abspann.tex).
%% ---------------------------------------------------------------

\normalsize

% Das esempio-Environment wird nur in der Leseansicht benötigt
\ifkorrekturansicht\else
\newenvironment{esempio}[3]%
{
    \vspace{1.5ex}
    \rlap{\underline{#1}}
    \par
    \setlength{\parindent}{0cm}
    \nopagebreak
    \leftskip=#2cm
    \rightskip=#3cm
}
{
    \par
}
\fi

\doendnotes{C}
\bigskip
\vfill

\clearpage

\footnotesize

\ifkorrekturansicht
  \lohead{\textsc{register}}
\fi

% theindex-Environment neu definieren ohne reledmac
\makeatletter
\renewenvironment{theindex}{%
  \ifkorrekturansicht
    \section*{\indexname}%
  \else
    \subsubsection*{Index der erwähnten Entitäten}%
  \fi
  \setlength{\parindent}{0pt}%
  \setlength{\parskip}{0pt plus 0.3pt}%
  \let\item\@idxitem
}{%
  \ifkorrekturansicht\clearpage\fi
}
\makeatother

\IfFileExists{\jobname-pw.ind}{\input{\jobname-pw.ind}}{}

% Quellenangabe nur in der Leseansicht
\ifkorrekturansicht\else
% Fallback-Definitionen, falls die .tex-Datei \titel etc. nicht gesetzt hat
\providecommand{\titel}{}
\providecommand{\editorInnen}{}
\providecommand{\dateiname}{\jobname}

\vspace{3cm}

\vfill

\footnotesize
\textsc{Quelle}: \titel. Herausgegeben von {\editorInnen}. In: \emph{Arthur Schnitzler: Briefwechsel mit Autorinnen und Autoren}.
 Digitale Edition, https://schnitzler-briefe.acdh.oeaw.ac.at/{\dateiname}.html (Stand \today)
\fi

\end{document}


