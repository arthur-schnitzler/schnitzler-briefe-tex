%% latex-leseansicht-vorspann.tex
%% Vorspann für die Leseansicht.
%% Lädt die gemeinsame Datei latex-vorspann.tex mit nicht gesetztem Schalter.

\newif\ifkorrekturansicht
\korrekturansichtfalse

\input{../tex-inputs/latex-vorspann}


\section[Arthur Schnitzler an Richard Beer-Hofmann, 14. 2. 1912]{L02056 Arthur Schnitzler an Richard Beer-Hofmann, 14. 2. 1912}
\nopagebreak\mylabel{L02056v}
\rehead{ }\normalsize\beginnumbering\briefempfaengerindex{Beer-Hofmann, Richard@\textsc{Beer-Hofmann, Richard}!zzzSchnitzler, Arthur@\emph{von Arthur Schnitzler}!1912-02-142@{14. 2. 1912}|(be}
\toendnotes[C]{\smallbreak\pagebreak[2]}
\correspDesc{Versand  durch Arthur Schnitzler am 14. 2. 1912 in Wien
\newline{}Erhalt  durch Richard Beer-Hofmann am 14. 2. 1912 in Wien}\toendnotes[C]{\smallbreak}
\Standort{YCGL, MSS 31.}
\physDesc{Briefkarte, , Kuvert, 342 Zeichen (Karte und Umschlag mit Trauerrand )
\newline{}Handschrift: Bleistift, lateinische Kurrent
\newline{}Versand: ohne postalischen Übermittlungsvermerk }
\buchAbdrucke{\weitereDrucke{Arthur Schnitzler, Richard Beer-Hofmann: \emph{Briefwechsel 1891–1931}. Herausgegeben von Konstanze Fliedl. Wien, Zürich: \emph{Europaverlag} 1992, S. 216.} }\toendnotes[C]{\smallbreak}\pstart{}{\pb}\textcolor{gray}{\textbf{XVIII., STERNWARTESTRASSE 71\oindex{Wien@\textbf{Wien}!XVIII., Währing@\textbf{XVIII., Währing}!Sternwartestraße 71@\textbf{Sternwartestraße 71}, \emph{Wohngebäude}|pw}.}}\pend{}{\bigskip}\pstart{}{\pb}Herrn Doctor Richard Beer-Hofmann\pend{}\pstart{}\label{T_L02056-1v}\edtext{Wien}{\lemma{\textnormal{\emph{Wien}}}\Cendnote{\textnormal{Abweichend vom restlichen
                        Korrespondenzstück ist dies nicht in Lateinschrift geschrieben.}}}\label{T_L02056-1}\oindex{Wien@\textbf{Wien}, \emph{Verwaltungsgebiet}|pw}\pend{}{\bigskip}\vspace{1em}
\pstart
           {\pb}\textcolor{gray}{\textbf{A. S.}}\hfill \textcolor{gray}{\textbf{XVIII., STERNWARTESTRASSE 71\oindex{Wien@\textbf{Wien}!XVIII., Währing@\textbf{XVIII., Währing}!Sternwartestraße 71@\textbf{Sternwartestraße 71}, \emph{Wohngebäude}|pw}.}}\pend
           
\pstart
           \raggedleft{}14. 2.\pend
           \vspace{0.5em}
\pstart
           lieber Richard,{ }Rosenbaums\pwindex{Rosenbaum, Richard 4.\,11.\,1867 Žikov – 25.\,6.\,1942 Konzentrationslager Theresienstadt@\textsc{Rosenbaum, Richard} (4.\,11.\,1867 Žikov – 25.\,6.\,1942 Konzentrationslager Theresienstadt), \emph{Dramaturg, Verleger}|pw} Privat-Teleph. Nu{\geminationm}er mir unbeka{\geminationn}t, will mich
               auch im Burg. Th.\oindex{Wien@\textbf{Wien}!I., Innere Stadt@\textbf{I., Innere Stadt}!Burgtheater@\textbf{Burgtheater}, \emph{Theater}|pw} nicht erkundigen, da ich einen
               Refus fürchte – oder feurige Kohlen. Stucken\pwindex{Stucken, Eduard 18.\,3.\,1865 Moskau – 9.\,3.\,1936 Berlin@\textsc{Stucken, Eduard} (18.\,3.\,1865 Moskau – 9.\,3.\,1936 Berlin), \emph{Schriftsteller}|pw}\pwindex{Stucken, Ania 1877/1878 – 16.\,8.\,1924 Saaleck@\textsc{Stucken, Ania} (1877/1878 – 16.\,8.\,1924 Saaleck)|pw}’s wohnen Hotel Regina\oindex{Wien@\textbf{Wien}!IX., Alsergrund@\textbf{IX., Alsergrund}!Hotel Regina [Wien]@\textbf{Hotel Regina [Wien]}, \emph{Hotel}|pw}. Sie
               kommen \label{K_L02056-1v}\edtext{Samstag}{\lemma{\textnormal{\emph{Samstag}}}\Cendnote{\textnormal{Siehe A. S.: \emph{Tagebuch}, 17. 2. 1912.
               }}}\label{K_L02056-1} gegen {\pb}5 Uhr zum Thee zu uns und wir bitten Sie mit Paula\pwindex{Beer-Hofmann, Paula 25.\,2.\,1879 Wien – 30.\,10.\,1939 Zürich@\textsc{Beer-Hofmann, Paula} (25.\,2.\,1879 Wien – 30.\,10.\,1939 Zürich)|pw} gleichfalls zu erscheinen.\pend
           \pstart Herzlichst Ihr \spacefill\mbox{A. \strikeout{S.}}\pend{}\selectlanguage{ngerman}\endnumbering\briefempfaengerindex{Beer-Hofmann, Richard@\textsc{Beer-Hofmann, Richard}!zzzSchnitzler, Arthur@\emph{von Arthur Schnitzler}!1912-02-142@{14. 2. 1912}|)be}\mylabel{L02056h}  \newcommand{\dateiname}{L02056}\newcommand{\titel}{Arthur Schnitzler an Richard Beer-Hofmann, 14. 2. 1912}\newcommand{\editorInnen}{Martin Anton Müller und Gerd-Hermann Susen}%% latex-leseansicht-abspann.tex
%% Abspann für die Leseansicht.
%% Der Schalter \ifkorrekturansicht ist bereits durch den Vorspann gesetzt.

%% latex-abspann.tex
%% Gemeinsamer Abspann für Korrekturansicht und Leseansicht.
%% Setzt den Schalter \ifkorrekturansicht voraus (gesetzt in den
%% einbindenden Dateien latex-korrekturansicht-abspann.tex bzw.
%% latex-leseansicht-abspann.tex).
%% ---------------------------------------------------------------

\normalsize

% Das esempio-Environment wird nur in der Leseansicht benötigt
\ifkorrekturansicht\else
\newenvironment{esempio}[3]%
{
    \vspace{1.5ex}
    \rlap{\underline{#1}}
    \par
    \setlength{\parindent}{0cm}
    \nopagebreak
    \leftskip=#2cm
    \rightskip=#3cm
}
{
    \par
}
\fi

\doendnotes{C}
\bigskip
\vfill

\clearpage

\footnotesize

\ifkorrekturansicht
  \lohead{\textsc{register}}
\fi

% theindex-Environment neu definieren ohne reledmac
\makeatletter
\renewenvironment{theindex}{%
  \ifkorrekturansicht
    \section*{\indexname}%
  \else
    \subsubsection*{Index der erwähnten Entitäten}%
  \fi
  \setlength{\parindent}{0pt}%
  \setlength{\parskip}{0pt plus 0.3pt}%
  \let\item\@idxitem
}{%
  \ifkorrekturansicht\clearpage\fi
}
\makeatother

\IfFileExists{\jobname-pw.ind}{\input{\jobname-pw.ind}}{}

% Quellenangabe nur in der Leseansicht
\ifkorrekturansicht\else
% Fallback-Definitionen, falls die .tex-Datei \titel etc. nicht gesetzt hat
\providecommand{\titel}{}
\providecommand{\editorInnen}{}
\providecommand{\dateiname}{\jobname}

\vspace{3cm}

\vfill

\footnotesize
\textsc{Quelle}: \titel. Herausgegeben von {\editorInnen}. In: \emph{Arthur Schnitzler: Briefwechsel mit Autorinnen und Autoren}.
 Digitale Edition, https://schnitzler-briefe.acdh.oeaw.ac.at/{\dateiname}.html (Stand \today)
\fi

\end{document}


