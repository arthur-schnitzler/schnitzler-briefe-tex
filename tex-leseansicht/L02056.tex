%% latex-korrekturansicht-vorspann.tex
%% Vorspann für die Korrekturansicht.
%% Lädt die gemeinsame Datei latex-vorspann.tex mit gesetztem Schalter.

\newif\ifkorrekturansicht
\korrekturansichttrue

\input{../tex-inputs/latex-vorspann}


\section[Arthur Schnitzler an Richard Beer-Hofmann, 14. 2. 1912]{L02056 Arthur Schnitzler an Richard Beer-Hofmann, 14. 2. 1912}
\nopagebreak\mylabel{L02056v}
\rehead{ }\normalsize\beginnumbering\briefempfaengerindex{Beer-Hofmann, Richard@\textsc{Beer-Hofmann, Richard}!zzzSchnitzler, Arthur@\emph{von Arthur Schnitzler}!1912-02-142@{14. 2. 1912}|(be}
\toendnotes[C]{\smallbreak\pagebreak[2]}\Standort{YCGL, MSS 31.}
\physDesc{Briefkarte, , Umschlag, 342 Zeichen (Karte und Umschlag mit Trauerrand )
\newline{}Handschrift: Bleistift, lateinische Kurrent
\newline{}Versand: ohne postalischen Übermittlungsvermerk }
\buchAbdrucke{\weitereDrucke{Arthur Schnitzler, Richard Beer-Hofmann: \emph{Briefwechsel 1891–1931}. Wien, Zürich: \emph{Europaverlag} 1992, S. 216.} }\toendnotes[C]{\smallbreak}\pstart{}{\pb}\textcolor{gray}{\textbf{XVIII., STERNWARTESTRASSE 71\oindex{Sternwartestrasse 71@\textbf{Sternwartestraße 71}, \emph{Wohngebäude (K.WHS)}|pw}.}}\pend{}{\bigskip}\pstart{}{\pb}Herrn Doctor Richard Beer-Hofmann\pend{}\pstart{}\label{T_L02056-1v}\edtext{Wien}{\lemma{\textnormal{\emph{Wien}}}\Cendnote{\textnormal{Abweichend vom restlichen
                        Korrespondenzstück ist dies nicht in Lateinschrift geschrieben.}}}\label{T_L02056-1}\oindex{Wien@\textbf{Wien}, \emph{A.ADM2}|pw}\pend{}{\bigskip}\vspace{1em}
\pstart
           {\pb}\textcolor{gray}{\textbf{A. S.}}\hfill \textcolor{gray}{\textbf{XVIII., STERNWARTESTRASSE 71\oindex{Sternwartestrasse 71@\textbf{Sternwartestraße 71}, \emph{Wohngebäude (K.WHS)}|pw}.}}\pend
           
\pstart
           \raggedleft{}14. 2.\pend
           \vspace{0.5em}
\pstart
           lieber Richard,{ }Rosenbaums\pwindex{Rosenbaum, Richard 04.11.1867 – 25.06.1942@\textsc{Rosenbaum, Richard} (04.11.1867 – 25.06.1942), \emph{Dramaturg/Dramaturgin, Verleger/Verlegerin}|pw} Privat-Teleph. Nu{\geminationm}er mir unbeka{\geminationn}t, will mich
               auch im Burg. Th.\oindex{Burgtheater@\textbf{Burgtheater}, \emph{S.THTR}|pw} nicht erkundigen, da ich einen
               Refus fürchte – oder feurige Kohlen. Stucken\pwindex{Stucken, Eduard 18.03.1865 – 09.03.1936@\textsc{Stucken, Eduard} (18.03.1865 – 09.03.1936), \emph{Schriftsteller/Schriftstellerin}|pw}\pwindex{Stucken, Ania 1877/1878 – 16.8.1924@\textsc{Stucken, Ania} (1877/1878 – 16.8.1924)|pw}’s wohnen Hotel Regina\oindex{Hotel Regina [Wien]@\textbf{Hotel Regina [Wien]}, \emph{Hotel (K.HTL)}|pw}. Sie
               kommen \label{K_L02056-1v}\edtext{Samstag}{\lemma{\textnormal{\emph{Samstag}}}\Cendnote{\textnormal{Siehe A. S.: \emph{Tagebuch}, 17. 2. 1912.
               }}}\label{K_L02056-1} gegen {\pb}5 Uhr zum Thee zu uns und wir bitten Sie mit Paula\pwindex{Beer-Hofmann, Paula 25.02.1879 – 30.10.1939@\textsc{Beer-Hofmann, Paula} (25.02.1879 – 30.10.1939)|pw} gleichfalls zu erscheinen.\pend
           \pstart Herzlichst Ihr \spacefill\mbox{A. \strikeout{S.}}\pend{}\selectlanguage{ngerman}\endnumbering\briefempfaengerindex{Beer-Hofmann, Richard@\textsc{Beer-Hofmann, Richard}!zzzSchnitzler, Arthur@\emph{von Arthur Schnitzler}!1912-02-142@{14. 2. 1912}|)be}\mylabel{L02056h}  \normalsize

\doendnotes{C}
\bigskip
\vfill

\clearpage

\footnotesize

\lohead{\textsc{register}}

% Definiere theindex-Environment komplett neu ohne reledmac
\makeatletter
\renewenvironment{theindex}{%
  \section*{\indexname}%
  \setlength{\parindent}{0pt}%
  \setlength{\parskip}{0pt plus 0.3pt}%
  \let\item\@idxitem
}{%
  \clearpage
}
\makeatother

\IfFileExists{\jobname-pw.ind}{\input{\jobname-pw.ind}}{}

\end{document}

      