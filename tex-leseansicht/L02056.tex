\input{../tex-inputs/latex-pdf-vorspann}
\begin{center}
            \textcolor{red}{ENTWURF. ENTZIFFERUNG NOCH NICHT KORREKTURGELESEN}
                      \end{center}
            
               \section[Arthur Schnitzler an Richard Beer-Hofmann, 14. 2. 1912]{ Arthur Schnitzler an Richard Beer-Hofmann, 14. 2. 1912}\nopagebreak\mylabel{v}\rehead{ }\begin{ledgroupsized}[t]{13cm}\normalsize\beginnumbering\briefempfaengerindex{Beer-Hofmann, Richard@\textsc{Beer-Hofmann, Richard}!zzzSchnitzler, Arthur@\emph{von Arthur Schnitzler}!1912-02-142@{14. 2. 1912}|(be} \toendnotes[C]{\smallbreak\pagebreak[2]} \Standort{YCGL, MSS 31.}
\physDesc{Briefkarte mit Trauerrand, Umschlag mit Trauerrand
\newline{}Handschrift: Bleistift, lateinische Kurrent\newline{}Versand: ohne postalischen Übermittlungsvermerk }\buchAbdrucke{\weitereDrucke{Arthur Schnitzler, Richard Beer-Hofmann: \emph{Briefwechsel 1891–1931}. Hg. Konstanze Fliedl. Wien, Zürich: \emph{Europaverlag} 1992, S. 216.} }\toendnotes[C]{\smallbreak}\pstart{}{\pb}\textcolor{gray}{\textbf{XVIII., STERNWARTESTRASSE 71\oindex{Sternwartestrasse@\textbf{Sternwartestraße}|pw}.}}\pend{}{\bigskip}\pstart{}{\pb}Herrn Doctor Richard Beer-Hofmann\pend{}\pstart{}\label{TLL02056_Beer-Hofmann-1v}\edtext{Wien}{\lemma{\textnormal{\emph{Wien}}}\Cendnote{\textnormal{Abweichend vom restlichen
                        Korrespondenzstück ist dies nicht in Lateinschrift geschrieben.}}}\label{TLL02056_Beer-Hofmann-1h}\oindex{Wien@\textbf{Wien}|pw}\pend{}{\bigskip}\pstart
           \noindent{}{\pb}\textcolor{gray}{\textbf{A. S.}}\hfill \textcolor{gray}{\textbf{XVIII., STERNWARTESTRASSE 71\oindex{Sternwartestrasse@\textbf{Sternwartestraße}|pw}.}}\pend
           \pstart
           \raggedleft{}14. 2.\pend
           \pstart
           lieber Richard, Rosenbaums\pwindex{Rosenbaum, Richard 04.11.1867 – 25.06.1942@\textsc{Rosenbaum, Richard} (04.11.1867 – 25.06.1942), \emph{Dramaturg, Verleger}|pw} Privat-Teleph. Nu{\geminationm}er mir unbeka{\geminationn}t, will mich
               auch im Burg. Th.\oindex{Burgtheater@\textbf{Burgtheater}|pw} nicht erkundigen, da ich einen
               Refus fürchte – oder feurige Kohlen. Stucken\pwindex{Stucken, Eduard 18.03.1865 – 09.03.1936@\textsc{Stucken, Eduard} (18.03.1865 – 09.03.1936), \emph{Schriftsteller}|pw}\pwindex{Stucken, Ania 1877/1878 – 16.8.1924@\textsc{Stucken, Ania} (1877/1878 – 16.8.1924)|pw}’s wohnen Hotel Regina\oindex{Hotel Regina@\textbf{Hotel Regina}|pw}. Sie kommen
                  \label{KLL02056_Beer-Hofmann-1v}\edtext{Samstag}{\lemma{\textnormal{\emph{Samstag}}}\Cendnote{\textnormal{siehe A. S.: \emph{Tagebuch}, 17. 2. 1912}}}\label{KLL02056_Beer-Hofmann-1h} gegen {\pb}5 Uhr zum Thee zu uns und wir bitten Sie mit Paula\pwindex{Beer-Hofmann, Paula 25.02.1879 – 30.10.1939@\textsc{Beer-Hofmann, Paula} (25.02.1879 – 30.10.1939)|pw} gleichfalls zu erscheinen.\pend
           \pstart Herzlichst Ihr \spacefill\mbox{A. \strikeout{S.}}\pend{}\endnumbering\briefempfaengerindex{Beer-Hofmann, Richard@\textsc{Beer-Hofmann, Richard}!zzzSchnitzler, Arthur@\emph{von Arthur Schnitzler}!1912-02-142@{14. 2. 1912}|)be}\mylabel{h}\end{ledgroupsized}  \newcommand{\dateiname}{L02056}\newcommand{\titel}{Arthur Schnitzler an Richard Beer-Hofmann, 14. 2. 1912}\newcommand{\editorInnen}{Martin Anton Müller und Gerd-Hermann Susen}\input{../tex-inputs/latex-pdf-abspann}
      