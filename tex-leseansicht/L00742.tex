%% latex-leseansicht-vorspann.tex
%% Vorspann für die Leseansicht.
%% Lädt die gemeinsame Datei latex-vorspann.tex mit nicht gesetztem Schalter.

\newif\ifkorrekturansicht
\korrekturansichtfalse

\input{../tex-inputs/latex-vorspann}


         
         \renewcommand{\erwaehntePersonen}{Personen: Leopold von Andrian-Werburg, Eberhard von Bodenhausen, Friedrich Hess-Diller, Hugo von Hofmannsthal, Robert Lutz, Hans Bernhard Schlesinger}
         \renewcommand{\erwaehnteOrte}{Orte: Wien}
         \renewcommand{\erwaehnteWerke}{Werke: Der weiße Fächer. Ein Zwischenspiel, Die Hochzeit der Sobeide, Die Schwestern}
               \section[Arthur Schnitzler an Hugo von Hofmannsthal, {[}16. 11. 1897{]}]{ Arthur Schnitzler an Hugo von Hofmannsthal, {[}16. 11. 1897{]}}\nopagebreak\mylabel{v}\rehead{ }\begin{ledgroupsized}[t]{13cm}\normalsize\beginnumbering \toendnotes[C]{\smallbreak\pagebreak[2]} \Standort{FDH, Hs-30885,65.}
\physDesc{Brief, 1 Blatt, 3 Seiten, 971 Zeichen
\newline{}Handschrift: schwarze Tinte, deutsche Kurrent
\newline{}Hofmannsthal: mit Bleistift die 4. (leere) Seite beschriftet: »\noindent{}{\pb}\strikeout{Lutz\pwindex{Lutz, Robert @\textsc{Lutz, Robert}, \emph{Oberingenieur, Militär}|pw}}{ / }Poldy\pwindex{Andrian-Werburg, Leopold von 09.05.1875 – 19.11.1951@\textsc{Andrian-Werburg, Leopold von} (09.05.1875 – 19.11.1951), \emph{Schriftsteller, Diplomat}|pw}{ / }B\textsuperscript{\textcolor{gray}{rn}} Hess\pwindex{Hess-Diller, Friedrich 1847-03-31 – 1922-09-08@\textsc{Hess-Diller, Friedrich} (1847-03-31 – 1922-09-08)|pwu}{ / }Bodenhausen\pwindex{Bodenhausen, Eberhard von 12.06.1868 – 06.05.1918@\textsc{Bodenhausen, Eberhard von} (12.06.1868 – 06.05.1918)|pw}{ / }\strikeout{Hansl\pwindex{Schlesinger, Hans Bernhard 20.07.1875 – 13.3.1932@\textsc{Schlesinger, Hans Bernhard} (20.07.1875 – 13.3.1932), \emph{Maler}|pw}}« 
\newline{}Ordnung: mit Bleistift von Schnitzler mutmaßlich bei der Durchsicht der Korrespondenz
                                    1929  beschriftet: »Datum?
                                       92? 96?« }\buchAbdrucke{\weitereDrucke{Hugo von Hofmannsthal, Arthur Schnitzler: \emph{Briefwechsel}. Hg. Therese Nickl und Heinrich Schnitzler. Frankfurt am Main: \emph{S. Fischer} 1964, S. 97–98.} }\toendnotes[C]{\smallbreak}\pstart
           \raggedleft{}{\pb}Dinstag{ }Früh.\pend
           \pstart
           Lieber Hugo, ich vergaſs Ihnen zu ſchreiben, dſs heute
                  Dinſtag{ }Abend{ }\uline{nichts} bei mir iſt. – Ihre Antwort \substVorne{}\textsuperscript{hatte}\substDazwischen{}geſtern\substHinten{} Früh hatte ich wohl erwartet; aber ich konnte den Verſuch nicht weigern. Im
               übrigen mußte auch ich abſagen und hätte auch Ihnen abgeſagt, da ich ſchrecklich
               verkühlt bin. –\pend
           \pstart
           Hier ſind Ihre drei Stücke\pwindex{Hofmannsthal, Hugo von 1874-02-01 – 1929-07-15@\textsc{Hofmannsthal, Hugo von} (1874-02-01 – 1929-07-15), \emph{Schriftsteller}!weisse Faecher. Ein Zwischenspiel05. 02. 1898@\strich\emph{Der weiße Fächer. Ein Zwischenspiel} {[}05. 02. 1898{]}|pwv}\pwindex{Hofmannsthal, Hugo von 1874-02-01 – 1929-07-15@\textsc{Hofmannsthal, Hugo von} (1874-02-01 – 1929-07-15), \emph{Schriftsteller}!Hochzeit der Sobeide1899-03-18@\strich\emph{Die Hochzeit der Sobeide} {[}1899-03-18{]}|pwv}\pwindex{Hofmannsthal, Hugo von 1874-02-01 – 1929-07-15@\textsc{Hofmannsthal, Hugo von} (1874-02-01 – 1929-07-15), \emph{Schriftsteller}!Schwestern1982@\strich\emph{Die Schwestern} {[}1982{]}|pwv}. Ich habe mich {\pb}beim
               Leſen ſehr gefreut. Am reinſten hat der weiße
                  Fächer\pwindex{Hofmannsthal, Hugo von 1874-02-01 – 1929-07-15@\textsc{Hofmannsthal, Hugo von} (1874-02-01 – 1929-07-15), \emph{Schriftsteller}!weisse Faecher. Ein Zwischenspiel05. 02. 1898@\strich\emph{Der weiße Fächer. Ein Zwischenspiel} {[}05. 02. 1898{]}|pw} auf mich gewirkt; käme es zwiſchen Fortunio\pwindex{Hofmannsthal, Hugo von 1874-02-01 – 1929-07-15@\textsc{Hofmannsthal, Hugo von} (1874-02-01 – 1929-07-15), \emph{Schriftsteller}!weisse Faecher. Ein Zwischenspiel05. 02. 1898@\strich\emph{Der weiße Fächer. Ein Zwischenspiel} {[}05. 02. 1898{]}|pwv} und Miranda\pwindex{Hofmannsthal, Hugo von 1874-02-01 – 1929-07-15@\textsc{Hofmannsthal, Hugo von} (1874-02-01 – 1929-07-15), \emph{Schriftsteller}!weisse Faecher. Ein Zwischenspiel05. 02. 1898@\strich\emph{Der weiße Fächer. Ein Zwischenspiel} {[}05. 02. 1898{]}|pwv} irgendwo, am beſten wohl am Schluſs, zu einem
               lebhaften Sichſelber und Einanderverſtehn – ganz kurz, aber ſtark, ſo wäre das Stück\pwindex{Hofmannsthal, Hugo von 1874-02-01 – 1929-07-15@\textsc{Hofmannsthal, Hugo von} (1874-02-01 – 1929-07-15), \emph{Schriftsteller}!weisse Faecher. Ein Zwischenspiel05. 02. 1898@\strich\emph{Der weiße Fächer. Ein Zwischenspiel} {[}05. 02. 1898{]}|pwv} etwas vollko{\geminationm}enes. Bei der jungen
                  Frau\pwindex{Hofmannsthal, Hugo von 1874-02-01 – 1929-07-15@\textsc{Hofmannsthal, Hugo von} (1874-02-01 – 1929-07-15), \emph{Schriftsteller}!Hochzeit der Sobeide1899-03-18@\strich\emph{Die Hochzeit der Sobeide} {[}1899-03-18{]}|pw} hab ich zum Schluſs meinen lieben Kaufmann\pwindex{Hofmannsthal, Hugo von 1874-02-01 – 1929-07-15@\textsc{Hofmannsthal, Hugo von} (1874-02-01 – 1929-07-15), \emph{Schriftsteller}!Hochzeit der Sobeide1899-03-18@\strich\emph{Die Hochzeit der Sobeide} {[}1899-03-18{]}|pwv} wieder herbeigeſehnt. Hoffentlich laſſen Sie ihn
               erſcheinen, bei welcher Gelegenheit {\pb}er vielleicht auch
               aufklären könnte, wieſo die junge Frau\pwindex{Hofmannsthal, Hugo von 1874-02-01 – 1929-07-15@\textsc{Hofmannsthal, Hugo von} (1874-02-01 – 1929-07-15), \emph{Schriftsteller}!Hochzeit der Sobeide1899-03-18@\strich\emph{Die Hochzeit der Sobeide} {[}1899-03-18{]}|pwv}{ }ſich über den Sohn des Teppichhändlers\pwindex{Hofmannsthal, Hugo von 1874-02-01 – 1929-07-15@\textsc{Hofmannsthal, Hugo von} (1874-02-01 – 1929-07-15), \emph{Schriftsteller}!Hochzeit der Sobeide1899-03-18@\strich\emph{Die Hochzeit der Sobeide} {[}1899-03-18{]}|pwv} in ſo furchtbarer Weiſe durch
               viele Jahre täuſchen konnte.\pend
           \pstart
           Meine Karte mit dem Brief von Andrian\pwindex{Andrian-Werburg, Leopold von 09.05.1875 – 19.11.1951@\textsc{Andrian-Werburg, Leopold von} (09.05.1875 – 19.11.1951), \emph{Schriftsteller, Diplomat}|pw} haben
               Sie bekommen? –\pend
           \pstart
           Herzlichen Gruſs.{\\[\baselineskip]}Ihr \spacefill\mbox{Arthur}\pend
           \leftskip=0em{}
         
         \endnumbering\mylabel{h}\end{ledgroupsized}  \newcommand{\dateiname}{L00742}\newcommand{\titel}{Arthur Schnitzler an Hugo von Hofmannsthal, [16. 11. 1897]}\newcommand{\editorInnen}{Martin Anton Müller und Gerd-Hermann Susen}%% latex-leseansicht-abspann.tex
%% Abspann für die Leseansicht.
%% Der Schalter \ifkorrekturansicht ist bereits durch den Vorspann gesetzt.

%% latex-abspann.tex
%% Gemeinsamer Abspann für Korrekturansicht und Leseansicht.
%% Setzt den Schalter \ifkorrekturansicht voraus (gesetzt in den
%% einbindenden Dateien latex-korrekturansicht-abspann.tex bzw.
%% latex-leseansicht-abspann.tex).
%% ---------------------------------------------------------------

\normalsize

% Das esempio-Environment wird nur in der Leseansicht benötigt
\ifkorrekturansicht\else
\newenvironment{esempio}[3]%
{
    \vspace{1.5ex}
    \rlap{\underline{#1}}
    \par
    \setlength{\parindent}{0cm}
    \nopagebreak
    \leftskip=#2cm
    \rightskip=#3cm
}
{
    \par
}
\fi

\doendnotes{C}
\bigskip
\vfill

\clearpage

\footnotesize

\ifkorrekturansicht
  \lohead{\textsc{register}}
\fi

% theindex-Environment neu definieren ohne reledmac
\makeatletter
\renewenvironment{theindex}{%
  \ifkorrekturansicht
    \section*{\indexname}%
  \else
    \subsubsection*{Index der erwähnten Entitäten}%
  \fi
  \setlength{\parindent}{0pt}%
  \setlength{\parskip}{0pt plus 0.3pt}%
  \let\item\@idxitem
}{%
  \ifkorrekturansicht\clearpage\fi
}
\makeatother

\IfFileExists{\jobname-pw.ind}{\input{\jobname-pw.ind}}{}

% Quellenangabe nur in der Leseansicht
\ifkorrekturansicht\else
% Fallback-Definitionen, falls die .tex-Datei \titel etc. nicht gesetzt hat
\providecommand{\titel}{}
\providecommand{\editorInnen}{}
\providecommand{\dateiname}{\jobname}

\vspace{3cm}

\vfill

\footnotesize
\textsc{Quelle}: \titel. Herausgegeben von {\editorInnen}. In: \emph{Arthur Schnitzler: Briefwechsel mit Autorinnen und Autoren}.
 Digitale Edition, https://schnitzler-briefe.acdh.oeaw.ac.at/{\dateiname}.html (Stand \today)
\fi

\end{document}


      