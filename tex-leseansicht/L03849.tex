%% latex-leseansicht-vorspann.tex
%% Vorspann für die Leseansicht.
%% Lädt die gemeinsame Datei latex-vorspann.tex mit nicht gesetztem Schalter.

\newif\ifkorrekturansicht
\korrekturansichtfalse

\input{../tex-inputs/latex-vorspann}


\section[Theodor Herzl an Arthur Schnitzler, 18. 2. 1895]{L03849 Theodor Herzl an Arthur Schnitzler, 18. 2. 1895}
\nopagebreak\mylabel{L03849v}
\rehead{ }\normalsize\beginnumbering\briefempfaengerindex{Schnitzler, Arthur@\textsc{Schnitzler, Arthur}!zzzHerzl, Theodor@\emph{von Theodor Herzl}!1895-02-183@{18. 2. 1895}|(be}
\toendnotes[C]{\smallbreak\pagebreak[2]}
\correspDesc{Versand  durch Theodor Herzl am 18. 2. 1895 in Paris
\newline{}Erhalt  durch Arthur Schnitzler im Zeitraum [19. 2. 1895 – 23. 2. 1895?] in Wien}\toendnotes[C]{\smallbreak}
\Standort{CUL, Schnitzler, B 39.}
\physDesc{Brief, 1 Blatt, 4 Seiten, 2621 Zeichen
\newline{}Handschrift: schwarze Tinte, lateinische Kurrent
\newline{}Ordnung: mit Bleistift von unbekannter Hand nummeriert: »28« }
\buchAbdrucke{\weitereDrucke{Theodor Herzl: \emph{Briefe und
                        autobiographische Notizen 1866–1895}. Bearbeitet von Johannes Wachten in Zusammenarbeit mit Chaya Harel, Daisy Tycho und Manfred Winkler. Berlin, Frankfurt am Main, Wien: \emph{Propyläen} 1983, S. 573–574 (Briefe und Tagebücher. Herausgegeben von Alex Bein, Hermann Greive, Moshe Schaerf, Julius H. Schoeps und Johannes Wachten, 1).} }\toendnotes[C]{\smallbreak}
\pstart
           \raggedleft{}{\pb}Palais Bourbon\oindex{Palais Bourbon@\textbf{Palais Bourbon}, \emph{Regierungsgebäude}|pw}\pend
           
\pstart
           \raggedleft{}18. II. 95\pend
           
\pstart{}Mein lieber Freund!\pend\vspace{0.5em}
\pstart
           Schon abgekühlt, also ohne sonderlichen Schmerz erhielt ich heute wieder
               das zweite \label{K_L03849-1v}\edtext{»Subscriptionsresultat«}{\lemma{\textnormal{\emph{»Subscriptionsresultat«}}}\Cendnote{\textnormal{Herzl\pwindex{Herzl, Theodor 2.\,5.\,1860 Budapest – 3.\,7.\,1904 Edlach@\textsc{Herzl, Theodor} (2.\,5.\,1860 Budapest – 3.\,7.\,1904 Edlach), \emph{Schriftsteller, Journalist}|pwk} bat Schnitzler, in Bezug auf das Schauspiel \emph{Das neue Ghetto}\pwindex{Herzl, Theodor 2.\,5.\,1860 Budapest – 3.\,7.\,1904 Edlach@\textsc{Herzl, Theodor} (2.\,5.\,1860 Budapest – 3.\,7.\,1904 Edlach), \emph{Schriftsteller, Journalist}!neue Ghetto. Schauspiel in vier Acten@\strich\emph{Das neue Ghetto. Schauspiel in vier Acten}|pwk} und seine Einreichung auch in der
                  Korrespondenz die Anonymität streng zu wahren und alles diesbezügliche zu
                  verklausulieren, vgl. XXXX Auszeichnungsfehler: Dokument L03836 nicht gefunden. Die
                     Ablehnung des Stückes\pwindex{Herzl, Theodor 2.\,5.\,1860 Budapest – 3.\,7.\,1904 Edlach@\textsc{Herzl, Theodor} (2.\,5.\,1860 Budapest – 3.\,7.\,1904 Edlach), \emph{Schriftsteller, Journalist}!neue Ghetto. Schauspiel in vier Acten@\strich\emph{Das neue Ghetto. Schauspiel in vier Acten}|pwkv} durch ein Theater sollte Schnitzler per Telegramm durch das Codewort
                  »Subskriptionsresultat« vermelden, vgl. XXXX Auszeichnungsfehler: Dokument L03844 nicht gefunden.}}}\label{K_L03849-1}.\pend
           
\pstart
           Ihre Aufgabe nähert sich ihrem Ende, und mein armes Manuscript\pwindex{Herzl, Theodor 2.\,5.\,1860 Budapest – 3.\,7.\,1904 Edlach@\textsc{Herzl, Theodor} (2.\,5.\,1860 Budapest – 3.\,7.\,1904 Edlach), \emph{Schriftsteller, Journalist}!neue Ghetto. Schauspiel in vier Acten@\strich\emph{Das neue Ghetto. Schauspiel in vier Acten}|pwv} dem Grabe.\pend
           
\pstart
           Ich bitte Sie jetzt noch um Folgendes. Sobald Sie das Mscpt\pwindex{Herzl, Theodor 2.\,5.\,1860 Budapest – 3.\,7.\,1904 Edlach@\textsc{Herzl, Theodor} (2.\,5.\,1860 Budapest – 3.\,7.\,1904 Edlach), \emph{Schriftsteller, Journalist}!neue Ghetto. Schauspiel in vier Acten@\strich\emph{Das neue Ghetto. Schauspiel in vier Acten}|pwv} von Berlin\oindex{Berlin@\textbf{Berlin}, \emph{Hauptstadt}|pw} erhalten haben, bitte
               ich Sie damit zu Müller Guttenbrunn\pwindex{Müller-Guttenbrunn, Adam 22.\,10.\,1852 Zăbrani – 5.\,1.\,1923 Wien@\textsc{Müller-Guttenbrunn, Adam} (22.\,10.\,1852 Zăbrani – 5.\,1.\,1923 Wien), \emph{Schriftsteller, Theaterleiter, Beamter}|pw} zu gehen.
               Er ist so unausstehlich, dass Sie mit ihm vielleicht in schlechten Beziehungen sind;
               aber so anständig dass er das Manuscript\pwindex{Herzl, Theodor 2.\,5.\,1860 Budapest – 3.\,7.\,1904 Edlach@\textsc{Herzl, Theodor} (2.\,5.\,1860 Budapest – 3.\,7.\,1904 Edlach), \emph{Schriftsteller, Journalist}!neue Ghetto. Schauspiel in vier Acten@\strich\emph{Das neue Ghetto. Schauspiel in vier Acten}|pwv} selbst eines
               Feindes aufmerksam behandeln wird. Sollte freilich Ernstes vorliegen, so will ich Sie
               nicht mit dem Auftrag belästigen, sondern Schick\pwindex{Schik, Friedrich *~6.\,9.\,1857 Wien@\textsc{Schik, Friedrich} (*~6.\,9.\,1857 Wien), \emph{Notar, Journalist, Dramaturg}|pw}{ }{\pb}bitten, zu ihm zu gehen.\pend
           
\pstart
           Sie oder
               Schick\pwindex{Schik, Friedrich *~6.\,9.\,1857 Wien@\textsc{Schik, Friedrich} (*~6.\,9.\,1857 Wien), \emph{Notar, Journalist, Dramaturg}|pw} wollen gütigst Folgendes sagen: der in München\oindex{München@\textbf{München}|pw} lebende Verfasser hat das Stück dem
                  Deutschen\orgindex{Deutsches Theater Berlin@Deutsches Theater Berlin|pw}, dann dem Lessingtheater\orgindex{Lessing-Theater@Lessing-Theater|pw} in Berlin\oindex{Berlin@\textbf{Berlin}, \emph{Hauptstadt}|pw} eingereicht.
                  Beide haben abgelehnt. Jetzt gibt er es dem Raimund-Th\orgindex{Raimund-Theater@Raimund-Theater|pw}.
                  Er ist aber vom Warten schon so entnervt, dass er sich den Bescheid innerhalb \uline{acht} Tagen erbittet.\pend
           
\pstart
           Wenn Sie gehen, lassen Sie sich
                        von Müller\pwindex{Müller-Guttenbrunn, Adam 22.\,10.\,1852 Zăbrani – 5.\,1.\,1923 Wien@\textsc{Müller-Guttenbrunn, Adam} (22.\,10.\,1852 Zăbrani – 5.\,1.\,1923 Wien), \emph{Schriftsteller, Theaterleiter, Beamter}|pw} das Ehrenwort geben, dass er Niemandem
                  Sie als Einreicher nennt. Mein Grund dafür ist, dass man nach der – allerdings
                  nicht mehr recht zu erhoffenden – Aufführung durch \strikeout{Sie} unseren Verkehr auf die richtige Fährte kommen könnte.\pend
           
\pstart
           Wenn Schick\pwindex{Schik, Friedrich *~6.\,9.\,1857 Wien@\textsc{Schik, Friedrich} (*~6.\,9.\,1857 Wien), \emph{Notar, Journalist, Dramaturg}|pw} geht, ist diese Vorsicht {\pb}überflüssig, da ich ihn nicht kenne
                  u. er mich nicht.\pend
           
\pstart
           \introOben{}Aber Ihr Gang hat natürlich mehr Autorität.\introOben{} So kann
                     ich Ende der nächsten Woche im Reinen sein.\pend
           
\pstart
           Von Müller\pwindex{Müller-Guttenbrunn, Adam 22.\,10.\,1852 Zăbrani – 5.\,1.\,1923 Wien@\textsc{Müller-Guttenbrunn, Adam} (22.\,10.\,1852 Zăbrani – 5.\,1.\,1923 Wien), \emph{Schriftsteller, Theaterleiter, Beamter}|pw}
                        abgelehnt, geht das Stück\pwindex{Herzl, Theodor 2.\,5.\,1860 Budapest – 3.\,7.\,1904 Edlach@\textsc{Herzl, Theodor} (2.\,5.\,1860 Budapest – 3.\,7.\,1904 Edlach), \emph{Schriftsteller, Journalist}!neue Ghetto. Schauspiel in vier Acten@\strich\emph{Das neue Ghetto. Schauspiel in vier Acten}|pwv} auf seine letzte Reise: nach
                     Prag\oindex{Prag@\textbf{Prag}, \emph{Land}|pw}. Dort lebt mir ein Freund, das ist Teweles\pwindex{Teweles, Heinrich 13.\,11.\,1856 Prag – 9.\,8.\,1927 Prein an der Rax@\textsc{Teweles, Heinrich} (13.\,11.\,1856 Prag – 9.\,8.\,1927 Prein an der Rax), \emph{Schriftsteller, Journalist, Theaterleiter}|pw}, der
               Dramaturg des Prager Theaters\orgindex{Neues Deutsches Theater@Neues Deutsches Theater|pw}. Den müsste man einweihen.\pend
           
\pstart
           Dem würde ich
                  sobald ich Ihre Subscriptionsresultattdepesche habe schreiben, ihm das Silentium
                  ehrenwörtlich abnehmen und Sie dann bitten, ihm das Mscpt\pwindex{Herzl, Theodor 2.\,5.\,1860 Budapest – 3.\,7.\,1904 Edlach@\textsc{Herzl, Theodor} (2.\,5.\,1860 Budapest – 3.\,7.\,1904 Edlach), \emph{Schriftsteller, Journalist}!neue Ghetto. Schauspiel in vier Acten@\strich\emph{Das neue Ghetto. Schauspiel in vier Acten}|pwv} zu schicken u. auch weiterhin im Verkehr mit ihm zu bleiben, weil
                  zu mir keine Spur aus den Theaterkanzleien führen soll.\pend
           
\pstart
           Einer der Gründe zweiter
                  Klasse für mein Anonymat war, dass {\pb}ich – \label{K_L03849-2v}\edtext{\begin{otherlanguage}{french}à mon age\end{otherlanguage}}{\lemma{\textnormal{\emph{à mon age}}}\Cendnote{\textnormal{französisch: in meinem Alter}}}\label{K_L03849-2}! — keine Körbe von
                  den Directionen einstecken wollte. Das wenigstens habe ich erreicht.\pend
           
\pstart
           Es bleibt das
                  für junge Autoren, wie es unser fingirter Albert ist, interessante Problem, ob die
                  Sache eine andere Wendung genommen hätte, wenn mein Journalisten-Name bekannt
                  gewesen wäre.\pend
           
\pstart
           Vielleicht ist das Stück\pwindex{Herzl, Theodor 2.\,5.\,1860 Budapest – 3.\,7.\,1904 Edlach@\textsc{Herzl, Theodor} (2.\,5.\,1860 Budapest – 3.\,7.\,1904 Edlach), \emph{Schriftsteller, Journalist}!neue Ghetto. Schauspiel in vier Acten@\strich\emph{Das neue Ghetto. Schauspiel in vier Acten}|pwv} so schlecht,
                  dass nicht einmal die Zeitungspression was geholfen hätte. Sie haben mir
                  vielleicht nicht das Richtige über mein Stück\pwindex{Herzl, Theodor 2.\,5.\,1860 Budapest – 3.\,7.\,1904 Edlach@\textsc{Herzl, Theodor} (2.\,5.\,1860 Budapest – 3.\,7.\,1904 Edlach), \emph{Schriftsteller, Journalist}!neue Ghetto. Schauspiel in vier Acten@\strich\emph{Das neue Ghetto. Schauspiel in vier Acten}|pwv} gesagt.
                  Geben Sie acht, ich werde Ihnen zum Schluss noch Vorwürfe machen – –\pend
           
\pstart
           Soll ich also
                  meine erbärmliche Feder zerbrechen?\pend
           
\pstart
           Nein, sie kostet zehn \begin{otherlanguage}{french}francs 50c\end{otherlanguage}.{[},{]} ist von
                  Gold, der Stiel mit Tinte angefüllt, so dass man nicht einzutauchen braucht. –
                  Aber untertauchen!\pend
           \pstart Herzlich Ihr Freund \spacefill\mbox{Th H.}\pend{}\selectlanguage{ngerman}\endnumbering\briefempfaengerindex{Schnitzler, Arthur@\textsc{Schnitzler, Arthur}!zzzHerzl, Theodor@\emph{von Theodor Herzl}!1895-02-183@{18. 2. 1895}|)be}\mylabel{L03849h}
\begin{anhang}
\end{anhang}\newcommand{\dateiname}{L03849}\newcommand{\titel}{Theodor Herzl an Arthur Schnitzler, 18. 2. 1895}\newcommand{\editorInnen}{Selma Jahnke und Martin Anton Müller}%% latex-leseansicht-abspann.tex
%% Abspann für die Leseansicht.
%% Der Schalter \ifkorrekturansicht ist bereits durch den Vorspann gesetzt.

%% latex-abspann.tex
%% Gemeinsamer Abspann für Korrekturansicht und Leseansicht.
%% Setzt den Schalter \ifkorrekturansicht voraus (gesetzt in den
%% einbindenden Dateien latex-korrekturansicht-abspann.tex bzw.
%% latex-leseansicht-abspann.tex).
%% ---------------------------------------------------------------

\normalsize

% Das esempio-Environment wird nur in der Leseansicht benötigt
\ifkorrekturansicht\else
\newenvironment{esempio}[3]%
{
    \vspace{1.5ex}
    \rlap{\underline{#1}}
    \par
    \setlength{\parindent}{0cm}
    \nopagebreak
    \leftskip=#2cm
    \rightskip=#3cm
}
{
    \par
}
\fi

\doendnotes{C}
\bigskip
\vfill

\clearpage

\footnotesize

\ifkorrekturansicht
  \lohead{\textsc{register}}
\fi

% theindex-Environment neu definieren ohne reledmac
\makeatletter
\renewenvironment{theindex}{%
  \ifkorrekturansicht
    \section*{\indexname}%
  \else
    \subsubsection*{Index der erwähnten Entitäten}%
  \fi
  \setlength{\parindent}{0pt}%
  \setlength{\parskip}{0pt plus 0.3pt}%
  \let\item\@idxitem
}{%
  \ifkorrekturansicht\clearpage\fi
}
\makeatother

\IfFileExists{\jobname-pw.ind}{\input{\jobname-pw.ind}}{}

% Quellenangabe nur in der Leseansicht
\ifkorrekturansicht\else
% Fallback-Definitionen, falls die .tex-Datei \titel etc. nicht gesetzt hat
\providecommand{\titel}{}
\providecommand{\editorInnen}{}
\providecommand{\dateiname}{\jobname}

\vspace{3cm}

\vfill

\footnotesize
\textsc{Quelle}: \titel. Herausgegeben von {\editorInnen}. In: \emph{Arthur Schnitzler: Briefwechsel mit Autorinnen und Autoren}.
 Digitale Edition, https://schnitzler-briefe.acdh.oeaw.ac.at/{\dateiname}.html (Stand \today)
\fi

\end{document}


