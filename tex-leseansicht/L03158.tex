%% latex-korrekturansicht-vorspann.tex
%% Vorspann für die Korrekturansicht.
%% Lädt die gemeinsame Datei latex-vorspann.tex mit gesetztem Schalter.

\newif\ifkorrekturansicht
\korrekturansichttrue

\input{../tex-inputs/latex-vorspann}


\section[ Felix Salten an Arthur Schnitzler, 16. 7. {[}1895{]}]{L03158 Felix Salten an Arthur Schnitzler, 16. 7. {[}1895{]}}
\nopagebreak\mylabel{L03158v}
\rehead{ }\normalsize\beginnumbering\briefempfaengerindex{Schnitzler, Arthur@\textsc{Schnitzler, Arthur}!zzzSalten, Felix@\emph{von Felix Salten}!1895-07-161@{16. 7. {[}1895{]}}|(be}
\toendnotes[C]{\smallbreak\pagebreak[2]}\Standort{CUL, Schnitzler, B 89, A 1.}
\physDesc{Brief, 1 Blatt, 4 Seiten, 1050 Zeichen
\newline{}Handschrift: Bleistift, lateinische Kurrent
\newline{}Schnitzler: mit Bleistift die Jahreszahl ergänzt: »95« 
\newline{}Ordnung: mit Bleistift von unbekannter Hand nummeriert: »57« }\toendnotes[C]{\smallbreak}
\pstart
           \raggedleft{}{\pb}Montag, 16. VII.\pend
           \vspace{0.5em}
\pstart
           Lieber Arthur, so viel ich zu sagen hätte, so wenig
               hab’ ich zu schreiben, wie ja Sie auch. Nur so viel, dass es mir leidlich geht, dass
               ich einiges arbeite, und hie und da aufs Land fahre. Von Hugo\pwindex{Hofmannsthal, Hugo von 1874-02-01 – 1929-07-15@\textsc{Hofmannsthal, Hugo von} (1874-02-01 – 1929-07-15), \emph{Schriftsteller/Schriftstellerin}|pw} habe ich ein paarmal schöne Briefe gehabt, und habe ihm
               das zweite Heft des Pan\pwindex{Pan@\emph{Pan}|pw} gesendet, welches soeben
                  {\pb}erschienen, seine \label{K_L03158-1v}\edtext{Terzinen\pwindex{Terzinen@\emph{Terzinen}|pw}}{\lemma{\textnormal{\emph{Terzinen}}}\Cendnote{\textnormal{Loris\pwindex{Hofmannsthal, Hugo von 1874-02-01 – 1929-07-15@\textsc{Hofmannsthal, Hugo von} (1874-02-01 – 1929-07-15), \emph{Schriftsteller/Schriftstellerin}|pwk}: \emph{Terzinen}\pwindex{Terzinen@\emph{Terzinen}|pwk}. In: \emph{Pan}\pwindex{Pan@\emph{Pan}|pwk}, H. 2,
                        Juni, Juli, August 1895, S. 86–88.}}}\label{K_L03158-1} bringt. Ich
               mühe mich in \label{K_L03158-2v}\edtext{Umständen}{\lemma{\textnormal{\emph{Umständen}}}\Cendnote{\textnormal{Es dürfte sich um eine Bezugnahme auf die schwierige Beziehung
                  mit Charlotte Glas\pwindex{Pohl-Glas, Charlotte 1873-01-01 – 1944-02-15@\textsc{Pohl-Glas, Charlotte} (1873-01-01 – 1944-02-15), \emph{Schriftsteller/Schriftstellerin, Politiker/Politikerin, Sozialist/Sozialistin}|pwk} handeln (vgl. Felix Salten an Arthur Schnitzler, 22. 7. 1895).}}}\label{K_L03158-2}, die Sie ja
               kennen, und trachte \strikeout{\textcolor{gray}{nur}}, so wenig Kräfte zu verbrauchen als möglich. Das hindert nicht, dass mir
               darüber manche Stunden vergehen, die ich besser hätte anwenden
                  können\textcolor{gray}{.}\pend
           
\pstart
           Ich möchten gerne wissen, wie es mit \label{K_L03158-3v}\edtext{Kopenhagen\oindex{Kopenhagen@\textbf{Kopenhagen}, \emph{P.PPLC}|pw}}{\lemma{\textnormal{\emph{Kopenhagen}}}\Cendnote{\textnormal{Zu Schnitzlers erster Skandinavien\oindex{Skandinavien@\textbf{Skandinavien}, \emph{Region}|pwk}reise
                  kam es erst ein Jahr später, im August 1896, aber ohne
                     Salten\pwindex{Salten, Felix 06.09.1869 – 08.10.1945@\textsc{Salten, Felix} (06.09.1869 – 08.10.1945), \emph{Schriftsteller/Schriftstellerin, Journalist/Journalistin, Chefredakteur/Chefredakteurin}|pwk}, dafür mit Paul Goldmann\pwindex{Goldmann, Paul 31.01.1865 – 25.09.1935@\textsc{Goldmann, Paul} (31.01.1865 – 25.09.1935), \emph{Schriftsteller/Schriftstellerin, Journalist/Journalistin}|pwk} und Richard Beer-Hofmann\pwindex{Beer-Hofmann, Richard 1866-07-11 – 1945-09-26@\textsc{Beer-Hofmann, Richard} (1866-07-11 – 1945-09-26), \emph{Schriftsteller/Schriftstellerin}|pwk}.}}}\label{K_L03158-3}{ }{\pb}steht. Ich möchte das gerne
               bald und genau wissen, weil ich mich danach einrichten muss. Vielleicht können Sie
               mir jetzt schon etwas darüber mitttheilen. Fährt B-H.\pwindex{Beer-Hofmann, Richard 1866-07-11 – 1945-09-26@\textsc{Beer-Hofmann, Richard} (1866-07-11 – 1945-09-26), \emph{Schriftsteller/Schriftstellerin}|pw}, von dem ich Nichts höre, auch? Ich habe ihm, \strikeout{auf} wie die \label{K_L03158-4v}\edtext{L.\pwindex{Pohl-Glas, Charlotte 1873-01-01 – 1944-02-15@\textsc{Pohl-Glas, Charlotte} (1873-01-01 – 1944-02-15), \emph{Schriftsteller/Schriftstellerin, Politiker/Politikerin, Sozialist/Sozialistin}|pwu}}{\lemma{\textnormal{\emph{L.}}}\Cendnote{\textnormal{vermutlich 
                  Lotte
                     Glas\pwindex{Pohl-Glas, Charlotte 1873-01-01 – 1944-02-15@\textsc{Pohl-Glas, Charlotte} (1873-01-01 – 1944-02-15), \emph{Schriftsteller/Schriftstellerin, Politiker/Politikerin, Sozialist/Sozialistin}|pwk}}}}\label{K_L03158-4} mir ausgerichtet, den \label{K_L03158-5v}\edtext{Wurstelprater\pwindex{Quer durch den Wurstelprater@\emph{Quer durch den Wurstelprater}|pw}}{\lemma{\textnormal{\emph{Wurstelprater}}}\Cendnote{\textnormal{Felix Salten\pwindex{Salten, Felix 06.09.1869 – 08.10.1945@\textsc{Salten, Felix} (06.09.1869 – 08.10.1945), \emph{Schriftsteller/Schriftstellerin, Journalist/Journalistin, Chefredakteur/Chefredakteurin}|pwk}:
                           \emph{Quer durch den Wurstelprater}\pwindex{Quer durch den Wurstelprater@\emph{Quer durch den Wurstelprater}|pwk}. In:
                           \emph{Wiener Allgemeine Zeitung}\pwindex{Wiener Allgemeine Zeitung@\emph{Wiener Allgemeine Zeitung}|pwk}, Jg. 16, Nr. 5174,
                           2. 6. 1895, Pfingst-Beilage, S. [1]–[4] und Nr. 5179,
                           9. 6. 1895, S. 2–4 
                           (Illustrationen von
                           Theo Zasche\pwindex{Zasche, Theodor 18.10.1862 – 15.11.1922@\textsc{Zasche, Theodor} (18.10.1862 – 15.11.1922), \emph{Zeichner/Zeichnerin, Karikaturist/Karikaturistin}|pwk}). }}}\label{K_L03158-5} geschickt, aber ich weiss nicht, ob er {\pb}ihn erhalten hat. Also bitte,
               theilen Sie mir mit, ob es mit Kphg\oindex{Kopenhagen@\textbf{Kopenhagen}, \emph{P.PPLC}|pw}. etwas ist,
               weil ich ja doch etwas anfangen möchte.\pend
           
\pstart
           Grüßen Sie Beer-Hofmann\pwindex{Beer-Hofmann, Richard 1866-07-11 – 1945-09-26@\textsc{Beer-Hofmann, Richard} (1866-07-11 – 1945-09-26), \emph{Schriftsteller/Schriftstellerin}|pw}\textcolor{gray}{,} herzlichst Ihr {\\[\baselineskip]}\spacefill\mbox{Salten}\pend
           \leftskip=0em{}\selectlanguage{ngerman}\endnumbering\briefempfaengerindex{Schnitzler, Arthur@\textsc{Schnitzler, Arthur}!zzzSalten, Felix@\emph{von Felix Salten}!1895-07-161@{16. 7. {[}1895{]}}|)be}\mylabel{L03158h}  \normalsize

\doendnotes{C}
\bigskip
\vfill

\clearpage

\footnotesize

\lohead{\textsc{register}}

% Definiere theindex-Environment komplett neu ohne reledmac
\makeatletter
\renewenvironment{theindex}{%
  \section*{\indexname}%
  \setlength{\parindent}{0pt}%
  \setlength{\parskip}{0pt plus 0.3pt}%
  \let\item\@idxitem
}{%
  \clearpage
}
\makeatother

\IfFileExists{\jobname-pw.ind}{\input{\jobname-pw.ind}}{}

\end{document}

      