%% latex-leseansicht-vorspann.tex
%% Vorspann für die Leseansicht.
%% Lädt die gemeinsame Datei latex-vorspann.tex mit nicht gesetztem Schalter.

\newif\ifkorrekturansicht
\korrekturansichtfalse

\input{../tex-inputs/latex-vorspann}

\begin{center}
            \textcolor{red}{ENTWURF, NICHT FERTIG KORRIGIERT}
                      \end{center}
            
         
         \renewcommand{\erwaehntePersonen}{Personen: Richard Beer-Hofmann, Hugo von Hofmannsthal, Charlotte Pohl-Glas}
         \renewcommand{\erwaehnteOrte}{Orte: Bad Ischl, Kopenhagen, Wien}
         \renewcommand{\erwaehnteWerke}{Werke: Pan, Quer durch den Wurstelprater, Terzinen}
               \section[Felix Salten an Arthur Schnitzler, 16. 7. {[}1895{]}]{ Felix Salten an Arthur Schnitzler, 16. 7. {[}1895{]}}\nopagebreak\mylabel{v}\rehead{ }\begin{ledgroupsized}[t]{13cm}\normalsize\beginnumbering \toendnotes[C]{\smallbreak\pagebreak[2]} \Standort{CUL, Schnitzler, B 89, A 1.}
\physDesc{Brief, 1 Blatt, 4 Seiten
\newline{}Handschrift: Bleistift, lateinische Kurrent
\newline{}Schnitzler: mit Bleistift die Jahreszahl ergänzt: »95« }\toendnotes[C]{\smallbreak}\pstart
           \raggedleft{}{\pb}Montag, 16. VII.\pend
           \pstart
           Lieber Arthur, so viel ich zu sagen hätte, so wenig hab' ich zu
               schreiben, wie ja Sie auch. Nur so viel, dass es mir leidlich geht, dass ich einiges
               arbeite, und hie und da aufs Land fahre. Von Hugo\pwindex{Hofmannsthal, Hugo von 1874-02-01 – 1929-07-15@\textsc{Hofmannsthal, Hugo von} (1874-02-01 – 1929-07-15), \emph{Schriftsteller}|pw} habe ich ein paarmal schöne Briefe gehabt, und habe ihm das zweite Heft
               des Pan\pwindex{Pan1895 – 1915@\emph{Pan} {[}1895 – 1915{]}|pw} gesandt, welches soeben {\pb}erschienen, seine \label{K_L03158-1v}\edtext{Terzinen\pwindex{Terzinen1895-07-15@\emph{Terzinen} {[}1895-07-15{]}|pw}}{\lemma{\textnormal{\emph{Terzinen}}}\Cendnote{\textnormal{Loris\pwindex{Hofmannsthal, Hugo von 1874-02-01 – 1929-07-15@\textsc{Hofmannsthal, Hugo von} (1874-02-01 – 1929-07-15), \emph{Schriftsteller}|pwk}: \emph{Terzinen}\pwindex{Terzinen1895-07-15@\emph{Terzinen} {[}1895-07-15{]}|pwk}. In: \emph{Pan}\pwindex{Pan1895 – 1915@\emph{Pan} {[}1895 – 1915{]}|pwk}, H. 2,
                        Juni, Juli, August 1892, S. 86–88.}}}\label{K_L03158-1h} bringt. Ich
               mühe mich in Umständen, die Sie ja kennen, und trachte nur, so wenig Kräfte zu
               verbrauchen als möglich. Das hindert nicht, dass mir darüber manche Stunden vergehen,
               die ich besser hätte anwenden können. \pend
           \pstart
           Ich möchten gerne wissen, wie es mit Kopenhagen\oindex{Kopenhagen@\textbf{Kopenhagen}|pw}{ }{\pb}steht. Ich möchte das gerne
               bald und genau wissen, weil ich mich danach einrichten muss. Vielleicht können Sie
               mir jetzt schon etwas darüber mitttheilen. Fährt B-H.\pwindex{Beer-Hofmann, Richard 1866-07-11 – 1945-09-26@\textsc{Beer-Hofmann, Richard} (1866-07-11 – 1945-09-26), \emph{Schriftsteller}|pw}, von dann ich Nichts höre, auch? \pend
           \pstart
           Ich habe ihm, \strikeout{auf} wie die L.\pwindex{Pohl-Glas, Charlotte 1873-01-01 – 1944-02-15@\textsc{Pohl-Glas, Charlotte} (1873-01-01 – 1944-02-15), \emph{Schriftstellerin, Politikerin, Sozialistin}|pwu} mir ausgerichtet, den Wurstelprater\pwindex{Salten, Felix 06.09.1869 – 08.10.1945@\textsc{Salten, Felix} (06.09.1869 – 08.10.1945), \emph{Schriftsteller, Journalist}!Quer durch den Wurstelprater1895-06-02 – 1895-06-09@\strich\emph{Quer durch den Wurstelprater} {[}1895-06-02 – 1895-06-09{]}|pw} geschickt, aber ich weiss nicht, ob er {\pb}ihn erhalten hat. Also bitte,
               theilen Sie mir mit, ob es mit Kphg\oindex{Kopenhagen@\textbf{Kopenhagen}|pw}. etwas ist,
               weil ich ja doch etwas anfangen möchte. \pend
           \pstart
           Grüßen Sie Beer-Hofmann\pwindex{Beer-Hofmann, Richard 1866-07-11 – 1945-09-26@\textsc{Beer-Hofmann, Richard} (1866-07-11 – 1945-09-26), \emph{Schriftsteller}|pw}, {\\[\baselineskip]}herzlichst.
               Ihr {\\[\baselineskip]}\spacefill\mbox{Salten}\pend
           \leftskip=0em{}
         
         \endnumbering\mylabel{h}\end{ledgroupsized}\begin{anhang}\end{anhang}\newcommand{\dateiname}{L03158}\newcommand{\titel}{Felix Salten an Arthur Schnitzler, 16. 7. [1895]}\newcommand{\editorInnen}{Martin Anton Müller und Laura Untner}%% latex-leseansicht-abspann.tex
%% Abspann für die Leseansicht.
%% Der Schalter \ifkorrekturansicht ist bereits durch den Vorspann gesetzt.

%% latex-abspann.tex
%% Gemeinsamer Abspann für Korrekturansicht und Leseansicht.
%% Setzt den Schalter \ifkorrekturansicht voraus (gesetzt in den
%% einbindenden Dateien latex-korrekturansicht-abspann.tex bzw.
%% latex-leseansicht-abspann.tex).
%% ---------------------------------------------------------------

\normalsize

% Das esempio-Environment wird nur in der Leseansicht benötigt
\ifkorrekturansicht\else
\newenvironment{esempio}[3]%
{
    \vspace{1.5ex}
    \rlap{\underline{#1}}
    \par
    \setlength{\parindent}{0cm}
    \nopagebreak
    \leftskip=#2cm
    \rightskip=#3cm
}
{
    \par
}
\fi

\doendnotes{C}
\bigskip
\vfill

\clearpage

\footnotesize

\ifkorrekturansicht
  \lohead{\textsc{register}}
\fi

% theindex-Environment neu definieren ohne reledmac
\makeatletter
\renewenvironment{theindex}{%
  \ifkorrekturansicht
    \section*{\indexname}%
  \else
    \subsubsection*{Index der erwähnten Entitäten}%
  \fi
  \setlength{\parindent}{0pt}%
  \setlength{\parskip}{0pt plus 0.3pt}%
  \let\item\@idxitem
}{%
  \ifkorrekturansicht\clearpage\fi
}
\makeatother

\IfFileExists{\jobname-pw.ind}{\input{\jobname-pw.ind}}{}

% Quellenangabe nur in der Leseansicht
\ifkorrekturansicht\else
% Fallback-Definitionen, falls die .tex-Datei \titel etc. nicht gesetzt hat
\providecommand{\titel}{}
\providecommand{\editorInnen}{}
\providecommand{\dateiname}{\jobname}

\vspace{3cm}

\vfill

\footnotesize
\textsc{Quelle}: \titel. Herausgegeben von {\editorInnen}. In: \emph{Arthur Schnitzler: Briefwechsel mit Autorinnen und Autoren}.
 Digitale Edition, https://schnitzler-briefe.acdh.oeaw.ac.at/{\dateiname}.html (Stand \today)
\fi

\end{document}


      