%% latex-leseansicht-vorspann.tex
%% Vorspann für die Leseansicht.
%% Lädt die gemeinsame Datei latex-vorspann.tex mit nicht gesetztem Schalter.

\newif\ifkorrekturansicht
\korrekturansichtfalse

\input{../tex-inputs/latex-vorspann}


\section[ Felix Salten an Arthur Schnitzler, 16. 7. {[}1895{]}]{L03158 Felix Salten an Arthur Schnitzler,  16. 7. [1895]}
\nopagebreak\mylabel{L03158v}
\rehead{ }\normalsize\beginnumbering\briefempfaengerindex{Schnitzler, Arthur@\textsc{Schnitzler, Arthur}!zzzSalten, Felix@\emph{von Felix Salten}!1895-07-161@{16. 7. [1895]}|(be}
\toendnotes[C]{\smallbreak\pagebreak[2]}
\correspDesc{Versand  durch Felix Salten am 16. 7. [1895] in Wien
\newline{}Erhalt  durch Arthur Schnitzler im Zeitraum [17. 7. 1895
                  – 21. 7. 1895?] in Bad Ischl}\toendnotes[C]{\smallbreak}
\Standort{CUL, Schnitzler, B 89, A 1.}
\physDesc{Brief, 1 Blatt, 4 Seiten, 1050 Zeichen
\newline{}Handschrift: Bleistift, lateinische Kurrent
\newline{}Schnitzler: mit Bleistift die Jahreszahl ergänzt: »95« 
\newline{}Ordnung: mit Bleistift von unbekannter Hand nummeriert: »57« }\toendnotes[C]{\smallbreak}
\pstart
           \raggedleft{}{\pb}Montag, 16. VII.\pend
           \vspace{0.5em}
\pstart
           Lieber Arthur, so viel ich zu sagen hätte, so wenig
               hab’ ich zu schreiben, wie ja Sie auch. Nur so viel, dass es mir leidlich geht, dass
               ich einiges arbeite, und hie und da aufs Land fahre. Von Hugo\pwindex{Hofmannsthal, Hugo von 1.\,2.\,1874 Wien – 15.\,7.\,1929 Rodaun@\textsc{Hofmannsthal, Hugo von} (1.\,2.\,1874 Wien – 15.\,7.\,1929 Rodaun), \emph{Schriftsteller}|pw} habe ich ein paarmal schöne Briefe gehabt, und habe ihm
               das zweite Heft des Pan\pwindex{Pan@\emph{Pan}|pw} gesendet, welches soeben
                  {\pb}erschienen, seine \label{K_L03158-1v}\edtext{Terzinen\pwindex{Hofmannsthal, Hugo von 1.\,2.\,1874 Wien – 15.\,7.\,1929 Rodaun@\textsc{Hofmannsthal, Hugo von} (1.\,2.\,1874 Wien – 15.\,7.\,1929 Rodaun), \emph{Schriftsteller}!Terzinen@\strich\emph{Terzinen}|pw}}{\lemma{\textnormal{\emph{Terzinen}}}\Cendnote{\textnormal{Loris\pwindex{Hofmannsthal, Hugo von 1.\,2.\,1874 Wien – 15.\,7.\,1929 Rodaun@\textsc{Hofmannsthal, Hugo von} (1.\,2.\,1874 Wien – 15.\,7.\,1929 Rodaun), \emph{Schriftsteller}|pwk}: \emph{Terzinen}\pwindex{Hofmannsthal, Hugo von 1.\,2.\,1874 Wien – 15.\,7.\,1929 Rodaun@\textsc{Hofmannsthal, Hugo von} (1.\,2.\,1874 Wien – 15.\,7.\,1929 Rodaun), \emph{Schriftsteller}!Terzinen@\strich\emph{Terzinen}|pwk}. In: \emph{Pan}\pwindex{Pan@\emph{Pan}|pwk}, H. 2,
                        Juni, Juli, August 1895, S. 86–88.}}}\label{K_L03158-1} bringt. Ich
               mühe mich in \label{K_L03158-2v}\edtext{Umständen}{\lemma{\textnormal{\emph{Umständen}}}\Cendnote{\textnormal{Es dürfte sich um eine Bezugnahme auf die schwierige Beziehung
                  mit Charlotte Glas\pwindex{Pohl-Glas, Charlotte 1.\,1.\,1873 Wien – 15.\,2.\,1944 Zürich@\textsc{Pohl-Glas, Charlotte} (1.\,1.\,1873 Wien – 15.\,2.\,1944 Zürich), \emph{Schriftstellerin, Politikerin, Sozialistin}|pwk} handeln (vgl. XXXX Auszeichnungsfehler: Dokument L03159 nicht gefunden).}}}\label{K_L03158-2}, die Sie ja
               kennen, und trachte \strikeout{\textcolor{gray}{nur}}, so wenig Kräfte zu verbrauchen als möglich. Das hindert nicht, dass mir
               darüber manche Stunden vergehen, die ich besser hätte anwenden
                  können\textcolor{gray}{.}\pend
           
\pstart
           Ich möchten gerne wissen, wie es mit \label{K_L03158-3v}\edtext{Kopenhagen\oindex{Kopenhagen@\textbf{Kopenhagen}, \emph{Hauptstadt}|pw}}{\lemma{\textnormal{\emph{Kopenhagen}}}\Cendnote{\textnormal{Zu Schnitzlers erster Skandinavien\oindex{Skandinavien@\textbf{Skandinavien}|pwk}reise
                  kam es erst ein Jahr später, im August 1896, aber ohne
                     Salten\pwindex{Salten, Felix 6.\,9.\,1869 Budapest – 8.\,10.\,1945 Zürich@\textsc{Salten, Felix} (6.\,9.\,1869 Budapest – 8.\,10.\,1945 Zürich), \emph{Schriftsteller, Journalist, Chefredakteur}|pwk}, dafür mit Paul Goldmann\pwindex{Goldmann, Paul 31.\,1.\,1865 Breslau – 25.\,9.\,1935 Wien@\textsc{Goldmann, Paul} (31.\,1.\,1865 Breslau – 25.\,9.\,1935 Wien), \emph{Schriftsteller, Journalist}|pwk} und Richard Beer-Hofmann\pwindex{Beer-Hofmann, Richard 11.\,7.\,1866 Wien – 26.\,9.\,1945 New York City@\textsc{Beer-Hofmann, Richard} (11.\,7.\,1866 Wien – 26.\,9.\,1945 New York City), \emph{Schriftsteller}|pwk}.}}}\label{K_L03158-3}{ }{\pb}steht. Ich möchte das gerne
               bald und genau wissen, weil ich mich danach einrichten muss. Vielleicht können Sie
               mir jetzt schon etwas darüber mitttheilen. Fährt B-H.\pwindex{Beer-Hofmann, Richard 11.\,7.\,1866 Wien – 26.\,9.\,1945 New York City@\textsc{Beer-Hofmann, Richard} (11.\,7.\,1866 Wien – 26.\,9.\,1945 New York City), \emph{Schriftsteller}|pw}, von dem ich Nichts höre, auch? Ich habe ihm, \strikeout{auf} wie die \label{K_L03158-4v}\edtext{L.\pwindex{Pohl-Glas, Charlotte 1.\,1.\,1873 Wien – 15.\,2.\,1944 Zürich@\textsc{Pohl-Glas, Charlotte} (1.\,1.\,1873 Wien – 15.\,2.\,1944 Zürich), \emph{Schriftstellerin, Politikerin, Sozialistin}|pwu}}{\lemma{\textnormal{\emph{L.}}}\Cendnote{\textnormal{vermutlich 
                  Lotte
                     Glas\pwindex{Pohl-Glas, Charlotte 1.\,1.\,1873 Wien – 15.\,2.\,1944 Zürich@\textsc{Pohl-Glas, Charlotte} (1.\,1.\,1873 Wien – 15.\,2.\,1944 Zürich), \emph{Schriftstellerin, Politikerin, Sozialistin}|pwk}}}}\label{K_L03158-4} mir ausgerichtet, den \label{K_L03158-5v}\edtext{Wurstelprater\pwindex{Salten, Felix 6.\,9.\,1869 Budapest – 8.\,10.\,1945 Zürich@\textsc{Salten, Felix} (6.\,9.\,1869 Budapest – 8.\,10.\,1945 Zürich), \emph{Schriftsteller, Journalist, Chefredakteur}!Quer durch den Wurstelprater@\strich\emph{Quer durch den Wurstelprater}|pw}}{\lemma{\textnormal{\emph{Wurstelprater}}}\Cendnote{\textnormal{Felix Salten\pwindex{Salten, Felix 6.\,9.\,1869 Budapest – 8.\,10.\,1945 Zürich@\textsc{Salten, Felix} (6.\,9.\,1869 Budapest – 8.\,10.\,1945 Zürich), \emph{Schriftsteller, Journalist, Chefredakteur}|pwk}:
                           \emph{Quer durch den Wurstelprater}\pwindex{Salten, Felix 6.\,9.\,1869 Budapest – 8.\,10.\,1945 Zürich@\textsc{Salten, Felix} (6.\,9.\,1869 Budapest – 8.\,10.\,1945 Zürich), \emph{Schriftsteller, Journalist, Chefredakteur}!Quer durch den Wurstelprater@\strich\emph{Quer durch den Wurstelprater}|pwk}. In:
                           \emph{Wiener Allgemeine Zeitung}\pwindex{Wiener Allgemeine Zeitung@\emph{Wiener Allgemeine Zeitung}|pwk}, Jg. 16, Nr. 5174,
                           2.\,6.\,1895, Pfingst-Beilage, S. [1]–[4] und Nr. 5179,
                           9.\,6.\,1895, S. 2–4 
                           (Illustrationen von
                           Theo Zasche\pwindex{Zasche, Theodor 18.\,10.\,1862 Wien – 15.\,11.\,1922 ebd.@\textsc{Zasche, Theodor} (18.\,10.\,1862 Wien – 15.\,11.\,1922 ebd.), \emph{Zeichner, Karikaturist}|pwk}). }}}\label{K_L03158-5} geschickt, aber ich weiss nicht, ob er {\pb}ihn erhalten hat. Also bitte,
               theilen Sie mir mit, ob es mit Kphg\oindex{Kopenhagen@\textbf{Kopenhagen}, \emph{Hauptstadt}|pw}. etwas ist,
               weil ich ja doch etwas anfangen möchte.\pend
           
\pstart
           Grüßen Sie Beer-Hofmann\pwindex{Beer-Hofmann, Richard 11.\,7.\,1866 Wien – 26.\,9.\,1945 New York City@\textsc{Beer-Hofmann, Richard} (11.\,7.\,1866 Wien – 26.\,9.\,1945 New York City), \emph{Schriftsteller}|pw}\textcolor{gray}{,} herzlichst Ihr {\\[\baselineskip]}\spacefill\mbox{Salten}\pend
           \leftskip=0em{}\selectlanguage{ngerman}\endnumbering\briefempfaengerindex{Schnitzler, Arthur@\textsc{Schnitzler, Arthur}!zzzSalten, Felix@\emph{von Felix Salten}!1895-07-161@{16. 7. [1895]}|)be}\mylabel{L03158h}  \newcommand{\dateiname}{L03158}\newcommand{\titel}{Felix Salten an Arthur Schnitzler, 16. 7. [1895]}\newcommand{\editorInnen}{Martin Anton Müller und Laura Untner}%% latex-leseansicht-abspann.tex
%% Abspann für die Leseansicht.
%% Der Schalter \ifkorrekturansicht ist bereits durch den Vorspann gesetzt.

%% latex-abspann.tex
%% Gemeinsamer Abspann für Korrekturansicht und Leseansicht.
%% Setzt den Schalter \ifkorrekturansicht voraus (gesetzt in den
%% einbindenden Dateien latex-korrekturansicht-abspann.tex bzw.
%% latex-leseansicht-abspann.tex).
%% ---------------------------------------------------------------

\normalsize

% Das esempio-Environment wird nur in der Leseansicht benötigt
\ifkorrekturansicht\else
\newenvironment{esempio}[3]%
{
    \vspace{1.5ex}
    \rlap{\underline{#1}}
    \par
    \setlength{\parindent}{0cm}
    \nopagebreak
    \leftskip=#2cm
    \rightskip=#3cm
}
{
    \par
}
\fi

\doendnotes{C}
\bigskip
\vfill

\clearpage

\footnotesize

\ifkorrekturansicht
  \lohead{\textsc{register}}
\fi

% theindex-Environment neu definieren ohne reledmac
\makeatletter
\renewenvironment{theindex}{%
  \ifkorrekturansicht
    \section*{\indexname}%
  \else
    \subsubsection*{Index der erwähnten Entitäten}%
  \fi
  \setlength{\parindent}{0pt}%
  \setlength{\parskip}{0pt plus 0.3pt}%
  \let\item\@idxitem
}{%
  \ifkorrekturansicht\clearpage\fi
}
\makeatother

\IfFileExists{\jobname-pw.ind}{\input{\jobname-pw.ind}}{}

% Quellenangabe nur in der Leseansicht
\ifkorrekturansicht\else
% Fallback-Definitionen, falls die .tex-Datei \titel etc. nicht gesetzt hat
\providecommand{\titel}{}
\providecommand{\editorInnen}{}
\providecommand{\dateiname}{\jobname}

\vspace{3cm}

\vfill

\footnotesize
\textsc{Quelle}: \titel. Herausgegeben von {\editorInnen}. In: \emph{Arthur Schnitzler: Briefwechsel mit Autorinnen und Autoren}.
 Digitale Edition, https://schnitzler-briefe.acdh.oeaw.ac.at/{\dateiname}.html (Stand \today)
\fi

\end{document}


