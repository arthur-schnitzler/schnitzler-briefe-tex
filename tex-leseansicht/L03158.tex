%% latex-leseansicht-vorspann.tex
%% Vorspann für die Leseansicht.
%% Lädt die gemeinsame Datei latex-vorspann.tex mit nicht gesetztem Schalter.

\newif\ifkorrekturansicht
\korrekturansichtfalse

\input{../tex-inputs/latex-vorspann}


         
         \renewcommand{\erwaehntePersonen}{Personen: Richard Beer-Hofmann, Paul Goldmann, Hugo von Hofmannsthal, Charlotte Pohl-Glas, Theodor Zasche}
         \renewcommand{\erwaehnteOrte}{Orte: Bad Ischl, Kopenhagen, Skandinavien, Wien}
         \renewcommand{\erwaehnteWerke}{Werke: Pan, Quer durch den Wurstelprater, Terzinen, Wiener Allgemeine Zeitung}
               \section[ Felix Salten an Arthur Schnitzler, 16. 7. {[}1895{]}]{ Felix Salten an Arthur Schnitzler, 16. 7. {[}1895{]}}\nopagebreak\mylabel{v}\rehead{ }\begin{ledgroupsized}[t]{13cm}\normalsize\beginnumbering \toendnotes[C]{\smallbreak\pagebreak[2]} \Standort{CUL, Schnitzler, B 89, A 1.}
\physDesc{Brief, 1 Blatt, 4 Seiten, 1050 Zeichen
\newline{}Handschrift: Bleistift, lateinische Kurrent
\newline{}Schnitzler: mit Bleistift die Jahreszahl ergänzt: »95« 
\newline{}Ordnung: mit Bleistift von unbekannter Hand nummeriert: »57« }\toendnotes[C]{\smallbreak}\pstart
           \raggedleft{}{\pb}Montag, 16. VII.\pend
           \pstart
           Lieber Arthur, so viel ich zu sagen hätte, so wenig
               hab’ ich zu schreiben, wie ja Sie auch. Nur so viel, dass es mir leidlich geht, dass
               ich einiges arbeite, und hie und da aufs Land fahre. Von Hugo\pwindex{Hofmannsthal, Hugo von 1874-02-01 – 1929-07-15@\textsc{Hofmannsthal, Hugo von} (1874-02-01 – 1929-07-15), \emph{Schriftsteller}|pw} habe ich ein paarmal schöne Briefe gehabt, und habe ihm
               das zweite Heft des Pan\pwindex{Pan1895 – 1915@\emph{Pan} {[}1895 – 1915{]}|pw} gesendet, welches soeben
                  {\pb}erschienen, seine \label{K_L03158-1v}\edtext{Terzinen\pwindex{Terzinen1895-07-15@\emph{Terzinen} {[}1895-07-15{]}|pw}}{\lemma{\textnormal{\emph{Terzinen}}}\Cendnote{\textnormal{Loris\pwindex{Hofmannsthal, Hugo von 1874-02-01 – 1929-07-15@\textsc{Hofmannsthal, Hugo von} (1874-02-01 – 1929-07-15), \emph{Schriftsteller}|pwk}: \emph{Terzinen}\pwindex{Terzinen1895-07-15@\emph{Terzinen} {[}1895-07-15{]}|pwk}. In: \emph{Pan}\pwindex{Pan1895 – 1915@\emph{Pan} {[}1895 – 1915{]}|pwk}, H. 2,
                        Juni, Juli, August 1895, S. 86–88.}}}\label{K_L03158-1h} bringt. Ich
               mühe mich in \label{K_L03158-2v}\edtext{Umständen}{\lemma{\textnormal{\emph{Umständen}}}\Cendnote{\textnormal{Bezugnahme auf die die schwierige Beziehung
                  mit Charlotte Glas\pwindex{Pohl-Glas, Charlotte 1873-01-01 – 1944-02-15@\textsc{Pohl-Glas, Charlotte} (1873-01-01 – 1944-02-15), \emph{Schriftstellerin, Politikerin, Sozialistin}|pwk} (vgl. Felix Salten an Arthur Schnitzler, 22. 7. 1895)?}}}\label{K_L03158-2h}, die Sie ja
               kennen, und trachte \strikeout{\textcolor{gray}{nur}}, so wenig Kräfte zu verbrauchen als möglich. Das hindert nicht, dass mir
               darüber manche Stunden vergehen, die ich besser hätte anwenden
                  können\textcolor{gray}{.}\pend
           \pstart
           Ich möchten gerne wissen, wie es mit \label{K_L03158-3v}\edtext{Kopenhagen\oindex{Kopenhagen@\textbf{Kopenhagen}|pw}}{\lemma{\textnormal{\emph{Kopenhagen}}}\Cendnote{\textnormal{Zu Schnitzler\pwindex{Schnitzler, Arthur 15.05.1862 – 21.10.1931@\textsc{Schnitzler, Arthur} (15.05.1862 – 21.10.1931), \emph{Schriftsteller, Mediziner}|pwk}s erster Skandinavien\oindex{Skandinavien@\textbf{Skandinavien}|pwk}reise
                  kam es erst ein Jahr später, im August 1896, aber ohne
                     Salten\pwindex{Salten, Felix 06.09.1869 – 08.10.1945@\textsc{Salten, Felix} (06.09.1869 – 08.10.1945), \emph{Schriftsteller, Journalist}|pwk}, dafür mit Paul Goldmann\pwindex{Goldmann, Paul 31.01.1865 – 25.09.1935@\textsc{Goldmann, Paul} (31.01.1865 – 25.09.1935), \emph{Schriftsteller, Journalist}|pwk} und Richard Beer-Hofmann\pwindex{Beer-Hofmann, Richard 1866-07-11 – 1945-09-26@\textsc{Beer-Hofmann, Richard} (1866-07-11 – 1945-09-26), \emph{Schriftsteller}|pwk}.}}}\label{K_L03158-3h}{ }{\pb}steht. Ich möchte das gerne
               bald und genau wissen, weil ich mich danach einrichten muss. Vielleicht können Sie
               mir jetzt schon etwas darüber mitttheilen. Fährt B-H.\pwindex{Beer-Hofmann, Richard 1866-07-11 – 1945-09-26@\textsc{Beer-Hofmann, Richard} (1866-07-11 – 1945-09-26), \emph{Schriftsteller}|pw}, von dem ich Nichts höre, auch? Ich habe ihm, \strikeout{auf} wie die \label{K_L03158-4v}\edtext{L.\pwindex{Pohl-Glas, Charlotte 1873-01-01 – 1944-02-15@\textsc{Pohl-Glas, Charlotte} (1873-01-01 – 1944-02-15), \emph{Schriftstellerin, Politikerin, Sozialistin}|pwu}}{\lemma{\textnormal{\emph{L.}}}\Cendnote{\textnormal{Lotte
                     Glas\pwindex{Pohl-Glas, Charlotte 1873-01-01 – 1944-02-15@\textsc{Pohl-Glas, Charlotte} (1873-01-01 – 1944-02-15), \emph{Schriftstellerin, Politikerin, Sozialistin}|pwk}?}}}\label{K_L03158-4h} mir ausgerichtet, den \label{K_L03158-5v}\edtext{Wurstelprater\pwindex{Salten, Felix 06.09.1869 – 08.10.1945@\textsc{Salten, Felix} (06.09.1869 – 08.10.1945), \emph{Schriftsteller, Journalist}!Quer durch den Wurstelprater1895-06-02 – 1895-06-09@\strich\emph{Quer durch den Wurstelprater} {[}1895-06-02 – 1895-06-09{]}|pw}}{\lemma{\textnormal{\emph{Wurstelprater}}}\Cendnote{\textnormal{Felix Salten\pwindex{Salten, Felix 06.09.1869 – 08.10.1945@\textsc{Salten, Felix} (06.09.1869 – 08.10.1945), \emph{Schriftsteller, Journalist}|pwk}:
                           \emph{Quer durch den Wurstelprater}\pwindex{Salten, Felix 06.09.1869 – 08.10.1945@\textsc{Salten, Felix} (06.09.1869 – 08.10.1945), \emph{Schriftsteller, Journalist}!Quer durch den Wurstelprater1895-06-02 – 1895-06-09@\strich\emph{Quer durch den Wurstelprater} {[}1895-06-02 – 1895-06-09{]}|pwk}. In:
                           \emph{Wiener Allgemeine Zeitung}\pwindex{?? Werk@Nicht ermittelte Verfasserinnen und Verfasser!Wiener Allgemeine Zeitung1.3.1880 – 11.2.1934@\emph{Wiener Allgemeine Zeitung} {[}1.3.1880 – 11.2.1934{]}|pwk}, Jg. 16, Nr. 5.174,
                           2. 6. 1895, Pfingst-Beilage, S. [1]–[4] und Nr. 5.179,
                           9. 6. 1895, S. 2–4. 
                           (Mit Illustrationen von
                           Theo Zasche\pwindex{Zasche, Theodor 18.10.1862 – 15.11.1922@\textsc{Zasche, Theodor} (18.10.1862 – 15.11.1922), \emph{Zeichner, Karikaturist}|pwk}). }}}\label{K_L03158-5h} geschickt, aber ich weiss nicht, ob er {\pb}ihn erhalten hat. Also bitte,
               theilen Sie mir mit, ob es mit Kphg\oindex{Kopenhagen@\textbf{Kopenhagen}|pw}. etwas ist,
               weil ich ja doch etwas anfangen möchte.\pend
           \pstart
           Grüßen Sie Beer-Hofmann\pwindex{Beer-Hofmann, Richard 1866-07-11 – 1945-09-26@\textsc{Beer-Hofmann, Richard} (1866-07-11 – 1945-09-26), \emph{Schriftsteller}|pw}\textcolor{gray}{,} herzlichst Ihr {\\[\baselineskip]}\spacefill\mbox{Salten}\pend
           \leftskip=0em{}
         
         \endnumbering\mylabel{h}\end{ledgroupsized}  \newcommand{\dateiname}{L03158}\newcommand{\titel}{Felix Salten an Arthur Schnitzler, 16. 7. [1895]}\newcommand{\editorInnen}{Martin Anton Müller und Laura Untner}%% latex-leseansicht-abspann.tex
%% Abspann für die Leseansicht.
%% Der Schalter \ifkorrekturansicht ist bereits durch den Vorspann gesetzt.

%% latex-abspann.tex
%% Gemeinsamer Abspann für Korrekturansicht und Leseansicht.
%% Setzt den Schalter \ifkorrekturansicht voraus (gesetzt in den
%% einbindenden Dateien latex-korrekturansicht-abspann.tex bzw.
%% latex-leseansicht-abspann.tex).
%% ---------------------------------------------------------------

\normalsize

% Das esempio-Environment wird nur in der Leseansicht benötigt
\ifkorrekturansicht\else
\newenvironment{esempio}[3]%
{
    \vspace{1.5ex}
    \rlap{\underline{#1}}
    \par
    \setlength{\parindent}{0cm}
    \nopagebreak
    \leftskip=#2cm
    \rightskip=#3cm
}
{
    \par
}
\fi

\doendnotes{C}
\bigskip
\vfill

\clearpage

\footnotesize

\ifkorrekturansicht
  \lohead{\textsc{register}}
\fi

% theindex-Environment neu definieren ohne reledmac
\makeatletter
\renewenvironment{theindex}{%
  \ifkorrekturansicht
    \section*{\indexname}%
  \else
    \subsubsection*{Index der erwähnten Entitäten}%
  \fi
  \setlength{\parindent}{0pt}%
  \setlength{\parskip}{0pt plus 0.3pt}%
  \let\item\@idxitem
}{%
  \ifkorrekturansicht\clearpage\fi
}
\makeatother

\IfFileExists{\jobname-pw.ind}{\input{\jobname-pw.ind}}{}

% Quellenangabe nur in der Leseansicht
\ifkorrekturansicht\else
% Fallback-Definitionen, falls die .tex-Datei \titel etc. nicht gesetzt hat
\providecommand{\titel}{}
\providecommand{\editorInnen}{}
\providecommand{\dateiname}{\jobname}

\vspace{3cm}

\vfill

\footnotesize
\textsc{Quelle}: \titel. Herausgegeben von {\editorInnen}. In: \emph{Arthur Schnitzler: Briefwechsel mit Autorinnen und Autoren}.
 Digitale Edition, https://schnitzler-briefe.acdh.oeaw.ac.at/{\dateiname}.html (Stand \today)
\fi

\end{document}


      