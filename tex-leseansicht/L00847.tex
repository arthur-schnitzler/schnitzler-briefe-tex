\input{../tex-inputs/latex-pdf-vorspann}
\begin{center}
            \textcolor{red}{ENTWURF. ENTZIFFERUNG NOCH NICHT KORREKTURGELESEN}
                      \end{center}
            
               \section[Stefan Großmann an Arthur Schnitzler, 16. 9. 1898]{ Stefan Großmann an Arthur Schnitzler, 16. 9. 1898}\nopagebreak\mylabel{v}\rehead{ }\begin{ledgroupsized}[t]{13cm}\normalsize\beginnumbering\briefempfaengerindex{Schnitzler, Arthur@\textsc{Schnitzler, Arthur}!zzzGrossmann, Stefan@\emph{von Stefan Großmann}!1898-09-162@{16. 9. 1898}|(be} \toendnotes[C]{\smallbreak\pagebreak[2]} \Standort{CUL, Schnitzler, B 34.}
\physDesc{Brief, 1 Blatt, 2 Seiten
\newline{}Handschrift: schwarze Tinte, deutsche Kurrent
\newline{}Schnitzler: mit rotem Buntstift drei Unterstreichungen \newline{}Ordnung: mit Bleistift von unbekannter Hand nummeriert: »1« }\toendnotes[C]{\smallbreak}\pstart
           \noindent{}{\pb}\textcolor{gray}{\textbf{WIENER RUNDSCHAU\orgindex{Wiener Rundschau@Wiener Rundschau|pw}.}}\pend
           \pstart
           \textcolor{gray}{\textbf{HERAUSGEBER}}\pend
           \pstart
           \textcolor{gray}{\textbf{GUSTAV SCHOENAICH\pwindex{Schoenaich, Gustav 1840-11-24 – 1906-04-04@\textsc{Schönaich, Gustav} (1840-11-24 – 1906-04-04), \emph{Journalist}|pw}.}}\pend
           \pstart
           \textcolor{gray}{\textbf{FELIX RAPPAPORT\pwindex{Rappaport, Felix 1874-11-17 – 1939-12-08@\textsc{Rappaport, Felix} (1874-11-17 – 1939-12-08), \emph{Herausgeber, Finanzier}|pw}.}}\hfill \textcolor{gray}{\textbf{Wien\oindex{Wien@\textbf{Wien}|pw},}}{ }16. September \textcolor{gray}{\textbf{189}}8\pend
           \pstart
           \textcolor{gray}{\textbf{REDACTION UND ADMINISTRATION:}}\pend
           \pstart
           \textcolor{gray}{\textbf{WIEN\oindex{Wien@\textbf{Wien}|pw}}}\pend
           \pstart
           \textcolor{gray}{\textbf{I/1 SPIEGELGASSE 11\oindex{Spiegelgasse@\textbf{Spiegelgasse}|pw}.}}\pend
           \pstart
           \textcolor{gray}{\textbf{TELEPHON NR. 2579.}}\pend
           \pstart\center{}Sehr geehrter Herr Doctor!\pend\pstart
           Ich leſe in den Zeitungen von \introOben{}Ihren\introOben{} drei neuen Einactern\pwindex{Schnitzler, Arthur 15.05.1862 – 21.10.1931@\textsc{Schnitzler, Arthur} (15.05.1862 – 21.10.1931), \emph{Schriftsteller, Mediziner}!gruene Kakadu – Paracelsus – Die Gefaehrtin. Drei Einakter1.3.1899 – 1.3.1899@\strich\emph{Der grüne Kakadu – Paracelsus – Die Gefährtin. Drei Einakter} {[}1.3.1899 – 1.3.1899{]}|pwv}, die D\textsuperscript{r} \textsc{Brahm\pwindex{Brahm, Otto 05.02.1856 – 28.11.1912@\textsc{Brahm, Otto} (05.02.1856 – 28.11.1912), \emph{Theaterleiter, Regisseur}|pw}} im »Deutſchen Theater\orgindex{Deutsches Theater Berlin@Deutsches Theater Berlin|pw}« aufführen wird.\pend
           \pstart
           Darf ich Sie nochmals, aufrichtig und innigſt bitten, ob Sie mir einen von dieſen zum
               Abdruck in der »Rundſchau\orgindex{Wiener Rundschau@Wiener Rundschau|pw}« überlaſſen möchten? Ich
               gebe Ihnen die Verſicherung, daſs ich glücklich wäre, wenn Sie meine Bitte erfüllen
               würden, daſs ich von Tag zu Tag \strikeout{\textcolor{gray}{×}\-\textcolor{gray}{×}} mehr einſehe, wie bornirt, leicht-fertig meine \strikeout{Radi} literariſchen Radicalismen von ſeinerzeit waren.
               Ich brauche nur an die \uline{nach} Ihnen Kommenden zu denken
               u bin beſchämt.\pend
           \pstart
           Überdies würden Sie \substVorne{}\textsuperscript{ſich}\substDazwischen{}mich\substHinten{} hiedurch beſonders verpflichten, weil mir Ihre Gabe eine moraliſche Unter{\pb}ſtützung wäre, gerade jetzt beſonders
               werthvoll, wo die literariſchen Schwarzkünſtler aller Art meinem Herausgeber\pwindex{Schoenaich, Gustav 1840-11-24 – 1906-04-04@\textsc{Schönaich, Gustav} (1840-11-24 – 1906-04-04), \emph{Journalist}|pwv}\pwindex{Rappaport, Felix 1874-11-17 – 1939-12-08@\textsc{Rappaport, Felix} (1874-11-17 – 1939-12-08), \emph{Herausgeber, Finanzier}|pwv} in den
               Ohren liegen.\pend
           \pstart
           Verzeihen Sie, bitte, die Beläſtigung und erfüllen Sie – bitte – bald mein
               Anſuchen.\pend
           \pstart
           Ich bin{\\[\baselineskip]} Ihr \uline{sehr}{ }\uline{ergebener}{\\[\baselineskip]}\spacefill\mbox{Stefan Großmann}\pend
           \leftskip=0em{}\endnumbering\briefempfaengerindex{Schnitzler, Arthur@\textsc{Schnitzler, Arthur}!zzzGrossmann, Stefan@\emph{von Stefan Großmann}!1898-09-162@{16. 9. 1898}|)be}\mylabel{h}\end{ledgroupsized}  \newcommand{\dateiname}{L00847}\newcommand{\titel}{Stefan Großmann an Arthur Schnitzler, 16. 9. 1898}\newcommand{\editorInnen}{ Martin Anton Müller und Gerd-Hermann Susen}\input{../tex-inputs/latex-pdf-abspann}
      