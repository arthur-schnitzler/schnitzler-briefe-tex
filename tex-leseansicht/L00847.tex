%% latex-leseansicht-vorspann.tex
%% Vorspann für die Leseansicht.
%% Lädt die gemeinsame Datei latex-vorspann.tex mit nicht gesetztem Schalter.

\newif\ifkorrekturansicht
\korrekturansichtfalse

\input{../tex-inputs/latex-vorspann}


         
         \newcommand{\erwaehntePersonen}{Personen: }
         \newcommand{\erwaehnteInstitutionen}{}
         \newcommand{\erwaehnteOrte}{}
         \newcommand{\erwaehnteWerke}{
               \section[Stefan Großmann an Arthur Schnitzler, 16. 9. 1898]{ Stefan Großmann an Arthur Schnitzler, 16. 9. 1898}\nopagebreak\mylabel{v}\rehead{ }\begin{ledgroupsized}[t]{13cm}\normalsize\beginnumbering \toendnotes[C]{\smallbreak\pagebreak[2]} \Standort{CUL, Schnitzler, B 34.}
\physDesc{Brief, 1 Blatt, 2 Seiten
\newline{}Handschrift: schwarze Tinte, deutsche Kurrent
\newline{}Schnitzler: mit rotem Buntstift drei Unterstreichungen \newline{}Ordnung: mit Bleistift von unbekannter Hand nummeriert: »1« }\toendnotes[C]{\smallbreak}\pstart
           \noindent{}{\pb}\textcolor{gray}{\textbf{WIENER RUNDSCHAUXXXX ORGangabe fehlt.}}\pend
           \pstart
           \textcolor{gray}{\textbf{HERAUSGEBER}}\pend
           \pstart
           \textcolor{gray}{\textbf{GUSTAV SCHOENAICH\pwindex{\textcolor{red}{\textsuperscript{XXXX1 indx}}|pw}.}}\pend
           \pstart
           \textcolor{gray}{\textbf{FELIX RAPPAPORT\pwindex{\textcolor{red}{\textsuperscript{XXXX1 indx}}|pw}.}}\hfill \textcolor{gray}{\textbf{Wien\oindex{XXXX Ortsangabe fehlt|pw},}}{ }16. September \textcolor{gray}{\textbf{189}}8\pend
           \pstart
           \textcolor{gray}{\textbf{REDACTION UND ADMINISTRATION:}}\pend
           \pstart
           \textcolor{gray}{\textbf{WIEN\oindex{XXXX Ortsangabe fehlt|pw}}}\pend
           \pstart
           \textcolor{gray}{\textbf{I/1 SPIEGELGASSE 11\oindex{XXXX Ortsangabe fehlt|pw}.}}\pend
           \pstart
           \textcolor{gray}{\textbf{TELEPHON NR. 2579.}}\pend
           \pstart\center{}Sehr geehrter Herr Doctor!\pend\pstart
           Ich leſe in den Zeitungen von \introOben{}Ihren\introOben{} drei neuen Einactern\textcolor{red}{\textsuperscript{XXXX indx}}, die D\textsuperscript{r} \textsc{Brahm\pwindex{\textcolor{red}{\textsuperscript{XXXX1 indx}}|pw}} im »Deutſchen TheaterXXXX ORGangabe fehlt« aufführen wird.\pend
           \pstart
           Darf ich Sie nochmals, aufrichtig und innigſt bitten, ob Sie mir einen von dieſen zum
               Abdruck in der »RundſchauXXXX ORGangabe fehlt« überlaſſen möchten? Ich
               gebe Ihnen die Verſicherung, daſs ich glücklich wäre, wenn Sie meine Bitte erfüllen
               würden, daſs ich von Tag zu Tag \strikeout{\textcolor{gray}{×}\-\textcolor{gray}{×}} mehr einſehe, wie bornirt, leicht-fertig meine \strikeout{Radi} literariſchen Radicalismen von ſeinerzeit waren.
               Ich brauche nur an die \uline{nach} Ihnen Kommenden zu denken
               u bin beſchämt.\pend
           \pstart
           Überdies würden Sie \substVorne{}\textsuperscript{ſich}\substDazwischen{}mich\substHinten{} hiedurch beſonders verpflichten, weil mir Ihre Gabe eine moraliſche Unter{\pb}ſtützung wäre, gerade jetzt beſonders
               werthvoll, wo die literariſchen Schwarzkünſtler aller Art meinem Herausgeber\pwindex{\textcolor{red}{\textsuperscript{XXXX1 indx}}|pwv}\pwindex{\textcolor{red}{\textsuperscript{XXXX1 indx}}|pwv} in den
               Ohren liegen.\pend
           \pstart
           Verzeihen Sie, bitte, die Beläſtigung und erfüllen Sie – bitte – bald mein
               Anſuchen.\pend
           \pstart
           Ich bin{\\[\baselineskip]} Ihr \uline{sehr}{ }\uline{ergebener}{\\[\baselineskip]}\spacefill\mbox{Stefan Großmann}\pend
           \leftskip=0em{}
         
         \endnumbering\mylabel{h}\end{ledgroupsized}  \newcommand{\dateiname}{L00847}\newcommand{\titel}{Stefan Großmann an Arthur Schnitzler, 16. 9. 1898}\newcommand{\editorInnen}{ Martin Anton Müller und Gerd-Hermann Susen}%% latex-leseansicht-abspann.tex
%% Abspann für die Leseansicht.
%% Der Schalter \ifkorrekturansicht ist bereits durch den Vorspann gesetzt.

%% latex-abspann.tex
%% Gemeinsamer Abspann für Korrekturansicht und Leseansicht.
%% Setzt den Schalter \ifkorrekturansicht voraus (gesetzt in den
%% einbindenden Dateien latex-korrekturansicht-abspann.tex bzw.
%% latex-leseansicht-abspann.tex).
%% ---------------------------------------------------------------

\normalsize

% Das esempio-Environment wird nur in der Leseansicht benötigt
\ifkorrekturansicht\else
\newenvironment{esempio}[3]%
{
    \vspace{1.5ex}
    \rlap{\underline{#1}}
    \par
    \setlength{\parindent}{0cm}
    \nopagebreak
    \leftskip=#2cm
    \rightskip=#3cm
}
{
    \par
}
\fi

\doendnotes{C}
\bigskip
\vfill

\clearpage

\footnotesize

\ifkorrekturansicht
  \lohead{\textsc{register}}
\fi

% theindex-Environment neu definieren ohne reledmac
\makeatletter
\renewenvironment{theindex}{%
  \ifkorrekturansicht
    \section*{\indexname}%
  \else
    \subsubsection*{Index der erwähnten Entitäten}%
  \fi
  \setlength{\parindent}{0pt}%
  \setlength{\parskip}{0pt plus 0.3pt}%
  \let\item\@idxitem
}{%
  \ifkorrekturansicht\clearpage\fi
}
\makeatother

\IfFileExists{\jobname-pw.ind}{\input{\jobname-pw.ind}}{}

% Quellenangabe nur in der Leseansicht
\ifkorrekturansicht\else
% Fallback-Definitionen, falls die .tex-Datei \titel etc. nicht gesetzt hat
\providecommand{\titel}{}
\providecommand{\editorInnen}{}
\providecommand{\dateiname}{\jobname}

\vspace{3cm}

\vfill

\footnotesize
\textsc{Quelle}: \titel. Herausgegeben von {\editorInnen}. In: \emph{Arthur Schnitzler: Briefwechsel mit Autorinnen und Autoren}.
 Digitale Edition, https://schnitzler-briefe.acdh.oeaw.ac.at/{\dateiname}.html (Stand \today)
\fi

\end{document}


      