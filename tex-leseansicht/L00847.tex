%% latex-leseansicht-vorspann.tex
%% Vorspann für die Leseansicht.
%% Lädt die gemeinsame Datei latex-vorspann.tex mit nicht gesetztem Schalter.

\newif\ifkorrekturansicht
\korrekturansichtfalse

\input{../tex-inputs/latex-vorspann}


\section[Stefan Großmann an Arthur Schnitzler, 16. 9. 1898]{L00847 Stefan Großmann an Arthur Schnitzler, 16. 9. 1898}
\nopagebreak\mylabel{L00847v}
\rehead{ }\normalsize\beginnumbering\briefempfaengerindex{Schnitzler, Arthur@\textsc{Schnitzler, Arthur}!zzzGroßmann, Stefan@\emph{von Stefan Großmann}!1898-09-162@{16. 9. 1898}|(be}
\toendnotes[C]{\smallbreak\pagebreak[2]}
\correspDesc{Versand  durch Stefan Großmann am 16. 9. 1898 in Wien
\newline{}Erhalt  durch Arthur Schnitzler im Zeitraum [16. 9. 1898
                  – 20. 9. 1898?] in Wien}\toendnotes[C]{\smallbreak}
\Standort{CUL, Schnitzler, B 34.}
\physDesc{Brief, 1 Blatt, 2 Seiten, 920 Zeichen
\newline{}Handschrift: schwarze Tinte, deutsche Kurrent
\newline{}Schnitzler: mit rotem Buntstift drei Unterstreichungen 
\newline{}Ordnung: mit Bleistift von unbekannter Hand nummeriert:
                                 »1« }\toendnotes[C]{\smallbreak}
\pstart
           {\pb}\textcolor{gray}{\textbf{WIENER RUNDSCHAU\orgindex{Wiener Rundschau@Wiener Rundschau|pw}.}}\pend
           
\pstart
           \textcolor{gray}{\textbf{HERAUSGEBER}}\pend
           
\pstart
           \textcolor{gray}{\textbf{GUSTAV SCHOENAICH\pwindex{Schönaich, Gustav 24.\,11.\,1840 Wien – 4.\,4.\,1906@\textsc{Schönaich, Gustav} (24.\,11.\,1840 Wien – 4.\,4.\,1906), \emph{Journalist, Musikwissenschaftler, Jurist}|pw}.}}\pend
           
\pstart
           \textcolor{gray}{\textbf{FELIX RAPPAPORT\pwindex{Rappaport, Felix 17.\,11.\,1874 Wien – 8.\,12.\,1939 ebd.@\textsc{Rappaport, Felix} (17.\,11.\,1874 Wien – 8.\,12.\,1939 ebd.), \emph{Herausgeber, Finanzier, Lyriker}|pw}.}}\hfill \textcolor{gray}{\textbf{Wien\oindex{Wien@\textbf{Wien}, \emph{Verwaltungsgebiet}|pw},}}{ }16. September \textcolor{gray}{\textbf{189}}8\pend
           
\pstart
           \textcolor{gray}{\textbf{REDACTION UND ADMINISTRATION:}}\pend
           
\pstart
           \textcolor{gray}{\textbf{WIEN\oindex{Wien@\textbf{Wien}, \emph{Verwaltungsgebiet}|pw}}}\pend
           
\pstart
           \textcolor{gray}{\textbf{I/1 SPIEGELGASSE 11\oindex{Wien@\textbf{Wien}!I., Innere Stadt@\textbf{I., Innere Stadt}!Spiegelgasse@\textbf{Spiegelgasse}, \emph{Straße}|pw}.}}\pend
           
\pstart
           \textcolor{gray}{\textbf{TELEPHON NR. 2579.}}\pend
           
\pstart\center{}Sehr geehrter Herr Doctor!\pend\vspace{0.5em}
\pstart
           Ich leſe in den Zeitungen von \introOben{}Ihren\introOben{} drei neuen Einactern\pwindex{Schnitzler, Arthur 15.\,5.\,1862 Wien – 21.\,10.\,1931 ebd.@\textsc{Schnitzler, Arthur} (15.\,5.\,1862 Wien – 21.\,10.\,1931 ebd.), \emph{Schriftsteller, Mediziner}!grüne Kakadu – Paracelsus – Die Gefährtin. Drei Einakter@\strich\emph{Der grüne Kakadu – Paracelsus – Die Gefährtin. Drei Einakter}|pwv}, die D\textsuperscript{r}\textsc{Brahm\pwindex{Brahm, Otto 5.\,2.\,1856 Hamburg – 28.\,11.\,1912 Berlin@\textsc{Brahm, Otto} (5.\,2.\,1856 Hamburg – 28.\,11.\,1912 Berlin), \emph{Theaterleiter, Regisseur}|pw}} im »Deutſchen Theater\orgindex{Deutsches Theater Berlin@Deutsches Theater Berlin|pw}« aufführen wird.\pend
           
\pstart
           Darf ich Sie nochmals, aufrichtig und innigſt bitten, ob Sie mir einen von dieſen zum
               Abdruck in der »Rundſchau\orgindex{Wiener Rundschau@Wiener Rundschau|pw}« überlaſſen möchten?
               Ich gebe Ihnen die Verſicherung, daſs ich glücklich wäre, wenn Sie meine Bitte
               erfüllen würden, daſs ich von Tag zu Tag \strikeout{\textcolor{gray}{×}\-\textcolor{gray}{×}} mehr einſehe, wie bornirt, leicht-fertig meine \strikeout{Radi} literariſchen Radicalismen von{ }ſeinerzeit waren. Ich brauche nur an
               die \uline{nach} Ihnen Kommenden zu denken u bin
               beſchämt.\pend
           
\pstart
           Überdies würden Sie \substVorne{}\textsuperscript{ſich}\substDazwischen{}mich\substHinten{} hiedurch beſonders verpflichten, weil mir Ihre Gabe eine moraliſche Unter{\pb}ſtützung wäre, gerade jetzt beſonders
               werthvoll, wo die literariſchen Schwarzkünſtler aller Art meinem Herausgeber\pwindex{Schönaich, Gustav 24.\,11.\,1840 Wien – 4.\,4.\,1906@\textsc{Schönaich, Gustav} (24.\,11.\,1840 Wien – 4.\,4.\,1906), \emph{Journalist, Musikwissenschaftler, Jurist}|pwv}\pwindex{Rappaport, Felix 17.\,11.\,1874 Wien – 8.\,12.\,1939 ebd.@\textsc{Rappaport, Felix} (17.\,11.\,1874 Wien – 8.\,12.\,1939 ebd.), \emph{Herausgeber, Finanzier, Lyriker}|pwv} in den Ohren
               liegen.\pend
           
\pstart
           Verzeihen Sie, bitte, die Beläſtigung und erfüllen Sie – bitte – bald mein
               Anſuchen.\pend
           
\pstart
           Ich bin{\\[\baselineskip]} Ihr \uline{sehr}{ }\uline{ergebener}{\\[\baselineskip]}\spacefill\mbox{Stefan Großmann}\pend
           \leftskip=0em{}\selectlanguage{ngerman}\endnumbering\briefempfaengerindex{Schnitzler, Arthur@\textsc{Schnitzler, Arthur}!zzzGroßmann, Stefan@\emph{von Stefan Großmann}!1898-09-162@{16. 9. 1898}|)be}\mylabel{L00847h}  \newcommand{\dateiname}{L00847}\newcommand{\titel}{Stefan Großmann an Arthur Schnitzler, 16. 9. 1898}\newcommand{\editorInnen}{Herausgegeben von Martin Anton Müller}%% latex-leseansicht-abspann.tex
%% Abspann für die Leseansicht.
%% Der Schalter \ifkorrekturansicht ist bereits durch den Vorspann gesetzt.

%% latex-abspann.tex
%% Gemeinsamer Abspann für Korrekturansicht und Leseansicht.
%% Setzt den Schalter \ifkorrekturansicht voraus (gesetzt in den
%% einbindenden Dateien latex-korrekturansicht-abspann.tex bzw.
%% latex-leseansicht-abspann.tex).
%% ---------------------------------------------------------------

\normalsize

% Das esempio-Environment wird nur in der Leseansicht benötigt
\ifkorrekturansicht\else
\newenvironment{esempio}[3]%
{
    \vspace{1.5ex}
    \rlap{\underline{#1}}
    \par
    \setlength{\parindent}{0cm}
    \nopagebreak
    \leftskip=#2cm
    \rightskip=#3cm
}
{
    \par
}
\fi

\doendnotes{C}
\bigskip
\vfill

\clearpage

\footnotesize

\ifkorrekturansicht
  \lohead{\textsc{register}}
\fi

% theindex-Environment neu definieren ohne reledmac
\makeatletter
\renewenvironment{theindex}{%
  \ifkorrekturansicht
    \section*{\indexname}%
  \else
    \subsubsection*{Index der erwähnten Entitäten}%
  \fi
  \setlength{\parindent}{0pt}%
  \setlength{\parskip}{0pt plus 0.3pt}%
  \let\item\@idxitem
}{%
  \ifkorrekturansicht\clearpage\fi
}
\makeatother

\IfFileExists{\jobname-pw.ind}{\input{\jobname-pw.ind}}{}

% Quellenangabe nur in der Leseansicht
\ifkorrekturansicht\else
% Fallback-Definitionen, falls die .tex-Datei \titel etc. nicht gesetzt hat
\providecommand{\titel}{}
\providecommand{\editorInnen}{}
\providecommand{\dateiname}{\jobname}

\vspace{3cm}

\vfill

\footnotesize
\textsc{Quelle}: \titel. Herausgegeben von {\editorInnen}. In: \emph{Arthur Schnitzler: Briefwechsel mit Autorinnen und Autoren}.
 Digitale Edition, https://schnitzler-briefe.acdh.oeaw.ac.at/{\dateiname}.html (Stand \today)
\fi

\end{document}


