%% latex-leseansicht-vorspann.tex
%% Vorspann für die Leseansicht.
%% Lädt die gemeinsame Datei latex-vorspann.tex mit nicht gesetztem Schalter.

\newif\ifkorrekturansicht
\korrekturansichtfalse

\input{../tex-inputs/latex-vorspann}


         
         \renewcommand{\erwaehntePersonen}{Personen: Richard Beer-Hofmann, Leopold Sonnemann}
         \renewcommand{\erwaehnteInstitutionen}{Institutionen: Frankfurter Zeitung}
         \renewcommand{\erwaehnteOrte}{Orte: Paris, Wien, rue Richelieu}
         \renewcommand{\erwaehnteWerke}{Werke: Abschiedssouper, Agonie, Anatol}
               \section[Paul Goldmann an Arthur Schnitzler, 2. 11. {[}1892{]}]{ Paul Goldmann an Arthur Schnitzler, 2. 11. {[}1892{]}}\nopagebreak\mylabel{v}\rehead{ }\begin{ledgroupsized}[t]{13cm}\normalsize\beginnumbering \toendnotes[C]{\smallbreak\pagebreak[2]} \Standort{DLA, A:Schnitzler, HS.NZ85.1.3163.}
\physDesc{Brief, 2 Blätter, 6 Seiten, 3112 Zeichen
\newline{}Handschrift: schwarze Tinte, deutsche Kurrent
\newline{}Schnitzler: 1) mit Bleistift das Jahr »92« ergänzt, sowie, vermutlich am »7/1 08« das Schlagwort »(Zukunftsverſprechungen)«
                                 vermerkt  2) mit rotem Buntstift drei vertikale Markierungen}\toendnotes[C]{\smallbreak}\pstart
           \noindent{}{\pb}\textcolor{gray}{\textbf{Frankfurter Zeitung\orgindex{Frankfurter Zeitung@Frankfurter Zeitung|pw}.}}\hfill \textsc{Paris\oindex{Paris@\textbf{Paris}|pw}}, 2. November. \pend
           \pstart
           \textcolor{gray}{\textbf{(Gazette de
                     Francfort\orgindex{Frankfurter Zeitung@Frankfurter Zeitung|pw}.)}}\pend
           \pstart
           \textcolor{gray}{\textbf{\begin{otherlanguage}{french}Directeur\end{otherlanguage}: \textbf{M. L. Sonnemann\pwindex{Sonnemann, Leopold 1831-10-29 – 1909-10-30@\textsc{Sonnemann, Leopold} (1831-10-29 – 1909-10-30), \emph{Journalist, Herausgeber}|pw}}.}}\pend
           \pstart
           \textcolor{gray}{\textbf{\begin{otherlanguage}{french}Journal politique, financier,\end{otherlanguage}}}\pend
           \pstart
           \textcolor{gray}{\textbf{\begin{otherlanguage}{french}commercial et litteraire.\end{otherlanguage}}}\pend
           \pstart
           \textcolor{gray}{\textbf{\begin{otherlanguage}{french}\textbf{Paraissant trois fois par jour}\end{otherlanguage}}}\pend
           \pstart
           \textcolor{gray}{\textbf{\begin{otherlanguage}{french}\textbf{Bureaux à Paris\oindex{Paris@\textbf{Paris}|pw}:}\end{otherlanguage}}}\pend
           \pstart
           \textcolor{gray}{\textbf{\begin{otherlanguage}{french}\textbf{rue Richelieu 75\oindex{rue Richelieu@\textbf{rue Richelieu}|pw}.}\end{otherlanguage}}}\pend
           \pstart\center{}Mein lieber Arthur!\pend\pstart
           Ich habe die mit ungeduldiger Spannung erwartete Sendung\pwindex{Schnitzler, Arthur 15.05.1862 – 21.10.1931@\textsc{Schnitzler, Arthur} (15.05.1862 – 21.10.1931), \emph{Schriftsteller, Mediziner}!Anatol1892-10-29@\strich\emph{Anatol} {[}1892-10-29{]}|pwv} erhalten. Habe mich zunächſt an dem äußeren Eindruck
               geweidet und mich mit der merkwürdigen Thatſache befreundet, daß da vor mir auf
               blauem Einband \strikeout{\textcolor{gray}{×}} ein mir theurer Name ſtand, ein Stück Literatur geworden. Und habe mich dann
               athemlos, athemlos an die Lectüre gemacht und die lieben Seiten\pwindex{Schnitzler, Arthur 15.05.1862 – 21.10.1931@\textsc{Schnitzler, Arthur} (15.05.1862 – 21.10.1931), \emph{Schriftsteller, Mediziner}!Anatol1892-10-29@\strich\emph{Anatol} {[}1892-10-29{]}|pwv} verſchlungen, was ich nicht kannte
               zuerſt – »Abſchiedsſouper\pwindex{Schnitzler, Arthur 15.05.1862 – 21.10.1931@\textsc{Schnitzler, Arthur} (15.05.1862 – 21.10.1931), \emph{Schriftsteller, Mediziner}!Abschiedssouper1892@\strich\emph{Abschiedssouper} {[}1892{]}|pw}«, »Agonie\pwindex{Schnitzler, Arthur 15.05.1862 – 21.10.1931@\textsc{Schnitzler, Arthur} (15.05.1862 – 21.10.1931), \emph{Schriftsteller, Mediziner}!Agonie1892@\strich\emph{Agonie} {[}1892{]}|pw}«, wo ich beſonders in letzterem {\pb}einfach göttliche Sachen gefunden habe – und was ich
               kannte darauf. Und es war eine köſtliche Stunde, und ich ſtand wieder unter dem Banne
               Deines lieben Geiſtes, mit all’ dem Warmen, Weichen und Traulichen, das er für mich
               hat und das in meinem wüſten Leben eines der wenigen guten Dinge geweſen iſt. Aber
               ich habe auch als Literat geleſen, als Kritiker wenn Du willſt. Ich habe zugleich als
               Freund geleſen und dann wieder als der Mann, der das Buch\pwindex{Schnitzler, Arthur 15.05.1862 – 21.10.1931@\textsc{Schnitzler, Arthur} (15.05.1862 – 21.10.1931), \emph{Schriftsteller, Mediziner}!Anatol1892-10-29@\strich\emph{Anatol} {[}1892-10-29{]}|pwv} des blauen Einbands wegen aufſchlägt
               und fragt: »\textsc{Arthur Schnitzler}? Wer iſt das?« Und ich
               ſchwöre Dir, nach abermaliger Prüfung Deiner und meiner ſelbſt, nach einer Prüfung,
                  {\pb}die von jener neidvollen Strenge des Erfolgloſen
               gegen den Erfolgreichen, des Zurückgebliebenen gegen den Vorwärtsſchreitenden erfüllt
               war, nach alledem kann ich Dir nur Eines verſichern: So wie Dein Buch\pwindex{Schnitzler, Arthur 15.05.1862 – 21.10.1931@\textsc{Schnitzler, Arthur} (15.05.1862 – 21.10.1931), \emph{Schriftsteller, Mediziner}!Anatol1892-10-29@\strich\emph{Anatol} {[}1892-10-29{]}|pwv} Dich mir zeigt, biſt Du ein großes,
               herzerquickendes, gottbegnadetes, zukunftsreiches Talent. Ich drücke Dir
               glückwünſchend beide Hände angeſichts dieſes kleinen erſten Band\pwindex{Schnitzler, Arthur 15.05.1862 – 21.10.1931@\textsc{Schnitzler, Arthur} (15.05.1862 – 21.10.1931), \emph{Schriftsteller, Mediziner}!Anatol1892-10-29@\strich\emph{Anatol} {[}1892-10-29{]}|pwv}es, der mir die Kunde davon bringt, daß
               für Dich die Zukunft beginnt, die ich für Dich geträumt habe. Und ich glaube mich zu
               der Verheißung berechtigt, daß dieſe Zukunft groß und reich ſein wird, wenn Du jetzt
                  {\pb}\substVorne{}\textsuperscript{\textcolor{gray}{M}}\substDazwischen{}ſ\substHinten{}ta\substVorne{}\textsuperscript{\textcolor{gray}{×}\-\textcolor{gray}{×}\-\textcolor{gray}{×}}\substDazwischen{}rk\substHinten{} bleibſt, wo die ernſten Prüfungen Deiner harren, welche keinem Künſtler
               erſpart werden, wenn er in die Öffentlichkeit tritt. Ich weiß nicht, wie ich es
               machen ſoll, damit Dir dieſe Worte nicht altweiberhaft klingen, ſondern ſo treu und
               ehrlich wie ſie gemeint ſind. Ich weiß nur, daß ich es gerade jetzt dringender als je
               wünſche, \strikeout{\textcolor{gray}{and}} an Deiner Seite zu ſein. Und es thut mir in der Seele weh, daß ich Dir nur aus
               der Ferne ſagen kann in einem Briefe, der nur einmal zu Worte kommt und dann in einer
               Schublade verſchwindet! {\pb}Laß’ Dich nicht ablenken
               oder entmuthigen, wenn hier und da die große Dummheit ihre Stimme gegen Dich erheben
                  ſollte\strikeout{\textcolor{gray}{n}}. Glatt geht es nicht hinauf. Und das »\label{K_L02703-1v}\edtext{\textsc{\begin{otherlanguage}{french}Il faut se maintenir tout-de-même\end{otherlanguage}}}{\lemma{\textnormal{\emph{Il … tout-de-même}}}\Cendnote{\textnormal{französisch: man muss sich demungeachtet
                  behaupten}}}\label{K_L02703-1h}«, das mir ein Mal ein armer Teufel von einem \label{K_L02703-12v}\edtext{Collegen}{\lemma{\textnormal{\emph{Collegen}}}\Cendnote{\textnormal{nicht identifiziert}}}\label{K_L02703-12h} ſagte, der gar hart mit der
               Dummheit und Gemeinheit zu ringen hatte, iſt ein furchtbar platter und alltäglicher
               Wahlſpruch, aber man kann doch daraus unter Umſtänden eine Rieſenmenge von {\pb}Troſt und Stärke ziehen.\pend
           \pstart
           So hab’ ich getreulich Alles erwogen, das Gute und das Schlimme. Und zuletzt kehre
               ich nochmals zum Guten zurück und danke Dir für die Freude, die das kleine blaue Buch\pwindex{Schnitzler, Arthur 15.05.1862 – 21.10.1931@\textsc{Schnitzler, Arthur} (15.05.1862 – 21.10.1931), \emph{Schriftsteller, Mediziner}!Anatol1892-10-29@\strich\emph{Anatol} {[}1892-10-29{]}|pwv} in mein Zimmer gebracht
               hat, und ſcheide von Dir mit dem allerwärmſten aller Glückwünſche{\dotstwo}\pend
           \pstart
           Ich umarme Dich herzlichſt {\\[\baselineskip]}Dein {\\[\baselineskip]}\spacefill\mbox{Paul Goldmn}\pend
           \leftskip=0em{}\pstart
           \noindent{}Beſprechungen? Wollen ſehen.\pend
           \pstart
           Schlecht haſt Du aber Correctur geleſen. Warum haſt Du mir nicht die Bogen
                  geſchickt?\pend
           \pstart
           \label{T_L02703-1v}\edtext{Und \textsc{Richard\pwindex{Beer-Hofmann, Richard 1866-07-11 – 1945-09-26@\textsc{Beer-Hofmann, Richard} (1866-07-11 – 1945-09-26), \emph{Schriftsteller}|pw}} ſoll mir ſchreiben, bitte!}{\lemma{\textnormal{\emph{Und … bitte!}}}\Cendnote{\textnormal{seitlich entlang des Mittelfalzes}}}\label{T_L02703-1h}\pend
           
         
         \endnumbering\mylabel{h}\end{ledgroupsized}  \newcommand{\dateiname}{L02703}\newcommand{\titel}{Paul Goldmann an Arthur Schnitzler, 2. 11. [1892]}\newcommand{\editorInnen}{Martin Anton Müller und Laura Untner}%% latex-leseansicht-abspann.tex
%% Abspann für die Leseansicht.
%% Der Schalter \ifkorrekturansicht ist bereits durch den Vorspann gesetzt.

%% latex-abspann.tex
%% Gemeinsamer Abspann für Korrekturansicht und Leseansicht.
%% Setzt den Schalter \ifkorrekturansicht voraus (gesetzt in den
%% einbindenden Dateien latex-korrekturansicht-abspann.tex bzw.
%% latex-leseansicht-abspann.tex).
%% ---------------------------------------------------------------

\normalsize

% Das esempio-Environment wird nur in der Leseansicht benötigt
\ifkorrekturansicht\else
\newenvironment{esempio}[3]%
{
    \vspace{1.5ex}
    \rlap{\underline{#1}}
    \par
    \setlength{\parindent}{0cm}
    \nopagebreak
    \leftskip=#2cm
    \rightskip=#3cm
}
{
    \par
}
\fi

\doendnotes{C}
\bigskip
\vfill

\clearpage

\footnotesize

\ifkorrekturansicht
  \lohead{\textsc{register}}
\fi

% theindex-Environment neu definieren ohne reledmac
\makeatletter
\renewenvironment{theindex}{%
  \ifkorrekturansicht
    \section*{\indexname}%
  \else
    \subsubsection*{Index der erwähnten Entitäten}%
  \fi
  \setlength{\parindent}{0pt}%
  \setlength{\parskip}{0pt plus 0.3pt}%
  \let\item\@idxitem
}{%
  \ifkorrekturansicht\clearpage\fi
}
\makeatother

\IfFileExists{\jobname-pw.ind}{\input{\jobname-pw.ind}}{}

% Quellenangabe nur in der Leseansicht
\ifkorrekturansicht\else
% Fallback-Definitionen, falls die .tex-Datei \titel etc. nicht gesetzt hat
\providecommand{\titel}{}
\providecommand{\editorInnen}{}
\providecommand{\dateiname}{\jobname}

\vspace{3cm}

\vfill

\footnotesize
\textsc{Quelle}: \titel. Herausgegeben von {\editorInnen}. In: \emph{Arthur Schnitzler: Briefwechsel mit Autorinnen und Autoren}.
 Digitale Edition, https://schnitzler-briefe.acdh.oeaw.ac.at/{\dateiname}.html (Stand \today)
\fi

\end{document}


      