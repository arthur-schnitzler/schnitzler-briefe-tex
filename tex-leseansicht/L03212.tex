%% latex-korrekturansicht-vorspann.tex
%% Vorspann für die Korrekturansicht.
%% Lädt die gemeinsame Datei latex-vorspann.tex mit gesetztem Schalter.

\newif\ifkorrekturansicht
\korrekturansichttrue

\input{../tex-inputs/latex-vorspann}


\section[ Paul Goldmann an Arthur Schnitzler, 24. 6. 1902]{L03212 Paul Goldmann an Arthur Schnitzler, 24. 6. 1902}
\nopagebreak\mylabel{L03212v}
\rehead{ }\normalsize\beginnumbering\briefempfaengerindex{Schnitzler, Arthur@\textsc{Schnitzler, Arthur}!zzzGoldmann, Paul@\emph{von Paul Goldmann}!1902-06-241@{24. 6. 1902}|(be}
\toendnotes[C]{\smallbreak\pagebreak[2]}\Standort{DLA, A:Schnitzler, HS.NZ85.1.3172.}
\physDesc{Bildpostkarte, 88 Zeichen
\newline{}Handschrift: 1) schwarze Tinte, deutsche Kurrent\hspace{1em}2) schwarze Tinte, lateinische Kurrent (\noindent{}Adresse)\hspace{1em}
\newline{}Versand: 1) Stempel: »\nobreak{}\oindex{Dresden@\textbf{Dresden}, \emph{P.PPLA}|pwk}Dresden, 24. 6. 02, 11½–12\nobreak{}«.   2) Stempel: »\nobreak{}\oindex{IX., Alsergrund@\textbf{IX., Alsergrund}, \emph{A.ADM3}|pwk}9/3 Wi{[}en{]}, 25{[}.{]} 6. \textcolor{gray}{02}, 7. N, Beste{[}llt{]}\nobreak{}«. }\pstart{}{\pb}Herrn\pend{}\pstart{}Dr. Arthur Schnitzler\pend{}\pstart{}Wien\oindex{Wien@\textbf{Wien}, \emph{A.ADM2}|pw}\pend{}\pstart{}IX. Frankgaſse 1\oindex{Frankgasse 1@\textbf{Frankgasse 1}, \emph{Wohngebäude (K.WHS)}|pw}.\pend{}{\bigskip}
\pstart
           {\pb}\textcolor{gray}{\textbf{Neustädter Markt\oindex{Neustaedter Markt@\textbf{Neustädter Markt}, \emph{Platz (K.PLT)}|pw} mit Wache\oindex{Blockhaus@\textbf{Blockhaus}, \emph{Gebäude (K.GBD)}|pw}}}\hfill \textcolor{gray}{\textbf{Dresden\oindex{Dresden@\textbf{Dresden}, \emph{P.PPLA}|pw}}}\pend
           \vspace{1em}
\pstart
           \centering{}{\pb}24. Juni.\pend
           \vspace{0.5em}
\pstart
           Herzlichſte Grüße!\pend
           \pstart \spacefill\mbox{Paul Goldmann.}\pend{}\selectlanguage{ngerman}\endnumbering\briefempfaengerindex{Schnitzler, Arthur@\textsc{Schnitzler, Arthur}!zzzGoldmann, Paul@\emph{von Paul Goldmann}!1902-06-241@{24. 6. 1902}|)be}\mylabel{L03212h}  \normalsize

\doendnotes{C}
\bigskip
\vfill

\clearpage

\footnotesize

\lohead{\textsc{register}}

% Definiere theindex-Environment komplett neu ohne reledmac
\makeatletter
\renewenvironment{theindex}{%
  \section*{\indexname}%
  \setlength{\parindent}{0pt}%
  \setlength{\parskip}{0pt plus 0.3pt}%
  \let\item\@idxitem
}{%
  \clearpage
}
\makeatother

\IfFileExists{\jobname-pw.ind}{\input{\jobname-pw.ind}}{}

\end{document}

      