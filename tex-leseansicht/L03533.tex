%% latex-korrekturansicht-vorspann.tex
%% Vorspann für die Korrekturansicht.
%% Lädt die gemeinsame Datei latex-vorspann.tex mit gesetztem Schalter.

\newif\ifkorrekturansicht
\korrekturansichttrue

\input{../tex-inputs/latex-vorspann}


\section[ Felix Salten an Arthur Schnitzler, 17. 5. 1930]{L03533 Felix Salten an Arthur Schnitzler, 17. 5. 1930}
\nopagebreak\mylabel{L03533v}
\rehead{ }\normalsize\beginnumbering\briefempfaengerindex{Schnitzler, Arthur@\textsc{Schnitzler, Arthur}!zzzSalten, Felix@\emph{von Felix Salten}!1930-05-171@{17. 5. 1930}|(be}
\toendnotes[C]{\smallbreak\pagebreak[2]}\Standort{CUL, Schnitzler, B 89, B 2.}
\physDesc{Bildpostkarte, 248 Zeichen
\newline{}Handschrift: schwarze Tinte, lateinische Kurrent
\newline{}Versand: Stempel: »\nobreak{}18 5 \textcolor{gray}{3}0
                                    , Deutsche Seepost Linie Bremen
                                          – New York\orgindex{Deutsch-amerikanische Seepost@Deutsch-amerikanische Seepost|pw}\nobreak{}«.  
\newline{}Ordnung: mit Bleistift von unbekannter Hand nummeriert: »304« }\toendnotes[C]{\smallbreak}\pstart{}{\pb}Europa\oindex{Europa@\textbf{Europa}, \emph{Kontinent (A.KNT)}|pw}\pend{}\pstart{}Austria\oindex{Oesterreich@\textbf{Österreich}, \emph{A.PCLI}|pw}\pend{}\pstart{}Herrn\pend{}\pstart{}D\textsuperscript{r} Arthur Schnitzler\pend{}\pstart{}Wien\oindex{Wien@\textbf{Wien}, \emph{A.ADM2}|pw}\pend{}\pstart{}18. Sternwartestrasse 71\oindex{Sternwartestrasse 71@\textbf{Sternwartestraße 71}, \emph{Wohngebäude (K.WHS)}|pw}\pend{}{\bigskip}
\pstart
           \noindent{}\centering{}{\pb}\textcolor{gray}{\textbf{NORDDEUTSCHER LLOYD\orgindex{Norddeutscher Lloyd@Norddeutscher Lloyd|pw}, BREMEN\oindex{Bremen@\textbf{Bremen}, \emph{P.PPLA}|pw}}}\pend
           
\pstart
           \centering{}\textcolor{gray}{\textbf{D. »Berlin«}}\orgindex{Dampfer Berlin@Dampfer Berlin|pw}\pend
           
\pstart
           \centering{}\textcolor{gray}{\textbf{\label{T_L03533-1v}\edtext{Gesellschaftshalle}{\lemma{\textnormal{\emph{Gesellschaftshalle}}}\Cendnote{\textnormal{Druckfehler: »Gesellscaaftshalle«}}}\label{T_L03533-1} 1. Kl.}}\pend
           \vspace{1em}
\pstart
           \raggedleft{}{\pb}vor New York\oindex{New York City@\textbf{New York City}, \emph{P.PPL}|pw}{\\}17. 5. 30\pend
           \vspace{0.5em}
\pstart
           Lieber, die Fahrt, die nach zehn unruhigen Tagen morgen{ }früh endigt, war trotz allem sehr schön. Ich denke viel und gut an Sie
               und grüße sie herzlichst\pend
           
\pstart
           Ihr {\\[\baselineskip]}\spacefill\mbox{Felix Salten}\pend
           \leftskip=0em{}\selectlanguage{ngerman}\endnumbering\briefempfaengerindex{Schnitzler, Arthur@\textsc{Schnitzler, Arthur}!zzzSalten, Felix@\emph{von Felix Salten}!1930-05-171@{17. 5. 1930}|)be}\mylabel{L03533h}  \normalsize

\doendnotes{C}
\bigskip
\vfill

\clearpage

\footnotesize

\lohead{\textsc{register}}

% Definiere theindex-Environment komplett neu ohne reledmac
\makeatletter
\renewenvironment{theindex}{%
  \section*{\indexname}%
  \setlength{\parindent}{0pt}%
  \setlength{\parskip}{0pt plus 0.3pt}%
  \let\item\@idxitem
}{%
  \clearpage
}
\makeatother

\IfFileExists{\jobname-pw.ind}{\input{\jobname-pw.ind}}{}

\end{document}

      