%% latex-korrekturansicht-vorspann.tex
%% Vorspann für die Korrekturansicht.
%% Lädt die gemeinsame Datei latex-vorspann.tex mit gesetztem Schalter.

\newif\ifkorrekturansicht
\korrekturansichttrue

\input{../tex-inputs/latex-vorspann}


\section[ Paul Goldmann an Olga Gussmann, 20. 6. {[}1902{]}]{L03531 Paul Goldmann an Olga Gussmann, 20. 6. {[}1902{]}}
\nopagebreak\mylabel{L03531v}
\rehead{ }\normalsize\beginnumbering\briefempfaengerindex{Schnitzler, Olga@\textsc{Schnitzler, Olga}!zzzGoldmann, Paul@\emph{von Paul Goldmann}!1902-06-202@{20. 6. {[}1902{]}}|(be}
\toendnotes[C]{\smallbreak\pagebreak[2]}\Standort{DLA, A:Schnitzler, HS.NZ85.1.5247.}
\physDesc{Brief, 1 Blatt, 1 Seite, 389 Zeichen
\newline{}Handschrift: blaue Tinte, deutsche Kurrent}\toendnotes[C]{\smallbreak}
\pstart
           \raggedleft{}{\pb}\textcolor{gray}{\textbf{DESSAUERSTRASSE 19\oindex{Dessauer Strasse@\textbf{Dessauer Straße}, \emph{Straße (K.STR)}|pw}}}\pend
           
\pstart
           Berlin\oindex{Berlin@\textbf{Berlin}, \emph{P.PPLC}|pw}, 20. Juni.\pend
           
\pstart\center{}Liebe Freundin,\pend\vspace{0.5em}
\pstart
           Eben bekomme ich \label{K_L03531-1v}\edtext{Marſchordre}{\lemma{\textnormal{\emph{Marſchordre}}}\Cendnote{\textnormal{Goldmann\pwindex{Goldmann, Paul 31.01.1865 – 25.09.1935@\textsc{Goldmann, Paul} (31.01.1865 – 25.09.1935), \emph{Schriftsteller/Schriftstellerin, Journalist/Journalistin}|pwk} verwendet das französische Wort »ordre«, wechselt aber nicht, 
            wie bei ihm zu erwarten wäre, für das Fremdwort von deutscher in lateinische Kurrentschrift.}}}\label{K_L03531-1} nach \label{K_L03531-2v}\edtext{Dresden\oindex{Dresden@\textbf{Dresden}, \emph{P.PPLA}|pw}}{\lemma{\textnormal{\emph{Dresden}}}\Cendnote{\textnormal{Die fehlende Jahresangabe des Korrespondenzstücks lässt sich über den 
                  Inhalt erschließen. Albert von
                     Sachsen\pwindex{Albert von Sachsen 1828-04-23 – 1902-06-19@\textsc{Albert von Sachsen} (1828-04-23 – 1902-06-19), \emph{König/Königin}|pwk} starb am 19. 6. 1902, am 23. 6. 1902 wurde er in Dresden\oindex{Dresden@\textbf{Dresden}, \emph{P.PPLA}|pwk} beerdigt.
                  Goldmann\pwindex{Goldmann, Paul 31.01.1865 – 25.09.1935@\textsc{Goldmann, Paul} (31.01.1865 – 25.09.1935), \emph{Schriftsteller/Schriftstellerin, Journalist/Journalistin}|pwk} hielt sich nachweislich am 24. 6. 1902 in der Stadt auf.}}}\label{K_L03531-2} (Beerdigung des König\pwindex{Albert von Sachsen 1828-04-23 – 1902-06-19@\textsc{Albert von Sachsen} (1828-04-23 – 1902-06-19), \emph{König/Königin}|pwv}s). In fliegender Eile alſo: vielen
               Dank für Ihren lieben Brief! Sorgen Sie, bitte, dafür, daß \textsc{Liesl\pwindex{Steinrueck, Elisabeth 19.11.1885 – 07.04.1920@\textsc{Steinrück, Elisabeth} (19.11.1885 – 07.04.1920)|pw}} die \label{K_L03531-3v}\edtext{Angelegenheit mit \textsc{Löwenfeld\pwindex{Loewenfeld, Raphael 11.02.1854 – 28.12.1910@\textsc{Löwenfeld, Raphael} (11.02.1854 – 28.12.1910), \emph{Theaterleiter/Theaterleiterin}|pw}}}{\lemma{\textnormal{\emph{Angelegenheit mit Löwenfeld}}}\Cendnote{\textnormal{Siehe Paul Goldmann an Arthur Schnitzler, 16. 6. [1902].
               }}}\label{K_L03531-3} nicht verſchlampt. \textsc{Paul\pwindex{Marx, Paul 04.06.1861 – 27.11.1919@\textsc{Marx, Paul} (04.06.1861 – 27.11.1919), \emph{Journalist/Journalistin, Kritiker/Kritikerin}|pw}} werde ich in meine \label{K_L03531-4v}\edtext{Obhut}{\lemma{\textnormal{\emph{Obhut}}}\Cendnote{\textnormal{Das steht womöglich in Zusammenhang mit dem
                  im September des Jahres beginnenden Engagement Paul Marx\pwindex{Marx, Paul 04.06.1861 – 27.11.1919@\textsc{Marx, Paul} (04.06.1861 – 27.11.1919), \emph{Journalist/Journalistin, Kritiker/Kritikerin}|pwk}’ am \emph{Deutschen Theater Berlin}\orgindex{Deutsches Theater Berlin@Deutsches Theater Berlin|pwk}.}}}\label{K_L03531-4} nehmen. Ihnen wünſche ich von Herzen das
               Allerbeſte und ſende Ihnen viele Grüße.\pend
           
\pstart
           Ihr getreuer, halb todt gehetzter {\\[\baselineskip]}\spacefill\mbox{Paul Goldmann.}\pend
           \leftskip=0em{}\selectlanguage{ngerman}\endnumbering\briefempfaengerindex{Schnitzler, Olga@\textsc{Schnitzler, Olga}!zzzGoldmann, Paul@\emph{von Paul Goldmann}!1902-06-202@{20. 6. {[}1902{]}}|)be}\mylabel{L03531h}  \normalsize

\doendnotes{C}
\bigskip
\vfill

\clearpage

\footnotesize

\lohead{\textsc{register}}

% Definiere theindex-Environment komplett neu ohne reledmac
\makeatletter
\renewenvironment{theindex}{%
  \section*{\indexname}%
  \setlength{\parindent}{0pt}%
  \setlength{\parskip}{0pt plus 0.3pt}%
  \let\item\@idxitem
}{%
  \clearpage
}
\makeatother

\IfFileExists{\jobname-pw.ind}{\input{\jobname-pw.ind}}{}

\end{document}

      