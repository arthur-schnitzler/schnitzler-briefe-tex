%% latex-leseansicht-vorspann.tex
%% Vorspann für die Leseansicht.
%% Lädt die gemeinsame Datei latex-vorspann.tex mit nicht gesetztem Schalter.

\newif\ifkorrekturansicht
\korrekturansichtfalse

\input{../tex-inputs/latex-vorspann}

\begin{center}
            \textcolor{red}{ENTWURF, NICHT FERTIG KORRIGIERT}
                      \end{center}
            
         
         \renewcommand{\erwaehntePersonen}{Personen: Olga Schnitzler, Elisabeth Steinrück}
         \renewcommand{\erwaehnteOrte}{Orte: Berlin, Wien}
         \renewcommand{\erwaehnteWerke}{}
               \section[ Paul Goldmann an Olga XXXX Gussmann/Schnitzler, 20. 6. {[}XXXX{]}]{ Paul Goldmann an Olga XXXX Gussmann/Schnitzler, 20. 6. {[}XXXX{]}}\nopagebreak\mylabel{v}\rehead{ }\begin{ledgroupsized}[t]{13cm}\normalsize\beginnumbering \toendnotes[C]{\smallbreak\pagebreak[2]} \Standort{DLA, A:Schnitzler, HS.1985.1.5247.}
\physDesc{,  Blätter,  Seiten
\newline{}Handschrift: , deutsche Kurrent}{\pb}\textcolor{gray}{\textbf{DESSAUERSTRASSE 19\oindex{XXXX Ortsangabe fehlt|pw}}}\textcolor{red}{\textsuperscript{\textbf{KEY}}}\pstart
           Berlin\oindex{Berlin@\textbf{Berlin}|pw}, 20. Juni. 20.
                     Juni.\pend
           \pstart{}Liebe Freundin,\pend\pstart
           \pend
           \pstart
           Eben bekomme ich Marſchorder nach Dresden\textcolor{red}{\textsuperscript{\textbf{KEY}}}
               (Beerdigung des König\textcolor{red}{\textsuperscript{\textbf{KEY}}}s). In fliegender Eile alſo:
               vielen Dank für Ihren lieben Brief! Sorgen Sie, bitte, dafür, daß \textsc{Liesl\pwindex{Steinrueck, Elisabeth 19.11.1885 – 07.04.1920@\textsc{Steinrück, Elisabeth} (19.11.1885 – 07.04.1920)|pw}} die Angelegenheit mit \textsc{Löwenfeld\textcolor{red}{\textsuperscript{\textbf{KEY}}}} nicht verſchlampt. \textsc{Paul\textcolor{red}{\textsuperscript{\textbf{KEY}}}} werde ich in meine Obhut nehmen. Ihnen wünſche ich von Herzen das Allerbeſte
               und ſende Ihnen viele Grüße. {\\[\baselineskip]}Ihr getreuer, halb todt gehetzter\pend
           \leftskip=0em{}\pstart
           {\\[\baselineskip]}\spacefill\mbox{Paul Goldmann.}\pend
           \leftskip=0em{}
         
         \endnumbering\mylabel{h}\end{ledgroupsized}\begin{anhang}\end{anhang}\newcommand{\dateiname}{L03531}\newcommand{\titel}{Paul Goldmann an Olga XXXX Gussmann/Schnitzler, 20. 6. [XXXX]}\newcommand{\editorInnen}{Martin Anton Müller und Laura Untner}%% latex-leseansicht-abspann.tex
%% Abspann für die Leseansicht.
%% Der Schalter \ifkorrekturansicht ist bereits durch den Vorspann gesetzt.

%% latex-abspann.tex
%% Gemeinsamer Abspann für Korrekturansicht und Leseansicht.
%% Setzt den Schalter \ifkorrekturansicht voraus (gesetzt in den
%% einbindenden Dateien latex-korrekturansicht-abspann.tex bzw.
%% latex-leseansicht-abspann.tex).
%% ---------------------------------------------------------------

\normalsize

% Das esempio-Environment wird nur in der Leseansicht benötigt
\ifkorrekturansicht\else
\newenvironment{esempio}[3]%
{
    \vspace{1.5ex}
    \rlap{\underline{#1}}
    \par
    \setlength{\parindent}{0cm}
    \nopagebreak
    \leftskip=#2cm
    \rightskip=#3cm
}
{
    \par
}
\fi

\doendnotes{C}
\bigskip
\vfill

\clearpage

\footnotesize

\ifkorrekturansicht
  \lohead{\textsc{register}}
\fi

% theindex-Environment neu definieren ohne reledmac
\makeatletter
\renewenvironment{theindex}{%
  \ifkorrekturansicht
    \section*{\indexname}%
  \else
    \subsubsection*{Index der erwähnten Entitäten}%
  \fi
  \setlength{\parindent}{0pt}%
  \setlength{\parskip}{0pt plus 0.3pt}%
  \let\item\@idxitem
}{%
  \ifkorrekturansicht\clearpage\fi
}
\makeatother

\IfFileExists{\jobname-pw.ind}{\input{\jobname-pw.ind}}{}

% Quellenangabe nur in der Leseansicht
\ifkorrekturansicht\else
% Fallback-Definitionen, falls die .tex-Datei \titel etc. nicht gesetzt hat
\providecommand{\titel}{}
\providecommand{\editorInnen}{}
\providecommand{\dateiname}{\jobname}

\vspace{3cm}

\vfill

\footnotesize
\textsc{Quelle}: \titel. Herausgegeben von {\editorInnen}. In: \emph{Arthur Schnitzler: Briefwechsel mit Autorinnen und Autoren}.
 Digitale Edition, https://schnitzler-briefe.acdh.oeaw.ac.at/{\dateiname}.html (Stand \today)
\fi

\end{document}


      