%% latex-leseansicht-vorspann.tex
%% Vorspann für die Leseansicht.
%% Lädt die gemeinsame Datei latex-vorspann.tex mit nicht gesetztem Schalter.

\newif\ifkorrekturansicht
\korrekturansichtfalse

\input{../tex-inputs/latex-vorspann}


\section[ Paul Goldmann an Olga Gussmann, 20. 6. [1902]]{L03531 Paul Goldmann an Olga Gussmann,  20. 6. [1902]}
\nopagebreak\mylabel{L03531v}
\rehead{ }\normalsize\beginnumbering\briefempfaengerindex{Schnitzler, Olga@\textsc{Schnitzler, Olga}!zzzGoldmann, Paul@\emph{von Paul Goldmann}!1902-06-202@{20. 6. [1902]}|(be}
\toendnotes[C]{\smallbreak\pagebreak[2]}
\correspDesc{Versand  durch Paul Goldmann am 20. 6. [1902] in Berlin
\newline{}Erhalt  durch Olga Gussmann im Zeitraum [21. 6. 1902
                  – 25. 6. 1902?] in Wien?}\toendnotes[C]{\smallbreak}
\Standort{DLA, A:Schnitzler, HS.NZ85.1.5247.}
\physDesc{Brief, 1 Blatt, 1 Seite, 389 Zeichen
\newline{}Handschrift: blaue Tinte, deutsche Kurrent}\toendnotes[C]{\smallbreak}
\pstart
           \raggedleft{}{\pb}\textcolor{gray}{\textbf{DESSAUERSTRASSE 19\oindex{Dessauer Straße@\textbf{Dessauer Straße}, \emph{Straße}|pw}}}\pend
           
\pstart
           Berlin\oindex{Berlin@\textbf{Berlin}, \emph{Hauptstadt}|pw}, 20. Juni.\pend
           
\pstart\center{}Liebe Freundin,\pend\vspace{0.5em}
\pstart
           Eben bekomme ich \label{K_L03531-1v}\edtext{Marſchordre}{\lemma{\textnormal{\emph{Marschordre}}}\Cendnote{\textnormal{Goldmann\pwindex{Goldmann, Paul 31.\,1.\,1865 Breslau – 25.\,9.\,1935 Wien@\textsc{Goldmann, Paul} (31.\,1.\,1865 Breslau – 25.\,9.\,1935 Wien), \emph{Schriftsteller, Journalist}|pwk} verwendet das französische Wort »ordre«, wechselt aber nicht, 
            wie bei ihm zu erwarten wäre, für das Fremdwort von deutscher in lateinische Kurrentschrift.}}}\label{K_L03531-1} nach \label{K_L03531-2v}\edtext{Dresden\oindex{Dresden@\textbf{Dresden}|pw}}{\lemma{\textnormal{\emph{Dresden}}}\Cendnote{\textnormal{Die fehlende Jahresangabe des Korrespondenzstücks lässt sich über den 
                  Inhalt erschließen. Albert von
                     Sachsen\pwindex{Albert von Sachsen 23.\,4.\,1828 Dresden – 19.\,6.\,1902 Szczodre@\textsc{Albert von Sachsen} (23.\,4.\,1828 Dresden – 19.\,6.\,1902 Szczodre), \emph{König}|pwk} starb am 19. 6. 1902, am 23. 6. 1902 wurde er in Dresden\oindex{Dresden@\textbf{Dresden}|pwk} beerdigt.
                  Goldmann\pwindex{Goldmann, Paul 31.\,1.\,1865 Breslau – 25.\,9.\,1935 Wien@\textsc{Goldmann, Paul} (31.\,1.\,1865 Breslau – 25.\,9.\,1935 Wien), \emph{Schriftsteller, Journalist}|pwk} hielt sich nachweislich am XXXX Auszeichnungsfehler: Dokument L03212 nicht gefunden in der Stadt auf.}}}\label{K_L03531-2} (Beerdigung des König\pwindex{Albert von Sachsen 23.\,4.\,1828 Dresden – 19.\,6.\,1902 Szczodre@\textsc{Albert von Sachsen} (23.\,4.\,1828 Dresden – 19.\,6.\,1902 Szczodre), \emph{König}|pwv}s). In fliegender Eile alſo: vielen
               Dank für Ihren lieben Brief! Sorgen Sie, bitte, dafür, daß \textsc{Liesl\pwindex{Steinrück, Elisabeth 19.\,11.\,1885 – 7.\,4.\,1920 Partenkirchen@\textsc{Steinrück, Elisabeth} (19.\,11.\,1885 – 7.\,4.\,1920 Partenkirchen)|pw}} die \label{K_L03531-3v}\edtext{Angelegenheit mit \textsc{Löwenfeld\pwindex{Löwenfeld, Raphael 11.\,2.\,1854 Poznan – 28.\,12.\,1910 Berlin@\textsc{Löwenfeld, Raphael} (11.\,2.\,1854 Poznan – 28.\,12.\,1910 Berlin), \emph{Theaterleiter}|pw}}}{\lemma{\textnormal{\emph{Angelegenheit mit Löwenfeld}}}\Cendnote{\textnormal{Siehe XXXX Auszeichnungsfehler: Dokument L03211 nicht gefunden.
               }}}\label{K_L03531-3} nicht verſchlampt. \textsc{Paul\pwindex{Marx, Paul 4.\,6.\,1861 Breslau – 27.\,11.\,1919 Berlin@\textsc{Marx, Paul} (4.\,6.\,1861 Breslau – 27.\,11.\,1919 Berlin), \emph{Journalist, Kritiker}|pw}} werde ich in meine \label{K_L03531-4v}\edtext{Obhut}{\lemma{\textnormal{\emph{Obhut}}}\Cendnote{\textnormal{Das steht womöglich in Zusammenhang mit dem
                  im September des Jahres beginnenden Engagement Paul Marx\pwindex{Marx, Paul 4.\,6.\,1861 Breslau – 27.\,11.\,1919 Berlin@\textsc{Marx, Paul} (4.\,6.\,1861 Breslau – 27.\,11.\,1919 Berlin), \emph{Journalist, Kritiker}|pwk}’ am \emph{Deutschen Theater Berlin}\orgindex{Deutsches Theater Berlin@Deutsches Theater Berlin|pwk}.}}}\label{K_L03531-4} nehmen. Ihnen wünſche ich von Herzen das
               Allerbeſte und{ }ſende Ihnen viele Grüße.\pend
           
\pstart
           Ihr getreuer, halb todt gehetzter {\\[\baselineskip]}\spacefill\mbox{Paul Goldmann.}\pend
           \leftskip=0em{}\selectlanguage{ngerman}\endnumbering\briefempfaengerindex{Schnitzler, Olga@\textsc{Schnitzler, Olga}!zzzGoldmann, Paul@\emph{von Paul Goldmann}!1902-06-202@{20. 6. [1902]}|)be}\mylabel{L03531h}  \newcommand{\dateiname}{L03531}\newcommand{\titel}{Paul Goldmann an Olga Gussmann, 20. 6. [1902]}\newcommand{\editorInnen}{Martin Anton Müller und Laura Untner}%% latex-leseansicht-abspann.tex
%% Abspann für die Leseansicht.
%% Der Schalter \ifkorrekturansicht ist bereits durch den Vorspann gesetzt.

%% latex-abspann.tex
%% Gemeinsamer Abspann für Korrekturansicht und Leseansicht.
%% Setzt den Schalter \ifkorrekturansicht voraus (gesetzt in den
%% einbindenden Dateien latex-korrekturansicht-abspann.tex bzw.
%% latex-leseansicht-abspann.tex).
%% ---------------------------------------------------------------

\normalsize

% Das esempio-Environment wird nur in der Leseansicht benötigt
\ifkorrekturansicht\else
\newenvironment{esempio}[3]%
{
    \vspace{1.5ex}
    \rlap{\underline{#1}}
    \par
    \setlength{\parindent}{0cm}
    \nopagebreak
    \leftskip=#2cm
    \rightskip=#3cm
}
{
    \par
}
\fi

\doendnotes{C}
\bigskip
\vfill

\clearpage

\footnotesize

\ifkorrekturansicht
  \lohead{\textsc{register}}
\fi

% theindex-Environment neu definieren ohne reledmac
\makeatletter
\renewenvironment{theindex}{%
  \ifkorrekturansicht
    \section*{\indexname}%
  \else
    \subsubsection*{Index der erwähnten Entitäten}%
  \fi
  \setlength{\parindent}{0pt}%
  \setlength{\parskip}{0pt plus 0.3pt}%
  \let\item\@idxitem
}{%
  \ifkorrekturansicht\clearpage\fi
}
\makeatother

\IfFileExists{\jobname-pw.ind}{\input{\jobname-pw.ind}}{}

% Quellenangabe nur in der Leseansicht
\ifkorrekturansicht\else
% Fallback-Definitionen, falls die .tex-Datei \titel etc. nicht gesetzt hat
\providecommand{\titel}{}
\providecommand{\editorInnen}{}
\providecommand{\dateiname}{\jobname}

\vspace{3cm}

\vfill

\footnotesize
\textsc{Quelle}: \titel. Herausgegeben von {\editorInnen}. In: \emph{Arthur Schnitzler: Briefwechsel mit Autorinnen und Autoren}.
 Digitale Edition, https://schnitzler-briefe.acdh.oeaw.ac.at/{\dateiname}.html (Stand \today)
\fi

\end{document}


