%% latex-leseansicht-vorspann.tex
%% Vorspann für die Leseansicht.
%% Lädt die gemeinsame Datei latex-vorspann.tex mit nicht gesetztem Schalter.

\newif\ifkorrekturansicht
\korrekturansichtfalse

\input{../tex-inputs/latex-vorspann}

\begin{center}
            \textcolor{red}{ENTWURF, NICHT FERTIG KORRIGIERT}
                      \end{center}
            
         
         \renewcommand{\erwaehntePersonen}{Personen:  Albert von Sachsen, Paul Goldmann, Raphael Löwenfeld, Paul Marx, Olga Schnitzler, Elisabeth Steinrück}
         \renewcommand{\erwaehnteInstitutionen}{Institutionen: Deutsches Theater Berlin}
         \renewcommand{\erwaehnteOrte}{Orte: Berlin, Dessauer Straße, Dresden, Wien}
         \renewcommand{\erwaehnteWerke}{}
               \section[ Paul Goldmann an Olga Gussmann, 20. 6. {[}1902{]}]{ Paul Goldmann an Olga Gussmann, 20. 6. {[}1902{]}}\nopagebreak\mylabel{v}\rehead{ }\begin{ledgroupsized}[t]{13cm}\normalsize\beginnumbering \toendnotes[C]{\smallbreak\pagebreak[2]} \Standort{DLA, A:Schnitzler, HS.NZ85.1.5247.}
\physDesc{Brief, 1 Blatt, 1 Seite, 391 Zeichen
\newline{}Handschrift: blaue Tinte, deutsche Kurrent}\toendnotes[C]{\smallbreak}\pstart
           \noindent{}\raggedleft{}{\pb}\textcolor{gray}{\textbf{DESSAUERSTRASSE 19\oindex{Dessauer Strasse@\textbf{Dessauer Straße}|pw}}}\pend
           \pstart
           Berlin\oindex{Berlin@\textbf{Berlin}|pw}, 20. Juni.\pend
           \pstart\center{}Liebe Freundin,\pend\pstart
           Eben bekomme ich Marſchorder nach \label{K_L03531-1v}\edtext{Dresden\oindex{Dresden@\textbf{Dresden}|pw}}{\lemma{\textnormal{\emph{Dresden}}}\Cendnote{\textnormal{Da sich Goldmann\pwindex{Goldmann, Paul 31.01.1865 – 25.09.1935@\textsc{Goldmann, Paul} (31.01.1865 – 25.09.1935), \emph{Schriftsteller, Journalist}|pwk} jedenfalls am 24. 6. 1902 in Dresden\oindex{Dresden@\textbf{Dresden}|pwk} aufhielt und Albert von
                     Sachsen\pwindex{Albert von Sachsen 1828-04-23 – 1902-06-19@\textsc{Albert von Sachsen} (1828-04-23 – 1902-06-19), \emph{König}|pwk} am 19. 6. 1902 verstorben war und
                  am 23. 6. 1902 in Dresden\oindex{Dresden@\textbf{Dresden}|pwk} beerdigt wurde, kann davon ausgegangen werden, dass dieser Brief
                  wenige Tage zuvor verfasst wurde, also aus dem Jahr 1902
                  stammt.}}}\label{K_L03531-1h} (Beerdigung des König\pwindex{Albert von Sachsen 1828-04-23 – 1902-06-19@\textsc{Albert von Sachsen} (1828-04-23 – 1902-06-19), \emph{König}|pwv}s). In fliegender Eile alſo: vielen Dank für Ihren
               lieben Brief! Sorgen Sie, bitte, dafür, daß \textsc{Liesl\pwindex{Steinrueck, Elisabeth 19.11.1885 – 07.04.1920@\textsc{Steinrück, Elisabeth} (19.11.1885 – 07.04.1920)|pw}} die \label{K_L03531-2v}\edtext{Angelegenheit mit \textsc{Löwenfeld\pwindex{Loewenfeld, Raphael 11.02.1854 – 28.12.1910@\textsc{Löwenfeld, Raphael} (11.02.1854 – 28.12.1910), \emph{Theaterleiter}|pw}}}{\lemma{\textnormal{\emph{Angelegenheit mit Löwenfeld}}}\Cendnote{\textnormal{siehe Paul Goldmann an Arthur Schnitzler, 16. 6. [1902]}}}\label{K_L03531-2h} nicht verſchlampt. \textsc{Paul\pwindex{Marx, Paul 04.06.1861 – 27.11.1919@\textsc{Marx, Paul} (04.06.1861 – 27.11.1919), \emph{Journalist, Kritiker}|pw}} werde ich in meine \label{K_L03531-3v}\edtext{Obhut}{\lemma{\textnormal{\emph{Obhut}}}\Cendnote{\textnormal{Das steht womöglich in Zusammenhang mit Paul Marx\pwindex{Marx, Paul 04.06.1861 – 27.11.1919@\textsc{Marx, Paul} (04.06.1861 – 27.11.1919), \emph{Journalist, Kritiker}|pwk}’ Engagement am \emph{Deutschen Theater Berlin}\orgindex{Deutsches Theater Berlin@Deutsches Theater Berlin|pwk} ab dem 1. 9. 1902. Wie der Korrespondenz zwischen Goldmann\pwindex{Goldmann, Paul 31.01.1865 – 25.09.1935@\textsc{Goldmann, Paul} (31.01.1865 – 25.09.1935), \emph{Schriftsteller, Journalist}|pwk} und Elisabeth
                     Gussmann\pwindex{Steinrueck, Elisabeth 19.11.1885 – 07.04.1920@\textsc{Steinrück, Elisabeth} (19.11.1885 – 07.04.1920)|pwk} zu entnehmen ist, erhielt er dafür nur 100 Mark (vgl.
                        \emph{DLA}, HS.1985.1.5246).}}}\label{K_L03531-3h} nehmen. Ihnen
               wünſche ich von Herzen das Allerbeſte und ſende Ihnen viele Grüße.\pend
           \pstart
           Ihr getreuer, halb todt gehetzter {\\[\baselineskip]}\spacefill\mbox{Paul Goldmann.}\pend
           \leftskip=0em{}
         
         \endnumbering\mylabel{h}\end{ledgroupsized}\begin{anhang}\end{anhang}\newcommand{\dateiname}{L03531}\newcommand{\titel}{Paul Goldmann an Olga Gussmann, 20. 6. [1902]}\newcommand{\editorInnen}{Martin Anton Müller und Laura Untner}%% latex-leseansicht-abspann.tex
%% Abspann für die Leseansicht.
%% Der Schalter \ifkorrekturansicht ist bereits durch den Vorspann gesetzt.

%% latex-abspann.tex
%% Gemeinsamer Abspann für Korrekturansicht und Leseansicht.
%% Setzt den Schalter \ifkorrekturansicht voraus (gesetzt in den
%% einbindenden Dateien latex-korrekturansicht-abspann.tex bzw.
%% latex-leseansicht-abspann.tex).
%% ---------------------------------------------------------------

\normalsize

% Das esempio-Environment wird nur in der Leseansicht benötigt
\ifkorrekturansicht\else
\newenvironment{esempio}[3]%
{
    \vspace{1.5ex}
    \rlap{\underline{#1}}
    \par
    \setlength{\parindent}{0cm}
    \nopagebreak
    \leftskip=#2cm
    \rightskip=#3cm
}
{
    \par
}
\fi

\doendnotes{C}
\bigskip
\vfill

\clearpage

\footnotesize

\ifkorrekturansicht
  \lohead{\textsc{register}}
\fi

% theindex-Environment neu definieren ohne reledmac
\makeatletter
\renewenvironment{theindex}{%
  \ifkorrekturansicht
    \section*{\indexname}%
  \else
    \subsubsection*{Index der erwähnten Entitäten}%
  \fi
  \setlength{\parindent}{0pt}%
  \setlength{\parskip}{0pt plus 0.3pt}%
  \let\item\@idxitem
}{%
  \ifkorrekturansicht\clearpage\fi
}
\makeatother

\IfFileExists{\jobname-pw.ind}{\input{\jobname-pw.ind}}{}

% Quellenangabe nur in der Leseansicht
\ifkorrekturansicht\else
% Fallback-Definitionen, falls die .tex-Datei \titel etc. nicht gesetzt hat
\providecommand{\titel}{}
\providecommand{\editorInnen}{}
\providecommand{\dateiname}{\jobname}

\vspace{3cm}

\vfill

\footnotesize
\textsc{Quelle}: \titel. Herausgegeben von {\editorInnen}. In: \emph{Arthur Schnitzler: Briefwechsel mit Autorinnen und Autoren}.
 Digitale Edition, https://schnitzler-briefe.acdh.oeaw.ac.at/{\dateiname}.html (Stand \today)
\fi

\end{document}


      