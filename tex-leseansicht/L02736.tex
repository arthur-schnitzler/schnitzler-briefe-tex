%% latex-leseansicht-vorspann.tex
%% Vorspann für die Leseansicht.
%% Lädt die gemeinsame Datei latex-vorspann.tex mit nicht gesetztem Schalter.

\newif\ifkorrekturansicht
\korrekturansichtfalse

\input{../tex-inputs/latex-vorspann}


         
         \renewcommand{\erwaehntePersonen}{Personen: Max Eugen Burckhard, Isidor Fuchs, Paul Goldmann, Theodor Herzl, Max Nordau, Mizi Rosner, Leopold Sonnemann, Agnes Sorma, Edgar von Spiegl-Thurnsee, Theodor Wolff}
         \renewcommand{\erwaehnteInstitutionen}{Institutionen: Berliner Tageblatt, Burgtheater, Concordia, Frankfurter Zeitung, Neue Freie Presse}
         \renewcommand{\erwaehnteOrte}{Orte: Berlin, Paris, Wien, rue Feydeau}
         \renewcommand{\erwaehnteWerke}{Werke: Die Kunst in den elysäischen Feldern, Liebelei. Schauspiel in drei Akten, Marsfeldsalon-Typen, Neue Freie Presse}
               \section[Paul Goldmann an Arthur Schnitzler, 7. 6. {[}1895{]}]{ Paul Goldmann an Arthur Schnitzler, 7. 6. {[}1895{]}}\nopagebreak\mylabel{v}\rehead{ }\begin{ledgroupsized}[t]{13cm}\normalsize\beginnumbering \toendnotes[C]{\smallbreak\pagebreak[2]} \Standort{DLA, A:Schnitzler, HS.NZ85.1.3165.}
\physDesc{Brief, 2 Blätter, 8 Seiten, 2315 Zeichen
\newline{}Handschrift: schwarze Tinte, deutsche Kurrent
\newline{}Schnitzler: 1) mit Bleistift das Jahr »95« vermerkt  2) mit rotem Buntstift fünf Unterstreichungen}\toendnotes[C]{\smallbreak}\pstart
           \noindent{}{\pb}\textcolor{gray}{\textbf{\textbf{Frankfurter Zeitung\orgindex{Frankfurter Zeitung@Frankfurter Zeitung|pw}}}}\pend
           \pstart
           \textcolor{gray}{\textbf{(\begin{otherlanguage}{french}Gazette de Francfort\end{otherlanguage}\orgindex{Frankfurter Zeitung@Frankfurter Zeitung|pw}). }}\pend
           \pstart
           \textcolor{gray}{\textbf{\textbf{\begin{otherlanguage}{french}Fondateur M. L.
                              Sonnemann\pwindex{Sonnemann, Leopold 1831-10-29 – 1909-10-30@\textsc{Sonnemann, Leopold} (1831-10-29 – 1909-10-30), \emph{Journalist, Herausgeber}|pw}\end{otherlanguage}.}}}\pend
           \pstart
           \begin{otherlanguage}{french}\textcolor{gray}{\textbf{Journal politique, financier,}}\end{otherlanguage}\pend
           \pstart
           \begin{otherlanguage}{french}\textcolor{gray}{\textbf{commercial et littéraire.}}\end{otherlanguage}\pend
           \pstart
           \begin{otherlanguage}{french}\textcolor{gray}{\textbf{\textbf{Paraissant trois fois par jour.}}}\end{otherlanguage}\hfill \textsc{Paris\oindex{Paris@\textbf{Paris}|pw}}, 7. Juni.\pend
           \pstart
           \begin{otherlanguage}{french}\textcolor{gray}{\textbf{\textbf{Bureau à Paris\oindex{Paris@\textbf{Paris}|pw}}}}\end{otherlanguage}\pend
           \pstart
           \begin{otherlanguage}{french}\textcolor{gray}{\textbf{\textbf{24. Rue Feydeau\oindex{rue Feydeau@\textbf{rue Feydeau}|pw}.}}}\end{otherlanguage}\pend
           \pstart\center{}Mein lieber Freund,\pend\pstart
           Noch immer nicht der große Brief. Ich bin zu lebensmüde, zu hoffnungslos. Von allen
               Seiten wird es enge um mich, und kein Ausweg, keiner!\pend
           \pstart
           Nur Folgendes: \textsc{Isidor Fuchs\pwindex{Fuchs, Isidor 25.09.1849 – um den 20.8.1920@\textsc{Fuchs, Isidor} (25.09.1849 – um den 20.8.1920), \emph{Schriftsteller, Journalist}|pw}}, der ein verläßlicher Vertrauensmann\pwindex{Fuchs, Isidor 25.09.1849 – um den 20.8.1920@\textsc{Fuchs, Isidor} (25.09.1849 – um den 20.8.1920), \emph{Schriftsteller, Journalist}|pwv} iſt, frug mich um Dein Stück\pwindex{Schnitzler, Arthur 15.05.1862 – 21.10.1931@\textsc{Schnitzler, Arthur} (15.05.1862 – 21.10.1931), \emph{Schriftsteller, Mediziner}!Liebelei. Schauspiel in drei Akten1895-10-09@\strich\emph{Liebelei. Schauspiel in drei Akten} {[}1895-10-09{]}|pwv}. Ich ſagte ihm, die Schwierig{\pb}keiten, die ſich ihm bisher entgegengeſtellt, lagen
               wohl in den Kühnheiten, die es hat. Worauf \textsc{Fuchs\pwindex{Fuchs, Isidor 25.09.1849 – um den 20.8.1920@\textsc{Fuchs, Isidor} (25.09.1849 – um den 20.8.1920), \emph{Schriftsteller, Journalist}|pw}} ſolgenden Vorſchlag machte: Man ſolle es zuerſt in einer jener Vorſtellungen
               zum Benefiz der »\textsc{Concordia\orgindex{Concordia@Concordia|pw}}« geben, bei denen die Burg\orgindex{Burgtheater@Burgtheater|pwv}ſchauſpieler alljährlich mitwirken. Präcedenzfälle ſind da\substVorne{}\textsuperscript{.}\substDazwischen{},\substHinten{} wo ein Burgtheater\orgindex{Burgtheater@Burgtheater|pw}-Direktor ein Stück
               auf dieſe Weiſe zuerſt dem Publikum vorführte\substVorne{}\textsuperscript{.}\substDazwischen{},\substHinten{}{ }{\pb}gleichſam probeweiſe, um \strikeout{den} die Stimmung des Publikums zu ſondiren. \textsc{Fuchs\pwindex{Fuchs, Isidor 25.09.1849 – um den 20.8.1920@\textsc{Fuchs, Isidor} (25.09.1849 – um den 20.8.1920), \emph{Schriftsteller, Journalist}|pw}}, der, wie Du weißt, ein einflußreiches Mitglied\pwindex{Fuchs, Isidor 25.09.1849 – um den 20.8.1920@\textsc{Fuchs, Isidor} (25.09.1849 – um den 20.8.1920), \emph{Schriftsteller, Journalist}|pwv} der »\textsc{Concordia\orgindex{Concordia@Concordia|pw}}« iſt, will Dir gern die Sache bei \label{K_L02736-1v}\edtext{\textsc{Spigl\pwindex{Spiegl-Thurnsee, Edgar von 01.05.1839 – 29.06.1908@\textsc{Spiegl-Thurnsee, Edgar von} (01.05.1839 – 29.06.1908), \emph{Vereinspräsident, Zeitungsverleger}|pw}}}{\lemma{\textnormal{\emph{Spigl}}}\Cendnote{\textnormal{Edgar von Spiegl-Thurnsee\pwindex{Spiegl-Thurnsee, Edgar von 01.05.1839 – 29.06.1908@\textsc{Spiegl-Thurnsee, Edgar von} (01.05.1839 – 29.06.1908), \emph{Vereinspräsident, Zeitungsverleger}|pwk}, Vizepräsident\pwindex{Spiegl-Thurnsee, Edgar von 01.05.1839 – 29.06.1908@\textsc{Spiegl-Thurnsee, Edgar von} (01.05.1839 – 29.06.1908), \emph{Vereinspräsident, Zeitungsverleger}|pwkv} der \emph{Concordia}\orgindex{Concordia@Concordia|pwk}. Es sind keine Bemühungen um eine
                  Aufführung der \emph{Liebelei}\pwindex{Schnitzler, Arthur 15.05.1862 – 21.10.1931@\textsc{Schnitzler, Arthur} (15.05.1862 – 21.10.1931), \emph{Schriftsteller, Mediziner}!Liebelei. Schauspiel in drei Akten1895-10-09@\strich\emph{Liebelei. Schauspiel in drei Akten} {[}1895-10-09{]}|pwk} bei einer \emph{Concordia}\orgindex{Concordia@Concordia|pwk}-Veranstaltung bekannt.}}}\label{K_L02736-1h}
               richten. Er meint, auch \textsc{Burckhardt\pwindex{Burckhard, Max Eugen 14.07.1854 – 16.03.1912@\textsc{Burckhard, Max Eugen} (14.07.1854 – 16.03.1912), \emph{Schriftsteller, Wissenschaftler, Theaterleiter}|pw}} würde mit Freuden zuſtimmen, und ſo könnte man am Besten ein weiteres
               Hinausſchieben der Aufführung\pwindex{Schnitzler, Arthur 15.05.1862 – 21.10.1931@\textsc{Schnitzler, Arthur} (15.05.1862 – 21.10.1931), \emph{Schriftsteller, Mediziner}!Liebelei. Schauspiel in drei Akten1895-10-09@\strich\emph{Liebelei. Schauspiel in drei Akten} {[}1895-10-09{]}|pwv}
               verhindern. Außerdem gibt eine \textsc{Concordia\orgindex{Concordia@Concordia|pw}}-Vor{\pb}ſtellung eine gewiſſe Garantie für
               günſtige Referate. Was ſagſt Du zu dem Vorſchlag? Du ſollteſt ihn meiner Anſicht nach
               freilich nur annehmen, wenn Du nicht ein \uline{bindendes}
               Verſprechen von \textsc{Burckhardt\pwindex{Burckhard, Max Eugen 14.07.1854 – 16.03.1912@\textsc{Burckhard, Max Eugen} (14.07.1854 – 16.03.1912), \emph{Schriftsteller, Wissenschaftler, Theaterleiter}|pw}} erhalten könnteſt, Dich\pwindex{Schnitzler, Arthur 15.05.1862 – 21.10.1931@\textsc{Schnitzler, Arthur} (15.05.1862 – 21.10.1931), \emph{Schriftsteller, Mediziner}!Liebelei. Schauspiel in drei Akten1895-10-09@\strich\emph{Liebelei. Schauspiel in drei Akten} {[}1895-10-09{]}|pwv}{ }\uline{bald} aufzuführen. Es wäre aber nur eine Brücke für
               die Director\pwindex{Burckhard, Max Eugen 14.07.1854 – 16.03.1912@\textsc{Burckhard, Max Eugen} (14.07.1854 – 16.03.1912), \emph{Schriftsteller, Wissenschaftler, Theaterleiter}|pwv}en-Feigheit.\pend
           \pstart
           Die \textsc{Sorma\pwindex{Sorma, Agnes 17.05.1862 – 10.02.1927@\textsc{Sorma, Agnes} (17.05.1862 – 10.02.1927), \emph{Schauspielerin}|pw}} iſt in \textsc{Paris\oindex{Paris@\textbf{Paris}|pw}}. \textsc{Th. Wolff\pwindex{Wolff, Theodor 1868-08-02 – 1943-09-23@\textsc{Wolff, Theodor} (1868-08-02 – 1943-09-23), \emph{Schriftsteller, Journalist}|pw}}, der hier Correſpondent\pwindex{Wolff, Theodor 1868-08-02 – 1943-09-23@\textsc{Wolff, Theodor} (1868-08-02 – 1943-09-23), \emph{Schriftsteller, Journalist}|pwv}{ }{\pb}des »Berliner
                  Tageblatt\orgindex{Berliner Tageblatt@Berliner Tageblatt|pw}« iſt, wird mich ihr vorſtellen, und ich werde ihr von Dir
               ſprechen.\pend
           \pstart
           \textsc{\begin{otherlanguage}{french}À propos\end{otherlanguage}{ }Wolff\pwindex{Wolff, Theodor 1868-08-02 – 1943-09-23@\textsc{Wolff, Theodor} (1868-08-02 – 1943-09-23), \emph{Schriftsteller, Journalist}|pw}}: er hat in Berlin\oindex{Berlin@\textbf{Berlin}|pw} eine Geliebte\pwindex{Rosner, Mizi @\textsc{Rosner, Mizi}, \emph{Schauspielerin}|pwv}{ }\strikeout{\textcolor{gray}{f}} gehabt, die ihm lieber war, als alle andern: \label{K_L02736-2v}\edtext{\textsc{Mizzi Rosner\pwindex{Rosner, Mizi @\textsc{Rosner, Mizi}, \emph{Schauspielerin}|pw}}}{\lemma{\textnormal{\emph{Mizzi Rosner}}}\Cendnote{\textnormal{Schauspielerin\pwindex{Rosner, Mizi @\textsc{Rosner, Mizi}, \emph{Schauspielerin}|pwkv} und
                  ehemalige Geliebte\pwindex{Rosner, Mizi @\textsc{Rosner, Mizi}, \emph{Schauspielerin}|pwkv}{ }Schnitzler\pwindex{Schnitzler, Arthur 15.05.1862 – 21.10.1931@\textsc{Schnitzler, Arthur} (15.05.1862 – 21.10.1931), \emph{Schriftsteller, Mediziner}|pwk}s}}}\label{K_L02736-2h}. Die Fäden, die
               Fäden!\pend
           \pstart
           Und \textsc{Nordau\pwindex{Nordau, Max 29.07.1849 – 22.01.1923@\textsc{Nordau, Max} (29.07.1849 – 22.01.1923), \emph{Schriftsteller, Mediziner}|pw}s}{ }{\pb}\label{K_L02736-3v}\edtext{Debüt\pwindex{Nordau, Max 29.07.1849 – 22.01.1923@\textsc{Nordau, Max} (29.07.1849 – 22.01.1923), \emph{Schriftsteller, Mediziner}!Kunst in den elysaeischen Feldern1895-05-18@\strich\emph{Die Kunst in den elysäischen Feldern} {[}1895-05-18{]}|pwv}\pwindex{Nordau, Max 29.07.1849 – 22.01.1923@\textsc{Nordau, Max} (29.07.1849 – 22.01.1923), \emph{Schriftsteller, Mediziner}!Marsfeldsalon-Typen1895-05-07@\strich\emph{Marsfeldsalon-Typen} {[}1895-05-07{]}|pwv}}{\lemma{\textnormal{\emph{Debüt}}}\Cendnote{\textnormal{Im Mai 1895 erschienen zwei
                     Feuilletons\pwindex{Nordau, Max 29.07.1849 – 22.01.1923@\textsc{Nordau, Max} (29.07.1849 – 22.01.1923), \emph{Schriftsteller, Mediziner}!Kunst in den elysaeischen Feldern1895-05-18@\strich\emph{Die Kunst in den elysäischen Feldern} {[}1895-05-18{]}|pwkv}\pwindex{Nordau, Max 29.07.1849 – 22.01.1923@\textsc{Nordau, Max} (29.07.1849 – 22.01.1923), \emph{Schriftsteller, Mediziner}!Marsfeldsalon-Typen1895-05-07@\strich\emph{Marsfeldsalon-Typen} {[}1895-05-07{]}|pwkv}
                  von Max Nordau\pwindex{Nordau, Max 29.07.1849 – 22.01.1923@\textsc{Nordau, Max} (29.07.1849 – 22.01.1923), \emph{Schriftsteller, Mediziner}|pwk} in der \emph{Neuen Freie Presse}\pwindex{Neue Freie Presse1864 – 1939@\emph{Neue Freie Presse} {[}1864 – 1939{]}|pwk}: \emph{Marsfeldsalon-Typen}\pwindex{Nordau, Max 29.07.1849 – 22.01.1923@\textsc{Nordau, Max} (29.07.1849 – 22.01.1923), \emph{Schriftsteller, Mediziner}!Marsfeldsalon-Typen1895-05-07@\strich\emph{Marsfeldsalon-Typen} {[}1895-05-07{]}|pwk}. In: \emph{Neue Freie Presse}\pwindex{Neue Freie Presse1864 – 1939@\emph{Neue Freie Presse} {[}1864 – 1939{]}|pwk}, Nr. 11.027, 7. 5. 1895,
                     Morgenblatt, S. 1–4 und \emph{Die Kunst in den elysäischen Feldern}\pwindex{Nordau, Max 29.07.1849 – 22.01.1923@\textsc{Nordau, Max} (29.07.1849 – 22.01.1923), \emph{Schriftsteller, Mediziner}!Kunst in den elysaeischen Feldern1895-05-18@\strich\emph{Die Kunst in den elysäischen Feldern} {[}1895-05-18{]}|pwk}. In:
                        \emph{Neue Freie Presse}\pwindex{Neue Freie Presse1864 – 1939@\emph{Neue Freie Presse} {[}1864 – 1939{]}|pwk}, Nr. 11.038,
                        18. 5. 1895, Morgenblatt, S. 1–3. }}}\label{K_L02736-3h} in der »Neuen Freien Presse\pwindex{Neue Freie Presse1864 – 1939@\emph{Neue Freie Presse} {[}1864 – 1939{]}|pw}«? \strikeout{D\textcolor{gray}{i}} Die langſame Vorbereitung zu \label{K_L02736-4v}\edtext{\textsc{Herzl\pwindex{Herzl, Theodor 1860-05-02 – 1904-07-03@\textsc{Herzl, Theodor} (1860-05-02 – 1904-07-03), \emph{Schriftsteller, Journalist}|pw}s} Nachfolgerſchaft}{\lemma{\textnormal{\emph{Herzls Nachfolgerſchaft}}}\Cendnote{\textnormal{Nordau\pwindex{Nordau, Max 29.07.1849 – 22.01.1923@\textsc{Nordau, Max} (29.07.1849 – 22.01.1923), \emph{Schriftsteller, Mediziner}|pwk} wurde Paris\oindex{Paris@\textbf{Paris}|pwk}er Kultur-Korrespondent\pwindex{Nordau, Max 29.07.1849 – 22.01.1923@\textsc{Nordau, Max} (29.07.1849 – 22.01.1923), \emph{Schriftsteller, Mediziner}|pwkv} der \emph{Neuen Freien
                     Presse}\orgindex{Neue Freie Presse@Neue Freie Presse|pwk}.}}}\label{K_L02736-4h}. Du ahnſt gar nicht, was für frecher Blödſinn in dieſen Kunſtartikeln\pwindex{Nordau, Max 29.07.1849 – 22.01.1923@\textsc{Nordau, Max} (29.07.1849 – 22.01.1923), \emph{Schriftsteller, Mediziner}!Kunst in den elysaeischen Feldern1895-05-18@\strich\emph{Die Kunst in den elysäischen Feldern} {[}1895-05-18{]}|pwv}\pwindex{Nordau, Max 29.07.1849 – 22.01.1923@\textsc{Nordau, Max} (29.07.1849 – 22.01.1923), \emph{Schriftsteller, Mediziner}!Marsfeldsalon-Typen1895-05-07@\strich\emph{Marsfeldsalon-Typen} {[}1895-05-07{]}|pwv} ſtand.
               Aber er iſt der große Schriftſteller\pwindex{Nordau, Max 29.07.1849 – 22.01.1923@\textsc{Nordau, Max} (29.07.1849 – 22.01.1923), \emph{Schriftsteller, Mediziner}|pwv}, \textsc{Herzl\pwindex{Herzl, Theodor 1860-05-02 – 1904-07-03@\textsc{Herzl, Theodor} (1860-05-02 – 1904-07-03), \emph{Schriftsteller, Journalist}|pw}} ſelbſt hat ihn candidirt, ich bin ein guter Reporter und zähle nicht mit. Von
                  \textsc{Herzl\pwindex{Herzl, Theodor 1860-05-02 – 1904-07-03@\textsc{Herzl, Theodor} (1860-05-02 – 1904-07-03), \emph{Schriftsteller, Journalist}|pw}} überraſcht mich das nicht. {\pb}Trotz aller
               äußeren Collegialitäts-Tünche haben wir uns im Grunde immer gehaßt, und ich habe auch
               nichts gemeinſam mit dieſem engherzigen, doktrinär vernagelten Menſchen\pwindex{Herzl, Theodor 1860-05-02 – 1904-07-03@\textsc{Herzl, Theodor} (1860-05-02 – 1904-07-03), \emph{Schriftsteller, Journalist}|pwv} von echt rabbiniſtiſchem Spitz-
               und Dürr-Geiſte.\pend
           \pstart
           Nur thut es eben gar ſo weh, ſich ſo übergangen zu ſehen {\pb}und immer und ewig der Menſch zweiten oder dritten
               Ranges zu ſein.\pend
           \pstart
           Grüß’ Dich Gott, mein lieber Freund, und laß wieder von Dir hören!\pend
           \pstart
           Dein {\\[\baselineskip]}treuer {\\[\baselineskip]}\spacefill\mbox{Paul Goldmann}\pend
           \leftskip=0em{}
         
         \endnumbering\mylabel{h}\end{ledgroupsized}  \newcommand{\dateiname}{L02736}\newcommand{\titel}{Paul Goldmann an Arthur Schnitzler, 7. 6. [1895]}\newcommand{\editorInnen}{Martin Anton Müller und Laura Untner}%% latex-leseansicht-abspann.tex
%% Abspann für die Leseansicht.
%% Der Schalter \ifkorrekturansicht ist bereits durch den Vorspann gesetzt.

%% latex-abspann.tex
%% Gemeinsamer Abspann für Korrekturansicht und Leseansicht.
%% Setzt den Schalter \ifkorrekturansicht voraus (gesetzt in den
%% einbindenden Dateien latex-korrekturansicht-abspann.tex bzw.
%% latex-leseansicht-abspann.tex).
%% ---------------------------------------------------------------

\normalsize

% Das esempio-Environment wird nur in der Leseansicht benötigt
\ifkorrekturansicht\else
\newenvironment{esempio}[3]%
{
    \vspace{1.5ex}
    \rlap{\underline{#1}}
    \par
    \setlength{\parindent}{0cm}
    \nopagebreak
    \leftskip=#2cm
    \rightskip=#3cm
}
{
    \par
}
\fi

\doendnotes{C}
\bigskip
\vfill

\clearpage

\footnotesize

\ifkorrekturansicht
  \lohead{\textsc{register}}
\fi

% theindex-Environment neu definieren ohne reledmac
\makeatletter
\renewenvironment{theindex}{%
  \ifkorrekturansicht
    \section*{\indexname}%
  \else
    \subsubsection*{Index der erwähnten Entitäten}%
  \fi
  \setlength{\parindent}{0pt}%
  \setlength{\parskip}{0pt plus 0.3pt}%
  \let\item\@idxitem
}{%
  \ifkorrekturansicht\clearpage\fi
}
\makeatother

\IfFileExists{\jobname-pw.ind}{\input{\jobname-pw.ind}}{}

% Quellenangabe nur in der Leseansicht
\ifkorrekturansicht\else
% Fallback-Definitionen, falls die .tex-Datei \titel etc. nicht gesetzt hat
\providecommand{\titel}{}
\providecommand{\editorInnen}{}
\providecommand{\dateiname}{\jobname}

\vspace{3cm}

\vfill

\footnotesize
\textsc{Quelle}: \titel. Herausgegeben von {\editorInnen}. In: \emph{Arthur Schnitzler: Briefwechsel mit Autorinnen und Autoren}.
 Digitale Edition, https://schnitzler-briefe.acdh.oeaw.ac.at/{\dateiname}.html (Stand \today)
\fi

\end{document}


      