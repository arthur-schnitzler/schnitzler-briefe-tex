%% latex-korrekturansicht-vorspann.tex
%% Vorspann für die Korrekturansicht.
%% Lädt die gemeinsame Datei latex-vorspann.tex mit gesetztem Schalter.

\newif\ifkorrekturansicht
\korrekturansichttrue

\input{../tex-inputs/latex-vorspann}


\section[Paul Goldmann an Arthur Schnitzler, {[}5.? 2. 1896{]}]{L02688 Paul Goldmann an Arthur Schnitzler, {[}5.? 2. 1896{]}}
\nopagebreak\mylabel{L02688v}
\rehead{ }\normalsize\beginnumbering\briefempfaengerindex{Schnitzler, Arthur@\textsc{Schnitzler, Arthur}!zzzGoldmann, Paul@\emph{von Paul Goldmann}!1896-02-052@{{[}5.? 2. 1896{]}}|(be}
\toendnotes[C]{\smallbreak\pagebreak[2]}\Standort{DLA, A:Schnitzler, HS.NZ85.1.3166.}
\physDesc{Telegramm, 48 Zeichen
\newline{}maschinell
\newline{}Schnitzler: mit Bleistift datiert: »Feber 99« 
\newline{}Ordnung: beschnitten }\toendnotes[C]{\smallbreak}
\pstart
           \noindent{}{\pb}\label{K_L02688-1v}\edtext{glueckwunsch}{\lemma{\textnormal{\emph{glueckwunsch}}}\Cendnote{\textnormal{Wohl zur Premiere von \emph{Liebelei}\pwindex{Liebelei. Schauspiel in drei Akten@\emph{Liebelei. Schauspiel in drei Akten}|pwk} (gemeinsam mit \emph{Der zerbrochene
                     Krug}\pwindex{zerbrochene Krug. Ein Lustspiel in drei Aufzuegen@\emph{Der zerbrochene Krug. Ein Lustspiel in drei Aufzügen}|pwk}) am 4. 2. 1896 im \emph{Deutschen
                     Theater}\orgindex{Deutsches Theater Berlin@Deutsches Theater Berlin|pwk} in Berlin\oindex{Berlin@\textbf{Berlin}, \emph{P.PPLC}|pwk}. Schnitzler war anwesend, weswegen dieses Telegramm nach Berlin\oindex{Berlin@\textbf{Berlin}, \emph{P.PPLC}|pwk} gerichtet gewesen sein dürfte.}}}\label{K_L02688-1}
               von ganzem herzen. gruss. \spacefill\mbox{goldmann =«}\pend
           \selectlanguage{ngerman}\endnumbering\briefempfaengerindex{Schnitzler, Arthur@\textsc{Schnitzler, Arthur}!zzzGoldmann, Paul@\emph{von Paul Goldmann}!1896-02-052@{{[}5.? 2. 1896{]}}|)be}\mylabel{L02688h}  \normalsize

\doendnotes{C}
\bigskip
\vfill

\clearpage

\footnotesize

\lohead{\textsc{register}}

% Definiere theindex-Environment komplett neu ohne reledmac
\makeatletter
\renewenvironment{theindex}{%
  \section*{\indexname}%
  \setlength{\parindent}{0pt}%
  \setlength{\parskip}{0pt plus 0.3pt}%
  \let\item\@idxitem
}{%
  \clearpage
}
\makeatother

\IfFileExists{\jobname-pw.ind}{\input{\jobname-pw.ind}}{}

\end{document}

      