%% latex-leseansicht-vorspann.tex
%% Vorspann für die Leseansicht.
%% Lädt die gemeinsame Datei latex-vorspann.tex mit nicht gesetztem Schalter.

\newif\ifkorrekturansicht
\korrekturansichtfalse

\input{../tex-inputs/latex-vorspann}


\section[ Felix Salten an Arthur Schnitzler, 1. 3. [1924]]{L03594 Felix Salten an Arthur Schnitzler,  1. 3. [1924]}
\nopagebreak\mylabel{L03594v}
\rehead{ }\normalsize\beginnumbering\briefempfaengerindex{Schnitzler, Arthur@\textsc{Schnitzler, Arthur}!zzzSalten, Felix@\emph{von Felix Salten}!1924-03-011@{1. 3. [1924]}|(be}
\toendnotes[C]{\smallbreak\pagebreak[2]}
\correspDesc{Versand  durch Felix Salten am 1. 3. [1924] in Assuan
\newline{}Übermittlung  am 3. 3. 1924 in Assuan
\newline{}Erhalt  durch Arthur Schnitzler im Zeitraum [6. 3. 1924
                  – 16. 3. 1924?] in Wien}\toendnotes[C]{\smallbreak}
\Standort{CUL, Schnitzler, B 89, B 2.}
\physDesc{Bildpostkarte, 244 Zeichen
\newline{}Handschrift: schwarze Tinte, lateinische Kurrent
\newline{}Versand: Stempel: »\nobreak{}\oindex{Old Cataract Hotel@\textbf{Old Cataract Hotel}, \emph{Hotel}|pwk}Cata\textcolor{gray}{ract Hote}{[}l{]}, 3. \textcolor{gray}{Mar}{[}ch 1924{]}\nobreak{}«.  
\newline{}Ordnung: mit Bleistift von unbekannter Hand nummeriert: »295« }\pstart{}{\pb}Austria\oindex{Österreich@\textbf{Österreich}|pw}\pend{}\pstart{}Herrn D\textsuperscript{r} Arthur Schnitzler\pend{}\pstart{}\begin{otherlanguage}{english}Vienna\oindex{Wien@\textbf{Wien}, \emph{Verwaltungsgebiet}|pw}\end{otherlanguage}\pend{}\pstart{}XVIII. Sternwartestrasse 71\oindex{Wien@\textbf{Wien}!XVIII., Währing@\textbf{XVIII., Währing}!Sternwartestraße 71@\textbf{Sternwartestraße 71}, \emph{Wohngebäude}|pw}\pend{}{\bigskip}
\pstart
           \noindent{}{\pb}\textcolor{gray}{\textbf{\textsc{No. 36 – Karnak. Ptolomey Gateways and the Temple of Khonsu, God
                        of the Moon\oindex{Karnak Tempelanlage@\textbf{Karnak Tempelanlage}|pw}}.}}\pend
           \vspace{1em}
\pstart
           \raggedleft{}{\pb}Assuan\oindex{Assuan@\textbf{Assuan}|pw}, 1. III\pend
           \vspace{0.5em}
\pstart
           Es ist halt doch sehr schön, schon um diese Zeit 30° Hitze zu haben – aber leben
               möchte man hier trotzdem nicht. Herzliche Grüße, auch an Heini\pwindex{Schnitzler, Heinrich 9.\,8.\,1902 Hinterbrühl – 12.\,7.\,1982 Wien@\textsc{Schnitzler, Heinrich} (9.\,8.\,1902 Hinterbrühl – 12.\,7.\,1982 Wien), \emph{Regisseur, Schauspieler}|pw} und Lilli\pwindex{Schnitzler, Lilly 3.\,7.\,1911 Wien – 17.\,5.\,2009 ebd.@\textsc{Schnitzler, Lilly} (3.\,7.\,1911 Wien – 17.\,5.\,2009 ebd.), \emph{Violinistin}|pw}.
               {\\}Ihr {\\}\spacefill\mbox{Felix Salten}\pend
           \selectlanguage{ngerman}\endnumbering\briefempfaengerindex{Schnitzler, Arthur@\textsc{Schnitzler, Arthur}!zzzSalten, Felix@\emph{von Felix Salten}!1924-03-011@{1. 3. [1924]}|)be}\mylabel{L03594h}  \newcommand{\dateiname}{L03594}\newcommand{\titel}{Felix Salten an Arthur Schnitzler, 1. 3. [1924]}\newcommand{\editorInnen}{Martin Anton Müller und Laura Untner}%% latex-leseansicht-abspann.tex
%% Abspann für die Leseansicht.
%% Der Schalter \ifkorrekturansicht ist bereits durch den Vorspann gesetzt.

%% latex-abspann.tex
%% Gemeinsamer Abspann für Korrekturansicht und Leseansicht.
%% Setzt den Schalter \ifkorrekturansicht voraus (gesetzt in den
%% einbindenden Dateien latex-korrekturansicht-abspann.tex bzw.
%% latex-leseansicht-abspann.tex).
%% ---------------------------------------------------------------

\normalsize

% Das esempio-Environment wird nur in der Leseansicht benötigt
\ifkorrekturansicht\else
\newenvironment{esempio}[3]%
{
    \vspace{1.5ex}
    \rlap{\underline{#1}}
    \par
    \setlength{\parindent}{0cm}
    \nopagebreak
    \leftskip=#2cm
    \rightskip=#3cm
}
{
    \par
}
\fi

\doendnotes{C}
\bigskip
\vfill

\clearpage

\footnotesize

\ifkorrekturansicht
  \lohead{\textsc{register}}
\fi

% theindex-Environment neu definieren ohne reledmac
\makeatletter
\renewenvironment{theindex}{%
  \ifkorrekturansicht
    \section*{\indexname}%
  \else
    \subsubsection*{Index der erwähnten Entitäten}%
  \fi
  \setlength{\parindent}{0pt}%
  \setlength{\parskip}{0pt plus 0.3pt}%
  \let\item\@idxitem
}{%
  \ifkorrekturansicht\clearpage\fi
}
\makeatother

\IfFileExists{\jobname-pw.ind}{\input{\jobname-pw.ind}}{}

% Quellenangabe nur in der Leseansicht
\ifkorrekturansicht\else
% Fallback-Definitionen, falls die .tex-Datei \titel etc. nicht gesetzt hat
\providecommand{\titel}{}
\providecommand{\editorInnen}{}
\providecommand{\dateiname}{\jobname}

\vspace{3cm}

\vfill

\footnotesize
\textsc{Quelle}: \titel. Herausgegeben von {\editorInnen}. In: \emph{Arthur Schnitzler: Briefwechsel mit Autorinnen und Autoren}.
 Digitale Edition, https://schnitzler-briefe.acdh.oeaw.ac.at/{\dateiname}.html (Stand \today)
\fi

\end{document}


