%% latex-leseansicht-vorspann.tex
%% Vorspann für die Leseansicht.
%% Lädt die gemeinsame Datei latex-vorspann.tex mit nicht gesetztem Schalter.

\newif\ifkorrekturansicht
\korrekturansichtfalse

\input{../tex-inputs/latex-vorspann}


\section[Hugo von Hofmannsthal an Arthur Schnitzler, {[}16. 1. 1897{]}]{L00640 Hugo von Hofmannsthal an Arthur Schnitzler, {[}16. 1. 1897{]}}
\nopagebreak\mylabel{L00640v}
\rehead{ }\normalsize\beginnumbering\briefempfaengerindex{Schnitzler, Arthur@\textsc{Schnitzler, Arthur}!zzzHofmannsthal, Hugo von@\emph{von Hugo von Hofmannsthal}!1897-01-163@{{[}16. 1. 1897{]}}|(be}
\toendnotes[C]{\smallbreak\pagebreak[2]}
\correspDesc{Versand  durch Hugo von Hofmannsthal am [16. 1. 1897] in Wien
\newline{}Erhalt  durch Arthur Schnitzler im Zeitraum [16. 1. 1897
                  – 20. 1. 1897?] in Wien}\toendnotes[C]{\smallbreak}
\Standort{CUL, Schnitzler, B 43.}
\physDesc{Brief, 1 Blatt, 3 Seiten, 775 Zeichen
\newline{}Handschrift: schwarze Tinte, deutsche Kurrent
\newline{}Schnitzler: mit Bleistift datiert: »16/1 97« 
\newline{}Ordnung: mit Bleistift von unbekannter Hand nummeriert:
                                    »85« }
\buchAbdrucke{\weitereDrucke{Hugo von Hofmannsthal, Arthur Schnitzler: \emph{Briefwechsel}. Herausgegeben von Therese Nickl und Heinrich Schnitzler. Frankfurt am Main: \emph{S. Fischer} 1964, S. 77.} }\toendnotes[C]{\smallbreak}
\pstart
           {\pb}\textcolor{gray}{\textbf{\label{T_L00640-1v}\edtext{hvH}{\lemma{\textnormal{\emph{hvH}}}\Cendnote{\textnormal{gedrucktes Monogramm mit Krone in blauer Farbe}}}\label{T_L00640-1}}}\pend
           
\pstart
           \raggedleft{}\label{K_L00640-1v}\edtext{Samstag}{\lemma{\textnormal{\emph{Samstag}}}\Cendnote{\textnormal{Am Samstag, dem 16. 1. 1897
                        erschien der dritte und letzte Teil des Erstdrucks von \emph{Die Frau des Weisen. Erzählung}\pwindex{Schnitzler, Arthur 15.\,5.\,1862 Wien – 21.\,10.\,1931 ebd.@\textsc{Schnitzler, Arthur} (15.\,5.\,1862 Wien – 21.\,10.\,1931 ebd.), \emph{Schriftsteller, Mediziner}!Frau des Weisen. Erzählung@\strich\emph{Die Frau des Weisen. Erzählung}|pwk} in der Wochenschrift \emph{Die Zeit}\orgindex{Zeit. Wiener Wochenschrift@Die Zeit. Wiener Wochenschrift|pwk} (Bd. 10, Nr. 118,
                              2. 1. 1897, S. 15–16; Nr. 119, 9. 1. 1897,
                           S. 31–32; Nr. 120, 16. 1. 1897, S. 47–48).}}}\label{K_L00640-1}.\pend
           
\pstart{}mein lieber Arthur\pend\vspace{0.5em}
\pstart
           ich{ }ſehe Sie, glaub ich, weder heute im Café noch \label{K_L00640-2v}\edtext{morgen}{\lemma{\textnormal{\emph{morgen}}}\Cendnote{\textnormal{Am
                     17. 1. 1897 war Hofmannsthal\pwindex{Hofmannsthal, Hugo von 1.\,2.\,1874 Wien – 15.\,7.\,1929 Rodaun@\textsc{Hofmannsthal, Hugo von} (1.\,2.\,1874 Wien – 15.\,7.\,1929 Rodaun), \emph{Schriftsteller}|pwk} bei Louis\pwindex{Loeb, Louis 29.\,6.\,1842 Mattersdorf – 6.\,6.\,1921 Wien@\textsc{Loeb, Louis} (29.\,6.\,1842 Mattersdorf – 6.\,6.\,1921 Wien), \emph{Bankier}|pwk} und Regina Loeb\pwindex{Loeb, Regina 1850 – 5.\,2.\,1918 Wien@\textsc{Loeb, Regina} (1850 – 5.\,2.\,1918 Wien)|pwk} (Hugo von Hofmannsthal\pwindex{Hofmannsthal, Hugo von 1.\,2.\,1874 Wien – 15.\,7.\,1929 Rodaun@\textsc{Hofmannsthal, Hugo von} (1.\,2.\,1874 Wien – 15.\,7.\,1929 Rodaun), \emph{Schriftsteller}|pwk}: \emph{Aufzeichnungen}. Herausgegeben von Rudolf Hirsch † und Ellen Ritter † in
                     Zusammenarbeit mit Konrad Heumann und Peter Michael Braunwarth. Frankfurt am
                     Main: \emph{S. Fischer}\orgindex{S. Fischer Verlag@S. Fischer Verlag|pwk}{ }2013, S. 378 (\emph{Sämtliche Werke},
                     XXXIX)).}}}\label{K_L00640-2} bei L.\pwindex{Loeb, Louis 29.\,6.\,1842 Mattersdorf – 6.\,6.\,1921 Wien@\textsc{Loeb, Louis} (29.\,6.\,1842 Mattersdorf – 6.\,6.\,1921 Wien), \emph{Bankier}|pw}\pwindex{Loeb, Regina 1850 – 5.\,2.\,1918 Wien@\textsc{Loeb, Regina} (1850 – 5.\,2.\,1918 Wien)|pw}
               und möchte Ihnen doch{ }ſagen, daſs die »Frau des
                  Weiſen\pwindex{Schnitzler, Arthur 15.\,5.\,1862 Wien – 21.\,10.\,1931 ebd.@\textsc{Schnitzler, Arthur} (15.\,5.\,1862 Wien – 21.\,10.\,1931 ebd.), \emph{Schriftsteller, Mediziner}!Frau des Weisen. Erzählung@\strich\emph{Die Frau des Weisen. Erzählung}|pw}« eine{ }ſehr{ }ſchöne Novelle iſt. Ich war von der Führung des Schluſſes
               überraſcht wie von einer völlig unerwarteten und {\pb}doch unendlich einfachen
               naheliegenden Löſung einer Rechenaufgabe, das was man in der Mathematik eine »ſchöne
               Löſung« nennt. Auch iſt alles Äußerliche, das den Fortgang der Handlung unterſtützt,
               wunderſchön{ }ſparſam und durchſichtig. Man{ }ſieht die Landſchaft nicht, man glaubt{ }ſich
               in ihr zu bewegen\strikeout{d}, und {\pb}fühlt \uline{unmittelbar} ihre Wirkung auf’s Gemüth der handelnden Perſonen.\pend
           
\pstart
           Ich bin{ }ſchläfrig, und kann mich nicht gut ausdrücken. Sie waren übrigens in den
               letzten Tagen beſonders lieb und nett gegen mich.\pend
           
\pstart
           Herzlich Ihr{\\[\baselineskip]}\spacefill\mbox{Hugo.}\pend
           \leftskip=0em{}\selectlanguage{ngerman}\endnumbering\briefempfaengerindex{Schnitzler, Arthur@\textsc{Schnitzler, Arthur}!zzzHofmannsthal, Hugo von@\emph{von Hugo von Hofmannsthal}!1897-01-163@{{[}16. 1. 1897{]}}|)be}\mylabel{L00640h}  \newcommand{\dateiname}{L00640}\newcommand{\titel}{Hugo von Hofmannsthal an Arthur Schnitzler, [16. 1. 1897]}\newcommand{\editorInnen}{Martin Anton Müller und Gerd-Hermann Susen}%% latex-leseansicht-abspann.tex
%% Abspann für die Leseansicht.
%% Der Schalter \ifkorrekturansicht ist bereits durch den Vorspann gesetzt.

%% latex-abspann.tex
%% Gemeinsamer Abspann für Korrekturansicht und Leseansicht.
%% Setzt den Schalter \ifkorrekturansicht voraus (gesetzt in den
%% einbindenden Dateien latex-korrekturansicht-abspann.tex bzw.
%% latex-leseansicht-abspann.tex).
%% ---------------------------------------------------------------

\normalsize

% Das esempio-Environment wird nur in der Leseansicht benötigt
\ifkorrekturansicht\else
\newenvironment{esempio}[3]%
{
    \vspace{1.5ex}
    \rlap{\underline{#1}}
    \par
    \setlength{\parindent}{0cm}
    \nopagebreak
    \leftskip=#2cm
    \rightskip=#3cm
}
{
    \par
}
\fi

\doendnotes{C}
\bigskip
\vfill

\clearpage

\footnotesize

\ifkorrekturansicht
  \lohead{\textsc{register}}
\fi

% theindex-Environment neu definieren ohne reledmac
\makeatletter
\renewenvironment{theindex}{%
  \ifkorrekturansicht
    \section*{\indexname}%
  \else
    \subsubsection*{Index der erwähnten Entitäten}%
  \fi
  \setlength{\parindent}{0pt}%
  \setlength{\parskip}{0pt plus 0.3pt}%
  \let\item\@idxitem
}{%
  \ifkorrekturansicht\clearpage\fi
}
\makeatother

\IfFileExists{\jobname-pw.ind}{\input{\jobname-pw.ind}}{}

% Quellenangabe nur in der Leseansicht
\ifkorrekturansicht\else
% Fallback-Definitionen, falls die .tex-Datei \titel etc. nicht gesetzt hat
\providecommand{\titel}{}
\providecommand{\editorInnen}{}
\providecommand{\dateiname}{\jobname}

\vspace{3cm}

\vfill

\footnotesize
\textsc{Quelle}: \titel. Herausgegeben von {\editorInnen}. In: \emph{Arthur Schnitzler: Briefwechsel mit Autorinnen und Autoren}.
 Digitale Edition, https://schnitzler-briefe.acdh.oeaw.ac.at/{\dateiname}.html (Stand \today)
\fi

\end{document}


