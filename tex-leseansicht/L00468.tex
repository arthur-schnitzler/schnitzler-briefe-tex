%% latex-leseansicht-vorspann.tex
%% Vorspann für die Leseansicht.
%% Lädt die gemeinsame Datei latex-vorspann.tex mit nicht gesetztem Schalter.

\newif\ifkorrekturansicht
\korrekturansichtfalse

\input{../tex-inputs/latex-vorspann}


         
         \renewcommand{\erwaehntePersonen}{Personen: Peter Altenberg, Richard Beer-Hofmann}
         \renewcommand{\erwaehnteOrte}{Orte: Bad Ischl, Gmunden, Karlsbad, Marienbad, Ostende, Scheveningen, Schweiz, Traunstein (Berg), Wien}
         \renewcommand{\erwaehnteWerke}{}
               \section[Peter Altenberg an Arthur Schnitzler, {[}30. 7. 1895{]}]{ Peter Altenberg an Arthur Schnitzler, {[}30. 7. 1895{]}}\nopagebreak\mylabel{v}\rehead{ }\begin{ledgroupsized}[t]{13cm}\normalsize\beginnumbering\briefempfaengerindex{Schnitzler, Arthur@\textsc{Schnitzler, Arthur}!zzzAltenberg, Peter@\emph{von Peter Altenberg}!1895-07-301@{{[}30. 7. 1895{]}}|(be} \toendnotes[C]{\smallbreak\pagebreak[2]} \Standort{CUL, Schnitzler, B 2.}
\physDesc{Brief, 1 Blatt, 4 Seiten, 1382 Zeichen
\newline{}Handschrift: schwarze Tinte, deutsche Kurrent
\newline{}Schnitzler: 1) mit Bleistift beschrieben: »Gmunden\oindex{Gmunden@\textbf{Gmunden}|pw}{ }30/7 95« und nummeriert: »4«  2) mit rotem Buntstift eine Unterstreichung
\newline{}Ordnung: mit Bleistift von unbekannter Hand nummeriert:
                                 »3« }\buchAbdrucke{\weitereDrucke{1) Kurt Bergel: \emph{Arthur Schnitzlers unveröffentlichte Tragikomödie Das Wort.} In: \emph{Studies in Arthur Schnitzler. Centennial Commemorative
                        Volume}. Hg. Herbert W. Reichert und Herman Salinger. Chapel Hill: \emph{University of North Carolina Press} 1963, S. 19–20 (UNC Studies in the Germanic Languages and Literatures, 42).} \weitereDrucke{2) Arthur Schnitzler: \emph{Das Wort. Tragikomödie in fünf Akten. Fragment}. Aus dem Nachlaß hg. und eingeleitet von Kurt Bergel. Frankfurt am Main: \emph{S. Fischer Verlag} 1966, S. 7–8.} \weitereDrucke{3) Peter Altenberg: \emph{Die Selbsterfindung eines Dichters. Briefe und Dokumente
                        1892–1896}. Hg. und mit einem Nachwort von Leo A. Lensing. Göttingen: \emph{Wallstein} 2009, S. 32.} }\toendnotes[C]{\smallbreak}\pstart{}{\pb}Lieber \textsc{D\textsuperscript{r.}} Arthur Schnitzler.\pend\pstart
           Ich habe nach Wien\oindex{Wien@\textbf{Wien}|pw} geſchrieben in ihrer \label{K_L00468-1v}\edtext{Angelegenheit}{\lemma{\textnormal{\emph{Angelegenheit}}}\Cendnote{\textnormal{Schnitzler\pwindex{Schnitzler, Arthur 15.05.1862 – 21.10.1931@\textsc{Schnitzler, Arthur} (15.05.1862 – 21.10.1931), \emph{Schriftsteller, Mediziner}|pwk} dürfte um die Lieferung von
                  Zigaretten gebeten haben. Vgl. Kommentar zum Brief in \emph{Die
                        Selbsterfindung eines Dichters}, S. 142.}}}\label{K_L00468-1h}, glaube aber, daß es mit Schwierigkeiten verbunden ſein dürfte. Jedenfalls
               benachrichtige ich Sie. Kommen Sie doch \label{K_L00468-2v}\edtext{herüber}{\lemma{\textnormal{\emph{herüber}}}\Cendnote{\textnormal{Bereits am Folgetag radelte
                     Schnitzler\pwindex{Schnitzler, Arthur 15.05.1862 – 21.10.1931@\textsc{Schnitzler, Arthur} (15.05.1862 – 21.10.1931), \emph{Schriftsteller, Mediziner}|pwk} nach Gmunden\oindex{Gmunden@\textbf{Gmunden}|pwk}.}}}\label{K_L00468-2h}. Sie ſind geſund u. mobil. Kommen Sie mit Richard Beer-Hofmann\pwindex{Beer-Hofmann, Richard 1866-07-11 – 1945-09-26@\textsc{Beer-Hofmann, Richard} (1866-07-11 – 1945-09-26), \emph{Schriftsteller}|pw}. Ich bin wie ſtets von Gmunden\oindex{Gmunden@\textbf{Gmunden}|pw} tief entzückt. Es iſt gleichſam für {\pb}mich geſchaffen. Und dann, es muß mir halt
               die Welten-Schönheit \strikeout{rp} repräſentiren. Wenn die Leute
               am Strande hin u. hertrippeln, iſt es Oſtende\oindex{Ostende@\textbf{Ostende}|pw},
                  Sch\introOben{}e\introOben{}weningen\oindex{Scheveningen@\textbf{Scheveningen}|pw}, wenn
               die Muſik ſpielt u. Damen in \textsc{Chiné}-Seide erſcheinen, iſt es
               Karlsbad\oindex{Karlsbad@\textbf{Karlsbad}|pw}, Marienbad\oindex{Marienbad@\textbf{Marienbad}|pw}, wenn der Traunſtein\oindex{Traunstein (Berg)@\textbf{Traunstein (Berg)}|pw} ziegelroth wird, iſt es
               die Schweiz\oindex{Schweiz@\textbf{Schweiz}|pw} u. wenn der Abendfriede ko{\geminationm}t, iſt \strikeout{d} es die \strikeout{?} Welt, die Zukunft, \introOben{}das Ende.\introOben{}
               Glauben Sie mir, lieber \textsc{D\textsuperscript{r}}. Arthur, wir Armen ſind wie gewiſſe {\pb}Kranke. Gewiſſe Organe verfeinern ſich, erhöhen ihre Leiſtungsfähigkeiten, um den
               Ausfall anderer zu decken. So iſt es mit der Potenz in jeder Form. Ekonomiſche
               Kräfte, \textsc{sexuelle} Kräfte, werden durch erhöhte ſeeliſche
               ausgeglichen. Das Gehirn überni{\geminationm}t gleichſam ihre Aufgabe
               u. macht ſich die Verkümmerung zu Nutze.\pend
           \pstart
           Sie werden ſagen: »Das iſt nicht Harmonie, mein Lieber – – –.« {\pb}Wenn Sie das aber nicht antworten, werde
               ich Sie noch höher ſchätzen, nach meinem berühmten\introOben{}!?\introOben{}
               Ausſpruch: »\uline{Weiſe ſein heißt, \strikeout{h} auch das noch verſtehen, was man nicht mehr verſteht!!}«\pend
           \pstart
           Adieu, alſo ko{\geminationm}en Sie doch herüber.\pend
           \pstart
           Ihr aufrichtig freundſchaftlicher{\\[\baselineskip]}\spacefill\mbox{Richard Engländer.}\pend
           \leftskip=0em{}
         
         \endnumbering\mylabel{h}\end{ledgroupsized}  \newcommand{\dateiname}{L00468}\newcommand{\titel}{Peter Altenberg an Arthur Schnitzler, [30. 7. 1895]}\newcommand{\editorInnen}{Martin Anton Müller und Gerd-Hermann Susen}%% latex-leseansicht-abspann.tex
%% Abspann für die Leseansicht.
%% Der Schalter \ifkorrekturansicht ist bereits durch den Vorspann gesetzt.

%% latex-abspann.tex
%% Gemeinsamer Abspann für Korrekturansicht und Leseansicht.
%% Setzt den Schalter \ifkorrekturansicht voraus (gesetzt in den
%% einbindenden Dateien latex-korrekturansicht-abspann.tex bzw.
%% latex-leseansicht-abspann.tex).
%% ---------------------------------------------------------------

\normalsize

% Das esempio-Environment wird nur in der Leseansicht benötigt
\ifkorrekturansicht\else
\newenvironment{esempio}[3]%
{
    \vspace{1.5ex}
    \rlap{\underline{#1}}
    \par
    \setlength{\parindent}{0cm}
    \nopagebreak
    \leftskip=#2cm
    \rightskip=#3cm
}
{
    \par
}
\fi

\doendnotes{C}
\bigskip
\vfill

\clearpage

\footnotesize

\ifkorrekturansicht
  \lohead{\textsc{register}}
\fi

% theindex-Environment neu definieren ohne reledmac
\makeatletter
\renewenvironment{theindex}{%
  \ifkorrekturansicht
    \section*{\indexname}%
  \else
    \subsubsection*{Index der erwähnten Entitäten}%
  \fi
  \setlength{\parindent}{0pt}%
  \setlength{\parskip}{0pt plus 0.3pt}%
  \let\item\@idxitem
}{%
  \ifkorrekturansicht\clearpage\fi
}
\makeatother

\IfFileExists{\jobname-pw.ind}{\input{\jobname-pw.ind}}{}

% Quellenangabe nur in der Leseansicht
\ifkorrekturansicht\else
% Fallback-Definitionen, falls die .tex-Datei \titel etc. nicht gesetzt hat
\providecommand{\titel}{}
\providecommand{\editorInnen}{}
\providecommand{\dateiname}{\jobname}

\vspace{3cm}

\vfill

\footnotesize
\textsc{Quelle}: \titel. Herausgegeben von {\editorInnen}. In: \emph{Arthur Schnitzler: Briefwechsel mit Autorinnen und Autoren}.
 Digitale Edition, https://schnitzler-briefe.acdh.oeaw.ac.at/{\dateiname}.html (Stand \today)
\fi

\end{document}


      