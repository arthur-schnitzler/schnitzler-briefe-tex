%% latex-korrekturansicht-vorspann.tex
%% Vorspann für die Korrekturansicht.
%% Lädt die gemeinsame Datei latex-vorspann.tex mit gesetztem Schalter.

\newif\ifkorrekturansicht
\korrekturansichttrue

\input{../tex-inputs/latex-vorspann}


\section[Arthur Schnitzler an Richard Beer-Hofmann, 22. 6. 1895]{L00456 Arthur Schnitzler an Richard Beer-Hofmann, 22. 6. 1895}
\nopagebreak\mylabel{L00456v}
\rehead{ }\normalsize\beginnumbering\briefempfaengerindex{Beer-Hofmann, Richard@\textsc{Beer-Hofmann, Richard}!zzzSchnitzler, Arthur@\emph{von Arthur Schnitzler}!1895-06-221@{22. 6. 1895}|(be}
\toendnotes[C]{\smallbreak\pagebreak[2]}\Standort{YCGL, MSS 31.}
\physDesc{Brief, 2 Blätter, 7 Seiten, Umschlag, 943 Zeichen
\newline{}Handschrift: 1) Bleistift, deutsche Kurrent\hspace{1em}2) schwarze Tinte, deutsche Kurrent (\noindent{}Umschlag)\hspace{1em}
\newline{}Versand: 1) Stempel: »\nobreak{}\oindex{I., Innere Stadt@\textbf{I., Innere Stadt}, \emph{A.ADM3}|pwk}Wien
                                       {[}1{]}/1, 22. {[}6{]}. 95, 8–9\nobreak{}«.   2) Stempel: »\nobreak{}\oindex{Cáslav@\textbf{Čáslav}, \emph{P.PPL}|pwk}{\pb}Časlau Časlav, 23 / 6 / 95, 8–9\nobreak{}«. }
\buchAbdrucke{\weitereDrucke{Arthur Schnitzler, Richard Beer-Hofmann: \emph{Briefwechsel 1891–1931}. Wien, Zürich: \emph{Europaverlag} 1992, S. 75.} }\toendnotes[C]{\smallbreak}\pstart{}{\pb}Herrn kuk. u. a. Lieutenant\pend{}\pstart{}\textsc{Dr. Richard Beer-Hofmann}\pend{}\pstart{}im k. k. Landw. Inf.-Regmt\pend{}\pstart{}\textsc{Caslau Nr 12\oindex{Cáslav@\textbf{Čáslav}, \emph{P.PPL}|pw}.}\pend{}{\bigskip}\vspace{1em}
\pstart{}{\pb}Lieber Richard\pend\vspace{0.5em}
\pstart
           wann ko{\geminationm}en Sie? Werden Sie mich noch hier antreffen? Ich
               verreiſe wahrscheinlich am 2. Juli.\pend
           
\pstart
           {\pb}\textsc{Hugo}\pwindex{Hofmannsthal, Hugo von 1874-02-01 – 1929-07-15@\textsc{Hofmannsthal, Hugo von} (1874-02-01 – 1929-07-15), \emph{Schriftsteller/Schriftstellerin}|pw}{ }ſoll heute in Wien\oindex{Wien@\textbf{Wien}, \emph{A.ADM2}|pw}{ }ſein, telephonirte mir ſein Vater\pwindex{Hofmannsthal, Hugo August von 21.12.1841 – 08.12.1915@\textsc{Hofmannsthal, Hugo August von} (21.12.1841 – 08.12.1915), \emph{Bankdirektor/Bankdirektorin}|pwv}; vielleicht treff ich ihn heute
               Abend. – \textsc{Salten}\pwindex{Salten, Felix 06.09.1869 – 08.10.1945@\textsc{Salten, Felix} (06.09.1869 – 08.10.1945), \emph{Schriftsteller/Schriftstellerin, Journalist/Journalistin, Chefredakteur/Chefredakteurin}|pw}{ }ſeh ich ſelten, \textsc{Schwarzkopf}\pwindex{Schwarzkopf, Gustav 07.11.1853 – 13.11.1939@\textsc{Schwarzkopf, Gustav} (07.11.1853 – 13.11.1939), \emph{Schriftsteller/Schriftstellerin}|pw} faſt gar nicht. {\pb}Daſs ich ein Stück\pwindex{Freiwild. Schauspiel in 3 Akten@\emph{Freiwild. Schauspiel in 3 Akten}|pwv}{ }ſchreibe, wiſſen Sie? Vielleicht beend’ ich den
               1. Akt noch in Wien\oindex{Wien@\textbf{Wien}, \emph{A.ADM2}|pw}. – Burckhard\pwindex{Burckhard, Max Eugen 14.07.1854 – 16.03.1912@\textsc{Burckhard, Max Eugen} (14.07.1854 – 16.03.1912), \emph{Schriftsteller/Schriftstellerin, Rechtswissenschaftler/Rechtswissenschaftlerin, Theaterleiter/Theaterleiterin}|pw}{ }ſprach ich neulich; Nachts – im Dunkel unsrer {\pb}gemeinſchaftlichen Stiege. Er iſt ein Wurſtl. – Ich
               war bei \textsc{Sonnenthal}\pwindex{Sonnenthal, Adolf von 1834-12-21 – 1909-04-04@\textsc{Sonnenthal, Adolf von} (1834-12-21 – 1909-04-04), \emph{Schauspieler/Schauspielerin}|pw} – der wird nemlich den Vater\pwindex{Liebelei. Schauspiel in drei Akten@\emph{Liebelei. Schauspiel in drei Akten}|pwv} geben. Und, wie B.\pwindex{Burckhard, Max Eugen 14.07.1854 – 16.03.1912@\textsc{Burckhard, Max Eugen} (14.07.1854 – 16.03.1912), \emph{Schriftsteller/Schriftstellerin, Rechtswissenschaftler/Rechtswissenschaftlerin, Theaterleiter/Theaterleiterin}|pw} verſichert,
                  Mitter{\pb}wurzer\pwindex{Mitterwurzer, Friedrich 16.10.1844 – 13.02.1897@\textsc{Mitterwurzer, Friedrich} (16.10.1844 – 13.02.1897), \emph{Schauspieler/Schauspielerin}|pw} den »Herrn\pwindex{Liebelei. Schauspiel in drei Akten@\emph{Liebelei. Schauspiel in drei Akten}|pwv}«. –\pend
           
\pstart
           Ich habe geradezu eine Sehnſucht, wieder mit Ihnen zu plaudern. »Geradezu« – das ſoll
               der Sentimentalität den Kragen umdrehen.\pend
           
\pstart
           {\pb}Wie geht’s Ihnen? Schreiben Sie bitte. –\pend
           
\pstart
           Den »alten Dichter\pwindex{Spaeter Ruhm@\emph{Später Ruhm}|pw}« werd ich dem \textsc{Bahr}\pwindex{Bahr, Hermann 19.07.1863 – 15.01.1934@\textsc{Bahr, Hermann} (19.07.1863 – 15.01.1934), \emph{Schriftsteller/Schriftstellerin, Kritiker/Kritikerin}|pw} für die Zeit\orgindex{Zeit. Wiener Wochenschrift@Die Zeit. Wiener Wochenschrift|pw} geben, we{\geminationn} er ihn bringen will. Im Prinzip iſt er ein{\pb}verſtanden.\pend
           
\pstart
           Seien Sie herzlich gegrüßt.{\\[\baselineskip]}Ihr \spacefill\mbox{Arthur}\pend
           \leftskip=0em{}\selectlanguage{ngerman}\endnumbering\briefempfaengerindex{Beer-Hofmann, Richard@\textsc{Beer-Hofmann, Richard}!zzzSchnitzler, Arthur@\emph{von Arthur Schnitzler}!1895-06-221@{22. 6. 1895}|)be}\mylabel{L00456h}  \normalsize

\doendnotes{C}
\bigskip
\vfill

\clearpage

\footnotesize

\lohead{\textsc{register}}

% Definiere theindex-Environment komplett neu ohne reledmac
\makeatletter
\renewenvironment{theindex}{%
  \section*{\indexname}%
  \setlength{\parindent}{0pt}%
  \setlength{\parskip}{0pt plus 0.3pt}%
  \let\item\@idxitem
}{%
  \clearpage
}
\makeatother

\IfFileExists{\jobname-pw.ind}{\input{\jobname-pw.ind}}{}

\end{document}

      