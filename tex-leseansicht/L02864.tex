%% latex-leseansicht-vorspann.tex
%% Vorspann für die Leseansicht.
%% Lädt die gemeinsame Datei latex-vorspann.tex mit nicht gesetztem Schalter.

\newif\ifkorrekturansicht
\korrekturansichtfalse

\input{../tex-inputs/latex-vorspann}


         
         \renewcommand{\erwaehntePersonen}{Personen: Marie Reinhard}
         \renewcommand{\erwaehnteOrte}{Orte: Japan, Kyoto, The Oriental Hotel, Wien, Yokohama}
         \renewcommand{\erwaehnteWerke}{}
               \section[ Paul Goldmann an Arthur Schnitzler, 3. 11. {[}1898{]}]{ Paul Goldmann an Arthur Schnitzler, 3. 11. {[}1898{]}}\nopagebreak\mylabel{v}\rehead{ }\begin{ledgroupsized}[t]{13cm}\normalsize\beginnumbering \toendnotes[C]{\smallbreak\pagebreak[2]} \Standort{DLA, A:Schnitzler, HS.NZ85.1.3168.}
\physDesc{Brief, 1 Blatt, 2 Seiten, 1017 Zeichen
\newline{}Handschrift: schwarze Tinte, deutsche Kurrent
\newline{}Schnitzler: mit Bleistift das Jahr »98« vermerkt }\toendnotes[C]{\smallbreak}\pstart
           \noindent{}{\pb}\textcolor{gray}{\textbf{The Oriental Hotel\oindex{The Oriental Hotel@\textbf{The Oriental Hotel}|pw},}}\pend
           \pstart
           \textcolor{gray}{\textbf{YOKOHAMA, JAPAN\oindex{Yokohama@\textbf{Yokohama}|pw}.}}\pend
           \pstart
           \raggedleft{}\textsc{Yokohama\oindex{Yokohama@\textbf{Yokohama}|pw}}, 3. November.\pend
           \pstart\center{}Mein lieber Freund,\pend\pstart
           Ich habe drei Tage in \textsc{Kyoto\oindex{Kyoto@\textbf{Kyoto}|pw}}, der alten japan\oindex{Japan@\textbf{Japan}|pw}iſchen Hauptſtadt\oindex{Kyoto@\textbf{Kyoto}|pwv}, verlebt, die zu den ſchönſten
               meines Lebens gehören. Das einzige Mal, daß ich den Eindruck hatte, ganz aus der
               Wirklichkeit heraus zu ſein! Ich bin gerade ſo kurze Zeit dageweſen, daß der Zauber
               nicht verfliegen konnte. Und ich ſpreche vom Lande allein, \strikeout{von die} nicht von den \label{K_L02864-3v}\edtext{\textsc{Musmes}}{\lemma{\textnormal{\emph{Musmes}}}\Cendnote{\textnormal{junge Mädchen; eventuell von Goldmann\pwindex{Goldmann, Paul 31.01.1865 – 25.09.1935@\textsc{Goldmann, Paul} (31.01.1865 – 25.09.1935), \emph{Schriftsteller, Journalist}|pwk} hier als Synonym für »süßes Mädel«
                  gebraucht?}}}\label{K_L02864-3h} und leichter Liebe, – nein, allein von dem Zauber dieſer
               herrlichen Berge mit ihren Nadelwäldern und herbſtrothen Ahorn-Bäumen, von dem {\pb}Zauber dieſer ſeltſamen, ſeltſamen Stadt\oindex{Kyoto@\textbf{Kyoto}|pwv} mit ihren wundervollen \strikeout{\textcolor{gray}{Straßen} und ihr} Tempeln und den ſtillen Straßen, in denen
               das ſanfte Flötenſpiel der Prieſter klingt, welche Almoſen einſammeln. Keine Feder
               vermag das zu beſchreiben. Jetzt fällt der Regen, und ich ſitze in dem reizloſen
               kosmopolitiſchen \textsc{Yokohama\oindex{Yokohama@\textbf{Yokohama}|pw}} und ſehne mich nach \textsc{Kyoto\oindex{Kyoto@\textbf{Kyoto}|pw}}, wie ich mich mein ganzes Leben danach ſehnen werde.\pend
           \pstart
           Von Dir habe ich lange nichts gehört. Wie mag es Dir nur gehen?\pend
           \pstart
           Viele treue Grüße! {\\[\baselineskip]}Dein {\\[\baselineskip]}\spacefill\mbox{Paul Goldmann}\pend
           \leftskip=0em{}\pstart
           \noindent{}Grüße an Deine Freundin\pwindex{Reinhard, Marie 1871-03-13 – 1899-03-18@\textsc{Reinhard, Marie} (1871-03-13 – 1899-03-18), \emph{Gesangspädagogin}|pwv}!\pend
           
         
         \endnumbering\mylabel{h}\end{ledgroupsized}  \newcommand{\dateiname}{L02864}\newcommand{\titel}{Paul Goldmann an Arthur Schnitzler, 3. 11. [1898]}\newcommand{\editorInnen}{Martin Anton Müller und Laura Untner}%% latex-leseansicht-abspann.tex
%% Abspann für die Leseansicht.
%% Der Schalter \ifkorrekturansicht ist bereits durch den Vorspann gesetzt.

%% latex-abspann.tex
%% Gemeinsamer Abspann für Korrekturansicht und Leseansicht.
%% Setzt den Schalter \ifkorrekturansicht voraus (gesetzt in den
%% einbindenden Dateien latex-korrekturansicht-abspann.tex bzw.
%% latex-leseansicht-abspann.tex).
%% ---------------------------------------------------------------

\normalsize

% Das esempio-Environment wird nur in der Leseansicht benötigt
\ifkorrekturansicht\else
\newenvironment{esempio}[3]%
{
    \vspace{1.5ex}
    \rlap{\underline{#1}}
    \par
    \setlength{\parindent}{0cm}
    \nopagebreak
    \leftskip=#2cm
    \rightskip=#3cm
}
{
    \par
}
\fi

\doendnotes{C}
\bigskip
\vfill

\clearpage

\footnotesize

\ifkorrekturansicht
  \lohead{\textsc{register}}
\fi

% theindex-Environment neu definieren ohne reledmac
\makeatletter
\renewenvironment{theindex}{%
  \ifkorrekturansicht
    \section*{\indexname}%
  \else
    \subsubsection*{Index der erwähnten Entitäten}%
  \fi
  \setlength{\parindent}{0pt}%
  \setlength{\parskip}{0pt plus 0.3pt}%
  \let\item\@idxitem
}{%
  \ifkorrekturansicht\clearpage\fi
}
\makeatother

\IfFileExists{\jobname-pw.ind}{\input{\jobname-pw.ind}}{}

% Quellenangabe nur in der Leseansicht
\ifkorrekturansicht\else
% Fallback-Definitionen, falls die .tex-Datei \titel etc. nicht gesetzt hat
\providecommand{\titel}{}
\providecommand{\editorInnen}{}
\providecommand{\dateiname}{\jobname}

\vspace{3cm}

\vfill

\footnotesize
\textsc{Quelle}: \titel. Herausgegeben von {\editorInnen}. In: \emph{Arthur Schnitzler: Briefwechsel mit Autorinnen und Autoren}.
 Digitale Edition, https://schnitzler-briefe.acdh.oeaw.ac.at/{\dateiname}.html (Stand \today)
\fi

\end{document}


      