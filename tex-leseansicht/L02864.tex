%% latex-korrekturansicht-vorspann.tex
%% Vorspann für die Korrekturansicht.
%% Lädt die gemeinsame Datei latex-vorspann.tex mit gesetztem Schalter.

\newif\ifkorrekturansicht
\korrekturansichttrue

\input{../tex-inputs/latex-vorspann}


\section[ Paul Goldmann an Arthur Schnitzler, 3. 11. {[}1898{]}]{L02864 Paul Goldmann an Arthur Schnitzler, 3. 11. {[}1898{]}}
\nopagebreak\mylabel{L02864v}
\rehead{ }\normalsize\beginnumbering\briefempfaengerindex{Schnitzler, Arthur@\textsc{Schnitzler, Arthur}!zzzGoldmann, Paul@\emph{von Paul Goldmann}!1898-11-031@{3. 11. {[}1898{]}}|(be}
\toendnotes[C]{\smallbreak\pagebreak[2]}\Standort{DLA, A:Schnitzler, HS.NZ85.1.3168.}
\physDesc{Brief, 1 Blatt, 2 Seiten, 1017 Zeichen
\newline{}Handschrift: schwarze Tinte, deutsche Kurrent
\newline{}Schnitzler: mit Bleistift das Jahr »98« vermerkt }\toendnotes[C]{\smallbreak}
\pstart
           {\pb}\textcolor{gray}{\textbf{The Oriental Hotel\oindex{The Oriental Hotel@\textbf{The Oriental Hotel}, \emph{Hotel (K.HTL)}|pw},}}\pend
           
\pstart
           \textcolor{gray}{\textbf{YOKOHAMA, JAPAN\oindex{Yokohama@\textbf{Yokohama}, \emph{P.PPLA}|pw}.}}\pend
           
\pstart
           \raggedleft{}\textsc{Yokohama\oindex{Yokohama@\textbf{Yokohama}, \emph{P.PPLA}|pw}}, 3. November.\pend
           
\pstart\center{}Mein lieber Freund,\pend\vspace{0.5em}
\pstart
           Ich habe drei Tage in \textsc{Kyoto\oindex{Kyoto@\textbf{Kyoto}, \emph{P.PPLA}|pw}}, der alten japan\oindex{Japan@\textbf{Japan}, \emph{A.PCLI}|pw}iſchen Hauptſtadt\oindex{Kyoto@\textbf{Kyoto}, \emph{P.PPLA}|pwv}, verlebt, die zu den ſchönſten
               meines Lebens gehören. Das einzige Mal, daß ich den Eindruck hatte, ganz aus der
               Wirklichkeit heraus zu ſein! Ich bin gerade ſo kurze Zeit dageweſen, daß der Zauber
               nicht verfliegen konnte. Und ich ſpreche vom Lande allein, \strikeout{von die} nicht von den \label{K_L02864-1v}\edtext{\textsc{Musmes}}{\lemma{\textnormal{\emph{Musmes}}}\Cendnote{\textnormal{junge Mädchen; eventuell wird es hier von Goldmann\pwindex{Goldmann, Paul 31.01.1865 – 25.09.1935@\textsc{Goldmann, Paul} (31.01.1865 – 25.09.1935), \emph{Schriftsteller/Schriftstellerin, Journalist/Journalistin}|pwk} als Synonym für »süßes Mädel«
                  gebraucht.}}}\label{K_L02864-1} und leichter Liebe, – nein, allein von dem Zauber dieſer
               herrlichen Berge mit ihren Nadelwäldern und herbſtrothen Ahorn-Bäumen, von dem {\pb}Zauber dieſer ſeltſamen, ſeltſamen Stadt\oindex{Kyoto@\textbf{Kyoto}, \emph{P.PPLA}|pwv} mit ihren wundervollen \strikeout{\textcolor{gray}{Straßen} und ihr} Tempeln und den ſtillen Straßen, in denen
               das ſanfte Flötenſpiel der Prieſter klingt, welche Almoſen einſammeln. Keine Feder
               vermag das zu beſchreiben. Jetzt fällt der Regen, und ich ſitze in dem reizloſen
               kosmopolitiſchen \textsc{Yokohama\oindex{Yokohama@\textbf{Yokohama}, \emph{P.PPLA}|pw}} und ſehne mich nach \textsc{Kyoto\oindex{Kyoto@\textbf{Kyoto}, \emph{P.PPLA}|pw}}, wie ich mich mein ganzes Leben danach ſehnen werde.\pend
           
\pstart
           Von Dir habe ich lange nichts gehört. Wie mag es Dir nur gehen?\pend
           
\pstart
           Viele treue Grüße! {\\[\baselineskip]}Dein {\\[\baselineskip]}\spacefill\mbox{Paul Goldmann}\pend
           \leftskip=0em{}
\pstart
           \noindent{}Grüße an Deine Freundin\pwindex{Reinhard, Marie 1871-03-13 – 1899-03-18@\textsc{Reinhard, Marie} (1871-03-13 – 1899-03-18), \emph{Gesangspädagoge/Gesangspädagogin}|pwv}!\pend
           \selectlanguage{ngerman}\endnumbering\briefempfaengerindex{Schnitzler, Arthur@\textsc{Schnitzler, Arthur}!zzzGoldmann, Paul@\emph{von Paul Goldmann}!1898-11-031@{3. 11. {[}1898{]}}|)be}\mylabel{L02864h}  \normalsize

\doendnotes{C}
\bigskip
\vfill

\clearpage

\footnotesize

\lohead{\textsc{register}}

% Definiere theindex-Environment komplett neu ohne reledmac
\makeatletter
\renewenvironment{theindex}{%
  \section*{\indexname}%
  \setlength{\parindent}{0pt}%
  \setlength{\parskip}{0pt plus 0.3pt}%
  \let\item\@idxitem
}{%
  \clearpage
}
\makeatother

\IfFileExists{\jobname-pw.ind}{\input{\jobname-pw.ind}}{}

\end{document}

      