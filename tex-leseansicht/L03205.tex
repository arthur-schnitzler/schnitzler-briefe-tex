%% latex-leseansicht-vorspann.tex
%% Vorspann für die Leseansicht.
%% Lädt die gemeinsame Datei latex-vorspann.tex mit nicht gesetztem Schalter.

\newif\ifkorrekturansicht
\korrekturansichtfalse

\input{../tex-inputs/latex-vorspann}

\begin{center}
            \textcolor{red}{ENTWURF, NICHT FERTIG KORRIGIERT}
                      \end{center}
            
         \renewcommand{\erwaehnteInstitutionen}{Institutionen: Berliner Tageblatt, Tägliche Rundschau}
         \renewcommand{\erwaehnteOrte}{Orte: Berlin, Deutschland, Wien, Österreich}
         \renewcommand{\erwaehnteWerke}{Werke: Tägliche Rundschau}
               \section[ Paul Goldmann an Arthur Schnitzler, 29. 4. {[}1902{]}]{ Paul Goldmann an Arthur Schnitzler, 29. 4. {[}1902{]}}\nopagebreak\mylabel{v}\rehead{ }\begin{ledgroupsized}[t]{13cm}\normalsize\beginnumbering \toendnotes[C]{\smallbreak\pagebreak[2]} \Standort{DLA, A:Schnitzler, HS.NZ85.1.3172.}
\physDesc{Brief, 1 Blatt, 4 Seiten
\newline{}Handschrift: blaue Tinte, deutsche Kurrent
\newline{}Schnitzler: 1) mit Bleistift das Jahr »{[}1{]}902«
                                            vermerkt  2) mit rotem Buntstift eine Unterstreichung}\toendnotes[C]{\smallbreak}\pstart
           \centering{}{\pb}Berlin\oindex{Berlin@\textbf{Berlin}|pw}, 29. April.\pend
           \pstart{}Mein lieber Freund,\pend\pstart
           Die »Tägliche Rundſchau\orgindex{Taegliche Rundschau@Tägliche Rundschau|pw}« hat auch heut{ }Morgen noch nicht für nöthig befunden, nachdem ſie in überaus
                    taktloſer Weiſe Deinen \label{K_L03205-1v}\edtext{Namen genannt\textcolor{red}{\textsuperscript{\textbf{KEY}}}}{\lemma{\textnormal{\emph{Namen genannt}}}\Cendnote{\textnormal{XXXX}}}\label{K_L03205-1h} und ſogar von einem »Fall \textsc{Schnitzler}\textcolor{red}{\textsuperscript{\textbf{KEY}}}« geſprochen hat, von Deinem \label{K_L03205-2v}\edtext{\textsc{Dementi\textcolor{red}{\textsuperscript{\textbf{KEY}}}}}{\lemma{\textnormal{\emph{Dementi}}}\Cendnote{\textnormal{XXXX}}}\label{K_L03205-2h} Notiz zu nehmen.
                    Die »Tgl. Rundſchau\pwindex{?? Werk@Nicht ermittelte Verfasserinnen und Verfasser!Taegliche Rundschau1881 – 1933@\emph{Tägliche Rundschau} {[}1881 – 1933{]}|pw}« iſt ein alldeutſches
                    und antiſemitiſches Blatt\strikeout{, gilt} und gilt für
                    ſehr »literariſch«, ebenſo wie der Herr \textsc{Karl Strecker\textcolor{red}{\textsuperscript{\textbf{KEY}}}} (der ein \strikeout{germa} germaniſtiſcher Schwätzer
                    iſt) für einen »vornehmen Kritiker« gilt. Es iſt möglich, daß das {\pb}Schweigen der Tgl. Rdſch.\orgindex{Taegliche Rundschau@Tägliche Rundschau|pw} nur Schlamperei iſt, daß der Herr \textsc{Strecker\textcolor{red}{\textsuperscript{\textbf{KEY}}}} vielleicht die Angelegenheit in ſeinem nächſten \label{K_L03205-3v}\edtext{Referat\textcolor{red}{\textsuperscript{\textbf{KEY}}}}{\lemma{\textnormal{\emph{Referat}}}\Cendnote{\textnormal{XXXX (auch zu
                        Goldis Forderung, Schnitzler solle klagen, sollte keine Berichtigung kommen,
                        auch zu Anwalt und Staatsanwaltschaft)}}}\label{K_L03205-3h} berühren will. Aber ſchon
                    dieſes Warten, nachdem er das Maul ſo voll genommen und eine »ofſene Frage\textcolor{red}{\textsuperscript{\textbf{KEY}}}« an Dich gerichtet hat, iſt
                    unanſtändig. Ich bitte Dich daher, ihm in gemeſſenem Ton einen \label{K_L03205-4v}\edtext{Brief}{\lemma{\textnormal{\emph{Brief}}}\Cendnote{\textnormal{XXXX}}}\label{K_L03205-4h} zu ſchreiben, Dein Erſtaunen über ſein
                    ganzes Vorgehen, Dein noch größeres Erſtaunen über \strikeout{das} die Nichtveröffentlichung Deiner Antwort\textcolor{red}{\textsuperscript{\textbf{KEY}}} auszudrücken, ihn um ſofortige Publikation\textcolor{red}{\textsuperscript{\textbf{KEY}}} Deiner Antwort zu
                    erſuchen und die Hoffnung auszuſprechen, daß {\pb}er\textcolor{red}{\textsuperscript{\textbf{KEY}}} Dich nicht dazu
                    nöthigen wird, die Veröffentlichung dieſer Antwort\textcolor{red}{\textsuperscript{\textbf{KEY}}}, die eine ſchlicht literariſchen Anſtandes iſt, auf
                    andere Weiſe zu erzwingen. Wenn das nicht hilft, wirſt Du das Blatt\orgindex{Taegliche Rundschau@Tägliche Rundschau|pwv} ſelbſtverſtändlich klagen. Hier\oindex{Deutschland@\textbf{Deutschland}|pwv} liegen die
                    Verhältniſſe anders als in Öſterreich\oindex{Oesterreich@\textbf{Österreich}|pw}, und
                    jedes Gericht wird Dir Recht geben. Ich übernehme die Angelegenheit und beſorge
                    Dir einen guten Advokat\textcolor{red}{\textsuperscript{\textbf{KEY}}}en.
                    Ebenſo würde ich rathen, daß Du bei der Wien\oindex{Wien@\textbf{Wien}|pw}er Staatsanwaltſchaft\textcolor{red}{\textsuperscript{\textbf{KEY}}} Anzeige erſtatteſt.
                    Dieſem ſauberen Herrn von \label{K_L03205-5v}\edtext{\textsc{Jarro\textcolor{red}{\textsuperscript{\textbf{KEY}}}}}{\lemma{\textnormal{\emph{Jarro}}}\Cendnote{\textnormal{XXXX}}}\label{K_L03205-5h} muß doch das
                    Handwerk gelegt werden. Auch an die {\pb}Direktion
                    des \label{K_L03205-6v}\edtext{\textsc{Carl Weiss} Theater\textcolor{red}{\textsuperscript{\textbf{KEY}}}}{\lemma{\textnormal{\emph{Carl Weiss Theater}}}\Cendnote{\textnormal{XXXX}}}\label{K_L03205-6h}s ſollteſt Du ſchreiben und Dir die Nennung
                    des wirklichen Namens des Herrn \label{K_L03205-7v}\edtext{\textsc{von Jurca\textcolor{red}{\textsuperscript{\textbf{KEY}}}}}{\lemma{\textnormal{\emph{von Jurca}}}\Cendnote{\textnormal{XXXX (falls nicht ohnehin
                        durch bibl schon geklärt)}}}\label{K_L03205-7h} erbitten. Die Direktion\textcolor{red}{\textsuperscript{\textbf{KEY}}} hat dem \strikeout{H\textcolor{gray}{e}}{ }Berliner
                        Tageblatt\orgindex{Berliner Tageblatt@Berliner Tageblatt|pw}{ }\strikeout{\textcolor{gray}{×}} auf eine
                    telephoniſche Anfrage geantwortet, daß \strikeout{ſih} ſich
                        \strikeout{\textcolor{gray}{u}n} unter dieſem Pſeudonym ein Autor\textcolor{red}{\textsuperscript{\textbf{KEY}}} aus »guter Wien\oindex{Wien@\textbf{Wien}|pw}er Familie« verberge, deſſen Namen allerdings die Direktion\textcolor{red}{\textsuperscript{\textbf{KEY}}} nicht nennen könne.\pend
           \pstart
           Hebe Dir (für den Fall, daß es zum Prozeß kommt) alle Berlin\oindex{Berlin@\textbf{Berlin}|pw}er Zeitungen auf, die ich Dir ſchicke,
                        ſ\textcolor{gray}{ende} eine \label{K_L03205-8v}\edtext{Copie Deines Briefes an \textsc{Strecker\textcolor{red}{\textsuperscript{\textbf{KEY}}}}}{\lemma{\textnormal{\emph{Copie … Strecker}}}\Cendnote{\textnormal{XXXX}}}\label{K_L03205-8h}.\pend
           \pstart
           Viele treue Grüße! {\\[\baselineskip]}Dein \spacefill\mbox{Paul Goldmann}\pend
           \leftskip=0em{}
         
         \endnumbering\mylabel{h}\end{ledgroupsized}\begin{anhang}\end{anhang}\newcommand{\dateiname}{L03205}\newcommand{\titel}{Paul Goldmann an Arthur Schnitzler, 29. 4. [1902]}\newcommand{\editorInnen}{Martin Anton Müller und Laura Untner}%% latex-leseansicht-abspann.tex
%% Abspann für die Leseansicht.
%% Der Schalter \ifkorrekturansicht ist bereits durch den Vorspann gesetzt.

%% latex-abspann.tex
%% Gemeinsamer Abspann für Korrekturansicht und Leseansicht.
%% Setzt den Schalter \ifkorrekturansicht voraus (gesetzt in den
%% einbindenden Dateien latex-korrekturansicht-abspann.tex bzw.
%% latex-leseansicht-abspann.tex).
%% ---------------------------------------------------------------

\normalsize

% Das esempio-Environment wird nur in der Leseansicht benötigt
\ifkorrekturansicht\else
\newenvironment{esempio}[3]%
{
    \vspace{1.5ex}
    \rlap{\underline{#1}}
    \par
    \setlength{\parindent}{0cm}
    \nopagebreak
    \leftskip=#2cm
    \rightskip=#3cm
}
{
    \par
}
\fi

\doendnotes{C}
\bigskip
\vfill

\clearpage

\footnotesize

\ifkorrekturansicht
  \lohead{\textsc{register}}
\fi

% theindex-Environment neu definieren ohne reledmac
\makeatletter
\renewenvironment{theindex}{%
  \ifkorrekturansicht
    \section*{\indexname}%
  \else
    \subsubsection*{Index der erwähnten Entitäten}%
  \fi
  \setlength{\parindent}{0pt}%
  \setlength{\parskip}{0pt plus 0.3pt}%
  \let\item\@idxitem
}{%
  \ifkorrekturansicht\clearpage\fi
}
\makeatother

\IfFileExists{\jobname-pw.ind}{\input{\jobname-pw.ind}}{}

% Quellenangabe nur in der Leseansicht
\ifkorrekturansicht\else
% Fallback-Definitionen, falls die .tex-Datei \titel etc. nicht gesetzt hat
\providecommand{\titel}{}
\providecommand{\editorInnen}{}
\providecommand{\dateiname}{\jobname}

\vspace{3cm}

\vfill

\footnotesize
\textsc{Quelle}: \titel. Herausgegeben von {\editorInnen}. In: \emph{Arthur Schnitzler: Briefwechsel mit Autorinnen und Autoren}.
 Digitale Edition, https://schnitzler-briefe.acdh.oeaw.ac.at/{\dateiname}.html (Stand \today)
\fi

\end{document}


      