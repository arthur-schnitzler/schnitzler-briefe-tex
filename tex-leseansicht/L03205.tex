%% latex-leseansicht-vorspann.tex
%% Vorspann für die Leseansicht.
%% Lädt die gemeinsame Datei latex-vorspann.tex mit nicht gesetztem Schalter.

\newif\ifkorrekturansicht
\korrekturansichtfalse

\input{../tex-inputs/latex-vorspann}

\begin{center}
            \textcolor{red}{ENTWURF, NICHT FERTIG KORRIGIERT}
                      \end{center}
            
         
         \renewcommand{\erwaehntePersonen}{Personen: Paul Goldmann, Ernest von Gréger-Jurco, Alfred Spitzer, Karl Strecker}
         \renewcommand{\erwaehnteInstitutionen}{Institutionen: Berliner Tageblatt, Rose-Theater, Staatsanwaltschaft, Tägliche Rundschau}
         \renewcommand{\erwaehnteOrte}{Orte: Berlin, Deutschland, Wien, Österreich}
         \renewcommand{\erwaehnteWerke}{Werke: Das angebliche Telegramm Arthur Schnitzlers, Die Kinder der Armen, Ein litterarisch-dramatisches Hochstapler-Stücklein, Tägliche Rundschau}
               \section[ Paul Goldmann an Arthur Schnitzler, 29. 4. {[}1902{]}]{ Paul Goldmann an Arthur Schnitzler, 29. 4. {[}1902{]}}\nopagebreak\mylabel{v}\rehead{ }\begin{ledgroupsized}[t]{13cm}\normalsize\beginnumbering\briefempfaengerindex{Schnitzler, Arthur@\textsc{Schnitzler, Arthur}!zzzGoldmann, Paul@\emph{von Paul Goldmann}!1902-04-291@{29. 4. {[}1902{]}}|(be} \toendnotes[C]{\smallbreak\pagebreak[2]} \Standort{DLA, A:Schnitzler, HS.NZ85.1.3172.}
\physDesc{Brief, 1 Blatt, 4 Seiten, 2108 Zeichen
\newline{}Handschrift: blaue Tinte, deutsche Kurrent
\newline{}Schnitzler: 1) mit Bleistift das Jahr »902« vermerkt  2) mit rotem Buntstift eine Unterstreichung}\toendnotes[C]{\smallbreak}\pstart
           \centering{}{\pb}Berlin\oindex{Berlin@\textbf{Berlin}|pw}, 29. April.\pend
           \pstart{}Mein lieber Freund,\pend\pstart
           Die »Tägliche Rundſchau\orgindex{Taegliche Rundschau@Tägliche Rundschau|pw}« hat auch heut{ }Morgen noch nicht für nöthig befunden, nachdem ſie in überaus taktloſer
               Weiſe Deinen \label{K_L03205-1v}\edtext{Namen genannt\pwindex{Strecker, Karl 1862-04-08 – 1933-02-19@\textsc{Strecker, Karl} (1862-04-08 – 1933-02-19), \emph{Theaterkritiker}!litterarisch-dramatisches Hochstapler-Stuecklein1902-04-26@\strich\emph{Ein litterarisch-dramatisches Hochstapler-Stücklein} {[}1902-04-26{]}|pwv}}{\lemma{\textnormal{\emph{Namen genannt}}}\Cendnote{\textnormal{siehe Paul Goldmann an Arthur Schnitzler, 26. 4. 1902}}}\label{K_L03205-1h} und ſogar von einem »Fall
                     \textsc{Schnitzler}\pwindex{Strecker, Karl 1862-04-08 – 1933-02-19@\textsc{Strecker, Karl} (1862-04-08 – 1933-02-19), \emph{Theaterkritiker}!litterarisch-dramatisches Hochstapler-Stuecklein1902-04-26@\strich\emph{Ein litterarisch-dramatisches Hochstapler-Stücklein} {[}1902-04-26{]}|pwv}« geſprochen hat, von Deinem \label{K_L03205-2v}\edtext{\textsc{Dementi\pwindex{Strecker, Karl 1862-04-08 – 1933-02-19@\textsc{Strecker, Karl} (1862-04-08 – 1933-02-19), \emph{Theaterkritiker}!angebliche Telegramm Arthur Schnitzlers1902-06-24@\strich\emph{Das angebliche Telegramm Arthur Schnitzlers} {[}1902-06-24{]}|pwv}}}{\lemma{\textnormal{\emph{Dementi}}}\Cendnote{\textnormal{Goldmann\pwindex{Goldmann, Paul 31.01.1865 – 25.09.1935@\textsc{Goldmann, Paul} (31.01.1865 – 25.09.1935), \emph{Schriftsteller, Journalist}|pwk} dürfte diese Notiz\pwindex{Strecker, Karl 1862-04-08 – 1933-02-19@\textsc{Strecker, Karl} (1862-04-08 – 1933-02-19), \emph{Theaterkritiker}!angebliche Telegramm Arthur Schnitzlers1902-06-24@\strich\emph{Das angebliche Telegramm Arthur Schnitzlers} {[}1902-06-24{]}|pwkv} übersehen haben: Karl Strecker\pwindex{Strecker, Karl 1862-04-08 – 1933-02-19@\textsc{Strecker, Karl} (1862-04-08 – 1933-02-19), \emph{Theaterkritiker}|pwk}: \emph{Das angebliche Telegramm Arthur Schnitzlers}\pwindex{Strecker, Karl 1862-04-08 – 1933-02-19@\textsc{Strecker, Karl} (1862-04-08 – 1933-02-19), \emph{Theaterkritiker}!angebliche Telegramm Arthur Schnitzlers1902-06-24@\strich\emph{Das angebliche Telegramm Arthur Schnitzlers} {[}1902-06-24{]}|pwk}. In: \emph{Tägliche Rundschau}\pwindex{?? Werk@Nicht ermittelte Verfasserinnen und Verfasser!Taegliche Rundschau1881 – 1933@\emph{Tägliche Rundschau} {[}1881 – 1933{]}|pwk}, Jg. 22, Nr. 194, 26. 4. 1902, Abend-Blatt, Unterhaltungsbeilage,
                     Nr. 97, S. 388. Siehe A. S.: \emph{»Das Zeitlose ist von kürzester Dauer«}, Karl Strecker: Das angebliche Telegramm Arthur Schnitzlers, 26. 4. 1902}}}\label{K_L03205-2h} Notiz zu nehmen. Die »Tgl. Rundſchau\pwindex{?? Werk@Nicht ermittelte Verfasserinnen und Verfasser!Taegliche Rundschau1881 – 1933@\emph{Tägliche Rundschau} {[}1881 – 1933{]}|pw}«
               iſt ein alldeutſches und antiſemitiſches Blatt\strikeout{, gilt}
               und gilt für ſehr »literariſch«, ebenſo wie der Herr \textsc{Karl Strecker\pwindex{Strecker, Karl 1862-04-08 – 1933-02-19@\textsc{Strecker, Karl} (1862-04-08 – 1933-02-19), \emph{Theaterkritiker}|pw}} (der ein \strikeout{germa} germaniſtiſcher Schwätzer iſt)
               für einen »vornehmen Kritiker« gilt. Es iſt möglich, daß das {\pb}Schweigen der Tgl.
                  Rdſch.\orgindex{Taegliche Rundschau@Tägliche Rundschau|pw} nur Schlamperei iſt, daß der Herr \textsc{Strecker\pwindex{Strecker, Karl 1862-04-08 – 1933-02-19@\textsc{Strecker, Karl} (1862-04-08 – 1933-02-19), \emph{Theaterkritiker}|pw}} vielleicht die Angelegenheit in ſeinem nächſten Referat berühren will. Aber
               ſchon dieſes Warten, nachdem er das Maul ſo voll genommen und eine »offene Frage\pwindex{Strecker, Karl 1862-04-08 – 1933-02-19@\textsc{Strecker, Karl} (1862-04-08 – 1933-02-19), \emph{Theaterkritiker}!litterarisch-dramatisches Hochstapler-Stuecklein1902-04-26@\strich\emph{Ein litterarisch-dramatisches Hochstapler-Stücklein} {[}1902-04-26{]}|pwv}« an Dich gerichtet hat, iſt
               unanſtändig. Ich bitte Dich daher, ihm in gemeſſenem Ton einen Brief zu ſchreiben,
               Dein Erſtaunen über ſein ganzes Vorgehen, Dein noch größeres Erſtaunen über \strikeout{das} die Nichtveröffentlichung Deiner Antwort\pwindex{Strecker, Karl 1862-04-08 – 1933-02-19@\textsc{Strecker, Karl} (1862-04-08 – 1933-02-19), \emph{Theaterkritiker}!angebliche Telegramm Arthur Schnitzlers1902-06-24@\strich\emph{Das angebliche Telegramm Arthur Schnitzlers} {[}1902-06-24{]}|pwv} auszudrücken, ihn um ſofortige
               Publikation Deiner Antwort\pwindex{Strecker, Karl 1862-04-08 – 1933-02-19@\textsc{Strecker, Karl} (1862-04-08 – 1933-02-19), \emph{Theaterkritiker}!angebliche Telegramm Arthur Schnitzlers1902-06-24@\strich\emph{Das angebliche Telegramm Arthur Schnitzlers} {[}1902-06-24{]}|pwv} zu
               erſuchen und die Hoffnung auszuſprechen, daß {\pb}er\pwindex{Strecker, Karl 1862-04-08 – 1933-02-19@\textsc{Strecker, Karl} (1862-04-08 – 1933-02-19), \emph{Theaterkritiker}|pwv} Dich nicht dazu nöthigen
               wird, die Veröffentlichung dieſer Antwort\pwindex{Strecker, Karl 1862-04-08 – 1933-02-19@\textsc{Strecker, Karl} (1862-04-08 – 1933-02-19), \emph{Theaterkritiker}!angebliche Telegramm Arthur Schnitzlers1902-06-24@\strich\emph{Das angebliche Telegramm Arthur Schnitzlers} {[}1902-06-24{]}|pwv}, die eine ſchlicht literariſchen Anſtandes iſt, auf andere Weiſe zu
               erzwingen. Wenn das nicht \substVorne{}\textsuperscript{h\textcolor{gray}{elf}}\substDazwischen{}hilft\substHinten{}, wirſt Du das Blatt\orgindex{Taegliche Rundschau@Tägliche Rundschau|pwv}
               ſelbſtverſtändlich klagen. Hier\oindex{Deutschland@\textbf{Deutschland}|pwv}
               liegen die Verhältniſſe anders als in Öſterreich\oindex{Oesterreich@\textbf{Österreich}|pw}, und jedes Gericht wird Dir Recht geben. Ich übernehme die
               Angelegenheit und beſorge Dir einen guten \label{K_L03205-3v}\edtext{Advokaten}{\lemma{\textnormal{\emph{Advokaten}}}\Cendnote{\textnormal{Schnitzler\pwindex{Schnitzler, Arthur 15.05.1862 – 21.10.1931@\textsc{Schnitzler, Arthur} (15.05.1862 – 21.10.1931), \emph{Schriftsteller, Mediziner}|pwk} sprach am 5. 5. 1902 jedenfalls
                  mit dem Rechtsanwalt Alfred Spitzer\pwindex{Spitzer, Alfred 03.12.1861 – 26.04.1923@\textsc{Spitzer, Alfred} (03.12.1861 – 26.04.1923), \emph{Rechtsanwalt}|pwk} über
                  die Angelegenheit.}}}\label{K_L03205-3h}. Ebenſo würde ich rathen, daß Du bei der Wien\oindex{Wien@\textbf{Wien}|pw}er Staatsanwaltſchaft\orgindex{Staatsanwaltschaft@Staatsanwaltschaft|pw} Anzeige erſtatteſt. Dieſem ſauberen Herrn von \label{K_L03205-4v}\edtext{\textsc{Jurco\pwindex{Greger-Jurco, Ernest von *~11.08.1860@\textsc{Gréger-Jurco, Ernest von} (*~11.08.1860), \emph{Schriftsteller}|pw}}}{\lemma{\textnormal{\emph{Jurco}}}\Cendnote{\textnormal{Ernest von Jurco-Gréger\pwindex{Greger-Jurco, Ernest von *~11.08.1860@\textsc{Gréger-Jurco, Ernest von} (*~11.08.1860), \emph{Schriftsteller}|pwk}, dessen Stück \emph{Die Kinder der Armen}\pwindex{Greger-Jurco, Ernest von *~11.08.1860@\textsc{Gréger-Jurco, Ernest von} (*~11.08.1860), \emph{Schriftsteller}!Kinder der Armen1902-04-25@\strich\emph{Die Kinder der Armen} {[}1902-04-25{]}|pwk} in dem gefälschten Telegramm\pwindex{Strecker, Karl 1862-04-08 – 1933-02-19@\textsc{Strecker, Karl} (1862-04-08 – 1933-02-19), \emph{Theaterkritiker}!litterarisch-dramatisches Hochstapler-Stuecklein1902-04-26@\strich\emph{Ein litterarisch-dramatisches Hochstapler-Stücklein} {[}1902-04-26{]}|pwkv}{ }Schnitzler\pwindex{Schnitzler, Arthur 15.05.1862 – 21.10.1931@\textsc{Schnitzler, Arthur} (15.05.1862 – 21.10.1931), \emph{Schriftsteller, Mediziner}|pwk}s empfohlen worden war}}}\label{K_L03205-4h} muß
               doch das Handwerk gelegt werden. Auch an die {\pb}Direktion des \label{K_L03205-5v}\edtext{\textsc{Carl Weiss} Theater\orgindex{Rose-Theater@Rose-Theater|pw}}{\lemma{\textnormal{\emph{Carl Weiss Theater}}}\Cendnote{\textnormal{An diesem Berlin\oindex{Berlin@\textbf{Berlin}|pwk}er Theater fand am 25. 4. 1902 die Uraufführung von
                     \emph{Die Kinder der Armen}\pwindex{Greger-Jurco, Ernest von *~11.08.1860@\textsc{Gréger-Jurco, Ernest von} (*~11.08.1860), \emph{Schriftsteller}!Kinder der Armen1902-04-25@\strich\emph{Die Kinder der Armen} {[}1902-04-25{]}|pwk} statt.}}}\label{K_L03205-5h}s
               ſollteſt Du ſchreiben und Dir die Nennung des wirklichen Namens des Herrn \textsc{von Jurco\pwindex{Greger-Jurco, Ernest von *~11.08.1860@\textsc{Gréger-Jurco, Ernest von} (*~11.08.1860), \emph{Schriftsteller}|pw}} erbitten. Die Direktion\orgindex{Rose-Theater@Rose-Theater|pwv}
               hat dem \strikeout{H\textcolor{gray}{e}}{ }Berliner Tageblatt\orgindex{Berliner Tageblatt@Berliner Tageblatt|pw}{ }\strikeout{\textcolor{gray}{×}} auf eine telephoniſche Anfrage geantwortet, daß \strikeout{ſih} ſich \strikeout{\textcolor{gray}{u}n} unter dieſem Pſeudonym ein Autor\pwindex{Greger-Jurco, Ernest von *~11.08.1860@\textsc{Gréger-Jurco, Ernest von} (*~11.08.1860), \emph{Schriftsteller}|pwv} aus »guter Wien\oindex{Wien@\textbf{Wien}|pw}er Familie« verberge, deſſen Namen allerdings die Direktion\orgindex{Rose-Theater@Rose-Theater|pwv} nicht nennen
               könne.\pend
           \pstart
           Hebe Dir (für den Fall, daß es zum Prozeß kommt) alle Berlin\oindex{Berlin@\textbf{Berlin}|pw}er Zeitungen auf, die ich Dir ſchicke, ſowie eine \label{K_L03205-6v}\edtext{Copie Deines Briefes an \textsc{Strecker\pwindex{Strecker, Karl 1862-04-08 – 1933-02-19@\textsc{Strecker, Karl} (1862-04-08 – 1933-02-19), \emph{Theaterkritiker}|pw}}}{\lemma{\textnormal{\emph{Copie … Strecker}}}\Cendnote{\textnormal{nicht nachweisbar}}}\label{K_L03205-6h}.\pend
           \pstart
           Viele treue Grüße! {\\[\baselineskip]}Dein \spacefill\mbox{Paul Goldmann}\pend
           \leftskip=0em{}
         
         \endnumbering\mylabel{h}\end{ledgroupsized}\begin{anhang}\end{anhang}\newcommand{\dateiname}{L03205}\newcommand{\titel}{Paul Goldmann an Arthur Schnitzler, 29. 4. [1902]}\newcommand{\editorInnen}{Martin Anton Müller und Laura Untner}%% latex-leseansicht-abspann.tex
%% Abspann für die Leseansicht.
%% Der Schalter \ifkorrekturansicht ist bereits durch den Vorspann gesetzt.

%% latex-abspann.tex
%% Gemeinsamer Abspann für Korrekturansicht und Leseansicht.
%% Setzt den Schalter \ifkorrekturansicht voraus (gesetzt in den
%% einbindenden Dateien latex-korrekturansicht-abspann.tex bzw.
%% latex-leseansicht-abspann.tex).
%% ---------------------------------------------------------------

\normalsize

% Das esempio-Environment wird nur in der Leseansicht benötigt
\ifkorrekturansicht\else
\newenvironment{esempio}[3]%
{
    \vspace{1.5ex}
    \rlap{\underline{#1}}
    \par
    \setlength{\parindent}{0cm}
    \nopagebreak
    \leftskip=#2cm
    \rightskip=#3cm
}
{
    \par
}
\fi

\doendnotes{C}
\bigskip
\vfill

\clearpage

\footnotesize

\ifkorrekturansicht
  \lohead{\textsc{register}}
\fi

% theindex-Environment neu definieren ohne reledmac
\makeatletter
\renewenvironment{theindex}{%
  \ifkorrekturansicht
    \section*{\indexname}%
  \else
    \subsubsection*{Index der erwähnten Entitäten}%
  \fi
  \setlength{\parindent}{0pt}%
  \setlength{\parskip}{0pt plus 0.3pt}%
  \let\item\@idxitem
}{%
  \ifkorrekturansicht\clearpage\fi
}
\makeatother

\IfFileExists{\jobname-pw.ind}{\input{\jobname-pw.ind}}{}

% Quellenangabe nur in der Leseansicht
\ifkorrekturansicht\else
% Fallback-Definitionen, falls die .tex-Datei \titel etc. nicht gesetzt hat
\providecommand{\titel}{}
\providecommand{\editorInnen}{}
\providecommand{\dateiname}{\jobname}

\vspace{3cm}

\vfill

\footnotesize
\textsc{Quelle}: \titel. Herausgegeben von {\editorInnen}. In: \emph{Arthur Schnitzler: Briefwechsel mit Autorinnen und Autoren}.
 Digitale Edition, https://schnitzler-briefe.acdh.oeaw.ac.at/{\dateiname}.html (Stand \today)
\fi

\end{document}


      