%% latex-leseansicht-vorspann.tex
%% Vorspann für die Leseansicht.
%% Lädt die gemeinsame Datei latex-vorspann.tex mit nicht gesetztem Schalter.

\newif\ifkorrekturansicht
\korrekturansichtfalse

\input{../tex-inputs/latex-vorspann}


\section[ Paul Goldmann an Arthur Schnitzler, 29. 4. [1902]]{L03205 Paul Goldmann an Arthur Schnitzler,  29. 4. [1902]}
\nopagebreak\mylabel{L03205v}
\rehead{ }\normalsize\beginnumbering\briefempfaengerindex{Schnitzler, Arthur@\textsc{Schnitzler, Arthur}!zzzGoldmann, Paul@\emph{von Paul Goldmann}!1902-04-291@{29. 4. [1902]}|(be}
\toendnotes[C]{\smallbreak\pagebreak[2]}
\correspDesc{Versand  durch Paul Goldmann am 29. 4. [1902] in Berlin
\newline{}Erhalt  durch Arthur Schnitzler im Zeitraum [30. 4. 1902
                  – 4. 5. 1902?] in Wien}\toendnotes[C]{\smallbreak}
\Standort{DLA, A:Schnitzler, HS.NZ85.1.3172.}
\physDesc{Brief, 1 Blatt, 4 Seiten, 2108 Zeichen
\newline{}Handschrift: blaue Tinte, deutsche Kurrent
\newline{}Schnitzler: 1) mit Bleistift das Jahr »902« vermerkt  2) mit rotem Buntstift eine Unterstreichung}\toendnotes[C]{\smallbreak}
\pstart
           \centering{}{\pb}Berlin\oindex{Berlin@\textbf{Berlin}, \emph{Hauptstadt}|pw}, 29. April.\pend
           
\pstart{}Mein lieber Freund,\pend\vspace{0.5em}
\pstart
           Die »Tägliche Rundſchau\orgindex{Tägliche Rundschau@Tägliche Rundschau|pw}« hat auch heut{ }Morgen noch nicht für nöthig befunden, nachdem{ }ſie in überaus taktloſer
               Weiſe Deinen \label{K_L03205-1v}\edtext{Namen genannt\pwindex{Strecker, Karl 8.\,4.\,1862 Tąpadły – 19.\,2.\,1933 Garmisch-Partenkirchen@\textsc{Strecker, Karl} (8.\,4.\,1862 Tąpadły – 19.\,2.\,1933 Garmisch-Partenkirchen), \emph{Theaterkritiker}!litterarisch-dramatisches Hochstapler-Stücklein@\strich\emph{Ein litterarisch-dramatisches Hochstapler-Stücklein}|pwv}}{\lemma{\textnormal{\emph{Namen genannt}}}\Cendnote{\textnormal{Siehe XXXX Auszeichnungsfehler: Dokument L02634 nicht gefunden.
               }}}\label{K_L03205-1} und{ }ſogar von einem »Fall
                     \textsc{Schnitzler}\pwindex{Strecker, Karl 8.\,4.\,1862 Tąpadły – 19.\,2.\,1933 Garmisch-Partenkirchen@\textsc{Strecker, Karl} (8.\,4.\,1862 Tąpadły – 19.\,2.\,1933 Garmisch-Partenkirchen), \emph{Theaterkritiker}!litterarisch-dramatisches Hochstapler-Stücklein@\strich\emph{Ein litterarisch-dramatisches Hochstapler-Stücklein}|pwv}« geſprochen hat, von Deinem \label{K_L03205-2v}\edtext{\textsc{Dementi\pwindex{Strecker, Karl 8.\,4.\,1862 Tąpadły – 19.\,2.\,1933 Garmisch-Partenkirchen@\textsc{Strecker, Karl} (8.\,4.\,1862 Tąpadły – 19.\,2.\,1933 Garmisch-Partenkirchen), \emph{Theaterkritiker}!angebliche Telegramm Arthur Schnitzlers@\strich\emph{Das angebliche Telegramm Arthur Schnitzlers}|pwv}}}{\lemma{\textnormal{\emph{Dementi}}}\Cendnote{\textnormal{Goldmann\pwindex{Goldmann, Paul 31.\,1.\,1865 Breslau – 25.\,9.\,1935 Wien@\textsc{Goldmann, Paul} (31.\,1.\,1865 Breslau – 25.\,9.\,1935 Wien), \emph{Schriftsteller, Journalist}|pwk} dürfte diese Notiz\pwindex{Strecker, Karl 8.\,4.\,1862 Tąpadły – 19.\,2.\,1933 Garmisch-Partenkirchen@\textsc{Strecker, Karl} (8.\,4.\,1862 Tąpadły – 19.\,2.\,1933 Garmisch-Partenkirchen), \emph{Theaterkritiker}!angebliche Telegramm Arthur Schnitzlers@\strich\emph{Das angebliche Telegramm Arthur Schnitzlers}|pwkv} übersehen haben: Karl Strecker\pwindex{Strecker, Karl 8.\,4.\,1862 Tąpadły – 19.\,2.\,1933 Garmisch-Partenkirchen@\textsc{Strecker, Karl} (8.\,4.\,1862 Tąpadły – 19.\,2.\,1933 Garmisch-Partenkirchen), \emph{Theaterkritiker}|pwk}: \emph{Das angebliche Telegramm Arthur Schnitzlers}\pwindex{Strecker, Karl 8.\,4.\,1862 Tąpadły – 19.\,2.\,1933 Garmisch-Partenkirchen@\textsc{Strecker, Karl} (8.\,4.\,1862 Tąpadły – 19.\,2.\,1933 Garmisch-Partenkirchen), \emph{Theaterkritiker}!angebliche Telegramm Arthur Schnitzlers@\strich\emph{Das angebliche Telegramm Arthur Schnitzlers}|pwk}. In: \emph{Tägliche Rundschau}\pwindex{Tägliche Rundschau@\emph{Tägliche Rundschau}|pwk}, Jg. 22, Nr. 194, 26. 4. 1902, Abend-Blatt, Unterhaltungsbeilage,
                     Nr. 97, S. 388. Siehe A. S.: \emph{»Das Zeitlose ist von kürzester Dauer«}, Karl Strecker: Das angebliche Telegramm Arthur Schnitzlers, 26. 4. 1902.}}}\label{K_L03205-2} Notiz zu nehmen. Die »Tgl. Rundſchau\pwindex{Tägliche Rundschau@\emph{Tägliche Rundschau}|pw}« iſt ein alldeutſches und antiſemitiſches Blatt\strikeout{, gilt} und gilt für{ }ſehr »literariſch«, ebenſo wie der
               Herr \textsc{Karl Strecker\pwindex{Strecker, Karl 8.\,4.\,1862 Tąpadły – 19.\,2.\,1933 Garmisch-Partenkirchen@\textsc{Strecker, Karl} (8.\,4.\,1862 Tąpadły – 19.\,2.\,1933 Garmisch-Partenkirchen), \emph{Theaterkritiker}|pw}} (der ein \strikeout{germa} germaniſtiſcher Schwätzer iſt)
               für einen »vornehmen Kritiker« gilt. Es iſt möglich, daß das {\pb}Schweigen der Tgl.
                  Rdſch.\orgindex{Tägliche Rundschau@Tägliche Rundschau|pw} nur Schlamperei iſt, daß der Herr \textsc{Strecker\pwindex{Strecker, Karl 8.\,4.\,1862 Tąpadły – 19.\,2.\,1933 Garmisch-Partenkirchen@\textsc{Strecker, Karl} (8.\,4.\,1862 Tąpadły – 19.\,2.\,1933 Garmisch-Partenkirchen), \emph{Theaterkritiker}|pw}} vielleicht die Angelegenheit in{ }ſeinem nächſten Referat berühren will. Aber{ }ſchon dieſes Warten, nachdem er das Maul{ }ſo voll genommen und eine »offene Frage\pwindex{Strecker, Karl 8.\,4.\,1862 Tąpadły – 19.\,2.\,1933 Garmisch-Partenkirchen@\textsc{Strecker, Karl} (8.\,4.\,1862 Tąpadły – 19.\,2.\,1933 Garmisch-Partenkirchen), \emph{Theaterkritiker}!litterarisch-dramatisches Hochstapler-Stücklein@\strich\emph{Ein litterarisch-dramatisches Hochstapler-Stücklein}|pwv}« an Dich gerichtet hat, iſt
               unanſtändig. Ich bitte Dich daher, ihm in gemeſſenem Ton einen Brief zu{ }ſchreiben,
               Dein Erſtaunen über{ }ſein ganzes Vorgehen, Dein noch größeres Erſtaunen über \strikeout{das} die Nichtveröffentlichung Deiner Antwort\pwindex{Strecker, Karl 8.\,4.\,1862 Tąpadły – 19.\,2.\,1933 Garmisch-Partenkirchen@\textsc{Strecker, Karl} (8.\,4.\,1862 Tąpadły – 19.\,2.\,1933 Garmisch-Partenkirchen), \emph{Theaterkritiker}!angebliche Telegramm Arthur Schnitzlers@\strich\emph{Das angebliche Telegramm Arthur Schnitzlers}|pwv} auszudrücken, ihn um{ }ſofortige
               Publikation Deiner Antwort\pwindex{Strecker, Karl 8.\,4.\,1862 Tąpadły – 19.\,2.\,1933 Garmisch-Partenkirchen@\textsc{Strecker, Karl} (8.\,4.\,1862 Tąpadły – 19.\,2.\,1933 Garmisch-Partenkirchen), \emph{Theaterkritiker}!angebliche Telegramm Arthur Schnitzlers@\strich\emph{Das angebliche Telegramm Arthur Schnitzlers}|pwv} zu
               erſuchen und die Hoffnung auszuſprechen, daß {\pb}er\pwindex{Strecker, Karl 8.\,4.\,1862 Tąpadły – 19.\,2.\,1933 Garmisch-Partenkirchen@\textsc{Strecker, Karl} (8.\,4.\,1862 Tąpadły – 19.\,2.\,1933 Garmisch-Partenkirchen), \emph{Theaterkritiker}|pwv} Dich nicht dazu nöthigen
               wird, die Veröffentlichung dieſer Antwort\pwindex{Strecker, Karl 8.\,4.\,1862 Tąpadły – 19.\,2.\,1933 Garmisch-Partenkirchen@\textsc{Strecker, Karl} (8.\,4.\,1862 Tąpadły – 19.\,2.\,1933 Garmisch-Partenkirchen), \emph{Theaterkritiker}!angebliche Telegramm Arthur Schnitzlers@\strich\emph{Das angebliche Telegramm Arthur Schnitzlers}|pwv}, die eine{ }ſchlicht literariſchen Anſtandes iſt, auf andere Weiſe zu
               erzwingen. Wenn das nicht \substVorne{}\textsuperscript{h\textcolor{gray}{elf}}\substDazwischen{}hilft\substHinten{}, wirſt Du das Blatt\orgindex{Tägliche Rundschau@Tägliche Rundschau|pwv}{ }ſelbſtverſtändlich klagen. Hier\oindex{Deutschland@\textbf{Deutschland}|pwv}
               liegen die Verhältniſſe anders als in Öſterreich\oindex{Österreich@\textbf{Österreich}|pw}, und jedes Gericht wird Dir Recht geben. Ich übernehme die
               Angelegenheit und beſorge Dir einen guten \label{K_L03205-3v}\edtext{Advokaten}{\lemma{\textnormal{\emph{Advokaten}}}\Cendnote{\textnormal{Schnitzler sprach am 5. 5. 1902 jedenfalls
                  mit dem Rechtsanwalt Alfred Spitzer\pwindex{Spitzer, Alfred 3.\,12.\,1861 Frýdek-Místek – 26.\,4.\,1923 Wien@\textsc{Spitzer, Alfred} (3.\,12.\,1861 Frýdek-Místek – 26.\,4.\,1923 Wien), \emph{Rechtsanwalt}|pwk} über
                  die Angelegenheit.}}}\label{K_L03205-3}. Ebenſo würde ich rathen, daß Du bei der Wien\oindex{Wien@\textbf{Wien}, \emph{Verwaltungsgebiet}|pw}er Staatsanwaltſchaft\orgindex{Staatsanwaltschaft@Staatsanwaltschaft|pw} Anzeige erſtatteſt. Dieſem{ }ſauberen Herrn von \label{K_L03205-4v}\edtext{\textsc{Jurco\pwindex{Gréger-Jurco, Ernest von *~11.\,8.\,1860 Orăştie@\textsc{Gréger-Jurco, Ernest von} (*~11.\,8.\,1860 Orăştie), \emph{Schriftsteller}|pw}}}{\lemma{\textnormal{\emph{Jurco}}}\Cendnote{\textnormal{Ernest von Jurco-Gréger\pwindex{Gréger-Jurco, Ernest von *~11.\,8.\,1860 Orăştie@\textsc{Gréger-Jurco, Ernest von} (*~11.\,8.\,1860 Orăştie), \emph{Schriftsteller}|pwk}, dessen Stück \emph{Die Kinder der Armen}\pwindex{Gréger-Jurco, Ernest von *~11.\,8.\,1860 Orăştie@\textsc{Gréger-Jurco, Ernest von} (*~11.\,8.\,1860 Orăştie), \emph{Schriftsteller}!Kinder der Armen@\strich\emph{Die Kinder der Armen}|pwk} in dem gefälschten Telegramm\pwindex{Strecker, Karl 8.\,4.\,1862 Tąpadły – 19.\,2.\,1933 Garmisch-Partenkirchen@\textsc{Strecker, Karl} (8.\,4.\,1862 Tąpadły – 19.\,2.\,1933 Garmisch-Partenkirchen), \emph{Theaterkritiker}!litterarisch-dramatisches Hochstapler-Stücklein@\strich\emph{Ein litterarisch-dramatisches Hochstapler-Stücklein}|pwkv}{ }Schnitzlers empfohlen worden war}}}\label{K_L03205-4} muß
               doch das Handwerk gelegt werden. Auch an die {\pb}Direktion des \label{K_L03205-5v}\edtext{\textsc{Carl Weiss} Theaters\orgindex{Rose-Theater@Rose-Theater|pw}}{\lemma{\textnormal{\emph{Carl Weiss Theaters}}}\Cendnote{\textnormal{An diesem Berlin\oindex{Berlin@\textbf{Berlin}, \emph{Hauptstadt}|pwk}er Theater fand am 25. 4. 1902 die Uraufführung von
                     \emph{Die Kinder der Armen}\pwindex{Gréger-Jurco, Ernest von *~11.\,8.\,1860 Orăştie@\textsc{Gréger-Jurco, Ernest von} (*~11.\,8.\,1860 Orăştie), \emph{Schriftsteller}!Kinder der Armen@\strich\emph{Die Kinder der Armen}|pwk}\eventindex{Berlin@\textbf{Berlin}!Uraufführung Die Kinder der Armen, 25.4.1902@Uraufführung Die Kinder der Armen, 25.4.1902|pwk} statt.}}}\label{K_L03205-5}{ }ſollteſt Du{ }ſchreiben und Dir die Nennung des wirklichen Namens des Herrn \textsc{von Jurco\pwindex{Gréger-Jurco, Ernest von *~11.\,8.\,1860 Orăştie@\textsc{Gréger-Jurco, Ernest von} (*~11.\,8.\,1860 Orăştie), \emph{Schriftsteller}|pw}} erbitten. Die Direktion\orgindex{Rose-Theater@Rose-Theater|pwv}
               hat dem \strikeout{H\textcolor{gray}{e}}{ }Berliner Tageblatt\orgindex{Berliner Tageblatt@Berliner Tageblatt|pw}{ }\strikeout{\textcolor{gray}{×}} auf eine telephoniſche Anfrage geantwortet, daß \strikeout{ſih}{ }ſich \strikeout{\textcolor{gray}{u}n} unter dieſem Pſeudonym ein Autor\pwindex{Gréger-Jurco, Ernest von *~11.\,8.\,1860 Orăştie@\textsc{Gréger-Jurco, Ernest von} (*~11.\,8.\,1860 Orăştie), \emph{Schriftsteller}|pwv} aus »guter Wien\oindex{Wien@\textbf{Wien}, \emph{Verwaltungsgebiet}|pw}er Familie« verberge, deſſen Namen allerdings die Direktion\orgindex{Rose-Theater@Rose-Theater|pwv} nicht nennen
               könne.\pend
           
\pstart
           Hebe Dir (für den Fall, daß es zum Prozeß kommt) alle Berlin\oindex{Berlin@\textbf{Berlin}, \emph{Hauptstadt}|pw}er Zeitungen auf, die ich Dir{ }ſchicke,{ }ſowie eine \label{K_L03205-6v}\edtext{Copie Deines Briefes an \textsc{Strecker\pwindex{Strecker, Karl 8.\,4.\,1862 Tąpadły – 19.\,2.\,1933 Garmisch-Partenkirchen@\textsc{Strecker, Karl} (8.\,4.\,1862 Tąpadły – 19.\,2.\,1933 Garmisch-Partenkirchen), \emph{Theaterkritiker}|pw}}}{\lemma{\textnormal{\emph{Copie … Strecker}}}\Cendnote{\textnormal{nicht nachweisbar}}}\label{K_L03205-6}.\pend
           
\pstart
           Viele treue Grüße! {\\[\baselineskip]}Dein \spacefill\mbox{Paul Goldmann}\pend
           \leftskip=0em{}\selectlanguage{ngerman}\endnumbering\briefempfaengerindex{Schnitzler, Arthur@\textsc{Schnitzler, Arthur}!zzzGoldmann, Paul@\emph{von Paul Goldmann}!1902-04-291@{29. 4. [1902]}|)be}\mylabel{L03205h}  \newcommand{\dateiname}{L03205}\newcommand{\titel}{Paul Goldmann an Arthur Schnitzler, 29. 4. [1902]}\newcommand{\editorInnen}{Martin Anton Müller und Laura Untner}%% latex-leseansicht-abspann.tex
%% Abspann für die Leseansicht.
%% Der Schalter \ifkorrekturansicht ist bereits durch den Vorspann gesetzt.

%% latex-abspann.tex
%% Gemeinsamer Abspann für Korrekturansicht und Leseansicht.
%% Setzt den Schalter \ifkorrekturansicht voraus (gesetzt in den
%% einbindenden Dateien latex-korrekturansicht-abspann.tex bzw.
%% latex-leseansicht-abspann.tex).
%% ---------------------------------------------------------------

\normalsize

% Das esempio-Environment wird nur in der Leseansicht benötigt
\ifkorrekturansicht\else
\newenvironment{esempio}[3]%
{
    \vspace{1.5ex}
    \rlap{\underline{#1}}
    \par
    \setlength{\parindent}{0cm}
    \nopagebreak
    \leftskip=#2cm
    \rightskip=#3cm
}
{
    \par
}
\fi

\doendnotes{C}
\bigskip
\vfill

\clearpage

\footnotesize

\ifkorrekturansicht
  \lohead{\textsc{register}}
\fi

% theindex-Environment neu definieren ohne reledmac
\makeatletter
\renewenvironment{theindex}{%
  \ifkorrekturansicht
    \section*{\indexname}%
  \else
    \subsubsection*{Index der erwähnten Entitäten}%
  \fi
  \setlength{\parindent}{0pt}%
  \setlength{\parskip}{0pt plus 0.3pt}%
  \let\item\@idxitem
}{%
  \ifkorrekturansicht\clearpage\fi
}
\makeatother

\IfFileExists{\jobname-pw.ind}{\input{\jobname-pw.ind}}{}

% Quellenangabe nur in der Leseansicht
\ifkorrekturansicht\else
% Fallback-Definitionen, falls die .tex-Datei \titel etc. nicht gesetzt hat
\providecommand{\titel}{}
\providecommand{\editorInnen}{}
\providecommand{\dateiname}{\jobname}

\vspace{3cm}

\vfill

\footnotesize
\textsc{Quelle}: \titel. Herausgegeben von {\editorInnen}. In: \emph{Arthur Schnitzler: Briefwechsel mit Autorinnen und Autoren}.
 Digitale Edition, https://schnitzler-briefe.acdh.oeaw.ac.at/{\dateiname}.html (Stand \today)
\fi

\end{document}


