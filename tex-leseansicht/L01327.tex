%% latex-korrekturansicht-vorspann.tex
%% Vorspann für die Korrekturansicht.
%% Lädt die gemeinsame Datei latex-vorspann.tex mit gesetztem Schalter.

\newif\ifkorrekturansicht
\korrekturansichttrue

\input{../tex-inputs/latex-vorspann}


\section[Franz Blei an Arthur Schnitzler, 12. 10. 1903]{L01327 Franz Blei an Arthur Schnitzler, 12. 10. 1903}
\nopagebreak\mylabel{L01327v}
\rehead{ }\normalsize\beginnumbering\briefempfaengerindex{Schnitzler, Arthur@\textsc{Schnitzler, Arthur}!zzzBlei, Franz@\emph{von Franz Blei}!1903-10-124@{12. 10. 1903}|(be}
\toendnotes[C]{\smallbreak\pagebreak[2]}\Standort{CUL, Schnitzler, B 14.}
\physDesc{Brief, 1 Blatt, 2 Seiten, 593 Zeichen
\newline{}Handschrift: schwarze Tinte, lateinische Kurrent
\newline{}Schnitzler: 1) mit Bleistift beschriftet: »\textsc{Blei}« und datiert »12/10 903«  2) mit rotem Buntstift zwei Unterstreichungen
\newline{}Ordnung: 1) mit Bleistift von unbekannter Hand nummeriert: »\strikeout{7}«  2) mit Bleistift von unbekannter Hand nummeriert:
                                 »2«}\toendnotes[C]{\smallbreak}
\pstart
           \centering{}{\pb}München, Arcisstrasse 19\oindex{Arcisstrasse@\textbf{Arcisstraße}, \emph{Straße (K.STR)}|pw}\pend
           
\pstart{}Sehr geehrter Herr Arthur Schnitzler,\pend\vspace{0.5em}
\pstart
           Miss Johnson\pwindex{Johnson, Fanny 1862 – 07.02.1943@\textsc{Johnson, Fanny} (1862 – 07.02.1943), \emph{Schriftsteller/Schriftstellerin}|pw} kam mit Empfehlungen von sehr
               guten Engländern\oindex{England@\textbf{England}, \emph{A.ADM1}|pw}, wie Yeats\pwindex{Yeats, William Butler 13.06.1865 – 28.01.1939@\textsc{Yeats, William Butler} (13.06.1865 – 28.01.1939), \emph{Schriftsteller/Schriftstellerin}|pw} und A. Symons\pwindex{Symons, Arthur 28.02.1865 – 22.01.1945@\textsc{Symons, Arthur} (28.02.1865 – 22.01.1945), \emph{Schriftsteller/Schriftstellerin}|pw} zu
               mir und auf die Frage, was sie übersetzen solle, rieth ich ihr zu dem Grünen Kakadu\pwindex{gruene Kakadu. Groteske in einem Akt@\emph{Der grüne Kakadu. Groteske in einem Akt}|pw}. Die Dame\pwindex{Johnson, Fanny 1862 – 07.02.1943@\textsc{Johnson, Fanny} (1862 – 07.02.1943), \emph{Schriftsteller/Schriftstellerin}|pwv} wird sicher eine sehr gute Übertragung
               zu stand bringen und dass sie damit bei den englischen\oindex{England@\textbf{England}, \emph{A.ADM1}|pw} Bühnen mehr Glück haben wird wie mit ihren eigenhändigen Stücken
               ist keine Frage. Wenn Sie {\pb}daher nicht
               andere entscheidende Gründe dagegen haben, möchte ich mir erlauben, Ihnen Miss Johnson\pwindex{Johnson, Fanny 1862 – 07.02.1943@\textsc{Johnson, Fanny} (1862 – 07.02.1943), \emph{Schriftsteller/Schriftstellerin}|pw} für die Übertragung\pwindex{gruene Kakadu. Groteske in einem Akt@\emph{Der grüne Kakadu. Groteske in einem Akt}|pwv} zu empfehlen.\pend
           
\pstart
           Ich bin Ihr ganz ergebener{\\[\baselineskip]}\spacefill\mbox{Franz Blei.}\pend
           \leftskip=0em{}
\pstart
           12. 10. 1903.\pend
           \selectlanguage{ngerman}\endnumbering\briefempfaengerindex{Schnitzler, Arthur@\textsc{Schnitzler, Arthur}!zzzBlei, Franz@\emph{von Franz Blei}!1903-10-124@{12. 10. 1903}|)be}\mylabel{L01327h}  \normalsize

\doendnotes{C}
\bigskip
\vfill

\clearpage

\footnotesize

\lohead{\textsc{register}}

% Definiere theindex-Environment komplett neu ohne reledmac
\makeatletter
\renewenvironment{theindex}{%
  \section*{\indexname}%
  \setlength{\parindent}{0pt}%
  \setlength{\parskip}{0pt plus 0.3pt}%
  \let\item\@idxitem
}{%
  \clearpage
}
\makeatother

\IfFileExists{\jobname-pw.ind}{\input{\jobname-pw.ind}}{}

\end{document}

      