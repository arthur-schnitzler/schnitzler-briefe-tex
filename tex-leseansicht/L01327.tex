\input{../tex-inputs/latex-pdf-vorspann}
\begin{center}
            \textcolor{red}{ENTWURF. ENTZIFFERUNG NOCH NICHT KORREKTURGELESEN}
                      \end{center}
            
               \section[Franz Blei an Arthur Schnitzler, 12. 10. 1903]{ Franz Blei an Arthur Schnitzler, 12. 10. 1903}\nopagebreak\mylabel{v}\rehead{ }\begin{ledgroupsized}[t]{13cm}\normalsize\beginnumbering\briefempfaengerindex{Schnitzler, Arthur@\textsc{Schnitzler, Arthur}!zzzBlei, Franz@\emph{von Franz Blei}!1903-10-121@{12. 10. 1903}|(be} \toendnotes[C]{\smallbreak\pagebreak[2]} \Standort{CUL, Schnitzler, B 14.}
\physDesc{Brief, 1 Blatt, 2 Seiten
\newline{}Handschrift: schwarze Tinte, lateinische Kurrent
\newline{}Schnitzler: 1) mit Bleistift beschriftet: »\textsc{Blei}« und datiert »12/10 903« 2) mit rotem Buntstift zwei Unterstreichungen\newline{}Ordnung: 1) mit Bleistift von unbekannter Hand nummeriert: »\strikeout{7}« 2) mit Bleistift von unbekannter Hand nummeriert: »2«}\toendnotes[C]{\smallbreak}\pstart
           \centering{}{\pb}München, Arcisstrasse 19\oindex{Arcisstrasse@\textbf{Arcisstraße}|pw}\pend
           \pstart{}Sehr geehrter Herr Arthur Schnitzler,\pend\pstart
           Miss Johnson\pwindex{Johnson, Fanny 1862 – 07.02.1943@\textsc{Johnson, Fanny} (1862 – 07.02.1943), \emph{Schriftsteller/Schriftstellerin}|pw} kam mit Empfehlungen von sehr
                    guten Engländern\oindex{England@\textbf{England}|pw}, wie Yeats\pwindex{Yeats, William Butler 13.06.1865 – 28.01.1939@\textsc{Yeats, William Butler} (13.06.1865 – 28.01.1939), \emph{Schriftsteller}|pw} und A. Symons\pwindex{Symons, Arthur 28.02.1865 – 22.01.1945@\textsc{Symons, Arthur} (28.02.1865 – 22.01.1945), \emph{Schriftsteller}|pw} zu
                    mir und auf die Frage, was sie übersetzen solle, rieth ich ihr zu dem Grünen Kakadu\pwindex{Schnitzler, Arthur 15.05.1862 – 21.10.1931@\textsc{Schnitzler, Arthur} (15.05.1862 – 21.10.1931), \emph{Schriftsteller, Mediziner}!gruene Kakadu. Groteske in einem Akt1.3.1899 – 1.3.1899@\strich\emph{Der grüne Kakadu. Groteske in einem Akt} {[}1.3.1899 – 1.3.1899{]}|pw}. Die Dame\pwindex{Johnson, Fanny 1862 – 07.02.1943@\textsc{Johnson, Fanny} (1862 – 07.02.1943), \emph{Schriftsteller/Schriftstellerin}|pwv} wird sicher eine sehr gute
                    Übertragung zu stand bringen und dass sie damit bei den englischen\oindex{England@\textbf{England}|pw} Bühnen mehr Glück haben wird wie mit ihren
                    eigenhändigen Stücken ist keine Frage. Wenn Sie {\pb}daher nicht andere entscheidende
                    Gründe dagegen haben, möchte ich mir erlauben, Ihnen Miss Johnson\pwindex{Johnson, Fanny 1862 – 07.02.1943@\textsc{Johnson, Fanny} (1862 – 07.02.1943), \emph{Schriftsteller/Schriftstellerin}|pw} für die Übertragung\pwindex{Schnitzler, Arthur 15.05.1862 – 21.10.1931@\textsc{Schnitzler, Arthur} (15.05.1862 – 21.10.1931), \emph{Schriftsteller, Mediziner}!gruene Kakadu. Groteske in einem Akt1.3.1899 – 1.3.1899@\strich\emph{Der grüne Kakadu. Groteske in einem Akt} {[}1.3.1899 – 1.3.1899{]}|pwv} zu empfehlen.\pend
           \pstart
           Ich bin Ihr ganz ergebener{\\[\baselineskip]}\spacefill\mbox{Franz Blei.}\pend
           \leftskip=0em{}\pstart
           12. 10. 1903.\pend
           \endnumbering\briefempfaengerindex{Schnitzler, Arthur@\textsc{Schnitzler, Arthur}!zzzBlei, Franz@\emph{von Franz Blei}!1903-10-121@{12. 10. 1903}|)be}\mylabel{h}\end{ledgroupsized}  \newcommand{\dateiname}{L01327}\newcommand{\titel}{Franz Blei an Arthur Schnitzler, 12. 10. 1903}\newcommand{\editorInnen}{Martin Anton Müller und Gerd-Hermann Susen}\input{../tex-inputs/latex-pdf-abspann}
      