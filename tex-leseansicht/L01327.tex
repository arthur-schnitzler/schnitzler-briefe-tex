%% latex-leseansicht-vorspann.tex
%% Vorspann für die Leseansicht.
%% Lädt die gemeinsame Datei latex-vorspann.tex mit nicht gesetztem Schalter.

\newif\ifkorrekturansicht
\korrekturansichtfalse

\input{../tex-inputs/latex-vorspann}


\section[Franz Blei an Arthur Schnitzler, 12. 10. 1903]{L01327 Franz Blei an Arthur Schnitzler, 12. 10. 1903}
\nopagebreak\mylabel{L01327v}
\rehead{ }\normalsize\beginnumbering\briefempfaengerindex{Schnitzler, Arthur@\textsc{Schnitzler, Arthur}!zzzBlei, Franz@\emph{von Franz Blei}!1903-10-124@{12. 10. 1903}|(be}
\toendnotes[C]{\smallbreak\pagebreak[2]}
\correspDesc{Versand  durch Franz Blei am 12. 10. 1903 \textbf{Ort fehlend} 
\newline{}Erhalt  durch Arthur Schnitzler im Zeitraum [12. 10. 1903 – 16. 10. 1903?] in Wien}\toendnotes[C]{\smallbreak}
\Standort{CUL, Schnitzler, B 14.}
\physDesc{Brief, 1 Blatt, 2 Seiten, 593 Zeichen
\newline{}Handschrift: schwarze Tinte, lateinische Kurrent
\newline{}Schnitzler: 1) mit Bleistift beschriftet: »\textsc{Blei}« und datiert »12/10 903«  2) mit rotem Buntstift zwei Unterstreichungen
\newline{}Ordnung: 1) mit Bleistift von unbekannter Hand nummeriert: »\strikeout{7}«  2) mit Bleistift von unbekannter Hand nummeriert:
                                 »2«}\toendnotes[C]{\smallbreak}
\pstart
           \centering{}{\pb}München, Arcisstrasse 19\oindex{Arcisstraße@\textbf{Arcisstraße}, \emph{Straße}|pw}\pend
           
\pstart{}Sehr geehrter Herr Arthur Schnitzler,\pend\vspace{0.5em}
\pstart
           Miss Johnson\pwindex{Johnson, Fanny 1862 Worcester – 7.\,2.\,1943 Cambridge@\textsc{Johnson, Fanny} (1862 Worcester – 7.\,2.\,1943 Cambridge), \emph{Schriftstellerin}|pw} kam mit Empfehlungen von sehr
               guten Engländern\oindex{England@\textbf{England}, \emph{Land}|pw}, wie Yeats\pwindex{Yeats, William Butler 13.\,6.\,1865 Sandymount – 28.\,1.\,1939 Menton@\textsc{Yeats, William Butler} (13.\,6.\,1865 Sandymount – 28.\,1.\,1939 Menton), \emph{Schriftsteller}|pw} und A. Symons\pwindex{Symons, Arthur 28.\,2.\,1865 Milford Haven – 22.\,1.\,1945 Wittersham@\textsc{Symons, Arthur} (28.\,2.\,1865 Milford Haven – 22.\,1.\,1945 Wittersham), \emph{Schriftsteller}|pw} zu
               mir und auf die Frage, was sie übersetzen solle, rieth ich ihr zu dem Grünen Kakadu\pwindex{Schnitzler, Arthur 15.\,5.\,1862 Wien – 21.\,10.\,1931 ebd.@\textsc{Schnitzler, Arthur} (15.\,5.\,1862 Wien – 21.\,10.\,1931 ebd.), \emph{Schriftsteller, Mediziner}!grüne Kakadu. Groteske in einem Akt@\strich\emph{Der grüne Kakadu. Groteske in einem Akt}|pw}. Die Dame\pwindex{Johnson, Fanny 1862 Worcester – 7.\,2.\,1943 Cambridge@\textsc{Johnson, Fanny} (1862 Worcester – 7.\,2.\,1943 Cambridge), \emph{Schriftstellerin}|pwv} wird sicher eine sehr gute Übertragung
               zu stand bringen und dass sie damit bei den englischen\oindex{England@\textbf{England}, \emph{Land}|pw} Bühnen mehr Glück haben wird wie mit ihren eigenhändigen Stücken
               ist keine Frage. Wenn Sie {\pb}daher nicht
               andere entscheidende Gründe dagegen haben, möchte ich mir erlauben, Ihnen Miss Johnson\pwindex{Johnson, Fanny 1862 Worcester – 7.\,2.\,1943 Cambridge@\textsc{Johnson, Fanny} (1862 Worcester – 7.\,2.\,1943 Cambridge), \emph{Schriftstellerin}|pw} für die Übertragung\pwindex{Schnitzler, Arthur 15.\,5.\,1862 Wien – 21.\,10.\,1931 ebd.@\textsc{Schnitzler, Arthur} (15.\,5.\,1862 Wien – 21.\,10.\,1931 ebd.), \emph{Schriftsteller, Mediziner}!grüne Kakadu. Groteske in einem Akt@\strich\emph{Der grüne Kakadu. Groteske in einem Akt}|pwv} zu empfehlen.\pend
           
\pstart
           Ich bin Ihr ganz ergebener{\\[\baselineskip]}\spacefill\mbox{Franz Blei.}\pend
           \leftskip=0em{}
\pstart
           12. 10. 1903.\pend
           \selectlanguage{ngerman}\endnumbering\briefempfaengerindex{Schnitzler, Arthur@\textsc{Schnitzler, Arthur}!zzzBlei, Franz@\emph{von Franz Blei}!1903-10-124@{12. 10. 1903}|)be}\mylabel{L01327h}  \newcommand{\dateiname}{L01327}\newcommand{\titel}{Franz Blei an Arthur Schnitzler, 12. 10. 1903}\newcommand{\editorInnen}{Martin Anton Müller und Gerd-Hermann Susen}%% latex-leseansicht-abspann.tex
%% Abspann für die Leseansicht.
%% Der Schalter \ifkorrekturansicht ist bereits durch den Vorspann gesetzt.

%% latex-abspann.tex
%% Gemeinsamer Abspann für Korrekturansicht und Leseansicht.
%% Setzt den Schalter \ifkorrekturansicht voraus (gesetzt in den
%% einbindenden Dateien latex-korrekturansicht-abspann.tex bzw.
%% latex-leseansicht-abspann.tex).
%% ---------------------------------------------------------------

\normalsize

% Das esempio-Environment wird nur in der Leseansicht benötigt
\ifkorrekturansicht\else
\newenvironment{esempio}[3]%
{
    \vspace{1.5ex}
    \rlap{\underline{#1}}
    \par
    \setlength{\parindent}{0cm}
    \nopagebreak
    \leftskip=#2cm
    \rightskip=#3cm
}
{
    \par
}
\fi

\doendnotes{C}
\bigskip
\vfill

\clearpage

\footnotesize

\ifkorrekturansicht
  \lohead{\textsc{register}}
\fi

% theindex-Environment neu definieren ohne reledmac
\makeatletter
\renewenvironment{theindex}{%
  \ifkorrekturansicht
    \section*{\indexname}%
  \else
    \subsubsection*{Index der erwähnten Entitäten}%
  \fi
  \setlength{\parindent}{0pt}%
  \setlength{\parskip}{0pt plus 0.3pt}%
  \let\item\@idxitem
}{%
  \ifkorrekturansicht\clearpage\fi
}
\makeatother

\IfFileExists{\jobname-pw.ind}{\input{\jobname-pw.ind}}{}

% Quellenangabe nur in der Leseansicht
\ifkorrekturansicht\else
% Fallback-Definitionen, falls die .tex-Datei \titel etc. nicht gesetzt hat
\providecommand{\titel}{}
\providecommand{\editorInnen}{}
\providecommand{\dateiname}{\jobname}

\vspace{3cm}

\vfill

\footnotesize
\textsc{Quelle}: \titel. Herausgegeben von {\editorInnen}. In: \emph{Arthur Schnitzler: Briefwechsel mit Autorinnen und Autoren}.
 Digitale Edition, https://schnitzler-briefe.acdh.oeaw.ac.at/{\dateiname}.html (Stand \today)
\fi

\end{document}


