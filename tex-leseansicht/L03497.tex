%% latex-leseansicht-vorspann.tex
%% Vorspann für die Leseansicht.
%% Lädt die gemeinsame Datei latex-vorspann.tex mit nicht gesetztem Schalter.

\newif\ifkorrekturansicht
\korrekturansichtfalse

\input{../tex-inputs/latex-vorspann}


\section[ Felix Salten an Arthur Schnitzler, 5. 7. 1908]{L03497 Felix Salten an Arthur Schnitzler,  5. 7. 1908}
\nopagebreak\mylabel{L03497v}
\rehead{ }\normalsize\beginnumbering\briefempfaengerindex{Schnitzler, Arthur@\textsc{Schnitzler, Arthur}!zzzSalten, Felix@\emph{von Felix Salten}!1908-07-052@{5. 7. 1908}|(be}
\toendnotes[C]{\smallbreak\pagebreak[2]}
\correspDesc{Versand  durch Felix Salten am 5. 7. 1908 in Wien
\newline{}Erhalt  durch Arthur Schnitzler im Zeitraum [6. 7. 1908
                  – 10. 7. 1908?] \textbf{Ort fehlend} }\toendnotes[C]{\smallbreak}
\Standort{CUL, Schnitzler, B 89, B 1.}
\physDesc{Brief, 1 Blatt, 1 Seite, 1268 Zeichen (Briefpapier mit Trauerrand)
\newline{}Handschrift: schwarze Tinte, lateinische Kurrent
\newline{}Schnitzler: mit Bleistift Vermerk: »\textsc{Salten}« 
\newline{}Ordnung: mit Bleistift von unbekannter Hand nummeriert: »246« }\toendnotes[C]{\smallbreak}
\pstart
           \raggedleft{}{\pb}Wien\oindex{Wien@\textbf{Wien}, \emph{Verwaltungsgebiet}|pw}, 5. Juli 08\pend
           \vspace{0.5em}
\pstart
           Lieber, vielen Dank für Ihren \label{K_L03497-1v}\edtext{teilnehmenden Brief}{\lemma{\textnormal{\emph{teilnehmenden Brief}}}\Cendnote{\textnormal{XXXX Auszeichnungsfehler: Dokument L03014 nicht gefunden.
               }}}\label{K_L03497-1}, und danken Sie, bitte, auch Ihrer \label{K_L03497-2v}\edtext{l.}{\lemma{\textnormal{\emph{l.}}}\Cendnote{\textnormal{lieben}}}\label{K_L03497-2}{ }Frau\pwindex{Schnitzler, Olga 17.\,1.\,1882 Wien – 13.\,1.\,1970 Lugano@\textsc{Schnitzler, Olga} (17.\,1.\,1882 Wien – 13.\,1.\,1970 Lugano), \emph{Schauspielerin, Sängerin}|pwv} für ihre Teilnahme. Mein
               armer Bruder\pwindex{Salzmann, Michael Emil 19.\,1.\,1858 Szigetvár – 26.\,6.\,1908 Wien@\textsc{Salzmann, Michael Emil} (19.\,1.\,1858 Szigetvár – 26.\,6.\,1908 Wien), \emph{Versicherungsbeamter}|pwv} hat uns bis vor
               etwa vier Wochen noch immer Hoffnung gelaßen. Sein Befinden war schwankend, aber
               nicht verzweifelt. Gelitten hat er in den ganzen fünfzehn Monaten beständig, mehr als
               sich sagen läßt. Dann aber begann ganz plötzlich das sterben und dauerte mit allen
               Qualen, die sich nur denken laßen, vier Wochen lang. Ich war viel, namentlich aber in
               den allerletzten Stunden bei ihm, und habe versucht, ihm durch fortwährendes Morphium
               wenigstens einen Teil seiner ungemein frischen Besinnung zu nehmen. Was für eine
               Krankheit ihn weggenommen hat von uns, das wissen wir noch immer nicht. Aber \uline{jetzt} braucht man’s auch nicht mehr wißen. Ich bin
               jetzt an den Nerven wieder total herunter und von meinen Darmzuständen in peinigender
               Weise, mehr als je, heimgesucht. Trotz alledem muß ich sehr viel arbeiten, und muß
               den ganzen Sommer an die Arbeit wenden. Wir reisen Mittwoch{ }früh nach Noordwijk\oindex{Noordwijk@\textbf{Noordwijk}, \emph{Verwaltungsgebiet}|pw}, wo wir Sonntag eintreffen und bis 15. August bleiben. Otti\pwindex{Salten, Ottilie 7.\,3.\,1868 Prag – 22.\,6.\,1942 Zürich@\textsc{Salten, Ottilie} (7.\,3.\,1868 Prag – 22.\,6.\,1942 Zürich), \emph{Schauspielerin}|pw} geht von
               dort nach Franzensbad\oindex{Franzensbad@\textbf{Franzensbad}|pw}; ich über Baireuth\oindex{Bayreuth@\textbf{Bayreuth}, \emph{Hauptstadt}|pw} und Salzkammergut\oindex{Salzkammergut@\textbf{Salzkammergut}, \emph{Region}|pw}
               nach Wien\oindex{Wien@\textbf{Wien}, \emph{Verwaltungsgebiet}|pw}.\pend
           
\pstart
           Nochmals vielen Dank Ihnen Beiden\pwindex{Schnitzler, Olga 17.\,1.\,1882 Wien – 13.\,1.\,1970 Lugano@\textsc{Schnitzler, Olga} (17.\,1.\,1882 Wien – 13.\,1.\,1970 Lugano), \emph{Schauspielerin, Sängerin}|pwv}, auch meine Mama\pwindex{Salzmann, Marie 27.\,10.\,1833 Budapest – 1.\,12.\,1909 Wien@\textsc{Salzmann, Marie} (27.\,10.\,1833 Budapest – 1.\,12.\,1909 Wien)|pwv} dankt Ihnen vielmals. Schönste Grüße von uns zu Ihnen. {\\[\baselineskip]}Ihr
                  \spacefill\mbox{Salten}\pend
           \leftskip=0em{}\selectlanguage{ngerman}\endnumbering\briefempfaengerindex{Schnitzler, Arthur@\textsc{Schnitzler, Arthur}!zzzSalten, Felix@\emph{von Felix Salten}!1908-07-052@{5. 7. 1908}|)be}\mylabel{L03497h}  \newcommand{\dateiname}{L03497}\newcommand{\titel}{Felix Salten an Arthur Schnitzler, 5. 7. 1908}\newcommand{\editorInnen}{Martin Anton Müller und Laura Untner}%% latex-leseansicht-abspann.tex
%% Abspann für die Leseansicht.
%% Der Schalter \ifkorrekturansicht ist bereits durch den Vorspann gesetzt.

%% latex-abspann.tex
%% Gemeinsamer Abspann für Korrekturansicht und Leseansicht.
%% Setzt den Schalter \ifkorrekturansicht voraus (gesetzt in den
%% einbindenden Dateien latex-korrekturansicht-abspann.tex bzw.
%% latex-leseansicht-abspann.tex).
%% ---------------------------------------------------------------

\normalsize

% Das esempio-Environment wird nur in der Leseansicht benötigt
\ifkorrekturansicht\else
\newenvironment{esempio}[3]%
{
    \vspace{1.5ex}
    \rlap{\underline{#1}}
    \par
    \setlength{\parindent}{0cm}
    \nopagebreak
    \leftskip=#2cm
    \rightskip=#3cm
}
{
    \par
}
\fi

\doendnotes{C}
\bigskip
\vfill

\clearpage

\footnotesize

\ifkorrekturansicht
  \lohead{\textsc{register}}
\fi

% theindex-Environment neu definieren ohne reledmac
\makeatletter
\renewenvironment{theindex}{%
  \ifkorrekturansicht
    \section*{\indexname}%
  \else
    \subsubsection*{Index der erwähnten Entitäten}%
  \fi
  \setlength{\parindent}{0pt}%
  \setlength{\parskip}{0pt plus 0.3pt}%
  \let\item\@idxitem
}{%
  \ifkorrekturansicht\clearpage\fi
}
\makeatother

\IfFileExists{\jobname-pw.ind}{\input{\jobname-pw.ind}}{}

% Quellenangabe nur in der Leseansicht
\ifkorrekturansicht\else
% Fallback-Definitionen, falls die .tex-Datei \titel etc. nicht gesetzt hat
\providecommand{\titel}{}
\providecommand{\editorInnen}{}
\providecommand{\dateiname}{\jobname}

\vspace{3cm}

\vfill

\footnotesize
\textsc{Quelle}: \titel. Herausgegeben von {\editorInnen}. In: \emph{Arthur Schnitzler: Briefwechsel mit Autorinnen und Autoren}.
 Digitale Edition, https://schnitzler-briefe.acdh.oeaw.ac.at/{\dateiname}.html (Stand \today)
\fi

\end{document}


