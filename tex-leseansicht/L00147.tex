%% latex-leseansicht-vorspann.tex
%% Vorspann für die Leseansicht.
%% Lädt die gemeinsame Datei latex-vorspann.tex mit nicht gesetztem Schalter.

\newif\ifkorrekturansicht
\korrekturansichtfalse

\input{../tex-inputs/latex-vorspann}


\section[Arthur Schnitzler an Richard Beer-Hofmann, {[}27. 12. 1892?{]}]{L00147 Arthur Schnitzler an Richard Beer-Hofmann, {[}27. 12. 1892?{]}}
\nopagebreak\mylabel{L00147v}
\rehead{ }\normalsize\beginnumbering\briefempfaengerindex{Beer-Hofmann, Richard@\textsc{Beer-Hofmann, Richard}!zzzSchnitzler, Arthur@\emph{von Arthur Schnitzler}!1892-12-272@{{[}27. 12. 1892?{]}}|(be}
\toendnotes[C]{\smallbreak\pagebreak[2]}
\correspDesc{Versand  durch Arthur Schnitzler am [27. 12. 1892?] in Wien
\newline{}Erhalt  durch Richard Beer-Hofmann im Zeitraum [27. 12. 1892 – 31. 12. 1892?] in Wien}\toendnotes[C]{\smallbreak}
\Standort{YCGL, MSS 31.}
\physDesc{Brief, 1 Blatt, 2 Seiten, 165 Zeichen
\newline{}Handschrift: Bleistift, deutsche Kurrent}\toendnotes[C]{\smallbreak}
\pstart{}{\pb}Mein lieber Richard,\pend\vspace{0.5em}
\pstart
           ich muſs Ihnen dieſe \label{K_L00147-1v}\edtext{Karte}{\lemma{\textnormal{\emph{Karte}}}\Cendnote{\textnormal{nicht überlieferte Beilage. Im Folgenden
                  wird klar, dass es sich um eine Bertha
                     Flegmann\pwindex{Flegmann, Bertha 27.\,5.\,1852 Dubrovsky, Polen – 24.\,6.\,1933 Bad Ischl@\textsc{Flegmann, Bertha} (27.\,5.\,1852 Dubrovsky, Polen – 24.\,6.\,1933 Bad Ischl), \emph{Salonnière}|pwk} betreffende Sache handelt. Es könnte von einer Eintrittskarte für
                  die Liebhaberaufführung der \emph{Aspasia}\pwindex{\textcolor{red}{\textsuperscript{XXXX indx1}}!Aspasia@\strich\emph{Aspasia}|pwk} in ihrem
                  Salon Ende 1892 die Rede sein, was auch Schnitzlers unwilligen Ton in Einklang mit seinen anderen
                  Äußerungen zur Sache brächte.}}}\label{K_L00147-1}{ }ſchicken. Wenn Sie
                  lie\textcolor{gray}{b}ens\textcolor{gray}{würd}{ }ſind, antworten Sie mir.\pend
           
\pstart
           {\pb}herzlich{\\[\baselineskip]}Ihr{\\[\baselineskip]}\spacefill\mbox{Arth}\pend
           \leftskip=0em{}
\pstart
           \noindent{}Aber{ }ſo, dſs ich Ihren Brief der Frau F.\pwindex{Flegmann, Bertha 27.\,5.\,1852 Dubrovsky, Polen – 24.\,6.\,1933 Bad Ischl@\textsc{Flegmann, Bertha} (27.\,5.\,1852 Dubrovsky, Polen – 24.\,6.\,1933 Bad Ischl), \emph{Salonnière}|pw}
                  zeigen kann.\pend
           \selectlanguage{ngerman}\endnumbering\briefempfaengerindex{Beer-Hofmann, Richard@\textsc{Beer-Hofmann, Richard}!zzzSchnitzler, Arthur@\emph{von Arthur Schnitzler}!1892-12-272@{{[}27. 12. 1892?{]}}|)be}\mylabel{L00147h}  \newcommand{\dateiname}{L00147}\newcommand{\titel}{Arthur Schnitzler an Richard Beer-Hofmann, [27. 12. 1892?]}\newcommand{\editorInnen}{Martin Anton Müller und Gerd-Hermann Susen}%% latex-leseansicht-abspann.tex
%% Abspann für die Leseansicht.
%% Der Schalter \ifkorrekturansicht ist bereits durch den Vorspann gesetzt.

%% latex-abspann.tex
%% Gemeinsamer Abspann für Korrekturansicht und Leseansicht.
%% Setzt den Schalter \ifkorrekturansicht voraus (gesetzt in den
%% einbindenden Dateien latex-korrekturansicht-abspann.tex bzw.
%% latex-leseansicht-abspann.tex).
%% ---------------------------------------------------------------

\normalsize

% Das esempio-Environment wird nur in der Leseansicht benötigt
\ifkorrekturansicht\else
\newenvironment{esempio}[3]%
{
    \vspace{1.5ex}
    \rlap{\underline{#1}}
    \par
    \setlength{\parindent}{0cm}
    \nopagebreak
    \leftskip=#2cm
    \rightskip=#3cm
}
{
    \par
}
\fi

\doendnotes{C}
\bigskip
\vfill

\clearpage

\footnotesize

\ifkorrekturansicht
  \lohead{\textsc{register}}
\fi

% theindex-Environment neu definieren ohne reledmac
\makeatletter
\renewenvironment{theindex}{%
  \ifkorrekturansicht
    \section*{\indexname}%
  \else
    \subsubsection*{Index der erwähnten Entitäten}%
  \fi
  \setlength{\parindent}{0pt}%
  \setlength{\parskip}{0pt plus 0.3pt}%
  \let\item\@idxitem
}{%
  \ifkorrekturansicht\clearpage\fi
}
\makeatother

\IfFileExists{\jobname-pw.ind}{\input{\jobname-pw.ind}}{}

% Quellenangabe nur in der Leseansicht
\ifkorrekturansicht\else
% Fallback-Definitionen, falls die .tex-Datei \titel etc. nicht gesetzt hat
\providecommand{\titel}{}
\providecommand{\editorInnen}{}
\providecommand{\dateiname}{\jobname}

\vspace{3cm}

\vfill

\footnotesize
\textsc{Quelle}: \titel. Herausgegeben von {\editorInnen}. In: \emph{Arthur Schnitzler: Briefwechsel mit Autorinnen und Autoren}.
 Digitale Edition, https://schnitzler-briefe.acdh.oeaw.ac.at/{\dateiname}.html (Stand \today)
\fi

\end{document}


