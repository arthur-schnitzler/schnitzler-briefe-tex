%% latex-korrekturansicht-vorspann.tex
%% Vorspann für die Korrekturansicht.
%% Lädt die gemeinsame Datei latex-vorspann.tex mit gesetztem Schalter.

\newif\ifkorrekturansicht
\korrekturansichttrue

\input{../tex-inputs/latex-vorspann}


\section[Arthur Schnitzler an Richard Beer-Hofmann, {[}27. 12. 1892?{]}]{L00147 Arthur Schnitzler an Richard Beer-Hofmann, {[}27. 12. 1892?{]}}
\nopagebreak\mylabel{L00147v}
\rehead{ }\normalsize\beginnumbering\briefempfaengerindex{Beer-Hofmann, Richard@\textsc{Beer-Hofmann, Richard}!zzzSchnitzler, Arthur@\emph{von Arthur Schnitzler}!1892-12-272@{{[}27. 12. 1892?{]}}|(be}
\toendnotes[C]{\smallbreak\pagebreak[2]}\Standort{YCGL, MSS 31.}
\physDesc{Brief, 1 Blatt, 2 Seiten, 165 Zeichen
\newline{}Handschrift: Bleistift, deutsche Kurrent}\toendnotes[C]{\smallbreak}
\pstart{}{\pb}Mein lieber Richard, \pend\vspace{0.5em}
\pstart
           ich muſs Ihnen dieſe \label{K_L00147-1v}\edtext{Karte}{\lemma{\textnormal{\emph{Karte}}}\Cendnote{\textnormal{nicht überlieferte Beilage. Im Folgenden
                  wird klar, dass es sich um eine Bertha
                     Flegmann\pwindex{Flegmann, Bertha 27.05.1852 – 24.6.1933@\textsc{Flegmann, Bertha} (27.05.1852 – 24.6.1933), \emph{männliche Salonnière/Salonnière}|pwk} betreffende Sache handelt. Es könnte von einer Eintrittskarte für
                  die Liebhaberaufführung der \emph{Aspasia}\pwindex{Aspasia@\emph{Aspasia}|pwk} in ihrem
                  Salon Ende 1892 die Rede sein, was auch Schnitzlers unwilligen Ton in Einklang mit seinen anderen
                  Äußerungen zur Sache brächte.}}}\label{K_L00147-1} ſchicken. Wenn Sie
                  lie\textcolor{gray}{b}ens\textcolor{gray}{würd} ſind, antworten Sie mir.\pend
           
\pstart
           {\pb}herzlich{\\[\baselineskip]}Ihr{\\[\baselineskip]}\spacefill\mbox{Arth}\pend
           \leftskip=0em{}
\pstart
           \noindent{}Aber ſo, dſs ich Ihren Brief der Frau F.\pwindex{Flegmann, Bertha 27.05.1852 – 24.6.1933@\textsc{Flegmann, Bertha} (27.05.1852 – 24.6.1933), \emph{männliche Salonnière/Salonnière}|pw}
                  zeigen kann.\pend
           \selectlanguage{ngerman}\endnumbering\briefempfaengerindex{Beer-Hofmann, Richard@\textsc{Beer-Hofmann, Richard}!zzzSchnitzler, Arthur@\emph{von Arthur Schnitzler}!1892-12-272@{{[}27. 12. 1892?{]}}|)be}\mylabel{L00147h}  \normalsize

\doendnotes{C}
\bigskip
\vfill

\clearpage

\footnotesize

\lohead{\textsc{register}}

% Definiere theindex-Environment komplett neu ohne reledmac
\makeatletter
\renewenvironment{theindex}{%
  \section*{\indexname}%
  \setlength{\parindent}{0pt}%
  \setlength{\parskip}{0pt plus 0.3pt}%
  \let\item\@idxitem
}{%
  \clearpage
}
\makeatother

\IfFileExists{\jobname-pw.ind}{\input{\jobname-pw.ind}}{}

\end{document}

      