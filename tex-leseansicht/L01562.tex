%% latex-korrekturansicht-vorspann.tex
%% Vorspann für die Korrekturansicht.
%% Lädt die gemeinsame Datei latex-vorspann.tex mit gesetztem Schalter.

\newif\ifkorrekturansicht
\korrekturansichttrue

\input{../tex-inputs/latex-vorspann}


\section[Arthur Schnitzler an Hermann Bahr, 13. 10. 1905]{L01562 Arthur Schnitzler an Hermann Bahr, 13. 10. 1905}
\nopagebreak\mylabel{L01562v}
\rehead{ }\normalsize\beginnumbering\briefempfaengerindex{Bahr, Hermann@\textsc{Bahr, Hermann}!zzzSchnitzler, Arthur@\emph{von Arthur Schnitzler}!1905-10-132@{13. 10. 1905}|(be}
\toendnotes[C]{\smallbreak\pagebreak[2]}\Standort{TMW, HS AM 60177 Ba.}
\physDesc{Briefkarte, 627 Zeichen
\newline{}Handschrift: schwarze Tinte, deutsche Kurrent
\newline{}Ordnung: Lochung }
\buchAbdrucke{\weitereDrucke{1) Arthur Schnitzler: \emph{The Letters of Arthur Schnitzler to Hermann Bahr}. Chapel Hill: \emph{The University of North Carolina Press} 1978, S. 93.} \weitereDrucke{2) Hermann Bahr, Arthur Schnitzler: \emph{Briefwechsel, Aufzeichnungen, Dokumente (1891–1931)}. Göttingen: \emph{Wallstein} 2018, S. 361.} }\toendnotes[C]{\smallbreak}
\pstart
           {\pb}\textcolor{gray}{\textbf{Dr. Arthur Schnitzler}}\hfill 13. X. 905\pend
           
\pstart
           \textcolor{gray}{\textbf{Wien, XVIII. Spoettelgasse 7\oindex{Edmund-Weiss-Gasse 7@\textbf{Edmund-Weiß-Gasse 7}, \emph{Wohngebäude (K.WHS)}|pw}.}}\pend
           \vspace{0.5em}
\pstart
           eben, lieber Hermann, ko{\geminationm}t der \textsc{Klub} der Erlöſer\pwindex{Klub der Erloeser. Ein Akt@\emph{Der Klub der Erlöser. Ein Akt}|pw}, und dazu, zum 2. Mal, der \textsc{arme Narr\pwindex{arme Narr. Lustspiel in einem Akt@\emph{Der arme Narr. Lustspiel in einem Akt}|pw}}, den ich alſo ſchon geleſen, der mir eines deiner \label{LL087-1v}merkwürdigſten Produkte\label{LL087-1h} zu ſein ſcheint, und den ich am
               liebſten als eine Art von Vorſpiel zu \damage{e}inem ganz voll tönenden Drama auf dem Theater {\pb}ſehen möchte, das aber
               natürlich auch von \damage{dir}{ }ſein müßte, und zu dem mir alle Elemente in
               geheimnisvoller Weiſe ſchon in dieſem ſeltſamen Akt zu liegen ſcheinen.\pend
           
\pstart
           Darf ich dir bei dieſer Gelegenheit gleich für deine lieben Worte\pwindex{Zwischenspiel. (Komoedie in drei Akten von Arthur Schnitzler. Zum erstenmal aufgefuehrt im Burgtheater am 12. Oktober 1905)@\emph{Zwischenspiel. (Komödie in drei Akten von Arthur Schnitzler. Zum erstenmal aufgeführt im Burgtheater am 12. Oktober 1905)}|pwv} in der Volkszeitg\orgindex{Oesterreichische Volks-Zeitung@Österreichische Volks-Zeitung|pw} die Hand drücken?\pend
           
\pstart
           \label{K_L01562-1v}\edtext{So{\geminationn}tag
               oder Montag}{\lemma{\textnormal{\emph{Sonntag
               oder Montag}}}\Cendnote{\textnormal{Am Montag, dem 16. 10. 1905 fuhr Schnitzler mit Brahm\pwindex{Brahm, Otto 05.02.1856 – 28.11.1912@\textsc{Brahm, Otto} (05.02.1856 – 28.11.1912), \emph{Theaterleiter/Theaterleiterin, Regisseur/Regisseurin}|pwk} auf den Semmering\oindex{Semmering@\textbf{Semmering}, \emph{A.ADM3}|pwk}.}}}\label{K_L01562-1} fahr ich fort, auf einige Tage nur, dann auf
               Wiederſehen.\pend
           \pstart Von Herzen dein \spacefill\mbox{A.}\pend{}\selectlanguage{ngerman}\endnumbering\briefempfaengerindex{Bahr, Hermann@\textsc{Bahr, Hermann}!zzzSchnitzler, Arthur@\emph{von Arthur Schnitzler}!1905-10-132@{13. 10. 1905}|)be}\mylabel{L01562h}  \normalsize

\doendnotes{C}
\bigskip
\vfill

\clearpage

\footnotesize

\lohead{\textsc{register}}

% Definiere theindex-Environment komplett neu ohne reledmac
\makeatletter
\renewenvironment{theindex}{%
  \section*{\indexname}%
  \setlength{\parindent}{0pt}%
  \setlength{\parskip}{0pt plus 0.3pt}%
  \let\item\@idxitem
}{%
  \clearpage
}
\makeatother

\IfFileExists{\jobname-pw.ind}{\input{\jobname-pw.ind}}{}

\end{document}

      