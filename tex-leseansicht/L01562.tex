%% latex-leseansicht-vorspann.tex
%% Vorspann für die Leseansicht.
%% Lädt die gemeinsame Datei latex-vorspann.tex mit nicht gesetztem Schalter.

\newif\ifkorrekturansicht
\korrekturansichtfalse

\input{../tex-inputs/latex-vorspann}


         
         \renewcommand{\erwaehntePersonen}{Personen: Hermann Bahr, Otto Brahm}
         \renewcommand{\erwaehnteInstitutionen}{Institutionen: Österreichische Volks-Zeitung}
         \renewcommand{\erwaehnteOrte}{Orte: Edmund-Weiß-Gasse 7, Semmering, Wien}
         \renewcommand{\erwaehnteWerke}{Werke: Der Klub der Erlöser. Ein Akt, Der arme Narr. Lustspiel in einem Akt, Zwischenspiel. (Komödie in drei Akten von Arthur Schnitzler. Zum erstenmal aufgeführt im Burgtheater am 12. Oktober 1905)}
               \section[Arthur Schnitzler an Hermann Bahr, 13. 10. 1905]{ Arthur Schnitzler an Hermann Bahr, 13. 10. 1905}\nopagebreak\mylabel{v}\rehead{ }\begin{ledgroupsized}[t]{13cm}\normalsize\beginnumbering\briefempfaengerindex{Bahr, Hermann@\textsc{Bahr, Hermann}!zzzSchnitzler, Arthur@\emph{von Arthur Schnitzler}!1905-10-132@{13. 10. 1905}|(be} \toendnotes[C]{\smallbreak\pagebreak[2]} \Standort{TMW, HS AM 60177 Ba.}
\physDesc{Briefkarte, 627 Zeichen
\newline{}Handschrift: schwarze Tinte, deutsche Kurrent
\newline{}Ordnung: Lochung }\buchAbdrucke{\weitereDrucke{1) \emph{13. 10. 1905, Abschrift.} In: Arthur Schnitzler: \emph{The Letters of Arthur Schnitzler to Hermann Bahr}. Edited, annotated, and with an introduction, by Donald G.
                        Daviau. Chapel Hill: \emph{The University of North Carolina Press} 1978, S. 93 (University of North Carolina studies in the Germanic languages
                        and literatures, 89).} \weitereDrucke{2) Hermann Bahr, Arthur Schnitzler: \emph{Briefwechsel, Aufzeichnungen, Dokumente (1891–1931)}. Hg. Kurt Ifkovits und Martin Anton Müller. Göttingen: \emph{Wallstein} 2018, S. 361.} }\toendnotes[C]{\smallbreak}\pstart
           \noindent{}{\pb}\textcolor{gray}{\textbf{Dr. Arthur Schnitzler}}\hfill 13. X. 905\pend
           \pstart
           \textcolor{gray}{\textbf{Wien, XVIII. Spoettelgasse 7\oindex{Edmund-Weiss-Gasse 7@\textbf{Edmund-Weiß-Gasse 7}|pw}.}}\pend
           \pstart
           eben, lieber Hermann, ko{\geminationm}t der \textsc{Klub} der Erlöſer\pwindex{Bahr, Hermann 19.07.1863 – 15.01.1934@\textsc{Bahr, Hermann} (19.07.1863 – 15.01.1934), \emph{Schriftsteller, Kritiker}!Klub der Erloeser. Ein Akt01. 03. 1906@\strich\emph{Der Klub der Erlöser. Ein Akt} {[}01. 03. 1906{]}|pw}, und dazu, zum 2. Mal, der \textsc{arme Narr\pwindex{Bahr, Hermann 19.07.1863 – 15.01.1934@\textsc{Bahr, Hermann} (19.07.1863 – 15.01.1934), \emph{Schriftsteller, Kritiker}!arme Narr. Lustspiel in einem Akt28.09.1905 – 12.10.1905,@\strich\emph{Der arme Narr. Lustspiel in einem Akt} {[}28.09.1905 – 12.10.1905,{]}|pw}}, den ich alſo ſchon geleſen, der mir eines deiner \label{LL087-1v}merkwürdigſten Produkte\label{LL087-1h} zu ſein ſcheint, und den ich am
               liebſten als eine Art von Vorſpiel zu \damage{e}inem ganz voll tönenden Drama auf dem Theater {\pb}ſehen möchte, das aber
               natürlich auch von \damage{dir}{ }ſein müßte, und zu dem mir alle Elemente in
               geheimnisvoller Weiſe ſchon in dieſem ſeltſamen Akt zu liegen ſcheinen.\pend
           \pstart
           Darf ich dir bei dieſer Gelegenheit gleich für deine lieben Worte\pwindex{Bahr, Hermann 19.07.1863 – 15.01.1934@\textsc{Bahr, Hermann} (19.07.1863 – 15.01.1934), \emph{Schriftsteller, Kritiker}!Zwischenspiel. (Komoedie in drei Akten von Arthur Schnitzler. Zum erstenmal
                  aufgefuehrt im Burgtheater am 12. Oktober 1905)13. 10. 1905@\strich\emph{Zwischenspiel. (Komödie in drei Akten von Arthur Schnitzler. Zum erstenmal aufgeführt im Burgtheater am 12. Oktober 1905)} {[}13. 10. 1905{]}|pwv} in der Volkszeitg\orgindex{Oesterreichische Volks-Zeitung@Österreichische Volks-Zeitung|pw} die Hand drücken?\pend
           \pstart
           \label{K_L01562-1v}\edtext{So{\geminationn}tag
               oder Montag}{\lemma{\textnormal{\emph{Sonntag
               oder Montag}}}\Cendnote{\textnormal{Am Montag, dem 16. 10. 1905 fuhr Schnitzler\pwindex{Schnitzler, Arthur 15.05.1862 – 21.10.1931@\textsc{Schnitzler, Arthur} (15.05.1862 – 21.10.1931), \emph{Schriftsteller, Mediziner}|pwk} mit Brahm\pwindex{Brahm, Otto 05.02.1856 – 28.11.1912@\textsc{Brahm, Otto} (05.02.1856 – 28.11.1912), \emph{Theaterleiter, Regisseur}|pwk} auf den Semmering\oindex{Semmering@\textbf{Semmering}|pwk}.}}}\label{K_L01562-1h} fahr ich fort, auf einige Tage nur, dann auf
               Wiederſehen.\pend
           \pstart Von Herzen dein \spacefill\mbox{A.}\pend{}
         
         \endnumbering\mylabel{h}\end{ledgroupsized}  \newcommand{\dateiname}{L01562}\newcommand{\titel}{Arthur Schnitzler an Hermann Bahr, 13. 10. 1905}\newcommand{\editorInnen}{ Kurt Ifkovits,  Martin Anton Müller}%% latex-leseansicht-abspann.tex
%% Abspann für die Leseansicht.
%% Der Schalter \ifkorrekturansicht ist bereits durch den Vorspann gesetzt.

%% latex-abspann.tex
%% Gemeinsamer Abspann für Korrekturansicht und Leseansicht.
%% Setzt den Schalter \ifkorrekturansicht voraus (gesetzt in den
%% einbindenden Dateien latex-korrekturansicht-abspann.tex bzw.
%% latex-leseansicht-abspann.tex).
%% ---------------------------------------------------------------

\normalsize

% Das esempio-Environment wird nur in der Leseansicht benötigt
\ifkorrekturansicht\else
\newenvironment{esempio}[3]%
{
    \vspace{1.5ex}
    \rlap{\underline{#1}}
    \par
    \setlength{\parindent}{0cm}
    \nopagebreak
    \leftskip=#2cm
    \rightskip=#3cm
}
{
    \par
}
\fi

\doendnotes{C}
\bigskip
\vfill

\clearpage

\footnotesize

\ifkorrekturansicht
  \lohead{\textsc{register}}
\fi

% theindex-Environment neu definieren ohne reledmac
\makeatletter
\renewenvironment{theindex}{%
  \ifkorrekturansicht
    \section*{\indexname}%
  \else
    \subsubsection*{Index der erwähnten Entitäten}%
  \fi
  \setlength{\parindent}{0pt}%
  \setlength{\parskip}{0pt plus 0.3pt}%
  \let\item\@idxitem
}{%
  \ifkorrekturansicht\clearpage\fi
}
\makeatother

\IfFileExists{\jobname-pw.ind}{\input{\jobname-pw.ind}}{}

% Quellenangabe nur in der Leseansicht
\ifkorrekturansicht\else
% Fallback-Definitionen, falls die .tex-Datei \titel etc. nicht gesetzt hat
\providecommand{\titel}{}
\providecommand{\editorInnen}{}
\providecommand{\dateiname}{\jobname}

\vspace{3cm}

\vfill

\footnotesize
\textsc{Quelle}: \titel. Herausgegeben von {\editorInnen}. In: \emph{Arthur Schnitzler: Briefwechsel mit Autorinnen und Autoren}.
 Digitale Edition, https://schnitzler-briefe.acdh.oeaw.ac.at/{\dateiname}.html (Stand \today)
\fi

\end{document}


      