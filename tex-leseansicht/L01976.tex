%% latex-korrekturansicht-vorspann.tex
%% Vorspann für die Korrekturansicht.
%% Lädt die gemeinsame Datei latex-vorspann.tex mit gesetztem Schalter.

\newif\ifkorrekturansicht
\korrekturansichttrue

\input{../tex-inputs/latex-vorspann}


\section[Richard Beer-Hofmann an Arthur Schnitzler, 3. 11. 1910]{L01976 Richard Beer-Hofmann an Arthur Schnitzler, 3. 11. 1910}
\nopagebreak\mylabel{L01976v}
\rehead{ }\normalsize\beginnumbering\briefempfaengerindex{Schnitzler, Arthur@\textsc{Schnitzler, Arthur}!zzzBeer-Hofmann, Richard@\emph{von Richard Beer-Hofmann}!1910-11-031@{3. 11. 1910}|(be}
\toendnotes[C]{\smallbreak\pagebreak[2]}\Standort{CUL, Schnitzler, B 8.}
\physDesc{Brief, 1 Blatt, 2 Seiten, 922 Zeichen
\newline{}Handschrift: Bleistift, lateinische Kurrent
\newline{}Schnitzler: mit Bleistift beschriftet: »\textsc{B. H}« 
\newline{}Ordnung: mit Bleistift von unbekannter Hand nummeriert:
                                    »238« }
\buchAbdrucke{\weitereDrucke{Arthur Schnitzler, Richard Beer-Hofmann: \emph{Briefwechsel 1891–1931}. Wien, Zürich: \emph{Europaverlag} 1992, S. 213.} }
\pstart
           \raggedleft{}{\pb}3. November 1910\pend
           \vspace{0.5em}
\pstart
           Lieber Arthur!{ }Leo\pwindex{Van-Jung, Leo 15.10.1866 – 02.07.1939@\textsc{Van-Jung, Leo} (15.10.1866 – 02.07.1939), \emph{Gesangspädagoge/Gesangspädagogin, Mathematiker/Mathematikerin}|pw} – den ich gestern sah, – bittet um
               Folgendes: Eine Frau Moller\pwindex{Moller, Alice 24.04.1871 – Oktober 1962@\textsc{Moller, Alice} (24.04.1871 – Oktober 1962), \emph{Kassier/Kassierin}|pw} (etwas Snob),
               Schülerin von ihm, will einen Autorenabend zu Gunsten des Vereines »Mutterschutz\orgindex{Bund fuer Mutterschutz@Bund für Mutterschutz|pw}« machen. Möchte dass Sie – gegen von
               Ihnen zu besti{\geminationm}endes Honorar – lesen. Ausser Ihnen nur
               »würdige Entourage«. Salten\pwindex{Salten, Felix 06.09.1869 – 08.10.1945@\textsc{Salten, Felix} (06.09.1869 – 08.10.1945), \emph{Schriftsteller/Schriftstellerin, Journalist/Journalistin, Chefredakteur/Chefredakteurin}|pw} soll principiell
               nichts dagegen haben. Im Kl. Musikvereinssaal\oindex{Musikverein@\textbf{Musikverein}, \emph{Konzertsaal (K.KNZ)}|pw}.
                  Leo\pwindex{Van-Jung, Leo 15.10.1866 – 02.07.1939@\textsc{Van-Jung, Leo} (15.10.1866 – 02.07.1939), \emph{Gesangspädagoge/Gesangspädagogin, Mathematiker/Mathematikerin}|pw} frägt bei Ihnen an, um Ihnen – u Frau
                  M.\pwindex{Moller, Alice 24.04.1871 – Oktober 1962@\textsc{Moller, Alice} (24.04.1871 – Oktober 1962), \emph{Kassier/Kassierin}|pw} den Besuch \introOben{}eventuell\introOben{} zu ersparen. Er bittet mich Ihnen zu sagen, dass er gar nichts
               bei der Sache zu tun hat, Sie sich um seinetwillen nicht mehr {\pb}Freundlichkeit i. d. Absage (oder
               Annahme) auferlegen sollen, als \strikeout{es} Ihnen passt. Er
               hat nur Frau M.\pwindex{Moller, Alice 24.04.1871 – Oktober 1962@\textsc{Moller, Alice} (24.04.1871 – Oktober 1962), \emph{Kassier/Kassierin}|pw} zugesagt Sie vorerst zu
               fragen, da im Falle Ihrer princip. Abgeneigtheit jede weitere Belästigung für Sie
               entfällt\pend
           
\pstart
           Er erwartet – durch mich – von Ihnen nur ein »Ja« oder »Nein«; \introOben{}Mit\introOben{} Motivirungen sollen Sie Sich nicht mühen –\pend
           
\pstart
           Bitte noch heute um Antwort. Herzlichst\pend
           
\pstart
           Ihr{\\[\baselineskip]}\spacefill\mbox{Richard}\pend
           \leftskip=0em{}\selectlanguage{ngerman}\endnumbering\briefempfaengerindex{Schnitzler, Arthur@\textsc{Schnitzler, Arthur}!zzzBeer-Hofmann, Richard@\emph{von Richard Beer-Hofmann}!1910-11-031@{3. 11. 1910}|)be}\mylabel{L01976h}  \normalsize

\doendnotes{C}
\bigskip
\vfill

\clearpage

\footnotesize

\lohead{\textsc{register}}

% Definiere theindex-Environment komplett neu ohne reledmac
\makeatletter
\renewenvironment{theindex}{%
  \section*{\indexname}%
  \setlength{\parindent}{0pt}%
  \setlength{\parskip}{0pt plus 0.3pt}%
  \let\item\@idxitem
}{%
  \clearpage
}
\makeatother

\IfFileExists{\jobname-pw.ind}{\input{\jobname-pw.ind}}{}

\end{document}

      