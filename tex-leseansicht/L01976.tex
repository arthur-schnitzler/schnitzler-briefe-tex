%% latex-leseansicht-vorspann.tex
%% Vorspann für die Leseansicht.
%% Lädt die gemeinsame Datei latex-vorspann.tex mit nicht gesetztem Schalter.

\newif\ifkorrekturansicht
\korrekturansichtfalse

\input{../tex-inputs/latex-vorspann}


\section[Richard Beer-Hofmann an Arthur Schnitzler, 3. 11. 1910]{L01976 Richard Beer-Hofmann an Arthur Schnitzler, 3. 11. 1910}
\nopagebreak\mylabel{L01976v}
\rehead{ }\normalsize\beginnumbering\briefempfaengerindex{Schnitzler, Arthur@\textsc{Schnitzler, Arthur}!zzzBeer-Hofmann, Richard@\emph{von Richard Beer-Hofmann}!1910-11-031@{3. 11. 1910}|(be}
\toendnotes[C]{\smallbreak\pagebreak[2]}
\correspDesc{Versand  durch Richard Beer-Hofmann am 3. 11. 1910 in Wien
\newline{}Erhalt  durch Arthur Schnitzler im Zeitraum [3. 11. 1910
                  – 7. 11. 1910?] in Wien}\toendnotes[C]{\smallbreak}
\Standort{CUL, Schnitzler, B 8.}
\physDesc{Brief, 1 Blatt, 2 Seiten, 922 Zeichen
\newline{}Handschrift: Bleistift, lateinische Kurrent
\newline{}Schnitzler: mit Bleistift beschriftet: »\textsc{B. H}« 
\newline{}Ordnung: mit Bleistift von unbekannter Hand nummeriert:
                                    »238« }
\buchAbdrucke{\weitereDrucke{Arthur Schnitzler, Richard Beer-Hofmann: \emph{Briefwechsel 1891–1931}. Herausgegeben von Konstanze Fliedl. Wien, Zürich: \emph{Europaverlag} 1992, S. 213.} }
\pstart
           \raggedleft{}{\pb}3. November 1910\pend
           \vspace{0.5em}
\pstart
           Lieber Arthur!{ }Leo\pwindex{Van-Jung, Leo 15.\,10.\,1866 Odessa – 2.\,7.\,1939 Riga@\textsc{Van-Jung, Leo} (15.\,10.\,1866 Odessa – 2.\,7.\,1939 Riga), \emph{Gesangspädagoge, Mathematiker}|pw} – den ich gestern sah, – bittet um
               Folgendes: Eine Frau Moller\pwindex{Moller, Alice 24.\,4.\,1871 Wien – Oktober 1962@\textsc{Moller, Alice} (24.\,4.\,1871 Wien – Oktober 1962), \emph{Kassierin}|pw} (etwas Snob),
               Schülerin von ihm, will einen Autorenabend zu Gunsten des Vereines »Mutterschutz\orgindex{Bund für Mutterschutz@Bund für Mutterschutz|pw}« machen. Möchte dass Sie – gegen von
               Ihnen zu besti{\geminationm}endes Honorar – lesen. Ausser Ihnen nur
               »würdige Entourage«. Salten\pwindex{Salten, Felix 6.\,9.\,1869 Budapest – 8.\,10.\,1945 Zürich@\textsc{Salten, Felix} (6.\,9.\,1869 Budapest – 8.\,10.\,1945 Zürich), \emph{Schriftsteller, Journalist, Chefredakteur}|pw} soll principiell
               nichts dagegen haben. Im Kl. Musikvereinssaal\oindex{Wien@\textbf{Wien}!I., Innere Stadt@\textbf{I., Innere Stadt}!Musikverein@\textbf{Musikverein}, \emph{Konzertsaal}|pw}.
                  Leo\pwindex{Van-Jung, Leo 15.\,10.\,1866 Odessa – 2.\,7.\,1939 Riga@\textsc{Van-Jung, Leo} (15.\,10.\,1866 Odessa – 2.\,7.\,1939 Riga), \emph{Gesangspädagoge, Mathematiker}|pw} frägt bei Ihnen an, um Ihnen – u Frau
                  M.\pwindex{Moller, Alice 24.\,4.\,1871 Wien – Oktober 1962@\textsc{Moller, Alice} (24.\,4.\,1871 Wien – Oktober 1962), \emph{Kassierin}|pw} den Besuch \introOben{}eventuell\introOben{} zu ersparen. Er bittet mich Ihnen zu sagen, dass er gar nichts
               bei der Sache zu tun hat, Sie sich um seinetwillen nicht mehr {\pb}Freundlichkeit i. d. Absage (oder
               Annahme) auferlegen sollen, als \strikeout{es} Ihnen passt. Er
               hat nur Frau M.\pwindex{Moller, Alice 24.\,4.\,1871 Wien – Oktober 1962@\textsc{Moller, Alice} (24.\,4.\,1871 Wien – Oktober 1962), \emph{Kassierin}|pw} zugesagt Sie vorerst zu
               fragen, da im Falle Ihrer princip. Abgeneigtheit jede weitere Belästigung für Sie
               entfällt\pend
           
\pstart
           Er erwartet – durch mich – von Ihnen nur ein »Ja« oder »Nein«; \introOben{}Mit\introOben{} Motivirungen sollen Sie Sich nicht mühen –\pend
           
\pstart
           Bitte noch heute um Antwort. Herzlichst\pend
           
\pstart
           Ihr{\\[\baselineskip]}\spacefill\mbox{Richard}\pend
           \leftskip=0em{}\selectlanguage{ngerman}\endnumbering\briefempfaengerindex{Schnitzler, Arthur@\textsc{Schnitzler, Arthur}!zzzBeer-Hofmann, Richard@\emph{von Richard Beer-Hofmann}!1910-11-031@{3. 11. 1910}|)be}\mylabel{L01976h}  \newcommand{\dateiname}{L01976}\newcommand{\titel}{Richard Beer-Hofmann an Arthur Schnitzler, 3. 11. 1910}\newcommand{\editorInnen}{Martin Anton Müller und Gerd-Hermann Susen}%% latex-leseansicht-abspann.tex
%% Abspann für die Leseansicht.
%% Der Schalter \ifkorrekturansicht ist bereits durch den Vorspann gesetzt.

%% latex-abspann.tex
%% Gemeinsamer Abspann für Korrekturansicht und Leseansicht.
%% Setzt den Schalter \ifkorrekturansicht voraus (gesetzt in den
%% einbindenden Dateien latex-korrekturansicht-abspann.tex bzw.
%% latex-leseansicht-abspann.tex).
%% ---------------------------------------------------------------

\normalsize

% Das esempio-Environment wird nur in der Leseansicht benötigt
\ifkorrekturansicht\else
\newenvironment{esempio}[3]%
{
    \vspace{1.5ex}
    \rlap{\underline{#1}}
    \par
    \setlength{\parindent}{0cm}
    \nopagebreak
    \leftskip=#2cm
    \rightskip=#3cm
}
{
    \par
}
\fi

\doendnotes{C}
\bigskip
\vfill

\clearpage

\footnotesize

\ifkorrekturansicht
  \lohead{\textsc{register}}
\fi

% theindex-Environment neu definieren ohne reledmac
\makeatletter
\renewenvironment{theindex}{%
  \ifkorrekturansicht
    \section*{\indexname}%
  \else
    \subsubsection*{Index der erwähnten Entitäten}%
  \fi
  \setlength{\parindent}{0pt}%
  \setlength{\parskip}{0pt plus 0.3pt}%
  \let\item\@idxitem
}{%
  \ifkorrekturansicht\clearpage\fi
}
\makeatother

\IfFileExists{\jobname-pw.ind}{\input{\jobname-pw.ind}}{}

% Quellenangabe nur in der Leseansicht
\ifkorrekturansicht\else
% Fallback-Definitionen, falls die .tex-Datei \titel etc. nicht gesetzt hat
\providecommand{\titel}{}
\providecommand{\editorInnen}{}
\providecommand{\dateiname}{\jobname}

\vspace{3cm}

\vfill

\footnotesize
\textsc{Quelle}: \titel. Herausgegeben von {\editorInnen}. In: \emph{Arthur Schnitzler: Briefwechsel mit Autorinnen und Autoren}.
 Digitale Edition, https://schnitzler-briefe.acdh.oeaw.ac.at/{\dateiname}.html (Stand \today)
\fi

\end{document}


