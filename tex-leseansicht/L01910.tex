%% latex-leseansicht-vorspann.tex
%% Vorspann für die Leseansicht.
%% Lädt die gemeinsame Datei latex-vorspann.tex mit nicht gesetztem Schalter.

\newif\ifkorrekturansicht
\korrekturansichtfalse

\input{../tex-inputs/latex-vorspann}


\section[Hugo von Hofmannsthal an Olga Schnitzler, 26. 12. 1909]{L01910 Hugo von Hofmannsthal an Olga Schnitzler, 26. 12. 1909}
\nopagebreak\mylabel{L01910v}
\rehead{ }\normalsize\beginnumbering\briefempfaengerindex{Schnitzler, Olga@\textsc{Schnitzler, Olga}!zzzHofmannsthal, Hugo von@\emph{von Hugo von Hofmannsthal}!1909-12-261@{26. 12. 1909}|(be}
\toendnotes[C]{\smallbreak\pagebreak[2]}
\correspDesc{Versand  durch Hugo von Hofmannsthal am 26. 12. 1909 \textbf{Ort fehlend} 
\newline{}Erhalt  durch Olga Schnitzler im Zeitraum [26. 12. 1909 – 30. 12. 1909?] in Wien}\toendnotes[C]{\smallbreak}
\Standort{CUL, Schnitzler, B 43.}
\physDesc{Brief, 1 Blatt, 1 Seite, 304 Zeichen
\newline{}Handschrift: schwarze Tinte, lateinische Kurrent
\newline{}Ordnung: 1) mit Bleistift von unbekannter Hand nummeriert: »\strikeout{306}«  2) mit Bleistift von unbekannter Hand nummeriert:
                                    »313«}
\buchAbdrucke{\weitereDrucke{Hugo von Hofmannsthal, Arthur Schnitzler: \emph{Briefwechsel}. Herausgegeben von Therese Nickl und Heinrich Schnitzler. Frankfurt am Main: \emph{S. Fischer} 1964, S. 380–381.} }\toendnotes[C]{\smallbreak}\stanza{}{\pb}Seit Olga uns ein Zweites\pwindex{Cappellini, Lili 13.\,9.\,1909 Wien – 26.\,7.\,1928 Venedig@\textsc{Cappellini, Lili} (13.\,9.\,1909 Wien – 26.\,7.\,1928 Venedig)|pwv} bracht\newverse{}Wird sie noch doppelt hochgeacht\newverse{}und gar \uline{noch}{ }\uline{schöner} sie zu machen\newverse{}schenkt man ihr nette \label{K_L01910-1v}\edtext{Siebensachen}{\lemma{\textnormal{\emph{Siebensachen}}}\Cendnote{\textnormal{Sie bekam ein
                     Medaillon aus dem Atelier der \emph{Wiener
                        Werkstätten}\orgindex{Wiener Werkstätte@Wiener Werkstätte|pwk} geschenkt.}}}\label{K_L01910-1}.\newverse{}Worauf sie fröhlich sich bespiegelt\newverse{}und seufzt: Ach ist der Hugo frech!\newverse{}{\dotsfour}\newverse{}Das Schächtelchen ist nicht –  –»versiegelt\pwindex{Blech, Leo 21.\,4.\,1871 Aachen – 24.\,8.\,1958 Berlin@\textsc{Blech, Leo} (21.\,4.\,1871 Aachen – 24.\,8.\,1958 Berlin), \emph{Komponist, Dirigent}!Versiegelt. Komische Oper@\strich\emph{Versiegelt. Komische Oper}|pwv}«\newverse{}und was darin ist – nicht von Blech\pwindex{Blech, Leo 21.\,4.\,1871 Aachen – 24.\,8.\,1958 Berlin@\textsc{Blech, Leo} (21.\,4.\,1871 Aachen – 24.\,8.\,1958 Berlin), \emph{Komponist, Dirigent}|pw}.\stanzaend{}
\pstart
           An Olga.\hspace*{1.5em}26. XII. 1909.\pend
           \selectlanguage{ngerman}\endnumbering\briefempfaengerindex{Schnitzler, Olga@\textsc{Schnitzler, Olga}!zzzHofmannsthal, Hugo von@\emph{von Hugo von Hofmannsthal}!1909-12-261@{26. 12. 1909}|)be}\mylabel{L01910h}  \newcommand{\dateiname}{L01910}\newcommand{\titel}{Hugo von Hofmannsthal an Olga Schnitzler, 26. 12. 1909}\newcommand{\editorInnen}{Martin Anton Müller und Gerd-Hermann Susen}%% latex-leseansicht-abspann.tex
%% Abspann für die Leseansicht.
%% Der Schalter \ifkorrekturansicht ist bereits durch den Vorspann gesetzt.

%% latex-abspann.tex
%% Gemeinsamer Abspann für Korrekturansicht und Leseansicht.
%% Setzt den Schalter \ifkorrekturansicht voraus (gesetzt in den
%% einbindenden Dateien latex-korrekturansicht-abspann.tex bzw.
%% latex-leseansicht-abspann.tex).
%% ---------------------------------------------------------------

\normalsize

% Das esempio-Environment wird nur in der Leseansicht benötigt
\ifkorrekturansicht\else
\newenvironment{esempio}[3]%
{
    \vspace{1.5ex}
    \rlap{\underline{#1}}
    \par
    \setlength{\parindent}{0cm}
    \nopagebreak
    \leftskip=#2cm
    \rightskip=#3cm
}
{
    \par
}
\fi

\doendnotes{C}
\bigskip
\vfill

\clearpage

\footnotesize

\ifkorrekturansicht
  \lohead{\textsc{register}}
\fi

% theindex-Environment neu definieren ohne reledmac
\makeatletter
\renewenvironment{theindex}{%
  \ifkorrekturansicht
    \section*{\indexname}%
  \else
    \subsubsection*{Index der erwähnten Entitäten}%
  \fi
  \setlength{\parindent}{0pt}%
  \setlength{\parskip}{0pt plus 0.3pt}%
  \let\item\@idxitem
}{%
  \ifkorrekturansicht\clearpage\fi
}
\makeatother

\IfFileExists{\jobname-pw.ind}{\input{\jobname-pw.ind}}{}

% Quellenangabe nur in der Leseansicht
\ifkorrekturansicht\else
% Fallback-Definitionen, falls die .tex-Datei \titel etc. nicht gesetzt hat
\providecommand{\titel}{}
\providecommand{\editorInnen}{}
\providecommand{\dateiname}{\jobname}

\vspace{3cm}

\vfill

\footnotesize
\textsc{Quelle}: \titel. Herausgegeben von {\editorInnen}. In: \emph{Arthur Schnitzler: Briefwechsel mit Autorinnen und Autoren}.
 Digitale Edition, https://schnitzler-briefe.acdh.oeaw.ac.at/{\dateiname}.html (Stand \today)
\fi

\end{document}


