%% latex-korrekturansicht-vorspann.tex
%% Vorspann für die Korrekturansicht.
%% Lädt die gemeinsame Datei latex-vorspann.tex mit gesetztem Schalter.

\newif\ifkorrekturansicht
\korrekturansichttrue

\input{../tex-inputs/latex-vorspann}


\section[Paul Goldmann an Arthur Schnitzler, {[}24. 8. 1896?{]}]{L02691 Paul Goldmann an Arthur Schnitzler, {[}24. 8. 1896?{]}}
\nopagebreak\mylabel{L02691v}
\rehead{ }\normalsize\beginnumbering\briefempfaengerindex{Schnitzler, Arthur@\textsc{Schnitzler, Arthur}!zzzGoldmann, Paul@\emph{von Paul Goldmann}!1896-08-241@{{[}24. 8. 1896?{]}}|(be}
\toendnotes[C]{\smallbreak\pagebreak[2]}\Standort{DLA, A:Schnitzler, HS.NZ85.1.3166.}
\physDesc{Telegramm, 125 Zeichen
\newline{}maschinell
\newline{}Ordnung: beschnitten }\toendnotes[C]{\smallbreak}
\pstart
           \noindent{}{\pb}tausend dank fuer frohe nachricht und von \label{T_L02691-1v}\edtext{ganzem}{\lemma{\textnormal{\emph{ganzem}}}\Cendnote{\textnormal{In der Vorlage steht: »ganzen«.}}}\label{T_L02691-1} herzen
               glueckwunsch jetzt ist dir das \label{K_L02691-1v}\edtext{stueck\pwindex{Freiwild. Schauspiel in 3 Akten@\emph{Freiwild. Schauspiel in 3 Akten}|pwuv}}{\lemma{\textnormal{\emph{stueck}}}\Cendnote{\textnormal{Das Telegramm weist keine Datierung auf,
                  wird aber in Schnitzlers Nachlass mit den
                  Korrespondenzstücken des Jahres 1896 aufbewahrt.
                  Inhaltlich passt es in diesem Jahr am besten zu \emph{Freiwild}\pwindex{Freiwild. Schauspiel in 3 Akten@\emph{Freiwild. Schauspiel in 3 Akten}|pwk}, dem Schnitzler selbst skeptisch gegenüberstand, das aber bei einem Treffen
                  am 23. 8. 1896 von
                     Otto Brahm\pwindex{Brahm, Otto 05.02.1856 – 28.11.1912@\textsc{Brahm, Otto} (05.02.1856 – 28.11.1912), \emph{Theaterleiter/Theaterleiterin, Regisseur/Regisseurin}|pwk} für das \emph{Deutsche Theater}\orgindex{Deutsches Theater Berlin@Deutsches Theater Berlin|pwk} angenommen wurde.}}}\label{K_L02691-1} hoffentlich
               sympathischer \spacefill\mbox{goldmann =}\pend
           \selectlanguage{ngerman}\endnumbering\briefempfaengerindex{Schnitzler, Arthur@\textsc{Schnitzler, Arthur}!zzzGoldmann, Paul@\emph{von Paul Goldmann}!1896-08-241@{{[}24. 8. 1896?{]}}|)be}\mylabel{L02691h}  \normalsize

\doendnotes{C}
\bigskip
\vfill

\clearpage

\footnotesize

\lohead{\textsc{register}}

% Definiere theindex-Environment komplett neu ohne reledmac
\makeatletter
\renewenvironment{theindex}{%
  \section*{\indexname}%
  \setlength{\parindent}{0pt}%
  \setlength{\parskip}{0pt plus 0.3pt}%
  \let\item\@idxitem
}{%
  \clearpage
}
\makeatother

\IfFileExists{\jobname-pw.ind}{\input{\jobname-pw.ind}}{}

\end{document}

      