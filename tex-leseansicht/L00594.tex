%% latex-leseansicht-vorspann.tex
%% Vorspann für die Leseansicht.
%% Lädt die gemeinsame Datei latex-vorspann.tex mit nicht gesetztem Schalter.

\newif\ifkorrekturansicht
\korrekturansichtfalse

\input{../tex-inputs/latex-vorspann}


         
         \newcommand{\erwaehntePersonen}{Personen: Felix Dörmann, Paul Goldmann, Friedrich Mitterwurzer, Adele Sandrock, Adolf von Sonnenthal}
         \newcommand{\erwaehnteOrte}{Orte: Bad Ischl, Baden bei Wien, Raimund-Theater, Theater in der Josefstadt, VIII., Josefstadt, Wien}
         \newcommand{\erwaehnteWerke}{Werke: Freiwild. Schauspiel in 3 Akten, Liebelei. Schauspiel in drei Akten, Sein Sohn}
               \section[Richard Beer-Hofmann an Arthur Schnitzler, 19. 9. 1896]{ Richard Beer-Hofmann an Arthur Schnitzler, 19. 9. 1896}\nopagebreak\mylabel{v}\rehead{ }\begin{ledgroupsized}[t]{13cm}\normalsize\beginnumbering \toendnotes[C]{\smallbreak\pagebreak[2]} \Standort{CUL, Schnitzler, B 8.}
\physDesc{Brief, 3 Blätter, 9 Seiten
\newline{}Handschrift: 1) Bleistift, lateinische Kurrent (\noindent{}3. Blatt)\hspace{1em}2) blauer Buntstift, lateinische Kurrent (\noindent{}1.–2. Blatt)\hspace{1em}\newline{}Ordnung: mit Bleistift von unbekannter Hand nummeriert:
                                    »86« beziehungsweise
                                 »86a?« }\buchAbdrucke{\weitereDrucke{Arthur Schnitzler, Richard Beer-Hofmann: \emph{Briefwechsel 1891–1931}. Hg. Konstanze Fliedl. Wien, Zürich: \emph{Europaverlag} 1992, S. 97–98.} }\toendnotes[C]{\smallbreak}\pstart
           \centering{}{\pb}Baden\oindex{Baden bei Wien@\textbf{Baden bei Wien}|pw}{ }19/IX 96\pend
           \pstart
           Lieber Arthur! Ich bin schon Mittwoch{ }Abends in Wien\oindex{Wien@\textbf{Wien}|pw} und möchte gerne den
                  Abend mit Ihnen beisa{\geminationm}en sein. Schreiben
               Sie mir ob Sie frei sind und wann Sie mich abholen möchten. Außerdem, bitte, nehmen
               Sie mir für Donnerstag (Dörmann\pwindex{Doermann, Felix 29.05.1870 – 26.10.1928@\textsc{Dörmann, Felix} (29.05.1870 – 26.10.1928), \emph{Schriftsteller}|pw}?)
               einen Sitz\pwindex{Doermann, Felix 29.05.1870 – 26.10.1928@\textsc{Dörmann, Felix} (29.05.1870 – 26.10.1928), \emph{Schriftsteller}!Sein Sohn16. 10. 1896@\strich\emph{Sein Sohn} {[}16. 10. 1896{]}|pwv} (neben sich – oder {\pb}Ecke) ins Raimundtheater\oindex{Raimund-Theater@\textbf{Raimund-Theater}|pw} – ja?\pend
           \pstart
           Schließlich dachte ich heute Nachmittag an »Liebelei\pwindex{Schnitzler, Arthur 15.05.1862 – 21.10.1931@\textsc{Schnitzler, Arthur} (15.05.1862 – 21.10.1931), \emph{Schriftsteller, Mediziner}!Liebelei. Schauspiel in drei Akten1895-10-09@\strich\emph{Liebelei. Schauspiel in drei Akten} {[}1895-10-09{]}|pw}« und »Freiwild\pwindex{Schnitzler, Arthur 15.05.1862 – 21.10.1931@\textsc{Schnitzler, Arthur} (15.05.1862 – 21.10.1931), \emph{Schriftsteller, Mediziner}!Freiwild. Schauspiel in 3 Akten1896@\strich\emph{Freiwild. Schauspiel in 3 Akten} {[}1896{]}|pw}«. Sie machen das
               Leben – wissen Sie das \uuline{Leben} (nicht das Leben das »so
               ist {\pb}wie – –{[}«{]})
               sehr schwer. Duellirt man sich – wird man unfehlbar erschossen; Duellirt man sich
               nicht, – no da wird man doch erst recht erschossen – das ist schrecklich. Im übrigen
               könnten Sie nicht 6 Akte aus den zwei Stücken {\pb}machen? Nur i{\geminationm}er abwechselnd einen Akt von Liebelei\pwindex{Schnitzler, Arthur 15.05.1862 – 21.10.1931@\textsc{Schnitzler, Arthur} (15.05.1862 – 21.10.1931), \emph{Schriftsteller, Mediziner}!Liebelei. Schauspiel in drei Akten1895-10-09@\strich\emph{Liebelei. Schauspiel in drei Akten} {[}1895-10-09{]}|pw} und Freiwild\pwindex{Schnitzler, Arthur 15.05.1862 – 21.10.1931@\textsc{Schnitzler, Arthur} (15.05.1862 – 21.10.1931), \emph{Schriftsteller, Mediziner}!Freiwild. Schauspiel in 3 Akten1896@\strich\emph{Freiwild. Schauspiel in 3 Akten} {[}1896{]}|pw} spielen
               lassen?\pend
           \pstart
           Der Lobheimer\pwindex{Schnitzler, Arthur 15.05.1862 – 21.10.1931@\textsc{Schnitzler, Arthur} (15.05.1862 – 21.10.1931), \emph{Schriftsteller, Mediziner}!Liebelei. Schauspiel in drei Akten1895-10-09@\strich\emph{Liebelei. Schauspiel in drei Akten} {[}1895-10-09{]}|pwv} wird im I Akt nicht
               gefordert, sondern statt des \label{K_L00594_1v}\edtext{Mitterwurzer\pwindex{Mitterwurzer, Friedrich 16.10.1844 – 13.02.1897@\textsc{Mitterwurzer, Friedrich} (16.10.1844 – 13.02.1897), \emph{Schauspieler}|pw}}{\lemma{\textnormal{\emph{Mitterwurzer}}}\Cendnote{\textnormal{Dieser hatte in der Uraufführung den
                  »Herrn«, den betrogenen Ehemann, gespielt.}}}\label{K_L00594_1h} ko{\geminationm}t ein Briefträger – der auch zweimal läutet, {\pb}mit einem Expressbrief – der \strikeout{Pau\pwindex{Goldmann, Paul 31.01.1865 – 25.09.1935@\textsc{Goldmann, Paul} (31.01.1865 – 25.09.1935), \emph{Schriftsteller, Journalist}|pwu}}{ }Fritz\pwindex{Schnitzler, Arthur 15.05.1862 – 21.10.1931@\textsc{Schnitzler, Arthur} (15.05.1862 – 21.10.1931), \emph{Schriftsteller, Mediziner}!Liebelei. Schauspiel in drei Akten1895-10-09@\strich\emph{Liebelei. Schauspiel in drei Akten} {[}1895-10-09{]}|pwv} soll aufs Land zu seinen
               Eltern. Im II Akt (I. Akt \substVorne{}\textsuperscript{Liebelei\pwindex{Schnitzler, Arthur 15.05.1862 – 21.10.1931@\textsc{Schnitzler, Arthur} (15.05.1862 – 21.10.1931), \emph{Schriftsteller, Mediziner}!Liebelei. Schauspiel in drei Akten1895-10-09@\strich\emph{Liebelei. Schauspiel in drei Akten} {[}1895-10-09{]}|pw}}{\allowbreak}\substDazwischen{}Freiwild\pwindex{Schnitzler, Arthur 15.05.1862 – 21.10.1931@\textsc{Schnitzler, Arthur} (15.05.1862 – 21.10.1931), \emph{Schriftsteller, Mediziner}!Freiwild. Schauspiel in 3 Akten1896@\strich\emph{Freiwild. Schauspiel in 3 Akten} {[}1896{]}|pw}\substHinten{}) \substVorne{}\textsuperscript{wird er gefordert}{\allowbreak}\substDazwischen{}beleidigt er –\substHinten{}.\pend
           \pstart
           Im III Akt fährt er nach Wien\oindex{Wien@\textbf{Wien}|pw} Abschied nehmen (II Akt
                  Liebelei\pwindex{Schnitzler, Arthur 15.05.1862 – 21.10.1931@\textsc{Schnitzler, Arthur} (15.05.1862 – 21.10.1931), \emph{Schriftsteller, Mediziner}!Liebelei. Schauspiel in drei Akten1895-10-09@\strich\emph{Liebelei. Schauspiel in drei Akten} {[}1895-10-09{]}|pw}).\pend
           \pstart
           Im IV Akt (II Akt {\pb}Freiwild\pwindex{Schnitzler, Arthur 15.05.1862 – 21.10.1931@\textsc{Schnitzler, Arthur} (15.05.1862 – 21.10.1931), \emph{Schriftsteller, Mediziner}!Freiwild. Schauspiel in 3 Akten1896@\strich\emph{Freiwild. Schauspiel in 3 Akten} {[}1896{]}|pw}) überlegt er sich die Sache. Im V Akt
               (III Akt Freiwild\pwindex{Schnitzler, Arthur 15.05.1862 – 21.10.1931@\textsc{Schnitzler, Arthur} (15.05.1862 – 21.10.1931), \emph{Schriftsteller, Mediziner}!Freiwild. Schauspiel in 3 Akten1896@\strich\emph{Freiwild. Schauspiel in 3 Akten} {[}1896{]}|pw}) wird er todtgeschossen –
               »Gruppe« sagt die Sandrock\pwindex{Sandrock, Adele 1863-08-19 – 1937-08-30@\textsc{Sandrock, Adele} (1863-08-19 – 1937-08-30), \emph{Schauspielerin}|pw}. Im VI Akt (III Akt
                  Liebelei\pwindex{Schnitzler, Arthur 15.05.1862 – 21.10.1931@\textsc{Schnitzler, Arthur} (15.05.1862 – 21.10.1931), \emph{Schriftsteller, Mediziner}!Liebelei. Schauspiel in drei Akten1895-10-09@\strich\emph{Liebelei. Schauspiel in drei Akten} {[}1895-10-09{]}|pw}) teilt mans {\pb}dem »süßen Mädel« mit. Sehr feine
               Verkettung: Sonnenthal\pwindex{Sonnenthal, Adolf von 1834-12-21 – 1909-04-04@\textsc{Sonnenthal, Adolf von} (1834-12-21 – 1909-04-04), \emph{Schauspieler}|pw} ist Geigenspieler am Josefstädtertheater\oindex{Theater in der Josefstadt@\textbf{Theater in der Josefstadt}|pw}! Die Schauspielerin ist an der Josefstadt\oindex{VIII., Josefstadt@\textbf{VIII., Josefstadt}|pw}, im So{\geminationm}er im
                  Bade{\pb}ort – Ischl\oindex{Bad Ischl@\textbf{Bad Ischl}|pw} – Ha! Bitte schlagen Sie mich nicht todt.\pend
           \pstart
           Herzlichst{\\[\baselineskip]}\spacefill\mbox{Richard}\pend
           \leftskip=0em{}\pstart
           \noindent{}{\pb}Da ich sehe daß das Couvert
                  durchsichtig ist und das »Todtschlagen« die Polizei beunruhigen könnte so nehme
                  ich noch ein Couvert drüber.\pend
           \pstart
           \raggedleft{}R\pend
           
         
         \endnumbering\mylabel{h}\end{ledgroupsized}  \newcommand{\dateiname}{L00594}\newcommand{\titel}{Richard Beer-Hofmann an Arthur Schnitzler, 19. 9. 1896}\newcommand{\editorInnen}{Martin Anton Müller und Gerd-Hermann Susen}%% latex-leseansicht-abspann.tex
%% Abspann für die Leseansicht.
%% Der Schalter \ifkorrekturansicht ist bereits durch den Vorspann gesetzt.

%% latex-abspann.tex
%% Gemeinsamer Abspann für Korrekturansicht und Leseansicht.
%% Setzt den Schalter \ifkorrekturansicht voraus (gesetzt in den
%% einbindenden Dateien latex-korrekturansicht-abspann.tex bzw.
%% latex-leseansicht-abspann.tex).
%% ---------------------------------------------------------------

\normalsize

% Das esempio-Environment wird nur in der Leseansicht benötigt
\ifkorrekturansicht\else
\newenvironment{esempio}[3]%
{
    \vspace{1.5ex}
    \rlap{\underline{#1}}
    \par
    \setlength{\parindent}{0cm}
    \nopagebreak
    \leftskip=#2cm
    \rightskip=#3cm
}
{
    \par
}
\fi

\doendnotes{C}
\bigskip
\vfill

\clearpage

\footnotesize

\ifkorrekturansicht
  \lohead{\textsc{register}}
\fi

% theindex-Environment neu definieren ohne reledmac
\makeatletter
\renewenvironment{theindex}{%
  \ifkorrekturansicht
    \section*{\indexname}%
  \else
    \subsubsection*{Index der erwähnten Entitäten}%
  \fi
  \setlength{\parindent}{0pt}%
  \setlength{\parskip}{0pt plus 0.3pt}%
  \let\item\@idxitem
}{%
  \ifkorrekturansicht\clearpage\fi
}
\makeatother

\IfFileExists{\jobname-pw.ind}{\input{\jobname-pw.ind}}{}

% Quellenangabe nur in der Leseansicht
\ifkorrekturansicht\else
% Fallback-Definitionen, falls die .tex-Datei \titel etc. nicht gesetzt hat
\providecommand{\titel}{}
\providecommand{\editorInnen}{}
\providecommand{\dateiname}{\jobname}

\vspace{3cm}

\vfill

\footnotesize
\textsc{Quelle}: \titel. Herausgegeben von {\editorInnen}. In: \emph{Arthur Schnitzler: Briefwechsel mit Autorinnen und Autoren}.
 Digitale Edition, https://schnitzler-briefe.acdh.oeaw.ac.at/{\dateiname}.html (Stand \today)
\fi

\end{document}


      