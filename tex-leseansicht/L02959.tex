%% latex-korrekturansicht-vorspann.tex
%% Vorspann für die Korrekturansicht.
%% Lädt die gemeinsame Datei latex-vorspann.tex mit gesetztem Schalter.

\newif\ifkorrekturansicht
\korrekturansichttrue

\input{../tex-inputs/latex-vorspann}


\section[Arthur Schnitzler an Felix Salten, 9. 7. 1893]{L02959 Arthur Schnitzler an Felix Salten, 9. 7. 1893}
\nopagebreak\mylabel{L02959v}
\rehead{ }\normalsize\beginnumbering\briefempfaengerindex{Salten, Felix@\textsc{Salten, Felix}!zzzSchnitzler, Arthur@\emph{von Arthur Schnitzler}!1893-07-091@{9. 7. 1893}|(be}
\toendnotes[C]{\smallbreak\pagebreak[2]}\Standort{Wienbibliothek im Rathaus, ZPH 1681, 2.1.516.}
\physDesc{Kartenbrief, 446 Zeichen
\newline{}Handschrift: Bleistift, deutsche Kurrent
\newline{}Versand: Stempel: »\nobreak{}\oindex{Bad Ischl@\textbf{Bad Ischl}, \emph{P.PPL}|pwk}I{[}sch{]}l, 9/7 93, 9 A\nobreak{}«. Stempel: »\nobreak{}\oindex{IX., Alsergrund@\textbf{IX., Alsergrund}, \emph{A.ADM3}|pwk}Wien 9/1 66, 10. 7. 93, 9 V., Bestellt\nobreak{}«.  
\newline{}Ordnung: mit Bleistift von unbekannter Hand nummeriert:
                                    »80« }\toendnotes[C]{\smallbreak}\pstart{}{\pb}Hrn \textsc{Felix
                     Salten}\pend{}\pstart{}Wien\oindex{Wien@\textbf{Wien}, \emph{A.ADM2}|pw}\pend{}\pstart{}\textsc{IX Berggasse 13\oindex{Berggasse@\textbf{Berggasse}, \emph{Straße (K.STR)}|pw}}.\pend{}{\bigskip}\vspace{1em}
\pstart
           \noindent{}{\pb}Lieber Freund! – Mein \label{K_L02959-1v}\edtext{Stück\pwindex{Abschiedssouper@\emph{Abschiedssouper}|pwv}{ }hier\orgindex{Saisontheater Ischl@Saisontheater Ischl|pwv}{ }Freitag}{\lemma{\textnormal{\emph{Stück hier Freitag}}}\Cendnote{\textnormal{Siehe Arthur Schnitzler an Felix Salten, 5. 7. 1893.
               }}}\label{K_L02959-1}. \textsc{\textsc{\uline{Anatol\pwindex{Abschiedssouper@\emph{Abschiedssouper}|pwv}}}}{ }\textsc{Hoefer\pwindex{Hoefer, Emil 14.05.1864 – 01.05.1940@\textsc{Höfer, Emil} (14.05.1864 – 01.05.1940), \emph{Schauspieler/Schauspielerin}|pw}}, \textsc{\textsc{\uline{Max\pwindex{Abschiedssouper@\emph{Abschiedssouper}|pwv}}}}{ }\textsc{Jarno\pwindex{Jarno, Josef 24.08.1865 – 11.01.1932@\textsc{Jarno, Josef} (24.08.1865 – 11.01.1932), \emph{Theaterleiter/Theaterleiterin, Schauspieler/Schauspielerin}|pw}}. \uline{\textsc{Cora\pwindex{Frage an das Schicksal@\emph{Die Frage an das Schicksal}|pwv}}}{ }\textsc{Wreden\pwindex{Wreden, Grethe @\textsc{Wreden, Grethe}, \emph{Schauspieler/Schauspielerin}|pwv}}{ }\textsc{\uline{Annie\pwindex{Abschiedssouper@\emph{Abschiedssouper}|pwv}}}{ }\textsc{\substVorne{}\textsuperscript{M}\substDazwischen{}Gr\substHinten{}iebl\pwindex{Griebl, Marie 1872-02-27 – 1952-06-08@\textsc{Griebl, Marie} (1872-02-27 – 1952-06-08), \emph{Schauspieler/Schauspielerin, Sänger/Sängerin}|pw}} (Volkstheater\orgindex{Volkstheater@Volkstheater|pw}.) – \label{K_L02959-2v}\edtext{War beim Bezhauptm.\pwindex{Aichelburg-Labia, Franz von 08.04.1856 – 25.01.1927@\textsc{Aichelburg-Labia, Franz von} (08.04.1856 – 25.01.1927), \emph{Beamter/Beamte}|pwv}}{\lemma{\textnormal{\emph{War beim Bezhauptm.}}}\Cendnote{\textnormal{Siehe A. S.: \emph{Tagebuch}, 7. 7. 1893.
               }}}\label{K_L02959-2} in Gmunden\oindex{Gmunden@\textbf{Gmunden}, \emph{P.PPL}|pw} von wegen Cenſur.\pend
           
\pstart
           – Aus Wien\oindex{Wien@\textbf{Wien}, \emph{A.ADM2}|pw} von \label{K_L02959-3v}\edtext{Frl. \textsc{G.\pwindex{Gluemer, Marie 03.07.1867 – 16.11.1925@\textsc{Glümer, Marie} (03.07.1867 – 16.11.1925), \emph{Schauspieler/Schauspielerin}|pwu}}}{\lemma{\textnormal{\emph{Frl. G.}}}\Cendnote{\textnormal{Marie Glümer\pwindex{Gluemer, Marie 03.07.1867 – 16.11.1925@\textsc{Glümer, Marie} (03.07.1867 – 16.11.1925), \emph{Schauspieler/Schauspielerin}|pwk}, siehe A. S.: \emph{Tagebuch}, 8. 7. 1893 und 9. 7. 1893.
               }}}\label{K_L02959-3} Verzweiflungsſchreie entſetzlicher Art. Ich habe kein Wort geſchrieben. –\pend
           
\pstart
           – Ein paar Verſe weiter»gedichtet« an de\textcolor{gray}{m}{ }allegor. Gedicht\pwindex{Artifex@\emph{Artifex}|pwv}. – – Schreibe
               dieſe Zeilen bei Frau \textsc{Flegmann\pwindex{Flegmann, Bertha 27.05.1852 – 24.6.1933@\textsc{Flegmann, Bertha} (27.05.1852 – 24.6.1933), \emph{männliche Salonnière/Salonnière}|pw}}. – Eben ging \textsc{Brahms\pwindex{Brahms, Johannes 07.05.1833 – 03.04.1897@\textsc{Brahms, Johannes} (07.05.1833 – 03.04.1897), \emph{Komponist/Komponistin, Dirigent/Dirigentin, Pianist/Pianistin}|pw}} weg. – \textsc{Richard\pwindex{Beer-Hofmann, Richard 1866-07-11 – 1945-09-26@\textsc{Beer-Hofmann, Richard} (1866-07-11 – 1945-09-26), \emph{Schriftsteller/Schriftstellerin}|pw}} iſt da, grüßt Sie herzlich. Ihr \spacefill\mbox{Arthur}\pend
           \selectlanguage{ngerman}\endnumbering\briefempfaengerindex{Salten, Felix@\textsc{Salten, Felix}!zzzSchnitzler, Arthur@\emph{von Arthur Schnitzler}!1893-07-091@{9. 7. 1893}|)be}\mylabel{L02959h}  \normalsize

\doendnotes{C}
\bigskip
\vfill

\clearpage

\footnotesize

\lohead{\textsc{register}}

% Definiere theindex-Environment komplett neu ohne reledmac
\makeatletter
\renewenvironment{theindex}{%
  \section*{\indexname}%
  \setlength{\parindent}{0pt}%
  \setlength{\parskip}{0pt plus 0.3pt}%
  \let\item\@idxitem
}{%
  \clearpage
}
\makeatother

\IfFileExists{\jobname-pw.ind}{\input{\jobname-pw.ind}}{}

\end{document}

      