%% latex-leseansicht-vorspann.tex
%% Vorspann für die Leseansicht.
%% Lädt die gemeinsame Datei latex-vorspann.tex mit nicht gesetztem Schalter.

\newif\ifkorrekturansicht
\korrekturansichtfalse

\input{../tex-inputs/latex-vorspann}


\section[Arthur Schnitzler an Felix Salten, 9. 7. 1893]{L02959 Arthur Schnitzler an Felix Salten, 9. 7. 1893}
\nopagebreak\mylabel{L02959v}
\rehead{ }\normalsize\beginnumbering\briefempfaengerindex{Salten, Felix@\textsc{Salten, Felix}!zzzSchnitzler, Arthur@\emph{von Arthur Schnitzler}!1893-07-091@{9. 7. 1893}|(be}
\toendnotes[C]{\smallbreak\pagebreak[2]}
\correspDesc{Versand  durch Arthur Schnitzler am 9. 7. 1893 in Bad Ischl
\newline{}Erhalt  durch Felix Salten am 10. 7. 1893 in Wien}\toendnotes[C]{\smallbreak}
\Standort{Wienbibliothek im Rathaus, ZPH 1681, 2.1.516.}
\physDesc{Kartenbrief, 446 Zeichen
\newline{}Handschrift: Bleistift, deutsche Kurrent
\newline{}Versand: Stempel: »\nobreak{}\oindex{Bad Ischl@\textbf{Bad Ischl}|pwk}I{[}sch{]}l, 9/7 93, 9 A\nobreak{}«. Stempel: »\nobreak{}\oindex{IX., Alsergrund@\textbf{IX., Alsergrund}, \emph{Verwaltungsgebiet}|pwk}Wien 9/1 66, 10. 7. 93, 9 V., Bestellt\nobreak{}«.  
\newline{}Ordnung: mit Bleistift von unbekannter Hand nummeriert:
                                    »80« }\toendnotes[C]{\smallbreak}\pstart{}{\pb}Hrn \textsc{Felix
                     Salten}\pend{}\pstart{}Wien\oindex{Wien@\textbf{Wien}, \emph{Verwaltungsgebiet}|pw}\pend{}\pstart{}\textsc{IX Berggasse 13\oindex{Wien@\textbf{Wien}!IX., Alsergrund@\textbf{IX., Alsergrund}!Berggasse@\textbf{Berggasse}, \emph{Straße}|pw}}.\pend{}{\bigskip}\vspace{1em}
\pstart
           \noindent{}{\pb}Lieber Freund! – Mein \label{K_L02959-1v}\edtext{Stück\pwindex{Schnitzler, Arthur 15.\,5.\,1862 Wien – 21.\,10.\,1931 ebd.@\textsc{Schnitzler, Arthur} (15.\,5.\,1862 Wien – 21.\,10.\,1931 ebd.), \emph{Schriftsteller, Mediziner}!Abschiedssouper@\strich\emph{Abschiedssouper}|pwv}{ }hier\orgindex{Saisontheater Ischl@Saisontheater Ischl|pwv}{ }Freitag}{\lemma{\textnormal{\emph{Stück hier Freitag}}}\Cendnote{\textnormal{Siehe XXXX Auszeichnungsfehler: Dokument L02958 nicht gefunden.
               }}}\label{K_L02959-1}. \textsc{\textsc{\uline{Anatol\pwindex{Schnitzler, Arthur 15.\,5.\,1862 Wien – 21.\,10.\,1931 ebd.@\textsc{Schnitzler, Arthur} (15.\,5.\,1862 Wien – 21.\,10.\,1931 ebd.), \emph{Schriftsteller, Mediziner}!Abschiedssouper@\strich\emph{Abschiedssouper}|pwv}}}}{ }\textsc{Hoefer\pwindex{Höfer, Emil 14.\,5.\,1864 Wien – 1.\,5.\,1940 München@\textsc{Höfer, Emil} (14.\,5.\,1864 Wien – 1.\,5.\,1940 München), \emph{Schauspieler}|pw}}, \textsc{\textsc{\uline{Max\pwindex{Schnitzler, Arthur 15.\,5.\,1862 Wien – 21.\,10.\,1931 ebd.@\textsc{Schnitzler, Arthur} (15.\,5.\,1862 Wien – 21.\,10.\,1931 ebd.), \emph{Schriftsteller, Mediziner}!Abschiedssouper@\strich\emph{Abschiedssouper}|pwv}}}}{ }\textsc{Jarno\pwindex{Jarno, Josef 24.\,8.\,1865 Budapest – 11.\,1.\,1932 Wien@\textsc{Jarno, Josef} (24.\,8.\,1865 Budapest – 11.\,1.\,1932 Wien), \emph{Theaterleiter, Schauspieler}|pw}}. \uline{\textsc{Cora\pwindex{Schnitzler, Arthur 15.\,5.\,1862 Wien – 21.\,10.\,1931 ebd.@\textsc{Schnitzler, Arthur} (15.\,5.\,1862 Wien – 21.\,10.\,1931 ebd.), \emph{Schriftsteller, Mediziner}!Frage an das Schicksal@\strich\emph{Die Frage an das Schicksal}|pwv}}}{ }\textsc{Wreden\pwindex{Wreden, Grethe @\textsc{Wreden, Grethe}, \emph{Schauspielerin}|pwv}}{ }\textsc{\uline{Annie\pwindex{Schnitzler, Arthur 15.\,5.\,1862 Wien – 21.\,10.\,1931 ebd.@\textsc{Schnitzler, Arthur} (15.\,5.\,1862 Wien – 21.\,10.\,1931 ebd.), \emph{Schriftsteller, Mediziner}!Abschiedssouper@\strich\emph{Abschiedssouper}|pwv}}}{ }\textsc{\substVorne{}\textsuperscript{M}\substDazwischen{}Gr\substHinten{}iebl\pwindex{Griebl, Marie 27.\,2.\,1872 Baden bei Wien – 8.\,6.\,1952 Wien@\textsc{Griebl, Marie} (27.\,2.\,1872 Baden bei Wien – 8.\,6.\,1952 Wien), \emph{Schauspielerin, Sängerin}|pw}} (Volkstheater\orgindex{Volkstheater@Volkstheater|pw}.) – \label{K_L02959-2v}\edtext{War beim Bezhauptm.\pwindex{Aichelburg-Labia, Franz von 8.\,4.\,1856 Klagenfurt – 25.\,1.\,1927 Meran@\textsc{Aichelburg-Labia, Franz von} (8.\,4.\,1856 Klagenfurt – 25.\,1.\,1927 Meran), \emph{Beamter}|pwv}}{\lemma{\textnormal{\emph{War beim Bezhauptm.}}}\Cendnote{\textnormal{Siehe A. S.: \emph{Tagebuch}, 7. 7. 1893.
               }}}\label{K_L02959-2} in Gmunden\oindex{Gmunden@\textbf{Gmunden}|pw} von wegen Cenſur.\pend
           
\pstart
           – Aus Wien\oindex{Wien@\textbf{Wien}, \emph{Verwaltungsgebiet}|pw} von \label{K_L02959-3v}\edtext{Frl. \textsc{G.\pwindex{Glümer, Marie 3.\,7.\,1867 Wien – 16.\,11.\,1925 München@\textsc{Glümer, Marie} (3.\,7.\,1867 Wien – 16.\,11.\,1925 München), \emph{Schauspielerin}|pwu}}}{\lemma{\textnormal{\emph{Frl. G.}}}\Cendnote{\textnormal{Marie Glümer\pwindex{Glümer, Marie 3.\,7.\,1867 Wien – 16.\,11.\,1925 München@\textsc{Glümer, Marie} (3.\,7.\,1867 Wien – 16.\,11.\,1925 München), \emph{Schauspielerin}|pwk}, siehe A. S.: \emph{Tagebuch}, 8. 7. 1893 und 9. 7. 1893.
               }}}\label{K_L02959-3} Verzweiflungsſchreie entſetzlicher Art. Ich habe kein Wort geſchrieben. –\pend
           
\pstart
           – Ein paar Verſe weiter»gedichtet« an de\textcolor{gray}{m}{ }allegor. Gedicht\pwindex{Schnitzler, Arthur 15.\,5.\,1862 Wien – 21.\,10.\,1931 ebd.@\textsc{Schnitzler, Arthur} (15.\,5.\,1862 Wien – 21.\,10.\,1931 ebd.), \emph{Schriftsteller, Mediziner}!Artifex@\strich\emph{Artifex}|pwv}. – – Schreibe
               dieſe Zeilen bei Frau \textsc{Flegmann\pwindex{Flegmann, Bertha 27.\,5.\,1852 Dubrovsky, Polen – 24.\,6.\,1933 Bad Ischl@\textsc{Flegmann, Bertha} (27.\,5.\,1852 Dubrovsky, Polen – 24.\,6.\,1933 Bad Ischl), \emph{Salonnière}|pw}}. – Eben ging \textsc{Brahms\pwindex{Brahms, Johannes 7.\,5.\,1833 Hamburg – 3.\,4.\,1897 Wien@\textsc{Brahms, Johannes} (7.\,5.\,1833 Hamburg – 3.\,4.\,1897 Wien), \emph{Komponist, Dirigent, Pianist}|pw}} weg. – \textsc{Richard\pwindex{Beer-Hofmann, Richard 11.\,7.\,1866 Wien – 26.\,9.\,1945 New York City@\textsc{Beer-Hofmann, Richard} (11.\,7.\,1866 Wien – 26.\,9.\,1945 New York City), \emph{Schriftsteller}|pw}} iſt da, grüßt Sie herzlich. Ihr \spacefill\mbox{Arthur}\pend
           \selectlanguage{ngerman}\endnumbering\briefempfaengerindex{Salten, Felix@\textsc{Salten, Felix}!zzzSchnitzler, Arthur@\emph{von Arthur Schnitzler}!1893-07-091@{9. 7. 1893}|)be}\mylabel{L02959h}  \newcommand{\dateiname}{L02959}\newcommand{\titel}{Arthur Schnitzler an Felix Salten, 9. 7. 1893}\newcommand{\editorInnen}{Martin Anton Müller und Laura Untner}%% latex-leseansicht-abspann.tex
%% Abspann für die Leseansicht.
%% Der Schalter \ifkorrekturansicht ist bereits durch den Vorspann gesetzt.

%% latex-abspann.tex
%% Gemeinsamer Abspann für Korrekturansicht und Leseansicht.
%% Setzt den Schalter \ifkorrekturansicht voraus (gesetzt in den
%% einbindenden Dateien latex-korrekturansicht-abspann.tex bzw.
%% latex-leseansicht-abspann.tex).
%% ---------------------------------------------------------------

\normalsize

% Das esempio-Environment wird nur in der Leseansicht benötigt
\ifkorrekturansicht\else
\newenvironment{esempio}[3]%
{
    \vspace{1.5ex}
    \rlap{\underline{#1}}
    \par
    \setlength{\parindent}{0cm}
    \nopagebreak
    \leftskip=#2cm
    \rightskip=#3cm
}
{
    \par
}
\fi

\doendnotes{C}
\bigskip
\vfill

\clearpage

\footnotesize

\ifkorrekturansicht
  \lohead{\textsc{register}}
\fi

% theindex-Environment neu definieren ohne reledmac
\makeatletter
\renewenvironment{theindex}{%
  \ifkorrekturansicht
    \section*{\indexname}%
  \else
    \subsubsection*{Index der erwähnten Entitäten}%
  \fi
  \setlength{\parindent}{0pt}%
  \setlength{\parskip}{0pt plus 0.3pt}%
  \let\item\@idxitem
}{%
  \ifkorrekturansicht\clearpage\fi
}
\makeatother

\IfFileExists{\jobname-pw.ind}{\input{\jobname-pw.ind}}{}

% Quellenangabe nur in der Leseansicht
\ifkorrekturansicht\else
% Fallback-Definitionen, falls die .tex-Datei \titel etc. nicht gesetzt hat
\providecommand{\titel}{}
\providecommand{\editorInnen}{}
\providecommand{\dateiname}{\jobname}

\vspace{3cm}

\vfill

\footnotesize
\textsc{Quelle}: \titel. Herausgegeben von {\editorInnen}. In: \emph{Arthur Schnitzler: Briefwechsel mit Autorinnen und Autoren}.
 Digitale Edition, https://schnitzler-briefe.acdh.oeaw.ac.at/{\dateiname}.html (Stand \today)
\fi

\end{document}


