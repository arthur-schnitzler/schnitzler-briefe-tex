%% latex-leseansicht-vorspann.tex
%% Vorspann für die Leseansicht.
%% Lädt die gemeinsame Datei latex-vorspann.tex mit nicht gesetztem Schalter.

\newif\ifkorrekturansicht
\korrekturansichtfalse

\input{../tex-inputs/latex-vorspann}

\begin{center}
            \textcolor{red}{ENTWURF, NICHT FERTIG KORRIGIERT}
                      \end{center}
            
         
         \renewcommand{\erwaehntePersonen}{Personen: Franz von Aichelburg-Labia, Richard Beer-Hofmann, Johannes Brahms, Bertha Flegmann, Marie Glümer, Marie Griebl, Emil Höfer, Josef Jarno, Felix Salten, Grethe Wreden}
         \renewcommand{\erwaehnteInstitutionen}{Institutionen: Volkstheater}
         \renewcommand{\erwaehnteOrte}{Orte: Berggasse, Gmunden, Wien}
         \renewcommand{\erwaehnteWerke}{Werke: Abschiedssouper, Die Frage an das Schicksal}
               \section[Arthur Schnitzler an Felix Salten, 9. 7. 1893]{ Arthur Schnitzler an Felix Salten, 9. 7. 1893}\nopagebreak\mylabel{v}\rehead{ }\begin{ledgroupsized}[t]{13cm}\normalsize\beginnumbering \toendnotes[C]{\smallbreak\pagebreak[2]} \Standort{Wienbibliothek im Rathaus, ZPH 1681, 2.1.516.}
\physDesc{Kartenbrief
\newline{}Handschrift: Bleistift, deutsche Kurrent\newline{}Ordnung: mit Bleistift von unbekannter Hand Nummerierung der ungeraden Seiten:
                                 »80« }\toendnotes[C]{\smallbreak}\pstart{}{\pb}Hrn \textsc{Felix
                     Salten}\pend{}\pstart{}Wien\oindex{Wien@\textbf{Wien}|pw}\pend{}\pstart{}\textsc{IX Berggasse 13\oindex{Berggasse@\textbf{Berggasse}|pw}}.\pend{}{\bigskip}\pstart
           \noindent{}{\pb}Lieber Freund! – Mein Stück\pwindex{Schnitzler, Arthur 15.05.1862 – 21.10.1931@\textsc{Schnitzler, Arthur} (15.05.1862 – 21.10.1931), \emph{Schriftsteller, Mediziner}!Abschiedssouper1892@\strich\emph{Abschiedssouper} {[}1892{]}|pwv} hier
               Freitag. \textsc{\textsc{\uline{Anatol\pwindex{Schnitzler, Arthur 15.05.1862 – 21.10.1931@\textsc{Schnitzler, Arthur} (15.05.1862 – 21.10.1931), \emph{Schriftsteller, Mediziner}!Abschiedssouper1892@\strich\emph{Abschiedssouper} {[}1892{]}|pw}}}}{ }\textsc{Hoefer\pwindex{Hoefer, Emil 14.05.1864 – 01.05.1940@\textsc{Höfer, Emil} (14.05.1864 – 01.05.1940), \emph{Schauspieler}|pw}}, \textsc{\textsc{\uline{Max \pwindex{Schnitzler, Arthur 15.05.1862 – 21.10.1931@\textsc{Schnitzler, Arthur} (15.05.1862 – 21.10.1931), \emph{Schriftsteller, Mediziner}!Abschiedssouper1892@\strich\emph{Abschiedssouper} {[}1892{]}|pw}}}}\textsc{Jarno\pwindex{Jarno, Josef 24.08.1865 – 11.01.1932@\textsc{Jarno, Josef} (24.08.1865 – 11.01.1932), \emph{Theaterleiter, Schauspieler}|pw}}. \uline{\textsc{Cora\pwindex{Schnitzler, Arthur 15.05.1862 – 21.10.1931@\textsc{Schnitzler, Arthur} (15.05.1862 – 21.10.1931), \emph{Schriftsteller, Mediziner}!Frage an das Schicksal01. 05. 1890@\strich\emph{Die Frage an das Schicksal} {[}01. 05. 1890{]}|pw}}}{ }\textsc{Wreden\pwindex{Wreden, Grethe @\textsc{Wreden, Grethe}, \emph{Schauspielerin}|pw}}{ }\textsc{\uline{Annie \pwindex{Schnitzler, Arthur 15.05.1862 – 21.10.1931@\textsc{Schnitzler, Arthur} (15.05.1862 – 21.10.1931), \emph{Schriftsteller, Mediziner}!Abschiedssouper1892@\strich\emph{Abschiedssouper} {[}1892{]}|pw}}}{ }\textsc{Griebl\pwindex{Griebl, Marie 1872-02-27 – 1952-06-08@\textsc{Griebl, Marie} (1872-02-27 – 1952-06-08), \emph{Schauspielerin, Sängerin}|pw}} (Volkstheater\orgindex{Volkstheater@Volkstheater|pw}.) – \label{K_L02959-22v}\edtext{War beim Bezhaupt.\pwindex{Aichelburg-Labia, Franz von 08.04.1856 – 25.01.1927@\textsc{Aichelburg-Labia, Franz von} (08.04.1856 – 25.01.1927), \emph{Beamter}|pwv}}{\lemma{\textnormal{\emph{War beim Bezhaupt.}}}\Cendnote{\textnormal{siehe A. S.: \emph{Tagebuch}, 7. 7. 1893}}}\label{K_L02959-22h} in Gmunden\oindex{Gmunden@\textbf{Gmunden}|pw} von wegen Cenſur. – Aus Wien\oindex{Wien@\textbf{Wien}|pw} von Frl. \textsc{G.\pwindex{Gluemer, Marie 03.07.1867 – 16.11.1925@\textsc{Glümer, Marie} (03.07.1867 – 16.11.1925), \emph{Schauspielerin}|pwu}} Verzweiflungsſchreie entſetzlicher Art. Ich habe kein Wort geſchrieben.– – Ein
               paar Verſe weiter»gedichtet« an de\textcolor{gray}{m}{ }allegor. Gedicht\textcolor{red}{\textsuperscript{\textbf{KEY}}}. – – Schreibe dieſe Zeilen bei Frau \textsc{Flegmann\pwindex{Flegmann, Bertha 27.05.1852 – 24.6.1933@\textsc{Flegmann, Bertha} (27.05.1852 – 24.6.1933), \emph{Salonnière}|pw}}.– Eben ging \textsc{Brahms\pwindex{Brahms, Johannes 07.05.1833 – 03.04.1897@\textsc{Brahms, Johannes} (07.05.1833 – 03.04.1897), \emph{Komponist, Dirigent, Pianist}|pw}} weg. – \textsc{Richard\pwindex{Beer-Hofmann, Richard 1866-07-11 – 1945-09-26@\textsc{Beer-Hofmann, Richard} (1866-07-11 – 1945-09-26), \emph{Schriftsteller}|pw}} iſt da, grüßt Sie herzlich. Ihr \spacefill\mbox{Arthur}\pend
           
         
         \endnumbering\mylabel{h}\end{ledgroupsized}\begin{anhang}\end{anhang}\newcommand{\dateiname}{L02959}\newcommand{\titel}{Arthur Schnitzler an Felix Salten, 9. 7. 1893}\newcommand{\editorInnen}{Martin Anton Müller und Laura Untner}%% latex-leseansicht-abspann.tex
%% Abspann für die Leseansicht.
%% Der Schalter \ifkorrekturansicht ist bereits durch den Vorspann gesetzt.

%% latex-abspann.tex
%% Gemeinsamer Abspann für Korrekturansicht und Leseansicht.
%% Setzt den Schalter \ifkorrekturansicht voraus (gesetzt in den
%% einbindenden Dateien latex-korrekturansicht-abspann.tex bzw.
%% latex-leseansicht-abspann.tex).
%% ---------------------------------------------------------------

\normalsize

% Das esempio-Environment wird nur in der Leseansicht benötigt
\ifkorrekturansicht\else
\newenvironment{esempio}[3]%
{
    \vspace{1.5ex}
    \rlap{\underline{#1}}
    \par
    \setlength{\parindent}{0cm}
    \nopagebreak
    \leftskip=#2cm
    \rightskip=#3cm
}
{
    \par
}
\fi

\doendnotes{C}
\bigskip
\vfill

\clearpage

\footnotesize

\ifkorrekturansicht
  \lohead{\textsc{register}}
\fi

% theindex-Environment neu definieren ohne reledmac
\makeatletter
\renewenvironment{theindex}{%
  \ifkorrekturansicht
    \section*{\indexname}%
  \else
    \subsubsection*{Index der erwähnten Entitäten}%
  \fi
  \setlength{\parindent}{0pt}%
  \setlength{\parskip}{0pt plus 0.3pt}%
  \let\item\@idxitem
}{%
  \ifkorrekturansicht\clearpage\fi
}
\makeatother

\IfFileExists{\jobname-pw.ind}{\input{\jobname-pw.ind}}{}

% Quellenangabe nur in der Leseansicht
\ifkorrekturansicht\else
% Fallback-Definitionen, falls die .tex-Datei \titel etc. nicht gesetzt hat
\providecommand{\titel}{}
\providecommand{\editorInnen}{}
\providecommand{\dateiname}{\jobname}

\vspace{3cm}

\vfill

\footnotesize
\textsc{Quelle}: \titel. Herausgegeben von {\editorInnen}. In: \emph{Arthur Schnitzler: Briefwechsel mit Autorinnen und Autoren}.
 Digitale Edition, https://schnitzler-briefe.acdh.oeaw.ac.at/{\dateiname}.html (Stand \today)
\fi

\end{document}


      