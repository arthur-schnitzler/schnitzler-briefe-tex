%% latex-leseansicht-vorspann.tex
%% Vorspann für die Leseansicht.
%% Lädt die gemeinsame Datei latex-vorspann.tex mit nicht gesetztem Schalter.

\newif\ifkorrekturansicht
\korrekturansichtfalse

\input{../tex-inputs/latex-vorspann}


\section[Arthur Schnitzler an Richard Beer-Hofmann, 26. 9. 1895]{L00494 Arthur Schnitzler an Richard Beer-Hofmann, 26. 9. 1895}
\nopagebreak\mylabel{L00494v}
\rehead{ }\normalsize\beginnumbering\briefempfaengerindex{Beer-Hofmann, Richard@\textsc{Beer-Hofmann, Richard}!zzzSchnitzler, Arthur@\emph{von Arthur Schnitzler}!1895-09-261@{26. 9. 1895}|(be}
\toendnotes[C]{\smallbreak\pagebreak[2]}
\correspDesc{Versand  durch Arthur Schnitzler am 26. 9. 1895 in Wien
\newline{}Weiterleitung  am 28. 9. 1895 in Gardone Riviera
\newline{}Erhalt  durch Richard Beer-Hofmann am 1. 10. 1895 in Wien}\toendnotes[C]{\smallbreak}
\Standort{YCGL, MSS 31.}
\physDesc{Brief, 2 Blätter, 7 Seiten, Kuvert, 1904 Zeichen
\newline{}Handschrift: 1) schwarze Tinte, deutsche Kurrent (\noindent{}Umschlag)\hspace{1em}2) Bleistift, deutsche Kurrent\hspace{1em}
\newline{}Versand: 1) Stempel: »\nobreak{}\oindex{Wien@\textbf{Wien}, \emph{Verwaltungsgebiet}|pwk}Wien, 26. 9. 95, 7–8\nobreak{}«.   2) Stempel: »\nobreak{}\oindex{Gardone Riviera@\textbf{Gardone Riviera}, \emph{Verwaltungsgebiet}|pwk}Gardone Riva, 28 9 95\nobreak{}«.  3) Stempel: »\nobreak{}\oindex{I., Innere Stadt@\textbf{I., Innere Stadt}, \emph{Verwaltungsgebiet}|pwk}Wien 1/1, 1/10 95, 8–9½ V., Bestellt\nobreak{}«.  4) mit blauer Tinte von unbekannter Hand die Nachsendeadresse
                                 vermerkt: »I Wollzeile 15\oindex{Wien@\textbf{Wien}!I., Innere Stadt@\textbf{I., Innere Stadt}!Wollzeile 15 (»Berthahof«)@\textbf{Wollzeile 15 (»Berthahof«)}, \emph{Wohngebäude}|pw}. Wien I.\oindex{Wien@\textbf{Wien}, \emph{Verwaltungsgebiet}|pw}«}
\buchAbdrucke{\weitereDrucke{Arthur Schnitzler, Richard Beer-Hofmann: \emph{Briefwechsel 1891–1931}. Herausgegeben von Konstanze Fliedl. Wien, Zürich: \emph{Europaverlag} 1992, S. 85–86.} }\toendnotes[C]{\smallbreak}\pstart{}{\pb}Herrn \textsc{Dr. Richard
                     Beer-Hofmann}\pend{}\pstart{}\textsc{Gardone}\oindex{Gardone Riviera@\textbf{Gardone Riviera}, \emph{Verwaltungsgebiet}|pw}\pend{}\pstart{}\textsc{am Gardasee\oindex{Lago di Garda@\textbf{Lago di Garda}, \emph{See}|pw}}\pend{}\pstart{}\textsc{Italien}\oindex{Italien@\textbf{Italien}|pw}\pend{}{\bigskip}\vspace{1em}
\pstart
           \raggedleft{}{\pb}Wien\oindex{Wien@\textbf{Wien}, \emph{Verwaltungsgebiet}|pw}{ }26. 9. 95\pend
           \vspace{0.5em}
\pstart
           Lieber Richard, heute kam zugleich Ihre Karte vom 23.
               und Ihr Brief vom 24. an. Ich{ }ſende also dieſe Zeilen hier nach Gardone\oindex{Gardone Riviera@\textbf{Gardone Riviera}, \emph{Verwaltungsgebiet}|pw}; warum{ }ſchreiben Sie nicht, wohin Sie von
               da aus gehen? Eben hat mir die Tragödin\pwindex{Sandrock, Adele 19.\,8.\,1863 Rotterdam – 30.\,8.\,1937 Berlin@\textsc{Sandrock, Adele} (19.\,8.\,1863 Rotterdam – 30.\,8.\,1937 Berlin), \emph{Schauspielerin}|pwv} telephonirt, es war heut Probe von Liebelei\pwindex{Schnitzler, Arthur 15.\,5.\,1862 Wien – 21.\,10.\,1931 ebd.@\textsc{Schnitzler, Arthur} (15.\,5.\,1862 Wien – 21.\,10.\,1931 ebd.), \emph{Schriftsteller, Mediziner}!Liebelei. Schauspiel in drei Akten@\strich\emph{Liebelei. Schauspiel in drei Akten}|pw} (ſtatt Don \textsc{Carlos}\pwindex{\textcolor{red}{\textsuperscript{XXXX indx1}}!Dom Karlos, Infant von Spanien@\strich\emph{Dom Karlos, Infant von Spanien}|pw}) von der ich nichts wußte, und{ }ſie überbot{ }ſich{ }ſelbſt an Liebenswürdigkeiten
               für mich, mein Stück\pwindex{Schnitzler, Arthur 15.\,5.\,1862 Wien – 21.\,10.\,1931 ebd.@\textsc{Schnitzler, Arthur} (15.\,5.\,1862 Wien – 21.\,10.\,1931 ebd.), \emph{Schriftsteller, Mediziner}!Liebelei. Schauspiel in drei Akten@\strich\emph{Liebelei. Schauspiel in drei Akten}|pwv} und ihre
               Rolle. {\pb}Sie hat heute auf der Probe einen
               »großartigen« Erfolg gehabt, und na, und{ }ſo weiter. Ich denke, die \textsc{Premiere\pwindex{Schnitzler, Arthur 15.\,5.\,1862 Wien – 21.\,10.\,1931 ebd.@\textsc{Schnitzler, Arthur} (15.\,5.\,1862 Wien – 21.\,10.\,1931 ebd.), \emph{Schriftsteller, Mediziner}!Liebelei. Schauspiel in drei Akten@\strich\emph{Liebelei. Schauspiel in drei Akten}|pwv}} wird am 7. oder 8. oder 9.{ }ſein. Dazu gibt man \textsc{Giacosa}\pwindex{Giacosa, Giuseppe 21.\,10.\,1847 Colleretto Giacosa – 2.\,9.\,1906 ebd.@\textsc{Giacosa, Giuseppe} (21.\,10.\,1847 Colleretto Giacosa – 2.\,9.\,1906 ebd.), \emph{Schriftsteller}|pw}, Rechte der Seele\pwindex{Giacosa, Giuseppe 21.\,10.\,1847 Colleretto Giacosa – 2.\,9.\,1906 ebd.@\textsc{Giacosa, Giuseppe} (21.\,10.\,1847 Colleretto Giacosa – 2.\,9.\,1906 ebd.), \emph{Schriftsteller}!Rechte der Seele. Schauspiel in einem Act@\strich\emph{Rechte der Seele. Schauspiel in einem Act}|pw}. Für einen guten Sitz{ }ſoll geſorgt sein. –\pend
           
\pstart
           Allmälig hab ich zu arbeiten angefangen. Begonnen hab ich damit, daſs ich ein Stück\pwindex{Schnitzler, Arthur 15.\,5.\,1862 Wien – 21.\,10.\,1931 ebd.@\textsc{Schnitzler, Arthur} (15.\,5.\,1862 Wien – 21.\,10.\,1931 ebd.), \emph{Schriftsteller, Mediziner}!Portrait@\strich\emph{Das Portrait}|pwv} (Einakter) in Verſen, {\pb}den ich vorigen Winter{ }ſchrieb, in mein\introOben{}em\introOben{}{ }\substVorne{}\textsuperscript{\textcolor{gray}{Kästchen}}\substDazwischen{}Schreibtiſch\substHinten{} vergrub, – wo e\substVorne{}\textsuperscript{s}\substDazwischen{}r\substHinten{} am tiefſten iſt. Ich hab manchmal die{ }ſtarke Empfindung, daſs mir nie mehr
               etwas gelingen wird – wie \textsc{Ibsen}\pwindex{Ibsen, Henrik 20.\,3.\,1828 Skien – 23.\,5.\,1906 Oslo@\textsc{Ibsen, Henrik} (20.\,3.\,1828 Skien – 23.\,5.\,1906 Oslo), \emph{Schriftsteller}|pw} und – \textsc{Paul Lindau}\pwindex{Lindau, Paul 3.\,6.\,1839 Magdeburg – 31.\,1.\,1919 Berlin@\textsc{Lindau, Paul} (3.\,6.\,1839 Magdeburg – 31.\,1.\,1919 Berlin), \emph{Schriftsteller, Kritiker, Theaterleiter}|pw}. –\pend
           
\pstart
           Da die Läufigkeit der Frauen manchmal angenehm war, haben Sie wohl auch was »erlebt«
                  {\dots} wenigſtens {\pb}Anfänge.
               Da drin{ }ſtecken ja die ganzen Erlebniſſe, die Schlüſſe{ }ſind ja dieſelben. (Anatol\pwindex{Schnitzler, Arthur 15.\,5.\,1862 Wien – 21.\,10.\,1931 ebd.@\textsc{Schnitzler, Arthur} (15.\,5.\,1862 Wien – 21.\,10.\,1931 ebd.), \emph{Schriftsteller, Mediziner}!Anatol@\strich\emph{Anatol}|pwv} reibt{ }ſich die Augen. Er{ }ſchlu{\geminationm}ert{ }ſofort wieder ein. Bald{ }ſchläfſt du {\dots}{ }\textsc{etc}.{ }ſiehe \textsc{Hänsel u Grethel}\pwindex{\textcolor{red}{\textsuperscript{XXXX indx1}}!Hänsel und Gretel. Märchenspiel in drei Bildern@\strich\emph{Hänsel und Gretel. Märchenspiel in drei Bildern}|pw}) Ich beneide Sie{ }ſo um die Natur. Es iſt{ }ſo{ }ſchön jetzt und ich möchte ganz wo
               anders{ }ſein. Neulich war ich {\pb}in der Brühl\oindex{Brühl@\textbf{Brühl}, \emph{Tal}|pw}. Tini\pwindex{Kepert, Christine 17.\,11.\,1875 – 3.\,2.\,1971 Wien@\textsc{Kepert, Christine} (17.\,11.\,1875 – 3.\,2.\,1971 Wien), \emph{Gastwirtin}|pw} iſt{ }ſehr{ }ſtolz geworden. Auch war ein Jägerlieutenant draußen. Dem Hugo\pwindex{Hofmannsthal, Hugo von 1.\,2.\,1874 Wien – 15.\,7.\,1929 Rodaun@\textsc{Hofmannsthal, Hugo von} (1.\,2.\,1874 Wien – 15.\,7.\,1929 Rodaun), \emph{Schriftsteller}|pw} hab ich Ihre Kränkung ausgerichtet, er iſt
               auch gekränkt. –\pend
           
\pstart
           Wie weit iſt der Liebling der Götter\pwindex{Beer-Hofmann, Richard 11.\,7.\,1866 Wien – 26.\,9.\,1945 New York City@\textsc{Beer-Hofmann, Richard} (11.\,7.\,1866 Wien – 26.\,9.\,1945 New York City), \emph{Schriftsteller}!Tod Georgs@\strich\emph{Der Tod Georgs}|pw} und
               hoffentlich vieler Menschen? –\pend
           
\pstart
           {\pb}Leben Sie wohl und{ }ſchreiben Sie mir.
                  Samſtag werde ich wohl das Datum der \textsc{Prém.}\pwindex{Schnitzler, Arthur 15.\,5.\,1862 Wien – 21.\,10.\,1931 ebd.@\textsc{Schnitzler, Arthur} (15.\,5.\,1862 Wien – 21.\,10.\,1931 ebd.), \emph{Schriftsteller, Mediziner}!Liebelei. Schauspiel in drei Akten@\strich\emph{Liebelei. Schauspiel in drei Akten}|pwv}{ }\textsc{def}\substVorne{}\textsuperscript{.}\substDazwischen{}\textsc{initiv}\substHinten{} kennen.\pend
           
\pstart
           Man erkundigt{ }ſich i{\geminationm}erfort und allſeitig nach Ihnen,
               was keine Broſamen,{ }ſondern naive Wahrheiten {\pb}ſind.
               Warum{ }ſoll ichs Ihnen denn verſchweigen? Dazu bin ich nicht 999gradig genug.\pend
           
\pstart
           Herzlichen Gruſs, ich freu mich{ }ſchon{ }ſehr auf Sie.{\\[\baselineskip]}Ihr
                  \spacefill\mbox{Arthur.}\pend
           \leftskip=0em{}\selectlanguage{ngerman}\endnumbering\briefempfaengerindex{Beer-Hofmann, Richard@\textsc{Beer-Hofmann, Richard}!zzzSchnitzler, Arthur@\emph{von Arthur Schnitzler}!1895-09-261@{26. 9. 1895}|)be}\mylabel{L00494h}  \newcommand{\dateiname}{L00494}\newcommand{\titel}{Arthur Schnitzler an Richard Beer-Hofmann, 26. 9. 1895}\newcommand{\editorInnen}{Martin Anton Müller und Gerd-Hermann Susen}%% latex-leseansicht-abspann.tex
%% Abspann für die Leseansicht.
%% Der Schalter \ifkorrekturansicht ist bereits durch den Vorspann gesetzt.

%% latex-abspann.tex
%% Gemeinsamer Abspann für Korrekturansicht und Leseansicht.
%% Setzt den Schalter \ifkorrekturansicht voraus (gesetzt in den
%% einbindenden Dateien latex-korrekturansicht-abspann.tex bzw.
%% latex-leseansicht-abspann.tex).
%% ---------------------------------------------------------------

\normalsize

% Das esempio-Environment wird nur in der Leseansicht benötigt
\ifkorrekturansicht\else
\newenvironment{esempio}[3]%
{
    \vspace{1.5ex}
    \rlap{\underline{#1}}
    \par
    \setlength{\parindent}{0cm}
    \nopagebreak
    \leftskip=#2cm
    \rightskip=#3cm
}
{
    \par
}
\fi

\doendnotes{C}
\bigskip
\vfill

\clearpage

\footnotesize

\ifkorrekturansicht
  \lohead{\textsc{register}}
\fi

% theindex-Environment neu definieren ohne reledmac
\makeatletter
\renewenvironment{theindex}{%
  \ifkorrekturansicht
    \section*{\indexname}%
  \else
    \subsubsection*{Index der erwähnten Entitäten}%
  \fi
  \setlength{\parindent}{0pt}%
  \setlength{\parskip}{0pt plus 0.3pt}%
  \let\item\@idxitem
}{%
  \ifkorrekturansicht\clearpage\fi
}
\makeatother

\IfFileExists{\jobname-pw.ind}{\input{\jobname-pw.ind}}{}

% Quellenangabe nur in der Leseansicht
\ifkorrekturansicht\else
% Fallback-Definitionen, falls die .tex-Datei \titel etc. nicht gesetzt hat
\providecommand{\titel}{}
\providecommand{\editorInnen}{}
\providecommand{\dateiname}{\jobname}

\vspace{3cm}

\vfill

\footnotesize
\textsc{Quelle}: \titel. Herausgegeben von {\editorInnen}. In: \emph{Arthur Schnitzler: Briefwechsel mit Autorinnen und Autoren}.
 Digitale Edition, https://schnitzler-briefe.acdh.oeaw.ac.at/{\dateiname}.html (Stand \today)
\fi

\end{document}


