%% latex-korrekturansicht-vorspann.tex
%% Vorspann für die Korrekturansicht.
%% Lädt die gemeinsame Datei latex-vorspann.tex mit gesetztem Schalter.

\newif\ifkorrekturansicht
\korrekturansichttrue

\input{../tex-inputs/latex-vorspann}


\section[Arthur Schnitzler an Richard Beer-Hofmann, 26. 9. 1895]{L00494 Arthur Schnitzler an Richard Beer-Hofmann, 26. 9. 1895}
\nopagebreak\mylabel{L00494v}
\rehead{ }\normalsize\beginnumbering\briefempfaengerindex{Beer-Hofmann, Richard@\textsc{Beer-Hofmann, Richard}!zzzSchnitzler, Arthur@\emph{von Arthur Schnitzler}!1895-09-261@{26. 9. 1895}|(be}
\toendnotes[C]{\smallbreak\pagebreak[2]}\Standort{YCGL, MSS 31.}
\physDesc{Brief, 2 Blätter, 7 Seiten, Umschlag, 1904 Zeichen
\newline{}Handschrift: 1) schwarze Tinte, deutsche Kurrent (\noindent{}Umschlag)\hspace{1em}2) Bleistift, deutsche Kurrent\hspace{1em}
\newline{}Versand: 1) Stempel: »\nobreak{}Wien, 26. 9. 95, 7–8\nobreak{}«.   2) Stempel: »\nobreak{}\oindex{Gardone Riviera@\textbf{Gardone Riviera}, \emph{A.ADM3}|pwk}Gardone Riva, 28 9 95\nobreak{}«.  3) Stempel: »\nobreak{}\oindex{I., Innere Stadt@\textbf{I., Innere Stadt}, \emph{A.ADM3}|pwk}Wien 1/1, 1/10 95, 8–9½ V., Bestellt\nobreak{}«.  4) mit blauer Tinte von unbekannter Hand die Nachsendeadresse
                                 vermerkt: »I Wollzeile 15\oindex{Wollzeile@\textbf{Wollzeile}, \emph{Straße (K.STR)}|pw}. Wien I.\oindex{Wien@\textbf{Wien}, \emph{A.ADM2}|pw}«}
\buchAbdrucke{\weitereDrucke{Arthur Schnitzler, Richard Beer-Hofmann: \emph{Briefwechsel 1891–1931}. Wien, Zürich: \emph{Europaverlag} 1992, S. 85–86.} }\toendnotes[C]{\smallbreak}\pstart{}{\pb}Herrn \textsc{Dr. Richard
                     Beer-Hofmann}\pend{}\pstart{}\textsc{Gardone}\oindex{Gardone Riviera@\textbf{Gardone Riviera}, \emph{A.ADM3}|pw}\pend{}\pstart{}\textsc{am Gardasee\oindex{Lago di Garda@\textbf{Lago di Garda}, \emph{See (N.SEE)}|pw}}\pend{}\pstart{}\textsc{Italien}\oindex{Italien@\textbf{Italien}, \emph{A.PCLI}|pw}\pend{}{\bigskip}\vspace{1em}
\pstart
           \raggedleft{}{\pb}Wien\oindex{Wien@\textbf{Wien}, \emph{A.ADM2}|pw}{ }26. 9. 95\pend
           \vspace{0.5em}
\pstart
           Lieber Richard, heute kam zugleich Ihre Karte vom 23.
               und Ihr Brief vom 24. an. Ich ſende also dieſe Zeilen hier nach Gardone\oindex{Gardone Riviera@\textbf{Gardone Riviera}, \emph{A.ADM3}|pw}; warum ſchreiben Sie nicht, wohin Sie von
               da aus gehen? Eben hat mir die Tragödin\pwindex{Sandrock, Adele 1863-08-19 – 1937-08-30@\textsc{Sandrock, Adele} (1863-08-19 – 1937-08-30), \emph{Schauspieler/Schauspielerin}|pwv} telephonirt, es war heut Probe von Liebelei\pwindex{Liebelei. Schauspiel in drei Akten@\emph{Liebelei. Schauspiel in drei Akten}|pw} (ſtatt Don \textsc{Carlos}\pwindex{Don Karlos, Infant von Spanien@\emph{Don Karlos, Infant von Spanien}|pw}) von der ich nichts wußte, und ſie überbot ſich ſelbſt an Liebenswürdigkeiten
               für mich, mein Stück\pwindex{Liebelei. Schauspiel in drei Akten@\emph{Liebelei. Schauspiel in drei Akten}|pwv} und ihre
               Rolle. {\pb}Sie hat heute auf der Probe einen
               »großartigen« Erfolg gehabt, und na, und ſo weiter. Ich denke, die \textsc{Premiere\pwindex{Liebelei. Schauspiel in drei Akten@\emph{Liebelei. Schauspiel in drei Akten}|pwv}} wird am 7. oder 8. oder 9.{ }ſein. Dazu gibt man \textsc{Giacosa}\pwindex{Giacosa, Giuseppe 21.10.1847 – 02.09.1906@\textsc{Giacosa, Giuseppe} (21.10.1847 – 02.09.1906), \emph{Schriftsteller/Schriftstellerin}|pw}, Rechte der Seele\pwindex{Rechte der Seele. Schauspiel in einem Act@\emph{Rechte der Seele. Schauspiel in einem Act}|pw}. Für einen guten Sitz
               ſoll geſorgt sein. –\pend
           
\pstart
           Allmälig hab ich zu arbeiten angefangen. Begonnen hab ich damit, daſs ich ein Stück\pwindex{Portrait@\emph{Das Portrait}|pwv} (Einakter) in Verſen, {\pb}den ich vorigen Winter ſchrieb, in mein\introOben{}em\introOben{}{ }\substVorne{}\textsuperscript{\textcolor{gray}{Kästchen}}\substDazwischen{}Schreibtiſch\substHinten{} vergrub, – wo e\substVorne{}\textsuperscript{s}\substDazwischen{}r\substHinten{} am tiefſten iſt. Ich hab manchmal die ſtarke Empfindung, daſs mir nie mehr
               etwas gelingen wird – wie \textsc{Ibsen}\pwindex{Ibsen, Henrik 20.03.1828 – 23.05.1906@\textsc{Ibsen, Henrik} (20.03.1828 – 23.05.1906), \emph{Schriftsteller/Schriftstellerin}|pw} und – \textsc{Paul Lindau}\pwindex{Lindau, Paul 03.06.1839 – 31.01.1919@\textsc{Lindau, Paul} (03.06.1839 – 31.01.1919), \emph{Schriftsteller/Schriftstellerin, Kritiker/Kritikerin, Theaterleiter/Theaterleiterin}|pw}. –\pend
           
\pstart
           Da die Läufigkeit der Frauen manchmal angenehm war, haben Sie wohl auch was »erlebt«
                  {\dots} wenigſtens {\pb}Anfänge.
               Da drin ſtecken ja die ganzen Erlebniſſe, die Schlüſſe ſind ja dieſelben. (Anatol\pwindex{Anatol@\emph{Anatol}|pwv} reibt ſich die Augen. Er
                  ſchlu{\geminationm}ert ſofort wieder ein. Bald ſchläfſt du {\dots}{ }\textsc{etc}. ſiehe \textsc{Hänsel u Grethel}\pwindex{Haensel und Gretel. Maerchenspiel in drei Bildern@\emph{Hänsel und Gretel. Märchenspiel in drei Bildern}|pw}) Ich beneide Sie ſo um die Natur. Es iſt ſo ſchön jetzt und ich möchte ganz wo
               anders ſein. Neulich war ich {\pb}in der Brühl\oindex{Bruehl@\textbf{Brühl}, \emph{Tal (N.TAL)}|pw}. Tini\pwindex{Schoenberger, Christine 1875-11-17 – 1971-02-03@\textsc{Schönberger, Christine} (1875-11-17 – 1971-02-03), \emph{Gastwirt/Gastwirtin}|pw} iſt ſehr ſtolz geworden. Auch war ein Jägerlieutenant draußen. Dem Hugo\pwindex{Hofmannsthal, Hugo von 1874-02-01 – 1929-07-15@\textsc{Hofmannsthal, Hugo von} (1874-02-01 – 1929-07-15), \emph{Schriftsteller/Schriftstellerin}|pw} hab ich Ihre Kränkung ausgerichtet, er iſt
               auch gekränkt. –\pend
           
\pstart
           Wie weit iſt der Liebling der Götter\pwindex{Tod Georgs@\emph{Der Tod Georgs}|pw} und
               hoffentlich vieler Menschen? – \pend
           
\pstart
           {\pb}Leben Sie wohl und ſchreiben Sie mir.
                  Samſtag werde ich wohl das Datum der \textsc{Prém.}\pwindex{Liebelei. Schauspiel in drei Akten@\emph{Liebelei. Schauspiel in drei Akten}|pwv}{ }\textsc{def}\substVorne{}\textsuperscript{.}\substDazwischen{}\textsc{initiv}\substHinten{} kennen.\pend
           
\pstart
           Man erkundigt ſich i{\geminationm}erfort und allſeitig nach Ihnen,
               was keine Broſamen, ſondern naive Wahrheiten {\pb}ſind.
               Warum ſoll ichs Ihnen denn verſchweigen? Dazu bin ich nicht 999gradig genug.\pend
           
\pstart
           Herzlichen Gruſs, ich freu mich ſchon ſehr auf Sie.{\\[\baselineskip]}Ihr
                  \spacefill\mbox{Arthur.}\pend
           \leftskip=0em{}\selectlanguage{ngerman}\endnumbering\briefempfaengerindex{Beer-Hofmann, Richard@\textsc{Beer-Hofmann, Richard}!zzzSchnitzler, Arthur@\emph{von Arthur Schnitzler}!1895-09-261@{26. 9. 1895}|)be}\mylabel{L00494h}  \normalsize

\doendnotes{C}
\bigskip
\vfill

\clearpage

\footnotesize

\lohead{\textsc{register}}

% Definiere theindex-Environment komplett neu ohne reledmac
\makeatletter
\renewenvironment{theindex}{%
  \section*{\indexname}%
  \setlength{\parindent}{0pt}%
  \setlength{\parskip}{0pt plus 0.3pt}%
  \let\item\@idxitem
}{%
  \clearpage
}
\makeatother

\IfFileExists{\jobname-pw.ind}{\input{\jobname-pw.ind}}{}

\end{document}

      