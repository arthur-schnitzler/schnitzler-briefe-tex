%% latex-leseansicht-vorspann.tex
%% Vorspann für die Leseansicht.
%% Lädt die gemeinsame Datei latex-vorspann.tex mit nicht gesetztem Schalter.

\newif\ifkorrekturansicht
\korrekturansichtfalse

\input{../tex-inputs/latex-vorspann}


         
         \newcommand{\erwaehntePersonen}{Personen: Richard Beer-Hofmann, Giuseppe Giacosa, Hugo von Hofmannsthal, Henrik Ibsen, Paul Lindau, Adele Sandrock, Christine Schönberger}
         \newcommand{\erwaehnteOrte}{Orte: Brühl, Gardone Riviera, I., Innere Stadt, Italien, Lago di Garda, Wien, Wollzeile}
         \newcommand{\erwaehnteWerke}{Werke: Anatol, Das Portrait, Der Tod Georgs, Don Karlos, Infant von Spanien, Hänsel und Grethel, Liebelei. Schauspiel in drei Akten, Rechte der Seele. Schauspiel in einem Act}
               \section[Arthur Schnitzler an Richard Beer-Hofmann, 26. 9. 1895]{ Arthur Schnitzler an Richard Beer-Hofmann, 26. 9. 1895}\nopagebreak\mylabel{v}\rehead{ }\begin{ledgroupsized}[t]{13cm}\normalsize\beginnumbering \toendnotes[C]{\smallbreak\pagebreak[2]} \Standort{YCGL, MSS 31.}
\physDesc{Brief, 2 Blätter, 7 Seiten, Umschlag
\newline{}Handschrift: 1) schwarze Tinte, deutsche Kurrent (\noindent{}Umschlag)\hspace{1em}2) Bleistift, deutsche Kurrent\hspace{1em}\newline{}Versand: 1) Stempel: »\nobreak{}Wien, 26. 9. 95, 7–8\nobreak{}«.   2) Stempel: »\nobreak{}\oindex{Gardone Riviera@\textbf{Gardone Riviera}|pwk}Gardone Riva, 28 9 95\nobreak{}«.  3) Stempel: »\nobreak{}\oindex{I., Innere Stadt@\textbf{I., Innere Stadt}|pwk}Wien 1/1, 1/10 95, 8–9½ V., Bestellt\nobreak{}«.  4) mit blauer Tinte von unbekannter Hand die Nachsendeadresse vermerkt: »I Wollzeile 15\oindex{Wollzeile@\textbf{Wollzeile}|pw}. Wien I.\oindex{Wien@\textbf{Wien}|pw}«}\buchAbdrucke{\weitereDrucke{Arthur Schnitzler, Richard Beer-Hofmann: \emph{Briefwechsel 1891–1931}. Hg. Konstanze Fliedl. Wien, Zürich: \emph{Europaverlag} 1992, S. 85–86.} }\toendnotes[C]{\smallbreak}\pstart{}{\pb}Herrn \textsc{Dr. Richard
                     Beer-Hofmann}\pend{}\pstart{}\textsc{Gardone}\oindex{Gardone Riviera@\textbf{Gardone Riviera}|pw}\pend{}\pstart{}\textsc{am Gardasee\oindex{Lago di Garda@\textbf{Lago di Garda}|pw}}\pend{}\pstart{}\textsc{Italien}\oindex{Italien@\textbf{Italien}|pw}\pend{}{\bigskip}\pstart
           \raggedleft{}{\pb}Wien\oindex{Wien@\textbf{Wien}|pw}{ }26. 9. 95\pend
           \pstart
           Lieber Richard, heute kam zugleich Ihre Karte vom 23.
               und Ihr Brief vom 24. an. Ich ſende also dieſe Zeilen hier nach Gardone\oindex{Gardone Riviera@\textbf{Gardone Riviera}|pw}; warum ſchreiben Sie nicht, wohin Sie von da
               aus gehen? Eben hat mir die Tragödin\pwindex{Sandrock, Adele 1863-08-19 – 1937-08-30@\textsc{Sandrock, Adele} (1863-08-19 – 1937-08-30), \emph{Schauspielerin}|pwv} telephonirt, es war heut Probe von Liebelei\pwindex{Schnitzler, Arthur 15.05.1862 – 21.10.1931@\textsc{Schnitzler, Arthur} (15.05.1862 – 21.10.1931), \emph{Schriftsteller, Mediziner}!Liebelei. Schauspiel in drei Akten1895-10-09@\strich\emph{Liebelei. Schauspiel in drei Akten} {[}1895-10-09{]}|pw} (ſtatt Don \textsc{Carlos}\pwindex{\textcolor{red}{\textsuperscript{XXXX1 indx}}!Don Karlos, Infant von Spanien1787@\strich\emph{Don Karlos, Infant von Spanien} {[}1787{]}|pw}) von der ich nichts wußte, und ſie überbot ſich ſelbſt an Liebenswürdigkeiten
               für mich, mein Stück\pwindex{Schnitzler, Arthur 15.05.1862 – 21.10.1931@\textsc{Schnitzler, Arthur} (15.05.1862 – 21.10.1931), \emph{Schriftsteller, Mediziner}!Liebelei. Schauspiel in drei Akten1895-10-09@\strich\emph{Liebelei. Schauspiel in drei Akten} {[}1895-10-09{]}|pwv} und ihre
               Rolle. {\pb}Sie hat heute auf der Probe einen
               »großartigen« Erfolg gehabt, und na, und ſo weiter. Ich denke, die \textsc{Premiere\pwindex{Schnitzler, Arthur 15.05.1862 – 21.10.1931@\textsc{Schnitzler, Arthur} (15.05.1862 – 21.10.1931), \emph{Schriftsteller, Mediziner}!Liebelei. Schauspiel in drei Akten1895-10-09@\strich\emph{Liebelei. Schauspiel in drei Akten} {[}1895-10-09{]}|pwv}} wird am 7. oder 8. oder 9.{ }ſein. Dazu gibt man \textsc{Giacosa}\pwindex{Giacosa, Giuseppe 21.10.1847 – 02.09.1906@\textsc{Giacosa, Giuseppe} (21.10.1847 – 02.09.1906), \emph{Schriftsteller}|pw}, Rechte der Seele\pwindex{Giacosa, Giuseppe 21.10.1847 – 02.09.1906@\textsc{Giacosa, Giuseppe} (21.10.1847 – 02.09.1906), \emph{Schriftsteller}!Rechte der Seele. Schauspiel in einem Act1894@\strich\emph{Rechte der Seele. Schauspiel in einem Act} {[}1894{]}|pw}. Für einen guten Sitz ſoll
               geſorgt sein. –\pend
           \pstart
           Allmälig hab ich zu arbeiten angefangen. Begonnen hab ich damit, daſs ich ein Stück\pwindex{Schnitzler, Arthur 15.05.1862 – 21.10.1931@\textsc{Schnitzler, Arthur} (15.05.1862 – 21.10.1931), \emph{Schriftsteller, Mediziner}!PortraitNone@\strich\emph{Das Portrait} {[}None{]}|pwv} (Einakter) in Verſen, {\pb}den ich vorigen Winter ſchrieb, in mein\introOben{}em\introOben{}{ }\substVorne{}\textsuperscript{\textcolor{gray}{Kästchen}}{\allowbreak}\substDazwischen{}Schreibtiſch\substHinten{} vergrub, – wo e\substVorne{}\textsuperscript{s}\substDazwischen{}r\substHinten{} am tiefſten iſt. Ich hab manchmal die ſtarke Empfindung, daſs mir nie mehr
               etwas gelingen wird – wie \textsc{Ibsen}\pwindex{Ibsen, Henrik 20.03.1828 – 23.05.1906@\textsc{Ibsen, Henrik} (20.03.1828 – 23.05.1906), \emph{Schriftsteller}|pw} und – \textsc{Paul Lindau}\pwindex{Lindau, Paul 03.06.1839 – 31.01.1919@\textsc{Lindau, Paul} (03.06.1839 – 31.01.1919), \emph{Schriftsteller, Kritiker, Theaterleiter}|pw}. –\pend
           \pstart
           Da die Läufigkeit der Frauen manchmal angenehm war, haben Sie wohl auch was »erlebt«
                  {\dots} wenigſtens {\pb}Anfänge.
               Da drin ſtecken ja die ganzen Erlebniſſe, die Schlüſſe ſind ja dieſelben. (Anatol\pwindex{Schnitzler, Arthur 15.05.1862 – 21.10.1931@\textsc{Schnitzler, Arthur} (15.05.1862 – 21.10.1931), \emph{Schriftsteller, Mediziner}!Anatol1892-10-29@\strich\emph{Anatol} {[}1892-10-29{]}|pwv} reibt ſich die Augen. Er
                  ſchlu{\geminationm}ert ſofort wieder ein. Bald ſchläfſt du {\dots}{ }\textsc{etc}. ſiehe \textsc{Hänsel u Grethel}\pwindex{\textcolor{red}{\textsuperscript{XXXX1 indx}}!Haensel und Grethel1893@\strich\emph{Hänsel und Grethel} {[}1893{]}|pw}) Ich beneide Sie ſo um die Natur. Es iſt ſo ſchön jetzt und ich möchte ganz wo
               anders ſein. Neulich war ich {\pb}in der Brühl\oindex{Bruehl@\textbf{Brühl}|pw}. Tini\pwindex{Schoenberger, Christine 1875-11-17 – 1971-02-03@\textsc{Schönberger, Christine} (1875-11-17 – 1971-02-03), \emph{Gastwirtin}|pw} iſt ſehr ſtolz
               geworden. Auch war ein Jägerlieutenant draußen. Dem Hugo\pwindex{Hofmannsthal, Hugo von 1874-02-01 – 1929-07-15@\textsc{Hofmannsthal, Hugo von} (1874-02-01 – 1929-07-15), \emph{Schriftsteller}|pw} hab ich Ihre Kränkung ausgerichtet, er iſt auch gekränkt. –\pend
           \pstart
           Wie weit iſt der Liebling der Götter\pwindex{Beer-Hofmann, Richard 1866-07-11 – 1945-09-26@\textsc{Beer-Hofmann, Richard} (1866-07-11 – 1945-09-26), \emph{Schriftsteller}!Tod Georgs1900@\strich\emph{Der Tod Georgs} {[}1900{]}|pw} und
               hoffentlich vieler Menschen? – \pend
           \pstart
           {\pb}Leben Sie wohl und ſchreiben Sie mir.
                  Samſtag werde ich wohl das Datum der \textsc{Prém.}\pwindex{Schnitzler, Arthur 15.05.1862 – 21.10.1931@\textsc{Schnitzler, Arthur} (15.05.1862 – 21.10.1931), \emph{Schriftsteller, Mediziner}!Liebelei. Schauspiel in drei Akten1895-10-09@\strich\emph{Liebelei. Schauspiel in drei Akten} {[}1895-10-09{]}|pwv}{ }\textsc{def}\substVorne{}\textsuperscript{.}\substDazwischen{}\textsc{initiv}\substHinten{} kennen.\pend
           \pstart
           Man erkundigt ſich i{\geminationm}erfort und allſeitig nach Ihnen,
               was keine Broſamen, ſondern naive Wahrheiten {\pb}ſind.
               Warum ſoll ichs Ihnen denn verſchweigen? Dazu bin ich nicht 999gradig genug.\pend
           \pstart
           Herzlichen Gruſs, ich freu mich ſchon ſehr auf Sie.{\\[\baselineskip]}Ihr
                  \spacefill\mbox{Arthur.}\pend
           \leftskip=0em{}
         
         \endnumbering\mylabel{h}\end{ledgroupsized}  \newcommand{\dateiname}{L00494}\newcommand{\titel}{Arthur Schnitzler an Richard Beer-Hofmann, 26. 9. 1895}\newcommand{\editorInnen}{Martin Anton Müller und Gerd-Hermann Susen}%% latex-leseansicht-abspann.tex
%% Abspann für die Leseansicht.
%% Der Schalter \ifkorrekturansicht ist bereits durch den Vorspann gesetzt.

%% latex-abspann.tex
%% Gemeinsamer Abspann für Korrekturansicht und Leseansicht.
%% Setzt den Schalter \ifkorrekturansicht voraus (gesetzt in den
%% einbindenden Dateien latex-korrekturansicht-abspann.tex bzw.
%% latex-leseansicht-abspann.tex).
%% ---------------------------------------------------------------

\normalsize

% Das esempio-Environment wird nur in der Leseansicht benötigt
\ifkorrekturansicht\else
\newenvironment{esempio}[3]%
{
    \vspace{1.5ex}
    \rlap{\underline{#1}}
    \par
    \setlength{\parindent}{0cm}
    \nopagebreak
    \leftskip=#2cm
    \rightskip=#3cm
}
{
    \par
}
\fi

\doendnotes{C}
\bigskip
\vfill

\clearpage

\footnotesize

\ifkorrekturansicht
  \lohead{\textsc{register}}
\fi

% theindex-Environment neu definieren ohne reledmac
\makeatletter
\renewenvironment{theindex}{%
  \ifkorrekturansicht
    \section*{\indexname}%
  \else
    \subsubsection*{Index der erwähnten Entitäten}%
  \fi
  \setlength{\parindent}{0pt}%
  \setlength{\parskip}{0pt plus 0.3pt}%
  \let\item\@idxitem
}{%
  \ifkorrekturansicht\clearpage\fi
}
\makeatother

\IfFileExists{\jobname-pw.ind}{\input{\jobname-pw.ind}}{}

% Quellenangabe nur in der Leseansicht
\ifkorrekturansicht\else
% Fallback-Definitionen, falls die .tex-Datei \titel etc. nicht gesetzt hat
\providecommand{\titel}{}
\providecommand{\editorInnen}{}
\providecommand{\dateiname}{\jobname}

\vspace{3cm}

\vfill

\footnotesize
\textsc{Quelle}: \titel. Herausgegeben von {\editorInnen}. In: \emph{Arthur Schnitzler: Briefwechsel mit Autorinnen und Autoren}.
 Digitale Edition, https://schnitzler-briefe.acdh.oeaw.ac.at/{\dateiname}.html (Stand \today)
\fi

\end{document}


      