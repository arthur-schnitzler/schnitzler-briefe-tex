%% latex-korrekturansicht-vorspann.tex
%% Vorspann für die Korrekturansicht.
%% Lädt die gemeinsame Datei latex-vorspann.tex mit gesetztem Schalter.

\newif\ifkorrekturansicht
\korrekturansichttrue

\input{../tex-inputs/latex-vorspann}


\section[Georg Brandes an Arthur Schnitzler, 23. 8. 1914]{L02195 Georg Brandes an Arthur Schnitzler, 23. 8. 1914}
\nopagebreak\mylabel{L02195v}
\rehead{ }\normalsize\beginnumbering\briefempfaengerindex{Schnitzler, Arthur@\textsc{Schnitzler, Arthur}!zzzBrandes, Georg@\emph{von Georg Brandes}!1914-08-231@{23. 8. 1914}|(be}
\toendnotes[C]{\smallbreak\pagebreak[2]}\Standort{CUL, Schnitzler, B 17.}
\physDesc{Brief, 1 Blatt, 4 Seiten, 2896 Zeichen
\newline{}Handschrift: schwarze Tinte, lateinische Kurrent
\newline{}Schnitzler: mit Bleistift unterhalb des Datums wohl der Tag der Zustellung
                                 ergänzt: »am 10. 9. 14« 
\newline{}Ordnung: mit Bleistift von unbekannter Hand nummeriert:
                                    »=42?« }
\buchAbdrucke{\weitereDrucke{Georg Brandes, Arthur Schnitzler: \emph{Ein Briefwechsel}. Bern: \emph{Francke} 1956, S. 109–110.} }\toendnotes[C]{\smallbreak}
\pstart
           \raggedleft{}{\pb}Kopenhagen\oindex{Kopenhagen@\textbf{Kopenhagen}, \emph{P.PPLC}|pw}{\\}23 August 14\pend
           
\pstart{}Verehrter und lieber Freund\pend\vspace{0.5em}
\pstart
           Erst jetzt erhalte ich Ihren Schweiz\oindex{Schweiz@\textbf{Schweiz}, \emph{A.PCLI}|pw}erbrief vom
                  3 August. Er war 20 Tage unterwegs.\pend
           
\pstart
           Ich brauche kaum zu sagen, wie gerne ich etwas für Sie thun möchte. Sie wissen, wie
               lieb ich Sie habe und wie sehr ich Sie schätze.\pend
           
\pstart
           Leider bin ich nicht der rechte Mann. Ich bin in der schwedischen Akademie\orgindex{Kungliga Vetenskapsakademien@Kungliga Vetenskapsakademien|pw} ganz unbeliebt.\pend
           
\pstart
           \uline{Erstens}: Ich glaube nicht, dass der
                  Schwede\pwindex{?? [Schwede, mit dem Arthur Schnitzler ueber den Nobelpreis spricht] *~1914@\textsc{?? [Schwede, mit dem Arthur Schnitzler über den Nobelpreis spricht]} (*~1914)|pwv} der Ihnen von \uline{Oesterreich}\oindex{Oesterreich@\textbf{Österreich}, \emph{A.PCLI}|pw} sprach, wirklich etwas \uline{wusste}. Jedes Jahr
               werden völlig unrichtige Gerüchte in Umlauf gesetzt. Die Eingeweihten \uline{dürfen} nichts sagen. Der Preis wird 1914 gar
               nicht vertheilt, erst Frühling 1915. Man hat November abgeschafft, Juni
               eingeführt.\pend
           
\pstart
           {\pb}\uline{Zweitens}. Man fragt nicht speciell im Ministerium\oindex{Ministerium fuer Unterricht@\textbf{Ministerium für Unterricht}, \emph{Ministerium (K.MIN)}|pwv} oder in der Akademie\orgindex{Oesterreichische Akademie der Wissenschaften@Österreichische Akademie der Wissenschaften|pw}. Jedes Jahr haben alle Mitglieder einer
                  \uline{Universität} und alle Mitglieder der \uline{Akademien} des Landes eine Stimme. So haben hier
               Universitätsprofessoren und Akademiemitglieder jeder eine Stimme.\pend
           
\pstart
           \uline{Ich} habe keine. Denn obwohl Ehrendoctor an schottischen\oindex{Schottland@\textbf{Schottland}, \emph{A.ADM1}|pw} Universitäten und Ehrenmitglied der
                  amerikanischen Akademie der Wissenschaften und
                  Künste\orgindex{American Academy of Arts and Sciences@American Academy of Arts and Sciences|pw}, der italiänischen\orgindex{Società Italiana delle Scienze detta dei XL@Società Italiana delle Scienze detta dei XL|pwv}, der norwegischen\orgindex{Det Kongelige Norske Videnskabers Selskab@Det Kongelige Norske Videnskabers Selskab|pwv}, der Royal Society\orgindex{Royal Society@Royal Society|pw} usw. bin
               ich nicht einmal ordinäres Mitglied der \uline{dänischen} Akademie\orgindex{Kongelige Danske Videnskabernes Selskab@Kongelige Danske Videnskabernes Selskab|pw}, noch angestellt an der
                  \uline{dänischen} Universität\orgindex{Københavns Universitet@Københavns Universitet|pw}.\pend
           
\pstart
           Bin also \uline{nie} gefragt worden.\pend
           
\pstart
           \uline{Drittens}. Schon vor zehn Jahren schlugen viele fremde
               Schriftsteller (u. a. Anatole France\pwindex{France, Anatole 16.04.1844 – 12.10.1924@\textsc{France, Anatole} (16.04.1844 – 12.10.1924), \emph{Schriftsteller/Schriftstellerin}|pw}) mich zum
                  Nobelpreis\orgindex{Nobelpreis@Nobelpreis|pw} vor; schon vor 9 Jahren schlug {\pb}die dänische Akademie der Wissenschaften\orgindex{Kongelige Danske Videnskabernes Selskab@Kongelige Danske Videnskabernes Selskab|pw} mich einstimmig zum Nobelpreis\orgindex{Nobelpreis@Nobelpreis|pw} vor und hat nie später einen anderen Vorschlag machen
               wollen. Die Schweden\oindex{Schweden@\textbf{Schweden}, \emph{A.PCLI}|pw} aber, die mich hassen,
               weil ich einen russischen\oindex{Russland@\textbf{Russland}, \emph{A.PCLI}|pw}{ }Flüchtling\pwindex{?? [Russischer Fluechtling in Stockholm] *~1914@\textsc{?? [Russischer Flüchtling in Stockholm]} (*~1914)|pwv}, der in Stockholm\oindex{Stockholm@\textbf{Stockholm}, \emph{P.PPLC}|pw} gefesselt war, gegen Auslieferung
               schützte, haben erklärt, dass von mir \uline{nie} die Rede
               sein konnte. So unpopulär bin ich dort. Sie sehen also, dass ich ganz ausser Lage
               bin, jemand offiziell zu empfehlen.\pend
           
\pstart
           \uline{Viertens}. Ich kenne indessen privat einige
               einflussreiche Mitglieder der Akademie\orgindex{Kungliga Vetenskapsakademien@Kungliga Vetenskapsakademien|pwv} und ich werde Ihnen schreiben.\pend
           
\pstart
           Nur ist dies nicht der Moment. Kein Mensch in Schweden\oindex{Schweden@\textbf{Schweden}, \emph{A.PCLI}|pw} denkt an anderes als an den Krieg; das ganze Land ist zur
               Vertheidigung gegen Rusland\oindex{Russland@\textbf{Russland}, \emph{A.PCLI}|pw} gerüstet.\pend
           
\pstart
           {\pb}Ich lernte im vergangenen Sommer
               einigermassen englisch reden, hielt im November–December
               mit viel Erfolg Vorlesungen in allen Städten Englands\oindex{England@\textbf{England}, \emph{A.ADM1}|pw} und Schottlands\oindex{Schottland@\textbf{Schottland}, \emph{A.ADM1}|pw}.\hspace*{2em}Mai und Juni redete ich in Nordamerika\oindex{Amerika@\textbf{Amerika}, \emph{kein passender Code gefunden}|pw}, in New Haven\oindex{New Haven@\textbf{New Haven}, \emph{P.PPL}|pw}, Chicago\oindex{Chicago@\textbf{Chicago}, \emph{P.PPLA2}|pw}, Minneapolis\oindex{Minneapolis@\textbf{Minneapolis}, \emph{P.PPLA2}|pw} und New York\oindex{New York City@\textbf{New York City}, \emph{P.PPL}|pw}. An meinem
               letzten Abend in New York\oindex{New York City@\textbf{New York City}, \emph{P.PPL}|pw} im Juni
                  (93 {\%} Fahrenheit) hatte ich das Comedy Theatre\oindex{Comedy Theatre@\textbf{Comedy Theatre}, \emph{Theater (K.THE)}|pw} so voll dass über tausend Personen mit
               unverrichteteter Sache weggehen müssten.\pend
           
\pstart
           Und nun haben wir den schrecklichen Weltkrieg. Ich möchte Untergang für Rusland\oindex{Russland@\textbf{Russland}, \emph{A.PCLI}|pw}, Rettung für Frankreich\oindex{Frankreich@\textbf{Frankreich}, \emph{A.PCLI}|pw}. Aber wer fragt nach unsern Wünschen! Meine Tochter\pwindex{Philipp, Edith 17.01.1879 – 1968-02-16@\textsc{Philipp, Edith} (17.01.1879 – 1968-02-16)|pwv} hat einen jungen
               deutschen Artillerieofficier\pwindex{Philipp, Reinhold 15.08.1883 – 1968@\textsc{Philipp, Reinhold} (15.08.1883 – 1968), \emph{Fabrikant/Fabrikantin}|pwv}
               von 32 Jahren zum Gatten. Sie ist hier mit einem kl. Mädchen\pwindex{Philipp, Gerda 27.11.1907 – 1968@\textsc{Philipp, Gerda} (27.11.1907 – 1968)|pwv} von 6 Jahren und einem kl. Jungen\pwindex{Philipp, Georg 1912-06-21 – 1995-11-08@\textsc{Philipp, Georg} (1912-06-21 – 1995-11-08), \emph{Schauspieler/Schauspielerin}|pwv} von 2 Jahren in
               grosser Angst für ihren Mann\pwindex{Philipp, Reinhold 15.08.1883 – 1968@\textsc{Philipp, Reinhold} (15.08.1883 – 1968), \emph{Fabrikant/Fabrikantin}|pwv},
               den sie leidenschaftlich liebt.\pend
           
\pstart
           Mein ehrerbietiger Gruss an Ihre liebe Frau Gemahlin\pwindex{Schnitzler, Olga 17.01.1882 – 13.01.1970@\textsc{Schnitzler, Olga} (17.01.1882 – 13.01.1970), \emph{Schauspieler/Schauspielerin, Sänger/Sängerin}|pwv}.\hspace*{2em}Ich bin Ihr
               treuer Freund{\\[\baselineskip]}\spacefill\mbox{Georg Brandes}\pend
           \leftskip=0em{}\selectlanguage{ngerman}\endnumbering\briefempfaengerindex{Schnitzler, Arthur@\textsc{Schnitzler, Arthur}!zzzBrandes, Georg@\emph{von Georg Brandes}!1914-08-231@{23. 8. 1914}|)be}\mylabel{L02195h}  \normalsize

\doendnotes{C}
\bigskip
\vfill

\clearpage

\footnotesize

\lohead{\textsc{register}}

% Definiere theindex-Environment komplett neu ohne reledmac
\makeatletter
\renewenvironment{theindex}{%
  \section*{\indexname}%
  \setlength{\parindent}{0pt}%
  \setlength{\parskip}{0pt plus 0.3pt}%
  \let\item\@idxitem
}{%
  \clearpage
}
\makeatother

\IfFileExists{\jobname-pw.ind}{\input{\jobname-pw.ind}}{}

\end{document}

      