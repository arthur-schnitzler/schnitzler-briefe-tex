%% latex-leseansicht-vorspann.tex
%% Vorspann für die Leseansicht.
%% Lädt die gemeinsame Datei latex-vorspann.tex mit nicht gesetztem Schalter.

\newif\ifkorrekturansicht
\korrekturansichtfalse

\input{../tex-inputs/latex-vorspann}


         
         \newcommand{\erwaehntePersonen}{Personen:  ?? [Russischer Flüchtling in Stockholm],  ?? [Schwede, mit dem Arthur Schnitzler über den Nobelpreis spricht], Anatole France, Edith Philipp, Reinhold Philipp, Gerda Philipp, Georg Philipp, Olga Schnitzler}
         \newcommand{\erwaehnteInstitutionen}{Institutionen: American Academy of Arts and Sciences, Det Kongelige Norske Videnskabers Selskab, Kongelige Danske Videnskabernes Selskab, Kungliga Vetenskapsakademien, Københavns Universitet, Nobelpreis, Royal Society, Società Italiana delle Scienze detta dei XL, Österreichische Akademie der Wissenschaften}
         \newcommand{\erwaehnteOrte}{Orte: Amerika, Bad Ischl, Chicago, Comedy Theatre, England, Frankreich, Kopenhagen, Ministerium für Unterricht, Minneapolis, New Haven, New York City, Russland, Schottland, Schweden, Schweiz, Stockholm, Österreich}
         \newcommand{\erwaehnteWerke}{
               \section[Georg Brandes an Arthur Schnitzler, 23. 8. 1914]{ Georg Brandes an Arthur Schnitzler, 23. 8. 1914}\nopagebreak\mylabel{v}\rehead{ }\begin{ledgroupsized}[t]{13cm}\normalsize\beginnumbering \toendnotes[C]{\smallbreak\pagebreak[2]} \Standort{CUL, Schnitzler, B 17.}
\physDesc{Brief, 1 Blatt, 4 Seiten
\newline{}Handschrift: schwarze Tinte, lateinische Kurrent
\newline{}Schnitzler: mit Bleistift unterhalb des Datums wohl der Tag der Zustellung
            ergänzt: »am 10. 9. 14« \newline{}Ordnung: mit Bleistift von unbekannter Hand nummeriert:
                                        »=42?« }\buchAbdrucke{\weitereDrucke{Georg Brandes, Arthur Schnitzler: \emph{Ein Briefwechsel}. Hg. Kurt Bergel. Bern: \emph{Francke} 1956, S. 109–110.} }\toendnotes[C]{\smallbreak}\pstart
           \raggedleft{}{\pb}Kopenhagen\oindex{Kopenhagen@\textbf{Kopenhagen}|pw}{\\}23 August 14\pend
           \pstart{}Verehrter und lieber Freund\pend\pstart
           Erst jetzt erhalte ich Ihren Schweiz\oindex{Schweiz@\textbf{Schweiz}|pw}erbrief
                    vom 3 August. Er war 20 Tage unterwegs.\pend
           \pstart
           Ich brauche kaum zu sagen, wie gerne ich etwas für Sie thun möchte. Sie wissen,
                    wie lieb ich Sie habe und wie sehr ich Sie schätze.\pend
           \pstart
           Leider bin ich nicht der rechte Mann. Ich bin in der schwedischen Akademie\orgindex{Kungliga Vetenskapsakademien@Kungliga Vetenskapsakademien|pw} ganz unbeliebt.\pend
           \pstart
           \uline{Erstens}: Ich glaube nicht, dass
                    der Schwede\pwindex{?? [Schwede, mit dem Arthur Schnitzler ueber den Nobelpreis spricht] *~1914@\textsc{?? [Schwede, mit dem Arthur Schnitzler über den Nobelpreis spricht]} (*~1914)|pwv} der Ihnen von
                        \uline{Oesterreich}\oindex{Oesterreich@\textbf{Österreich}|pw} sprach, wirklich etwas \uline{wusste}. Jedes Jahr
                    werden völlig unrichtige Gerüchte in Umlauf gesetzt. Die Eingeweihten \uline{dürfen} nichts sagen. Der Preis wird 1914
                    gar nicht vertheilt, erst Frühling 1915. Man hat November
                    abgeschafft, Juni eingeführt.\pend
           \pstart
           {\pb}\uline{Zweitens}. Man fragt nicht speciell im Ministerium\oindex{Ministerium fuer Unterricht@\textbf{Ministerium für Unterricht}|pwv} oder in der Akademie\orgindex{Oesterreichische Akademie der Wissenschaften@Österreichische Akademie der Wissenschaften|pw}. Jedes Jahr haben alle Mitglieder
                    einer \uline{Universität} und alle Mitglieder der \uline{Akademien} des Landes eine Stimme. So haben hier
                    Universitätsprofessoren und Akademiemitglieder jeder eine Stimme.\pend
           \pstart
           \uline{Ich} habe keine. Denn obwohl Ehrendoctor an schottischen\oindex{Schottland@\textbf{Schottland}|pw} Universitäten und Ehrenmitglied
                    der amerikanischen Akademie der Wissenschaften und
                        Künste\orgindex{American Academy of Arts and Sciences@American Academy of Arts and Sciences|pw}, der italiänischen\orgindex{Società Italiana delle Scienze detta dei XL@Società Italiana delle Scienze detta dei XL|pwv}, der norwegischen\orgindex{Det Kongelige Norske Videnskabers Selskab@Det Kongelige Norske Videnskabers Selskab|pwv}, der Royal Society\orgindex{Royal Society@Royal Society|pw} usw.
                    bin ich nicht einmal ordinäres Mitglied der \uline{dänischen} Akademie\orgindex{Kongelige Danske Videnskabernes Selskab@Kongelige Danske Videnskabernes Selskab|pw}, noch angestellt an
                    der \uline{dänischen} Universität\orgindex{Københavns Universitet@Københavns Universitet|pw}.\pend
           \pstart
           Bin also \uline{nie} gefragt worden.\pend
           \pstart
           \uline{Drittens}. Schon vor zehn Jahren schlugen viele
                    fremde Schriftsteller (u. a. Anatole France\pwindex{France, Anatole 16.04.1844 – 12.10.1924@\textsc{France, Anatole} (16.04.1844 – 12.10.1924), \emph{Schriftsteller}|pw})
                    mich zum Nobelpreis\orgindex{Nobelpreis@Nobelpreis|pw} vor; schon vor 9 Jahren
                    schlug {\pb}die dänische Akademie der Wissenschaften\orgindex{Kongelige Danske Videnskabernes Selskab@Kongelige Danske Videnskabernes Selskab|pw} mich einstimmig zum Nobelpreis\orgindex{Nobelpreis@Nobelpreis|pw} vor und hat nie später einen anderen
                    Vorschlag machen wollen. Die Schweden\oindex{Schweden@\textbf{Schweden}|pw} aber,
                    die mich hassen, weil ich einen russischen\oindex{Russland@\textbf{Russland}|pw}{ }Flüchtling\pwindex{?? [Russischer Fluechtling in Stockholm] *~1914@\textsc{?? [Russischer Flüchtling in Stockholm]} (*~1914)|pwv}, der in Stockholm\oindex{Stockholm@\textbf{Stockholm}|pw} gefesselt war, gegen Auslieferung
                    schützte, haben erklärt, dass von mir \uline{nie} die
                    Rede sein konnte. So unpopulär bin ich dort. Sie sehen also, dass ich ganz
                    ausser Lage bin, jemand offiziell zu empfehlen.\pend
           \pstart
           \uline{Viertens}. Ich kenne indessen privat einige
                    einflussreiche Mitglieder der Akademie\orgindex{Kungliga Vetenskapsakademien@Kungliga Vetenskapsakademien|pwv} und ich werde Ihnen schreiben.\pend
           \pstart
           Nur ist dies nicht der Moment. Kein Mensch in Schweden\oindex{Schweden@\textbf{Schweden}|pw} denkt an anderes als an den Krieg; das ganze Land ist zur
                    Vertheidigung gegen Rusland\oindex{Russland@\textbf{Russland}|pw} gerüstet.\pend
           \pstart
           {\pb}Ich lernte im vergangenen
                    Sommer einigermassen englisch reden, hielt im
                        November–December mit viel Erfolg Vorlesungen in
                    allen Städten Englands\oindex{England@\textbf{England}|pw} und Schottlands\oindex{Schottland@\textbf{Schottland}|pw}.\hspace*{2em}Mai und Juni redete ich in Nordamerika\oindex{Amerika@\textbf{Amerika}|pw}, in New Haven\oindex{New Haven@\textbf{New Haven}|pw}, Chicago\oindex{Chicago@\textbf{Chicago}|pw}, Minneapolis\oindex{Minneapolis@\textbf{Minneapolis}|pw} und New York\oindex{New York City@\textbf{New York City}|pw}. An meinem
                    letzten Abend in New York\oindex{New York City@\textbf{New York City}|pw} im
                        Juni (93 {\%} Fahrenheit) hatte ich das
                        Comedy Theatre\oindex{Comedy Theatre@\textbf{Comedy Theatre}|pw} so voll dass über tausend
                    Personen mit unverrichteteter Sache weggehen müssten.\pend
           \pstart
           Und nun haben wir den schrecklichen Weltkrieg. Ich möchte Untergang für Rusland\oindex{Russland@\textbf{Russland}|pw}, Rettung für Frankreich\oindex{Frankreich@\textbf{Frankreich}|pw}. Aber wer fragt nach unsern Wünschen! Meine Tochter\pwindex{Philipp, Edith 17.01.1879 – 1968-02-16@\textsc{Philipp, Edith} (17.01.1879 – 1968-02-16)|pwv} hat einen jungen
                    deutschen Artillerieofficier\pwindex{Philipp, Reinhold 15.08.1883 – 1968@\textsc{Philipp, Reinhold} (15.08.1883 – 1968), \emph{Fabrikant}|pwv} von 32 Jahren zum Gatten. Sie ist hier mit einem
                    kl. Mädchen\pwindex{Philipp, Gerda 27.11.1907 – 1968@\textsc{Philipp, Gerda} (27.11.1907 – 1968)|pwv} von 6 Jahren
                    und einem kl. Jungen\pwindex{Philipp, Georg 1912-06-21 – 1995-11-08@\textsc{Philipp, Georg} (1912-06-21 – 1995-11-08), \emph{Schauspieler}|pwv} von
                    2 Jahren in grosser Angst für ihren Mann\pwindex{Philipp, Reinhold 15.08.1883 – 1968@\textsc{Philipp, Reinhold} (15.08.1883 – 1968), \emph{Fabrikant}|pwv}, den sie leidenschaftlich liebt.\pend
           \pstart
           Mein ehrerbietiger Gruss an Ihre liebe Frau Gemahlin\pwindex{Schnitzler, Olga 17.01.1882 – 13.01.1970@\textsc{Schnitzler, Olga} (17.01.1882 – 13.01.1970), \emph{Schauspielerin, Sängerin}|pwv}.\hspace*{2em}Ich bin
                    Ihr treuer Freund{\\[\baselineskip]}\spacefill\mbox{Georg Brandes}\pend
           \leftskip=0em{}
         
         \endnumbering\mylabel{h}\end{ledgroupsized}  \newcommand{\dateiname}{L02195}\newcommand{\titel}{Georg Brandes an Arthur Schnitzler, 23. 8. 1914}\newcommand{\editorInnen}{Martin Anton Müller und Gerd-Hermann Susen}%% latex-leseansicht-abspann.tex
%% Abspann für die Leseansicht.
%% Der Schalter \ifkorrekturansicht ist bereits durch den Vorspann gesetzt.

%% latex-abspann.tex
%% Gemeinsamer Abspann für Korrekturansicht und Leseansicht.
%% Setzt den Schalter \ifkorrekturansicht voraus (gesetzt in den
%% einbindenden Dateien latex-korrekturansicht-abspann.tex bzw.
%% latex-leseansicht-abspann.tex).
%% ---------------------------------------------------------------

\normalsize

% Das esempio-Environment wird nur in der Leseansicht benötigt
\ifkorrekturansicht\else
\newenvironment{esempio}[3]%
{
    \vspace{1.5ex}
    \rlap{\underline{#1}}
    \par
    \setlength{\parindent}{0cm}
    \nopagebreak
    \leftskip=#2cm
    \rightskip=#3cm
}
{
    \par
}
\fi

\doendnotes{C}
\bigskip
\vfill

\clearpage

\footnotesize

\ifkorrekturansicht
  \lohead{\textsc{register}}
\fi

% theindex-Environment neu definieren ohne reledmac
\makeatletter
\renewenvironment{theindex}{%
  \ifkorrekturansicht
    \section*{\indexname}%
  \else
    \subsubsection*{Index der erwähnten Entitäten}%
  \fi
  \setlength{\parindent}{0pt}%
  \setlength{\parskip}{0pt plus 0.3pt}%
  \let\item\@idxitem
}{%
  \ifkorrekturansicht\clearpage\fi
}
\makeatother

\IfFileExists{\jobname-pw.ind}{\input{\jobname-pw.ind}}{}

% Quellenangabe nur in der Leseansicht
\ifkorrekturansicht\else
% Fallback-Definitionen, falls die .tex-Datei \titel etc. nicht gesetzt hat
\providecommand{\titel}{}
\providecommand{\editorInnen}{}
\providecommand{\dateiname}{\jobname}

\vspace{3cm}

\vfill

\footnotesize
\textsc{Quelle}: \titel. Herausgegeben von {\editorInnen}. In: \emph{Arthur Schnitzler: Briefwechsel mit Autorinnen und Autoren}.
 Digitale Edition, https://schnitzler-briefe.acdh.oeaw.ac.at/{\dateiname}.html (Stand \today)
\fi

\end{document}


      