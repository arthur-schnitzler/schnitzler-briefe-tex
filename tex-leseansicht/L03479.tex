%% latex-korrekturansicht-vorspann.tex
%% Vorspann für die Korrekturansicht.
%% Lädt die gemeinsame Datei latex-vorspann.tex mit gesetztem Schalter.

\newif\ifkorrekturansicht
\korrekturansichttrue

\input{../tex-inputs/latex-vorspann}


\section[ Paul Goldmann an Arthur Schnitzler, 24. 10. 1925]{L03479 Paul Goldmann an Arthur Schnitzler, 24. 10. 1925}
\nopagebreak\mylabel{L03479v}
\rehead{ }\normalsize\beginnumbering\briefempfaengerindex{Schnitzler, Arthur@\textsc{Schnitzler, Arthur}!zzzGoldmann, Paul@\emph{von Paul Goldmann}!1925-10-241@{24. 10. 1925}|(be}
\toendnotes[C]{\smallbreak\pagebreak[2]}\Standort{DLA, A:Schnitzler, HS.NZ85.1.3176.}
\physDesc{Brief, 1 Blatt, 2 Seiten, 831 Zeichen
\newline{}Handschrift: lila Tinte, deutsche Kurrent
\newline{}Schnitzler: mit rotem Buntstift zwei Unterstreichungen }\toendnotes[C]{\smallbreak}
\pstart
           {\pb}Berlin\oindex{Berlin@\textbf{Berlin}, \emph{P.PPLC}|pw}, 24. 10. 25.\pend
           
\pstart{}Lieber Freund,\pend\vspace{0.5em}
\pstart
           Es war ſehr lieb von Dir, daß Du gleich nach Deiner \label{K_L03479-1v}\edtext{Heimkehr}{\lemma{\textnormal{\emph{Heimkehr}}}\Cendnote{\textnormal{Schnitzler langte am 21. 10. 1925, aus Berlin\oindex{Berlin@\textbf{Berlin}, \emph{P.PPLC}|pwk} kommend, in Wien\oindex{Wien@\textbf{Wien}, \emph{A.ADM2}|pwk} an.}}}\label{K_L03479-1}{ }uns\pwindex{Goldmann, Franziska 1911-05-29 – 1963-08-19@\textsc{Goldmann, Franziska} (1911-05-29 – 1963-08-19), \emph{Schauspieler/Schauspielerin}|pwv} die Bücher\pwindex{Fraeulein Else@\emph{Fräulein Else}|pwv}\pwindex{Komoedie der Verfuehrung. In drei Akten@\emph{Komödie der Verführung. In drei Akten}|pwv} geſchickt haſt. Tochter\pwindex{Goldmann, Franziska 1911-05-29 – 1963-08-19@\textsc{Goldmann, Franziska} (1911-05-29 – 1963-08-19), \emph{Schauspieler/Schauspielerin}|pwv} u. Vater danken Dir
               auf das Herzlichſte. Franzi\pwindex{Goldmann, Franziska 1911-05-29 – 1963-08-19@\textsc{Goldmann, Franziska} (1911-05-29 – 1963-08-19), \emph{Schauspieler/Schauspielerin}|pw} iſt bereits in
                  »Fräulein Elſe\pwindex{Fraeulein Else@\emph{Fräulein Else}|pw}« vertieft u. erklärt, es ſei
               das Schönſte, das ſie je geleſen habe, – dankt Dir auch für die eigenhändige Widmung,
               mit der ſie in ihrer Klaſſe großen Eindruck zu machen hofft. Ich freue mich darauf,
               das Buch\pwindex{Fraeulein Else@\emph{Fräulein Else}|pwv} nach meiner Tochter\pwindex{Goldmann, Franziska 1911-05-29 – 1963-08-19@\textsc{Goldmann, Franziska} (1911-05-29 – 1963-08-19), \emph{Schauspieler/Schauspielerin}|pwv} zu leſen. »Komödie der Verführung\pwindex{Komoedie der Verfuehrung. In drei Akten@\emph{Komödie der Verführung. In drei Akten}|pw}« iſt mir bereits bekannt.
               Für die Widmung danke ich Dir noch beſonders – ebenſo wie für Deinen lieben \label{K_L03479-2v}\edtext{Beſuch}{\lemma{\textnormal{\emph{Beſuch}}}\Cendnote{\textnormal{Am 17. 10. 1925 trafen Goldmann\pwindex{Goldmann, Paul 31.01.1865 – 25.09.1935@\textsc{Goldmann, Paul} (31.01.1865 – 25.09.1935), \emph{Schriftsteller/Schriftstellerin, Journalist/Journalistin}|pwk}
                  und seine Tochter Franziska\pwindex{Goldmann, Franziska 1911-05-29 – 1963-08-19@\textsc{Goldmann, Franziska} (1911-05-29 – 1963-08-19), \emph{Schauspieler/Schauspielerin}|pwk} mit Schnitzler zusammen, am 20. 10. 1925 besuchte
                     Schnitzler die beiden zu Hause.}}}\label{K_L03479-2},
               der für mich eine ſehr große Freude war. Wirklich – Du {\pb}biſt kaum gealtert – biſt innerlich derſelbe
               geblieben u. haſt Dich auch äußerlich nur wenig verändert.\pend
           
\pstart
           Und nun wollen wir zuſammen bleiben – in alter Freundſchaft – bis zum Schluß!
               {\\[\baselineskip]}Herzlichſt {\\[\baselineskip]}Dein {\\[\baselineskip]}\spacefill\mbox{Paul Goldmann.}\pend
           \leftskip=0em{}\selectlanguage{ngerman}\endnumbering\briefempfaengerindex{Schnitzler, Arthur@\textsc{Schnitzler, Arthur}!zzzGoldmann, Paul@\emph{von Paul Goldmann}!1925-10-241@{24. 10. 1925}|)be}\mylabel{L03479h}  \normalsize

\doendnotes{C}
\bigskip
\vfill

\clearpage

\footnotesize

\lohead{\textsc{register}}

% Definiere theindex-Environment komplett neu ohne reledmac
\makeatletter
\renewenvironment{theindex}{%
  \section*{\indexname}%
  \setlength{\parindent}{0pt}%
  \setlength{\parskip}{0pt plus 0.3pt}%
  \let\item\@idxitem
}{%
  \clearpage
}
\makeatother

\IfFileExists{\jobname-pw.ind}{\input{\jobname-pw.ind}}{}

\end{document}

      