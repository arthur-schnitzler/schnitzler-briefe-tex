%% latex-leseansicht-vorspann.tex
%% Vorspann für die Leseansicht.
%% Lädt die gemeinsame Datei latex-vorspann.tex mit nicht gesetztem Schalter.

\newif\ifkorrekturansicht
\korrekturansichtfalse

\input{../tex-inputs/latex-vorspann}


\section[ Paul Goldmann an Arthur Schnitzler, 24. 10. 1925]{L03479 Paul Goldmann an Arthur Schnitzler,  24. 10. 1925}
\nopagebreak\mylabel{L03479v}
\rehead{ }\normalsize\beginnumbering\briefempfaengerindex{Schnitzler, Arthur@\textsc{Schnitzler, Arthur}!zzzGoldmann, Paul@\emph{von Paul Goldmann}!1925-10-241@{24. 10. 1925}|(be}
\toendnotes[C]{\smallbreak\pagebreak[2]}
\correspDesc{Versand  durch Paul Goldmann am 24. 10. 1925 in Berlin
\newline{}Erhalt  durch Arthur Schnitzler im Zeitraum [25. 10. 1925 – 29. 10. 1925?] in Wien}\toendnotes[C]{\smallbreak}
\Standort{DLA, A:Schnitzler, HS.NZ85.1.3176.}
\physDesc{Brief, 1 Blatt, 2 Seiten, 831 Zeichen
\newline{}Handschrift: lila Tinte, deutsche Kurrent
\newline{}Schnitzler: mit rotem Buntstift zwei Unterstreichungen }\toendnotes[C]{\smallbreak}
\pstart
           {\pb}Berlin\oindex{Berlin@\textbf{Berlin}, \emph{Hauptstadt}|pw}, 24. 10. 25.\pend
           
\pstart{}Lieber Freund,\pend\vspace{0.5em}
\pstart
           Es war{ }ſehr lieb von Dir, daß Du gleich nach Deiner \label{K_L03479-1v}\edtext{Heimkehr}{\lemma{\textnormal{\emph{Heimkehr}}}\Cendnote{\textnormal{Schnitzler langte am 21. 10. 1925, aus Berlin\oindex{Berlin@\textbf{Berlin}, \emph{Hauptstadt}|pwk} kommend, in Wien\oindex{Wien@\textbf{Wien}, \emph{Verwaltungsgebiet}|pwk} an.}}}\label{K_L03479-1}{ }uns\pwindex{Goldmann, Franziska 29.\,5.\,1911 Berlin – 19.\,8.\,1963 Rio de Janeiro@\textsc{Goldmann, Franziska} (29.\,5.\,1911 Berlin – 19.\,8.\,1963 Rio de Janeiro), \emph{Schauspielerin}|pwv} die Bücher\pwindex{Schnitzler, Arthur 15.\,5.\,1862 Wien – 21.\,10.\,1931 ebd.@\textsc{Schnitzler, Arthur} (15.\,5.\,1862 Wien – 21.\,10.\,1931 ebd.), \emph{Schriftsteller, Mediziner}!Fräulein Else@\strich\emph{Fräulein Else}|pwv}\pwindex{Schnitzler, Arthur 15.\,5.\,1862 Wien – 21.\,10.\,1931 ebd.@\textsc{Schnitzler, Arthur} (15.\,5.\,1862 Wien – 21.\,10.\,1931 ebd.), \emph{Schriftsteller, Mediziner}!Komödie der Verführung. In drei Akten@\strich\emph{Komödie der Verführung. In drei Akten}|pwv} geſchickt haſt. Tochter\pwindex{Goldmann, Franziska 29.\,5.\,1911 Berlin – 19.\,8.\,1963 Rio de Janeiro@\textsc{Goldmann, Franziska} (29.\,5.\,1911 Berlin – 19.\,8.\,1963 Rio de Janeiro), \emph{Schauspielerin}|pwv} u. Vater danken Dir
               auf das Herzlichſte. Franzi\pwindex{Goldmann, Franziska 29.\,5.\,1911 Berlin – 19.\,8.\,1963 Rio de Janeiro@\textsc{Goldmann, Franziska} (29.\,5.\,1911 Berlin – 19.\,8.\,1963 Rio de Janeiro), \emph{Schauspielerin}|pw} iſt bereits in
                  »Fräulein Elſe\pwindex{Schnitzler, Arthur 15.\,5.\,1862 Wien – 21.\,10.\,1931 ebd.@\textsc{Schnitzler, Arthur} (15.\,5.\,1862 Wien – 21.\,10.\,1931 ebd.), \emph{Schriftsteller, Mediziner}!Fräulein Else@\strich\emph{Fräulein Else}|pw}« vertieft u. erklärt, es{ }ſei
               das Schönſte, das{ }ſie je geleſen habe, – dankt Dir auch für die eigenhändige Widmung,
               mit der{ }ſie in ihrer Klaſſe großen Eindruck zu machen hofft. Ich freue mich darauf,
               das Buch\pwindex{Schnitzler, Arthur 15.\,5.\,1862 Wien – 21.\,10.\,1931 ebd.@\textsc{Schnitzler, Arthur} (15.\,5.\,1862 Wien – 21.\,10.\,1931 ebd.), \emph{Schriftsteller, Mediziner}!Fräulein Else@\strich\emph{Fräulein Else}|pwv} nach meiner Tochter\pwindex{Goldmann, Franziska 29.\,5.\,1911 Berlin – 19.\,8.\,1963 Rio de Janeiro@\textsc{Goldmann, Franziska} (29.\,5.\,1911 Berlin – 19.\,8.\,1963 Rio de Janeiro), \emph{Schauspielerin}|pwv} zu leſen. »Komödie der Verführung\pwindex{Schnitzler, Arthur 15.\,5.\,1862 Wien – 21.\,10.\,1931 ebd.@\textsc{Schnitzler, Arthur} (15.\,5.\,1862 Wien – 21.\,10.\,1931 ebd.), \emph{Schriftsteller, Mediziner}!Komödie der Verführung. In drei Akten@\strich\emph{Komödie der Verführung. In drei Akten}|pw}« iſt mir bereits bekannt.
               Für die Widmung danke ich Dir noch beſonders – ebenſo wie für Deinen lieben \label{K_L03479-2v}\edtext{Beſuch}{\lemma{\textnormal{\emph{Besuch}}}\Cendnote{\textnormal{Am 17. 10. 1925 trafen Goldmann\pwindex{Goldmann, Paul 31.\,1.\,1865 Breslau – 25.\,9.\,1935 Wien@\textsc{Goldmann, Paul} (31.\,1.\,1865 Breslau – 25.\,9.\,1935 Wien), \emph{Schriftsteller, Journalist}|pwk}
                  und seine Tochter Franziska\pwindex{Goldmann, Franziska 29.\,5.\,1911 Berlin – 19.\,8.\,1963 Rio de Janeiro@\textsc{Goldmann, Franziska} (29.\,5.\,1911 Berlin – 19.\,8.\,1963 Rio de Janeiro), \emph{Schauspielerin}|pwk} mit Schnitzler zusammen, am 20. 10. 1925 besuchte
                     Schnitzler die beiden zu Hause.}}}\label{K_L03479-2},
               der für mich eine{ }ſehr große Freude war. Wirklich – Du {\pb}biſt kaum gealtert – biſt innerlich derſelbe
               geblieben u. haſt Dich auch äußerlich nur wenig verändert.\pend
           
\pstart
           Und nun wollen wir zuſammen bleiben – in alter Freundſchaft – bis zum Schluß!
               {\\[\baselineskip]}Herzlichſt {\\[\baselineskip]}Dein {\\[\baselineskip]}\spacefill\mbox{Paul Goldmann.}\pend
           \leftskip=0em{}\selectlanguage{ngerman}\endnumbering\briefempfaengerindex{Schnitzler, Arthur@\textsc{Schnitzler, Arthur}!zzzGoldmann, Paul@\emph{von Paul Goldmann}!1925-10-241@{24. 10. 1925}|)be}\mylabel{L03479h}  \newcommand{\dateiname}{L03479}\newcommand{\titel}{Paul Goldmann an Arthur Schnitzler, 24. 10. 1925}\newcommand{\editorInnen}{Martin Anton Müller und Laura Untner}%% latex-leseansicht-abspann.tex
%% Abspann für die Leseansicht.
%% Der Schalter \ifkorrekturansicht ist bereits durch den Vorspann gesetzt.

%% latex-abspann.tex
%% Gemeinsamer Abspann für Korrekturansicht und Leseansicht.
%% Setzt den Schalter \ifkorrekturansicht voraus (gesetzt in den
%% einbindenden Dateien latex-korrekturansicht-abspann.tex bzw.
%% latex-leseansicht-abspann.tex).
%% ---------------------------------------------------------------

\normalsize

% Das esempio-Environment wird nur in der Leseansicht benötigt
\ifkorrekturansicht\else
\newenvironment{esempio}[3]%
{
    \vspace{1.5ex}
    \rlap{\underline{#1}}
    \par
    \setlength{\parindent}{0cm}
    \nopagebreak
    \leftskip=#2cm
    \rightskip=#3cm
}
{
    \par
}
\fi

\doendnotes{C}
\bigskip
\vfill

\clearpage

\footnotesize

\ifkorrekturansicht
  \lohead{\textsc{register}}
\fi

% theindex-Environment neu definieren ohne reledmac
\makeatletter
\renewenvironment{theindex}{%
  \ifkorrekturansicht
    \section*{\indexname}%
  \else
    \subsubsection*{Index der erwähnten Entitäten}%
  \fi
  \setlength{\parindent}{0pt}%
  \setlength{\parskip}{0pt plus 0.3pt}%
  \let\item\@idxitem
}{%
  \ifkorrekturansicht\clearpage\fi
}
\makeatother

\IfFileExists{\jobname-pw.ind}{\input{\jobname-pw.ind}}{}

% Quellenangabe nur in der Leseansicht
\ifkorrekturansicht\else
% Fallback-Definitionen, falls die .tex-Datei \titel etc. nicht gesetzt hat
\providecommand{\titel}{}
\providecommand{\editorInnen}{}
\providecommand{\dateiname}{\jobname}

\vspace{3cm}

\vfill

\footnotesize
\textsc{Quelle}: \titel. Herausgegeben von {\editorInnen}. In: \emph{Arthur Schnitzler: Briefwechsel mit Autorinnen und Autoren}.
 Digitale Edition, https://schnitzler-briefe.acdh.oeaw.ac.at/{\dateiname}.html (Stand \today)
\fi

\end{document}


