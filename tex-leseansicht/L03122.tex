%% latex-leseansicht-vorspann.tex
%% Vorspann für die Leseansicht.
%% Lädt die gemeinsame Datei latex-vorspann.tex mit nicht gesetztem Schalter.

\newif\ifkorrekturansicht
\korrekturansichtfalse

\input{../tex-inputs/latex-vorspann}


\section[Felix Salten an Arthur Schnitzler, {[}2. 5. 1893{]}]{L03122 Felix Salten an Arthur Schnitzler, {[}2. 5. 1893{]}}
\nopagebreak\mylabel{L03122v}
\rehead{ }\normalsize\beginnumbering\briefempfaengerindex{Schnitzler, Arthur@\textsc{Schnitzler, Arthur}!zzzSalten, Felix@\emph{von Felix Salten}!1893-05-023@{{[}2. 5. 1893{]}}|(be}
\toendnotes[C]{\smallbreak\pagebreak[2]}
\correspDesc{Versand  durch Felix Salten am [2. 5. 1893] in Wien
\newline{}Erhalt  durch Arthur Schnitzler im Zeitraum [2. 5. 1893
                  – 3. 5. 1893?] in Wien}\toendnotes[C]{\smallbreak}
\Standort{CUL, Schnitzler, B 89, A 1.}
\physDesc{Briefkarte, 417 Zeichen
\newline{}Handschrift: Bleistift, lateinische Kurrent
\newline{}Schnitzler: mit Bleistift datiert: »2/5 93« 
\newline{}Ordnung: mit Bleistift von unbekannter Hand nummeriert: »25« }\toendnotes[C]{\smallbreak}
\pstart
           \noindent{}{\pb}Theuerster Freund! Ich bin so furchtbar \label{K_L03122-1v}\edtext{erschüttert}{\lemma{\textnormal{\emph{erschüttert}}}\Cendnote{\textnormal{Am 2. 5. 1893 war Schnitzlers Vater Johann Schnitzler\pwindex{Schnitzler, Johann 10.\,4.\,1835 Nagykanizsa – 2.\,5.\,1893 Wien@\textsc{Schnitzler, Johann} (10.\,4.\,1835 Nagykanizsa – 2.\,5.\,1893 Wien), \emph{Laryngologe}|pwk} verstorben.}}}\label{K_L03122-1},
               dass ich nicht weiss, was ich Ihnen sagen, was ich denken soll, Ich habe nur einen
               Wunsch, u. das ist, Ihnen tragen helfen, was ja doch {\pb}zu schwer sein muss für Sie, zu
               schwer. – Bitte, Sie wissen ja, wie sehr ich Sie liebe, laßen Sie mich, wenn es
               Ihnen Erleichterung ist an Ihrer Seite sein so oft Sie es \label{T_L03122-1v}\edtext{immer}{\lemma{\textnormal{\emph{immer}}}\Cendnote{\textnormal{In der
                  Vorlage steht »i{\geminationm}mer«.}}}\label{T_L03122-1} wollen –\pend
           
\pstart
           Ich weine, es ist doch zu traurig alles\pend
           \pstart \uline{Ihr}{ }\spacefill\mbox{Salten}\pend{}\selectlanguage{ngerman}\endnumbering\briefempfaengerindex{Schnitzler, Arthur@\textsc{Schnitzler, Arthur}!zzzSalten, Felix@\emph{von Felix Salten}!1893-05-023@{{[}2. 5. 1893{]}}|)be}\mylabel{L03122h}  \newcommand{\dateiname}{L03122}\newcommand{\titel}{Felix Salten an Arthur Schnitzler, [2. 5. 1893]}\newcommand{\editorInnen}{Martin Anton Müller und Laura Untner}%% latex-leseansicht-abspann.tex
%% Abspann für die Leseansicht.
%% Der Schalter \ifkorrekturansicht ist bereits durch den Vorspann gesetzt.

%% latex-abspann.tex
%% Gemeinsamer Abspann für Korrekturansicht und Leseansicht.
%% Setzt den Schalter \ifkorrekturansicht voraus (gesetzt in den
%% einbindenden Dateien latex-korrekturansicht-abspann.tex bzw.
%% latex-leseansicht-abspann.tex).
%% ---------------------------------------------------------------

\normalsize

% Das esempio-Environment wird nur in der Leseansicht benötigt
\ifkorrekturansicht\else
\newenvironment{esempio}[3]%
{
    \vspace{1.5ex}
    \rlap{\underline{#1}}
    \par
    \setlength{\parindent}{0cm}
    \nopagebreak
    \leftskip=#2cm
    \rightskip=#3cm
}
{
    \par
}
\fi

\doendnotes{C}
\bigskip
\vfill

\clearpage

\footnotesize

\ifkorrekturansicht
  \lohead{\textsc{register}}
\fi

% theindex-Environment neu definieren ohne reledmac
\makeatletter
\renewenvironment{theindex}{%
  \ifkorrekturansicht
    \section*{\indexname}%
  \else
    \subsubsection*{Index der erwähnten Entitäten}%
  \fi
  \setlength{\parindent}{0pt}%
  \setlength{\parskip}{0pt plus 0.3pt}%
  \let\item\@idxitem
}{%
  \ifkorrekturansicht\clearpage\fi
}
\makeatother

\IfFileExists{\jobname-pw.ind}{\input{\jobname-pw.ind}}{}

% Quellenangabe nur in der Leseansicht
\ifkorrekturansicht\else
% Fallback-Definitionen, falls die .tex-Datei \titel etc. nicht gesetzt hat
\providecommand{\titel}{}
\providecommand{\editorInnen}{}
\providecommand{\dateiname}{\jobname}

\vspace{3cm}

\vfill

\footnotesize
\textsc{Quelle}: \titel. Herausgegeben von {\editorInnen}. In: \emph{Arthur Schnitzler: Briefwechsel mit Autorinnen und Autoren}.
 Digitale Edition, https://schnitzler-briefe.acdh.oeaw.ac.at/{\dateiname}.html (Stand \today)
\fi

\end{document}


