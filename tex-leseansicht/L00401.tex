%% latex-leseansicht-vorspann.tex
%% Vorspann für die Leseansicht.
%% Lädt die gemeinsame Datei latex-vorspann.tex mit nicht gesetztem Schalter.

\newif\ifkorrekturansicht
\korrekturansichtfalse

\input{../tex-inputs/latex-vorspann}

\begin{center}
            \textcolor{red}{ENTWURF. ENTZIFFERUNG NOCH NICHT KORREKTURGELESEN}
                      \end{center}
            
               \section[Friedrich M. Fels an Arthur Schnitzler, {[}12. 11. 1894{]}]{ Friedrich M. Fels an Arthur Schnitzler, {[}12. 11. 1894{]}}\nopagebreak\mylabel{v}\rehead{ }\begin{ledgroupsized}[t]{13cm}\normalsize\beginnumbering\briefempfaengerindex{Schnitzler, Arthur@\textsc{Schnitzler, Arthur}!zzzFels, Friedrich Michael@\emph{von Friedrich Michael Fels}!1894-11-121@{{[}12. 11. 1894{]}}|(be} \toendnotes[C]{\smallbreak\pagebreak[2]} \Standort{DLA, A:Schnitzler, HS.NZ85.1.2956.}
\physDesc{Brief, 2 Blätter (auf Bürstenabzug), 2 Seiten
\newline{}Handschrift: schwarze Tinte, lateinische Kurrent
\newline{}Schnitzler: 1) mit Bleistift datiert: »12/11 94« und nummeriert: »19« bzw. auf
                                            dem zweiten Blatt »19a«. 2) mit rotem Buntstift eine Unterstreichung}\toendnotes[C]{\smallbreak}\pstart{}{\pb}Lieber Doktor Schnitzler!\pend\pstart
           Da ich gerade ein paar Minuten Zeit habe, will ich Ihnen eine Unterredung
                    berichten, die ich heute abend mit meinem Philister\pwindex{?? [Vermieter von F. M. Fels] 1893 – 1894@\textsc{?? [Vermieter von F. M. Fels]} (1893 – 1894)|pwv} hatte; vielleicht haben Sie ein paar
                    Sekunden Zeit, sie zu lesen.\pend
           \pstart
           Auf der Straſse las mich der Herr auf und bega{\geminationn},
                    über schlechten Geschäftsgang zu reden, um mich zu fragen, wie eigentlich »mein
                    Geschäft« gehe. Darauf erbot er sich, da er in der hiesigen Journalistik
                    Beziehungen habe, meinetwegen anzufragen; jedenfalls werde er möglichst bald mit
                        Jak. Herzog\pwindex{Herzog, Jakob 17.5.1842 – 10.4.1915@\textsc{Herzog, Jakob} (17.5.1842 – 10.4.1915), \emph{Herausgeber}|pw} reden, dem Hrsg. der Montagsrevue\orgindex{Montags-Revue@Montags-Revue|pw}, mit dem er sehr gut stehe.\pend
           \pstart
           Da{\geminationn} kamen wir auf die Korff\pwindex{Korff, Heinrich von 05.06.1868 – 18.08.1938@\textsc{Korff, Heinrich von} (05.06.1868 – 18.08.1938), \emph{Journalist}|pwu}sche Denunziation, wobei er mir mitteilte, in
                    letzter Zeit sei niemand von der \uline{Polizei}
                    meinetwegen bei ihnen gewesen, doch drei Tage nach meinem Einzug, also vor fünf
                    Wochen, sei ein Herr erschienen, habe sich seiner Schwägerin, die allein zu
                    Hause gewesen, als Polizeiko{\geminationm}issär (??!)
                    vorgestellt und erklärt, er müſse sie vor mir warnen, da ich ein stadtbeka{\geminationn}ter Schwindler sei. Ih\substVorne{}\textsuperscript{m}\substDazwischen{}n\substHinten{} (dem Philister\pwindex{?? [Vermieter von F. M. Fels] 1893 – 1894@\textsc{?? [Vermieter von F. M. Fels]} (1893 – 1894)|pwv})
                    habe dieses Anzeige nicht bekü{\geminationm}ert; weil er ihr
                    nicht geglaubt habe.\pend
           \pstart
           Nun – so viel dürfte sicher sein: ein Kommiſsär war der Herr nicht,
                        de{\geminationn} ein solcher geht nicht zu den Leuten,
                    sondern läſst sie zu sich ko{\geminationm}en; ein Detektiv auch
                    nicht, de{\geminationn} der {\pb}hätte seinen Adler vorgezeigt und sich ausserdem nicht für einen Ko{\geminationm}issär angegeben. Auſserdem, we{\geminationn} die Polizei bereits seit 5 Wochen auf mich
                    aufmerksam gemacht wäre, wäre es unerfindlich, weshalb ich jetzt erst zitiert
                    worden bin. Es ka{\geminationn} also nur eine Privatperson
                    gewesen sein, die sich den Polizeititel angemasst hat. Wer sie aber war oder von
                    wem sie geschickt worden ist, das ist mir kein Rätsel. Früh übt sich, wer ein
                    Meister werden will.\pend
           \pstart
           Besten Gruſs{\\[\baselineskip]}\spacefill\mbox{Fels}\pend
           \leftskip=0em{}\endnumbering\briefempfaengerindex{Schnitzler, Arthur@\textsc{Schnitzler, Arthur}!zzzFels, Friedrich Michael@\emph{von Friedrich Michael Fels}!1894-11-121@{{[}12. 11. 1894{]}}|)be}\mylabel{h}\end{ledgroupsized}  \newcommand{\dateiname}{L00401}\newcommand{\titel}{Friedrich M. Fels an Arthur Schnitzler, [12. 11. 1894]}\newcommand{\editorInnen}{Martin Anton Müller und Gerd-Hermann Susen}%% latex-leseansicht-abspann.tex
%% Abspann für die Leseansicht.
%% Der Schalter \ifkorrekturansicht ist bereits durch den Vorspann gesetzt.

%% latex-abspann.tex
%% Gemeinsamer Abspann für Korrekturansicht und Leseansicht.
%% Setzt den Schalter \ifkorrekturansicht voraus (gesetzt in den
%% einbindenden Dateien latex-korrekturansicht-abspann.tex bzw.
%% latex-leseansicht-abspann.tex).
%% ---------------------------------------------------------------

\normalsize

% Das esempio-Environment wird nur in der Leseansicht benötigt
\ifkorrekturansicht\else
\newenvironment{esempio}[3]%
{
    \vspace{1.5ex}
    \rlap{\underline{#1}}
    \par
    \setlength{\parindent}{0cm}
    \nopagebreak
    \leftskip=#2cm
    \rightskip=#3cm
}
{
    \par
}
\fi

\doendnotes{C}
\bigskip
\vfill

\clearpage

\footnotesize

\ifkorrekturansicht
  \lohead{\textsc{register}}
\fi

% theindex-Environment neu definieren ohne reledmac
\makeatletter
\renewenvironment{theindex}{%
  \ifkorrekturansicht
    \section*{\indexname}%
  \else
    \subsubsection*{Index der erwähnten Entitäten}%
  \fi
  \setlength{\parindent}{0pt}%
  \setlength{\parskip}{0pt plus 0.3pt}%
  \let\item\@idxitem
}{%
  \ifkorrekturansicht\clearpage\fi
}
\makeatother

\IfFileExists{\jobname-pw.ind}{\input{\jobname-pw.ind}}{}

% Quellenangabe nur in der Leseansicht
\ifkorrekturansicht\else
% Fallback-Definitionen, falls die .tex-Datei \titel etc. nicht gesetzt hat
\providecommand{\titel}{}
\providecommand{\editorInnen}{}
\providecommand{\dateiname}{\jobname}

\vspace{3cm}

\vfill

\footnotesize
\textsc{Quelle}: \titel. Herausgegeben von {\editorInnen}. In: \emph{Arthur Schnitzler: Briefwechsel mit Autorinnen und Autoren}.
 Digitale Edition, https://schnitzler-briefe.acdh.oeaw.ac.at/{\dateiname}.html (Stand \today)
\fi

\end{document}


      