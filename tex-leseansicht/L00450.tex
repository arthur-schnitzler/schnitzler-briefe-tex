%% latex-leseansicht-vorspann.tex
%% Vorspann für die Leseansicht.
%% Lädt die gemeinsame Datei latex-vorspann.tex mit nicht gesetztem Schalter.

\newif\ifkorrekturansicht
\korrekturansichtfalse

\input{../tex-inputs/latex-vorspann}


         
         \renewcommand{\erwaehntePersonen}{Personen:  ?? [Sexualpartnerin von Richard Beer-Hofmann], Richard Beer-Hofmann, Friedrich Michael Fels, Samuel Fischer, Gerhart Hauptmann, Hugo von Hofmannsthal, Gabor Nobl, Heinrich Osten, Clemens Sokal, Edmund Wengraf, Theodor Zasche}
         \renewcommand{\erwaehnteOrte}{Orte: Böhmen, Caslau, Deutschland, IX., Alsergrund, Schweiz, Wien}
         \renewcommand{\erwaehnteWerke}{Werke: Collection Fischer, Das Märchen der 672. Nacht, Die kleine Komödie, Freiwild. Schauspiel in 3 Akten, Neue Deutsche Rundschau, Neue Revue. Wiener Literatur-Zeitung, Sterben, Sterben. Novelle}
               \section[Arthur Schnitzler an Richard Beer-Hofmann, 7. 6. 1895]{ Arthur Schnitzler an Richard Beer-Hofmann, 7. 6. 1895}\nopagebreak\mylabel{v}\rehead{ }\begin{ledgroupsized}[t]{13cm}\normalsize\beginnumbering\briefempfaengerindex{Beer-Hofmann, Richard@\textsc{Beer-Hofmann, Richard}!zzzSchnitzler, Arthur@\emph{von Arthur Schnitzler}!1895-06-071@{7. 6. 1895}|(be} \toendnotes[C]{\smallbreak\pagebreak[2]} \Standort{YCGL, MSS 31.}
\physDesc{Brief, 1 Blatt, 5 Seiten, Umschlag, 1512 Zeichen (Briefpapier mit Trauerrand)
\newline{}Handschrift: schwarze Tinte, deutsche Kurrent
\newline{}Versand: 1) Stempel: »\nobreak{}\oindex{IX., Alsergrund@\textbf{IX., Alsergrund}|pwk}Wien 9/3, 7. 6. 95, 5–6 N\nobreak{}«.   2) Stempel: »\nobreak{}\oindex{Caslau@\textbf{Caslau}|pwk}Časlau, 8 6 95\nobreak{}«. }\buchAbdrucke{\weitereDrucke{Arthur Schnitzler, Richard Beer-Hofmann: \emph{Briefwechsel 1891–1931}. Hg. Konstanze Fliedl. Wien, Zürich: \emph{Europaverlag} 1992, S. 73.} }\toendnotes[C]{\smallbreak}\pstart{}Herrn n. a. Lieutenant\pend{}\pstart{}{\pb}\textsc{Dr. Richard Beer-Hofmann}\pend{}\pstart{}im k. k. Landw.-Inf-Regiment\pend{}\pstart{}\textsc{Caslau\oindex{Caslau@\textbf{Caslau}|pw} Nr 12}.\pend{}\pstart{}\textsc{Böhmen\oindex{Boehmen@\textbf{Böhmen}|pw}}\pend{}{\bigskip}\pstart
           \noindent{}{\pb}Lieber Richard, warum ſchreiben Sie mir denn gar nicht?\pend
           \pstart
           Mit Fels\pwindex{Fels, Friedrich Michael *~1864@\textsc{Fels, Friedrich Michael} (*~1864), \emph{Journalist}|pw} gehn einige Dinge vor, die ausführlich
               zu erzählen zu langweilig wäre. Er muſs fort, in die Schweiz\oindex{Schweiz@\textbf{Schweiz}|pw} – deutſche\oindex{Deutschland@\textbf{Deutschland}|pw} Militärgeſchichte.
               Ich erlaube mir ihm in Ihrem Namen wie in dem Hugo\pwindex{Hofmannsthal, Hugo von 1874-02-01 – 1929-07-15@\textsc{Hofmannsthal, Hugo von} (1874-02-01 – 1929-07-15), \emph{Schriftsteller}|pw}s (mit dem ich ſchon geſprochen – er war ein paar Tage da, wieder
               Catarrh – abſolut unbedenklich) wie in dem meinen je zehn Gulden zu geben. Geht nicht
               anders.\pend
           \pstart
           {\pb}– Warum ſchreiben Sie mir eigentlich nicht? –\pend
           \pstart
           \textsc{Fischer}\pwindex{Fischer, Samuel 24.12.1859 – 15.10.1934@\textsc{Fischer, Samuel} (24.12.1859 – 15.10.1934), \emph{Verleger}|pw} hat mir geſchrieben, mir einen Contract auf 5 Jahre für alle meine Werke,
               angeblich denſelben wie \textsc{Hauptmann}\pwindex{Hauptmann, Gerhart 15.11.1862 – 06.06.1946@\textsc{Hauptmann, Gerhart} (15.11.1862 – 06.06.1946), \emph{Schriftsteller}|pw}{ }\textsc{etc} überſandt (Unterſchr\textcolor{gray}{ieb} noch
               nicht.) Will die \textsc{Kleine Komödie}\pwindex{Schnitzler, Arthur 15.05.1862 – 21.10.1931@\textsc{Schnitzler, Arthur} (15.05.1862 – 21.10.1931), \emph{Schriftsteller, Mediziner}!kleine Komoedie1895-08-01@\strich\emph{Die kleine Komödie} {[}1895-08-01{]}|pw} (die ihm ſehr gut gefällt was mir unheimlich iſt) in der \textsc{Collect. Fischer}\pwindex{?? Werk@Nicht ermittelte Verfasserinnen und Verfasser!Collection Fischer1894 – 1899@\emph{Collection Fischer} {[}1894 – 1899{]}|pw} mit {\pb}\textsc{Zasche\pwindex{Zasche, Theodor 18.10.1862 – 15.11.1922@\textsc{Zasche, Theodor} (18.10.1862 – 15.11.1922), \emph{Zeichner, Karikaturist}|pw}}’ſchen Illuſtr. bringen, will sie aber zuerſt in der \textsc{Freien Bühne}\pwindex{Neue Deutsche Rundschau1894-01-01 – 1903-12-31@\emph{Neue Deutsche Rundschau} {[}1894-01-01 – 1903-12-31{]}|pw} (Auguſtheft, ohne Illuſtr.) veröffentlichen. Wie denken Sie? –\pend
           \pstart
           An N.\pwindex{Nobl, Gabor 12.10.1864 – 14.03.1938@\textsc{Nobl, Gabor} (12.10.1864 – 14.03.1938), \emph{Mediziner, Dermatologe}|pw} hab ich die 20 fl. geſandt; ich ſprach
               ihn zufällig am ſelben Tag, und er wollte ſie nicht nehmen, was ich aber {\pb}heftig abwehrte. – Die betreffende Dame\pwindex{?? [Sexualpartnerin von Richard Beer-Hofmann] 1.5.1895 – 7.6.1895@\textsc{?? [Sexualpartnerin von Richard Beer-Hofmann]} (1.5.1895 – 7.6.1895)|pwv} – nun ſind Sie ja aus allen Sorgen –
               hat natürlich doch \textsc{Lues} gehabt – ſecundäre; auch im Mund.
               Wenn wir alſo bei dem Hugo\pwindex{Hofmannsthal, Hugo von 1874-02-01 – 1929-07-15@\textsc{Hofmannsthal, Hugo von} (1874-02-01 – 1929-07-15), \emph{Schriftsteller}|pw}’ſchen Märchen\pwindex{Hofmannsthal, Hugo von 1874-02-01 – 1929-07-15@\textsc{Hofmannsthal, Hugo von} (1874-02-01 – 1929-07-15), \emph{Schriftsteller}!Maerchen der 672. Nacht2.11.1895 – 16.11.1895@\strich\emph{Das Märchen der 672. Nacht} {[}2.11.1895 – 16.11.1895{]}|pw} bleiben, ka{\geminationn}
               man ſagen: Alles ist eingetroffen, nur – unberufen – hat das \label{K_L00450-1v}\edtext{Pferd}{\lemma{\textnormal{\emph{Pferd}}}\Cendnote{\textnormal{Der Protagonist von \emph{Das Märchen
                     der 672. Nacht}\pwindex{Hofmannsthal, Hugo von 1874-02-01 – 1929-07-15@\textsc{Hofmannsthal, Hugo von} (1874-02-01 – 1929-07-15), \emph{Schriftsteller}!Maerchen der 672. Nacht2.11.1895 – 16.11.1895@\strich\emph{Das Märchen der 672. Nacht} {[}2.11.1895 – 16.11.1895{]}|pwk} stirbt am Hufschlag eines Pferdes.}}}\label{K_L00450-1h} nicht
               ausgeſchlagen. – Daſs Sie {\pb}mir nicht ſchreiben, ist
               durchaus nicht ſchön. –\pend
           \pstart Herzlich der Ihre \spacefill\mbox{Arthur}\pend{}\pstart
           Haben Sie die Kritik\pwindex{Sokal, Clemens *~21.01.1867@\textsc{Sokal, Clemens} (*~21.01.1867), \emph{Journalist, Rechtsanwalt}!Sterben29. 5. 1895@\strich\emph{Sterben} {[}29. 5. 1895{]}|pwv}{ }\textsc{Sokals}\pwindex{Sokal, Clemens *~21.01.1867@\textsc{Sokal, Clemens} (*~21.01.1867), \emph{Journalist, Rechtsanwalt}|pw} über Sterben\pwindex{Schnitzler, Arthur 15.05.1862 – 21.10.1931@\textsc{Schnitzler, Arthur} (15.05.1862 – 21.10.1931), \emph{Schriftsteller, Mediziner}!Sterben. Novelle1894-10-01 – 1894-12-01@\strich\emph{Sterben. Novelle} {[}1894-10-01 – 1894-12-01{]}|pw} geleſen? Merkwürdig von
                  \label{K_L00450-2v}\edtext{\textsc{Osten\pwindex{Osten, Heinrich 16.08.1855 – 01.08.1931@\textsc{Osten, Heinrich} (16.08.1855 – 01.08.1931), \emph{Schriftsteller, Journalist}|pw}-Wengraf\pwindex{Wengraf, Edmund 09.01.1860 – 08.12.1933@\textsc{Wengraf, Edmund} (09.01.1860 – 08.12.1933), \emph{Journalist}|pw}}ſcher Animosität}{\lemma{\textnormal{\emph{Osten-Wengrafſcher Animosität}}}\Cendnote{\textnormal{die beiden
                  Herausgeber der \emph{Neuen Revue}\pwindex{Neue Revue. Wiener Literatur-Zeitung1894 – 1898-05-31@\emph{Neue Revue. Wiener Literatur-Zeitung} {[}1894 – 1898-05-31{]}|pwk}, in der am
                     29. 5. 1895 die Rezension\pwindex{Sokal, Clemens *~21.01.1867@\textsc{Sokal, Clemens} (*~21.01.1867), \emph{Journalist, Rechtsanwalt}!Sterben29. 5. 1895@\strich\emph{Sterben} {[}29. 5. 1895{]}|pwkv} erschienen war.}}}\label{K_L00450-2h} durchtränkt.\pend
           \pstart
           Ich ſchreib jetzt an einem Stück\pwindex{Schnitzler, Arthur 15.05.1862 – 21.10.1931@\textsc{Schnitzler, Arthur} (15.05.1862 – 21.10.1931), \emph{Schriftsteller, Mediziner}!Freiwild. Schauspiel in 3 Akten1896@\strich\emph{Freiwild. Schauspiel in 3 Akten} {[}1896{]}|pwv}. –\pend
           
         
         \endnumbering\mylabel{h}\end{ledgroupsized}  \newcommand{\dateiname}{L00450}\newcommand{\titel}{Arthur Schnitzler an Richard Beer-Hofmann, 7. 6. 1895}\newcommand{\editorInnen}{Martin Anton Müller und Gerd-Hermann Susen}%% latex-leseansicht-abspann.tex
%% Abspann für die Leseansicht.
%% Der Schalter \ifkorrekturansicht ist bereits durch den Vorspann gesetzt.

%% latex-abspann.tex
%% Gemeinsamer Abspann für Korrekturansicht und Leseansicht.
%% Setzt den Schalter \ifkorrekturansicht voraus (gesetzt in den
%% einbindenden Dateien latex-korrekturansicht-abspann.tex bzw.
%% latex-leseansicht-abspann.tex).
%% ---------------------------------------------------------------

\normalsize

% Das esempio-Environment wird nur in der Leseansicht benötigt
\ifkorrekturansicht\else
\newenvironment{esempio}[3]%
{
    \vspace{1.5ex}
    \rlap{\underline{#1}}
    \par
    \setlength{\parindent}{0cm}
    \nopagebreak
    \leftskip=#2cm
    \rightskip=#3cm
}
{
    \par
}
\fi

\doendnotes{C}
\bigskip
\vfill

\clearpage

\footnotesize

\ifkorrekturansicht
  \lohead{\textsc{register}}
\fi

% theindex-Environment neu definieren ohne reledmac
\makeatletter
\renewenvironment{theindex}{%
  \ifkorrekturansicht
    \section*{\indexname}%
  \else
    \subsubsection*{Index der erwähnten Entitäten}%
  \fi
  \setlength{\parindent}{0pt}%
  \setlength{\parskip}{0pt plus 0.3pt}%
  \let\item\@idxitem
}{%
  \ifkorrekturansicht\clearpage\fi
}
\makeatother

\IfFileExists{\jobname-pw.ind}{\input{\jobname-pw.ind}}{}

% Quellenangabe nur in der Leseansicht
\ifkorrekturansicht\else
% Fallback-Definitionen, falls die .tex-Datei \titel etc. nicht gesetzt hat
\providecommand{\titel}{}
\providecommand{\editorInnen}{}
\providecommand{\dateiname}{\jobname}

\vspace{3cm}

\vfill

\footnotesize
\textsc{Quelle}: \titel. Herausgegeben von {\editorInnen}. In: \emph{Arthur Schnitzler: Briefwechsel mit Autorinnen und Autoren}.
 Digitale Edition, https://schnitzler-briefe.acdh.oeaw.ac.at/{\dateiname}.html (Stand \today)
\fi

\end{document}


      