%% latex-korrekturansicht-vorspann.tex
%% Vorspann für die Korrekturansicht.
%% Lädt die gemeinsame Datei latex-vorspann.tex mit gesetztem Schalter.

\newif\ifkorrekturansicht
\korrekturansichttrue

\input{../tex-inputs/latex-vorspann}


\section[Arthur Schnitzler an Richard Beer-Hofmann, 7. 6. 1895]{L00450 Arthur Schnitzler an Richard Beer-Hofmann, 7. 6. 1895}
\nopagebreak\mylabel{L00450v}
\rehead{ }\normalsize\beginnumbering\briefempfaengerindex{Beer-Hofmann, Richard@\textsc{Beer-Hofmann, Richard}!zzzSchnitzler, Arthur@\emph{von Arthur Schnitzler}!1895-06-071@{7. 6. 1895}|(be}
\toendnotes[C]{\smallbreak\pagebreak[2]}\Standort{YCGL, MSS 31.}
\physDesc{Brief, 1 Blatt, 5 Seiten, Umschlag, 1512 Zeichen
\newline{}Handschrift: schwarze Tinte, deutsche Kurrent
\newline{}Versand: 1) Stempel: »\nobreak{}\oindex{IX., Alsergrund@\textbf{IX., Alsergrund}, \emph{A.ADM3}|pwk}Wien 9/3, 7. 6. 95, 5–6 N\nobreak{}«.   2) Stempel: »\nobreak{}\oindex{Cáslav@\textbf{Čáslav}, \emph{P.PPL}|pwk}Časlau, 8 6 95\nobreak{}«. }
\buchAbdrucke{\weitereDrucke{Arthur Schnitzler, Richard Beer-Hofmann: \emph{Briefwechsel 1891–1931}. Wien, Zürich: \emph{Europaverlag} 1992, S. 73.} }\toendnotes[C]{\smallbreak}\pstart{}Herrn n. a. Lieutenant\pend{}\pstart{}{\pb}\textsc{Dr. Richard Beer-Hofmann}\pend{}\pstart{}im k. k. Landw.-Inf-Regiment\pend{}\pstart{}\textsc{Caslau\oindex{Cáslav@\textbf{Čáslav}, \emph{P.PPL}|pw} Nr 12}.\pend{}\pstart{}\textsc{Böhmen\oindex{Boehmen@\textbf{Böhmen}, \emph{L.RGN}|pw}}\pend{}{\bigskip}\vspace{1em}
\pstart
           \noindent{}{\pb}Lieber Richard, warum ſchreiben Sie mir denn gar nicht?\pend
           
\pstart
           Mit Fels\pwindex{Fels, Friedrich Michael *~1864@\textsc{Fels, Friedrich Michael} (*~1864), \emph{Journalist/Journalistin}|pw} gehn einige Dinge vor, die ausführlich
               zu erzählen zu langweilig wäre. Er muſs fort, in die Schweiz\oindex{Schweiz@\textbf{Schweiz}, \emph{A.PCLI}|pw} – deutſche\oindex{Deutschland@\textbf{Deutschland}, \emph{A.PCLI}|pw} Militärgeſchichte.
               Ich erlaube mir ihm in Ihrem Namen wie in dem Hugos\pwindex{Hofmannsthal, Hugo von 1874-02-01 – 1929-07-15@\textsc{Hofmannsthal, Hugo von} (1874-02-01 – 1929-07-15), \emph{Schriftsteller/Schriftstellerin}|pw} (mit dem ich ſchon geſprochen – er war ein paar Tage da, wieder
               Catarrh – abſolut unbedenklich) wie in dem meinen je zehn Gulden zu geben. Geht nicht
               anders.\pend
           
\pstart
           {\pb}– Warum ſchreiben Sie mir eigentlich nicht? –\pend
           
\pstart
           \textsc{Fischer}\pwindex{Fischer, Samuel 24.12.1859 – 15.10.1934@\textsc{Fischer, Samuel} (24.12.1859 – 15.10.1934), \emph{Verleger/Verlegerin}|pw} hat mir geſchrieben, mir einen Contract auf 5 Jahre für alle meine Werke,
               angeblich denſelben wie \textsc{Hauptmann}\pwindex{Hauptmann, Gerhart 15.11.1862 – 06.06.1946@\textsc{Hauptmann, Gerhart} (15.11.1862 – 06.06.1946), \emph{Schriftsteller/Schriftstellerin}|pw}{ }\textsc{etc} überſandt (Unterſchr\textcolor{gray}{ieb} noch
               nicht.) Will die \textsc{Kleine Komödie}\pwindex{kleine Komoedie@\emph{Die kleine Komödie}|pw} (die ihm ſehr gut gefällt was mir unheimlich iſt) in der \textsc{Collect. Fischer}\pwindex{Collection Fischer@\emph{Collection Fischer}|pw} mit {\pb}\textsc{Zasche\pwindex{Zasche, Theodor 18.10.1862 – 15.11.1922@\textsc{Zasche, Theodor} (18.10.1862 – 15.11.1922), \emph{Zeichner/Zeichnerin, Karikaturist/Karikaturistin}|pw}}’ſchen Illuſtr. bringen, will sie aber zuerſt in der \textsc{Freien Bühne}\pwindex{Neue Deutsche Rundschau@\emph{Neue Deutsche Rundschau}|pw} (Auguſtheft, ohne Illuſtr.) veröffentlichen. Wie denken Sie? –\pend
           
\pstart
           An N.\pwindex{Nobl, Gabor 12.10.1864 – 14.03.1938@\textsc{Nobl, Gabor} (12.10.1864 – 14.03.1938), \emph{Mediziner/Medizinerin, Dermatologe/Dermatologin}|pw} hab ich die 20 fl. geſandt; ich ſprach
               ihn zufällig am ſelben Tag, und er wollte ſie nicht nehmen, was ich aber {\pb}heftig abwehrte. – Die betreffende Dame\pwindex{?? [Sexualpartnerin von Richard Beer-Hofmann] @\textsc{?? [Sexualpartnerin von Richard Beer-Hofmann]}|pwv} – nun ſind Sie ja aus allen Sorgen –
               hat natürlich doch \textsc{Lues} gehabt – ſecundäre; auch im Mund.
               Wenn wir alſo bei dem Hugo\pwindex{Hofmannsthal, Hugo von 1874-02-01 – 1929-07-15@\textsc{Hofmannsthal, Hugo von} (1874-02-01 – 1929-07-15), \emph{Schriftsteller/Schriftstellerin}|pw}’ſchen Märchen\pwindex{Maerchen der 672. Nacht@\emph{Das Märchen der 672. Nacht}|pw} bleiben, ka{\geminationn}
               man ſagen: Alles ist eingetroffen, nur – unberufen – hat das \label{K_L00450-1v}\edtext{Pferd}{\lemma{\textnormal{\emph{Pferd}}}\Cendnote{\textnormal{Der Protagonist von \emph{Das Märchen
                     der 672. Nacht}\pwindex{Maerchen der 672. Nacht@\emph{Das Märchen der 672. Nacht}|pwk} stirbt am Hufschlag eines Pferdes.}}}\label{K_L00450-1} nicht
               ausgeſchlagen. – Daſs Sie {\pb}mir nicht ſchreiben, ist
               durchaus nicht ſchön. –\pend
           \pstart Herzlich der Ihre \spacefill\mbox{Arthur}\pend{}
\pstart
           Haben Sie die Kritik\pwindex{Sterben@\emph{Sterben}|pwv}{ }\textsc{Sokals}\pwindex{Sokal, Clemens *~21.01.1867@\textsc{Sokal, Clemens} (*~21.01.1867), \emph{Journalist/Journalistin, Rechtsanwalt/Rechtsanwältin}|pw} über Sterben\pwindex{Sterben. Novelle@\emph{Sterben. Novelle}|pw} geleſen? Merkwürdig von
                  \label{K_L00450-2v}\edtext{\textsc{Osten\pwindex{Osten, Heinrich 16.08.1855 – 01.08.1931@\textsc{Osten, Heinrich} (16.08.1855 – 01.08.1931), \emph{Schriftsteller/Schriftstellerin, Journalist/Journalistin}|pw}-Wengraf\pwindex{Wengraf, Edmund 09.01.1860 – 08.12.1933@\textsc{Wengraf, Edmund} (09.01.1860 – 08.12.1933), \emph{Schriftsteller/Schriftstellerin, Journalist/Journalistin, Kaufmann/Kauffrau}|pw}}ſcher Animosität}{\lemma{\textnormal{\emph{Osten-Wengrafſcher Animosität}}}\Cendnote{\textnormal{die beiden
                  Herausgeber der \emph{Neuen Revue}\pwindex{Neue Revue. Wiener Literatur-Zeitung@\emph{Neue Revue. Wiener Literatur-Zeitung}|pwk}, in der am
                     29. 5. 1895 die Rezension\pwindex{Sterben@\emph{Sterben}|pwkv} erschienen war.}}}\label{K_L00450-2} durchtränkt.\pend
           
\pstart
           Ich ſchreib jetzt an einem Stück\pwindex{Freiwild. Schauspiel in 3 Akten@\emph{Freiwild. Schauspiel in 3 Akten}|pwv}. –\pend
           \selectlanguage{ngerman}\endnumbering\briefempfaengerindex{Beer-Hofmann, Richard@\textsc{Beer-Hofmann, Richard}!zzzSchnitzler, Arthur@\emph{von Arthur Schnitzler}!1895-06-071@{7. 6. 1895}|)be}\mylabel{L00450h}  \normalsize

\doendnotes{C}
\bigskip
\vfill

\clearpage

\footnotesize

\lohead{\textsc{register}}

% Definiere theindex-Environment komplett neu ohne reledmac
\makeatletter
\renewenvironment{theindex}{%
  \section*{\indexname}%
  \setlength{\parindent}{0pt}%
  \setlength{\parskip}{0pt plus 0.3pt}%
  \let\item\@idxitem
}{%
  \clearpage
}
\makeatother

\IfFileExists{\jobname-pw.ind}{\input{\jobname-pw.ind}}{}

\end{document}

      