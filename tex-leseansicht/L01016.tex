%% latex-korrekturansicht-vorspann.tex
%% Vorspann für die Korrekturansicht.
%% Lädt die gemeinsame Datei latex-vorspann.tex mit gesetztem Schalter.

\newif\ifkorrekturansicht
\korrekturansichttrue

\input{../tex-inputs/latex-vorspann}


\section[Richard Beer-Hofmann an Arthur Schnitzler, 22. 2. 1900]{L01016 Richard Beer-Hofmann an Arthur Schnitzler, 22. 2. 1900}
\nopagebreak\mylabel{L01016v}
\rehead{ }\normalsize\beginnumbering\briefempfaengerindex{Schnitzler, Arthur@\textsc{Schnitzler, Arthur}!zzzBeer-Hofmann, Richard@\emph{von Richard Beer-Hofmann}!1900-02-221@{22. 2. 1900}|(be}
\toendnotes[C]{\smallbreak\pagebreak[2]}\Standort{CUL, Schnitzler, B 8.}
\physDesc{Brief, 2 Blätter, 4 Seiten, 2975 Zeichen
\newline{}Handschrift: schwarze Tinte, lateinische Kurrent
\newline{}Ordnung: mit Bleistift von unbekannter Hand nummeriert:
                                    »151« }
\buchAbdrucke{\weitereDrucke{Arthur Schnitzler, Richard Beer-Hofmann: \emph{Briefwechsel 1891–1931}. Wien, Zürich: \emph{Europaverlag} 1992, S. 142–143.} }\toendnotes[C]{\smallbreak}
\pstart
           \raggedleft{}{\pb}Sanremo\oindex{Sanremo@\textbf{Sanremo}, \emph{P.PPLA3}|pw}{ }22/II 1900\pend
           \vspace{0.5em}
\pstart
           Mein lieber Arthur! »Beneiden«! Mein Gott! Wissen Sie was »beneiden«
               heißt? »Das Andere nicht wissen.« Im übrigen, dieser demonstrative »Süden« mit
               »Nachtkastel-Palmen«, der um 5 Uhr Abends die Maske abwirft, ist recht traurig.
               Überhaupt versagt Italien\oindex{Italien@\textbf{Italien}, \emph{A.PCLI}|pw} zum erstenmal bei
               mir; vielleicht wirds in Florenz\oindex{Florenz@\textbf{Florenz}, \emph{P.PPLA}|pw} besser. Ich
               vertrage es offenbar nicht irgendwohin direkt des schönen Wetters halber zu gehen.
               Sofort fang ich an aufs Wetter aufzupassen, bemerke wenn es blufft, und finde
               schließlich daß es, wie alle Dinge wenn man ihnen auf die Finger sieht, auch »in
               seinem Fach ein Esel« ist, und gar nicht weiß wie schönes Wetter eigentlich sein
               soll. Man darf gar nichts genau ansehen wollen; {\pb}Vielleicht heisst das große
               Geheimniß eines erträglichen Daseins: Oberflächlichkeit. Unsereiner, der einmal zu
               graben begonnen hat, kann freilich nicht mehr zurück; aber vielleicht geht es an so
               tief zu graben bis man auf der anderen Seite wieder herausko{\geminationm}t; das ist dann unsere »Oberflächlichkeit«. Der nächste
               Weg ist das nicht! »\label{K_L01016-1v}\edtext{Pollak wo hast Du
               Dein linkes Ohr?}{\lemma{\textnormal{\emph{Pollak … Ohr?}}}\Cendnote{\textnormal{Es handelt sich um eine stehende Redewendung für
                  den Griff mit der rechten Hand über den Kopf zum linken Ohr. Ein (jüdischer)
                  Junge, der vom Lehrer gefragt wurde, wo er sein linkes Ohr habe, soll diese
                  umständliche Geste gemacht haben. Vgl. Arthur Schnitzler an Richard Beer-Hofmann, 15. 10. 1894.}}}\label{K_L01016-1}«\pend
           
\pstart
           Meine Frau\pwindex{Beer-Hofmann, Paula 25.02.1879 – 30.10.1939@\textsc{Beer-Hofmann, Paula} (25.02.1879 – 30.10.1939)|pwv} hat sich bisher
               nicht erholt, ich habe hier einen Husten beko{\geminationm}en, die
               Einzige die sich wol fühlt ist Mirjam\pwindex{Beer-Hofmann, Mirjam 04.09.1897 – 24.12.1984@\textsc{Beer-Hofmann, Mirjam} (04.09.1897 – 24.12.1984)|pw}; bis sie
               größer sein wird, wirds schon besser werden. Frau Professor \label{K_L01016-2v}\edtext{Döppler\pwindex{Doepler, Berta 1822? – 1902-02-10@\textsc{Doepler, Berta} (1822? – 1902-02-10)|pw}}{\lemma{\textnormal{\emph{Döppler}}}\Cendnote{\textnormal{Berta Doepler\pwindex{Doepler, Berta 1822? – 1902-02-10@\textsc{Doepler, Berta} (1822? – 1902-02-10)|pwk} ist am
                     25. 7. 1895 auf der Kurliste von Bad Ischl\oindex{Bad Ischl@\textbf{Bad Ischl}, \emph{P.PPL}|pwk} verzeichnet, wodurch eine frühere Bekanntschaft anzunehmen
                  ist.}}}\label{K_L01016-2} habe ich hier getroffen und mir von ihr vortratschen lassen, was sie
               amüsant und eifrig hat; Ideenassociation: Elly
                  H.\pwindex{Petersen, Elly 26.02.1874 – 29.12.1965@\textsc{Petersen, Elly} (26.02.1874 – 29.12.1965), \emph{Schriftsteller/Schriftstellerin}|pw} hat sich richtig, wie ich herzloser Weise schon vorher zu Meyer\pwindex{Mayer, Oskar 1876 – 15.05.1915@\textsc{Mayer, Oskar} (1876 – 15.05.1915), \emph{Schriftsteller/Schriftstellerin, Beamter/Beamte}|pw}
                sagte, mit ihrer Krankheit eine {\pb}Position bei uns gemacht; man kann
               nicht sagen daß es mit wenig Einsatz geschehen ist. Wenn ihr Mann\pwindex{Hirschfeld, Georg 11.02.1873 – 17.01.1942@\textsc{Hirschfeld, Georg} (11.02.1873 – 17.01.1942), \emph{Schriftsteller/Schriftstellerin}|pwv} jetzt noch kein Geld verdienen würde,
               wäre er ein Dichter – für uns – nur um nicht roh zu sein. Frau Professor D.\pwindex{Doepler, Berta 1822? – 1902-02-10@\textsc{Doepler, Berta} (1822? – 1902-02-10)|pw} hat ihn – sie findet ihn überschätzt – mit
               dem Zeichner \uline{Allers\pwindex{Allers, Christian Wilhelm 1857-08-06 – 1915-10-19@\textsc{Allers, Christian Wilhelm} (1857-08-06 – 1915-10-19), \emph{Maler/Malerin, Zeichner/Zeichnerin, Illustrator/Illustratorin}|pw}} verglichen; wer von H.s\pwindex{Hirschfeld, Georg 11.02.1873 – 17.01.1942@\textsc{Hirschfeld, Georg} (11.02.1873 – 17.01.1942), \emph{Schriftsteller/Schriftstellerin}|pw} Freunden ihr das
               beigebracht haben mag? Auf ihrem eigenen Mist ist das nicht gewachsen; ich glaube
               übrigens sie hat überhaupt keinen eigenen Mist. Daß Sie sich die Lektüre von Georgs Tod\pwindex{Tod Georgs@\emph{Der Tod Georgs}|pw} für einen Frühlingstag auf dem Land
               aufheben ist sicher für das Buch gut; ob auch für den Tag? Wenn Sie mir durchaus das
                  Buch\pwindex{Wiener Bummelgeschichten@\emph{Wiener Bummelgeschichten}|pwv} des »dampfenden Jünglings\pwindex{Messer, Max 05.07.1875 – 25.12.1930@\textsc{Messer, Max} (05.07.1875 – 25.12.1930), \emph{Schriftsteller/Schriftstellerin, Journalist/Journalistin, Rechtsanwalt/Rechtsanwältin}|pwv}« schicken wollen,
               schicken Sie es nach Florenz\oindex{Florenz@\textbf{Florenz}, \emph{P.PPLA}|pw}, poste {\pb}restante. Nicht vielleicht deshalb
               weil ich hier bin, sondern weil ich am 27. dort sein will.\pend
           
\pstart
           Ich arbeite natürlich nichts. Von Hugo\pwindex{Hofmannsthal, Hugo von 1874-02-01 – 1929-07-15@\textsc{Hofmannsthal, Hugo von} (1874-02-01 – 1929-07-15), \emph{Schriftsteller/Schriftstellerin}|pw} habe
               ich keinerlei Nachricht. An Brandes\pwindex{Brandes, Georg 04.02.1842 – 19.02.1927@\textsc{Brandes, Georg} (04.02.1842 – 19.02.1927)|pw} habe ich
               heute mein Buch\pwindex{Tod Georgs@\emph{Der Tod Georgs}|pwv} geschickt Ich
               glaube nicht, daß er was damit anfangen kann. Auch Robert Hirschfeld\pwindex{Hirschfeld, Robert 17.09.1857 – 02.04.1914@\textsc{Hirschfeld, Robert} (17.09.1857 – 02.04.1914), \emph{Journalist/Journalistin, Musikkritiker/Musikkritikerin}|pw} der mich vor meiner Abreise becomplimentirte scheint keine
               Ahnung zu haben was der Inhalt des Buches\pwindex{Tod Georgs@\emph{Der Tod Georgs}|pwv} ist. Was macht Gustav\pwindex{Schwarzkopf, Gustav 07.11.1853 – 13.11.1939@\textsc{Schwarzkopf, Gustav} (07.11.1853 – 13.11.1939), \emph{Schriftsteller/Schriftstellerin}|pw}; während
               ich seinen Vornamen niederschreibe werde ich so verlegen, als sähe ich sein
               ungläubiges Lächeln zu dieser Intimität. Grüßen Sie ihn, dann à discretion die
               Übrigen, aber in gemessenen Distanzen.\pend
           
\pstart
           Wie ich meinen Brief überlese, finde ich daß er »\uline{witzig}« ist. »Gott sei
                  Dank er wird witzig\pwindex{Kabale und Liebe@\emph{Kabale und Liebe}|pwv}«! Aber der Hofmarschall Kalb\pwindex{Kabale und Liebe@\emph{Kabale und Liebe}|pwv}, der das sagt weiß nicht daß das für den Ferdinand\pwindex{Kabale und Liebe@\emph{Kabale und Liebe}|pwv} ein schlechtes
               Symptom ist. Für mich auch.\pend
           
\pstart
           Von Herzen\hspace*{1.5em}Ihr{\\[\baselineskip]}\spacefill\mbox{Richard}\pend
           \leftskip=0em{}\selectlanguage{ngerman}\endnumbering\briefempfaengerindex{Schnitzler, Arthur@\textsc{Schnitzler, Arthur}!zzzBeer-Hofmann, Richard@\emph{von Richard Beer-Hofmann}!1900-02-221@{22. 2. 1900}|)be}\mylabel{L01016h}  \normalsize

\doendnotes{C}
\bigskip
\vfill

\clearpage

\footnotesize

\lohead{\textsc{register}}

% Definiere theindex-Environment komplett neu ohne reledmac
\makeatletter
\renewenvironment{theindex}{%
  \section*{\indexname}%
  \setlength{\parindent}{0pt}%
  \setlength{\parskip}{0pt plus 0.3pt}%
  \let\item\@idxitem
}{%
  \clearpage
}
\makeatother

\IfFileExists{\jobname-pw.ind}{\input{\jobname-pw.ind}}{}

\end{document}

      