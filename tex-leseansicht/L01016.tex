%% latex-leseansicht-vorspann.tex
%% Vorspann für die Leseansicht.
%% Lädt die gemeinsame Datei latex-vorspann.tex mit nicht gesetztem Schalter.

\newif\ifkorrekturansicht
\korrekturansichtfalse

\input{../tex-inputs/latex-vorspann}


               \section[Richard Beer-Hofmann an Arthur Schnitzler, 22. 2. 1900]{ Richard Beer-Hofmann an Arthur Schnitzler, 22. 2. 1900}\nopagebreak\mylabel{v}\rehead{ }\begin{ledgroupsized}[t]{13cm}\normalsize\beginnumbering\briefempfaengerindex{Schnitzler, Arthur@\textsc{Schnitzler, Arthur}!zzzBeer-Hofmann, Richard@\emph{von Richard Beer-Hofmann}!1900-02-221@{22. 2. 1900}|(be} \toendnotes[C]{\smallbreak\pagebreak[2]} \Standort{CUL, Schnitzler, B 8.}
\physDesc{Brief, 2 Blätter, 4 Seiten
\newline{}Handschrift: schwarze Tinte, lateinische Kurrent\newline{}Ordnung: mit Bleistift von unbekannter Hand nummeriert: »151« }\buchAbdrucke{\weitereDrucke{Arthur Schnitzler, Richard Beer-Hofmann: \emph{Briefwechsel 1891–1931}. Hg. Konstanze Fliedl. Wien, Zürich: \emph{Europaverlag} 1992, S. 142–143.} }\toendnotes[C]{\smallbreak}\pstart
           \raggedleft{}{\pb}Sanremo\oindex{Sanremo@\textbf{Sanremo}|pw}{ }22/II 1900\pend
           \pstart
           Mein lieber Arthur! »Beneiden«! Mein Gott! Wissen Sie was »beneiden«
               heißt? »Das Andere nicht wissen.« Im übrigen, dieser demonstrative »Süden« mit
               »Nachtkastel-Palmen«, der um 5 Uhr Abends die Maske abwirft, ist recht traurig.
               Überhaupt versagt Italien\oindex{Italien@\textbf{Italien}|pw} zum erstenmal bei mir;
               vielleicht wirds in Florenz\oindex{Florenz@\textbf{Florenz}|pw} besser. Ich vertrage es
               offenbar nicht irgendwohin direkt des schönen Wetters halber zu gehen. Sofort fang
               ich an aufs Wetter aufzupassen, bemerke wenn es blufft, und finde schließlich daß es,
               wie alle Dinge wenn man ihnen auf die Finger sieht, auch »in seinem Fach ein Esel«
               ist, und gar nicht weiß wie schönes Wetter eigentlich sein soll. Man darf gar nichts
               genau ansehen wollen; {\pb}Vielleicht
               heisst das große Geheimniß eines erträglichen Daseins: Oberflächlichkeit. Unsereiner,
               der einmal zu graben begonnen hat, kann freilich nicht mehr zurück; aber vielleicht
               geht es an so tief zu graben bis man auf der anderen Seite wieder herausko{\geminationm}t; das ist dann unsere »Oberflächlichkeit«. Der nächste
               Weg ist das nicht! »\label{K_L01016_1v}\edtext{Pollak wo hast Du Dein linkes
                  Ohr?}{\lemma{\textnormal{\emph{Pollak … Ohr?}}}\Cendnote{\textnormal{Stehende
                  Redewendung für den Griff mit der rechten Hand über den Kopf zum linken Ohr. Ein
                  (jüdischer) Junge, der vom Lehrer gefragt wurde, wo er sein linkes Ohr habe, soll
                  diese umständliche Geste gemacht haben. Vgl. Arthur Schnitzler an Richard Beer-Hofmann, 15. 10. 1894}}}\label{K_L01016_1h}«\pend
           \pstart
           Meine Frau\pwindex{Beer-Hofmann, Paula 25.02.1879 – 30.10.1939@\textsc{Beer-Hofmann, Paula} (25.02.1879 – 30.10.1939)|pwv} hat sich bisher
               nicht erholt, ich habe hier einen Husten beko{\geminationm}en, die
               Einzige die sich wol fühlt ist Mirjam\pwindex{Beer-Hofmann, Mirjam 04.09.1897 – 24.12.1984@\textsc{Beer-Hofmann, Mirjam} (04.09.1897 – 24.12.1984)|pw}; bis sie
               größer sein wird, wirds schon besser werden. Frau Professor \label{K_L01016_2v}\edtext{Döppler\pwindex{Doepler, Berta 1822? – 1902-02-10@\textsc{Doepler, Berta} (1822? – 1902-02-10)|pw}}{\lemma{\textnormal{\emph{Döppler}}}\Cendnote{\textnormal{Berta Doepler\pwindex{Doepler, Berta 1822? – 1902-02-10@\textsc{Doepler, Berta} (1822? – 1902-02-10)|pwk} ist am 25. 7. 1895
                  auf der Kurliste von Bad Ischl\oindex{Bad Ischl@\textbf{Bad Ischl}|pwk} verzeichnet,
                  wodurch eine frühere Bekanntschaft anzunehmen ist.}}}\label{K_L01016_2h} habe ich hier getroffen
               und mir von ihr vortratschen lassen, was sie amüsant und eifrig hat;
               Ideenassociation: Elly H.\pwindex{Petersen, Elly 26.02.1874 – 29.12.1965@\textsc{Petersen, Elly} (26.02.1874 – 29.12.1965), \emph{Schriftstellerin}|pw} hat sich richtig, wie
               ich herzloser Weise schon vorher zu Meyer\pwindex{Mayer, Oskar 1876 – 15.05.1915@\textsc{Mayer, Oskar} (1876 – 15.05.1915), \emph{Schriftsteller, Beamter}|pw} sagte, mit ihrer Krankheit eine {\pb}Position bei uns gemacht; man kann
               nicht sagen daß es mit wenig Einsatz geschehen ist. Wenn ihr Mann\pwindex{Hirschfeld, Georg 11.02.1873 – 17.01.1942@\textsc{Hirschfeld, Georg} (11.02.1873 – 17.01.1942), \emph{Schriftsteller}|pwv} jetzt noch kein Geld verdienen würde,
               wäre er ein Dichter – für uns – nur um nicht roh zu sein. Frau Professor D.\pwindex{Doepler, Berta 1822? – 1902-02-10@\textsc{Doepler, Berta} (1822? – 1902-02-10)|pw} hat ihn – sie findet ihn überschätzt – mit dem
               Zeichner \uline{Allers\pwindex{Allers, Christian Wilhelm 1857-08-06 – 1915-10-19@\textsc{Allers, Christian Wilhelm} (1857-08-06 – 1915-10-19), \emph{Maler, Zeichner, Illustrator}|pw}} verglichen; wer von H.\pwindex{Hirschfeld, Georg 11.02.1873 – 17.01.1942@\textsc{Hirschfeld, Georg} (11.02.1873 – 17.01.1942), \emph{Schriftsteller}|pw}s Freunden ihr das
               beigebracht haben mag? Auf ihrem eigenen Mist ist das nicht gewachsen; ich glaube
               übrigens sie hat überhaupt keinen eigenen Mist. Daß Sie sich die Lektüre von Georgs Tod\pwindex{Beer-Hofmann, Richard 11.07.1866 – 26.09.1945@\textsc{Beer-Hofmann, Richard} (11.07.1866 – 26.09.1945), \emph{Schriftsteller}!Tod Georgs1900@\strich\emph{Der Tod Georgs} {[}1900{]}|pw} für einen Frühlingstag auf dem Land
               aufheben ist sicher für das Buch gut; ob auch für den Tag? Wenn Sie mir durchaus das
                  Buch\pwindex{Messer, Max 05.07.1875 – 25.12.1930@\textsc{Messer, Max} (05.07.1875 – 25.12.1930), \emph{Schriftsteller, Journalist, Rechtsanwalt}!Wiener Bummelgeschichten1900 – 1900@\strich\emph{Wiener Bummelgeschichten} {[}1900 – 1900{]}|pwv} des »dampfenden Jünglings\pwindex{Messer, Max 05.07.1875 – 25.12.1930@\textsc{Messer, Max} (05.07.1875 – 25.12.1930), \emph{Schriftsteller, Journalist, Rechtsanwalt}|pwv}« schicken wollen, schicken
               Sie es nach Florenz\oindex{Florenz@\textbf{Florenz}|pw}, poste {\pb}restante. Nicht vielleicht deshalb
               weil ich hier bin, sondern weil ich am 27. dort sein will.\pend
           \pstart
           Ich arbeite natürlich nichts. Von Hugo\pwindex{Hofmannsthal, Hugo von 01.02.1874 – 15.07.1929@\textsc{Hofmannsthal, Hugo von} (01.02.1874 – 15.07.1929), \emph{Schriftsteller}|pw} habe ich
               keinerlei Nachricht. An Brandes\pwindex{Brandes, Georg 04.02.1842 – 19.02.1927@\textsc{Brandes, Georg} (04.02.1842 – 19.02.1927)|pw} habe ich heute
               mein Buch\pwindex{Beer-Hofmann, Richard 11.07.1866 – 26.09.1945@\textsc{Beer-Hofmann, Richard} (11.07.1866 – 26.09.1945), \emph{Schriftsteller}!Tod Georgs1900@\strich\emph{Der Tod Georgs} {[}1900{]}|pwv} geschickt Ich glaube nicht, daß er was damit anfangen kann. Auch Robert Hirschfeld\pwindex{Hirschfeld, Robert 17.09.1857 – 02.04.1914@\textsc{Hirschfeld, Robert} (17.09.1857 – 02.04.1914), \emph{Journalist, Musikkritiker}|pw} der mich vor meiner Abreise
               becomplimentirte scheint keine Ahnung zu haben was der Inhalt des Buches\pwindex{Beer-Hofmann, Richard 11.07.1866 – 26.09.1945@\textsc{Beer-Hofmann, Richard} (11.07.1866 – 26.09.1945), \emph{Schriftsteller}!Tod Georgs1900@\strich\emph{Der Tod Georgs} {[}1900{]}|pwv} ist. Was macht Gustav\pwindex{Schwarzkopf, Gustav 07.11.1853 – 13.11.1939@\textsc{Schwarzkopf, Gustav} (07.11.1853 – 13.11.1939), \emph{Schriftsteller}|pw}; während ich seinen Vornamen niederschreibe werde ich so
               verlegen, als sähe ich sein ungläubiges Lächeln zu dieser Intimität. Grüßen Sie ihn,
               dann à discretion die Übrigen, aber in gemessenen Distanzen.\pend
           \pstart
           Wie ich meinen Brief überlese, finde ich daß er »\uline{witzig}« ist. »Gott sei Dank er wird witzig\pwindex{\textcolor{red}{\textsuperscript{XXXX1 indx}}!Kabale und Liebe1784@\strich\emph{Kabale und Liebe} {[}1784{]}|pwv}«! Aber der
               Hofmarschall Kalb\pwindex{\textcolor{red}{\textsuperscript{XXXX1 indx}}!Kabale und Liebe1784@\strich\emph{Kabale und Liebe} {[}1784{]}|pwv}, der das sagt
               weiß nicht daß das für den Ferdinand\pwindex{\textcolor{red}{\textsuperscript{XXXX1 indx}}!Kabale und Liebe1784@\strich\emph{Kabale und Liebe} {[}1784{]}|pwv} ein schlechtes Symptom ist. Für mich auch.\pend
           \pstart
           Von Herzen\hspace*{1.5em}Ihr{\\[\baselineskip]}\spacefill\mbox{Richard}\pend
           \leftskip=0em{}\endnumbering\briefempfaengerindex{Schnitzler, Arthur@\textsc{Schnitzler, Arthur}!zzzBeer-Hofmann, Richard@\emph{von Richard Beer-Hofmann}!1900-02-221@{22. 2. 1900}|)be}\mylabel{h}\end{ledgroupsized}  \newcommand{\dateiname}{L01016}\newcommand{\titel}{Richard Beer-Hofmann an Arthur Schnitzler, 22. 2. 1900}\newcommand{\editorInnen}{Martin Anton Müller und Gerd-Hermann Susen}%% latex-leseansicht-abspann.tex
%% Abspann für die Leseansicht.
%% Der Schalter \ifkorrekturansicht ist bereits durch den Vorspann gesetzt.

%% latex-abspann.tex
%% Gemeinsamer Abspann für Korrekturansicht und Leseansicht.
%% Setzt den Schalter \ifkorrekturansicht voraus (gesetzt in den
%% einbindenden Dateien latex-korrekturansicht-abspann.tex bzw.
%% latex-leseansicht-abspann.tex).
%% ---------------------------------------------------------------

\normalsize

% Das esempio-Environment wird nur in der Leseansicht benötigt
\ifkorrekturansicht\else
\newenvironment{esempio}[3]%
{
    \vspace{1.5ex}
    \rlap{\underline{#1}}
    \par
    \setlength{\parindent}{0cm}
    \nopagebreak
    \leftskip=#2cm
    \rightskip=#3cm
}
{
    \par
}
\fi

\doendnotes{C}
\bigskip
\vfill

\clearpage

\footnotesize

\ifkorrekturansicht
  \lohead{\textsc{register}}
\fi

% theindex-Environment neu definieren ohne reledmac
\makeatletter
\renewenvironment{theindex}{%
  \ifkorrekturansicht
    \section*{\indexname}%
  \else
    \subsubsection*{Index der erwähnten Entitäten}%
  \fi
  \setlength{\parindent}{0pt}%
  \setlength{\parskip}{0pt plus 0.3pt}%
  \let\item\@idxitem
}{%
  \ifkorrekturansicht\clearpage\fi
}
\makeatother

\IfFileExists{\jobname-pw.ind}{\input{\jobname-pw.ind}}{}

% Quellenangabe nur in der Leseansicht
\ifkorrekturansicht\else
% Fallback-Definitionen, falls die .tex-Datei \titel etc. nicht gesetzt hat
\providecommand{\titel}{}
\providecommand{\editorInnen}{}
\providecommand{\dateiname}{\jobname}

\vspace{3cm}

\vfill

\footnotesize
\textsc{Quelle}: \titel. Herausgegeben von {\editorInnen}. In: \emph{Arthur Schnitzler: Briefwechsel mit Autorinnen und Autoren}.
 Digitale Edition, https://schnitzler-briefe.acdh.oeaw.ac.at/{\dateiname}.html (Stand \today)
\fi

\end{document}


      