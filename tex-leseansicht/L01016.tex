%% latex-leseansicht-vorspann.tex
%% Vorspann für die Leseansicht.
%% Lädt die gemeinsame Datei latex-vorspann.tex mit nicht gesetztem Schalter.

\newif\ifkorrekturansicht
\korrekturansichtfalse

\input{../tex-inputs/latex-vorspann}


\section[Richard Beer-Hofmann an Arthur Schnitzler, 22. 2. 1900]{L01016 Richard Beer-Hofmann an Arthur Schnitzler, 22. 2. 1900}
\nopagebreak\mylabel{L01016v}
\rehead{ }\normalsize\beginnumbering\briefempfaengerindex{Schnitzler, Arthur@\textsc{Schnitzler, Arthur}!zzzBeer-Hofmann, Richard@\emph{von Richard Beer-Hofmann}!1900-02-221@{22. 2. 1900}|(be}
\toendnotes[C]{\smallbreak\pagebreak[2]}
\correspDesc{Versand  durch Richard Beer-Hofmann am 22. 2. 1900 in Sanremo
\newline{}Erhalt  durch Arthur Schnitzler im Zeitraum [23. 2. 1900
                  – 27. 2. 1900?] in Wien}\toendnotes[C]{\smallbreak}
\Standort{CUL, Schnitzler, B 8.}
\physDesc{Brief, 2 Blätter, 4 Seiten, 2975 Zeichen
\newline{}Handschrift: schwarze Tinte, lateinische Kurrent
\newline{}Ordnung: mit Bleistift von unbekannter Hand nummeriert:
                                    »151« }
\buchAbdrucke{\weitereDrucke{Arthur Schnitzler, Richard Beer-Hofmann: \emph{Briefwechsel 1891–1931}. Herausgegeben von Konstanze Fliedl. Wien, Zürich: \emph{Europaverlag} 1992, S. 142–143.} }\toendnotes[C]{\smallbreak}
\pstart
           \raggedleft{}{\pb}Sanremo\oindex{Sanremo@\textbf{Sanremo}, \emph{Hauptstadt}|pw}{ }22/II 1900\pend
           \vspace{0.5em}
\pstart
           Mein lieber Arthur! »Beneiden«! Mein Gott! Wissen Sie was »beneiden«
               heißt? »Das Andere nicht wissen.« Im übrigen, dieser demonstrative »Süden« mit
               »Nachtkastel-Palmen«, der um 5 Uhr Abends die Maske abwirft, ist recht traurig.
               Überhaupt versagt Italien\oindex{Italien@\textbf{Italien}|pw} zum erstenmal bei
               mir; vielleicht wirds in Florenz\oindex{Florenz@\textbf{Florenz}|pw} besser. Ich
               vertrage es offenbar nicht irgendwohin direkt des schönen Wetters halber zu gehen.
               Sofort fang ich an aufs Wetter aufzupassen, bemerke wenn es blufft, und finde
               schließlich daß es, wie alle Dinge wenn man ihnen auf die Finger sieht, auch »in
               seinem Fach ein Esel« ist, und gar nicht weiß wie schönes Wetter eigentlich sein
               soll. Man darf gar nichts genau ansehen wollen; {\pb}Vielleicht heisst das große
               Geheimniß eines erträglichen Daseins: Oberflächlichkeit. Unsereiner, der einmal zu
               graben begonnen hat, kann freilich nicht mehr zurück; aber vielleicht geht es an so
               tief zu graben bis man auf der anderen Seite wieder herausko{\geminationm}t; das ist dann unsere »Oberflächlichkeit«. Der nächste
               Weg ist das nicht! »\label{K_L01016-1v}\edtext{Pollak wo hast Du
               Dein linkes Ohr?}{\lemma{\textnormal{\emph{Pollak … Ohr?}}}\Cendnote{\textnormal{Es handelt sich um eine stehende Redewendung für
                  den Griff mit der rechten Hand über den Kopf zum linken Ohr. Ein (jüdischer)
                  Junge, der vom Lehrer gefragt wurde, wo er sein linkes Ohr habe, soll diese
                  umständliche Geste gemacht haben. Vgl. XXXX Auszeichnungsfehler: Dokument L00382 nicht gefunden.}}}\label{K_L01016-1}«\pend
           
\pstart
           Meine Frau\pwindex{Beer-Hofmann, Paula 25.\,2.\,1879 Wien – 30.\,10.\,1939 Zürich@\textsc{Beer-Hofmann, Paula} (25.\,2.\,1879 Wien – 30.\,10.\,1939 Zürich)|pwv} hat sich bisher
               nicht erholt, ich habe hier einen Husten beko{\geminationm}en, die
               Einzige die sich wol fühlt ist Mirjam\pwindex{Beer-Hofmann, Mirjam 4.\,9.\,1897 Wien – 24.\,12.\,1984 New York City@\textsc{Beer-Hofmann, Mirjam} (4.\,9.\,1897 Wien – 24.\,12.\,1984 New York City)|pw}; bis sie
               größer sein wird, wirds schon besser werden. Frau Professor \label{K_L01016-2v}\edtext{Döppler\pwindex{Doepler, Berta 1822? – 10.\,2.\,1902@\textsc{Doepler, Berta} (1822? – 10.\,2.\,1902)|pw}}{\lemma{\textnormal{\emph{Döppler}}}\Cendnote{\textnormal{Berta Doepler\pwindex{Doepler, Berta 1822? – 10.\,2.\,1902@\textsc{Doepler, Berta} (1822? – 10.\,2.\,1902)|pwk} ist am
                     25. 7. 1895 auf der Kurliste von Bad Ischl\oindex{Bad Ischl@\textbf{Bad Ischl}|pwk} verzeichnet, wodurch eine frühere Bekanntschaft anzunehmen
                  ist.}}}\label{K_L01016-2} habe ich hier getroffen und mir von ihr vortratschen lassen, was sie
               amüsant und eifrig hat; Ideenassociation: Elly
                  H.\pwindex{Petersen, Elly 26.\,2.\,1874 Berlin – 29.\,12.\,1965 München@\textsc{Petersen, Elly} (26.\,2.\,1874 Berlin – 29.\,12.\,1965 München), \emph{Schriftstellerin}|pw} hat sich richtig, wie ich herzloser Weise schon vorher zu Meyer\pwindex{Mayer, Oskar 1876 – 15.\,5.\,1915 München@\textsc{Mayer, Oskar} (1876 – 15.\,5.\,1915 München), \emph{Schriftsteller, Beamter}|pw}
                sagte, mit ihrer Krankheit eine {\pb}Position bei uns gemacht; man kann
               nicht sagen daß es mit wenig Einsatz geschehen ist. Wenn ihr Mann\pwindex{Hirschfeld, Georg 11.\,2.\,1873 Berlin – 17.\,1.\,1942 München@\textsc{Hirschfeld, Georg} (11.\,2.\,1873 Berlin – 17.\,1.\,1942 München), \emph{Schriftsteller}|pwv} jetzt noch kein Geld verdienen würde,
               wäre er ein Dichter – für uns – nur um nicht roh zu sein. Frau Professor D.\pwindex{Doepler, Berta 1822? – 10.\,2.\,1902@\textsc{Doepler, Berta} (1822? – 10.\,2.\,1902)|pw} hat ihn – sie findet ihn überschätzt – mit
               dem Zeichner \uline{Allers\pwindex{Allers, Christian Wilhelm 6.\,8.\,1857 Hamburg – 19.\,10.\,1915 Karlsruhe@\textsc{Allers, Christian Wilhelm} (6.\,8.\,1857 Hamburg – 19.\,10.\,1915 Karlsruhe), \emph{Maler, Zeichner, Illustrator}|pw}} verglichen; wer von H.s\pwindex{Hirschfeld, Georg 11.\,2.\,1873 Berlin – 17.\,1.\,1942 München@\textsc{Hirschfeld, Georg} (11.\,2.\,1873 Berlin – 17.\,1.\,1942 München), \emph{Schriftsteller}|pw} Freunden ihr das
               beigebracht haben mag? Auf ihrem eigenen Mist ist das nicht gewachsen; ich glaube
               übrigens sie hat überhaupt keinen eigenen Mist. Daß Sie sich die Lektüre von Georgs Tod\pwindex{Beer-Hofmann, Richard 11.\,7.\,1866 Wien – 26.\,9.\,1945 New York City@\textsc{Beer-Hofmann, Richard} (11.\,7.\,1866 Wien – 26.\,9.\,1945 New York City), \emph{Schriftsteller}!Tod Georgs@\strich\emph{Der Tod Georgs}|pw} für einen Frühlingstag auf dem Land
               aufheben ist sicher für das Buch gut; ob auch für den Tag? Wenn Sie mir durchaus das
                  Buch\pwindex{Messer, Max 5.\,7.\,1875 Wien – 25.\,12.\,1930 ebd.@\textsc{Messer, Max} (5.\,7.\,1875 Wien – 25.\,12.\,1930 ebd.), \emph{Schriftsteller, Journalist, Rechtsanwalt}!Wiener Bummelgeschichten@\strich\emph{Wiener Bummelgeschichten}|pwv} des »dampfenden Jünglings\pwindex{Messer, Max 5.\,7.\,1875 Wien – 25.\,12.\,1930 ebd.@\textsc{Messer, Max} (5.\,7.\,1875 Wien – 25.\,12.\,1930 ebd.), \emph{Schriftsteller, Journalist, Rechtsanwalt}|pwv}« schicken wollen,
               schicken Sie es nach Florenz\oindex{Florenz@\textbf{Florenz}|pw}, poste {\pb}restante. Nicht vielleicht deshalb
               weil ich hier bin, sondern weil ich am 27. dort sein will.\pend
           
\pstart
           Ich arbeite natürlich nichts. Von Hugo\pwindex{Hofmannsthal, Hugo von 1.\,2.\,1874 Wien – 15.\,7.\,1929 Rodaun@\textsc{Hofmannsthal, Hugo von} (1.\,2.\,1874 Wien – 15.\,7.\,1929 Rodaun), \emph{Schriftsteller}|pw} habe
               ich keinerlei Nachricht. An Brandes\pwindex{Brandes, Georg 4.\,2.\,1842 Kopenhagen – 19.\,2.\,1927 ebd.@\textsc{Brandes, Georg} (4.\,2.\,1842 Kopenhagen – 19.\,2.\,1927 ebd.)|pw} habe ich
               heute mein Buch\pwindex{Beer-Hofmann, Richard 11.\,7.\,1866 Wien – 26.\,9.\,1945 New York City@\textsc{Beer-Hofmann, Richard} (11.\,7.\,1866 Wien – 26.\,9.\,1945 New York City), \emph{Schriftsteller}!Tod Georgs@\strich\emph{Der Tod Georgs}|pwv} geschickt Ich
               glaube nicht, daß er was damit anfangen kann. Auch Robert Hirschfeld\pwindex{Hirschfeld, Robert 17.\,9.\,1857 Žďár nad Sázavou – 2.\,4.\,1914 Salzburg@\textsc{Hirschfeld, Robert} (17.\,9.\,1857 Žďár nad Sázavou – 2.\,4.\,1914 Salzburg), \emph{Journalist, Musikkritiker}|pw} der mich vor meiner Abreise becomplimentirte scheint keine
               Ahnung zu haben was der Inhalt des Buches\pwindex{Beer-Hofmann, Richard 11.\,7.\,1866 Wien – 26.\,9.\,1945 New York City@\textsc{Beer-Hofmann, Richard} (11.\,7.\,1866 Wien – 26.\,9.\,1945 New York City), \emph{Schriftsteller}!Tod Georgs@\strich\emph{Der Tod Georgs}|pwv} ist. Was macht Gustav\pwindex{Schwarzkopf, Gustav 7.\,11.\,1853 Wien – 13.\,11.\,1939 ebd.@\textsc{Schwarzkopf, Gustav} (7.\,11.\,1853 Wien – 13.\,11.\,1939 ebd.), \emph{Schriftsteller}|pw}; während
               ich seinen Vornamen niederschreibe werde ich so verlegen, als sähe ich sein
               ungläubiges Lächeln zu dieser Intimität. Grüßen Sie ihn, dann à discretion die
               Übrigen, aber in gemessenen Distanzen.\pend
           
\pstart
           Wie ich meinen Brief überlese, finde ich daß er »\uline{witzig}« ist. »Gott sei
                  Dank er wird witzig\pwindex{\textcolor{red}{\textsuperscript{XXXX indx1}}!Kabale und Liebe. Ein bürgerliches Trauerspiel in fünf Aufzügen@\strich\emph{Kabale und Liebe. Ein bürgerliches Trauerspiel in fünf Aufzügen}|pwv}«! Aber der Hofmarschall Kalb\pwindex{\textcolor{red}{\textsuperscript{XXXX indx1}}!Kabale und Liebe. Ein bürgerliches Trauerspiel in fünf Aufzügen@\strich\emph{Kabale und Liebe. Ein bürgerliches Trauerspiel in fünf Aufzügen}|pwv}, der das sagt weiß nicht daß das für den Ferdinand\pwindex{\textcolor{red}{\textsuperscript{XXXX indx1}}!Kabale und Liebe. Ein bürgerliches Trauerspiel in fünf Aufzügen@\strich\emph{Kabale und Liebe. Ein bürgerliches Trauerspiel in fünf Aufzügen}|pwv} ein schlechtes
               Symptom ist. Für mich auch.\pend
           
\pstart
           Von Herzen\hspace*{1.5em}Ihr{\\[\baselineskip]}\spacefill\mbox{Richard}\pend
           \leftskip=0em{}\selectlanguage{ngerman}\endnumbering\briefempfaengerindex{Schnitzler, Arthur@\textsc{Schnitzler, Arthur}!zzzBeer-Hofmann, Richard@\emph{von Richard Beer-Hofmann}!1900-02-221@{22. 2. 1900}|)be}\mylabel{L01016h}  \newcommand{\dateiname}{L01016}\newcommand{\titel}{Richard Beer-Hofmann an Arthur Schnitzler, 22. 2. 1900}\newcommand{\editorInnen}{Martin Anton Müller und Gerd-Hermann Susen}%% latex-leseansicht-abspann.tex
%% Abspann für die Leseansicht.
%% Der Schalter \ifkorrekturansicht ist bereits durch den Vorspann gesetzt.

%% latex-abspann.tex
%% Gemeinsamer Abspann für Korrekturansicht und Leseansicht.
%% Setzt den Schalter \ifkorrekturansicht voraus (gesetzt in den
%% einbindenden Dateien latex-korrekturansicht-abspann.tex bzw.
%% latex-leseansicht-abspann.tex).
%% ---------------------------------------------------------------

\normalsize

% Das esempio-Environment wird nur in der Leseansicht benötigt
\ifkorrekturansicht\else
\newenvironment{esempio}[3]%
{
    \vspace{1.5ex}
    \rlap{\underline{#1}}
    \par
    \setlength{\parindent}{0cm}
    \nopagebreak
    \leftskip=#2cm
    \rightskip=#3cm
}
{
    \par
}
\fi

\doendnotes{C}
\bigskip
\vfill

\clearpage

\footnotesize

\ifkorrekturansicht
  \lohead{\textsc{register}}
\fi

% theindex-Environment neu definieren ohne reledmac
\makeatletter
\renewenvironment{theindex}{%
  \ifkorrekturansicht
    \section*{\indexname}%
  \else
    \subsubsection*{Index der erwähnten Entitäten}%
  \fi
  \setlength{\parindent}{0pt}%
  \setlength{\parskip}{0pt plus 0.3pt}%
  \let\item\@idxitem
}{%
  \ifkorrekturansicht\clearpage\fi
}
\makeatother

\IfFileExists{\jobname-pw.ind}{\input{\jobname-pw.ind}}{}

% Quellenangabe nur in der Leseansicht
\ifkorrekturansicht\else
% Fallback-Definitionen, falls die .tex-Datei \titel etc. nicht gesetzt hat
\providecommand{\titel}{}
\providecommand{\editorInnen}{}
\providecommand{\dateiname}{\jobname}

\vspace{3cm}

\vfill

\footnotesize
\textsc{Quelle}: \titel. Herausgegeben von {\editorInnen}. In: \emph{Arthur Schnitzler: Briefwechsel mit Autorinnen und Autoren}.
 Digitale Edition, https://schnitzler-briefe.acdh.oeaw.ac.at/{\dateiname}.html (Stand \today)
\fi

\end{document}


