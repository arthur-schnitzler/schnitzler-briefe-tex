%% latex-leseansicht-vorspann.tex
%% Vorspann für die Leseansicht.
%% Lädt die gemeinsame Datei latex-vorspann.tex mit nicht gesetztem Schalter.

\newif\ifkorrekturansicht
\korrekturansichtfalse

\input{../tex-inputs/latex-vorspann}


\section[Arthur und Olga Schnitzler an Hugo von Hofmannsthal, 22. 5. 1904]{L01403 Arthur und Olga Schnitzler an Hugo von Hofmannsthal, 22. 5. 1904}
\nopagebreak\mylabel{L01403v}
\rehead{ }\normalsize\beginnumbering\briefempfaengerindex{Hofmannsthal, Hugo von@\textsc{Hofmannsthal, Hugo von}!zzzSchnitzler, Olga@\emph{von Olga Schnitzler}!1904-05-221@{22. 5. 1904}|(be}\briefempfaengerindex{Hofmannsthal, Hugo von@\textsc{Hofmannsthal, Hugo von}!zzzSchnitzler, Arthur@\emph{von Arthur Schnitzler}!1904-05-221@{22. 5. 1904}|(be}
\toendnotes[C]{\smallbreak\pagebreak[2]}
\correspDesc{Versand  durch Arthur Schnitzler, Olga Schnitzler am 22. 5. 1904 in Taormina
\newline{}Erhalt  durch Hugo von Hofmannsthal im Zeitraum [23. 5. 1904
                  – 27. 5. 1904?] in Wien}\toendnotes[C]{\smallbreak}
\Standort{FDH, Hs-30885,7.}
\physDesc{Bildpostkarte, 168 Zeichen
\newline{}Handschrift Arthur Schnitzler: Bleistift, deutsche Kurrent
\newline{}Handschrift Olga Schnitzler: Bleistift
\newline{}Versand: 1) Stempel: »\nobreak{}\oindex{Taormina@\textbf{Taormina}, \emph{Hauptstadt}|pwk}{[}Taormi{]}na
                                       \textcolor{gray}{Mes}sina\nobreak{}«.   2) Stempel: »\nobreak{}\oindex{Wien@\textbf{Wien}!XXIII., Liesing@\textbf{XXIII., Liesing}!Badgasse@\textbf{Badgasse}, \emph{Straße}|pwk}Rodaun\nobreak{}«. }
\buchAbdrucke{\weitereDrucke{Hugo von Hofmannsthal, Arthur Schnitzler: \emph{Briefwechsel}. Herausgegeben von Therese Nickl und Heinrich Schnitzler. Frankfurt am Main: \emph{S. Fischer} 1964, S. 187.} }\pstart{}{\pb}Herrn \textsc{Dr Hugo von
                     Hofmannsthal}\pend{}\pstart{}\textsc{Rodaun bei Wien\oindex{Wien@\textbf{Wien}!XXIII., Liesing@\textbf{XXIII., Liesing}!Rodaun@\textbf{Rodaun}, \emph{Region}|pw}}\pend{}\pstart{}\textsc{Badgasse 5\oindex{Wien@\textbf{Wien}!XXIII., Liesing@\textbf{XXIII., Liesing}!Badgasse@\textbf{Badgasse}, \emph{Straße}|pw}}\pend{}\pstart{}\textsc{Austria\oindex{Österreich@\textbf{Österreich}|pw}}\pend{}{\bigskip}
\pstart
           \noindent{}\centering{}{\pb}\textcolor{gray}{\textbf{Taormina\oindex{Taormina@\textbf{Taormina}, \emph{Hauptstadt}|pw}}}\pend
           
\pstart
           \centering{}\textcolor{gray}{\textbf{Castel Mola\oindex{Castelmola@\textbf{Castelmola}, \emph{Verwaltungsgebiet}|pw} dal Grand Hôtel S. Domenico\oindex{Grand Hotel San Domenico@\textbf{Grand Hotel San Domenico}, \emph{Hotel}|pw}}}\pend
           \vspace{1em}
\pstart
           {\pb}22. 5. 904\pend
           \vspace{0.5em}
\pstart
           Herzliche Grüße, die Sie wohl nach Holland\oindex{Niederlande@\textbf{Niederlande}|pw}
               nachgeſchickt bekommen? Auch Gerty\pwindex{Hofmannsthal, Gertrude von 16.\,3.\,1880 Wien – 9.\,11.\,1959 Paddington@\textsc{Hofmannsthal, Gertrude von} (16.\,3.\,1880 Wien – 9.\,11.\,1959 Paddington)|pw} u Ihrem Papa\pwindex{Hofmannsthal, Hugo August von 21.\,12.\,1841 Wien – 8.\,12.\,1915 ebd.@\textsc{Hofmannsthal, Hugo August von} (21.\,12.\,1841 Wien – 8.\,12.\,1915 ebd.), \emph{Bankdirektor}|pw}. –\pend
           
\pstart
           Ihr \spacefill\mbox{A.}{\\[\baselineskip]}\spacefill\mbox{{[}hs. Schnitzler:{]} Olga}\pend
           \leftskip=0em{}\selectlanguage{ngerman}\endnumbering\briefempfaengerindex{Hofmannsthal, Hugo von@\textsc{Hofmannsthal, Hugo von}!zzzSchnitzler, Olga@\emph{von Olga Schnitzler}!1904-05-221@{22. 5. 1904}|)be}\briefempfaengerindex{Hofmannsthal, Hugo von@\textsc{Hofmannsthal, Hugo von}!zzzSchnitzler, Arthur@\emph{von Arthur Schnitzler}!1904-05-221@{22. 5. 1904}|)be}\mylabel{L01403h}  \newcommand{\dateiname}{L01403}\newcommand{\titel}{Arthur und Olga Schnitzler an Hugo von Hofmannsthal, 22. 5. 1904}\newcommand{\editorInnen}{Martin Anton Müller und Gerd-Hermann Susen}%% latex-leseansicht-abspann.tex
%% Abspann für die Leseansicht.
%% Der Schalter \ifkorrekturansicht ist bereits durch den Vorspann gesetzt.

%% latex-abspann.tex
%% Gemeinsamer Abspann für Korrekturansicht und Leseansicht.
%% Setzt den Schalter \ifkorrekturansicht voraus (gesetzt in den
%% einbindenden Dateien latex-korrekturansicht-abspann.tex bzw.
%% latex-leseansicht-abspann.tex).
%% ---------------------------------------------------------------

\normalsize

% Das esempio-Environment wird nur in der Leseansicht benötigt
\ifkorrekturansicht\else
\newenvironment{esempio}[3]%
{
    \vspace{1.5ex}
    \rlap{\underline{#1}}
    \par
    \setlength{\parindent}{0cm}
    \nopagebreak
    \leftskip=#2cm
    \rightskip=#3cm
}
{
    \par
}
\fi

\doendnotes{C}
\bigskip
\vfill

\clearpage

\footnotesize

\ifkorrekturansicht
  \lohead{\textsc{register}}
\fi

% theindex-Environment neu definieren ohne reledmac
\makeatletter
\renewenvironment{theindex}{%
  \ifkorrekturansicht
    \section*{\indexname}%
  \else
    \subsubsection*{Index der erwähnten Entitäten}%
  \fi
  \setlength{\parindent}{0pt}%
  \setlength{\parskip}{0pt plus 0.3pt}%
  \let\item\@idxitem
}{%
  \ifkorrekturansicht\clearpage\fi
}
\makeatother

\IfFileExists{\jobname-pw.ind}{\input{\jobname-pw.ind}}{}

% Quellenangabe nur in der Leseansicht
\ifkorrekturansicht\else
% Fallback-Definitionen, falls die .tex-Datei \titel etc. nicht gesetzt hat
\providecommand{\titel}{}
\providecommand{\editorInnen}{}
\providecommand{\dateiname}{\jobname}

\vspace{3cm}

\vfill

\footnotesize
\textsc{Quelle}: \titel. Herausgegeben von {\editorInnen}. In: \emph{Arthur Schnitzler: Briefwechsel mit Autorinnen und Autoren}.
 Digitale Edition, https://schnitzler-briefe.acdh.oeaw.ac.at/{\dateiname}.html (Stand \today)
\fi

\end{document}


