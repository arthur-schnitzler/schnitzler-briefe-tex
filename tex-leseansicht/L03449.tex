%% latex-leseansicht-vorspann.tex
%% Vorspann für die Leseansicht.
%% Lädt die gemeinsame Datei latex-vorspann.tex mit nicht gesetztem Schalter.

\newif\ifkorrekturansicht
\korrekturansichtfalse

\input{../tex-inputs/latex-vorspann}


\section[ Paul Goldmann an Arthur Schnitzler, 4. 8. [1904]]{L03449 Paul Goldmann an Arthur Schnitzler,  4. 8. [1904]}
\nopagebreak\mylabel{L03449v}
\rehead{ }\normalsize\beginnumbering\briefempfaengerindex{Schnitzler, Arthur@\textsc{Schnitzler, Arthur}!zzzGoldmann, Paul@\emph{von Paul Goldmann}!1904-08-041@{4. 8. [1904]}|(be}
\toendnotes[C]{\smallbreak\pagebreak[2]}
\correspDesc{Versand  durch Paul Goldmann am 4. 8. [1904] in Berlin
\newline{}Erhalt  durch Arthur Schnitzler im Zeitraum [5. 8. 1904
                  – 9. 8. 1904?] in Wien}\toendnotes[C]{\smallbreak}
\Standort{DLA, A:Schnitzler, HS.NZ85.1.3174.}
\physDesc{Brief, 2 Blätter, 6 Seiten, 2223 Zeichen
\newline{}Handschrift: blaue Tinte, deutsche Kurrent
\newline{}Schnitzler: 1) mit Bleistift das Jahr »904« vermerkt  2) mit rotem Buntstift drei Unterstreichungen}\toendnotes[C]{\smallbreak}
\pstart
           \raggedleft{}{\pb}\textcolor{gray}{\textbf{DESSAUERSTRASSE 19\oindex{Dessauer Straße@\textbf{Dessauer Straße}, \emph{Straße}|pw}}}\pend
           
\pstart
           Berlin\oindex{Berlin@\textbf{Berlin}, \emph{Hauptstadt}|pw}, 4. Auguſt.\pend
           
\pstart{}Mein lieber Freund,\pend\vspace{0.5em}
\pstart
           \textsc{Peter Dorners\pwindex{Dorner, Peter 17.\,2.\,1857 Welsberg-Taisten – 1.\,4.\,1931 ebd.@\textsc{Dorner, Peter} (17.\,2.\,1857 Welsberg-Taisten – 1.\,4.\,1931 ebd.), \emph{Schmied, Kunsthandwerker, Kunstschmied}|pw}} Verlobung mag im
               Zuſammenhang mit{ }ſeinem \label{K_L03449-1v}\edtext{»Ruhm«}{\lemma{\textnormal{\emph{»Ruhm«}}}\Cendnote{\textnormal{Der Kunstschmied Peter Dorner\pwindex{Dorner, Peter 17.\,2.\,1857 Welsberg-Taisten – 1.\,4.\,1931 ebd.@\textsc{Dorner, Peter} (17.\,2.\,1857 Welsberg-Taisten – 1.\,4.\,1931 ebd.), \emph{Schmied, Kunsthandwerker, Kunstschmied}|pwk} war wegen seiner Vorliebe, Schlangen
                  darzustellen, als »Schlangenschmied von Welsberg\oindex{Welsberg-Taisten@\textbf{Welsberg-Taisten}, \emph{Verwaltungsgebiet}|pwk}« bekannt. Am 28. 4. 1904 hatte er
                  in der \emph{Gießerei Gladenbeck}\orgindex{H. Gladenbeck und Sohn Bildgießerei@H. Gladenbeck {\kaufmannsund}  Sohn Bildgießerei|pwk} seine Arbeiten
                  erstmals in Berlin\oindex{Berlin@\textbf{Berlin}, \emph{Hauptstadt}|pwk} ausgestellt.}}}\label{K_L03449-1}{ }ſtehen.
               Sicherlich aber macht er eine \label{K_L03449-2v}\edtext{»gute Parthie\pwindex{?? [Verlobte von Peter Dorner] @\textsc{?? [Verlobte von Peter Dorner]}|pwv}«}{\lemma{\textnormal{\emph{»gute Parthie«}}}\Cendnote{\textnormal{nicht ermittelt}}}\label{K_L03449-2}. Das Haus des Schwiegervaters\pwindex{?? [Vater von Peter Dorners Verlobter] @\textsc{?? [Vater von Peter Dorners Verlobter]}|pwv} auf der
               Karte, die er mir geſchickt hat, \strikeout{ſpricht} läßt das mit
               aller Deutlichkeit erkennen. Ich habe dieſen \textsc{Peter Dorner\pwindex{Dorner, Peter 17.\,2.\,1857 Welsberg-Taisten – 1.\,4.\,1931 ebd.@\textsc{Dorner, Peter} (17.\,2.\,1857 Welsberg-Taisten – 1.\,4.\,1931 ebd.), \emph{Schmied, Kunsthandwerker, Kunstschmied}|pw}}, den »weltverlorenen \strikeout{\textcolor{gray}{×}} Bauern«{[},{]}{ }hier\oindex{Berlin@\textbf{Berlin}, \emph{Hauptstadt}|pwv} als einen \strikeout{G\textcolor{gray}{e} G\textcolor{gray}{ef}\textcolor{gray}{×}\-\textcolor{gray}{×}\-\textcolor{gray}{×}\-\textcolor{gray}{×}\-\textcolor{gray}{×}\-\textcolor{gray}{×}} Gefühlsmann kennen gelernt, der die geriſſenſten Börſenjuden übertrifft.\pend
           
\pstart
           Sehr bedauert habe ich, zu erfahren, {\pb}daß Du \substVorne{}\textsuperscript{\textcolor{gray}{×}\-\textcolor{gray}{×}\-\textcolor{gray}{×}}\substDazwischen{}14\substHinten{} Tage \label{K_L03449-3v}\edtext{krank}{\lemma{\textnormal{\emph{krank}}}\Cendnote{\textnormal{Vgl. A. S.: \emph{Tagebuch}, 18. 7. 1904 bis 23. 7. 1904.
               }}}\label{K_L03449-3} warſt. Hoffentlich haſt Du, außer einiger »Gelbheit«, keine großen
               Beſchwerden gehabt, und ich freue mich, daß Du wiederhergeſtellt und arbeitsluſtig
               und arbeitskräftig biſt.\pend
           
\pstart
           Der \label{K_L03449-4v}\edtext{Tod \textsc{Herzls\pwindex{Herzl, Theodor 2.\,5.\,1860 Budapest – 3.\,7.\,1904 Edlach@\textsc{Herzl, Theodor} (2.\,5.\,1860 Budapest – 3.\,7.\,1904 Edlach), \emph{Schriftsteller, Journalist}|pw}}}{\lemma{\textnormal{\emph{Tod Herzls}}}\Cendnote{\textnormal{Theodor Herzl\pwindex{Herzl, Theodor 2.\,5.\,1860 Budapest – 3.\,7.\,1904 Edlach@\textsc{Herzl, Theodor} (2.\,5.\,1860 Budapest – 3.\,7.\,1904 Edlach), \emph{Schriftsteller, Journalist}|pwk} war am 3. 7. 1904 an Herzleiden verstorben.}}}\label{K_L03449-4} hat auch
               mich{ }ſehr ergriffen. Er war der Anſtändigſten und Begabteſten \strikeout{e\textcolor{gray}{i}} einer, und \strikeout{\textcolor{gray}{×}} man{ }ſchätzt ihn umſo höher, wenn man \strikeout{bed\textcolor{gray}{enkt, was nac}h} ihn mit dem Nachwuchs vergleicht.
               Nur was{ }ſeinen \label{K_L03449-5v}\edtext{zioniſtiſchen
                  Lebensplan}{\lemma{\textnormal{\emph{zionistischen
                  Lebensplan}}}\Cendnote{\textnormal{Siehe zu Goldmanns\pwindex{Goldmann, Paul 31.\,1.\,1865 Breslau – 25.\,9.\,1935 Wien@\textsc{Goldmann, Paul} (31.\,1.\,1865 Breslau – 25.\,9.\,1935 Wien), \emph{Schriftsteller, Journalist}|pwk} Ablehnung gegenüber Herzls\pwindex{Herzl, Theodor 2.\,5.\,1860 Budapest – 3.\,7.\,1904 Edlach@\textsc{Herzl, Theodor} (2.\,5.\,1860 Budapest – 3.\,7.\,1904 Edlach), \emph{Schriftsteller, Journalist}|pwk} zionistischen Visionen etwa XXXX Auszeichnungsfehler: Dokument L02742 nicht gefunden, XXXX Auszeichnungsfehler: Dokument L02769 nicht gefunden und XXXX Auszeichnungsfehler: Dokument L02841 nicht gefunden.
               }}}\label{K_L03449-5} anlangt,{ }ſo iſt {\pb}er, glaube ich, zur rechten
               Zeit geſtorben. Denn die Bewegung{ }ſtand, wie ich höre, am Vorabend{ }ſchwerer \label{K_L03449-6v}\edtext{Kriſen}{\lemma{\textnormal{\emph{Krisen}}}\Cendnote{\textnormal{womöglich Bezug auf die wiederholte Ablehnung eines jüdischen
                  Staats durch Autoritäten wie Papst Pius X.\pwindex{Pius X. 2.\,6.\,1835 Riese Pio X – 20.\,8.\,1914 Rom@\textsc{Pius X.} (2.\,6.\,1835 Riese Pio X – 20.\,8.\,1914 Rom), \emph{Papst}|pwk}
                  und Kaiser Wilhelm II.\pwindex{Wilhelm II. von Preußen 27.\,1.\,1859 Potsdam – 4.\,6.\,1941 Gemeente Utrechtse Heuvelrug@\textsc{Wilhelm II. von Preußen} (27.\,1.\,1859 Potsdam – 4.\,6.\,1941 Gemeente Utrechtse Heuvelrug), \emph{Kaiser}|pwk}}}}\label{K_L03449-6}.\pend
           
\pstart
           Daß ich{ }ſein \label{K_L03449-7v}\edtext{Nachfolger\orgindex{Neue Freie Presse@Neue Freie Presse|pwv}}{\lemma{\textnormal{\emph{Nachfolger}}}\Cendnote{\textnormal{als Feuilletonredakteur der \emph{Neuen Freien Presse}\orgindex{Neue Freie Presse@Neue Freie Presse|pwk}}}}\label{K_L03449-7} werde, halte ich für ausgeſchloſſen. Die Herausgeber\pwindex{Bacher, Eduard 7.\,3.\,1846 Postoloprty – 16.\,1.\,1908 Wien@\textsc{Bacher, Eduard} (7.\,3.\,1846 Postoloprty – 16.\,1.\,1908 Wien), \emph{Journalist, Herausgeber}|pwv}\pwindex{Benedikt, Moriz 27.\,5.\,1849 Kvačice – 18.\,3.\,1920 Wien@\textsc{Benedikt, Moriz} (27.\,5.\,1849 Kvačice – 18.\,3.\,1920 Wien), \emph{Journalist, Herausgeber}|pwv} machen keine Anſtalten, mir die
               Stellung anzubieten, und ich habe nicht die Abſicht, mich darum zu bewerben, da die
               Stellung mir nicht die Freiheit gewährt, zu leiſten, was ich leiſten möchte, und da
               außerdem meine Luſt, nach Wien\oindex{Wien@\textbf{Wien}, \emph{Verwaltungsgebiet}|pw} zurückzukehren,
               immer geringer wird.\pend
           
\pstart
           {\pb}Meine \label{K_L03449-8v}\edtext{Äußerung}{\lemma{\textnormal{\emph{Äußerung}}}\Cendnote{\textnormal{Siehe XXXX Auszeichnungsfehler: Dokument L03445 nicht gefunden und XXXX Auszeichnungsfehler: Dokument L01406 nicht gefunden. }}}\label{K_L03449-8} über \textsc{Hoffmannsthal\pwindex{Hofmannsthal, Hugo von 1.\,2.\,1874 Wien – 15.\,7.\,1929 Rodaun@\textsc{Hofmannsthal, Hugo von} (1.\,2.\,1874 Wien – 15.\,7.\,1929 Rodaun), \emph{Schriftsteller}|pw}} haſt Du wieder einmal mißverſtanden. Mich hat es nicht überraſcht, daß Du die
               Fehler, die Deine Freunde begehen, offen als{ }ſolche bezeichneſt (ich kenne Deine
               Offenheit{ }ſehr wohl und{ }ſchätze{ }ſie{ }ſehr hoch),{ }ſondern mich hat es überraſcht, daß
               Du einen Fehler \textsc{Hoffmannsthals\pwindex{Hofmannsthal, Hugo von 1.\,2.\,1874 Wien – 15.\,7.\,1929 Rodaun@\textsc{Hofmannsthal, Hugo von} (1.\,2.\,1874 Wien – 15.\,7.\,1929 Rodaun), \emph{Schriftsteller}|pw}} als{ }ſolchen erkannt
               haſt, da Du{ }ſonſt, meiner Anſicht nach, \textsc{Hoffmannsthal\pwindex{Hofmannsthal, Hugo von 1.\,2.\,1874 Wien – 15.\,7.\,1929 Rodaun@\textsc{Hofmannsthal, Hugo von} (1.\,2.\,1874 Wien – 15.\,7.\,1929 Rodaun), \emph{Schriftsteller}|pw}}{ }{\pb}nicht richtig beurtheilſt. Im Übrigen überraſcht
               mich wieder der Ausdruck »Eſelei«, den Du gebrauchſt. Jemanden, der{ }ſich abfällig
               über einen Schriftſteller geäußert hat und dieſe Äußerung dann beſtreitet, \strikeout{nenn} nenne ich nicht einen Eſel,{ }ſondern einen
               Lügner.\pend
           
\pstart
           Ich trete Ende dieſer Woche meinen Urlaub an. Wohin ich gehe, weiß {\pb}ich immer noch nicht. \strikeout{Wa} Wahrſcheinlich gehe ich nach Tirol\oindex{Tirol@\textbf{Tirol}, \emph{Land}|pw},
               über Wien\oindex{Wien@\textbf{Wien}, \emph{Verwaltungsgebiet}|pw}, und in dieſem Falle werde ich mich{ }ſehr
               freuen, Dir \label{K_L03449-9v}\edtext{nächſte Woche}{\lemma{\textnormal{\emph{nächste Woche}}}\Cendnote{\textnormal{Goldmann\pwindex{Goldmann, Paul 31.\,1.\,1865 Breslau – 25.\,9.\,1935 Wien@\textsc{Goldmann, Paul} (31.\,1.\,1865 Breslau – 25.\,9.\,1935 Wien), \emph{Schriftsteller, Journalist}|pwk} war jedenfalls am 10. 8. 1904 und am 11. 8. 1904 in Wien\oindex{Wien@\textbf{Wien}, \emph{Verwaltungsgebiet}|pwk}. Am 11. 8. 1904 besuchte er Arthur
                  und Olga Schnitzler\pwindex{Schnitzler, Olga 17.\,1.\,1882 Wien – 13.\,1.\,1970 Lugano@\textsc{Schnitzler, Olga} (17.\,1.\,1882 Wien – 13.\,1.\,1970 Lugano), \emph{Schauspielerin, Sängerin}|pwk}.}}}\label{K_L03449-9} die Hand zu
               drücken.\pend
           
\pstart
           Herzliche Grüße Dir und Deiner Frau\pwindex{Schnitzler, Olga 17.\,1.\,1882 Wien – 13.\,1.\,1970 Lugano@\textsc{Schnitzler, Olga} (17.\,1.\,1882 Wien – 13.\,1.\,1970 Lugano), \emph{Schauspielerin, Sängerin}|pwv}! {\\[\baselineskip]}Dein {\\[\baselineskip]}\spacefill\mbox{Paul Goldmn}\pend
           \leftskip=0em{}\selectlanguage{ngerman}\endnumbering\briefempfaengerindex{Schnitzler, Arthur@\textsc{Schnitzler, Arthur}!zzzGoldmann, Paul@\emph{von Paul Goldmann}!1904-08-041@{4. 8. [1904]}|)be}\mylabel{L03449h}  \newcommand{\dateiname}{L03449}\newcommand{\titel}{Paul Goldmann an Arthur Schnitzler, 4. 8. [1904]}\newcommand{\editorInnen}{Martin Anton Müller und Laura Untner}%% latex-leseansicht-abspann.tex
%% Abspann für die Leseansicht.
%% Der Schalter \ifkorrekturansicht ist bereits durch den Vorspann gesetzt.

%% latex-abspann.tex
%% Gemeinsamer Abspann für Korrekturansicht und Leseansicht.
%% Setzt den Schalter \ifkorrekturansicht voraus (gesetzt in den
%% einbindenden Dateien latex-korrekturansicht-abspann.tex bzw.
%% latex-leseansicht-abspann.tex).
%% ---------------------------------------------------------------

\normalsize

% Das esempio-Environment wird nur in der Leseansicht benötigt
\ifkorrekturansicht\else
\newenvironment{esempio}[3]%
{
    \vspace{1.5ex}
    \rlap{\underline{#1}}
    \par
    \setlength{\parindent}{0cm}
    \nopagebreak
    \leftskip=#2cm
    \rightskip=#3cm
}
{
    \par
}
\fi

\doendnotes{C}
\bigskip
\vfill

\clearpage

\footnotesize

\ifkorrekturansicht
  \lohead{\textsc{register}}
\fi

% theindex-Environment neu definieren ohne reledmac
\makeatletter
\renewenvironment{theindex}{%
  \ifkorrekturansicht
    \section*{\indexname}%
  \else
    \subsubsection*{Index der erwähnten Entitäten}%
  \fi
  \setlength{\parindent}{0pt}%
  \setlength{\parskip}{0pt plus 0.3pt}%
  \let\item\@idxitem
}{%
  \ifkorrekturansicht\clearpage\fi
}
\makeatother

\IfFileExists{\jobname-pw.ind}{\input{\jobname-pw.ind}}{}

% Quellenangabe nur in der Leseansicht
\ifkorrekturansicht\else
% Fallback-Definitionen, falls die .tex-Datei \titel etc. nicht gesetzt hat
\providecommand{\titel}{}
\providecommand{\editorInnen}{}
\providecommand{\dateiname}{\jobname}

\vspace{3cm}

\vfill

\footnotesize
\textsc{Quelle}: \titel. Herausgegeben von {\editorInnen}. In: \emph{Arthur Schnitzler: Briefwechsel mit Autorinnen und Autoren}.
 Digitale Edition, https://schnitzler-briefe.acdh.oeaw.ac.at/{\dateiname}.html (Stand \today)
\fi

\end{document}


