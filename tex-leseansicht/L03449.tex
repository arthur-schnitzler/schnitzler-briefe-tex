%% latex-korrekturansicht-vorspann.tex
%% Vorspann für die Korrekturansicht.
%% Lädt die gemeinsame Datei latex-vorspann.tex mit gesetztem Schalter.

\newif\ifkorrekturansicht
\korrekturansichttrue

\input{../tex-inputs/latex-vorspann}


\section[ Paul Goldmann an Arthur Schnitzler, 4. 8. {[}1904{]}]{L03449 Paul Goldmann an Arthur Schnitzler, 4. 8. {[}1904{]}}
\nopagebreak\mylabel{L03449v}
\rehead{ }\normalsize\beginnumbering\briefempfaengerindex{Schnitzler, Arthur@\textsc{Schnitzler, Arthur}!zzzGoldmann, Paul@\emph{von Paul Goldmann}!1904-08-041@{4. 8. {[}1904{]}}|(be}
\toendnotes[C]{\smallbreak\pagebreak[2]}\Standort{DLA, A:Schnitzler, HS.NZ85.1.3174.}
\physDesc{Brief, 2 Blätter, 6 Seiten, 2223 Zeichen
\newline{}Handschrift: blaue Tinte, deutsche Kurrent
\newline{}Schnitzler: 1) mit Bleistift das Jahr »904« vermerkt  2) mit rotem Buntstift drei Unterstreichungen}\toendnotes[C]{\smallbreak}
\pstart
           \raggedleft{}{\pb}\textcolor{gray}{\textbf{DESSAUERSTRASSE 19\oindex{Dessauer Strasse@\textbf{Dessauer Straße}, \emph{Straße (K.STR)}|pw}}}\pend
           
\pstart
           Berlin\oindex{Berlin@\textbf{Berlin}, \emph{P.PPLC}|pw}, 4. Auguſt.\pend
           
\pstart{}Mein lieber Freund,\pend\vspace{0.5em}
\pstart
           \textsc{Peter Dorners\pwindex{Dorner, Peter 17.02.1857 – 01.04.1931@\textsc{Dorner, Peter} (17.02.1857 – 01.04.1931), \emph{Schmied/Schmiedin, Kunsthandwerker/Kunsthandwerkerin, Kunstschmied/Kunstschmiedin}|pw}} Verlobung mag im
               Zuſammenhang mit ſeinem \label{K_L03449-1v}\edtext{»Ruhm«}{\lemma{\textnormal{\emph{»Ruhm«}}}\Cendnote{\textnormal{Der Kunstschmied Peter Dorner\pwindex{Dorner, Peter 17.02.1857 – 01.04.1931@\textsc{Dorner, Peter} (17.02.1857 – 01.04.1931), \emph{Schmied/Schmiedin, Kunsthandwerker/Kunsthandwerkerin, Kunstschmied/Kunstschmiedin}|pwk} war wegen seiner Vorliebe, Schlangen
                  darzustellen, als »Schlangenschmied von Welsberg\oindex{Welsberg-Taisten@\textbf{Welsberg-Taisten}, \emph{A.ADM3}|pwk}« bekannt. Am 28. 4. 1904 hatte er
                  in der \emph{Gießerei Gladenbeck}\orgindex{H. Gladenbeck und Sohn Bildgiesserei@H. Gladenbeck {\kaufmannsund}  Sohn Bildgießerei|pwk} seine Arbeiten
                  erstmals in Berlin\oindex{Berlin@\textbf{Berlin}, \emph{P.PPLC}|pwk} ausgestellt.}}}\label{K_L03449-1} ſtehen.
               Sicherlich aber macht er eine \label{K_L03449-2v}\edtext{»gute Parthie\pwindex{?? [Verlobte von Peter Dorner] @\textsc{?? [Verlobte von Peter Dorner]}|pwv}«}{\lemma{\textnormal{\emph{»gute Parthie«}}}\Cendnote{\textnormal{nicht ermittelt}}}\label{K_L03449-2}. Das Haus des Schwiegervaters\pwindex{?? [Vater von Peter Dorners Verlobter] @\textsc{?? [Vater von Peter Dorners Verlobter]}|pwv} auf der
               Karte, die er mir geſchickt hat, \strikeout{ſpricht} läßt das mit
               aller Deutlichkeit erkennen. Ich habe dieſen \textsc{Peter Dorner\pwindex{Dorner, Peter 17.02.1857 – 01.04.1931@\textsc{Dorner, Peter} (17.02.1857 – 01.04.1931), \emph{Schmied/Schmiedin, Kunsthandwerker/Kunsthandwerkerin, Kunstschmied/Kunstschmiedin}|pw}}, den »weltverlorenen \strikeout{\textcolor{gray}{×}} Bauern«{[},{]}{ }hier\oindex{Berlin@\textbf{Berlin}, \emph{P.PPLC}|pwv} als einen \strikeout{G\textcolor{gray}{e} G\textcolor{gray}{ef}\textcolor{gray}{×}\-\textcolor{gray}{×}\-\textcolor{gray}{×}\-\textcolor{gray}{×}\-\textcolor{gray}{×}\-\textcolor{gray}{×}} Gefühlsmann kennen gelernt, der die geriſſenſten Börſenjuden übertrifft.\pend
           
\pstart
           Sehr bedauert habe ich, zu erfahren, {\pb}daß Du \substVorne{}\textsuperscript{\textcolor{gray}{×}\-\textcolor{gray}{×}\-\textcolor{gray}{×}}\substDazwischen{}14\substHinten{} Tage \label{K_L03449-3v}\edtext{krank}{\lemma{\textnormal{\emph{krank}}}\Cendnote{\textnormal{Vgl. A. S.: \emph{Tagebuch}, 18. 7. 1904 bis 23. 7. 1904.
               }}}\label{K_L03449-3} warſt. Hoffentlich haſt Du, außer einiger »Gelbheit«, keine großen
               Beſchwerden gehabt, und ich freue mich, daß Du wiederhergeſtellt und arbeitsluſtig
               und arbeitskräftig biſt.\pend
           
\pstart
           Der \label{K_L03449-4v}\edtext{Tod \textsc{Herzls\pwindex{Herzl, Theodor 1860-05-02 – 1904-07-03@\textsc{Herzl, Theodor} (1860-05-02 – 1904-07-03), \emph{Schriftsteller/Schriftstellerin, Journalist/Journalistin}|pw}}}{\lemma{\textnormal{\emph{Tod Herzls}}}\Cendnote{\textnormal{Theodor Herzl\pwindex{Herzl, Theodor 1860-05-02 – 1904-07-03@\textsc{Herzl, Theodor} (1860-05-02 – 1904-07-03), \emph{Schriftsteller/Schriftstellerin, Journalist/Journalistin}|pwk} war am 3. 7. 1904 an Herzleiden verstorben.}}}\label{K_L03449-4} hat auch
               mich ſehr ergriffen. Er war der Anſtändigſten und Begabteſten \strikeout{e\textcolor{gray}{i}} einer, und \strikeout{\textcolor{gray}{×}} man ſchätzt ihn umſo höher, wenn man \strikeout{bed\textcolor{gray}{enkt, was nac}h} ihn mit dem Nachwuchs vergleicht.
               Nur was ſeinen \label{K_L03449-5v}\edtext{zioniſtiſchen
                  Lebensplan}{\lemma{\textnormal{\emph{zioniſtiſchen
                  Lebensplan}}}\Cendnote{\textnormal{Siehe zu Goldmanns\pwindex{Goldmann, Paul 31.01.1865 – 25.09.1935@\textsc{Goldmann, Paul} (31.01.1865 – 25.09.1935), \emph{Schriftsteller/Schriftstellerin, Journalist/Journalistin}|pwk} Ablehnung gegenüber Herzls\pwindex{Herzl, Theodor 1860-05-02 – 1904-07-03@\textsc{Herzl, Theodor} (1860-05-02 – 1904-07-03), \emph{Schriftsteller/Schriftstellerin, Journalist/Journalistin}|pwk} zionistischen Visionen etwa Paul Goldmann an Arthur Schnitzler, 29. 7. [1895], 1. 4. [1896] und 7. 3. [1898].
               }}}\label{K_L03449-5} anlangt, ſo iſt {\pb}er, glaube ich, zur rechten
               Zeit geſtorben. Denn die Bewegung ſtand, wie ich höre, am Vorabend ſchwerer \label{K_L03449-6v}\edtext{Kriſen}{\lemma{\textnormal{\emph{Kriſen}}}\Cendnote{\textnormal{womöglich Bezug auf die wiederholte Ablehnung eines jüdischen
                  Staats durch Autoritäten wie Papst Pius X.\pwindex{Pius X. 1835-06-02 – 1914-08-20@\textsc{Pius X.} (1835-06-02 – 1914-08-20), \emph{Papst/Päpstin}|pwk}
                  und Kaiser Wilhelm II.\pwindex{Wilhelm II. von Preussen 27.1.1859 – 4.6.1941@\textsc{Wilhelm II. von Preußen} (27.1.1859 – 4.6.1941), \emph{Kaiser/Kaiserin}|pwk}}}}\label{K_L03449-6}.\pend
           
\pstart
           Daß ich ſein \label{K_L03449-7v}\edtext{Nachfolger\orgindex{Neue Freie Presse@Neue Freie Presse|pwv}}{\lemma{\textnormal{\emph{Nachfolger}}}\Cendnote{\textnormal{als Feuilletonredakteur der \emph{Neuen Freien Presse}\orgindex{Neue Freie Presse@Neue Freie Presse|pwk}}}}\label{K_L03449-7} werde, halte ich für ausgeſchloſſen. Die Herausgeber\pwindex{Bacher, Eduard 07.03.1846 – 16.01.1908@\textsc{Bacher, Eduard} (07.03.1846 – 16.01.1908), \emph{Journalist/Journalistin, Herausgeber/Herausgeberin}|pwv}\pwindex{Benedikt, Moriz 27.05.1849 – 18.03.1920@\textsc{Benedikt, Moriz} (27.05.1849 – 18.03.1920), \emph{Journalist/Journalistin, Herausgeber/Herausgeberin}|pwv} machen keine Anſtalten, mir die
               Stellung anzubieten, und ich habe nicht die Abſicht, mich darum zu bewerben, da die
               Stellung mir nicht die Freiheit gewährt, zu leiſten, was ich leiſten möchte, und da
               außerdem meine Luſt, nach Wien\oindex{Wien@\textbf{Wien}, \emph{A.ADM2}|pw} zurückzukehren,
               immer geringer wird.\pend
           
\pstart
           {\pb}Meine \label{K_L03449-8v}\edtext{Äußerung}{\lemma{\textnormal{\emph{Äußerung}}}\Cendnote{\textnormal{Siehe Paul Goldmann an Arthur Schnitzler, 23. 6. [1904] und Hugo von Hofmannsthal an Arthur Schnitzler, 1[9?]. 6. [1904]. }}}\label{K_L03449-8} über \textsc{Hoffmannsthal\pwindex{Hofmannsthal, Hugo von 1874-02-01 – 1929-07-15@\textsc{Hofmannsthal, Hugo von} (1874-02-01 – 1929-07-15), \emph{Schriftsteller/Schriftstellerin}|pw}} haſt Du wieder einmal mißverſtanden. Mich hat es nicht überraſcht, daß Du die
               Fehler, die Deine Freunde begehen, offen als ſolche bezeichneſt (ich kenne Deine
               Offenheit ſehr wohl und ſchätze ſie ſehr hoch), ſondern mich hat es überraſcht, daß
               Du einen Fehler \textsc{Hoffmannsthals\pwindex{Hofmannsthal, Hugo von 1874-02-01 – 1929-07-15@\textsc{Hofmannsthal, Hugo von} (1874-02-01 – 1929-07-15), \emph{Schriftsteller/Schriftstellerin}|pw}} als ſolchen erkannt
               haſt, da Du ſonſt, meiner Anſicht nach, \textsc{Hoffmannsthal\pwindex{Hofmannsthal, Hugo von 1874-02-01 – 1929-07-15@\textsc{Hofmannsthal, Hugo von} (1874-02-01 – 1929-07-15), \emph{Schriftsteller/Schriftstellerin}|pw}}{ }{\pb}nicht richtig beurtheilſt. Im Übrigen überraſcht
               mich wieder der Ausdruck »Eſelei«, den Du gebrauchſt. Jemanden, der ſich abfällig
               über einen Schriftſteller geäußert hat und dieſe Äußerung dann beſtreitet, \strikeout{nenn} nenne ich nicht einen Eſel, ſondern einen
               Lügner.\pend
           
\pstart
           Ich trete Ende dieſer Woche meinen Urlaub an. Wohin ich gehe, weiß {\pb}ich immer noch nicht. \strikeout{Wa} Wahrſcheinlich gehe ich nach Tirol\oindex{Tirol@\textbf{Tirol}, \emph{A.ADM1}|pw},
               über Wien\oindex{Wien@\textbf{Wien}, \emph{A.ADM2}|pw}, und in dieſem Falle werde ich mich ſehr
               freuen, Dir \label{K_L03449-9v}\edtext{nächſte Woche}{\lemma{\textnormal{\emph{nächſte Woche}}}\Cendnote{\textnormal{Goldmann\pwindex{Goldmann, Paul 31.01.1865 – 25.09.1935@\textsc{Goldmann, Paul} (31.01.1865 – 25.09.1935), \emph{Schriftsteller/Schriftstellerin, Journalist/Journalistin}|pwk} war jedenfalls am 10. 8. 1904 und am 11. 8. 1904 in Wien\oindex{Wien@\textbf{Wien}, \emph{A.ADM2}|pwk}. Am 11. 8. 1904 besuchte er Arthur
                  und Olga Schnitzler\pwindex{Schnitzler, Olga 17.01.1882 – 13.01.1970@\textsc{Schnitzler, Olga} (17.01.1882 – 13.01.1970), \emph{Schauspieler/Schauspielerin, Sänger/Sängerin}|pwk}.}}}\label{K_L03449-9} die Hand zu
               drücken.\pend
           
\pstart
           Herzliche Grüße Dir und Deiner Frau\pwindex{Schnitzler, Olga 17.01.1882 – 13.01.1970@\textsc{Schnitzler, Olga} (17.01.1882 – 13.01.1970), \emph{Schauspieler/Schauspielerin, Sänger/Sängerin}|pwv}! {\\[\baselineskip]}Dein {\\[\baselineskip]}\spacefill\mbox{Paul Goldmn}\pend
           \leftskip=0em{}\selectlanguage{ngerman}\endnumbering\briefempfaengerindex{Schnitzler, Arthur@\textsc{Schnitzler, Arthur}!zzzGoldmann, Paul@\emph{von Paul Goldmann}!1904-08-041@{4. 8. {[}1904{]}}|)be}\mylabel{L03449h}  \normalsize

\doendnotes{C}
\bigskip
\vfill

\clearpage

\footnotesize

\lohead{\textsc{register}}

% Definiere theindex-Environment komplett neu ohne reledmac
\makeatletter
\renewenvironment{theindex}{%
  \section*{\indexname}%
  \setlength{\parindent}{0pt}%
  \setlength{\parskip}{0pt plus 0.3pt}%
  \let\item\@idxitem
}{%
  \clearpage
}
\makeatother

\IfFileExists{\jobname-pw.ind}{\input{\jobname-pw.ind}}{}

\end{document}

      