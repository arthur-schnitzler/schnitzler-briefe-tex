%% latex-leseansicht-vorspann.tex
%% Vorspann für die Leseansicht.
%% Lädt die gemeinsame Datei latex-vorspann.tex mit nicht gesetztem Schalter.

\newif\ifkorrekturansicht
\korrekturansichtfalse

\input{../tex-inputs/latex-vorspann}

\begin{center}
            \textcolor{red}{ENTWURF, NICHT FERTIG KORRIGIERT}
                      \end{center}
            
         
         \renewcommand{\erwaehntePersonen}{Personen: Eduard Bacher, Moriz Benedikt, Peter Dorner, Theodor Herzl, Hugo von Hofmannsthal,  Papst Pius X., Olga Schnitzler,  Wilhelm II. von Preußen, ?? [Vater von Peter Dorners Verlobter], ?? [Verlobte von Peter Dorner]}
         \renewcommand{\erwaehnteInstitutionen}{Institutionen: Neue Freie Presse}
         \renewcommand{\erwaehnteOrte}{Orte: Berlin, Dessauer Straße, Tirol, Welsberg-Taisten, Wien}
         \renewcommand{\erwaehnteWerke}{Werke: Tagebuch}
               \section[ Paul Goldmann an Arthur Schnitzler, 4. 8. {[}1904{]}]{ Paul Goldmann an Arthur Schnitzler, 4. 8. {[}1904{]}}\nopagebreak\mylabel{v}\rehead{ }\begin{ledgroupsized}[t]{13cm}\normalsize\beginnumbering \toendnotes[C]{\smallbreak\pagebreak[2]} \Standort{DLA, A:Schnitzler, HS.NZ85.1.3174.}
\physDesc{Brief, 2 Blätter, 5 Seiten
\newline{}Handschrift: blaue Tinte, deutsche Kurrent
\newline{}Schnitzler: 1) mit Bleistift das Jahr »{[}1{]}904« vermerkt  2) mit rotem Buntstift drei Unterstreichungen}\toendnotes[C]{\smallbreak}\pstart
           \noindent{}\raggedleft{}{\pb}\textcolor{gray}{\textbf{DESSAUERSTRASSE 19\oindex{Dessauer Strasse@\textbf{Dessauer Straße}|pw}}}\pend
           \pstart
           Berlin\oindex{Berlin@\textbf{Berlin}|pw}, 4. Auguſt.\pend
           \pstart{}Mein lieber Freund,\pend\pstart
           \textsc{Peter Dorner\pwindex{Dorner, Peter 17.02.1857 – 01.04.1931@\textsc{Dorner, Peter} (17.02.1857 – 01.04.1931), \emph{Schmied, Kunsthandwerker, Kunstschmied}|pw}s} Verlobung mag im
               Zuſammenhang mit ſeinem \label{K_L03449-1v}\edtext{»Ruhm«}{\lemma{\textnormal{\emph{»Ruhm«}}}\Cendnote{\textnormal{Peter Dorner\pwindex{Dorner, Peter 17.02.1857 – 01.04.1931@\textsc{Dorner, Peter} (17.02.1857 – 01.04.1931), \emph{Schmied, Kunsthandwerker, Kunstschmied}|pwk} war ein erfolreicher
                  Kunstschmied, der als »Schlangenschmied von Welsberg\oindex{Welsberg-Taisten@\textbf{Welsberg-Taisten}|pwk}« bekannt war.}}}\label{K_L03449-1h} ſtehen. Sicherlich aber macht er eine
                  \label{K_L03449-2v}\edtext{»gute Parthie\pwindex{[Verlobte von Peter Dorner], ?? @\textsc{[Verlobte von Peter Dorner], ??}|pwv}«}{\lemma{\textnormal{\emph{»gute Parthie«}}}\Cendnote{\textnormal{nicht ermittelt
                  }}}\label{K_L03449-2h}.
               Das Haus des Schwiegervater\pwindex{[Vater von Peter Dorners Verlobter], ?? @\textsc{[Vater von Peter Dorners Verlobter], ??}|pwv}s auf der Karte, die er mir geſchickt hat, \strikeout{ſpricht} läßt das mit aller Deutlichkeit erkennen. Ich
               habe dieſen \textsc{Peter Dorner\pwindex{Dorner, Peter 17.02.1857 – 01.04.1931@\textsc{Dorner, Peter} (17.02.1857 – 01.04.1931), \emph{Schmied, Kunsthandwerker, Kunstschmied}|pw}}, den »weltverlorenen \strikeout{\textcolor{gray}{×}} Bauern«{[},{]}hier\oindex{Berlin@\textbf{Berlin}|pwv} als einen \strikeout{G\textcolor{gray}{e} Ge\textcolor{gray}{f}\textcolor{gray}{×}\-\textcolor{gray}{×}\-\textcolor{gray}{×}\-\textcolor{gray}{×}\-\textcolor{gray}{×}\-\textcolor{gray}{×}} Gefühlsmann kennen gelernt, der die geriſſenſten Börſenjuden übertrifft.\pend
           \pstart
           Sehr bedauert habe ich, zu erfahren, {\pb} daß du \strikeout{\textcolor{gray}{×}\-\textcolor{gray}{×}\-\textcolor{gray}{×}}\introOben{}14\introOben{} Tage \label{K_L03449-3v}\edtext{krank}{\lemma{\textnormal{\emph{krank}}}\Cendnote{\textnormal{vgl. \emph{Tagebuch}\pwindex{Schnitzler, Arthur 15.05.1862 – 21.10.1931@\textsc{Schnitzler, Arthur} (15.05.1862 – 21.10.1931), \emph{Schriftsteller, Mediziner}!Tagebuch1981 – 2000@\strich\emph{Tagebuch} {[}1981 – 2000{]}|pwk}}}}\label{K_L03449-3h} warſt. Hoffentlich haſt Du, außer einiger »Gelbheit«, keine großen
               Beſchwerden gehabt, und ich freue mich, daß Du wiederhergeſtellt und arbeitsluſtig
               und arbeitskräftig biſt.\pend
           \pstart
           Der \label{K_L03449-4v}\edtext{Tod \textsc{Herzl\pwindex{Herzl, Theodor 1860-05-02 – 1904-07-03@\textsc{Herzl, Theodor} (1860-05-02 – 1904-07-03), \emph{Schriftsteller, Journalist}|pw}s}}{\lemma{\textnormal{\emph{Tod Herzls}}}\Cendnote{\textnormal{Theodor Herzl\pwindex{Herzl, Theodor 1860-05-02 – 1904-07-03@\textsc{Herzl, Theodor} (1860-05-02 – 1904-07-03), \emph{Schriftsteller, Journalist}|pwk} war am 3. 7. 1904 an Herzleiden verstorben.}}}\label{K_L03449-4h} hat auch
               mich ſehr ergriffen. Er war der Anſtändigſten und Begabteſten \strikeout{e\textcolor{gray}{i}} einer, und \strikeout{\textcolor{gray}{×}} man ſchätzt ihn umſo höher, wenn man \strikeout{bed\textcolor{gray}{enkt, was nac}h} ihn mit dem Nachwuchs vergleicht.
               Nur was ſeinen \label{K_L03449-5v}\edtext{zioniſtiſchen
                  Lebensplan}{\lemma{\textnormal{\emph{zioniſtiſchen
                  Lebensplan}}}\Cendnote{\textnormal{siehe zu Goldmann\pwindex{Goldmann, Paul 31.01.1865 – 25.09.1935@\textsc{Goldmann, Paul} (31.01.1865 – 25.09.1935), \emph{Schriftsteller, Journalist}|pwk}s Ablehnung gegenüber Herzl\pwindex{Herzl, Theodor 1860-05-02 – 1904-07-03@\textsc{Herzl, Theodor} (1860-05-02 – 1904-07-03), \emph{Schriftsteller, Journalist}|pwk}s zionistischen Visionen etwa Paul Goldmann an Arthur Schnitzler, 29. 7. [1895], 1. 4. [1896] und 7. 3. [1898]}}}\label{K_L03449-5h} anlangt, ſo iſt {\pb}er, glaube ich, zur rechten
               Zeit geſtorben. Denn die Bewegung ſtand, wie ich höre, am Vorabend ſchwerer \label{K_L03449-6v}\edtext{Kriſen}{\lemma{\textnormal{\emph{Kriſen}}}\Cendnote{\textnormal{womöglich Bezug auf die wiederholte Ablehnung eines jüdischen
                  Staats durch Autoritäten wie Papst Pius X.\pwindex{Papst Pius X. 1835-06-02 – 1914-08-20@\textsc{Papst Pius X.} (1835-06-02 – 1914-08-20), \emph{Papst}|pwk}
                  und Kaiser Wilhelm II.\pwindex{Wilhelm II. von Preussen 27.1.1859 – 4.6.1941@\textsc{Wilhelm II. von Preußen} (27.1.1859 – 4.6.1941), \emph{Kaiser}|pwk}}}}\label{K_L03449-6h}.\pend
           \pstart
           Daß ich ſein \label{K_L03449-7v}\edtext{Nachfolger\orgindex{Neue Freie Presse@Neue Freie Presse|pwv}}{\lemma{\textnormal{\emph{Nachfolger}}}\Cendnote{\textnormal{als Feuilletonredakteur der \emph{Neuen Freien Presse}\orgindex{Neue Freie Presse@Neue Freie Presse|pwk}}}}\label{K_L03449-7h} werde, halte ich für ausgeſchloſſen. Die Herausgeber\pwindex{Bacher, Eduard 07.03.1846 – 16.01.1908@\textsc{Bacher, Eduard} (07.03.1846 – 16.01.1908), \emph{Journalist, Herausgeber}|pwv}\pwindex{Benedikt, Moriz 27.05.1849 – 18.03.1920@\textsc{Benedikt, Moriz} (27.05.1849 – 18.03.1920), \emph{Journalist, Herausgeber}|pwv} machen keine Anſtalten, mir die
               Stellung anzubieten, und ich habe nicht die Abſicht, mich darum zu bewerben, da die
               Stellung mir nicht die Freiheit gewährt, zu leiſten, was ich leiſten möchte, und da
               außerdem meine Luſt, nach Wien\oindex{Wien@\textbf{Wien}|pw} zurückzukehren,
               immer geringer wird.\pend
           \pstart
           {\pb}Meine \label{K_L03449-8v}\edtext{Äußerung}{\lemma{\textnormal{\emph{Äußerung}}}\Cendnote{\textnormal{siehe Paul Goldmann an Arthur Schnitzler, 23. 6. [1904]}}}\label{K_L03449-8h} über \textsc{Hoffmannsthal\pwindex{Hofmannsthal, Hugo von 1874-02-01 – 1929-07-15@\textsc{Hofmannsthal, Hugo von} (1874-02-01 – 1929-07-15), \emph{Schriftsteller}|pw}} haſt Du wieder einmal mißverſtanden. Mich hat es nicht überraſcht, daß Du die
               Fehler, die Deine Freunde begehen, offen als ſolche bezeichneſt (ich kenne Deine
               Offenheit ſehr wohl und ſchätze ſie ſehr hoch), ſondern mich hat es überraſcht, daß
               Du einen Fehler \textsc{Hoffmannsthal\pwindex{Hofmannsthal, Hugo von 1874-02-01 – 1929-07-15@\textsc{Hofmannsthal, Hugo von} (1874-02-01 – 1929-07-15), \emph{Schriftsteller}|pw}s} als ſolchen erkannt
               haſt, da Du ſonſt, meiner Anſicht nach, \textsc{Hoffmannsthal\pwindex{Hofmannsthal, Hugo von 1874-02-01 – 1929-07-15@\textsc{Hofmannsthal, Hugo von} (1874-02-01 – 1929-07-15), \emph{Schriftsteller}|pw}}\pend
           \pstart
           {\pb}ich immer noch nicht. \strikeout{Wa} Wahrſcheinlich gehe ich nach Tirol\oindex{Tirol@\textbf{Tirol}|pw},
               über Wien\oindex{Wien@\textbf{Wien}|pw}, und in dieſem Falle werde ich ſehr
               freuen, Dir \label{K_L03449-9v}\edtext{nächſte Woche}{\lemma{\textnormal{\emph{nächſte Woche}}}\Cendnote{\textnormal{Goldmann\pwindex{Goldmann, Paul 31.01.1865 – 25.09.1935@\textsc{Goldmann, Paul} (31.01.1865 – 25.09.1935), \emph{Schriftsteller, Journalist}|pwk} war jedenfalls am 10. 8. 1904 und am 11. 8. 1904 in Wien\oindex{Wien@\textbf{Wien}|pwk}. Am 11. 8. 1904 besuchte er Arthur und Olga Schnitzler\pwindex{Schnitzler, Arthur 15.05.1862 – 21.10.1931@\textsc{Schnitzler, Arthur} (15.05.1862 – 21.10.1931), \emph{Schriftsteller, Mediziner}|pwk}\pwindex{Schnitzler, Olga 17.01.1882 – 13.01.1970@\textsc{Schnitzler, Olga} (17.01.1882 – 13.01.1970), \emph{Schauspielerin, Sängerin}|pwk}.}}}\label{K_L03449-9h} die Hand zu drücken.\pend
           \pstart
           Herzliche Grüße Dir und Deiner Frau\pwindex{Schnitzler, Olga 17.01.1882 – 13.01.1970@\textsc{Schnitzler, Olga} (17.01.1882 – 13.01.1970), \emph{Schauspielerin, Sängerin}|pwv}! {\\[\baselineskip]}Dein {\\[\baselineskip]}\spacefill\mbox{Paul Goldmn}\pend
           \leftskip=0em{}
         
         \endnumbering\mylabel{h}\end{ledgroupsized}\begin{anhang}\end{anhang}\newcommand{\dateiname}{L03449}\newcommand{\titel}{Paul Goldmann an Arthur Schnitzler, 4. 8. [1904]}\newcommand{\editorInnen}{Martin Anton Müller und Laura Untner}%% latex-leseansicht-abspann.tex
%% Abspann für die Leseansicht.
%% Der Schalter \ifkorrekturansicht ist bereits durch den Vorspann gesetzt.

%% latex-abspann.tex
%% Gemeinsamer Abspann für Korrekturansicht und Leseansicht.
%% Setzt den Schalter \ifkorrekturansicht voraus (gesetzt in den
%% einbindenden Dateien latex-korrekturansicht-abspann.tex bzw.
%% latex-leseansicht-abspann.tex).
%% ---------------------------------------------------------------

\normalsize

% Das esempio-Environment wird nur in der Leseansicht benötigt
\ifkorrekturansicht\else
\newenvironment{esempio}[3]%
{
    \vspace{1.5ex}
    \rlap{\underline{#1}}
    \par
    \setlength{\parindent}{0cm}
    \nopagebreak
    \leftskip=#2cm
    \rightskip=#3cm
}
{
    \par
}
\fi

\doendnotes{C}
\bigskip
\vfill

\clearpage

\footnotesize

\ifkorrekturansicht
  \lohead{\textsc{register}}
\fi

% theindex-Environment neu definieren ohne reledmac
\makeatletter
\renewenvironment{theindex}{%
  \ifkorrekturansicht
    \section*{\indexname}%
  \else
    \subsubsection*{Index der erwähnten Entitäten}%
  \fi
  \setlength{\parindent}{0pt}%
  \setlength{\parskip}{0pt plus 0.3pt}%
  \let\item\@idxitem
}{%
  \ifkorrekturansicht\clearpage\fi
}
\makeatother

\IfFileExists{\jobname-pw.ind}{\input{\jobname-pw.ind}}{}

% Quellenangabe nur in der Leseansicht
\ifkorrekturansicht\else
% Fallback-Definitionen, falls die .tex-Datei \titel etc. nicht gesetzt hat
\providecommand{\titel}{}
\providecommand{\editorInnen}{}
\providecommand{\dateiname}{\jobname}

\vspace{3cm}

\vfill

\footnotesize
\textsc{Quelle}: \titel. Herausgegeben von {\editorInnen}. In: \emph{Arthur Schnitzler: Briefwechsel mit Autorinnen und Autoren}.
 Digitale Edition, https://schnitzler-briefe.acdh.oeaw.ac.at/{\dateiname}.html (Stand \today)
\fi

\end{document}


      