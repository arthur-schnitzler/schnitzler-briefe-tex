%% latex-leseansicht-vorspann.tex
%% Vorspann für die Leseansicht.
%% Lädt die gemeinsame Datei latex-vorspann.tex mit nicht gesetztem Schalter.

\newif\ifkorrekturansicht
\korrekturansichtfalse

\input{../tex-inputs/latex-vorspann}


\section[Arthur Schnitzler an Georg Brandes, 12. 10. 1911]{L02035 Arthur Schnitzler an Georg Brandes, 12. 10. 1911}
\nopagebreak\mylabel{L02035v}
\rehead{ }\normalsize\beginnumbering\briefempfaengerindex{Brandes, Georg@\textsc{Brandes, Georg}!zzzSchnitzler, Arthur@\emph{von Arthur Schnitzler}!1911-10-121@{12. 10. 1911}|(be}
\toendnotes[C]{\smallbreak\pagebreak[2]}
\correspDesc{Versand  durch Arthur Schnitzler am 12. 10. 1911 in Wien
\newline{}Erhalt  durch Georg Brandes im Zeitraum [13. 10. 1911 – 17. 10. 1911?] in Kopenhagen}\toendnotes[C]{\smallbreak}
\Standort{Kopenhagen, Det Kongelige Bibliotek, Georg Brandes Arkiv, box 125.}
\physDesc{Brief, 1 Blatt, 4 Seiten, 1552 Zeichen (Briefpapier mit Trauerrand)
\newline{}Handschrift: schwarze Tinte, lateinische Kurrent
\newline{}Ordnung: mit Bleistift von unbekannter Hand beschriftet:
                                    »Schnitzler« und »Arthur
                                    Schnitzler«, nummeriert: »32.« und
                                 mehrere Unterstreichungen }
\buchAbdrucke{\weitereDrucke{Georg Brandes, Arthur Schnitzler: \emph{Ein Briefwechsel}. Herausgegeben von Kurt Bergel. Bern: \emph{Francke} 1956, S. 102.} }\toendnotes[C]{\smallbreak}
\pstart
           \raggedleft{}{\pb}Wien, XVIII.\oindex{XVIII., Währing@\textbf{XVIII., Währing}, \emph{Verwaltungsgebiet}|pw}{\\}Sternwartestr. 71\oindex{Wien@\textbf{Wien}!XVIII., Währing@\textbf{XVIII., Währing}!Sternwartestraße 71@\textbf{Sternwartestraße 71}, \emph{Wohngebäude}|pw}{\\}12. X. 911\pend
           
\pstart{}Lieber und verehrter Herr Brandes,\pend\vspace{0.5em}
\pstart
           ich habe mich in der Angelegenheit des Frl. Prozor\pwindex{Prozor, Grete 28.\,12.\,1885 Paris – 14.\,2.\,1978 Genf@\textsc{Prozor, Grete} (28.\,12.\,1885 Paris – 14.\,2.\,1978 Genf), \emph{Schauspielerin}|pw} gleich an die Neue Freie Presse\orgindex{Neue Freie Presse@Neue Freie Presse|pw}
               gewendet; hier das \label{K_L02035-1v}\edtext{Resultat\pwindex{Ibsen in Frankreich@\emph{Ibsen in Frankreich}|pwv}}{\lemma{\textnormal{\emph{Resultat}}}\Cendnote{\textnormal{Ein Interview mit Grete Prozor\pwindex{Prozor, Grete 28.\,12.\,1885 Paris – 14.\,2.\,1978 Genf@\textsc{Prozor, Grete} (28.\,12.\,1885 Paris – 14.\,2.\,1978 Genf), \emph{Schauspielerin}|pwk} enthält: [O. V.]: \emph{Ibsen in Frankreich}\pwindex{Ibsen in Frankreich@\emph{Ibsen in Frankreich}|pwk}. In: \emph{Neue Freie Presse}\pwindex{Neue Freie Presse@\emph{Neue Freie Presse}|pwk}, Nr. 16.933, 12. 10. 1911, Morgenblatt,
                     S. 10.}}}\label{K_L02035-1}.\pend
           
\pstart
           Sie reisen überall hin – nur nach Wien\oindex{Wien@\textbf{Wien}, \emph{Verwaltungsgebiet}|pw} wollen Sie
               niemals kommen! Nun, vielleicht führt uns der nächste Sommer wieder nordwärts, und
               man sieht einander wieder. Es freut mich immer so sehr in Ihren Briefen zu lesen, daß
               Sie meiner {\pb}in Sympathie gedenken;– was Sie, mein
               verehrter und lieber Freund mir bedeuten – mir schon bedeutet haben, lang eh Sie von
               meiner Existenz wußten, das fühlen Sie wohl! Nur schade, daß man sich meist an diesem
               Wissen u Fühlen muß genügen lassen – und in so vielen vielen Jahren innerer Zusa{\geminationm}engehörigkeit keine fünfzig Stunden miteinander
               verbracht hat!\pend
           
\pstart
           – Ich bin nun mit den Proben meiner {\pb}neuen
               Tragikomödie »das weite Land\pwindex{Schnitzler, Arthur 15.\,5.\,1862 Wien – 21.\,10.\,1931 ebd.@\textsc{Schnitzler, Arthur} (15.\,5.\,1862 Wien – 21.\,10.\,1931 ebd.), \emph{Schriftsteller, Mediziner}!weite Land. Tragikomödie in fünf Akten@\strich\emph{Das weite Land. Tragikomödie in fünf Akten}|pw}« beschäftigt – am
               Sonntag ist die Première zugleich am Burgtheater\orgindex{Burgtheater@Burgtheater|pw},
               in Berlin\oindex{Berlin@\textbf{Berlin}, \emph{Hauptstadt}|pw}, München\oindex{München@\textbf{München}|pw}, Hamburg\oindex{Hamburg@\textbf{Hamburg}|pw}, Frankfurt\oindex{Frankfurt am Main@\textbf{Frankfurt am Main}, \emph{Hauptstadt}|pw} und noch etlichen andern Städten. Sie werden das Buch\pwindex{Schnitzler, Arthur 15.\,5.\,1862 Wien – 21.\,10.\,1931 ebd.@\textsc{Schnitzler, Arthur} (15.\,5.\,1862 Wien – 21.\,10.\,1931 ebd.), \emph{Schriftsteller, Mediziner}!weite Land. Tragikomödie in fünf Akten@\strich\emph{Das weite Land. Tragikomödie in fünf Akten}|pwv} in diesen Tagen \substVorne{}\textsuperscript{haben}\substDazwischen{}beko{\geminationm}en\substHinten{}; hoffentlich werden Sie einige Freude daran haben.\pend
           
\pstart
           – Der schwarze Rand dieses Blattes besagt, daß meine Mutter\pwindex{Schnitzler, Louise 8.\,7.\,1840 Kőszeg – 9.\,9.\,1911 Wien@\textsc{Schnitzler, Louise} (8.\,7.\,1840 Kőszeg – 9.\,9.\,1911 Wien)|pwv} gestorben ist. Es sind nun fünf
               Wochen her – nach einer {\pb}Lungenentzündung, von der
               sie gar nichts verspürte (sie glaubte im Sanatorium eine Mastkur zu gebrauchen,) ist
               sie ruhig eingeschlafen für ewige Zeit. –\pend
           
\pstart
           Leben Sie wohl, erhalten Sie mir Ihre Freundschaft, und lassen Sie uns ein
               Wiedersehen in guter Gesundheit erhoffen.\pend
           
\pstart
           Herzlichst der{\\[\baselineskip]}Ihre{\\[\baselineskip]}\spacefill\mbox{ArthurSchnitzler}\pend
           \leftskip=0em{}
\pstart
           \noindent{}Meine Frau\pwindex{Schnitzler, Olga 17.\,1.\,1882 Wien – 13.\,1.\,1970 Lugano@\textsc{Schnitzler, Olga} (17.\,1.\,1882 Wien – 13.\,1.\,1970 Lugano), \emph{Schauspielerin, Sängerin}|pwv} grüßt Sie. Auch
                  sie möchte so gern wieder einmal Georg Brandes sehen!\pend
           \selectlanguage{ngerman}\endnumbering\briefempfaengerindex{Brandes, Georg@\textsc{Brandes, Georg}!zzzSchnitzler, Arthur@\emph{von Arthur Schnitzler}!1911-10-121@{12. 10. 1911}|)be}\mylabel{L02035h}  \newcommand{\dateiname}{L02035}\newcommand{\titel}{Arthur Schnitzler an Georg Brandes, 12. 10. 1911}\newcommand{\editorInnen}{Martin Anton Müller und Gerd-Hermann Susen}%% latex-leseansicht-abspann.tex
%% Abspann für die Leseansicht.
%% Der Schalter \ifkorrekturansicht ist bereits durch den Vorspann gesetzt.

%% latex-abspann.tex
%% Gemeinsamer Abspann für Korrekturansicht und Leseansicht.
%% Setzt den Schalter \ifkorrekturansicht voraus (gesetzt in den
%% einbindenden Dateien latex-korrekturansicht-abspann.tex bzw.
%% latex-leseansicht-abspann.tex).
%% ---------------------------------------------------------------

\normalsize

% Das esempio-Environment wird nur in der Leseansicht benötigt
\ifkorrekturansicht\else
\newenvironment{esempio}[3]%
{
    \vspace{1.5ex}
    \rlap{\underline{#1}}
    \par
    \setlength{\parindent}{0cm}
    \nopagebreak
    \leftskip=#2cm
    \rightskip=#3cm
}
{
    \par
}
\fi

\doendnotes{C}
\bigskip
\vfill

\clearpage

\footnotesize

\ifkorrekturansicht
  \lohead{\textsc{register}}
\fi

% theindex-Environment neu definieren ohne reledmac
\makeatletter
\renewenvironment{theindex}{%
  \ifkorrekturansicht
    \section*{\indexname}%
  \else
    \subsubsection*{Index der erwähnten Entitäten}%
  \fi
  \setlength{\parindent}{0pt}%
  \setlength{\parskip}{0pt plus 0.3pt}%
  \let\item\@idxitem
}{%
  \ifkorrekturansicht\clearpage\fi
}
\makeatother

\IfFileExists{\jobname-pw.ind}{\input{\jobname-pw.ind}}{}

% Quellenangabe nur in der Leseansicht
\ifkorrekturansicht\else
% Fallback-Definitionen, falls die .tex-Datei \titel etc. nicht gesetzt hat
\providecommand{\titel}{}
\providecommand{\editorInnen}{}
\providecommand{\dateiname}{\jobname}

\vspace{3cm}

\vfill

\footnotesize
\textsc{Quelle}: \titel. Herausgegeben von {\editorInnen}. In: \emph{Arthur Schnitzler: Briefwechsel mit Autorinnen und Autoren}.
 Digitale Edition, https://schnitzler-briefe.acdh.oeaw.ac.at/{\dateiname}.html (Stand \today)
\fi

\end{document}


