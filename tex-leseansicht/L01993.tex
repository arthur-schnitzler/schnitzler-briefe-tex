%% latex-leseansicht-vorspann.tex
%% Vorspann für die Leseansicht.
%% Lädt die gemeinsame Datei latex-vorspann.tex mit nicht gesetztem Schalter.

\newif\ifkorrekturansicht
\korrekturansichtfalse

\input{../tex-inputs/latex-vorspann}


         
         \renewcommand{\erwaehntePersonen}{Personen: Richard Beer-Hofmann, Josef Berger}
         \renewcommand{\erwaehnteOrte}{Orte: Wien}
         \renewcommand{\erwaehnteWerke}{
               \section[Arthur Schnitzler an Richard Beer-Hofmann, 22. 12. 1910]{ Arthur Schnitzler an Richard Beer-Hofmann, 22. 12. 1910}\nopagebreak\mylabel{v}\rehead{ }\begin{ledgroupsized}[t]{13cm}\normalsize\beginnumbering \toendnotes[C]{\smallbreak\pagebreak[2]} \Standort{YCGL, MSS 31.}
\physDesc{Kartenbrief
\newline{}Handschrift: Bleistift, deutsche Kurrent\newline{}Versand: ohne postalischen Übermittlungsvermerk }\buchAbdrucke{\weitereDrucke{Arthur Schnitzler, Richard Beer-Hofmann: \emph{Briefwechsel 1891–1931}. Hg. Konstanze Fliedl. Wien, Zürich: \emph{Europaverlag} 1992, S. 214.} }\pstart{}{\pb}\textsc{Herrn Dr. Rich Beer-Hofmann}\pend{}\pstart{}Wien\oindex{Wien@\textbf{Wien}|pw}\pend{}{\bigskip}\pstart
           \raggedleft{}{\pb}22/12 910\pend
           \pstart{}lieber Richard,\pend\pstart
           wir reflectiren \uline{nicht} auf die Commode. (Ich ka{\geminationn} nichts dafür, daſs \textsc{Berger}\pwindex{Berger, Josef 20.01.1848 – 02.09.1932@\textsc{Berger, Josef} (20.01.1848 – 02.09.1932), \emph{Antiquitätenhändler}|pw} Sie um unſretwillen telephoniſch behelligt – entſchuldigen Sie!)\pend
           \pstart
           Auf Wiederſehen.\pend
           \pstart Herzlichſt Ihr \spacefill\mbox{A.}\pend{}
         
         \endnumbering\mylabel{h}\end{ledgroupsized}  \newcommand{\dateiname}{L01993}\newcommand{\titel}{Arthur Schnitzler an Richard Beer-Hofmann, 22. 12. 1910}\newcommand{\editorInnen}{Martin Anton Müller und Gerd-Hermann Susen}%% latex-leseansicht-abspann.tex
%% Abspann für die Leseansicht.
%% Der Schalter \ifkorrekturansicht ist bereits durch den Vorspann gesetzt.

%% latex-abspann.tex
%% Gemeinsamer Abspann für Korrekturansicht und Leseansicht.
%% Setzt den Schalter \ifkorrekturansicht voraus (gesetzt in den
%% einbindenden Dateien latex-korrekturansicht-abspann.tex bzw.
%% latex-leseansicht-abspann.tex).
%% ---------------------------------------------------------------

\normalsize

% Das esempio-Environment wird nur in der Leseansicht benötigt
\ifkorrekturansicht\else
\newenvironment{esempio}[3]%
{
    \vspace{1.5ex}
    \rlap{\underline{#1}}
    \par
    \setlength{\parindent}{0cm}
    \nopagebreak
    \leftskip=#2cm
    \rightskip=#3cm
}
{
    \par
}
\fi

\doendnotes{C}
\bigskip
\vfill

\clearpage

\footnotesize

\ifkorrekturansicht
  \lohead{\textsc{register}}
\fi

% theindex-Environment neu definieren ohne reledmac
\makeatletter
\renewenvironment{theindex}{%
  \ifkorrekturansicht
    \section*{\indexname}%
  \else
    \subsubsection*{Index der erwähnten Entitäten}%
  \fi
  \setlength{\parindent}{0pt}%
  \setlength{\parskip}{0pt plus 0.3pt}%
  \let\item\@idxitem
}{%
  \ifkorrekturansicht\clearpage\fi
}
\makeatother

\IfFileExists{\jobname-pw.ind}{\input{\jobname-pw.ind}}{}

% Quellenangabe nur in der Leseansicht
\ifkorrekturansicht\else
% Fallback-Definitionen, falls die .tex-Datei \titel etc. nicht gesetzt hat
\providecommand{\titel}{}
\providecommand{\editorInnen}{}
\providecommand{\dateiname}{\jobname}

\vspace{3cm}

\vfill

\footnotesize
\textsc{Quelle}: \titel. Herausgegeben von {\editorInnen}. In: \emph{Arthur Schnitzler: Briefwechsel mit Autorinnen und Autoren}.
 Digitale Edition, https://schnitzler-briefe.acdh.oeaw.ac.at/{\dateiname}.html (Stand \today)
\fi

\end{document}


      