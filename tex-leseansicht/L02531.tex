%% latex-korrekturansicht-vorspann.tex
%% Vorspann für die Korrekturansicht.
%% Lädt die gemeinsame Datei latex-vorspann.tex mit gesetztem Schalter.

\newif\ifkorrekturansicht
\korrekturansichttrue

\input{../tex-inputs/latex-vorspann}


\section[Arthur Schnitzler an Hermann Bahr, 16. 2. 1930]{L02531 Arthur Schnitzler an Hermann Bahr, 16. 2. 1930}
\nopagebreak\mylabel{L02531v}
\rehead{ }\normalsize\beginnumbering\briefempfaengerindex{Bahr, Hermann@\textsc{Bahr, Hermann}!zzzSchnitzler, Arthur@\emph{von Arthur Schnitzler}!1930-02-161@{16. 2. 1930}|(be}
\toendnotes[C]{\smallbreak\pagebreak[2]}\Standort{TMW, HS AM 23398 Ba.}
\physDesc{Brief, 1 Blatt, 2 Seiten, 1138 Zeichen
\newline{}Handschrift: schwarze Tinte, lateinische Kurrent
\newline{}Bahr: mit rotem Buntstift beschriftet: »\uline{Schnitzler}« }
\buchAbdrucke{\weitereDrucke{1) Arthur Schnitzler: \emph{The Letters of Arthur Schnitzler to Hermann Bahr}. Chapel Hill: \emph{The University of North Carolina Press} 1978, S. 116–117.} \weitereDrucke{2) Hermann Bahr, Arthur Schnitzler: \emph{Briefwechsel, Aufzeichnungen, Dokumente (1891–1931)}. Göttingen: \emph{Wallstein} 2018, S. 594.} }\toendnotes[C]{\smallbreak}
\pstart
           \raggedleft{}{\pb}Wien\oindex{Wien@\textbf{Wien}, \emph{A.ADM2}|pw}{ }16. \textcolor{gray}{2}. 930\pend
           \vspace{0.5em}
\pstart
           Mein lieber Hermann, nach so langer Zeit hör\pwindex{Tagebuch. 10. Januar [1930]@\emph{Tagebuch. 10. Januar [1930]}|pwv} ich wieder was von dir – und da
               verleihst du mir gleich den Nobelpreis\orgindex{Bauernfeld-Preis@Bauernfeld-Preis|pw}!\pend
           
\pstart
           Ich fühle ganz wie du: daſs Hugo\pwindex{Hofmannsthal, Hugo von 1874-02-01 – 1929-07-15@\textsc{Hofmannsthal, Hugo von} (1874-02-01 – 1929-07-15), \emph{Schriftsteller/Schriftstellerin}|pw} derjenige
               gewesen ist, der ihn hätte beko{\geminationm}en müssen. Leider könnte
               ich diesmal nicht wieder aussprechen – wie seinerzeit als ich (noch dazu für das Zwischenspiel\pwindex{Zwischenspiel. Komoedie in drei Akten@\emph{Zwischenspiel. Komödie in drei Akten}|pw}!) den \label{K_L02531-1v}\edtext{Grillparzerpreis\orgindex{Franz-Grillparzer-Preis@Franz-Grillparzer-Preis|pw} erhielt}{\lemma{\textnormal{\emph{Grillparzerpreis erhielt}}}\Cendnote{\textnormal{Vgl. A. S.: \emph{Tagebuch}, 15. 1. 1908.
               }}}\label{K_L02531-1}, daſs der eigentlich Hofmannsthal\pwindex{Hofmannsthal, Hugo von 1874-02-01 – 1929-07-15@\textsc{Hofmannsthal, Hugo von} (1874-02-01 – 1929-07-15), \emph{Schriftsteller/Schriftstellerin}|pw}
               gebühre. Auch damit hast du recht: »\uline{melden} werd ich
                  \introOben{}mich\introOben{} nicht, vielleicht weniger aus
                  »Bescheidenheit{[}«{]}, als aus Bequemlichkeit und einer immer
               wachsenden {\pb}Gleichgiltigkeit gegen alle Arten von äußeren »Ehrungen« u was man so nennt.\pend
           
\pstart
           D\substVorne{}\textsuperscript{as}\substDazwischen{}ein\substHinten{} »Tagebuch\pwindex{Tagebuch [Kolumne im Neuen Wiener Journal]@\emph{Tagebuch [Kolumne im Neuen Wiener Journal]}|pw}{[}«{]}{ }\label{LL140-1v}les ich natürlich immer\label{LL140-1h} – so bedürfte
               es also kaum einer \label{K_L02531-2v}\edtext{freundlichen
               persönlichen Bemerkung}{\lemma{\textnormal{\emph{freundlichen … Bemerkung}}}\Cendnote{\textnormal{Bahr\pwindex{Bahr, Hermann 19.07.1863 – 15.01.1934@\textsc{Bahr, Hermann} (19.07.1863 – 15.01.1934), \emph{Schriftsteller/Schriftstellerin, Kritiker/Kritikerin}|pwk} ließ regelmäßig seine Kolumne den
                  darin behandelten Personen zukommen. Diese Textstelle deutet an, dass das \emph{Tagebuch. 10. Januar}\pwindex{Tagebuch. 10. Januar [1930]@\emph{Tagebuch. 10. Januar [1930]}|pwk} in einer Fassung mit
                  einem handschriftlichen Gruß im Besitz Schnitzlers gewesen sein dürfte. In seinen Zeitungsausschnitten (heute in
                     Exeter\oindex{Exeter@\textbf{Exeter}, \emph{P.PPLA2}|pwk}) findet sich aber nur ein Abzug,
                  auf dem sich außer einem Datumsvermerk von Schnitzler keine Beschriftung findet (University of Exeter,
                        \emph{Schnitzler Press Cuttings Archive}, Box
                  42/2).}}}\label{K_L02531-2}, – und umso mehr dank ich dir. Ich weiſs nicht, ob du meine
               kleinen Bücher »Geist im Wort und in der That\pwindex{Geist im Wort und der Geist in der Tat@\emph{Der Geist im Wort und der Geist in der Tat}|pw}«,
               u mein Buch der Sprüche u Bedenken\pwindex{Buch der Sprueche und Bedenken@\emph{Buch der Sprüche und Bedenken}|pw} erhalten hast
               – ich würde sie dir gern schicken, auf die Gefahr hin, dass du mit vielem nicht
               einverstanden sein wirst.\pend
           
\pstart
           Es wär schön wenn man einander wieder sähe {\dotstwo} »\label{K_L02531-3v}\edtext{Einer von uns wird es einmal bedauern}{\lemma{\textnormal{\emph{Einer … bedauern}}}\Cendnote{\textnormal{Siehe Arthur Schnitzler an Hermann Bahr, 22. 6. 1909, Hermann Bahr an Arthur Schnitzler, 28. 6. 1909, Hugo von Hofmannsthal an Olga Schnitzler, 5. 7. [1912].
               }}}\label{K_L02531-3}{ }{\dotstwo}« wie Hugo\pwindex{Hofmannsthal, Hugo von 1874-02-01 – 1929-07-15@\textsc{Hofmannsthal, Hugo von} (1874-02-01 – 1929-07-15), \emph{Schriftsteller/Schriftstellerin}|pw} immer
               sagte. – \pend
           
\pstart
           Ich grüße dich herzlich in alter Freundschaft{\\[\baselineskip]}Dein \spacefill\mbox{Arthur}\pend
           \leftskip=0em{}\selectlanguage{ngerman}\endnumbering\briefempfaengerindex{Bahr, Hermann@\textsc{Bahr, Hermann}!zzzSchnitzler, Arthur@\emph{von Arthur Schnitzler}!1930-02-161@{16. 2. 1930}|)be}\mylabel{L02531h}  \normalsize

\doendnotes{C}
\bigskip
\vfill

\clearpage

\footnotesize

\lohead{\textsc{register}}

% Definiere theindex-Environment komplett neu ohne reledmac
\makeatletter
\renewenvironment{theindex}{%
  \section*{\indexname}%
  \setlength{\parindent}{0pt}%
  \setlength{\parskip}{0pt plus 0.3pt}%
  \let\item\@idxitem
}{%
  \clearpage
}
\makeatother

\IfFileExists{\jobname-pw.ind}{\input{\jobname-pw.ind}}{}

\end{document}

      