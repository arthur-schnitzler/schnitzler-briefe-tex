%% latex-leseansicht-vorspann.tex
%% Vorspann für die Leseansicht.
%% Lädt die gemeinsame Datei latex-vorspann.tex mit nicht gesetztem Schalter.

\newif\ifkorrekturansicht
\korrekturansichtfalse

\input{../tex-inputs/latex-vorspann}


         \renewcommand{\erwaehnteInstitutionen}{Institutionen: Deutschösterreichischer Autorenverband, Vereinigung der österreichischen Richterinnen und Richter}
         \renewcommand{\erwaehnteOrte}{Orte: Wien}
         \renewcommand{\erwaehnteWerke}{}
               \section[Robert Adam an Arthur Schnitzler, 1. – 3. 11. 1920]{ Robert Adam an Arthur Schnitzler, 1. – 3. 11. 1920}\nopagebreak\mylabel{v}\rehead{ }\begin{ledgroupsized}[t]{13cm}\normalsize\beginnumbering \toendnotes[C]{\smallbreak\pagebreak[2]} \Standort{CUL, Schnitzler, B 1.}
\physDesc{Brief, 1 Blatt, 4 Seiten
\newline{}Handschrift: blaue Tinte, deutsche Kurrent
\newline{}Schnitzler: 1) mit Bleistift beschriftet: »\textsc{Adam}«  2) mit rotem Buntstift eine Unterstreichung\newline{}Ordnung: mit Bleistift von unbekannter Hand nummeriert:
                                        »16« }\Standort{Wien, Österreichische Nationalbibliothek, Cod.ser. 52.268, 95 verso, 96.}
\physDesc{Handschriftliche Abschrift, 2 Blätter, 2 Seiten
\newline{}Handschrift: schwarze Tinte, Gabelsberger Kurzschrift}\Standort{Wien, Österreichische Nationalbibliothek, Cod.ser. 52.268, 95 verso, 96.}
\physDesc{Maschinenschriftliche Abschrift, 2 Blätter, 2 Seiten
\newline{}Schreibmaschine}\toendnotes[C]{\smallbreak}\pstart
           \raggedleft{}{\pb}Wien\oindex{Wien@\textbf{Wien}|pw}, am 1. November{\\}1920\pend
           \pstart\center{}Hochverehrter Herr Doktor!\pend\pstart
           Ich habe Ihr Schreiben mit größter Freude geleſen – und mit ebenſogroßem
                    Bedauern; mit Freude darüber, daß Sie die Güte hatten, mich zu einem ſo
                    ehrenvollen und mir in jedem Sinne erſtrebenswerten Amte in Vorſchlag zu
                    bringen; mit Bedauern – denn es iſt mir nach dem derzeitigen Stande der
                    öſterreichiſchen Geſetzgebung unmöglich, dem Rufe Folge zu leiſten. § 578 der
                    Zivilprozeßordnung lautet nämlich: »Richterliche Beamte dürfen, ſolange ſie im
                    richterlichen Dienste ſtehen, die Beſtellung als Schiedsrichter nicht annehmen«,
                    und dieſes Verbot findet im § 595 Z. 3 ſeine Sanktion, wonach Schiedsſprüche
                        wirkungs{\pb}los ſind, wenn
                    hinſichtlich der Beſetzung des Schiedsgerichtes eine geſetzliche Beſtimmung
                    verletzt wurde. Die Teilnahme eines noch aktiven Berufsrichters an dem
                    fraglichen Schiedsgerichte iſt alſo leider unmöglich.\pend
           \pstart
           Sie können ſich leicht vorſtellen, mit welch bitteren Gefühlen ich dieſe
                    unbarmherzigen Paragraphen zitiere.\pend
           \pstart
           Ich werde in den nächſten Tagen im Ausſchuß der Richtervereinigung\orgindex{Vereinigung der oesterreichischen Richterinnen und Richter@Vereinigung der österreichischen Richterinnen und Richter|pw} anregen, daß unter die anläßlich der
                    Beſoldungsreform von den Richtern zu ſtellenden Forderungen auch die nach
                    Streichung des § 578 ZPO – der jetzt vollkommen obſolet und der unnötige
                    Ausdruck eines den Richtern gegenüber bei Schaffung des Geſetzes gehegten
                    Mißtrauens iſt – aufgenommen werde, und ich bin ziemlich ſicher, mit meiner
                    Anregung durchzudringen: \label{K_L02357_1v}\edtext{ob aber
                    die Streichung ſo bald erfolgen wird}{\lemma{\textnormal{\emph{ob … wird}}}\Cendnote{\textnormal{§ 578 der Zivilprozessordnung vom 1. Januar 1898 blieb in Kraft bis zum
                        30. Juni 2006.}}}\label{K_L02357_1h}, daß für den Verein\orgindex{Deutschoesterreichischer Autorenverband@Deutschösterreichischer Autorenverband|pwv} meine Perſon noch in Betracht kommen könnte,
                    iſt doch ſehr zweifelhaft.\pend
           \pstart
           {\pb}Es bleibt mir demnach nichts übrig,
                    als Ihnen, hochverehrter Herr Doktor, auf’s herzlichſte zu danken und Sie zu
                    bitten, meinen Dank den andern Herren der Genoſſenſchaft\orgindex{Deutschoesterreichischer Autorenverband@Deutschösterreichischer Autorenverband|pwv} zugleich mit der Verſicherung zu
                    übermitteln, daß \uline{nur} die erwähnte
                    Geſetzesbeſtimmung mich abhält, das Anerbieten anzunehmen.\pend
           \pstart
           Mit den beſten Grüßen Ihr{\\[\baselineskip]}ſehr ergebener{\\[\baselineskip]}\spacefill\mbox{D\textsuperscript{r}RAdam.}\pend
           \leftskip=0em{}\pstart
           \noindent{}Nachſchrift vom 3. November:\pend
           \pstart
           Ich bitte wegen Verzögerung der Abſendung des Briefes um Entſchuldigung. Ich
                    wollte vorher durch Nachfrage bei Kollegen mir Sicherheit verſchaffen, ob meine
                    Rechtsanſicht wirklich die richtige ſei und ob nicht etwa doch für mich eine
                    Möglichkeit beſtehe, Ihnen – wie ich gerne wünſchte – andern Beſcheid zu ſenden.
                    Aber {\pb}das Geſetz ſteht ſtarr und
                    unbeugſam da.\pend
           \pstart
           Nochmals die beſten Grüße und vielen Dank!{\\[\baselineskip]}Ihr{\\[\baselineskip]}\spacefill\mbox{D\textsuperscript{r}RAdam}\pend
           \leftskip=0em{}
         
         \endnumbering\mylabel{h}\end{ledgroupsized}  \newcommand{\dateiname}{L02357}\newcommand{\titel}{Robert Adam an Arthur Schnitzler, 1. – 3. 11. 1920}\newcommand{\editorInnen}{Martin Anton Müller und Gerd-Hermann Susen}%% latex-leseansicht-abspann.tex
%% Abspann für die Leseansicht.
%% Der Schalter \ifkorrekturansicht ist bereits durch den Vorspann gesetzt.

%% latex-abspann.tex
%% Gemeinsamer Abspann für Korrekturansicht und Leseansicht.
%% Setzt den Schalter \ifkorrekturansicht voraus (gesetzt in den
%% einbindenden Dateien latex-korrekturansicht-abspann.tex bzw.
%% latex-leseansicht-abspann.tex).
%% ---------------------------------------------------------------

\normalsize

% Das esempio-Environment wird nur in der Leseansicht benötigt
\ifkorrekturansicht\else
\newenvironment{esempio}[3]%
{
    \vspace{1.5ex}
    \rlap{\underline{#1}}
    \par
    \setlength{\parindent}{0cm}
    \nopagebreak
    \leftskip=#2cm
    \rightskip=#3cm
}
{
    \par
}
\fi

\doendnotes{C}
\bigskip
\vfill

\clearpage

\footnotesize

\ifkorrekturansicht
  \lohead{\textsc{register}}
\fi

% theindex-Environment neu definieren ohne reledmac
\makeatletter
\renewenvironment{theindex}{%
  \ifkorrekturansicht
    \section*{\indexname}%
  \else
    \subsubsection*{Index der erwähnten Entitäten}%
  \fi
  \setlength{\parindent}{0pt}%
  \setlength{\parskip}{0pt plus 0.3pt}%
  \let\item\@idxitem
}{%
  \ifkorrekturansicht\clearpage\fi
}
\makeatother

\IfFileExists{\jobname-pw.ind}{\input{\jobname-pw.ind}}{}

% Quellenangabe nur in der Leseansicht
\ifkorrekturansicht\else
% Fallback-Definitionen, falls die .tex-Datei \titel etc. nicht gesetzt hat
\providecommand{\titel}{}
\providecommand{\editorInnen}{}
\providecommand{\dateiname}{\jobname}

\vspace{3cm}

\vfill

\footnotesize
\textsc{Quelle}: \titel. Herausgegeben von {\editorInnen}. In: \emph{Arthur Schnitzler: Briefwechsel mit Autorinnen und Autoren}.
 Digitale Edition, https://schnitzler-briefe.acdh.oeaw.ac.at/{\dateiname}.html (Stand \today)
\fi

\end{document}


      