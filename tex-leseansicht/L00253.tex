%% latex-korrekturansicht-vorspann.tex
%% Vorspann für die Korrekturansicht.
%% Lädt die gemeinsame Datei latex-vorspann.tex mit gesetztem Schalter.

\newif\ifkorrekturansicht
\korrekturansichttrue

\input{../tex-inputs/latex-vorspann}


\section[Oscar Blumenthal an Arthur Schnitzler, 12. 8. 1893]{L00253 Oscar Blumenthal an Arthur Schnitzler, 12. 8. 1893}
\nopagebreak\mylabel{L00253v}
\rehead{ }\normalsize\beginnumbering\briefempfaengerindex{Schnitzler, Arthur@\textsc{Schnitzler, Arthur}!zzzBlumenthal, Oskar@\emph{von Oskar Blumenthal}!1893-08-122@{12. 8. 1893}|(be}
\toendnotes[C]{\smallbreak\pagebreak[2]}\Standort{CUL, Schnitzler, B 15.}
\physDesc{Brief, 1 Blatt, 1 Seite, 671 Zeichen
\newline{}Handschrift Schreibkraft: schwarze Tinte, deutsche Kurrent
\newline{}Handschrift Oskar Blumenthal: schwarze Tinte, deutsche Kurrent (\noindent{}Unterschrift)
\newline{}Schnitzler: 1) mit Bleistift auf der Rückseite beschriftet: »\textsc{Blumenthal}«  2) mit rotem Buntstift eine Unterstreichung und nummeriert:
                                    »4«}\toendnotes[C]{\smallbreak}
\pstart
           \centering{}{\pb}\textcolor{gray}{\textbf{\textsc{Lessing-Theater\orgindex{Lessing-Theater@Lessing-Theater|pw}}}}\pend
           
\pstart
           \centering{}\textcolor{gray}{\textbf{\textsc{Director}:}}{\\}\textcolor{gray}{\textbf{DR. OSCAR BLUMENTHAL.}}\pend
           
\pstart
           \raggedleft{}\textcolor{gray}{\textbf{Berlin N.W.\oindex{Berlin@\textbf{Berlin}, \emph{P.PPLC}|pw}, den}}{ }12. August \textcolor{gray}{\textbf{189}}3.\pend
           
\pstart\center{}Werther Herr Doktor!\pend\vspace{0.5em}
\pstart
           Es iſt nicht richtig, daß ich eine Aufführung des »Märchens\pwindex{Maerchen. Schauspiel in drei Aufzuegen@\emph{Das Märchen. Schauspiel in drei Aufzügen}|pw}« für die Sommermonate in Ausſicht genommen hätte. Die
               bisherige Verzögerung erklärt ſich aus der berechtigten Erwägung, daß gerade auf dem
                  Leſſing-Theater\orgindex{Lessing-Theater@Lessing-Theater|pw}{ }ſowohl in der letzten wie in der vorletzten Saiſon
               die in Ihrem Stücke\pwindex{Maerchen. Schauspiel in drei Aufzuegen@\emph{Das Märchen. Schauspiel in drei Aufzügen}|pwv}
               aufgeworfene Frage, in welcher Weiſe die Vergangenheit eines Mädchens auf ihr
               gegenwärtiges Schickſal einwirkt, allzu oft behandelt iſt, ſo daß augenblicklich
               dieſes Thema auf ermüdete Hörer treffen würde. Ich habe gleichwohl den Plan der
               Darſtellung keineswegs aufgegeben und werde Sie zur Zeit verſtändigen.\hspace*{2.5em}Mit hochachtungsvollem Gruß\pend
           \pstart \spacefill\mbox{{[}hs. :{]} Dr. Osc. Blumenthal}\pend{}\selectlanguage{ngerman}\endnumbering\briefempfaengerindex{Schnitzler, Arthur@\textsc{Schnitzler, Arthur}!zzzBlumenthal, Oskar@\emph{von Oskar Blumenthal}!1893-08-122@{12. 8. 1893}|)be}\mylabel{L00253h}  \normalsize

\doendnotes{C}
\bigskip
\vfill

\clearpage

\footnotesize

\lohead{\textsc{register}}

% Definiere theindex-Environment komplett neu ohne reledmac
\makeatletter
\renewenvironment{theindex}{%
  \section*{\indexname}%
  \setlength{\parindent}{0pt}%
  \setlength{\parskip}{0pt plus 0.3pt}%
  \let\item\@idxitem
}{%
  \clearpage
}
\makeatother

\IfFileExists{\jobname-pw.ind}{\input{\jobname-pw.ind}}{}

\end{document}

      