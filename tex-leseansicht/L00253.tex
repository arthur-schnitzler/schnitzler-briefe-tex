\input{../tex-inputs/latex-pdf-vorspann}
\begin{center}
            \textcolor{red}{ENTWURF. ENTZIFFERUNG NOCH NICHT KORREKTURGELESEN}
                      \end{center}
            
               \section[Oscar Blumenthal an Arthur Schnitzler, 12. 8. 1893]{ Oscar Blumenthal an Arthur Schnitzler, 12. 8. 1893}\nopagebreak\mylabel{v}\rehead{ }\begin{ledgroupsized}[t]{13cm}\normalsize\beginnumbering\briefempfaengerindex{Schnitzler, Arthur@\textsc{Schnitzler, Arthur}!zzzBlumenthal, Oskar@\emph{von Oskar Blumenthal}!1893-08-122@{12. 8. 1893}|(be} \toendnotes[C]{\smallbreak\pagebreak[2]} \Standort{CUL, Schnitzler, B 15.}
\physDesc{Brief, 1 Blatt, 1 Seite
\newline{}Handschrift  : schwarze Tinte, deutsche Kurrent\newline{}Handschrift Oskar Blumenthal: schwarze Tinte, deutsche Kurrent (\noindent{}Unterschrift)
\newline{}Schnitzler: 1) mit Bleistift auf der Rückseite beschriftet: »\textsc{Blumenthal}« 2) mit rotem Buntstift eine Unterstreichung und nummeriert:
                                    »4«}\toendnotes[C]{\smallbreak}\pstart
           \noindent{}\centering{}{\pb}\textcolor{gray}{\textbf{\textsc{Lessing-Theater\orgindex{Lessing-Theater@Lessing-Theater|pw}}}}\pend
           \pstart
           \noindent{}\centering{}\textcolor{gray}{\textbf{\textsc{Director}:}}{\\}\textcolor{gray}{\textbf{DR. OSCAR BLUMENTHAL.}}\pend
           \pstart
           \noindent{}\raggedleft{}\textcolor{gray}{\textbf{Berlin N.W.\oindex{Berlin@\textbf{Berlin}|pw}, den}}{ }12. August \textcolor{gray}{\textbf{189}}3.\pend
           \pstart\center{}Werther Herr Doktor!\pend\pstart
           Es iſt nicht richtig, daß ich eine Aufführung des »Märchens\pwindex{Schnitzler, Arthur 15.05.1862 – 21.10.1931@\textsc{Schnitzler, Arthur} (15.05.1862 – 21.10.1931), \emph{Schriftsteller, Mediziner}!Maerchen. Schauspiel in drei Aufzuegen1891 – 1891@\strich\emph{Das Märchen. Schauspiel in drei Aufzügen} {[}1891 – 1891{]}|pw}« für die Sommermonate in Ausſicht genommen hätte. Die
               bisherige Verzögerung erklärt ſich aus der berechtigten Erwägung, daß gerade auf dem
                  Leſſing-Theater\orgindex{Lessing-Theater@Lessing-Theater|pw}{ }ſowohl in der letzten wie in der vorletzten Saiſon
               die in Ihrem Stücke\pwindex{Schnitzler, Arthur 15.05.1862 – 21.10.1931@\textsc{Schnitzler, Arthur} (15.05.1862 – 21.10.1931), \emph{Schriftsteller, Mediziner}!Maerchen. Schauspiel in drei Aufzuegen1891 – 1891@\strich\emph{Das Märchen. Schauspiel in drei Aufzügen} {[}1891 – 1891{]}|pwv} aufgeworfene
               Frage, in welcher Weiſe die Vergangenheit eines Mädchens auf ihr gegenwärtiges
               Schickſal einwirkt, allzu oft behandelt iſt, ſo daß augenblicklich dieſes Thema auf
               ermüdete Hörer treffen würde. Ich habe gleichwohl den Plan der Darſtellung keineswegs
               aufgegeben und werde Sie zur Zeit verſtändigen.\hspace*{2.5em}Mit
               hochachtungsvollem Gruß\pend
           \pstart \spacefill\mbox{{[}hs. Blumenthal:{]} Dr. Osc. Blumenthal}\pend{}\endnumbering\briefempfaengerindex{Schnitzler, Arthur@\textsc{Schnitzler, Arthur}!zzzBlumenthal, Oskar@\emph{von Oskar Blumenthal}!1893-08-122@{12. 8. 1893}|)be}\mylabel{h}\end{ledgroupsized}  \newcommand{\dateiname}{L00253}\newcommand{\titel}{Oscar Blumenthal an Arthur Schnitzler, 12. 8. 1893}\newcommand{\editorInnen}{Martin Anton Müller und Gerd-Hermann Susen}\input{../tex-inputs/latex-pdf-abspann}
      