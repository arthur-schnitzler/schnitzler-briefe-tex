%% latex-leseansicht-vorspann.tex
%% Vorspann für die Leseansicht.
%% Lädt die gemeinsame Datei latex-vorspann.tex mit nicht gesetztem Schalter.

\newif\ifkorrekturansicht
\korrekturansichtfalse

\input{../tex-inputs/latex-vorspann}


         
         \renewcommand{\erwaehntePersonen}{Personen: Josef Kaufmann, Marie Reinhard, Felix Salten}
         \renewcommand{\erwaehnteInstitutionen}{Institutionen: Jung-Wiener Theater zum Lieben Augustin}
         \renewcommand{\erwaehnteOrte}{Orte: Raimund-Theater, Wien}
         \renewcommand{\erwaehnteWerke}{Werke: Tagebuch}
               \section[ Arthur Schnitzler an Felix Salten, {[}14. 5. 1896?{]}]{ Arthur Schnitzler an Felix Salten, {[}14. 5. 1896?{]}}\nopagebreak\mylabel{v}\rehead{ }\begin{ledgroupsized}[t]{13cm}\normalsize\beginnumbering\briefempfaengerindex{Salten, Felix@\textsc{Salten, Felix}!zzzSchnitzler, Arthur@\emph{von Arthur Schnitzler}!1896-05-142@{{[}14. 5. 1896?{]}}|(be} \toendnotes[C]{\smallbreak\pagebreak[2]} \Standort{Wienbibliothek im Rathaus, ZPH 1681, 2.1.516.}
\physDesc{Brief, 1 Blatt, 2 Seiten, 374 Zeichen
\newline{}Handschrift: Bleistift, deutsche Kurrent
\newline{}Ordnung: mit Bleistift von unbekannter Hand nummeriert: »2« }\toendnotes[C]{\smallbreak}\pstart
           \noindent{}{\pb}lieber, wenn es Ihnen alſo keine Umſtände macht, bitte
               ſehr, laſſen Sie mir folgendes für den \label{K_L03035-1v}\edtext{16.}{\lemma{\textnormal{\emph{16.}}}\Cendnote{\textnormal{Das Korrespondenzstück ist undatiert. Vier enthaltene
                  Details geben Hinweise für eine mögliche Datierung. 1.) Es werden (Theater-)Karten für eine
                  Aufführung am 16. eines Monats erbeten. 2.) Dieser Tag ist ein Samstag. 3.) Das 
                  Theater enthält eine »2. Gallerie«. 4.) Schnitzler\pwindex{Schnitzler, Arthur 15.05.1862 – 21.10.1931@\textsc{Schnitzler, Arthur} (15.05.1862 – 21.10.1931), \emph{Schriftsteller, Mediziner}|pwk}
                  möchte mit jemandem ins Theater gehen, ohne gemeinsam aufzutreten. Die letzten beiden Punkte 
                  machen es unwahrscheinlich,
                  dass die am 16. 11. 1901 stattfindende
                  Premiere des von Salten\pwindex{Salten, Felix 06.09.1869 – 08.10.1945@\textsc{Salten, Felix} (06.09.1869 – 08.10.1945), \emph{Schriftsteller, Journalist, Chefredakteur}|pwk} geleiteten Kabaretts \emph{Jung-Wiener Theater zum Lieben Augustin}\orgindex{Jung-Wiener Theater zum Lieben Augustin@Jung-Wiener Theater zum Lieben Augustin|pwk}
                  gemeint ist. (Auch besuchte Schnitzler\pwindex{Schnitzler, Arthur 15.05.1862 – 21.10.1931@\textsc{Schnitzler, Arthur} (15.05.1862 – 21.10.1931), \emph{Schriftsteller, Mediziner}|pwk} die Generalprobe, sodass
                  potentiell auch dies hier mitdiskutiert werden müsste.) Berücksichtigt man ausschließlich Theater, die
                  über mehrere Galerien verfügten, so reduziert sich die Zahl möglicher Termine stark. Eine Anwesenheit von
                  Salten\pwindex{Salten, Felix 06.09.1869 – 08.10.1945@\textsc{Salten, Felix} (06.09.1869 – 08.10.1945), \emph{Schriftsteller, Journalist, Chefredakteur}|pwk} kann nur zu einem der infrage kommenden Termine belegt werden
                  (16. 10. 1897), doch spricht die Erwähnung
                  mehrerer anderer Kollegen in und nach dem Theater im \emph{Tagebuch}\pwindex{\textcolor{red}{\textsuperscript{XXXX1 indx}}!Tagebuch1981 – 2000@\strich\emph{Tagebuch} {[}Hrsg., 1981 – 2000{]}|pwk}-Eintrag 
                  dagegen. So bleibt Schnitzlers\pwindex{Schnitzler, Arthur 15.05.1862 – 21.10.1931@\textsc{Schnitzler, Arthur} (15.05.1862 – 21.10.1931), \emph{Schriftsteller, Mediziner}|pwk} Besuch
                  im Raimund-Theater\oindex{Raimund-Theater@\textbf{Raimund-Theater}|pwk} am 16. 5. 1896.
                  Einer der Sitze in der anderen Reihe wäre dann für Schnitzlers\pwindex{Schnitzler, Arthur 15.05.1862 – 21.10.1931@\textsc{Schnitzler, Arthur} (15.05.1862 – 21.10.1931), \emph{Schriftsteller, Mediziner}|pwk} Partnerin Marie Reinhard\pwindex{Reinhard, Marie 1871-03-13 – 1899-03-18@\textsc{Reinhard, Marie} (1871-03-13 – 1899-03-18), \emph{Gesangspädagogin}|pwk}
                  gedacht, ein weiterer Platz für Josef Kaufmann\pwindex{Kaufmann, Josef 15.06.1868 – 23.12.1922@\textsc{Kaufmann, Josef} (15.06.1868 – 23.12.1922), \emph{Kaufmann}|pwk}. 
                  Für wen der vierte Sitz vorgesehen war, bleibt offen.}}}\label{K_L03035-1h} reſerviren\pend
           \pstart
           \uline{2. Gallerie}, 1. Reihe\pend
           \pstart
           \textcolor{gray}{W}e{\geminationn} irgend möglich Mittelgang
               Ecke 2 Sitze und \introOben{}(etwa)\introOben{} gleich dahinter 2. Reihe – noch 2,
               alſo {\pb}im ganzen 4 Sitze.\pend
           \pstart
           Vielleicht ſtecken Sie die Sitze zu ſich? oder ſchicken Sie mir? oder ich hol ſie ab?
               oder Sie bringen ſie mir Samſtag –?\pend
           \pstart
           Herzlichſt Ihr {\\[\baselineskip]}\spacefill\mbox{Arthur}\pend
           \leftskip=0em{}
         
         \endnumbering\mylabel{h}\end{ledgroupsized}  \newcommand{\dateiname}{L03035}\newcommand{\titel}{Arthur Schnitzler an Felix Salten, [14. 5. 1896?]}\newcommand{\editorInnen}{Martin Anton Müller und Laura Untner}%% latex-leseansicht-abspann.tex
%% Abspann für die Leseansicht.
%% Der Schalter \ifkorrekturansicht ist bereits durch den Vorspann gesetzt.

%% latex-abspann.tex
%% Gemeinsamer Abspann für Korrekturansicht und Leseansicht.
%% Setzt den Schalter \ifkorrekturansicht voraus (gesetzt in den
%% einbindenden Dateien latex-korrekturansicht-abspann.tex bzw.
%% latex-leseansicht-abspann.tex).
%% ---------------------------------------------------------------

\normalsize

% Das esempio-Environment wird nur in der Leseansicht benötigt
\ifkorrekturansicht\else
\newenvironment{esempio}[3]%
{
    \vspace{1.5ex}
    \rlap{\underline{#1}}
    \par
    \setlength{\parindent}{0cm}
    \nopagebreak
    \leftskip=#2cm
    \rightskip=#3cm
}
{
    \par
}
\fi

\doendnotes{C}
\bigskip
\vfill

\clearpage

\footnotesize

\ifkorrekturansicht
  \lohead{\textsc{register}}
\fi

% theindex-Environment neu definieren ohne reledmac
\makeatletter
\renewenvironment{theindex}{%
  \ifkorrekturansicht
    \section*{\indexname}%
  \else
    \subsubsection*{Index der erwähnten Entitäten}%
  \fi
  \setlength{\parindent}{0pt}%
  \setlength{\parskip}{0pt plus 0.3pt}%
  \let\item\@idxitem
}{%
  \ifkorrekturansicht\clearpage\fi
}
\makeatother

\IfFileExists{\jobname-pw.ind}{\input{\jobname-pw.ind}}{}

% Quellenangabe nur in der Leseansicht
\ifkorrekturansicht\else
% Fallback-Definitionen, falls die .tex-Datei \titel etc. nicht gesetzt hat
\providecommand{\titel}{}
\providecommand{\editorInnen}{}
\providecommand{\dateiname}{\jobname}

\vspace{3cm}

\vfill

\footnotesize
\textsc{Quelle}: \titel. Herausgegeben von {\editorInnen}. In: \emph{Arthur Schnitzler: Briefwechsel mit Autorinnen und Autoren}.
 Digitale Edition, https://schnitzler-briefe.acdh.oeaw.ac.at/{\dateiname}.html (Stand \today)
\fi

\end{document}


      