%% latex-korrekturansicht-vorspann.tex
%% Vorspann für die Korrekturansicht.
%% Lädt die gemeinsame Datei latex-vorspann.tex mit gesetztem Schalter.

\newif\ifkorrekturansicht
\korrekturansichttrue

\input{../tex-inputs/latex-vorspann}


\section[Arthur Schnitzler an Richard Beer-Hofmann, {[}zwischen 3. und 15. 7. 1893?{]}]{L00232 Arthur Schnitzler an Richard Beer-Hofmann, {[}zwischen 3. und
               15. 7. 1893?{]}}
\nopagebreak\mylabel{L00232v}
\rehead{ }\normalsize\beginnumbering\briefempfaengerindex{Beer-Hofmann, Richard@\textsc{Beer-Hofmann, Richard}!zzzSchnitzler, Arthur@\emph{von Arthur Schnitzler}!1893-07-152@{{[}zwischen 3. und
                  15. 7. 1893?{]}}|(be}
\toendnotes[C]{\smallbreak\pagebreak[2]}\Standort{YCGL, MSS 31.}
\physDesc{Brief, 1 Blatt, 2 Seiten, Umschlag, 185 Zeichen
\newline{}Handschrift: Bleistift, deutsche Kurrent
\newline{}Versand: ohne postalischen Übermittlungsvermerk }\toendnotes[C]{\smallbreak}\pstart{}{\pb}Herrn \textsc{Dr. Richard
                     Beer-Hofmann}\pend{}\pstart{}\textsc{bei Hr. Johann Strauss\pwindex{Strauss, Johann 25.10.1825 – 03.06.1899@\textsc{Strauss, Johann} (25.10.1825 – 03.06.1899), \emph{Komponist/Komponistin, Dirigent/Dirigentin}|pw}}\pend{}\pstart{}\textsc{Villa Erdödy}\oindex{Villa Erdoedy@\textbf{Villa Erdödy}, \emph{Gebäude (K.GBD)}|pw}.\pend{}{\bigskip}\vspace{1em}
\pstart
           \noindent{}{\pb}Lieber Richard, – ich bleibe Nachmittags zu Hauſe. Ko{\geminationm}en Sie einfach direct von \label{K_L00232-1v}\edtext{\textsc{Str.}’s\pwindex{Strauss, Johann 25.10.1825 – 03.06.1899@\textsc{Strauss, Johann} (25.10.1825 – 03.06.1899), \emph{Komponist/Komponistin, Dirigent/Dirigentin}|pw}}{\lemma{\textnormal{\emph{Str.’s}}}\Cendnote{\textnormal{Das Korrespondenzstück ist undatiert.
                  Einzig für den Aufenthalt Schnitzlers vom
                  2. 7. 1893 – 15. 7. 1893 lassen sich Begegnungen mit Johann Strauss\pwindex{Strauss, Johann 25.10.1825 – 03.06.1899@\textsc{Strauss, Johann} (25.10.1825 – 03.06.1899), \emph{Komponist/Komponistin, Dirigent/Dirigentin}|pwk} in seinem \emph{Tagebuch}\pwindex{Tagebuch@\emph{Tagebuch}|pwk} ausmachen. Ob Beer-Hofmanns\pwindex{Beer-Hofmann, Richard 1866-07-11 – 1945-09-26@\textsc{Beer-Hofmann, Richard} (1866-07-11 – 1945-09-26), \emph{Schriftsteller/Schriftstellerin}|pwk} Kontakt in der gleichen Zeit stattfand oder länger bestand,
                  ist nicht zu klären.}}}\label{K_L00232-1} zu mir {\pb}herüber.\pend
           \pstart Herzlich grüßt Ihr \spacefill\mbox{Arthur}\pend{}\selectlanguage{ngerman}\endnumbering\briefempfaengerindex{Beer-Hofmann, Richard@\textsc{Beer-Hofmann, Richard}!zzzSchnitzler, Arthur@\emph{von Arthur Schnitzler}!1893-07-032@{{[}zwischen 3. und
                  15. 7. 1893?{]}}|)be}\mylabel{L00232h}  \normalsize

\doendnotes{C}
\bigskip
\vfill

\clearpage

\footnotesize

\lohead{\textsc{register}}

% Definiere theindex-Environment komplett neu ohne reledmac
\makeatletter
\renewenvironment{theindex}{%
  \section*{\indexname}%
  \setlength{\parindent}{0pt}%
  \setlength{\parskip}{0pt plus 0.3pt}%
  \let\item\@idxitem
}{%
  \clearpage
}
\makeatother

\IfFileExists{\jobname-pw.ind}{\input{\jobname-pw.ind}}{}

\end{document}

      