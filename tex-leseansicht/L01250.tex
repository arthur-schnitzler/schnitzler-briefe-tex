%% latex-leseansicht-vorspann.tex
%% Vorspann für die Leseansicht.
%% Lädt die gemeinsame Datei latex-vorspann.tex mit nicht gesetztem Schalter.

\newif\ifkorrekturansicht
\korrekturansichtfalse

\input{../tex-inputs/latex-vorspann}


         
         \renewcommand{\erwaehntePersonen}{Personen: Heinrich von Kleist, Adalbert Franz Seligmann, William Shakespeare}
         \renewcommand{\erwaehnteOrte}{Orte: Wien}
         \renewcommand{\erwaehnteWerke}{Werke: Der Schleier der Beatrice. Schauspiel in fünf Akten, Lebendige Stunden. Vier Einakter}
               \section[Adalbert Seligmann an Arthur Schnitzler, 21. 11. 1902]{ Adalbert Seligmann an Arthur Schnitzler, 21. 11. 1902}\nopagebreak\mylabel{v}\rehead{ }\begin{ledgroupsized}[t]{13cm}\normalsize\beginnumbering \toendnotes[C]{\smallbreak\pagebreak[2]} \Standort{CUL, Schnitzler, B 97.}
\physDesc{Brief, 1 Blatt, 4 Seiten, 1618 Zeichen
\newline{}Handschrift: schwarze Tinte, deutsche Kurrent
\newline{}Schnitzler: 1) mit Bleistift beschriftet: »\textsc{Seligmann}« und nummeriert: »4«   2) mit rotem Buntstift eine Unterstreichung}\toendnotes[C]{\smallbreak}\pstart
           \noindent{}{\pb}Verehrter Freund! Vor allem Verzeihung, daſs ich Ihnen bis jetzt
               nicht für die Ueberſendung Ihrer beiden \label{K_L01250-1v}\edtext{Werke\pwindex{Schnitzler, Arthur 15.05.1862 – 21.10.1931@\textsc{Schnitzler, Arthur} (15.05.1862 – 21.10.1931), \emph{Schriftsteller, Mediziner}!Schleier der Beatrice. Schauspiel in fuenf Akten1900-12-01@\strich\emph{Der Schleier der Beatrice. Schauspiel in fünf Akten} {[}1900-12-01{]}|pwv}\pwindex{Schnitzler, Arthur 15.05.1862 – 21.10.1931@\textsc{Schnitzler, Arthur} (15.05.1862 – 21.10.1931), \emph{Schriftsteller, Mediziner}!Lebendige Stunden. Vier Einakter1901-12-23@\strich\emph{Lebendige Stunden. Vier Einakter} {[}1901-12-23{]}|pwv}}{\lemma{\textnormal{\emph{Werke}}}\Cendnote{\textnormal{Obzwar im Folgenden nicht genannt,
                  dürfte es sich um Schnitzler\pwindex{Schnitzler, Arthur 15.05.1862 – 21.10.1931@\textsc{Schnitzler, Arthur} (15.05.1862 – 21.10.1931), \emph{Schriftsteller, Mediziner}|pwk}s einzige
                  Neuerscheinung in Buchform des Jahres 1902 handeln, die vier Einakter
                     \emph{Lebendige Stunden}\pwindex{Schnitzler, Arthur 15.05.1862 – 21.10.1931@\textsc{Schnitzler, Arthur} (15.05.1862 – 21.10.1931), \emph{Schriftsteller, Mediziner}!Lebendige Stunden. Vier Einakter1901-12-23@\strich\emph{Lebendige Stunden. Vier Einakter} {[}1901-12-23{]}|pwk}.}}}\label{K_L01250-1h} gedankt habe.
               Aber ich wollte nicht früher ſchreiben, als bis ich den »Schleier der Beatrice\pwindex{Schnitzler, Arthur 15.05.1862 – 21.10.1931@\textsc{Schnitzler, Arthur} (15.05.1862 – 21.10.1931), \emph{Schriftsteller, Mediziner}!Schleier der Beatrice. Schauspiel in fuenf Akten1900-12-01@\strich\emph{Der Schleier der Beatrice. Schauspiel in fünf Akten} {[}1900-12-01{]}|pw}{[}«{]}, über den ich mancherlei gehört, auch geleſen hätte; und ich
               bin in diesen Tagen durch mannigfache Arbeit und ſonſtige Scherereien nicht gleich
               dazu gekommen. – Ich weiß, daſs nichts lächerlicher iſt, als wenn man einem Künſtler
               über sein {\pb}Werk\strikeout{e} Dinge ſagt, die er ſelber viel beſſer weiß. Darum nur ſo viel: Ich halte
               dieſe Arbeit für Ihre dichteriſch bedeutendſte. Die Idee, eine Handlung unter dem
               Hochdruck, den das Vorgefühl \substVorne{}\textsuperscript{eines}\substDazwischen{}des\substHinten{} unentrinnbaren Untergangs erzeugt, ſpielen zu laſſen, und dadurch alle
               Hemmungen fortzuſchaffen, die ſich den immerhin etwas wunderlichen Begebenheiten
               ſonſt hindernd in den Weg ſtellen möchten, finde ich genial! Die Geſtalt der Beatrice\pwindex{Schnitzler, Arthur 15.05.1862 – 21.10.1931@\textsc{Schnitzler, Arthur} (15.05.1862 – 21.10.1931), \emph{Schriftsteller, Mediziner}!Schleier der Beatrice. Schauspiel in fuenf Akten1900-12-01@\strich\emph{Der Schleier der Beatrice. Schauspiel in fünf Akten} {[}1900-12-01{]}|pwv}{ }{\pb}unglaublich rührend und – wahr! Dabei
               alles trotz der ſchwülen Atmosphäre keinen Augenblick verletzend oder unfein!
               Allerdings geſteh’ ich, begreife ich ganz gut daſs ein Theaterdirector das Werk ſich
               nicht aufzuführen getraut. Unſer Publicum, das täglich gemeiner wird – beachten Sie,
               bei welchen Stellen in einem Shakeſpeare\pwindex{Shakespeare, William 23.4.1564? – 03.05.1616@\textsc{Shakespeare, William} (23.4.1564? – 03.05.1616), \emph{Schauspieler, Dramatiker}|pw}ſtück
               gelacht wird – würde die Subtilität der pſychologiſchen Vorgänge gewiß nicht
               verſtehen – da es ſich um das Werk eines Zeitgenoſſen handelt. Wenn Sie {\pb}Kleiſt\pwindex{Kleist, Heinrich von 18.10.1777 – 21.11.1811@\textsc{Kleist, Heinrich von} (18.10.1777 – 21.11.1811), \emph{Schriftsteller}|pw} oder ſo jemand wären – \textsc{\label{K_L01250-2v}\edtext{à la bonheur}{\lemma{\textnormal{\emph{à la bonheur}}}\Cendnote{\textnormal{französisch: auf gut Glück}}}\label{K_L01250-2h}}! Aber für einen Kreis verſtändiger und dichteriſch empfindender Menſchen wird
               Ihr Werk ein wahrer Genuß ſein und bleiben. Ich danke Ihnen noch \uline{ſehr} für Ihre Liebenswürdigkeit und\pend
           \pstart
           bin Ihr{\\[\baselineskip]}ſtets ergebener{\\[\baselineskip]}\spacefill\mbox{Seligmann}\pend
           \leftskip=0em{}\pstart
           Wien\oindex{Wien@\textbf{Wien}|pw}{ }21 Nov. 1902.\pend
           
         
         \endnumbering\mylabel{h}\end{ledgroupsized}  \newcommand{\dateiname}{L01250}\newcommand{\titel}{Adalbert Seligmann an Arthur Schnitzler, 21. 11. 1902}\newcommand{\editorInnen}{Martin Anton Müller und Gerd-Hermann Susen}%% latex-leseansicht-abspann.tex
%% Abspann für die Leseansicht.
%% Der Schalter \ifkorrekturansicht ist bereits durch den Vorspann gesetzt.

%% latex-abspann.tex
%% Gemeinsamer Abspann für Korrekturansicht und Leseansicht.
%% Setzt den Schalter \ifkorrekturansicht voraus (gesetzt in den
%% einbindenden Dateien latex-korrekturansicht-abspann.tex bzw.
%% latex-leseansicht-abspann.tex).
%% ---------------------------------------------------------------

\normalsize

% Das esempio-Environment wird nur in der Leseansicht benötigt
\ifkorrekturansicht\else
\newenvironment{esempio}[3]%
{
    \vspace{1.5ex}
    \rlap{\underline{#1}}
    \par
    \setlength{\parindent}{0cm}
    \nopagebreak
    \leftskip=#2cm
    \rightskip=#3cm
}
{
    \par
}
\fi

\doendnotes{C}
\bigskip
\vfill

\clearpage

\footnotesize

\ifkorrekturansicht
  \lohead{\textsc{register}}
\fi

% theindex-Environment neu definieren ohne reledmac
\makeatletter
\renewenvironment{theindex}{%
  \ifkorrekturansicht
    \section*{\indexname}%
  \else
    \subsubsection*{Index der erwähnten Entitäten}%
  \fi
  \setlength{\parindent}{0pt}%
  \setlength{\parskip}{0pt plus 0.3pt}%
  \let\item\@idxitem
}{%
  \ifkorrekturansicht\clearpage\fi
}
\makeatother

\IfFileExists{\jobname-pw.ind}{\input{\jobname-pw.ind}}{}

% Quellenangabe nur in der Leseansicht
\ifkorrekturansicht\else
% Fallback-Definitionen, falls die .tex-Datei \titel etc. nicht gesetzt hat
\providecommand{\titel}{}
\providecommand{\editorInnen}{}
\providecommand{\dateiname}{\jobname}

\vspace{3cm}

\vfill

\footnotesize
\textsc{Quelle}: \titel. Herausgegeben von {\editorInnen}. In: \emph{Arthur Schnitzler: Briefwechsel mit Autorinnen und Autoren}.
 Digitale Edition, https://schnitzler-briefe.acdh.oeaw.ac.at/{\dateiname}.html (Stand \today)
\fi

\end{document}


      