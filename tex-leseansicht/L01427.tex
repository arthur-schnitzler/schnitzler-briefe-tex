%% latex-leseansicht-vorspann.tex
%% Vorspann für die Leseansicht.
%% Lädt die gemeinsame Datei latex-vorspann.tex mit nicht gesetztem Schalter.

\newif\ifkorrekturansicht
\korrekturansichtfalse

\input{../tex-inputs/latex-vorspann}


         
         \newcommand{\erwaehntePersonen}{Personen: Hermann Bahr, Olga Schnitzler}
         \newcommand{\erwaehnteOrte}{Orte: Kuffner Bierbrauerei, Ober Sankt Veit, Ottakringer Bräu, Veitlissengasse, Wien, XIII., Hietzing, XVI., Ottakring, XVIII., Währing}
         \newcommand{\erwaehnteWerke}{
               \section[Arthur Schnitzler an Hermann Bahr, 17. 8. 1904]{ Arthur Schnitzler an Hermann Bahr, 17. 8. 1904}\nopagebreak\mylabel{v}\rehead{ }\begin{ledgroupsized}[t]{13cm}\normalsize\beginnumbering \toendnotes[C]{\smallbreak\pagebreak[2]} \Standort{TMW, HS AM 23366 Ba.}
\physDesc{Kartenbrief
\newline{}Handschrift: Bleistift, deutsche Kurrent\newline{}Versand: 1) Stempel: »\nobreak{}\oindex{XVIII., Waehring@\textbf{XVIII., Währing}|pwk}Wien 18, 17. VIII 04\nobreak{}«.   2) Stempel: »\nobreak{}\oindex{XIII., Hietzing@\textbf{XIII., Hietzing}|pwk}Wien 13/7, 18. 8. 04, 8. V, Bestellt\nobreak{}«. }\buchAbdrucke{\weitereDrucke{1) \emph{17. 8. 1904.} In: Arthur Schnitzler: \emph{The Letters of Arthur Schnitzler to Hermann Bahr}. Edited, annotated, and with an introduction, by Donald G.
                        Daviau. Chapel Hill: \emph{The University of North Carolina Press} 1978, S. 85 (University of North Carolina studies in the Germanic languages
                        and literatures, 89).} \weitereDrucke{2) Hermann Bahr, Arthur Schnitzler: \emph{Briefwechsel, Aufzeichnungen, Dokumente (1891–1931)}. Hg. Kurt Ifkovits und Martin Anton Müller. Göttingen: \emph{Wallstein} 2018, S. 316.} }\toendnotes[C]{\smallbreak}\pstart{}{\pb}Herrn Hermann
                  Bahr\pend{}\pstart{}Wien Ob. St. Veit\oindex{Ober Sankt Veit@\textbf{Ober Sankt Veit}|pw}\pend{}\pstart{}Veitliſſengaſſe.\oindex{Veitlissengasse@\textbf{Veitlissengasse}|pw}\pend{}{\bigskip}\pstart
           \raggedleft{}{\pb}17. 8.\pend
           \pstart
           lieber Hermann, wir\pwindex{Schnitzler, Olga 17.01.1882 – 13.01.1970@\textsc{Schnitzler, Olga} (17.01.1882 – 13.01.1970), \emph{Schauspielerin, Sängerin}|pwv} wollen Freitag um 8 in dem
                  \label{K_L01427_1v}\edtext{\textsc{Kuffner} Garten\oindex{Ottakringer Braeu@\textbf{Ottakringer Bräu}|pw}}{\lemma{\textnormal{\emph{Kuffner Garten}}}\Cendnote{\textnormal{Gastgarten des Ottakringer Bräu\oindex{Ottakringer Braeu@\textbf{Ottakringer Bräu}|pwk}; das Lokal gehörte zu der im 16. Wien\oindex{Wien@\textbf{Wien}|pwk}er Gemeindebezirk (Ottakring\oindex{XVI., Ottakring@\textbf{XVI., Ottakring}|pwk}) angesiedelten Brauerei Kuffner\oindex{Kuffner Bierbrauerei@\textbf{Kuffner Bierbrauerei}|pwk}.}}}\label{K_L01427_1h} in Hietzing\oindex{XIII., Hietzing@\textbf{XIII., Hietzing}|pw} nachtmahlen; –
               und hoffen ſehr, wenn du nichts anderes vorhaſt, dich dort zu treffen.\pend
           \pstart
           Ziehſt du ein andres Rendezvous vor, so verſtändige mich.\pend
           \pstart
           Von Herzen{\\[\baselineskip]}dein \spacefill\mbox{Arthur}\pend
           \leftskip=0em{}
         
         \endnumbering\mylabel{h}\end{ledgroupsized}  \newcommand{\dateiname}{L01427}\newcommand{\titel}{Arthur Schnitzler an Hermann Bahr, 17. 8. 1904}\newcommand{\editorInnen}{ Kurt Ifkovits,  Martin Anton Müller}%% latex-leseansicht-abspann.tex
%% Abspann für die Leseansicht.
%% Der Schalter \ifkorrekturansicht ist bereits durch den Vorspann gesetzt.

%% latex-abspann.tex
%% Gemeinsamer Abspann für Korrekturansicht und Leseansicht.
%% Setzt den Schalter \ifkorrekturansicht voraus (gesetzt in den
%% einbindenden Dateien latex-korrekturansicht-abspann.tex bzw.
%% latex-leseansicht-abspann.tex).
%% ---------------------------------------------------------------

\normalsize

% Das esempio-Environment wird nur in der Leseansicht benötigt
\ifkorrekturansicht\else
\newenvironment{esempio}[3]%
{
    \vspace{1.5ex}
    \rlap{\underline{#1}}
    \par
    \setlength{\parindent}{0cm}
    \nopagebreak
    \leftskip=#2cm
    \rightskip=#3cm
}
{
    \par
}
\fi

\doendnotes{C}
\bigskip
\vfill

\clearpage

\footnotesize

\ifkorrekturansicht
  \lohead{\textsc{register}}
\fi

% theindex-Environment neu definieren ohne reledmac
\makeatletter
\renewenvironment{theindex}{%
  \ifkorrekturansicht
    \section*{\indexname}%
  \else
    \subsubsection*{Index der erwähnten Entitäten}%
  \fi
  \setlength{\parindent}{0pt}%
  \setlength{\parskip}{0pt plus 0.3pt}%
  \let\item\@idxitem
}{%
  \ifkorrekturansicht\clearpage\fi
}
\makeatother

\IfFileExists{\jobname-pw.ind}{\input{\jobname-pw.ind}}{}

% Quellenangabe nur in der Leseansicht
\ifkorrekturansicht\else
% Fallback-Definitionen, falls die .tex-Datei \titel etc. nicht gesetzt hat
\providecommand{\titel}{}
\providecommand{\editorInnen}{}
\providecommand{\dateiname}{\jobname}

\vspace{3cm}

\vfill

\footnotesize
\textsc{Quelle}: \titel. Herausgegeben von {\editorInnen}. In: \emph{Arthur Schnitzler: Briefwechsel mit Autorinnen und Autoren}.
 Digitale Edition, https://schnitzler-briefe.acdh.oeaw.ac.at/{\dateiname}.html (Stand \today)
\fi

\end{document}


      