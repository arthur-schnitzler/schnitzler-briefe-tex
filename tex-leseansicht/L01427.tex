%% latex-korrekturansicht-vorspann.tex
%% Vorspann für die Korrekturansicht.
%% Lädt die gemeinsame Datei latex-vorspann.tex mit gesetztem Schalter.

\newif\ifkorrekturansicht
\korrekturansichttrue

\input{../tex-inputs/latex-vorspann}


\section[Arthur Schnitzler an Hermann Bahr, 17. 8. 1904]{L01427 Arthur Schnitzler an Hermann Bahr, 17. 8. 1904}
\nopagebreak\mylabel{L01427v}
\rehead{ }\normalsize\beginnumbering\briefempfaengerindex{Bahr, Hermann@\textsc{Bahr, Hermann}!zzzSchnitzler, Arthur@\emph{von Arthur Schnitzler}!1904-08-171@{17. 8. 1904}|(be}
\toendnotes[C]{\smallbreak\pagebreak[2]}\Standort{TMW, HS AM 23366 Ba.}
\physDesc{Kartenbrief, 287 Zeichen
\newline{}Handschrift: Bleistift, deutsche Kurrent
\newline{}Versand: 1) Stempel: »\nobreak{}\oindex{XVIII., Waehring@\textbf{XVIII., Währing}, \emph{A.ADM3}|pwk}Wien 18, 17. VIII 04\nobreak{}«.   2) Stempel: »\nobreak{}\oindex{XIII., Hietzing@\textbf{XIII., Hietzing}, \emph{A.ADM3}|pwk}Wien 13/7, 18. 8. 04, 8. V, Bestellt\nobreak{}«. }
\buchAbdrucke{\weitereDrucke{1) Arthur Schnitzler: \emph{The Letters of Arthur Schnitzler to Hermann Bahr}. Chapel Hill: \emph{The University of North Carolina Press} 1978, S. 85.} \weitereDrucke{2) Hermann Bahr, Arthur Schnitzler: \emph{Briefwechsel, Aufzeichnungen, Dokumente (1891–1931)}. Göttingen: \emph{Wallstein} 2018, S. 316.} }\toendnotes[C]{\smallbreak}\pstart{}{\pb}Herrn Hermann
                  Bahr\pend{}\pstart{}Wien Ob. St. Veit\oindex{Ober Sankt Veit@\textbf{Ober Sankt Veit}, \emph{P.PPLX}|pw}\pend{}\pstart{}Veitliſſengaſſe.\oindex{Veitlissengasse@\textbf{Veitlissengasse}, \emph{Straße (K.STR)}|pw}\pend{}{\bigskip}\vspace{1em}
\pstart
           \raggedleft{}{\pb}17. 8.\pend
           \vspace{0.5em}
\pstart
           lieber Hermann,{ }wir\pwindex{Schnitzler, Olga 17.01.1882 – 13.01.1970@\textsc{Schnitzler, Olga} (17.01.1882 – 13.01.1970), \emph{Schauspieler/Schauspielerin, Sänger/Sängerin}|pwv} wollen Freitag um 8 in
               dem \label{K_L01427-1v}\edtext{\textsc{Kuffner} Garten\oindex{Ottakringer Braeu@\textbf{Ottakringer Bräu}, \emph{Bierhaus (K.BIR)}|pw}}{\lemma{\textnormal{\emph{Kuffner Garten}}}\Cendnote{\textnormal{Gastgarten des Ottakringer Bräu\oindex{Ottakringer Braeu@\textbf{Ottakringer Bräu}, \emph{Bierhaus (K.BIR)}|pwk}; das Lokal gehörte zu der im 16. Wien\oindex{Wien@\textbf{Wien}, \emph{A.ADM2}|pwk}er Gemeindebezirk (Ottakring\oindex{XVI., Ottakring@\textbf{XVI., Ottakring}, \emph{A.ADM3}|pwk}) angesiedelten Brauerei Kuffner\oindex{Kuffner Bierbrauerei@\textbf{Kuffner Bierbrauerei}, \emph{Bierhaus (K.BIR)}|pwk}.}}}\label{K_L01427-1} in Hietzing\oindex{XIII., Hietzing@\textbf{XIII., Hietzing}, \emph{A.ADM3}|pw} nachtmahlen; – und hoffen ſehr, wenn du nichts anderes vorhaſt,
               dich dort zu treffen.\pend
           
\pstart
           Ziehſt du ein andres Rendezvous vor, so verſtändige mich.\pend
           
\pstart
           Von Herzen{\\[\baselineskip]}dein \spacefill\mbox{Arthur}\pend
           \leftskip=0em{}\selectlanguage{ngerman}\endnumbering\briefempfaengerindex{Bahr, Hermann@\textsc{Bahr, Hermann}!zzzSchnitzler, Arthur@\emph{von Arthur Schnitzler}!1904-08-171@{17. 8. 1904}|)be}\mylabel{L01427h}  \normalsize

\doendnotes{C}
\bigskip
\vfill

\clearpage

\footnotesize

\lohead{\textsc{register}}

% Definiere theindex-Environment komplett neu ohne reledmac
\makeatletter
\renewenvironment{theindex}{%
  \section*{\indexname}%
  \setlength{\parindent}{0pt}%
  \setlength{\parskip}{0pt plus 0.3pt}%
  \let\item\@idxitem
}{%
  \clearpage
}
\makeatother

\IfFileExists{\jobname-pw.ind}{\input{\jobname-pw.ind}}{}

\end{document}

      