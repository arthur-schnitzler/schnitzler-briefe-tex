%% latex-korrekturansicht-vorspann.tex
%% Vorspann für die Korrekturansicht.
%% Lädt die gemeinsame Datei latex-vorspann.tex mit gesetztem Schalter.

\newif\ifkorrekturansicht
\korrekturansichttrue

\input{../tex-inputs/latex-vorspann}


\section[Hugo von Hofmannsthal an Arthur Schnitzler, 4. 12. 1891]{L00046 Hugo von Hofmannsthal an Arthur Schnitzler, 4. 12. 1891}
\nopagebreak\mylabel{L00046v}
\rehead{ }\normalsize\beginnumbering\briefempfaengerindex{Schnitzler, Arthur@\textsc{Schnitzler, Arthur}!zzzHofmannsthal, Hugo von@\emph{von Hugo von Hofmannsthal}!1891-12-041@{4. 12. 1891}|(be}
\toendnotes[C]{\smallbreak\pagebreak[2]}\Standort{CUL, Schnitzler, B 43.}
\physDesc{Postkarte, 180 Zeichen
\newline{}Handschrift: schwarze Tinte, lateinische Kurrent
\newline{}Versand: Stempel: »\nobreak{}Wien, 4/12 91, 10 A\nobreak{}«.  
\newline{}Schnitzler: mit Bleistift jeweils auf der Text- und auf der Anschriftenseite
                                 datiert: »\strikeout{2}4 12 91.«
                               
\newline{}Ordnung: mit Bleistift von unbekannter Hand nummeriert:
                                 »2« }
\buchAbdrucke{\weitereDrucke{1) Hugo von Hofmannsthal, Arthur Schnitzler: \emph{Briefwechsel}. Frankfurt am Main: \emph{S. Fischer} 1964, S. 7.} \weitereDrucke{2) Hermann Bahr, Arthur Schnitzler: \emph{Briefwechsel, Aufzeichnungen, Dokumente (1891–1931)}. Göttingen: \emph{Wallstein} 2018, S. 14.} }\toendnotes[C]{\smallbreak}\pstart{}{\pb}D\textsuperscript{r}
                  Arthur Schnitzler\pend{}\pstart{}Wien\oindex{Wien@\textbf{Wien}, \emph{A.ADM2}|pw}\pend{}\pstart{}I Kärnthnerring 12\oindex{Kaerntnerring 12/Boesendorferstrasse 11@\textbf{Kärntnerring 12/Bösendorferstraße 11}, \emph{Wohngebäude (K.WHS)}|pw}\pend{}{\bigskip}\vspace{1em}
\pstart
           \noindent{}{\pb}Bahr\pwindex{Bahr, Hermann 19.07.1863 – 15.01.1934@\textsc{Bahr, Hermann} (19.07.1863 – 15.01.1934), \emph{Schriftsteller/Schriftstellerin, Kritiker/Kritikerin}|pw} wohnt Heumarkt 9, 3 Stiege, 3. Stock Thür 37\oindex{Am Heumarkt@\textbf{Am Heumarkt}, \emph{Straße (K.STR)}|pw}. Kommt aber, wenn sie ihm nichts
               anderes schreiben, ebenso wie ich \label{K_L00046-1v}\edtext{Sonntag}{\lemma{\textnormal{\emph{Sonntag}}}\Cendnote{\textnormal{Am Sonntag, den 6. 12. 1891 besuchten
                     Schnitzler und Hofmannsthal\pwindex{Hofmannsthal, Hugo von 1874-02-01 – 1929-07-15@\textsc{Hofmannsthal, Hugo von} (1874-02-01 – 1929-07-15), \emph{Schriftsteller/Schriftstellerin}|pwk} am Nachmittag den erkrankten Bahr\pwindex{Bahr, Hermann 19.07.1863 – 15.01.1934@\textsc{Bahr, Hermann} (19.07.1863 – 15.01.1934), \emph{Schriftsteller/Schriftstellerin, Kritiker/Kritikerin}|pwk}.}}}\label{K_L00046-1} um 5 zu ihnen.\pend
           \pstart \spacefill\mbox{Loris.}\pend{}\selectlanguage{ngerman}\endnumbering\briefempfaengerindex{Schnitzler, Arthur@\textsc{Schnitzler, Arthur}!zzzHofmannsthal, Hugo von@\emph{von Hugo von Hofmannsthal}!1891-12-041@{4. 12. 1891}|)be}\mylabel{L00046h}  \normalsize

\doendnotes{C}
\bigskip
\vfill

\clearpage

\footnotesize

\lohead{\textsc{register}}

% Definiere theindex-Environment komplett neu ohne reledmac
\makeatletter
\renewenvironment{theindex}{%
  \section*{\indexname}%
  \setlength{\parindent}{0pt}%
  \setlength{\parskip}{0pt plus 0.3pt}%
  \let\item\@idxitem
}{%
  \clearpage
}
\makeatother

\IfFileExists{\jobname-pw.ind}{\input{\jobname-pw.ind}}{}

\end{document}

      