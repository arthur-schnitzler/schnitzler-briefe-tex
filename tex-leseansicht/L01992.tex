%% latex-korrekturansicht-vorspann.tex
%% Vorspann für die Korrekturansicht.
%% Lädt die gemeinsame Datei latex-vorspann.tex mit gesetztem Schalter.

\newif\ifkorrekturansicht
\korrekturansichttrue

\input{../tex-inputs/latex-vorspann}


\section[Richard Beer-Hofmann an Arthur Schnitzler, {[}22. 12. 1910{]}]{L01992 Richard Beer-Hofmann an Arthur Schnitzler, {[}22. 12. 1910{]}}
\nopagebreak\mylabel{L01992v}
\rehead{ }\normalsize\beginnumbering\briefempfaengerindex{Schnitzler, Arthur@\textsc{Schnitzler, Arthur}!zzzBeer-Hofmann, Richard@\emph{von Richard Beer-Hofmann}!1910-12-221@{{[}22. 12. 1910{]}}|(be}
\toendnotes[C]{\smallbreak\pagebreak[2]}\Standort{CUL, Schnitzler, B 8.}
\physDesc{Kartenbrief, 249 Zeichen
\newline{}Handschrift: blauer Buntstift, lateinische Kurrent
\newline{}Versand: ohne postalischen Übermittlungsvermerk 
\newline{}Schnitzler: mit Bleistift datiert: »22/12 910« 
\newline{}Ordnung: mit Bleistift von unbekannter Hand nummeriert:
                                    »239« }
\buchAbdrucke{\weitereDrucke{Arthur Schnitzler, Richard Beer-Hofmann: \emph{Briefwechsel 1891–1931}. Wien, Zürich: \emph{Europaverlag} 1992, S. 213.} }\toendnotes[C]{\smallbreak}\pstart{}{\pb}Herrn\pend{}\pstart{}Arthur Schnitzler\pend{}{\bigskip}\vspace{1em}
\pstart{}{\pb}Lieber Arthur!\pend\vspace{0.5em}
\pstart
           Berger\pwindex{Berger, Josef 20.01.1848 – 02.09.1932@\textsc{Berger, Josef} (20.01.1848 – 02.09.1932), \emph{Antiquitätenhändler/Antiquitätenhändlerin}|pw} – der \label{K_L01992-1v}\edtext{ehrliche}{\lemma{\textnormal{\emph{ehrliche}}}\Cendnote{\textnormal{gemeint
                  in Abgrenzung zum \emph{Burgtheaterdirektor}\orgindex{Burgtheater@Burgtheater|pwk}{ }Alfred von Berger\pwindex{Berger, Alfred von 30.04.1853 – 24.08.1912@\textsc{Berger, Alfred von} (30.04.1853 – 24.08.1912), \emph{Schriftsteller/Schriftstellerin, Journalist/Journalistin, Theaterleiter/Theaterleiterin}|pwk}}}}\label{K_L01992-1} – fragt teleph. an, ob Sie auf die kleine Roccococo{\geminationm}ode reflectiren, da \strikeout{S}
               sonst eine Dame – die auf Antwort wartet – sie möchte. Bitte um Antwort, u. Trauer
                  \label{T_L01992-1v}\edtext{dass}{\lemma{\textnormal{\emph{dass}}}\Cendnote{\textnormal{geschrieben: das}}}\label{T_L01992-1} Sie uns heute Vorm. verfehlten. Ihr
                  \spacefill\mbox{R}\pend
           \selectlanguage{ngerman}\endnumbering\briefempfaengerindex{Schnitzler, Arthur@\textsc{Schnitzler, Arthur}!zzzBeer-Hofmann, Richard@\emph{von Richard Beer-Hofmann}!1910-12-221@{{[}22. 12. 1910{]}}|)be}\mylabel{L01992h}  \normalsize

\doendnotes{C}
\bigskip
\vfill

\clearpage

\footnotesize

\lohead{\textsc{register}}

% Definiere theindex-Environment komplett neu ohne reledmac
\makeatletter
\renewenvironment{theindex}{%
  \section*{\indexname}%
  \setlength{\parindent}{0pt}%
  \setlength{\parskip}{0pt plus 0.3pt}%
  \let\item\@idxitem
}{%
  \clearpage
}
\makeatother

\IfFileExists{\jobname-pw.ind}{\input{\jobname-pw.ind}}{}

\end{document}

      