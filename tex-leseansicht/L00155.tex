%% latex-leseansicht-vorspann.tex
%% Vorspann für die Leseansicht.
%% Lädt die gemeinsame Datei latex-vorspann.tex mit nicht gesetztem Schalter.

\newif\ifkorrekturansicht
\korrekturansichtfalse

\input{../tex-inputs/latex-vorspann}


         
         \newcommand{\erwaehntePersonen}{Personen: Friedrich Michael Fels, Robert Hirschfeld, Hugo von Hofmannsthal}
         \newcommand{\erwaehnteInstitutionen}{}
         \newcommand{\erwaehnteOrte}{Orte: Café Griensteidl, Café Pfob, Wien}
         \newcommand{\erwaehnteWerke}{Werke: Algernon Charles Swinburne, Anatol, Deutsche Zeitung, Wiener Sonn- und Montagszeitung, »Anatol« von Arthur Schnitzler}
               \section[Arthur Schnitzler an Hugo von Hofmannsthal, 7. 1. 1893]{ Arthur Schnitzler an Hugo von Hofmannsthal, 7. 1. 1893}\nopagebreak\mylabel{v}\rehead{ }\begin{ledgroupsized}[t]{13cm}\normalsize\beginnumbering \toendnotes[C]{\smallbreak\pagebreak[2]} \Standort{FDH, Hs-30885,32.}
\physDesc{Briefkarte
\newline{}Handschrift: schwarze Tinte, deutsche Kurrent\newline{}Ordnung: von Schnitzler mutmaßlich bei der Durchsicht der Korrespondenz
                                    1929 mit Bleistift datiert: »7. 1. 93« }\buchAbdrucke{\weitereDrucke{Hugo von Hofmannsthal, Arthur Schnitzler: \emph{Briefwechsel}. Hg. Therese Nickl und Heinrich Schnitzler. Frankfurt am Main: \emph{S. Fischer} 1964, S. 33.} }\toendnotes[C]{\smallbreak}\pstart{}{\pb}Lieber Hugo,\pend\pstart
           verſpäteten Dank für die liebenswürdige Überſendung der Ballkarten. – Morgen iſt
               nichts bei mir; alſo Dienſtag im \textsc{Pfob}\oindex{Cafe Pfob@\textbf{Café Pfob}|pw} oder we{\geminationn} da nicht, Mittwoch auf dem \label{K_L00155_1v}\edtext{Ball}{\lemma{\textnormal{\emph{Ball}}}\Cendnote{\textnormal{Am 11. 1. 1893 fand der Juristenball statt.}}}\label{K_L00155_1h}.
               Aber da{\geminationn} werden wir gefälligſt wieder vernünftig, –
               entſchuldigen Sie das »wir«.\pend
           \pstart
           »\label{K_L00155_2v}\edtext{\textsc{Swinburne}\pwindex{Algernon Charles Swinburne05. 01. 1893@\emph{Algernon Charles Swinburne} {[}05. 01. 1893{]}|pw}}{\lemma{\textnormal{\emph{Swinburne}}}\Cendnote{\textnormal{Loris\pwindex{Hofmannsthal, Hugo von 1874-02-01 – 1929-07-15@\textsc{Hofmannsthal, Hugo von} (1874-02-01 – 1929-07-15), \emph{Schriftsteller}|pwk}: \emph{Charles
                        Algernon Swinburne}\pwindex{Algernon Charles Swinburne05. 01. 1893@\emph{Algernon Charles Swinburne} {[}05. 01. 1893{]}|pwk}. In: \emph{Deutsche
                        Zeitung}\pwindex{?? Werk@Nicht ermittelte Verfasserinnen und Verfasser!Deutsche Zeitung1871 – 1907@\emph{Deutsche Zeitung} {[}1871 – 1907{]}|pwk}, Nr. 7551, 5. 1. 1893, Morgen-Ausgabe,
                     S. 1–2.}}}\label{K_L00155_2h}« war wunderſchön, eins Ihrer ſchönſten meiner Anſicht
               nach. – \pend
           \pstart
           \textsc{Fels}\pwindex{Fels, Friedrich Michael *~1864@\textsc{Fels, Friedrich Michael} (*~1864), \emph{Journalist}|pw} bereits wohler; von Ihrer Güte wird gelegentlich Gebrauch gemacht werden; ich
               ſprach mit ihm viertgradig über alles. – Waren Sie mit der So{\geminationn}- u {\pb}\textsc{Montagszeitung}\pwindex{?? Werk@Nicht ermittelte Verfasserinnen und Verfasser!Wiener Sonn- und Montagszeitung1863 – 1936@\emph{Wiener Sonn- und Montagszeitung} {[}1863 – 1936{]}|pw}{ }\label{K_L00155_3v}\edtext{zufrieden\pwindex{Anatol« von Arthur Schnitzler1893-01-02@\emph{»Anatol« von Arthur Schnitzler} {[}1893-01-02{]}|pwv}}{\lemma{\textnormal{\emph{zufrieden}}}\Cendnote{\textnormal{l.a.t. [=Robert Hirschfeld]\pwindex{Hirschfeld, Robert 17.09.1857 – 02.04.1914@\textsc{Hirschfeld, Robert} (17.09.1857 – 02.04.1914), \emph{Journalist, Musikkritiker}|pwk}: \emph{»Anatol\pwindex{Schnitzler, Arthur 15.05.1862 – 21.10.1931@\textsc{Schnitzler, Arthur} (15.05.1862 – 21.10.1931), \emph{Schriftsteller, Mediziner}!Anatol1892-10-29@\strich\emph{Anatol} {[}1892-10-29{]}|pwkv}« von Arthur Schnitzler}\pwindex{Anatol« von Arthur Schnitzler1893-01-02@\emph{»Anatol« von Arthur Schnitzler} {[}1893-01-02{]}|pwk}. In: \emph{Wiener Sonn- und Montagszeitung}\pwindex{?? Werk@Nicht ermittelte Verfasserinnen und Verfasser!Wiener Sonn- und Montagszeitung1863 – 1936@\emph{Wiener Sonn- und Montagszeitung} {[}1863 – 1936{]}|pwk}, Jg. 31, Nr. 1,
                        2. 1. 1893, S. 2–3.}}}\label{K_L00155_3h}? – Nicht unmöglich iſt es,
               daß ich morgen So{\geminationn}tag nach etwelchen
               Beſuchen um 7 ins \textsc{Griensteidl}\oindex{Cafe Griensteidl@\textbf{Café Griensteidl}|pw} ko{\geminationm}e. –\pend
           \pstart
           Herzlichſt der Ihre{\\[\baselineskip]}\spacefill\mbox{Arthur.}\pend
           \leftskip=0em{}\pstart
           Samſtag 7. 1. 93.\pend
           
         
         \endnumbering\mylabel{h}\end{ledgroupsized}  \newcommand{\dateiname}{L00155}\newcommand{\titel}{Arthur Schnitzler an Hugo von Hofmannsthal, 7. 1. 1893}\newcommand{\editorInnen}{Martin Anton Müller und Gerd-Hermann Susen}%% latex-leseansicht-abspann.tex
%% Abspann für die Leseansicht.
%% Der Schalter \ifkorrekturansicht ist bereits durch den Vorspann gesetzt.

%% latex-abspann.tex
%% Gemeinsamer Abspann für Korrekturansicht und Leseansicht.
%% Setzt den Schalter \ifkorrekturansicht voraus (gesetzt in den
%% einbindenden Dateien latex-korrekturansicht-abspann.tex bzw.
%% latex-leseansicht-abspann.tex).
%% ---------------------------------------------------------------

\normalsize

% Das esempio-Environment wird nur in der Leseansicht benötigt
\ifkorrekturansicht\else
\newenvironment{esempio}[3]%
{
    \vspace{1.5ex}
    \rlap{\underline{#1}}
    \par
    \setlength{\parindent}{0cm}
    \nopagebreak
    \leftskip=#2cm
    \rightskip=#3cm
}
{
    \par
}
\fi

\doendnotes{C}
\bigskip
\vfill

\clearpage

\footnotesize

\ifkorrekturansicht
  \lohead{\textsc{register}}
\fi

% theindex-Environment neu definieren ohne reledmac
\makeatletter
\renewenvironment{theindex}{%
  \ifkorrekturansicht
    \section*{\indexname}%
  \else
    \subsubsection*{Index der erwähnten Entitäten}%
  \fi
  \setlength{\parindent}{0pt}%
  \setlength{\parskip}{0pt plus 0.3pt}%
  \let\item\@idxitem
}{%
  \ifkorrekturansicht\clearpage\fi
}
\makeatother

\IfFileExists{\jobname-pw.ind}{\input{\jobname-pw.ind}}{}

% Quellenangabe nur in der Leseansicht
\ifkorrekturansicht\else
% Fallback-Definitionen, falls die .tex-Datei \titel etc. nicht gesetzt hat
\providecommand{\titel}{}
\providecommand{\editorInnen}{}
\providecommand{\dateiname}{\jobname}

\vspace{3cm}

\vfill

\footnotesize
\textsc{Quelle}: \titel. Herausgegeben von {\editorInnen}. In: \emph{Arthur Schnitzler: Briefwechsel mit Autorinnen und Autoren}.
 Digitale Edition, https://schnitzler-briefe.acdh.oeaw.ac.at/{\dateiname}.html (Stand \today)
\fi

\end{document}


      