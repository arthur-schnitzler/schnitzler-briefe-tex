%% latex-leseansicht-vorspann.tex
%% Vorspann für die Leseansicht.
%% Lädt die gemeinsame Datei latex-vorspann.tex mit nicht gesetztem Schalter.

\newif\ifkorrekturansicht
\korrekturansichtfalse

\input{../tex-inputs/latex-vorspann}


\section[Arthur Schnitzler an Hugo von Hofmannsthal, 7. 1. 1893]{L00155 Arthur Schnitzler an Hugo von Hofmannsthal, 7. 1. 1893}
\nopagebreak\mylabel{L00155v}
\rehead{ }\normalsize\beginnumbering\briefempfaengerindex{Hofmannsthal, Hugo von@\textsc{Hofmannsthal, Hugo von}!zzzSchnitzler, Arthur@\emph{von Arthur Schnitzler}!1893-01-071@{7. 1. 1893}|(be}
\toendnotes[C]{\smallbreak\pagebreak[2]}
\correspDesc{Versand  durch Arthur Schnitzler am 7. 1. 1893 in Wien
\newline{}Erhalt  durch Hugo von Hofmannsthal im Zeitraum [7. 1. 1893 – 11. 1. 1893?] in Wien}\toendnotes[C]{\smallbreak}
\Standort{FDH, Hs-30885,32.}
\physDesc{Briefkarte, 627 Zeichen
\newline{}Handschrift: schwarze Tinte, deutsche Kurrent
\newline{}Ordnung: mit Bleistift von Schnitzler mutmaßlich bei der Durchsicht der
                                            Korrespondenz 1929 datiert: »7. 1. 93« }
\buchAbdrucke{\weitereDrucke{Hugo von Hofmannsthal, Arthur Schnitzler: \emph{Briefwechsel}. Herausgegeben von Therese Nickl und Heinrich Schnitzler. Frankfurt am Main: \emph{S. Fischer} 1964, S. 33.} }\toendnotes[C]{\smallbreak}
\pstart{}{\pb}Lieber Hugo,\pend\vspace{0.5em}
\pstart
           verſpäteten Dank für die liebenswürdige Überſendung der \label{K_L00155-1v}\edtext{Ballkarten\eventindex{Wien@\textbf{Wien}!Hausball Stein, 7.1.1893@Hausball Stein, 7.1.1893|pwv}}{\lemma{\textnormal{\emph{Ballkarten}}}\Cendnote{\textnormal{Vgl. A. S.: \emph{Kulturveranstaltungen}, 7. 1. 1893.}}}\label{K_L00155-1}. –
                    Morgen iſt nichts bei mir; alſo \label{K_L00155-2v}\edtext{Dienſtag im \textsc{Pfob}\oindex{Wien@\textbf{Wien}!I., Innere Stadt@\textbf{I., Innere Stadt}!Café Pfob@\textbf{Café Pfob}, \emph{Kaffeehaus}|pw}}{\lemma{\textnormal{\emph{Dienstag im Pfob}}}\Cendnote{\textnormal{Vgl. A. S.: \emph{Tagebuch}, 10. 1. 1893. Schnitzler erwähnt keine Anwesenheit
                            Hofmannsthals\pwindex{Hofmannsthal, Hugo von 1.\,2.\,1874 Wien – 15.\,7.\,1929 Rodaun@\textsc{Hofmannsthal, Hugo von} (1.\,2.\,1874 Wien – 15.\,7.\,1929 Rodaun), \emph{Schriftsteller}|pwk}.}}}\label{K_L00155-2} oder we{\geminationn} da nicht, Mittwoch auf dem \label{K_L00155-3v}\edtext{Ball\eventindex{Sofiensäle@\textbf{Sofiensäle}!Juristenball, 11.1.1893@Juristenball, 11.1.1893|pwv}}{\lemma{\textnormal{\emph{Ball}}}\Cendnote{\textnormal{Am
                            11. 1. 1893 fand der Juristenball\eventindex{Sofiensäle@\textbf{Sofiensäle}!Juristenball, 11.1.1893@Juristenball, 11.1.1893|pwk} statt, vgl. A. S.: \emph{Kulturveranstaltungen}, 11. 1. 1893.}}}\label{K_L00155-3}. Aber da{\geminationn} werden wir gefälligſt wieder vernünftig, –
                    entſchuldigen Sie das »wir«.\pend
           
\pstart
           »\label{K_L00155-4v}\edtext{\textsc{Swinburne}\pwindex{Hofmannsthal, Hugo von 1.\,2.\,1874 Wien – 15.\,7.\,1929 Rodaun@\textsc{Hofmannsthal, Hugo von} (1.\,2.\,1874 Wien – 15.\,7.\,1929 Rodaun), \emph{Schriftsteller}!Algernon Charles Swinburne@\strich\emph{Algernon Charles Swinburne}|pw}}{\lemma{\textnormal{\emph{Swinburne}}}\Cendnote{\textnormal{Loris\pwindex{Hofmannsthal, Hugo von 1.\,2.\,1874 Wien – 15.\,7.\,1929 Rodaun@\textsc{Hofmannsthal, Hugo von} (1.\,2.\,1874 Wien – 15.\,7.\,1929 Rodaun), \emph{Schriftsteller}|pwk}: \emph{Charles Algernon Swinburne}\pwindex{Hofmannsthal, Hugo von 1.\,2.\,1874 Wien – 15.\,7.\,1929 Rodaun@\textsc{Hofmannsthal, Hugo von} (1.\,2.\,1874 Wien – 15.\,7.\,1929 Rodaun), \emph{Schriftsteller}!Algernon Charles Swinburne@\strich\emph{Algernon Charles Swinburne}|pwk}. In: \emph{Deutsche Zeitung}\pwindex{Deutsche Zeitung@\emph{Deutsche Zeitung}|pwk}, Nr. 7551,
                            5. 1. 1893, Morgen-Ausgabe, S. 1–2.}}}\label{K_L00155-4}« war
                    wunderſchön, eins Ihrer{ }ſchönſten meiner Anſicht nach. –\pend
           
\pstart
           \textsc{Fels}\pwindex{Fels, Friedrich Michael *~1864 Bad Dürkheim@\textsc{Fels, Friedrich Michael} (*~1864 Bad Dürkheim), \emph{Journalist}|pw} bereits wohler; von Ihrer Güte wird gelegentlich Gebrauch gemacht werden;
                    ich{ }ſprach mit ihm viertgradig über alles. – Waren Sie mit der \textsc{So{\geminationn}}- u {\pb}\textsc{Montagszeitung}\pwindex{Wiener Sonn- und Montagszeitung@\emph{Wiener Sonn- und Montagszeitung}|pw}{ }\label{K_L00155-5v}\edtext{zufrieden\pwindex{Hirschfeld, Robert 17.\,9.\,1857 Žďár nad Sázavou – 2.\,4.\,1914 Salzburg@\textsc{Hirschfeld, Robert} (17.\,9.\,1857 Žďár nad Sázavou – 2.\,4.\,1914 Salzburg), \emph{Journalist, Musikkritiker}!Anatol« von Arthur Schnitzler@\strich\emph{»Anatol« von Arthur Schnitzler}|pwv}}{\lemma{\textnormal{\emph{zufrieden}}}\Cendnote{\textnormal{l.a.t. [ = Robert Hirschfeld]\pwindex{Hirschfeld, Robert 17.\,9.\,1857 Žďár nad Sázavou – 2.\,4.\,1914 Salzburg@\textsc{Hirschfeld, Robert} (17.\,9.\,1857 Žďár nad Sázavou – 2.\,4.\,1914 Salzburg), \emph{Journalist, Musikkritiker}|pwk}:
                                \emph{»Anatol« von Arthur Schnitzler}\pwindex{Hirschfeld, Robert 17.\,9.\,1857 Žďár nad Sázavou – 2.\,4.\,1914 Salzburg@\textsc{Hirschfeld, Robert} (17.\,9.\,1857 Žďár nad Sázavou – 2.\,4.\,1914 Salzburg), \emph{Journalist, Musikkritiker}!Anatol« von Arthur Schnitzler@\strich\emph{»Anatol« von Arthur Schnitzler}|pwk}. In:
                                \emph{Wiener Sonn- und
                                Montags-Zeitung}\pwindex{Wiener Sonn- und Montagszeitung@\emph{Wiener Sonn- und Montagszeitung}|pwk}, Jg. 31, Nr. 1, 2. 1. 1893,
                            S. 2–3.}}}\label{K_L00155-5}? – Nicht unmöglich iſt es, daß ich morgen
                            So{\geminationn}tag nach etwelchen Beſuchen
                        um 7 ins \textsc{Griensteidl}\oindex{Wien@\textbf{Wien}!I., Innere Stadt@\textbf{I., Innere Stadt}!Café Griensteidl@\textbf{Café Griensteidl}, \emph{Kaffeehaus}|pw} ko{\geminationm}e. –\pend
           
\pstart
           Herzlichſt der Ihre{\\[\baselineskip]}\spacefill\mbox{Arthur.}\pend
           \leftskip=0em{}
\pstart
           Samſtag 7. 1. 93.\pend
           \selectlanguage{ngerman}\endnumbering\briefempfaengerindex{Hofmannsthal, Hugo von@\textsc{Hofmannsthal, Hugo von}!zzzSchnitzler, Arthur@\emph{von Arthur Schnitzler}!1893-01-071@{7. 1. 1893}|)be}\mylabel{L00155h}  \newcommand{\dateiname}{L00155}\newcommand{\titel}{Arthur Schnitzler an Hugo von Hofmannsthal, 7. 1. 1893}\newcommand{\editorInnen}{Martin Anton Müller und Gerd-Hermann Susen}%% latex-leseansicht-abspann.tex
%% Abspann für die Leseansicht.
%% Der Schalter \ifkorrekturansicht ist bereits durch den Vorspann gesetzt.

%% latex-abspann.tex
%% Gemeinsamer Abspann für Korrekturansicht und Leseansicht.
%% Setzt den Schalter \ifkorrekturansicht voraus (gesetzt in den
%% einbindenden Dateien latex-korrekturansicht-abspann.tex bzw.
%% latex-leseansicht-abspann.tex).
%% ---------------------------------------------------------------

\normalsize

% Das esempio-Environment wird nur in der Leseansicht benötigt
\ifkorrekturansicht\else
\newenvironment{esempio}[3]%
{
    \vspace{1.5ex}
    \rlap{\underline{#1}}
    \par
    \setlength{\parindent}{0cm}
    \nopagebreak
    \leftskip=#2cm
    \rightskip=#3cm
}
{
    \par
}
\fi

\doendnotes{C}
\bigskip
\vfill

\clearpage

\footnotesize

\ifkorrekturansicht
  \lohead{\textsc{register}}
\fi

% theindex-Environment neu definieren ohne reledmac
\makeatletter
\renewenvironment{theindex}{%
  \ifkorrekturansicht
    \section*{\indexname}%
  \else
    \subsubsection*{Index der erwähnten Entitäten}%
  \fi
  \setlength{\parindent}{0pt}%
  \setlength{\parskip}{0pt plus 0.3pt}%
  \let\item\@idxitem
}{%
  \ifkorrekturansicht\clearpage\fi
}
\makeatother

\IfFileExists{\jobname-pw.ind}{\input{\jobname-pw.ind}}{}

% Quellenangabe nur in der Leseansicht
\ifkorrekturansicht\else
% Fallback-Definitionen, falls die .tex-Datei \titel etc. nicht gesetzt hat
\providecommand{\titel}{}
\providecommand{\editorInnen}{}
\providecommand{\dateiname}{\jobname}

\vspace{3cm}

\vfill

\footnotesize
\textsc{Quelle}: \titel. Herausgegeben von {\editorInnen}. In: \emph{Arthur Schnitzler: Briefwechsel mit Autorinnen und Autoren}.
 Digitale Edition, https://schnitzler-briefe.acdh.oeaw.ac.at/{\dateiname}.html (Stand \today)
\fi

\end{document}


