%% latex-korrekturansicht-vorspann.tex
%% Vorspann für die Korrekturansicht.
%% Lädt die gemeinsame Datei latex-vorspann.tex mit gesetztem Schalter.

\newif\ifkorrekturansicht
\korrekturansichttrue

\input{../tex-inputs/latex-vorspann}


\section[Arthur Schnitzler an Hugo von Hofmannsthal, 7. 1. 1893]{L00155 Arthur Schnitzler an Hugo von Hofmannsthal, 7. 1. 1893}
\nopagebreak\mylabel{L00155v}
\rehead{ }\normalsize\beginnumbering\briefempfaengerindex{Hofmannsthal, Hugo von@\textsc{Hofmannsthal, Hugo von}!zzzSchnitzler, Arthur@\emph{von Arthur Schnitzler}!1893-01-071@{7. 1. 1893}|(be}
\toendnotes[C]{\smallbreak\pagebreak[2]}\Standort{FDH, Hs-30885,32.}
\physDesc{Briefkarte, 628 Zeichen
\newline{}Handschrift: schwarze Tinte, deutsche Kurrent
\newline{}Ordnung: mit Bleistift von Schnitzler mutmaßlich bei der Durchsicht der Korrespondenz
                                    1929 datiert: »7. 1. 93« }
\buchAbdrucke{\weitereDrucke{Hugo von Hofmannsthal, Arthur Schnitzler: \emph{Briefwechsel}. Frankfurt am Main: \emph{S. Fischer} 1964, S. 33.} }\toendnotes[C]{\smallbreak}
\pstart{}{\pb}Lieber Hugo,\pend\vspace{0.5em}
\pstart
           verſpäteten Dank für die liebenswürdige Überſendung der Ballkarten. – Morgen iſt
               nichts bei mir; alſo Dienſtag im \textsc{Pfob}\oindex{Cafe Pfob@\textbf{Café Pfob}, \emph{Kaffeehaus (K.KAF)}|pw} oder we{\geminationn} da nicht, Mittwoch auf dem \label{K_L00155-1v}\edtext{Ball}{\lemma{\textnormal{\emph{Ball}}}\Cendnote{\textnormal{Am 11. 1. 1893 fand der Juristenball statt.}}}\label{K_L00155-1}.
               Aber da{\geminationn} werden wir gefälligſt wieder vernünftig, –
               entſchuldigen Sie das »wir«.\pend
           
\pstart
           »\label{K_L00155-2v}\edtext{\textsc{Swinburne}\pwindex{Algernon Charles Swinburne@\emph{Algernon Charles Swinburne}|pw}}{\lemma{\textnormal{\emph{Swinburne}}}\Cendnote{\textnormal{Loris\pwindex{Hofmannsthal, Hugo von 1874-02-01 – 1929-07-15@\textsc{Hofmannsthal, Hugo von} (1874-02-01 – 1929-07-15), \emph{Schriftsteller/Schriftstellerin}|pwk}: \emph{Charles Algernon Swinburne}\pwindex{Algernon Charles Swinburne@\emph{Algernon Charles Swinburne}|pwk}. In: \emph{Deutsche Zeitung}\pwindex{Deutsche Zeitung@\emph{Deutsche Zeitung}|pwk}, Nr. 7551, 5. 1. 1893, Morgen-Ausgabe,
                     S. 1–2.}}}\label{K_L00155-2}« war wunderſchön, eins Ihrer{ }ſchönſten meiner Anſicht
               nach. – \pend
           
\pstart
           \textsc{Fels}\pwindex{Fels, Friedrich Michael *~1864@\textsc{Fels, Friedrich Michael} (*~1864), \emph{Journalist/Journalistin}|pw} bereits wohler; von Ihrer Güte wird gelegentlich Gebrauch gemacht werden; ich{ }ſprach mit ihm viertgradig über alles. – Waren Sie mit der So{\geminationn}- u {\pb}\textsc{Montagszeitung}\pwindex{Wiener Sonn- und Montagszeitung@\emph{Wiener Sonn- und Montagszeitung}|pw}{ }\label{K_L00155-3v}\edtext{zufrieden\pwindex{Anatol« von Arthur Schnitzler@\emph{»Anatol« von Arthur Schnitzler}|pwv}}{\lemma{\textnormal{\emph{zufrieden}}}\Cendnote{\textnormal{l.a.t. [ = Robert Hirschfeld]\pwindex{Hirschfeld, Robert 17.09.1857 – 02.04.1914@\textsc{Hirschfeld, Robert} (17.09.1857 – 02.04.1914), \emph{Journalist/Journalistin, Musikkritiker/Musikkritikerin}|pwk}: \emph{»Anatol\pwindex{Anatol@\emph{Anatol}|pwkv}« von Arthur Schnitzler}\pwindex{Anatol« von Arthur Schnitzler@\emph{»Anatol« von Arthur Schnitzler}|pwk}. In: \emph{Wiener Sonn- und Montags-Zeitung}\pwindex{Wiener Sonn- und Montagszeitung@\emph{Wiener Sonn- und Montagszeitung}|pwk}, Jg. 31,
                     Nr. 1, 2. 1. 1893, S. 2–3.}}}\label{K_L00155-3}? – Nicht unmöglich iſt
               es, daß ich morgen So{\geminationn}tag nach etwelchen
               Beſuchen um 7 ins \textsc{Griensteidl}\oindex{Cafe Griensteidl@\textbf{Café Griensteidl}, \emph{Kaffeehaus (K.KAF)}|pw} ko{\geminationm}e. –\pend
           
\pstart
           Herzlichſt der Ihre{\\[\baselineskip]}\spacefill\mbox{Arthur.}\pend
           \leftskip=0em{}
\pstart
           Samſtag 7. 1. 93.\pend
           \selectlanguage{ngerman}\endnumbering\briefempfaengerindex{Hofmannsthal, Hugo von@\textsc{Hofmannsthal, Hugo von}!zzzSchnitzler, Arthur@\emph{von Arthur Schnitzler}!1893-01-071@{7. 1. 1893}|)be}\mylabel{L00155h}  \normalsize

\doendnotes{C}
\bigskip
\vfill

\clearpage

\footnotesize

\lohead{\textsc{register}}

% Definiere theindex-Environment komplett neu ohne reledmac
\makeatletter
\renewenvironment{theindex}{%
  \section*{\indexname}%
  \setlength{\parindent}{0pt}%
  \setlength{\parskip}{0pt plus 0.3pt}%
  \let\item\@idxitem
}{%
  \clearpage
}
\makeatother

\IfFileExists{\jobname-pw.ind}{\input{\jobname-pw.ind}}{}

\end{document}

      