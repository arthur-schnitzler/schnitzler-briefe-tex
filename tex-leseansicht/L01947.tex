%% latex-leseansicht-vorspann.tex
%% Vorspann für die Leseansicht.
%% Lädt die gemeinsame Datei latex-vorspann.tex mit nicht gesetztem Schalter.

\newif\ifkorrekturansicht
\korrekturansichtfalse

\input{../tex-inputs/latex-vorspann}


\section[Hugo von Hofmannsthal an Arthur Schnitzler, 13. 7. {[}1910{]}]{L01947 Hugo von Hofmannsthal an Arthur Schnitzler, 13. 7. [1910]}
\nopagebreak\mylabel{L01947v}
\rehead{ }\normalsize\beginnumbering\briefempfaengerindex{Schnitzler, Arthur@\textsc{Schnitzler, Arthur}!zzzHofmannsthal, Hugo von@\emph{von Hugo von Hofmannsthal}!1910-07-131@{13. 7. [1910]}|(be}
\toendnotes[C]{\smallbreak\pagebreak[2]}
\correspDesc{Versand  durch Hugo von Hofmannsthal am 13. 7. [1910] in Rodaun
\newline{}Erhalt  durch Arthur Schnitzler im Zeitraum [14. 7. 1910
                  – 18. 7. 1910?] in Wien}\toendnotes[C]{\smallbreak}
\Standort{CUL, Schnitzler, B 43.}
\physDesc{Brief, 2 Blätter, 8 Seiten, 2213 Zeichen (das zweite Blatt mit »2«
                                 gekennzeichnet)
\newline{}Handschrift: schwarze Tinte, deutsche Kurrent
\newline{}Schnitzler: mit Bleistift die Jahreszahl ergänzt: »910« und beschriftet: »\textsc{Hugo}« 
\newline{}Ordnung: 1) mit Bleistift von unbekannter Hand nummeriert: »\strikeout{313}«  2) mit Bleistift von unbekannter Hand nummeriert:
                                    »320«}
\buchAbdrucke{\weitereDrucke{Hugo von Hofmannsthal, Arthur Schnitzler: \emph{Briefwechsel}. Herausgegeben von Therese Nickl und Heinrich Schnitzler. Frankfurt am Main: \emph{S. Fischer} 1964, S. 250.} }\toendnotes[C]{\smallbreak}
\pstart
           \raggedleft{}{\pb}Rodaun\oindex{Wien@\textbf{Wien}!XXIII., Liesing@\textbf{XXIII., Liesing}!Rodaun@\textbf{Rodaun}, \emph{Region}|pw}{ }13 Juli.\pend
           
\pstart{}Mein lieber Arthur,\pend\vspace{0.5em}
\pstart
           neulich hatte ich einmal den Gedanken: man wohnt doch in der{ }ſelben Stadt –{ }ſo kann
               man doch \uline{ein Mal}, wenn man{ }ſich wünſcht, den andern
               zu{ }ſehen, auch Glück haben, ohne erſt einen Brief zu{ }ſchreiben oder ein Telegramm zu{ }ſchicken – und als ich dann bei Euch die Treppe heruntergehen {\pb}muſste, war ich unverhältnismäßig
               traurig. Freilich das einzelne iſt ja immer ein Zufall oder ein unbeträchtliches
               Detail, aber das Ganze macht mich wachſend traurig, ich kann mir nicht helfen. Man
               iſt{ }ſeit 20 Jahren gut miteinander, man iſt{ }ſich weder fremder, noch unintereſſanter,
               noch weniger lieb geworden, {\pb}ſondern im Gegentheil vielleicht, man gehört demſelben Berufe an, man wohnt in \uline{einer} Stadt – und man verbringt keine 20 Stunden im
               Jahr miteinander! Mir geht es furchtbar ab – Euch, Ihnen und Richard\pwindex{Beer-Hofmann, Richard 11.\,7.\,1866 Wien – 26.\,9.\,1945 New York City@\textsc{Beer-Hofmann, Richard} (11.\,7.\,1866 Wien – 26.\,9.\,1945 New York City), \emph{Schriftsteller}|pw} offenbar viel weniger, das iſt ja Temperamentsſache. Am
                  \label{K_L01947-1v}\edtext{Lido\oindex{Lido@\textbf{Lido}|pw}}{\lemma{\textnormal{\emph{Lido}}}\Cendnote{\textnormal{Sie hielten sich von
                     12. 6. 1910 bis zum Ende des Monats im Grand Hotel Excelsior\oindex{Grand Hotel Excelsior [Lido]@\textbf{Grand Hotel Excelsior [Lido]}, \emph{Hotel}|pwk} auf.}}}\label{K_L01947-1}{ }{\pb}hatte ich oft daran gedacht,
               hatte{ }ſo{ }ſicher gehofft, in dieſen drei Wochen Juli würde man{ }ſich mehr
               als einmal{ }ſehen, – es{ }ſind Jahre her, daſs Sie nicht in meinem Haus\oindex{Wien@\textbf{Wien}!XXIII., Liesing@\textbf{XXIII., Liesing}!Hofmannsthal-Schlössl@\textbf{Hofmannsthal-Schlössl}, \emph{Schloss}|pwv} waren! – und nun ko{\geminationm}t es \uline{ſo}. In dieſer
               Woche, wo wir noch hier{ }ſind, \strikeout{trafen} überſiedelt Ihr,
               zu \label{K_L01947-2v}\edtext{Anfang der nächſten Woche}{\lemma{\textnormal{\emph{Anfang … Woche}}}\Cendnote{\textnormal{Am 21. 7. 1910 reisten sie
                  ab.}}}\label{K_L01947-2} fahren wir mit {\pb}den
                  Friedmanns\pwindex{Friedmann, Rose 12.\,2.\,1864 – 14.\,1.\,1919 Baden bei Wien@\textsc{Friedmann, Rose} (12.\,2.\,1864 – 14.\,1.\,1919 Baden bei Wien)|pw}\pwindex{Friedmann, Louis Philipp 29.\,6.\,1861 Paris – 1.\,4.\,1939 Wien@\textsc{Friedmann, Louis Philipp} (29.\,6.\,1861 Paris – 1.\,4.\,1939 Wien), \emph{Industrieller, Bergsteiger}|pw} fort, über München\oindex{München@\textbf{München}|pw} an den Bodenſee\oindex{Bodensee@\textbf{Bodensee}, \emph{See}|pw} (eine Landschaft die ich nicht kenne und mir lange wünſche) dann
               über den Arlberg\oindex{Arlberg@\textbf{Arlberg}, \emph{Berg}|pw} nach Tirol\oindex{Tirol@\textbf{Tirol}, \emph{Land}|pw} hinein und{ }ſind ungefähr die erſten 10 Tage des
                  Auguſt in \textsc{Canazei}\oindex{Canazei@\textbf{Canazei}, \emph{Hauptstadt}|pw}. Dann{ }ſind wir für viele Wochen {\pb}in Auſſee\oindex{Bad Aussee@\textbf{Bad Aussee}, \emph{Hauptstadt}|pw}. Ko{\geminationm}t doch im September ein
               biſſl dorthin, da iſt gewöhnlich eine{ }ſo{ }ſchöne Zeit.\pend
           
\pstart
           Wenn Ihr jemals wieder nach Tirol\oindex{Tirol@\textbf{Tirol}, \emph{Land}|pw} geht, will ich
               alles tun, um für eine Zeit an den gleichen Ort zu ko{\geminationm}en; ich habe {\pb}eine{ }ſo{ }ſchöne
               liebe Erinnerung an die Tage in Welsberg\oindex{Welsberg-Taisten@\textbf{Welsberg-Taisten}, \emph{Verwaltungsgebiet}|pw} – das
               iſt aber auch schon wieder \label{K_L01947-3v}\edtext{4 Jahre}{\lemma{\textnormal{\emph{4 Jahre}}}\Cendnote{\textnormal{Wie Schnitzler in seiner Antwort (siehe XXXX Auszeichnungsfehler: Dokument L01952 nicht gefunden) bemerkt, nur drei, im Juli 1907.}}}\label{K_L01947-3} her.\pend
           
\pstart
           Vielen Dank für Ihre{ }ſo lieben Zeilen nach der Cristina\pwindex{Hofmannsthal, Hugo von 1.\,2.\,1874 Wien – 15.\,7.\,1929 Rodaun@\textsc{Hofmannsthal, Hugo von} (1.\,2.\,1874 Wien – 15.\,7.\,1929 Rodaun), \emph{Schriftsteller}!Cristinas Heimreise. Komödie@\strich\emph{Cristinas Heimreise. Komödie}|pw}.\pend
           
\pstart
           Ein Wort über eine Arbeit von Ihnen (auch die Einſchränkungen, die ich mir ganz zu
               eigen machen kann) das iſt{ }ſo {\pb}ganz dasſelbe was es vor 18 Jahren war, und ganz etwas anderes, als was von
               fremderem Mund ko{\geminationm}t.\pend
           
\pstart
           Wie gerne hätte ich wieder ein neues Buch von Ihnen in der Hand. Wie gerne möchte ich
               Ihnen meine Spieloper\pwindex{Hofmannsthal, Hugo von 1.\,2.\,1874 Wien – 15.\,7.\,1929 Rodaun@\textsc{Hofmannsthal, Hugo von} (1.\,2.\,1874 Wien – 15.\,7.\,1929 Rodaun), \emph{Schriftsteller}!Rosenkavalier. Komödie für Musik@\strich\emph{Der Rosenkavalier. Komödie für Musik}|pwv}
                  vorleſen.\hspace*{1.5em}Schicken Sie mir ein paar Zeilen nach
                  \textsc{Canazei, Südtirol}\oindex{Canazei@\textbf{Canazei}, \emph{Hauptstadt}|pw}, dann{ }ſpäter.\pend
           
\pstart
           Von Herzen Ihr{\\[\baselineskip]}\spacefill\mbox{Hugo.}\pend
           \leftskip=0em{}
\pstart
           \noindent{}\label{T_L01947-1v}\edtext{Alles Gute Olga\pwindex{Schnitzler, Olga 17.\,1.\,1882 Wien – 13.\,1.\,1970 Lugano@\textsc{Schnitzler, Olga} (17.\,1.\,1882 Wien – 13.\,1.\,1970 Lugano), \emph{Schauspielerin, Sängerin}|pw} und den Kleinen\pwindex{Schnitzler, Heinrich 9.\,8.\,1902 Hinterbrühl – 12.\,7.\,1982 Wien@\textsc{Schnitzler, Heinrich} (9.\,8.\,1902 Hinterbrühl – 12.\,7.\,1982 Wien), \emph{Regisseur, Schauspieler}|pwv}\pwindex{Schnitzler, Lilly 3.\,7.\,1911 Wien – 17.\,5.\,2009 ebd.@\textsc{Schnitzler, Lilly} (3.\,7.\,1911 Wien – 17.\,5.\,2009 ebd.), \emph{Violinistin}|pwv} von uns beiden.}{\lemma{\textnormal{\emph{Alles … beiden.}}}\Cendnote{\textnormal{quer am linken Rand der letzten Seite}}}\label{T_L01947-1}\pend
           \selectlanguage{ngerman}\endnumbering\briefempfaengerindex{Schnitzler, Arthur@\textsc{Schnitzler, Arthur}!zzzHofmannsthal, Hugo von@\emph{von Hugo von Hofmannsthal}!1910-07-131@{13. 7. [1910]}|)be}\mylabel{L01947h}  \newcommand{\dateiname}{L01947}\newcommand{\titel}{Hugo von Hofmannsthal an Arthur Schnitzler, 13. 7. [1910]}\newcommand{\editorInnen}{Martin Anton Müller und Gerd-Hermann Susen}%% latex-leseansicht-abspann.tex
%% Abspann für die Leseansicht.
%% Der Schalter \ifkorrekturansicht ist bereits durch den Vorspann gesetzt.

%% latex-abspann.tex
%% Gemeinsamer Abspann für Korrekturansicht und Leseansicht.
%% Setzt den Schalter \ifkorrekturansicht voraus (gesetzt in den
%% einbindenden Dateien latex-korrekturansicht-abspann.tex bzw.
%% latex-leseansicht-abspann.tex).
%% ---------------------------------------------------------------

\normalsize

% Das esempio-Environment wird nur in der Leseansicht benötigt
\ifkorrekturansicht\else
\newenvironment{esempio}[3]%
{
    \vspace{1.5ex}
    \rlap{\underline{#1}}
    \par
    \setlength{\parindent}{0cm}
    \nopagebreak
    \leftskip=#2cm
    \rightskip=#3cm
}
{
    \par
}
\fi

\doendnotes{C}
\bigskip
\vfill

\clearpage

\footnotesize

\ifkorrekturansicht
  \lohead{\textsc{register}}
\fi

% theindex-Environment neu definieren ohne reledmac
\makeatletter
\renewenvironment{theindex}{%
  \ifkorrekturansicht
    \section*{\indexname}%
  \else
    \subsubsection*{Index der erwähnten Entitäten}%
  \fi
  \setlength{\parindent}{0pt}%
  \setlength{\parskip}{0pt plus 0.3pt}%
  \let\item\@idxitem
}{%
  \ifkorrekturansicht\clearpage\fi
}
\makeatother

\IfFileExists{\jobname-pw.ind}{\input{\jobname-pw.ind}}{}

% Quellenangabe nur in der Leseansicht
\ifkorrekturansicht\else
% Fallback-Definitionen, falls die .tex-Datei \titel etc. nicht gesetzt hat
\providecommand{\titel}{}
\providecommand{\editorInnen}{}
\providecommand{\dateiname}{\jobname}

\vspace{3cm}

\vfill

\footnotesize
\textsc{Quelle}: \titel. Herausgegeben von {\editorInnen}. In: \emph{Arthur Schnitzler: Briefwechsel mit Autorinnen und Autoren}.
 Digitale Edition, https://schnitzler-briefe.acdh.oeaw.ac.at/{\dateiname}.html (Stand \today)
\fi

\end{document}


