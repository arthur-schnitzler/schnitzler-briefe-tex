%% latex-leseansicht-vorspann.tex
%% Vorspann für die Leseansicht.
%% Lädt die gemeinsame Datei latex-vorspann.tex mit nicht gesetztem Schalter.

\newif\ifkorrekturansicht
\korrekturansichtfalse

\input{../tex-inputs/latex-vorspann}


\section[Arthur Schnitzler an Paul Goldmann, nicht abgesandt, 28. 1. 1907]{L03520 Arthur Schnitzler an Paul Goldmann, nicht abgesandt, 28. 1. 1907}
\nopagebreak\mylabel{L03520v}
\rehead{ }\normalsize\beginnumbering\briefempfaengerindex{Goldmann, Paul@\textsc{Goldmann, Paul}!zzzSchnitzler, Arthur@\emph{von Arthur Schnitzler}!1907-01-281@{28. 1. 1907}|(be}
\toendnotes[C]{\smallbreak\pagebreak[2]}
\correspDesc{Versand  durch Arthur Schnitzler am 28. 1. 1907 in Wien
\newline{}Erhalt  durch Paul Goldmann am [nicht erhalten] \textbf{Ort fehlend} }\toendnotes[C]{\smallbreak}
\Standort{CUL, Schnitzler, A 20,6.}
\physDesc{Briefentwurf, Durchschlag, 5 Blätter, 5 Seiten, 4162 Zeichen
\newline{}Schreibmaschine
\newline{}Handschrift Arthur Schnitzler: 1) Bleistift (\noindent{}eine Unterstreichung)\hspace{1em}2) Bleistift, lateinische Kurrent (\noindent{}Vermerk »(An Goldmann\pwindex{Goldmann, Paul 31.\,1.\,1865 Breslau – 25.\,9.\,1935 Wien@\textsc{Goldmann, Paul} (31.\,1.\,1865 Breslau – 25.\,9.\,1935 Wien), \emph{Schriftsteller, Journalist}|pw}.)« und zwei sprachliche Eingriffe)\hspace{1em}
\newline{}Handschrift Schreibkraft: Bleistift, lateinische Kurrent (\noindent{}Vermerk
                                       »n{[}icht{]}. a{[}bgesandt{]}.«,
                                 marginale Korrekturen, Paginierung der 1. Seite)}\toendnotes[C]{\smallbreak}
\pstart
           \raggedleft{}{\pb}28. I. 907.\pend
           \vspace{0.5em}
\pstart
           \label{K_L03520-1v}\edtext{Auch Schweigen wäre
                  Unaufrichtigkeit.}{\lemma{\textnormal{\emph{Auch … Unaufrichtigkeit.}}}\Cendnote{\textnormal{Der unmittelbare
                  Anlass für diesen nicht abgesandten Brief ist unklar. Womöglich war das
                  letzte Feuilleton Goldmanns\pwindex{Goldmann, Paul 31.\,1.\,1865 Breslau – 25.\,9.\,1935 Wien@\textsc{Goldmann, Paul} (31.\,1.\,1865 Breslau – 25.\,9.\,1935 Wien), \emph{Schriftsteller, Journalist}|pwk}
                  ausschlaggebend: \emph{Berliner Theater. »Mensch und
                     Übermensch« von Bernard Shaw}\pwindex{Goldmann, Paul 31.\,1.\,1865 Breslau – 25.\,9.\,1935 Wien@\textsc{Goldmann, Paul} (31.\,1.\,1865 Breslau – 25.\,9.\,1935 Wien), \emph{Schriftsteller, Journalist}!Berliner Theater. »Mensch und Übermensch« von Bernard Shaw@\strich\emph{Berliner Theater. »Mensch und Übermensch« von Bernard Shaw}|pwk} (\emph{Neue Freie Presse}\pwindex{Neue Freie Presse@\emph{Neue Freie Presse}|pwk}, Nr. 15.241,
                        25. 1. 1907, Morgenblatt, S. 1–4). Zumindest mit der
                  Wiederaufnahme des Begriffs ›polemisieren‹ aus dem vorangegangenen (erhaltenen)
                  Brief Goldmanns\pwindex{Goldmann, Paul 31.\,1.\,1865 Breslau – 25.\,9.\,1935 Wien@\textsc{Goldmann, Paul} (31.\,1.\,1865 Breslau – 25.\,9.\,1935 Wien), \emph{Schriftsteller, Journalist}|pwk} vom XXXX Auszeichnungsfehler: Dokument L03251 nicht gefunden bettet sich
                  dieses Schreiben in die Korrespondenz ein. Zugleich verfügt er durch die
                  Unterscheidung zwischen Dichter und Literat (ersterer arbeitet unter Einsatz
                  seiner Persönlichkeit), eine bedeutsame biografisch-werkästhetische Aussage für
                     Schnitzler. Das dürfte dieser selbst so
                  gesehen haben, denn dieser Text wird (zusammen mit den Abschriften seiner Briefe
                  an Goldmann\pwindex{Goldmann, Paul 31.\,1.\,1865 Breslau – 25.\,9.\,1935 Wien@\textsc{Goldmann, Paul} (31.\,1.\,1865 Breslau – 25.\,9.\,1935 Wien), \emph{Schriftsteller, Journalist}|pwk}, vgl. auch XXXX Auszeichnungsfehler: Dokument L03521 nicht gefunden) nicht bei den restlichen Briefen im \emph{Deutschen Literaturarchiv in Marbach} aufbewahrt, sondern findet sich im literarischen Teil des Nachlasses in
                  der \emph{Cambridge University Library}.}}}\label{K_L03520-1} Ich muss es Dir wieder einmal sagen. \label{T_L03520-1v}\edtext{Seit Jahren{[},{]} Du weisst es, verfolge ich
               Deine Feuilletons mit wachsendem Widerstand}{\lemma{\textnormal{\emph{Seit … Widerstand}}}\Cendnote{\textnormal{Die Unterstreichung, die sich hier in der Vorlage befindet,
                  dürfte von Schnitzler stammen und einem
                  archivalischen Zweck dienen, nicht der Textauszeichnung des Originals.}}}\label{T_L03520-1}. Es
               braucht nicht erst gesagt zu werden, dass auch meinem Geschmack \label{T_L03520-2v}\edtext{Einzelheiten}{\lemma{\textnormal{\emph{Einzelheiten}}}\Cendnote{\textnormal{korrigiert aus »einzelheiten«}}}\label{T_L03520-2} zusagen.
               Dass Du in einzelnem Recht hast. Aber als ganzes verwerf ich sie durchaus. Gesinnung
               und Ton. Ich wünsche nicht mit Dir zu polemisieren{[},{]}
                  vielmehr{[},{]} ich betone ausdrücklich, dass ich Unrecht haben
                  \label{T_L03520-3v}\edtext{kann,}{\lemma{\textnormal{\emph{kann,}}}\Cendnote{\textnormal{korrigiert aus »kann.,«}}}\label{T_L03520-3} dass Du
               sachlich Recht
               haben, dass Du sogar gut schreiben magst. Alles das ist möglich. Aber erst in einer
               fernen Zukunft wird das zu entscheiden sein. Und wir haben keine Zeit das abzuwarten.
               Das Wesentliche ist nur, dass Du und ich wie zwei fremde Welten einander gegenüber
               stehen. Dass unser Verhältnis zu dem, was heute gesagt, gedacht, geschrieben wird, in
               den wesentlichsten Punkten völlig von einander verschieden ist. Wir sind vor fünf {\pb}Jahren anlässlich Deiner \label{K_L03520-2v}\edtext{Stellungna{[}h{]}me\pwindex{Goldmann, Paul 31.\,1.\,1865 Breslau – 25.\,9.\,1935 Wien@\textsc{Goldmann, Paul} (31.\,1.\,1865 Breslau – 25.\,9.\,1935 Wien), \emph{Schriftsteller, Journalist}!Michael Kramer.«@\strich\emph{»Michael Kramer.«}|pwuv}{ }Hauptmann\pwindex{Hauptmann, Gerhart 15.\,11.\,1862 Szczawno-Zdrój – 6.\,6.\,1946 Jagniątków@\textsc{Hauptmann, Gerhart} (15.\,11.\,1862 Szczawno-Zdrój – 6.\,6.\,1946 Jagniątków), \emph{Schriftsteller}|pw}}{\lemma{\textnormal{\emph{Stellungnahme Hauptmann}}}\Cendnote{\textnormal{Höchstwahrscheinlich bezog sich Schnitzler auf ein älteres Feuilleton\pwindex{Goldmann, Paul 31.\,1.\,1865 Breslau – 25.\,9.\,1935 Wien@\textsc{Goldmann, Paul} (31.\,1.\,1865 Breslau – 25.\,9.\,1935 Wien), \emph{Schriftsteller, Journalist}!Michael Kramer.«@\strich\emph{»Michael Kramer.«}|pwkv}, nämlich Paul Goldmann\pwindex{Goldmann, Paul 31.\,1.\,1865 Breslau – 25.\,9.\,1935 Wien@\textsc{Goldmann, Paul} (31.\,1.\,1865 Breslau – 25.\,9.\,1935 Wien), \emph{Schriftsteller, Journalist}|pwk}: \emph{»Michael Kramer«}\pwindex{Goldmann, Paul 31.\,1.\,1865 Breslau – 25.\,9.\,1935 Wien@\textsc{Goldmann, Paul} (31.\,1.\,1865 Breslau – 25.\,9.\,1935 Wien), \emph{Schriftsteller, Journalist}!Michael Kramer.«@\strich\emph{»Michael Kramer.«}|pwk}. In: \emph{Neue Freie Presse}\pwindex{Neue Freie Presse@\emph{Neue Freie Presse}|pwk}, Nr. 13.055, 28. 12. 1900, Morgenblatt, S. 1–3, bzw. auf darauffolgende
                  Feuilletons und damit einhergehende Auseinandersetzungen, vgl. XXXX Auszeichnungsfehler: Dokument L02947 nicht gefunden, XXXX Auszeichnungsfehler: Dokument L03090 nicht gefunden und XXXX Auszeichnungsfehler: Dokument L03091 nicht gefunden.}}}\label{K_L03520-2}
               gegenüber zum erstenmal brieflich an einander geraten. Ich habe es
                  vorgezogen{[},{]} eine Disskussion abzubrechen, deren
               Hoffnungslosigkeit vom ersten \substVorne{}\textsuperscript{Tag}\substDazwischen{}Augenblick\substHinten{} an klar zu Tage lag. Der Verdacht, den später einmal andre, die mich nicht
                  kennen{[},{]} äussern könnten, dass erst persönliche
               Empfindlichkeit mich die Verschiederheit unserer Anschauungen\introOben{},\introOben{} unserer Naturen entdecken liess, fällt damit fort. Nun bin ich aber fern
               davon zu glauben, dass es zu den lebhaften inneren Differenzen gekommen wäre, wie sie
               nun bestehen, wenn nicht auch meine rein persönliche Sache zur Verhandlung stünde.
               Auch hier schalt ich gleich die Frage des Recht- oder Unrechthabens aus. Vielleicht
               wird Dir die Zukunft beistimmen und wird bei allen Dichtern deutscher Sprache, die
               heute leben und schaffen{[},{]} konstatieren{[},{]} was
               Du heute konstatierst, dass sie Dramen schreiben, in denen alles mangelt, {\pb}was einem Gedanken auch nur von fern
               ähnlich sieht. Und dass man überall in Deutschland\oindex{Deutschland@\textbf{Deutschland}|pw} Ideen finden kann{[},{]} nur nicht im modernen
               deutschen Drama. Sehr möglich, dass Du recht hast. Jedenfalls steht für mich die
               Sache so, dass ich nicht umhin kann{[},{]} mich mit den Dingen, die
               ich schreibe zu identifizieren. Es ist mir selbstverständlich bis heute noch nicht
               gelungen mich und meine Welt völlig zum Ausdruck zu bringen, aber die Arbeit\introOben{}en\introOben{} der letzten Zeit enthalten so viel von mir, \label{T_L03520-4v}\edtext{dass der,}{\lemma{\textnormal{\emph{dass der,}}}\Cendnote{\textnormal{korrigiert aus »dass, der«}}}\label{T_L03520-4} der sie
                  ablehnt{[},{]} von mir als Ganze\substVorne{}\textsuperscript{n}\substDazwischen{}m\substHinten{} sich abwenden muss. Das hat nichts mit persönlicher Eitelkeit zu tun. Es
               gibt Schriftsteller bei denen es möglich ist ihr Schaffen von ihrem Dasein zu
               trennen. Ich gehöre nicht zu ihnen. Ich vermeide es mich hinter der Legende von einer
               Persönlichkeit zu verstecken, die es verschmäht oder nicht imstande ist, ihr bestes,
               ihr Eigenstes in ihren Werken zum Ausdruck zu bringen. Man kann es zum Beispiel {\pb}bei Lothar\pwindex{Lothar, Rudolf 23.\,2.\,1865 Budapest – 2.\,10.\,1943 ebd.@\textsc{Lothar, Rudolf} (23.\,2.\,1865 Budapest – 2.\,10.\,1943 ebd.), \emph{Schriftsteller, Journalist, Theaterdirektor}|pw} trennen, was er ist und was er schreibt, kann es vielleicht in anderm
                  Sin{[}n{]} bei Hofmannsthal\pwindex{Hofmannsthal, Hugo von 1.\,2.\,1874 Wien – 15.\,7.\,1929 Rodaun@\textsc{Hofmannsthal, Hugo von} (1.\,2.\,1874 Wien – 15.\,7.\,1929 Rodaun), \emph{Schriftsteller}|pw},
               wieder in anderm bei Fulda\pwindex{Fulda, Ludwig 15.\,7.\,1862 Frankfurt am Main – 30.\,3.\,1939 Berlin@\textsc{Fulda, Ludwig} (15.\,7.\,1862 Frankfurt am Main – 30.\,3.\,1939 Berlin), \emph{Schriftsteller, Übersetzer}|pw}, gerade bei mir
               kann man es nicht. Ich bin\introOben{},\introOben{} was wieder die Zukunft zu
               entscheiden haben wird, vielleicht ein niederträchtiger Dichter, aber ich bin ein
               Dichter und kein Literat. Und übernehme die Verantwortung so gut für den Reigen\pwindex{Schnitzler, Arthur 15.\,5.\,1862 Wien – 21.\,10.\,1931 ebd.@\textsc{Schnitzler, Arthur} (15.\,5.\,1862 Wien – 21.\,10.\,1931 ebd.), \emph{Schriftsteller, Mediziner}!Reigen. Zehn Dialoge@\strich\emph{Reigen. Zehn Dialoge}|pw}, wie für den einsamen Weg\pwindex{Schnitzler, Arthur 15.\,5.\,1862 Wien – 21.\,10.\,1931 ebd.@\textsc{Schnitzler, Arthur} (15.\,5.\,1862 Wien – 21.\,10.\,1931 ebd.), \emph{Schriftsteller, Mediziner}!einsame Weg. Schauspiel in fünf Akten@\strich\emph{Der einsame Weg. Schauspiel in fünf Akten}|pw}, für den blinden
                  Geronimo\pwindex{Schnitzler, Arthur 15.\,5.\,1862 Wien – 21.\,10.\,1931 ebd.@\textsc{Schnitzler, Arthur} (15.\,5.\,1862 Wien – 21.\,10.\,1931 ebd.), \emph{Schriftsteller, Mediziner}!blinde Geronimo und sein Bruder@\strich\emph{Der blinde Geronimo und sein Bruder}|pw}, wie für die Berta Garlan\pwindex{Schnitzler, Arthur 15.\,5.\,1862 Wien – 21.\,10.\,1931 ebd.@\textsc{Schnitzler, Arthur} (15.\,5.\,1862 Wien – 21.\,10.\,1931 ebd.), \emph{Schriftsteller, Mediziner}!Frau Bertha Garlan. Roman@\strich\emph{Frau Bertha Garlan. Roman}|pw} u. s.
               w. Natürlich weiss ich sehr gut, dass mir formal einiges mehr,
                  a{[}n{]}deres minder gelungen ist und verstehe ohne weiters, dass
               auch jemand\introOben{}em\introOben{}, der mich schätzt, das eine oder das andre
               meiner Werke zuwider ist. Aber ich bestreite es, dass irgend ein Mensch, der beinah
               zu keinem dieser Werke ein Verhältnis zu finden imstande ist (und ihren Gehalt nicht
               spüren heisst für mich\introOben{}:\introOben{} kein Verhältnis zu ihnen finden) zu
               mir persönlich in irgend einem wirklichen Verhältnis zu stehen imstande {\pb}ist. Sind diese Werke ideenlos und
                  gering{[},{]} so muss ich es selbst auch sein. Und es ist nur ein
               Gebot der Selbstachtung{[},{]} eine menschliche \label{T_L03520-5v}\edtext{Beziehung}{\lemma{\textnormal{\emph{Beziehung}}}\Cendnote{\textnormal{korrigiert aus »Beziehung,«}}}\label{T_L03520-5} jener
               schönen Lüge zu entkleiden, die sie durch die \label{T_L03520-6v}\edtext{Ursupierung}{\lemma{\textnormal{\emph{Ursupierung}}}\Cendnote{\textnormal{korrigiert aus »Ursopierung«}}}\label{T_L03520-6} des \strikeout{schönen} Wortes Freundschaft um die Schultern schlägt. Und die Erinnerung
               unserer früheren Freundschaft steht mir zu hoch, als dass ich die Illusion aufrecht
               erhalten dürfte, zwei Menschen{[},{]} die so ziemlich über alle Dinge
               der Welt so verschieden denken, wie ich und Du könnten Freunde bleiben oder weiter
               Freunde heissen.\pend
           \selectlanguage{ngerman}\endnumbering\briefempfaengerindex{Goldmann, Paul@\textsc{Goldmann, Paul}!zzzSchnitzler, Arthur@\emph{von Arthur Schnitzler}!1907-01-281@{28. 1. 1907}|)be}\mylabel{L03520h}  \newcommand{\dateiname}{L03520}\newcommand{\titel}{Arthur Schnitzler an Paul Goldmann, nicht abgesandt, 28. 1. 1907}\newcommand{\editorInnen}{Martin Anton Müller und Laura Untner}%% latex-leseansicht-abspann.tex
%% Abspann für die Leseansicht.
%% Der Schalter \ifkorrekturansicht ist bereits durch den Vorspann gesetzt.

%% latex-abspann.tex
%% Gemeinsamer Abspann für Korrekturansicht und Leseansicht.
%% Setzt den Schalter \ifkorrekturansicht voraus (gesetzt in den
%% einbindenden Dateien latex-korrekturansicht-abspann.tex bzw.
%% latex-leseansicht-abspann.tex).
%% ---------------------------------------------------------------

\normalsize

% Das esempio-Environment wird nur in der Leseansicht benötigt
\ifkorrekturansicht\else
\newenvironment{esempio}[3]%
{
    \vspace{1.5ex}
    \rlap{\underline{#1}}
    \par
    \setlength{\parindent}{0cm}
    \nopagebreak
    \leftskip=#2cm
    \rightskip=#3cm
}
{
    \par
}
\fi

\doendnotes{C}
\bigskip
\vfill

\clearpage

\footnotesize

\ifkorrekturansicht
  \lohead{\textsc{register}}
\fi

% theindex-Environment neu definieren ohne reledmac
\makeatletter
\renewenvironment{theindex}{%
  \ifkorrekturansicht
    \section*{\indexname}%
  \else
    \subsubsection*{Index der erwähnten Entitäten}%
  \fi
  \setlength{\parindent}{0pt}%
  \setlength{\parskip}{0pt plus 0.3pt}%
  \let\item\@idxitem
}{%
  \ifkorrekturansicht\clearpage\fi
}
\makeatother

\IfFileExists{\jobname-pw.ind}{\input{\jobname-pw.ind}}{}

% Quellenangabe nur in der Leseansicht
\ifkorrekturansicht\else
% Fallback-Definitionen, falls die .tex-Datei \titel etc. nicht gesetzt hat
\providecommand{\titel}{}
\providecommand{\editorInnen}{}
\providecommand{\dateiname}{\jobname}

\vspace{3cm}

\vfill

\footnotesize
\textsc{Quelle}: \titel. Herausgegeben von {\editorInnen}. In: \emph{Arthur Schnitzler: Briefwechsel mit Autorinnen und Autoren}.
 Digitale Edition, https://schnitzler-briefe.acdh.oeaw.ac.at/{\dateiname}.html (Stand \today)
\fi

\end{document}


