%% latex-leseansicht-vorspann.tex
%% Vorspann für die Leseansicht.
%% Lädt die gemeinsame Datei latex-vorspann.tex mit nicht gesetztem Schalter.

\newif\ifkorrekturansicht
\korrekturansichtfalse

\input{../tex-inputs/latex-vorspann}


         
         \renewcommand{\erwaehntePersonen}{Personen: Richard Beer-Hofmann, Hugo von Hofmannsthal, Olga Schnitzler, Gustav Schwarzkopf}
         \renewcommand{\erwaehnteInstitutionen}{Institutionen: Wiener Verlag}
         \renewcommand{\erwaehnteOrte}{Orte: IX., Alsergrund, Liesingerstraße, Palasthotel Berlin, Rodaun, Wien}
         \renewcommand{\erwaehnteWerke}{Werke: Der einsame Weg. Schauspiel in fünf Akten, Professor Bernhardi. Komödie in fünf Akten, Reigen. Zehn Dialoge}
               \section[Arthur Schnitzler an Richard Beer-Hofmann, 20. 2. 1903]{ Arthur Schnitzler an Richard Beer-Hofmann, 20. 2. 1903}\nopagebreak\mylabel{v}\rehead{ }\begin{ledgroupsized}[t]{13cm}\normalsize\beginnumbering \toendnotes[C]{\smallbreak\pagebreak[2]} \Standort{YCGL, MSS 31.}
\physDesc{Brief, 1 Blatt, 4 Seiten, Umschlag, 971 Zeichen (Briefpapier mit Trauerrand)
\newline{}Handschrift: schwarze Tinte, deutsche Kurrent
\newline{}Versand: 1) Stempel: »\nobreak{}\oindex{IX., Alsergrund@\textbf{IX., Alsergrund}|pwk}9/3 Wien, 20. 2. 03, 5–6N\nobreak{}«.   2) Stempel: »\nobreak{}\oindex{Rodaun@\textbf{Rodaun}|pwk}{\pb}R{[}odau{]}n, 21. 2. {[}03{]}, 7–\textcolor{gray}{9}V\nobreak{}«. }\buchAbdrucke{\weitereDrucke{Arthur Schnitzler, Richard Beer-Hofmann: \emph{Briefwechsel 1891–1931}. Hg. Konstanze Fliedl. Wien, Zürich: \emph{Europaverlag} 1992, S. 161–162.} }\toendnotes[C]{\smallbreak}\pstart{}{\pb}Herrn \textsc{Dr Richard
                     Beer-Hofmann}\pend{}\pstart{}Rodaun\oindex{Rodaun@\textbf{Rodaun}|pw}\pend{}\pstart{}Lieſinger Hauptstraße 2\oindex{Liesingerstrasse@\textbf{Liesingerstraße}|pw}\pend{}{\bigskip}\pstart
           \raggedleft{}{\pb}20. 2. 903\pend
           \pstart{}Lieber Richard,\pend\pstart
           Ihnen und Hugo\pwindex{Hofmannsthal, Hugo von 1874-02-01 – 1929-07-15@\textsc{Hofmannsthal, Hugo von} (1874-02-01 – 1929-07-15), \emph{Schriftsteller}|pw} danke ich für das Gutachten und
               theile Ihnen mit, dſs ich heute gegen vorherige Honorirung von \introOben{}3\introOben{} Auflagen\pwindex{Schnitzler, Arthur 15.05.1862 – 21.10.1931@\textsc{Schnitzler, Arthur} (15.05.1862 – 21.10.1931), \emph{Schriftsteller, Mediziner}!Reigen. Zehn Dialoge1900@\strich\emph{Reigen. Zehn Dialoge} {[}1900{]}|pwv} mit dem Wiener Verlag\orgindex{Wiener Verlag@Wiener Verlag|pw}{ }\label{K_L01271_1v}\edtext{abgeſchloſſen}{\lemma{\textnormal{\emph{abgeſchloſſen}}}\Cendnote{\textnormal{für die Veröffentlichung des \emph{Reigen}\pwindex{Schnitzler, Arthur 15.05.1862 – 21.10.1931@\textsc{Schnitzler, Arthur} (15.05.1862 – 21.10.1931), \emph{Schriftsteller, Mediziner}!Reigen. Zehn Dialoge1900@\strich\emph{Reigen. Zehn Dialoge} {[}1900{]}|pwk}, der im April erscheinen
                  sollte}}}\label{K_L01271_1h} habe. Auch die Ausſtattung wird Ihren Wünſchen entſprechend
               ausfallen. –\pend
           \pstart
           Im übrigen reiſe ich morgen {\pb}nach Berlin, \uline{Palaſthotel}\oindex{Palasthotel Berlin@\textbf{Palasthotel Berlin}|pw} woſelbſt ich alſo bis etwa 8. März zu bleiben denke.\pend
           \pstart
           Mein neues Stück\pwindex{Schnitzler, Arthur 15.05.1862 – 21.10.1931@\textsc{Schnitzler, Arthur} (15.05.1862 – 21.10.1931), \emph{Schriftsteller, Mediziner}!einsame Weg. Schauspiel in fuenf Akten1904@\strich\emph{Der einsame Weg. Schauspiel in fünf Akten} {[}1904{]}|pw}\pwindex{Schnitzler, Arthur 15.05.1862 – 21.10.1931@\textsc{Schnitzler, Arthur} (15.05.1862 – 21.10.1931), \emph{Schriftsteller, Mediziner}!Professor Bernhardi. Komoedie in fuenf Akten1912@\strich\emph{Professor Bernhardi. Komödie in fünf Akten} {[}1912{]}|pw} in jetziger Faſſung
               iſt, nach theilweiſer Mittheilung an Olga\pwindex{Schnitzler, Olga 17.01.1882 – 13.01.1970@\textsc{Schnitzler, Olga} (17.01.1882 – 13.01.1970), \emph{Schauspielerin, Sängerin}|pw} und
                  Schwarzkopf\pwindex{Schwarzkopf, Gustav 07.11.1853 – 13.11.1939@\textsc{Schwarzkopf, Gustav} (07.11.1853 – 13.11.1939), \emph{Schriftsteller}|pw}, meinem eigenen Antrag
               entſprechend, misbilligt und damit erledigt worden. Es iſt ein ſiameſiſches {\pb}\label{K_L01271_2v}\edtext{Zwilling}{\lemma{\textnormal{\emph{Zwilling}}}\Cendnote{\textnormal{Gemeint ist die Trennung der Stoffe in \emph{Der einsame Weg}\pwindex{Schnitzler, Arthur 15.05.1862 – 21.10.1931@\textsc{Schnitzler, Arthur} (15.05.1862 – 21.10.1931), \emph{Schriftsteller, Mediziner}!einsame Weg. Schauspiel in fuenf Akten1904@\strich\emph{Der einsame Weg. Schauspiel in fünf Akten} {[}1904{]}|pwk} und dem späteren \emph{Professor Bernhardi}\pwindex{Schnitzler, Arthur 15.05.1862 – 21.10.1931@\textsc{Schnitzler, Arthur} (15.05.1862 – 21.10.1931), \emph{Schriftsteller, Mediziner}!Professor Bernhardi. Komoedie in fuenf Akten1912@\strich\emph{Professor Bernhardi. Komödie in fünf Akten} {[}1912{]}|pwk}.}}}\label{K_L01271_2h}; vielleicht hilft eine
               Operation, und Sie ſehen, zur rechten und zur linken je einen Siam
               herunterſinken.\pend
           \pstart
           – Immerhin, – es iſt eine »fertige Sach« – und ſomit bin ich beſſer gelaunt als alle
               dieſe letzten Tage{\dots}\pend
           \pstart
           Überdies, Frühling!. Soll man daran glauben?{\dots} Nun,
               genug.\pend
           \pstart
           {\pb}Ich hoffe, wir ſehen uns alle, in 3 Wochen etwa,
               geſund wieder.\pend
           \pstart
           Grüßen Sie allerorten.\pend
           \pstart
           Herzlichſt Ihr{\\[\baselineskip]}\spacefill\mbox{A.}\pend
           \leftskip=0em{}
         
         \endnumbering\mylabel{h}\end{ledgroupsized}  \newcommand{\dateiname}{L01271}\newcommand{\titel}{Arthur Schnitzler an Richard Beer-Hofmann, 20. 2. 1903}\newcommand{\editorInnen}{Martin Anton Müller und Gerd-Hermann Susen}%% latex-leseansicht-abspann.tex
%% Abspann für die Leseansicht.
%% Der Schalter \ifkorrekturansicht ist bereits durch den Vorspann gesetzt.

%% latex-abspann.tex
%% Gemeinsamer Abspann für Korrekturansicht und Leseansicht.
%% Setzt den Schalter \ifkorrekturansicht voraus (gesetzt in den
%% einbindenden Dateien latex-korrekturansicht-abspann.tex bzw.
%% latex-leseansicht-abspann.tex).
%% ---------------------------------------------------------------

\normalsize

% Das esempio-Environment wird nur in der Leseansicht benötigt
\ifkorrekturansicht\else
\newenvironment{esempio}[3]%
{
    \vspace{1.5ex}
    \rlap{\underline{#1}}
    \par
    \setlength{\parindent}{0cm}
    \nopagebreak
    \leftskip=#2cm
    \rightskip=#3cm
}
{
    \par
}
\fi

\doendnotes{C}
\bigskip
\vfill

\clearpage

\footnotesize

\ifkorrekturansicht
  \lohead{\textsc{register}}
\fi

% theindex-Environment neu definieren ohne reledmac
\makeatletter
\renewenvironment{theindex}{%
  \ifkorrekturansicht
    \section*{\indexname}%
  \else
    \subsubsection*{Index der erwähnten Entitäten}%
  \fi
  \setlength{\parindent}{0pt}%
  \setlength{\parskip}{0pt plus 0.3pt}%
  \let\item\@idxitem
}{%
  \ifkorrekturansicht\clearpage\fi
}
\makeatother

\IfFileExists{\jobname-pw.ind}{\input{\jobname-pw.ind}}{}

% Quellenangabe nur in der Leseansicht
\ifkorrekturansicht\else
% Fallback-Definitionen, falls die .tex-Datei \titel etc. nicht gesetzt hat
\providecommand{\titel}{}
\providecommand{\editorInnen}{}
\providecommand{\dateiname}{\jobname}

\vspace{3cm}

\vfill

\footnotesize
\textsc{Quelle}: \titel. Herausgegeben von {\editorInnen}. In: \emph{Arthur Schnitzler: Briefwechsel mit Autorinnen und Autoren}.
 Digitale Edition, https://schnitzler-briefe.acdh.oeaw.ac.at/{\dateiname}.html (Stand \today)
\fi

\end{document}


      