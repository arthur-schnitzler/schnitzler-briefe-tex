%% latex-korrekturansicht-vorspann.tex
%% Vorspann für die Korrekturansicht.
%% Lädt die gemeinsame Datei latex-vorspann.tex mit gesetztem Schalter.

\newif\ifkorrekturansicht
\korrekturansichttrue

\input{../tex-inputs/latex-vorspann}


\section[Arthur Schnitzler an Richard Beer-Hofmann, 20. 2. 1903]{L01271 Arthur Schnitzler an Richard Beer-Hofmann, 20. 2. 1903}
\nopagebreak\mylabel{L01271v}
\rehead{ }\normalsize\beginnumbering\briefempfaengerindex{Beer-Hofmann, Richard@\textsc{Beer-Hofmann, Richard}!zzzSchnitzler, Arthur@\emph{von Arthur Schnitzler}!1903-02-202@{20. 2. 1903}|(be}
\toendnotes[C]{\smallbreak\pagebreak[2]}\Standort{YCGL, MSS 31.}
\physDesc{Brief, 1 Blatt, 4 Seiten, Umschlag, 971 Zeichen
\newline{}Handschrift: schwarze Tinte, deutsche Kurrent
\newline{}Versand: 1) Stempel: »\nobreak{}\oindex{IX., Alsergrund@\textbf{IX., Alsergrund}, \emph{A.ADM3}|pwk}9/3 Wien, 20. 2. 03, 5–6N\nobreak{}«.   2) Stempel: »\nobreak{}\oindex{Rodaun@\textbf{Rodaun}, \emph{A.ADM4}|pwk}{\pb}R{[}odau{]}n, 21. 2. {[}03{]}, 7–\textcolor{gray}{9}V\nobreak{}«. }
\buchAbdrucke{\weitereDrucke{Arthur Schnitzler, Richard Beer-Hofmann: \emph{Briefwechsel 1891–1931}. Wien, Zürich: \emph{Europaverlag} 1992, S. 161–162.} }\toendnotes[C]{\smallbreak}\pstart{}{\pb}Herrn \textsc{Dr Richard
                     Beer-Hofmann}\pend{}\pstart{}Rodaun\oindex{Rodaun@\textbf{Rodaun}, \emph{A.ADM4}|pw}\pend{}\pstart{}Lieſinger Hauptstraße 2\oindex{Liesingerstrasse@\textbf{Liesingerstraße}, \emph{Straße (K.STR)}|pw}\pend{}{\bigskip}\vspace{1em}
\pstart
           \raggedleft{}{\pb}20. 2. 903\pend
           
\pstart{}Lieber Richard,\pend\vspace{0.5em}
\pstart
           Ihnen und Hugo\pwindex{Hofmannsthal, Hugo von 1874-02-01 – 1929-07-15@\textsc{Hofmannsthal, Hugo von} (1874-02-01 – 1929-07-15), \emph{Schriftsteller/Schriftstellerin}|pw} danke ich für das Gutachten und
               theile Ihnen mit, dſs ich heute gegen vorherige Honorirung von \introOben{}3\introOben{} Auflagen\pwindex{Reigen. Zehn Dialoge@\emph{Reigen. Zehn Dialoge}|pwv} mit dem Wiener Verlag\orgindex{Wiener Verlag@Wiener Verlag|pw}{ }\label{K_L01271-1v}\edtext{abgeſchloſſen}{\lemma{\textnormal{\emph{abgeſchloſſen}}}\Cendnote{\textnormal{für die Veröffentlichung des \emph{Reigen}\pwindex{Reigen. Zehn Dialoge@\emph{Reigen. Zehn Dialoge}|pwk}, der im April erscheinen
                  sollte}}}\label{K_L01271-1} habe. Auch die Ausſtattung wird Ihren Wünſchen entſprechend
               ausfallen. –\pend
           
\pstart
           Im übrigen reiſe ich morgen {\pb}nach Berlin, \uline{Palaſthotel}\oindex{Palasthotel Berlin@\textbf{Palasthotel Berlin}, \emph{Hotel (K.HTL)}|pw} woſelbſt ich alſo bis etwa 8. März zu bleiben denke.\pend
           
\pstart
           Mein neues Stück\pwindex{einsame Weg. Schauspiel in fuenf Akten@\emph{Der einsame Weg. Schauspiel in fünf Akten}|pw}\pwindex{Professor Bernhardi. Komoedie in fuenf Akten@\emph{Professor Bernhardi. Komödie in fünf Akten}|pw} in jetziger Faſſung
               iſt, nach theilweiſer Mittheilung an Olga\pwindex{Schnitzler, Olga 17.01.1882 – 13.01.1970@\textsc{Schnitzler, Olga} (17.01.1882 – 13.01.1970), \emph{Schauspieler/Schauspielerin, Sänger/Sängerin}|pw} und
                  Schwarzkopf\pwindex{Schwarzkopf, Gustav 07.11.1853 – 13.11.1939@\textsc{Schwarzkopf, Gustav} (07.11.1853 – 13.11.1939), \emph{Schriftsteller/Schriftstellerin}|pw}, meinem eigenen Antrag
               entſprechend, misbilligt und damit erledigt worden. Es iſt ein ſiameſiſches {\pb}\label{K_L01271-2v}\edtext{Zwilling}{\lemma{\textnormal{\emph{Zwilling}}}\Cendnote{\textnormal{Gemeint ist die Trennung der Stoffe in \emph{Der einsame Weg}\pwindex{einsame Weg. Schauspiel in fuenf Akten@\emph{Der einsame Weg. Schauspiel in fünf Akten}|pwk} und \emph{Professor Bernhardi}\pwindex{Professor Bernhardi. Komoedie in fuenf Akten@\emph{Professor Bernhardi. Komödie in fünf Akten}|pwk}.}}}\label{K_L01271-2}; vielleicht hilft eine
               Operation, und Sie ſehen, zur rechten und zur linken je einen Siam
               herunterſinken.\pend
           
\pstart
           – Immerhin, – es iſt eine »fertige Sach« – und ſomit bin ich beſſer gelaunt als alle
               dieſe letzten Tage{\dots}\pend
           
\pstart
           Überdies, Frühling!. Soll man daran glauben?{\dots} Nun,
               genug.\pend
           
\pstart
           {\pb}Ich hoffe, wir ſehen uns alle, in 3 Wochen etwa,
               geſund wieder.\pend
           
\pstart
           Grüßen Sie allerorten.\pend
           
\pstart
           Herzlichſt Ihr{\\[\baselineskip]}\spacefill\mbox{A.}\pend
           \leftskip=0em{}\selectlanguage{ngerman}\endnumbering\briefempfaengerindex{Beer-Hofmann, Richard@\textsc{Beer-Hofmann, Richard}!zzzSchnitzler, Arthur@\emph{von Arthur Schnitzler}!1903-02-202@{20. 2. 1903}|)be}\mylabel{L01271h}  \normalsize

\doendnotes{C}
\bigskip
\vfill

\clearpage

\footnotesize

\lohead{\textsc{register}}

% Definiere theindex-Environment komplett neu ohne reledmac
\makeatletter
\renewenvironment{theindex}{%
  \section*{\indexname}%
  \setlength{\parindent}{0pt}%
  \setlength{\parskip}{0pt plus 0.3pt}%
  \let\item\@idxitem
}{%
  \clearpage
}
\makeatother

\IfFileExists{\jobname-pw.ind}{\input{\jobname-pw.ind}}{}

\end{document}

      