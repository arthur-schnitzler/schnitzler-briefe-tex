%% latex-leseansicht-vorspann.tex
%% Vorspann für die Leseansicht.
%% Lädt die gemeinsame Datei latex-vorspann.tex mit nicht gesetztem Schalter.

\newif\ifkorrekturansicht
\korrekturansichtfalse

\input{../tex-inputs/latex-vorspann}


\section[Arthur Schnitzler an Theodor Herzl, 7. 1. 189[5?]]{L03918 Arthur Schnitzler an Theodor Herzl, 7. 1. 189[5?]}
\nopagebreak\mylabel{L03918v}
\rehead{ }\normalsize\beginnumbering\briefempfaengerindex{Herzl, Theodor@\textsc{Herzl, Theodor}!zzzSchnitzler, Arthur@\emph{von Arthur Schnitzler}!1895-01-071@{7. 1. 189[5?]}|(be}
\toendnotes[C]{\smallbreak\pagebreak[2]}
\correspDesc{Versand  durch Arthur Schnitzler am 7. 1. 189[5?] in Wien
\newline{}Erhalt  durch Theodor Herzl in Wien}\toendnotes[C]{\smallbreak}
\Standort{Jerusalem, Central Zionist Archives, H1:1925-3.}
\physDesc{,  Blätter,  Seiten
\newline{}Handschrift: , deutsche Kurrent}
\buchAbdrucke{\weitereDrucke{Arthur Schnitzler: \emph{Briefe 1875–1912}. Herausgegeben von Therese Nickl und Heinrich Schnitzler. Frankfurt am Main: \emph{S. Fischer} 1981, S. 247–249.} }\toendnotes[C]{\smallbreak}
\pstart
           {\pb}Wien\oindex{Wien@\textbf{Wien}, \emph{Verwaltungsgebiet}|pw}, 7. Jä{\geminationn}er \label{K_L03918-1v}\edtext{94}{\lemma{\textnormal{\emph{94}}}\Cendnote{\textnormal{Schreibirrtum, wie aus dem Inhalt
                        hervorgeht, stammt es von Anfang 1895.}}}\label{K_L03918-1}.\pend
           
\pstart{}Mein lieber Freund!\pend\vspace{0.5em}
\pstart
           Heute iſt das \textsc{Ghetto\pwindex{Herzl, Theodor 2.\,5.\,1860 Budapest – 3.\,7.\,1904 Edlach@\textsc{Herzl, Theodor} (2.\,5.\,1860 Budapest – 3.\,7.\,1904 Edlach), \emph{Schriftsteller, Journalist}!neue Ghetto. Schauspiel in vier Acten@\strich\emph{Das neue Ghetto. Schauspiel in vier Acten}|pw}} abgeſandt worden. Ans deutſche Theater\orgindex{Deutsches Theater Berlin@Deutsches Theater Berlin|pw}. Das
               iſt auch deshalb vortheilhafter, weil Leſſingtheater\orgindex{Lessing-Theater@Lessing-Theater|pw} u. Berliner Theater\orgindex{Berliner Theater@Berliner Theater|pw} jetzt
               denſelben Director\pwindex{Blumenthal, Oskar 13.\,3.\,1852 Berlin – 24.\,4.\,1917 ebd.@\textsc{Blumenthal, Oskar} (13.\,3.\,1852 Berlin – 24.\,4.\,1917 ebd.), \emph{Schriftsteller, Journalist, Theaterleiter}|pwv} haben –
                  \textsc{Blumenthal\pwindex{Blumenthal, Oskar 13.\,3.\,1852 Berlin – 24.\,4.\,1917 ebd.@\textsc{Blumenthal, Oskar} (13.\,3.\,1852 Berlin – 24.\,4.\,1917 ebd.), \emph{Schriftsteller, Journalist, Theaterleiter}|pw}}. – Das Stück\pwindex{Herzl, Theodor 2.\,5.\,1860 Budapest – 3.\,7.\,1904 Edlach@\textsc{Herzl, Theodor} (2.\,5.\,1860 Budapest – 3.\,7.\,1904 Edlach), \emph{Schriftsteller, Journalist}!neue Ghetto. Schauspiel in vier Acten@\strich\emph{Das neue Ghetto. Schauspiel in vier Acten}|pwv} iſt ſehr
               gut geſchrieben, die Schrift auf der Rechnung, die ich Ihrem Wunsche gemäſs (den Sie
               mir nicht gar ſo ſtreng hätten ausdrücken müſſen!) beilege, iſt die des Abſchreibers;
               –{ }ſie wird {\pb}ſie über dieſen Punkt beruhigen. der
               Begleitbrief iſt vor den Anfang hingeklebt worden (\uline{natürlich auch in Abſchrift.}) Ich habe das Stück\pwindex{Herzl, Theodor 2.\,5.\,1860 Budapest – 3.\,7.\,1904 Edlach@\textsc{Herzl, Theodor} (2.\,5.\,1860 Budapest – 3.\,7.\,1904 Edlach), \emph{Schriftsteller, Journalist}!neue Ghetto. Schauspiel in vier Acten@\strich\emph{Das neue Ghetto. Schauspiel in vier Acten}|pwv}{ }ſehr ſorgfältig in der Abſchrift durchgeleſen, Ihrer
               Angabe nach unterſtrichen und kleine Correcturen angebracht, welche durch kleine
               Verſehen des Abſchreibers\pwindex{?? [Schreibkraft, die Das neue Ghetto abschreibt] @\textsc{?? [Schreibkraft, die Das neue Ghetto abschreibt]}|pwv} nothwendig
               wurde. Das Geheimnis ist vollko{\geminationm}en gewahrt; auch {\pb}Herrn \textsc{Schick\pwindex{Schik, Friedrich *~6.\,9.\,1857 Wien@\textsc{Schik, Friedrich} (*~6.\,9.\,1857 Wien), \emph{Notar, Journalist, Dramaturg}|pw}} hab ich Ihren Namen nicht genannt, und ſoweit bin ich, we{\geminationn} Sie nicht ſelbſt noch jemanden eingeweiht haben, der
               einzige lebende Menſch außer Ihnen, der den Verfaſſer weiſs. Es war nicht
               durchführbar, daß die Abſchrift bei mir gemacht wurde; da ich zu den Zeiten, in
               welchem der \substVorne{}\textsuperscript{ſelbe}\substDazwischen{}Abſchreiber\substHinten{} arbeiten konnte, nicht an{\pb}weſend hätte ſein können;
               er hat in ſeiner Wohnung geſchrieben, mitgegeben hab ich ihm das \textsc{Mscr.\pwindex{Herzl, Theodor 2.\,5.\,1860 Budapest – 3.\,7.\,1904 Edlach@\textsc{Herzl, Theodor} (2.\,5.\,1860 Budapest – 3.\,7.\,1904 Edlach), \emph{Schriftsteller, Journalist}!neue Ghetto. Schauspiel in vier Acten@\strich\emph{Das neue Ghetto. Schauspiel in vier Acten}|pwv}} Abends nach 6; einmal \strikeout{aber} um ½ 3, ſorgfältig
               verſchloſſen. Sie können verſichert ſein, daſs kein Unberufener Einſicht geno{\geminationm}en hat.– Morgen geht der Brief II ans Deutſche Theater\orgindex{Deutsches Theater Berlin@Deutsches Theater Berlin|pw} ab. {\pb}Und nun, viel
               Glück!– Ich habe gute Hoffnung. Es iſt Ihnen gelungen ein »literariſches« Stück\pwindex{Herzl, Theodor 2.\,5.\,1860 Budapest – 3.\,7.\,1904 Edlach@\textsc{Herzl, Theodor} (2.\,5.\,1860 Budapest – 3.\,7.\,1904 Edlach), \emph{Schriftsteller, Journalist}!neue Ghetto. Schauspiel in vier Acten@\strich\emph{Das neue Ghetto. Schauspiel in vier Acten}|pwv} zu ſchreiben, das
               zugleich gutes Theater iſt. Sie wiſſen, wie mein erſter Eindruck war. Diesmal hat mir
               das Schauſpiel\pwindex{Herzl, Theodor 2.\,5.\,1860 Budapest – 3.\,7.\,1904 Edlach@\textsc{Herzl, Theodor} (2.\,5.\,1860 Budapest – 3.\,7.\,1904 Edlach), \emph{Schriftsteller, Journalist}!neue Ghetto. Schauspiel in vier Acten@\strich\emph{Das neue Ghetto. Schauspiel in vier Acten}|pwv} noch viel
               beſſer gefallen; die Lebendigkeit der Geſtalten wurde mir eindringlicher und es ſind
               eine {\pb}ganze Reihe von Scenen\pwindex{Herzl, Theodor 2.\,5.\,1860 Budapest – 3.\,7.\,1904 Edlach@\textsc{Herzl, Theodor} (2.\,5.\,1860 Budapest – 3.\,7.\,1904 Edlach), \emph{Schriftsteller, Journalist}!neue Ghetto. Schauspiel in vier Acten@\strich\emph{Das neue Ghetto. Schauspiel in vier Acten}|pwv} darin, die nicht nur in menſchlicher Hinſicht
               ergreifen, die auch auf der Bühne packen müſſen. Einige Figuren habe ich jetzt erſt{ }ſo recht lieb gewonnen. Wie prächtig dieſe alten Samuels\pwindex{Herzl, Theodor 2.\,5.\,1860 Budapest – 3.\,7.\,1904 Edlach@\textsc{Herzl, Theodor} (2.\,5.\,1860 Budapest – 3.\,7.\,1904 Edlach), \emph{Schriftsteller, Journalist}!neue Ghetto. Schauspiel in vier Acten@\strich\emph{Das neue Ghetto. Schauspiel in vier Acten}|pwv}! Schwer zu beſetzen wird es sein –
               Menſchen! Menschen! – Noch immer ſcheint mir Jacob\pwindex{Herzl, Theodor 2.\,5.\,1860 Budapest – 3.\,7.\,1904 Edlach@\textsc{Herzl, Theodor} (2.\,5.\,1860 Budapest – 3.\,7.\,1904 Edlach), \emph{Schriftsteller, Journalist}!neue Ghetto. Schauspiel in vier Acten@\strich\emph{Das neue Ghetto. Schauspiel in vier Acten}|pwv} der {\pb}bläßeſte zu ſein.
               Durch ſeine Haut{ }ſchimmert zu ſtark, was ja ſchon kräftig genug in den Vorgängen des
                  Dramas\pwindex{Herzl, Theodor 2.\,5.\,1860 Budapest – 3.\,7.\,1904 Edlach@\textsc{Herzl, Theodor} (2.\,5.\,1860 Budapest – 3.\,7.\,1904 Edlach), \emph{Schriftsteller, Journalist}!neue Ghetto. Schauspiel in vier Acten@\strich\emph{Das neue Ghetto. Schauspiel in vier Acten}|pwv} ſich ausdrückt, –
               die Idee des Stücks\pwindex{Herzl, Theodor 2.\,5.\,1860 Budapest – 3.\,7.\,1904 Edlach@\textsc{Herzl, Theodor} (2.\,5.\,1860 Budapest – 3.\,7.\,1904 Edlach), \emph{Schriftsteller, Journalist}!neue Ghetto. Schauspiel in vier Acten@\strich\emph{Das neue Ghetto. Schauspiel in vier Acten}|pwv}. Der
               Grundfehler aller modernen Helden beinahe?–\pend
           
\pstart
           Es iſt wahrhaftig ein Stück\pwindex{Herzl, Theodor 2.\,5.\,1860 Budapest – 3.\,7.\,1904 Edlach@\textsc{Herzl, Theodor} (2.\,5.\,1860 Budapest – 3.\,7.\,1904 Edlach), \emph{Schriftsteller, Journalist}!neue Ghetto. Schauspiel in vier Acten@\strich\emph{Das neue Ghetto. Schauspiel in vier Acten}|pwv},
               an dem man ſich freuen kann, ein Stück\pwindex{Herzl, Theodor 2.\,5.\,1860 Budapest – 3.\,7.\,1904 Edlach@\textsc{Herzl, Theodor} (2.\,5.\,1860 Budapest – 3.\,7.\,1904 Edlach), \emph{Schriftsteller, Journalist}!neue Ghetto. Schauspiel in vier Acten@\strich\emph{Das neue Ghetto. Schauspiel in vier Acten}|pwv} des Lebens, das rings um uns iſt, {\pb}das Stück\pwindex{Herzl, Theodor 2.\,5.\,1860 Budapest – 3.\,7.\,1904 Edlach@\textsc{Herzl, Theodor} (2.\,5.\,1860 Budapest – 3.\,7.\,1904 Edlach), \emph{Schriftsteller, Journalist}!neue Ghetto. Schauspiel in vier Acten@\strich\emph{Das neue Ghetto. Schauspiel in vier Acten}|pwv} eines ganz lebendigen, mit Geberden, die heftig und doch zielbewußt,
               abſichtsvoll und doch mannigfaltig{ }ſind. Was Sie überhaupt für ein Meiſter der
               Plaſtik ſind. Ihr letztes \label{K_L03918-55v}\edtext{Feu{[}i{]}lleton\pwindex{Herzl, Theodor 2.\,5.\,1860 Budapest – 3.\,7.\,1904 Edlach@\textsc{Herzl, Theodor} (2.\,5.\,1860 Budapest – 3.\,7.\,1904 Edlach), \emph{Schriftsteller, Journalist}!Palais Bourbon. V. »Sprechen wir über Politik«@\strich\emph{Das Palais Bourbon. V. »Sprechen wir über Politik{\rufezeichen}«}|pwv}}{\lemma{\textnormal{\emph{Feuilleton}}}\Cendnote{\textnormal{Theodor Herzl\pwindex{Herzl, Theodor 2.\,5.\,1860 Budapest – 3.\,7.\,1904 Edlach@\textsc{Herzl, Theodor} (2.\,5.\,1860 Budapest – 3.\,7.\,1904 Edlach), \emph{Schriftsteller, Journalist}|pwk}: \emph{Das Palais Bourbon. V. »Sprechen wir von Politik!«}\pwindex{Herzl, Theodor 2.\,5.\,1860 Budapest – 3.\,7.\,1904 Edlach@\textsc{Herzl, Theodor} (2.\,5.\,1860 Budapest – 3.\,7.\,1904 Edlach), \emph{Schriftsteller, Journalist}!Palais Bourbon. V. »Sprechen wir über Politik«@\strich\emph{Das Palais Bourbon. V. »Sprechen wir über Politik{\rufezeichen}«}|pwk}
                     In: \emph{Neue Freie Presse}\pwindex{Neue Freie Presse@\emph{Neue Freie Presse}|pwk}, Nr. 10.906,
                        3. 1. 1895, Morgenblatt, S. 1–4.}}}\label{K_L03918-55}
               war unvergleichlich. Wie man dieſe Leute ſieht, hört – es gibt keinen, der diese
               Schärfe in den Linien träfe, der so Figuren umreißen könnte.–\pend
           
\pstart
           Leben Sie wohl, mein lieben Freund! Die treueſten Wünſche begleiten Ihr Stück\pwindex{Herzl, Theodor 2.\,5.\,1860 Budapest – 3.\,7.\,1904 Edlach@\textsc{Herzl, Theodor} (2.\,5.\,1860 Budapest – 3.\,7.\,1904 Edlach), \emph{Schriftsteller, Journalist}!neue Ghetto. Schauspiel in vier Acten@\strich\emph{Das neue Ghetto. Schauspiel in vier Acten}|pwv} nach Berlin\oindex{Berlin@\textbf{Berlin}, \emph{Hauptstadt}|pw}.\pend
           \pstart Ihr \spacefill\mbox{Arth Schn}\pend{}\selectlanguage{ngerman}\endnumbering\briefempfaengerindex{Herzl, Theodor@\textsc{Herzl, Theodor}!zzzSchnitzler, Arthur@\emph{von Arthur Schnitzler}!1895-01-071@{7. 1. 189[5?]}|)be}\mylabel{L03918h}
\begin{anhang}
\end{anhang}\newcommand{\dateiname}{L03918}\newcommand{\titel}{Arthur Schnitzler an Theodor Herzl, 7. 1. 189[5?]}\newcommand{\editorInnen}{Herausgegeben von Jahnke, SelmaMüller, Martin Anton}%% latex-leseansicht-abspann.tex
%% Abspann für die Leseansicht.
%% Der Schalter \ifkorrekturansicht ist bereits durch den Vorspann gesetzt.

%% latex-abspann.tex
%% Gemeinsamer Abspann für Korrekturansicht und Leseansicht.
%% Setzt den Schalter \ifkorrekturansicht voraus (gesetzt in den
%% einbindenden Dateien latex-korrekturansicht-abspann.tex bzw.
%% latex-leseansicht-abspann.tex).
%% ---------------------------------------------------------------

\normalsize

% Das esempio-Environment wird nur in der Leseansicht benötigt
\ifkorrekturansicht\else
\newenvironment{esempio}[3]%
{
    \vspace{1.5ex}
    \rlap{\underline{#1}}
    \par
    \setlength{\parindent}{0cm}
    \nopagebreak
    \leftskip=#2cm
    \rightskip=#3cm
}
{
    \par
}
\fi

\doendnotes{C}
\bigskip
\vfill

\clearpage

\footnotesize

\ifkorrekturansicht
  \lohead{\textsc{register}}
\fi

% theindex-Environment neu definieren ohne reledmac
\makeatletter
\renewenvironment{theindex}{%
  \ifkorrekturansicht
    \section*{\indexname}%
  \else
    \subsubsection*{Index der erwähnten Entitäten}%
  \fi
  \setlength{\parindent}{0pt}%
  \setlength{\parskip}{0pt plus 0.3pt}%
  \let\item\@idxitem
}{%
  \ifkorrekturansicht\clearpage\fi
}
\makeatother

\IfFileExists{\jobname-pw.ind}{\input{\jobname-pw.ind}}{}

% Quellenangabe nur in der Leseansicht
\ifkorrekturansicht\else
% Fallback-Definitionen, falls die .tex-Datei \titel etc. nicht gesetzt hat
\providecommand{\titel}{}
\providecommand{\editorInnen}{}
\providecommand{\dateiname}{\jobname}

\vspace{3cm}

\vfill

\footnotesize
\textsc{Quelle}: \titel. Herausgegeben von {\editorInnen}. In: \emph{Arthur Schnitzler: Briefwechsel mit Autorinnen und Autoren}.
 Digitale Edition, https://schnitzler-briefe.acdh.oeaw.ac.at/{\dateiname}.html (Stand \today)
\fi

\end{document}


