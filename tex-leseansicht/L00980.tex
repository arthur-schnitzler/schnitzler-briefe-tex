%% latex-korrekturansicht-vorspann.tex
%% Vorspann für die Korrekturansicht.
%% Lädt die gemeinsame Datei latex-vorspann.tex mit gesetztem Schalter.

\newif\ifkorrekturansicht
\korrekturansichttrue

\input{../tex-inputs/latex-vorspann}


\section[Arthur Schnitzler an Richard Beer-Hofmann, 24. 9. 1899]{L00980 Arthur Schnitzler an Richard Beer-Hofmann, 24. 9. 1899}
\nopagebreak\mylabel{L00980v}
\rehead{ }\normalsize\beginnumbering\briefempfaengerindex{Beer-Hofmann, Richard@\textsc{Beer-Hofmann, Richard}!zzzSchnitzler, Arthur@\emph{von Arthur Schnitzler}!1899-09-241@{24. 9. 1899}|(be}
\toendnotes[C]{\smallbreak\pagebreak[2]}\Standort{YCGL, MSS 31.}
\physDesc{Bildpostkarte, 237 Zeichen
\newline{}Handschrift: Bleistift, deutsche Kurrent
\newline{}Versand: 1) Stempel: »\nobreak{}\oindex{Wiesbaden@\textbf{Wiesbaden}, \emph{P.PPLA}|pwk}Wiesbaden, 24. 9. 99, 6–7N\nobreak{}«.   2) Stempel: »\nobreak{}\oindex{Vahrn@\textbf{Vahrn}, \emph{P.PPLA3}|pwk}V{[}ahrn{]}, 2\textcolor{gray}{6.} 9. 99\nobreak{}«. 
\newline{}Ordnung: mit Bleistift von unbekannter Hand datiert: »24. 9.« }\toendnotes[C]{\smallbreak}\pstart{}{\pb}\textsc{Dr. Richard Beer-Hofmann}\pend{}\pstart{}\textsc{Vahrn}\oindex{Vahrn@\textbf{Vahrn}, \emph{P.PPLA3}|pw}\pend{}\pstart{}bei \textsc{Brixen\oindex{Brixen@\textbf{Brixen}, \emph{P.PPLA3}|pw}}\pend{}\pstart{}\textsc{Tirol}\oindex{Tirol@\textbf{Tirol}, \emph{A.ADM1}|pw}\pend{}{\bigskip}
\pstart
           \noindent{}\centering{}{\pb}\textcolor{gray}{\textbf{Nerobergbahn\oindex{Nerobergbahn@\textbf{Nerobergbahn}, \emph{Bahn}|pw}{ }Wiesbaden\oindex{Wiesbaden@\textbf{Wiesbaden}, \emph{P.PPLA}|pw}}}\pend
           \vspace{1em}
\pstart
           \raggedleft{}{\pb}{\pb}24. 9. 99.\pend
           \vspace{0.5em}
\pstart
           Will hier 8 Tage bleiben, arbeiten\pend
           
\pstart
           Bitte ſchreiben Sie mir, auch Hugo\pwindex{Hofmannsthal, Hugo von 1874-02-01 – 1929-07-15@\textsc{Hofmannsthal, Hugo von} (1874-02-01 – 1929-07-15), \emph{Schriftsteller/Schriftstellerin}|pw}, wie’s
               Ihnen geht, und was die Arbeit macht. Ich wohne Parkhotel\oindex{Hôtel du Parc {\kaufmannsund} Bristol@\textbf{Hôtel du Parc {\kaufmannsund} Bristol}, \emph{Hotel (K.HTL)}|pw}.\pend
           
\pstart
           Die \label{K_L00980-1v}\edtext{Ovation}{\lemma{\textnormal{\emph{Ovation}}}\Cendnote{\textnormal{Vgl. Richard Beer-Hofmann und Hugo von Hofmannsthal an Arthur Schnitzler,
               19. 9. 1899.
               }}}\label{K_L00980-1} hab ich erhalten. Paul\pwindex{Goldmann, Paul 31.01.1865 – 25.09.1935@\textsc{Goldmann, Paul} (31.01.1865 – 25.09.1935), \emph{Schriftsteller/Schriftstellerin, Journalist/Journalistin}|pw} ist heut nach
                  Florenz\oindex{Florenz@\textbf{Florenz}, \emph{P.PPLA}|pw}.\pend
           \selectlanguage{ngerman}\endnumbering\briefempfaengerindex{Beer-Hofmann, Richard@\textsc{Beer-Hofmann, Richard}!zzzSchnitzler, Arthur@\emph{von Arthur Schnitzler}!1899-09-241@{24. 9. 1899}|)be}\mylabel{L00980h}  \normalsize

\doendnotes{C}
\bigskip
\vfill

\clearpage

\footnotesize

\lohead{\textsc{register}}

% Definiere theindex-Environment komplett neu ohne reledmac
\makeatletter
\renewenvironment{theindex}{%
  \section*{\indexname}%
  \setlength{\parindent}{0pt}%
  \setlength{\parskip}{0pt plus 0.3pt}%
  \let\item\@idxitem
}{%
  \clearpage
}
\makeatother

\IfFileExists{\jobname-pw.ind}{\input{\jobname-pw.ind}}{}

\end{document}

      