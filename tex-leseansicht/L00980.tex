%% latex-leseansicht-vorspann.tex
%% Vorspann für die Leseansicht.
%% Lädt die gemeinsame Datei latex-vorspann.tex mit nicht gesetztem Schalter.

\newif\ifkorrekturansicht
\korrekturansichtfalse

\input{../tex-inputs/latex-vorspann}


\section[Arthur Schnitzler an Richard Beer-Hofmann, 24. 9. 1899]{L00980 Arthur Schnitzler an Richard Beer-Hofmann, 24. 9. 1899}
\nopagebreak\mylabel{L00980v}
\rehead{ }\normalsize\beginnumbering\briefempfaengerindex{Beer-Hofmann, Richard@\textsc{Beer-Hofmann, Richard}!zzzSchnitzler, Arthur@\emph{von Arthur Schnitzler}!1899-09-241@{24. 9. 1899}|(be}
\toendnotes[C]{\smallbreak\pagebreak[2]}
\correspDesc{Versand  durch Arthur Schnitzler am 24. 9. 1899 in Wiesbaden
\newline{}Erhalt  durch Richard Beer-Hofmann am 26. 9. 1899 in Vahrn}\toendnotes[C]{\smallbreak}
\Standort{YCGL, MSS 31.}
\physDesc{Bildpostkarte, 237 Zeichen
\newline{}Handschrift: Bleistift, deutsche Kurrent
\newline{}Versand: 1) Stempel: »\nobreak{}\oindex{Wiesbaden@\textbf{Wiesbaden}|pwk}Wiesbaden, 24. 9. 99, 6–7N\nobreak{}«.   2) Stempel: »\nobreak{}\oindex{Vahrn@\textbf{Vahrn}, \emph{Hauptstadt}|pwk}V{[}ahrn{]}, 2\textcolor{gray}{6.} 9. 99\nobreak{}«. 
\newline{}Ordnung: mit Bleistift von unbekannter Hand datiert: »24. 9.« }\toendnotes[C]{\smallbreak}\pstart{}{\pb}\textsc{Dr. Richard Beer-Hofmann}\pend{}\pstart{}\textsc{Vahrn}\oindex{Vahrn@\textbf{Vahrn}, \emph{Hauptstadt}|pw}\pend{}\pstart{}bei \textsc{Brixen\oindex{Brixen@\textbf{Brixen}, \emph{Hauptstadt}|pw}}\pend{}\pstart{}\textsc{Tirol}\oindex{Tirol@\textbf{Tirol}, \emph{Land}|pw}\pend{}{\bigskip}
\pstart
           \noindent{}\centering{}{\pb}\textcolor{gray}{\textbf{Nerobergbahn\oindex{Nerobergbahn@\textbf{Nerobergbahn}, \emph{Bahn}|pw}{ }Wiesbaden\oindex{Wiesbaden@\textbf{Wiesbaden}|pw}}}\pend
           \vspace{1em}
\pstart
           \raggedleft{}{\pb}{\pb}24. 9. 99.\pend
           \vspace{0.5em}
\pstart
           Will hier 8 Tage bleiben, arbeiten\pend
           
\pstart
           Bitte{ }ſchreiben Sie mir, auch Hugo\pwindex{Hofmannsthal, Hugo von 1.\,2.\,1874 Wien – 15.\,7.\,1929 Rodaun@\textsc{Hofmannsthal, Hugo von} (1.\,2.\,1874 Wien – 15.\,7.\,1929 Rodaun), \emph{Schriftsteller}|pw}, wie’s
               Ihnen geht, und was die Arbeit macht. Ich wohne Parkhotel\oindex{Hôtel du Parc {\kaufmannsund} Bristol@\textbf{Hôtel du Parc {\kaufmannsund} Bristol}, \emph{Hotel}|pw}.\pend
           
\pstart
           Die \label{K_L00980-1v}\edtext{Ovation}{\lemma{\textnormal{\emph{Ovation}}}\Cendnote{\textnormal{Vgl. XXXX Auszeichnungsfehler: Dokument L00978 nicht gefunden.
               }}}\label{K_L00980-1} hab ich erhalten. Paul\pwindex{Goldmann, Paul 31.\,1.\,1865 Breslau – 25.\,9.\,1935 Wien@\textsc{Goldmann, Paul} (31.\,1.\,1865 Breslau – 25.\,9.\,1935 Wien), \emph{Schriftsteller, Journalist}|pw} ist heut nach
                  Florenz\oindex{Florenz@\textbf{Florenz}|pw}.\pend
           \selectlanguage{ngerman}\endnumbering\briefempfaengerindex{Beer-Hofmann, Richard@\textsc{Beer-Hofmann, Richard}!zzzSchnitzler, Arthur@\emph{von Arthur Schnitzler}!1899-09-241@{24. 9. 1899}|)be}\mylabel{L00980h}  \newcommand{\dateiname}{L00980}\newcommand{\titel}{Arthur Schnitzler an Richard Beer-Hofmann, 24. 9. 1899}\newcommand{\editorInnen}{Martin Anton Müller und Gerd-Hermann Susen}%% latex-leseansicht-abspann.tex
%% Abspann für die Leseansicht.
%% Der Schalter \ifkorrekturansicht ist bereits durch den Vorspann gesetzt.

%% latex-abspann.tex
%% Gemeinsamer Abspann für Korrekturansicht und Leseansicht.
%% Setzt den Schalter \ifkorrekturansicht voraus (gesetzt in den
%% einbindenden Dateien latex-korrekturansicht-abspann.tex bzw.
%% latex-leseansicht-abspann.tex).
%% ---------------------------------------------------------------

\normalsize

% Das esempio-Environment wird nur in der Leseansicht benötigt
\ifkorrekturansicht\else
\newenvironment{esempio}[3]%
{
    \vspace{1.5ex}
    \rlap{\underline{#1}}
    \par
    \setlength{\parindent}{0cm}
    \nopagebreak
    \leftskip=#2cm
    \rightskip=#3cm
}
{
    \par
}
\fi

\doendnotes{C}
\bigskip
\vfill

\clearpage

\footnotesize

\ifkorrekturansicht
  \lohead{\textsc{register}}
\fi

% theindex-Environment neu definieren ohne reledmac
\makeatletter
\renewenvironment{theindex}{%
  \ifkorrekturansicht
    \section*{\indexname}%
  \else
    \subsubsection*{Index der erwähnten Entitäten}%
  \fi
  \setlength{\parindent}{0pt}%
  \setlength{\parskip}{0pt plus 0.3pt}%
  \let\item\@idxitem
}{%
  \ifkorrekturansicht\clearpage\fi
}
\makeatother

\IfFileExists{\jobname-pw.ind}{\input{\jobname-pw.ind}}{}

% Quellenangabe nur in der Leseansicht
\ifkorrekturansicht\else
% Fallback-Definitionen, falls die .tex-Datei \titel etc. nicht gesetzt hat
\providecommand{\titel}{}
\providecommand{\editorInnen}{}
\providecommand{\dateiname}{\jobname}

\vspace{3cm}

\vfill

\footnotesize
\textsc{Quelle}: \titel. Herausgegeben von {\editorInnen}. In: \emph{Arthur Schnitzler: Briefwechsel mit Autorinnen und Autoren}.
 Digitale Edition, https://schnitzler-briefe.acdh.oeaw.ac.at/{\dateiname}.html (Stand \today)
\fi

\end{document}


