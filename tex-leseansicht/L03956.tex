%% latex-leseansicht-vorspann.tex
%% Vorspann für die Leseansicht.
%% Lädt die gemeinsame Datei latex-vorspann.tex mit nicht gesetztem Schalter.

\newif\ifkorrekturansicht
\korrekturansichtfalse

\input{../tex-inputs/latex-vorspann}


\section[Arthur Schnitzler an Berta Zuckerkandl, 1. 12. 1924]{L03956 Arthur Schnitzler an Berta Zuckerkandl, 1. 12. 1924}
\nopagebreak\mylabel{L03956v}
\rehead{ }\normalsize\beginnumbering\briefempfaengerindex{Zuckerkandl, Berta@\textsc{Zuckerkandl, Berta}!zzzSchnitzler, Arthur@\emph{von Arthur Schnitzler}!1924-12-011@{1. 12. 1924}|(be}
\toendnotes[C]{\smallbreak\pagebreak[2]}
\correspDesc{Versand  durch Arthur Schnitzler am 1. 12. 1924 in Wien
\newline{}Erhalt  durch Berta Zuckerkandl im Zeitraum [2. 12. 1924
                  – 6. 12. 1924?] in Paris}\toendnotes[C]{\smallbreak}
\Standort{DLA, HS.1985.1.2282.}
\physDesc{Brief, Durchschlag, 1 Blatt, 2 Seiten, 2614 Zeichen
\newline{}Schreibmaschine
\newline{}Handschrift: roter Buntstift, lateinische Kurrent (\noindent{}beschriftet: »\uline{Zuckerkandl}«, fünfzehn Unterstreichungen)}\toendnotes[C]{\smallbreak}
\pstart
           \raggedleft{}{\pb}1. 12. 1924.\pend
           
\pstart{}Liebe und verehrte Frau Hofrätin.\pend\vspace{0.5em}
\pstart
           Ich sende Ihnen heute den \label{K_L03956-1v}\edtext{Beginn einer
                  französischen\oindex{Frankreich@\textbf{Frankreich}|pw}{ }Uebersetzung\pwindex{Schnitzler, Arthur 15. 5. 1862 Wien – 21. 10. 1931 ebd.@\textsc{Schnitzler, Arthur} (15. 5. 1862 Wien – 21. 10. 1931 ebd.), \emph{Schriftsteller, Mediziner}!?? [Beginn einer französischen Übersetzung von Fräulein Else]@\strich\emph{?? [Beginn einer französischen Übersetzung von Fräulein Else]}|pwv}}{\lemma{\textnormal{\emph{Beginn … Uebersetzung}}}\Cendnote{\textnormal{Die Beilage ist nicht überliefert, um
                  welche Übersetzung es sich handelt, ist nicht ermittelt.}}}\label{K_L03956-1} von »Fräulein Else\pwindex{Schnitzler, Arthur 15. 5. 1862 Wien – 21. 10. 1931 ebd.@\textsc{Schnitzler, Arthur} (15. 5. 1862 Wien – 21. 10. 1931 ebd.), \emph{Schriftsteller, Mediziner}!Fräulein Else@\strich\emph{Fräulein Else}|pw}« ein. Es wäre mir höchst
               interessant zu wissen, wie Sie sie finden, ob Sie sich für so gelungen halten, dass
               man sie eventuell Géraldy\pwindex{Géraldy, Paul 6.\,3.\,1885 Paris – 9.\,3.\,1983 Neuilly-sur-Seine@\textsc{Géraldy, Paul} (6.\,3.\,1885 Paris – 9.\,3.\,1983 Neuilly-sur-Seine), \emph{Schriftsteller}|pw} zur Begutachtung
               übergeben und eventuell daran denken könnte dem Uebersetzer\pwindex{?? [Verfasser einer französischen Übersetzung von Fräulein Else] @\textsc{?? [Verfasser einer französischen Übersetzung von Fräulein Else]}|pw} eine Autorisation für die ganze Novelle\pwindex{Schnitzler, Arthur 15. 5. 1862 Wien – 21. 10. 1931 ebd.@\textsc{Schnitzler, Arthur} (15. 5. 1862 Wien – 21. 10. 1931 ebd.), \emph{Schriftsteller, Mediziner}!Fräulein Else@\strich\emph{Fräulein Else}|pwv} zu erteilen. So könnte man dann das Werk\pwindex{Schnitzler, Arthur 15. 5. 1862 Wien – 21. 10. 1931 ebd.@\textsc{Schnitzler, Arthur} (15. 5. 1862 Wien – 21. 10. 1931 ebd.), \emph{Schriftsteller, Mediziner}!?? [Beginn einer französischen Übersetzung von Fräulein Else]@\strich\emph{?? [Beginn einer französischen Übersetzung von Fräulein Else]}|pwv} gleich einem französischen\oindex{Frankreich@\textbf{Frankreich}|pw} Verleger übergeben. Aber es ist wohl
               wahrscheinlich, dass jeder Verleger sich selbst seinen Uebersetzer zu engagieren
               wünscht. Jedenfalls bin ich auf Ihre Antwort sehr gespannt.\pend
           
\pstart
           Eben heute habe ich von Mme. Bianquis\pwindex{Bianquis, Geneviève 19.\,9.\,1886 Rouen – 24.\,3.\,1972 Antony@\textsc{Bianquis, Geneviève} (19.\,9.\,1886 Rouen – 24.\,3.\,1972 Antony), \emph{Übersetzerin, Literaturhistorikerin}|pw} neuerlich ein \label{K_L03956-2v}\edtext{Schreiben}{\lemma{\textnormal{\emph{Schreiben}}}\Cendnote{\textnormal{Das Schreiben von Geneviève Bianquis\pwindex{Bianquis, Geneviève 19.\,9.\,1886 Rouen – 24.\,3.\,1972 Antony@\textsc{Bianquis, Geneviève} (19.\,9.\,1886 Rouen – 24.\,3.\,1972 Antony), \emph{Übersetzerin, Literaturhistorikerin}|pwk} ist nicht überliefert,
                  jedoch Schnitzlers Antwort darauf: Arthur Schnitzler an Geneviève Bianquis\pwindex{Bianquis, Geneviève 19.\,9.\,1886 Rouen – 24.\,3.\,1972 Antony@\textsc{Bianquis, Geneviève} (19.\,9.\,1886 Rouen – 24.\,3.\,1972 Antony), \emph{Übersetzerin, Literaturhistorikerin}|pwk}, 1. 12. 1924, \emph{Deutsches Literaturarchiv Marbach},
                     HS.1985.1.387,4.}}}\label{K_L03956-2} erhalt{[}e{]}n. Sie teilt mir
               mit, dass die Revue de Paris\orgindex{Revue de Paris@La Revue de Paris|pw} nicht den »Einsamen Weg\pwindex{Schnitzler, Arthur 15. 5. 1862 Wien – 21. 10. 1931 ebd.@\textsc{Schnitzler, Arthur} (15. 5. 1862 Wien – 21. 10. 1931 ebd.), \emph{Schriftsteller, Mediziner}!einsame Weg. Schauspiel in fünf Akten@\strich\emph{Der einsame Weg. Schauspiel in fünf Akten}|pw}\pwindex{Schnitzler, Arthur 15. 5. 1862 Wien – 21. 10. 1931 ebd.@\textsc{Schnitzler, Arthur} (15. 5. 1862 Wien – 21. 10. 1931 ebd.), \emph{Schriftsteller, Mediziner}!?? [französische Übersetzung von Der einsame Weg]@\strich\emph{?? [französische Übersetzung von Der einsame Weg]}|pw}«, sondern vorläufig das
                  »Bachusfest\pwindex{Schnitzler, Arthur 15. 5. 1862 Wien – 21. 10. 1931 ebd.@\textsc{Schnitzler, Arthur} (15. 5. 1862 Wien – 21. 10. 1931 ebd.), \emph{Schriftsteller, Mediziner}!Bacchusfest@\strich\emph{Das Bacchusfest}|pw}\pwindex{Schnitzler, Arthur 15. 5. 1862 Wien – 21. 10. 1931 ebd.@\textsc{Schnitzler, Arthur} (15. 5. 1862 Wien – 21. 10. 1931 ebd.), \emph{Schriftsteller, Mediziner}!?? [französische Übersetzung von Das Bacchusfest]@\strich\emph{?? [französische Übersetzung von Das Bacchusfest]}|pw}« drucken will (was ja
               allerdings betrachtlich bequemer ist). Man offeriert mir 20 Francs per Seite, die ich
               mit ihr zu \label{T_L03956-1v}\edtext{teilen}{\lemma{\textnormal{\emph{teilen}}}\Cendnote{\textnormal{Im Typoscript steht: »teillein«.}}}\label{T_L03956-1} hätte.\pend
           
\pstart
           Da Maurice Rémon\pwindex{Rémon, Maurice 27.\,11.\,1861 Paris – 20.\,6.\,1945 Mérignac@\textsc{Rémon, Maurice} (27.\,11.\,1861 Paris – 20.\,6.\,1945 Mérignac), \emph{Übersetzer}|pw} die »Stunde des Erkennens\pwindex{Schnitzler, Arthur 15. 5. 1862 Wien – 21. 10. 1931 ebd.@\textsc{Schnitzler, Arthur} (15. 5. 1862 Wien – 21. 10. 1931 ebd.), \emph{Schriftsteller, Mediziner}!Stunde des Erkennens@\strich\emph{Stunde des Erkennens}|pw}« schon übersetzt\pwindex{Schnitzler, Arthur 15. 5. 1862 Wien – 21. 10. 1931 ebd.@\textsc{Schnitzler, Arthur} (15. 5. 1862 Wien – 21. 10. 1931 ebd.), \emph{Schriftsteller, Mediziner}!?? [französische Übersetzung von Stunde des Erkennens]@\strich\emph{?? [französische Übersetzung von Stunde des Erkennens]}|pwv} hat und Mme. Bianquis\pwindex{Bianquis, Geneviève 19.\,9.\,1886 Rouen – 24.\,3.\,1972 Antony@\textsc{Bianquis, Geneviève} (19.\,9.\,1886 Rouen – 24.\,3.\,1972 Antony), \emph{Übersetzerin, Literaturhistorikerin}|pw} ausser dem »Bachusfest\pwindex{Schnitzler, Arthur 15. 5. 1862 Wien – 21. 10. 1931 ebd.@\textsc{Schnitzler, Arthur} (15. 5. 1862 Wien – 21. 10. 1931 ebd.), \emph{Schriftsteller, Mediziner}!Bacchusfest@\strich\emph{Das Bacchusfest}|pw}\pwindex{Schnitzler, Arthur 15. 5. 1862 Wien – 21. 10. 1931 ebd.@\textsc{Schnitzler, Arthur} (15. 5. 1862 Wien – 21. 10. 1931 ebd.), \emph{Schriftsteller, Mediziner}!?? [französische Übersetzung von Das Bacchusfest]@\strich\emph{?? [französische Übersetzung von Das Bacchusfest]}|pw}« die »Grosse
                  Szene\pwindex{Schnitzler, Arthur 15. 5. 1862 Wien – 21. 10. 1931 ebd.@\textsc{Schnitzler, Arthur} (15. 5. 1862 Wien – 21. 10. 1931 ebd.), \emph{Schriftsteller, Mediziner}!Große Szene@\strich\emph{Große Szene}|pw}\pwindex{Schnitzler, Arthur 15. 5. 1862 Wien – 21. 10. 1931 ebd.@\textsc{Schnitzler, Arthur} (15. 5. 1862 Wien – 21. 10. 1931 ebd.), \emph{Schriftsteller, Mediziner}!?? [französische Übersetzung von Große Szene]@\strich\emph{?? [französische Übersetzung von Große Szene]}|pw}« schon fertig hat, liegt eigentlich der ganze Zyklus »Komödie der Worte\pwindex{Schnitzler, Arthur 15. 5. 1862 Wien – 21. 10. 1931 ebd.@\textsc{Schnitzler, Arthur} (15. 5. 1862 Wien – 21. 10. 1931 ebd.), \emph{Schriftsteller, Mediziner}!Komödie der Worte. Drei Einakter@\strich\emph{Komödie der Worte. Drei Einakter}|pw}« in französischer\oindex{Frankreich@\textbf{Frankreich}|pw}{ }Uebersetzung\pwindex{Schnitzler, Arthur 15. 5. 1862 Wien – 21. 10. 1931 ebd.@\textsc{Schnitzler, Arthur} (15. 5. 1862 Wien – 21. 10. 1931 ebd.), \emph{Schriftsteller, Mediziner}!?? [französische Übersetzung von Stunde des Erkennens]@\strich\emph{?? [französische Übersetzung von Stunde des Erkennens]}|pwv}\pwindex{Schnitzler, Arthur 15. 5. 1862 Wien – 21. 10. 1931 ebd.@\textsc{Schnitzler, Arthur} (15. 5. 1862 Wien – 21. 10. 1931 ebd.), \emph{Schriftsteller, Mediziner}!?? [französische Übersetzung von Das Bacchusfest]@\strich\emph{?? [französische Übersetzung von Das Bacchusfest]}|pwv}\pwindex{Schnitzler, Arthur 15. 5. 1862 Wien – 21. 10. 1931 ebd.@\textsc{Schnitzler, Arthur} (15. 5. 1862 Wien – 21. 10. 1931 ebd.), \emph{Schriftsteller, Mediziner}!?? [französische Übersetzung von Große Szene]@\strich\emph{?? [französische Übersetzung von Große Szene]}|pwv} vor.\pend
           
\pstart
           Es wundert mich, dass Gèraldy\pwindex{Géraldy, Paul 6.\,3.\,1885 Paris – 9.\,3.\,1983 Neuilly-sur-Seine@\textsc{Géraldy, Paul} (6.\,3.\,1885 Paris – 9.\,3.\,1983 Neuilly-sur-Seine), \emph{Schriftsteller}|pw} neuerdings eine
               Frage wegen des Einakters für Fabre\pwindex{Fabre, Émile 24.\,3.\,1869 Metz – 25.\,9.\,1955 Paris@\textsc{Fabre, Émile} (24.\,3.\,1869 Metz – 25.\,9.\,1955 Paris), \emph{Schriftsteller, Journalist, Theaterleiter}|pw} stellt.
                  \label{K_L03956-3v}\edtext{Ich schrieb ihm}{\lemma{\textnormal{\emph{Ich schrieb ihm}}}\Cendnote{\textnormal{Arthur Schnitzler an Paul Géraldy\pwindex{Géraldy, Paul 6.\,3.\,1885 Paris – 9.\,3.\,1983 Neuilly-sur-Seine@\textsc{Géraldy, Paul} (6.\,3.\,1885 Paris – 9.\,3.\,1983 Neuilly-sur-Seine), \emph{Schriftsteller}|pwk}, 31. 7. 1924, \emph{Deutsches Literaturarchiv Marbach},
                     HS.1985.1.811,8.}}}\label{K_L03956-3} schon im Sommer wegen des »Grünen Kakadu\pwindex{Schnitzler, Arthur 15. 5. 1862 Wien – 21. 10. 1931 ebd.@\textsc{Schnitzler, Arthur} (15. 5. 1862 Wien – 21. 10. 1931 ebd.), \emph{Schriftsteller, Mediziner}!grüne Kakadu. Groteske in einem Akt@\strich\emph{Der grüne Kakadu. Groteske in einem Akt}|pw}«, Sie, liebe Frau Hofrätin, waren
               ja mit mir der Ansicht, dass abgesehen von der »Gr{[}o{]}ssen Szene\pwindex{Schnitzler, Arthur 15. 5. 1862 Wien – 21. 10. 1931 ebd.@\textsc{Schnitzler, Arthur} (15. 5. 1862 Wien – 21. 10. 1931 ebd.), \emph{Schriftsteller, Mediziner}!Große Szene@\strich\emph{Große Szene}|pw}« der »Kakadu\pwindex{Schnitzler, Arthur 15. 5. 1862 Wien – 21. 10. 1931 ebd.@\textsc{Schnitzler, Arthur} (15. 5. 1862 Wien – 21. 10. 1931 ebd.), \emph{Schriftsteller, Mediziner}!grüne Kakadu. Groteske in einem Akt@\strich\emph{Der grüne Kakadu. Groteske in einem Akt}|pw}« {\pb}wohl als der repräsentativste meiner
               Einakter für Paris\oindex{Paris@\textbf{Paris}, \emph{Hauptstadt}|pw} in Betracht käme. Wie nun die
               Dinge stehen, wäre ja immerhin auch das »Bachusfest\pwindex{Schnitzler, Arthur 15. 5. 1862 Wien – 21. 10. 1931 ebd.@\textsc{Schnitzler, Arthur} (15. 5. 1862 Wien – 21. 10. 1931 ebd.), \emph{Schriftsteller, Mediziner}!Bacchusfest@\strich\emph{Das Bacchusfest}|pw}« zu erwägen, obzwar es mir wenig sympathisch wäre gerade mit
               diesem, nicht eben bedeutenden Stück\pwindex{Schnitzler, Arthur 15. 5. 1862 Wien – 21. 10. 1931 ebd.@\textsc{Schnitzler, Arthur} (15. 5. 1862 Wien – 21. 10. 1931 ebd.), \emph{Schriftsteller, Mediziner}!Bacchusfest@\strich\emph{Das Bacchusfest}|pwv} am Theatre français\orgindex{Comédie-Française@Comédie-Française|pw} zu
                  erscheinen{[}.{]} Auch von meinen andern Einaktern schiene mir
               keiner recht geeignet mich im Theatre francais\oindex{Comédie française@\textbf{Comédie française}, \emph{Theater}|pw}
               einzuführen, eventuell könnte man an die »Frau mit
                  dem Dolch\pwindex{Schnitzler, Arthur 15. 5. 1862 Wien – 21. 10. 1931 ebd.@\textsc{Schnitzler, Arthur} (15. 5. 1862 Wien – 21. 10. 1931 ebd.), \emph{Schriftsteller, Mediziner}!Frau mit dem Dolche@\strich\emph{Die Frau mit dem Dolche}|pw}« denken.\pend
           
\pstart
           Von Grasset\pwindex{Grasset, Bernard 6.\,3.\,1881 Chambéry – 20.\,10.\,1955 Paris@\textsc{Grasset, Bernard} (6.\,3.\,1881 Chambéry – 20.\,10.\,1955 Paris), \emph{Verleger}|pw} habe ich nach wie vor nichts
               weiter gehört. Er hat mir auch noch nichts über die Auswahl der Novellen geschrieben,
               die er herauszugeben gedenkt und über die wir uns wohl doch erst einigen müssten.\pend
           
\pstart
           Vielleicht ist manches von diesem Briefe bereits überholt, wenn er eintrifft. Ich
               danke Ihnen heute noch ganz besonders dafür, dass Sie Mademoiselle Bianquis\pwindex{Bianquis, Geneviève 19.\,9.\,1886 Rouen – 24.\,3.\,1972 Antony@\textsc{Bianquis, Geneviève} (19.\,9.\,1886 Rouen – 24.\,3.\,1972 Antony), \emph{Übersetzerin, Literaturhistorikerin}|pw} empfangen und mit ihr über meine Angelegenheiten
               konferiert haben; ebenso wie für Ihren \label{K_L03956-4v}\edtext{liebenswürdigen Brief}{\lemma{\textnormal{\emph{liebenswürdigen Brief}}}\Cendnote{\textnormal{nicht
                  überliefert}}}\label{K_L03956-4}, und alle Ihre freundlichen Bemühungen.\pend
           
\pstart
           Ich freue mich schon sehr \label{K_L03956-5v}\edtext{Lenormands\pwindex{Lenormand, Henri-René 3.\,5.\,1882 Paris – 16.\,2.\,1951 ebd.@\textsc{Lenormand, Henri-René} (3.\,5.\,1882 Paris – 16.\,2.\,1951 ebd.), \emph{Schriftsteller}|pw} neues Stück\pwindex{Lenormand, Henri-René 3.\,5.\,1882 Paris – 16.\,2.\,1951 ebd.@\textsc{Lenormand, Henri-René} (3.\,5.\,1882 Paris – 16.\,2.\,1951 ebd.), \emph{Schriftsteller}!Stimmen aus dem Dunkel. Ein Don Juan-Spiel in 16 Bildern@\strich\emph{Stimmen aus dem Dunkel. Ein Don Juan-Spiel in 16 Bildern}|pwv} und ihn selbst kennen zu lernen}{\lemma{\textnormal{\emph{Lenormands … lernen}}}\Cendnote{\textnormal{Henri-René Lenormands\pwindex{Lenormand, Henri-René 3.\,5.\,1882 Paris – 16.\,2.\,1951 ebd.@\textsc{Lenormand, Henri-René} (3.\,5.\,1882 Paris – 16.\,2.\,1951 ebd.), \emph{Schriftsteller}|pwk} Drama \emph{Stimmen aus dem Dunkel}\pwindex{Lenormand, Henri-René 3.\,5.\,1882 Paris – 16.\,2.\,1951 ebd.@\textsc{Lenormand, Henri-René} (3.\,5.\,1882 Paris – 16.\,2.\,1951 ebd.), \emph{Schriftsteller}!Stimmen aus dem Dunkel. Ein Don Juan-Spiel in 16 Bildern@\strich\emph{Stimmen aus dem Dunkel. Ein Don Juan-Spiel in 16 Bildern}|pwk} in der Übersetzung von Berta Zuckerkandl\pwindex{Zuckerkandl, Berta 13.\,4.\,1864 Wien – 16.\,10.\,1945 Paris@\textsc{Zuckerkandl, Berta} (13.\,4.\,1864 Wien – 16.\,10.\,1945 Paris), \emph{Schriftstellerin, Journalistin, Übersetzerin}|pwk} wurde in dieser Saison am
                  \emph{Burgtheater}\orgindex{Burgtheater@Burgtheater|pwk} gegeben. Schnitzler besuchte die Premiere\eventindex{Burgtheater@\textbf{Burgtheater}!Premiere von Stimmen aus dem Dunkel, 3.1.1925@Premiere von Stimmen aus dem Dunkel, 3.1.1925|pwkv} am 3. 1. 1925. Zuvor las er jedoch das Stück\pwindex{Lenormand, Henri-René 3.\,5.\,1882 Paris – 16.\,2.\,1951 ebd.@\textsc{Lenormand, Henri-René} (3.\,5.\,1882 Paris – 16.\,2.\,1951 ebd.), \emph{Schriftsteller}!Stimmen aus dem Dunkel. Ein Don Juan-Spiel in 16 Bildern@\strich\emph{Stimmen aus dem Dunkel. Ein Don Juan-Spiel in 16 Bildern}|pwkv}, wenn
                  auch »ohne Zustimmung«, vgl. A. S.: \emph{Tagebuch}, 24. 12. 1924, und lernte den Schriftsteller\pwindex{Lenormand, Henri-René 3.\,5.\,1882 Paris – 16.\,2.\,1951 ebd.@\textsc{Lenormand, Henri-René} (3.\,5.\,1882 Paris – 16.\,2.\,1951 ebd.), \emph{Schriftsteller}|pwkv} bei Zuckerkandl\pwindex{Zuckerkandl, Berta 13.\,4.\,1864 Wien – 16.\,10.\,1945 Paris@\textsc{Zuckerkandl, Berta} (13.\,4.\,1864 Wien – 16.\,10.\,1945 Paris), \emph{Schriftstellerin, Journalistin, Übersetzerin}|pwk} persönlich kennen, vgl. A. S.: \emph{Tagebuch}, 28. 12. 1924.}}}\label{K_L03956-5}, ganz
               besonders aber darauf, Sie selbst, liebe und verehrte Freundin, recht bald und
               hoffentlich bei gutem Befinden und in leidlicher Stimmung \label{K_L03956-6v}\edtext{wiederzusehen}{\lemma{\textnormal{\emph{wiederzusehen}}}\Cendnote{\textnormal{Vgl. A. S.: \emph{Tagebuch}, 23. 12. 1924.}}}\label{K_L03956-6}.\pend
           
\pstart
           Herzlichst{\\[\baselineskip]} Ihr\pend
           \leftskip=0em{}{\vspace{1\baselineskip}}
\pstart
           \noindent{}Frau Hofrätin Bertha Zuckerkandl,{\\}Paris\oindex{Paris@\textbf{Paris}, \emph{Hauptstadt}|pw}.\pend
           
\pstart
           1 Beilage.\pend
           \selectlanguage{ngerman}\endnumbering\briefempfaengerindex{Zuckerkandl, Berta@\textsc{Zuckerkandl, Berta}!zzzSchnitzler, Arthur@\emph{von Arthur Schnitzler}!1924-12-011@{1. 12. 1924}|)be}\mylabel{L03956h}
\begin{anhang}
\end{anhang}\newcommand{\dateiname}{L03956}\newcommand{\titel}{Arthur Schnitzler an Berta Zuckerkandl, 1. 12. 1924}\newcommand{\editorInnen}{Herausgegeben von Jahnke, SelmaMüller, Martin Anton}%% latex-leseansicht-abspann.tex
%% Abspann für die Leseansicht.
%% Der Schalter \ifkorrekturansicht ist bereits durch den Vorspann gesetzt.

%% latex-abspann.tex
%% Gemeinsamer Abspann für Korrekturansicht und Leseansicht.
%% Setzt den Schalter \ifkorrekturansicht voraus (gesetzt in den
%% einbindenden Dateien latex-korrekturansicht-abspann.tex bzw.
%% latex-leseansicht-abspann.tex).
%% ---------------------------------------------------------------

\normalsize

% Das esempio-Environment wird nur in der Leseansicht benötigt
\ifkorrekturansicht\else
\newenvironment{esempio}[3]%
{
    \vspace{1.5ex}
    \rlap{\underline{#1}}
    \par
    \setlength{\parindent}{0cm}
    \nopagebreak
    \leftskip=#2cm
    \rightskip=#3cm
}
{
    \par
}
\fi

\doendnotes{C}
\bigskip
\vfill

\clearpage

\footnotesize

\ifkorrekturansicht
  \lohead{\textsc{register}}
\fi

% theindex-Environment neu definieren ohne reledmac
\makeatletter
\renewenvironment{theindex}{%
  \ifkorrekturansicht
    \section*{\indexname}%
  \else
    \subsubsection*{Index der erwähnten Entitäten}%
  \fi
  \setlength{\parindent}{0pt}%
  \setlength{\parskip}{0pt plus 0.3pt}%
  \let\item\@idxitem
}{%
  \ifkorrekturansicht\clearpage\fi
}
\makeatother

\IfFileExists{\jobname-pw.ind}{\input{\jobname-pw.ind}}{}

% Quellenangabe nur in der Leseansicht
\ifkorrekturansicht\else
% Fallback-Definitionen, falls die .tex-Datei \titel etc. nicht gesetzt hat
\providecommand{\titel}{}
\providecommand{\editorInnen}{}
\providecommand{\dateiname}{\jobname}

\vspace{3cm}

\vfill

\footnotesize
\textsc{Quelle}: \titel. Herausgegeben von {\editorInnen}. In: \emph{Arthur Schnitzler: Briefwechsel mit Autorinnen und Autoren}.
 Digitale Edition, https://schnitzler-briefe.acdh.oeaw.ac.at/{\dateiname}.html (Stand \today)
\fi

\end{document}


