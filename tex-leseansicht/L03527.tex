%% latex-leseansicht-vorspann.tex
%% Vorspann für die Leseansicht.
%% Lädt die gemeinsame Datei latex-vorspann.tex mit nicht gesetztem Schalter.

\newif\ifkorrekturansicht
\korrekturansichtfalse

\input{../tex-inputs/latex-vorspann}

\begin{center}
            \textcolor{red}{ENTWURF, NICHT FERTIG KORRIGIERT}
                      \end{center}
            
         
         \renewcommand{\erwaehntePersonen}{Personen: Hermann Bahr, Olga Schnitzler}
         \renewcommand{\erwaehnteOrte}{Orte: Berlin, Dessauer Straße, Wien, Wörthersee}
         \renewcommand{\erwaehnteWerke}{Werke: Neue Freie Presse}
               \section[ Paul Goldmann an Olga Gussmann, 10. 5. {[}1901{]}]{ Paul Goldmann an Olga Gussmann, 10. 5. {[}1901{]}}\nopagebreak\mylabel{v}\rehead{ }\begin{ledgroupsized}[t]{13cm}\normalsize\beginnumbering \toendnotes[C]{\smallbreak\pagebreak[2]} \Standort{DLA, A:Schnitzler, HS.NZ85.1.5247.}
\physDesc{Brief, 1 Blatt, 4 Seiten
\newline{}Handschrift: blaue Tinte, deutsche Kurrent\newline{}Ordnung: mit Bleistift von Arthur
                                    Schnitzler\pwindex{Schnitzler, Arthur 15.05.1862 – 21.10.1931@\textsc{Schnitzler, Arthur} (15.05.1862 – 21.10.1931), \emph{Schriftsteller, Mediziner}|pw} das Jahr »1901.« vermerkt }\toendnotes[C]{\smallbreak}\pstart
           \noindent{}\raggedleft{}{\pb}\textcolor{gray}{\textbf{DESSAUERSTRASSE 19\oindex{Dessauer Strasse@\textbf{Dessauer Straße}|pw}}}\pend
           \pstart
           Berlin\oindex{Berlin@\textbf{Berlin}|pw}, 10. Mai.\pend
           \pstart\center{}Liebes Fräulein \textsc{Olga},\pend\pstart
           Haben Sie vielen herzlichen Dank für das ſchöne \label{K_L03527-1v}\edtext{Bild}{\lemma{\textnormal{\emph{Bild}}}\Cendnote{\textnormal{}}}\label{K_L03527-1h}! Es ſoll mir ein lieber Beſitz ſein. Diese Wien\oindex{Wien@\textbf{Wien}|pw}er Photographen ſind doch mehr Künſtler. Man bekommt nach
               dieſem Bilde wirklich ein lebendige Vorſtellung von Ihnen, und Ihre Perſönlichkeit
               iſt ſehr reizvoll darin ausgedrückt.\pend
           \pstart
           Mit Dank ſende ich Ihnen die \label{K_L03527-2v}\edtext{Zeitungausſchnitte\textcolor{red}{\textsuperscript{\textbf{KEY}}}}{\lemma{\textnormal{\emph{Zeitungausſchnitte}}}\Cendnote{\textnormal{}}}\label{K_L03527-2h} zurück. \textsc{Bahr\pwindex{Bahr, Hermann 19.07.1863 – 15.01.1934@\textsc{Bahr, Hermann} (19.07.1863 – 15.01.1934), \emph{Schriftsteller, Kritiker}|pw}} hat, {\pb}wie gewöhnlich, \substVorne{}\textsuperscript{Blech}\substDazwischen{}Blech\substHinten{}{ }geſchrieben\textcolor{red}{\textsuperscript{\textbf{KEY}}}. Das ſpürt man heraus, wenn man auch die Vorſtellung\textcolor{red}{\textsuperscript{\textbf{KEY}}} ſelbſt nicht geſehen hat. Ich freue
               mich, daß Alles gut gegangen iſt. Auf die N. Fr.
                  Pr.\pwindex{Neue Freie Presse1864 – 1939@\emph{Neue Freie Presse} {[}1864 – 1939{]}|pw} bin ich neugierig. Oder iſt das \label{K_L03527-3v}\edtext{Referat\textcolor{red}{\textsuperscript{\textbf{KEY}}}}{\lemma{\textnormal{\emph{Referat}}}\Cendnote{\textnormal{}}}\label{K_L03527-3h} vielleicht ſchon
               erſchienen und habe ich es überſehen?\pend
           \pstart
           Ob ich Sie \label{K_L03527-4v}\edtext{im Sommer wiederſehen}{\lemma{\textnormal{\emph{im Sommer wiederſehen}}}\Cendnote{\textnormal{}}}\label{K_L03527-4h}
               werde, weiß ich noch nicht. Jedenfalls kann ich nur im Auguſt auf Urlaub gehen, {\pb}und auch dann
               will ich nicht herumreiſen, ſondern irgendwo feſtſitzen, etwa am Wörtherſee\oindex{Woerthersee@\textbf{Wörthersee}|pw}. Ich bat \textsc{Arthur\pwindex{Schnitzler, Arthur 15.05.1862 – 21.10.1931@\textsc{Schnitzler, Arthur} (15.05.1862 – 21.10.1931), \emph{Schriftsteller, Mediziner}|pw}}{ }\strikeout{darum} deshalb,
               daß er mit Ihnen im Auguſt an den Wörtherſee\oindex{Woerthersee@\textbf{Wörthersee}|pw} kommen möge. Wenn das nicht geht, ſehen wir uns
               hoffentlich auf meiner Rückreiſe in Wien\oindex{Wien@\textbf{Wien}|pw}.\pend
           \pstart
           Sie ſelbſt werden mit \textsc{Arthur\pwindex{Schnitzler, Arthur 15.05.1862 – 21.10.1931@\textsc{Schnitzler, Arthur} (15.05.1862 – 21.10.1931), \emph{Schriftsteller, Mediziner}|pw}} gewiß einige ſchöne {\pb}Sommermonate verleben.
               Laſſen Sie alle trüben Gedanken zu Hauſe und genießen Sie die ſchöne Welt, die ja
               überhaupt nur dann wirklich ſchön iſt, wenn man Jemanden neben ſich hat, den man \strikeout{\textcolor{gray}{l}} liebt. Auch der Naturgenuß kann nur aus dem Herzen kommen; und das Herz bleibt
               ungerührt, wenn nicht eine Liebe es bewegt. Es gibt keine ſchönen Landſchaften (ohne
               Liebe nämlich).\pend
           \pstart
           Seien Sie herzlichſt gegrüßt von Ihrem ergebenen {\\[\baselineskip]}\spacefill\mbox{Dr. Paul
                  Goldmann.}\pend
           \leftskip=0em{}
         
         \endnumbering\mylabel{h}\end{ledgroupsized}\begin{anhang}\end{anhang}\newcommand{\dateiname}{L03527}\newcommand{\titel}{Paul Goldmann an Olga Gussmann, 10. 5. [1901]}\newcommand{\editorInnen}{Martin Anton Müller und Laura Untner}%% latex-leseansicht-abspann.tex
%% Abspann für die Leseansicht.
%% Der Schalter \ifkorrekturansicht ist bereits durch den Vorspann gesetzt.

%% latex-abspann.tex
%% Gemeinsamer Abspann für Korrekturansicht und Leseansicht.
%% Setzt den Schalter \ifkorrekturansicht voraus (gesetzt in den
%% einbindenden Dateien latex-korrekturansicht-abspann.tex bzw.
%% latex-leseansicht-abspann.tex).
%% ---------------------------------------------------------------

\normalsize

% Das esempio-Environment wird nur in der Leseansicht benötigt
\ifkorrekturansicht\else
\newenvironment{esempio}[3]%
{
    \vspace{1.5ex}
    \rlap{\underline{#1}}
    \par
    \setlength{\parindent}{0cm}
    \nopagebreak
    \leftskip=#2cm
    \rightskip=#3cm
}
{
    \par
}
\fi

\doendnotes{C}
\bigskip
\vfill

\clearpage

\footnotesize

\ifkorrekturansicht
  \lohead{\textsc{register}}
\fi

% theindex-Environment neu definieren ohne reledmac
\makeatletter
\renewenvironment{theindex}{%
  \ifkorrekturansicht
    \section*{\indexname}%
  \else
    \subsubsection*{Index der erwähnten Entitäten}%
  \fi
  \setlength{\parindent}{0pt}%
  \setlength{\parskip}{0pt plus 0.3pt}%
  \let\item\@idxitem
}{%
  \ifkorrekturansicht\clearpage\fi
}
\makeatother

\IfFileExists{\jobname-pw.ind}{\input{\jobname-pw.ind}}{}

% Quellenangabe nur in der Leseansicht
\ifkorrekturansicht\else
% Fallback-Definitionen, falls die .tex-Datei \titel etc. nicht gesetzt hat
\providecommand{\titel}{}
\providecommand{\editorInnen}{}
\providecommand{\dateiname}{\jobname}

\vspace{3cm}

\vfill

\footnotesize
\textsc{Quelle}: \titel. Herausgegeben von {\editorInnen}. In: \emph{Arthur Schnitzler: Briefwechsel mit Autorinnen und Autoren}.
 Digitale Edition, https://schnitzler-briefe.acdh.oeaw.ac.at/{\dateiname}.html (Stand \today)
\fi

\end{document}


      