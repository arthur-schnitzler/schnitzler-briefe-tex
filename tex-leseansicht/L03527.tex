%% latex-leseansicht-vorspann.tex
%% Vorspann für die Leseansicht.
%% Lädt die gemeinsame Datei latex-vorspann.tex mit nicht gesetztem Schalter.

\newif\ifkorrekturansicht
\korrekturansichtfalse

\input{../tex-inputs/latex-vorspann}


\section[ Paul Goldmann an Olga Gussmann, 10. 5. [1901]]{L03527 Paul Goldmann an Olga Gussmann,  10. 5. [1901]}
\nopagebreak\mylabel{L03527v}
\rehead{ }\normalsize\beginnumbering\briefempfaengerindex{Schnitzler, Olga@\textsc{Schnitzler, Olga}!zzzGoldmann, Paul@\emph{von Paul Goldmann}!1901-05-102@{10. 5. [1901]}|(be}
\toendnotes[C]{\smallbreak\pagebreak[2]}
\correspDesc{Versand  durch Paul Goldmann am 10. 5. [1901] in Berlin
\newline{}Erhalt  durch Olga Gussmann im Zeitraum [11. 5. 1901
                  – 15. 5. 1901?] in Wien}\toendnotes[C]{\smallbreak}
\Standort{DLA, A:Schnitzler, HS.NZ85.1.5247.}
\physDesc{Brief, 1 Blatt, 4 Seiten, 1470 Zeichen
\newline{}Handschrift: blaue Tinte, deutsche Kurrent
\newline{}Ordnung: mit Bleistift von Arthur Schnitzler das
                                 Jahr »1901.« vermerkt }\toendnotes[C]{\smallbreak}
\pstart
           \raggedleft{}{\pb}\textcolor{gray}{\textbf{DESSAUERSTRASSE 19\oindex{Dessauer Straße@\textbf{Dessauer Straße}, \emph{Straße}|pw}}}\pend
           
\pstart
           Berlin\oindex{Berlin@\textbf{Berlin}, \emph{Hauptstadt}|pw}, 10. Mai.\pend
           
\pstart\center{}Liebes Fräulein \textsc{Olga},\pend\vspace{0.5em}
\pstart
           Haben Sie vielen herzlichen Dank für das{ }ſchöne \label{K_L03527-1v}\edtext{Bild\pwindex{Grillich, Ludwig  Wien – 21.\,5.\,1926 ebd.@\textsc{Grillich, Ludwig} ( Wien – 21.\,5.\,1926 ebd.), \emph{Fotograf}!Portraitfoto von Olga Gussmann]@\strich\emph{[Portraitfoto von Olga Gussmann]}|pwuv}}{\lemma{\textnormal{\emph{Bild}}}\Cendnote{\textnormal{Es dürfte das von Ludwig Grillich\pwindex{Grillich, Ludwig  Wien – 21.\,5.\,1926 ebd.@\textsc{Grillich, Ludwig} ( Wien – 21.\,5.\,1926 ebd.), \emph{Fotograf}|pwk} angefertige Porträtfoto\pwindex{Grillich, Ludwig  Wien – 21.\,5.\,1926 ebd.@\textsc{Grillich, Ludwig} ( Wien – 21.\,5.\,1926 ebd.), \emph{Fotograf}!Portraitfoto von Olga Gussmann]@\strich\emph{[Portraitfoto von Olga Gussmann]}|pwkv} gemeint sein (\emph{DLA}, B 1989.Q 0249).}}}\label{K_L03527-1}! Es{ }ſoll mir ein
               lieber Beſitz{ }ſein. Diese Wien\oindex{Wien@\textbf{Wien}, \emph{Verwaltungsgebiet}|pw}er Photographen{ }ſind
               doch wahre Künſtler. Man bekommt nach dieſem Bilde\pwindex{Grillich, Ludwig  Wien – 21.\,5.\,1926 ebd.@\textsc{Grillich, Ludwig} ( Wien – 21.\,5.\,1926 ebd.), \emph{Fotograf}!Portraitfoto von Olga Gussmann]@\strich\emph{[Portraitfoto von Olga Gussmann]}|pwv} wirklich eine lebendige Vorſtellung von Ihnen, und
               Ihre Perſönlichkeit iſt{ }ſehr reizvoll darin ausgedrückt.\pend
           
\pstart
           Mit Dank{ }ſende ich Ihnen die \label{K_L03527-2v}\edtext{Zeitungausſchnitt\pwindex{Bahr, Hermann 19.\,7.\,1863 Linz – 15.\,1.\,1934 München@\textsc{Bahr, Hermann} (19.\,7.\,1863 Linz – 15.\,1.\,1934 München), \emph{Schriftsteller, Kritiker}!Theater, Kunst und Literatur [Vorstellung des Konservatoriums]@\strich\emph{Theater, Kunst und Literatur [Vorstellung des Konservatoriums]}|pwv}e}{\lemma{\textnormal{\emph{Zeitungausschnitte}}}\Cendnote{\textnormal{Beilagen nicht erhalten. Bahr\pwindex{Bahr, Hermann 19.\,7.\,1863 Linz – 15.\,1.\,1934 München@\textsc{Bahr, Hermann} (19.\,7.\,1863 Linz – 15.\,1.\,1934 München), \emph{Schriftsteller, Kritiker}|pwk} hatte folgende lobende Notiz\pwindex{Bahr, Hermann 19.\,7.\,1863 Linz – 15.\,1.\,1934 München@\textsc{Bahr, Hermann} (19.\,7.\,1863 Linz – 15.\,1.\,1934 München), \emph{Schriftsteller, Kritiker}!Theater, Kunst und Literatur [Vorstellung des Konservatoriums]@\strich\emph{Theater, Kunst und Literatur [Vorstellung des Konservatoriums]}|pwkv} über die Schulaufführung von \emph{Maria Magdalena}\pwindex{\textcolor{red}{\textsuperscript{XXXX indx1}}!Maria Magdalena. Ein bürgerliches Trauerspiel in drei Akten@\strich\emph{Maria Magdalena. Ein bürgerliches Trauerspiel in drei Akten}|pwk} mit Olga Gussmann\pwindex{Schnitzler, Olga 17.\,1.\,1882 Wien – 13.\,1.\,1970 Lugano@\textsc{Schnitzler, Olga} (17.\,1.\,1882 Wien – 13.\,1.\,1970 Lugano), \emph{Schauspielerin, Sängerin}|pwk} verfasst: H. B.\pwindex{Bahr, Hermann 19.\,7.\,1863 Linz – 15.\,1.\,1934 München@\textsc{Bahr, Hermann} (19.\,7.\,1863 Linz – 15.\,1.\,1934 München), \emph{Schriftsteller, Kritiker}|pwkv} [ = Hermann Bahr\pwindex{Bahr, Hermann 19.\,7.\,1863 Linz – 15.\,1.\,1934 München@\textsc{Bahr, Hermann} (19.\,7.\,1863 Linz – 15.\,1.\,1934 München), \emph{Schriftsteller, Kritiker}|pwk}]: \emph{Theater, Kunst und Literatur}\pwindex{Bahr, Hermann 19.\,7.\,1863 Linz – 15.\,1.\,1934 München@\textsc{Bahr, Hermann} (19.\,7.\,1863 Linz – 15.\,1.\,1934 München), \emph{Schriftsteller, Kritiker}!Theater, Kunst und Literatur [Vorstellung des Konservatoriums]@\strich\emph{Theater, Kunst und Literatur [Vorstellung des Konservatoriums]}|pwk}. In: \emph{Neues Wiener Tagblatt}\pwindex{Neues Wiener Tagblatt@\emph{Neues Wiener Tagblatt}|pwk}, Jg. 35, Nr. 118, 1. 5. 1901, S. 7. Siehe XXXX Auszeichnungsfehler: Dokument L01110 nicht gefunden. Vgl. Martin Anton Müller:
                        \emph{Hermann Bahr und Arthur Schnitzler im Konvervatorium.
                        Cherchez la femme!} In: \emph{Ein Zoll Dankfest. Texte
                        für Germanistik. Konstanze Fliedl zum 60. Geburtstag}.
                     Würzburg: \emph{Königshausen {\kaufmannsund} Neumann}{ }2015, S. 43–49.}}}\label{K_L03527-2} zurück. \textsc{Bahr\pwindex{Bahr, Hermann 19.\,7.\,1863 Linz – 15.\,1.\,1934 München@\textsc{Bahr, Hermann} (19.\,7.\,1863 Linz – 15.\,1.\,1934 München), \emph{Schriftsteller, Kritiker}|pw}} hat, {\pb}wie gewöhnlich, \substVorne{}\textsuperscript{Blech}\substDazwischen{}Blech\substHinten{}{ }geſchrieben\pwindex{Bahr, Hermann 19.\,7.\,1863 Linz – 15.\,1.\,1934 München@\textsc{Bahr, Hermann} (19.\,7.\,1863 Linz – 15.\,1.\,1934 München), \emph{Schriftsteller, Kritiker}!Theater, Kunst und Literatur [Vorstellung des Konservatoriums]@\strich\emph{Theater, Kunst und Literatur [Vorstellung des Konservatoriums]}|pwv}. Das{ }ſpürt man
               heraus, wenn man auch die Vorſtellung\pwindex{\textcolor{red}{\textsuperscript{XXXX indx1}}!Maria Magdalena. Ein bürgerliches Trauerspiel in drei Akten@\strich\emph{Maria Magdalena. Ein bürgerliches Trauerspiel in drei Akten}|pwv}{ }ſelbſt nicht geſehen hat. Ich freue mich, daß Alles gut gegangen
               iſt. Auf die N. Fr. Pr.\pwindex{Neue Freie Presse@\emph{Neue Freie Presse}|pw} bin ich neugierig. Oder
               iſt das \label{K_L03527-3v}\edtext{Referat}{\lemma{\textnormal{\emph{Referat}}}\Cendnote{\textnormal{Die \emph{Neue Freie Presse}\pwindex{Neue Freie Presse@\emph{Neue Freie Presse}|pwk} besprach die Aufführung nicht.}}}\label{K_L03527-3} vielleicht{ }ſchon erſchienen und habe
               ich es überſehen?\pend
           
\pstart
           Ob ich Sie \label{K_L03527-4v}\edtext{im Sommer wiederſehen}{\lemma{\textnormal{\emph{im Sommer wiedersehen}}}\Cendnote{\textnormal{Siehe XXXX Auszeichnungsfehler: Dokument L03064 nicht gefunden.
               }}}\label{K_L03527-4} werde, weiß ich noch nicht. Jedenfalls kann ich nur im Auguſt auf Urlaub gehen, {\pb}und auch dann
               will ich nicht herumreiſen,{ }ſondern irgendwo feſtſitzen, etwa am Wörtherſee\oindex{Wörthersee@\textbf{Wörthersee}, \emph{See}|pw}. Ich bat \textsc{Arthur}{ }\strikeout{drum} deshalb, daß er mit Ihnen im Auguſt an den Wörtherſee\oindex{Wörthersee@\textbf{Wörthersee}, \emph{See}|pw}
               kommen möge. Wenn das nicht geht,{ }ſehen wir uns hoffentlich auf meiner Rückreiſe in
                  Wien\oindex{Wien@\textbf{Wien}, \emph{Verwaltungsgebiet}|pw}.\pend
           
\pstart
           Sie{ }ſelbſt werden mit \textsc{Arthur} gewiß einige{ }ſchöne {\pb}Sommermonate verleben.
               Laſſen Sie alle trüben Gedanken zu Hauſe und genießen Sie die{ }ſchöne Welt, die ja
               überhaupt nur dann wirklich{ }ſchön iſt, wenn man Jemanden neben{ }ſich hat, den man \strikeout{g} liebt. Auch der Naturgenuß kann nur aus dem Herzen
               kommen; und das Herz bleibt ungerührt, wenn nicht eine Liebe es bewegt. Es gibt keine{ }ſchönen Landſchaften (ohne Liebe nämlich).\pend
           
\pstart
           Seien Sie herzlichſt gegrüßt von Ihrem ergebenen {\\[\baselineskip]}\spacefill\mbox{Dr. Paul Goldmann.}\pend
           \leftskip=0em{}\selectlanguage{ngerman}\endnumbering\briefempfaengerindex{Schnitzler, Olga@\textsc{Schnitzler, Olga}!zzzGoldmann, Paul@\emph{von Paul Goldmann}!1901-05-102@{10. 5. [1901]}|)be}\mylabel{L03527h}  \newcommand{\dateiname}{L03527}\newcommand{\titel}{Paul Goldmann an Olga Gussmann, 10. 5. [1901]}\newcommand{\editorInnen}{Martin Anton Müller und Laura Untner}%% latex-leseansicht-abspann.tex
%% Abspann für die Leseansicht.
%% Der Schalter \ifkorrekturansicht ist bereits durch den Vorspann gesetzt.

%% latex-abspann.tex
%% Gemeinsamer Abspann für Korrekturansicht und Leseansicht.
%% Setzt den Schalter \ifkorrekturansicht voraus (gesetzt in den
%% einbindenden Dateien latex-korrekturansicht-abspann.tex bzw.
%% latex-leseansicht-abspann.tex).
%% ---------------------------------------------------------------

\normalsize

% Das esempio-Environment wird nur in der Leseansicht benötigt
\ifkorrekturansicht\else
\newenvironment{esempio}[3]%
{
    \vspace{1.5ex}
    \rlap{\underline{#1}}
    \par
    \setlength{\parindent}{0cm}
    \nopagebreak
    \leftskip=#2cm
    \rightskip=#3cm
}
{
    \par
}
\fi

\doendnotes{C}
\bigskip
\vfill

\clearpage

\footnotesize

\ifkorrekturansicht
  \lohead{\textsc{register}}
\fi

% theindex-Environment neu definieren ohne reledmac
\makeatletter
\renewenvironment{theindex}{%
  \ifkorrekturansicht
    \section*{\indexname}%
  \else
    \subsubsection*{Index der erwähnten Entitäten}%
  \fi
  \setlength{\parindent}{0pt}%
  \setlength{\parskip}{0pt plus 0.3pt}%
  \let\item\@idxitem
}{%
  \ifkorrekturansicht\clearpage\fi
}
\makeatother

\IfFileExists{\jobname-pw.ind}{\input{\jobname-pw.ind}}{}

% Quellenangabe nur in der Leseansicht
\ifkorrekturansicht\else
% Fallback-Definitionen, falls die .tex-Datei \titel etc. nicht gesetzt hat
\providecommand{\titel}{}
\providecommand{\editorInnen}{}
\providecommand{\dateiname}{\jobname}

\vspace{3cm}

\vfill

\footnotesize
\textsc{Quelle}: \titel. Herausgegeben von {\editorInnen}. In: \emph{Arthur Schnitzler: Briefwechsel mit Autorinnen und Autoren}.
 Digitale Edition, https://schnitzler-briefe.acdh.oeaw.ac.at/{\dateiname}.html (Stand \today)
\fi

\end{document}


