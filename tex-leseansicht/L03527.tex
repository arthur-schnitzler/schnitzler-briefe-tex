%% latex-leseansicht-vorspann.tex
%% Vorspann für die Leseansicht.
%% Lädt die gemeinsame Datei latex-vorspann.tex mit nicht gesetztem Schalter.

\newif\ifkorrekturansicht
\korrekturansichtfalse

\input{../tex-inputs/latex-vorspann}

\begin{center}
            \textcolor{red}{ENTWURF, NICHT FERTIG KORRIGIERT}
                      \end{center}
            
         
         \renewcommand{\erwaehntePersonen}{Personen: Hermann Bahr, Ludwig Grillich, Olga Schnitzler}
         \renewcommand{\erwaehnteOrte}{Orte: Berlin, Dessauer Straße, Wien, Wörthersee}
         \renewcommand{\erwaehnteWerke}{Werke: Maria Magdalena. Ein bürgerliches Trauerspiel in drei Akten, Neue Freie Presse, Neues Wiener Tagblatt, Theater, Kunst und Literatur [Vorstellung des Konservatoriums], [Portraitfoto von Olga Gussmann]}
               \section[ Paul Goldmann an Olga Gussmann, 10. 5. {[}1901{]}]{ Paul Goldmann an Olga Gussmann, 10. 5. {[}1901{]}}\nopagebreak\mylabel{v}\rehead{ }\begin{ledgroupsized}[t]{13cm}\normalsize\beginnumbering \toendnotes[C]{\smallbreak\pagebreak[2]} \Standort{DLA, A:Schnitzler, HS.NZ85.1.5247.}
\physDesc{Brief, 1 Blatt, 4 Seiten, 1471 Zeichen
\newline{}Handschrift: blaue Tinte, deutsche Kurrent
\newline{}Ordnung: mit Bleistift von Arthur
                                    Schnitzler\pwindex{Schnitzler, Arthur 15.05.1862 – 21.10.1931@\textsc{Schnitzler, Arthur} (15.05.1862 – 21.10.1931), \emph{Schriftsteller, Mediziner}|pw} das Jahr »1901.« vermerkt }\toendnotes[C]{\smallbreak}\pstart
           \noindent{}\raggedleft{}{\pb}\textcolor{gray}{\textbf{DESSAUERSTRASSE 19\oindex{Dessauer Strasse@\textbf{Dessauer Straße}|pw}}}\pend
           \pstart
           Berlin\oindex{Berlin@\textbf{Berlin}|pw}, 10. Mai.\pend
           \pstart\center{}Liebes Fräulein \textsc{Olga},\pend\pstart
           Haben Sie vielen herzlichen Dank für das ſchöne \label{K_L03527-1v}\edtext{Bild\pwindex{Grillich, Ludwig †~1926-05-21@\textsc{Grillich, Ludwig} (†~1926-05-21), \emph{Fotograf}!Portraitfoto von Olga Gussmann]@\strich\emph{[Portraitfoto von Olga Gussmann]}|pwv}}{\lemma{\textnormal{\emph{Bild}}}\Cendnote{\textnormal{höchstwahrscheinlich das von Ludwig Grillich\pwindex{Grillich, Ludwig †~1926-05-21@\textsc{Grillich, Ludwig} (†~1926-05-21), \emph{Fotograf}|pwk} angefertige Portraitfoto\pwindex{Grillich, Ludwig †~1926-05-21@\textsc{Grillich, Ludwig} (†~1926-05-21), \emph{Fotograf}!Portraitfoto von Olga Gussmann]@\strich\emph{[Portraitfoto von Olga Gussmann]}|pwkv} (\emph{DLA}, B 1989.Q 0249)}}}\label{K_L03527-1h}! Es ſoll mir ein
               lieber Beſitz ſein. Diese Wien\oindex{Wien@\textbf{Wien}|pw}er Photographen ſind
               doch mehr Künſtler. Man bekommt nach dieſem Bilde wirklich ein lebendige Vorſtellung
               von Ihnen, und Ihre Perſönlichkeit iſt ſehr reizvoll darin ausgedrückt.\pend
           \pstart
           Mit Dank ſende ich Ihnen die \label{K_L03527-2v}\edtext{Zeitungausſchnitt\pwindex{Theater, Kunst und Literatur [Vorstellung des Konservatoriums]1901-05-01@\emph{Theater, Kunst und Literatur [Vorstellung des Konservatoriums]} {[}1901-05-01{]}|pwv}e}{\lemma{\textnormal{\emph{Zeitungausſchnitte}}}\Cendnote{\textnormal{Beilagen nicht erhalten. Bahr\pwindex{Bahr, Hermann 19.07.1863 – 15.01.1934@\textsc{Bahr, Hermann} (19.07.1863 – 15.01.1934), \emph{Schriftsteller, Kritiker}|pwk} hatte folgende lobende Notiz\pwindex{Theater, Kunst und Literatur [Vorstellung des Konservatoriums]1901-05-01@\emph{Theater, Kunst und Literatur [Vorstellung des Konservatoriums]} {[}1901-05-01{]}|pwkv} über die Aufführung von \emph{Maria Magdalena}\pwindex{\textcolor{red}{\textsuperscript{XXXX1 indx}}!Maria Magdalena. Ein buergerliches Trauerspiel in drei Akten1844@\strich\emph{Maria Magdalena. Ein bürgerliches Trauerspiel in drei Akten} {[}1844{]}|pwk} mit Olga Gussmann\pwindex{Schnitzler, Olga 17.01.1882 – 13.01.1970@\textsc{Schnitzler, Olga} (17.01.1882 – 13.01.1970), \emph{Schauspielerin, Sängerin}|pwk} (siehe Arthur Schnitzler an Hermann Bahr, 19. 4. 1901) verfasst: H. B.\pwindex{Bahr, Hermann 19.07.1863 – 15.01.1934@\textsc{Bahr, Hermann} (19.07.1863 – 15.01.1934), \emph{Schriftsteller, Kritiker}|pwkv} [ = Hermann Bahr\pwindex{Bahr, Hermann 19.07.1863 – 15.01.1934@\textsc{Bahr, Hermann} (19.07.1863 – 15.01.1934), \emph{Schriftsteller, Kritiker}|pwk}]: \emph{Theater, Kunst und Literatur}\pwindex{Theater, Kunst und Literatur [Vorstellung des Konservatoriums]1901-05-01@\emph{Theater, Kunst und Literatur [Vorstellung des Konservatoriums]} {[}1901-05-01{]}|pwk}. In: \emph{Neues Wiener Tagblatt}\pwindex{?? Werk@Nicht ermittelte Verfasserinnen und Verfasser!Neues Wiener Tagblatt1867 – 1945@\emph{Neues Wiener Tagblatt} {[}1867 – 1945{]}|pwk}, Jg. 35, Nr. 118, 1. 5. 1901, S. 7.}}}\label{K_L03527-2h} zurück. \textsc{Bahr\pwindex{Bahr, Hermann 19.07.1863 – 15.01.1934@\textsc{Bahr, Hermann} (19.07.1863 – 15.01.1934), \emph{Schriftsteller, Kritiker}|pw}} hat, {\pb}wie gewöhnlich, \substVorne{}\textsuperscript{Blech}\substDazwischen{}Blech\substHinten{}{ }geſchrieben\pwindex{Theater, Kunst und Literatur [Vorstellung des Konservatoriums]1901-05-01@\emph{Theater, Kunst und Literatur [Vorstellung des Konservatoriums]} {[}1901-05-01{]}|pwv}. Das ſpürt man
               heraus, wenn man auch die Vorſtellung\pwindex{\textcolor{red}{\textsuperscript{XXXX1 indx}}!Maria Magdalena. Ein buergerliches Trauerspiel in drei Akten1844@\strich\emph{Maria Magdalena. Ein bürgerliches Trauerspiel in drei Akten} {[}1844{]}|pwv} ſelbſt nicht geſehen hat. Ich freue mich, daß Alles gut gegangen
               iſt. Auf die N. Fr. Pr.\pwindex{Neue Freie Presse1864 – 1939@\emph{Neue Freie Presse} {[}1864 – 1939{]}|pw} bin ich neugierig. Oder
               iſt das \label{K_L03527-3v}\edtext{Referat }{\lemma{\textnormal{\emph{Referat }}}\Cendnote{\textnormal{Es konnte kein entsprechender
                  Zeitungsbericht ermittelt werden.}}}\label{K_L03527-3h} vielleicht ſchon erſchienen und habe ich
               es überſehen?\pend
           \pstart
           Ob ich Sie \label{K_L03527-4v}\edtext{im Sommer wiederſehen}{\lemma{\textnormal{\emph{im Sommer wiederſehen}}}\Cendnote{\textnormal{siehe Paul Goldmann an Arthur Schnitzler, 26. 4. [1901]}}}\label{K_L03527-4h} werde, weiß ich noch nicht. Jedenfalls kann ich nur im Auguſt auf Urlaub gehen, {\pb}und auch dann
               will ich nicht herumreiſen, ſondern irgendwo feſtſitzen, etwa am Wörtherſee\oindex{Woerthersee@\textbf{Wörthersee}|pw}. Ich bat \textsc{Arthur\pwindex{Schnitzler, Arthur 15.05.1862 – 21.10.1931@\textsc{Schnitzler, Arthur} (15.05.1862 – 21.10.1931), \emph{Schriftsteller, Mediziner}|pw}}{ }\strikeout{darum} deshalb, daß er mit Ihnen im Auguſt an den Wörtherſee\oindex{Woerthersee@\textbf{Wörthersee}|pw}
               kommen möge. Wenn das nicht geht, ſehen wir uns hoffentlich auf meiner Rückreiſe in
                  Wien\oindex{Wien@\textbf{Wien}|pw}.\pend
           \pstart
           Sie ſelbſt werden mit \textsc{Arthur\pwindex{Schnitzler, Arthur 15.05.1862 – 21.10.1931@\textsc{Schnitzler, Arthur} (15.05.1862 – 21.10.1931), \emph{Schriftsteller, Mediziner}|pw}} gewiß einige ſchöne {\pb}Sommermonate verleben.
               Laſſen Sie alle trüben Gedanken zu Hauſe und genießen Sie die ſchöne Welt, die ja
               überhaupt nur dann wirklich ſchön iſt, wenn man Jemanden neben ſich hat, den man \strikeout{\textcolor{gray}{l}} liebt. Auch der Naturgenuß kann nur aus dem Herzen kommen; und das Herz bleibt
               ungerührt, wenn nicht eine Liebe es bewegt. Es gibt keine ſchönen Landſchaften (ohne
               Liebe nämlich).\pend
           \pstart
           Seien Sie herzlichſt gegrüßt von Ihrem ergebenen {\\[\baselineskip]}\spacefill\mbox{Dr. Paul Goldmann.}\pend
           \leftskip=0em{}
         
         \endnumbering\mylabel{h}\end{ledgroupsized}\begin{anhang}\end{anhang}\newcommand{\dateiname}{L03527}\newcommand{\titel}{Paul Goldmann an Olga Gussmann, 10. 5. [1901]}\newcommand{\editorInnen}{Martin Anton Müller und Laura Untner}%% latex-leseansicht-abspann.tex
%% Abspann für die Leseansicht.
%% Der Schalter \ifkorrekturansicht ist bereits durch den Vorspann gesetzt.

%% latex-abspann.tex
%% Gemeinsamer Abspann für Korrekturansicht und Leseansicht.
%% Setzt den Schalter \ifkorrekturansicht voraus (gesetzt in den
%% einbindenden Dateien latex-korrekturansicht-abspann.tex bzw.
%% latex-leseansicht-abspann.tex).
%% ---------------------------------------------------------------

\normalsize

% Das esempio-Environment wird nur in der Leseansicht benötigt
\ifkorrekturansicht\else
\newenvironment{esempio}[3]%
{
    \vspace{1.5ex}
    \rlap{\underline{#1}}
    \par
    \setlength{\parindent}{0cm}
    \nopagebreak
    \leftskip=#2cm
    \rightskip=#3cm
}
{
    \par
}
\fi

\doendnotes{C}
\bigskip
\vfill

\clearpage

\footnotesize

\ifkorrekturansicht
  \lohead{\textsc{register}}
\fi

% theindex-Environment neu definieren ohne reledmac
\makeatletter
\renewenvironment{theindex}{%
  \ifkorrekturansicht
    \section*{\indexname}%
  \else
    \subsubsection*{Index der erwähnten Entitäten}%
  \fi
  \setlength{\parindent}{0pt}%
  \setlength{\parskip}{0pt plus 0.3pt}%
  \let\item\@idxitem
}{%
  \ifkorrekturansicht\clearpage\fi
}
\makeatother

\IfFileExists{\jobname-pw.ind}{\input{\jobname-pw.ind}}{}

% Quellenangabe nur in der Leseansicht
\ifkorrekturansicht\else
% Fallback-Definitionen, falls die .tex-Datei \titel etc. nicht gesetzt hat
\providecommand{\titel}{}
\providecommand{\editorInnen}{}
\providecommand{\dateiname}{\jobname}

\vspace{3cm}

\vfill

\footnotesize
\textsc{Quelle}: \titel. Herausgegeben von {\editorInnen}. In: \emph{Arthur Schnitzler: Briefwechsel mit Autorinnen und Autoren}.
 Digitale Edition, https://schnitzler-briefe.acdh.oeaw.ac.at/{\dateiname}.html (Stand \today)
\fi

\end{document}


      