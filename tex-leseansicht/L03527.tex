%% latex-korrekturansicht-vorspann.tex
%% Vorspann für die Korrekturansicht.
%% Lädt die gemeinsame Datei latex-vorspann.tex mit gesetztem Schalter.

\newif\ifkorrekturansicht
\korrekturansichttrue

\input{../tex-inputs/latex-vorspann}


\section[ Paul Goldmann an Olga Gussmann, 10. 5. {[}1901{]}]{L03527 Paul Goldmann an Olga Gussmann, 10. 5. {[}1901{]}}
\nopagebreak\mylabel{L03527v}
\rehead{ }\normalsize\beginnumbering\briefempfaengerindex{Schnitzler, Olga@\textsc{Schnitzler, Olga}!zzzGoldmann, Paul@\emph{von Paul Goldmann}!1901-05-102@{10. 5. {[}1901{]}}|(be}
\toendnotes[C]{\smallbreak\pagebreak[2]}\Standort{DLA, A:Schnitzler, HS.NZ85.1.5247.}
\physDesc{Brief, 1 Blatt, 4 Seiten, 1470 Zeichen
\newline{}Handschrift: blaue Tinte, deutsche Kurrent
\newline{}Ordnung: mit Bleistift von Arthur Schnitzler das
                                 Jahr »1901.« vermerkt }\toendnotes[C]{\smallbreak}
\pstart
           \raggedleft{}{\pb}\textcolor{gray}{\textbf{DESSAUERSTRASSE 19\oindex{Dessauer Strasse@\textbf{Dessauer Straße}, \emph{Straße (K.STR)}|pw}}}\pend
           
\pstart
           Berlin\oindex{Berlin@\textbf{Berlin}, \emph{P.PPLC}|pw}, 10. Mai.\pend
           
\pstart\center{}Liebes Fräulein \textsc{Olga},\pend\vspace{0.5em}
\pstart
           Haben Sie vielen herzlichen Dank für das ſchöne \label{K_L03527-1v}\edtext{Bild\pwindex{Portraitfoto von Olga Gussmann]@\emph{[Portraitfoto von Olga Gussmann]}|pwuv}}{\lemma{\textnormal{\emph{Bild}}}\Cendnote{\textnormal{Es dürfte das von Ludwig Grillich\pwindex{Grillich, Ludwig †~1926-05-21@\textsc{Grillich, Ludwig} (†~1926-05-21), \emph{Fotograf/Fotografin}|pwk} angefertige Porträtfoto\pwindex{Portraitfoto von Olga Gussmann]@\emph{[Portraitfoto von Olga Gussmann]}|pwkv} gemeint sein (\emph{DLA}, B 1989.Q 0249).}}}\label{K_L03527-1}! Es ſoll mir ein
               lieber Beſitz ſein. Diese Wien\oindex{Wien@\textbf{Wien}, \emph{A.ADM2}|pw}er Photographen ſind
               doch wahre Künſtler. Man bekommt nach dieſem Bilde\pwindex{Portraitfoto von Olga Gussmann]@\emph{[Portraitfoto von Olga Gussmann]}|pwv} wirklich eine lebendige Vorſtellung von Ihnen, und
               Ihre Perſönlichkeit iſt ſehr reizvoll darin ausgedrückt.\pend
           
\pstart
           Mit Dank ſende ich Ihnen die \label{K_L03527-2v}\edtext{Zeitungausſchnitt\pwindex{Theater, Kunst und Literatur [Vorstellung des Konservatoriums]@\emph{Theater, Kunst und Literatur [Vorstellung des Konservatoriums]}|pwv}e}{\lemma{\textnormal{\emph{Zeitungausſchnitte}}}\Cendnote{\textnormal{Beilagen nicht erhalten. Bahr\pwindex{Bahr, Hermann 19.07.1863 – 15.01.1934@\textsc{Bahr, Hermann} (19.07.1863 – 15.01.1934), \emph{Schriftsteller/Schriftstellerin, Kritiker/Kritikerin}|pwk} hatte folgende lobende Notiz\pwindex{Theater, Kunst und Literatur [Vorstellung des Konservatoriums]@\emph{Theater, Kunst und Literatur [Vorstellung des Konservatoriums]}|pwkv} über die Schulaufführung von \emph{Maria Magdalena}\pwindex{Maria Magdalena. Ein buergerliches Trauerspiel in drei Akten@\emph{Maria Magdalena. Ein bürgerliches Trauerspiel in drei Akten}|pwk} mit Olga Gussmann\pwindex{Schnitzler, Olga 17.01.1882 – 13.01.1970@\textsc{Schnitzler, Olga} (17.01.1882 – 13.01.1970), \emph{Schauspieler/Schauspielerin, Sänger/Sängerin}|pwk} verfasst: H. B.\pwindex{Bahr, Hermann 19.07.1863 – 15.01.1934@\textsc{Bahr, Hermann} (19.07.1863 – 15.01.1934), \emph{Schriftsteller/Schriftstellerin, Kritiker/Kritikerin}|pwkv} [ = Hermann Bahr\pwindex{Bahr, Hermann 19.07.1863 – 15.01.1934@\textsc{Bahr, Hermann} (19.07.1863 – 15.01.1934), \emph{Schriftsteller/Schriftstellerin, Kritiker/Kritikerin}|pwk}]: \emph{Theater, Kunst und Literatur}\pwindex{Theater, Kunst und Literatur [Vorstellung des Konservatoriums]@\emph{Theater, Kunst und Literatur [Vorstellung des Konservatoriums]}|pwk}. In: \emph{Neues Wiener Tagblatt}\pwindex{Neues Wiener Tagblatt@\emph{Neues Wiener Tagblatt}|pwk}, Jg. 35, Nr. 118, 1. 5. 1901, S. 7. Siehe Arthur Schnitzler an Hermann Bahr, 19. 4. 1901. Vgl. Martin Anton Müller:
                        \emph{Hermann Bahr und Arthur Schnitzler im Konvervatorium.
                        Cherchez la femme!} In: \emph{Ein Zoll Dankfest. Texte
                        für Germanistik. Konstanze Fliedl zum 60. Geburtstag}.
                     Würzburg: \emph{Königshausen {\kaufmannsund} Neumann}{ }2015, S. 43–49.}}}\label{K_L03527-2} zurück. \textsc{Bahr\pwindex{Bahr, Hermann 19.07.1863 – 15.01.1934@\textsc{Bahr, Hermann} (19.07.1863 – 15.01.1934), \emph{Schriftsteller/Schriftstellerin, Kritiker/Kritikerin}|pw}} hat, {\pb}wie gewöhnlich, \substVorne{}\textsuperscript{Blech}\substDazwischen{}Blech\substHinten{}{ }geſchrieben\pwindex{Theater, Kunst und Literatur [Vorstellung des Konservatoriums]@\emph{Theater, Kunst und Literatur [Vorstellung des Konservatoriums]}|pwv}. Das ſpürt man
               heraus, wenn man auch die Vorſtellung\pwindex{Maria Magdalena. Ein buergerliches Trauerspiel in drei Akten@\emph{Maria Magdalena. Ein bürgerliches Trauerspiel in drei Akten}|pwv} ſelbſt nicht geſehen hat. Ich freue mich, daß Alles gut gegangen
               iſt. Auf die N. Fr. Pr.\pwindex{Neue Freie Presse@\emph{Neue Freie Presse}|pw} bin ich neugierig. Oder
               iſt das \label{K_L03527-3v}\edtext{Referat}{\lemma{\textnormal{\emph{Referat}}}\Cendnote{\textnormal{Die \emph{Neue Freie Presse}\pwindex{Neue Freie Presse@\emph{Neue Freie Presse}|pwk} besprach die Aufführung nicht.}}}\label{K_L03527-3} vielleicht ſchon erſchienen und habe
               ich es überſehen?\pend
           
\pstart
           Ob ich Sie \label{K_L03527-4v}\edtext{im Sommer wiederſehen}{\lemma{\textnormal{\emph{im Sommer wiederſehen}}}\Cendnote{\textnormal{Siehe Paul Goldmann an Arthur Schnitzler, 26. 4. [1901].
               }}}\label{K_L03527-4} werde, weiß ich noch nicht. Jedenfalls kann ich nur im Auguſt auf Urlaub gehen, {\pb}und auch dann
               will ich nicht herumreiſen, ſondern irgendwo feſtſitzen, etwa am Wörtherſee\oindex{Woerthersee@\textbf{Wörthersee}, \emph{H.LK}|pw}. Ich bat \textsc{Arthur}{ }\strikeout{drum} deshalb, daß er mit Ihnen im Auguſt an den Wörtherſee\oindex{Woerthersee@\textbf{Wörthersee}, \emph{H.LK}|pw}
               kommen möge. Wenn das nicht geht, ſehen wir uns hoffentlich auf meiner Rückreiſe in
                  Wien\oindex{Wien@\textbf{Wien}, \emph{A.ADM2}|pw}.\pend
           
\pstart
           Sie ſelbſt werden mit \textsc{Arthur} gewiß einige ſchöne {\pb}Sommermonate verleben.
               Laſſen Sie alle trüben Gedanken zu Hauſe und genießen Sie die ſchöne Welt, die ja
               überhaupt nur dann wirklich ſchön iſt, wenn man Jemanden neben ſich hat, den man \strikeout{g} liebt. Auch der Naturgenuß kann nur aus dem Herzen
               kommen; und das Herz bleibt ungerührt, wenn nicht eine Liebe es bewegt. Es gibt keine
               ſchönen Landſchaften (ohne Liebe nämlich).\pend
           
\pstart
           Seien Sie herzlichſt gegrüßt von Ihrem ergebenen {\\[\baselineskip]}\spacefill\mbox{Dr. Paul Goldmann.}\pend
           \leftskip=0em{}\selectlanguage{ngerman}\endnumbering\briefempfaengerindex{Schnitzler, Olga@\textsc{Schnitzler, Olga}!zzzGoldmann, Paul@\emph{von Paul Goldmann}!1901-05-102@{10. 5. {[}1901{]}}|)be}\mylabel{L03527h}  \normalsize

\doendnotes{C}
\bigskip
\vfill

\clearpage

\footnotesize

\lohead{\textsc{register}}

% Definiere theindex-Environment komplett neu ohne reledmac
\makeatletter
\renewenvironment{theindex}{%
  \section*{\indexname}%
  \setlength{\parindent}{0pt}%
  \setlength{\parskip}{0pt plus 0.3pt}%
  \let\item\@idxitem
}{%
  \clearpage
}
\makeatother

\IfFileExists{\jobname-pw.ind}{\input{\jobname-pw.ind}}{}

\end{document}

      