%% latex-leseansicht-vorspann.tex
%% Vorspann für die Leseansicht.
%% Lädt die gemeinsame Datei latex-vorspann.tex mit nicht gesetztem Schalter.

\newif\ifkorrekturansicht
\korrekturansichtfalse

\input{../tex-inputs/latex-vorspann}


\section[Wanda Sacher-Masoch an Arthur Schnitzler, 9. 12. 1908]{L03880 Wanda Sacher-Masoch an Arthur Schnitzler, 9. 12. 1908}
\nopagebreak\mylabel{L03880v}
\rehead{ }\normalsize\beginnumbering\briefempfaengerindex{Schnitzler, Arthur@\textsc{Schnitzler, Arthur}!zzzSacher-Masoch, Wanda von@\emph{von Wanda von Sacher-Masoch}!1908-12-091@{9. 12. 1908}|(be}
\toendnotes[C]{\smallbreak\pagebreak[2]}
\correspDesc{Versand  durch Wanda Sacher-Masoch am 9. 12. 1908 in Morges
\newline{}Erhalt  durch Arthur Schnitzler im Zeitraum [10. 12. 1908 – 14. 12. 1908?] \textbf{Ort fehlend} }\toendnotes[C]{\smallbreak}
\Standort{DLA, A:Schnitzler, 1985.1.4377.}
\physDesc{Briefkarte, 278 Zeichen
\newline{}Handschrift: schwarze Tinte, deutsche Kurrent
\newline{}Schnitzler: mit rotem Buntstift eine Unterstreichung 
\newline{}Ordnung: mit Bleistift eine Unterstreichung }\toendnotes[C]{\smallbreak}
\pstart{}{\pb}Sehr geehrter Herr!\pend\vspace{0.5em}
\pstart
           Wären Sie bereit ein \label{K_L03880-1v}\edtext{Stück mit mir zuſammen zu{ }ſchreiben}{\lemma{\textnormal{\emph{Stück … schreiben}}}\Cendnote{\textnormal{Diese Karte ist
                  der einzige erhaltene Beleg für einen Kontakt zwischen Schnitzler und Wanda von Sacher-Masoch\pwindex{Sacher-Masoch, Wanda von 14.\,3.\,1845 Graz – 1917?@\textsc{Sacher-Masoch, Wanda von} (14.\,3.\,1845 Graz – 1917?), \emph{Übersetzerin, Schriftstellerin}|pwk}.
                  Ob sie auf die Karte eine Antwort erhalten hat, ist offen. In diesem Fall dürfte er darauf verwiesen haben, dass er nie Kooperationen bei Texten einging.}}}\label{K_L03880-1}?\pend
           
\pstart
           Es handelt{ }ſich um ein{ }ſehr \uline{dramatiſches
                  Erlebnis}\textcolor{gray}{,} das ich in München\oindex{München@\textbf{München}|pw} hatte.\pend
           
\pstart
           {\pb}Durch eine raſche Rückäußerung würden Sie mich{ }ſehr
               verbinden.\pend
           
\pstart
           Hochachtungsvoll{\\[\baselineskip]}\spacefill\mbox{Wanda v. Sacher-Maſoch.}\pend
           \leftskip=0em{}
\pstart
           \textsc{Morges, Schweiz\oindex{Morges@\textbf{Morges}, \emph{Verwaltungsgebiet}|pw}
                        (Vaud\oindex{Kanton Waadt@\textbf{Kanton Waadt}, \emph{Land}|pw})}{ }9/12–08\pend
           \selectlanguage{ngerman}\endnumbering\briefempfaengerindex{Schnitzler, Arthur@\textsc{Schnitzler, Arthur}!zzzSacher-Masoch, Wanda von@\emph{von Wanda von Sacher-Masoch}!1908-12-091@{9. 12. 1908}|)be}\mylabel{L03880h}
\begin{anhang}
\end{anhang}\newcommand{\dateiname}{L03880}\newcommand{\titel}{Wanda Sacher-Masoch an Arthur Schnitzler, 9. 12. 1908}\newcommand{\editorInnen}{Selma Jahnke und Martin Anton Müller}%% latex-leseansicht-abspann.tex
%% Abspann für die Leseansicht.
%% Der Schalter \ifkorrekturansicht ist bereits durch den Vorspann gesetzt.

%% latex-abspann.tex
%% Gemeinsamer Abspann für Korrekturansicht und Leseansicht.
%% Setzt den Schalter \ifkorrekturansicht voraus (gesetzt in den
%% einbindenden Dateien latex-korrekturansicht-abspann.tex bzw.
%% latex-leseansicht-abspann.tex).
%% ---------------------------------------------------------------

\normalsize

% Das esempio-Environment wird nur in der Leseansicht benötigt
\ifkorrekturansicht\else
\newenvironment{esempio}[3]%
{
    \vspace{1.5ex}
    \rlap{\underline{#1}}
    \par
    \setlength{\parindent}{0cm}
    \nopagebreak
    \leftskip=#2cm
    \rightskip=#3cm
}
{
    \par
}
\fi

\doendnotes{C}
\bigskip
\vfill

\clearpage

\footnotesize

\ifkorrekturansicht
  \lohead{\textsc{register}}
\fi

% theindex-Environment neu definieren ohne reledmac
\makeatletter
\renewenvironment{theindex}{%
  \ifkorrekturansicht
    \section*{\indexname}%
  \else
    \subsubsection*{Index der erwähnten Entitäten}%
  \fi
  \setlength{\parindent}{0pt}%
  \setlength{\parskip}{0pt plus 0.3pt}%
  \let\item\@idxitem
}{%
  \ifkorrekturansicht\clearpage\fi
}
\makeatother

\IfFileExists{\jobname-pw.ind}{\input{\jobname-pw.ind}}{}

% Quellenangabe nur in der Leseansicht
\ifkorrekturansicht\else
% Fallback-Definitionen, falls die .tex-Datei \titel etc. nicht gesetzt hat
\providecommand{\titel}{}
\providecommand{\editorInnen}{}
\providecommand{\dateiname}{\jobname}

\vspace{3cm}

\vfill

\footnotesize
\textsc{Quelle}: \titel. Herausgegeben von {\editorInnen}. In: \emph{Arthur Schnitzler: Briefwechsel mit Autorinnen und Autoren}.
 Digitale Edition, https://schnitzler-briefe.acdh.oeaw.ac.at/{\dateiname}.html (Stand \today)
\fi

\end{document}


