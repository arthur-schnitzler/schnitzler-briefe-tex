%% latex-leseansicht-vorspann.tex
%% Vorspann für die Leseansicht.
%% Lädt die gemeinsame Datei latex-vorspann.tex mit nicht gesetztem Schalter.

\newif\ifkorrekturansicht
\korrekturansichtfalse

\input{../tex-inputs/latex-vorspann}


         
         \newcommand{\erwaehntePersonen}{Personen: Hermann Bahr, Olga Schnitzler}
         \newcommand{\erwaehnteInstitutionen}{Institutionen: Burgtheater}
         \newcommand{\erwaehnteOrte}{Orte: Sternwartestraße, Wien}
         \newcommand{\erwaehnteWerke}{
               \section[Arthur Schnitzler an Hermann Bahr, 23. 9. 1918]{ Arthur Schnitzler an Hermann Bahr, 23. 9. 1918}\nopagebreak\mylabel{v}\rehead{ }\begin{ledgroupsized}[t]{13cm}\normalsize\beginnumbering \toendnotes[C]{\smallbreak\pagebreak[2]} \Standort{TMW, HS AM 96991 Ba.}
\physDesc{Brief, 1 Blatt, 1 Seite
\newline{}Schreibmaschine
\newline{}Handschrift: schwarze Tinte (\noindent{}Unterschrift)\newline{}Ordnung: Erwerb bei einer Auktion des Dorotheum, 19.12.2022, Lot. 154 }\Standort{DLA, A:Schnitzler, 85.1.294/6.}
\physDesc{Brief, 1 Blatt, 1 Seite, maschineller Durchschlag
\newline{}Schreibmaschine
\newline{}Handschrift: roter Buntstift, lateinische Kurrent (\noindent{}Beschriftung: »Bahr« und zwei Unterstreichungen)}\buchAbdrucke{\weitereDrucke{Hermann Bahr, Arthur Schnitzler: \emph{Briefwechsel, Aufzeichnungen, Dokumente (1891–1931)}. Hg. Kurt Ifkovits und Martin Anton Müller. Göttingen: \emph{Wallstein} 2018, S. 520.} }\toendnotes[C]{\smallbreak}\pstart
           \textcolor{gray}{\textbf{{\pb}Dr. Arthur Schnitzler}}\hfill 23. 9. 1918.\pend
           \pstart
           \textcolor{gray}{\textbf{Wien XVIII. Sternwartestrasse 71\oindex{Sternwartestrasse@\textbf{Sternwartestraße}|pw}}}\pend
           \pstart\center{}Lieber Hermann.\pend\pstart
           Kann ich darauf rechnen, dass ich wie in den abgelaufenen Jahren für mich und meine
                  Frau\pwindex{Schnitzler, Olga 17.01.1882 – 13.01.1970@\textsc{Schnitzler, Olga} (17.01.1882 – 13.01.1970), \emph{Schauspielerin, Sängerin}|pwv} Einlass zu den
               Generalproben im Burgtheater\orgindex{Burgtheater@Burgtheater|pw} erhalte? Falls ja, bitte
               ich Dich recht sehr einen darauf bezüglichen Auftrag zu erteilen.\pend
           \pstart
           Herzlichst{\\[\baselineskip]}Dein{\\[\baselineskip]}{[}hs.:{]} \spacefill\mbox{ArthurSchn}\pend
           \leftskip=0em{}
         
         \endnumbering\mylabel{h}\end{ledgroupsized}  \newcommand{\dateiname}{L02304}\newcommand{\titel}{Arthur Schnitzler an Hermann Bahr, 23. 9. 1918}\newcommand{\editorInnen}{ Kurt Ifkovits,  Martin Anton Müller}%% latex-leseansicht-abspann.tex
%% Abspann für die Leseansicht.
%% Der Schalter \ifkorrekturansicht ist bereits durch den Vorspann gesetzt.

%% latex-abspann.tex
%% Gemeinsamer Abspann für Korrekturansicht und Leseansicht.
%% Setzt den Schalter \ifkorrekturansicht voraus (gesetzt in den
%% einbindenden Dateien latex-korrekturansicht-abspann.tex bzw.
%% latex-leseansicht-abspann.tex).
%% ---------------------------------------------------------------

\normalsize

% Das esempio-Environment wird nur in der Leseansicht benötigt
\ifkorrekturansicht\else
\newenvironment{esempio}[3]%
{
    \vspace{1.5ex}
    \rlap{\underline{#1}}
    \par
    \setlength{\parindent}{0cm}
    \nopagebreak
    \leftskip=#2cm
    \rightskip=#3cm
}
{
    \par
}
\fi

\doendnotes{C}
\bigskip
\vfill

\clearpage

\footnotesize

\ifkorrekturansicht
  \lohead{\textsc{register}}
\fi

% theindex-Environment neu definieren ohne reledmac
\makeatletter
\renewenvironment{theindex}{%
  \ifkorrekturansicht
    \section*{\indexname}%
  \else
    \subsubsection*{Index der erwähnten Entitäten}%
  \fi
  \setlength{\parindent}{0pt}%
  \setlength{\parskip}{0pt plus 0.3pt}%
  \let\item\@idxitem
}{%
  \ifkorrekturansicht\clearpage\fi
}
\makeatother

\IfFileExists{\jobname-pw.ind}{\input{\jobname-pw.ind}}{}

% Quellenangabe nur in der Leseansicht
\ifkorrekturansicht\else
% Fallback-Definitionen, falls die .tex-Datei \titel etc. nicht gesetzt hat
\providecommand{\titel}{}
\providecommand{\editorInnen}{}
\providecommand{\dateiname}{\jobname}

\vspace{3cm}

\vfill

\footnotesize
\textsc{Quelle}: \titel. Herausgegeben von {\editorInnen}. In: \emph{Arthur Schnitzler: Briefwechsel mit Autorinnen und Autoren}.
 Digitale Edition, https://schnitzler-briefe.acdh.oeaw.ac.at/{\dateiname}.html (Stand \today)
\fi

\end{document}


      