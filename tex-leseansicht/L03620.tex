%% latex-leseansicht-vorspann.tex
%% Vorspann für die Leseansicht.
%% Lädt die gemeinsame Datei latex-vorspann.tex mit nicht gesetztem Schalter.

\newif\ifkorrekturansicht
\korrekturansichtfalse

\input{../tex-inputs/latex-vorspann}


\section[ Karl Emil Franzos an Arthur Schnitzler, 8. 9. 1900]{L03620 Karl Emil Franzos an Arthur Schnitzler,  8. 9. 1900}
\nopagebreak\mylabel{L03620v}
\rehead{ }\normalsize\beginnumbering\briefempfaengerindex{Schnitzler, Arthur@\textsc{Schnitzler, Arthur}!zzzFranzos, Karl Emil@\emph{von Karl Emil Franzos}!1900-09-082@{8. 9. 1900}|(be}
\toendnotes[C]{\smallbreak\pagebreak[2]}
\correspDesc{Versand  durch Karl Emil Franzos am 8. 9. 1900 in Berlin
\newline{}Erhalt  durch Arthur Schnitzler am 10. 9. 1900 in Wien}\toendnotes[C]{\smallbreak}
\Standort{DLA, A:Schnitzler, HS.1985.1.3025.}
\physDesc{Postkarte, 661 Zeichen
\newline{}Handschrift: schwarze Tinte, deutsche Kurrent
\newline{}Versand: 1) Stempel: »\nobreak{}\oindex{Berlin@\textbf{Berlin}, \emph{Hauptstadt}|pwk}Berl{[}lin{]}
                                       10, 8. 9. 00, 8–9\nobreak{}«.   2) Stempel: »\nobreak{}\oindex{IX., Alsergrund@\textbf{IX., Alsergrund}, \emph{Verwaltungsgebiet}|pwk}Wien 9/3, 10. 9. 00, 8.V, Bestellt\nobreak{}«. }\pstart{}{\pb}Herrn Dr. A.
                     Schnitzler\pend{}\pstart{}Wien IX\oindex{IX., Alsergrund@\textbf{IX., Alsergrund}, \emph{Verwaltungsgebiet}|pw}\pend{}\pstart{}Frankgasse 1\oindex{Wien@\textbf{Wien}!IX., Alsergrund@\textbf{IX., Alsergrund}!Frankgasse 1@\textbf{Frankgasse 1}, \emph{Wohngebäude}|pw}.\pend{}{\bigskip}\vspace{1em}
\pstart
           \centering{}{\pb}\textcolor{gray}{\textbf{Redaction der »Deutschen
                        Dichtung\orgindex{Deutsche Dichtung@Deutsche Dichtung|pw}«}}\pend
           
\pstart
           \raggedleft{}\textcolor{gray}{\textbf{\textbf{\emph{Berlin W. 10\oindex{Berlin W@\textbf{Berlin W}, \emph{Bezirk}|pw}}},}}{ }\substVorne{}\textsuperscript{9}\substDazwischen{}8\substHinten{}. IX \textcolor{gray}{\textbf{1\strikeout{8}9}}00\pend
           
\pstart
           \raggedleft{}\textcolor{gray}{\textbf{\emph{Friedrich Wilhelm-Strasse 6\oindex{Friedrich Wilhelm-Strasse 6@\textbf{Friedrich Wilhelm-Strasse 6}, \emph{Wohngebäude}|pw}}.}}\pend
           
\pstart{}Verehrter Herr Doctor!\pend\vspace{0.5em}
\pstart
           Es thut mir{ }ſehr leid, daß zunächſt nichts von Ihnen zu haben ist, doch hoffe ich auf
               Ihre freundliche Zuſage, beim Nächſten an mich zu denken. Wir kö{\geminationn}en Längeres und Kürzeres brauchen; haben Sie was,{ }ſo{ }ſchicken Sies und fügen Sie Ihren Honorar-Anspruch bei; wir ko{\geminationm}en da{\geminationn} schon zu einem
               Gehalt et\textcolor{gray}{c}. Am liebſten brächte ich ein Drama von Ihnen; da
               Ihnen dadurch weder die Bühnen-Tantième noch das Honorar der Buchgausgabe irgend
               tangirt wär,{ }ſo iſt dies vielleicht auch Ihnen das Genehmſte!\pend
           
\pstart
           Mit beſten Empfehlungen Ihr{ }ſehr ergebner{\\[\baselineskip]}\spacefill\mbox{K. E. Franzos}\pend
           \leftskip=0em{}
\pstart
           \noindent{}Herrn \textsc{Dr. A. Schnitzler, Wien
                        IX.\oindex{IX., Alsergrund@\textbf{IX., Alsergrund}, \emph{Verwaltungsgebiet}|pw}{ }Frankgasse 1\oindex{Wien@\textbf{Wien}!IX., Alsergrund@\textbf{IX., Alsergrund}!Frankgasse 1@\textbf{Frankgasse 1}, \emph{Wohngebäude}|pw}.}\pend
           \selectlanguage{ngerman}\endnumbering\briefempfaengerindex{Schnitzler, Arthur@\textsc{Schnitzler, Arthur}!zzzFranzos, Karl Emil@\emph{von Karl Emil Franzos}!1900-09-082@{8. 9. 1900}|)be}\mylabel{L03620h}  \newcommand{\dateiname}{L03620}\newcommand{\titel}{Karl Emil Franzos an Arthur Schnitzler, 8. 9. 1900}\newcommand{\editorInnen}{Selma Jahnke und Martin Anton Müller}%% latex-leseansicht-abspann.tex
%% Abspann für die Leseansicht.
%% Der Schalter \ifkorrekturansicht ist bereits durch den Vorspann gesetzt.

%% latex-abspann.tex
%% Gemeinsamer Abspann für Korrekturansicht und Leseansicht.
%% Setzt den Schalter \ifkorrekturansicht voraus (gesetzt in den
%% einbindenden Dateien latex-korrekturansicht-abspann.tex bzw.
%% latex-leseansicht-abspann.tex).
%% ---------------------------------------------------------------

\normalsize

% Das esempio-Environment wird nur in der Leseansicht benötigt
\ifkorrekturansicht\else
\newenvironment{esempio}[3]%
{
    \vspace{1.5ex}
    \rlap{\underline{#1}}
    \par
    \setlength{\parindent}{0cm}
    \nopagebreak
    \leftskip=#2cm
    \rightskip=#3cm
}
{
    \par
}
\fi

\doendnotes{C}
\bigskip
\vfill

\clearpage

\footnotesize

\ifkorrekturansicht
  \lohead{\textsc{register}}
\fi

% theindex-Environment neu definieren ohne reledmac
\makeatletter
\renewenvironment{theindex}{%
  \ifkorrekturansicht
    \section*{\indexname}%
  \else
    \subsubsection*{Index der erwähnten Entitäten}%
  \fi
  \setlength{\parindent}{0pt}%
  \setlength{\parskip}{0pt plus 0.3pt}%
  \let\item\@idxitem
}{%
  \ifkorrekturansicht\clearpage\fi
}
\makeatother

\IfFileExists{\jobname-pw.ind}{\input{\jobname-pw.ind}}{}

% Quellenangabe nur in der Leseansicht
\ifkorrekturansicht\else
% Fallback-Definitionen, falls die .tex-Datei \titel etc. nicht gesetzt hat
\providecommand{\titel}{}
\providecommand{\editorInnen}{}
\providecommand{\dateiname}{\jobname}

\vspace{3cm}

\vfill

\footnotesize
\textsc{Quelle}: \titel. Herausgegeben von {\editorInnen}. In: \emph{Arthur Schnitzler: Briefwechsel mit Autorinnen und Autoren}.
 Digitale Edition, https://schnitzler-briefe.acdh.oeaw.ac.at/{\dateiname}.html (Stand \today)
\fi

\end{document}


