%% latex-korrekturansicht-vorspann.tex
%% Vorspann für die Korrekturansicht.
%% Lädt die gemeinsame Datei latex-vorspann.tex mit gesetztem Schalter.

\newif\ifkorrekturansicht
\korrekturansichttrue

\input{../tex-inputs/latex-vorspann}


\section[Arthur Schnitzler an Richard Beer-Hofmann, 24. 8. 1892]{L00118 Arthur Schnitzler an Richard Beer-Hofmann, 24. 8. 1892}
\nopagebreak\mylabel{L00118v}
\rehead{ }\normalsize\beginnumbering\briefempfaengerindex{Beer-Hofmann, Richard@\textsc{Beer-Hofmann, Richard}!zzzSchnitzler, Arthur@\emph{von Arthur Schnitzler}!1892-08-241@{24. 8. 1892}|(be}
\toendnotes[C]{\smallbreak\pagebreak[2]}\Standort{YCGL, MSS 31.}
\physDesc{Brief, 1 Blatt, 3 Seiten, Umschlag, 515 Zeichen
\newline{}Handschrift: Bleistift, deutsche Kurrent
\newline{}Versand: 1) Stempel: »\nobreak{}{\pb}Wien
                                       \textcolor{gray}{4}/1, 24 8 92, 7–8N\nobreak{}«.   2) Stempel: »\nobreak{}\oindex{Bad Ischl@\textbf{Bad Ischl}, \emph{P.PPL}|pwk}Ischl, 25 8 9{[}2{]}, 10 F\nobreak{}«. }
\buchAbdrucke{\weitereDrucke{Arthur Schnitzler, Richard Beer-Hofmann: \emph{Briefwechsel 1891–1931}. Wien, Zürich: \emph{Europaverlag} 1992, S. 38.} }\pstart{}{\pb}Herrn Dr \textsc{Richard Beer
                     Hofmann}\pend{}\pstart{}\textsc{Ischl\oindex{Bad Ischl@\textbf{Bad Ischl}, \emph{P.PPL}|pw}.}\pend{}\pstart{}\textsc{Grazerstraße 6\oindex{Grazer Strasse [Bad Ischl]@\textbf{Grazer Straße [Bad Ischl]}, \emph{Straße (K.STR)}|pw}}.\pend{}{\bigskip}\vspace{1em}
\pstart{}{\pb}Lieber Richard,\pend\vspace{0.5em}
\pstart
           ich theile Ihnen mit, dß ich Samſtag in Iſchl\oindex{Bad Ischl@\textbf{Bad Ischl}, \emph{P.PPL}|pw} eintreffen werde; wo ich wohne, iſt noch nicht beſtimmt – \textsc{Leopold}\oindex{Hotel und Pension Rudolfshoehe (Leopold Petter)@\textbf{Hotel und Pension Rudolfshöhe (Leopold Petter)}, \emph{Hotel (K.HTL)}|pw} wahrſcheinlich – möglich \textsc{Elisabeth}\oindex{Hotel Kaiserin Elisabeth [Bad Ischl]@\textbf{Hotel Kaiserin Elisabeth [Bad Ischl]}, \emph{Hotel (K.HTL)}|pw}. –\pend
           
\pstart
           {\pb}Viele herzliche Grüße bis dahin! – \pend
           
\pstart
           Meine Abſicht iſt es, Touren zu machen; jawohl, lachen Sie nicht; ich brauche
               nothwendig phyſiſche Bewegung, vielleicht ſogar Abmattung, um mich aus einer {\pb}unerträglichen Dumpfheit des Seeliſchen zu retten.\pend
           
\pstart
           Ich freue mich auf Sie, ich hoffe ſogar auf Sie.\pend
           \pstart Ihr \spacefill\mbox{Arthur}\pend{}
\pstart
           24. 8. 92{ }Wien\oindex{Wien@\textbf{Wien}, \emph{A.ADM2}|pw}\pend
           \selectlanguage{ngerman}\endnumbering\briefempfaengerindex{Beer-Hofmann, Richard@\textsc{Beer-Hofmann, Richard}!zzzSchnitzler, Arthur@\emph{von Arthur Schnitzler}!1892-08-241@{24. 8. 1892}|)be}\mylabel{L00118h}  \normalsize

\doendnotes{C}
\bigskip
\vfill

\clearpage

\footnotesize

\lohead{\textsc{register}}

% Definiere theindex-Environment komplett neu ohne reledmac
\makeatletter
\renewenvironment{theindex}{%
  \section*{\indexname}%
  \setlength{\parindent}{0pt}%
  \setlength{\parskip}{0pt plus 0.3pt}%
  \let\item\@idxitem
}{%
  \clearpage
}
\makeatother

\IfFileExists{\jobname-pw.ind}{\input{\jobname-pw.ind}}{}

\end{document}

      