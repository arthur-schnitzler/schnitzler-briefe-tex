%% latex-leseansicht-vorspann.tex
%% Vorspann für die Leseansicht.
%% Lädt die gemeinsame Datei latex-vorspann.tex mit nicht gesetztem Schalter.

\newif\ifkorrekturansicht
\korrekturansichtfalse

\input{../tex-inputs/latex-vorspann}


\section[Arthur Schnitzler an Gerhart Hauptmann, 15. 7. 1899]{L00943 Arthur Schnitzler an Gerhart Hauptmann, 15. 7. 1899}
\nopagebreak\mylabel{L00943v}
\rehead{ }\normalsize\beginnumbering\briefempfaengerindex{Hauptmann, Gerhart@\textsc{Hauptmann, Gerhart}!zzzSchnitzler, Arthur@\emph{von Arthur Schnitzler}!1899-07-151@{15. 7. 1899}|(be}
\toendnotes[C]{\smallbreak\pagebreak[2]}
\correspDesc{Versand  durch Arthur Schnitzler am 15. 7. 1899 in Wien
\newline{}Weiterleitung  durch Otto Brahm in Berlin
\newline{}Erhalt  durch Gerhart Hauptmann im Zeitraum [16. 7. 1899
                  – 20. 7. 1899?] in Szklarska Poręba}\toendnotes[C]{\smallbreak}
\Standort{Staatsbibliothek Berlin – Preußischer Kulturbesitz, GHBrBl A:Schnitzler (4).}
\physDesc{Brief, 2 Blätter, 7 Seiten, 1956 Zeichen
\newline{}Handschrift: schwarze Tinte, deutsche Kurrent}
\buchAbdrucke{\weitereDrucke{1) Arthur Schnitzler: \emph{Briefe 1875–1912}. Herausgegeben von Therese Nickl und Heinrich Schnitzler. Frankfurt am Main: \emph{S. Fischer} 1981, S. 372–373.} \weitereDrucke{2) Hermann Bahr, Arthur Schnitzler: \emph{Briefwechsel, Aufzeichnungen, Dokumente (1891–1931)}. Herausgegeben von Kurt Ifkovits und Martin Anton Müller. Göttingen: \emph{Wallstein} 2018, S. 171.} }\toendnotes[C]{\smallbreak}
\pstart{}{\pb}Verehrteſter Herr Hauptmann,\pend\vspace{0.5em}
\pstart
           die Redaction der Zeit\orgindex{Zeit. Wiener Wochenschrift@Die Zeit. Wiener Wochenschrift|pw}, Singer\pwindex{Singer, Isidor 16.\,1.\,1857 Budapest – 8.\,12.\,1927 Wien@\textsc{Singer, Isidor} (16.\,1.\,1857 Budapest – 8.\,12.\,1927 Wien), \emph{Journalist, Herausgeber, Soziologe}|pw}, wendet{ }ſich mit einem Erſuchen an mich. Bahr\pwindex{Bahr, Hermann 19.\,7.\,1863 Linz – 15.\,1.\,1934 München@\textsc{Bahr, Hermann} (19.\,7.\,1863 Linz – 15.\,1.\,1934 München), \emph{Schriftsteller, Kritiker}|pw} verläßt im October d. J. das Blatt, und
               nun{ }ſoll es nach verſchiedenen Richtungen hin reorganiſirt werden. So wollen die Herausgeber\pwindex{Singer, Isidor 16.\,1.\,1857 Budapest – 8.\,12.\,1927 Wien@\textsc{Singer, Isidor} (16.\,1.\,1857 Budapest – 8.\,12.\,1927 Wien), \emph{Journalist, Herausgeber, Soziologe}|pwv}\pwindex{Kanner, Heinrich 9.\,11.\,1864 Galați – 15.\,2.\,1930 Wien@\textsc{Kanner, Heinrich} (9.\,11.\,1864 Galați – 15.\,2.\,1930 Wien), \emph{Herausgeber, Publizist}|pwv} z. B.
               daſs Hof{\pb}mannsthal\pwindex{Hofmannsthal, Hugo von 1.\,2.\,1874 Wien – 15.\,7.\,1929 Rodaun@\textsc{Hofmannsthal, Hugo von} (1.\,2.\,1874 Wien – 15.\,7.\,1929 Rodaun), \emph{Schriftsteller}|pw}, Burckhard\pwindex{Burckhard, Max Eugen 14.\,7.\,1854 Korneuburg – 16.\,3.\,1912 Wien@\textsc{Burckhard, Max Eugen} (14.\,7.\,1854 Korneuburg – 16.\,3.\,1912 Wien), \emph{Schriftsteller, Rechtswissenschaftler, Theaterleiter}|pw} und ich als{ }ſtändig Mitwirkende{ }ſich nicht nur betheiligen{ }ſondern{ }ſich in dieſer Eigenſchaft
               auch aufs Blatt{ }ſetzen laſſen. Wir hätten Oeſterreich\oindex{Österreich@\textbf{Österreich}|pw} zu vertreten. Was nun Deutſchland\oindex{Deutschland@\textbf{Deutschland}|pw} anbelangt,{ }ſo hätte Prof. Singer\pwindex{Singer, Isidor 16.\,1.\,1857 Budapest – 8.\,12.\,1927 Wien@\textsc{Singer, Isidor} (16.\,1.\,1857 Budapest – 8.\,12.\,1927 Wien), \emph{Journalist, Herausgeber, Soziologe}|pw} keinen lebhaftern Wunſch, als Sie {\pb}in gleicher Weiſe wie uns zu gewinnen. Er wäre
               glücklich, bei irgd einer Gelegenheit etwas von Ihnen zur Veröffentlichung zu beko{\geminationm}en – und wenn Sie nun gar die Erlaubnis gäben, Ihren
               Namen neben die unſern als den eines Mitwirkenden zu{ }ſetzen,{ }ſo glaubt er, daſs damit
               das Weſen und der Geiſt{ }ſeiner Zeitung{ }ſtärker {\pb}ausgedrückt werden könnte, als mit jedem Programm. Er hat mich gebeten, Ihnen das
               zu{ }ſagen; in der Hoffnung, daſs Ihnen perſönliche Beka{\geminationn}tſchaft das Antworten zu einer minder läſtigen Verpflichtung macht. Man wird{ }ſich
               vorläufig an keinen andern Dichter oder Schriftſteller Deutſch{\pb}lands\oindex{Deutschland@\textbf{Deutschland}|pw} wenden, da man im
               Falle einer Zuſage Ihrerſeits jedenfalls auf Ihre Zuſti{\geminationm}ung ev. auch auf Ihre Rathſchläge reflectiren möchte. –\pend
           
\pstart
           Hiemit endet mein Auftrag. Perſönlich{ }ſetze ich lieber nichts hinzu; – daſs Sie in
               keiner{ }ſchlechten Geſellschaft wären,{ }ſehen Sie ja – und gebunden{ }ſind {\pb}Sie in keiner Weiſe.\pend
           
\pstart
           Ich{ }ſende dieſen Brief an Brahm\pwindex{Brahm, Otto 5.\,2.\,1856 Hamburg – 28.\,11.\,1912 Berlin@\textsc{Brahm, Otto} (5.\,2.\,1856 Hamburg – 28.\,11.\,1912 Berlin), \emph{Theaterleiter, Regisseur}|pw} zu
               freundlicher Beförderung, da ich nicht weiſs, wo Sie{ }ſind. Wo immer: ich hoffe Sie
               wohlgeſtimmt und eben daran, neues zu{ }ſchaffen.\pend
           
\pstart
           Von mir kann ich gleiches nicht{ }ſagen; vielleicht daſs der Sommer noch gute Tage \strikeout{ver}bringt.\pend
           
\pstart
           {\pb}– Sie hätten hier eine große Freude gehabt,
               wie die Leute Ihr Friedensfeſt\pwindex{Hauptmann, Gerhart 15.\,11.\,1862 Szczawno-Zdrój – 6.\,6.\,1946 Jagniątków@\textsc{Hauptmann, Gerhart} (15.\,11.\,1862 Szczawno-Zdrój – 6.\,6.\,1946 Jagniątków), \emph{Schriftsteller}!Friedensfest. Eine Familienkatastrophe@\strich\emph{Das Friedensfest. Eine Familienkatastrophe}|pw} aufgenommen
               haben. Beſonders der Schluſs des zweiten Aktes hat mächtig eingeſchlagen. Bekämen wir
               doch hier einmal die Weber\pwindex{Hauptmann, Gerhart 15.\,11.\,1862 Szczawno-Zdrój – 6.\,6.\,1946 Jagniątków@\textsc{Hauptmann, Gerhart} (15.\,11.\,1862 Szczawno-Zdrój – 6.\,6.\,1946 Jagniątków), \emph{Schriftsteller}!Weber. Schauspiel aus den vierziger Jahren@\strich\emph{Die Weber. Schauspiel aus den vierziger Jahren}|pw} zu{ }ſehn.\pend
           
\pstart
           Herzlich grüßt Sie Ihr Ihnen{\\[\baselineskip]}wärmſtens ergebner{\\[\baselineskip]}\spacefill\mbox{Arthur Schnitzler}\pend
           \leftskip=0em{}
\pstart
           15. 7. 99.{\\}IX. Frankgaſſe 1\oindex{Wien@\textbf{Wien}!IX., Alsergrund@\textbf{IX., Alsergrund}!Frankgasse 1@\textbf{Frankgasse 1}, \emph{Wohngebäude}|pw}.\pend
           \selectlanguage{ngerman}\endnumbering\briefempfaengerindex{Hauptmann, Gerhart@\textsc{Hauptmann, Gerhart}!zzzSchnitzler, Arthur@\emph{von Arthur Schnitzler}!1899-07-151@{15. 7. 1899}|)be}\mylabel{L00943h}  \newcommand{\dateiname}{L00943}\newcommand{\titel}{Arthur Schnitzler an Gerhart Hauptmann, 15. 7. 1899}\newcommand{\editorInnen}{Herausgegeben von Martin Anton Müller}%% latex-leseansicht-abspann.tex
%% Abspann für die Leseansicht.
%% Der Schalter \ifkorrekturansicht ist bereits durch den Vorspann gesetzt.

%% latex-abspann.tex
%% Gemeinsamer Abspann für Korrekturansicht und Leseansicht.
%% Setzt den Schalter \ifkorrekturansicht voraus (gesetzt in den
%% einbindenden Dateien latex-korrekturansicht-abspann.tex bzw.
%% latex-leseansicht-abspann.tex).
%% ---------------------------------------------------------------

\normalsize

% Das esempio-Environment wird nur in der Leseansicht benötigt
\ifkorrekturansicht\else
\newenvironment{esempio}[3]%
{
    \vspace{1.5ex}
    \rlap{\underline{#1}}
    \par
    \setlength{\parindent}{0cm}
    \nopagebreak
    \leftskip=#2cm
    \rightskip=#3cm
}
{
    \par
}
\fi

\doendnotes{C}
\bigskip
\vfill

\clearpage

\footnotesize

\ifkorrekturansicht
  \lohead{\textsc{register}}
\fi

% theindex-Environment neu definieren ohne reledmac
\makeatletter
\renewenvironment{theindex}{%
  \ifkorrekturansicht
    \section*{\indexname}%
  \else
    \subsubsection*{Index der erwähnten Entitäten}%
  \fi
  \setlength{\parindent}{0pt}%
  \setlength{\parskip}{0pt plus 0.3pt}%
  \let\item\@idxitem
}{%
  \ifkorrekturansicht\clearpage\fi
}
\makeatother

\IfFileExists{\jobname-pw.ind}{\input{\jobname-pw.ind}}{}

% Quellenangabe nur in der Leseansicht
\ifkorrekturansicht\else
% Fallback-Definitionen, falls die .tex-Datei \titel etc. nicht gesetzt hat
\providecommand{\titel}{}
\providecommand{\editorInnen}{}
\providecommand{\dateiname}{\jobname}

\vspace{3cm}

\vfill

\footnotesize
\textsc{Quelle}: \titel. Herausgegeben von {\editorInnen}. In: \emph{Arthur Schnitzler: Briefwechsel mit Autorinnen und Autoren}.
 Digitale Edition, https://schnitzler-briefe.acdh.oeaw.ac.at/{\dateiname}.html (Stand \today)
\fi

\end{document}


