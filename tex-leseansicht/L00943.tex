%% latex-leseansicht-vorspann.tex
%% Vorspann für die Leseansicht.
%% Lädt die gemeinsame Datei latex-vorspann.tex mit nicht gesetztem Schalter.

\newif\ifkorrekturansicht
\korrekturansichtfalse

\input{../tex-inputs/latex-vorspann}


         
         \newcommand{\erwaehntePersonen}{Personen: Hermann Bahr, Otto Brahm, Max Eugen Burckhard, Gerhart Hauptmann, Hugo von Hofmannsthal, Heinrich Kanner, Isidor Singer}
         \newcommand{\erwaehnteInstitutionen}{Institutionen: Die Zeit. Wiener Wochenschrift}
         \newcommand{\erwaehnteOrte}{Orte: Berlin, Deutschland, Frankgasse, Szklarska Poręba, Wien, Österreich}
         \newcommand{\erwaehnteWerke}{Werke: Das Friedensfest, Die Weber. Schauspiel aus den vierziger Jahren}
               \section[Arthur Schnitzler an Gerhart Hauptmann, 15. 7. 1899]{ Arthur Schnitzler an Gerhart Hauptmann,
                    15. 7. 1899}\nopagebreak\mylabel{v}\rehead{ }\begin{ledgroupsized}[t]{13cm}\normalsize\beginnumbering \toendnotes[C]{\smallbreak\pagebreak[2]} \Standort{Staatsbibliothek Berlin – Preußischer Kulturbesitz, GHBrBl A:Schnitzler (4).}
\physDesc{Brief, 2 Blätter, 7 Seiten
\newline{}Handschrift: schwarze Tinte, deutsche Kurrent}\buchAbdrucke{\weitereDrucke{1) Arthur Schnitzler: \emph{Briefe 1875–1912}. Hg. Therese Nickl und Heinrich Schnitzler. Frankfurt am Main: \emph{S. Fischer} 1981, S. 372–373.} \weitereDrucke{2) Hermann Bahr, Arthur Schnitzler: \emph{Briefwechsel, Aufzeichnungen, Dokumente
                                (1891–1931)}. Hg. Kurt Ifkovits und Martin Anton Müller. Göttingen: \emph{Wallstein} 2018, S. 171.} }\toendnotes[C]{\smallbreak}\pstart{}{\pb}Verehrteſter Herr Hauptmann,\pend\pstart
           die Redaction der Zeit\orgindex{Zeit. Wiener Wochenschrift@Die Zeit. Wiener Wochenschrift|pw}, Singer\pwindex{Singer, Isidor 16.01.1857 – 08.12.1927@\textsc{Singer, Isidor} (16.01.1857 – 08.12.1927), \emph{Journalist, Herausgeber, Soziologe}|pw}, wendet ſich mit einem Erſuchen an mich. Bahr\pwindex{Bahr, Hermann 19.07.1863 – 15.01.1934@\textsc{Bahr, Hermann} (19.07.1863 – 15.01.1934), \emph{Schriftsteller, Kritiker}|pw} verläßt im October d. J. das Blatt, und
                    nun ſoll es nach verſchiedenen Richtungen hin reorganiſirt werden. So wollen die
                        Herausgeber\pwindex{Singer, Isidor 16.01.1857 – 08.12.1927@\textsc{Singer, Isidor} (16.01.1857 – 08.12.1927), \emph{Journalist, Herausgeber, Soziologe}|pwv}\pwindex{Kanner, Heinrich 09.11.1864 – 15.02.1930@\textsc{Kanner, Heinrich} (09.11.1864 – 15.02.1930), \emph{Herausgeber, Publizist}|pwv}
                    z. B. daſs Hof{\pb}mannsthal\pwindex{Hofmannsthal, Hugo von 1874-02-01 – 1929-07-15@\textsc{Hofmannsthal, Hugo von} (1874-02-01 – 1929-07-15), \emph{Schriftsteller}|pw}, Burckhard\pwindex{Burckhard, Max Eugen 14.07.1854 – 16.03.1912@\textsc{Burckhard, Max Eugen} (14.07.1854 – 16.03.1912), \emph{Schriftsteller, Rechtswissenschaftler, Theaterleiter}|pw} und ich als
                    ſtändig Mitwirkende ſich nicht nur betheiligen ſondern ſich in dieſer
                    Eigenſchaft auch aufs Blatt ſetzen laſſen. Wir hätten Oeſterreich\oindex{Oesterreich@\textbf{Österreich}|pw} zu vertreten. Was nun Deutſchland\oindex{Deutschland@\textbf{Deutschland}|pw} anbelangt, ſo hätte Prof. Singer\pwindex{Singer, Isidor 16.01.1857 – 08.12.1927@\textsc{Singer, Isidor} (16.01.1857 – 08.12.1927), \emph{Journalist, Herausgeber, Soziologe}|pw} keinen lebhaftern Wunſch, als Sie {\pb}in gleicher Weiſe wie uns zu gewinnen. Er wäre glücklich,
                    bei irgd einer Gelegenheit etwas von Ihnen zur Veröffentlichung zu beko{\geminationm}en – und wenn Sie nun gar die Erlaubnis gäben,
                    Ihren Namen neben die unſern als den eines Mitwirkenden zu ſetzen, ſo glaubt er,
                    daſs damit das Weſen und der Geiſt ſeiner Zeitung ſtärker {\pb}ausgedrückt werden könnte, als mit jedem Programm. Er hat mich gebeten, Ihnen
                    das zu ſagen; in der Hoffnung, daſs Ihnen perſönliche Beka{\geminationn}tſchaft das Antworten zu einer minder läſtigen
                    Verpflichtung macht. Man wird ſich vorläufig an keinen andern Dichter oder
                    Schriftſteller Deutſch{\pb}land\oindex{Deutschland@\textbf{Deutschland}|pw}s wenden, da man im Falle einer Zuſage Ihrerſeits jedenfalls auf Ihre
                        Zuſti{\geminationm}ung ev. auch auf Ihre Rathſchläge
                    reflectiren möchte. –\pend
           \pstart
           Hiemit endet mein Auftrag. Perſönlich ſetze ich lieber nichts hinzu; – daſs Sie
                    in keiner ſchlechten Geſellschaft wären, ſehen Sie ja – und gebunden ſind {\pb}Sie in keiner Weiſe.\pend
           \pstart
           Ich ſende dieſen Brief an Brahm\pwindex{Brahm, Otto 05.02.1856 – 28.11.1912@\textsc{Brahm, Otto} (05.02.1856 – 28.11.1912), \emph{Theaterleiter, Regisseur}|pw} zu
                    freundlicher Beförderung, da ich nicht weiſs, wo Sie ſind. Wo immer: ich hoffe
                    Sie wohlgeſtimmt und eben daran, neues zu ſchaffen.\pend
           \pstart
           Von mir kann ich gleiches nicht ſagen; vielleicht daſs der Sommer noch gute Tage
                        \strikeout{ver}bringt.\pend
           \pstart
           {\pb}– Sie hätten hier eine große Freude gehabt, wie die Leute
                    Ihr Friedensfeſt\pwindex{Hauptmann, Gerhart 15.11.1862 – 06.06.1946@\textsc{Hauptmann, Gerhart} (15.11.1862 – 06.06.1946), \emph{Schriftsteller}!Friedensfest1890@\strich\emph{Das Friedensfest} {[}1890{]}|pw} aufgenommen haben. Beſonders
                    der Schluſs des zweiten Aktes hat mächtig eingeſchlagen. Bekämen wir doch hier
                    einmal die Weber\pwindex{Hauptmann, Gerhart 15.11.1862 – 06.06.1946@\textsc{Hauptmann, Gerhart} (15.11.1862 – 06.06.1946), \emph{Schriftsteller}!Weber. Schauspiel aus den vierziger Jahren1892@\strich\emph{Die Weber. Schauspiel aus den vierziger Jahren} {[}1892{]}|pw} zu ſehn.\pend
           \pstart
           Herzlich grüßt Sie Ihr Ihnen{\\[\baselineskip]}wärmſtens ergebner{\\[\baselineskip]}\spacefill\mbox{Arthur Schnitzler}\pend
           \leftskip=0em{}\pstart
           15. 7. 99.{\\}IX. Frankgaſſe 1\oindex{Frankgasse@\textbf{Frankgasse}|pw}.
                \pend
           
         
         \endnumbering\mylabel{h}\end{ledgroupsized}  \newcommand{\dateiname}{L00943}\newcommand{\titel}{Arthur Schnitzler an Gerhart Hauptmann, 15. 7. 1899}\newcommand{\editorInnen}{ Martin Anton Müller und Gerd-Hermann Susen}%% latex-leseansicht-abspann.tex
%% Abspann für die Leseansicht.
%% Der Schalter \ifkorrekturansicht ist bereits durch den Vorspann gesetzt.

%% latex-abspann.tex
%% Gemeinsamer Abspann für Korrekturansicht und Leseansicht.
%% Setzt den Schalter \ifkorrekturansicht voraus (gesetzt in den
%% einbindenden Dateien latex-korrekturansicht-abspann.tex bzw.
%% latex-leseansicht-abspann.tex).
%% ---------------------------------------------------------------

\normalsize

% Das esempio-Environment wird nur in der Leseansicht benötigt
\ifkorrekturansicht\else
\newenvironment{esempio}[3]%
{
    \vspace{1.5ex}
    \rlap{\underline{#1}}
    \par
    \setlength{\parindent}{0cm}
    \nopagebreak
    \leftskip=#2cm
    \rightskip=#3cm
}
{
    \par
}
\fi

\doendnotes{C}
\bigskip
\vfill

\clearpage

\footnotesize

\ifkorrekturansicht
  \lohead{\textsc{register}}
\fi

% theindex-Environment neu definieren ohne reledmac
\makeatletter
\renewenvironment{theindex}{%
  \ifkorrekturansicht
    \section*{\indexname}%
  \else
    \subsubsection*{Index der erwähnten Entitäten}%
  \fi
  \setlength{\parindent}{0pt}%
  \setlength{\parskip}{0pt plus 0.3pt}%
  \let\item\@idxitem
}{%
  \ifkorrekturansicht\clearpage\fi
}
\makeatother

\IfFileExists{\jobname-pw.ind}{\input{\jobname-pw.ind}}{}

% Quellenangabe nur in der Leseansicht
\ifkorrekturansicht\else
% Fallback-Definitionen, falls die .tex-Datei \titel etc. nicht gesetzt hat
\providecommand{\titel}{}
\providecommand{\editorInnen}{}
\providecommand{\dateiname}{\jobname}

\vspace{3cm}

\vfill

\footnotesize
\textsc{Quelle}: \titel. Herausgegeben von {\editorInnen}. In: \emph{Arthur Schnitzler: Briefwechsel mit Autorinnen und Autoren}.
 Digitale Edition, https://schnitzler-briefe.acdh.oeaw.ac.at/{\dateiname}.html (Stand \today)
\fi

\end{document}


      