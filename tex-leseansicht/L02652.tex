%% latex-korrekturansicht-vorspann.tex
%% Vorspann für die Korrekturansicht.
%% Lädt die gemeinsame Datei latex-vorspann.tex mit gesetztem Schalter.

\newif\ifkorrekturansicht
\korrekturansichttrue

\input{../tex-inputs/latex-vorspann}


\section[Paul Goldmann an Arthur Schnitzler, 20. 12. 1890]{L02652 Paul Goldmann an Arthur Schnitzler, 20. 12. 1890}
\nopagebreak\mylabel{L02652v}
\rehead{ }\normalsize\beginnumbering\briefempfaengerindex{Schnitzler, Arthur@\textsc{Schnitzler, Arthur}!zzzGoldmann, Paul@\emph{von Paul Goldmann}!1890-12-201@{20. 12. 1890}|(be}
\toendnotes[C]{\smallbreak\pagebreak[2]}\Standort{DLA, A:Schnitzler, HS.NZ85.1.3162.}
\physDesc{Brief, 1 Blatt, 2 Seiten, 821 Zeichen
\newline{}Handschrift: schwarze Tinte, deutsche Kurrent}\toendnotes[C]{\smallbreak}
\pstart
           {\pb}Wien\oindex{Wien@\textbf{Wien}, \emph{A.ADM2}|pw} den \textsuperscript{20}/\textsubscript{12}
                  1890.\pend
           \vspace{0.5em}
\pstart
           Lieber Arthur! Ich ſchreibe dieſe Zeilen in fliegender
               Eile in einem \textsc{Café} auf der Mariahilferſtraße\oindex{Mariahilfer Strasse@\textbf{Mariahilfer Straße}, \emph{Straße (K.STR)}|pw}. Soeben iſt ein ſcharfer Conflict zwiſchen dem \label{K_L02652-1v}\edtext{bisherigen Verleger\pwindex{Eberle, Joseph 1884-08-02 – 1947@\textsc{Eberle, Joseph} (1884-08-02 – 1947), \emph{Journalist/Journalistin, Herausgeber/Herausgeberin}|pwv}}{\lemma{\textnormal{\emph{bisherigen Verleger}}}\Cendnote{\textnormal{Die ersten fünf Jahrgänge von \emph{An der schönen blauen Donau}\orgindex{der schoenen blauen Donau@An der schönen blauen Donau|pwk} wurden von der
                  Druckerei \emph{Josef Eberle}\orgindex{Josef Eberle Stein-, Buch und Musikaliendruckerei@Josef Eberle Stein-, Buch und Musikaliendruckerei|pwk} in der Seidengasse\oindex{Seidengasse@\textbf{Seidengasse}, \emph{Straße (K.STR)}|pwk} nahe der Mariahilferstraße\oindex{Mariahilfer Strasse@\textbf{Mariahilfer Straße}, \emph{Straße (K.STR)}|pwk} hergestellt. Ab dem 6. Jahrgang bzw. ab
                     1891 erschien die Zeitschrift\pwindex{der schoenen blauen Donau@\emph{An der schönen blauen Donau}|pwkv} als Beilage der Tageszeitung \emph{Die Presse}\orgindex{Presse@Die Presse|pwk}, womit diese für die Produktion verantwortlich
                  wurde.}}}\label{K_L02652-1} der »Blauen Donau\orgindex{der schoenen blauen Donau@An der schönen blauen Donau|pw}« und der »Preſſe\orgindex{Presse@Die Presse|pw}« zum Ausbruch gekommen. Erſteren\orgindex{Josef Eberle Stein-, Buch und Musikaliendruckerei@Josef Eberle Stein-, Buch und Musikaliendruckerei|pwv} verärgert die Ausfolgung des
               Materials; ich habe ſoeben mit ihm und ſeinem \label{K_L02652-2v}\edtext{Advocaten\pwindex{?? [Anwalt der Buchdruckerei Eberle, 1891] @\textsc{?? [Anwalt der Buchdruckerei Eberle, 1891]}|pwv}}{\lemma{\textnormal{\emph{Advocaten}}}\Cendnote{\textnormal{nicht identifiziert}}}\label{K_L02652-2} conferirt und
               muß ſofort wieder einer zweiten Conferenz beiwohnen. Theile dies, bitte, deiner Frau
                  Schweſter\pwindex{Hajek, Gisela 20.12.1867 – 03.02.1953@\textsc{Hajek, Gisela} (20.12.1867 – 03.02.1953)|pwv} u. Deinem
                  Herr\textcolor{gray}{n}{ }Schwager\pwindex{Hajek, Markus 25.11.1861 – 04.04.1941@\textsc{Hajek, Markus} (25.11.1861 – 04.04.1941), \emph{Mediziner/Medizinerin, Laryngologe/Laryngologin}|pwv} – unter Discretion –
               mit! Unter dieſen Umſtänden {\pb}werden ſie mein
               Nichterſcheinen wohl entſchuldigen. Ich bedaure unendlich, daß mir die Freude
               verſtört wir{[}d{]}, dieſen Abend bei ihnen zubringen zu können. Und
               wie verſtört! Näheres mündlich!\pend
           
\pstart
           Ich habe auch nicht früher ſchreiben können, weil ſich die ganze Geſchichte erſt um
                  7 Uhr Abends begeben hat\textcolor{gray}{.}\pend
           
\pstart
           Viele Grüße!{\\[\baselineskip]}Dein{\\[\baselineskip]}\spacefill\mbox{Paul.}\pend
           \leftskip=0em{}\selectlanguage{ngerman}\endnumbering\briefempfaengerindex{Schnitzler, Arthur@\textsc{Schnitzler, Arthur}!zzzGoldmann, Paul@\emph{von Paul Goldmann}!1890-12-201@{20. 12. 1890}|)be}\mylabel{L02652h}  \normalsize

\doendnotes{C}
\bigskip
\vfill

\clearpage

\footnotesize

\lohead{\textsc{register}}

% Definiere theindex-Environment komplett neu ohne reledmac
\makeatletter
\renewenvironment{theindex}{%
  \section*{\indexname}%
  \setlength{\parindent}{0pt}%
  \setlength{\parskip}{0pt plus 0.3pt}%
  \let\item\@idxitem
}{%
  \clearpage
}
\makeatother

\IfFileExists{\jobname-pw.ind}{\input{\jobname-pw.ind}}{}

\end{document}

      