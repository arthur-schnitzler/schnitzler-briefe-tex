%% latex-leseansicht-vorspann.tex
%% Vorspann für die Leseansicht.
%% Lädt die gemeinsame Datei latex-vorspann.tex mit nicht gesetztem Schalter.

\newif\ifkorrekturansicht
\korrekturansichtfalse

\input{../tex-inputs/latex-vorspann}

\begin{center}
            \textcolor{red}{ENTWURF, NICHT FERTIG KORRIGIERT}
                      \end{center}
            
               \section[Paul Goldmann an Arthur Schnitzler, 20. 12. 1890]{ Paul Goldmann an Arthur Schnitzler, 20. 12. 1890}\nopagebreak\mylabel{v}\rehead{ }\begin{ledgroupsized}[t]{13cm}\normalsize\beginnumbering\briefempfaengerindex{Schnitzler, Arthur@\textsc{Schnitzler, Arthur}!zzzGoldmann, Paul@\emph{von Paul Goldmann}!1890-12-201@{20. 12. 1890}|(be} \toendnotes[C]{\smallbreak\pagebreak[2]} \Standort{DLA, A:Schnitzler, HS.NZ85.1.3162.}
\physDesc{Brief, 1 Blatt, 2 Seiten
\newline{}Handschrift: schwarze Tinte, deutsche Kurrent}\toendnotes[C]{\smallbreak}\pstart
           {\pb}Wien\oindex{Wien@\textbf{Wien}|pw} den \textsuperscript{20}/\textsubscript{12}
                  1890.\pend
           \pstart
           Lieber Arthur! Ich ſchreibe dieſe Zeilen in fliegender
               Eile in einem \textsc{Café} auf der Mariahilferſtraße\oindex{Mariahilferstrasse@\textbf{Mariahilferstraße}|pw}. Soeben iſt ein ſcharfer Conflict zwiſchen dem \label{K_L02652-1v}\edtext{bisherigen Verleger\pwindex{Eberle, Joseph 1884 – 1947@\textsc{Eberle, Joseph} (1884 – 1947), \emph{Journalist, Herausgeber}|pwv}}{\lemma{\textnormal{\emph{bisherigen Verleger}}}\Cendnote{\textnormal{Die ersten fünf Jahrgänge von \emph{An der schönen blauen Donau}\orgindex{der schoenen blauen Donau@An der schönen blauen Donau|pwk} wurden von der
                  Druckerei \emph{Josef Eberle}\orgindex{Josef Eberle Stein-, Buch und Musikaliendruckerei@Josef Eberle Stein-, Buch und Musikaliendruckerei|pwk} in der Seidengasse\oindex{Seidengasse@\textbf{Seidengasse}|pwk} nahe der Mariahilferſtraße\oindex{Mariahilferstrasse@\textbf{Mariahilferstraße}|pwk} hergestellt. Mit dem 6. Jahrgang übernahm ab
                     1891 die Druckerei der Tageszeitung \emph{Die Presse}\orgindex{Presse@Die Presse|pwk} die Produktion.}}}\label{K_L02652-1h} der »Blauen Donau\orgindex{der schoenen blauen Donau@An der schönen blauen Donau|pw}« und der »Preſſe\orgindex{Presse@Die Presse|pw}« zum
               Ausbruch gekommen. Erſteren\orgindex{Josef Eberle Stein-, Buch und Musikaliendruckerei@Josef Eberle Stein-, Buch und Musikaliendruckerei|pwv}
               verärgert die Ausfolgung des Materials; ich habe ſoeben mit ihm und ſeinem \label{K_L02652-2v}\edtext{Advocaten\pwindex{?? [Anwalt der Buchdruckerei Eberle, 1891] @\textsc{?? [Anwalt der Buchdruckerei Eberle, 1891]}|pwv}}{\lemma{\textnormal{\emph{Advocaten}}}\Cendnote{\textnormal{nicht identifiziert}}}\label{K_L02652-2h} conferirt und
               muß ſofort wieder einer zweiten Conferenz beiwohnen. Theile dies, bitte, deiner Frau
                  Schweſter\pwindex{Hajek, Gisela 20.12.1867 – 03.02.1953@\textsc{Hajek, Gisela} (20.12.1867 – 03.02.1953)|pwv} u. Deinem Herrn
                  Schwager\pwindex{Hajek, Markus 25.11.1861 – 04.04.1941@\textsc{Hajek, Markus} (25.11.1861 – 04.04.1941), \emph{Mediziner, Laryngologe}|pwv} – unter
               Discretion – mit! Unter dieſen Umſtänden {\pb}werden ſie mein Nichterſcheinen
               wohl entſchuldigen. Ich bedaure unendlich, daß mir die Freude verſtört wird, dieſen
               Abend bei ihnen zubringen zu können. Und wie verſtört! Näheres mündlich!\pend
           \pstart
           Ich habe auch nicht früher ſchreiben können, weil ſich die ganze Geſchichte erſt um
                  7 Uhr Abends begeben hat.\pend
           \pstart
           Viele Grüße!{\\[\baselineskip]}Dein{\\[\baselineskip]}\spacefill\mbox{Paul.}\pend
           \leftskip=0em{}          \endnumbering\briefempfaengerindex{Schnitzler, Arthur@\textsc{Schnitzler, Arthur}!zzzGoldmann, Paul@\emph{von Paul Goldmann}!1890-12-201@{20. 12. 1890}|)be}\mylabel{h}\end{ledgroupsized}  \newcommand{\dateiname}{L02652}\newcommand{\titel}{Paul Goldmann an Arthur Schnitzler, 20. 12. 1890}\newcommand{\editorInnen}{Martin Anton Müller und Laura Untner}
            \footnotesize
\begin{ledgroupsized}[t]{11.5cm}
\doendnotes{C}
\end{ledgroupsized}
         %% latex-leseansicht-abspann.tex
%% Abspann für die Leseansicht.
%% Der Schalter \ifkorrekturansicht ist bereits durch den Vorspann gesetzt.

%% latex-abspann.tex
%% Gemeinsamer Abspann für Korrekturansicht und Leseansicht.
%% Setzt den Schalter \ifkorrekturansicht voraus (gesetzt in den
%% einbindenden Dateien latex-korrekturansicht-abspann.tex bzw.
%% latex-leseansicht-abspann.tex).
%% ---------------------------------------------------------------

\normalsize

% Das esempio-Environment wird nur in der Leseansicht benötigt
\ifkorrekturansicht\else
\newenvironment{esempio}[3]%
{
    \vspace{1.5ex}
    \rlap{\underline{#1}}
    \par
    \setlength{\parindent}{0cm}
    \nopagebreak
    \leftskip=#2cm
    \rightskip=#3cm
}
{
    \par
}
\fi

\doendnotes{C}
\bigskip
\vfill

\clearpage

\footnotesize

\ifkorrekturansicht
  \lohead{\textsc{register}}
\fi

% theindex-Environment neu definieren ohne reledmac
\makeatletter
\renewenvironment{theindex}{%
  \ifkorrekturansicht
    \section*{\indexname}%
  \else
    \subsubsection*{Index der erwähnten Entitäten}%
  \fi
  \setlength{\parindent}{0pt}%
  \setlength{\parskip}{0pt plus 0.3pt}%
  \let\item\@idxitem
}{%
  \ifkorrekturansicht\clearpage\fi
}
\makeatother

\IfFileExists{\jobname-pw.ind}{\input{\jobname-pw.ind}}{}

% Quellenangabe nur in der Leseansicht
\ifkorrekturansicht\else
% Fallback-Definitionen, falls die .tex-Datei \titel etc. nicht gesetzt hat
\providecommand{\titel}{}
\providecommand{\editorInnen}{}
\providecommand{\dateiname}{\jobname}

\vspace{3cm}

\vfill

\footnotesize
\textsc{Quelle}: \titel. Herausgegeben von {\editorInnen}. In: \emph{Arthur Schnitzler: Briefwechsel mit Autorinnen und Autoren}.
 Digitale Edition, https://schnitzler-briefe.acdh.oeaw.ac.at/{\dateiname}.html (Stand \today)
\fi

\end{document}


      