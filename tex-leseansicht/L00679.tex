%% latex-leseansicht-vorspann.tex
%% Vorspann für die Leseansicht.
%% Lädt die gemeinsame Datei latex-vorspann.tex mit nicht gesetztem Schalter.

\newif\ifkorrekturansicht
\korrekturansichtfalse

\input{../tex-inputs/latex-vorspann}


         
         \renewcommand{\erwaehntePersonen}{Personen: Richard Beer-Hofmann, Paul Goldmann, Hugo von Hofmannsthal, Hugo August von Hofmannsthal, Anna von Hofmannsthal, Felix Markbreiter}
         \renewcommand{\erwaehnteOrte}{Orte: Bad Ischl, Bayreuth, Honor Oak, London, Paris, Wien}
         \renewcommand{\erwaehnteWerke}{}
               \section[Arthur Schnitzler an Hugo von Hofmannsthal, 20. 5. 1897]{ Arthur Schnitzler an Hugo von Hofmannsthal, 20. 5. 1897}\nopagebreak\mylabel{v}\rehead{ }\begin{ledgroupsized}[t]{13cm}\normalsize\beginnumbering \toendnotes[C]{\smallbreak\pagebreak[2]} \Standort{FDH, Hs-30885,12.}
\physDesc{Brief, 1 Blatt, 4 Seiten
\newline{}Handschrift: schwarze Tinte, deutsche Kurrent}\buchAbdrucke{\weitereDrucke{Hugo von Hofmannsthal, Arthur Schnitzler: \emph{Briefwechsel}. Hg. Therese Nickl und Heinrich Schnitzler. Frankfurt am Main: \emph{S. Fischer} 1964, S. 86–87.} }\toendnotes[C]{\smallbreak}\pstart
           \raggedleft{}{\pb}\textsc{Paris}\oindex{Paris@\textbf{Paris}|pw}{ }20. 5. 97\pend
           \pstart
           Mein lieber Hugo, Sagen Sie, haben Sie alle meine Briefe
                    bekommen? Dieſer iſt der \uline{vierte}.\pend
           \pstart
           Ich reiſe Montag von hier nach London\oindex{London@\textbf{London}|pw}; meine
                    Adreſſe dort: bei \textsc{Felix Markbreiter\pwindex{Markbreiter, Felix 20.11.1855 – 15.09.1914@\textsc{Markbreiter, Felix} (20.11.1855 – 15.09.1914), \emph{Kaufmann}|pw}, London S. E. Woodville Hall, Honor Oak\oindex{Honor Oak@\textbf{Honor Oak}|pw}.}\pend
           \pstart
           Um den erſten herum bin ich in Wien\oindex{Wien@\textbf{Wien}|pw}.
                    Es war ſehr geſcheit, daſs ich fortgefahren bin; für {\pb}das gegenwärtige ſicher; aber es wird ſicher auch für die Zukunft was zu
                    bedeuten ha\substVorne{}\textsuperscript{tt}\substDazwischen{}b\substHinten{}en, wenn nicht alles Erleben Unſinn iſt. Man weiſs ja nie, was man von
                    irgendwoher mitni{\geminationm}t; wenn man den Koffer auspackt,
                    ſo wundert man ſich über die ſchönen Dinge, die man ſich gar nicht mehr erinnern
                        {\pb}kann hineingeſtopft zu haben.\pend
           \pstart
           – Ich freue mich ſehr, dſs ich Sie noch in Wien\oindex{Wien@\textbf{Wien}|pw}
                    finde. Werden wir miteinander Radfahren? – – Rieſengebirge? Und wie wär es im Auguſt mit ein
                    paar Bayreuth\oindex{Bayreuth@\textbf{Bayreuth}|pw}er Tagen? Goldmann\pwindex{Goldmann, Paul 31.01.1865 – 25.09.1935@\textsc{Goldmann, Paul} (31.01.1865 – 25.09.1935), \emph{Schriftsteller, Journalist}|pw} wird wohl nach Iſchl\oindex{Bad Ischl@\textbf{Bad Ischl}|pw} kommen, möchte auch gern nach Bay{\pb}reuth\oindex{Bayreuth@\textbf{Bayreuth}|pw}. Bitte ſagen Sie das dem Richard\pwindex{Beer-Hofmann, Richard 1866-07-11 – 1945-09-26@\textsc{Beer-Hofmann, Richard} (1866-07-11 – 1945-09-26), \emph{Schriftsteller}|pw}, ich hab vergeſſen ihm das zu
                    ſchreiben. –\pend
           \pstart
           – Nach dem Arbeiten glaub ich hab ich mich in meinem ganzen Leben nicht ſo
                    geſehnt wie jetzt! –\pend
           \pstart
           Bitte grüßen Sie Ihre Eltern\pwindex{Hofmannsthal, Hugo August von 21.12.1841 – 08.12.1915@\textsc{Hofmannsthal, Hugo August von} (21.12.1841 – 08.12.1915), \emph{Bankdirektor}|pwv}\pwindex{Hofmannsthal, Anna von 27.01.1849 – 22.03.1904@\textsc{Hofmannsthal, Anna von} (27.01.1849 – 22.03.1904)|pwv} von mir.\pend
           \pstart
           Herzlich der Ihre{\\[\baselineskip]}\spacefill\mbox{Arthur.}\pend
           \leftskip=0em{}
         
         \endnumbering\mylabel{h}\end{ledgroupsized}  \newcommand{\dateiname}{L00679}\newcommand{\titel}{Arthur Schnitzler an Hugo von Hofmannsthal, 20. 5. 1897}\newcommand{\editorInnen}{Martin Anton Müller und Gerd-Hermann Susen}%% latex-leseansicht-abspann.tex
%% Abspann für die Leseansicht.
%% Der Schalter \ifkorrekturansicht ist bereits durch den Vorspann gesetzt.

%% latex-abspann.tex
%% Gemeinsamer Abspann für Korrekturansicht und Leseansicht.
%% Setzt den Schalter \ifkorrekturansicht voraus (gesetzt in den
%% einbindenden Dateien latex-korrekturansicht-abspann.tex bzw.
%% latex-leseansicht-abspann.tex).
%% ---------------------------------------------------------------

\normalsize

% Das esempio-Environment wird nur in der Leseansicht benötigt
\ifkorrekturansicht\else
\newenvironment{esempio}[3]%
{
    \vspace{1.5ex}
    \rlap{\underline{#1}}
    \par
    \setlength{\parindent}{0cm}
    \nopagebreak
    \leftskip=#2cm
    \rightskip=#3cm
}
{
    \par
}
\fi

\doendnotes{C}
\bigskip
\vfill

\clearpage

\footnotesize

\ifkorrekturansicht
  \lohead{\textsc{register}}
\fi

% theindex-Environment neu definieren ohne reledmac
\makeatletter
\renewenvironment{theindex}{%
  \ifkorrekturansicht
    \section*{\indexname}%
  \else
    \subsubsection*{Index der erwähnten Entitäten}%
  \fi
  \setlength{\parindent}{0pt}%
  \setlength{\parskip}{0pt plus 0.3pt}%
  \let\item\@idxitem
}{%
  \ifkorrekturansicht\clearpage\fi
}
\makeatother

\IfFileExists{\jobname-pw.ind}{\input{\jobname-pw.ind}}{}

% Quellenangabe nur in der Leseansicht
\ifkorrekturansicht\else
% Fallback-Definitionen, falls die .tex-Datei \titel etc. nicht gesetzt hat
\providecommand{\titel}{}
\providecommand{\editorInnen}{}
\providecommand{\dateiname}{\jobname}

\vspace{3cm}

\vfill

\footnotesize
\textsc{Quelle}: \titel. Herausgegeben von {\editorInnen}. In: \emph{Arthur Schnitzler: Briefwechsel mit Autorinnen und Autoren}.
 Digitale Edition, https://schnitzler-briefe.acdh.oeaw.ac.at/{\dateiname}.html (Stand \today)
\fi

\end{document}


      