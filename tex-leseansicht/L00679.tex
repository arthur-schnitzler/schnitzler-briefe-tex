%% latex-leseansicht-vorspann.tex
%% Vorspann für die Leseansicht.
%% Lädt die gemeinsame Datei latex-vorspann.tex mit nicht gesetztem Schalter.

\newif\ifkorrekturansicht
\korrekturansichtfalse

\input{../tex-inputs/latex-vorspann}


\section[Arthur Schnitzler an Hugo von Hofmannsthal, 20. 5. 1897]{L00679 Arthur Schnitzler an Hugo von Hofmannsthal, 20. 5. 1897}
\nopagebreak\mylabel{L00679v}
\rehead{ }\normalsize\beginnumbering\briefempfaengerindex{Hofmannsthal, Hugo von@\textsc{Hofmannsthal, Hugo von}!zzzSchnitzler, Arthur@\emph{von Arthur Schnitzler}!1897-05-202@{20. 5. 1897}|(be}
\toendnotes[C]{\smallbreak\pagebreak[2]}
\correspDesc{Versand  durch Arthur Schnitzler am 20. 5. 1897 in Paris
\newline{}Erhalt  durch Hugo von Hofmannsthal im Zeitraum [21. 5. 1897
                  – 25. 5. 1897?] in Wien}\toendnotes[C]{\smallbreak}
\Standort{FDH, Hs-30885,12.}
\physDesc{Brief, 1 Blatt, 4 Seiten, 1075 Zeichen
\newline{}Handschrift: schwarze Tinte, deutsche Kurrent}
\buchAbdrucke{\weitereDrucke{Hugo von Hofmannsthal, Arthur Schnitzler: \emph{Briefwechsel}. Herausgegeben von Therese Nickl und Heinrich Schnitzler. Frankfurt am Main: \emph{S. Fischer} 1964, S. 86–87.} }\toendnotes[C]{\smallbreak}
\pstart
           \raggedleft{}{\pb}\textsc{Paris}\oindex{Paris@\textbf{Paris}, \emph{Hauptstadt}|pw}{ }20. 5. 97\pend
           \vspace{0.5em}
\pstart
           Mein lieber Hugo, Sagen Sie, haben Sie alle meine Briefe bekommen?
               Dieſer iſt der \uline{\label{K_L00679-1v}\edtext{vierte}{\lemma{\textnormal{\emph{vierte}}}\Cendnote{\textnormal{Vgl. XXXX Auszeichnungsfehler: Dokument L00671 nicht gefunden, XXXX Auszeichnungsfehler: Dokument L00672 nicht gefunden und
                  XXXX Auszeichnungsfehler: Dokument L00674 nicht gefunden.
                  }}}\label{K_L00679-1}}.\pend
           
\pstart
           Ich reiſe Montag von hier nach London\oindex{London@\textbf{London}, \emph{Hauptstadt}|pw}; meine
               Adreſſe dort: bei \textsc{Felix Markbreiter\pwindex{Markbreiter, Felix 20.\,11.\,1855 Wien – 15.\,9.\,1914 London@\textsc{Markbreiter, Felix} (20.\,11.\,1855 Wien – 15.\,9.\,1914 London), \emph{Kaufmann}|pw}, London S. E. Woodville Hall, Honor Oak\oindex{Honor Oak@\textbf{Honor Oak}, \emph{Teil eines besiedelten Ortes}|pw}.}\pend
           
\pstart
           Um den erſten herum bin ich in Wien\oindex{Wien@\textbf{Wien}, \emph{Verwaltungsgebiet}|pw}.
               Es war{ }ſehr geſcheit, daſs ich fortgefahren bin; für {\pb}das
               gegenwärtige{ }ſicher; aber es wird{ }ſicher auch für die Zukunft was zu bedeuten ha\substVorne{}\textsuperscript{tt}\substDazwischen{}b\substHinten{}en, wenn nicht alles Erleben Unſinn iſt. Man weiſs ja nie, was man von
               irgendwoher mitni{\geminationm}t; wenn man den Koffer auspackt,{ }ſo
               wundert man{ }ſich über die{ }ſchönen Dinge, die man{ }ſich gar nicht mehr erinnern {\pb}kann hineingeſtopft zu haben.\pend
           
\pstart
           – Ich freue mich{ }ſehr, dſs ich Sie noch in Wien\oindex{Wien@\textbf{Wien}, \emph{Verwaltungsgebiet}|pw}
               finde. Werden wir miteinander Radfahren? – – Rieſengebirge\oindex{Riesengebirge@\textbf{Riesengebirge}, \emph{Gebirge}|pw}? Und wie wär es im Auguſt mit ein paar Bayreuth\oindex{Bayreuth@\textbf{Bayreuth}, \emph{Hauptstadt}|pw}er Tagen? Goldmann\pwindex{Goldmann, Paul 31.\,1.\,1865 Breslau – 25.\,9.\,1935 Wien@\textsc{Goldmann, Paul} (31.\,1.\,1865 Breslau – 25.\,9.\,1935 Wien), \emph{Schriftsteller, Journalist}|pw} wird wohl nach Iſchl\oindex{Bad Ischl@\textbf{Bad Ischl}|pw} kommen, möchte auch gern nach Bay{\pb}reuth\oindex{Bayreuth@\textbf{Bayreuth}, \emph{Hauptstadt}|pw}. Bitte{ }ſagen Sie das dem Richard\pwindex{Beer-Hofmann, Richard 11.\,7.\,1866 Wien – 26.\,9.\,1945 New York City@\textsc{Beer-Hofmann, Richard} (11.\,7.\,1866 Wien – 26.\,9.\,1945 New York City), \emph{Schriftsteller}|pw}, ich hab vergeſſen ihm das zu{ }ſchreiben. –\pend
           
\pstart
           – Nach dem Arbeiten glaub ich hab ich mich in meinem ganzen Leben nicht{ }ſo geſehnt
               wie jetzt! –\pend
           
\pstart
           Bitte grüßen Sie Ihre Eltern\pwindex{Hofmannsthal, Hugo August von 21.\,12.\,1841 Wien – 8.\,12.\,1915 ebd.@\textsc{Hofmannsthal, Hugo August von} (21.\,12.\,1841 Wien – 8.\,12.\,1915 ebd.), \emph{Bankdirektor}|pwv}\pwindex{Hofmannsthal, Anna von 27.\,1.\,1849 Wien – 22.\,3.\,1904 Sanatorium Fürth@\textsc{Hofmannsthal, Anna von} (27.\,1.\,1849 Wien – 22.\,3.\,1904 Sanatorium Fürth)|pwv} von mir.\pend
           
\pstart
           Herzlich der Ihre{\\[\baselineskip]}\spacefill\mbox{Arthur.}\pend
           \leftskip=0em{}\selectlanguage{ngerman}\endnumbering\briefempfaengerindex{Hofmannsthal, Hugo von@\textsc{Hofmannsthal, Hugo von}!zzzSchnitzler, Arthur@\emph{von Arthur Schnitzler}!1897-05-202@{20. 5. 1897}|)be}\mylabel{L00679h}  \newcommand{\dateiname}{L00679}\newcommand{\titel}{Arthur Schnitzler an Hugo von Hofmannsthal, 20. 5. 1897}\newcommand{\editorInnen}{Martin Anton Müller und Gerd-Hermann Susen}%% latex-leseansicht-abspann.tex
%% Abspann für die Leseansicht.
%% Der Schalter \ifkorrekturansicht ist bereits durch den Vorspann gesetzt.

%% latex-abspann.tex
%% Gemeinsamer Abspann für Korrekturansicht und Leseansicht.
%% Setzt den Schalter \ifkorrekturansicht voraus (gesetzt in den
%% einbindenden Dateien latex-korrekturansicht-abspann.tex bzw.
%% latex-leseansicht-abspann.tex).
%% ---------------------------------------------------------------

\normalsize

% Das esempio-Environment wird nur in der Leseansicht benötigt
\ifkorrekturansicht\else
\newenvironment{esempio}[3]%
{
    \vspace{1.5ex}
    \rlap{\underline{#1}}
    \par
    \setlength{\parindent}{0cm}
    \nopagebreak
    \leftskip=#2cm
    \rightskip=#3cm
}
{
    \par
}
\fi

\doendnotes{C}
\bigskip
\vfill

\clearpage

\footnotesize

\ifkorrekturansicht
  \lohead{\textsc{register}}
\fi

% theindex-Environment neu definieren ohne reledmac
\makeatletter
\renewenvironment{theindex}{%
  \ifkorrekturansicht
    \section*{\indexname}%
  \else
    \subsubsection*{Index der erwähnten Entitäten}%
  \fi
  \setlength{\parindent}{0pt}%
  \setlength{\parskip}{0pt plus 0.3pt}%
  \let\item\@idxitem
}{%
  \ifkorrekturansicht\clearpage\fi
}
\makeatother

\IfFileExists{\jobname-pw.ind}{\input{\jobname-pw.ind}}{}

% Quellenangabe nur in der Leseansicht
\ifkorrekturansicht\else
% Fallback-Definitionen, falls die .tex-Datei \titel etc. nicht gesetzt hat
\providecommand{\titel}{}
\providecommand{\editorInnen}{}
\providecommand{\dateiname}{\jobname}

\vspace{3cm}

\vfill

\footnotesize
\textsc{Quelle}: \titel. Herausgegeben von {\editorInnen}. In: \emph{Arthur Schnitzler: Briefwechsel mit Autorinnen und Autoren}.
 Digitale Edition, https://schnitzler-briefe.acdh.oeaw.ac.at/{\dateiname}.html (Stand \today)
\fi

\end{document}


