%% latex-korrekturansicht-vorspann.tex
%% Vorspann für die Korrekturansicht.
%% Lädt die gemeinsame Datei latex-vorspann.tex mit gesetztem Schalter.

\newif\ifkorrekturansicht
\korrekturansichttrue

\input{../tex-inputs/latex-vorspann}


\section[Arthur Schnitzler an Richard Beer-Hofmann, 9. 5. 1905]{L01515 Arthur Schnitzler an Richard Beer-Hofmann, 9. 5. 1905}
\nopagebreak\mylabel{L01515v}
\rehead{ }\normalsize\beginnumbering\briefempfaengerindex{Beer-Hofmann, Richard@\textsc{Beer-Hofmann, Richard}!zzzSchnitzler, Arthur@\emph{von Arthur Schnitzler}!1905-05-091@{9. 5. 1905}|(be}
\toendnotes[C]{\smallbreak\pagebreak[2]}\Standort{YCGL, MSS 31.}
\physDesc{Brief, 1 Blatt, 1 Seite, Umschlag, 214 Zeichen
\newline{}Handschrift: schwarze Tinte, deutsche Kurrent
\newline{}Versand: 1) Stempel: »\nobreak{}\oindex{XVIII., Waehring@\textbf{XVIII., Währing}, \emph{A.ADM3}|pwk}18/1 Wien 110, 10. V. 05, X\nobreak{}«.   2) Stempel: »\nobreak{}\oindex{Rodaun@\textbf{Rodaun}, \emph{A.ADM4}|pwk}{\pb}Ro\textcolor{gray}{da}un, 10. 5. 05, 2–4N\nobreak{}«. }
\buchAbdrucke{\weitereDrucke{Arthur Schnitzler, Richard Beer-Hofmann: \emph{Briefwechsel 1891–1931}. Wien, Zürich: \emph{Europaverlag} 1992, S. 172.} }\toendnotes[C]{\smallbreak}\pstart{}{\pb}\textcolor{gray}{\textbf{Dr. Arthur Schnitzler}}\pend{}\pstart{}\textcolor{gray}{\textbf{Wien, XVIII Spoettelgasse 7\oindex{Edmund-Weiss-Gasse 7@\textbf{Edmund-Weiß-Gasse 7}, \emph{Wohngebäude (K.WHS)}|pw}.}}\pend{}{\bigskip}\pstart{}{\pb}Herrn \textsc{Dr Richard
                     Beer-Hofmann}\pend{}\pstart{}\textsc{Rodaun}\oindex{Rodaun@\textbf{Rodaun}, \emph{A.ADM4}|pw}\pend{}\pstart{}\textsc{Liesingerstr. 2}\oindex{Liesingerstrasse@\textbf{Liesingerstraße}, \emph{Straße (K.STR)}|pw}\pend{}{\bigskip}\vspace{1em}
\pstart
           \raggedleft{}{\pb}Wien\oindex{Wien@\textbf{Wien}, \emph{A.ADM2}|pw}, 9. 5. 905\pend
           \vspace{0.5em}
\pstart
           lieber Richard, heute erſt erfahre ich daſs Ihr \label{K_L01515-1v}\edtext{Onkel\pwindex{Beer, Sigmund 1826-09-14 – 1905-05-07@\textsc{Beer, Sigmund} (1826-09-14 – 1905-05-07)|pwv}}{\lemma{\textnormal{\emph{Onkel}}}\Cendnote{\textnormal{Sigmund Beer\pwindex{Beer, Sigmund 1826-09-14 – 1905-05-07@\textsc{Beer, Sigmund} (1826-09-14 – 1905-05-07)|pwk} war am 7. 5. 1905
                  gestorben. Die Beerdigung fand am 9. 5. 1905 statt.}}}\label{K_L01515-1} geſtorben
               iſt. Meine herzlichſte Theilnahme! Und hoffentlich auf ſehr baldiges Wiederſehn.\pend
           
\pstart
           Ihr{\\[\baselineskip]}\spacefill\mbox{Arthur.}\pend
           \leftskip=0em{}\selectlanguage{ngerman}\endnumbering\briefempfaengerindex{Beer-Hofmann, Richard@\textsc{Beer-Hofmann, Richard}!zzzSchnitzler, Arthur@\emph{von Arthur Schnitzler}!1905-05-091@{9. 5. 1905}|)be}\mylabel{L01515h}  \normalsize

\doendnotes{C}
\bigskip
\vfill

\clearpage

\footnotesize

\lohead{\textsc{register}}

% Definiere theindex-Environment komplett neu ohne reledmac
\makeatletter
\renewenvironment{theindex}{%
  \section*{\indexname}%
  \setlength{\parindent}{0pt}%
  \setlength{\parskip}{0pt plus 0.3pt}%
  \let\item\@idxitem
}{%
  \clearpage
}
\makeatother

\IfFileExists{\jobname-pw.ind}{\input{\jobname-pw.ind}}{}

\end{document}

      