%% latex-korrekturansicht-vorspann.tex
%% Vorspann für die Korrekturansicht.
%% Lädt die gemeinsame Datei latex-vorspann.tex mit gesetztem Schalter.

\newif\ifkorrekturansicht
\korrekturansichttrue

\input{../tex-inputs/latex-vorspann}


\section[Stefan Zweig an Arthur Schnitzler, {[}zwischen 5. und 10. 2. 1915{]}]{L03651 Stefan Zweig an Arthur Schnitzler, {[}zwischen 5. und 10. 2. 1915{]}}
\nopagebreak\mylabel{L03651v}
\rehead{ }\normalsize\beginnumbering\briefempfaengerindex{Schnitzler, Arthur@\textsc{Schnitzler, Arthur}!zzzZweig, Stefan@\emph{von Stefan Zweig}!1915-02-101@{{[}zwischen 5. und 10. 2. 1915{]}}|(be}
\toendnotes[C]{\smallbreak\pagebreak[2]}\Standort{CUL, Schnitzler, B 118.}
\physDesc{Briefkarte, 1038 Zeichen
\newline{}Handschrift: lila Tinte, lateinische Kurrent}
\buchAbdrucke{\weitereDrucke{Stefan Zweig: \emph{Briefwechsel mit Hermann Bahr, Sigmund Freud, Rainer Maria
                        Rilke und Arthur Schnitzler}. Frankfurt am Main: \emph{S. Fischer} 1987, S. 390–391.} }\toendnotes[C]{\smallbreak}
\pstart
           {\pb}\textcolor{gray}{\textbf{SZ}}\hfill \textcolor{gray}{\textbf{VIII. KOCHGASSE 8\oindex{Kochgasse 8@\textbf{Kochgasse 8}, \emph{Wohngebäude (K.WHS)}|pw}}}\pend
           \vspace{0.5em}
\pstart
           Verehrter lieber Herr Doktor,{ }Romain Rolland\pwindex{Rolland, Romain 29.01.1866 – 30.12.1944@\textsc{Rolland, Romain} (29.01.1866 – 30.12.1944), \emph{Schriftsteller/Schriftstellerin}|pw} hat nur endlich wieder einen
                  \label{K_L03651-1v}\edtext{Brief}{\lemma{\textnormal{\emph{Brief}}}\Cendnote{\textnormal{Romain Rolland\pwindex{Rolland, Romain 29.01.1866 – 30.12.1944@\textsc{Rolland, Romain} (29.01.1866 – 30.12.1944), \emph{Schriftsteller/Schriftstellerin}|pwk} an Stefan Zweig\pwindex{Zweig, Stefan 28.11.1881 – 23.02.1942@\textsc{Zweig, Stefan} (28.11.1881 – 23.02.1942), \emph{Schriftsteller/Schriftstellerin}|pwk},
                     5. 2. 1915, abgedruckt in: Romain Rolland\pwindex{Rolland, Romain 29.01.1866 – 30.12.1944@\textsc{Rolland, Romain} (29.01.1866 – 30.12.1944), \emph{Schriftsteller/Schriftstellerin}|pwk}, Stefan Zweig\pwindex{Zweig, Stefan 28.11.1881 – 23.02.1942@\textsc{Zweig, Stefan} (28.11.1881 – 23.02.1942), \emph{Schriftsteller/Schriftstellerin}|pwk}: \emph{Von Welt zu
                           Welt. Briefe einer Freundschaft 1914–1918}. Mit einem Begleitwort von Peter Handke. Aus dem Französischen von Eva
                     und Gerhard Schwewe (Briefe Rollands) und Christel Gersch (Briefe Zweigs). Berlin:
                     \emph{Aufbau Verlag}{ }2014.}}}\label{K_L03651-1} ungehindert
               schreiben können. Er spricht auch von Ihnen darin – er hat offenbar in Berl. Tag.\pwindex{Berliner Tageblatt@\emph{Berliner Tageblatt}|pw} jenen \label{K_L03651-2v}\edtext{Artikel\pwindex{Kultur–?@\emph{Kultur–?}|pwv}}{\lemma{\textnormal{\emph{Artikel}}}\Cendnote{\textnormal{P. B.\pwindex{Block, Paul 1862-05-30 – 1934-08-15@\textsc{Block, Paul} (1862-05-30 – 1934-08-15), \emph{Schriftsteller/Schriftstellerin, Journalist/Journalistin}|pwk} [=Paul Block\pwindex{Block, Paul 1862-05-30 – 1934-08-15@\textsc{Block, Paul} (1862-05-30 – 1934-08-15), \emph{Schriftsteller/Schriftstellerin, Journalist/Journalistin}|pwk}]: \emph{Kultur–?}\pwindex{Kultur–?@\emph{Kultur–?}|pwk} In: \emph{Berliner Tageblatt}\pwindex{Berliner Tageblatt@\emph{Berliner Tageblatt}|pwk},
                     Jg. 44, Nr. 33, 21. 1. 1915, Abendausgabe, S. [3].}}}\label{K_L03651-2} gegen und für Sie gelesen – und schreibt »\label{K_L03651-3v}\edtext{Le voici logé a la même enseigne en Allemagne\oindex{Deutschland@\textbf{Deutschland}, \emph{A.PCLI}|pw} que je le suis en France\oindex{Frankreich@\textbf{Frankreich}, \emph{A.PCLI}|pw}! Exprimez lui de ma part toute ma sympathie
               confraternelle – si toutefois elle {\pb}ne le
                  comp{[}r{]}omet pas encore plus. Ah que les gens sont fous! C’est
               presque comique.}{\lemma{\textnormal{\emph{Le … comique.}}}\Cendnote{\textnormal{französisch: Und wie
                  finden Sie, was unserem armen Arthur
                     Schnitzler widerfahren ist? Da gerät er in Deutschland\oindex{Deutschland@\textbf{Deutschland}, \emph{A.PCLI}|pwk} in die gleiche Lage wie ich in Frankreich\oindex{Frankreich@\textbf{Frankreich}, \emph{A.PCLI}|pwk}! Drücken Sie ihm meine brüderliche Verbundenheit aus, falls
                  ihn das nicht noch mehr kompromittiert. Ach, wie töricht die Menschen doch sind!
                  Es ist schon fast komisch. (Zitiert nach \emph{Von Welt zu
                        Welt}.) }}}\label{K_L03651-3}« Wirklich – dieser Versuch auch Sie jetzt
               einzubeziehen in den grossen deutschen Bann war schon zu ärgerlich! Wird man all dies
               in zehn Jahren noch verstehen können? Ich denke Ihrer oft und in Herzlichkeit:
               hoffentlich hat die tragische Monotonie der andauernden bewegungslosen Kämpfe auch in
               Ihnen wieder die Arbeit als Gegengewalt hochgebracht\substVorne{}\textsuperscript{{ }und}\substDazwischen{}.\substHinten{} Ich habe keinen bessern Wunsch als Sie wieder schaffend und gesammelt
               zu wissen, dass wenigstens hier im Geistigen etwas Fruchtbares bleibe aus diesen
               sinnlosen Tagen der Vernichtung.\pend
           
\pstart
           Ihnen und Ihrer lieben Frau\pwindex{Schnitzler, Olga 17.01.1882 – 13.01.1970@\textsc{Schnitzler, Olga} (17.01.1882 – 13.01.1970), \emph{Schauspieler/Schauspielerin, Sänger/Sängerin}|pwv} herzlich getreu{\\[\baselineskip]}\spacefill\mbox{Stefan Zweig}\pend
           \leftskip=0em{}\selectlanguage{ngerman}\endnumbering\briefempfaengerindex{Schnitzler, Arthur@\textsc{Schnitzler, Arthur}!zzzZweig, Stefan@\emph{von Stefan Zweig}!1915-02-051@{{[}zwischen 5. und 10. 2. 1915{]}}|)be}\mylabel{L03651h}
\begin{anhang}
\end{anhang}\normalsize

\doendnotes{C}
\bigskip
\vfill

\clearpage

\footnotesize

\lohead{\textsc{register}}

% Definiere theindex-Environment komplett neu ohne reledmac
\makeatletter
\renewenvironment{theindex}{%
  \section*{\indexname}%
  \setlength{\parindent}{0pt}%
  \setlength{\parskip}{0pt plus 0.3pt}%
  \let\item\@idxitem
}{%
  \clearpage
}
\makeatother

\IfFileExists{\jobname-pw.ind}{\input{\jobname-pw.ind}}{}

\end{document}

      