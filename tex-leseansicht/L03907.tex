%% latex-leseansicht-vorspann.tex
%% Vorspann für die Leseansicht.
%% Lädt die gemeinsame Datei latex-vorspann.tex mit nicht gesetztem Schalter.

\newif\ifkorrekturansicht
\korrekturansichtfalse

\input{../tex-inputs/latex-vorspann}


\section[Arthur Schnitzler an Theodor Herzl, 10. 12. 1893]{L03907 Arthur Schnitzler an Theodor Herzl, 10. 12. 1893}
\nopagebreak\mylabel{L03907v}
\rehead{ }\normalsize\beginnumbering\briefempfaengerindex{Herzl, Theodor@\textsc{Herzl, Theodor}!zzzSchnitzler, Arthur@\emph{von Arthur Schnitzler}!1893-12-101@{10. 12. 1893}|(be}
\toendnotes[C]{\smallbreak\pagebreak[2]}
\correspDesc{Versand  durch Arthur Schnitzler am 10. 12. 1893 in Wien
\newline{}Erhalt  durch Theodor Herzl in Paris}\toendnotes[C]{\smallbreak}
\Standort{Jerusalem, Central Zionist Archives, H1:1924-12.}
\physDesc{,  Blätter,  Seiten
\newline{}Handschrift: , deutsche Kurrent}
\buchAbdrucke{\weitereDrucke{1) Arthur Schnitzler: \emph{Briefe 1875–1912}. Herausgegeben von Therese Nickl und Heinrich Schnitzler. Frankfurt am Main: \emph{S. Fischer} 1981, S. 220–221.} \weitereDrucke{2) Hermann Bahr, Arthur Schnitzler: \emph{Briefwechsel, Aufzeichnungen, Dokumente (1891–1931)}. Herausgegeben von Kurt Ifkovits und Martin Anton Müller. (2018) \url{https://schnitzler-bahr.acdh.oeaw.ac.at/L041378.html}.} }\toendnotes[C]{\smallbreak}
\pstart
           \raggedleft{}{\pb}Wien, \textsc{IX. Frankgasse 1}\oindex{Wien@\textbf{Wien}!IX., Alsergrund@\textbf{IX., Alsergrund}!Frankgasse 1@\textbf{Frankgasse 1}, \emph{Wohngebäude}|pw}.\pend
           
\pstart{}Mein verehrte Freund,\pend\vspace{0.5em}
\pstart
           Herzlichen Dank! Zum gratulieren liegt eigentlich wenig Anlaſs vor, denn ich bin durchgefallen\eventindex{Volkstheater@\textbf{Volkstheater}!Uraufführung von Das Märchen, 1.12.1893@Uraufführung von Das Märchen, 1.12.1893|pwv}\eventindex{Volkstheater@\textbf{Volkstheater}!2. (und letzte) Aufführung Das Märchen, 2.12.1893@2. (und letzte) Aufführung Das Märchen, 2.12.1893|pwv}. Aber
               das thut nichts, denn in meiner nahtloſen Arroganz verachte ich das \label{K_L03907-1v}\edtext{Urtheil\pwindex{Theater- und Kunstnachrichten [Uraufführung Das Märchen]@\emph{Theater- und Kunstnachrichten [Uraufführung Das Märchen]}|pwv}}{\lemma{\textnormal{\emph{Urtheil}}}\Cendnote{\textnormal{[Friedrich Schütz\pwindex{Schütz, Friedrich 24.\,4.\,1844 Prag – 22.\,12.\,1908 Wien@\textsc{Schütz, Friedrich} (24.\,4.\,1844 Prag – 22.\,12.\,1908 Wien), \emph{Schriftsteller, Journalist}|pwk}]: \emph{Theater- und Kunstnachrichten}\pwindex{Theater- und Kunstnachrichten [Uraufführung Das Märchen]@\emph{Theater- und Kunstnachrichten [Uraufführung Das Märchen]}|pwk}. In: \emph{Neue Freie Presse}\pwindex{Neue Freie Presse@\emph{Neue Freie Presse}|pwk}, Jg. 30, Nr. 10.518,
                        2. 12. 1893, S. 7.}}}\label{K_L03907-1} des Schütz\pwindex{Schütz, Friedrich 24.\,4.\,1844 Prag – 22.\,12.\,1908 Wien@\textsc{Schütz, Friedrich} (24.\,4.\,1844 Prag – 22.\,12.\,1908 Wien), \emph{Schriftsteller, Journalist}|pw}, und{ }ſelbſt das Verschwinden des Stücks\pwindex{Schnitzler, Arthur 15.\,5.\,1862 Wien – 21.\,10.\,1931 ebd.@\textsc{Schnitzler, Arthur} (15.\,5.\,1862 Wien – 21.\,10.\,1931 ebd.), \emph{Schriftsteller, Mediziner}!Märchen. Schauspiel in drei Aufzügen@\strich\emph{Das Märchen. Schauspiel in drei Aufzügen}|pwv} vom Repertoire nach zwei
                  Vorſtellungen\eventindex{Volkstheater@\textbf{Volkstheater}!Uraufführung von Das Märchen, 1.12.1893@Uraufführung von Das Märchen, 1.12.1893|pwv}\eventindex{Volkstheater@\textbf{Volkstheater}!2. (und letzte) Aufführung Das Märchen, 2.12.1893@2. (und letzte) Aufführung Das Märchen, 2.12.1893|pwv}
               kann mich vor der Ueberzeugung nicht abbringen, {\pb}daſs es
               beſſer iſt, wie manche, die man dreimal geſpielt hat. –\pend
           
\pstart
           Daſs es Ihnen endlich wieder gut geht, freut mich herzlich; ich habe mich bei meinem
               Freund Paul\pwindex{Goldmann, Paul 31.\,1.\,1865 Breslau – 25.\,9.\,1935 Wien@\textsc{Goldmann, Paul} (31.\,1.\,1865 Breslau – 25.\,9.\,1935 Wien), \emph{Schriftsteller, Journalist}|pw} erkundigt, wie es \strikeout{geht} mit Ihrem Befinden{ }ſteht – und habe Sie eigentlich{ }ſchon{ }ſeit einigen Wochen für ganz geſund gehalten. Werden Sie zu völliger Erholung
               einer Urlaub {\pb}nehmen? –\pend
           
\pstart
           Die Feu{[}i{]}lletons\pwindex{Bahr, Hermann 19.\,7.\,1863 Linz – 15.\,1.\,1934 München@\textsc{Bahr, Hermann} (19.\,7.\,1863 Linz – 15.\,1.\,1934 München), \emph{Schriftsteller, Kritiker}!junge Österreich@\strich\emph{Das junge Österreich}|pwv} von \textsc{Bahr\pwindex{Bahr, Hermann 19.\,7.\,1863 Linz – 15.\,1.\,1934 München@\textsc{Bahr, Hermann} (19.\,7.\,1863 Linz – 15.\,1.\,1934 München), \emph{Schriftsteller, Kritiker}|pw}} beſorge ich Ihnen eheſtens; in 2 Tagen haben Sie{ }ſie. –\pend
           
\pstart
           Wie geht es de{\geminationn} Ihnen? Bitte empfehlen Sie mich Ihrer w. Frau Gemahlin\pwindex{Herzl, Julie 1.\,2.\,1868 Budapest – 10.\,11.\,1907 Wien@\textsc{Herzl, Julie} (1.\,2.\,1868 Budapest – 10.\,11.\,1907 Wien)|pwv} und laſſen Sie recht bald von{ }ſich
               hören.\pend
           
\pstart
           Mit vielen herzlichen Grüßen{\\[\baselineskip]}Ihr treu ergebner{\\[\baselineskip]}\spacefill\mbox{ArthurSchnitzler}\pend
           \leftskip=0em{}
\pstart
           Wien\oindex{Wien@\textbf{Wien}, \emph{Verwaltungsgebiet}|pw},
                  10. 12. 93.\pend
           \selectlanguage{ngerman}\endnumbering\briefempfaengerindex{Herzl, Theodor@\textsc{Herzl, Theodor}!zzzSchnitzler, Arthur@\emph{von Arthur Schnitzler}!1893-12-101@{10. 12. 1893}|)be}\mylabel{L03907h}
\begin{anhang}
\end{anhang}\newcommand{\dateiname}{L03907}\newcommand{\titel}{Arthur Schnitzler an Theodor Herzl, 10. 12. 1893}\newcommand{\editorInnen}{Herausgegeben von Jahnke, SelmaMüller, Martin Anton}%% latex-leseansicht-abspann.tex
%% Abspann für die Leseansicht.
%% Der Schalter \ifkorrekturansicht ist bereits durch den Vorspann gesetzt.

%% latex-abspann.tex
%% Gemeinsamer Abspann für Korrekturansicht und Leseansicht.
%% Setzt den Schalter \ifkorrekturansicht voraus (gesetzt in den
%% einbindenden Dateien latex-korrekturansicht-abspann.tex bzw.
%% latex-leseansicht-abspann.tex).
%% ---------------------------------------------------------------

\normalsize

% Das esempio-Environment wird nur in der Leseansicht benötigt
\ifkorrekturansicht\else
\newenvironment{esempio}[3]%
{
    \vspace{1.5ex}
    \rlap{\underline{#1}}
    \par
    \setlength{\parindent}{0cm}
    \nopagebreak
    \leftskip=#2cm
    \rightskip=#3cm
}
{
    \par
}
\fi

\doendnotes{C}
\bigskip
\vfill

\clearpage

\footnotesize

\ifkorrekturansicht
  \lohead{\textsc{register}}
\fi

% theindex-Environment neu definieren ohne reledmac
\makeatletter
\renewenvironment{theindex}{%
  \ifkorrekturansicht
    \section*{\indexname}%
  \else
    \subsubsection*{Index der erwähnten Entitäten}%
  \fi
  \setlength{\parindent}{0pt}%
  \setlength{\parskip}{0pt plus 0.3pt}%
  \let\item\@idxitem
}{%
  \ifkorrekturansicht\clearpage\fi
}
\makeatother

\IfFileExists{\jobname-pw.ind}{\input{\jobname-pw.ind}}{}

% Quellenangabe nur in der Leseansicht
\ifkorrekturansicht\else
% Fallback-Definitionen, falls die .tex-Datei \titel etc. nicht gesetzt hat
\providecommand{\titel}{}
\providecommand{\editorInnen}{}
\providecommand{\dateiname}{\jobname}

\vspace{3cm}

\vfill

\footnotesize
\textsc{Quelle}: \titel. Herausgegeben von {\editorInnen}. In: \emph{Arthur Schnitzler: Briefwechsel mit Autorinnen und Autoren}.
 Digitale Edition, https://schnitzler-briefe.acdh.oeaw.ac.at/{\dateiname}.html (Stand \today)
\fi

\end{document}


