%% latex-korrekturansicht-vorspann.tex
%% Vorspann für die Korrekturansicht.
%% Lädt die gemeinsame Datei latex-vorspann.tex mit gesetztem Schalter.

\newif\ifkorrekturansicht
\korrekturansichttrue

\input{../tex-inputs/latex-vorspann}


\section[Joseph Victor Widmann an Arthur Schnitzler, 26. 3. 1893]{L00193 Joseph Victor Widmann an Arthur Schnitzler, 26. 3. 1893}
\nopagebreak\mylabel{L00193v}
\rehead{ }\normalsize\beginnumbering\briefempfaengerindex{Schnitzler, Arthur@\textsc{Schnitzler, Arthur}!zzzWidmann, Joseph Victor@\emph{von Joseph Victor Widmann}!1893-03-261@{26. 3. 1893}|(be}
\toendnotes[C]{\smallbreak\pagebreak[2]}\Standort{CUL, Schnitzler, B 113.}
\physDesc{Postkarte, 420 Zeichen
\newline{}Handschrift: schwarze Tinte, deutsche Kurrent
\newline{}Versand: Stempel: »\nobreak{}\oindex{Bern@\textbf{Bern}, \emph{P.PPLC}|pwk}Bern Brf. Exp., 25 III. 93., 1\nobreak{}«.  }\toendnotes[C]{\smallbreak}\pstart{}{\pb}\textsc{
                     Herrn D
                     \textsuperscript{r}
                      Arthur Schnitzler
                  }\pend{}\pstart{}in\pend{}\pstart{}\textsc{Wien}
                      I
                  \oindex{I., Innere Stadt@\textbf{I., Innere Stadt}, \emph{A.ADM3}|pw}
                  . 
               \pend{}\pstart{}\textsc{I. Grillparzerstraſse 7\oindex{Grillparzerstrasse@\textbf{Grillparzerstraße}, \emph{R.ST}|pw}
                     .
                  }\pend{}{\bigskip}\vspace{1em}
\pstart
           \raggedleft{}{\pb}Bern\oindex{Bern@\textbf{Bern}, \emph{P.PPLC}|pw}
                  , d.
                  
                  26. März 1893
                  .
               \pend
           
\pstart{}Verehrteſter Herr!\pend\vspace{0.5em}
\pstart
           
               Die 
               Beſprechung\pwindex{Kunst und Litteratur@\emph{Kunst und Litteratur}|pwv}
                Ihres 
               Anatol\pwindex{Anatol@\emph{Anatol}|pw}
                war von mir ſelbſt, wie ich überhaupt die
               meiſten literariſchen Referate des »
               Bund\orgindex{Bund@Der Bund|pw}
               «
               ſchreibe. Es freut mich, Ihrer vom 
               14. d.
                an die Redaktion gerichteten
               Zuſchrift zu entnehmen, daß ſie Ihnen Spaß machte. 
               \strikeout{A}
               
               Ihren 
               Anatol\pwindex{Anatol@\emph{Anatol}|pw}
                habe ich ſehr wohl gelobt.
            \pend
           \pstart 
               Hochachtungsvoll
               \spacefill\mbox{J. V. Widmann}\pend{}
\pstart
           \raggedleft{}
               liter. Redakteur des »
               Bund\orgindex{Bund@Der Bund|pw}
               «
            \pend
           \selectlanguage{ngerman}\endnumbering\briefempfaengerindex{Schnitzler, Arthur@\textsc{Schnitzler, Arthur}!zzzWidmann, Joseph Victor@\emph{von Joseph Victor Widmann}!1893-03-261@{26. 3. 1893}|)be}\mylabel{L00193h}  \normalsize

\doendnotes{C}
\bigskip
\vfill

\clearpage

\footnotesize

\lohead{\textsc{register}}

% Definiere theindex-Environment komplett neu ohne reledmac
\makeatletter
\renewenvironment{theindex}{%
  \section*{\indexname}%
  \setlength{\parindent}{0pt}%
  \setlength{\parskip}{0pt plus 0.3pt}%
  \let\item\@idxitem
}{%
  \clearpage
}
\makeatother

\IfFileExists{\jobname-pw.ind}{\input{\jobname-pw.ind}}{}

\end{document}

      