%% latex-leseansicht-vorspann.tex
%% Vorspann für die Leseansicht.
%% Lädt die gemeinsame Datei latex-vorspann.tex mit nicht gesetztem Schalter.

\newif\ifkorrekturansicht
\korrekturansichtfalse

\input{../tex-inputs/latex-vorspann}


         
         \newcommand{\erwaehntePersonen}{Personen: }
         \newcommand{\erwaehnteInstitutionen}{Institutionen: Der Bund}
         \newcommand{\erwaehnteOrte}{Orte: Bern, Grillparzerstraße, I., Innere Stadt, Wien}
         \newcommand{\erwaehnteWerke}{Werke: Anatol, Kunst und Litteratur}
               \section[Joseph Victor Widmann an Arthur Schnitzler, 26. 3. 1893]{ Joseph Victor Widmann an Arthur Schnitzler,
                    26. 3. 1893}\nopagebreak\mylabel{v}\rehead{ }\begin{ledgroupsized}[t]{13cm}\normalsize\beginnumbering \toendnotes[C]{\smallbreak\pagebreak[2]} \Standort{CUL, Schnitzler, B 113.}
\physDesc{Postkarte
\newline{}Handschrift: schwarze Tinte, deutsche Kurrent\newline{}Versand: Stempel: »\nobreak{}\oindex{Bern@\textbf{Bern}|pwk}Bern Brf. Exp., 25 III. 93., 1\nobreak{}«.  }\toendnotes[C]{\smallbreak}\pstart{}{\pb}\textsc{Herrn
                                D\textsuperscript{r} Arthur Schnitzler}\pend{}\pstart{}in\pend{}\pstart{}\textsc{Wien} I\oindex{I., Innere Stadt@\textbf{I., Innere Stadt}|pw}. \pend{}\pstart{}\textsc{I.
                                Grillparzerstraſse 7\oindex{Grillparzerstrasse@\textbf{Grillparzerstraße}|pw}.}\pend{}{\bigskip}\pstart
           \raggedleft{}{\pb}Bern\oindex{Bern@\textbf{Bern}|pw}, d.
                            26. März 1893.\pend
           \pstart{}Verehrteſter Herr!\pend\pstart
           Die Beſprechung\pwindex{Kunst und Litteratur12.02.1893 – 12.2.1893@\emph{Kunst und Litteratur} {[}12.02.1893 – 12.2.1893{]}|pwv} Ihres Anatol\pwindex{Schnitzler, Arthur 15.05.1862 – 21.10.1931@\textsc{Schnitzler, Arthur} (15.05.1862 – 21.10.1931), \emph{Schriftsteller, Mediziner}!Anatol1892-10-29@\strich\emph{Anatol} {[}1892-10-29{]}|pw} war von mir ſelbſt, wie ich überhaupt die meiſten
                    literariſchen Referate des »Bund\orgindex{Bund@Der Bund|pw}« ſchreibe. Es
                    freut mich, Ihrer vom 14. d. an die Redaktion gerichteten Zuſchrift zu
                    entnehmen, daß ſie Ihnen Spaß machte. \strikeout{A} Ihren
                        Anatol\pwindex{Schnitzler, Arthur 15.05.1862 – 21.10.1931@\textsc{Schnitzler, Arthur} (15.05.1862 – 21.10.1931), \emph{Schriftsteller, Mediziner}!Anatol1892-10-29@\strich\emph{Anatol} {[}1892-10-29{]}|pw} habe ich ſehr wohl gelobt.\pend
           \pstart Hochachtungsvoll\spacefill\mbox{J. V. Widmann}\pend{}\pstart
           \raggedleft{}liter. Redakteur des »Bund\orgindex{Bund@Der Bund|pw}«\pend
           
         
         \endnumbering\mylabel{h}\end{ledgroupsized}  \newcommand{\dateiname}{L00193}\newcommand{\titel}{Joseph Victor Widmann an Arthur Schnitzler, 26. 3. 1893}\newcommand{\editorInnen}{Martin Anton Müller und Gerd-Hermann Susen}%% latex-leseansicht-abspann.tex
%% Abspann für die Leseansicht.
%% Der Schalter \ifkorrekturansicht ist bereits durch den Vorspann gesetzt.

%% latex-abspann.tex
%% Gemeinsamer Abspann für Korrekturansicht und Leseansicht.
%% Setzt den Schalter \ifkorrekturansicht voraus (gesetzt in den
%% einbindenden Dateien latex-korrekturansicht-abspann.tex bzw.
%% latex-leseansicht-abspann.tex).
%% ---------------------------------------------------------------

\normalsize

% Das esempio-Environment wird nur in der Leseansicht benötigt
\ifkorrekturansicht\else
\newenvironment{esempio}[3]%
{
    \vspace{1.5ex}
    \rlap{\underline{#1}}
    \par
    \setlength{\parindent}{0cm}
    \nopagebreak
    \leftskip=#2cm
    \rightskip=#3cm
}
{
    \par
}
\fi

\doendnotes{C}
\bigskip
\vfill

\clearpage

\footnotesize

\ifkorrekturansicht
  \lohead{\textsc{register}}
\fi

% theindex-Environment neu definieren ohne reledmac
\makeatletter
\renewenvironment{theindex}{%
  \ifkorrekturansicht
    \section*{\indexname}%
  \else
    \subsubsection*{Index der erwähnten Entitäten}%
  \fi
  \setlength{\parindent}{0pt}%
  \setlength{\parskip}{0pt plus 0.3pt}%
  \let\item\@idxitem
}{%
  \ifkorrekturansicht\clearpage\fi
}
\makeatother

\IfFileExists{\jobname-pw.ind}{\input{\jobname-pw.ind}}{}

% Quellenangabe nur in der Leseansicht
\ifkorrekturansicht\else
% Fallback-Definitionen, falls die .tex-Datei \titel etc. nicht gesetzt hat
\providecommand{\titel}{}
\providecommand{\editorInnen}{}
\providecommand{\dateiname}{\jobname}

\vspace{3cm}

\vfill

\footnotesize
\textsc{Quelle}: \titel. Herausgegeben von {\editorInnen}. In: \emph{Arthur Schnitzler: Briefwechsel mit Autorinnen und Autoren}.
 Digitale Edition, https://schnitzler-briefe.acdh.oeaw.ac.at/{\dateiname}.html (Stand \today)
\fi

\end{document}


      