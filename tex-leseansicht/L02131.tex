%% latex-leseansicht-vorspann.tex
%% Vorspann für die Leseansicht.
%% Lädt die gemeinsame Datei latex-vorspann.tex mit nicht gesetztem Schalter.

\newif\ifkorrekturansicht
\korrekturansichtfalse

\input{../tex-inputs/latex-vorspann}


         
         \renewcommand{\erwaehntePersonen}{Personen: Peter Altenberg, Georg Engländer, Margarethe Engländer, Franz Hansy}
         \renewcommand{\erwaehnteOrte}{Orte: Seidlgasse, Semmering, Wien}
         \renewcommand{\erwaehnteWerke}{}
               \section[Georg Engländer an Arthur Schnitzler, 25. 4. 1913]{ Georg Engländer an Arthur Schnitzler, 25. 4. 1913}\nopagebreak\mylabel{v}\rehead{ }\begin{ledgroupsized}[t]{13cm}\normalsize\beginnumbering\briefempfaengerindex{Schnitzler, Arthur@\textsc{Schnitzler, Arthur}!zzzEnglaender, Georg@\emph{von Georg Engländer}!1913-04-251@{25. 4. 1913}|(be} \toendnotes[C]{\smallbreak\pagebreak[2]} \Standort{DLA, A:Schnitzler, HS.NZ85.1.2889.}
\physDesc{Brief, 1 Blatt, 2 Seiten, 1229 Zeichen
\newline{}Schreibmaschine
\newline{}Handschrift: schwarze Tinte (\noindent{}Unterschrift)}\toendnotes[C]{\smallbreak}\pstart
           \noindent{}{\pb}\textcolor{gray}{\textbf{\textit{Georg Engländer}}}\pend
           \pstart
           \textcolor{gray}{\textbf{\textit{Wien, III. Seidlgasse 23\oindex{Seidlgasse@\textbf{Seidlgasse}|pw}.}}}\pend
           \pstart
           \raggedleft{}Wien\oindex{Wien@\textbf{Wien}|pw}, 25. April 1913\pend
           \pstart{}Hochgeehrter Herr!\pend\pstart
           Ich freue mich Ihnen die Mitteilung machen zu könne{[}n{]}, dass ich
               heute von Dr. Hansy\pwindex{Hansy, Franz 23.07.1865 – 25.05.1944@\textsc{Hansy, Franz} (23.07.1865 – 25.05.1944), \emph{Mediziner}|pw}{ }Semmering\oindex{Semmering@\textbf{Semmering}|pw} die Antwort erhielt, dass er nicht nur
               gerne bereit ist, meinem Bruder Peter\pwindex{Altenberg, Peter 09.03.1859 – 08.01.1919@\textsc{Altenberg, Peter} (09.03.1859 – 08.01.1919), \emph{Schriftsteller}|pw} für
               einige Zeit, quasi als Nachkur, in seiner Anstalt aufzunehmen, sondern ihm auch in
               entgegenkommendster Weise einen ausserordentlich bescheidenen Preis per Tag notiert
               hat.\pend
           \pstart
           Ihre besonders freundschaftliche Teilnahme sowie Ihre besonders liebenswürdige Mühe,
               die Sie hierauf verwendet, verpflichten mich selbstverständlich, Ihnen sofort hievon
               Bericht zu geben, wie Ihnen auch gleichzeitig zu melden, dass ich
                  Sonntag{ }{\pb}Nachmittag mit dem Bruder\pwindex{Altenberg, Peter 09.03.1859 – 08.01.1919@\textsc{Altenberg, Peter} (09.03.1859 – 08.01.1919), \emph{Schriftsteller}|pwv} die diesbezügliche Entscheidung treffen werde und es
               seinem Belieben überlassen werde, ob er Montag vorerst für ein od. zwei
               Tage unter meiner Aufsicht in Wien\oindex{Wien@\textbf{Wien}|pw} verbringen
               will, oder sofort schon Montag mit mir od. meiner Schwester\pwindex{Englaender, Margarethe 1873-10-21 – 1942@\textsc{Engländer, Margarethe} (1873-10-21 – 1942)|pwv} auf den Semmering\oindex{Semmering@\textbf{Semmering}|pw} fahren will.\pend
           \pstart
           Ich hoffe nunmehr, dass der peinliche Konflikt zwischen unserer Verantwortung und dem
               natürlichen Drange meines Bruders\pwindex{Altenberg, Peter 09.03.1859 – 08.01.1919@\textsc{Altenberg, Peter} (09.03.1859 – 08.01.1919), \emph{Schriftsteller}|pwv} zu seiner möglichsten Unabhängigkeit beigelegt sein dürfte und
               verbleibe mit nochmaligem ausserordentlichen und herzlichstem Danke Ihr in\pend
           \pstart
           Hochachtung ergebenster{\\[\baselineskip]}\spacefill\mbox{{[}hs.:{]} Georg Engländer}\pend
           \leftskip=0em{}
         
         \endnumbering\mylabel{h}\end{ledgroupsized}  \newcommand{\dateiname}{L02131}\newcommand{\titel}{Georg Engländer an Arthur Schnitzler, 25. 4. 1913}\newcommand{\editorInnen}{Martin Anton Müller und Gerd-Hermann Susen}%% latex-leseansicht-abspann.tex
%% Abspann für die Leseansicht.
%% Der Schalter \ifkorrekturansicht ist bereits durch den Vorspann gesetzt.

%% latex-abspann.tex
%% Gemeinsamer Abspann für Korrekturansicht und Leseansicht.
%% Setzt den Schalter \ifkorrekturansicht voraus (gesetzt in den
%% einbindenden Dateien latex-korrekturansicht-abspann.tex bzw.
%% latex-leseansicht-abspann.tex).
%% ---------------------------------------------------------------

\normalsize

% Das esempio-Environment wird nur in der Leseansicht benötigt
\ifkorrekturansicht\else
\newenvironment{esempio}[3]%
{
    \vspace{1.5ex}
    \rlap{\underline{#1}}
    \par
    \setlength{\parindent}{0cm}
    \nopagebreak
    \leftskip=#2cm
    \rightskip=#3cm
}
{
    \par
}
\fi

\doendnotes{C}
\bigskip
\vfill

\clearpage

\footnotesize

\ifkorrekturansicht
  \lohead{\textsc{register}}
\fi

% theindex-Environment neu definieren ohne reledmac
\makeatletter
\renewenvironment{theindex}{%
  \ifkorrekturansicht
    \section*{\indexname}%
  \else
    \subsubsection*{Index der erwähnten Entitäten}%
  \fi
  \setlength{\parindent}{0pt}%
  \setlength{\parskip}{0pt plus 0.3pt}%
  \let\item\@idxitem
}{%
  \ifkorrekturansicht\clearpage\fi
}
\makeatother

\IfFileExists{\jobname-pw.ind}{\input{\jobname-pw.ind}}{}

% Quellenangabe nur in der Leseansicht
\ifkorrekturansicht\else
% Fallback-Definitionen, falls die .tex-Datei \titel etc. nicht gesetzt hat
\providecommand{\titel}{}
\providecommand{\editorInnen}{}
\providecommand{\dateiname}{\jobname}

\vspace{3cm}

\vfill

\footnotesize
\textsc{Quelle}: \titel. Herausgegeben von {\editorInnen}. In: \emph{Arthur Schnitzler: Briefwechsel mit Autorinnen und Autoren}.
 Digitale Edition, https://schnitzler-briefe.acdh.oeaw.ac.at/{\dateiname}.html (Stand \today)
\fi

\end{document}


      