%% latex-leseansicht-vorspann.tex
%% Vorspann für die Leseansicht.
%% Lädt die gemeinsame Datei latex-vorspann.tex mit nicht gesetztem Schalter.

\newif\ifkorrekturansicht
\korrekturansichtfalse

\input{../tex-inputs/latex-vorspann}


\section[Georg Engländer an Arthur Schnitzler, 25. 4. 1913]{L02131 Georg Engländer an Arthur Schnitzler, 25. 4. 1913}
\nopagebreak\mylabel{L02131v}
\rehead{ }\normalsize\beginnumbering\briefempfaengerindex{Schnitzler, Arthur@\textsc{Schnitzler, Arthur}!zzzEngländer, Georg@\emph{von Georg Engländer}!1913-04-251@{25. 4. 1913}|(be}
\toendnotes[C]{\smallbreak\pagebreak[2]}
\correspDesc{Versand  durch Georg Engländer am 25. 4. 1913 in Wien
\newline{}Erhalt  durch Arthur Schnitzler im Zeitraum [25. 4. 1913
                  – 29. 4. 1913?] in Wien}\toendnotes[C]{\smallbreak}
\Standort{DLA, A:Schnitzler, HS.NZ85.1.2889.}
\physDesc{Brief, 1 Blatt, 2 Seiten, 1229 Zeichen
\newline{}Schreibmaschine
\newline{}Handschrift: schwarze Tinte (\noindent{}Unterschrift)}\toendnotes[C]{\smallbreak}
\pstart
           {\pb}\textcolor{gray}{\textbf{\textit{Georg Engländer}}}\pend
           
\pstart
           \textcolor{gray}{\textbf{\textit{Wien, III. Seidlgasse 23\oindex{Wien@\textbf{Wien}!III., Landstraße@\textbf{III., Landstraße}!Seidlgasse@\textbf{Seidlgasse}, \emph{Straße}|pw}.}}}\pend
           
\pstart
           \raggedleft{}Wien\oindex{Wien@\textbf{Wien}, \emph{Verwaltungsgebiet}|pw}, 25. April 1913\pend
           
\pstart{}Hochgeehrter Herr!\pend\vspace{0.5em}
\pstart
           Ich freue mich Ihnen die Mitteilung machen zu könne{[}n{]}, dass ich
               heute von Dr. Hansy\pwindex{Hansy, Franz 23.\,7.\,1865 Baden bei Wien – 25.\,5.\,1944 Wien@\textsc{Hansy, Franz} (23.\,7.\,1865 Baden bei Wien – 25.\,5.\,1944 Wien), \emph{Mediziner}|pw}{ }Semmering\oindex{Semmering@\textbf{Semmering}, \emph{Verwaltungsgebiet}|pw} die Antwort erhielt, dass er nicht nur
               gerne bereit ist, meinem Bruder Peter\pwindex{Altenberg, Peter 9.\,3.\,1859 Wien – 8.\,1.\,1919 ebd.@\textsc{Altenberg, Peter} (9.\,3.\,1859 Wien – 8.\,1.\,1919 ebd.), \emph{Schriftsteller}|pw} für
               einige Zeit, quasi als Nachkur, in seiner Anstalt aufzunehmen, sondern ihm auch in
               entgegenkommendster Weise einen ausserordentlich bescheidenen Preis per Tag notiert
               hat.\pend
           
\pstart
           Ihre besonders freundschaftliche Teilnahme sowie Ihre besonders liebenswürdige Mühe,
               die Sie hierauf verwendet, verpflichten mich selbstverständlich, Ihnen sofort hievon
               Bericht zu geben, wie Ihnen auch gleichzeitig zu melden, dass ich
                  Sonntag{ }{\pb}Nachmittag mit dem Bruder\pwindex{Altenberg, Peter 9.\,3.\,1859 Wien – 8.\,1.\,1919 ebd.@\textsc{Altenberg, Peter} (9.\,3.\,1859 Wien – 8.\,1.\,1919 ebd.), \emph{Schriftsteller}|pwv} die diesbezügliche Entscheidung treffen werde und es
               seinem Belieben überlassen werde, ob er Montag vorerst für ein od. zwei
               Tage unter meiner Aufsicht in Wien\oindex{Wien@\textbf{Wien}, \emph{Verwaltungsgebiet}|pw} verbringen
               will, oder sofort schon Montag mit mir od. meiner Schwester\pwindex{Engländer, Margarethe 21.\,10.\,1873 Wien – 1942 Vernichtungslager Treblinka@\textsc{Engländer, Margarethe} (21.\,10.\,1873 Wien – 1942 Vernichtungslager Treblinka)|pwv} auf den Semmering\oindex{Semmering@\textbf{Semmering}, \emph{Verwaltungsgebiet}|pw} fahren will.\pend
           
\pstart
           Ich hoffe nunmehr, dass der peinliche Konflikt zwischen unserer Verantwortung und dem
               natürlichen Drange meines Bruders\pwindex{Altenberg, Peter 9.\,3.\,1859 Wien – 8.\,1.\,1919 ebd.@\textsc{Altenberg, Peter} (9.\,3.\,1859 Wien – 8.\,1.\,1919 ebd.), \emph{Schriftsteller}|pwv} zu seiner möglichsten Unabhängigkeit beigelegt sein dürfte und
               verbleibe mit nochmaligem ausserordentlichen und herzlichstem Danke Ihr in\pend
           
\pstart
           Hochachtung ergebenster{\\[\baselineskip]}\spacefill\mbox{{[}hs.:{]} Georg Engländer}\pend
           \leftskip=0em{}\selectlanguage{ngerman}\endnumbering\briefempfaengerindex{Schnitzler, Arthur@\textsc{Schnitzler, Arthur}!zzzEngländer, Georg@\emph{von Georg Engländer}!1913-04-251@{25. 4. 1913}|)be}\mylabel{L02131h}  \newcommand{\dateiname}{L02131}\newcommand{\titel}{Georg Engländer an Arthur Schnitzler, 25. 4. 1913}\newcommand{\editorInnen}{Martin Anton Müller und Gerd-Hermann Susen}%% latex-leseansicht-abspann.tex
%% Abspann für die Leseansicht.
%% Der Schalter \ifkorrekturansicht ist bereits durch den Vorspann gesetzt.

%% latex-abspann.tex
%% Gemeinsamer Abspann für Korrekturansicht und Leseansicht.
%% Setzt den Schalter \ifkorrekturansicht voraus (gesetzt in den
%% einbindenden Dateien latex-korrekturansicht-abspann.tex bzw.
%% latex-leseansicht-abspann.tex).
%% ---------------------------------------------------------------

\normalsize

% Das esempio-Environment wird nur in der Leseansicht benötigt
\ifkorrekturansicht\else
\newenvironment{esempio}[3]%
{
    \vspace{1.5ex}
    \rlap{\underline{#1}}
    \par
    \setlength{\parindent}{0cm}
    \nopagebreak
    \leftskip=#2cm
    \rightskip=#3cm
}
{
    \par
}
\fi

\doendnotes{C}
\bigskip
\vfill

\clearpage

\footnotesize

\ifkorrekturansicht
  \lohead{\textsc{register}}
\fi

% theindex-Environment neu definieren ohne reledmac
\makeatletter
\renewenvironment{theindex}{%
  \ifkorrekturansicht
    \section*{\indexname}%
  \else
    \subsubsection*{Index der erwähnten Entitäten}%
  \fi
  \setlength{\parindent}{0pt}%
  \setlength{\parskip}{0pt plus 0.3pt}%
  \let\item\@idxitem
}{%
  \ifkorrekturansicht\clearpage\fi
}
\makeatother

\IfFileExists{\jobname-pw.ind}{\input{\jobname-pw.ind}}{}

% Quellenangabe nur in der Leseansicht
\ifkorrekturansicht\else
% Fallback-Definitionen, falls die .tex-Datei \titel etc. nicht gesetzt hat
\providecommand{\titel}{}
\providecommand{\editorInnen}{}
\providecommand{\dateiname}{\jobname}

\vspace{3cm}

\vfill

\footnotesize
\textsc{Quelle}: \titel. Herausgegeben von {\editorInnen}. In: \emph{Arthur Schnitzler: Briefwechsel mit Autorinnen und Autoren}.
 Digitale Edition, https://schnitzler-briefe.acdh.oeaw.ac.at/{\dateiname}.html (Stand \today)
\fi

\end{document}


