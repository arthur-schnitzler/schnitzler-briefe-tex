%% latex-leseansicht-vorspann.tex
%% Vorspann für die Leseansicht.
%% Lädt die gemeinsame Datei latex-vorspann.tex mit nicht gesetztem Schalter.

\newif\ifkorrekturansicht
\korrekturansichtfalse

\input{../tex-inputs/latex-vorspann}


\section[Stefan Zweig an Arthur Schnitzler, 28. 7. 1923]{L03686 Stefan Zweig an Arthur Schnitzler, 28. 7. 1923}
\nopagebreak\mylabel{L03686v}
\rehead{ }\normalsize\beginnumbering\briefempfaengerindex{Schnitzler, Arthur@\textsc{Schnitzler, Arthur}!zzzZweig, Stefan@\emph{von Stefan Zweig}!1923-07-281@{28. 7. 1923}|(be}
\toendnotes[C]{\smallbreak\pagebreak[2]}
\correspDesc{Versand  durch Stefan Zweig am 28. 7. 1923 in Salzburg
\newline{}Erhalt  durch Arthur Schnitzler im Zeitraum [29. 7. 1923
                  – 30. 7. 1923?] in Wien}\toendnotes[C]{\smallbreak}
\Standort{CUL, Schnitzler, B 118.}
\physDesc{Brief, 1 Blatt, 1 Seite, 732 Zeichen
\newline{}Schreibmaschine
\newline{}Handschrift: blaue Tinte, lateinische Kurrent (\noindent{}Unterschrift und Postskriptum)
\newline{}Schnitzler: mit rotem Buntstift zwei Unterstreichungen }
\buchAbdrucke{\weitereDrucke{1) Stefan Zweig: \emph{Briefwechsel mit Hermann Bahr, Sigmund Freud, Rainer Maria
                        Rilke und Arthur Schnitzler}. Herausgegeben von Jeffrey B. Berlin, Hans-Ulrich Lindken und Donald A. Prater. Frankfurt am Main: \emph{S. Fischer} 1987, S. 417.} \weitereDrucke{2) Hermann Bahr, Arthur Schnitzler: \emph{Briefwechsel, Aufzeichnungen, Dokumente (1891–1931)}. Herausgegeben von Kurt Ifkovits und Martin Anton Müller. Göttingen: \emph{Wallstein} 2018, S. 578–579.} }\toendnotes[C]{\smallbreak}
\pstart
           {\pb}\textcolor{gray}{\textbf{SZ}}\hfill \textcolor{gray}{\textbf{KAPUZINERBERG 5\oindex{Paschinger Schlössl@\textbf{Paschinger Schlössl}, \emph{Wohngebäude}|pw}}}\pend
           
\pstart
           \raggedleft{}\textcolor{gray}{\textbf{SALZBURG\oindex{Salzburg@\textbf{Salzburg}, \emph{Verwaltungsgebiet}|pw},}}\pend
           
\pstart
           \raggedleft{}am 28. Juli 1923.\pend
           
\pstart\center{}Verehrter Herr Doktor!\pend\vspace{0.5em}
\pstart
           Ich empfange freudig Ihre \label{K_L03686-1v}\edtext{Nachricht}{\lemma{\textnormal{\emph{Nachricht}}}\Cendnote{\textnormal{XXXX Auszeichnungsfehler: Dokument L03750 nicht gefunden.}}}\label{K_L03686-1} und brauche nicht zu sagen, dass ich Ihnen gern,
               wenn Sie mich rechtzeitig verständigen, im »Oesterreichischen Hof\oindex{Österreichischer Hof@\textbf{Österreichischer Hof}, \emph{Hotel}|pw}« ein Zimmer reserviere. Ich hätte Sie lieber zu uns
               gebeten, aber wir sind durch die Gegenwart Rollands\pwindex{Rolland, Romain 29.\,1.\,1866 Clamecy – 30.\,12.\,1944 Vézelay@\textsc{Rolland, Romain} (29.\,1.\,1866 Clamecy – 30.\,12.\,1944 Vézelay), \emph{Schriftsteller}|pw} besetzt. Das Hotel Europe\oindex{Grand Hotel de L’Europe, G. Jung@\textbf{Grand Hotel de L’Europe, G. Jung}, \emph{Hotel}|pw}
               ist aber momentan wirklich etwas kostspielig und Sie werden im »Oesterreichischen Hof\oindex{Österreichischer Hof@\textbf{Österreichischer Hof}, \emph{Hotel}|pw}{[}«{]} ebenso zufrieden sein.\pend
           
\pstart
           Gestern und heute waren wir mit Bahr\pwindex{Bahr, Hermann 19.\,7.\,1863 Linz – 15.\,1.\,1934 München@\textsc{Bahr, Hermann} (19.\,7.\,1863 Linz – 15.\,1.\,1934 München), \emph{Schriftsteller, Kritiker}|pw} und heute ging er in einem Zuge zur Gaissbergspitze\oindex{Gaisberg@\textbf{Gaisberg}, \emph{Berg}|pw} hinauf. Es war ein rechtes
               Vergnügen, ihn so heiter und wohlgelaunt, wie seit Jahren nicht, zu sehen.\pend
           
\pstart
           In herzlicher Erwartung Ihnen entgegen und aufrichtig ergeben Ihr{\\[\baselineskip]}\spacefill\mbox{{[}hs.:{]} Stefan Zweig}\pend
           \leftskip=0em{}
\pstart
           \noindent{}\uline{P. S.} Auch Bahr\pwindex{Bahr, Hermann 1858/1859 – 1939 Wien@\textsc{Bahr, Hermann} (1858/1859 – 1939 Wien), \emph{Privatbeamter}|pw} kommt in jenen Tagen aus München\oindex{München@\textbf{München}|pw} herüber.\pend
           \selectlanguage{ngerman}\endnumbering\briefempfaengerindex{Schnitzler, Arthur@\textsc{Schnitzler, Arthur}!zzzZweig, Stefan@\emph{von Stefan Zweig}!1923-07-281@{28. 7. 1923}|)be}\mylabel{L03686h}  \newcommand{\dateiname}{L03686}\newcommand{\titel}{Stefan Zweig an Arthur Schnitzler, 28. 7. 1923}\newcommand{\editorInnen}{Selma Jahnke und Martin Anton Müller}%% latex-leseansicht-abspann.tex
%% Abspann für die Leseansicht.
%% Der Schalter \ifkorrekturansicht ist bereits durch den Vorspann gesetzt.

%% latex-abspann.tex
%% Gemeinsamer Abspann für Korrekturansicht und Leseansicht.
%% Setzt den Schalter \ifkorrekturansicht voraus (gesetzt in den
%% einbindenden Dateien latex-korrekturansicht-abspann.tex bzw.
%% latex-leseansicht-abspann.tex).
%% ---------------------------------------------------------------

\normalsize

% Das esempio-Environment wird nur in der Leseansicht benötigt
\ifkorrekturansicht\else
\newenvironment{esempio}[3]%
{
    \vspace{1.5ex}
    \rlap{\underline{#1}}
    \par
    \setlength{\parindent}{0cm}
    \nopagebreak
    \leftskip=#2cm
    \rightskip=#3cm
}
{
    \par
}
\fi

\doendnotes{C}
\bigskip
\vfill

\clearpage

\footnotesize

\ifkorrekturansicht
  \lohead{\textsc{register}}
\fi

% theindex-Environment neu definieren ohne reledmac
\makeatletter
\renewenvironment{theindex}{%
  \ifkorrekturansicht
    \section*{\indexname}%
  \else
    \subsubsection*{Index der erwähnten Entitäten}%
  \fi
  \setlength{\parindent}{0pt}%
  \setlength{\parskip}{0pt plus 0.3pt}%
  \let\item\@idxitem
}{%
  \ifkorrekturansicht\clearpage\fi
}
\makeatother

\IfFileExists{\jobname-pw.ind}{\input{\jobname-pw.ind}}{}

% Quellenangabe nur in der Leseansicht
\ifkorrekturansicht\else
% Fallback-Definitionen, falls die .tex-Datei \titel etc. nicht gesetzt hat
\providecommand{\titel}{}
\providecommand{\editorInnen}{}
\providecommand{\dateiname}{\jobname}

\vspace{3cm}

\vfill

\footnotesize
\textsc{Quelle}: \titel. Herausgegeben von {\editorInnen}. In: \emph{Arthur Schnitzler: Briefwechsel mit Autorinnen und Autoren}.
 Digitale Edition, https://schnitzler-briefe.acdh.oeaw.ac.at/{\dateiname}.html (Stand \today)
\fi

\end{document}


