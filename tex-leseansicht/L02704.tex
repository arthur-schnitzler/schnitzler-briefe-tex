\input{../tex-inputs/latex-pdf-vorspann}
\begin{center}
            \textcolor{red}{ENTWURF. ENTZIFFERUNG NOCH NICHT KORREKTURGELESEN}
                      \end{center}
            
               \section[Paul Goldmann an Arthur Schnitzler, 24. 12. {[}1892{]}]{ Paul Goldmann an Arthur Schnitzler, 24. 12. {[}1892{]}}\nopagebreak\mylabel{v}\rehead{ }\begin{ledgroupsized}[t]{13cm}\normalsize\beginnumbering\briefempfaengerindex{Schnitzler, Arthur@\textsc{Schnitzler, Arthur}!zzzGoldmann, Paul@\emph{von Paul Goldmann}!1892-12-241@{24. 12. {[}1892{]}}|(be} \toendnotes[C]{\smallbreak\pagebreak[2]} \Standort{DLA, A:Schnitzler, HS.NZ85.1.3163.}
\physDesc{Brief, 6 Blätter, 22 Seiten
\newline{}Handschrift: schwarze Tinte, deutsche Kurrent
\newline{}Schnitzler: mit Bleistift das Jahr »92« vermerkt }\toendnotes[C]{\smallbreak}\pstart
           \centering{}{\pb}\textsc{Paris\oindex{Paris@\textbf{Paris}|pw}}, 24. December.\pend
           \pstart
           Alſo Weihnachtsabend. Aber nicht ſentimental,
               beileibe. Das thun wir hier nicht, das hält auf, das iſt reacſionär. Wir wollen
               vorwärts. Und darum müſſen wir ſtark werden. Was für einen ſchwachen Menſchen wohl
               nur ſoviel bedeutet, daß er daran vergißt, daß er eigentlich ſchwach iſt.\pend
           \pstart
           Mein theurer Freund! Es iſt Weihnachtsabend, und ich
               hätte \strikeout{\textcolor{gray}{g}} unter keinen Umſtänden Zeit, Dir zu ſchreiben, wenn ich nicht die \textsc{Chance} gehabt hätte, {\pb}vorgeſtern beim Herunterſteigen von der Tramway zu
               ſtürzen und mir die linke Schulter auszurenken. Man nennt das hier eine \textsc{\begin{otherlanguage}{french}luxation de l’épaule\end{otherlanguage}}, renkt das gewohnheitsmäßig falſch ein, renkt das dann wieder aus – 
                  \begin{otherlanguage}{french}\textsc{remettre}\end{otherlanguage}{ }\textsc{und}{ }\textsc{\begin{otherlanguage}{french}démettre\end{otherlanguage}}
                – und conſtatirt jedesmal, daß eine neue Gelenkkapſel oder Gelenkband – ich
               weiß nicht, wie das Zeug auf deutſch heißt – zerriſſen iſt. Der Tag geht für den
               Patienten unter dieſen Umſtänden nicht ohne heitere Zerſtreuungen vor{\pb}über. \label{K_L02704-1v}\edtext{\textsc{\begin{otherlanguage}{french}Mais, enfin\end{otherlanguage}}}{\lemma{\textnormal{\emph{Mais, enfin}}}\Cendnote{\textnormal{französisch: aber letztendlich}}}\label{K_L02704-1h}
               ich bin genöthigt, für einige Tage meinen Dienſt einzuſtellen – wenn nicht die
               Kurpfuſcher, in deren Händen ich hier bin, einige Wochen daraus machen – und vor
               Allem, ich ſitze heut{ }Abends müßig zuhauſe. Habe ich alſo geſucht, an der Sache eine gute
               Seite zu finden, habe eine ſehr künſtliche Inſtallation auf meinem Schreibtiſch
               gemacht, um das Papier ſeſthalten zu können, und habe mich dann niedergeſetzt, um {\pb}endlich einmal wieder mit Dir, Liebſter, zu plaudern.
               Und ſiehe da, es geht.\pend
           \pstart
           Ich ſehe zu meiner großen Herzenserleichterung – habe mir wirklich viel Sorge darüber
               gemacht – daß Du mir nicht bös biſt, weil ich Dir nicht antworte. Aber, weiß Gott, es
               geht nicht! Das Leben, das wir in dieſer böſen Zeit zu führen gezwungen ſind, iſt
               einfach unmenſchlich. Der Dienſt verſchlingt Alles, Eſſenszeit, Schlafenszeit, und
               nun gar erſt die {\pb}Zeit zum freundſchaftlichen
               Briefwechſel. An Dich gedacht? Oh, mein lieber Freund, wie oft, wie oft! Mitten im
               Sturm der \label{T_L02704-1v}\edtext{Eindrücke}{\lemma{\textnormal{\emph{Eindrücke}}}\Cendnote{\textnormal{Goldmann schreibt
                  »Eindrücken«}}}\label{T_L02704-1h}, mitten im feinem Kunſtgenuß, wo ich immer gar
               ſo gern mit Dir getheilt hätte. Und beſonders auch in dieſen Stunden der
               verzweifelten Verlaſſenheit und Lebensmüdigkeit, wo ich mich nach Dir geſehnt, als
               nach einem \uline{Menſchen}! Denn das gibt es hier nun wohl
               gar nicht. Ich habe immer den gleich ſtarken Wunſch, Dich {\pb}wiederzuſehen. Aber ich würde mich anderſeits doch
               davor fürchten; denn einmal habe ich Sorge davor, du würdeſt mich in Vielem verändert
               und nicht mehr ſo mit Dir zuſammenſtimmend finden; und dann fürchte ich, ich würde
               die Verlaſſenheit wieder ſchwerer ertragen und würde wieder arg mit meiner Wien\oindex{Paris@\textbf{Paris}|pw}-Sehnſucht zu ringen haben, die eine Form
               meiner Sentimentalität iſt, will ſagen meines Nichtvorwärtskommens, will ſagen \textsc{etc.} ſiehe oben. {\pb}Aber
               Eines begreiſe ich doch nicht: Ganz abgeſehen von dem zwiſchen mir und Dir. Sag’ mir:
               warum kommſt Du nicht nach \textsc{Paris\oindex{Paris@\textbf{Paris}|pw}}? Und zwar auf lange? Um jeden Preis? Glaub’ mir – ich ſehe es jetzt ſo
               deutlich, wie nur irgend etwas auf der Welt – es iſt für Deine ganze Entwickelung
               einfach unentbehrlich. Es wird Dir ekelhaft, abſcheulich, unerträglich ſein. Aber Du
               weißt ja, daß das die {\pb}Formen ſind, in denen die
               Entwickelungs-Kriſis aufzutreten pflegt. Und Du würdeſt hier eine ſolche Fülle neuer
               Ideen, – würdeſt ſo gewaltige \textsc{\begin{otherlanguage}{french}Chocs\end{otherlanguage}} bekommen – daß Du \strikeout{\textcolor{gray}{von}} am Ende wie ein neuer Menſch daſtehen und mit ganz anderen Augen ſehen
               würdeſt. Specieller: Das Leben in \textsc{Paris\oindex{Paris@\textbf{Paris}|pw}} entſubjectivirt, es objectivirt – und Du biſt unter allen Umſtänden
               verpflichtet, es auch damit zu verſuchen{[}.{]}{ }{\pb}Alſo komm’ her, mein lieber \textsc{Arthur}, – nicht meinetwegen. Ich würde Dich vielleicht alle drei Wochen
               einmal ſehen können, um Dich zu bitten, daß Du mir ein Nachtmahl zahlſt. Aber
               Deinetwegen! Folge mir! Du wirſt es nicht zu bereuen haben. Das heißt, Du wirſt es
               furchtbar bereuen. Aber es wird Dir ganz enorm geſund ſein.\pend
           \pstart
           Woraus Du nicht etwa ſchließen darfſt, daß ich mich hier wohl fühle. {\pb}Im Gegentheil! Entſetzlich elend. Heimathlos,
               verſtoßen, zuſchanden gearbeitet, angewidert, unbefriedigt \textsc{etc}. Aber eine große Compenſation dafür iſt da: Ich fühle, daß ich \uline{lerne}. Und ſolange das Gefühl anhält, will ich es
               muthig hier aushalten. Vom eigentlichen Lebensziel freilich ferner als je. Keine
               Selbſtändigkeit zu erblicken – kein Erwerb, kein Vermögen. Tagelohn und Schulden. {\pb}Keinen Weg zu den 12000 \textsc{frcs} Rente, die ich brauche. Weißt Du mir vielleicht einen? Dann komme ich
               gleich wieder, und dann bleiben und ſchaffen wir mitſammen. Oder irgend eine ſicher
               nicht-journaliſtiſche Stellung? Wenn Dir ſo etwas unter die Augen komm, denk’ bitte
               an mich! {\dotsfour}\pend
           \pstart
           Und nun Du. Vielen Dank für die Kritiken. Werth hat nur \label{K_L02704-3v}\edtext{die von Dr. \textsc{Meyer\pwindex{Meyer, F. @\textsc{Meyer, F.}, \emph{Journalist}|pw}}}{\lemma{\textnormal{\emph{die von Dr. Meyer}}}\Cendnote{\textnormal{f. m.\pwindex{Meyer, F. @\textsc{Meyer, F.}, \emph{Journalist}|pwk} [=F. Meyer\pwindex{Meyer, F. @\textsc{Meyer, F.}, \emph{Journalist}|pwk}]: \emph{[Mit unſerer öſterreichiſchen Literatur]}\pwindex{Mit unſerer oeſterreichiſchen Literatur]1892-11-06 – 1892-11-06@\emph{[Mit unſerer öſterreichiſchen Literatur]} {[}1892-11-06 – 1892-11-06{]}|pwk}. In: \emph{Berliner Neueste Nachrichten}\pwindex{Berliner Neueste Nachrichten1881 – 1919@\emph{Berliner Neueste Nachrichten}|pwk}, Jg. 12,
                     Nr. 563, 6. 11. 1892, S. [3].}}}\label{K_L02704-3h}. Es
               erhöht meinen {\pb}Reſpekt vor dem Mann\pwindex{Meyer, F. @\textsc{Meyer, F.}, \emph{Journalist}|pwv} beträchtlich, daß er einem Freunde\pwindex{David, Jakob Julius 06.02.1859 – 20.11.1906@\textsc{David, Jakob Julius} (06.02.1859 – 20.11.1906), \emph{Schriftsteller, Journalist}|pwv} ſo derb ſeine \label{K_L02704-4v}\edtext{Meinung}{\lemma{\textnormal{\emph{Meinung}}}\Cendnote{\textnormal{Meyer\pwindex{Meyer, F. @\textsc{Meyer, F.}, \emph{Journalist}|pwk} kritisierte in seinem kurzen Absatz\pwindex{Mit unſerer oeſterreichiſchen Literatur]1892-11-06 – 1892-11-06@\emph{[Mit unſerer öſterreichiſchen Literatur]} {[}1892-11-06 – 1892-11-06{]}|pwkv} zur österreichischen
                  Literatur auch Jakob Julius David\pwindex{David, Jakob Julius 06.02.1859 – 20.11.1906@\textsc{David, Jakob Julius} (06.02.1859 – 20.11.1906), \emph{Schriftsteller, Journalist}|pwk}
                  beziehungsweise dessen Stil (»Seine Probleme und Charaktere ſind einfach,
                     ſeine Sprache iſt knapp und alterthümelnd.«) vor der besprochenen Erzählsammlung\pwindex{David, Jakob Julius 06.02.1859 – 20.11.1906@\textsc{David, Jakob Julius} (06.02.1859 – 20.11.1906), \emph{Schriftsteller, Journalist}!Probleme1892 – 1892@\strich\emph{Probleme} {[}1892 – 1892{]}|pwkv}{ }\emph{Probleme}\pwindex{David, Jakob Julius 06.02.1859 – 20.11.1906@\textsc{David, Jakob Julius} (06.02.1859 – 20.11.1906), \emph{Schriftsteller, Journalist}!Probleme1892 – 1892@\strich\emph{Probleme} {[}1892 – 1892{]}|pwk}.}}}\label{K_L02704-4h} ſagt. Er hat zwar in der
               Sache meiner Anſicht nach Unrecht, aber als Offenheit iſt es werthzuſchätzen. Alle
               übrigen verſtehen Dich nicht, außer etwa \textsc{\label{K_L02704-5v}\edtext{Ludassy\pwindex{Gans-Ludassy, Julius von 13.04.1858 – 30.09.1922@\textsc{Gans-Ludassy, Julius von} (13.04.1858 – 30.09.1922), \emph{Schriftsteller, Journalist}|pw}}{\lemma{\textnormal{\emph{Ludassy}}}\Cendnote{\textnormal{Julius von Gans-Ludassy\pwindex{Gans-Ludassy, Julius von 13.04.1858 – 30.09.1922@\textsc{Gans-Ludassy, Julius von} (13.04.1858 – 30.09.1922), \emph{Schriftsteller, Journalist}|pwk}: \emph{Bücher}\pwindex{Gans-Ludassy, Julius von 13.04.1858 – 30.09.1922@\textsc{Gans-Ludassy, Julius von} (13.04.1858 – 30.09.1922), \emph{Schriftsteller, Journalist}!Buecher1892-12-19 – 1892-12-19@\strich\emph{Bücher} {[}1892-12-19 – 1892-12-19{]}|pwk}. In: \emph{Fremden-Blatt}\pwindex{Fremden-Blatt1.7.1847 – 22.3.1919@\emph{Fremden-Blatt}|pwk}, Jg. 46, Nr. 351, 19. 12. 1892, S. XXXX.}}}\label{K_L02704-5h}}. \label{K_L02704-6v}\edtext{\textsc{Bauer\pwindex{Bauer, Julius 15.10.1853 – 11.06.1941@\textsc{Bauer, Julius} (15.10.1853 – 11.06.1941), \emph{Schriftsteller, Journalist, Kritiker}|pw}}}{\lemma{\textnormal{\emph{Bauer}}}\Cendnote{\textnormal{[Julius Bauer\pwindex{Bauer, Julius 15.10.1853 – 11.06.1941@\textsc{Bauer, Julius} (15.10.1853 – 11.06.1941), \emph{Schriftsteller, Journalist, Kritiker}|pwk}]: \emph{Oeſterreichiſche Autoren}\pwindex{Oeſterreichiſche Autoren1892-12-19 – 1892-12-19@\emph{Oeſterreichiſche Autoren} {[}1892-12-19 – 1892-12-19{]}|pwk}. In: \emph{Neues Wiener Abendblatt}\pwindex{Neues Wiener Abendblatt@\emph{Neues Wiener Abendblatt}|pwk}, Jg. 26, Nr. 351, 19. 12. 1892, S. 3–4, hier: S. 3.}}}\label{K_L02704-6h}:
               eine lobende Notiz\pwindex{Oeſterreichiſche Autoren1892-12-19 – 1892-12-19@\emph{Oeſterreichiſche Autoren} {[}1892-12-19 – 1892-12-19{]}|pwv} mit
               Rückſicht darauf, daß man in dem Hauſe dinirt und ſich die Beziehung zu Deinem Papa-\label{K_L02704-7v}\edtext{Regierungsrath}{\lemma{\textnormal{\emph{Regierungsrath}}}\Cendnote{\textnormal{Johann Schnitzler\pwindex{Schnitzler, Johann 10.04.1835 – 02.05.1893@\textsc{Schnitzler, Johann} (10.04.1835 – 02.05.1893), \emph{Laryngologe}|pwk} wurde
                        1883 zum Regierungsrat\pwindex{Schnitzler, Johann 10.04.1835 – 02.05.1893@\textsc{Schnitzler, Johann} (10.04.1835 – 02.05.1893), \emph{Laryngologe}|pwkv} ernannt. Julius
                        Bauer\pwindex{Bauer, Julius 15.10.1853 – 11.06.1941@\textsc{Bauer, Julius} (15.10.1853 – 11.06.1941), \emph{Schriftsteller, Journalist, Kritiker}|pwk} nannte Arthur Schnitzler in seiner Rezension\pwindex{Oeſterreichiſche Autoren1892-12-19 – 1892-12-19@\emph{Oeſterreichiſche Autoren} {[}1892-12-19 – 1892-12-19{]}|pwkv} den »Sohn des
                        bekannten Profeſſors\pwindex{Schnitzler, Johann 10.04.1835 – 02.05.1893@\textsc{Schnitzler, Johann} (10.04.1835 – 02.05.1893), \emph{Laryngologe}|pwv} Dr. Schnitzler\pwindex{Schnitzler, Johann 10.04.1835 – 02.05.1893@\textsc{Schnitzler, Johann} (10.04.1835 – 02.05.1893), \emph{Laryngologe}|pwv}«.}}}\label{K_L02704-7h}\pwindex{Schnitzler, Johann 10.04.1835 – 02.05.1893@\textsc{Schnitzler, Johann} (10.04.1835 – 02.05.1893), \emph{Laryngologe}|pwv} erhalten will. \label{K_L02704-8v}\edtext{\textsc{Nossig\pwindex{Nossig, Alfred 18.04.1863 – 22.02.1943@\textsc{Nossig, Alfred} (18.04.1863 – 22.02.1943), \emph{Journalist/Journalistin}|pw}}}{\lemma{\textnormal{\emph{Nossig}}}\Cendnote{\textnormal{Rezension\pwindex{Nossig, Alfred 18.04.1863 – 22.02.1943@\textsc{Nossig, Alfred} (18.04.1863 – 22.02.1943), \emph{Journalist/Journalistin}!Anatol-Rezension]1892 – 1892@\strich\emph{[Anatol-Rezension]} {[}1892 – 1892{]}|pwkv} nicht ermittelt
                  XXXX evtl. könnte es die Rezension vom 3. 12. 1892 im Extrablatt sein, aber das
                  lässt sich nicht verifizieren; andere Möglichkeit wäre z. B. auch Wiener
                  Allgemeine Zeitung XXXX}}}\label{K_L02704-8h}: {\pb}einer, der auf
               Beides – die \strikeout{Dine} Diners und die Beziehung –
               candidirt. Macht aber nichts; ſie ſollen nur von dir ſprechen. Der Ruf wird ja nicht
               dadurch zunächſt gemacht, daß man verſtanden, ſondern dadurch, daß überſaupt von
               Einem geſprochen wird. Ich ſelbſt hätte längſt über Dich ſchreiben ſollen. Aber wann?
               Pure phyſiſche Unmöglichkeit, das ich Dich doch nicht damit {\pb}beſchimpfen will, daß ich eine Reklamnotiz für Dich
               zuſammenſchmiere. Die Sache mußte künſtleriſch verarbeitet werden. Aber ich habe
               nicht eine Stunde dafür gehabt. Soll alſo inzwiſchen der \label{K_L02704-12v}\edtext{Andere}{\lemma{\textnormal{\emph{Andere}}}\Cendnote{\textnormal{nicht
                  identifiziert}}}\label{K_L02704-12h} ſchreiben – der Berlin\oindex{Berlin@\textbf{Berlin}|pw}er
               – ein ganz braver Menſch, \strikeout{b\textcolor{gray}{o}} bornirt, aber nach der guten Richtung bornirt, d.h. mit einem dummen
               Vorurtheil für das Moderne be{\pb}haftet, was Dir
               zuſtatten kommen wird. Er wird wohl bald \label{K_L02704-13v}\edtext{losſchießen}{\lemma{\textnormal{\emph{losſchießen}}}\Cendnote{\textnormal{XXXX Anatol-Rezension des Berliners erschienen? XXXX}}}\label{K_L02704-13h}. Und dann kann ich ja
               immer noch das Wort nehmen, wie es mein ſehnlicher Wunſch und feſter Vorſatz iſt. \textsc{Herzl\pwindex{Herzl, Theodor 02.05.1860 – 03.07.1904@\textsc{Herzl, Theodor} (02.05.1860 – 03.07.1904), \emph{Schriftsteller, Journalist}|pw}} aber wird nicht ſchreiben. Ich habe mein Möglichſtes gethan – ich bin ſoweit
               gegangen, als ich gehen konnte, – aber, ein ſo braver Menſch\pwindex{Herzl, Theodor 02.05.1860 – 03.07.1904@\textsc{Herzl, Theodor} (02.05.1860 – 03.07.1904), \emph{Schriftsteller, Journalist}|pwv} er iſt, ſo kennſt Du doch auch
               ſeinen {\pb}Größenwahn. Und er hat mir auf meine
               Andeutungen in einer Weiſe geantwortet, daß ich nicht mehr darauf zurückkommen
               konnte, ohne Dich bloßzuſtellen. (»Wenn er mir ſein Buch\pwindex{Schnitzler, Arthur 15.05.1862 – 21.10.1931@\textsc{Schnitzler, Arthur} (15.05.1862 – 21.10.1931), \emph{Schriftsteller, Mediziner}!Anatol1892-10-29 – 1892-10-29@\strich\emph{Anatol} {[}1892-10-29 – 1892-10-29{]}|pwv} deshalb geſchickt hat, damit ich darüber ſchreibe \textsc{etc}« {\dotsfour})\pend
           \pstart
           Und nun Dein Stück\pwindex{Schnitzler, Arthur 15.05.1862 – 21.10.1931@\textsc{Schnitzler, Arthur} (15.05.1862 – 21.10.1931), \emph{Schriftsteller, Mediziner}!Maerchen. Schauspiel in drei Aufzuegen1891 – 1891@\strich\emph{Das Märchen. Schauspiel in drei Aufzügen} {[}1891 – 1891{]}|pwv}? Auf wann
               die \label{K_L02704-9v}\edtext{Aufführung}{\lemma{\textnormal{\emph{Aufführung}}}\Cendnote{\textnormal{Erst ein knappes Jahr später, am 1. 12. 1893, kam es zur Uraufführung des \emph{Märchen}\pwindex{Schnitzler, Arthur 15.05.1862 – 21.10.1931@\textsc{Schnitzler, Arthur} (15.05.1862 – 21.10.1931), \emph{Schriftsteller, Mediziner}!Maerchen. Schauspiel in drei Aufzuegen1891 – 1891@\strich\emph{Das Märchen. Schauspiel in drei Aufzügen} {[}1891 – 1891{]}|pwk}s am \emph{Deutschen
                     Volkstheater}\orgindex{Volkstheater@Volkstheater|pwk} in Wien\oindex{Wien@\textbf{Wien}|pwk}. Zuvor lehnte das
                     \emph{Burgtheater}\orgindex{Burgtheater@Burgtheater|pwk} das \emph{Märchen}\pwindex{Schnitzler, Arthur 15.05.1862 – 21.10.1931@\textsc{Schnitzler, Arthur} (15.05.1862 – 21.10.1931), \emph{Schriftsteller, Mediziner}!Maerchen. Schauspiel in drei Aufzuegen1891 – 1891@\strich\emph{Das Märchen. Schauspiel in drei Aufzügen} {[}1891 – 1891{]}|pwk} ab, wie Schnitzler am 19. 11. 1892 im \emph{Tagebuch}\pwindex{Schnitzler, Arthur 15.05.1862 – 21.10.1931@\textsc{Schnitzler, Arthur} (15.05.1862 – 21.10.1931), \emph{Schriftsteller, Mediziner}!Tagebuch1981 – 2000@\strich\emph{Tagebuch} {[}1981 – 2000{]}|pwk} notierte. Außerdem war eine Aufführung in der zweiten Hälfte des
                     Januars 1893 am \emph{Neuen Deutschen Theater}\orgindex{Neues Deutsches Theater@Neues Deutsches Theater|pwk} in Prag\oindex{Prag@\textbf{Prag}|pwk}
                  geplant, die jedoch ebenso nicht stattfand Siehe Paul Goldmann an Arthur Schnitzler, 27. 6. [1892] wie Bemühungen um eine Aufführung am Berlin\oindex{Berlin@\textbf{Berlin}|pwk}er \emph{Lessing-Theater}\orgindex{Lessing-Theater@Lessing-Theater|pwk}
                  gelingen wollten Siehe A. S.: \emph{Tagebuch}, 18. 3. 1893.}}}\label{K_L02704-9h}? Und das neue Stück\pwindex{Schnitzler, Arthur 15.05.1862 – 21.10.1931@\textsc{Schnitzler, Arthur} (15.05.1862 – 21.10.1931), \emph{Schriftsteller, Mediziner}!Abschiedssouper1892@\strich\emph{Abschiedssouper} {[}1892{]}|pwuv}? Und Deine \label{K_L02704-14v}\edtext{Novellen\pwindex{Schnitzler, Arthur 15.05.1862 – 21.10.1931@\textsc{Schnitzler, Arthur} (15.05.1862 – 21.10.1931), \emph{Schriftsteller, Mediziner}!Sterben. Novelle1.10.1894 – 1.12.1894@\strich\emph{Sterben. Novelle} {[}1.10.1894 – 1.12.1894{]}|pwv}}{\lemma{\textnormal{\emph{Novellen}}}\Cendnote{\textnormal{Goldmann bezog sich hier womöglich auf
                  den \emph{Anatol}\pwindex{Schnitzler, Arthur 15.05.1862 – 21.10.1931@\textsc{Schnitzler, Arthur} (15.05.1862 – 21.10.1931), \emph{Schriftsteller, Mediziner}!Anatol1892-10-29 – 1892-10-29@\strich\emph{Anatol} {[}1892-10-29 – 1892-10-29{]}|pwk}-Einakter\pwindex{Schnitzler, Arthur 15.05.1862 – 21.10.1931@\textsc{Schnitzler, Arthur} (15.05.1862 – 21.10.1931), \emph{Schriftsteller, Mediziner}!Abschiedssouper1892@\strich\emph{Abschiedssouper} {[}1892{]}|pwkv}{ }\emph{Abschiedssouper}\pwindex{Schnitzler, Arthur 15.05.1862 – 21.10.1931@\textsc{Schnitzler, Arthur} (15.05.1862 – 21.10.1931), \emph{Schriftsteller, Mediziner}!Abschiedssouper1892@\strich\emph{Abschiedssouper} {[}1892{]}|pwk}, der am 14. 7. 1893 im Bad Ischl\oindex{Bad Ischl@\textbf{Bad Ischl}|pwk}er Stadttheater\oindex{Stadttheater (Bad Ischl)@\textbf{Stadttheater (Bad Ischl)}|pwk} uraufgeführt wurde. Eine der erwähnten
                     »Novellen« könnte \emph{Sterben}\pwindex{Schnitzler, Arthur 15.05.1862 – 21.10.1931@\textsc{Schnitzler, Arthur} (15.05.1862 – 21.10.1931), \emph{Schriftsteller, Mediziner}!Sterben. Novelle1.10.1894 – 1.12.1894@\strich\emph{Sterben. Novelle} {[}1.10.1894 – 1.12.1894{]}|pwk}
                  sein.}}}\label{K_L02704-14h}? Und, ſag mir nur, warum {\pb}biſt Du
               ein ſo elender Menſch und ich ſchreibſt mir nichts PerſönIiches mehr? Weißt Du, daß
               Du mich glücklich aus Deinem Leben herausgeworfen haſt? Und daß Du mich auf
               literariſche Diät geſetzt haſt? Literariſcher Beirath! Aber Arthur! Pfui Teufel!
               Schämſt Du Dich denn gar nicht? {\dots}\pend
           \pstart
           Ich habe Jemanden\pwindex{Kanner, Heinrich 09.11.1864 – 15.02.1930@\textsc{Kanner, Heinrich} (09.11.1864 – 15.02.1930), \emph{Publizist}|pwv} für Euren
               lieben Kreis. {\pb}Das ſympathiſcheſte Mitglied\pwindex{Kanner, Heinrich 09.11.1864 – 15.02.1930@\textsc{Kanner, Heinrich} (09.11.1864 – 15.02.1930), \emph{Publizist}|pwv} hat ſich aus unſerer Redaction\orgindex{Frankfurter Zeitung@Frankfurter Zeitung|pwv} losgelöſt, weil es von
                  \textsc{Sonnemann\pwindex{Sonnemann, Leopold 1831-10-29 – 1909-10-30@\textsc{Sonnemann, Leopold} (1831-10-29 – 1909-10-30), \emph{Journalist, Herausgeber}|pw}} denn doch gar zu ſehr chicanirt wurde, und iſt – Wien\oindex{Wien@\textbf{Wien}|pw}er von \label{K_L02704-2v}\edtext{Geburt}{\lemma{\textnormal{\emph{Geburt}}}\Cendnote{\textnormal{Heinrich Kanner\pwindex{Kanner, Heinrich 09.11.1864 – 15.02.1930@\textsc{Kanner, Heinrich} (09.11.1864 – 15.02.1930), \emph{Publizist}|pwk} wurde in Galatz\oindex{Galatz@\textbf{Galatz}|pwk} (Rumänien\oindex{Rumaenien@\textbf{Rumänien}|pwk})
                  geboren, zog aber als Kleinkind im Jahr 1866 mit seiner Familie nach
                     Wien\oindex{Wien@\textbf{Wien}|pwkv}.}}}\label{K_L02704-2h} und Erziehung
               – unſer Wien\oindex{Wien@\textbf{Wien}|pw}er Correſpondent\pwindex{Kanner, Heinrich 09.11.1864 – 15.02.1930@\textsc{Kanner, Heinrich} (09.11.1864 – 15.02.1930), \emph{Publizist}|pwv} geworden. \textsc{Dr. Heinrich Kanner\pwindex{Kanner, Heinrich 09.11.1864 – 15.02.1930@\textsc{Kanner, Heinrich} (09.11.1864 – 15.02.1930), \emph{Publizist}|pw}} – Adreſſe wird Dir Dr. \textsc{Joachim\pwindex{Joachim, Jaques 24.11.1866 – 07.11.1925@\textsc{Joachim, Jaques} (24.11.1866 – 07.11.1925), \emph{Rechtswissenschaftler, Rechtsanwalt, Herausgeber}|pw}} ſagen, oder ich ſchreib’ ſie Dir auf – einer\pwindex{Kanner, Heinrich 09.11.1864 – 15.02.1930@\textsc{Kanner, Heinrich} (09.11.1864 – 15.02.1930), \emph{Publizist}|pwv} der liebſten Leute, die mir überhaupt be{\pb}gegnet ſind. Kein Künſtler ſondern Volkswirth\pwindex{Kanner, Heinrich 09.11.1864 – 15.02.1930@\textsc{Kanner, Heinrich} (09.11.1864 – 15.02.1930), \emph{Publizist}|pwv} und Politiker\pwindex{Kanner, Heinrich 09.11.1864 – 15.02.1930@\textsc{Kanner, Heinrich} (09.11.1864 – 15.02.1930), \emph{Publizist}|pwv}. Aber doch vielleicht Künſtlernatur\pwindex{Kanner, Heinrich 09.11.1864 – 15.02.1930@\textsc{Kanner, Heinrich} (09.11.1864 – 15.02.1930), \emph{Publizist}|pwv}, vor Allem aber
               ein wahres Ideal\pwindex{Kanner, Heinrich 09.11.1864 – 15.02.1930@\textsc{Kanner, Heinrich} (09.11.1864 – 15.02.1930), \emph{Publizist}|pwv} an
               Geſcheitheit, Feinſinn und \textsc{Noblesse}. Geh’, ſetz Dich mit
               ihm in \label{K_L02704-10v}\edtext{Verbindung}{\lemma{\textnormal{\emph{Verbindung}}}\Cendnote{\textnormal{Es sind keine Briefe zwischen Schnitzler
                  und Heinrich Kanner\pwindex{Kanner, Heinrich 09.11.1864 – 15.02.1930@\textsc{Kanner, Heinrich} (09.11.1864 – 15.02.1930), \emph{Publizist}|pwk}, der außerdem erst am
                     24. 9. 1896 im \emph{Tagebuch}\pwindex{Schnitzler, Arthur 15.05.1862 – 21.10.1931@\textsc{Schnitzler, Arthur} (15.05.1862 – 21.10.1931), \emph{Schriftsteller, Mediziner}!Tagebuch1981 – 2000@\strich\emph{Tagebuch} {[}1981 – 2000{]}|pwk} erwähnt wurde, bekannt.}}}\label{K_L02704-10h}. Wirſt
               Deine Freude daran haben{\dotsfive}\pend
           \pstart
           Von ganzem Herzen ein frohes neues Jahr, mein theurer Freund! {\pb}Arbeitsluſt! Erfolg! Und vorwärts! Die allerwärmſten
               Grüße an \textsc{Loris\pwindex{Hofmannsthal, Hugo von 01.02.1874 – 15.07.1929@\textsc{Hofmannsthal, Hugo von} (01.02.1874 – 15.07.1929), \emph{Schriftsteller}|pw}} und \textsc{Richard\pwindex{Beer-Hofmann, Richard 11.07.1866 – 26.09.1945@\textsc{Beer-Hofmann, Richard} (11.07.1866 – 26.09.1945), \emph{Schriftsteller}|pw}} (\textsc{Richard\pwindex{Beer-Hofmann, Richard 11.07.1866 – 26.09.1945@\textsc{Beer-Hofmann, Richard} (11.07.1866 – 26.09.1945), \emph{Schriftsteller}|pw}} ſoll mir ſchreiben!!!). Ergebene Empfehlungen und Neujahrswünſche an Deine Eltern\pwindex{Schnitzler, Louise 08.07.1840 – 09.09.1911@\textsc{Schnitzler, Louise} (08.07.1840 – 09.09.1911)|pwv}\pwindex{Schnitzler, Johann 10.04.1835 – 02.05.1893@\textsc{Schnitzler, Johann} (10.04.1835 – 02.05.1893), \emph{Laryngologe}|pwv}. Grüße an
               Deinen Bruder\pwindex{Schnitzler, Julius 13.07.1865 – 29.06.1939@\textsc{Schnitzler, Julius} (13.07.1865 – 29.06.1939), \emph{Chirurg}|pwv}, \textsc{Kapper\pwindex{Kapper, Friedrich 21.04.1861 – 22.07.1939@\textsc{Kapper, Friedrich} (21.04.1861 – 22.07.1939), \emph{Mediziner}|pw}} und wen ich ſonſt noch in Wien\oindex{Wien@\textbf{Wien}|pw} lieb habe,
               was Du ja ebenſo wohl weißt wie ich.\pend
           \pstart
           Und ich umarme Dich von ganzem Herzen, {\pb}in alter,
               unwandelbarer, treuer Freundſchaft.\pend
           \pstart
           Dein {\\[\baselineskip]}\spacefill\mbox{Paul Goldm}\pend
           \leftskip=0em{}\pstart
           \noindent{}Der kleinen Elſe\pwindex{Singer, Else 25.06.1878 – 1943?@\textsc{Singer, Else} (25.06.1878 – 1943?), \emph{Schriftstellerin, Sprachlehrerin}|pw}: Handkuß, und ich hab’ die
                  Sachen leider ſelbſt nicht mehr. Liegt auch ſo weit hinter mir. Will mich auch gar
                  nicht mehr daran erinnern, daß ich einmal Künſtler werden wollte und daß es kleine
                     Elſe\pwindex{Singer, Else 25.06.1878 – 1943?@\textsc{Singer, Else} (25.06.1878 – 1943?), \emph{Schriftstellerin, Sprachlehrerin}|pw}n in der {\pb}Welt gibt. Das thut ſo weh!\pend
           \pstart
           Und ſag’ einmal: Könnteſt Du nicht unter der Hand einmal und ganz zufällig
                  erfahren, was \label{K_L02704-15v}\edtext{\textsc{Hilda\pwindex{Mitis, Hilda von 1876-08-30 – 1894-12-14@\textsc{Mitis, Hilda von} (1876-08-30 – 1894-12-14), \emph{Schriftstellerin, Telefonistin}|pw}}}{\lemma{\textnormal{\emph{Hilda}}}\Cendnote{\textnormal{Siehe Paul Goldmann an Arthur Schnitzler, 27. 4. 1891}}}\label{K_L02704-15h} macht? Ich glaube, ich habe mich da doch wie ein Schaf benommen. Dieſes
                  aber unter uns.\pend
           \pstart
           Bald einen Brief, nicht wahr? Theils literariſch, theils perſönlich!\pend
           \endnumbering\briefempfaengerindex{Schnitzler, Arthur@\textsc{Schnitzler, Arthur}!zzzGoldmann, Paul@\emph{von Paul Goldmann}!1892-12-241@{24. 12. {[}1892{]}}|)be}\mylabel{h}\end{ledgroupsized}\begin{anhang}\end{anhang}\newcommand{\dateiname}{L02704}\newcommand{\titel}{Paul Goldmann an Arthur Schnitzler, 24. 12. [1892]}\newcommand{\editorInnen}{Martin Anton Müller und Laura Untner}\input{../tex-inputs/latex-pdf-abspann}
      