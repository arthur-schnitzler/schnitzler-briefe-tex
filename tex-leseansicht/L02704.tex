%% latex-leseansicht-vorspann.tex
%% Vorspann für die Leseansicht.
%% Lädt die gemeinsame Datei latex-vorspann.tex mit nicht gesetztem Schalter.

\newif\ifkorrekturansicht
\korrekturansichtfalse

\input{../tex-inputs/latex-vorspann}

\begin{center}
            \textcolor{red}{ENTWURF, NICHT FERTIG KORRIGIERT}
                      \end{center}
            
               \section[Paul Goldmann an Arthur Schnitzler, 24. 12. {[}1892{]}]{ Paul Goldmann an Arthur Schnitzler, 24. 12. {[}1892{]}}\nopagebreak\mylabel{v}\rehead{ }\begin{ledgroupsized}[t]{13cm}\normalsize\beginnumbering\briefempfaengerindex{Schnitzler, Arthur@\textsc{Schnitzler, Arthur}!zzzGoldmann, Paul@\emph{von Paul Goldmann}!1892-12-241@{24. 12. {[}1892{]}}|(be} \toendnotes[C]{\smallbreak\pagebreak[2]} \Standort{DLA, A:Schnitzler, HS.NZ85.1.3163.}
\physDesc{Brief, 6 Blätter, 22 Seiten
\newline{}Handschrift: schwarze Tinte, deutsche Kurrent
\newline{}Schnitzler: mit Bleistift das Jahr »92« vermerkt }\toendnotes[C]{\smallbreak}\pstart
           \raggedleft{}{\pb}\textsc{Paris\oindex{Paris@\textbf{Paris}|pw}}, 24. December.\pend
           \pstart
           Alſo Weihnachtsabend. Aber nicht ſentimental,
               beileibe. Das thun wir hier nicht, das hält auf, das iſt reacſionär. Wir wollen
               vorwärts. Und darum müſſen wir ſtark werden. Was für einen ſchwachen Menſchen wohl
               nur ſoviel bedeutet, daß er daran vergißt, daß er eigentlich ſchwach iſt.\pend
           \pstart
           Mein theurer Freund! Es iſt Weihnachtsabend, und ich
               hätte \strikeout{\textcolor{gray}{g}} unter keinen Umſtänden Zeit, Dir zu ſchreiben –, wenn ich nicht die \textsc{Chance} gehabt hätte, {\pb}vorgeſtern beim Herunterſteigen von der Tramway zu
               ſtürzen und mir die linke Schulter auszurenken. Man nennt das hier eine \textsc{\begin{otherlanguage}{french}luxation de l’épaule\end{otherlanguage}}, renkt das gewohnheitsmäßig falſch ein, renkt das dann wieder aus – \begin{otherlanguage}{french}\textsc{remettre}\end{otherlanguage}{ }\textsc{und}{ }\textsc{\begin{otherlanguage}{french}démettre\end{otherlanguage}} – und conſtatirt jedesmal, daß eine neue Gelenkkapſel oder Gelenkband – ich
               weiß nicht, wie das Zeug auf deutſch heißt – zerriſſen iſt. Der Tag geht für den
               Patienten unter dieſen Umſtänden nicht ohne heitere Zerſtreuungen vor{\pb}über. \label{K_L02704-1v}\edtext{\textsc{\begin{otherlanguage}{french}Mais, enfin\end{otherlanguage}}}{\lemma{\textnormal{\emph{Mais, enfin}}}\Cendnote{\textnormal{französisch: aber letztendlich}}}\label{K_L02704-1h} –
               ich bin genöthigt, für einige Tage meinen Dienſt einzuſtellen – wenn nicht die
               Kurpfuſcher, in deren Händen ich hier bin, einige Wochen daraus machen – und vor
               Allem, ich ſitze heut{ }Abends müßig zuhauſe. Habe ich alſo geſucht, an der Sache eine gute
               Seite zu finden, habe eine ſehr künſtliche Inſtallation auf meinem Schreibtiſch
               gemacht, um das Papier ſeſthalten zu können, und habe mich dann niedergeſetzt, um {\pb}endlich einmal wieder mit Dir, Liebſter, zu plaudern.
               Und ſiehe da, es geht.\pend
           \pstart
           Ich ſehe zu meiner großen Herzenserleichterung – habe mir wirklich viel Sorge darüber
               gemacht – daß Du mir nicht bös biſt, weil ich Dir nicht antworte. Aber, weiß Gott, es
               geht nicht! Das Leben, das wir in dieſer böſen Zeit zu führen gezwungen ſind, iſt
               einfach unmenſchlich. Der Dienſt verſchlingt Alles, Eſſenszeit, Schlafenszeit, und
               nun gar erſt die {\pb}Zeit zum freundſchaftlichen
               Briefwechſel. An Dich gedacht? Oh, mein lieber Freund, wie oft, wie oft! Mitten im
               Sturm der \label{T_L02704-1v}\edtext{Eindrücke}{\lemma{\textnormal{\emph{Eindrücke}}}\Cendnote{\textnormal{Goldmann schreibt
                  »Eindrücken«}}}\label{T_L02704-1h}, mitten im feinem Kunſtgenuß, wo ich immer gar
               ſo gern mit Dir getheilt hätte. Und beſonders auch in dieſen Stunden der
               verzweifelten Verlaſſenheit und Lebensmüdigkeit, wo ich mich nach Dir geſehnt, als
               nach einem \uline{Menſchen}! Denn das gibt es hier nun wohl
               gar nicht. Ich habe immer den gleich ſtarken Wunſch, Dich {\pb}wiederzuſehen. Aber ich würde mich anderſeits doch
               davor fürchten; denn einmal habe ich Sorge davor, du würdeſt mich in Vielem verändert
               und nicht mehr ſo mit Dir zuſammenſtimmend finden; und dann fürchte ich, ich würde
               die Verlaſſenheit wieder ſchwerer ertragen und würde wieder arg mit meiner Wien\oindex{Paris@\textbf{Paris}|pw}-Sehnſucht zu ringen haben, die eine Form
               meiner Sentimentalität iſt, will ſagen meines Nichtvorwärtskommens, will ſagen \textsc{etc.} ſiehe oben. {\pb}Aber
               Eines begreiſe ich doch nicht: Ganz abgeſehen von dem zwiſchen mir und Dir. Sag’ mir:
               warum kommſt Du nicht nach \textsc{Paris\oindex{Paris@\textbf{Paris}|pw}}? Und zwar auf lange? Um jeden Preis? Glaub’ mir – ich ſehe es jetzt ſo
               deutlich, wie nur irgend etwas auf der Welt – es iſt für Deine ganze Entwickelung
               einfach unentbehrlich. Es wird Dir ekelhaft, abſcheulich, unerträglich ſein. Aber Du
               weißt ja, daß das die {\pb}Formen ſind, in denen die
               Entwickelungs-Kriſis aufzutreten pflegt. Und Du würdeſt hier eine ſolche Fülle neuer
               Ideen, – würdeſt ſo gewaltige \textsc{\begin{otherlanguage}{french}Chocs\end{otherlanguage}} bekommen, – daß Du \strikeout{\textcolor{gray}{von}} am Ende wie ein neuer Menſch daſtehen und mit ganz anderen Augen ſehen
               würdeſt. Specieller: Das Leben in \textsc{Paris\oindex{Paris@\textbf{Paris}|pw}} entſubjectivirt, es objectivirt – und Du biſt unter allen Umſtänden
               verpflichtet, es auch damit zu verſuchen{[}.{]}{ }{\pb}Alſo komm’ her, mein lieber \textsc{Arthur}, – nicht meinetwegen. Ich würde Dich vielleicht alle drei Wochen
               einmal ſehen können, um Dich zu bitten, daß Du mir ein Nachtmahl zahlſt. Aber
               Deinetwegen! Folge mir! Du wirſt es nicht zu bereuen haben! Das heißt, Du wirſt es
               furchtbar bereuen. Aber es wird Dir ganz enorm geſund ſein.\pend
           \pstart
           Woraus Du nicht etwa ſchließen darfſt, daß ich mich hier wohl fühle. {\pb}Im Gegentheil! Entſetzlich elend. Heimathlos,
               verſtoßen, zuſchanden gearbeitet, angewidert, unbefriedigt \textsc{etc}. Aber eine große Compenſation dafür iſt da: Ich fühle, daß ich \uline{lerne}. Und ſolange das Gefühl anhält, will ich es
               muthig hier aushalten. Vom eigentlichen Lebensziel freilich ferner als je. Keine
               Selbſtändigkeit zu erblicken – kein Erwerb, kein Vermögen. Tagelohn und Schulden. {\pb}Keinen Weg zu den 12000 \textsc{frcs} Rente, die ich brauche. Weißt Du mir vielleicht einen? Dann komme ich
               gleich wieder, und dann bleiben und ſchaffen wir mitſammen. Oder irgend eine ſicher
               nicht-journaliſtiſche Stellung? Wenn Dir ſo etwas unter die Augen komm, denk’ bitte
               an mich! {\dotsfour}\pend
           \pstart
           Und nun Du. Vielen Dank für die Kritiken. Werth hat nur \label{K_L02704-3v}\edtext{die von Dr. \textsc{Meyer\pwindex{Fels, Friedrich Michael *~1864@\textsc{Fels, Friedrich Michael} (*~1864), \emph{Journalist}|pw}}}{\lemma{\textnormal{\emph{die von Dr. Meyer}}}\Cendnote{\textnormal{f. m.\pwindex{Fels, Friedrich Michael *~1864@\textsc{Fels, Friedrich Michael} (*~1864), \emph{Journalist}|pwk} [=Friedrich M. Fels\pwindex{Fels, Friedrich Michael *~1864@\textsc{Fels, Friedrich Michael} (*~1864), \emph{Journalist}|pwk}]: \emph{[Mit unſerer
                        öſterreichiſchen Literatur]}\pwindex{Mit unſerer oeſterreichiſchen Literatur]1892-11-06@\emph{[Mit unſerer öſterreichiſchen Literatur]} {[}1892-11-06{]}|pwk}. In: \emph{Berliner Neueste Nachrichten}\pwindex{Berliner Neueste Nachrichten1881 – 1919@\emph{Berliner Neueste Nachrichten}|pwk}, Jg. 12, Nr. 563, 6. 11. 1892, S. [3]. Die »Entschlüsselung« des Kürzels
                  erfolgt einerseits durch Goldmann\pwindex{Goldmann, Paul 31.01.1865 – 25.09.1935@\textsc{Goldmann, Paul} (31.01.1865 – 25.09.1935), \emph{Schriftsteller, Journalist}|pwk} selbst,
                  indem er ihn als »Dr. Meyer« und Freund von Jakob Julius David\pwindex{David, Jakob Julius 1859-02-06 – 1906-11-20@\textsc{David, Jakob Julius} (1859-02-06 – 1906-11-20), \emph{Schriftsteller, Journalist}|pwk} identifiziert. Andererseits weist die
                  ausführlichere – und kritische – Rezension von \emph{Anatol}\pwindex{Schnitzler, Arthur 15.05.1862 – 21.10.1931@\textsc{Schnitzler, Arthur} (15.05.1862 – 21.10.1931), \emph{Schriftsteller, Mediziner}!Anatol1892-10-29@\strich\emph{Anatol} {[}1892-10-29{]}|pwk} durch Friedrich M. Fels\pwindex{Fels, Friedrich Michael *~1864@\textsc{Fels, Friedrich Michael} (*~1864), \emph{Journalist}|pwk}
                  einige sprachliche Gemeinsamkeiten auf (»graziöse«, »feinsinnige Plaudereien«),
                  die die gleiche Quelle erkennbar machen. (F. M. F.\pwindex{Fels, Friedrich Michael *~1864@\textsc{Fels, Friedrich Michael} (*~1864), \emph{Journalist}|pwk}: \emph{»Anatol.« Von Arthur Schnitzler}\pwindex{Anatol.« Von Arthur Schnitzler1892-11-15@\emph{»Anatol.« Von Arthur Schnitzler} {[}1892-11-15{]}|pwk}. In: \emph{Allgemeine Kunst-Chronik}\pwindex{Allgemeine Kunst-Chronik1881 – 1896@\emph{Allgemeine Kunst-Chronik}|pwk}, Bd. 16, Nr. 24, 2. November-Heft
                        1892, S. 614.)}}}\label{K_L02704-3h}. Es erhöht meinen {\pb}Reſpekt vor dem Mann\pwindex{Fels, Friedrich Michael *~1864@\textsc{Fels, Friedrich Michael} (*~1864), \emph{Journalist}|pwv} beträchtlich, daß er einem Freunde\pwindex{David, Jakob Julius 1859-02-06 – 1906-11-20@\textsc{David, Jakob Julius} (1859-02-06 – 1906-11-20), \emph{Schriftsteller, Journalist}|pwv} ſo derb ſeine \label{K_L02704-4v}\edtext{Meinung}{\lemma{\textnormal{\emph{Meinung}}}\Cendnote{\textnormal{Fels\pwindex{Fels, Friedrich Michael *~1864@\textsc{Fels, Friedrich Michael} (*~1864), \emph{Journalist}|pwk} kritisierte in seiner kurzen Besprechung\pwindex{Mit unſerer oeſterreichiſchen Literatur]1892-11-06@\emph{[Mit unſerer öſterreichiſchen Literatur]} {[}1892-11-06{]}|pwkv} den Stil der
                  Erzählsammlung { }\emph{Probleme}\pwindex{David, Jakob Julius 1859-02-06 – 1906-11-20@\textsc{David, Jakob Julius} (1859-02-06 – 1906-11-20), \emph{Schriftsteller, Journalist}!Probleme1892@\strich\emph{Probleme} {[}1892{]}|pwk} von Jakob Julius David\pwindex{David, Jakob Julius 1859-02-06 – 1906-11-20@\textsc{David, Jakob Julius} (1859-02-06 – 1906-11-20), \emph{Schriftsteller, Journalist}|pwk} (»Seine Probleme und Charaktere sind einfach,
                     seine Sprache ist knapp und alterthümelnd.«).}}}\label{K_L02704-4h} ſagt. Er hat zwar
               in der Sache meiner Anſicht nach Unrecht, aber als Offenheit iſt es werthzuſchätzen.
               Alle übrigen verſtehen Dich nicht, außer etwa \textsc{\label{K_L02704-5v}\edtext{Ludassy\pwindex{Gans-Ludassy, Julius von 13.04.1858 – 30.09.1922@\textsc{Gans-Ludassy, Julius von} (13.04.1858 – 30.09.1922), \emph{Schriftsteller, Journalist}|pw}}{\lemma{\textnormal{\emph{Ludassy}}}\Cendnote{\textnormal{Julius von Gans-Ludassy\pwindex{Gans-Ludassy, Julius von 13.04.1858 – 30.09.1922@\textsc{Gans-Ludassy, Julius von} (13.04.1858 – 30.09.1922), \emph{Schriftsteller, Journalist}|pwk}: \emph{Bücher}\pwindex{Gans-Ludassy, Julius von 13.04.1858 – 30.09.1922@\textsc{Gans-Ludassy, Julius von} (13.04.1858 – 30.09.1922), \emph{Schriftsteller, Journalist}!Buecher1892-12-19@\strich\emph{Bücher} {[}1892-12-19{]}|pwk}. In: \emph{Fremden-Blatt}\pwindex{Fremden-Blatt1.7.1847 – 22.3.1919@\emph{Fremden-Blatt}|pwk}, Jg. 46, Nr. 351, 19. 12. 1892, S. XXXX.}}}\label{K_L02704-5h}}. \label{K_L02704-6v}\edtext{\textsc{Bauer\pwindex{Bauer, Julius 15.10.1853 – 11.06.1941@\textsc{Bauer, Julius} (15.10.1853 – 11.06.1941), \emph{Schriftsteller, Journalist, Kritiker}|pw}}}{\lemma{\textnormal{\emph{Bauer}}}\Cendnote{\textnormal{[O. V. = Julius Bauer\pwindex{Bauer, Julius 15.10.1853 – 11.06.1941@\textsc{Bauer, Julius} (15.10.1853 – 11.06.1941), \emph{Schriftsteller, Journalist, Kritiker}|pwk}?]: \emph{XXXX}\pwindex{Vom Lesetische. Oesterreichische Autoren1892-12-19@\emph{Vom Lesetische. Österreichische Autoren} {[}1892-12-19{]}|pwk}. In: \emph{Illustriertes Wiener Extrablatt}\pwindex{Illustriertes Wiener Extrablatt1872 – 1928@\emph{Illustriertes Wiener Extrablatt}|pwk}, Jg. 20, Nr. YYYY,
                        3. 12. 1892, S. YYYY.}}}\label{K_L02704-6h}: eine
               lobende Notiz\pwindex{Vom Lesetische. Oesterreichische Autoren1892-12-19@\emph{Vom Lesetische. Österreichische Autoren} {[}1892-12-19{]}|pwv} mit Rückſicht
               darauf, daß man in dem Hauſe dinirt und ſich die Beziehung zu Deinem \label{K_L02704-7v}\edtext{Papa-Regierungsrath\pwindex{Schnitzler, Johann 10.04.1835 – 02.05.1893@\textsc{Schnitzler, Johann} (10.04.1835 – 02.05.1893), \emph{Laryngologe}|pwv}}{\lemma{\textnormal{\emph{Papa-Regierungsrath}}}\Cendnote{\textnormal{Die Rezension\pwindex{Vom Lesetische. Oesterreichische Autoren1892-12-19@\emph{Vom Lesetische. Österreichische Autoren} {[}1892-12-19{]}|pwkv} ist knapp: »Anatol ist ein sentimentaler
                     Roué, der täglich bereits zum Frühstück ein oder zwei Balett-Tänzerinnen oder
                     Circusreiterinnen consumirt, bei diesen Letzteren aber in Hinblick auf seine
                     Unwiderstehlichkeit dauernde Gefühle voraussetzt. Die Persiflage ist stellweise
                     wirklich köstlich durchgeführt. Lesern, die gern über gute Einfälle lachen und
                     hinterdrein ebenso gerne über die Tendenz schimpfen, wird das Büchlein eine
                     willkommene Gabe sein.« Die Zuschreibung an Julius Bauer\pwindex{Bauer, Julius 15.10.1853 – 11.06.1941@\textsc{Bauer, Julius} (15.10.1853 – 11.06.1941), \emph{Schriftsteller, Journalist, Kritiker}|pwk} stützt Schnitzler\pwindex{Schnitzler, Arthur 15.05.1862 – 21.10.1931@\textsc{Schnitzler, Arthur} (15.05.1862 – 21.10.1931), \emph{Schriftsteller, Mediziner}|pwk}s \emph{Tagebuch}\pwindex{Schnitzler, Arthur 15.05.1862 – 21.10.1931@\textsc{Schnitzler, Arthur} (15.05.1862 – 21.10.1931), \emph{Schriftsteller, Mediziner}!Tagebuch1981 – 2000@\strich\emph{Tagebuch} {[}1981 – 2000{]}|pwk}, das am 19. 12. 1892
                  vier Rezensenten und vier Publikationsorgane nennt. Die Reihung der beiden Listen
                  dürfte übereinstimmen, zumindest trifft es für die beiden nachweisbaren
                  Rezensionen auf den Plätzen 2 und 3 zu.}}}\label{K_L02704-7h} erhalten will. \label{K_L02704-88v}\edtext{\textsc{Nossig\pwindex{Nossig, Alfred 18.04.1863 – 22.02.1943@\textsc{Nossig, Alfred} (18.04.1863 – 22.02.1943), \emph{Journalist}|pw}}}{\lemma{\textnormal{\emph{Nossig}}}\Cendnote{\textnormal{[O. V. = Alfred Nossig\pwindex{Nossig, Alfred 18.04.1863 – 22.02.1943@\textsc{Nossig, Alfred} (18.04.1863 – 22.02.1943), \emph{Journalist}|pwk}? oder Clemens Sokal\pwindex{Sokal, Clemens *~21.01.1867@\textsc{Sokal, Clemens} (*~21.01.1867), \emph{Journalist, Rechtsanwalt}|pwk}?]: \emph{Vom Lesetische. Österreichische Literatur}\pwindex{Vom Lesetische. Oesterreichische Autoren1892-12-19@\emph{Vom Lesetische. Österreichische Autoren} {[}1892-12-19{]}|pwk}. In: \emph{Neues Wiener Abendblatt}\pwindex{Neues Wiener Abendblatt@\emph{Neues Wiener Abendblatt}|pwk}, Jg. 26, Nr. 351,
                        19. 12. 1892, S. 3–4, hier: S. 3.
                  Darin wird Arthur Schnitzler\pwindex{Schnitzler, Arthur 15.05.1862 – 21.10.1931@\textsc{Schnitzler, Arthur} (15.05.1862 – 21.10.1931), \emph{Schriftsteller, Mediziner}|pwk} als
                     »Sohn des bekannten Professors\pwindex{Schnitzler, Johann 10.04.1835 – 02.05.1893@\textsc{Schnitzler, Johann} (10.04.1835 – 02.05.1893), \emph{Laryngologe}|pwv} Dr. Schnitzler\pwindex{Schnitzler, Johann 10.04.1835 – 02.05.1893@\textsc{Schnitzler, Johann} (10.04.1835 – 02.05.1893), \emph{Laryngologe}|pwv}« eingeführt. Im erwähnten \emph{Tagebuch}\pwindex{Schnitzler, Arthur 15.05.1862 – 21.10.1931@\textsc{Schnitzler, Arthur} (15.05.1862 – 21.10.1931), \emph{Schriftsteller, Mediziner}!Tagebuch1981 – 2000@\strich\emph{Tagebuch} {[}1981 – 2000{]}|pwk}-Eintrag hat die Rezension Clemens
                     Sokal\pwindex{Sokal, Clemens *~21.01.1867@\textsc{Sokal, Clemens} (*~21.01.1867), \emph{Journalist, Rechtsanwalt}|pwk} geschrieben, hingegen geht Goldmann\pwindex{Goldmann, Paul 31.01.1865 – 25.09.1935@\textsc{Goldmann, Paul} (31.01.1865 – 25.09.1935), \emph{Schriftsteller, Journalist}|pwk} von Alfred Nossig\pwindex{Nossig, Alfred 18.04.1863 – 22.02.1943@\textsc{Nossig, Alfred} (18.04.1863 – 22.02.1943), \emph{Journalist}|pwk}
                  aus.}}}\label{K_L02704-88h}: {\pb}einer, der auf Beides – die \strikeout{Dine} Diners und die Beziehung – candidirt. Macht aber
               nichts; ſie ſollen nur von dir ſprechen. Der Ruf wird ja nicht dadurch zunächſt
               gemacht, daß man verſtanden, ſondern dadurch, daß überſaupt von Einem geſprochen
               wird. Ich ſelbſt hätte längſt über Dich ſchreiben ſollen. Aber wann? Pure phyſiſche
               Unmöglichkeit, das ich Dich doch nicht damit {\pb}beſchimpfen will, daß ich eine Reklamenotiz für Dich zuſammenſchmiere. Die Sache
               mußte künſtleriſch verarbeitet werden. Aber ich habe nicht eine Stunde dafür gehabt.
               Soll alſo inzwiſchen der \label{K_L02704-12v}\edtext{Andere\pwindex{Eisner, Kurt 14.05.1867 – 21.02.1919@\textsc{Eisner, Kurt} (14.05.1867 – 21.02.1919), \emph{Schriftsteller, Politiker}|pwuv}\pwindex{Stein, August 1851-06-02 – 1920-10-12@\textsc{Stein, August} (1851-06-02 – 1920-10-12), \emph{Schriftsteller, Journalist}|pwuv}}{\lemma{\textnormal{\emph{Andere}}}\Cendnote{\textnormal{Eventuell ist August Stein\pwindex{Stein, August 1851-06-02 – 1920-10-12@\textsc{Stein, August} (1851-06-02 – 1920-10-12), \emph{Schriftsteller, Journalist}|pwk} oder Kurt Eisner\pwindex{Eisner, Kurt 14.05.1867 – 21.02.1919@\textsc{Eisner, Kurt} (14.05.1867 – 21.02.1919), \emph{Schriftsteller, Politiker}|pwk} gemeint?}}}\label{K_L02704-12h} ſchreiben – der Berlin\oindex{Berlin@\textbf{Berlin}|pw}er – ein
               ganz braver Menſch, \strikeout{b\textcolor{gray}{o}} bornirt, aber nach der guten Richtung bornirt, d.h. mit einem dummen
               Vorurtheil für das Moderne be{\pb}haftet, was Dir
               zuſtatten kommen wird. Er wird wohl bald \label{K_L02704-13v}\edtext{losſchießen}{\lemma{\textnormal{\emph{losſchießen}}}\Cendnote{\textnormal{Nicht nachgewiesen. Da Goldmann\pwindex{Goldmann, Paul 31.01.1865 – 25.09.1935@\textsc{Goldmann, Paul} (31.01.1865 – 25.09.1935), \emph{Schriftsteller, Journalist}|pwk} eine
                  solche Anfang 1894 (vgl. Paul Goldmann an Arthur Schnitzler, 8. 1. [1894]) ankündigt, ist eine zeitnahe Besprechung unwahrscheinlich.}}}\label{K_L02704-13h}. Und dann
               kann ich ja immer noch das Wort nehmen, wie es mein ſehnlicher Wunſch und feſter
               Vorſatz iſt. \textsc{Herzl\pwindex{Herzl, Theodor 1860-05-02 – 1904-07-03@\textsc{Herzl, Theodor} (1860-05-02 – 1904-07-03), \emph{Schriftsteller, Journalist}|pw}} aber wird nicht ſchreiben. Ich habe mein Möglichſtes gethan – ich bin ſoweit
               gegangen, als ich gehen konnte, – aber, ein ſo braver Menſch\pwindex{Herzl, Theodor 1860-05-02 – 1904-07-03@\textsc{Herzl, Theodor} (1860-05-02 – 1904-07-03), \emph{Schriftsteller, Journalist}|pwv} er iſt, ſo kennſt Du doch auch
               ſeinen {\pb}Größenwahn. Und er hat mir auf meine
               Andeutungen in einer Weiſe geantwortet, daß ich nicht mehr darauf zurückkommen
               konnte, ohne Dich bloszuſtellen. (»Wenn er mir ſein Buch\pwindex{Schnitzler, Arthur 15.05.1862 – 21.10.1931@\textsc{Schnitzler, Arthur} (15.05.1862 – 21.10.1931), \emph{Schriftsteller, Mediziner}!Anatol1892-10-29@\strich\emph{Anatol} {[}1892-10-29{]}|pwv} deshalb geſchickt hat, damit ich darüber ſchreibe \textsc{etc}« {\dotsfour})\pend
           \pstart
           Und nun Dein Stück\pwindex{Schnitzler, Arthur 15.05.1862 – 21.10.1931@\textsc{Schnitzler, Arthur} (15.05.1862 – 21.10.1931), \emph{Schriftsteller, Mediziner}!Maerchen. Schauspiel in drei Aufzuegen1893-12-01@\strich\emph{Das Märchen. Schauspiel in drei Aufzügen} {[}1893-12-01{]}|pwv}? Auf wann
               die \label{K_L02704-9v}\edtext{Aufführung}{\lemma{\textnormal{\emph{Aufführung}}}\Cendnote{\textnormal{Erst ein knappes Jahr später, am 1. 12. 1893, kam es zur Uraufführung des \emph{Märchen}\pwindex{Schnitzler, Arthur 15.05.1862 – 21.10.1931@\textsc{Schnitzler, Arthur} (15.05.1862 – 21.10.1931), \emph{Schriftsteller, Mediziner}!Maerchen. Schauspiel in drei Aufzuegen1893-12-01@\strich\emph{Das Märchen. Schauspiel in drei Aufzügen} {[}1893-12-01{]}|pwk}s am \emph{Deutschen
                     Volkstheater}\orgindex{Volkstheater@Volkstheater|pwk} in Wien\oindex{Wien@\textbf{Wien}|pwk}. Zuvor lehnte das
                     \emph{Burgtheater}\orgindex{Burgtheater@Burgtheater|pwk} das \emph{Märchen}\pwindex{Schnitzler, Arthur 15.05.1862 – 21.10.1931@\textsc{Schnitzler, Arthur} (15.05.1862 – 21.10.1931), \emph{Schriftsteller, Mediziner}!Maerchen. Schauspiel in drei Aufzuegen1893-12-01@\strich\emph{Das Märchen. Schauspiel in drei Aufzügen} {[}1893-12-01{]}|pwk} ab, wie Schnitzler am 19. 11. 1892 im \emph{Tagebuch}\pwindex{Schnitzler, Arthur 15.05.1862 – 21.10.1931@\textsc{Schnitzler, Arthur} (15.05.1862 – 21.10.1931), \emph{Schriftsteller, Mediziner}!Tagebuch1981 – 2000@\strich\emph{Tagebuch} {[}1981 – 2000{]}|pwk} notierte. Außerdem war eine Aufführung in der zweiten Hälfte des
                     Januars 1893 am \emph{Neuen Deutschen Theater}\orgindex{Neues Deutsches Theater@Neues Deutsches Theater|pwk} in Prag\oindex{Prag@\textbf{Prag}|pwk}
                  geplant, die jedoch ebenso nicht stattfand Siehe Paul Goldmann an Arthur Schnitzler, 27. 6. [1892] wie Bemühungen um eine Aufführung am Berlin\oindex{Berlin@\textbf{Berlin}|pwk}er \emph{Lessing-Theater}\orgindex{Lessing-Theater@Lessing-Theater|pwk}
                  gelingen wollten Siehe A. S.: \emph{Tagebuch}, 18. 3. 1893.}}}\label{K_L02704-9h}? Und das neue \label{K_L02704-77v}\edtext{Stück\pwindex{Schnitzler, Arthur 15.05.1862 – 21.10.1931@\textsc{Schnitzler, Arthur} (15.05.1862 – 21.10.1931), \emph{Schriftsteller, Mediziner}!Liebelei. Schauspiel in drei Akten1895-10-09@\strich\emph{Liebelei. Schauspiel in drei Akten} {[}1895-10-09{]}|pwv}}{\lemma{\textnormal{\emph{Stück}}}\Cendnote{\textnormal{vermutlich \emph{Liebelei}\pwindex{Schnitzler, Arthur 15.05.1862 – 21.10.1931@\textsc{Schnitzler, Arthur} (15.05.1862 – 21.10.1931), \emph{Schriftsteller, Mediziner}!Liebelei. Schauspiel in drei Akten1895-10-09@\strich\emph{Liebelei. Schauspiel in drei Akten} {[}1895-10-09{]}|pwk},
                  das aber erst im Herbst 1893 in die Schreibphase trat}}}\label{K_L02704-77h}? Und
               Deine Novellen? Und, ſag’ mir nur, warum {\pb}biſt Du
               ein ſo elender Menſch und ich \label{K_L02704-56v}\edtext{ſchreibſt mir nichts Perſönliches mehr}{\lemma{\textnormal{\emph{XXXX Lemmafehler}}}\Cendnote{\textnormal{Eine mögliche Antwort findet sich in Schnitzler\pwindex{Schnitzler, Arthur 15.05.1862 – 21.10.1931@\textsc{Schnitzler, Arthur} (15.05.1862 – 21.10.1931), \emph{Schriftsteller, Mediziner}|pwk}s \emph{Tagebuch}\pwindex{Schnitzler, Arthur 15.05.1862 – 21.10.1931@\textsc{Schnitzler, Arthur} (15.05.1862 – 21.10.1931), \emph{Schriftsteller, Mediziner}!Tagebuch1981 – 2000@\strich\emph{Tagebuch} {[}1981 – 2000{]}|pwk} vom 15. 9. 1892: »Paul Goldmann\pwindex{Goldmann, Paul 31.01.1865 – 25.09.1935@\textsc{Goldmann, Paul} (31.01.1865 – 25.09.1935), \emph{Schriftsteller, Journalist}|pw} zu weit – in Briefen theil’
                     ich mich nicht gern mit.«}}}\label{K_L02704-55h}? Weißt Du, daß Du mich glücklich aus
               Deinem Leben herausgeworfen haſt? Und daß Du mich auf literariſche Diät geſetzt haſt?
               Literariſcher Beirath! Aber Arthur! Pfui Teufel! Schämſt Du Dich denn gar nicht? {\dots}\pend
           \pstart
           Ich habe Jemanden\pwindex{Kanner, Heinrich 09.11.1864 – 15.02.1930@\textsc{Kanner, Heinrich} (09.11.1864 – 15.02.1930), \emph{Publizist}|pwv} für Euren
               lieben Kreis. {\pb}Das ſympathiſcheſte Mitglied\pwindex{Kanner, Heinrich 09.11.1864 – 15.02.1930@\textsc{Kanner, Heinrich} (09.11.1864 – 15.02.1930), \emph{Publizist}|pwv} hat ſich aus unſerer Redaktion\orgindex{Frankfurter Zeitung@Frankfurter Zeitung|pwv} losgelöſt, weil es von
                  \textsc{Sonnemann\pwindex{Sonnemann, Leopold 1831-10-29 – 1909-10-30@\textsc{Sonnemann, Leopold} (1831-10-29 – 1909-10-30), \emph{Journalist, Herausgeber}|pw}} denn doch gar zu ſehr chicanirt wurde, und iſt – Wien\oindex{Wien@\textbf{Wien}|pw}er von \label{K_L02704-2v}\edtext{Geburt}{\lemma{\textnormal{\emph{Geburt}}}\Cendnote{\textnormal{Heinrich Kanner\pwindex{Kanner, Heinrich 09.11.1864 – 15.02.1930@\textsc{Kanner, Heinrich} (09.11.1864 – 15.02.1930), \emph{Publizist}|pwk} wurde in Galatz\oindex{Galatz@\textbf{Galatz}|pwk} (Rumänien\oindex{Rumaenien@\textbf{Rumänien}|pwk})
                  geboren, zog aber als Kleinkind im Jahr 1866 mit seiner Familie nach
                     Wien\oindex{Wien@\textbf{Wien}|pwkv}.}}}\label{K_L02704-2h} und Erziehung
               – unſer Wien\oindex{Wien@\textbf{Wien}|pw}er Correſpondent\pwindex{Kanner, Heinrich 09.11.1864 – 15.02.1930@\textsc{Kanner, Heinrich} (09.11.1864 – 15.02.1930), \emph{Publizist}|pwv} geworden. \textsc{Dr. Heinrich Kanner\pwindex{Kanner, Heinrich 09.11.1864 – 15.02.1930@\textsc{Kanner, Heinrich} (09.11.1864 – 15.02.1930), \emph{Publizist}|pw}} – Adreſſe wird Dir Dr. \textsc{Joachim\pwindex{Joachim, Jaques 24.11.1866 – 07.11.1925@\textsc{Joachim, Jaques} (24.11.1866 – 07.11.1925), \emph{Rechtswissenschaftler, Rechtsanwalt, Herausgeber}|pw}} ſagen, oder ich ſchreib’ ſie Dir auf – einer der liebſten Leute, die mir
               überhaupt be{\pb}gegnet ſind. Kein Künſtler ſondern
               Volkswirth und Politiker. Aber doch vielleicht Künſtlernatur, vor Allem aber ein
               wahres Ideal an
               Geſcheitheit, Feinſinn und \textsc{Noblesse}. Geh’, ſetz’ Dich mit
               ihm in \label{K_L02704-10v}\edtext{Verbindung}{\lemma{\textnormal{\emph{Verbindung}}}\Cendnote{\textnormal{Es sind keine Briefe zwischen Schnitzler
                  und Heinrich Kanner\pwindex{Kanner, Heinrich 09.11.1864 – 15.02.1930@\textsc{Kanner, Heinrich} (09.11.1864 – 15.02.1930), \emph{Publizist}|pwk}, der außerdem erst am
                     24. 9. 1896 im \emph{Tagebuch}\pwindex{Schnitzler, Arthur 15.05.1862 – 21.10.1931@\textsc{Schnitzler, Arthur} (15.05.1862 – 21.10.1931), \emph{Schriftsteller, Mediziner}!Tagebuch1981 – 2000@\strich\emph{Tagebuch} {[}1981 – 2000{]}|pwk} erwähnt wurde, bekannt.}}}\label{K_L02704-10h}. Wirſt
               Deine Freude daran haben{\dotsfive}\pend
           \pstart
           Von ganzem Herzen ein frohes neues Jahr, mein theurer Freund! {\pb}Arbeitsluſt! Erfolg! Und vorwärts! Die allerwärmſten
               Grüße an \textsc{Loris\pwindex{Hofmannsthal, Hugo von 01.02.1874 – 15.07.1929@\textsc{Hofmannsthal, Hugo von} (01.02.1874 – 15.07.1929), \emph{Schriftsteller}|pw}} und \textsc{Richard\pwindex{Beer-Hofmann, Richard 1866-07-11 – 1945-09-26@\textsc{Beer-Hofmann, Richard} (1866-07-11 – 1945-09-26), \emph{Schriftsteller}|pw}} (\textsc{Richard\pwindex{Beer-Hofmann, Richard 1866-07-11 – 1945-09-26@\textsc{Beer-Hofmann, Richard} (1866-07-11 – 1945-09-26), \emph{Schriftsteller}|pw}} ſoll mir ſchreiben!!!). Ergebene Empfehlungen und Neujahrswünſche an Deine Eltern\pwindex{Schnitzler, Louise 1840-07-08 – 1911-09-09@\textsc{Schnitzler, Louise} (1840-07-08 – 1911-09-09)|pwv}\pwindex{Schnitzler, Johann 10.04.1835 – 02.05.1893@\textsc{Schnitzler, Johann} (10.04.1835 – 02.05.1893), \emph{Laryngologe}|pwv}. Grüße an
               Deinen Bruder\pwindex{Schnitzler, Julius 13.07.1865 – 29.06.1939@\textsc{Schnitzler, Julius} (13.07.1865 – 29.06.1939), \emph{Chirurg}|pwv}, \textsc{Kapper\pwindex{Kapper, Friedrich 21.04.1861 – 22.07.1939@\textsc{Kapper, Friedrich} (21.04.1861 – 22.07.1939), \emph{Mediziner}|pw}} und wen ich ſonſt noch in Wien\oindex{Wien@\textbf{Wien}|pw} lieb habe,
               was Du ja ebenſo wohl weißt wie ich.\pend
           \pstart
           Und ich umarme Dich von ganzem Herzen, {\pb}in alter,
               unwandelbarer, treuer Freundſchaft.\pend
           \pstart
           Dein {\\[\baselineskip]}\spacefill\mbox{Paul Goldm}\pend
           \leftskip=0em{}\pstart
           \noindent{}Der kleinen Elſe\pwindex{Singer, Else 25.06.1878 – 1943?@\textsc{Singer, Else} (25.06.1878 – 1943?), \emph{Schriftstellerin, Sprachlehrerin}|pw}: Handkuß, und ich hab’ die
                  Sachen leider ſelbſt nicht mehr. Liegt auch ſo weit hinter mir. Will mich auch gar
                  nicht mehr daran erinnern, daß ich einmal Künſtler werden wollte und daß es kleine
                     Elſe\pwindex{Singer, Else 25.06.1878 – 1943?@\textsc{Singer, Else} (25.06.1878 – 1943?), \emph{Schriftstellerin, Sprachlehrerin}|pw}n in der {\pb}Welt gibt. Das thut ſo weh!\pend
           \pstart
           Und ſag’ einmal: Könnteſt Du nicht unter der Hand einmal und ganz zufällig
                  erfahren, was \label{K_L02704-15v}\edtext{\textsc{Hilda\pwindex{Mitis, Hilda von 1876-08-30 – 1894-12-14@\textsc{Mitis, Hilda von} (1876-08-30 – 1894-12-14), \emph{Schriftstellerin, Telefonistin}|pw}}}{\lemma{\textnormal{\emph{Hilda}}}\Cendnote{\textnormal{Siehe Paul Goldmann an Arthur Schnitzler, 27. 4. 1891}}}\label{K_L02704-15h} macht? Ich glaube, ich habe mich da doch wie ein Schaf benommen. Dieſes
                  aber unter uns.\pend
           \pstart
           Bald einen Brief, nicht wahr? Theils literariſch, theils perſönlich!\pend
           
         
         \endnumbering\briefempfaengerindex{Schnitzler, Arthur@\textsc{Schnitzler, Arthur}!zzzGoldmann, Paul@\emph{von Paul Goldmann}!1892-12-241@{24. 12. {[}1892{]}}|)be}\mylabel{h}\end{ledgroupsized}  \newcommand{\dateiname}{L02704}\newcommand{\titel}{Paul Goldmann an Arthur Schnitzler, 24. 12. [1892]}\newcommand{\editorInnen}{Martin Anton Müller und Laura Untner}
            \footnotesize
\begin{ledgroupsized}[t]{11.5cm}
\doendnotes{C}
\end{ledgroupsized}
         %% latex-leseansicht-abspann.tex
%% Abspann für die Leseansicht.
%% Der Schalter \ifkorrekturansicht ist bereits durch den Vorspann gesetzt.

%% latex-abspann.tex
%% Gemeinsamer Abspann für Korrekturansicht und Leseansicht.
%% Setzt den Schalter \ifkorrekturansicht voraus (gesetzt in den
%% einbindenden Dateien latex-korrekturansicht-abspann.tex bzw.
%% latex-leseansicht-abspann.tex).
%% ---------------------------------------------------------------

\normalsize

% Das esempio-Environment wird nur in der Leseansicht benötigt
\ifkorrekturansicht\else
\newenvironment{esempio}[3]%
{
    \vspace{1.5ex}
    \rlap{\underline{#1}}
    \par
    \setlength{\parindent}{0cm}
    \nopagebreak
    \leftskip=#2cm
    \rightskip=#3cm
}
{
    \par
}
\fi

\doendnotes{C}
\bigskip
\vfill

\clearpage

\footnotesize

\ifkorrekturansicht
  \lohead{\textsc{register}}
\fi

% theindex-Environment neu definieren ohne reledmac
\makeatletter
\renewenvironment{theindex}{%
  \ifkorrekturansicht
    \section*{\indexname}%
  \else
    \subsubsection*{Index der erwähnten Entitäten}%
  \fi
  \setlength{\parindent}{0pt}%
  \setlength{\parskip}{0pt plus 0.3pt}%
  \let\item\@idxitem
}{%
  \ifkorrekturansicht\clearpage\fi
}
\makeatother

\IfFileExists{\jobname-pw.ind}{\input{\jobname-pw.ind}}{}

% Quellenangabe nur in der Leseansicht
\ifkorrekturansicht\else
% Fallback-Definitionen, falls die .tex-Datei \titel etc. nicht gesetzt hat
\providecommand{\titel}{}
\providecommand{\editorInnen}{}
\providecommand{\dateiname}{\jobname}

\vspace{3cm}

\vfill

\footnotesize
\textsc{Quelle}: \titel. Herausgegeben von {\editorInnen}. In: \emph{Arthur Schnitzler: Briefwechsel mit Autorinnen und Autoren}.
 Digitale Edition, https://schnitzler-briefe.acdh.oeaw.ac.at/{\dateiname}.html (Stand \today)
\fi

\end{document}


      