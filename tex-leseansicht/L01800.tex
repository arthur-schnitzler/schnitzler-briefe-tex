%% latex-leseansicht-vorspann.tex
%% Vorspann für die Leseansicht.
%% Lädt die gemeinsame Datei latex-vorspann.tex mit nicht gesetztem Schalter.

\newif\ifkorrekturansicht
\korrekturansichtfalse

\input{../tex-inputs/latex-vorspann}


\section[Hermann Bahr an Arthur Schnitzler, 10. 11. 1908]{L01800 Hermann Bahr an Arthur Schnitzler, 10. 11. 1908}
\nopagebreak\mylabel{L01800v}
\rehead{ }\normalsize\beginnumbering\briefempfaengerindex{Schnitzler, Arthur@\textsc{Schnitzler, Arthur}!zzzBahr, Hermann@\emph{von Hermann Bahr}!1908-11-101@{10. 11. 1908}|(be}
\toendnotes[C]{\smallbreak\pagebreak[2]}
\correspDesc{Versand  durch Hermann Bahr am 10. 11. 1908 in Lindau am Bodensee
\newline{}Erhalt  durch Arthur Schnitzler im Zeitraum [11. 11. 1908 – 15. 11. 1908?] in Wien}\toendnotes[C]{\smallbreak}
\Standort{CUL, Schnitzler, B 5b.}
\physDesc{Bildpostkarte, 296 Zeichen
\newline{}Handschrift: Bleistift, deutsche Kurrent
\newline{}Versand: Stempel: »\nobreak{}\oindex{Bahnhof@\textbf{Bahnhof}, \emph{Bahnhofsgebäude}|pwk}Lind. K. B. Bahnhof, 10 Nov. 08\nobreak{}«.  
\newline{}Schnitzler: mit Bleistift ergänzt »Bahr« 
\newline{}Ordnung: mit Bleistift von unbekannter Hand nummeriert:
                                    »161« }
\buchAbdrucke{\weitereDrucke{Hermann Bahr, Arthur Schnitzler: \emph{Briefwechsel, Aufzeichnungen, Dokumente (1891–1931)}. Herausgegeben von Kurt Ifkovits und Martin Anton Müller. Göttingen: \emph{Wallstein} 2018, S. 406.} }\toendnotes[C]{\smallbreak}\pstart{}\textsc{{\pb}Artur Schnitzler}\pend{}\pstart{}\textsc{Wien XVIII\oindex{XVIII., Währing@\textbf{XVIII., Währing}, \emph{Verwaltungsgebiet}|pw}}\pend{}\pstart{}\textsc{Spöttelgasse 7\oindex{Wien@\textbf{Wien}!XVIII., Währing@\textbf{XVIII., Währing}!Edmund-Weiß-Gasse 7@\textbf{Edmund-Weiß-Gasse 7}, \emph{Wohngebäude}|pw}}\pend{}{\bigskip}
\pstart
           \noindent{}\centering{}{\pb}\textcolor{gray}{\textbf{Lindau i. B.\oindex{Lindau am Bodensee@\textbf{Lindau am Bodensee}, \emph{Hauptstadt}|pw}}}\pend
           
\pstart
           \centering{}\textcolor{gray}{\textbf{Partie im Hafen mit Bayrischen
                     Hof\oindex{Bayerischer Hof@\textbf{Bayerischer Hof}, \emph{Hotel}|pw} und alten Leuchtturm\oindex{Mangturm@\textbf{Mangturm}, \emph{Gebäude}|pw}}}\pend
           \vspace{1em}
\pstart
           \raggedleft{}{\pb}10. 11.\pend
           \vspace{0.5em}
\pstart
           Ich habe Dich am 5. in Frankfurt\oindex{Frankfurt am Main@\textbf{Frankfurt am Main}, \emph{Hauptstadt}|pw} und
               geſtern \label{K_L01800-1v}\edtext{in Zürich\oindex{Zürich@\textbf{Zürich}|pw}}{\lemma{\textnormal{\emph{in Zürich}}}\Cendnote{\textnormal{Zur Lesung am 9. 11. 1908
                  im \emph{Lesezirkel Hottingen}\orgindex{Lesezirkel Hottingen@Lesezirkel Hottingen|pwk} ist sowohl in Bahrs\pwindex{Bahr, Hermann 19.\,7.\,1863 Linz – 15.\,1.\,1934 München@\textsc{Bahr, Hermann} (19.\,7.\,1863 Linz – 15.\,1.\,1934 München), \emph{Schriftsteller, Kritiker}|pwk} wie auch in Schnitzlers Papieren (University of Exeter,
                        \emph{The Schnitzler Press-Cuttings Archive}, Box
                  1/6) das Programmheft überliefert. Als Ablauf wird angegeben: »1.
                     Über Schnitzler. 2. Schnitzlers Novelle: ›Die
                        Toten schweigen\pwindex{Schnitzler, Arthur 15.\,5.\,1862 Wien – 21.\,10.\,1931 ebd.@\textsc{Schnitzler, Arthur} (15.\,5.\,1862 Wien – 21.\,10.\,1931 ebd.), \emph{Schriftsteller, Mediziner}!Toten schweigen@\strich\emph{Die Toten schweigen}|pw}‹«.}}}\label{K_L01800-1} beſungen, \introOben{}über\introOben{}morgen wirſt Dus auch noch nicht in Mannheim\oindex{Mannheim@\textbf{Mannheim}, \emph{Hauptstadt}|pw}. Verſchaff Dir das letzte Heft des »Morgen\pwindex{Morgen. Wochenschrift für deutsche Kultur@\emph{Morgen. Wochenschrift für deutsche Kultur}|pw}«, wo ich einiges\pwindex{Bahr, Hermann 19.\,7.\,1863 Linz – 15.\,1.\,1934 München@\textsc{Bahr, Hermann} (19.\,7.\,1863 Linz – 15.\,1.\,1934 München), \emph{Schriftsteller, Kritiker}!Tagebuch. 10. Juni [1908]@\strich\emph{Tagebuch. 10. Juni [1908]}|pwv} zum »Weg ins Freie\pwindex{Schnitzler, Arthur 15.\,5.\,1862 Wien – 21.\,10.\,1931 ebd.@\textsc{Schnitzler, Arthur} (15.\,5.\,1862 Wien – 21.\,10.\,1931 ebd.), \emph{Schriftsteller, Mediziner}!Weg ins Freie. Roman@\strich\emph{Der Weg ins Freie. Roman}|pw}« geſagt
               habe.\pend
           
\pstart
           Mit vielen Grüßen an Deine liebe Frau\pwindex{Schnitzler, Olga 17.\,1.\,1882 Wien – 13.\,1.\,1970 Lugano@\textsc{Schnitzler, Olga} (17.\,1.\,1882 Wien – 13.\,1.\,1970 Lugano), \emph{Schauspielerin, Sängerin}|pwv}{\\[\baselineskip]}herzlichſt\hspace*{1.5em}\spacefill\mbox{Hermann}\pend
           \leftskip=0em{}\selectlanguage{ngerman}\endnumbering\briefempfaengerindex{Schnitzler, Arthur@\textsc{Schnitzler, Arthur}!zzzBahr, Hermann@\emph{von Hermann Bahr}!1908-11-101@{10. 11. 1908}|)be}\mylabel{L01800h}  \newcommand{\dateiname}{L01800}\newcommand{\titel}{Hermann Bahr an Arthur Schnitzler, 10. 11. 1908}\newcommand{\editorInnen}{Herausgegeben von Martin Anton Müller}%% latex-leseansicht-abspann.tex
%% Abspann für die Leseansicht.
%% Der Schalter \ifkorrekturansicht ist bereits durch den Vorspann gesetzt.

%% latex-abspann.tex
%% Gemeinsamer Abspann für Korrekturansicht und Leseansicht.
%% Setzt den Schalter \ifkorrekturansicht voraus (gesetzt in den
%% einbindenden Dateien latex-korrekturansicht-abspann.tex bzw.
%% latex-leseansicht-abspann.tex).
%% ---------------------------------------------------------------

\normalsize

% Das esempio-Environment wird nur in der Leseansicht benötigt
\ifkorrekturansicht\else
\newenvironment{esempio}[3]%
{
    \vspace{1.5ex}
    \rlap{\underline{#1}}
    \par
    \setlength{\parindent}{0cm}
    \nopagebreak
    \leftskip=#2cm
    \rightskip=#3cm
}
{
    \par
}
\fi

\doendnotes{C}
\bigskip
\vfill

\clearpage

\footnotesize

\ifkorrekturansicht
  \lohead{\textsc{register}}
\fi

% theindex-Environment neu definieren ohne reledmac
\makeatletter
\renewenvironment{theindex}{%
  \ifkorrekturansicht
    \section*{\indexname}%
  \else
    \subsubsection*{Index der erwähnten Entitäten}%
  \fi
  \setlength{\parindent}{0pt}%
  \setlength{\parskip}{0pt plus 0.3pt}%
  \let\item\@idxitem
}{%
  \ifkorrekturansicht\clearpage\fi
}
\makeatother

\IfFileExists{\jobname-pw.ind}{\input{\jobname-pw.ind}}{}

% Quellenangabe nur in der Leseansicht
\ifkorrekturansicht\else
% Fallback-Definitionen, falls die .tex-Datei \titel etc. nicht gesetzt hat
\providecommand{\titel}{}
\providecommand{\editorInnen}{}
\providecommand{\dateiname}{\jobname}

\vspace{3cm}

\vfill

\footnotesize
\textsc{Quelle}: \titel. Herausgegeben von {\editorInnen}. In: \emph{Arthur Schnitzler: Briefwechsel mit Autorinnen und Autoren}.
 Digitale Edition, https://schnitzler-briefe.acdh.oeaw.ac.at/{\dateiname}.html (Stand \today)
\fi

\end{document}


