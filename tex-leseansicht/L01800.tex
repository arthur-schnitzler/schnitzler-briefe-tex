%% latex-korrekturansicht-vorspann.tex
%% Vorspann für die Korrekturansicht.
%% Lädt die gemeinsame Datei latex-vorspann.tex mit gesetztem Schalter.

\newif\ifkorrekturansicht
\korrekturansichttrue

\input{../tex-inputs/latex-vorspann}


\section[Hermann Bahr an Arthur Schnitzler, 10. 11. 1908]{L01800 Hermann Bahr an Arthur Schnitzler, 10. 11. 1908}
\nopagebreak\mylabel{L01800v}
\rehead{ }\normalsize\beginnumbering\briefempfaengerindex{Schnitzler, Arthur@\textsc{Schnitzler, Arthur}!zzzBahr, Hermann@\emph{von Hermann Bahr}!1908-11-101@{10. 11. 1908}|(be}
\toendnotes[C]{\smallbreak\pagebreak[2]}\Standort{CUL, Schnitzler, B 5b.}
\physDesc{Bildpostkarte, 296 Zeichen
\newline{}Handschrift: 1) Bleistift, deutsche Kurrent\hspace{1em}2) Bleistift, lateinische Kurrent (\noindent{}Adresse)\hspace{1em}
\newline{}Versand: Stempel: »\nobreak{}\oindex{Bahnhof@\textbf{Bahnhof}, \emph{Bahnhofsgebäude (K.BHF)}|pwk}Lind. K. B. Bahnhof, 10 Nov. 08\nobreak{}«.  
\newline{}Schnitzler: mit Bleistift ergänzt »Bahr« 
\newline{}Ordnung: mit Bleistift von unbekannter Hand nummeriert:
                                    »161« }
\buchAbdrucke{\weitereDrucke{Hermann Bahr, Arthur Schnitzler: \emph{Briefwechsel, Aufzeichnungen, Dokumente (1891–1931)}. Göttingen: \emph{Wallstein} 2018, S. 406.} }\toendnotes[C]{\smallbreak}\pstart{}{\pb}Artur Schnitzler\pend{}\pstart{}Wien XVIII\oindex{XVIII., Waehring@\textbf{XVIII., Währing}, \emph{A.ADM3}|pw}\pend{}\pstart{}Spöttelgasse 7\oindex{Edmund-Weiss-Gasse 7@\textbf{Edmund-Weiß-Gasse 7}, \emph{Wohngebäude (K.WHS)}|pw}\pend{}{\bigskip}
\pstart
           \noindent{}\centering{}{\pb}\textcolor{gray}{\textbf{Lindau i. B.\oindex{Lindau am Bodensee@\textbf{Lindau am Bodensee}, \emph{P.PPLA3}|pw}}}\pend
           
\pstart
           \centering{}\textcolor{gray}{\textbf{Partie im Hafen mit Bayrischen
                     Hof\oindex{Bayerischer Hof@\textbf{Bayerischer Hof}, \emph{Hotel (K.HTL)}|pw} und alten Leuchtturm\oindex{Mangturm@\textbf{Mangturm}, \emph{Gebäude (K.GBD)}|pw}}}\pend
           \vspace{1em}
\pstart
           \raggedleft{}{\pb}10. 11.\pend
           \vspace{0.5em}
\pstart
           Ich habe Dich am 5. in Frankfurt\oindex{Frankfurt am Main@\textbf{Frankfurt am Main}, \emph{P.PPLA3}|pw} und
               geſtern \label{K_L01800-1v}\edtext{in Zürich\oindex{Zuerich@\textbf{Zürich}, \emph{P.PPLA}|pw}}{\lemma{\textnormal{\emph{in Zürich}}}\Cendnote{\textnormal{Zur Lesung am 9. 11. 1908
                  im \emph{Lesezirkel Hottingen}\orgindex{Lesezirkel Hottingen@Lesezirkel Hottingen|pwk} ist sowohl in Bahrs\pwindex{Bahr, Hermann 19.07.1863 – 15.01.1934@\textsc{Bahr, Hermann} (19.07.1863 – 15.01.1934), \emph{Schriftsteller/Schriftstellerin, Kritiker/Kritikerin}|pwk} wie auch in Schnitzlers Papieren (University of Exeter,
                        \emph{The Schnitzler Press-Cuttings Archive}, Box
                  1/6) das Programmheft überliefert. Als Ablauf wird angegeben: »1.
                     Über Schnitzler. 2. Schnitzlers Novelle: ›Die
                        Toten schweigen\pwindex{Toten schweigen@\emph{Die Toten schweigen}|pw}‹«.}}}\label{K_L01800-1} beſungen, \introOben{}über\introOben{}morgen wirſt Dus auch noch nicht in Mannheim\oindex{Mannheim@\textbf{Mannheim}, \emph{P.PPLA3}|pw}. Verſchaff Dir das letzte Heft des »Morgen\pwindex{Morgen. Wochenschrift fuer deutsche Kultur@\emph{Morgen. Wochenschrift für deutsche Kultur}|pw}«, wo ich einiges\pwindex{Tagebuch. 10. Juni [1908]@\emph{Tagebuch. 10. Juni [1908]}|pwv} zum »Weg ins Freie\pwindex{Weg ins Freie. Roman@\emph{Der Weg ins Freie. Roman}|pw}« geſagt
               habe.\pend
           
\pstart
           Mit vielen Grüßen an Deine liebe Frau\pwindex{Schnitzler, Olga 17.01.1882 – 13.01.1970@\textsc{Schnitzler, Olga} (17.01.1882 – 13.01.1970), \emph{Schauspieler/Schauspielerin, Sänger/Sängerin}|pwv}{\\[\baselineskip]}herzlichſt\hspace*{1.5em}\spacefill\mbox{Hermann}\pend
           \leftskip=0em{}\selectlanguage{ngerman}\endnumbering\briefempfaengerindex{Schnitzler, Arthur@\textsc{Schnitzler, Arthur}!zzzBahr, Hermann@\emph{von Hermann Bahr}!1908-11-101@{10. 11. 1908}|)be}\mylabel{L01800h}  \normalsize

\doendnotes{C}
\bigskip
\vfill

\clearpage

\footnotesize

\lohead{\textsc{register}}

% Definiere theindex-Environment komplett neu ohne reledmac
\makeatletter
\renewenvironment{theindex}{%
  \section*{\indexname}%
  \setlength{\parindent}{0pt}%
  \setlength{\parskip}{0pt plus 0.3pt}%
  \let\item\@idxitem
}{%
  \clearpage
}
\makeatother

\IfFileExists{\jobname-pw.ind}{\input{\jobname-pw.ind}}{}

\end{document}

      