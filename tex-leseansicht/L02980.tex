%% latex-korrekturansicht-vorspann.tex
%% Vorspann für die Korrekturansicht.
%% Lädt die gemeinsame Datei latex-vorspann.tex mit gesetztem Schalter.

\newif\ifkorrekturansicht
\korrekturansichttrue

\input{../tex-inputs/latex-vorspann}


\section[ Arthur Schnitzler an Felix Salten, 4. 3. 1903]{L02980 Arthur Schnitzler an Felix Salten, 4. 3. 1903}
\nopagebreak\mylabel{L02980v}
\rehead{ }\normalsize\beginnumbering\briefempfaengerindex{Salten, Felix@\textsc{Salten, Felix}!zzzSchnitzler, Arthur@\emph{von Arthur Schnitzler}!1903-03-041@{4. 3. 1903}|(be}
\toendnotes[C]{\smallbreak\pagebreak[2]}\Standort{Wienbibliothek im Rathaus, ZPH 1681, 2.1.516.}
\physDesc{Brief, 1 Blatt, 4 Seiten, 1351 Zeichen
\newline{}Handschrift: Bleistift, deutsche Kurrent
\newline{}Ordnung: mit Bleistift von unbekannter Hand Nummerierung der Doppelseiten des Konvoluts:
                                    »55«–»56« }\toendnotes[C]{\smallbreak}
\pstart
           \raggedleft{}{\pb}4. \textcolor{gray}{3}. 903.\pend
           \vspace{0.5em}
\pstart
           lieber Freund, mit \label{K_L02980-1v}\edtext{\textsc{M. H.\pwindex{Horwitz, Mirjam 1882-06-15 – 1967-09-26@\textsc{Horwitz, Mirjam} (1882-06-15 – 1967-09-26), \emph{Theaterleiter/Theaterleiterin, Schauspieler/Schauspielerin}|pw}}}{\lemma{\textnormal{\emph{M. H.}}}\Cendnote{\textnormal{Siehe Felix Salten an Arthur Schnitzler, 3. 3. 1903.
               }}}\label{K_L02980-1} konnte ich bisher kaum hundert Worte unauffällg ſprechen; der Brief, den Sie
               erhalten, iſt natürlich die Reaction auf meine Mittheilg; – in dieſen Tagen habe ich
               jedenfalls weiter Gelegenheit ſie zu \label{K_L02980-2v}\edtext{ſehen (vielleicht heute)}{\lemma{\textnormal{\emph{ſehen (vielleicht heute)}}}\Cendnote{\textnormal{Siehe A. S.: \emph{Tagebuch}, 4. 3. 1903.
               }}}\label{K_L02980-2} und bringe das gewünſchte ſchonend bei. Ich habe nicht den Eindruck, daſs
               Gefahren drohen. Nicht »Verlogenheit«, aber naive Unechtheit ſozuſagen. Glauben Sie
               nicht? –\pend
           
\pstart
           {\pb}– Die \label{K_L02980-3v}\edtext{Proben\pwindex{Schleier der Beatrice. Schauspiel in fuenf Akten@\emph{Der Schleier der Beatrice. Schauspiel in fünf Akten}|pwv} haben mir keine
               beſondre Freude gemacht}{\lemma{\textnormal{\emph{Proben … gemacht}}}\Cendnote{\textnormal{Siehe A. S.: \emph{Tagebuch}, 23. 2. 1903, 24. 2. 1903 und 26. 2. 1903.
               }}}\label{K_L02980-3}; i{\geminationm}erhin ko{\geminationm}t
               einiges beſſer heraus als ich dachte. Mit Leſſing\pwindex{Lessing, Emil 06.05.1857 – 01.11.1921@\textsc{Lessing, Emil} (06.05.1857 – 01.11.1921), \emph{Regisseur/Regisseurin, Schauspieler/Schauspielerin}|pw} vertrag ich mich ſchlecht. Brahm\pwindex{Brahm, Otto 05.02.1856 – 28.11.1912@\textsc{Brahm, Otto} (05.02.1856 – 28.11.1912), \emph{Theaterleiter/Theaterleiterin, Regisseur/Regisseurin}|pw} iſt klug und quälend wie i{\geminationm}er. Paul G.\pwindex{Goldmann, Paul 31.01.1865 – 25.09.1935@\textsc{Goldmann, Paul} (31.01.1865 – 25.09.1935), \emph{Schriftsteller/Schriftstellerin, Journalist/Journalistin}|pw} geht als »\label{K_L02980-4v}\edtext{verbloedeter Thor}{\lemma{\textnormal{\emph{verbloedeter Thor}}}\Cendnote{\textnormal{Siehe A. S.: \emph{Tagebuch}, 22. 2. 1903.
               }}}\label{K_L02980-4}« herum. (So ne{\geminationn}t er ſich ſelbſt, in Anſchluſs an
               eine \introOben{}unglückliche\introOben{}{ }Liebesgeſchichte\pwindex{Rottenberg, Theodore 1875-09-07 – 1945-04-05@\textsc{Rottenberg, Theodore} (1875-09-07 – 1945-04-05)|pwv}, die er in
               ganz Berlin\oindex{Berlin@\textbf{Berlin}, \emph{P.PPLC}|pw} ſelber erzählt hat.) – \label{K_L02980-5v}\edtext{Heute{ }Abend ko{\geminationm}t Olga\pwindex{Schnitzler, Olga 17.01.1882 – 13.01.1970@\textsc{Schnitzler, Olga} (17.01.1882 – 13.01.1970), \emph{Schauspieler/Schauspielerin, Sänger/Sängerin}|pw} an}{\lemma{\textnormal{\emph{Heute … an}}}\Cendnote{\textnormal{Siehe A. S.: \emph{Tagebuch}, 4. 3. 1903.
               }}}\label{K_L02980-5}, { }{\pb}\label{K_L02980-6v}\edtext{Samſtag mein Bruder\pwindex{Schnitzler, Julius 13.07.1865 – 29.06.1939@\textsc{Schnitzler, Julius} (13.07.1865 – 29.06.1939), \emph{Chirurg/Chirurgin}|pwv} (wahrſcheinlich}{\lemma{\textnormal{\emph{Samſtag … (wahrſcheinlich}}}\Cendnote{\textnormal{Er dürfte nicht angereist sein, jedenfalls erwähnt ihn Schnitzler in diesen Tagen nicht im \emph{Tagebuch}\pwindex{Tagebuch@\emph{Tagebuch}|pwk}.}}}\label{K_L02980-6}.) – Ich hoffe \label{K_L02980-7v}\edtext{Dinſtg{ }früh zu Hauſe}{\lemma{\textnormal{\emph{Dinſtg früh zu Hauſe}}}\Cendnote{\textnormal{Siehe A. S.: \emph{Tagebuch}, 10. 3. 1903.
               }}}\label{K_L02980-7} zu ſein und \label{K_L02980-8v}\edtext{ſpreche Sie wohl
                  gleich}{\lemma{\textnormal{\emph{ſpreche Sie wohl
                  gleich}}}\Cendnote{\textnormal{Nachweislich sahen sie sich
                  bereits einen Tag nach Schnitzlers Rückkehr,
                     vgl. A. S.: \emph{Tagebuch}, 11. 3. 1903.}}}\label{K_L02980-8} in
               den erſten Tagen. – Zu dem neuen \label{K_L02980-9v}\edtext{»\begin{otherlanguage}{french}Avancement\end{otherlanguage}«}{\lemma{\textnormal{\emph{»Avancement«}}}\Cendnote{\textnormal{französisch: Beförderung}}}\label{K_L02980-9} gratulir ich herzlich.
                  \label{K_L02980-10v}\edtext{Herr \textsc{Wigand\pwindex{Wigand, Curt 28.03.1865 – 1913@\textsc{Wigand, Curt} (28.03.1865 – 1913), \emph{Schriftsteller/Schriftstellerin, Verleger/Verlegerin}|pw}} war hier}{\lemma{\textnormal{\emph{Herr Wigand war hier}}}\Cendnote{\textnormal{Siehe A. S.: \emph{Tagebuch}, 3. 3. 1903.
               }}}\label{K_L02980-10} bei mir; ſolang ich nur durch \textsc{Lantz\pwindex{Lantz, Adolf 10.11.1882 – 19.08.1949@\textsc{Lantz, Adolf} (10.11.1882 – 19.08.1949), \emph{Schriftsteller/Schriftstellerin, Theaterleiter/Theaterleiterin, Dramaturg/Dramaturgin}|pw}} von den \label{K_L02980-11v}\edtext{adminiſtr Zuſtänden der »Zeit\orgindex{Zeit@Die Zeit|pw}}{\lemma{\textnormal{\emph{adminiſtr … »Zeit}}}\Cendnote{\textnormal{Die
                  Unzufriedenheit an der Führung der Tageszeitung dürfte sich auf die Person von Heinrich Kanner\pwindex{Kanner, Heinrich 09.11.1864 – 15.02.1930@\textsc{Kanner, Heinrich} (09.11.1864 – 15.02.1930), \emph{Herausgeber/Herausgeberin, Publizist/Publizistin}|pwk}
                  konzentriert haben, vgl. Felix Salten an Arthur Schnitzler, 9. 3. 1906.}}}\label{K_L02980-11}«
               erfahren hatte, konnte ich einige für {\pb}unbewußt übertrieben halten, aber nach den Berichten des Hrn W.\pwindex{Wigand, Curt 28.03.1865 – 1913@\textsc{Wigand, Curt} (28.03.1865 – 1913), \emph{Schriftsteller/Schriftstellerin, Verleger/Verlegerin}|pw} find ich das Verhalten des hier in Betracht ko{\geminationm}enden Hinaus\textcolor{gray}{ſ}chmeißer\substVorne{}\textsuperscript{ un\textcolor{gray}{d}}\substDazwischen{},\substHinten{} Gageverkürzer und Proceſsführer einfach ſkandalös. –\pend
           
\pstart
           – Leben Sie wohl, ſeien Sie herzlich gegrüßt, auf Wiederſehn {\\[\baselineskip]}Ich hoffe, Ihre
                  Frau\pwindex{Salten, Ottilie 07.03.1868 – 22.06.1942@\textsc{Salten, Ottilie} (07.03.1868 – 22.06.1942), \emph{Schauspieler/Schauspielerin}|pwv} iſt wohl. {\\[\baselineskip]}Ihr {\\[\baselineskip]}\spacefill\mbox{A.}\pend
           \leftskip=0em{}\selectlanguage{ngerman}\endnumbering\briefempfaengerindex{Salten, Felix@\textsc{Salten, Felix}!zzzSchnitzler, Arthur@\emph{von Arthur Schnitzler}!1903-03-041@{4. 3. 1903}|)be}\mylabel{L02980h}  \normalsize

\doendnotes{C}
\bigskip
\vfill

\clearpage

\footnotesize

\lohead{\textsc{register}}

% Definiere theindex-Environment komplett neu ohne reledmac
\makeatletter
\renewenvironment{theindex}{%
  \section*{\indexname}%
  \setlength{\parindent}{0pt}%
  \setlength{\parskip}{0pt plus 0.3pt}%
  \let\item\@idxitem
}{%
  \clearpage
}
\makeatother

\IfFileExists{\jobname-pw.ind}{\input{\jobname-pw.ind}}{}

\end{document}

      