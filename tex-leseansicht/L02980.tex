%% latex-leseansicht-vorspann.tex
%% Vorspann für die Leseansicht.
%% Lädt die gemeinsame Datei latex-vorspann.tex mit nicht gesetztem Schalter.

\newif\ifkorrekturansicht
\korrekturansichtfalse

\input{../tex-inputs/latex-vorspann}


\section[ Arthur Schnitzler an Felix Salten, 4. 3. 1903]{L02980 Arthur Schnitzler an Felix Salten,  4. 3. 1903}
\nopagebreak\mylabel{L02980v}
\rehead{ }\normalsize\beginnumbering\briefempfaengerindex{Salten, Felix@\textsc{Salten, Felix}!zzzSchnitzler, Arthur@\emph{von Arthur Schnitzler}!1903-03-041@{4. 3. 1903}|(be}
\toendnotes[C]{\smallbreak\pagebreak[2]}
\correspDesc{Versand  durch Arthur Schnitzler am 4. 3. 1903 in Berlin
\newline{}Erhalt  durch Felix Salten im Zeitraum [5. 3. 1903
                  – 9. 3. 1903?] in Wien}\toendnotes[C]{\smallbreak}
\Standort{Wienbibliothek im Rathaus, ZPH 1681, 2.1.516.}
\physDesc{Brief, 1 Blatt, 4 Seiten, 1351 Zeichen
\newline{}Handschrift: Bleistift, deutsche Kurrent
\newline{}Ordnung: mit Bleistift von unbekannter Hand Nummerierung der Doppelseiten des Konvoluts:
                                    »55«–»56« }\toendnotes[C]{\smallbreak}
\pstart
           \raggedleft{}{\pb}4. \textcolor{gray}{3}. 903.\pend
           \vspace{0.5em}
\pstart
           lieber Freund, mit \label{K_L02980-1v}\edtext{\textsc{M. H.\pwindex{Horwitz, Mirjam 15.\,6.\,1882 Berlin – 26.\,9.\,1967 Lütjensee@\textsc{Horwitz, Mirjam} (15.\,6.\,1882 Berlin – 26.\,9.\,1967 Lütjensee), \emph{Theaterleiterin, Schauspielerin}|pw}}}{\lemma{\textnormal{\emph{M. H.}}}\Cendnote{\textnormal{Siehe XXXX Auszeichnungsfehler: Dokument L03339 nicht gefunden.
               }}}\label{K_L02980-1} konnte ich bisher kaum hundert Worte unauffällg{ }ſprechen; der Brief, den Sie
               erhalten, iſt natürlich die Reaction auf meine Mittheilg; – in dieſen Tagen habe ich
               jedenfalls weiter Gelegenheit{ }ſie zu \label{K_L02980-2v}\edtext{ſehen (vielleicht heute)}{\lemma{\textnormal{\emph{sehen (vielleicht heute)}}}\Cendnote{\textnormal{Siehe A. S.: \emph{Tagebuch}, 4. 3. 1903.
               }}}\label{K_L02980-2} und bringe das gewünſchte{ }ſchonend bei. Ich habe nicht den Eindruck, daſs
               Gefahren drohen. Nicht »Verlogenheit«, aber naive Unechtheit{ }ſozuſagen. Glauben Sie
               nicht? –\pend
           
\pstart
           {\pb}– Die \label{K_L02980-3v}\edtext{Proben\pwindex{Schnitzler, Arthur 15.\,5.\,1862 Wien – 21.\,10.\,1931 ebd.@\textsc{Schnitzler, Arthur} (15.\,5.\,1862 Wien – 21.\,10.\,1931 ebd.), \emph{Schriftsteller, Mediziner}!Schleier der Beatrice. Schauspiel in fünf Akten@\strich\emph{Der Schleier der Beatrice. Schauspiel in fünf Akten}|pwv} haben mir keine
               beſondre Freude gemacht}{\lemma{\textnormal{\emph{Proben … gemacht}}}\Cendnote{\textnormal{Siehe A. S.: \emph{Tagebuch}, 23. 2. 1903, 24. 2. 1903 und 26. 2. 1903.
               }}}\label{K_L02980-3}; i{\geminationm}erhin ko{\geminationm}t
               einiges beſſer heraus als ich dachte. Mit Leſſing\pwindex{Lessing, Emil 6.\,5.\,1857 Berlin – 1.\,11.\,1921 ebd.@\textsc{Lessing, Emil} (6.\,5.\,1857 Berlin – 1.\,11.\,1921 ebd.), \emph{Regisseur, Schauspieler}|pw} vertrag ich mich{ }ſchlecht. Brahm\pwindex{Brahm, Otto 5.\,2.\,1856 Hamburg – 28.\,11.\,1912 Berlin@\textsc{Brahm, Otto} (5.\,2.\,1856 Hamburg – 28.\,11.\,1912 Berlin), \emph{Theaterleiter, Regisseur}|pw} iſt klug und quälend wie i{\geminationm}er. Paul G.\pwindex{Goldmann, Paul 31.\,1.\,1865 Breslau – 25.\,9.\,1935 Wien@\textsc{Goldmann, Paul} (31.\,1.\,1865 Breslau – 25.\,9.\,1935 Wien), \emph{Schriftsteller, Journalist}|pw} geht als »\label{K_L02980-4v}\edtext{verbloedeter Thor}{\lemma{\textnormal{\emph{verbloedeter Thor}}}\Cendnote{\textnormal{Siehe A. S.: \emph{Tagebuch}, 22. 2. 1903.
               }}}\label{K_L02980-4}« herum. (So ne{\geminationn}t er{ }ſich{ }ſelbſt, in Anſchluſs an
               eine \introOben{}unglückliche\introOben{}{ }Liebesgeſchichte\pwindex{Rottenberg, Theodore 7.\,9.\,1875 – 5.\,4.\,1945 Limburg an der Lahn@\textsc{Rottenberg, Theodore} (7.\,9.\,1875 – 5.\,4.\,1945 Limburg an der Lahn)|pwv}, die er in
               ganz Berlin\oindex{Berlin@\textbf{Berlin}, \emph{Hauptstadt}|pw}{ }ſelber erzählt hat.) – \label{K_L02980-5v}\edtext{Heute{ }Abend ko{\geminationm}t Olga\pwindex{Schnitzler, Olga 17.\,1.\,1882 Wien – 13.\,1.\,1970 Lugano@\textsc{Schnitzler, Olga} (17.\,1.\,1882 Wien – 13.\,1.\,1970 Lugano), \emph{Schauspielerin, Sängerin}|pw} an}{\lemma{\textnormal{\emph{Heute … an}}}\Cendnote{\textnormal{Siehe A. S.: \emph{Tagebuch}, 4. 3. 1903.
               }}}\label{K_L02980-5}, { }{\pb}\label{K_L02980-6v}\edtext{Samſtag mein Bruder\pwindex{Schnitzler, Julius 13.\,7.\,1865 Wien – 29.\,6.\,1939 ebd.@\textsc{Schnitzler, Julius} (13.\,7.\,1865 Wien – 29.\,6.\,1939 ebd.), \emph{Chirurg}|pwv} (wahrſcheinlich}{\lemma{\textnormal{\emph{Samstag … (wahrscheinlich}}}\Cendnote{\textnormal{Er dürfte nicht angereist sein, jedenfalls erwähnt ihn Schnitzler in diesen Tagen nicht im \emph{Tagebuch}\pwindex{Schnitzler, Arthur 15.\,5.\,1862 Wien – 21.\,10.\,1931 ebd.@\textsc{Schnitzler, Arthur} (15.\,5.\,1862 Wien – 21.\,10.\,1931 ebd.), \emph{Schriftsteller, Mediziner}!Tagebuch@\strich\emph{Tagebuch}|pwk}.}}}\label{K_L02980-6}.) – Ich hoffe \label{K_L02980-7v}\edtext{Dinſtg{ }früh zu Hauſe}{\lemma{\textnormal{\emph{Dinstg früh zu Hause}}}\Cendnote{\textnormal{Siehe A. S.: \emph{Tagebuch}, 10. 3. 1903.
               }}}\label{K_L02980-7} zu{ }ſein und \label{K_L02980-8v}\edtext{ſpreche Sie wohl
                  gleich}{\lemma{\textnormal{\emph{spreche Sie wohl
                  gleich}}}\Cendnote{\textnormal{Nachweislich sahen sie sich
                  bereits einen Tag nach Schnitzlers Rückkehr,
                     vgl. A. S.: \emph{Tagebuch}, 11. 3. 1903.}}}\label{K_L02980-8} in
               den erſten Tagen. – Zu dem neuen \label{K_L02980-9v}\edtext{»\begin{otherlanguage}{french}Avancement\end{otherlanguage}«}{\lemma{\textnormal{\emph{»Avancement«}}}\Cendnote{\textnormal{französisch: Beförderung}}}\label{K_L02980-9} gratulir ich herzlich.
                  \label{K_L02980-10v}\edtext{Herr \textsc{Wigand\pwindex{Wigand, Curt 28.\,3.\,1865 Kassel – 1913@\textsc{Wigand, Curt} (28.\,3.\,1865 Kassel – 1913), \emph{Schriftsteller, Verleger}|pw}} war hier}{\lemma{\textnormal{\emph{Herr Wigand war hier}}}\Cendnote{\textnormal{Siehe A. S.: \emph{Tagebuch}, 3. 3. 1903.
               }}}\label{K_L02980-10} bei mir;{ }ſolang ich nur durch \textsc{Lantz\pwindex{Lantz, Adolf 10.\,11.\,1882 Wien – 19.\,8.\,1949 London@\textsc{Lantz, Adolf} (10.\,11.\,1882 Wien – 19.\,8.\,1949 London), \emph{Schriftsteller, Theaterleiter, Dramaturg}|pw}} von den \label{K_L02980-11v}\edtext{adminiſtr Zuſtänden der »Zeit\orgindex{Zeit@Die Zeit|pw}}{\lemma{\textnormal{\emph{administr … »Zeit}}}\Cendnote{\textnormal{Die
                  Unzufriedenheit an der Führung der Tageszeitung dürfte sich auf die Person von Heinrich Kanner\pwindex{Kanner, Heinrich 9.\,11.\,1864 Galați – 15.\,2.\,1930 Wien@\textsc{Kanner, Heinrich} (9.\,11.\,1864 Galați – 15.\,2.\,1930 Wien), \emph{Herausgeber, Publizist}|pwk}
                  konzentriert haben, vgl. XXXX Auszeichnungsfehler: Dokument L03415 nicht gefunden.}}}\label{K_L02980-11}«
               erfahren hatte, konnte ich einige für {\pb}unbewußt übertrieben halten, aber nach den Berichten des Hrn W.\pwindex{Wigand, Curt 28.\,3.\,1865 Kassel – 1913@\textsc{Wigand, Curt} (28.\,3.\,1865 Kassel – 1913), \emph{Schriftsteller, Verleger}|pw} find ich das Verhalten des hier in Betracht ko{\geminationm}enden Hinaus\textcolor{gray}{ſ}chmeißer\substVorne{}\textsuperscript{{ }un\textcolor{gray}{d}}\substDazwischen{},\substHinten{} Gageverkürzer und Proceſsführer einfach{ }ſkandalös. –\pend
           
\pstart
           – Leben Sie wohl,{ }ſeien Sie herzlich gegrüßt, auf Wiederſehn {\\[\baselineskip]}Ich hoffe, Ihre
                  Frau\pwindex{Salten, Ottilie 7.\,3.\,1868 Prag – 22.\,6.\,1942 Zürich@\textsc{Salten, Ottilie} (7.\,3.\,1868 Prag – 22.\,6.\,1942 Zürich), \emph{Schauspielerin}|pwv} iſt wohl. {\\[\baselineskip]}Ihr {\\[\baselineskip]}\spacefill\mbox{A.}\pend
           \leftskip=0em{}\selectlanguage{ngerman}\endnumbering\briefempfaengerindex{Salten, Felix@\textsc{Salten, Felix}!zzzSchnitzler, Arthur@\emph{von Arthur Schnitzler}!1903-03-041@{4. 3. 1903}|)be}\mylabel{L02980h}  \newcommand{\dateiname}{L02980}\newcommand{\titel}{Arthur Schnitzler an Felix Salten, 4. 3. 1903}\newcommand{\editorInnen}{Martin Anton Müller und Laura Untner}%% latex-leseansicht-abspann.tex
%% Abspann für die Leseansicht.
%% Der Schalter \ifkorrekturansicht ist bereits durch den Vorspann gesetzt.

%% latex-abspann.tex
%% Gemeinsamer Abspann für Korrekturansicht und Leseansicht.
%% Setzt den Schalter \ifkorrekturansicht voraus (gesetzt in den
%% einbindenden Dateien latex-korrekturansicht-abspann.tex bzw.
%% latex-leseansicht-abspann.tex).
%% ---------------------------------------------------------------

\normalsize

% Das esempio-Environment wird nur in der Leseansicht benötigt
\ifkorrekturansicht\else
\newenvironment{esempio}[3]%
{
    \vspace{1.5ex}
    \rlap{\underline{#1}}
    \par
    \setlength{\parindent}{0cm}
    \nopagebreak
    \leftskip=#2cm
    \rightskip=#3cm
}
{
    \par
}
\fi

\doendnotes{C}
\bigskip
\vfill

\clearpage

\footnotesize

\ifkorrekturansicht
  \lohead{\textsc{register}}
\fi

% theindex-Environment neu definieren ohne reledmac
\makeatletter
\renewenvironment{theindex}{%
  \ifkorrekturansicht
    \section*{\indexname}%
  \else
    \subsubsection*{Index der erwähnten Entitäten}%
  \fi
  \setlength{\parindent}{0pt}%
  \setlength{\parskip}{0pt plus 0.3pt}%
  \let\item\@idxitem
}{%
  \ifkorrekturansicht\clearpage\fi
}
\makeatother

\IfFileExists{\jobname-pw.ind}{\input{\jobname-pw.ind}}{}

% Quellenangabe nur in der Leseansicht
\ifkorrekturansicht\else
% Fallback-Definitionen, falls die .tex-Datei \titel etc. nicht gesetzt hat
\providecommand{\titel}{}
\providecommand{\editorInnen}{}
\providecommand{\dateiname}{\jobname}

\vspace{3cm}

\vfill

\footnotesize
\textsc{Quelle}: \titel. Herausgegeben von {\editorInnen}. In: \emph{Arthur Schnitzler: Briefwechsel mit Autorinnen und Autoren}.
 Digitale Edition, https://schnitzler-briefe.acdh.oeaw.ac.at/{\dateiname}.html (Stand \today)
\fi

\end{document}


