%% latex-leseansicht-vorspann.tex
%% Vorspann für die Leseansicht.
%% Lädt die gemeinsame Datei latex-vorspann.tex mit nicht gesetztem Schalter.

\newif\ifkorrekturansicht
\korrekturansichtfalse

\input{../tex-inputs/latex-vorspann}


         
         \renewcommand{\erwaehntePersonen}{Personen: Otto Brahm, Paul Goldmann, Mirjam Horwitz, Heinrich Kanner, Adolf Lantz, Emil Lessing, Theodore Rottenberg, Felix Salten, Ottilie Salten, Olga Schnitzler, Julius Schnitzler, Curt Wigand}
         \renewcommand{\erwaehnteInstitutionen}{Institutionen: Die Zeit}
         \renewcommand{\erwaehnteOrte}{Orte: Berlin, Wien}
         \renewcommand{\erwaehnteWerke}{Werke: Der Schleier der Beatrice. Schauspiel in fünf Akten, Tagebuch}
               \section[ Arthur Schnitzler an Felix Salten, 4. 3. 1903]{ Arthur Schnitzler an Felix Salten, 4. 3. 1903}\nopagebreak\mylabel{v}\rehead{ }\begin{ledgroupsized}[t]{13cm}\normalsize\beginnumbering\briefempfaengerindex{Salten, Felix@\textsc{Salten, Felix}!zzzSchnitzler, Arthur@\emph{von Arthur Schnitzler}!1903-03-041@{4. 3. 1903}|(be} \toendnotes[C]{\smallbreak\pagebreak[2]} \Standort{Wienbibliothek im Rathaus, ZPH 1681, 2.1.516.}
\physDesc{Brief, 1 Blatt, 4 Seiten, 1351 Zeichen
\newline{}Handschrift: Bleistift, deutsche Kurrent
\newline{}Ordnung: mit Bleistift von unbekannter Hand Nummerierung der Doppelseiten des Konvoluts:
                                    »55«–»56« }\toendnotes[C]{\smallbreak}\pstart
           \raggedleft{}{\pb}4. \textcolor{gray}{3}. 903.\pend
           \pstart
           lieber Freund, mit \label{K_L02980-1v}\edtext{\textsc{M. H.\pwindex{Horwitz, Mirjam 1882-06-15 – 1967-09-26@\textsc{Horwitz, Mirjam} (1882-06-15 – 1967-09-26), \emph{Theaterleiterin, Schauspielerin}|pw}}}{\lemma{\textnormal{\emph{M. H.}}}\Cendnote{\textnormal{siehe Felix Salten an Arthur Schnitzler, 3. 3. 1903}}}\label{K_L02980-1h} konnte ich bisher kaum hundert Worte unauffällg ſprechen; der Brief, den Sie
               erhalten, iſt natürlich die Reaction auf meine Mittheilg; – in dieſen Tagen habe ich
               jedenfalls weiter Gelegenheit ſie zu \label{K_L02980-2v}\edtext{ſehen (vielleicht heute)}{\lemma{\textnormal{\emph{ſehen (vielleicht heute)}}}\Cendnote{\textnormal{siehe A. S.: \emph{Tagebuch}, 4. 3. 1903}}}\label{K_L02980-2h} und bringe das gewünſchte ſchonend bei. Ich habe nicht den Eindruck, daſs
               Gefahren drohen. Nicht »Verlogenheit«, aber naive Unechtheit ſozuſagen. Glauben Sie
               nicht? –\pend
           \pstart
           {\pb}– Die \label{K_L02980-3v}\edtext{Proben\pwindex{Schnitzler, Arthur 15.05.1862 – 21.10.1931@\textsc{Schnitzler, Arthur} (15.05.1862 – 21.10.1931), \emph{Schriftsteller, Mediziner}!Schleier der Beatrice. Schauspiel in fuenf Akten1900-12-01@\strich\emph{Der Schleier der Beatrice. Schauspiel in fünf Akten} {[}1900-12-01{]}|pwv} haben mir keine
               beſondre Freude gemacht}{\lemma{\textnormal{\emph{Proben … gemacht}}}\Cendnote{\textnormal{siehe A. S.: \emph{Tagebuch}, 23. 2. 1903, 24. 2. 1903 und 26. 2. 1903}}}\label{K_L02980-3h}; i{\geminationm}erhin ko{\geminationm}t
               einiges beſſer heraus als ich dachte. Mit Leſſing\pwindex{Lessing, Emil 06.05.1857 – 01.11.1921@\textsc{Lessing, Emil} (06.05.1857 – 01.11.1921), \emph{Regisseur, Schauspieler}|pw} vertrag ich mich ſchlecht. Brahm\pwindex{Brahm, Otto 05.02.1856 – 28.11.1912@\textsc{Brahm, Otto} (05.02.1856 – 28.11.1912), \emph{Theaterleiter, Regisseur}|pw} iſt klug und quälend wie i{\geminationm}er. Paul G.\pwindex{Goldmann, Paul 31.01.1865 – 25.09.1935@\textsc{Goldmann, Paul} (31.01.1865 – 25.09.1935), \emph{Schriftsteller, Journalist}|pw} geht als »\label{K_L02980-4v}\edtext{verbloedeter Thor}{\lemma{\textnormal{\emph{verbloedeter Thor}}}\Cendnote{\textnormal{siehe A. S.: \emph{Tagebuch}, 22. 2. 1903}}}\label{K_L02980-4h}« herum. (So ne{\geminationn}t er ſich ſelbſt, in Anſchluſs an
               eine \introOben{}unglückliche\introOben{}{ }Liebesgeſchichte\pwindex{Rottenberg, Theodore 1875-09-07 – 1945-04-05@\textsc{Rottenberg, Theodore} (1875-09-07 – 1945-04-05)|pwv}, die er in
               ganz Berlin\oindex{Berlin@\textbf{Berlin}|pw} ſelber erzählt hat.) – \label{K_L02980-5v}\edtext{Heute{ }Abend ko{\geminationm}t Olga\pwindex{Schnitzler, Olga 17.01.1882 – 13.01.1970@\textsc{Schnitzler, Olga} (17.01.1882 – 13.01.1970), \emph{Schauspielerin, Sängerin}|pw} an}{\lemma{\textnormal{\emph{Heute … an}}}\Cendnote{\textnormal{siehe A. S.: \emph{Tagebuch}, 4. 3. 1903}}}\label{K_L02980-5h}, { }{\pb}\label{K_L02980-6v}\edtext{Samſtag mein Bruder\pwindex{Schnitzler, Julius 13.07.1865 – 29.06.1939@\textsc{Schnitzler, Julius} (13.07.1865 – 29.06.1939), \emph{Chirurg}|pwv} (wahrſcheinlich}{\lemma{\textnormal{\emph{Samſtag … (wahrſcheinlich}}}\Cendnote{\textnormal{Er dürfte nicht angereist sein, jedenfalls erwähnt in Schnitzler\pwindex{Schnitzler, Arthur 15.05.1862 – 21.10.1931@\textsc{Schnitzler, Arthur} (15.05.1862 – 21.10.1931), \emph{Schriftsteller, Mediziner}|pwk} in diesen Tagen nicht im \emph{Tagebuch}\pwindex{\textcolor{red}{\textsuperscript{XXXX1 indx}}!Tagebuch1981 – 2000@\strich\emph{Tagebuch} {[}Hrsg., 1981 – 2000{]}|pwk}.}}}\label{K_L02980-6h}.) – Ich hoffe \label{K_L02980-7v}\edtext{Dinſtg{ }früh zu Hauſe}{\lemma{\textnormal{\emph{Dinſtg früh zu Hauſe}}}\Cendnote{\textnormal{siehe A. S.: \emph{Tagebuch}, 10. 3. 1903}}}\label{K_L02980-7h} zu ſein und \label{K_L02980-8v}\edtext{ſpreche Sie wohl
                  gleich}{\lemma{\textnormal{\emph{ſpreche Sie wohl
                  gleich}}}\Cendnote{\textnormal{Nachweislich sahen sie sich
                  bereits einen Tag nach Schnitzler\pwindex{Schnitzler, Arthur 15.05.1862 – 21.10.1931@\textsc{Schnitzler, Arthur} (15.05.1862 – 21.10.1931), \emph{Schriftsteller, Mediziner}|pwk}s Rückkehr,
                     vgl. A. S.: \emph{Tagebuch}, 11. 3. 1903.}}}\label{K_L02980-8h} in
               den erſten Tagen. – Zu dem neuen \label{K_L02980-9v}\edtext{»\begin{otherlanguage}{french}Avancement\end{otherlanguage}«}{\lemma{\textnormal{\emph{»Avancement«}}}\Cendnote{\textnormal{französisch: Beförderung}}}\label{K_L02980-9h} gratulir ich herzlich.
                  \label{K_L02980-10v}\edtext{Herr \textsc{Wigand\pwindex{Wigand, Curt 28.03.1865 – 1913@\textsc{Wigand, Curt} (28.03.1865 – 1913), \emph{Schriftsteller, Verleger}|pw}} war hier}{\lemma{\textnormal{\emph{Herr Wigand war hier}}}\Cendnote{\textnormal{siehe A. S.: \emph{Tagebuch}, 3. 3. 1903}}}\label{K_L02980-10h} bei mir; ſolang ich nur durch \textsc{Lantz\pwindex{Lantz, Adolf 10.11.1882 – 19.08.1949@\textsc{Lantz, Adolf} (10.11.1882 – 19.08.1949), \emph{Schriftsteller, Theaterleiter, Dramaturg}|pw}} von den \label{K_L02980-11v}\edtext{adminiſtr Zuſtänden der »Zeit\orgindex{Zeit@Die Zeit|pw}}{\lemma{\textnormal{\emph{adminiſtr … »Zeit}}}\Cendnote{\textnormal{Die
                  Unzufriedenheit an der Führung der Tageszeitung dürfte sich auf die Person von Heinrich Kanner\pwindex{Kanner, Heinrich 09.11.1864 – 15.02.1930@\textsc{Kanner, Heinrich} (09.11.1864 – 15.02.1930), \emph{Herausgeber, Publizist}|pwk}
                  konzentriert haben, vgl. Felix Salten an Arthur Schnitzler, 9. 3. 1906.}}}\label{K_L02980-11h}«
               erfahren hatte, konnte ich einige für {\pb}unbewußt übertrieben halten, aber nach den Berichten des Hrn W.\pwindex{Wigand, Curt 28.03.1865 – 1913@\textsc{Wigand, Curt} (28.03.1865 – 1913), \emph{Schriftsteller, Verleger}|pw} find ich das Verhalten des hier in Betracht ko{\geminationm}enden Hinaus\textcolor{gray}{ſ}chmeißer\substVorne{}\textsuperscript{ un\textcolor{gray}{d}}\substDazwischen{},\substHinten{} Gageverkürzer und Proceſsführer einfach ſkandalös. –\pend
           \pstart
           – Leben Sie wohl, ſeien Sie herzlich gegrüßt, auf Wiederſehn {\\[\baselineskip]}Ich hoffe, Ihre
                  Frau\pwindex{Salten, Ottilie 07.03.1868 – 22.06.1942@\textsc{Salten, Ottilie} (07.03.1868 – 22.06.1942), \emph{Schauspielerin}|pwv} iſt wohl. {\\[\baselineskip]}Ihr {\\[\baselineskip]}\spacefill\mbox{A.}\pend
           \leftskip=0em{}
         
         \endnumbering\mylabel{h}\end{ledgroupsized}  \newcommand{\dateiname}{L02980}\newcommand{\titel}{Arthur Schnitzler an Felix Salten, 4. 3. 1903}\newcommand{\editorInnen}{Martin Anton Müller und Laura Untner}%% latex-leseansicht-abspann.tex
%% Abspann für die Leseansicht.
%% Der Schalter \ifkorrekturansicht ist bereits durch den Vorspann gesetzt.

%% latex-abspann.tex
%% Gemeinsamer Abspann für Korrekturansicht und Leseansicht.
%% Setzt den Schalter \ifkorrekturansicht voraus (gesetzt in den
%% einbindenden Dateien latex-korrekturansicht-abspann.tex bzw.
%% latex-leseansicht-abspann.tex).
%% ---------------------------------------------------------------

\normalsize

% Das esempio-Environment wird nur in der Leseansicht benötigt
\ifkorrekturansicht\else
\newenvironment{esempio}[3]%
{
    \vspace{1.5ex}
    \rlap{\underline{#1}}
    \par
    \setlength{\parindent}{0cm}
    \nopagebreak
    \leftskip=#2cm
    \rightskip=#3cm
}
{
    \par
}
\fi

\doendnotes{C}
\bigskip
\vfill

\clearpage

\footnotesize

\ifkorrekturansicht
  \lohead{\textsc{register}}
\fi

% theindex-Environment neu definieren ohne reledmac
\makeatletter
\renewenvironment{theindex}{%
  \ifkorrekturansicht
    \section*{\indexname}%
  \else
    \subsubsection*{Index der erwähnten Entitäten}%
  \fi
  \setlength{\parindent}{0pt}%
  \setlength{\parskip}{0pt plus 0.3pt}%
  \let\item\@idxitem
}{%
  \ifkorrekturansicht\clearpage\fi
}
\makeatother

\IfFileExists{\jobname-pw.ind}{\input{\jobname-pw.ind}}{}

% Quellenangabe nur in der Leseansicht
\ifkorrekturansicht\else
% Fallback-Definitionen, falls die .tex-Datei \titel etc. nicht gesetzt hat
\providecommand{\titel}{}
\providecommand{\editorInnen}{}
\providecommand{\dateiname}{\jobname}

\vspace{3cm}

\vfill

\footnotesize
\textsc{Quelle}: \titel. Herausgegeben von {\editorInnen}. In: \emph{Arthur Schnitzler: Briefwechsel mit Autorinnen und Autoren}.
 Digitale Edition, https://schnitzler-briefe.acdh.oeaw.ac.at/{\dateiname}.html (Stand \today)
\fi

\end{document}


      