%% latex-leseansicht-vorspann.tex
%% Vorspann für die Leseansicht.
%% Lädt die gemeinsame Datei latex-vorspann.tex mit nicht gesetztem Schalter.

\newif\ifkorrekturansicht
\korrekturansichtfalse

\input{../tex-inputs/latex-vorspann}

\begin{center}
            \textcolor{red}{ENTWURF, NICHT FERTIG KORRIGIERT}
                      \end{center}
            
         
         \renewcommand{\erwaehntePersonen}{Personen: Otto Brahm, Paul Goldmann, Mirjam Horwitz, Emil Lessing, Felix Salten, Olga Schnitzler}
         \renewcommand{\erwaehnteInstitutionen}{Institutionen: Die Zeit}
         \renewcommand{\erwaehnteOrte}{Orte: Berlin, Wien}
         \renewcommand{\erwaehnteWerke}{
               \section[Arthur Schnitzler an Felix Salten, 4. 1. 1903]{ Arthur Schnitzler an Felix Salten, 4. 1. 1903}\nopagebreak\mylabel{v}\rehead{ }\begin{ledgroupsized}[t]{13cm}\normalsize\beginnumbering \toendnotes[C]{\smallbreak\pagebreak[2]} \Standort{Wienbibliothek im Rathaus, ZPH 1681, 2.1.516.}
\physDesc{
\newline{}Handschrift: , deutsche Kurrent}\toendnotes[C]{\smallbreak}\pstart
           \raggedleft{}{\pb}4. 1. 903. \pend
           \pstart
           lieber Freund, mit \textsc{M. H.\pwindex{Horwitz, Mirjam 1882-06-15 – 1967-09-26@\textsc{Horwitz, Mirjam} (1882-06-15 – 1967-09-26), \emph{Theaterleiterin, Schauspielerin}|pw}} konnte ich bisher kaum hundert Worte unauffällg ſprechen; der Brief, den Sie
               erhalten, iſt natürlich die Reaction auf meine Mittheilg;– in dieſen Tagen habe ich
               jedenfalls w\textcolor{gray}{ied}er Gelegenheit ſie zu ſehen (vielleicht heute) und
               bringe das gewünſchte ſch\textcolor{gray}{were} bei. Ich habe nicht den Eindruck,
               daſs Gefahren drohen. Nicht »Verlogenheit«, aber naive Unechtheit ſozuſagen. Glauben
               Sie nicht?– {\pb}– Die Proben haben mir keine
               beſondere Freude gemacht; i{\geminationm}erhin ko{\geminationm}t einiges beſſer heraus als ich dachte. Mit Leſſing\pwindex{Lessing, Emil 06.05.1857 – 01.11.1921@\textsc{Lessing, Emil} (06.05.1857 – 01.11.1921), \emph{Regisseur, Schauspieler}|pw} vertrag ich mich ſchlecht. Brahm\pwindex{Brahm, Otto 05.02.1856 – 28.11.1912@\textsc{Brahm, Otto} (05.02.1856 – 28.11.1912), \emph{Theaterleiter, Regisseur}|pw} iſt klug und quälend i{\geminationm}er. Paul G.\pwindex{Goldmann, Paul 31.01.1865 – 25.09.1935@\textsc{Goldmann, Paul} (31.01.1865 – 25.09.1935), \emph{Schriftsteller, Journalist}|pw} geht
               als »\label{K_L02980-1v}\edtext{verbloedeter Thor}{\lemma{\textnormal{\emph{verbloedeter Thor}}}\Cendnote{\textnormal{vgl. A. S.: \emph{Tagebuch}, 22. 2. 1903}}}\label{K_L02980-1h}« herum. (So ne{\geminationn}t er ſich ſelbſt, in Anſchluſs an
               eine unglückliche Liebesgeſchichte, die er in ganz Berlin\oindex{Berlin@\textbf{Berlin}|pw} ſelber erzählt hat.)– Heut Abend ko{\geminationm}t Olga\pwindex{Schnitzler, Olga 17.01.1882 – 13.01.1970@\textsc{Schnitzler, Olga} (17.01.1882 – 13.01.1970), \emph{Schauspielerin, Sängerin}|pw} an {\pb}Samſtag mein Bruder\textcolor{red}{\textsuperscript{\textbf{KEY}}} (wahrſcheinlich.)– Ich hoffe Dinſtg früh zu Hauſe zu
               ſein und ſpreche Sie wohl gleich in den erſten Tagen. Zu dem neuen »Avancement«
               gratulir ich herzlich. Herr \textsc{Wigand\textcolor{red}{\textsuperscript{\textbf{KEY}}}} war hier bei mir; ſolang ich nur durch \textsc{Lantz\textcolor{red}{\textsuperscript{\textbf{KEY}}}} von den adminiſtr. Zuſtänden der »Zeit\orgindex{Zeit@Die Zeit|pw}«
               erfahren hatte, konnte ich einige für {\pb}unbewußt übertrieben halten, aber nach den Berichten des Herrn W.\textcolor{red}{\textsuperscript{\textbf{KEY}}} find ich das Verhalten des hier in Betracht ko{\geminationm}enden Hinausſch\textcolor{gray}{miß}es \strikeout{un\textcolor{gray}{d}}\textcolor{gray}{wie} Gageverkürzen \textcolor{gray}{wie }Proceſs \textcolor{gray}{führen} einfach
               ſkandalös.– \pend
           \pstart
           – Leben Sie wohl, ſeien Sie {\\[\baselineskip]}herzlich gegrüßt, auf Wiederſehen {\\[\baselineskip]}Ich
               hoffe Ihre Frau\textcolor{red}{\textsuperscript{\textbf{KEY}}} iſt wohl, {\\[\baselineskip]}Ihr {\\[\baselineskip]}\spacefill\mbox{A.}\pend
           \leftskip=0em{}
         
         \endnumbering\mylabel{h}\end{ledgroupsized}\begin{anhang}\end{anhang}\newcommand{\dateiname}{L02980}\newcommand{\titel}{Arthur Schnitzler an Felix Salten, 4. 1. 1903}\newcommand{\editorInnen}{Martin Anton Müller und Laura Untner}%% latex-leseansicht-abspann.tex
%% Abspann für die Leseansicht.
%% Der Schalter \ifkorrekturansicht ist bereits durch den Vorspann gesetzt.

%% latex-abspann.tex
%% Gemeinsamer Abspann für Korrekturansicht und Leseansicht.
%% Setzt den Schalter \ifkorrekturansicht voraus (gesetzt in den
%% einbindenden Dateien latex-korrekturansicht-abspann.tex bzw.
%% latex-leseansicht-abspann.tex).
%% ---------------------------------------------------------------

\normalsize

% Das esempio-Environment wird nur in der Leseansicht benötigt
\ifkorrekturansicht\else
\newenvironment{esempio}[3]%
{
    \vspace{1.5ex}
    \rlap{\underline{#1}}
    \par
    \setlength{\parindent}{0cm}
    \nopagebreak
    \leftskip=#2cm
    \rightskip=#3cm
}
{
    \par
}
\fi

\doendnotes{C}
\bigskip
\vfill

\clearpage

\footnotesize

\ifkorrekturansicht
  \lohead{\textsc{register}}
\fi

% theindex-Environment neu definieren ohne reledmac
\makeatletter
\renewenvironment{theindex}{%
  \ifkorrekturansicht
    \section*{\indexname}%
  \else
    \subsubsection*{Index der erwähnten Entitäten}%
  \fi
  \setlength{\parindent}{0pt}%
  \setlength{\parskip}{0pt plus 0.3pt}%
  \let\item\@idxitem
}{%
  \ifkorrekturansicht\clearpage\fi
}
\makeatother

\IfFileExists{\jobname-pw.ind}{\input{\jobname-pw.ind}}{}

% Quellenangabe nur in der Leseansicht
\ifkorrekturansicht\else
% Fallback-Definitionen, falls die .tex-Datei \titel etc. nicht gesetzt hat
\providecommand{\titel}{}
\providecommand{\editorInnen}{}
\providecommand{\dateiname}{\jobname}

\vspace{3cm}

\vfill

\footnotesize
\textsc{Quelle}: \titel. Herausgegeben von {\editorInnen}. In: \emph{Arthur Schnitzler: Briefwechsel mit Autorinnen und Autoren}.
 Digitale Edition, https://schnitzler-briefe.acdh.oeaw.ac.at/{\dateiname}.html (Stand \today)
\fi

\end{document}


      