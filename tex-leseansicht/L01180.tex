%% latex-leseansicht-vorspann.tex
%% Vorspann für die Leseansicht.
%% Lädt die gemeinsame Datei latex-vorspann.tex mit nicht gesetztem Schalter.

\newif\ifkorrekturansicht
\korrekturansichtfalse

\input{../tex-inputs/latex-vorspann}


\section[Hermann Bahr an Arthur Schnitzler, 7. 10. {[}1901{]}]{L01180 Hermann Bahr an Arthur Schnitzler, 7. 10. [1901]}
\nopagebreak\mylabel{L01180v}
\rehead{ }\normalsize\beginnumbering\briefempfaengerindex{Schnitzler, Arthur@\textsc{Schnitzler, Arthur}!zzzBahr, Hermann@\emph{von Hermann Bahr}!1901-10-071@{7. 10. 1901}|(be}
\toendnotes[C]{\smallbreak\pagebreak[2]}
\correspDesc{Versand  durch Hermann Bahr am 7. 10. 1901 in Wien
\newline{}Erhalt  durch Arthur Schnitzler im Zeitraum [7. 10. 1901
                  – 11. 10. 1901?] in Wien}\toendnotes[C]{\smallbreak}
\Standort{CUL, Schnitzler, B 5b.}
\physDesc{Brief, 1 Blatt, 2 Seiten, 484 Zeichen
\newline{}Handschrift: blaue Tinte, deutsche Kurrent
\newline{}Schnitzler: mit Bleistift die Jahreszahl »901« ergänzt 
\newline{}Ordnung: mit Bleistift von unbekannter Hand nummeriert:
                                    »80« }
\buchAbdrucke{\weitereDrucke{Hermann Bahr, Arthur Schnitzler: \emph{Briefwechsel, Aufzeichnungen, Dokumente (1891–1931)}. Herausgegeben von Kurt Ifkovits und Martin Anton Müller. Göttingen: \emph{Wallstein} 2018, S. 215.} }\toendnotes[C]{\smallbreak}
\pstart
           \raggedleft{}{\pb}7. October\pend
           
\pstart\center{}Lieber Arthur!\pend\vspace{0.5em}
\pstart
           Morgen kann ich leider nicht, da ich zur \label{K_L01180-1v}\edtext{Probe\eventindex{Volkstheater@\textbf{Volkstheater}!Probe von Der neue Simson@Probe von Der neue Simson|pw} des »neuen
                    Simſon\pwindex{Karlweis, Carl 23.\,11.\,1850 Wien – 27.\,10.\,1901 ebd.@\textsc{Karlweis, Carl} (23.\,11.\,1850 Wien – 27.\,10.\,1901 ebd.), \emph{Schriftsteller}!neue Simson@\strich\emph{Der neue Simson}|pw}«}{\lemma{\textnormal{\emph{Probe des »neuen
                    Simson«}}}\Cendnote{\textnormal{Das Schauspiel\pwindex{Karlweis, Carl 23.\,11.\,1850 Wien – 27.\,10.\,1901 ebd.@\textsc{Karlweis, Carl} (23.\,11.\,1850 Wien – 27.\,10.\,1901 ebd.), \emph{Schriftsteller}!neue Simson@\strich\emph{Der neue Simson}|pwkv} von C. Karlweis\pwindex{Karlweis, Carl 23.\,11.\,1850 Wien – 27.\,10.\,1901 ebd.@\textsc{Karlweis, Carl} (23.\,11.\,1850 Wien – 27.\,10.\,1901 ebd.), \emph{Schriftsteller}|pwk} hatte die Uraufführung\eventindex{Volkstheater@\textbf{Volkstheater}!Uraufführung von Der neue Simson@Uraufführung von Der neue Simson|pwkv} am
                     19. 10. 1901 im Deutschen
                     Volkstheater\oindex{Wien@\textbf{Wien}!VII., Neubau@\textbf{VII., Neubau}!Volkstheater@\textbf{Volkstheater}, \emph{Theater}|pwk}.}}}\label{K_L01180-1} muß, und eben um dieſes Stückes willen kann ich auch
               über die nächſten Tage nicht wohl disponieren. Dagegen bin ich \label{K_L01180-2v}\edtext{Samſtag, Sonntag, Montag}{\lemma{\textnormal{\emph{Samstag, Sonntag, Montag}}}\Cendnote{\textnormal{14., 15., 16. 10. 1901.}}}\label{K_L01180-2} immer mit
               Vergnügen bereit und, wenn es{ }ſchön iſt, könnte ich Dir dann auch die {\pb}hübſchen kleinen Spazierwege zeigen, die es hier
               gibt. Übrigens hoffe ich Dich noch vorher zu{ }ſehen, da ich in den nächſten Tagen
               einmal zu Dir springen will.\pend
           
\pstart
           Herzlichſt{\\[\baselineskip]}Dein{\\[\baselineskip]}\spacefill\mbox{Hermann}\pend
           \leftskip=0em{}\selectlanguage{ngerman}\endnumbering\briefempfaengerindex{Schnitzler, Arthur@\textsc{Schnitzler, Arthur}!zzzBahr, Hermann@\emph{von Hermann Bahr}!1901-10-071@{7. 10. 1901}|)be}\mylabel{L01180h}  \newcommand{\dateiname}{L01180}\newcommand{\titel}{Hermann Bahr an Arthur Schnitzler, 7. 10. [1901]}\newcommand{\editorInnen}{Herausgegeben von Martin Anton Müller}%% latex-leseansicht-abspann.tex
%% Abspann für die Leseansicht.
%% Der Schalter \ifkorrekturansicht ist bereits durch den Vorspann gesetzt.

%% latex-abspann.tex
%% Gemeinsamer Abspann für Korrekturansicht und Leseansicht.
%% Setzt den Schalter \ifkorrekturansicht voraus (gesetzt in den
%% einbindenden Dateien latex-korrekturansicht-abspann.tex bzw.
%% latex-leseansicht-abspann.tex).
%% ---------------------------------------------------------------

\normalsize

% Das esempio-Environment wird nur in der Leseansicht benötigt
\ifkorrekturansicht\else
\newenvironment{esempio}[3]%
{
    \vspace{1.5ex}
    \rlap{\underline{#1}}
    \par
    \setlength{\parindent}{0cm}
    \nopagebreak
    \leftskip=#2cm
    \rightskip=#3cm
}
{
    \par
}
\fi

\doendnotes{C}
\bigskip
\vfill

\clearpage

\footnotesize

\ifkorrekturansicht
  \lohead{\textsc{register}}
\fi

% theindex-Environment neu definieren ohne reledmac
\makeatletter
\renewenvironment{theindex}{%
  \ifkorrekturansicht
    \section*{\indexname}%
  \else
    \subsubsection*{Index der erwähnten Entitäten}%
  \fi
  \setlength{\parindent}{0pt}%
  \setlength{\parskip}{0pt plus 0.3pt}%
  \let\item\@idxitem
}{%
  \ifkorrekturansicht\clearpage\fi
}
\makeatother

\IfFileExists{\jobname-pw.ind}{\input{\jobname-pw.ind}}{}

% Quellenangabe nur in der Leseansicht
\ifkorrekturansicht\else
% Fallback-Definitionen, falls die .tex-Datei \titel etc. nicht gesetzt hat
\providecommand{\titel}{}
\providecommand{\editorInnen}{}
\providecommand{\dateiname}{\jobname}

\vspace{3cm}

\vfill

\footnotesize
\textsc{Quelle}: \titel. Herausgegeben von {\editorInnen}. In: \emph{Arthur Schnitzler: Briefwechsel mit Autorinnen und Autoren}.
 Digitale Edition, https://schnitzler-briefe.acdh.oeaw.ac.at/{\dateiname}.html (Stand \today)
\fi

\end{document}


