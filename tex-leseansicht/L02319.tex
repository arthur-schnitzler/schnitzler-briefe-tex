%% latex-korrekturansicht-vorspann.tex
%% Vorspann für die Korrekturansicht.
%% Lädt die gemeinsame Datei latex-vorspann.tex mit gesetztem Schalter.

\newif\ifkorrekturansicht
\korrekturansichttrue

\input{../tex-inputs/latex-vorspann}


\section[Robert Adam an Arthur Schnitzler, 17. 1. 1919]{L02319 Robert Adam an Arthur Schnitzler, 17. 1. 1919}
\nopagebreak\mylabel{L02319v}
\rehead{ }\normalsize\beginnumbering\briefempfaengerindex{Schnitzler, Arthur@\textsc{Schnitzler, Arthur}!zzzAdam, Robert@\emph{von Robert Adam}!1919-01-171@{17. 1. 1919}|(be}
\toendnotes[C]{\smallbreak\pagebreak[2]}\Standort{CUL, Schnitzler, B 1.}
\physDesc{Brief, 1 Blatt, 4 Seiten, 3058 Zeichen
\newline{}Handschrift: schwarze Tinte, deutsche Kurrent
\newline{}Schnitzler: 1) mit Bleistift beschriftet: »\textsc{Adam}«  2) mit rotem Buntstift mehrere Unterstreichungen
\newline{}Ordnung: mit Bleistift von unbekannter Hand nummeriert:
                                    »12« }\Standort{Wien, Österreichische Nationalbibliothek, Cod.ser. 52.269, 231.}
\physDesc{Brief, maschinenschriftliche Abschrift1 Blatt, 1 Seite, 3058 Zeichen
\newline{}Schreibmaschine}\toendnotes[C]{\smallbreak}
\pstart
           \raggedleft{}{\pb}Wien\oindex{Wien@\textbf{Wien}, \emph{A.ADM2}|pw}, 17. Januar 1919\pend
           
\pstart{}Hochverehrter Herr Doktor!\pend\vspace{0.5em}
\pstart
           Da ich vom Deutſchen Volkstheater\oindex{Volkstheater@\textbf{Volkstheater}, \emph{Theater (K.THE)}|pw} zwei Monate lang
               nichts zu hören bekam, konnte ich meine verzweifelte Ungeduld nicht mehr bezwingen
               und ſchrieb an die Direktion, ſie möchte ſo gut ſein, mich vom Stande der Dinge zu
               verſtändigen (was wir in der Amtsſprache eine »Betreibung« nennen). Heut erhielt ich
               nun vom Dramaturgen D\textsuperscript{r}{ }\textsc{Glücksmann}\pwindex{Gluecksmann, Heinrich 08.07.1864 – 01.03.1943@\textsc{Glücksmann, Heinrich} (08.07.1864 – 01.03.1943), \emph{Schriftsteller/Schriftstellerin, Journalist/Journalistin, Dramaturg/Dramaturgin}|pw} folgenden Brief:\pend
           
\pstart
           »Ihre beiden dramatiſchen Arbeiten ›Der Fremde\pwindex{Fremde@\emph{Der Fremde}|pw}‹
               und ›Yppl\pwindex{Fremde@\emph{Der Fremde}|pw}‹ ſind längſt geleſen. Ich geſtehe
               ſofort: mit lebhafteſtem Genuß. Die Chriſtus\pwindex{Jesus 7–4 v. u. Z. – 30/31@\textsc{Jesus} (7–4 v. u. Z. – 30/31), \emph{Wanderprediger/Wanderpredigerin}|pw}-Szenen ſind nicht alle gleichwertig, aber doch zumeiſt ſchön und tief
               und nachklingend. Vielleicht wird es möglich ſein, ſie im Rahmen einer {\pb}literariſchen Veranſtaltung zu bringen.
               Das 3. Bild wird man wohl auslaſſen müſſen, aus inneren Gründen, und vielleicht iſt
               auch das 4., nur ein undramatiſches Gleichnis, von der Bühne herab nicht
                  wirkſam. 1.{ }2.{ }5. und 6 dürften jedoch ihre Probe beſtehen.\pend
           
\pstart
           »Was ›Yppl\pwindex{Yppl. Idylle in fuenf Akten@\emph{Yppl. Idylle in fünf Akten}|pw}‹ anbelangt, ſo iſt es eine gute
               Satire auf das kleinſtädtiſche Beamtenleben. Die Löſung des Konfliktes erſcheint mir
               freilich gewaltſam und nicht überzeugend, die Wiederholung der Probe des
               Dilettanten-Stückes wäre zu vermeiden, weil ſie ein bischen auf den Gang der Handlung
               drückt. Jedenfalls habe ich es für meine Pflicht gehalten, Herrn Direktor Bernau\pwindex{Bernau, Alfred 06.03.1879 – 20.08.1950@\textsc{Bernau, Alfred} (06.03.1879 – 20.08.1950), \emph{Theaterleiter/Theaterleiterin, Schauspieler/Schauspielerin}|pw} für die beiden Arbeiten zu intereſſieren.
               Sobald er dazu kommt, wird er ſie auch leſen.«\pend
           
\pstart
           Nach Erhalt dieſes Briefes, des angenehmſten, den ich noch in Theaterdingen bekommen
               habe, begab ich mich – heut vormittags – in das Theater und ſuchte D\textsuperscript{r}{ }\textsc{Glücksmann}\pwindex{Gluecksmann, Heinrich 08.07.1864 – 01.03.1943@\textsc{Glücksmann, Heinrich} (08.07.1864 – 01.03.1943), \emph{Schriftsteller/Schriftstellerin, Journalist/Journalistin, Dramaturg/Dramaturgin}|pw} auf. Er äußerte ſich ſehr liebenswürdig {\pb}über beide Stücke\pwindex{Yppl. Idylle in fuenf Akten@\emph{Yppl. Idylle in fünf Akten}|pwv}\pwindex{Fremde@\emph{Der Fremde}|pwv} und ſagte, er habe ſie dem
                  Direktor\pwindex{Bernau, Alfred 06.03.1879 – 20.08.1950@\textsc{Bernau, Alfred} (06.03.1879 – 20.08.1950), \emph{Theaterleiter/Theaterleiterin, Schauspieler/Schauspielerin}|pwv}{ }ſchon längſt als erſte der von ihm zu leſenden
               vorbereitet, doch ſei er immer noch abgehalten geweſen, die Lektüre vorzunehmen.\pend
           
\pstart
           Daß D\textsuperscript{r}{ }\textsc{Glücksmann}\pwindex{Gluecksmann, Heinrich 08.07.1864 – 01.03.1943@\textsc{Glücksmann, Heinrich} (08.07.1864 – 01.03.1943), \emph{Schriftsteller/Schriftstellerin, Journalist/Journalistin, Dramaturg/Dramaturgin}|pw} gerade das 3. und 4. Bild des »Fremden\pwindex{Yppl. Idylle in fuenf Akten@\emph{Yppl. Idylle in fünf Akten}|pw}«
                  (»Die Hure\pwindex{Yppl. Idylle in fuenf Akten@\emph{Yppl. Idylle in fünf Akten}|pwv}« und »Der Hund\pwindex{Yppl. Idylle in fuenf Akten@\emph{Yppl. Idylle in fünf Akten}|pwv}«) für untheatraliſch
               hält, iſt mir nicht recht begreiflich, da ich immer gerade dieſe beiden Szenen für
               die dramatiſch allein wirkſamen gehalten habe, und auch Sie, hochverehrter Herr
               Doktor, haben eine ähnliche Meinung geäußert.\pend
           
\pstart
           Welche Szenen aber zur Aufführung kommen, ſcheint mir von ſekundärer Wichtigkeit;
               wenn nur überhaupt eine Annahme erfolgte! Denn damit wäre wohl die Möglichkeit
               gegeben, einen Verleger zu finden, und ich ſehne mich unbändig danach, juſt den »Fremden\pwindex{Yppl. Idylle in fuenf Akten@\emph{Yppl. Idylle in fünf Akten}|pw}« gedruckt zu ſehen.\pend
           
\pstart
           Ich will nun den Verſuch machen, Direktor Bernau\pwindex{Bernau, Alfred 06.03.1879 – 20.08.1950@\textsc{Bernau, Alfred} (06.03.1879 – 20.08.1950), \emph{Theaterleiter/Theaterleiterin, Schauspieler/Schauspielerin}|pw} im Theater anzu{\pb}treffen
               und ihn zu Beſchleunigung ſeiner Entſcheidung zu veranlaſſen. Sollten Sie,
               hochverehrter Herr Doktor, in der nächſten Zeit einmal mit ihm zuſammentreffen, ſo
               bitte ihn bei dieſer Gelegenheit meine Stücke in Erinnerung zu bringen (ſofern es
               Ihnen nicht unangenehm iſt).\pend
           
\pstart
           Außer dieſem Ereignis weiß ich aus der Monotonie meiner Exiſtenz nichts zu berichten:
               ich arbeite im Amt und leſe daheim, halbſatt und halbwarm und halb im
               Winterſchlaf.\pend
           
\pstart
           Wenn ich Sie nicht ſtöre, möchte ich Sie gerne wieder einmal aufſuchen; ich habe
               einige kleine Lektüreentdeckungen gemacht, die Sie vielleicht intereſſieren
               könnten.\pend
           
\pstart
           Mit beſten Grüßen Ihr ſehr{\\[\baselineskip]}ergebener{\\[\baselineskip]}\spacefill\mbox{D\textsuperscript{r}RAdam}\pend
           \leftskip=0em{}\selectlanguage{ngerman}\endnumbering\briefempfaengerindex{Schnitzler, Arthur@\textsc{Schnitzler, Arthur}!zzzAdam, Robert@\emph{von Robert Adam}!1919-01-171@{17. 1. 1919}|)be}\mylabel{L02319h}  \normalsize

\doendnotes{C}
\bigskip
\vfill

\clearpage

\footnotesize

\lohead{\textsc{register}}

% Definiere theindex-Environment komplett neu ohne reledmac
\makeatletter
\renewenvironment{theindex}{%
  \section*{\indexname}%
  \setlength{\parindent}{0pt}%
  \setlength{\parskip}{0pt plus 0.3pt}%
  \let\item\@idxitem
}{%
  \clearpage
}
\makeatother

\IfFileExists{\jobname-pw.ind}{\input{\jobname-pw.ind}}{}

\end{document}

      