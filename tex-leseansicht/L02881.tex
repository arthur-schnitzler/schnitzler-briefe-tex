%% latex-korrekturansicht-vorspann.tex
%% Vorspann für die Korrekturansicht.
%% Lädt die gemeinsame Datei latex-vorspann.tex mit gesetztem Schalter.

\newif\ifkorrekturansicht
\korrekturansichttrue

\input{../tex-inputs/latex-vorspann}


\section[ Paul Goldmann an Arthur Schnitzler, 27. 7. {[}1899{]}]{L02881 Paul Goldmann an Arthur Schnitzler, 27. 7. {[}1899{]}}
\nopagebreak\mylabel{L02881v}
\rehead{ }\normalsize\beginnumbering\briefempfaengerindex{Schnitzler, Arthur@\textsc{Schnitzler, Arthur}!zzzGoldmann, Paul@\emph{von Paul Goldmann}!1899-07-272@{27. 7. {[}1899{]}}|(be}
\toendnotes[C]{\smallbreak\pagebreak[2]}\Standort{DLA, A:Schnitzler, HS.NZ85.1.3169.}
\physDesc{Brief, 1 Blatt, 4 Seiten, 1544 Zeichen
\newline{}Handschrift: blaue Tinte, deutsche Kurrent
\newline{}Schnitzler: mit Bleistift das Jahr »99.« vermerkt }\toendnotes[C]{\smallbreak}
\pstart
           \raggedleft{}{\pb}Bayreuth\oindex{Bayreuth@\textbf{Bayreuth}, \emph{P.PPLA2}|pw}, 27. Juli.\pend
           
\pstart\center{}Mein lieber Freund,\pend\vspace{0.5em}
\pstart
           Ich erhielt hier Deinen Brief aus \label{K_L02881-1v}\edtext{\textsc{Velden\oindex{Velden am Woerthersee@\textbf{Velden am Wörthersee}, \emph{P.PPL}|pw}}}{\lemma{\textnormal{\emph{Velden}}}\Cendnote{\textnormal{Schnitzler hielt sich vom 18. 7. 1899 bis zum 28. 7. 1899 in Velden am Wörthersee\oindex{Velden am Woerthersee@\textbf{Velden am Wörthersee}, \emph{P.PPL}|pwk} auf.}}}\label{K_L02881-1} und freue
               mich, Dich in guter Geſellſchaft an einem ſchönen See\oindex{Woerthersee@\textbf{Wörthersee}, \emph{H.LK}|pwv} zu wiſſen.\pend
           
\pstart
           Hier\oindex{Bayreuth@\textbf{Bayreuth}, \emph{P.PPLA2}|pwv} habe ich eben die vier
               Tage des \label{K_L02881-2v}\edtext{Nibelungen-Ring\pwindex{Ring des Nibelungen@\emph{Der Ring des Nibelungen}|pw}}{\lemma{\textnormal{\emph{Nibelungen-Ring}}}\Cendnote{\textnormal{Richard Wagners\pwindex{Wagner, Richard 22.05.1813 – 13.02.1883@\textsc{Wagner, Richard} (22.05.1813 – 13.02.1883), \emph{Komponist/Komponistin}|pwk} Opernzyklus \emph{Der Ring des Nibelungen}\pwindex{Ring des Nibelungen@\emph{Der Ring des Nibelungen}|pwk} wurde bei den \emph{Bayreuther Festspielen}\orgindex{Bayreuther Festspiele@Bayreuther Festspiele|pwk}{ }1899 unter der Leitung von Cosima Wagner\pwindex{Wagner, Cosima 25.12.1837 – 01.04.1930@\textsc{Wagner, Cosima} (25.12.1837 – 01.04.1930)|pwk} aufgeführt.}}}\label{K_L02881-2} abſolvirt. Halbtodt vor
               Arbeit und Anſtrengung. Aber gewaltige Eindrücke. Es iſt unverzeihlich, daß Du noch
               nicht in Bayreuth\oindex{Bayreuth@\textbf{Bayreuth}, \emph{P.PPLA2}|pw}\orgindex{Bayreuther Festspiele@Bayreuther Festspiele|pwv} warſt, und Du ſollteſt es möglich machen, wenigſtens zum zweiten Cyclus\pwindex{Ring des Nibelungen@\emph{Der Ring des Nibelungen}|pwv} im Auguſt{ }\strikeout{her}{ }\label{K_L02881-3v}\edtext{herzukommen}{\lemma{\textnormal{\emph{herzukommen}}}\Cendnote{\textnormal{Dazu kam es nicht. Schnitzler besuchte die \emph{Bayreuther
                     Festspiele}\orgindex{Bayreuther Festspiele@Bayreuther Festspiele|pwk} nie.}}}\label{K_L02881-3}. Dieſes Bayreuth\oindex{Bayreuth@\textbf{Bayreuth}, \emph{P.PPLA2}|pw}\orgindex{Bayreuther Festspiele@Bayreuther Festspiele|pwv} gehört zum Größten i\substVorne{}\textsuperscript{\textcolor{gray}{m}}\substDazwischen{}n\substHinten{}{ }\introOben{}der\introOben{} gegenwärtigen deutſch\oindex{Deutschland@\textbf{Deutschland}, \emph{A.PCLI}|pwv}en Kunſt, und man muß es einmal miterlebt haben.\pend
           
\pstart
           {\pb}Von hier gehe ich nach \textsc{Rennes\oindex{Rennes@\textbf{Rennes}, \emph{P.PPLA}|pw}}, und alle meine Urlaubspläne hängen ab von dem Zeitpunkt, an dem der \label{K_L02881-4v}\edtext{Prozeß\pwindex{Dreyfus, Alfred 1859-10-09 – 1935-07-12@\textsc{Dreyfus, Alfred} (1859-10-09 – 1935-07-12), \emph{Militär/Militärin}|pwv}}{\lemma{\textnormal{\emph{Prozeß}}}\Cendnote{\textnormal{Gemeint war der neue
                  Kriegsgerichtsprozess in der Affäre Dreyfus\pwindex{Dreyfus, Alfred 1859-10-09 – 1935-07-12@\textsc{Dreyfus, Alfred} (1859-10-09 – 1935-07-12), \emph{Militär/Militärin}|pwk},
                  der bis in den September fortdauerte. Es kam zum
                  Schuldspruch am 9. 9. 1899 und zur Begnadigung am
                     19. 9. 1899.}}}\label{K_L02881-4} zu Ende iſt. Dauert er, wie
               ich vorausſehe, bis tief in den Auguſt hinein, ſo kann
               ich dann fürs Erſte nicht mehr fort, da \strikeout{i\textcolor{gray}{ch}} im September{ }\textsc{Dr. Mamroth\pwindex{Mamroth, Fedor 21.02.1851 – 25.06.1907@\textsc{Mamroth, Fedor} (21.02.1851 – 25.06.1907), \emph{Journalist/Journalistin, Kritiker/Kritikerin}|pw}} auf Urlaub geht, den ich vertreten muß. In dieſem Falle würde ich nach \textsc{Mamroths\pwindex{Mamroth, Fedor 21.02.1851 – 25.06.1907@\textsc{Mamroth, Fedor} (21.02.1851 – 25.06.1907), \emph{Journalist/Journalistin, Kritiker/Kritikerin}|pw}} Rückkehr (wenn die Ereigniſſe
               in der Welt es erlauben u. ich zu dieſem Zeitpunkt nicht etwa nach Grönland\oindex{Groenland@\textbf{Grönland}, \emph{A.PCLD}|pw} muß, um dort über eine Revolution der Eskimos zu
               berichten) ſo zwiſchen dem \strikeout{5.}{ }5. und 10. Oktober
               meinen Urlaub {\pb}antreten u. nach Italien\oindex{Italien@\textbf{Italien}, \emph{A.PCLI}|pw} gehen. Die Ausſicht, daß Du \label{K_L02881-5v}\edtext{mitgehſt}{\lemma{\textnormal{\emph{mitgehſt}}}\Cendnote{\textnormal{Dazu kam es nicht.}}}\label{K_L02881-5}, iſt entzückend. Vielleicht kommt auch \textsc{Richard\pwindex{Beer-Hofmann, Richard 1866-07-11 – 1945-09-26@\textsc{Beer-Hofmann, Richard} (1866-07-11 – 1945-09-26), \emph{Schriftsteller/Schriftstellerin}|pw}} mit. Defintives aber kann ich Dir erſt nach meiner Rückkehr von \textsc{Rennes\oindex{Rennes@\textbf{Rennes}, \emph{P.PPLA}|pw}} ſagen.\pend
           
\pstart
           Ich wünſche Dir weiter recht viel Sommer-Erholung und recht viel ſonnige Tage, in
               denen keine Schatten umgehen. Komm’ nur hierher\oindex{Bayreuth@\textbf{Bayreuth}, \emph{P.PPLA2}|pwv} und laß’ Dir vom Bayreuth\oindex{Bayreuth@\textbf{Bayreuth}, \emph{P.PPLA2}|pw}er Orcheſter\orgindex{Bayreuther Festspiele@Bayreuther Festspiele|pwv} Freude am Leben und
               Freude an der Kunſt \strikeout{ins} im Herzen entzünden!\pend
           
\pstart
           In Treue {\\[\baselineskip]}Dein \spacefill\mbox{Paul Goldmann}\pend
           \leftskip=0em{}
\pstart
           \noindent{}{\pb}Viele Grüße an Deine Familie, wenn Du bei ihr
                  biſt!\pend
           \selectlanguage{ngerman}\endnumbering\briefempfaengerindex{Schnitzler, Arthur@\textsc{Schnitzler, Arthur}!zzzGoldmann, Paul@\emph{von Paul Goldmann}!1899-07-272@{27. 7. {[}1899{]}}|)be}\mylabel{L02881h}  \normalsize

\doendnotes{C}
\bigskip
\vfill

\clearpage

\footnotesize

\lohead{\textsc{register}}

% Definiere theindex-Environment komplett neu ohne reledmac
\makeatletter
\renewenvironment{theindex}{%
  \section*{\indexname}%
  \setlength{\parindent}{0pt}%
  \setlength{\parskip}{0pt plus 0.3pt}%
  \let\item\@idxitem
}{%
  \clearpage
}
\makeatother

\IfFileExists{\jobname-pw.ind}{\input{\jobname-pw.ind}}{}

\end{document}

      