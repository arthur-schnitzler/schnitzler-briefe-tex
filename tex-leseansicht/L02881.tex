%% latex-leseansicht-vorspann.tex
%% Vorspann für die Leseansicht.
%% Lädt die gemeinsame Datei latex-vorspann.tex mit nicht gesetztem Schalter.

\newif\ifkorrekturansicht
\korrekturansichtfalse

\input{../tex-inputs/latex-vorspann}


\section[ Paul Goldmann an Arthur Schnitzler, 27. 7. {[}1899{]}]{L02881 Paul Goldmann an Arthur Schnitzler,  27. 7. [1899]}
\nopagebreak\mylabel{L02881v}
\rehead{ }\normalsize\beginnumbering\briefempfaengerindex{Schnitzler, Arthur@\textsc{Schnitzler, Arthur}!zzzGoldmann, Paul@\emph{von Paul Goldmann}!1899-07-273@{27. 7. [1899]}|(be}
\toendnotes[C]{\smallbreak\pagebreak[2]}
\correspDesc{Versand  durch Paul Goldmann am 27. 7. [1899] in Bayreuth
\newline{}Erhalt  durch Arthur Schnitzler im Zeitraum [29. 7. 1899
                  – 1. 8. 1899?] in Villach?}\toendnotes[C]{\smallbreak}
\Standort{DLA, A:Schnitzler, HS.NZ85.1.3169.}
\physDesc{Brief, 1 Blatt, 4 Seiten, 1544 Zeichen
\newline{}Handschrift: blaue Tinte, deutsche Kurrent
\newline{}Schnitzler: mit Bleistift das Jahr »99.« vermerkt }\toendnotes[C]{\smallbreak}
\pstart
           \raggedleft{}{\pb}Bayreuth\oindex{Bayreuth@\textbf{Bayreuth}, \emph{Hauptstadt}|pw}, 27. Juli.\pend
           
\pstart\center{}Mein lieber Freund,\pend\vspace{0.5em}
\pstart
           Ich erhielt hier Deinen Brief aus \label{K_L02881-1v}\edtext{\textsc{Velden\oindex{Velden am Wörthersee@\textbf{Velden am Wörthersee}|pw}}}{\lemma{\textnormal{\emph{Velden}}}\Cendnote{\textnormal{Schnitzler hielt sich vom 18. 7. 1899 bis zum 28. 7. 1899 in Velden am Wörthersee\oindex{Velden am Wörthersee@\textbf{Velden am Wörthersee}|pwk} auf.}}}\label{K_L02881-1} und freue
               mich, Dich in guter Geſellſchaft an einem{ }ſchönen See\oindex{Wörthersee@\textbf{Wörthersee}, \emph{See}|pwv} zu wiſſen.\pend
           
\pstart
           Hier\oindex{Bayreuth@\textbf{Bayreuth}, \emph{Hauptstadt}|pwv} habe ich eben die vier
               Tage des \label{K_L02881-2v}\edtext{Nibelungen-Ring\pwindex{Wagner, Richard 22.\,5.\,1813 Leipzig – 13.\,2.\,1883 Venedig@\textsc{Wagner, Richard} (22.\,5.\,1813 Leipzig – 13.\,2.\,1883 Venedig), \emph{Komponist}!Ring des Nibelungen@\strich\emph{Der Ring des Nibelungen}|pw}}{\lemma{\textnormal{\emph{Nibelungen-Ring}}}\Cendnote{\textnormal{Richard Wagners\pwindex{Wagner, Richard 22.\,5.\,1813 Leipzig – 13.\,2.\,1883 Venedig@\textsc{Wagner, Richard} (22.\,5.\,1813 Leipzig – 13.\,2.\,1883 Venedig), \emph{Komponist}|pwk} Opernzyklus \emph{Der Ring des Nibelungen}\pwindex{Wagner, Richard 22.\,5.\,1813 Leipzig – 13.\,2.\,1883 Venedig@\textsc{Wagner, Richard} (22.\,5.\,1813 Leipzig – 13.\,2.\,1883 Venedig), \emph{Komponist}!Ring des Nibelungen@\strich\emph{Der Ring des Nibelungen}|pwk} wurde bei den \emph{Bayreuther Festspielen}\orgindex{Bayreuther Festspiele@Bayreuther Festspiele|pwk}{ }1899 unter der Leitung von Cosima Wagner\pwindex{Wagner, Cosima 25.\,12.\,1837 Como – 1.\,4.\,1930 Bayreuth@\textsc{Wagner, Cosima} (25.\,12.\,1837 Como – 1.\,4.\,1930 Bayreuth)|pwk} aufgeführt.}}}\label{K_L02881-2} abſolvirt. Halbtodt vor
               Arbeit und Anſtrengung. Aber gewaltige Eindrücke. Es iſt unverzeihlich, daß Du noch
               nicht in Bayreuth\oindex{Bayreuth@\textbf{Bayreuth}, \emph{Hauptstadt}|pw}\orgindex{Bayreuther Festspiele@Bayreuther Festspiele|pwv} warſt, und Du{ }ſollteſt es möglich machen, wenigſtens zum zweiten Cyclus\pwindex{Wagner, Richard 22.\,5.\,1813 Leipzig – 13.\,2.\,1883 Venedig@\textsc{Wagner, Richard} (22.\,5.\,1813 Leipzig – 13.\,2.\,1883 Venedig), \emph{Komponist}!Ring des Nibelungen@\strich\emph{Der Ring des Nibelungen}|pwv} im Auguſt{ }\strikeout{her}{ }\label{K_L02881-3v}\edtext{herzukommen}{\lemma{\textnormal{\emph{herzukommen}}}\Cendnote{\textnormal{Dazu kam es nicht. Schnitzler besuchte die \emph{Bayreuther
                     Festspiele}\orgindex{Bayreuther Festspiele@Bayreuther Festspiele|pwk} nie.}}}\label{K_L02881-3}. Dieſes Bayreuth\oindex{Bayreuth@\textbf{Bayreuth}, \emph{Hauptstadt}|pw}\orgindex{Bayreuther Festspiele@Bayreuther Festspiele|pwv} gehört zum Größten i\substVorne{}\textsuperscript{\textcolor{gray}{m}}\substDazwischen{}n\substHinten{}{ }\introOben{}der\introOben{} gegenwärtigen deutſch\oindex{Deutschland@\textbf{Deutschland}|pwv}en Kunſt, und man muß es einmal miterlebt haben.\pend
           
\pstart
           {\pb}Von hier gehe ich nach \textsc{Rennes\oindex{Rennes@\textbf{Rennes}|pw}}, und alle meine Urlaubspläne hängen ab von dem Zeitpunkt, an dem der \label{K_L02881-4v}\edtext{Prozeß\pwindex{Dreyfus, Alfred 9.\,10.\,1859 Mulhouse – 12.\,7.\,1935 Paris@\textsc{Dreyfus, Alfred} (9.\,10.\,1859 Mulhouse – 12.\,7.\,1935 Paris), \emph{Militär}|pwv}}{\lemma{\textnormal{\emph{Prozeß}}}\Cendnote{\textnormal{Gemeint war der neue
                  Kriegsgerichtsprozess in der Affäre Dreyfus\pwindex{Dreyfus, Alfred 9.\,10.\,1859 Mulhouse – 12.\,7.\,1935 Paris@\textsc{Dreyfus, Alfred} (9.\,10.\,1859 Mulhouse – 12.\,7.\,1935 Paris), \emph{Militär}|pwk},
                  der bis in den September fortdauerte. Es kam zum
                  Schuldspruch am 9. 9. 1899 und zur Begnadigung am
                     19. 9. 1899.}}}\label{K_L02881-4} zu Ende iſt. Dauert er, wie
               ich vorausſehe, bis tief in den Auguſt hinein,{ }ſo kann
               ich dann fürs Erſte nicht mehr fort, da \strikeout{i\textcolor{gray}{ch}} im September{ }\textsc{Dr. Mamroth\pwindex{Mamroth, Fedor 21.\,2.\,1851 Breslau – 25.\,6.\,1907 Frankfurt am Main@\textsc{Mamroth, Fedor} (21.\,2.\,1851 Breslau – 25.\,6.\,1907 Frankfurt am Main), \emph{Journalist, Kritiker}|pw}} auf Urlaub geht, den ich vertreten muß. In dieſem Falle würde ich nach \textsc{Mamroths\pwindex{Mamroth, Fedor 21.\,2.\,1851 Breslau – 25.\,6.\,1907 Frankfurt am Main@\textsc{Mamroth, Fedor} (21.\,2.\,1851 Breslau – 25.\,6.\,1907 Frankfurt am Main), \emph{Journalist, Kritiker}|pw}} Rückkehr (wenn die Ereigniſſe
               in der Welt es erlauben u. ich zu dieſem Zeitpunkt nicht etwa nach Grönland\oindex{Grönland@\textbf{Grönland}, \emph{Exterritoriales Gebiet}|pw} muß, um dort über eine Revolution der Eskimos zu
               berichten){ }ſo zwiſchen dem \strikeout{5.}{ }5. und 10. Oktober
               meinen Urlaub {\pb}antreten u. nach Italien\oindex{Italien@\textbf{Italien}|pw} gehen. Die Ausſicht, daß Du \label{K_L02881-5v}\edtext{mitgehſt}{\lemma{\textnormal{\emph{mitgehst}}}\Cendnote{\textnormal{Dazu kam es nicht.}}}\label{K_L02881-5}, iſt entzückend. Vielleicht kommt auch \textsc{Richard\pwindex{Beer-Hofmann, Richard 11.\,7.\,1866 Wien – 26.\,9.\,1945 New York City@\textsc{Beer-Hofmann, Richard} (11.\,7.\,1866 Wien – 26.\,9.\,1945 New York City), \emph{Schriftsteller}|pw}} mit. Defintives aber kann ich Dir erſt nach meiner Rückkehr von \textsc{Rennes\oindex{Rennes@\textbf{Rennes}|pw}}{ }ſagen.\pend
           
\pstart
           Ich wünſche Dir weiter recht viel Sommer-Erholung und recht viel{ }ſonnige Tage, in
               denen keine Schatten umgehen. Komm’ nur hierher\oindex{Bayreuth@\textbf{Bayreuth}, \emph{Hauptstadt}|pwv} und laß’ Dir vom Bayreuth\oindex{Bayreuth@\textbf{Bayreuth}, \emph{Hauptstadt}|pw}er Orcheſter\orgindex{Bayreuther Festspiele@Bayreuther Festspiele|pwv} Freude am Leben und
               Freude an der Kunſt \strikeout{ins} im Herzen entzünden!\pend
           
\pstart
           In Treue {\\[\baselineskip]}Dein \spacefill\mbox{Paul Goldmann}\pend
           \leftskip=0em{}
\pstart
           \noindent{}{\pb}Viele Grüße an Deine Familie, wenn Du bei ihr
                  biſt!\pend
           \selectlanguage{ngerman}\endnumbering\briefempfaengerindex{Schnitzler, Arthur@\textsc{Schnitzler, Arthur}!zzzGoldmann, Paul@\emph{von Paul Goldmann}!1899-07-273@{27. 7. [1899]}|)be}\mylabel{L02881h}  \newcommand{\dateiname}{L02881}\newcommand{\titel}{Paul Goldmann an Arthur Schnitzler, 27. 7. [1899]}\newcommand{\editorInnen}{Martin Anton Müller und Laura Untner}%% latex-leseansicht-abspann.tex
%% Abspann für die Leseansicht.
%% Der Schalter \ifkorrekturansicht ist bereits durch den Vorspann gesetzt.

%% latex-abspann.tex
%% Gemeinsamer Abspann für Korrekturansicht und Leseansicht.
%% Setzt den Schalter \ifkorrekturansicht voraus (gesetzt in den
%% einbindenden Dateien latex-korrekturansicht-abspann.tex bzw.
%% latex-leseansicht-abspann.tex).
%% ---------------------------------------------------------------

\normalsize

% Das esempio-Environment wird nur in der Leseansicht benötigt
\ifkorrekturansicht\else
\newenvironment{esempio}[3]%
{
    \vspace{1.5ex}
    \rlap{\underline{#1}}
    \par
    \setlength{\parindent}{0cm}
    \nopagebreak
    \leftskip=#2cm
    \rightskip=#3cm
}
{
    \par
}
\fi

\doendnotes{C}
\bigskip
\vfill

\clearpage

\footnotesize

\ifkorrekturansicht
  \lohead{\textsc{register}}
\fi

% theindex-Environment neu definieren ohne reledmac
\makeatletter
\renewenvironment{theindex}{%
  \ifkorrekturansicht
    \section*{\indexname}%
  \else
    \subsubsection*{Index der erwähnten Entitäten}%
  \fi
  \setlength{\parindent}{0pt}%
  \setlength{\parskip}{0pt plus 0.3pt}%
  \let\item\@idxitem
}{%
  \ifkorrekturansicht\clearpage\fi
}
\makeatother

\IfFileExists{\jobname-pw.ind}{\input{\jobname-pw.ind}}{}

% Quellenangabe nur in der Leseansicht
\ifkorrekturansicht\else
% Fallback-Definitionen, falls die .tex-Datei \titel etc. nicht gesetzt hat
\providecommand{\titel}{}
\providecommand{\editorInnen}{}
\providecommand{\dateiname}{\jobname}

\vspace{3cm}

\vfill

\footnotesize
\textsc{Quelle}: \titel. Herausgegeben von {\editorInnen}. In: \emph{Arthur Schnitzler: Briefwechsel mit Autorinnen und Autoren}.
 Digitale Edition, https://schnitzler-briefe.acdh.oeaw.ac.at/{\dateiname}.html (Stand \today)
\fi

\end{document}


