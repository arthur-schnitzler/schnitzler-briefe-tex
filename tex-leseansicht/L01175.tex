%% latex-korrekturansicht-vorspann.tex
%% Vorspann für die Korrekturansicht.
%% Lädt die gemeinsame Datei latex-vorspann.tex mit gesetztem Schalter.

\newif\ifkorrekturansicht
\korrekturansichttrue

\input{../tex-inputs/latex-vorspann}


\section[Arthur Schnitzler an Richard Beer-Hofmann, 21. 9. 1901]{L01175 Arthur Schnitzler an Richard Beer-Hofmann, 21. 9. 1901}
\nopagebreak\mylabel{L01175v}
\rehead{ }\normalsize\beginnumbering\briefempfaengerindex{Beer-Hofmann, Richard@\textsc{Beer-Hofmann, Richard}!zzzSchnitzler, Arthur@\emph{von Arthur Schnitzler}!1901-09-211@{21. 9. 1901}|(be}
\toendnotes[C]{\smallbreak\pagebreak[2]}\Standort{YCGL, MSS 31.}
\physDesc{Brief, 1 Blatt, 1 Seite, Umschlag, 294 Zeichen
\newline{}Handschrift: Bleistift, deutsche Kurrent
\newline{}Versand: 1) Rohrpost  2) Stempel: »\nobreak{}\oindex{I., Innere Stadt@\textbf{I., Innere Stadt}, \emph{A.ADM3}|pwk}Wien 1/1, 21 IX 01, 10 40\nobreak{}«.  3) Stempel: »\nobreak{}\oindex{I., Innere Stadt@\textbf{I., Innere Stadt}, \emph{A.ADM3}|pwk}Wien 1/1, 21 IX 01, 10 50N\nobreak{}«. 
\newline{}Ordnung: mit Bleistift von unbekannter Hand datiert: »{\pb}21. 9.« }\toendnotes[C]{\smallbreak}\pstart{}{\pb}Herrn Dr. \textsc{Rich.
                     Beer-Hofmann}\pend{}\pstart{}Wien\oindex{Wien@\textbf{Wien}, \emph{A.ADM2}|pw}\pend{}\pstart{}\textsc{I. Wollzeile 15\oindex{Wollzeile@\textbf{Wollzeile}, \emph{Straße (K.STR)}|pw}.}\pend{}{\bigskip}\vspace{1em}
\pstart
           \raggedleft{}{\pb}Samſtag\pend
           \vspace{0.5em}
\pstart
           lieber Richard, ohne \label{K_L01175-1v}\edtext{\textcolor{gray}{Praejudiz}}{\lemma{\textnormal{\emph{Praejudiz}}}\Cendnote{\textnormal{Vgl. Arthur Schnitzler an Felix Salten, [14. 9. 1901?].
               }}}\label{K_L01175-1} für event.
               Eintritt werde ich heute Abend, am Ende ſchon zum Nachtmahl, jedenfalls aber um
                  10, im \label{K_L01175-2v}\edtext{Club\orgindex{?? [Wiener Club September 1901]@?? [Wiener Club September 1901]|pwv}}{\lemma{\textnormal{\emph{Club}}}\Cendnote{\textnormal{Siehe Arthur Schnitzler an Richard Beer-Hofmann, 6. 9. 1901.
               }}}\label{K_L01175-2} ſein.\pend
           
\pstart
           Herzlichſt{\\[\baselineskip]}Ihr{\\[\baselineskip]}\spacefill\mbox{Arthur}\pend
           \leftskip=0em{}
\pstart
           \noindent{}Guſtav\pwindex{Schwarzkopf, Gustav 07.11.1853 – 13.11.1939@\textsc{Schwarzkopf, Gustav} (07.11.1853 – 13.11.1939), \emph{Schriftsteller/Schriftstellerin}|pw} wohnt nach wie vor Tief. Gr. 23\oindex{Tiefer Graben@\textbf{Tiefer Graben}, \emph{Straße (K.STR)}|pw}; vielleicht iſt er aber in dieſen Tagen in
                  der Brühl\oindex{Bruehl@\textbf{Brühl}, \emph{Tal (N.TAL)}|pw}\pend
           \selectlanguage{ngerman}\endnumbering\briefempfaengerindex{Beer-Hofmann, Richard@\textsc{Beer-Hofmann, Richard}!zzzSchnitzler, Arthur@\emph{von Arthur Schnitzler}!1901-09-211@{21. 9. 1901}|)be}\mylabel{L01175h}  \normalsize

\doendnotes{C}
\bigskip
\vfill

\clearpage

\footnotesize

\lohead{\textsc{register}}

% Definiere theindex-Environment komplett neu ohne reledmac
\makeatletter
\renewenvironment{theindex}{%
  \section*{\indexname}%
  \setlength{\parindent}{0pt}%
  \setlength{\parskip}{0pt plus 0.3pt}%
  \let\item\@idxitem
}{%
  \clearpage
}
\makeatother

\IfFileExists{\jobname-pw.ind}{\input{\jobname-pw.ind}}{}

\end{document}

      