%% latex-leseansicht-vorspann.tex
%% Vorspann für die Leseansicht.
%% Lädt die gemeinsame Datei latex-vorspann.tex mit nicht gesetztem Schalter.

\newif\ifkorrekturansicht
\korrekturansichtfalse

\input{../tex-inputs/latex-vorspann}


         
         \renewcommand{\erwaehntePersonen}{Personen: Hermann Bahr}
         \renewcommand{\erwaehnteOrte}{Orte: Ober Sankt Veit, Veitlissengasse, Wien}
         \renewcommand{\erwaehnteWerke}{
               \section[Arthur Schnitzler an Hermann Bahr, 10. 1. 1907]{ Arthur Schnitzler an Hermann Bahr, 10. 1. 1907}\nopagebreak\mylabel{v}\rehead{ }\begin{ledgroupsized}[t]{13cm}\normalsize\beginnumbering \toendnotes[C]{\smallbreak\pagebreak[2]} \Standort{TMW, HS AM 23381 Ba.}
\physDesc{Telegramm
\newline{}Handschrift einer Schreibkraft: Bleistift, lateinische Kurrent\newline{}Versand: »\noindent{}\textcolor{gray}{\textbf{Aufgenommen von {\dots} auf Ltg. Nr. {\dots} am }}{ }10\textcolor{gray}{\textbf{/}}1 \textcolor{gray}{\textbf{190}}7{ }11 \textcolor{gray}{\textbf{Uhr}} 30 \textcolor{gray}{\textbf{M. {\dots} Mitt. durch {\dots}}}{ / }\textcolor{gray}{\textbf{Nr.}} 765 \textcolor{gray}{\textbf{Taxw.}} /7 \textcolor{gray}{\textbf{(W.{\dots} Ch{\dots}) aufgegeben am}}{ }10/I \textcolor{gray}{\textbf{190}}7{ }\textcolor{gray}{\textbf{um}}{ }9 \textcolor{gray}{\textbf{Uhr}} 50 \textcolor{gray}{\textbf{M.}} V.\textcolor{gray}{\textbf{Mitt.}}« \newline{}Ordnung: Lochung }\buchAbdrucke{\weitereDrucke{1) \emph{10. 1. 1907.} In: Arthur Schnitzler: \emph{The Letters of Arthur Schnitzler to Hermann Bahr}. Edited, annotated, and with an introduction, by Donald G.
                        Daviau. Chapel Hill: \emph{The University of North Carolina Press} 1978, S. 96 (University of North Carolina studies in the Germanic languages
                        and literatures, 89).} \weitereDrucke{2) Hermann Bahr, Arthur Schnitzler: \emph{Briefwechsel, Aufzeichnungen, Dokumente (1891–1931)}. Hg. Kurt Ifkovits und Martin Anton Müller. Göttingen: \emph{Wallstein} 2018, S. 387.} }\pstart{}{\pb}Hermann
                  Bahr\pend{}\pstart{}Obersanktveit\oindex{Ober Sankt Veit@\textbf{Ober Sankt Veit}|pw}\pend{}\pstart{}Veitlissengasse \damage{\textcolor{gray}{Wien}}\oindex{Veitlissengasse@\textbf{Veitlissengasse}|pw}\pend{}{\bigskip}\pstart
           \noindent{}{\pb}Kann ich dich Sonntag
               Vormittag oder anderem Vormittag besuchen? herzlichst\pend
           \pstart \spacefill\mbox{Arthur}\pend{}
         
         \endnumbering\mylabel{h}\end{ledgroupsized}  \newcommand{\dateiname}{L01649}\newcommand{\titel}{Arthur Schnitzler an Hermann Bahr, 10. 1. 1907}\newcommand{\editorInnen}{ Kurt Ifkovits,  Martin Anton Müller}%% latex-leseansicht-abspann.tex
%% Abspann für die Leseansicht.
%% Der Schalter \ifkorrekturansicht ist bereits durch den Vorspann gesetzt.

%% latex-abspann.tex
%% Gemeinsamer Abspann für Korrekturansicht und Leseansicht.
%% Setzt den Schalter \ifkorrekturansicht voraus (gesetzt in den
%% einbindenden Dateien latex-korrekturansicht-abspann.tex bzw.
%% latex-leseansicht-abspann.tex).
%% ---------------------------------------------------------------

\normalsize

% Das esempio-Environment wird nur in der Leseansicht benötigt
\ifkorrekturansicht\else
\newenvironment{esempio}[3]%
{
    \vspace{1.5ex}
    \rlap{\underline{#1}}
    \par
    \setlength{\parindent}{0cm}
    \nopagebreak
    \leftskip=#2cm
    \rightskip=#3cm
}
{
    \par
}
\fi

\doendnotes{C}
\bigskip
\vfill

\clearpage

\footnotesize

\ifkorrekturansicht
  \lohead{\textsc{register}}
\fi

% theindex-Environment neu definieren ohne reledmac
\makeatletter
\renewenvironment{theindex}{%
  \ifkorrekturansicht
    \section*{\indexname}%
  \else
    \subsubsection*{Index der erwähnten Entitäten}%
  \fi
  \setlength{\parindent}{0pt}%
  \setlength{\parskip}{0pt plus 0.3pt}%
  \let\item\@idxitem
}{%
  \ifkorrekturansicht\clearpage\fi
}
\makeatother

\IfFileExists{\jobname-pw.ind}{\input{\jobname-pw.ind}}{}

% Quellenangabe nur in der Leseansicht
\ifkorrekturansicht\else
% Fallback-Definitionen, falls die .tex-Datei \titel etc. nicht gesetzt hat
\providecommand{\titel}{}
\providecommand{\editorInnen}{}
\providecommand{\dateiname}{\jobname}

\vspace{3cm}

\vfill

\footnotesize
\textsc{Quelle}: \titel. Herausgegeben von {\editorInnen}. In: \emph{Arthur Schnitzler: Briefwechsel mit Autorinnen und Autoren}.
 Digitale Edition, https://schnitzler-briefe.acdh.oeaw.ac.at/{\dateiname}.html (Stand \today)
\fi

\end{document}


      