%% latex-korrekturansicht-vorspann.tex
%% Vorspann für die Korrekturansicht.
%% Lädt die gemeinsame Datei latex-vorspann.tex mit gesetztem Schalter.

\newif\ifkorrekturansicht
\korrekturansichttrue

\input{../tex-inputs/latex-vorspann}


\section[Richard Beer-Hofmann an Arthur Schnitzler, 17. 7. 1927]{L02490 Richard Beer-Hofmann an Arthur Schnitzler, 17. 7. 1927}
\nopagebreak\mylabel{L02490v}
\rehead{ }\normalsize\beginnumbering\briefempfaengerindex{Schnitzler, Arthur@\textsc{Schnitzler, Arthur}!zzzBeer-Hofmann, Richard@\emph{von Richard Beer-Hofmann}!1927-07-172@{17. 7. 1927}|(be}
\toendnotes[C]{\smallbreak\pagebreak[2]}\Standort{CUL, Schnitzler, B 8.}
\physDesc{Brief, 1 Blatt, 1 Seite, 98 Zeichen
\newline{}Handschrift: Bleistift, lateinische Kurrent
\newline{}Ordnung: mit Bleistift von unbekannter Hand nummeriert:
                                    »274« }
\pstart
           \raggedleft{}{\pb}17. VII. 27.\pend
           \vspace{0.5em}
\pstart
           Lieber Arthur! Wir kommen gerne. Um 8 Uhr. D\textsuperscript{r}{ }M\pwindex{Menczel, Philipp 09.01.1872 – 26.10.1941@\textsc{Menczel, Philipp} (09.01.1872 – 26.10.1941), \emph{Journalist/Journalistin, Rechtsanwalt/Rechtsanwältin}|pw}. ist schon in Gastein\oindex{Bad Gastein@\textbf{Bad Gastein}, \emph{P.PPLA3}|pw}.\pend
           
\pstart
           Herzlichst{\\[\baselineskip]}Ihr{\\[\baselineskip]}\spacefill\mbox{Richard}\pend
           \leftskip=0em{}\selectlanguage{ngerman}\endnumbering\briefempfaengerindex{Schnitzler, Arthur@\textsc{Schnitzler, Arthur}!zzzBeer-Hofmann, Richard@\emph{von Richard Beer-Hofmann}!1927-07-172@{17. 7. 1927}|)be}\mylabel{L02490h}  \normalsize

\doendnotes{C}
\bigskip
\vfill

\clearpage

\footnotesize

\lohead{\textsc{register}}

% Definiere theindex-Environment komplett neu ohne reledmac
\makeatletter
\renewenvironment{theindex}{%
  \section*{\indexname}%
  \setlength{\parindent}{0pt}%
  \setlength{\parskip}{0pt plus 0.3pt}%
  \let\item\@idxitem
}{%
  \clearpage
}
\makeatother

\IfFileExists{\jobname-pw.ind}{\input{\jobname-pw.ind}}{}

\end{document}

      