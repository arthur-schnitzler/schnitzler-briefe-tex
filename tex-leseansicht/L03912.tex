%% latex-leseansicht-vorspann.tex
%% Vorspann für die Leseansicht.
%% Lädt die gemeinsame Datei latex-vorspann.tex mit nicht gesetztem Schalter.

\newif\ifkorrekturansicht
\korrekturansichtfalse

\input{../tex-inputs/latex-vorspann}


\section[Arthur Schnitzler an Theodor Herzl, 15. 12. 1894]{L03912 Arthur Schnitzler an Theodor Herzl, 15. 12. 1894}
\nopagebreak\mylabel{L03912v}
\rehead{ }\normalsize\beginnumbering\briefempfaengerindex{Herzl, Theodor@\textsc{Herzl, Theodor}!zzzSchnitzler, Arthur@\emph{von Arthur Schnitzler}!1894-12-151@{15. 12. 1894}|(be}
\toendnotes[C]{\smallbreak\pagebreak[2]}
\correspDesc{Versand  durch Arthur Schnitzler am 15. 12. 1894 in Wien
\newline{}Erhalt  durch Theodor Herzl in Wien}\toendnotes[C]{\smallbreak}
\Standort{Jerusalem, Central Zionist Archives, H1:1924-17.}
\physDesc{,  Blätter,  Seiten
\newline{}Handschrift: , deutsche Kurrent}
\buchAbdrucke{\weitereDrucke{Arthur Schnitzler: \emph{Briefe 1875–1912}. Herausgegeben von Therese Nickl und Heinrich Schnitzler. Frankfurt am Main: \emph{S. Fischer} 1981, S. 243.} }\toendnotes[C]{\smallbreak}
\pstart{}{\pb}Lieber Freund,\pend\vspace{0.5em}
\pstart
           die Adreſſe iſt: \textsc{Friedrich Schick\pwindex{Schik, Friedrich *~6.\,9.\,1857 Wien@\textsc{Schik, Friedrich} (*~6.\,9.\,1857 Wien), \emph{Notar, Journalist, Dramaturg}|pw}}, Wien III\oindex{III., Landstraße@\textbf{III., Landstraße}, \emph{Verwaltungsgebiet}|pw}{ }Reiſnerſtraße 35\oindex{Wien@\textbf{Wien}!III., Landstraße@\textbf{III., Landstraße}!Reisnerstraße 35@\textbf{Reisnerstraße 35}, \emph{Wohngebäude}|pw}. Ich hoffe auch durch \textsc{Sch.\pwindex{Schik, Friedrich *~6.\,9.\,1857 Wien@\textsc{Schik, Friedrich} (*~6.\,9.\,1857 Wien), \emph{Notar, Journalist, Dramaturg}|pw}} einen ſchreibmaſchinkundigen Abſchreiber zu bekommen, und \strikeout{dahin},
               daſs alles ſo raſch als möglich beſorgt wird, können Sie verſichert{ }ſein.
               Aviſiren Sie mich bitte 2 Tage vor Eintreffen des \textsc{Mscr.\pwindex{Herzl, Theodor 2.\,5.\,1860 Budapest – 3.\,7.\,1904 Edlach@\textsc{Herzl, Theodor} (2.\,5.\,1860 Budapest – 3.\,7.\,1904 Edlach), \emph{Schriftsteller, Journalist}!neue Ghetto. Schauspiel in vier Acten@\strich\emph{Das neue Ghetto. Schauspiel in vier Acten}|pwv}} ſo dſs der Schreiber bereit
               iſt.–\pend
           
\pstart
           {\pb}Die \textsc{Glosse\pwindex{Herzl, Theodor 2.\,5.\,1860 Budapest – 3.\,7.\,1904 Edlach@\textsc{Herzl, Theodor} (2.\,5.\,1860 Budapest – 3.\,7.\,1904 Edlach), \emph{Schriftsteller, Journalist}!Glosse. Lustspiel in einem Act@\strich\emph{Die Glosse. Lustspiel in einem Act}|pw}} habe ich erhalten und habe eine große Freude gehabt, ſie wiederzuleſen. Oder
               wiederzuhören; denn ich habe Ihre Stimme noch im Ohr gehabt. Nun haben Sie nichts
               mehr damit verſucht?– Ein vornehmes Theater hat doch eigentlich abſolut keinen Grund,
               dieſes Stück nicht aufzuführen? Riskirt wird gar nichts, der literariſche Erfolg
               iſt ſicher und ich muſs ſagen, daſs auch die Bühnen{\pb}wirkſamkeit, wo gute Schauſpieler{ }ſind, kaum in Frage ſteht. Haben Sie ſchon beim
                  Deutſchen Theater\orgindex{Deutsches Theater Berlin@Deutsches Theater Berlin|pw} in Berlin\oindex{Berlin@\textbf{Berlin}, \emph{Hauptstadt}|pw} verſucht? Hübſch wär’ es, wenn – beide\pwindex{Herzl, Theodor 2.\,5.\,1860 Budapest – 3.\,7.\,1904 Edlach@\textsc{Herzl, Theodor} (2.\,5.\,1860 Budapest – 3.\,7.\,1904 Edlach), \emph{Schriftsteller, Journalist}!neue Ghetto. Schauspiel in vier Acten@\strich\emph{Das neue Ghetto. Schauspiel in vier Acten}|pwv}\pwindex{Herzl, Theodor 2.\,5.\,1860 Budapest – 3.\,7.\,1904 Edlach@\textsc{Herzl, Theodor} (2.\,5.\,1860 Budapest – 3.\,7.\,1904 Edlach), \emph{Schriftsteller, Journalist}!Glosse. Lustspiel in einem Act@\strich\emph{Die Glosse. Lustspiel in einem Act}|pwv} am ſelben Abend gegeben
               würden – ob man da wohl den Verfaſſer des anonymen\pwindex{Herzl, Theodor 2.\,5.\,1860 Budapest – 3.\,7.\,1904 Edlach@\textsc{Herzl, Theodor} (2.\,5.\,1860 Budapest – 3.\,7.\,1904 Edlach), \emph{Schriftsteller, Journalist}!neue Ghetto. Schauspiel in vier Acten@\strich\emph{Das neue Ghetto. Schauspiel in vier Acten}|pw} erreicht? – Seien Sie herzlich gegrüßt und verfügen
               Sie ſtets über mich\pend
           
\pstart
           Ihr allzeitergebener{\\[\baselineskip]}\spacefill\mbox{ArthSchnitzler}\pend
           \leftskip=0em{}
\pstart
           Wien\oindex{Wien@\textbf{Wien}, \emph{Verwaltungsgebiet}|pw}{ }15. 12. 94\pend
           \selectlanguage{ngerman}\endnumbering\briefempfaengerindex{Herzl, Theodor@\textsc{Herzl, Theodor}!zzzSchnitzler, Arthur@\emph{von Arthur Schnitzler}!1894-12-151@{15. 12. 1894}|)be}\mylabel{L03912h}
\begin{anhang}
\end{anhang}\newcommand{\dateiname}{L03912}\newcommand{\titel}{Arthur Schnitzler an Theodor Herzl, 15. 12. 1894}\newcommand{\editorInnen}{Herausgegeben von Jahnke, SelmaMüller, Martin Anton}%% latex-leseansicht-abspann.tex
%% Abspann für die Leseansicht.
%% Der Schalter \ifkorrekturansicht ist bereits durch den Vorspann gesetzt.

%% latex-abspann.tex
%% Gemeinsamer Abspann für Korrekturansicht und Leseansicht.
%% Setzt den Schalter \ifkorrekturansicht voraus (gesetzt in den
%% einbindenden Dateien latex-korrekturansicht-abspann.tex bzw.
%% latex-leseansicht-abspann.tex).
%% ---------------------------------------------------------------

\normalsize

% Das esempio-Environment wird nur in der Leseansicht benötigt
\ifkorrekturansicht\else
\newenvironment{esempio}[3]%
{
    \vspace{1.5ex}
    \rlap{\underline{#1}}
    \par
    \setlength{\parindent}{0cm}
    \nopagebreak
    \leftskip=#2cm
    \rightskip=#3cm
}
{
    \par
}
\fi

\doendnotes{C}
\bigskip
\vfill

\clearpage

\footnotesize

\ifkorrekturansicht
  \lohead{\textsc{register}}
\fi

% theindex-Environment neu definieren ohne reledmac
\makeatletter
\renewenvironment{theindex}{%
  \ifkorrekturansicht
    \section*{\indexname}%
  \else
    \subsubsection*{Index der erwähnten Entitäten}%
  \fi
  \setlength{\parindent}{0pt}%
  \setlength{\parskip}{0pt plus 0.3pt}%
  \let\item\@idxitem
}{%
  \ifkorrekturansicht\clearpage\fi
}
\makeatother

\IfFileExists{\jobname-pw.ind}{\input{\jobname-pw.ind}}{}

% Quellenangabe nur in der Leseansicht
\ifkorrekturansicht\else
% Fallback-Definitionen, falls die .tex-Datei \titel etc. nicht gesetzt hat
\providecommand{\titel}{}
\providecommand{\editorInnen}{}
\providecommand{\dateiname}{\jobname}

\vspace{3cm}

\vfill

\footnotesize
\textsc{Quelle}: \titel. Herausgegeben von {\editorInnen}. In: \emph{Arthur Schnitzler: Briefwechsel mit Autorinnen und Autoren}.
 Digitale Edition, https://schnitzler-briefe.acdh.oeaw.ac.at/{\dateiname}.html (Stand \today)
\fi

\end{document}


