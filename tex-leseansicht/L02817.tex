%% latex-korrekturansicht-vorspann.tex
%% Vorspann für die Korrekturansicht.
%% Lädt die gemeinsame Datei latex-vorspann.tex mit gesetztem Schalter.

\newif\ifkorrekturansicht
\korrekturansichttrue

\input{../tex-inputs/latex-vorspann}


\section[ Paul Goldmann an Arthur Schnitzler, 13. 7. {[}1897{]}]{L02817 Paul Goldmann an Arthur Schnitzler, 13. 7. {[}1897{]}}
\nopagebreak\mylabel{L02817v}
\rehead{ }\normalsize\beginnumbering\briefempfaengerindex{Schnitzler, Arthur@\textsc{Schnitzler, Arthur}!zzzGoldmann, Paul@\emph{von Paul Goldmann}!1897-07-132@{13. 7. {[}1897{]}}|(be}
\toendnotes[C]{\smallbreak\pagebreak[2]}\Standort{DLA, A:Schnitzler, HS.NZ85.1.3167.}
\physDesc{Brief, 1 Blatt, 3 Seiten, 1239 Zeichen
\newline{}Handschrift: blaue Tinte, deutsche Kurrent
\newline{}Schnitzler: mit Bleistift das Jahr »97« vermerkt }\toendnotes[C]{\smallbreak}
\pstart
           {\pb}\textcolor{gray}{\textbf{\textbf{Frankfurter Zeitung\orgindex{Frankfurter Zeitung@Frankfurter Zeitung|pw}}}}\pend
           
\pstart
           \textcolor{gray}{\textbf{(\begin{otherlanguage}{french}Gazette de Francfort\end{otherlanguage}\orgindex{Frankfurter Zeitung@Frankfurter Zeitung|pw}).}}\pend
           
\pstart
           \textcolor{gray}{\textbf{\textbf{\begin{otherlanguage}{french}Fondateur M.\end{otherlanguage}{ }L. Sonnemann\pwindex{Sonnemann, Leopold 1831-10-29 – 1909-10-30@\textsc{Sonnemann, Leopold} (1831-10-29 – 1909-10-30), \emph{Journalist/Journalistin, Herausgeber/Herausgeberin}|pw}.}}}\pend
           
\pstart
           \begin{otherlanguage}{french}\textcolor{gray}{\textbf{Journal politique, financier,}}\end{otherlanguage}\hfill \textsc{Paris\oindex{Paris@\textbf{Paris}, \emph{P.PPLC}|pw}}, 13. Juli.\pend
           
\pstart
           \begin{otherlanguage}{french}\textcolor{gray}{\textbf{commercial et littéraire.}}\end{otherlanguage}\pend
           
\pstart
           \begin{otherlanguage}{french}\textcolor{gray}{\textbf{\textbf{Paraissant trois fois par jour.}}}\end{otherlanguage}\pend
           
\pstart
           \begin{otherlanguage}{french}\textcolor{gray}{\textbf{\textbf{Bureau à Paris\oindex{Paris@\textbf{Paris}, \emph{P.PPLC}|pw}}}}\end{otherlanguage}\pend
           
\pstart
           \begin{otherlanguage}{french}\textcolor{gray}{\textbf{\textbf{10 \so{Rue de la Bourse}\oindex{rue de la Bourse@\textbf{rue de la Bourse}, \emph{Straße (K.STR)}|pw}.}}}\end{otherlanguage}\pend
           
\pstart\center{}Mein lieber Freund,\pend\vspace{0.5em}
\pstart
           Eine ausführliche Beantwortung Deiner lieben Briefe behalte ich mir für demnächſt
               vor. Heut nur in aller Eile:\pend
           
\pstart
           Ich habe geſtern von der Redaction\orgindex{Frankfurter Zeitung@Frankfurter Zeitung|pwv} meinen Urlaub für Anfang Auguſt verlangt. Ob ich ihn bekommen werde und ob man mich
               nicht zwingen wird, bis Ende Auguſt (während der \label{K_L02817-1v}\edtext{Reiſe des Präſident\pwindex{Faure, Felix 1841-01-30 – 1899-02-16@\textsc{Faure, Félix} (1841-01-30 – 1899-02-16), \emph{Politiker/Politikerin, Präsident/Präsidentin}|pwv}en}{\lemma{\textnormal{\emph{Reiſe des Präſidenten}}}\Cendnote{\textnormal{Siehe Paul Goldmann an Arthur Schnitzler, 15. 6. [1897]. }}}\label{K_L02817-1} der Republik\oindex{Frankreich@\textbf{Frankreich}, \emph{A.PCLI}|pwv}) hierzubleiben, weiß
               ich nicht. Jedenfalls habe ich mir in \label{K_L02817-2v}\edtext{\textsc{Bayreuth\oindex{Bayreuth@\textbf{Bayreuth}, \emph{P.PPLA2}|pw}\orgindex{Bayreuther Festspiele@Bayreuther Festspiele|pwv}}}{\lemma{\textnormal{\emph{Bayreuth}}}\Cendnote{\textnormal{Siehe Paul Goldmann an Arthur Schnitzler, 15. 6. [1897]. }}}\label{K_L02817-2} Sitze
               beſtellt und deren {\pb}zwei für die \textsc{Parsifal\pwindex{Parsifal@\emph{Parsifal}|pw}}-Aufführung\orgindex{Bayreuther Festspiele@Bayreuther Festspiele|pwv} vom 11. Auguſt bekommen. Wenn
               Du nicht mitkommen kannſt, ſo frage doch den \label{K_L02817-3v}\edtext{\textsc{Richard\pwindex{Beer-Hofmann, Richard 1866-07-11 – 1945-09-26@\textsc{Beer-Hofmann, Richard} (1866-07-11 – 1945-09-26), \emph{Schriftsteller/Schriftstellerin}|pw}}}{\lemma{\textnormal{\emph{Richard}}}\Cendnote{\textnormal{Schnitzler hatte Richard Beer-Hofmann\pwindex{Beer-Hofmann, Richard 1866-07-11 – 1945-09-26@\textsc{Beer-Hofmann, Richard} (1866-07-11 – 1945-09-26), \emph{Schriftsteller/Schriftstellerin}|pwk} bereits wegen einer früheren
                  Vorstellung gefragt, worauf sich Beer-Hofmann\pwindex{Beer-Hofmann, Richard 1866-07-11 – 1945-09-26@\textsc{Beer-Hofmann, Richard} (1866-07-11 – 1945-09-26), \emph{Schriftsteller/Schriftstellerin}|pwk} aber nicht festlegen wollte (vgl. Arthur Schnitzler an Richard Beer-Hofmann, 12. 6. 1897 und Richard Beer-Hofmann an Arthur Schnitzler, 13. 6. 1897). Einem Brief Goldmanns\pwindex{Goldmann, Paul 31.01.1865 – 25.09.1935@\textsc{Goldmann, Paul} (31.01.1865 – 25.09.1935), \emph{Schriftsteller/Schriftstellerin, Journalist/Journalistin}|pwk} an Beer-Hofmann\pwindex{Beer-Hofmann, Richard 1866-07-11 – 1945-09-26@\textsc{Beer-Hofmann, Richard} (1866-07-11 – 1945-09-26), \emph{Schriftsteller/Schriftstellerin}|pwk} vom
                     24. 7. {[}1897{]} (\emph{Houghton Library}\orgindex{Houghton Library@Houghton Library|pwk},
                     Harvard (Signatur 825.978)) ist zu entnehmen, dass
                  er hoffte, ihn bereits in München\oindex{Muenchen@\textbf{München}, \emph{P.PPLA}|pwk} zu
                  treffen. Das schließt nicht nur eine Teilnahme in Bayreuth\oindex{Bayreuth@\textbf{Bayreuth}, \emph{P.PPLA2}|pwk}\orgindex{Bayreuther Festspiele@Bayreuther Festspiele|pwkv} aus, sondern lässt vermuten, dass auch dieses Treffen nicht
                  stattgefunden hat. Beer-Hofmanns\pwindex{Beer-Hofmann, Richard 1866-07-11 – 1945-09-26@\textsc{Beer-Hofmann, Richard} (1866-07-11 – 1945-09-26), \emph{Schriftsteller/Schriftstellerin}|pwk} »Daten«
                  ist außerdem zu entnehmen, dass er 1897 nicht ins Ausland
                  reiste (vgl. Eugene Weber: \emph{Richard Beer-Hofmann:
                        Daten}. In: \emph{Modern Austrian Literature},
                     Jg. 17, 1984, Nr. 2, S. 13–42, hier:
                  S. 22).}}}\label{K_L02817-3}, ob er nicht den zweiten Sitz benutzen will? Er müßte mir
               aber \uuline{ſofort} antworten, da ich bis 20. Juli Beſcheid ſagen muß. Ginge ich nun nach Bayreuth\oindex{Bayreuth@\textbf{Bayreuth}, \emph{P.PPLA2}|pw}, was ſollte ich dann von 11 bis 20. Auguſt
               anfangen, ehe Du nach \label{K_L02817-4v}\edtext{\textsc{Muenchen\oindex{Muenchen@\textbf{München}, \emph{P.PPLA}|pw}}}{\lemma{\textnormal{\emph{Muenchen}}}\Cendnote{\textnormal{Schnitzler verreiste im
                     Sommer 1897 nicht nach München\oindex{Muenchen@\textbf{München}, \emph{P.PPLA}|pwk}.}}}\label{K_L02817-4} kommen kannſt? Auch liegt es mir daran, möglichſt viel
               Zeit in guter Luft, im Gebirge zu verbringen, nicht in der großen Stadt. {\pb}Wäreſt Du nicht für Süd-Tirol\oindex{Suedtirol@\textbf{Südtirol}, \emph{A.ADM2}|pw} zu haben? Das iſt doch das herrlichſte Land\oindex{Suedtirol@\textbf{Südtirol}, \emph{A.ADM2}|pwv} der Welt, und ich begreife nicht, daß
               Ihr das ſo wenig mögt.\pend
           
\pstart
           Sobald ich von meiner Redaction\orgindex{Frankfurter Zeitung@Frankfurter Zeitung|pwv}
               Beſcheid habe, ſchreibe ich Dir.\pend
           
\pstart
           Viele treue Grüße!\pend
           
\pstart
           Dein {\\[\baselineskip]}\spacefill\mbox{Paul Goldmann}\pend
           \leftskip=0em{}
\pstart
           \noindent{}Ich habe nicht nach \label{K_L02817-5v}\edtext{\textsc{Andermatt\oindex{Andermatt@\textbf{Andermatt}, \emph{P.PPL}|pw}\pwindex{Reinhard, Marie 1871-03-13 – 1899-03-18@\textsc{Reinhard, Marie} (1871-03-13 – 1899-03-18), \emph{Gesangspädagoge/Gesangspädagogin}|pwv}}}{\lemma{\textnormal{\emph{Andermatt}}}\Cendnote{\textnormal{an die sich dort aufhaltende Marie Reinhard\pwindex{Reinhard, Marie 1871-03-13 – 1899-03-18@\textsc{Reinhard, Marie} (1871-03-13 – 1899-03-18), \emph{Gesangspädagoge/Gesangspädagogin}|pwk}, siehe Paul Goldmann an Arthur Schnitzler, 2. 7. [1897].}}}\label{K_L02817-5} ſchreiben
                  können, weil ich nicht weiß, wie ich adreſſiren ſoll. Soll ich »\textsc{Madame}« ſchreiben? Und welchen Namen?\pend
           \selectlanguage{ngerman}\endnumbering\briefempfaengerindex{Schnitzler, Arthur@\textsc{Schnitzler, Arthur}!zzzGoldmann, Paul@\emph{von Paul Goldmann}!1897-07-132@{13. 7. {[}1897{]}}|)be}\mylabel{L02817h}  \normalsize

\doendnotes{C}
\bigskip
\vfill

\clearpage

\footnotesize

\lohead{\textsc{register}}

% Definiere theindex-Environment komplett neu ohne reledmac
\makeatletter
\renewenvironment{theindex}{%
  \section*{\indexname}%
  \setlength{\parindent}{0pt}%
  \setlength{\parskip}{0pt plus 0.3pt}%
  \let\item\@idxitem
}{%
  \clearpage
}
\makeatother

\IfFileExists{\jobname-pw.ind}{\input{\jobname-pw.ind}}{}

\end{document}

      