%% latex-leseansicht-vorspann.tex
%% Vorspann für die Leseansicht.
%% Lädt die gemeinsame Datei latex-vorspann.tex mit nicht gesetztem Schalter.

\newif\ifkorrekturansicht
\korrekturansichtfalse

\input{../tex-inputs/latex-vorspann}


\section[ Paul Goldmann an Arthur Schnitzler, 13. 7. [1897]]{L02817 Paul Goldmann an Arthur Schnitzler,  13. 7. [1897]}
\nopagebreak\mylabel{L02817v}
\rehead{ }\normalsize\beginnumbering\briefempfaengerindex{Schnitzler, Arthur@\textsc{Schnitzler, Arthur}!zzzGoldmann, Paul@\emph{von Paul Goldmann}!1897-07-132@{13. 7. [1897]}|(be}
\toendnotes[C]{\smallbreak\pagebreak[2]}
\correspDesc{Versand  durch Paul Goldmann am 13. 7. [1897] in Paris
\newline{}Erhalt  durch Arthur Schnitzler im Zeitraum [14. 7. 1897
                  – 18. 7. 1897?] in Bad Ischl}\toendnotes[C]{\smallbreak}
\Standort{DLA, A:Schnitzler, HS.NZ85.1.3167.}
\physDesc{Brief, 1 Blatt, 3 Seiten, 1239 Zeichen
\newline{}Handschrift: blaue Tinte, deutsche Kurrent
\newline{}Schnitzler: mit Bleistift das Jahr »97« vermerkt }\toendnotes[C]{\smallbreak}
\pstart
           {\pb}\textcolor{gray}{\textbf{\textbf{Frankfurter Zeitung\orgindex{Frankfurter Zeitung@Frankfurter Zeitung|pw}}}}\pend
           
\pstart
           \textcolor{gray}{\textbf{(\begin{otherlanguage}{french}Gazette de Francfort\end{otherlanguage}\orgindex{Frankfurter Zeitung@Frankfurter Zeitung|pw}).}}\pend
           
\pstart
           \textcolor{gray}{\textbf{\textbf{\begin{otherlanguage}{french}Fondateur M.\end{otherlanguage}{ }L. Sonnemann\pwindex{Sonnemann, Leopold 29.\,10.\,1831 Höchberg – 30.\,10.\,1909 Frankfurt am Main@\textsc{Sonnemann, Leopold} (29.\,10.\,1831 Höchberg – 30.\,10.\,1909 Frankfurt am Main), \emph{Journalist, Herausgeber}|pw}.}}}\pend
           
\pstart
           \begin{otherlanguage}{french}\textcolor{gray}{\textbf{Journal politique, financier,}}\end{otherlanguage}\hfill \textsc{Paris\oindex{Paris@\textbf{Paris}, \emph{Hauptstadt}|pw}}, 13. Juli.\pend
           
\pstart
           \begin{otherlanguage}{french}\textcolor{gray}{\textbf{commercial et littéraire.}}\end{otherlanguage}\pend
           
\pstart
           \begin{otherlanguage}{french}\textcolor{gray}{\textbf{\textbf{Paraissant trois fois par jour.}}}\end{otherlanguage}\pend
           
\pstart
           \begin{otherlanguage}{french}\textcolor{gray}{\textbf{\textbf{Bureau à Paris\oindex{Paris@\textbf{Paris}, \emph{Hauptstadt}|pw}}}}\end{otherlanguage}\pend
           
\pstart
           \begin{otherlanguage}{french}\textcolor{gray}{\textbf{\textbf{10 \so{Rue de la Bourse}\oindex{rue de la Bourse@\textbf{rue de la Bourse}, \emph{Straße}|pw}.}}}\end{otherlanguage}\pend
           
\pstart\center{}Mein lieber Freund,\pend\vspace{0.5em}
\pstart
           Eine ausführliche Beantwortung Deiner lieben Briefe behalte ich mir für demnächſt
               vor. Heut nur in aller Eile:\pend
           
\pstart
           Ich habe geſtern von der Redaction\orgindex{Frankfurter Zeitung@Frankfurter Zeitung|pwv} meinen Urlaub für Anfang Auguſt verlangt. Ob ich ihn bekommen werde und ob man mich
               nicht zwingen wird, bis Ende Auguſt (während der \label{K_L02817-1v}\edtext{Reiſe des Präſident\pwindex{Faure, Félix 30.\,1.\,1841 Paris – 16.\,2.\,1899 ebd.@\textsc{Faure, Félix} (30.\,1.\,1841 Paris – 16.\,2.\,1899 ebd.), \emph{Politiker, Präsident}|pwv}en}{\lemma{\textnormal{\emph{Reise des Präsidenten}}}\Cendnote{\textnormal{Siehe XXXX Auszeichnungsfehler: Dokument L02814 nicht gefunden. }}}\label{K_L02817-1} der Republik\oindex{Frankreich@\textbf{Frankreich}|pwv}) hierzubleiben, weiß
               ich nicht. Jedenfalls habe ich mir in \label{K_L02817-2v}\edtext{\textsc{Bayreuth\oindex{Bayreuth@\textbf{Bayreuth}, \emph{Hauptstadt}|pw}\orgindex{Bayreuther Festspiele@Bayreuther Festspiele|pwv}}}{\lemma{\textnormal{\emph{Bayreuth}}}\Cendnote{\textnormal{Siehe XXXX Auszeichnungsfehler: Dokument L02814 nicht gefunden. }}}\label{K_L02817-2} Sitze
               beſtellt und deren {\pb}zwei für die \textsc{Parsifal\pwindex{\textcolor{red}{\textsuperscript{XXXX indx1}}!Parsifal@\strich\emph{Parsifal}|pw}}-Aufführung\orgindex{Bayreuther Festspiele@Bayreuther Festspiele|pwv} vom 11. Auguſt bekommen. Wenn
               Du nicht mitkommen kannſt,{ }ſo frage doch den \label{K_L02817-3v}\edtext{\textsc{Richard\pwindex{Beer-Hofmann, Richard 11.\,7.\,1866 Wien – 26.\,9.\,1945 New York City@\textsc{Beer-Hofmann, Richard} (11.\,7.\,1866 Wien – 26.\,9.\,1945 New York City), \emph{Schriftsteller}|pw}}}{\lemma{\textnormal{\emph{Richard}}}\Cendnote{\textnormal{Schnitzler hatte Richard Beer-Hofmann\pwindex{Beer-Hofmann, Richard 11.\,7.\,1866 Wien – 26.\,9.\,1945 New York City@\textsc{Beer-Hofmann, Richard} (11.\,7.\,1866 Wien – 26.\,9.\,1945 New York City), \emph{Schriftsteller}|pwk} bereits wegen einer früheren
                  Vorstellung gefragt, worauf sich Beer-Hofmann\pwindex{Beer-Hofmann, Richard 11.\,7.\,1866 Wien – 26.\,9.\,1945 New York City@\textsc{Beer-Hofmann, Richard} (11.\,7.\,1866 Wien – 26.\,9.\,1945 New York City), \emph{Schriftsteller}|pwk} aber nicht festlegen wollte (vgl. XXXX Auszeichnungsfehler: Dokument L00685 nicht gefunden und XXXX Auszeichnungsfehler: Dokument L00686 nicht gefunden). Einem Brief Goldmanns\pwindex{Goldmann, Paul 31.\,1.\,1865 Breslau – 25.\,9.\,1935 Wien@\textsc{Goldmann, Paul} (31.\,1.\,1865 Breslau – 25.\,9.\,1935 Wien), \emph{Schriftsteller, Journalist}|pwk} an Beer-Hofmann\pwindex{Beer-Hofmann, Richard 11.\,7.\,1866 Wien – 26.\,9.\,1945 New York City@\textsc{Beer-Hofmann, Richard} (11.\,7.\,1866 Wien – 26.\,9.\,1945 New York City), \emph{Schriftsteller}|pwk} vom
                     24. 7. [1897] (\emph{Houghton Library}\orgindex{Houghton Library@Houghton Library|pwk},
                     Harvard (Signatur 825.978)) ist zu entnehmen, dass
                  er hoffte, ihn bereits in München\oindex{München@\textbf{München}|pwk} zu
                  treffen. Das schließt nicht nur eine Teilnahme in Bayreuth\oindex{Bayreuth@\textbf{Bayreuth}, \emph{Hauptstadt}|pwk}\orgindex{Bayreuther Festspiele@Bayreuther Festspiele|pwkv} aus, sondern lässt vermuten, dass auch dieses Treffen nicht
                  stattgefunden hat. Beer-Hofmanns\pwindex{Beer-Hofmann, Richard 11.\,7.\,1866 Wien – 26.\,9.\,1945 New York City@\textsc{Beer-Hofmann, Richard} (11.\,7.\,1866 Wien – 26.\,9.\,1945 New York City), \emph{Schriftsteller}|pwk} »Daten«
                  ist außerdem zu entnehmen, dass er 1897 nicht ins Ausland
                  reiste (vgl. Eugene Weber: \emph{Richard Beer-Hofmann:
                        Daten}. In: \emph{Modern Austrian Literature},
                     Jg. 17, 1984, Nr. 2, S. 13–42, hier:
                  S. 22).}}}\label{K_L02817-3}, ob er nicht den zweiten Sitz benutzen will? Er müßte mir
               aber \uuline{ſofort} antworten, da ich bis 20. Juli Beſcheid{ }ſagen muß. Ginge ich nun nach Bayreuth\oindex{Bayreuth@\textbf{Bayreuth}, \emph{Hauptstadt}|pw}, was{ }ſollte ich dann von 11 bis 20. Auguſt
               anfangen, ehe Du nach \label{K_L02817-4v}\edtext{\textsc{Muenchen\oindex{München@\textbf{München}|pw}}}{\lemma{\textnormal{\emph{Muenchen}}}\Cendnote{\textnormal{Schnitzler verreiste im
                     Sommer 1897 nicht nach München\oindex{München@\textbf{München}|pwk}.}}}\label{K_L02817-4} kommen kannſt? Auch liegt es mir daran, möglichſt viel
               Zeit in guter Luft, im Gebirge zu verbringen, nicht in der großen Stadt. {\pb}Wäreſt Du nicht für Süd-Tirol\oindex{Südtirol@\textbf{Südtirol}, \emph{Verwaltungsgebiet}|pw} zu haben? Das iſt doch das herrlichſte Land\oindex{Südtirol@\textbf{Südtirol}, \emph{Verwaltungsgebiet}|pwv} der Welt, und ich begreife nicht, daß
               Ihr das{ }ſo wenig mögt.\pend
           
\pstart
           Sobald ich von meiner Redaction\orgindex{Frankfurter Zeitung@Frankfurter Zeitung|pwv}
               Beſcheid habe,{ }ſchreibe ich Dir.\pend
           
\pstart
           Viele treue Grüße!\pend
           
\pstart
           Dein {\\[\baselineskip]}\spacefill\mbox{Paul Goldmann}\pend
           \leftskip=0em{}
\pstart
           \noindent{}Ich habe nicht nach \label{K_L02817-5v}\edtext{\textsc{Andermatt\oindex{Andermatt@\textbf{Andermatt}|pw}\pwindex{Reinhard, Marie 13.\,3.\,1871 Wien – 18.\,3.\,1899 ebd.@\textsc{Reinhard, Marie} (13.\,3.\,1871 Wien – 18.\,3.\,1899 ebd.), \emph{Gesangspädagogin}|pwv}}}{\lemma{\textnormal{\emph{Andermatt}}}\Cendnote{\textnormal{an die sich dort aufhaltende Marie Reinhard\pwindex{Reinhard, Marie 13.\,3.\,1871 Wien – 18.\,3.\,1899 ebd.@\textsc{Reinhard, Marie} (13.\,3.\,1871 Wien – 18.\,3.\,1899 ebd.), \emph{Gesangspädagogin}|pwk}, siehe XXXX Auszeichnungsfehler: Dokument L02816 nicht gefunden.}}}\label{K_L02817-5}{ }ſchreiben
                  können, weil ich nicht weiß, wie ich adreſſiren{ }ſoll. Soll ich »\textsc{Madame}«{ }ſchreiben? Und welchen Namen?\pend
           \selectlanguage{ngerman}\endnumbering\briefempfaengerindex{Schnitzler, Arthur@\textsc{Schnitzler, Arthur}!zzzGoldmann, Paul@\emph{von Paul Goldmann}!1897-07-132@{13. 7. [1897]}|)be}\mylabel{L02817h}  \newcommand{\dateiname}{L02817}\newcommand{\titel}{Paul Goldmann an Arthur Schnitzler, 13. 7. [1897]}\newcommand{\editorInnen}{Martin Anton Müller und Laura Untner}%% latex-leseansicht-abspann.tex
%% Abspann für die Leseansicht.
%% Der Schalter \ifkorrekturansicht ist bereits durch den Vorspann gesetzt.

%% latex-abspann.tex
%% Gemeinsamer Abspann für Korrekturansicht und Leseansicht.
%% Setzt den Schalter \ifkorrekturansicht voraus (gesetzt in den
%% einbindenden Dateien latex-korrekturansicht-abspann.tex bzw.
%% latex-leseansicht-abspann.tex).
%% ---------------------------------------------------------------

\normalsize

% Das esempio-Environment wird nur in der Leseansicht benötigt
\ifkorrekturansicht\else
\newenvironment{esempio}[3]%
{
    \vspace{1.5ex}
    \rlap{\underline{#1}}
    \par
    \setlength{\parindent}{0cm}
    \nopagebreak
    \leftskip=#2cm
    \rightskip=#3cm
}
{
    \par
}
\fi

\doendnotes{C}
\bigskip
\vfill

\clearpage

\footnotesize

\ifkorrekturansicht
  \lohead{\textsc{register}}
\fi

% theindex-Environment neu definieren ohne reledmac
\makeatletter
\renewenvironment{theindex}{%
  \ifkorrekturansicht
    \section*{\indexname}%
  \else
    \subsubsection*{Index der erwähnten Entitäten}%
  \fi
  \setlength{\parindent}{0pt}%
  \setlength{\parskip}{0pt plus 0.3pt}%
  \let\item\@idxitem
}{%
  \ifkorrekturansicht\clearpage\fi
}
\makeatother

\IfFileExists{\jobname-pw.ind}{\input{\jobname-pw.ind}}{}

% Quellenangabe nur in der Leseansicht
\ifkorrekturansicht\else
% Fallback-Definitionen, falls die .tex-Datei \titel etc. nicht gesetzt hat
\providecommand{\titel}{}
\providecommand{\editorInnen}{}
\providecommand{\dateiname}{\jobname}

\vspace{3cm}

\vfill

\footnotesize
\textsc{Quelle}: \titel. Herausgegeben von {\editorInnen}. In: \emph{Arthur Schnitzler: Briefwechsel mit Autorinnen und Autoren}.
 Digitale Edition, https://schnitzler-briefe.acdh.oeaw.ac.at/{\dateiname}.html (Stand \today)
\fi

\end{document}


