%% latex-leseansicht-vorspann.tex
%% Vorspann für die Leseansicht.
%% Lädt die gemeinsame Datei latex-vorspann.tex mit nicht gesetztem Schalter.

\newif\ifkorrekturansicht
\korrekturansichtfalse

\input{../tex-inputs/latex-vorspann}


         
         \renewcommand{\erwaehntePersonen}{Personen: Caroline Burger, Paul Goldmann, Marie Reinhard}
         \renewcommand{\erwaehnteInstitutionen}{Institutionen: Deutsche Post in China}
         \renewcommand{\erwaehnteOrte}{Orte: Bad Tölz, Frankgasse, Tegernsee, Tianjin, Wien, Österreich}
         \renewcommand{\erwaehnteWerke}{}
               \section[ Paul Goldmann an Arthur Schnitzler, 10. 9. 1898]{ Paul Goldmann an Arthur Schnitzler, 10. 9. 1898}\nopagebreak\mylabel{v}\rehead{ }\begin{ledgroupsized}[t]{13cm}\normalsize\beginnumbering\briefempfaengerindex{Schnitzler, Arthur@\textsc{Schnitzler, Arthur}!zzzGoldmann, Paul@\emph{von Paul Goldmann}!1898-09-101@{10. 9. 1898}|(be} \toendnotes[C]{\smallbreak\pagebreak[2]} \Standort{DLA, A:Schnitzler, HS.NZ85.1.3168.}
\physDesc{Bildpostkarte, 144 Zeichen
\newline{}Handschrift: 1) blaue Tinte, deutsche Kurrent\hspace{1em}2) blaue Tinte, lateinische Kurrent (\noindent{}Adresse)\hspace{1em}
\newline{}Versand: 1) Stempel: »\nobreak{}\oindex{Tianjin@\textbf{Tianjin}|pwk}Tientsin, 10/9 98, Kaiserl. deutsche
                                          Postagentur\orgindex{Deutsche Post in China@Deutsche Post in China|pw}\nobreak{}«.   2) Stempel: »\nobreak{}\oindex{XXXX Ortsangabe fehlt|pwk}Wien 9/3 72, 22. 10. 98, 1. N, Bestellt\nobreak{}«. 
\newline{}Schnitzler: mit Bleistift das Jahr »98« vermerkt }\toendnotes[C]{\smallbreak}\pstart{}{\pb}\begin{otherlanguage}{english}Austria\oindex{Oesterreich@\textbf{Österreich}|pw}\end{otherlanguage}.\pend{}\pstart{}Herrn Dr. Arthur Schnitzler\pend{}\pstart{}Wien\oindex{Wien@\textbf{Wien}|pw}\pend{}\pstart{}IX. Frankgaſse 1\oindex{XXXX Ortsangabe fehlt|pw}.\pend{}{\bigskip}\pstart
           \raggedleft{}{\pb}\textcolor{gray}{\textbf{TIENTSIN\oindex{Tianjin@\textbf{Tianjin}|pw}}}, 10. September.\pend
           \pstart
           Viele Grüße und herzlichen Dank für die Karte aus \label{K_L02857-1v}\edtext{\textsc{Toelz\oindex{Bad Toelz@\textbf{Bad Tölz}|pw}}}{\lemma{\textnormal{\emph{Toelz}}}\Cendnote{\textnormal{Schnitzler\pwindex{Schnitzler, Arthur 15.05.1862 – 21.10.1931@\textsc{Schnitzler, Arthur} (15.05.1862 – 21.10.1931), \emph{Schriftsteller, Mediziner}|pwk} unternahm am 3. 8. 1898 mit Marie Reinhard\pwindex{Reinhard, Marie 1871-03-13 – 1899-03-18@\textsc{Reinhard, Marie} (1871-03-13 – 1899-03-18), \emph{Gesangspädagogin}|pwk} und deren Schwester Caroline Burger\pwindex{Burger, Caroline 11.07.1869 – 15.03.1959@\textsc{Burger, Caroline} (11.07.1869 – 15.03.1959)|pwk} einen Radausflug von Tegernsee\oindex{Tegernsee@\textbf{Tegernsee}|pwk} nach Bad Tölz\oindex{Bad Toelz@\textbf{Bad Tölz}|pwk} und wieder zurück.}}}\label{K_L02857-1h}!\pend
           \pstart
           Dein {\\[\baselineskip]}\spacefill\mbox{Paul Goldmann.}\pend
           \leftskip=0em{}
         
         \endnumbering\mylabel{h}\end{ledgroupsized}  \newcommand{\dateiname}{L02857}\newcommand{\titel}{Paul Goldmann an Arthur Schnitzler, 10. 9. 1898}\newcommand{\editorInnen}{Martin Anton Müller und Laura Untner}%% latex-leseansicht-abspann.tex
%% Abspann für die Leseansicht.
%% Der Schalter \ifkorrekturansicht ist bereits durch den Vorspann gesetzt.

%% latex-abspann.tex
%% Gemeinsamer Abspann für Korrekturansicht und Leseansicht.
%% Setzt den Schalter \ifkorrekturansicht voraus (gesetzt in den
%% einbindenden Dateien latex-korrekturansicht-abspann.tex bzw.
%% latex-leseansicht-abspann.tex).
%% ---------------------------------------------------------------

\normalsize

% Das esempio-Environment wird nur in der Leseansicht benötigt
\ifkorrekturansicht\else
\newenvironment{esempio}[3]%
{
    \vspace{1.5ex}
    \rlap{\underline{#1}}
    \par
    \setlength{\parindent}{0cm}
    \nopagebreak
    \leftskip=#2cm
    \rightskip=#3cm
}
{
    \par
}
\fi

\doendnotes{C}
\bigskip
\vfill

\clearpage

\footnotesize

\ifkorrekturansicht
  \lohead{\textsc{register}}
\fi

% theindex-Environment neu definieren ohne reledmac
\makeatletter
\renewenvironment{theindex}{%
  \ifkorrekturansicht
    \section*{\indexname}%
  \else
    \subsubsection*{Index der erwähnten Entitäten}%
  \fi
  \setlength{\parindent}{0pt}%
  \setlength{\parskip}{0pt plus 0.3pt}%
  \let\item\@idxitem
}{%
  \ifkorrekturansicht\clearpage\fi
}
\makeatother

\IfFileExists{\jobname-pw.ind}{\input{\jobname-pw.ind}}{}

% Quellenangabe nur in der Leseansicht
\ifkorrekturansicht\else
% Fallback-Definitionen, falls die .tex-Datei \titel etc. nicht gesetzt hat
\providecommand{\titel}{}
\providecommand{\editorInnen}{}
\providecommand{\dateiname}{\jobname}

\vspace{3cm}

\vfill

\footnotesize
\textsc{Quelle}: \titel. Herausgegeben von {\editorInnen}. In: \emph{Arthur Schnitzler: Briefwechsel mit Autorinnen und Autoren}.
 Digitale Edition, https://schnitzler-briefe.acdh.oeaw.ac.at/{\dateiname}.html (Stand \today)
\fi

\end{document}


      