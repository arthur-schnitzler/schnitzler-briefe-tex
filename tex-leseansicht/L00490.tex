%% latex-korrekturansicht-vorspann.tex
%% Vorspann für die Korrekturansicht.
%% Lädt die gemeinsame Datei latex-vorspann.tex mit gesetztem Schalter.

\newif\ifkorrekturansicht
\korrekturansichttrue

\input{../tex-inputs/latex-vorspann}


\section[Arthur Schnitzler an Richard Beer-Hofmann, 23. 9. 1895]{L00490 Arthur Schnitzler an Richard Beer-Hofmann, 23. 9. 1895}
\nopagebreak\mylabel{L00490v}
\rehead{ }\normalsize\beginnumbering\briefempfaengerindex{Beer-Hofmann, Richard@\textsc{Beer-Hofmann, Richard}!zzzSchnitzler, Arthur@\emph{von Arthur Schnitzler}!1895-09-231@{23. 9. 1895}|(be}
\toendnotes[C]{\smallbreak\pagebreak[2]}\Standort{YCGL, MSS 31.}
\physDesc{Postkarte, 468 Zeichen
\newline{}Handschrift: 1) schwarze Tinte, deutsche Kurrent\hspace{1em}2) schwarze Tinte, lateinische Kurrent (\noindent{}Adresse)\hspace{1em}
\newline{}Versand: Stempel: »\nobreak{}\oindex{IX., Alsergrund@\textbf{IX., Alsergrund}, \emph{A.ADM3}|pwk}Wien 9/3 72, 23. 9. 95, 3–4N\nobreak{}«.  }
\buchAbdrucke{\weitereDrucke{Arthur Schnitzler, Richard Beer-Hofmann: \emph{Briefwechsel 1891–1931}. Wien, Zürich: \emph{Europaverlag} 1992, S. 84.} }\toendnotes[C]{\smallbreak}\pstart{}{\pb}Dr. Richard Beer-Hofmann\pend{}\pstart{}Gardone\oindex{Gardone Riviera@\textbf{Gardone Riviera}, \emph{A.ADM3}|pw}\pend{}\pstart{}am Gardasee\oindex{Lago di Garda@\textbf{Lago di Garda}, \emph{See (N.SEE)}|pw}\pend{}\pstart{}Italia.\oindex{Italien@\textbf{Italien}, \emph{A.PCLI}|pw}\pend{}{\bigskip}\vspace{1em}
\pstart
           \noindent{}{\pb}Lieber Richard, nach \textsc{Riva}\oindex{Riva del Garda@\textbf{Riva del Garda}, \emph{P.PPLA3}|pw} hab ich Ihnen nicht nur eine Karte, ſondern einen längern Brief geſchrieben,
               den Sie gef. reclamiren wollen. Schreiben Sie mir endlich auch einmal wieder.\pend
           
\pstart
           Vom Burgth.\orgindex{Burgtheater@Burgtheater|pw} nichts Neues. –\pend
           
\pstart
           »\label{K_L00490-1v}\edtext{\textsc{Mourir}\pwindex{Mourir. Roman@\emph{Mourir. Roman}|pw}}{\lemma{\textnormal{\emph{Mourir}}}\Cendnote{\textnormal{Zuvor war \emph{Sterben}\pwindex{Sterben. Novelle@\emph{Sterben. Novelle}|pwk} in der Übersetzung von Gaspard Vallette\pwindex{Vallette, Gaspard 13.5.1865 – 6.8.1911@\textsc{Vallette, Gaspard} (13.5.1865 – 6.8.1911), \emph{Journalist/Journalistin, Übersetzer/Übersetzerin}|pwk} in sechs Teilen zwischen
                     27. 4. 1895 und 1. 6. 1895 in der \emph{Semaine littéraire}\orgindex{Semaine Litteraire@La Semaine Littéraire|pwk} erschienen. Die gebundene Ausgabe hatte Schnitzler am 12. 4. 1896 in der Hand.}}}\label{K_L00490-1}« erſcheint
               bei \textsc{Perrin}\orgindex{Editions Perrin@Éditions Perrin|pw} in \textsc{Paris}\oindex{Paris@\textbf{Paris}, \emph{P.PPLC}|pw} (durch Vermittlung der Red. der \textsc{Sem. litt.}\orgindex{Semaine Litteraire@La Semaine Littéraire|pw})\pend
           
\pstart
           – Sie müſſen es jetzt da unten herrlich haben. Ich denke an den Gardaſee\oindex{Lago di Garda@\textbf{Lago di Garda}, \emph{See (N.SEE)}|pw} bei Gardone\oindex{Gardone Riviera@\textbf{Gardone Riviera}, \emph{A.ADM3}|pw} zurück
               wie an ein Meer.\pend
           \pstart Seien Sie herzlich gegrüßt! Ihr \spacefill\mbox{Arthur}\pend{}\selectlanguage{ngerman}\endnumbering\briefempfaengerindex{Beer-Hofmann, Richard@\textsc{Beer-Hofmann, Richard}!zzzSchnitzler, Arthur@\emph{von Arthur Schnitzler}!1895-09-231@{23. 9. 1895}|)be}\mylabel{L00490h}  \normalsize

\doendnotes{C}
\bigskip
\vfill

\clearpage

\footnotesize

\lohead{\textsc{register}}

% Definiere theindex-Environment komplett neu ohne reledmac
\makeatletter
\renewenvironment{theindex}{%
  \section*{\indexname}%
  \setlength{\parindent}{0pt}%
  \setlength{\parskip}{0pt plus 0.3pt}%
  \let\item\@idxitem
}{%
  \clearpage
}
\makeatother

\IfFileExists{\jobname-pw.ind}{\input{\jobname-pw.ind}}{}

\end{document}

      