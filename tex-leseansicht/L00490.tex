%% latex-leseansicht-vorspann.tex
%% Vorspann für die Leseansicht.
%% Lädt die gemeinsame Datei latex-vorspann.tex mit nicht gesetztem Schalter.

\newif\ifkorrekturansicht
\korrekturansichtfalse

\input{../tex-inputs/latex-vorspann}

\begin{center}
            \textcolor{red}{ENTWURF. ENTZIFFERUNG NOCH NICHT KORREKTURGELESEN}
                      \end{center}
            
               \section[Arthur Schnitzler an Richard Beer-Hofmann, 23. 9. 1895]{ Arthur Schnitzler an Richard Beer-Hofmann, 23. 9. 1895}\nopagebreak\mylabel{v}\rehead{ }\begin{ledgroupsized}[t]{13cm}\normalsize\beginnumbering\briefempfaengerindex{Beer-Hofmann, Richard@\textsc{Beer-Hofmann, Richard}!zzzSchnitzler, Arthur@\emph{von Arthur Schnitzler}!1895-09-231@{23. 9. 1895}|(be} \toendnotes[C]{\smallbreak\pagebreak[2]} \Standort{YCGL, MSS 31.}
\physDesc{Postkarte
\newline{}Handschrift: schwarze Tinte, deutsche Kurrent\newline{}Versand: Stempel: »\nobreak{}\oindex{IX., Alsergrund@\textbf{IX., Alsergrund}|pwk}Wien 9/3 72, 23. 9. 95, 3–4N\nobreak{}«.  }\buchAbdrucke{\weitereDrucke{Arthur Schnitzler, Richard Beer-Hofmann: \emph{Briefwechsel 1891–1931}. Hg. Konstanze Fliedl. Wien, Zürich: \emph{Europaverlag} 1992, S. 84.} }\toendnotes[C]{\smallbreak}\pstart{}{\pb}\textsc{Dr. Richard Beer-Hofmann}\pend{}\pstart{}\textsc{Gardone}\oindex{Gardone Riviera@\textbf{Gardone Riviera}|pw}\pend{}\pstart{}\textsc{am Gardasee\oindex{Lago di Garda@\textbf{Lago di Garda}|pw}}\pend{}\pstart{}\textsc{Italia.}\oindex{Italien@\textbf{Italien}|pw}\pend{}{\bigskip}\pstart
           \noindent{}{\pb}Lieber Richard, nach \textsc{Riva}\oindex{Riva del Garda@\textbf{Riva del Garda}|pw} hab ich Ihnen nicht nur eine Karte, ſondern einen längern Brief geſchrieben,
               den Sie gef. reclamiren wollen. Schreiben Sie mir endlich auch einmal wieder.\pend
           \pstart
           Vom Burgth.\orgindex{Burgtheater@Burgtheater|pw} nichts Neues. –\pend
           \pstart
           »\label{K_L00490_1v}\edtext{\textsc{Mourir}\pwindex{Schnitzler, Arthur 15.05.1862 – 21.10.1931@\textsc{Schnitzler, Arthur} (15.05.1862 – 21.10.1931), \emph{Schriftsteller, Mediziner}!Mourir1896-04-12 – 1896-04-12@\strich\emph{Mourir} {[}1896-04-12 – 1896-04-12{]}|pw}}{\lemma{\textnormal{\emph{Mourir}}}\Cendnote{\textnormal{Zuvor war \emph{Sterben}\pwindex{Schnitzler, Arthur 15.05.1862 – 21.10.1931@\textsc{Schnitzler, Arthur} (15.05.1862 – 21.10.1931), \emph{Schriftsteller, Mediziner}!Sterben. Novelle1.10.1894 – 1.12.1894@\strich\emph{Sterben. Novelle} {[}1.10.1894 – 1.12.1894{]}|pwk} in der Übersetzung von Gaspard
                     Vallette\pwindex{Vallette, Gaspard 13.5.1865 – 6.8.1911@\textsc{Vallette, Gaspard} (13.5.1865 – 6.8.1911), \emph{Journalist, Übersetzer}|pwk} in sechs Teilen zwischen 27. 4. 1895 und
                     1. 6.1895 in der \emph{Semaine
                     littéraire}\orgindex{Semaine Litteraire@La Semaine Littéraire|pwk} erschienen. Die gebundene Ausgabe hatte Schnitzler\pwindex{Schnitzler, Arthur 15.05.1862 – 21.10.1931@\textsc{Schnitzler, Arthur} (15.05.1862 – 21.10.1931), \emph{Schriftsteller, Mediziner}|pwk} am 12. 4. 1896 in der Hand.}}}\label{K_L00490_1h}« erſcheint bei \textsc{Perrin}\orgindex{Editions Perrin@Éditions Perrin|pw} in \textsc{Paris}\oindex{Paris@\textbf{Paris}|pw} (durch Vermittlung der Red. der \textsc{Sem. litt.}\orgindex{Semaine Litteraire@La Semaine Littéraire|pw})\pend
           \pstart
           – Sie müſſen es jetzt da unten herrlich haben. Ich denke an den Gardaſee\oindex{Lago di Garda@\textbf{Lago di Garda}|pw} bei Gardone\oindex{Gardone Riviera@\textbf{Gardone Riviera}|pw} zurück wie
               an ein Meer.\pend
           \pstart Seien Sie herzlich gegrüßt! Ihr \spacefill\mbox{Arthur}\pend{}\endnumbering\briefempfaengerindex{Beer-Hofmann, Richard@\textsc{Beer-Hofmann, Richard}!zzzSchnitzler, Arthur@\emph{von Arthur Schnitzler}!1895-09-231@{23. 9. 1895}|)be}\mylabel{h}\end{ledgroupsized}  \newcommand{\dateiname}{L00490}\newcommand{\titel}{Arthur Schnitzler an Richard Beer-Hofmann, 23. 9. 1895}\newcommand{\editorInnen}{Martin Anton Müller und Gerd-Hermann Susen}%% latex-leseansicht-abspann.tex
%% Abspann für die Leseansicht.
%% Der Schalter \ifkorrekturansicht ist bereits durch den Vorspann gesetzt.

%% latex-abspann.tex
%% Gemeinsamer Abspann für Korrekturansicht und Leseansicht.
%% Setzt den Schalter \ifkorrekturansicht voraus (gesetzt in den
%% einbindenden Dateien latex-korrekturansicht-abspann.tex bzw.
%% latex-leseansicht-abspann.tex).
%% ---------------------------------------------------------------

\normalsize

% Das esempio-Environment wird nur in der Leseansicht benötigt
\ifkorrekturansicht\else
\newenvironment{esempio}[3]%
{
    \vspace{1.5ex}
    \rlap{\underline{#1}}
    \par
    \setlength{\parindent}{0cm}
    \nopagebreak
    \leftskip=#2cm
    \rightskip=#3cm
}
{
    \par
}
\fi

\doendnotes{C}
\bigskip
\vfill

\clearpage

\footnotesize

\ifkorrekturansicht
  \lohead{\textsc{register}}
\fi

% theindex-Environment neu definieren ohne reledmac
\makeatletter
\renewenvironment{theindex}{%
  \ifkorrekturansicht
    \section*{\indexname}%
  \else
    \subsubsection*{Index der erwähnten Entitäten}%
  \fi
  \setlength{\parindent}{0pt}%
  \setlength{\parskip}{0pt plus 0.3pt}%
  \let\item\@idxitem
}{%
  \ifkorrekturansicht\clearpage\fi
}
\makeatother

\IfFileExists{\jobname-pw.ind}{\input{\jobname-pw.ind}}{}

% Quellenangabe nur in der Leseansicht
\ifkorrekturansicht\else
% Fallback-Definitionen, falls die .tex-Datei \titel etc. nicht gesetzt hat
\providecommand{\titel}{}
\providecommand{\editorInnen}{}
\providecommand{\dateiname}{\jobname}

\vspace{3cm}

\vfill

\footnotesize
\textsc{Quelle}: \titel. Herausgegeben von {\editorInnen}. In: \emph{Arthur Schnitzler: Briefwechsel mit Autorinnen und Autoren}.
 Digitale Edition, https://schnitzler-briefe.acdh.oeaw.ac.at/{\dateiname}.html (Stand \today)
\fi

\end{document}


      