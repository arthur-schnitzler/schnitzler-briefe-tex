%% latex-leseansicht-vorspann.tex
%% Vorspann für die Leseansicht.
%% Lädt die gemeinsame Datei latex-vorspann.tex mit nicht gesetztem Schalter.

\newif\ifkorrekturansicht
\korrekturansichtfalse

\input{../tex-inputs/latex-vorspann}


\section[Arthur Schnitzler an Richard Beer-Hofmann, 23. 9. 1895]{L00490 Arthur Schnitzler an Richard Beer-Hofmann, 23. 9. 1895}
\nopagebreak\mylabel{L00490v}
\rehead{ }\normalsize\beginnumbering\briefempfaengerindex{Beer-Hofmann, Richard@\textsc{Beer-Hofmann, Richard}!zzzSchnitzler, Arthur@\emph{von Arthur Schnitzler}!1895-09-231@{23. 9. 1895}|(be}
\toendnotes[C]{\smallbreak\pagebreak[2]}
\correspDesc{Versand  durch Arthur Schnitzler am 23. 9. 1895 in Wien
\newline{}Erhalt  durch Richard Beer-Hofmann im Zeitraum [24. 9. 1895
                  – 28. 9. 1895?] in Gardone Riviera}\toendnotes[C]{\smallbreak}
\Standort{YCGL, MSS 31.}
\physDesc{Postkarte, 468 Zeichen
\newline{}Handschrift: schwarze Tinte, deutsche Kurrent
\newline{}Versand: Stempel: »\nobreak{}\oindex{IX., Alsergrund@\textbf{IX., Alsergrund}, \emph{Verwaltungsgebiet}|pwk}Wien 9/3 72, 23. 9. 95, 3–4N\nobreak{}«.  }
\buchAbdrucke{\weitereDrucke{Arthur Schnitzler, Richard Beer-Hofmann: \emph{Briefwechsel 1891–1931}. Herausgegeben von Konstanze Fliedl. Wien, Zürich: \emph{Europaverlag} 1992, S. 84.} }\toendnotes[C]{\smallbreak}\pstart{}\textsc{{\pb}Dr. Richard Beer-Hofmann}\pend{}\pstart{}\textsc{Gardone\oindex{Gardone Riviera@\textbf{Gardone Riviera}, \emph{Verwaltungsgebiet}|pw}}\pend{}\pstart{}\textsc{am Gardasee\oindex{Lago di Garda@\textbf{Lago di Garda}, \emph{See}|pw}}\pend{}\pstart{}\textsc{Italia.\oindex{Italien@\textbf{Italien}|pw}}\pend{}{\bigskip}\vspace{1em}
\pstart
           \noindent{}{\pb}Lieber Richard, nach \textsc{Riva}\oindex{Riva del Garda@\textbf{Riva del Garda}, \emph{Hauptstadt}|pw} hab ich Ihnen nicht nur eine Karte,{ }ſondern einen längern Brief geſchrieben,
               den Sie gef. reclamiren wollen. Schreiben Sie mir endlich auch einmal wieder.\pend
           
\pstart
           Vom Burgth.\orgindex{Burgtheater@Burgtheater|pw} nichts Neues. –\pend
           
\pstart
           »\label{K_L00490-1v}\edtext{\textsc{Mourir}\pwindex{Schnitzler, Arthur 15.\,5.\,1862 Wien – 21.\,10.\,1931 ebd.@\textsc{Schnitzler, Arthur} (15.\,5.\,1862 Wien – 21.\,10.\,1931 ebd.), \emph{Schriftsteller, Mediziner}!Mourir. Roman@\strich\emph{Mourir. Roman}|pw}}{\lemma{\textnormal{\emph{Mourir}}}\Cendnote{\textnormal{Zuvor war \emph{Sterben}\pwindex{Schnitzler, Arthur 15.\,5.\,1862 Wien – 21.\,10.\,1931 ebd.@\textsc{Schnitzler, Arthur} (15.\,5.\,1862 Wien – 21.\,10.\,1931 ebd.), \emph{Schriftsteller, Mediziner}!Sterben. Novelle@\strich\emph{Sterben. Novelle}|pwk} in der Übersetzung von Gaspard Vallette\pwindex{Vallette, Gaspard 13.\,5.\,1865 Jussy – 6.\,8.\,1911 La Tène@\textsc{Vallette, Gaspard} (13.\,5.\,1865 Jussy – 6.\,8.\,1911 La Tène), \emph{Journalist, Übersetzer}|pwk} in sechs Teilen zwischen
                     27. 4. 1895 und 1. 6. 1895 in der \emph{Semaine littéraire}\orgindex{Semaine Littéraire@La Semaine Littéraire|pwk} erschienen. Die gebundene Ausgabe hatte Schnitzler am 12. 4. 1896 in der Hand.}}}\label{K_L00490-1}« erſcheint
               bei \textsc{Perrin}\orgindex{Éditions Perrin@Éditions Perrin|pw} in \textsc{Paris}\oindex{Paris@\textbf{Paris}, \emph{Hauptstadt}|pw} (durch Vermittlung der Red. der \textsc{Sem. litt.}\orgindex{Semaine Littéraire@La Semaine Littéraire|pw})\pend
           
\pstart
           – Sie müſſen es jetzt da unten herrlich haben. Ich denke an den Gardaſee\oindex{Lago di Garda@\textbf{Lago di Garda}, \emph{See}|pw} bei Gardone\oindex{Gardone Riviera@\textbf{Gardone Riviera}, \emph{Verwaltungsgebiet}|pw} zurück
               wie an ein Meer.\pend
           \pstart Seien Sie herzlich gegrüßt! Ihr \spacefill\mbox{Arthur}\pend{}\selectlanguage{ngerman}\endnumbering\briefempfaengerindex{Beer-Hofmann, Richard@\textsc{Beer-Hofmann, Richard}!zzzSchnitzler, Arthur@\emph{von Arthur Schnitzler}!1895-09-231@{23. 9. 1895}|)be}\mylabel{L00490h}  \newcommand{\dateiname}{L00490}\newcommand{\titel}{Arthur Schnitzler an Richard Beer-Hofmann, 23. 9. 1895}\newcommand{\editorInnen}{Martin Anton Müller und Gerd-Hermann Susen}%% latex-leseansicht-abspann.tex
%% Abspann für die Leseansicht.
%% Der Schalter \ifkorrekturansicht ist bereits durch den Vorspann gesetzt.

%% latex-abspann.tex
%% Gemeinsamer Abspann für Korrekturansicht und Leseansicht.
%% Setzt den Schalter \ifkorrekturansicht voraus (gesetzt in den
%% einbindenden Dateien latex-korrekturansicht-abspann.tex bzw.
%% latex-leseansicht-abspann.tex).
%% ---------------------------------------------------------------

\normalsize

% Das esempio-Environment wird nur in der Leseansicht benötigt
\ifkorrekturansicht\else
\newenvironment{esempio}[3]%
{
    \vspace{1.5ex}
    \rlap{\underline{#1}}
    \par
    \setlength{\parindent}{0cm}
    \nopagebreak
    \leftskip=#2cm
    \rightskip=#3cm
}
{
    \par
}
\fi

\doendnotes{C}
\bigskip
\vfill

\clearpage

\footnotesize

\ifkorrekturansicht
  \lohead{\textsc{register}}
\fi

% theindex-Environment neu definieren ohne reledmac
\makeatletter
\renewenvironment{theindex}{%
  \ifkorrekturansicht
    \section*{\indexname}%
  \else
    \subsubsection*{Index der erwähnten Entitäten}%
  \fi
  \setlength{\parindent}{0pt}%
  \setlength{\parskip}{0pt plus 0.3pt}%
  \let\item\@idxitem
}{%
  \ifkorrekturansicht\clearpage\fi
}
\makeatother

\IfFileExists{\jobname-pw.ind}{\input{\jobname-pw.ind}}{}

% Quellenangabe nur in der Leseansicht
\ifkorrekturansicht\else
% Fallback-Definitionen, falls die .tex-Datei \titel etc. nicht gesetzt hat
\providecommand{\titel}{}
\providecommand{\editorInnen}{}
\providecommand{\dateiname}{\jobname}

\vspace{3cm}

\vfill

\footnotesize
\textsc{Quelle}: \titel. Herausgegeben von {\editorInnen}. In: \emph{Arthur Schnitzler: Briefwechsel mit Autorinnen und Autoren}.
 Digitale Edition, https://schnitzler-briefe.acdh.oeaw.ac.at/{\dateiname}.html (Stand \today)
\fi

\end{document}


