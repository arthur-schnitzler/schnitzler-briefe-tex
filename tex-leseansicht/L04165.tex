%% latex-leseansicht-vorspann.tex
%% Vorspann für die Leseansicht.
%% Lädt die gemeinsame Datei latex-vorspann.tex mit nicht gesetztem Schalter.

\newif\ifkorrekturansicht
\korrekturansichtfalse

\input{../tex-inputs/latex-vorspann}


\section[Arthur Schnitzler an Gustav Schwarzkopf, 25. 1. 1910]{L04165 Arthur Schnitzler an Gustav Schwarzkopf, 25. 1. 1910}
\nopagebreak\mylabel{L04165v}
\rehead{ }\normalsize\beginnumbering\briefempfaengerindex{Schwarzkopf, Gustav@\textsc{Schwarzkopf, Gustav}!zzzSchnitzler, Arthur@\emph{von Arthur Schnitzler}!1910-01-251@{25. 1. 1910}|(be}
\toendnotes[C]{\smallbreak\pagebreak[2]}
\correspDesc{Versand  durch Arthur Schnitzler am 25. 1. 1910 in Wien
\newline{}Erhalt  durch Gustav Schwarzkopf im Zeitraum [25. 1. 1910 – 28. 1. 1910?] in Wien}\toendnotes[C]{\smallbreak}
\Standort{CUL, Schnitzler, B 96.}
\physDesc{Briefkarte, 326 Zeichen
\newline{}Handschrift: schwarze Tinte, deutsche Kurrent}\toendnotes[C]{\smallbreak}
\pstart
           {\pb}\textcolor{gray}{\textbf{Dr Arthur Schnitzler}}\hfill 25. 1. 10\pend
           
\pstart
           \textcolor{gray}{\textbf{Wien XVIII.
                        Spoettelgasse 7\oindex{Wien@\textbf{Wien}!XVIII., Währing@\textbf{XVIII., Währing}!Edmund-Weiß-Gasse@\textbf{Edmund-Weiß-Gasse}, \emph{Straße}|pw}.}}\pend
           \vspace{0.5em}
\pstart
           lieber Guſtav, ich finde Ihren Brief erſt bei meiner \label{K_L04165-1v}\edtext{Rückkunft aus
               Dresden\oindex{Dresden@\textbf{Dresden}|pw}}{\lemma{\textnormal{\emph{Rückkunft aus
               Dresden}}}\Cendnote{\textnormal{Er war bei der Uraufführung\eventindex{Semperoper@\textbf{Semperoper}!Uraufführung von Der Schleier der Pierrette, Premiere von Versiegelt, 22.1.1910@Uraufführung von Der Schleier der Pierrette, Premiere von Versiegelt, 22.1.1910|pwk} von
                  \emph{Der Schleier der Pierrette}\pwindex{Schnitzler, Arthur 15. 5. 1862 Wien – 21. 10. 1931 ebd.@\textsc{Schnitzler, Arthur} (15. 5. 1862 Wien – 21. 10. 1931 ebd.), \emph{Schriftsteller, Mediziner}!Schleier der Pierrette. Pantomime in drei Bildern@\strich\emph{Der Schleier der Pierrette. Pantomime in drei Bildern}|pwk} gewesen und war am 24. 1. 1910
                  in der Früh wieder in Wien\oindex{Wien@\textbf{Wien}, \emph{Verwaltungsgebiet}|pwk} angekommen.}}}\label{K_L04165-1} vor. Meinem Bruder\pwindex{Schnitzler, Julius 13.\,7.\,1865 Wien – 29.\,6.\,1939 ebd.@\textsc{Schnitzler, Julius} (13.\,7.\,1865 Wien – 29.\,6.\,1939 ebd.), \emph{Chirurg}|pwv} will ich Holzers\pwindex{Holzer, Rudolf 28.\,7.\,1875 Wien – 17.\,7.\,1965 ebd.@\textsc{Holzer, Rudolf} (28.\,7.\,1875 Wien – 17.\,7.\,1965 ebd.), \emph{Schriftsteller, Journalist}|pw}
               Anliegen gern beſtellen. Hoffentlich gehts der \label{K_L04165-2v}\edtext{Kleinen\pwindex{?? [Tocher von Alice Hétsey-Holzer und Rudolf Holzer] vor Juli 1909 – nach Juli 1915@\textsc{?? [Tocher von Alice Hétsey-Holzer und Rudolf Holzer]} (vor Juli 1909 – nach Juli 1915)|pwv}}{\lemma{\textnormal{\emph{Kleinen}}}\Cendnote{\textnormal{Die gemeinsame
                  Tochter von Rudolf Holzer\pwindex{Holzer, Rudolf 28.\,7.\,1875 Wien – 17.\,7.\,1965 ebd.@\textsc{Holzer, Rudolf} (28.\,7.\,1875 Wien – 17.\,7.\,1965 ebd.), \emph{Schriftsteller, Journalist}|pwk} und Alice Holzer-Hétsey\pwindex{Hétsey-Holzer, Alice 3.\,9.\,1875 Wien – 11.\,2.\,1939 ebd.@\textsc{Hétsey-Holzer, Alice} (3.\,9.\,1875 Wien – 11.\,2.\,1939 ebd.), \emph{Schauspielerin}|pwk} hieß
                  Dorothea Maria\pwindex{?? [Tocher von Alice Hétsey-Holzer und Rudolf Holzer] vor Juli 1909 – nach Juli 1915@\textsc{?? [Tocher von Alice Hétsey-Holzer und Rudolf Holzer]} (vor Juli 1909 – nach Juli 1915)|pwk} und verwendete den Rufnamen Thea\pwindex{?? [Tocher von Alice Hétsey-Holzer und Rudolf Holzer] vor Juli 1909 – nach Juli 1915@\textsc{?? [Tocher von Alice Hétsey-Holzer und Rudolf Holzer]} (vor Juli 1909 – nach Juli 1915)|pwk}. Sie
                  dürfte 1919 mit ihrer Mutter\pwindex{Hétsey-Holzer, Alice 3.\,9.\,1875 Wien – 11.\,2.\,1939 ebd.@\textsc{Hétsey-Holzer, Alice} (3.\,9.\,1875 Wien – 11.\,2.\,1939 ebd.), \emph{Schauspielerin}|pwkv} in dem Film \emph{Wer das Kleine nicht ehrt}\textcolor{red}{\textsuperscript{XXXX indx2}}
                  aufgetreten sein und dafür den Bühnennamen »Little Thea\pwindex{?? [Tocher von Alice Hétsey-Holzer und Rudolf Holzer] vor Juli 1909 – nach Juli 1915@\textsc{?? [Tocher von Alice Hétsey-Holzer und Rudolf Holzer]} (vor Juli 1909 – nach Juli 1915)|pwk}« verwendet haben. Ihr Alter wird 
                  mit 15 angegeben, so dass sie um 1904 geboren sein dürfte. Bis 1926 lassen sich vereinzelt von ihr Texte in Zeitungen
                  nachweisen, danach verliert sich ihre Spur.}}}\label{K_L04165-2} bald ganz gut. Auf Wiedersehen. Dresden\oindex{Dresden@\textbf{Dresden}|pw} war ſehr {\pb}angenehm. Die Aufführung\eventindex{Semperoper@\textbf{Semperoper}!Uraufführung von Der Schleier der Pierrette, Premiere von Versiegelt, 22.1.1910@Uraufführung von Der Schleier der Pierrette, Premiere von Versiegelt, 22.1.1910|pwv} musikaliſch
               außerordentlich; manches einfach hinreißend.\pend
           
\pstart
           Auf bald alſo. Herzlichſt Ihr{\\[\baselineskip]}\spacefill\mbox{Arthur.}\pend
           \leftskip=0em{}\selectlanguage{ngerman}\endnumbering\briefempfaengerindex{Schwarzkopf, Gustav@\textsc{Schwarzkopf, Gustav}!zzzSchnitzler, Arthur@\emph{von Arthur Schnitzler}!1910-01-251@{25. 1. 1910}|)be}\mylabel{L04165h}
\begin{anhang}
\end{anhang}\newcommand{\dateiname}{L04165}\newcommand{\titel}{Arthur Schnitzler an Gustav Schwarzkopf, 25. 1. 1910}\newcommand{\editorInnen}{Herausgegeben von Jahnke, SelmaMüller, Martin Anton}%% latex-leseansicht-abspann.tex
%% Abspann für die Leseansicht.
%% Der Schalter \ifkorrekturansicht ist bereits durch den Vorspann gesetzt.

%% latex-abspann.tex
%% Gemeinsamer Abspann für Korrekturansicht und Leseansicht.
%% Setzt den Schalter \ifkorrekturansicht voraus (gesetzt in den
%% einbindenden Dateien latex-korrekturansicht-abspann.tex bzw.
%% latex-leseansicht-abspann.tex).
%% ---------------------------------------------------------------

\normalsize

% Das esempio-Environment wird nur in der Leseansicht benötigt
\ifkorrekturansicht\else
\newenvironment{esempio}[3]%
{
    \vspace{1.5ex}
    \rlap{\underline{#1}}
    \par
    \setlength{\parindent}{0cm}
    \nopagebreak
    \leftskip=#2cm
    \rightskip=#3cm
}
{
    \par
}
\fi

\doendnotes{C}
\bigskip
\vfill

\clearpage

\footnotesize

\ifkorrekturansicht
  \lohead{\textsc{register}}
\fi

% theindex-Environment neu definieren ohne reledmac
\makeatletter
\renewenvironment{theindex}{%
  \ifkorrekturansicht
    \section*{\indexname}%
  \else
    \subsubsection*{Index der erwähnten Entitäten}%
  \fi
  \setlength{\parindent}{0pt}%
  \setlength{\parskip}{0pt plus 0.3pt}%
  \let\item\@idxitem
}{%
  \ifkorrekturansicht\clearpage\fi
}
\makeatother

\IfFileExists{\jobname-pw.ind}{\input{\jobname-pw.ind}}{}

% Quellenangabe nur in der Leseansicht
\ifkorrekturansicht\else
% Fallback-Definitionen, falls die .tex-Datei \titel etc. nicht gesetzt hat
\providecommand{\titel}{}
\providecommand{\editorInnen}{}
\providecommand{\dateiname}{\jobname}

\vspace{3cm}

\vfill

\footnotesize
\textsc{Quelle}: \titel. Herausgegeben von {\editorInnen}. In: \emph{Arthur Schnitzler: Briefwechsel mit Autorinnen und Autoren}.
 Digitale Edition, https://schnitzler-briefe.acdh.oeaw.ac.at/{\dateiname}.html (Stand \today)
\fi

\end{document}


