%% latex-korrekturansicht-vorspann.tex
%% Vorspann für die Korrekturansicht.
%% Lädt die gemeinsame Datei latex-vorspann.tex mit gesetztem Schalter.

\newif\ifkorrekturansicht
\korrekturansichttrue

\input{../tex-inputs/latex-vorspann}


\section[Richard Beer-Hofmann an Arthur Schnitzler, 3. 7. 1899]{L00932 Richard Beer-Hofmann an Arthur Schnitzler, 3. 7. 1899}
\nopagebreak\mylabel{L00932v}
\rehead{ }\normalsize\beginnumbering\briefempfaengerindex{Schnitzler, Arthur@\textsc{Schnitzler, Arthur}!zzzBeer-Hofmann, Richard@\emph{von Richard Beer-Hofmann}!1899-07-031@{3. 7. 1899}|(be}
\toendnotes[C]{\smallbreak\pagebreak[2]}\Standort{CUL, Schnitzler, B 8.}
\physDesc{Bildpostkarte, 281 Zeichen
\newline{}Handschrift: Bleistift, lateinische Kurrent
\newline{}Versand: 1) Stempel: »\nobreak{}\oindex{Seeboden@\textbf{Seeboden}, \emph{A.ADM3}|pwk}\textcolor{gray}{S}eebod\textcolor{gray}{en}, 3 \textcolor{gray}{7 99}\nobreak{}«.   2) Stempel: »\nobreak{}\oindex{IX., Alsergrund@\textbf{IX., Alsergrund}, \emph{A.ADM3}|pwk}Wien 9/3 72, 4. 7. 99, 10.V, Bestellt\nobreak{}«. 
\newline{}Schnitzler: mit Bleistift mit Empfangsdatum versehen: »4/7 99« 
\newline{}Ordnung: mit Bleistift von unbekannter Hand nummeriert:
                                    »130« }
\buchAbdrucke{\weitereDrucke{Arthur Schnitzler, Richard Beer-Hofmann: \emph{Briefwechsel 1891–1931}. Wien, Zürich: \emph{Europaverlag} 1992, S. 131.} }\toendnotes[C]{\smallbreak}\pstart{}{\pb}Herrn\pend{}\pstart{}D\textsuperscript{r} Arthur Schnitzler\pend{}\pstart{}Wien\oindex{Wien@\textbf{Wien}, \emph{A.ADM2}|pw}\pend{}\pstart{}IX Frankgasse 1\oindex{Frankgasse 1@\textbf{Frankgasse 1}, \emph{Wohngebäude (K.WHS)}|pw}\pend{}{\bigskip}
\pstart
           \noindent{}\centering{}{\pb}\textcolor{gray}{\textbf{Gruss aus Seeboden\oindex{Seeboden@\textbf{Seeboden}, \emph{A.ADM3}|pw}.}}\pend
           \vspace{1em}
\pstart
           \noindent{}{\pb}Lieber Arthur! Im Vorhinein mit Ihren Plänen einverstanden. Bitte
               verständigen Sie auch gelegentlich Mayer\pwindex{Mayer, Oskar 1876 – 15.05.1915@\textsc{Mayer, Oskar} (1876 – 15.05.1915), \emph{Schriftsteller/Schriftstellerin, Beamter/Beamte}|pw}, dem
               ich übrigens auch schreiben werde. Das bewußte, heißt »\label{K_L00932-1v}\edtext{Sanitas}{\lemma{\textnormal{\emph{Sanitas}}}\Cendnote{\textnormal{Ein
                  Verhütungsmittel. In der
                  Werbung wurde Sanitas als »Das neue antiseptische desinficirende und
                     hygienische Mittel«, »Unentbehrlich für jeden Haushalt«
                  angepriesen.}}}\label{K_L00932-1}« erhältlich in d. großen Papiergeschäft\orgindex{Joseph Lustig und Co.@Joseph Lustig {\kaufmannsund}  Co.|pwv} am Hohen
                  Markt\oindex{Hoher Markt@\textbf{Hoher Markt}, \emph{Platz (K.PLT)}|pw}. Herzlichst \spacefill\mbox{R.}\pend
           \selectlanguage{ngerman}\endnumbering\briefempfaengerindex{Schnitzler, Arthur@\textsc{Schnitzler, Arthur}!zzzBeer-Hofmann, Richard@\emph{von Richard Beer-Hofmann}!1899-07-031@{3. 7. 1899}|)be}\mylabel{L00932h}  \normalsize

\doendnotes{C}
\bigskip
\vfill

\clearpage

\footnotesize

\lohead{\textsc{register}}

% Definiere theindex-Environment komplett neu ohne reledmac
\makeatletter
\renewenvironment{theindex}{%
  \section*{\indexname}%
  \setlength{\parindent}{0pt}%
  \setlength{\parskip}{0pt plus 0.3pt}%
  \let\item\@idxitem
}{%
  \clearpage
}
\makeatother

\IfFileExists{\jobname-pw.ind}{\input{\jobname-pw.ind}}{}

\end{document}

      