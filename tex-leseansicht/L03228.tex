%% latex-leseansicht-vorspann.tex
%% Vorspann für die Leseansicht.
%% Lädt die gemeinsame Datei latex-vorspann.tex mit nicht gesetztem Schalter.

\newif\ifkorrekturansicht
\korrekturansichtfalse

\input{../tex-inputs/latex-vorspann}


\section[ Paul Goldmann an Arthur Schnitzler, 26. 10. {[}1902{]}]{L03228 Paul Goldmann an Arthur Schnitzler,  26. 10. [1902]}
\nopagebreak\mylabel{L03228v}
\rehead{ }\normalsize\beginnumbering\briefempfaengerindex{Schnitzler, Arthur@\textsc{Schnitzler, Arthur}!zzzGoldmann, Paul@\emph{von Paul Goldmann}!1902-10-261@{26. 10. [1902]}|(be}
\toendnotes[C]{\smallbreak\pagebreak[2]}
\correspDesc{Versand  durch Paul Goldmann am 26. 10. [1902] in Berlin
\newline{}Erhalt  durch Arthur Schnitzler im Zeitraum [27. 10. 1902 – 31. 10. 1902?] in Wien}\toendnotes[C]{\smallbreak}
\Standort{DLA, A:Schnitzler, HS.NZ85.1.3172.}
\physDesc{Brief, 1 Blatt, 4 Seiten, 1138 Zeichen
\newline{}Handschrift: blaue Tinte, deutsche Kurrent
\newline{}Schnitzler: mit Bleistift das Jahr »902« vermerkt }\toendnotes[C]{\smallbreak}
\pstart
           \raggedleft{}{\pb}\textcolor{gray}{\textbf{DESSAUERSTRASSE 19}}\oindex{Dessauer Straße@\textbf{Dessauer Straße}, \emph{Straße}|pw}\pend
           
\pstart
           Berlin\oindex{Berlin@\textbf{Berlin}, \emph{Hauptstadt}|pw}, 26. Oktober.\pend
           
\pstart\center{}Mein lieber Freund,\pend\vspace{0.5em}
\pstart
           Ich danke Dir vielmals für Deinen lieben Brief, der mich{ }ſehr erfreut hat. Was Du von
                  \label{K_L03228-1v}\edtext{Agnetendorf\oindex{Jagniątków@\textbf{Jagniątków}|pw}}{\lemma{\textnormal{\emph{Agnetendorf}}}\Cendnote{\textnormal{Schnitzler war vom 19. 10. 1902 bis zum
                     20. 10. 1902 bei
                     Gerhart Hauptmann\pwindex{Hauptmann, Gerhart 15.\,11.\,1862 Szczawno-Zdrój – 6.\,6.\,1946 Jagniątków@\textsc{Hauptmann, Gerhart} (15.\,11.\,1862 Szczawno-Zdrój – 6.\,6.\,1946 Jagniątków), \emph{Schriftsteller}|pwk} zu Besuch
                  gewesen.}}}\label{K_L03228-1} erzählſt, hat mich natürlich ganz beſonders intereſſirt. Es thut
               mir aufrichtig leid, daß ich einen Mann\pwindex{Hauptmann, Gerhart 15.\,11.\,1862 Szczawno-Zdrój – 6.\,6.\,1946 Jagniątków@\textsc{Hauptmann, Gerhart} (15.\,11.\,1862 Szczawno-Zdrój – 6.\,6.\,1946 Jagniątków), \emph{Schriftsteller}|pwv}, den Du als{ }ſo{ }ſympathiſch{ }ſchilderſt, \label{K_L03228-2v}\edtext{öffentlich bekämpfen}{\lemma{\textnormal{\emph{öffentlich bekämpfen}}}\Cendnote{\textnormal{Ausdruck findet das in seiner eine Woche zuvor erschienenen
                     Feuilletonsammlung\pwindex{Goldmann, Paul 31.\,1.\,1865 Breslau – 25.\,9.\,1935 Wien@\textsc{Goldmann, Paul} (31.\,1.\,1865 Breslau – 25.\,9.\,1935 Wien), \emph{Schriftsteller, Journalist}!»neue Richtung«. Polemische Aufsätze über Berliner Theater-Aufführungen@\strich\emph{Die »neue Richtung«. Polemische Aufsätze über Berliner Theater-Aufführungen}|pwkv}: Paul Goldmann\pwindex{Goldmann, Paul 31.\,1.\,1865 Breslau – 25.\,9.\,1935 Wien@\textsc{Goldmann, Paul} (31.\,1.\,1865 Breslau – 25.\,9.\,1935 Wien), \emph{Schriftsteller, Journalist}|pwk}: \emph{Die »neue Richtung«. Polemische Aufsätze über Berliner
                        Theater-Aufführungen}\pwindex{Goldmann, Paul 31.\,1.\,1865 Breslau – 25.\,9.\,1935 Wien@\textsc{Goldmann, Paul} (31.\,1.\,1865 Breslau – 25.\,9.\,1935 Wien), \emph{Schriftsteller, Journalist}!»neue Richtung«. Polemische Aufsätze über Berliner Theater-Aufführungen@\strich\emph{Die »neue Richtung«. Polemische Aufsätze über Berliner Theater-Aufführungen}|pwk}. Wien\oindex{Wien@\textbf{Wien}, \emph{Verwaltungsgebiet}|pwk}: \emph{Buchhandlung L. Rosner}\orgindex{Buchhandlung L. Rosner@Buchhandlung L. Rosner|pwk}{ }1902, vordatiert auf 1903.}}}\label{K_L03228-2} und dadurch manchmal kränken muß.\pend
           
\pstart
           {\pb}Daß die \label{K_L03228-3v}\edtext{\textsc{Sorma\pwindex{Sorma, Agnes 17.\,5.\,1862 Breslau – 10.\,2.\,1927 Crown King@\textsc{Sorma, Agnes} (17.\,5.\,1862 Breslau – 10.\,2.\,1927 Crown King), \emph{Schauspielerin}|pw}}}{\lemma{\textnormal{\emph{Sorma}}}\Cendnote{\textnormal{Agnes Sorma\pwindex{Sorma, Agnes 17.\,5.\,1862 Breslau – 10.\,2.\,1927 Crown King@\textsc{Sorma, Agnes} (17.\,5.\,1862 Breslau – 10.\,2.\,1927 Crown King), \emph{Schauspielerin}|pwk} wäre Schnitzlers Wunschkandidatin für die Titelrolle in der
                  Inszenierung von \emph{Der Schleier der Beatrice}\pwindex{Schnitzler, Arthur 15.\,5.\,1862 Wien – 21.\,10.\,1931 ebd.@\textsc{Schnitzler, Arthur} (15.\,5.\,1862 Wien – 21.\,10.\,1931 ebd.), \emph{Schriftsteller, Mediziner}!Schleier der Beatrice. Schauspiel in fünf Akten@\strich\emph{Der Schleier der Beatrice. Schauspiel in fünf Akten}|pwk}
                  gewesen, gastierte aber am \emph{Berliner Theater}\orgindex{Berliner Theater@Berliner Theater|pwk},
                  während die Inszenierung von \emph{Der Schleier der
                     Beatrice}\pwindex{Schnitzler, Arthur 15.\,5.\,1862 Wien – 21.\,10.\,1931 ebd.@\textsc{Schnitzler, Arthur} (15.\,5.\,1862 Wien – 21.\,10.\,1931 ebd.), \emph{Schriftsteller, Mediziner}!Schleier der Beatrice. Schauspiel in fünf Akten@\strich\emph{Der Schleier der Beatrice. Schauspiel in fünf Akten}|pwk} am 7. 3. 1903 am \emph{Deutschen Theater
                     Berlin}\orgindex{Deutsches Theater Berlin@Deutsches Theater Berlin|pwk} stattfand. Vgl. A. S.: \emph{Tagebuch}, 19. 10. 1902.}}}\label{K_L03228-3} nicht zu haben iſt, iſt{ }ſehr bedauerlich. Jetzt rathe
               ich \strikeout{\textcolor{gray}{ſelbſt}} ganz entſchieden zum »Deutſchen Theater\orgindex{Deutsches Theater Berlin@Deutsches Theater Berlin|pw}«.
               Da Du{ }ſelbſt die Proben\pwindex{Schnitzler, Arthur 15.\,5.\,1862 Wien – 21.\,10.\,1931 ebd.@\textsc{Schnitzler, Arthur} (15.\,5.\,1862 Wien – 21.\,10.\,1931 ebd.), \emph{Schriftsteller, Mediziner}!Schleier der Beatrice. Schauspiel in fünf Akten@\strich\emph{Der Schleier der Beatrice. Schauspiel in fünf Akten}|pwv} leiten
               wirſt, iſt eine Chance mehr, daß die Aufführung beſſer wird als die der »\textsc{\label{K_L03228-4v}\edtext{Monna Vanna\pwindex{Maeterlinck, Maurice 29.\,8.\,1862 Gent – 6.\,5.\,1949 Nizza@\textsc{Maeterlinck, Maurice} (29.\,8.\,1862 Gent – 6.\,5.\,1949 Nizza), \emph{Schriftsteller}!Monna Vanna. Schauspiel in drei Akten@\strich\emph{Monna Vanna. Schauspiel in drei Akten}|pw}}{\lemma{\textnormal{\emph{Monna Vanna}}}\Cendnote{\textnormal{Vgl. A. S.: \emph{Tagebuch}, 24. 11. 1902 und 12. 12. 1902.
                  }}}\label{K_L03228-4}}«, bei deren Vorbereitung der Dichter\pwindex{Maeterlinck, Maurice 29.\,8.\,1862 Gent – 6.\,5.\,1949 Nizza@\textsc{Maeterlinck, Maurice} (29.\,8.\,1862 Gent – 6.\,5.\,1949 Nizza), \emph{Schriftsteller}|pwv} nicht mitgewirkt hat. Komm’ nur zu den \label{K_L03228-5v}\edtext{Proben}{\lemma{\textnormal{\emph{Proben}}}\Cendnote{\textnormal{Schnitzler kam am 22. 2. 1903 in Berlin\oindex{Berlin@\textbf{Berlin}, \emph{Hauptstadt}|pwk} an. Zwischen 23. 2. 1903 und 6. 3. 1903 war er,
                  abgesehen von einer Pause am Sonntag und Mittwoch vor der Premiere\pwindex{Schnitzler, Arthur 15.\,5.\,1862 Wien – 21.\,10.\,1931 ebd.@\textsc{Schnitzler, Arthur} (15.\,5.\,1862 Wien – 21.\,10.\,1931 ebd.), \emph{Schriftsteller, Mediziner}!Schleier der Beatrice. Schauspiel in fünf Akten@\strich\emph{Der Schleier der Beatrice. Schauspiel in fünf Akten}|pwkv}, täglich bei den Proben.}}}\label{K_L03228-5} recht bald nach
                  Berlin\oindex{Berlin@\textbf{Berlin}, \emph{Hauptstadt}|pw} und {\pb}bringe Dir gleich das Geld mit, um Dir die gewiſſe \strikeout{klei\textcolor{gray}{e}} kleine \label{K_L03228-6v}\edtext{Villa\oindex{Haus Wiesenstein@\textbf{Haus Wiesenstein}, \emph{Wohngebäude}|pwv} im Grunewald\oindex{Grunewald@\textbf{Grunewald}, \emph{Ehemaliger Ort}|pw}}{\lemma{\textnormal{\emph{Villa im Grunewald}}}\Cendnote{\textnormal{Die Stelle liest sich, als hätte Schnitzler Gefallen an einem bestimmten Haus
                  in der bevorzugt von Reichen bewohnten Kolonie Grunewald\oindex{Grunewald@\textbf{Grunewald}, \emph{Ehemaliger Ort}|pwk} geäußert, die bis 1920 außerhalb Berlins\oindex{Berlin@\textbf{Berlin}, \emph{Hauptstadt}|pwk} lag.}}}\label{K_L03228-6} zu
               kaufen.\pend
           
\pstart
           Daß Dein Sohn\pwindex{Schnitzler, Heinrich 9.\,8.\,1902 Hinterbrühl – 12.\,7.\,1982 Wien@\textsc{Schnitzler, Heinrich} (9.\,8.\,1902 Hinterbrühl – 12.\,7.\,1982 Wien), \emph{Regisseur, Schauspieler}|pwv} gedeiht, freut
               mich zu hören. Wenn er{ }ſo viel Symptome von Intelligenz zeigt, wird er{ }ſicherlich ein
               Kritiker werden und gegen die »neue Richtung\pwindex{Goldmann, Paul 31.\,1.\,1865 Breslau – 25.\,9.\,1935 Wien@\textsc{Goldmann, Paul} (31.\,1.\,1865 Breslau – 25.\,9.\,1935 Wien), \emph{Schriftsteller, Journalist}!»neue Richtung«. Polemische Aufsätze über Berliner Theater-Aufführungen@\strich\emph{Die »neue Richtung«. Polemische Aufsätze über Berliner Theater-Aufführungen}|pwv}« auftreten. Grüße ihn und{ }ſeine Mutter\pwindex{Schnitzler, Olga 17.\,1.\,1882 Wien – 13.\,1.\,1970 Lugano@\textsc{Schnitzler, Olga} (17.\,1.\,1882 Wien – 13.\,1.\,1970 Lugano), \emph{Schauspielerin, Sängerin}|pwv} vielmals von mir.\pend
           
\pstart
           Beſprechungen über mein Buch\pwindex{Goldmann, Paul 31.\,1.\,1865 Breslau – 25.\,9.\,1935 Wien@\textsc{Goldmann, Paul} (31.\,1.\,1865 Breslau – 25.\,9.\,1935 Wien), \emph{Schriftsteller, Journalist}!»neue Richtung«. Polemische Aufsätze über Berliner Theater-Aufführungen@\strich\emph{Die »neue Richtung«. Polemische Aufsätze über Berliner Theater-Aufführungen}|pwv}
               kann ich Dir nicht {\pb}ſchicken, weil keine erſcheinen.
               Es wird todtgeſchwiegen, von den Gegnern wie von den Freunden.\pend
           
\pstart
           Viele herzliche Grüße! {\\[\baselineskip]}Dein {\\[\baselineskip]}\spacefill\mbox{Paul Goldm}\pend
           \leftskip=0em{}\selectlanguage{ngerman}\endnumbering\briefempfaengerindex{Schnitzler, Arthur@\textsc{Schnitzler, Arthur}!zzzGoldmann, Paul@\emph{von Paul Goldmann}!1902-10-261@{26. 10. [1902]}|)be}\mylabel{L03228h}  \newcommand{\dateiname}{L03228}\newcommand{\titel}{Paul Goldmann an Arthur Schnitzler, 26. 10. [1902]}\newcommand{\editorInnen}{Martin Anton Müller und Laura Untner}%% latex-leseansicht-abspann.tex
%% Abspann für die Leseansicht.
%% Der Schalter \ifkorrekturansicht ist bereits durch den Vorspann gesetzt.

%% latex-abspann.tex
%% Gemeinsamer Abspann für Korrekturansicht und Leseansicht.
%% Setzt den Schalter \ifkorrekturansicht voraus (gesetzt in den
%% einbindenden Dateien latex-korrekturansicht-abspann.tex bzw.
%% latex-leseansicht-abspann.tex).
%% ---------------------------------------------------------------

\normalsize

% Das esempio-Environment wird nur in der Leseansicht benötigt
\ifkorrekturansicht\else
\newenvironment{esempio}[3]%
{
    \vspace{1.5ex}
    \rlap{\underline{#1}}
    \par
    \setlength{\parindent}{0cm}
    \nopagebreak
    \leftskip=#2cm
    \rightskip=#3cm
}
{
    \par
}
\fi

\doendnotes{C}
\bigskip
\vfill

\clearpage

\footnotesize

\ifkorrekturansicht
  \lohead{\textsc{register}}
\fi

% theindex-Environment neu definieren ohne reledmac
\makeatletter
\renewenvironment{theindex}{%
  \ifkorrekturansicht
    \section*{\indexname}%
  \else
    \subsubsection*{Index der erwähnten Entitäten}%
  \fi
  \setlength{\parindent}{0pt}%
  \setlength{\parskip}{0pt plus 0.3pt}%
  \let\item\@idxitem
}{%
  \ifkorrekturansicht\clearpage\fi
}
\makeatother

\IfFileExists{\jobname-pw.ind}{\input{\jobname-pw.ind}}{}

% Quellenangabe nur in der Leseansicht
\ifkorrekturansicht\else
% Fallback-Definitionen, falls die .tex-Datei \titel etc. nicht gesetzt hat
\providecommand{\titel}{}
\providecommand{\editorInnen}{}
\providecommand{\dateiname}{\jobname}

\vspace{3cm}

\vfill

\footnotesize
\textsc{Quelle}: \titel. Herausgegeben von {\editorInnen}. In: \emph{Arthur Schnitzler: Briefwechsel mit Autorinnen und Autoren}.
 Digitale Edition, https://schnitzler-briefe.acdh.oeaw.ac.at/{\dateiname}.html (Stand \today)
\fi

\end{document}


