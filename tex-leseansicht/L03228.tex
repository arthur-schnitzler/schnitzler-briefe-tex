%% latex-leseansicht-vorspann.tex
%% Vorspann für die Leseansicht.
%% Lädt die gemeinsame Datei latex-vorspann.tex mit nicht gesetztem Schalter.

\newif\ifkorrekturansicht
\korrekturansichtfalse

\input{../tex-inputs/latex-vorspann}

\begin{center}
            \textcolor{red}{ENTWURF, NICHT FERTIG KORRIGIERT}
                      \end{center}
            
         \renewcommand{\erwaehnteOrte}{Orte: Berlin, Dessauer Straße, Wien}
         \renewcommand{\erwaehnteWerke}{}
               \section[ Paul Goldmann an Arthur Schnitzler, 26. 10. {[}1902{]}]{ Paul Goldmann an Arthur Schnitzler, 26. 10. {[}1902{]}}\nopagebreak\mylabel{v}\rehead{ }\begin{ledgroupsized}[t]{13cm}\normalsize\beginnumbering \toendnotes[C]{\smallbreak\pagebreak[2]} \Standort{DLA, A:Schnitzler, HS.NZ85.1.3172.}
\physDesc{Brief, 1 Blatt, 4 Seiten
\newline{}Handschrift: blaue Tinte, deutsche Kurrent
\newline{}Schnitzler: mit Bleistift das Jahr »{[}1{]}902«
                                            vermerkt }\toendnotes[C]{\smallbreak}\pstart
           \noindent{}\raggedleft{}{\pb}\textcolor{gray}{\textbf{DESSAUERSTRASSE 19}}\oindex{Dessauer Strasse@\textbf{Dessauer Straße}|pw}\pend
           \pstart
           Berlin\oindex{Berlin@\textbf{Berlin}|pw}, 26. Oktober.\pend
           \pstart\center{}Mein lieber Freund,\pend\pstart
           Ich danke Dir vielmals für Deinen lieben Bief, der mich ſehr erfreut hat. Was Du
                    von \label{XXXXv}\edtext{Agnetendorf\textcolor{red}{\textsuperscript{\textbf{KEY}}}[Kommentar:
                        Besuch\u0020bei\u0020Hauptmann?]}{\lemma{\textnormal{\emph{XXXX Lemmafehler}}}\Cendnote{\textnormal{}}}\label{XXXX} erzählſt, hat mich natürlich ganz
                    beſonders intereſſirt. Es thut mir aufrichtig leid, daß ich einen Mann\textcolor{red}{\textsuperscript{\textbf{KEY}}}, den Du als ſo ſympathiſch ſchilderſt,
                    öffentlich bekämpfen und dadurch manchmal kränken muß. {\pb}\pend
           \pstart
           Daß die \textsc{Sorma\textcolor{red}{\textsuperscript{\textbf{KEY}}}} nicht zu haben iſt, iſt ſehr bedauerlich. Jetzt rathe ich \strikeout{\textcolor{gray}{ſelbſt}} ganz entſchieden zum »Deutſchen\textcolor{red}{\textsuperscript{\textbf{KEY}}}Theater\textcolor{red}{\textsuperscript{\textbf{KEY}}}«. Da Du ſelbſt die Proben leiten wirſt, iſt
                    eine Chance mehr, daß die Aufführung beſſer wird als die der »\textsc{Monna Vanna\textcolor{red}{\textsuperscript{\textbf{KEY}}}}«, bei deren Vorbereitung der Dichter\textcolor{red}{\textsuperscript{\textbf{KEY}}} nicht
                    mitgewirkt hat. Komm’ nur zu den Proben recht bald nach Berlin\textcolor{red}{\textsuperscript{\textbf{KEY}}} und {\pb} bringe Dir gleich das
                    Geld mit, um Dir die gewiſſe \strikeout{klei\textcolor{gray}{e}} kleine Villa\textcolor{red}{\textsuperscript{\textbf{KEY}}} im Grunewald\textcolor{red}{\textsuperscript{\textbf{KEY}}} zu kaufen. \pend
           \pstart
           Daß Dein Sohn\textcolor{red}{\textsuperscript{\textbf{KEY}}} gedeiht, freut mich zu hören. Wenn er
                    ſo viel Symptome von Intelligenz zeigt, wird er ſicherlich ein Kritiker werden
                    und gegen die »neue Richtung« auftreten. Grüße ihn und ſeine Mutter\textcolor{red}{\textsuperscript{\textbf{KEY}}} vielmals von mir. \pend
           \pstart
           Beſprechungen über mein Buch\textcolor{red}{\textsuperscript{\textbf{KEY}}} kann ich Dir nicht
                        {\pb} ſchicken, weil keine erſcheinen. Es wird
                    todtgeſchwiegen, von den Gegnern wie von den Freunden. {\\[\baselineskip]}Viele herzliche
                    Grüße!\pend
           \leftskip=0em{}\pstart
           {\\[\baselineskip]}Dein\pend
           \leftskip=0em{}\pstart
           {\\[\baselineskip]}\spacefill\mbox{Paul Goldm }\pend
           \leftskip=0em{}
         
         \endnumbering\mylabel{h}\end{ledgroupsized}\begin{anhang}\end{anhang}\newcommand{\dateiname}{L03228}\newcommand{\titel}{Paul Goldmann an Arthur Schnitzler, 26. 10. [1902]}\newcommand{\editorInnen}{Martin Anton Müller und Laura Untner}%% latex-leseansicht-abspann.tex
%% Abspann für die Leseansicht.
%% Der Schalter \ifkorrekturansicht ist bereits durch den Vorspann gesetzt.

%% latex-abspann.tex
%% Gemeinsamer Abspann für Korrekturansicht und Leseansicht.
%% Setzt den Schalter \ifkorrekturansicht voraus (gesetzt in den
%% einbindenden Dateien latex-korrekturansicht-abspann.tex bzw.
%% latex-leseansicht-abspann.tex).
%% ---------------------------------------------------------------

\normalsize

% Das esempio-Environment wird nur in der Leseansicht benötigt
\ifkorrekturansicht\else
\newenvironment{esempio}[3]%
{
    \vspace{1.5ex}
    \rlap{\underline{#1}}
    \par
    \setlength{\parindent}{0cm}
    \nopagebreak
    \leftskip=#2cm
    \rightskip=#3cm
}
{
    \par
}
\fi

\doendnotes{C}
\bigskip
\vfill

\clearpage

\footnotesize

\ifkorrekturansicht
  \lohead{\textsc{register}}
\fi

% theindex-Environment neu definieren ohne reledmac
\makeatletter
\renewenvironment{theindex}{%
  \ifkorrekturansicht
    \section*{\indexname}%
  \else
    \subsubsection*{Index der erwähnten Entitäten}%
  \fi
  \setlength{\parindent}{0pt}%
  \setlength{\parskip}{0pt plus 0.3pt}%
  \let\item\@idxitem
}{%
  \ifkorrekturansicht\clearpage\fi
}
\makeatother

\IfFileExists{\jobname-pw.ind}{\input{\jobname-pw.ind}}{}

% Quellenangabe nur in der Leseansicht
\ifkorrekturansicht\else
% Fallback-Definitionen, falls die .tex-Datei \titel etc. nicht gesetzt hat
\providecommand{\titel}{}
\providecommand{\editorInnen}{}
\providecommand{\dateiname}{\jobname}

\vspace{3cm}

\vfill

\footnotesize
\textsc{Quelle}: \titel. Herausgegeben von {\editorInnen}. In: \emph{Arthur Schnitzler: Briefwechsel mit Autorinnen und Autoren}.
 Digitale Edition, https://schnitzler-briefe.acdh.oeaw.ac.at/{\dateiname}.html (Stand \today)
\fi

\end{document}


      