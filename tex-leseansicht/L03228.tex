%% latex-leseansicht-vorspann.tex
%% Vorspann für die Leseansicht.
%% Lädt die gemeinsame Datei latex-vorspann.tex mit nicht gesetztem Schalter.

\newif\ifkorrekturansicht
\korrekturansichtfalse

\input{../tex-inputs/latex-vorspann}


         
         \renewcommand{\erwaehntePersonen}{Personen: Paul Goldmann, Gerhart Hauptmann, Maurice Maeterlinck, Heinrich Schnitzler, Olga Schnitzler, Agnes Sorma}
         \renewcommand{\erwaehnteInstitutionen}{Institutionen: Berliner Theater, Buchhandlung L. Rosner, Deutsches Theater Berlin}
         \renewcommand{\erwaehnteOrte}{Orte: Agnetendorf, Berlin, Dessauer Straße, Deutsches Theater Berlin, Grunewald, Haus Wiesenstein, Wien}
         \renewcommand{\erwaehnteWerke}{Werke: Der Schleier der Beatrice. Schauspiel in fünf Akten, Die »neue Richtung«. Polemische Aufsätze über Berliner Theater-Aufführungen, Monna Vanna. Schauspiel in drei Akten}
               \section[ Paul Goldmann an Arthur Schnitzler, 26. 10. {[}1902{]}]{ Paul Goldmann an Arthur Schnitzler, 26. 10. {[}1902{]}}\nopagebreak\mylabel{v}\rehead{ }\begin{ledgroupsized}[t]{13cm}\normalsize\beginnumbering \toendnotes[C]{\smallbreak\pagebreak[2]} \Standort{DLA, A:Schnitzler, HS.NZ85.1.3172.}
\physDesc{Brief, 1 Blatt, 4 Seiten, 1137 Zeichen
\newline{}Handschrift: blaue Tinte, deutsche Kurrent
\newline{}Schnitzler: mit Bleistift das Jahr »902« vermerkt }\toendnotes[C]{\smallbreak}\pstart
           \noindent{}\raggedleft{}{\pb}\textcolor{gray}{\textbf{DESSAUERSTRASSE 19}}\oindex{Dessauer Strasse@\textbf{Dessauer Straße}|pw}\pend
           \pstart
           Berlin\oindex{Berlin@\textbf{Berlin}|pw}, 26. Oktober.\pend
           \pstart\center{}Mein lieber Freund,\pend\pstart
           Ich danke Dir vielmals für Deinen lieben Brief, der mich ſehr erfreut hat. Was Du von
                  \label{K_L03228-1v}\edtext{Agnetendorf\oindex{Agnetendorf@\textbf{Agnetendorf}|pw}}{\lemma{\textnormal{\emph{Agnetendorf}}}\Cendnote{\textnormal{Schnitzler\pwindex{Schnitzler, Arthur 15.05.1862 – 21.10.1931@\textsc{Schnitzler, Arthur} (15.05.1862 – 21.10.1931), \emph{Schriftsteller, Mediziner}|pwk} war von 19. 10. 1902 bis zum
                     20. 10. 1902 bei
                     Gerhart Hauptmann\pwindex{Hauptmann, Gerhart 15.11.1862 – 06.06.1946@\textsc{Hauptmann, Gerhart} (15.11.1862 – 06.06.1946), \emph{Schriftsteller}|pwk} zu Besuch
                  gewesen.}}}\label{K_L03228-1h} erzählſt, hat mich natürlich ganz beſonders intereſſirt. Es thut
               mir aufrichtig leid, daß ich einen Mann\pwindex{Hauptmann, Gerhart 15.11.1862 – 06.06.1946@\textsc{Hauptmann, Gerhart} (15.11.1862 – 06.06.1946), \emph{Schriftsteller}|pwv}, den Du als ſo ſympathiſch ſchilderſt, \label{K_L03228-2v}\edtext{öffentlich bekämpfen}{\lemma{\textnormal{\emph{öffentlich bekämpfen}}}\Cendnote{\textnormal{Ausdruck findet das in seiner eine Woche zuvor erschienenen
                     Feuilletonsammlung\pwindex{Goldmann, Paul 31.01.1865 – 25.09.1935@\textsc{Goldmann, Paul} (31.01.1865 – 25.09.1935), \emph{Schriftsteller, Journalist}!»neue Richtung«. Polemische Aufsaetze ueber Berliner Theater-Auffuehrungen1902-10-17@\strich\emph{Die »neue Richtung«. Polemische Aufsätze über Berliner Theater-Aufführungen} {[}1902-10-17{]}|pwkv}: Paul Goldmann\pwindex{Goldmann, Paul 31.01.1865 – 25.09.1935@\textsc{Goldmann, Paul} (31.01.1865 – 25.09.1935), \emph{Schriftsteller, Journalist}|pwk}: \emph{Die »neue Richtung«. Polemische Aufsätze über Berliner
                        Theater-Aufführungen}\pwindex{Goldmann, Paul 31.01.1865 – 25.09.1935@\textsc{Goldmann, Paul} (31.01.1865 – 25.09.1935), \emph{Schriftsteller, Journalist}!»neue Richtung«. Polemische Aufsaetze ueber Berliner Theater-Auffuehrungen1902-10-17@\strich\emph{Die »neue Richtung«. Polemische Aufsätze über Berliner Theater-Aufführungen} {[}1902-10-17{]}|pwk}. Wien\oindex{Wien@\textbf{Wien}|pwk}: \emph{Buchhandlung L. Rosner}\orgindex{Buchhandlung L. Rosner@Buchhandlung L. Rosner|pwk}{ }1902, vordatiert auf 1903.}}}\label{K_L03228-2h} und dadurch manchmal kränken muß.\pend
           \pstart
           {\pb}Daß die \label{K_L03228-3v}\edtext{\textsc{Sorma\pwindex{Sorma, Agnes 17.05.1862 – 10.02.1927@\textsc{Sorma, Agnes} (17.05.1862 – 10.02.1927), \emph{Schauspielerin}|pw}}}{\lemma{\textnormal{\emph{Sorma}}}\Cendnote{\textnormal{Agnes
                     Sorma\pwindex{Sorma, Agnes 17.05.1862 – 10.02.1927@\textsc{Sorma, Agnes} (17.05.1862 – 10.02.1927), \emph{Schauspielerin}|pwk} wäre Schnitzler\pwindex{Schnitzler, Arthur 15.05.1862 – 21.10.1931@\textsc{Schnitzler, Arthur} (15.05.1862 – 21.10.1931), \emph{Schriftsteller, Mediziner}|pwk}s
                  Wunschkandidatin für die Titelrolle in der Inszenierung von \emph{Der Schleier der Beatrice}\pwindex{Schnitzler, Arthur 15.05.1862 – 21.10.1931@\textsc{Schnitzler, Arthur} (15.05.1862 – 21.10.1931), \emph{Schriftsteller, Mediziner}!Schleier der Beatrice. Schauspiel in fuenf Akten1900-12-01@\strich\emph{Der Schleier der Beatrice. Schauspiel in fünf Akten} {[}1900-12-01{]}|pwk} am Deutschen Theater Berlin\oindex{Deutsches Theater Berlin@\textbf{Deutsches Theater Berlin}|pwk} gewesen. Sie gastierte zur Zeit der Premiere,
                     Anfang März 1903, am \emph{Berliner Theater}\orgindex{Berliner Theater@Berliner Theater|pwk}. Vgl. A. S.: \emph{Tagebuch}, 19. 10. 1902.}}}\label{K_L03228-3h} nicht zu haben iſt, iſt ſehr bedauerlich. Jetzt rathe
               ich \strikeout{\textcolor{gray}{ſelbſt}} ganz entſchieden zum \label{K_L03228-4v}\edtext{»Deutſchen Theater\orgindex{Deutsches Theater Berlin@Deutsches Theater Berlin|pw}«}{\lemma{\textnormal{\emph{»Deutſchen Theater«}}}\Cendnote{\textnormal{für die Berlin\oindex{Berlin@\textbf{Berlin}|pwk}er Premiere
                  von \emph{Der Schleier der Beatrice}\pwindex{Schnitzler, Arthur 15.05.1862 – 21.10.1931@\textsc{Schnitzler, Arthur} (15.05.1862 – 21.10.1931), \emph{Schriftsteller, Mediziner}!Schleier der Beatrice. Schauspiel in fuenf Akten1900-12-01@\strich\emph{Der Schleier der Beatrice. Schauspiel in fünf Akten} {[}1900-12-01{]}|pwk}, wo sie am 7. 3. 1903 auch
                  stattfand; siehe zu \emph{Monna Vanna}\pwindex{Maeterlinck, Maurice 29.08.1862 – 06.05.1949@\textsc{Maeterlinck, Maurice} (29.08.1862 – 06.05.1949), \emph{Schriftsteller}!Monna Vanna. Schauspiel in drei Akten1903@\strich\emph{Monna Vanna. Schauspiel in drei Akten} {[}1903{]}|pwk} auch A. S.: \emph{Tagebuch}, 24. 11. 1902 und 12. 12. 1902}}}\label{K_L03228-4h}. Da Du ſelbſt die Proben\pwindex{Schnitzler, Arthur 15.05.1862 – 21.10.1931@\textsc{Schnitzler, Arthur} (15.05.1862 – 21.10.1931), \emph{Schriftsteller, Mediziner}!Schleier der Beatrice. Schauspiel in fuenf Akten1900-12-01@\strich\emph{Der Schleier der Beatrice. Schauspiel in fünf Akten} {[}1900-12-01{]}|pwv} leiten wirſt, iſt eine Chance mehr, daß die Aufführung beſſer wird
               als die der »\textsc{Monna Vanna\pwindex{Maeterlinck, Maurice 29.08.1862 – 06.05.1949@\textsc{Maeterlinck, Maurice} (29.08.1862 – 06.05.1949), \emph{Schriftsteller}!Monna Vanna. Schauspiel in drei Akten1903@\strich\emph{Monna Vanna. Schauspiel in drei Akten} {[}1903{]}|pw}}«, bei deren Vorbereitung der Dichter\pwindex{Maeterlinck, Maurice 29.08.1862 – 06.05.1949@\textsc{Maeterlinck, Maurice} (29.08.1862 – 06.05.1949), \emph{Schriftsteller}|pwv} nicht mitgewirkt hat. Komm’ nur zu den \label{K_L03228-5v}\edtext{Proben}{\lemma{\textnormal{\emph{Proben}}}\Cendnote{\textnormal{Schnitzler\pwindex{Schnitzler, Arthur 15.05.1862 – 21.10.1931@\textsc{Schnitzler, Arthur} (15.05.1862 – 21.10.1931), \emph{Schriftsteller, Mediziner}|pwk} kam am 22. 2. 1903 in Berlin\oindex{Berlin@\textbf{Berlin}|pwk} an. Zwischen 23. 2. 1903 und 6. 3. 1903 war er,
                  abgesehen von einer Pause am Sonntag und Mittwoch vor der Premiere\pwindex{Schnitzler, Arthur 15.05.1862 – 21.10.1931@\textsc{Schnitzler, Arthur} (15.05.1862 – 21.10.1931), \emph{Schriftsteller, Mediziner}!Schleier der Beatrice. Schauspiel in fuenf Akten1900-12-01@\strich\emph{Der Schleier der Beatrice. Schauspiel in fünf Akten} {[}1900-12-01{]}|pwkv}, täglich bei den Proben.}}}\label{K_L03228-5h} recht bald nach
                  Berlin\oindex{Berlin@\textbf{Berlin}|pw} und {\pb}bringe Dir gleich das Geld mit, um Dir die gewiſſe \strikeout{klei\textcolor{gray}{e}} kleine \label{K_L03228-6v}\edtext{Villa\oindex{Haus Wiesenstein@\textbf{Haus Wiesenstein}|pwv} im Grunewald\oindex{Grunewald@\textbf{Grunewald}|pw}}{\lemma{\textnormal{\emph{Villa im Grunewald}}}\Cendnote{\textnormal{Die Stelle liest sich, als hätte Schnitzler\pwindex{Schnitzler, Arthur 15.05.1862 – 21.10.1931@\textsc{Schnitzler, Arthur} (15.05.1862 – 21.10.1931), \emph{Schriftsteller, Mediziner}|pwk} Gefallen an einem bestimmten Haus
                  im bevorzugt von Reichen bewohnten Ortsteil Grunewald\oindex{Grunewald@\textbf{Grunewald}|pwk} geäußert. Eventuell handelte es sich auch um eine Anspielung
                  auf Gerhart Hauptmann\pwindex{Hauptmann, Gerhart 15.11.1862 – 06.06.1946@\textsc{Hauptmann, Gerhart} (15.11.1862 – 06.06.1946), \emph{Schriftsteller}|pwk}s dortiges Wohnhaus\oindex{Haus Wiesenstein@\textbf{Haus Wiesenstein}|pwkv}.}}}\label{K_L03228-6h} zu
               kaufen.\pend
           \pstart
           Daß Dein Sohn\pwindex{Schnitzler, Heinrich 09.08.1902 – 12.07.1982@\textsc{Schnitzler, Heinrich} (09.08.1902 – 12.07.1982), \emph{Regisseur, Schauspieler}|pwv} gedeiht, freut
               mich zu hören. Wenn er ſo viel Symptome von Intelligenz zeigt, wird er ſicherlich ein
               Kritiker werden und gegen die »neue Richtung\pwindex{Goldmann, Paul 31.01.1865 – 25.09.1935@\textsc{Goldmann, Paul} (31.01.1865 – 25.09.1935), \emph{Schriftsteller, Journalist}!»neue Richtung«. Polemische Aufsaetze ueber Berliner Theater-Auffuehrungen1902-10-17@\strich\emph{Die »neue Richtung«. Polemische Aufsätze über Berliner Theater-Aufführungen} {[}1902-10-17{]}|pwv}« auftreten. Grüße ihn und ſeine Mutter\pwindex{Schnitzler, Olga 17.01.1882 – 13.01.1970@\textsc{Schnitzler, Olga} (17.01.1882 – 13.01.1970), \emph{Schauspielerin, Sängerin}|pwv} vielmals von mir.\pend
           \pstart
           Beſprechungen über mein Buch\pwindex{Goldmann, Paul 31.01.1865 – 25.09.1935@\textsc{Goldmann, Paul} (31.01.1865 – 25.09.1935), \emph{Schriftsteller, Journalist}!»neue Richtung«. Polemische Aufsaetze ueber Berliner Theater-Auffuehrungen1902-10-17@\strich\emph{Die »neue Richtung«. Polemische Aufsätze über Berliner Theater-Aufführungen} {[}1902-10-17{]}|pwv}
               kann ich Dir nicht {\pb}ſchicken, weil keine erſcheinen.
               Es wird todtgeſchwiegen, von den Gegnern wie von den Freunden.\pend
           \pstart
           Viele herzliche Grüße! {\\[\baselineskip]}Dein {\\[\baselineskip]}\spacefill\mbox{Paul Goldm}\pend
           \leftskip=0em{}
         
         \endnumbering\mylabel{h}\end{ledgroupsized}  \newcommand{\dateiname}{L03228}\newcommand{\titel}{Paul Goldmann an Arthur Schnitzler, 26. 10. [1902]}\newcommand{\editorInnen}{Martin Anton Müller und Laura Untner}%% latex-leseansicht-abspann.tex
%% Abspann für die Leseansicht.
%% Der Schalter \ifkorrekturansicht ist bereits durch den Vorspann gesetzt.

%% latex-abspann.tex
%% Gemeinsamer Abspann für Korrekturansicht und Leseansicht.
%% Setzt den Schalter \ifkorrekturansicht voraus (gesetzt in den
%% einbindenden Dateien latex-korrekturansicht-abspann.tex bzw.
%% latex-leseansicht-abspann.tex).
%% ---------------------------------------------------------------

\normalsize

% Das esempio-Environment wird nur in der Leseansicht benötigt
\ifkorrekturansicht\else
\newenvironment{esempio}[3]%
{
    \vspace{1.5ex}
    \rlap{\underline{#1}}
    \par
    \setlength{\parindent}{0cm}
    \nopagebreak
    \leftskip=#2cm
    \rightskip=#3cm
}
{
    \par
}
\fi

\doendnotes{C}
\bigskip
\vfill

\clearpage

\footnotesize

\ifkorrekturansicht
  \lohead{\textsc{register}}
\fi

% theindex-Environment neu definieren ohne reledmac
\makeatletter
\renewenvironment{theindex}{%
  \ifkorrekturansicht
    \section*{\indexname}%
  \else
    \subsubsection*{Index der erwähnten Entitäten}%
  \fi
  \setlength{\parindent}{0pt}%
  \setlength{\parskip}{0pt plus 0.3pt}%
  \let\item\@idxitem
}{%
  \ifkorrekturansicht\clearpage\fi
}
\makeatother

\IfFileExists{\jobname-pw.ind}{\input{\jobname-pw.ind}}{}

% Quellenangabe nur in der Leseansicht
\ifkorrekturansicht\else
% Fallback-Definitionen, falls die .tex-Datei \titel etc. nicht gesetzt hat
\providecommand{\titel}{}
\providecommand{\editorInnen}{}
\providecommand{\dateiname}{\jobname}

\vspace{3cm}

\vfill

\footnotesize
\textsc{Quelle}: \titel. Herausgegeben von {\editorInnen}. In: \emph{Arthur Schnitzler: Briefwechsel mit Autorinnen und Autoren}.
 Digitale Edition, https://schnitzler-briefe.acdh.oeaw.ac.at/{\dateiname}.html (Stand \today)
\fi

\end{document}


      