%% latex-leseansicht-vorspann.tex
%% Vorspann für die Leseansicht.
%% Lädt die gemeinsame Datei latex-vorspann.tex mit nicht gesetztem Schalter.

\newif\ifkorrekturansicht
\korrekturansichtfalse

\input{../tex-inputs/latex-vorspann}


\section[ Felix Salten an Arthur Schnitzler, 8. 2. 1908]{L03492 Felix Salten an Arthur Schnitzler,  8. 2. 1908}
\nopagebreak\mylabel{L03492v}
\rehead{ }\normalsize\beginnumbering\briefempfaengerindex{Schnitzler, Arthur@\textsc{Schnitzler, Arthur}!zzzSalten, Felix@\emph{von Felix Salten}!1908-02-081@{8. 2. 1908}|(be}
\toendnotes[C]{\smallbreak\pagebreak[2]}
\correspDesc{Versand  durch Felix Salten am 8. 2. 1908 in Wien
\newline{}Erhalt  durch Arthur Schnitzler im Zeitraum [9. 2. 1908
                  – 13. 2. 1908?] in Semmering}\toendnotes[C]{\smallbreak}
\Standort{CUL, Schnitzler, B 89, B 1.}
\physDesc{Postkarte, 699 Zeichen
\newline{}Handschrift: schwarze Tinte, lateinische Kurrent
\newline{}Versand: 1) Stempel: »\nobreak{}\oindex{I., Innere Stadt@\textbf{I., Innere Stadt}, \emph{Verwaltungsgebiet}|pwk}1/\textsubscript{1} Wien 1, 8. II. 08, 12\nobreak{}«.   2) mit Bleistift von unbekannter Hand der Vorname Schnitzlers in der Adressangabe gestrichen
\newline{}Schnitzler: mit Bleistift datiert: »8/2 908« 
\newline{}Ordnung: mit Bleistift von unbekannter Hand nummeriert: »{\pb}242« }\toendnotes[C]{\smallbreak}\pstart{}{\pb}Herrn D\textsuperscript{r} Arthur Schnitzler\pend{}\pstart{}Semmering\oindex{Semmering@\textbf{Semmering}, \emph{Verwaltungsgebiet}|pw}\pend{}\pstart{}Südbahnhotel\oindex{Südbahnhotel [Semmering]@\textbf{Südbahnhotel [Semmering]}, \emph{Hotel}|pw}\pend{}{\bigskip}\vspace{1em}
\pstart
           \noindent{}{\pb}Lieber, wir waren erst gegen 2\textsuperscript{h} in Wien\oindex{Wien@\textbf{Wien}, \emph{Verwaltungsgebiet}|pw}, \label{K_L03492-1v}\edtext{¾ 3}{\lemma{\textnormal{\emph{¾ 3}}}\Cendnote{\textnormal{14 Uhr 45}}}\label{K_L03492-1} in Heiligenstadt\oindex{Wien@\textbf{Wien}!XIX., Döbling@\textbf{XIX., Döbling}!Heiligenstadt@\textbf{Heiligenstadt}|pw}, wo wir
               essen mußten. Wir haben Ihrem Herrn Bruder\pwindex{Schnitzler, Julius 13.\,7.\,1865 Wien – 29.\,6.\,1939 ebd.@\textsc{Schnitzler, Julius} (13.\,7.\,1865 Wien – 29.\,6.\,1939 ebd.), \emph{Chirurg}|pwv} gleich telefonirt, fuhren auch ohne Verzögerung in
               die Stadt\oindex{I., Innere Stadt@\textbf{I., Innere Stadt}, \emph{Verwaltungsgebiet}|pwv}, aber bei dem heftigen
               Sturm kamen die Pferde nur schwer vorwärts. Und als wir mit einer Verspätung um \uline{10} Minuten in die \label{K_L03492-2v}\edtext{Biberstraße\oindex{Wien@\textbf{Wien}!I., Innere Stadt@\textbf{I., Innere Stadt}!Biberstraße@\textbf{Biberstraße}, \emph{Straße}|pw}}{\lemma{\textnormal{\emph{Biberstraße}}}\Cendnote{\textnormal{In der Biberstraße 8\oindex{Wien@\textbf{Wien}!I., Innere Stadt@\textbf{I., Innere Stadt}!Biberstraße@\textbf{Biberstraße}, \emph{Straße}|pwk} befand sich Julius
                     Schnitzlers\pwindex{Schnitzler, Julius 13.\,7.\,1865 Wien – 29.\,6.\,1939 ebd.@\textsc{Schnitzler, Julius} (13.\,7.\,1865 Wien – 29.\,6.\,1939 ebd.), \emph{Chirurg}|pwk}{ }Privatpraxis\oindex{Ordination Julius Schnitzler@\textbf{Ordination Julius Schnitzler}, \emph{Ordination}|pwk}.}}}\label{K_L03492-2} kamen, wurden wir nicht mehr angenommen. Mir that es sehr
               leid, umso mehr, als ich ja eigens wegen dieser Consultation um 10.17
               vom Semmering\oindex{Semmering@\textbf{Semmering}, \emph{Verwaltungsgebiet}|pw} weg bin und nicht mit dem
               Schnell-Zug.\pend
           
\pstart
           \label{K_L03492-3v}\edtext{Vielleicht komme ich am Montag}{\lemma{\textnormal{\emph{Vielleicht … Montag}}}\Cendnote{\textnormal{Das dürfte er nicht umgesetzt haben.}}}\label{K_L03492-3}{ }früh, oder um 2\textsuperscript{h.} von Brünn\oindex{Brünn@\textbf{Brünn}|pw} aus noch einmal für einen Tag
                  hinauf\oindex{Semmering@\textbf{Semmering}, \emph{Verwaltungsgebiet}|pwv}. Grüßen Sie \uline{Alle}, Ihre Frau\pwindex{Schnitzler, Olga 17.\,1.\,1882 Wien – 13.\,1.\,1970 Lugano@\textsc{Schnitzler, Olga} (17.\,1.\,1882 Wien – 13.\,1.\,1970 Lugano), \emph{Schauspielerin, Sängerin}|pwv}, Ihre Mama\pwindex{Schnitzler, Louise 8.\,7.\,1840 Kőszeg – 9.\,9.\,1911 Wien@\textsc{Schnitzler, Louise} (8.\,7.\,1840 Kőszeg – 9.\,9.\,1911 Wien)|pwv}, Hofmannsthal\pwindex{Hofmannsthal, Hugo von 1.\,2.\,1874 Wien – 15.\,7.\,1929 Rodaun@\textsc{Hofmannsthal, Hugo von} (1.\,2.\,1874 Wien – 15.\,7.\,1929 Rodaun), \emph{Schriftsteller}|pw},
                  Wassermann\pwindex{Wassermann, Jakob 10.\,3.\,1873 Fürth – 1.\,1.\,1934 Altaussee@\textsc{Wassermann, Jakob} (10.\,3.\,1873 Fürth – 1.\,1.\,1934 Altaussee), \emph{Schriftsteller}|pw} u. Frau Kainz\pwindex{Kainz, Margarethe 31.\,12.\,1885 Berlin – 12.\,2.\,1950 Wien@\textsc{Kainz, Margarethe} (31.\,12.\,1885 Berlin – 12.\,2.\,1950 Wien), \emph{Schauspielerin}|pw}. Herzlichst\pend
           \pstart Ihr \spacefill\mbox{Salten}\pend{}\selectlanguage{ngerman}\endnumbering\briefempfaengerindex{Schnitzler, Arthur@\textsc{Schnitzler, Arthur}!zzzSalten, Felix@\emph{von Felix Salten}!1908-02-081@{8. 2. 1908}|)be}\mylabel{L03492h}  \newcommand{\dateiname}{L03492}\newcommand{\titel}{Felix Salten an Arthur Schnitzler, 8. 2. 1908}\newcommand{\editorInnen}{Martin Anton Müller und Laura Untner}%% latex-leseansicht-abspann.tex
%% Abspann für die Leseansicht.
%% Der Schalter \ifkorrekturansicht ist bereits durch den Vorspann gesetzt.

%% latex-abspann.tex
%% Gemeinsamer Abspann für Korrekturansicht und Leseansicht.
%% Setzt den Schalter \ifkorrekturansicht voraus (gesetzt in den
%% einbindenden Dateien latex-korrekturansicht-abspann.tex bzw.
%% latex-leseansicht-abspann.tex).
%% ---------------------------------------------------------------

\normalsize

% Das esempio-Environment wird nur in der Leseansicht benötigt
\ifkorrekturansicht\else
\newenvironment{esempio}[3]%
{
    \vspace{1.5ex}
    \rlap{\underline{#1}}
    \par
    \setlength{\parindent}{0cm}
    \nopagebreak
    \leftskip=#2cm
    \rightskip=#3cm
}
{
    \par
}
\fi

\doendnotes{C}
\bigskip
\vfill

\clearpage

\footnotesize

\ifkorrekturansicht
  \lohead{\textsc{register}}
\fi

% theindex-Environment neu definieren ohne reledmac
\makeatletter
\renewenvironment{theindex}{%
  \ifkorrekturansicht
    \section*{\indexname}%
  \else
    \subsubsection*{Index der erwähnten Entitäten}%
  \fi
  \setlength{\parindent}{0pt}%
  \setlength{\parskip}{0pt plus 0.3pt}%
  \let\item\@idxitem
}{%
  \ifkorrekturansicht\clearpage\fi
}
\makeatother

\IfFileExists{\jobname-pw.ind}{\input{\jobname-pw.ind}}{}

% Quellenangabe nur in der Leseansicht
\ifkorrekturansicht\else
% Fallback-Definitionen, falls die .tex-Datei \titel etc. nicht gesetzt hat
\providecommand{\titel}{}
\providecommand{\editorInnen}{}
\providecommand{\dateiname}{\jobname}

\vspace{3cm}

\vfill

\footnotesize
\textsc{Quelle}: \titel. Herausgegeben von {\editorInnen}. In: \emph{Arthur Schnitzler: Briefwechsel mit Autorinnen und Autoren}.
 Digitale Edition, https://schnitzler-briefe.acdh.oeaw.ac.at/{\dateiname}.html (Stand \today)
\fi

\end{document}


