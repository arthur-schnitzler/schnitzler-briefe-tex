%% latex-korrekturansicht-vorspann.tex
%% Vorspann für die Korrekturansicht.
%% Lädt die gemeinsame Datei latex-vorspann.tex mit gesetztem Schalter.

\newif\ifkorrekturansicht
\korrekturansichttrue

\input{../tex-inputs/latex-vorspann}


\section[ Felix Salten an Arthur Schnitzler, 8. 2. 1908]{L03492 Felix Salten an Arthur Schnitzler, 8. 2. 1908}
\nopagebreak\mylabel{L03492v}
\rehead{ }\normalsize\beginnumbering\briefempfaengerindex{Schnitzler, Arthur@\textsc{Schnitzler, Arthur}!zzzSalten, Felix@\emph{von Felix Salten}!1908-02-081@{8. 2. 1908}|(be}
\toendnotes[C]{\smallbreak\pagebreak[2]}\Standort{CUL, Schnitzler, B 89, B 1.}
\physDesc{Postkarte, 699 Zeichen
\newline{}Handschrift: schwarze Tinte, lateinische Kurrent
\newline{}Versand: 1) Stempel: »\nobreak{}\oindex{I., Innere Stadt@\textbf{I., Innere Stadt}, \emph{A.ADM3}|pwk}1/\textsubscript{1} Wien 1, 8. II. 08, 12\nobreak{}«.   2) mit Bleistift von unbekannter Hand der Vorname Schnitzlers in der Adressangabe gestrichen
\newline{}Schnitzler: mit Bleistift datiert: »8/2 908« 
\newline{}Ordnung: mit Bleistift von unbekannter Hand nummeriert: »{\pb}242« }\toendnotes[C]{\smallbreak}\pstart{}{\pb}Herrn D\textsuperscript{r} Arthur Schnitzler\pend{}\pstart{}Semmering\oindex{Semmering@\textbf{Semmering}, \emph{A.ADM3}|pw}\pend{}\pstart{}Südbahnhotel\oindex{Suedbahnhotel [Semmering]@\textbf{Südbahnhotel [Semmering]}, \emph{Hotel (K.HTL)}|pw}\pend{}{\bigskip}\vspace{1em}
\pstart
           \noindent{}{\pb}Lieber, wir waren erst gegen 2\textsuperscript{h} in Wien\oindex{Wien@\textbf{Wien}, \emph{A.ADM2}|pw}, \label{K_L03492-1v}\edtext{¾ 3}{\lemma{\textnormal{\emph{¾ 3}}}\Cendnote{\textnormal{14 Uhr 45}}}\label{K_L03492-1} in Heiligenstadt\oindex{Heiligenstadt@\textbf{Heiligenstadt}, \emph{P.PPL}|pw}, wo wir
               essen mußten. Wir haben Ihrem Herrn Bruder\pwindex{Schnitzler, Julius 13.07.1865 – 29.06.1939@\textsc{Schnitzler, Julius} (13.07.1865 – 29.06.1939), \emph{Chirurg/Chirurgin}|pwv} gleich telefonirt, fuhren auch ohne Verzögerung in
               die Stadt\oindex{I., Innere Stadt@\textbf{I., Innere Stadt}, \emph{A.ADM3}|pwv}, aber bei dem heftigen
               Sturm kamen die Pferde nur schwer vorwärts. Und als wir mit einer Verspätung um \uline{10} Minuten in die \label{K_L03492-2v}\edtext{Biberstraße\oindex{Biberstrasse@\textbf{Biberstraße}, \emph{Straße (K.STR)}|pw}}{\lemma{\textnormal{\emph{Biberstraße}}}\Cendnote{\textnormal{In der Biberstraße 8\oindex{Biberstrasse@\textbf{Biberstraße}, \emph{Straße (K.STR)}|pwk} befand sich Julius
                     Schnitzlers\pwindex{Schnitzler, Julius 13.07.1865 – 29.06.1939@\textsc{Schnitzler, Julius} (13.07.1865 – 29.06.1939), \emph{Chirurg/Chirurgin}|pwk}{ }Privatpraxis\oindex{Ordination Julius Schnitzler@\textbf{Ordination Julius Schnitzler}, \emph{Ordination}|pwk}.}}}\label{K_L03492-2} kamen, wurden wir nicht mehr angenommen. Mir that es sehr
               leid, umso mehr, als ich ja eigens wegen dieser Consultation um 10.17
               vom Semmering\oindex{Semmering@\textbf{Semmering}, \emph{A.ADM3}|pw} weg bin und nicht mit dem
               Schnell-Zug.\pend
           
\pstart
           \label{K_L03492-3v}\edtext{Vielleicht komme ich am Montag}{\lemma{\textnormal{\emph{Vielleicht … Montag}}}\Cendnote{\textnormal{Das dürfte er nicht umgesetzt haben.}}}\label{K_L03492-3}{ }früh, oder um 2\textsuperscript{h.} von Brünn\oindex{Bruenn@\textbf{Brünn}, \emph{P.PPLA}|pw} aus noch einmal für einen Tag
                  hinauf\oindex{Semmering@\textbf{Semmering}, \emph{A.ADM3}|pwv}. Grüßen Sie \uline{Alle}, Ihre Frau\pwindex{Schnitzler, Olga 17.01.1882 – 13.01.1970@\textsc{Schnitzler, Olga} (17.01.1882 – 13.01.1970), \emph{Schauspieler/Schauspielerin, Sänger/Sängerin}|pwv}, Ihre Mama\pwindex{Schnitzler, Louise 1840-07-08 – 1911-09-09@\textsc{Schnitzler, Louise} (1840-07-08 – 1911-09-09)|pwv}, Hofmannsthal\pwindex{Hofmannsthal, Hugo von 1874-02-01 – 1929-07-15@\textsc{Hofmannsthal, Hugo von} (1874-02-01 – 1929-07-15), \emph{Schriftsteller/Schriftstellerin}|pw},
                  Wassermann\pwindex{Wassermann, Jakob 10.03.1873 – 01.01.1934@\textsc{Wassermann, Jakob} (10.03.1873 – 01.01.1934), \emph{Schriftsteller/Schriftstellerin}|pw} u. Frau Kainz\pwindex{Kainz, Margarethe 13.12.1858 – 12.02.1950@\textsc{Kainz, Margarethe} (13.12.1858 – 12.02.1950), \emph{Schauspieler/Schauspielerin}|pw}. Herzlichst 
            \pend
           \pstart Ihr \spacefill\mbox{Salten}\pend{}\selectlanguage{ngerman}\endnumbering\briefempfaengerindex{Schnitzler, Arthur@\textsc{Schnitzler, Arthur}!zzzSalten, Felix@\emph{von Felix Salten}!1908-02-081@{8. 2. 1908}|)be}\mylabel{L03492h}  \normalsize

\doendnotes{C}
\bigskip
\vfill

\clearpage

\footnotesize

\lohead{\textsc{register}}

% Definiere theindex-Environment komplett neu ohne reledmac
\makeatletter
\renewenvironment{theindex}{%
  \section*{\indexname}%
  \setlength{\parindent}{0pt}%
  \setlength{\parskip}{0pt plus 0.3pt}%
  \let\item\@idxitem
}{%
  \clearpage
}
\makeatother

\IfFileExists{\jobname-pw.ind}{\input{\jobname-pw.ind}}{}

\end{document}

      