%% latex-leseansicht-vorspann.tex
%% Vorspann für die Leseansicht.
%% Lädt die gemeinsame Datei latex-vorspann.tex mit nicht gesetztem Schalter.

\newif\ifkorrekturansicht
\korrekturansichtfalse

\input{../tex-inputs/latex-vorspann}

\begin{center}
            \textcolor{red}{ENTWURF, NICHT FERTIG KORRIGIERT}
                      \end{center}
            
         
         \renewcommand{\erwaehntePersonen}{Personen: Hugo von Hofmannsthal, Margarethe Kainz, Julius Schnitzler, Olga Schnitzler, Louise Schnitzler, Jakob Wassermann}
         \renewcommand{\erwaehnteOrte}{Orte: Biberstraße, Brünn, Heiligenstadt, I., Innere Stadt, Semmering, Südbahnhotel, Wien}
         \renewcommand{\erwaehnteWerke}{}
               \section[Felix Salten an Arthur Schnitzler, 8. 2. 1908]{ Felix Salten an Arthur Schnitzler, 8. 2. 1908}\nopagebreak\mylabel{v}\rehead{ }\begin{ledgroupsized}[t]{13cm}\normalsize\beginnumbering \toendnotes[C]{\smallbreak\pagebreak[2]} \Standort{CUL, Schnitzler, B 89, B 1.}
\physDesc{Postkarte
\newline{}Handschrift: schwarze Tinte, lateinische Kurrent\newline{}Versand: 1) Stempel: »\nobreak{}\oindex{I., Innere Stadt@\textbf{I., Innere Stadt}|pwk}1/\textsubscript{1} Wien
                                       1, 8. II. 08, 12\nobreak{}«.   2) mit Bleistift von unbekannter Hand der Vorname Schnitzler\pwindex{Schnitzler, Arthur 15.05.1862 – 21.10.1931@\textsc{Schnitzler, Arthur} (15.05.1862 – 21.10.1931), \emph{Schriftsteller, Mediziner}|pw}s durchgestrichen
\newline{}Schnitzler: mit Bleistift datiert:
                                          »8/2 {[}1{]}908« }\toendnotes[C]{\smallbreak}\pstart{}{\pb}Herrn D\textsuperscript{r} Arthur Schnitzler\pend{}\pstart{}Semmering\oindex{Semmering@\textbf{Semmering}|pw}\pend{}\pstart{}Südbahnhotel\oindex{Suedbahnhotel@\textbf{Südbahnhotel}|pw}\pend{}{\bigskip}\pstart
           \noindent{}{\pb}Lieber, wir waren erst gegen 2\textsuperscript{h} in Wien\oindex{Wien@\textbf{Wien}|pw}, \label{K_L03492-1v}\edtext{¾ 3}{\lemma{\textnormal{\emph{¾ 3}}}\Cendnote{\textnormal{14 Uhr 45}}}\label{K_L03492-1h} in Heiligenstadt\oindex{Heiligenstadt@\textbf{Heiligenstadt}|pw}, wo wir
               essen muſsten. Wir haben Ihrem Herrn Bruder\pwindex{Schnitzler, Julius 13.07.1865 – 29.06.1939@\textsc{Schnitzler, Julius} (13.07.1865 – 29.06.1939), \emph{Chirurg}|pwv} gleich telefonirt, fuhren auch ohne Verzögerung in
               die Stadt, aber bei dem heftigen Sturm kamen die Pferde nur schwer vorwärts. Und als
               wir mit einer Verspätung um 10 Minuten in die Biberstraße\oindex{Biberstrasse@\textbf{Biberstraße}|pw} kamen, wurden wir nicht mehr angenommen. Mir that es sehr leid,
               umso mehr, als ich ja eigens wegen dieser Consultation um 10.17 vom Semmering\oindex{Semmering@\textbf{Semmering}|pw} weg bin und nicht mit dem Schnell-Zug. \pend
           \pstart
           Vielleicht komme ich am Montag früh, oder um 2\textsuperscript{h.} von Brünn\oindex{Bruenn@\textbf{Brünn}|pw} aus noch einmal für einen Tag
               hinauf. Grüßen Sie \uline{Alle}, Ihre Frau\pwindex{Schnitzler, Olga 17.01.1882 – 13.01.1970@\textsc{Schnitzler, Olga} (17.01.1882 – 13.01.1970), \emph{Schauspielerin, Sängerin}|pwv}, Ihre Mama\pwindex{Schnitzler, Louise 1840-07-08 – 1911-09-09@\textsc{Schnitzler, Louise} (1840-07-08 – 1911-09-09)|pwv}, Hofmannsthal\pwindex{Hofmannsthal, Hugo von 1874-02-01 – 1929-07-15@\textsc{Hofmannsthal, Hugo von} (1874-02-01 – 1929-07-15), \emph{Schriftsteller}|pw},
                  Wassermann\pwindex{Wassermann, Jakob 10.03.1873 – 01.01.1934@\textsc{Wassermann, Jakob} (10.03.1873 – 01.01.1934), \emph{Schriftsteller}|pw} u. Frau Kainz\pwindex{Kainz, Margarethe 13.12.1858 – 12.02.1950@\textsc{Kainz, Margarethe} (13.12.1858 – 12.02.1950), \emph{Schauspielerin}|pw}. Herzlichst\pend
           \pstart Ihr \spacefill\mbox{Salten}\pend{}
         
         \endnumbering\mylabel{h}\end{ledgroupsized}\begin{anhang}\end{anhang}\newcommand{\dateiname}{L03492}\newcommand{\titel}{Felix Salten an Arthur Schnitzler, 8. 2. 1908}\newcommand{\editorInnen}{Martin Anton Müller und Laura Untner}%% latex-leseansicht-abspann.tex
%% Abspann für die Leseansicht.
%% Der Schalter \ifkorrekturansicht ist bereits durch den Vorspann gesetzt.

%% latex-abspann.tex
%% Gemeinsamer Abspann für Korrekturansicht und Leseansicht.
%% Setzt den Schalter \ifkorrekturansicht voraus (gesetzt in den
%% einbindenden Dateien latex-korrekturansicht-abspann.tex bzw.
%% latex-leseansicht-abspann.tex).
%% ---------------------------------------------------------------

\normalsize

% Das esempio-Environment wird nur in der Leseansicht benötigt
\ifkorrekturansicht\else
\newenvironment{esempio}[3]%
{
    \vspace{1.5ex}
    \rlap{\underline{#1}}
    \par
    \setlength{\parindent}{0cm}
    \nopagebreak
    \leftskip=#2cm
    \rightskip=#3cm
}
{
    \par
}
\fi

\doendnotes{C}
\bigskip
\vfill

\clearpage

\footnotesize

\ifkorrekturansicht
  \lohead{\textsc{register}}
\fi

% theindex-Environment neu definieren ohne reledmac
\makeatletter
\renewenvironment{theindex}{%
  \ifkorrekturansicht
    \section*{\indexname}%
  \else
    \subsubsection*{Index der erwähnten Entitäten}%
  \fi
  \setlength{\parindent}{0pt}%
  \setlength{\parskip}{0pt plus 0.3pt}%
  \let\item\@idxitem
}{%
  \ifkorrekturansicht\clearpage\fi
}
\makeatother

\IfFileExists{\jobname-pw.ind}{\input{\jobname-pw.ind}}{}

% Quellenangabe nur in der Leseansicht
\ifkorrekturansicht\else
% Fallback-Definitionen, falls die .tex-Datei \titel etc. nicht gesetzt hat
\providecommand{\titel}{}
\providecommand{\editorInnen}{}
\providecommand{\dateiname}{\jobname}

\vspace{3cm}

\vfill

\footnotesize
\textsc{Quelle}: \titel. Herausgegeben von {\editorInnen}. In: \emph{Arthur Schnitzler: Briefwechsel mit Autorinnen und Autoren}.
 Digitale Edition, https://schnitzler-briefe.acdh.oeaw.ac.at/{\dateiname}.html (Stand \today)
\fi

\end{document}


      