%% latex-leseansicht-vorspann.tex
%% Vorspann für die Leseansicht.
%% Lädt die gemeinsame Datei latex-vorspann.tex mit nicht gesetztem Schalter.

\newif\ifkorrekturansicht
\korrekturansichtfalse

\input{../tex-inputs/latex-vorspann}


\section[Arthur Schnitzler an Gustav Schwarzkopf, {{[}}23. 7. 1893?{{]}}]{L04208 Arthur Schnitzler an Gustav Schwarzkopf, {[}23. 7. 1893?{]}}
\nopagebreak\mylabel{L04208v}
\rehead{ }\normalsize\beginnumbering\briefempfaengerindex{Schwarzkopf, Gustav@\textsc{Schwarzkopf, Gustav}!zzzSchnitzler, Arthur@\emph{von Arthur Schnitzler}!1893-07-231@{{[}23. 7. 1893?{]}}|(be}
\toendnotes[C]{\smallbreak\pagebreak[2]}
\correspDesc{Versand  durch Arthur Schnitzler am [23. 7. 1893?] in Wien
\newline{}Erhalt  durch Gustav Schwarzkopf im Zeitraum [24. 7. 1893 – 28. 7. 1893?] in Brühl}\toendnotes[C]{\smallbreak}
\Standort{CUL, Schnitzler, B 96.}
\physDesc{Briefkarte, 254 Zeichen (Briefkarte mit Trauerrand)
\newline{}Handschrift: Bleistift, deutsche Kurrent}\toendnotes[C]{\smallbreak}
\pstart{}{\pb}Verehrteſter Herr
                  Schwarzkopf,\pend\vspace{0.5em}
\pstart
           ob Sie dieſe Zeilen antreffen, weiſs ich nicht. Für alle Fälle theile ich Ihnen mit,
               dß ich um 11{ }\textsc{per} Bic. nach der Brühl\oindex{Brühl@\textbf{Brühl}, \emph{Tal}|pw} fahre. Ich freue mich ſchon {\pb}ſehr, \introOben{}mit\introOben{} Ihnen
               wieder \label{K_L04208-1v}\edtext{beiſa{\geminationm}en zu{ }ſein}{\lemma{\textnormal{\emph{beisammen zu sein}}}\Cendnote{\textnormal{Vgl. A. S.: \emph{Tagebuch}, 23. 7. 1893.
               }}}\label{K_L04208-1}\pend
           
\pstart
           Herzlichen Gruſs{\\[\baselineskip]} Ihr ſtets ergebner{\\[\baselineskip]}\spacefill\mbox{Arth Schn}\pend
           \leftskip=0em{}\selectlanguage{ngerman}\endnumbering\briefempfaengerindex{Schwarzkopf, Gustav@\textsc{Schwarzkopf, Gustav}!zzzSchnitzler, Arthur@\emph{von Arthur Schnitzler}!1893-07-231@{{[}23. 7. 1893?{]}}|)be}\mylabel{L04208h}
\begin{anhang}
\end{anhang}\newcommand{\dateiname}{L04208}\newcommand{\titel}{Arthur Schnitzler an Gustav Schwarzkopf, [23. 7. 1893?]}\newcommand{\editorInnen}{Herausgegeben von Jahnke, SelmaMüller, Martin Anton}%% latex-leseansicht-abspann.tex
%% Abspann für die Leseansicht.
%% Der Schalter \ifkorrekturansicht ist bereits durch den Vorspann gesetzt.

%% latex-abspann.tex
%% Gemeinsamer Abspann für Korrekturansicht und Leseansicht.
%% Setzt den Schalter \ifkorrekturansicht voraus (gesetzt in den
%% einbindenden Dateien latex-korrekturansicht-abspann.tex bzw.
%% latex-leseansicht-abspann.tex).
%% ---------------------------------------------------------------

\normalsize

% Das esempio-Environment wird nur in der Leseansicht benötigt
\ifkorrekturansicht\else
\newenvironment{esempio}[3]%
{
    \vspace{1.5ex}
    \rlap{\underline{#1}}
    \par
    \setlength{\parindent}{0cm}
    \nopagebreak
    \leftskip=#2cm
    \rightskip=#3cm
}
{
    \par
}
\fi

\doendnotes{C}
\bigskip
\vfill

\clearpage

\footnotesize

\ifkorrekturansicht
  \lohead{\textsc{register}}
\fi

% theindex-Environment neu definieren ohne reledmac
\makeatletter
\renewenvironment{theindex}{%
  \ifkorrekturansicht
    \section*{\indexname}%
  \else
    \subsubsection*{Index der erwähnten Entitäten}%
  \fi
  \setlength{\parindent}{0pt}%
  \setlength{\parskip}{0pt plus 0.3pt}%
  \let\item\@idxitem
}{%
  \ifkorrekturansicht\clearpage\fi
}
\makeatother

\IfFileExists{\jobname-pw.ind}{\input{\jobname-pw.ind}}{}

% Quellenangabe nur in der Leseansicht
\ifkorrekturansicht\else
% Fallback-Definitionen, falls die .tex-Datei \titel etc. nicht gesetzt hat
\providecommand{\titel}{}
\providecommand{\editorInnen}{}
\providecommand{\dateiname}{\jobname}

\vspace{3cm}

\vfill

\footnotesize
\textsc{Quelle}: \titel. Herausgegeben von {\editorInnen}. In: \emph{Arthur Schnitzler: Briefwechsel mit Autorinnen und Autoren}.
 Digitale Edition, https://schnitzler-briefe.acdh.oeaw.ac.at/{\dateiname}.html (Stand \today)
\fi

\end{document}


