%% latex-leseansicht-vorspann.tex
%% Vorspann für die Leseansicht.
%% Lädt die gemeinsame Datei latex-vorspann.tex mit nicht gesetztem Schalter.

\newif\ifkorrekturansicht
\korrekturansichtfalse

\input{../tex-inputs/latex-vorspann}


         
         \renewcommand{\erwaehntePersonen}{Personen: Alfred Bates, Franz Blei, William Heinemann, Fanny Johnson}
         \renewcommand{\erwaehnteInstitutionen}{Institutionen: William Heinemann Ltd}
         \renewcommand{\erwaehnteOrte}{Orte: Arcisstraße, England, München, Oxford, Wien}
         \renewcommand{\erwaehnteWerke}{Werke: Der grüne Kakadu. Groteske in einem Akt}
               \section[Franz Blei an Arthur Schnitzler, 4. 1. 1904]{ Franz Blei an Arthur Schnitzler, 4. 1. 1904}\nopagebreak\mylabel{v}\rehead{ }\begin{ledgroupsized}[t]{13cm}\normalsize\beginnumbering\briefempfaengerindex{Schnitzler, Arthur@\textsc{Schnitzler, Arthur}!zzzBlei, Franz@\emph{von Franz Blei}!1904-01-041@{4. 1. 1904}|(be} \toendnotes[C]{\smallbreak\pagebreak[2]} \Standort{CUL, Schnitzler, B 14.}
\physDesc{Brief, 1 Blatt, 3 Seiten, 1071 Zeichen
\newline{}Handschrift: schwarze Tinte, lateinische Kurrent
\newline{}Schnitzler: 1) mit Bleistift beschriftet: »\textsc{Blei}« und datiert: »4. 1. 904«  2) mit rotem Buntstift mehrere Unterstreichungen
\newline{}Ordnung: 1) mit Bleistift von unbekannter Hand nummeriert: »\strikeout{2}«  2) mit Bleistift von unbekannter Hand nummeriert:
                                 »3«}\toendnotes[C]{\smallbreak}\pstart
           \raggedleft{}{\pb}München, Arcisstrasse 19\oindex{Arcisstrasse@\textbf{Arcisstraße}|pw}\pend
           \pstart{}Verehrter Herr Doktor,\pend\pstart
           meine unvorhergesehene frühe Abreise von Wien\oindex{Wien@\textbf{Wien}|pw}
               liess es nicht dazu kommen, dass ich Sie, wie ich so gern gethan hätte, besuchte. Was
               mir sehr leid thut.\pend
           \pstart
           Heute schreibt mir Miss Johnson\pwindex{Johnson, Fanny 1862 – 07.02.1943@\textsc{Johnson, Fanny} (1862 – 07.02.1943), \emph{Schriftstellerin}|pw}, Oxford\oindex{Oxford@\textbf{Oxford}|pw}, dass Ihr englischer\oindex{England@\textbf{England}|pw}{ }Verleger\pwindex{Bates, Alfred @\textsc{Bates, Alfred}, \emph{Herausgeber}|pwv} »\begin{otherlanguage}{english}distinctly shady\end{otherlanguage}« sei, was sie in Ihrem wie in ihrem
               Interesse bedauert. Doch lässt sie sich dadurch nicht abhalten, die angefangene
               Übertragung des »Grünen Kakadu\pwindex{Schnitzler, Arthur 15.05.1862 – 21.10.1931@\textsc{Schnitzler, Arthur} (15.05.1862 – 21.10.1931), \emph{Schriftsteller, Mediziner}!gruene Kakadu. Groteske in einem Akt1. 3. 1899@\strich\emph{Der grüne Kakadu. Groteske in einem Akt} {[}1. 3. 1899{]}|pw}« {\pb}zu beenden, aus Freude an der Sache, denn
               ihr Honorar sei eine leere Versprechung. Ich berichte damit nur was die Dame\pwindex{Johnson, Fanny 1862 – 07.02.1943@\textsc{Johnson, Fanny} (1862 – 07.02.1943), \emph{Schriftstellerin}|pwv} schreibt und kann
               meinerseits nur sagen, dass sie soweit ich sie kenne, recht haben wird wenn sie den
                  Verleger\pwindex{Bates, Alfred @\textsc{Bates, Alfred}, \emph{Herausgeber}|pwv} nicht \begin{otherlanguage}{english}reputable\end{otherlanguage} findet. Für die Zukunft möchte ich Ihnen Heinemann\orgindex{William Heinemann Ltd@William Heinemann Ltd|pw}, den ich persönlich und als einen sehr
               noblen Geschäftsmann {\pb}kenne{[},
                  anempfehlen{]}. Wenn Sie Miss Johnson\pwindex{Johnson, Fanny 1862 – 07.02.1943@\textsc{Johnson, Fanny} (1862 – 07.02.1943), \emph{Schriftstellerin}|pw} die Übersetzung Ihrer Novellen anvertrauen, werden Sie dazu in W. Heinemann\pwindex{Heinemann, William 18.05.1863 – 05.10.1920@\textsc{Heinemann, William} (18.05.1863 – 05.10.1920), \emph{Verleger}|pw} einen in jeder Beziehung
               vortrefflichen Verleger haben, sowohl was Reputation als Ausstattung als besonders
               Honorar betrifft.\pend
           \pstart
           Mit bestem Gruss{\\[\baselineskip]}Ihr ganz ergebener{\\[\baselineskip]}\spacefill\mbox{Dr Franz Blei}\pend
           \leftskip=0em{}\pstart
           4. 1. 1904\pend
           
         
         \endnumbering\mylabel{h}\end{ledgroupsized}  \newcommand{\dateiname}{L01356}\newcommand{\titel}{Franz Blei an Arthur Schnitzler, 4. 1. 1904}\newcommand{\editorInnen}{Martin Anton Müller und Gerd-Hermann Susen}%% latex-leseansicht-abspann.tex
%% Abspann für die Leseansicht.
%% Der Schalter \ifkorrekturansicht ist bereits durch den Vorspann gesetzt.

%% latex-abspann.tex
%% Gemeinsamer Abspann für Korrekturansicht und Leseansicht.
%% Setzt den Schalter \ifkorrekturansicht voraus (gesetzt in den
%% einbindenden Dateien latex-korrekturansicht-abspann.tex bzw.
%% latex-leseansicht-abspann.tex).
%% ---------------------------------------------------------------

\normalsize

% Das esempio-Environment wird nur in der Leseansicht benötigt
\ifkorrekturansicht\else
\newenvironment{esempio}[3]%
{
    \vspace{1.5ex}
    \rlap{\underline{#1}}
    \par
    \setlength{\parindent}{0cm}
    \nopagebreak
    \leftskip=#2cm
    \rightskip=#3cm
}
{
    \par
}
\fi

\doendnotes{C}
\bigskip
\vfill

\clearpage

\footnotesize

\ifkorrekturansicht
  \lohead{\textsc{register}}
\fi

% theindex-Environment neu definieren ohne reledmac
\makeatletter
\renewenvironment{theindex}{%
  \ifkorrekturansicht
    \section*{\indexname}%
  \else
    \subsubsection*{Index der erwähnten Entitäten}%
  \fi
  \setlength{\parindent}{0pt}%
  \setlength{\parskip}{0pt plus 0.3pt}%
  \let\item\@idxitem
}{%
  \ifkorrekturansicht\clearpage\fi
}
\makeatother

\IfFileExists{\jobname-pw.ind}{\input{\jobname-pw.ind}}{}

% Quellenangabe nur in der Leseansicht
\ifkorrekturansicht\else
% Fallback-Definitionen, falls die .tex-Datei \titel etc. nicht gesetzt hat
\providecommand{\titel}{}
\providecommand{\editorInnen}{}
\providecommand{\dateiname}{\jobname}

\vspace{3cm}

\vfill

\footnotesize
\textsc{Quelle}: \titel. Herausgegeben von {\editorInnen}. In: \emph{Arthur Schnitzler: Briefwechsel mit Autorinnen und Autoren}.
 Digitale Edition, https://schnitzler-briefe.acdh.oeaw.ac.at/{\dateiname}.html (Stand \today)
\fi

\end{document}


      