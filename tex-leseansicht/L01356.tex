%% latex-korrekturansicht-vorspann.tex
%% Vorspann für die Korrekturansicht.
%% Lädt die gemeinsame Datei latex-vorspann.tex mit gesetztem Schalter.

\newif\ifkorrekturansicht
\korrekturansichttrue

\input{../tex-inputs/latex-vorspann}


\section[Franz Blei an Arthur Schnitzler, 4. 1. 1904]{L01356 Franz Blei an Arthur Schnitzler, 4. 1. 1904}
\nopagebreak\mylabel{L01356v}
\rehead{ }\normalsize\beginnumbering\briefempfaengerindex{Schnitzler, Arthur@\textsc{Schnitzler, Arthur}!zzzBlei, Franz@\emph{von Franz Blei}!1904-01-041@{4. 1. 1904}|(be}
\toendnotes[C]{\smallbreak\pagebreak[2]}\Standort{CUL, Schnitzler, B 14.}
\physDesc{Brief, 1 Blatt, 3 Seiten, 1071 Zeichen
\newline{}Handschrift: schwarze Tinte, lateinische Kurrent
\newline{}Schnitzler: 1) mit Bleistift beschriftet: »\textsc{Blei}« und datiert: »4. 1. 904«  2) mit rotem Buntstift mehrere Unterstreichungen
\newline{}Ordnung: 1) mit Bleistift von unbekannter Hand nummeriert: »\strikeout{2}«  2) mit Bleistift von unbekannter Hand nummeriert:
                                 »3«}\toendnotes[C]{\smallbreak}
\pstart
           \raggedleft{}{\pb}München, Arcisstrasse 19\oindex{Arcisstrasse@\textbf{Arcisstraße}, \emph{Straße (K.STR)}|pw}\pend
           
\pstart{}Verehrter Herr Doktor,\pend\vspace{0.5em}
\pstart
           meine unvorhergesehene frühe Abreise von Wien\oindex{Wien@\textbf{Wien}, \emph{A.ADM2}|pw}
               liess es nicht dazu kommen, dass ich Sie, wie ich so gern gethan hätte, besuchte. Was
               mir sehr leid thut.\pend
           
\pstart
           Heute schreibt mir Miss Johnson\pwindex{Johnson, Fanny 1862 – 07.02.1943@\textsc{Johnson, Fanny} (1862 – 07.02.1943), \emph{Schriftsteller/Schriftstellerin}|pw}, Oxford\oindex{Oxford@\textbf{Oxford}, \emph{P.PPLA2}|pw}, dass Ihr englischer\oindex{England@\textbf{England}, \emph{A.ADM1}|pw}{ }Verleger\pwindex{Bates, Alfred @\textsc{Bates, Alfred}, \emph{Herausgeber/Herausgeberin}|pwv} »\begin{otherlanguage}{english}distinctly shady\end{otherlanguage}« sei, was sie in Ihrem wie in ihrem
               Interesse bedauert. Doch lässt sie sich dadurch nicht abhalten, die angefangene
               Übertragung des »Grünen Kakadu\pwindex{gruene Kakadu. Groteske in einem Akt@\emph{Der grüne Kakadu. Groteske in einem Akt}|pw}« {\pb}zu beenden, aus Freude an der Sache, denn
               ihr Honorar sei eine leere Versprechung. Ich berichte damit nur was die Dame\pwindex{Johnson, Fanny 1862 – 07.02.1943@\textsc{Johnson, Fanny} (1862 – 07.02.1943), \emph{Schriftsteller/Schriftstellerin}|pwv} schreibt und kann
               meinerseits nur sagen, dass sie soweit ich sie kenne, recht haben wird wenn sie den
                  Verleger\pwindex{Bates, Alfred @\textsc{Bates, Alfred}, \emph{Herausgeber/Herausgeberin}|pwv} nicht \begin{otherlanguage}{english}reputable\end{otherlanguage} findet. Für die Zukunft möchte ich Ihnen Heinemann\orgindex{William Heinemann Ltd@William Heinemann Ltd|pw}, den ich persönlich und als einen sehr
               noblen Geschäftsmann {\pb}kenne{[},
                  anempfehlen{]}. Wenn Sie Miss Johnson\pwindex{Johnson, Fanny 1862 – 07.02.1943@\textsc{Johnson, Fanny} (1862 – 07.02.1943), \emph{Schriftsteller/Schriftstellerin}|pw} die Übersetzung Ihrer Novellen anvertrauen, werden Sie dazu in W. Heinemann\pwindex{Heinemann, William 18.05.1863 – 05.10.1920@\textsc{Heinemann, William} (18.05.1863 – 05.10.1920), \emph{Verleger/Verlegerin}|pw} einen in jeder Beziehung
               vortrefflichen Verleger haben, sowohl was Reputation als Ausstattung als besonders
               Honorar betrifft.\pend
           
\pstart
           Mit bestem Gruss{\\[\baselineskip]}Ihr ganz ergebener{\\[\baselineskip]}\spacefill\mbox{Dr Franz Blei}\pend
           \leftskip=0em{}
\pstart
           4. 1. 1904\pend
           \selectlanguage{ngerman}\endnumbering\briefempfaengerindex{Schnitzler, Arthur@\textsc{Schnitzler, Arthur}!zzzBlei, Franz@\emph{von Franz Blei}!1904-01-041@{4. 1. 1904}|)be}\mylabel{L01356h}  \normalsize

\doendnotes{C}
\bigskip
\vfill

\clearpage

\footnotesize

\lohead{\textsc{register}}

% Definiere theindex-Environment komplett neu ohne reledmac
\makeatletter
\renewenvironment{theindex}{%
  \section*{\indexname}%
  \setlength{\parindent}{0pt}%
  \setlength{\parskip}{0pt plus 0.3pt}%
  \let\item\@idxitem
}{%
  \clearpage
}
\makeatother

\IfFileExists{\jobname-pw.ind}{\input{\jobname-pw.ind}}{}

\end{document}

      