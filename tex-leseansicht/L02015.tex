%% latex-korrekturansicht-vorspann.tex
%% Vorspann für die Korrekturansicht.
%% Lädt die gemeinsame Datei latex-vorspann.tex mit gesetztem Schalter.

\newif\ifkorrekturansicht
\korrekturansichttrue

\input{../tex-inputs/latex-vorspann}


\section[Arthur Schnitzler an Hugo von Hofmannsthal, 27. 3. 1911]{L02015 Arthur Schnitzler an Hugo von Hofmannsthal, 27. 3. 1911}
\nopagebreak\mylabel{L02015v}
\rehead{ }\normalsize\beginnumbering\briefempfaengerindex{Hofmannsthal, Hugo von@\textsc{Hofmannsthal, Hugo von}!zzzSchnitzler, Arthur@\emph{von Arthur Schnitzler}!1911-03-271@{27. 3. 1911}|(be}
\toendnotes[C]{\smallbreak\pagebreak[2]}\Standort{FDH, Hs-30885,3.}
\physDesc{Briefkarte, 304 Zeichen
\newline{}Handschrift: schwarze Tinte, deutsche Kurrent}\Standort{FDH, Hs-30885,143.}
\physDesc{Briefkarte, Fotokopie304 Zeichen
\newline{}Handschrift: schwarze Tinte, deutsche Kurrent}
\buchAbdrucke{\weitereDrucke{Hugo von Hofmannsthal, Arthur Schnitzler: \emph{Briefwechsel}. Frankfurt am Main: \emph{S. Fischer} 1964, S. 261.} }\toendnotes[C]{\smallbreak}
\pstart
           {\pb}\textcolor{gray}{\textbf{Dr. Arthur Schnitzler}}\hfill 27. 3. 911\pend
           
\pstart
           \textcolor{gray}{\textbf{Wien XVIII. Sternwartestrasse 71\oindex{Sternwartestrasse 71@\textbf{Sternwartestraße 71}, \emph{Wohngebäude (K.WHS)}|pw}}}\pend
           \vspace{0.5em}
\pstart
           lieber Hugo, auch wir ſind am 8. April nicht mehr in
                  Wien\oindex{Wien@\textbf{Wien}, \emph{A.ADM2}|pw}; fahren am 5. od.
                  6. zuerſt nach München\oindex{Muenchen@\textbf{München}, \emph{P.PPLA}|pw} (\textsc{Partenkirchen}\oindex{Partenkirchen@\textbf{Partenkirchen}, \emph{Teil eines besiedelten Ortes (A.BSOX)}|pw}, zu \textsc{Lisl}\pwindex{Steinrueck, Elisabeth 19.11.1885 – 07.04.1920@\textsc{Steinrück, Elisabeth} (19.11.1885 – 07.04.1920)|pw},) dann vorausſichtlich weiter nach \textsc{Genua}\oindex{Genua@\textbf{Genua}, \emph{P.PPLA}|pw}, \textsc{Mentone}\oindex{Menton@\textbf{Menton}, \emph{P.PPL}|pw}, auf 2–3 Wochen.\pend
           
\pstart
           Ich hatte die Première\pwindex{Rosenkavalier@\emph{Der Rosenkavalier}|pwv} für
               früher angeno{\geminationm}en. {\pb}Es thut uns
               leid, ſie nicht \label{K_L02015-1v}\edtext{mitmachen}{\lemma{\textnormal{\emph{mitmachen}}}\Cendnote{\textnormal{Die Abreise verzögerte sich bis zum
                     10. 4. 1911, er nahm aber trotzdem nicht teil.}}}\label{K_L02015-1} zu kö{\geminationn}en.\pend
           
\pstart
           Schönen Dank und Gruß.{\\[\baselineskip]}Ihr{\\[\baselineskip]}\spacefill\mbox{Arthur}\pend
           \leftskip=0em{}\selectlanguage{ngerman}\endnumbering\briefempfaengerindex{Hofmannsthal, Hugo von@\textsc{Hofmannsthal, Hugo von}!zzzSchnitzler, Arthur@\emph{von Arthur Schnitzler}!1911-03-271@{27. 3. 1911}|)be}\mylabel{L02015h}  \normalsize

\doendnotes{C}
\bigskip
\vfill

\clearpage

\footnotesize

\lohead{\textsc{register}}

% Definiere theindex-Environment komplett neu ohne reledmac
\makeatletter
\renewenvironment{theindex}{%
  \section*{\indexname}%
  \setlength{\parindent}{0pt}%
  \setlength{\parskip}{0pt plus 0.3pt}%
  \let\item\@idxitem
}{%
  \clearpage
}
\makeatother

\IfFileExists{\jobname-pw.ind}{\input{\jobname-pw.ind}}{}

\end{document}

      