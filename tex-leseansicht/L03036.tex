%% latex-leseansicht-vorspann.tex
%% Vorspann für die Leseansicht.
%% Lädt die gemeinsame Datei latex-vorspann.tex mit nicht gesetztem Schalter.

\newif\ifkorrekturansicht
\korrekturansichtfalse

\input{../tex-inputs/latex-vorspann}

\begin{center}
            \textcolor{red}{ENTWURF, NICHT FERTIG KORRIGIERT}
                      \end{center}
            
         
         \renewcommand{\erwaehntePersonen}{Personen:  ?? [Anstandsdame von Anna und Clara Loeb], Marianne Benedict, Anna Epstein, Clara Katharina Pollaczek, Felix Salten}
         \renewcommand{\erwaehnteOrte}{Orte: Café Arkaden, Wien}
         \renewcommand{\erwaehnteWerke}{Werke: Tagebuch}
               \section[Arthur Schnitzler an Felix Salten, {[}17. 12. 1896?{]}]{ Arthur Schnitzler an Felix Salten, {[}17. 12. 1896?{]}}\nopagebreak\mylabel{v}\rehead{ }\begin{ledgroupsized}[t]{13cm}\normalsize\beginnumbering \toendnotes[C]{\smallbreak\pagebreak[2]} \Standort{Wienbibliothek im Rathaus, ZPH 1681, 2.1.516.}
\physDesc{Brief, 1 Blatt, 2 Seiten
\newline{}Handschrift: Bleistift, deutsche Kurrent\newline{}Ordnung: mit Bleistift von unbekannter Hand Nummerierung der ungeraden Seiten:
                                 »15«–»16« }\toendnotes[C]{\smallbreak}\pstart
           \noindent{}{\pb}Lieber, ich habe \label{K_L03036-1v}\edtext{\textsc{Mademoiselle\pwindex{?? [Anstandsdame von Anna und Clara Loeb] @\textsc{?? [Anstandsdame von Anna und Clara Loeb]}|pwv}} und die 2
                  Mädel\pwindex{Pollaczek, Clara Katharina 15.01.1875 – 22.07.1951@\textsc{Pollaczek, Clara Katharina} (15.01.1875 – 22.07.1951), \emph{Schriftstellerin}|pwv}\pwindex{Epstein, Anna 6.3.1877 – 16.3.1943@\textsc{Epstein, Anna} (6.3.1877 – 16.3.1943)|pwv}}{\lemma{\textnormal{\emph{Mademoiselle … Mädel}}}\Cendnote{\textnormal{Die Datierung dieses Korrespondenzstücks
                  gelingt möglicherweise, wenn die beiden jungen Frauen als die Schwestern Clara\pwindex{Pollaczek, Clara Katharina 15.01.1875 – 22.07.1951@\textsc{Pollaczek, Clara Katharina} (15.01.1875 – 22.07.1951), \emph{Schriftstellerin}|pwk} und Anna Loeb\pwindex{Epstein, Anna 6.3.1877 – 16.3.1943@\textsc{Epstein, Anna} (6.3.1877 – 16.3.1943)|pwk} identifiziert werden. Am 17. 12. 1896 plauderten sie auf einer Soirée bei
                     Marianne Benedict\pwindex{Benedict, Marianne 01.01.1848 – 12.05.1930@\textsc{Benedict, Marianne} (01.01.1848 – 12.05.1930)|pwk}, am Folgetag wird am
                  Nachmittag im \emph{Tagebuch}\pwindex{Schnitzler, Arthur 15.05.1862 – 21.10.1931@\textsc{Schnitzler, Arthur} (15.05.1862 – 21.10.1931), \emph{Schriftsteller, Mediziner}!Tagebuch1981 – 2000@\strich\emph{Tagebuch} {[}1981 – 2000{]}|pwk} die »Anstandsdame\pwindex{?? [Anstandsdame von Anna und Clara Loeb] @\textsc{?? [Anstandsdame von Anna und Clara Loeb]}|pwkv}« erwähnt. Da dies wiederum
                  keine Erwähnung findet, dürfte das Schriftstück am Vormittag des 18. 12. 1896 verfasst
                  sein.}}}\label{K_L03036-1h} eine viertel Minute vor Ihnen getroffen – \textsc{Cl.\pwindex{Pollaczek, Clara Katharina 15.01.1875 – 22.07.1951@\textsc{Pollaczek, Clara Katharina} (15.01.1875 – 22.07.1951), \emph{Schriftstellerin}|pw}} fragt mich, warum ich \uline{nicht} telephonirt habe?
               ich: ich ka{\geminationn} heut nicht ko{\geminationm}en :\textsc{Cl.\pwindex{Pollaczek, Clara Katharina 15.01.1875 – 22.07.1951@\textsc{Pollaczek, Clara Katharina} (15.01.1875 – 22.07.1951), \emph{Schriftstellerin}|pw}}: Schade, {\pb}zu ſprechen, wir ſind allein.
                  Anna\pwindex{Epstein, Anna 6.3.1877 – 16.3.1943@\textsc{Epstein, Anna} (6.3.1877 – 16.3.1943)|pw}: Sehn Sie S.? Ich: Ich ka{\geminationn} ihm ſchreiben. \uline{Anna\pwindex{Epstein, Anna 6.3.1877 – 16.3.1943@\textsc{Epstein, Anna} (6.3.1877 – 16.3.1943)|pw}}: Er ſoll beſti{\geminationm}t um ½ 5 zu uns ko{\geminationm}en.\pend
           \pstart
           – Gehn Sie vielleicht {\pb}auf eine halbe Stunde hinauf?– \pend
           \pstart
           Ja, »angfangt iſt’ leicht!« Ich hoff Sie Abends im Arkaden\oindex{Cafe Arkaden@\textbf{Café Arkaden}|pw}, nicht zu ſpät, zu ſehen. Herzlichſt \pend
           \pstart Ihr \spacefill\mbox{Arth}\pend{}
         
         \endnumbering\mylabel{h}\end{ledgroupsized}\begin{anhang}\end{anhang}\newcommand{\dateiname}{L03036}\newcommand{\titel}{Arthur Schnitzler an Felix Salten, [17. 12. 1896?]}\newcommand{\editorInnen}{Martin Anton Müller und Laura Untner}%% latex-leseansicht-abspann.tex
%% Abspann für die Leseansicht.
%% Der Schalter \ifkorrekturansicht ist bereits durch den Vorspann gesetzt.

%% latex-abspann.tex
%% Gemeinsamer Abspann für Korrekturansicht und Leseansicht.
%% Setzt den Schalter \ifkorrekturansicht voraus (gesetzt in den
%% einbindenden Dateien latex-korrekturansicht-abspann.tex bzw.
%% latex-leseansicht-abspann.tex).
%% ---------------------------------------------------------------

\normalsize

% Das esempio-Environment wird nur in der Leseansicht benötigt
\ifkorrekturansicht\else
\newenvironment{esempio}[3]%
{
    \vspace{1.5ex}
    \rlap{\underline{#1}}
    \par
    \setlength{\parindent}{0cm}
    \nopagebreak
    \leftskip=#2cm
    \rightskip=#3cm
}
{
    \par
}
\fi

\doendnotes{C}
\bigskip
\vfill

\clearpage

\footnotesize

\ifkorrekturansicht
  \lohead{\textsc{register}}
\fi

% theindex-Environment neu definieren ohne reledmac
\makeatletter
\renewenvironment{theindex}{%
  \ifkorrekturansicht
    \section*{\indexname}%
  \else
    \subsubsection*{Index der erwähnten Entitäten}%
  \fi
  \setlength{\parindent}{0pt}%
  \setlength{\parskip}{0pt plus 0.3pt}%
  \let\item\@idxitem
}{%
  \ifkorrekturansicht\clearpage\fi
}
\makeatother

\IfFileExists{\jobname-pw.ind}{\input{\jobname-pw.ind}}{}

% Quellenangabe nur in der Leseansicht
\ifkorrekturansicht\else
% Fallback-Definitionen, falls die .tex-Datei \titel etc. nicht gesetzt hat
\providecommand{\titel}{}
\providecommand{\editorInnen}{}
\providecommand{\dateiname}{\jobname}

\vspace{3cm}

\vfill

\footnotesize
\textsc{Quelle}: \titel. Herausgegeben von {\editorInnen}. In: \emph{Arthur Schnitzler: Briefwechsel mit Autorinnen und Autoren}.
 Digitale Edition, https://schnitzler-briefe.acdh.oeaw.ac.at/{\dateiname}.html (Stand \today)
\fi

\end{document}


      