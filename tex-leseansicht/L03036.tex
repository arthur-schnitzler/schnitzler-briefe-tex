%% latex-korrekturansicht-vorspann.tex
%% Vorspann für die Korrekturansicht.
%% Lädt die gemeinsame Datei latex-vorspann.tex mit gesetztem Schalter.

\newif\ifkorrekturansicht
\korrekturansichttrue

\input{../tex-inputs/latex-vorspann}


\section[Arthur Schnitzler an Felix Salten, {[}21. 11. 1897?{]}]{L03036 Arthur Schnitzler an Felix Salten, {[}21. 11. 1897?{]}}
\nopagebreak\mylabel{L03036v}
\rehead{ }\normalsize\beginnumbering\briefempfaengerindex{Salten, Felix@\textsc{Salten, Felix}!zzzSchnitzler, Arthur@\emph{von Arthur Schnitzler}!1897-11-212@{{[}21. 11. 1897?{]}}|(be}
\toendnotes[C]{\smallbreak\pagebreak[2]}\Standort{Wienbibliothek im Rathaus, ZPH 1681, 2.1.516.}
\physDesc{Brief, 1 Blatt, 3 Seiten, 429 Zeichen
\newline{}Handschrift: Bleistift, deutsche Kurrent
\newline{}Ordnung: mit Bleistift von unbekannter Hand Nummerierung der Doppelseiten des
                                 Konvoluts: »15«–»16« }\toendnotes[C]{\smallbreak}
\pstart
           \noindent{}{\pb}Lieber, ich habe \label{K_L03036-1v}\edtext{\textsc{Mademoiselle\pwindex{?? [Anstandsdame von Anna und Clara Loeb] @\textsc{?? [Anstandsdame von Anna und Clara Loeb]}|pwv}} und die 2
                  Mädel\pwindex{Pollaczek, Clara Katharina 15.01.1875 – 22.07.1951@\textsc{Pollaczek, Clara Katharina} (15.01.1875 – 22.07.1951), \emph{Schriftsteller/Schriftstellerin}|pwv}\pwindex{Epstein, Anna 6.3.1877 – 16.3.1943@\textsc{Epstein, Anna} (6.3.1877 – 16.3.1943)|pwv}}{\lemma{\textnormal{\emph{Mademoiselle … Mädel}}}\Cendnote{\textnormal{Bei diesem Korrespondenzstück
                     dürfte es sich um die Antwort auf Felix Salten an Arthur Schnitzler, [21. 11. 1897] handeln.
                     Die zeitliche Einordnung wird zusätzlich gestützt durch die gemeinsamen Ausflüge der
                     Schwestern Clara\pwindex{Pollaczek, Clara Katharina 15.01.1875 – 22.07.1951@\textsc{Pollaczek, Clara Katharina} (15.01.1875 – 22.07.1951), \emph{Schriftsteller/Schriftstellerin}|pwk} und Anna Loeb\pwindex{Epstein, Anna 6.3.1877 – 16.3.1943@\textsc{Epstein, Anna} (6.3.1877 – 16.3.1943)|pwk}, die sich zu diesem Zeitpunkt in Schnitzlers{ }\emph{Tagebuch}\pwindex{Tagebuch@\emph{Tagebuch}|pwk} belegen lassen, vor allem aber durch das für den
                     12. 11. 1897 dokumentierte Interesse
                     von Anna Loeb\pwindex{Epstein, Anna 6.3.1877 – 16.3.1943@\textsc{Epstein, Anna} (6.3.1877 – 16.3.1943)|pwk} an Salten\pwindex{Salten, Felix 06.09.1869 – 08.10.1945@\textsc{Salten, Felix} (06.09.1869 – 08.10.1945), \emph{Schriftsteller/Schriftstellerin, Journalist/Journalistin, Chefredakteur/Chefredakteurin}|pwk}.}}}\label{K_L03036-1} eine
                  viertel Minute vor Ihnen
                  getroffen –\pend
           
\pstart
           \textsc{Cl.\pwindex{Pollaczek, Clara Katharina 15.01.1875 – 22.07.1951@\textsc{Pollaczek, Clara Katharina} (15.01.1875 – 22.07.1951), \emph{Schriftsteller/Schriftstellerin}|pw}} fragt mich, warum ich \uline{nicht} telephonirt habe?
               ich: ich ka{\geminationn}{ }heut nicht ko{\geminationm}en\textcolor{gray}{!}{ }\textsc{Cl\pwindex{Pollaczek, Clara Katharina 15.01.1875 – 22.07.1951@\textsc{Pollaczek, Clara Katharina} (15.01.1875 – 22.07.1951), \emph{Schriftsteller/Schriftstellerin}|pw}}: \label{K_L03036-2v}\edtext{Schade, {\pb}zu ſprechen}{\lemma{\textnormal{\emph{Schade, zu ſprechen}}}\Cendnote{\textnormal{Hier
               dürfte Schnitzler beim Wechsel der Seiten ein Versehen passiert sein
                  und er überging einen Halbsatz wie ›ich hatte gehofft, Sie zu sprechen‹.}}}\label{K_L03036-2}, wir ſind allein.
                  Anna\pwindex{Epstein, Anna 6.3.1877 – 16.3.1943@\textsc{Epstein, Anna} (6.3.1877 – 16.3.1943)|pw}: Sehn Sie S.? Ich: Ich ka{\geminationn} ihm ſchreiben. \uline{Anna\pwindex{Epstein, Anna 6.3.1877 – 16.3.1943@\textsc{Epstein, Anna} (6.3.1877 – 16.3.1943)|pw}}: Er ſoll beſti{\geminationm}t um ½ 5 zu uns ko{\geminationm}en.\pend
           
\pstart
           – Gehn Sie vielleicht {\pb}auf eine halbe Stunde
               hinauf? –\pend
           
\pstart
           Ja, »\label{K_L03036-3v}\edtext{angfangt iſt leicht}{\lemma{\textnormal{\emph{angfangt iſt leicht}}}\Cendnote{\textnormal{Redewendung: anfangen ist leicht, beharren
                  eine Kunst}}}\label{K_L03036-3}«!\pend
           
\pstart
           Ich hoff Sie Abends im Arkaden\oindex{Cafe Arkaden@\textbf{Café Arkaden}, \emph{Kaffeehaus (K.KAF)}|pw},
               nicht ſpät, zu ſehen. Herzlichſt\pend
           \pstart Ihr \spacefill\mbox{Arth}\pend{}\selectlanguage{ngerman}\endnumbering\briefempfaengerindex{Salten, Felix@\textsc{Salten, Felix}!zzzSchnitzler, Arthur@\emph{von Arthur Schnitzler}!1897-11-212@{{[}21. 11. 1897?{]}}|)be}\mylabel{L03036h}  \normalsize

\doendnotes{C}
\bigskip
\vfill

\clearpage

\footnotesize

\lohead{\textsc{register}}

% Definiere theindex-Environment komplett neu ohne reledmac
\makeatletter
\renewenvironment{theindex}{%
  \section*{\indexname}%
  \setlength{\parindent}{0pt}%
  \setlength{\parskip}{0pt plus 0.3pt}%
  \let\item\@idxitem
}{%
  \clearpage
}
\makeatother

\IfFileExists{\jobname-pw.ind}{\input{\jobname-pw.ind}}{}

\end{document}

      