%% latex-leseansicht-vorspann.tex
%% Vorspann für die Leseansicht.
%% Lädt die gemeinsame Datei latex-vorspann.tex mit nicht gesetztem Schalter.

\newif\ifkorrekturansicht
\korrekturansichtfalse

\input{../tex-inputs/latex-vorspann}


\section[Arthur Schnitzler an Felix Salten, {[}21. 11. 1897?{]}]{L03036 Arthur Schnitzler an Felix Salten, {[}21. 11. 1897?{]}}
\nopagebreak\mylabel{L03036v}
\rehead{ }\normalsize\beginnumbering\briefempfaengerindex{Salten, Felix@\textsc{Salten, Felix}!zzzSchnitzler, Arthur@\emph{von Arthur Schnitzler}!1897-11-212@{{[}21. 11. 1897?{]}}|(be}
\toendnotes[C]{\smallbreak\pagebreak[2]}
\correspDesc{Versand  durch Arthur Schnitzler am [21. 11. 1897?] in Wien
\newline{}Erhalt  durch Felix Salten am [21. 11. 1897?] in Wien}\toendnotes[C]{\smallbreak}
\Standort{Wienbibliothek im Rathaus, ZPH 1681, 2.1.516.}
\physDesc{Brief, 1 Blatt, 3 Seiten, 429 Zeichen
\newline{}Handschrift: Bleistift, deutsche Kurrent
\newline{}Ordnung: mit Bleistift von unbekannter Hand Nummerierung der Doppelseiten des
                                 Konvoluts: »15«–»16« }\toendnotes[C]{\smallbreak}
\pstart
           \noindent{}{\pb}Lieber, ich habe \label{K_L03036-1v}\edtext{\textsc{Mademoiselle\pwindex{?? [Anstandsdame von Anna und Clara Loeb] @\textsc{?? [Anstandsdame von Anna und Clara Loeb]}|pwv}} und die 2
                  Mädel\pwindex{Pollaczek, Clara Katharina 15.\,1.\,1875 Wien – 22.\,7.\,1951 ebd.@\textsc{Pollaczek, Clara Katharina} (15.\,1.\,1875 Wien – 22.\,7.\,1951 ebd.), \emph{Schriftstellerin}|pwv}\pwindex{Epstein, Anna 6.\,3.\,1877 Wien – 16.\,3.\,1943 Konzentrationslager Theresienstadt@\textsc{Epstein, Anna} (6.\,3.\,1877 Wien – 16.\,3.\,1943 Konzentrationslager Theresienstadt)|pwv}}{\lemma{\textnormal{\emph{Mademoiselle … Mädel}}}\Cendnote{\textnormal{Bei diesem Korrespondenzstück
                     dürfte es sich um die Antwort auf XXXX Auszeichnungsfehler: Dokument L03276 nicht gefunden handeln.
                     Die zeitliche Einordnung wird zusätzlich gestützt durch die gemeinsamen Ausflüge der
                     Schwestern Clara\pwindex{Pollaczek, Clara Katharina 15.\,1.\,1875 Wien – 22.\,7.\,1951 ebd.@\textsc{Pollaczek, Clara Katharina} (15.\,1.\,1875 Wien – 22.\,7.\,1951 ebd.), \emph{Schriftstellerin}|pwk} und Anna Loeb\pwindex{Epstein, Anna 6.\,3.\,1877 Wien – 16.\,3.\,1943 Konzentrationslager Theresienstadt@\textsc{Epstein, Anna} (6.\,3.\,1877 Wien – 16.\,3.\,1943 Konzentrationslager Theresienstadt)|pwk}, die sich zu diesem Zeitpunkt in Schnitzlers{ }\emph{Tagebuch}\pwindex{Schnitzler, Arthur 15.\,5.\,1862 Wien – 21.\,10.\,1931 ebd.@\textsc{Schnitzler, Arthur} (15.\,5.\,1862 Wien – 21.\,10.\,1931 ebd.), \emph{Schriftsteller, Mediziner}!Tagebuch@\strich\emph{Tagebuch}|pwk} belegen lassen, vor allem aber durch das für den
                     12. 11. 1897 dokumentierte Interesse
                     von Anna Loeb\pwindex{Epstein, Anna 6.\,3.\,1877 Wien – 16.\,3.\,1943 Konzentrationslager Theresienstadt@\textsc{Epstein, Anna} (6.\,3.\,1877 Wien – 16.\,3.\,1943 Konzentrationslager Theresienstadt)|pwk} an Salten\pwindex{Salten, Felix 6.\,9.\,1869 Budapest – 8.\,10.\,1945 Zürich@\textsc{Salten, Felix} (6.\,9.\,1869 Budapest – 8.\,10.\,1945 Zürich), \emph{Schriftsteller, Journalist, Chefredakteur}|pwk}.}}}\label{K_L03036-1} eine
                  viertel Minute vor Ihnen
                  getroffen –\pend
           
\pstart
           \textsc{Cl.\pwindex{Pollaczek, Clara Katharina 15.\,1.\,1875 Wien – 22.\,7.\,1951 ebd.@\textsc{Pollaczek, Clara Katharina} (15.\,1.\,1875 Wien – 22.\,7.\,1951 ebd.), \emph{Schriftstellerin}|pw}} fragt mich, warum ich \uline{nicht} telephonirt habe?
               ich: ich ka{\geminationn}{ }heut nicht ko{\geminationm}en\textcolor{gray}{!}{ }\textsc{Cl\pwindex{Pollaczek, Clara Katharina 15.\,1.\,1875 Wien – 22.\,7.\,1951 ebd.@\textsc{Pollaczek, Clara Katharina} (15.\,1.\,1875 Wien – 22.\,7.\,1951 ebd.), \emph{Schriftstellerin}|pw}}: \label{K_L03036-2v}\edtext{Schade, {\pb}zu{ }ſprechen}{\lemma{\textnormal{\emph{Schade, zu sprechen}}}\Cendnote{\textnormal{Hier
               dürfte Schnitzler beim Wechsel der Seiten ein Versehen passiert sein
                  und er überging einen Halbsatz wie ›ich hatte gehofft, Sie zu sprechen‹.}}}\label{K_L03036-2}, wir{ }ſind allein.
                  Anna\pwindex{Epstein, Anna 6.\,3.\,1877 Wien – 16.\,3.\,1943 Konzentrationslager Theresienstadt@\textsc{Epstein, Anna} (6.\,3.\,1877 Wien – 16.\,3.\,1943 Konzentrationslager Theresienstadt)|pw}: Sehn Sie S.? Ich: Ich ka{\geminationn} ihm{ }ſchreiben. \uline{Anna\pwindex{Epstein, Anna 6.\,3.\,1877 Wien – 16.\,3.\,1943 Konzentrationslager Theresienstadt@\textsc{Epstein, Anna} (6.\,3.\,1877 Wien – 16.\,3.\,1943 Konzentrationslager Theresienstadt)|pw}}: Er{ }ſoll beſti{\geminationm}t um ½ 5 zu uns ko{\geminationm}en.\pend
           
\pstart
           – Gehn Sie vielleicht {\pb}auf eine halbe Stunde
               hinauf? –\pend
           
\pstart
           Ja, »\label{K_L03036-3v}\edtext{angfangt iſt leicht}{\lemma{\textnormal{\emph{angfangt ist leicht}}}\Cendnote{\textnormal{Redewendung: anfangen ist leicht, beharren
                  eine Kunst}}}\label{K_L03036-3}«!\pend
           
\pstart
           Ich hoff Sie Abends im Arkaden\oindex{Wien@\textbf{Wien}!I., Innere Stadt@\textbf{I., Innere Stadt}!Café Arkaden@\textbf{Café Arkaden}, \emph{Kaffeehaus}|pw},
               nicht{ }ſpät, zu{ }ſehen. Herzlichſt\pend
           \pstart Ihr \spacefill\mbox{Arth}\pend{}\selectlanguage{ngerman}\endnumbering\briefempfaengerindex{Salten, Felix@\textsc{Salten, Felix}!zzzSchnitzler, Arthur@\emph{von Arthur Schnitzler}!1897-11-212@{{[}21. 11. 1897?{]}}|)be}\mylabel{L03036h}  \newcommand{\dateiname}{L03036}\newcommand{\titel}{Arthur Schnitzler an Felix Salten, [21. 11. 1897?]}\newcommand{\editorInnen}{Martin Anton Müller und Laura Untner}%% latex-leseansicht-abspann.tex
%% Abspann für die Leseansicht.
%% Der Schalter \ifkorrekturansicht ist bereits durch den Vorspann gesetzt.

%% latex-abspann.tex
%% Gemeinsamer Abspann für Korrekturansicht und Leseansicht.
%% Setzt den Schalter \ifkorrekturansicht voraus (gesetzt in den
%% einbindenden Dateien latex-korrekturansicht-abspann.tex bzw.
%% latex-leseansicht-abspann.tex).
%% ---------------------------------------------------------------

\normalsize

% Das esempio-Environment wird nur in der Leseansicht benötigt
\ifkorrekturansicht\else
\newenvironment{esempio}[3]%
{
    \vspace{1.5ex}
    \rlap{\underline{#1}}
    \par
    \setlength{\parindent}{0cm}
    \nopagebreak
    \leftskip=#2cm
    \rightskip=#3cm
}
{
    \par
}
\fi

\doendnotes{C}
\bigskip
\vfill

\clearpage

\footnotesize

\ifkorrekturansicht
  \lohead{\textsc{register}}
\fi

% theindex-Environment neu definieren ohne reledmac
\makeatletter
\renewenvironment{theindex}{%
  \ifkorrekturansicht
    \section*{\indexname}%
  \else
    \subsubsection*{Index der erwähnten Entitäten}%
  \fi
  \setlength{\parindent}{0pt}%
  \setlength{\parskip}{0pt plus 0.3pt}%
  \let\item\@idxitem
}{%
  \ifkorrekturansicht\clearpage\fi
}
\makeatother

\IfFileExists{\jobname-pw.ind}{\input{\jobname-pw.ind}}{}

% Quellenangabe nur in der Leseansicht
\ifkorrekturansicht\else
% Fallback-Definitionen, falls die .tex-Datei \titel etc. nicht gesetzt hat
\providecommand{\titel}{}
\providecommand{\editorInnen}{}
\providecommand{\dateiname}{\jobname}

\vspace{3cm}

\vfill

\footnotesize
\textsc{Quelle}: \titel. Herausgegeben von {\editorInnen}. In: \emph{Arthur Schnitzler: Briefwechsel mit Autorinnen und Autoren}.
 Digitale Edition, https://schnitzler-briefe.acdh.oeaw.ac.at/{\dateiname}.html (Stand \today)
\fi

\end{document}


