%% latex-korrekturansicht-vorspann.tex
%% Vorspann für die Korrekturansicht.
%% Lädt die gemeinsame Datei latex-vorspann.tex mit gesetztem Schalter.

\newif\ifkorrekturansicht
\korrekturansichttrue

\input{../tex-inputs/latex-vorspann}


\section[Arthur Schnitzler an Richard Beer-Hofmann, 28. 11. 1908]{L01813 Arthur Schnitzler an Richard Beer-Hofmann, 28. 11. 1908}
\nopagebreak\mylabel{L01813v}
\rehead{ }\normalsize\beginnumbering\briefempfaengerindex{Beer-Hofmann, Richard@\textsc{Beer-Hofmann, Richard}!zzzSchnitzler, Arthur@\emph{von Arthur Schnitzler}!1908-11-283@{28. 11. 1908}|(be}
\toendnotes[C]{\smallbreak\pagebreak[2]}\Standort{YCGL, MSS 31.}
\physDesc{Brief, 1 Blatt, 3 Seiten, Umschlag, 413 Zeichen
\newline{}Handschrift: 1) Bleistift, deutsche Kurrent\hspace{1em}2) schwarze Tinte, deutsche Kurrent (\noindent{}Umschlag)\hspace{1em}
\newline{}Versand: Stempel: »\nobreak{}Wien 3, 2\textcolor{gray}{8}. XI. 08, 4\nobreak{}«.  }
\buchAbdrucke{\weitereDrucke{Arthur Schnitzler, Richard Beer-Hofmann: \emph{Briefwechsel 1891–1931}. Wien, Zürich: \emph{Europaverlag} 1992, S. 192.} }\toendnotes[C]{\smallbreak}\pstart{}{\pb}\textcolor{gray}{\textbf{Dr. Arthur Schnitzler}}\pend{}\pstart{}\textcolor{gray}{\textbf{Wien XVIII. Spoettelgasse 7\oindex{Edmund-Weiss-Gasse 7@\textbf{Edmund-Weiß-Gasse 7}, \emph{Wohngebäude (K.WHS)}|pw}.}}\pend{}{\bigskip}\pstart{}{\pb}\textsc{Dr. Richard Beer Hofmann}\pend{}\pstart{}Wien XVIII\oindex{XVIII., Waehring@\textbf{XVIII., Währing}, \emph{A.ADM3}|pw}\pend{}\pstart{}\textsc{Hasenauerstr. 59}\oindex{Hasenauerstrasse 59@\textbf{Hasenauerstraße 59}, \emph{Wohngebäude (K.WHS)}|pw}.\pend{}{\bigskip}\vspace{1em}
\pstart
           {\pb}\textcolor{gray}{\textbf{Dr. Arthur Schnitzler}}\hfill \label{K_L01813-1v}\edtext{So{\geminationn}tag 28. 11 08}{\lemma{\textnormal{\emph{Sonntag 28. 11 08}}}\Cendnote{\textnormal{Der 28. 11. 1908 war ein Samstag. An
                        diesem Tag besuchte Schnitzler ein Konzert\pwindex{Dohnányi, Ernst von 27.07.1877 – 09.02.1960@\textsc{Dohnányi, Ernst von} (27.07.1877 – 09.02.1960), \emph{Komponist/Komponistin, Pianist/Pianistin}|pwkv}, was die
                        Datierung erlaubt. Trotzdem fordert das Fehlen eines Gegenbriefes (oder
                        handelte es sich um eine mündlich übermittelte Nachricht?) einige
                        Spekulation, wie die Reihenfolge des zweiten und dritten Briefes vom
                           28. 11. 1908 anzusetzen ist. Im ersten Brief erfährt Beer-Hofmann\pwindex{Beer-Hofmann, Richard 1866-07-11 – 1945-09-26@\textsc{Beer-Hofmann, Richard} (1866-07-11 – 1945-09-26), \emph{Schriftsteller/Schriftstellerin}|pwk} vom Konzert\pwindex{Dohnányi, Ernst von 27.07.1877 – 09.02.1960@\textsc{Dohnányi, Ernst von} (27.07.1877 – 09.02.1960), \emph{Komponist/Komponistin, Pianist/Pianistin}|pwkv} und von der geplanten Reise
                        auf den Semmering\oindex{Semmering@\textbf{Semmering}, \emph{A.ADM3}|pwk}. In der nicht
                        erhaltenen Reaktion Beer-Hofmanns\pwindex{Beer-Hofmann, Richard 1866-07-11 – 1945-09-26@\textsc{Beer-Hofmann, Richard} (1866-07-11 – 1945-09-26), \emph{Schriftsteller/Schriftstellerin}|pwk}
                        gibt dieser bekannt, ebenfalls ins Konzert\pwindex{Dohnányi, Ernst von 27.07.1877 – 09.02.1960@\textsc{Dohnányi, Ernst von} (27.07.1877 – 09.02.1960), \emph{Komponist/Komponistin, Pianist/Pianistin}|pwkv} zu wollen und lädt für den
                           29. 11. 1908 zu zwei Treffen, eins bei ihm zu Hause für den
                        Mittag und eins nach der Lesung am Abend. Im vorliegenden Schreiben versucht
                           Schnitzler die beiden Treffen auf
                        eines – das am Abend – zu reduzieren. In seiner zweiten Mitteilung versucht
                        er, es vom Privathaus in das Restaurant Meißl
                              {\kaufmannsund} Schadn\oindex{Meissl {\kaufmannsund} Schadn@\textbf{Meissl {\kaufmannsund} Schadn}, \emph{Hotel (K.HTL)}|pwk} zu verlegen.}}}\label{K_L01813-1}\pend
           
\pstart
           \textcolor{gray}{\textbf{Wien XVIII. Spoettelgasse 7\oindex{Edmund-Weiss-Gasse 7@\textbf{Edmund-Weiß-Gasse 7}, \emph{Wohngebäude (K.WHS)}|pw}.}}\pend
           \vspace{0.5em}
\pstart
           lieber Richard, wir fahren nicht auf den Se{\geminationm}ering\oindex{Semmering@\textbf{Semmering}, \emph{A.ADM3}|pw}, hingegen erlaube ich mir
               folgenden Vorſchlag. Wollen Sie nicht Kerr\pwindex{Kerr, Alfred 25.12.1867 – 12.10.1948@\textsc{Kerr, Alfred} (25.12.1867 – 12.10.1948), \emph{Schriftsteller/Schriftstellerin, Kritiker/Kritikerin}|pw}{ }ſtatt Mittag morgen Abend laden, wir (oder ich
               allein, (denn {\pb}Olga\pwindex{Schnitzler, Olga 17.01.1882 – 13.01.1970@\textsc{Schnitzler, Olga} (17.01.1882 – 13.01.1970), \emph{Schauspieler/Schauspielerin, Sänger/Sängerin}|pw} iſt nicht ſehr wohl, (daher \introOben{}auch\introOben{} möchte ich die morg. Einladg verreinen)))
               kämen nach dem Nachtmahl hinüber – \pend
           
\pstart
           Sie ſagen mirs Abend im Concert\pwindex{Dohnányi, Ernst von 27.07.1877 – 09.02.1960@\textsc{Dohnányi, Ernst von} (27.07.1877 – 09.02.1960), \emph{Komponist/Komponistin, Pianist/Pianistin}|pwv}\pend
           
\pstart
           Herzlichſt Ihr{\\[\baselineskip]}\spacefill\mbox{A.}\pend
           \leftskip=0em{}
\pstart
           \noindent{}{\pb}Wie gehts Noemi\pwindex{Beer-Hofmann, Naemah 20.12.1898 – 10.11.1971@\textsc{Beer-Hofmann, Naëmah} (20.12.1898 – 10.11.1971)|pw}\pend
           \selectlanguage{ngerman}\endnumbering\briefempfaengerindex{Beer-Hofmann, Richard@\textsc{Beer-Hofmann, Richard}!zzzSchnitzler, Arthur@\emph{von Arthur Schnitzler}!1908-11-283@{28. 11. 1908}|)be}\mylabel{L01813h}  \normalsize

\doendnotes{C}
\bigskip
\vfill

\clearpage

\footnotesize

\lohead{\textsc{register}}

% Definiere theindex-Environment komplett neu ohne reledmac
\makeatletter
\renewenvironment{theindex}{%
  \section*{\indexname}%
  \setlength{\parindent}{0pt}%
  \setlength{\parskip}{0pt plus 0.3pt}%
  \let\item\@idxitem
}{%
  \clearpage
}
\makeatother

\IfFileExists{\jobname-pw.ind}{\input{\jobname-pw.ind}}{}

\end{document}

      