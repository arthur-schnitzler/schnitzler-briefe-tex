%% latex-leseansicht-vorspann.tex
%% Vorspann für die Leseansicht.
%% Lädt die gemeinsame Datei latex-vorspann.tex mit nicht gesetztem Schalter.

\newif\ifkorrekturansicht
\korrekturansichtfalse

\input{../tex-inputs/latex-vorspann}


\section[Arthur Schnitzler an Richard Beer-Hofmann, 28. 11. 1908]{L01813 Arthur Schnitzler an Richard Beer-Hofmann, 28. 11. 1908}
\nopagebreak\mylabel{L01813v}
\rehead{ }\normalsize\beginnumbering\briefempfaengerindex{Beer-Hofmann, Richard@\textsc{Beer-Hofmann, Richard}!zzzSchnitzler, Arthur@\emph{von Arthur Schnitzler}!1908-11-283@{28. 11. 1908}|(be}
\toendnotes[C]{\smallbreak\pagebreak[2]}
\correspDesc{Versand  durch Arthur Schnitzler am 28. 11. 1908 in Wien
\newline{}Erhalt  durch Richard Beer-Hofmann im Zeitraum [28. 11. 1908 – 2. 12. 1908?] in Wien}\toendnotes[C]{\smallbreak}
\Standort{YCGL, MSS 31.}
\physDesc{Brief, 1 Blatt, 3 Seiten, Kuvert, 413 Zeichen
\newline{}Handschrift: 1) Bleistift, deutsche Kurrent\hspace{1em}2) schwarze Tinte, deutsche Kurrent (\noindent{}Umschlag)\hspace{1em}
\newline{}Versand: Stempel: »\nobreak{}\oindex{Wien@\textbf{Wien}, \emph{Verwaltungsgebiet}|pwk}Wien 3, 2\textcolor{gray}{8}. XI. 08, 4\nobreak{}«.  }
\buchAbdrucke{\weitereDrucke{Arthur Schnitzler, Richard Beer-Hofmann: \emph{Briefwechsel 1891–1931}. Herausgegeben von Konstanze Fliedl. Wien, Zürich: \emph{Europaverlag} 1992, S. 192.} }\toendnotes[C]{\smallbreak}\pstart{}{\pb}\textcolor{gray}{\textbf{Dr. Arthur Schnitzler}}\pend{}\pstart{}\textcolor{gray}{\textbf{Wien XVIII. Spoettelgasse 7\oindex{Wien@\textbf{Wien}!XVIII., Währing@\textbf{XVIII., Währing}!Edmund-Weiß-Gasse 7@\textbf{Edmund-Weiß-Gasse 7}, \emph{Wohngebäude}|pw}.}}\pend{}{\bigskip}\pstart{}{\pb}\textsc{Dr. Richard Beer Hofmann}\pend{}\pstart{}Wien XVIII\oindex{XVIII., Währing@\textbf{XVIII., Währing}, \emph{Verwaltungsgebiet}|pw}\pend{}\pstart{}\textsc{Hasenauerstr. 59}\oindex{Wien@\textbf{Wien}!XVIII., Währing@\textbf{XVIII., Währing}!Hasenauerstraße 59@\textbf{Hasenauerstraße 59}, \emph{Wohngebäude}|pw}.\pend{}{\bigskip}\vspace{1em}
\pstart
           {\pb}\textcolor{gray}{\textbf{Dr. Arthur Schnitzler}}\hfill \label{K_L01813-1v}\edtext{So{\geminationn}tag 28. 11 08}{\lemma{\textnormal{\emph{Sonntag 28. 11 08}}}\Cendnote{\textnormal{Der 28. 11. 1908 war ein Samstag. An
                        diesem Tag besuchte Schnitzler ein Konzert\pwindex{Dohnányi, Ernst von 27.\,7.\,1877 Bratislava – 9.\,2.\,1960 New York City@\textsc{Dohnányi, Ernst von} (27.\,7.\,1877 Bratislava – 9.\,2.\,1960 New York City), \emph{Komponist, Pianist}|pwkv}, was die
                        Datierung erlaubt. Trotzdem fordert das Fehlen eines Gegenbriefes (oder
                        handelte es sich um eine mündlich übermittelte Nachricht?) einige
                        Spekulation, wie die Reihenfolge des zweiten und dritten Briefes vom
                           28. 11. 1908 anzusetzen ist. Im ersten Brief erfährt Beer-Hofmann\pwindex{Beer-Hofmann, Richard 11.\,7.\,1866 Wien – 26.\,9.\,1945 New York City@\textsc{Beer-Hofmann, Richard} (11.\,7.\,1866 Wien – 26.\,9.\,1945 New York City), \emph{Schriftsteller}|pwk} vom Konzert\pwindex{Dohnányi, Ernst von 27.\,7.\,1877 Bratislava – 9.\,2.\,1960 New York City@\textsc{Dohnányi, Ernst von} (27.\,7.\,1877 Bratislava – 9.\,2.\,1960 New York City), \emph{Komponist, Pianist}|pwkv} und von der geplanten Reise
                        auf den Semmering\oindex{Semmering@\textbf{Semmering}, \emph{Verwaltungsgebiet}|pwk}. In der nicht
                        erhaltenen Reaktion Beer-Hofmanns\pwindex{Beer-Hofmann, Richard 11.\,7.\,1866 Wien – 26.\,9.\,1945 New York City@\textsc{Beer-Hofmann, Richard} (11.\,7.\,1866 Wien – 26.\,9.\,1945 New York City), \emph{Schriftsteller}|pwk}
                        gibt dieser bekannt, ebenfalls ins Konzert\pwindex{Dohnányi, Ernst von 27.\,7.\,1877 Bratislava – 9.\,2.\,1960 New York City@\textsc{Dohnányi, Ernst von} (27.\,7.\,1877 Bratislava – 9.\,2.\,1960 New York City), \emph{Komponist, Pianist}|pwkv} zu wollen und lädt für den
                           29. 11. 1908 zu zwei Treffen, eins bei ihm zu Hause für den
                        Mittag und eins nach der Lesung am Abend. Im vorliegenden Schreiben versucht
                           Schnitzler die beiden Treffen auf
                        eines – das am Abend – zu reduzieren. In seiner zweiten Mitteilung versucht
                        er, es vom Privathaus in das Restaurant Meißl
                              {\kaufmannsund} Schadn\oindex{Wien@\textbf{Wien}!I., Innere Stadt@\textbf{I., Innere Stadt}!Meissl {\kaufmannsund} Schadn@\textbf{Meissl {\kaufmannsund} Schadn}, \emph{Hotel}|pwk} zu verlegen.}}}\label{K_L01813-1}\pend
           
\pstart
           \textcolor{gray}{\textbf{Wien XVIII. Spoettelgasse 7\oindex{Wien@\textbf{Wien}!XVIII., Währing@\textbf{XVIII., Währing}!Edmund-Weiß-Gasse 7@\textbf{Edmund-Weiß-Gasse 7}, \emph{Wohngebäude}|pw}.}}\pend
           \vspace{0.5em}
\pstart
           lieber Richard, wir fahren nicht auf den Se{\geminationm}ering\oindex{Semmering@\textbf{Semmering}, \emph{Verwaltungsgebiet}|pw}, hingegen erlaube ich mir
               folgenden Vorſchlag. Wollen Sie nicht Kerr\pwindex{Kerr, Alfred 25.\,12.\,1867 Breslau – 12.\,10.\,1948 Hamburg@\textsc{Kerr, Alfred} (25.\,12.\,1867 Breslau – 12.\,10.\,1948 Hamburg), \emph{Schriftsteller, Kritiker}|pw}{ }ſtatt Mittag morgen Abend laden, wir (oder ich
               allein, (denn {\pb}Olga\pwindex{Schnitzler, Olga 17.\,1.\,1882 Wien – 13.\,1.\,1970 Lugano@\textsc{Schnitzler, Olga} (17.\,1.\,1882 Wien – 13.\,1.\,1970 Lugano), \emph{Schauspielerin, Sängerin}|pw} iſt nicht{ }ſehr wohl, (daher \introOben{}auch\introOben{} möchte ich die morg. Einladg verreinen)))
               kämen nach dem Nachtmahl hinüber –\pend
           
\pstart
           Sie{ }ſagen mirs Abend im Concert\pwindex{Dohnányi, Ernst von 27.\,7.\,1877 Bratislava – 9.\,2.\,1960 New York City@\textsc{Dohnányi, Ernst von} (27.\,7.\,1877 Bratislava – 9.\,2.\,1960 New York City), \emph{Komponist, Pianist}|pwv}\pend
           
\pstart
           Herzlichſt Ihr{\\[\baselineskip]}\spacefill\mbox{A.}\pend
           \leftskip=0em{}
\pstart
           \noindent{}{\pb}Wie gehts Noemi\pwindex{Beer-Hofmann, Naëmah 20.\,12.\,1898 Wien – 10.\,11.\,1971 New York City@\textsc{Beer-Hofmann, Naëmah} (20.\,12.\,1898 Wien – 10.\,11.\,1971 New York City)|pw}\pend
           \selectlanguage{ngerman}\endnumbering\briefempfaengerindex{Beer-Hofmann, Richard@\textsc{Beer-Hofmann, Richard}!zzzSchnitzler, Arthur@\emph{von Arthur Schnitzler}!1908-11-283@{28. 11. 1908}|)be}\mylabel{L01813h}  \newcommand{\dateiname}{L01813}\newcommand{\titel}{Arthur Schnitzler an Richard Beer-Hofmann, 28. 11. 1908}\newcommand{\editorInnen}{Martin Anton Müller und Gerd-Hermann Susen}%% latex-leseansicht-abspann.tex
%% Abspann für die Leseansicht.
%% Der Schalter \ifkorrekturansicht ist bereits durch den Vorspann gesetzt.

%% latex-abspann.tex
%% Gemeinsamer Abspann für Korrekturansicht und Leseansicht.
%% Setzt den Schalter \ifkorrekturansicht voraus (gesetzt in den
%% einbindenden Dateien latex-korrekturansicht-abspann.tex bzw.
%% latex-leseansicht-abspann.tex).
%% ---------------------------------------------------------------

\normalsize

% Das esempio-Environment wird nur in der Leseansicht benötigt
\ifkorrekturansicht\else
\newenvironment{esempio}[3]%
{
    \vspace{1.5ex}
    \rlap{\underline{#1}}
    \par
    \setlength{\parindent}{0cm}
    \nopagebreak
    \leftskip=#2cm
    \rightskip=#3cm
}
{
    \par
}
\fi

\doendnotes{C}
\bigskip
\vfill

\clearpage

\footnotesize

\ifkorrekturansicht
  \lohead{\textsc{register}}
\fi

% theindex-Environment neu definieren ohne reledmac
\makeatletter
\renewenvironment{theindex}{%
  \ifkorrekturansicht
    \section*{\indexname}%
  \else
    \subsubsection*{Index der erwähnten Entitäten}%
  \fi
  \setlength{\parindent}{0pt}%
  \setlength{\parskip}{0pt plus 0.3pt}%
  \let\item\@idxitem
}{%
  \ifkorrekturansicht\clearpage\fi
}
\makeatother

\IfFileExists{\jobname-pw.ind}{\input{\jobname-pw.ind}}{}

% Quellenangabe nur in der Leseansicht
\ifkorrekturansicht\else
% Fallback-Definitionen, falls die .tex-Datei \titel etc. nicht gesetzt hat
\providecommand{\titel}{}
\providecommand{\editorInnen}{}
\providecommand{\dateiname}{\jobname}

\vspace{3cm}

\vfill

\footnotesize
\textsc{Quelle}: \titel. Herausgegeben von {\editorInnen}. In: \emph{Arthur Schnitzler: Briefwechsel mit Autorinnen und Autoren}.
 Digitale Edition, https://schnitzler-briefe.acdh.oeaw.ac.at/{\dateiname}.html (Stand \today)
\fi

\end{document}


