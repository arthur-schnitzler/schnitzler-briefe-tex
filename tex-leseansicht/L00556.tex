%% latex-leseansicht-vorspann.tex
%% Vorspann für die Leseansicht.
%% Lädt die gemeinsame Datei latex-vorspann.tex mit nicht gesetztem Schalter.

\newif\ifkorrekturansicht
\korrekturansichtfalse

\input{../tex-inputs/latex-vorspann}


\section[Hugo von Hofmannsthal an Arthur Schnitzler, 27. 6. [1896]]{L00556 Hugo von Hofmannsthal an Arthur Schnitzler, 27. 6. [1896]}
\nopagebreak\mylabel{L00556v}
\rehead{ }\normalsize\beginnumbering\briefempfaengerindex{Schnitzler, Arthur@\textsc{Schnitzler, Arthur}!zzzHofmannsthal, Hugo von@\emph{von Hugo von Hofmannsthal}!1896-06-272@{27. 6. [1896]}|(be}
\toendnotes[C]{\smallbreak\pagebreak[2]}
\correspDesc{Versand  durch Hugo von Hofmannsthal am 27. 6. [1896] in Bad Fusch
\newline{}Erhalt  durch Arthur Schnitzler im Zeitraum [28. 6. 1896
                  – 2. 7. 1896?] in Wien}\toendnotes[C]{\smallbreak}
\Standort{CUL, Schnitzler, B 43.}
\physDesc{Brief, 1 Blatt, 3 Seiten, 948 Zeichen (aufgeprägtes Wappen)
\newline{}Handschrift: schwarze Tinte, deutsche Kurrent
\newline{}Schnitzler: mit Bleistift die Jahreszahl ergänzt: »96.« 
\newline{}Ordnung: mit Bleistift von unbekannter Hand nummeriert:
                                    »77« }
\buchAbdrucke{\weitereDrucke{1) Hugo von Hofmannsthal: \emph{Briefe. 1890–1901}. Berlin: \emph{S. Fischer} 1935, S. 204.} \weitereDrucke{2) Hugo von Hofmannsthal, Arthur Schnitzler: \emph{Briefwechsel}. Herausgegeben von Therese Nickl und Heinrich Schnitzler. Frankfurt am Main: \emph{S. Fischer} 1964, S. 67–68.} }\toendnotes[C]{\smallbreak}
\pstart
           \raggedleft{}{\pb}Bad Fuſch\oindex{Bad Fusch@\textbf{Bad Fusch}|pw}, 27\textsuperscript{ten} 6.\pend
           
\pstart{}lieber Arthur!\pend\vspace{0.5em}
\pstart
           ich denke, dieſer Brief erreicht Sie noch gerade vor Ihrer Abreiſe. Es wird mir dann{ }ſehr viel Freude machen, Sie auf dem Schiff und in fremden Gegenden zu denken. Zu
               meinem Vergnügen am Daſein gehört es{ }ſehr{ }ſtark, mir das Leben meiner Freunde
               merkwürdig und{ }ſchön vorzuſtellen. Es iſt das geheimnis{\pb}voll wie die Zuſammenſetzung von{ }ſchönen Gegenſtänden auf einem Bild.\pend
           
\pstart
           Ich lebe hier ganz{ }ſtill. Ich{ }ſchreibe eine Novelle\pwindex{Hofmannsthal, Hugo von 1.\,2.\,1874 Wien – 15.\,7.\,1929 Rodaun@\textsc{Hofmannsthal, Hugo von} (1.\,2.\,1874 Wien – 15.\,7.\,1929 Rodaun), \emph{Schriftsteller}!Geschichte der beiden Liebespaare@\strich\emph{Geschichte der beiden Liebespaare}|pwv} und{ }ſehe 5, 6 andere vor mir. Nur kommt mir{ }ſonderbarer Weiſe immer während des Arbeitens gerade die weſentliche Schönheit des
               Stoffes wie erblindet vor. Das muſs man wahrſcheinlich überwinden. Ich kann es nur
               nicht, weil ich bis jetzt eigentlich immer nur {\pb}kurze Gedichte gemacht habe.\pend
           
\pstart
           Sie laſſen mich dann immer wiſſen, wo Sie Briefe finden wollen, nicht wahr? (Vom
                     15\textsuperscript{ten} Juli ab{ }ſchreiben Sie mir nach
                  Wien\oindex{Wien@\textbf{Wien}, \emph{Verwaltungsgebiet}|pw}, weil ich nicht genau weiß wo ich{ }ſein
               werde.) Leben Sie wohl, lieber Arthur.\pend
           
\pstart
           Herzlich Ihr{\\[\baselineskip]}\spacefill\mbox{Hugo.}\pend
           \leftskip=0em{}\selectlanguage{ngerman}\endnumbering\briefempfaengerindex{Schnitzler, Arthur@\textsc{Schnitzler, Arthur}!zzzHofmannsthal, Hugo von@\emph{von Hugo von Hofmannsthal}!1896-06-272@{27. 6. [1896]}|)be}\mylabel{L00556h}  \newcommand{\dateiname}{L00556}\newcommand{\titel}{Hugo von Hofmannsthal an Arthur Schnitzler, 27. 6. [1896]}\newcommand{\editorInnen}{Martin Anton Müller und Gerd-Hermann Susen}%% latex-leseansicht-abspann.tex
%% Abspann für die Leseansicht.
%% Der Schalter \ifkorrekturansicht ist bereits durch den Vorspann gesetzt.

%% latex-abspann.tex
%% Gemeinsamer Abspann für Korrekturansicht und Leseansicht.
%% Setzt den Schalter \ifkorrekturansicht voraus (gesetzt in den
%% einbindenden Dateien latex-korrekturansicht-abspann.tex bzw.
%% latex-leseansicht-abspann.tex).
%% ---------------------------------------------------------------

\normalsize

% Das esempio-Environment wird nur in der Leseansicht benötigt
\ifkorrekturansicht\else
\newenvironment{esempio}[3]%
{
    \vspace{1.5ex}
    \rlap{\underline{#1}}
    \par
    \setlength{\parindent}{0cm}
    \nopagebreak
    \leftskip=#2cm
    \rightskip=#3cm
}
{
    \par
}
\fi

\doendnotes{C}
\bigskip
\vfill

\clearpage

\footnotesize

\ifkorrekturansicht
  \lohead{\textsc{register}}
\fi

% theindex-Environment neu definieren ohne reledmac
\makeatletter
\renewenvironment{theindex}{%
  \ifkorrekturansicht
    \section*{\indexname}%
  \else
    \subsubsection*{Index der erwähnten Entitäten}%
  \fi
  \setlength{\parindent}{0pt}%
  \setlength{\parskip}{0pt plus 0.3pt}%
  \let\item\@idxitem
}{%
  \ifkorrekturansicht\clearpage\fi
}
\makeatother

\IfFileExists{\jobname-pw.ind}{\input{\jobname-pw.ind}}{}

% Quellenangabe nur in der Leseansicht
\ifkorrekturansicht\else
% Fallback-Definitionen, falls die .tex-Datei \titel etc. nicht gesetzt hat
\providecommand{\titel}{}
\providecommand{\editorInnen}{}
\providecommand{\dateiname}{\jobname}

\vspace{3cm}

\vfill

\footnotesize
\textsc{Quelle}: \titel. Herausgegeben von {\editorInnen}. In: \emph{Arthur Schnitzler: Briefwechsel mit Autorinnen und Autoren}.
 Digitale Edition, https://schnitzler-briefe.acdh.oeaw.ac.at/{\dateiname}.html (Stand \today)
\fi

\end{document}


