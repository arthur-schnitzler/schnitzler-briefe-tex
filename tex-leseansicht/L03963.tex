%% latex-leseansicht-vorspann.tex
%% Vorspann für die Leseansicht.
%% Lädt die gemeinsame Datei latex-vorspann.tex mit nicht gesetztem Schalter.

\newif\ifkorrekturansicht
\korrekturansichtfalse

\input{../tex-inputs/latex-vorspann}


\section[Arthur Schnitzler an Berta Zuckerkandl, 2. 1. 1926]{L03963 Arthur Schnitzler an Berta Zuckerkandl, 2. 1. 1926}
\nopagebreak\mylabel{L03963v}
\rehead{ }\normalsize\beginnumbering\briefempfaengerindex{Zuckerkandl, Berta@\textsc{Zuckerkandl, Berta}!zzzSchnitzler, Arthur@\emph{von Arthur Schnitzler}!1926-01-021@{2. 1. 1926}|(be}
\toendnotes[C]{\smallbreak\pagebreak[2]}
\correspDesc{Versand  durch Arthur Schnitzler am 2. 1. 1926 in Wien
\newline{}Erhalt  durch Berta Zuckerkandl im Zeitraum [3. 1. 1926
                  – 7. 1. 1926?] in Paris}\toendnotes[C]{\smallbreak}
\Standort{DLA, HS.1985.1.2282.}
\physDesc{Brief, Durchschlag, 1 Blatt, 1 Seite, 1141 Zeichen
\newline{}Schreibmaschine
\newline{}Handschrift: roter Buntstift, lateinische Kurrent (\noindent{}beschriftet: »\uline{Zuckerkandl}« und »\uline{Frankreich}«, acht Unterstreichungen)}\toendnotes[C]{\smallbreak}
\pstart
           \raggedleft{}{\pb}2. 1. 1926.\pend
           
\pstart{}Liebe und verehrte Frau Hofrätin.\pend\vspace{0.5em}
\pstart
           Ich möchte Ihnen für heute nur mitteilen, dass ich von Herrn Nathan\pwindex{Nathan, Nicolas @\textsc{Nathan, Nicolas}, \emph{Übersetzer}|pw}, der sich augenblicklich in Vevey\oindex{Vevey@\textbf{Vevey}, \emph{Hauptstadt}|pw} (Hotel
                  d’Angleterre\oindex{Hôtel d' Angleterre@\textbf{Hôtel d' Angleterre}, \emph{Hotel}|pw}) {[}befindet,{]} eine \label{K_L03963-1v}\edtext{Bitte um Verlängerung}{\lemma{\textnormal{\emph{Bitte um Verlängerung}}}\Cendnote{\textnormal{Der Brief ist nicht überliefert, wohl aber die Antwort: Arthur Schnitzler an Nicolas Nathan\pwindex{Nathan, Nicolas @\textsc{Nathan, Nicolas}, \emph{Übersetzer}|pwk}, 2. 1. 1926, \emph{Deutsches Literaturarchiv Marbach},
                  HS.1985.1.1485. }}}\label{K_L03963-1} der Autorisation (»Casanovas Heimfahrt\pwindex{Schnitzler, Arthur 15. 5. 1862 Wien – 21. 10. 1931 ebd.@\textsc{Schnitzler, Arthur} (15. 5. 1862 Wien – 21. 10. 1931 ebd.), \emph{Schriftsteller, Mediziner}!Casanovas Heimfahrt@\strich\emph{Casanovas Heimfahrt}|pw}«) für ein halbes Jahr und aufklärende Mitteilungen über
               die Gründe der bisherigen Verzögerung erhalten habe.\pend
           
\pstart
           Die Uebersetzung\pwindex{Schnitzler, Arthur 15. 5. 1862 Wien – 21. 10. 1931 ebd.@\textsc{Schnitzler, Arthur} (15. 5. 1862 Wien – 21. 10. 1931 ebd.), \emph{Schriftsteller, Mediziner}!Madmoiselle Else@\strich\emph{Madmoiselle Else}|pw}, die Frau Pollaczek\pwindex{Pollaczek, Clara Katharina 15.\,1.\,1875 Wien – 22.\,7.\,1951 ebd.@\textsc{Pollaczek, Clara Katharina} (15.\,1.\,1875 Wien – 22.\,7.\,1951 ebd.), \emph{Schriftstellerin}|pw} von »Fräulein
                  Else\pwindex{Schnitzler, Arthur 15. 5. 1862 Wien – 21. 10. 1931 ebd.@\textsc{Schnitzler, Arthur} (15. 5. 1862 Wien – 21. 10. 1931 ebd.), \emph{Schriftsteller, Mediziner}!Fräulein Else@\strich\emph{Fräulein Else}|pw}« anfertigt, schreitet rasch fort und so viel ich beurteilen kann, wird
               sie sich sehr gut verwenden lassen, vielleicht sogar in einem weiteren Sinn als
               einfache Rohübersetzung. Wollen Sie mir gütigst schreiben, verehrte Freundin, \label{K_L03963-2v}\edtext{wie lange}{\lemma{\textnormal{\emph{wie lange}}}\Cendnote{\textnormal{Die erste Begegnung nach der Reise fand laut \emph{Tagebuch}\pwindex{Schnitzler, Arthur 15. 5. 1862 Wien – 21. 10. 1931 ebd.@\textsc{Schnitzler, Arthur} (15. 5. 1862 Wien – 21. 10. 1931 ebd.), \emph{Schriftsteller, Mediziner}!Tagebuch@\strich\emph{Tagebuch}|pwk} am 21. 2. 1926 statt.}}}\label{K_L03963-2} Sie sich in Paris\oindex{Paris@\textbf{Paris}, \emph{Hauptstadt}|pw} aufzuhalten gedenken.\pend
           
\pstart
           Gestern erhielt ich ein Exemplar der Revue de Geneve\pwindex{Revue de Genève@\emph{Revue de Genève}|pw} mit einer (\label{K_L03963-3v}\edtext{ich glaube der 3.}{\lemma{\textnormal{\emph{ich glaube der 3.}}}\Cendnote{\textnormal{Eine Übersetzung\pwindex{Schnitzler, Arthur 15. 5. 1862 Wien – 21. 10. 1931 ebd.@\textsc{Schnitzler, Arthur} (15. 5. 1862 Wien – 21. 10. 1931 ebd.), \emph{Schriftsteller, Mediziner}!Morts se taisent@\strich\emph{Les Morts se taisent}|pwkv} der Erzählung\pwindex{Schnitzler, Arthur 15. 5. 1862 Wien – 21. 10. 1931 ebd.@\textsc{Schnitzler, Arthur} (15. 5. 1862 Wien – 21. 10. 1931 ebd.), \emph{Schriftsteller, Mediziner}!Toten schweigen@\strich\emph{Die Toten schweigen}|pwkv} von Maurice Rémon\pwindex{Rémon, Maurice 27.\,11.\,1861 Paris – 20.\,6.\,1945 Mérignac@\textsc{Rémon, Maurice} (27.\,11.\,1861 Paris – 20.\,6.\,1945 Mérignac), \emph{Übersetzer}|pwk} und Noémie
                     Valentin\pwindex{Valentin, Noémi @\textsc{Valentin, Noémi}, \emph{Übersetzerin}|pwk} erschien 1902 in der \emph{Revue de Paris}\pwindex{Revue de Paris@\emph{La Revue de Paris}|pwk}, eine weitere unter dem Titel »\emph{Les morts ne parlent pas}\pwindex{Schnitzler, Arthur 15. 5. 1862 Wien – 21. 10. 1931 ebd.@\textsc{Schnitzler, Arthur} (15. 5. 1862 Wien – 21. 10. 1931 ebd.), \emph{Schriftsteller, Mediziner}!morts ne parlent pas@\strich\emph{Les morts ne parlent pas}|pwk}« ohne Angabe des
                  Übersetzers befindet sich im Nachlass, siehe Arthur Schnitzler: \emph{Mikrofilme}, \url{https://schnitzler\_mikrofilme.acdh.oeaw.ac.at/1429076}.}}}\label{K_L03963-3}) Uebersetzung\pwindex{Schnitzler, Arthur 15. 5. 1862 Wien – 21. 10. 1931 ebd.@\textsc{Schnitzler, Arthur} (15. 5. 1862 Wien – 21. 10. 1931 ebd.), \emph{Schriftsteller, Mediziner}!Morts se taisent@\strich\emph{Les Morts se taisent}|pwv} von »Die Toten
                  schweigen\pwindex{Schnitzler, Arthur 15. 5. 1862 Wien – 21. 10. 1931 ebd.@\textsc{Schnitzler, Arthur} (15. 5. 1862 Wien – 21. 10. 1931 ebd.), \emph{Schriftsteller, Mediziner}!Toten schweigen@\strich\emph{Die Toten schweigen}|pw}«, sowie nachträgliche \label{K_L03963-4v}\edtext{Bewerbung}{\lemma{\textnormal{\emph{Bewerbung}}}\Cendnote{\textnormal{Der Brief von Maury\pwindex{Maury, Geneviève 23.\,5.\,1886 Vevey – 21.\,8.\,1956 Paris@\textsc{Maury, Geneviève} (23.\,5.\,1886 Vevey – 21.\,8.\,1956 Paris), \emph{Übersetzerin, Bibliothekarin, Schriftstellerin}|pwk} ist nicht überliefert, aber die
                  Antwort darauf: Arthur Schnitzler an Geneviève Maury\pwindex{Maury, Geneviève 23.\,5.\,1886 Vevey – 21.\,8.\,1956 Paris@\textsc{Maury, Geneviève} (23.\,5.\,1886 Vevey – 21.\,8.\,1956 Paris), \emph{Übersetzerin, Bibliothekarin, Schriftstellerin}|pwk}, 2. 1. 1926, \emph{Deutsches Literaturarchiv Marbach}, HS.1985.1.1390
                  .}}}\label{K_L03963-4} um \label{T_L03963-1v}\edtext{die
                  Autorisation}{\lemma{\textnormal{\emph{die
                  Autorisation}}}\Cendnote{\textnormal{In der Vorlage steht:
                     »due Aztorisation«.}}}\label{T_L03963-1} von Seite der Uebersetzerin Geneviève Maury\pwindex{Maury, Geneviève 23.\,5.\,1886 Vevey – 21.\,8.\,1956 Paris@\textsc{Maury, Geneviève} (23.\,5.\,1886 Vevey – 21.\,8.\,1956 Paris), \emph{Übersetzerin, Bibliothekarin, Schriftstellerin}|pw}.\pend
           
\pstart
           Ich hoffe, Sie haben angenehme und in jedem Sinn fruchtbringende Tage in Paris\oindex{Paris@\textbf{Paris}, \emph{Hauptstadt}|pw}, hoffe sehr bald und Gutes von Ihnen zu
               hören und bin {[}mit{]} den herzlichsten \label{T_L03963-2v}\edtext{Neujahrsgrüssen}{\lemma{\textnormal{\emph{Neujahrsgrüssen}}}\Cendnote{\textnormal{In
                  der Vorlage steht: »Neujajrsgrüssen«.}}}\label{T_L03963-2} und Wünschen für Sie
               und die Ihrigen\pend
           \pstart Ihr auf richtig ergebener\pend{}{\vspace{1\baselineskip}}
\pstart
           \noindent{}Frau Hofrätin Berta Zuckerkandl,{\\}Paris\oindex{Paris@\textbf{Paris}, \emph{Hauptstadt}|pw}.\pend
           \selectlanguage{ngerman}\endnumbering\briefempfaengerindex{Zuckerkandl, Berta@\textsc{Zuckerkandl, Berta}!zzzSchnitzler, Arthur@\emph{von Arthur Schnitzler}!1926-01-021@{2. 1. 1926}|)be}\mylabel{L03963h}
\begin{anhang}
\end{anhang}\newcommand{\dateiname}{L03963}\newcommand{\titel}{Arthur Schnitzler an Berta Zuckerkandl, 2. 1. 1926}\newcommand{\editorInnen}{Herausgegeben von Jahnke, SelmaMüller, Martin Anton}%% latex-leseansicht-abspann.tex
%% Abspann für die Leseansicht.
%% Der Schalter \ifkorrekturansicht ist bereits durch den Vorspann gesetzt.

%% latex-abspann.tex
%% Gemeinsamer Abspann für Korrekturansicht und Leseansicht.
%% Setzt den Schalter \ifkorrekturansicht voraus (gesetzt in den
%% einbindenden Dateien latex-korrekturansicht-abspann.tex bzw.
%% latex-leseansicht-abspann.tex).
%% ---------------------------------------------------------------

\normalsize

% Das esempio-Environment wird nur in der Leseansicht benötigt
\ifkorrekturansicht\else
\newenvironment{esempio}[3]%
{
    \vspace{1.5ex}
    \rlap{\underline{#1}}
    \par
    \setlength{\parindent}{0cm}
    \nopagebreak
    \leftskip=#2cm
    \rightskip=#3cm
}
{
    \par
}
\fi

\doendnotes{C}
\bigskip
\vfill

\clearpage

\footnotesize

\ifkorrekturansicht
  \lohead{\textsc{register}}
\fi

% theindex-Environment neu definieren ohne reledmac
\makeatletter
\renewenvironment{theindex}{%
  \ifkorrekturansicht
    \section*{\indexname}%
  \else
    \subsubsection*{Index der erwähnten Entitäten}%
  \fi
  \setlength{\parindent}{0pt}%
  \setlength{\parskip}{0pt plus 0.3pt}%
  \let\item\@idxitem
}{%
  \ifkorrekturansicht\clearpage\fi
}
\makeatother

\IfFileExists{\jobname-pw.ind}{\input{\jobname-pw.ind}}{}

% Quellenangabe nur in der Leseansicht
\ifkorrekturansicht\else
% Fallback-Definitionen, falls die .tex-Datei \titel etc. nicht gesetzt hat
\providecommand{\titel}{}
\providecommand{\editorInnen}{}
\providecommand{\dateiname}{\jobname}

\vspace{3cm}

\vfill

\footnotesize
\textsc{Quelle}: \titel. Herausgegeben von {\editorInnen}. In: \emph{Arthur Schnitzler: Briefwechsel mit Autorinnen und Autoren}.
 Digitale Edition, https://schnitzler-briefe.acdh.oeaw.ac.at/{\dateiname}.html (Stand \today)
\fi

\end{document}


