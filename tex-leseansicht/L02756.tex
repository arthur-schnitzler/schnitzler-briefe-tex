%% latex-leseansicht-vorspann.tex
%% Vorspann für die Leseansicht.
%% Lädt die gemeinsame Datei latex-vorspann.tex mit nicht gesetztem Schalter.

\newif\ifkorrekturansicht
\korrekturansichtfalse

\input{../tex-inputs/latex-vorspann}


\section[Paul Goldmann an Arthur Schnitzler, 17. 10. {[}1895{]}]{L02756 Paul Goldmann an Arthur Schnitzler, 17. 10. [1895]}
\nopagebreak\mylabel{L02756v}
\rehead{ }\normalsize\beginnumbering\briefempfaengerindex{Schnitzler, Arthur@\textsc{Schnitzler, Arthur}!zzzGoldmann, Paul@\emph{von Paul Goldmann}!1895-10-171@{17. 10. [1895]}|(be}
\toendnotes[C]{\smallbreak\pagebreak[2]}
\correspDesc{Versand  durch Paul Goldmann am 17. 10. [1895] in Paris
\newline{}Erhalt  durch Arthur Schnitzler im Zeitraum [18. 10. 1895 – 22. 10. 1895?] in Wien}\toendnotes[C]{\smallbreak}
\Standort{DLA, A:Schnitzler, HS.NZ85.1.3165.}
\physDesc{Brief, 2 Blätter, 7 Seiten, 2822 Zeichen
\newline{}Handschrift: blaue Tinte, deutsche Kurrent
\newline{}Schnitzler: 1) mit Bleistift eine Unterstreichung, eine seitliche Markierung
                                 und das Jahr »95« vermerkt  2) mit rotem Buntstift acht Unterstreichungen}\toendnotes[C]{\smallbreak}
\pstart
           {\pb}\textcolor{gray}{\textbf{\textbf{Frankfurter Zeitung\orgindex{Frankfurter Zeitung@Frankfurter Zeitung|pw}}}}\pend
           
\pstart
           \textcolor{gray}{\textbf{(\begin{otherlanguage}{french}Gazette de Francfort\end{otherlanguage}\orgindex{Frankfurter Zeitung@Frankfurter Zeitung|pw}).}}\pend
           
\pstart
           \textcolor{gray}{\textbf{\textbf{\begin{otherlanguage}{french}Fondateur M. L.
                              Sonnemann\pwindex{Sonnemann, Leopold 29.\,10.\,1831 Höchberg – 30.\,10.\,1909 Frankfurt am Main@\textsc{Sonnemann, Leopold} (29.\,10.\,1831 Höchberg – 30.\,10.\,1909 Frankfurt am Main), \emph{Journalist, Herausgeber}|pw}\end{otherlanguage}.}}}\pend
           
\pstart
           \begin{otherlanguage}{french}\textcolor{gray}{\textbf{Journal politique, financier,}}\end{otherlanguage}\hfill \textsc{Paris\oindex{Paris@\textbf{Paris}, \emph{Hauptstadt}|pw}}, 17. Oktober.\pend
           
\pstart
           \begin{otherlanguage}{french}\textcolor{gray}{\textbf{commercial et littéraire.}}\end{otherlanguage}\pend
           
\pstart
           \begin{otherlanguage}{french}\textcolor{gray}{\textbf{\textbf{Paraissant trois fois par jour.}}}\end{otherlanguage}\pend
           
\pstart
           \begin{otherlanguage}{french}\textcolor{gray}{\textbf{\textbf{Bureau à Paris\oindex{Paris@\textbf{Paris}, \emph{Hauptstadt}|pw}}}}\end{otherlanguage}\pend
           
\pstart
           \begin{otherlanguage}{french}\textcolor{gray}{\textbf{\textbf{24. Rue Feydeau\oindex{rue Feydeau@\textbf{rue Feydeau}, \emph{Straße}|pw}.}}}\end{otherlanguage}\pend
           
\pstart\center{}Mein lieber Freund,\pend\vspace{0.5em}
\pstart
           Herzlichſten Dank für die Kritiken! Das iſt gar eine amüſante Lectüre. Wie \strikeout{d} Dein Bild da aus \strikeout{all}
               all’ den Spiegeln der Öffentlichkeit zurückgeworfen wird! Aber manchmal{ }ſieht es mich
               auch fremd an,{ }ſchmerzlich fremd, und meine trüben Ahnungen kommen wieder. Ja, ja,
               laß’ nur! Es iſt Unſinn, ich weiß{\dotsfive}\pend
           
\pstart
           Sehr intereſſant, dieſe Lectüre. Über \textsc{Speidel\pwindex{Speidel, Ludwig 11.\,4.\,1830 Ulm – 3.\,2.\,1906 Wien@\textsc{Speidel, Ludwig} (11.\,4.\,1830 Ulm – 3.\,2.\,1906 Wien), \emph{Journalist, Kritiker}|pw}\pwindex{Theater- und Kunstnachrichten. [Burgtheater] [Liebelei, Rechte der Seele]@\emph{Theater- und Kunstnachrichten. [Burgtheater] [Liebelei, Rechte der Seele]}|pwv}}{ }ſchrieb {\pb}ich Dir{ }ſchon. \label{K_L02756-1v}\edtext{\textsc{Kalbeck\pwindex{Kalbeck, Max 4.\,1.\,1850 Breslau – 4.\,5.\,1921 Wien@\textsc{Kalbeck, Max} (4.\,1.\,1850 Breslau – 4.\,5.\,1921 Wien), \emph{Journalist}!Theater, Kunst und Literatur. Burgtheater [Liebelei, Rechte der Seele]@\strich\emph{Theater, Kunst und Literatur. Burgtheater [Liebelei, Rechte der Seele]}|pwv}\pwindex{Kalbeck, Max 4.\,1.\,1850 Breslau – 4.\,5.\,1921 Wien@\textsc{Kalbeck, Max} (4.\,1.\,1850 Breslau – 4.\,5.\,1921 Wien), \emph{Journalist}!Burgtheater. »Liebelei«, Schauspiel in drei Acten von Arthur Schnitzler. – »Rechte der Seele«, Schauspiel in einem Acte von Guiseppe Giacosa; deutsch von Otto Eisenschitz@\strich\emph{Burgtheater. »Liebelei«, Schauspiel in drei Acten von Arthur Schnitzler. – »Rechte der Seele«, Schauspiel in einem Acte von Guiseppe Giacosa; deutsch von Otto Eisenschitz}|pwv}\pwindex{Kalbeck, Max 4.\,1.\,1850 Breslau – 4.\,5.\,1921 Wien@\textsc{Kalbeck, Max} (4.\,1.\,1850 Breslau – 4.\,5.\,1921 Wien), \emph{Journalist}|pw}}}{\lemma{\textnormal{\emph{Kalbeck}}}\Cendnote{\textnormal{Nachtkritik\pwindex{Kalbeck, Max 4.\,1.\,1850 Breslau – 4.\,5.\,1921 Wien@\textsc{Kalbeck, Max} (4.\,1.\,1850 Breslau – 4.\,5.\,1921 Wien), \emph{Journalist}!Theater, Kunst und Literatur. Burgtheater [Liebelei, Rechte der Seele]@\strich\emph{Theater, Kunst und Literatur. Burgtheater [Liebelei, Rechte der Seele]}|pwkv}: M. K.\pwindex{Kalbeck, Max 4.\,1.\,1850 Breslau – 4.\,5.\,1921 Wien@\textsc{Kalbeck, Max} (4.\,1.\,1850 Breslau – 4.\,5.\,1921 Wien), \emph{Journalist}|pwk} [ = Max Kalbeck\pwindex{Kalbeck, Max 4.\,1.\,1850 Breslau – 4.\,5.\,1921 Wien@\textsc{Kalbeck, Max} (4.\,1.\,1850 Breslau – 4.\,5.\,1921 Wien), \emph{Journalist}|pwk}]: \emph{Theater,
                        Kunst und Literatur. Burgtheater}\pwindex{Kalbeck, Max 4.\,1.\,1850 Breslau – 4.\,5.\,1921 Wien@\textsc{Kalbeck, Max} (4.\,1.\,1850 Breslau – 4.\,5.\,1921 Wien), \emph{Journalist}!Theater, Kunst und Literatur. Burgtheater [Liebelei, Rechte der Seele]@\strich\emph{Theater, Kunst und Literatur. Burgtheater [Liebelei, Rechte der Seele]}|pwk}. In: \emph{Neues Wiener Tagblatt}\pwindex{Neues Wiener Tagblatt@\emph{Neues Wiener Tagblatt}|pwk}, Jg. 29, Nr. 278, 10. 10. 1895, S. 7 und Feuilleton\pwindex{Kalbeck, Max 4.\,1.\,1850 Breslau – 4.\,5.\,1921 Wien@\textsc{Kalbeck, Max} (4.\,1.\,1850 Breslau – 4.\,5.\,1921 Wien), \emph{Journalist}!Burgtheater. »Liebelei«, Schauspiel in drei Acten von Arthur Schnitzler. – »Rechte der Seele«, Schauspiel in einem Acte von Guiseppe Giacosa; deutsch von Otto Eisenschitz@\strich\emph{Burgtheater. »Liebelei«, Schauspiel in drei Acten von Arthur Schnitzler. – »Rechte der Seele«, Schauspiel in einem Acte von Guiseppe Giacosa; deutsch von Otto Eisenschitz}|pwkv}: Max Kalbeck\pwindex{Kalbeck, Max 4.\,1.\,1850 Breslau – 4.\,5.\,1921 Wien@\textsc{Kalbeck, Max} (4.\,1.\,1850 Breslau – 4.\,5.\,1921 Wien), \emph{Journalist}|pwk}: \emph{Burgtheater. »Liebelei«, Schauspiel in drei Acten von Arthur
                        Schnitzler. – »Rechte der Seele«, Schauspiel in einem Acte von Guiseppe
                        Giacosa; deutsch von Otto Eisenschitz}\pwindex{Kalbeck, Max 4.\,1.\,1850 Breslau – 4.\,5.\,1921 Wien@\textsc{Kalbeck, Max} (4.\,1.\,1850 Breslau – 4.\,5.\,1921 Wien), \emph{Journalist}!Burgtheater. »Liebelei«, Schauspiel in drei Acten von Arthur Schnitzler. – »Rechte der Seele«, Schauspiel in einem Acte von Guiseppe Giacosa; deutsch von Otto Eisenschitz@\strich\emph{Burgtheater. »Liebelei«, Schauspiel in drei Acten von Arthur Schnitzler. – »Rechte der Seele«, Schauspiel in einem Acte von Guiseppe Giacosa; deutsch von Otto Eisenschitz}|pwk}. In: \emph{Neues Wiener Tagblatt}\pwindex{Neues Wiener Tagblatt@\emph{Neues Wiener Tagblatt}|pwk}, Jg. 29, Nr. 279, 11. 10. 1895, S. 1–3.}}}\label{K_L02756-1} iſt
               unerträglich{ }ſchwülſtig geſchrieben. Gefällt ihm das Stück\pwindex{Schnitzler, Arthur 15.\,5.\,1862 Wien – 21.\,10.\,1931 ebd.@\textsc{Schnitzler, Arthur} (15.\,5.\,1862 Wien – 21.\,10.\,1931 ebd.), \emph{Schriftsteller, Mediziner}!Liebelei. Schauspiel in drei Akten@\strich\emph{Liebelei. Schauspiel in drei Akten}|pwv} wirklich{ }ſo? Oder hat er nur
               vernommen, daß es \textsc{Speidel\pwindex{Speidel, Ludwig 11.\,4.\,1830 Ulm – 3.\,2.\,1906 Wien@\textsc{Speidel, Ludwig} (11.\,4.\,1830 Ulm – 3.\,2.\,1906 Wien), \emph{Journalist, Kritiker}|pw}} loben würde und{ }ſich darum beeilt, um die Wette zu loben, – auf Seiten der
               Mächtigen, wie immer? Ich glaube, der iſt kein echter, auf den kannſt Du Dich nicht verlaſſen\substVorne{}\textsuperscript{.}\substDazwischen{},\substHinten{} – wohl aber auf \textsc{Speidel\pwindex{Speidel, Ludwig 11.\,4.\,1830 Ulm – 3.\,2.\,1906 Wien@\textsc{Speidel, Ludwig} (11.\,4.\,1830 Ulm – 3.\,2.\,1906 Wien), \emph{Journalist, Kritiker}|pw}}. Schön iſt das Wohlwollen u. die Sympathie, die faſt bei \uline{Allen} zutage tritt. Einiges davon iſt wohl auf Rechnung des Wien\oindex{Wien@\textbf{Wien}, \emph{Verwaltungsgebiet}|pw}eriſchen zu{ }ſetzen, die {\pb}Hauptſache aber kommt aus der Achtung und dem
               Reſpect vor dem \uline{Menſchen}{ }\textsc{Schnitzler}. Durch warmen, \strikeout{\textcolor{gray}{×}u\textcolor{gray}{×}} herzlichen, neidloſen Ton ragt vor Allem \label{K_L02756-2v}\edtext{\uline{\textsc{Hirschfeld\pwindex{Hirschfeld, Robert 17.\,9.\,1857 Žďár nad Sázavou – 2.\,4.\,1914 Salzburg@\textsc{Hirschfeld, Robert} (17.\,9.\,1857 Žďár nad Sázavou – 2.\,4.\,1914 Salzburg), \emph{Journalist, Musikkritiker}|pw}\pwindex{Hirschfeld, Robert 17.\,9.\,1857 Žďár nad Sázavou – 2.\,4.\,1914 Salzburg@\textsc{Hirschfeld, Robert} (17.\,9.\,1857 Žďár nad Sázavou – 2.\,4.\,1914 Salzburg), \emph{Journalist, Musikkritiker}!Burgtheater. (»Liebelei« von Arthur Schnitzler. – »Rechte der Seele« von Giacosa.)@\strich\emph{Burgtheater. (»Liebelei« von Arthur Schnitzler. – »Rechte der Seele« von Giacosa.)}|pwv}}}}{\lemma{\textnormal{\emph{Hirschfeld}}}\Cendnote{\textnormal{L. A. Terne\pwindex{Hirschfeld, Robert 17.\,9.\,1857 Žďár nad Sázavou – 2.\,4.\,1914 Salzburg@\textsc{Hirschfeld, Robert} (17.\,9.\,1857 Žďár nad Sázavou – 2.\,4.\,1914 Salzburg), \emph{Journalist, Musikkritiker}|pwk} [ = Robert Hirschfeld\pwindex{Hirschfeld, Robert 17.\,9.\,1857 Žďár nad Sázavou – 2.\,4.\,1914 Salzburg@\textsc{Hirschfeld, Robert} (17.\,9.\,1857 Žďár nad Sázavou – 2.\,4.\,1914 Salzburg), \emph{Journalist, Musikkritiker}|pwk}]: \emph{Burgtheater. (»Liebelei« von Arthur Schnitzler. – »Rechte der Seele« von
                        Giacosa.)}\pwindex{Hirschfeld, Robert 17.\,9.\,1857 Žďár nad Sázavou – 2.\,4.\,1914 Salzburg@\textsc{Hirschfeld, Robert} (17.\,9.\,1857 Žďár nad Sázavou – 2.\,4.\,1914 Salzburg), \emph{Journalist, Musikkritiker}!Burgtheater. (»Liebelei« von Arthur Schnitzler. – »Rechte der Seele« von Giacosa.)@\strich\emph{Burgtheater. (»Liebelei« von Arthur Schnitzler. – »Rechte der Seele« von Giacosa.)}|pwk} In: \emph{Wiener Sonn- und
                        Montags-Zeitung}\pwindex{Wiener Sonn- und Montagszeitung@\emph{Wiener Sonn- und Montagszeitung}|pwk}, Jg. 33, Nr. 41, 14. 10. 1895, S. 1–3.}}}\label{K_L02756-2} hervor. Das iſt Einer, der{ }ſich
               wirklich mit Deinem Talent und Deinem Erfolge freut. Das Schönſte aber iſt – es iſt{ }ſeltſam, daß ich dieſem widerwärtigen Menſchen\pwindex{David, Jakob Julius 6.\,2.\,1859 Hranice – 20.\,11.\,1906 Wien@\textsc{David, Jakob Julius} (6.\,2.\,1859 Hranice – 20.\,11.\,1906 Wien), \emph{Schriftsteller, Journalist}|pwv} das Zugeſtändniß machen muß – \label{K_L02756-3v}\edtext{\textsc{\uline{J. J. Davids\pwindex{David, Jakob Julius 6.\,2.\,1859 Hranice – 20.\,11.\,1906 Wien@\textsc{David, Jakob Julius} (6.\,2.\,1859 Hranice – 20.\,11.\,1906 Wien), \emph{Schriftsteller, Journalist}|pw}}}{ }Feuilleton\pwindex{David, Jakob Julius 6.\,2.\,1859 Hranice – 20.\,11.\,1906 Wien@\textsc{David, Jakob Julius} (6.\,2.\,1859 Hranice – 20.\,11.\,1906 Wien), \emph{Schriftsteller, Journalist}!Arthur Schnitzler@\strich\emph{Arthur Schnitzler}|pwv}}{\lemma{\textnormal{\emph{J. J. Davids Feuilleton}}}\Cendnote{\textnormal{Am Tag der Uraufführung\eventindex{Burgtheater@\textbf{Burgtheater}!Uraufführung von Liebelei, Premiere von Rechte der Seele, 9.10.1895@Uraufführung von Liebelei, Premiere von Rechte der Seele, 9.10.1895|pwkv} erschien: 
                  –v–\pwindex{David, Jakob Julius 6.\,2.\,1859 Hranice – 20.\,11.\,1906 Wien@\textsc{David, Jakob Julius} (6.\,2.\,1859 Hranice – 20.\,11.\,1906 Wien), \emph{Schriftsteller, Journalist}|pwk} [ = J.
                        J. David\pwindex{David, Jakob Julius 6.\,2.\,1859 Hranice – 20.\,11.\,1906 Wien@\textsc{David, Jakob Julius} (6.\,2.\,1859 Hranice – 20.\,11.\,1906 Wien), \emph{Schriftsteller, Journalist}|pwk}]: \emph{Arthur Schnitzler}\pwindex{David, Jakob Julius 6.\,2.\,1859 Hranice – 20.\,11.\,1906 Wien@\textsc{David, Jakob Julius} (6.\,2.\,1859 Hranice – 20.\,11.\,1906 Wien), \emph{Schriftsteller, Journalist}!Arthur Schnitzler@\strich\emph{Arthur Schnitzler}|pwk}. In:
                        \emph{Neues Wiener Journal}\pwindex{Neues Wiener Journal@\emph{Neues Wiener Journal}|pwk}, Jg. 3, Nr. 703,
                        9. 10. 1895, S. 1–2. Zusätzlich dazu verfasste David\pwindex{David, Jakob Julius 6.\,2.\,1859 Hranice – 20.\,11.\,1906 Wien@\textsc{David, Jakob Julius} (6.\,2.\,1859 Hranice – 20.\,11.\,1906 Wien), \emph{Schriftsteller, Journalist}|pwk} eine Nachtkritik\pwindex{David, Jakob Julius 6.\,2.\,1859 Hranice – 20.\,11.\,1906 Wien@\textsc{David, Jakob Julius} (6.\,2.\,1859 Hranice – 20.\,11.\,1906 Wien), \emph{Schriftsteller, Journalist}!Theater und Kunst. (Burgtheater.) [Liebelei, Rechte der Seele]@\strich\emph{Theater und Kunst. (Burgtheater.) [Liebelei, Rechte der Seele]}|pwkv}: –v–\pwindex{David, Jakob Julius 6.\,2.\,1859 Hranice – 20.\,11.\,1906 Wien@\textsc{David, Jakob Julius} (6.\,2.\,1859 Hranice – 20.\,11.\,1906 Wien), \emph{Schriftsteller, Journalist}|pwk} [ = J.
                        J. David\pwindex{David, Jakob Julius 6.\,2.\,1859 Hranice – 20.\,11.\,1906 Wien@\textsc{David, Jakob Julius} (6.\,2.\,1859 Hranice – 20.\,11.\,1906 Wien), \emph{Schriftsteller, Journalist}|pwk}]: \emph{Theater und Kunst.
                        (Burgtheater.)}\pwindex{David, Jakob Julius 6.\,2.\,1859 Hranice – 20.\,11.\,1906 Wien@\textsc{David, Jakob Julius} (6.\,2.\,1859 Hranice – 20.\,11.\,1906 Wien), \emph{Schriftsteller, Journalist}!Theater und Kunst. (Burgtheater.) [Liebelei, Rechte der Seele]@\strich\emph{Theater und Kunst. (Burgtheater.) [Liebelei, Rechte der Seele]}|pwk} In: \emph{Neues Wiener
                        Journal}\pwindex{Neues Wiener Journal@\emph{Neues Wiener Journal}|pwk}, Jg. 3, Nr. 704, 10. 10. 1895, S. 5.}}}\label{K_L02756-3}
               über Dich. Das iſt prächtig geſchrieben, das iſt ein klug und wahr gezeichnetes
               Seelenbild von Dir, und das{ }ſchlägt {\pb}in meinem
               Innern liebe Saiten an, die lange nicht geklungen. Es hat mich tief berührt, und ich
               will dem Manne\pwindex{David, Jakob Julius 6.\,2.\,1859 Hranice – 20.\,11.\,1906 Wien@\textsc{David, Jakob Julius} (6.\,2.\,1859 Hranice – 20.\,11.\,1906 Wien), \emph{Schriftsteller, Journalist}|pwv} Manches um
               deßwillen verzeihen. \label{K_L02756-4v}\edtext{\textsc{\uline{Bauer\pwindex{Hofburgtheater [Rechte der Seele, Liebelei]@\emph{Hofburgtheater [Rechte der Seele, Liebelei]}|pwv}\pwindex{Bauer, Julius 15.\,10.\,1853 Szigetvár – 11.\,6.\,1941 Wien@\textsc{Bauer, Julius} (15.\,10.\,1853 Szigetvár – 11.\,6.\,1941 Wien), \emph{Schriftsteller, Journalist, Kritiker}|pw}}}}{\lemma{\textnormal{\emph{Bauer}}}\Cendnote{\textnormal{[Julius Bauer\pwindex{Bauer, Julius 15.\,10.\,1853 Szigetvár – 11.\,6.\,1941 Wien@\textsc{Bauer, Julius} (15.\,10.\,1853 Szigetvár – 11.\,6.\,1941 Wien), \emph{Schriftsteller, Journalist, Kritiker}|pwk}]: \emph{Hofburgtheater}\pwindex{Hofburgtheater [Rechte der Seele, Liebelei]@\emph{Hofburgtheater [Rechte der Seele, Liebelei]}|pwk}. In: \emph{Illustrirtes Wiener Extrablatt}\pwindex{Illustrirtes Wiener Extrablatt@\emph{Illustrirtes Wiener Extrablatt}|pwk}, Jg. 24, Nr. 278, 10. 10. 1895, S. 5.}}}\label{K_L02756-4} tadelt den Schluß\pwindex{Schnitzler, Arthur 15.\,5.\,1862 Wien – 21.\,10.\,1931 ebd.@\textsc{Schnitzler, Arthur} (15.\,5.\,1862 Wien – 21.\,10.\,1931 ebd.), \emph{Schriftsteller, Mediziner}!Liebelei. Schauspiel in drei Akten@\strich\emph{Liebelei. Schauspiel in drei Akten}|pwv}, und hat vielleicht
               nicht Unrecht. \label{K_L02756-5v}\edtext{\textsc{\uline{Hevesi\pwindex{Hevesi, Ludwig 20.\,12.\,1843 Heves – 27.\,2.\,1910 Wien@\textsc{Hevesi, Ludwig} (20.\,12.\,1843 Heves – 27.\,2.\,1910 Wien), \emph{Schriftsteller, Journalist}!Burgtheater. (Herr Mitterwurzer als König Philipp. – »Rechte der Seele«, von Guiseppe Giacosa. – »Liebelei«, von Arthur Schnitzler.)@\strich\emph{Burgtheater. (Herr Mitterwurzer als König Philipp. – »Rechte der Seele«, von Guiseppe Giacosa. – »Liebelei«, von Arthur Schnitzler.)}|pwv}\pwindex{Hevesi, Ludwig 20.\,12.\,1843 Heves – 27.\,2.\,1910 Wien@\textsc{Hevesi, Ludwig} (20.\,12.\,1843 Heves – 27.\,2.\,1910 Wien), \emph{Schriftsteller, Journalist}!Burgtheater. (»Rechte der Seele«, Schauspiel in einem Akt von Giuseppe Giacosa. – »Liebelei«, Schauspiel in drei Aufzügen von Arthur Schnitzler.)@\strich\emph{Burgtheater. (»Rechte der Seele«, Schauspiel in einem Akt von Giuseppe Giacosa. – »Liebelei«, Schauspiel in drei Aufzügen von Arthur Schnitzler.)}|pwv}}\pwindex{Hevesi, Ludwig 20.\,12.\,1843 Heves – 27.\,2.\,1910 Wien@\textsc{Hevesi, Ludwig} (20.\,12.\,1843 Heves – 27.\,2.\,1910 Wien), \emph{Schriftsteller, Journalist}|pw}}}{\lemma{\textnormal{\emph{Hevesi}}}\Cendnote{\textnormal{L. H–i\pwindex{Hevesi, Ludwig 20.\,12.\,1843 Heves – 27.\,2.\,1910 Wien@\textsc{Hevesi, Ludwig} (20.\,12.\,1843 Heves – 27.\,2.\,1910 Wien), \emph{Schriftsteller, Journalist}|pwk} [ = Ludwig Hevesi\pwindex{Hevesi, Ludwig 20.\,12.\,1843 Heves – 27.\,2.\,1910 Wien@\textsc{Hevesi, Ludwig} (20.\,12.\,1843 Heves – 27.\,2.\,1910 Wien), \emph{Schriftsteller, Journalist}|pwk}]: \emph{Burgtheater. (»Rechte der Seele«, Schauspiel in einem Akt von Giuseppe
                        Giacosa. – »Liebelei«, Schauspiel in drei Aufzügen von Arthur
                        Schnitzler.)}\pwindex{Hevesi, Ludwig 20.\,12.\,1843 Heves – 27.\,2.\,1910 Wien@\textsc{Hevesi, Ludwig} (20.\,12.\,1843 Heves – 27.\,2.\,1910 Wien), \emph{Schriftsteller, Journalist}!Burgtheater. (»Rechte der Seele«, Schauspiel in einem Akt von Giuseppe Giacosa. – »Liebelei«, Schauspiel in drei Aufzügen von Arthur Schnitzler.)@\strich\emph{Burgtheater. (»Rechte der Seele«, Schauspiel in einem Akt von Giuseppe Giacosa. – »Liebelei«, Schauspiel in drei Aufzügen von Arthur Schnitzler.)}|pwk} In: \emph{Fremden-Blatt}\pwindex{Fremden-Blatt@\emph{Fremden-Blatt}|pwk},
                     Jg. 51, Nr. 279, 11. 10. 1895, S. 13–14.
                  Unter den Zeitungsausschnitten Schnitzlers findet sich auch eine zweite Fassung, offenbar für eine Zeitung außerhalb Wiens\oindex{Wien@\textbf{Wien}, \emph{Verwaltungsgebiet}|pwk} verfasst (\emph{Breslauer Zeitung}\pwindex{Breslauer Zeitung@\emph{Breslauer Zeitung}|pwk}?): L. H–i\pwindex{Hevesi, Ludwig 20.\,12.\,1843 Heves – 27.\,2.\,1910 Wien@\textsc{Hevesi, Ludwig} (20.\,12.\,1843 Heves – 27.\,2.\,1910 Wien), \emph{Schriftsteller, Journalist}|pwk} [ = Ludwig Hevesi\pwindex{Hevesi, Ludwig 20.\,12.\,1843 Heves – 27.\,2.\,1910 Wien@\textsc{Hevesi, Ludwig} (20.\,12.\,1843 Heves – 27.\,2.\,1910 Wien), \emph{Schriftsteller, Journalist}|pwk}]: \emph{Burgtheater. (Herr Mitterwurzer als König Philipp. – »Rechte der Seele«,
                        von Guiseppe Giacosa. – »Liebelei«, von Arthur Schnitzler.)}\pwindex{Hevesi, Ludwig 20.\,12.\,1843 Heves – 27.\,2.\,1910 Wien@\textsc{Hevesi, Ludwig} (20.\,12.\,1843 Heves – 27.\,2.\,1910 Wien), \emph{Schriftsteller, Journalist}!Burgtheater. (Herr Mitterwurzer als König Philipp. – »Rechte der Seele«, von Guiseppe Giacosa. – »Liebelei«, von Arthur Schnitzler.)@\strich\emph{Burgtheater. (Herr Mitterwurzer als König Philipp. – »Rechte der Seele«, von Guiseppe Giacosa. – »Liebelei«, von Arthur Schnitzler.)}|pwk}.}}}\label{K_L02756-5}{ }\strikeout{m} iſt vortrefflich und geſcheit; beſonders \strikeout{das}, was er über die Paradoxe{ }ſagt,{ }ſind goldene Worte.
                  \label{K_L02756-6v}\edtext{\textsc{\uline{Uhl\pwindex{K. k. Hofburgtheater: »Rechte der Seele«, Schauspiel in einem Acte von Giuseppe Giacosa. – »Liebelei«, Schauspiel in drei Acten von Arthur Schnitzler. Zum ersten Male aufgeführt am 9. October@\emph{K. k. Hofburgtheater: »Rechte der Seele«, Schauspiel in einem Acte von Giuseppe Giacosa. – »Liebelei«, Schauspiel in drei Acten von Arthur Schnitzler. Zum ersten Male aufgeführt am 9. October}|pwv}\pwindex{Uhl, Friedrich 14.\,5.\,1825 Cieszyn – 20.\,1.\,1906 Mondsee@\textsc{Uhl, Friedrich} (14.\,5.\,1825 Cieszyn – 20.\,1.\,1906 Mondsee), \emph{Journalist}|pw}}}}{\lemma{\textnormal{\emph{Uhl}}}\Cendnote{\textnormal{[Friedrich Uhl\pwindex{Uhl, Friedrich 14.\,5.\,1825 Cieszyn – 20.\,1.\,1906 Mondsee@\textsc{Uhl, Friedrich} (14.\,5.\,1825 Cieszyn – 20.\,1.\,1906 Mondsee), \emph{Journalist}|pwk}]: \emph{K. k. Hofburgtheater: »Rechte der Seele«, Schauspiel in
                        einem Acte von Giuseppe Giacosa. – »Liebelei«, Schauspiel in drei Acten von
                        Arthur Schnitzler. Zum ersten Male aufgeführt am 9. October}\pwindex{K. k. Hofburgtheater: »Rechte der Seele«, Schauspiel in einem Acte von Giuseppe Giacosa. – »Liebelei«, Schauspiel in drei Acten von Arthur Schnitzler. Zum ersten Male aufgeführt am 9. October@\emph{K. k. Hofburgtheater: »Rechte der Seele«, Schauspiel in einem Acte von Giuseppe Giacosa. – »Liebelei«, Schauspiel in drei Acten von Arthur Schnitzler. Zum ersten Male aufgeführt am 9. October}|pwk}. In: \emph{Wiener Abendpost}\pwindex{Wiener Abendpost@\emph{Wiener Abendpost}|pwk}, Nr. 234, 10. 10. 1895, S. 1–2.}}}\label{K_L02756-6} iſt merkwürdig
               boshaft, hat \strikeout{ſichtlich}{ }ſichtlich in der Abſicht
               geſchrieben, Dir wehzuthun, packt das Stück\pwindex{Schnitzler, Arthur 15.\,5.\,1862 Wien – 21.\,10.\,1931 ebd.@\textsc{Schnitzler, Arthur} (15.\,5.\,1862 Wien – 21.\,10.\,1931 ebd.), \emph{Schriftsteller, Mediziner}!Liebelei. Schauspiel in drei Akten@\strich\emph{Liebelei. Schauspiel in drei Akten}|pwv} viel zu{ }ſchwer {\pb}an,{ }ſagt aber{ }ſchließlich doch manches Beherzigenswerthe;{ }ſein Tadel gegen die Figur des Vaters\pwindex{Schnitzler, Arthur 15.\,5.\,1862 Wien – 21.\,10.\,1931 ebd.@\textsc{Schnitzler, Arthur} (15.\,5.\,1862 Wien – 21.\,10.\,1931 ebd.), \emph{Schriftsteller, Mediziner}!Liebelei. Schauspiel in drei Akten@\strich\emph{Liebelei. Schauspiel in drei Akten}|pwv} iſt viel zu
                  \strikeout{heft\textcolor{gray}{i}} heftig ausgedrückt, aber im Grunde{ }ſcheint er Recht zu haben. Durch beſondere
               Dummheit zeichnet{ }ſich \label{K_L02756-7v}\edtext{\textsc{\uline{Bunzl\pwindex{Bunzl, Arthur 25.\,12.\,1849 Prag – 26.\,3.\,1899 Wien@\textsc{Bunzl, Arthur} (25.\,12.\,1849 Prag – 26.\,3.\,1899 Wien), \emph{Journalist, Chefredakteur}!Burgtheater. »Rechte der Seele«, Schauspiel in einem Akt von Giuseppe Giacosa. Deutsch von Otto Eisenschütz. – »Liebelei«, Schauspiel in drei Akten von Arthur Schnitzler. Zum erstenmale aufgeführt am 9. Oktober@\strich\emph{Burgtheater. »Rechte der Seele«, Schauspiel in einem Akt von Giuseppe Giacosa. Deutsch von Otto Eisenschütz. – »Liebelei«, Schauspiel in drei Akten von Arthur Schnitzler. Zum erstenmale aufgeführt am 9. Oktober}|pwv}\pwindex{Bunzl, Arthur 25.\,12.\,1849 Prag – 26.\,3.\,1899 Wien@\textsc{Bunzl, Arthur} (25.\,12.\,1849 Prag – 26.\,3.\,1899 Wien), \emph{Journalist, Chefredakteur}|pw}}}}{\lemma{\textnormal{\emph{Bunzl}}}\Cendnote{\textnormal{Arthur Bunzl\pwindex{Bunzl, Arthur 25.\,12.\,1849 Prag – 26.\,3.\,1899 Wien@\textsc{Bunzl, Arthur} (25.\,12.\,1849 Prag – 26.\,3.\,1899 Wien), \emph{Journalist, Chefredakteur}|pwk}: \emph{Burgtheater. »Rechte der Seele«, Schauspiel in einem Akt von
                        Giuseppe Giacosa. Deutsch von Otto Eisenschütz. – »Liebelei«, Schauspiel in
                        drei Akten von Arthur Schnitzler. Zum erstenmale aufgeführt am
                        9. Oktober}\pwindex{Bunzl, Arthur 25.\,12.\,1849 Prag – 26.\,3.\,1899 Wien@\textsc{Bunzl, Arthur} (25.\,12.\,1849 Prag – 26.\,3.\,1899 Wien), \emph{Journalist, Chefredakteur}!Burgtheater. »Rechte der Seele«, Schauspiel in einem Akt von Giuseppe Giacosa. Deutsch von Otto Eisenschütz. – »Liebelei«, Schauspiel in drei Akten von Arthur Schnitzler. Zum erstenmale aufgeführt am 9. Oktober@\strich\emph{Burgtheater. »Rechte der Seele«, Schauspiel in einem Akt von Giuseppe Giacosa. Deutsch von Otto Eisenschütz. – »Liebelei«, Schauspiel in drei Akten von Arthur Schnitzler. Zum erstenmale aufgeführt am 9. Oktober}|pwk}. In: \emph{Österreichische
                        Volks-Zeitung}\pwindex{Österreichische Volks-Zeitung@\emph{Österreichische Volks-Zeitung}|pwk}, Jg. 41, Nr. 279, 11. 10. 1895, S. 1–2.}}}\label{K_L02756-7} aus; er war aber immer ein Ochs.
               Köſtlich iſt die künſtleriſche Strenge des »\label{K_L02756-8v}\edtext{Neuigkeits-Weltblatts\pwindex{Neuigkeits-Welt-Blatt@\emph{Neuigkeits-Welt-Blatt}|pw}\pwindex{Alpha @\textsc{Alpha}, \emph{Journalist/Journalistin}!Hofburgtheater. (»Rechte der Seele«, Schauspiel in einem Akte von Guiseppe Giacosa. – »Liebelei«, Schauspiel in drei Akten von Arthur Schnitzler. – Erstaufführung am 9. Oktober 1895.)@\strich\emph{Hofburgtheater. (»Rechte der Seele«, Schauspiel in einem Akte von Guiseppe Giacosa. – »Liebelei«, Schauspiel in drei Akten von Arthur Schnitzler. – Erstaufführung am 9. Oktober 1895.)}|pwv}}{\lemma{\textnormal{\emph{Neuigkeits-Weltblatts}}}\Cendnote{\textnormal{Alpha\pwindex{Alpha @\textsc{Alpha}, \emph{Journalist/Journalistin}|pwk}: \emph{Hofburgtheater. (»Rechte der Seele«, Schauspiel in einem Akte von Guiseppe
                        Giacosa. – »Liebelei«, Schauspiel in drei Akten von Arthur Schnitzler. –
                        Erstaufführung am 9. Oktober 1895.)}\pwindex{Alpha @\textsc{Alpha}, \emph{Journalist/Journalistin}!Hofburgtheater. (»Rechte der Seele«, Schauspiel in einem Akte von Guiseppe Giacosa. – »Liebelei«, Schauspiel in drei Akten von Arthur Schnitzler. – Erstaufführung am 9. Oktober 1895.)@\strich\emph{Hofburgtheater. (»Rechte der Seele«, Schauspiel in einem Akte von Guiseppe Giacosa. – »Liebelei«, Schauspiel in drei Akten von Arthur Schnitzler. – Erstaufführung am 9. Oktober 1895.)}|pwk} In: \emph{Neuigkeits-Welt-Blatt}\pwindex{Neuigkeits-Welt-Blatt@\emph{Neuigkeits-Welt-Blatt}|pwk}, Jg. 22, Nr. 235, 12. 10. 1895, S. 10.}}}\label{K_L02756-8}«. Hübſch{ }ſind auch die \label{K_L02756-9v}\edtext{\uline{Socialiſten}\pwindex{Wengraf, Edmund 9.\,1.\,1860 Mikulov – 8.\,12.\,1933 Wien@\textsc{Wengraf, Edmund} (9.\,1.\,1860 Mikulov – 8.\,12.\,1933 Wien), \emph{Schriftsteller, Journalist, Kaufmann}!Burgtheater. [Rechte der Seele, Liebelei]@\strich\emph{Burgtheater. [Rechte der Seele, Liebelei]}|pwv}\pwindex{Wengraf, Edmund 9.\,1.\,1860 Mikulov – 8.\,12.\,1933 Wien@\textsc{Wengraf, Edmund} (9.\,1.\,1860 Mikulov – 8.\,12.\,1933 Wien), \emph{Schriftsteller, Journalist, Kaufmann}|pwv}}{\lemma{\textnormal{\emph{Socialisten}}}\Cendnote{\textnormal{e. w.\pwindex{Wengraf, Edmund 9.\,1.\,1860 Mikulov – 8.\,12.\,1933 Wien@\textsc{Wengraf, Edmund} (9.\,1.\,1860 Mikulov – 8.\,12.\,1933 Wien), \emph{Schriftsteller, Journalist, Kaufmann}|pwk} [ = Edmund Wengraf\pwindex{Wengraf, Edmund 9.\,1.\,1860 Mikulov – 8.\,12.\,1933 Wien@\textsc{Wengraf, Edmund} (9.\,1.\,1860 Mikulov – 8.\,12.\,1933 Wien), \emph{Schriftsteller, Journalist, Kaufmann}|pwk}]: \emph{Burgtheater}\pwindex{Wengraf, Edmund 9.\,1.\,1860 Mikulov – 8.\,12.\,1933 Wien@\textsc{Wengraf, Edmund} (9.\,1.\,1860 Mikulov – 8.\,12.\,1933 Wien), \emph{Schriftsteller, Journalist, Kaufmann}!Burgtheater. [Rechte der Seele, Liebelei]@\strich\emph{Burgtheater. [Rechte der Seele, Liebelei]}|pwk}. In: \emph{Arbeiter-Zeitung}\pwindex{Arbeiter-Zeitung@\emph{Arbeiter-Zeitung}|pwk}, Jg. 7, Nr. 279, 11. 10. 1895, Morgenblatt, S. 5.}}}\label{K_L02756-9}, welche unzufrieden{ }ſind, {\pb}weil das Stück\pwindex{Schnitzler, Arthur 15.\,5.\,1862 Wien – 21.\,10.\,1931 ebd.@\textsc{Schnitzler, Arthur} (15.\,5.\,1862 Wien – 21.\,10.\,1931 ebd.), \emph{Schriftsteller, Mediziner}!Liebelei. Schauspiel in drei Akten@\strich\emph{Liebelei. Schauspiel in drei Akten}|pwv} nicht nach Dreck{ }ſtinkt: »\label{K_L02756-10v}\edtext{Das iſt nicht das wahre Volk\pwindex{Wengraf, Edmund 9.\,1.\,1860 Mikulov – 8.\,12.\,1933 Wien@\textsc{Wengraf, Edmund} (9.\,1.\,1860 Mikulov – 8.\,12.\,1933 Wien), \emph{Schriftsteller, Journalist, Kaufmann}!Burgtheater. [Rechte der Seele, Liebelei]@\strich\emph{Burgtheater. [Rechte der Seele, Liebelei]}|pwv}}{\lemma{\textnormal{\emph{Das … Volk}}}\Cendnote{\textnormal{Paraphrase, kein direktes
               Zitat}}}\label{K_L02756-10}«. Daß{ }ſelbſt die \uline{Antiſemiten}\pwindex{r. p. @\textsc{r. p.}, \emph{Journalist/Journalistin}|pwv} über Dich{ }ſympathiſch{ }ſchreiben (»\label{K_L02756-11v}\edtext{Reichspoſt\pwindex{Reichspost@\emph{Reichspost}|pw}\pwindex{r. p. @\textsc{r. p.}, \emph{Journalist/Journalistin}!k. k. Hofburgtheater [Rechte der Seele, Liebelei]@\strich\emph{k. k. Hofburgtheater [Rechte der Seele, Liebelei]}|pwv}}{\lemma{\textnormal{\emph{Reichspost}}}\Cendnote{\textnormal{r. p.\pwindex{r. p. @\textsc{r. p.}, \emph{Journalist/Journalistin}|pwk}: \emph{k. k. Hofburgtheater}\pwindex{r. p. @\textsc{r. p.}, \emph{Journalist/Journalistin}!k. k. Hofburgtheater [Rechte der Seele, Liebelei]@\strich\emph{k. k. Hofburgtheater [Rechte der Seele, Liebelei]}|pwk}. In: \emph{Reichspost}\pwindex{Reichspost@\emph{Reichspost}|pwk}, Jg. 2, Nr. 235, 12. 10. 1895, S. 1.}}}\label{K_L02756-11}{[}«{]}), iſt ein wahrer Triumph für Dich und beweiſt abermals, daß
               der Antiſemitismus{ }ſich nur gegen die widerlichen Saujuden richtet und vor dem
               ehrenhaften und tüchtigen Juden entwaffnen muß. \label{K_L02756-12v}\edtext{\textsc{\uline{Granichstaedten\pwindex{Granichstaedten, Emil 8.\,7.\,1847 Wien – 2.\,7.\,1904 Berlin@\textsc{Granichstaedten, Emil} (8.\,7.\,1847 Wien – 2.\,7.\,1904 Berlin), \emph{Journalist, Rechtswissenschaftler}!Burgtheater. Zwei Schauspiele: »Rechte der Seele« von Giuseppe Giacosa. – »Liebelei« von Arthur Schnitzler@\strich\emph{Burgtheater. Zwei Schauspiele: »Rechte der Seele« von Giuseppe Giacosa. – »Liebelei« von Arthur Schnitzler}|pwv}\pwindex{Granichstaedten, Emil 8.\,7.\,1847 Wien – 2.\,7.\,1904 Berlin@\textsc{Granichstaedten, Emil} (8.\,7.\,1847 Wien – 2.\,7.\,1904 Berlin), \emph{Journalist, Rechtswissenschaftler}|pw}}}}{\lemma{\textnormal{\emph{Granichstaedten}}}\Cendnote{\textnormal{Emil Granichstaedten\pwindex{Granichstaedten, Emil 8.\,7.\,1847 Wien – 2.\,7.\,1904 Berlin@\textsc{Granichstaedten, Emil} (8.\,7.\,1847 Wien – 2.\,7.\,1904 Berlin), \emph{Journalist, Rechtswissenschaftler}|pwk}: \emph{Burgtheater. Zwei Schauspiele: »Rechte der Seele« von
                        Giuseppe Giacosa. – »Liebelei« von Arthur Schnitzler}\pwindex{Granichstaedten, Emil 8.\,7.\,1847 Wien – 2.\,7.\,1904 Berlin@\textsc{Granichstaedten, Emil} (8.\,7.\,1847 Wien – 2.\,7.\,1904 Berlin), \emph{Journalist, Rechtswissenschaftler}!Burgtheater. Zwei Schauspiele: »Rechte der Seele« von Giuseppe Giacosa. – »Liebelei« von Arthur Schnitzler@\strich\emph{Burgtheater. Zwei Schauspiele: »Rechte der Seele« von Giuseppe Giacosa. – »Liebelei« von Arthur Schnitzler}|pwk}. In: \emph{Die Presse}\pwindex{Presse@\emph{Die Presse}|pwk}, Jg. 48, Nr. 279, 11. 10. 1895, S. 1–2.}}}\label{K_L02756-12} iſt{ }ſo
               ungeſchickt und offen gemein, daß es {\pb}nicht einmal
               empört; jede Zeile{ }ſagt{ }ſelbſt dem \strikeout{\textcolor{gray}{enh}} nichteingeweihten Leſer im Vertrauen, daß der Verfaſſer\pwindex{Granichstaedten, Emil 8.\,7.\,1847 Wien – 2.\,7.\,1904 Berlin@\textsc{Granichstaedten, Emil} (8.\,7.\,1847 Wien – 2.\,7.\,1904 Berlin), \emph{Journalist, Rechtswissenschaftler}|pwv} lügt{\dotsfive}\pend
           
\pstart
           Das Geſammtbild iſt glänzend; und der Erfolg iſt{ }ſo groß, wie ich ihn nur irgend für
               Dich wünſchen konnte. Jetzt mach’ Dich bald und frohen Muthes an die neue Arbeit!\pend
           
\pstart
           Viele treue Grüße! {\\[\baselineskip]}Dein {\\[\baselineskip]}\spacefill\mbox{Paul Goldmnn.}\pend
           \leftskip=0em{}\selectlanguage{ngerman}\endnumbering\briefempfaengerindex{Schnitzler, Arthur@\textsc{Schnitzler, Arthur}!zzzGoldmann, Paul@\emph{von Paul Goldmann}!1895-10-171@{17. 10. [1895]}|)be}\mylabel{L02756h}  \newcommand{\dateiname}{L02756}\newcommand{\titel}{Paul Goldmann an Arthur Schnitzler, 17. 10. [1895]}\newcommand{\editorInnen}{Martin Anton Müller und Laura Untner}%% latex-leseansicht-abspann.tex
%% Abspann für die Leseansicht.
%% Der Schalter \ifkorrekturansicht ist bereits durch den Vorspann gesetzt.

%% latex-abspann.tex
%% Gemeinsamer Abspann für Korrekturansicht und Leseansicht.
%% Setzt den Schalter \ifkorrekturansicht voraus (gesetzt in den
%% einbindenden Dateien latex-korrekturansicht-abspann.tex bzw.
%% latex-leseansicht-abspann.tex).
%% ---------------------------------------------------------------

\normalsize

% Das esempio-Environment wird nur in der Leseansicht benötigt
\ifkorrekturansicht\else
\newenvironment{esempio}[3]%
{
    \vspace{1.5ex}
    \rlap{\underline{#1}}
    \par
    \setlength{\parindent}{0cm}
    \nopagebreak
    \leftskip=#2cm
    \rightskip=#3cm
}
{
    \par
}
\fi

\doendnotes{C}
\bigskip
\vfill

\clearpage

\footnotesize

\ifkorrekturansicht
  \lohead{\textsc{register}}
\fi

% theindex-Environment neu definieren ohne reledmac
\makeatletter
\renewenvironment{theindex}{%
  \ifkorrekturansicht
    \section*{\indexname}%
  \else
    \subsubsection*{Index der erwähnten Entitäten}%
  \fi
  \setlength{\parindent}{0pt}%
  \setlength{\parskip}{0pt plus 0.3pt}%
  \let\item\@idxitem
}{%
  \ifkorrekturansicht\clearpage\fi
}
\makeatother

\IfFileExists{\jobname-pw.ind}{\input{\jobname-pw.ind}}{}

% Quellenangabe nur in der Leseansicht
\ifkorrekturansicht\else
% Fallback-Definitionen, falls die .tex-Datei \titel etc. nicht gesetzt hat
\providecommand{\titel}{}
\providecommand{\editorInnen}{}
\providecommand{\dateiname}{\jobname}

\vspace{3cm}

\vfill

\footnotesize
\textsc{Quelle}: \titel. Herausgegeben von {\editorInnen}. In: \emph{Arthur Schnitzler: Briefwechsel mit Autorinnen und Autoren}.
 Digitale Edition, https://schnitzler-briefe.acdh.oeaw.ac.at/{\dateiname}.html (Stand \today)
\fi

\end{document}


