%% latex-leseansicht-vorspann.tex
%% Vorspann für die Leseansicht.
%% Lädt die gemeinsame Datei latex-vorspann.tex mit nicht gesetztem Schalter.

\newif\ifkorrekturansicht
\korrekturansichtfalse

\input{../tex-inputs/latex-vorspann}


         
         \newcommand{\erwaehntePersonen}{Personen:  Alpha, Julius Bauer, Arthur Bunzl, Jakob Julius David, Emil Granichstaedten, Ludwig Hevesi, Robert Hirschfeld, Max Kalbeck, Leopold Sonnemann, Ludwig Speidel, Friedrich Uhl, Edmund Wengraf,  r. p.}
         \newcommand{\erwaehnteInstitutionen}{Institutionen: Frankfurter Zeitung}
         \newcommand{\erwaehnteOrte}{Orte: Paris, Wien, rue Feydeau}
         \newcommand{\erwaehnteWerke}{Werke: Arbeiter-Zeitung, Arthur Schnitzler, Breslauer Zeitung, Burgtheater. (Herr Mitterwurzer als König Philipp. – »Rechte der Seele«, von Guiseppe Giacosa. – »Liebelei«, von Arthur Schnitzler.), Burgtheater. (»Liebelei« von Arthur Schnitzler. – »Rechte der Seele« von Giacosa.), Burgtheater. (»Rechte der Seele«, Schauspiel in einem Akt von Giuseppe Giacosa. – »Liebelei«, Schauspiel in drei Aufzügen von Arthur Schnitzler.), Burgtheater. Zwei Schauspiele: »Rechte der Seele« von Giuseppe Giacosa. – »Liebelei« von Arthur Schnitzler, Burgtheater. [Rechte der Seele, Liebelei], Burgtheater. »Liebelei«, Schauspiel in drei Acten von Arthur Schnitzler. – »Rechte der Seele«, Schauspiel in einem Acte von Guiseppe Giacosa; deutsch von Otto Eisenschitz, Burgtheater. »Rechte der Seele«, Schauspiel in einem Akt von Giuseppe Giacosa. Deutsch von Otto Eisenschütz. – »Liebelei«, Schauspiel in drei Akten von Arthur Schnitzler. Zum erstenmale aufgeführt am 9. Oktober, Die Presse, Fremden-Blatt, Hofburgtheater [Rechte der Seele, Liebelei], Hofburgtheater. (»Rechte der Seele«, Schauspiel in einem Akte von Guiseppe Giacosa. – »Liebelei«, Schauspiel in drei Akten von Arthur Schnitzler. – Erstaufführung am 9. Oktober 1895.), Illustrirtes Wiener Extrablatt, K. k. Hofburgtheater: »Rechte der Seele«, Schauspiel in einem Acte von Giuseppe Giacosa. – »Liebelei«, Schauspiel in drei Acten von Arthur Schnitzler. Zum ersten Male aufgeführt am 9. October, Liebelei. Schauspiel in drei Akten, Neues Wiener Journal, Neues Wiener Tagblatt, Neuigkeits-Welt-Blatt, Reichspost, Theater und Kunst. (Burgtheater.) [Liebelei, Rechte der Seele], Theater, Kunst und Literatur. Burgtheater [Liebelei, Rechte der Seele], Theater- und Kunstnachrichten. [Burgtheater] [Liebelei, Rechte der Seele], Wiener Abendpost, Wiener Sonn- und Montagszeitung, k. k. Hofburgtheater [Rechte der Seele, Liebelei], Österreichische Volks-Zeitung}
               \section[Paul Goldmann an Arthur Schnitzler, 17. 10. {[}1895{]}]{ Paul Goldmann an Arthur Schnitzler, 17. 10. {[}1895{]}}\nopagebreak\mylabel{v}\rehead{ }\begin{ledgroupsized}[t]{13cm}\normalsize\beginnumbering \toendnotes[C]{\smallbreak\pagebreak[2]} \Standort{DLA, A:Schnitzler, HS.NZ85.1.3165.}
\physDesc{Brief, 2 Blätter, 7 Seiten
\newline{}Handschrift: blaue Tinte, deutsche Kurrent
\newline{}Schnitzler: 1) mit Bleistift eine Unterstreichung, eine seitliche Markierung
                                 und das Jahr »95« vermerkt  2) mit rotem Buntstift acht Unterstreichungen}\toendnotes[C]{\smallbreak}\pstart
           \noindent{}{\pb}\textcolor{gray}{\textbf{\textbf{Frankfurter Zeitung\orgindex{Frankfurter Zeitung@Frankfurter Zeitung|pw}}}}\pend
           \pstart
           \textcolor{gray}{\textbf{(\begin{otherlanguage}{french}Gazette de Francfort\end{otherlanguage}\orgindex{Frankfurter Zeitung@Frankfurter Zeitung|pw}). }}\pend
           \pstart
           \textcolor{gray}{\textbf{\textbf{\begin{otherlanguage}{french}Fondateur M. L.
                              Sonnemann\pwindex{Sonnemann, Leopold 1831-10-29 – 1909-10-30@\textsc{Sonnemann, Leopold} (1831-10-29 – 1909-10-30), \emph{Journalist, Herausgeber}|pw}\end{otherlanguage}.}}}\pend
           \pstart
           \begin{otherlanguage}{french}\textcolor{gray}{\textbf{Journal politique, financier,}}\end{otherlanguage}\hfill \textsc{Paris\oindex{Paris@\textbf{Paris}|pw}}, 17. Oktober.\pend
           \pstart
           \begin{otherlanguage}{french}\textcolor{gray}{\textbf{commercial et littéraire.}}\end{otherlanguage}\pend
           \pstart
           \begin{otherlanguage}{french}\textcolor{gray}{\textbf{\textbf{Paraissant trois fois par jour.}}}\end{otherlanguage}\pend
           \pstart
           \begin{otherlanguage}{french}\textcolor{gray}{\textbf{\textbf{Bureau à Paris\oindex{Paris@\textbf{Paris}|pw}}}}\end{otherlanguage}\pend
           \pstart
           \begin{otherlanguage}{french}\textcolor{gray}{\textbf{\textbf{24. Rue Feydeau\oindex{rue Feydeau@\textbf{rue Feydeau}|pw}.}}}\end{otherlanguage}\pend
           \pstart\center{}Mein lieber Freund,\pend\pstart
           Herzlichſten Dank für die Kritiken! Das iſt gar eine amüſante Lectüre. Wie \strikeout{d} Dein Bild da aus \strikeout{all}
               all’ den Spiegeln der Öffentlichkeit zurückgeworfen wird! Aber manchmal ſieht es mich
               auch fremd an, ſchmerzlich fremd, und meine trüben Ahnungen kommen wieder. Ja, ja,
               laß’ nur! Es iſt Unſinn, ich weiß{\dotsfive}\pend
           \pstart
           Sehr intereſſant, dieſe Lectüre. Über \textsc{Speidel\pwindex{Speidel, Ludwig 1830-04-11 – 1906-02-03@\textsc{Speidel, Ludwig} (1830-04-11 – 1906-02-03), \emph{Journalist, Kritiker}|pw}\pwindex{Theater- und Kunstnachrichten. [Burgtheater] [Liebelei, Rechte der Seele]1895-10-10@\emph{Theater- und Kunstnachrichten. [Burgtheater] [Liebelei, Rechte der Seele]} {[}1895-10-10{]}|pwv}} ſchrieb {\pb}ich Dir ſchon. \label{K_L02756-1v}\edtext{\textsc{Kalbeck\pwindex{Theater, Kunst und Literatur. Burgtheater [Liebelei, Rechte der Seele]1895-10-10@\emph{Theater, Kunst und Literatur. Burgtheater [Liebelei, Rechte der Seele]} {[}1895-10-10{]}|pwv}\pwindex{Kalbeck, Max 1850-01-04 – 1921-05-04@\textsc{Kalbeck, Max} (1850-01-04 – 1921-05-04), \emph{Journalist}!Burgtheater. »Liebelei«, Schauspiel in drei Acten von Arthur Schnitzler. – »Rechte der Seele«, Schauspiel in einem Acte von Guiseppe Giacosa; deutsch von Otto Eisenschitz1895-10-11@\strich\emph{Burgtheater. »Liebelei«, Schauspiel in drei Acten von Arthur Schnitzler. – »Rechte der Seele«, Schauspiel in einem Acte von Guiseppe Giacosa; deutsch von Otto Eisenschitz} {[}1895-10-11{]}|pwv}\pwindex{Kalbeck, Max 1850-01-04 – 1921-05-04@\textsc{Kalbeck, Max} (1850-01-04 – 1921-05-04), \emph{Journalist}|pw}}}{\lemma{\textnormal{\emph{Kalbeck}}}\Cendnote{\textnormal{Nachtkritik\pwindex{Theater, Kunst und Literatur. Burgtheater [Liebelei, Rechte der Seele]1895-10-10@\emph{Theater, Kunst und Literatur. Burgtheater [Liebelei, Rechte der Seele]} {[}1895-10-10{]}|pwkv}: M. K.\pwindex{Kalbeck, Max 1850-01-04 – 1921-05-04@\textsc{Kalbeck, Max} (1850-01-04 – 1921-05-04), \emph{Journalist}|pwk} [=Max Kalbeck\pwindex{Kalbeck, Max 1850-01-04 – 1921-05-04@\textsc{Kalbeck, Max} (1850-01-04 – 1921-05-04), \emph{Journalist}|pwk}]: \emph{Theater,
                        Kunst und Literatur. Burgtheater}\pwindex{Theater, Kunst und Literatur. Burgtheater [Liebelei, Rechte der Seele]1895-10-10@\emph{Theater, Kunst und Literatur. Burgtheater [Liebelei, Rechte der Seele]} {[}1895-10-10{]}|pwk}. In: \emph{Neues Wiener Tagblatt}\pwindex{?? Werk@Nicht ermittelte Verfasserinnen und Verfasser!Neues Wiener Tagblatt1867 – 1945@\emph{Neues Wiener Tagblatt} {[}1867 – 1945{]}|pwk}, Jg. 29, Nr. 278, 10. 10. 1895, S. 7 und Feuilleton\pwindex{Kalbeck, Max 1850-01-04 – 1921-05-04@\textsc{Kalbeck, Max} (1850-01-04 – 1921-05-04), \emph{Journalist}!Burgtheater. »Liebelei«, Schauspiel in drei Acten von Arthur Schnitzler. – »Rechte der Seele«, Schauspiel in einem Acte von Guiseppe Giacosa; deutsch von Otto Eisenschitz1895-10-11@\strich\emph{Burgtheater. »Liebelei«, Schauspiel in drei Acten von Arthur Schnitzler. – »Rechte der Seele«, Schauspiel in einem Acte von Guiseppe Giacosa; deutsch von Otto Eisenschitz} {[}1895-10-11{]}|pwkv}: Max Kalbeck\pwindex{Kalbeck, Max 1850-01-04 – 1921-05-04@\textsc{Kalbeck, Max} (1850-01-04 – 1921-05-04), \emph{Journalist}|pwk}: \emph{Burgtheater. »Liebelei«, Schauspiel in drei Acten von Arthur
                        Schnitzler. – »Rechte der Seele«, Schauspiel in einem Acte von Guiseppe
                        Giacosa; deutsch von Otto Eisenschitz}\pwindex{Kalbeck, Max 1850-01-04 – 1921-05-04@\textsc{Kalbeck, Max} (1850-01-04 – 1921-05-04), \emph{Journalist}!Burgtheater. »Liebelei«, Schauspiel in drei Acten von Arthur Schnitzler. – »Rechte der Seele«, Schauspiel in einem Acte von Guiseppe Giacosa; deutsch von Otto Eisenschitz1895-10-11@\strich\emph{Burgtheater. »Liebelei«, Schauspiel in drei Acten von Arthur Schnitzler. – »Rechte der Seele«, Schauspiel in einem Acte von Guiseppe Giacosa; deutsch von Otto Eisenschitz} {[}1895-10-11{]}|pwk}. In: \emph{Neues Wiener Tagblatt}\pwindex{?? Werk@Nicht ermittelte Verfasserinnen und Verfasser!Neues Wiener Tagblatt1867 – 1945@\emph{Neues Wiener Tagblatt} {[}1867 – 1945{]}|pwk}, Jg. 29, Nr. 279, 11. 10. 1895, S. 1–3.}}}\label{K_L02756-1h} iſt
               unerträglich ſchwülſtig geſchrieben. Gefällt ihm das Stück\pwindex{Schnitzler, Arthur 15.05.1862 – 21.10.1931@\textsc{Schnitzler, Arthur} (15.05.1862 – 21.10.1931), \emph{Schriftsteller, Mediziner}!Liebelei. Schauspiel in drei Akten1895-10-09@\strich\emph{Liebelei. Schauspiel in drei Akten} {[}1895-10-09{]}|pwv} wirklich ſo? Oder hat er nur
               vernommen, daß es \textsc{Speidel\pwindex{Speidel, Ludwig 1830-04-11 – 1906-02-03@\textsc{Speidel, Ludwig} (1830-04-11 – 1906-02-03), \emph{Journalist, Kritiker}|pw}} loben würde und ſich darum beeilt, um die Wette zu loben, – auf Seiten der
               Mächtigen, wie immer? Ich glaube, der iſt kein echter, auf den kannſt Du Dich nicht verlaſſen\substVorne{}\textsuperscript{.}\substDazwischen{},\substHinten{} – wohl aber auf \textsc{Speidel\pwindex{Speidel, Ludwig 1830-04-11 – 1906-02-03@\textsc{Speidel, Ludwig} (1830-04-11 – 1906-02-03), \emph{Journalist, Kritiker}|pw}}. Schön iſt das Wohlwollen u. die Sympathie, die faſt bei \uline{Allen} zutage tritt. Einiges davon iſt wohl auf Rechnung des Wien\oindex{Wien@\textbf{Wien}|pw}eriſchen zu ſetzen, die {\pb}Hauptſache aber kommt aus der Achtung und dem
               Reſpect vor dem \uline{Menſchen}{ }\textsc{Schnitzler}. Durch warmen, \strikeout{\textcolor{gray}{×}u\textcolor{gray}{×}} herzlichen, neidloſen Ton ragt vor Allem \label{K_L02756-2v}\edtext{\uline{\textsc{Hirschfeld\pwindex{Hirschfeld, Robert 17.09.1857 – 02.04.1914@\textsc{Hirschfeld, Robert} (17.09.1857 – 02.04.1914), \emph{Journalist, Musikkritiker}|pw}\pwindex{Burgtheater. (»Liebelei« von Arthur Schnitzler. – »Rechte der Seele« von Giacosa.)1895-10-14@\emph{Burgtheater. (»Liebelei« von Arthur Schnitzler. – »Rechte der Seele« von Giacosa.)} {[}1895-10-14{]}|pwv}}}}{\lemma{\textnormal{\emph{Hirschfeld}}}\Cendnote{\textnormal{L. A. Terne\pwindex{Hirschfeld, Robert 17.09.1857 – 02.04.1914@\textsc{Hirschfeld, Robert} (17.09.1857 – 02.04.1914), \emph{Journalist, Musikkritiker}|pwk} [=Robert Hirschfeld\pwindex{Hirschfeld, Robert 17.09.1857 – 02.04.1914@\textsc{Hirschfeld, Robert} (17.09.1857 – 02.04.1914), \emph{Journalist, Musikkritiker}|pwk}]: \emph{Burgtheater. (»Liebelei« von Arthur Schnitzler. – »Rechte der Seele« von
                        Giacosa.)}\pwindex{Burgtheater. (»Liebelei« von Arthur Schnitzler. – »Rechte der Seele« von Giacosa.)1895-10-14@\emph{Burgtheater. (»Liebelei« von Arthur Schnitzler. – »Rechte der Seele« von Giacosa.)} {[}1895-10-14{]}|pwk} In: \emph{Wiener Sonn- und
                        Montags-Zeitung}\pwindex{?? Werk@Nicht ermittelte Verfasserinnen und Verfasser!Wiener Sonn- und Montagszeitung1863 – 1936@\emph{Wiener Sonn- und Montagszeitung} {[}1863 – 1936{]}|pwk}, Jg. 33, Nr. 41, 14. 10. 1895, S. 1–3.}}}\label{K_L02756-2h} hervor. Das iſt Einer, der ſich
               wirklich mit Deinem Talent und Deinem Erfolge freut. Das Schönſte aber iſt – es iſt
               ſeltſam, daß ich dieſem widerwärtigen Menſchen\pwindex{David, Jakob Julius 1859-02-06 – 1906-11-20@\textsc{David, Jakob Julius} (1859-02-06 – 1906-11-20), \emph{Schriftsteller, Journalist}|pwv} das Zugeſtändniß machen muß – \label{K_L02756-3v}\edtext{\textsc{\uline{J. J. David\pwindex{David, Jakob Julius 1859-02-06 – 1906-11-20@\textsc{David, Jakob Julius} (1859-02-06 – 1906-11-20), \emph{Schriftsteller, Journalist}|pw}s}}{ }Feuilleton\pwindex{Arthur Schnitzler9. 10. 1895@\emph{Arthur Schnitzler} {[}9. 10. 1895{]}|pwv}}{\lemma{\textnormal{\emph{J. J. Davids Feuilleton}}}\Cendnote{\textnormal{–v–\pwindex{David, Jakob Julius 1859-02-06 – 1906-11-20@\textsc{David, Jakob Julius} (1859-02-06 – 1906-11-20), \emph{Schriftsteller, Journalist}|pwk} [=J.
                        J. David\pwindex{David, Jakob Julius 1859-02-06 – 1906-11-20@\textsc{David, Jakob Julius} (1859-02-06 – 1906-11-20), \emph{Schriftsteller, Journalist}|pwk}]: \emph{Arthur Schnitzler}\pwindex{Arthur Schnitzler9. 10. 1895@\emph{Arthur Schnitzler} {[}9. 10. 1895{]}|pwk}. In:
                        \emph{Neues Wiener Journal}\pwindex{Neues Wiener Journal1893 – 1939@\emph{Neues Wiener Journal} {[}1893 – 1939{]}|pwk}, Jg. 3, Nr. 703,
                        9. 10. 1895, S. 1–2. (Am Tag der Uraufführung). Zusätzlich dazu verfasste David\pwindex{David, Jakob Julius 1859-02-06 – 1906-11-20@\textsc{David, Jakob Julius} (1859-02-06 – 1906-11-20), \emph{Schriftsteller, Journalist}|pwk} eine Nachtkritik\pwindex{Theater und Kunst. (Burgtheater.) [Liebelei, Rechte der Seele]1895-10-10@\emph{Theater und Kunst. (Burgtheater.) [Liebelei, Rechte der Seele]} {[}1895-10-10{]}|pwkv}: –v–\pwindex{David, Jakob Julius 1859-02-06 – 1906-11-20@\textsc{David, Jakob Julius} (1859-02-06 – 1906-11-20), \emph{Schriftsteller, Journalist}|pwk} [=J.
                        J. David\pwindex{David, Jakob Julius 1859-02-06 – 1906-11-20@\textsc{David, Jakob Julius} (1859-02-06 – 1906-11-20), \emph{Schriftsteller, Journalist}|pwk}]: \emph{Theater und Kunst.
                        (Burgtheater.)}\pwindex{Theater und Kunst. (Burgtheater.) [Liebelei, Rechte der Seele]1895-10-10@\emph{Theater und Kunst. (Burgtheater.) [Liebelei, Rechte der Seele]} {[}1895-10-10{]}|pwk} In: \emph{Neues Wiener
                        Journal}\pwindex{Neues Wiener Journal1893 – 1939@\emph{Neues Wiener Journal} {[}1893 – 1939{]}|pwk}, Jg. 3, Nr. 704, 10. 10. 1895, S. 5.}}}\label{K_L02756-3h}
               über Dich. Das iſt prächtig geſchrieben, das iſt ein klug und wahr gezeichnetes
               Seelenbild von Dir, und das ſchlägt {\pb}in meinem Innern
               liebe Saiten an, die lange nicht geklungen. Es hat mich tief berührt, und ich will
               dem Manne\pwindex{David, Jakob Julius 1859-02-06 – 1906-11-20@\textsc{David, Jakob Julius} (1859-02-06 – 1906-11-20), \emph{Schriftsteller, Journalist}|pwv} Manches um
               deßwillen verzeihen. \label{K_L02756-886v}\edtext{\textsc{\uline{Bauer\pwindex{Hofburgtheater [Rechte der Seele, Liebelei]1895-10-10@\emph{Hofburgtheater [Rechte der Seele, Liebelei]} {[}1895-10-10{]}|pwv}\pwindex{Bauer, Julius 15.10.1853 – 11.06.1941@\textsc{Bauer, Julius} (15.10.1853 – 11.06.1941), \emph{Schriftsteller, Journalist, Kritiker}|pw}}}}{\lemma{\textnormal{\emph{Bauer}}}\Cendnote{\textnormal{[Julius Bauer\pwindex{Bauer, Julius 15.10.1853 – 11.06.1941@\textsc{Bauer, Julius} (15.10.1853 – 11.06.1941), \emph{Schriftsteller, Journalist, Kritiker}|pwk}]: \emph{Hofburgtheater}\pwindex{Hofburgtheater [Rechte der Seele, Liebelei]1895-10-10@\emph{Hofburgtheater [Rechte der Seele, Liebelei]} {[}1895-10-10{]}|pwk}. In: \emph{Illustriertes Wiener Extrablatt}\pwindex{Illustrirtes Wiener Extrablatt1872 – 1928@\emph{Illustrirtes Wiener Extrablatt} {[}1872 – 1928{]}|pwk}, Jg. 24, Nr. 278, 10. 10. 1895, S. 5.}}}\label{K_L02756-886h} tadelt den Schluß\pwindex{Schnitzler, Arthur 15.05.1862 – 21.10.1931@\textsc{Schnitzler, Arthur} (15.05.1862 – 21.10.1931), \emph{Schriftsteller, Mediziner}!Liebelei. Schauspiel in drei Akten1895-10-09@\strich\emph{Liebelei. Schauspiel in drei Akten} {[}1895-10-09{]}|pwv}, und hat vielleicht
               nicht Unrecht. \label{K_L02756-88v}\edtext{\textsc{\uline{Hevesi\pwindex{Burgtheater. (Herr Mitterwurzer als Koenig Philipp. – »Rechte der Seele«, von Guiseppe Giacosa. – »Liebelei«, von Arthur Schnitzler.)None@\emph{Burgtheater. (Herr Mitterwurzer als König Philipp. – »Rechte der Seele«, von Guiseppe Giacosa. – »Liebelei«, von Arthur Schnitzler.)} {[}None{]}|pwv}\pwindex{Burgtheater. (»Rechte der Seele«, Schauspiel in einem Akt von Giuseppe Giacosa. – »Liebelei«, Schauspiel in drei Aufzuegen von Arthur Schnitzler.)1895-10-11@\emph{Burgtheater. (»Rechte der Seele«, Schauspiel in einem Akt von Giuseppe Giacosa. – »Liebelei«, Schauspiel in drei Aufzügen von Arthur Schnitzler.)} {[}1895-10-11{]}|pwv}}\pwindex{Hevesi, Ludwig 20.12.1843 – 27.02.1910@\textsc{Hevesi, Ludwig} (20.12.1843 – 27.02.1910), \emph{Schriftsteller, Journalist}|pw}}}{\lemma{\textnormal{\emph{Hevesi}}}\Cendnote{\textnormal{L. H–i\pwindex{Hevesi, Ludwig 20.12.1843 – 27.02.1910@\textsc{Hevesi, Ludwig} (20.12.1843 – 27.02.1910), \emph{Schriftsteller, Journalist}|pwk} [=Ludwig Hevesi\pwindex{Hevesi, Ludwig 20.12.1843 – 27.02.1910@\textsc{Hevesi, Ludwig} (20.12.1843 – 27.02.1910), \emph{Schriftsteller, Journalist}|pwk}]: \emph{Burgtheater. (»Rechte der Seele«, Schauspiel in einem Akt von Giuseppe
                        Giacosa. – »Liebelei«, Schauspiel in drei Aufzügen von Arthur
                        Schnitzler.)}\pwindex{Burgtheater. (»Rechte der Seele«, Schauspiel in einem Akt von Giuseppe Giacosa. – »Liebelei«, Schauspiel in drei Aufzuegen von Arthur Schnitzler.)1895-10-11@\emph{Burgtheater. (»Rechte der Seele«, Schauspiel in einem Akt von Giuseppe Giacosa. – »Liebelei«, Schauspiel in drei Aufzügen von Arthur Schnitzler.)} {[}1895-10-11{]}|pwk} In: \emph{Fremden-Blatt}\pwindex{?? Werk@Nicht ermittelte Verfasserinnen und Verfasser!Fremden-Blatt1.7.1847 – 22.3.1919@\emph{Fremden-Blatt} {[}1.7.1847 – 22.3.1919{]}|pwk},
                     Jg. 51, Nr. 279, 11. 10. 1895, S. 13–14.
                  Unter den Zeitungsausschnitten Schnitzler\pwindex{Schnitzler, Arthur 15.05.1862 – 21.10.1931@\textsc{Schnitzler, Arthur} (15.05.1862 – 21.10.1931), \emph{Schriftsteller, Mediziner}|pwk}s
                  findet sich auch eine zweite Fassung, offenbar für eine Zeitung außerhalb Wien\oindex{Wien@\textbf{Wien}|pwk}s verfasst (\emph{Breslauer Zeitung}\pwindex{?? Werk@Nicht ermittelte Verfasserinnen und Verfasser!Breslauer Zeitung1820 – 1933@\emph{Breslauer Zeitung} {[}1820 – 1933{]}|pwk}?): L. H–i\pwindex{Hevesi, Ludwig 20.12.1843 – 27.02.1910@\textsc{Hevesi, Ludwig} (20.12.1843 – 27.02.1910), \emph{Schriftsteller, Journalist}|pwk} [=Ludwig Hevesi\pwindex{Hevesi, Ludwig 20.12.1843 – 27.02.1910@\textsc{Hevesi, Ludwig} (20.12.1843 – 27.02.1910), \emph{Schriftsteller, Journalist}|pwk}]: \emph{Burgtheater. (Herr Mitterwurzer als König Philipp. – »Rechte der Seele«,
                        von Guiseppe Giacosa. – »Liebelei«, von Arthur Schnitzler.)}\pwindex{Burgtheater. (Herr Mitterwurzer als Koenig Philipp. – »Rechte der Seele«, von Guiseppe Giacosa. – »Liebelei«, von Arthur Schnitzler.)None@\emph{Burgtheater. (Herr Mitterwurzer als König Philipp. – »Rechte der Seele«, von Guiseppe Giacosa. – »Liebelei«, von Arthur Schnitzler.)} {[}None{]}|pwk}.}}}\label{K_L02756-88h}{ }\strikeout{m} iſt vortrefflich und geſcheit; beſonders \strikeout{das}, was er über die Paradoxe ſagt, ſind goldene Worte.
                  \label{K_L02756-5v}\edtext{\textsc{\uline{Uhl\pwindex{K. k. Hofburgtheater: »Rechte der Seele«, Schauspiel in einem Acte von Giuseppe Giacosa. – »Liebelei«, Schauspiel in drei Acten von Arthur Schnitzler. Zum ersten Male aufgefuehrt am 9. October1895-10-10@\emph{K. k. Hofburgtheater: »Rechte der Seele«, Schauspiel in einem Acte von Giuseppe Giacosa. – »Liebelei«, Schauspiel in drei Acten von Arthur Schnitzler. Zum ersten Male aufgeführt am 9. October} {[}1895-10-10{]}|pwv}\pwindex{Uhl, Friedrich 14.05.1825 – 20.01.1906@\textsc{Uhl, Friedrich} (14.05.1825 – 20.01.1906), \emph{Journalist}|pw}}}}{\lemma{\textnormal{\emph{Uhl}}}\Cendnote{\textnormal{[Friedrich Uhl\pwindex{Uhl, Friedrich 14.05.1825 – 20.01.1906@\textsc{Uhl, Friedrich} (14.05.1825 – 20.01.1906), \emph{Journalist}|pwk}]: \emph{K. k. Hofburgtheater: »Rechte der Seele«, Schauspiel in
                        einem Acte von Giuseppe Giacosa. – »Liebelei«, Schauspiel in drei Acten von
                        Arthur Schnitzler. Zum ersten Male aufgeführt am 9. October}\pwindex{K. k. Hofburgtheater: »Rechte der Seele«, Schauspiel in einem Acte von Giuseppe Giacosa. – »Liebelei«, Schauspiel in drei Acten von Arthur Schnitzler. Zum ersten Male aufgefuehrt am 9. October1895-10-10@\emph{K. k. Hofburgtheater: »Rechte der Seele«, Schauspiel in einem Acte von Giuseppe Giacosa. – »Liebelei«, Schauspiel in drei Acten von Arthur Schnitzler. Zum ersten Male aufgeführt am 9. October} {[}1895-10-10{]}|pwk}. In: \emph{Wiener Abendpost}\pwindex{?? Werk@Nicht ermittelte Verfasserinnen und Verfasser!Wiener Abendpost1.7.1863 – 31.12.1921@\emph{Wiener Abendpost} {[}1.7.1863 – 31.12.1921{]}|pwk}, Nr. 234, 10. 10. 1895, S. 1–2.}}}\label{K_L02756-5h} iſt merkwürdig
               boshaft, hat \strikeout{ſichtlich} ſichtlich in der Abſicht
               geſchrieben, Dir wehzuthun, packt das Stück\pwindex{Schnitzler, Arthur 15.05.1862 – 21.10.1931@\textsc{Schnitzler, Arthur} (15.05.1862 – 21.10.1931), \emph{Schriftsteller, Mediziner}!Liebelei. Schauspiel in drei Akten1895-10-09@\strich\emph{Liebelei. Schauspiel in drei Akten} {[}1895-10-09{]}|pwv} viel zu ſchwer {\pb}an,
               ſagt aber ſchließlich doch manches Beherzigenswerthe; ſein Tadel gegen die Figur des Vaters\pwindex{Schnitzler, Arthur 15.05.1862 – 21.10.1931@\textsc{Schnitzler, Arthur} (15.05.1862 – 21.10.1931), \emph{Schriftsteller, Mediziner}!Liebelei. Schauspiel in drei Akten1895-10-09@\strich\emph{Liebelei. Schauspiel in drei Akten} {[}1895-10-09{]}|pwv} iſt viel zu
                  \strikeout{heft\textcolor{gray}{i}} heftig ausgedrückt, aber im
               Grunde ſcheint er Recht zu haben. Durch beſondere Dummheit zeichnet ſich \label{K_L02756-65v}\edtext{\textsc{\uline{Bunzl\pwindex{Bunzl, Arthur 25.12.1849 – 26.03.1899@\textsc{Bunzl, Arthur} (25.12.1849 – 26.03.1899), \emph{Journalist}!Burgtheater. »Rechte der Seele«, Schauspiel in einem Akt von Giuseppe Giacosa. Deutsch von Otto Eisenschuetz. – »Liebelei«, Schauspiel in drei Akten von Arthur Schnitzler. Zum erstenmale aufgefuehrt am 9. Oktober1895-10-11@\strich\emph{Burgtheater. »Rechte der Seele«, Schauspiel in einem Akt von Giuseppe Giacosa. Deutsch von Otto Eisenschütz. – »Liebelei«, Schauspiel in drei Akten von Arthur Schnitzler. Zum erstenmale aufgeführt am 9. Oktober} {[}1895-10-11{]}|pwv}\pwindex{Bunzl, Arthur 25.12.1849 – 26.03.1899@\textsc{Bunzl, Arthur} (25.12.1849 – 26.03.1899), \emph{Journalist}|pw}}}}{\lemma{\textnormal{\emph{Bunzl}}}\Cendnote{\textnormal{Arthur Bunzl\pwindex{Bunzl, Arthur 25.12.1849 – 26.03.1899@\textsc{Bunzl, Arthur} (25.12.1849 – 26.03.1899), \emph{Journalist}|pwk}: \emph{Burgtheater. »Rechte der Seele«, Schauspiel in einem Akt von
                        Giuseppe Giacosa. Deutsch von Otto Eisenschütz. – »Liebelei«, Schauspiel in
                        drei Akten von Arthur Schnitzler. Zum erstenmale aufgeführt am 9. Oktober}\pwindex{Bunzl, Arthur 25.12.1849 – 26.03.1899@\textsc{Bunzl, Arthur} (25.12.1849 – 26.03.1899), \emph{Journalist}!Burgtheater. »Rechte der Seele«, Schauspiel in einem Akt von Giuseppe Giacosa. Deutsch von Otto Eisenschuetz. – »Liebelei«, Schauspiel in drei Akten von Arthur Schnitzler. Zum erstenmale aufgefuehrt am 9. Oktober1895-10-11@\strich\emph{Burgtheater. »Rechte der Seele«, Schauspiel in einem Akt von Giuseppe Giacosa. Deutsch von Otto Eisenschütz. – »Liebelei«, Schauspiel in drei Akten von Arthur Schnitzler. Zum erstenmale aufgeführt am 9. Oktober} {[}1895-10-11{]}|pwk}. In: \emph{Österreichische
                        Volks-Zeitung}\pwindex{?? Werk@Nicht ermittelte Verfasserinnen und Verfasser!Oesterreichische Volks-Zeitung1888 – 13.11.1918@\emph{Österreichische Volks-Zeitung} {[}1888 – 13.11.1918{]}|pwk}, Jg. 41, Nr. 279, 11. 10. 1895, S. 1–2.}}}\label{K_L02756-65h} aus; er war aber immer ein Ochs.
               Köſtlich iſt die künſtleriſche Strenge des »\label{K_L02756-22v}\edtext{Neuigkeits-Weltblatts\pwindex{?? Werk@Nicht ermittelte Verfasserinnen und Verfasser!Neuigkeits-Welt-Blatt1874-01-06 – 1918-12-31@\emph{Neuigkeits-Welt-Blatt} {[}1874-01-06 – 1918-12-31{]}|pw}\pwindex{Hofburgtheater. (»Rechte der Seele«, Schauspiel in einem Akte von Guiseppe Giacosa. – »Liebelei«, Schauspiel in drei Akten von Arthur Schnitzler. – Erstauffuehrung am 9. Oktober 1895.)1895-10-12@\emph{Hofburgtheater. (»Rechte der Seele«, Schauspiel in einem Akte von Guiseppe Giacosa. – »Liebelei«, Schauspiel in drei Akten von Arthur Schnitzler. – Erstaufführung am 9. Oktober 1895.)} {[}1895-10-12{]}|pwv}}{\lemma{\textnormal{\emph{Neuigkeits-Weltblatts}}}\Cendnote{\textnormal{Alpha\pwindex{Alpha @\textsc{Alpha}, \emph{Journalist/Journalistin}|pwk}: \emph{Hofburgtheater. (»Rechte der Seele«, Schauspiel in einem Akte von Guiseppe
                        Giacosa. – »Liebelei«, Schauspiel in drei Akten von Arthur Schnitzler. –
                        Erstaufführung am 9. Oktober 1895.)}\pwindex{Hofburgtheater. (»Rechte der Seele«, Schauspiel in einem Akte von Guiseppe Giacosa. – »Liebelei«, Schauspiel in drei Akten von Arthur Schnitzler. – Erstauffuehrung am 9. Oktober 1895.)1895-10-12@\emph{Hofburgtheater. (»Rechte der Seele«, Schauspiel in einem Akte von Guiseppe Giacosa. – »Liebelei«, Schauspiel in drei Akten von Arthur Schnitzler. – Erstaufführung am 9. Oktober 1895.)} {[}1895-10-12{]}|pwk} In: \emph{Neuigkeits-Welt-Blatt}\pwindex{?? Werk@Nicht ermittelte Verfasserinnen und Verfasser!Neuigkeits-Welt-Blatt1874-01-06 – 1918-12-31@\emph{Neuigkeits-Welt-Blatt} {[}1874-01-06 – 1918-12-31{]}|pwk}, Jg. 22, Nr. 235, 12. 10. 1895, S. 10.}}}\label{K_L02756-22h}«. Hübſch ſind auch die \label{K_L02756-999v}\edtext{\uline{Socialiſten}\pwindex{Burgtheater. [Rechte der Seele, Liebelei]1895-10-11@\emph{Burgtheater. [Rechte der Seele, Liebelei]} {[}1895-10-11{]}|pwv}\pwindex{Wengraf, Edmund 09.01.1860 – 08.12.1933@\textsc{Wengraf, Edmund} (09.01.1860 – 08.12.1933), \emph{Journalist}|pwv}}{\lemma{\textnormal{\emph{Socialiſten}}}\Cendnote{\textnormal{e. w.\pwindex{Wengraf, Edmund 09.01.1860 – 08.12.1933@\textsc{Wengraf, Edmund} (09.01.1860 – 08.12.1933), \emph{Journalist}|pwk} [=Edmund Wengraf\pwindex{Wengraf, Edmund 09.01.1860 – 08.12.1933@\textsc{Wengraf, Edmund} (09.01.1860 – 08.12.1933), \emph{Journalist}|pwk}]: \emph{Burgtheater}\pwindex{Burgtheater. [Rechte der Seele, Liebelei]1895-10-11@\emph{Burgtheater. [Rechte der Seele, Liebelei]} {[}1895-10-11{]}|pwk}. In: \emph{Arbeiter-Zeitung}\pwindex{Arbeiter-Zeitung12.7.1881 – 31.10.1991@\emph{Arbeiter-Zeitung} {[}12.7.1881 – 31.10.1991{]}|pwk}, Jg. 7, Nr. 279, 11. 10. 1895, Morgenblatt, S. 5.}}}\label{K_L02756-999h}, welche unzufrieden
               ſind, {\pb}weil das Stück\pwindex{Schnitzler, Arthur 15.05.1862 – 21.10.1931@\textsc{Schnitzler, Arthur} (15.05.1862 – 21.10.1931), \emph{Schriftsteller, Mediziner}!Liebelei. Schauspiel in drei Akten1895-10-09@\strich\emph{Liebelei. Schauspiel in drei Akten} {[}1895-10-09{]}|pwv} nicht nach Dreck ſtinkt: »\label{K_L02756-444v}\edtext{Das iſt nicht das wahre Volk\pwindex{Burgtheater. [Rechte der Seele, Liebelei]1895-10-11@\emph{Burgtheater. [Rechte der Seele, Liebelei]} {[}1895-10-11{]}|pwv}}{\lemma{\textnormal{\emph{Das … Volk}}}\Cendnote{\textnormal{Paraphrase, kein direktes
               Zitat}}}\label{K_L02756-444h}«. Daß ſelbſt die \uline{Antiſemiten}\pwindex{r. p. @\textsc{r. p.}, \emph{Journalist/Journalistin}|pwv} über Dich ſympathiſch ſchreiben (»\label{K_L02756-111v}\edtext{Reichspoſt\pwindex{Reichspost1893 – 1938@\emph{Reichspost} {[}1893 – 1938{]}|pw}\pwindex{k. k. Hofburgtheater [Rechte der Seele, Liebelei]1895-10-12@\emph{k. k. Hofburgtheater [Rechte der Seele, Liebelei]} {[}1895-10-12{]}|pwv}}{\lemma{\textnormal{\emph{Reichspoſt}}}\Cendnote{\textnormal{r. p.\pwindex{r. p. @\textsc{r. p.}, \emph{Journalist/Journalistin}|pwk}: \emph{k. k. Hofburgtheater}\pwindex{k. k. Hofburgtheater [Rechte der Seele, Liebelei]1895-10-12@\emph{k. k. Hofburgtheater [Rechte der Seele, Liebelei]} {[}1895-10-12{]}|pwk}. In: \emph{Reichspost}\pwindex{Reichspost1893 – 1938@\emph{Reichspost} {[}1893 – 1938{]}|pwk}, Jg. 2, Nr. 235, 12. 10. 1895, S. 1.}}}\label{K_L02756-111h}{[}«{]}), iſt ein
               wahrer Triumph für Dich und beweiſt abermals, daß der Antiſemitismus ſich nur gegen
               die widerlichen Saujuden richtet und vor dem ehrenhaften und tüchtigen Juden
               entwaffnen muß. \label{K_L02756-87v}\edtext{\textsc{\uline{Granichstaedten\pwindex{Granichstaedten, Emil 1847-07-08 – 1904-07-02@\textsc{Granichstaedten, Emil} (1847-07-08 – 1904-07-02), \emph{Journalist, Rechtswissenschaftler}!Burgtheater. Zwei Schauspiele: »Rechte der Seele« von Giuseppe Giacosa. – »Liebelei« von Arthur Schnitzler1895-10-11@\strich\emph{Burgtheater. Zwei Schauspiele: »Rechte der Seele« von Giuseppe Giacosa. – »Liebelei« von Arthur Schnitzler} {[}1895-10-11{]}|pwv}\pwindex{Granichstaedten, Emil 1847-07-08 – 1904-07-02@\textsc{Granichstaedten, Emil} (1847-07-08 – 1904-07-02), \emph{Journalist, Rechtswissenschaftler}|pw}}}}{\lemma{\textnormal{\emph{Granichstaedten}}}\Cendnote{\textnormal{Emil Granichstaedten\pwindex{Granichstaedten, Emil 1847-07-08 – 1904-07-02@\textsc{Granichstaedten, Emil} (1847-07-08 – 1904-07-02), \emph{Journalist, Rechtswissenschaftler}|pwk}: \emph{Burgtheater. Zwei Schauspiele: »Rechte der Seele« von
                        Giuseppe Giacosa. – »Liebelei« von Arthur Schnitzler}\pwindex{Granichstaedten, Emil 1847-07-08 – 1904-07-02@\textsc{Granichstaedten, Emil} (1847-07-08 – 1904-07-02), \emph{Journalist, Rechtswissenschaftler}!Burgtheater. Zwei Schauspiele: »Rechte der Seele« von Giuseppe Giacosa. – »Liebelei« von Arthur Schnitzler1895-10-11@\strich\emph{Burgtheater. Zwei Schauspiele: »Rechte der Seele« von Giuseppe Giacosa. – »Liebelei« von Arthur Schnitzler} {[}1895-10-11{]}|pwk}. In: \emph{Die Presse}\pwindex{?? Werk@Nicht ermittelte Verfasserinnen und Verfasser!Presse1848-07-03@\emph{Die Presse} {[}1848-07-03{]}|pwk}, Jg. 48, Nr. 279, 11. 10. 1895, S. 1–2.}}}\label{K_L02756-87h} iſt ſo
               ungeſchickt und offen gemein, daß es {\pb}nicht einmal
               empört; jede Zeile ſagt ſelbſt dem \strikeout{\textcolor{gray}{enh}} nichteingeweihten Leſer im Vertrauen, daß der Verfaſſer\pwindex{Granichstaedten, Emil 1847-07-08 – 1904-07-02@\textsc{Granichstaedten, Emil} (1847-07-08 – 1904-07-02), \emph{Journalist, Rechtswissenschaftler}|pwv} lügt{\dotsfive}\pend
           \pstart
           Das Geſammtbild iſt glänzend; und der Erfolg iſt ſo groß, wie ich ihn nur irgend für
               Dich wünſchen konnte. Jetzt mach’ Dich bald und frohen Muthes an die neue Arbeit!\pend
           \pstart
           Viele treue Grüße! {\\[\baselineskip]}Dein {\\[\baselineskip]}\spacefill\mbox{Paul Goldmnn.}\pend
           \leftskip=0em{}
         
         \endnumbering\mylabel{h}\end{ledgroupsized}  \newcommand{\dateiname}{L02756}\newcommand{\titel}{Paul Goldmann an Arthur Schnitzler, 17. 10. [1895]}\newcommand{\editorInnen}{Martin Anton Müller und Laura Untner}%% latex-leseansicht-abspann.tex
%% Abspann für die Leseansicht.
%% Der Schalter \ifkorrekturansicht ist bereits durch den Vorspann gesetzt.

%% latex-abspann.tex
%% Gemeinsamer Abspann für Korrekturansicht und Leseansicht.
%% Setzt den Schalter \ifkorrekturansicht voraus (gesetzt in den
%% einbindenden Dateien latex-korrekturansicht-abspann.tex bzw.
%% latex-leseansicht-abspann.tex).
%% ---------------------------------------------------------------

\normalsize

% Das esempio-Environment wird nur in der Leseansicht benötigt
\ifkorrekturansicht\else
\newenvironment{esempio}[3]%
{
    \vspace{1.5ex}
    \rlap{\underline{#1}}
    \par
    \setlength{\parindent}{0cm}
    \nopagebreak
    \leftskip=#2cm
    \rightskip=#3cm
}
{
    \par
}
\fi

\doendnotes{C}
\bigskip
\vfill

\clearpage

\footnotesize

\ifkorrekturansicht
  \lohead{\textsc{register}}
\fi

% theindex-Environment neu definieren ohne reledmac
\makeatletter
\renewenvironment{theindex}{%
  \ifkorrekturansicht
    \section*{\indexname}%
  \else
    \subsubsection*{Index der erwähnten Entitäten}%
  \fi
  \setlength{\parindent}{0pt}%
  \setlength{\parskip}{0pt plus 0.3pt}%
  \let\item\@idxitem
}{%
  \ifkorrekturansicht\clearpage\fi
}
\makeatother

\IfFileExists{\jobname-pw.ind}{\input{\jobname-pw.ind}}{}

% Quellenangabe nur in der Leseansicht
\ifkorrekturansicht\else
% Fallback-Definitionen, falls die .tex-Datei \titel etc. nicht gesetzt hat
\providecommand{\titel}{}
\providecommand{\editorInnen}{}
\providecommand{\dateiname}{\jobname}

\vspace{3cm}

\vfill

\footnotesize
\textsc{Quelle}: \titel. Herausgegeben von {\editorInnen}. In: \emph{Arthur Schnitzler: Briefwechsel mit Autorinnen und Autoren}.
 Digitale Edition, https://schnitzler-briefe.acdh.oeaw.ac.at/{\dateiname}.html (Stand \today)
\fi

\end{document}


      