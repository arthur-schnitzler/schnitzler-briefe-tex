%% latex-leseansicht-vorspann.tex
%% Vorspann für die Leseansicht.
%% Lädt die gemeinsame Datei latex-vorspann.tex mit nicht gesetztem Schalter.

\newif\ifkorrekturansicht
\korrekturansichtfalse

\input{../tex-inputs/latex-vorspann}


\section[Arthur Schnitzler an Hugo von Hofmannsthal, 14. 7. 1892]{L00104 Arthur Schnitzler an Hugo von Hofmannsthal, 14. 7. 1892}
\nopagebreak\mylabel{L00104v}
\rehead{ }\normalsize\beginnumbering\briefempfaengerindex{Hofmannsthal, Hugo von@\textsc{Hofmannsthal, Hugo von}!zzzSchnitzler, Arthur@\emph{von Arthur Schnitzler}!1892-07-141@{14. 7. 1892}|(be}
\toendnotes[C]{\smallbreak\pagebreak[2]}
\correspDesc{Versand  durch Arthur Schnitzler am 14. 7. 1892 in Wien
\newline{}Erhalt  durch Hugo von Hofmannsthal im Zeitraum [14. 7. 1892
                  – 18. 7. 1892?] in Wien}\toendnotes[C]{\smallbreak}
\Standort{FDH, Hs-30885,21.}
\physDesc{Brief, 1 Blatt, 4 Seiten, 1254 Zeichen
\newline{}Handschrift: schwarze Tinte, deutsche Kurrent
\newline{}Ordnung: mit Bleistift von Schnitzler auf der ersten Seite mutmaßlich bei der
                                 Durchsicht der Korrespondenz 1929 
                                 datiert »14. 7. 92« }
\buchAbdrucke{\weitereDrucke{Hugo von Hofmannsthal, Arthur Schnitzler: \emph{Briefwechsel}. Herausgegeben von Therese Nickl und Heinrich Schnitzler. Frankfurt am Main: \emph{S. Fischer} 1964, S. 22.} }\toendnotes[C]{\smallbreak}
\pstart{}{\pb}Lieber Hugo,\pend\vspace{0.5em}
\pstart
           von \textsc{Salten}\pwindex{Salten, Felix 6.\,9.\,1869 Budapest – 8.\,10.\,1945 Zürich@\textsc{Salten, Felix} (6.\,9.\,1869 Budapest – 8.\,10.\,1945 Zürich), \emph{Schriftsteller, Journalist, Chefredakteur}|pw} erfahre ich, daſs Ihr Vater\pwindex{Hofmannsthal, Hugo August von 21.\,12.\,1841 Wien – 8.\,12.\,1915 ebd.@\textsc{Hofmannsthal, Hugo August von} (21.\,12.\,1841 Wien – 8.\,12.\,1915 ebd.), \emph{Bankdirektor}|pwv} krank war, aber bereits wiederhergeſtellt iſt. Hoffentlich erholen Sie{ }ſich zugleich von Ihrer Verſti{\geminationm}ung und Abſpa{\geminationn}ung und verbringen den ko{\geminationm}enden So{\geminationm}er und Herbſt in{ }ſo reicher Fülle des I{\geminationn}ern und Äußern, wie ichs Ihnen von Herzen
               wünſche. –\pend
           
\pstart
           Geſtern{ }ſtarb mein Großvater\pwindex{Markbreiter, Philipp 1.\,1.\,1811 Rajka – 13.\,7.\,1892 Wien@\textsc{Markbreiter, Philipp} (1.\,1.\,1811 Rajka – 13.\,7.\,1892 Wien), \emph{Mediziner}|pwv};
                  {\pb}in wenigen Tagen reiſen meine Eltern\pwindex{Schnitzler, Johann 10.\,4.\,1835 Nagykanizsa – 2.\,5.\,1893 Wien@\textsc{Schnitzler, Johann} (10.\,4.\,1835 Nagykanizsa – 2.\,5.\,1893 Wien), \emph{Laryngologe}|pwv}\pwindex{Schnitzler, Louise 8.\,7.\,1840 Kőszeg – 9.\,9.\,1911 Wien@\textsc{Schnitzler, Louise} (8.\,7.\,1840 Kőszeg – 9.\,9.\,1911 Wien)|pwv} ab, und ich übernehme die
               Praxis meines Papa\pwindex{Schnitzler, Johann 10.\,4.\,1835 Nagykanizsa – 2.\,5.\,1893 Wien@\textsc{Schnitzler, Johann} (10.\,4.\,1835 Nagykanizsa – 2.\,5.\,1893 Wien), \emph{Laryngologe}|pwv}.\pend
           
\pstart
           Seit einiger Zeit bring ich es zuwege, auch nachts literariſch zu arbeiten, und ich
               hoffe, meine angefangenen Sachen werden trotz anderweitiger Thätigkeit wohl
               fortſchreiten können.\pend
           
\pstart
           – Hebbels\pwindex{Hebbel, Friedrich 18.\,3.\,1813 Wesselburen – 13.\,12.\,1863 Wien@\textsc{Hebbel, Friedrich} (18.\,3.\,1813 Wesselburen – 13.\,12.\,1863 Wien), \emph{Schriftsteller}|pw}{ }Briefe\pwindex{Hebbel, Friedrich 18.\,3.\,1813 Wesselburen – 13.\,12.\,1863 Wien@\textsc{Hebbel, Friedrich} (18.\,3.\,1813 Wesselburen – 13.\,12.\,1863 Wien), \emph{Schriftsteller}!Briefwechsel mit Freunden und berühmten Zeitgenossen@\strich\emph{Briefwechsel mit Freunden und berühmten Zeitgenossen}|pw} leſe ich jetzt, Leſſing\pwindex{Lessing, Gotthold Ephraim 22.\,1.\,1729 Kamenz – 15.\,2.\,1781 Braunschweig@\textsc{Lessing, Gotthold Ephraim} (22.\,1.\,1729 Kamenz – 15.\,2.\,1781 Braunschweig), \emph{Schriftsteller, Kritiker, Philosoph}|pw}’s Leben\pwindex{Lessing, Karl Gotthelf 10.\,7.\,1740 Kamenz – 17.\,2.\,1812 Breslau@\textsc{Lessing, Karl Gotthelf} (10.\,7.\,1740 Kamenz – 17.\,2.\,1812 Breslau), \emph{Schriftsteller, Münzmeister}!G. E. Lessings Leben@\strich\emph{G. E. Lessings Leben}|pwv} von{ }ſeinem Bruder\pwindex{Lessing, Karl Gotthelf 10.\,7.\,1740 Kamenz – 17.\,2.\,1812 Breslau@\textsc{Lessing, Karl Gotthelf} (10.\,7.\,1740 Kamenz – 17.\,2.\,1812 Breslau), \emph{Schriftsteller, Münzmeister}|pwv} geſchildert, Annalen\pwindex{Goethe, Johann Wolfgang von 28.\,8.\,1749 Frankfurt am Main – 22.\,3.\,1832 Weimar@\textsc{Goethe, Johann Wolfgang von} (28.\,8.\,1749 Frankfurt am Main – 22.\,3.\,1832 Weimar), \emph{Schriftsteller}!Tag- und Jahreshefte@\strich\emph{Tag- und Jahreshefte}|pw} von Goethe\pwindex{Goethe, Johann Wolfgang von 28.\,8.\,1749 Frankfurt am Main – 22.\,3.\,1832 Weimar@\textsc{Goethe, Johann Wolfgang von} (28.\,8.\,1749 Frankfurt am Main – 22.\,3.\,1832 Weimar), \emph{Schriftsteller}|pw}. {\pb}Hebbel\pwindex{Hebbel, Friedrich 18.\,3.\,1813 Wesselburen – 13.\,12.\,1863 Wien@\textsc{Hebbel, Friedrich} (18.\,3.\,1813 Wesselburen – 13.\,12.\,1863 Wien), \emph{Schriftsteller}|pw} war wohl nach Goethe\pwindex{Goethe, Johann Wolfgang von 28.\,8.\,1749 Frankfurt am Main – 22.\,3.\,1832 Weimar@\textsc{Goethe, Johann Wolfgang von} (28.\,8.\,1749 Frankfurt am Main – 22.\,3.\,1832 Weimar), \emph{Schriftsteller}|pw} der größte Geiſt, den die Deutſchen in dem Jahrhundert
               gehabt haben; manchmal ko{\geminationm}t mir vor, daſs man ihn vor
                  \textsc{Nietz}ſche\pwindex{Nietzsche, Friedrich 15.\,10.\,1844 Röcken – 25.\,8.\,1900 Weimar@\textsc{Nietzsche, Friedrich} (15.\,10.\,1844 Röcken – 25.\,8.\,1900 Weimar), \emph{Schriftsteller, Philosoph}|pw} wird ne{\geminationn}en
               müſſen. Ich bin jetzt bei der Periode{ }ſeines Lebens, wo er auf der Verlegerſuche iſt
               und auf Gutzkow\pwindex{Gutzkow, Karl 17.\,3.\,1811 Berlin – 16.\,12.\,1878 Frankfurt am Main@\textsc{Gutzkow, Karl} (17.\,3.\,1811 Berlin – 16.\,12.\,1878 Frankfurt am Main), \emph{Schriftsteller}|pw}, Laube\pwindex{Laube, Heinrich 18.\,9.\,1806 Sprottau – 1.\,8.\,1884 Wien@\textsc{Laube, Heinrich} (18.\,9.\,1806 Sprottau – 1.\,8.\,1884 Wien), \emph{Schriftsteller, Theaterleiter}|pw}, Mundt\pwindex{Mundt, Theodor 19.\,9.\,1808 Potsdam – 30.\,11.\,1861 Berlin@\textsc{Mundt, Theodor} (19.\,9.\,1808 Potsdam – 30.\,11.\,1861 Berlin), \emph{Schriftsteller}|pw}, Körner\pwindex{Körner, Christian Gottfried 2.\,7.\,1756 Leipzig – 13.\,5.\,1831 Berlin@\textsc{Körner, Christian Gottfried} (2.\,7.\,1756 Leipzig – 13.\,5.\,1831 Berlin), \emph{Schriftsteller}|pw}, zuweilen wohl auch auf Schiller\pwindex{Schiller, Friedrich von 10.\,11.\,1759 Marbach am Neckar – 9.\,5.\,1805 Weimar@\textsc{Schiller, Friedrich von} (10.\,11.\,1759 Marbach am Neckar – 9.\,5.\,1805 Weimar), \emph{Schriftsteller, Historiker, Philosoph}|pw}{ }ſchimpft. Er hat aber auch noch manches andre zu{ }ſagen. – Wiſſen Sie, daſs er eine {\pb}Jungfrau von Orleans\pwindex{Arc, Jeanne d’ um 1412 Domrémy-la-Pucelle – 30.\,5.\,1431 Rouen@\textsc{Arc, Jeanne d’} (um 1412 Domrémy-la-Pucelle – 30.\,5.\,1431 Rouen), \emph{Nationalheilige, Militärin}|pwv}{ }ſchreiben wollte? –\pend
           
\pstart
           Von Richard\pwindex{Beer-Hofmann, Richard 11.\,7.\,1866 Wien – 26.\,9.\,1945 New York City@\textsc{Beer-Hofmann, Richard} (11.\,7.\,1866 Wien – 26.\,9.\,1945 New York City), \emph{Schriftsteller}|pwv} hör ich nichts.
               Sie? –\pend
           
\pstart
           Von Ihnen hoffe ich bald{ }ſchönes und gutes zu erfahren; empfehlen Sie mich bitte
               den Ihren aufs wärmſte.{\\[\baselineskip]}Ihr\hspace*{3.5em}\spacefill\mbox{Arthur}\pend
           \leftskip=0em{}
\pstart
           14. 7. 92.\pend
           
\pstart
           Wien\oindex{Wien@\textbf{Wien}, \emph{Verwaltungsgebiet}|pw}.\pend
           \selectlanguage{ngerman}\endnumbering\briefempfaengerindex{Hofmannsthal, Hugo von@\textsc{Hofmannsthal, Hugo von}!zzzSchnitzler, Arthur@\emph{von Arthur Schnitzler}!1892-07-141@{14. 7. 1892}|)be}\mylabel{L00104h}  \newcommand{\dateiname}{L00104}\newcommand{\titel}{Arthur Schnitzler an Hugo von Hofmannsthal, 14. 7. 1892}\newcommand{\editorInnen}{Martin Anton Müller und Gerd-Hermann Susen}%% latex-leseansicht-abspann.tex
%% Abspann für die Leseansicht.
%% Der Schalter \ifkorrekturansicht ist bereits durch den Vorspann gesetzt.

%% latex-abspann.tex
%% Gemeinsamer Abspann für Korrekturansicht und Leseansicht.
%% Setzt den Schalter \ifkorrekturansicht voraus (gesetzt in den
%% einbindenden Dateien latex-korrekturansicht-abspann.tex bzw.
%% latex-leseansicht-abspann.tex).
%% ---------------------------------------------------------------

\normalsize

% Das esempio-Environment wird nur in der Leseansicht benötigt
\ifkorrekturansicht\else
\newenvironment{esempio}[3]%
{
    \vspace{1.5ex}
    \rlap{\underline{#1}}
    \par
    \setlength{\parindent}{0cm}
    \nopagebreak
    \leftskip=#2cm
    \rightskip=#3cm
}
{
    \par
}
\fi

\doendnotes{C}
\bigskip
\vfill

\clearpage

\footnotesize

\ifkorrekturansicht
  \lohead{\textsc{register}}
\fi

% theindex-Environment neu definieren ohne reledmac
\makeatletter
\renewenvironment{theindex}{%
  \ifkorrekturansicht
    \section*{\indexname}%
  \else
    \subsubsection*{Index der erwähnten Entitäten}%
  \fi
  \setlength{\parindent}{0pt}%
  \setlength{\parskip}{0pt plus 0.3pt}%
  \let\item\@idxitem
}{%
  \ifkorrekturansicht\clearpage\fi
}
\makeatother

\IfFileExists{\jobname-pw.ind}{\input{\jobname-pw.ind}}{}

% Quellenangabe nur in der Leseansicht
\ifkorrekturansicht\else
% Fallback-Definitionen, falls die .tex-Datei \titel etc. nicht gesetzt hat
\providecommand{\titel}{}
\providecommand{\editorInnen}{}
\providecommand{\dateiname}{\jobname}

\vspace{3cm}

\vfill

\footnotesize
\textsc{Quelle}: \titel. Herausgegeben von {\editorInnen}. In: \emph{Arthur Schnitzler: Briefwechsel mit Autorinnen und Autoren}.
 Digitale Edition, https://schnitzler-briefe.acdh.oeaw.ac.at/{\dateiname}.html (Stand \today)
\fi

\end{document}


