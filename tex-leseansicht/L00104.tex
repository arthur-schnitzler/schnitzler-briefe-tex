%% latex-leseansicht-vorspann.tex
%% Vorspann für die Leseansicht.
%% Lädt die gemeinsame Datei latex-vorspann.tex mit nicht gesetztem Schalter.

\newif\ifkorrekturansicht
\korrekturansichtfalse

\input{../tex-inputs/latex-vorspann}


               \section[Arthur Schnitzler an Hugo von Hofmannsthal, 14. 7. 1892]{ Arthur Schnitzler an Hugo von Hofmannsthal, 14. 7. 1892}\nopagebreak\mylabel{v}\rehead{ }\begin{ledgroupsized}[t]{13cm}\normalsize\beginnumbering\briefempfaengerindex{Hofmannsthal, Hugo von@\textsc{Hofmannsthal, Hugo von}!zzzSchnitzler, Arthur@\emph{von Arthur Schnitzler}!1892-07-141@{14. 7. 1892}|(be} \toendnotes[C]{\smallbreak\pagebreak[2]} \Standort{FDH, Hs-30885,21.}
\physDesc{Brief, 1 Blatt, 4 Seiten
\newline{}Handschrift: schwarze Tinte, deutsche Kurrent\newline{}Ordnung: von Schnitzler auf der
                                  ersten Seite mutmaßlich bei der Durchsicht der Korrespondenz 1929 mit Bleistift datiert »14. 7. 92« }\buchAbdrucke{\weitereDrucke{Hugo von Hofmannsthal, Arthur Schnitzler: \emph{Briefwechsel}. Hg. Therese Nickl und Heinrich Schnitzler. Frankfurt am Main: \emph{S. Fischer} 1964, S. 22.} }\toendnotes[C]{\smallbreak}\pstart{}{\pb}Lieber Hugo,\pend\pstart
           von \textsc{Salten}\pwindex{Salten, Felix 06.09.1869 – 08.10.1945@\textsc{Salten, Felix} (06.09.1869 – 08.10.1945), \emph{Schriftsteller, Journalist}|pw} erfahre
                    ich, daſs Ihr Vater\pwindex{Hofmannsthal, Hugo August von 21.12.1841 – 08.12.1915@\textsc{Hofmannsthal, Hugo August von} (21.12.1841 – 08.12.1915), \emph{Bankdirektor}|pwv} krank war, aber
                    bereits wiederhergeſtellt iſt. Hoffentlich erholen Sie ſich zugleich von Ihrer
                        Verſti{\geminationm}ung und Abſpa{\geminationn}ung und verbringen den ko{\geminationm}enden So{\geminationm}er und
                    Herbſt in ſo reicher Fülle des I{\geminationn}ern und Äußern,
                    wie ichs Ihnen von Herzen wünſche. –\pend
           \pstart
           Geſtern ſtarb mein Großvater\pwindex{Markbreiter, Phillip 01.01.1811 – 13.07.1892@\textsc{Markbreiter, Phillip} (01.01.1811 – 13.07.1892), \emph{Mediziner}|pwv}; {\pb}in wenigen Tagen reiſen meine Eltern\pwindex{Schnitzler, Johann 10.04.1835 – 02.05.1893@\textsc{Schnitzler, Johann} (10.04.1835 – 02.05.1893), \emph{Laryngologe}|pwv}\pwindex{Schnitzler, Louise 08.07.1840 – 09.09.1911@\textsc{Schnitzler, Louise} (08.07.1840 – 09.09.1911)|pwv} ab, und ich übernehme die Praxis meines
                        Papa\pwindex{Schnitzler, Johann 10.04.1835 – 02.05.1893@\textsc{Schnitzler, Johann} (10.04.1835 – 02.05.1893), \emph{Laryngologe}|pwv}.\pend
           \pstart
           Seit einiger Zeit bring ich es zuwege, auch nachts literariſch zu arbeiten, und
                    ich hoffe, meine angefangenen Sachen werden trotz anderweitiger Thätigkeit wohl
                    fortſchreiten können.\pend
           \pstart
           – Hebbel\pwindex{Hebbel, Friedrich 18.03.1813 – 13.12.1863@\textsc{Hebbel, Friedrich} (18.03.1813 – 13.12.1863), \emph{Schriftsteller}|pw}s Briefe\pwindex{Hebbel, Friedrich 18.03.1813 – 13.12.1863@\textsc{Hebbel, Friedrich} (18.03.1813 – 13.12.1863), \emph{Schriftsteller}!Briefwechsel mit Freunden und beruehmten Zeitgenossen1890@\strich\emph{Briefwechsel mit Freunden und berühmten Zeitgenossen} {[}1890{]}|pw} leſe ich jetzt, Leſſing\pwindex{Lessing, Gotthold Ephraim 22.01.1729 – 15.02.1781@\textsc{Lessing, Gotthold Ephraim} (22.01.1729 – 15.02.1781), \emph{Schriftsteller, Bibliothekar}|pw}’s Leben\pwindex{Lessing, Karl Gotthelf 10.07.1740 – 17.02.1812@\textsc{Lessing, Karl Gotthelf} (10.07.1740 – 17.02.1812), \emph{Schriftsteller, Münzmeister}!G. E. Lessings Leben1793 – 1795@\strich\emph{G. E. Lessings Leben} {[}1793 – 1795{]}|pwv} von ſeinem
                        Bruder\pwindex{Lessing, Karl Gotthelf 10.07.1740 – 17.02.1812@\textsc{Lessing, Karl Gotthelf} (10.07.1740 – 17.02.1812), \emph{Schriftsteller, Münzmeister}|pwv} geſchildert, Annalen\pwindex{Goethe, Johann Wolfgang von 28.08.1749 – 22.03.1832@\textsc{Goethe, Johann Wolfgang von} (28.08.1749 – 22.03.1832), \emph{Schriftsteller}!Tag- und Jahreshefte1830@\strich\emph{Tag- und Jahreshefte} {[}1830{]}|pw} von Goethe\pwindex{Goethe, Johann Wolfgang von 28.08.1749 – 22.03.1832@\textsc{Goethe, Johann Wolfgang von} (28.08.1749 – 22.03.1832), \emph{Schriftsteller}|pw}. {\pb}Hebbel\pwindex{Hebbel, Friedrich 18.03.1813 – 13.12.1863@\textsc{Hebbel, Friedrich} (18.03.1813 – 13.12.1863), \emph{Schriftsteller}|pw} war wohl nach Goethe\pwindex{Goethe, Johann Wolfgang von 28.08.1749 – 22.03.1832@\textsc{Goethe, Johann Wolfgang von} (28.08.1749 – 22.03.1832), \emph{Schriftsteller}|pw} der größte Geiſt, den die Deutſchen in dem Jahrhundert
                    gehabt haben; manchmal ko{\geminationm}t mir vor, daſs man ihn
                    vor \textsc{Nietz}ſche\pwindex{Nietzsche, Friedrich 15.10.1844 – 25.08.1900@\textsc{Nietzsche, Friedrich} (15.10.1844 – 25.08.1900), \emph{Schriftsteller, Philosoph}|pw} wird ne{\geminationn}en müſſen. Ich bin jetzt bei der Periode ſeines
                    Lebens, wo er auf der Verlegerſuche iſt und auf Gutzkow\pwindex{Gutzkow, Karl 17.03.1811 – 16.12.1878@\textsc{Gutzkow, Karl} (17.03.1811 – 16.12.1878), \emph{Schriftsteller}|pw}, Laube\pwindex{Laube, Heinrich 1806-09-18 – 1884-08-01@\textsc{Laube, Heinrich} (1806-09-18 – 1884-08-01), \emph{Schriftsteller, Theaterleiter}|pw}, Mundt\pwindex{Mundt, Theodor 19.09.1808 – 30.11.1861@\textsc{Mundt, Theodor} (19.09.1808 – 30.11.1861), \emph{Schriftsteller}|pw}, Körner\pwindex{Koerner, Christian Gottfried 1756-07-02 – 1831-05-13@\textsc{Körner, Christian Gottfried} (1756-07-02 – 1831-05-13), \emph{Schriftsteller}|pw},
                    zuweilen wohl auch auf Schiller\pwindex{Schiller, Friedrich von 10.11.1759 – 09.05.1805@\textsc{Schiller, Friedrich von} (10.11.1759 – 09.05.1805), \emph{Schriftsteller, Historiker, Philosoph}|pw}{ }ſchimpft. Er
                    hat aber auch noch manches andre zu ſagen. – Wiſſen Sie, daſs er eine {\pb}Jungfrau von Orleans\pwindex{Arc, Jeanne D um 1412 – 1431-05-30@\textsc{Arc, Jeanne d’} (um 1412 – 1431-05-30), \emph{Nationalheilige, Militärin}|pwv}{ }ſchreiben
                    wollte? –\pend
           \pstart
           Von Richard\pwindex{Beer-Hofmann, Richard 11.07.1866 – 26.09.1945@\textsc{Beer-Hofmann, Richard} (11.07.1866 – 26.09.1945), \emph{Schriftsteller}|pwv} hör ich nichts. Sie? –\pend
           \pstart
           Von Ihnen hoffe ich bald ſchönes und gutes zu erfahren; empfehlen Sie mich
                    bitte den Ihren aufs wärmſte.{\\[\baselineskip]}Ihr\hspace*{3.5em}\spacefill\mbox{Arthur}\pend
           \leftskip=0em{}\pstart
           14. 7. 92.\pend
           \pstart
           Wien\oindex{Wien@\textbf{Wien}|pw}.\pend
                     \endnumbering\briefempfaengerindex{Hofmannsthal, Hugo von@\textsc{Hofmannsthal, Hugo von}!zzzSchnitzler, Arthur@\emph{von Arthur Schnitzler}!1892-07-141@{14. 7. 1892}|)be}\mylabel{h}\end{ledgroupsized}  \newcommand{\dateiname}{L00104}\newcommand{\titel}{Arthur Schnitzler an Hugo von Hofmannsthal, 14. 7. 1892}\newcommand{\editorInnen}{Martin Anton Müller und Gerd-Hermann Susen}%% latex-leseansicht-abspann.tex
%% Abspann für die Leseansicht.
%% Der Schalter \ifkorrekturansicht ist bereits durch den Vorspann gesetzt.

%% latex-abspann.tex
%% Gemeinsamer Abspann für Korrekturansicht und Leseansicht.
%% Setzt den Schalter \ifkorrekturansicht voraus (gesetzt in den
%% einbindenden Dateien latex-korrekturansicht-abspann.tex bzw.
%% latex-leseansicht-abspann.tex).
%% ---------------------------------------------------------------

\normalsize

% Das esempio-Environment wird nur in der Leseansicht benötigt
\ifkorrekturansicht\else
\newenvironment{esempio}[3]%
{
    \vspace{1.5ex}
    \rlap{\underline{#1}}
    \par
    \setlength{\parindent}{0cm}
    \nopagebreak
    \leftskip=#2cm
    \rightskip=#3cm
}
{
    \par
}
\fi

\doendnotes{C}
\bigskip
\vfill

\clearpage

\footnotesize

\ifkorrekturansicht
  \lohead{\textsc{register}}
\fi

% theindex-Environment neu definieren ohne reledmac
\makeatletter
\renewenvironment{theindex}{%
  \ifkorrekturansicht
    \section*{\indexname}%
  \else
    \subsubsection*{Index der erwähnten Entitäten}%
  \fi
  \setlength{\parindent}{0pt}%
  \setlength{\parskip}{0pt plus 0.3pt}%
  \let\item\@idxitem
}{%
  \ifkorrekturansicht\clearpage\fi
}
\makeatother

\IfFileExists{\jobname-pw.ind}{\input{\jobname-pw.ind}}{}

% Quellenangabe nur in der Leseansicht
\ifkorrekturansicht\else
% Fallback-Definitionen, falls die .tex-Datei \titel etc. nicht gesetzt hat
\providecommand{\titel}{}
\providecommand{\editorInnen}{}
\providecommand{\dateiname}{\jobname}

\vspace{3cm}

\vfill

\footnotesize
\textsc{Quelle}: \titel. Herausgegeben von {\editorInnen}. In: \emph{Arthur Schnitzler: Briefwechsel mit Autorinnen und Autoren}.
 Digitale Edition, https://schnitzler-briefe.acdh.oeaw.ac.at/{\dateiname}.html (Stand \today)
\fi

\end{document}


      