%% latex-leseansicht-vorspann.tex
%% Vorspann für die Leseansicht.
%% Lädt die gemeinsame Datei latex-vorspann.tex mit nicht gesetztem Schalter.

\newif\ifkorrekturansicht
\korrekturansichtfalse

\input{../tex-inputs/latex-vorspann}

\begin{center}
            \textcolor{red}{ENTWURF, NICHT FERTIG KORRIGIERT}
                      \end{center}
            
         
         \newcommand{\erwaehntePersonen}{Personen: Fedor Mamroth, Louise Schnitzler, Jakob Wassermann}
         \newcommand{\erwaehnteInstitutionen}{Institutionen: Frankfurter Zeitung, Neue Freie Presse}
         \newcommand{\erwaehnteOrte}{Orte: Berlin, Frankfurt am Main, Wien}
         \newcommand{\erwaehnteWerke}{Werke: Agnes Jordan. Schauspiel in fünf Akten, Frankfurter Zeitung, Tagebuch, Wiener Burgtheater. (»Agnes Jordan« von Georg Hirschfeld.)}
               \section[ Paul Goldmann an Arthur Schnitzler, 23. 10. {[}1899{]}]{ Paul Goldmann an Arthur Schnitzler, 23. 10. {[}1899{]}}\nopagebreak\mylabel{v}\rehead{ }\begin{ledgroupsized}[t]{13cm}\normalsize\beginnumbering \toendnotes[C]{\smallbreak\pagebreak[2]} \Standort{DLA, A:Schnitzler, HS.NZ85.1.3169.}
\physDesc{Brief, 1 Blatt, 2 Seiten
\newline{}Handschrift: blaue Tinte, deutsche Kurrent
\newline{}Schnitzler: mit Bleistift das Jahr »99.« vermerkt }\toendnotes[C]{\smallbreak}\pstart
           \centering{}{\pb}23. Oktober.\pend
           \pstart{}Mein lieber Freund,\pend\pstart
           Ich bin ſeit geſtern wieder hier\oindex{Frankfurt am Main@\textbf{Frankfurt am Main}|pwv}, ſtecke im Büreau\orgindex{Frankfurter Zeitung@Frankfurter Zeitung|pwv}dienſt und will Dir nur raſch noch
               einmal und von ganzem Herzen danken für alle die ſo wohlthuende Freundſchaft, die Du
               mir während meines \label{K_L02891-1v}\edtext{Aufenthaltes in Wien\oindex{Wien@\textbf{Wien}|pw}}{\lemma{\textnormal{\emph{Aufenthaltes in Wien}}}\Cendnote{\textnormal{Goldmann\pwindex{Goldmann, Paul 31.01.1865 – 25.09.1935@\textsc{Goldmann, Paul} (31.01.1865 – 25.09.1935), \emph{Schriftsteller, Journalist}|pwk} war von 13. 10. 1899 bis 21. 10. 1899 in Wien\oindex{Wien@\textbf{Wien}|pwk}. Er wohnte bei Schnitzler\pwindex{Schnitzler, Arthur 15.05.1862 – 21.10.1931@\textsc{Schnitzler, Arthur} (15.05.1862 – 21.10.1931), \emph{Schriftsteller, Mediziner}|pwk}.}}}\label{K_L02891-1h} erwieſen haſt.\pend
           \pstart
           Auch Deiner Frau Mutter\pwindex{Schnitzler, Louise 1840-07-08 – 1911-09-09@\textsc{Schnitzler, Louise} (1840-07-08 – 1911-09-09)|pwv}
               bitte ich meinen herzlichſten Dank für ihre Güte zu ſagen.\pend
           \pstart
           Meine Familie hier iſt der \textcolor{gray}{A}nſicht, daß ich nach Berlin\oindex{Berlin@\textbf{Berlin}|pw} für die Neue {\pb}Freie Preſſe\orgindex{Neue Freie Presse@Neue Freie Presse|pw} gehen ſoll. Ich habe heut an die Leute\orgindex{Neue Freie Presse@Neue Freie Presse|pwv} wegen der materiellen Bedingungen geſchrieben und werde
               Dir von dem Ausgang der Verhandlungen ſofort berichten.\pend
           \pstart
           Dein \label{K_L02891-3v}\edtext{Burgtheater-Referat\pwindex{Wiener Burgtheater. (»Agnes Jordan« von Georg Hirschfeld.)1899-10-25@\emph{Wiener Burgtheater. (»Agnes Jordan« von Georg Hirschfeld.)} {[}1899-10-25{]}|pwv}}{\lemma{\textnormal{\emph{Burgtheater-Referat}}}\Cendnote{\textnormal{–rm–\pwindex{Schnitzler, Arthur 15.05.1862 – 21.10.1931@\textsc{Schnitzler, Arthur} (15.05.1862 – 21.10.1931), \emph{Schriftsteller, Mediziner}|pwkv} [=Arthur Schnitzler\pwindex{Schnitzler, Arthur 15.05.1862 – 21.10.1931@\textsc{Schnitzler, Arthur} (15.05.1862 – 21.10.1931), \emph{Schriftsteller, Mediziner}|pwk}]: \emph{[Rezension über Agnes Jordan]}\pwindex{Wiener Burgtheater. (»Agnes Jordan« von Georg Hirschfeld.)1899-10-25@\emph{Wiener Burgtheater. (»Agnes Jordan« von Georg Hirschfeld.)} {[}1899-10-25{]}|pwk}. In: \emph{Frankfurter Zeitung}\pwindex{?? Werk@Nicht ermittelte Verfasserinnen und Verfasser!Frankfurter Zeitung1856 – 1943@\emph{Frankfurter Zeitung} {[}1856 – 1943{]}|pwk}, Jg. 44, Nr. 296, 25. 10. 1899, 2. Morgenblatt, S. 1. Die
                     Kritik\pwindex{Wiener Burgtheater. (»Agnes Jordan« von Georg Hirschfeld.)1899-10-25@\emph{Wiener Burgtheater. (»Agnes Jordan« von Georg Hirschfeld.)} {[}1899-10-25{]}|pwkv} erschien unter
                     Jakob Wassermann\pwindex{Wassermann, Jakob 10.03.1873 – 01.01.1934@\textsc{Wassermann, Jakob} (10.03.1873 – 01.01.1934), \emph{Schriftsteller}|pwk}s Kürzel. Schnitzler\pwindex{Schnitzler, Arthur 15.05.1862 – 21.10.1931@\textsc{Schnitzler, Arthur} (15.05.1862 – 21.10.1931), \emph{Schriftsteller, Mediziner}|pwk} besuchte die Premiere\pwindex{\textcolor{red}{\textsuperscript{XXXX1 indx}}!Agnes Jordan. Schauspiel in fuenf Akten1897@\strich\emph{Agnes Jordan. Schauspiel in fünf Akten} {[}1897{]}|pwkv} für ihn, weil dieser krank war.
                  Vgl. \emph{Jakob Wassermann. 1873–1934. Ein Weg als Deutscher und Jude.
                        Lesebuch zu einer Ausstellung}. Hg. v. Dierk Rodewald.
                     Bonn: \emph{Bouvier Verlag Herbert
                        Grundmann}{ }1984, S. 34. (Schriften des Arbeitskreises
                     selbstständiger Kultur-Institute, 3)}}}\label{K_L02891-3h} hat mein Onkel\pwindex{Mamroth, Fedor 21.02.1851 – 25.06.1907@\textsc{Mamroth, Fedor} (21.02.1851 – 25.06.1907), \emph{Journalist, Kritiker}|pwv} mit
               Dank in Empfang genommen.\pend
           \pstart
           Von der langen Eiſenbahnfahrt iſt mein Darm in einem ſchauerlichen Zuſtande{\dotsseven}\pend
           \pstart
           Liebſter Freund, es that mir ſehr weh, als ich Samſtag{ }Abend aus Wien\oindex{Wien@\textbf{Wien}|pw} hinausfahren mußte und
               Dich ſo einſam fortgehen ſah. Haſt Du zu \label{K_L02891-5v}\edtext{arbeiten}{\lemma{\textnormal{\emph{arbeiten}}}\Cendnote{\textnormal{Im \emph{Tagebuch}\pwindex{Schnitzler, Arthur 15.05.1862 – 21.10.1931@\textsc{Schnitzler, Arthur} (15.05.1862 – 21.10.1931), \emph{Schriftsteller, Mediziner}!Tagebuch1981 – 2000@\strich\emph{Tagebuch} {[}1981 – 2000{]}|pwk} ist zumindest nichts festgehalten, 
                  das darauf hinwiese.}}}\label{K_L02891-5h} angefangen?\pend
           \pstart
           Viele treue Grüße! {\\[\baselineskip]}Dein {\\[\baselineskip]}\spacefill\mbox{Paul Goldmn}\pend
           \leftskip=0em{}
         
         \endnumbering\mylabel{h}\end{ledgroupsized}  \newcommand{\dateiname}{L02891}\newcommand{\titel}{Paul Goldmann an Arthur Schnitzler, 23. 10. [1899]}\newcommand{\editorInnen}{Martin Anton Müller und Laura Untner}%% latex-leseansicht-abspann.tex
%% Abspann für die Leseansicht.
%% Der Schalter \ifkorrekturansicht ist bereits durch den Vorspann gesetzt.

%% latex-abspann.tex
%% Gemeinsamer Abspann für Korrekturansicht und Leseansicht.
%% Setzt den Schalter \ifkorrekturansicht voraus (gesetzt in den
%% einbindenden Dateien latex-korrekturansicht-abspann.tex bzw.
%% latex-leseansicht-abspann.tex).
%% ---------------------------------------------------------------

\normalsize

% Das esempio-Environment wird nur in der Leseansicht benötigt
\ifkorrekturansicht\else
\newenvironment{esempio}[3]%
{
    \vspace{1.5ex}
    \rlap{\underline{#1}}
    \par
    \setlength{\parindent}{0cm}
    \nopagebreak
    \leftskip=#2cm
    \rightskip=#3cm
}
{
    \par
}
\fi

\doendnotes{C}
\bigskip
\vfill

\clearpage

\footnotesize

\ifkorrekturansicht
  \lohead{\textsc{register}}
\fi

% theindex-Environment neu definieren ohne reledmac
\makeatletter
\renewenvironment{theindex}{%
  \ifkorrekturansicht
    \section*{\indexname}%
  \else
    \subsubsection*{Index der erwähnten Entitäten}%
  \fi
  \setlength{\parindent}{0pt}%
  \setlength{\parskip}{0pt plus 0.3pt}%
  \let\item\@idxitem
}{%
  \ifkorrekturansicht\clearpage\fi
}
\makeatother

\IfFileExists{\jobname-pw.ind}{\input{\jobname-pw.ind}}{}

% Quellenangabe nur in der Leseansicht
\ifkorrekturansicht\else
% Fallback-Definitionen, falls die .tex-Datei \titel etc. nicht gesetzt hat
\providecommand{\titel}{}
\providecommand{\editorInnen}{}
\providecommand{\dateiname}{\jobname}

\vspace{3cm}

\vfill

\footnotesize
\textsc{Quelle}: \titel. Herausgegeben von {\editorInnen}. In: \emph{Arthur Schnitzler: Briefwechsel mit Autorinnen und Autoren}.
 Digitale Edition, https://schnitzler-briefe.acdh.oeaw.ac.at/{\dateiname}.html (Stand \today)
\fi

\end{document}


      