%% latex-korrekturansicht-vorspann.tex
%% Vorspann für die Korrekturansicht.
%% Lädt die gemeinsame Datei latex-vorspann.tex mit gesetztem Schalter.

\newif\ifkorrekturansicht
\korrekturansichttrue

\input{../tex-inputs/latex-vorspann}


\section[ Paul Goldmann an Arthur Schnitzler, 23. 10. {[}1899{]}]{L02891 Paul Goldmann an Arthur Schnitzler, 23. 10. {[}1899{]}}
\nopagebreak\mylabel{L02891v}
\rehead{ }\normalsize\beginnumbering\briefempfaengerindex{Schnitzler, Arthur@\textsc{Schnitzler, Arthur}!zzzGoldmann, Paul@\emph{von Paul Goldmann}!1899-10-231@{23. 10. {[}1899{]}}|(be}
\toendnotes[C]{\smallbreak\pagebreak[2]}\Standort{DLA, A:Schnitzler, HS.NZ85.1.3169.}
\physDesc{Brief, 1 Blatt, 2 Seiten, 878 Zeichen
\newline{}Handschrift: blaue Tinte, deutsche Kurrent
\newline{}Schnitzler: mit Bleistift das Jahr »99.« vermerkt }\toendnotes[C]{\smallbreak}
\pstart
           \centering{}{\pb}23. Oktober.\pend
           
\pstart{}Mein lieber Freund,\pend\vspace{0.5em}
\pstart
           Ich bin ſeit geſtern wieder hier\oindex{Frankfurt am Main@\textbf{Frankfurt am Main}, \emph{P.PPLA3}|pwv}, ſtecke im Büreau\orgindex{Frankfurter Zeitung@Frankfurter Zeitung|pwv}dienſt und will Dir nur raſch noch
               einmal und von ganzem Herzen danken für alle die ſo wohlthuende Freundſchaft, die Du
               mir während meines \label{K_L02891-1v}\edtext{Aufenthaltes in Wien\oindex{Wien@\textbf{Wien}, \emph{A.ADM2}|pw}}{\lemma{\textnormal{\emph{Aufenthaltes in Wien}}}\Cendnote{\textnormal{Goldmann\pwindex{Goldmann, Paul 31.01.1865 – 25.09.1935@\textsc{Goldmann, Paul} (31.01.1865 – 25.09.1935), \emph{Schriftsteller/Schriftstellerin, Journalist/Journalistin}|pwk} war vom 13. 10. 1899 bis zum 21. 10. 1899 in Wien\oindex{Wien@\textbf{Wien}, \emph{A.ADM2}|pwk} gewesen. Er hatte bei Schnitzler gewohnt.}}}\label{K_L02891-1} erwieſen haſt.\pend
           
\pstart
           Auch Deiner Frau Mutter\pwindex{Schnitzler, Louise 1840-07-08 – 1911-09-09@\textsc{Schnitzler, Louise} (1840-07-08 – 1911-09-09)|pwv}
               bitte ich meinen herzlichſten Dank für ihre Güte zu ſagen.\pend
           
\pstart
           Meine Familie hier iſt der \textcolor{gray}{A}nſicht, daß ich nach Berlin\oindex{Berlin@\textbf{Berlin}, \emph{P.PPLC}|pw} für die Neue {\pb}Freie Preſſe\orgindex{Neue Freie Presse@Neue Freie Presse|pw} gehen ſoll. Ich habe heut an die Leute\orgindex{Neue Freie Presse@Neue Freie Presse|pwv} wegen der materiellen Bedingungen geſchrieben und werde
               Dir von dem Ausgang der Verhandlungen ſofort berichten.\pend
           
\pstart
           Dein \label{K_L02891-2v}\edtext{Burgtheater-Referat\pwindex{Wiener Burgtheater. (»Agnes Jordan« von Georg Hirschfeld.)@\emph{Wiener Burgtheater. (»Agnes Jordan« von Georg Hirschfeld.)}|pwv}}{\lemma{\textnormal{\emph{Burgtheater-Referat}}}\Cendnote{\textnormal{–rm– [ = Arthur Schnitzler]: \emph{Wiener Burgtheater. (»Agnes Jordan« von Georg
                        Hirschfeld)}\pwindex{Wiener Burgtheater. (»Agnes Jordan« von Georg Hirschfeld.)@\emph{Wiener Burgtheater. (»Agnes Jordan« von Georg Hirschfeld.)}|pwk}. In: \emph{Frankfurter
                           Zeitung}\pwindex{Frankfurter Zeitung@\emph{Frankfurter Zeitung}|pwk}, Jg. 44, Nr. 296, 25. 10. 1899, Zweites Morgenblatt, S. 1. Die Kritik\pwindex{Wiener Burgtheater. (»Agnes Jordan« von Georg Hirschfeld.)@\emph{Wiener Burgtheater. (»Agnes Jordan« von Georg Hirschfeld.)}|pwkv} erschien unter Jakob Wassermanns\pwindex{Wassermann, Jakob 10.03.1873 – 01.01.1934@\textsc{Wassermann, Jakob} (10.03.1873 – 01.01.1934), \emph{Schriftsteller/Schriftstellerin}|pwk} Kürzel. Schnitzler besuchte die Premiere\pwindex{Agnes Jordan. Schauspiel in fuenf Akten@\emph{Agnes Jordan. Schauspiel in fünf Akten}|pwkv} für Wassermann\pwindex{Wassermann, Jakob 10.03.1873 – 01.01.1934@\textsc{Wassermann, Jakob} (10.03.1873 – 01.01.1934), \emph{Schriftsteller/Schriftstellerin}|pwk}, weil dieser krank war. Vgl. 
                     \emph{Jakob Wassermann. 1873–1934. Ein Weg als Deutscher und Jude.
                        Lesebuch zu einer Ausstellung}. Herausgegeben von Dierk Rodewald.
                     Bonn: \emph{Bouvier Verlag Herbert
                        Grundmann}{ }1984, S. 34 (Schriften des Arbeitskreises
                              selbstständiger Kultur-Institute, 3). Siehe A. S.: \emph{»Das Zeitlose ist von kürzester Dauer«}, –rm–: Wiener Burgtheater. (»Agnes Jordan« von Georg Hirschfeld), 25. 10. 1899.
               }}}\label{K_L02891-2} hat mein Onkel\pwindex{Mamroth, Fedor 21.02.1851 – 25.06.1907@\textsc{Mamroth, Fedor} (21.02.1851 – 25.06.1907), \emph{Journalist/Journalistin, Kritiker/Kritikerin}|pwv} mit
               Dank in Empfang genommen.\pend
           
\pstart
           Von der langen Eiſenbahnfahrt iſt mein Darm in einem ſchauerlichen Zuſtande{\dotsseven}\pend
           
\pstart
           Liebſter Freund, es that mir ſehr weh, als ich Samſtag{ }Abend aus Wien\oindex{Wien@\textbf{Wien}, \emph{A.ADM2}|pw} hinausfahren mußte und
               Dich ſo einſam fortgehen ſah. Haſt Du zu \label{K_L02891-3v}\edtext{arbeiten}{\lemma{\textnormal{\emph{arbeiten}}}\Cendnote{\textnormal{Im \emph{Tagebuch}\pwindex{Tagebuch@\emph{Tagebuch}|pwk} ist zumindest nichts festgehalten,
                  das darauf hinwiese.}}}\label{K_L02891-3} angefangen?\pend
           
\pstart
           Viele treue Grüße! {\\[\baselineskip]}Dein {\\[\baselineskip]}\spacefill\mbox{Paul Goldmn}\pend
           \leftskip=0em{}\selectlanguage{ngerman}\endnumbering\briefempfaengerindex{Schnitzler, Arthur@\textsc{Schnitzler, Arthur}!zzzGoldmann, Paul@\emph{von Paul Goldmann}!1899-10-231@{23. 10. {[}1899{]}}|)be}\mylabel{L02891h}  \normalsize

\doendnotes{C}
\bigskip
\vfill

\clearpage

\footnotesize

\lohead{\textsc{register}}

% Definiere theindex-Environment komplett neu ohne reledmac
\makeatletter
\renewenvironment{theindex}{%
  \section*{\indexname}%
  \setlength{\parindent}{0pt}%
  \setlength{\parskip}{0pt plus 0.3pt}%
  \let\item\@idxitem
}{%
  \clearpage
}
\makeatother

\IfFileExists{\jobname-pw.ind}{\input{\jobname-pw.ind}}{}

\end{document}

      