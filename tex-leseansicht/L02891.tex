%% latex-leseansicht-vorspann.tex
%% Vorspann für die Leseansicht.
%% Lädt die gemeinsame Datei latex-vorspann.tex mit nicht gesetztem Schalter.

\newif\ifkorrekturansicht
\korrekturansichtfalse

\input{../tex-inputs/latex-vorspann}


\section[ Paul Goldmann an Arthur Schnitzler, 23. 10. {[}1899{]}]{L02891 Paul Goldmann an Arthur Schnitzler,  23. 10. [1899]}
\nopagebreak\mylabel{L02891v}
\rehead{ }\normalsize\beginnumbering\briefempfaengerindex{Schnitzler, Arthur@\textsc{Schnitzler, Arthur}!zzzGoldmann, Paul@\emph{von Paul Goldmann}!1899-10-231@{23. 10. [1899]}|(be}
\toendnotes[C]{\smallbreak\pagebreak[2]}
\correspDesc{Versand  durch Paul Goldmann am 23. 10. [1899] in Frankfurt am Main
\newline{}Erhalt  durch Arthur Schnitzler im Zeitraum [24. 10. 1899 – 28. 10. 1899?] in Wien}\toendnotes[C]{\smallbreak}
\Standort{DLA, A:Schnitzler, HS.NZ85.1.3169.}
\physDesc{Brief, 1 Blatt, 2 Seiten, 878 Zeichen
\newline{}Handschrift: blaue Tinte, deutsche Kurrent
\newline{}Schnitzler: mit Bleistift das Jahr »99.« vermerkt }\toendnotes[C]{\smallbreak}
\pstart
           \centering{}{\pb}23. Oktober.\pend
           
\pstart{}Mein lieber Freund,\pend\vspace{0.5em}
\pstart
           Ich bin{ }ſeit geſtern wieder hier\oindex{Frankfurt am Main@\textbf{Frankfurt am Main}, \emph{Hauptstadt}|pwv},{ }ſtecke im Büreau\orgindex{Frankfurter Zeitung@Frankfurter Zeitung|pwv}dienſt und will Dir nur raſch noch
               einmal und von ganzem Herzen danken für alle die{ }ſo wohlthuende Freundſchaft, die Du
               mir während meines \label{K_L02891-1v}\edtext{Aufenthaltes in Wien\oindex{Wien@\textbf{Wien}, \emph{Verwaltungsgebiet}|pw}}{\lemma{\textnormal{\emph{Aufenthaltes in Wien}}}\Cendnote{\textnormal{Goldmann\pwindex{Goldmann, Paul 31.\,1.\,1865 Breslau – 25.\,9.\,1935 Wien@\textsc{Goldmann, Paul} (31.\,1.\,1865 Breslau – 25.\,9.\,1935 Wien), \emph{Schriftsteller, Journalist}|pwk} war vom 13. 10. 1899 bis zum 21. 10. 1899 in Wien\oindex{Wien@\textbf{Wien}, \emph{Verwaltungsgebiet}|pwk} gewesen. Er hatte bei Schnitzler gewohnt.}}}\label{K_L02891-1} erwieſen haſt.\pend
           
\pstart
           Auch Deiner Frau Mutter\pwindex{Schnitzler, Louise 8.\,7.\,1840 Kőszeg – 9.\,9.\,1911 Wien@\textsc{Schnitzler, Louise} (8.\,7.\,1840 Kőszeg – 9.\,9.\,1911 Wien)|pwv}
               bitte ich meinen herzlichſten Dank für ihre Güte zu{ }ſagen.\pend
           
\pstart
           Meine Familie hier iſt der \textcolor{gray}{A}nſicht, daß ich nach Berlin\oindex{Berlin@\textbf{Berlin}, \emph{Hauptstadt}|pw} für die Neue {\pb}Freie Preſſe\orgindex{Neue Freie Presse@Neue Freie Presse|pw} gehen{ }ſoll. Ich habe heut an die Leute\orgindex{Neue Freie Presse@Neue Freie Presse|pwv} wegen der materiellen Bedingungen geſchrieben und werde
               Dir von dem Ausgang der Verhandlungen{ }ſofort berichten.\pend
           
\pstart
           Dein \label{K_L02891-2v}\edtext{Burgtheater-Referat\pwindex{Schnitzler, Arthur 15.\,5.\,1862 Wien – 21.\,10.\,1931 ebd.@\textsc{Schnitzler, Arthur} (15.\,5.\,1862 Wien – 21.\,10.\,1931 ebd.), \emph{Schriftsteller, Mediziner}!Wiener Burgtheater. (»Agnes Jordan« von Georg Hirschfeld.)@\strich\emph{Wiener Burgtheater. (»Agnes Jordan« von Georg Hirschfeld.)}|pwv}}{\lemma{\textnormal{\emph{Burgtheater-Referat}}}\Cendnote{\textnormal{–rm– [ = Arthur Schnitzler]: \emph{Wiener Burgtheater. (»Agnes Jordan« von Georg
                        Hirschfeld)}\pwindex{Schnitzler, Arthur 15.\,5.\,1862 Wien – 21.\,10.\,1931 ebd.@\textsc{Schnitzler, Arthur} (15.\,5.\,1862 Wien – 21.\,10.\,1931 ebd.), \emph{Schriftsteller, Mediziner}!Wiener Burgtheater. (»Agnes Jordan« von Georg Hirschfeld.)@\strich\emph{Wiener Burgtheater. (»Agnes Jordan« von Georg Hirschfeld.)}|pwk}. In: \emph{Frankfurter
                           Zeitung}\pwindex{Frankfurter Zeitung@\emph{Frankfurter Zeitung}|pwk}, Jg. 44, Nr. 296, 25. 10. 1899, Zweites Morgenblatt, S. 1. Die Kritik\pwindex{Schnitzler, Arthur 15.\,5.\,1862 Wien – 21.\,10.\,1931 ebd.@\textsc{Schnitzler, Arthur} (15.\,5.\,1862 Wien – 21.\,10.\,1931 ebd.), \emph{Schriftsteller, Mediziner}!Wiener Burgtheater. (»Agnes Jordan« von Georg Hirschfeld.)@\strich\emph{Wiener Burgtheater. (»Agnes Jordan« von Georg Hirschfeld.)}|pwkv} erschien unter Jakob Wassermanns\pwindex{Wassermann, Jakob 10.\,3.\,1873 Fürth – 1.\,1.\,1934 Altaussee@\textsc{Wassermann, Jakob} (10.\,3.\,1873 Fürth – 1.\,1.\,1934 Altaussee), \emph{Schriftsteller}|pwk} Kürzel. Schnitzler besuchte die Premiere\pwindex{\textcolor{red}{\textsuperscript{XXXX indx1}}!Agnes Jordan. Schauspiel in fünf Akten@\strich\emph{Agnes Jordan. Schauspiel in fünf Akten}|pwkv} für Wassermann\pwindex{Wassermann, Jakob 10.\,3.\,1873 Fürth – 1.\,1.\,1934 Altaussee@\textsc{Wassermann, Jakob} (10.\,3.\,1873 Fürth – 1.\,1.\,1934 Altaussee), \emph{Schriftsteller}|pwk}, weil dieser krank war. Vgl. 
                     \emph{Jakob Wassermann. 1873–1934. Ein Weg als Deutscher und Jude.
                        Lesebuch zu einer Ausstellung}. Herausgegeben von Dierk Rodewald.
                     Bonn: \emph{Bouvier Verlag Herbert
                        Grundmann}{ }1984, S. 34 (Schriften des Arbeitskreises
                              selbstständiger Kultur-Institute, 3). Siehe A. S.: \emph{»Das Zeitlose ist von kürzester Dauer«}, –rm–: Wiener Burgtheater. (»Agnes Jordan« von Georg Hirschfeld), 25. 10. 1899.
               }}}\label{K_L02891-2} hat mein Onkel\pwindex{Mamroth, Fedor 21.\,2.\,1851 Breslau – 25.\,6.\,1907 Frankfurt am Main@\textsc{Mamroth, Fedor} (21.\,2.\,1851 Breslau – 25.\,6.\,1907 Frankfurt am Main), \emph{Journalist, Kritiker}|pwv} mit
               Dank in Empfang genommen.\pend
           
\pstart
           Von der langen Eiſenbahnfahrt iſt mein Darm in einem{ }ſchauerlichen Zuſtande{\dotsseven}\pend
           
\pstart
           Liebſter Freund, es that mir{ }ſehr weh, als ich Samſtag{ }Abend aus Wien\oindex{Wien@\textbf{Wien}, \emph{Verwaltungsgebiet}|pw} hinausfahren mußte und
               Dich{ }ſo einſam fortgehen{ }ſah. Haſt Du zu \label{K_L02891-3v}\edtext{arbeiten}{\lemma{\textnormal{\emph{arbeiten}}}\Cendnote{\textnormal{Im \emph{Tagebuch}\pwindex{Schnitzler, Arthur 15.\,5.\,1862 Wien – 21.\,10.\,1931 ebd.@\textsc{Schnitzler, Arthur} (15.\,5.\,1862 Wien – 21.\,10.\,1931 ebd.), \emph{Schriftsteller, Mediziner}!Tagebuch@\strich\emph{Tagebuch}|pwk} ist zumindest nichts festgehalten,
                  das darauf hinwiese.}}}\label{K_L02891-3} angefangen?\pend
           
\pstart
           Viele treue Grüße! {\\[\baselineskip]}Dein {\\[\baselineskip]}\spacefill\mbox{Paul Goldmn}\pend
           \leftskip=0em{}\selectlanguage{ngerman}\endnumbering\briefempfaengerindex{Schnitzler, Arthur@\textsc{Schnitzler, Arthur}!zzzGoldmann, Paul@\emph{von Paul Goldmann}!1899-10-231@{23. 10. [1899]}|)be}\mylabel{L02891h}  \newcommand{\dateiname}{L02891}\newcommand{\titel}{Paul Goldmann an Arthur Schnitzler, 23. 10. [1899]}\newcommand{\editorInnen}{Martin Anton Müller und Laura Untner}%% latex-leseansicht-abspann.tex
%% Abspann für die Leseansicht.
%% Der Schalter \ifkorrekturansicht ist bereits durch den Vorspann gesetzt.

%% latex-abspann.tex
%% Gemeinsamer Abspann für Korrekturansicht und Leseansicht.
%% Setzt den Schalter \ifkorrekturansicht voraus (gesetzt in den
%% einbindenden Dateien latex-korrekturansicht-abspann.tex bzw.
%% latex-leseansicht-abspann.tex).
%% ---------------------------------------------------------------

\normalsize

% Das esempio-Environment wird nur in der Leseansicht benötigt
\ifkorrekturansicht\else
\newenvironment{esempio}[3]%
{
    \vspace{1.5ex}
    \rlap{\underline{#1}}
    \par
    \setlength{\parindent}{0cm}
    \nopagebreak
    \leftskip=#2cm
    \rightskip=#3cm
}
{
    \par
}
\fi

\doendnotes{C}
\bigskip
\vfill

\clearpage

\footnotesize

\ifkorrekturansicht
  \lohead{\textsc{register}}
\fi

% theindex-Environment neu definieren ohne reledmac
\makeatletter
\renewenvironment{theindex}{%
  \ifkorrekturansicht
    \section*{\indexname}%
  \else
    \subsubsection*{Index der erwähnten Entitäten}%
  \fi
  \setlength{\parindent}{0pt}%
  \setlength{\parskip}{0pt plus 0.3pt}%
  \let\item\@idxitem
}{%
  \ifkorrekturansicht\clearpage\fi
}
\makeatother

\IfFileExists{\jobname-pw.ind}{\input{\jobname-pw.ind}}{}

% Quellenangabe nur in der Leseansicht
\ifkorrekturansicht\else
% Fallback-Definitionen, falls die .tex-Datei \titel etc. nicht gesetzt hat
\providecommand{\titel}{}
\providecommand{\editorInnen}{}
\providecommand{\dateiname}{\jobname}

\vspace{3cm}

\vfill

\footnotesize
\textsc{Quelle}: \titel. Herausgegeben von {\editorInnen}. In: \emph{Arthur Schnitzler: Briefwechsel mit Autorinnen und Autoren}.
 Digitale Edition, https://schnitzler-briefe.acdh.oeaw.ac.at/{\dateiname}.html (Stand \today)
\fi

\end{document}


