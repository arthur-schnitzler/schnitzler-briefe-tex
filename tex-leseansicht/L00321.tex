%% latex-korrekturansicht-vorspann.tex
%% Vorspann für die Korrekturansicht.
%% Lädt die gemeinsame Datei latex-vorspann.tex mit gesetztem Schalter.

\newif\ifkorrekturansicht
\korrekturansichttrue

\input{../tex-inputs/latex-vorspann}


\section[Detlev von Liliencron an Arthur Schnitzler, 7. 5. 1894]{L00321 Detlev von Liliencron an Arthur Schnitzler, 7. 5. 1894}
\nopagebreak\mylabel{L00321v}
\rehead{ }\normalsize\beginnumbering\briefempfaengerindex{Schnitzler, Arthur@\textsc{Schnitzler, Arthur}!zzzLiliencron, Detlev von@\emph{von Detlev von Liliencron}!1894-05-071@{7. 5. 1894}|(be}
\toendnotes[C]{\smallbreak\pagebreak[2]}\Standort{DLA, A:Schnitzler, HS.NZ85.1.3896, S. 1.}
\physDesc{Brief, maschinenschriftliche Abschrift1 Blatt, 1 Seite, 784 Zeichen
\newline{}Schreibmaschine}\toendnotes[C]{\smallbreak}
\pstart
           \raggedleft{}{\pb}Altona (Elbe), Palmaille 5\oindex{Palmaille@\textbf{Palmaille}, \emph{Straße (K.STR)}|pw},{\\}Den
                     7. 5. 94.\pend
           
\pstart{}Sehr geehrter Herr Doctor,\pend\vspace{0.5em}
\pstart
           Sie hatten die Güte mir Ihr Schauspiel: Das
                  Märchen\pwindex{Maerchen. Schauspiel in drei Aufzuegen@\emph{Das Märchen. Schauspiel in drei Aufzügen}|pw} zu übersenden.\pend
           
\pstart
           Ich habs jetzt in einem Zuge durchgelesen. Ich habe keine Ahnung von Dramatik. Ich
               kann also nur das aussprechen, was ich beim Lesen gefühlt habe. Und das ist in erster
               Reihe: dass ich bis zur letzten Zeile gefesselt war von Ihrem Stück\pwindex{Maerchen. Schauspiel in drei Aufzuegen@\emph{Das Märchen. Schauspiel in drei Aufzügen}|pwv}, mit allen Fibern! Es ist ein Stück\pwindex{Maerchen. Schauspiel in drei Aufzuegen@\emph{Das Märchen. Schauspiel in drei Aufzügen}|pwv} aus \uline{unserm} Leben und aus dem Leben der \uline{Zukunft}. Ungemein fein haben Sie die Frauenfrage gestreift. Ich \uline{sah} beim Lesen alle Ihre Menschen ganz leibhaftig vor
               mir. Und ich hoffe sehr, dass das Märchen\pwindex{Maerchen. Schauspiel in drei Aufzuegen@\emph{Das Märchen. Schauspiel in drei Aufzügen}|pw} nicht
               nur die Freien Bühnen beschäftigen wird, sondern erst recht unsere grossen Theater,
               wenn diesen noch ein letzter Ernst geblieben ist.\pend
           
\pstart
           Ihr hochachtungsvoll ergebener{\\[\baselineskip]}\spacefill\mbox{Baron Detlev Liliencron.}\pend
           \leftskip=0em{}\selectlanguage{ngerman}\endnumbering\briefempfaengerindex{Schnitzler, Arthur@\textsc{Schnitzler, Arthur}!zzzLiliencron, Detlev von@\emph{von Detlev von Liliencron}!1894-05-071@{7. 5. 1894}|)be}\mylabel{L00321h}  \normalsize

\doendnotes{C}
\bigskip
\vfill

\clearpage

\footnotesize

\lohead{\textsc{register}}

% Definiere theindex-Environment komplett neu ohne reledmac
\makeatletter
\renewenvironment{theindex}{%
  \section*{\indexname}%
  \setlength{\parindent}{0pt}%
  \setlength{\parskip}{0pt plus 0.3pt}%
  \let\item\@idxitem
}{%
  \clearpage
}
\makeatother

\IfFileExists{\jobname-pw.ind}{\input{\jobname-pw.ind}}{}

\end{document}

      