%% latex-leseansicht-vorspann.tex
%% Vorspann für die Leseansicht.
%% Lädt die gemeinsame Datei latex-vorspann.tex mit nicht gesetztem Schalter.

\newif\ifkorrekturansicht
\korrekturansichtfalse

\input{../tex-inputs/latex-vorspann}


\section[Detlev von Liliencron an Arthur Schnitzler, 7. 5. 1894]{L00321 Detlev von Liliencron an Arthur Schnitzler, 7. 5. 1894}
\nopagebreak\mylabel{L00321v}
\rehead{ }\normalsize\beginnumbering\briefempfaengerindex{Schnitzler, Arthur@\textsc{Schnitzler, Arthur}!zzzLiliencron, Detlev von@\emph{von Detlev von Liliencron}!1894-05-071@{7. 5. 1894}|(be}
\toendnotes[C]{\smallbreak\pagebreak[2]}
\correspDesc{Versand  durch Detlev von Liliencron am 7. 5. 1894 in Hamburg
\newline{}Erhalt  durch Arthur Schnitzler im Zeitraum [8. 5. 1894
                  – 12. 5. 1894?] in Wien}\toendnotes[C]{\smallbreak}
\Standort{DLA, A:Schnitzler, HS.NZ85.1.3896, S. 1.}
\physDesc{Brief, maschinenschriftliche Abschrift, 1 Blatt, 1 Seite, 784 Zeichen
\newline{}Schreibmaschine}\toendnotes[C]{\smallbreak}
\pstart
           \raggedleft{}{\pb}Altona (Elbe), Palmaille 5\oindex{Palmaille@\textbf{Palmaille}, \emph{Straße}|pw},{\\}Den
                     7. 5. 94.\pend
           
\pstart{}Sehr geehrter Herr Doctor,\pend\vspace{0.5em}
\pstart
           Sie hatten die Güte mir Ihr Schauspiel: Das
                  Märchen\pwindex{Schnitzler, Arthur 15.\,5.\,1862 Wien – 21.\,10.\,1931 ebd.@\textsc{Schnitzler, Arthur} (15.\,5.\,1862 Wien – 21.\,10.\,1931 ebd.), \emph{Schriftsteller, Mediziner}!Märchen. Schauspiel in drei Aufzügen@\strich\emph{Das Märchen. Schauspiel in drei Aufzügen}|pw} zu übersenden.\pend
           
\pstart
           Ich habs jetzt in einem Zuge durchgelesen. Ich habe keine Ahnung von Dramatik. Ich
               kann also nur das aussprechen, was ich beim Lesen gefühlt habe. Und das ist in erster
               Reihe: dass ich bis zur letzten Zeile gefesselt war von Ihrem Stück\pwindex{Schnitzler, Arthur 15.\,5.\,1862 Wien – 21.\,10.\,1931 ebd.@\textsc{Schnitzler, Arthur} (15.\,5.\,1862 Wien – 21.\,10.\,1931 ebd.), \emph{Schriftsteller, Mediziner}!Märchen. Schauspiel in drei Aufzügen@\strich\emph{Das Märchen. Schauspiel in drei Aufzügen}|pwv}, mit allen Fibern! Es ist ein Stück\pwindex{Schnitzler, Arthur 15.\,5.\,1862 Wien – 21.\,10.\,1931 ebd.@\textsc{Schnitzler, Arthur} (15.\,5.\,1862 Wien – 21.\,10.\,1931 ebd.), \emph{Schriftsteller, Mediziner}!Märchen. Schauspiel in drei Aufzügen@\strich\emph{Das Märchen. Schauspiel in drei Aufzügen}|pwv} aus \uline{unserm} Leben und aus dem Leben der \uline{Zukunft}. Ungemein fein haben Sie die Frauenfrage gestreift. Ich \uline{sah} beim Lesen alle Ihre Menschen ganz leibhaftig vor
               mir. Und ich hoffe sehr, dass das Märchen\pwindex{Schnitzler, Arthur 15.\,5.\,1862 Wien – 21.\,10.\,1931 ebd.@\textsc{Schnitzler, Arthur} (15.\,5.\,1862 Wien – 21.\,10.\,1931 ebd.), \emph{Schriftsteller, Mediziner}!Märchen. Schauspiel in drei Aufzügen@\strich\emph{Das Märchen. Schauspiel in drei Aufzügen}|pw} nicht
               nur die Freien Bühnen beschäftigen wird, sondern erst recht unsere grossen Theater,
               wenn diesen noch ein letzter Ernst geblieben ist.\pend
           
\pstart
           Ihr hochachtungsvoll ergebener{\\[\baselineskip]}\spacefill\mbox{Baron Detlev Liliencron.}\pend
           \leftskip=0em{}\selectlanguage{ngerman}\endnumbering\briefempfaengerindex{Schnitzler, Arthur@\textsc{Schnitzler, Arthur}!zzzLiliencron, Detlev von@\emph{von Detlev von Liliencron}!1894-05-071@{7. 5. 1894}|)be}\mylabel{L00321h}  \newcommand{\dateiname}{L00321}\newcommand{\titel}{Detlev von Liliencron an Arthur Schnitzler, 7. 5. 1894}\newcommand{\editorInnen}{Martin Anton Müller und Gerd-Hermann Susen}%% latex-leseansicht-abspann.tex
%% Abspann für die Leseansicht.
%% Der Schalter \ifkorrekturansicht ist bereits durch den Vorspann gesetzt.

%% latex-abspann.tex
%% Gemeinsamer Abspann für Korrekturansicht und Leseansicht.
%% Setzt den Schalter \ifkorrekturansicht voraus (gesetzt in den
%% einbindenden Dateien latex-korrekturansicht-abspann.tex bzw.
%% latex-leseansicht-abspann.tex).
%% ---------------------------------------------------------------

\normalsize

% Das esempio-Environment wird nur in der Leseansicht benötigt
\ifkorrekturansicht\else
\newenvironment{esempio}[3]%
{
    \vspace{1.5ex}
    \rlap{\underline{#1}}
    \par
    \setlength{\parindent}{0cm}
    \nopagebreak
    \leftskip=#2cm
    \rightskip=#3cm
}
{
    \par
}
\fi

\doendnotes{C}
\bigskip
\vfill

\clearpage

\footnotesize

\ifkorrekturansicht
  \lohead{\textsc{register}}
\fi

% theindex-Environment neu definieren ohne reledmac
\makeatletter
\renewenvironment{theindex}{%
  \ifkorrekturansicht
    \section*{\indexname}%
  \else
    \subsubsection*{Index der erwähnten Entitäten}%
  \fi
  \setlength{\parindent}{0pt}%
  \setlength{\parskip}{0pt plus 0.3pt}%
  \let\item\@idxitem
}{%
  \ifkorrekturansicht\clearpage\fi
}
\makeatother

\IfFileExists{\jobname-pw.ind}{\input{\jobname-pw.ind}}{}

% Quellenangabe nur in der Leseansicht
\ifkorrekturansicht\else
% Fallback-Definitionen, falls die .tex-Datei \titel etc. nicht gesetzt hat
\providecommand{\titel}{}
\providecommand{\editorInnen}{}
\providecommand{\dateiname}{\jobname}

\vspace{3cm}

\vfill

\footnotesize
\textsc{Quelle}: \titel. Herausgegeben von {\editorInnen}. In: \emph{Arthur Schnitzler: Briefwechsel mit Autorinnen und Autoren}.
 Digitale Edition, https://schnitzler-briefe.acdh.oeaw.ac.at/{\dateiname}.html (Stand \today)
\fi

\end{document}


