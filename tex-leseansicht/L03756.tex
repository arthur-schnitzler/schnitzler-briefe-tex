%% latex-leseansicht-vorspann.tex
%% Vorspann für die Leseansicht.
%% Lädt die gemeinsame Datei latex-vorspann.tex mit nicht gesetztem Schalter.

\newif\ifkorrekturansicht
\korrekturansichtfalse

\input{../tex-inputs/latex-vorspann}


\section[Arthur Schnitzler an Stefan Zweig, 29. 11. 1922]{L03756 Arthur Schnitzler an Stefan Zweig, 29. 11. 1922}
\nopagebreak\mylabel{L03756v}
\rehead{ }\normalsize\beginnumbering\briefempfaengerindex{Zweig, Stefan@\textsc{Zweig, Stefan}!zzzSchnitzler, Arthur@\emph{von Arthur Schnitzler}!1922-11-291@{29. 11. 1922}|(be}
\toendnotes[C]{\smallbreak\pagebreak[2]}
\correspDesc{Versand  durch Arthur Schnitzler am 29. 11. 1922 in Wien
\newline{}Erhalt  durch Stefan Zweig im Zeitraum [30. 11. 1922 – 4. 12. 1922?] in Salzburg}\toendnotes[C]{\smallbreak}
\Standort{Jerusalem, National Library of Israel, ARC. Ms. Var. 305 1 58 Stefan Zweig Collection.}
\physDesc{Postkarte, 501 Zeichen
\newline{}Handschrift: Bleistift, lateinische Kurrent
\newline{}Versand: Stempel: »\nobreak{}\oindex{XVIII., Währing@\textbf{XVIII., Währing}, \emph{Verwaltungsgebiet}|pwk} 1\textcolor{gray}{8}/1 Wien
                                       110, 29. XI. 22\nobreak{}«.  
\newline{}Ordnung: mit Bleistift Vermerk: »Schnitzler« }\toendnotes[C]{\smallbreak}\pstart{}{\pb}Herrn Dr.\pend{}\pstart{}Stefan Zweig.\pend{}\pstart{}Salzburg\oindex{Salzburg@\textbf{Salzburg}, \emph{Verwaltungsgebiet}|pw}\pend{}\pstart{}Kapuzinerberg 5\oindex{Paschinger Schlössl@\textbf{Paschinger Schlössl}, \emph{Wohngebäude}|pw}\pend{}{\bigskip}\vspace{1em}
\pstart{}{\pb}lieber und verehrter Herr Doctor,\pend\vspace{0.5em}
\pstart
           vielen Dank für die liebenswürdige Übersendg Ihrer Novellen\pwindex{Zweig, Stefan 28.\,11.\,1881 Wien – 23.\,2.\,1942 Petrópolis@\textsc{Zweig, Stefan} (28.\,11.\,1881 Wien – 23.\,2.\,1942 Petrópolis), \emph{Schriftsteller}!Amok. Novellen einer Leidenschaft@\strich\emph{Amok. Novellen einer Leidenschaft}|pwv}, mit denen eine werthvolle Beka{\geminationn}tschaft zu erneuern ich mich herzlich freue. Hab ich
               Ihnen das schon gesagt, wie besonders schön ich Ihren \label{K_L03756-1v}\edtext{Geburtstagsartikel\pwindex{Arthur Schnitzler. Zu seinem sechzigsten Geburtstag (15. Mai 1922)@\emph{Arthur Schnitzler. Zu seinem sechzigsten Geburtstag (15. Mai 1922)}|pwv}}{\lemma{\textnormal{\emph{Geburtstagsartikel}}}\Cendnote{\textnormal{Raoul Auernheimer\pwindex{Auernheimer, Raoul 15.\,4.\,1876 Wien – 6.\,1.\,1948 Oakland@\textsc{Auernheimer, Raoul} (15.\,4.\,1876 Wien – 6.\,1.\,1948 Oakland), \emph{Schriftsteller, Journalist, Kritiker}|pwk}, Hermann Bahr\pwindex{Bahr, Hermann 19.\,7.\,1863 Linz – 15.\,1.\,1934 München@\textsc{Bahr, Hermann} (19.\,7.\,1863 Linz – 15.\,1.\,1934 München), \emph{Schriftsteller, Kritiker}|pwk}, Franz
                        Blei\pwindex{Blei, Franz 18.\,1.\,1871 Wien – 10.\,7.\,1942 Westbury@\textsc{Blei, Franz} (18.\,1.\,1871 Wien – 10.\,7.\,1942 Westbury), \emph{Schriftsteller}|pwk}, Samuel Fischer\pwindex{Fischer, Samuel 24.\,12.\,1859 Liptovský Mikuláš – 15.\,10.\,1934 Berlin@\textsc{Fischer, Samuel} (24.\,12.\,1859 Liptovský Mikuláš – 15.\,10.\,1934 Berlin), \emph{Verleger}|pwk}, Otto Flake\pwindex{Flake, Otto 29.\,10.\,1880 Metz – 10.\,11.\,1963 Baden-Baden@\textsc{Flake, Otto} (29.\,10.\,1880 Metz – 10.\,11.\,1963 Baden-Baden), \emph{Schriftsteller}|pwk}, Egon Friedell\pwindex{Friedell, Egon 21.\,1.\,1878 Wien – 16.\,3.\,1938 ebd.@\textsc{Friedell, Egon} (21.\,1.\,1878 Wien – 16.\,3.\,1938 ebd.), \emph{Schriftsteller, Journalist, Kulturphilosoph}|pwk}, Gerhart
                        Hauptmann\pwindex{Hauptmann, Gerhart 15.\,11.\,1862 Szczawno-Zdrój – 6.\,6.\,1946 Jagniątków@\textsc{Hauptmann, Gerhart} (15.\,11.\,1862 Szczawno-Zdrój – 6.\,6.\,1946 Jagniątków), \emph{Schriftsteller}|pwk}, Hugo von
                     Hofmannsthal\pwindex{Hofmannsthal, Hugo von 1.\,2.\,1874 Wien – 15.\,7.\,1929 Rodaun@\textsc{Hofmannsthal, Hugo von} (1.\,2.\,1874 Wien – 15.\,7.\,1929 Rodaun), \emph{Schriftsteller}|pwk}, Felix Hollaender\pwindex{Hollaender, Felix 1.\,11.\,1867 Głubczyce – 29.\,5.\,1931 Berlin@\textsc{Hollaender, Felix} (1.\,11.\,1867 Głubczyce – 29.\,5.\,1931 Berlin), \emph{Schriftsteller, Theaterleiter, Regisseur}|pwk}, Alfred Kerr\pwindex{Kerr, Alfred 25.\,12.\,1867 Breslau – 12.\,10.\,1948 Hamburg@\textsc{Kerr, Alfred} (25.\,12.\,1867 Breslau – 12.\,10.\,1948 Hamburg), \emph{Schriftsteller, Kritiker}|pwk}, Heinrich Mann\pwindex{Mann, Heinrich 27.\,3.\,1871 Lübeck – 11.\,3.\,1950 Santa Monica@\textsc{Mann, Heinrich} (27.\,3.\,1871 Lübeck – 11.\,3.\,1950 Santa Monica), \emph{Schriftsteller}|pwk}, Thomas
                        Mann\pwindex{Mann, Thomas 6.\,6.\,1875 Lübeck – 12.\,8.\,1955 Zürich@\textsc{Mann, Thomas} (6.\,6.\,1875 Lübeck – 12.\,8.\,1955 Zürich), \emph{Schriftsteller}|pwk}, Jakob Wassermann\pwindex{Wassermann, Jakob 10.\,3.\,1873 Fürth – 1.\,1.\,1934 Altaussee@\textsc{Wassermann, Jakob} (10.\,3.\,1873 Fürth – 1.\,1.\,1934 Altaussee), \emph{Schriftsteller}|pwk}, Franz Werfel\pwindex{Werfel, Franz 10.\,9.\,1890 Prag – 26.\,8.\,1945 Beverly Hills@\textsc{Werfel, Franz} (10.\,9.\,1890 Prag – 26.\,8.\,1945 Beverly Hills), \emph{Schriftsteller}|pwk}, Stefan Zweig\pwindex{Zweig, Stefan 28.\,11.\,1881 Wien – 23.\,2.\,1942 Petrópolis@\textsc{Zweig, Stefan} (28.\,11.\,1881 Wien – 23.\,2.\,1942 Petrópolis), \emph{Schriftsteller}|pwk}: \emph{Arthur
                        Schnitzler. Zu seinem sechzigsten Geburtstag (15. Mai 1922)}\pwindex{Arthur Schnitzler. Zu seinem sechzigsten Geburtstag (15. Mai 1922)@\emph{Arthur Schnitzler. Zu seinem sechzigsten Geburtstag (15. Mai 1922)}|pwk}. In: \emph{Die neue Rundschau}\pwindex{neue Rundschau@\emph{Die neue Rundschau}|pwk}, Jg. 33, Nr. 5,
                        1. 5. 1922, S. 498–513. }}}\label{K_L03756-1} in der N. R.\pwindex{neue Rundschau@\emph{Die neue Rundschau}|pw} gefunden habe? Jetzt les ich Ihr ausge{\pb}zeichnetes Desbord-Valmore\pwindex{Desbordes-Valmore, Marceline 20.\,6.\,1786 Douai – 23.\,7.\,1859 Paris@\textsc{Desbordes-Valmore, Marceline} (20.\,6.\,1786 Douai – 23.\,7.\,1859 Paris), \emph{Schauspielerin, Sängerin, Schriftstellerin}|pw}-Buch\pwindex{Zweig, Stefan 28.\,11.\,1881 Wien – 23.\,2.\,1942 Petrópolis@\textsc{Zweig, Stefan} (28.\,11.\,1881 Wien – 23.\,2.\,1942 Petrópolis), \emph{Schriftsteller}!Marceline Desbordes-Valmore. Das Lebensbild einer Dichterin@\strich\emph{Marceline Desbordes-Valmore. Das Lebensbild einer Dichterin}|pwv}, das mir ein
               dafür schwärmender Vetter\pwindex{?? [Vetter von Schnitzler, der ihm Buch schenkt] @\textsc{?? [Vetter von Schnitzler, der ihm Buch schenkt]}|pwv}
               zum Geschenk gemacht hat. Auf Wiedersehen!\pend
           \pstart Mit vielen Grüßen Ihr sehr ergebener \spacefill\mbox{ArthSchnitzler}\pend{}\selectlanguage{ngerman}\endnumbering\briefempfaengerindex{Zweig, Stefan@\textsc{Zweig, Stefan}!zzzSchnitzler, Arthur@\emph{von Arthur Schnitzler}!1922-11-291@{29. 11. 1922}|)be}\mylabel{L03756h}  \newcommand{\dateiname}{L03756}\newcommand{\titel}{Arthur Schnitzler an Stefan Zweig, 29. 11. 1922}\newcommand{\editorInnen}{Selma Jahnke und Martin Anton Müller}%% latex-leseansicht-abspann.tex
%% Abspann für die Leseansicht.
%% Der Schalter \ifkorrekturansicht ist bereits durch den Vorspann gesetzt.

%% latex-abspann.tex
%% Gemeinsamer Abspann für Korrekturansicht und Leseansicht.
%% Setzt den Schalter \ifkorrekturansicht voraus (gesetzt in den
%% einbindenden Dateien latex-korrekturansicht-abspann.tex bzw.
%% latex-leseansicht-abspann.tex).
%% ---------------------------------------------------------------

\normalsize

% Das esempio-Environment wird nur in der Leseansicht benötigt
\ifkorrekturansicht\else
\newenvironment{esempio}[3]%
{
    \vspace{1.5ex}
    \rlap{\underline{#1}}
    \par
    \setlength{\parindent}{0cm}
    \nopagebreak
    \leftskip=#2cm
    \rightskip=#3cm
}
{
    \par
}
\fi

\doendnotes{C}
\bigskip
\vfill

\clearpage

\footnotesize

\ifkorrekturansicht
  \lohead{\textsc{register}}
\fi

% theindex-Environment neu definieren ohne reledmac
\makeatletter
\renewenvironment{theindex}{%
  \ifkorrekturansicht
    \section*{\indexname}%
  \else
    \subsubsection*{Index der erwähnten Entitäten}%
  \fi
  \setlength{\parindent}{0pt}%
  \setlength{\parskip}{0pt plus 0.3pt}%
  \let\item\@idxitem
}{%
  \ifkorrekturansicht\clearpage\fi
}
\makeatother

\IfFileExists{\jobname-pw.ind}{\input{\jobname-pw.ind}}{}

% Quellenangabe nur in der Leseansicht
\ifkorrekturansicht\else
% Fallback-Definitionen, falls die .tex-Datei \titel etc. nicht gesetzt hat
\providecommand{\titel}{}
\providecommand{\editorInnen}{}
\providecommand{\dateiname}{\jobname}

\vspace{3cm}

\vfill

\footnotesize
\textsc{Quelle}: \titel. Herausgegeben von {\editorInnen}. In: \emph{Arthur Schnitzler: Briefwechsel mit Autorinnen und Autoren}.
 Digitale Edition, https://schnitzler-briefe.acdh.oeaw.ac.at/{\dateiname}.html (Stand \today)
\fi

\end{document}


