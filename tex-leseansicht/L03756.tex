%% latex-korrekturansicht-vorspann.tex
%% Vorspann für die Korrekturansicht.
%% Lädt die gemeinsame Datei latex-vorspann.tex mit gesetztem Schalter.

\newif\ifkorrekturansicht
\korrekturansichttrue

\input{../tex-inputs/latex-vorspann}


\section[Arthur Schnitzler an Stefan Zweig, 29. 11. 1922]{L03756 Arthur Schnitzler an Stefan Zweig, 29. 11. 1922}
\nopagebreak\mylabel{L03756v}
\rehead{ }\normalsize\beginnumbering\briefempfaengerindex{Zweig, Stefan@\textsc{Zweig, Stefan}!zzzSchnitzler, Arthur@\emph{von Arthur Schnitzler}!1922-11-291@{29. 11. 1922}|(be}
\toendnotes[C]{\smallbreak\pagebreak[2]}\Standort{Jerusalem, National Library of Israel, ARC. Ms. Var. 305 1 58 Stefan Zweig Collection.}
\physDesc{Postkarte, 1 Blatt, 2 Seiten, 502 Zeichen
\newline{}Handschrift: Bleistift, lateinische Kurrent
\newline{}Versand: Stempel: »\nobreak{}\oindex{XVIII., Waehring@\textbf{XVIII., Währing}, \emph{A.ADM3}|pwk}\textcolor{gray}{18}/\textcolor{gray}{1} Wien 110, 29. XI. 22, \textcolor{gray}{3}\nobreak{}«.  
\newline{}Zweig: mit Bleistift beschriftet: »Schnitzler« }\toendnotes[C]{\smallbreak}\pstart{}{\pb}Herrn Dr.\pend{}\pstart{}Stefan Zweig.\pend{}\pstart{}Salzburg\oindex{Salzburg@\textbf{Salzburg}, \emph{A.ADM2}|pw}\pend{}\pstart{}Kapuzinerberg 5\oindex{Paschinger Schloessl@\textbf{Paschinger Schlössl}, \emph{Wohngebäude (K.WHS)}|pw}\pend{}{\bigskip}\vspace{1em}
\pstart{}{\pb}lieber und verehrter Herr Doctor,\pend\vspace{0.5em}
\pstart
           vielen Dank für die liebenswürdige Übersendg Ihrer Novellen\pwindex{Amok@\emph{Amok}|pwv}, mit denen eine wertvolle Beka{\geminationn}tschaft zu
               erneuern ich mich herzlich freue. Hab ich Ihnen das schon gesagt, wie besonders schön
               ich Ihren \label{K_L03756-1v}\edtext{Geburtstagsartikel\pwindex{Arthur Schnitzler. Zu seinem sechzigsten Geburtstag (15. Mai 1922)@\emph{Arthur Schnitzler. Zu seinem sechzigsten Geburtstag (15. Mai 1922)}|pwv}}{\lemma{\textnormal{\emph{Geburtstagsartikel}}}\Cendnote{\textnormal{Raoul Auernheimer\pwindex{Auernheimer, Raoul 15.04.1876 – 06.01.1948@\textsc{Auernheimer, Raoul} (15.04.1876 – 06.01.1948), \emph{Schriftsteller/Schriftstellerin, Journalist/Journalistin, Kritiker/Kritikerin}|pwk}, Hermann Bahr\pwindex{Bahr, Hermann 19.07.1863 – 15.01.1934@\textsc{Bahr, Hermann} (19.07.1863 – 15.01.1934), \emph{Schriftsteller/Schriftstellerin, Kritiker/Kritikerin}|pwk}, Franz
                        Blei\pwindex{Blei, Franz 18.01.1871 – 10.07.1942@\textsc{Blei, Franz} (18.01.1871 – 10.07.1942), \emph{Schriftsteller/Schriftstellerin}|pwk}, Samuel Fischer\pwindex{Fischer, Samuel 24.12.1859 – 15.10.1934@\textsc{Fischer, Samuel} (24.12.1859 – 15.10.1934), \emph{Verleger/Verlegerin}|pwk}, Otto Flake\pwindex{Flake, Otto 29.10.1880 – 10.11.1963@\textsc{Flake, Otto} (29.10.1880 – 10.11.1963), \emph{Schriftsteller/Schriftstellerin}|pwk}, Egon Friedell\pwindex{Friedell, Egon 21.01.1878 – 16.03.1938@\textsc{Friedell, Egon} (21.01.1878 – 16.03.1938), \emph{Schriftsteller/Schriftstellerin, Journalist/Journalistin, Kulturphilosoph/Kulturphilosophin}|pwk}, Gerhart
                        Hauptmann\pwindex{Hauptmann, Gerhart 15.11.1862 – 06.06.1946@\textsc{Hauptmann, Gerhart} (15.11.1862 – 06.06.1946), \emph{Schriftsteller/Schriftstellerin}|pwk}, Hugo von
                     Hofmannsthal\pwindex{Hofmannsthal, Hugo von 1874-02-01 – 1929-07-15@\textsc{Hofmannsthal, Hugo von} (1874-02-01 – 1929-07-15), \emph{Schriftsteller/Schriftstellerin}|pwk}, Felix Hollaender\pwindex{Hollaender, Felix 01.11.1867 – 29.05.1931@\textsc{Hollaender, Felix} (01.11.1867 – 29.05.1931), \emph{Schriftsteller/Schriftstellerin, Theaterleiter/Theaterleiterin, Regisseur/Regisseurin}|pwk}, Alfred Kerr\pwindex{Kerr, Alfred 25.12.1867 – 12.10.1948@\textsc{Kerr, Alfred} (25.12.1867 – 12.10.1948), \emph{Schriftsteller/Schriftstellerin, Kritiker/Kritikerin}|pwk}, Heinrich Mann\pwindex{Mann, Heinrich 27.03.1871 – 11.03.1950@\textsc{Mann, Heinrich} (27.03.1871 – 11.03.1950), \emph{Schriftsteller/Schriftstellerin}|pwk}, Thomas
                        Mann\pwindex{Mann, Thomas 06.06.1875 – 12.08.1955@\textsc{Mann, Thomas} (06.06.1875 – 12.08.1955), \emph{Schriftsteller/Schriftstellerin}|pwk}, Jakob Wassermann\pwindex{Wassermann, Jakob 10.03.1873 – 01.01.1934@\textsc{Wassermann, Jakob} (10.03.1873 – 01.01.1934), \emph{Schriftsteller/Schriftstellerin}|pwk}, Franz Werfel\pwindex{Werfel, Franz 10.09.1890 – 26.08.1945@\textsc{Werfel, Franz} (10.09.1890 – 26.08.1945), \emph{Schriftsteller/Schriftstellerin}|pwk}, Stefan Zweig\pwindex{Zweig, Stefan 28.11.1881 – 23.02.1942@\textsc{Zweig, Stefan} (28.11.1881 – 23.02.1942), \emph{Schriftsteller/Schriftstellerin}|pwk}: \emph{Arthur
                        Schnitzler. Zu seinem sechzigsten Geburtstag (15. Mai 1922)}\pwindex{Arthur Schnitzler. Zu seinem sechzigsten Geburtstag (15. Mai 1922)@\emph{Arthur Schnitzler. Zu seinem sechzigsten Geburtstag (15. Mai 1922)}|pwk}. In: \emph{Die neue Rundschau}\pwindex{neue Rundschau@\emph{Die neue Rundschau}|pwk}, Jg. 33,
                     Nr. 5, 1. 5. 1922, S. 498–513. }}}\label{K_L03756-1}
               in der n. R.\pwindex{neue Rundschau@\emph{Die neue Rundschau}|pw} gefunden habe? Jetzt les ich Ihr
                  ausge{\pb}zeichnetes Desbord-Valmore\pwindex{Desbordes-Valmore, Marceline 1786-06-20 – 1859-07-23@\textsc{Desbordes-Valmore, Marceline} (1786-06-20 – 1859-07-23), \emph{Schauspieler/Schauspielerin, Sänger/Sängerin, Schriftsteller/Schriftstellerin}|pw}
                  Buch\pwindex{Marceline Desbordes-Valmore. Das Lebensbild einer Dichterin@\emph{Marceline Desbordes-Valmore. Das Lebensbild einer Dichterin}|pwv}, das mir ein dafür schwärmender Vetter\pwindex{?? [Vetter von Schnitzler, der ihm Buch schenkt] @\textsc{?? [Vetter von Schnitzler, der ihm Buch schenkt]}|pwv} zum Geschenk gemacht hat. Auf Wiedersehen!\pend
           \pstart Mit vielen Grüßen Ihr sehr ergebener \spacefill\mbox{Arth Schnitzler}\pend{}\selectlanguage{ngerman}\endnumbering\briefempfaengerindex{Zweig, Stefan@\textsc{Zweig, Stefan}!zzzSchnitzler, Arthur@\emph{von Arthur Schnitzler}!1922-11-291@{29. 11. 1922}|)be}\mylabel{L03756h}
\begin{anhang}
\end{anhang}\normalsize

\doendnotes{C}
\bigskip
\vfill

\clearpage

\footnotesize

\lohead{\textsc{register}}

% Definiere theindex-Environment komplett neu ohne reledmac
\makeatletter
\renewenvironment{theindex}{%
  \section*{\indexname}%
  \setlength{\parindent}{0pt}%
  \setlength{\parskip}{0pt plus 0.3pt}%
  \let\item\@idxitem
}{%
  \clearpage
}
\makeatother

\IfFileExists{\jobname-pw.ind}{\input{\jobname-pw.ind}}{}

\end{document}

      