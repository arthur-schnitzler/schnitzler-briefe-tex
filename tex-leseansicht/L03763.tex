%% latex-korrekturansicht-vorspann.tex
%% Vorspann für die Korrekturansicht.
%% Lädt die gemeinsame Datei latex-vorspann.tex mit gesetztem Schalter.

\newif\ifkorrekturansicht
\korrekturansichttrue

\input{../tex-inputs/latex-vorspann}


\section[Arthur Schnitzler an Stefan Zweig, 20. 8. 1920]{L03763 Arthur Schnitzler an Stefan Zweig, 20. 8. 1920}
\nopagebreak\mylabel{L03763v}
\rehead{ }\normalsize\beginnumbering\briefempfaengerindex{Zweig, Stefan@\textsc{Zweig, Stefan}!zzzSchnitzler, Arthur@\emph{von Arthur Schnitzler}!1920-08-201@{20. 8. 1920}|(be}
\toendnotes[C]{\smallbreak\pagebreak[2]}\Standort{Jerusalem, National Library of Israel, ARC. Ms. Var. 305 1 58 Stefan Zweig Collection.}
\physDesc{Brief, 1 Blatt, 1 Seite, 689 Zeichen
\newline{}Schreibmaschine
\newline{}Handschrift: schwarze Tinte (\noindent{}Unterschrift)}\toendnotes[C]{\smallbreak}
\pstart
           {\pb}\textcolor{gray}{\textbf{D\textsuperscript{R} ARTHUR SCHNITZLER}}\hfill 20. 8. 1920. \pend
           
\pstart
           \textcolor{gray}{\textbf{WIEN, XVIII.
                              STERNWARTESTRASSE 71\oindex{Sternwartestrasse 71@\textbf{Sternwartestraße 71}, \emph{Wohngebäude (K.WHS)}|pw}.}}\pend
           
\pstart{}Lieber Herr Dr. Zweig.\pend\vspace{0.5em}
\pstart
           Vielen Dank für Ihren \label{K_L03763-1v}\edtext{Brief}{\lemma{\textnormal{\emph{Brief}}}\Cendnote{\textnormal{Stefan Zweig an Arthur Schnitzler, 18. 8. 1920.}}}\label{K_L03763-1} und für Ihr \label{K_L03763-2v}\edtext{Telegramm}{\lemma{\textnormal{\emph{Telegramm}}}\Cendnote{\textnormal{Das Telegramm ist nicht erhalten, vgl. Stefan Zweig an Arthur Schnitzler, 18. 8. 1920.}}}\label{K_L03763-2}, das um einen Tag später ankam als
               Ihr Brief. Zu den 10{\%} habe ich mich auch entschlossen. Mit
               dem Vorschuss bin ich etwas höher gegangen. Ich glaube, wir sollten nicht immer
               umrechnen. Hundert Dollars sind doch nicht mehr als fünfhundert Kronen, nicht
               zwanzigtausend, wie uns die Amerikaner\oindex{Vereinigte Staaten von Amerika [USA]@\textbf{Vereinigte Staaten von Amerika [USA]}, \emph{A.PCLI}|pw} jetzt
               einreden wollen. Und ich stelle meine Honorarforderungen, wenn irgend möglich, von
               diesem Standpunkt aus. Dass ich damit bisher immer reussiert hätte, will ich
               allerdings nicht behaupten. \pend
           
\pstart
           Auf baldiges Wiedersehen entweder in Salzburg\oindex{Salzburg@\textbf{Salzburg}, \emph{A.ADM2}|pw} oder
               in Wien\oindex{Wien@\textbf{Wien}, \emph{A.ADM2}|pw}. \pend
           
\pstart
           Seien Sie herzlichst gegrüsst von Ihrem sehr ergebenen{\\[\baselineskip]}\spacefill\mbox{{[}hs.:{]} Arthur Schnitzler}\pend
           \leftskip=0em{}\selectlanguage{ngerman}\endnumbering\briefempfaengerindex{Zweig, Stefan@\textsc{Zweig, Stefan}!zzzSchnitzler, Arthur@\emph{von Arthur Schnitzler}!1920-08-201@{20. 8. 1920}|)be}\mylabel{L03763h}
\begin{anhang}
\end{anhang}\normalsize

\doendnotes{C}
\bigskip
\vfill

\clearpage

\footnotesize

\lohead{\textsc{register}}

% Definiere theindex-Environment komplett neu ohne reledmac
\makeatletter
\renewenvironment{theindex}{%
  \section*{\indexname}%
  \setlength{\parindent}{0pt}%
  \setlength{\parskip}{0pt plus 0.3pt}%
  \let\item\@idxitem
}{%
  \clearpage
}
\makeatother

\IfFileExists{\jobname-pw.ind}{\input{\jobname-pw.ind}}{}

\end{document}

      