%% latex-korrekturansicht-vorspann.tex
%% Vorspann für die Korrekturansicht.
%% Lädt die gemeinsame Datei latex-vorspann.tex mit gesetztem Schalter.

\newif\ifkorrekturansicht
\korrekturansichttrue

\input{../tex-inputs/latex-vorspann}


\section[Richard Beer-Hofmann an Arthur Schnitzler, {[}16. 12. 1891{]}]{L00054 Richard Beer-Hofmann an Arthur Schnitzler, {[}16. 12. 1891{]}}
\nopagebreak\mylabel{L00054v}
\rehead{ }\normalsize\beginnumbering\briefempfaengerindex{Schnitzler, Arthur@\textsc{Schnitzler, Arthur}!zzzBeer-Hofmann, Richard@\emph{von Richard Beer-Hofmann}!1891-12-161@{{[}16. 12. 1891{]}}|(be}
\toendnotes[C]{\smallbreak\pagebreak[2]}\Standort{CUL, Schnitzler, B 43.}
\physDesc{Brief, 1 Blatt, 2 Seiten, 251 Zeichen
\newline{}Handschrift: schwarze Tinte, lateinische Kurrent
\newline{}Schnitzler: mit Bleistift datiert: »16/12 91« und nummeriert: »5« }
\buchAbdrucke{\weitereDrucke{1) Arthur Schnitzler, Richard Beer-Hofmann: \emph{Briefwechsel 1891–1931}. Wien, Zürich: \emph{Europaverlag} 1992, S. 32.} \weitereDrucke{2) Hermann Bahr, Arthur Schnitzler: \emph{Briefwechsel, Aufzeichnungen, Dokumente (1891–1931)}. Göttingen: \emph{Wallstein} 2018, S. 16.} }
\pstart
           \raggedleft{}{\pb}Im Caffée\pend
           
\pstart\center{}Lieber Arthur!\pend\vspace{0.5em}
\pstart
           Hermann Bahr\pwindex{Bahr, Hermann 19.07.1863 – 15.01.1934@\textsc{Bahr, Hermann} (19.07.1863 – 15.01.1934), \emph{Schriftsteller/Schriftstellerin, Kritiker/Kritikerin}|pw} erzählt mir soeben: Er hat Brief
               von Reicher\pwindex{Reicher, Emanuel 18.06.1849 – 15.05.1924@\textsc{Reicher, Emanuel} (18.06.1849 – 15.05.1924), \emph{Schauspieler/Schauspielerin}|pw}, das Märchen\pwindex{Maerchen. Schauspiel in drei Aufzuegen@\emph{Das Märchen. Schauspiel in drei Aufzügen}|pw} ist am Lessingtheater\orgindex{Lessing-Theater@Lessing-Theater|pw}
                  angeno{\geminationm}en; Blumenthal\pwindex{Blumenthal, Oskar 13.03.1852 – 24.04.1917@\textsc{Blumenthal, Oskar} (13.03.1852 – 24.04.1917), \emph{Schriftsteller/Schriftstellerin, Journalist/Journalistin, Theaterleiter/Theaterleiterin}|pw} ist entzückt \strikeout{und} wird ihnen
                  \introOben{}aber\introOben{} eine Reihe von »unbedeutenden« (?)
               Aenderungen vorschlagen.\pend
           
\pstart
           {\pb}Es grüßt sie von
                  Herzen\hspace*{2em}Ihr{\\[\baselineskip]}\spacefill\mbox{Richard}\pend
           \leftskip=0em{}\selectlanguage{ngerman}\endnumbering\briefempfaengerindex{Schnitzler, Arthur@\textsc{Schnitzler, Arthur}!zzzBeer-Hofmann, Richard@\emph{von Richard Beer-Hofmann}!1891-12-161@{{[}16. 12. 1891{]}}|)be}\mylabel{L00054h}  \normalsize

\doendnotes{C}
\bigskip
\vfill

\clearpage

\footnotesize

\lohead{\textsc{register}}

% Definiere theindex-Environment komplett neu ohne reledmac
\makeatletter
\renewenvironment{theindex}{%
  \section*{\indexname}%
  \setlength{\parindent}{0pt}%
  \setlength{\parskip}{0pt plus 0.3pt}%
  \let\item\@idxitem
}{%
  \clearpage
}
\makeatother

\IfFileExists{\jobname-pw.ind}{\input{\jobname-pw.ind}}{}

\end{document}

      