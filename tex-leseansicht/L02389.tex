%% latex-korrekturansicht-vorspann.tex
%% Vorspann für die Korrekturansicht.
%% Lädt die gemeinsame Datei latex-vorspann.tex mit gesetztem Schalter.

\newif\ifkorrekturansicht
\korrekturansichttrue

\input{../tex-inputs/latex-vorspann}


\section[Arthur Schnitzler an Hermann Bahr, 26. 6. 1922]{L02389 Arthur Schnitzler an Hermann Bahr, 26. 6. 1922}
\nopagebreak\mylabel{L02389v}
\rehead{ }\normalsize\beginnumbering\briefempfaengerindex{Bahr, Hermann@\textsc{Bahr, Hermann}!zzzSchnitzler, Arthur@\emph{von Arthur Schnitzler}!1922-06-261@{26. 6. 192{[}2{]}}|(be}
\toendnotes[C]{\smallbreak\pagebreak[2]}\Standort{TMW, HS AM 60133 Ba.}
\physDesc{Briefkarte, 379 Zeichen
\newline{}Handschrift: schwarze Tinte, lateinische Kurrent}
\buchAbdrucke{\weitereDrucke{1) Arthur Schnitzler: \emph{The Letters of Arthur Schnitzler to Hermann Bahr}. Chapel Hill: \emph{The University of North Carolina Press} 1978, S. 115.} \weitereDrucke{2) Hermann Bahr, Arthur Schnitzler: \emph{Briefwechsel, Aufzeichnungen, Dokumente (1891–1931)}. Göttingen: \emph{Wallstein} 2018, S. 562.} }
\pstart
           \raggedleft{}{\pb}Wien\oindex{Wien@\textbf{Wien}, \emph{A.ADM2}|pw}{ }26. 6. 22\pend
           \vspace{0.5em}
\pstart
           lieber Hermann, darf ich dir Mr Scofield Thayer\pwindex{Thayer, Scofield 12.12.1889 – 09.07.1982@\textsc{Thayer, Scofield} (12.12.1889 – 09.07.1982), \emph{Journalist/Journalistin, Herausgeber/Herausgeberin}|pw} vorstellen, Herausgeber der Dial\orgindex{Dial@The Dial|pw}, einer der charmantesten und anregendsten jungen Amerikaner\oindex{Amerika@\textbf{Amerika}, \emph{kein passender Code gefunden}|pw}, die mir begegnet sind! Ich sage nicht mehr, denn ich
               hoffe, du wirst dir die Zeit nehmen Mr Thayer\pwindex{Thayer, Scofield 12.12.1889 – 09.07.1982@\textsc{Thayer, Scofield} (12.12.1889 – 09.07.1982), \emph{Journalist/Journalistin, Herausgeber/Herausgeberin}|pw}{ }{\pb}einmal zu empfangen
               und ebenso persönlich kennen lernen.\pend
           
\pstart
           Er bringt dir meine herzlichsten Grüße.\pend
           
\pstart
           Auf Wiedersehen!{\\[\baselineskip]}Dein getreuer{\\[\baselineskip]}\spacefill\mbox{Arthur}\pend
           \leftskip=0em{}\selectlanguage{ngerman}\endnumbering\briefempfaengerindex{Bahr, Hermann@\textsc{Bahr, Hermann}!zzzSchnitzler, Arthur@\emph{von Arthur Schnitzler}!1922-06-261@{26. 6. 192{[}2{]}}|)be}\mylabel{L02389h}  \normalsize

\doendnotes{C}
\bigskip
\vfill

\clearpage

\footnotesize

\lohead{\textsc{register}}

% Definiere theindex-Environment komplett neu ohne reledmac
\makeatletter
\renewenvironment{theindex}{%
  \section*{\indexname}%
  \setlength{\parindent}{0pt}%
  \setlength{\parskip}{0pt plus 0.3pt}%
  \let\item\@idxitem
}{%
  \clearpage
}
\makeatother

\IfFileExists{\jobname-pw.ind}{\input{\jobname-pw.ind}}{}

\end{document}

      