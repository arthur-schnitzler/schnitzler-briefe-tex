%% latex-leseansicht-vorspann.tex
%% Vorspann für die Leseansicht.
%% Lädt die gemeinsame Datei latex-vorspann.tex mit nicht gesetztem Schalter.

\newif\ifkorrekturansicht
\korrekturansichtfalse

\input{../tex-inputs/latex-vorspann}


\section[Arthur Schnitzler an Marie Herzfeld, 24.\,1.\,1908]{L02599 Arthur Schnitzler an Marie Herzfeld, 24.\,1.\,1908}
\nopagebreak\mylabel{L02599v}
\rehead{ }\normalsize\beginnumbering\briefempfaengerindex{Herzfeld, Marie@\textsc{Herzfeld, Marie}!zzzSchnitzler, Arthur@\emph{von Arthur Schnitzler}!1908-01-242@{24.\,1.\,1908}|(be}
\toendnotes[C]{\smallbreak\pagebreak[2]}
\correspDesc{Versand  durch Arthur Schnitzler am 24. 1. 1908 in Wien
\newline{}Erhalt  durch Marie Herzfeld im Zeitraum [24. 1. 1908
                  – 28. 1. 1908?] in Wien}\toendnotes[C]{\smallbreak}
\Standort{Wien, Privatbesitz Reinhard Urbach, \emph{ohne Signatur}.}
\physDesc{Brief, Fotokopie, 1 Blatt, 3 Seiten, 1360 Zeichen
\newline{}Handschrift: schwarze Tinte, lateinische Kurrent
\newline{}Zusatz: Das Original des Briefes ist verschollen. Evtl. könnte es sich
                                 beim Schreibmedium auch um blaue Tinte handeln. }
\buchAbdrucke{\weitereDrucke{Marie Herzfeld: \emph{Briefe an Hugo von Hofmannsthal.}. Mitgeteilt von Reinhard Urbach In: \emph{Hofmannsthal-Blätter} (1971) Nr. 6, S. 442.} }\toendnotes[C]{\smallbreak}
\pstart
           
\pstart
           {\pb}\textcolor{gray}{\textbf{Dr. Arthur Schnitzler}}\pend
           
\pstart
           \raggedleft{}24/1 908\pend
           \pend
           
\pstart
           \textcolor{gray}{\textbf{Wien XVIII. Spoettelgasse 7\oindex{Wien@\textbf{Wien}!XVIII., Währing@\textbf{XVIII., Währing}!Edmund-Weiß-Gasse 7@\textbf{Edmund-Weiß-Gasse 7}, \emph{Wohngebäude}|pw}.}}\pend
           
\pstart{}verehrtes Fräulein,\pend\vspace{0.5em}
\pstart
           ich danke Ihnen herzlich für Ihren liebenswürdgn Brief. Sie sind aber gewissenhaft!
               Es als Fehler einzubekennen, dass Sie mich nach meinem ersten Buch »verkannt« haben–!
               Dazu ist man ja geradezu verpflichtet. Ich glaube, ich habs selber auch gethan. Und
               thue es auch jetzt noch oft genug, in schlimmen Stunden (die einem in diesen schli{\geminationm}en Stunden selbst als die einsichtsvollen erscheinen.)
               Im übrigen, we{\geminationn} man die Wahl hätte zwischen verka{\geminationn}t und {\pb}»falsch
               gekannt« sein – ? Dies letztere passirt einem allerdings nach dem siebzehnten oder
               achtundzwanzigsten Buche eher als nach dem ersten. Und man erholt sich schwerer. Den
                  Stein der Weisen\pwindex{Schnitzler, Arthur 15.\,5.\,1862 Wien – 21.\,10.\,1931 ebd.@\textsc{Schnitzler, Arthur} (15.\,5.\,1862 Wien – 21.\,10.\,1931 ebd.), \emph{Schriftsteller, Mediziner}!Frau des Weisen. Novelletten@\strich\emph{Die Frau des Weisen. Novelletten}|pwv} (den Sie
               schätzen) hab ich nicht gefunden und nicht geschrieben. Sie meinen das
               Novellettenbuch »die Frau des Weisen\pwindex{Schnitzler, Arthur 15.\,5.\,1862 Wien – 21.\,10.\,1931 ebd.@\textsc{Schnitzler, Arthur} (15.\,5.\,1862 Wien – 21.\,10.\,1931 ebd.), \emph{Schriftsteller, Mediziner}!Frau des Weisen. Novelletten@\strich\emph{Die Frau des Weisen. Novelletten}|pw}«. Ich bin
               wohl vor dem Verdacht geschützt mich revanchiren zu wollen, we{\geminationn} ich Ihnen sage, verehrtes Fräulein, wie stark Ihr
                  \label{K_L02599-1v}\edtext{Leonardobuch\pwindex{Herzfeld, Marie 20.\,3.\,1855 Kőszeg – 22.\,9.\,1940 Mining@\textsc{Herzfeld, Marie} (20.\,3.\,1855 Kőszeg – 22.\,9.\,1940 Mining), \emph{Schriftstellerin, Übersetzerin}!Leonardo da Vinci. Der Denker, Forscher und Poet. Nach den veröffentlichten Handschriften@\strich\emph{Leonardo da Vinci. Der Denker, Forscher und Poet. Nach den veröffentlichten Handschriften}|pwv}}{\lemma{\textnormal{\emph{Leonardobuch}}}\Cendnote{\textnormal{\emph{Leonardo da Vinci. Der Denker, Forscher und
                        Poet}\pwindex{Herzfeld, Marie 20.\,3.\,1855 Kőszeg – 22.\,9.\,1940 Mining@\textsc{Herzfeld, Marie} (20.\,3.\,1855 Kőszeg – 22.\,9.\,1940 Mining), \emph{Schriftstellerin, Übersetzerin}!Leonardo da Vinci. Der Denker, Forscher und Poet. Nach den veröffentlichten Handschriften@\strich\emph{Leonardo da Vinci. Der Denker, Forscher und Poet. Nach den veröffentlichten Handschriften}|pwk}. Nach den veröffentlichten Handschriften. Auswahl, Übersetzung {\kaufmannsund} Einleitung von Marie Herzfeld\pwindex{Herzfeld, Marie 20.\,3.\,1855 Kőszeg – 22.\,9.\,1940 Mining@\textsc{Herzfeld, Marie} (20.\,3.\,1855 Kőszeg – 22.\,9.\,1940 Mining), \emph{Schriftstellerin, Übersetzerin}|pwk}. Jena\oindex{Jena@\textbf{Jena}, \emph{Hauptstadt}|pwk}: \emph{Eugen Diederichs Verlag}\orgindex{Eugen Diederichs Verlag@Eugen Diederichs Verlag|pwk}{ }1904.}}}\label{K_L02599-1} auf mich gewirkt hat. Ich benütze eben die Gelegenheit. Da wir
               einander leider nie begegnen, sind wir auf Gelegen{\pb}heiten angewiesen, um uns gegenseitig schmeichelhafte Dinge zu sagen. Und da Sie
               sogar meine Lyrik nicht ungelobt lassen (was ich als Originalitätshascherei auffasse)
               so müssen Sie es auch geduldg hinnehmen, dass ich mich Ihrer reizvollen \label{K_L02599-2v}\edtext{Bang\pwindex{Bang, Herman 20.\,4.\,1857 Asserballe – 29.\,1.\,1912 Ogden@\textsc{Bang, Herman} (20.\,4.\,1857 Asserballe – 29.\,1.\,1912 Ogden), \emph{Schriftsteller}|pw}{ }Silhouette\pwindex{Herzfeld, Marie 20.\,3.\,1855 Kőszeg – 22.\,9.\,1940 Mining@\textsc{Herzfeld, Marie} (20.\,3.\,1855 Kőszeg – 22.\,9.\,1940 Mining), \emph{Schriftstellerin, Übersetzerin}!Hermann Bang. Eine Silhouette@\strich\emph{Hermann Bang. Eine Silhouette}|pwv}}{\lemma{\textnormal{\emph{Bang Silhouette}}}\Cendnote{\textnormal{\emph{Hermann Bang. Eine Silhouette}\pwindex{Herzfeld, Marie 20.\,3.\,1855 Kőszeg – 22.\,9.\,1940 Mining@\textsc{Herzfeld, Marie} (20.\,3.\,1855 Kőszeg – 22.\,9.\,1940 Mining), \emph{Schriftstellerin, Übersetzerin}!Hermann Bang. Eine Silhouette@\strich\emph{Hermann Bang. Eine Silhouette}|pwk} von Marie Herzfeld\pwindex{Herzfeld, Marie 20.\,3.\,1855 Kőszeg – 22.\,9.\,1940 Mining@\textsc{Herzfeld, Marie} (20.\,3.\,1855 Kőszeg – 22.\,9.\,1940 Mining), \emph{Schriftstellerin, Übersetzerin}|pwk}. In: \emph{Neue Freie Presse}\pwindex{Neue Freie Presse@\emph{Neue Freie Presse}|pwk}, Nr. 15.590,
                     16. 1. 1908, Morgenblatt, S. 1–2.}}}\label{K_L02599-2} mit Vergnügen
               erinnere.\pend
           
\pstart
           Mit herzlichem Gruß Ihr sehr ergebener{\\[\baselineskip]}\spacefill\mbox{Arthur Schnitzler}\pend
           \leftskip=0em{}\selectlanguage{ngerman}\endnumbering\briefempfaengerindex{Herzfeld, Marie@\textsc{Herzfeld, Marie}!zzzSchnitzler, Arthur@\emph{von Arthur Schnitzler}!1908-01-242@{24.\,1.\,1908}|)be}\mylabel{L02599h}  \newcommand{\dateiname}{L02599}\newcommand{\titel}{Arthur Schnitzler an Marie Herzfeld, 24. 1. 1908}\newcommand{\editorInnen}{Martin Anton Müller und Laura Untner}%% latex-leseansicht-abspann.tex
%% Abspann für die Leseansicht.
%% Der Schalter \ifkorrekturansicht ist bereits durch den Vorspann gesetzt.

%% latex-abspann.tex
%% Gemeinsamer Abspann für Korrekturansicht und Leseansicht.
%% Setzt den Schalter \ifkorrekturansicht voraus (gesetzt in den
%% einbindenden Dateien latex-korrekturansicht-abspann.tex bzw.
%% latex-leseansicht-abspann.tex).
%% ---------------------------------------------------------------

\normalsize

% Das esempio-Environment wird nur in der Leseansicht benötigt
\ifkorrekturansicht\else
\newenvironment{esempio}[3]%
{
    \vspace{1.5ex}
    \rlap{\underline{#1}}
    \par
    \setlength{\parindent}{0cm}
    \nopagebreak
    \leftskip=#2cm
    \rightskip=#3cm
}
{
    \par
}
\fi

\doendnotes{C}
\bigskip
\vfill

\clearpage

\footnotesize

\ifkorrekturansicht
  \lohead{\textsc{register}}
\fi

% theindex-Environment neu definieren ohne reledmac
\makeatletter
\renewenvironment{theindex}{%
  \ifkorrekturansicht
    \section*{\indexname}%
  \else
    \subsubsection*{Index der erwähnten Entitäten}%
  \fi
  \setlength{\parindent}{0pt}%
  \setlength{\parskip}{0pt plus 0.3pt}%
  \let\item\@idxitem
}{%
  \ifkorrekturansicht\clearpage\fi
}
\makeatother

\IfFileExists{\jobname-pw.ind}{\input{\jobname-pw.ind}}{}

% Quellenangabe nur in der Leseansicht
\ifkorrekturansicht\else
% Fallback-Definitionen, falls die .tex-Datei \titel etc. nicht gesetzt hat
\providecommand{\titel}{}
\providecommand{\editorInnen}{}
\providecommand{\dateiname}{\jobname}

\vspace{3cm}

\vfill

\footnotesize
\textsc{Quelle}: \titel. Herausgegeben von {\editorInnen}. In: \emph{Arthur Schnitzler: Briefwechsel mit Autorinnen und Autoren}.
 Digitale Edition, https://schnitzler-briefe.acdh.oeaw.ac.at/{\dateiname}.html (Stand \today)
\fi

\end{document}


