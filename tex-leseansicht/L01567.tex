%% latex-korrekturansicht-vorspann.tex
%% Vorspann für die Korrekturansicht.
%% Lädt die gemeinsame Datei latex-vorspann.tex mit gesetztem Schalter.

\newif\ifkorrekturansicht
\korrekturansichttrue

\input{../tex-inputs/latex-vorspann}


\section[Arthur Schnitzler: Widmungsexemplar Zwischenspiel für Hugo von Hofmannsthal, 24. 11. 1905]{L01567 Arthur Schnitzler: Widmungsexemplar Zwischenspiel für Hugo von
               Hofmannsthal, 24. 11. 1905}
\nopagebreak\mylabel{L01567v}
\rehead{ }\normalsize\beginnumbering\briefempfaengerindex{Hofmannsthal, Hugo von@\textsc{Hofmannsthal, Hugo von}!zzzSchnitzler, Arthur@\emph{von Arthur Schnitzler}!1905-11-241@{24. 11. 1905}|(be}
\toendnotes[C]{\smallbreak\pagebreak[2]}\Standort{FDH, FDH 3242.}
\physDesc{Widmung am Schmutztitel, 49 Zeichen
\newline{}Handschrift: schwarze Tinte, deutsche Kurrent}
\buchAbdrucke{\weitereDrucke{Hugo von Hofmannsthal: \emph{Bibliothek}. Frankfurt am Main: \emph{S. Fischer} 2011, S. 606.} }\toendnotes[C]{\smallbreak}
\pstart
           \noindent{}{\pb}Meinem lieben Hugo\pend
           \pstart \spacefill\mbox{ArthSchnitz}\pend{}
\pstart
           Berlin\oindex{Berlin@\textbf{Berlin}, \emph{P.PPLC}|pw}{ }24. 11. 905\pend
           \selectlanguage{ngerman}\vspace{1em}{\vspace{1\baselineskip}}
\pstart
           \centering{}{\pb}\textcolor{gray}{\textbf{Zwiſchenſpiel\pwindex{Zwischenspiel. Komoedie in drei Akten@\emph{Zwischenspiel. Komödie in drei Akten}|pw}}}\pend
           
\pstart
           \centering{}\textcolor{gray}{\textbf{Komödie in drei Akten von}}{\\}\textcolor{gray}{\textbf{\textbf{Arthur Schnitzler}}}\pend
           {\vspace{1\baselineskip}}
\pstart
           \centering{}\textcolor{gray}{\textbf{\so{Berlin}\oindex{Berlin@\textbf{Berlin}, \emph{P.PPLC}|pw}{ }\label{K_L01567-1v}\edtext{\so{1906}}{\lemma{\textnormal{\emph{1906}}}\Cendnote{\textnormal{am 28. 10. 1905 vom
                           \emph{Börsenblatt für den deutschen
                           Buchhandel}\pwindex{Boersenblatt fuer den Deutschen Buchhandel@\emph{Börsenblatt für den Deutschen Buchhandel}|pwk} als Neuerscheinung gemeldet}}}\label{K_L01567-1}}}\pend
           
\pstart
           \centering{}\textcolor{gray}{\textbf{\so{S. Fiſcher, Verlag}\orgindex{S. Fischer Verlag@S. Fischer Verlag|pw}}}\pend
           \selectlanguage{ngerman}\endnumbering\briefempfaengerindex{Hofmannsthal, Hugo von@\textsc{Hofmannsthal, Hugo von}!zzzSchnitzler, Arthur@\emph{von Arthur Schnitzler}!1905-11-241@{24. 11. 1905}|)be}\mylabel{L01567h}  \normalsize

\doendnotes{C}
\bigskip
\vfill

\clearpage

\footnotesize

\lohead{\textsc{register}}

% Definiere theindex-Environment komplett neu ohne reledmac
\makeatletter
\renewenvironment{theindex}{%
  \section*{\indexname}%
  \setlength{\parindent}{0pt}%
  \setlength{\parskip}{0pt plus 0.3pt}%
  \let\item\@idxitem
}{%
  \clearpage
}
\makeatother

\IfFileExists{\jobname-pw.ind}{\input{\jobname-pw.ind}}{}

\end{document}

      