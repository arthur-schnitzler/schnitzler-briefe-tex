%% latex-leseansicht-vorspann.tex
%% Vorspann für die Leseansicht.
%% Lädt die gemeinsame Datei latex-vorspann.tex mit nicht gesetztem Schalter.

\newif\ifkorrekturansicht
\korrekturansichtfalse

\input{../tex-inputs/latex-vorspann}


         
         \renewcommand{\erwaehntePersonen}{Personen:  Charles I von England,  Hintermayer,  Hintermayer}
         \renewcommand{\erwaehnteOrte}{Orte: Andorf, Bayern, Innviertel, Schärding, Wien}
         \renewcommand{\erwaehnteWerke}{Werke: David Copperfield, Robert, Siebenkäs}
               \section[Robert Adam an Arthur Schnitzler, 17. 7. 1918]{ Robert Adam an Arthur Schnitzler, 17. 7. 1918}\nopagebreak\mylabel{v}\rehead{ }\begin{ledgroupsized}[t]{13cm}\normalsize\beginnumbering \toendnotes[C]{\smallbreak\pagebreak[2]} \Standort{CUL, Schnitzler, B 1.}
\physDesc{Brief, 1 Blatt, 4 Seiten, 3005 Zeichen
\newline{}Handschrift: schwarze Tinte, deutsche Kurrent
\newline{}Schnitzler: 1) mit Bleistift beschriftet: »\textsc{Adam}«  2) mit rotem Buntstift zwei Unterstreichungen
\newline{}Ordnung: von unbekannter Hand nummeriert: »4« }\Standort{Wien, Österreichische Nationalbibliothek, Cod.ser. 52.263, 217.}
\physDesc{Brief, Maschinenschriftliche Abschrift, 1 Blatt, 1 Seite, 3005 Zeichen
\newline{}Schreibmaschine}\toendnotes[C]{\smallbreak}\pstart
           \raggedleft{}{\pb}Andorf\oindex{Andorf@\textbf{Andorf}|pw}, 17. Juli
                  1918.\pend
           \pstart{}Hochverehrter Herr Doktor!\pend\pstart
           Ich bin auf meiner Suche nach einem einſamen Erholungsorte – infolge einer während
               der Eiſenbahnfahrt vernommenen Äußerung einer Mitreiſenden – in dieſen kleinen
               bäuerlichen Ort des Innviertel\oindex{Innviertel@\textbf{Innviertel}|pw}s, nicht weit von
                  Schärding\oindex{Schaerding@\textbf{Schärding}|pw} entfernt, geraten und habe das
               gefunden, was ich geſucht hatte: ungeſtörte Einſamkeit – nur manchmal verſucht ſich
               die ältere Wirtstochter\pwindex{Hintermayer @\textsc{Hintermayer}, \emph{Gastwirtin}|pwv} oder
               ein ſtrebſamer Jüngling der Nachbarſchaft im Klavierüben; letzteres hat ſeinen Grund
               darin, daß mein Wirt\pwindex{Hintermayer @\textsc{Hintermayer}, \emph{Gastwirt}|pwv} im
               Beſitze des Ortsklaviers iſt –, wundervolle fruchtbare Wieſen und Felder ringsum im
               Hügelland, weite Strec{\pb}ken
               abwechslungsreicher Nadelwälder, in denen es außer vielem Wild, das jetzt für mich
               leider nicht in Betracht kommt, Beeren und Schwämme gibt und endlich eine ſehr gute,
               reichliche und nach Wien\oindex{Wien@\textbf{Wien}|pw}er Begriffen äußerſt
               wohlfeile Friedenskoſt; denn man verfügt hier noch über Nahrungsmittel, deren
               Exiſtenz in Wien\oindex{Wien@\textbf{Wien}|pw} längſt zur Sage geworden iſt, vor
               allem reichlich über Mehl, Butter und Milch. Dieſes Phänomen iſt zum Teil darauf
               zurückzuführen, daß man Sommergäſte mit wenigen Ausnahmen rückſichtslos abweiſt und
               ſich Hamſterverſuchen gegenüber ſehr ſpröde zeigt; weshalb man mit mir eine Ausnahme
               gemacht hat, weiß ich eigentlich nicht recht, aber es geſchah – nach urſprünglicher
               Abweiſung – und ich bin dem Schickſal dafür ſehr dankbar. Ich glaube bereits die
               günſtigen Wirkungen der unſparſamen {\pb}Verköſtigung nicht nur auf meinen körperlichen, ſondern auch auf meinen geiſtigen
               Zuſtand wahrzunehmen, eine gewiſſe Fähigkeit, freier und ungenierter Gedankengängen
               nachzugehen, ohne beſorgen zu müſſen, daß ſie plötzlich – wie es in Wien\oindex{Wien@\textbf{Wien}|pw}{ }ſo oft geſchah – in die Sackgaſſe der Nahrungsfrage
               einzulaufen: dies Kriegsthema des Eſſens ſchien mir in Geſpräch und Denken ſchon ſo
               unvermeidlich wie der Kopf Karls I.\pwindex{Charles I von England 19.11.1600 – 30.01.1649@\textsc{Charles I von England} (19.11.1600 – 30.01.1649), \emph{Regent}|pw} in den Promemorien des armen \textsc{Dick}\pwindex{\textcolor{red}{\textsuperscript{XXXX1 indx}}!David Copperfield1849@\strich\emph{David Copperfield} {[}1849{]}|pwv} im \textsc{David Copperfield}\pwindex{\textcolor{red}{\textsuperscript{XXXX1 indx}}!David Copperfield1849@\strich\emph{David Copperfield} {[}1849{]}|pw}.\pend
           \pstart
           Meine Lebensweiſe hier iſt von äußerſter Einfachheit: ich gehe nach dem Frühſtück in
               den Wald, laufe und liege drin bis zum Mittageſſen; bis zur Jauſe ſitze oder liege
               ich in oder beim Hauſe; dann gehe ich wieder in den Wald und verlaſſe ihn erſt, um
               zum Nachtmahl zu gehen; nach dem Nachtmahl ſpaziere ich ein wenig auf den Feldern
               umher und ſitze dann mit Bauern und Schul{\pb}lehrer beim \label{K_L02289_1v}\edtext{Moſt}{\lemma{\textnormal{\emph{Moſt}}}\Cendnote{\textnormal{gegärter Fruchtsaft}}}\label{K_L02289_1h}. Ich habe in
               zwei Wochen – außer der Zeitung – keine 20 Seiten im »Siebenkäs\pwindex{\textcolor{red}{\textsuperscript{XXXX1 indx}}!Siebenkaes1796 – 1797@\strich\emph{Siebenkäs} {[}1796 – 1797{]}|pw}« geleſen und nur ſehr wenig geſchrieben. Trotzdem bin ich mit
               jener Kriegstragödie\pwindex{Adam, Robert 20.04.1877 – 16.10.1961@\textsc{Adam, Robert} (20.04.1877 – 16.10.1961), \emph{Schriftsteller, Richter}!RobertNone@\strich\emph{Robert} {[}None{]}|pwv}, von
                  der\strikeout{n} ich Ihnen erzählte, (der
               Kannibalengeſchichte) ziemlich weit gekommen; zum Niederſchreiben bin ich nur viel zu
               faul. Aber dieſes läßt ſich hoffentlich in Wien\oindex{Wien@\textbf{Wien}|pw}
               nachholen.\pend
           \pstart
           Die Kriegsſtimmung der hieſigen Bevölkerung, die durch die letzte Niederlage
               ſchreckliche Verluſte erlitten hat, iſt nicht viel beſſer als die der Wien\oindex{Wien@\textbf{Wien}|pw}er; vor Äußerungen der Erregung bewahrt ſie
               wohl nur ihre felſenhafte Zuverſicht, demnächſt zu Baiern\oindex{Bayern@\textbf{Bayern}|pw} zu gehören: – worauf dieſer Glaube beruht, iſt nicht zu eruieren.\pend
           \pstart
           Mein Urlaub endet leider ſchon in 10 Tagen.\pend
           \pstart
           Mit den herzlichſten Grüßen\pend
           \pstart
           Ihr ergebener{\\[\baselineskip]}\spacefill\mbox{Robert Adam}\pend
           \leftskip=0em{}
         
         \endnumbering\mylabel{h}\end{ledgroupsized}  \newcommand{\dateiname}{L02289}\newcommand{\titel}{Robert Adam an Arthur Schnitzler, 17. 7. 1918}\newcommand{\editorInnen}{Martin Anton Müller und Gerd-Hermann Susen}%% latex-leseansicht-abspann.tex
%% Abspann für die Leseansicht.
%% Der Schalter \ifkorrekturansicht ist bereits durch den Vorspann gesetzt.

%% latex-abspann.tex
%% Gemeinsamer Abspann für Korrekturansicht und Leseansicht.
%% Setzt den Schalter \ifkorrekturansicht voraus (gesetzt in den
%% einbindenden Dateien latex-korrekturansicht-abspann.tex bzw.
%% latex-leseansicht-abspann.tex).
%% ---------------------------------------------------------------

\normalsize

% Das esempio-Environment wird nur in der Leseansicht benötigt
\ifkorrekturansicht\else
\newenvironment{esempio}[3]%
{
    \vspace{1.5ex}
    \rlap{\underline{#1}}
    \par
    \setlength{\parindent}{0cm}
    \nopagebreak
    \leftskip=#2cm
    \rightskip=#3cm
}
{
    \par
}
\fi

\doendnotes{C}
\bigskip
\vfill

\clearpage

\footnotesize

\ifkorrekturansicht
  \lohead{\textsc{register}}
\fi

% theindex-Environment neu definieren ohne reledmac
\makeatletter
\renewenvironment{theindex}{%
  \ifkorrekturansicht
    \section*{\indexname}%
  \else
    \subsubsection*{Index der erwähnten Entitäten}%
  \fi
  \setlength{\parindent}{0pt}%
  \setlength{\parskip}{0pt plus 0.3pt}%
  \let\item\@idxitem
}{%
  \ifkorrekturansicht\clearpage\fi
}
\makeatother

\IfFileExists{\jobname-pw.ind}{\input{\jobname-pw.ind}}{}

% Quellenangabe nur in der Leseansicht
\ifkorrekturansicht\else
% Fallback-Definitionen, falls die .tex-Datei \titel etc. nicht gesetzt hat
\providecommand{\titel}{}
\providecommand{\editorInnen}{}
\providecommand{\dateiname}{\jobname}

\vspace{3cm}

\vfill

\footnotesize
\textsc{Quelle}: \titel. Herausgegeben von {\editorInnen}. In: \emph{Arthur Schnitzler: Briefwechsel mit Autorinnen und Autoren}.
 Digitale Edition, https://schnitzler-briefe.acdh.oeaw.ac.at/{\dateiname}.html (Stand \today)
\fi

\end{document}


      