%% latex-leseansicht-vorspann.tex
%% Vorspann für die Leseansicht.
%% Lädt die gemeinsame Datei latex-vorspann.tex mit nicht gesetztem Schalter.

\newif\ifkorrekturansicht
\korrekturansichtfalse

\input{../tex-inputs/latex-vorspann}


               \section[Olga und Arthur Schnitzler an Hugo von Hofmannsthal, 7. 5. 1904]{ Olga und Arthur Schnitzler an Hugo von Hofmannsthal, 7. 5. 1904}\nopagebreak\mylabel{v}\rehead{ }\begin{ledgroupsized}[t]{13cm}\normalsize\beginnumbering\briefempfaengerindex{Hofmannsthal, Hugo von@\textsc{Hofmannsthal, Hugo von}!zzzSchnitzler, Arthur@\emph{von Arthur Schnitzler}!1904-05-072@{7. 5. 1904}|(be}\briefempfaengerindex{Hofmannsthal, Hugo von@\textsc{Hofmannsthal, Hugo von}!zzzSchnitzler, Olga@\emph{von Olga Schnitzler}!1904-05-072@{7. 5. 1904}|(be} \toendnotes[C]{\smallbreak\pagebreak[2]} \buchAlsQuelle{Hugo von Hofmannsthal, Arthur Schnitzler: \emph{Briefwechsel}. Hg. Therese Nickl und Heinrich Schnitzler. Frankfurt am Main: \emph{S. Fischer} 1964, S. 186.}\toendnotes[C]{\smallbreak}\pstart
           \raggedleft{}{\pb}7. Mai 1904\pend
           \pstart
           »Ital. Reise\pwindex{Goethe, Johann Wolfgang von 28.08.1749 – 22.03.1832@\textsc{Goethe, Johann Wolfgang von} (28.08.1749 – 22.03.1832), \emph{Schriftsteller}!Italienische Reise1816 – 1817@\strich\emph{Italienische Reise} {[}1816 – 1817{]}|pw}. Venedig\oindex{Venedig@\textbf{Venedig}|pw}, 7. October{ }{\dots} Nur betrug sie sich von \label{K_L01399_1v}\edtext{Anfang bis zu Ende}{\lemma{\textnormal{\emph{Anfang bis zu Ende}}}\Cendnote{\textnormal{In Goethe\pwindex{Goethe, Johann Wolfgang von 28.08.1749 – 22.03.1832@\textsc{Goethe, Johann Wolfgang von} (28.08.1749 – 22.03.1832), \emph{Schriftsteller}|pwk}s \emph{Italienischer Reise}\pwindex{Goethe, Johann Wolfgang von 28.08.1749 – 22.03.1832@\textsc{Goethe, Johann Wolfgang von} (28.08.1749 – 22.03.1832), \emph{Schriftsteller}!Italienische Reise1816 – 1817@\strich\emph{Italienische Reise} {[}1816 – 1817{]}|pwk} steht
                        »Anfang bis zum Ende«.}}}\label{K_L01399_1h} toll, wie es leider die Rolle
                  verlangt\pwindex{Goethe, Johann Wolfgang von 28.08.1749 – 22.03.1832@\textsc{Goethe, Johann Wolfgang von} (28.08.1749 – 22.03.1832), \emph{Schriftsteller}!Italienische Reise1816 – 1817@\strich\emph{Italienische Reise} {[}1816 – 1817{]}|pwv}« Sehn Sie!\pend
           \pstart
            Herzl. Grüße.{\\[\baselineskip]}\spacefill\mbox{Olga}{\\[\baselineskip]}{[}hs. Schnitzler:{]} Herzlichst \spacefill\mbox{Arthur}\pend
           \leftskip=0em{}\endnumbering\briefempfaengerindex{Hofmannsthal, Hugo von@\textsc{Hofmannsthal, Hugo von}!zzzSchnitzler, Arthur@\emph{von Arthur Schnitzler}!1904-05-072@{7. 5. 1904}|)be}\briefempfaengerindex{Hofmannsthal, Hugo von@\textsc{Hofmannsthal, Hugo von}!zzzSchnitzler, Olga@\emph{von Olga Schnitzler}!1904-05-072@{7. 5. 1904}|)be}\mylabel{h}\end{ledgroupsized}  \newcommand{\dateiname}{L01399}\newcommand{\titel}{Olga und Arthur Schnitzler an Hugo von Hofmannsthal, 7. 5. 1904}\newcommand{\editorInnen}{Martin Anton Müller und Gerd-Hermann Susen}
            \footnotesize
\begin{ledgroupsized}[t]{11.5cm}
\doendnotes{C}
\end{ledgroupsized}
         %% latex-leseansicht-abspann.tex
%% Abspann für die Leseansicht.
%% Der Schalter \ifkorrekturansicht ist bereits durch den Vorspann gesetzt.

%% latex-abspann.tex
%% Gemeinsamer Abspann für Korrekturansicht und Leseansicht.
%% Setzt den Schalter \ifkorrekturansicht voraus (gesetzt in den
%% einbindenden Dateien latex-korrekturansicht-abspann.tex bzw.
%% latex-leseansicht-abspann.tex).
%% ---------------------------------------------------------------

\normalsize

% Das esempio-Environment wird nur in der Leseansicht benötigt
\ifkorrekturansicht\else
\newenvironment{esempio}[3]%
{
    \vspace{1.5ex}
    \rlap{\underline{#1}}
    \par
    \setlength{\parindent}{0cm}
    \nopagebreak
    \leftskip=#2cm
    \rightskip=#3cm
}
{
    \par
}
\fi

\doendnotes{C}
\bigskip
\vfill

\clearpage

\footnotesize

\ifkorrekturansicht
  \lohead{\textsc{register}}
\fi

% theindex-Environment neu definieren ohne reledmac
\makeatletter
\renewenvironment{theindex}{%
  \ifkorrekturansicht
    \section*{\indexname}%
  \else
    \subsubsection*{Index der erwähnten Entitäten}%
  \fi
  \setlength{\parindent}{0pt}%
  \setlength{\parskip}{0pt plus 0.3pt}%
  \let\item\@idxitem
}{%
  \ifkorrekturansicht\clearpage\fi
}
\makeatother

\IfFileExists{\jobname-pw.ind}{\input{\jobname-pw.ind}}{}

% Quellenangabe nur in der Leseansicht
\ifkorrekturansicht\else
% Fallback-Definitionen, falls die .tex-Datei \titel etc. nicht gesetzt hat
\providecommand{\titel}{}
\providecommand{\editorInnen}{}
\providecommand{\dateiname}{\jobname}

\vspace{3cm}

\vfill

\footnotesize
\textsc{Quelle}: \titel. Herausgegeben von {\editorInnen}. In: \emph{Arthur Schnitzler: Briefwechsel mit Autorinnen und Autoren}.
 Digitale Edition, https://schnitzler-briefe.acdh.oeaw.ac.at/{\dateiname}.html (Stand \today)
\fi

\end{document}


      