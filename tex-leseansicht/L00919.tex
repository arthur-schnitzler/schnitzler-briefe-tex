%% latex-leseansicht-vorspann.tex
%% Vorspann für die Leseansicht.
%% Lädt die gemeinsame Datei latex-vorspann.tex mit nicht gesetztem Schalter.

\newif\ifkorrekturansicht
\korrekturansichtfalse

\input{../tex-inputs/latex-vorspann}


         
         \renewcommand{\erwaehntePersonen}{Personen: Wilhelm Bombast von Hohenheim, Theofrastus Bombastus Paracelsus}
         \renewcommand{\erwaehnteOrte}{Orte: Villach, Wien}
         \renewcommand{\erwaehnteWerke}{Werke: Der grüne Kakadu. Groteske in einem Akt}
               \section[Richard Beer-Hofmann an Arthur Schnitzler, 29. 5. 1899]{ Richard Beer-Hofmann an Arthur Schnitzler, 29. 5. 1899}\nopagebreak\mylabel{v}\rehead{ }\begin{ledgroupsized}[t]{13cm}\normalsize\beginnumbering \toendnotes[C]{\smallbreak\pagebreak[2]} \Standort{CUL, Schnitzler, B 8.}
\physDesc{Briefkarte,  (Faltkarte
                              )
\newline{}Handschrift: Bleistift, lateinische Kurrent}\buchAbdrucke{\weitereDrucke{Arthur Schnitzler, Richard Beer-Hofmann: \emph{Briefwechsel 1891–1931}. Hg. Konstanze Fliedl. Wien, Zürich: \emph{Europaverlag} 1992, S. 127–128.} }\toendnotes[C]{\smallbreak}\pstart
           \noindent{}\centering{}{\pb}\textcolor{gray}{\textbf{Platz}}.\pend
           \pstart
           \noindent{}\centering{}\textcolor{gray}{\textbf{{\pb}Besten Gruss aus Villach\oindex{Villach@\textbf{Villach}|pw} sendet}}\pend
           \pstart\center{}Lieber Arthur!\pend\pstart
           In \label{T_L00919_1v}\edtext{diesem
                  Hause}{\lemma{\textnormal{\emph{diesem
                  Hause}}}\Cendnote{\textnormal{Ein Pfeil
                  mit Bleistift markiert das Gebäude auf der gedruckten Abbildung.}}}\label{T_L00919_1h} lebte von
                  1502 bis zu seinem Tode 8 Sept 1534 als Stadtarzt von
                  Villach\oindex{Villach@\textbf{Villach}|pw}, Wilhelm
                  Bombast von Hohenheim\pwindex{Bombast von Hohenheim, Wilhelm 1457 – 1534@\textsc{Bombast von Hohenheim, Wilhelm} (1457 – 1534)|pw}; sein Sohn, der durch Sie — so \label{K_L00919_1v}\edtext{berühmte\pwindex{Schnitzler, Arthur 15.05.1862 – 21.10.1931@\textsc{Schnitzler, Arthur} (15.05.1862 – 21.10.1931), \emph{Schriftsteller, Mediziner}!gruene Kakadu. Groteske in einem Akt1. 3. 1899@\strich\emph{Der grüne Kakadu. Groteske in einem Akt} {[}1. 3. 1899{]}|pw}}{\lemma{\textnormal{\emph{berühmte}}}\Cendnote{\textnormal{Anspielung auf Schnitzler\pwindex{Schnitzler, Arthur 15.05.1862 – 21.10.1931@\textsc{Schnitzler, Arthur} (15.05.1862 – 21.10.1931), \emph{Schriftsteller, Mediziner}|pwk}s Einakter \emph{Paracelsus}\pwindex{Schnitzler, Arthur 15.05.1862 – 21.10.1931@\textsc{Schnitzler, Arthur} (15.05.1862 – 21.10.1931), \emph{Schriftsteller, Mediziner}!gruene Kakadu. Groteske in einem Akt1. 3. 1899@\strich\emph{Der grüne Kakadu. Groteske in einem Akt} {[}1. 3. 1899{]}|pwk}.}}}\label{K_L00919_1h}{ }Paracelsus\pwindex{Paracelsus, Theofrastus Bombastus 1493/1494 – 24.9.1541@\textsc{Paracelsus, Theofrastus Bombastus} (1493/1494 – 24.9.1541), \emph{Mediziner, Philosoph, Chemiker}|pw} lebte hier von
                  1502–1516, und Richard Beer-Hofmann trank am
                  29/V 1899 hier schwarzen Kaffee; das letzte kann natürlich heute noch
               nicht auf der Gedenktafel stehen.\pend
           \pstart
           Herzlichst{\\[\baselineskip]}\spacefill\mbox{Richard}\pend
           \leftskip=0em{}
         
         \endnumbering\mylabel{h}\end{ledgroupsized}  \newcommand{\dateiname}{L00919}\newcommand{\titel}{Richard Beer-Hofmann an Arthur Schnitzler, 29. 5. 1899}\newcommand{\editorInnen}{Martin Anton Müller und Gerd-Hermann Susen}%% latex-leseansicht-abspann.tex
%% Abspann für die Leseansicht.
%% Der Schalter \ifkorrekturansicht ist bereits durch den Vorspann gesetzt.

%% latex-abspann.tex
%% Gemeinsamer Abspann für Korrekturansicht und Leseansicht.
%% Setzt den Schalter \ifkorrekturansicht voraus (gesetzt in den
%% einbindenden Dateien latex-korrekturansicht-abspann.tex bzw.
%% latex-leseansicht-abspann.tex).
%% ---------------------------------------------------------------

\normalsize

% Das esempio-Environment wird nur in der Leseansicht benötigt
\ifkorrekturansicht\else
\newenvironment{esempio}[3]%
{
    \vspace{1.5ex}
    \rlap{\underline{#1}}
    \par
    \setlength{\parindent}{0cm}
    \nopagebreak
    \leftskip=#2cm
    \rightskip=#3cm
}
{
    \par
}
\fi

\doendnotes{C}
\bigskip
\vfill

\clearpage

\footnotesize

\ifkorrekturansicht
  \lohead{\textsc{register}}
\fi

% theindex-Environment neu definieren ohne reledmac
\makeatletter
\renewenvironment{theindex}{%
  \ifkorrekturansicht
    \section*{\indexname}%
  \else
    \subsubsection*{Index der erwähnten Entitäten}%
  \fi
  \setlength{\parindent}{0pt}%
  \setlength{\parskip}{0pt plus 0.3pt}%
  \let\item\@idxitem
}{%
  \ifkorrekturansicht\clearpage\fi
}
\makeatother

\IfFileExists{\jobname-pw.ind}{\input{\jobname-pw.ind}}{}

% Quellenangabe nur in der Leseansicht
\ifkorrekturansicht\else
% Fallback-Definitionen, falls die .tex-Datei \titel etc. nicht gesetzt hat
\providecommand{\titel}{}
\providecommand{\editorInnen}{}
\providecommand{\dateiname}{\jobname}

\vspace{3cm}

\vfill

\footnotesize
\textsc{Quelle}: \titel. Herausgegeben von {\editorInnen}. In: \emph{Arthur Schnitzler: Briefwechsel mit Autorinnen und Autoren}.
 Digitale Edition, https://schnitzler-briefe.acdh.oeaw.ac.at/{\dateiname}.html (Stand \today)
\fi

\end{document}


      