%% latex-korrekturansicht-vorspann.tex
%% Vorspann für die Korrekturansicht.
%% Lädt die gemeinsame Datei latex-vorspann.tex mit gesetztem Schalter.

\newif\ifkorrekturansicht
\korrekturansichttrue

\input{../tex-inputs/latex-vorspann}


\section[Richard Beer-Hofmann an Arthur Schnitzler, 29. 5. 1899]{L00919 Richard Beer-Hofmann an Arthur Schnitzler, 29. 5. 1899}
\nopagebreak\mylabel{L00919v}
\rehead{ }\normalsize\beginnumbering\briefempfaengerindex{Schnitzler, Arthur@\textsc{Schnitzler, Arthur}!zzzBeer-Hofmann, Richard@\emph{von Richard Beer-Hofmann}!1899-05-291@{29. 5. 1899}|(be}
\toendnotes[C]{\smallbreak\pagebreak[2]}\Standort{CUL, Schnitzler, B 8.}
\physDesc{Briefkarte, 352 Zeichen (Faltkarte )
\newline{}Handschrift: Bleistift, lateinische Kurrent}
\buchAbdrucke{\weitereDrucke{Arthur Schnitzler, Richard Beer-Hofmann: \emph{Briefwechsel 1891–1931}. Wien, Zürich: \emph{Europaverlag} 1992, S. 127–128.} }\toendnotes[C]{\smallbreak}
\pstart
           \centering{}{\pb}\textcolor{gray}{\textbf{Platz}}.\pend
           
\pstart
           \centering{}\textcolor{gray}{\textbf{{\pb}Besten Gruss aus Villach\oindex{Villach@\textbf{Villach}, \emph{A.ADM3}|pw}
                     sendet}}\pend
           
\pstart\center{}Lieber Arthur!\pend\vspace{0.5em}
\pstart
           In \label{T_L00919-1v}\edtext{diesem Hause}{\lemma{\textnormal{\emph{diesem Hause}}}\Cendnote{\textnormal{Ein Pfeil mit Bleistift markiert das Gebäude
                  auf der gedruckten Abbildung.}}}\label{T_L00919-1} lebte von 1502 bis zu seinem
               Tode 8 Sept 1534 als Stadtarzt von Villach\oindex{Villach@\textbf{Villach}, \emph{A.ADM3}|pw}, Wilhelm Bombast von
                  Hohenheim\pwindex{Bombast von Hohenheim, Wilhelm 1457 – 1534@\textsc{Bombast von Hohenheim, Wilhelm} (1457 – 1534)|pw}; sein Sohn, der durch Sie – so \label{K_L00919-1v}\edtext{berühmte\pwindex{gruene Kakadu. Groteske in einem Akt@\emph{Der grüne Kakadu. Groteske in einem Akt}|pw}}{\lemma{\textnormal{\emph{berühmte}}}\Cendnote{\textnormal{Anspielung auf Schnitzlers Einakter \emph{Paracelsus}\pwindex{gruene Kakadu. Groteske in einem Akt@\emph{Der grüne Kakadu. Groteske in einem Akt}|pwk}.}}}\label{K_L00919-1}{ }Paracelsus\pwindex{Paracelsus, Theofrastus Bombastus 1493/1494 – 24.9.1541@\textsc{Paracelsus, Theofrastus Bombastus} (1493/1494 – 24.9.1541), \emph{Mediziner/Medizinerin, Philosoph/Philosophin, Chemiker/Chemikerin}|pw} lebte hier von
                  1502–1516, und Richard Beer-Hofmann trank am
                  29/V 1899 hier schwarzen Kaffee; das letzte kann natürlich heute noch
               nicht auf der Gedenktafel stehen.\pend
           
\pstart
           Herzlichst{\\[\baselineskip]}\spacefill\mbox{Richard}\pend
           \leftskip=0em{}\selectlanguage{ngerman}\endnumbering\briefempfaengerindex{Schnitzler, Arthur@\textsc{Schnitzler, Arthur}!zzzBeer-Hofmann, Richard@\emph{von Richard Beer-Hofmann}!1899-05-291@{29. 5. 1899}|)be}\mylabel{L00919h}  \normalsize

\doendnotes{C}
\bigskip
\vfill

\clearpage

\footnotesize

\lohead{\textsc{register}}

% Definiere theindex-Environment komplett neu ohne reledmac
\makeatletter
\renewenvironment{theindex}{%
  \section*{\indexname}%
  \setlength{\parindent}{0pt}%
  \setlength{\parskip}{0pt plus 0.3pt}%
  \let\item\@idxitem
}{%
  \clearpage
}
\makeatother

\IfFileExists{\jobname-pw.ind}{\input{\jobname-pw.ind}}{}

\end{document}

      