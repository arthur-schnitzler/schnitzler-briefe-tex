%% latex-korrekturansicht-vorspann.tex
%% Vorspann für die Korrekturansicht.
%% Lädt die gemeinsame Datei latex-vorspann.tex mit gesetztem Schalter.

\newif\ifkorrekturansicht
\korrekturansichttrue

\input{../tex-inputs/latex-vorspann}


\section[Olga Schnitzler an Richard Beer-Hofmann, {[}28. 11. 1908?{]}]{L01812 Olga Schnitzler an Richard Beer-Hofmann, {[}28. 11. 1908?{]}}
\nopagebreak\mylabel{L01812v}
\rehead{ }\normalsize\beginnumbering\briefempfaengerindex{Beer-Hofmann, Richard@\textsc{Beer-Hofmann, Richard}!zzzSchnitzler, Olga@\emph{von Olga Schnitzler}!1908-11-282@{{[}28. 11. 1908?{]}}|(be}
\toendnotes[C]{\smallbreak\pagebreak[2]}\Standort{YCGL, MSS 31.}
\physDesc{Brief, 1 Blatt, 1 Seite, Umschlag, 244 Zeichen
\newline{}Handschrift: schwarze Tinte, lateinische Kurrent
\newline{}Versand: ohne postalischen Übermittlungsvermerk }\toendnotes[C]{\smallbreak}\pstart{}{\pb}\textcolor{gray}{\textbf{O. S.}}\pend{}{\bigskip}\pstart{}{\pb}Herrn D\textsuperscript{r} Richard
                  Beer-Hofmann\pend{}{\bigskip}\vspace{1em}
\pstart
           {\pb}\textcolor{gray}{\textbf{O. S.}}\pend
           
\pstart{}Lieber Herr Doctor,\pend\vspace{0.5em}
\pstart
           Arthur lässt sagen: wir wollten zuhause
               nachtmalen, möchten Sie und Kerr\pwindex{Kerr, Alfred 25.12.1867 – 12.10.1948@\textsc{Kerr, Alfred} (25.12.1867 – 12.10.1948), \emph{Schriftsteller/Schriftstellerin, Kritiker/Kritikerin}|pw} aber sehr
               gerne treffen, daher sind wir nach dem Concert bei Meissl {\kaufmannsund} Schadn\oindex{Meissl {\kaufmannsund} Schadn@\textbf{Meissl {\kaufmannsund} Schadn}, \emph{Hotel (K.HTL)}|pw}, so gegen ½ oder
                  ¾ 10.\pend
           
\pstart
           Herzliche Grüsse an Sie Beide\pwindex{Beer-Hofmann, Paula 25.02.1879 – 30.10.1939@\textsc{Beer-Hofmann, Paula} (25.02.1879 – 30.10.1939)|pwv}{\\[\baselineskip]}\spacefill\mbox{OlgaS.}\pend
           \leftskip=0em{}\selectlanguage{ngerman}\endnumbering\briefempfaengerindex{Beer-Hofmann, Richard@\textsc{Beer-Hofmann, Richard}!zzzSchnitzler, Olga@\emph{von Olga Schnitzler}!1908-11-282@{{[}28. 11. 1908?{]}}|)be}\mylabel{L01812h}  \normalsize

\doendnotes{C}
\bigskip
\vfill

\clearpage

\footnotesize

\lohead{\textsc{register}}

% Definiere theindex-Environment komplett neu ohne reledmac
\makeatletter
\renewenvironment{theindex}{%
  \section*{\indexname}%
  \setlength{\parindent}{0pt}%
  \setlength{\parskip}{0pt plus 0.3pt}%
  \let\item\@idxitem
}{%
  \clearpage
}
\makeatother

\IfFileExists{\jobname-pw.ind}{\input{\jobname-pw.ind}}{}

\end{document}

      