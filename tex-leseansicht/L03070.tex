%% latex-korrekturansicht-vorspann.tex
%% Vorspann für die Korrekturansicht.
%% Lädt die gemeinsame Datei latex-vorspann.tex mit gesetztem Schalter.

\newif\ifkorrekturansicht
\korrekturansichttrue

\input{../tex-inputs/latex-vorspann}


\section[ Paul Goldmann an Arthur Schnitzler, 21. 6. {[}1901{]}]{L03070 Paul Goldmann an Arthur Schnitzler, 21. 6. {[}1901{]}}
\nopagebreak\mylabel{L03070v}
\rehead{ }\normalsize\beginnumbering\briefempfaengerindex{Schnitzler, Arthur@\textsc{Schnitzler, Arthur}!zzzGoldmann, Paul@\emph{von Paul Goldmann}!1901-06-211@{21. 6. {[}1901{]}}|(be}
\toendnotes[C]{\smallbreak\pagebreak[2]}\Standort{DLA, A:Schnitzler, HS.NZ85.1.3171.}
\physDesc{Brief, 1 Blatt, 4 Seiten, 1360 Zeichen
\newline{}Handschrift: blaue Tinte, deutsche Kurrent
\newline{}Schnitzler: mit Bleistift das Jahr »901« vermerkt }\toendnotes[C]{\smallbreak}
\pstart
           \raggedleft{}{\pb}\textcolor{gray}{\textbf{DESSAUERSTRASSE 19}}\oindex{Dessauer Strasse@\textbf{Dessauer Straße}, \emph{Straße (K.STR)}|pw}\pend
           
\pstart
           Berlin\oindex{Berlin@\textbf{Berlin}, \emph{P.PPLC}|pw}, 21. Juni.\pend
           
\pstart\center{}Mein lieber Freund,\pend\vspace{0.5em}
\pstart
           Wir haben heut hier telegraphiſch die Kunde erhalten,
               daß Du \label{K_L03070-1v}\edtext{aus dem Offizierſtande
                  geſtrichen}{\lemma{\textnormal{\emph{aus … geſtrichen}}}\Cendnote{\textnormal{Für die Veröffentlichung
                  von \emph{Lieutenant Gustl}\pwindex{Lieutenant Gustl. Novelle@\emph{Lieutenant Gustl. Novelle}|pwk} wurde Schnitzler am 21. 6. 1901 der Offiziersrang aberkannt.}}}\label{K_L03070-1}
               biſt. \strikeout{Es iſt} Ich weiß, es wird Dir ſchrecklich ſein,
               daß Du künftig den bewaffneten Schaaren nicht als Heerführer voranziehen ſollſt, aber
               Du wirſt das Unglück zu tragen wiſſen. Die \label{K_L03070-2v}\edtext{Begründung}{\lemma{\textnormal{\emph{Begründung}}}\Cendnote{\textnormal{Siehe
                  etwa den Leitartikel\pwindex{Wien, 20. Juni@\emph{Wien, 20. Juni}|pwkv} der
                     \emph{Neuen Freien Presse}\pwindex{Neue Freie Presse@\emph{Neue Freie Presse}|pwk} zum Thema: [Moriz Benedikt\pwindex{Benedikt, Moriz 27.05.1849 – 18.03.1920@\textsc{Benedikt, Moriz} (27.05.1849 – 18.03.1920), \emph{Journalist/Journalistin, Herausgeber/Herausgeberin}|pwk}]: \emph{Wien, 20. Juni}\pwindex{Wien, 20. Juni@\emph{Wien, 20. Juni}|pwk}. In: \emph{Neue Freie Presse}\pwindex{Neue Freie Presse@\emph{Neue Freie Presse}|pwk}, Nr. 13.226, 21. 6. 1901, Morgenblatt, S. 1–2.}}}\label{K_L03070-2} des ehrenräthlichen
               Erkenntniſſes iſt perfid und verräth gute jeſuitiſche Schulung. Wenn Du noch eines
               Mittels bedurft {\pb}hätteſt, um in ganz Deutſchland\oindex{Deutschland@\textbf{Deutschland}, \emph{A.PCLI}|pw} und Öſterreich\oindex{Oesterreich@\textbf{Österreich}, \emph{A.PCLI}|pw} Sympathien zu gewinnen, ſo wäre dieſer Streich jedenfalls das
               beſte Mittel dieſer Art. Immerhin werden die Sympathien, die \substVorne{}\textsuperscript{man}\substDazwischen{}man\substHinten{} für Dich hegt, überall an Herzlichkeit zunehmen, und die Herren vom
               Ehrenrathe haben durch ihr Verdikt für Deine Perſon und Deine Werke eine ſehr
               löbliche Propaganda gemacht. Da ſie aber das Gegentheil beabſichtigt haben, {\pb}ſo wirſt Du hoffentlich die Antwort \strikeout{\textcolor{gray}{u}} nicht ſchuldig bleiben. Eine kräftige und doch vornehme Abſage an den \strikeout{g\textcolor{gray}{e}} Ehrenrath und den Militarismus überhaupt wäre wohl angemeſſen, und die »Neue Freie Preſſe\orgindex{Neue Freie Presse@Neue Freie Presse|pw}« könnte einer ſolchen \label{K_L03070-3v}\edtext{Antwort}{\lemma{\textnormal{\emph{Antwort}}}\Cendnote{\textnormal{Eine solche Antwort gab es nie, Schnitzler entschied sich auf Anraten von Max Burckhard\pwindex{Burckhard, Max Eugen 14.07.1854 – 16.03.1912@\textsc{Burckhard, Max Eugen} (14.07.1854 – 16.03.1912), \emph{Schriftsteller/Schriftstellerin, Rechtswissenschaftler/Rechtswissenschaftlerin, Theaterleiter/Theaterleiterin}|pwk}, sich weder dem Geheimprozess zu stellen
                  noch Stellung zu beziehen. Schnitzler
                  verfasste jedoch zu einem nicht näher bestimmten Zeitpunkt eine fünfseitige, zu
                  Lebzeiten nicht veröffentlichte Parodie\pwindex{Leutnant Gustl. Parodie@\emph{Leutnant Gustl. Parodie}|pwkv} auf seine Novelle\pwindex{Lieutenant Gustl. Novelle@\emph{Lieutenant Gustl. Novelle}|pwkv}, betitelt \emph{Leutnant Gustl}\pwindex{Leutnant Gustl. Parodie@\emph{Leutnant Gustl. Parodie}|pwk}.
                  Darin wird Gustl\pwindex{Lieutenant Gustl. Novelle@\emph{Lieutenant Gustl. Novelle}|pwkv}
                  übertrieben sittlich-korrekt dargestellt und die antisemitisch geprägte
                  Berichterstattung humorvoll thematisiert.}}}\label{K_L03070-3} aus Deiner Feder die Aufnahme
               kaum verweigern.\pend
           
\pstart
           Ich drücke Dir herzlichſt die \strikeout{H\textcolor{gray}{a}} Hand und grüße Dich in Treue, – obwohl ich es für meinen Theil lebhaft
               bedaure, {\pb}nicht mehr einen k. u. k. Regimentsarzt,
               ſondern einen ganz gemeinen Reſerviſten als Freund zu beſitzen. {\\[\baselineskip]}Dein {\\[\baselineskip]}\spacefill\mbox{Paul Goldmann.}\pend
           \leftskip=0em{}
\pstart
           \noindent{}Herzlichen Gruß an Fräulein \textsc{Olga\pwindex{Schnitzler, Olga 17.01.1882 – 13.01.1970@\textsc{Schnitzler, Olga} (17.01.1882 – 13.01.1970), \emph{Schauspieler/Schauspielerin, Sänger/Sängerin}|pw}}!\pend
           \selectlanguage{ngerman}\endnumbering\briefempfaengerindex{Schnitzler, Arthur@\textsc{Schnitzler, Arthur}!zzzGoldmann, Paul@\emph{von Paul Goldmann}!1901-06-211@{21. 6. {[}1901{]}}|)be}\mylabel{L03070h}  \normalsize

\doendnotes{C}
\bigskip
\vfill

\clearpage

\footnotesize

\lohead{\textsc{register}}

% Definiere theindex-Environment komplett neu ohne reledmac
\makeatletter
\renewenvironment{theindex}{%
  \section*{\indexname}%
  \setlength{\parindent}{0pt}%
  \setlength{\parskip}{0pt plus 0.3pt}%
  \let\item\@idxitem
}{%
  \clearpage
}
\makeatother

\IfFileExists{\jobname-pw.ind}{\input{\jobname-pw.ind}}{}

\end{document}

      