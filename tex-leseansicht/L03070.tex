%% latex-leseansicht-vorspann.tex
%% Vorspann für die Leseansicht.
%% Lädt die gemeinsame Datei latex-vorspann.tex mit nicht gesetztem Schalter.

\newif\ifkorrekturansicht
\korrekturansichtfalse

\input{../tex-inputs/latex-vorspann}


         
         \renewcommand{\erwaehntePersonen}{Personen: Moriz Benedikt, Max Eugen Burckhard, Olga Schnitzler}
         \renewcommand{\erwaehnteInstitutionen}{Institutionen: Neue Freie Presse}
         \renewcommand{\erwaehnteOrte}{Orte: Berlin, Dessauer Straße, Deutschland, Salzburg, Österreich}
         \renewcommand{\erwaehnteWerke}{Werke: Leutnant Gustl. Parodie, Lieutenant Gustl. Novelle, Neue Freie Presse, Wien, 20. Juni}
               \section[ Paul Goldmann an Arthur Schnitzler, 21. 6. {[}1901{]}]{ Paul Goldmann an Arthur Schnitzler, 21. 6. {[}1901{]}}\nopagebreak\mylabel{v}\rehead{ }\begin{ledgroupsized}[t]{13cm}\normalsize\beginnumbering \toendnotes[C]{\smallbreak\pagebreak[2]} \Standort{DLA, A:Schnitzler, HS.NZ85.1.3171.}
\physDesc{Brief, 1 Blatt, 4 Seiten, 1360 Zeichen
\newline{}Handschrift: blaue Tinte, deutsche Kurrent
\newline{}Schnitzler: mit Bleistift das Jahr »901« vermerkt }\toendnotes[C]{\smallbreak}\pstart
           \noindent{}\raggedleft{}{\pb}\textcolor{gray}{\textbf{DESSAUERSTRASSE 19}}\oindex{Dessauer Strasse@\textbf{Dessauer Straße}|pw}\pend
           \pstart
           Berlin\oindex{Berlin@\textbf{Berlin}|pw}, 21. Juni.\pend
           \pstart\center{}Mein lieber Freund,\pend\pstart
           Wir haben heut hier telegraphiſch die Kunde erhalten,
               daß Du \label{K_L03070-1v}\edtext{aus dem Offizierſtande
                  geſtrichen}{\lemma{\textnormal{\emph{aus … geſtrichen}}}\Cendnote{\textnormal{Für die Veröffentlichung
                  von \emph{Lieutenant Gustl}\pwindex{Schnitzler, Arthur 15.05.1862 – 21.10.1931@\textsc{Schnitzler, Arthur} (15.05.1862 – 21.10.1931), \emph{Schriftsteller, Mediziner}!Lieutenant Gustl. Novelle1900-12-25@\strich\emph{Lieutenant Gustl. Novelle} {[}1900-12-25{]}|pwk} wurde Schnitzler\pwindex{Schnitzler, Arthur 15.05.1862 – 21.10.1931@\textsc{Schnitzler, Arthur} (15.05.1862 – 21.10.1931), \emph{Schriftsteller, Mediziner}|pwk} am 21. 6. 1901 der Offiziersrang aberkannt.}}}\label{K_L03070-1h}
               biſt. \strikeout{Es iſt} Ich weiß, es wird Dir ſchrecklich ſein,
               daß Du künftig den bewaffneten Schaaren nicht als Heerführer voranziehen ſollſt, aber
               Du wirſt das Unglück zu tragen wiſſen. Die \label{K_L03070-2v}\edtext{Begründung}{\lemma{\textnormal{\emph{Begründung}}}\Cendnote{\textnormal{Siehe
                  etwa den Leitartikel\pwindex{Wien, 20. Juni21. 6. 1901@\emph{Wien, 20. Juni} {[}21. 6. 1901{]}|pwkv} der
                     \emph{Neuen Freien Presse}\pwindex{Neue Freie Presse1864 – 1939@\emph{Neue Freie Presse} {[}1864 – 1939{]}|pwk} zum Thema: [Moriz Benedikt\pwindex{Benedikt, Moriz 27.05.1849 – 18.03.1920@\textsc{Benedikt, Moriz} (27.05.1849 – 18.03.1920), \emph{Journalist, Herausgeber}|pwk}]: \emph{Wien, 20. Juni}\pwindex{Wien, 20. Juni21. 6. 1901@\emph{Wien, 20. Juni} {[}21. 6. 1901{]}|pwk}. In: \emph{Neue Freie Presse}\pwindex{Neue Freie Presse1864 – 1939@\emph{Neue Freie Presse} {[}1864 – 1939{]}|pwk}, Nr. 13.226, 21. 6. 1901, Morgenblatt, S. 1–2.}}}\label{K_L03070-2h} des ehrenräthlichen
               Erkenntniſſes iſt perfid und verräth gute jeſuitiſche Schulung. Wenn Du noch eines
               Mittels bedurft {\pb}hätteſt, um in ganz Deutſchland\oindex{Deutschland@\textbf{Deutschland}|pw} und Öſterreich\oindex{Oesterreich@\textbf{Österreich}|pw} Sympathien zu gewinnen, ſo wäre dieſer Streich jedenfalls das
               beſte Mittel dieſer Art. Immerhin werden die Sympathien, die \substVorne{}\textsuperscript{man}\substDazwischen{}man\substHinten{} für Dich hegt, überall an Herzlichkeit zunehmen, und die Herren vom
               Ehrenrathe haben durch ihr Verdikt für Deine Perſon und Deine Werke eine ſehr
               löbliche Propaganda gemacht. Da ſie aber das Gegentheil beabſichtigt haben, {\pb}ſo wirſt Du hoffentlich die Antwort \strikeout{\textcolor{gray}{u}} nicht ſchuldig bleiben. Eine kräftige und doch vornehme Abſage an den \strikeout{g\textcolor{gray}{e}} Ehrenrath und den Militarismus überhaupt wäre wohl angemeſſen, und die »Neue Freie Preſſe\orgindex{Neue Freie Presse@Neue Freie Presse|pw}« könnte einer ſolchen \label{K_L03070-3v}\edtext{Antwort}{\lemma{\textnormal{\emph{Antwort}}}\Cendnote{\textnormal{Eine solche Antwort gab es nie, Schnitzler\pwindex{Schnitzler, Arthur 15.05.1862 – 21.10.1931@\textsc{Schnitzler, Arthur} (15.05.1862 – 21.10.1931), \emph{Schriftsteller, Mediziner}|pwk} entschied sich auf Anraten von Max Burckhard\pwindex{Burckhard, Max Eugen 14.07.1854 – 16.03.1912@\textsc{Burckhard, Max Eugen} (14.07.1854 – 16.03.1912), \emph{Schriftsteller, Rechtswissenschaftler, Theaterleiter}|pwk}, sich weder dem Geheimprozess zu stellen
                  noch Stellung zu beziehen. Schnitzler\pwindex{Schnitzler, Arthur 15.05.1862 – 21.10.1931@\textsc{Schnitzler, Arthur} (15.05.1862 – 21.10.1931), \emph{Schriftsteller, Mediziner}|pwk}
                  verfasste jedoch zu einem nicht näher bestimmten Zeitpunkt eine fünfseitige, zu
                  Lebzeiten nicht veröffentlichte Parodie\pwindex{Schnitzler, Arthur 15.05.1862 – 21.10.1931@\textsc{Schnitzler, Arthur} (15.05.1862 – 21.10.1931), \emph{Schriftsteller, Mediziner}!Leutnant Gustl. Parodie@\strich\emph{Leutnant Gustl. Parodie}|pwkv} auf seine Novelle\pwindex{Schnitzler, Arthur 15.05.1862 – 21.10.1931@\textsc{Schnitzler, Arthur} (15.05.1862 – 21.10.1931), \emph{Schriftsteller, Mediziner}!Lieutenant Gustl. Novelle1900-12-25@\strich\emph{Lieutenant Gustl. Novelle} {[}1900-12-25{]}|pwkv}, betitelt \emph{Leutnant Gustl}\pwindex{Schnitzler, Arthur 15.05.1862 – 21.10.1931@\textsc{Schnitzler, Arthur} (15.05.1862 – 21.10.1931), \emph{Schriftsteller, Mediziner}!Leutnant Gustl. Parodie@\strich\emph{Leutnant Gustl. Parodie}|pwk}.
                  Darin wird Gustl\pwindex{Schnitzler, Arthur 15.05.1862 – 21.10.1931@\textsc{Schnitzler, Arthur} (15.05.1862 – 21.10.1931), \emph{Schriftsteller, Mediziner}!Lieutenant Gustl. Novelle1900-12-25@\strich\emph{Lieutenant Gustl. Novelle} {[}1900-12-25{]}|pwkv}
                  übertrieben sittlich-korrekt dargestellt und die antisemitisch geprägte
                  Berichterstattung humorvoll thematisiert.}}}\label{K_L03070-3h} aus Deiner Feder die Aufnahme
               kaum verweigern.\pend
           \pstart
           Ich drücke Dir herzlichſt die \strikeout{H\textcolor{gray}{a}} Hand und grüße Dich in Treue, – obwohl ich es für meinen Theil lebhaft
               bedaure, {\pb}nicht mehr einen k. u. k. Regimentsarzt,
               ſondern einen ganz gemeinen Reſerviſten als Freund zu beſitzen. {\\[\baselineskip]}Dein {\\[\baselineskip]}\spacefill\mbox{Paul Goldmann.}\pend
           \leftskip=0em{}\pstart
           \noindent{}Herzlichen Gruß an Fräulein \textsc{Olga\pwindex{Schnitzler, Olga 17.01.1882 – 13.01.1970@\textsc{Schnitzler, Olga} (17.01.1882 – 13.01.1970), \emph{Schauspielerin, Sängerin}|pw}}!\pend
           
         
         \endnumbering\mylabel{h}\end{ledgroupsized}  \newcommand{\dateiname}{L03070}\newcommand{\titel}{Paul Goldmann an Arthur Schnitzler, 21. 6. [1901]}\newcommand{\editorInnen}{Martin Anton Müller und Laura Untner}%% latex-leseansicht-abspann.tex
%% Abspann für die Leseansicht.
%% Der Schalter \ifkorrekturansicht ist bereits durch den Vorspann gesetzt.

%% latex-abspann.tex
%% Gemeinsamer Abspann für Korrekturansicht und Leseansicht.
%% Setzt den Schalter \ifkorrekturansicht voraus (gesetzt in den
%% einbindenden Dateien latex-korrekturansicht-abspann.tex bzw.
%% latex-leseansicht-abspann.tex).
%% ---------------------------------------------------------------

\normalsize

% Das esempio-Environment wird nur in der Leseansicht benötigt
\ifkorrekturansicht\else
\newenvironment{esempio}[3]%
{
    \vspace{1.5ex}
    \rlap{\underline{#1}}
    \par
    \setlength{\parindent}{0cm}
    \nopagebreak
    \leftskip=#2cm
    \rightskip=#3cm
}
{
    \par
}
\fi

\doendnotes{C}
\bigskip
\vfill

\clearpage

\footnotesize

\ifkorrekturansicht
  \lohead{\textsc{register}}
\fi

% theindex-Environment neu definieren ohne reledmac
\makeatletter
\renewenvironment{theindex}{%
  \ifkorrekturansicht
    \section*{\indexname}%
  \else
    \subsubsection*{Index der erwähnten Entitäten}%
  \fi
  \setlength{\parindent}{0pt}%
  \setlength{\parskip}{0pt plus 0.3pt}%
  \let\item\@idxitem
}{%
  \ifkorrekturansicht\clearpage\fi
}
\makeatother

\IfFileExists{\jobname-pw.ind}{\input{\jobname-pw.ind}}{}

% Quellenangabe nur in der Leseansicht
\ifkorrekturansicht\else
% Fallback-Definitionen, falls die .tex-Datei \titel etc. nicht gesetzt hat
\providecommand{\titel}{}
\providecommand{\editorInnen}{}
\providecommand{\dateiname}{\jobname}

\vspace{3cm}

\vfill

\footnotesize
\textsc{Quelle}: \titel. Herausgegeben von {\editorInnen}. In: \emph{Arthur Schnitzler: Briefwechsel mit Autorinnen und Autoren}.
 Digitale Edition, https://schnitzler-briefe.acdh.oeaw.ac.at/{\dateiname}.html (Stand \today)
\fi

\end{document}


      