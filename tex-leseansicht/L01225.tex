\input{../tex-inputs/latex-pdf-vorspann}
\begin{center}
            \textcolor{red}{ENTWURF. ENTZIFFERUNG NOCH NICHT KORREKTURGELESEN}
                      \end{center}
            
               \section[Hugo von Hofmannsthal an Arthur Schnitzler, 26. 6. 1902]{ Hugo von Hofmannsthal an Arthur Schnitzler, 26. 6. 1902}\nopagebreak\mylabel{v}\rehead{ }\begin{ledgroupsized}[t]{13cm}\normalsize\beginnumbering\briefempfaengerindex{Schnitzler, Arthur@\textsc{Schnitzler, Arthur}!zzzHofmannsthal, Hugo von@\emph{von Hugo von Hofmannsthal}!1902-06-261@{26. 6. 1902}|(be} \toendnotes[C]{\smallbreak\pagebreak[2]} \Standort{CUL, Schnitzler, B 43.}
\physDesc{Postkarte
\newline{}Handschrift: schwarze Tinte, deutsche Kurrent\newline{}Versand: 1) Stempel: »\nobreak{}\oindex{Rodaun@\textbf{Rodaun}|pwk}Rodaun, 26 6 02\nobreak{}«.  2) Stempel: »\nobreak{}\oindex{Salzburg@\textbf{Salzburg}|pwk}Salzburg-Stadt, 27 6 02, 9 F.\nobreak{}«. 
\newline{}Schnitzler: mit Bleistift datiert: »26/6 902.« \newline{}Ordnung: 1) mit Bleistift von unbekannter Hand nummeriert: »\strikeout{198}« 2) mit Bleistift von unbekannter Hand nummeriert: »181«}\buchAbdrucke{\weitereDrucke{Hugo von Hofmannsthal, Arthur Schnitzler: \emph{Briefwechsel}. Hg. Therese Nickl und Heinrich Schnitzler. Frankfurt am Main: \emph{S. Fischer} 1964, S. 159.} }\pstart{}{\pb}\textsc{Herrn D\textsuperscript{r} Arthur Schnitzler}\pend{}\pstart{}\textsc{Salzburg\oindex{Salzburg@\textbf{Salzburg}|pw}}\pend{}\pstart{}\textsc{Hôtel oesterr. Hof\oindex{Oesterreichischer Hof@\textbf{Österreichischer Hof}|pw}}\pend{}{\bigskip}\pstart
           \centering{}{\pb}Donnerstag.\pend
           \pstart
           Da ſich’s ausheitert, hoffe ich – wenn nichts unberechenbares dazwiſchenkommt – falls
               nicht abtelegrafiere, Samstag mit gleichem Zug Ihnen nachzuko{\geminationm}en.\pend
           \pstart Ihr\spacefill\mbox{Hugo.}\pend{}\endnumbering\briefempfaengerindex{Schnitzler, Arthur@\textsc{Schnitzler, Arthur}!zzzHofmannsthal, Hugo von@\emph{von Hugo von Hofmannsthal}!1902-06-261@{26. 6. 1902}|)be}\mylabel{h}\end{ledgroupsized}  \newcommand{\dateiname}{L01225}\newcommand{\titel}{Hugo von Hofmannsthal an Arthur Schnitzler, 26. 6. 1902}\newcommand{\editorInnen}{Martin Anton Müller und Gerd-Hermann Susen}\input{../tex-inputs/latex-pdf-abspann}
      