%% latex-korrekturansicht-vorspann.tex
%% Vorspann für die Korrekturansicht.
%% Lädt die gemeinsame Datei latex-vorspann.tex mit gesetztem Schalter.

\newif\ifkorrekturansicht
\korrekturansichttrue

\input{../tex-inputs/latex-vorspann}


\section[ Paul Goldmann an Arthur Schnitzler, 20. 4. {[}1900{]}]{L02912 Paul Goldmann an Arthur Schnitzler, 20. 4. {[}1900{]}}
\nopagebreak\mylabel{L02912v}
\rehead{ }\normalsize\beginnumbering\briefempfaengerindex{Schnitzler, Arthur@\textsc{Schnitzler, Arthur}!zzzGoldmann, Paul@\emph{von Paul Goldmann}!1900-04-202@{20. 4. {[}1900{]}}|(be}
\toendnotes[C]{\smallbreak\pagebreak[2]}\Standort{DLA, A:Schnitzler, HS.NZ85.1.3170.}
\physDesc{Brief, 1 Blatt, 2 Seiten, 669 Zeichen
\newline{}Handschrift: blaue Tinte, deutsche Kurrent
\newline{}Schnitzler: 1) mit Bleistift das Jahr »900« vermerkt  2) mit rotem Buntstift eine Unterstreichung}\toendnotes[C]{\smallbreak}
\pstart
           {\pb}\textcolor{gray}{\textbf{DESSAUERSTRASSE 19}}\oindex{Dessauer Strasse@\textbf{Dessauer Straße}, \emph{Straße (K.STR)}|pw}\pend
           
\pstart
           \raggedleft{}Berlin\oindex{Berlin@\textbf{Berlin}, \emph{P.PPLC}|pw}, 20. April.\pend
           
\pstart{}Mein lieber Freund,\pend\vspace{0.5em}
\pstart
           Ich danke Dir vielmals für den \label{K_L02912-1v}\edtext{»Reigen\pwindex{Reigen. Zehn Dialoge@\emph{Reigen. Zehn Dialoge}|pw}«}{\lemma{\textnormal{\emph{»Reigen«}}}\Cendnote{\textnormal{Schnitzler hatte einen Privatdruck des \emph{Reigen}\pwindex{Reigen. Zehn Dialoge@\emph{Reigen. Zehn Dialoge}|pwk} herstellen lassen, die Auflage
                  betrug 200 Stück. Diese verschenkte er an Freunde, im Buchhandel war das Werk\pwindex{Reigen. Zehn Dialoge@\emph{Reigen. Zehn Dialoge}|pwkv} erst 1903 zu erstehen.}}}\label{K_L02912-1}. Ich habe es in einem Zuge \label{K_L02912-2v}\edtext{noch einmal durchgeleſen}{\lemma{\textnormal{\emph{noch einmal durchgeleſen}}}\Cendnote{\textnormal{Die erste Lektüre ist nicht belegt. Sie
                  dürfte bei Goldmanns\pwindex{Goldmann, Paul 31.01.1865 – 25.09.1935@\textsc{Goldmann, Paul} (31.01.1865 – 25.09.1935), \emph{Schriftsteller/Schriftstellerin, Journalist/Journalistin}|pwk} Aufenthalt in Wien\oindex{Wien@\textbf{Wien}, \emph{A.ADM2}|pwk} im Oktober 1899
                  stattgefunden haben.}}}\label{K_L02912-2}. Köſtlich, köſtlich! Aber die \strikeout{K\textcolor{gray}{ro}} Krone des Ganzen iſt doch die Schauſpielerin\pwindex{Reigen. Zehn Dialoge@\emph{Reigen. Zehn Dialoge}|pwv}. Eine Figur von unvergleichlicher Komik. Ich
               habe mich geſchüttelt vor Lachen. Wie ſchade, daß dieſes Buch\pwindex{Reigen. Zehn Dialoge@\emph{Reigen. Zehn Dialoge}|pwv}, das zu Deinen beſten gehört, dem
               Publikum \label{K_L02912-3v}\edtext{nicht bekannt werden
                  ſoll}{\lemma{\textnormal{\emph{nicht … ſoll}}}\Cendnote{\textnormal{Schnitzler antizipierte damit den Skandal,
                  den eine reguläre Veröffentlichung und nachmalige Aufführungen mit sich bringen
                  würden.}}}\label{K_L02912-3}! Druck und Ausſtattung ſind vornehm und geſchmackvoll.\pend
           
\pstart
           {\pb}Geſtern ſprach ich \textsc{Gusti Gl}.\pwindex{Gluemer, Auguste 1862-03-16 – 1956@\textsc{Glümer, Auguste} (1862-03-16 – 1956), \emph{Lehrer/Lehrerin}|pw} und ſagte ihr, daß Du nach ihr gefragt
               haſt. Sie antwortete, ſie ſei jetzt nicht in der Stimmung, aber ſie werde Dir ſchon
               ſchreiben. Sieht übrigens angegriffen und gealtert aus.\pend
           
\pstart
           Viele treue Grüße! {\\[\baselineskip]}Dein {\\[\baselineskip]}\spacefill\mbox{Paul Goldmann}\pend
           \leftskip=0em{}\selectlanguage{ngerman}\endnumbering\briefempfaengerindex{Schnitzler, Arthur@\textsc{Schnitzler, Arthur}!zzzGoldmann, Paul@\emph{von Paul Goldmann}!1900-04-202@{20. 4. {[}1900{]}}|)be}\mylabel{L02912h}  \normalsize

\doendnotes{C}
\bigskip
\vfill

\clearpage

\footnotesize

\lohead{\textsc{register}}

% Definiere theindex-Environment komplett neu ohne reledmac
\makeatletter
\renewenvironment{theindex}{%
  \section*{\indexname}%
  \setlength{\parindent}{0pt}%
  \setlength{\parskip}{0pt plus 0.3pt}%
  \let\item\@idxitem
}{%
  \clearpage
}
\makeatother

\IfFileExists{\jobname-pw.ind}{\input{\jobname-pw.ind}}{}

\end{document}

      