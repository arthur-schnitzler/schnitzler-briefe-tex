%% latex-leseansicht-vorspann.tex
%% Vorspann für die Leseansicht.
%% Lädt die gemeinsame Datei latex-vorspann.tex mit nicht gesetztem Schalter.

\newif\ifkorrekturansicht
\korrekturansichtfalse

\input{../tex-inputs/latex-vorspann}


\section[Stefan Zweig an Arthur Schnitzler, {{[}}4.{]} 10. {[}1926?{{]}}]{L03663 Stefan Zweig an Arthur Schnitzler, {[}4.] 10. [1926?{]}}
\nopagebreak\mylabel{L03663v}
\rehead{ }\normalsize\beginnumbering\briefempfaengerindex{Schnitzler, Arthur@\textsc{Schnitzler, Arthur}!zzzZweig, Stefan@\emph{von Stefan Zweig}!1926-10-041@{5. 10. [1926?]}|(be}
\toendnotes[C]{\smallbreak\pagebreak[2]}
\correspDesc{Versand  durch Stefan Zweig am 5. 10. [1926?] in Salzburg
\newline{}Übermittlung  am 6. 10. [1926?] in Salzburg
\newline{}Erhalt  durch Arthur Schnitzler im Zeitraum [7. 10. 1926
                  – 9. 10. 1926?] in Wien}\toendnotes[C]{\smallbreak}
\Standort{CUL, Schnitzler, B 118.}
\physDesc{Bildpostkarte, 531 Zeichen
\newline{}Handschrift: blaue Tinte, lateinische Kurrent
\newline{}Versand: 1) Stempel: »\nobreak{}\oindex{Salzburg@\textbf{Salzburg}, \emph{Verwaltungsgebiet}|pwk}S{[}alzburg{]}, 5. X. [1926], 18\nobreak{}«.   2) Stempel: »\nobreak{}\oindex{Salzburg@\textbf{Salzburg}, \emph{Verwaltungsgebiet}|pwk}Sa{[}lzbu{]}\textcolor{gray}{rg}, 6. X. [1926], 1\textcolor{gray}{9}\nobreak{}«. 
\newline{}Schnitzler: 1) mit Bleistift datiert: »4/10 2{[}6{]}«  2) mit rotem Buntstift eine Unterstreichung}
\buchAbdrucke{\weitereDrucke{Stefan Zweig: \emph{Briefwechsel mit Hermann Bahr, Sigmund Freud, Rainer Maria
                        Rilke und Arthur Schnitzler}. Herausgegeben von Jeffrey B. Berlin, Hans-Ulrich Lindken und Donald A. Prater. Frankfurt am Main: \emph{S. Fischer} 1987, S. 423–424.} }\toendnotes[C]{\smallbreak}\pstart{}{\pb}D\textsuperscript{r} Arthur
                  Schnitzler\pend{}\pstart{}Wien – Cottage\oindex{Wien@\textbf{Wien}!XVIII., Währing@\textbf{XVIII., Währing}!Währinger Cottage@\textbf{Währinger Cottage}, \emph{Teil eines besiedelten Ortes}|pw}\pend{}\pstart{}Sternwartestrasse 71\oindex{Wien@\textbf{Wien}!XVIII., Währing@\textbf{XVIII., Währing}!Sternwartestraße 71@\textbf{Sternwartestraße 71}, \emph{Wohngebäude}|pw}\pend{}{\bigskip}
\pstart
           \noindent{}\centering{}{\pb}\textcolor{gray}{\textbf{Salzburg\oindex{Salzburg@\textbf{Salzburg}, \emph{Verwaltungsgebiet}|pw}}}\pend
           \vspace{1em}
\pstart
           \noindent{}{\pb}Lieber verehrter Herr Doktor, von \label{K_L03663-1v}\edtext{Zermatt\oindex{Zermatt@\textbf{Zermatt}|pw} her}{\lemma{\textnormal{\emph{Zermatt her}}}\Cendnote{\textnormal{Die Jahreszuordnung dieses Korrespondenzstücks ist über
                  Schlussfolgerungen vorzunehmen. Schnitzler
                  vermerkte zwar mit dem »4.« einen bestimmten Tag, es ist aber unklar,
                  woher er die Information besaß, wann es abgefasst war. (Die beiden Poststempel sind mit »5.«
                  respektive »6.« datiert.) Auch vergaß er auf die zweite Ziffer
                  der Jahresangabe. Das erwähnte Treffen in Zermatt\oindex{Zermatt@\textbf{Zermatt}|pwk} fand am 20. 8. 1926 statt. Das fügt sich gut ein und macht aus der
                  vorliegenden Karte die Antwort auf Schnitzlers Brief vom XXXX Auszeichnungsfehler: Dokument L03747 nicht gefunden.}}}\label{K_L03663-1} weiss ich noch, wie gut es ist, mit Ihnen zu
               sprechen und weiss es nun wieder, wie wohltuend auch Ihr geschriebenes Wort einen grüsst: Dankbar
               habe ich Ihren Eindruck und \substVorne{}\textsuperscript{Ihr}\substDazwischen{}sein\substHinten{}e mich sehr anregende Ausführlichkeit empfangen und
               Ihre Zustimmung gilt mir noch genau wie vor zwanzig Jahren als innere Bestärkung. Ich
               werde froh sein, bald Ihnen wieder die Hand reichen zu dürfen, viel Grüsse in alter
               Verehrung voraus!\pend
           
\pstart
           Ihr getreuer{\\[\baselineskip]}\spacefill\mbox{Stefan Zweig}\pend
           \leftskip=0em{}\selectlanguage{ngerman}\endnumbering\briefempfaengerindex{Schnitzler, Arthur@\textsc{Schnitzler, Arthur}!zzzZweig, Stefan@\emph{von Stefan Zweig}!1926-10-041@{5. 10. [1926?]}|)be}\mylabel{L03663h}  \newcommand{\dateiname}{L03663}\newcommand{\titel}{Stefan Zweig an Arthur Schnitzler, [4.] 10. [1926?]}\newcommand{\editorInnen}{Selma Jahnke und Martin Anton Müller}%% latex-leseansicht-abspann.tex
%% Abspann für die Leseansicht.
%% Der Schalter \ifkorrekturansicht ist bereits durch den Vorspann gesetzt.

%% latex-abspann.tex
%% Gemeinsamer Abspann für Korrekturansicht und Leseansicht.
%% Setzt den Schalter \ifkorrekturansicht voraus (gesetzt in den
%% einbindenden Dateien latex-korrekturansicht-abspann.tex bzw.
%% latex-leseansicht-abspann.tex).
%% ---------------------------------------------------------------

\normalsize

% Das esempio-Environment wird nur in der Leseansicht benötigt
\ifkorrekturansicht\else
\newenvironment{esempio}[3]%
{
    \vspace{1.5ex}
    \rlap{\underline{#1}}
    \par
    \setlength{\parindent}{0cm}
    \nopagebreak
    \leftskip=#2cm
    \rightskip=#3cm
}
{
    \par
}
\fi

\doendnotes{C}
\bigskip
\vfill

\clearpage

\footnotesize

\ifkorrekturansicht
  \lohead{\textsc{register}}
\fi

% theindex-Environment neu definieren ohne reledmac
\makeatletter
\renewenvironment{theindex}{%
  \ifkorrekturansicht
    \section*{\indexname}%
  \else
    \subsubsection*{Index der erwähnten Entitäten}%
  \fi
  \setlength{\parindent}{0pt}%
  \setlength{\parskip}{0pt plus 0.3pt}%
  \let\item\@idxitem
}{%
  \ifkorrekturansicht\clearpage\fi
}
\makeatother

\IfFileExists{\jobname-pw.ind}{\input{\jobname-pw.ind}}{}

% Quellenangabe nur in der Leseansicht
\ifkorrekturansicht\else
% Fallback-Definitionen, falls die .tex-Datei \titel etc. nicht gesetzt hat
\providecommand{\titel}{}
\providecommand{\editorInnen}{}
\providecommand{\dateiname}{\jobname}

\vspace{3cm}

\vfill

\footnotesize
\textsc{Quelle}: \titel. Herausgegeben von {\editorInnen}. In: \emph{Arthur Schnitzler: Briefwechsel mit Autorinnen und Autoren}.
 Digitale Edition, https://schnitzler-briefe.acdh.oeaw.ac.at/{\dateiname}.html (Stand \today)
\fi

\end{document}


