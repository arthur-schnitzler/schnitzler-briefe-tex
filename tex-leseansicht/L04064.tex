%% latex-leseansicht-vorspann.tex
%% Vorspann für die Leseansicht.
%% Lädt die gemeinsame Datei latex-vorspann.tex mit nicht gesetztem Schalter.

\newif\ifkorrekturansicht
\korrekturansichtfalse

\input{../tex-inputs/latex-vorspann}


\section[Arthur Schnitzler an Gustav Schwarzkopf, 29. 6. 1905]{L04064 Arthur Schnitzler an Gustav Schwarzkopf, 29. 6. 1905}
\nopagebreak\mylabel{L04064v}
\rehead{ }\normalsize\beginnumbering\briefempfaengerindex{Schwarzkopf, Gustav@\textsc{Schwarzkopf, Gustav}!zzzSchnitzler, Arthur@\emph{von Arthur Schnitzler}!1905-06-291@{29. 6. 1905}|(be}
\toendnotes[C]{\smallbreak\pagebreak[2]}
\correspDesc{Versand  durch Arthur Schnitzler am 29. 6. 1905 in Wien
\newline{}Übermittlung  am 30. 6. 1905 in Wien
\newline{}Erhalt  durch Gustav Schwarzkopf im Zeitraum [29. 6. 1905 – 2. 7. 1905?] in Wien}\toendnotes[C]{\smallbreak}
\Standort{CUL, Schnitzler, B 96.}
\physDesc{Kartenbrief, 409 Zeichen
\newline{}Handschrift: schwarze Tinte, deutsche Kurrent
\newline{}Versand: Stempel: »\nobreak{}\oindex{VII., Neubau@\textbf{VII., Neubau}, \emph{Verwaltungsgebiet}|pwk}7/2 Wien, 30. VI. 05, VIII\nobreak{}«.  }\toendnotes[C]{\smallbreak}\pstart{}{\pb}Herrn Guſtav
                  Schwarzkopf\pend{}\pstart{}Wien I\oindex{I., Innere Stadt@\textbf{I., Innere Stadt}, \emph{Verwaltungsgebiet}|pw}\pend{}\pstart{}Tiefer Graben 23\oindex{Wien@\textbf{Wien}!I., Innere Stadt@\textbf{I., Innere Stadt}!Tiefer Graben 23@\textbf{Tiefer Graben 23}, \emph{Wohngebäude}|pw}.\pend{}{\bigskip}\vspace{1em}
\pstart
           \raggedleft{}{\pb}29. 6. 905\pend
           \vspace{0.5em}
\pstart
           lieber Guſtav,{ }\textsc{Paul Marx\pwindex{Marx, Paul 21.\,7.\,1879 Wien – 30.\,10.\,1956 ebd.@\textsc{Marx, Paul} (21.\,7.\,1879 Wien – 30.\,10.\,1956 ebd.), \emph{Regisseur, Schauspieler}|pw}} möchte Ihnen für Samſtag{ }Juden\pwindex{Čirikov, Evgenij N. 5.\,8.\,1864 Kazan – 18.\,1.\,1932 Prag@\textsc{Čirikov, Evgenij N.} (5.\,8.\,1864 Kazan – 18.\,1.\,1932 Prag), \emph{Schriftsteller}!Juden. Schauspiel in drei Aufzügen@\strich\emph{Die Juden. Schauspiel in drei Aufzügen}|pw}karten\eventindex{Volkstheater@\textbf{Volkstheater}!Aufführung von Die Juden, 1.7.1905@Aufführung von Die Juden, 1.7.1905|pwv}{ }ſchicken; und we{\geminationn} Ihnen der Tag \uline{nicht}
               componirt, bitte ihm ins Deutſche Volkstheater\orgindex{Volkstheater@Volkstheater|pw}
               brieflich abzuwinken.\pend
           
\pstart
           Den Schluſs des namenlosen \label{K_L04064-1v}\edtext{Stückes\pwindex{Schnitzler, Arthur 15. 5. 1862 Wien – 21. 10. 1931 ebd.@\textsc{Schnitzler, Arthur} (15. 5. 1862 Wien – 21. 10. 1931 ebd.), \emph{Schriftsteller, Mediziner}!Zwischenspiel. Komödie in drei Akten@\strich\emph{Zwischenspiel. Komödie in drei Akten}|pwv} hab ich geändert}{\lemma{\textnormal{\emph{Stückes hab ich geändert}}}\Cendnote{\textnormal{Vgl. A. S.: \emph{Tagebuch}, 29. 6. 1905.}}}\label{K_L04064-1}, ſo daſs kein Menſch mehr den \textcolor{gray}{Meiſter\pwindex{Bahr, Hermann 19.\,7.\,1863 Linz – 15.\,1.\,1934 München@\textsc{Bahr, Hermann} (19.\,7.\,1863 Linz – 15.\,1.\,1934 München), \emph{Schriftsteller, Kritiker}!Meister. Komödie in drei Akten@\strich\emph{Der Meister. Komödie in drei Akten}|pw}\pwindex{Bahr, Hermann 19.\,7.\,1863 Linz – 15.\,1.\,1934 München@\textsc{Bahr, Hermann} (19.\,7.\,1863 Linz – 15.\,1.\,1934 München), \emph{Schriftsteller, Kritiker}!Meister. Komödie in drei Akten@\strich\emph{Der Meister. Komödie in drei Akten}|pw}} ahnen wird. In der Hoffnung daſs Sie von dieſem Doppelſinn i{\geminationn}erlich keinen Gebrauch mache\textcolor{gray}{n}\pend
           \pstart Herzlichſt Ihr \spacefill\mbox{A.}\pend{}\selectlanguage{ngerman}\endnumbering\briefempfaengerindex{Schwarzkopf, Gustav@\textsc{Schwarzkopf, Gustav}!zzzSchnitzler, Arthur@\emph{von Arthur Schnitzler}!1905-06-291@{29. 6. 1905}|)be}\mylabel{L04064h}
\begin{anhang}
\end{anhang}\newcommand{\dateiname}{L04064}\newcommand{\titel}{Arthur Schnitzler an Gustav Schwarzkopf, 29. 6. 1905}\newcommand{\editorInnen}{Herausgegeben von Jahnke, SelmaMüller, Martin Anton}%% latex-leseansicht-abspann.tex
%% Abspann für die Leseansicht.
%% Der Schalter \ifkorrekturansicht ist bereits durch den Vorspann gesetzt.

%% latex-abspann.tex
%% Gemeinsamer Abspann für Korrekturansicht und Leseansicht.
%% Setzt den Schalter \ifkorrekturansicht voraus (gesetzt in den
%% einbindenden Dateien latex-korrekturansicht-abspann.tex bzw.
%% latex-leseansicht-abspann.tex).
%% ---------------------------------------------------------------

\normalsize

% Das esempio-Environment wird nur in der Leseansicht benötigt
\ifkorrekturansicht\else
\newenvironment{esempio}[3]%
{
    \vspace{1.5ex}
    \rlap{\underline{#1}}
    \par
    \setlength{\parindent}{0cm}
    \nopagebreak
    \leftskip=#2cm
    \rightskip=#3cm
}
{
    \par
}
\fi

\doendnotes{C}
\bigskip
\vfill

\clearpage

\footnotesize

\ifkorrekturansicht
  \lohead{\textsc{register}}
\fi

% theindex-Environment neu definieren ohne reledmac
\makeatletter
\renewenvironment{theindex}{%
  \ifkorrekturansicht
    \section*{\indexname}%
  \else
    \subsubsection*{Index der erwähnten Entitäten}%
  \fi
  \setlength{\parindent}{0pt}%
  \setlength{\parskip}{0pt plus 0.3pt}%
  \let\item\@idxitem
}{%
  \ifkorrekturansicht\clearpage\fi
}
\makeatother

\IfFileExists{\jobname-pw.ind}{\input{\jobname-pw.ind}}{}

% Quellenangabe nur in der Leseansicht
\ifkorrekturansicht\else
% Fallback-Definitionen, falls die .tex-Datei \titel etc. nicht gesetzt hat
\providecommand{\titel}{}
\providecommand{\editorInnen}{}
\providecommand{\dateiname}{\jobname}

\vspace{3cm}

\vfill

\footnotesize
\textsc{Quelle}: \titel. Herausgegeben von {\editorInnen}. In: \emph{Arthur Schnitzler: Briefwechsel mit Autorinnen und Autoren}.
 Digitale Edition, https://schnitzler-briefe.acdh.oeaw.ac.at/{\dateiname}.html (Stand \today)
\fi

\end{document}


