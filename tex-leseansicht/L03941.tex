%% latex-leseansicht-vorspann.tex
%% Vorspann für die Leseansicht.
%% Lädt die gemeinsame Datei latex-vorspann.tex mit nicht gesetztem Schalter.

\newif\ifkorrekturansicht
\korrekturansichtfalse

\input{../tex-inputs/latex-vorspann}


\section[Arthur Schnitzler an Theodor Herzl, 12.–14. 11. 1892]{L03941 Arthur Schnitzler an Theodor Herzl, 12.–14. 11. 1892}
\nopagebreak\mylabel{L03941v}
\rehead{ }\normalsize\beginnumbering\briefempfaengerindex{Herzl, Theodor@\textsc{Herzl, Theodor}!zzzSchnitzler, Arthur@\emph{von Arthur Schnitzler}!1892-11-142@{12.–14. 11. 1892}|(be}
\toendnotes[C]{\smallbreak\pagebreak[2]}
\correspDesc{Versand  durch Arthur Schnitzler im Zeitraum 12. – 14. 11. 1892 in Wien
\newline{}Erhalt  durch Theodor Herzl im Zeitraum [15. 11. 1892 – 16. 11. 1892?] in Paris}\toendnotes[C]{\smallbreak}
\Standort{Jerusalem, Central Zionist Archives, H1:1924-3.}
\physDesc{Brief, 2 Blätter, 8 Seiten, 3161 Zeichen
\newline{}Handschrift: schwarze Tinte, deutsche Kurrent
\newline{}Ordnung: mit Bleistift von unbekannter Hand innerhalb das Konvoluts paginiert:
                                    »11«–»14« }
\buchAbdrucke{\weitereDrucke{1) \emph{Unveröffentlichtes aus Arthur Schnitzlers Nachlaß.} In: \emph{Neue Zürcher Zeitung. Beilage Literatur und Kunst}, Nr. 91/92, 9. 1. 1966, S. 4–5.} \weitereDrucke{2) Arthur Schnitzler: \emph{Briefe 1875–1912}. Herausgegeben von Therese Nickl und Heinrich Schnitzler. Frankfurt am Main: \emph{S. Fischer} 1981, S. 142–143.} }\toendnotes[C]{\smallbreak}
\pstart
           \raggedleft{}{\pb}\uline{Wien\oindex{Wien@\textbf{Wien}, \emph{Verwaltungsgebiet}|pw}, 12. November 892.}\pend
           
\pstart{}Verehrteſter Freund,\pend\vspace{0.5em}
\pstart
           zuerſt will ich Ihnen für Ihre \label{K_L03941-1v}\edtext{liebenswürdigen Worte}{\lemma{\textnormal{\emph{liebenswürdigen Worte}}}\Cendnote{\textnormal{XXXX Auszeichnungsfehler: Dokument L03825 nicht gefunden.}}}\label{K_L03941-1} herzlich danken, u. da{\geminationn}
               gleich{ }ſagen, wer \textsc{Loris}\pwindex{Hofmannsthal, Hugo von 1.\,2.\,1874 Wien – 15.\,7.\,1929 Rodaun@\textsc{Hofmannsthal, Hugo von} (1.\,2.\,1874 Wien – 15.\,7.\,1929 Rodaun), \emph{Schriftsteller}|pw} ist. Räthſelhaft, daſs
               Sie es von Goldma{\geminationn}\pwindex{Goldmann, Paul 31.\,1.\,1865 Breslau – 25.\,9.\,1935 Wien@\textsc{Goldmann, Paul} (31.\,1.\,1865 Breslau – 25.\,9.\,1935 Wien), \emph{Schriftsteller, Journalist}|pw} nicht wiſſen. Ich{ }ſelber
               bin es leider nicht. Erſtens wäre ich dann um 12 Jahre jünger und zweitens hätte ich
                  »Geſtern\pwindex{Hofmannsthal, Hugo von 1.\,2.\,1874 Wien – 15.\,7.\,1929 Rodaun@\textsc{Hofmannsthal, Hugo von} (1.\,2.\,1874 Wien – 15.\,7.\,1929 Rodaun), \emph{Schriftsteller}!Gestern. Dramatische Studie in einem Akt in Versen@\strich\emph{Gestern. Dramatische Studie in einem Akt in Versen}|pw}« geſchrieben, den{ }ſchönſten Einakter
               in Verſen, der{ }ſeit{ }ſehr,{ }ſehr langer Zeit in deutſcher Sprache erſchienen iſt. Von
               dieſem merk{\pb}würdigen Achtzehnjährigen wird noch{ }ſehr viel geſprochen werden. We{\geminationn} Sie{ }ſchon die Einleitungsverſe\pwindex{Hofmannsthal, Hugo von 1.\,2.\,1874 Wien – 15.\,7.\,1929 Rodaun@\textsc{Hofmannsthal, Hugo von} (1.\,2.\,1874 Wien – 15.\,7.\,1929 Rodaun), \emph{Schriftsteller}!Prolog [zum Anatol]@\strich\emph{Prolog [zum Anatol]}|pwv} zum
                  Anatol\pwindex{Schnitzler, Arthur 15.\,5.\,1862 Wien – 21.\,10.\,1931 ebd.@\textsc{Schnitzler, Arthur} (15.\,5.\,1862 Wien – 21.\,10.\,1931 ebd.), \emph{Schriftsteller, Mediziner}!Anatol@\strich\emph{Anatol}|pw} »zum küſſen« finden,{ }ſo will ich Sie
               vor den unzüchtigen Gedanken warnen, die in Ihnen beim Genuſs{ }ſeiner andern Sachen
               aufſteigen könnten. In Wirklichkeit heißt der Herr Hugo von Hofmannsthal\pwindex{Hofmannsthal, Hugo von 1.\,2.\,1874 Wien – 15.\,7.\,1929 Rodaun@\textsc{Hofmannsthal, Hugo von} (1.\,2.\,1874 Wien – 15.\,7.\,1929 Rodaun), \emph{Schriftsteller}|pw}, hat im Juli maturiert und studiert Jus an der Wr. Univerſität\orgindex{Universität Wien@Universität Wien|pw}. Sie wiſſen ja, verehrteſter, wie
               wenig {\pb}wörtlich das zu nehmen iſt. Wenn es geſtattet ist,{ }ſeiner Biographie
               vorzugreifen,{ }ſo will ich Ihnen auch mittheilen, daſs ich heute Abend nach der \textsc{Première} von Musotte\pwindex{\textcolor{red}{\textsuperscript{XXXX indx1}}!Musotte. Schauspiel in drei Akten@\strich\emph{Musotte. Schauspiel in drei Akten}|pw}\pwindex{\textcolor{red}{\textsuperscript{XXXX indx1}}!Musotte. Schauspiel in drei Akten@\strich\emph{Musotte. Schauspiel in drei Akten}|pw}\eventindex{Volkstheater@\textbf{Volkstheater}!Premiere von Musotte, 12.11.1892@Premiere von Musotte, 12.11.1892|pw} mit ihm{ }ſoupiren und ihm von Ihrem freundlichen Intereſſe erzählen will. Im
               übrigen, fragen Sie doch \label{K_L03941-2v}\edtext{Goldma{\geminationn}\pwindex{Goldmann, Paul 31.\,1.\,1865 Breslau – 25.\,9.\,1935 Wien@\textsc{Goldmann, Paul} (31.\,1.\,1865 Breslau – 25.\,9.\,1935 Wien), \emph{Schriftsteller, Journalist}|pw} nach ihm; – er hat ihn ja
                  entdeckt}{\lemma{\textnormal{\emph{Goldmann … entdeckt}}}\Cendnote{\textnormal{Vgl. XXXX Auszeichnungsfehler: Dokument L00545 nicht gefunden.}}}\label{K_L03941-2}! –\pend
           
\pstart
           – Von Wiener\oindex{Wien@\textbf{Wien}, \emph{Verwaltungsgebiet}|pw} Kunſt{ }ſoll ich Ihnen was berichten? –
               Nun, die literariſche Bewegung äußert{ }ſich darin, daß im Wiedener {\pb}Theater\orgindex{Theater an der Wien@Theater an der Wien|pw} oder Carltheater\orgindex{Carl-Theater@Carl-Theater|pw}{ }\textsc{Couplets} gegen den Naturalismus geſungen werden (»brutal–!«
               »Skandal!«), daſs es keine Verleger, keine neuen Stücke, dagegen{ }ſehr viele
               Kaffeehäuſer gibt, in denen alle Literaten, denen Vormittags nichts eingefallen iſt,
               Nachmittag ihre Gedanken austauſchen. Sitzen zwei zuſa{\geminationm}en,{ }ſo ne{\geminationn}t man{ }ſie eine
                  \label{K_L03941-3v}\edtext{\textsc{Clique}}{\lemma{\textnormal{\emph{Clique}}}\Cendnote{\textnormal{Vgl. A. S.: \emph{Tagebuch}, 9. 10. 1891.}}}\label{K_L03941-3} – und{ }ſitzen gar drei zuſa{\geminationm}en, –{ }ſo{ }ſind{ }ſie es {\pb}wirklich. Man glaubt weder an{ }ſich, noch
               an die andern – und hat großentheils Recht. – Ihr
                  Feuilleton\pwindex{Herzl, Theodor 2.\,5.\,1860 Budapest – 3.\,7.\,1904 Edlach@\textsc{Herzl, Theodor} (2.\,5.\,1860 Budapest – 3.\,7.\,1904 Edlach), \emph{Schriftsteller, Journalist}!Kaffeehaus der »neuen Richtung«@\strich\emph{Kaffeehaus der »neuen Richtung«}|pwv} von dazumal
               fällt mir ein: \label{K_L03941-4v}\edtext{Kaffeehaus der neuen Richtung\pwindex{Herzl, Theodor 2.\,5.\,1860 Budapest – 3.\,7.\,1904 Edlach@\textsc{Herzl, Theodor} (2.\,5.\,1860 Budapest – 3.\,7.\,1904 Edlach), \emph{Schriftsteller, Journalist}!Kaffeehaus der »neuen Richtung«@\strich\emph{Kaffeehaus der »neuen Richtung«}|pw}}{\lemma{\textnormal{\emph{Kaffeehaus … Richtung}}}\Cendnote{\textnormal{Theodor Herzl\pwindex{Herzl, Theodor 2.\,5.\,1860 Budapest – 3.\,7.\,1904 Edlach@\textsc{Herzl, Theodor} (2.\,5.\,1860 Budapest – 3.\,7.\,1904 Edlach), \emph{Schriftsteller, Journalist}|pwk}: \emph{Das Kaffeehaus der »neuen Richtung«}\pwindex{Herzl, Theodor 2.\,5.\,1860 Budapest – 3.\,7.\,1904 Edlach@\textsc{Herzl, Theodor} (2.\,5.\,1860 Budapest – 3.\,7.\,1904 Edlach), \emph{Schriftsteller, Journalist}!Kaffeehaus der »neuen Richtung«@\strich\emph{Kaffeehaus der »neuen Richtung«}|pwk}. In: \emph{Wiener Allgemeine Zeitung}\pwindex{Wiener Allgemeine Zeitung@\emph{Wiener Allgemeine Zeitung}|pwk}, Nr. 783,
                        4. 5. 1882, Morgenblatt, S. 1–4.}}}\label{K_L03941-4} hieß es,
               nicht? – wenn Sie mir gelegentlich dasſelbe{ }ſchicken wollten (Sie haben es doch wohl)
               freute es mich{ }ſehr. Und noch nach einem andern Werk gelüſtet es mich wieder; das
               iſt der Tabarin\pwindex{Herzl, Theodor 2.\,5.\,1860 Budapest – 3.\,7.\,1904 Edlach@\textsc{Herzl, Theodor} (2.\,5.\,1860 Budapest – 3.\,7.\,1904 Edlach), \emph{Schriftsteller, Journalist}!Tabarin. Schauspiel in einem Act. Frei nach Catulle Mendès@\strich\emph{Tabarin. Schauspiel in einem Act. Frei nach Catulle Mendès}|pw}. Nun aber will ich noch mit
               einer ganz beſonderen Bitte heraus (die {\pb}einleitenden \substVorne{}\textsuperscript{\textcolor{gray}{F}}\substDazwischen{}P\substHinten{}hraſen{ }ſchenken Sie mir ja) ich
               möchte{ }ſehr gern diejenigen Ihrer Stücke leſen, auf die Sie{ }ſelbſt was halten u. die
               \uline{nicht} aufgeführt worden sind. – Sie würden meinem literariſchen u perſönlichen
               Intereſſe in gleicher Weiſe durch Berückſichtigg dieſes Erſuchens
               entgegenko{\geminationm}en. \textcolor{gray}{–}\pend
           
\pstart
           – Ihre Schlußpointe zu den Weihnachtseinkäufen\pwindex{Schnitzler, Arthur 15.\,5.\,1862 Wien – 21.\,10.\,1931 ebd.@\textsc{Schnitzler, Arthur} (15.\,5.\,1862 Wien – 21.\,10.\,1931 ebd.), \emph{Schriftsteller, Mediziner}!Weihnachts-Einkäufe@\strich\emph{Weihnachts-Einkäufe}|pw}
               gefällt mir vorzüglich; nur glaub’ ich wär{ }ſie aus der einen Scene{ }ſchwierig
               herauszuentwickeln. Es wäre überhaupt {\pb}was andres; in Ihrer Pointe liegt ganz einfach
               ein{ }ſehr reizendes Luſt- oder vielleicht gar Schauſpiel verſteckt, welches zu{ }ſchreiben Sie höflichſt gebeten werden. – Neugierig bin ich, ob Sie eins von den
               Dingen bühnenwirkſam finden werden. –\pend
           \selectlanguage{ngerman}\vspace{1em}
\pstart
           \raggedleft{}14. 11.\pend
           \vspace{0.5em}
\pstart
           Ich wurde neulich unterbrochen, u. ko{\geminationm}e erſt heute zum Abschluſs meines Briefes\pend
           
\pstart
           Laſſen Sie mich Ihnen alſo nur noch einmal{ }ſagen, wie{ }ſehr mich Ihre Freundlichkeit
               und Antheil{\pb}nahme ehrt und wie es mich freuen würde, bald wieder was von Ihnen zu
               hören. Sie haben mir nun \label{K_L03941-5v}\edtext{zwei
                  Briefe}{\lemma{\textnormal{\emph{zwei
                  Briefe}}}\Cendnote{\textnormal{XXXX Auszeichnungsfehler: Dokument L03823 nicht gefunden, XXXX Auszeichnungsfehler: Dokument L03825 nicht gefunden.}}}\label{K_L03941-5} über mich geſchrieben; ich darf nun wohl
               einen über Sie erwarten?\pend
           \pstart Mit herzlichen Grüßen Ihr{ }ſehr ergebner \spacefill\mbox{Arthur Schnitzler}\pend{}
\pstart
           \noindent{}I Grillparzerstrasse 7\oindex{Wien@\textbf{Wien}!I., Innere Stadt@\textbf{I., Innere Stadt}!Wohnung und Ordination Arthur Schnitzler Grillparzerstraße 7/3. Stock@\textbf{Wohnung und Ordination Arthur Schnitzler Grillparzerstraße 7/3. Stock}, \emph{Ordination}|pw}.\pend
           \selectlanguage{ngerman}\endnumbering\briefempfaengerindex{Herzl, Theodor@\textsc{Herzl, Theodor}!zzzSchnitzler, Arthur@\emph{von Arthur Schnitzler}!2@{12.–14. 11. 1892}|)be}\mylabel{L03941h}
\begin{anhang}
\end{anhang}\newcommand{\dateiname}{L03941}\newcommand{\titel}{Arthur Schnitzler an Theodor Herzl, 12. – 14. 11. 1892}\newcommand{\editorInnen}{Herausgegeben von Jahnke, SelmaMüller, Martin Anton}%% latex-leseansicht-abspann.tex
%% Abspann für die Leseansicht.
%% Der Schalter \ifkorrekturansicht ist bereits durch den Vorspann gesetzt.

%% latex-abspann.tex
%% Gemeinsamer Abspann für Korrekturansicht und Leseansicht.
%% Setzt den Schalter \ifkorrekturansicht voraus (gesetzt in den
%% einbindenden Dateien latex-korrekturansicht-abspann.tex bzw.
%% latex-leseansicht-abspann.tex).
%% ---------------------------------------------------------------

\normalsize

% Das esempio-Environment wird nur in der Leseansicht benötigt
\ifkorrekturansicht\else
\newenvironment{esempio}[3]%
{
    \vspace{1.5ex}
    \rlap{\underline{#1}}
    \par
    \setlength{\parindent}{0cm}
    \nopagebreak
    \leftskip=#2cm
    \rightskip=#3cm
}
{
    \par
}
\fi

\doendnotes{C}
\bigskip
\vfill

\clearpage

\footnotesize

\ifkorrekturansicht
  \lohead{\textsc{register}}
\fi

% theindex-Environment neu definieren ohne reledmac
\makeatletter
\renewenvironment{theindex}{%
  \ifkorrekturansicht
    \section*{\indexname}%
  \else
    \subsubsection*{Index der erwähnten Entitäten}%
  \fi
  \setlength{\parindent}{0pt}%
  \setlength{\parskip}{0pt plus 0.3pt}%
  \let\item\@idxitem
}{%
  \ifkorrekturansicht\clearpage\fi
}
\makeatother

\IfFileExists{\jobname-pw.ind}{\input{\jobname-pw.ind}}{}

% Quellenangabe nur in der Leseansicht
\ifkorrekturansicht\else
% Fallback-Definitionen, falls die .tex-Datei \titel etc. nicht gesetzt hat
\providecommand{\titel}{}
\providecommand{\editorInnen}{}
\providecommand{\dateiname}{\jobname}

\vspace{3cm}

\vfill

\footnotesize
\textsc{Quelle}: \titel. Herausgegeben von {\editorInnen}. In: \emph{Arthur Schnitzler: Briefwechsel mit Autorinnen und Autoren}.
 Digitale Edition, https://schnitzler-briefe.acdh.oeaw.ac.at/{\dateiname}.html (Stand \today)
\fi

\end{document}


