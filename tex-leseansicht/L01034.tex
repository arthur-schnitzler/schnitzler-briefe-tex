%% latex-leseansicht-vorspann.tex
%% Vorspann für die Leseansicht.
%% Lädt die gemeinsame Datei latex-vorspann.tex mit nicht gesetztem Schalter.

\newif\ifkorrekturansicht
\korrekturansichtfalse

\input{../tex-inputs/latex-vorspann}


               \section[Arthur Schnitzler an Georg Brandes, 3. 5. 1900]{ Arthur Schnitzler an Georg Brandes, 3. 5. 1900}\nopagebreak\mylabel{v}\rehead{ }\begin{ledgroupsized}[t]{13cm}\normalsize\beginnumbering\briefempfaengerindex{Brandes, Georg@\textsc{Brandes, Georg}!zzzSchnitzler, Arthur@\emph{von Arthur Schnitzler}!1900-05-031@{3. 5. 1900}|(be} \toendnotes[C]{\smallbreak\pagebreak[2]} \Standort{Kopenhagen, Det Kongelige Bibliotek, Georg Brandes Arkiv, box 125.}
\physDesc{Brief, 3 Blätter, 10 Seiten
\newline{}Handschrift: schwarze Tinte, deutsche Kurrent\newline{}Ordnung: auf der ersten Seite von unbekannter Hand mit Bleistift
                                 nummeriert: »20. \textsc{Schnitzler}« und datiert: »3/5 00«, die Datierung jeweils auf den ersten Seiten der weiteren
                                 Blätter mit Bleistift wiederholt, diesmal in Verbindung mit einem
                                 vorangestellten »?« }\buchAbdrucke{\weitereDrucke{1) Georg Brandes, Arthur Schnitzler: \emph{Ein Briefwechsel}. Hg. Kurt Bergel. Bern: \emph{Francke} 1956, S. 81–83.} \weitereDrucke{2) Arthur Schnitzler: \emph{Briefe 1875–1912}. Hg. Therese Nickl und Heinrich Schnitzler. Frankfurt am Main: \emph{S. Fischer} 1981, S. 382–384.} }\toendnotes[C]{\smallbreak}\pstart{}{\pb}Mein lieber und verehrter Herr
                  Brandes,\pend\pstart
           ſchon vor einigen Tagen las ich in einer Zeitung, daſs Sie ſich wieder \label{K_L01034_1v}\edtext{leidend befinden und in ein Sanatorium\oindex{Kommunehospitalet@\textbf{Kommunehospitalet}|pwv}}{\lemma{\textnormal{\emph{leidend … Sanatorium}}}\Cendnote{\textnormal{Vermutlich bezieht er sich auf diese
                  Meldung: [O. V.:] \emph{Personal-Nachrichten.
                        [Dr. Georg Brandes]}\pwindex{?? Werk@Nicht ermittelte Verfasserinnen und Verfasser!Personal-Nachrichten [Dr. Georg Brandes]24. 04. 1900@\emph{Personal-Nachrichten [Dr. Georg Brandes]} {[}24. 04. 1900{]}|pwk}. In: \emph{Neue Freie
                        Presse}\pwindex{Neue Freie Presse1864 – 1939@\emph{Neue Freie Presse}|pwk}, Nr. 12811, 24. 4. 1900, S. 6:
                        »Dr. Georg \so{Brandes}\pwindex{Brandes, Georg 04.02.1842 – 19.02.1927@\textsc{Brandes, Georg} (04.02.1842 – 19.02.1927)|pw}, dessen rheumatisches Leiden wieder heftiger aufgetreten ist, hat sich,
                     um eine so sachverständige und sorgfältige Behandlung als möglich zu finden, in
                     das Commune-Hospital in
                        Kopenhagen\oindex{Kommunehospitalet@\textbf{Kommunehospitalet}|pwv} begeben. Sein Zustand gibt nicht zu Besorgnissen
                     Anlaß.«}}}\label{K_L01034_1h} gegangen wären; aber nach dem ganzen Tun u auch nach der
               Schrift Ihres Briefes ſcheint mir, daſs die Krankheit diesmal leichter auftritt als
               die erſten Male, und hoffentlich ſtehn Sie bald wieder auf und ſind endlich ganz
               geſund. Es iſt gewiſs ein gutes Zeichen, wenn \label{K_L01034_2v}\edtext{Recidive}{\lemma{\textnormal{\emph{Recidive}}}\Cendnote{\textnormal{Rückfall}}}\label{K_L01034_2h} in abgeſchwächter Form auftreten; {\pb}ich wünſche von Herzen, daſs es das letzte iſt. –
               Sehr bedauert hab ich dſs ich in Abbazia\oindex{Opatija@\textbf{Opatija}|pw} Ihren
               Abſagebrief fand nicht Sie ſelbſt. Ich habe auf der dalmatiniſchen\oindex{Dalmatien@\textbf{Dalmatien}|pw} Reiſe meiſt ſchlechtes Wetter gehabt; nur in Raguſa\oindex{Dubrovnik@\textbf{Dubrovnik}|pw} zwei ſonnige Tage; überdies gerieth ich anfangs in einen
                  \label{K_L01034_3v}\edtext{Balneologen}{\lemma{\textnormal{\emph{Balneologen}}}\Cendnote{\textnormal{Balneologie: die Lehre von den
                  Heilbädern.}}}\label{K_L01034_3h}congreſs, deſſen Mitglieder Schiffe und Hotels füllten, von
               denen ich auch manche perſönlich kannte, es war ziemlich unangenehm. Unter ſolchen
                  Halb{\pb}bekannten ſein iſt die ſchli{\geminationm}ſte Form – der Einſamkeit, nicht der Geſelligkeit. Von
                  Abbazia\oindex{Opatija@\textbf{Opatija}|pw} aus, wo es ununterbrochen regnete,
               flüchtete ich bald nach Hauſe. Das ſchönſte was ich mitbrachte, iſt die Eri{\geminationn}erung an die Trümmer von Salona\oindex{Solin@\textbf{Solin}|pw}, ich ka{\geminationn} gar nicht verſtehen, warum man
               da nicht immer und immer weitergräbt; die Erde wegkratzen und die Vergangenheit
               finden – wie ko{\geminationm}t es, dſs darüber noch keiner wahnſi{\geminationn}ig {\pb}geworden
               iſt? –\pend
           \pstart
           Auch die albernen Angriffe gegen Sie wegen Ihrer \label{K_L01034_4v}\edtext{Budapeſt\oindex{Budapest@\textbf{Budapest}|pw}er Einleitung}{\lemma{\textnormal{\emph{Budapeſter Einleitung}}}\Cendnote{\textnormal{Möglicherweise bezieht sich Schnitzler auf diese Meldung:
                     [O. V.:] \emph{Ein recht ungezogener Mensch}\pwindex{?? Werk@Nicht ermittelte Verfasserinnen und Verfasser!recht ungezogener Mensch1900-04-15 – 1900-04-15@\emph{Ein recht ungezogener Mensch} {[}1900-04-15 – 1900-04-15{]}|pwk}.
                     In: \emph{Arbeiter-Zeitung}\pwindex{Arbeiter-Zeitung12.7.1881 – 31.10.1991@\emph{Arbeiter-Zeitung}|pwk}, Nr. 103,
                        15. 4. 1900, S. 6–7, hier S. 6: »Ein recht
                     ungezogener Mensch scheint Herr Georg \so{Brandes}\pwindex{Brandes, Georg 04.02.1842 – 19.02.1927@\textsc{Brandes, Georg} (04.02.1842 – 19.02.1927)|pw}, der dänische\oindex{Daenemark@\textbf{Dänemark}|pw} Literaturkritiker, zu
                     sein. Er hielt am letzten des vorigen Monats in einem Budapester Klub\oindex{Leopoldstaedter Kasino@\textbf{Leopoldstädter Kasino}|pwv} einen Vortrag über Ibsen\pwindex{Ibsen, Henrik 20.03.1828 – 23.05.1906@\textsc{Ibsen, Henrik} (20.03.1828 – 23.05.1906), \emph{Schriftsteller}|pw}. Da Herr Brandes\pwindex{Brandes, Georg 04.02.1842 – 19.02.1927@\textsc{Brandes, Georg} (04.02.1842 – 19.02.1927)|pw} nicht ungarisch\oindex{Ungarn@\textbf{Ungarn}|pw}{ }spricht, die Budapest\oindex{Budapest@\textbf{Budapest}|pw}er aber wenig dänisch\oindex{Daenemark@\textbf{Dänemark}|pw}
                     verstehen, so sprach Herr Brandes\pwindex{Brandes, Georg 04.02.1842 – 19.02.1927@\textsc{Brandes, Georg} (04.02.1842 – 19.02.1927)|pw} –
                     natürlich \so{deutsch}. Er begann nun seine Rede mit
                     folgenden Worten: ›Meine Damen und Herren! Die Sprache, in der ich zu ihnen
                     rede, ist nicht die ihrige, und sie ist auch nicht die meine. Ich gestehe, \so{daß ich die deutsche Sprache nicht sehr liebe}; wie
                     ich weiß, ist sie \so{auch bei ihnen nicht sehr beliebt}.
                     Allein dieses einemal muß ich mich ihrer dennoch bedienen, denn schließlich ist
                     es doch die Hauptsache, daß wir einander verstehen. Ich habe das Deutsche erst
                     in meinem 30. Lebensjahr gelernt, und obwohl ich es vollkommen beherrsche, so
                     ist doch meine Aussprache mangelhaft. Deshalb ist es keine Phrase, wenn ich um
                     Nachsicht bitte.‹ Man braucht nicht viel Worte zu machen, um zu sagen, was das
                     ist, dessen sich Herr Brandes\pwindex{Brandes, Georg 04.02.1842 – 19.02.1927@\textsc{Brandes, Georg} (04.02.1842 – 19.02.1927)|pw} hier schuldig
                     gemacht hat: eine \so{Unanständigkeit}. Niemand hat
                     weniger Anlaß, über das deutsche Volk Klage zu führen, wie Herr Brandes\pwindex{Brandes, Georg 04.02.1842 – 19.02.1927@\textsc{Brandes, Georg} (04.02.1842 – 19.02.1927)|pw}, der in deutschen
                     Schriftstellerkreisen stets mit der größten Unbefangenheit und mit warmem
                     Wohlwollen aufgenommen worden ist. Es ist also eine Unziemlichkeit sehr arger
                     Art, wenn Herr Brandes\pwindex{Brandes, Georg 04.02.1842 – 19.02.1927@\textsc{Brandes, Georg} (04.02.1842 – 19.02.1927)|pw}, der kurz vorher in
                        Wien\oindex{Wien@\textbf{Wien}|pw} der deutschen Sprache so große
                     Komplimente gemacht hat, den deutschfresserischen Instinkten der Budapest\oindex{Budapest@\textbf{Budapest}|pw}er Clique so niedrige Konzessionen
                     bereitet.«}}}\label{K_L01034_4h} habe ich geleſen. Es iſt ja wirklich gar nicht
               ernſthaft darüber zu reden. Und doch ſcheint es, ka{\geminationn} man
               die Empfindlichkeit gegenüber dem dü{\geminationm}ſten, we{\geminationn} es nur einmal gedruckt iſt, nicht ganz verlieren. Ich
               erinnere mich, wie ich ſeinerzeit mit einigem Staunen im Briefwechſel von Goethe\pwindex{Goethe, Johann Wolfgang von 28.08.1749 – 22.03.1832@\textsc{Goethe, Johann Wolfgang von} (28.08.1749 – 22.03.1832), \emph{Schriftsteller}|pw} und Schiller\pwindex{Schiller, Friedrich von 10.11.1759 – 09.05.1805@\textsc{Schiller, Friedrich von} (10.11.1759 – 09.05.1805), \emph{Schriftsteller, Historiker, Philosoph}|pw} Denkmäler ihres Aergers über die nichtigſten Scribenten antraf.
               Seither ſtaune ich {\pb}aber nicht mehr, we{\geminationn} ich ſehe, wie ſich zuweilen die Klügſten über die
               Thörichteſten ärgern. Die Philoſophie hilft wohl gegen die Todesangſt, aber nicht
               gegen Flohſtiche.\pend
           \pstart
           Daſs Sie auch mir für Wien\oindex{Wien@\textbf{Wien}|pw} danken, iſt zu
               liebenswürdig; ich fühle, daſs ich Ihnen, beſonders diesmal, nicht viel ſein konnte.
               Im Anfang waren dieſe langweiligen Zahngeſchichten; und dann liegen die Schatten von
               jenem traurigen Ereignis oft, und nun gar in dieſen Frühlingstagen ſchwer auf meiner
               Seele. Dazu kommen noch mancherlei zum {\pb}Theil
               nervöſe Dinge (aber nur zum Theil), über die ich nicht gern rede, hauptſächlich ein
               quälendes Ohrenſauſen, an dem ich nun ſeit drei einhalb Jahren ununterbrochen leide,
               mit beginnender Verſchlechterung des Gehörs – das macht mich natürlich auch nicht
               viel froher. Immerhin arbeite ich ſeit einiger Zeit mehr als je und mit einer
               Empfindung – wenigſtens zuweilen – von innerer Fülle wie niemals früher. Ich bin
               jetzt daran eine Novelle\pwindex{Schnitzler, Arthur 15.05.1862 – 21.10.1931@\textsc{Schnitzler, Arthur} (15.05.1862 – 21.10.1931), \emph{Schriftsteller, Mediziner}!Frau Bertha Garlan. Roman15.1.1901 – 15.3.1901@\strich\emph{Frau Bertha Garlan. Roman} {[}15.1.1901 – 15.3.1901{]}|pwv} zu
               dictiren, die vor ein paar Wochen beendet wurde, ſchreibe jetzt einige {\pb}kleinere und möchte im Sommer eine Komödie
               ſchreiben. Der Schleier der \textsc{Beatrice}\pwindex{Schnitzler, Arthur 15.05.1862 – 21.10.1931@\textsc{Schnitzler, Arthur} (15.05.1862 – 21.10.1931), \emph{Schriftsteller, Mediziner}!Schleier der Beatrice. Schauspiel in fuenf Akten1900-12-01 – 1900-12-01@\strich\emph{Der Schleier der Beatrice. Schauspiel in fünf Akten} {[}1900-12-01 – 1900-12-01{]}|pw} wird wahrſcheinlich im \substVorne{}\textsuperscript{Sommer}{\allowbreak}\substDazwischen{}Herbſt\substHinten{} an der Burg\oindex{Burgtheater@\textbf{Burgtheater}|pw} aufgeführt; wo ich aber mit den
               neuen Sachen hin ſoll die ich im Kopf habe weiſs ich nicht recht. Es wird nemlich
               kaum möglich ſein in der nächſten Zeit etwas wien\oindex{Wien@\textbf{Wien}|pw}eriſches zu ſchreiben, in das nicht die antiſemitiſche Frage hineinſpielt –
               und meine Art darüber zu denken wird weder den Chriſten noch den Juden recht ſein. –
               Das neue Buch\pwindex{Bourget, Paul 02.09.1852 – 25.12.1935@\textsc{Bourget, Paul} (02.09.1852 – 25.12.1935), \emph{Schriftsteller}!Familiendramen1900@\strich\emph{Familiendramen} {[}1900{]}|pwv} von \textsc{Bour{\pb}get}\pwindex{Bourget, Paul 02.09.1852 – 25.12.1935@\textsc{Bourget, Paul} (02.09.1852 – 25.12.1935), \emph{Schriftsteller}|pw} ke{\geminationn} ich nicht, habe ſchon lange nicht von ihm
               geleſen; auch das Reiſewerk\pwindex{Lanckoroński, Karl 04.11.1848 – 15.07.1934@\textsc{Lanckoroński, Karl} (04.11.1848 – 15.07.1934), \emph{Schriftsteller, Sammler, Forscher}!Rund um die Erde 1888–891891@\strich\emph{Rund um die Erde 1888–89} {[}1891{]}|pwv} von
                  \textsc{Lanckoronsky}\pwindex{Lanckoroński, Karl 04.11.1848 – 15.07.1934@\textsc{Lanckoroński, Karl} (04.11.1848 – 15.07.1934), \emph{Schriftsteller, Sammler, Forscher}|pw} iſt mir noch unbekannt. Ich leſe jetzt – denken Sie! zum erſten Mal – we{\geminationn} ich von einer Jugendbearbeitung abſehe – den \textsc{Don Quixote}\pwindex{\textcolor{red}{\textsuperscript{XXXX1 indx}}!Don Quijote1605@\strich\emph{Don Quijote} {[}1605{]}|pw}; da{\geminationn} ein vorzügliches Buch\pwindex{Federn, Karl 02.02.1868 – 22.03.1943@\textsc{Federn, Karl} (02.02.1868 – 22.03.1943), \emph{Schriftsteller/Schriftstellerin, Übersetzer/Übersetzerin}!Dante1899@\strich\emph{Dante} {[}1899{]}|pwv} über \textsc{\uline{Dante}}\pwindex{Dante Alighieri um 1265 – 22.09.1321@\textsc{Dante Alighieri} (um 1265 – 22.09.1321), \emph{Schriftsteller}|pw} von \textsc{Federn}\pwindex{Federn, Karl 02.02.1868 – 22.03.1943@\textsc{Federn, Karl} (02.02.1868 – 22.03.1943), \emph{Schriftsteller/Schriftstellerin, Übersetzer/Übersetzerin}|pw}, demſelben, der den \textsc{Emerson}\pwindex{Emerson, Ralph Waldo 25.05.1803 – 27.04.1882@\textsc{Emerson, Ralph Waldo} (25.05.1803 – 27.04.1882), \emph{Schriftsteller}|pw} trefflich überſetzt\pwindex{Emerson, Ralph Waldo 25.05.1803 – 27.04.1882@\textsc{Emerson, Ralph Waldo} (25.05.1803 – 27.04.1882), \emph{Schriftsteller}!Essays1894@\strich\emph{Essays} {[}1894{]}|pwv} hat.
                  \textsc{Gibbon}\pwindex{Gibbon, Edward 08.05.1737 – 16.01.1794@\textsc{Gibbon, Edward} (08.05.1737 – 16.01.1794), \emph{Politiker, Historiker}|pw}\pwindex{Gibbon, Edward 08.05.1737 – 16.01.1794@\textsc{Gibbon, Edward} (08.05.1737 – 16.01.1794), \emph{Politiker, Historiker}!Verfall und Untergang des Roemischen Reiches1776 – 1789@\strich\emph{Verfall und Untergang des Römischen Reiches} {[}1776 – 1789{]}|pwv} begleitet mich bereits längere Zeit.\pend
           \pstart
           Seit das Wetter ſchön iſt, radl ich auch manchmal aufs Land, und für den Sommer hab
               ich {\pb}größere Touren auf dem Rad vor. Vielleicht
               entſchließen Sie ſich einmal, in der heißen Zeit ins Gebirge zu gehen; ich habe mich
               ſchon darauf  gefreut,
               einmal mit Ihnen im Freien zu ſein, außerhalb von Stadt und Mauern herumzuſpaziren.
               Vielleicht läßt es ſich gar machen, dſs Sie, Goldmann\pwindex{Goldmann, Paul 31.01.1865 – 25.09.1935@\textsc{Goldmann, Paul} (31.01.1865 – 25.09.1935), \emph{Schriftsteller, Journalist}|pw} und Beer Hofma{\geminationn}\pwindex{Beer-Hofmann, Richard 11.07.1866 – 26.09.1945@\textsc{Beer-Hofmann, Richard} (11.07.1866 – 26.09.1945), \emph{Schriftsteller}|pw} u ich irgendwo zuſammentreffen, fern von allen Zeitungen – und am Ende auch von
               aller »Lit\damage{\textcolor{gray}{era}}tur«. –\pend
           \pstart
           Jedenfalls hoff ich Sie ſagen mir bald wieder ein Wort, wies Ihnen {\pb}geht. Es iſt eine meiner wirklichen Freuden, daſs
               Sie meiner mit Sympathie gedenken. Ich grüße Sie herzlich.\pend
           \pstart Ihr \spacefill\mbox{Arthur Schnitzler}\pend{}\pstart
           Wien\oindex{Wien@\textbf{Wien}|pw}, 3. 5. 900.\pend
           \endnumbering\briefempfaengerindex{Brandes, Georg@\textsc{Brandes, Georg}!zzzSchnitzler, Arthur@\emph{von Arthur Schnitzler}!1900-05-031@{3. 5. 1900}|)be}\mylabel{h}\end{ledgroupsized}  \newcommand{\dateiname}{L01034}\newcommand{\titel}{Arthur Schnitzler an Georg Brandes, 3. 5. 1900}\newcommand{\editorInnen}{Martin Anton Müller und Gerd-Hermann Susen}%% latex-leseansicht-abspann.tex
%% Abspann für die Leseansicht.
%% Der Schalter \ifkorrekturansicht ist bereits durch den Vorspann gesetzt.

%% latex-abspann.tex
%% Gemeinsamer Abspann für Korrekturansicht und Leseansicht.
%% Setzt den Schalter \ifkorrekturansicht voraus (gesetzt in den
%% einbindenden Dateien latex-korrekturansicht-abspann.tex bzw.
%% latex-leseansicht-abspann.tex).
%% ---------------------------------------------------------------

\normalsize

% Das esempio-Environment wird nur in der Leseansicht benötigt
\ifkorrekturansicht\else
\newenvironment{esempio}[3]%
{
    \vspace{1.5ex}
    \rlap{\underline{#1}}
    \par
    \setlength{\parindent}{0cm}
    \nopagebreak
    \leftskip=#2cm
    \rightskip=#3cm
}
{
    \par
}
\fi

\doendnotes{C}
\bigskip
\vfill

\clearpage

\footnotesize

\ifkorrekturansicht
  \lohead{\textsc{register}}
\fi

% theindex-Environment neu definieren ohne reledmac
\makeatletter
\renewenvironment{theindex}{%
  \ifkorrekturansicht
    \section*{\indexname}%
  \else
    \subsubsection*{Index der erwähnten Entitäten}%
  \fi
  \setlength{\parindent}{0pt}%
  \setlength{\parskip}{0pt plus 0.3pt}%
  \let\item\@idxitem
}{%
  \ifkorrekturansicht\clearpage\fi
}
\makeatother

\IfFileExists{\jobname-pw.ind}{\input{\jobname-pw.ind}}{}

% Quellenangabe nur in der Leseansicht
\ifkorrekturansicht\else
% Fallback-Definitionen, falls die .tex-Datei \titel etc. nicht gesetzt hat
\providecommand{\titel}{}
\providecommand{\editorInnen}{}
\providecommand{\dateiname}{\jobname}

\vspace{3cm}

\vfill

\footnotesize
\textsc{Quelle}: \titel. Herausgegeben von {\editorInnen}. In: \emph{Arthur Schnitzler: Briefwechsel mit Autorinnen und Autoren}.
 Digitale Edition, https://schnitzler-briefe.acdh.oeaw.ac.at/{\dateiname}.html (Stand \today)
\fi

\end{document}


      