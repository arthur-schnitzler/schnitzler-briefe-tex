%% latex-leseansicht-vorspann.tex
%% Vorspann für die Leseansicht.
%% Lädt die gemeinsame Datei latex-vorspann.tex mit nicht gesetztem Schalter.

\newif\ifkorrekturansicht
\korrekturansichtfalse

\input{../tex-inputs/latex-vorspann}


\section[Arthur Schnitzler an Georg Brandes, 3. 5. 1900]{L01034 Arthur Schnitzler an Georg Brandes, 3. 5. 1900}
\nopagebreak\mylabel{L01034v}
\rehead{ }\normalsize\beginnumbering\briefempfaengerindex{Brandes, Georg@\textsc{Brandes, Georg}!zzzSchnitzler, Arthur@\emph{von Arthur Schnitzler}!1900-05-031@{3. 5. 1900}|(be}
\toendnotes[C]{\smallbreak\pagebreak[2]}
\correspDesc{Versand  durch Arthur Schnitzler am 3. 5. 1900 in Wien
\newline{}Erhalt  durch Georg Brandes im Zeitraum [4. 5. 1900
                  – 8. 5. 1900?] in Kopenhagen}\toendnotes[C]{\smallbreak}
\Standort{Kopenhagen, Det Kongelige Bibliotek, Georg Brandes Arkiv, box 125.}
\physDesc{Brief, 3 Blätter, 10 Seiten, 4265 Zeichen
\newline{}Handschrift: schwarze Tinte, deutsche Kurrent
\newline{}Ordnung: auf der ersten Seite von unbekannter Hand mit Bleistift
                                 nummeriert: »20. \textsc{Schnitzler}« und datiert: »3/5 00«, die Datierung jeweils auf den ersten Seiten der weiteren
                                 Blätter mit Bleistift wiederholt, diesmal in Verbindung mit einem
                                 vorangestellten »?« }
\buchAbdrucke{\weitereDrucke{1) Georg Brandes, Arthur Schnitzler: \emph{Ein Briefwechsel}. Herausgegeben von Kurt Bergel. Bern: \emph{Francke} 1956, S. 81–83.} \weitereDrucke{2) Arthur Schnitzler: \emph{Briefe 1875–1912}. Herausgegeben von Therese Nickl und Heinrich Schnitzler. Frankfurt am Main: \emph{S. Fischer} 1981, S. 382–384.} }\toendnotes[C]{\smallbreak}
\pstart{}{\pb}Mein lieber und verehrter Herr
                  Brandes,\pend\vspace{0.5em}
\pstart
           ſchon vor einigen Tagen las ich in einer Zeitung, daſs Sie{ }ſich wieder \label{K_L01034-1v}\edtext{leidend befinden und in ein Sanatorium\oindex{Kommunehospitalet@\textbf{Kommunehospitalet}, \emph{Krankenhaus}|pwv}}{\lemma{\textnormal{\emph{leidend … Sanatorium}}}\Cendnote{\textnormal{Vermutlich bezieht Schnitzler sich auf diese
                  Meldung: [O. V.]: \emph{Personal-Nachrichten.
                        [Dr. Georg Brandes]}\pwindex{Personal-Nachrichten [Dr. Georg Brandes]@\emph{Personal-Nachrichten [Dr. Georg Brandes]}|pwk}. In: \emph{Neue Freie
                        Presse}\pwindex{Neue Freie Presse@\emph{Neue Freie Presse}|pwk}, Nr. 12.811, 24. 4. 1900, S. 6:
                        »Dr. Georg \so{Brandes}\pwindex{Brandes, Georg 4.\,2.\,1842 Kopenhagen – 19.\,2.\,1927 ebd.@\textsc{Brandes, Georg} (4.\,2.\,1842 Kopenhagen – 19.\,2.\,1927 ebd.)|pw}, dessen rheumatisches Leiden wieder heftiger aufgetreten ist, hat sich,
                     um eine so sachverständige und sorgfältige Behandlung als möglich zu finden, in
                     das Commune-Hospital in
                        Kopenhagen\oindex{Kommunehospitalet@\textbf{Kommunehospitalet}, \emph{Krankenhaus}|pwv} begeben. Sein Zustand gibt nicht zu Besorgnissen
                     Anlaß.«}}}\label{K_L01034-1} gegangen wären; aber nach dem ganzen Tun u auch nach der
               Schrift Ihres Briefes{ }ſcheint mir, daſs die Krankheit diesmal leichter auftritt als
               die erſten Male, und hoffentlich{ }ſtehn Sie bald wieder auf und{ }ſind endlich ganz
               geſund. Es iſt gewiſs ein gutes Zeichen, wenn \label{K_L01034-2v}\edtext{Recidive}{\lemma{\textnormal{\emph{Recidive}}}\Cendnote{\textnormal{Rückfall}}}\label{K_L01034-2} in abgeſchwächter Form auftreten; {\pb}ich wünſche von Herzen, daſs es das letzte iſt. –
               Sehr bedauert hab ich dſs ich in Abbazia\oindex{Opatija@\textbf{Opatija}, \emph{Hauptstadt}|pw} Ihren
               Abſagebrief fand nicht Sie{ }ſelbſt. Ich habe auf der dalmatiniſchen\oindex{Dalmatien@\textbf{Dalmatien}, \emph{Ehemalige Region}|pw} Reiſe meiſt{ }ſchlechtes Wetter gehabt; nur in Raguſa\oindex{Dubrovnik@\textbf{Dubrovnik}|pw} zwei{ }ſonnige Tage; überdies gerieth ich
               anfangs in einen \label{K_L01034-3v}\edtext{Balneologen}{\lemma{\textnormal{\emph{Balneologen}}}\Cendnote{\textnormal{Balneologie: die Lehre von den
                  Heilbädern}}}\label{K_L01034-3}congreſs, deſſen Mitglieder Schiffe und Hotels füllten, von
               denen ich auch manche perſönlich kannte, es war ziemlich unangenehm. Unter{ }ſolchen
                  Halb{\pb}bekannten{ }ſein iſt die{ }ſchli{\geminationm}ſte Form – der Einſamkeit, nicht der Geſelligkeit. Von
                  Abbazia\oindex{Opatija@\textbf{Opatija}, \emph{Hauptstadt}|pw} aus, wo es ununterbrochen regnete,
               flüchtete ich bald nach Hauſe. Das{ }ſchönſte was ich mitbrachte, iſt die Eri{\geminationn}erung an die Trümmer von Salona\oindex{Solin@\textbf{Solin}, \emph{Verwaltungsgebiet}|pw}, ich ka{\geminationn} gar nicht verſtehen, warum man
               da nicht immer und immer weitergräbt; die Erde wegkratzen und die Vergangenheit
               finden – wie ko{\geminationm}t es, dſs darüber noch keiner wahnſi{\geminationn}ig {\pb}geworden
               iſt? –\pend
           
\pstart
           Auch die albernen Angriffe gegen Sie wegen Ihrer \label{K_L01034-4v}\edtext{Budapeſt\oindex{Budapest@\textbf{Budapest}, \emph{Hauptstadt}|pw}er Einleitung}{\lemma{\textnormal{\emph{Budapester Einleitung}}}\Cendnote{\textnormal{Möglicherweise bezieht sich Schnitzler auf diese Meldung:
                     [O. V.]: \emph{Ein recht ungezogener
                        Mensch}\pwindex{recht ungezogener Mensch@\emph{Ein recht ungezogener Mensch}|pwk}. In: \emph{Arbeiter-Zeitung}\pwindex{Arbeiter-Zeitung@\emph{Arbeiter-Zeitung}|pwk},
                     Nr. 103, 15. 4. 1900, S. 6–7, hier S. 6: »Ein recht
                     ungezogener Mensch scheint Herr Georg \so{Brandes}\pwindex{Brandes, Georg 4.\,2.\,1842 Kopenhagen – 19.\,2.\,1927 ebd.@\textsc{Brandes, Georg} (4.\,2.\,1842 Kopenhagen – 19.\,2.\,1927 ebd.)|pw}, der dänische\oindex{Dänemark@\textbf{Dänemark}|pw} Literaturkritiker, zu
                     sein. Er hielt am letzten des vorigen Monats in einem Budapester Klub\oindex{Leopoldstädter Kasino@\textbf{Leopoldstädter Kasino}, \emph{Vergnügungspark}|pwv} einen Vortrag über
                        Ibsen\pwindex{Ibsen, Henrik 20.\,3.\,1828 Skien – 23.\,5.\,1906 Oslo@\textsc{Ibsen, Henrik} (20.\,3.\,1828 Skien – 23.\,5.\,1906 Oslo), \emph{Schriftsteller}|pw}. Da Herr Brandes\pwindex{Brandes, Georg 4.\,2.\,1842 Kopenhagen – 19.\,2.\,1927 ebd.@\textsc{Brandes, Georg} (4.\,2.\,1842 Kopenhagen – 19.\,2.\,1927 ebd.)|pw} nicht ungarisch\oindex{Ungarn@\textbf{Ungarn}|pw}{ }spricht, die Budapest\oindex{Budapest@\textbf{Budapest}, \emph{Hauptstadt}|pw}er aber wenig dänisch\oindex{Dänemark@\textbf{Dänemark}|pw}
                     verstehen, so sprach Herr Brandes\pwindex{Brandes, Georg 4.\,2.\,1842 Kopenhagen – 19.\,2.\,1927 ebd.@\textsc{Brandes, Georg} (4.\,2.\,1842 Kopenhagen – 19.\,2.\,1927 ebd.)|pw} –
                     natürlich \so{deutsch}. Er begann nun seine Rede mit
                     folgenden Worten: ›Meine Damen und Herren! Die Sprache, in der ich zu ihnen
                     rede, ist nicht die ihrige, und sie ist auch nicht die meine. Ich gestehe, \so{daß ich die deutsche Sprache nicht sehr liebe}; wie
                     ich weiß, ist sie \so{auch bei ihnen nicht sehr beliebt}.
                     Allein dieses einemal muß ich mich ihrer dennoch bedienen, denn schließlich ist
                     es doch die Hauptsache, daß wir einander verstehen. Ich habe das Deutsche erst
                     in meinem 30. Lebensjahr gelernt, und obwohl ich es vollkommen beherrsche, so
                     ist doch meine Aussprache mangelhaft. Deshalb ist es keine Phrase, wenn ich um
                     Nachsicht bitte.‹ Man braucht nicht viel Worte zu machen, um zu sagen, was das
                     ist, dessen sich Herr Brandes\pwindex{Brandes, Georg 4.\,2.\,1842 Kopenhagen – 19.\,2.\,1927 ebd.@\textsc{Brandes, Georg} (4.\,2.\,1842 Kopenhagen – 19.\,2.\,1927 ebd.)|pw} hier
                     schuldig gemacht hat: eine \so{Unanständigkeit}. Niemand
                     hat weniger Anlaß, über das deutsche Volk Klage zu führen, wie Herr Brandes\pwindex{Brandes, Georg 4.\,2.\,1842 Kopenhagen – 19.\,2.\,1927 ebd.@\textsc{Brandes, Georg} (4.\,2.\,1842 Kopenhagen – 19.\,2.\,1927 ebd.)|pw}, der in deutschen
                     Schriftstellerkreisen stets mit der größten Unbefangenheit und mit warmem
                     Wohlwollen aufgenommen worden ist. Es ist also eine Unziemlichkeit sehr arger
                     Art, wenn Herr Brandes\pwindex{Brandes, Georg 4.\,2.\,1842 Kopenhagen – 19.\,2.\,1927 ebd.@\textsc{Brandes, Georg} (4.\,2.\,1842 Kopenhagen – 19.\,2.\,1927 ebd.)|pw}, der kurz vorher
                     in Wien\oindex{Wien@\textbf{Wien}, \emph{Verwaltungsgebiet}|pw} der deutschen Sprache so große
                     Komplimente gemacht hat, den deutschfresserischen Instinkten der Budapest\oindex{Budapest@\textbf{Budapest}, \emph{Hauptstadt}|pw}er Clique so niedrige Konzessionen
                     bereitet.«}}}\label{K_L01034-4} habe ich geleſen. Es iſt ja wirklich gar nicht
               ernſthaft darüber zu reden. Und doch{ }ſcheint es, ka{\geminationn} man
               die Empfindlichkeit gegenüber dem dü{\geminationm}ſten, we{\geminationn} es nur einmal gedruckt iſt, nicht ganz verlieren. Ich
               erinnere mich, wie ich{ }ſeinerzeit mit einigem Staunen im Briefwechſel von Goethe\pwindex{Goethe, Johann Wolfgang von 28.\,8.\,1749 Frankfurt am Main – 22.\,3.\,1832 Weimar@\textsc{Goethe, Johann Wolfgang von} (28.\,8.\,1749 Frankfurt am Main – 22.\,3.\,1832 Weimar), \emph{Schriftsteller}|pw} und Schiller\pwindex{Schiller, Friedrich von 10.\,11.\,1759 Marbach am Neckar – 9.\,5.\,1805 Weimar@\textsc{Schiller, Friedrich von} (10.\,11.\,1759 Marbach am Neckar – 9.\,5.\,1805 Weimar), \emph{Schriftsteller, Historiker, Philosoph}|pw} Denkmäler ihres Aergers über die nichtigſten Scribenten antraf.
               Seither{ }ſtaune ich {\pb}aber nicht mehr, we{\geminationn} ich{ }ſehe, wie{ }ſich zuweilen die Klügſten über die
               Thörichteſten ärgern. Die Philoſophie hilft wohl gegen die Todesangſt, aber nicht
               gegen Flohſtiche.\pend
           
\pstart
           Daſs Sie auch mir für Wien\oindex{Wien@\textbf{Wien}, \emph{Verwaltungsgebiet}|pw} danken, iſt zu
               liebenswürdig; ich fühle, daſs ich Ihnen, beſonders diesmal, nicht viel{ }ſein konnte.
               Im Anfang waren dieſe langweiligen Zahngeſchichten; und dann liegen die Schatten von
               jenem traurigen Ereignis oft, und nun gar in dieſen Frühlingstagen{ }ſchwer auf meiner
               Seele. Dazu kommen noch mancherlei zum {\pb}Theil
               nervöſe Dinge (aber nur zum Theil), über die ich nicht gern rede, hauptſächlich ein
               quälendes Ohrenſauſen, an dem ich nun{ }ſeit drei einhalb Jahren ununterbrochen leide,
               mit beginnender Verſchlechterung des Gehörs – das macht mich natürlich auch nicht
               viel froher. Immerhin arbeite ich{ }ſeit einiger Zeit mehr als je und mit einer
               Empfindung – wenigſtens zuweilen – von innerer Fülle wie niemals früher. Ich bin
               jetzt daran eine Novelle\pwindex{Schnitzler, Arthur 15.\,5.\,1862 Wien – 21.\,10.\,1931 ebd.@\textsc{Schnitzler, Arthur} (15.\,5.\,1862 Wien – 21.\,10.\,1931 ebd.), \emph{Schriftsteller, Mediziner}!Frau Bertha Garlan. Roman@\strich\emph{Frau Bertha Garlan. Roman}|pwv} zu
               dictiren, die vor ein paar Wochen beendet wurde,{ }ſchreibe jetzt einige {\pb}kleinere und möchte im Sommer eine Komödie{ }ſchreiben. Der Schleier der \textsc{Beatrice}\pwindex{Schnitzler, Arthur 15.\,5.\,1862 Wien – 21.\,10.\,1931 ebd.@\textsc{Schnitzler, Arthur} (15.\,5.\,1862 Wien – 21.\,10.\,1931 ebd.), \emph{Schriftsteller, Mediziner}!Schleier der Beatrice. Schauspiel in fünf Akten@\strich\emph{Der Schleier der Beatrice. Schauspiel in fünf Akten}|pw} wird wahrſcheinlich im \substVorne{}\textsuperscript{Sommer}\substDazwischen{}Herbſt\substHinten{} an der Burg\oindex{Wien@\textbf{Wien}!I., Innere Stadt@\textbf{I., Innere Stadt}!Burgtheater@\textbf{Burgtheater}, \emph{Theater}|pw} aufgeführt; wo ich aber mit
               den neuen Sachen hin{ }ſoll die ich im Kopf habe weiſs ich nicht recht. Es wird nemlich
               kaum möglich{ }ſein in der nächſten Zeit etwas wien\oindex{Wien@\textbf{Wien}, \emph{Verwaltungsgebiet}|pw}eriſches zu{ }ſchreiben, in das nicht die antiſemitiſche Frage hineinſpielt –
               und meine Art darüber zu denken wird weder den Chriſten noch den Juden recht{ }ſein. –
               Das neue Buch\pwindex{Bourget, Paul 2.\,9.\,1852 Amiens – 25.\,12.\,1935 Paris@\textsc{Bourget, Paul} (2.\,9.\,1852 Amiens – 25.\,12.\,1935 Paris), \emph{Schriftsteller}!Familiendramen@\strich\emph{Familiendramen}|pwv} von \textsc{Bour{\pb}get}\pwindex{Bourget, Paul 2.\,9.\,1852 Amiens – 25.\,12.\,1935 Paris@\textsc{Bourget, Paul} (2.\,9.\,1852 Amiens – 25.\,12.\,1935 Paris), \emph{Schriftsteller}|pw} ke{\geminationn} ich nicht, habe{ }ſchon lange nicht von ihm
               geleſen; auch das Reiſewerk\pwindex{Lanckoroński, Karl 4.\,11.\,1848 Wien – 15.\,7.\,1934 ebd.@\textsc{Lanckoroński, Karl} (4.\,11.\,1848 Wien – 15.\,7.\,1934 ebd.), \emph{Schriftsteller, Sammler, Forscher}!Rund um die Erde 1888–89@\strich\emph{Rund um die Erde 1888–89}|pwv}
               von \textsc{Lanckoronsky}\pwindex{Lanckoroński, Karl 4.\,11.\,1848 Wien – 15.\,7.\,1934 ebd.@\textsc{Lanckoroński, Karl} (4.\,11.\,1848 Wien – 15.\,7.\,1934 ebd.), \emph{Schriftsteller, Sammler, Forscher}|pw} iſt mir noch unbekannt. Ich leſe jetzt – denken Sie! zum erſten Mal – we{\geminationn} ich von einer Jugendbearbeitung abſehe – den \textsc{Don Quixote}\pwindex{\textcolor{red}{\textsuperscript{XXXX indx1}}!Don Quijote@\strich\emph{Don Quijote}|pw}; da{\geminationn} ein vorzügliches Buch\pwindex{Federn, Karl 2.\,2.\,1868 Wien – 22.\,3.\,1943 London@\textsc{Federn, Karl} (2.\,2.\,1868 Wien – 22.\,3.\,1943 London), \emph{Schriftsteller, Übersetzer}!Dante@\strich\emph{Dante}|pwv} über \textsc{\uline{Dante}}\pwindex{Dante Alighieri um 1265 Florenz – 22.\,9.\,1321 Ravenna@\textsc{Dante Alighieri} (um 1265 Florenz – 22.\,9.\,1321 Ravenna), \emph{Schriftsteller}|pw} von \textsc{Federn}\pwindex{Federn, Karl 2.\,2.\,1868 Wien – 22.\,3.\,1943 London@\textsc{Federn, Karl} (2.\,2.\,1868 Wien – 22.\,3.\,1943 London), \emph{Schriftsteller, Übersetzer}|pw}, demſelben, der den \textsc{Emerson}\pwindex{Emerson, Ralph Waldo 25.\,5.\,1803 Boston – 27.\,4.\,1882 Concord@\textsc{Emerson, Ralph Waldo} (25.\,5.\,1803 Boston – 27.\,4.\,1882 Concord), \emph{Schriftsteller}|pw} trefflich überſetzt\pwindex{Emerson, Ralph Waldo 25.\,5.\,1803 Boston – 27.\,4.\,1882 Concord@\textsc{Emerson, Ralph Waldo} (25.\,5.\,1803 Boston – 27.\,4.\,1882 Concord), \emph{Schriftsteller}!Essays@\strich\emph{Essays}|pwv} hat.
                  \textsc{Gibbon}\pwindex{Gibbon, Edward 8.\,5.\,1737 Putney – 16.\,1.\,1794 London@\textsc{Gibbon, Edward} (8.\,5.\,1737 Putney – 16.\,1.\,1794 London), \emph{Politiker, Historiker}|pw}\pwindex{Gibbon, Edward 8.\,5.\,1737 Putney – 16.\,1.\,1794 London@\textsc{Gibbon, Edward} (8.\,5.\,1737 Putney – 16.\,1.\,1794 London), \emph{Politiker, Historiker}!Verfall und Untergang des Römischen Reiches@\strich\emph{Verfall und Untergang des Römischen Reiches}|pwv} begleitet mich bereits längere Zeit.\pend
           
\pstart
           Seit das Wetter{ }ſchön iſt, radl ich auch manchmal aufs Land, und für den Sommer hab
               ich {\pb}größere Touren auf dem Rad vor. Vielleicht
               entſchließen Sie{ }ſich einmal, in der heißen Zeit ins Gebirge zu gehen; ich habe mich{ }ſchon darauf  gefreut,
               einmal mit Ihnen im Freien zu{ }ſein, außerhalb von Stadt und Mauern herumzuſpaziren.
               Vielleicht läßt es{ }ſich gar machen, dſs Sie, Goldmann\pwindex{Goldmann, Paul 31.\,1.\,1865 Breslau – 25.\,9.\,1935 Wien@\textsc{Goldmann, Paul} (31.\,1.\,1865 Breslau – 25.\,9.\,1935 Wien), \emph{Schriftsteller, Journalist}|pw} und Beer Hofma{\geminationn}\pwindex{Beer-Hofmann, Richard 11.\,7.\,1866 Wien – 26.\,9.\,1945 New York City@\textsc{Beer-Hofmann, Richard} (11.\,7.\,1866 Wien – 26.\,9.\,1945 New York City), \emph{Schriftsteller}|pw} u ich irgendwo zuſammentreffen, fern von allen Zeitungen – und am Ende auch von
               aller »Lit\damage{\textcolor{gray}{era}}tur«. –\pend
           
\pstart
           Jedenfalls hoff ich Sie{ }ſagen mir bald wieder ein Wort, wies Ihnen {\pb}geht. Es iſt eine meiner wirklichen Freuden, daſs
               Sie meiner mit Sympathie gedenken. Ich grüße Sie herzlich.\pend
           \pstart Ihr \spacefill\mbox{Arthur Schnitzler}\pend{}
\pstart
           Wien\oindex{Wien@\textbf{Wien}, \emph{Verwaltungsgebiet}|pw}, 3. 5. 900.\pend
           \selectlanguage{ngerman}\endnumbering\briefempfaengerindex{Brandes, Georg@\textsc{Brandes, Georg}!zzzSchnitzler, Arthur@\emph{von Arthur Schnitzler}!1900-05-031@{3. 5. 1900}|)be}\mylabel{L01034h}  \newcommand{\dateiname}{L01034}\newcommand{\titel}{Arthur Schnitzler an Georg Brandes, 3. 5. 1900}\newcommand{\editorInnen}{Martin Anton Müller und Gerd-Hermann Susen}%% latex-leseansicht-abspann.tex
%% Abspann für die Leseansicht.
%% Der Schalter \ifkorrekturansicht ist bereits durch den Vorspann gesetzt.

%% latex-abspann.tex
%% Gemeinsamer Abspann für Korrekturansicht und Leseansicht.
%% Setzt den Schalter \ifkorrekturansicht voraus (gesetzt in den
%% einbindenden Dateien latex-korrekturansicht-abspann.tex bzw.
%% latex-leseansicht-abspann.tex).
%% ---------------------------------------------------------------

\normalsize

% Das esempio-Environment wird nur in der Leseansicht benötigt
\ifkorrekturansicht\else
\newenvironment{esempio}[3]%
{
    \vspace{1.5ex}
    \rlap{\underline{#1}}
    \par
    \setlength{\parindent}{0cm}
    \nopagebreak
    \leftskip=#2cm
    \rightskip=#3cm
}
{
    \par
}
\fi

\doendnotes{C}
\bigskip
\vfill

\clearpage

\footnotesize

\ifkorrekturansicht
  \lohead{\textsc{register}}
\fi

% theindex-Environment neu definieren ohne reledmac
\makeatletter
\renewenvironment{theindex}{%
  \ifkorrekturansicht
    \section*{\indexname}%
  \else
    \subsubsection*{Index der erwähnten Entitäten}%
  \fi
  \setlength{\parindent}{0pt}%
  \setlength{\parskip}{0pt plus 0.3pt}%
  \let\item\@idxitem
}{%
  \ifkorrekturansicht\clearpage\fi
}
\makeatother

\IfFileExists{\jobname-pw.ind}{\input{\jobname-pw.ind}}{}

% Quellenangabe nur in der Leseansicht
\ifkorrekturansicht\else
% Fallback-Definitionen, falls die .tex-Datei \titel etc. nicht gesetzt hat
\providecommand{\titel}{}
\providecommand{\editorInnen}{}
\providecommand{\dateiname}{\jobname}

\vspace{3cm}

\vfill

\footnotesize
\textsc{Quelle}: \titel. Herausgegeben von {\editorInnen}. In: \emph{Arthur Schnitzler: Briefwechsel mit Autorinnen und Autoren}.
 Digitale Edition, https://schnitzler-briefe.acdh.oeaw.ac.at/{\dateiname}.html (Stand \today)
\fi

\end{document}


