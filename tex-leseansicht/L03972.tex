%% latex-leseansicht-vorspann.tex
%% Vorspann für die Leseansicht.
%% Lädt die gemeinsame Datei latex-vorspann.tex mit nicht gesetztem Schalter.

\newif\ifkorrekturansicht
\korrekturansichtfalse

\input{../tex-inputs/latex-vorspann}


\section[Arthur Schnitzler an Berta Zuckerkandl, 19. 11. 1927]{L03972 Arthur Schnitzler an Berta Zuckerkandl, 19. 11. 1927}
\nopagebreak\mylabel{L03972v}
\rehead{ }\normalsize\beginnumbering\briefempfaengerindex{Zuckerkandl, Berta@\textsc{Zuckerkandl, Berta}!zzzSchnitzler, Arthur@\emph{von Arthur Schnitzler}!1927-11-191@{19. 11. 1927}|(be}
\toendnotes[C]{\smallbreak\pagebreak[2]}
\correspDesc{Versand  durch Arthur Schnitzler am 19. 11. 1927 in Wien
\newline{}Erhalt  durch Berta Zuckerkandl im Zeitraum [20. 11. 1927 – 24. 11. 1927?] in Paris}\toendnotes[C]{\smallbreak}
\Standort{DLA, HS.1985.1.2282.}
\physDesc{Brief, Durchschlag, 1 Blatt, 2 Seiten, 2216 Zeichen
\newline{}Schreibmaschine
\newline{}Handschrift: roter Buntstift, lateinische Kurrent (\noindent{}beschriftet: »\uline{Zuckerkandl}«, »(Blum)«, »\uline{Paris}« und »Frkr.«, zwölf Unterstreichungen)}\toendnotes[C]{\smallbreak}
\pstart
           \raggedleft{}{\pb}19. 11. 1927.\pend
           
\pstart{}Liebe und verehrte Freundin.\pend\vspace{0.5em}
\pstart
           Ich beeile mich Ihren eben angelangten \label{K_L03972-1v}\edtext{Brief}{\lemma{\textnormal{\emph{Brief}}}\Cendnote{\textnormal{nicht überliefert}}}\label{K_L03972-1} zu
               beantworten. Schon vor zirka zehn Tagen hat sich die Sekretärin\pwindex{Reichelt @\textsc{Reichelt}, \emph{Sekretärin}|pwv} des Herrn Blum\pwindex{Blum, Robert 17.\,4.\,1881 Wien – 3.\,7.\,1952 Paris@\textsc{Blum, Robert} (17.\,4.\,1881 Wien – 3.\,7.\,1952 Paris), \emph{Schriftsteller, Journalist, Theaterleiter}|pw} telefonisch an mich gewandt. Ich habe die Sache mit
               mässigem Vertrauen behandelt und nach meinen bisherigen Erfahrungen liess ich Herrn
                  Blum\pwindex{Blum, Robert 17.\,4.\,1881 Wien – 3.\,7.\,1952 Paris@\textsc{Blum, Robert} (17.\,4.\,1881 Wien – 3.\,7.\,1952 Paris), \emph{Schriftsteller, Journalist, Theaterleiter}|pw} mitteilen, dass ich vor allem einmal
               einen direkten Brief von Herrn Gemier\pwindex{Gémier, Firmin 21.\,2.\,1865 Aubervilliers – 26.\,11.\,1933 Paris@\textsc{Gémier, Firmin} (21.\,2.\,1865 Aubervilliers – 26.\,11.\,1933 Paris), \emph{Theaterleiter, Schauspieler, Drehbuchautor}|pw} erwarte
               und nur auf ein direktes Anerbieten von ihm mich ausführlich zu äussern gedächte. Was
               von mir übersetzt ist weiss ja Herr Gemier\pwindex{Gémier, Firmin 21.\,2.\,1865 Aubervilliers – 26.\,11.\,1933 Paris@\textsc{Gémier, Firmin} (21.\,2.\,1865 Aubervilliers – 26.\,11.\,1933 Paris), \emph{Theaterleiter, Schauspieler, Drehbuchautor}|pw}
               längst, ich glaube Sie selbst, liebe Freundin, haben ihm seinerzeit \label{K_L03972-2v}\edtext{die Liste dieser Stücke}{\lemma{\textnormal{\emph{die Liste dieser Stücke}}}\Cendnote{\textnormal{Am 6. 2. 1925 vermerkte Schnitzler im \emph{Tagebuch}\pwindex{Schnitzler, Arthur 15. 5. 1862 Wien – 21. 10. 1931 ebd.@\textsc{Schnitzler, Arthur} (15. 5. 1862 Wien – 21. 10. 1931 ebd.), \emph{Schriftsteller, Mediziner}!Tagebuch@\strich\emph{Tagebuch}|pwk}
                  nach einem Gespräch mit Berta Zuckerkandl\pwindex{Zuckerkandl, Berta 13.\,4.\,1864 Wien – 16.\,10.\,1945 Paris@\textsc{Zuckerkandl, Berta} (13.\,4.\,1864 Wien – 16.\,10.\,1945 Paris), \emph{Schriftstellerin, Journalistin, Übersetzerin}|pwk},
                     Gémier\pwindex{Gémier, Firmin 21.\,2.\,1865 Aubervilliers – 26.\,11.\,1933 Paris@\textsc{Gémier, Firmin} (21.\,2.\,1865 Aubervilliers – 26.\,11.\,1933 Paris), \emph{Theaterleiter, Schauspieler, Drehbuchautor}|pwk} wolle ein großes Werk von ihn
                  aufführen, am Tag darauf übersandte er eine nicht (nicht überlieferte) Liste, siehe XXXX Auszeichnungsfehler: Dokument L03958 nicht gefunden.}}}\label{K_L03972-2} übergeben. Nur
               weniges kommt ja für eine Aufführung in Paris\oindex{Paris@\textbf{Paris}, \emph{Hauptstadt}|pw} in
               Betracht. »Das weite Land\pwindex{Schnitzler, Arthur 15. 5. 1862 Wien – 21. 10. 1931 ebd.@\textsc{Schnitzler, Arthur} (15. 5. 1862 Wien – 21. 10. 1931 ebd.), \emph{Schriftsteller, Mediziner}!weite Land. Tragikomödie in fünf Akten@\strich\emph{Das weite Land. Tragikomödie in fünf Akten}|pw}\pwindex{Schnitzler, Arthur 15. 5. 1862 Wien – 21. 10. 1931 ebd.@\textsc{Schnitzler, Arthur} (15. 5. 1862 Wien – 21. 10. 1931 ebd.), \emph{Schriftsteller, Mediziner}!Le Pays de l’âme. Drame en 5 actes@\strich\emph{Le Pays de l’âme. Drame en 5 actes}|pw}« scheint
               Herrn Gemier\pwindex{Gémier, Firmin 21.\,2.\,1865 Aubervilliers – 26.\,11.\,1933 Paris@\textsc{Gémier, Firmin} (21.\,2.\,1865 Aubervilliers – 26.\,11.\,1933 Paris), \emph{Theaterleiter, Schauspieler, Drehbuchautor}|pw} nun doch nicht zu passen, bleibt
               übrig der »Einsame Weg\pwindex{Schnitzler, Arthur 15. 5. 1862 Wien – 21. 10. 1931 ebd.@\textsc{Schnitzler, Arthur} (15. 5. 1862 Wien – 21. 10. 1931 ebd.), \emph{Schriftsteller, Mediziner}!einsame Weg. Schauspiel in fünf Akten@\strich\emph{Der einsame Weg. Schauspiel in fünf Akten}|pw}« \label{K_L03972-3v}\edtext{überse{[}t{]}zt von Emma Cabire\pwindex{Cabire, Emma @\textsc{Cabire, Emma}, \emph{Übersetzerin, Redakteurin, Literaturagentin}|pw}}{\lemma{\textnormal{\emph{übersetzt … Cabire}}}\Cendnote{\textnormal{Die betreffende Übersetzung\pwindex{Schnitzler, Arthur 15. 5. 1862 Wien – 21. 10. 1931 ebd.@\textsc{Schnitzler, Arthur} (15. 5. 1862 Wien – 21. 10. 1931 ebd.), \emph{Schriftsteller, Mediziner}!?? [französische Übersetzung von Der einsame Weg]@\strich\emph{?? [französische Übersetzung von Der einsame Weg]}|pwkv} von \emph{Der
                        einsame Weg}\pwindex{Schnitzler, Arthur 15. 5. 1862 Wien – 21. 10. 1931 ebd.@\textsc{Schnitzler, Arthur} (15. 5. 1862 Wien – 21. 10. 1931 ebd.), \emph{Schriftsteller, Mediziner}!einsame Weg. Schauspiel in fünf Akten@\strich\emph{Der einsame Weg. Schauspiel in fünf Akten}|pwk} stammte nicht von Emma Cabire\pwindex{Cabire, Emma @\textsc{Cabire, Emma}, \emph{Übersetzerin, Redakteurin, Literaturagentin}|pwk} sondern von
                  Geneviève Bianquis\pwindex{Bianquis, Geneviève 19.\,9.\,1886 Rouen – 24.\,3.\,1972 Antony@\textsc{Bianquis, Geneviève} (19.\,9.\,1886 Rouen – 24.\,3.\,1972 Antony), \emph{Übersetzerin, Literaturhistorikerin}|pwk}.}}}\label{K_L03972-3}, ein Stück\pwindex{Schnitzler, Arthur 15. 5. 1862 Wien – 21. 10. 1931 ebd.@\textsc{Schnitzler, Arthur} (15. 5. 1862 Wien – 21. 10. 1931 ebd.), \emph{Schriftsteller, Mediziner}!einsame Weg. Schauspiel in fünf Akten@\strich\emph{Der einsame Weg. Schauspiel in fünf Akten}|pwv}, das sich gewiss nicht für Paris\oindex{Paris@\textbf{Paris}, \emph{Hauptstadt}|pw}
               eignet, »Zwischenspiel\pwindex{Schnitzler, Arthur 15. 5. 1862 Wien – 21. 10. 1931 ebd.@\textsc{Schnitzler, Arthur} (15. 5. 1862 Wien – 21. 10. 1931 ebd.), \emph{Schriftsteller, Mediziner}!Zwischenspiel. Komödie in drei Akten@\strich\emph{Zwischenspiel. Komödie in drei Akten}|pw}«, übersetzt\pwindex{Schnitzler, Arthur 15. 5. 1862 Wien – 21. 10. 1931 ebd.@\textsc{Schnitzler, Arthur} (15. 5. 1862 Wien – 21. 10. 1931 ebd.), \emph{Schriftsteller, Mediziner}!?? [französische Übersetzung von Zwischenspiel]@\strich\emph{?? [französische Übersetzung von Zwischenspiel]}|pwv} von Remon\pwindex{Rémon, Maurice 27.\,11.\,1861 Paris – 20.\,6.\,1945 Mérignac@\textsc{Rémon, Maurice} (27.\,11.\,1861 Paris – 20.\,6.\,1945 Mérignac), \emph{Übersetzer}|pw}, endlich einige Einakter, »Kakadu\pwindex{Schnitzler, Arthur 15. 5. 1862 Wien – 21. 10. 1931 ebd.@\textsc{Schnitzler, Arthur} (15. 5. 1862 Wien – 21. 10. 1931 ebd.), \emph{Schriftsteller, Mediziner}!grüne Kakadu. Groteske in einem Akt@\strich\emph{Der grüne Kakadu. Groteske in einem Akt}|pw}\pwindex{Schnitzler, Arthur 15. 5. 1862 Wien – 21. 10. 1931 ebd.@\textsc{Schnitzler, Arthur} (15. 5. 1862 Wien – 21. 10. 1931 ebd.), \emph{Schriftsteller, Mediziner}!Au Perroquet Vert@\strich\emph{Au Perroquet Vert}|pw}«, »Gefährtin\pwindex{Schnitzler, Arthur 15. 5. 1862 Wien – 21. 10. 1931 ebd.@\textsc{Schnitzler, Arthur} (15. 5. 1862 Wien – 21. 10. 1931 ebd.), \emph{Schriftsteller, Mediziner}!Gefährtin. Schauspiel in einem Akt@\strich\emph{Die Gefährtin. Schauspiel in einem Akt}|pw}\pwindex{Schnitzler, Arthur 15. 5. 1862 Wien – 21. 10. 1931 ebd.@\textsc{Schnitzler, Arthur} (15. 5. 1862 Wien – 21. 10. 1931 ebd.), \emph{Schriftsteller, Mediziner}!Compagne. Comédie en une acte@\strich\emph{La Compagne. Comédie en une acte}|pw}«, »Letzte Masken\pwindex{Schnitzler, Arthur 15. 5. 1862 Wien – 21. 10. 1931 ebd.@\textsc{Schnitzler, Arthur} (15. 5. 1862 Wien – 21. 10. 1931 ebd.), \emph{Schriftsteller, Mediziner}!letzten Masken@\strich\emph{Die letzten Masken}|pw}\pwindex{Schnitzler, Arthur 15. 5. 1862 Wien – 21. 10. 1931 ebd.@\textsc{Schnitzler, Arthur} (15. 5. 1862 Wien – 21. 10. 1931 ebd.), \emph{Schriftsteller, Mediziner}!Derniers masques. Comédie en un act@\strich\emph{Les Derniers masques. Comédie en un act}|pw}«,
               um nur die übersetzten zu nennen, ausserdem zogen wir in Betracht »Liebelei\pwindex{Schnitzler, Arthur 15. 5. 1862 Wien – 21. 10. 1931 ebd.@\textsc{Schnitzler, Arthur} (15. 5. 1862 Wien – 21. 10. 1931 ebd.), \emph{Schriftsteller, Mediziner}!Liebelei. Schauspiel in drei Akten@\strich\emph{Liebelei. Schauspiel in drei Akten}|pw}«, das ich im vorigen Jahr in einer
               alten Uebersetzung\pwindex{Schnitzler, Arthur 15. 5. 1862 Wien – 21. 10. 1931 ebd.@\textsc{Schnitzler, Arthur} (15. 5. 1862 Wien – 21. 10. 1931 ebd.), \emph{Schriftsteller, Mediziner}!Amourette. Pièce en trois actes. Adaptée de Arthur Schnitzler@\strich\emph{Amourette. Pièce en trois actes. Adaptée de Arthur Schnitzler}|pwv} von Thorel\pwindex{Thorel, Jean 11.\,9.\,1859 Éragny – 20.\,8.\,1916 Enghien-les-Bains@\textsc{Thorel, Jean} (11.\,9.\,1859 Éragny – 20.\,8.\,1916 Enghien-les-Bains), \emph{Übersetzer, Dramatiker}|pw} mit kleinen Verbesserungen, die Frau Pollaczek\pwindex{Pollaczek, Clara Katharina 15.\,1.\,1875 Wien – 22.\,7.\,1951 ebd.@\textsc{Pollaczek, Clara Katharina} (15.\,1.\,1875 Wien – 22.\,7.\,1951 ebd.), \emph{Schriftstellerin}|pw} besorgte, \label{K_L03972-4v}\edtext{nach Paris\oindex{Paris@\textbf{Paris}, \emph{Hauptstadt}|pw} gesandt}{\lemma{\textnormal{\emph{nach Paris gesandt}}}\Cendnote{\textnormal{Die Übersetzung\pwindex{Schnitzler, Arthur 15. 5. 1862 Wien – 21. 10. 1931 ebd.@\textsc{Schnitzler, Arthur} (15. 5. 1862 Wien – 21. 10. 1931 ebd.), \emph{Schriftsteller, Mediziner}!Amourette. Pièce en trois actes. Adaptée de Arthur Schnitzler@\strich\emph{Amourette. Pièce en trois actes. Adaptée de Arthur Schnitzler}|pwkv} wurde übersandt mit dem Brief von Arthur Schnitzler an Henri-René Lenormand\pwindex{Lenormand, Henri-René 3.\,5.\,1882 Paris – 16.\,2.\,1951 ebd.@\textsc{Lenormand, Henri-René} (3.\,5.\,1882 Paris – 16.\,2.\,1951 ebd.), \emph{Schriftsteller}|pwk}, 30. 7. 1925, \emph{Deutsches Literaturarchiv Marbach},
                  HS.1985.1.1280. Eine weitere Abschrift\pwindex{Schnitzler, Arthur 15. 5. 1862 Wien – 21. 10. 1931 ebd.@\textsc{Schnitzler, Arthur} (15. 5. 1862 Wien – 21. 10. 1931 ebd.), \emph{Schriftsteller, Mediziner}!Amourette. Pièce en trois actes. Adaptée de Arthur Schnitzler@\strich\emph{Amourette. Pièce en trois actes. Adaptée de Arthur Schnitzler}|pwkv} ging an Zuckerkandl\pwindex{Zuckerkandl, Berta 13.\,4.\,1864 Wien – 16.\,10.\,1945 Paris@\textsc{Zuckerkandl, Berta} (13.\,4.\,1864 Wien – 16.\,10.\,1945 Paris), \emph{Schriftstellerin, Journalistin, Übersetzerin}|pwk}, siehe XXXX Auszeichnungsfehler: Dokument L03961 nicht gefunden.}}}\label{K_L03972-4} habe entweder an Sie oder an Gemier\pwindex{Gémier, Firmin 21.\,2.\,1865 Aubervilliers – 26.\,11.\,1933 Paris@\textsc{Gémier, Firmin} (21.\,2.\,1865 Aubervilliers – 26.\,11.\,1933 Paris), \emph{Theaterleiter, Schauspieler, Drehbuchautor}|pw}, oder an Lenormand\pwindex{Lenormand, Henri-René 3.\,5.\,1882 Paris – 16.\,2.\,1951 ebd.@\textsc{Lenormand, Henri-René} (3.\,5.\,1882 Paris – 16.\,2.\,1951 ebd.), \emph{Schriftsteller}|pw}, ich weiss
               mich nicht mehr genau zu erinnern. Ich will meine Korrespondenz daraufhin noch
               durchsehen. Für Ihre Bemühungen danke ich Ihnen sehr. Das Interesse von Gemier\pwindex{Gémier, Firmin 21.\,2.\,1865 Aubervilliers – 26.\,11.\,1933 Paris@\textsc{Gémier, Firmin} (21.\,2.\,1865 Aubervilliers – 26.\,11.\,1933 Paris), \emph{Theaterleiter, Schauspieler, Drehbuchautor}|pw} ist mir durchaus zweifelhaft und mir
               läge natürlich nur daran, dass irgendwelche meiner Stücke sehr gut und mit voller
               Anteilnahme des betreffenden Direktors oder Regisseurs, in glänzender Uebersetzung
               herausgebracht werde{[}n{]}, nicht etwa darauf, dass Gemier\pwindex{Gémier, Firmin 21.\,2.\,1865 Aubervilliers – 26.\,11.\,1933 Paris@\textsc{Gémier, Firmin} (21.\,2.\,1865 Aubervilliers – 26.\,11.\,1933 Paris), \emph{Theaterleiter, Schauspieler, Drehbuchautor}|pw} glaubt irgend einer Verpflichtung nachkommen zu
               müssen. Beim Theater gibt es nur Kontrakte, aber keine Verpflichtungen. Ich nehme ja
               an, dass sich die Sekretärin\pwindex{Reichelt @\textsc{Reichelt}, \emph{Sekretärin}|pwv}
               des Herrn Blum\pwindex{Blum, Robert 17.\,4.\,1881 Wien – 3.\,7.\,1952 Paris@\textsc{Blum, Robert} (17.\,4.\,1881 Wien – 3.\,7.\,1952 Paris), \emph{Schriftsteller, Journalist, Theaterleiter}|pw} noch einmal \label{K_L03972-5v}\edtext{an mich wenden}{\lemma{\textnormal{\emph{an mich wenden}}}\Cendnote{\textnormal{Dies geschah laut \emph{Tagebuch}\pwindex{Schnitzler, Arthur 15. 5. 1862 Wien – 21. 10. 1931 ebd.@\textsc{Schnitzler, Arthur} (15. 5. 1862 Wien – 21. 10. 1931 ebd.), \emph{Schriftsteller, Mediziner}!Tagebuch@\strich\emph{Tagebuch}|pwk} am 23. 11. 1927.}}}\label{K_L03972-5} wird.\pend
           
\pstart
           {\pb}Wann kommen Sie nach Wien\oindex{Wien@\textbf{Wien}, \emph{Verwaltungsgebiet}|pw} zurück? Ich fahre Ende d. M. wahrscheinlich auf zehn Tage
                  \label{K_L03972-6v}\edtext{nach Berlin\oindex{Berlin@\textbf{Berlin}, \emph{Hauptstadt}|pw}}{\lemma{\textnormal{\emph{nach Berlin}}}\Cendnote{\textnormal{Schnitzler hielt sich vom 29. 11. 1927 bis zum
                     9. 12. 1927 in
                     Berlin\oindex{Berlin@\textbf{Berlin}, \emph{Hauptstadt}|pwk} auf.}}}\label{K_L03972-6}, bin jedesfalls
                  Mitte Dezember wieder hier, umso sicherer, als ich \label{K_L03972-7v}\edtext{für diese Zeit}{\lemma{\textnormal{\emph{für diese Zeit}}}\Cendnote{\textnormal{Lili\pwindex{Cappellini, Lili 13.\,9.\,1909 Wien – 26.\,7.\,1928 Venedig@\textsc{Cappellini, Lili} (13.\,9.\,1909 Wien – 26.\,7.\,1928 Venedig)|pwk} und Arnoldo Capellini\pwindex{Cappellini, Arnoldo 10.\,8.\,1889 Venedig – 8.\,5.\,1954 Rom@\textsc{Cappellini, Arnoldo} (10.\,8.\,1889 Venedig – 8.\,5.\,1954 Rom)|pwk} kamen vom 15. 12. 1927 bis zum 3. 1. 1928 nach Wien\oindex{Wien@\textbf{Wien}, \emph{Verwaltungsgebiet}|pwk}.}}}\label{K_L03972-7} meine Tochter\pwindex{Cappellini, Lili 13.\,9.\,1909 Wien – 26.\,7.\,1928 Venedig@\textsc{Cappellini, Lili} (13.\,9.\,1909 Wien – 26.\,7.\,1928 Venedig)|pwv} mit dem Faschisten\pwindex{Cappellini, Arnoldo 10.\,8.\,1889 Venedig – 8.\,5.\,1954 Rom@\textsc{Cappellini, Arnoldo} (10.\,8.\,1889 Venedig – 8.\,5.\,1954 Rom)|pwv} erwarte, wie Lili\pwindex{Cappellini, Lili 13.\,9.\,1909 Wien – 26.\,7.\,1928 Venedig@\textsc{Cappellini, Lili} (13.\,9.\,1909 Wien – 26.\,7.\,1928 Venedig)|pw} ihren Gatten\pwindex{Cappellini, Arnoldo 10.\,8.\,1889 Venedig – 8.\,5.\,1954 Rom@\textsc{Cappellini, Arnoldo} (10.\,8.\,1889 Venedig – 8.\,5.\,1954 Rom)|pwv} in ihren Briefen immer noch nennt. Ein nicht allzu fernes gutes
               Wiedersehen erhoffend, bin ich mit den herzlichsten Grüssen\pend
           
\pstart
           Ihr getreuer {\\[\baselineskip]}{\vspace{1\baselineskip}}\pend
           \leftskip=0em{}
\pstart
           \noindent{}Frau Hofrätin Bertha Zuckerkandl,{\\}Paris\oindex{Paris@\textbf{Paris}, \emph{Hauptstadt}|pw}.\pend
           \selectlanguage{ngerman}\endnumbering\briefempfaengerindex{Zuckerkandl, Berta@\textsc{Zuckerkandl, Berta}!zzzSchnitzler, Arthur@\emph{von Arthur Schnitzler}!1927-11-191@{19. 11. 1927}|)be}\mylabel{L03972h}
\begin{anhang}
\end{anhang}\newcommand{\dateiname}{L03972}\newcommand{\titel}{Arthur Schnitzler an Berta Zuckerkandl, 19. 11. 1927}\newcommand{\editorInnen}{Herausgegeben von Jahnke, SelmaMüller, Martin Anton}%% latex-leseansicht-abspann.tex
%% Abspann für die Leseansicht.
%% Der Schalter \ifkorrekturansicht ist bereits durch den Vorspann gesetzt.

%% latex-abspann.tex
%% Gemeinsamer Abspann für Korrekturansicht und Leseansicht.
%% Setzt den Schalter \ifkorrekturansicht voraus (gesetzt in den
%% einbindenden Dateien latex-korrekturansicht-abspann.tex bzw.
%% latex-leseansicht-abspann.tex).
%% ---------------------------------------------------------------

\normalsize

% Das esempio-Environment wird nur in der Leseansicht benötigt
\ifkorrekturansicht\else
\newenvironment{esempio}[3]%
{
    \vspace{1.5ex}
    \rlap{\underline{#1}}
    \par
    \setlength{\parindent}{0cm}
    \nopagebreak
    \leftskip=#2cm
    \rightskip=#3cm
}
{
    \par
}
\fi

\doendnotes{C}
\bigskip
\vfill

\clearpage

\footnotesize

\ifkorrekturansicht
  \lohead{\textsc{register}}
\fi

% theindex-Environment neu definieren ohne reledmac
\makeatletter
\renewenvironment{theindex}{%
  \ifkorrekturansicht
    \section*{\indexname}%
  \else
    \subsubsection*{Index der erwähnten Entitäten}%
  \fi
  \setlength{\parindent}{0pt}%
  \setlength{\parskip}{0pt plus 0.3pt}%
  \let\item\@idxitem
}{%
  \ifkorrekturansicht\clearpage\fi
}
\makeatother

\IfFileExists{\jobname-pw.ind}{\input{\jobname-pw.ind}}{}

% Quellenangabe nur in der Leseansicht
\ifkorrekturansicht\else
% Fallback-Definitionen, falls die .tex-Datei \titel etc. nicht gesetzt hat
\providecommand{\titel}{}
\providecommand{\editorInnen}{}
\providecommand{\dateiname}{\jobname}

\vspace{3cm}

\vfill

\footnotesize
\textsc{Quelle}: \titel. Herausgegeben von {\editorInnen}. In: \emph{Arthur Schnitzler: Briefwechsel mit Autorinnen und Autoren}.
 Digitale Edition, https://schnitzler-briefe.acdh.oeaw.ac.at/{\dateiname}.html (Stand \today)
\fi

\end{document}


