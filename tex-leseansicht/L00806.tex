%% latex-korrekturansicht-vorspann.tex
%% Vorspann für die Korrekturansicht.
%% Lädt die gemeinsame Datei latex-vorspann.tex mit gesetztem Schalter.

\newif\ifkorrekturansicht
\korrekturansichttrue

\input{../tex-inputs/latex-vorspann}


\section[Arthur Schnitzler an Richard Beer-Hofmann, 17. 6. 1898]{L00806 Arthur Schnitzler an Richard Beer-Hofmann, 17. 6. 1898}
\nopagebreak\mylabel{L00806v}
\rehead{ }\normalsize\beginnumbering\briefempfaengerindex{Beer-Hofmann, Richard@\textsc{Beer-Hofmann, Richard}!zzzSchnitzler, Arthur@\emph{von Arthur Schnitzler}!1898-06-171@{17. 6. 1898}|(be}
\toendnotes[C]{\smallbreak\pagebreak[2]}\Standort{CUL, Schnitzler, B 8.1, S. 71.}
\physDesc{Brief, maschinenschriftliche Abschrift1 Blatt, 1 Seite, 3043 Zeichen
\newline{}Schreibmaschine
\newline{}Ordnung: von unbekannter Hand nummeriert: »119« }\Standort{YCGL, MSS 31.}
\physDesc{Korrekturen zu Schlaflied für Mirjam, 1 Blatt, 1 Seite, 3043 Zeichen
\newline{}Handschrift: Bleistift, deutsche Kurrent
\newline{}Ordnung: 1) mit Bleistift von unbekannter Hand beschriftet
                                    »Schnitzler: Korrekturen zu Beer-Hofmanns ›Schlaflied für
                                    Mirjam‹«  2) mit schwarzer Tinte von unbekannter Hand zur Zeile 6 der 2. Strophe:
                                    »doch«}\Standort{CUL, Schnitzler, B 8.}
\physDesc{Gedichtabschrift, handschriftliche Abschrift1 Blatt, 2 Seiten, 3043 Zeichen
\newline{}Handschrift: Bleistift, deutsche Kurrent
\newline{}Ordnung: mit Bleistift von unbekannter Hand nummeriert:
                                    »116« }
\buchAbdrucke{\weitereDrucke{Arthur Schnitzler, Richard Beer-Hofmann: \emph{Briefwechsel 1891–1931}. Wien, Zürich: \emph{Europaverlag} 1992, S. 119–120, 118–119.} }\toendnotes[C]{\smallbreak}
\pstart
           \raggedleft{}{\pb}Wien\oindex{Wien@\textbf{Wien}, \emph{A.ADM2}|pw}, 17. 6. 98.\pend
           \vspace{0.5em}
\pstart
           Lieber Richard, beiliegend mein Interpunktionsgefühl. Im
               wesentlichen liegt ja nicht viel dran. Hugo\pwindex{Hofmannsthal, Hugo von 1874-02-01 – 1929-07-15@\textsc{Hofmannsthal, Hugo von} (1874-02-01 – 1929-07-15), \emph{Schriftsteller/Schriftstellerin}|pw}
               ist in der Brühl\oindex{Bruehl@\textbf{Brühl}, \emph{Tal (N.TAL)}|pw}, ich wollte gestern zu ihm;
               aber es regnete. Am Tag meiner Abfahrt hatte ich Regen bis Wr. Neustadt\oindex{Wiener Neustadt@\textbf{Wiener Neustadt}, \emph{A.ADM2}|pw} – dann war es schön und blieb so bis gestern. Meine
               Sommerpläne sind verpfuscht. Man lässt sie\pwindex{Reinhard, Marie 1871-03-13 – 1899-03-18@\textsc{Reinhard, Marie} (1871-03-13 – 1899-03-18), \emph{Gesangspädagoge/Gesangspädagogin}|pwv} nicht mit mir reisen, so wird ein enervirendes Hin und
               Her herauskommen. Ich bleibe vor allem einmal bis Mitte Juli in Wien\oindex{Wien@\textbf{Wien}, \emph{A.ADM2}|pw}; bin dann ein paar Tage mit ihr\pwindex{Reinhard, Marie 1871-03-13 – 1899-03-18@\textsc{Reinhard, Marie} (1871-03-13 – 1899-03-18), \emph{Gesangspädagoge/Gesangspädagogin}|pwv} und ihrer Schwester\pwindex{Burger, Caroline 11.07.1869 – 15.03.1959@\textsc{Burger, Caroline} (11.07.1869 – 15.03.1959)|pwv} sowie Schwager\pwindex{Burger, Rudolf *~06.12.1866@\textsc{Burger, Rudolf} (*~06.12.1866), \emph{Versicherungsdirektor/Versicherungsdirektorin}|pwv} in Gr.\oindex{Graz@\textbf{Graz}, \emph{A.ADM2}|pw} zusammen – wohin ich vom 20.–27. Juli gehe,
               weiss ich nicht. (Wollen Sie irgendwo mit mir zusammen sein? Aber nicht in Steindorf\oindex{Steindorf am Ossiacher See@\textbf{Steindorf am Ossiacher See}, \emph{A.ADM3}|pw}) Dann per Rad mit ihr\pwindex{Reinhard, Marie 1871-03-13 – 1899-03-18@\textsc{Reinhard, Marie} (1871-03-13 – 1899-03-18), \emph{Gesangspädagoge/Gesangspädagogin}|pwv} und den Ihren\pwindex{Burger, Caroline 11.07.1869 – 15.03.1959@\textsc{Burger, Caroline} (11.07.1869 – 15.03.1959)|pw}\pwindex{Reinhard, Carl 01.03.1868 – 1904-09-29@\textsc{Reinhard, Carl} (01.03.1868 – 1904-09-29), \emph{Kapellmeister/Kapellmeisterin}|pw}\pwindex{Reinhard, Therese 13.12.1844 – 25.03.1926@\textsc{Reinhard, Therese} (13.12.1844 – 25.03.1926)|pw} nach Tegernsee\oindex{Tegernsee@\textbf{Tegernsee}, \emph{P.PPL}|pw}. – Von dort verschwind ich sofort; –
               wahrscheinlich in die Schweiz. Da werd ich eine Zeitlang mit der Mama\pwindex{Schnitzler, Louise 1840-07-08 – 1911-09-09@\textsc{Schnitzler, Louise} (1840-07-08 – 1911-09-09)|pwv} zusammen sein. (Vierwaldstädtersee\oindex{Vierwaldstaettersee@\textbf{Vierwaldstättersee}, \emph{See (N.SEE)}|pw}). Die letzte Augustwoche
               wahrscheinlich in Tegernsee\oindex{Tegernsee@\textbf{Tegernsee}, \emph{P.PPL}|pw} – dann in den
               ersten Septembertagen wenns geht, durchs Ampezzo\oindex{Ampezzo@\textbf{Ampezzo}, \emph{P.PPLA3}|pw} per Rad nach Venedig\oindex{Venedig@\textbf{Venedig}, \emph{P.PPLA}|pw}. –\pend
           
\pstart
           Im übrigen arbeite ich und fühl mich aus den bekannten Ursachen nicht wohl. – (Milder
               Ausdruck.)\pend
           
\pstart
           Brief und Carton hab ich erhalten, danke sehr. Wie gehts Ihnen? Machen Sie was? Paul G.\pwindex{Goldmann, Paul 31.01.1865 – 25.09.1935@\textsc{Goldmann, Paul} (31.01.1865 – 25.09.1935), \emph{Schriftsteller/Schriftstellerin, Journalist/Journalistin}|pw} hat Recht, sag ich Ihnen! – Gustav Schw.\pwindex{Schwarzkopf, Gustav 07.11.1853 – 13.11.1939@\textsc{Schwarzkopf, Gustav} (07.11.1853 – 13.11.1939), \emph{Schriftsteller/Schriftstellerin}|pw} und Leo V.\pwindex{Van-Jung, Leo 15.10.1866 – 02.07.1939@\textsc{Van-Jung, Leo} (15.10.1866 – 02.07.1939), \emph{Gesangspädagoge/Gesangspädagogin, Mathematiker/Mathematikerin}|pw} werden sicher Ihre Grüsse erwidern, sobald ich sie ihnen
               ausgerichtet habe. – Das gleiche nehm ich von Paula\pwindex{Beer-Hofmann, Paula 25.02.1879 – 30.10.1939@\textsc{Beer-Hofmann, Paula} (25.02.1879 – 30.10.1939)|pw}, ja beinah von Mirjam\pwindex{Beer-Hofmann, Mirjam 04.09.1897 – 24.12.1984@\textsc{Beer-Hofmann, Mirjam} (04.09.1897 – 24.12.1984)|pw} an. Sie
               wird einmal sehr gerührt sein, wenn sie als alte Frau ihrer Enkelin das Gedicht\pwindex{Schlaflied fuer Mirjam@\emph{Schlaflied für Mirjam}|pwv} vom Urgrosspapa
               vorlesen wird. Und auch Ihrer Urenkelin werden vielleicht Thränen ins Auge kommen.
               Auf Wiedersehen, womöglich noch vorher.\pend
           \pstart Herzlich Ihr \spacefill\mbox{Arthur.}\pend{}
\pstart
           \noindent{}(nach Steindorf\oindex{Steindorf am Ossiacher See@\textbf{Steindorf am Ossiacher See}, \emph{A.ADM3}|pw})\pend
           {\vspace{1\baselineskip}}
\pstart
           {\pb}Strophe I\pwindex{Schlaflied fuer Mirjam@\emph{Schlaflied für Mirjam}|pwv}\pend
           \settowidth{\longeste}{}\settowidth{\longestz}{}\settowidth{\longestd}{}\settowidth{\longestv}{}\settowidth{\longestf}{}\addtolength\longeste{1em}
        \addtolength\longestz{1em}
      
\pstart
           Strophe II\pwindex{Schlaflied fuer Mirjam@\emph{Schlaflied für Mirjam}|pwv}\pend
           \settowidth{\longeste}{}\settowidth{\longestz}{}\settowidth{\longestd}{}\settowidth{\longestv}{}\settowidth{\longestf}{}\addtolength\longeste{1em}
        \addtolength\longestz{1em}
      
\pstart
           \uline{Strophe III\pwindex{Schlaflied fuer Mirjam@\emph{Schlaflied für Mirjam}|pwv}}\pend
           \settowidth{\longeste}{}\settowidth{\longestz}{}\settowidth{\longestd}{}\settowidth{\longestv}{}\settowidth{\longestf}{}\addtolength\longeste{1em}
        \addtolength\longestz{1em}
      
\pstart
           Strophe IV\pwindex{Schlaflied fuer Mirjam@\emph{Schlaflied für Mirjam}|pwv}\pend
           \settowidth{\longeste}{}\settowidth{\longestz}{}\settowidth{\longestd}{}\settowidth{\longestv}{}\settowidth{\longestf}{}\addtolength\longeste{1em}
        \addtolength\longestz{1em}
      \selectlanguage{ngerman}\vspace{1em}{\vspace{1\baselineskip}}
\pstart
           {\pb}Schlaflied für Mirjam\pwindex{Beer-Hofmann, Mirjam 04.09.1897 – 24.12.1984@\textsc{Beer-Hofmann, Mirjam} (04.09.1897 – 24.12.1984)|pw}\pend
           {\vspace{1\baselineskip}}\stanza{}Schlaf mein Kind – schlaf, es iſt spät.Sieh, wie die Sonne zur Ruh dort geht;Hinter den Bergen ſtirbt ſie im Roth.Du, – du weißt nichts von Sonne und Tod,Wendeſt die Augen zum Licht und zum ScheinSchlaf – es ſind ſo viel Sonnen noch dein,Schlaf mein Kind – mein Kind, ſchlaf ein.\stanzaend{}\stanza{}– Schlaf mein Kind – der Abendwind wehtWeiß man, woher er ko{\geminationm}t – wohin er geht?Dunkel, verborgen die Wege hier ſindDir, und mir, und uns allen mein Kind.Blinde ſo geh’n wir, und gehen alleinKeiner kann Keinem Gefährte hier ſein –Schlaf mein Kind {[}–{]} mein Kind ſchlaf ein\stanzaend{}\stanza{}{\pb}Schlaf mein Kind – und horch
                     nicht auf mich;Sinn hat’s für mich nur – und Schall iſts für dich.Schall nur, wie Windeswehn, Waſſergerinn,Worte – vielleicht eines Lebens Gewinn.Was ich gewonnen, gräbt mit mir man ein,Keiner ka{\geminationn} Keinem ein Erbe hier sein,Schlaf mein Kind – mein Kind ſchlaf ein.\stanzaend{}\stanza{}Schläfſt du Mirjam\pwindex{Beer-Hofmann, Mirjam 04.09.1897 – 24.12.1984@\textsc{Beer-Hofmann, Mirjam} (04.09.1897 – 24.12.1984)|pw}? – Mirjam\pwindex{Beer-Hofmann, Mirjam 04.09.1897 – 24.12.1984@\textsc{Beer-Hofmann, Mirjam} (04.09.1897 – 24.12.1984)|pw} mein Kind,Ufer nur ſind wir, und tief in uns rinntBlut von Geweſ’nen – zu Ko{\geminationm}enden rollt’s;Blut unſrer Väter, voll Unruh und Stolz.In uns sind alle; wer fühlt ſich allein?Du biſt ihr Leben – ihr Leben iſt dein,Mirjam\pwindex{Beer-Hofmann, Mirjam 04.09.1897 – 24.12.1984@\textsc{Beer-Hofmann, Mirjam} (04.09.1897 – 24.12.1984)|pw} mein Leben – mein Kind ſchlaf
                     ein.\stanzaend{}
\pstart
           \spacefill\mbox{Richard Beer-Hofmann}\pend
           \selectlanguage{ngerman}\endnumbering\briefempfaengerindex{Beer-Hofmann, Richard@\textsc{Beer-Hofmann, Richard}!zzzSchnitzler, Arthur@\emph{von Arthur Schnitzler}!1898-06-171@{17. 6. 1898}|)be}\mylabel{L00806h}  \normalsize

\doendnotes{C}
\bigskip
\vfill

\clearpage

\footnotesize

\lohead{\textsc{register}}

% Definiere theindex-Environment komplett neu ohne reledmac
\makeatletter
\renewenvironment{theindex}{%
  \section*{\indexname}%
  \setlength{\parindent}{0pt}%
  \setlength{\parskip}{0pt plus 0.3pt}%
  \let\item\@idxitem
}{%
  \clearpage
}
\makeatother

\IfFileExists{\jobname-pw.ind}{\input{\jobname-pw.ind}}{}

\end{document}

      