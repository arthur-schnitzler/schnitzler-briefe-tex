%% latex-leseansicht-vorspann.tex
%% Vorspann für die Leseansicht.
%% Lädt die gemeinsame Datei latex-vorspann.tex mit nicht gesetztem Schalter.

\newif\ifkorrekturansicht
\korrekturansichtfalse

\input{../tex-inputs/latex-vorspann}


\section[Arthur Schnitzler an Richard Beer-Hofmann, 17. 6. 1898]{L00806 Arthur Schnitzler an Richard Beer-Hofmann, 17. 6. 1898}
\nopagebreak\mylabel{L00806v}
\rehead{ }\normalsize\beginnumbering\briefempfaengerindex{Beer-Hofmann, Richard@\textsc{Beer-Hofmann, Richard}!zzzSchnitzler, Arthur@\emph{von Arthur Schnitzler}!1898-06-171@{17. 6. 1898}|(be}
\toendnotes[C]{\smallbreak\pagebreak[2]}
\correspDesc{Versand  durch Arthur Schnitzler am 17. 6. 1898 in Wien
\newline{}Erhalt  durch Richard Beer-Hofmann im Zeitraum [18. 6. 1898
                  – 22. 6. 1898?] in Steindorf am Ossiacher See}\toendnotes[C]{\smallbreak}
\Standort{CUL, Schnitzler, B 8.1, S. 71.}
\physDesc{Brief, maschinenschriftliche Abschrift, 1 Blatt, 1 Seite, 3043 Zeichen
\newline{}Schreibmaschine
\newline{}Ordnung: von unbekannter Hand nummeriert: »119« }\Standort{YCGL, MSS 31.}
\physDesc{Korrekturen zu Schlaflied für Mirjam, 1 Blatt, 1 Seite, 3043 Zeichen
\newline{}Handschrift: Bleistift, deutsche Kurrent
\newline{}Ordnung: 1) mit Bleistift von unbekannter Hand beschriftet
                                    »Schnitzler: Korrekturen zu Beer-Hofmanns ›Schlaflied für
                                    Mirjam‹«  2) mit schwarzer Tinte von unbekannter Hand zur Zeile 6 der 2. Strophe:
                                    »doch«}\Standort{CUL, Schnitzler, B 8.}
\physDesc{Gedichtabschrift, handschriftliche Abschrift. 1 Blatt, 2 Seiten, 3043 Zeichen
\newline{}Handschrift: Bleistift, deutsche Kurrent
\newline{}Ordnung: mit Bleistift von unbekannter Hand nummeriert:
                                    »116« }
\buchAbdrucke{\weitereDrucke{Arthur Schnitzler, Richard Beer-Hofmann: \emph{Briefwechsel 1891–1931}. Herausgegeben von Konstanze Fliedl. Wien, Zürich: \emph{Europaverlag} 1992, S. 119–120, 118–119.} }\toendnotes[C]{\smallbreak}
\pstart
           \raggedleft{}{\pb}Wien\oindex{Wien@\textbf{Wien}, \emph{Verwaltungsgebiet}|pw}, 17. 6. 98.\pend
           \vspace{0.5em}
\pstart
           Lieber Richard, beiliegend mein Interpunktionsgefühl. Im
               wesentlichen liegt ja nicht viel dran. Hugo\pwindex{Hofmannsthal, Hugo von 1.\,2.\,1874 Wien – 15.\,7.\,1929 Rodaun@\textsc{Hofmannsthal, Hugo von} (1.\,2.\,1874 Wien – 15.\,7.\,1929 Rodaun), \emph{Schriftsteller}|pw}
               ist in der Brühl\oindex{Brühl@\textbf{Brühl}, \emph{Tal}|pw}, ich wollte gestern zu ihm;
               aber es regnete. Am Tag meiner Abfahrt hatte ich Regen bis Wr. Neustadt\oindex{Wiener Neustadt@\textbf{Wiener Neustadt}, \emph{Verwaltungsgebiet}|pw} – dann war es schön und blieb so bis gestern. Meine
               Sommerpläne sind verpfuscht. Man lässt sie\pwindex{Reinhard, Marie 13.\,3.\,1871 Wien – 18.\,3.\,1899 ebd.@\textsc{Reinhard, Marie} (13.\,3.\,1871 Wien – 18.\,3.\,1899 ebd.), \emph{Gesangspädagogin}|pwv} nicht mit mir reisen, so wird ein enervirendes Hin und
               Her herauskommen. Ich bleibe vor allem einmal bis Mitte Juli in Wien\oindex{Wien@\textbf{Wien}, \emph{Verwaltungsgebiet}|pw}; bin dann ein paar Tage mit ihr\pwindex{Reinhard, Marie 13.\,3.\,1871 Wien – 18.\,3.\,1899 ebd.@\textsc{Reinhard, Marie} (13.\,3.\,1871 Wien – 18.\,3.\,1899 ebd.), \emph{Gesangspädagogin}|pwv} und ihrer Schwester\pwindex{Burger, Caroline 11.\,7.\,1869 Wien – 15.\,3.\,1959 ebd.@\textsc{Burger, Caroline} (11.\,7.\,1869 Wien – 15.\,3.\,1959 ebd.)|pwv} sowie Schwager\pwindex{Burger, Rudolf *~6.\,12.\,1866 Wien@\textsc{Burger, Rudolf} (*~6.\,12.\,1866 Wien), \emph{Versicherungsdirektor}|pwv} in Gr.\oindex{Graz@\textbf{Graz}, \emph{Verwaltungsgebiet}|pw} zusammen – wohin ich vom 20.–27. Juli gehe,
               weiss ich nicht. (Wollen Sie irgendwo mit mir zusammen sein? Aber nicht in Steindorf\oindex{Steindorf am Ossiacher See@\textbf{Steindorf am Ossiacher See}, \emph{Verwaltungsgebiet}|pw}) Dann per Rad mit ihr\pwindex{Reinhard, Marie 13.\,3.\,1871 Wien – 18.\,3.\,1899 ebd.@\textsc{Reinhard, Marie} (13.\,3.\,1871 Wien – 18.\,3.\,1899 ebd.), \emph{Gesangspädagogin}|pwv} und den Ihren\pwindex{Burger, Caroline 11.\,7.\,1869 Wien – 15.\,3.\,1959 ebd.@\textsc{Burger, Caroline} (11.\,7.\,1869 Wien – 15.\,3.\,1959 ebd.)|pw}\pwindex{Reinhard, Carl 1.\,3.\,1868 Wien – 29.\,9.\,1904 ebd.@\textsc{Reinhard, Carl} (1.\,3.\,1868 Wien – 29.\,9.\,1904 ebd.), \emph{Kapellmeister}|pw}\pwindex{Reinhard, Therese 13.\,12.\,1844 Wien – 25.\,3.\,1926 ebd.@\textsc{Reinhard, Therese} (13.\,12.\,1844 Wien – 25.\,3.\,1926 ebd.)|pw} nach Tegernsee\oindex{Tegernsee@\textbf{Tegernsee}|pw}. – Von dort verschwind ich sofort; –
               wahrscheinlich in die Schweiz. Da werd ich eine Zeitlang mit der Mama\pwindex{Schnitzler, Louise 8.\,7.\,1840 Kőszeg – 9.\,9.\,1911 Wien@\textsc{Schnitzler, Louise} (8.\,7.\,1840 Kőszeg – 9.\,9.\,1911 Wien)|pwv} zusammen sein. (Vierwaldstädtersee\oindex{Vierwaldstättersee@\textbf{Vierwaldstättersee}, \emph{See}|pw}). Die letzte Augustwoche
               wahrscheinlich in Tegernsee\oindex{Tegernsee@\textbf{Tegernsee}|pw} – dann in den
               ersten Septembertagen wenns geht, durchs Ampezzo\oindex{Ampezzo@\textbf{Ampezzo}, \emph{Hauptstadt}|pw} per Rad nach Venedig\oindex{Venedig@\textbf{Venedig}|pw}. –\pend
           
\pstart
           Im übrigen arbeite ich und fühl mich aus den bekannten Ursachen nicht wohl. – (Milder
               Ausdruck.)\pend
           
\pstart
           Brief und Carton hab ich erhalten, danke sehr. Wie gehts Ihnen? Machen Sie was? Paul G.\pwindex{Goldmann, Paul 31.\,1.\,1865 Breslau – 25.\,9.\,1935 Wien@\textsc{Goldmann, Paul} (31.\,1.\,1865 Breslau – 25.\,9.\,1935 Wien), \emph{Schriftsteller, Journalist}|pw} hat Recht, sag ich Ihnen! – Gustav Schw.\pwindex{Schwarzkopf, Gustav 7.\,11.\,1853 Wien – 13.\,11.\,1939 ebd.@\textsc{Schwarzkopf, Gustav} (7.\,11.\,1853 Wien – 13.\,11.\,1939 ebd.), \emph{Schriftsteller}|pw} und Leo V.\pwindex{Van-Jung, Leo 15.\,10.\,1866 Odessa – 2.\,7.\,1939 Riga@\textsc{Van-Jung, Leo} (15.\,10.\,1866 Odessa – 2.\,7.\,1939 Riga), \emph{Gesangspädagoge, Mathematiker}|pw} werden sicher Ihre Grüsse erwidern, sobald ich sie ihnen
               ausgerichtet habe. – Das gleiche nehm ich von Paula\pwindex{Beer-Hofmann, Paula 25.\,2.\,1879 Wien – 30.\,10.\,1939 Zürich@\textsc{Beer-Hofmann, Paula} (25.\,2.\,1879 Wien – 30.\,10.\,1939 Zürich)|pw}, ja beinah von Mirjam\pwindex{Beer-Hofmann, Mirjam 4.\,9.\,1897 Wien – 24.\,12.\,1984 New York City@\textsc{Beer-Hofmann, Mirjam} (4.\,9.\,1897 Wien – 24.\,12.\,1984 New York City)|pw} an. Sie
               wird einmal sehr gerührt sein, wenn sie als alte Frau ihrer Enkelin das Gedicht\pwindex{Beer-Hofmann, Richard 11.\,7.\,1866 Wien – 26.\,9.\,1945 New York City@\textsc{Beer-Hofmann, Richard} (11.\,7.\,1866 Wien – 26.\,9.\,1945 New York City), \emph{Schriftsteller}!Schlaflied für Mirjam@\strich\emph{Schlaflied für Mirjam}|pwv} vom Urgrosspapa
               vorlesen wird. Und auch Ihrer Urenkelin werden vielleicht Thränen ins Auge kommen.
               Auf Wiedersehen, womöglich noch vorher.\pend
           \pstart Herzlich Ihr \spacefill\mbox{Arthur.}\pend{}
\pstart
           \noindent{}(nach Steindorf\oindex{Steindorf am Ossiacher See@\textbf{Steindorf am Ossiacher See}, \emph{Verwaltungsgebiet}|pw})\pend
           {\vspace{1\baselineskip}}
\pstart
           {\pb}Strophe I\pwindex{Beer-Hofmann, Richard 11.\,7.\,1866 Wien – 26.\,9.\,1945 New York City@\textsc{Beer-Hofmann, Richard} (11.\,7.\,1866 Wien – 26.\,9.\,1945 New York City), \emph{Schriftsteller}!Schlaflied für Mirjam@\strich\emph{Schlaflied für Mirjam}|pwv}\pend
           \settowidth{\longeste}{Zeile}\settowidth{\longestz}{5}\settowidth{\longestd}{nach ; ein –}\settowidth{\longestv}{}\settowidth{\longestf}{}\addtolength\longeste{1em}
        \addtolength\longestz{1em}
        \addtolength\longestd{1em}
      \pstart\noindent\makebox[\the\longeste][l]{Zeile}\makebox[\the\longestz][l]{2}
                  \makebox[\the\longestd][l]{nach Sieh \uuline{,}}\pend\pstart\noindent\makebox[\the\longeste][l]{Zeile}\makebox[\the\longestz][l]{3}
                  \makebox[\the\longestd][l]{– fort!}\pend\pstart\noindent\makebox[\the\longeste][l]{Zeile}\makebox[\the\longestz][l]{5}
                  \makebox[\the\longestd][l]{nach ; ein –}\pend
\pstart
           Strophe II\pwindex{Beer-Hofmann, Richard 11.\,7.\,1866 Wien – 26.\,9.\,1945 New York City@\textsc{Beer-Hofmann, Richard} (11.\,7.\,1866 Wien – 26.\,9.\,1945 New York City), \emph{Schriftsteller}!Schlaflied für Mirjam@\strich\emph{Schlaflied für Mirjam}|pwv}\pend
           \settowidth{\longeste}{Zeile}\settowidth{\longestz}{6,}\settowidth{\longestd}{dasxauchstört nicht.}\settowidth{\longestv}{}\settowidth{\longestf}{}\addtolength\longeste{1em}
        \addtolength\longestz{1em}
        \addtolength\longestd{1em}
      \pstart\noindent\makebox[\the\longeste][l]{Zeile}\makebox[\the\longestz][l]{2}
                  \makebox[\the\longestd][l]{ſtatt – lieber ,}\pend\pstart\noindent\makebox[\the\longeste][l]{}\makebox[\the\longestz][l]{4}
                  \makebox[\the\longestd][l]{das \uline{auch} stört nicht.}\pend\pstart\noindent\makebox[\the\longeste][l]{Zeile}\makebox[\the\longestz][l]{6,}
                  \makebox[\the\longestd][l]{lieber kein –}\pend
\pstart
           \uline{Strophe III\pwindex{Beer-Hofmann, Richard 11.\,7.\,1866 Wien – 26.\,9.\,1945 New York City@\textsc{Beer-Hofmann, Richard} (11.\,7.\,1866 Wien – 26.\,9.\,1945 New York City), \emph{Schriftsteller}!Schlaflied für Mirjam@\strich\emph{Schlaflied für Mirjam}|pwv}}\pend
           \settowidth{\longeste}{Zeile}\settowidth{\longestz}{7}\settowidth{\longestd}{ist ein Beistrich; an den gleichen Stellen Str I u II fehlt er –mm}\settowidth{\longestv}{}\settowidth{\longestf}{}\addtolength\longeste{1em}
        \addtolength\longestz{1em}
        \addtolength\longestd{1em}
      \pstart\noindent\makebox[\the\longeste][l]{Zeile}\makebox[\the\longestz][l]{1}
                  \makebox[\the\longestd][l]{– fort!}\pend\pstart\noindent\makebox[\the\longeste][l]{Zeile}\makebox[\the\longestz][l]{2}
                  \makebox[\the\longestd][l]{ebenſo}\pend\pstart\noindent\makebox[\the\longeste][l]{Zeile}\makebox[\the\longestz][l]{7}
                  \makebox[\the\longestd][l]{iſt ein Beiſtrich; an den gleichen Stellen Str I u II fehlt er –}\pend\pstart\noindent\makebox[\the\longeste][l]{}\makebox[\the\longestz][l]{}
                  \makebox[\the\longestd][l]{eins von beiden! –}\pend
\pstart
           Strophe IV\pwindex{Beer-Hofmann, Richard 11.\,7.\,1866 Wien – 26.\,9.\,1945 New York City@\textsc{Beer-Hofmann, Richard} (11.\,7.\,1866 Wien – 26.\,9.\,1945 New York City), \emph{Schriftsteller}!Schlaflied für Mirjam@\strich\emph{Schlaflied für Mirjam}|pwv}\pend
           \settowidth{\longeste}{Zeile}\settowidth{\longestz}{6,}\settowidth{\longestd}{der erste – fortm}\settowidth{\longestv}{}\settowidth{\longestf}{}\addtolength\longeste{1em}
        \addtolength\longestz{1em}
        \addtolength\longestd{1em}
      \pstart\noindent\makebox[\the\longeste][l]{Zeile}\makebox[\the\longestz][l]{4}
                  \makebox[\the\longestd][l]{lieber \uline{,} statt –}\pend\pstart\noindent\makebox[\the\longeste][l]{Zeile}\makebox[\the\longestz][l]{6,}
                  \makebox[\the\longestd][l]{der erſte – fort}\pend\pstart\noindent\makebox[\the\longeste][l]{Zeile}\makebox[\the\longestz][l]{7}
                  \makebox[\the\longestd][l]{der letzte –}\pend\selectlanguage{ngerman}\vspace{1em}{\vspace{1\baselineskip}}
\pstart
           {\pb}Schlaflied für Mirjam\pwindex{Beer-Hofmann, Mirjam 4.\,9.\,1897 Wien – 24.\,12.\,1984 New York City@\textsc{Beer-Hofmann, Mirjam} (4.\,9.\,1897 Wien – 24.\,12.\,1984 New York City)|pw}\pend
           {\vspace{1\baselineskip}}\stanza{}Schlaf mein Kind – schlaf, es iſt spät.\newverse{}Sieh, wie die Sonne zur Ruh dort geht;\newverse{}Hinter den Bergen{ }ſtirbt{ }ſie im Roth.\newverse{}Du, – du weißt nichts von Sonne und Tod,\newverse{}Wendeſt die Augen zum Licht und zum Schein\newverse{}Schlaf – es{ }ſind{ }ſo viel Sonnen noch dein,\newverse{}Schlaf mein Kind – mein Kind,{ }ſchlaf ein.\stanzaend{}\stanza{}– Schlaf mein Kind – der Abendwind weht\newverse{}Weiß man, woher er ko{\geminationm}t – wohin er geht?\newverse{}Dunkel, verborgen die Wege hier{ }ſind\newverse{}Dir, und mir, und uns allen mein Kind.\newverse{}Blinde{ }ſo geh’n wir, und gehen allein\newverse{}Keiner kann Keinem Gefährte hier{ }ſein –\newverse{}Schlaf mein Kind {[}–{]} mein Kind{ }ſchlaf ein\stanzaend{}\stanza{}{\pb}Schlaf mein Kind – und horch
                     nicht auf mich;\newverse{}Sinn hat’s für mich nur – und Schall iſts für dich.\newverse{}Schall nur, wie Windeswehn, Waſſergerinn,\newverse{}Worte – vielleicht eines Lebens Gewinn.\newverse{}Was ich gewonnen, gräbt mit mir man ein,\newverse{}Keiner ka{\geminationn} Keinem ein Erbe hier sein,\newverse{}Schlaf mein Kind – mein Kind{ }ſchlaf ein.\stanzaend{}\stanza{}Schläfſt du Mirjam\pwindex{Beer-Hofmann, Mirjam 4.\,9.\,1897 Wien – 24.\,12.\,1984 New York City@\textsc{Beer-Hofmann, Mirjam} (4.\,9.\,1897 Wien – 24.\,12.\,1984 New York City)|pw}? – Mirjam\pwindex{Beer-Hofmann, Mirjam 4.\,9.\,1897 Wien – 24.\,12.\,1984 New York City@\textsc{Beer-Hofmann, Mirjam} (4.\,9.\,1897 Wien – 24.\,12.\,1984 New York City)|pw} mein Kind,\newverse{}Ufer nur{ }ſind wir, und tief in uns rinnt\newverse{}Blut von Geweſ’nen – zu Ko{\geminationm}enden rollt’s;\newverse{}Blut unſrer Väter, voll Unruh und Stolz.\newverse{}In uns sind alle; wer fühlt{ }ſich allein?\newverse{}Du biſt ihr Leben – ihr Leben iſt dein,\newverse{}Mirjam\pwindex{Beer-Hofmann, Mirjam 4.\,9.\,1897 Wien – 24.\,12.\,1984 New York City@\textsc{Beer-Hofmann, Mirjam} (4.\,9.\,1897 Wien – 24.\,12.\,1984 New York City)|pw} mein Leben – mein Kind{ }ſchlaf
                     ein.\stanzaend{}
\pstart
           \spacefill\mbox{Richard Beer-Hofmann}\pend
           \selectlanguage{ngerman}\endnumbering\briefempfaengerindex{Beer-Hofmann, Richard@\textsc{Beer-Hofmann, Richard}!zzzSchnitzler, Arthur@\emph{von Arthur Schnitzler}!1898-06-171@{17. 6. 1898}|)be}\mylabel{L00806h}  \newcommand{\dateiname}{L00806}\newcommand{\titel}{Arthur Schnitzler an Richard Beer-Hofmann, 17. 6. 1898}\newcommand{\editorInnen}{Martin Anton Müller und Gerd-Hermann Susen}%% latex-leseansicht-abspann.tex
%% Abspann für die Leseansicht.
%% Der Schalter \ifkorrekturansicht ist bereits durch den Vorspann gesetzt.

%% latex-abspann.tex
%% Gemeinsamer Abspann für Korrekturansicht und Leseansicht.
%% Setzt den Schalter \ifkorrekturansicht voraus (gesetzt in den
%% einbindenden Dateien latex-korrekturansicht-abspann.tex bzw.
%% latex-leseansicht-abspann.tex).
%% ---------------------------------------------------------------

\normalsize

% Das esempio-Environment wird nur in der Leseansicht benötigt
\ifkorrekturansicht\else
\newenvironment{esempio}[3]%
{
    \vspace{1.5ex}
    \rlap{\underline{#1}}
    \par
    \setlength{\parindent}{0cm}
    \nopagebreak
    \leftskip=#2cm
    \rightskip=#3cm
}
{
    \par
}
\fi

\doendnotes{C}
\bigskip
\vfill

\clearpage

\footnotesize

\ifkorrekturansicht
  \lohead{\textsc{register}}
\fi

% theindex-Environment neu definieren ohne reledmac
\makeatletter
\renewenvironment{theindex}{%
  \ifkorrekturansicht
    \section*{\indexname}%
  \else
    \subsubsection*{Index der erwähnten Entitäten}%
  \fi
  \setlength{\parindent}{0pt}%
  \setlength{\parskip}{0pt plus 0.3pt}%
  \let\item\@idxitem
}{%
  \ifkorrekturansicht\clearpage\fi
}
\makeatother

\IfFileExists{\jobname-pw.ind}{\input{\jobname-pw.ind}}{}

% Quellenangabe nur in der Leseansicht
\ifkorrekturansicht\else
% Fallback-Definitionen, falls die .tex-Datei \titel etc. nicht gesetzt hat
\providecommand{\titel}{}
\providecommand{\editorInnen}{}
\providecommand{\dateiname}{\jobname}

\vspace{3cm}

\vfill

\footnotesize
\textsc{Quelle}: \titel. Herausgegeben von {\editorInnen}. In: \emph{Arthur Schnitzler: Briefwechsel mit Autorinnen und Autoren}.
 Digitale Edition, https://schnitzler-briefe.acdh.oeaw.ac.at/{\dateiname}.html (Stand \today)
\fi

\end{document}


