%% latex-leseansicht-vorspann.tex
%% Vorspann für die Leseansicht.
%% Lädt die gemeinsame Datei latex-vorspann.tex mit nicht gesetztem Schalter.

\newif\ifkorrekturansicht
\korrekturansichtfalse

\input{../tex-inputs/latex-vorspann}


         
         \newcommand{\erwaehntePersonen}{Personen: }
         \newcommand{\erwaehnteInstitutionen}{}
         \newcommand{\erwaehnteOrte}{}
         \newcommand{\erwaehnteWerke}{
               \section[Arthur Schnitzler an Richard Beer-Hofmann, 17. 6. 1898]{ Arthur Schnitzler an Richard Beer-Hofmann, 17. 6. 1898}\nopagebreak\mylabel{v}\rehead{ }\begin{ledgroupsized}[t]{13cm}\normalsize\beginnumbering \toendnotes[C]{\smallbreak\pagebreak[2]} \Standort{CUL, Schnitzler, B 8.1, S. 71.}
\physDesc{maschinelle Abschrift
\newline{}Schreibmaschine\newline{}Ordnung: von unbekannter Hand nummeriert: »119« }\Standort{YCGL, MSS 31.}
\physDesc{Korrekturen zu Schlaflied für Mirjam1 Blatt, 1 Seite
\newline{}Handschrift: Bleistift, deutsche Kurrent\newline{}Ordnung: 1) mit Bleistift von unbekannter
                                 Hand beschriftet »Schnitzler: Korrekturen zu Beer-Hofmanns
                                    ›Schlaflied für Mirjam‹«  2) mit Tinte von unbekannter Hand zur Zeile 6 der 2. Strophe:
                                    »doch«}\Standort{CUL, Schnitzler, B 8.}
\physDesc{1 Blatt, 2 Seiten, Gedichtabschrift
\newline{}Handschrift: Bleistift, deutsche Kurrent\newline{}Ordnung: mit Bleistift von unbekannter Hand nummeriert:
                                    »116« }\buchAbdrucke{\weitereDrucke{Arthur Schnitzler, Richard Beer-Hofmann: \emph{Briefwechsel 1891–1931}. Hg. Konstanze Fliedl. Wien, Zürich: \emph{Europaverlag} 1992, S. 119–120, 118–119.} }\toendnotes[C]{\smallbreak}\pstart
           \raggedleft{}{\pb}Wien\oindex{XXXX Ortsangabe fehlt|pw}, 17. 6. 98.\pend
           \pstart
           Lieber Richard, beiliegend mein Interpunktionsgefühl. Im
               wesentlichen liegt ja nicht viel dran. Hugo\pwindex{\textcolor{red}{\textsuperscript{XXXX1 indx}}|pw} ist
               in der Brühl\oindex{XXXX Ortsangabe fehlt|pw}, ich wollte gestern zu ihm; aber es
               regnete. Am Tag meiner Abfahrt hatte ich Regen bis Wr.
                  Neustadt\oindex{XXXX Ortsangabe fehlt|pw} – dann war es schön und blieb so bis gestern. Meine Sommerpläne sind
               verpfuscht. Man lässt sie\pwindex{\textcolor{red}{\textsuperscript{XXXX1 indx}}|pwv} nicht
               mit mir reisen, so wird ein enervirendes Hin und Her herauskommen. Ich bleibe vor
               allem einmal bis Mitte Juli in Wien\oindex{XXXX Ortsangabe fehlt|pw}; bin
               dann ein paar Tage mit ihr\pwindex{\textcolor{red}{\textsuperscript{XXXX1 indx}}|pwv} und
               ihrer Schwester\pwindex{\textcolor{red}{\textsuperscript{XXXX1 indx}}|pwv} sowie Schwager\pwindex{\textcolor{red}{\textsuperscript{XXXX1 indx}}|pwv} in Gr.\oindex{XXXX Ortsangabe fehlt|pw} zusammen – wohin ich vom
                  20.–27. Juli gehe, weiss ich nicht. (Wollen Sie irgendwo
               mit mir zusammen sein? Aber nicht in Steindorf\oindex{XXXX Ortsangabe fehlt|pw}) Dann per Rad mit ihr\pwindex{\textcolor{red}{\textsuperscript{XXXX1 indx}}|pwv} und den Ihren\pwindex{\textcolor{red}{\textsuperscript{XXXX1 indx}}|pw}\pwindex{\textcolor{red}{\textsuperscript{XXXX1 indx}}|pw}\pwindex{\textcolor{red}{\textsuperscript{XXXX1 indx}}|pw} nach Tegernsee\oindex{XXXX Ortsangabe fehlt|pw}. – Von dort verschwind ich sofort; –
               wahrscheinlich in die Schweiz. Da werd ich eine Zeitlang mit der Mama\pwindex{\textcolor{red}{\textsuperscript{XXXX1 indx}}|pwv} zusammen sein. (Vierwaldstädtersee\oindex{XXXX Ortsangabe fehlt|pw}). Die letzte Augustwoche
               wahrscheinlich in Tegernsee\oindex{XXXX Ortsangabe fehlt|pw} – dann in den ersten
                  Septembertagen wenns geht, durchs Ampezzo\oindex{XXXX Ortsangabe fehlt|pw} per Rad nach Venedig\oindex{XXXX Ortsangabe fehlt|pw}. –\pend
           \pstart
           Im übrigen arbeite ich und fühl mich aus den bekannten Ursachen nicht wohl. – (Milder
               Ausdruck.)\pend
           \pstart
           Brief und Carton hab ich erhalten, danke sehr. Wie gehts Ihnen? Machen Sie was? Paul G.\pwindex{\textcolor{red}{\textsuperscript{XXXX1 indx}}|pw} hat Recht, sag ich Ihnen! – Gustav Schw.\pwindex{\textcolor{red}{\textsuperscript{XXXX1 indx}}|pw} und Leo V.\pwindex{\textcolor{red}{\textsuperscript{XXXX1 indx}}|pw} werden sicher Ihre Grüsse erwidern, sobald ich sie ihnen ausgerichtet
               habe. – Das gleiche nehm ich von Paula\pwindex{\textcolor{red}{\textsuperscript{XXXX1 indx}}|pw}, ja beinah
               von Mirjam\pwindex{\textcolor{red}{\textsuperscript{XXXX1 indx}}|pw} an. Sie wird einmal sehr gerührt sein,
               wenn sie als alte Frau ihrer Enkelin das Gedicht\textcolor{red}{\textsuperscript{XXXX indx}} vom Urgrosspapa vorlesen wird. Und auch Ihrer Urenkelin
               werden vielleicht Thränen ins Auge kommen. Auf Wiedersehen, womöglich noch
               vorher.\pend
           \pstart Herzlich Ihr \spacefill\mbox{Arthur.}\pend{}\pstart
           \noindent{}(nach Steindorf\oindex{XXXX Ortsangabe fehlt|pw})\pend
           {\bigskip}\pstart
           \noindent{}{\pb}Strophe I\textcolor{red}{\textsuperscript{XXXX indx}}\pend
           \settowidth{\longeste}{Zeile}\settowidth{\longestz}{5}\settowidth{\longestd}{nach ; ein –}\settowidth{\longestv}{}\settowidth{\longestf}{}\addtolength\longeste{1em}
        \addtolength\longestz{1em}
        \addtolength\longestd{1em}
      \pstart\noindent\makebox[\the\longeste][l]{Zeile}\makebox[\the\longestz][l]{2}
                  \makebox[\the\longestd][l]{nach Sieh \uuline{,}}\pend\pstart\noindent\makebox[\the\longeste][l]{Zeile}\makebox[\the\longestz][l]{3}
                  \makebox[\the\longestd][l]{– fort!}\pend\pstart\noindent\makebox[\the\longeste][l]{Zeile}\makebox[\the\longestz][l]{5}
                  \makebox[\the\longestd][l]{nach ; ein –}\pend\pstart
           Strophe II\textcolor{red}{\textsuperscript{XXXX indx}}\pend
           \settowidth{\longeste}{Zeile}\settowidth{\longestz}{6,}\settowidth{\longestd}{das auch stört nicht.}\settowidth{\longestv}{}\settowidth{\longestf}{}\addtolength\longeste{1em}
        \addtolength\longestz{1em}
        \addtolength\longestd{1em}
      \pstart\noindent\makebox[\the\longeste][l]{Zeile}\makebox[\the\longestz][l]{2}
                  \makebox[\the\longestd][l]{ſtatt – lieber ,}\pend\pstart\noindent\makebox[\the\longeste][l]{}\makebox[\the\longestz][l]{4}
                  \makebox[\the\longestd][l]{das \uline{auch} stört nicht.}\pend\pstart\noindent\makebox[\the\longeste][l]{Zeile}\makebox[\the\longestz][l]{6,}
                  \makebox[\the\longestd][l]{lieber kein –}\pend\pstart
           \uline{Strophe III\textcolor{red}{\textsuperscript{XXXX indx}}}\pend
           \settowidth{\longeste}{Zeile}\settowidth{\longestz}{7}\settowidth{\longestd}{iſt ein Beiſtrich; an den gleichen Stellen Str I u II fehlt er –}\settowidth{\longestv}{}\settowidth{\longestf}{}\addtolength\longeste{1em}
        \addtolength\longestz{1em}
        \addtolength\longestd{1em}
      \pstart\noindent\makebox[\the\longeste][l]{Zeile}\makebox[\the\longestz][l]{1}
                  \makebox[\the\longestd][l]{– fort!}\pend\pstart\noindent\makebox[\the\longeste][l]{Zeile}\makebox[\the\longestz][l]{2}
                  \makebox[\the\longestd][l]{ebenſo}\pend\pstart\noindent\makebox[\the\longeste][l]{Zeile}\makebox[\the\longestz][l]{7}
                  \makebox[\the\longestd][l]{iſt ein Beiſtrich; an den gleichen Stellen Str I u II fehlt er –}\pend\pstart\noindent\makebox[\the\longeste][l]{}\makebox[\the\longestz][l]{}
                  \makebox[\the\longestd][l]{eins von beiden! –}\pend\pstart
           Strophe IV\textcolor{red}{\textsuperscript{XXXX indx}}\pend
           \settowidth{\longeste}{Zeile}\settowidth{\longestz}{6,}\settowidth{\longestd}{der erſte – fort}\settowidth{\longestv}{}\settowidth{\longestf}{}\addtolength\longeste{1em}
        \addtolength\longestz{1em}
        \addtolength\longestd{1em}
      \pstart\noindent\makebox[\the\longeste][l]{Zeile}\makebox[\the\longestz][l]{4}
                  \makebox[\the\longestd][l]{lieber \uline{,} statt –}\pend\pstart\noindent\makebox[\the\longeste][l]{Zeile}\makebox[\the\longestz][l]{6,}
                  \makebox[\the\longestd][l]{der erſte – fort}\pend\pstart\noindent\makebox[\the\longeste][l]{Zeile}\makebox[\the\longestz][l]{7}
                  \makebox[\the\longestd][l]{der letzte –}\pend{\bigskip}\pstart
           \noindent{}{\pb}Schlaflied für Mirjam\pwindex{\textcolor{red}{\textsuperscript{XXXX1 indx}}|pw}\pend
           {\bigskip}\stanza{}Schlaf mein Kind – schlaf, es iſt spät.\newverse{}Sieh, wie die Sonne zur Ruh dort geht;\newverse{}Hinter den Bergen ſtirbt ſie im Roth.\newverse{}Du, – du weißt nichts von Sonne und Tod,\newverse{}Wendeſt die Augen zum Licht und zum Schein\newverse{}Schlaf – es ſind ſo viel Sonnen noch dein,\newverse{}Schlaf mein Kind – mein Kind, ſchlaf ein.\stanzaend{}\stanza{}– Schlaf mein Kind – der Abendwind weht\newverse{}Weiß man, woher er ko{\geminationm}t – wohin er geht?\newverse{}Dunkel, verborgen die Wege hier ſind\newverse{}Dir, und mir, und uns allen mein Kind.\newverse{}Blinde ſo geh’n wir, und gehen allein\newverse{}Keiner kann Keinem Gefährte hier ſein –\newverse{}Schlaf mein Kind {[}–{]} mein Kind ſchlaf ein\stanzaend{}\stanza{}{\pb}Schlaf mein Kind – und horch
                     nicht auf mich;\newverse{}Sinn hat’s für mich nur – und Schall iſts für dich.\newverse{}Schall nur, wie Windeswehn, Waſſergerinn,\newverse{}Worte – vielleicht eines Lebens Gewinn.\newverse{}Was ich gewonnen, gräbt mit mir man ein,\newverse{}Keiner ka{\geminationn} Keinem ein Erbe hier sein,\newverse{}Schlaf mein Kind – mein Kind ſchlaf ein.\stanzaend{}\stanza{}Schläfſt du Mirjam\pwindex{\textcolor{red}{\textsuperscript{XXXX1 indx}}|pw}? – Mirjam\pwindex{\textcolor{red}{\textsuperscript{XXXX1 indx}}|pw} mein Kind,\newverse{}Ufer nur ſind wir, und tief in uns rinnt\newverse{}Blut von Geweſ’nen – zu Ko{\geminationm}enden rollt’s;\newverse{}Blut unſrer Väter, voll Unruh und Stolz.\newverse{}In uns sind alle; wer fühlt ſich allein?\newverse{}Du biſt ihr Leben – ihr Leben iſt dein,\newverse{}Mirjam\pwindex{\textcolor{red}{\textsuperscript{XXXX1 indx}}|pw} mein Leben – mein Kind ſchlaf
                     ein.\stanzaend{}\pstart
           \spacefill\mbox{Richard Beer-Hofmann}\pend
           
         
         \endnumbering\mylabel{h}\end{ledgroupsized}  \newcommand{\dateiname}{L00806}\newcommand{\titel}{Arthur Schnitzler an Richard Beer-Hofmann, 17. 6. 1898}\newcommand{\editorInnen}{Martin Anton Müller und Gerd-Hermann Susen}%% latex-leseansicht-abspann.tex
%% Abspann für die Leseansicht.
%% Der Schalter \ifkorrekturansicht ist bereits durch den Vorspann gesetzt.

%% latex-abspann.tex
%% Gemeinsamer Abspann für Korrekturansicht und Leseansicht.
%% Setzt den Schalter \ifkorrekturansicht voraus (gesetzt in den
%% einbindenden Dateien latex-korrekturansicht-abspann.tex bzw.
%% latex-leseansicht-abspann.tex).
%% ---------------------------------------------------------------

\normalsize

% Das esempio-Environment wird nur in der Leseansicht benötigt
\ifkorrekturansicht\else
\newenvironment{esempio}[3]%
{
    \vspace{1.5ex}
    \rlap{\underline{#1}}
    \par
    \setlength{\parindent}{0cm}
    \nopagebreak
    \leftskip=#2cm
    \rightskip=#3cm
}
{
    \par
}
\fi

\doendnotes{C}
\bigskip
\vfill

\clearpage

\footnotesize

\ifkorrekturansicht
  \lohead{\textsc{register}}
\fi

% theindex-Environment neu definieren ohne reledmac
\makeatletter
\renewenvironment{theindex}{%
  \ifkorrekturansicht
    \section*{\indexname}%
  \else
    \subsubsection*{Index der erwähnten Entitäten}%
  \fi
  \setlength{\parindent}{0pt}%
  \setlength{\parskip}{0pt plus 0.3pt}%
  \let\item\@idxitem
}{%
  \ifkorrekturansicht\clearpage\fi
}
\makeatother

\IfFileExists{\jobname-pw.ind}{\input{\jobname-pw.ind}}{}

% Quellenangabe nur in der Leseansicht
\ifkorrekturansicht\else
% Fallback-Definitionen, falls die .tex-Datei \titel etc. nicht gesetzt hat
\providecommand{\titel}{}
\providecommand{\editorInnen}{}
\providecommand{\dateiname}{\jobname}

\vspace{3cm}

\vfill

\footnotesize
\textsc{Quelle}: \titel. Herausgegeben von {\editorInnen}. In: \emph{Arthur Schnitzler: Briefwechsel mit Autorinnen und Autoren}.
 Digitale Edition, https://schnitzler-briefe.acdh.oeaw.ac.at/{\dateiname}.html (Stand \today)
\fi

\end{document}


      