%% latex-leseansicht-vorspann.tex
%% Vorspann für die Leseansicht.
%% Lädt die gemeinsame Datei latex-vorspann.tex mit nicht gesetztem Schalter.

\newif\ifkorrekturansicht
\korrekturansichtfalse

\input{../tex-inputs/latex-vorspann}


\section[Arthur Schnitzler an Gustav Schwarzkopf, {{[}}1. 10. 1909?{{]}}]{L04182 Arthur Schnitzler an Gustav Schwarzkopf, {[}1. 10. 1909?{]}}
\nopagebreak\mylabel{L04182v}
\rehead{ }\normalsize\beginnumbering\briefempfaengerindex{Schwarzkopf, Gustav@\textsc{Schwarzkopf, Gustav}!zzzSchnitzler, Arthur@\emph{von Arthur Schnitzler}!1909-10-011@{{[}1. 10. 1909?{]}}|(be}
\toendnotes[C]{\smallbreak\pagebreak[2]}
\correspDesc{Versand  durch Arthur Schnitzler am [1. 10. 1909?] in Wien
\newline{}Erhalt  durch Gustav Schwarzkopf im Zeitraum [1. 10. 1909 – 2. 10. 1909?] in Wien}\toendnotes[C]{\smallbreak}
\Standort{CUL, Schnitzler, B 96.}
\physDesc{Karte, 293 Zeichen
\newline{}Handschrift: Bleistift, deutsche Kurrent}\toendnotes[C]{\smallbreak}
\pstart
           \noindent{}{\pb}lieber Guſtav, we{\geminationn} Sie Luſt haben, \label{K_L04182-1v}\edtext{morgen
              Samſtag}{\lemma{\textnormal{\emph{morgen
              Samstag}}}\Cendnote{\textnormal{Die Karte ist nicht datiert. Mit der Einschränkung, dass 
                 Olga Schnitzler\pwindex{Schnitzler, Olga 17.\,1.\,1882 Wien – 13.\,1.\,1970 Lugano@\textsc{Schnitzler, Olga} (17.\,1.\,1882 Wien – 13.\,1.\,1970 Lugano), \emph{Schauspielerin, Sängerin}|pwk} mehrere Tage gelegen sein muss – und also weder dabei war, noch am
                 selben Tag eine andere größere Aktivität unternommen haben kann, reduziert es sich auf den Ausflug am 2. 10. 1909,
                 den Arthur Schnitzler allein unternahm. Olga\pwindex{Schnitzler, Olga 17.\,1.\,1882 Wien – 13.\,1.\,1970 Lugano@\textsc{Schnitzler, Olga} (17.\,1.\,1882 Wien – 13.\,1.\,1970 Lugano), \emph{Schauspielerin, Sängerin}|pwk} war nach
                 der Geburt von Lili\pwindex{Cappellini, Lili 13.\,9.\,1909 Wien – 26.\,7.\,1928 Venedig@\textsc{Cappellini, Lili} (13.\,9.\,1909 Wien – 26.\,7.\,1928 Venedig)|pwk} am 13. 9. 1909
                 bis zumindest 15. 10. 1909 nur eingeschränkt auf.}}}\label{K_L04182-1} früh mit mir einen »Ausflug«
              (nach Pötzleinsdorf\oindex{Wien@\textbf{Wien}!XVIII., Währing@\textbf{XVIII., Währing}!Pötzleinsdorf@\textbf{Pötzleinsdorf}, \emph{Ehemaliger Ort}|pw}, elektriſch!) zu machen,{ }ſo läuten Sie mich bitte zwiſchen
      ½  u ¾ 11 herunter. Abſage überflüſſig.
      Seh ich Sie morgen früh nicht, ſo \textcolor{gray}{event}{ }ſieht Aben\textcolor{gray}{d} im Kfhaus? O.\pwindex{Schnitzler, Olga 17.\,1.\,1882 Wien – 13.\,1.\,1970 Lugano@\textsc{Schnitzler, Olga} (17.\,1.\,1882 Wien – 13.\,1.\,1970 Lugano), \emph{Schauspielerin, Sängerin}|pw} liegt
               {\pb}noch immer und grüßt herzlich\pend
           
\pstart
           Ihr{\\[\baselineskip]}\spacefill\mbox{Arth}\pend
           \leftskip=0em{}\selectlanguage{ngerman}\endnumbering\briefempfaengerindex{Schwarzkopf, Gustav@\textsc{Schwarzkopf, Gustav}!zzzSchnitzler, Arthur@\emph{von Arthur Schnitzler}!1909-10-011@{{[}1. 10. 1909?{]}}|)be}\mylabel{L04182h}
\begin{anhang}
\end{anhang}\newcommand{\dateiname}{L04182}\newcommand{\titel}{Arthur Schnitzler an Gustav Schwarzkopf, [1. 10. 1909?]}\newcommand{\editorInnen}{Herausgegeben von Jahnke, SelmaMüller, Martin Anton}%% latex-leseansicht-abspann.tex
%% Abspann für die Leseansicht.
%% Der Schalter \ifkorrekturansicht ist bereits durch den Vorspann gesetzt.

%% latex-abspann.tex
%% Gemeinsamer Abspann für Korrekturansicht und Leseansicht.
%% Setzt den Schalter \ifkorrekturansicht voraus (gesetzt in den
%% einbindenden Dateien latex-korrekturansicht-abspann.tex bzw.
%% latex-leseansicht-abspann.tex).
%% ---------------------------------------------------------------

\normalsize

% Das esempio-Environment wird nur in der Leseansicht benötigt
\ifkorrekturansicht\else
\newenvironment{esempio}[3]%
{
    \vspace{1.5ex}
    \rlap{\underline{#1}}
    \par
    \setlength{\parindent}{0cm}
    \nopagebreak
    \leftskip=#2cm
    \rightskip=#3cm
}
{
    \par
}
\fi

\doendnotes{C}
\bigskip
\vfill

\clearpage

\footnotesize

\ifkorrekturansicht
  \lohead{\textsc{register}}
\fi

% theindex-Environment neu definieren ohne reledmac
\makeatletter
\renewenvironment{theindex}{%
  \ifkorrekturansicht
    \section*{\indexname}%
  \else
    \subsubsection*{Index der erwähnten Entitäten}%
  \fi
  \setlength{\parindent}{0pt}%
  \setlength{\parskip}{0pt plus 0.3pt}%
  \let\item\@idxitem
}{%
  \ifkorrekturansicht\clearpage\fi
}
\makeatother

\IfFileExists{\jobname-pw.ind}{\input{\jobname-pw.ind}}{}

% Quellenangabe nur in der Leseansicht
\ifkorrekturansicht\else
% Fallback-Definitionen, falls die .tex-Datei \titel etc. nicht gesetzt hat
\providecommand{\titel}{}
\providecommand{\editorInnen}{}
\providecommand{\dateiname}{\jobname}

\vspace{3cm}

\vfill

\footnotesize
\textsc{Quelle}: \titel. Herausgegeben von {\editorInnen}. In: \emph{Arthur Schnitzler: Briefwechsel mit Autorinnen und Autoren}.
 Digitale Edition, https://schnitzler-briefe.acdh.oeaw.ac.at/{\dateiname}.html (Stand \today)
\fi

\end{document}


