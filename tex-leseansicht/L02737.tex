%% latex-leseansicht-vorspann.tex
%% Vorspann für die Leseansicht.
%% Lädt die gemeinsame Datei latex-vorspann.tex mit nicht gesetztem Schalter.

\newif\ifkorrekturansicht
\korrekturansichtfalse

\input{../tex-inputs/latex-vorspann}


\section[Paul Goldmann an Arthur Schnitzler, 24. 6. {[}1895{]}]{L02737 Paul Goldmann an Arthur Schnitzler, 24. 6. [1895]}
\nopagebreak\mylabel{L02737v}
\rehead{ }\normalsize\beginnumbering\briefempfaengerindex{Schnitzler, Arthur@\textsc{Schnitzler, Arthur}!zzzGoldmann, Paul@\emph{von Paul Goldmann}!1895-06-242@{24. 6. [1895]}|(be}
\toendnotes[C]{\smallbreak\pagebreak[2]}
\correspDesc{Versand  durch Paul Goldmann am 24. 6. [1895] in Paris
\newline{}Erhalt  durch Arthur Schnitzler im Zeitraum [25. 6. 1895
                  – 29. 6. 1895?] in Wien}\toendnotes[C]{\smallbreak}
\Standort{DLA, A:Schnitzler, HS.NZ85.1.3165.}
\physDesc{Brief, 3 Blätter, 12 Seiten, 3187 Zeichen
\newline{}Handschrift: schwarze Tinte, deutsche Kurrent
\newline{}Schnitzler: 1) mit Bleistift das Jahr »95« vermerkt  2) mit rotem Buntstift fünf Unterstreichungen}\toendnotes[C]{\smallbreak}
\pstart
           {\pb}\textcolor{gray}{\textbf{\textbf{Frankfurter Zeitung\orgindex{Frankfurter Zeitung@Frankfurter Zeitung|pw}}}}\pend
           
\pstart
           \textcolor{gray}{\textbf{(\begin{otherlanguage}{french}Gazette de Francfort\end{otherlanguage}\orgindex{Frankfurter Zeitung@Frankfurter Zeitung|pw}).}}\pend
           
\pstart
           \textcolor{gray}{\textbf{\textbf{\begin{otherlanguage}{french}Fondateur M. L.
                                 Sonnemann\pwindex{Sonnemann, Leopold 29.\,10.\,1831 Höchberg – 30.\,10.\,1909 Frankfurt am Main@\textsc{Sonnemann, Leopold} (29.\,10.\,1831 Höchberg – 30.\,10.\,1909 Frankfurt am Main), \emph{Journalist, Herausgeber}|pw}\end{otherlanguage}.}}}\hfill \textsc{Paris\oindex{Paris@\textbf{Paris}, \emph{Hauptstadt}|pw}}, 24. Juni.\pend
           
\pstart
           \begin{otherlanguage}{french}\textcolor{gray}{\textbf{Journal politique, financier,}}\end{otherlanguage}\pend
           
\pstart
           \begin{otherlanguage}{french}\textcolor{gray}{\textbf{commercial et littéraire.}}\end{otherlanguage}\pend
           
\pstart
           \begin{otherlanguage}{french}\textcolor{gray}{\textbf{\textbf{Paraissant trois fois par jour.}}}\end{otherlanguage}\pend
           
\pstart
           \begin{otherlanguage}{french}\textcolor{gray}{\textbf{\textbf{Bureau à Paris\oindex{Paris@\textbf{Paris}, \emph{Hauptstadt}|pw}}}}\end{otherlanguage}\pend
           
\pstart
           \begin{otherlanguage}{french}\textcolor{gray}{\textbf{\textbf{24. Rue Feydeau\oindex{rue Feydeau@\textbf{rue Feydeau}, \emph{Straße}|pw}.}}}\end{otherlanguage}\pend
           
\pstart\center{}Mein lieber Freund,\pend\vspace{0.5em}
\pstart
           Eben bekomme ich Deinen lieben Brief. Nur raſch ein paar Zeilen. Mit Deinen Urtheilen
               über die geſandten \label{K_L02737-1v}\edtext{Druckſachen}{\lemma{\textnormal{\emph{Drucksachen}}}\Cendnote{\textnormal{Siehe XXXX Auszeichnungsfehler: Dokument L02735 nicht gefunden.
               }}}\label{K_L02737-1} – es iſt wirklich zu viel Mühe, daß Du mir lange darüber{ }ſchreibſt – bin ich
               Wort für Wort einverſtanden. Du mußt bedenken, daß {\pb}ich Dir kunterbunt durcheinander{ }ſchicke, was mir intereſſant erſcheint – Einiges
               wegen{ }ſtyliſtiſcher Schönheiten oder origineller Anſchauungen – Anderes wieder nur,
               weil es ein beachtenswerther Abſurditäts-Fall iſt (z. B. \label{K_L02737-2v}\edtext{\textsc{Rochefort\pwindex{Rochefort, Henri de 31.\,1.\,1830 Paris – 30.\,6.\,1913 Aix-les-Bains@\textsc{Rochefort, Henri de} (31.\,1.\,1830 Paris – 30.\,6.\,1913 Aix-les-Bains), \emph{Schriftsteller, Politiker, Journalist}|pw}}). \substVorne{}\textsuperscript{d}\substDazwischen{}D\substHinten{}er Fall \textsc{Wilde\pwindex{Wilde, Oscar 16.\,10.\,1854 Dublin – 30.\,11.\,1900 Paris@\textsc{Wilde, Oscar} (16.\,10.\,1854 Dublin – 30.\,11.\,1900 Paris), \emph{Schriftsteller}|pw}\pwindex{Adam, Paul 6.\,12.\,1862 Paris – 2.\,1.\,1920 ebd.@\textsc{Adam, Paul} (6.\,12.\,1862 Paris – 2.\,1.\,1920 ebd.), \emph{Schriftsteller, Kunstkritiker}!Assaut malicieux«@\strich\emph{»L’Assaut malicieux«}|pwv}}}{\lemma{\textnormal{\emph{Rochefort). … Wilde}}}\Cendnote{\textnormal{Der Polemiker Victor Henri de Rochefort\pwindex{Rochefort, Henri de 31.\,1.\,1830 Paris – 30.\,6.\,1913 Aix-les-Bains@\textsc{Rochefort, Henri de} (31.\,1.\,1830 Paris – 30.\,6.\,1913 Aix-les-Bains), \emph{Schriftsteller, Politiker, Journalist}|pwk}, der nach seiner politischen
                  Verfolgung sechs Jahre im London\oindex{London@\textbf{London}, \emph{Hauptstadt}|pwk}er Exil lebte,
                  wurde im Februar 1895 amnestiert und kehrte ruhmvoll
                  nach Paris\oindex{Paris@\textbf{Paris}, \emph{Hauptstadt}|pwk} zurück, wo er sich unter anderem
                  zur Dreyfus\pwindex{Dreyfus, Alfred 9.\,10.\,1859 Mulhouse – 12.\,7.\,1935 Paris@\textsc{Dreyfus, Alfred} (9.\,10.\,1859 Mulhouse – 12.\,7.\,1935 Paris), \emph{Militär}|pwk}-Affäre zu Wort meldete. Es ist
                  unklar, auf welchen Text Goldmann\pwindex{Goldmann, Paul 31.\,1.\,1865 Breslau – 25.\,9.\,1935 Wien@\textsc{Goldmann, Paul} (31.\,1.\,1865 Breslau – 25.\,9.\,1935 Wien), \emph{Schriftsteller, Journalist}|pwk} hier
                  Bezug nahm. Oscar Wilde\pwindex{Wilde, Oscar 16.\,10.\,1854 Dublin – 30.\,11.\,1900 Paris@\textsc{Wilde, Oscar} (16.\,10.\,1854 Dublin – 30.\,11.\,1900 Paris), \emph{Schriftsteller}|pwk} wurde wegen
                  »Unzucht« am 25. 5. 1895 zu zwei Jahren Zuchthaus
                  mit schwerer Zwangsarbeit verurteilt. Vgl. den gesandten Text\pwindex{Adam, Paul 6.\,12.\,1862 Paris – 2.\,1.\,1920 ebd.@\textsc{Adam, Paul} (6.\,12.\,1862 Paris – 2.\,1.\,1920 ebd.), \emph{Schriftsteller, Kunstkritiker}!Assaut malicieux«@\strich\emph{»L’Assaut malicieux«}|pwkv} von Paul Adam\pwindex{Adam, Paul 6.\,12.\,1862 Paris – 2.\,1.\,1920 ebd.@\textsc{Adam, Paul} (6.\,12.\,1862 Paris – 2.\,1.\,1920 ebd.), \emph{Schriftsteller, Kunstkritiker}|pwk}: \emph{»L’Assaut malicieux«}\pwindex{Adam, Paul 6.\,12.\,1862 Paris – 2.\,1.\,1920 ebd.@\textsc{Adam, Paul} (6.\,12.\,1862 Paris – 2.\,1.\,1920 ebd.), \emph{Schriftsteller, Kunstkritiker}!Assaut malicieux«@\strich\emph{»L’Assaut malicieux«}|pwk}. In: \emph{La Revue blanche}\pwindex{Revue blanche@\emph{La Revue blanche}|pwk}, Jg. 8, Nr. 47, 15. 5. 1895, S. 458–462. }}}\label{K_L02737-2} empört mich{ }ſchon lange. Das
                  engl\oindex{England@\textbf{England}, \emph{Land}|pwv}iſche Zuchthaus begreife
               ich {\pb}übrigens zur Noth, das{ }ſind dumme heuchleriſche
                  \begin{otherlanguage}{french}\textsc{Bourgeois}\end{otherlanguage}, in England\oindex{England@\textbf{England}, \emph{Land}|pw} – damit hat man{ }ſich
               abgefunden. Aber da gibt es dieſen Kerl\pwindex{Nordau, Max 29.\,7.\,1849 Budapest – 22.\,1.\,1923 Paris@\textsc{Nordau, Max} (29.\,7.\,1849 Budapest – 22.\,1.\,1923 Paris), \emph{Schriftsteller, Mediziner}|pwv}, den \textsc{Dr. Nordau\pwindex{Nordau, Max 29.\,7.\,1849 Budapest – 22.\,1.\,1923 Paris@\textsc{Nordau, Max} (29.\,7.\,1849 Budapest – 22.\,1.\,1923 Paris), \emph{Schriftsteller, Mediziner}|pw}}, der nach dem Urtheil an franzöſiſche\oindex{Frankreich@\textbf{Frankreich}|pwv} und deutschs\oindex{Deutschland@\textbf{Deutschland}|pwv} Blätter Briefe richtet, um zu{ }ſagen: man möge nur in{ }ſeinem Briefe
               nachleſen, wie er das \label{K_L02737-3v}\edtext{Schickſal \textsc{Wildes\pwindex{Wilde, Oscar 16.\,10.\,1854 Dublin – 30.\,11.\,1900 Paris@\textsc{Wilde, Oscar} (16.\,10.\,1854 Dublin – 30.\,11.\,1900 Paris), \emph{Schriftsteller}|pw}} voraus {\pb}berechnet}{\lemma{\textnormal{\emph{Schicksal … berechnet}}}\Cendnote{\textnormal{Max Nordau\pwindex{Nordau, Max 29.\,7.\,1849 Budapest – 22.\,1.\,1923 Paris@\textsc{Nordau, Max} (29.\,7.\,1849 Budapest – 22.\,1.\,1923 Paris), \emph{Schriftsteller, Mediziner}|pwk} hatte sich bereits in
                  seinem zweibändigen Buch \emph{Entartung}\pwindex{Nordau, Max 29.\,7.\,1849 Budapest – 22.\,1.\,1923 Paris@\textsc{Nordau, Max} (29.\,7.\,1849 Budapest – 22.\,1.\,1923 Paris), \emph{Schriftsteller, Mediziner}!Entartung (2 Bde.)@\strich\emph{Entartung (2 Bde.)}|pwk} (1892–1893) mit Oscar Wilde\pwindex{Wilde, Oscar 16.\,10.\,1854 Dublin – 30.\,11.\,1900 Paris@\textsc{Wilde, Oscar} (16.\,10.\,1854 Dublin – 30.\,11.\,1900 Paris), \emph{Schriftsteller}|pwk} beschäftigt, dessen vermeintliche Degeneration er
                  analysierte. Dass er damit den »Fall Wilde\pwindex{Wilde, Oscar 16.\,10.\,1854 Dublin – 30.\,11.\,1900 Paris@\textsc{Wilde, Oscar} (16.\,10.\,1854 Dublin – 30.\,11.\,1900 Paris), \emph{Schriftsteller}|pwk}«
                  hervorgesagt habe, betonte er beispielsweise in einem Interview\pwindex{Roche, Paul †~nach 1901@\textsc{Roche, Paul} (†~nach 1901), \emph{Journalist}!Oscar Wilde judgé par le docteur Max Nordau@\strich\emph{Oscar Wilde judgé par le docteur Max Nordau}|pwkv}: Paul Roche\pwindex{Roche, Paul †~nach 1901@\textsc{Roche, Paul} (†~nach 1901), \emph{Journalist}|pwk}: \emph{Oscar Wilde\pwindex{Wilde, Oscar 16.\,10.\,1854 Dublin – 30.\,11.\,1900 Paris@\textsc{Wilde, Oscar} (16.\,10.\,1854 Dublin – 30.\,11.\,1900 Paris), \emph{Schriftsteller}|pwk} judgé par le docteur Max Nordau\pwindex{Nordau, Max 29.\,7.\,1849 Budapest – 22.\,1.\,1923 Paris@\textsc{Nordau, Max} (29.\,7.\,1849 Budapest – 22.\,1.\,1923 Paris), \emph{Schriftsteller, Mediziner}|pwk}}\pwindex{Roche, Paul †~nach 1901@\textsc{Roche, Paul} (†~nach 1901), \emph{Journalist}!Oscar Wilde judgé par le docteur Max Nordau@\strich\emph{Oscar Wilde judgé par le docteur Max Nordau}|pwk}. In: \emph{Le Gaulois}\pwindex{Le Gaulois@\emph{Le Gaulois}|pwk}, Jg. 29, Nr. 5443,
                        10. 4. 1895, S. 1–2.}}}\label{K_L02737-3} – um alſo
               aus dem Schickſal dieſes Bemitleidenswerthen\pwindex{Wilde, Oscar 16.\,10.\,1854 Dublin – 30.\,11.\,1900 Paris@\textsc{Wilde, Oscar} (16.\,10.\,1854 Dublin – 30.\,11.\,1900 Paris), \emph{Schriftsteller}|pwv}{ }ſich eine Reklame für{ }ſeinen Dekadenz-Schwindel zu
               machen. \uline{Das} macht mir das Blut kochen – da möchte ich
               losprügeln können mit Fäuſten und Stiefel-Abſätzen.\pend
           
\pstart
           Über einen fran\oindex{Frankreich@\textbf{Frankreich}|pwv}zöſiſchen
                  \label{K_L02737-4v}\edtext{Verleger}{\lemma{\textnormal{\emph{Verleger}}}\Cendnote{\textnormal{\emph{Liebelei}\pwindex{Schnitzler, Arthur 15.\,5.\,1862 Wien – 21.\,10.\,1931 ebd.@\textsc{Schnitzler, Arthur} (15.\,5.\,1862 Wien – 21.\,10.\,1931 ebd.), \emph{Schriftsteller, Mediziner}!Liebelei. Schauspiel in drei Akten@\strich\emph{Liebelei. Schauspiel in drei Akten}|pwk} wurde 1896 und 1897 von Jean Thorel\pwindex{Thorel, Jean 11.\,9.\,1859 Éragny – 20.\,8.\,1916 Enghien-les-Bains@\textsc{Thorel, Jean} (11.\,9.\,1859 Éragny – 20.\,8.\,1916 Enghien-les-Bains), \emph{Übersetzer, Dramatiker}|pwk} ins Französische übersetzt, jedoch erst in der
                  Übersetzung von Suzanne Clauser\pwindex{Clauser, Suzanne 16.\,5.\,1898 Wien – 11.\,9.\,1981 Paris@\textsc{Clauser, Suzanne} (16.\,5.\,1898 Wien – 11.\,9.\,1981 Paris), \emph{Schriftstellerin, Übersetzerin}|pwk} im Jahr
                     1933 unter dem Titel \emph{Liebelei (\begin{otherlanguage}{french}amourette\end{otherlanguage})}\pwindex{Schnitzler, Arthur 15.\,5.\,1862 Wien – 21.\,10.\,1931 ebd.@\textsc{Schnitzler, Arthur} (15.\,5.\,1862 Wien – 21.\,10.\,1931 ebd.), \emph{Schriftsteller, Mediziner}!Liebelei (amourette): pièce en trois actes@\strich\emph{Liebelei (amourette): pièce en trois actes}|pwk} gedruckt.}}}\label{K_L02737-4}
               aus einer Aufführung {\pb}in \textsc{Paris\oindex{Paris@\textbf{Paris}, \emph{Hauptstadt}|pw}} denke ich{ }ſeit Empfang Deines letzten lieben Briefes nach. Das wird aber{ }ſchwer{ }ſein. Die Pariſ\oindex{Paris@\textbf{Paris}, \emph{Hauptstadt}|pw}er Verleger{ }ſind noch{ }ſchlimmeres
               Geſindel als die deutſch\oindex{Deutschland@\textbf{Deutschland}|pwv}en. Die
                  deutsch\oindex{Deutschland@\textbf{Deutschland}|pwv}en zahlen nur nichts,
               die fran\oindex{Frankreich@\textbf{Frankreich}|pwv}zöſiſchen verlangen,
               daß man ihnen {\pb}zahlt. Wärſt Du dazu bereit? Eine Aufführung\pwindex{Schnitzler, Arthur 15.\,5.\,1862 Wien – 21.\,10.\,1931 ebd.@\textsc{Schnitzler, Arthur} (15.\,5.\,1862 Wien – 21.\,10.\,1931 ebd.), \emph{Schriftsteller, Mediziner}!Liebelei. Schauspiel in drei Akten@\strich\emph{Liebelei. Schauspiel in drei Akten}|pwv} wäre eher möglich –
               aber erſt \uline{nach} einer Aufführung\pwindex{Schnitzler, Arthur 15.\,5.\,1862 Wien – 21.\,10.\,1931 ebd.@\textsc{Schnitzler, Arthur} (15.\,5.\,1862 Wien – 21.\,10.\,1931 ebd.), \emph{Schriftsteller, Mediziner}!Liebelei. Schauspiel in drei Akten@\strich\emph{Liebelei. Schauspiel in drei Akten}|pwv} in Berlin\oindex{Berlin@\textbf{Berlin}, \emph{Hauptstadt}|pw} oder Wien\oindex{Wien@\textbf{Wien}, \emph{Verwaltungsgebiet}|pw}, nicht
               zugleich. Wir reden noch darüber. Ich hab’ die Sache{ }ſchon lange im Auge und hab’
               auch{ }ſchon einige Schritte gethan.\pend
           
\pstart
           {\pb}Das iſt aber immer noch nicht der große Brief – nur
               ein paar raſche Worte, ehe die \strikeout{Ka}{ }Kammer\orgindex{Französische Abgeordnetenkammer@Französische Abgeordnetenkammer|pwv} beginnt. Darum{ }ſchreibe
               ich nicht über allerlei Perſönliches, das ich längſt berühren möchte.\pend
           
\pstart
           Es wäre mir eine große Freude, könnt’ ich Dich im Sommer{ }ſehen; aber ich möchte keine
                  {\pb}Störung bringen in Deine Reiſe-Pläne. \strikeout{\textcolor{gray}{×}} Ich muß nach \textsc{Toelz\oindex{Bad Tölz@\textbf{Bad Tölz}, \emph{Hauptstadt}|pw}} gehen u. muß dort vier Wochen bleiben. Das iſt nicht weit von \textsc{Muenchen\oindex{München@\textbf{München}|pw}}. Wie machen wirs alſo?\pend
           
\pstart
           Reiſe glücklich, liebſter Freund! Ich weiß, wie gern Du hinausfährſt, und freue mich
               für Dich. Laß’ die \strikeout{Hypoch\textcolor{gray}{ond}}{ }{\pb}\label{K_L02737-5v}\edtext{Hypochondrien}{\lemma{\textnormal{\emph{Hypochondrien}}}\Cendnote{\textnormal{Schnitzler notierte 1895 immer wieder hypochondrische Zustände im \emph{Tagebuch}\pwindex{Schnitzler, Arthur 15.\,5.\,1862 Wien – 21.\,10.\,1931 ebd.@\textsc{Schnitzler, Arthur} (15.\,5.\,1862 Wien – 21.\,10.\,1931 ebd.), \emph{Schriftsteller, Mediziner}!Tagebuch@\strich\emph{Tagebuch}|pwk}, zuletzt am 22. 6. 1895.}}}\label{K_L02737-5} in Wien\oindex{Wien@\textbf{Wien}, \emph{Verwaltungsgebiet}|pw}! Die Welt iſt{ }ſchön, und Du biſt jung und ein glücklicher
               Menſch, – ja, glaub’ mir, ein glücklicher Menſch.\pend
           
\pstart
           Ich höre wohl Deine Unterwegs-Adreſſe.\pend
           
\pstart
           \textsc{Burckhardt\pwindex{Burckhard, Max Eugen 14.\,7.\,1854 Korneuburg – 16.\,3.\,1912 Wien@\textsc{Burckhard, Max Eugen} (14.\,7.\,1854 Korneuburg – 16.\,3.\,1912 Wien), \emph{Schriftsteller, Rechtswissenschaftler, Theaterleiter}|pw}} iſt \label{K_L02737-6v}\edtext{unglaublich}{\lemma{\textnormal{\emph{unglaublich}}}\Cendnote{\textnormal{Am XXXX Auszeichnungsfehler: Dokument L00454 nicht gefunden schrieb Schnitzler an Richard Beer-Hofmann\pwindex{Beer-Hofmann, Richard 11.\,7.\,1866 Wien – 26.\,9.\,1945 New York City@\textsc{Beer-Hofmann, Richard} (11.\,7.\,1866 Wien – 26.\,9.\,1945 New York City), \emph{Schriftsteller}|pwk}
                  von dem Gerücht, \emph{Liebelei}\pwindex{Schnitzler, Arthur 15.\,5.\,1862 Wien – 21.\,10.\,1931 ebd.@\textsc{Schnitzler, Arthur} (15.\,5.\,1862 Wien – 21.\,10.\,1931 ebd.), \emph{Schriftsteller, Mediziner}!Liebelei. Schauspiel in drei Akten@\strich\emph{Liebelei. Schauspiel in drei Akten}|pwk} würde am \emph{Burgtheater}\orgindex{Burgtheater@Burgtheater|pwk} nicht mehr aufgeführt werden. Schnitzler konfrontierte Max Burckhard\pwindex{Burckhard, Max Eugen 14.\,7.\,1854 Korneuburg – 16.\,3.\,1912 Wien@\textsc{Burckhard, Max Eugen} (14.\,7.\,1854 Korneuburg – 16.\,3.\,1912 Wien), \emph{Schriftsteller, Rechtswissenschaftler, Theaterleiter}|pwk} damit, doch der machte deutlich, dass er es
                  unter allen Umständen aufführen werde. Vgl. A. S.: \emph{Tagebuch}, 16. 6. 1895.}}}\label{K_L02737-6}. Es wäre {\pb}ſogar{ }ſchon komiſch, wenns Dich nicht gerade träfe.
               Aber auch ich bin feſt überzeugt: das Stück\pwindex{Schnitzler, Arthur 15.\,5.\,1862 Wien – 21.\,10.\,1931 ebd.@\textsc{Schnitzler, Arthur} (15.\,5.\,1862 Wien – 21.\,10.\,1931 ebd.), \emph{Schriftsteller, Mediziner}!Liebelei. Schauspiel in drei Akten@\strich\emph{Liebelei. Schauspiel in drei Akten}|pwv}{ }\uline{wird} aufgeführt.\pend
           
\pstart
           Dem \textsc{Fuchs\pwindex{Fuchs, Isidor 25.\,9.\,1849 Lipnik Górny – um den 20.8.1920 Schruns@\textsc{Fuchs, Isidor} (25.\,9.\,1849 Lipnik Górny – um den 20.8.1920 Schruns), \emph{Schriftsteller, Journalist}|pw}} thatſt \strikeout{\textcolor{gray}{o}h} Du Unrecht. Er iſt kein \textsc{Concordia\orgindex{Concordia. Journalisten- und Schriftstellerverein@Concordia. Journalisten- und Schriftstellerverein|pw}}-Literat mehr,{ }ſondern ein lieber, neidloſer, treuer, einfacher Menſch\pwindex{Fuchs, Isidor 25.\,9.\,1849 Lipnik Górny – um den 20.8.1920 Schruns@\textsc{Fuchs, Isidor} (25.\,9.\,1849 Lipnik Górny – um den 20.8.1920 Schruns), \emph{Schriftsteller, Journalist}|pwv}, der alt wird und gut
               wird. Als Menſch {\pb}tauſendmal mehr werth, wie \textsc{Herzl\pwindex{Herzl, Theodor 2.\,5.\,1860 Budapest – 3.\,7.\,1904 Edlach@\textsc{Herzl, Theodor} (2.\,5.\,1860 Budapest – 3.\,7.\,1904 Edlach), \emph{Schriftsteller, Journalist}|pw}}.\pend
           
\pstart
           \textsc{Herzl\pwindex{Herzl, Theodor 2.\,5.\,1860 Budapest – 3.\,7.\,1904 Edlach@\textsc{Herzl, Theodor} (2.\,5.\,1860 Budapest – 3.\,7.\,1904 Edlach), \emph{Schriftsteller, Journalist}|pw}}{ }ſchreibt einen \label{K_L02737-7v}\edtext{Roman}{\lemma{\textnormal{\emph{Roman}}}\Cendnote{\textnormal{Im Sommer 1895,
                  kurz vor seiner Rückkehr nach Wien\oindex{Wien@\textbf{Wien}, \emph{Verwaltungsgebiet}|pwk}, spielte Theodor Herzl\pwindex{Herzl, Theodor 2.\,5.\,1860 Budapest – 3.\,7.\,1904 Edlach@\textsc{Herzl, Theodor} (2.\,5.\,1860 Budapest – 3.\,7.\,1904 Edlach), \emph{Schriftsteller, Journalist}|pwk} mit der Idee, einen
                  politischen Roman zu schreiben. Vgl. Shlomo Avineri: Herzl\pwindex{Herzl, Theodor 2.\,5.\,1860 Budapest – 3.\,7.\,1904 Edlach@\textsc{Herzl, Theodor} (2.\,5.\,1860 Budapest – 3.\,7.\,1904 Edlach), \emph{Schriftsteller, Journalist}|pwk}. Theodor
                        Herzl\pwindex{Herzl, Theodor 2.\,5.\,1860 Budapest – 3.\,7.\,1904 Edlach@\textsc{Herzl, Theodor} (2.\,5.\,1860 Budapest – 3.\,7.\,1904 Edlach), \emph{Schriftsteller, Journalist}|pwk} und die Gründung des jüdischen Staates. Berlin\oindex{Berlin@\textbf{Berlin}, \emph{Hauptstadt}|pwk}: eBook Jüdischer Verlag im \emph{Suhrkamp}\orgindex{Suhrkamp Verlag@Suhrkamp Verlag|pwk} Verlag 2016,
                     S. 181.}}}\label{K_L02737-7}.\pend
           
\pstart
           Was macht \textsc{\strikeout{Ric}\pwindex{Beer-Hofmann, Richard 11.\,7.\,1866 Wien – 26.\,9.\,1945 New York City@\textsc{Beer-Hofmann, Richard} (11.\,7.\,1866 Wien – 26.\,9.\,1945 New York City), \emph{Schriftsteller}|pwv}}{ }\textsc{Richard\pwindex{Beer-Hofmann, Richard 11.\,7.\,1866 Wien – 26.\,9.\,1945 New York City@\textsc{Beer-Hofmann, Richard} (11.\,7.\,1866 Wien – 26.\,9.\,1945 New York City), \emph{Schriftsteller}|pw}}? Schreibt er was? Und{ }ſehe ich ihn?\pend
           
\pstart
           Wie geht die »Zeit\orgindex{Zeit. Wiener Wochenschrift@Die Zeit. Wiener Wochenschrift|pw}«?\pend
           
\pstart
           Die Überſetzung\pwindex{Schnitzler, Arthur 15.\,5.\,1862 Wien – 21.\,10.\,1931 ebd.@\textsc{Schnitzler, Arthur} (15.\,5.\,1862 Wien – 21.\,10.\,1931 ebd.), \emph{Schriftsteller, Mediziner}!Mourir. Roman@\strich\emph{Mourir. Roman}|pwv} von »Sterben\pwindex{Schnitzler, Arthur 15.\,5.\,1862 Wien – 21.\,10.\,1931 ebd.@\textsc{Schnitzler, Arthur} (15.\,5.\,1862 Wien – 21.\,10.\,1931 ebd.), \emph{Schriftsteller, Mediziner}!Sterben. Novelle@\strich\emph{Sterben. Novelle}|pw}« iſt nicht übel. Dank für die
               Zuſendung.\pend
           
\pstart
           {\pb}\textsc{Bahr\pwindex{Bahr, Hermann 19.\,7.\,1863 Linz – 15.\,1.\,1934 München@\textsc{Bahr, Hermann} (19.\,7.\,1863 Linz – 15.\,1.\,1934 München), \emph{Schriftsteller, Kritiker}|pw}} hat hierher geſchrieben, um die Unterſchriften der fran\oindex{Frankreich@\textbf{Frankreich}|pwv}zöſiſchen Schriftſteller-Welt \strikeout{zur} zum Verlangen einer Aufführung eines \label{K_L02737-8v}\edtext{\textsc{Goldschmidtschen\pwindex{Goldschmidt, Adalbert von 5.\,5.\,1848 Wien – 21.\,12.\,1906 ebd.@\textsc{Goldschmidt, Adalbert von} (5.\,5.\,1848 Wien – 21.\,12.\,1906 ebd.), \emph{Schriftsteller, Komponist}|pw}}{ }Muſik-Dramas\pwindex{Goldschmidt, Adalbert von 5.\,5.\,1848 Wien – 21.\,12.\,1906 ebd.@\textsc{Goldschmidt, Adalbert von} (5.\,5.\,1848 Wien – 21.\,12.\,1906 ebd.), \emph{Schriftsteller, Komponist}!Gaea. Musikdrama@\strich\emph{Gaea. Musikdrama}|pwv}}{\lemma{\textnormal{\emph{Goldschmidtschen Musik-Dramas}}}\Cendnote{\textnormal{Das monumentale Musikdrama \emph{Gäa}\pwindex{Goldschmidt, Adalbert von 5.\,5.\,1848 Wien – 21.\,12.\,1906 ebd.@\textsc{Goldschmidt, Adalbert von} (5.\,5.\,1848 Wien – 21.\,12.\,1906 ebd.), \emph{Schriftsteller, Komponist}!Gaea. Musikdrama@\strich\emph{Gaea. Musikdrama}|pwk} von Adalbert von
                     Goldschmidt\pwindex{Goldschmidt, Adalbert von 5.\,5.\,1848 Wien – 21.\,12.\,1906 ebd.@\textsc{Goldschmidt, Adalbert von} (5.\,5.\,1848 Wien – 21.\,12.\,1906 ebd.), \emph{Schriftsteller, Komponist}|pwk} wurde seit 1892 von Bahr\pwindex{Bahr, Hermann 19.\,7.\,1863 Linz – 15.\,1.\,1934 München@\textsc{Bahr, Hermann} (19.\,7.\,1863 Linz – 15.\,1.\,1934 München), \emph{Schriftsteller, Kritiker}|pwk} für die Aufführung propagiert (vgl. Hermann Bahr\pwindex{Bahr, Hermann 19.\,7.\,1863 Linz – 15.\,1.\,1934 München@\textsc{Bahr, Hermann} (19.\,7.\,1863 Linz – 15.\,1.\,1934 München), \emph{Schriftsteller, Kritiker}|pwk}: \emph{Adalbert von Goldschmidt}\pwindex{Bahr, Hermann 19.\,7.\,1863 Linz – 15.\,1.\,1934 München@\textsc{Bahr, Hermann} (19.\,7.\,1863 Linz – 15.\,1.\,1934 München), \emph{Schriftsteller, Kritiker}!Adalbert von Goldschmidt@\strich\emph{Adalbert von Goldschmidt}|pwk}. In: \emph{Deutsche Zeitung}\pwindex{Deutsche Zeitung@\emph{Deutsche Zeitung}|pwk}, Jg. 22, Nr. 7490,
                        4. 11. 1892, Morgen-Ausgabe, S. 6). Erster Anlass war
                  dazu das Erscheinen einer französischen Übersetzung durch Catulle Mendès\pwindex{Mendès, Catulle 20.\,5.\,1841 Bordeaux – 8.\,2.\,1909 Saint-Germain-en-Laye@\textsc{Mendès, Catulle} (20.\,5.\,1841 Bordeaux – 8.\,2.\,1909 Saint-Germain-en-Laye), \emph{Schriftsteller}|pwk} (\emph{Ghea. Poeme dramatique}\pwindex{Goldschmidt, Adalbert von 5.\,5.\,1848 Wien – 21.\,12.\,1906 ebd.@\textsc{Goldschmidt, Adalbert von} (5.\,5.\,1848 Wien – 21.\,12.\,1906 ebd.), \emph{Schriftsteller, Komponist}!Ghea. Poeme dramatique@\strich\emph{Ghea. Poeme dramatique}|pwk}. Mis en Français
                     par Catulle Mendès\pwindex{Mendès, Catulle 20.\,5.\,1841 Bordeaux – 8.\,2.\,1909 Saint-Germain-en-Laye@\textsc{Mendès, Catulle} (20.\,5.\,1841 Bordeaux – 8.\,2.\,1909 Saint-Germain-en-Laye), \emph{Schriftsteller}|pwk}.
                     Paris: \emph{G. Charpentier et E.
                        Fasquelle}\orgindex{Charpentier@Charpentier|pwk}{ }1893.) Eine vollständige Inszenierung würde drei Tage dauern. Auf Initiative
                  von Bahr\pwindex{Bahr, Hermann 19.\,7.\,1863 Linz – 15.\,1.\,1934 München@\textsc{Bahr, Hermann} (19.\,7.\,1863 Linz – 15.\,1.\,1934 München), \emph{Schriftsteller, Kritiker}|pwk} entstanden Komitees in Wien\oindex{Wien@\textbf{Wien}, \emph{Verwaltungsgebiet}|pwk}, Berlin\oindex{Berlin@\textbf{Berlin}, \emph{Hauptstadt}|pwk}
                  und Paris\oindex{Paris@\textbf{Paris}, \emph{Hauptstadt}|pwk}, die die Aufführung bewerkstelligen
                  sollten. Goldmann\pwindex{Goldmann, Paul 31.\,1.\,1865 Breslau – 25.\,9.\,1935 Wien@\textsc{Goldmann, Paul} (31.\,1.\,1865 Breslau – 25.\,9.\,1935 Wien), \emph{Schriftsteller, Journalist}|pwk} irrte sich jedoch in der
                  Bereitwilligkeit fran\oindex{Frankreich@\textbf{Frankreich}|pwkv}zösischer Kulturgrößen, ihren Namen dafür herzugeben. Im März 1896 erschien eine Petition\pwindex{Gäa«@\emph{»Gäa«}|pwkv}, die die Aufführung forderte (\emph{»Gäa«}\pwindex{Gäa«@\emph{»Gäa«}|pwk}. In: \emph{Neue Deutsche Rundschau}\pwindex{Neue Deutsche Rundschau@\emph{Neue Deutsche Rundschau}|pwk}, Jg. 7, H. 3, März 1896,
                     S. 3039. Sie war unterzeichnet von Julius Bauer\pwindex{Bauer, Julius 15.\,10.\,1853 Szigetvár – 11.\,6.\,1941 Wien@\textsc{Bauer, Julius} (15.\,10.\,1853 Szigetvár – 11.\,6.\,1941 Wien), \emph{Schriftsteller, Journalist, Kritiker}|pwk}, Reinhold Begas\pwindex{Begas, Reinhold 15.\,7.\,1831 Berlin – 3.\,8.\,1911 ebd.@\textsc{Begas, Reinhold} (15.\,7.\,1831 Berlin – 3.\,8.\,1911 ebd.), \emph{Maler, Bildhauer}|pwk}, Alfred von Berger\pwindex{Berger, Alfred von 30.\,4.\,1853 Wien – 24.\,8.\,1912 ebd.@\textsc{Berger, Alfred von} (30.\,4.\,1853 Wien – 24.\,8.\,1912 ebd.), \emph{Schriftsteller, Journalist, Theaterleiter}|pwk}, Otto Julius Bierbaum\pwindex{Bierbaum, Otto Julius 28.\,6.\,1865 Zielona Góra – 1.\,2.\,1910 Dresden@\textsc{Bierbaum, Otto Julius} (28.\,6.\,1865 Zielona Góra – 1.\,2.\,1910 Dresden)|pwk}, Max Eugen Burckhard\pwindex{Burckhard, Max Eugen 14.\,7.\,1854 Korneuburg – 16.\,3.\,1912 Wien@\textsc{Burckhard, Max Eugen} (14.\,7.\,1854 Korneuburg – 16.\,3.\,1912 Wien), \emph{Schriftsteller, Rechtswissenschaftler, Theaterleiter}|pwk}, Alphonse
                     Daudet\pwindex{Daudet, Alphonse 13.\,5.\,1840 Nîmes – 16.\,11.\,1897 Paris@\textsc{Daudet, Alphonse} (13.\,5.\,1840 Nîmes – 16.\,11.\,1897 Paris), \emph{Schriftsteller}|pwk}, Georg Davidsohn\pwindex{Davidsohn, Georg 20.\,8.\,1872 Gniezno – 15.\,7.\,1942 Berlin@\textsc{Davidsohn, Georg} (20.\,8.\,1872 Gniezno – 15.\,7.\,1942 Berlin), \emph{Politiker, Journalist}|pwk}, Max Halbe\pwindex{Halbe, Max 4.\,10.\,1865 Gmina Suchy Dąb – 30.\,11.\,1944 Neuötting@\textsc{Halbe, Max} (4.\,10.\,1865 Gmina Suchy Dąb – 30.\,11.\,1944 Neuötting), \emph{Schriftsteller}|pwk}, Wilhelm Kienzl\pwindex{Kienzl, Wilhelm 17.\,1.\,1857 Waizenkirchen – 3.\,10.\,1941 Wien@\textsc{Kienzl, Wilhelm} (17.\,1.\,1857 Waizenkirchen – 3.\,10.\,1941 Wien), \emph{Schriftsteller, Komponist, Musikkritiker}|pwk}, Wilhelm von
                  Knigge\pwindex{Knigge, Wilhelm von 10.\,6.\,1863 – 2.\,7.\,1932 Piła@\textsc{Knigge, Wilhelm von} (10.\,6.\,1863 – 2.\,7.\,1932 Piła), \emph{Politiker}|pwk}, Maurice Kufferath\pwindex{Kufferath, Maurice 8.\,1.\,1852 Saint-Josse-ten-Noode – 8.\,12.\,1919@\textsc{Kufferath, Maurice} (8.\,1.\,1852 Saint-Josse-ten-Noode – 8.\,12.\,1919), \emph{Dirigent, Theaterdirektor, Violoncellist}|pwk}, Charles Lamoureux\pwindex{Lamoureux, Charles 28.\,9.\,1834 Bordeaux – 21.\,12.\,1899@\textsc{Lamoureux, Charles} (28.\,9.\,1834 Bordeaux – 21.\,12.\,1899), \emph{Dirigent}|pwk}, Eduard Lassen\pwindex{Lassen, Eduard 13.\,4.\,1830 Kopenhagen – 15.\,1.\,1904@\textsc{Lassen, Eduard} (13.\,4.\,1830 Kopenhagen – 15.\,1.\,1904), \emph{Komponist, Dirigent, Kapellmeister}|pwk}, Ruggero
                     Leoncavallo\pwindex{Leoncavallo, Ruggero 23.\,4.\,1857 Neapel – 9.\,8.\,1919 Montecatini Terme@\textsc{Leoncavallo, Ruggero} (23.\,4.\,1857 Neapel – 9.\,8.\,1919 Montecatini Terme), \emph{Komponist, Dirigent, Musiker}|pwk}, Arthur Levysohn\pwindex{Levysohn, Arthur 23.\,3.\,1841 Zielona Góra – 11.\,4.\,1908 Meran@\textsc{Levysohn, Arthur} (23.\,3.\,1841 Zielona Góra – 11.\,4.\,1908 Meran), \emph{Chefredakteur}|pwk}, Josef Lewinsky\pwindex{Lewinsky, Josef 20.\,9.\,1835 Wien – 27.\,2.\,1907 ebd.@\textsc{Lewinsky, Josef} (20.\,9.\,1835 Wien – 27.\,2.\,1907 ebd.), \emph{Schauspieler}|pwk}, Detlev von Liliencron\pwindex{Liliencron, Detlev von 3.\,6.\,1844 Kiel – 22.\,7.\,1909 Rahlstedt@\textsc{Liliencron, Detlev von} (3.\,6.\,1844 Kiel – 22.\,7.\,1909 Rahlstedt), \emph{Schriftsteller, Dichter, Dramatiker}|pwk}, Paul Lindau\pwindex{Lindau, Paul 3.\,6.\,1839 Magdeburg – 31.\,1.\,1919 Berlin@\textsc{Lindau, Paul} (3.\,6.\,1839 Magdeburg – 31.\,1.\,1919 Berlin), \emph{Schriftsteller, Kritiker, Theaterleiter}|pwk}, Rudolf Lothar\pwindex{Lothar, Rudolf 23.\,2.\,1865 Budapest – 2.\,10.\,1943 ebd.@\textsc{Lothar, Rudolf} (23.\,2.\,1865 Budapest – 2.\,10.\,1943 ebd.), \emph{Schriftsteller, Journalist, Theaterdirektor}|pwk}, Maurice Maeterlinck\pwindex{Maeterlinck, Maurice 29.\,8.\,1862 Gent – 6.\,5.\,1949 Nizza@\textsc{Maeterlinck, Maurice} (29.\,8.\,1862 Gent – 6.\,5.\,1949 Nizza), \emph{Schriftsteller}|pwk}, Jules Massenet\pwindex{Massenet, Jules 12.\,5.\,1842 Saint-Etienne – 13.\,8.\,1912 Paris@\textsc{Massenet, Jules} (12.\,5.\,1842 Saint-Etienne – 13.\,8.\,1912 Paris), \emph{Komponist}|pwk}, Catulle
                     Mendès\pwindex{Mendès, Catulle 20.\,5.\,1841 Bordeaux – 8.\,2.\,1909 Saint-Germain-en-Laye@\textsc{Mendès, Catulle} (20.\,5.\,1841 Bordeaux – 8.\,2.\,1909 Saint-Germain-en-Laye), \emph{Schriftsteller}|pwk}, Moritz Moszkowski\pwindex{Moszkowski, Moritz 23.\,8.\,1854 Breslau – 8.\,3.\,1925 Paris@\textsc{Moszkowski, Moritz} (23.\,8.\,1854 Breslau – 8.\,3.\,1925 Paris), \emph{Komponist, Dirigent, Pianist}|pwk}, Felix Mottl\pwindex{Mottl, Felix 24.\,8.\,1856 Wien – 2.\,7.\,1911 München@\textsc{Mottl, Felix} (24.\,8.\,1856 Wien – 2.\,7.\,1911 München), \emph{Dirigent}|pwk}, Vittorio Pica\pwindex{Pica, Vittorio 21.\,4.\,1862 Neapel – 1.\,5.\,1930 Mailand@\textsc{Pica, Vittorio} (21.\,4.\,1862 Neapel – 1.\,5.\,1930 Mailand), \emph{Schriftsteller, Kunstkritiker, Kunsthistoriker}|pwk}, Emanuel
                     Reicher\pwindex{Reicher, Emanuel 18.\,6.\,1849 Bochnia – 15.\,5.\,1924 Berlin@\textsc{Reicher, Emanuel} (18.\,6.\,1849 Bochnia – 15.\,5.\,1924 Berlin), \emph{Schauspieler}|pwk}, Marcel Schwob\pwindex{Schwob, Marcel 23.\,8.\,1867 Chaville – 27.\,2.\,1905@\textsc{Schwob, Marcel} (23.\,8.\,1867 Chaville – 27.\,2.\,1905), \emph{Schriftsteller, Journalist, Übersetzer}|pwk}, Johann Strauss\pwindex{Strauss, Johann 25.\,10.\,1825 Wien – 3.\,6.\,1899 ebd.@\textsc{Strauss, Johann} (25.\,10.\,1825 Wien – 3.\,6.\,1899 ebd.), \emph{Komponist, Dirigent}|pwk}, Hermann Sudermann\pwindex{Sudermann, Hermann 30.\,9.\,1857 Macikai – 21.\,11.\,1928 Berlin@\textsc{Sudermann, Hermann} (30.\,9.\,1857 Macikai – 21.\,11.\,1928 Berlin), \emph{Schriftsteller}|pwk}, Viktor Oskar Tilgner\pwindex{Tilgner, Viktor Oskar 25.\,10.\,1844 Bratislava – 16.\,4.\,1896 Wien@\textsc{Tilgner, Viktor Oskar} (25.\,10.\,1844 Bratislava – 16.\,4.\,1896 Wien), \emph{Bildhauer}|pwk}, Ernest Van
                        Dyck\pwindex{Van Dyck, Ernest 2.\,4.\,1861 Antwerpen – 31.\,8.\,1923 Berlaar@\textsc{Van Dyck, Ernest} (2.\,4.\,1861 Antwerpen – 31.\,8.\,1923 Berlaar), \emph{Sänger, Tenor}|pwk}, Sidney Whitman\pwindex{Whitman, Sidney 1848 Blendon – 2.\,11.\,1925 Derby@\textsc{Whitman, Sidney} (1848 Blendon – 2.\,11.\,1925 Derby), \emph{Historiker}|pwk}, Hermann Wolff\pwindex{Wolff, Hermann 4.\,9.\,1845 Köln – 3.\,2.\,1902 Berlin@\textsc{Wolff, Hermann} (4.\,9.\,1845 Köln – 3.\,2.\,1902 Berlin), \emph{Konzertveranstalter, Konzertagent, Inhaber einer Konzertagentur}|pwk} und Émile Zola\pwindex{Zola, Émile 2.\,4.\,1840 Paris – 29.\,9.\,1902 ebd.@\textsc{Zola, Émile} (2.\,4.\,1840 Paris – 29.\,9.\,1902 ebd.), \emph{Schriftsteller, Journalist}|pwk}.}}}\label{K_L02737-8} zu erhalten, das er, wenn ich nicht
               irre, als das größte dieſes Jahrhunderts bezeichnet. Man hat ihn
                  ausgelacht\textcolor{gray}{.} Aber iſt das nicht ekelhaft?\pend
           
\pstart
           Grüß’ Dich Gott, mein lieber Freund, und{ }ſchreib’ mir bald.\pend
           
\pstart
           Dein treuer{\\[\baselineskip]}\spacefill\mbox{Paul Goldmann.}\pend
           \leftskip=0em{}\selectlanguage{ngerman}\endnumbering\briefempfaengerindex{Schnitzler, Arthur@\textsc{Schnitzler, Arthur}!zzzGoldmann, Paul@\emph{von Paul Goldmann}!1895-06-242@{24. 6. [1895]}|)be}\mylabel{L02737h}  \newcommand{\dateiname}{L02737}\newcommand{\titel}{Paul Goldmann an Arthur Schnitzler, 24. 6. [1895]}\newcommand{\editorInnen}{Martin Anton Müller und Laura Untner}%% latex-leseansicht-abspann.tex
%% Abspann für die Leseansicht.
%% Der Schalter \ifkorrekturansicht ist bereits durch den Vorspann gesetzt.

%% latex-abspann.tex
%% Gemeinsamer Abspann für Korrekturansicht und Leseansicht.
%% Setzt den Schalter \ifkorrekturansicht voraus (gesetzt in den
%% einbindenden Dateien latex-korrekturansicht-abspann.tex bzw.
%% latex-leseansicht-abspann.tex).
%% ---------------------------------------------------------------

\normalsize

% Das esempio-Environment wird nur in der Leseansicht benötigt
\ifkorrekturansicht\else
\newenvironment{esempio}[3]%
{
    \vspace{1.5ex}
    \rlap{\underline{#1}}
    \par
    \setlength{\parindent}{0cm}
    \nopagebreak
    \leftskip=#2cm
    \rightskip=#3cm
}
{
    \par
}
\fi

\doendnotes{C}
\bigskip
\vfill

\clearpage

\footnotesize

\ifkorrekturansicht
  \lohead{\textsc{register}}
\fi

% theindex-Environment neu definieren ohne reledmac
\makeatletter
\renewenvironment{theindex}{%
  \ifkorrekturansicht
    \section*{\indexname}%
  \else
    \subsubsection*{Index der erwähnten Entitäten}%
  \fi
  \setlength{\parindent}{0pt}%
  \setlength{\parskip}{0pt plus 0.3pt}%
  \let\item\@idxitem
}{%
  \ifkorrekturansicht\clearpage\fi
}
\makeatother

\IfFileExists{\jobname-pw.ind}{\input{\jobname-pw.ind}}{}

% Quellenangabe nur in der Leseansicht
\ifkorrekturansicht\else
% Fallback-Definitionen, falls die .tex-Datei \titel etc. nicht gesetzt hat
\providecommand{\titel}{}
\providecommand{\editorInnen}{}
\providecommand{\dateiname}{\jobname}

\vspace{3cm}

\vfill

\footnotesize
\textsc{Quelle}: \titel. Herausgegeben von {\editorInnen}. In: \emph{Arthur Schnitzler: Briefwechsel mit Autorinnen und Autoren}.
 Digitale Edition, https://schnitzler-briefe.acdh.oeaw.ac.at/{\dateiname}.html (Stand \today)
\fi

\end{document}


