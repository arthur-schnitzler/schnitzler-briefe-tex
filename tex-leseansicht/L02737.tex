%% latex-leseansicht-vorspann.tex
%% Vorspann für die Leseansicht.
%% Lädt die gemeinsame Datei latex-vorspann.tex mit nicht gesetztem Schalter.

\newif\ifkorrekturansicht
\korrekturansichtfalse

\input{../tex-inputs/latex-vorspann}


         
         \renewcommand{\erwaehntePersonen}{Personen: Paul Adam, Hermann Bahr, Julius Bauer, Richard Beer-Hofmann, Reinhold Begas, Alfred von Berger, Otto Julius Bierbaum, Max Eugen Burckhard, Suzanne Clauser, Alphonse Daudet, Georg Davidsohn, Alfred Dreyfus, Isidor Fuchs, Paul Goldmann, Adalbert von Goldschmidt, Max Halbe, Theodor Herzl, Wilhelm Kienzl, Wilhelm von Knigge, Maurice Kufferath, Charles Lamoureux, Eduard Lassen, Ruggero Leoncavallo, Arthur Levysohn, Josef Lewinsky, Detlev von Liliencron, Paul Lindau, Rudolf Lothar, Maurice Maeterlinck, Jules Massenet, Catulle Mendès, Moritz Moszkowski, Felix Mottl, Max Nordau, Vittorio Pica, Emanuel Reicher, Paul Roche, Henri de Rochefort, Marcel Schwob, Leopold Sonnemann, Johann Strauss, Hermann Sudermann, Jean Thorel, Viktor Oskar Tilgner, Ernest Van Dyck, Sidney Whitman, Oscar Wilde, Hermann Wolff, Émile Zola}
         \renewcommand{\erwaehnteInstitutionen}{Institutionen: Burgtheater, Charpentier, Concordia, Die Zeit. Wiener Wochenschrift, Frankfurter Zeitung, Französische Abgeordnetenkammer, Suhrkamp Verlag}
         \renewcommand{\erwaehnteOrte}{Orte: Bad Tölz, Berlin, Deutschland, England, Frankreich, London, München, Paris, Wien, rue Feydeau}
         \renewcommand{\erwaehnteWerke}{Werke: Adalbert von Goldschmidt, Deutsche Zeitung, Entartung (2 Bde.), Gaea. Musikdrama, Ghea. Poeme dramatique, La Revue blanche, Le Gaulois, Liebelei (amourette): pièce en trois actes, Liebelei. Schauspiel in drei Akten, Mourir. Roman, Neue Deutsche Rundschau, Oscar Wilde judgé par le docteur Max Nordau, Sterben. Novelle, Tagebuch, »Gäa«, »L’Assaut malicieux«}
               \section[Paul Goldmann an Arthur Schnitzler, 24. 6. {[}1895{]}]{ Paul Goldmann an Arthur Schnitzler, 24. 6. {[}1895{]}}\nopagebreak\mylabel{v}\rehead{ }\begin{ledgroupsized}[t]{13cm}\normalsize\beginnumbering\briefempfaengerindex{Schnitzler, Arthur@\textsc{Schnitzler, Arthur}!zzzGoldmann, Paul@\emph{von Paul Goldmann}!1895-06-242@{24. 6. {[}1895{]}}|(be} \toendnotes[C]{\smallbreak\pagebreak[2]} \Standort{DLA, A:Schnitzler, HS.NZ85.1.3165.}
\physDesc{Brief, 3 Blätter, 12 Seiten, 3187 Zeichen
\newline{}Handschrift: schwarze Tinte, deutsche Kurrent
\newline{}Schnitzler: 1) mit Bleistift das Jahr »95« vermerkt  2) mit rotem Buntstift fünf Unterstreichungen}\toendnotes[C]{\smallbreak}\pstart
           \noindent{}{\pb}\textcolor{gray}{\textbf{\textbf{Frankfurter Zeitung\orgindex{Frankfurter Zeitung@Frankfurter Zeitung|pw}}}}\pend
           \pstart
           \textcolor{gray}{\textbf{(\begin{otherlanguage}{french}Gazette de Francfort\end{otherlanguage}\orgindex{Frankfurter Zeitung@Frankfurter Zeitung|pw}). }}\pend
           \pstart
           \textcolor{gray}{\textbf{\textbf{\begin{otherlanguage}{french}Fondateur M. L.
                                 Sonnemann\pwindex{Sonnemann, Leopold 1831-10-29 – 1909-10-30@\textsc{Sonnemann, Leopold} (1831-10-29 – 1909-10-30), \emph{Journalist, Herausgeber}|pw}\end{otherlanguage}.}}}\hfill \textsc{Paris\oindex{Paris@\textbf{Paris}|pw}}, 24. Juni.\pend
           \pstart
           \begin{otherlanguage}{french}\textcolor{gray}{\textbf{Journal politique, financier,}}\end{otherlanguage}\pend
           \pstart
           \begin{otherlanguage}{french}\textcolor{gray}{\textbf{commercial et littéraire.}}\end{otherlanguage}\pend
           \pstart
           \begin{otherlanguage}{french}\textcolor{gray}{\textbf{\textbf{Paraissant trois fois par jour.}}}\end{otherlanguage}\pend
           \pstart
           \begin{otherlanguage}{french}\textcolor{gray}{\textbf{\textbf{Bureau à Paris\oindex{Paris@\textbf{Paris}|pw}}}}\end{otherlanguage}\pend
           \pstart
           \begin{otherlanguage}{french}\textcolor{gray}{\textbf{\textbf{24. Rue Feydeau\oindex{rue Feydeau@\textbf{rue Feydeau}|pw}.}}}\end{otherlanguage}\pend
           \pstart\center{}Mein lieber Freund,\pend\pstart
           Eben bekomme ich Deinen lieben Brief. Nur raſch ein paar Zeilen. Mit Deinen Urtheilen
               über die geſandten \label{K_L02737-1v}\edtext{Druckſachen}{\lemma{\textnormal{\emph{Druckſachen}}}\Cendnote{\textnormal{Siehe Paul Goldmann an Arthur Schnitzler, 19. 5. [1895].
               }}}\label{K_L02737-1h} – es iſt wirklich zu viel Mühe, daß Du mir lange darüber ſchreibſt – bin ich
               Wort für Wort einverſtanden. Du mußt bedenken, daß {\pb}ich Dir kunterbunt durcheinander ſchicke, was mir intereſſant erſcheint – Einiges
               wegen ſtyliſtiſcher Schönheiten oder origineller Anſchauungen – Anderes wieder nur,
               weil es ein beachtenswerther Abſurditäts-Fall iſt (z. B. \label{K_L02737-2v}\edtext{\textsc{Rochefort\pwindex{Rochefort, Henri de 1830-01-31 – 1913-06-30@\textsc{Rochefort, Henri de} (1830-01-31 – 1913-06-30), \emph{Schriftsteller, Politiker, Journalist}|pw}}). \substVorne{}\textsuperscript{d}\substDazwischen{}D\substHinten{}er Fall \textsc{Wilde\pwindex{Wilde, Oscar 16.10.1854 – 30.11.1900@\textsc{Wilde, Oscar} (16.10.1854 – 30.11.1900), \emph{Schriftsteller}|pw}\pwindex{Adam, Paul 1862-12-06 – 1920-01-02@\textsc{Adam, Paul} (1862-12-06 – 1920-01-02), \emph{Schriftsteller, Kunstkritiker}!Assaut malicieux«1895-05-15@\strich\emph{»L’Assaut malicieux«} {[}1895-05-15{]}|pwv}}}{\lemma{\textnormal{\emph{Rochefort). … Wilde}}}\Cendnote{\textnormal{Der Polemiker Victor Henri de Rochefort\pwindex{Rochefort, Henri de 1830-01-31 – 1913-06-30@\textsc{Rochefort, Henri de} (1830-01-31 – 1913-06-30), \emph{Schriftsteller, Politiker, Journalist}|pwk}, der nach seiner politischen
                  Verfolgung sechs Jahre im London\oindex{London@\textbf{London}|pwk}er Exil lebte,
                  wurde im Februar 1895 amnestiert und kehrte ruhmvoll
                  nach Paris\oindex{Paris@\textbf{Paris}|pwk} zurück, wo er sich unter anderem
                  zur Dreyfus\pwindex{Dreyfus, Alfred 1859-10-09 – 1935-07-12@\textsc{Dreyfus, Alfred} (1859-10-09 – 1935-07-12), \emph{Militär}|pwk}-Affäre zu Wort meldete. Es ist
                  unklar, auf welchen Text Goldmann\pwindex{Goldmann, Paul 31.01.1865 – 25.09.1935@\textsc{Goldmann, Paul} (31.01.1865 – 25.09.1935), \emph{Schriftsteller, Journalist}|pwk} hier
                  Bezug nahm. Oscar Wilde\pwindex{Wilde, Oscar 16.10.1854 – 30.11.1900@\textsc{Wilde, Oscar} (16.10.1854 – 30.11.1900), \emph{Schriftsteller}|pwk} wurde wegen
                  »Unzucht« am 25. 5. 1895 zu zwei Jahren Zuchthaus
                  mit schwerer Zwangsarbeit verurteilt. Vgl. den gesandten Text\pwindex{Adam, Paul 1862-12-06 – 1920-01-02@\textsc{Adam, Paul} (1862-12-06 – 1920-01-02), \emph{Schriftsteller, Kunstkritiker}!Assaut malicieux«1895-05-15@\strich\emph{»L’Assaut malicieux«} {[}1895-05-15{]}|pwkv} von Paul Adam\pwindex{Adam, Paul 1862-12-06 – 1920-01-02@\textsc{Adam, Paul} (1862-12-06 – 1920-01-02), \emph{Schriftsteller, Kunstkritiker}|pwk}: \emph{»L’Assaut malicieux«}\pwindex{Adam, Paul 1862-12-06 – 1920-01-02@\textsc{Adam, Paul} (1862-12-06 – 1920-01-02), \emph{Schriftsteller, Kunstkritiker}!Assaut malicieux«1895-05-15@\strich\emph{»L’Assaut malicieux«} {[}1895-05-15{]}|pwk}. In: \emph{La Revue blanche}\pwindex{?? Werk@Nicht ermittelte Verfasserinnen und Verfasser!Revue blanche1889 – 1903@\emph{La Revue blanche} {[}1889 – 1903{]}|pwk}, Jg. 8, Nr. 47, 15. 5. 1895, S. 458–462. }}}\label{K_L02737-2h} empört mich ſchon lange. Das
                  engl\oindex{England@\textbf{England}|pwv}iſche Zuchthaus begreife
               ich {\pb}übrigens zur Noth, das ſind dumme heuchleriſche
                  \begin{otherlanguage}{french}\textsc{Bourgeois}\end{otherlanguage}, in England\oindex{England@\textbf{England}|pw} – damit hat man ſich
               abgefunden. Aber da gibt es dieſen Kerl\pwindex{Nordau, Max 29.07.1849 – 22.01.1923@\textsc{Nordau, Max} (29.07.1849 – 22.01.1923), \emph{Schriftsteller, Mediziner}|pwv}, den \textsc{Dr. Nordau\pwindex{Nordau, Max 29.07.1849 – 22.01.1923@\textsc{Nordau, Max} (29.07.1849 – 22.01.1923), \emph{Schriftsteller, Mediziner}|pw}}, der nach dem Urtheil an franzöſiſche\oindex{Frankreich@\textbf{Frankreich}|pwv} und deutschs\oindex{Deutschland@\textbf{Deutschland}|pwv} Blätter Briefe richtet, um zu ſagen: man möge nur in ſeinem Briefe
               nachleſen, wie er das \label{K_L02737-3v}\edtext{Schickſal \textsc{Wildes\pwindex{Wilde, Oscar 16.10.1854 – 30.11.1900@\textsc{Wilde, Oscar} (16.10.1854 – 30.11.1900), \emph{Schriftsteller}|pw}} voraus {\pb}berechnet}{\lemma{\textnormal{\emph{Schickſal … berechnet}}}\Cendnote{\textnormal{Max Nordau\pwindex{Nordau, Max 29.07.1849 – 22.01.1923@\textsc{Nordau, Max} (29.07.1849 – 22.01.1923), \emph{Schriftsteller, Mediziner}|pwk} hatte sich bereits in
                  seinem zweibändigen Buch \emph{Entartung}\pwindex{Nordau, Max 29.07.1849 – 22.01.1923@\textsc{Nordau, Max} (29.07.1849 – 22.01.1923), \emph{Schriftsteller, Mediziner}!Entartung (2 Bde.)1892 – 1893@\strich\emph{Entartung (2 Bde.)} {[}1892 – 1893{]}|pwk} (1892–1893) mit Oscar Wilde\pwindex{Wilde, Oscar 16.10.1854 – 30.11.1900@\textsc{Wilde, Oscar} (16.10.1854 – 30.11.1900), \emph{Schriftsteller}|pwk} beschäftigt, dessen vermeintliche Degeneration er
                  analysierte. Dass er damit den »Fall Wilde\pwindex{Wilde, Oscar 16.10.1854 – 30.11.1900@\textsc{Wilde, Oscar} (16.10.1854 – 30.11.1900), \emph{Schriftsteller}|pwk}«
                  hervorgesagt habe, betonte er beispielsweise in einem Interview\pwindex{Nordau, Max 29.07.1849 – 22.01.1923@\textsc{Nordau, Max} (29.07.1849 – 22.01.1923), \emph{Schriftsteller, Mediziner}!Oscar Wilde judge par le docteur Max Nordau1895-04-10@\strich\emph{Oscar Wilde judgé par le docteur Max Nordau} {[}1895-04-10{]}|pwkv}: Paul Roche\pwindex{Roche, Paul †~nach 1901@\textsc{Roche, Paul} (†~nach 1901), \emph{Journalist}|pwk}: \emph{Oscar Wilde\pwindex{Wilde, Oscar 16.10.1854 – 30.11.1900@\textsc{Wilde, Oscar} (16.10.1854 – 30.11.1900), \emph{Schriftsteller}|pwk} judgé par le docteur Max Nordau\pwindex{Nordau, Max 29.07.1849 – 22.01.1923@\textsc{Nordau, Max} (29.07.1849 – 22.01.1923), \emph{Schriftsteller, Mediziner}|pwk}}\pwindex{Nordau, Max 29.07.1849 – 22.01.1923@\textsc{Nordau, Max} (29.07.1849 – 22.01.1923), \emph{Schriftsteller, Mediziner}!Oscar Wilde judge par le docteur Max Nordau1895-04-10@\strich\emph{Oscar Wilde judgé par le docteur Max Nordau} {[}1895-04-10{]}|pwk}. In: \emph{Le Gaulois}\pwindex{?? Werk@Nicht ermittelte Verfasserinnen und Verfasser!Le Gaulois1868 – 1929@\emph{Le Gaulois} {[}1868 – 1929{]}|pwk}, Jg. 29, Nr. 5443,
                        10. 4. 1895, S. 1–2.}}}\label{K_L02737-3h} – um alſo
               aus dem Schickſal dieſes Bemitleidenswerthen\pwindex{Wilde, Oscar 16.10.1854 – 30.11.1900@\textsc{Wilde, Oscar} (16.10.1854 – 30.11.1900), \emph{Schriftsteller}|pwv} ſich eine Reklame für ſeinen Dekadenz-Schwindel zu
               machen. \uline{Das} macht mir das Blut kochen – da möchte ich
               losprügeln können mit Fäuſten und Stiefel-Abſätzen.\pend
           \pstart
           Über einen fran\oindex{Frankreich@\textbf{Frankreich}|pwv}zöſiſchen
                  \label{K_L02737-4v}\edtext{Verleger}{\lemma{\textnormal{\emph{Verleger}}}\Cendnote{\textnormal{\emph{Liebelei}\pwindex{Schnitzler, Arthur 15.05.1862 – 21.10.1931@\textsc{Schnitzler, Arthur} (15.05.1862 – 21.10.1931), \emph{Schriftsteller, Mediziner}!Liebelei. Schauspiel in drei Akten1895-10-09@\strich\emph{Liebelei. Schauspiel in drei Akten} {[}1895-10-09{]}|pwk} wurde 1896 und 1897 von Jean Thorel\pwindex{Thorel, Jean 1859-09-11 – 1916-08-20@\textsc{Thorel, Jean} (1859-09-11 – 1916-08-20), \emph{Übersetzer, Dramatiker}|pwk} ins Französische übersetzt, jedoch erst in der
                  Übersetzung von Suzanne Clauser\pwindex{Clauser, Suzanne 1898-05-16 – 1981-09-11@\textsc{Clauser, Suzanne} (1898-05-16 – 1981-09-11), \emph{Schriftstellerin, Übersetzerin}|pwk} im Jahr
                     1933 unter dem Titel \emph{Liebelei (\begin{otherlanguage}{french}amourette\end{otherlanguage})}\pwindex{Clauser, Suzanne 1898-05-16 – 1981-09-11@\textsc{Clauser, Suzanne} (1898-05-16 – 1981-09-11), \emph{Schriftstellerin, Übersetzerin}!Liebelei (amourette): piece en trois actes1933@\strich\emph{Liebelei (amourette): pièce en trois actes} {[}Übersetzung, 1933{]}|pwk} gedruckt.}}}\label{K_L02737-4h}
               aus einer Aufführung {\pb}in \textsc{Paris\oindex{Paris@\textbf{Paris}|pw}} denke ich ſeit Empfang Deines letzten lieben Briefes nach. Das wird aber ſchwer
               ſein. Die Pariſ\oindex{Paris@\textbf{Paris}|pw}er Verleger ſind noch ſchlimmeres
               Geſindel als die deutſch\oindex{Deutschland@\textbf{Deutschland}|pwv}en. Die
                  deutsch\oindex{Deutschland@\textbf{Deutschland}|pwv}en zahlen nur nichts,
               die fran\oindex{Frankreich@\textbf{Frankreich}|pwv}zöſiſchen verlangen,
               daß man ihnen {\pb}zahlt. Wärſt Du dazu bereit? Eine Aufführung\pwindex{Schnitzler, Arthur 15.05.1862 – 21.10.1931@\textsc{Schnitzler, Arthur} (15.05.1862 – 21.10.1931), \emph{Schriftsteller, Mediziner}!Liebelei. Schauspiel in drei Akten1895-10-09@\strich\emph{Liebelei. Schauspiel in drei Akten} {[}1895-10-09{]}|pwv} wäre eher möglich –
               aber erſt \uline{nach} einer Aufführung\pwindex{Schnitzler, Arthur 15.05.1862 – 21.10.1931@\textsc{Schnitzler, Arthur} (15.05.1862 – 21.10.1931), \emph{Schriftsteller, Mediziner}!Liebelei. Schauspiel in drei Akten1895-10-09@\strich\emph{Liebelei. Schauspiel in drei Akten} {[}1895-10-09{]}|pwv} in Berlin\oindex{Berlin@\textbf{Berlin}|pw} oder Wien\oindex{Wien@\textbf{Wien}|pw}, nicht
               zugleich. Wir reden noch darüber. Ich hab’ die Sache ſchon lange im Auge und hab’
               auch ſchon einige Schritte gethan.\pend
           \pstart
           {\pb}Das iſt aber immer noch nicht der große Brief – nur
               ein paar raſche Worte, ehe die \strikeout{Ka}{ }Kammer\orgindex{Franzoesische Abgeordnetenkammer@Französische Abgeordnetenkammer|pwv} beginnt. Darum ſchreibe
               ich nicht über allerlei Perſönliches, das ich längſt berühren möchte.\pend
           \pstart
           Es wäre mir eine große Freude, könnt’ ich Dich im Sommer ſehen; aber ich möchte keine
                  {\pb}Störung bringen in Deine Reiſe-Pläne. \strikeout{\textcolor{gray}{×}} Ich muß nach \textsc{Toelz\oindex{Bad Toelz@\textbf{Bad Tölz}|pw}} gehen u. muß dort vier Wochen bleiben. Das iſt nicht weit von \textsc{Muenchen\oindex{Muenchen@\textbf{München}|pw}}. Wie machen wirs alſo?\pend
           \pstart
           Reiſe glücklich, liebſter Freund! Ich weiß, wie gern Du hinausfährſt, und freue mich
               für Dich. Laß’ die \strikeout{Hypoch\textcolor{gray}{ond}}{ }{\pb}\label{K_L02737-5v}\edtext{Hypochondrien}{\lemma{\textnormal{\emph{Hypochondrien}}}\Cendnote{\textnormal{Schnitzler\pwindex{Schnitzler, Arthur 15.05.1862 – 21.10.1931@\textsc{Schnitzler, Arthur} (15.05.1862 – 21.10.1931), \emph{Schriftsteller, Mediziner}|pwk} notierte 1895 immer wieder hypochondrische Zustände im \emph{Tagebuch}\pwindex{Schnitzler, Arthur 15.05.1862 – 21.10.1931@\textsc{Schnitzler, Arthur} (15.05.1862 – 21.10.1931), \emph{Schriftsteller, Mediziner}!Tagebuch1981 – 2000@\strich\emph{Tagebuch} {[}1981 – 2000{]}|pwk}, zuletzt am 22. 6. 1895.}}}\label{K_L02737-5h} in Wien\oindex{Wien@\textbf{Wien}|pw}! Die Welt iſt ſchön, und Du biſt jung und ein glücklicher
               Menſch, – ja, glaub’ mir, ein glücklicher Menſch.\pend
           \pstart
           Ich höre wohl Deine Unterwegs-Adreſſe.\pend
           \pstart
           \textsc{Burckhardt\pwindex{Burckhard, Max Eugen 14.07.1854 – 16.03.1912@\textsc{Burckhard, Max Eugen} (14.07.1854 – 16.03.1912), \emph{Schriftsteller, Rechtswissenschaftler, Theaterleiter}|pw}} iſt \label{K_L02737-6v}\edtext{unglaublich}{\lemma{\textnormal{\emph{unglaublich}}}\Cendnote{\textnormal{Am 15. 6. 1895 schrieb Schnitzler\pwindex{Schnitzler, Arthur 15.05.1862 – 21.10.1931@\textsc{Schnitzler, Arthur} (15.05.1862 – 21.10.1931), \emph{Schriftsteller, Mediziner}|pwk} an Richard Beer-Hofmann\pwindex{Beer-Hofmann, Richard 1866-07-11 – 1945-09-26@\textsc{Beer-Hofmann, Richard} (1866-07-11 – 1945-09-26), \emph{Schriftsteller}|pwk}
                  von dem Gerücht, \emph{Liebelei}\pwindex{Schnitzler, Arthur 15.05.1862 – 21.10.1931@\textsc{Schnitzler, Arthur} (15.05.1862 – 21.10.1931), \emph{Schriftsteller, Mediziner}!Liebelei. Schauspiel in drei Akten1895-10-09@\strich\emph{Liebelei. Schauspiel in drei Akten} {[}1895-10-09{]}|pwk} würde am \emph{Burgtheater}\orgindex{Burgtheater@Burgtheater|pwk} nicht mehr aufgeführt werden. Schnitzler\pwindex{Schnitzler, Arthur 15.05.1862 – 21.10.1931@\textsc{Schnitzler, Arthur} (15.05.1862 – 21.10.1931), \emph{Schriftsteller, Mediziner}|pwk} konfrontierte Max Burckhard\pwindex{Burckhard, Max Eugen 14.07.1854 – 16.03.1912@\textsc{Burckhard, Max Eugen} (14.07.1854 – 16.03.1912), \emph{Schriftsteller, Rechtswissenschaftler, Theaterleiter}|pwk} damit, doch der machte deutlich, dass er es
                  unter allen Umständen aufführen werde. Vgl. A. S.: \emph{Tagebuch}, 16. 6. 1895.}}}\label{K_L02737-6h}. Es wäre {\pb}ſogar ſchon komiſch, wenns Dich nicht gerade träfe.
               Aber auch ich bin feſt überzeugt: das Stück\pwindex{Schnitzler, Arthur 15.05.1862 – 21.10.1931@\textsc{Schnitzler, Arthur} (15.05.1862 – 21.10.1931), \emph{Schriftsteller, Mediziner}!Liebelei. Schauspiel in drei Akten1895-10-09@\strich\emph{Liebelei. Schauspiel in drei Akten} {[}1895-10-09{]}|pwv}{ }\uline{wird} aufgeführt.\pend
           \pstart
           Dem \textsc{Fuchs\pwindex{Fuchs, Isidor 25.09.1849 – um den 20.8.1920@\textsc{Fuchs, Isidor} (25.09.1849 – um den 20.8.1920), \emph{Schriftsteller, Journalist}|pw}} thatſt \strikeout{\textcolor{gray}{o}h} Du Unrecht. Er iſt kein \textsc{Concordia\orgindex{Concordia@Concordia|pw}}-Literat mehr, ſondern ein lieber, neidloſer, treuer, einfacher Menſch\pwindex{Fuchs, Isidor 25.09.1849 – um den 20.8.1920@\textsc{Fuchs, Isidor} (25.09.1849 – um den 20.8.1920), \emph{Schriftsteller, Journalist}|pwv}, der alt wird und gut
               wird. Als Menſch {\pb}tauſendmal mehr werth, wie \textsc{Herzl\pwindex{Herzl, Theodor 1860-05-02 – 1904-07-03@\textsc{Herzl, Theodor} (1860-05-02 – 1904-07-03), \emph{Schriftsteller, Journalist}|pw}}.\pend
           \pstart
           \textsc{Herzl\pwindex{Herzl, Theodor 1860-05-02 – 1904-07-03@\textsc{Herzl, Theodor} (1860-05-02 – 1904-07-03), \emph{Schriftsteller, Journalist}|pw}} ſchreibt einen \label{K_L02737-7v}\edtext{Roman}{\lemma{\textnormal{\emph{Roman}}}\Cendnote{\textnormal{Im Sommer 1895,
                  kurz vor seiner Rückkehr nach Wien\oindex{Wien@\textbf{Wien}|pwk}, spielte Theodor Herzl\pwindex{Herzl, Theodor 1860-05-02 – 1904-07-03@\textsc{Herzl, Theodor} (1860-05-02 – 1904-07-03), \emph{Schriftsteller, Journalist}|pwk} mit der Idee, einen
                  politischen Roman zu schreiben. Vgl. Shlomo Avineri: Herzl\pwindex{Herzl, Theodor 1860-05-02 – 1904-07-03@\textsc{Herzl, Theodor} (1860-05-02 – 1904-07-03), \emph{Schriftsteller, Journalist}|pwk}. Theodor
                        Herzl\pwindex{Herzl, Theodor 1860-05-02 – 1904-07-03@\textsc{Herzl, Theodor} (1860-05-02 – 1904-07-03), \emph{Schriftsteller, Journalist}|pwk} und die Gründung des jüdischen Staates. Berlin\oindex{Berlin@\textbf{Berlin}|pwk}: eBook Jüdischer Verlag im \emph{Suhrkamp}\orgindex{Suhrkamp Verlag@Suhrkamp Verlag|pwk} Verlag 2016,
                     S. 181.}}}\label{K_L02737-7h}.\pend
           \pstart
           Was macht \textsc{\strikeout{Ric}\pwindex{Beer-Hofmann, Richard 1866-07-11 – 1945-09-26@\textsc{Beer-Hofmann, Richard} (1866-07-11 – 1945-09-26), \emph{Schriftsteller}|pwv}}{ }\textsc{Richard\pwindex{Beer-Hofmann, Richard 1866-07-11 – 1945-09-26@\textsc{Beer-Hofmann, Richard} (1866-07-11 – 1945-09-26), \emph{Schriftsteller}|pw}}? Schreibt er was? Und ſehe ich ihn?\pend
           \pstart
           Wie geht die »Zeit\orgindex{Zeit. Wiener Wochenschrift@Die Zeit. Wiener Wochenschrift|pw}«?\pend
           \pstart
           Die Überſetzung\pwindex{Schnitzler, Arthur 15.05.1862 – 21.10.1931@\textsc{Schnitzler, Arthur} (15.05.1862 – 21.10.1931), \emph{Schriftsteller, Mediziner}!Mourir. Roman1895-04-27 – 1895-06-01@\strich\emph{Mourir. Roman} {[}1895-04-27 – 1895-06-01{]}|pwv} von »Sterben\pwindex{Schnitzler, Arthur 15.05.1862 – 21.10.1931@\textsc{Schnitzler, Arthur} (15.05.1862 – 21.10.1931), \emph{Schriftsteller, Mediziner}!Sterben. Novelle1894-10-01 – 1894-12-01@\strich\emph{Sterben. Novelle} {[}1894-10-01 – 1894-12-01{]}|pw}« iſt nicht übel. Dank für die
               Zuſendung.\pend
           \pstart
           {\pb}\textsc{Bahr\pwindex{Bahr, Hermann 19.07.1863 – 15.01.1934@\textsc{Bahr, Hermann} (19.07.1863 – 15.01.1934), \emph{Schriftsteller, Kritiker}|pw}} hat hierher geſchrieben, um die Unterſchriften der fran\oindex{Frankreich@\textbf{Frankreich}|pwv}zöſiſchen Schriftſteller-Welt \strikeout{zur} zum Verlangen einer Aufführung eines \label{K_L02737-8v}\edtext{\textsc{Goldschmidtschen\pwindex{Goldschmidt, Adalbert von 1848-05-05 – 1906-12-21@\textsc{Goldschmidt, Adalbert von} (1848-05-05 – 1906-12-21), \emph{Schriftsteller, Komponist}|pw}}{ }Muſik-Dramas\pwindex{Goldschmidt, Adalbert von 1848-05-05 – 1906-12-21@\textsc{Goldschmidt, Adalbert von} (1848-05-05 – 1906-12-21), \emph{Schriftsteller, Komponist}!Gaea. Musikdrama1893@\strich\emph{Gaea. Musikdrama} {[}1893{]}|pwv}}{\lemma{\textnormal{\emph{Goldschmidtschen Muſik-Dramas}}}\Cendnote{\textnormal{Das monumentale Musikdrama \emph{Gäa}\pwindex{Goldschmidt, Adalbert von 1848-05-05 – 1906-12-21@\textsc{Goldschmidt, Adalbert von} (1848-05-05 – 1906-12-21), \emph{Schriftsteller, Komponist}!Gaea. Musikdrama1893@\strich\emph{Gaea. Musikdrama} {[}1893{]}|pwk} von Adalbert von
                     Goldschmidt\pwindex{Goldschmidt, Adalbert von 1848-05-05 – 1906-12-21@\textsc{Goldschmidt, Adalbert von} (1848-05-05 – 1906-12-21), \emph{Schriftsteller, Komponist}|pwk} wurde seit 1892 von Bahr\pwindex{Bahr, Hermann 19.07.1863 – 15.01.1934@\textsc{Bahr, Hermann} (19.07.1863 – 15.01.1934), \emph{Schriftsteller, Kritiker}|pwk} für die Aufführung propagiert (vgl. Hermann Bahr\pwindex{Bahr, Hermann 19.07.1863 – 15.01.1934@\textsc{Bahr, Hermann} (19.07.1863 – 15.01.1934), \emph{Schriftsteller, Kritiker}|pwk}: \emph{Adalbert von Goldschmidt}\pwindex{Bahr, Hermann 19.07.1863 – 15.01.1934@\textsc{Bahr, Hermann} (19.07.1863 – 15.01.1934), \emph{Schriftsteller, Kritiker}!Adalbert von Goldschmidt1892-11-04@\strich\emph{Adalbert von Goldschmidt} {[}1892-11-04{]}|pwk}. In: \emph{Deutsche Zeitung}\pwindex{?? Werk@Nicht ermittelte Verfasserinnen und Verfasser!Deutsche Zeitung1871 – 1907@\emph{Deutsche Zeitung} {[}1871 – 1907{]}|pwk}, Jg. 22, Nr. 7490,
                        4. 11. 1892, Morgen-Ausgabe, S. 6). Erster Anlass war
                  dazu das Erscheinen einer französischen Übersetzung durch Catulle Mendès\pwindex{Mendes, Catulle 20.05.1841 – 08.02.1909@\textsc{Mendès, Catulle} (20.05.1841 – 08.02.1909), \emph{Schriftsteller}|pwk} (\emph{Ghea. Poeme dramatique}\pwindex{Goldschmidt, Adalbert von 1848-05-05 – 1906-12-21@\textsc{Goldschmidt, Adalbert von} (1848-05-05 – 1906-12-21), \emph{Schriftsteller, Komponist}!Ghea. Poeme dramatique1893@\strich\emph{Ghea. Poeme dramatique} {[}1893{]}|pwk}. Mis en Français
                     par Catulle Mendès\pwindex{Mendes, Catulle 20.05.1841 – 08.02.1909@\textsc{Mendès, Catulle} (20.05.1841 – 08.02.1909), \emph{Schriftsteller}|pwk}.
                     Paris: \emph{G. Charpentier et E.
                        Fasquelle}\orgindex{Charpentier@Charpentier|pwk}{ }1893.) Eine vollständige Inszenierung würde drei Tage dauern. Auf Initiative
                  von Bahr\pwindex{Bahr, Hermann 19.07.1863 – 15.01.1934@\textsc{Bahr, Hermann} (19.07.1863 – 15.01.1934), \emph{Schriftsteller, Kritiker}|pwk} entstanden Komitees in Wien\oindex{Wien@\textbf{Wien}|pwk}, Berlin\oindex{Berlin@\textbf{Berlin}|pwk}
                  und Paris\oindex{Paris@\textbf{Paris}|pwk}, die die Aufführung bewerkstelligen
                  sollten. Goldmann\pwindex{Goldmann, Paul 31.01.1865 – 25.09.1935@\textsc{Goldmann, Paul} (31.01.1865 – 25.09.1935), \emph{Schriftsteller, Journalist}|pwk} irrte sich jedoch in der
                  Bereitwilligkeit fran\oindex{Frankreich@\textbf{Frankreich}|pwkv}zösischer Kulturgrößen, ihren Namen dafür herzugeben. Im März
                     1896 erschien eine Petition\pwindex{Gaea«1896-03-01@\emph{»Gäa«} {[}1896-03-01{]}|pwkv}, die die Aufführung forderte (\emph{»Gäa«}\pwindex{Gaea«1896-03-01@\emph{»Gäa«} {[}1896-03-01{]}|pwk}. In: \emph{Neue Deutsche Rundschau}\pwindex{Neue Deutsche Rundschau1894-01-01 – 1903-12-31@\emph{Neue Deutsche Rundschau} {[}1894-01-01 – 1903-12-31{]}|pwk}, Jg. 7, H. 3, März 1896,
                     S. 3039. Sie war unterzeichnet von Julius Bauer\pwindex{Bauer, Julius 15.10.1853 – 11.06.1941@\textsc{Bauer, Julius} (15.10.1853 – 11.06.1941), \emph{Schriftsteller, Journalist, Kritiker}|pwk}, Reinhold Begas\pwindex{Begas, Reinhold 1831-07-15 – 1911-08-03@\textsc{Begas, Reinhold} (1831-07-15 – 1911-08-03), \emph{Maler, Bildhauer}|pwk}, Alfred von Berger\pwindex{Berger, Alfred von 30.04.1853 – 24.08.1912@\textsc{Berger, Alfred von} (30.04.1853 – 24.08.1912), \emph{Schriftsteller, Journalist, Theaterleiter}|pwk}, Otto Julius Bierbaum\pwindex{Bierbaum, Otto Julius 28.06.1865 – 01.02.1910@\textsc{Bierbaum, Otto Julius} (28.06.1865 – 01.02.1910)|pwk}, Max Eugen Burckhard\pwindex{Burckhard, Max Eugen 14.07.1854 – 16.03.1912@\textsc{Burckhard, Max Eugen} (14.07.1854 – 16.03.1912), \emph{Schriftsteller, Rechtswissenschaftler, Theaterleiter}|pwk}, Alphonse
                     Daudet\pwindex{Daudet, Alphonse 13.05.1840 – 16.11.1897@\textsc{Daudet, Alphonse} (13.05.1840 – 16.11.1897), \emph{Schriftsteller}|pwk}, Georg Davidsohn\pwindex{Davidsohn, Georg 1872-08-20 – 1942-07-15@\textsc{Davidsohn, Georg} (1872-08-20 – 1942-07-15), \emph{Politiker, Journalist}|pwk}, Max Halbe\pwindex{Halbe, Max 04.10.1865 – 30.11.1944@\textsc{Halbe, Max} (04.10.1865 – 30.11.1944), \emph{Schriftsteller}|pwk}, Wilhelm Kienzl\pwindex{Kienzl, Wilhelm 17.01.1857 – 03.10.1941@\textsc{Kienzl, Wilhelm} (17.01.1857 – 03.10.1941), \emph{Schriftsteller, Komponist, Musikkritiker}|pwk}, Wilhelm von
                  Knigge\pwindex{Knigge, Wilhelm von 10.6.1863 – 2.7.1932@\textsc{Knigge, Wilhelm von} (10.6.1863 – 2.7.1932), \emph{Politiker}|pwk}, Maurice Kufferath\pwindex{Kufferath, Maurice 1852-01-08 – 1919-12-08@\textsc{Kufferath, Maurice} (1852-01-08 – 1919-12-08), \emph{Dirigent, Theaterdirektor, Violoncellist}|pwk}, Charles Lamoureux\pwindex{Lamoureux, Charles 1834-09-28 – 1899-12-21@\textsc{Lamoureux, Charles} (1834-09-28 – 1899-12-21), \emph{Dirigent}|pwk}, Eduard Lassen\pwindex{Lassen, Eduard 1830-04-13 – 1904-01-15@\textsc{Lassen, Eduard} (1830-04-13 – 1904-01-15), \emph{Komponist, Dirigent, Kapellmeister}|pwk}, Ruggero
                     Leoncavallo\pwindex{Leoncavallo, Ruggero 23.04.1857 – 09.08.1919@\textsc{Leoncavallo, Ruggero} (23.04.1857 – 09.08.1919), \emph{Komponist, Dirigent, Musiker}|pwk}, Arthur Levysohn\pwindex{Levysohn, Arthur 23.3.1841 – 11.4.1908@\textsc{Levysohn, Arthur} (23.3.1841 – 11.4.1908), \emph{Chefredakteur}|pwk}, Josef Lewinsky\pwindex{Lewinsky, Josef 20.09.1835 – 27.02.1907@\textsc{Lewinsky, Josef} (20.09.1835 – 27.02.1907), \emph{Schauspieler}|pwk}, Detlev von Liliencron\pwindex{Liliencron, Detlev von 03.06.1844 – 22.07.1909@\textsc{Liliencron, Detlev von} (03.06.1844 – 22.07.1909)|pwk}, Paul Lindau\pwindex{Lindau, Paul 03.06.1839 – 31.01.1919@\textsc{Lindau, Paul} (03.06.1839 – 31.01.1919), \emph{Schriftsteller, Kritiker, Theaterleiter}|pwk}, Rudolf Lothar\pwindex{Lothar, Rudolf 23.2.1865 – 2.10.1943@\textsc{Lothar, Rudolf} (23.2.1865 – 2.10.1943), \emph{Schriftsteller, Journalist, Theaterdirektor}|pwk}, Maurice Maeterlinck\pwindex{Maeterlinck, Maurice 29.08.1862 – 06.05.1949@\textsc{Maeterlinck, Maurice} (29.08.1862 – 06.05.1949), \emph{Schriftsteller}|pwk}, Jules Massenet\pwindex{Massenet, Jules 1842-05-12 – 1912-08-13@\textsc{Massenet, Jules} (1842-05-12 – 1912-08-13), \emph{Komponist}|pwk}, Catulle
                     Mendès\pwindex{Mendes, Catulle 20.05.1841 – 08.02.1909@\textsc{Mendès, Catulle} (20.05.1841 – 08.02.1909), \emph{Schriftsteller}|pwk}, Moritz Moszkowski\pwindex{Moszkowski, Moritz 1854-08-23 – 1925-03-08@\textsc{Moszkowski, Moritz} (1854-08-23 – 1925-03-08), \emph{Komponist, Dirigent, Pianist}|pwk}, Felix Mottl\pwindex{Mottl, Felix 24.08.1856 – 02.07.1911@\textsc{Mottl, Felix} (24.08.1856 – 02.07.1911), \emph{Dirigent}|pwk}, Vittorio Pica\pwindex{Pica, Vittorio 1862-04-21 – 1930-05-01@\textsc{Pica, Vittorio} (1862-04-21 – 1930-05-01), \emph{Schriftsteller, Kunstkritiker, Kunsthistoriker}|pwk}, Emanuel
                     Reicher\pwindex{Reicher, Emanuel 18.06.1849 – 15.05.1924@\textsc{Reicher, Emanuel} (18.06.1849 – 15.05.1924), \emph{Schauspieler}|pwk}, Marcel Schwob\pwindex{Schwob, Marcel 1867-08-23 – 1905-02-27@\textsc{Schwob, Marcel} (1867-08-23 – 1905-02-27), \emph{Schriftsteller, Journalist, Übersetzer}|pwk}, Johann Strauss\pwindex{Strauss, Johann 25.10.1825 – 03.06.1899@\textsc{Strauss, Johann} (25.10.1825 – 03.06.1899), \emph{Komponist, Dirigent}|pwk}, Hermann Sudermann\pwindex{Sudermann, Hermann 30.09.1857 – 21.11.1928@\textsc{Sudermann, Hermann} (30.09.1857 – 21.11.1928), \emph{Schriftsteller}|pwk}, Viktor Oskar Tilgner\pwindex{Tilgner, Viktor Oskar 1844-10-25 – 1896-04-16@\textsc{Tilgner, Viktor Oskar} (1844-10-25 – 1896-04-16), \emph{Bildhauer}|pwk}, Ernest Van
                        Dyck\pwindex{Van Dyck, Ernest 1861-04-02 – 1923-08-31@\textsc{Van Dyck, Ernest} (1861-04-02 – 1923-08-31), \emph{Sänger, Tenor}|pwk}, Sidney Whitman\pwindex{Whitman, Sidney 1845-01-01 – 1925@\textsc{Whitman, Sidney} (1845-01-01 – 1925), \emph{Historiker}|pwk}, Hermann Wolff\pwindex{Wolff, Hermann 1845-09-04 – 1902-02-03@\textsc{Wolff, Hermann} (1845-09-04 – 1902-02-03), \emph{Konzertveranstalter, Konzertagent, Inhaber einer Konzertagentur}|pwk} und Émile Zola\pwindex{Zola, Emile 02.04.1840 – 29.09.1902@\textsc{Zola, Émile} (02.04.1840 – 29.09.1902), \emph{Schriftsteller, Journalist}|pwk}.}}}\label{K_L02737-8h} zu erhalten, das er, wenn ich nicht
               irre, als das größte dieſes Jahrhunderts bezeichnet. Man hat ihn
                  ausgelacht\textcolor{gray}{.} Aber iſt das nicht ekelhaft?\pend
           \pstart
           Grüß’ Dich Gott, mein lieber Freund, und ſchreib’ mir bald.\pend
           \pstart
           Dein treuer{\\[\baselineskip]}\spacefill\mbox{Paul Goldmann.}\pend
           \leftskip=0em{}
         
         \endnumbering\mylabel{h}\end{ledgroupsized}  \newcommand{\dateiname}{L02737}\newcommand{\titel}{Paul Goldmann an Arthur Schnitzler, 24. 6. [1895]}\newcommand{\editorInnen}{Martin Anton Müller und Laura Untner}%% latex-leseansicht-abspann.tex
%% Abspann für die Leseansicht.
%% Der Schalter \ifkorrekturansicht ist bereits durch den Vorspann gesetzt.

%% latex-abspann.tex
%% Gemeinsamer Abspann für Korrekturansicht und Leseansicht.
%% Setzt den Schalter \ifkorrekturansicht voraus (gesetzt in den
%% einbindenden Dateien latex-korrekturansicht-abspann.tex bzw.
%% latex-leseansicht-abspann.tex).
%% ---------------------------------------------------------------

\normalsize

% Das esempio-Environment wird nur in der Leseansicht benötigt
\ifkorrekturansicht\else
\newenvironment{esempio}[3]%
{
    \vspace{1.5ex}
    \rlap{\underline{#1}}
    \par
    \setlength{\parindent}{0cm}
    \nopagebreak
    \leftskip=#2cm
    \rightskip=#3cm
}
{
    \par
}
\fi

\doendnotes{C}
\bigskip
\vfill

\clearpage

\footnotesize

\ifkorrekturansicht
  \lohead{\textsc{register}}
\fi

% theindex-Environment neu definieren ohne reledmac
\makeatletter
\renewenvironment{theindex}{%
  \ifkorrekturansicht
    \section*{\indexname}%
  \else
    \subsubsection*{Index der erwähnten Entitäten}%
  \fi
  \setlength{\parindent}{0pt}%
  \setlength{\parskip}{0pt plus 0.3pt}%
  \let\item\@idxitem
}{%
  \ifkorrekturansicht\clearpage\fi
}
\makeatother

\IfFileExists{\jobname-pw.ind}{\input{\jobname-pw.ind}}{}

% Quellenangabe nur in der Leseansicht
\ifkorrekturansicht\else
% Fallback-Definitionen, falls die .tex-Datei \titel etc. nicht gesetzt hat
\providecommand{\titel}{}
\providecommand{\editorInnen}{}
\providecommand{\dateiname}{\jobname}

\vspace{3cm}

\vfill

\footnotesize
\textsc{Quelle}: \titel. Herausgegeben von {\editorInnen}. In: \emph{Arthur Schnitzler: Briefwechsel mit Autorinnen und Autoren}.
 Digitale Edition, https://schnitzler-briefe.acdh.oeaw.ac.at/{\dateiname}.html (Stand \today)
\fi

\end{document}


      