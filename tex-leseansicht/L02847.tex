%% latex-leseansicht-vorspann.tex
%% Vorspann für die Leseansicht.
%% Lädt die gemeinsame Datei latex-vorspann.tex mit nicht gesetztem Schalter.

\newif\ifkorrekturansicht
\korrekturansichtfalse

\input{../tex-inputs/latex-vorspann}


         
         \renewcommand{\erwaehntePersonen}{Personen: Marie Reinhard}
         \renewcommand{\erwaehnteInstitutionen}{Institutionen: Frankfurter Zeitung}
         \renewcommand{\erwaehnteOrte}{Orte: Berlin, Deutsches Postamt in Shanghai, Deutschland, Frankfurt am Main, Frankreich, Genua, Guangzhou, Hong Kong, Incheon, Italien, Jangtsekiang, Japan, Kiautschou, Paris, Peking, Perlfluss, Seoul, Shanghai, Südkorea, Tianjin, Wien, Wuhan}
         \renewcommand{\erwaehnteWerke}{}
               \section[ Paul Goldmann an Arthur Schnitzler, 4. 4. 1898]{ Paul Goldmann an Arthur Schnitzler, 4. 4. 1898}\nopagebreak\mylabel{v}\rehead{ }\begin{ledgroupsized}[t]{13cm}\normalsize\beginnumbering \toendnotes[C]{\smallbreak\pagebreak[2]} \Standort{DLA, A:Schnitzler, HS.NZ85.1.3168.}
\physDesc{Brief, 1 Blatt, 4 Seiten, 1866 Zeichen
\newline{}Handschrift: schwarze Tinte, deutsche Kurrent}\toendnotes[C]{\smallbreak}\pstart
           \noindent{}{\pb}\textcolor{gray}{\textbf{Frankfurter Zeitung}}\orgindex{Frankfurter Zeitung@Frankfurter Zeitung|pw}\hfill \substVorne{}\textsuperscript{\textcolor{gray}{\textbf{Frankfurt a. M.\oindex{Frankfurt am Main@\textbf{Frankfurt am Main}|pw},}}}{\allowbreak}\substDazwischen{}\textsc{Genua\oindex{Genua@\textbf{Genua}|pw}}\substHinten{}{ }4. April \textcolor{gray}{\textbf{189}}8.\pend
           \pstart
           \textcolor{gray}{\textbf{und}}\pend
           \pstart
           \textcolor{gray}{\textbf{Handelsblatt.}}\pend
           \pstart
           \textcolor{gray}{\textbf{Redaktion\orgindex{Frankfurter Zeitung@Frankfurter Zeitung|pwv}.\footnote{\noindent{}\textcolor{gray}{\textbf{Für die Redaktion\orgindex{Frankfurter Zeitung@Frankfurter Zeitung|pwv} beſtimmte Briefe und Sendungen wolle man
                                 \so{nicht} an die Perſon eines Redakteurs,
                              ſondern ſtets \textbf{an die Redaktion der Frankfurter Zeitung\orgindex{Frankfurter Zeitung@Frankfurter Zeitung|pw}} adreſſiren. }}}}}\pend
           \pstart
           \textcolor{gray}{\textbf{Telegramm-Adreſſe:}}\pend
           \pstart
           \textcolor{gray}{\textbf{Zeitung\orgindex{Frankfurter Zeitung@Frankfurter Zeitung|pwv}{ }Frankfurt Main\oindex{Frankfurt am Main@\textbf{Frankfurt am Main}|pw}. }}\pend
           \pstart{}Mein lieber Freund,\pend\pstart
           Tauſend Dank für Deinen ſo lieben Brief! Es thut wohl, zum Abſchied ſo gute Worte zu
               hören.\pend
           \pstart
           Ich gehe morgen{ }früh aufs Schiff, fahre zuerſt nach \textsc{Hongkong\oindex{Hong Kong@\textbf{Hong Kong}|pw}} (5. Mai), von dort den \strikeout{Per}{ }Perlfluß\oindex{Perlfluss@\textbf{Perlfluss}|pw} hinauf nach \textsc{Canton\oindex{Guangzhou@\textbf{Guangzhou}|pw}}, zurück nach \textsc{Hongkong\oindex{Hong Kong@\textbf{Hong Kong}|pw}}, zur See nach \textsc{Shanghai\oindex{Shanghai@\textbf{Shanghai}|pw}}, von da den \textsc{Yang-tse-kiang\oindex{Jangtsekiang@\textbf{Jangtsekiang}|pw}} hinauf, vielleicht bis \textsc{Hankau\oindex{Wuhan@\textbf{Wuhan}|pw}}, zurück nach \textsc{Shanghai\oindex{Shanghai@\textbf{Shanghai}|pw}}, von da nach \textsc{Kiao-tschau\oindex{Kiautschou@\textbf{Kiautschou}|pw}}, von da nach \textsc{Tientsin\oindex{Tianjin@\textbf{Tianjin}|pw}}, von da nach \textsc{\strikeout{P\textcolor{gray}{or}}}{ }\textsc{Peking\oindex{Peking@\textbf{Peking}|pw}}, zurück nach \strikeout{\textsc{Peking\oindex{Peking@\textbf{Peking}|pw}}}{ }{\pb}\textsc{Tientsin\oindex{Tianjin@\textbf{Tianjin}|pw}}, von da zur See nach \textsc{Chemulpo\oindex{Incheon@\textbf{Incheon}|pw}} (\textsc{Korea\oindex{Suedkorea@\textbf{Südkorea}|pwv}}) und landeinwärts bis \textsc{Söul\oindex{Seoul@\textbf{Seoul}|pw}}, von da nach \textsc{Japan\oindex{Japan@\textbf{Japan}|pw}}.\pend
           \pstart
           Das iſt der vorläufige Entwurf. Bitte, ſchreib’ mir nach \textsc{Shanghai\oindex{Shanghai@\textbf{Shanghai}|pw}, Deutsches Post Amt (\begin{otherlanguage}{english}German Post
                        Office\end{otherlanguage})\oindex{Deutsches Postamt in Shanghai@\textbf{Deutsches Postamt in Shanghai}|pwv} Poste Restante}. Ich bin dort vorausſichtlich
               Ende Mai, aber \strikeout{es wird}
               während der ganzen Dauer meiner Reiſe wird meine Adreſſe ſo lauten, da ich mir von
                  \textsc{Shanghai\oindex{Shanghai@\textbf{Shanghai}|pw}} immer die Briefe nachſchicken laſſen werde.\pend
           \pstart
           Was nach meiner Rückkehr ſein wird, weiß ich nicht. \textsc{Berlin\oindex{Berlin@\textbf{Berlin}|pw}} wohl kaum. Es ſind noch andere Projecte in der {\pb}Luſt, aber das Alles wird ſich wohl zerſchlagen, und ich werde ins Joch nach \textsc{Paris\oindex{Paris@\textbf{Paris}|pw}} zurück müſſen.\pend
           \pstart
           Wie ſchön iſt \textsc{Genua\oindex{Genua@\textbf{Genua}|pw}}. Nie in meinem Leben habe ich ſolche Paläſte geſehen. Kennſt Du es? Die italien\oindex{Italien@\textbf{Italien}|pw}iſche \textsc{Renaissance} iſt doch unübertroffen, ſelbſt im Großartigen. Die fran\oindex{Frankreich@\textbf{Frankreich}|pwv}zöſiſche und deutſch\oindex{Deutschland@\textbf{Deutschland}|pwv}e Renaiſſance iſt nur
               nachempfunden.\pend
           \pstart
           Und dieſe liebe goldene Sonne! Armer Freund Du in Deinem Winter! .\textcolor{gray}{{\dotstwo}}\pend
           \pstart
           Ich umarme Dich im Geiſte, mein lieber Arthur, und grüße Dich noch einmal von ganzem
                  {\pb}Herzen! Ich will von unterwegs viel an Dich
               denken. Bleib’ mir gut, liebſter Freund!\pend
           \pstart
           Dein treuer {\\[\baselineskip]}\spacefill\mbox{Paul Goldmann.}\pend
           \leftskip=0em{}\pstart
           \noindent{}Viele herzliche Grüße an Deine Freundin\pwindex{Reinhard, Marie 1871-03-13 – 1899-03-18@\textsc{Reinhard, Marie} (1871-03-13 – 1899-03-18), \emph{Gesangspädagogin}|pwv}!\pend
           \pstart
           Erhole Dich im Sommer und geh’ auch ein wenig in die Welt hinaus aus Deinem
                  Hypochondrie-Winkel, wo Du Dich mit ſchwarzen Gedanken eingeſponnen haſt! Du wirſt
                  ſehen, wie das Alles in der Sonne zerfliegt! Gerade geht ſie drüben über dem Meere
                  unter. Ich ſage Dir, draußen iſt Licht und Wärme!\pend
           \pstart
           Und nochmals Lebewohl!!!!\pend
           
         
         \endnumbering\mylabel{h}\end{ledgroupsized}  \newcommand{\dateiname}{L02847}\newcommand{\titel}{Paul Goldmann an Arthur Schnitzler, 4. 4. 1898}\newcommand{\editorInnen}{Martin Anton Müller und Laura Untner}%% latex-leseansicht-abspann.tex
%% Abspann für die Leseansicht.
%% Der Schalter \ifkorrekturansicht ist bereits durch den Vorspann gesetzt.

%% latex-abspann.tex
%% Gemeinsamer Abspann für Korrekturansicht und Leseansicht.
%% Setzt den Schalter \ifkorrekturansicht voraus (gesetzt in den
%% einbindenden Dateien latex-korrekturansicht-abspann.tex bzw.
%% latex-leseansicht-abspann.tex).
%% ---------------------------------------------------------------

\normalsize

% Das esempio-Environment wird nur in der Leseansicht benötigt
\ifkorrekturansicht\else
\newenvironment{esempio}[3]%
{
    \vspace{1.5ex}
    \rlap{\underline{#1}}
    \par
    \setlength{\parindent}{0cm}
    \nopagebreak
    \leftskip=#2cm
    \rightskip=#3cm
}
{
    \par
}
\fi

\doendnotes{C}
\bigskip
\vfill

\clearpage

\footnotesize

\ifkorrekturansicht
  \lohead{\textsc{register}}
\fi

% theindex-Environment neu definieren ohne reledmac
\makeatletter
\renewenvironment{theindex}{%
  \ifkorrekturansicht
    \section*{\indexname}%
  \else
    \subsubsection*{Index der erwähnten Entitäten}%
  \fi
  \setlength{\parindent}{0pt}%
  \setlength{\parskip}{0pt plus 0.3pt}%
  \let\item\@idxitem
}{%
  \ifkorrekturansicht\clearpage\fi
}
\makeatother

\IfFileExists{\jobname-pw.ind}{\input{\jobname-pw.ind}}{}

% Quellenangabe nur in der Leseansicht
\ifkorrekturansicht\else
% Fallback-Definitionen, falls die .tex-Datei \titel etc. nicht gesetzt hat
\providecommand{\titel}{}
\providecommand{\editorInnen}{}
\providecommand{\dateiname}{\jobname}

\vspace{3cm}

\vfill

\footnotesize
\textsc{Quelle}: \titel. Herausgegeben von {\editorInnen}. In: \emph{Arthur Schnitzler: Briefwechsel mit Autorinnen und Autoren}.
 Digitale Edition, https://schnitzler-briefe.acdh.oeaw.ac.at/{\dateiname}.html (Stand \today)
\fi

\end{document}


      