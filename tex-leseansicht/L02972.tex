%% latex-leseansicht-vorspann.tex
%% Vorspann für die Leseansicht.
%% Lädt die gemeinsame Datei latex-vorspann.tex mit nicht gesetztem Schalter.

\newif\ifkorrekturansicht
\korrekturansichtfalse

\input{../tex-inputs/latex-vorspann}


         
         \renewcommand{\erwaehntePersonen}{Personen: Felix Salten, Arthur Strasser, Viktor Oskar Tilgner}
         \renewcommand{\erwaehnteOrte}{Orte: Wien}
         \renewcommand{\erwaehnteWerke}{Werke: Das Mozartdenkmal, Das österreichische Antlitz. Essays, Künstlerhaus, Moderne Rundschau, Secession. (Arthur Strasser), Victor Tilgner †, Wiener Allgemeine Zeitung}
               \section[ Arthur Schnitzler an Felix Salten, 25. 3. {[}1902{]}]{ Arthur Schnitzler an Felix Salten, 25. 3. {[}1902{]}}\nopagebreak\mylabel{v}\rehead{ }\begin{ledgroupsized}[t]{13cm}\normalsize\beginnumbering\briefempfaengerindex{Salten, Felix@\textsc{Salten, Felix}!zzzSchnitzler, Arthur@\emph{von Arthur Schnitzler}!1902-03-251@{25. 3. {[}1902{]}}|(be} \toendnotes[C]{\smallbreak\pagebreak[2]} \Standort{Wienbibliothek im Rathaus, ZPH 1681, 2.1.516.}
\physDesc{Brief, 1 Blatt, 3 Seiten, 801 Zeichen
\newline{}Handschrift: Bleistift, deutsche Kurrent
\newline{}Ordnung: mit Bleistift von unbekannter Hand Nummerierung der Doppelseiten des
                                 Konvoluts: »3«–»4« }\toendnotes[C]{\smallbreak}\pstart
           \raggedleft{}{\pb}\label{K_L02972-1v}\edtext{Dinſtag 25. 3.}{\lemma{\textnormal{\emph{Dinſtag 25. 3.}}}\Cendnote{\textnormal{Die Datierung auf das Jahr 1902 ist möglich, da im in Frage kommenden Zeitraum
                        nur in diesem Jahr der 25. 3. ein Dienstag
                        war.}}}\label{K_L02972-1h}\pend
           \pstart
           liebſter Freund, ich habe heut{ }Nachmittg einen Theil der \label{K_L02972-2v}\edtext{Aufſätze}{\lemma{\textnormal{\emph{Aufſätze}}}\Cendnote{\textnormal{Salten\pwindex{Salten, Felix 06.09.1869 – 08.10.1945@\textsc{Salten, Felix} (06.09.1869 – 08.10.1945), \emph{Schriftsteller, Journalist}|pwk} plante eine Zusammenstellung seiner
                  kritischen Zeitungsarbeiten zu publizieren (vgl. A. S.: \emph{Tagebuch}, 30. 3. 1902). Dazu kam es jedoch nicht. Inwiefern das
                  Jahre später verfolgte Projekt »Aus einem Wiener Kreis« (vgl. A. S.: \emph{Tagebuch}, 18. 2. 1909) darauf Bezug
                  nimmt, und wie sich dieses wiederum zur im selben Jahr veröffentlichten
                  Textsammlung \emph{Das österreichische Antlitz}\pwindex{Salten, Felix 06.09.1869 – 08.10.1945@\textsc{Salten, Felix} (06.09.1869 – 08.10.1945), \emph{Schriftsteller, Journalist}!oesterreichische Antlitz. Essays1909@\strich\emph{Das österreichische Antlitz. Essays} {[}1909{]}|pwk}
                  verhält, lässt sich nicht bestimmen.}}}\label{K_L02972-2h} geleſen, darunter die zwei großen, Sie
               wiſſen, wie beträchtlich meine Schätzung \introOben{}ſchon\introOben{} bisher
               geweſen iſt, aber ich ka{\geminationn} Sie verſichern, die Sachen
               ſtehn auf einem noch höhern Niveau als wir geglaubt haben. Nebenbei {\pb}– das wird hoffentlich dem äußern Erfolg zu
               ſtatten kommen, – ſchreiben Sie ſo (entſchuldigen Sie das Wort) amuſant, dſs mir
               beinah die Phraſe vom »Nicht aus der Hand legen können« in die Feder gekommen
               wäre. –\pend
           \pstart
           Die Aufſätze über \textsc{\label{K_L02972-3v}\edtext{Strasser\pwindex{Strasser, Arthur 1854-04-08 – 1927-08-11@\textsc{Strasser, Arthur} (1854-04-08 – 1927-08-11), \emph{Maler, Bildhauer, Künstler}|pw}}{\lemma{\textnormal{\emph{Strasser}}}\Cendnote{\textnormal{Gemeint sein dürfte: Felix Salten\pwindex{Salten, Felix 06.09.1869 – 08.10.1945@\textsc{Salten, Felix} (06.09.1869 – 08.10.1945), \emph{Schriftsteller, Journalist}|pwk}: \emph{Secession. (Arthur Strasser)}\pwindex{Salten, Felix 06.09.1869 – 08.10.1945@\textsc{Salten, Felix} (06.09.1869 – 08.10.1945), \emph{Schriftsteller, Journalist}!Secession. (Arthur Strasser)1899-03-18@\strich\emph{Secession. (Arthur Strasser)} {[}1899-03-18{]}|pwk}. In: \emph{Wiener Allgemeine Zeitung}\pwindex{Wiener Allgemeine Zeitung1.3.1880 – 11.2.1934@\emph{Wiener Allgemeine Zeitung} {[}1.3.1880 – 11.2.1934{]}|pwk}, Nr. 6.313, 18. 3. 1899, S. 2–3.}}}\label{K_L02972-3h}} u \textsc{\label{K_L02972-4v}\edtext{Tilgner\pwindex{Tilgner, Viktor Oskar 1844-10-25 – 1896-04-16@\textsc{Tilgner, Viktor Oskar} (1844-10-25 – 1896-04-16), \emph{Bildhauer}|pw}}{\lemma{\textnormal{\emph{Tilgner}}}\Cendnote{\textnormal{Salten\pwindex{Salten, Felix 06.09.1869 – 08.10.1945@\textsc{Salten, Felix} (06.09.1869 – 08.10.1945), \emph{Schriftsteller, Journalist}|pwk} hatte mehrfach über den Bildhauer
                        Viktor Tilgner\pwindex{Tilgner, Viktor Oskar 1844-10-25 – 1896-04-16@\textsc{Tilgner, Viktor Oskar} (1844-10-25 – 1896-04-16), \emph{Bildhauer}|pwk} geschrieben, darunter: Felix Salten\pwindex{Salten, Felix 06.09.1869 – 08.10.1945@\textsc{Salten, Felix} (06.09.1869 – 08.10.1945), \emph{Schriftsteller, Journalist}|pwk}: \emph{Das Mozartdenkmal}\pwindex{Salten, Felix 06.09.1869 – 08.10.1945@\textsc{Salten, Felix} (06.09.1869 – 08.10.1945), \emph{Schriftsteller, Journalist}!Mozartdenkmal1891-04-01@\strich\emph{Das Mozartdenkmal} {[}1891-04-01{]}|pwk}. In: \emph{Moderne Rundschau}\pwindex{Moderne Rundschau1.4.1891 – 31.12.1891@\emph{Moderne Rundschau} {[}1.4.1891 – 31.12.1891{]}|pwk}, Jg. 1, Bd. 3, H. 1, 1. 4. 1891, S. 35–36; f. s.\pwindex{Salten, Felix 06.09.1869 – 08.10.1945@\textsc{Salten, Felix} (06.09.1869 – 08.10.1945), \emph{Schriftsteller, Journalist}|pwk}: \emph{Victor Tilgner †}\pwindex{Victor Tilgner †1896-04-17@\emph{Victor Tilgner †} {[}1896-04-17{]}|pwk}. In: \emph{Wiener Allgemeine Zeitung}\pwindex{Wiener Allgemeine Zeitung1.3.1880 – 11.2.1934@\emph{Wiener Allgemeine Zeitung} {[}1.3.1880 – 11.2.1934{]}|pwk}, Nr. 5.441, 17. 4. 1896, S. 3; Felix Salten\pwindex{Salten, Felix 06.09.1869 – 08.10.1945@\textsc{Salten, Felix} (06.09.1869 – 08.10.1945), \emph{Schriftsteller, Journalist}|pwk}: \emph{Künstlerhaus}\pwindex{Kuenstlerhaus1896-11-11@\emph{Künstlerhaus} {[}1896-11-11{]}|pwk}. In: \emph{Wiener Allgemeine Zeitung}\pwindex{Wiener Allgemeine Zeitung1.3.1880 – 11.2.1934@\emph{Wiener Allgemeine Zeitung} {[}1.3.1880 – 11.2.1934{]}|pwk}, Nr. 5.612, 11. 11. 1896, S. 3.}}}\label{K_L02972-4h}} heben Sie vielleicht {\pb}beſſer für eine
               ſpätere Sa{\geminationm}lung auf, um das »moderne \uline{Theater}« nicht zu ſtören? –\pend
           \pstart
           Zu überlegen, ob die Aufſätze über Literatur 48–98 und ü Theater 48–98 nicht bis auf
               den heutigen Tag fortzuſetzen wären. (Event. als Anhang?)\pend
           \pstart
           Auf Wiederſehen. Herzlichſt {\\[\baselineskip]}Ihr \spacefill\mbox{A.}\pend
           \leftskip=0em{}
         
         \endnumbering\mylabel{h}\end{ledgroupsized}  \newcommand{\dateiname}{L02972}\newcommand{\titel}{Arthur Schnitzler an Felix Salten, 25. 3. [1902]}\newcommand{\editorInnen}{Martin Anton Müller und Laura Untner}%% latex-leseansicht-abspann.tex
%% Abspann für die Leseansicht.
%% Der Schalter \ifkorrekturansicht ist bereits durch den Vorspann gesetzt.

%% latex-abspann.tex
%% Gemeinsamer Abspann für Korrekturansicht und Leseansicht.
%% Setzt den Schalter \ifkorrekturansicht voraus (gesetzt in den
%% einbindenden Dateien latex-korrekturansicht-abspann.tex bzw.
%% latex-leseansicht-abspann.tex).
%% ---------------------------------------------------------------

\normalsize

% Das esempio-Environment wird nur in der Leseansicht benötigt
\ifkorrekturansicht\else
\newenvironment{esempio}[3]%
{
    \vspace{1.5ex}
    \rlap{\underline{#1}}
    \par
    \setlength{\parindent}{0cm}
    \nopagebreak
    \leftskip=#2cm
    \rightskip=#3cm
}
{
    \par
}
\fi

\doendnotes{C}
\bigskip
\vfill

\clearpage

\footnotesize

\ifkorrekturansicht
  \lohead{\textsc{register}}
\fi

% theindex-Environment neu definieren ohne reledmac
\makeatletter
\renewenvironment{theindex}{%
  \ifkorrekturansicht
    \section*{\indexname}%
  \else
    \subsubsection*{Index der erwähnten Entitäten}%
  \fi
  \setlength{\parindent}{0pt}%
  \setlength{\parskip}{0pt plus 0.3pt}%
  \let\item\@idxitem
}{%
  \ifkorrekturansicht\clearpage\fi
}
\makeatother

\IfFileExists{\jobname-pw.ind}{\input{\jobname-pw.ind}}{}

% Quellenangabe nur in der Leseansicht
\ifkorrekturansicht\else
% Fallback-Definitionen, falls die .tex-Datei \titel etc. nicht gesetzt hat
\providecommand{\titel}{}
\providecommand{\editorInnen}{}
\providecommand{\dateiname}{\jobname}

\vspace{3cm}

\vfill

\footnotesize
\textsc{Quelle}: \titel. Herausgegeben von {\editorInnen}. In: \emph{Arthur Schnitzler: Briefwechsel mit Autorinnen und Autoren}.
 Digitale Edition, https://schnitzler-briefe.acdh.oeaw.ac.at/{\dateiname}.html (Stand \today)
\fi

\end{document}


      