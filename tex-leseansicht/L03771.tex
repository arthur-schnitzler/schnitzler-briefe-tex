%% latex-korrekturansicht-vorspann.tex
%% Vorspann für die Korrekturansicht.
%% Lädt die gemeinsame Datei latex-vorspann.tex mit gesetztem Schalter.

\newif\ifkorrekturansicht
\korrekturansichttrue

\input{../tex-inputs/latex-vorspann}


\section[Arthur Schnitzler an Stefan Zweig, 15. 1. 1915]{L03771 Arthur Schnitzler an Stefan Zweig, 15. 1. 1915}
\nopagebreak\mylabel{L03771v}
\rehead{ }\normalsize\beginnumbering\briefempfaengerindex{Zweig, Stefan@\textsc{Zweig, Stefan}!zzzSchnitzler, Arthur@\emph{von Arthur Schnitzler}!1915-01-151@{15. 1. 1915}|(be}
\toendnotes[C]{\smallbreak\pagebreak[2]}\Standort{Jerusalem, National Library of Israel, ARC. Ms. Var. 305 1 58 Stefan Zweig Collection.}
\physDesc{Briefkarte, 1 Blatt, 1 Seite, 533 Zeichen
\newline{}Schreibmaschine
\newline{}Handschrift: schwarze Tinte (\noindent{}Korrekturen, Unterschrift)}\toendnotes[C]{\smallbreak}
\pstart
           {\pb}\textcolor{gray}{\textbf{Dr. Arthur Schnitzler}}\hfill 15. 1. 1915. \pend
           
\pstart
           \textcolor{gray}{\textbf{Wien XVIII. Sternwartestrasse 71\oindex{Sternwartestrasse 71@\textbf{Sternwartestraße 71}, \emph{Wohngebäude (K.WHS)}|pw}}}\pend
           
\pstart\center{}Lieber Herr Doktor Zweig.\pend\vspace{0.5em}
\pstart
           Rolland\pwindex{Rolland, Romain 29.01.1866 – 30.12.1944@\textsc{Rolland, Romain} (29.01.1866 – 30.12.1944), \emph{Schriftsteller/Schriftstellerin}|pw}{ }\label{K_L03771-1v}\edtext{schreibt mir (am
                  11. d.)}{\lemma{\textnormal{\emph{schreibt mir (am
                  11. d.)}}}\Cendnote{\textnormal{Das
                  Korrespondenzstück ist nicht überliefert.}}}\label{K_L03771-1}, dass er ihnen dreimal geschrieben
               und Ihnen einen Brief von Richard Bloch\pwindex{Bloch, Richard 03.03.1856 – 1928@\textsc{Bloch, Richard} (03.03.1856 – 1928), \emph{Theaterverleger/Theaterverlegerin}|pw} für
                  Paul Amann\pwindex{Amann, Paul 1884-03-06 – 1958-02-24@\textsc{Amann, Paul} (1884-03-06 – 1958-02-24), \emph{Übersetzer/Übersetzerin, Philologe/Philologin, Lehrer/Lehrerin}|pw} geschickt hätte. Das Journal de Genève\pwindex{Journal de Geneve@\emph{Journal de Genève}|pw}\strikeout{,} ist mir nicht zugekommen, ja bisher nicht einmal
               der betreffende \label{K_L03771-2v}\edtext{Ausschnitt\pwindex{Une protestation DArthur Schnitzler@\emph{Une protestation d’Arthur Schnitzler}|pwv} mit der Rolland\pwindex{Rolland, Romain 29.01.1866 – 30.12.1944@\textsc{Rolland, Romain} (29.01.1866 – 30.12.1944), \emph{Schriftsteller/Schriftstellerin}|pw}’schen Uebersetzung\pwindex{Une protestation DArthur Schnitzler@\emph{Une protestation d’Arthur Schnitzler}|pwv}}{\lemma{\textnormal{\emph{Ausschnitt … Uebersetzung}}}\Cendnote{\textnormal{Arthur Schnitzler, Romain Rolland\pwindex{Rolland, Romain 29.01.1866 – 30.12.1944@\textsc{Rolland, Romain} (29.01.1866 – 30.12.1944), \emph{Schriftsteller/Schriftstellerin}|pwk} [Einleitung und Übersetzung]: \emph{Une protestation d’Arthur Schnitzler}\pwindex{Une protestation DArthur Schnitzler@\emph{Une protestation d’Arthur Schnitzler}|pwk}. In:
                        \emph{Journal de Genève}\pwindex{Journal de Geneve@\emph{Journal de Genève}|pwk}, Jg. 85, 21. 12. 1914, 3. Ausgabe,
                  S. [1].}}}\label{K_L03771-2} meines \label{K_L03771-3v}\edtext{Protestes\pwindex{Brief Artur Schnitzlers@\emph{Ein Brief Artur Schnitzlers}|pwv}}{\lemma{\textnormal{\emph{Protestes}}}\Cendnote{\textnormal{Die deutschsprachige Veröffentlichung des
                  Protestes erschien am Tag nach der französischen: \emph{Ein Brief Artur Schnitzlers}\pwindex{Brief Artur Schnitzlers@\emph{Ein Brief Artur Schnitzlers}|pwk}. In: \emph{Neue Zürcher Zeitung}\pwindex{Neue Zuercher Zeitung@\emph{Neue Zürcher Zeitung}|pwk}, Jg. 135, Nr. 1700,
                        22. 12. 1914, 2. Mittagsblatt,
                  S. 2.}}}\label{K_L03771-3}, den er, Rolland\pwindex{Rolland, Romain 29.01.1866 – 30.12.1944@\textsc{Rolland, Romain} (29.01.1866 – 30.12.1944), \emph{Schriftsteller/Schriftstellerin}|pw},
               nach Nichteinlangen des J. d. G.\pwindex{Journal de Geneve@\emph{Journal de Genève}|pw}, \introOben{}mir\introOben{} zuzusenden versucht hat. So nehme ich an, dass auch einige
               von den Rolland\pwindex{Rolland, Romain 29.01.1866 – 30.12.1944@\textsc{Rolland, Romain} (29.01.1866 – 30.12.1944), \emph{Schriftsteller/Schriftstellerin}|pw}’schen Briefen an Sie von der
               Zensur zurückgehalten wurden.\pend
           
\pstart
           Herzlichen Gruss{\\[\baselineskip]}Ihr{\\[\baselineskip]}\spacefill\mbox{{[}hs.:{]} ArthurSchnitzler}\pend
           \leftskip=0em{}\selectlanguage{ngerman}\endnumbering\briefempfaengerindex{Zweig, Stefan@\textsc{Zweig, Stefan}!zzzSchnitzler, Arthur@\emph{von Arthur Schnitzler}!1915-01-151@{15. 1. 1915}|)be}\mylabel{L03771h}
\begin{anhang}
\end{anhang}\normalsize

\doendnotes{C}
\bigskip
\vfill

\clearpage

\footnotesize

\lohead{\textsc{register}}

% Definiere theindex-Environment komplett neu ohne reledmac
\makeatletter
\renewenvironment{theindex}{%
  \section*{\indexname}%
  \setlength{\parindent}{0pt}%
  \setlength{\parskip}{0pt plus 0.3pt}%
  \let\item\@idxitem
}{%
  \clearpage
}
\makeatother

\IfFileExists{\jobname-pw.ind}{\input{\jobname-pw.ind}}{}

\end{document}

      