%% latex-leseansicht-vorspann.tex
%% Vorspann für die Leseansicht.
%% Lädt die gemeinsame Datei latex-vorspann.tex mit nicht gesetztem Schalter.

\newif\ifkorrekturansicht
\korrekturansichtfalse

\input{../tex-inputs/latex-vorspann}


\section[Arthur Schnitzler an Stefan Zweig, 15. 1. 1915]{L03771 Arthur Schnitzler an Stefan Zweig, 15. 1. 1915}
\nopagebreak\mylabel{L03771v}
\rehead{ }\normalsize\beginnumbering\briefempfaengerindex{Zweig, Stefan@\textsc{Zweig, Stefan}!zzzSchnitzler, Arthur@\emph{von Arthur Schnitzler}!1915-01-151@{15. 1. 1915}|(be}
\toendnotes[C]{\smallbreak\pagebreak[2]}
\correspDesc{Versand  durch Arthur Schnitzler am 15. 1. 1915 in Wien
\newline{}Erhalt  durch Stefan Zweig im Zeitraum [15. 1. 1915
                  – 18. 1. 1915?] in Wien}\toendnotes[C]{\smallbreak}
\Standort{Jerusalem, National Library of Israel, ARC. Ms. Var. 305 1 58 Stefan Zweig Collection.}
\physDesc{Briefkarte, 534 Zeichen
\newline{}Schreibmaschine
\newline{}Handschrift: schwarze Tinte (\noindent{}Korrekturen, Unterschrift)}\toendnotes[C]{\smallbreak}
\pstart
           {\pb}\textcolor{gray}{\textbf{Dr. Arthur Schnitzler}}\hfill 15. 1. 1915.\pend
           
\pstart
           \textcolor{gray}{\textbf{Wien XVIII. Sternwartestrasse 71\oindex{Wien@\textbf{Wien}!XVIII., Währing@\textbf{XVIII., Währing}!Sternwartestraße 71@\textbf{Sternwartestraße 71}, \emph{Wohngebäude}|pw}}}\pend
           
\pstart\center{}Lieber Herr Doktor Zweig.\pend\vspace{0.5em}
\pstart
           Rolland\pwindex{Rolland, Romain 29.\,1.\,1866 Clamecy – 30.\,12.\,1944 Vézelay@\textsc{Rolland, Romain} (29.\,1.\,1866 Clamecy – 30.\,12.\,1944 Vézelay), \emph{Schriftsteller}|pw}{ }\label{K_L03771-1v}\edtext{schreibt mir (am
                  11. d.)}{\lemma{\textnormal{\emph{schreibt mir (am
                  11. d.)}}}\Cendnote{\textnormal{Das
                  Korrespondenzstück ist nicht überliefert. Im Nachlass Schnitzlers in der 
                  \emph{Cambridge University Library} (B 86)
                  sind zwei Briefe überliefert, XXXX Auszeichnungsfehler: Dokument L03882 nicht gefunden und
                     XXXX Auszeichnungsfehler: Dokument L03883 nicht gefunden. Auf der Mappe selbst steht mit
                     Bleistift von unbekannter Hand: »entnommen ein Brief«. Dabei könnte
                     es sich um den hier erwähnten gehandelt haben.}}}\label{K_L03771-1}, dass er ihnen dreimal geschrieben
               und Ihnen einen Brief von Richard Bloch\pwindex{Bloch, Richard 3.\,3.\,1856 Berlin – 1928 ebd.@\textsc{Bloch, Richard} (3.\,3.\,1856 Berlin – 1928 ebd.), \emph{Theaterverleger}|pw} für
                  Paul Amann\pwindex{Amann, Paul 6.\,3.\,1884 Prag – 24.\,2.\,1958 Fairfield County@\textsc{Amann, Paul} (6.\,3.\,1884 Prag – 24.\,2.\,1958 Fairfield County), \emph{Übersetzer, Philologe, Lehrer}|pw} geschickt hätte. Das Journal de Genève\pwindex{Journal de Genève@\emph{Journal de Genève}|pw}\strikeout{,} ist mir nicht zugekommen, ja bisher nicht einmal
               der betreffende \label{K_L03771-2v}\edtext{Ausschnitt\pwindex{Schnitzler, Arthur 15.\,5.\,1862 Wien – 21.\,10.\,1931 ebd.@\textsc{Schnitzler, Arthur} (15.\,5.\,1862 Wien – 21.\,10.\,1931 ebd.), \emph{Schriftsteller, Mediziner}!Une protestation d’Arthur Schnitzler@\strich\emph{Une protestation d’Arthur Schnitzler}|pwv} mit der Rolland\pwindex{Rolland, Romain 29.\,1.\,1866 Clamecy – 30.\,12.\,1944 Vézelay@\textsc{Rolland, Romain} (29.\,1.\,1866 Clamecy – 30.\,12.\,1944 Vézelay), \emph{Schriftsteller}|pw}’schen Uebersetzung\pwindex{Schnitzler, Arthur 15.\,5.\,1862 Wien – 21.\,10.\,1931 ebd.@\textsc{Schnitzler, Arthur} (15.\,5.\,1862 Wien – 21.\,10.\,1931 ebd.), \emph{Schriftsteller, Mediziner}!Une protestation d’Arthur Schnitzler@\strich\emph{Une protestation d’Arthur Schnitzler}|pwv}}{\lemma{\textnormal{\emph{Ausschnitt … Uebersetzung}}}\Cendnote{\textnormal{Arthur Schnitzler, Romain Rolland\pwindex{Rolland, Romain 29.\,1.\,1866 Clamecy – 30.\,12.\,1944 Vézelay@\textsc{Rolland, Romain} (29.\,1.\,1866 Clamecy – 30.\,12.\,1944 Vézelay), \emph{Schriftsteller}|pwk} [Einleitung und Übersetzung]: \emph{Une protestation d’Arthur Schnitzler}\pwindex{Schnitzler, Arthur 15.\,5.\,1862 Wien – 21.\,10.\,1931 ebd.@\textsc{Schnitzler, Arthur} (15.\,5.\,1862 Wien – 21.\,10.\,1931 ebd.), \emph{Schriftsteller, Mediziner}!Une protestation d’Arthur Schnitzler@\strich\emph{Une protestation d’Arthur Schnitzler}|pwk}. In:
                        \emph{Journal de Genève}\pwindex{Journal de Genève@\emph{Journal de Genève}|pwk}, Jg. 85, 21.\,12.\,1914, 3. Ausgabe,
                  S. [1].}}}\label{K_L03771-2} meines \label{K_L03771-3v}\edtext{Protestes\pwindex{Schnitzler, Arthur 15.\,5.\,1862 Wien – 21.\,10.\,1931 ebd.@\textsc{Schnitzler, Arthur} (15.\,5.\,1862 Wien – 21.\,10.\,1931 ebd.), \emph{Schriftsteller, Mediziner}!Brief Artur Schnitzlers@\strich\emph{Ein Brief Artur Schnitzlers}|pwv}}{\lemma{\textnormal{\emph{Protestes}}}\Cendnote{\textnormal{Die deutschsprachige Veröffentlichung des
                  Protestes erschien am Tag nach der französischen: \emph{Ein Brief Artur Schnitzlers}\pwindex{Schnitzler, Arthur 15.\,5.\,1862 Wien – 21.\,10.\,1931 ebd.@\textsc{Schnitzler, Arthur} (15.\,5.\,1862 Wien – 21.\,10.\,1931 ebd.), \emph{Schriftsteller, Mediziner}!Brief Artur Schnitzlers@\strich\emph{Ein Brief Artur Schnitzlers}|pwk}. In: \emph{Neue Zürcher Zeitung}\pwindex{Neue Zürcher Zeitung@\emph{Neue Zürcher Zeitung}|pwk}, Jg. 135, Nr. 1700,
                        22. 12. 1914, 2. Mittagsblatt,
                  S. 2.}}}\label{K_L03771-3}, den er, Rolland\pwindex{Rolland, Romain 29.\,1.\,1866 Clamecy – 30.\,12.\,1944 Vézelay@\textsc{Rolland, Romain} (29.\,1.\,1866 Clamecy – 30.\,12.\,1944 Vézelay), \emph{Schriftsteller}|pw},
               nach Nichteinlangen des J. d. G.\pwindex{Journal de Genève@\emph{Journal de Genève}|pw}, \introOben{}mir\introOben{} zuzusenden versucht hat. So nehme ich an, dass auch einige
               von den Rolland\pwindex{Rolland, Romain 29.\,1.\,1866 Clamecy – 30.\,12.\,1944 Vézelay@\textsc{Rolland, Romain} (29.\,1.\,1866 Clamecy – 30.\,12.\,1944 Vézelay), \emph{Schriftsteller}|pw}’schen Briefen an Sie von der
               Zensur zurückgehalten wurden.\pend
           
\pstart
           Herzlichen Gruss{\\[\baselineskip]}Ihr{\\[\baselineskip]}\spacefill\mbox{{[}hs.:{]} ArthurSchnitzler}\pend
           \leftskip=0em{}\selectlanguage{ngerman}\endnumbering\briefempfaengerindex{Zweig, Stefan@\textsc{Zweig, Stefan}!zzzSchnitzler, Arthur@\emph{von Arthur Schnitzler}!1915-01-151@{15. 1. 1915}|)be}\mylabel{L03771h}
\begin{anhang}
\end{anhang}\newcommand{\dateiname}{L03771}\newcommand{\titel}{Arthur Schnitzler an Stefan Zweig, 15. 1. 1915}\newcommand{\editorInnen}{Selma Jahnke und Martin Anton Müller}%% latex-leseansicht-abspann.tex
%% Abspann für die Leseansicht.
%% Der Schalter \ifkorrekturansicht ist bereits durch den Vorspann gesetzt.

%% latex-abspann.tex
%% Gemeinsamer Abspann für Korrekturansicht und Leseansicht.
%% Setzt den Schalter \ifkorrekturansicht voraus (gesetzt in den
%% einbindenden Dateien latex-korrekturansicht-abspann.tex bzw.
%% latex-leseansicht-abspann.tex).
%% ---------------------------------------------------------------

\normalsize

% Das esempio-Environment wird nur in der Leseansicht benötigt
\ifkorrekturansicht\else
\newenvironment{esempio}[3]%
{
    \vspace{1.5ex}
    \rlap{\underline{#1}}
    \par
    \setlength{\parindent}{0cm}
    \nopagebreak
    \leftskip=#2cm
    \rightskip=#3cm
}
{
    \par
}
\fi

\doendnotes{C}
\bigskip
\vfill

\clearpage

\footnotesize

\ifkorrekturansicht
  \lohead{\textsc{register}}
\fi

% theindex-Environment neu definieren ohne reledmac
\makeatletter
\renewenvironment{theindex}{%
  \ifkorrekturansicht
    \section*{\indexname}%
  \else
    \subsubsection*{Index der erwähnten Entitäten}%
  \fi
  \setlength{\parindent}{0pt}%
  \setlength{\parskip}{0pt plus 0.3pt}%
  \let\item\@idxitem
}{%
  \ifkorrekturansicht\clearpage\fi
}
\makeatother

\IfFileExists{\jobname-pw.ind}{\input{\jobname-pw.ind}}{}

% Quellenangabe nur in der Leseansicht
\ifkorrekturansicht\else
% Fallback-Definitionen, falls die .tex-Datei \titel etc. nicht gesetzt hat
\providecommand{\titel}{}
\providecommand{\editorInnen}{}
\providecommand{\dateiname}{\jobname}

\vspace{3cm}

\vfill

\footnotesize
\textsc{Quelle}: \titel. Herausgegeben von {\editorInnen}. In: \emph{Arthur Schnitzler: Briefwechsel mit Autorinnen und Autoren}.
 Digitale Edition, https://schnitzler-briefe.acdh.oeaw.ac.at/{\dateiname}.html (Stand \today)
\fi

\end{document}


