%% latex-leseansicht-vorspann.tex
%% Vorspann für die Leseansicht.
%% Lädt die gemeinsame Datei latex-vorspann.tex mit nicht gesetztem Schalter.

\newif\ifkorrekturansicht
\korrekturansichtfalse

\input{../tex-inputs/latex-vorspann}

\begin{center}
            \textcolor{red}{ENTWURF. ENTZIFFERUNG NOCH NICHT KORREKTURGELESEN}
                      \end{center}
            
               \section[Arthur Schnitzler an Hugo Hofmannsthal, 11. 3. 1926]{ Arthur Schnitzler an Hugo Hofmannsthal, 11. 3. 1926}\nopagebreak\mylabel{v}\rehead{ }\begin{ledgroupsized}[t]{13cm}\normalsize\beginnumbering\briefempfaengerindex{Hofmannsthal, Hugo von@\textsc{Hofmannsthal, Hugo von}!zzzSchnitzler, Arthur@\emph{von Arthur Schnitzler}!1926-03-111@{11. 3. 1926}|(be} \toendnotes[C]{\smallbreak\pagebreak[2]} \Standort{FDH, Hs-30885,156.}
\physDesc{Brief, 1 Blatt, 2 Seiten
\newline{}Handschrift: Bleistift, lateinische Kurrent}\buchAbdrucke{\weitereDrucke{Hugo von Hofmannsthal, Arthur Schnitzler: \emph{Briefwechsel}. Hg. Therese Nickl und Heinrich Schnitzler. Frankfurt am Main: \emph{S. Fischer} 1964, S. 306.} }\pstart
           \raggedleft{}{\pb}Wien\oindex{Wien@\textbf{Wien}|pw}, 11. März 1926\pend
           \pstart
           lieber Hugo, vom Verlag Reiss\orgindex{Erich Reiss@Erich Reiß|pw}
                    weiſs ich nur, dſs dort einige sehr gute und etliche bedeutende Bücher
                        herausgeko{\geminationm}en sind, (was alle Leute wissen) –
                        \introOben{}in Hinsicht aufs\introOben{} menschliche und geschäftliche bin
                    ich absolut nicht informirt – bin mir also gar nicht klar, wie ich solch eine
                    Bescheinigung abzufassen hätte, daſs sie für den Verlag nur einigermaßen
                    nutzbringend sich erweisen könnte. Worum handelt es sich de{\geminationn} eigentlich –? Um Sanirung? Um Verkauf? – Mir ist
                    der Sinn der Action nicht evident. Genügt meine Erklärung, daſs ich den
                    Zusa{\geminationm}enbruch eines Verlags bedauern würde, in dem viel vortreffliches
                    erschienen ist, so steh ich gern zur Verfügung. Ich lege für alle Fälle gleich
                    ein Blatt bei, vielleicht genügt es.\pend
           \pstart
           {\pb}Sonderbar, dſs ich gerade gestern, mit Andacht
                    fast könnt ich sagen, und jedenfalls mit \uline{tiefster} Bewegung eine ganze Anzahl Ihrer Gedichte \introOben{}wieder\introOben{}gelesen u empfunden habe, wie unerhört neu die Melodie und der
                    Rythmus ist, den Sie in die deutsche Dichtung gebracht haben, – und wie er durch
                    die Zeiten weiterschwingt.\pend
           \pstart
           Auf Wiedersehen also, sobald freundlichere Tage ko{\geminationm}en.\pend
           \pstart
           Von Herzen Ihr{\\[\baselineskip]}\spacefill\mbox{Arth}\pend
           \leftskip=0em{}\pstart
           \noindent{}Lili\pwindex{Schnitzler, Lili 13.09.1909 – 26.07.1928@\textsc{Schnitzler, Lili} (13.09.1909 – 26.07.1928)|pw} bestell ich alles, sie wird sehr
                        stolz sein daſs sie Ihnen »freundlichst verzeihen soll« – (und daſs sie zu
                        so interessanten allgemeinen Bemerkungen Anlaſs gab).\pend
           \endnumbering\briefempfaengerindex{Hofmannsthal, Hugo von@\textsc{Hofmannsthal, Hugo von}!zzzSchnitzler, Arthur@\emph{von Arthur Schnitzler}!1926-03-111@{11. 3. 1926}|)be}\mylabel{h}\end{ledgroupsized}  \newcommand{\dateiname}{L02467}\newcommand{\titel}{Arthur Schnitzler an Hugo Hofmannsthal, 11. 3. 1926}\newcommand{\editorInnen}{Martin Anton Müller und Gerd-Hermann Susen}%% latex-leseansicht-abspann.tex
%% Abspann für die Leseansicht.
%% Der Schalter \ifkorrekturansicht ist bereits durch den Vorspann gesetzt.

%% latex-abspann.tex
%% Gemeinsamer Abspann für Korrekturansicht und Leseansicht.
%% Setzt den Schalter \ifkorrekturansicht voraus (gesetzt in den
%% einbindenden Dateien latex-korrekturansicht-abspann.tex bzw.
%% latex-leseansicht-abspann.tex).
%% ---------------------------------------------------------------

\normalsize

% Das esempio-Environment wird nur in der Leseansicht benötigt
\ifkorrekturansicht\else
\newenvironment{esempio}[3]%
{
    \vspace{1.5ex}
    \rlap{\underline{#1}}
    \par
    \setlength{\parindent}{0cm}
    \nopagebreak
    \leftskip=#2cm
    \rightskip=#3cm
}
{
    \par
}
\fi

\doendnotes{C}
\bigskip
\vfill

\clearpage

\footnotesize

\ifkorrekturansicht
  \lohead{\textsc{register}}
\fi

% theindex-Environment neu definieren ohne reledmac
\makeatletter
\renewenvironment{theindex}{%
  \ifkorrekturansicht
    \section*{\indexname}%
  \else
    \subsubsection*{Index der erwähnten Entitäten}%
  \fi
  \setlength{\parindent}{0pt}%
  \setlength{\parskip}{0pt plus 0.3pt}%
  \let\item\@idxitem
}{%
  \ifkorrekturansicht\clearpage\fi
}
\makeatother

\IfFileExists{\jobname-pw.ind}{\input{\jobname-pw.ind}}{}

% Quellenangabe nur in der Leseansicht
\ifkorrekturansicht\else
% Fallback-Definitionen, falls die .tex-Datei \titel etc. nicht gesetzt hat
\providecommand{\titel}{}
\providecommand{\editorInnen}{}
\providecommand{\dateiname}{\jobname}

\vspace{3cm}

\vfill

\footnotesize
\textsc{Quelle}: \titel. Herausgegeben von {\editorInnen}. In: \emph{Arthur Schnitzler: Briefwechsel mit Autorinnen und Autoren}.
 Digitale Edition, https://schnitzler-briefe.acdh.oeaw.ac.at/{\dateiname}.html (Stand \today)
\fi

\end{document}


      