%% latex-leseansicht-vorspann.tex
%% Vorspann für die Leseansicht.
%% Lädt die gemeinsame Datei latex-vorspann.tex mit nicht gesetztem Schalter.

\newif\ifkorrekturansicht
\korrekturansichtfalse

\input{../tex-inputs/latex-vorspann}


\section[ Paul Goldmann an Arthur Schnitzler, 27. 2. [1903]]{L03365 Paul Goldmann an Arthur Schnitzler,  27. 2. [1903]}
\nopagebreak\mylabel{L03365v}
\rehead{ }\normalsize\beginnumbering\briefempfaengerindex{Schnitzler, Arthur@\textsc{Schnitzler, Arthur}!zzzGoldmann, Paul@\emph{von Paul Goldmann}!1903-02-272@{27. 2. [1903]}|(be}
\toendnotes[C]{\smallbreak\pagebreak[2]}
\correspDesc{Versand  durch Paul Goldmann am 27. 2. [1903] in Berlin
\newline{}Erhalt  durch Arthur Schnitzler im Zeitraum [27. 2. 1903
                  – 28. 2. 1903?] in Berlin}\toendnotes[C]{\smallbreak}
\Standort{DLA, A:Schnitzler, HS.NZ85.1.3173.}
\physDesc{Brief, 1 Blatt, 3 Seiten, 645 Zeichen
\newline{}Handschrift: blaue Tinte, deutsche Kurrent}\toendnotes[C]{\smallbreak}
\pstart
           \raggedleft{}{\pb}\textcolor{gray}{\textbf{DESSAUERSTRASSE 19\oindex{Dessauer Straße@\textbf{Dessauer Straße}, \emph{Straße}|pw}}}\pend
           
\pstart
           Berlin\oindex{Berlin@\textbf{Berlin}, \emph{Hauptstadt}|pw}, 27. Februar.\pend
           
\pstart\center{}Liebſter Freund,\pend\vspace{0.5em}
\pstart
           Bis \label{K_L03365-1v}\edtext{½ 8}{\lemma{\textnormal{\emph{½ 8}}}\Cendnote{\textnormal{Es dürfte sich, wie aus dem Folgenden
                  hervorgeht, um 7:30 morgens gehandelt haben. Wo der Treffpunkt
                  angesetzt war, ist nicht zu bestimmen. Schnitzler dürfte danach zur Probe von \emph{Der Schleier der Beatrice}\pwindex{Schnitzler, Arthur 15.\,5.\,1862 Wien – 21.\,10.\,1931 ebd.@\textsc{Schnitzler, Arthur} (15.\,5.\,1862 Wien – 21.\,10.\,1931 ebd.), \emph{Schriftsteller, Mediziner}!Schleier der Beatrice. Schauspiel in fünf Akten@\strich\emph{Der Schleier der Beatrice. Schauspiel in fünf Akten}|pwk} gegangen sein.}}}\label{K_L03365-1} habe ich auf Dich
               gewartet. Dann mußte ich fort, um allerlei Informations-Wünſche der Wien\oindex{Wien@\textbf{Wien}, \emph{Verwaltungsgebiet}|pw}er Redaktion\orgindex{Neue Freie Presse@Neue Freie Presse|pwv} zu befriedigen, glaubte auch, Du würdeſt nicht mehr kommen. Um
                  10 Uhr komme ich zurück und höre, daß Du da {\pb}warſt. Es thut mir unendlich leid, daß wir uns
               verfehlt haben. Ich habe um 10 Uhr noch in Dein \textsc{Hotel\orgindex{Palasthotel Berlin@Palasthotel Berlin|pw}} telephonirt, höre aber, daß Du nicht mehr dort zu finden biſt. Kann ich Dich
                  \label{K_L03365-2v}\edtext{morgen, Samſtag}{\lemma{\textnormal{\emph{morgen, Samstag}}}\Cendnote{\textnormal{Ein Treffen am Samstag, dem 28. 2. 1903 kam
                  zustande. Am Sonntag, dem 1. 3. 1903 sahen sie sich nicht.}}}\label{K_L03365-2}, Abend
                  nach 10 Uhr{ }ſehen? Wenn Du kannſt,{ }ſo komme doch, bitte, \substVorne{}\textsuperscript{um}\substDazwischen{}gegen\substHinten{}{ }1\substVorne{}\textsuperscript{\textcolor{gray}{×}}\substDazwischen{}7\substHinten{} Uhr zu mir hinauf. Wenn {\pb}nicht,{ }ſo laſſe
               mir Nachricht zukommen, ob ich Dich Sonntag{ }Nachmittag oder Abend{ }ſprechen kann.\pend
           
\pstart
           Herzlichſt {\\[\baselineskip]}Dein {\\[\baselineskip]}\spacefill\mbox{Paul Goldm}\pend
           \leftskip=0em{}\selectlanguage{ngerman}\endnumbering\briefempfaengerindex{Schnitzler, Arthur@\textsc{Schnitzler, Arthur}!zzzGoldmann, Paul@\emph{von Paul Goldmann}!1903-02-272@{27. 2. [1903]}|)be}\mylabel{L03365h}  \newcommand{\dateiname}{L03365}\newcommand{\titel}{Paul Goldmann an Arthur Schnitzler, 27. 2. [1903]}\newcommand{\editorInnen}{Martin Anton Müller und Laura Untner}%% latex-leseansicht-abspann.tex
%% Abspann für die Leseansicht.
%% Der Schalter \ifkorrekturansicht ist bereits durch den Vorspann gesetzt.

%% latex-abspann.tex
%% Gemeinsamer Abspann für Korrekturansicht und Leseansicht.
%% Setzt den Schalter \ifkorrekturansicht voraus (gesetzt in den
%% einbindenden Dateien latex-korrekturansicht-abspann.tex bzw.
%% latex-leseansicht-abspann.tex).
%% ---------------------------------------------------------------

\normalsize

% Das esempio-Environment wird nur in der Leseansicht benötigt
\ifkorrekturansicht\else
\newenvironment{esempio}[3]%
{
    \vspace{1.5ex}
    \rlap{\underline{#1}}
    \par
    \setlength{\parindent}{0cm}
    \nopagebreak
    \leftskip=#2cm
    \rightskip=#3cm
}
{
    \par
}
\fi

\doendnotes{C}
\bigskip
\vfill

\clearpage

\footnotesize

\ifkorrekturansicht
  \lohead{\textsc{register}}
\fi

% theindex-Environment neu definieren ohne reledmac
\makeatletter
\renewenvironment{theindex}{%
  \ifkorrekturansicht
    \section*{\indexname}%
  \else
    \subsubsection*{Index der erwähnten Entitäten}%
  \fi
  \setlength{\parindent}{0pt}%
  \setlength{\parskip}{0pt plus 0.3pt}%
  \let\item\@idxitem
}{%
  \ifkorrekturansicht\clearpage\fi
}
\makeatother

\IfFileExists{\jobname-pw.ind}{\input{\jobname-pw.ind}}{}

% Quellenangabe nur in der Leseansicht
\ifkorrekturansicht\else
% Fallback-Definitionen, falls die .tex-Datei \titel etc. nicht gesetzt hat
\providecommand{\titel}{}
\providecommand{\editorInnen}{}
\providecommand{\dateiname}{\jobname}

\vspace{3cm}

\vfill

\footnotesize
\textsc{Quelle}: \titel. Herausgegeben von {\editorInnen}. In: \emph{Arthur Schnitzler: Briefwechsel mit Autorinnen und Autoren}.
 Digitale Edition, https://schnitzler-briefe.acdh.oeaw.ac.at/{\dateiname}.html (Stand \today)
\fi

\end{document}


