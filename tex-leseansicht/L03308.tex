%% latex-leseansicht-vorspann.tex
%% Vorspann für die Leseansicht.
%% Lädt die gemeinsame Datei latex-vorspann.tex mit nicht gesetztem Schalter.

\newif\ifkorrekturansicht
\korrekturansichtfalse

\input{../tex-inputs/latex-vorspann}


\section[ Felix Salten an Arthur Schnitzler, 7. 8. 1900]{L03308 Felix Salten an Arthur Schnitzler,  7. 8. 1900}
\nopagebreak\mylabel{L03308v}
\rehead{ }\normalsize\beginnumbering\briefempfaengerindex{Schnitzler, Arthur@\textsc{Schnitzler, Arthur}!zzzSalten, Felix@\emph{von Felix Salten}!1900-08-072@{7. 8. 1900}|(be}
\toendnotes[C]{\smallbreak\pagebreak[2]}
\correspDesc{Versand  durch Felix Salten am 7. 8. 1900 in Wien
\newline{}Erhalt  durch Arthur Schnitzler im Zeitraum [8. 8. 1900
                  – 12. 8. 1900?] in Bad Ischl}\toendnotes[C]{\smallbreak}
\Standort{CUL, Schnitzler, B 89, A 2.}
\physDesc{Brief, 1 Blatt, 2 Seiten, 674 Zeichen
\newline{}Handschrift: schwarze Tinte, lateinische Kurrent
\newline{}Ordnung: mit Bleistift von unbekannter Hand nummeriert: »132« }\toendnotes[C]{\smallbreak}
\pstart
           \raggedleft{}{\pb}Wien\oindex{Wien@\textbf{Wien}, \emph{Verwaltungsgebiet}|pw}, 7. Aug. 00.\pend
           \vspace{0.5em}
\pstart
           Lieber, haben Sie meinen \label{K_L03308-1v}\edtext{Brief aus Pressbaum\oindex{Pressbaum@\textbf{Pressbaum}, \emph{Hauptstadt}|pw}}{\lemma{\textnormal{\emph{Brief aus Pressbaum}}}\Cendnote{\textnormal{XXXX Auszeichnungsfehler: Dokument L03307 nicht gefunden.
               }}}\label{K_L03308-1} nicht bekommen? Ich muß nun heute{ }Abend nach Karlsbad\oindex{Karlsbad@\textbf{Karlsbad}|pw} fahren, wodurch
               meine \label{K_L03308-2v}\edtext{Ankunft in Ischl\oindex{Bad Ischl@\textbf{Bad Ischl}|pw}}{\lemma{\textnormal{\emph{Ankunft in Ischl}}}\Cendnote{\textnormal{Salten\pwindex{Salten, Felix 6.\,9.\,1869 Budapest – 8.\,10.\,1945 Zürich@\textsc{Salten, Felix} (6.\,9.\,1869 Budapest – 8.\,10.\,1945 Zürich), \emph{Schriftsteller, Journalist, Chefredakteur}|pwk} kam am XXXX Auszeichnungsfehler: Dokument L03310 nicht gefunden in Ischl\oindex{Bad Ischl@\textbf{Bad Ischl}|pwk} an.}}}\label{K_L03308-2} sich bis Sonntag
               verzögert. Nach Vorarlberg\oindex{Vorarlberg@\textbf{Vorarlberg}|pw} komme ich ganz
               gewiss. Bitte, theilen Sie mir nur immer mit, wo Sie sind. Wenn man so gegen 20. od. 22. in \label{K_L03308-3v}\edtext{Schruns\oindex{Schruns@\textbf{Schruns}, \emph{Verwaltungsgebiet}|pw}}{\lemma{\textnormal{\emph{Schruns}}}\Cendnote{\textnormal{Siehe XXXX Auszeichnungsfehler: Dokument L03307 nicht gefunden.
               }}}\label{K_L03308-3} wäre, das könnte gerade für mich recht sein.\pend
           
\pstart
           Vielleicht ist es möglich, darauf ein wenig Rücksicht zu nehmen.\pend
           
\pstart
           {\pb}Leben Sie recht wol, und
               laßen Sie mir genaue Nachricht zukommen. Am besten \uline{Postlagernd Ischl\oindex{Bad Ischl@\textbf{Bad Ischl}|pw}}.\pend
           
\pstart
           Solle ich Sie Sonntag, wie ich aus dem heutigen Brief
               vermuthe, \label{K_L03308-4v}\edtext{nicht mehr antreffen}{\lemma{\textnormal{\emph{nicht mehr antreffen}}}\Cendnote{\textnormal{Schnitzler reiste am 10. 8. 1900 aus Ischl\oindex{Bad Ischl@\textbf{Bad Ischl}|pwk} ab.}}}\label{K_L03308-4}, so hole ich mir die
               Reisedispositionen von der Post.\pend
           
\pstart
           Auf Wiedersehen da oder dort. {\\[\baselineskip]}Herzlichst {\\[\baselineskip]}\spacefill\mbox{Salten.}\pend
           \leftskip=0em{}\selectlanguage{ngerman}\endnumbering\briefempfaengerindex{Schnitzler, Arthur@\textsc{Schnitzler, Arthur}!zzzSalten, Felix@\emph{von Felix Salten}!1900-08-072@{7. 8. 1900}|)be}\mylabel{L03308h}  \newcommand{\dateiname}{L03308}\newcommand{\titel}{Felix Salten an Arthur Schnitzler, 7. 8. 1900}\newcommand{\editorInnen}{Martin Anton Müller und Laura Untner}%% latex-leseansicht-abspann.tex
%% Abspann für die Leseansicht.
%% Der Schalter \ifkorrekturansicht ist bereits durch den Vorspann gesetzt.

%% latex-abspann.tex
%% Gemeinsamer Abspann für Korrekturansicht und Leseansicht.
%% Setzt den Schalter \ifkorrekturansicht voraus (gesetzt in den
%% einbindenden Dateien latex-korrekturansicht-abspann.tex bzw.
%% latex-leseansicht-abspann.tex).
%% ---------------------------------------------------------------

\normalsize

% Das esempio-Environment wird nur in der Leseansicht benötigt
\ifkorrekturansicht\else
\newenvironment{esempio}[3]%
{
    \vspace{1.5ex}
    \rlap{\underline{#1}}
    \par
    \setlength{\parindent}{0cm}
    \nopagebreak
    \leftskip=#2cm
    \rightskip=#3cm
}
{
    \par
}
\fi

\doendnotes{C}
\bigskip
\vfill

\clearpage

\footnotesize

\ifkorrekturansicht
  \lohead{\textsc{register}}
\fi

% theindex-Environment neu definieren ohne reledmac
\makeatletter
\renewenvironment{theindex}{%
  \ifkorrekturansicht
    \section*{\indexname}%
  \else
    \subsubsection*{Index der erwähnten Entitäten}%
  \fi
  \setlength{\parindent}{0pt}%
  \setlength{\parskip}{0pt plus 0.3pt}%
  \let\item\@idxitem
}{%
  \ifkorrekturansicht\clearpage\fi
}
\makeatother

\IfFileExists{\jobname-pw.ind}{\input{\jobname-pw.ind}}{}

% Quellenangabe nur in der Leseansicht
\ifkorrekturansicht\else
% Fallback-Definitionen, falls die .tex-Datei \titel etc. nicht gesetzt hat
\providecommand{\titel}{}
\providecommand{\editorInnen}{}
\providecommand{\dateiname}{\jobname}

\vspace{3cm}

\vfill

\footnotesize
\textsc{Quelle}: \titel. Herausgegeben von {\editorInnen}. In: \emph{Arthur Schnitzler: Briefwechsel mit Autorinnen und Autoren}.
 Digitale Edition, https://schnitzler-briefe.acdh.oeaw.ac.at/{\dateiname}.html (Stand \today)
\fi

\end{document}


