%% latex-korrekturansicht-vorspann.tex
%% Vorspann für die Korrekturansicht.
%% Lädt die gemeinsame Datei latex-vorspann.tex mit gesetztem Schalter.

\newif\ifkorrekturansicht
\korrekturansichttrue

\input{../tex-inputs/latex-vorspann}


\section[ Felix Salten an Arthur Schnitzler, 7. 8. 1900]{L03308 Felix Salten an Arthur Schnitzler, 7. 8. 1900}
\nopagebreak\mylabel{L03308v}
\rehead{ }\normalsize\beginnumbering\briefempfaengerindex{Schnitzler, Arthur@\textsc{Schnitzler, Arthur}!zzzSalten, Felix@\emph{von Felix Salten}!1900-08-072@{7. 8. 1900}|(be}
\toendnotes[C]{\smallbreak\pagebreak[2]}\Standort{CUL, Schnitzler, B 89, A 2.}
\physDesc{Brief, 1 Blatt, 2 Seiten, 674 Zeichen
\newline{}Handschrift: schwarze Tinte, lateinische Kurrent
\newline{}Ordnung: mit Bleistift von unbekannter Hand nummeriert: »132« }\toendnotes[C]{\smallbreak}
\pstart
           \raggedleft{}{\pb}Wien\oindex{Wien@\textbf{Wien}, \emph{A.ADM2}|pw}, 7. Aug. 00.\pend
           \vspace{0.5em}
\pstart
           Lieber, haben Sie meinen \label{K_L03308-1v}\edtext{Brief aus Pressbaum\oindex{Pressbaum@\textbf{Pressbaum}, \emph{P.PPLA3}|pw}}{\lemma{\textnormal{\emph{Brief aus Pressbaum}}}\Cendnote{\textnormal{Felix Salten an Arthur Schnitzler, 5. 8. 1900.
               }}}\label{K_L03308-1} nicht bekommen? Ich muß nun heute{ }Abend nach Karlsbad\oindex{Karlsbad@\textbf{Karlsbad}, \emph{P.PPLA}|pw} fahren, wodurch
               meine \label{K_L03308-2v}\edtext{Ankunft in Ischl\oindex{Bad Ischl@\textbf{Bad Ischl}, \emph{P.PPL}|pw}}{\lemma{\textnormal{\emph{Ankunft in Ischl}}}\Cendnote{\textnormal{Salten\pwindex{Salten, Felix 06.09.1869 – 08.10.1945@\textsc{Salten, Felix} (06.09.1869 – 08.10.1945), \emph{Schriftsteller/Schriftstellerin, Journalist/Journalistin, Chefredakteur/Chefredakteurin}|pwk} kam am 14. 8. 1900 in Ischl\oindex{Bad Ischl@\textbf{Bad Ischl}, \emph{P.PPL}|pwk} an.}}}\label{K_L03308-2} sich bis Sonntag
               verzögert. Nach Vorarlberg\oindex{Vorarlberg@\textbf{Vorarlberg}, \emph{Teil eines Landes (A.LNDX)}|pw} komme ich ganz
               gewiss. Bitte, theilen Sie mir nur immer mit, wo Sie sind. Wenn man so gegen 20. od. 22. in \label{K_L03308-3v}\edtext{Schruns\oindex{Schruns@\textbf{Schruns}, \emph{A.ADM3}|pw}}{\lemma{\textnormal{\emph{Schruns}}}\Cendnote{\textnormal{Siehe Felix Salten an Arthur Schnitzler, 5. 8. 1900.
               }}}\label{K_L03308-3} wäre, das könnte gerade für mich recht sein.\pend
           
\pstart
           Vielleicht ist es möglich, darauf ein wenig Rücksicht zu nehmen.\pend
           
\pstart
           {\pb}Leben Sie recht wol, und
               laßen Sie mir genaue Nachricht zukommen. Am besten \uline{Postlagernd Ischl\oindex{Bad Ischl@\textbf{Bad Ischl}, \emph{P.PPL}|pw}}.\pend
           
\pstart
           Solle ich Sie Sonntag, wie ich aus dem heutigen Brief
               vermuthe, \label{K_L03308-4v}\edtext{nicht mehr antreffen}{\lemma{\textnormal{\emph{nicht mehr antreffen}}}\Cendnote{\textnormal{Schnitzler reiste am 10. 8. 1900 aus Ischl\oindex{Bad Ischl@\textbf{Bad Ischl}, \emph{P.PPL}|pwk} ab.}}}\label{K_L03308-4}, so hole ich mir die
               Reisedispositionen von der Post.\pend
           
\pstart
           Auf Wiedersehen da oder dort. {\\[\baselineskip]}Herzlichst {\\[\baselineskip]}\spacefill\mbox{Salten.}\pend
           \leftskip=0em{}\selectlanguage{ngerman}\endnumbering\briefempfaengerindex{Schnitzler, Arthur@\textsc{Schnitzler, Arthur}!zzzSalten, Felix@\emph{von Felix Salten}!1900-08-072@{7. 8. 1900}|)be}\mylabel{L03308h}  \normalsize

\doendnotes{C}
\bigskip
\vfill

\clearpage

\footnotesize

\lohead{\textsc{register}}

% Definiere theindex-Environment komplett neu ohne reledmac
\makeatletter
\renewenvironment{theindex}{%
  \section*{\indexname}%
  \setlength{\parindent}{0pt}%
  \setlength{\parskip}{0pt plus 0.3pt}%
  \let\item\@idxitem
}{%
  \clearpage
}
\makeatother

\IfFileExists{\jobname-pw.ind}{\input{\jobname-pw.ind}}{}

\end{document}

      