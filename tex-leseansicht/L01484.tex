%% latex-leseansicht-vorspann.tex
%% Vorspann für die Leseansicht.
%% Lädt die gemeinsame Datei latex-vorspann.tex mit nicht gesetztem Schalter.

\newif\ifkorrekturansicht
\korrekturansichtfalse

\input{../tex-inputs/latex-vorspann}


\section[Arthur und Olga Schnitzler an Richard Beer-Hofmann, 28. 12. 1904]{L01484 Arthur und Olga Schnitzler an Richard Beer-Hofmann, 28. 12. 1904}
\nopagebreak\mylabel{L01484v}
\rehead{ }\normalsize\beginnumbering\briefempfaengerindex{Beer-Hofmann, Richard@\textsc{Beer-Hofmann, Richard}!zzzSchnitzler, Olga@\emph{von Olga Schnitzler}!1904-12-282@{28. 12. 1904}|(be}\briefempfaengerindex{Beer-Hofmann, Richard@\textsc{Beer-Hofmann, Richard}!zzzSchnitzler, Arthur@\emph{von Arthur Schnitzler}!1904-12-282@{28. 12. 1904}|(be}
\toendnotes[C]{\smallbreak\pagebreak[2]}
\correspDesc{Versand  durch Arthur Schnitzler, Olga Schnitzler am 28. 12. 1904 in St. Gilgen
\newline{}Erhalt  durch Richard Beer-Hofmann am 29. 12. 1904 in Rodaun}\toendnotes[C]{\smallbreak}
\Standort{YCGL, MSS 31.}
\physDesc{Bildpostkarte, 201 Zeichen
\newline{}Handschrift Arthur Schnitzler: Bleistift, deutsche Kurrent
\newline{}Handschrift Olga Schnitzler: Bleistift
\newline{}Versand: 1) Stempel: »\nobreak{}\oindex{St. Gilgen@\textbf{St. Gilgen}, \emph{Verwaltungsgebiet}|pwk}St.
                                       Gil{[}ge{]}n, 28 12 \textcolor{gray}{04}\nobreak{}«.   2) Stempel: »\nobreak{}\oindex{Wien@\textbf{Wien}!XXIII., Liesing@\textbf{XXIII., Liesing}!Rodaun@\textbf{Rodaun}, \emph{Region}|pwk}Ro{[}dau{]}n, 29 12 04, 9{[}–{]}12V\nobreak{}«. }\toendnotes[C]{\smallbreak}\pstart{}{\pb}\textsc{Herrn Dr. Richard Beer-Hofmann}\pend{}\pstart{}\textsc{Rodaun\oindex{Wien@\textbf{Wien}!XXIII., Liesing@\textbf{XXIII., Liesing}!Rodaun@\textbf{Rodaun}, \emph{Region}|pw}}{ }\introOben{}\textsc{bei}{ }Wien\oindex{Wien@\textbf{Wien}, \emph{Verwaltungsgebiet}|pw}, Südbahn\orgindex{Südbahnstrecke@Südbahnstrecke|pw}\introOben{}\pend{}\pstart{}\textsc{Liesingerstraße 2}\oindex{Liesingerstraße@\textbf{Liesingerstraße}, \emph{Straße}|pw}\pend{}{\bigskip}
\pstart
           \noindent{}\centering{}{\pb}\textcolor{gray}{\textbf{St. Gilgen\oindex{St. Gilgen@\textbf{St. Gilgen}, \emph{Verwaltungsgebiet}|pw}}}\pend
           \vspace{1em}
\pstart
           {\pb}\textsc{Lueg}\oindex{St. Gilgen@\textbf{St. Gilgen}, \emph{Verwaltungsgebiet}|pw}, 28. 12. 904.\pend
           \vspace{0.5em}
\pstart
           \label{K_L01484-1v}\edtext{Montag}{\lemma{\textnormal{\emph{Montag}}}\Cendnote{\textnormal{Siehe A. S.: \emph{Tagebuch}, 2. 1. 1905.
               }}}\label{K_L01484-1}{ }Abend den 2.{ }ſind wir (hoffentlich) in Hietzing\oindex{XIII., Hietzing@\textbf{XIII., Hietzing}, \emph{Verwaltungsgebiet}|pw} bei Kuffner\oindex{Wien@\textbf{Wien}!XIII., Hietzing@\textbf{XIII., Hietzing}!Ottakringer Bräu@\textbf{Ottakringer Bräu}, \emph{Bierhaus}|pw}. Möchte
               Sie gern bald{ }ſehen\pend
           
\pstart
           Ihr{\\[\baselineskip]}\spacefill\mbox{A.}\pend
           \leftskip=0em{}\selectlanguage{ngerman}\vspace{1em}
\pstart
           {[}hs. Schnitzler:{]} Herzliche Grüße! {\\[\baselineskip]}\spacefill\mbox{Olga.}\pend
           \leftskip=0em{}\selectlanguage{ngerman}\endnumbering\briefempfaengerindex{Beer-Hofmann, Richard@\textsc{Beer-Hofmann, Richard}!zzzSchnitzler, Olga@\emph{von Olga Schnitzler}!1904-12-282@{28. 12. 1904}|)be}\briefempfaengerindex{Beer-Hofmann, Richard@\textsc{Beer-Hofmann, Richard}!zzzSchnitzler, Arthur@\emph{von Arthur Schnitzler}!1904-12-282@{28. 12. 1904}|)be}\mylabel{L01484h}  \newcommand{\dateiname}{L01484}\newcommand{\titel}{Arthur und Olga Schnitzler an Richard Beer-Hofmann, 28. 12. 1904}\newcommand{\editorInnen}{Martin Anton Müller und Gerd-Hermann Susen}%% latex-leseansicht-abspann.tex
%% Abspann für die Leseansicht.
%% Der Schalter \ifkorrekturansicht ist bereits durch den Vorspann gesetzt.

%% latex-abspann.tex
%% Gemeinsamer Abspann für Korrekturansicht und Leseansicht.
%% Setzt den Schalter \ifkorrekturansicht voraus (gesetzt in den
%% einbindenden Dateien latex-korrekturansicht-abspann.tex bzw.
%% latex-leseansicht-abspann.tex).
%% ---------------------------------------------------------------

\normalsize

% Das esempio-Environment wird nur in der Leseansicht benötigt
\ifkorrekturansicht\else
\newenvironment{esempio}[3]%
{
    \vspace{1.5ex}
    \rlap{\underline{#1}}
    \par
    \setlength{\parindent}{0cm}
    \nopagebreak
    \leftskip=#2cm
    \rightskip=#3cm
}
{
    \par
}
\fi

\doendnotes{C}
\bigskip
\vfill

\clearpage

\footnotesize

\ifkorrekturansicht
  \lohead{\textsc{register}}
\fi

% theindex-Environment neu definieren ohne reledmac
\makeatletter
\renewenvironment{theindex}{%
  \ifkorrekturansicht
    \section*{\indexname}%
  \else
    \subsubsection*{Index der erwähnten Entitäten}%
  \fi
  \setlength{\parindent}{0pt}%
  \setlength{\parskip}{0pt plus 0.3pt}%
  \let\item\@idxitem
}{%
  \ifkorrekturansicht\clearpage\fi
}
\makeatother

\IfFileExists{\jobname-pw.ind}{\input{\jobname-pw.ind}}{}

% Quellenangabe nur in der Leseansicht
\ifkorrekturansicht\else
% Fallback-Definitionen, falls die .tex-Datei \titel etc. nicht gesetzt hat
\providecommand{\titel}{}
\providecommand{\editorInnen}{}
\providecommand{\dateiname}{\jobname}

\vspace{3cm}

\vfill

\footnotesize
\textsc{Quelle}: \titel. Herausgegeben von {\editorInnen}. In: \emph{Arthur Schnitzler: Briefwechsel mit Autorinnen und Autoren}.
 Digitale Edition, https://schnitzler-briefe.acdh.oeaw.ac.at/{\dateiname}.html (Stand \today)
\fi

\end{document}


