\input{../tex-inputs/latex-pdf-vorspann}
\begin{center}
            \textcolor{red}{ENTWURF. ENTZIFFERUNG NOCH NICHT KORREKTURGELESEN}
                      \end{center}
            
               \section[Arthur und Olga Schnitzler an Richard Beer-Hofmann, 28. 12. 1904]{ Arthur und Olga Schnitzler an Richard Beer-Hofmann,
               28. 12. 1904}\nopagebreak\mylabel{v}\rehead{ }\begin{ledgroupsized}[t]{13cm}\normalsize\beginnumbering\briefempfaengerindex{Beer-Hofmann, Richard@\textsc{Beer-Hofmann, Richard}!zzzSchnitzler, Olga@\emph{von Olga Schnitzler}!1904-12-282@{28. 12. 1904}|(be}\briefempfaengerindex{Beer-Hofmann, Richard@\textsc{Beer-Hofmann, Richard}!zzzSchnitzler, Arthur@\emph{von Arthur Schnitzler}!1904-12-282@{28. 12. 1904}|(be} \toendnotes[C]{\smallbreak\pagebreak[2]} \Standort{YCGL, MSS 31.}
\physDesc{Bildpostkarte
\newline{}Handschrift Arthur Schnitzler: Bleistift, deutsche Kurrent\newline{}Handschrift Olga Schnitzler: Bleistift\newline{}Versand: 1) Stempel: »\nobreak{}\oindex{St. Gilgen@\textbf{St. Gilgen}|pwk}St. Gil{[}ge{]}n, 28 12 \textcolor{gray}{04}\nobreak{}«.  2) Stempel: »\nobreak{}\oindex{Rodaun@\textbf{Rodaun}|pwk}Ro{[}dau{]}n, 29 12 04, 9{[}–{]}12V\nobreak{}«. }\toendnotes[C]{\smallbreak}\pstart{}{\pb}\textsc{Herrn Dr. Richard Beer-Hofmann}\pend{}\pstart{}\textsc{Rodaun\oindex{Rodaun@\textbf{Rodaun}|pw}}{ }\introOben{}\textsc{bei}{ }Wien\oindex{Wien@\textbf{Wien}|pw}, Südbahn\orgindex{Suedbahnstrecke@Südbahnstrecke|pw}\introOben{}\pend{}\pstart{}\textsc{Liesingerstraße 2}\oindex{Liesingerstrasse@\textbf{Liesingerstraße}|pw}\pend{}{\bigskip}\pstart
           \noindent{}\centering{}\textcolor{gray}{\textbf{{\pb}St. Gilgen\oindex{St. Gilgen@\textbf{St. Gilgen}|pw}}}\pend
           \pstart
           \textsc{Lueg}\oindex{St. Gilgen@\textbf{St. Gilgen}|pw}, 28. 12. 904.\pend
           \pstart
           \label{K_L01484_1v}\edtext{Montag}{\lemma{\textnormal{\emph{Montag}}}\Cendnote{\textnormal{siehe A. S.: \emph{Tagebuch}, 2. 1. 1905}}}\label{K_L01484_1h}{ }Abend den 2. ſind wir (hoffentlich) in Hietzing\oindex{XIII., Hietzing@\textbf{XIII., Hietzing}|pw} bei Kuffner\oindex{Ottakringer Braeu@\textbf{Ottakringer Bräu}|pw}. Möchte Sie
               gern bald ſehen\pend
           \pstart
           Ihr{\\[\baselineskip]}\spacefill\mbox{A.}\pend
           \leftskip=0em{}\pstart
           {[}hs. O. Schnitzler:{]} Herzliche Grüße! {\\[\baselineskip]}\spacefill\mbox{Olga.}\pend
           \leftskip=0em{}\endnumbering\briefempfaengerindex{Beer-Hofmann, Richard@\textsc{Beer-Hofmann, Richard}!zzzSchnitzler, Olga@\emph{von Olga Schnitzler}!1904-12-282@{28. 12. 1904}|)be}\briefempfaengerindex{Beer-Hofmann, Richard@\textsc{Beer-Hofmann, Richard}!zzzSchnitzler, Arthur@\emph{von Arthur Schnitzler}!1904-12-282@{28. 12. 1904}|)be}\mylabel{h}\end{ledgroupsized}  \newcommand{\dateiname}{L01484}\newcommand{\titel}{Arthur und Olga Schnitzler an Richard Beer-Hofmann, 28. 12. 1904}\newcommand{\editorInnen}{Martin Anton Müller und Gerd-Hermann Susen}\input{../tex-inputs/latex-pdf-abspann}
      