%% latex-korrekturansicht-vorspann.tex
%% Vorspann für die Korrekturansicht.
%% Lädt die gemeinsame Datei latex-vorspann.tex mit gesetztem Schalter.

\newif\ifkorrekturansicht
\korrekturansichttrue

\input{../tex-inputs/latex-vorspann}


\section[Joseph Victor Widmann an Arthur Schnitzler, 28. 5. 1900]{L01044 Joseph Victor Widmann an Arthur Schnitzler, 28. 5. 1900}
\nopagebreak\mylabel{L01044v}
\rehead{ }\normalsize\beginnumbering\briefempfaengerindex{Schnitzler, Arthur@\textsc{Schnitzler, Arthur}!zzzWidmann, Joseph Victor@\emph{von Joseph Victor Widmann}!1900-05-283@{28. 5. 1900}|(be}
\toendnotes[C]{\smallbreak\pagebreak[2]}\Standort{TMW, HS Schn 4/104/1.}
\physDesc{Brief, 1 Blatt, 3 Seiten, 1031 Zeichen
\newline{}Handschrift: schwarze Tinte, deutsche Kurrent
\newline{}Schnitzler: 1) mit Bleistift beschriftet: »Widmann«  2) mit rotem Buntstift eine Unterstreichung}\toendnotes[C]{\smallbreak}
\pstart
           \raggedleft{}{\pb}\textsc{Bern\oindex{Bern@\textbf{Bern}, \emph{P.PPLC}|pw}, 28. Mai 1900}.\pend
           
\pstart{}Verehrter Herr!\pend\vspace{0.5em}
\pstart
           Erſt geſtern habe ich über allerlei Rezenſionsbüchervolk Ihren »Reigen\pwindex{Reigen. Zehn Dialoge@\emph{Reigen. Zehn Dialoge}|pw}« und in dem kleinen Buche die große Liebenswürdigkeit
               entdeckt, die in einer ſo auszeichnenden perſönlichen {\pb}Sendung und Widmung
               eines als \textsc{Manuscript} gedruckten Werkes liegt.\pend
           
\pstart
           Und wie gut ich mich dann nachher mit dem aus ſo echter Menſchenkenntniß geſchöpften,
               feinen Buche unterhalten habe, das wird Ihnen der Bewunderer\pwindex{Kunst und Litteratur@\emph{Kunst und Litteratur}|pwv} Ihrer \textsc{Anatole\pwindex{Anatol@\emph{Anatol}|pw}}-Dialoge nicht erſt zu verſichern brauchen.\pend
           
\pstart
           Ich beglückwünſche Sie zu dem poetiſchen Einfall eines solchen Venusreigens, bei dem
               der komiſche Plumpsack, den wir alle ke{\geminationn}en, von einer
               Hand in die andere gleitet. Wir ſind da wieder bei der freien Kunſt angelangt, wie
               wir ſie aus fröhlichen Bildern des alten Pompeji\oindex{Pompeji@\textbf{Pompeji}, \emph{S.ANS}|pw}
               kennen. Und wie Ihr Soldat\pwindex{Reigen. Zehn Dialoge@\emph{Reigen. Zehn Dialoge}|pwv} zum
                  Stubenmädchen\pwindex{Reigen. Zehn Dialoge@\emph{Reigen. Zehn Dialoge}|pwv} dürfen Sie in
               dieſen manchmal {\pb}etwas
               dunkeln Zeiten zu Ihrer Mühe ſagen: »Gott ſei Dank! Mir sein mir!«\pend
           
\pstart
           Seien Sie alſo ſchönſtens bedankt für Ihr Buch\pwindex{Reigen. Zehn Dialoge@\emph{Reigen. Zehn Dialoge}|pwv} u. für die Ehre, die Sie mir mit der Zuſendung erwieſen
               haben.\pend
           
\pstart
           In herzlicher Verehrung{\\[\baselineskip]}Ihr{\\[\baselineskip]}\spacefill\mbox{J. V. Widmann}\pend
           \leftskip=0em{}\selectlanguage{ngerman}\endnumbering\briefempfaengerindex{Schnitzler, Arthur@\textsc{Schnitzler, Arthur}!zzzWidmann, Joseph Victor@\emph{von Joseph Victor Widmann}!1900-05-283@{28. 5. 1900}|)be}\mylabel{L01044h}  \normalsize

\doendnotes{C}
\bigskip
\vfill

\clearpage

\footnotesize

\lohead{\textsc{register}}

% Definiere theindex-Environment komplett neu ohne reledmac
\makeatletter
\renewenvironment{theindex}{%
  \section*{\indexname}%
  \setlength{\parindent}{0pt}%
  \setlength{\parskip}{0pt plus 0.3pt}%
  \let\item\@idxitem
}{%
  \clearpage
}
\makeatother

\IfFileExists{\jobname-pw.ind}{\input{\jobname-pw.ind}}{}

\end{document}

      