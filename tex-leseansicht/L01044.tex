%% latex-leseansicht-vorspann.tex
%% Vorspann für die Leseansicht.
%% Lädt die gemeinsame Datei latex-vorspann.tex mit nicht gesetztem Schalter.

\newif\ifkorrekturansicht
\korrekturansichtfalse

\input{../tex-inputs/latex-vorspann}


         
         \newcommand{\erwaehntePersonen}{Personen: }
         \newcommand{\erwaehnteInstitutionen}{}
         \newcommand{\erwaehnteOrte}{Orte: Bern, Pompei, Wien}
         \newcommand{\erwaehnteWerke}{Werke: Anatol, Kunst und Litteratur, Reigen. Zehn Dialoge}
               \section[Joseph Victor Widmann an Arthur Schnitzler, 28. 5. 1900]{ Joseph Victor Widmann an Arthur Schnitzler, 28. 5. 1900}\nopagebreak\mylabel{v}\rehead{ }\begin{ledgroupsized}[t]{13cm}\normalsize\beginnumbering \toendnotes[C]{\smallbreak\pagebreak[2]} \Standort{TMW, HS Schn 4/104/1.}
\physDesc{Brief, 1 Blatt, 3 Seiten
\newline{}Handschrift: schwarze Tinte, deutsche Kurrent
\newline{}Schnitzler: 1) mit Bleistift beschriftet: »Widmann«  2) mit rotem Buntstift eine
            Unterstreichung}\toendnotes[C]{\smallbreak}\pstart
           \raggedleft{}{\pb}\textsc{Bern\oindex{Bern@\textbf{Bern}|pw}, 28. Mai 1900}.\pend
           \pstart{}Verehrter Herr!\pend\pstart
           Erſt geſtern habe ich über allerlei Rezenſionsbüchervolk Ihren »Reigen\pwindex{Schnitzler, Arthur 15.05.1862 – 21.10.1931@\textsc{Schnitzler, Arthur} (15.05.1862 – 21.10.1931), \emph{Schriftsteller, Mediziner}!Reigen. Zehn Dialoge1900@\strich\emph{Reigen. Zehn Dialoge} {[}1900{]}|pw}« und in dem kleinen Buche die große
                    Liebenswürdigkeit entdeckt, die in einer ſo auszeichnenden perſönlichen {\pb}Sendung und Widmung eines als \textsc{Manuscript} gedruckten Werkes liegt.\pend
           \pstart
           Und wie gut ich mich dann nachher mit dem aus ſo echter Menſchenkenntniß
                    geſchöpften, feinen Buche unterhalten habe, das wird Ihnen der Bewunderer\pwindex{Kunst und Litteratur12.02.1893 – 12.2.1893@\emph{Kunst und Litteratur} {[}12.02.1893 – 12.2.1893{]}|pwv} Ihrer \textsc{Anatole\pwindex{Schnitzler, Arthur 15.05.1862 – 21.10.1931@\textsc{Schnitzler, Arthur} (15.05.1862 – 21.10.1931), \emph{Schriftsteller, Mediziner}!Anatol1892-10-29@\strich\emph{Anatol} {[}1892-10-29{]}|pw}}-Dialoge nicht erſt zu verſichern brauchen.\pend
           \pstart
           Ich beglückwünſche Sie zu dem poetiſchen Einfall eines solchen Venusreigens, bei
                    dem der komiſche Plumpsack, den wir alle ke{\geminationn}en, von
                    einer Hand in die andere gleitet. Wir ſind da wieder bei der freien Kunſt
                    angelangt, wie wir ſie aus fröhlichen Bildern des alten Pompeji\oindex{Pompei@\textbf{Pompei}|pw} kennen. Und wie Ihr Soldat\pwindex{Schnitzler, Arthur 15.05.1862 – 21.10.1931@\textsc{Schnitzler, Arthur} (15.05.1862 – 21.10.1931), \emph{Schriftsteller, Mediziner}!Reigen. Zehn Dialoge1900@\strich\emph{Reigen. Zehn Dialoge} {[}1900{]}|pwv} zum Stubenmädchen\pwindex{Schnitzler, Arthur 15.05.1862 – 21.10.1931@\textsc{Schnitzler, Arthur} (15.05.1862 – 21.10.1931), \emph{Schriftsteller, Mediziner}!Reigen. Zehn Dialoge1900@\strich\emph{Reigen. Zehn Dialoge} {[}1900{]}|pwv} dürfen Sie in dieſen
                    manchmal {\pb}etwas dunkeln Zeiten zu Ihrer Mühe
                    ſagen: »Gott ſei Dank! Mir sein mir!«\pend
           \pstart
           Seien Sie alſo ſchönſtens bedankt für Ihr Buch\pwindex{Schnitzler, Arthur 15.05.1862 – 21.10.1931@\textsc{Schnitzler, Arthur} (15.05.1862 – 21.10.1931), \emph{Schriftsteller, Mediziner}!Reigen. Zehn Dialoge1900@\strich\emph{Reigen. Zehn Dialoge} {[}1900{]}|pwv} u. für die Ehre, die Sie mir mit der Zuſendung
                    erwieſen haben.\pend
           \pstart
           In herzlicher Verehrung{\\[\baselineskip]}Ihr{\\[\baselineskip]}\spacefill\mbox{J. V. Widmann}\pend
           \leftskip=0em{}
         
         \endnumbering\mylabel{h}\end{ledgroupsized}  \newcommand{\dateiname}{L01044}\newcommand{\titel}{Joseph Victor Widmann an Arthur Schnitzler, 28. 5. 1900}\newcommand{\editorInnen}{Martin Anton Müller und Gerd-Hermann Susen}%% latex-leseansicht-abspann.tex
%% Abspann für die Leseansicht.
%% Der Schalter \ifkorrekturansicht ist bereits durch den Vorspann gesetzt.

%% latex-abspann.tex
%% Gemeinsamer Abspann für Korrekturansicht und Leseansicht.
%% Setzt den Schalter \ifkorrekturansicht voraus (gesetzt in den
%% einbindenden Dateien latex-korrekturansicht-abspann.tex bzw.
%% latex-leseansicht-abspann.tex).
%% ---------------------------------------------------------------

\normalsize

% Das esempio-Environment wird nur in der Leseansicht benötigt
\ifkorrekturansicht\else
\newenvironment{esempio}[3]%
{
    \vspace{1.5ex}
    \rlap{\underline{#1}}
    \par
    \setlength{\parindent}{0cm}
    \nopagebreak
    \leftskip=#2cm
    \rightskip=#3cm
}
{
    \par
}
\fi

\doendnotes{C}
\bigskip
\vfill

\clearpage

\footnotesize

\ifkorrekturansicht
  \lohead{\textsc{register}}
\fi

% theindex-Environment neu definieren ohne reledmac
\makeatletter
\renewenvironment{theindex}{%
  \ifkorrekturansicht
    \section*{\indexname}%
  \else
    \subsubsection*{Index der erwähnten Entitäten}%
  \fi
  \setlength{\parindent}{0pt}%
  \setlength{\parskip}{0pt plus 0.3pt}%
  \let\item\@idxitem
}{%
  \ifkorrekturansicht\clearpage\fi
}
\makeatother

\IfFileExists{\jobname-pw.ind}{\input{\jobname-pw.ind}}{}

% Quellenangabe nur in der Leseansicht
\ifkorrekturansicht\else
% Fallback-Definitionen, falls die .tex-Datei \titel etc. nicht gesetzt hat
\providecommand{\titel}{}
\providecommand{\editorInnen}{}
\providecommand{\dateiname}{\jobname}

\vspace{3cm}

\vfill

\footnotesize
\textsc{Quelle}: \titel. Herausgegeben von {\editorInnen}. In: \emph{Arthur Schnitzler: Briefwechsel mit Autorinnen und Autoren}.
 Digitale Edition, https://schnitzler-briefe.acdh.oeaw.ac.at/{\dateiname}.html (Stand \today)
\fi

\end{document}


      