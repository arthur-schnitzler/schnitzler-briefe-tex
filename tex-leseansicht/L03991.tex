%% latex-leseansicht-vorspann.tex
%% Vorspann für die Leseansicht.
%% Lädt die gemeinsame Datei latex-vorspann.tex mit nicht gesetztem Schalter.

\newif\ifkorrekturansicht
\korrekturansichtfalse

\input{../tex-inputs/latex-vorspann}


\section[Arthur Schnitzler: Widmungsexemplar von Casanovas Heimfahrt an Berta Zuckerkandl, Jänner 1919]{L03991 Arthur Schnitzler: Widmungsexemplar von Casanovas Heimfahrt an Berta
               Zuckerkandl, Jänner 1919}
\nopagebreak\mylabel{L03991v}
\rehead{ }\normalsize\beginnumbering\briefempfaengerindex{Zuckerkandl, Berta@\textsc{Zuckerkandl, Berta}!zzzSchnitzler, Arthur@\emph{von Arthur Schnitzler}!1919-01-311@{Jänner 1919}|(be}
\toendnotes[C]{\smallbreak\pagebreak[2]}
\correspDesc{Versand  durch Arthur Schnitzler im Zeitraum Jänner 1919 in Wien
\newline{}Erhalt  durch Berta Zuckerkandl im Zeitraum Jänner 1919 in Wien}\toendnotes[C]{\smallbreak}
\Standort{Wien, Österreichische Nationalbibliothek, ZUC.5.1.SchCas LIT MAG.}
\physDesc{Widmung am Schmutztitel, 125 Zeichen
\newline{}Handschrift: schwarze Tinte, lateinische Kurrent}\toendnotes[C]{\smallbreak}
\pstart
           {\pb}{[}S. Fischer Verlag\orgindex{S. Fischer Verlag@S. Fischer Verlag|pw}{]}\hfill {[}S. Fischer Verlag\orgindex{S. Fischer Verlag – Filiale Wien@S. Fischer Verlag – Filiale Wien|pw}{]}\pend
           
\pstart
           \textcolor{gray}{\textbf{BERLIN\oindex{Berlin@\textbf{Berlin}, \emph{Hauptstadt}|pw}}}\hfill \textcolor{gray}{\textbf{WIEN\oindex{Wien@\textbf{Wien}, \emph{Verwaltungsgebiet}|pw}}}\pend
           \vspace{0.5em}
\pstart
           Frau Hofrätin Bertha Zuckerkandl\pend
           \pstart in herzlicher Verehrung \spacefill\mbox{ArthurSchnitzler}\pend{}
\pstart
           Wien\oindex{Wien@\textbf{Wien}, \emph{Verwaltungsgebiet}|pw}, \label{K_L03990-1v}\edtext{Januar 1919}{\lemma{\textnormal{\emph{Januar 1919}}}\Cendnote{\textnormal{Weil Papiermangel herrschte, betreute in den Monaten nach
                     der Erstausgabe die Wiener\oindex{Wien@\textbf{Wien}, \emph{Verwaltungsgebiet}|pwk} Filiale des \emph{S.-Fischer-Verlags}\orgindex{S. Fischer Verlag – Filiale Wien@S. Fischer Verlag – Filiale Wien|pwk} die weiteren
                     Auflagen.}}}\label{K_L03990-1}.\pend
           {\vspace{1\baselineskip}}\selectlanguage{ngerman}\vspace{1em}{\vspace{1\baselineskip}}
\pstart
           \centering{}{\pb}\textcolor{gray}{\textbf{CASANOVAS HEIMFAHRT\pwindex{Schnitzler, Arthur 15.\,5.\,1862 Wien – 21.\,10.\,1931 ebd.@\textsc{Schnitzler, Arthur} (15.\,5.\,1862 Wien – 21.\,10.\,1931 ebd.), \emph{Schriftsteller, Mediziner}!Casanovas Heimfahrt@\strich\emph{Casanovas Heimfahrt}|pw}}}\pend
           
\pstart
           \centering{}\textcolor{gray}{\textbf{NOVELLE}}\pend
           
\pstart
           \centering{}\textcolor{gray}{\textbf{VON}}\pend
           
\pstart
           \centering{}\textcolor{gray}{\textbf{ARTHUR SCHNITZLER}}\pend
           {\vspace{1\baselineskip}}
\pstart
           \centering{}\textcolor{gray}{\textbf{1919}}\pend
           
\pstart
           \centering{}\textcolor{gray}{\textbf{S. FISCHER, VERLAG\orgindex{S. Fischer Verlag@S. Fischer Verlag|pw}, BERLIN\oindex{Berlin@\textbf{Berlin}, \emph{Hauptstadt}|pw}}}\pend
           \selectlanguage{ngerman}\endnumbering\briefempfaengerindex{Zuckerkandl, Berta@\textsc{Zuckerkandl, Berta}!zzzSchnitzler, Arthur@\emph{von Arthur Schnitzler}!1919-01-011@{Jänner 1919}|)be}\mylabel{L03991h}
\begin{anhang}
\end{anhang}\newcommand{\dateiname}{L03991}\newcommand{\titel}{Arthur Schnitzler: Widmungsexemplar von Casanovas Heimfahrt an Berta Zuckerkandl, Jänner 1919}\newcommand{\editorInnen}{Herausgegeben von Jahnke, SelmaMüller, Martin Anton}%% latex-leseansicht-abspann.tex
%% Abspann für die Leseansicht.
%% Der Schalter \ifkorrekturansicht ist bereits durch den Vorspann gesetzt.

%% latex-abspann.tex
%% Gemeinsamer Abspann für Korrekturansicht und Leseansicht.
%% Setzt den Schalter \ifkorrekturansicht voraus (gesetzt in den
%% einbindenden Dateien latex-korrekturansicht-abspann.tex bzw.
%% latex-leseansicht-abspann.tex).
%% ---------------------------------------------------------------

\normalsize

% Das esempio-Environment wird nur in der Leseansicht benötigt
\ifkorrekturansicht\else
\newenvironment{esempio}[3]%
{
    \vspace{1.5ex}
    \rlap{\underline{#1}}
    \par
    \setlength{\parindent}{0cm}
    \nopagebreak
    \leftskip=#2cm
    \rightskip=#3cm
}
{
    \par
}
\fi

\doendnotes{C}
\bigskip
\vfill

\clearpage

\footnotesize

\ifkorrekturansicht
  \lohead{\textsc{register}}
\fi

% theindex-Environment neu definieren ohne reledmac
\makeatletter
\renewenvironment{theindex}{%
  \ifkorrekturansicht
    \section*{\indexname}%
  \else
    \subsubsection*{Index der erwähnten Entitäten}%
  \fi
  \setlength{\parindent}{0pt}%
  \setlength{\parskip}{0pt plus 0.3pt}%
  \let\item\@idxitem
}{%
  \ifkorrekturansicht\clearpage\fi
}
\makeatother

\IfFileExists{\jobname-pw.ind}{\input{\jobname-pw.ind}}{}

% Quellenangabe nur in der Leseansicht
\ifkorrekturansicht\else
% Fallback-Definitionen, falls die .tex-Datei \titel etc. nicht gesetzt hat
\providecommand{\titel}{}
\providecommand{\editorInnen}{}
\providecommand{\dateiname}{\jobname}

\vspace{3cm}

\vfill

\footnotesize
\textsc{Quelle}: \titel. Herausgegeben von {\editorInnen}. In: \emph{Arthur Schnitzler: Briefwechsel mit Autorinnen und Autoren}.
 Digitale Edition, https://schnitzler-briefe.acdh.oeaw.ac.at/{\dateiname}.html (Stand \today)
\fi

\end{document}


