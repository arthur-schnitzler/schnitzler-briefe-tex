%% latex-korrekturansicht-vorspann.tex
%% Vorspann für die Korrekturansicht.
%% Lädt die gemeinsame Datei latex-vorspann.tex mit gesetztem Schalter.

\newif\ifkorrekturansicht
\korrekturansichttrue

\input{../tex-inputs/latex-vorspann}


\section[Hugo August von Hofmannsthal an Arthur Schnitzler, 5. 12. 1904]{L01476 Hugo August von Hofmannsthal an Arthur Schnitzler, 5. 12. 1904}
\nopagebreak\mylabel{L01476v}
\rehead{ }\normalsize\beginnumbering\briefempfaengerindex{Schnitzler, Arthur@\textsc{Schnitzler, Arthur}!zzzHofmannsthal, Hugo August von@\emph{von Hugo August von Hofmannsthal}!1904-12-052@{5. 12. 1904}|(be}
\toendnotes[C]{\smallbreak\pagebreak[2]}\Standort{DLA, A:Schnitzler, HS.NZ85.1.3483.}
\physDesc{Brief, 1 Blatt, 2 Seiten, 666 Zeichen (Briefpapier mit Trauerrand)
\newline{}Handschrift: schwarze Tinte, deutsche Kurrent
\newline{}Schnitzler: mit rotem Buntstift beschriftet: »\textsc{(Hugos\pwindex{Hofmannsthal, Hugo von 1874-02-01 – 1929-07-15@\textsc{Hofmannsthal, Hugo von} (1874-02-01 – 1929-07-15), \emph{Schriftsteller/Schriftstellerin}|pw} Vater)}« }\toendnotes[C]{\smallbreak}
\pstart
           \raggedleft{}{\pb}Wien\oindex{Wien@\textbf{Wien}, \emph{A.ADM2}|pw} den 5 December.{\\}1904\pend
           
\pstart{}Geehrter Freund!\pend\vspace{0.5em}
\pstart
           Ich beeile mich Ihnen mitzuteilen, dß ich mich meiner \textsc{diplomatischen Miſsionen} betreffs der \textsc{Tantième} von
               der \label{K_L01476-1v}\edtext{Woltätigkeitsvorſtellung}{\lemma{\textnormal{\emph{Woltätigkeitsvorſtellung}}}\Cendnote{\textnormal{Es handelt sich um den am 12. 12. 1904
                  stattfindenden »Arthur-Schnitzler-Abend« im Carl-Theater\oindex{Carl-Theater@\textbf{Carl-Theater}, \emph{Theater (K.THE)}|pwk}. Dieser wurde für das seit 1787 bestehende \emph{Erste öffentliche Kinderkrankeninstitut}\orgindex{Erstes oeffentliches Kinderkrankeninstitut@Erstes öffentliches Kinderkrankeninstitut|pwk}
                  abgehalten, dessen Leitung Carl Hochsinger\pwindex{Hochsinger, Carl 12.07.1860 – 28.10.1942@\textsc{Hochsinger, Carl} (12.07.1860 – 28.10.1942), \emph{Pädiater/Pädiaterin}|pwk}
                  innehatte.}}}\label{K_L01476-1}, geſtern pflichtgemäß entledigt habe. Die \textsc{Arrangeure} waren ſehr erſchüttert, weil ſie natürlich an den Dichter, der ja
               bekanntlich von der Luft zu leben verpflichtet iſt, nicht gedacht hatten, aber ich
               habe \strikeout{pf} ihnen den \textcolor{gray}{Stand}pönal klar
               gemacht. Baron Haas\pwindex{Haas-Teichen, Philipp von 15.11.1859 – 26.02.1926@\textsc{Haas-Teichen, Philipp von} (15.11.1859 – 26.02.1926), \emph{Schriftsteller/Schriftstellerin, Industrieller/Industrielle, Dramatiker/Dramatikerin}|pw} hat wegen des Ablebens
               ſeines {\pb}Schwagers \textsc{Grfen Castell\pwindex{Castell-Ruedenhausen, Wilhelm zu 1841-08-13 – 1904-12-03@\textsc{Castell-Rüdenhausen, Wilhelm zu} (1841-08-13 – 1904-12-03)|pw}} abſagen müſſen u D\textsuperscript{r}{ }\textsc{Hochsinger}\pwindex{Hochsinger, Carl 12.07.1860 – 28.10.1942@\textsc{Hochsinger, Carl} (12.07.1860 – 28.10.1942), \emph{Pädiater/Pädiaterin}|pw} iſt bemüht mit Hilfe \textsc{Heines\pwindex{Heine, Albert 16.11.1867 – 13.4.1949@\textsc{Heine, Albert} (16.11.1867 – 13.4.1949), \emph{Theaterleiter/Theaterleiterin, Schauspieler/Schauspielerin}|pw}} u \textsc{Treßler}\pwindex{Tressler, Otto 13.04.1871 – 27.04.1965@\textsc{Tressler, Otto} (13.04.1871 – 27.04.1965), \emph{Schauspieler/Schauspielerin, Bildhauer/Bildhauerin}|pw} einen paſſenden Erſatz zu finden.\pend
           
\pstart
           Empfehlen Sie mich gütigſt Ihrer Gnädigen\pwindex{Schnitzler, Olga 17.01.1882 – 13.01.1970@\textsc{Schnitzler, Olga} (17.01.1882 – 13.01.1970), \emph{Schauspieler/Schauspielerin, Sänger/Sängerin}|pwv} und ſein Sie beſtens gegrüßt von Ihrem\pend
           
\pstart
           ergebenſten{\\[\baselineskip]}\spacefill\mbox{D\textsuperscript{r} Hofmannsthal}\pend
           \leftskip=0em{}\selectlanguage{ngerman}\endnumbering\briefempfaengerindex{Schnitzler, Arthur@\textsc{Schnitzler, Arthur}!zzzHofmannsthal, Hugo August von@\emph{von Hugo August von Hofmannsthal}!1904-12-052@{5. 12. 1904}|)be}\mylabel{L01476h}  \normalsize

\doendnotes{C}
\bigskip
\vfill

\clearpage

\footnotesize

\lohead{\textsc{register}}

% Definiere theindex-Environment komplett neu ohne reledmac
\makeatletter
\renewenvironment{theindex}{%
  \section*{\indexname}%
  \setlength{\parindent}{0pt}%
  \setlength{\parskip}{0pt plus 0.3pt}%
  \let\item\@idxitem
}{%
  \clearpage
}
\makeatother

\IfFileExists{\jobname-pw.ind}{\input{\jobname-pw.ind}}{}

\end{document}

      