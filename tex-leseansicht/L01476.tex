%% latex-leseansicht-vorspann.tex
%% Vorspann für die Leseansicht.
%% Lädt die gemeinsame Datei latex-vorspann.tex mit nicht gesetztem Schalter.

\newif\ifkorrekturansicht
\korrekturansichtfalse

\input{../tex-inputs/latex-vorspann}


\section[Hugo August von Hofmannsthal an Arthur Schnitzler, 5. 12. 1904]{L01476 Hugo August von Hofmannsthal an Arthur Schnitzler, 5. 12. 1904}
\nopagebreak\mylabel{L01476v}
\rehead{ }\normalsize\beginnumbering\briefempfaengerindex{Schnitzler, Arthur@\textsc{Schnitzler, Arthur}!zzzHofmannsthal, Hugo August von@\emph{von Hugo August von Hofmannsthal}!1904-12-053@{5. 12. 1904}|(be}
\toendnotes[C]{\smallbreak\pagebreak[2]}
\correspDesc{Versand  durch Hugo August von Hofmannsthal am 5. 12. 1904 in Wien
\newline{}Erhalt  durch Arthur Schnitzler am 5. 12. 1904 in Wien}\toendnotes[C]{\smallbreak}
\Standort{DLA, A:Schnitzler, HS.NZ85.1.3483.}
\physDesc{Brief, 1 Blatt, 2 Seiten, 666 Zeichen (Briefpapier mit Trauerrand)
\newline{}Handschrift: schwarze Tinte, deutsche Kurrent
\newline{}Schnitzler: mit rotem Buntstift beschriftet: »\textsc{(Hugos\pwindex{Hofmannsthal, Hugo von 1.\,2.\,1874 Wien – 15.\,7.\,1929 Rodaun@\textsc{Hofmannsthal, Hugo von} (1.\,2.\,1874 Wien – 15.\,7.\,1929 Rodaun), \emph{Schriftsteller}|pw} Vater)}« }\toendnotes[C]{\smallbreak}
\pstart
           \raggedleft{}{\pb}Wien\oindex{Wien@\textbf{Wien}, \emph{Verwaltungsgebiet}|pw} den 5 December.{\\}1904\pend
           
\pstart{}Geehrter Freund!\pend\vspace{0.5em}
\pstart
           Ich beeile mich Ihnen mitzuteilen, dß ich mich meiner \textsc{diplomatischen Miſsionen} betreffs der \textsc{Tantième} von
               der \label{K_L01476-1v}\edtext{Woltätigkeitsvorſtellung}{\lemma{\textnormal{\emph{Woltätigkeitsvorstellung}}}\Cendnote{\textnormal{Es handelt sich um den am 12. 12. 1904
                  stattfindenden »Arthur-Schnitzler-Abend« im Carl-Theater\oindex{Wien@\textbf{Wien}!II., Leopoldstadt@\textbf{II., Leopoldstadt}!Carl-Theater@\textbf{Carl-Theater}, \emph{Theater}|pwk}. Dieser wurde für das seit 1787 bestehende \emph{Erste öffentliche Kinderkrankeninstitut}\orgindex{Erstes öffentliches Kinderkrankeninstitut@Erstes öffentliches Kinderkrankeninstitut|pwk}
                  abgehalten, dessen Leitung Carl Hochsinger\pwindex{Hochsinger, Carl 12.\,7.\,1860 Wien – 28.\,10.\,1942 Konzentrationslager Theresienstadt@\textsc{Hochsinger, Carl} (12.\,7.\,1860 Wien – 28.\,10.\,1942 Konzentrationslager Theresienstadt), \emph{Pädiater}|pwk}
                  innehatte.}}}\label{K_L01476-1}, geſtern pflichtgemäß entledigt habe. Die \textsc{Arrangeure} waren{ }ſehr erſchüttert, weil{ }ſie natürlich an den Dichter, der ja
               bekanntlich von der Luft zu leben verpflichtet iſt, nicht gedacht hatten, aber ich
               habe \strikeout{pf} ihnen den \textcolor{gray}{Stand}pönal klar
               gemacht. Baron Haas\pwindex{Haas-Teichen, Philipp von 15.\,11.\,1859 Wien – 26.\,2.\,1926 ebd.@\textsc{Haas-Teichen, Philipp von} (15.\,11.\,1859 Wien – 26.\,2.\,1926 ebd.), \emph{Schriftsteller, Industrieller, Dramatiker}|pw} hat wegen des Ablebens{ }ſeines {\pb}Schwagers \textsc{Grfen Castell\pwindex{Castell-Rüdenhausen, Wilhelm zu 13.\,8.\,1841 Rüdenhausen – 3.\,12.\,1904 Wien@\textsc{Castell-Rüdenhausen, Wilhelm zu} (13.\,8.\,1841 Rüdenhausen – 3.\,12.\,1904 Wien)|pw}} abſagen müſſen u D\textsuperscript{r}{ }\textsc{Hochsinger}\pwindex{Hochsinger, Carl 12.\,7.\,1860 Wien – 28.\,10.\,1942 Konzentrationslager Theresienstadt@\textsc{Hochsinger, Carl} (12.\,7.\,1860 Wien – 28.\,10.\,1942 Konzentrationslager Theresienstadt), \emph{Pädiater}|pw} iſt bemüht mit Hilfe \textsc{Heines\pwindex{Heine, Albert 16.\,11.\,1867 Braunschweig – 13.\,4.\,1949 Westerland@\textsc{Heine, Albert} (16.\,11.\,1867 Braunschweig – 13.\,4.\,1949 Westerland), \emph{Theaterleiter, Schauspieler}|pw}} u \textsc{Treßler}\pwindex{Tressler, Otto 13.\,4.\,1871 Stuttgart – 27.\,4.\,1965 Wien@\textsc{Tressler, Otto} (13.\,4.\,1871 Stuttgart – 27.\,4.\,1965 Wien), \emph{Schauspieler, Bildhauer}|pw} einen paſſenden Erſatz zu finden.\pend
           
\pstart
           Empfehlen Sie mich gütigſt Ihrer Gnädigen\pwindex{Schnitzler, Olga 17.\,1.\,1882 Wien – 13.\,1.\,1970 Lugano@\textsc{Schnitzler, Olga} (17.\,1.\,1882 Wien – 13.\,1.\,1970 Lugano), \emph{Schauspielerin, Sängerin}|pwv} und{ }ſein Sie beſtens gegrüßt von Ihrem\pend
           
\pstart
           ergebenſten{\\[\baselineskip]}\spacefill\mbox{D\textsuperscript{r} Hofmannsthal}\pend
           \leftskip=0em{}\selectlanguage{ngerman}\endnumbering\briefempfaengerindex{Schnitzler, Arthur@\textsc{Schnitzler, Arthur}!zzzHofmannsthal, Hugo August von@\emph{von Hugo August von Hofmannsthal}!1904-12-053@{5. 12. 1904}|)be}\mylabel{L01476h}  \newcommand{\dateiname}{L01476}\newcommand{\titel}{Hugo August von Hofmannsthal an Arthur Schnitzler, 5. 12. 1904}\newcommand{\editorInnen}{Martin Anton Müller und Gerd-Hermann Susen}%% latex-leseansicht-abspann.tex
%% Abspann für die Leseansicht.
%% Der Schalter \ifkorrekturansicht ist bereits durch den Vorspann gesetzt.

%% latex-abspann.tex
%% Gemeinsamer Abspann für Korrekturansicht und Leseansicht.
%% Setzt den Schalter \ifkorrekturansicht voraus (gesetzt in den
%% einbindenden Dateien latex-korrekturansicht-abspann.tex bzw.
%% latex-leseansicht-abspann.tex).
%% ---------------------------------------------------------------

\normalsize

% Das esempio-Environment wird nur in der Leseansicht benötigt
\ifkorrekturansicht\else
\newenvironment{esempio}[3]%
{
    \vspace{1.5ex}
    \rlap{\underline{#1}}
    \par
    \setlength{\parindent}{0cm}
    \nopagebreak
    \leftskip=#2cm
    \rightskip=#3cm
}
{
    \par
}
\fi

\doendnotes{C}
\bigskip
\vfill

\clearpage

\footnotesize

\ifkorrekturansicht
  \lohead{\textsc{register}}
\fi

% theindex-Environment neu definieren ohne reledmac
\makeatletter
\renewenvironment{theindex}{%
  \ifkorrekturansicht
    \section*{\indexname}%
  \else
    \subsubsection*{Index der erwähnten Entitäten}%
  \fi
  \setlength{\parindent}{0pt}%
  \setlength{\parskip}{0pt plus 0.3pt}%
  \let\item\@idxitem
}{%
  \ifkorrekturansicht\clearpage\fi
}
\makeatother

\IfFileExists{\jobname-pw.ind}{\input{\jobname-pw.ind}}{}

% Quellenangabe nur in der Leseansicht
\ifkorrekturansicht\else
% Fallback-Definitionen, falls die .tex-Datei \titel etc. nicht gesetzt hat
\providecommand{\titel}{}
\providecommand{\editorInnen}{}
\providecommand{\dateiname}{\jobname}

\vspace{3cm}

\vfill

\footnotesize
\textsc{Quelle}: \titel. Herausgegeben von {\editorInnen}. In: \emph{Arthur Schnitzler: Briefwechsel mit Autorinnen und Autoren}.
 Digitale Edition, https://schnitzler-briefe.acdh.oeaw.ac.at/{\dateiname}.html (Stand \today)
\fi

\end{document}


