%% latex-leseansicht-vorspann.tex
%% Vorspann für die Leseansicht.
%% Lädt die gemeinsame Datei latex-vorspann.tex mit nicht gesetztem Schalter.

\newif\ifkorrekturansicht
\korrekturansichtfalse

\input{../tex-inputs/latex-vorspann}


         
         \renewcommand{\erwaehntePersonen}{Personen: Wilhelm zu Castell-Rüdenhausen, Philipp von Haas-Teichen, Albert Heine, Carl Hochsinger, Hugo August von Hofmannsthal, Hugo von Hofmannsthal, Olga Schnitzler, Otto Tressler}
         \renewcommand{\erwaehnteInstitutionen}{Institutionen: Erstes öffentliches Kinderkrankeninstitut}
         \renewcommand{\erwaehnteOrte}{Orte: Carl-Theater, Wien}
         \renewcommand{\erwaehnteWerke}{}
               \section[Hugo August von Hofmannsthal an Arthur Schnitzler, 5. 12. 1904]{ Hugo August von Hofmannsthal an Arthur Schnitzler, 5. 12. 1904}\nopagebreak\mylabel{v}\rehead{ }\begin{ledgroupsized}[t]{13cm}\normalsize\beginnumbering\briefempfaengerindex{Schnitzler, Arthur@\textsc{Schnitzler, Arthur}!zzzHofmannsthal, Hugo August von@\emph{von Hugo August von Hofmannsthal}!1904-12-052@{5. 12. 1904}|(be} \toendnotes[C]{\smallbreak\pagebreak[2]} \Standort{DLA, A:Schnitzler, HS.NZ85.1.3483.}
\physDesc{Brief, 1 Blatt, 2 Seiten, 666 Zeichen (Briefpapier mit Trauerrand)
\newline{}Handschrift: schwarze Tinte, deutsche Kurrent
\newline{}Schnitzler: mit rotem Buntstift beschriftet: »\textsc{(Hugos\pwindex{Hofmannsthal, Hugo von 1874-02-01 – 1929-07-15@\textsc{Hofmannsthal, Hugo von} (1874-02-01 – 1929-07-15), \emph{Schriftsteller}|pw} Vater)}« }\toendnotes[C]{\smallbreak}\pstart
           \raggedleft{}{\pb}Wien\oindex{Wien@\textbf{Wien}|pw} den 5 December.{\\}1904\pend
           \pstart{}Geehrter Freund!\pend\pstart
           Ich beeile mich Ihnen mitzuteilen, dß ich mich meiner \textsc{diplomatischen Miſsionen} betreffs der \textsc{Tantième} von
               der \label{K_L01476-1v}\edtext{Woltätigkeitsvorſtellung}{\lemma{\textnormal{\emph{Woltätigkeitsvorſtellung}}}\Cendnote{\textnormal{Es handelt sich um den am 12. 12. 1904
                  stattfindenden »Arthur-Schnitzler-Abend« im Carl-Theater\oindex{Carl-Theater@\textbf{Carl-Theater}|pwk}. Dieser wurde für das seit 1787 bestehende \emph{Erste öffentliche Kinderkrankeninstitut}\orgindex{Erstes oeffentliches Kinderkrankeninstitut@Erstes öffentliches Kinderkrankeninstitut|pwk}
                  abgehalten, dessen Leitung Carl Hochsinger\pwindex{Hochsinger, Carl 12.07.1860 – 28.10.1942@\textsc{Hochsinger, Carl} (12.07.1860 – 28.10.1942), \emph{Pädiater}|pwk}
                  inne hatte.}}}\label{K_L01476-1h}, geſtern pflichtgemäß entledigt habe. Die \textsc{Arrangeure} waren ſehr erſchüttert, weil ſie natürlich an den Dichter, der ja
               bekanntlich von der Luft zu leben verpflichtet iſt, nicht gedacht hatten, aber ich
               habe \strikeout{pf} ihnen den \textcolor{gray}{Stand}pönal klar
               gemacht. Baron Haas\pwindex{Haas-Teichen, Philipp von 15.11.1859 – 26.02.1926@\textsc{Haas-Teichen, Philipp von} (15.11.1859 – 26.02.1926), \emph{Schriftsteller, Industrieller, Dramatiker}|pw} hat wegen des Ablebens
               ſeines {\pb}Schwagers \textsc{Grfen Castell\pwindex{Castell-Ruedenhausen, Wilhelm zu 1841-08-13 – 1904-12-03@\textsc{Castell-Rüdenhausen, Wilhelm zu} (1841-08-13 – 1904-12-03)|pw}} abſagen müſſen u D\textsuperscript{r}{ }\textsc{Hochsinger}\pwindex{Hochsinger, Carl 12.07.1860 – 28.10.1942@\textsc{Hochsinger, Carl} (12.07.1860 – 28.10.1942), \emph{Pädiater}|pw} iſt bemüht mit Hilfe \textsc{Heine\pwindex{Heine, Albert 16.11.1867 – 13.4.1949@\textsc{Heine, Albert} (16.11.1867 – 13.4.1949), \emph{Theaterleiter, Schauspieler}|pw}s} u \textsc{Treßler}\pwindex{Tressler, Otto 13.04.1871 – 27.04.1965@\textsc{Tressler, Otto} (13.04.1871 – 27.04.1965), \emph{Schauspieler, Bildhauer}|pw} einen paſſenden Erſatz zu finden.\pend
           \pstart
           Empfehlen Sie mich gütigſt Ihrer Gnädigen\pwindex{Schnitzler, Olga 17.01.1882 – 13.01.1970@\textsc{Schnitzler, Olga} (17.01.1882 – 13.01.1970), \emph{Schauspielerin, Sängerin}|pwv} und ſein Sie beſtens gegrüßt von Ihrem\pend
           \pstart
           ergebenſten{\\[\baselineskip]}\spacefill\mbox{D\textsuperscript{r} Hofmannsthal}\pend
           \leftskip=0em{}
         
         \endnumbering\mylabel{h}\end{ledgroupsized}  \newcommand{\dateiname}{L01476}\newcommand{\titel}{Hugo August von Hofmannsthal an Arthur Schnitzler, 5. 12. 1904}\newcommand{\editorInnen}{Martin Anton Müller und Gerd-Hermann Susen}%% latex-leseansicht-abspann.tex
%% Abspann für die Leseansicht.
%% Der Schalter \ifkorrekturansicht ist bereits durch den Vorspann gesetzt.

%% latex-abspann.tex
%% Gemeinsamer Abspann für Korrekturansicht und Leseansicht.
%% Setzt den Schalter \ifkorrekturansicht voraus (gesetzt in den
%% einbindenden Dateien latex-korrekturansicht-abspann.tex bzw.
%% latex-leseansicht-abspann.tex).
%% ---------------------------------------------------------------

\normalsize

% Das esempio-Environment wird nur in der Leseansicht benötigt
\ifkorrekturansicht\else
\newenvironment{esempio}[3]%
{
    \vspace{1.5ex}
    \rlap{\underline{#1}}
    \par
    \setlength{\parindent}{0cm}
    \nopagebreak
    \leftskip=#2cm
    \rightskip=#3cm
}
{
    \par
}
\fi

\doendnotes{C}
\bigskip
\vfill

\clearpage

\footnotesize

\ifkorrekturansicht
  \lohead{\textsc{register}}
\fi

% theindex-Environment neu definieren ohne reledmac
\makeatletter
\renewenvironment{theindex}{%
  \ifkorrekturansicht
    \section*{\indexname}%
  \else
    \subsubsection*{Index der erwähnten Entitäten}%
  \fi
  \setlength{\parindent}{0pt}%
  \setlength{\parskip}{0pt plus 0.3pt}%
  \let\item\@idxitem
}{%
  \ifkorrekturansicht\clearpage\fi
}
\makeatother

\IfFileExists{\jobname-pw.ind}{\input{\jobname-pw.ind}}{}

% Quellenangabe nur in der Leseansicht
\ifkorrekturansicht\else
% Fallback-Definitionen, falls die .tex-Datei \titel etc. nicht gesetzt hat
\providecommand{\titel}{}
\providecommand{\editorInnen}{}
\providecommand{\dateiname}{\jobname}

\vspace{3cm}

\vfill

\footnotesize
\textsc{Quelle}: \titel. Herausgegeben von {\editorInnen}. In: \emph{Arthur Schnitzler: Briefwechsel mit Autorinnen und Autoren}.
 Digitale Edition, https://schnitzler-briefe.acdh.oeaw.ac.at/{\dateiname}.html (Stand \today)
\fi

\end{document}


      