%% latex-leseansicht-vorspann.tex
%% Vorspann für die Leseansicht.
%% Lädt die gemeinsame Datei latex-vorspann.tex mit nicht gesetztem Schalter.

\newif\ifkorrekturansicht
\korrekturansichtfalse

\input{../tex-inputs/latex-vorspann}


\section[Fedor Mamroth an Arthur Schnitzler, 5. 3. 1893]{L00186 Fedor Mamroth an Arthur Schnitzler, 5. 3. 1893}
\nopagebreak\mylabel{L00186v}
\rehead{ }\normalsize\beginnumbering\briefempfaengerindex{Schnitzler, Arthur@\textsc{Schnitzler, Arthur}!zzzMamroth, Fedor@\emph{von Fedor Mamroth}!1893-03-052@{5. 3. 1893}|(be}
\toendnotes[C]{\smallbreak\pagebreak[2]}
\correspDesc{Versand  durch Fedor Mamroth am 5. 3. 1893 in Frankfurt am Main
\newline{}Erhalt  durch Arthur Schnitzler im Zeitraum [6. 3. 1893
                  – 10. 3. 1893?] in Wien}\toendnotes[C]{\smallbreak}
\Standort{CUL, Schnitzler, B 68.}
\physDesc{Brief, 1 Blatt, 2 Seiten, 2811 Zeichen
\newline{}Handschrift: blaue Tinte, deutsche Kurrent
\newline{}Schnitzler: 1) mit Bleistift nummeriert: »4.«  2) mit rotem Buntstift eine Unterstreichung}\toendnotes[C]{\smallbreak}
\pstart
           {\pb}\textcolor{gray}{\textbf{\textsc{Frankfurter Zeitung}}}{\\}\textsc{\textcolor{gray}{\textbf{und}}}{\\}\textcolor{gray}{\textbf{\textsc{Handelsblatt.}}}\orgindex{Frankfurter Zeitung@Frankfurter Zeitung|pw}\pend
           
\pstart
           
\pstart
           \textcolor{gray}{\textbf{\textsc{Redaction.}}}\pend
           
\pstart
           \raggedleft{}\textcolor{gray}{\textbf{\textsc{Frankfurt a. M.\oindex{Frankfurt am Main@\textbf{Frankfurt am Main}, \emph{Hauptstadt}|pw},}}}{ }5. März \textsc{\textcolor{gray}{\textbf{189}}}3\pend
           \pend
           
\pstart
           \textcolor{gray}{\textbf{\textsc{Telegramm-Adresse:}}}\pend
           
\pstart
           \textcolor{gray}{\textbf{\textsc{Zeitung Frankfurt Main.}}}\pend
           
\pstart{}Mein{ }ſehr verehrter Herr Doctor!\pend\vspace{0.5em}
\pstart
           Ich habe letzten Sonntag – heute vor 8 Tagen – Ihren Roman\pwindex{Schnitzler, Arthur 15.\,5.\,1862 Wien – 21.\,10.\,1931 ebd.@\textsc{Schnitzler, Arthur} (15.\,5.\,1862 Wien – 21.\,10.\,1931 ebd.), \emph{Schriftsteller, Mediziner}!Sterben. Novelle@\strich\emph{Sterben. Novelle}|pwv} in einem Zuge ausgeleſen, was mir bei
               einem Manuſcript{ }ſchon lange nicht paſſiert iſt, und darüber{ }ſogar das Theater
               verſäumt, was mir noch nie paſſiert iſt. Die ganze Woche über kam ich nicht dazu,
               Ihnen zu{ }ſchreiben, u. erſt heute vermag ich Ihnen mitzutheilen, daß ich die
               Erzählung \uline{nicht} acceptiere.\pend
           
\pstart
           Warum? Nicht mit Rückſicht auf die Prüderie des Publikums, denn die paar Stellen, die
               als bedenklich in Betracht kämen, ließen{ }ſich leicht beſeitigen. Nein, aus einem
               Grunde, den Sie von Ihrem Standpunkt aus gar nicht verſtehen dürften: Der Roman\pwindex{Schnitzler, Arthur 15.\,5.\,1862 Wien – 21.\,10.\,1931 ebd.@\textsc{Schnitzler, Arthur} (15.\,5.\,1862 Wien – 21.\,10.\,1931 ebd.), \emph{Schriftsteller, Mediziner}!Sterben. Novelle@\strich\emph{Sterben. Novelle}|pwv} ist mir viel zu ernſt u.
               düſter, mir, dem man beſtändig den Vorwurf macht, daß unſer Roman-Feuilleton »viel zu
               ernſt u. düſter«{ }ſei. Berückſichtigen Sie gefälligſt, daß ich nichts weiter bin als
               ein Knecht \label{T_L00186-1v}\edtext{und}{\lemma{\textnormal{\emph{und}}}\Cendnote{\textnormal{Er schreibt »und und«.}}}\label{T_L00186-1} daß ich aus
               tiefſter Knechts-Überzeugung ablehnen muß, unſer Publikum mit einer{ }ſo wenig
               fröhlichen und erbaulichen Erzählung,{ }ſchon in aller Frühe beim Morgenkaffee zu
               verſtimmen.\pend
           
\pstart
           Alſo ich nehme Ihren Roman\pwindex{Schnitzler, Arthur 15.\,5.\,1862 Wien – 21.\,10.\,1931 ebd.@\textsc{Schnitzler, Arthur} (15.\,5.\,1862 Wien – 21.\,10.\,1931 ebd.), \emph{Schriftsteller, Mediziner}!Sterben. Novelle@\strich\emph{Sterben. Novelle}|pwv}
               nicht, und das iſt wohl die Hauptſache, für Sie, aber nicht für mich; denn ich muß
               Ihnen noch etwas{ }ſagen, was an u. für{ }ſich{ }ſehr gleichgiltig iſt, Ihnen, aber nicht
               mir, nämlich daß {\pb}\uline{ich} der Lektüre Ihrer Erzählung\pwindex{Schnitzler, Arthur 15.\,5.\,1862 Wien – 21.\,10.\,1931 ebd.@\textsc{Schnitzler, Arthur} (15.\,5.\,1862 Wien – 21.\,10.\,1931 ebd.), \emph{Schriftsteller, Mediziner}!Sterben. Novelle@\strich\emph{Sterben. Novelle}|pwv} eine große Freude verdanke, – nein, das iſt
                  \damage{wohl} nicht das richtige Wort: eine zunehmende Aufregung, eine innige
               Antheilnahme, eine{ }ſtarke Erſchütterung. Es iſt eine glänzende Arbeit, mit der Sie
               einen{ }ſchönen Erfolg haben werden, nicht in einer Zeitung,{ }ſondern im Buche. Ich
               würde mir an Ihrer Stelle erſt keine Mühe geben,{ }ſie bei einer Redaction
               einzureichen; wenn \uline{ich}{ }ſie nicht nehme, nimmt{ }ſie Niemand;{ }ſoweit glaube
               ich den Geiſt der deutſchen\oindex{Deutschland@\textbf{Deutschland}|pw} u. öſterreichiſchen\oindex{Österreich@\textbf{Österreich}|pw} Preſſe zu kennen. Alſo im Buche
               u. ich wäre glücklich, Ihnen, falls dies nötig wäre, in irgend einer Weiſe dabei
               behilflich{ }ſein zu können. Und mit einem anderen Titel. »Der{ }ſterbende Herr\pwindex{Schnitzler, Arthur 15.\,5.\,1862 Wien – 21.\,10.\,1931 ebd.@\textsc{Schnitzler, Arthur} (15.\,5.\,1862 Wien – 21.\,10.\,1931 ebd.), \emph{Schriftsteller, Mediziner}!Sterben. Novelle@\strich\emph{Sterben. Novelle}|pw}« iſt gar nichts. Da müſſen Sie{ }ſchon etwas
               anderes finden. Aber um auf die Qualität der Arbeit zurückzukommen: ich müßte außer
               Landes gehen, um einen Vergleich zu finden. Erinnern Sie{ }ſich des Todes des Fürſten
               Andrej in »Krieg und Frieden\pwindex{Tolstoi, Lew Nikolajewitsch 9.\,9.\,1828 Yasnaya Polyana – 20.\,11.\,1910 Lev Tolstoy@\textsc{Tolstoi, Lew Nikolajewitsch} (9.\,9.\,1828 Yasnaya Polyana – 20.\,11.\,1910 Lev Tolstoy), \emph{Schriftsteller}!Krieg und Frieden@\strich\emph{Krieg und Frieden}|pw}«? Das hat ein Dichter\pwindex{Tolstoi, Lew Nikolajewitsch 9.\,9.\,1828 Yasnaya Polyana – 20.\,11.\,1910 Lev Tolstoy@\textsc{Tolstoi, Lew Nikolajewitsch} (9.\,9.\,1828 Yasnaya Polyana – 20.\,11.\,1910 Lev Tolstoy), \emph{Schriftsteller}|pwv} geſchrieben, der kein
               Arzt war. Ihren Roman hat ein Arzt geſchrieben, der ein Dichter iſt. Es iſt die erſte
               zugleich künſtleriſche und wahrheitstreue Darſtellung des Grundverhältniſſes zwiſchen
               Tod u. Leben einerſeits u. der phyſiſchen Auflöſung andrerſeits, die ich kenne.
               Welche Fülle von Beobachtungen u. welche überzeugende Richtigkeit in Auffaſſung und
               Entwicklung zweier einfacher Menſchenſchickſale! Ich beglückwünſche Sie aufrichtig zu
               dieſer Arbeit, mein{ }ſehr verehrter Herr Doctor, jetzt weiß ich ganz genau, wer Sie{ }ſind, und jetzt bin ich der Erſte, der für Ihren Beruf mit Freuden Zeugniß
               ablegt.\pend
           
\pstart
           Ihr\hspace*{1.5em}ergebener{\\[\baselineskip]}\spacefill\mbox{FMamroth}\pend
           \leftskip=0em{}\selectlanguage{ngerman}\endnumbering\briefempfaengerindex{Schnitzler, Arthur@\textsc{Schnitzler, Arthur}!zzzMamroth, Fedor@\emph{von Fedor Mamroth}!1893-03-052@{5. 3. 1893}|)be}\mylabel{L00186h}  \newcommand{\dateiname}{L00186}\newcommand{\titel}{Fedor Mamroth an Arthur Schnitzler, 5. 3. 1893}\newcommand{\editorInnen}{Martin Anton Müller und Gerd-Hermann Susen}%% latex-leseansicht-abspann.tex
%% Abspann für die Leseansicht.
%% Der Schalter \ifkorrekturansicht ist bereits durch den Vorspann gesetzt.

%% latex-abspann.tex
%% Gemeinsamer Abspann für Korrekturansicht und Leseansicht.
%% Setzt den Schalter \ifkorrekturansicht voraus (gesetzt in den
%% einbindenden Dateien latex-korrekturansicht-abspann.tex bzw.
%% latex-leseansicht-abspann.tex).
%% ---------------------------------------------------------------

\normalsize

% Das esempio-Environment wird nur in der Leseansicht benötigt
\ifkorrekturansicht\else
\newenvironment{esempio}[3]%
{
    \vspace{1.5ex}
    \rlap{\underline{#1}}
    \par
    \setlength{\parindent}{0cm}
    \nopagebreak
    \leftskip=#2cm
    \rightskip=#3cm
}
{
    \par
}
\fi

\doendnotes{C}
\bigskip
\vfill

\clearpage

\footnotesize

\ifkorrekturansicht
  \lohead{\textsc{register}}
\fi

% theindex-Environment neu definieren ohne reledmac
\makeatletter
\renewenvironment{theindex}{%
  \ifkorrekturansicht
    \section*{\indexname}%
  \else
    \subsubsection*{Index der erwähnten Entitäten}%
  \fi
  \setlength{\parindent}{0pt}%
  \setlength{\parskip}{0pt plus 0.3pt}%
  \let\item\@idxitem
}{%
  \ifkorrekturansicht\clearpage\fi
}
\makeatother

\IfFileExists{\jobname-pw.ind}{\input{\jobname-pw.ind}}{}

% Quellenangabe nur in der Leseansicht
\ifkorrekturansicht\else
% Fallback-Definitionen, falls die .tex-Datei \titel etc. nicht gesetzt hat
\providecommand{\titel}{}
\providecommand{\editorInnen}{}
\providecommand{\dateiname}{\jobname}

\vspace{3cm}

\vfill

\footnotesize
\textsc{Quelle}: \titel. Herausgegeben von {\editorInnen}. In: \emph{Arthur Schnitzler: Briefwechsel mit Autorinnen und Autoren}.
 Digitale Edition, https://schnitzler-briefe.acdh.oeaw.ac.at/{\dateiname}.html (Stand \today)
\fi

\end{document}


