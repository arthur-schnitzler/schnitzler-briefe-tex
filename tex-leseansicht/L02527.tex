%% latex-leseansicht-vorspann.tex
%% Vorspann für die Leseansicht.
%% Lädt die gemeinsame Datei latex-vorspann.tex mit nicht gesetztem Schalter.

\newif\ifkorrekturansicht
\korrekturansichtfalse

\input{../tex-inputs/latex-vorspann}


\section[Robert Adam an Arthur Schnitzler, 22. 12. 1929]{L02527 Robert Adam an Arthur Schnitzler, 22. 12. 1929}
\nopagebreak\mylabel{L02527v}
\rehead{ }\normalsize\beginnumbering\briefempfaengerindex{Schnitzler, Arthur@\textsc{Schnitzler, Arthur}!zzzAdam, Robert@\emph{von Robert Adam}!1929-12-221@{22. 12. 1929}|(be}
\toendnotes[C]{\smallbreak\pagebreak[2]}
\correspDesc{Versand  durch Robert Adam am 22. 12. 1929 in Wien
\newline{}Erhalt  durch Arthur Schnitzler im Zeitraum [22. 12. 1929 – 26. 12. 1929?] in Wien}\toendnotes[C]{\smallbreak}
\Standort{CUL, Schnitzler, B 1.}
\physDesc{Brief, 1 Blatt, 2 Seiten, 1451 Zeichen
\newline{}Handschrift: schwarze Tinte, deutsche Kurrent
\newline{}Schnitzler: mit rotem Buntstift Vermerk »\textsc{So{\geminationm}erlüfte}\pwindex{Schnitzler, Arthur 15.\,5.\,1862 Wien – 21.\,10.\,1931 ebd.@\textsc{Schnitzler, Arthur} (15.\,5.\,1862 Wien – 21.\,10.\,1931 ebd.), \emph{Schriftsteller, Mediziner}!Im Spiel der Sommerlüfte. In drei Aufzügen@\strich\emph{Im Spiel der Sommerlüfte. In drei Aufzügen}|pw}« und vereinzelte Unterstreichungen 
\newline{}Ordnung: mit Bleistift von unbekannter Hand nummeriert:
                                    »23« }\Standort{Wien, Österreichische Nationalbibliothek, Cod.ser. 52.269, 149 recto.}
\physDesc{handschriftliche Abschrift. 1 Blatt, 1 Seite, 1451 Zeichen
\newline{}Handschrift: schwarze Tinte, Gabelsberger Kurzschrift}\Standort{Wien, Österreichische Nationalbibliothek, Cod.ser. 52.269, 43.}
\physDesc{maschinenschriftliche Abschrift, 1 Blatt, 1 Seite, 1451 Zeichen
\newline{}Schreibmaschine}\toendnotes[C]{\smallbreak}
\pstart
           \raggedleft{}{\pb}Wien\oindex{Wien@\textbf{Wien}, \emph{Verwaltungsgebiet}|pw}, am 22. Dezember 1929\pend
           
\pstart{}Hochverehrter Herr Doktor!\pend\vspace{0.5em}
\pstart
           Nehmen Sie meinen herzlichſten Dank für die Überſendung Ihrer Komödie »Im Spiel der Sommerlüfte\pwindex{Schnitzler, Arthur 15.\,5.\,1862 Wien – 21.\,10.\,1931 ebd.@\textsc{Schnitzler, Arthur} (15.\,5.\,1862 Wien – 21.\,10.\,1931 ebd.), \emph{Schriftsteller, Mediziner}!Im Spiel der Sommerlüfte. In drei Aufzügen@\strich\emph{Im Spiel der Sommerlüfte. In drei Aufzügen}|pw}« entgegen!\pend
           
\pstart
           Wenn ich{ }ſo meine eigenen Produkte, auch die letzten und auch die noch gar nicht
               geſchriebenen,{ }ſondern erſt geplanten – es gibt leider{ }ſolche noch immer –, im Geiſt
               Revue paſſieren laſſe und Ihr Stück danebenhalte, dann erkenne ich{ }ſo recht, wie tief
               ich im Dilettantiſmus und in der Barbarei{ }ſtecke: denn ich verkenne gar nicht, daß
               allen meinen Hervorbringungen, und mögen{ }ſie{ }ſich noch{ }ſo kultiviert gehaben, etwas
               Barbariſches, das nun einmal mit meinem innerſten Weſen verbunden{ }ſein mag und
               vielleicht eine gewiſſe Eigenheit bewirkt, immerzu anhaftet.\pend
           
\pstart
           Wie wundervoll rein und klar iſt wieder Ihr neues Stück\pwindex{Schnitzler, Arthur 15.\,5.\,1862 Wien – 21.\,10.\,1931 ebd.@\textsc{Schnitzler, Arthur} (15.\,5.\,1862 Wien – 21.\,10.\,1931 ebd.), \emph{Schriftsteller, Mediziner}!Im Spiel der Sommerlüfte. In drei Aufzügen@\strich\emph{Im Spiel der Sommerlüfte. In drei Aufzügen}|pwv} gefügt und auf {\pb}welch einheitlichem Niveau{ }ſtehen und
               gebahren{ }ſich Ihre Menſchen! Wie jugendfriſch betaut iſt alles, vor und nach dem
               Gewitter, das die Luft von Leidenſchaften reinigt! Und welch geiſtreiche Ergänzung
               der von Ihnen geſchaffenen Welt iſt dieſes Eindringen der im Kaplan verkörperten
               religiöſen Idee in die Weltlichkeit des Weiten
                  Lands\pwindex{Schnitzler, Arthur 15.\,5.\,1862 Wien – 21.\,10.\,1931 ebd.@\textsc{Schnitzler, Arthur} (15.\,5.\,1862 Wien – 21.\,10.\,1931 ebd.), \emph{Schriftsteller, Mediziner}!weite Land. Tragikomödie in fünf Akten@\strich\emph{Das weite Land. Tragikomödie in fünf Akten}|pw}! Man möchte, wenn man den Kreis Ihrer Menſchen verlaſſen muß, noch
               einmal wiederholen: »Ich werd’ oft
                  zurückdenken an den Garten, an das liebe Haus, an die Landſchaft\pwindex{Schnitzler, Arthur 15.\,5.\,1862 Wien – 21.\,10.\,1931 ebd.@\textsc{Schnitzler, Arthur} (15.\,5.\,1862 Wien – 21.\,10.\,1931 ebd.), \emph{Schriftsteller, Mediziner}!Im Spiel der Sommerlüfte. In drei Aufzügen@\strich\emph{Im Spiel der Sommerlüfte. In drei Aufzügen}|pwv}« und an die,
               die drin lebten.\pend
           
\pstart
           Indem ich Ihnen freudige Weihnachtsfeiertage von Herzen wünſche, verbleibe ich mit
               vielem Dank und vielen Empfehlungen\pend
           
\pstart
           Ihr ergebener{\\[\baselineskip]}\spacefill\mbox{D\textsuperscript{r}Adam}\pend
           \leftskip=0em{}\selectlanguage{ngerman}\endnumbering\briefempfaengerindex{Schnitzler, Arthur@\textsc{Schnitzler, Arthur}!zzzAdam, Robert@\emph{von Robert Adam}!1929-12-221@{22. 12. 1929}|)be}\mylabel{L02527h}  \newcommand{\dateiname}{L02527}\newcommand{\titel}{Robert Adam an Arthur Schnitzler, 22. 12. 1929}\newcommand{\editorInnen}{Martin Anton Müller und Gerd-Hermann Susen}%% latex-leseansicht-abspann.tex
%% Abspann für die Leseansicht.
%% Der Schalter \ifkorrekturansicht ist bereits durch den Vorspann gesetzt.

%% latex-abspann.tex
%% Gemeinsamer Abspann für Korrekturansicht und Leseansicht.
%% Setzt den Schalter \ifkorrekturansicht voraus (gesetzt in den
%% einbindenden Dateien latex-korrekturansicht-abspann.tex bzw.
%% latex-leseansicht-abspann.tex).
%% ---------------------------------------------------------------

\normalsize

% Das esempio-Environment wird nur in der Leseansicht benötigt
\ifkorrekturansicht\else
\newenvironment{esempio}[3]%
{
    \vspace{1.5ex}
    \rlap{\underline{#1}}
    \par
    \setlength{\parindent}{0cm}
    \nopagebreak
    \leftskip=#2cm
    \rightskip=#3cm
}
{
    \par
}
\fi

\doendnotes{C}
\bigskip
\vfill

\clearpage

\footnotesize

\ifkorrekturansicht
  \lohead{\textsc{register}}
\fi

% theindex-Environment neu definieren ohne reledmac
\makeatletter
\renewenvironment{theindex}{%
  \ifkorrekturansicht
    \section*{\indexname}%
  \else
    \subsubsection*{Index der erwähnten Entitäten}%
  \fi
  \setlength{\parindent}{0pt}%
  \setlength{\parskip}{0pt plus 0.3pt}%
  \let\item\@idxitem
}{%
  \ifkorrekturansicht\clearpage\fi
}
\makeatother

\IfFileExists{\jobname-pw.ind}{\input{\jobname-pw.ind}}{}

% Quellenangabe nur in der Leseansicht
\ifkorrekturansicht\else
% Fallback-Definitionen, falls die .tex-Datei \titel etc. nicht gesetzt hat
\providecommand{\titel}{}
\providecommand{\editorInnen}{}
\providecommand{\dateiname}{\jobname}

\vspace{3cm}

\vfill

\footnotesize
\textsc{Quelle}: \titel. Herausgegeben von {\editorInnen}. In: \emph{Arthur Schnitzler: Briefwechsel mit Autorinnen und Autoren}.
 Digitale Edition, https://schnitzler-briefe.acdh.oeaw.ac.at/{\dateiname}.html (Stand \today)
\fi

\end{document}


