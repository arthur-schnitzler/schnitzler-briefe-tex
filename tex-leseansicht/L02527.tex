%% latex-korrekturansicht-vorspann.tex
%% Vorspann für die Korrekturansicht.
%% Lädt die gemeinsame Datei latex-vorspann.tex mit gesetztem Schalter.

\newif\ifkorrekturansicht
\korrekturansichttrue

\input{../tex-inputs/latex-vorspann}


\section[Robert Adam an Arthur Schnitzler, 22. 12. 1929]{L02527 Robert Adam an Arthur Schnitzler, 22. 12. 1929}
\nopagebreak\mylabel{L02527v}
\rehead{ }\normalsize\beginnumbering\briefempfaengerindex{Schnitzler, Arthur@\textsc{Schnitzler, Arthur}!zzzAdam, Robert@\emph{von Robert Adam}!1929-12-221@{22. 12. 1929}|(be}
\toendnotes[C]{\smallbreak\pagebreak[2]}\Standort{CUL, Schnitzler, B 1.}
\physDesc{Brief, 1 Blatt, 2 Seiten, 1451 Zeichen
\newline{}Handschrift: schwarze Tinte, deutsche Kurrent
\newline{}Schnitzler: 1) mit rotem Buntstift beschriftet: »\textsc{So{\geminationm}erlüfte}\pwindex{Im Spiel der Sommerluefte. In drei Aufzuegen@\emph{Im Spiel der Sommerlüfte. In drei Aufzügen}|pw}«  2) mit rotem Buntstift vereinzelte Unterstreichungen
\newline{}Ordnung: mit Bleistift von unbekannter Hand nummeriert:
                                    »23« }\Standort{Wien, Österreichische Nationalbibliothek, Cod.ser. 52.269, 149 recto.}
\physDesc{handschriftliche Abschrift1 Blatt, 1 Seite, 1451 Zeichen
\newline{}Handschrift: schwarze Tinte, Gabelsberger Kurzschrift}\Standort{Wien, Österreichische Nationalbibliothek, Cod.ser. 52.269, 43.}
\physDesc{maschinenschriftliche Abschrift1 Blatt, 1 Seite, 1451 Zeichen
\newline{}Schreibmaschine}\toendnotes[C]{\smallbreak}
\pstart
           \raggedleft{}{\pb}Wien\oindex{Wien@\textbf{Wien}, \emph{A.ADM2}|pw}, am 22. Dezember 1929\pend
           
\pstart{}Hochverehrter Herr Doktor!\pend\vspace{0.5em}
\pstart
           Nehmen Sie meinen herzlichſten Dank für die Überſendung Ihrer Komödie »Im Spiel der Sommerlüfte\pwindex{Im Spiel der Sommerluefte. In drei Aufzuegen@\emph{Im Spiel der Sommerlüfte. In drei Aufzügen}|pw}« entgegen!\pend
           
\pstart
           Wenn ich ſo meine eigenen Produkte, auch die letzten und auch die noch gar nicht
               geſchriebenen, ſondern erſt geplanten – es gibt leider ſolche noch immer –, im Geiſt
               Revue paſſieren laſſe und Ihr Stück danebenhalte, dann erkenne ich ſo recht, wie tief
               ich im Dilettantiſmus und in der Barbarei ſtecke: denn ich verkenne gar nicht, daß
               allen meinen Hervorbringungen, und mögen ſie ſich noch ſo kultiviert gehaben, etwas
               Barbariſches, das nun einmal mit meinem innerſten Weſen verbunden ſein mag und
               vielleicht eine gewiſſe Eigenheit bewirkt, immerzu anhaftet.\pend
           
\pstart
           Wie wundervoll rein und klar iſt wieder Ihr neues Stück\pwindex{Im Spiel der Sommerluefte. In drei Aufzuegen@\emph{Im Spiel der Sommerlüfte. In drei Aufzügen}|pwv} gefügt und auf {\pb}welch einheitlichem Niveau ſtehen und
               gebahren ſich Ihre Menſchen! Wie jugendfriſch betaut iſt alles, vor und nach dem
               Gewitter, das die Luft von Leidenſchaften reinigt! Und welch geiſtreiche Ergänzung
               der von Ihnen geſchaffenen Welt iſt dieſes Eindringen der im Kaplan verkörperten
               religiöſen Idee in die Weltlichkeit des Weiten
                  Lands\pwindex{weite Land. Tragikomoedie in fuenf Akten@\emph{Das weite Land. Tragikomödie in fünf Akten}|pw}! Man möchte, wenn man den Kreis Ihrer Menſchen verlaſſen muß, noch
               einmal wiederholen: »Ich werd’ oft
                  zurückdenken an den Garten, an das liebe Haus, an die Landſchaft\pwindex{Im Spiel der Sommerluefte. In drei Aufzuegen@\emph{Im Spiel der Sommerlüfte. In drei Aufzügen}|pwv}« und an die,
               die drin lebten.\pend
           
\pstart
           Indem ich Ihnen freudige Weihnachtsfeiertage von Herzen wünſche, verbleibe ich mit
               vielem Dank und vielen Empfehlungen\pend
           
\pstart
           Ihr ergebener{\\[\baselineskip]}\spacefill\mbox{D\textsuperscript{r}Adam}\pend
           \leftskip=0em{}\selectlanguage{ngerman}\endnumbering\briefempfaengerindex{Schnitzler, Arthur@\textsc{Schnitzler, Arthur}!zzzAdam, Robert@\emph{von Robert Adam}!1929-12-221@{22. 12. 1929}|)be}\mylabel{L02527h}  \normalsize

\doendnotes{C}
\bigskip
\vfill

\clearpage

\footnotesize

\lohead{\textsc{register}}

% Definiere theindex-Environment komplett neu ohne reledmac
\makeatletter
\renewenvironment{theindex}{%
  \section*{\indexname}%
  \setlength{\parindent}{0pt}%
  \setlength{\parskip}{0pt plus 0.3pt}%
  \let\item\@idxitem
}{%
  \clearpage
}
\makeatother

\IfFileExists{\jobname-pw.ind}{\input{\jobname-pw.ind}}{}

\end{document}

      