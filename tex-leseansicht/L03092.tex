%% latex-leseansicht-vorspann.tex
%% Vorspann für die Leseansicht.
%% Lädt die gemeinsame Datei latex-vorspann.tex mit nicht gesetztem Schalter.

\newif\ifkorrekturansicht
\korrekturansichtfalse

\input{../tex-inputs/latex-vorspann}


\section[ Paul Goldmann an Arthur Schnitzler, 29. 11. [1901]]{L03092 Paul Goldmann an Arthur Schnitzler,  29. 11. [1901]}
\nopagebreak\mylabel{L03092v}
\rehead{ }\normalsize\beginnumbering\briefempfaengerindex{Schnitzler, Arthur@\textsc{Schnitzler, Arthur}!zzzGoldmann, Paul@\emph{von Paul Goldmann}!1901-11-291@{29. 11. [1901]}|(be}
\toendnotes[C]{\smallbreak\pagebreak[2]}
\correspDesc{Versand  durch Paul Goldmann am 29. 11. [1901] in Berlin
\newline{}Erhalt  durch Arthur Schnitzler im Zeitraum [30. 11. 1901 – 4. 12. 1901?] in Wien}\toendnotes[C]{\smallbreak}
\Standort{DLA, A:Schnitzler, HS.NZ85.1.3171.}
\physDesc{Brief, 1 Blatt, 4 Seiten, 1564 Zeichen
\newline{}Handschrift: blaue Tinte, deutsche Kurrent
\newline{}Schnitzler: mit Bleistift das Jahr »901.« vermerkt }\toendnotes[C]{\smallbreak}
\pstart
           \raggedleft{}{\pb}\textcolor{gray}{\textbf{DESSAUERSTRASSE 19}}\oindex{Dessauer Straße@\textbf{Dessauer Straße}, \emph{Straße}|pw}\pend
           
\pstart
           Berlin\oindex{Berlin@\textbf{Berlin}, \emph{Hauptstadt}|pw}, 29. November.\pend
           
\pstart\center{}Mein lieber Freund,\pend\vspace{0.5em}
\pstart
           »Ungütig«! Du greifſt mich an, – greifſt mich an der Stelle an, wo ich am
               Verwundbarſten bin, – da, wo mein Lebensnerv{ }ſitzt. Ich wehre mich gegen Deinen
               Angriff. Und das nennſt Du »ungütig aufnehmen«. Das iſt ein glänzender
               Luſtſpiel-Einfall, und Du{ }ſollſt Dir ihn aufnotiren.\pend
           
\pstart
           »Zurechtweiſen«. Gewiß, \label{K_L03092-1v}\edtext{\textsc{Olga\pwindex{Schnitzler, Olga 17.\,1.\,1882 Wien – 13.\,1.\,1970 Lugano@\textsc{Schnitzler, Olga} (17.\,1.\,1882 Wien – 13.\,1.\,1970 Lugano), \emph{Schauspielerin, Sängerin}|pw}}}{\lemma{\textnormal{\emph{Olga}}}\Cendnote{\textnormal{Siehe XXXX Auszeichnungsfehler: Dokument L03091 nicht gefunden.
               }}}\label{K_L03092-1} hat mich nicht zurechtweiſen {\pb}\uline{gewollt}. Aber{ }ſie hat’s \uline{gethan}. Und was mich{ }ſo{ }ſehr erregte, \strikeout{war,}
               war, daß ich plötzlich erkennen mußte, wie dieſes Mädchen\pwindex{Schnitzler, Olga 17.\,1.\,1882 Wien – 13.\,1.\,1970 Lugano@\textsc{Schnitzler, Olga} (17.\,1.\,1882 Wien – 13.\,1.\,1970 Lugano), \emph{Schauspielerin, Sängerin}|pwv}, dem ich in aufrichtigſter Freundſchaft zugethan
               bin, die \strikeout{d\textcolor{gray}{e}} die Freundin\pwindex{Schnitzler, Olga 17.\,1.\,1882 Wien – 13.\,1.\,1970 Lugano@\textsc{Schnitzler, Olga} (17.\,1.\,1882 Wien – 13.\,1.\,1970 Lugano), \emph{Schauspielerin, Sängerin}|pwv} meines
               liebſten Freundes iſt, weltenweit davon entfernt iſt, mich zu verſtehen!\pend
           
\pstart
           Im Übrigen iſt wirklich genug geredet; und es iſt{ }ſehr blöd, daß wir uns da
               gegenſeitig allerlei Grobheiten{ }ſchreiben, wo wir uns doch {\pb}wirklich Wichtigeres zu{ }ſagen hätten.\pend
           
\pstart
           Mein lieber Freund, ich kann Dir heut nicht{ }ſo
               ausführlich{ }ſchreiben, als ich möchte. Ich habe wahnſinnig zu thun. In einigen Tagen
               hoffe ich Zeit zu einem längeren Brief zu finden.\pend
           
\pstart
           Der »\label{K_L03092-2v}\edtext{Rothe Hahn\eventindex{Deutsches Theater Berlin@\textbf{Deutsches Theater Berlin}!Uraufführung Der rothe Hahn, 27.11.1901@Uraufführung Der rothe Hahn, 27.11.1901|pwv}\pwindex{Hauptmann, Gerhart 15.\,11.\,1862 Szczawno-Zdrój – 6.\,6.\,1946 Jagniątków@\textsc{Hauptmann, Gerhart} (15.\,11.\,1862 Szczawno-Zdrój – 6.\,6.\,1946 Jagniątków), \emph{Schriftsteller}!rothe Hahn. Tragikomödie in vier Akten@\strich\emph{Der rothe Hahn. Tragikomödie in vier Akten}|pw}}{\lemma{\textnormal{\emph{Rothe Hahn}}}\Cendnote{\textnormal{Die Uraufführung von 
                     \emph{Der rothe Hahn. Tragikomödie in vier Akten}\pwindex{Hauptmann, Gerhart 15.\,11.\,1862 Szczawno-Zdrój – 6.\,6.\,1946 Jagniątków@\textsc{Hauptmann, Gerhart} (15.\,11.\,1862 Szczawno-Zdrój – 6.\,6.\,1946 Jagniątków), \emph{Schriftsteller}!rothe Hahn. Tragikomödie in vier Akten@\strich\emph{Der rothe Hahn. Tragikomödie in vier Akten}|pwk}\eventindex{Deutsches Theater Berlin@\textbf{Deutsches Theater Berlin}!Uraufführung Der rothe Hahn, 27.11.1901@Uraufführung Der rothe Hahn, 27.11.1901|pwk}
                  von Gerhart Hauptmann\pwindex{Hauptmann, Gerhart 15.\,11.\,1862 Szczawno-Zdrój – 6.\,6.\,1946 Jagniątków@\textsc{Hauptmann, Gerhart} (15.\,11.\,1862 Szczawno-Zdrój – 6.\,6.\,1946 Jagniątków), \emph{Schriftsteller}|pwk} fand am 27. 11. 1901 am \emph{Deutschen
                     Theater Berlin}\orgindex{Deutsches Theater Berlin@Deutsches Theater Berlin|pwk} statt. Siehe auch XXXX Auszeichnungsfehler: Dokument L03094 nicht gefunden.}}}\label{K_L03092-2}« war gräßlich, \label{K_L03092-3v}\edtext{\textsc{Wolzogen\pwindex{Wolzogen, Ernst von 23.\,4.\,1855 Breslau – 30.\,7.\,1934 Puppling@\textsc{Wolzogen, Ernst von} (23.\,4.\,1855 Breslau – 30.\,7.\,1934 Puppling), \emph{Schriftsteller}|pw}}}{\lemma{\textnormal{\emph{Wolzogen}}}\Cendnote{\textnormal{Siehe XXXX Auszeichnungsfehler: Dokument L03059 nicht gefunden.
               }}}\label{K_L03092-3} »Überbrettl\orgindex{Überbrettl@Überbrettl|pw}« fürchterlich.\pend
           
\pstart
           Was Du mir über Dein \label{K_L03092-4v}\edtext{Ohr}{\lemma{\textnormal{\emph{Ohr}}}\Cendnote{\textnormal{Bezug auf Schnitzlers Otosklerose – einer Verknöcherung des Innenohrs mit
                  zunehmender Schwerhörigkeit –, an der er seit Herbst 1896 litt. Goldmann\pwindex{Goldmann, Paul 31.\,1.\,1865 Breslau – 25.\,9.\,1935 Wien@\textsc{Goldmann, Paul} (31.\,1.\,1865 Breslau – 25.\,9.\,1935 Wien), \emph{Schriftsteller, Journalist}|pwk} nahm Schnitzlers Klagen zumeist nicht ernst, vgl. XXXX Auszeichnungsfehler: Dokument L02806 nicht gefunden, XXXX Auszeichnungsfehler: Dokument L02823 nicht gefunden und XXXX Auszeichnungsfehler: Dokument L02839 nicht gefunden.}}}\label{K_L03092-4}{ }ſchreibſt, iſt betrübend. Aber ich {\pb}kann mir nicht
               helfen, ich habe{ }ſo eine Ahnung, daß \strikeout{D\textcolor{gray}{ir
                     das}} Du mit Deinem Ohrenleiden vielleicht viel weniger zu \strikeout{\textcolor{gray}{×}}{ }ſchaffen hätteſt,
               wenn Du nicht{ }ſo oft zum Ohrenarzt gingeſt. Verringerung der Hörweite! \strikeout{Ich} Das wechſelt, wie alle Sinnesfunktionen bei allen
               nervöſen Menſchen. Von der Verringerung der Hörweite müßten doch diejenigen etwas
               merken, die mit Dir{ }ſprechen. Ich habe davon auch nicht das leiſeſte Anzeichen
               bemerkt.\pend
           
\pstart
           Tauſend Grüße! {\\[\baselineskip]}Dein \spacefill\mbox{Paul Goldmann.}\pend
           \leftskip=0em{}\selectlanguage{ngerman}\endnumbering\briefempfaengerindex{Schnitzler, Arthur@\textsc{Schnitzler, Arthur}!zzzGoldmann, Paul@\emph{von Paul Goldmann}!1901-11-291@{29. 11. [1901]}|)be}\mylabel{L03092h}  \newcommand{\dateiname}{L03092}\newcommand{\titel}{Paul Goldmann an Arthur Schnitzler, 29. 11. [1901]}\newcommand{\editorInnen}{Martin Anton Müller und Laura Untner}%% latex-leseansicht-abspann.tex
%% Abspann für die Leseansicht.
%% Der Schalter \ifkorrekturansicht ist bereits durch den Vorspann gesetzt.

%% latex-abspann.tex
%% Gemeinsamer Abspann für Korrekturansicht und Leseansicht.
%% Setzt den Schalter \ifkorrekturansicht voraus (gesetzt in den
%% einbindenden Dateien latex-korrekturansicht-abspann.tex bzw.
%% latex-leseansicht-abspann.tex).
%% ---------------------------------------------------------------

\normalsize

% Das esempio-Environment wird nur in der Leseansicht benötigt
\ifkorrekturansicht\else
\newenvironment{esempio}[3]%
{
    \vspace{1.5ex}
    \rlap{\underline{#1}}
    \par
    \setlength{\parindent}{0cm}
    \nopagebreak
    \leftskip=#2cm
    \rightskip=#3cm
}
{
    \par
}
\fi

\doendnotes{C}
\bigskip
\vfill

\clearpage

\footnotesize

\ifkorrekturansicht
  \lohead{\textsc{register}}
\fi

% theindex-Environment neu definieren ohne reledmac
\makeatletter
\renewenvironment{theindex}{%
  \ifkorrekturansicht
    \section*{\indexname}%
  \else
    \subsubsection*{Index der erwähnten Entitäten}%
  \fi
  \setlength{\parindent}{0pt}%
  \setlength{\parskip}{0pt plus 0.3pt}%
  \let\item\@idxitem
}{%
  \ifkorrekturansicht\clearpage\fi
}
\makeatother

\IfFileExists{\jobname-pw.ind}{\input{\jobname-pw.ind}}{}

% Quellenangabe nur in der Leseansicht
\ifkorrekturansicht\else
% Fallback-Definitionen, falls die .tex-Datei \titel etc. nicht gesetzt hat
\providecommand{\titel}{}
\providecommand{\editorInnen}{}
\providecommand{\dateiname}{\jobname}

\vspace{3cm}

\vfill

\footnotesize
\textsc{Quelle}: \titel. Herausgegeben von {\editorInnen}. In: \emph{Arthur Schnitzler: Briefwechsel mit Autorinnen und Autoren}.
 Digitale Edition, https://schnitzler-briefe.acdh.oeaw.ac.at/{\dateiname}.html (Stand \today)
\fi

\end{document}


