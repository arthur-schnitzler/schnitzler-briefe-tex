%% latex-korrekturansicht-vorspann.tex
%% Vorspann für die Korrekturansicht.
%% Lädt die gemeinsame Datei latex-vorspann.tex mit gesetztem Schalter.

\newif\ifkorrekturansicht
\korrekturansichttrue

\input{../tex-inputs/latex-vorspann}


\section[ Paul Goldmann an Arthur Schnitzler, 29. 11. {[}1901{]}]{L03092 Paul Goldmann an Arthur Schnitzler, 29. 11. {[}1901{]}}
\nopagebreak\mylabel{L03092v}
\rehead{ }\normalsize\beginnumbering\briefempfaengerindex{Schnitzler, Arthur@\textsc{Schnitzler, Arthur}!zzzGoldmann, Paul@\emph{von Paul Goldmann}!1901-11-291@{29. 11. {[}1901{]}}|(be}
\toendnotes[C]{\smallbreak\pagebreak[2]}\Standort{DLA, A:Schnitzler, HS.NZ85.1.3171.}
\physDesc{Brief, 1 Blatt, 4 Seiten, 1564 Zeichen
\newline{}Handschrift: blaue Tinte, deutsche Kurrent
\newline{}Schnitzler: mit Bleistift das Jahr »901.« vermerkt }\toendnotes[C]{\smallbreak}
\pstart
           \raggedleft{}{\pb}\textcolor{gray}{\textbf{DESSAUERSTRASSE 19}}\oindex{Dessauer Strasse@\textbf{Dessauer Straße}, \emph{Straße (K.STR)}|pw}\pend
           
\pstart
           Berlin\oindex{Berlin@\textbf{Berlin}, \emph{P.PPLC}|pw}, 29. November.\pend
           
\pstart\center{}Mein lieber Freund,\pend\vspace{0.5em}
\pstart
           »Ungütig«! Du greifſt mich an, – greifſt mich an der Stelle an, wo ich am
               Verwundbarſten bin, – da, wo mein Lebensnerv ſitzt. Ich wehre mich gegen Deinen
               Angriff. Und das nennſt Du »ungütig aufnehmen«. Das iſt ein glänzender
               Luſtſpiel-Einfall, und Du ſollſt Dir ihn aufnotiren.\pend
           
\pstart
           »Zurechtweiſen«. Gewiß, \label{K_L03092-1v}\edtext{\textsc{Olga\pwindex{Schnitzler, Olga 17.01.1882 – 13.01.1970@\textsc{Schnitzler, Olga} (17.01.1882 – 13.01.1970), \emph{Schauspieler/Schauspielerin, Sänger/Sängerin}|pw}}}{\lemma{\textnormal{\emph{Olga}}}\Cendnote{\textnormal{Siehe Paul Goldmann an Arthur Schnitzler, 23. 11. [1901].
               }}}\label{K_L03092-1} hat mich nicht zurechtweiſen {\pb}\uline{gewollt}. Aber ſie hat’s \uline{gethan}. Und was mich ſo ſehr erregte, \strikeout{war,}
               war, daß ich plötzlich erkennen mußte, wie dieſes Mädchen\pwindex{Schnitzler, Olga 17.01.1882 – 13.01.1970@\textsc{Schnitzler, Olga} (17.01.1882 – 13.01.1970), \emph{Schauspieler/Schauspielerin, Sänger/Sängerin}|pwv}, dem ich in aufrichtigſter Freundſchaft zugethan
               bin, die \strikeout{d\textcolor{gray}{e}} die Freundin\pwindex{Schnitzler, Olga 17.01.1882 – 13.01.1970@\textsc{Schnitzler, Olga} (17.01.1882 – 13.01.1970), \emph{Schauspieler/Schauspielerin, Sänger/Sängerin}|pwv} meines
               liebſten Freundes iſt, weltenweit davon entfernt iſt, mich zu verſtehen!\pend
           
\pstart
           Im Übrigen iſt wirklich genug geredet; und es iſt ſehr blöd, daß wir uns da
               gegenſeitig allerlei Grobheiten ſchreiben, wo wir uns doch {\pb}wirklich Wichtigeres zu ſagen hätten.\pend
           
\pstart
           Mein lieber Freund, ich kann Dir heut nicht ſo
               ausführlich ſchreiben, als ich möchte. Ich habe wahnſinnig zu thun. In einigen Tagen
               hoffe ich Zeit zu einem längeren Brief zu finden.\pend
           
\pstart
           Der »\label{K_L03092-2v}\edtext{Rothe Hahn\pwindex{rothe Hahn. Tragikomoedie in vier Akten@\emph{Der rothe Hahn. Tragikomödie in vier Akten}|pw}}{\lemma{\textnormal{\emph{Rothe Hahn}}}\Cendnote{\textnormal{Die Uraufführung von 
                  \emph{Der rothe Hahn. Tragikomödie in vier Akten}\pwindex{rothe Hahn. Tragikomoedie in vier Akten@\emph{Der rothe Hahn. Tragikomödie in vier Akten}|pwk}
                  von Gerhart Hauptmann\pwindex{Hauptmann, Gerhart 15.11.1862 – 06.06.1946@\textsc{Hauptmann, Gerhart} (15.11.1862 – 06.06.1946), \emph{Schriftsteller/Schriftstellerin}|pwk} fand am 27. 11. 1901 am \emph{Deutschen
                     Theater Berlin}\orgindex{Deutsches Theater Berlin@Deutsches Theater Berlin|pwk} statt. Siehe auch Paul Goldmann an Arthur Schnitzler, 6. 12. [1901].}}}\label{K_L03092-2}« war gräßlich, \label{K_L03092-3v}\edtext{\textsc{Wolzogen\pwindex{Wolzogen, Ernst von 23.04.1855 – 30.07.1934@\textsc{Wolzogen, Ernst von} (23.04.1855 – 30.07.1934), \emph{Schriftsteller/Schriftstellerin}|pw}}}{\lemma{\textnormal{\emph{Wolzogen}}}\Cendnote{\textnormal{Siehe Paul Goldmann an Arthur Schnitzler, 18. 2. [1901].
               }}}\label{K_L03092-3} »Überbrettl\orgindex{Ueberbrettl@Überbrettl|pw}« fürchterlich.\pend
           
\pstart
           Was Du mir über Dein \label{K_L03092-4v}\edtext{Ohr}{\lemma{\textnormal{\emph{Ohr}}}\Cendnote{\textnormal{Bezug auf Schnitzlers Otosklerose – einer Verknöcherung des Innenohrs mit
                  zunehmender Schwerhörigkeit –, an der er seit Herbst 1896 litt. Goldmann\pwindex{Goldmann, Paul 31.01.1865 – 25.09.1935@\textsc{Goldmann, Paul} (31.01.1865 – 25.09.1935), \emph{Schriftsteller/Schriftstellerin, Journalist/Journalistin}|pwk} nahm Schnitzlers Klagen zumeist nicht ernst, vgl. Paul Goldmann an Arthur Schnitzler, 22. 3. [1897], 13. 9. 1897 und 28. 2. [1898].}}}\label{K_L03092-4}
               ſchreibſt, iſt betrübend. Aber ich {\pb}kann mir nicht
               helfen, ich habe ſo eine Ahnung, daß \strikeout{D\textcolor{gray}{ir
                     das}} Du mit Deinem Ohrenleiden vielleicht viel weniger zu \strikeout{\textcolor{gray}{×}} ſchaffen hätteſt,
               wenn Du nicht ſo oft zum Ohrenarzt gingeſt. Verringerung der Hörweite! \strikeout{Ich} Das wechſelt, wie alle Sinnesfunktionen bei allen
               nervöſen Menſchen. Von der Verringerung der Hörweite müßten doch diejenigen etwas
               merken, die mit Dir ſprechen. Ich habe davon auch nicht das leiſeſte Anzeichen
               bemerkt.\pend
           
\pstart
           Tauſend Grüße! {\\[\baselineskip]}Dein \spacefill\mbox{Paul Goldmann.}\pend
           \leftskip=0em{}\selectlanguage{ngerman}\endnumbering\briefempfaengerindex{Schnitzler, Arthur@\textsc{Schnitzler, Arthur}!zzzGoldmann, Paul@\emph{von Paul Goldmann}!1901-11-291@{29. 11. {[}1901{]}}|)be}\mylabel{L03092h}  \normalsize

\doendnotes{C}
\bigskip
\vfill

\clearpage

\footnotesize

\lohead{\textsc{register}}

% Definiere theindex-Environment komplett neu ohne reledmac
\makeatletter
\renewenvironment{theindex}{%
  \section*{\indexname}%
  \setlength{\parindent}{0pt}%
  \setlength{\parskip}{0pt plus 0.3pt}%
  \let\item\@idxitem
}{%
  \clearpage
}
\makeatother

\IfFileExists{\jobname-pw.ind}{\input{\jobname-pw.ind}}{}

\end{document}

      