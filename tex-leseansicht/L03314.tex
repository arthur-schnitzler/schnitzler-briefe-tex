%% latex-leseansicht-vorspann.tex
%% Vorspann für die Leseansicht.
%% Lädt die gemeinsame Datei latex-vorspann.tex mit nicht gesetztem Schalter.

\newif\ifkorrekturansicht
\korrekturansichtfalse

\input{../tex-inputs/latex-vorspann}

\begin{center}
            \textcolor{red}{ENTWURF, NICHT FERTIG KORRIGIERT}
                      \end{center}
            
         
         \renewcommand{\erwaehntePersonen}{Personen: Richard Beer-Hofmann, Leo Van-Jung}
         \renewcommand{\erwaehnteOrte}{Orte: Frankgasse, Hauptbahnhof Salzburg, Hotel Castiglione, IX., Alsergrund, München, Wien, Zürich}
         \renewcommand{\erwaehnteWerke}{Werke: Olga Frohgemuth. Erzählung}
               \section[Felix Salten an Arthur Schnitzler, 2{[}3{]}. 6. 1901]{ Felix Salten an Arthur Schnitzler, 2{[}3{]}. 6. 1901}\nopagebreak\mylabel{v}\rehead{ }\begin{ledgroupsized}[t]{13cm}\normalsize\beginnumbering \toendnotes[C]{\smallbreak\pagebreak[2]} \Standort{CUL, Schnitzler, B 89, A 2.}
\physDesc{Postkarte, 743 Zeichen
\newline{}Handschrift: Bleistift, lateinische Kurrent
\newline{}Ordnung: mit Bleistift von unbekannter Hand nummeriert:
                                    »138« }\toendnotes[C]{\smallbreak}\pstart{}{\pb}Herrn D\textsuperscript{r} Arthur Schnitzler\pend{}\pstart{}Wien IX\oindex{IX., Alsergrund@\textbf{IX., Alsergrund}|pw}\pend{}\pstart{}Frankgaße 1\oindex{Frankgasse@\textbf{Frankgasse}|pw}\pend{}{\bigskip}\pstart
           \raggedleft{}{\pb}Salzburg, Bahnhof\oindex{Hauptbahnhof Salzburg@\textbf{Hauptbahnhof Salzburg}|pw}, \label{K_L03314-1v}\edtext{22. Juni 01}{\lemma{\textnormal{\emph{22. Juni 01}}}\Cendnote{\textnormal{Die genaue Datierung scheint
                     widersprüchlich, da die Karte den Poststempel vom 23. 6. 1901
                     trägt. Wahrscheinlich scheint, dass sie in der Nacht vom 22. auf
                     den 23. verfasst wurde und zwar, wenn man die Angabe der Uhrzeit
                     heranzieht, schon am 23.}}}\label{K_L03314-1h}. \pend
           \pstart
           \raggedleft{}½ 2 Nachts\pend
           \pstart
           Lieber Freund, ich komme soeben von München\oindex{Muenchen@\textbf{München}|pw} herüber, warte hier auf den Zug nach Zürich\oindex{Zuerich@\textbf{Zürich}|pw}. Hätte ich Ihre Adreße hier gewußt, ich hätte Ihnen gerne
               geschrieben, dass Sie auf die Bahn kommen, denn ich bin seit 12 Uhr Nachts hier.
               Heute früh erhielt ich in München\oindex{Muenchen@\textbf{München}|pw} Ihren Brief,
               der mir – wie alles – nachgesandt wurde.\pend
           \pstart
           Meine nächste Adreße ist Paris, Hotel
                  Castiglione\oindex{Hotel Castiglione@\textbf{Hotel Castiglione}|pw}. Ich freue mich, dass Sie arbeiten. Ich arbeite hoffentlich auf
               der Reise meinen Professor\pwindex{Salten, Felix 06.09.1869 – 08.10.1945@\textsc{Salten, Felix} (06.09.1869 – 08.10.1945), \emph{Schriftsteller, Journalist}!Olga Frohgemuth. Erzaehlung1910@\strich\emph{Olga Frohgemuth. Erzählung} {[}1910{]}|pwuv}, wozu ich viel Lust habe. \pend
           \pstart
           Wissen Sie, wo Beer-Hofmann\pwindex{Beer-Hofmann, Richard 1866-07-11 – 1945-09-26@\textsc{Beer-Hofmann, Richard} (1866-07-11 – 1945-09-26), \emph{Schriftsteller}|pw} ist? Ich möchte
               ihm drängen, den Text zu Van-Jung\pwindex{Van-Jung, Leo 15.10.1866 – 02.07.1939@\textsc{Van-Jung, Leo} (15.10.1866 – 02.07.1939), \emph{Gesangspädagoge, Mathematiker}|pw}s Pfeifertrio
               fertig zu stellen. \pend
           \pstart
           Leben Sie wol und laßen sich’s gut gehen, und grüßen von mir. {\\[\baselineskip]}Herzlichst
               Ihr \spacefill\mbox{Salten.}\pend
           \leftskip=0em{}
         
         \endnumbering\mylabel{h}\end{ledgroupsized}\begin{anhang}\end{anhang}\newcommand{\dateiname}{L03314}\newcommand{\titel}{Felix Salten an Arthur Schnitzler, 2[3]. 6. 1901}\newcommand{\editorInnen}{Martin Anton Müller und Laura Untner}%% latex-leseansicht-abspann.tex
%% Abspann für die Leseansicht.
%% Der Schalter \ifkorrekturansicht ist bereits durch den Vorspann gesetzt.

%% latex-abspann.tex
%% Gemeinsamer Abspann für Korrekturansicht und Leseansicht.
%% Setzt den Schalter \ifkorrekturansicht voraus (gesetzt in den
%% einbindenden Dateien latex-korrekturansicht-abspann.tex bzw.
%% latex-leseansicht-abspann.tex).
%% ---------------------------------------------------------------

\normalsize

% Das esempio-Environment wird nur in der Leseansicht benötigt
\ifkorrekturansicht\else
\newenvironment{esempio}[3]%
{
    \vspace{1.5ex}
    \rlap{\underline{#1}}
    \par
    \setlength{\parindent}{0cm}
    \nopagebreak
    \leftskip=#2cm
    \rightskip=#3cm
}
{
    \par
}
\fi

\doendnotes{C}
\bigskip
\vfill

\clearpage

\footnotesize

\ifkorrekturansicht
  \lohead{\textsc{register}}
\fi

% theindex-Environment neu definieren ohne reledmac
\makeatletter
\renewenvironment{theindex}{%
  \ifkorrekturansicht
    \section*{\indexname}%
  \else
    \subsubsection*{Index der erwähnten Entitäten}%
  \fi
  \setlength{\parindent}{0pt}%
  \setlength{\parskip}{0pt plus 0.3pt}%
  \let\item\@idxitem
}{%
  \ifkorrekturansicht\clearpage\fi
}
\makeatother

\IfFileExists{\jobname-pw.ind}{\input{\jobname-pw.ind}}{}

% Quellenangabe nur in der Leseansicht
\ifkorrekturansicht\else
% Fallback-Definitionen, falls die .tex-Datei \titel etc. nicht gesetzt hat
\providecommand{\titel}{}
\providecommand{\editorInnen}{}
\providecommand{\dateiname}{\jobname}

\vspace{3cm}

\vfill

\footnotesize
\textsc{Quelle}: \titel. Herausgegeben von {\editorInnen}. In: \emph{Arthur Schnitzler: Briefwechsel mit Autorinnen und Autoren}.
 Digitale Edition, https://schnitzler-briefe.acdh.oeaw.ac.at/{\dateiname}.html (Stand \today)
\fi

\end{document}


      