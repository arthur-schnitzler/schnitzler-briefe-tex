%% latex-leseansicht-vorspann.tex
%% Vorspann für die Leseansicht.
%% Lädt die gemeinsame Datei latex-vorspann.tex mit nicht gesetztem Schalter.

\newif\ifkorrekturansicht
\korrekturansichtfalse

\input{../tex-inputs/latex-vorspann}


\section[ Felix Salten an Arthur Schnitzler, 2{[}3{]}. 6. 1901]{L03314 Felix Salten an Arthur Schnitzler,  2[3]. 6. 1901}
\nopagebreak\mylabel{L03314v}
\rehead{ }\normalsize\beginnumbering\briefempfaengerindex{Schnitzler, Arthur@\textsc{Schnitzler, Arthur}!zzzSalten, Felix@\emph{von Felix Salten}!1901-06-231@{2[3]. 6. 1901}|(be}
\toendnotes[C]{\smallbreak\pagebreak[2]}
\correspDesc{Versand  durch Felix Salten am 2[3]. 6. 1901 in Salzburg
\newline{}Übermittlung  am 23. 6. 1901 in Salzburg
\newline{}Umleitung  am 24. 6. 1901 in Wien
\newline{}Erhalt  durch Arthur Schnitzler am [25. 6. 1901?] in Salzburg}\toendnotes[C]{\smallbreak}
\Standort{CUL, Schnitzler, B 89, A 2.}
\physDesc{Postkarte, 738 Zeichen
\newline{}Handschrift: Bleistift, lateinische Kurrent
\newline{}Versand: Stempel: »\nobreak{}\oindex{Hauptbahnhof Salzburg@\textbf{Hauptbahnhof Salzburg}, \emph{Bahnhofsgebäude}|pwk}Salzburg-Bahnhof, 23/6 \textcolor{gray}{01}, 3-F.\nobreak{}«. Stempel: »\nobreak{}\oindex{IX., Alsergrund@\textbf{IX., Alsergrund}, \emph{Verwaltungsgebiet}|pwk}Wien 9/3 72, 24. 6. 01, 8. \textcolor{gray}{V}, Bestellt\nobreak{}«.  
\newline{}Ordnung: mit Bleistift von unbekannter Hand nummeriert: »138« }\toendnotes[C]{\smallbreak}\pstart{}{\pb}Herrn D\textsuperscript{r} Arthur Schnitzler\pend{}\pstart{}Wien IX.\oindex{IX., Alsergrund@\textbf{IX., Alsergrund}, \emph{Verwaltungsgebiet}|pw}\pend{}\pstart{}Frankgaße 1\oindex{Wien@\textbf{Wien}!IX., Alsergrund@\textbf{IX., Alsergrund}!Frankgasse 1@\textbf{Frankgasse 1}, \emph{Wohngebäude}|pw}\pend{}{\bigskip}\vspace{1em}
\pstart
           \raggedleft{}{\pb}Salzburg, Bahnhof\oindex{Hauptbahnhof Salzburg@\textbf{Hauptbahnhof Salzburg}, \emph{Bahnhofsgebäude}|pw}, \label{K_L03314-1v}\edtext{22. Juni 01}{\lemma{\textnormal{\emph{22. Juni 01}}}\Cendnote{\textnormal{Zwischen der Datums- und der
                     Uhrzeitangabe besteht ein Widerspruch, wenn man den Poststempel vom 23. 6. 1901 hinzuzieht. Es ist davon auszugehen,
                     dass Salten\pwindex{Salten, Felix 6.\,9.\,1869 Budapest – 8.\,10.\,1945 Zürich@\textsc{Salten, Felix} (6.\,9.\,1869 Budapest – 8.\,10.\,1945 Zürich), \emph{Schriftsteller, Journalist, Chefredakteur}|pwk} die Karte in der Nacht
                     vom 22. auf den 23. verfasste, korrekterweise also bereits am 23. schrieb. Alternativ wäre die Karte unbearbeitet
                     über 24h liegengeblieben.}}}\label{K_L03314-1}.\pend
           
\pstart
           \raggedleft{}½ 2. Nachts.\pend
           \vspace{0.5em}
\pstart
           Lieber Freund, ich komme soeben von München\oindex{München@\textbf{München}|pw} herüber, warte hier\oindex{Hauptbahnhof Salzburg@\textbf{Hauptbahnhof Salzburg}, \emph{Bahnhofsgebäude}|pwv} auf den Zug nach Zürich\oindex{Zürich@\textbf{Zürich}|pw}. Hätte ich
               Ihre \label{K_L03314-2v}\edtext{Adreße}{\lemma{\textnormal{\emph{Adreße}}}\Cendnote{\textnormal{Schnitzler war seit dem 12. 6. 1901 in Salzburg\oindex{Salzburg@\textbf{Salzburg}, \emph{Verwaltungsgebiet}|pwk}.}}}\label{K_L03314-2} hier gewußt, ich hätte Ihnen
               gerne geschrieben\substVorne{}\textsuperscript{,}\substDazwischen{} (\substHinten{}dass Sie auf die Bahn\oindex{Hauptbahnhof Salzburg@\textbf{Hauptbahnhof Salzburg}, \emph{Bahnhofsgebäude}|pwv}
                  kommen{[}){]}, denn ich bin seit 12 Uhr Nachts hier.
                  Heute{ }früh erhielt ich in München\oindex{München@\textbf{München}|pw} Ihren
               Brief, der mir, – wie alles – nachgesendet wurde. Meine nächste Adreße ist Paris, Hotel
                  Castiglione\oindex{Hotel Castiglione@\textbf{Hotel Castiglione}, \emph{Hotel}|pw}. Ich freue mich, dass Sie arbeiten. Ich arbeite hoffentlich auf
               der Reise meinen \label{K_L03314-3v}\edtext{Professor\pwindex{Salten, Felix 6.\,9.\,1869 Budapest – 8.\,10.\,1945 Zürich@\textsc{Salten, Felix} (6.\,9.\,1869 Budapest – 8.\,10.\,1945 Zürich), \emph{Schriftsteller, Journalist, Chefredakteur}!Olga Frohgemuth. Erzählung@\strich\emph{Olga Frohgemuth. Erzählung}|pwuv}}{\lemma{\textnormal{\emph{Professor}}}\Cendnote{\textnormal{Die Erzählung 
                  \emph{Olga Frohgemuth}\pwindex{Salten, Felix 6.\,9.\,1869 Budapest – 8.\,10.\,1945 Zürich@\textsc{Salten, Felix} (6.\,9.\,1869 Budapest – 8.\,10.\,1945 Zürich), \emph{Schriftsteller, Journalist, Chefredakteur}!Olga Frohgemuth. Erzählung@\strich\emph{Olga Frohgemuth. Erzählung}|pwk}?}}}\label{K_L03314-3}, wozu ich viel Lust
               habe.\pend
           
\pstart
           Wissen Sie, wo \label{K_L03314-4v}\edtext{Beer-Hofmann\pwindex{Beer-Hofmann, Richard 11.\,7.\,1866 Wien – 26.\,9.\,1945 New York City@\textsc{Beer-Hofmann, Richard} (11.\,7.\,1866 Wien – 26.\,9.\,1945 New York City), \emph{Schriftsteller}|pw}}{\lemma{\textnormal{\emph{Beer-Hofmann}}}\Cendnote{\textnormal{Richard Beer-Hofmann\pwindex{Beer-Hofmann, Richard 11.\,7.\,1866 Wien – 26.\,9.\,1945 New York City@\textsc{Beer-Hofmann, Richard} (11.\,7.\,1866 Wien – 26.\,9.\,1945 New York City), \emph{Schriftsteller}|pwk} hielt sich
                  höchstwahrscheinlich bereits in Pörtschach am
                     Wörthersee\oindex{Pörtschach am Wörthersee@\textbf{Pörtschach am Wörthersee}|pwk} auf, vgl. XXXX Auszeichnungsfehler: Dokument L01133 nicht gefunden.}}}\label{K_L03314-4} ist? Ich möchte ihn drängen, den \label{K_L03314-5v}\edtext{Text\pwindex{Beer-Hofmann, Richard 11.\,7.\,1866 Wien – 26.\,9.\,1945 New York City@\textsc{Beer-Hofmann, Richard} (11.\,7.\,1866 Wien – 26.\,9.\,1945 New York City), \emph{Schriftsteller}!?? [Text zu Van-Jungs Pfeifertrio]@\strich\emph{?? [Text zu Van-Jungs Pfeifertrio]}|pwv} zu Van-Jungs\pwindex{Van-Jung, Leo 15.\,10.\,1866 Odessa – 2.\,7.\,1939 Riga@\textsc{Van-Jung, Leo} (15.\,10.\,1866 Odessa – 2.\,7.\,1939 Riga), \emph{Gesangspädagoge, Mathematiker}|pw} Pfeifertrio}{\lemma{\textnormal{\emph{Text … Pfeifertrio}}}\Cendnote{\textnormal{nicht ermittelt}}}\label{K_L03314-5} fertig zu stellen.\pend
           
\pstart
           Leben Sie wol und laßen sich’s gut gehen, und grüßen von mir. {\\[\baselineskip]}Herzlichst
               Ihr \spacefill\mbox{Salten.}\pend
           \leftskip=0em{}\selectlanguage{ngerman}\endnumbering\briefempfaengerindex{Schnitzler, Arthur@\textsc{Schnitzler, Arthur}!zzzSalten, Felix@\emph{von Felix Salten}!1901-06-231@{2[3]. 6. 1901}|)be}\mylabel{L03314h}  \newcommand{\dateiname}{L03314}\newcommand{\titel}{Felix Salten an Arthur Schnitzler, 2[3]. 6. 1901}\newcommand{\editorInnen}{Martin Anton Müller und Laura Untner}%% latex-leseansicht-abspann.tex
%% Abspann für die Leseansicht.
%% Der Schalter \ifkorrekturansicht ist bereits durch den Vorspann gesetzt.

%% latex-abspann.tex
%% Gemeinsamer Abspann für Korrekturansicht und Leseansicht.
%% Setzt den Schalter \ifkorrekturansicht voraus (gesetzt in den
%% einbindenden Dateien latex-korrekturansicht-abspann.tex bzw.
%% latex-leseansicht-abspann.tex).
%% ---------------------------------------------------------------

\normalsize

% Das esempio-Environment wird nur in der Leseansicht benötigt
\ifkorrekturansicht\else
\newenvironment{esempio}[3]%
{
    \vspace{1.5ex}
    \rlap{\underline{#1}}
    \par
    \setlength{\parindent}{0cm}
    \nopagebreak
    \leftskip=#2cm
    \rightskip=#3cm
}
{
    \par
}
\fi

\doendnotes{C}
\bigskip
\vfill

\clearpage

\footnotesize

\ifkorrekturansicht
  \lohead{\textsc{register}}
\fi

% theindex-Environment neu definieren ohne reledmac
\makeatletter
\renewenvironment{theindex}{%
  \ifkorrekturansicht
    \section*{\indexname}%
  \else
    \subsubsection*{Index der erwähnten Entitäten}%
  \fi
  \setlength{\parindent}{0pt}%
  \setlength{\parskip}{0pt plus 0.3pt}%
  \let\item\@idxitem
}{%
  \ifkorrekturansicht\clearpage\fi
}
\makeatother

\IfFileExists{\jobname-pw.ind}{\input{\jobname-pw.ind}}{}

% Quellenangabe nur in der Leseansicht
\ifkorrekturansicht\else
% Fallback-Definitionen, falls die .tex-Datei \titel etc. nicht gesetzt hat
\providecommand{\titel}{}
\providecommand{\editorInnen}{}
\providecommand{\dateiname}{\jobname}

\vspace{3cm}

\vfill

\footnotesize
\textsc{Quelle}: \titel. Herausgegeben von {\editorInnen}. In: \emph{Arthur Schnitzler: Briefwechsel mit Autorinnen und Autoren}.
 Digitale Edition, https://schnitzler-briefe.acdh.oeaw.ac.at/{\dateiname}.html (Stand \today)
\fi

\end{document}


