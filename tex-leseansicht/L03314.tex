%% latex-korrekturansicht-vorspann.tex
%% Vorspann für die Korrekturansicht.
%% Lädt die gemeinsame Datei latex-vorspann.tex mit gesetztem Schalter.

\newif\ifkorrekturansicht
\korrekturansichttrue

\input{../tex-inputs/latex-vorspann}


\section[ Felix Salten an Arthur Schnitzler, 2{[}3{]}. 6. 1901]{L03314 Felix Salten an Arthur Schnitzler, 2{[}3{]}. 6. 1901}
\nopagebreak\mylabel{L03314v}
\rehead{ }\normalsize\beginnumbering\briefempfaengerindex{Schnitzler, Arthur@\textsc{Schnitzler, Arthur}!zzzSalten, Felix@\emph{von Felix Salten}!1901-06-231@{2{[}3{]}. 6. 1901}|(be}
\toendnotes[C]{\smallbreak\pagebreak[2]}\Standort{CUL, Schnitzler, B 89, A 2.}
\physDesc{Postkarte, 738 Zeichen
\newline{}Handschrift: Bleistift, lateinische Kurrent
\newline{}Versand: Stempel: »\nobreak{}\oindex{Hauptbahnhof Salzburg@\textbf{Hauptbahnhof Salzburg}, \emph{Bahnhofsgebäude (K.BHF)}|pwk}Salzburg-Bahnhof, 23/6 \textcolor{gray}{01}, 3-F.\nobreak{}«. Stempel: »\nobreak{}\oindex{IX., Alsergrund@\textbf{IX., Alsergrund}, \emph{A.ADM3}|pwk}Wien 9/3 72, 24. 6. 01, 8. \textcolor{gray}{V}, Bestellt\nobreak{}«.  
\newline{}Ordnung: mit Bleistift von unbekannter Hand nummeriert: »138« }\toendnotes[C]{\smallbreak}\pstart{}{\pb}Herrn D\textsuperscript{r} Arthur Schnitzler\pend{}\pstart{}Wien IX.\oindex{IX., Alsergrund@\textbf{IX., Alsergrund}, \emph{A.ADM3}|pw}\pend{}\pstart{}Frankgaße 1\oindex{Frankgasse 1@\textbf{Frankgasse 1}, \emph{Wohngebäude (K.WHS)}|pw}\pend{}{\bigskip}\vspace{1em}
\pstart
           \raggedleft{}{\pb}Salzburg, Bahnhof\oindex{Hauptbahnhof Salzburg@\textbf{Hauptbahnhof Salzburg}, \emph{Bahnhofsgebäude (K.BHF)}|pw}, \label{K_L03314-1v}\edtext{22. Juni 01}{\lemma{\textnormal{\emph{22. Juni 01}}}\Cendnote{\textnormal{Zwischen der Datums- und der
                     Uhrzeitangabe besteht ein Widerspruch, wenn man den Poststempel vom 23. 6. 1901 hinzuzieht. Es ist davon auszugehen,
                     dass Salten\pwindex{Salten, Felix 06.09.1869 – 08.10.1945@\textsc{Salten, Felix} (06.09.1869 – 08.10.1945), \emph{Schriftsteller/Schriftstellerin, Journalist/Journalistin, Chefredakteur/Chefredakteurin}|pwk} die Karte in der Nacht
                     vom 22. auf den 23. verfasste, korrekterweise also bereits am 23. schrieb. Alternativ wäre die Karte unbearbeitet
                     über 24h liegengeblieben.}}}\label{K_L03314-1}.\pend
           
\pstart
           \raggedleft{}½ 2. Nachts. \pend
           \vspace{0.5em}
\pstart
           Lieber Freund, ich komme soeben von München\oindex{Muenchen@\textbf{München}, \emph{P.PPLA}|pw} herüber, warte hier\oindex{Hauptbahnhof Salzburg@\textbf{Hauptbahnhof Salzburg}, \emph{Bahnhofsgebäude (K.BHF)}|pwv} auf den Zug nach Zürich\oindex{Zuerich@\textbf{Zürich}, \emph{P.PPLA}|pw}. Hätte ich
               Ihre \label{K_L03314-2v}\edtext{Adreße}{\lemma{\textnormal{\emph{Adreße}}}\Cendnote{\textnormal{Schnitzler war seit dem 12. 6. 1901 in Salzburg\oindex{Salzburg@\textbf{Salzburg}, \emph{A.ADM2}|pwk}.}}}\label{K_L03314-2} hier gewußt, ich hätte Ihnen
               gerne geschrieben\substVorne{}\textsuperscript{,}\substDazwischen{} (\substHinten{}dass Sie auf die Bahn\oindex{Hauptbahnhof Salzburg@\textbf{Hauptbahnhof Salzburg}, \emph{Bahnhofsgebäude (K.BHF)}|pwv}
                  kommen{[}){]}, denn ich bin seit 12 Uhr Nachts hier.
                  Heute{ }früh erhielt ich in München\oindex{Muenchen@\textbf{München}, \emph{P.PPLA}|pw} Ihren
               Brief, der mir, – wie alles – nachgesendet wurde. Meine nächste Adreße ist Paris, Hotel
                  Castiglione\oindex{Hotel Castiglione@\textbf{Hotel Castiglione}, \emph{Hotel (K.HTL)}|pw}. Ich freue mich, dass Sie arbeiten. Ich arbeite hoffentlich auf
               der Reise meinen \label{K_L03314-3v}\edtext{Professor\pwindex{Olga Frohgemuth. Erzaehlung@\emph{Olga Frohgemuth. Erzählung}|pwuv}}{\lemma{\textnormal{\emph{Professor}}}\Cendnote{\textnormal{Die Erzählung 
                  \emph{Olga Frohgemuth}\pwindex{Olga Frohgemuth. Erzaehlung@\emph{Olga Frohgemuth. Erzählung}|pwk}?}}}\label{K_L03314-3}, wozu ich viel Lust
               habe.\pend
           
\pstart
           Wissen Sie, wo \label{K_L03314-4v}\edtext{Beer-Hofmann\pwindex{Beer-Hofmann, Richard 1866-07-11 – 1945-09-26@\textsc{Beer-Hofmann, Richard} (1866-07-11 – 1945-09-26), \emph{Schriftsteller/Schriftstellerin}|pw}}{\lemma{\textnormal{\emph{Beer-Hofmann}}}\Cendnote{\textnormal{Richard Beer-Hofmann\pwindex{Beer-Hofmann, Richard 1866-07-11 – 1945-09-26@\textsc{Beer-Hofmann, Richard} (1866-07-11 – 1945-09-26), \emph{Schriftsteller/Schriftstellerin}|pwk} hielt sich
                  höchstwahrscheinlich bereits in Pörtschach am
                     Wörthersee\oindex{Poertschach am Woerthersee@\textbf{Pörtschach am Wörthersee}, \emph{P.PPL}|pwk} auf, vgl. Arthur Schnitzler an Richard Beer-Hofmann, 25. 6. 1901.}}}\label{K_L03314-4} ist? Ich möchte ihn drängen, den \label{K_L03314-5v}\edtext{Text\pwindex{?? [Text zu Van-Jungs Pfeifertrio]@\emph{?? [Text zu Van-Jungs Pfeifertrio]}|pwv} zu Van-Jungs\pwindex{Van-Jung, Leo 15.10.1866 – 02.07.1939@\textsc{Van-Jung, Leo} (15.10.1866 – 02.07.1939), \emph{Gesangspädagoge/Gesangspädagogin, Mathematiker/Mathematikerin}|pw} Pfeifertrio}{\lemma{\textnormal{\emph{Text … Pfeifertrio}}}\Cendnote{\textnormal{nicht ermittelt}}}\label{K_L03314-5} fertig zu stellen.\pend
           
\pstart
           Leben Sie wol und laßen sich’s gut gehen, und grüßen von mir. {\\[\baselineskip]}Herzlichst
               Ihr \spacefill\mbox{Salten.}\pend
           \leftskip=0em{}\selectlanguage{ngerman}\endnumbering\briefempfaengerindex{Schnitzler, Arthur@\textsc{Schnitzler, Arthur}!zzzSalten, Felix@\emph{von Felix Salten}!1901-06-231@{2{[}3{]}. 6. 1901}|)be}\mylabel{L03314h}  \normalsize

\doendnotes{C}
\bigskip
\vfill

\clearpage

\footnotesize

\lohead{\textsc{register}}

% Definiere theindex-Environment komplett neu ohne reledmac
\makeatletter
\renewenvironment{theindex}{%
  \section*{\indexname}%
  \setlength{\parindent}{0pt}%
  \setlength{\parskip}{0pt plus 0.3pt}%
  \let\item\@idxitem
}{%
  \clearpage
}
\makeatother

\IfFileExists{\jobname-pw.ind}{\input{\jobname-pw.ind}}{}

\end{document}

      