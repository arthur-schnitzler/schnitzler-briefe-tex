\input{../tex-inputs/latex-pdf-vorspann}
\begin{center}
            \textcolor{red}{ENTWURF. ENTZIFFERUNG NOCH NICHT KORREKTURGELESEN}
                      \end{center}
            
               \section[Richard Beer-Hofmann an Arthur Schnitzler, 13. 7. 1906]{ Richard Beer-Hofmann an Arthur Schnitzler, 13. 7. 1906}\nopagebreak\mylabel{v}\rehead{ }\begin{ledgroupsized}[t]{13cm}\normalsize\beginnumbering\briefempfaengerindex{Schnitzler, Arthur@\textsc{Schnitzler, Arthur}!zzzBeer-Hofmann, Richard@\emph{von Richard Beer-Hofmann}!1906-07-132@{13. 7. 1906}|(be} \toendnotes[C]{\smallbreak\pagebreak[2]} \Standort{CUL, Schnitzler, B 8.}
\physDesc{Bildpostkarte
\newline{}Handschrift: Bleistift, lateinische Kurrent\newline{}Versand: 1) Stempel: »\nobreak{}\oindex{Rodaun@\textbf{Rodaun}|pwk}{[}Roda{]}un\nobreak{}«.  2) Stempel: »\nobreak{}\oindex{Helsingør@\textbf{Helsingør}|pwk}Helsingør, 16. 7. 06, 10-11E\nobreak{}«. \newline{}Ordnung: mit Bleistift von unbekannter Hand nummeriert: »206« }\buchAbdrucke{\weitereDrucke{Arthur Schnitzler, Richard Beer-Hofmann: \emph{Briefwechsel 1891–1931}. Hg. Konstanze Fliedl. Wien, Zürich: \emph{Europaverlag} 1992, S. 179.} }\toendnotes[C]{\smallbreak}\pstart{}{\pb}Herrn\pend{}\pstart{}Dr Arthur Schnitzler\pend{}\pstart{}Marienlyst\oindex{Marienlyst@\textbf{Marienlyst}|pw}\pend{}\pstart{}Kurhaus\oindex{Kurhotellet@\textbf{Kurhotellet}|pw}\pend{}\pstart{}Dänemark\oindex{Daenemark@\textbf{Dänemark}|pw}\pend{}{\bigskip}\pstart
           \noindent{}\centering{}{\pb}\textcolor{gray}{\textbf{Kirche und Dreifaltigkeitssäule\oindex{Stift Heiligenkreuz@\textbf{Stift Heiligenkreuz}|pw}.}}\hspace*{1.5em}\textcolor{gray}{\textbf{Gruss aus Heiligenkreuz\oindex{Heiligenkreuz@\textbf{Heiligenkreuz}|pw} im Wienerwald\oindex{Wienerwald@\textbf{Wienerwald}|pw}.}}\pend
           \pstart
           \raggedleft{}13/VII 06\pend
           \stanza{}»Lang ist es her«\newverse{}spielt ein Kind\newverse{}auf dem versti{\geminationm}ten Klavier\newverse{}»\label{K_L01611_1v}\edtext{Es ist ein
                     alter Klimperkasten}{\lemma{\textnormal{\emph{Es … Klimperkasten}}}\Cendnote{\textnormal{Es handelt sich um die letzten beiden Verse von Schnitzler\pwindex{Schnitzler, Arthur 15.05.1862 – 21.10.1931@\textsc{Schnitzler, Arthur} (15.05.1862 – 21.10.1931), \emph{Schriftsteller, Mediziner}|pwk}s Gedicht \emph{Am Flügel}\pwindex{Schnitzler, Arthur 15.05.1862 – 21.10.1931@\textsc{Schnitzler, Arthur} (15.05.1862 – 21.10.1931), \emph{Schriftsteller, Mediziner}!Am Fluegel1. 1. 1889@\strich\emph{Am Flügel} {[}1. 1. 1889{]}|pwk} mit der Textabweichung »alter« statt dem in
                        der ersten gedruckten Fassung stehenden »dummer«.}}}\label{K_L01611_1h}\pwindex{Schnitzler, Arthur 15.05.1862 – 21.10.1931@\textsc{Schnitzler, Arthur} (15.05.1862 – 21.10.1931), \emph{Schriftsteller, Mediziner}!Am Fluegel1. 1. 1889@\strich\emph{Am Flügel} {[}1. 1. 1889{]}|pwv}\newverse{}wahrscheinlich war er’s damals
                     schon!«\pwindex{Schnitzler, Arthur 15.05.1862 – 21.10.1931@\textsc{Schnitzler, Arthur} (15.05.1862 – 21.10.1931), \emph{Schriftsteller, Mediziner}!Am Fluegel1. 1. 1889@\strich\emph{Am Flügel} {[}1. 1. 1889{]}|pwv}\stanzaend{}\pstart \spacefill\mbox{Richard}\pend{}\endnumbering\briefempfaengerindex{Schnitzler, Arthur@\textsc{Schnitzler, Arthur}!zzzBeer-Hofmann, Richard@\emph{von Richard Beer-Hofmann}!1906-07-132@{13. 7. 1906}|)be}\mylabel{h}\end{ledgroupsized}  \newcommand{\dateiname}{L01611}\newcommand{\titel}{Richard Beer-Hofmann an Arthur Schnitzler, 13. 7. 1906}\newcommand{\editorInnen}{Martin Anton Müller und Gerd-Hermann Susen}\input{../tex-inputs/latex-pdf-abspann}
      