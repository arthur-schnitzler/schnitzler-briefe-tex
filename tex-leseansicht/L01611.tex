%% latex-korrekturansicht-vorspann.tex
%% Vorspann für die Korrekturansicht.
%% Lädt die gemeinsame Datei latex-vorspann.tex mit gesetztem Schalter.

\newif\ifkorrekturansicht
\korrekturansichttrue

\input{../tex-inputs/latex-vorspann}


\section[Richard Beer-Hofmann an Arthur Schnitzler, 13. 7. 1906]{L01611 Richard Beer-Hofmann an Arthur Schnitzler, 13. 7. 1906}
\nopagebreak\mylabel{L01611v}
\rehead{ }\normalsize\beginnumbering\briefempfaengerindex{Schnitzler, Arthur@\textsc{Schnitzler, Arthur}!zzzBeer-Hofmann, Richard@\emph{von Richard Beer-Hofmann}!1906-07-132@{13. 7. 1906}|(be}
\toendnotes[C]{\smallbreak\pagebreak[2]}\Standort{CUL, Schnitzler, B 8.}
\physDesc{Bildpostkarte, 194 Zeichen
\newline{}Handschrift: Bleistift, lateinische Kurrent
\newline{}Versand: 1) Stempel: »\nobreak{}\oindex{Rodaun@\textbf{Rodaun}, \emph{A.ADM4}|pwk}{[}Roda{]}un\nobreak{}«.   2) Stempel: »\nobreak{}\oindex{Helsingør@\textbf{Helsingør}, \emph{P.PPLA2}|pwk}Helsingør, 16. 7. 06, 10-11E\nobreak{}«. 
\newline{}Ordnung: mit Bleistift von unbekannter Hand nummeriert:
                                    »206« }
\buchAbdrucke{\weitereDrucke{Arthur Schnitzler, Richard Beer-Hofmann: \emph{Briefwechsel 1891–1931}. Wien, Zürich: \emph{Europaverlag} 1992, S. 179.} }\toendnotes[C]{\smallbreak}\pstart{}{\pb}Herrn\pend{}\pstart{}Dr Arthur Schnitzler\pend{}\pstart{}Marienlyst\oindex{Marienlyst@\textbf{Marienlyst}, \emph{S.EST}|pw}\pend{}\pstart{}Kurhaus\oindex{Kurhotellet@\textbf{Kurhotellet}, \emph{Hotel (K.HTL)}|pw}\pend{}\pstart{}Dänemark\oindex{Daenemark@\textbf{Dänemark}, \emph{A.PCLI}|pw}\pend{}{\bigskip}
\pstart
           \noindent{}\centering{}{\pb}\textcolor{gray}{\textbf{Kirche und Dreifaltigkeitssäule\oindex{Stift Heiligenkreuz@\textbf{Stift Heiligenkreuz}, \emph{Gebäude (K.GBD)}|pw}.}}\hspace*{1.5em}\textcolor{gray}{\textbf{Gruss aus Heiligenkreuz\oindex{Heiligenkreuz@\textbf{Heiligenkreuz}, \emph{A.ADM3}|pw}
                  im Wienerwald\oindex{Wienerwald@\textbf{Wienerwald}, \emph{Ausflugsziel}|pw}.}}\pend
           \vspace{1em}
\pstart
           \raggedleft{}{\pb}13/VII 06\pend
           \vspace{0.5em}\stanza{}»Lang ist es her«spielt ein Kindauf dem versti{\geminationm}ten Klavier»\label{K_L01611-1v}\edtext{Es ist ein alter Klimperkasten}{\lemma{\textnormal{\emph{Es … Klimperkasten}}}\Cendnote{\textnormal{Es handelt sich um die letzten beiden Verse von Schnitzlers Gedicht \emph{Am Flügel}\pwindex{Am Fluegel@\emph{Am Flügel}|pwk} mit der Textabweichung »alter« statt dem
                        in der ersten gedruckten Fassung stehenden »dummer«.}}}\label{K_L01611-1}\pwindex{Am Fluegel@\emph{Am Flügel}|pwv}wahrscheinlich war er’s damals
                     schon!«\pwindex{Am Fluegel@\emph{Am Flügel}|pwv}\stanzaend{}\pstart \spacefill\mbox{Richard}\pend{}\selectlanguage{ngerman}\endnumbering\briefempfaengerindex{Schnitzler, Arthur@\textsc{Schnitzler, Arthur}!zzzBeer-Hofmann, Richard@\emph{von Richard Beer-Hofmann}!1906-07-132@{13. 7. 1906}|)be}\mylabel{L01611h}  \normalsize

\doendnotes{C}
\bigskip
\vfill

\clearpage

\footnotesize

\lohead{\textsc{register}}

% Definiere theindex-Environment komplett neu ohne reledmac
\makeatletter
\renewenvironment{theindex}{%
  \section*{\indexname}%
  \setlength{\parindent}{0pt}%
  \setlength{\parskip}{0pt plus 0.3pt}%
  \let\item\@idxitem
}{%
  \clearpage
}
\makeatother

\IfFileExists{\jobname-pw.ind}{\input{\jobname-pw.ind}}{}

\end{document}

      