%% latex-korrekturansicht-vorspann.tex
%% Vorspann für die Korrekturansicht.
%% Lädt die gemeinsame Datei latex-vorspann.tex mit gesetztem Schalter.

\newif\ifkorrekturansicht
\korrekturansichttrue

\input{../tex-inputs/latex-vorspann}


\section[Oscar Blumenthal an Arthur Schnitzler, 14. 11. 1896]{L00622 Oscar Blumenthal an Arthur Schnitzler, 14. 11. 1896}
\nopagebreak\mylabel{L00622v}
\rehead{ }\normalsize\beginnumbering\briefempfaengerindex{Schnitzler, Arthur@\textsc{Schnitzler, Arthur}!zzzBlumenthal, Oskar@\emph{von Oskar Blumenthal}!1896-11-142@{14. 11. 1896}|(be}
\toendnotes[C]{\smallbreak\pagebreak[2]}\Standort{CUL, Schnitzler, B 15.}
\physDesc{Brief, 1 Blatt, 2 Seiten, 1402 Zeichen
\newline{}Schreibmaschine
\newline{}Handschrift: schwarze Tinte (\noindent{}Unterschrift)
\newline{}Schnitzler: 1) mit Bleistift auf der leeren Rückseite beschriftet: »\textsc{(Blumenthal)}«  2) mit rotem Buntstift zwei Unterstreichungen
\newline{}Ordnung: mit Bleistift von unbekannter Hand nummeriert:
                                 »7« }
\pstart
           \centering{}{\pb}\textcolor{gray}{\textbf{\textsc{Lessing-Theater}\orgindex{Lessing-Theater@Lessing-Theater|pw}}}\pend
           
\pstart
           \centering{}\textcolor{gray}{\textbf{\textsc{Director}:}}{ }\textcolor{gray}{\textbf{\textsc{Dr.}{ }OSCAR BLUMENTHAL.}}\pend
           
\pstart
           \raggedleft{}\textcolor{gray}{\textbf{Berlin N.W. (40)\oindex{Berlin@\textbf{Berlin}, \emph{P.PPLC}|pw}, den}}{ }14. November 1896.\pend
           
\pstart\center{}Werther Herr Doctor!\pend\vspace{0.5em}
\pstart
           Während meiner Anwesenheit in Wien\oindex{Wien@\textbf{Wien}, \emph{A.ADM2}|pw} habe ich leider
               keine Gelegenheit gefunden, Sie zu sehen, und möchte Ihnen deshalb auf diesem Wege
               eine Idee unterbreiten, die ich zunächst mit Friedrich Mitterwurzer\pwindex{Mitterwurzer, Friedrich 16.10.1844 – 13.02.1897@\textsc{Mitterwurzer, Friedrich} (16.10.1844 – 13.02.1897), \emph{Schauspieler/Schauspielerin}|pw} besprochen habe, und zwar mit begeisterter Zustimmung
               von seiner Seite. Da bei dem Einacter-Cyclus »MORITURI\pwindex{Morituri@\emph{Morituri}|pw}« das Publikum sich geneigt gefunden hat, eine Reihe von einactigen dramatischen
               Genrebildern für ein Ganzes zu nehmen, wenn sie auch nur durch einen losen Faden mit
               einander verknüpft sind, so ist mir der Gedanke gekommen, ob nicht Ihr prächtiger
                  »ANATOL\pwindex{Anatol@\emph{Anatol}|pw}« in ähnlicher Weise für das Theater erobert werden könnte. Ich denke mir unter
               dem Gesammt-Titel »ANATOL\pwindex{Anatol@\emph{Anatol}|pw}«, fünf Capitel aus einem Liebesleben von ARTHUR SCHNITZLER, eine Zusammenfassung etwa der fünf {\pb}einactigen Plaudereien aus Ihrem Buch:
                  »EINE FRAGE AN DAS SCHICKSAL\pwindex{Frage an das Schicksal@\emph{Die Frage an das Schicksal}|pw}«, – »WEIHNACHTS-AUSVERKAUF\pwindex{Weihnachts-Einkaeufe@\emph{Weihnachts-Einkäufe}|pw}«, – »EPISODE\pwindex{Episode@\emph{Episode}|pw}« – {[}»{]}DAS ABSCHIEDSSOUPER AM HOCHZEITSMORGEN\pwindex{Abschiedssouper@\emph{Abschiedssouper}|pw}«, – und glaube, dass es leicht gelingen könnte, durch Hinzufügung einzelner
               Sätze, besonders in das erste und letzte Stück dieser Serie einen inneren Halt und
               volle Abrundung zu geben. MITTERWURZER\pwindex{Mitterwurzer, Friedrich 16.10.1844 – 13.02.1897@\textsc{Mitterwurzer, Friedrich} (16.10.1844 – 13.02.1897), \emph{Schauspieler/Schauspielerin}|pw} ist mit Begeisterung bereit, den ANATOL\pwindex{Anatol@\emph{Anatol}|pw} bei seinem, den ganzen Monat April umfassenden, Gastspiel zur
               Darstellung zu bringen, und ich bitte freundlichst um Nachricht, wie Sie sich zu
               dieser Idee stellen würden.\pend
           
\pstart
           Mit besten Grüssen{\\[\baselineskip]} Ihr ergebener{\\[\baselineskip]}\spacefill\mbox{{[}hs. :{]} Dr. Osc. Blumenthal}\pend
           \leftskip=0em{}\selectlanguage{ngerman}\endnumbering\briefempfaengerindex{Schnitzler, Arthur@\textsc{Schnitzler, Arthur}!zzzBlumenthal, Oskar@\emph{von Oskar Blumenthal}!1896-11-142@{14. 11. 1896}|)be}\mylabel{L00622h}  \normalsize

\doendnotes{C}
\bigskip
\vfill

\clearpage

\footnotesize

\lohead{\textsc{register}}

% Definiere theindex-Environment komplett neu ohne reledmac
\makeatletter
\renewenvironment{theindex}{%
  \section*{\indexname}%
  \setlength{\parindent}{0pt}%
  \setlength{\parskip}{0pt plus 0.3pt}%
  \let\item\@idxitem
}{%
  \clearpage
}
\makeatother

\IfFileExists{\jobname-pw.ind}{\input{\jobname-pw.ind}}{}

\end{document}

      