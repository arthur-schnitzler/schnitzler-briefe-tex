%% latex-korrekturansicht-vorspann.tex
%% Vorspann für die Korrekturansicht.
%% Lädt die gemeinsame Datei latex-vorspann.tex mit gesetztem Schalter.

\newif\ifkorrekturansicht
\korrekturansichttrue

\input{../tex-inputs/latex-vorspann}


\section[Arthur und Olga Schnitzler an Richard Beer-Hofmann, 20. 7. 1908]{L01784 Arthur und Olga Schnitzler an Richard Beer-Hofmann, 20. 7. 1908}
\nopagebreak\mylabel{L01784v}
\rehead{ }\normalsize\beginnumbering\briefempfaengerindex{Beer-Hofmann, Richard@\textsc{Beer-Hofmann, Richard}!zzzSchnitzler, Olga@\emph{von Olga Schnitzler}!1908-07-201@{20. 7. 1908}|(be}\briefempfaengerindex{Beer-Hofmann, Richard@\textsc{Beer-Hofmann, Richard}!zzzSchnitzler, Arthur@\emph{von Arthur Schnitzler}!1908-07-201@{20. 7. 1908}|(be}
\toendnotes[C]{\smallbreak\pagebreak[2]}\Standort{YCGL, MSS 31.}
\physDesc{Bildpostkarte, 345 Zeichen
\newline{}Handschrift Arthur Schnitzler: 1) Bleistift, deutsche Kurrent\hspace{1em}2) Bleistift, lateinische Kurrent (\noindent{}Adresse)\hspace{1em}
\newline{}Handschrift Olga Schnitzler: Bleistift, lateinische Kurrent
\newline{}Versand: Stempel: »\nobreak{}\oindex{Seis am Schlern@\textbf{Seis am Schlern}, \emph{P.PPL}|pwk}{[}S{]}eis, 20–7–{[}1908{]}\nobreak{}«.  
\newline{}Ordnung: mit Bleistift von unbekannter Hand datiert: »20. 7.« }\toendnotes[C]{\smallbreak}\pstart{}{\pb}Dr. Richard Beer-Hofmann\pend{}\pstart{}Strobl\oindex{Strobl@\textbf{Strobl}, \emph{A.ADM3}|pw}\pend{}\pstart{}Hotel am See\oindex{Hotel am See@\textbf{Hotel am See}, \emph{Hotel (K.HTL)}|pw}\pend{}\pstart{}Salzka{\geminationm}ergut\oindex{Salzkammergut@\textbf{Salzkammergut}, \emph{L.RGN}|pw}.\pend{}{\bigskip}
\pstart
           \noindent{}\centering{}{\pb}\textcolor{gray}{\textbf{Tirol\oindex{Suedtirol@\textbf{Südtirol}, \emph{A.ADM2}|pw}: Partie in Seis a. Schlern\oindex{Seis am Schlern@\textbf{Seis am Schlern}, \emph{P.PPL}|pw}, 1000m. N. d. Aquarell\pwindex{Partie in Seis am Schlern@\emph{Partie in Seis am Schlern}|pwv} v. F. A. C. M\textcolor{gray}{.} Reisch\pwindex{Reisch, Franz August Carl Maria 1862-05-01 – 1942?@\textsc{Reisch, Franz August Carl Maria} (1862-05-01 – 1942?), \emph{Maler/Malerin}|pw}, Meran\oindex{Meran@\textbf{Meran}, \emph{P.PPLA3}|pw}.}}\pend
           \vspace{1em}
\pstart
           \noindent{}{\pb}Herzliche Grüße. Uns geht es hier und gefällt’s hier
               weiter ſehr gut. Seit \label{K_L01784-1v}\edtext{10.}{\lemma{\textnormal{\emph{10.}}}\Cendnote{\textnormal{Siehe A. S.: \emph{Tagebuch}, 10. 7. 1908. }}}\label{K_L01784-1} iſt
                  Brahm\pwindex{Brahm, Otto 05.02.1856 – 28.11.1912@\textsc{Brahm, Otto} (05.02.1856 – 28.11.1912), \emph{Theaterleiter/Theaterleiterin, Regisseur/Regisseurin}|pw} da.\noindent{}und grüsst herzlich. Wir denken bis zweite Hälfte Auguſt zu bleiben. Da{\geminationn} Reiſe. Wohin unbeſti{\geminationm}t.
                  \textsc{Martino}\oindex{San Martino di Castrozza@\textbf{San Martino di Castrozza}, \emph{P.PPL}|pw}? \textsc{Campiglio}\oindex{Madonna di Campiglio@\textbf{Madonna di Campiglio}, \emph{P.PPL}|pw}? \textsc{Engadin}\oindex{Engadin@\textbf{Engadin}, \emph{T.VAL}|pw}? – Schreiben Sie recht bald, wie’s Ihnen geht und was Sie {\pb}treiben. \pend
           \pstart Von Herzen Ihr \spacefill\mbox{A.}\pend{}\selectlanguage{ngerman}\vspace{1em}
\pstart
           \noindent{}{[}hs. :{]} Herzliche Grüsse!\pend
           \pstart \spacefill\mbox{OlgaS.}\pend{}\selectlanguage{ngerman}\endnumbering\briefempfaengerindex{Beer-Hofmann, Richard@\textsc{Beer-Hofmann, Richard}!zzzSchnitzler, Olga@\emph{von Olga Schnitzler}!1908-07-201@{20. 7. 1908}|)be}\briefempfaengerindex{Beer-Hofmann, Richard@\textsc{Beer-Hofmann, Richard}!zzzSchnitzler, Arthur@\emph{von Arthur Schnitzler}!1908-07-201@{20. 7. 1908}|)be}\mylabel{L01784h}  \normalsize

\doendnotes{C}
\bigskip
\vfill

\clearpage

\footnotesize

\lohead{\textsc{register}}

% Definiere theindex-Environment komplett neu ohne reledmac
\makeatletter
\renewenvironment{theindex}{%
  \section*{\indexname}%
  \setlength{\parindent}{0pt}%
  \setlength{\parskip}{0pt plus 0.3pt}%
  \let\item\@idxitem
}{%
  \clearpage
}
\makeatother

\IfFileExists{\jobname-pw.ind}{\input{\jobname-pw.ind}}{}

\end{document}

      