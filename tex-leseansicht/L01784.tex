%% latex-leseansicht-vorspann.tex
%% Vorspann für die Leseansicht.
%% Lädt die gemeinsame Datei latex-vorspann.tex mit nicht gesetztem Schalter.

\newif\ifkorrekturansicht
\korrekturansichtfalse

\input{../tex-inputs/latex-vorspann}


\section[Arthur und Olga Schnitzler an Richard Beer-Hofmann, 20. 7. 1908]{L01784 Arthur und Olga Schnitzler an Richard Beer-Hofmann, 20. 7. 1908}
\nopagebreak\mylabel{L01784v}
\rehead{ }\normalsize\beginnumbering\briefempfaengerindex{Beer-Hofmann, Richard@\textsc{Beer-Hofmann, Richard}!zzzSchnitzler, Olga@\emph{von Olga Schnitzler}!1908-07-201@{20. 7. 1908}|(be}\briefempfaengerindex{Beer-Hofmann, Richard@\textsc{Beer-Hofmann, Richard}!zzzSchnitzler, Arthur@\emph{von Arthur Schnitzler}!1908-07-201@{20. 7. 1908}|(be}
\toendnotes[C]{\smallbreak\pagebreak[2]}
\correspDesc{Versand  durch Arthur Schnitzler, Olga Schnitzler am 20. 7. 1908 in Seis am Schlern
\newline{}Erhalt  durch Richard Beer-Hofmann im Zeitraum [21. 7. 1908
                  – 25. 7. 1908?] in Strobl}\toendnotes[C]{\smallbreak}
\Standort{YCGL, MSS 31.}
\physDesc{Bildpostkarte, 345 Zeichen
\newline{}Handschrift Arthur Schnitzler: Bleistift, deutsche Kurrent
\newline{}Handschrift Olga Schnitzler: Bleistift, lateinische Kurrent
\newline{}Versand: Stempel: »\nobreak{}\oindex{Seis am Schlern@\textbf{Seis am Schlern}|pwk}{[}S{]}eis, 20–7–{[}1908{]}\nobreak{}«.  
\newline{}Ordnung: mit Bleistift von unbekannter Hand datiert: »20. 7.« }\toendnotes[C]{\smallbreak}\pstart{}\textsc{{\pb}Dr. Richard Beer-Hofmann}\pend{}\pstart{}\textsc{Strobl\oindex{Strobl@\textbf{Strobl}, \emph{Verwaltungsgebiet}|pw}}\pend{}\pstart{}\textsc{Hotel am See\oindex{Hotel am See@\textbf{Hotel am See}, \emph{Hotel}|pw}}\pend{}\pstart{}\textsc{Salzka{\geminationm}ergut\oindex{Salzkammergut@\textbf{Salzkammergut}, \emph{Region}|pw}.}\pend{}{\bigskip}
\pstart
           \noindent{}\centering{}{\pb}\textcolor{gray}{\textbf{Tirol\oindex{Südtirol@\textbf{Südtirol}, \emph{Verwaltungsgebiet}|pw}: Partie in Seis a. Schlern\oindex{Seis am Schlern@\textbf{Seis am Schlern}|pw}, 1000m. N. d. Aquarell\pwindex{Reisch, Franz August Carl Maria 1.\,5.\,1862 Wien – 1942? Meran@\textsc{Reisch, Franz August Carl Maria} (1.\,5.\,1862 Wien – 1942? Meran), \emph{Maler}!Partie in Seis am Schlern@\strich\emph{Partie in Seis am Schlern}|pwv} v. F. A. C. M\textcolor{gray}{.} Reisch\pwindex{Reisch, Franz August Carl Maria 1.\,5.\,1862 Wien – 1942? Meran@\textsc{Reisch, Franz August Carl Maria} (1.\,5.\,1862 Wien – 1942? Meran), \emph{Maler}|pw}, Meran\oindex{Meran@\textbf{Meran}, \emph{Hauptstadt}|pw}.}}\pend
           \vspace{1em}
\pstart
           \noindent{}{\pb}Herzliche Grüße. Uns geht es hier und gefällt’s hier
               weiter{ }ſehr gut. Seit \label{K_L01784-1v}\edtext{10.}{\lemma{\textnormal{\emph{10.}}}\Cendnote{\textnormal{Siehe A. S.: \emph{Tagebuch}, 10. 7. 1908. }}}\label{K_L01784-1} iſt
                  Brahm\pwindex{Brahm, Otto 5.\,2.\,1856 Hamburg – 28.\,11.\,1912 Berlin@\textsc{Brahm, Otto} (5.\,2.\,1856 Hamburg – 28.\,11.\,1912 Berlin), \emph{Theaterleiter, Regisseur}|pw} da.\footnote{\noindent{}und grüsst herzlich.} Wir denken bis zweite Hälfte Auguſt zu bleiben. Da{\geminationn} Reiſe. Wohin unbeſti{\geminationm}t.
                  \textsc{Martino}\oindex{San Martino di Castrozza@\textbf{San Martino di Castrozza}|pw}? \textsc{Campiglio}\oindex{Madonna di Campiglio@\textbf{Madonna di Campiglio}|pw}? \textsc{Engadin}\oindex{Engadin@\textbf{Engadin}, \emph{Tal}|pw}? – Schreiben Sie recht bald, wie’s Ihnen geht und was Sie {\pb}treiben.\pend
           \pstart Von Herzen Ihr \spacefill\mbox{A.}\pend{}\selectlanguage{ngerman}\vspace{1em}
\pstart
           \noindent{}{[}hs. Schnitzler:{]} Herzliche Grüsse!\pend
           \pstart \spacefill\mbox{OlgaS.}\pend{}\selectlanguage{ngerman}\endnumbering\briefempfaengerindex{Beer-Hofmann, Richard@\textsc{Beer-Hofmann, Richard}!zzzSchnitzler, Olga@\emph{von Olga Schnitzler}!1908-07-201@{20. 7. 1908}|)be}\briefempfaengerindex{Beer-Hofmann, Richard@\textsc{Beer-Hofmann, Richard}!zzzSchnitzler, Arthur@\emph{von Arthur Schnitzler}!1908-07-201@{20. 7. 1908}|)be}\mylabel{L01784h}  \newcommand{\dateiname}{L01784}\newcommand{\titel}{Arthur und Olga Schnitzler an Richard Beer-Hofmann, 20. 7. 1908}\newcommand{\editorInnen}{Martin Anton Müller und Gerd-Hermann Susen}%% latex-leseansicht-abspann.tex
%% Abspann für die Leseansicht.
%% Der Schalter \ifkorrekturansicht ist bereits durch den Vorspann gesetzt.

%% latex-abspann.tex
%% Gemeinsamer Abspann für Korrekturansicht und Leseansicht.
%% Setzt den Schalter \ifkorrekturansicht voraus (gesetzt in den
%% einbindenden Dateien latex-korrekturansicht-abspann.tex bzw.
%% latex-leseansicht-abspann.tex).
%% ---------------------------------------------------------------

\normalsize

% Das esempio-Environment wird nur in der Leseansicht benötigt
\ifkorrekturansicht\else
\newenvironment{esempio}[3]%
{
    \vspace{1.5ex}
    \rlap{\underline{#1}}
    \par
    \setlength{\parindent}{0cm}
    \nopagebreak
    \leftskip=#2cm
    \rightskip=#3cm
}
{
    \par
}
\fi

\doendnotes{C}
\bigskip
\vfill

\clearpage

\footnotesize

\ifkorrekturansicht
  \lohead{\textsc{register}}
\fi

% theindex-Environment neu definieren ohne reledmac
\makeatletter
\renewenvironment{theindex}{%
  \ifkorrekturansicht
    \section*{\indexname}%
  \else
    \subsubsection*{Index der erwähnten Entitäten}%
  \fi
  \setlength{\parindent}{0pt}%
  \setlength{\parskip}{0pt plus 0.3pt}%
  \let\item\@idxitem
}{%
  \ifkorrekturansicht\clearpage\fi
}
\makeatother

\IfFileExists{\jobname-pw.ind}{\input{\jobname-pw.ind}}{}

% Quellenangabe nur in der Leseansicht
\ifkorrekturansicht\else
% Fallback-Definitionen, falls die .tex-Datei \titel etc. nicht gesetzt hat
\providecommand{\titel}{}
\providecommand{\editorInnen}{}
\providecommand{\dateiname}{\jobname}

\vspace{3cm}

\vfill

\footnotesize
\textsc{Quelle}: \titel. Herausgegeben von {\editorInnen}. In: \emph{Arthur Schnitzler: Briefwechsel mit Autorinnen und Autoren}.
 Digitale Edition, https://schnitzler-briefe.acdh.oeaw.ac.at/{\dateiname}.html (Stand \today)
\fi

\end{document}


