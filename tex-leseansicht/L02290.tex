%% latex-leseansicht-vorspann.tex
%% Vorspann für die Leseansicht.
%% Lädt die gemeinsame Datei latex-vorspann.tex mit nicht gesetztem Schalter.

\newif\ifkorrekturansicht
\korrekturansichtfalse

\input{../tex-inputs/latex-vorspann}


\section[Arthur Schnitzler an Robert Adam, 24. 7. 1918]{L02290 Arthur Schnitzler an Robert Adam, 24. 7. 1918}
\nopagebreak\mylabel{L02290v}
\rehead{ }\normalsize\beginnumbering\briefempfaengerindex{Adam, Robert@\textsc{Adam, Robert}!zzzSchnitzler, Arthur@\emph{von Arthur Schnitzler}!1918-07-241@{24. 7. 1918}|(be}
\toendnotes[C]{\smallbreak\pagebreak[2]}
\correspDesc{Versand  durch Arthur Schnitzler am 24. 7. 1918 in Wien
\newline{}Erhalt  durch Robert Adam im Zeitraum [25. 7. 1918
                  – 29. 7. 1918?] in Andorf}\toendnotes[C]{\smallbreak}
\Standort{DLA, 96.34.2/10.}
\physDesc{Postkarte, 559 Zeichen
\newline{}Handschrift: Bleistift, lateinische Kurrent
\newline{}Versand: Stempel: »\nobreak{}24. VII. 18, 5\nobreak{}«.  }\pstart{}{\pb}\textcolor{gray}{\textbf{D\textsuperscript{R} ARTHUR SCHNITZLER}}\pend{}\pstart{}\textcolor{gray}{\textbf{WIEN, XVIII. STERNWARTESTRASSE 71\oindex{Wien@\textbf{Wien}!XVIII., Währing@\textbf{XVIII., Währing}!Sternwartestraße 71@\textbf{Sternwartestraße 71}, \emph{Wohngebäude}|pw}.}}\pend{}{\bigskip}\pstart{}Herrn Dr.\pend{}\pstart{}Robert Adam Pollak\pend{}\pstart{}Andorf nahe Schärding\oindex{Andorf@\textbf{Andorf}, \emph{Hauptstadt}|pw}.\pend{}\pstart{}Innviertel\oindex{Innviertel@\textbf{Innviertel}|pw}.\pend{}{\bigskip}\vspace{1em}
\pstart
           \raggedleft{}{\pb}24. 7. 18\pend
           \vspace{0.5em}
\pstart
           lieber Herr Doctor,  vielen Dank für Ihre freundliche Nachricht; ich
               freue mich, daß Sie’s so gut getroffen haben und hoffe Sie bringen we{\geminationn} schon nichts geschriebenes doch eine fruchtbare Sti{\geminationm}ung mit nach Hause. Ich bleibe wohl bis Mitte August hier, um da{\geminationn} nach Bayern\oindex{Bayern@\textbf{Bayern}, \emph{Land}|pw} abzureisen, u. zw. hätte ich nicht übel Lust, donau\oindex{Donau@\textbf{Donau}, \emph{Fluss}|pw}aufwärts bis Passau\oindex{Passau@\textbf{Passau}, \emph{Hauptstadt}|pw} zu fahren, u von dort erst nach München\oindex{München@\textbf{München}|pw}{ }{\pb}u. in weiterem Verlauf Partenkirchen\oindex{Partenkirchen@\textbf{Partenkirchen}, \emph{Teil eines besiedelten Ortes}|pw} abzubiegen. Wir sehen uns vielleicht noch vorher? Herzlich
               grüßt Sie Ihr ergebner\pend
           \pstart \spacefill\mbox{A. S.}\pend{}\selectlanguage{ngerman}\endnumbering\briefempfaengerindex{Adam, Robert@\textsc{Adam, Robert}!zzzSchnitzler, Arthur@\emph{von Arthur Schnitzler}!1918-07-241@{24. 7. 1918}|)be}\mylabel{L02290h}  \newcommand{\dateiname}{L02290}\newcommand{\titel}{Arthur Schnitzler an Robert Adam, 24. 7. 1918}\newcommand{\editorInnen}{Martin Anton Müller und Gerd-Hermann Susen}%% latex-leseansicht-abspann.tex
%% Abspann für die Leseansicht.
%% Der Schalter \ifkorrekturansicht ist bereits durch den Vorspann gesetzt.

%% latex-abspann.tex
%% Gemeinsamer Abspann für Korrekturansicht und Leseansicht.
%% Setzt den Schalter \ifkorrekturansicht voraus (gesetzt in den
%% einbindenden Dateien latex-korrekturansicht-abspann.tex bzw.
%% latex-leseansicht-abspann.tex).
%% ---------------------------------------------------------------

\normalsize

% Das esempio-Environment wird nur in der Leseansicht benötigt
\ifkorrekturansicht\else
\newenvironment{esempio}[3]%
{
    \vspace{1.5ex}
    \rlap{\underline{#1}}
    \par
    \setlength{\parindent}{0cm}
    \nopagebreak
    \leftskip=#2cm
    \rightskip=#3cm
}
{
    \par
}
\fi

\doendnotes{C}
\bigskip
\vfill

\clearpage

\footnotesize

\ifkorrekturansicht
  \lohead{\textsc{register}}
\fi

% theindex-Environment neu definieren ohne reledmac
\makeatletter
\renewenvironment{theindex}{%
  \ifkorrekturansicht
    \section*{\indexname}%
  \else
    \subsubsection*{Index der erwähnten Entitäten}%
  \fi
  \setlength{\parindent}{0pt}%
  \setlength{\parskip}{0pt plus 0.3pt}%
  \let\item\@idxitem
}{%
  \ifkorrekturansicht\clearpage\fi
}
\makeatother

\IfFileExists{\jobname-pw.ind}{\input{\jobname-pw.ind}}{}

% Quellenangabe nur in der Leseansicht
\ifkorrekturansicht\else
% Fallback-Definitionen, falls die .tex-Datei \titel etc. nicht gesetzt hat
\providecommand{\titel}{}
\providecommand{\editorInnen}{}
\providecommand{\dateiname}{\jobname}

\vspace{3cm}

\vfill

\footnotesize
\textsc{Quelle}: \titel. Herausgegeben von {\editorInnen}. In: \emph{Arthur Schnitzler: Briefwechsel mit Autorinnen und Autoren}.
 Digitale Edition, https://schnitzler-briefe.acdh.oeaw.ac.at/{\dateiname}.html (Stand \today)
\fi

\end{document}


