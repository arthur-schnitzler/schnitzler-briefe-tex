%% latex-korrekturansicht-vorspann.tex
%% Vorspann für die Korrekturansicht.
%% Lädt die gemeinsame Datei latex-vorspann.tex mit gesetztem Schalter.

\newif\ifkorrekturansicht
\korrekturansichttrue

\input{../tex-inputs/latex-vorspann}


\section[Paul Goldmann an Arthur Schnitzler, 19. 6. {[}1894{]}]{L02627 Paul Goldmann an Arthur Schnitzler, 19. 6. {[}1894{]}}
\nopagebreak\mylabel{L02627v}
\rehead{ }\normalsize\beginnumbering\briefempfaengerindex{Schnitzler, Arthur@\textsc{Schnitzler, Arthur}!zzzGoldmann, Paul@\emph{von Paul Goldmann}!1894-06-191@{19. 6. {[}1894{]}}|(be}
\toendnotes[C]{\smallbreak\pagebreak[2]}\Standort{DLA, A:Schnitzler, HS.NZ85.1.3164.}
\physDesc{Brief, 3 Blätter, 11 Seiten, 4701 Zeichen
\newline{}Handschrift: schwarze Tinte, deutsche Kurrent
\newline{}Schnitzler: 1) mit Bleistift auf dem ersten Blatt die Jahreszahl »94« vermerkt  2) mit rotem Buntstift vier Unterstreichungen}\toendnotes[C]{\smallbreak}
\pstart
           {\pb}\textcolor{gray}{\textbf{Frankfurter Zeitung\orgindex{Frankfurter Zeitung@Frankfurter Zeitung|pw}.}}\pend
           
\pstart
           \textcolor{gray}{\textbf{(Gazette de
                        Francfort\orgindex{Frankfurter Zeitung@Frankfurter Zeitung|pw}).}}\hfill \textsc{Paris\oindex{Paris@\textbf{Paris}, \emph{P.PPLC}|pw}}, 19. Juni.\pend
           
\pstart
           \textcolor{gray}{\textbf{Fondateur \textbf{M. L. Sonnemann\pwindex{Sonnemann, Leopold 1831-10-29 – 1909-10-30@\textsc{Sonnemann, Leopold} (1831-10-29 – 1909-10-30), \emph{Journalist/Journalistin, Herausgeber/Herausgeberin}|pw}}.}}\pend
           
\pstart
           \textcolor{gray}{\textbf{\begin{otherlanguage}{french}Journal politique, financier,\end{otherlanguage}}}\pend
           
\pstart
           \textcolor{gray}{\textbf{\begin{otherlanguage}{french}commercial et littéraire.\end{otherlanguage}}}\pend
           
\pstart
           \textcolor{gray}{\textbf{\begin{otherlanguage}{french}\textbf{Paraissant trois fois par jour.}\end{otherlanguage}}}\pend
           
\pstart
           \textcolor{gray}{\textbf{\begin{otherlanguage}{french}\textbf{Bureau à Paris\oindex{Paris@\textbf{Paris}, \emph{P.PPLC}|pw}:}\end{otherlanguage}}}\pend
           
\pstart
           \textcolor{gray}{\textbf{\begin{otherlanguage}{french}24. Rue Feydeau\oindex{rue Feydeau@\textbf{rue Feydeau}, \emph{Straße (K.STR)}|pw}.\end{otherlanguage}}}\pend
           
\pstart\center{}Mein lieber Freund,\pend\vspace{0.5em}
\pstart
           Gern hätte ich Dir ſchon vor einigen Tagen geſchrieben, weil mich Dein letzter Brief
               ſo hoch erfreut hat und ich Dir den friſchen Eindruck davon geben wollte. Es ſtand ſo
               viel Schönes darin; er war ſo frei und ſo leicht. Heut
               lagern wieder alle Nebel über meinem Gehirn. Mein Kopf iſt wüſt. Eindrücke und
               Sprache ſind unſicher. Und über dem ſchönen Lichtbild, das ich von Deinem letzten
               Briefe gehabt, liegt ſchon {\pb}wieder allerlei
               Schwarzes und Verfinſterndes.\pend
           
\pstart
           Ich ſchreib’ Dir trotzdem heute, um meinen guten
               Willen zu zeigen.\pend
           
\pstart
           Reden wir zunächſt einmal von dem Praktiſchen, von der \label{K_L02627-1v}\edtext{Reiſe}{\lemma{\textnormal{\emph{Reiſe}}}\Cendnote{\textnormal{Vom 23. 8. 1894 bis zum 3. 9. 1894
                  verbrachten Schnitzler und Goldmann\pwindex{Goldmann, Paul 31.01.1865 – 25.09.1935@\textsc{Goldmann, Paul} (31.01.1865 – 25.09.1935), \emph{Schriftsteller/Schriftstellerin, Journalist/Journalistin}|pwk} einige Zeit gemeinsam in Bad Ischl\oindex{Bad Ischl@\textbf{Bad Ischl}, \emph{P.PPL}|pwk} und Bad
                  Aussee\oindex{Bad Aussee@\textbf{Bad Aussee}, \emph{P.PPLA3}|pwk}. Dem \emph{Tagebuch}\pwindex{Tagebuch@\emph{Tagebuch}|pwk} ist zu entnehmen,
                  dass sie auch viel Zeit mit Richard
                     Beer-Hofmann\pwindex{Beer-Hofmann, Richard 1866-07-11 – 1945-09-26@\textsc{Beer-Hofmann, Richard} (1866-07-11 – 1945-09-26), \emph{Schriftsteller/Schriftstellerin}|pwk} verbrachten.}}}\label{K_L02627-1}. Ich hab’ mir meinen Urlaub diesmal
               überhaupt nur in der Form eines Beiſammenſeins mit Euch\pwindex{Beer-Hofmann, Richard 1866-07-11 – 1945-09-26@\textsc{Beer-Hofmann, Richard} (1866-07-11 – 1945-09-26), \emph{Schriftsteller/Schriftstellerin}|pwv}\pwindex{Hofmannsthal, Hugo von 1874-02-01 – 1929-07-15@\textsc{Hofmannsthal, Hugo von} (1874-02-01 – 1929-07-15), \emph{Schriftsteller/Schriftstellerin}|pwv} vorgeſtellt. Es wäre
               traurig, wenn daraus nichts würde. Die äußerſte Conceſſion, die ich machen kann, iſt
               die: am 15. Auguſt wegzugehen bis zum 15. September. Aber
               ich muß jedenfalls vor Ende September zurück ſein, weil die
                  Kammer\textcolor{gray}{n} wegen der \label{K_L02627-2v}\edtext{Präſidenten-{\pb}Wahl}{\lemma{\textnormal{\emph{Präſidenten-Wahl}}}\Cendnote{\textnormal{In Frankreich\oindex{Frankreich@\textbf{Frankreich}, \emph{A.PCLI}|pwk} wurde am
                     27. 6. 1894{ }Jean Casimir-Perier\pwindex{Casimir-Perier, Jean 1847-11-08 – 1907-03-11@\textsc{Casimir-Perier, Jean} (1847-11-08 – 1907-03-11), \emph{Politiker/Politikerin, Präsident/Präsidentin}|pwk} zum neuen Präsidenten
                  gewählt.}}}\label{K_L02627-2} diesmal zeitiger zuſammentreten. Nun könnteſt Du vielleicht in der
                  letzten Auguſt-Woche fort. Oder ich könnte mich vielleicht mit einem
               der andern \label{K_L02627-3v}\edtext{Zwei\pwindex{Beer-Hofmann, Richard 1866-07-11 – 1945-09-26@\textsc{Beer-Hofmann, Richard} (1866-07-11 – 1945-09-26), \emph{Schriftsteller/Schriftstellerin}|pwv}\pwindex{Hofmannsthal, Hugo von 1874-02-01 – 1929-07-15@\textsc{Hofmannsthal, Hugo von} (1874-02-01 – 1929-07-15), \emph{Schriftsteller/Schriftstellerin}|pwv}}{\lemma{\textnormal{\emph{Zwei}}}\Cendnote{\textnormal{Neben Richard Beer-Hofmann\pwindex{Beer-Hofmann, Richard 1866-07-11 – 1945-09-26@\textsc{Beer-Hofmann, Richard} (1866-07-11 – 1945-09-26), \emph{Schriftsteller/Schriftstellerin}|pwk} dürfte Hugo von
                     Hofmannsthal\pwindex{Hofmannsthal, Hugo von 1874-02-01 – 1929-07-15@\textsc{Hofmannsthal, Hugo von} (1874-02-01 – 1929-07-15), \emph{Schriftsteller/Schriftstellerin}|pwk} gemeint sein, der jedoch nur gelegentlich seinen Urlaub mit
                     Goldmann\pwindex{Goldmann, Paul 31.01.1865 – 25.09.1935@\textsc{Goldmann, Paul} (31.01.1865 – 25.09.1935), \emph{Schriftsteller/Schriftstellerin, Journalist/Journalistin}|pwk} und Schnitzler verbrachte.}}}\label{K_L02627-3} inzwiſchen treffen, und Du
               kämeſt nach. Ich möchte freilich nicht gerne die oberitalien\oindex{Italien@\textbf{Italien}, \emph{A.PCLI}|pwv}iſchen Seen, denn ich war dort erſt
               im vorigen Jahre. Hingegen kenne ich Florenz\oindex{Florenz@\textbf{Florenz}, \emph{P.PPLA}|pw} noch nicht und möchte gern irgend ein \label{K_L02627-4v}\edtext{\textsc{Itinerarium}}{\lemma{\textnormal{\emph{Itinerarium}}}\Cendnote{\textnormal{lateinisch: Reiseroute}}}\label{K_L02627-4} haben, das
                  dorthin\oindex{Florenz@\textbf{Florenz}, \emph{P.PPLA}|pwv} abzielt. Ich bitte
               Dich alſo: überleg’ Dirs und ſprich’ mit den Freunden\pwindex{Beer-Hofmann, Richard 1866-07-11 – 1945-09-26@\textsc{Beer-Hofmann, Richard} (1866-07-11 – 1945-09-26), \emph{Schriftsteller/Schriftstellerin}|pwv}\pwindex{Hofmannsthal, Hugo von 1874-02-01 – 1929-07-15@\textsc{Hofmannsthal, Hugo von} (1874-02-01 – 1929-07-15), \emph{Schriftsteller/Schriftstellerin}|pwv} und mach’ mir dann nähere
               Vorſchläge. Vielleicht können wir {\pb}doch etwas
               zuſammencombiniren. Es wäre ſo ſchön! Nur muß ich Dich um möglichſt baldige Antwort
               bitten. Zwei, drei Tage mit Dir zu ſein iſt mir zu wenig. Man braucht ſoviel, um
               wieder den alten Ton zu finden. Im Augenblick, wo man ſich \strikeout{\textcolor{gray}{a}} dann gerade gefunden hat, geht man auseinander. Außerdem haſt Du bekanntlich
               in den zwei bis drei Tagen den Schnupfen. Nein, ich möchte etwas Ausgiebiges – etwas,
               was am Anfang wie »für immer« ausſieht – alſo zum Beiſpiel vierzehn Tage{\dotsfive}\pend
           
\pstart
           Es thut mir leid, Dich {\pb}mit meinen \label{K_L02627-5v}\edtext{Andeutungen über \textsc{Bahr\pwindex{Bahr, Hermann 19.07.1863 – 15.01.1934@\textsc{Bahr, Hermann} (19.07.1863 – 15.01.1934), \emph{Schriftsteller/Schriftstellerin, Kritiker/Kritikerin}|pw}}}{\lemma{\textnormal{\emph{Andeutungen über Bahr}}}\Cendnote{\textnormal{Siehe Paul Goldmann an Arthur Schnitzler, 15. 6. [1894].
               }}}\label{K_L02627-5} nervös gemacht zu haben. Es läßt ſich ſo ſchwer ſagen. Im Übrigen ſind durch
               Deine letzten lieben Briefe die Geſpenſter beinahe zerſtreut. Es kam mir ſo vor, als
               ſei er zwiſchen mich und Euch\pwindex{Beer-Hofmann, Richard 1866-07-11 – 1945-09-26@\textsc{Beer-Hofmann, Richard} (1866-07-11 – 1945-09-26), \emph{Schriftsteller/Schriftstellerin}|pwv}\pwindex{Hofmannsthal, Hugo von 1874-02-01 – 1929-07-15@\textsc{Hofmannsthal, Hugo von} (1874-02-01 – 1929-07-15), \emph{Schriftsteller/Schriftstellerin}|pwv} getreten, und ich habe ihn im
               Verdacht, daß er dieſe quälende Vorſtellung abſichtlich genährt hat, durch \strikeout{geſ} allerlei geſchickt Hingeworfenes. Weniges zwiſchen
               mich und Dich – denn Deiner fühle {\pb}ich mich doch
               ſicher – als zwiſchen mich und die Andern\pwindex{Beer-Hofmann, Richard 1866-07-11 – 1945-09-26@\textsc{Beer-Hofmann, Richard} (1866-07-11 – 1945-09-26), \emph{Schriftsteller/Schriftstellerin}|pwv}\pwindex{Hofmannsthal, Hugo von 1874-02-01 – 1929-07-15@\textsc{Hofmannsthal, Hugo von} (1874-02-01 – 1929-07-15), \emph{Schriftsteller/Schriftstellerin}|pwv}, beſonders \textsc{Loris\pwindex{Hofmannsthal, Hugo von 1874-02-01 – 1929-07-15@\textsc{Hofmannsthal, Hugo von} (1874-02-01 – 1929-07-15), \emph{Schriftsteller/Schriftstellerin}|pw}}, mit dem ich keine Berührung mehr habe. Und das Letztere ſcheint mir übrigens
               noch heut ſo.\pend
           
\pstart
           Weißt Du übrigens – ganz unter uns Beiden geſagt – daß mir der letzte \label{K_L02627-6v}\edtext{Artikel\pwindex{Ueber moderne englische Malerei. Rueckblick auf die internationale Ausstellung Wien 1894@\emph{Über moderne englische Malerei. Rückblick auf die internationale Ausstellung Wien 1894}|pwv} von \textsc{Loris\pwindex{Hofmannsthal, Hugo von 1874-02-01 – 1929-07-15@\textsc{Hofmannsthal, Hugo von} (1874-02-01 – 1929-07-15), \emph{Schriftsteller/Schriftstellerin}|pw}}}{\lemma{\textnormal{\emph{Artikel von Loris}}}\Cendnote{\textnormal{Loris\pwindex{Hofmannsthal, Hugo von 1874-02-01 – 1929-07-15@\textsc{Hofmannsthal, Hugo von} (1874-02-01 – 1929-07-15), \emph{Schriftsteller/Schriftstellerin}|pwk}: \emph{Über moderne englische Malerei. Rückblick auf die internationale
                        Ausstellung Wien 1894}\pwindex{Ueber moderne englische Malerei. Rueckblick auf die internationale Ausstellung Wien 1894@\emph{Über moderne englische Malerei. Rückblick auf die internationale Ausstellung Wien 1894}|pwk}. In: \emph{Neue
                        Revue}\pwindex{Neue Revue. Wiener Literatur-Zeitung@\emph{Neue Revue. Wiener Literatur-Zeitung}|pwk}, Jg. 5, Bd. 1, Nr. 26, 13. 6. 1894,
                     S. 811–816.}}}\label{K_L02627-6} über die moderne engliſche Malerei in der »Neuen Revüe\pwindex{Neue Revue. Wiener Literatur-Zeitung@\emph{Neue Revue. Wiener Literatur-Zeitung}|pw}« gar nicht gefällt? Schon ſeit
               einiger Zeit merke ich, wenn ich hier und \strikeout{d\textcolor{gray}{a}} da etwas von ihm in die Hand bekomme, daß ſich in mir etwas regt, das nicht
               mitthun will. Ich weiß nur nicht {\pb}recht, welcher Art
               dieſe Regung iſt. Diesmal iſt es mir freilich \strikeout{e\textcolor{gray}{t}} ein wenig klarer geworden. Ich finde, er mangelt der \uline{Disciplin}. Er läßt ſeine Gedanken und ſeine Feder laufen, wohin ſie
               wollen. Er ſchreibt mir nicht einfach, nicht gerade, nicht ſicher genug. Es iſt mir
               auch zuviel Farbenſpiel in ſeinem \strikeout{Styl (d} Styl (da
               glaube ich ſicher den ungünſtigen Einfluß \textsc{Bahrs\pwindex{Bahr, Hermann 19.07.1863 – 15.01.1934@\textsc{Bahr, Hermann} (19.07.1863 – 15.01.1934), \emph{Schriftsteller/Schriftstellerin, Kritiker/Kritikerin}|pw}} zu erkennen.) Und dann, wie
               geſagt, das zügelloſe Herumſchweifen der Gedanken in allen Zeiten. Zum Beiſpiel:
                  »\label{K_L02627-7v}\edtext{Elementare Offenbarungen {\pb}des Genius\pwindex{Ueber moderne englische Malerei. Rueckblick auf die internationale Ausstellung Wien 1894@\emph{Über moderne englische Malerei. Rückblick auf die internationale Ausstellung Wien 1894}|pwv}}{\lemma{\textnormal{\emph{Elementare … Genius}}}\Cendnote{\textnormal{Zitat aus dem erwähnten Aufsatz}}}\label{K_L02627-7}«
               ſind nach ihm: Landſchaften von \textsc{Whistler\pwindex{Whistler, James McNeill 1834-07-11 – 1903-07-17@\textsc{Whistler, James McNeill} (1834-07-11 – 1903-07-17), \emph{Maler/Malerin}|pw}}, Menſchenköpſe von \textsc{Rembrandt\pwindex{Rembrandt van Rijn 15.07.1606 – 04.10.1669@\textsc{Rembrandt van Rijn} (15.07.1606 – 04.10.1669), \emph{Maler/Malerin}|pw}}, Muſik von \strikeout{Mo}{ }\textsc{Mozart\pwindex{Mozart, Wolfgang Amadeus 27.01.1756 – 05.12.1791@\textsc{Mozart, Wolfgang Amadeus} (27.01.1756 – 05.12.1791), \emph{Komponist/Komponistin}|pw}}. Ich finde in dieſer Combination irgendwo eine ſalſche Note, die mich
               erſchreckt. Das All\textcolor{gray}{e}s wird mir wohl übrigens noch klarer werden.
               Vielleicht thue ich ihm auch ſehr Unrecht, weil ich nur kleine Nebenarbeiten von ihm
               kenne und nichts Hauptſächliches{\dotsfive}\pend
           
\pstart
           Frau \textsc{Andreas\pwindex{Andreas-Salome, Lou 12.02.1861 – 05.02.1937@\textsc{Andreas-Salomé, Lou} (12.02.1861 – 05.02.1937), \emph{Schriftsteller/Schriftstellerin}|pw}} hat ſich mit Deinem \label{K_L02627-8v}\edtext{Briefe}{\lemma{\textnormal{\emph{Briefe}}}\Cendnote{\textnormal{Siehe Arthur Schnitzler an Lou Andreas-Salomé, 13. 6. 1894.
               }}}\label{K_L02627-8} ungemein gefreut. Wir zwei, ſie und ich, ſtehen merkwürdig zuſammen. Als wir
               uns kennen lernten, {\pb}\strikeout{th}{ }\label{K_L02627-9v}\edtext{ſtanden wir uns ſehr nahe}{\lemma{\textnormal{\emph{ſtanden … nahe}}}\Cendnote{\textnormal{Siehe Paul Goldmann an Arthur Schnitzler, 29. 5. [1894].
               }}}\label{K_L02627-9}. Jetzt thun ſich wahre \label{K_L02627-10v}\edtext{Abgründe}{\lemma{\textnormal{\emph{Abgründe}}}\Cendnote{\textnormal{Es ist davon auszugehen,
                  dass Goldmann\pwindex{Goldmann, Paul 31.01.1865 – 25.09.1935@\textsc{Goldmann, Paul} (31.01.1865 – 25.09.1935), \emph{Schriftsteller/Schriftstellerin, Journalist/Journalistin}|pwk} und Lou Andreas-Salomé\pwindex{Andreas-Salome, Lou 12.02.1861 – 05.02.1937@\textsc{Andreas-Salomé, Lou} (12.02.1861 – 05.02.1937), \emph{Schriftsteller/Schriftstellerin}|pwk}{ }1894 ein Verhältnis hatten. In Frieda von Bülows\pwindex{Buelow, Frieda von 12.10.1857 – 12.03.1909@\textsc{Bülow, Frieda von} (12.10.1857 – 12.03.1909), \emph{Schriftsteller/Schriftstellerin}|pwk} Novelle \emph{Zwei
                     Menschen}\pwindex{Zwei Menschen. Erzaehlung aus der heutigen Uebergangszeit@\emph{Zwei Menschen. Erzählung aus der heutigen Übergangszeit}|pwk}, auch »Die Goldmanniade\pwindex{Zwei Menschen. Erzaehlung aus der heutigen Uebergangszeit@\emph{Zwei Menschen. Erzählung aus der heutigen Übergangszeit}|pwv}« genannt, ist ein Brief der als Goldmann\pwindex{Goldmann, Paul 31.01.1865 – 25.09.1935@\textsc{Goldmann, Paul} (31.01.1865 – 25.09.1935), \emph{Schriftsteller/Schriftstellerin, Journalist/Journalistin}|pwk} erscheinenden Figur Dr. Siegfried Rosenfeld\pwindex{Zwei Menschen. Erzaehlung aus der heutigen Uebergangszeit@\emph{Zwei Menschen. Erzählung aus der heutigen Übergangszeit}|pwkv} zu finden, der im Ton mit dem hier
                  geschilderten Eindruck Goldmanns\pwindex{Goldmann, Paul 31.01.1865 – 25.09.1935@\textsc{Goldmann, Paul} (31.01.1865 – 25.09.1935), \emph{Schriftsteller/Schriftstellerin, Journalist/Journalistin}|pwk}
                  grundlegend übereinstimmt und das Ende eines angedeuteten Verhältnisses mit dem
                  alter ego Andreas-Salomés\pwindex{Andreas-Salome, Lou 12.02.1861 – 05.02.1937@\textsc{Andreas-Salomé, Lou} (12.02.1861 – 05.02.1937), \emph{Schriftsteller/Schriftstellerin}|pwk} in der Novelle\pwindex{Zwei Menschen. Erzaehlung aus der heutigen Uebergangszeit@\emph{Zwei Menschen. Erzählung aus der heutigen Übergangszeit}|pwkv} markiert. Vgl. Frieda von Bülow\pwindex{Buelow, Frieda von 12.10.1857 – 12.03.1909@\textsc{Bülow, Frieda von} (12.10.1857 – 12.03.1909), \emph{Schriftsteller/Schriftstellerin}|pwk}: \emph{Zwei Menschen}\pwindex{Zwei Menschen. Erzaehlung aus der heutigen Uebergangszeit@\emph{Zwei Menschen. Erzählung aus der heutigen Übergangszeit}|pwk}. In: Die schönsten Novellen der Frieda von Bülow\pwindex{Buelow, Frieda von 12.10.1857 – 12.03.1909@\textsc{Bülow, Frieda von} (12.10.1857 – 12.03.1909), \emph{Schriftsteller/Schriftstellerin}|pwk} über Lou Andreas-Salomé\pwindex{Andreas-Salome, Lou 12.02.1861 – 05.02.1937@\textsc{Andreas-Salomé, Lou} (12.02.1861 – 05.02.1937), \emph{Schriftsteller/Schriftstellerin}|pwk} und andere Frauen. Herausgegeben von Sabina
                     Streiter. Frankfurt a. M.\oindex{Frankfurt am Main@\textbf{Frankfurt am Main}, \emph{P.PPLA3}|pwk}, Berlin\oindex{Berlin@\textbf{Berlin}, \emph{P.PPLC}|pwk}: \emph{Ullstein}\orgindex{Ullstein Verlag@Ullstein Verlag|pwk}{ }1990, S. 60–61.}}}\label{K_L02627-10} zwiſchen uns auf. Ich glaube, ſie
               hat mich ſehr überſchätzt. Und für einen eitlen Menſchen, wie ich, iſt es furchtbar
               ſchmerzlich, wenn man zuſieht, wie die zu hohe Meinung langſam der richtigen weicht{\dotsfive}\pend
           
\pstart
           Über die Fortſchritte Deiner \label{K_L02627-11v}\edtext{Arbeiten}{\lemma{\textnormal{\emph{Arbeiten}}}\Cendnote{\textnormal{Schnitzler arbeitete seit dem Brief vom 1. 6. [1894], wie seinem \emph{Tagebuch}\pwindex{Tagebuch@\emph{Tagebuch}|pwk} zu entnehmen ist, an dem Schauspiel
                     \emph{Das Märchen}\pwindex{Maerchen. Schauspiel in drei Aufzuegen@\emph{Das Märchen. Schauspiel in drei Aufzügen}|pwk}. Außerdem arbeitete er unter
                  dem vorläufigen Titel »Armes
                     Mädel\pwindex{Liebelei. Schauspiel in drei Akten@\emph{Liebelei. Schauspiel in drei Akten}|pwkv}« an \emph{Liebelei}\pwindex{Liebelei. Schauspiel in drei Akten@\emph{Liebelei. Schauspiel in drei Akten}|pwk}. Mit
                  dem »ſiebzigjährigen Violin-Spieler« ist die Figur des Hans Weiring\pwindex{Liebelei. Schauspiel in drei Akten@\emph{Liebelei. Schauspiel in drei Akten}|pwkv} gemeint, der
                  Vater von Christine\pwindex{Liebelei. Schauspiel in drei Akten@\emph{Liebelei. Schauspiel in drei Akten}|pwkv}, der
                  bereits in Entwürfen aus dem Februar des Jahres
                  vorkommt. (\emph{Liebelei}\pwindex{Liebelei. Schauspiel in drei Akten@\emph{Liebelei. Schauspiel in drei Akten}|pwk}. Historisch-kritische Ausgabe.
                     Herausgegeben von  Peter Michael Braunwarth, Gerhard Hubmann und Isabella Schwentner. Berlin,
                     Boston: \emph{de Gruyter}{ }2014, T\textsuperscript{7}.)}}}\label{K_L02627-11} freue ich mich von Herzen. Den ſiebzigjährigen Violin-Spieler\pwindex{Liebelei. Schauspiel in drei Akten@\emph{Liebelei. Schauspiel in drei Akten}|pwv} begrüße ich freudig; denn in
               dieſe Hülle kannſt Du doch {\pb}unmöglich hinein, und ſo
               ſcheint die Löſung des Objectivirungs-Problems bevorzuſtehen. Sonſt aber wäre das
               beſte Mittel zur Objectivirung: \textsc{Paris\oindex{Paris@\textbf{Paris}, \emph{P.PPLC}|pw}}. Du haſt keine Ahnung, wie Einen dieſe Stadt\oindex{Paris@\textbf{Paris}, \emph{P.PPLC}|pwv} fortwährend nach außen reißt{\dotsfour}\pend
           
\pstart
           Von \label{K_L02627-12v}\edtext{\textsc{Duerer\pwindex{Duerer, Albrecht 21.05.1471 – 06.04.1528@\textsc{Dürer, Albrecht} (21.05.1471 – 06.04.1528), \emph{Maler/Malerin}|pw}} ſollſt Du die \uline{Briefe\pwindex{Duerers Briefe, Tagebuecher und Reime@\emph{Dürers Briefe, Tagebücher und Reime}|pwv}}}{\lemma{\textnormal{\emph{Duerer … Briefe}}}\Cendnote{\textnormal{\emph{Dürers Briefe, Tagebücher und Reime}\pwindex{Duerers Briefe, Tagebuecher und Reime@\emph{Dürers Briefe, Tagebücher und Reime}|pwk}. Nebst
                     einem Anhange von Zuschriften an und für Dürer\pwindex{Duerer, Albrecht 21.05.1471 – 06.04.1528@\textsc{Dürer, Albrecht} (21.05.1471 – 06.04.1528), \emph{Maler/Malerin}|pwk}. Übersetzt und mit Einleitung, Anmerkungen, Personenverzeichniß
                     und einer Reisekarte versehen von Moriz
                        Thausing\pwindex{Thausing, Moritz 1838-06-03 – 1884-08-11@\textsc{Thausing, Moritz} (1838-06-03 – 1884-08-11), \emph{Kunsthistoriker/Kunsthistorikerin}|pwk}. Wien\oindex{Wien@\textbf{Wien}, \emph{A.ADM2}|pwk}: \emph{Wilhelm Braumüller}\orgindex{Verlag Wilhelm Braumueller@Verlag Wilhelm Braumüller|pwk}{ }1872 (Quellenschriften für Kunstgeschichte und Kunsttechnik des
                     Mittelalters und der Renaissance 3). Eine Lektüre durch Schnitzler ist bislang nicht belegt.}}}\label{K_L02627-12}
               leſen, die \textsc{Thausing\pwindex{Thausing, Moritz 1838-06-03 – 1884-08-11@\textsc{Thausing, Moritz} (1838-06-03 – 1884-08-11), \emph{Kunsthistoriker/Kunsthistorikerin}|pw}} ſehr ſchön herausgegeben hat (bei \textsc{Braumueller\orgindex{Verlag Wilhelm Braumueller@Verlag Wilhelm Braumüller|pw}} in \textsc{Wien\oindex{Wien@\textbf{Wien}, \emph{A.ADM2}|pw}}).\pend
           
\pstart
           Grüß’ Dich Gott, mein lieber Freund! Und nochmals: mach’ es möglich, daß wir uns
                  {\pb}in Ruhe wiederſehen! {\\[\baselineskip]}In Treue {\\[\baselineskip]}Dein {\\[\baselineskip]}\spacefill\mbox{Paul Goldmann}\pend
           \leftskip=0em{}\selectlanguage{ngerman}\endnumbering\briefempfaengerindex{Schnitzler, Arthur@\textsc{Schnitzler, Arthur}!zzzGoldmann, Paul@\emph{von Paul Goldmann}!1894-06-191@{19. 6. {[}1894{]}}|)be}\mylabel{L02627h}  \normalsize

\doendnotes{C}
\bigskip
\vfill

\clearpage

\footnotesize

\lohead{\textsc{register}}

% Definiere theindex-Environment komplett neu ohne reledmac
\makeatletter
\renewenvironment{theindex}{%
  \section*{\indexname}%
  \setlength{\parindent}{0pt}%
  \setlength{\parskip}{0pt plus 0.3pt}%
  \let\item\@idxitem
}{%
  \clearpage
}
\makeatother

\IfFileExists{\jobname-pw.ind}{\input{\jobname-pw.ind}}{}

\end{document}

      