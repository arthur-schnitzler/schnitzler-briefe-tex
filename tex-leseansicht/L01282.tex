%% latex-korrekturansicht-vorspann.tex
%% Vorspann für die Korrekturansicht.
%% Lädt die gemeinsame Datei latex-vorspann.tex mit gesetztem Schalter.

\newif\ifkorrekturansicht
\korrekturansichttrue

\input{../tex-inputs/latex-vorspann}


\section[Hermann Bahr an Arthur Schnitzler, {[}29. 3. 1903?{]}]{L01282 Hermann Bahr an Arthur Schnitzler, {[}29. 3. 1903?{]}}
\nopagebreak\mylabel{L01282v}
\rehead{ }\normalsize\beginnumbering\briefempfaengerindex{Schnitzler, Arthur@\textsc{Schnitzler, Arthur}!zzzBahr, Hermann@\emph{von Hermann Bahr}!1903-03-291@{{[}29.  3. 1903?{]}}|(be}
\toendnotes[C]{\smallbreak\pagebreak[2]}\Standort{CUL, Schnitzler, B 5b.}
\physDesc{Brief, 1 Blatt, 4 Seiten, 2131 Zeichen
\newline{}Handschrift: schwarze Tinte, deutsche Kurrent
\newline{}Schnitzler: mit Datum »Ende März 903« versehen 
\newline{}Ordnung: mit Bleistift von unbekannter Hand nummeriert:
                                    »96« }
\buchAbdrucke{\weitereDrucke{Hermann Bahr, Arthur Schnitzler: \emph{Briefwechsel, Aufzeichnungen, Dokumente (1891–1931)}. Göttingen: \emph{Wallstein} 2018, S. 256.} }\toendnotes[C]{\smallbreak}
\pstart\center{}{\pb}Lieber Arthur,\pend\vspace{0.5em}
\pstart
           ſsehr gern und mit großer Freude ſchreibe ich über den »Reigen\pwindex{Reigen. Zehn Dialoge@\emph{Reigen. Zehn Dialoge}|pw}« und natürlich ſo bald als nur irgend möglich. Wann, das
               weiß ich freilich nicht und bitte Dich, damit nicht irgend eine Verſtimmung
               herauswächſt, folgendes zu bedenken. Ich muß dieſe Woche ſechs Mal ins Theater gehen
               und ſoll drei Feuilletons ſchreiben, »\label{K_L01282-1v}\edtext{Die Duse\pwindex{Duse, Eleonora 03.10.1858 – 21.04.1924@\textsc{Duse, Eleonora} (03.10.1858 – 21.04.1924), \emph{Schauspieler/Schauspielerin}|pw}\pwindex{Duse. (Als Gast im Carl-Theater vom 31. Maerz bis 8. April 1903)@\emph{Die Duse. (Als Gast im Carl-Theater vom 31. März bis 8. April 1903)}|pw}}{\lemma{\textnormal{\emph{Die Duse}}}\Cendnote{\textnormal{Hermann Bahr\pwindex{Bahr, Hermann 19.07.1863 – 15.01.1934@\textsc{Bahr, Hermann} (19.07.1863 – 15.01.1934), \emph{Schriftsteller/Schriftstellerin, Kritiker/Kritikerin}|pwk}: \emph{Die Duse. (Als Gast im Carl-Theater vom 31. März
                        bis 8. April 1903)}\pwindex{Duse. (Als Gast im Carl-Theater vom 31. Maerz bis 8. April 1903)@\emph{Die Duse. (Als Gast im Carl-Theater vom 31. März bis 8. April 1903)}|pwk}. In: \emph{Neues Wiener Tagblatt}\pwindex{Neues Wiener Tagblatt@\emph{Neues Wiener Tagblatt}|pwk}, Jg. 37, Nr. 89, 31. 3. 1903,
                     S. 1–2.}}}\label{K_L01282-1}«, »\label{K_L01282-2v}\edtext{l’altro pericolo\pwindex{autre danger@\emph{L’autre danger}|pw}\pwindex{autre danger@\emph{L’autre danger}|pw}}{\lemma{\textnormal{\emph{l’altro pericolo}}}\Cendnote{\textnormal{Hermann Bahr\pwindex{Bahr, Hermann 19.07.1863 – 15.01.1934@\textsc{Bahr, Hermann} (19.07.1863 – 15.01.1934), \emph{Schriftsteller/Schriftstellerin, Kritiker/Kritikerin}|pwk}: \emph{L’autre danger. (Komödie in vier Akten von Maurice Donnay. Zur
                        morgigen Aufführung im Carl-Theater durch die Truppe der Duse)}\pwindex{autre danger@\emph{L’autre danger}|pwk}. In: \emph{Neues Wiener Tagblatt}\pwindex{Neues Wiener Tagblatt@\emph{Neues Wiener Tagblatt}|pwk}, Jg. 37, Nr. 94,
                        4. 4. 1903, S. 1–3.}}}\label{K_L01282-2}«, »\label{K_L01282-3v}\edtext{Braut von Meſſina\pwindex{Braut von Messina oder die feindlichen Brueder@\emph{Die Braut von Messina oder die feindlichen Brüder}|pw}\pwindex{Theater und Kunst. (Burgtheater) [Die Braut von Messina]@\emph{Theater und Kunst. (Burgtheater) [Die Braut von Messina]}|pwv}}{\lemma{\textnormal{\emph{Braut von Meſſina}}}\Cendnote{\textnormal{Hermann Bahr\pwindex{Bahr, Hermann 19.07.1863 – 15.01.1934@\textsc{Bahr, Hermann} (19.07.1863 – 15.01.1934), \emph{Schriftsteller/Schriftstellerin, Kritiker/Kritikerin}|pwk}: \emph{Theater und Kunst. Burgtheater [Die Braut von Messina]}\pwindex{Theater und Kunst. (Burgtheater) [Die Braut von Messina]@\emph{Theater und Kunst. (Burgtheater) [Die Braut von Messina]}|pwk}. In:
                        \emph{Österreichische Volks-Zeitung}\pwindex{Oesterreichische Volks-Zeitung@\emph{Österreichische Volks-Zeitung}|pwk}, Jg. 49,
                     Nr. 96, 7. 4. 1903, S. 4.}}}\label{K_L01282-3}«, u. eigentlich auch noch
               eins über die »\label{K_L01282-4v}\edtext{Seceſſion\pwindex{Sezession. (Siebzehnte Ausstellung der Vereinigung bildender Kuenstler Oesterreichs)@\emph{Sezession. (Siebzehnte Ausstellung der Vereinigung bildender Künstler Österreichs)}|pwv}}{\lemma{\textnormal{\emph{Seceſſion}}}\Cendnote{\textnormal{Hermann Bahr\pwindex{Bahr, Hermann 19.07.1863 – 15.01.1934@\textsc{Bahr, Hermann} (19.07.1863 – 15.01.1934), \emph{Schriftsteller/Schriftstellerin, Kritiker/Kritikerin}|pwk}: \emph{Sezession. (Siebzehnte Ausstellung der Vereinigung bildender
                        Künstler Österreichs)}\pwindex{Sezession. (Siebzehnte Ausstellung der Vereinigung bildender Kuenstler Oesterreichs)@\emph{Sezession. (Siebzehnte Ausstellung der Vereinigung bildender Künstler Österreichs)}|pwk}. In: \emph{Österreichische Volks-Zeitung}\pwindex{Oesterreichische Volks-Zeitung@\emph{Österreichische Volks-Zeitung}|pwk}, Jg. 49, Nr. 96,
                     7. 4. 1903, S. 1.}}}\label{K_L01282-4}«. Du haſt aber keine Ahnung, wie
               mich der Theaterbeſuch jetzt aufregt u. wie unſinnig mich die geringſte Arbeit {\pb}anſtrengt. Geſtern habe ich außerdem wieder einen
               Anfall jener Herzbeklemmungen bekommen, diesmal auch noch mit ſolchem Schwindel
               verbunden, daß ich den Nachmittag nur auf dem Sopha ausgeſtreckt, die Augen feſt
               geschloſſen, beide Hände auf die Schläfen gedrückt zubringen konnte, immer mit dem
               Gefühl, es iſt ja doch alles aus und ich werde niemals mehr geſund. Unter dieſen
               Bedingungen arbeite ich jetzt und darf daher eigentlich gar nichts verſprechen, weil
               ich mich bei jedem Feuilleton wundere, wenn es ſchließlich doch fertig geworden
               iſt.\pend
           
\pstart
           Ferner mußt Du auch wiſſen, daß die Redacteure des {\pb}Neuen Wiener Tagblatt\orgindex{Neues Wiener Tagblatt@Neues Wiener Tagblatt|pw} (Wilhelm Singer\pwindex{Singer, Wilhelm 26.11.1847 – 10.10.1917@\textsc{Singer, Wilhelm} (26.11.1847 – 10.10.1917), \emph{Journalist/Journalistin, Chefredakteur/Chefredakteurin}|pw} und den braven Herrn Epſtein\pwindex{Epstein, Moritz 1844-01-01 – 1915@\textsc{Epstein, Moritz} (1844-01-01 – 1915), \emph{Journalist/Journalistin}|pw} ausgenommen) einen Bund bilden, deſſen einzige Sorge es
               zu ſein ſcheint, auszuſinnen, was etwa geeignet wäre, mich zu ärgern, und dies mit
               der Behendigkeit von Affen ſogleich ins Blatt zu ſetzen. Daß gegen Dich noch nicht
               eine ungeheuerliche Gemeinheit verübt worden iſt, wundert mich ſchon lange. Geht ſie
               vielleicht gelegentlich des »Reigens\pwindex{Reigen. Zehn Dialoge@\emph{Reigen. Zehn Dialoge}|pw}« los, ſo
               vergiß nicht, daß ſie, zwar an Dir executiert, aber Dir gar nicht zugedacht iſt.\pend
           
\pstart
           Bitte, ſchicke mir gleich ein Exemplar des »Reigens\pwindex{Reigen. Zehn Dialoge@\emph{Reigen. Zehn Dialoge}|pw}«. Meines iſt nemlich confisciert {\pb}worden, von der Cenſur. Das heißt: Der Herr Hofrath Jettel\pwindex{Jettel-Ettenach, Emil von 08.04.1846 – 25.04.1925@\textsc{Jettel-Ettenach, Emil von} (08.04.1846 – 25.04.1925), \emph{Rechtswissenschaftler/Rechtswissenschaftlerin, Ministerialbeamter/Ministerialbeamte, Zensor/Zensorin}|pw} hat es ſich bei mir ausleihen laſſen und ich habe es
               niemals mehr zurückbekommen.\pend
           
\pstart
           Das Incohärente dieſes Briefes mußt Du meinem Zuſtand vergeben. Wie ich nur Zeit
               habe, fahre ich zunächſt zu Julius\pwindex{Schnitzler, Julius 13.07.1865 – 29.06.1939@\textsc{Schnitzler, Julius} (13.07.1865 – 29.06.1939), \emph{Chirurg/Chirurgin}|pw}, der einmal
               doch mein Herz ordentlich unterſuchen muß.\pend
           
\pstart
           \label{K_L01282-5v}\edtext{Freitag}{\lemma{\textnormal{\emph{Freitag}}}\Cendnote{\textnormal{der verpasste Besuch vom 27. 3.}}}\label{K_L01282-5} war mir rieſig leid, ich war bei der Steuerbehörde, die mich auch noch
                  \label{K_L01282-6v}\edtext{ſekiert}{\lemma{\textnormal{\emph{ſekiert}}}\Cendnote{\textnormal{österreichisch sekkieren: ärgern, belästigen}}}\label{K_L01282-6}.\pend
           
\pstart
           Herzlichſt{\\[\baselineskip]}Dein{\\[\baselineskip]}\spacefill\mbox{Hermann}\pend
           \leftskip=0em{}\selectlanguage{ngerman}\endnumbering\briefempfaengerindex{Schnitzler, Arthur@\textsc{Schnitzler, Arthur}!zzzBahr, Hermann@\emph{von Hermann Bahr}!1903-03-291@{{[}29.  3. 1903?{]}}|)be}\mylabel{L01282h}  \normalsize

\doendnotes{C}
\bigskip
\vfill

\clearpage

\footnotesize

\lohead{\textsc{register}}

% Definiere theindex-Environment komplett neu ohne reledmac
\makeatletter
\renewenvironment{theindex}{%
  \section*{\indexname}%
  \setlength{\parindent}{0pt}%
  \setlength{\parskip}{0pt plus 0.3pt}%
  \let\item\@idxitem
}{%
  \clearpage
}
\makeatother

\IfFileExists{\jobname-pw.ind}{\input{\jobname-pw.ind}}{}

\end{document}

      