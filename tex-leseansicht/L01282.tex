\input{../tex-inputs/latex-pdf-vorspann}
\begin{center}
            \textcolor{red}{ENTWURF. ENTZIFFERUNG NOCH NICHT KORREKTURGELESEN}
                      \end{center}
            
               \section[Hermann Bahr an Arthur Schnitzler, {[}29. 3. 1903?{]}]{ Hermann Bahr an Arthur Schnitzler, {[}29. 3. 1903?{]}}\nopagebreak\mylabel{v}\rehead{ }\begin{ledgroupsized}[t]{13cm}\normalsize\beginnumbering\briefempfaengerindex{Schnitzler, Arthur@\textsc{Schnitzler, Arthur}!zzzBahr, Hermann@\emph{von Hermann Bahr}!1903-03-291@{{[}29.  3. 1903?{]}}|(be} \toendnotes[C]{\smallbreak\pagebreak[2]} \Standort{CUL, Schnitzler, B 5b.}
\physDesc{Brief, 1 Blatt, 4 Seiten
\newline{}Handschrift: schwarze Tinte, deutsche Kurrent
\newline{}Schnitzler: mit Datum »Ende März 903« versehen \newline{}Ordnung: mit Bleistift von unbekannter Hand nummeriert:
                              »96« }\buchAbdrucke{\weitereDrucke{Hermann Bahr, Arthur Schnitzler: \emph{Briefwechsel, Aufzeichnungen, Dokumente (1891–1931)}. Hg. Kurt Ifkovits und Martin Anton Müller. Göttingen: \emph{Wallstein} 2018, S. 256.} }\toendnotes[C]{\smallbreak}\pstart\center{}{\pb}Lieber Arthur,\pend\pstart
           ſsehr gern und mit großer Freude ſchreibe ich über den »Reigen\pwindex{Schnitzler, Arthur 15.05.1862 – 21.10.1931@\textsc{Schnitzler, Arthur} (15.05.1862 – 21.10.1931), \emph{Schriftsteller, Mediziner}!Reigen. Zehn Dialoge1900@\strich\emph{Reigen. Zehn Dialoge} {[}1900{]}|pw}« und natürlich ſo bald als nur irgend möglich. Wann, das
               weiß ich freilich nicht und bitte Dich, damit nicht irgend eine Verſtimmung
               herauswächſt, folgendes zu bedenken. Ich muß dieſe Woche ſechs Mal ins Theater gehen
               und ſoll drei Feuilletons ſchreiben, »\label{K_L01282_1v}\edtext{Die Duse\pwindex{Duse, Eleonora 03.10.1858 – 21.04.1924@\textsc{Duse, Eleonora} (03.10.1858 – 21.04.1924), \emph{Schauspielerin}|pw}\pwindex{Bahr, Hermann 19.07.1863 – 15.01.1934@\textsc{Bahr, Hermann} (19.07.1863 – 15.01.1934), \emph{Schriftsteller, Kritiker}!Duse. (Als Gast im Carl-Theater vom 31. Maerz bis 8. April 1903)31. 03. 1903@\strich\emph{Die Duse. (Als Gast im Carl-Theater vom 31. März bis 8. April 1903)} {[}31. 03. 1903{]}|pw}}{\lemma{\textnormal{\emph{Die Duse}}}\Cendnote{\textnormal{Hermann Bahr\pwindex{Bahr, Hermann 19.07.1863 – 15.01.1934@\textsc{Bahr, Hermann} (19.07.1863 – 15.01.1934), \emph{Schriftsteller, Kritiker}|pwk}: \emph{Die Duse. (Als Gast im Carl-Theater vom 31. März
                        bis 8. April 1903)}\pwindex{Bahr, Hermann 19.07.1863 – 15.01.1934@\textsc{Bahr, Hermann} (19.07.1863 – 15.01.1934), \emph{Schriftsteller, Kritiker}!Duse. (Als Gast im Carl-Theater vom 31. Maerz bis 8. April 1903)31. 03. 1903@\strich\emph{Die Duse. (Als Gast im Carl-Theater vom 31. März bis 8. April 1903)} {[}31. 03. 1903{]}|pwk}. In: \emph{Neues Wiener Tagblatt}\pwindex{Neues Wiener Tagblatt1867 – 1945@\emph{Neues Wiener Tagblatt}|pwk}, Jg. 37, Nr. 89,
                        31. 3. 1903, S. 1–2.}}}\label{K_L01282_1h}«, »\label{K_L01282_2v}\edtext{l’altro pericolo\pwindex{\textcolor{red}{\textsuperscript{XXXX1 indx}}!autre danger1902@\strich\emph{L’autre danger} {[}1902{]}|pw}\pwindex{Bahr, Hermann 19.07.1863 – 15.01.1934@\textsc{Bahr, Hermann} (19.07.1863 – 15.01.1934), \emph{Schriftsteller, Kritiker}!autre danger04. 04. 1903@\strich\emph{L’autre danger} {[}04. 04. 1903{]}|pw}}{\lemma{\textnormal{\emph{l’altro pericolo}}}\Cendnote{\textnormal{Hermann Bahr\pwindex{Bahr, Hermann 19.07.1863 – 15.01.1934@\textsc{Bahr, Hermann} (19.07.1863 – 15.01.1934), \emph{Schriftsteller, Kritiker}|pwk}: \emph{L’autre danger. (Komödie in vier Akten von Maurice Donnay. Zur
                        morgigen Aufführung im Carl-Theater durch die Truppe der Duse)}\pwindex{Bahr, Hermann 19.07.1863 – 15.01.1934@\textsc{Bahr, Hermann} (19.07.1863 – 15.01.1934), \emph{Schriftsteller, Kritiker}!autre danger04. 04. 1903@\strich\emph{L’autre danger} {[}04. 04. 1903{]}|pwk}. In: \emph{Neues Wiener Tagblatt}\pwindex{Neues Wiener Tagblatt1867 – 1945@\emph{Neues Wiener Tagblatt}|pwk}, Jg. 37,
                     Nr. 94, 4. 4. 1903, S. 1–3.}}}\label{K_L01282_2h}«,
                  »\label{K_L01282_3v}\edtext{Braut von Meſſina\pwindex{\textcolor{red}{\textsuperscript{XXXX1 indx}}!Braut von Messina1803@\strich\emph{Die Braut von Messina} {[}1803{]}|pw}\pwindex{Bahr, Hermann 19.07.1863 – 15.01.1934@\textsc{Bahr, Hermann} (19.07.1863 – 15.01.1934), \emph{Schriftsteller, Kritiker}!Theater und Kunst. (Burgtheater) [Die Braut von Messina]07. 04. 1903@\strich\emph{Theater und Kunst. (Burgtheater) [Die Braut von Messina]} {[}07. 04. 1903{]}|pwv}\pwindex{Bahr, Hermann 19.07.1863 – 15.01.1934@\textsc{Bahr, Hermann} (19.07.1863 – 15.01.1934), \emph{Schriftsteller, Kritiker}!Theater und Kunst. (Burgtheater) [Die Braut von Messina]07. 04. 1903@\strich\emph{Theater und Kunst. (Burgtheater) [Die Braut von Messina]} {[}07. 04. 1903{]}|pwv}}{\lemma{\textnormal{\emph{Braut von Meſſina}}}\Cendnote{\textnormal{Hermann Bahr\pwindex{Bahr, Hermann 19.07.1863 – 15.01.1934@\textsc{Bahr, Hermann} (19.07.1863 – 15.01.1934), \emph{Schriftsteller, Kritiker}|pwk}: \emph{Theater und Kunst. Burgtheater [Die Braut von Messina]}\pwindex{Bahr, Hermann 19.07.1863 – 15.01.1934@\textsc{Bahr, Hermann} (19.07.1863 – 15.01.1934), \emph{Schriftsteller, Kritiker}!Theater und Kunst. (Burgtheater) [Die Braut von Messina]07. 04. 1903@\strich\emph{Theater und Kunst. (Burgtheater) [Die Braut von Messina]} {[}07. 04. 1903{]}|pwk}\pwindex{Bahr, Hermann 19.07.1863 – 15.01.1934@\textsc{Bahr, Hermann} (19.07.1863 – 15.01.1934), \emph{Schriftsteller, Kritiker}!Theater und Kunst. (Burgtheater) [Die Braut von Messina]07. 04. 1903@\strich\emph{Theater und Kunst. (Burgtheater) [Die Braut von Messina]} {[}07. 04. 1903{]}|pwk}. In:
                     \emph{Österreichische Volks-Zeitung}\pwindex{Oesterreichische Volks-Zeitung1888 – 13.11.1918@\emph{Österreichische Volks-Zeitung}|pwk}, Jg. 49,
                     Nr. 96, 7. 4. 1903, S. 4.}}}\label{K_L01282_3h}«, u.
               eigentlich auch noch eins über die »\label{K_L01282_4v}\edtext{Seceſſion\pwindex{Bahr, Hermann 19.07.1863 – 15.01.1934@\textsc{Bahr, Hermann} (19.07.1863 – 15.01.1934), \emph{Schriftsteller, Kritiker}!Sezession. (Siebzehnte Ausstellung der Vereinigung bildender Kuenstler Oesterreichs)07. 04. 1903@\strich\emph{Sezession. (Siebzehnte Ausstellung der Vereinigung bildender Künstler Österreichs)} {[}07. 04. 1903{]}|pwv}}{\lemma{\textnormal{\emph{Seceſſion}}}\Cendnote{\textnormal{Hermann Bahr\pwindex{Bahr, Hermann 19.07.1863 – 15.01.1934@\textsc{Bahr, Hermann} (19.07.1863 – 15.01.1934), \emph{Schriftsteller, Kritiker}|pwk}: \emph{Sezession. (Siebzehnte Ausstellung der Vereinigung bildender
                        Künstler Österreichs)}\pwindex{Bahr, Hermann 19.07.1863 – 15.01.1934@\textsc{Bahr, Hermann} (19.07.1863 – 15.01.1934), \emph{Schriftsteller, Kritiker}!Sezession. (Siebzehnte Ausstellung der Vereinigung bildender Kuenstler Oesterreichs)07. 04. 1903@\strich\emph{Sezession. (Siebzehnte Ausstellung der Vereinigung bildender Künstler Österreichs)} {[}07. 04. 1903{]}|pwk}. In: \emph{Österreichische Volks-Zeitung}\pwindex{Oesterreichische Volks-Zeitung1888 – 13.11.1918@\emph{Österreichische Volks-Zeitung}|pwk}, Jg. 49, Nr. 96,
                        7. 4. 1903, S. 1.}}}\label{K_L01282_4h}«. Du haſt aber
               keine Ahnung, wie mich der Theaterbeſuch jetzt aufregt u. wie unſinnig mich die
               geringſte Arbeit {\pb}anſtrengt. Geſtern habe ich
               außerdem wieder einen Anfall jener Herzbeklemmungen bekommen, diesmal auch noch mit
               ſolchem Schwindel verbunden, daß ich den Nachmittag nur auf dem Sopha ausgeſtreckt,
               die Augen feſt geschloſſen, beide Hände auf die Schläfen gedrückt zubringen konnte,
               immer mit dem Gefühl, es iſt ja doch alles aus und ich werde niemals mehr geſund.
               Unter dieſen Bedingungen arbeite ich jetzt und darf daher eigentlich gar nichts
               verſprechen, weil ich mich bei jedem Feuilleton wundere, wenn es ſchließlich doch
               fertig geworden iſt.\pend
           \pstart
           Ferner mußt Du auch wiſſen, daß die Redacteure des {\pb}Neuen Wiener Tagblatt\orgindex{Neues Wiener Tagblatt@Neues Wiener Tagblatt|pw} (Wilhelm Singer\pwindex{Singer, Wilhelm 26.11.1847 – 10.10.1917@\textsc{Singer, Wilhelm} (26.11.1847 – 10.10.1917), \emph{Journalist, Chefredakteur}|pw} und den braven Herrn Epſtein\pwindex{Epstein, Moritz 1844 – 1915@\textsc{Epstein, Moritz} (1844 – 1915), \emph{Journalist/Journalistin}|pw} ausgenommen) einen Bund bilden, deſſen
               einzige Sorge es zu ſein ſcheint, auszuſinnen, was etwa geeignet wäre, mich zu
               ärgern, und dies mit der Behendigkeit von Affen ſogleich ins Blatt zu ſetzen. Daß
               gegen Dich noch nicht eine ungeheuerliche Gemeinheit verübt worden iſt, wundert mich
               ſchon lange. Geht ſie vielleicht gelegentlich des »Reigens\pwindex{Schnitzler, Arthur 15.05.1862 – 21.10.1931@\textsc{Schnitzler, Arthur} (15.05.1862 – 21.10.1931), \emph{Schriftsteller, Mediziner}!Reigen. Zehn Dialoge1900@\strich\emph{Reigen. Zehn Dialoge} {[}1900{]}|pw}« los, ſo vergiß nicht, daß ſie, zwar an Dir executiert, aber Dir gar
               nicht zugedacht iſt.\pend
           \pstart
           Bitte, ſchicke mir gleich ein Exemplar des »Reigens\pwindex{Schnitzler, Arthur 15.05.1862 – 21.10.1931@\textsc{Schnitzler, Arthur} (15.05.1862 – 21.10.1931), \emph{Schriftsteller, Mediziner}!Reigen. Zehn Dialoge1900@\strich\emph{Reigen. Zehn Dialoge} {[}1900{]}|pw}«. Meines iſt nemlich confisciert {\pb}worden, von der Cenſur. Das heißt: Der Herr Hofrath Jettel\pwindex{Jettel-Ettenach, Emil von 08.04.1846 – 25.04.1925@\textsc{Jettel-Ettenach, Emil von} (08.04.1846 – 25.04.1925), \emph{Rechtswissenschaftler, Ministerialbeamter, Zensor}|pw} hat es ſich bei mir ausleihen laſſen und ich habe es
               niemals mehr zurückbekommen.\pend
           \pstart
           Das Incohärente dieſes Briefes mußt Du meinem Zuſtand vergeben. Wie ich nur Zeit
               habe, fahre ich zunächſt zu Julius\pwindex{Schnitzler, Julius 13.07.1865 – 29.06.1939@\textsc{Schnitzler, Julius} (13.07.1865 – 29.06.1939), \emph{Chirurg}|pw}, der einmal
               doch mein Herz ordentlich unterſuchen muß.\pend
           \pstart
           \label{K_L01282_5v}\edtext{Freitag}{\lemma{\textnormal{\emph{Freitag}}}\Cendnote{\textnormal{der verpasste Besuch vom 27. 3.}}}\label{K_L01282_5h} war mir rieſig leid, ich war bei der Steuerbehörde, die mich auch noch
                  \label{K_L01282_6v}\edtext{ſekiert}{\lemma{\textnormal{\emph{ſekiert}}}\Cendnote{\textnormal{österreichisch sekkieren: ärgern, belästigen}}}\label{K_L01282_6h}.\pend
           \pstart
           Herzlichſt{\\[\baselineskip]}Dein{\\[\baselineskip]}\spacefill\mbox{Hermann}\pend
           \leftskip=0em{}\endnumbering\briefempfaengerindex{Schnitzler, Arthur@\textsc{Schnitzler, Arthur}!zzzBahr, Hermann@\emph{von Hermann Bahr}!1903-03-291@{{[}29.  3. 1903?{]}}|)be}\mylabel{h}\end{ledgroupsized}  \newcommand{\dateiname}{L01282}\newcommand{\titel}{Hermann Bahr an Arthur Schnitzler, [29. 3. 1903?]}\newcommand{\editorInnen}{ Kurt Ifkovits,  Martin Anton Müller}\input{../tex-inputs/latex-pdf-abspann}
      