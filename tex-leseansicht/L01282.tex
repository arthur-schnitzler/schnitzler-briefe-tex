%% latex-leseansicht-vorspann.tex
%% Vorspann für die Leseansicht.
%% Lädt die gemeinsame Datei latex-vorspann.tex mit nicht gesetztem Schalter.

\newif\ifkorrekturansicht
\korrekturansichtfalse

\input{../tex-inputs/latex-vorspann}


\section[Hermann Bahr an Arthur Schnitzler, {[}29. 3. 1903?{]}]{L01282 Hermann Bahr an Arthur Schnitzler, {[}29. 3. 1903?{]}}
\nopagebreak\mylabel{L01282v}
\rehead{ }\normalsize\beginnumbering\briefempfaengerindex{Schnitzler, Arthur@\textsc{Schnitzler, Arthur}!zzzBahr, Hermann@\emph{von Hermann Bahr}!1903-03-291@{{[}29.  3. 1903?{]}}|(be}
\toendnotes[C]{\smallbreak\pagebreak[2]}
\correspDesc{Versand  durch Hermann Bahr am [29.  3. 1903?] in Wien
\newline{}Erhalt  durch Arthur Schnitzler im Zeitraum [29. 3. 1903
                  – 2. 4. 1903?] in Wien}\toendnotes[C]{\smallbreak}
\Standort{CUL, Schnitzler, B 5b.}
\physDesc{Brief, 1 Blatt, 4 Seiten, 2131 Zeichen
\newline{}Handschrift: schwarze Tinte, deutsche Kurrent
\newline{}Schnitzler: mit Datum »Ende März 903« versehen 
\newline{}Ordnung: mit Bleistift von unbekannter Hand nummeriert:
                                    »96« }
\buchAbdrucke{\weitereDrucke{Hermann Bahr, Arthur Schnitzler: \emph{Briefwechsel, Aufzeichnungen, Dokumente (1891–1931)}. Herausgegeben von Kurt Ifkovits und Martin Anton Müller. Göttingen: \emph{Wallstein} 2018, S. 256.} }\toendnotes[C]{\smallbreak}
\pstart\center{}{\pb}Lieber Arthur,\pend\vspace{0.5em}
\pstart
           ſsehr gern und mit großer Freude{ }ſchreibe ich über den »Reigen\pwindex{Schnitzler, Arthur 15.\,5.\,1862 Wien – 21.\,10.\,1931 ebd.@\textsc{Schnitzler, Arthur} (15.\,5.\,1862 Wien – 21.\,10.\,1931 ebd.), \emph{Schriftsteller, Mediziner}!Reigen. Zehn Dialoge@\strich\emph{Reigen. Zehn Dialoge}|pw}« und natürlich{ }ſo bald als nur irgend möglich. Wann, das
               weiß ich freilich nicht und bitte Dich, damit nicht irgend eine Verſtimmung
               herauswächſt, folgendes zu bedenken. Ich muß dieſe Woche{ }ſechs Mal ins Theater gehen
               und{ }ſoll drei Feuilletons{ }ſchreiben, »\label{K_L01282-1v}\edtext{Die Duse\pwindex{Duse, Eleonora 3.\,10.\,1858 Vigevano – 21.\,4.\,1924 Pittsburgh@\textsc{Duse, Eleonora} (3.\,10.\,1858 Vigevano – 21.\,4.\,1924 Pittsburgh), \emph{Schauspielerin}|pw}\pwindex{Bahr, Hermann 19.\,7.\,1863 Linz – 15.\,1.\,1934 München@\textsc{Bahr, Hermann} (19.\,7.\,1863 Linz – 15.\,1.\,1934 München), \emph{Schriftsteller, Kritiker}!Duse. (Als Gast im Carl-Theater vom 31. März bis 8. April 1903)@\strich\emph{Die Duse. (Als Gast im Carl-Theater vom 31. März bis 8. April 1903)}|pw}}{\lemma{\textnormal{\emph{Die Duse}}}\Cendnote{\textnormal{Hermann Bahr\pwindex{Bahr, Hermann 19.\,7.\,1863 Linz – 15.\,1.\,1934 München@\textsc{Bahr, Hermann} (19.\,7.\,1863 Linz – 15.\,1.\,1934 München), \emph{Schriftsteller, Kritiker}|pwk}: \emph{Die Duse. (Als Gast im Carl-Theater vom 31. März
                        bis 8. April 1903)}\pwindex{Bahr, Hermann 19.\,7.\,1863 Linz – 15.\,1.\,1934 München@\textsc{Bahr, Hermann} (19.\,7.\,1863 Linz – 15.\,1.\,1934 München), \emph{Schriftsteller, Kritiker}!Duse. (Als Gast im Carl-Theater vom 31. März bis 8. April 1903)@\strich\emph{Die Duse. (Als Gast im Carl-Theater vom 31. März bis 8. April 1903)}|pwk}. In: \emph{Neues Wiener Tagblatt}\pwindex{Neues Wiener Tagblatt@\emph{Neues Wiener Tagblatt}|pwk}, Jg. 37, Nr. 89, 31. 3. 1903,
                     S. 1–2.}}}\label{K_L01282-1}«, »\label{K_L01282-2v}\edtext{l’altro pericolo\pwindex{\textcolor{red}{\textsuperscript{XXXX indx1}}!autre danger@\strich\emph{L’autre danger}|pw}\pwindex{Bahr, Hermann 19.\,7.\,1863 Linz – 15.\,1.\,1934 München@\textsc{Bahr, Hermann} (19.\,7.\,1863 Linz – 15.\,1.\,1934 München), \emph{Schriftsteller, Kritiker}!autre danger@\strich\emph{L’autre danger}|pw}}{\lemma{\textnormal{\emph{l’altro pericolo}}}\Cendnote{\textnormal{Hermann Bahr\pwindex{Bahr, Hermann 19.\,7.\,1863 Linz – 15.\,1.\,1934 München@\textsc{Bahr, Hermann} (19.\,7.\,1863 Linz – 15.\,1.\,1934 München), \emph{Schriftsteller, Kritiker}|pwk}: \emph{L’autre danger. (Komödie in vier Akten von Maurice Donnay. Zur
                        morgigen Aufführung im Carl-Theater durch die Truppe der Duse)}\pwindex{Bahr, Hermann 19.\,7.\,1863 Linz – 15.\,1.\,1934 München@\textsc{Bahr, Hermann} (19.\,7.\,1863 Linz – 15.\,1.\,1934 München), \emph{Schriftsteller, Kritiker}!autre danger@\strich\emph{L’autre danger}|pwk}. In: \emph{Neues Wiener Tagblatt}\pwindex{Neues Wiener Tagblatt@\emph{Neues Wiener Tagblatt}|pwk}, Jg. 37, Nr. 94,
                        4. 4. 1903, S. 1–3.}}}\label{K_L01282-2}«, »\label{K_L01282-3v}\edtext{Braut von Meſſina\pwindex{\textcolor{red}{\textsuperscript{XXXX indx1}}!Braut von Messina oder die feindlichen Brüder@\strich\emph{Die Braut von Messina oder die feindlichen Brüder}|pw}\pwindex{Bahr, Hermann 19.\,7.\,1863 Linz – 15.\,1.\,1934 München@\textsc{Bahr, Hermann} (19.\,7.\,1863 Linz – 15.\,1.\,1934 München), \emph{Schriftsteller, Kritiker}!Theater und Kunst. (Burgtheater) [Die Braut von Messina]@\strich\emph{Theater und Kunst. (Burgtheater) [Die Braut von Messina]}|pwv}}{\lemma{\textnormal{\emph{Braut von Messina}}}\Cendnote{\textnormal{Hermann Bahr\pwindex{Bahr, Hermann 19.\,7.\,1863 Linz – 15.\,1.\,1934 München@\textsc{Bahr, Hermann} (19.\,7.\,1863 Linz – 15.\,1.\,1934 München), \emph{Schriftsteller, Kritiker}|pwk}: \emph{Theater und Kunst. Burgtheater [Die Braut von Messina]}\pwindex{Bahr, Hermann 19.\,7.\,1863 Linz – 15.\,1.\,1934 München@\textsc{Bahr, Hermann} (19.\,7.\,1863 Linz – 15.\,1.\,1934 München), \emph{Schriftsteller, Kritiker}!Theater und Kunst. (Burgtheater) [Die Braut von Messina]@\strich\emph{Theater und Kunst. (Burgtheater) [Die Braut von Messina]}|pwk}. In:
                        \emph{Österreichische Volks-Zeitung}\pwindex{Österreichische Volks-Zeitung@\emph{Österreichische Volks-Zeitung}|pwk}, Jg. 49,
                     Nr. 96, 7. 4. 1903, S. 4.}}}\label{K_L01282-3}«, u. eigentlich auch noch
               eins über die »\label{K_L01282-4v}\edtext{Seceſſion\pwindex{Bahr, Hermann 19.\,7.\,1863 Linz – 15.\,1.\,1934 München@\textsc{Bahr, Hermann} (19.\,7.\,1863 Linz – 15.\,1.\,1934 München), \emph{Schriftsteller, Kritiker}!Sezession. (Siebzehnte Ausstellung der Vereinigung bildender Künstler Österreichs)@\strich\emph{Sezession. (Siebzehnte Ausstellung der Vereinigung bildender Künstler Österreichs)}|pwv}}{\lemma{\textnormal{\emph{Secession}}}\Cendnote{\textnormal{Hermann Bahr\pwindex{Bahr, Hermann 19.\,7.\,1863 Linz – 15.\,1.\,1934 München@\textsc{Bahr, Hermann} (19.\,7.\,1863 Linz – 15.\,1.\,1934 München), \emph{Schriftsteller, Kritiker}|pwk}: \emph{Sezession. (Siebzehnte Ausstellung der Vereinigung bildender
                        Künstler Österreichs)}\pwindex{Bahr, Hermann 19.\,7.\,1863 Linz – 15.\,1.\,1934 München@\textsc{Bahr, Hermann} (19.\,7.\,1863 Linz – 15.\,1.\,1934 München), \emph{Schriftsteller, Kritiker}!Sezession. (Siebzehnte Ausstellung der Vereinigung bildender Künstler Österreichs)@\strich\emph{Sezession. (Siebzehnte Ausstellung der Vereinigung bildender Künstler Österreichs)}|pwk}. In: \emph{Österreichische Volks-Zeitung}\pwindex{Österreichische Volks-Zeitung@\emph{Österreichische Volks-Zeitung}|pwk}, Jg. 49, Nr. 96,
                     7. 4. 1903, S. 1.}}}\label{K_L01282-4}«. Du haſt aber keine Ahnung, wie
               mich der Theaterbeſuch jetzt aufregt u. wie unſinnig mich die geringſte Arbeit {\pb}anſtrengt. Geſtern habe ich außerdem wieder einen
               Anfall jener Herzbeklemmungen bekommen, diesmal auch noch mit{ }ſolchem Schwindel
               verbunden, daß ich den Nachmittag nur auf dem Sopha ausgeſtreckt, die Augen feſt
               geschloſſen, beide Hände auf die Schläfen gedrückt zubringen konnte, immer mit dem
               Gefühl, es iſt ja doch alles aus und ich werde niemals mehr geſund. Unter dieſen
               Bedingungen arbeite ich jetzt und darf daher eigentlich gar nichts verſprechen, weil
               ich mich bei jedem Feuilleton wundere, wenn es{ }ſchließlich doch fertig geworden
               iſt.\pend
           
\pstart
           Ferner mußt Du auch wiſſen, daß die Redacteure des {\pb}Neuen Wiener Tagblatt\orgindex{Neues Wiener Tagblatt@Neues Wiener Tagblatt|pw} (Wilhelm Singer\pwindex{Singer, Wilhelm 26.\,11.\,1847 Bzenec – 10.\,10.\,1917 Wien@\textsc{Singer, Wilhelm} (26.\,11.\,1847 Bzenec – 10.\,10.\,1917 Wien), \emph{Journalist, Chefredakteur}|pw} und den braven Herrn Epſtein\pwindex{Epstein, Moritz 1.\,1.\,1844 Třebíč – 1915 Wien@\textsc{Epstein, Moritz} (1.\,1.\,1844 Třebíč – 1915 Wien), \emph{Journalist}|pw} ausgenommen) einen Bund bilden, deſſen einzige Sorge es
               zu{ }ſein{ }ſcheint, auszuſinnen, was etwa geeignet wäre, mich zu ärgern, und dies mit
               der Behendigkeit von Affen{ }ſogleich ins Blatt zu{ }ſetzen. Daß gegen Dich noch nicht
               eine ungeheuerliche Gemeinheit verübt worden iſt, wundert mich{ }ſchon lange. Geht{ }ſie
               vielleicht gelegentlich des »Reigens\pwindex{Schnitzler, Arthur 15.\,5.\,1862 Wien – 21.\,10.\,1931 ebd.@\textsc{Schnitzler, Arthur} (15.\,5.\,1862 Wien – 21.\,10.\,1931 ebd.), \emph{Schriftsteller, Mediziner}!Reigen. Zehn Dialoge@\strich\emph{Reigen. Zehn Dialoge}|pw}« los,{ }ſo
               vergiß nicht, daß{ }ſie, zwar an Dir executiert, aber Dir gar nicht zugedacht iſt.\pend
           
\pstart
           Bitte,{ }ſchicke mir gleich ein Exemplar des »Reigens\pwindex{Schnitzler, Arthur 15.\,5.\,1862 Wien – 21.\,10.\,1931 ebd.@\textsc{Schnitzler, Arthur} (15.\,5.\,1862 Wien – 21.\,10.\,1931 ebd.), \emph{Schriftsteller, Mediziner}!Reigen. Zehn Dialoge@\strich\emph{Reigen. Zehn Dialoge}|pw}«. Meines iſt nemlich confisciert {\pb}worden, von der Cenſur. Das heißt: Der Herr Hofrath Jettel\pwindex{Jettel-Ettenach, Emil von 8.\,4.\,1846 Wien – 25.\,4.\,1925 ebd.@\textsc{Jettel-Ettenach, Emil von} (8.\,4.\,1846 Wien – 25.\,4.\,1925 ebd.), \emph{Rechtswissenschaftler, Ministerialbeamter, Zensor}|pw} hat es{ }ſich bei mir ausleihen laſſen und ich habe es
               niemals mehr zurückbekommen.\pend
           
\pstart
           Das Incohärente dieſes Briefes mußt Du meinem Zuſtand vergeben. Wie ich nur Zeit
               habe, fahre ich zunächſt zu Julius\pwindex{Schnitzler, Julius 13.\,7.\,1865 Wien – 29.\,6.\,1939 ebd.@\textsc{Schnitzler, Julius} (13.\,7.\,1865 Wien – 29.\,6.\,1939 ebd.), \emph{Chirurg}|pw}, der einmal
               doch mein Herz ordentlich unterſuchen muß.\pend
           
\pstart
           \label{K_L01282-5v}\edtext{Freitag}{\lemma{\textnormal{\emph{Freitag}}}\Cendnote{\textnormal{der verpasste Besuch vom 27. 3.}}}\label{K_L01282-5} war mir rieſig leid, ich war bei der Steuerbehörde, die mich auch noch
                  \label{K_L01282-6v}\edtext{ſekiert}{\lemma{\textnormal{\emph{sekiert}}}\Cendnote{\textnormal{österreichisch sekkieren: ärgern, belästigen}}}\label{K_L01282-6}.\pend
           
\pstart
           Herzlichſt{\\[\baselineskip]}Dein{\\[\baselineskip]}\spacefill\mbox{Hermann}\pend
           \leftskip=0em{}\selectlanguage{ngerman}\endnumbering\briefempfaengerindex{Schnitzler, Arthur@\textsc{Schnitzler, Arthur}!zzzBahr, Hermann@\emph{von Hermann Bahr}!1903-03-291@{{[}29.  3. 1903?{]}}|)be}\mylabel{L01282h}  \newcommand{\dateiname}{L01282}\newcommand{\titel}{Hermann Bahr an Arthur Schnitzler, [29. 3. 1903?]}\newcommand{\editorInnen}{Herausgegeben von Martin Anton Müller}%% latex-leseansicht-abspann.tex
%% Abspann für die Leseansicht.
%% Der Schalter \ifkorrekturansicht ist bereits durch den Vorspann gesetzt.

%% latex-abspann.tex
%% Gemeinsamer Abspann für Korrekturansicht und Leseansicht.
%% Setzt den Schalter \ifkorrekturansicht voraus (gesetzt in den
%% einbindenden Dateien latex-korrekturansicht-abspann.tex bzw.
%% latex-leseansicht-abspann.tex).
%% ---------------------------------------------------------------

\normalsize

% Das esempio-Environment wird nur in der Leseansicht benötigt
\ifkorrekturansicht\else
\newenvironment{esempio}[3]%
{
    \vspace{1.5ex}
    \rlap{\underline{#1}}
    \par
    \setlength{\parindent}{0cm}
    \nopagebreak
    \leftskip=#2cm
    \rightskip=#3cm
}
{
    \par
}
\fi

\doendnotes{C}
\bigskip
\vfill

\clearpage

\footnotesize

\ifkorrekturansicht
  \lohead{\textsc{register}}
\fi

% theindex-Environment neu definieren ohne reledmac
\makeatletter
\renewenvironment{theindex}{%
  \ifkorrekturansicht
    \section*{\indexname}%
  \else
    \subsubsection*{Index der erwähnten Entitäten}%
  \fi
  \setlength{\parindent}{0pt}%
  \setlength{\parskip}{0pt plus 0.3pt}%
  \let\item\@idxitem
}{%
  \ifkorrekturansicht\clearpage\fi
}
\makeatother

\IfFileExists{\jobname-pw.ind}{\input{\jobname-pw.ind}}{}

% Quellenangabe nur in der Leseansicht
\ifkorrekturansicht\else
% Fallback-Definitionen, falls die .tex-Datei \titel etc. nicht gesetzt hat
\providecommand{\titel}{}
\providecommand{\editorInnen}{}
\providecommand{\dateiname}{\jobname}

\vspace{3cm}

\vfill

\footnotesize
\textsc{Quelle}: \titel. Herausgegeben von {\editorInnen}. In: \emph{Arthur Schnitzler: Briefwechsel mit Autorinnen und Autoren}.
 Digitale Edition, https://schnitzler-briefe.acdh.oeaw.ac.at/{\dateiname}.html (Stand \today)
\fi

\end{document}


