%% latex-korrekturansicht-vorspann.tex
%% Vorspann für die Korrekturansicht.
%% Lädt die gemeinsame Datei latex-vorspann.tex mit gesetztem Schalter.

\newif\ifkorrekturansicht
\korrekturansichttrue

\input{../tex-inputs/latex-vorspann}


\section[Hermann Bahr an Arthur Schnitzler, {[}19. 6. 1895{]}]{L00455 Hermann Bahr an Arthur Schnitzler, {[}19. 6. 1895{]}}
\nopagebreak\mylabel{L00455v}
\rehead{ }\normalsize\beginnumbering\briefempfaengerindex{Schnitzler, Arthur@\textsc{Schnitzler, Arthur}!zzzBahr, Hermann@\emph{von Hermann Bahr}!1895-06-191@{{[}19. 6. 1895{]}}|(be}
\toendnotes[C]{\smallbreak\pagebreak[2]}\Standort{CUL, Schnitzler, B 5b.}
\physDesc{Brief, 1 Blatt, 2 Seiten, 439 Zeichen
\newline{}Handschrift: schwarze Tinte, deutsche Kurrent
\newline{}Schnitzler: mit Bleistift datiert: »19/6 {[}189{]}5« 
\newline{}Ordnung: 1) mit rotem Buntstift von unbekannter Hand nummeriert:
                                    »29«  2) mit Bleistift von unbekannter Hand nummeriert:
                                    »29«}
\buchAbdrucke{\weitereDrucke{Hermann Bahr, Arthur Schnitzler: \emph{Briefwechsel, Aufzeichnungen, Dokumente (1891–1931)}. Göttingen: \emph{Wallstein} 2018, S. 102.} }\toendnotes[C]{\smallbreak}
\pstart
           {\pb}\textcolor{gray}{\textbf{»Die Zeit\orgindex{Zeit. Wiener Wochenschrift@Die Zeit. Wiener Wochenschrift|pw}«}}\hfill \textcolor{gray}{\textbf{\textbf{Wien\oindex{Wien@\textbf{Wien}, \emph{A.ADM2}|pw}}, den ..........189{\dotstwo}}}\pend
           
\pstart
           \textcolor{gray}{\textbf{Wiener Wochenſchrift}}\hfill \textcolor{gray}{\textbf{IX/3, Günthergaſſe 1\oindex{Guenthergasse@\textbf{Günthergasse}, \emph{Straße (K.STR)}|pw}.}}\pend
           
\pstart
           \textcolor{gray}{\textbf{\textbf{Herausgeber}:}}{\\}\textcolor{gray}{\textbf{Profeſſor Dr. I. Singer\pwindex{Singer, Isidor 16.01.1857 – 08.12.1927@\textsc{Singer, Isidor} (16.01.1857 – 08.12.1927), \emph{Journalist/Journalistin, Herausgeber/Herausgeberin, Soziologe/Soziologin}|pw}, Hermann Bahr\pwindex{Bahr, Hermann 19.07.1863 – 15.01.1934@\textsc{Bahr, Hermann} (19.07.1863 – 15.01.1934), \emph{Schriftsteller/Schriftstellerin, Kritiker/Kritikerin}|pw},
                        Dr. Heinrich Kanner\pwindex{Kanner, Heinrich 09.11.1864 – 15.02.1930@\textsc{Kanner, Heinrich} (09.11.1864 – 15.02.1930), \emph{Herausgeber/Herausgeberin, Publizist/Publizistin}|pw}.}}\pend
           
\pstart
           \textcolor{gray}{\textbf{Telephon Nr. 6415.}}\pend
           
\pstart{}Lieber Arthur!\pend\vspace{0.5em}
\pstart
           Ich möchte ſehr, ſehr gern etwas von Dir für die »Zeit\orgindex{Zeit. Wiener Wochenschrift@Die Zeit. Wiener Wochenschrift|pw}« haben. Lieber wäre mir eine kurze Geſchichte, nicht über 8 Spalten des
               Blattes. \textsc{Faute de mieux}, nehme ich auch eine lange, obwohl
               ich an \label{K_L00455-1v}\edtext{\textsc{d’Annunzio}\pwindex{DAnnunzio, Gabriele 12.03.1863 – 01.03.1938@\textsc{D’Annunzio, Gabriele} (12.03.1863 – 01.03.1938), \emph{Schriftsteller/Schriftstellerin}|pw}\pwindex{Giovanni Episcopo@\emph{Giovanni Episcopo}|pwv}}{\lemma{\textnormal{\emph{d’Annunzio}}}\Cendnote{\textnormal{Gabriele d’Annunzio\pwindex{DAnnunzio, Gabriele 12.03.1863 – 01.03.1938@\textsc{D’Annunzio, Gabriele} (12.03.1863 – 01.03.1938), \emph{Schriftsteller/Schriftstellerin}|pwk}: \emph{Giovanni Episcopo}\pwindex{Giovanni Episcopo@\emph{Giovanni Episcopo}|pwk}. In: \emph{Die
                        Zeit}\pwindex{Zeit. Wiener Wochenschrift@\emph{Die Zeit. Wiener Wochenschrift}|pwk}, Bd. 1, Nr. 9, 1. 12. 1894 – Bd. 2, Nr. 16,
                        19. 1. 1895 (8 Teile).}}}\label{K_L00455-1} erfahren habe, daß das
               Zerreißen in Fortſetzungen auch die stärksten Sachen umbringt.\pend
           
\pstart
           Deine Novelle\pwindex{Spaeter Ruhm@\emph{Später Ruhm}|pwv} könnte im
               Oktober er{\pb}ſcheinen.\pend
           
\pstart
           \label{K_L00455-2v}\edtext{Ich fahre heute Abend}{\lemma{\textnormal{\emph{Ich fahre heute Abend}}}\Cendnote{\textnormal{Vom 19. 6. bis zum
                     12. 7. 1895 machte Bahr\pwindex{Bahr, Hermann 19.07.1863 – 15.01.1934@\textsc{Bahr, Hermann} (19.07.1863 – 15.01.1934), \emph{Schriftsteller/Schriftstellerin, Kritiker/Kritikerin}|pwk}{ }Sommerurlaub. Er besuchte drei Tage München\oindex{Muenchen@\textbf{München}, \emph{P.PPLA}|pwk}, dann Schliersee\oindex{Schliersee@\textbf{Schliersee}, \emph{P.PPL}|pwk} und den Starnberger See\oindex{Starnberger See@\textbf{Starnberger See}, \emph{H.LK}|pwk} sowie
                     Innsbruck\oindex{Innsbruck@\textbf{Innsbruck}, \emph{A.ADM2}|pwk} und die Gegend von Kufstein\oindex{Kufstein@\textbf{Kufstein}, \emph{P.PPLA3}|pwk}.}}}\label{K_L00455-2} nach München\oindex{Muenchen@\textbf{München}, \emph{P.PPLA}|pw} und dann auf drei Wochen ins bairiſche Gebirg\oindex{Bayern@\textbf{Bayern}, \emph{A.ADM1}|pw}.\pend
           
\pstart
           Herzlichst{\\[\baselineskip]}Dein{\\[\baselineskip]}\spacefill\mbox{Hermann}\pend
           \leftskip=0em{}
\pstart
           \textcolor{gray}{\textbf{\label{T_L00455-1v}\edtext{Alle für »Die Zeit\orgindex{Zeit. Wiener Wochenschrift@Die Zeit. Wiener Wochenschrift|pw}« beſtimmten Zuſchriften und Sendungen ſind an die
                  Redaction der »Zeit\orgindex{Zeit. Wiener Wochenschrift@Die Zeit. Wiener Wochenschrift|pw}« und \textbf{nicht} an die Perſon eines der Herausgeber zu richten.}{\lemma{\textnormal{\emph{Alle … richten.}}}\Cendnote{\textnormal{am unteren Rand der ersten Seite}}}\label{T_L00455-1}}}\pend
           \selectlanguage{ngerman}\endnumbering\briefempfaengerindex{Schnitzler, Arthur@\textsc{Schnitzler, Arthur}!zzzBahr, Hermann@\emph{von Hermann Bahr}!1895-06-191@{{[}19. 6. 1895{]}}|)be}\mylabel{L00455h}  \normalsize

\doendnotes{C}
\bigskip
\vfill

\clearpage

\footnotesize

\lohead{\textsc{register}}

% Definiere theindex-Environment komplett neu ohne reledmac
\makeatletter
\renewenvironment{theindex}{%
  \section*{\indexname}%
  \setlength{\parindent}{0pt}%
  \setlength{\parskip}{0pt plus 0.3pt}%
  \let\item\@idxitem
}{%
  \clearpage
}
\makeatother

\IfFileExists{\jobname-pw.ind}{\input{\jobname-pw.ind}}{}

\end{document}

      