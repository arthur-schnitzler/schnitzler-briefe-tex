%% latex-korrekturansicht-vorspann.tex
%% Vorspann für die Korrekturansicht.
%% Lädt die gemeinsame Datei latex-vorspann.tex mit gesetztem Schalter.

\newif\ifkorrekturansicht
\korrekturansichttrue

\input{../tex-inputs/latex-vorspann}


\section[Richard Beer-Hofmann an Arthur Schnitzler, 18. 2. 1900]{L01015 Richard Beer-Hofmann an Arthur Schnitzler, 18. 2. 1900}
\nopagebreak\mylabel{L01015v}
\rehead{ }\normalsize\beginnumbering\briefempfaengerindex{Schnitzler, Arthur@\textsc{Schnitzler, Arthur}!zzzBeer-Hofmann, Richard@\emph{von Richard Beer-Hofmann}!1900-02-181@{18. 2. 1900}|(be}
\toendnotes[C]{\smallbreak\pagebreak[2]}\Standort{CUL, Schnitzler, B 8.}
\physDesc{Bildpostkarte, 232 Zeichen
\newline{}Handschrift: schwarze Tinte, lateinische Kurrent
\newline{}Versand: 1) Stempel: »\nobreak{}\oindex{Genua@\textbf{Genua}, \emph{P.PPLA}|pwk}{[}Ferrov{]}ia Genova, 18 \textcolor{gray}{2} 00\nobreak{}«.   2) Stempel: »\nobreak{}\oindex{IX., Alsergrund@\textbf{IX., Alsergrund}, \emph{A.ADM3}|pwk}Wien 9/3 72, 20. 2. 00, 10.V, Beste{[}llt{]}\nobreak{}«. 
\newline{}Ordnung: mit Bleistift von unbekannter Hand nummeriert:
                                    »150« }
\buchAbdrucke{\weitereDrucke{Arthur Schnitzler, Richard Beer-Hofmann: \emph{Briefwechsel 1891–1931}. Wien, Zürich: \emph{Europaverlag} 1992, S. 142.} }\pstart{}{\pb}D\textsuperscript{r}
                  Arthur Schnitzler\pend{}\pstart{}Wien\oindex{Wien@\textbf{Wien}, \emph{A.ADM2}|pw}\pend{}\pstart{}IX Frankgasse 1\oindex{Frankgasse 1@\textbf{Frankgasse 1}, \emph{Wohngebäude (K.WHS)}|pw}\pend{}\pstart{}Austria\oindex{Oesterreich@\textbf{Österreich}, \emph{A.PCLI}|pw}\pend{}{\bigskip}
\pstart
           \noindent{}\centering{}{\pb}\textcolor{gray}{\textbf{Manuel Wielandt\pwindex{Wielandt, Manuel 1863-12-20 – 1922-05-11@\textsc{Wielandt, Manuel} (1863-12-20 – 1922-05-11), \emph{Maler/Malerin, Radierer/Radiererin}|pw}.}}\pend
           
\pstart
           \centering{}\textcolor{gray}{\textbf{SAN REMO\oindex{Sanremo@\textbf{Sanremo}, \emph{P.PPLA3}|pw}}}\pend
           \vspace{1em}
\pstart
           
\pstart
           {\pb}\uline{Hôtel de Rome\oindex{Hôtel Victoria et de Rome@\textbf{Hôtel Victoria et de Rome}, \emph{Hotel (K.HTL)}|pw}}\pend
           
\pstart
           \raggedleft{}18/II 1900.\pend
           \pend
           \vspace{0.5em}
\pstart
           Lieber Arthur, ko{\geminationm}en Sie mit Mayer\pwindex{Mayer, Oskar 1876 – 15.05.1915@\textsc{Mayer, Oskar} (1876 – 15.05.1915), \emph{Schriftsteller/Schriftstellerin, Beamter/Beamte}|pw} doch her, wir gehen dann nach Florenz\oindex{Florenz@\textbf{Florenz}, \emph{P.PPLA}|pw}. – Ich weiß übrigens: weil Sie nichts in
                  Wien\oindex{Wien@\textbf{Wien}, \emph{A.ADM2}|pw} hält, können Sie nicht fort. Herzlichst
               Ihr\pend
           \pstart \spacefill\mbox{Richard}\pend{}\selectlanguage{ngerman}\endnumbering\briefempfaengerindex{Schnitzler, Arthur@\textsc{Schnitzler, Arthur}!zzzBeer-Hofmann, Richard@\emph{von Richard Beer-Hofmann}!1900-02-181@{18. 2. 1900}|)be}\mylabel{L01015h}  \normalsize

\doendnotes{C}
\bigskip
\vfill

\clearpage

\footnotesize

\lohead{\textsc{register}}

% Definiere theindex-Environment komplett neu ohne reledmac
\makeatletter
\renewenvironment{theindex}{%
  \section*{\indexname}%
  \setlength{\parindent}{0pt}%
  \setlength{\parskip}{0pt plus 0.3pt}%
  \let\item\@idxitem
}{%
  \clearpage
}
\makeatother

\IfFileExists{\jobname-pw.ind}{\input{\jobname-pw.ind}}{}

\end{document}

      