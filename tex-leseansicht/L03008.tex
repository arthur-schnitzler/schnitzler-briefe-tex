%% latex-korrekturansicht-vorspann.tex
%% Vorspann für die Korrekturansicht.
%% Lädt die gemeinsame Datei latex-vorspann.tex mit gesetztem Schalter.

\newif\ifkorrekturansicht
\korrekturansichttrue

\input{../tex-inputs/latex-vorspann}


\section[ Arthur Schnitzler an Felix Salten, 18. 4. 1907]{L03008 Arthur Schnitzler an Felix Salten, 18. 4. 1907}
\nopagebreak\mylabel{L03008v}
\rehead{ }\normalsize\beginnumbering\briefempfaengerindex{Salten, Felix@\textsc{Salten, Felix}!zzzSchnitzler, Arthur@\emph{von Arthur Schnitzler}!1907-04-181@{18. 4. 1907}|(be}
\toendnotes[C]{\smallbreak\pagebreak[2]}\Standort{Wienbibliothek im Rathaus, ZPH 1681, 2.1.516.}
\physDesc{Brief, 1 Blatt, 1 Seite, 350 Zeichen
\newline{}Handschrift: schwarze Tinte, deutsche Kurrent
\newline{}Ordnung: mit Bleistift von unbekannter Hand nummeriert: »12« }\toendnotes[C]{\smallbreak}
\pstart
           \raggedleft{}{\pb}18. 4. 907\pend
           
\pstart{}lieber,\pend\vspace{0.5em}
\pstart
           Herr \textsc{Rotenstern\pwindex{Rotenstern, Peter 10.01.1868 – 1944@\textsc{Rotenstern, Peter} (10.01.1868 – 1944), \emph{Journalist/Journalistin, Übersetzer/Übersetzerin}|pw}}, mein ruſſ.\oindex{Russland@\textbf{Russland}, \emph{A.PCLI}|pw} Überſetzer iſt jetzt in \textsc{Paris\oindex{Paris@\textbf{Paris}, \emph{P.PPLC}|pw}} und möchte gern »Vertreter« \textsc{Lautenburgs\pwindex{Lautenburg, Sigmund 11.09.1851 – 21.07.1918@\textsc{Lautenburg, Sigmund} (11.09.1851 – 21.07.1918), \emph{Theaterleiter/Theaterleiterin, Schauspieler/Schauspielerin}|pw}}, \textsc{resp.} des Raimundtheaters\orgindex{Raimund-Theater@Raimund-Theater|pw} dort ſein. We{\geminationn} es Ihnen bei
               Gelegenheit möglich \strikeout{iſt} und nicht aus irgd einem
               Grund unangenehm iſt, könnten Sie zu L.\pwindex{Lautenburg, Sigmund 11.09.1851 – 21.07.1918@\textsc{Lautenburg, Sigmund} (11.09.1851 – 21.07.1918), \emph{Theaterleiter/Theaterleiterin, Schauspieler/Schauspielerin}|pw} ein
               Wort in dieſem Sinne äußern?\pend
           \pstart Herzlichſt mit Grüßen von Haus zu Haus Ihr \spacefill\mbox{Arthur.}\pend{}
\pstart
           \noindent{}Vielleicht \label{K_L03008-1v}\edtext{morgen{ }\textsc{Tennis}}{\lemma{\textnormal{\emph{morgen Tennis}}}\Cendnote{\textnormal{Am 19. 4. 1907 spielte Schnitzler nur mit Felix Speidel\pwindex{Speidel, Felix 02.07.1875 – 1952-10-03@\textsc{Speidel, Felix} (02.07.1875 – 1952-10-03), \emph{Schriftsteller/Schriftstellerin, Verleger/Verlegerin}|pwk}.}}}\label{K_L03008-1}?\pend
           \selectlanguage{ngerman}\endnumbering\briefempfaengerindex{Salten, Felix@\textsc{Salten, Felix}!zzzSchnitzler, Arthur@\emph{von Arthur Schnitzler}!1907-04-181@{18. 4. 1907}|)be}\mylabel{L03008h}  \normalsize

\doendnotes{C}
\bigskip
\vfill

\clearpage

\footnotesize

\lohead{\textsc{register}}

% Definiere theindex-Environment komplett neu ohne reledmac
\makeatletter
\renewenvironment{theindex}{%
  \section*{\indexname}%
  \setlength{\parindent}{0pt}%
  \setlength{\parskip}{0pt plus 0.3pt}%
  \let\item\@idxitem
}{%
  \clearpage
}
\makeatother

\IfFileExists{\jobname-pw.ind}{\input{\jobname-pw.ind}}{}

\end{document}

      