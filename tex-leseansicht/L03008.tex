%% latex-leseansicht-vorspann.tex
%% Vorspann für die Leseansicht.
%% Lädt die gemeinsame Datei latex-vorspann.tex mit nicht gesetztem Schalter.

\newif\ifkorrekturansicht
\korrekturansichtfalse

\input{../tex-inputs/latex-vorspann}


         
         \renewcommand{\erwaehntePersonen}{Personen: Sigmund Lautenburg, Peter Rotenstern, Felix Salten, Felix Speidel}
         \renewcommand{\erwaehnteInstitutionen}{Institutionen: Raimund-Theater}
         \renewcommand{\erwaehnteOrte}{Orte: Paris, Russland, Wien}
         \renewcommand{\erwaehnteWerke}{}
               \section[ Arthur Schnitzler an Felix Salten, 18. 4. 1907]{ Arthur Schnitzler an Felix Salten, 18. 4. 1907}\nopagebreak\mylabel{v}\rehead{ }\begin{ledgroupsized}[t]{13cm}\normalsize\beginnumbering\briefempfaengerindex{Salten, Felix@\textsc{Salten, Felix}!zzzSchnitzler, Arthur@\emph{von Arthur Schnitzler}!1907-04-181@{18. 4. 1907}|(be} \toendnotes[C]{\smallbreak\pagebreak[2]} \Standort{Wienbibliothek im Rathaus, ZPH 1681, 2.1.516.}
\physDesc{Brief, 1 Blatt, 1 Seite, 350 Zeichen
\newline{}Handschrift: schwarze Tinte, deutsche Kurrent
\newline{}Ordnung: mit Bleistift von unbekannter Hand nummeriert: »12« }\toendnotes[C]{\smallbreak}\pstart
           \raggedleft{}{\pb}18. 4. 907\pend
           \pstart{}lieber,\pend\pstart
           Herr \textsc{Rotenstern\pwindex{Rotenstern, Peter 10.01.1868 – 1944@\textsc{Rotenstern, Peter} (10.01.1868 – 1944), \emph{Journalist, Übersetzer}|pw}}, mein ruſſ.\oindex{Russland@\textbf{Russland}|pw} Überſetzer iſt jetzt in \textsc{Paris\oindex{Paris@\textbf{Paris}|pw}} und möchte gern »Vertreter« \textsc{Lautenburg\pwindex{Lautenburg, Sigmund 11.09.1851 – 21.07.1918@\textsc{Lautenburg, Sigmund} (11.09.1851 – 21.07.1918), \emph{Theaterleiter, Schauspieler}|pw}}s, \textsc{resp.} des Raimundtheater\orgindex{Raimund-Theater@Raimund-Theater|pw}s dort ſein. We{\geminationn} es Ihnen bei
               Gelegenheit möglich \strikeout{iſt} und nicht aus irgd einem
               Grund unangenehm iſt, könnten Sie zu L.\pwindex{Lautenburg, Sigmund 11.09.1851 – 21.07.1918@\textsc{Lautenburg, Sigmund} (11.09.1851 – 21.07.1918), \emph{Theaterleiter, Schauspieler}|pw} ein
               Wort in dieſem Sinne äußern?\pend
           \pstart Herzlichſt mit Grüßen von Haus zu Haus Ihr \spacefill\mbox{Arthur.}\pend{}\pstart
           \noindent{}Vielleicht \label{K_L03008-1v}\edtext{morgen{ }\textsc{Tennis}}{\lemma{\textnormal{\emph{morgen Tennis}}}\Cendnote{\textnormal{Am 19. 4. 1907 spielte Schnitzler\pwindex{Schnitzler, Arthur 15.05.1862 – 21.10.1931@\textsc{Schnitzler, Arthur} (15.05.1862 – 21.10.1931), \emph{Schriftsteller, Mediziner}|pwk} nur mit Felix Speidel\pwindex{Speidel, Felix 02.07.1875 – 1952-10-03@\textsc{Speidel, Felix} (02.07.1875 – 1952-10-03), \emph{Schriftsteller, Verleger}|pwk}.}}}\label{K_L03008-1h}?\pend
           
         
         \endnumbering\mylabel{h}\end{ledgroupsized}  \newcommand{\dateiname}{L03008}\newcommand{\titel}{Arthur Schnitzler an Felix Salten, 18. 4. 1907}\newcommand{\editorInnen}{Martin Anton Müller und Laura Untner}%% latex-leseansicht-abspann.tex
%% Abspann für die Leseansicht.
%% Der Schalter \ifkorrekturansicht ist bereits durch den Vorspann gesetzt.

%% latex-abspann.tex
%% Gemeinsamer Abspann für Korrekturansicht und Leseansicht.
%% Setzt den Schalter \ifkorrekturansicht voraus (gesetzt in den
%% einbindenden Dateien latex-korrekturansicht-abspann.tex bzw.
%% latex-leseansicht-abspann.tex).
%% ---------------------------------------------------------------

\normalsize

% Das esempio-Environment wird nur in der Leseansicht benötigt
\ifkorrekturansicht\else
\newenvironment{esempio}[3]%
{
    \vspace{1.5ex}
    \rlap{\underline{#1}}
    \par
    \setlength{\parindent}{0cm}
    \nopagebreak
    \leftskip=#2cm
    \rightskip=#3cm
}
{
    \par
}
\fi

\doendnotes{C}
\bigskip
\vfill

\clearpage

\footnotesize

\ifkorrekturansicht
  \lohead{\textsc{register}}
\fi

% theindex-Environment neu definieren ohne reledmac
\makeatletter
\renewenvironment{theindex}{%
  \ifkorrekturansicht
    \section*{\indexname}%
  \else
    \subsubsection*{Index der erwähnten Entitäten}%
  \fi
  \setlength{\parindent}{0pt}%
  \setlength{\parskip}{0pt plus 0.3pt}%
  \let\item\@idxitem
}{%
  \ifkorrekturansicht\clearpage\fi
}
\makeatother

\IfFileExists{\jobname-pw.ind}{\input{\jobname-pw.ind}}{}

% Quellenangabe nur in der Leseansicht
\ifkorrekturansicht\else
% Fallback-Definitionen, falls die .tex-Datei \titel etc. nicht gesetzt hat
\providecommand{\titel}{}
\providecommand{\editorInnen}{}
\providecommand{\dateiname}{\jobname}

\vspace{3cm}

\vfill

\footnotesize
\textsc{Quelle}: \titel. Herausgegeben von {\editorInnen}. In: \emph{Arthur Schnitzler: Briefwechsel mit Autorinnen und Autoren}.
 Digitale Edition, https://schnitzler-briefe.acdh.oeaw.ac.at/{\dateiname}.html (Stand \today)
\fi

\end{document}


      