%% latex-korrekturansicht-vorspann.tex
%% Vorspann für die Korrekturansicht.
%% Lädt die gemeinsame Datei latex-vorspann.tex mit gesetztem Schalter.

\newif\ifkorrekturansicht
\korrekturansichttrue

\input{../tex-inputs/latex-vorspann}


\section[Arthur Schnitzler an Hugo von Hofmannsthal, {[}12. 3. 1892{]}]{L00080 Arthur Schnitzler an Hugo von Hofmannsthal, {[}12. 3. 1892{]}}
\nopagebreak\mylabel{L00080v}
\rehead{ }\normalsize\beginnumbering\briefempfaengerindex{Hofmannsthal, Hugo von@\textsc{Hofmannsthal, Hugo von}!zzzSchnitzler, Arthur@\emph{von Arthur Schnitzler}!1892-03-121@{{[}12. 3. 1892{]}}|(be}
\toendnotes[C]{\smallbreak\pagebreak[2]}\Standort{FDH, Hs-30885,16.}
\physDesc{Briefkarte, 261 Zeichen
\newline{}Handschrift: Bleistift, deutsche Kurrent
\newline{}Ordnung: mit Bleistift von Schnitzler mutmaßlich während der Durchsicht der Briefe
                                    1929 mit einer Jahreszahl versehen: »9\substVorne{}\textsuperscript{2}\substDazwischen{}1\substHinten{}« }
\buchAbdrucke{\weitereDrucke{Hugo von Hofmannsthal, Arthur Schnitzler: \emph{Briefwechsel}. Frankfurt am Main: \emph{S. Fischer} 1964, S. 30.} }\toendnotes[C]{\smallbreak}
\pstart
           \noindent{}{\pb}Lieber Hugo, morgen So{\geminationn}tag bin ich Nachmittags in einem
                  \label{K_L00080-1v}\edtext{Concert}{\lemma{\textnormal{\emph{Concert}}}\Cendnote{\textnormal{Nur ein Konzert an einem Sonntagnachmittag
                     lässt sich nachweisen, das 
                     Werke von
                     Anton Rückauf\pwindex{Rueckauf, Anton 13.03.1855 – 19.09.1903@\textsc{Rückauf, Anton} (13.03.1855 – 19.09.1903), \emph{Komponist/Komponistin, Pianist/Pianistin}|pwk} auf dem Programm hatte. Es fand am 13. 3. 1892 statt. In Schnitzlers Aufzeichnungen gibt es keinen
                  Hinweis auf eine Teilnahme; für den Tag ist nur ein Besuch in der \emph{Schauspielschule Otto}\orgindex{Schauspielschule Otto@Schauspielschule Otto|pwk} vermerkt (\emph{Cambridge University Library} A 179a).}}}\label{K_L00080-1}, wo
                  Rückauf\pwindex{Rueckauf, Anton 13.03.1855 – 19.09.1903@\textsc{Rückauf, Anton} (13.03.1855 – 19.09.1903), \emph{Komponist/Komponistin, Pianist/Pianistin}|pw} (mein einſtiger Lehrer, der mich
               ſehr intereſſirt) aufgeführt wird. Alſo nicht {\pb}zu Hauſe.
                  Ko{\geminationm}en Sie möglichſt bald, damit wir noch einen Abend
               dieſer Woche verabreden können.\pend
           
\pstart
           Herzlichst{\\[\baselineskip]}Ihr\spacefill\mbox{Arth Sch}\pend
           \leftskip=0em{}\selectlanguage{ngerman}\endnumbering\briefempfaengerindex{Hofmannsthal, Hugo von@\textsc{Hofmannsthal, Hugo von}!zzzSchnitzler, Arthur@\emph{von Arthur Schnitzler}!1892-03-121@{{[}12. 3. 1892{]}}|)be}\mylabel{L00080h}  \normalsize

\doendnotes{C}
\bigskip
\vfill

\clearpage

\footnotesize

\lohead{\textsc{register}}

% Definiere theindex-Environment komplett neu ohne reledmac
\makeatletter
\renewenvironment{theindex}{%
  \section*{\indexname}%
  \setlength{\parindent}{0pt}%
  \setlength{\parskip}{0pt plus 0.3pt}%
  \let\item\@idxitem
}{%
  \clearpage
}
\makeatother

\IfFileExists{\jobname-pw.ind}{\input{\jobname-pw.ind}}{}

\end{document}

      