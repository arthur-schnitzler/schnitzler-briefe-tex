%% latex-leseansicht-vorspann.tex
%% Vorspann für die Leseansicht.
%% Lädt die gemeinsame Datei latex-vorspann.tex mit nicht gesetztem Schalter.

\newif\ifkorrekturansicht
\korrekturansichtfalse

\input{../tex-inputs/latex-vorspann}


         
         \renewcommand{\erwaehntePersonen}{Personen: Hugo von Hofmannsthal, Anton Rückauf}
         \renewcommand{\erwaehnteInstitutionen}{Institutionen: Schauspielschule Otto}
         \renewcommand{\erwaehnteOrte}{Orte: Wien}
         \renewcommand{\erwaehnteWerke}{}
               \section[Arthur Schnitzler an Hugo von Hofmannsthal, {[}12. 3. 1892{]}]{ Arthur Schnitzler an Hugo von Hofmannsthal, {[}12. 3. 1892{]}}\nopagebreak\mylabel{v}\rehead{ }\begin{ledgroupsized}[t]{13cm}\normalsize\beginnumbering\briefempfaengerindex{Hofmannsthal, Hugo von@\textsc{Hofmannsthal, Hugo von}!zzzSchnitzler, Arthur@\emph{von Arthur Schnitzler}!1892-03-121@{{[}12. 3. 1892{]}}|(be} \toendnotes[C]{\smallbreak\pagebreak[2]} \Standort{FDH, Hs-30885,16.}
\physDesc{Briefkarte, 261 Zeichen
\newline{}Handschrift: Bleistift, deutsche Kurrent
\newline{}Ordnung: mit Bleistift von Schnitzler mutmaßlich während der Durchsicht der Briefe
                                    1929 mit einer Jahreszahl versehen: »9\substVorne{}\textsuperscript{2}\substDazwischen{}1\substHinten{}« }\buchAbdrucke{\weitereDrucke{Hugo von Hofmannsthal, Arthur Schnitzler: \emph{Briefwechsel}. Hg. Therese Nickl und Heinrich Schnitzler. Frankfurt am Main: \emph{S. Fischer} 1964, S. 30.} }\toendnotes[C]{\smallbreak}\pstart
           \noindent{}{\pb}Lieber Hugo, morgen So{\geminationn}tag bin ich Nachmittags in einem
                  \label{K_L00080-1v}\edtext{Concert}{\lemma{\textnormal{\emph{Concert}}}\Cendnote{\textnormal{Nur ein Konzert an einem Sonntagnachmittag
                     lässt sich nachweisen, das 
                     Werke von
                     Anton Rückauf\pwindex{Rueckauf, Anton 13.03.1855 – 19.09.1903@\textsc{Rückauf, Anton} (13.03.1855 – 19.09.1903), \emph{Komponist, Pianist}|pwk} auf dem Programm hatte. Es fand am 13. 3. 1892 statt. In Schnitzlers\pwindex{Schnitzler, Arthur 15.05.1862 – 21.10.1931@\textsc{Schnitzler, Arthur} (15.05.1862 – 21.10.1931), \emph{Schriftsteller, Mediziner}|pwk} Aufzeichnungen gibt es keinen
                  Hinweis auf eine Teilnahme; für den Tag ist nur ein Besuch in der \emph{Schauspielschule Otto}\orgindex{Schauspielschule Otto@Schauspielschule Otto|pwk} vermerkt (\emph{Cambridge University Library} A 179a).}}}\label{K_L00080-1h}, wo
                  Rückauf\pwindex{Rueckauf, Anton 13.03.1855 – 19.09.1903@\textsc{Rückauf, Anton} (13.03.1855 – 19.09.1903), \emph{Komponist, Pianist}|pw} (mein einſtiger Lehrer, der mich
               ſehr intereſſirt) aufgeführt wird. Alſo nicht {\pb}zu Hauſe.
                  Ko{\geminationm}en Sie möglichſt bald, damit wir noch einen Abend
               dieſer Woche verabreden können.\pend
           \pstart
           Herzlichst{\\[\baselineskip]}Ihr\spacefill\mbox{Arth Sch}\pend
           \leftskip=0em{}
         
         \endnumbering\mylabel{h}\end{ledgroupsized}  \newcommand{\dateiname}{L00080}\newcommand{\titel}{Arthur Schnitzler an Hugo von Hofmannsthal, [12. 3. 1892]}\newcommand{\editorInnen}{Martin Anton Müller und Gerd-Hermann Susen}%% latex-leseansicht-abspann.tex
%% Abspann für die Leseansicht.
%% Der Schalter \ifkorrekturansicht ist bereits durch den Vorspann gesetzt.

%% latex-abspann.tex
%% Gemeinsamer Abspann für Korrekturansicht und Leseansicht.
%% Setzt den Schalter \ifkorrekturansicht voraus (gesetzt in den
%% einbindenden Dateien latex-korrekturansicht-abspann.tex bzw.
%% latex-leseansicht-abspann.tex).
%% ---------------------------------------------------------------

\normalsize

% Das esempio-Environment wird nur in der Leseansicht benötigt
\ifkorrekturansicht\else
\newenvironment{esempio}[3]%
{
    \vspace{1.5ex}
    \rlap{\underline{#1}}
    \par
    \setlength{\parindent}{0cm}
    \nopagebreak
    \leftskip=#2cm
    \rightskip=#3cm
}
{
    \par
}
\fi

\doendnotes{C}
\bigskip
\vfill

\clearpage

\footnotesize

\ifkorrekturansicht
  \lohead{\textsc{register}}
\fi

% theindex-Environment neu definieren ohne reledmac
\makeatletter
\renewenvironment{theindex}{%
  \ifkorrekturansicht
    \section*{\indexname}%
  \else
    \subsubsection*{Index der erwähnten Entitäten}%
  \fi
  \setlength{\parindent}{0pt}%
  \setlength{\parskip}{0pt plus 0.3pt}%
  \let\item\@idxitem
}{%
  \ifkorrekturansicht\clearpage\fi
}
\makeatother

\IfFileExists{\jobname-pw.ind}{\input{\jobname-pw.ind}}{}

% Quellenangabe nur in der Leseansicht
\ifkorrekturansicht\else
% Fallback-Definitionen, falls die .tex-Datei \titel etc. nicht gesetzt hat
\providecommand{\titel}{}
\providecommand{\editorInnen}{}
\providecommand{\dateiname}{\jobname}

\vspace{3cm}

\vfill

\footnotesize
\textsc{Quelle}: \titel. Herausgegeben von {\editorInnen}. In: \emph{Arthur Schnitzler: Briefwechsel mit Autorinnen und Autoren}.
 Digitale Edition, https://schnitzler-briefe.acdh.oeaw.ac.at/{\dateiname}.html (Stand \today)
\fi

\end{document}


      