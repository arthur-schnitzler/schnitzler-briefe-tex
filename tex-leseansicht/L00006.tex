%% latex-korrekturansicht-vorspann.tex
%% Vorspann für die Korrekturansicht.
%% Lädt die gemeinsame Datei latex-vorspann.tex mit gesetztem Schalter.

\newif\ifkorrekturansicht
\korrekturansichttrue

\input{../tex-inputs/latex-vorspann}


\section[Arthur Schnitzler an Wilhelm Bölsche, 14. 10. 1890]{L00006 Arthur Schnitzler an Wilhelm Bölsche, 14. 10. 1890}
\nopagebreak\mylabel{L00006v}
\rehead{ }\normalsize\beginnumbering\briefempfaengerindex{Boelsche, Wilhelm@\textsc{Bölsche, Wilhelm}!zzzSchnitzler, Arthur@\emph{von Arthur Schnitzler}!1890-10-141@{14. 10. 1890}|(be}
\toendnotes[C]{\smallbreak\pagebreak[2]}\Standort{Wrocław, Biblioteka Uniwersytecka, Böl.Pis 1759.}
\physDesc{Brief, 1 Blatt, 2 Seiten, 574 Zeichen
\newline{}Handschrift: schwarze Tinte, deutsche Kurrent}
\buchAbdrucke{\weitereDrucke{1) \emph{Germanica Wratislaviensia} (1987) Nr. 77, S. 458.} \weitereDrucke{2) Wilhelm Bölsche: \emph{Briefwechsel. Mit Autoren der Freien Bühne}. Berlin: \emph{Weidler} 2010, S. 668.} }\toendnotes[C]{\smallbreak}
\pstart{}{\pb}Sehr geehrter Herr Redakteur!\pend\vspace{0.5em}
\pstart
           Ihrer freundlichen Aufforderung gemäß, die ich mir erlaubt habe, nicht als einfache
               Höflichkeitsform zu betrachten, ſende ich Ihnen hier etwas andres – nur ein Gedicht\pwindex{Morgenandacht@\emph{Morgenandacht}|pwv}, wie Sie ſehen, von dem
               ich aber vielleicht annehmen kann, daſs es nicht ganz aus dem Stil Ihres Blattes\pwindex{Freie Buehne fuer modernes Leben@\emph{Freie Bühne für modernes Leben}|pwv} fällt. Wollen Sie die
               große Liebenswürdigkeit haben (bei Gedichten iſt das wirklich eine große
               Liebenswürdigkeit) mir {\pb}die »\label{K_L00006-1v}\edtext{Morgenandacht\pwindex{Morgenandacht@\emph{Morgenandacht}|pw}}{\lemma{\textnormal{\emph{Morgenandacht}}}\Cendnote{\textnormal{Nach der Ablehnung durch Bölsche am
                     25. 10. 1890{ }sandte Schnitzler das Gedicht\pwindex{Morgenandacht@\emph{Morgenandacht}|pwkv} umgehend an Michael Georg Conrad\pwindex{Conrad, Michael Georg 05.04.1846 – 20.12.1927@\textsc{Conrad, Michael Georg} (05.04.1846 – 20.12.1927), \emph{Schriftsteller/Schriftstellerin, Kritiker/Kritikerin}|pwk}; dieser druckte es in der \emph{Gesellschaft}\pwindex{Gesellschaft. Monatsschrift fuer Litteratur, Kunst und Sozialpolitik@\emph{Die Gesellschaft. Monatsschrift für Litteratur, Kunst und Sozialpolitik}|pwk} im Februar 1891;
                     vgl. Michael Georg Conrad an Arthur Schnitzler, 14. 11. 1890.
               }}}\label{K_L00006-1}« zurückzuſchicken, wenn Sie ſie nicht brauchen können? –\pend
           
\pstart
           Hochachtungsvoll{\\[\baselineskip]}\spacefill\mbox{Dr. med. Arthur Schnitzler}\pend
           \leftskip=0em{}
\pstart
           \noindent{}\textsc{Wien I. Giselastraße 11\oindex{Ordination Arthur Schnitzler [Boesendorferstrasse 11]@\textbf{Ordination Arthur Schnitzler [Bösendorferstraße 11]}, \emph{Ordination}|pw}.}{\\}\textsc{14. Oktober 1890}.\pend
           \selectlanguage{ngerman}\endnumbering\briefempfaengerindex{Boelsche, Wilhelm@\textsc{Bölsche, Wilhelm}!zzzSchnitzler, Arthur@\emph{von Arthur Schnitzler}!1890-10-141@{14. 10. 1890}|)be}\mylabel{L00006h}  \normalsize

\doendnotes{C}
\bigskip
\vfill

\clearpage

\footnotesize

\lohead{\textsc{register}}

% Definiere theindex-Environment komplett neu ohne reledmac
\makeatletter
\renewenvironment{theindex}{%
  \section*{\indexname}%
  \setlength{\parindent}{0pt}%
  \setlength{\parskip}{0pt plus 0.3pt}%
  \let\item\@idxitem
}{%
  \clearpage
}
\makeatother

\IfFileExists{\jobname-pw.ind}{\input{\jobname-pw.ind}}{}

\end{document}

      