%% latex-leseansicht-vorspann.tex
%% Vorspann für die Leseansicht.
%% Lädt die gemeinsame Datei latex-vorspann.tex mit nicht gesetztem Schalter.

\newif\ifkorrekturansicht
\korrekturansichtfalse

\input{../tex-inputs/latex-vorspann}


         
         \renewcommand{\erwaehntePersonen}{Personen: Wilhelm Bölsche, Michael Georg Conrad}
         \renewcommand{\erwaehnteOrte}{Orte: Berlin, Ordination Dr. Arthur Schnitzler Giselastraße 11, Wien}
         \renewcommand{\erwaehnteWerke}{Werke: Die Gesellschaft. Monatsschrift für Litteratur, Kunst und Sozialpolitik, Freie Bühne für modernes Leben, Morgenandacht}
               \section[Arthur Schnitzler an Wilhelm Bölsche, 14. 10. 1890]{ Arthur Schnitzler an Wilhelm Bölsche, 14. 10. 1890}\nopagebreak\mylabel{v}\rehead{ }\begin{ledgroupsized}[t]{13cm}\normalsize\beginnumbering\briefempfaengerindex{Boelsche, Wilhelm@\textsc{Bölsche, Wilhelm}!zzzSchnitzler, Arthur@\emph{von Arthur Schnitzler}!1890-10-141@{14. 10. 1890}|(be} \toendnotes[C]{\smallbreak\pagebreak[2]} \Standort{Wrocław, Biblioteka Uniwersytecka, Böl.Pis 1759.}
\physDesc{Brief, 1 Blatt, 2 Seiten, 574 Zeichen
\newline{}Handschrift: schwarze Tinte, deutsche Kurrent}\buchAbdrucke{\weitereDrucke{1) Alois Woldan: \emph{Arthur Schnitzler – Briefe an Wilhelm Bölsche.} In: \emph{Germanica Wratislaviensia} (1987) Nr. 77, S. 458.} \weitereDrucke{2) Wilhelm Bölsche: \emph{Briefwechsel. Mit Autoren der Freien Bühne}. Hg. Gerd-Hermann Susen. Berlin: \emph{Weidler} 2010, S. 668 (Werke und Briefe. Wissenschaftliche Ausgabe, Briefe I).} }\toendnotes[C]{\smallbreak}\pstart{}{\pb}Sehr geehrter Herr Redakteur!\pend\pstart
           Ihrer freundlichen Aufforderung gemäß, die ich mir erlaubt habe, nicht als einfache
               Höflichkeitsform zu betrachten, ſende ich Ihnen hier etwas andres – nur ein Gedicht\pwindex{Schnitzler, Arthur 15.05.1862 – 21.10.1931@\textsc{Schnitzler, Arthur} (15.05.1862 – 21.10.1931), \emph{Schriftsteller, Mediziner}!Morgenandacht1. 2. 1891@\strich\emph{Morgenandacht} {[}1. 2. 1891{]}|pwv}, wie Sie ſehen, von dem
               ich aber vielleicht annehmen kann, daſs es nicht ganz aus dem Stil Ihres Blattes\pwindex{Freie Buehne fuer modernes Leben1890 – 1891@\emph{Freie Bühne für modernes Leben} {[}1890 – 1891{]}|pwv} fällt. Wollen Sie die
               große Liebenswürdigkeit haben (bei Gedichten iſt das wirklich eine große
               Liebenswürdigkeit) mir {\pb}die »\label{K_L00006-1v}\edtext{Morgenandacht\pwindex{Schnitzler, Arthur 15.05.1862 – 21.10.1931@\textsc{Schnitzler, Arthur} (15.05.1862 – 21.10.1931), \emph{Schriftsteller, Mediziner}!Morgenandacht1. 2. 1891@\strich\emph{Morgenandacht} {[}1. 2. 1891{]}|pw}}{\lemma{\textnormal{\emph{Morgenandacht}}}\Cendnote{\textnormal{Nach der Ablehnung durch Bölsche am
                     25. 10. 1890{ }sandte Schnitzler das Gedicht\pwindex{Schnitzler, Arthur 15.05.1862 – 21.10.1931@\textsc{Schnitzler, Arthur} (15.05.1862 – 21.10.1931), \emph{Schriftsteller, Mediziner}!Morgenandacht1. 2. 1891@\strich\emph{Morgenandacht} {[}1. 2. 1891{]}|pwkv} umgehend an Michael Georg Conrad\pwindex{Conrad, Michael Georg 05.04.1846 – 20.12.1927@\textsc{Conrad, Michael Georg} (05.04.1846 – 20.12.1927), \emph{Schriftsteller, Kritiker}|pwk}; dieser druckte es in der \emph{Gesellschaft}\pwindex{Gesellschaft. Monatsschrift fuer Litteratur, Kunst und Sozialpolitik1885 – 1902@\emph{Die Gesellschaft. Monatsschrift für Litteratur, Kunst und Sozialpolitik} {[}1885 – 1902{]}|pwk} im Februar 1891;
                     vgl. Michael Georg Conrad an Arthur Schnitzler, 14. 11. 1890.
               }}}\label{K_L00006-1h}« zurückzuſchicken, wenn Sie ſie nicht brauchen können? –\pend
           \pstart
           Hochachtungsvoll{\\[\baselineskip]}\spacefill\mbox{Dr. med. Arthur Schnitzler}\pend
           \leftskip=0em{}\pstart
           \noindent{}\textsc{Wien I. Giselastraße 11\oindex{Ordination Dr. Arthur Schnitzler Giselastrasse 11@\textbf{Ordination Dr. Arthur Schnitzler Giselastraße 11}|pw}.}{\\}\textsc{14. Oktober 1890}.\pend
           
         
         \endnumbering\mylabel{h}\end{ledgroupsized}  \newcommand{\dateiname}{L00006}\newcommand{\titel}{Arthur Schnitzler an Wilhelm Bölsche, 14. 10. 1890}\newcommand{\editorInnen}{Martin Anton Müller und Gerd-Hermann Susen}%% latex-leseansicht-abspann.tex
%% Abspann für die Leseansicht.
%% Der Schalter \ifkorrekturansicht ist bereits durch den Vorspann gesetzt.

%% latex-abspann.tex
%% Gemeinsamer Abspann für Korrekturansicht und Leseansicht.
%% Setzt den Schalter \ifkorrekturansicht voraus (gesetzt in den
%% einbindenden Dateien latex-korrekturansicht-abspann.tex bzw.
%% latex-leseansicht-abspann.tex).
%% ---------------------------------------------------------------

\normalsize

% Das esempio-Environment wird nur in der Leseansicht benötigt
\ifkorrekturansicht\else
\newenvironment{esempio}[3]%
{
    \vspace{1.5ex}
    \rlap{\underline{#1}}
    \par
    \setlength{\parindent}{0cm}
    \nopagebreak
    \leftskip=#2cm
    \rightskip=#3cm
}
{
    \par
}
\fi

\doendnotes{C}
\bigskip
\vfill

\clearpage

\footnotesize

\ifkorrekturansicht
  \lohead{\textsc{register}}
\fi

% theindex-Environment neu definieren ohne reledmac
\makeatletter
\renewenvironment{theindex}{%
  \ifkorrekturansicht
    \section*{\indexname}%
  \else
    \subsubsection*{Index der erwähnten Entitäten}%
  \fi
  \setlength{\parindent}{0pt}%
  \setlength{\parskip}{0pt plus 0.3pt}%
  \let\item\@idxitem
}{%
  \ifkorrekturansicht\clearpage\fi
}
\makeatother

\IfFileExists{\jobname-pw.ind}{\input{\jobname-pw.ind}}{}

% Quellenangabe nur in der Leseansicht
\ifkorrekturansicht\else
% Fallback-Definitionen, falls die .tex-Datei \titel etc. nicht gesetzt hat
\providecommand{\titel}{}
\providecommand{\editorInnen}{}
\providecommand{\dateiname}{\jobname}

\vspace{3cm}

\vfill

\footnotesize
\textsc{Quelle}: \titel. Herausgegeben von {\editorInnen}. In: \emph{Arthur Schnitzler: Briefwechsel mit Autorinnen und Autoren}.
 Digitale Edition, https://schnitzler-briefe.acdh.oeaw.ac.at/{\dateiname}.html (Stand \today)
\fi

\end{document}


      