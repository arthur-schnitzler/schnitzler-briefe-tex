%% latex-leseansicht-vorspann.tex
%% Vorspann für die Leseansicht.
%% Lädt die gemeinsame Datei latex-vorspann.tex mit nicht gesetztem Schalter.

\newif\ifkorrekturansicht
\korrekturansichtfalse

\input{../tex-inputs/latex-vorspann}


\section[Hugo von Hofmannsthal an Arthur Schnitzler, 2. 10. [1899]]{L00986 Hugo von Hofmannsthal an Arthur Schnitzler, 2. 10. [1899]}
\nopagebreak\mylabel{L00986v}
\rehead{ }\normalsize\beginnumbering\briefempfaengerindex{Schnitzler, Arthur@\textsc{Schnitzler, Arthur}!zzzHofmannsthal, Hugo von@\emph{von Hugo von Hofmannsthal}!1899-10-021@{2. 10. [1899]}|(be}
\toendnotes[C]{\smallbreak\pagebreak[2]}
\correspDesc{Versand  durch Hugo von Hofmannsthal am 2. 10. [1899] in Venedig
\newline{}Erhalt  durch Arthur Schnitzler im Zeitraum [3. 10. 1899
                  – 7. 10. 1899?] in Wien}\toendnotes[C]{\smallbreak}
\Standort{CUL, Schnitzler, B 43.}
\physDesc{Brief, 1 Blatt, 4 Seiten, 1535 Zeichen
\newline{}Handschrift: schwarze Tinte, deutsche Kurrent
\newline{}Ordnung: mit Bleistift von unbekannter Hand nummeriert:
                                    »236« }
\buchAbdrucke{\weitereDrucke{Hugo von Hofmannsthal, Arthur Schnitzler: \emph{Briefwechsel}. Herausgegeben von Therese Nickl und Heinrich Schnitzler. Frankfurt am Main: \emph{S. Fischer} 1964, S. 131–132.} }\toendnotes[C]{\smallbreak}
\pstart
           \raggedleft{}{\pb}\textcolor{gray}{\textbf{Venice\oindex{Venedig@\textbf{Venedig}|pw}}}\pend
           
\pstart
           \centering{}\textcolor{gray}{\textbf{Grand Hôtel Britannia\oindex{Grand Hotel Britannia@\textbf{Grand Hotel Britannia}, \emph{Hotel}|pw}}}\pend
           
\pstart
           \textcolor{gray}{\textbf{Charles Walther}}\hfill \textcolor{gray}{\textbf{Electric light and steamheat in all rooms}}\pend
           
\pstart
           \textcolor{gray}{\textbf{Propr.}}\hfill \textcolor{gray}{\textbf{Hydraulic Lifts}}\pend
           
\pstart
           \centering{}\textcolor{gray}{\textbf{Mêmes Maisons}}\pend
           
\pstart
           \textcolor{gray}{\textbf{Hôtel Victoria\oindex{Hotel Victoria [Bozen]@\textbf{Hotel Victoria [Bozen]}, \emph{Hotel}|pw}}}\hfill \textcolor{gray}{\textbf{Hôtel de la Ville\oindex{Hotel de la Ville [Genua]@\textbf{Hotel de la Ville [Genua]}, \emph{Hotel}|pw}}}\pend
           
\pstart
           \textcolor{gray}{\textbf{Bozen\oindex{Bozen@\textbf{Bozen}, \emph{Hauptstadt}|pw} (Tyrol\oindex{Tirol@\textbf{Tirol}, \emph{Land}|pw})}}\hfill \textcolor{gray}{\textbf{Genoa\oindex{Genua@\textbf{Genua}|pw} – Gênes\oindex{Genua@\textbf{Genua}|pw} – Genúa\oindex{Genua@\textbf{Genua}|pw}}}\pend
           
\pstart
           \raggedleft{}\textcolor{gray}{\textbf{Venice}}, den 2\textsuperscript{ten} X.\pend
           
\pstart{}mein lieber Arthur\pend\vspace{0.5em}
\pstart
           was Sie mir{ }ſchreiben, iſt{ }ſo wahr: für die Momente dankbar{ }ſein, in denen man eine
               gewiſſe innere Fülle empfindet. Daſs aber das alles unter{ }ſo furchtbar dunklen
               Geſetzen{ }ſteht und daſs die Starrheit manchmal alles ergreifen {\pb}kann,{ }ſogar die Empfindung für die
               Exiſtenz aller andern Menschen!\pend
           
\pstart
           Mit meinem Stück\pwindex{Hofmannsthal, Hugo von 1.\,2.\,1874 Wien – 15.\,7.\,1929 Rodaun@\textsc{Hofmannsthal, Hugo von} (1.\,2.\,1874 Wien – 15.\,7.\,1929 Rodaun), \emph{Schriftsteller}!Bergwerk zu Falun@\strich\emph{Das Bergwerk zu Falun}|pwv} geht es{ }ſonderbar. Ich hab in Vahrn\oindex{Vahrn@\textbf{Vahrn}, \emph{Hauptstadt}|pw} nochmals einen ganz
               unbrauchbaren 3\textsuperscript{ten} Act gemacht, recht verſchieden von
               dem, den Sie in Iſchl\oindex{Bad Ischl@\textbf{Bad Ischl}|pw} geſehen haben, und doch
               falsch. Eine{ }ſchlechte Art, die Menſchen und ihr Schickſal anzuſehen. Der Grundfehler
               war, wie ich jetzt weiß, schon im \substVorne{}\textsuperscript{erſten}\substDazwischen{}zweiten\substHinten{}{ }Act\pwindex{Hofmannsthal, Hugo von 1.\,2.\,1874 Wien – 15.\,7.\,1929 Rodaun@\textsc{Hofmannsthal, Hugo von} (1.\,2.\,1874 Wien – 15.\,7.\,1929 Rodaun), \emph{Schriftsteller}!Bergwerk zu Falun@\strich\emph{Das Bergwerk zu Falun}|pwv} gelegen. Bin dann hier her
               gefahren. Wollte ganz aufhören, mich abſolut von dem Stoff losmachen. Das war ich
               aber auch nicht im Stande. Habe wieder den 2\textsuperscript{ten}{ }Act\pwindex{Hofmannsthal, Hugo von 1.\,2.\,1874 Wien – 15.\,7.\,1929 Rodaun@\textsc{Hofmannsthal, Hugo von} (1.\,2.\,1874 Wien – 15.\,7.\,1929 Rodaun), \emph{Schriftsteller}!Bergwerk zu Falun@\strich\emph{Das Bergwerk zu Falun}|pwv} vorgeno{\geminationm}en. In dieſer weichen helleren Luft hier {\pb}nimmt alles weichere Formen an;
               ich arbeite wieder mit Freude, die Bekanntſchaft mit den umgeſchmolzenen Figuren
               kommt mir zu Hilfe und ich hoffe hier{ }ſehr raſch weit zu kommen.\pend
           
\pstart
           Brahm\pwindex{Brahm, Otto 5.\,2.\,1856 Hamburg – 28.\,11.\,1912 Berlin@\textsc{Brahm, Otto} (5.\,2.\,1856 Hamburg – 28.\,11.\,1912 Berlin), \emph{Theaterleiter, Regisseur}|pw} will ich in dieſen Tagen{ }ſchreiben. Es
               liegt mir aus weitläufigen Gründen{ }ſehr viel daran, daſs das Stück\pwindex{Hofmannsthal, Hugo von 1.\,2.\,1874 Wien – 15.\,7.\,1929 Rodaun@\textsc{Hofmannsthal, Hugo von} (1.\,2.\,1874 Wien – 15.\,7.\,1929 Rodaun), \emph{Schriftsteller}!Bergwerk zu Falun@\strich\emph{Das Bergwerk zu Falun}|pwv} wenigſtens in einem der Theater noch
               in dieſem Spieljahr drankommt.\pend
           
\pstart
           Richards\pwindex{Beer-Hofmann, Richard 11.\,7.\,1866 Wien – 26.\,9.\,1945 New York City@\textsc{Beer-Hofmann, Richard} (11.\,7.\,1866 Wien – 26.\,9.\,1945 New York City), \emph{Schriftsteller}|pw}{ }Stück\pwindex{Beer-Hofmann, Richard 11.\,7.\,1866 Wien – 26.\,9.\,1945 New York City@\textsc{Beer-Hofmann, Richard} (11.\,7.\,1866 Wien – 26.\,9.\,1945 New York City), \emph{Schriftsteller}!Graf von Charolais. Ein Trauerspiel@\strich\emph{Der Graf von Charolais. Ein Trauerspiel}|pwv} iſt in der Anlage wunderſchön und er arbeitet gar
               nicht langſam, etwa 30–40 Verſe {\pb}im Tag.\hspace*{1.5em}Wie froh bin ich,{ }ſolche Menſchen zu haben
               wie Sie und Richard\pwindex{Beer-Hofmann, Richard 11.\,7.\,1866 Wien – 26.\,9.\,1945 New York City@\textsc{Beer-Hofmann, Richard} (11.\,7.\,1866 Wien – 26.\,9.\,1945 New York City), \emph{Schriftsteller}|pw}. Daſs man trotzdem{ }ſo \strikeout{vielfach} oft{ }ſo traurig, oed und{ }ſtarr{ }ſein kann.\pend
           
\pstart
           Ich bin vielleicht noch 14 Tage hier. Ko{\geminationm}en Sie nicht
               vorbei und leſen mir zur Ermuthigung was vor?\pend
           
\pstart
           Von Herzen Ihr{\\[\baselineskip]}\spacefill\mbox{Hugo.}\pend
           \leftskip=0em{}\selectlanguage{ngerman}\endnumbering\briefempfaengerindex{Schnitzler, Arthur@\textsc{Schnitzler, Arthur}!zzzHofmannsthal, Hugo von@\emph{von Hugo von Hofmannsthal}!1899-10-021@{2. 10. [1899]}|)be}\mylabel{L00986h}  \newcommand{\dateiname}{L00986}\newcommand{\titel}{Hugo von Hofmannsthal an Arthur Schnitzler, 2. 10. [1899]}\newcommand{\editorInnen}{Martin Anton Müller und Gerd-Hermann Susen}%% latex-leseansicht-abspann.tex
%% Abspann für die Leseansicht.
%% Der Schalter \ifkorrekturansicht ist bereits durch den Vorspann gesetzt.

%% latex-abspann.tex
%% Gemeinsamer Abspann für Korrekturansicht und Leseansicht.
%% Setzt den Schalter \ifkorrekturansicht voraus (gesetzt in den
%% einbindenden Dateien latex-korrekturansicht-abspann.tex bzw.
%% latex-leseansicht-abspann.tex).
%% ---------------------------------------------------------------

\normalsize

% Das esempio-Environment wird nur in der Leseansicht benötigt
\ifkorrekturansicht\else
\newenvironment{esempio}[3]%
{
    \vspace{1.5ex}
    \rlap{\underline{#1}}
    \par
    \setlength{\parindent}{0cm}
    \nopagebreak
    \leftskip=#2cm
    \rightskip=#3cm
}
{
    \par
}
\fi

\doendnotes{C}
\bigskip
\vfill

\clearpage

\footnotesize

\ifkorrekturansicht
  \lohead{\textsc{register}}
\fi

% theindex-Environment neu definieren ohne reledmac
\makeatletter
\renewenvironment{theindex}{%
  \ifkorrekturansicht
    \section*{\indexname}%
  \else
    \subsubsection*{Index der erwähnten Entitäten}%
  \fi
  \setlength{\parindent}{0pt}%
  \setlength{\parskip}{0pt plus 0.3pt}%
  \let\item\@idxitem
}{%
  \ifkorrekturansicht\clearpage\fi
}
\makeatother

\IfFileExists{\jobname-pw.ind}{\input{\jobname-pw.ind}}{}

% Quellenangabe nur in der Leseansicht
\ifkorrekturansicht\else
% Fallback-Definitionen, falls die .tex-Datei \titel etc. nicht gesetzt hat
\providecommand{\titel}{}
\providecommand{\editorInnen}{}
\providecommand{\dateiname}{\jobname}

\vspace{3cm}

\vfill

\footnotesize
\textsc{Quelle}: \titel. Herausgegeben von {\editorInnen}. In: \emph{Arthur Schnitzler: Briefwechsel mit Autorinnen und Autoren}.
 Digitale Edition, https://schnitzler-briefe.acdh.oeaw.ac.at/{\dateiname}.html (Stand \today)
\fi

\end{document}


