%% latex-leseansicht-vorspann.tex
%% Vorspann für die Leseansicht.
%% Lädt die gemeinsame Datei latex-vorspann.tex mit nicht gesetztem Schalter.

\newif\ifkorrekturansicht
\korrekturansichtfalse

\input{../tex-inputs/latex-vorspann}


         
         \renewcommand{\erwaehntePersonen}{Personen: Felix Salten, Michael Emil Salzmann, Marie Salzmann, Olga Schnitzler}
         \renewcommand{\erwaehnteOrte}{Orte: Edmund-Weiß-Gasse 7, Seis am Schlern, Wien}
         \renewcommand{\erwaehnteWerke}{}
               \section[ Arthur Schnitzler an Felix Salten, 29. 6. 1908]{ Arthur Schnitzler an Felix Salten, 29. 6. 1908}\nopagebreak\mylabel{v}\rehead{ }\begin{ledgroupsized}[t]{13cm}\normalsize\beginnumbering\briefempfaengerindex{Salten, Felix@\textsc{Salten, Felix}!zzzSchnitzler, Arthur@\emph{von Arthur Schnitzler}!1908-06-292@{29. 6. 1908}|(be} \toendnotes[C]{\smallbreak\pagebreak[2]} \Standort{Wienbibliothek im Rathaus, ZPH 1681, 2.1.516.}
\physDesc{Karte, 528 Zeichen
\newline{}Handschrift: schwarze Tinte, deutsche Kurrent
\newline{}Ordnung: mit Bleistift von unbekannter Hand nummeriert: »5« }\toendnotes[C]{\smallbreak}\pstart
           \noindent{}{\pb}\textcolor{gray}{\textbf{Dr. Arthur Schnitzler}}\hfill am 29. Juni 908\pend
           \pstart
           \textcolor{gray}{\textbf{Wien XVIII. Spoettelgasse 7\oindex{Edmund-Weiss-Gasse 7@\textbf{Edmund-Weiß-Gasse 7}|pw}.}}\hfill \textsc{Seis am Schlern\oindex{Seis am Schlern@\textbf{Seis am Schlern}|pw}}.\pend
           \pstart
           lieber, ich leſe eben, daſs Ihr \label{K_L03014-1v}\edtext{Bruder\pwindex{Salzmann, Michael Emil 1858-01-19 – 1908-06-26@\textsc{Salzmann, Michael Emil} (1858-01-19 – 1908-06-26), \emph{Versicherungsbeamter}|pwv} geſtorben}{\lemma{\textnormal{\emph{Bruder geſtorben}}}\Cendnote{\textnormal{(Michael) Emil Salzmann\pwindex{Salzmann, Michael Emil 1858-01-19 – 1908-06-26@\textsc{Salzmann, Michael Emil} (1858-01-19 – 1908-06-26), \emph{Versicherungsbeamter}|pwk} war am 26. 6. 1908 gestorben. Er war
                  das älteste Geschwister und die wichtigste familiäre Bezugsperson Saltens\pwindex{Salten, Felix 06.09.1869 – 08.10.1945@\textsc{Salten, Felix} (06.09.1869 – 08.10.1945), \emph{Schriftsteller, Journalist, Chefredakteur}|pwk}. Er hatte zeitlebens unverheiratet bei der
                     Mutter\pwindex{Salzmann, Marie 1833-10-27 – 1909-12-01@\textsc{Salzmann, Marie} (1833-10-27 – 1909-12-01)|pwkv} gelebt.}}}\label{K_L03014-1h} iſt,
               und bin um ſo tiefer ergriffen, als ich nicht wußte, daſs ſein Befinden ſich in der
               letzten Zeit verſchli{\geminationm}ert hatte. Glauben Sie mir, daſs
               ich an Ihrem Schmerze den herzlichſten Antheil nehme und ſagen Sie es auch den
               Ihrigen, vor allem Ihrer Mutter\pwindex{Salzmann, Marie 1833-10-27 – 1909-12-01@\textsc{Salzmann, Marie} (1833-10-27 – 1909-12-01)|pwv}, wie ſehr ich das {\pb}frühe Ende
               dieſes liebenswerthen Menſchen\pwindex{Salzmann, Michael Emil 1858-01-19 – 1908-06-26@\textsc{Salzmann, Michael Emil} (1858-01-19 – 1908-06-26), \emph{Versicherungsbeamter}|pwv} beklage. Auch Olga\pwindex{Schnitzler, Olga 17.01.1882 – 13.01.1970@\textsc{Schnitzler, Olga} (17.01.1882 – 13.01.1970), \emph{Schauspielerin, Sängerin}|pw} bittet Sie
               ihres Mitgefühls verſichert zu ſein. Wir\pwindex{Schnitzler, Olga 17.01.1882 – 13.01.1970@\textsc{Schnitzler, Olga} (17.01.1882 – 13.01.1970), \emph{Schauspielerin, Sängerin}|pwv} grüßen vielmals und hoffen baldmöglichſt
                  gutes von Ihnen zu hören.\pend
           \pstart
           Ihr {\\[\baselineskip]}\spacefill\mbox{Arthur}\pend
           \leftskip=0em{}
         
         \endnumbering\mylabel{h}\end{ledgroupsized}  \newcommand{\dateiname}{L03014}\newcommand{\titel}{Arthur Schnitzler an Felix Salten, 29. 6. 1908}\newcommand{\editorInnen}{Martin Anton Müller und Laura Untner}%% latex-leseansicht-abspann.tex
%% Abspann für die Leseansicht.
%% Der Schalter \ifkorrekturansicht ist bereits durch den Vorspann gesetzt.

%% latex-abspann.tex
%% Gemeinsamer Abspann für Korrekturansicht und Leseansicht.
%% Setzt den Schalter \ifkorrekturansicht voraus (gesetzt in den
%% einbindenden Dateien latex-korrekturansicht-abspann.tex bzw.
%% latex-leseansicht-abspann.tex).
%% ---------------------------------------------------------------

\normalsize

% Das esempio-Environment wird nur in der Leseansicht benötigt
\ifkorrekturansicht\else
\newenvironment{esempio}[3]%
{
    \vspace{1.5ex}
    \rlap{\underline{#1}}
    \par
    \setlength{\parindent}{0cm}
    \nopagebreak
    \leftskip=#2cm
    \rightskip=#3cm
}
{
    \par
}
\fi

\doendnotes{C}
\bigskip
\vfill

\clearpage

\footnotesize

\ifkorrekturansicht
  \lohead{\textsc{register}}
\fi

% theindex-Environment neu definieren ohne reledmac
\makeatletter
\renewenvironment{theindex}{%
  \ifkorrekturansicht
    \section*{\indexname}%
  \else
    \subsubsection*{Index der erwähnten Entitäten}%
  \fi
  \setlength{\parindent}{0pt}%
  \setlength{\parskip}{0pt plus 0.3pt}%
  \let\item\@idxitem
}{%
  \ifkorrekturansicht\clearpage\fi
}
\makeatother

\IfFileExists{\jobname-pw.ind}{\input{\jobname-pw.ind}}{}

% Quellenangabe nur in der Leseansicht
\ifkorrekturansicht\else
% Fallback-Definitionen, falls die .tex-Datei \titel etc. nicht gesetzt hat
\providecommand{\titel}{}
\providecommand{\editorInnen}{}
\providecommand{\dateiname}{\jobname}

\vspace{3cm}

\vfill

\footnotesize
\textsc{Quelle}: \titel. Herausgegeben von {\editorInnen}. In: \emph{Arthur Schnitzler: Briefwechsel mit Autorinnen und Autoren}.
 Digitale Edition, https://schnitzler-briefe.acdh.oeaw.ac.at/{\dateiname}.html (Stand \today)
\fi

\end{document}


      