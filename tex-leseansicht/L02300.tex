%% latex-korrekturansicht-vorspann.tex
%% Vorspann für die Korrekturansicht.
%% Lädt die gemeinsame Datei latex-vorspann.tex mit gesetztem Schalter.

\newif\ifkorrekturansicht
\korrekturansichttrue

\input{../tex-inputs/latex-vorspann}


\section[Hermann Bahr an Arthur Schnitzler, 25. 8. 1918]{L02300 Hermann Bahr an Arthur Schnitzler, 25. 8. 1918}
\nopagebreak\mylabel{L02300v}
\rehead{ }\normalsize\beginnumbering\briefempfaengerindex{Schnitzler, Arthur@\textsc{Schnitzler, Arthur}!zzzBahr, Hermann@\emph{von Hermann Bahr}!1918-08-251@{25. 8. 1918}|(be}
\toendnotes[C]{\smallbreak\pagebreak[2]}\Standort{CUL, Schnitzler, B 5b.}
\physDesc{Postkarte, 479 Zeichen
\newline{}Handschrift: schwarze Tinte, deutsche Kurrent
\newline{}Versand: Stempel: »\nobreak{}\oindex{Salzburg@\textbf{Salzburg}, \emph{A.ADM2}|pwk}Salzburg 2, 25. VIII. {[}1{]}8, 2\nobreak{}«.  
\newline{}Schnitzler: mit Bleistift Vermerk »\uline{A}«, vermutlich für »Abzuschreiben«/»Abschrift« 
\newline{}Ordnung: mit Bleistift von unbekannter Hand nummeriert:
                                    »182« }
\buchAbdrucke{\weitereDrucke{Hermann Bahr, Arthur Schnitzler: \emph{Briefwechsel, Aufzeichnungen, Dokumente (1891–1931)}. Göttingen: \emph{Wallstein} 2018, S. 512.} }\toendnotes[C]{\smallbreak}\pstart{}{\pb}Abs. Hermann Bahr\pend{}{\bigskip}\pstart{}Herrn\pend{}\pstart{}D\textsuperscript{r} Arthur Schnitzler\pend{}\pstart{}Wien XVIII\oindex{XVIII., Waehring@\textbf{XVIII., Währing}, \emph{A.ADM3}|pw}\pend{}\pstart{}Sternwarteſtr 71\oindex{Sternwartestrasse 71@\textbf{Sternwartestraße 71}, \emph{Wohngebäude (K.WHS)}|pw}\pend{}{\bigskip}\vspace{1em}
\pstart
           \raggedleft{}{\pb}25. 8. 18\pend
           \vspace{0.5em}
\pstart
           Herzlichſten Dank, lieber Arthur, für Deinen lieben Brief – Frau Kainz\pwindex{Kainz, Margarethe 13.12.1858 – 12.02.1950@\textsc{Kainz, Margarethe} (13.12.1858 – 12.02.1950), \emph{Schauspieler/Schauspielerin}|pw} verhieß uns\pwindex{Bahr-Mildenburg, Anna 29.11.1872 – 27.01.1947@\textsc{Bahr-Mildenburg, Anna} (29.11.1872 – 27.01.1947), \emph{Sänger/Sängerin}|pwv} immer Euren\pwindex{Schnitzler, Olga 17.01.1882 – 13.01.1970@\textsc{Schnitzler, Olga} (17.01.1882 – 13.01.1970), \emph{Schauspieler/Schauspielerin, Sänger/Sängerin}|pwv} erſehnten Beſuch und wir\pwindex{Bahr-Mildenburg, Anna 29.11.1872 – 27.01.1947@\textsc{Bahr-Mildenburg, Anna} (29.11.1872 – 27.01.1947), \emph{Sänger/Sängerin}|pwv} warteten den ganzen Sommer auf Euch\pwindex{Schnitzler, Olga 17.01.1882 – 13.01.1970@\textsc{Schnitzler, Olga} (17.01.1882 – 13.01.1970), \emph{Schauspieler/Schauspielerin, Sänger/Sängerin}|pwv}, leider vergeblich. So
               bald ich in Wien\oindex{Wien@\textbf{Wien}, \emph{A.ADM2}|pw} bin, melde ich mich bei Dir, um
               gleich in den erſten Tagen einmal zu Dir zu kommen. Bis dahin (wo wir dann auch über
               Deinen Muſiker\pwindex{Scholz, Arthur Johannes 16.11.1883 – 03.04.1945@\textsc{Scholz, Arthur Johannes} (16.11.1883 – 03.04.1945), \emph{Komponist/Komponistin, Musikpädagoge/Musikpädagogin, Dirigent/Dirigentin}|pwv}{ }ſprechen) mit den herzlichſten Grüßen von uns Beiden\pwindex{Bahr-Mildenburg, Anna 29.11.1872 – 27.01.1947@\textsc{Bahr-Mildenburg, Anna} (29.11.1872 – 27.01.1947), \emph{Sänger/Sängerin}|pwv} an Dich und Deine
               liebe Frau\pwindex{Schnitzler, Olga 17.01.1882 – 13.01.1970@\textsc{Schnitzler, Olga} (17.01.1882 – 13.01.1970), \emph{Schauspieler/Schauspielerin, Sänger/Sängerin}|pwv}\pend
           
\pstart
           Dein{\\[\baselineskip]}alter{\\[\baselineskip]}\spacefill\mbox{H.}\pend
           \leftskip=0em{}\selectlanguage{ngerman}\endnumbering\briefempfaengerindex{Schnitzler, Arthur@\textsc{Schnitzler, Arthur}!zzzBahr, Hermann@\emph{von Hermann Bahr}!1918-08-251@{25. 8. 1918}|)be}\mylabel{L02300h}  \normalsize

\doendnotes{C}
\bigskip
\vfill

\clearpage

\footnotesize

\lohead{\textsc{register}}

% Definiere theindex-Environment komplett neu ohne reledmac
\makeatletter
\renewenvironment{theindex}{%
  \section*{\indexname}%
  \setlength{\parindent}{0pt}%
  \setlength{\parskip}{0pt plus 0.3pt}%
  \let\item\@idxitem
}{%
  \clearpage
}
\makeatother

\IfFileExists{\jobname-pw.ind}{\input{\jobname-pw.ind}}{}

\end{document}

      