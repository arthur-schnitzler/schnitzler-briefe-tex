%% latex-korrekturansicht-vorspann.tex
%% Vorspann für die Korrekturansicht.
%% Lädt die gemeinsame Datei latex-vorspann.tex mit gesetztem Schalter.

\newif\ifkorrekturansicht
\korrekturansichttrue

\input{../tex-inputs/latex-vorspann}


\section[Felix Salten an Arthur Schnitzler, {[}8. 10. 1892{]}]{L03115 Felix Salten an Arthur Schnitzler, {[}8. 10. 1892{]}}
\nopagebreak\mylabel{L03115v}
\rehead{ }\normalsize\beginnumbering\briefempfaengerindex{Schnitzler, Arthur@\textsc{Schnitzler, Arthur}!zzzSalten, Felix@\emph{von Felix Salten}!1892-10-082@{{[}8. 10. 1892{]}}|(be}
\toendnotes[C]{\smallbreak\pagebreak[2]}\Standort{CUL, Schnitzler, B 89, A 1.}
\physDesc{Brief, 1 Blatt, 1 Seite, 200 Zeichen
\newline{}Handschrift: schwarze Tinte, lateinische Kurrent
\newline{}Schnitzler: mit Bleistift datiert: »8/10 9\textcolor{gray}{2}« 
\newline{}Ordnung: mit Bleistift von unbekannter Hand nummeriert: »19« }\toendnotes[C]{\smallbreak}
\pstart
           \noindent{}{\pb}lieber Freund! Bitte warten Sie morgen nicht auf mich. Ich bin, wie Sie ja \label{K_L03115-1v}\edtext{neulich}{\lemma{\textnormal{\emph{neulich}}}\Cendnote{\textnormal{womöglich am 7. 10. 1892}}}\label{K_L03115-1} durch Rosner\pwindex{Rosner, Karl Peter 05.02.1873 – 06.05.1951@\textsc{Rosner, Karl Peter} (05.02.1873 – 06.05.1951), \emph{Schriftsteller/Schriftstellerin}|pw} gehört, krank, – erst
               seit heute außer Bett und es geht mir garnicht
               gut.\pend
           
\pstart
           Jedenfalls besten Dank und Gruß {\\[\baselineskip]}Ihr {\\[\baselineskip]}\spacefill\mbox{Salten}\pend
           \leftskip=0em{}\selectlanguage{ngerman}\endnumbering\briefempfaengerindex{Schnitzler, Arthur@\textsc{Schnitzler, Arthur}!zzzSalten, Felix@\emph{von Felix Salten}!1892-10-082@{{[}8. 10. 1892{]}}|)be}\mylabel{L03115h}  \normalsize

\doendnotes{C}
\bigskip
\vfill

\clearpage

\footnotesize

\lohead{\textsc{register}}

% Definiere theindex-Environment komplett neu ohne reledmac
\makeatletter
\renewenvironment{theindex}{%
  \section*{\indexname}%
  \setlength{\parindent}{0pt}%
  \setlength{\parskip}{0pt plus 0.3pt}%
  \let\item\@idxitem
}{%
  \clearpage
}
\makeatother

\IfFileExists{\jobname-pw.ind}{\input{\jobname-pw.ind}}{}

\end{document}

      