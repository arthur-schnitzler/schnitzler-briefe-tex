%% latex-leseansicht-vorspann.tex
%% Vorspann für die Leseansicht.
%% Lädt die gemeinsame Datei latex-vorspann.tex mit nicht gesetztem Schalter.

\newif\ifkorrekturansicht
\korrekturansichtfalse

\input{../tex-inputs/latex-vorspann}


\section[Richard und Paula Beer-Hofmann an Arthur Schnitzler, 11. 7. 1910]{L01944 Richard und Paula Beer-Hofmann an Arthur Schnitzler, 11. 7. 1910}
\nopagebreak\mylabel{L01944v}
\rehead{ }\normalsize\beginnumbering\briefempfaengerindex{Schnitzler, Arthur@\textsc{Schnitzler, Arthur}!zzzBeer-Hofmann, Paula@\emph{von Paula Beer-Hofmann}!1910-07-111@{11. 7. 1910}|(be}\briefempfaengerindex{Schnitzler, Arthur@\textsc{Schnitzler, Arthur}!zzzBeer-Hofmann, Richard@\emph{von Richard Beer-Hofmann}!1910-07-111@{11. 7. 1910}|(be}
\toendnotes[C]{\smallbreak\pagebreak[2]}
\correspDesc{Versand  durch Richard Beer-Hofmann, Paula Beer-Hofmann am 11. 7. 1910 in Salzburg
\newline{}Erhalt  durch Arthur Schnitzler im Zeitraum [12. 7. 1910
                  – 16. 7. 1910?] in Wien}\toendnotes[C]{\smallbreak}
\Standort{CUL, Schnitzler, B 8.}
\physDesc{Bildpostkarte, 87 Zeichen
\newline{}Handschrift Richard Beer-Hofmann: Bleistift, lateinische Kurrent
\newline{}Handschrift Paula Beer-Hofmann: Bleistift
\newline{}Versand: Stempel: »\nobreak{}\oindex{Salzburg@\textbf{Salzburg}, \emph{Verwaltungsgebiet}|pwk}Salzburg 1, 11. VII. 10, 3\nobreak{}«.  
\newline{}Ordnung: mit Bleistift von unbekannter Hand nummeriert:
                                    »233« }\pstart{}{\pb}Herrn\pend{}\pstart{}Arthur Schnitzler\pend{}\pstart{}Wien\oindex{Wien@\textbf{Wien}, \emph{Verwaltungsgebiet}|pw}\pend{}\pstart{}XVIII. Spöttelgasse 7\oindex{Wien@\textbf{Wien}!XVIII., Währing@\textbf{XVIII., Währing}!Edmund-Weiß-Gasse 7@\textbf{Edmund-Weiß-Gasse 7}, \emph{Wohngebäude}|pw}.\pend{}{\bigskip}
\pstart
           \noindent{}\centering{}{\pb}\textcolor{gray}{\textbf{SALZBURG\oindex{Salzburg@\textbf{Salzburg}, \emph{Verwaltungsgebiet}|pw}.}}\pend
           \vspace{1em}
\pstart
           \raggedleft{}{\pb}11/VII. 10\pend
           \vspace{0.5em}
\pstart
           Herzliche Grüsse.\pend
           
\pstart
           \spacefill\mbox{Richard}{\\[\baselineskip]}\spacefill\mbox{{[}hs. Beer-Hofmann:{]} Paula}\pend
           \leftskip=0em{}\selectlanguage{ngerman}\endnumbering\briefempfaengerindex{Schnitzler, Arthur@\textsc{Schnitzler, Arthur}!zzzBeer-Hofmann, Paula@\emph{von Paula Beer-Hofmann}!1910-07-111@{11. 7. 1910}|)be}\briefempfaengerindex{Schnitzler, Arthur@\textsc{Schnitzler, Arthur}!zzzBeer-Hofmann, Richard@\emph{von Richard Beer-Hofmann}!1910-07-111@{11. 7. 1910}|)be}\mylabel{L01944h}  \newcommand{\dateiname}{L01944}\newcommand{\titel}{Richard und Paula Beer-Hofmann an Arthur Schnitzler, 11. 7. 1910}\newcommand{\editorInnen}{Martin Anton Müller und Gerd-Hermann Susen}%% latex-leseansicht-abspann.tex
%% Abspann für die Leseansicht.
%% Der Schalter \ifkorrekturansicht ist bereits durch den Vorspann gesetzt.

%% latex-abspann.tex
%% Gemeinsamer Abspann für Korrekturansicht und Leseansicht.
%% Setzt den Schalter \ifkorrekturansicht voraus (gesetzt in den
%% einbindenden Dateien latex-korrekturansicht-abspann.tex bzw.
%% latex-leseansicht-abspann.tex).
%% ---------------------------------------------------------------

\normalsize

% Das esempio-Environment wird nur in der Leseansicht benötigt
\ifkorrekturansicht\else
\newenvironment{esempio}[3]%
{
    \vspace{1.5ex}
    \rlap{\underline{#1}}
    \par
    \setlength{\parindent}{0cm}
    \nopagebreak
    \leftskip=#2cm
    \rightskip=#3cm
}
{
    \par
}
\fi

\doendnotes{C}
\bigskip
\vfill

\clearpage

\footnotesize

\ifkorrekturansicht
  \lohead{\textsc{register}}
\fi

% theindex-Environment neu definieren ohne reledmac
\makeatletter
\renewenvironment{theindex}{%
  \ifkorrekturansicht
    \section*{\indexname}%
  \else
    \subsubsection*{Index der erwähnten Entitäten}%
  \fi
  \setlength{\parindent}{0pt}%
  \setlength{\parskip}{0pt plus 0.3pt}%
  \let\item\@idxitem
}{%
  \ifkorrekturansicht\clearpage\fi
}
\makeatother

\IfFileExists{\jobname-pw.ind}{\input{\jobname-pw.ind}}{}

% Quellenangabe nur in der Leseansicht
\ifkorrekturansicht\else
% Fallback-Definitionen, falls die .tex-Datei \titel etc. nicht gesetzt hat
\providecommand{\titel}{}
\providecommand{\editorInnen}{}
\providecommand{\dateiname}{\jobname}

\vspace{3cm}

\vfill

\footnotesize
\textsc{Quelle}: \titel. Herausgegeben von {\editorInnen}. In: \emph{Arthur Schnitzler: Briefwechsel mit Autorinnen und Autoren}.
 Digitale Edition, https://schnitzler-briefe.acdh.oeaw.ac.at/{\dateiname}.html (Stand \today)
\fi

\end{document}


