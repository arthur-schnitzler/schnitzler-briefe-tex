%% latex-leseansicht-vorspann.tex
%% Vorspann für die Leseansicht.
%% Lädt die gemeinsame Datei latex-vorspann.tex mit nicht gesetztem Schalter.

\newif\ifkorrekturansicht
\korrekturansichtfalse

\input{../tex-inputs/latex-vorspann}


               \section[Arthur Schnitzler an Georg Brandes, 7. 8. 1896]{ Arthur Schnitzler an Georg Brandes, 7. 8. 1896}\nopagebreak\mylabel{v}\rehead{ }\begin{ledgroupsized}[t]{13cm}\normalsize\beginnumbering\briefempfaengerindex{Brandes, Georg@\textsc{Brandes, Georg}!zzzSchnitzler, Arthur@\emph{von Arthur Schnitzler}!1896-08-071@{7. 8. 1896}|(be} \toendnotes[C]{\smallbreak\pagebreak[2]} \Standort{Kopenhagen, Det Kongelige Bibliotek, Georg Brandes Arkiv, box 125.}
\physDesc{Brief, 1 Blatt, 2 Seiten
\newline{}Handschrift: schwarze Tinte, deutsche Kurrent\newline{}Ordnung: mit Bleistift von unbekannter Hand auf der ersten Seite datiert: »7. 7. 1896 (?)« und nummeriert: »4« }\buchAbdrucke{\weitereDrucke{Georg Brandes, Arthur Schnitzler: \emph{Ein Briefwechsel}. Hg. Kurt Bergel. Bern: \emph{Francke} 1956, S. 57.} }\pstart{}{\pb}Sehr geehrter Herr,\pend\pstart
           ſeit ein paar Tagen bin ich hier, in Skodsborg,
                        Badehotel\oindex{Badehotel@\textbf{Badehotel}|pw}, in Gesellſchaft von Dr \textsc{Richard Beer-Hofmann}\pwindex{Beer-Hofmann, Richard 11.07.1866 – 26.09.1945@\textsc{Beer-Hofmann, Richard} (11.07.1866 – 26.09.1945), \emph{Schriftsteller}|pw}, und bleibe wohl noch bis gegen den 20. da. Ich wäre
                    höchſt erfreut, wenn mir im Laufe dieſer Zeit einmal Gelegenheit geboten würde,
                    Sie zu ſprechen, und, wie ich aus ihrem Brief an Dr. B. H.\pwindex{Beer-Hofmann, Richard 11.07.1866 – 26.09.1945@\textsc{Beer-Hofmann, Richard} (11.07.1866 – 26.09.1945), \emph{Schriftsteller}|pw} entnehmen {\pb}möchte,
                    liegt das im Bereiche der Wahrſcheinlichkeit. Somit darf ich Sie heute in der
                    angenehmen Hoffnung verbindlichſt grüßen, Ihnen bald perſönlich die Hand drücken
                    zu können.\pend
           \pstart
           Ihr dankbar ergebener{\\[\baselineskip]}\spacefill\mbox{Arthur Schnitzler}\pend
           \leftskip=0em{}\pstart
           Skodsborg\oindex{Skodsborg@\textbf{Skodsborg}|pw}{ }7/8. 96.\pend
                     \endnumbering\briefempfaengerindex{Brandes, Georg@\textsc{Brandes, Georg}!zzzSchnitzler, Arthur@\emph{von Arthur Schnitzler}!1896-08-071@{7. 8. 1896}|)be}\mylabel{h}\end{ledgroupsized}  \newcommand{\dateiname}{L00578}\newcommand{\titel}{Arthur Schnitzler an Georg Brandes, 7. 8. 1896}\newcommand{\editorInnen}{Martin Anton Müller und Gerd-Hermann Susen}%% latex-leseansicht-abspann.tex
%% Abspann für die Leseansicht.
%% Der Schalter \ifkorrekturansicht ist bereits durch den Vorspann gesetzt.

%% latex-abspann.tex
%% Gemeinsamer Abspann für Korrekturansicht und Leseansicht.
%% Setzt den Schalter \ifkorrekturansicht voraus (gesetzt in den
%% einbindenden Dateien latex-korrekturansicht-abspann.tex bzw.
%% latex-leseansicht-abspann.tex).
%% ---------------------------------------------------------------

\normalsize

% Das esempio-Environment wird nur in der Leseansicht benötigt
\ifkorrekturansicht\else
\newenvironment{esempio}[3]%
{
    \vspace{1.5ex}
    \rlap{\underline{#1}}
    \par
    \setlength{\parindent}{0cm}
    \nopagebreak
    \leftskip=#2cm
    \rightskip=#3cm
}
{
    \par
}
\fi

\doendnotes{C}
\bigskip
\vfill

\clearpage

\footnotesize

\ifkorrekturansicht
  \lohead{\textsc{register}}
\fi

% theindex-Environment neu definieren ohne reledmac
\makeatletter
\renewenvironment{theindex}{%
  \ifkorrekturansicht
    \section*{\indexname}%
  \else
    \subsubsection*{Index der erwähnten Entitäten}%
  \fi
  \setlength{\parindent}{0pt}%
  \setlength{\parskip}{0pt plus 0.3pt}%
  \let\item\@idxitem
}{%
  \ifkorrekturansicht\clearpage\fi
}
\makeatother

\IfFileExists{\jobname-pw.ind}{\input{\jobname-pw.ind}}{}

% Quellenangabe nur in der Leseansicht
\ifkorrekturansicht\else
% Fallback-Definitionen, falls die .tex-Datei \titel etc. nicht gesetzt hat
\providecommand{\titel}{}
\providecommand{\editorInnen}{}
\providecommand{\dateiname}{\jobname}

\vspace{3cm}

\vfill

\footnotesize
\textsc{Quelle}: \titel. Herausgegeben von {\editorInnen}. In: \emph{Arthur Schnitzler: Briefwechsel mit Autorinnen und Autoren}.
 Digitale Edition, https://schnitzler-briefe.acdh.oeaw.ac.at/{\dateiname}.html (Stand \today)
\fi

\end{document}


      