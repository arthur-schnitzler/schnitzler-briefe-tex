\input{../tex-inputs/latex-pdf-vorspann}
\begin{center}
            \textcolor{red}{ENTWURF. ENTZIFFERUNG NOCH NICHT KORREKTURGELESEN}
                      \end{center}
            
               \section[Arthur Schnitzler an Hermann Bahr, 19. 7. {[}1913{]}]{ Arthur Schnitzler an Hermann Bahr, 19. 7. {[}1913{]}}\nopagebreak\mylabel{v}\rehead{ }\begin{ledgroupsized}[t]{13cm}\normalsize\beginnumbering\briefempfaengerindex{Bahr, Hermann@\textsc{Bahr, Hermann}!zzzSchnitzler, Arthur@\emph{von Arthur Schnitzler}!1913-07-191@{19. 7. {[}1913{]}}|(be} \toendnotes[C]{\smallbreak\pagebreak[2]} \Standort{TMW, HS AM 23397 Ba.}
\physDesc{Telegramm
\newline{}maschinell\newline{}Versand: Stempel: »\nobreak{}19 7 {[}1913{]}, Nm\nobreak{}«.  }\buchAbdrucke{\weitereDrucke{1) \emph{20. 7. [1913].} In: Arthur Schnitzler: \emph{The Letters of Arthur Schnitzler to Hermann Bahr}. Edited, annotated, and with an introduction, by Donald G.
                        Daviau. Chapel Hill: \emph{The University of North Carolina Press} 1978, S. 112 (University of North Carolina studies in the Germanic languages
                        and literatures, 89).} \weitereDrucke{2) Hermann Bahr, Arthur Schnitzler: \emph{Briefwechsel, Aufzeichnungen, Dokumente (1891–1931)}. Hg. Kurt Ifkovits und Martin Anton Müller. Göttingen: \emph{Wallstein} 2018, S. 488.} }\toendnotes[C]{\smallbreak}\pstart{}{\pb}herman bahr salzburg\oindex{Salzburg@\textbf{Salzburg}|pw}\pend{}\pstart{}schloss arenberg\oindex{Schloss Arenberg@\textbf{Schloss Arenberg}|pw}\pend{}{\bigskip}\pstart
           \noindent{}{\pb}wien\oindex{Wien@\textbf{Wien}|pw} 111+ 881 40 19{ }11 10 m =\pend
           \pstart
           = vergangner gemeinsamer stunden inigst gedenkend noch manche kuenftige erhoffend doch
               auch in getrennten dir freundschaftlich nach send ich dir zu gleich im namen meiner
                  frau\pwindex{Schnitzler, Olga 17.01.1882 – 13.01.1970@\textsc{Schnitzler, Olga} (17.01.1882 – 13.01.1970), \emph{Schauspielerin, Sängerin}|pwv} herzlichste wuensche und
               treue gruesze als dein alter \spacefill\mbox{= arthur schnitzler =''}\pend
           \endnumbering\briefempfaengerindex{Bahr, Hermann@\textsc{Bahr, Hermann}!zzzSchnitzler, Arthur@\emph{von Arthur Schnitzler}!1913-07-191@{19. 7. {[}1913{]}}|)be}\mylabel{h}\end{ledgroupsized}  \newcommand{\dateiname}{L02141}\newcommand{\titel}{Arthur Schnitzler an Hermann Bahr, 19. 7. [1913]}\newcommand{\editorInnen}{ Kurt Ifkovits,  Martin Anton Müller}\input{../tex-inputs/latex-pdf-abspann}
      