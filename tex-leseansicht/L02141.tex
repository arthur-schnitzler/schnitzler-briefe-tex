%% latex-korrekturansicht-vorspann.tex
%% Vorspann für die Korrekturansicht.
%% Lädt die gemeinsame Datei latex-vorspann.tex mit gesetztem Schalter.

\newif\ifkorrekturansicht
\korrekturansichttrue

\input{../tex-inputs/latex-vorspann}


\section[Arthur Schnitzler an Hermann Bahr, 19. 7. {[}1913{]}]{L02141 Arthur Schnitzler an Hermann Bahr, 19. 7. {[}1913{]}}
\nopagebreak\mylabel{L02141v}
\rehead{ }\normalsize\beginnumbering\briefempfaengerindex{Bahr, Hermann@\textsc{Bahr, Hermann}!zzzSchnitzler, Arthur@\emph{von Arthur Schnitzler}!1913-07-191@{19. 7. {[}1913{]}}|(be}
\toendnotes[C]{\smallbreak\pagebreak[2]}\Standort{TMW, HS AM 23397 Ba.}
\physDesc{Telegramm, 311 Zeichen
\newline{}maschinell
\newline{}Versand: Stempel: »\nobreak{}19 7 {[}1913{]}, Nm\nobreak{}«.  }
\buchAbdrucke{\weitereDrucke{1) Arthur Schnitzler: \emph{The Letters of Arthur Schnitzler to Hermann Bahr}. Chapel Hill: \emph{The University of North Carolina Press} 1978, S. 112.} \weitereDrucke{2) Hermann Bahr, Arthur Schnitzler: \emph{Briefwechsel, Aufzeichnungen, Dokumente (1891–1931)}. Göttingen: \emph{Wallstein} 2018, S. 488.} }\toendnotes[C]{\smallbreak}\pstart{}{\pb}herman bahr salzburg\oindex{Salzburg@\textbf{Salzburg}, \emph{A.ADM2}|pw}\pend{}\pstart{}schloss arenberg\oindex{Schloss Arenberg@\textbf{Schloss Arenberg}, \emph{Schloss (K.SLS)}|pw}\pend{}{\bigskip}\vspace{1em}
\pstart
           {\pb}wien\oindex{Wien@\textbf{Wien}, \emph{A.ADM2}|pw} 111+ 881 40 19{ }11 10 m =\pend
           \vspace{0.5em}
\pstart
           = vergangner gemeinsamer stunden inigst gedenkend noch manche kuenftige erhoffend
               doch auch in getrennten dir freundschaftlich nach send ich dir zu gleich im namen
               meiner frau\pwindex{Schnitzler, Olga 17.01.1882 – 13.01.1970@\textsc{Schnitzler, Olga} (17.01.1882 – 13.01.1970), \emph{Schauspieler/Schauspielerin, Sänger/Sängerin}|pwv} herzlichste
               wuensche und treue gruesze als dein alter \spacefill\mbox{= arthur schnitzler =''}\pend
           \selectlanguage{ngerman}\endnumbering\briefempfaengerindex{Bahr, Hermann@\textsc{Bahr, Hermann}!zzzSchnitzler, Arthur@\emph{von Arthur Schnitzler}!1913-07-191@{19. 7. {[}1913{]}}|)be}\mylabel{L02141h}  \normalsize

\doendnotes{C}
\bigskip
\vfill

\clearpage

\footnotesize

\lohead{\textsc{register}}

% Definiere theindex-Environment komplett neu ohne reledmac
\makeatletter
\renewenvironment{theindex}{%
  \section*{\indexname}%
  \setlength{\parindent}{0pt}%
  \setlength{\parskip}{0pt plus 0.3pt}%
  \let\item\@idxitem
}{%
  \clearpage
}
\makeatother

\IfFileExists{\jobname-pw.ind}{\input{\jobname-pw.ind}}{}

\end{document}

      