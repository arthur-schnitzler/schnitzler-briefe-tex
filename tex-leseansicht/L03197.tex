%% latex-leseansicht-vorspann.tex
%% Vorspann für die Leseansicht.
%% Lädt die gemeinsame Datei latex-vorspann.tex mit nicht gesetztem Schalter.

\newif\ifkorrekturansicht
\korrekturansichtfalse

\input{../tex-inputs/latex-vorspann}

\begin{center}
            \textcolor{red}{ENTWURF, NICHT FERTIG KORRIGIERT}
                      \end{center}
            
         
         \renewcommand{\erwaehntePersonen}{Personen: Ignacy Daszyński, Peter Dorner,  Leopold II. von Belgien,  Louise von Belgien, Fedor Mamroth, Géza von Mattachich, Olga Schnitzler, Elisabeth Steinrück, Irene Triesch, Émile Zola}
         \renewcommand{\erwaehnteInstitutionen}{Institutionen: Die Zeit. Wiener Wochenschrift, Neue Freie Presse, Reichsrat}
         \renewcommand{\erwaehnteOrte}{Orte: Berlin, Dessauer Straße, Deutsches Theater Berlin, Deutschland, Hoftheater Stuttgart, Kroatien, Mödling, Stuttgart, Wien, Österreich}
         \renewcommand{\erwaehnteWerke}{Werke: Die Frau mit dem Dolche, Extrapost. Unparteiische Montags-Zeitung, Lebendige Stunden. Vier Einakter, Literatur, Politische Glossen. Der Ernst der Volksvertreter, Sehnsucht}
               \section[ Paul Goldmann an Arthur Schnitzler, 10. 2. {[}1902{]}]{ Paul Goldmann an Arthur Schnitzler, 10. 2. {[}1902{]}}\nopagebreak\mylabel{v}\rehead{ }\begin{ledgroupsized}[t]{13cm}\normalsize\beginnumbering \toendnotes[C]{\smallbreak\pagebreak[2]} \Standort{DLA, A:Schnitzler, HS.NZ85.1.3172.}
\physDesc{Brief, 1 Blatt, 3 Seiten
\newline{}Handschrift: blaue Tinte, deutsche Kurrent\newline{}Beilage: ein Zeitungsausschnitt, beschnitten und eingeklebt 
\newline{}Schnitzler: 1) mit Bleistift das Jahr »{[}1{]}902« vermerkt  2) mit rotem Buntstift eine Unterstreichung}\toendnotes[C]{\smallbreak}\pstart
           \noindent{}\raggedleft{}{\pb}\textcolor{gray}{\textbf{DESSAUERSTRASSE 19}}\oindex{Dessauer Strasse@\textbf{Dessauer Straße}|pw}\pend
           \pstart
           Berlin\oindex{Berlin@\textbf{Berlin}|pw}, 10. Februar.\pend
           \pstart{}Mein lieber Freund,\pend\pstart
           Wenn ich \textsc{Arthur Schnitzler} wäre, weißt Du, was ich thäte?
               Ich hätte den Ehrgeiz, nach all’ den ſchönen literariſchen Leiſtungen auch noch eine
               menſchlich große That zu vollbringen. Und würde mich darum an die Spitze einer
               Bewegung ſtellen, die zum Zweck hätte, den \label{K_L03197-1v}\edtext{Fall \textsc{Matassich-Keglevich\pwindex{Mattachich, Geza von 1867-12-19 – 1923-09-29@\textsc{Mattachich, Géza von} (1867-12-19 – 1923-09-29), \emph{Oberstleutnant}|pw}}}{\lemma{\textnormal{\emph{Fall Matassich-Keglevich}}}\Cendnote{\textnormal{Oberstleutnant Géza von Mattachich\pwindex{Mattachich, Geza von 1867-12-19 – 1923-09-29@\textsc{Mattachich, Géza von} (1867-12-19 – 1923-09-29), \emph{Oberstleutnant}|pwk} hatte ab 1895 eine intime
                  Beziehung mit Louise von Belgien\pwindex{Louise von Belgien 1858-02-18 – 1924-03-01@\textsc{Louise von Belgien} (1858-02-18 – 1924-03-01), \emph{Prinzessin}|pwk}. Da sie
                  als älteste Tochter annehmen konnte, nach dem Tod ihres Vaters Leopold II. von Belgien\pwindex{Leopold II. von Belgien 1835-04-09 – 1909-12-17@\textsc{Leopold II. von Belgien} (1835-04-09 – 1909-12-17), \emph{König}|pwk} ein großes Vermögen zu erben,
                  lebte sie über ihre Verhältnisse und machte Schulden. Die beiden wurden im
                     Mai 1898 in Kroatien\oindex{Kroatien@\textbf{Kroatien}|pwk}
                  verhaftet und der Geldwechselfälschung beschuldigt. Während sie in eine
                  psychiatrische Verwahrung kam, wurde er zu sechs Jahren schwerem Kerker
                  verurteilt. Am 8. 2. 1902 hielt Ignacy Daszińsky\pwindex{Daszyński, Ignacy 1866-10-26 – 1936-10-31@\textsc{Daszyński, Ignacy} (1866-10-26 – 1936-10-31), \emph{Politiker, Ministerpräsident}|pwk} im \emph{Reichsrat}\orgindex{Reichsrat@Reichsrat|pwk} eine Rede ([O. V.:] \emph{Politische Glossen. Der Ernst der
                           Volksvertreter}\pwindex{?? Werk@Nicht ermittelte Verfasserinnen und Verfasser!Politische Glossen. Der Ernst der Volksvertreter1902-02-10@\emph{Politische Glossen. Der Ernst der Volksvertreter} {[}1902-02-10{]}|pwk}. In: \emph{Extrapost.
                              Unparteiische Montags-Zeitung}\pwindex{?? Werk@Nicht ermittelte Verfasserinnen und Verfasser!Extrapost. Unparteiische Montags-Zeitung1882 – 1905@\emph{Extrapost. Unparteiische Montags-Zeitung} {[}1882 – 1905{]}|pwk}. Jg. 21, Nr. 1.045, 10. 2. 1902, S. 2.). Noch
                  im selben Monat wurde Mattachich\pwindex{Mattachich, Geza von 1867-12-19 – 1923-09-29@\textsc{Mattachich, Géza von} (1867-12-19 – 1923-09-29), \emph{Oberstleutnant}|pwk} für unschuldig erklärt und
                  begnadigt.}}}\label{K_L03197-1h}, in dem ſicherlich ein gemeiner Juſtizmord verübt worden iſt,
               aufzuklären. \textsc{Zola\pwindex{Zola, Emile 02.04.1840 – 29.09.1902@\textsc{Zola, Émile} (02.04.1840 – 29.09.1902), \emph{Schriftsteller, Journalist}|pw}} gibt das große Vorbild. Ein \label{K_L03197-2v}\edtext{Artikel}{\lemma{\textnormal{\emph{Artikel}}}\Cendnote{\textnormal{Schnitzler\pwindex{Schnitzler, Arthur 15.05.1862 – 21.10.1931@\textsc{Schnitzler, Arthur} (15.05.1862 – 21.10.1931), \emph{Schriftsteller, Mediziner}|pwk} verzichtete nicht nur hier, sondern zeitlebens darauf, seinen Namen für
                  eine größere (kultur-)politische Kampagne zu verwenden.}}}\label{K_L03197-2h} in einem großen Wien\oindex{Wien@\textbf{Wien}|pw}er oder reichsdeutſch\oindex{Deutschland@\textbf{Deutschland}|pwv}en Blatte mit Darlegung des ganzen Materials (das ſicherlich in
                  Wien\oindex{Wien@\textbf{Wien}|pw}{ }{\pb}zu bekommen iſt, wahrſcheinlich vom Abg. \textsc{Daszinsky\pwindex{Daszyński, Ignacy 1866-10-26 – 1936-10-31@\textsc{Daszyński, Ignacy} (1866-10-26 – 1936-10-31), \emph{Politiker, Ministerpräsident}|pw}}), mit \textsc{Arthur Schnitzlers} klangvollem Namen
               unterzeichnet, würde die Bewegung einleiten und alle empfänglichen
                  He\textcolor{gray}{r}zen in Deutſchland\oindex{Deutschland@\textbf{Deutschland}|pw} und
                  Öſterreich\oindex{Oesterreich@\textbf{Österreich}|pw} für den Fall intereſſiren.
               Vielleicht iſt die Sache in Wien\oindex{Wien@\textbf{Wien}|pw} mit der »Zeit\orgindex{Zeit. Wiener Wochenschrift@Die Zeit. Wiener Wochenschrift|pw}« zu machen. Vielleicht auch mit der N. Fr. Pr.\orgindex{Neue Freie Presse@Neue Freie Presse|pw}\pend
           \pstart
           Wie geht es \textsc{Olga\pwindex{Schnitzler, Olga 17.01.1882 – 13.01.1970@\textsc{Schnitzler, Olga} (17.01.1882 – 13.01.1970), \emph{Schauspielerin, Sängerin}|pw}}? Seid Ihr ſchon in \label{K_L03197-43v}\edtext{\textsc{Mödling}\oindex{Moedling@\textbf{Mödling}|pw}}{\lemma{\textnormal{\emph{Mödling}}}\Cendnote{\textnormal{siehe Paul Goldmann an Arthur Schnitzler, 14. 1. [1902]}}}\label{K_L03197-43h}? Herzliche Grüße an die Mädels\pwindex{Schnitzler, Olga 17.01.1882 – 13.01.1970@\textsc{Schnitzler, Olga} (17.01.1882 – 13.01.1970), \emph{Schauspielerin, Sängerin}|pwv}\pwindex{Steinrueck, Elisabeth 19.11.1885 – 07.04.1920@\textsc{Steinrück, Elisabeth} (19.11.1885 – 07.04.1920)|pwv}!\pend
           \pstart
           Ich habe unbeſchreiblich viel zu thun.\pend
           \pstart
           Dank für Deinen letzten lieben Brief! {\\[\baselineskip]}Viele treue Grüße! {\\[\baselineskip]}Dein
                  \spacefill\mbox{Paul Goldm}\pend
           \leftskip=0em{}\pstart
           \noindent{}{\pb}Das Stück\pwindex{Sehnsucht1902-02-04@\emph{Sehnsucht} {[}1902-02-04{]}|pwv} meines Onkel\pwindex{Mamroth, Fedor 21.02.1851 – 25.06.1907@\textsc{Mamroth, Fedor} (21.02.1851 – 25.06.1907), \emph{Journalist, Kritiker}|pwv}s, das unter dem Namen »Sehnſucht\pwindex{Sehnsucht1902-02-04@\emph{Sehnsucht} {[}1902-02-04{]}|pw}« in \label{K_L03197-6v}\edtext{Stuttgart\oindex{Stuttgart@\textbf{Stuttgart}|pw}}{\lemma{\textnormal{\emph{Stuttgart}}}\Cendnote{\textnormal{Am 4. 2. 1902 wurde Fedor
                     Mamroth\pwindex{Mamroth, Fedor 21.02.1851 – 25.06.1907@\textsc{Mamroth, Fedor} (21.02.1851 – 25.06.1907), \emph{Journalist, Kritiker}|pwk}s vieraktige Komödie \emph{Sehnsucht}\pwindex{Sehnsucht1902-02-04@\emph{Sehnsucht} {[}1902-02-04{]}|pwk} (unter dem Pseudonym F. Albert\pwindex{Mamroth, Fedor 21.02.1851 – 25.06.1907@\textsc{Mamroth, Fedor} (21.02.1851 – 25.06.1907), \emph{Journalist, Kritiker}|pwkv}) am Stuttgarter
                        Hoftheater\oindex{Hoftheater Stuttgart@\textbf{Hoftheater Stuttgart}|pwk} uraufgeführt.}}}\label{K_L03197-6h} aufgeführt wurde, hatte dort einen ſehr
                  ſchönen Erfolg.\pend
           \pstart
           Wie hat ſich die \label{K_L03197-34v}\edtext{Angelegenheit \textsc{Peter Dorner\pwindex{Dorner, Peter 17.02.1857 – 01.04.1931@\textsc{Dorner, Peter} (17.02.1857 – 01.04.1931), \emph{Schmied, Kunsthandwerker, Kunstschmied}|pw}}}{\lemma{\textnormal{\emph{Angelegenheit Peter Dorner}}}\Cendnote{\textnormal{siehe Paul Goldmann an Arthur Schnitzler, 23. 9. [1901]}}}\label{K_L03197-34h} noch entwickelt?\pend
           {\bigskip}\pstart
           \noindent{}\textcolor{gray}{\textbf{\textbf{– }\label{K_L03197-32v}\edtext{\textbf{Arthur Schnitzler’s »}\textbf{Lebendige Stunden}\pwindex{Schnitzler, Arthur 15.05.1862 – 21.10.1931@\textsc{Schnitzler, Arthur} (15.05.1862 – 21.10.1931), \emph{Schriftsteller, Mediziner}!Lebendige Stunden. Vier Einakter1901-12-23@\strich\emph{Lebendige Stunden. Vier Einakter} {[}1901-12-23{]}|pw}\textbf{«,} die bisher in zwanzig Wiederholungen bei unverminderter
                     Zugkraft im \so{Deutſchen Theater}\oindex{Deutsches Theater Berlin@\textbf{Deutsches Theater Berlin}|pw} in Szene gingen, können in den folgenden Wochen nur je einmal auf dem
                     Spielplan erſcheinen, da \so{Irene Trieſch}\pwindex{Triesch, Irene 13.04.1877 – 24.11.1964@\textsc{Triesch, Irene} (13.04.1877 – 24.11.1964), \emph{Schauspielerin}|pw} einen kontraktlichen Urlaub angetreten hat, jedoch allwöchentlich einmal,
                     zunächſt am Mittwoch, den 12., nach Berlin\oindex{Berlin@\textbf{Berlin}|pw} zurückkehren wird, um die von ihr in
                     den »Lebendigen Stunden\pwindex{Schnitzler, Arthur 15.05.1862 – 21.10.1931@\textsc{Schnitzler, Arthur} (15.05.1862 – 21.10.1931), \emph{Schriftsteller, Mediziner}!Lebendige Stunden. Vier Einakter1901-12-23@\strich\emph{Lebendige Stunden. Vier Einakter} {[}1901-12-23{]}|pw}« geſpielten beiden
                     weiblichen Hauptrollen weiterhin darzuſtellen.}{\lemma{\textnormal{\emph{Arthur … darzuſtellen.}}}\Cendnote{\textnormal{Quelle nicht ermittelt; in \emph{Die Frau mit dem Dolche}\pwindex{Schnitzler, Arthur 15.05.1862 – 21.10.1931@\textsc{Schnitzler, Arthur} (15.05.1862 – 21.10.1931), \emph{Schriftsteller, Mediziner}!Frau mit dem Dolche1901@\strich\emph{Die Frau mit dem Dolche} {[}1901{]}|pwk} spielte Irene Triesch\pwindex{Triesch, Irene 13.04.1877 – 24.11.1964@\textsc{Triesch, Irene} (13.04.1877 – 24.11.1964), \emph{Schauspielerin}|pwk} die Rolle der Pauline\pwindex{Schnitzler, Arthur 15.05.1862 – 21.10.1931@\textsc{Schnitzler, Arthur} (15.05.1862 – 21.10.1931), \emph{Schriftsteller, Mediziner}!Frau mit dem Dolche1901@\strich\emph{Die Frau mit dem Dolche} {[}1901{]}|pwkv} und in \emph{Literatur}\pwindex{Schnitzler, Arthur 15.05.1862 – 21.10.1931@\textsc{Schnitzler, Arthur} (15.05.1862 – 21.10.1931), \emph{Schriftsteller, Mediziner}!Literatur1901@\strich\emph{Literatur} {[}1901{]}|pwk} jene der Margarete\pwindex{Schnitzler, Arthur 15.05.1862 – 21.10.1931@\textsc{Schnitzler, Arthur} (15.05.1862 – 21.10.1931), \emph{Schriftsteller, Mediziner}!Literatur1901@\strich\emph{Literatur} {[}1901{]}|pwkv}.}}}\label{K_L03197-32h}}}\pend
           
         
         \endnumbering\mylabel{h}\end{ledgroupsized}  \newcommand{\dateiname}{L03197}\newcommand{\titel}{Paul Goldmann an Arthur Schnitzler, 10. 2. [1902]}\newcommand{\editorInnen}{Martin Anton Müller und Laura Untner}%% latex-leseansicht-abspann.tex
%% Abspann für die Leseansicht.
%% Der Schalter \ifkorrekturansicht ist bereits durch den Vorspann gesetzt.

%% latex-abspann.tex
%% Gemeinsamer Abspann für Korrekturansicht und Leseansicht.
%% Setzt den Schalter \ifkorrekturansicht voraus (gesetzt in den
%% einbindenden Dateien latex-korrekturansicht-abspann.tex bzw.
%% latex-leseansicht-abspann.tex).
%% ---------------------------------------------------------------

\normalsize

% Das esempio-Environment wird nur in der Leseansicht benötigt
\ifkorrekturansicht\else
\newenvironment{esempio}[3]%
{
    \vspace{1.5ex}
    \rlap{\underline{#1}}
    \par
    \setlength{\parindent}{0cm}
    \nopagebreak
    \leftskip=#2cm
    \rightskip=#3cm
}
{
    \par
}
\fi

\doendnotes{C}
\bigskip
\vfill

\clearpage

\footnotesize

\ifkorrekturansicht
  \lohead{\textsc{register}}
\fi

% theindex-Environment neu definieren ohne reledmac
\makeatletter
\renewenvironment{theindex}{%
  \ifkorrekturansicht
    \section*{\indexname}%
  \else
    \subsubsection*{Index der erwähnten Entitäten}%
  \fi
  \setlength{\parindent}{0pt}%
  \setlength{\parskip}{0pt plus 0.3pt}%
  \let\item\@idxitem
}{%
  \ifkorrekturansicht\clearpage\fi
}
\makeatother

\IfFileExists{\jobname-pw.ind}{\input{\jobname-pw.ind}}{}

% Quellenangabe nur in der Leseansicht
\ifkorrekturansicht\else
% Fallback-Definitionen, falls die .tex-Datei \titel etc. nicht gesetzt hat
\providecommand{\titel}{}
\providecommand{\editorInnen}{}
\providecommand{\dateiname}{\jobname}

\vspace{3cm}

\vfill

\footnotesize
\textsc{Quelle}: \titel. Herausgegeben von {\editorInnen}. In: \emph{Arthur Schnitzler: Briefwechsel mit Autorinnen und Autoren}.
 Digitale Edition, https://schnitzler-briefe.acdh.oeaw.ac.at/{\dateiname}.html (Stand \today)
\fi

\end{document}


      