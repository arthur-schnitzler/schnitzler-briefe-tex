%% latex-leseansicht-vorspann.tex
%% Vorspann für die Leseansicht.
%% Lädt die gemeinsame Datei latex-vorspann.tex mit nicht gesetztem Schalter.

\newif\ifkorrekturansicht
\korrekturansichtfalse

\input{../tex-inputs/latex-vorspann}


\section[ Paul Goldmann an Arthur Schnitzler, 10. 2. [1902]]{L03197 Paul Goldmann an Arthur Schnitzler,  10. 2. [1902]}
\nopagebreak\mylabel{L03197v}
\rehead{ }\normalsize\beginnumbering\briefempfaengerindex{Schnitzler, Arthur@\textsc{Schnitzler, Arthur}!zzzGoldmann, Paul@\emph{von Paul Goldmann}!1902-02-101@{10. 2. [1902]}|(be}
\toendnotes[C]{\smallbreak\pagebreak[2]}
\correspDesc{Versand  durch Paul Goldmann am 10. 2. [1902] in Berlin
\newline{}Erhalt  durch Arthur Schnitzler im Zeitraum [11. 2. 1902
                  – 15. 2. 1902?] in Wien}\toendnotes[C]{\smallbreak}
\Standort{DLA, A:Schnitzler, HS.NZ85.1.3172.}
\physDesc{Brief, 1 Blatt, 3 Seiten, 1208 Zeichen
\newline{}Handschrift: blaue Tinte, deutsche Kurrent
\newline{}Beilage: ein Zeitungsausschnitt, beschnitten und eingeklebt 
\newline{}Schnitzler: 1) mit Bleistift das Jahr »902« vermerkt  2) mit rotem Buntstift eine Unterstreichung}\toendnotes[C]{\smallbreak}
\pstart
           \raggedleft{}{\pb}\textcolor{gray}{\textbf{DESSAUERSTRASSE 19}}\oindex{Dessauer Straße@\textbf{Dessauer Straße}, \emph{Straße}|pw}\pend
           
\pstart
           Berlin\oindex{Berlin@\textbf{Berlin}, \emph{Hauptstadt}|pw}, 10. Februar.\pend
           
\pstart{}Mein lieber Freund,\pend\vspace{0.5em}
\pstart
           Wenn ich \textsc{Arthur Schnitzler} wäre, weißt Du, was ich thäte?
               Ich hätte den Ehrgeiz, nach all’ den{ }ſchönen literariſchen Leiſtungen auch noch eine
               menſchlich große That zu vollbringen. Und würde mich darum an die Spitze einer
               Bewegung{ }ſtellen, die zum Zweck hätte, den \label{K_L03197-1v}\edtext{Fall \textsc{Matassich-Keglevich\pwindex{Mattachich, Géza von 19.\,12.\,1867 Tomaševac – 29.\,9.\,1923 Paris@\textsc{Mattachich, Géza von} (19.\,12.\,1867 Tomaševac – 29.\,9.\,1923 Paris), \emph{Oberstleutnant}|pw}}}{\lemma{\textnormal{\emph{Fall Matassich-Keglevich}}}\Cendnote{\textnormal{Oberstleutnant Géza von Mattachich\pwindex{Mattachich, Géza von 19.\,12.\,1867 Tomaševac – 29.\,9.\,1923 Paris@\textsc{Mattachich, Géza von} (19.\,12.\,1867 Tomaševac – 29.\,9.\,1923 Paris), \emph{Oberstleutnant}|pwk} hatte ab 1895 eine intime
                  Beziehung mit Louise von Belgien\pwindex{Louise von Belgien 18.\,2.\,1858 Laeken – 1.\,3.\,1924 Wiesbaden@\textsc{Louise von Belgien} (18.\,2.\,1858 Laeken – 1.\,3.\,1924 Wiesbaden), \emph{Prinzessin}|pwk}. Da sie
                  als älteste Tochter annehmen konnte, nach dem Tod ihres Vaters Leopold II. von Belgien\pwindex{Leopold II. von Belgien 9.\,4.\,1835 Brüssel – 17.\,12.\,1909 ebd.@\textsc{Leopold II. von Belgien} (9.\,4.\,1835 Brüssel – 17.\,12.\,1909 ebd.), \emph{König}|pwk} ein großes Vermögen zu erben,
                  lebte sie über ihre Verhältnisse und machte Schulden. Die beiden wurden im
                     Mai 1898 in Kroatien\oindex{Kroatien@\textbf{Kroatien}|pwk}
                  verhaftet und der Geldwechselfälschung beschuldigt. Während sie in eine
                  psychiatrische Verwahrung kam, wurde er zu sechs Jahren schwerem Kerker
                  verurteilt. Am 8. 2. 1902 hatte Ignacy Daszińsky\pwindex{Daszyński, Ignacy 26.\,10.\,1866 Zbarazh – 31.\,10.\,1936 Cieszyn@\textsc{Daszyński, Ignacy} (26.\,10.\,1866 Zbarazh – 31.\,10.\,1936 Cieszyn), \emph{Politiker, Ministerpräsident}|pwk} im \emph{Reichsrat}\orgindex{Reichsrat@Reichsrat|pwk} eine Rede gehalten (vgl. [O. V.]: \emph{Politische Glossen. Der Ernst der Volksvertreter}\pwindex{Politische Glossen. Der Ernst der Volksvertreter@\emph{Politische Glossen. Der Ernst der Volksvertreter}|pwk}. In:
                        \emph{Extrapost. Unparteiische
                        Montags-Zeitung}\pwindex{Extrapost. Unparteiische Montags-Zeitung@\emph{Extrapost. Unparteiische Montags-Zeitung}|pwk}. Jg. 21, Nr. 1045, 10. 2. 1902, S. 2). Noch im selben Monat wurde Mattachich\pwindex{Mattachich, Géza von 19.\,12.\,1867 Tomaševac – 29.\,9.\,1923 Paris@\textsc{Mattachich, Géza von} (19.\,12.\,1867 Tomaševac – 29.\,9.\,1923 Paris), \emph{Oberstleutnant}|pwk} für unschuldig erklärt und
                  begnadigt.}}}\label{K_L03197-1}, in dem{ }ſicherlich ein gemeiner Juſtizmord verübt worden iſt,
               aufzuklären. \textsc{Zola\pwindex{Zola, Émile 2.\,4.\,1840 Paris – 29.\,9.\,1902 ebd.@\textsc{Zola, Émile} (2.\,4.\,1840 Paris – 29.\,9.\,1902 ebd.), \emph{Schriftsteller, Journalist}|pw}} gibt das große Vorbild. Ein \label{K_L03197-2v}\edtext{Artikel}{\lemma{\textnormal{\emph{Artikel}}}\Cendnote{\textnormal{Schnitzler verzichtete nicht nur hier,
                  sondern zeitlebens darauf, seinen Namen für eine größere (kultur-)politische
                  Kampagne zu verwenden.}}}\label{K_L03197-2} in einem großen Wien\oindex{Wien@\textbf{Wien}, \emph{Verwaltungsgebiet}|pw}er oder reichsdeutſch\oindex{Deutschland@\textbf{Deutschland}|pwv}en Blatte mit Darlegung des ganzen Materials (das{ }ſicherlich in
                  Wien\oindex{Wien@\textbf{Wien}, \emph{Verwaltungsgebiet}|pw}{ }{\pb}zu bekommen iſt, wahrſcheinlich vom Abg. \textsc{Daszinsky\pwindex{Daszyński, Ignacy 26.\,10.\,1866 Zbarazh – 31.\,10.\,1936 Cieszyn@\textsc{Daszyński, Ignacy} (26.\,10.\,1866 Zbarazh – 31.\,10.\,1936 Cieszyn), \emph{Politiker, Ministerpräsident}|pw}}), mit \textsc{Arthur Schnitzlers} klangvollem Namen
               unterzeichnet, würde die Bewegung einleiten und alle empfänglichen
                  He\textcolor{gray}{r}zen in Deutſchland\oindex{Deutschland@\textbf{Deutschland}|pw} und
                  Öſterreich\oindex{Österreich@\textbf{Österreich}|pw} für den Fall intereſſiren.
               Vielleicht iſt die Sache in Wien\oindex{Wien@\textbf{Wien}, \emph{Verwaltungsgebiet}|pw} mit der »Zeit\orgindex{Zeit. Wiener Wochenschrift@Die Zeit. Wiener Wochenschrift|pw}« zu machen. Vielleicht auch mit der N. Fr. Pr.\orgindex{Neue Freie Presse@Neue Freie Presse|pw}\pend
           
\pstart
           Wie geht es \textsc{Olga\pwindex{Schnitzler, Olga 17.\,1.\,1882 Wien – 13.\,1.\,1970 Lugano@\textsc{Schnitzler, Olga} (17.\,1.\,1882 Wien – 13.\,1.\,1970 Lugano), \emph{Schauspielerin, Sängerin}|pw}}? Seid Ihr{ }ſchon in \label{K_L03197-3v}\edtext{\textsc{Mödling}\oindex{Mödling@\textbf{Mödling}, \emph{Hauptstadt}|pw}}{\lemma{\textnormal{\emph{Mödling}}}\Cendnote{\textnormal{Siehe XXXX Auszeichnungsfehler: Dokument L03192 nicht gefunden.
               }}}\label{K_L03197-3}? Herzliche Grüße an die Mädels\pwindex{Schnitzler, Olga 17.\,1.\,1882 Wien – 13.\,1.\,1970 Lugano@\textsc{Schnitzler, Olga} (17.\,1.\,1882 Wien – 13.\,1.\,1970 Lugano), \emph{Schauspielerin, Sängerin}|pwv}\pwindex{Steinrück, Elisabeth 19.\,11.\,1885 – 7.\,4.\,1920 Partenkirchen@\textsc{Steinrück, Elisabeth} (19.\,11.\,1885 – 7.\,4.\,1920 Partenkirchen)|pwv}!\pend
           
\pstart
           Ich habe unbeſchreiblich viel zu thun.\pend
           
\pstart
           Dank für Deinen letzten lieben Brief! {\\[\baselineskip]}Viele treue Grüße! {\\[\baselineskip]}Dein
                  \spacefill\mbox{Paul Goldm}\pend
           \leftskip=0em{}
\pstart
           \noindent{}{\pb}Das Stück\pwindex{Mamroth, Fedor 21.\,2.\,1851 Breslau – 25.\,6.\,1907 Frankfurt am Main@\textsc{Mamroth, Fedor} (21.\,2.\,1851 Breslau – 25.\,6.\,1907 Frankfurt am Main), \emph{Journalist, Kritiker}!Sehnsucht@\strich\emph{Sehnsucht}|pwv} meines Onkels\pwindex{Mamroth, Fedor 21.\,2.\,1851 Breslau – 25.\,6.\,1907 Frankfurt am Main@\textsc{Mamroth, Fedor} (21.\,2.\,1851 Breslau – 25.\,6.\,1907 Frankfurt am Main), \emph{Journalist, Kritiker}|pwv}, das unter dem Namen »Sehnſucht\pwindex{Mamroth, Fedor 21.\,2.\,1851 Breslau – 25.\,6.\,1907 Frankfurt am Main@\textsc{Mamroth, Fedor} (21.\,2.\,1851 Breslau – 25.\,6.\,1907 Frankfurt am Main), \emph{Journalist, Kritiker}!Sehnsucht@\strich\emph{Sehnsucht}|pw}« in \label{K_L03197-4v}\edtext{Stuttgart\oindex{Stuttgart@\textbf{Stuttgart}|pw}}{\lemma{\textnormal{\emph{Stuttgart}}}\Cendnote{\textnormal{Am 4. 2. 1902 war Fedor
                     Mamroths\pwindex{Mamroth, Fedor 21.\,2.\,1851 Breslau – 25.\,6.\,1907 Frankfurt am Main@\textsc{Mamroth, Fedor} (21.\,2.\,1851 Breslau – 25.\,6.\,1907 Frankfurt am Main), \emph{Journalist, Kritiker}|pwk} vieraktige Komödie \emph{Sehnsucht}\pwindex{Mamroth, Fedor 21.\,2.\,1851 Breslau – 25.\,6.\,1907 Frankfurt am Main@\textsc{Mamroth, Fedor} (21.\,2.\,1851 Breslau – 25.\,6.\,1907 Frankfurt am Main), \emph{Journalist, Kritiker}!Sehnsucht@\strich\emph{Sehnsucht}|pwk} (unter dem Pseudonym F. Albert\pwindex{Mamroth, Fedor 21.\,2.\,1851 Breslau – 25.\,6.\,1907 Frankfurt am Main@\textsc{Mamroth, Fedor} (21.\,2.\,1851 Breslau – 25.\,6.\,1907 Frankfurt am Main), \emph{Journalist, Kritiker}|pwkv}) am Stuttgarter
                        Hoftheater\oindex{Hoftheater Stuttgart@\textbf{Hoftheater Stuttgart}, \emph{Theater}|pwk} uraufgeführt worden.}}}\label{K_L03197-4} aufgeführt wurde, hatte dort einen{ }ſehr{ }ſchönen Erfolg.\pend
           
\pstart
           Wie hat{ }ſich die \label{K_L03197-5v}\edtext{Angelegenheit \textsc{Peter Dorner\pwindex{Dorner, Peter 17.\,2.\,1857 Welsberg-Taisten – 1.\,4.\,1931 ebd.@\textsc{Dorner, Peter} (17.\,2.\,1857 Welsberg-Taisten – 1.\,4.\,1931 ebd.), \emph{Schmied, Kunsthandwerker, Kunstschmied}|pw}}}{\lemma{\textnormal{\emph{Angelegenheit Peter Dorner}}}\Cendnote{\textnormal{Siehe XXXX Auszeichnungsfehler: Dokument L03085 nicht gefunden.
                  }}}\label{K_L03197-5} noch entwickelt?\pend
           {\vspace{1\baselineskip}}
\pstart
           \textcolor{gray}{\textbf{\textbf{– }\label{K_L03197-6v}\edtext{\textbf{Arthur Schnitzler’s »}\textbf{Lebendige Stunden}\pwindex{Schnitzler, Arthur 15.\,5.\,1862 Wien – 21.\,10.\,1931 ebd.@\textsc{Schnitzler, Arthur} (15.\,5.\,1862 Wien – 21.\,10.\,1931 ebd.), \emph{Schriftsteller, Mediziner}!Lebendige Stunden. Vier Einakter@\strich\emph{Lebendige Stunden. Vier Einakter}|pw}\textbf{«,} die bisher in zwanzig Wiederholungen bei unverminderter
                     Zugkraft im \so{Deutſchen Theater}\oindex{Deutsches Theater Berlin@\textbf{Deutsches Theater Berlin}, \emph{Theater}|pw} in Szene gingen, können in den folgenden Wochen nur je einmal auf dem
                     Spielplan erſcheinen, da \so{Irene Trieſch}\pwindex{Triesch, Irene 13.\,4.\,1877 Wien – 24.\,11.\,1964 Basel@\textsc{Triesch, Irene} (13.\,4.\,1877 Wien – 24.\,11.\,1964 Basel), \emph{Schauspielerin}|pw} einen kontraktlichen Urlaub angetreten hat, jedoch allwöchentlich einmal,
                     zunächſt am Mittwoch, den 12., nach Berlin\oindex{Berlin@\textbf{Berlin}, \emph{Hauptstadt}|pw} zurückkehren wird, um die von ihr in
                     den »Lebendigen Stunden\pwindex{Schnitzler, Arthur 15.\,5.\,1862 Wien – 21.\,10.\,1931 ebd.@\textsc{Schnitzler, Arthur} (15.\,5.\,1862 Wien – 21.\,10.\,1931 ebd.), \emph{Schriftsteller, Mediziner}!Lebendige Stunden. Vier Einakter@\strich\emph{Lebendige Stunden. Vier Einakter}|pw}« geſpielten beiden
                     weiblichen Hauptrollen weiterhin darzuſtellen.}{\lemma{\textnormal{\emph{Arthur … darzustellen.}}}\Cendnote{\textnormal{Quelle nicht ermittelt; in \emph{Die Frau mit dem Dolche}\pwindex{Schnitzler, Arthur 15.\,5.\,1862 Wien – 21.\,10.\,1931 ebd.@\textsc{Schnitzler, Arthur} (15.\,5.\,1862 Wien – 21.\,10.\,1931 ebd.), \emph{Schriftsteller, Mediziner}!Frau mit dem Dolche@\strich\emph{Die Frau mit dem Dolche}|pwk} spielte Irene Triesch\pwindex{Triesch, Irene 13.\,4.\,1877 Wien – 24.\,11.\,1964 Basel@\textsc{Triesch, Irene} (13.\,4.\,1877 Wien – 24.\,11.\,1964 Basel), \emph{Schauspielerin}|pwk} die Rolle der Pauline\pwindex{Schnitzler, Arthur 15.\,5.\,1862 Wien – 21.\,10.\,1931 ebd.@\textsc{Schnitzler, Arthur} (15.\,5.\,1862 Wien – 21.\,10.\,1931 ebd.), \emph{Schriftsteller, Mediziner}!Frau mit dem Dolche@\strich\emph{Die Frau mit dem Dolche}|pwkv} und in \emph{Literatur}\pwindex{Schnitzler, Arthur 15.\,5.\,1862 Wien – 21.\,10.\,1931 ebd.@\textsc{Schnitzler, Arthur} (15.\,5.\,1862 Wien – 21.\,10.\,1931 ebd.), \emph{Schriftsteller, Mediziner}!Literatur@\strich\emph{Literatur}|pwk} jene der Margarete\pwindex{Schnitzler, Arthur 15.\,5.\,1862 Wien – 21.\,10.\,1931 ebd.@\textsc{Schnitzler, Arthur} (15.\,5.\,1862 Wien – 21.\,10.\,1931 ebd.), \emph{Schriftsteller, Mediziner}!Literatur@\strich\emph{Literatur}|pwkv}.}}}\label{K_L03197-6}}}\pend
           \selectlanguage{ngerman}\endnumbering\briefempfaengerindex{Schnitzler, Arthur@\textsc{Schnitzler, Arthur}!zzzGoldmann, Paul@\emph{von Paul Goldmann}!1902-02-101@{10. 2. [1902]}|)be}\mylabel{L03197h}  \newcommand{\dateiname}{L03197}\newcommand{\titel}{Paul Goldmann an Arthur Schnitzler, 10. 2. [1902]}\newcommand{\editorInnen}{Martin Anton Müller und Laura Untner}%% latex-leseansicht-abspann.tex
%% Abspann für die Leseansicht.
%% Der Schalter \ifkorrekturansicht ist bereits durch den Vorspann gesetzt.

%% latex-abspann.tex
%% Gemeinsamer Abspann für Korrekturansicht und Leseansicht.
%% Setzt den Schalter \ifkorrekturansicht voraus (gesetzt in den
%% einbindenden Dateien latex-korrekturansicht-abspann.tex bzw.
%% latex-leseansicht-abspann.tex).
%% ---------------------------------------------------------------

\normalsize

% Das esempio-Environment wird nur in der Leseansicht benötigt
\ifkorrekturansicht\else
\newenvironment{esempio}[3]%
{
    \vspace{1.5ex}
    \rlap{\underline{#1}}
    \par
    \setlength{\parindent}{0cm}
    \nopagebreak
    \leftskip=#2cm
    \rightskip=#3cm
}
{
    \par
}
\fi

\doendnotes{C}
\bigskip
\vfill

\clearpage

\footnotesize

\ifkorrekturansicht
  \lohead{\textsc{register}}
\fi

% theindex-Environment neu definieren ohne reledmac
\makeatletter
\renewenvironment{theindex}{%
  \ifkorrekturansicht
    \section*{\indexname}%
  \else
    \subsubsection*{Index der erwähnten Entitäten}%
  \fi
  \setlength{\parindent}{0pt}%
  \setlength{\parskip}{0pt plus 0.3pt}%
  \let\item\@idxitem
}{%
  \ifkorrekturansicht\clearpage\fi
}
\makeatother

\IfFileExists{\jobname-pw.ind}{\input{\jobname-pw.ind}}{}

% Quellenangabe nur in der Leseansicht
\ifkorrekturansicht\else
% Fallback-Definitionen, falls die .tex-Datei \titel etc. nicht gesetzt hat
\providecommand{\titel}{}
\providecommand{\editorInnen}{}
\providecommand{\dateiname}{\jobname}

\vspace{3cm}

\vfill

\footnotesize
\textsc{Quelle}: \titel. Herausgegeben von {\editorInnen}. In: \emph{Arthur Schnitzler: Briefwechsel mit Autorinnen und Autoren}.
 Digitale Edition, https://schnitzler-briefe.acdh.oeaw.ac.at/{\dateiname}.html (Stand \today)
\fi

\end{document}


