%% latex-korrekturansicht-vorspann.tex
%% Vorspann für die Korrekturansicht.
%% Lädt die gemeinsame Datei latex-vorspann.tex mit gesetztem Schalter.

\newif\ifkorrekturansicht
\korrekturansichttrue

\input{../tex-inputs/latex-vorspann}


\section[ Paul Goldmann an Arthur Schnitzler, 10. 2. {[}1902{]}]{L03197 Paul Goldmann an Arthur Schnitzler, 10. 2. {[}1902{]}}
\nopagebreak\mylabel{L03197v}
\rehead{ }\normalsize\beginnumbering\briefempfaengerindex{Schnitzler, Arthur@\textsc{Schnitzler, Arthur}!zzzGoldmann, Paul@\emph{von Paul Goldmann}!1902-02-101@{10. 2. {[}1902{]}}|(be}
\toendnotes[C]{\smallbreak\pagebreak[2]}\Standort{DLA, A:Schnitzler, HS.NZ85.1.3172.}
\physDesc{Brief, 1 Blatt, 3 Seiten, 1208 Zeichen
\newline{}Handschrift: blaue Tinte, deutsche Kurrent
\newline{}Beilage: ein Zeitungsausschnitt, beschnitten und eingeklebt 
\newline{}Schnitzler: 1) mit Bleistift das Jahr »902« vermerkt  2) mit rotem Buntstift eine Unterstreichung}\toendnotes[C]{\smallbreak}
\pstart
           \raggedleft{}{\pb}\textcolor{gray}{\textbf{DESSAUERSTRASSE 19}}\oindex{Dessauer Strasse@\textbf{Dessauer Straße}, \emph{Straße (K.STR)}|pw}\pend
           
\pstart
           Berlin\oindex{Berlin@\textbf{Berlin}, \emph{P.PPLC}|pw}, 10. Februar.\pend
           
\pstart{}Mein lieber Freund,\pend\vspace{0.5em}
\pstart
           Wenn ich \textsc{Arthur Schnitzler} wäre, weißt Du, was ich thäte?
               Ich hätte den Ehrgeiz, nach all’ den ſchönen literariſchen Leiſtungen auch noch eine
               menſchlich große That zu vollbringen. Und würde mich darum an die Spitze einer
               Bewegung ſtellen, die zum Zweck hätte, den \label{K_L03197-1v}\edtext{Fall \textsc{Matassich-Keglevich\pwindex{Mattachich, Geza von 1867-12-19 – 1923-09-29@\textsc{Mattachich, Géza von} (1867-12-19 – 1923-09-29), \emph{Oberstleutnant/Oberstleutnantin}|pw}}}{\lemma{\textnormal{\emph{Fall Matassich-Keglevich}}}\Cendnote{\textnormal{Oberstleutnant Géza von Mattachich\pwindex{Mattachich, Geza von 1867-12-19 – 1923-09-29@\textsc{Mattachich, Géza von} (1867-12-19 – 1923-09-29), \emph{Oberstleutnant/Oberstleutnantin}|pwk} hatte ab 1895 eine intime
                  Beziehung mit Louise von Belgien\pwindex{Louise von Belgien 1858-02-18 – 1924-03-01@\textsc{Louise von Belgien} (1858-02-18 – 1924-03-01), \emph{Prinz/Prinzessin}|pwk}. Da sie
                  als älteste Tochter annehmen konnte, nach dem Tod ihres Vaters Leopold II. von Belgien\pwindex{Leopold II. von Belgien 1835-04-09 – 1909-12-17@\textsc{Leopold II. von Belgien} (1835-04-09 – 1909-12-17), \emph{König/Königin}|pwk} ein großes Vermögen zu erben,
                  lebte sie über ihre Verhältnisse und machte Schulden. Die beiden wurden im
                     Mai 1898 in Kroatien\oindex{Kroatien@\textbf{Kroatien}, \emph{A.PCLI}|pwk}
                  verhaftet und der Geldwechselfälschung beschuldigt. Während sie in eine
                  psychiatrische Verwahrung kam, wurde er zu sechs Jahren schwerem Kerker
                  verurteilt. Am 8. 2. 1902 hatte Ignacy Daszińsky\pwindex{Daszyński, Ignacy 1866-10-26 – 1936-10-31@\textsc{Daszyński, Ignacy} (1866-10-26 – 1936-10-31), \emph{Politiker/Politikerin, Ministerpräsident/Ministerpräsidentin}|pwk} im \emph{Reichsrat}\orgindex{Reichsrat@Reichsrat|pwk} eine Rede gehalten (vgl. [O. V.]: \emph{Politische Glossen. Der Ernst der Volksvertreter}\pwindex{Politische Glossen. Der Ernst der Volksvertreter@\emph{Politische Glossen. Der Ernst der Volksvertreter}|pwk}. In:
                        \emph{Extrapost. Unparteiische
                        Montags-Zeitung}\pwindex{Extrapost. Unparteiische Montags-Zeitung@\emph{Extrapost. Unparteiische Montags-Zeitung}|pwk}. Jg. 21, Nr. 1045, 10. 2. 1902, S. 2). Noch im selben Monat wurde Mattachich\pwindex{Mattachich, Geza von 1867-12-19 – 1923-09-29@\textsc{Mattachich, Géza von} (1867-12-19 – 1923-09-29), \emph{Oberstleutnant/Oberstleutnantin}|pwk} für unschuldig erklärt und
                  begnadigt.}}}\label{K_L03197-1}, in dem ſicherlich ein gemeiner Juſtizmord verübt worden iſt,
               aufzuklären. \textsc{Zola\pwindex{Zola, Emile 02.04.1840 – 29.09.1902@\textsc{Zola, Émile} (02.04.1840 – 29.09.1902), \emph{Schriftsteller/Schriftstellerin, Journalist/Journalistin}|pw}} gibt das große Vorbild. Ein \label{K_L03197-2v}\edtext{Artikel}{\lemma{\textnormal{\emph{Artikel}}}\Cendnote{\textnormal{Schnitzler verzichtete nicht nur hier,
                  sondern zeitlebens darauf, seinen Namen für eine größere (kultur-)politische
                  Kampagne zu verwenden.}}}\label{K_L03197-2} in einem großen Wien\oindex{Wien@\textbf{Wien}, \emph{A.ADM2}|pw}er oder reichsdeutſch\oindex{Deutschland@\textbf{Deutschland}, \emph{A.PCLI}|pwv}en Blatte mit Darlegung des ganzen Materials (das ſicherlich in
                  Wien\oindex{Wien@\textbf{Wien}, \emph{A.ADM2}|pw}{ }{\pb}zu bekommen iſt, wahrſcheinlich vom Abg. \textsc{Daszinsky\pwindex{Daszyński, Ignacy 1866-10-26 – 1936-10-31@\textsc{Daszyński, Ignacy} (1866-10-26 – 1936-10-31), \emph{Politiker/Politikerin, Ministerpräsident/Ministerpräsidentin}|pw}}), mit \textsc{Arthur Schnitzlers} klangvollem Namen
               unterzeichnet, würde die Bewegung einleiten und alle empfänglichen
                  He\textcolor{gray}{r}zen in Deutſchland\oindex{Deutschland@\textbf{Deutschland}, \emph{A.PCLI}|pw} und
                  Öſterreich\oindex{Oesterreich@\textbf{Österreich}, \emph{A.PCLI}|pw} für den Fall intereſſiren.
               Vielleicht iſt die Sache in Wien\oindex{Wien@\textbf{Wien}, \emph{A.ADM2}|pw} mit der »Zeit\orgindex{Zeit. Wiener Wochenschrift@Die Zeit. Wiener Wochenschrift|pw}« zu machen. Vielleicht auch mit der N. Fr. Pr.\orgindex{Neue Freie Presse@Neue Freie Presse|pw}\pend
           
\pstart
           Wie geht es \textsc{Olga\pwindex{Schnitzler, Olga 17.01.1882 – 13.01.1970@\textsc{Schnitzler, Olga} (17.01.1882 – 13.01.1970), \emph{Schauspieler/Schauspielerin, Sänger/Sängerin}|pw}}? Seid Ihr ſchon in \label{K_L03197-3v}\edtext{\textsc{Mödling}\oindex{Moedling@\textbf{Mödling}, \emph{P.PPLA3}|pw}}{\lemma{\textnormal{\emph{Mödling}}}\Cendnote{\textnormal{Siehe Paul Goldmann an Arthur Schnitzler, 14. 1. [1902].
               }}}\label{K_L03197-3}? Herzliche Grüße an die Mädels\pwindex{Schnitzler, Olga 17.01.1882 – 13.01.1970@\textsc{Schnitzler, Olga} (17.01.1882 – 13.01.1970), \emph{Schauspieler/Schauspielerin, Sänger/Sängerin}|pwv}\pwindex{Steinrueck, Elisabeth 19.11.1885 – 07.04.1920@\textsc{Steinrück, Elisabeth} (19.11.1885 – 07.04.1920)|pwv}!\pend
           
\pstart
           Ich habe unbeſchreiblich viel zu thun.\pend
           
\pstart
           Dank für Deinen letzten lieben Brief! {\\[\baselineskip]}Viele treue Grüße! {\\[\baselineskip]}Dein
                  \spacefill\mbox{Paul Goldm}\pend
           \leftskip=0em{}
\pstart
           \noindent{}{\pb}Das Stück\pwindex{Sehnsucht@\emph{Sehnsucht}|pwv} meines Onkels\pwindex{Mamroth, Fedor 21.02.1851 – 25.06.1907@\textsc{Mamroth, Fedor} (21.02.1851 – 25.06.1907), \emph{Journalist/Journalistin, Kritiker/Kritikerin}|pwv}, das unter dem Namen »Sehnſucht\pwindex{Sehnsucht@\emph{Sehnsucht}|pw}« in \label{K_L03197-4v}\edtext{Stuttgart\oindex{Stuttgart@\textbf{Stuttgart}, \emph{P.PPLA}|pw}}{\lemma{\textnormal{\emph{Stuttgart}}}\Cendnote{\textnormal{Am 4. 2. 1902 war Fedor
                     Mamroths\pwindex{Mamroth, Fedor 21.02.1851 – 25.06.1907@\textsc{Mamroth, Fedor} (21.02.1851 – 25.06.1907), \emph{Journalist/Journalistin, Kritiker/Kritikerin}|pwk} vieraktige Komödie \emph{Sehnsucht}\pwindex{Sehnsucht@\emph{Sehnsucht}|pwk} (unter dem Pseudonym F. Albert\pwindex{Mamroth, Fedor 21.02.1851 – 25.06.1907@\textsc{Mamroth, Fedor} (21.02.1851 – 25.06.1907), \emph{Journalist/Journalistin, Kritiker/Kritikerin}|pwkv}) am Stuttgarter
                        Hoftheater\oindex{Hoftheater Stuttgart@\textbf{Hoftheater Stuttgart}, \emph{Theater (K.THE)}|pwk} uraufgeführt worden.}}}\label{K_L03197-4} aufgeführt wurde, hatte dort einen ſehr
                  ſchönen Erfolg.\pend
           
\pstart
           Wie hat ſich die \label{K_L03197-5v}\edtext{Angelegenheit \textsc{Peter Dorner\pwindex{Dorner, Peter 17.02.1857 – 01.04.1931@\textsc{Dorner, Peter} (17.02.1857 – 01.04.1931), \emph{Schmied/Schmiedin, Kunsthandwerker/Kunsthandwerkerin, Kunstschmied/Kunstschmiedin}|pw}}}{\lemma{\textnormal{\emph{Angelegenheit Peter Dorner}}}\Cendnote{\textnormal{Siehe Paul Goldmann an Arthur Schnitzler, 23. 9. [1901].
                  }}}\label{K_L03197-5} noch entwickelt?\pend
           {\vspace{1\baselineskip}}
\pstart
           \textcolor{gray}{\textbf{\textbf{– }\label{K_L03197-6v}\edtext{\textbf{Arthur Schnitzler’s »}\textbf{Lebendige Stunden}\pwindex{Lebendige Stunden. Vier Einakter@\emph{Lebendige Stunden. Vier Einakter}|pw}\textbf{«,} die bisher in zwanzig Wiederholungen bei unverminderter
                     Zugkraft im \so{Deutſchen Theater}\oindex{Deutsches Theater Berlin@\textbf{Deutsches Theater Berlin}, \emph{Theater (K.THE)}|pw} in Szene gingen, können in den folgenden Wochen nur je einmal auf dem
                     Spielplan erſcheinen, da \so{Irene Trieſch}\pwindex{Triesch, Irene 13.04.1877 – 24.11.1964@\textsc{Triesch, Irene} (13.04.1877 – 24.11.1964), \emph{Schauspieler/Schauspielerin}|pw} einen kontraktlichen Urlaub angetreten hat, jedoch allwöchentlich einmal,
                     zunächſt am Mittwoch, den 12., nach Berlin\oindex{Berlin@\textbf{Berlin}, \emph{P.PPLC}|pw} zurückkehren wird, um die von ihr in
                     den »Lebendigen Stunden\pwindex{Lebendige Stunden. Vier Einakter@\emph{Lebendige Stunden. Vier Einakter}|pw}« geſpielten beiden
                     weiblichen Hauptrollen weiterhin darzuſtellen.}{\lemma{\textnormal{\emph{Arthur … darzuſtellen.}}}\Cendnote{\textnormal{Quelle nicht ermittelt; in \emph{Die Frau mit dem Dolche}\pwindex{Frau mit dem Dolche@\emph{Die Frau mit dem Dolche}|pwk} spielte Irene Triesch\pwindex{Triesch, Irene 13.04.1877 – 24.11.1964@\textsc{Triesch, Irene} (13.04.1877 – 24.11.1964), \emph{Schauspieler/Schauspielerin}|pwk} die Rolle der Pauline\pwindex{Frau mit dem Dolche@\emph{Die Frau mit dem Dolche}|pwkv} und in \emph{Literatur}\pwindex{Literatur@\emph{Literatur}|pwk} jene der Margarete\pwindex{Literatur@\emph{Literatur}|pwkv}.}}}\label{K_L03197-6}}}\pend
           \selectlanguage{ngerman}\endnumbering\briefempfaengerindex{Schnitzler, Arthur@\textsc{Schnitzler, Arthur}!zzzGoldmann, Paul@\emph{von Paul Goldmann}!1902-02-101@{10. 2. {[}1902{]}}|)be}\mylabel{L03197h}  \normalsize

\doendnotes{C}
\bigskip
\vfill

\clearpage

\footnotesize

\lohead{\textsc{register}}

% Definiere theindex-Environment komplett neu ohne reledmac
\makeatletter
\renewenvironment{theindex}{%
  \section*{\indexname}%
  \setlength{\parindent}{0pt}%
  \setlength{\parskip}{0pt plus 0.3pt}%
  \let\item\@idxitem
}{%
  \clearpage
}
\makeatother

\IfFileExists{\jobname-pw.ind}{\input{\jobname-pw.ind}}{}

\end{document}

      