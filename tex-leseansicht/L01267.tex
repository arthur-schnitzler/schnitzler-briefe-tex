%% latex-korrekturansicht-vorspann.tex
%% Vorspann für die Korrekturansicht.
%% Lädt die gemeinsame Datei latex-vorspann.tex mit gesetztem Schalter.

\newif\ifkorrekturansicht
\korrekturansichttrue

\input{../tex-inputs/latex-vorspann}


\section[Richard Beer-Hofmann an Arthur Schnitzler, 18. 1. 1903]{L01267 Richard Beer-Hofmann an Arthur Schnitzler, 18. 1. 1903}
\nopagebreak\mylabel{L01267v}
\rehead{ }\normalsize\beginnumbering\briefempfaengerindex{Schnitzler, Arthur@\textsc{Schnitzler, Arthur}!zzzBeer-Hofmann, Richard@\emph{von Richard Beer-Hofmann}!1903-01-181@{18. 1. 1903}|(be}
\toendnotes[C]{\smallbreak\pagebreak[2]}\Standort{CUL, Schnitzler, B 8.}
\physDesc{Brief, 1 Blatt, 2 Seiten, 728 Zeichen (Briefpapier mit Trauerrand)
\newline{}Handschrift: blauer Buntstift, lateinische Kurrent
\newline{}Ordnung: mit Bleistift von unbekannter Hand nummeriert:
                                    »177« }
\buchAbdrucke{\weitereDrucke{Arthur Schnitzler, Richard Beer-Hofmann: \emph{Briefwechsel 1891–1931}. Wien, Zürich: \emph{Europaverlag} 1992, S. 160.} }\toendnotes[C]{\smallbreak}
\pstart
           \raggedleft{}{\pb}Rodaun\oindex{Rodaun@\textbf{Rodaun}, \emph{A.ADM4}|pw}{ }18/1 1903\pend
           \vspace{0.5em}
\pstart
           Lieber Arthur! Vielen Dank für Ihren Antrag. Ich kann mich aber
               nicht entschließen »\label{K_L01267-1v}\edtext{Bern}{\lemma{\textnormal{\emph{Bern}}}\Cendnote{\textnormal{Vgl. Paul Goldmann an Arthur Schnitzler, 17. 4. [1902].
               }}}\label{K_L01267-1}« in unsere
               Familie aufzunehmen. Abgesehen vom Großfolio-Format würde ich – wenn – nur einen \uline{ganz jungen} Hund wieder nehmen damit er an die Kinder,
               und sie an ihn sich gewöhnen, und ich sein Inneres von seinen ersten Lebenswochen an
               bilden kann. Jedenfalls werde ich ihm aber demnächst einen Besuch abstatten. Um Salzburg\oindex{Salzburg@\textbf{Salzburg}, \emph{A.ADM2}|pw} beneide ich Sie na{\pb}türlich. Ich arbeite (ja!) und Hugo\pwindex{Hofmannsthal, Hugo von 1874-02-01 – 1929-07-15@\textsc{Hofmannsthal, Hugo von} (1874-02-01 – 1929-07-15), \emph{Schriftsteller/Schriftstellerin}|pw} ist mit dem Flohtheater beschäftigt –
               bestehend aus Ihren – Schwarzkopfs\pwindex{Schwarzkopf, Gustav 07.11.1853 – 13.11.1939@\textsc{Schwarzkopf, Gustav} (07.11.1853 – 13.11.1939), \emph{Schriftsteller/Schriftstellerin}|pw} etc. Flöhen
               die ihm ins Ohr gesetzt wurden. Vielleicht sehe ich Sie \label{K_L01267-2v}\edtext{Samstag (24){ }Nachm. (Akad. Verein\orgindex{Akademischer Verein fuer Kunst und Literatur@Akademischer Verein für Kunst und Literatur|pw})}{\lemma{\textnormal{\emph{Samstag … Verein)}}}\Cendnote{\textnormal{Der \emph{Akademische Verein für Kunst und Literatur}\orgindex{Akademischer Verein fuer Kunst und Literatur@Akademischer Verein für Kunst und Literatur|pwk} veranstaltete im Theater an der Wien\oindex{Theater an der Wien@\textbf{Theater an der Wien}, \emph{Theater (K.THE)}|pwk} die erste Wien\oindex{Wien@\textbf{Wien}, \emph{A.ADM2}|pwk}er Inszenierung von \emph{Elpenor}\pwindex{Elpenor. Trauerspiel@\emph{Elpenor. Trauerspiel}|pwk}. Schnitzler dürfte nicht
                  teilgenommen haben. Im Original steht der »Akad. Verein« in eckigen
                  Klammern.}}}\label{K_L01267-2}. Nochmals Dank und herzliche Grüße – auch an Mutter\pwindex{Schnitzler, Olga 17.01.1882 – 13.01.1970@\textsc{Schnitzler, Olga} (17.01.1882 – 13.01.1970), \emph{Schauspieler/Schauspielerin, Sänger/Sängerin}|pwv} und Kind\pwindex{Schnitzler, Heinrich 09.08.1902 – 12.07.1982@\textsc{Schnitzler, Heinrich} (09.08.1902 – 12.07.1982), \emph{Regisseur/Regisseurin, Schauspieler/Schauspielerin}|pwv}.\pend
           
\pstart
           Ihr{\\[\baselineskip]}\spacefill\mbox{Richard}\pend
           \leftskip=0em{}
\pstart
           \noindent{}\label{K_L01267-3v}\edtext{Hallein\oindex{Hallein@\textbf{Hallein}, \emph{P.PPLA3}|pw} erhalten}{\lemma{\textnormal{\emph{Hallein erhalten}}}\Cendnote{\textnormal{Arthur Schnitzler und Olga Gussmann an Richard Beer-Hofmann,
               16. 1. 1903.
                  }}}\label{K_L01267-3}\pend
           \selectlanguage{ngerman}\endnumbering\briefempfaengerindex{Schnitzler, Arthur@\textsc{Schnitzler, Arthur}!zzzBeer-Hofmann, Richard@\emph{von Richard Beer-Hofmann}!1903-01-181@{18. 1. 1903}|)be}\mylabel{L01267h}  \normalsize

\doendnotes{C}
\bigskip
\vfill

\clearpage

\footnotesize

\lohead{\textsc{register}}

% Definiere theindex-Environment komplett neu ohne reledmac
\makeatletter
\renewenvironment{theindex}{%
  \section*{\indexname}%
  \setlength{\parindent}{0pt}%
  \setlength{\parskip}{0pt plus 0.3pt}%
  \let\item\@idxitem
}{%
  \clearpage
}
\makeatother

\IfFileExists{\jobname-pw.ind}{\input{\jobname-pw.ind}}{}

\end{document}

      