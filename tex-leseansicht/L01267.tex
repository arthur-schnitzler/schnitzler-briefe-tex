%% latex-leseansicht-vorspann.tex
%% Vorspann für die Leseansicht.
%% Lädt die gemeinsame Datei latex-vorspann.tex mit nicht gesetztem Schalter.

\newif\ifkorrekturansicht
\korrekturansichtfalse

\input{../tex-inputs/latex-vorspann}


         
         \renewcommand{\erwaehntePersonen}{Personen: Hugo von Hofmannsthal, Olga Schnitzler, Heinrich Schnitzler, Gustav Schwarzkopf}
         \renewcommand{\erwaehnteInstitutionen}{Institutionen: Akademischer Verein für Kunst und Literatur}
         \renewcommand{\erwaehnteOrte}{Orte: Hallein, Rodaun, Salzburg, Theater an der Wien, Wien}
         \renewcommand{\erwaehnteWerke}{Werke: Elpenor. Trauerspiel}
               \section[Richard Beer-Hofmann an Arthur Schnitzler, 18. 1. 1903]{ Richard Beer-Hofmann an Arthur Schnitzler, 18. 1. 1903}\nopagebreak\mylabel{v}\rehead{ }\begin{ledgroupsized}[t]{13cm}\normalsize\beginnumbering \toendnotes[C]{\smallbreak\pagebreak[2]} \Standort{CUL, Schnitzler, B 8.}
\physDesc{Brief, 1 Blatt, 2 Seiten, 728 Zeichen (Briefpapier mit Trauerrand)
\newline{}Handschrift: blauer Buntstift, lateinische Kurrent
\newline{}Ordnung: mit Bleistift von unbekannter Hand nummeriert:
                                    »177« }\buchAbdrucke{\weitereDrucke{Arthur Schnitzler, Richard Beer-Hofmann: \emph{Briefwechsel 1891–1931}. Hg. Konstanze Fliedl. Wien, Zürich: \emph{Europaverlag} 1992, S. 160.} }\toendnotes[C]{\smallbreak}\pstart
           \raggedleft{}{\pb}Rodaun\oindex{Rodaun@\textbf{Rodaun}|pw}{ }18/1 1903\pend
           \pstart
           Lieber Arthur! Vielen Dank für Ihren Antrag. Ich kann mich aber
               nicht entschließen »\label{K_L01267_1v}\edtext{Bern}{\lemma{\textnormal{\emph{Bern}}}\Cendnote{\textnormal{den Bernhardinerhund}}}\label{K_L01267_1h}« in unsere
               Familie aufzunehmen. Abgesehen vom Großfolio-Format würde ich – wenn – nur einen \uline{ganz jungen} Hund wieder nehmen damit er an die Kinder,
               und sie an ihn sich gewöhnen, und ich sein Inneres von seinen ersten Lebenswochen an
               bilden kann. Jedenfalls werde ich ihm aber demnächst einen Besuch abstatten. Um Salzburg\oindex{Salzburg@\textbf{Salzburg}|pw} beneide ich Sie na{\pb}türlich. Ich arbeite (ja!) und Hugo\pwindex{Hofmannsthal, Hugo von 1874-02-01 – 1929-07-15@\textsc{Hofmannsthal, Hugo von} (1874-02-01 – 1929-07-15), \emph{Schriftsteller}|pw} ist mit dem Flohtheater beschäftigt –
               bestehend aus Ihren – Schwarzkopfs\pwindex{Schwarzkopf, Gustav 07.11.1853 – 13.11.1939@\textsc{Schwarzkopf, Gustav} (07.11.1853 – 13.11.1939), \emph{Schriftsteller}|pw} etc. Flöhen
               die ihm ins Ohr gesetzt wurden. Vielleicht sehe ich Sie \label{K_L01267_2v}\edtext{Samstag (24){ }Nachm. (Akad. Verein\orgindex{Akademischer Verein fuer Kunst und Literatur@Akademischer Verein für Kunst und Literatur|pw})}{\lemma{\textnormal{\emph{Samstag … Verein)}}}\Cendnote{\textnormal{Der \emph{Akademische Verein für Kunst und Literatur}\orgindex{Akademischer Verein fuer Kunst und Literatur@Akademischer Verein für Kunst und Literatur|pwk} veranstaltete im Theater an der Wien\oindex{Theater an der Wien@\textbf{Theater an der Wien}|pwk} die erste Wien\oindex{Wien@\textbf{Wien}|pwk}er Inszenierung von \emph{Elpenor}\pwindex{\textcolor{red}{\textsuperscript{XXXX1 indx}}!Elpenor. Trauerspiel1806@\strich\emph{Elpenor. Trauerspiel} {[}1806{]}|pwk}. Schnitzler\pwindex{Schnitzler, Arthur 15.05.1862 – 21.10.1931@\textsc{Schnitzler, Arthur} (15.05.1862 – 21.10.1931), \emph{Schriftsteller, Mediziner}|pwk} dürfte nicht
                  teilgenommen haben. Im Original steht der »Akad. Verein« in eckigen
                  Klammern.}}}\label{K_L01267_2h}. Nochmals Dank und herzliche Grüße – auch an Mutter\pwindex{Schnitzler, Olga 17.01.1882 – 13.01.1970@\textsc{Schnitzler, Olga} (17.01.1882 – 13.01.1970), \emph{Schauspielerin, Sängerin}|pwv} und Kind\pwindex{Schnitzler, Heinrich 09.08.1902 – 12.07.1982@\textsc{Schnitzler, Heinrich} (09.08.1902 – 12.07.1982), \emph{Regisseur, Schauspieler}|pwv}.\pend
           \pstart
           Ihr{\\[\baselineskip]}\spacefill\mbox{Richard}\pend
           \leftskip=0em{}\pstart
           \noindent{}\label{K_L01267_3v}\edtext{Hallein\oindex{Hallein@\textbf{Hallein}|pw} erhalten}{\lemma{\textnormal{\emph{Hallein erhalten}}}\Cendnote{\textnormal{Arthur Schnitzler und Olga Gussmann an Richard Beer-Hofmann,
               16. 1. 1903}}}\label{K_L01267_3h}\pend
           
         
         \endnumbering\mylabel{h}\end{ledgroupsized}  \newcommand{\dateiname}{L01267}\newcommand{\titel}{Richard Beer-Hofmann an Arthur Schnitzler, 18. 1. 1903}\newcommand{\editorInnen}{Martin Anton Müller und Gerd-Hermann Susen}%% latex-leseansicht-abspann.tex
%% Abspann für die Leseansicht.
%% Der Schalter \ifkorrekturansicht ist bereits durch den Vorspann gesetzt.

%% latex-abspann.tex
%% Gemeinsamer Abspann für Korrekturansicht und Leseansicht.
%% Setzt den Schalter \ifkorrekturansicht voraus (gesetzt in den
%% einbindenden Dateien latex-korrekturansicht-abspann.tex bzw.
%% latex-leseansicht-abspann.tex).
%% ---------------------------------------------------------------

\normalsize

% Das esempio-Environment wird nur in der Leseansicht benötigt
\ifkorrekturansicht\else
\newenvironment{esempio}[3]%
{
    \vspace{1.5ex}
    \rlap{\underline{#1}}
    \par
    \setlength{\parindent}{0cm}
    \nopagebreak
    \leftskip=#2cm
    \rightskip=#3cm
}
{
    \par
}
\fi

\doendnotes{C}
\bigskip
\vfill

\clearpage

\footnotesize

\ifkorrekturansicht
  \lohead{\textsc{register}}
\fi

% theindex-Environment neu definieren ohne reledmac
\makeatletter
\renewenvironment{theindex}{%
  \ifkorrekturansicht
    \section*{\indexname}%
  \else
    \subsubsection*{Index der erwähnten Entitäten}%
  \fi
  \setlength{\parindent}{0pt}%
  \setlength{\parskip}{0pt plus 0.3pt}%
  \let\item\@idxitem
}{%
  \ifkorrekturansicht\clearpage\fi
}
\makeatother

\IfFileExists{\jobname-pw.ind}{\input{\jobname-pw.ind}}{}

% Quellenangabe nur in der Leseansicht
\ifkorrekturansicht\else
% Fallback-Definitionen, falls die .tex-Datei \titel etc. nicht gesetzt hat
\providecommand{\titel}{}
\providecommand{\editorInnen}{}
\providecommand{\dateiname}{\jobname}

\vspace{3cm}

\vfill

\footnotesize
\textsc{Quelle}: \titel. Herausgegeben von {\editorInnen}. In: \emph{Arthur Schnitzler: Briefwechsel mit Autorinnen und Autoren}.
 Digitale Edition, https://schnitzler-briefe.acdh.oeaw.ac.at/{\dateiname}.html (Stand \today)
\fi

\end{document}


      