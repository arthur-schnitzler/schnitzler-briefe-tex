%% latex-leseansicht-vorspann.tex
%% Vorspann für die Leseansicht.
%% Lädt die gemeinsame Datei latex-vorspann.tex mit nicht gesetztem Schalter.

\newif\ifkorrekturansicht
\korrekturansichtfalse

\input{../tex-inputs/latex-vorspann}


\section[Richard Beer-Hofmann an Arthur Schnitzler, 18. 1. 1903]{L01267 Richard Beer-Hofmann an Arthur Schnitzler, 18. 1. 1903}
\nopagebreak\mylabel{L01267v}
\rehead{ }\normalsize\beginnumbering\briefempfaengerindex{Schnitzler, Arthur@\textsc{Schnitzler, Arthur}!zzzBeer-Hofmann, Richard@\emph{von Richard Beer-Hofmann}!1903-01-181@{18. 1. 1903}|(be}
\toendnotes[C]{\smallbreak\pagebreak[2]}
\correspDesc{Versand  durch Richard Beer-Hofmann am 18. 1. 1903 in Rodaun
\newline{}Erhalt  durch Arthur Schnitzler im Zeitraum [18. 1. 1903
                  – 22. 1. 1903?] \textbf{Ort fehlend} }\toendnotes[C]{\smallbreak}
\Standort{CUL, Schnitzler, B 8.}
\physDesc{Brief, 1 Blatt, 2 Seiten, 728 Zeichen (Briefpapier mit Trauerrand)
\newline{}Handschrift: blauer Buntstift, lateinische Kurrent
\newline{}Ordnung: mit Bleistift von unbekannter Hand nummeriert:
                                    »177« }
\buchAbdrucke{\weitereDrucke{Arthur Schnitzler, Richard Beer-Hofmann: \emph{Briefwechsel 1891–1931}. Herausgegeben von Konstanze Fliedl. Wien, Zürich: \emph{Europaverlag} 1992, S. 160.} }\toendnotes[C]{\smallbreak}
\pstart
           \raggedleft{}{\pb}Rodaun\oindex{Wien@\textbf{Wien}!XXIII., Liesing@\textbf{XXIII., Liesing}!Rodaun@\textbf{Rodaun}, \emph{Region}|pw}{ }18/1 1903\pend
           \vspace{0.5em}
\pstart
           Lieber Arthur! Vielen Dank für Ihren Antrag. Ich kann mich aber
               nicht entschließen »\label{K_L01267-1v}\edtext{Bern}{\lemma{\textnormal{\emph{Bern}}}\Cendnote{\textnormal{Vgl. XXXX Auszeichnungsfehler: Dokument L03204 nicht gefunden.
               }}}\label{K_L01267-1}« in unsere
               Familie aufzunehmen. Abgesehen vom Großfolio-Format würde ich – wenn – nur einen \uline{ganz jungen} Hund wieder nehmen damit er an die Kinder,
               und sie an ihn sich gewöhnen, und ich sein Inneres von seinen ersten Lebenswochen an
               bilden kann. Jedenfalls werde ich ihm aber demnächst einen Besuch abstatten. Um Salzburg\oindex{Salzburg@\textbf{Salzburg}, \emph{Verwaltungsgebiet}|pw} beneide ich Sie na{\pb}türlich. Ich arbeite (ja!) und Hugo\pwindex{Hofmannsthal, Hugo von 1.\,2.\,1874 Wien – 15.\,7.\,1929 Rodaun@\textsc{Hofmannsthal, Hugo von} (1.\,2.\,1874 Wien – 15.\,7.\,1929 Rodaun), \emph{Schriftsteller}|pw} ist mit dem Flohtheater beschäftigt –
               bestehend aus Ihren – Schwarzkopfs\pwindex{Schwarzkopf, Gustav 7.\,11.\,1853 Wien – 13.\,11.\,1939 ebd.@\textsc{Schwarzkopf, Gustav} (7.\,11.\,1853 Wien – 13.\,11.\,1939 ebd.), \emph{Schriftsteller}|pw} etc. Flöhen
               die ihm ins Ohr gesetzt wurden. Vielleicht sehe ich Sie \label{K_L01267-2v}\edtext{Samstag (24){ }Nachm. (Akad. Verein\orgindex{Akademischer Verein für Kunst und Literatur@Akademischer Verein für Kunst und Literatur|pw})}{\lemma{\textnormal{\emph{Samstag … Verein)}}}\Cendnote{\textnormal{Der \emph{Akademische Verein für Kunst und Literatur}\orgindex{Akademischer Verein für Kunst und Literatur@Akademischer Verein für Kunst und Literatur|pwk} veranstaltete im Theater an der Wien\oindex{Wien@\textbf{Wien}!VI., Mariahilf@\textbf{VI., Mariahilf}!Theater an der Wien@\textbf{Theater an der Wien}, \emph{Theater}|pwk} die erste Wien\oindex{Wien@\textbf{Wien}, \emph{Verwaltungsgebiet}|pwk}er Inszenierung von \emph{Elpenor}\pwindex{\textcolor{red}{\textsuperscript{XXXX indx1}}!Elpenor. Trauerspiel@\strich\emph{Elpenor. Trauerspiel}|pwk}. Schnitzler dürfte nicht
                  teilgenommen haben. Im Original steht der »Akad. Verein« in eckigen
                  Klammern.}}}\label{K_L01267-2}. Nochmals Dank und herzliche Grüße – auch an Mutter\pwindex{Schnitzler, Olga 17.\,1.\,1882 Wien – 13.\,1.\,1970 Lugano@\textsc{Schnitzler, Olga} (17.\,1.\,1882 Wien – 13.\,1.\,1970 Lugano), \emph{Schauspielerin, Sängerin}|pwv} und Kind\pwindex{Schnitzler, Heinrich 9.\,8.\,1902 Hinterbrühl – 12.\,7.\,1982 Wien@\textsc{Schnitzler, Heinrich} (9.\,8.\,1902 Hinterbrühl – 12.\,7.\,1982 Wien), \emph{Regisseur, Schauspieler}|pwv}.\pend
           
\pstart
           Ihr{\\[\baselineskip]}\spacefill\mbox{Richard}\pend
           \leftskip=0em{}
\pstart
           \noindent{}\label{K_L01267-3v}\edtext{Hallein\oindex{Hallein@\textbf{Hallein}, \emph{Hauptstadt}|pw} erhalten}{\lemma{\textnormal{\emph{Hallein erhalten}}}\Cendnote{\textnormal{XXXX Auszeichnungsfehler: Dokument L01266 nicht gefunden.
                  }}}\label{K_L01267-3}\pend
           \selectlanguage{ngerman}\endnumbering\briefempfaengerindex{Schnitzler, Arthur@\textsc{Schnitzler, Arthur}!zzzBeer-Hofmann, Richard@\emph{von Richard Beer-Hofmann}!1903-01-181@{18. 1. 1903}|)be}\mylabel{L01267h}  \newcommand{\dateiname}{L01267}\newcommand{\titel}{Richard Beer-Hofmann an Arthur Schnitzler, 18. 1. 1903}\newcommand{\editorInnen}{Martin Anton Müller und Gerd-Hermann Susen}%% latex-leseansicht-abspann.tex
%% Abspann für die Leseansicht.
%% Der Schalter \ifkorrekturansicht ist bereits durch den Vorspann gesetzt.

%% latex-abspann.tex
%% Gemeinsamer Abspann für Korrekturansicht und Leseansicht.
%% Setzt den Schalter \ifkorrekturansicht voraus (gesetzt in den
%% einbindenden Dateien latex-korrekturansicht-abspann.tex bzw.
%% latex-leseansicht-abspann.tex).
%% ---------------------------------------------------------------

\normalsize

% Das esempio-Environment wird nur in der Leseansicht benötigt
\ifkorrekturansicht\else
\newenvironment{esempio}[3]%
{
    \vspace{1.5ex}
    \rlap{\underline{#1}}
    \par
    \setlength{\parindent}{0cm}
    \nopagebreak
    \leftskip=#2cm
    \rightskip=#3cm
}
{
    \par
}
\fi

\doendnotes{C}
\bigskip
\vfill

\clearpage

\footnotesize

\ifkorrekturansicht
  \lohead{\textsc{register}}
\fi

% theindex-Environment neu definieren ohne reledmac
\makeatletter
\renewenvironment{theindex}{%
  \ifkorrekturansicht
    \section*{\indexname}%
  \else
    \subsubsection*{Index der erwähnten Entitäten}%
  \fi
  \setlength{\parindent}{0pt}%
  \setlength{\parskip}{0pt plus 0.3pt}%
  \let\item\@idxitem
}{%
  \ifkorrekturansicht\clearpage\fi
}
\makeatother

\IfFileExists{\jobname-pw.ind}{\input{\jobname-pw.ind}}{}

% Quellenangabe nur in der Leseansicht
\ifkorrekturansicht\else
% Fallback-Definitionen, falls die .tex-Datei \titel etc. nicht gesetzt hat
\providecommand{\titel}{}
\providecommand{\editorInnen}{}
\providecommand{\dateiname}{\jobname}

\vspace{3cm}

\vfill

\footnotesize
\textsc{Quelle}: \titel. Herausgegeben von {\editorInnen}. In: \emph{Arthur Schnitzler: Briefwechsel mit Autorinnen und Autoren}.
 Digitale Edition, https://schnitzler-briefe.acdh.oeaw.ac.at/{\dateiname}.html (Stand \today)
\fi

\end{document}


