%% latex-korrekturansicht-vorspann.tex
%% Vorspann für die Korrekturansicht.
%% Lädt die gemeinsame Datei latex-vorspann.tex mit gesetztem Schalter.

\newif\ifkorrekturansicht
\korrekturansichttrue

\input{../tex-inputs/latex-vorspann}


\section[Arthur Schnitzler an Richard Beer-Hofmann, 24. 6. 1895]{L00459 Arthur Schnitzler an Richard Beer-Hofmann, 24. 6. 1895}
\nopagebreak\mylabel{L00459v}
\rehead{ }\normalsize\beginnumbering\briefempfaengerindex{Beer-Hofmann, Richard@\textsc{Beer-Hofmann, Richard}!zzzSchnitzler, Arthur@\emph{von Arthur Schnitzler}!1895-06-241@{24. 6. 1895}|(be}
\toendnotes[C]{\smallbreak\pagebreak[2]}\Standort{YCGL, MSS 31.}
\physDesc{Brief, 2 Blätter, 8 Seiten, Umschlag, 1860 Zeichen
\newline{}Handschrift: 1) Bleistift, deutsche Kurrent\hspace{1em}2) schwarze Tinte, deutsche Kurrent (\noindent{}Umschlag)\hspace{1em}
\newline{}Versand: 1) Stempel: »\nobreak{}\oindex{I., Innere Stadt@\textbf{I., Innere Stadt}, \emph{A.ADM3}|pwk}Wien 1/1, 24. 6. 95, 9–10 N\nobreak{}«.   2) Stempel: »\nobreak{}\oindex{Cáslav@\textbf{Čáslav}, \emph{P.PPL}|pwk}Časlau, 25 6 95\nobreak{}«. }
\buchAbdrucke{\weitereDrucke{Arthur Schnitzler, Richard Beer-Hofmann: \emph{Briefwechsel 1891–1931}. Wien, Zürich: \emph{Europaverlag} 1992, S. 76–77.} }\toendnotes[C]{\smallbreak}\pstart{}{\pb}Herrn n. a. Lieutenant\pend{}\pstart{}\textsc{Dr. Richard Beer Hofmann}\pend{}\pstart{}im k.k. Landw Inf Regimt.\pend{}\pstart{}\textsc{Caslau\oindex{Cáslav@\textbf{Čáslav}, \emph{P.PPL}|pw} Nr 12}\pend{}{\bigskip}\vspace{1em}
\pstart
           \noindent{}{\pb}Lieber Richard. Ich freue mich ſehr, daſs ich Sie noch in Wien\oindex{Wien@\textbf{Wien}, \emph{A.ADM2}|pw}{ }ſehen werde. – \textsc{Nobl}\pwindex{Nobl, Gabor 12.10.1864 – 14.03.1938@\textsc{Nobl, Gabor} (12.10.1864 – 14.03.1938), \emph{Mediziner/Medizinerin, Dermatologe/Dermatologin}|pw}{ }ſprach ich vorgeſtern, er hat, »angeregt« durch
                  Ihr\introOben{}e\introOben{} perſönlich\textcolor{gray}{e}{ }\substVorne{}\textsuperscript{\textcolor{gray}{Epiſödchen}}\substDazwischen{}Beka{\geminationn}tſchaft\substHinten{}, das Kind\pwindex{Kind@\emph{Das Kind}|pw} geleſen. Sie werden erſucht,
               ſich nächſtens auf {\pb}gefahrloſere Weiſe Leſer zu
               verſchaffen. – Habe heute Kopfweh, nach einer »\so{un}gemeinen«
               Landpartie die ich geſtern gemacht und die – entſchuldigen – in zwei miſerabeln
               Betten einer niederoeſterreichiſchen
                  Stadt\oindex{Klosterneuburg@\textbf{Klosterneuburg}, \emph{P.PPLA3}|pwv} endete.\pend
           
\pstart
           – Von der \textsc{Lou Salomé}\pwindex{Andreas-Salome, Lou 12.02.1861 – 05.02.1937@\textsc{Andreas-Salomé, Lou} (12.02.1861 – 05.02.1937), \emph{Schriftsteller/Schriftstellerin}|pw} ha\textcolor{gray}{b} ich {\pb}noch i{\geminationm}er gar nichts gehört. Sie? – Wie wird es mit Kopenhagen\oindex{Kopenhagen@\textbf{Kopenhagen}, \emph{P.PPLC}|pw}{ }ſein? – Auch von \textsc{Paul}\pwindex{Goldmann, Paul 31.01.1865 – 25.09.1935@\textsc{Goldmann, Paul} (31.01.1865 – 25.09.1935), \emph{Schriftsteller/Schriftstellerin, Journalist/Journalistin}|pw} iſt noch nichts Definitives herauszubeko{\geminationm}en. –
                  Ke{\geminationn}en Sie den Briefwechſel \textsc{Lessing\pwindex{Lessing, Gotthold Ephraim 22.01.1729 – 15.02.1781@\textsc{Lessing, Gotthold Ephraim} (22.01.1729 – 15.02.1781), \emph{Schriftsteller/Schriftstellerin, Kritiker/Kritikerin, Philosoph/Philosophin}|pw} – Eva König\pwindex{Koenig, Eva 22.03.1736 – 10.01.1778@\textsc{König, Eva} (22.03.1736 – 10.01.1778)|pw}}\pwindex{Lessings Briefwechsel mit seiner Frau@\emph{Lessings Briefwechsel mit seiner Frau}|pwv}. Er iſt nicht ſehr intereſſant. Merkwürdig nur, wie ſie ſich i{\geminationm}er über Lotterienu{\geminationm}ern {\pb}berathen. – Leſen Sie den \textsc{Candide}\pwindex{Candide oder der Optimismus@\emph{Candide oder der Optimismus}|pw}. – Hingegen weniger nothwendig das »Gelächter\pwindex{Gelaechter@\emph{Gelächter}|pw}« von Dörmann\pwindex{Doermann, Felix 29.05.1870 – 26.10.1928@\textsc{Dörmann, Felix} (29.05.1870 – 26.10.1928), \emph{Schriftsteller/Schriftstellerin}|pw}. – Ich übe mich
               in erzählender Proſa: Schreibe »Hiſtorietten« – we{\geminationn} Sie
               wollen. Ja, den alten Dichter\pwindex{Spaeter Ruhm@\emph{Später Ruhm}|pw} hab ich erheblich
               geſtrichen; ich find ihn aber noch i{\geminationm}er {\pb}etwas langweilig. Die ſtiliſtiſchen Schlampereien
               (»ich bin erschrocken«) ſind wohl alle draußen. –\pend
           
\pstart
           – Für Iſchl\oindex{Bad Ischl@\textbf{Bad Ischl}, \emph{P.PPL}|pw} hab ich literariſch gute Hoffnungen –
               möchte mein Stück\pwindex{Liebelei. Schauspiel in drei Akten@\emph{Liebelei. Schauspiel in drei Akten}|pwv} gern
               beenden. – Von Dörmann\pwindex{Doermann, Felix 29.05.1870 – 26.10.1928@\textsc{Dörmann, Felix} (29.05.1870 – 26.10.1928), \emph{Schriftsteller/Schriftstellerin}|pw}{ }ſoll dort ein Einakter gegeben werden, den er mir
               auch zum leſen gegeben hat u über den ich {\pb}eigentlich
               nicht ſprechen darf. (»Auch von Frl. Albrecht\pwindex{Albrecht 1895 – 1895@\textsc{Albrecht} (1895 – 1895), \emph{Schauspieler/Schauspielerin}|pw}
               müſſen wir einige freundliche Worte sagen.«) – Er heißt »Der Eisbrecher\pwindex{Eisbrecher@\emph{Der Eisbrecher}|pw}«. – Jo. –\pend
           
\pstart
           – Hugo\pwindex{Hofmannsthal, Hugo von 1874-02-01 – 1929-07-15@\textsc{Hofmannsthal, Hugo von} (1874-02-01 – 1929-07-15), \emph{Schriftsteller/Schriftstellerin}|pw} war geſtern in Wien\oindex{Wien@\textbf{Wien}, \emph{A.ADM2}|pw}, ich hab ihn verſäumt. – Heut bin ich braver Sohn und hole
                  Mama\pwindex{Schnitzler, Louise 1840-07-08 – 1911-09-09@\textsc{Schnitzler, Louise} (1840-07-08 – 1911-09-09)|pwv} von der Bahn
               ab. –\pend
           
\pstart
           – In dieſem Augenblick {\pb}ſitzt der Schreiber\pwindex{?? [Schreibkraft fuer Arthur Schnitzler] @\textsc{?? [Schreibkraft für Arthur Schnitzler]}|pwv} im Nebenzi{\geminationm}er u paginirt den alten
                  Dichter\pwindex{Spaeter Ruhm@\emph{Später Ruhm}|pw}.\pend
           
\pstart
           Leben Sie wohl und nehmen Sie von Ihrer schönen Arbeitsſehnſucht recht viel ins Civil
               herüber. So kö{\geminationn}ten Sie z. B. den Götterliebling\pwindex{Tod Georgs@\emph{Der Tod Georgs}|pw} zu Ende ſchreiben. Finden Sie nicht? – Viele {\pb}herzliche Grüße\pend
           \pstart Ihr \spacefill\mbox{Arthur}\pend{}
\pstart
           24/6 95.\pend
           \selectlanguage{ngerman}\endnumbering\briefempfaengerindex{Beer-Hofmann, Richard@\textsc{Beer-Hofmann, Richard}!zzzSchnitzler, Arthur@\emph{von Arthur Schnitzler}!1895-06-241@{24. 6. 1895}|)be}\mylabel{L00459h}  \normalsize

\doendnotes{C}
\bigskip
\vfill

\clearpage

\footnotesize

\lohead{\textsc{register}}

% Definiere theindex-Environment komplett neu ohne reledmac
\makeatletter
\renewenvironment{theindex}{%
  \section*{\indexname}%
  \setlength{\parindent}{0pt}%
  \setlength{\parskip}{0pt plus 0.3pt}%
  \let\item\@idxitem
}{%
  \clearpage
}
\makeatother

\IfFileExists{\jobname-pw.ind}{\input{\jobname-pw.ind}}{}

\end{document}

      