%% latex-leseansicht-vorspann.tex
%% Vorspann für die Leseansicht.
%% Lädt die gemeinsame Datei latex-vorspann.tex mit nicht gesetztem Schalter.

\newif\ifkorrekturansicht
\korrekturansichtfalse

\input{../tex-inputs/latex-vorspann}


               \section[Arthur Schnitzler an Richard Beer-Hofmann, 24. 6. 1895]{ Arthur Schnitzler an Richard Beer-Hofmann,
               24. 6. 1895}\nopagebreak\mylabel{v}\rehead{ }\begin{ledgroupsized}[t]{13cm}\normalsize\beginnumbering\briefempfaengerindex{Beer-Hofmann, Richard@\textsc{Beer-Hofmann, Richard}!zzzSchnitzler, Arthur@\emph{von Arthur Schnitzler}!1895-06-241@{24. 6. 1895}|(be} \toendnotes[C]{\smallbreak\pagebreak[2]} \Standort{YCGL, MSS 31.}
\physDesc{Brief, 2 Blätter, 8 Seiten, Umschlag
\newline{}Handschrift: 1) Bleistift, deutsche Kurrent\hspace{1em}2) schwarze Tinte, deutsche Kurrent (\noindent{}Umschlag)\hspace{1em}\newline{}Versand: 1) Stempel: »\nobreak{}\oindex{I., Innere Stadt@\textbf{I., Innere Stadt}|pwk}Wien 1/1, 24. 6. 95, 9–10 N\nobreak{}«.  2) Stempel: »\nobreak{}\oindex{Caslau@\textbf{Caslau}|pwk}Časlau, 25 6 95\nobreak{}«. }\buchAbdrucke{\weitereDrucke{Arthur Schnitzler, Richard Beer-Hofmann: \emph{Briefwechsel 1891–1931}. Hg. Konstanze Fliedl. Wien, Zürich: \emph{Europaverlag} 1992, S. 76–77.} }\toendnotes[C]{\smallbreak}\pstart{}{\pb}Herrn n. a. Lieutenant\pend{}\pstart{}\textsc{Dr. Richard Beer Hofmann}\pend{}\pstart{}im k.k. Landw Inf Regimt.\pend{}\pstart{}\textsc{Caslau\oindex{Caslau@\textbf{Caslau}|pw} Nr 12}\pend{}{\bigskip}\pstart
           \noindent{}{\pb}Lieber Richard. Ich freue mich ſehr, daſs ich Sie noch in Wien\oindex{Wien@\textbf{Wien}|pw}{ }ſehen werde. – \textsc{Nobl}\pwindex{Nobl, Gabor 12.10.1864 – 14.03.1938@\textsc{Nobl, Gabor} (12.10.1864 – 14.03.1938), \emph{Mediziner, Dermatologe}|pw}{ }ſprach ich vorgeſtern, er hat, »angeregt« durch Ihr\introOben{}e\introOben{}
                  perſönlich\textcolor{gray}{e}{ }\substVorne{}\textsuperscript{\textcolor{gray}{Epiſödchen}}{\allowbreak}\substDazwischen{}Beka{\geminationn}tſchaft\substHinten{}, das Kind\pwindex{Beer-Hofmann, Richard 11.07.1866 – 26.09.1945@\textsc{Beer-Hofmann, Richard} (11.07.1866 – 26.09.1945), \emph{Schriftsteller}!Kind1893@\strich\emph{Das Kind} {[}1893{]}|pw} geleſen. Sie werden erſucht,
               ſich nächſtens auf {\pb}gefahrloſere Weiſe Leſer zu
               verſchaffen. – Habe heute Kopfweh, nach einer »\so{un}gemeinen« Landpartie die ich geſtern gemacht und die – entſchuldigen – in
               zwei miſerabeln Betten einer niederoeſterreichiſchen Stadt\oindex{Klosterneuburg@\textbf{Klosterneuburg}|pwv} endete.\pend
           \pstart
           – Von der \textsc{Lou Salomé}\pwindex{Andreas-Salome, Lou 12.02.1861 – 05.02.1937@\textsc{Andreas-Salomé, Lou} (12.02.1861 – 05.02.1937), \emph{Schriftstellerin}|pw} ha\textcolor{gray}{b} ich {\pb}noch i{\geminationm}er gar nichts gehört. Sie? – Wie wird es mit Kopenhagen\oindex{Kopenhagen@\textbf{Kopenhagen}|pw}{ }ſein? – Auch von \textsc{Paul}\pwindex{Goldmann, Paul 31.01.1865 – 25.09.1935@\textsc{Goldmann, Paul} (31.01.1865 – 25.09.1935), \emph{Schriftsteller, Journalist}|pw} iſt noch nichts Definitives
                  herauszubeko{\geminationm}en. – Ke{\geminationn}en Sie den Briefwechſel \textsc{Lessing\pwindex{Lessing, Gotthold Ephraim 22.01.1729 – 15.02.1781@\textsc{Lessing, Gotthold Ephraim} (22.01.1729 – 15.02.1781), \emph{Schriftsteller, Bibliothekar}|pw} – Eva König\pwindex{Koenig, Eva 22.03.1736 – 10.01.1778@\textsc{König, Eva} (22.03.1736 – 10.01.1778)|pw}}\pwindex{Lessing, Gotthold Ephraim 22.01.1729 – 15.02.1781@\textsc{Lessing, Gotthold Ephraim} (22.01.1729 – 15.02.1781), \emph{Schriftsteller, Bibliothekar}!Lessings Briefwechsel mit seiner Frau1789 – 1789@\strich\emph{Lessings Briefwechsel mit seiner Frau} {[}1789 – 1789{]}|pwv}\pwindex{Koenig, Eva 22.03.1736 – 10.01.1778@\textsc{König, Eva} (22.03.1736 – 10.01.1778)!Lessings Briefwechsel mit seiner Frau1789 – 1789@\strich\emph{Lessings Briefwechsel mit seiner Frau} {[}1789 – 1789{]}|pwv}. Er iſt nicht ſehr intereſſant. Merkwürdig nur, wie ſie ſich i{\geminationm}er über Lotterienu{\geminationm}ern {\pb}berathen. – Leſen Sie den \textsc{Candide}\pwindex{Candide oder der Optimismus1759 – 1759@\emph{Candide oder der Optimismus} {[}1759 – 1759{]}|pw}. – Hingegen weniger
               nothwendig das »Gelächter\pwindex{Doermann, Felix 29.05.1870 – 26.10.1928@\textsc{Dörmann, Felix} (29.05.1870 – 26.10.1928), \emph{Schriftsteller}!Gelaechter1895@\strich\emph{Gelächter} {[}1895{]}|pw}« von Dörmann\pwindex{Doermann, Felix 29.05.1870 – 26.10.1928@\textsc{Dörmann, Felix} (29.05.1870 – 26.10.1928), \emph{Schriftsteller}|pw}. – Ich übe mich in erzählender Proſa: Schreibe
               »Hiſtorietten« – we{\geminationn}
               Sie wollen. Ja, den alten Dichter\pwindex{Schnitzler, Arthur 15.05.1862 – 21.10.1931@\textsc{Schnitzler, Arthur} (15.05.1862 – 21.10.1931), \emph{Schriftsteller, Mediziner}!Spaeter Ruhm2014@\strich\emph{Später Ruhm} {[}2014{]}|pw} hab ich erheblich geſtrichen; ich find
               ihn aber noch i{\geminationm}er {\pb}etwas langweilig. Die ſtiliſtiſchen Schlampereien (»ich bin erschrocken«) ſind wohl
               alle draußen. –\pend
           \pstart
           – Für Iſchl\oindex{Bad Ischl@\textbf{Bad Ischl}|pw} hab ich literariſch gute Hoffnungen –
               möchte mein Stück\pwindex{Schnitzler, Arthur 15.05.1862 – 21.10.1931@\textsc{Schnitzler, Arthur} (15.05.1862 – 21.10.1931), \emph{Schriftsteller, Mediziner}!Liebelei. Schauspiel in drei Akten9. 10. 1895@\strich\emph{Liebelei. Schauspiel in drei Akten} {[}9. 10. 1895{]}|pwv} gern
               beenden. – Von Dörmann\pwindex{Doermann, Felix 29.05.1870 – 26.10.1928@\textsc{Dörmann, Felix} (29.05.1870 – 26.10.1928), \emph{Schriftsteller}|pw}{ }ſoll dort ein Einakter
               gegeben werden, den er mir auch zum leſen gegeben hat u über den ich {\pb}eigentlich nicht ſprechen darf. (»Auch von Frl.
                  Albrecht\pwindex{Albrecht 1895 – 1895@\textsc{Albrecht} (1895 – 1895), \emph{Schauspielerin}|pw} müſſen wir einige freundliche Worte
               sagen.«) – Er heißt »Der Eisbrecher\pwindex{Doermann, Felix 29.05.1870 – 26.10.1928@\textsc{Dörmann, Felix} (29.05.1870 – 26.10.1928), \emph{Schriftsteller}!Eisbrecher1895@\strich\emph{Der Eisbrecher} {[}1895{]}|pw}\pwindex{\textcolor{red}{\textsuperscript{XXXX1 indx}}!Eisbrecher1895@\strich\emph{Der Eisbrecher} {[}1895{]}|pw}«. –
               Jo. –\pend
           \pstart
           – Hugo\pwindex{Hofmannsthal, Hugo von 01.02.1874 – 15.07.1929@\textsc{Hofmannsthal, Hugo von} (01.02.1874 – 15.07.1929), \emph{Schriftsteller}|pw} war geſtern in Wien\oindex{Wien@\textbf{Wien}|pw}, ich hab ihn verſäumt. – Heut bin ich braver Sohn und
               hole Mama\pwindex{Schnitzler, Louise 08.07.1840 – 09.09.1911@\textsc{Schnitzler, Louise} (08.07.1840 – 09.09.1911)|pwv} von der Bahn
               ab. –\pend
           \pstart
           – In dieſem Augenblick {\pb}ſitzt der Schreiber\pwindex{?? [Schreibkraft fuer Arthur Schnitzler] @\textsc{?? [Schreibkraft für Arthur Schnitzler]}|pwv} im Nebenzi{\geminationm}er u paginirt den alten
                  Dichter\pwindex{Schnitzler, Arthur 15.05.1862 – 21.10.1931@\textsc{Schnitzler, Arthur} (15.05.1862 – 21.10.1931), \emph{Schriftsteller, Mediziner}!Spaeter Ruhm2014@\strich\emph{Später Ruhm} {[}2014{]}|pw}.\pend
           \pstart
           Leben Sie wohl und nehmen Sie von Ihrer schönen Arbeitsſehnſucht recht viel ins Civil
               herüber. So kö{\geminationn}ten Sie z. B. den Götterliebling\pwindex{Beer-Hofmann, Richard 11.07.1866 – 26.09.1945@\textsc{Beer-Hofmann, Richard} (11.07.1866 – 26.09.1945), \emph{Schriftsteller}!Tod Georgs1900@\strich\emph{Der Tod Georgs} {[}1900{]}|pw} zu Ende ſchreiben. Finden Sie nicht? – Viele {\pb}herzliche Grüße\pend
           \pstart Ihr \spacefill\mbox{Arthur}\pend{}\pstart
           24/6 95.\pend
                     \endnumbering\briefempfaengerindex{Beer-Hofmann, Richard@\textsc{Beer-Hofmann, Richard}!zzzSchnitzler, Arthur@\emph{von Arthur Schnitzler}!1895-06-241@{24. 6. 1895}|)be}\mylabel{h}\end{ledgroupsized}  \newcommand{\dateiname}{L00459}\newcommand{\titel}{Arthur Schnitzler an Richard Beer-Hofmann, 24. 6. 1895}\newcommand{\editorInnen}{Martin Anton Müller und Gerd-Hermann Susen}%% latex-leseansicht-abspann.tex
%% Abspann für die Leseansicht.
%% Der Schalter \ifkorrekturansicht ist bereits durch den Vorspann gesetzt.

%% latex-abspann.tex
%% Gemeinsamer Abspann für Korrekturansicht und Leseansicht.
%% Setzt den Schalter \ifkorrekturansicht voraus (gesetzt in den
%% einbindenden Dateien latex-korrekturansicht-abspann.tex bzw.
%% latex-leseansicht-abspann.tex).
%% ---------------------------------------------------------------

\normalsize

% Das esempio-Environment wird nur in der Leseansicht benötigt
\ifkorrekturansicht\else
\newenvironment{esempio}[3]%
{
    \vspace{1.5ex}
    \rlap{\underline{#1}}
    \par
    \setlength{\parindent}{0cm}
    \nopagebreak
    \leftskip=#2cm
    \rightskip=#3cm
}
{
    \par
}
\fi

\doendnotes{C}
\bigskip
\vfill

\clearpage

\footnotesize

\ifkorrekturansicht
  \lohead{\textsc{register}}
\fi

% theindex-Environment neu definieren ohne reledmac
\makeatletter
\renewenvironment{theindex}{%
  \ifkorrekturansicht
    \section*{\indexname}%
  \else
    \subsubsection*{Index der erwähnten Entitäten}%
  \fi
  \setlength{\parindent}{0pt}%
  \setlength{\parskip}{0pt plus 0.3pt}%
  \let\item\@idxitem
}{%
  \ifkorrekturansicht\clearpage\fi
}
\makeatother

\IfFileExists{\jobname-pw.ind}{\input{\jobname-pw.ind}}{}

% Quellenangabe nur in der Leseansicht
\ifkorrekturansicht\else
% Fallback-Definitionen, falls die .tex-Datei \titel etc. nicht gesetzt hat
\providecommand{\titel}{}
\providecommand{\editorInnen}{}
\providecommand{\dateiname}{\jobname}

\vspace{3cm}

\vfill

\footnotesize
\textsc{Quelle}: \titel. Herausgegeben von {\editorInnen}. In: \emph{Arthur Schnitzler: Briefwechsel mit Autorinnen und Autoren}.
 Digitale Edition, https://schnitzler-briefe.acdh.oeaw.ac.at/{\dateiname}.html (Stand \today)
\fi

\end{document}


      