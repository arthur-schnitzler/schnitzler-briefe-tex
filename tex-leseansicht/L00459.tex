%% latex-leseansicht-vorspann.tex
%% Vorspann für die Leseansicht.
%% Lädt die gemeinsame Datei latex-vorspann.tex mit nicht gesetztem Schalter.

\newif\ifkorrekturansicht
\korrekturansichtfalse

\input{../tex-inputs/latex-vorspann}


\section[Arthur Schnitzler an Richard Beer-Hofmann, 24. 6. 1895]{L00459 Arthur Schnitzler an Richard Beer-Hofmann, 24. 6. 1895}
\nopagebreak\mylabel{L00459v}
\rehead{ }\normalsize\beginnumbering\briefempfaengerindex{Beer-Hofmann, Richard@\textsc{Beer-Hofmann, Richard}!zzzSchnitzler, Arthur@\emph{von Arthur Schnitzler}!1895-06-241@{24. 6. 1895}|(be}
\toendnotes[C]{\smallbreak\pagebreak[2]}
\correspDesc{Versand  durch Arthur Schnitzler am 24. 6. 1895 in Wien
\newline{}Erhalt  durch Richard Beer-Hofmann am 25. 6. 1895 in Caslau}\toendnotes[C]{\smallbreak}
\Standort{YCGL, MSS 31.}
\physDesc{Brief, 2 Blätter, 8 Seiten, Kuvert, 1860 Zeichen
\newline{}Handschrift: 1) Bleistift, deutsche Kurrent\hspace{1em}2) schwarze Tinte, deutsche Kurrent (\noindent{}Umschlag)\hspace{1em}
\newline{}Versand: 1) Stempel: »\nobreak{}\oindex{I., Innere Stadt@\textbf{I., Innere Stadt}, \emph{Verwaltungsgebiet}|pwk}Wien 1/1, 24. 6. 95, 9–10 N\nobreak{}«.   2) Stempel: »\nobreak{}\oindex{Čáslav@\textbf{Čáslav}|pwk}Časlau, 25 6 95\nobreak{}«. }
\buchAbdrucke{\weitereDrucke{Arthur Schnitzler, Richard Beer-Hofmann: \emph{Briefwechsel 1891–1931}. Herausgegeben von Konstanze Fliedl. Wien, Zürich: \emph{Europaverlag} 1992, S. 76–77.} }\toendnotes[C]{\smallbreak}\pstart{}{\pb}Herrn n. a. Lieutenant\pend{}\pstart{}\textsc{Dr. Richard Beer Hofmann}\pend{}\pstart{}im k.k. Landw Inf Regimt.\pend{}\pstart{}\textsc{Caslau\oindex{Čáslav@\textbf{Čáslav}|pw} Nr 12}\pend{}{\bigskip}\vspace{1em}
\pstart
           \noindent{}{\pb}Lieber Richard. Ich freue mich{ }ſehr, daſs ich Sie noch in Wien\oindex{Wien@\textbf{Wien}, \emph{Verwaltungsgebiet}|pw}{ }ſehen werde. – \textsc{Nobl}\pwindex{Nobl, Gabor 12.\,10.\,1864 Szombathely – 14.\,3.\,1938 Wien@\textsc{Nobl, Gabor} (12.\,10.\,1864 Szombathely – 14.\,3.\,1938 Wien), \emph{Mediziner, Dermatologe}|pw}{ }ſprach ich vorgeſtern, er hat, »angeregt« durch
                  Ihr\introOben{}e\introOben{} perſönlich\textcolor{gray}{e}{ }\substVorne{}\textsuperscript{\textcolor{gray}{Epiſödchen}}\substDazwischen{}Beka{\geminationn}tſchaft\substHinten{}, das Kind\pwindex{Beer-Hofmann, Richard 11.\,7.\,1866 Wien – 26.\,9.\,1945 New York City@\textsc{Beer-Hofmann, Richard} (11.\,7.\,1866 Wien – 26.\,9.\,1945 New York City), \emph{Schriftsteller}!Kind@\strich\emph{Das Kind}|pw} geleſen. Sie werden erſucht,{ }ſich nächſtens auf {\pb}gefahrloſere Weiſe Leſer zu
               verſchaffen. – Habe heute Kopfweh, nach einer »\so{un}gemeinen«
               Landpartie die ich geſtern gemacht und die – entſchuldigen – in zwei miſerabeln
               Betten einer niederoeſterreichiſchen
                  Stadt\oindex{Klosterneuburg@\textbf{Klosterneuburg}, \emph{Hauptstadt}|pwv} endete.\pend
           
\pstart
           – Von der \textsc{Lou Salomé}\pwindex{Andreas-Salomé, Lou 12.\,2.\,1861 Sankt Petersburg – 5.\,2.\,1937 Göttingen@\textsc{Andreas-Salomé, Lou} (12.\,2.\,1861 Sankt Petersburg – 5.\,2.\,1937 Göttingen), \emph{Schriftstellerin}|pw} ha\textcolor{gray}{b} ich {\pb}noch i{\geminationm}er gar nichts gehört. Sie? – Wie wird es mit Kopenhagen\oindex{Kopenhagen@\textbf{Kopenhagen}, \emph{Hauptstadt}|pw}{ }ſein? – Auch von \textsc{Paul}\pwindex{Goldmann, Paul 31.\,1.\,1865 Breslau – 25.\,9.\,1935 Wien@\textsc{Goldmann, Paul} (31.\,1.\,1865 Breslau – 25.\,9.\,1935 Wien), \emph{Schriftsteller, Journalist}|pw} iſt noch nichts Definitives herauszubeko{\geminationm}en. –
                  Ke{\geminationn}en Sie den Briefwechſel \textsc{Lessing\pwindex{Lessing, Gotthold Ephraim 22.\,1.\,1729 Kamenz – 15.\,2.\,1781 Braunschweig@\textsc{Lessing, Gotthold Ephraim} (22.\,1.\,1729 Kamenz – 15.\,2.\,1781 Braunschweig), \emph{Schriftsteller, Kritiker, Philosoph}|pw} – Eva König\pwindex{König, Eva 22.\,3.\,1736 Heidelberg – 10.\,1.\,1778@\textsc{König, Eva} (22.\,3.\,1736 Heidelberg – 10.\,1.\,1778)|pw}}\pwindex{Lessing, Gotthold Ephraim 22.\,1.\,1729 Kamenz – 15.\,2.\,1781 Braunschweig@\textsc{Lessing, Gotthold Ephraim} (22.\,1.\,1729 Kamenz – 15.\,2.\,1781 Braunschweig), \emph{Schriftsteller, Kritiker, Philosoph}!Lessings Briefwechsel mit seiner Frau@\strich\emph{Lessings Briefwechsel mit seiner Frau}|pwv}\pwindex{König, Eva 22.\,3.\,1736 Heidelberg – 10.\,1.\,1778@\textsc{König, Eva} (22.\,3.\,1736 Heidelberg – 10.\,1.\,1778)!Lessings Briefwechsel mit seiner Frau@\strich\emph{Lessings Briefwechsel mit seiner Frau}|pwv}. Er iſt nicht{ }ſehr intereſſant. Merkwürdig nur, wie{ }ſie{ }ſich i{\geminationm}er über Lotterienu{\geminationm}ern {\pb}berathen. – Leſen Sie den \textsc{Candide}\pwindex{\textcolor{red}{\textsuperscript{XXXX indx1}}!Candide oder der Optimismus@\strich\emph{Candide oder der Optimismus}|pw}. – Hingegen weniger nothwendig das »Gelächter\pwindex{Dörmann, Felix 29.\,5.\,1870 Wien – 26.\,10.\,1928 ebd.@\textsc{Dörmann, Felix} (29.\,5.\,1870 Wien – 26.\,10.\,1928 ebd.), \emph{Schriftsteller}!Gelächter@\strich\emph{Gelächter}|pw}« von Dörmann\pwindex{Dörmann, Felix 29.\,5.\,1870 Wien – 26.\,10.\,1928 ebd.@\textsc{Dörmann, Felix} (29.\,5.\,1870 Wien – 26.\,10.\,1928 ebd.), \emph{Schriftsteller}|pw}. – Ich übe mich
               in erzählender Proſa: Schreibe »Hiſtorietten« – we{\geminationn} Sie
               wollen. Ja, den alten Dichter\pwindex{Schnitzler, Arthur 15.\,5.\,1862 Wien – 21.\,10.\,1931 ebd.@\textsc{Schnitzler, Arthur} (15.\,5.\,1862 Wien – 21.\,10.\,1931 ebd.), \emph{Schriftsteller, Mediziner}!Später Ruhm@\strich\emph{Später Ruhm}|pw} hab ich erheblich
               geſtrichen; ich find ihn aber noch i{\geminationm}er {\pb}etwas langweilig. Die{ }ſtiliſtiſchen Schlampereien
               (»ich bin erschrocken«){ }ſind wohl alle draußen. –\pend
           
\pstart
           – Für Iſchl\oindex{Bad Ischl@\textbf{Bad Ischl}|pw} hab ich literariſch gute Hoffnungen –
               möchte mein Stück\pwindex{Schnitzler, Arthur 15.\,5.\,1862 Wien – 21.\,10.\,1931 ebd.@\textsc{Schnitzler, Arthur} (15.\,5.\,1862 Wien – 21.\,10.\,1931 ebd.), \emph{Schriftsteller, Mediziner}!Liebelei. Schauspiel in drei Akten@\strich\emph{Liebelei. Schauspiel in drei Akten}|pwv} gern
               beenden. – Von Dörmann\pwindex{Dörmann, Felix 29.\,5.\,1870 Wien – 26.\,10.\,1928 ebd.@\textsc{Dörmann, Felix} (29.\,5.\,1870 Wien – 26.\,10.\,1928 ebd.), \emph{Schriftsteller}|pw}{ }ſoll dort ein Einakter gegeben werden, den er mir
               auch zum leſen gegeben hat u über den ich {\pb}eigentlich
               nicht{ }ſprechen darf. (»Auch von Frl. Albrecht\pwindex{Albrecht 1895 – 1895@\textsc{Albrecht} (1895 – 1895), \emph{Schauspielerin}|pw}
               müſſen wir einige freundliche Worte sagen.«) – Er heißt »Der Eisbrecher\pwindex{Dörmann, Felix 29.\,5.\,1870 Wien – 26.\,10.\,1928 ebd.@\textsc{Dörmann, Felix} (29.\,5.\,1870 Wien – 26.\,10.\,1928 ebd.), \emph{Schriftsteller}!Eisbrecher@\strich\emph{Der Eisbrecher}|pw}\pwindex{\textcolor{red}{\textsuperscript{XXXX indx1}}!Eisbrecher@\strich\emph{Der Eisbrecher}|pw}«. – Jo. –\pend
           
\pstart
           – Hugo\pwindex{Hofmannsthal, Hugo von 1.\,2.\,1874 Wien – 15.\,7.\,1929 Rodaun@\textsc{Hofmannsthal, Hugo von} (1.\,2.\,1874 Wien – 15.\,7.\,1929 Rodaun), \emph{Schriftsteller}|pw} war geſtern in Wien\oindex{Wien@\textbf{Wien}, \emph{Verwaltungsgebiet}|pw}, ich hab ihn verſäumt. – Heut bin ich braver Sohn und hole
                  Mama\pwindex{Schnitzler, Louise 8.\,7.\,1840 Kőszeg – 9.\,9.\,1911 Wien@\textsc{Schnitzler, Louise} (8.\,7.\,1840 Kőszeg – 9.\,9.\,1911 Wien)|pwv} von der Bahn
               ab. –\pend
           
\pstart
           – In dieſem Augenblick {\pb}ſitzt der Schreiber\pwindex{?? [Schreibkraft für Arthur Schnitzler] @\textsc{?? [Schreibkraft für Arthur Schnitzler]}|pwv} im Nebenzi{\geminationm}er u paginirt den alten
                  Dichter\pwindex{Schnitzler, Arthur 15.\,5.\,1862 Wien – 21.\,10.\,1931 ebd.@\textsc{Schnitzler, Arthur} (15.\,5.\,1862 Wien – 21.\,10.\,1931 ebd.), \emph{Schriftsteller, Mediziner}!Später Ruhm@\strich\emph{Später Ruhm}|pw}.\pend
           
\pstart
           Leben Sie wohl und nehmen Sie von Ihrer schönen Arbeitsſehnſucht recht viel ins Civil
               herüber. So kö{\geminationn}ten Sie z. B. den Götterliebling\pwindex{Beer-Hofmann, Richard 11.\,7.\,1866 Wien – 26.\,9.\,1945 New York City@\textsc{Beer-Hofmann, Richard} (11.\,7.\,1866 Wien – 26.\,9.\,1945 New York City), \emph{Schriftsteller}!Tod Georgs@\strich\emph{Der Tod Georgs}|pw} zu Ende{ }ſchreiben. Finden Sie nicht? – Viele {\pb}herzliche Grüße\pend
           \pstart Ihr \spacefill\mbox{Arthur}\pend{}
\pstart
           24/6 95.\pend
           \selectlanguage{ngerman}\endnumbering\briefempfaengerindex{Beer-Hofmann, Richard@\textsc{Beer-Hofmann, Richard}!zzzSchnitzler, Arthur@\emph{von Arthur Schnitzler}!1895-06-241@{24. 6. 1895}|)be}\mylabel{L00459h}  \newcommand{\dateiname}{L00459}\newcommand{\titel}{Arthur Schnitzler an Richard Beer-Hofmann, 24. 6. 1895}\newcommand{\editorInnen}{Martin Anton Müller und Gerd-Hermann Susen}%% latex-leseansicht-abspann.tex
%% Abspann für die Leseansicht.
%% Der Schalter \ifkorrekturansicht ist bereits durch den Vorspann gesetzt.

%% latex-abspann.tex
%% Gemeinsamer Abspann für Korrekturansicht und Leseansicht.
%% Setzt den Schalter \ifkorrekturansicht voraus (gesetzt in den
%% einbindenden Dateien latex-korrekturansicht-abspann.tex bzw.
%% latex-leseansicht-abspann.tex).
%% ---------------------------------------------------------------

\normalsize

% Das esempio-Environment wird nur in der Leseansicht benötigt
\ifkorrekturansicht\else
\newenvironment{esempio}[3]%
{
    \vspace{1.5ex}
    \rlap{\underline{#1}}
    \par
    \setlength{\parindent}{0cm}
    \nopagebreak
    \leftskip=#2cm
    \rightskip=#3cm
}
{
    \par
}
\fi

\doendnotes{C}
\bigskip
\vfill

\clearpage

\footnotesize

\ifkorrekturansicht
  \lohead{\textsc{register}}
\fi

% theindex-Environment neu definieren ohne reledmac
\makeatletter
\renewenvironment{theindex}{%
  \ifkorrekturansicht
    \section*{\indexname}%
  \else
    \subsubsection*{Index der erwähnten Entitäten}%
  \fi
  \setlength{\parindent}{0pt}%
  \setlength{\parskip}{0pt plus 0.3pt}%
  \let\item\@idxitem
}{%
  \ifkorrekturansicht\clearpage\fi
}
\makeatother

\IfFileExists{\jobname-pw.ind}{\input{\jobname-pw.ind}}{}

% Quellenangabe nur in der Leseansicht
\ifkorrekturansicht\else
% Fallback-Definitionen, falls die .tex-Datei \titel etc. nicht gesetzt hat
\providecommand{\titel}{}
\providecommand{\editorInnen}{}
\providecommand{\dateiname}{\jobname}

\vspace{3cm}

\vfill

\footnotesize
\textsc{Quelle}: \titel. Herausgegeben von {\editorInnen}. In: \emph{Arthur Schnitzler: Briefwechsel mit Autorinnen und Autoren}.
 Digitale Edition, https://schnitzler-briefe.acdh.oeaw.ac.at/{\dateiname}.html (Stand \today)
\fi

\end{document}


