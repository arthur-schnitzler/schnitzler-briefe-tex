%% latex-korrekturansicht-vorspann.tex
%% Vorspann für die Korrekturansicht.
%% Lädt die gemeinsame Datei latex-vorspann.tex mit gesetztem Schalter.

\newif\ifkorrekturansicht
\korrekturansichttrue

\input{../tex-inputs/latex-vorspann}


\section[Arthur Schnitzler an Hugo von Hofmannsthal, 5. 8. 1898]{L00830 Arthur Schnitzler an Hugo von Hofmannsthal, 5. 8. 1898}
\nopagebreak\mylabel{L00830v}
\rehead{ }\normalsize\beginnumbering\briefempfaengerindex{Hofmannsthal, Hugo von@\textsc{Hofmannsthal, Hugo von}!zzzSchnitzler, Arthur@\emph{von Arthur Schnitzler}!1898-08-051@{5. 8. 1898}|(be}
\toendnotes[C]{\smallbreak\pagebreak[2]}\Standort{FDH, Hs-30885,73.}
\physDesc{Brief, 1 Blatt, 4 Seiten, 1332 Zeichen
\newline{}Handschrift: Bleistift, deutsche Kurrent}
\buchAbdrucke{\weitereDrucke{Hugo von Hofmannsthal, Arthur Schnitzler: \emph{Briefwechsel}. Frankfurt am Main: \emph{S. Fischer} 1964, S. 108–109.} }\toendnotes[C]{\smallbreak}
\pstart
           \raggedleft{}{\pb}Tegernſee\oindex{Tegernsee@\textbf{Tegernsee}, \emph{P.PPL}|pw}{ }5. 8. 98\pend
           \vspace{0.5em}
\pstart
           Mein lieber Hugo, die Radtour, die wir vorhaben, iſt \introOben{}(\introOben{}ungefähr\introOben{})\introOben{}{ }\textsc{Basel}\oindex{Basel@\textbf{Basel}, \emph{P.PPLA}|pw}–\textsc{Biel}\oindex{Biel@\textbf{Biel}, \emph{P.PPLA2}|pw} bis hinunter zum Genferſee\oindex{Genfer See@\textbf{Genfer See}, \emph{H.LK}|pw}. Ob wir nur am
                  Genferſee\oindex{Genfer See@\textbf{Genfer See}, \emph{H.LK}|pw} bleiben oder da{\geminationn} ins italieniſche\oindex{Italien@\textbf{Italien}, \emph{A.PCLI}|pw}
               hinüber fahren, können wir uns an Ort u Stelle überlegen, jedenfalls ſteht die Sache
               heute ſo, dſs ich nicht nur bis zum 20. Zeit habe, ſondern bis
                  Ende Auguſt mit Ihnen bleiben kann und auch Luſt habe {\pb}mich an irgd einen See zu ſetzen. Dazu iſt ja auch Richard\pwindex{Beer-Hofmann, Richard 1866-07-11 – 1945-09-26@\textsc{Beer-Hofmann, Richard} (1866-07-11 – 1945-09-26), \emph{Schriftsteller/Schriftstellerin}|pw} vielleicht zu haben, es könnte ſehr
               ſchön ſein.\pend
           
\pstart
           Nun zu den Modalitäten unſrer Begegnung. \uline{Ich} bin am
                  12.{ }\substVorne{}\textsuperscript{a}\substDazwischen{}i\substHinten{}n München\oindex{Muenchen@\textbf{München}, \emph{P.PPLA}|pw} (aus verſchiedenen Gründen
                  \uline{muſs} ich nach München\oindex{Muenchen@\textbf{München}, \emph{P.PPLA}|pw}, u \uline{ka{\geminationn}
                  nicht} nach Innsbruck\oindex{Innsbruck@\textbf{Innsbruck}, \emph{A.ADM2}|pw}) und ſchlage Ihnen
               daher vor: treffen wir uns entweder am 12.{ }{\pb}in München\oindex{Muenchen@\textbf{München}, \emph{P.PPLA}|pw} oder,
               was Ihnen wahrſcheinlich bequemer ſein wird, \uuline{am
                     13. in Baſel\oindex{Basel@\textbf{Basel}, \emph{P.PPLA}|pw}}. (Sie führen da{\geminationn} direct Wien\oindex{Wien@\textbf{Wien}, \emph{A.ADM2}|pw}–\introOben{}I{\geminationn}sbruck\oindex{Innsbruck@\textbf{Innsbruck}, \emph{A.ADM2}|pw}–\introOben{}Baſel\oindex{Basel@\textbf{Basel}, \emph{P.PPLA}|pw}, \label{T_L00830-1v}\edtext{(}{\lemma{\textnormal{\emph{(}}}\Cendnote{\textnormal{In der Handschrift
                  setzt Schnitzler eine eckige Klammer für die öffnende und schließende Klammer
                  innerhalb der Klammer. Auf die Wiedergabe wurde, wegen der möglichen
                  Verwechslungen mit editorischen Zeichen, verzichtet.}}}\label{T_L00830-1}München\oindex{Muenchen@\textbf{München}, \emph{P.PPLA}|pw} iſt ein kleiner Umweg für Sie)). Ich
               denke, ſo iſt die Sache am einfachſten. Hier bin ich noch bis Dinſtag;
               jedenfalls bitte \uline{antworten Sie mir gleich}. Ob wir uns
               ſchon in Innsbruck\oindex{Innsbruck@\textbf{Innsbruck}, \emph{A.ADM2}|pw} oder erſt {\pb}in Baſel\oindex{Basel@\textbf{Basel}, \emph{P.PPLA}|pw} treffen, iſt
               bei dem Weſen unſrer Tour egal.\pend
           
\pstart
           Hoffentlich hat dieſe Correſpondenz ſchon endgiltige Bedeutung; ich freu mich rieſig
               auf die Reiſe, u. beſonders, dſs auch meine Zeit verhältnismäßg unbeſchränkt iſt.
               Alſo nochmals bitte \uline{gleich} Antwort. Von Herzen
                  Ihr\hspace*{1.5em}\spacefill\mbox{Arthur}\pend
           
\pstart
           \noindent{}\label{T_L00830-2v}\edtext{Richard\pwindex{Beer-Hofmann, Richard 1866-07-11 – 1945-09-26@\textsc{Beer-Hofmann, Richard} (1866-07-11 – 1945-09-26), \emph{Schriftsteller/Schriftstellerin}|pw} hat Schwarzk.\pwindex{Schwarzkopf, Gustav 07.11.1853 – 13.11.1939@\textsc{Schwarzkopf, Gustav} (07.11.1853 – 13.11.1939), \emph{Schriftsteller/Schriftstellerin}|pw} u mir in Salzburg\oindex{Salzburg@\textbf{Salzburg}, \emph{A.ADM2}|pw}{ }ſein 3. Capitel\pwindex{Tod Georgs@\emph{Der Tod Georgs}|pwv}{ }\label{K_L00830-1v}\edtext{vorgeleſen}{\lemma{\textnormal{\emph{vorgeleſen}}}\Cendnote{\textnormal{Siehe A. S.: \emph{Tagebuch}, 28. 7. 1898.
                  }}}\label{K_L00830-1}. Es iſt außerordentlich.}{\lemma{\textnormal{\emph{Richard … außerordentlich.}}}\Cendnote{\textnormal{am
                     unteren Blattrand auf dem Kopf}}}\label{T_L00830-2}\pend
           \selectlanguage{ngerman}\endnumbering\briefempfaengerindex{Hofmannsthal, Hugo von@\textsc{Hofmannsthal, Hugo von}!zzzSchnitzler, Arthur@\emph{von Arthur Schnitzler}!1898-08-051@{5. 8. 1898}|)be}\mylabel{L00830h}  \normalsize

\doendnotes{C}
\bigskip
\vfill

\clearpage

\footnotesize

\lohead{\textsc{register}}

% Definiere theindex-Environment komplett neu ohne reledmac
\makeatletter
\renewenvironment{theindex}{%
  \section*{\indexname}%
  \setlength{\parindent}{0pt}%
  \setlength{\parskip}{0pt plus 0.3pt}%
  \let\item\@idxitem
}{%
  \clearpage
}
\makeatother

\IfFileExists{\jobname-pw.ind}{\input{\jobname-pw.ind}}{}

\end{document}

      