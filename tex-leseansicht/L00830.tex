%% latex-leseansicht-vorspann.tex
%% Vorspann für die Leseansicht.
%% Lädt die gemeinsame Datei latex-vorspann.tex mit nicht gesetztem Schalter.

\newif\ifkorrekturansicht
\korrekturansichtfalse

\input{../tex-inputs/latex-vorspann}


         
         \renewcommand{\erwaehntePersonen}{Personen: Richard Beer-Hofmann, Hugo von Hofmannsthal, Gustav Schwarzkopf}
         \renewcommand{\erwaehnteOrte}{Orte: Basel, Biel, Genfer See, Hinterbrühl, Innsbruck, Italien, München, Salzburg, Tegernsee, Wien}
         \renewcommand{\erwaehnteWerke}{Werke: Der Tod Georgs}
               \section[Arthur Schnitzler an Hugo von Hofmannsthal, 5. 8. 1898]{ Arthur Schnitzler an Hugo von Hofmannsthal, 5. 8. 1898}\nopagebreak\mylabel{v}\rehead{ }\begin{ledgroupsized}[t]{13cm}\normalsize\beginnumbering\briefempfaengerindex{Hofmannsthal, Hugo von@\textsc{Hofmannsthal, Hugo von}!zzzSchnitzler, Arthur@\emph{von Arthur Schnitzler}!1898-08-051@{5. 8. 1898}|(be} \toendnotes[C]{\smallbreak\pagebreak[2]} \Standort{FDH, Hs-30885,73.}
\physDesc{Brief, 1 Blatt, 4 Seiten, 1332 Zeichen
\newline{}Handschrift: Bleistift, deutsche Kurrent}\buchAbdrucke{\weitereDrucke{Hugo von Hofmannsthal, Arthur Schnitzler: \emph{Briefwechsel}. Hg. Therese Nickl und Heinrich Schnitzler. Frankfurt am Main: \emph{S. Fischer} 1964, S. 108–109.} }\toendnotes[C]{\smallbreak}\pstart
           \raggedleft{}{\pb}Tegernſee\oindex{Tegernsee@\textbf{Tegernsee}|pw}{ }5. 8. 98\pend
           \pstart
           Mein lieber Hugo, die Radtour, die wir vorhaben, iſt \introOben{}(\introOben{}ungefähr\introOben{})\introOben{}{ }\textsc{Basel}\oindex{Basel@\textbf{Basel}|pw}–\textsc{Biel}\oindex{Biel@\textbf{Biel}|pw} bis hinunter zum Genferſee\oindex{Genfer See@\textbf{Genfer See}|pw}. Ob wir nur am
                  Genferſee\oindex{Genfer See@\textbf{Genfer See}|pw} bleiben oder da{\geminationn} ins italieniſche\oindex{Italien@\textbf{Italien}|pw}
               hinüber fahren, können wir uns an Ort u Stelle überlegen, jedenfalls ſteht die Sache
               heute ſo, dſs ich nicht nur bis zum 20. Zeit habe, ſondern bis
                  Ende Auguſt mit Ihnen bleiben kann und auch Luſt habe {\pb}mich an irgd einen See zu ſetzen. Dazu iſt ja auch Richard\pwindex{Beer-Hofmann, Richard 1866-07-11 – 1945-09-26@\textsc{Beer-Hofmann, Richard} (1866-07-11 – 1945-09-26), \emph{Schriftsteller}|pw} vielleicht zu haben, es könnte ſehr
               ſchön ſein.\pend
           \pstart
           Nun zu den Modalitäten unſrer Begegnung. \uline{Ich} bin am
                  12.{ }\substVorne{}\textsuperscript{a}\substDazwischen{}i\substHinten{}n München\oindex{Muenchen@\textbf{München}|pw} (aus verſchiedenen Gründen
                  \uline{muſs} ich nach München\oindex{Muenchen@\textbf{München}|pw}, u \uline{ka{\geminationn}
                  nicht} nach Innsbruck\oindex{Innsbruck@\textbf{Innsbruck}|pw}) und ſchlage Ihnen
               daher vor: treffen wir uns entweder am 12.{ }{\pb}in München\oindex{Muenchen@\textbf{München}|pw} oder,
               was Ihnen wahrſcheinlich bequemer ſein wird, \uuline{am
                     13. in Baſel\oindex{Basel@\textbf{Basel}|pw}}. (Sie führen da{\geminationn} direct Wien\oindex{Wien@\textbf{Wien}|pw}–\introOben{}I{\geminationn}sbruck\oindex{Innsbruck@\textbf{Innsbruck}|pw}–\introOben{}Baſel\oindex{Basel@\textbf{Basel}|pw}, \label{T_L00830-1v}\edtext{(}{\lemma{\textnormal{\emph{(}}}\Cendnote{\textnormal{In der Handschrift
                  setzt Schnitzler eine eckige Klammer für die öffnende und schließende Klammer
                  innerhalb der Klammer. Auf die Wiedergabe wurde, wegen der möglichen
                  Verwechslungen mit editorischen Zeichen, verzichtet.}}}\label{T_L00830-1h}München\oindex{Muenchen@\textbf{München}|pw} iſt ein kleiner Umweg für Sie)). Ich
               denke, ſo iſt die Sache am einfachſten. Hier bin ich noch bis Dinſtag;
               jedenfalls bitte \uline{antworten Sie mir gleich}. Ob wir uns
               ſchon in Innsbruck\oindex{Innsbruck@\textbf{Innsbruck}|pw} oder erſt {\pb}in Baſel\oindex{Basel@\textbf{Basel}|pw} treffen, iſt
               bei dem Weſen unſrer Tour egal.\pend
           \pstart
           Hoffentlich hat dieſe Correſpondenz ſchon endgiltige Bedeutung; ich freu mich rieſig
               auf die Reiſe, u. beſonders, dſs auch meine Zeit verhältnismäßg unbeſchränkt iſt.
               Alſo nochmals bitte \uline{gleich} Antwort. Von Herzen
                  Ihr\hspace*{1.5em}\spacefill\mbox{Arthur}\pend
           \pstart
           \noindent{}\label{T_L00830-2v}\edtext{Richard\pwindex{Beer-Hofmann, Richard 1866-07-11 – 1945-09-26@\textsc{Beer-Hofmann, Richard} (1866-07-11 – 1945-09-26), \emph{Schriftsteller}|pw} hat Schwarzk.\pwindex{Schwarzkopf, Gustav 07.11.1853 – 13.11.1939@\textsc{Schwarzkopf, Gustav} (07.11.1853 – 13.11.1939), \emph{Schriftsteller}|pw} u mir in Salzburg\oindex{Salzburg@\textbf{Salzburg}|pw}{ }ſein 3. Capitel\pwindex{Beer-Hofmann, Richard 1866-07-11 – 1945-09-26@\textsc{Beer-Hofmann, Richard} (1866-07-11 – 1945-09-26), \emph{Schriftsteller}!Tod Georgs1900@\strich\emph{Der Tod Georgs} {[}1900{]}|pwv}{ }\label{K_L00830-1v}\edtext{vorgeleſen}{\lemma{\textnormal{\emph{vorgeleſen}}}\Cendnote{\textnormal{Siehe A. S.: \emph{Tagebuch}, 28. 7. 1898.
                  }}}\label{K_L00830-1h}. Es iſt außerordentlich.}{\lemma{\textnormal{\emph{Richard … außerordentlich.}}}\Cendnote{\textnormal{am
                     unteren Blattrand auf dem Kopf}}}\label{T_L00830-2h}\pend
           
         
         \endnumbering\mylabel{h}\end{ledgroupsized}  \newcommand{\dateiname}{L00830}\newcommand{\titel}{Arthur Schnitzler an Hugo von Hofmannsthal, 5. 8. 1898}\newcommand{\editorInnen}{Martin Anton Müller und Gerd-Hermann Susen}%% latex-leseansicht-abspann.tex
%% Abspann für die Leseansicht.
%% Der Schalter \ifkorrekturansicht ist bereits durch den Vorspann gesetzt.

%% latex-abspann.tex
%% Gemeinsamer Abspann für Korrekturansicht und Leseansicht.
%% Setzt den Schalter \ifkorrekturansicht voraus (gesetzt in den
%% einbindenden Dateien latex-korrekturansicht-abspann.tex bzw.
%% latex-leseansicht-abspann.tex).
%% ---------------------------------------------------------------

\normalsize

% Das esempio-Environment wird nur in der Leseansicht benötigt
\ifkorrekturansicht\else
\newenvironment{esempio}[3]%
{
    \vspace{1.5ex}
    \rlap{\underline{#1}}
    \par
    \setlength{\parindent}{0cm}
    \nopagebreak
    \leftskip=#2cm
    \rightskip=#3cm
}
{
    \par
}
\fi

\doendnotes{C}
\bigskip
\vfill

\clearpage

\footnotesize

\ifkorrekturansicht
  \lohead{\textsc{register}}
\fi

% theindex-Environment neu definieren ohne reledmac
\makeatletter
\renewenvironment{theindex}{%
  \ifkorrekturansicht
    \section*{\indexname}%
  \else
    \subsubsection*{Index der erwähnten Entitäten}%
  \fi
  \setlength{\parindent}{0pt}%
  \setlength{\parskip}{0pt plus 0.3pt}%
  \let\item\@idxitem
}{%
  \ifkorrekturansicht\clearpage\fi
}
\makeatother

\IfFileExists{\jobname-pw.ind}{\input{\jobname-pw.ind}}{}

% Quellenangabe nur in der Leseansicht
\ifkorrekturansicht\else
% Fallback-Definitionen, falls die .tex-Datei \titel etc. nicht gesetzt hat
\providecommand{\titel}{}
\providecommand{\editorInnen}{}
\providecommand{\dateiname}{\jobname}

\vspace{3cm}

\vfill

\footnotesize
\textsc{Quelle}: \titel. Herausgegeben von {\editorInnen}. In: \emph{Arthur Schnitzler: Briefwechsel mit Autorinnen und Autoren}.
 Digitale Edition, https://schnitzler-briefe.acdh.oeaw.ac.at/{\dateiname}.html (Stand \today)
\fi

\end{document}


      