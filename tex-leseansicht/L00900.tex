%% latex-leseansicht-vorspann.tex
%% Vorspann für die Leseansicht.
%% Lädt die gemeinsame Datei latex-vorspann.tex mit nicht gesetztem Schalter.

\newif\ifkorrekturansicht
\korrekturansichtfalse

\input{../tex-inputs/latex-vorspann}


\section[Arthur Schnitzler an Julius Rodenberg, 7. 3. 1899]{L00900 Arthur Schnitzler an Julius Rodenberg, 7. 3. 1899}
\nopagebreak\mylabel{L00900v}
\rehead{ }\normalsize\beginnumbering\briefempfaengerindex{Rodenberg, Julius@\textsc{Rodenberg, Julius}!zzzSchnitzler, Arthur@\emph{von Arthur Schnitzler}!1899-03-072@{7. 3. 1899}|(be}
\toendnotes[C]{\smallbreak\pagebreak[2]}
\correspDesc{Versand  durch Arthur Schnitzler am 7. 3. 1899 in Wien
\newline{}Erhalt  durch Julius Rodenberg im Zeitraum [8. 3. 1899
                  – 12. 3. 1899?] in Berlin}\toendnotes[C]{\smallbreak}
\Standort{Weimar, Klassik Stiftung, 81/X,2,10.}
\physDesc{Brief, 1 Blatt, 2 Seiten, 506 Zeichen
\newline{}Handschrift: schwarze Tinte, deutsche Kurrent}
\pstart{}{\pb}Sehr geehrter Herr Doktor,\pend\vspace{0.5em}
\pstart
           noch immer ko{\geminationm}e ich mit keiner Novelle; – ich habe noch
               immer keine geſchrieben. Hingegen möchte ich Ihnen gern meinen in der Burg\oindex{Wien@\textbf{Wien}!I., Innere Stadt@\textbf{I., Innere Stadt}!Burgtheater@\textbf{Burgtheater}, \emph{Theater}|pw} aufgeführten Einakter »Die Gefährtin\pwindex{Schnitzler, Arthur 15.\,5.\,1862 Wien – 21.\,10.\,1931 ebd.@\textsc{Schnitzler, Arthur} (15.\,5.\,1862 Wien – 21.\,10.\,1931 ebd.), \emph{Schriftsteller, Mediziner}!Gefährtin. Schauspiel in einem Akt@\strich\emph{Die Gefährtin. Schauspiel in einem Akt}|pw}« für die »Deutſche
                  Rundſchau\orgindex{Deutsche Rundschau@Deutsche Rundschau|pw}« überreichen, und bitte Sie mir freundlichſt zu{ }ſagen, erſtens, ob
                  {\pb}Sie überhaupt dramatiſches bringen, zweitens ob
               Sie einen Einakter von mir haben wollen, drittens \uline{wann}{ }Sie das kleine Stück bringen könnten, wenn Sie es
               nehmen.\pend
           
\pstart
           Ihr hochachtungsvoll ergebener{\\[\baselineskip]}\spacefill\mbox{ArthurSchnitzler}\pend
           \leftskip=0em{}
\pstart
           Wien\oindex{Wien@\textbf{Wien}, \emph{Verwaltungsgebiet}|pw}{ }7. 3. 99.\pend
           \selectlanguage{ngerman}\endnumbering\briefempfaengerindex{Rodenberg, Julius@\textsc{Rodenberg, Julius}!zzzSchnitzler, Arthur@\emph{von Arthur Schnitzler}!1899-03-072@{7. 3. 1899}|)be}\mylabel{L00900h}  \newcommand{\dateiname}{L00900}\newcommand{\titel}{Arthur Schnitzler an Julius Rodenberg, 7. 3. 1899}\newcommand{\editorInnen}{Martin Anton Müller und Gerd-Hermann Susen}%% latex-leseansicht-abspann.tex
%% Abspann für die Leseansicht.
%% Der Schalter \ifkorrekturansicht ist bereits durch den Vorspann gesetzt.

%% latex-abspann.tex
%% Gemeinsamer Abspann für Korrekturansicht und Leseansicht.
%% Setzt den Schalter \ifkorrekturansicht voraus (gesetzt in den
%% einbindenden Dateien latex-korrekturansicht-abspann.tex bzw.
%% latex-leseansicht-abspann.tex).
%% ---------------------------------------------------------------

\normalsize

% Das esempio-Environment wird nur in der Leseansicht benötigt
\ifkorrekturansicht\else
\newenvironment{esempio}[3]%
{
    \vspace{1.5ex}
    \rlap{\underline{#1}}
    \par
    \setlength{\parindent}{0cm}
    \nopagebreak
    \leftskip=#2cm
    \rightskip=#3cm
}
{
    \par
}
\fi

\doendnotes{C}
\bigskip
\vfill

\clearpage

\footnotesize

\ifkorrekturansicht
  \lohead{\textsc{register}}
\fi

% theindex-Environment neu definieren ohne reledmac
\makeatletter
\renewenvironment{theindex}{%
  \ifkorrekturansicht
    \section*{\indexname}%
  \else
    \subsubsection*{Index der erwähnten Entitäten}%
  \fi
  \setlength{\parindent}{0pt}%
  \setlength{\parskip}{0pt plus 0.3pt}%
  \let\item\@idxitem
}{%
  \ifkorrekturansicht\clearpage\fi
}
\makeatother

\IfFileExists{\jobname-pw.ind}{\input{\jobname-pw.ind}}{}

% Quellenangabe nur in der Leseansicht
\ifkorrekturansicht\else
% Fallback-Definitionen, falls die .tex-Datei \titel etc. nicht gesetzt hat
\providecommand{\titel}{}
\providecommand{\editorInnen}{}
\providecommand{\dateiname}{\jobname}

\vspace{3cm}

\vfill

\footnotesize
\textsc{Quelle}: \titel. Herausgegeben von {\editorInnen}. In: \emph{Arthur Schnitzler: Briefwechsel mit Autorinnen und Autoren}.
 Digitale Edition, https://schnitzler-briefe.acdh.oeaw.ac.at/{\dateiname}.html (Stand \today)
\fi

\end{document}


