%% latex-korrekturansicht-vorspann.tex
%% Vorspann für die Korrekturansicht.
%% Lädt die gemeinsame Datei latex-vorspann.tex mit gesetztem Schalter.

\newif\ifkorrekturansicht
\korrekturansichttrue

\input{../tex-inputs/latex-vorspann}


\section[Arthur Schnitzler an Julius Rodenberg, 7. 3. 1899]{L00900 Arthur Schnitzler an Julius Rodenberg, 7. 3. 1899 }
\nopagebreak\mylabel{L00900v}
\rehead{ }\normalsize\beginnumbering\briefempfaengerindex{Rodenberg, Julius@\textsc{Rodenberg, Julius}!zzzSchnitzler, Arthur@\emph{von Arthur Schnitzler}!1899-03-072@{7. 3. 1899}|(be}
\toendnotes[C]{\smallbreak\pagebreak[2]}\Standort{Weimar, Klassik Stiftung, 81/X,2,10.}
\physDesc{Brief, 1 Blatt, 2 Seiten, 506 Zeichen
\newline{}Handschrift: schwarze Tinte, deutsche Kurrent}
\pstart{}{\pb}Sehr geehrter Herr Doktor,\pend\vspace{0.5em}
\pstart
           noch immer ko{\geminationm}e ich mit keiner Novelle; – ich habe noch
               immer keine geſchrieben. Hingegen möchte ich Ihnen gern meinen in der Burg\oindex{Burgtheater@\textbf{Burgtheater}, \emph{S.THTR}|pw} aufgeführten Einakter »Die Gefährtin\pwindex{Gefaehrtin. Schauspiel in einem Akt@\emph{Die Gefährtin. Schauspiel in einem Akt}|pw}« für die »Deutſche
                  Rundſchau\orgindex{Deutsche Rundschau@Deutsche Rundschau|pw}« überreichen, und bitte Sie mir freundlichſt zu ſagen, erſtens, ob
                  {\pb}Sie überhaupt dramatiſches bringen, zweitens ob
               Sie einen Einakter von mir haben wollen, drittens \uline{wann}{ }Sie das kleine Stück bringen könnten, wenn Sie es
               nehmen.\pend
           
\pstart
           Ihr hochachtungsvoll ergebener{\\[\baselineskip]}\spacefill\mbox{ArthurSchnitzler}\pend
           \leftskip=0em{}
\pstart
           Wien\oindex{Wien@\textbf{Wien}, \emph{A.ADM2}|pw}{ }7. 3. 99.\pend
           \selectlanguage{ngerman}\endnumbering\briefempfaengerindex{Rodenberg, Julius@\textsc{Rodenberg, Julius}!zzzSchnitzler, Arthur@\emph{von Arthur Schnitzler}!1899-03-072@{7. 3. 1899}|)be}\mylabel{L00900h}  \normalsize

\doendnotes{C}
\bigskip
\vfill

\clearpage

\footnotesize

\lohead{\textsc{register}}

% Definiere theindex-Environment komplett neu ohne reledmac
\makeatletter
\renewenvironment{theindex}{%
  \section*{\indexname}%
  \setlength{\parindent}{0pt}%
  \setlength{\parskip}{0pt plus 0.3pt}%
  \let\item\@idxitem
}{%
  \clearpage
}
\makeatother

\IfFileExists{\jobname-pw.ind}{\input{\jobname-pw.ind}}{}

\end{document}

      