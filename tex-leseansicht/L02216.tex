%% latex-leseansicht-vorspann.tex
%% Vorspann für die Leseansicht.
%% Lädt die gemeinsame Datei latex-vorspann.tex mit nicht gesetztem Schalter.

\newif\ifkorrekturansicht
\korrekturansichtfalse

\input{../tex-inputs/latex-vorspann}


         
         \renewcommand{\erwaehntePersonen}{Personen: Robert Adam}
         \renewcommand{\erwaehnteOrte}{Orte: Niederösterreich, Sternwartestraße, Wien, XVIII., Währing, Zistersdorf}
         \renewcommand{\erwaehnteWerke}{Werke: Gesellschaft [Eine Gaunerkomödie], Rechtsphilosophie}
               \section[Arthur Schnitzler an Robert Adam, 20. 7. 1915]{ Arthur Schnitzler an Robert Adam, 20. 7. 1915}\nopagebreak\mylabel{v}\rehead{ }\begin{ledgroupsized}[t]{13cm}\normalsize\beginnumbering \toendnotes[C]{\smallbreak\pagebreak[2]} \Standort{DLA, 96.34.1/15.}
\physDesc{Briefkarte, , , , , , Umschlag
\newline{}Handschrift: schwarze Tinte, lateinische Kurrent\newline{}Versand: Stempel: »\nobreak{}\oindex{XVIII., Waehring@\textbf{XVIII., Währing}|pwk}18/1 Wien 110, 21. VII. 15, 3\nobreak{}«.  }\pstart{}{\pb}\textcolor{gray}{\textbf{Dr. Arthur Schnitzler}}\pend{}\pstart{}\textcolor{gray}{\textbf{Wien XVIII. Sternwartestrasse 71\oindex{Sternwartestrasse@\textbf{Sternwartestraße}|pw}}}\pend{}{\bigskip}\pstart{}{\pb}Herrn Dr. Robert Adam
                        Pollak\pend{}\pstart{}Bezirksrichter in\pend{}\pstart{}Zistersdorf\oindex{Zistersdorf@\textbf{Zistersdorf}|pw}.\pend{}\pstart{}N. Oe.\oindex{Niederoesterreich@\textbf{Niederösterreich}|pw}\pend{}{\bigskip}\pstart
           \noindent{}{\pb}\textcolor{gray}{\textbf{Dr. Arthur Schnitzler}}\hfill 20/7 1915\pend
           \pstart
           \textcolor{gray}{\textbf{Wien XVIII. Sternwartestrasse 71\oindex{Sternwartestrasse@\textbf{Sternwartestraße}|pw}}}\pend
           \pstart
           verehrter Herr Doctor, es freut mich, daß Sie meine nicht
                    durchaus freundlichen Worte über die »Gesellschaft\pwindex{Adam, Robert 20.04.1877 – 16.10.1961@\textsc{Adam, Robert} (20.04.1877 – 16.10.1961), \emph{Schriftsteller, Richter}!Gesellschaft [Eine Gaunerkomoedie]None@\strich\emph{Gesellschaft [Eine Gaunerkomödie]} {[}None{]}|pw}« so liebenswürdg aufgeno{\geminationm}en
                    haben und ich möchte nur nochmals darauf hinweisen, daß ich eine Art von
                    Bühnenwirkung durchaus nicht ausgeschlossen halte{[}.{]} Was das
                    »gelegentliche Hinschmeißen« anbelangt, so bin ich übrigens ganz Ihrer Ansicht –
                    nur weiß man nicht im voraus, was der »Welt« gefallen wird – und die Nachwelt
                    (die bisweilen sehr früh anfängt) ent{\pb}scheidet nach ziemlich
                    geheimnisvollen Gesetzen, gerechter – aber im Sinne der Selbstkritik – die einem
                    gewissen Niveau des Talents continuierlich waltet (auch we{\geminationn} wir versuchen wegzuhören).\pend
           \pstart
           So sehe ich Ihrer »Rechtsphilosophie\pwindex{Adam, Robert 20.04.1877 – 16.10.1961@\textsc{Adam, Robert} (20.04.1877 – 16.10.1961), \emph{Schriftsteller, Richter}!RechtsphilosophieNone@\strich\emph{Rechtsphilosophie} {[}None{]}|pw}«, Ihrer
                    neuen Komödie und einer baldigen Wiederbegegnung mit Vergnügen entgegegen.\pend
           \pstart
           herzlich grüßend\hspace*{1.5em}Ihr sehr ergebner{\\[\baselineskip]}\spacefill\mbox{Arthur Schnitzler}\pend
           \leftskip=0em{}
         
         \endnumbering\mylabel{h}\end{ledgroupsized}  \newcommand{\dateiname}{L02216}\newcommand{\titel}{Arthur Schnitzler an Robert Adam, 20. 7. 1915}\newcommand{\editorInnen}{Martin Anton Müller und Gerd-Hermann Susen}%% latex-leseansicht-abspann.tex
%% Abspann für die Leseansicht.
%% Der Schalter \ifkorrekturansicht ist bereits durch den Vorspann gesetzt.

%% latex-abspann.tex
%% Gemeinsamer Abspann für Korrekturansicht und Leseansicht.
%% Setzt den Schalter \ifkorrekturansicht voraus (gesetzt in den
%% einbindenden Dateien latex-korrekturansicht-abspann.tex bzw.
%% latex-leseansicht-abspann.tex).
%% ---------------------------------------------------------------

\normalsize

% Das esempio-Environment wird nur in der Leseansicht benötigt
\ifkorrekturansicht\else
\newenvironment{esempio}[3]%
{
    \vspace{1.5ex}
    \rlap{\underline{#1}}
    \par
    \setlength{\parindent}{0cm}
    \nopagebreak
    \leftskip=#2cm
    \rightskip=#3cm
}
{
    \par
}
\fi

\doendnotes{C}
\bigskip
\vfill

\clearpage

\footnotesize

\ifkorrekturansicht
  \lohead{\textsc{register}}
\fi

% theindex-Environment neu definieren ohne reledmac
\makeatletter
\renewenvironment{theindex}{%
  \ifkorrekturansicht
    \section*{\indexname}%
  \else
    \subsubsection*{Index der erwähnten Entitäten}%
  \fi
  \setlength{\parindent}{0pt}%
  \setlength{\parskip}{0pt plus 0.3pt}%
  \let\item\@idxitem
}{%
  \ifkorrekturansicht\clearpage\fi
}
\makeatother

\IfFileExists{\jobname-pw.ind}{\input{\jobname-pw.ind}}{}

% Quellenangabe nur in der Leseansicht
\ifkorrekturansicht\else
% Fallback-Definitionen, falls die .tex-Datei \titel etc. nicht gesetzt hat
\providecommand{\titel}{}
\providecommand{\editorInnen}{}
\providecommand{\dateiname}{\jobname}

\vspace{3cm}

\vfill

\footnotesize
\textsc{Quelle}: \titel. Herausgegeben von {\editorInnen}. In: \emph{Arthur Schnitzler: Briefwechsel mit Autorinnen und Autoren}.
 Digitale Edition, https://schnitzler-briefe.acdh.oeaw.ac.at/{\dateiname}.html (Stand \today)
\fi

\end{document}


      