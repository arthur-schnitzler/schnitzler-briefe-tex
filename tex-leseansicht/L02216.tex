%% latex-leseansicht-vorspann.tex
%% Vorspann für die Leseansicht.
%% Lädt die gemeinsame Datei latex-vorspann.tex mit nicht gesetztem Schalter.

\newif\ifkorrekturansicht
\korrekturansichtfalse

\input{../tex-inputs/latex-vorspann}


\section[Arthur Schnitzler an Robert Adam, 20. 7. 1915]{L02216 Arthur Schnitzler an Robert Adam, 20. 7. 1915}
\nopagebreak\mylabel{L02216v}
\rehead{ }\normalsize\beginnumbering\briefempfaengerindex{Adam, Robert@\textsc{Adam, Robert}!zzzSchnitzler, Arthur@\emph{von Arthur Schnitzler}!1915-07-201@{20. 7. 1915}|(be}
\toendnotes[C]{\smallbreak\pagebreak[2]}
\correspDesc{Versand  durch Arthur Schnitzler am 20. 7. 1915 in Wien
\newline{}Erhalt  durch Robert Adam im Zeitraum [21. 7. 1915
                  – 25. 7. 1915?] in Zistersdorf}\toendnotes[C]{\smallbreak}
\Standort{DLA, 96.34.1/15.}
\physDesc{Briefkarte, , Kuvert, 894 Zeichen
\newline{}Handschrift: schwarze Tinte, lateinische Kurrent
\newline{}Versand: Stempel: »\nobreak{}\oindex{XVIII., Währing@\textbf{XVIII., Währing}, \emph{Verwaltungsgebiet}|pwk}18/1 Wien 110, 21. VII. 15, 3\nobreak{}«.  }\pstart{}{\pb}\textcolor{gray}{\textbf{Dr. Arthur Schnitzler}}\pend{}\pstart{}\textcolor{gray}{\textbf{Wien XVIII. Sternwartestrasse 71\oindex{Wien@\textbf{Wien}!XVIII., Währing@\textbf{XVIII., Währing}!Sternwartestraße 71@\textbf{Sternwartestraße 71}, \emph{Wohngebäude}|pw}}}\pend{}{\bigskip}\pstart{}{\pb}Herrn Dr. Robert Adam Pollak\pend{}\pstart{}Bezirksrichter in\pend{}\pstart{}Zistersdorf\oindex{Zistersdorf@\textbf{Zistersdorf}, \emph{Verwaltungsgebiet}|pw}.\pend{}\pstart{}N. Oe.\oindex{Niederösterreich@\textbf{Niederösterreich}, \emph{Land}|pw}\pend{}{\bigskip}\vspace{1em}
\pstart
           {\pb}\textcolor{gray}{\textbf{Dr. Arthur Schnitzler}}\hfill 20/7 1915\pend
           
\pstart
           \textcolor{gray}{\textbf{Wien XVIII. Sternwartestrasse 71\oindex{Wien@\textbf{Wien}!XVIII., Währing@\textbf{XVIII., Währing}!Sternwartestraße 71@\textbf{Sternwartestraße 71}, \emph{Wohngebäude}|pw}}}\pend
           \vspace{0.5em}
\pstart
           verehrter Herr Doctor, es freut mich, daß Sie meine nicht durchaus
               freundlichen Worte über die »Gesellschaft\pwindex{Adam, Robert 20.\,4.\,1877 Wien – 16.\,10.\,1961 Baden bei Wien@\textsc{Adam, Robert} (20.\,4.\,1877 Wien – 16.\,10.\,1961 Baden bei Wien), \emph{Schriftsteller, Richter}!Gesellschaft [Eine Gaunerkomödie]@\strich\emph{Gesellschaft [Eine Gaunerkomödie]}|pw}« so
               liebenswürdg aufgeno{\geminationm}en haben und ich möchte nur
               nochmals darauf hinweisen, daß ich eine Art von Bühnenwirkung durchaus nicht
               ausgeschlossen halte{[}.{]} Was das »gelegentliche Hinschmeißen«
               anbelangt, so bin ich übrigens ganz Ihrer Ansicht – nur weiß man nicht im voraus, was
               der »Welt« gefallen wird – und die Nachwelt (die bisweilen sehr früh anfängt) ent{\pb}scheidet nach ziemlich geheimnisvollen Gesetzen, gerechter
               – aber im Sinne der Selbstkritik – die einem gewissen Niveau des Talents
               continuierlich waltet (auch we{\geminationn} wir versuchen
               wegzuhören).\pend
           
\pstart
           So sehe ich Ihrer »Rechtsphilosophie\pwindex{Adam, Robert 20.\,4.\,1877 Wien – 16.\,10.\,1961 Baden bei Wien@\textsc{Adam, Robert} (20.\,4.\,1877 Wien – 16.\,10.\,1961 Baden bei Wien), \emph{Schriftsteller, Richter}!Rechtsphilosophie@\strich\emph{Rechtsphilosophie}|pw}«, Ihrer
               neuen Komödie und einer baldigen Wiederbegegnung mit Vergnügen entgegegen.\pend
           
\pstart
           herzlich grüßend\hspace*{1.5em}Ihr sehr ergebner{\\[\baselineskip]}\spacefill\mbox{Arthur Schnitzler}\pend
           \leftskip=0em{}\selectlanguage{ngerman}\endnumbering\briefempfaengerindex{Adam, Robert@\textsc{Adam, Robert}!zzzSchnitzler, Arthur@\emph{von Arthur Schnitzler}!1915-07-201@{20. 7. 1915}|)be}\mylabel{L02216h}  \newcommand{\dateiname}{L02216}\newcommand{\titel}{Arthur Schnitzler an Robert Adam, 20. 7. 1915}\newcommand{\editorInnen}{Martin Anton Müller und Gerd-Hermann Susen}%% latex-leseansicht-abspann.tex
%% Abspann für die Leseansicht.
%% Der Schalter \ifkorrekturansicht ist bereits durch den Vorspann gesetzt.

%% latex-abspann.tex
%% Gemeinsamer Abspann für Korrekturansicht und Leseansicht.
%% Setzt den Schalter \ifkorrekturansicht voraus (gesetzt in den
%% einbindenden Dateien latex-korrekturansicht-abspann.tex bzw.
%% latex-leseansicht-abspann.tex).
%% ---------------------------------------------------------------

\normalsize

% Das esempio-Environment wird nur in der Leseansicht benötigt
\ifkorrekturansicht\else
\newenvironment{esempio}[3]%
{
    \vspace{1.5ex}
    \rlap{\underline{#1}}
    \par
    \setlength{\parindent}{0cm}
    \nopagebreak
    \leftskip=#2cm
    \rightskip=#3cm
}
{
    \par
}
\fi

\doendnotes{C}
\bigskip
\vfill

\clearpage

\footnotesize

\ifkorrekturansicht
  \lohead{\textsc{register}}
\fi

% theindex-Environment neu definieren ohne reledmac
\makeatletter
\renewenvironment{theindex}{%
  \ifkorrekturansicht
    \section*{\indexname}%
  \else
    \subsubsection*{Index der erwähnten Entitäten}%
  \fi
  \setlength{\parindent}{0pt}%
  \setlength{\parskip}{0pt plus 0.3pt}%
  \let\item\@idxitem
}{%
  \ifkorrekturansicht\clearpage\fi
}
\makeatother

\IfFileExists{\jobname-pw.ind}{\input{\jobname-pw.ind}}{}

% Quellenangabe nur in der Leseansicht
\ifkorrekturansicht\else
% Fallback-Definitionen, falls die .tex-Datei \titel etc. nicht gesetzt hat
\providecommand{\titel}{}
\providecommand{\editorInnen}{}
\providecommand{\dateiname}{\jobname}

\vspace{3cm}

\vfill

\footnotesize
\textsc{Quelle}: \titel. Herausgegeben von {\editorInnen}. In: \emph{Arthur Schnitzler: Briefwechsel mit Autorinnen und Autoren}.
 Digitale Edition, https://schnitzler-briefe.acdh.oeaw.ac.at/{\dateiname}.html (Stand \today)
\fi

\end{document}


