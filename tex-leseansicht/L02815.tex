%% latex-korrekturansicht-vorspann.tex
%% Vorspann für die Korrekturansicht.
%% Lädt die gemeinsame Datei latex-vorspann.tex mit gesetztem Schalter.

\newif\ifkorrekturansicht
\korrekturansichttrue

\input{../tex-inputs/latex-vorspann}


\section[ Paul Goldmann an Arthur Schnitzler, 18. 6. {[}1897{]}]{L02815 Paul Goldmann an Arthur Schnitzler, 18. 6. {[}1897{]}}
\nopagebreak\mylabel{L02815v}
\rehead{ }\normalsize\beginnumbering\briefempfaengerindex{Schnitzler, Arthur@\textsc{Schnitzler, Arthur}!zzzGoldmann, Paul@\emph{von Paul Goldmann}!1897-06-181@{18. 6. {[}1897{]}}|(be}
\toendnotes[C]{\smallbreak\pagebreak[2]}\Standort{DLA, A:Schnitzler, HS.NZ85.1.3167.}
\physDesc{Brief, 1 Blatt, 2 Seiten, 608 Zeichen
\newline{}Handschrift: blaue Tinte, deutsche Kurrent
\newline{}Beilage: aufgeklebtes Brieffragment, Handschrift Clementine Goldmann\pwindex{Goldmann, Clementine 1842-05-15 – 1924-02-24@\textsc{Goldmann, Clementine} (1842-05-15 – 1924-02-24)|pw}, blaue Tinte, deutsche
                                 Kurrentschrift; das schließende Anführungszeichen wurde von Paul Goldmann\pwindex{Goldmann, Paul 31.01.1865 – 25.09.1935@\textsc{Goldmann, Paul} (31.01.1865 – 25.09.1935), \emph{Schriftsteller/Schriftstellerin, Journalist/Journalistin}|pw} ergänzt 
\newline{}Schnitzler: 1) mit Bleistift das Jahr »97« vermerkt  2) mit rotem Buntstift eine Unterstreichung}\toendnotes[C]{\smallbreak}
\pstart
           {\pb}\textcolor{gray}{\textbf{\textbf{Frankfurter Zeitung\orgindex{Frankfurter Zeitung@Frankfurter Zeitung|pw}}}}\pend
           
\pstart
           \textcolor{gray}{\textbf{(\begin{otherlanguage}{french}Gazette de Francfort\end{otherlanguage}\orgindex{Frankfurter Zeitung@Frankfurter Zeitung|pw}).}}\pend
           
\pstart
           \textcolor{gray}{\textbf{\textbf{\begin{otherlanguage}{french}Fondateur M.\end{otherlanguage}{ }L. Sonnemann\pwindex{Sonnemann, Leopold 1831-10-29 – 1909-10-30@\textsc{Sonnemann, Leopold} (1831-10-29 – 1909-10-30), \emph{Journalist/Journalistin, Herausgeber/Herausgeberin}|pw}.}}}\pend
           
\pstart
           \begin{otherlanguage}{french}\textcolor{gray}{\textbf{Journal politique, financier,}}\end{otherlanguage}\pend
           
\pstart
           \begin{otherlanguage}{french}\textcolor{gray}{\textbf{commercial et littéraire.}}\end{otherlanguage}\pend
           
\pstart
           \begin{otherlanguage}{french}\textcolor{gray}{\textbf{\textbf{Paraissant trois fois par jour.}}}\end{otherlanguage}\hfill \textsc{Paris\oindex{Paris@\textbf{Paris}, \emph{P.PPLC}|pw}}, 18. Juni.\pend
           
\pstart
           \begin{otherlanguage}{french}\textcolor{gray}{\textbf{\textbf{Bureau à Paris\oindex{Paris@\textbf{Paris}, \emph{P.PPLC}|pw}}}}\end{otherlanguage}\pend
           
\pstart
           \begin{otherlanguage}{french}\textcolor{gray}{\textbf{\textbf{10 \so{Rue de la Bourse}\oindex{rue de la Bourse@\textbf{rue de la Bourse}, \emph{Straße (K.STR)}|pw}.}}}\end{otherlanguage}\pend
           
\pstart\center{}Mein lieber Freund,\pend\vspace{0.5em}
\pstart
           Das \label{K_L02815-1v}\edtext{Manuſkript des \textsc{Nansen\pwindex{Nansen, Peter 20.01.1861 – 31.07.1918@\textsc{Nansen, Peter} (20.01.1861 – 31.07.1918), \emph{Schriftsteller/Schriftstellerin, Journalist/Journalistin, Verleger/Verlegerin}|pw}}-Artikels\pwindex{?? [Artikel von Peter Nansen, Mai/Juni 1897]@\emph{?? [Artikel von Peter Nansen, Mai/Juni 1897]}|pwv}}{\lemma{\textnormal{\emph{Manuſkript des Nansen-Artikels}}}\Cendnote{\textnormal{Auch wenn letztlich nicht zu klären ist,
                  von welchem Text die Rede ist, dürfte der Umstand, dass Clementine Goldmann\pwindex{Goldmann, Clementine 1842-05-15 – 1924-02-24@\textsc{Goldmann, Clementine} (1842-05-15 – 1924-02-24)|pwk} im Besitz des Textes war und ihn an
                  ihren Bruder Fedor Mamroth\pwindex{Mamroth, Fedor 21.02.1851 – 25.06.1907@\textsc{Mamroth, Fedor} (21.02.1851 – 25.06.1907), \emph{Journalist/Journalistin, Kritiker/Kritikerin}|pwk} weiterreichte, so
                  zu lesen sein, dass es sich nicht um einen bei der \emph{Frankfurter Zeitung}\orgindex{Frankfurter Zeitung@Frankfurter Zeitung|pwk} eingereichten Beitrag handelte, da sie ihn sonst
                  zurückgegeben hätte. Weiters deutet das Wort »damals« darauf hin,
                  dass es sich schon vor einiger Zeit abgespielt hatte und also kein neuer Text Nansens\pwindex{Nansen, Peter 20.01.1861 – 31.07.1918@\textsc{Nansen, Peter} (20.01.1861 – 31.07.1918), \emph{Schriftsteller/Schriftstellerin, Journalist/Journalistin, Verleger/Verlegerin}|pwk} gemeint war. Vermutlich ist schlicht
                  von einer (nicht überlieferten) deutschen Übersetzung des Aufsatzes \emph{Arthur Schnitzler. »Elskovsleg«s Forfatter}\pwindex{Arthur Schnitzler. »Elskovsleg«s Forfatter@\emph{Arthur Schnitzler. »Elskovsleg«s Forfatter}|pwk} (–n–\pwindex{Nansen, Peter 20.01.1861 – 31.07.1918@\textsc{Nansen, Peter} (20.01.1861 – 31.07.1918), \emph{Schriftsteller/Schriftstellerin, Journalist/Journalistin, Verleger/Verlegerin}|pwkv} [ = Peter Nansen\pwindex{Nansen, Peter 20.01.1861 – 31.07.1918@\textsc{Nansen, Peter} (20.01.1861 – 31.07.1918), \emph{Schriftsteller/Schriftstellerin, Journalist/Journalistin, Verleger/Verlegerin}|pwk}], in: \emph{Politiken}\pwindex{Politiken@\emph{Politiken}|pwk}, Nr. 68, 9. 3. 1897,
                     S. 1) die Rede (siehe Paul Goldmann an Arthur Schnitzler, 11. 3. [1897]).}}}\label{K_L02815-1} ſcheint leider futſch zu ſein. Meine Mutter\pwindex{Goldmann, Clementine 1842-05-15 – 1924-02-24@\textsc{Goldmann, Clementine} (1842-05-15 – 1924-02-24)|pwv} ſchreibt mir:\pend
           
\pstart
           »An \textsc{Dr. Schnitzler} konnte ich leider\pend
           
\pstart
           {[}hs. :{]} das \textsc{Nansen\pwindex{Nansen, Peter 20.01.1861 – 31.07.1918@\textsc{Nansen, Peter} (20.01.1861 – 31.07.1918), \emph{Schriftsteller/Schriftstellerin, Journalist/Journalistin, Verleger/Verlegerin}|pw}}{ }Manuskript\pwindex{?? [Artikel von Peter Nansen, Mai/Juni 1897]@\emph{?? [Artikel von Peter Nansen, Mai/Juni 1897]}|pwv} nicht ſchicken;
               ich gab es damals Onkel \textsc{Fedor\pwindex{Mamroth, Fedor 21.02.1851 – 25.06.1907@\textsc{Mamroth, Fedor} (21.02.1851 – 25.06.1907), \emph{Journalist/Journalistin, Kritiker/Kritikerin}|pw}}, ohne es zurückzubekommen. –«\pend
           
\pstart
           {[}hs. :{]} Was alſo thun?\pend
           
\pstart
           Suche Dich doch ſo einzurichten, daß Du am 8. Auguſt
               nach \label{K_L02815-2v}\edtext{\textsc{Bayreuth\oindex{Bayreuth@\textbf{Bayreuth}, \emph{P.PPLA2}|pw}\orgindex{Bayreuther Festspiele@Bayreuther Festspiele|pwv}}}{\lemma{\textnormal{\emph{Bayreuth}}}\Cendnote{\textnormal{Siehe Paul Goldmann an Arthur Schnitzler, 15. 6. [1897].
               }}}\label{K_L02815-2} gehſt. Du, der Du nicht Berufsſklave biſt, wie ich, kannſt Dir doch eher
               Deine {\pb}Zeit eintheilen.\pend
           
\pstart
           Haſt Du dieſe Beſtie, den \textsc{Graf\pwindex{Graf, Max 01.10.1873 – 24.06.1958@\textsc{Graf, Max} (01.10.1873 – 24.06.1958), \emph{Kritiker/Kritikerin}|pwv}}, geſehen? Hat er irgendwelchen Geſtank in Bezug auf mich verurſacht?\pend
           
\pstart
           Wie geht es ſonſt Dir und ihr\pwindex{Reinhard, Marie 1871-03-13 – 1899-03-18@\textsc{Reinhard, Marie} (1871-03-13 – 1899-03-18), \emph{Gesangspädagoge/Gesangspädagogin}|pwv}?\pend
           
\pstart
           Schreib’ recht bald\textcolor{gray}{!}\pend
           
\pstart
           Ich begrüße Dich von Herzen {\\[\baselineskip]}Dein {\\[\baselineskip]}\spacefill\mbox{Paul Goldm}\pend
           \leftskip=0em{}\selectlanguage{ngerman}\endnumbering\briefempfaengerindex{Schnitzler, Arthur@\textsc{Schnitzler, Arthur}!zzzGoldmann, Paul@\emph{von Paul Goldmann}!1897-06-181@{18. 6. {[}1897{]}}|)be}\mylabel{L02815h}  \normalsize

\doendnotes{C}
\bigskip
\vfill

\clearpage

\footnotesize

\lohead{\textsc{register}}

% Definiere theindex-Environment komplett neu ohne reledmac
\makeatletter
\renewenvironment{theindex}{%
  \section*{\indexname}%
  \setlength{\parindent}{0pt}%
  \setlength{\parskip}{0pt plus 0.3pt}%
  \let\item\@idxitem
}{%
  \clearpage
}
\makeatother

\IfFileExists{\jobname-pw.ind}{\input{\jobname-pw.ind}}{}

\end{document}

      