%% latex-leseansicht-vorspann.tex
%% Vorspann für die Leseansicht.
%% Lädt die gemeinsame Datei latex-vorspann.tex mit nicht gesetztem Schalter.

\newif\ifkorrekturansicht
\korrekturansichtfalse

\input{../tex-inputs/latex-vorspann}


         
         \renewcommand{\erwaehntePersonen}{Personen: Clementine Goldmann, Max Graf, Fedor Mamroth, Peter Nansen, Marie Reinhard, Leopold Sonnemann}
         \renewcommand{\erwaehnteInstitutionen}{Institutionen: Bayreuther Festspiele, Frankfurter Zeitung}
         \renewcommand{\erwaehnteOrte}{Orte: Bayreuth, Paris, Wien, rue de la Bourse}
         \renewcommand{\erwaehnteWerke}{Werke: ?? [Artikel von Peter Nansen, Mai/Juni 1897], Arthur Schnitzler. »Elskovsleg«s Forfatter, Politiken}
               \section[ Paul Goldmann an Arthur Schnitzler, 18. 6. {[}1897{]}]{ Paul Goldmann an Arthur Schnitzler, 18. 6. {[}1897{]}}\nopagebreak\mylabel{v}\rehead{ }\begin{ledgroupsized}[t]{13cm}\normalsize\beginnumbering \toendnotes[C]{\smallbreak\pagebreak[2]} \Standort{DLA, A:Schnitzler, HS.NZ85.1.3167.}
\physDesc{Brief, 1 Blatt, 2 Seiten, 608 Zeichen
\newline{}Handschrift: blaue Tinte, deutsche Kurrent
\newline{}Beilage: aufgeklebtes Brieffragment, Handschrift Clementine Goldmann\pwindex{Goldmann, Clementine 1842-05-15 – 1924-02-24@\textsc{Goldmann, Clementine} (1842-05-15 – 1924-02-24)|pw}, blaue Tinte, deutsche
                                 Kurrentschrift; das schließende Anführungszeichen wurde von Paul Goldmann\pwindex{Goldmann, Paul 31.01.1865 – 25.09.1935@\textsc{Goldmann, Paul} (31.01.1865 – 25.09.1935), \emph{Schriftsteller, Journalist}|pw} ergänzt 
\newline{}Schnitzler: 1) mit Bleistift das Jahr »97« vermerkt  2) mit rotem Buntstift eine Unterstreichung}\toendnotes[C]{\smallbreak}\pstart
           \noindent{}{\pb}\textcolor{gray}{\textbf{\textbf{Frankfurter Zeitung\orgindex{Frankfurter Zeitung@Frankfurter Zeitung|pw}}}}\pend
           \pstart
           \textcolor{gray}{\textbf{(\begin{otherlanguage}{french}Gazette de Francfort\end{otherlanguage}\orgindex{Frankfurter Zeitung@Frankfurter Zeitung|pw}).}}\pend
           \pstart
           \textcolor{gray}{\textbf{\textbf{\begin{otherlanguage}{french}Fondateur M.\end{otherlanguage}{ }L. Sonnemann\pwindex{Sonnemann, Leopold 1831-10-29 – 1909-10-30@\textsc{Sonnemann, Leopold} (1831-10-29 – 1909-10-30), \emph{Journalist, Herausgeber}|pw}.}}}\pend
           \pstart
           \begin{otherlanguage}{french}\textcolor{gray}{\textbf{Journal politique, financier,}}\end{otherlanguage}\pend
           \pstart
           \begin{otherlanguage}{french}\textcolor{gray}{\textbf{commercial et littéraire.}}\end{otherlanguage}\pend
           \pstart
           \begin{otherlanguage}{french}\textcolor{gray}{\textbf{\textbf{Paraissant trois fois par jour.}}}\end{otherlanguage}\hfill \textsc{Paris\oindex{Paris@\textbf{Paris}|pw}}, 18. Juni.\pend
           \pstart
           \begin{otherlanguage}{french}\textcolor{gray}{\textbf{\textbf{Bureau à Paris\oindex{Paris@\textbf{Paris}|pw}}}}\end{otherlanguage}\pend
           \pstart
           \begin{otherlanguage}{french}\textcolor{gray}{\textbf{\textbf{10 \so{Rue de la Bourse}\oindex{rue de la Bourse@\textbf{rue de la Bourse}|pw}.}}}\end{otherlanguage}\pend
           \pstart\center{}Mein lieber Freund,\pend\pstart
           Das \label{K_L02815-1v}\edtext{Manuſkript des \textsc{Nansen\pwindex{Nansen, Peter 20.01.1861 – 31.07.1918@\textsc{Nansen, Peter} (20.01.1861 – 31.07.1918), \emph{Schriftsteller, Journalist, Verleger}|pw}}-Artikel\pwindex{Nansen, Peter 20.01.1861 – 31.07.1918@\textsc{Nansen, Peter} (20.01.1861 – 31.07.1918), \emph{Schriftsteller, Journalist, Verleger}!?? [Artikel von Peter Nansen, Mai/Juni 1897]Mai/Juni 1897@\strich\emph{?? [Artikel von Peter Nansen, Mai/Juni 1897]} {[}Mai/Juni 1897{]}|pwv}s}{\lemma{\textnormal{\emph{Manuſkript des Nansen-Artikels}}}\Cendnote{\textnormal{Auch wenn letztlich nicht zu klären ist,
                  von welchem Text die Rede war, dürfte der Umstand, dass Clementine Goldmann\pwindex{Goldmann, Clementine 1842-05-15 – 1924-02-24@\textsc{Goldmann, Clementine} (1842-05-15 – 1924-02-24)|pwk} im Besitz des Textes war und ihn an
                  ihren Bruder Fedor Mamroth\pwindex{Mamroth, Fedor 21.02.1851 – 25.06.1907@\textsc{Mamroth, Fedor} (21.02.1851 – 25.06.1907), \emph{Journalist, Kritiker}|pwk} weiterreichte, so
                  zu lesen sein, dass es sich nicht um einen bei der \emph{Frankfurter Zeitung}\orgindex{Frankfurter Zeitung@Frankfurter Zeitung|pwk} eingereichten Beitrag handelte, da sie ihn sonst
                  zurückgegeben hätte. Weiters deutet das Wort »damals« darauf hin,
                  dass es sich schon vor einiger Zeit abgespielt hatte und also kein neuer Text Nansen\pwindex{Nansen, Peter 20.01.1861 – 31.07.1918@\textsc{Nansen, Peter} (20.01.1861 – 31.07.1918), \emph{Schriftsteller, Journalist, Verleger}|pwk}s gemeint war. Vermutlich ist schlicht
                  von einer (nicht überlieferten) deutschen Übersetzung des Aufsatzes \emph{Arthur Schnitzler. »Elskovsleg«s Forfatter}\pwindex{Arthur Schnitzler. »Elskovsleg«s Forfatter1897-03-09@\emph{Arthur Schnitzler. »Elskovsleg«s Forfatter} {[}1897-03-09{]}|pwk} (–n–\pwindex{Nansen, Peter 20.01.1861 – 31.07.1918@\textsc{Nansen, Peter} (20.01.1861 – 31.07.1918), \emph{Schriftsteller, Journalist, Verleger}|pwkv} [ = Peter Nansen\pwindex{Nansen, Peter 20.01.1861 – 31.07.1918@\textsc{Nansen, Peter} (20.01.1861 – 31.07.1918), \emph{Schriftsteller, Journalist, Verleger}|pwk}], in: \emph{Politiken}\pwindex{?? Werk@Nicht ermittelte Verfasserinnen und Verfasser!Politiken1. 1. 1884@\emph{Politiken} {[}1. 1. 1884{]}|pwk}, Nr. 68, 9. 3. 1897,
                     S. 1) die Rede (siehe Paul Goldmann an Arthur Schnitzler, 11. 3. [1897]).}}}\label{K_L02815-1h} ſcheint leider futſch zu ſein. Meine Mutter\pwindex{Goldmann, Clementine 1842-05-15 – 1924-02-24@\textsc{Goldmann, Clementine} (1842-05-15 – 1924-02-24)|pwv} ſchreibt mir:\pend
           \pstart
           »An \textsc{Dr. Schnitzler} konnte ich leider\pend
           \pstart
           {[}hs. Clementine Goldmann:{]} das \textsc{Nansen\pwindex{Nansen, Peter 20.01.1861 – 31.07.1918@\textsc{Nansen, Peter} (20.01.1861 – 31.07.1918), \emph{Schriftsteller, Journalist, Verleger}|pw}}{ }Manuskript\pwindex{Nansen, Peter 20.01.1861 – 31.07.1918@\textsc{Nansen, Peter} (20.01.1861 – 31.07.1918), \emph{Schriftsteller, Journalist, Verleger}!?? [Artikel von Peter Nansen, Mai/Juni 1897]Mai/Juni 1897@\strich\emph{?? [Artikel von Peter Nansen, Mai/Juni 1897]} {[}Mai/Juni 1897{]}|pwv} nicht ſchicken;
               ich gab es damals Onkel \textsc{Fedor\pwindex{Mamroth, Fedor 21.02.1851 – 25.06.1907@\textsc{Mamroth, Fedor} (21.02.1851 – 25.06.1907), \emph{Journalist, Kritiker}|pw}}, ohne es zurückzubekommen. –«\pend
           \pstart
           {[}hs. Paul Goldmann:{]} Was alſo thun?\pend
           \pstart
           Suche Dich doch ſo einzurichten, daß Du am 8. Auguſt
               nach \label{K_L02815-2v}\edtext{\textsc{Bayreuth\oindex{Bayreuth@\textbf{Bayreuth}|pw}\orgindex{Bayreuther Festspiele@Bayreuther Festspiele|pwv}}}{\lemma{\textnormal{\emph{Bayreuth}}}\Cendnote{\textnormal{siehe Paul Goldmann an Arthur Schnitzler, 15. 6. [1897]}}}\label{K_L02815-2h} gehſt. Du, der Du nicht Berufsſklave biſt, wie ich, kannſt Dir doch eher
               Deine {\pb}Zeit eintheilen.\pend
           \pstart
           Haſt Du dieſe Beſtie, den \textsc{Graf\pwindex{Graf, Max 01.10.1873 – 24.06.1958@\textsc{Graf, Max} (01.10.1873 – 24.06.1958), \emph{Kritiker}|pwv}}, geſehen? Hat er irgendwelchen Geſtank in Bezug auf mich verurſacht?\pend
           \pstart
           Wie geht es ſonſt Dir und ihr\pwindex{Reinhard, Marie 1871-03-13 – 1899-03-18@\textsc{Reinhard, Marie} (1871-03-13 – 1899-03-18), \emph{Gesangspädagogin}|pwv}?\pend
           \pstart
           Schreib’ recht bald\textcolor{gray}{!}\pend
           \pstart
           Ich begrüße Dich von Herzen {\\[\baselineskip]}Dein {\\[\baselineskip]}\spacefill\mbox{Paul Goldm}\pend
           \leftskip=0em{}
         
         \endnumbering\mylabel{h}\end{ledgroupsized}  \newcommand{\dateiname}{L02815}\newcommand{\titel}{Paul Goldmann an Arthur Schnitzler, 18. 6. [1897]}\newcommand{\editorInnen}{Martin Anton Müller und Laura Untner}%% latex-leseansicht-abspann.tex
%% Abspann für die Leseansicht.
%% Der Schalter \ifkorrekturansicht ist bereits durch den Vorspann gesetzt.

%% latex-abspann.tex
%% Gemeinsamer Abspann für Korrekturansicht und Leseansicht.
%% Setzt den Schalter \ifkorrekturansicht voraus (gesetzt in den
%% einbindenden Dateien latex-korrekturansicht-abspann.tex bzw.
%% latex-leseansicht-abspann.tex).
%% ---------------------------------------------------------------

\normalsize

% Das esempio-Environment wird nur in der Leseansicht benötigt
\ifkorrekturansicht\else
\newenvironment{esempio}[3]%
{
    \vspace{1.5ex}
    \rlap{\underline{#1}}
    \par
    \setlength{\parindent}{0cm}
    \nopagebreak
    \leftskip=#2cm
    \rightskip=#3cm
}
{
    \par
}
\fi

\doendnotes{C}
\bigskip
\vfill

\clearpage

\footnotesize

\ifkorrekturansicht
  \lohead{\textsc{register}}
\fi

% theindex-Environment neu definieren ohne reledmac
\makeatletter
\renewenvironment{theindex}{%
  \ifkorrekturansicht
    \section*{\indexname}%
  \else
    \subsubsection*{Index der erwähnten Entitäten}%
  \fi
  \setlength{\parindent}{0pt}%
  \setlength{\parskip}{0pt plus 0.3pt}%
  \let\item\@idxitem
}{%
  \ifkorrekturansicht\clearpage\fi
}
\makeatother

\IfFileExists{\jobname-pw.ind}{\input{\jobname-pw.ind}}{}

% Quellenangabe nur in der Leseansicht
\ifkorrekturansicht\else
% Fallback-Definitionen, falls die .tex-Datei \titel etc. nicht gesetzt hat
\providecommand{\titel}{}
\providecommand{\editorInnen}{}
\providecommand{\dateiname}{\jobname}

\vspace{3cm}

\vfill

\footnotesize
\textsc{Quelle}: \titel. Herausgegeben von {\editorInnen}. In: \emph{Arthur Schnitzler: Briefwechsel mit Autorinnen und Autoren}.
 Digitale Edition, https://schnitzler-briefe.acdh.oeaw.ac.at/{\dateiname}.html (Stand \today)
\fi

\end{document}


      