%% latex-leseansicht-vorspann.tex
%% Vorspann für die Leseansicht.
%% Lädt die gemeinsame Datei latex-vorspann.tex mit nicht gesetztem Schalter.

\newif\ifkorrekturansicht
\korrekturansichtfalse

\input{../tex-inputs/latex-vorspann}


\section[ Paul Goldmann an Arthur Schnitzler, 18. 6. [1897]]{L02815 Paul Goldmann an Arthur Schnitzler,  18. 6. [1897]}
\nopagebreak\mylabel{L02815v}
\rehead{ }\normalsize\beginnumbering\briefempfaengerindex{Schnitzler, Arthur@\textsc{Schnitzler, Arthur}!zzzGoldmann, Paul@\emph{von Paul Goldmann}!1897-06-181@{18. 6. [1897]}|(be}
\toendnotes[C]{\smallbreak\pagebreak[2]}
\correspDesc{Versand  durch Paul Goldmann am 18. 6. [1897] in Paris
\newline{}Erhalt  durch Arthur Schnitzler im Zeitraum [19. 6. 1897
                  – 23. 6. 1897?] in Wien}\toendnotes[C]{\smallbreak}
\Standort{DLA, A:Schnitzler, HS.NZ85.1.3167.}
\physDesc{Brief, 1 Blatt, 2 Seiten, 608 Zeichen
\newline{}Handschrift: blaue Tinte, deutsche Kurrent
\newline{}Beilage: aufgeklebtes Brieffragment, Handschrift Clementine Goldmann\pwindex{Goldmann, Clementine 15.\,5.\,1842 Breslau – 24.\,2.\,1924 Frankfurt am Main@\textsc{Goldmann, Clementine} (15.\,5.\,1842 Breslau – 24.\,2.\,1924 Frankfurt am Main)|pw}, blaue Tinte, deutsche
                                 Kurrentschrift; das schließende Anführungszeichen wurde von Paul Goldmann\pwindex{Goldmann, Paul 31.\,1.\,1865 Breslau – 25.\,9.\,1935 Wien@\textsc{Goldmann, Paul} (31.\,1.\,1865 Breslau – 25.\,9.\,1935 Wien), \emph{Schriftsteller, Journalist}|pw} ergänzt 
\newline{}Schnitzler: 1) mit Bleistift das Jahr »97« vermerkt  2) mit rotem Buntstift eine Unterstreichung}\toendnotes[C]{\smallbreak}
\pstart
           {\pb}\textcolor{gray}{\textbf{\textbf{Frankfurter Zeitung\orgindex{Frankfurter Zeitung@Frankfurter Zeitung|pw}}}}\pend
           
\pstart
           \textcolor{gray}{\textbf{(\begin{otherlanguage}{french}Gazette de Francfort\end{otherlanguage}\orgindex{Frankfurter Zeitung@Frankfurter Zeitung|pw}).}}\pend
           
\pstart
           \textcolor{gray}{\textbf{\textbf{\begin{otherlanguage}{french}Fondateur M.\end{otherlanguage}{ }L. Sonnemann\pwindex{Sonnemann, Leopold 29.\,10.\,1831 Höchberg – 30.\,10.\,1909 Frankfurt am Main@\textsc{Sonnemann, Leopold} (29.\,10.\,1831 Höchberg – 30.\,10.\,1909 Frankfurt am Main), \emph{Journalist, Herausgeber}|pw}.}}}\pend
           
\pstart
           \begin{otherlanguage}{french}\textcolor{gray}{\textbf{Journal politique, financier,}}\end{otherlanguage}\pend
           
\pstart
           \begin{otherlanguage}{french}\textcolor{gray}{\textbf{commercial et littéraire.}}\end{otherlanguage}\pend
           
\pstart
           \begin{otherlanguage}{french}\textcolor{gray}{\textbf{\textbf{Paraissant trois fois par jour.}}}\end{otherlanguage}\hfill \textsc{Paris\oindex{Paris@\textbf{Paris}, \emph{Hauptstadt}|pw}}, 18. Juni.\pend
           
\pstart
           \begin{otherlanguage}{french}\textcolor{gray}{\textbf{\textbf{Bureau à Paris\oindex{Paris@\textbf{Paris}, \emph{Hauptstadt}|pw}}}}\end{otherlanguage}\pend
           
\pstart
           \begin{otherlanguage}{french}\textcolor{gray}{\textbf{\textbf{10 \so{Rue de la Bourse}\oindex{rue de la Bourse@\textbf{rue de la Bourse}, \emph{Straße}|pw}.}}}\end{otherlanguage}\pend
           
\pstart\center{}Mein lieber Freund,\pend\vspace{0.5em}
\pstart
           Das \label{K_L02815-1v}\edtext{Manuſkript des \textsc{Nansen\pwindex{Nansen, Peter 20.\,1.\,1861 Kopenhagen – 31.\,7.\,1918 Mariager@\textsc{Nansen, Peter} (20.\,1.\,1861 Kopenhagen – 31.\,7.\,1918 Mariager), \emph{Schriftsteller, Journalist, Verleger}|pw}}-Artikels\pwindex{Nansen, Peter 20.\,1.\,1861 Kopenhagen – 31.\,7.\,1918 Mariager@\textsc{Nansen, Peter} (20.\,1.\,1861 Kopenhagen – 31.\,7.\,1918 Mariager), \emph{Schriftsteller, Journalist, Verleger}!?? [Artikel von Peter Nansen, Mai/Juni 1897]@\strich\emph{?? [Artikel von Peter Nansen, Mai/Juni 1897]}|pwv}}{\lemma{\textnormal{\emph{Manuskript des Nansen-Artikels}}}\Cendnote{\textnormal{Auch wenn letztlich nicht zu klären ist,
                  von welchem Text die Rede ist, dürfte der Umstand, dass Clementine Goldmann\pwindex{Goldmann, Clementine 15.\,5.\,1842 Breslau – 24.\,2.\,1924 Frankfurt am Main@\textsc{Goldmann, Clementine} (15.\,5.\,1842 Breslau – 24.\,2.\,1924 Frankfurt am Main)|pwk} im Besitz des Textes war und ihn an
                  ihren Bruder Fedor Mamroth\pwindex{Mamroth, Fedor 21.\,2.\,1851 Breslau – 25.\,6.\,1907 Frankfurt am Main@\textsc{Mamroth, Fedor} (21.\,2.\,1851 Breslau – 25.\,6.\,1907 Frankfurt am Main), \emph{Journalist, Kritiker}|pwk} weiterreichte, so
                  zu lesen sein, dass es sich nicht um einen bei der \emph{Frankfurter Zeitung}\orgindex{Frankfurter Zeitung@Frankfurter Zeitung|pwk} eingereichten Beitrag handelte, da sie ihn sonst
                  zurückgegeben hätte. Weiters deutet das Wort »damals« darauf hin,
                  dass es sich schon vor einiger Zeit abgespielt hatte und also kein neuer Text Nansens\pwindex{Nansen, Peter 20.\,1.\,1861 Kopenhagen – 31.\,7.\,1918 Mariager@\textsc{Nansen, Peter} (20.\,1.\,1861 Kopenhagen – 31.\,7.\,1918 Mariager), \emph{Schriftsteller, Journalist, Verleger}|pwk} gemeint war. Vermutlich ist schlicht
                  von einer (nicht überlieferten) deutschen Übersetzung des Aufsatzes \emph{Arthur Schnitzler. »Elskovsleg«s Forfatter}\pwindex{Nansen, Peter 20.\,1.\,1861 Kopenhagen – 31.\,7.\,1918 Mariager@\textsc{Nansen, Peter} (20.\,1.\,1861 Kopenhagen – 31.\,7.\,1918 Mariager), \emph{Schriftsteller, Journalist, Verleger}!Arthur Schnitzler. »Elskovsleg«s Forfatter@\strich\emph{Arthur Schnitzler. »Elskovsleg«s Forfatter}|pwk} (–n–\pwindex{Nansen, Peter 20.\,1.\,1861 Kopenhagen – 31.\,7.\,1918 Mariager@\textsc{Nansen, Peter} (20.\,1.\,1861 Kopenhagen – 31.\,7.\,1918 Mariager), \emph{Schriftsteller, Journalist, Verleger}|pwkv} [ = Peter Nansen\pwindex{Nansen, Peter 20.\,1.\,1861 Kopenhagen – 31.\,7.\,1918 Mariager@\textsc{Nansen, Peter} (20.\,1.\,1861 Kopenhagen – 31.\,7.\,1918 Mariager), \emph{Schriftsteller, Journalist, Verleger}|pwk}], in: \emph{Politiken}\pwindex{Politiken@\emph{Politiken}|pwk}, Nr. 68, 9. 3. 1897,
                     S. 1) die Rede (siehe XXXX Auszeichnungsfehler: Dokument L02805 nicht gefunden).}}}\label{K_L02815-1}{ }ſcheint leider futſch zu{ }ſein. Meine Mutter\pwindex{Goldmann, Clementine 15.\,5.\,1842 Breslau – 24.\,2.\,1924 Frankfurt am Main@\textsc{Goldmann, Clementine} (15.\,5.\,1842 Breslau – 24.\,2.\,1924 Frankfurt am Main)|pwv}{ }ſchreibt mir:\pend
           
\pstart
           »An \textsc{Dr. Schnitzler} konnte ich leider\pend
           
\pstart
           {[}hs. Goldmann:{]} das \textsc{Nansen\pwindex{Nansen, Peter 20.\,1.\,1861 Kopenhagen – 31.\,7.\,1918 Mariager@\textsc{Nansen, Peter} (20.\,1.\,1861 Kopenhagen – 31.\,7.\,1918 Mariager), \emph{Schriftsteller, Journalist, Verleger}|pw}}{ }Manuskript\pwindex{Nansen, Peter 20.\,1.\,1861 Kopenhagen – 31.\,7.\,1918 Mariager@\textsc{Nansen, Peter} (20.\,1.\,1861 Kopenhagen – 31.\,7.\,1918 Mariager), \emph{Schriftsteller, Journalist, Verleger}!?? [Artikel von Peter Nansen, Mai/Juni 1897]@\strich\emph{?? [Artikel von Peter Nansen, Mai/Juni 1897]}|pwv} nicht{ }ſchicken;
               ich gab es damals Onkel \textsc{Fedor\pwindex{Mamroth, Fedor 21.\,2.\,1851 Breslau – 25.\,6.\,1907 Frankfurt am Main@\textsc{Mamroth, Fedor} (21.\,2.\,1851 Breslau – 25.\,6.\,1907 Frankfurt am Main), \emph{Journalist, Kritiker}|pw}}, ohne es zurückzubekommen. –«\pend
           
\pstart
           {[}hs. Goldmann:{]} Was alſo thun?\pend
           
\pstart
           Suche Dich doch{ }ſo einzurichten, daß Du am 8. Auguſt
               nach \label{K_L02815-2v}\edtext{\textsc{Bayreuth\oindex{Bayreuth@\textbf{Bayreuth}, \emph{Hauptstadt}|pw}\orgindex{Bayreuther Festspiele@Bayreuther Festspiele|pwv}}}{\lemma{\textnormal{\emph{Bayreuth}}}\Cendnote{\textnormal{Siehe XXXX Auszeichnungsfehler: Dokument L02814 nicht gefunden.
               }}}\label{K_L02815-2} gehſt. Du, der Du nicht Berufsſklave biſt, wie ich, kannſt Dir doch eher
               Deine {\pb}Zeit eintheilen.\pend
           
\pstart
           Haſt Du dieſe Beſtie, den \textsc{Graf\pwindex{Graf, Max 1.\,10.\,1873 Wien – 24.\,6.\,1958 ebd.@\textsc{Graf, Max} (1.\,10.\,1873 Wien – 24.\,6.\,1958 ebd.), \emph{Kritiker}|pwv}}, geſehen? Hat er irgendwelchen Geſtank in Bezug auf mich verurſacht?\pend
           
\pstart
           Wie geht es{ }ſonſt Dir und ihr\pwindex{Reinhard, Marie 13.\,3.\,1871 Wien – 18.\,3.\,1899 ebd.@\textsc{Reinhard, Marie} (13.\,3.\,1871 Wien – 18.\,3.\,1899 ebd.), \emph{Gesangspädagogin}|pwv}?\pend
           
\pstart
           Schreib’ recht bald\textcolor{gray}{!}\pend
           
\pstart
           Ich begrüße Dich von Herzen {\\[\baselineskip]}Dein {\\[\baselineskip]}\spacefill\mbox{Paul Goldm}\pend
           \leftskip=0em{}\selectlanguage{ngerman}\endnumbering\briefempfaengerindex{Schnitzler, Arthur@\textsc{Schnitzler, Arthur}!zzzGoldmann, Paul@\emph{von Paul Goldmann}!1897-06-181@{18. 6. [1897]}|)be}\mylabel{L02815h}  \newcommand{\dateiname}{L02815}\newcommand{\titel}{Paul Goldmann an Arthur Schnitzler, 18. 6. [1897]}\newcommand{\editorInnen}{Martin Anton Müller und Laura Untner}%% latex-leseansicht-abspann.tex
%% Abspann für die Leseansicht.
%% Der Schalter \ifkorrekturansicht ist bereits durch den Vorspann gesetzt.

%% latex-abspann.tex
%% Gemeinsamer Abspann für Korrekturansicht und Leseansicht.
%% Setzt den Schalter \ifkorrekturansicht voraus (gesetzt in den
%% einbindenden Dateien latex-korrekturansicht-abspann.tex bzw.
%% latex-leseansicht-abspann.tex).
%% ---------------------------------------------------------------

\normalsize

% Das esempio-Environment wird nur in der Leseansicht benötigt
\ifkorrekturansicht\else
\newenvironment{esempio}[3]%
{
    \vspace{1.5ex}
    \rlap{\underline{#1}}
    \par
    \setlength{\parindent}{0cm}
    \nopagebreak
    \leftskip=#2cm
    \rightskip=#3cm
}
{
    \par
}
\fi

\doendnotes{C}
\bigskip
\vfill

\clearpage

\footnotesize

\ifkorrekturansicht
  \lohead{\textsc{register}}
\fi

% theindex-Environment neu definieren ohne reledmac
\makeatletter
\renewenvironment{theindex}{%
  \ifkorrekturansicht
    \section*{\indexname}%
  \else
    \subsubsection*{Index der erwähnten Entitäten}%
  \fi
  \setlength{\parindent}{0pt}%
  \setlength{\parskip}{0pt plus 0.3pt}%
  \let\item\@idxitem
}{%
  \ifkorrekturansicht\clearpage\fi
}
\makeatother

\IfFileExists{\jobname-pw.ind}{\input{\jobname-pw.ind}}{}

% Quellenangabe nur in der Leseansicht
\ifkorrekturansicht\else
% Fallback-Definitionen, falls die .tex-Datei \titel etc. nicht gesetzt hat
\providecommand{\titel}{}
\providecommand{\editorInnen}{}
\providecommand{\dateiname}{\jobname}

\vspace{3cm}

\vfill

\footnotesize
\textsc{Quelle}: \titel. Herausgegeben von {\editorInnen}. In: \emph{Arthur Schnitzler: Briefwechsel mit Autorinnen und Autoren}.
 Digitale Edition, https://schnitzler-briefe.acdh.oeaw.ac.at/{\dateiname}.html (Stand \today)
\fi

\end{document}


