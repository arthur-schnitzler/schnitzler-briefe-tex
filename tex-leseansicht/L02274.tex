%% latex-leseansicht-vorspann.tex
%% Vorspann für die Leseansicht.
%% Lädt die gemeinsame Datei latex-vorspann.tex mit nicht gesetztem Schalter.

\newif\ifkorrekturansicht
\korrekturansichtfalse

\input{../tex-inputs/latex-vorspann}


\section[Georg Brandes an Arthur Schnitzler, 28. 9. 1917]{L02274 Georg Brandes an Arthur Schnitzler, 28. 9. 1917}
\nopagebreak\mylabel{L02274v}
\rehead{ }\normalsize\beginnumbering\briefempfaengerindex{Schnitzler, Arthur@\textsc{Schnitzler, Arthur}!zzzBrandes, Georg@\emph{von Georg Brandes}!1917-09-281@{28. 9. 1917}|(be}
\toendnotes[C]{\smallbreak\pagebreak[2]}
\correspDesc{Versand  durch Georg Brandes am 28. 9. 1917 in Kopenhagen
\newline{}Erhalt  durch Arthur Schnitzler im Zeitraum [29. 9. 1917
                  – 3. 10. 1917?] in Wien}\toendnotes[C]{\smallbreak}
\Standort{CUL, Schnitzler, B 17.}
\physDesc{Postkarte, 1106 Zeichen
\newline{}Handschrift: schwarze Tinte, lateinische Kurrent
\newline{}Versand: 1) Stempel: »\nobreak{}\oindex{Kopenhagen@\textbf{Kopenhagen}, \emph{Hauptstadt}|pwk}Kj\textcolor{gray}{øbenha}vn, 2{[}8. 9.{]} 17, 2–3 E\nobreak{}«.   2) Stempel: »\nobreak{}Zensuriert {[}k. u. k.{]} Zensurstelle Wien\orgindex{K. u. k. Zensurstelle@K. u. k. Zensurstelle|pw}\nobreak{}«. 
\newline{}Schnitzler: mit rotem Buntstift vereinzelte Unterstreichungen 
\newline{}Ordnung: mit Bleistift von unbekannter Hand nummeriert:
                                    »47« }
\buchAbdrucke{\weitereDrucke{Georg Brandes, Arthur Schnitzler: \emph{Ein Briefwechsel}. Herausgegeben von Kurt Bergel. Bern: \emph{Francke} 1956, S. 121.} }\toendnotes[C]{\smallbreak}\pstart{}{\pb}Herrn Dr. Arthur
                  Schnitzler\pend{}\pstart{}Sternwartestrasse 71\oindex{Wien@\textbf{Wien}!XVIII., Währing@\textbf{XVIII., Währing}!Sternwartestraße 71@\textbf{Sternwartestraße 71}, \emph{Wohngebäude}|pw}\pend{}\pstart{}Wien \textsubscript{XVIII}\oindex{XVIII., Währing@\textbf{XVIII., Währing}, \emph{Verwaltungsgebiet}|pw}\pend{}{\bigskip}\vspace{1em}
\pstart
           \raggedleft{}{\pb}Kopenhagen\oindex{Kopenhagen@\textbf{Kopenhagen}, \emph{Hauptstadt}|pw}{ }28 Sept. 17\pend
           \vspace{0.5em}
\pstart
           Verehrter Freund!\hspace*{3.5em}Es hat mich riesig gefreut, dass Sie auch in dieser
               traurigen Zeit an mich gedacht haben. Ich habe Ihr lächelnd-wehmütiges Buch\pwindex{Schnitzler, Arthur 15.\,5.\,1862 Wien – 21.\,10.\,1931 ebd.@\textsc{Schnitzler, Arthur} (15.\,5.\,1862 Wien – 21.\,10.\,1931 ebd.), \emph{Schriftsteller, Mediziner}!Doktor Gräsler, Badearzt@\strich\emph{Doktor Gräsler, Badearzt}|pwv} mit grossem Behagen
               gelesen und die Bekanntschaft mit zwei reizenden jungen Damen, genannt Sabine\pwindex{Schnitzler, Arthur 15.\,5.\,1862 Wien – 21.\,10.\,1931 ebd.@\textsc{Schnitzler, Arthur} (15.\,5.\,1862 Wien – 21.\,10.\,1931 ebd.), \emph{Schriftsteller, Mediziner}!Doktor Gräsler, Badearzt@\strich\emph{Doktor Gräsler, Badearzt}|pwv} und Katharina\pwindex{Schnitzler, Arthur 15.\,5.\,1862 Wien – 21.\,10.\,1931 ebd.@\textsc{Schnitzler, Arthur} (15.\,5.\,1862 Wien – 21.\,10.\,1931 ebd.), \emph{Schriftsteller, Mediziner}!Doktor Gräsler, Badearzt@\strich\emph{Doktor Gräsler, Badearzt}|pwv}, gemacht. Auch eine gewisse
               Lebensphilosophie, eine überlegene, ist in dem Buch\pwindex{Schnitzler, Arthur 15.\,5.\,1862 Wien – 21.\,10.\,1931 ebd.@\textsc{Schnitzler, Arthur} (15.\,5.\,1862 Wien – 21.\,10.\,1931 ebd.), \emph{Schriftsteller, Mediziner}!Doktor Gräsler, Badearzt@\strich\emph{Doktor Gräsler, Badearzt}|pwv}. Während Sie erfinderisch schöpfen, muss ich mich
               begnügen, geschichtliche Gestalten neu zu formen. Ich habe während des Krieges ein
                  Buch über Goethe\pwindex{Goethe, Johann Wolfgang von 28.\,8.\,1749 Frankfurt am Main – 22.\,3.\,1832 Weimar@\textsc{Goethe, Johann Wolfgang von} (28.\,8.\,1749 Frankfurt am Main – 22.\,3.\,1832 Weimar), \emph{Schriftsteller}|pw}\pwindex{Brandes, Georg 4.\,2.\,1842 Kopenhagen – 19.\,2.\,1927 ebd.@\textsc{Brandes, Georg} (4.\,2.\,1842 Kopenhagen – 19.\,2.\,1927 ebd.)!Wolfgang Goethe@\strich\emph{Wolfgang Goethe}|pwv} und eins über Voltaire\pwindex{Voltaire 21.\,11.\,1694 Paris – 30.\,5.\,1778 ebd.@\textsc{Voltaire} (21.\,11.\,1694 Paris – 30.\,5.\,1778 ebd.), \emph{Schriftsteller, Philosoph}|pw}\pwindex{Brandes, Georg 4.\,2.\,1842 Kopenhagen – 19.\,2.\,1927 ebd.@\textsc{Brandes, Georg} (4.\,2.\,1842 Kopenhagen – 19.\,2.\,1927 ebd.)!Voltaire und sein Jahrhundert@\strich\emph{Voltaire und sein Jahrhundert}|pwv} publiciert, beide in zwei Bänden, ausserdem einen Band über den Weltkrieg\pwindex{Brandes, Georg 4.\,2.\,1842 Kopenhagen – 19.\,2.\,1927 ebd.@\textsc{Brandes, Georg} (4.\,2.\,1842 Kopenhagen – 19.\,2.\,1927 ebd.)!Verdenskrigen [The World at War]@\strich\emph{Verdenskrigen [The World at War]}|pwv}, nur hier und in
                  Amerika\oindex{Amerika@\textbf{Amerika}|pw} als \uline{The World at War}\pwindex{Brandes, Georg 4.\,2.\,1842 Kopenhagen – 19.\,2.\,1927 ebd.@\textsc{Brandes, Georg} (4.\,2.\,1842 Kopenhagen – 19.\,2.\,1927 ebd.)!Verdenskrigen [The World at War]@\strich\emph{Verdenskrigen [The World at War]}|pw} erschienen. Seit 5 Monaten bin ich so närrisch, an einer grösseren Arbeit über
                  \uline{Cäsar}\pwindex{Caesar, Gaius Iulius 13.7.100? v. Chr. Rom – 15.3.44 v. Chr. ebd.@\textsc{Caesar, Gaius Iulius} (13.7.100? v. Chr. Rom – 15.3.44 v. Chr. ebd.), \emph{Politiker, Kaiser, Heerführer}|pw}\pwindex{Brandes, Georg 4.\,2.\,1842 Kopenhagen – 19.\,2.\,1927 ebd.@\textsc{Brandes, Georg} (4.\,2.\,1842 Kopenhagen – 19.\,2.\,1927 ebd.)!Gaius Julius Cæsar@\strich\emph{Gaius Julius Cæsar}|pwv} zu pfuschen. Die wird wol mehr {\pb}als ein Jahr noch nehmen. Ich
               denke oft an Wien\oindex{Wien@\textbf{Wien}, \emph{Verwaltungsgebiet}|pw} und an die Freunde dort. Seit
               wir uns im November 1912 sahen, ist Alles verändert. Ich kann kaum
               verstehen, dass es fast schon 5 Jahre her ist.\pend
           
\pstart
           Bitte sehr mich in der Erinnerung Ihrer Frau Gemahlin\pwindex{Schnitzler, Olga 17.\,1.\,1882 Wien – 13.\,1.\,1970 Lugano@\textsc{Schnitzler, Olga} (17.\,1.\,1882 Wien – 13.\,1.\,1970 Lugano), \emph{Schauspielerin, Sängerin}|pwv} und Beer-Hofmanns\pwindex{Beer-Hofmann, Richard 11.\,7.\,1866 Wien – 26.\,9.\,1945 New York City@\textsc{Beer-Hofmann, Richard} (11.\,7.\,1866 Wien – 26.\,9.\,1945 New York City), \emph{Schriftsteller}|pw}\pwindex{Beer-Hofmann, Paula 25.\,2.\,1879 Wien – 30.\,10.\,1939 Zürich@\textsc{Beer-Hofmann, Paula} (25.\,2.\,1879 Wien – 30.\,10.\,1939 Zürich)|pw} zurückzurufen.\pend
           
\pstart
           Ihr ganz ergebener{\\[\baselineskip]}\spacefill\mbox{Georg Brandes}\pend
           \leftskip=0em{}\selectlanguage{ngerman}\endnumbering\briefempfaengerindex{Schnitzler, Arthur@\textsc{Schnitzler, Arthur}!zzzBrandes, Georg@\emph{von Georg Brandes}!1917-09-281@{28. 9. 1917}|)be}\mylabel{L02274h}  \newcommand{\dateiname}{L02274}\newcommand{\titel}{Georg Brandes an Arthur Schnitzler, 28. 9. 1917}\newcommand{\editorInnen}{Martin Anton Müller und Gerd-Hermann Susen}%% latex-leseansicht-abspann.tex
%% Abspann für die Leseansicht.
%% Der Schalter \ifkorrekturansicht ist bereits durch den Vorspann gesetzt.

%% latex-abspann.tex
%% Gemeinsamer Abspann für Korrekturansicht und Leseansicht.
%% Setzt den Schalter \ifkorrekturansicht voraus (gesetzt in den
%% einbindenden Dateien latex-korrekturansicht-abspann.tex bzw.
%% latex-leseansicht-abspann.tex).
%% ---------------------------------------------------------------

\normalsize

% Das esempio-Environment wird nur in der Leseansicht benötigt
\ifkorrekturansicht\else
\newenvironment{esempio}[3]%
{
    \vspace{1.5ex}
    \rlap{\underline{#1}}
    \par
    \setlength{\parindent}{0cm}
    \nopagebreak
    \leftskip=#2cm
    \rightskip=#3cm
}
{
    \par
}
\fi

\doendnotes{C}
\bigskip
\vfill

\clearpage

\footnotesize

\ifkorrekturansicht
  \lohead{\textsc{register}}
\fi

% theindex-Environment neu definieren ohne reledmac
\makeatletter
\renewenvironment{theindex}{%
  \ifkorrekturansicht
    \section*{\indexname}%
  \else
    \subsubsection*{Index der erwähnten Entitäten}%
  \fi
  \setlength{\parindent}{0pt}%
  \setlength{\parskip}{0pt plus 0.3pt}%
  \let\item\@idxitem
}{%
  \ifkorrekturansicht\clearpage\fi
}
\makeatother

\IfFileExists{\jobname-pw.ind}{\input{\jobname-pw.ind}}{}

% Quellenangabe nur in der Leseansicht
\ifkorrekturansicht\else
% Fallback-Definitionen, falls die .tex-Datei \titel etc. nicht gesetzt hat
\providecommand{\titel}{}
\providecommand{\editorInnen}{}
\providecommand{\dateiname}{\jobname}

\vspace{3cm}

\vfill

\footnotesize
\textsc{Quelle}: \titel. Herausgegeben von {\editorInnen}. In: \emph{Arthur Schnitzler: Briefwechsel mit Autorinnen und Autoren}.
 Digitale Edition, https://schnitzler-briefe.acdh.oeaw.ac.at/{\dateiname}.html (Stand \today)
\fi

\end{document}


