%% latex-korrekturansicht-vorspann.tex
%% Vorspann für die Korrekturansicht.
%% Lädt die gemeinsame Datei latex-vorspann.tex mit gesetztem Schalter.

\newif\ifkorrekturansicht
\korrekturansichttrue

\input{../tex-inputs/latex-vorspann}


\section[Georg Brandes an Arthur Schnitzler, 28. 9. 1917]{L02274 Georg Brandes an Arthur Schnitzler, 28. 9. 1917}
\nopagebreak\mylabel{L02274v}
\rehead{ }\normalsize\beginnumbering\briefempfaengerindex{Schnitzler, Arthur@\textsc{Schnitzler, Arthur}!zzzBrandes, Georg@\emph{von Georg Brandes}!1917-09-281@{28. 9. 1917}|(be}
\toendnotes[C]{\smallbreak\pagebreak[2]}\Standort{CUL, Schnitzler, B 17.}
\physDesc{Postkarte, 1106 Zeichen
\newline{}Handschrift: schwarze Tinte, lateinische Kurrent
\newline{}Versand: 1) Stempel: »\nobreak{}\oindex{Kopenhagen@\textbf{Kopenhagen}, \emph{P.PPLC}|pwk}Kj\textcolor{gray}{øbenha}vn, 2{[}8. 9.{]} 17, 2–3 E\nobreak{}«.   2) Stempel: »\nobreak{}Zensuriert {[}k. u. k.{]} Zensurstelle Wien\orgindex{K. u. k. Zensurstelle@K. u. k. Zensurstelle|pw}\nobreak{}«. 
\newline{}Schnitzler: mit rotem Buntstift vereinzelte Unterstreichungen 
\newline{}Ordnung: mit Bleistift von unbekannter Hand nummeriert:
                                    »47« }
\buchAbdrucke{\weitereDrucke{Georg Brandes, Arthur Schnitzler: \emph{Ein Briefwechsel}. Bern: \emph{Francke} 1956, S. 121.} }\toendnotes[C]{\smallbreak}\pstart{}{\pb}Herrn Dr. Arthur
                  Schnitzler\pend{}\pstart{}Sternwartestrasse 71\oindex{Sternwartestrasse 71@\textbf{Sternwartestraße 71}, \emph{Wohngebäude (K.WHS)}|pw}\pend{}\pstart{}Wien \textsubscript{XVIII}\oindex{XVIII., Waehring@\textbf{XVIII., Währing}, \emph{A.ADM3}|pw}\pend{}{\bigskip}\vspace{1em}
\pstart
           \raggedleft{}{\pb}Kopenhagen\oindex{Kopenhagen@\textbf{Kopenhagen}, \emph{P.PPLC}|pw}{ }28 Sept. 17\pend
           \vspace{0.5em}
\pstart
           Verehrter Freund!\hspace*{3.5em}Es hat mich riesig gefreut, dass Sie auch in dieser
               traurigen Zeit an mich gedacht haben. Ich habe Ihr lächelnd-wehmütiges Buch\pwindex{Doktor Graesler, Badearzt@\emph{Doktor Gräsler, Badearzt}|pwv} mit grossem Behagen
               gelesen und die Bekanntschaft mit zwei reizenden jungen Damen, genannt Sabine\pwindex{Doktor Graesler, Badearzt@\emph{Doktor Gräsler, Badearzt}|pwv} und Katharina\pwindex{Doktor Graesler, Badearzt@\emph{Doktor Gräsler, Badearzt}|pwv}, gemacht. Auch eine gewisse
               Lebensphilosophie, eine überlegene, ist in dem Buch\pwindex{Doktor Graesler, Badearzt@\emph{Doktor Gräsler, Badearzt}|pwv}. Während Sie erfinderisch schöpfen, muss ich mich
               begnügen, geschichtliche Gestalten neu zu formen. Ich habe während des Krieges ein
                  Buch über Goethe\pwindex{Goethe, Johann Wolfgang von 1749-08-28 – 1832-03-22@\textsc{Goethe, Johann Wolfgang von} (1749-08-28 – 1832-03-22), \emph{Schriftsteller/Schriftstellerin}|pw}\pwindex{Wolfgang Goethe@\emph{Wolfgang Goethe}|pwv} und eins über Voltaire\pwindex{Voltaire 21.11.1694 – 30.05.1778@\textsc{Voltaire} (21.11.1694 – 30.05.1778), \emph{Schriftsteller/Schriftstellerin, Philosoph/Philosophin}|pw}\pwindex{Voltaire und sein Jahrhundert@\emph{Voltaire und sein Jahrhundert}|pwv} publiciert, beide in zwei Bänden, ausserdem einen Band über den Weltkrieg\pwindex{Verdenskrigen [The World at War]@\emph{Verdenskrigen [The World at War]}|pwv}, nur hier und in
                  Amerika\oindex{Amerika@\textbf{Amerika}, \emph{kein passender Code gefunden}|pw} als \uline{The World at War}\pwindex{Verdenskrigen [The World at War]@\emph{Verdenskrigen [The World at War]}|pw} erschienen. Seit 5 Monaten bin ich so närrisch, an einer grösseren Arbeit über
                  \uline{Cäsar}\pwindex{Caesar, Gaius Iulius 13.7.100? v. Chr. – 15.3.44 v. Chr.@\textsc{Caesar, Gaius Iulius} (13.7.100? v. Chr. – 15.3.44 v. Chr.), \emph{Politiker/Politikerin, Kaiser/Kaiserin, Heerführer/Heerführerin}|pw}\pwindex{Gaius Julius Cæsar@\emph{Gaius Julius Cæsar}|pwv} zu pfuschen. Die wird wol mehr {\pb}als ein Jahr noch nehmen. Ich
               denke oft an Wien\oindex{Wien@\textbf{Wien}, \emph{A.ADM2}|pw} und an die Freunde dort. Seit
               wir uns im November 1912 sahen, ist Alles verändert. Ich kann kaum
               verstehen, dass es fast schon 5 Jahre her ist.\pend
           
\pstart
           Bitte sehr mich in der Erinnerung Ihrer Frau Gemahlin\pwindex{Schnitzler, Olga 17.01.1882 – 13.01.1970@\textsc{Schnitzler, Olga} (17.01.1882 – 13.01.1970), \emph{Schauspieler/Schauspielerin, Sänger/Sängerin}|pwv} und Beer-Hofmanns\pwindex{Beer-Hofmann, Richard 1866-07-11 – 1945-09-26@\textsc{Beer-Hofmann, Richard} (1866-07-11 – 1945-09-26), \emph{Schriftsteller/Schriftstellerin}|pw}\pwindex{Beer-Hofmann, Paula 25.02.1879 – 30.10.1939@\textsc{Beer-Hofmann, Paula} (25.02.1879 – 30.10.1939)|pw} zurückzurufen.\pend
           
\pstart
           Ihr ganz ergebener{\\[\baselineskip]}\spacefill\mbox{Georg Brandes}\pend
           \leftskip=0em{}\selectlanguage{ngerman}\endnumbering\briefempfaengerindex{Schnitzler, Arthur@\textsc{Schnitzler, Arthur}!zzzBrandes, Georg@\emph{von Georg Brandes}!1917-09-281@{28. 9. 1917}|)be}\mylabel{L02274h}  \normalsize

\doendnotes{C}
\bigskip
\vfill

\clearpage

\footnotesize

\lohead{\textsc{register}}

% Definiere theindex-Environment komplett neu ohne reledmac
\makeatletter
\renewenvironment{theindex}{%
  \section*{\indexname}%
  \setlength{\parindent}{0pt}%
  \setlength{\parskip}{0pt plus 0.3pt}%
  \let\item\@idxitem
}{%
  \clearpage
}
\makeatother

\IfFileExists{\jobname-pw.ind}{\input{\jobname-pw.ind}}{}

\end{document}

      