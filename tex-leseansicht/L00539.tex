%% latex-korrekturansicht-vorspann.tex
%% Vorspann für die Korrekturansicht.
%% Lädt die gemeinsame Datei latex-vorspann.tex mit gesetztem Schalter.

\newif\ifkorrekturansicht
\korrekturansichttrue

\input{../tex-inputs/latex-vorspann}


\section[Arthur Schnitzler an Richard Beer-Hofmann, 14. 3. 1896]{L00539 Arthur Schnitzler an Richard Beer-Hofmann, 14. 3. 1896}
\nopagebreak\mylabel{L00539v}
\rehead{ }\normalsize\beginnumbering\briefempfaengerindex{Beer-Hofmann, Richard@\textsc{Beer-Hofmann, Richard}!zzzSchnitzler, Arthur@\emph{von Arthur Schnitzler}!1896-03-141@{14. 3. 1896}|(be}
\toendnotes[C]{\smallbreak\pagebreak[2]}\Standort{Privatbesitz, Peter Michael Braunwarth, \emph{ohne Signatur}.}
\physDesc{Brief, 1 Blatt, 2 Seiten, 758 Zeichen
\newline{}Handschrift: schwarze Tinte, deutsche Kurrent}
\buchAbdrucke{\weitereDrucke{\emph{Die Presse}, 4. 5. 2002, Sec. Spectrum, S. II.} }
\pstart
           \noindent{}{\pb}lieber Richard, hätt ich nicht gewußt, daſs Sie meinen Brief ſo
               nehmen wie er geſchrieben iſt, ſo hätte ich ihn ja nicht geſchrieben. Aber ſo war’s
               wieder nicht gemeint, daſs Sie ſich einbilden \introOben{}müſſen\introOben{}, das
               Schreiben mit der Zeit ganz ſein zu laſſen. Wo wär ich heute, we{\geminationn} mich irgend was misglücktes i{\geminationm}er dahin gebracht hätte. I{\geminationm}erhin gefällt mir Ihre Idee, ſchöne fremde Sachen gut zu überſetzen, ausnehmend.
               Vielleicht wird es einen Weg für Sie bedeuten, der Sie zu Ihnen ſelbſt führt.\pend
           
\pstart
           {\pb}Ich schließe die gewünſchte Karte für Paul Goldmann\pwindex{Goldmann, Paul 31.01.1865 – 25.09.1935@\textsc{Goldmann, Paul} (31.01.1865 – 25.09.1935), \emph{Schriftsteller/Schriftstellerin, Journalist/Journalistin}|pw} bei; grüßen Sie ihn auch mündlich
               aufs herzlichſte von mir. Sie bald im Bild zu ſehn, freut mich, Ihnen in kurzer Zeit
               perſönlich die Hand drücken zu kö{\geminationn}en, freut mich noch
               viel mehr.\pend
           \pstart Herzlich der Ihre, \spacefill\mbox{ArthSchn}\pend{}
\pstart
           Wien\oindex{Wien@\textbf{Wien}, \emph{A.ADM2}|pw}{ }14. 3. 96.\pend
           \selectlanguage{ngerman}\endnumbering\briefempfaengerindex{Beer-Hofmann, Richard@\textsc{Beer-Hofmann, Richard}!zzzSchnitzler, Arthur@\emph{von Arthur Schnitzler}!1896-03-141@{14. 3. 1896}|)be}\mylabel{L00539h}  \normalsize

\doendnotes{C}
\bigskip
\vfill

\clearpage

\footnotesize

\lohead{\textsc{register}}

% Definiere theindex-Environment komplett neu ohne reledmac
\makeatletter
\renewenvironment{theindex}{%
  \section*{\indexname}%
  \setlength{\parindent}{0pt}%
  \setlength{\parskip}{0pt plus 0.3pt}%
  \let\item\@idxitem
}{%
  \clearpage
}
\makeatother

\IfFileExists{\jobname-pw.ind}{\input{\jobname-pw.ind}}{}

\end{document}

      