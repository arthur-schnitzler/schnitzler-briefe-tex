%% latex-leseansicht-vorspann.tex
%% Vorspann für die Leseansicht.
%% Lädt die gemeinsame Datei latex-vorspann.tex mit nicht gesetztem Schalter.

\newif\ifkorrekturansicht
\korrekturansichtfalse

\input{../tex-inputs/latex-vorspann}


               \section[Arthur Schnitzler an Richard Beer-Hofmann, 14. 3. 1896]{ Arthur Schnitzler an Richard Beer-Hofmann, 14. 3. 1896}\nopagebreak\mylabel{v}\rehead{ }\begin{ledgroupsized}[t]{13cm}\normalsize\beginnumbering\briefempfaengerindex{Beer-Hofmann, Richard@\textsc{Beer-Hofmann, Richard}!zzzSchnitzler, Arthur@\emph{von Arthur Schnitzler}!1896-03-141@{14. 3. 1896}|(be} \toendnotes[C]{\smallbreak\pagebreak[2]} \Standort{Privatbesitz, Peter Michael Braunwarth, \emph{ohne Signatur}.}
\physDesc{Brief, 1 Blatt, 2 Seiten
\newline{}Handschrift: schwarze Tinte, deutsche Kurrent}\buchAbdrucke{\weitereDrucke{Peter Michael Braunwarth: \emph{»Wo wär ich heute«.} In: \emph{Die Presse}, 4. 5. 2002, Sec. Spectrum, S. II.} }\pstart
           \noindent{}{\pb}lieber Richard, hätt ich nicht gewußt, daſs Sie meinen Brief ſo
               nehmen wie er geſchrieben iſt, ſo hätte ich ihn ja nicht geſchrieben. Aber ſo war’s
               wieder nicht gemeint, daſs Sie ſich einbilden \introOben{}müſſen\introOben{}, das
               Schreiben mit der Zeit ganz ſein zu laſſen. Wo wär ich heute, we{\geminationn} mich irgend was misglücktes i{\geminationm}er dahin gebracht hätte. I{\geminationm}erhin gefällt mir Ihre Idee, ſchöne fremde Sachen gut zu überſetzen, ausnehmend.
               Vielleicht wird es einen Weg für Sie bedeuten, der Sie zu Ihnen ſelbſt führt.\pend
           \pstart
           {\pb}Ich schließe die gewünſchte Karte für Paul Goldmann\pwindex{Goldmann, Paul 31.01.1865 – 25.09.1935@\textsc{Goldmann, Paul} (31.01.1865 – 25.09.1935), \emph{Schriftsteller, Journalist}|pw} bei; grüßen Sie ihn auch mündlich
               aufs herzlichſte von mir. Sie bald im Bild zu ſehn, freut mich, Ihnen in kurzer Zeit
               perſönlich die Hand drücken zu kö{\geminationn}en, freut mich noch
               viel mehr.\pend
           \pstart Herzlich der Ihre, \spacefill\mbox{ArthSchn}\pend{}\pstart
           Wien\oindex{Wien@\textbf{Wien}|pw}{ }14. 3. 96.\pend
                     \endnumbering\briefempfaengerindex{Beer-Hofmann, Richard@\textsc{Beer-Hofmann, Richard}!zzzSchnitzler, Arthur@\emph{von Arthur Schnitzler}!1896-03-141@{14. 3. 1896}|)be}\mylabel{h}\end{ledgroupsized}  \newcommand{\dateiname}{L00539}\newcommand{\titel}{Arthur Schnitzler an Richard Beer-Hofmann, 14. 3. 1896}\newcommand{\editorInnen}{Martin Anton Müller und Gerd-Hermann Susen}%% latex-leseansicht-abspann.tex
%% Abspann für die Leseansicht.
%% Der Schalter \ifkorrekturansicht ist bereits durch den Vorspann gesetzt.

%% latex-abspann.tex
%% Gemeinsamer Abspann für Korrekturansicht und Leseansicht.
%% Setzt den Schalter \ifkorrekturansicht voraus (gesetzt in den
%% einbindenden Dateien latex-korrekturansicht-abspann.tex bzw.
%% latex-leseansicht-abspann.tex).
%% ---------------------------------------------------------------

\normalsize

% Das esempio-Environment wird nur in der Leseansicht benötigt
\ifkorrekturansicht\else
\newenvironment{esempio}[3]%
{
    \vspace{1.5ex}
    \rlap{\underline{#1}}
    \par
    \setlength{\parindent}{0cm}
    \nopagebreak
    \leftskip=#2cm
    \rightskip=#3cm
}
{
    \par
}
\fi

\doendnotes{C}
\bigskip
\vfill

\clearpage

\footnotesize

\ifkorrekturansicht
  \lohead{\textsc{register}}
\fi

% theindex-Environment neu definieren ohne reledmac
\makeatletter
\renewenvironment{theindex}{%
  \ifkorrekturansicht
    \section*{\indexname}%
  \else
    \subsubsection*{Index der erwähnten Entitäten}%
  \fi
  \setlength{\parindent}{0pt}%
  \setlength{\parskip}{0pt plus 0.3pt}%
  \let\item\@idxitem
}{%
  \ifkorrekturansicht\clearpage\fi
}
\makeatother

\IfFileExists{\jobname-pw.ind}{\input{\jobname-pw.ind}}{}

% Quellenangabe nur in der Leseansicht
\ifkorrekturansicht\else
% Fallback-Definitionen, falls die .tex-Datei \titel etc. nicht gesetzt hat
\providecommand{\titel}{}
\providecommand{\editorInnen}{}
\providecommand{\dateiname}{\jobname}

\vspace{3cm}

\vfill

\footnotesize
\textsc{Quelle}: \titel. Herausgegeben von {\editorInnen}. In: \emph{Arthur Schnitzler: Briefwechsel mit Autorinnen und Autoren}.
 Digitale Edition, https://schnitzler-briefe.acdh.oeaw.ac.at/{\dateiname}.html (Stand \today)
\fi

\end{document}


      