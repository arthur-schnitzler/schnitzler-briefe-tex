%% latex-korrekturansicht-vorspann.tex
%% Vorspann für die Korrekturansicht.
%% Lädt die gemeinsame Datei latex-vorspann.tex mit gesetztem Schalter.

\newif\ifkorrekturansicht
\korrekturansichttrue

\input{../tex-inputs/latex-vorspann}


\section[Arthur Schnitzler an Georg Brandes, 12. 1. 1899]{L00880 Arthur Schnitzler an Georg Brandes, 12. 1. 1899}
\nopagebreak\mylabel{L00880v}
\rehead{ }\normalsize\beginnumbering\briefempfaengerindex{Brandes, Georg@\textsc{Brandes, Georg}!zzzSchnitzler, Arthur@\emph{von Arthur Schnitzler}!1899-01-121@{12. 1. 1899}|(be}
\toendnotes[C]{\smallbreak\pagebreak[2]}\Standort{Kopenhagen, Det Kongelige Bibliotek, Georg Brandes Arkiv, box 125.}
\physDesc{Brief, 3 Blätter, 11 Seiten, 4515 Zeichen
\newline{}Handschrift: schwarze Tinte, deutsche Kurrent
\newline{}Ordnung: mit Bleistift von unbekannter Hand beschriftet:
                                    »Schnitzler 12. 1.99.« und mit
                                 Bleistift nummeriert »13.«, das zweite Blatt mit
                                    »2« versehen und auf dieses und das dritte erneut
                                 das Datum vermerkt: »12/1 99« }
\buchAbdrucke{\weitereDrucke{1) Georg Brandes, Arthur Schnitzler: \emph{Ein Briefwechsel}. Bern: \emph{Francke} 1956, S. 70–72.} \weitereDrucke{2) Arthur Schnitzler: \emph{Briefe 1875–1912}. Frankfurt am Main: \emph{S. Fischer} 1981, S. 366–368.} }\toendnotes[C]{\smallbreak}
\pstart{}{\pb}Verehrter Herr Brandes,\pend\vspace{0.5em}
\pstart
           geſtern hab ich Ihren Brief bekommen und aus dem erfahren, dſs Sie wieder zu Bette
               liegen. Abends ſtand es \label{K_L00880-1v}\edtext{in einer Berliner Zeitung\pwindex{Berliner Tageblatt@\emph{Berliner Tageblatt}|pwv} zu leſen}{\lemma{\textnormal{\emph{in … leſen}}}\Cendnote{\textnormal{Vgl. 
                     V. A.\pwindex{V. A. @\textsc{V. A.}, \emph{Journalist/Journalistin}|pwk}: \emph{Bei Georg Brandes}\pwindex{Bei Georg Brandes@\emph{Bei Georg Brandes}|pwk}. In: \emph{Berliner
                        Tageblatt}\pwindex{Berliner Tageblatt@\emph{Berliner Tageblatt}|pwk}, Jg. 28, Nr. 16, 9. 1. 1899, Abend-Ausgabe,
                     S. 3: »Aus \so{Kopenhagen}\oindex{Kopenhagen@\textbf{Kopenhagen}, \emph{P.PPLC}|pw}{ }schreibt uns unser dortiger Korrespondent:
                           Dr. Georg Brandes\pwindex{Brandes, Georg 04.02.1842 – 19.02.1927@\textsc{Brandes, Georg} (04.02.1842 – 19.02.1927)|pw} muß leider
                        wieder das Bett hüten und zwar wegen seines alten Leidens: Venenentzündung.
                        Ich besuchte gestern den berühmten Autor. {[}{\dots}{]} ›Und jetzt liege ich hier seit drei Wochen auf meinem
                        Schmerzenslager,‹ sagte Brandes\pwindex{Brandes, Georg 04.02.1842 – 19.02.1927@\textsc{Brandes, Georg} (04.02.1842 – 19.02.1927)|pw} mit
                        einem matten Lächeln; ›wann und wie die Aerzte mir wieder auf die Beine
                        helfen können, wissen sie ja selber nicht.‹{ / }{[}{\dots}{]} Eine Besserung ist jedoch augenscheinlich eingetreten, welche
                        hoffentlich fortschreiten wird.«}}}\label{K_L00880-1}, mit dem Beiſatz, dſs Sie ſich ſchon auf dem Weg der Beſſerung
               befinden. Ich hoffe, daſs es ſich ſo verhält und daſs Sie bald ganz geſund \substVorne{}\textsuperscript{iſt}\substDazwischen{}ſind\substHinten{}. Meine innigſten Wünſche ſind bei Ihnen, {\pb}das wiſſen Sie. Auch von Ihrem Streit mit den \label{K_L00880-2v}\edtext{Deutſchen hab ich durch die Zeitung}{\lemma{\textnormal{\emph{Deutſchen … Zeitung}}}\Cendnote{\textnormal{Vgl. [O. V.]: \emph{Köllers
                        Erfolge}\pwindex{Koellers Erfolge@\emph{Köllers Erfolge}|pwk}. In: \emph{Berliner Tageblatt}\pwindex{Berliner Tageblatt@\emph{Berliner Tageblatt}|pwk},
                     Jg. 28, Nr. 9, 5. 1. 1899, Abend-Ausgabe, S. 2: »\so{Georg Brandes}\pwindex{Brandes, Georg 04.02.1842 – 19.02.1927@\textsc{Brandes, Georg} (04.02.1842 – 19.02.1927)|pw}, der vom ›\so{Verein Berliner Presse}\orgindex{Verein Berliner Presse@Verein Berliner Presse|pw}‹ aufgefordert worden war, nach Berlin\oindex{Berlin@\textbf{Berlin}, \emph{P.PPLC}|pw}
                     zu kommen, um einen Vortrag zum Besten der Hilfskasse des genannten Vereins zu
                     halten, hat geantwortet, daß ein \so{dänischer}\oindex{Daenemark@\textbf{Dänemark}, \emph{A.PCLI}|pw}\so{{ }Autor} während der gegenwärtigen Verhältnisse in Nordschleswig\oindex{Nordschleswig@\textbf{Nordschleswig}, \emph{Teil eines Landes (A.LNDX)}|pw} unmöglich Vorträge in Berlin\oindex{Berlin@\textbf{Berlin}, \emph{P.PPLC}|pw} halten könne.«}}}\label{K_L00880-2}
               erfahren; Sie ſollen irgend einen Vortrag abgeſagt haben, im Verein »Berliner Preſſe\orgindex{Verein Berliner Presse@Verein Berliner Presse|pw}«, aus »polit. Gründen«. Fügen Sie
               Ihren Antipathien gegen \strikeout{De}Preußen\oindex{Preussen@\textbf{Preußen}, \emph{Land (A.LND)}|pw} und Frankreich\oindex{Frankreich@\textbf{Frankreich}, \emph{A.PCLI}|pw} nur getroſt \introOben{}die\introOben{} gegen Oeſterreich\oindex{Oesterreich@\textbf{Österreich}, \emph{A.PCLI}|pw} bei. Leſen Sie manchmal Wien\oindex{Wien@\textbf{Wien}, \emph{A.ADM2}|pw}er Zeitungen, Parlaments- und
               Gemeinderathsberichte? Es iſt ſtaunenswerth, unter was für Schweinen wir hier
               leben; – und {\pb}ich denke i{\geminationm}er, ſelbſt Antiſemiten müßte es doch auffallen, daſs
               der Antiſemitismus – von allem andern abgeſehen – jedenfalls die ſonderbare Kraft
               hat, die verlogenſten Gemeinheiten der menſchlichen Natur zu Tage zu fördern und ſie
               aufs höchſte auszubilden. Wie merkwürdig, daſs ſogar die offenbaren Mängel, Fehler,
               meinetwegen Verbrechen der Judenpreſſe, die man als ſo ſpezifiſch jüdiſch hinſtellen
               wollte, von der Antiſemiten{\pb}preſſe ins
               ungeheuerliche ausgebildet worden ſind. Aber wir wollen über dieſe widerlichen Dinge
               lieber gar nicht reden.\pend
           
\pstart
           Ich freue mich, dſs das »Vermächtnis\pwindex{Vermaechtnis. Schauspiel in drei Akten@\emph{Das Vermächtnis. Schauspiel in drei Akten}|pw}« einigen
               Beifall bei Ihnen gefunden hat. Mir ſelbſt iſt nur der erſte Akt lieb; dann gewiſſe
               Partien des letzten. Solange die Hauptperſon auf der Scene iſt, hab ich das Stück
               nicht gern. Die iſt ganz unperſönlich geblieben find ich. Während der Proben fiel mir
               mancherlei ein, wodurch ich das Stück hätte höher bringen können; vor allem hätt ich
               das Kind {\pb}müſſen am Leben laſſen; – aber es
               ſcheint ich bin nicht anſtändig genug, um ein Stück noch auf der Probe zurückzuziehn,
               ſelbſt we{\geminationn} ich weiſs, wie es beſſer zu machen wäre. Es
               hat in Berlin\oindex{Berlin@\textbf{Berlin}, \emph{P.PPLC}|pw} un\textcolor{gray}{d}{ }Wien\oindex{Wien@\textbf{Wien}, \emph{A.ADM2}|pw} bei der Erſtaufführung viel Erfolg gehabt; in
                  Berlin\oindex{Berlin@\textbf{Berlin}, \emph{P.PPLC}|pw} verſchwand es bald; hier ſcheint es
               ſich zu halten. Irgend eine Zukunft hat es gewiſs nicht – und wahrhaftig nicht nur
               wegen ſeiner Traurigkeit –! – Nun hab ich was geſchrieben, das mir lieber iſt; drei
               kleine Stücke\pwindex{Gefaehrtin. Schauspiel in einem Akt@\emph{Die Gefährtin. Schauspiel in einem Akt}|pwv}\pwindex{Paracelsus. Versspiel in einem Akt@\emph{Paracelsus. Versspiel in einem Akt}|pwv}\pwindex{gruene Kakadu. Groteske in einem Akt@\emph{Der grüne Kakadu. Groteske in einem Akt}|pwv}, von denen das {\pb}eine »Der grüne Kakadu\pwindex{gruene Kakadu. Groteske in einem Akt@\emph{Der grüne Kakadu. Groteske in einem Akt}|pw}«, das beſte, großen
               Schwierigkeiten begegnet. In Berlin\oindex{Berlin@\textbf{Berlin}, \emph{P.PPLC}|pw} haben ſie es
               verboten; – hier will die Hofcenſur die unmöglichſten Aenderungen. Es ſpielt am Abend
               der Baſtille\oindex{Bastille@\textbf{Bastille}, \emph{Gebäude (K.GBD)}|pw}nerſtürmung zu Paris\oindex{Paris@\textbf{Paris}, \emph{P.PPLC}|pw} – aber ich ſoll den »Blutgeruch« herausſtreichen. Auch
               daſs ein Herzog umgebracht wird, will den Leuten nicht gefallen. Ich freu mich Ihnen
               das Ding bald zu ſchicken; es wird Sie wahrſcheinlich amuſiren.\pend
           
\pstart
           Und jetzt bin ich mit einer ganz phantaſtiſchen {\pb}fünfactigen Sache\pwindex{Schleier der Beatrice. Schauspiel in fuenf Akten@\emph{Der Schleier der Beatrice. Schauspiel in fünf Akten}|pwv}
               beſchäftigt; mir ſcheint überhaupt als käme ich jetzt in andere Gegenden. Wer weiſs,
               ob alles bisherige nicht doch nur Tagebuch war; wenigſtens von einer gewiſſen Zeit
               an. (Denn früher einmal, von meinem 9. bis zu meinem 20. Jahr hab ich geſchrieben,
               »wie der Vogel ſingt« – ich muſs damals ſehr glücklich geweſen ſein; de{\geminationn} ich eri{\geminationn}ere mich gar
               nicht, wie ichs eigentlich gemacht habe. Ich habe noch manches; Trauerſpiele und
               Faſtnachtsſpiele und {\pb}komiſche Romane; nahezu
               durchaus blödſinnig; aber ich habe ſelbſt zu der Zeit, da ich dieſe Dinge ſchrieb,
               nie das Bedürfnis gehabt, es irgend wem zu zeigen. So wird man zudringlicher,
               niedriger und unfröhlicher von Jahr zu Jahr. –)\pend
           
\pstart
           Hoffentlich ſchwingt ſich Beer-Hofma{\geminationn}\pwindex{Beer-Hofmann, Richard 1866-07-11 – 1945-09-26@\textsc{Beer-Hofmann, Richard} (1866-07-11 – 1945-09-26), \emph{Schriftsteller/Schriftstellerin}|pw} auf, Ihnen ſelbſt zu ſchreiben; faul iſt er allerdings enorm. Sie wiſſen
               wahrſcheinlich nicht einmal, dſs er \label{K_L00880-3v}\edtext{geheiratet}{\lemma{\textnormal{\emph{geheiratet}}}\Cendnote{\textnormal{Beer-Hofmanns\pwindex{Beer-Hofmann, Richard 1866-07-11 – 1945-09-26@\textsc{Beer-Hofmann, Richard} (1866-07-11 – 1945-09-26), \emph{Schriftsteller/Schriftstellerin}|pwk} Hochzeit hatte am
                     14. 5. 1898 in einer Synagoge in Wien\oindex{Wien@\textbf{Wien}, \emph{A.ADM2}|pwk} stattgefunden.}}}\label{K_L00880-3} hat, Paula\pwindex{Beer-Hofmann, Paula 25.02.1879 – 30.10.1939@\textsc{Beer-Hofmann, Paula} (25.02.1879 – 30.10.1939)|pw}, die Sie kennen {\pb}auch hat er ſchon
               zwei Töchter, die Mirjam\pwindex{Beer-Hofmann, Mirjam 04.09.1897 – 24.12.1984@\textsc{Beer-Hofmann, Mirjam} (04.09.1897 – 24.12.1984)|pw} und Naëmie\pwindex{Beer-Hofmann, Naemah 20.12.1898 – 10.11.1971@\textsc{Beer-Hofmann, Naëmah} (20.12.1898 – 10.11.1971)|pw} heißen. Aber ſeine neue Novelle\pwindex{Tod Georgs@\emph{Der Tod Georgs}|pwv} (was ich davon kenne iſt wunderſchön) iſt noch
               nicht fertig.\pend
           
\pstart
           Iſt Ihnen ein Roman bekannt, die Juden von
                  Zirndorf\pwindex{Juden von Zirndorf@\emph{Die Juden von Zirndorf}|pw}, von Waſſermann\pwindex{Wassermann, Jakob 10.03.1873 – 01.01.1934@\textsc{Wassermann, Jakob} (10.03.1873 – 01.01.1934), \emph{Schriftsteller/Schriftstellerin}|pw}? Ich glaube,
               das iſt derjenige Menſch, der den \introOben{}deutſchen\introOben{} Roman vom Anfang
               des nächſten Jahrhunderts ſchreiben wird. Sind Ihnen die Novelletten zugeko{\geminationm}en, die ich Ihnen im Frühjahr ſchickte? {\pb}(»Frau des
                  Weiſen\pwindex{Frau des Weisen. Novelletten@\emph{Die Frau des Weisen. Novelletten}|pw}«. –)\pend
           
\pstart
           Von Ihrem Ausflug nach Polen\oindex{Polen@\textbf{Polen}, \emph{A.PCLI}|pw} und Ihrem Empfang
               haben wir hier \label{K_L00880-4v}\edtext{geleſen}{\lemma{\textnormal{\emph{geleſen}}}\Cendnote{\textnormal{Die Wien\oindex{Wien@\textbf{Wien}, \emph{A.ADM2}|pwk}er Zeitungen hatten mehrfach über den Besuch Brandes\pwindex{Brandes, Georg 04.02.1842 – 19.02.1927@\textsc{Brandes, Georg} (04.02.1842 – 19.02.1927)|pwk}’ in Lemberg\oindex{Lviv@\textbf{Lviv}, \emph{P.PPLA}|pwk}
                  berichtet, so etwa die \emph{Neue Freie Presse}\orgindex{Neue Freie Presse@Neue Freie Presse|pwk} in
                  der ungezeichneten Meldung \emph{Georg Brandes in
                     Lemberg}\pwindex{Brandes in Lemberg@\emph{Brandes in Lemberg}|pwk} ([O. V.], Nr. 12.300, 19. 11. 1898,
                     Morgenausgabe, S. 4): »Georg Brandes\pwindex{Brandes, Georg 04.02.1842 – 19.02.1927@\textsc{Brandes, Georg} (04.02.1842 – 19.02.1927)|pw}, der einer Einladung nach
                        Lemberg\oindex{Lviv@\textbf{Lviv}, \emph{P.PPLA}|pw} zu der am
                        20. November stattfindenden Enthüllung des Sobiesky\pwindex{Sobieski, Johann III. 1629-08-17 – 1696-06-17@\textsc{Sobieski, Johann III.} (1629-08-17 – 1696-06-17), \emph{Politiker/Politikerin, König/Königin, Militär/Militärin}|pw}-Denkmals Folge gegeben hat, wurde bei seiner
                     Ankunft daselbst von einer Deputation feierlich empfangen. Die Spitzen der
                     Gesellschaft wetteifern in dem Bestreben, sich dem großen dänischen Schriftsteller\pwindex{Brandes, Georg 04.02.1842 – 19.02.1927@\textsc{Brandes, Georg} (04.02.1842 – 19.02.1927)|pwv} für die in
                     seinem ebenso geistvollen als anregenden Werke ›Polen\pwindex{Polen@\emph{Polen}|pw}‹ zum Ausdrucke gebrachten Sympathien erkenntlich
                     zu zeigen.« }}}\label{K_L00880-4}; dagegen hab ich von Ihren Gedichten\pwindex{Ungdomsvers [Jugendgedichte]@\emph{Ungdomsvers [Jugendgedichte]}|pwv} abſolut nichts gewußt\substVorne{}\textsuperscript{?}\substDazwischen{}.\substHinten{} Werden Sie ſie \label{K_L00880-5v}\edtext{überſetzen}{\lemma{\textnormal{\emph{überſetzen}}}\Cendnote{\textnormal{Eine deutsche Übersetzung
                  der Jugendgedichte erschien nicht.}}}\label{K_L00880-5} laſſen? Sind ſie ſchön? Haben Sie ſie
               gern? Wie viele Stunden hat Ihr Tag! Zu allem haben Sie Zeit. Und alles bewahren Sie
               auf, das iſt das Bewunderungswürdige, und darum {\pb}ſind Sie ſo reich.\pend
           
\pstart
           Ich wünſchte, Sie würden gleich geſund, reiſten wieder nach Italien\oindex{Italien@\textbf{Italien}, \emph{A.PCLI}|pw}, und blieben wieder ein paar Tage in Wien\oindex{Wien@\textbf{Wien}, \emph{A.ADM2}|pw}. Ein Wort von Ihnen, wie’s Ihnen geht, brächte mir
               jedenfalls viel Freude.\pend
           
\pstart
           Herzlich grüßt Sie Ihr Ihnen {\\[\baselineskip]}treuergebener{\\[\baselineskip]}\spacefill\mbox{ArthurSchnitzler}\pend
           \leftskip=0em{}
\pstart
           Wien\oindex{Wien@\textbf{Wien}, \emph{A.ADM2}|pw}{ }12. 1. 99.\pend
           \selectlanguage{ngerman}\endnumbering\briefempfaengerindex{Brandes, Georg@\textsc{Brandes, Georg}!zzzSchnitzler, Arthur@\emph{von Arthur Schnitzler}!1899-01-121@{12. 1. 1899}|)be}\mylabel{L00880h}  \normalsize

\doendnotes{C}
\bigskip
\vfill

\clearpage

\footnotesize

\lohead{\textsc{register}}

% Definiere theindex-Environment komplett neu ohne reledmac
\makeatletter
\renewenvironment{theindex}{%
  \section*{\indexname}%
  \setlength{\parindent}{0pt}%
  \setlength{\parskip}{0pt plus 0.3pt}%
  \let\item\@idxitem
}{%
  \clearpage
}
\makeatother

\IfFileExists{\jobname-pw.ind}{\input{\jobname-pw.ind}}{}

\end{document}

      