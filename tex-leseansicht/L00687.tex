%% latex-leseansicht-vorspann.tex
%% Vorspann für die Leseansicht.
%% Lädt die gemeinsame Datei latex-vorspann.tex mit nicht gesetztem Schalter.

\newif\ifkorrekturansicht
\korrekturansichtfalse

\input{../tex-inputs/latex-vorspann}


         
         \renewcommand{\erwaehntePersonen}{Personen: Christine Schönberger}
         \renewcommand{\erwaehnteOrte}{Orte: Brühl, Gießhüblerstraße, Hinterbrühl, Wien}
         \renewcommand{\erwaehnteWerke}{
               \section[Hugo von Hofmannsthal an Arthur Schnitzler, {[}13. 6. 1897{]}]{ Hugo von Hofmannsthal an Arthur Schnitzler, {[}13. 6. 1897{]}}\nopagebreak\mylabel{v}\rehead{ }\begin{ledgroupsized}[t]{13cm}\normalsize\beginnumbering \toendnotes[C]{\smallbreak\pagebreak[2]} \Standort{CUL, Schnitzler, B 43.}
\physDesc{Brief, 1 Blatt (gedrucktes Wappen in blauer Farbe), 3 Seiten
\newline{}Handschrift: Bleistift, deutsche Kurrent
\newline{}Schnitzler: mit Bleistift datiert: »13/6 97« \newline{}Ordnung: mit Bleistift von unbekannter Hand nummeriert:
                                        »91« }\buchAbdrucke{\weitereDrucke{Hugo von Hofmannsthal, Arthur Schnitzler: \emph{Briefwechsel}. Hg. Therese Nickl und Heinrich Schnitzler. Frankfurt am Main: \emph{S. Fischer} 1964, S. 87.} }\toendnotes[C]{\smallbreak}\pstart
           \raggedleft{}{\pb}Sonntag\pend
           \pstart{}lieber Arthur!\pend\pstart
           ich fahre wegen vielerlei Gründen (hauptſächlich Ruhe zum Arbeiten) ſchon heute
                    wieder in die Brühl\oindex{Bruehl@\textbf{Brühl}|pw}. Adreſſe Gießhüblerſtraße 2\oindex{Giesshueblerstrasse@\textbf{Gießhüblerstraße}|pw}, Hinterbrühl\oindex{Hinterbruehl@\textbf{Hinterbrühl}|pw}. Bitte machen Sie mir die Freude und ko{\geminationm}en morgen oder Dienstag
                    oder {\pb}\label{K_L00687_1v}\edtext{Do{\geminationn}erstag}{\lemma{\textnormal{\emph{Donnerstag}}}\Cendnote{\textnormal{Zum Treffen kam es am Donnerstag, dem
                            17. 6. 1897.}}}\label{K_L00687_1h} (nur nicht Mittwoch) gegen
                    Abend hinaus. Sie müſſen mir nur den Zug ſchreiben, ich hab ja nichts zu thuen
                    (von 4 Uhr an) und ko{\geminationm} dann auf die
                    Bahn Sie abholen oder wenn Sie mit dem Rad hinausfahren ſchreiben Sie mir genau,
                        {\pb}wann ich bei der Schönberger\pwindex{Schoenberger, Christine 1875-11-17 – 1971-02-03@\textsc{Schönberger, Christine} (1875-11-17 – 1971-02-03), \emph{Gastwirtin}|pw} auf Sie warten, oder
                    telegraphieren Sie mir.\pend
           \pstart
           Ich rechne ganz beſtimmt darauf. Herzlich Ihr{\\[\baselineskip]}\spacefill\mbox{Hugo.}\pend
           \leftskip=0em{}
         
         \endnumbering\mylabel{h}\end{ledgroupsized}  \newcommand{\dateiname}{L00687}\newcommand{\titel}{Hugo von Hofmannsthal an Arthur Schnitzler, [13. 6. 1897]}\newcommand{\editorInnen}{Martin Anton Müller und Gerd-Hermann Susen}%% latex-leseansicht-abspann.tex
%% Abspann für die Leseansicht.
%% Der Schalter \ifkorrekturansicht ist bereits durch den Vorspann gesetzt.

%% latex-abspann.tex
%% Gemeinsamer Abspann für Korrekturansicht und Leseansicht.
%% Setzt den Schalter \ifkorrekturansicht voraus (gesetzt in den
%% einbindenden Dateien latex-korrekturansicht-abspann.tex bzw.
%% latex-leseansicht-abspann.tex).
%% ---------------------------------------------------------------

\normalsize

% Das esempio-Environment wird nur in der Leseansicht benötigt
\ifkorrekturansicht\else
\newenvironment{esempio}[3]%
{
    \vspace{1.5ex}
    \rlap{\underline{#1}}
    \par
    \setlength{\parindent}{0cm}
    \nopagebreak
    \leftskip=#2cm
    \rightskip=#3cm
}
{
    \par
}
\fi

\doendnotes{C}
\bigskip
\vfill

\clearpage

\footnotesize

\ifkorrekturansicht
  \lohead{\textsc{register}}
\fi

% theindex-Environment neu definieren ohne reledmac
\makeatletter
\renewenvironment{theindex}{%
  \ifkorrekturansicht
    \section*{\indexname}%
  \else
    \subsubsection*{Index der erwähnten Entitäten}%
  \fi
  \setlength{\parindent}{0pt}%
  \setlength{\parskip}{0pt plus 0.3pt}%
  \let\item\@idxitem
}{%
  \ifkorrekturansicht\clearpage\fi
}
\makeatother

\IfFileExists{\jobname-pw.ind}{\input{\jobname-pw.ind}}{}

% Quellenangabe nur in der Leseansicht
\ifkorrekturansicht\else
% Fallback-Definitionen, falls die .tex-Datei \titel etc. nicht gesetzt hat
\providecommand{\titel}{}
\providecommand{\editorInnen}{}
\providecommand{\dateiname}{\jobname}

\vspace{3cm}

\vfill

\footnotesize
\textsc{Quelle}: \titel. Herausgegeben von {\editorInnen}. In: \emph{Arthur Schnitzler: Briefwechsel mit Autorinnen und Autoren}.
 Digitale Edition, https://schnitzler-briefe.acdh.oeaw.ac.at/{\dateiname}.html (Stand \today)
\fi

\end{document}


      