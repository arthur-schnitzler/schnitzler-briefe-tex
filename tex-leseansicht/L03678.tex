%% latex-leseansicht-vorspann.tex
%% Vorspann für die Leseansicht.
%% Lädt die gemeinsame Datei latex-vorspann.tex mit nicht gesetztem Schalter.

\newif\ifkorrekturansicht
\korrekturansichtfalse

\input{../tex-inputs/latex-vorspann}


\section[Olga Schnitzler und andere an Arthur Schnitzler, 11. 9. 1919]{L03678 Olga Schnitzler und andere an Arthur Schnitzler, 11. 9. 1919}
\nopagebreak\mylabel{L03678v}
\rehead{ }\normalsize\beginnumbering\briefempfaengerindex{Schnitzler, Arthur@\textsc{Schnitzler, Arthur}!zzzZotos, Iphigenie@\emph{von Iphigenie Zotos}!1919-09-111@{11. 9. 1909}|(be}\briefempfaengerindex{Schnitzler, Arthur@\textsc{Schnitzler, Arthur}!zzzPaumgartner, Bernhard@\emph{von Bernhard Paumgartner}!1919-09-111@{11. 9. 1909}|(be}\briefempfaengerindex{Schnitzler, Arthur@\textsc{Schnitzler, Arthur}!zzzZweig, Stefan@\emph{von Stefan Zweig}!1919-09-111@{11. 9. 1909}|(be}\briefempfaengerindex{Schnitzler, Arthur@\textsc{Schnitzler, Arthur}!zzzFoges, Arthur@\emph{von Arthur Foges}!1919-09-111@{11. 9. 1909}|(be}\briefempfaengerindex{Schnitzler, Arthur@\textsc{Schnitzler, Arthur}!zzzHarta, Felix Albrecht@\emph{von Felix Albrecht Harta}!1919-09-111@{11. 9. 1909}|(be}\briefempfaengerindex{Schnitzler, Arthur@\textsc{Schnitzler, Arthur}!zzzSalten, Felix@\emph{von Felix Salten}!1919-09-111@{11. 9. 1909}|(be}\briefempfaengerindex{Schnitzler, Arthur@\textsc{Schnitzler, Arthur}!zzzSchnitzler, Olga@\emph{von Olga Schnitzler}!1919-09-111@{11. 9. 1909}|(be}
\toendnotes[C]{\smallbreak\pagebreak[2]}
\correspDesc{Versand  durch Olga Schnitzler, Felix Salten, Felix Albrecht Harta, Arthur Foges, Stefan Zweig, Bernhard Paumgartner, Iphigenia Zotos am 11. 9. 1909 in Salzburg
\newline{}Erhalt  durch Arthur Schnitzler im Zeitraum [12. 9. 1919
                  – 16. 9. 1919?] in Wien}\toendnotes[C]{\smallbreak}
\Standort{DLA, A:Schnitzler, HS.NZ85.1.4547.}
\physDesc{Kartenbrief, 715 Zeichen
\newline{}Handschrift Olga Schnitzler: Bleistift, lateinische Kurrent
\newline{}Handschrift Felix Salten: Bleistift, lateinische Kurrent
\newline{}Handschrift Felix Albrecht Harta: Bleistift, lateinische Kurrent
\newline{}Handschrift Arthur Foges: Bleistift, lateinische Kurrent
\newline{}Handschrift Stefan Zweig: Bleistift, lateinische Kurrent
\newline{}Handschrift Bernhard Paumgartner: Bleistift, lateinische Kurrent
\newline{}Handschrift Iphigenie Zotos: Bleistift, lateinische Kurrent
\newline{}Versand: Stempel: »\nobreak{}\oindex{Salzburg@\textbf{Salzburg}, \emph{Verwaltungsgebiet}|pwk}Salzburg 2, 11. IX. 19, VII\nobreak{}«.  }\toendnotes[C]{\smallbreak}\pstart{}{\pb}{[}hs. Zweig:{]} Herrn\pend{}\pstart{}D\textsuperscript{r} Arthur Schnitzler\pend{}\pstart{}Wien – Cottage\oindex{Wien@\textbf{Wien}!XVIII., Währing@\textbf{XVIII., Währing}!Währinger Cottage@\textbf{Währinger Cottage}, \emph{Teil eines besiedelten Ortes}|pw}\pend{}\pstart{}Sternwartestrasse 71\oindex{Wien@\textbf{Wien}!XVIII., Währing@\textbf{XVIII., Währing}!Sternwartestraße 71@\textbf{Sternwartestraße 71}, \emph{Wohngebäude}|pw}\pend{}{\bigskip}\vspace{1em}
\pstart
           \noindent{}{\pb}{[}hs. Schnitzler:{]} Hier hält Felix mit D\textsuperscript{r} Paumgartner\introOben{}{[}hs. Salten:{]} \textsuperscript{*)}\introOben{} einen Concurrenzcurs in urjüdischen Worten – und das ist sehr komisch. Er
               (Felix) wird dir von diesem heiteren Tag erzälen. In seiner Vorlesung hab ich wieder
               einmal Concert-gelacht, – es war aber auch sehr lustig. Alles Liebe,\pend
           \pstart \spacefill\mbox{O.}\pend{}\selectlanguage{ngerman}\vspace{1em}{\vspace{1\baselineskip}}
\pstart
           {[}hs. Salten:{]} \textsuperscript{*)} Paumgartner – ein ungewöhnlich \label{K_L03678-1v}\edtext{betamter}{\lemma{\textnormal{\emph{betamter}}}\Cendnote{\textnormal{jiddisch betamt: geistvoll, geschickt}}}\label{K_L03678-1}{ }\label{K_L03678-2v}\edtext{Scheyeg}{\lemma{\textnormal{\emph{Scheyeg}}}\Cendnote{\textnormal{jiddisch Schaygetz: ein nichtjüdischer junger Mann}}}\label{K_L03678-2} –\pend
           \pstart herzlichst Ihr \spacefill\mbox{Felix}\pend{}\selectlanguage{ngerman}\vspace{1em}{\vspace{1\baselineskip}}
\pstart
           {[}hs. Harta:{]} Gesehen und richtig befunden\pend
           \pstart \spacefill\mbox{HARTA}\pend{}\selectlanguage{ngerman}\vspace{1em}{\vspace{1\baselineskip}}
\pstart
           {[}hs. Foges:{]} Schade, dass Sie nicht dabei sind\pend
           \pstart \spacefill\mbox{D\textsuperscript{r}Foges}\pend{}\selectlanguage{ngerman}\vspace{1em}{\vspace{1\baselineskip}}
\pstart
           {[}hs. Zweig:{]} Frau \label{K_L03678-3v}\edtext{Gu\substVorne{}\textsuperscript{ß}\substDazwischen{}z\substHinten{}man}{\lemma{\textnormal{\emph{Guzman}}}\Cendnote{\textnormal{Olga Schnitzler\pwindex{Schnitzler, Olga 17.\,1.\,1882 Wien – 13.\,1.\,1970 Lugano@\textsc{Schnitzler, Olga} (17.\,1.\,1882 Wien – 13.\,1.\,1970 Lugano), \emph{Schauspielerin, Sängerin}|pwk} benützte für
                  Konzertauftritte teilweise ihren Mädchennamen
                     »Gussmann«/»Guszmann«, um nicht durch ihren
                  Ehemann ein bestimmtes Bild in der Öffentlichkeit zu erwecken. Manchmal – offenbar
                  auch dieses Mal – verfremdete sie diesen Namen zu »Guzman«.}}}\label{K_L03678-3}\strikeout{n}\footnote{\noindent{}Eine junge Spanierin mit sehr schöner Stimme} lacht ausgezeichnet Coleratur. Wir freuen uns alle sehr auf das \label{K_L03678-4v}\edtext{Concert\eventindex{Mozarteum [Salzburg]@\textbf{Mozarteum [Salzburg]}!Konzert mit Olga Schnitzer, 19.9.1919@Konzert mit Olga Schnitzer, 19.9.1919|pwv}}{\lemma{\textnormal{\emph{Concert}}}\Cendnote{\textnormal{Das Koncert\eventindex{Mozarteum [Salzburg]@\textbf{Mozarteum [Salzburg]}!Konzert mit Olga Schnitzer, 19.9.1919@Konzert mit Olga Schnitzer, 19.9.1919|pwkv} fand am
                     19. 9. 1919 im \emph{Mozarteum}\orgindex{Mozarteum [Salzburg]@Mozarteum [Salzburg]|pwk}
                  statt. Vgl. Hermann Bahr, Arthur Schnitzler: \emph{Briefwechsel, Aufzeichnungen, Dokumente (1891–1931)}, Olga an Arthur Schnitzler, 20. 9. 1919. Ein für den
                     24. 9. 1919 geplantes Solokonzert von Olga Schnitzler\pwindex{Schnitzler, Olga 17.\,1.\,1882 Wien – 13.\,1.\,1970 Lugano@\textsc{Schnitzler, Olga} (17.\,1.\,1882 Wien – 13.\,1.\,1970 Lugano), \emph{Schauspielerin, Sängerin}|pwk} wurde kurzfristig abgesagt. }}}\label{K_L03678-4}!\pend
           \pstart \spacefill\mbox{StefanZweig}\pend{}\selectlanguage{ngerman}\vspace{1em}{\vspace{1\baselineskip}}
\pstart
           {[}hs. Paumgartner:{]} Die Direktion des Mozarteums\orgindex{Mozarteum [Salzburg]@Mozarteum [Salzburg]|pw} erbittet Ihr baldiges Erscheinen\pend
           
\pstart
           Ergebenst{\\[\baselineskip]}\spacefill\mbox{D\textsuperscript{r}BPaumgartner}\pend
           \leftskip=0em{}\selectlanguage{ngerman}\vspace{1em}{\vspace{1\baselineskip}}\pstart \spacefill\mbox{{[}hs. Zotos:{]} Iphigenia Zotos}\pend{}
\pstart
           \noindent{}ein »\textcolor{gray}{F}rl«\pend
           \selectlanguage{ngerman}\vspace{1em}{\vspace{1\baselineskip}}
\pstart
           \noindent{}{[}hs. Schnitzler:{]} \label{T_L03678-1v}\edtext{Es ist \label{K_L03678-5v}\edtext{11 Uhr, abends}{\lemma{\textnormal{\emph{11 Uhr, abends}}}\Cendnote{\textnormal{Vgl. Hermann Bahr, Arthur Schnitzler: \emph{Briefwechsel, Aufzeichnungen, Dokumente (1891–1931)}, Olga an Arthur Schnitzler, 13. 9. 1919. }}}\label{K_L03678-5} – wie Du
               bemerken wirst.}{\lemma{\textnormal{\emph{Es … wirst.}}}\Cendnote{\textnormal{seitlich entlang des
                  linken Rands}}}\label{T_L03678-1}\pend
           \selectlanguage{ngerman}\endnumbering\briefempfaengerindex{Schnitzler, Arthur@\textsc{Schnitzler, Arthur}!zzzZotos, Iphigenie@\emph{von Iphigenie Zotos}!1919-09-111@{11. 9. 1909}|)be}\briefempfaengerindex{Schnitzler, Arthur@\textsc{Schnitzler, Arthur}!zzzPaumgartner, Bernhard@\emph{von Bernhard Paumgartner}!1919-09-111@{11. 9. 1909}|)be}\briefempfaengerindex{Schnitzler, Arthur@\textsc{Schnitzler, Arthur}!zzzZweig, Stefan@\emph{von Stefan Zweig}!1919-09-111@{11. 9. 1909}|)be}\briefempfaengerindex{Schnitzler, Arthur@\textsc{Schnitzler, Arthur}!zzzFoges, Arthur@\emph{von Arthur Foges}!1919-09-111@{11. 9. 1909}|)be}\briefempfaengerindex{Schnitzler, Arthur@\textsc{Schnitzler, Arthur}!zzzHarta, Felix Albrecht@\emph{von Felix Albrecht Harta}!1919-09-111@{11. 9. 1909}|)be}\briefempfaengerindex{Schnitzler, Arthur@\textsc{Schnitzler, Arthur}!zzzSalten, Felix@\emph{von Felix Salten}!1919-09-111@{11. 9. 1909}|)be}\briefempfaengerindex{Schnitzler, Arthur@\textsc{Schnitzler, Arthur}!zzzSchnitzler, Olga@\emph{von Olga Schnitzler}!1919-09-111@{11. 9. 1909}|)be}\mylabel{L03678h}
\begin{anhang}
\end{anhang}\newcommand{\dateiname}{L03678}\newcommand{\titel}{Olga Schnitzler und andere an Arthur Schnitzler, 11. 9. 1919}\newcommand{\editorInnen}{Selma Jahnke und Martin Anton Müller}%% latex-leseansicht-abspann.tex
%% Abspann für die Leseansicht.
%% Der Schalter \ifkorrekturansicht ist bereits durch den Vorspann gesetzt.

%% latex-abspann.tex
%% Gemeinsamer Abspann für Korrekturansicht und Leseansicht.
%% Setzt den Schalter \ifkorrekturansicht voraus (gesetzt in den
%% einbindenden Dateien latex-korrekturansicht-abspann.tex bzw.
%% latex-leseansicht-abspann.tex).
%% ---------------------------------------------------------------

\normalsize

% Das esempio-Environment wird nur in der Leseansicht benötigt
\ifkorrekturansicht\else
\newenvironment{esempio}[3]%
{
    \vspace{1.5ex}
    \rlap{\underline{#1}}
    \par
    \setlength{\parindent}{0cm}
    \nopagebreak
    \leftskip=#2cm
    \rightskip=#3cm
}
{
    \par
}
\fi

\doendnotes{C}
\bigskip
\vfill

\clearpage

\footnotesize

\ifkorrekturansicht
  \lohead{\textsc{register}}
\fi

% theindex-Environment neu definieren ohne reledmac
\makeatletter
\renewenvironment{theindex}{%
  \ifkorrekturansicht
    \section*{\indexname}%
  \else
    \subsubsection*{Index der erwähnten Entitäten}%
  \fi
  \setlength{\parindent}{0pt}%
  \setlength{\parskip}{0pt plus 0.3pt}%
  \let\item\@idxitem
}{%
  \ifkorrekturansicht\clearpage\fi
}
\makeatother

\IfFileExists{\jobname-pw.ind}{\input{\jobname-pw.ind}}{}

% Quellenangabe nur in der Leseansicht
\ifkorrekturansicht\else
% Fallback-Definitionen, falls die .tex-Datei \titel etc. nicht gesetzt hat
\providecommand{\titel}{}
\providecommand{\editorInnen}{}
\providecommand{\dateiname}{\jobname}

\vspace{3cm}

\vfill

\footnotesize
\textsc{Quelle}: \titel. Herausgegeben von {\editorInnen}. In: \emph{Arthur Schnitzler: Briefwechsel mit Autorinnen und Autoren}.
 Digitale Edition, https://schnitzler-briefe.acdh.oeaw.ac.at/{\dateiname}.html (Stand \today)
\fi

\end{document}


