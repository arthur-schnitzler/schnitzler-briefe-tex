%% latex-korrekturansicht-vorspann.tex
%% Vorspann für die Korrekturansicht.
%% Lädt die gemeinsame Datei latex-vorspann.tex mit gesetztem Schalter.

\newif\ifkorrekturansicht
\korrekturansichttrue

\input{../tex-inputs/latex-vorspann}


\section[Olga Schnitzler und andere an Arthur Schnitzler, 11. 9. 1919]{L03678 Olga Schnitzler und andere an Arthur Schnitzler, 11. 9. 1919}
\nopagebreak\mylabel{L03678v}
\rehead{ }\normalsize\beginnumbering\briefempfaengerindex{Schnitzler, Arthur@\textsc{Schnitzler, Arthur}!zzzZotos, Iphigenie@\emph{von Iphigenie Zotos}!1919-09-111@{11. 9. 1909}|(be}\briefempfaengerindex{Schnitzler, Arthur@\textsc{Schnitzler, Arthur}!zzzPaumgartner, Bernhard@\emph{von Bernhard Paumgartner}!1919-09-111@{11. 9. 1909}|(be}\briefempfaengerindex{Schnitzler, Arthur@\textsc{Schnitzler, Arthur}!zzzZweig, Stefan@\emph{von Stefan Zweig}!1919-09-111@{11. 9. 1909}|(be}\briefempfaengerindex{Schnitzler, Arthur@\textsc{Schnitzler, Arthur}!zzzFoges, Arthur@\emph{von Arthur Foges}!1919-09-111@{11. 9. 1909}|(be}\briefempfaengerindex{Schnitzler, Arthur@\textsc{Schnitzler, Arthur}!zzzHarta, Felix Albrecht@\emph{von Felix Albrecht Harta}!1919-09-111@{11. 9. 1909}|(be}\briefempfaengerindex{Schnitzler, Arthur@\textsc{Schnitzler, Arthur}!zzzSalten, Felix@\emph{von Felix Salten}!1919-09-111@{11. 9. 1909}|(be}\briefempfaengerindex{Schnitzler, Arthur@\textsc{Schnitzler, Arthur}!zzzSchnitzler, Olga@\emph{von Olga Schnitzler}!1919-09-111@{11. 9. 1909}|(be}
\toendnotes[C]{\smallbreak\pagebreak[2]}\Standort{DLA, A:Schnitzler, HS.NZ85.1.4547.}
\physDesc{Kartenbrief, 1 Blatt, 2 Seiten, 715 Zeichen
\newline{}Handschrift Olga Schnitzler: , lateinische Kurrent
\newline{}Handschrift Felix Salten: schwarze Tinte, lateinische Kurrent
\newline{}Handschrift Felix Albrecht Harta: schwarze Tinte, lateinische Kurrent
\newline{}Handschrift Arthur Foges: schwarze Tinte, lateinische Kurrent
\newline{}Handschrift Stefan Zweig: schwarze Tinte, lateinische Kurrent
\newline{}Handschrift Bernhard Paumgartner: schwarze Tinte, lateinische Kurrent
\newline{}Handschrift Iphigenie Zotos: schwarze Tinte, lateinische Kurrent
\newline{}Versand: Stempel: »\nobreak{}\oindex{Salzburg@\textbf{Salzburg}, \emph{A.ADM2}|pwk}Salzburg 2\nobreak{}«.  }\toendnotes[C]{\smallbreak}\pstart{}{\pb}{[}hs. :{]} Herrn\pend{}\pstart{}D\textsuperscript{r} Arthur Schnitzler\pend{}\pstart{}Wien – Cottage\oindex{Waehringer Cottage@\textbf{Währinger Cottage}, \emph{Teil eines besiedelten Ortes (A.BSOX)}|pw}\pend{}\pstart{}Sternwartestrasse 71\oindex{Sternwartestrasse 71@\textbf{Sternwartestraße 71}, \emph{Wohngebäude (K.WHS)}|pw}\pend{}{\bigskip}\vspace{1em}
\pstart
           \noindent{}{\pb}{[}hs. :{]} Hier hält Felix mit D\textsuperscript{r} Paumgartner\introOben{}{[}hs. :{]} \textsuperscript{*)}\introOben{} einen Concurrenzcurs in urjüdischen Worten – und das ist sehr komisch. Er
               (Felix) wird dir von diesem heiteren Tag erzälen. In seiner Vorlesung hab ich wieder
               einmal Concert-gelacht, – es war aber auch sehr lustig. Alles Liebe,\pend
           \pstart \spacefill\mbox{O.}\pend{}\selectlanguage{ngerman}\vspace{1em}{\vspace{1\baselineskip}}
\pstart
           {[}hs. :{]} \textsuperscript{*)} Paumgartner – ein ungewöhnlich \label{K_L03678-1v}\edtext{betamter}{\lemma{\textnormal{\emph{betamter}}}\Cendnote{\textnormal{jiddisch betamt: geistvoll, geschickt}}}\label{K_L03678-1}{ }\label{K_L03678-2v}\edtext{Scheyeg}{\lemma{\textnormal{\emph{Scheyeg}}}\Cendnote{\textnormal{jiddisch Schaygetz: ein nichtjüdischer junger Mann}}}\label{K_L03678-2} – \pend
           \pstart herzlichst Ihr \spacefill\mbox{Felix}\pend{}\selectlanguage{ngerman}\vspace{1em}{\vspace{1\baselineskip}}
\pstart
           {[}hs. :{]} Gesehen und richtig befunden\pend
           \pstart \spacefill\mbox{HARTA}\pend{}\selectlanguage{ngerman}\vspace{1em}{\vspace{1\baselineskip}}
\pstart
           {[}hs. :{]} Schade, dass Sie nicht dabei sind\pend
           \pstart \spacefill\mbox{D\textsuperscript{r}Foges}\pend{}\selectlanguage{ngerman}\vspace{1em}{\vspace{1\baselineskip}}
\pstart
           {[}hs. :{]} Frau \label{K_L03678-3v}\edtext{Gu\substVorne{}\textsuperscript{ß}\substDazwischen{}z\substHinten{}man}{\lemma{\textnormal{\emph{Guzman}}}\Cendnote{\textnormal{Olga Schnitzler\pwindex{Schnitzler, Olga 17.01.1882 – 13.01.1970@\textsc{Schnitzler, Olga} (17.01.1882 – 13.01.1970), \emph{Schauspieler/Schauspielerin, Sänger/Sängerin}|pwk} benützte für
                  Konzertauftritte teilweise ihren Mädchennamen »Gussmann«, um nicht
                  durch ihren Ehemann ein bestimmtes Bild in der Öffentlichkeit zu erwecken.
                  Manchmal – offenbar auch dieses Mal – verfremdete sie diesen Namen zu
                     »Guzman«.}}}\label{K_L03678-3}\strikeout{n}\noindent{}Eine junge Spanierin mit sehr schöner Stimme lacht ausgezeichnet Coleratur. Wir freuen uns alle sehr auf das \label{K_L03678-4v}\edtext{Concert}{\lemma{\textnormal{\emph{Concert}}}\Cendnote{\textnormal{Dieses fand am 19. 9. 1919 im \emph{Mozarteum}\orgindex{Mozarteum [Salzburg]@Mozarteum [Salzburg]|pwk} statt. Vgl. Hermann Bahr, Arthur Schnitzler: \emph{Briefwechsel, Aufzeichnungen, Dokumente (1891–1931)}, Olga an Arthur Schnitzler, 20. 9. 1919. Ein für den 24. 9. 1919
                  geplantes Solokonzert von Olga Schnitzler\pwindex{Schnitzler, Olga 17.01.1882 – 13.01.1970@\textsc{Schnitzler, Olga} (17.01.1882 – 13.01.1970), \emph{Schauspieler/Schauspielerin, Sänger/Sängerin}|pwk}
                  wurde kurzfristig abgesagt. }}}\label{K_L03678-4}!\pend
           \pstart \spacefill\mbox{StefanZweig}\pend{}\selectlanguage{ngerman}\vspace{1em}{\vspace{1\baselineskip}}
\pstart
           {[}hs. :{]} Die Direktion des Mozarteums\orgindex{Mozarteum [Salzburg]@Mozarteum [Salzburg]|pw} erbittet Ihr baldiges Erscheinen\pend
           
\pstart
           Ergebenst{\\[\baselineskip]}\spacefill\mbox{D\textsuperscript{r}BPaumgartner}\pend
           \leftskip=0em{}\selectlanguage{ngerman}\vspace{1em}{\vspace{1\baselineskip}}\pstart \spacefill\mbox{{[}hs. :{]} Iphigenia Zotos}\pend{}
\pstart
           \noindent{}ein »\textcolor{gray}{F}rl«\pend
           \selectlanguage{ngerman}\vspace{1em}{\vspace{1\baselineskip}}
\pstart
           \noindent{}{[}hs. :{]} \label{T_L03678-1v}\edtext{Es ist \label{K_L03678-5v}\edtext{11 Uhr, abends}{\lemma{\textnormal{\emph{11 Uhr, abends}}}\Cendnote{\textnormal{vgl. Hermann Bahr, Arthur Schnitzler: \emph{Briefwechsel, Aufzeichnungen, Dokumente (1891–1931)}, Olga an Arthur Schnitzler, 13. 9. 1919. }}}\label{K_L03678-5} – wie Du
               bemerken wirst.}{\lemma{\textnormal{\emph{Es … wirst.}}}\Cendnote{\textnormal{seitlich entlang des
                  linken Rands}}}\label{T_L03678-1}\pend
           \selectlanguage{ngerman}\endnumbering\briefempfaengerindex{Schnitzler, Arthur@\textsc{Schnitzler, Arthur}!zzzZotos, Iphigenie@\emph{von Iphigenie Zotos}!1919-09-111@{11. 9. 1909}|)be}\briefempfaengerindex{Schnitzler, Arthur@\textsc{Schnitzler, Arthur}!zzzPaumgartner, Bernhard@\emph{von Bernhard Paumgartner}!1919-09-111@{11. 9. 1909}|)be}\briefempfaengerindex{Schnitzler, Arthur@\textsc{Schnitzler, Arthur}!zzzZweig, Stefan@\emph{von Stefan Zweig}!1919-09-111@{11. 9. 1909}|)be}\briefempfaengerindex{Schnitzler, Arthur@\textsc{Schnitzler, Arthur}!zzzFoges, Arthur@\emph{von Arthur Foges}!1919-09-111@{11. 9. 1909}|)be}\briefempfaengerindex{Schnitzler, Arthur@\textsc{Schnitzler, Arthur}!zzzHarta, Felix Albrecht@\emph{von Felix Albrecht Harta}!1919-09-111@{11. 9. 1909}|)be}\briefempfaengerindex{Schnitzler, Arthur@\textsc{Schnitzler, Arthur}!zzzSalten, Felix@\emph{von Felix Salten}!1919-09-111@{11. 9. 1909}|)be}\briefempfaengerindex{Schnitzler, Arthur@\textsc{Schnitzler, Arthur}!zzzSchnitzler, Olga@\emph{von Olga Schnitzler}!1919-09-111@{11. 9. 1909}|)be}\mylabel{L03678h}
\begin{anhang}
\end{anhang}\normalsize

\doendnotes{C}
\bigskip
\vfill

\clearpage

\footnotesize

\lohead{\textsc{register}}

% Definiere theindex-Environment komplett neu ohne reledmac
\makeatletter
\renewenvironment{theindex}{%
  \section*{\indexname}%
  \setlength{\parindent}{0pt}%
  \setlength{\parskip}{0pt plus 0.3pt}%
  \let\item\@idxitem
}{%
  \clearpage
}
\makeatother

\IfFileExists{\jobname-pw.ind}{\input{\jobname-pw.ind}}{}

\end{document}

      