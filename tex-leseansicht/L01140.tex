%% latex-korrekturansicht-vorspann.tex
%% Vorspann für die Korrekturansicht.
%% Lädt die gemeinsame Datei latex-vorspann.tex mit gesetztem Schalter.

\newif\ifkorrekturansicht
\korrekturansichttrue

\input{../tex-inputs/latex-vorspann}


\section[Arthur Schnitzler an Richard Beer-Hofmann, 4. 7. 1901]{L01140 Arthur Schnitzler an Richard Beer-Hofmann, 4. 7. 1901}
\nopagebreak\mylabel{L01140v}
\rehead{ }\normalsize\beginnumbering\briefempfaengerindex{Beer-Hofmann, Richard@\textsc{Beer-Hofmann, Richard}!zzzSchnitzler, Arthur@\emph{von Arthur Schnitzler}!1901-07-041@{4. 7. 1901}|(be}
\toendnotes[C]{\smallbreak\pagebreak[2]}\Standort{YCGL, MSS 31.}
\physDesc{Brief, 2 Blätter, 7 Seiten, Umschlag, 1789 Zeichen
\newline{}Handschrift: 1) Bleistift, deutsche Kurrent\hspace{1em}2) schwarze Tinte, deutsche Kurrent (\noindent{}Umschlag)\hspace{1em}
\newline{}Versand: 1) Stempel: »\nobreak{}\oindex{St. Anton am Arlberg@\textbf{St. Anton am Arlberg}, \emph{A.ADM3}|pwk}St. Anton am Arlberg, 4 7 01\nobreak{}«.   2) Stempel: »\nobreak{}\oindex{Poertschach am Woerthersee@\textbf{Pörtschach am Wörthersee}, \emph{P.PPL}|pwk}{\pb}Pörtschach am See, 5 7 01\nobreak{}«. }
\buchAbdrucke{\weitereDrucke{Arthur Schnitzler, Richard Beer-Hofmann: \emph{Briefwechsel 1891–1931}. Wien, Zürich: \emph{Europaverlag} 1992, S. 152–153.} }\toendnotes[C]{\smallbreak}\pstart{}{\pb}Herrn \textsc{Dr. Richard
                     Beer-Hofmann}\pend{}\pstart{}\textsc{Pörtschach\oindex{Poertschach am Woerthersee@\textbf{Pörtschach am Wörthersee}, \emph{P.PPL}|pw}}\pend{}\pstart{}\textsc{am Wörthersee\oindex{Woerthersee@\textbf{Wörthersee}, \emph{H.LK}|pw}}\pend{}\pstart{}\textsc{Villa Arnstein\oindex{Villa Arnstein@\textbf{Villa Arnstein}, \emph{Wohngebäude (K.WHS)}|pw}}\pend{}{\bigskip}\vspace{1em}
\pstart
           \raggedleft{}{\pb}\textsc{St. Anton a (Arlberg)}\oindex{St. Anton am Arlberg@\textbf{St. Anton am Arlberg}, \emph{A.ADM3}|pw}{\\}4. 7. 901.\pend
           \vspace{0.5em}
\pstart
           mein lieber Richard, ich war zuerſt 14 Tage in Salzburg\oindex{Salzburg@\textbf{Salzburg}, \emph{A.ADM2}|pw}, oeſterr Hof\oindex{Oesterreichischer Hof@\textbf{Österreichischer Hof}, \emph{Hotel (K.HTL)}|pw}, mit
                  ihr\pwindex{Schnitzler, Olga 17.01.1882 – 13.01.1970@\textsc{Schnitzler, Olga} (17.01.1882 – 13.01.1970), \emph{Schauspieler/Schauspielerin, Sänger/Sängerin}|pwv}, es war ſehr ſchön.
               Dann 2 Tage Innsbruck\oindex{Innsbruck@\textbf{Innsbruck}, \emph{A.ADM2}|pw} (daſs ich Schönberg\oindex{Schoenberg im Stubaital@\textbf{Schönberg im Stubaital}, \emph{P.PPLA3}|pw} aufgeſucht habe, wiſſen Sie), da{\geminationn} fuhren wir nach \textsc{Landeck}\oindex{Landeck@\textbf{Landeck}, \emph{P.PPLA3}|pw}, wo ihre Schweſter\pwindex{Steinrueck, Elisabeth 19.11.1885 – 07.04.1920@\textsc{Steinrück, Elisabeth} (19.11.1885 – 07.04.1920)|pwv} kam,
               und nun ſind wir in \textsc{St. Anton}\oindex{St. Anton am Arlberg@\textbf{St. Anton am Arlberg}, \emph{A.ADM3}|pw} – ich habe ein \introOben{}ſehr behagliches\introOben{} Zimmer zu 60 Kreuzer
               in einem Privat{\pb}haus, und es wäre ſehr nett, we{\geminationn} nicht das Wetter elend wäre. Wie lang ich hier bleibe,
                  ka{\geminationn} ich natürlich \introOben{}nicht\introOben{}
               ſagen (daher bitte ich um Nachricht nach \uline{Wien}\oindex{Wien@\textbf{Wien}, \emph{A.ADM2}|pw}) wahrſcheinlich fahre ich von hier aus in die Schweiz\oindex{Schweiz@\textbf{Schweiz}, \emph{A.PCLI}|pw}. Anfang August ſoll ich dort Mama\pwindex{Schnitzler, Louise 1840-07-08 – 1911-09-09@\textsc{Schnitzler, Louise} (1840-07-08 – 1911-09-09)|pwv} treffen (\textsc{Flims}\oindex{Flims@\textbf{Flims}, \emph{P.PPL}|pw} von \textsc{Reichenau}\oindex{Reichenau [Schweiz]@\textbf{Reichenau [Schweiz]}, \emph{P.PPL}|pw}
                – (\textsc{Chur}\oindex{Chur@\textbf{Chur}, \emph{P.PPLA}|pw} – \textsc{Tham}\oindex{Tamins@\textbf{Tamins}, \emph{A.ADM3}|pw}) aus 3 Stunden) auf etwa {\pb}8 Tage. Der \textsc{Wörther}ſee\oindex{Woerthersee@\textbf{Wörthersee}, \emph{H.LK}|pw} fiel ins Waſſer, weil Scharlach Gerüchte
               umgingen, und überdies wollte Mama\pwindex{Schnitzler, Louise 1840-07-08 – 1911-09-09@\textsc{Schnitzler, Louise} (1840-07-08 – 1911-09-09)|pwv} nicht zu \textsc{Pundschu}\oindex{Pension Pundschu@\textbf{Pension Pundschu}, \emph{Hotel (K.HTL)}|pw}, weil ich nicht wußte, auf wie lang ich hingehn würde. Nun bin ich ſo weit von
               dort, dſs ich Sie heuer im Sommer kaum ſehn werde, we{\geminationn}
               Sie nicht mir, \textsc{resp}. mir und {\pb}Paul Goldmann\pwindex{Goldmann, Paul 31.01.1865 – 25.09.1935@\textsc{Goldmann, Paul} (31.01.1865 – 25.09.1935), \emph{Schriftsteller/Schriftstellerin, Journalist/Journalistin}|pw} (von dem ich übrigens noch keine
                  beſti{\geminationm}te Nachricht habe) irgendwie entgegenko{\geminationm}en.\pend
           
\pstart
           Haben Sie ſchon irgendwelche Auguſtpläne? Sie ſchreiben mir wenig, faſt gar nichts
               über ſich; was thun Sie? Arbeiten Sie? Wie gehts Ihrer Frau\pwindex{Beer-Hofmann, Paula 25.02.1879 – 30.10.1939@\textsc{Beer-Hofmann, Paula} (25.02.1879 – 30.10.1939)|pwv} und den Kindern\pwindex{Beer-Hofmann, Naemah 20.12.1898 – 10.11.1971@\textsc{Beer-Hofmann, Naëmah} (20.12.1898 – 10.11.1971)|pwv}\pwindex{Beer-Hofmann, Mirjam 04.09.1897 – 24.12.1984@\textsc{Beer-Hofmann, Mirjam} (04.09.1897 – 24.12.1984)|pwv}?\pend
           
\pstart
           Salten\pwindex{Salten, Felix 06.09.1869 – 08.10.1945@\textsc{Salten, Felix} (06.09.1869 – 08.10.1945), \emph{Schriftsteller/Schriftstellerin, Journalist/Journalistin, Chefredakteur/Chefredakteurin}|pw} iſt auf Reiſen, {\pb}wie mir eine \label{K_L01140-1v}\edtext{Karte von ihm}{\lemma{\textnormal{\emph{Karte von ihm}}}\Cendnote{\textnormal{Siehe Felix Salten an Arthur Schnitzler, 2[3]. 6. 1901.
               }}}\label{K_L01140-1} flüchtig mittheilt, aus Brettlgründen\orgindex{Jung-Wiener Theater zum Lieben Augustin@Jung-Wiener Theater zum Lieben Augustin|pwv}. Ich ſchreibe ein 3aktiges Stück\pwindex{einsame Weg. Schauspiel in fuenf Akten@\emph{Der einsame Weg. Schauspiel in fünf Akten}|pwv}\pwindex{Professor Bernhardi. Komoedie in fuenf Akten@\emph{Professor Bernhardi. Komödie in fünf Akten}|pwv} und glaube im Sommer damit und
               auch mit 2 Einaktern\pwindex{Lebendige Stunden@\emph{Lebendige Stunden}|pwv}\pwindex{Frau mit dem Dolche@\emph{Die Frau mit dem Dolche}|pwv}
               fertig zu werden. – An \label{K_L01140-2v}\edtext{Hugo\pwindex{Hofmannsthal, Hugo von 1874-02-01 – 1929-07-15@\textsc{Hofmannsthal, Hugo von} (1874-02-01 – 1929-07-15), \emph{Schriftsteller/Schriftstellerin}|pw} und Gerty\pwindex{Hofmannsthal, Gertrude von 16.03.1880 – 09.11.1959@\textsc{Hofmannsthal, Gertrude von} (16.03.1880 – 09.11.1959)|pw}{ }ſauſte ich (\textsc{resp}. wir\pwindex{Schnitzler, Olga 17.01.1882 – 13.01.1970@\textsc{Schnitzler, Olga} (17.01.1882 – 13.01.1970), \emph{Schauspieler/Schauspielerin, Sänger/Sängerin}|pwv}) in Innsbruck\oindex{Innsbruck@\textbf{Innsbruck}, \emph{A.ADM2}|pw} in einem Einſpänner vorüber}{\lemma{\textnormal{\emph{Hugo … vorüber}}}\Cendnote{\textnormal{Vgl. A. S.: \emph{Tagebuch}, 27. 6. 1901.
               }}}\label{K_L01140-2}. – Innsbruck\oindex{Innsbruck@\textbf{Innsbruck}, \emph{A.ADM2}|pw} verſucht ich diesmal Tiroler {\pb}Hof\oindex{Tiroler Hof@\textbf{Tiroler Hof}, \emph{Hotel (K.HTL)}|pw}. Ich
               warne Sie. Es iſt ſchmierig und ver\textsc{snobt}. Das ſchönſte
               bisher war natürlich \textsc{Hel\introOben{}l\introOben{}brunn}\oindex{Hellbrunn@\textbf{Hellbrunn}, \emph{P.PPL}|pw}. Heuer zum erſten Mal hab ich auch das Schloſs\oindex{Hellbrunn@\textbf{Hellbrunn}, \emph{P.PPL}|pwv} geſehn, innen (nicht das »Monatsſchlößel\oindex{Monatsschloessl@\textbf{Monatsschlössl}, \emph{Schloss (K.SLS)}|pw}«, ſondern das ununterbrochene.) –\pend
           
\pstart
           Leben Sie wohl und ſchreiben Sie bald.\pend
           
\pstart
           {\pb}Von Herzen Ihr{\\[\baselineskip]}\spacefill\mbox{Arthur}\pend
           \leftskip=0em{}\selectlanguage{ngerman}\endnumbering\briefempfaengerindex{Beer-Hofmann, Richard@\textsc{Beer-Hofmann, Richard}!zzzSchnitzler, Arthur@\emph{von Arthur Schnitzler}!1901-07-041@{4. 7. 1901}|)be}\mylabel{L01140h}  \normalsize

\doendnotes{C}
\bigskip
\vfill

\clearpage

\footnotesize

\lohead{\textsc{register}}

% Definiere theindex-Environment komplett neu ohne reledmac
\makeatletter
\renewenvironment{theindex}{%
  \section*{\indexname}%
  \setlength{\parindent}{0pt}%
  \setlength{\parskip}{0pt plus 0.3pt}%
  \let\item\@idxitem
}{%
  \clearpage
}
\makeatother

\IfFileExists{\jobname-pw.ind}{\input{\jobname-pw.ind}}{}

\end{document}

      