%% latex-leseansicht-vorspann.tex
%% Vorspann für die Leseansicht.
%% Lädt die gemeinsame Datei latex-vorspann.tex mit nicht gesetztem Schalter.

\newif\ifkorrekturansicht
\korrekturansichtfalse

\input{../tex-inputs/latex-vorspann}


\section[Arthur Schnitzler an Richard Beer-Hofmann, 4. 7. 1901]{L01140 Arthur Schnitzler an Richard Beer-Hofmann, 4. 7. 1901}
\nopagebreak\mylabel{L01140v}
\rehead{ }\normalsize\beginnumbering\briefempfaengerindex{Beer-Hofmann, Richard@\textsc{Beer-Hofmann, Richard}!zzzSchnitzler, Arthur@\emph{von Arthur Schnitzler}!1901-07-041@{4. 7. 1901}|(be}
\toendnotes[C]{\smallbreak\pagebreak[2]}
\correspDesc{Versand  durch Arthur Schnitzler am 4. 7. 1901 in St. Anton am Arlberg
\newline{}Erhalt  durch Richard Beer-Hofmann am 5. 7. 1901 in Pörtschach am Wörthersee}\toendnotes[C]{\smallbreak}
\Standort{YCGL, MSS 31.}
\physDesc{Brief, 2 Blätter, 7 Seiten, Kuvert, 1788 Zeichen
\newline{}Handschrift: 1) Bleistift, deutsche Kurrent\hspace{1em}2) schwarze Tinte, deutsche Kurrent (\noindent{}Umschlag)\hspace{1em}
\newline{}Versand: 1) Stempel: »\nobreak{}\oindex{St. Anton am Arlberg@\textbf{St. Anton am Arlberg}, \emph{Verwaltungsgebiet}|pwk}St. Anton am Arlberg, 4 7 01\nobreak{}«.   2) Stempel: »\nobreak{}\oindex{Pörtschach am Wörthersee@\textbf{Pörtschach am Wörthersee}|pwk}{\pb}Pörtschach am See, 5 7 01\nobreak{}«. }
\buchAbdrucke{\weitereDrucke{Arthur Schnitzler, Richard Beer-Hofmann: \emph{Briefwechsel 1891–1931}. Herausgegeben von Konstanze Fliedl. Wien, Zürich: \emph{Europaverlag} 1992, S. 152–153.} }\toendnotes[C]{\smallbreak}\pstart{}{\pb}Herrn \textsc{Dr. Richard
                     Beer-Hofmann}\pend{}\pstart{}\textsc{Pörtschach\oindex{Pörtschach am Wörthersee@\textbf{Pörtschach am Wörthersee}|pw}}\pend{}\pstart{}\textsc{am Wörthersee\oindex{Wörthersee@\textbf{Wörthersee}, \emph{See}|pw}}\pend{}\pstart{}\textsc{Villa Arnstein\oindex{Villa Arnstein@\textbf{Villa Arnstein}, \emph{Wohngebäude}|pw}}\pend{}{\bigskip}\vspace{1em}
\pstart
           \raggedleft{}{\pb}\textsc{St. Anton a (Arlberg)}\oindex{St. Anton am Arlberg@\textbf{St. Anton am Arlberg}, \emph{Verwaltungsgebiet}|pw}{\\}4. 7. 901.\pend
           \vspace{0.5em}
\pstart
           mein lieber Richard, ich war zuerſt 14 Tage in Salzburg\oindex{Salzburg@\textbf{Salzburg}, \emph{Verwaltungsgebiet}|pw}, oeſterr Hof\oindex{Österreichischer Hof@\textbf{Österreichischer Hof}, \emph{Hotel}|pw}, mit
                  ihr\pwindex{Schnitzler, Olga 17.\,1.\,1882 Wien – 13.\,1.\,1970 Lugano@\textsc{Schnitzler, Olga} (17.\,1.\,1882 Wien – 13.\,1.\,1970 Lugano), \emph{Schauspielerin, Sängerin}|pwv}, es war{ }ſehr{ }ſchön.
               Dann 2 Tage Innsbruck\oindex{Innsbruck@\textbf{Innsbruck}, \emph{Verwaltungsgebiet}|pw} (daſs ich Schönberg\oindex{Schönberg im Stubaital@\textbf{Schönberg im Stubaital}, \emph{Hauptstadt}|pw} aufgeſucht habe, wiſſen Sie), da{\geminationn} fuhren wir nach \textsc{Landeck}\oindex{Landeck@\textbf{Landeck}, \emph{Hauptstadt}|pw}, wo ihre Schweſter\pwindex{Steinrück, Elisabeth 19.\,11.\,1885 – 7.\,4.\,1920 Partenkirchen@\textsc{Steinrück, Elisabeth} (19.\,11.\,1885 – 7.\,4.\,1920 Partenkirchen)|pwv} kam,
               und nun{ }ſind wir in \textsc{St. Anton}\oindex{St. Anton am Arlberg@\textbf{St. Anton am Arlberg}, \emph{Verwaltungsgebiet}|pw} – ich habe ein \introOben{}ſehr behagliches\introOben{} Zimmer zu 60 Kreuzer
                in einem Privat{\pb}haus\oindex{?? [Unterkunft von Schnitzler und Gussmann in St. Anton. 1901]@\textbf{?? [Unterkunft von Schnitzler und Gussmann in St. Anton. 1901]}, \emph{Wohngebäude}|pwv}, und es wäre{ }ſehr nett, we{\geminationn} nicht das Wetter elend wäre. Wie lang ich hier bleibe,
                  ka{\geminationn} ich natürlich \introOben{}nicht\introOben{}{ }ſagen (daher bitte ich um Nachricht nach \uline{Wien}\oindex{Wien@\textbf{Wien}, \emph{Verwaltungsgebiet}|pw}) wahrſcheinlich fahre ich von hier aus in die Schweiz\oindex{Schweiz@\textbf{Schweiz}|pw}. Anfang August{ }ſoll ich dort Mama\pwindex{Schnitzler, Louise 8.\,7.\,1840 Kőszeg – 9.\,9.\,1911 Wien@\textsc{Schnitzler, Louise} (8.\,7.\,1840 Kőszeg – 9.\,9.\,1911 Wien)|pwv} treffen (\textsc{Flims}\oindex{Flims@\textbf{Flims}|pw} von \textsc{Reichenau}\oindex{Reichenau [Schweiz]@\textbf{Reichenau [Schweiz]}|pw}
                – (\textsc{Chur}\oindex{Chur@\textbf{Chur}|pw} – \textsc{Tham}\oindex{Tamins@\textbf{Tamins}, \emph{Verwaltungsgebiet}|pw}) aus 3 Stunden) auf etwa {\pb}8 Tage. Der \textsc{Wörther}ſee\oindex{Wörthersee@\textbf{Wörthersee}, \emph{See}|pw} fiel ins Waſſer, weil Scharlach Gerüchte
               umgingen, und überdies wollte Mama\pwindex{Schnitzler, Louise 8.\,7.\,1840 Kőszeg – 9.\,9.\,1911 Wien@\textsc{Schnitzler, Louise} (8.\,7.\,1840 Kőszeg – 9.\,9.\,1911 Wien)|pwv} nicht zu \textsc{Pundschu}\oindex{Pension Pundschu@\textbf{Pension Pundschu}, \emph{Hotel}|pw}, weil ich nicht wußte, auf wie lang ich hingehn würde. Nun bin ich{ }ſo weit von
               dort, dſs ich Sie heuer im Sommer kaum{ }ſehn werde, we{\geminationn}
               Sie nicht mir, \textsc{resp}. mir und {\pb}Paul Goldmann\pwindex{Goldmann, Paul 31.\,1.\,1865 Breslau – 25.\,9.\,1935 Wien@\textsc{Goldmann, Paul} (31.\,1.\,1865 Breslau – 25.\,9.\,1935 Wien), \emph{Schriftsteller, Journalist}|pw} (von dem ich übrigens noch keine
                  beſti{\geminationm}te Nachricht habe) irgendwie entgegenko{\geminationm}en.\pend
           
\pstart
           Haben Sie{ }ſchon irgendwelche Auguſtpläne? Sie{ }ſchreiben mir wenig, faſt gar nichts
               über{ }ſich; was thun Sie? Arbeiten Sie? Wie gehts Ihrer Frau\pwindex{Beer-Hofmann, Paula 25.\,2.\,1879 Wien – 30.\,10.\,1939 Zürich@\textsc{Beer-Hofmann, Paula} (25.\,2.\,1879 Wien – 30.\,10.\,1939 Zürich)|pwv} und den Kindern\pwindex{Beer-Hofmann, Naëmah 20.\,12.\,1898 Wien – 10.\,11.\,1971 New York City@\textsc{Beer-Hofmann, Naëmah} (20.\,12.\,1898 Wien – 10.\,11.\,1971 New York City)|pwv}\pwindex{Beer-Hofmann, Mirjam 4.\,9.\,1897 Wien – 24.\,12.\,1984 New York City@\textsc{Beer-Hofmann, Mirjam} (4.\,9.\,1897 Wien – 24.\,12.\,1984 New York City)|pwv}?\pend
           
\pstart
           Salten\pwindex{Salten, Felix 6.\,9.\,1869 Budapest – 8.\,10.\,1945 Zürich@\textsc{Salten, Felix} (6.\,9.\,1869 Budapest – 8.\,10.\,1945 Zürich), \emph{Schriftsteller, Journalist, Chefredakteur}|pw} iſt auf Reiſen, {\pb}wie mir eine \label{K_L01140-1v}\edtext{Karte von ihm}{\lemma{\textnormal{\emph{Karte von ihm}}}\Cendnote{\textnormal{Siehe XXXX Auszeichnungsfehler: Dokument L03314 nicht gefunden.
               }}}\label{K_L01140-1} flüchtig mittheilt, aus Brettlgründen\orgindex{Jung-Wiener Theater zum Lieben Augustin@Jung-Wiener Theater zum Lieben Augustin|pwv}. Ich{ }ſchreibe ein 3aktiges Stück\pwindex{Schnitzler, Arthur 15.\,5.\,1862 Wien – 21.\,10.\,1931 ebd.@\textsc{Schnitzler, Arthur} (15.\,5.\,1862 Wien – 21.\,10.\,1931 ebd.), \emph{Schriftsteller, Mediziner}!einsame Weg. Schauspiel in fünf Akten@\strich\emph{Der einsame Weg. Schauspiel in fünf Akten}|pwv}\pwindex{Schnitzler, Arthur 15.\,5.\,1862 Wien – 21.\,10.\,1931 ebd.@\textsc{Schnitzler, Arthur} (15.\,5.\,1862 Wien – 21.\,10.\,1931 ebd.), \emph{Schriftsteller, Mediziner}!Professor Bernhardi. Komödie in fünf Akten@\strich\emph{Professor Bernhardi. Komödie in fünf Akten}|pwv} und glaube im Sommer damit und
               auch mit 2 Einaktern\pwindex{Schnitzler, Arthur 15.\,5.\,1862 Wien – 21.\,10.\,1931 ebd.@\textsc{Schnitzler, Arthur} (15.\,5.\,1862 Wien – 21.\,10.\,1931 ebd.), \emph{Schriftsteller, Mediziner}!Lebendige Stunden@\strich\emph{Lebendige Stunden}|pwv}\pwindex{Schnitzler, Arthur 15.\,5.\,1862 Wien – 21.\,10.\,1931 ebd.@\textsc{Schnitzler, Arthur} (15.\,5.\,1862 Wien – 21.\,10.\,1931 ebd.), \emph{Schriftsteller, Mediziner}!Frau mit dem Dolche@\strich\emph{Die Frau mit dem Dolche}|pwv}
               fertig zu werden. – An \label{K_L01140-2v}\edtext{Hugo\pwindex{Hofmannsthal, Hugo von 1.\,2.\,1874 Wien – 15.\,7.\,1929 Rodaun@\textsc{Hofmannsthal, Hugo von} (1.\,2.\,1874 Wien – 15.\,7.\,1929 Rodaun), \emph{Schriftsteller}|pw} und Gerty\pwindex{Hofmannsthal, Gertrude von 16.\,3.\,1880 Wien – 9.\,11.\,1959 Paddington@\textsc{Hofmannsthal, Gertrude von} (16.\,3.\,1880 Wien – 9.\,11.\,1959 Paddington)|pw}{ }ſauſte ich (\textsc{resp}. wir\pwindex{Schnitzler, Olga 17.\,1.\,1882 Wien – 13.\,1.\,1970 Lugano@\textsc{Schnitzler, Olga} (17.\,1.\,1882 Wien – 13.\,1.\,1970 Lugano), \emph{Schauspielerin, Sängerin}|pwv}) in Innsbruck\oindex{Innsbruck@\textbf{Innsbruck}, \emph{Verwaltungsgebiet}|pw} in einem Einſpänner vorüber}{\lemma{\textnormal{\emph{Hugo … vorüber}}}\Cendnote{\textnormal{Vgl. A. S.: \emph{Tagebuch}, 27. 6. 1901.
               }}}\label{K_L01140-2}. – Innsbruck\oindex{Innsbruck@\textbf{Innsbruck}, \emph{Verwaltungsgebiet}|pw} verſucht ich diesmal Tiroler {\pb}Hof\oindex{Tiroler Hof@\textbf{Tiroler Hof}, \emph{Hotel}|pw}. Ich
               warne Sie. Es iſt{ }ſchmierig und ver\textsc{snobt}. Das{ }ſchönſte
               bisher war natürlich \textsc{Hel\introOben{}l\introOben{}brunn}\oindex{Hellbrunn@\textbf{Hellbrunn}|pw}. Heuer zum erſten Mal hab ich auch das Schloſs\oindex{Hellbrunn@\textbf{Hellbrunn}|pwv} geſehn, innen (nicht das »Monatsſchlößel\oindex{Monatsschlössl@\textbf{Monatsschlössl}, \emph{Schloss}|pw}«,{ }ſondern das ununterbrochene.) –\pend
           
\pstart
           Leben Sie wohl und{ }ſchreiben Sie bald.\pend
           
\pstart
           {\pb}Von Herzen Ihr{\\[\baselineskip]}\spacefill\mbox{Arthur}\pend
           \leftskip=0em{}\selectlanguage{ngerman}\endnumbering\briefempfaengerindex{Beer-Hofmann, Richard@\textsc{Beer-Hofmann, Richard}!zzzSchnitzler, Arthur@\emph{von Arthur Schnitzler}!1901-07-041@{4. 7. 1901}|)be}\mylabel{L01140h}  \newcommand{\dateiname}{L01140}\newcommand{\titel}{Arthur Schnitzler an Richard Beer-Hofmann, 4. 7. 1901}\newcommand{\editorInnen}{Martin Anton Müller und Gerd-Hermann Susen}%% latex-leseansicht-abspann.tex
%% Abspann für die Leseansicht.
%% Der Schalter \ifkorrekturansicht ist bereits durch den Vorspann gesetzt.

%% latex-abspann.tex
%% Gemeinsamer Abspann für Korrekturansicht und Leseansicht.
%% Setzt den Schalter \ifkorrekturansicht voraus (gesetzt in den
%% einbindenden Dateien latex-korrekturansicht-abspann.tex bzw.
%% latex-leseansicht-abspann.tex).
%% ---------------------------------------------------------------

\normalsize

% Das esempio-Environment wird nur in der Leseansicht benötigt
\ifkorrekturansicht\else
\newenvironment{esempio}[3]%
{
    \vspace{1.5ex}
    \rlap{\underline{#1}}
    \par
    \setlength{\parindent}{0cm}
    \nopagebreak
    \leftskip=#2cm
    \rightskip=#3cm
}
{
    \par
}
\fi

\doendnotes{C}
\bigskip
\vfill

\clearpage

\footnotesize

\ifkorrekturansicht
  \lohead{\textsc{register}}
\fi

% theindex-Environment neu definieren ohne reledmac
\makeatletter
\renewenvironment{theindex}{%
  \ifkorrekturansicht
    \section*{\indexname}%
  \else
    \subsubsection*{Index der erwähnten Entitäten}%
  \fi
  \setlength{\parindent}{0pt}%
  \setlength{\parskip}{0pt plus 0.3pt}%
  \let\item\@idxitem
}{%
  \ifkorrekturansicht\clearpage\fi
}
\makeatother

\IfFileExists{\jobname-pw.ind}{\input{\jobname-pw.ind}}{}

% Quellenangabe nur in der Leseansicht
\ifkorrekturansicht\else
% Fallback-Definitionen, falls die .tex-Datei \titel etc. nicht gesetzt hat
\providecommand{\titel}{}
\providecommand{\editorInnen}{}
\providecommand{\dateiname}{\jobname}

\vspace{3cm}

\vfill

\footnotesize
\textsc{Quelle}: \titel. Herausgegeben von {\editorInnen}. In: \emph{Arthur Schnitzler: Briefwechsel mit Autorinnen und Autoren}.
 Digitale Edition, https://schnitzler-briefe.acdh.oeaw.ac.at/{\dateiname}.html (Stand \today)
\fi

\end{document}


