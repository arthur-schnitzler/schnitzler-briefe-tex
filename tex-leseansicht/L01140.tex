\input{../tex-inputs/latex-pdf-vorspann}
\begin{center}
            \textcolor{red}{ENTWURF. ENTZIFFERUNG NOCH NICHT KORREKTURGELESEN}
                      \end{center}
            
               \section[Arthur Schnitzler an Richard Beer-Hofmann, 4. 7. 1901]{ Arthur Schnitzler an Richard Beer-Hofmann, 4. 7. 1901}\nopagebreak\mylabel{v}\rehead{ }\begin{ledgroupsized}[t]{13cm}\normalsize\beginnumbering\briefempfaengerindex{Beer-Hofmann, Richard@\textsc{Beer-Hofmann, Richard}!zzzSchnitzler, Arthur@\emph{von Arthur Schnitzler}!1901-07-041@{4. 7. 1901}|(be} \toendnotes[C]{\smallbreak\pagebreak[2]} \Standort{YCGL, MSS 31.}
\physDesc{Brief, 2 Blätter, 7 Seiten, Umschlag
\newline{}Handschrift: 1) Bleistift, deutsche Kurrent\hspace{1em}2) schwarze Tinte, deutsche Kurrent (\noindent{}Umschlag)\hspace{1em}\newline{}Versand: 1) Stempel: »\nobreak{}\oindex{St. Anton am Arlberg@\textbf{St. Anton am Arlberg}|pwk}St. Anton am Arlberg, 4 7 01\nobreak{}«.  2) Stempel: »\nobreak{}\oindex{Poertschach@\textbf{Pörtschach}|pwk}{\pb}Pörtschach am See, 5 7 01\nobreak{}«. }\buchAbdrucke{\weitereDrucke{Arthur Schnitzler, Richard Beer-Hofmann: \emph{Briefwechsel 1891–1931}. Hg. Konstanze Fliedl. Wien, Zürich: \emph{Europaverlag} 1992, S. 152–153.} }\toendnotes[C]{\smallbreak}\pstart{}{\pb}Herrn \textsc{Dr. Richard
                     Beer-Hofmann}\pend{}\pstart{}\textsc{Pörtschach\oindex{Poertschach@\textbf{Pörtschach}|pw}}\pend{}\pstart{}\textsc{am Wörthersee\oindex{Woerthersee@\textbf{Wörthersee}|pw}}\pend{}\pstart{}\textsc{Villa Arnstein\oindex{Villa Arnstein@\textbf{Villa Arnstein}|pw}}\pend{}{\bigskip}\pstart
           \raggedleft{}{\pb}\textsc{St. Anton a (Arlberg)}\oindex{St. Anton am Arlberg@\textbf{St. Anton am Arlberg}|pw}{\\}4. 7. 901.\pend
           \pstart
           mein lieber Richard, ich war zuerſt 14 Tage in Salzburg\oindex{Salzburg@\textbf{Salzburg}|pw}, oeſterr Hof\oindex{Oesterreichischer Hof@\textbf{Österreichischer Hof}|pw}, mit ihr\pwindex{Schnitzler, Olga 17.01.1882 – 13.01.1970@\textsc{Schnitzler, Olga} (17.01.1882 – 13.01.1970), \emph{Schauspielerin, Sängerin}|pwv}, es war ſehr ſchön. Dann
               2 Tage Innsbruck\oindex{Innsbruck@\textbf{Innsbruck}|pw} (daſs ich Schönberg\oindex{Schoenberg im Stubaital@\textbf{Schönberg im Stubaital}|pw} aufgeſucht habe, wiſſen Sie), da{\geminationn} fuhren wir nach \textsc{Landeck}\oindex{Landeck@\textbf{Landeck}|pw}, wo ihre Schweſter\pwindex{Steinrueck, Elisabeth 19.11.1885 – 07.04.1920@\textsc{Steinrück, Elisabeth} (19.11.1885 – 07.04.1920)|pwv} kam,
               und nun ſind wir in \textsc{St. Anton}\oindex{St. Anton am Arlberg@\textbf{St. Anton am Arlberg}|pw} – ich habe ein \introOben{}ſehr behagliches\introOben{}
               Zimmer zu 60 Kreuzer in einem Privat{\pb}haus, und es wäre
               ſehr nett, we{\geminationn} nicht das Wetter elend wäre. Wie lang ich
               hier bleibe, ka{\geminationn} ich natürlich \introOben{}nicht\introOben{} ſagen (daher bitte ich um Nachricht nach \uline{Wien}\oindex{Wien@\textbf{Wien}|pw}) wahrſcheinlich fahre ich von hier aus in die Schweiz\oindex{Schweiz@\textbf{Schweiz}|pw}. Anfang August ſoll ich dort Mama\pwindex{Schnitzler, Louise 08.07.1840 – 09.09.1911@\textsc{Schnitzler, Louise} (08.07.1840 – 09.09.1911)|pwv} treffen (\textsc{Flims}\oindex{Flims@\textbf{Flims}|pw} von \textsc{Reichenau}\oindex{Reichenau@\textbf{Reichenau}|pw} – (\textsc{Chur}\oindex{Chur@\textbf{Chur}|pw} – \textsc{Tham}\oindex{Tamins@\textbf{Tamins}|pw}) aus 3 Stunden) auf etwa {\pb}8 Tage. Der \textsc{Wörther}ſee\oindex{Woerthersee@\textbf{Wörthersee}|pw} fiel ins Waſſer, weil Scharlach Gerüchte
               umgingen, und überdies wollte Mama\pwindex{Schnitzler, Louise 08.07.1840 – 09.09.1911@\textsc{Schnitzler, Louise} (08.07.1840 – 09.09.1911)|pwv} nicht zu \textsc{Pundschu}\oindex{Pension Pundschu@\textbf{Pension Pundschu}|pw}, weil ich nicht
               wußte, auf wie lang ich hingehn würde. Nun bin ich ſo weit von dort, dſs ich Sie
               heuer im Sommer kaum ſehn werde, we{\geminationn} Sie nicht mir, \textsc{resp}. mir und {\pb}Paul Goldmann\pwindex{Goldmann, Paul 31.01.1865 – 25.09.1935@\textsc{Goldmann, Paul} (31.01.1865 – 25.09.1935), \emph{Schriftsteller, Journalist}|pw} (von dem ich übrigens noch keine
                  beſti{\geminationm}te Nachricht habe) irgendwie entgegenko{\geminationm}en.\pend
           \pstart
           Haben Sie ſchon irgendwelche Auguſtpläne? Sie ſchreiben mir wenig, faſt gar nichts
               über ſich; was thun Sie? Arbeiten Sie? Wie gehts Ihrer Frau\pwindex{Beer-Hofmann, Paula 25.02.1879 – 30.10.1939@\textsc{Beer-Hofmann, Paula} (25.02.1879 – 30.10.1939)|pwv} und den Kindern\pwindex{Beer-Hofmann, Naemah 20.12.1898 – 10.11.1971@\textsc{Beer-Hofmann, Naëmah} (20.12.1898 – 10.11.1971)|pwv}\pwindex{Beer-Hofmann, Mirjam 04.09.1897 – 24.12.1984@\textsc{Beer-Hofmann, Mirjam} (04.09.1897 – 24.12.1984)|pwv}?\pend
           \pstart
           Salten\pwindex{Salten, Felix 06.09.1869 – 08.10.1945@\textsc{Salten, Felix} (06.09.1869 – 08.10.1945), \emph{Schriftsteller, Journalist}|pw} iſt auf Reiſen, {\pb}wie mir eine Karte von ihm flüchtig mittheilt, aus Brettlgründen\oindex{Jung-Wiener Theater zum Lieben Augustin@\textbf{Jung-Wiener Theater zum Lieben Augustin}|pwv}. Ich ſchreibe ein
               3aktiges Stück\pwindex{Schnitzler, Arthur 15.05.1862 – 21.10.1931@\textsc{Schnitzler, Arthur} (15.05.1862 – 21.10.1931), \emph{Schriftsteller, Mediziner}!einsame Weg. Schauspiel in fuenf Akten1904@\strich\emph{Der einsame Weg. Schauspiel in fünf Akten} {[}1904{]}|pwv}\pwindex{Schnitzler, Arthur 15.05.1862 – 21.10.1931@\textsc{Schnitzler, Arthur} (15.05.1862 – 21.10.1931), \emph{Schriftsteller, Mediziner}!Professor Bernhardi. Komoedie in fuenf Akten1912@\strich\emph{Professor Bernhardi. Komödie in fünf Akten} {[}1912{]}|pwv} und glaube
               im Sommer damit und auch mit 2 Einaktern\pwindex{Schnitzler, Arthur 15.05.1862 – 21.10.1931@\textsc{Schnitzler, Arthur} (15.05.1862 – 21.10.1931), \emph{Schriftsteller, Mediziner}!Lebendige Stunden01. 12. 1901@\strich\emph{Lebendige Stunden} {[}01. 12. 1901{]}|pwv}\pwindex{Schnitzler, Arthur 15.05.1862 – 21.10.1931@\textsc{Schnitzler, Arthur} (15.05.1862 – 21.10.1931), \emph{Schriftsteller, Mediziner}!Frau mit dem Dolche1901@\strich\emph{Die Frau mit dem Dolche} {[}1901{]}|pwv} fertig zu werden. – An Hugo\pwindex{Hofmannsthal, Hugo von 01.02.1874 – 15.07.1929@\textsc{Hofmannsthal, Hugo von} (01.02.1874 – 15.07.1929), \emph{Schriftsteller}|pw} und Gerty\pwindex{Hofmannsthal, Gertrude von 16.03.1880 – 09.11.1959@\textsc{Hofmannsthal, Gertrude von} (16.03.1880 – 09.11.1959)|pw}{ }ſauſte ich (\textsc{resp}. wir\pwindex{Schnitzler, Olga 17.01.1882 – 13.01.1970@\textsc{Schnitzler, Olga} (17.01.1882 – 13.01.1970), \emph{Schauspielerin, Sängerin}|pwv}) in
                  Innsbruck\oindex{Innsbruck@\textbf{Innsbruck}|pw} in einem Einſpänner vorüber. –
               Innsbruck\oindex{Innsbruck@\textbf{Innsbruck}|pw}
               verſucht ich diesmal Tiroler {\pb}Hof\oindex{Tiroler Hof@\textbf{Tiroler Hof}|pw}. Ich warne Sie. Es iſt ſchmierig und
                  ver\textsc{snobt}. Das ſchönſte bisher war natürlich \textsc{Hel\introOben{}l\introOben{}brunn}\oindex{Hellbrunn@\textbf{Hellbrunn}|pw}. Heuer zum erſten Mal hab ich auch das Schloſs\oindex{Hellbrunn@\textbf{Hellbrunn}|pwv} geſehn, innen (nicht das »Monatsſchlößel\oindex{Monatsschloessl@\textbf{Monatsschlössl}|pw}«, ſondern das ununterbrochene.) –\pend
           \pstart
           Leben Sie wohl und ſchreiben Sie bald.\pend
           \pstart
           {\pb}Von Herzen Ihr{\\[\baselineskip]}\spacefill\mbox{Arthur}\pend
           \leftskip=0em{}\endnumbering\briefempfaengerindex{Beer-Hofmann, Richard@\textsc{Beer-Hofmann, Richard}!zzzSchnitzler, Arthur@\emph{von Arthur Schnitzler}!1901-07-041@{4. 7. 1901}|)be}\mylabel{h}\end{ledgroupsized}  \newcommand{\dateiname}{L01140}\newcommand{\titel}{Arthur Schnitzler an Richard Beer-Hofmann, 4. 7. 1901}\newcommand{\editorInnen}{Martin Anton Müller und Gerd-Hermann Susen}\input{../tex-inputs/latex-pdf-abspann}
      