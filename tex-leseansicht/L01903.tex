%% latex-leseansicht-vorspann.tex
%% Vorspann für die Leseansicht.
%% Lädt die gemeinsame Datei latex-vorspann.tex mit nicht gesetztem Schalter.

\newif\ifkorrekturansicht
\korrekturansichtfalse

\input{../tex-inputs/latex-vorspann}


\section[Arthur Schnitzler an Richard Beer-Hofmann, 16. 12. 1909]{L01903 Arthur Schnitzler an Richard Beer-Hofmann, 16. 12. 1909}
\nopagebreak\mylabel{L01903v}
\rehead{ }\normalsize\beginnumbering\briefempfaengerindex{Beer-Hofmann, Richard@\textsc{Beer-Hofmann, Richard}!zzzSchnitzler, Arthur@\emph{von Arthur Schnitzler}!1909-12-161@{16. 12. 1909}|(be}
\toendnotes[C]{\smallbreak\pagebreak[2]}
\correspDesc{Versand  durch Arthur Schnitzler am 16. 12. 1909 in Wien
\newline{}Erhalt  durch Richard Beer-Hofmann am 16. 12. 1909 in Wien}\toendnotes[C]{\smallbreak}
\Standort{YCGL, MSS 31.}
\physDesc{Briefkarte, , Kuvert, 463 Zeichen
\newline{}Handschrift: Bleistift, deutsche Kurrent
\newline{}Versand: ohne postalischen Übermittlungsvermerk }
\buchAbdrucke{\weitereDrucke{Arthur Schnitzler, Richard Beer-Hofmann: \emph{Briefwechsel 1891–1931}. Herausgegeben von Konstanze Fliedl. Wien, Zürich: \emph{Europaverlag} 1992, S. 206.} }\toendnotes[C]{\smallbreak}\pstart{}{\pb}\textcolor{gray}{\textbf{Dr. Arthur Schnitzler}}\pend{}\pstart{}\textcolor{gray}{\textbf{Wien XVIII. Spoettelgasse 7\oindex{Wien@\textbf{Wien}!XVIII., Währing@\textbf{XVIII., Währing}!Edmund-Weiß-Gasse 7@\textbf{Edmund-Weiß-Gasse 7}, \emph{Wohngebäude}|pw}.}}\pend{}{\bigskip}\pstart{}{\pb}Hrn \textsc{Dr. Rich.
                     Beer-Hofmann}\pend{}\pstart{}Wien\oindex{Wien@\textbf{Wien}, \emph{Verwaltungsgebiet}|pw}\pend{}{\bigskip}\vspace{1em}
\pstart
           {\pb}\textcolor{gray}{\textbf{Dr. Arthur Schnitzler}}\hfill 16. 12. 09\pend
           
\pstart
           \textcolor{gray}{\textbf{Wien XVIII. Spoettelgasse 7\oindex{Wien@\textbf{Wien}!XVIII., Währing@\textbf{XVIII., Währing}!Edmund-Weiß-Gasse 7@\textbf{Edmund-Weiß-Gasse 7}, \emph{Wohngebäude}|pw}.}}\pend
           \vspace{0.5em}
\pstart
           lieber Richard, heute Abend ka{\geminationn} ich \textsc{Poldi}\pwindex{Andrian-Werburg, Leopold von 9.\,5.\,1875 Berlin – 19.\,11.\,1951 Fribourg@\textsc{Andrian-Werburg, Leopold von} (9.\,5.\,1875 Berlin – 19.\,11.\,1951 Fribourg), \emph{Schriftsteller, Diplomat}|pw} nicht erwarten, gehe eben, längſt geladen, mit Olga\pwindex{Schnitzler, Olga 17.\,1.\,1882 Wien – 13.\,1.\,1970 Lugano@\textsc{Schnitzler, Olga} (17.\,1.\,1882 Wien – 13.\,1.\,1970 Lugano), \emph{Schauspielerin, Sängerin}|pw} zu \textsc{Speidels}\pwindex{Speidel-Haeberle, Else 11.\,7.\,1877 Stuttgart – 21.\,7.\,1937 Augustenfeld@\textsc{Speidel-Haeberle, Else} (11.\,7.\,1877 Stuttgart – 21.\,7.\,1937 Augustenfeld), \emph{Schauspielerin}|pw}\pwindex{Speidel, Felix 2.\,7.\,1875 Stuttgart – 3.\,10.\,1952 Unterach am Attersee@\textsc{Speidel, Felix} (2.\,7.\,1875 Stuttgart – 3.\,10.\,1952 Unterach am Attersee), \emph{Schriftsteller, Verleger}|pw}; morgen früh{\dotstwo} doch eben{ }ſeh ich, daſs
               er{ }ſchon morgen früh abreiſt. {\pb}Nun, für alle Fälle,
               von ½ 10–10 bin ich zu Hauſe.\pend
           
\pstart
           Herzlichſt{\\[\baselineskip]}Ihr{\\[\baselineskip]}\spacefill\mbox{A.}\pend
           \leftskip=0em{}
\pstart
           \noindent{}(Aber, we{\geminationn} nicht dringend gewünſcht,{ }ſagen Sie’s
                  nicht.\pend
           
\pstart
           Gratulire herzlich zu Ihrem Telephon\pend
           
\pstart
           Der Einfachheit wegen könnten Sie eigentlich {\pb}\label{T_L01903-1v}\edtext{telephoniren, Ihre Karte habe uns
                  nicht mehr zu Haus getroffen{[}){]}}{\lemma{\textnormal{\emph{telephoniren, … getroffen)}}}\Cendnote{\textnormal{auf der ersten Seite am oberen Rand,
                     verkehrt zum Text}}}\label{T_L01903-1}\pend
           \selectlanguage{ngerman}\endnumbering\briefempfaengerindex{Beer-Hofmann, Richard@\textsc{Beer-Hofmann, Richard}!zzzSchnitzler, Arthur@\emph{von Arthur Schnitzler}!1909-12-161@{16. 12. 1909}|)be}\mylabel{L01903h}  \newcommand{\dateiname}{L01903}\newcommand{\titel}{Arthur Schnitzler an Richard Beer-Hofmann, 16. 12. 1909}\newcommand{\editorInnen}{Martin Anton Müller und Gerd-Hermann Susen}%% latex-leseansicht-abspann.tex
%% Abspann für die Leseansicht.
%% Der Schalter \ifkorrekturansicht ist bereits durch den Vorspann gesetzt.

%% latex-abspann.tex
%% Gemeinsamer Abspann für Korrekturansicht und Leseansicht.
%% Setzt den Schalter \ifkorrekturansicht voraus (gesetzt in den
%% einbindenden Dateien latex-korrekturansicht-abspann.tex bzw.
%% latex-leseansicht-abspann.tex).
%% ---------------------------------------------------------------

\normalsize

% Das esempio-Environment wird nur in der Leseansicht benötigt
\ifkorrekturansicht\else
\newenvironment{esempio}[3]%
{
    \vspace{1.5ex}
    \rlap{\underline{#1}}
    \par
    \setlength{\parindent}{0cm}
    \nopagebreak
    \leftskip=#2cm
    \rightskip=#3cm
}
{
    \par
}
\fi

\doendnotes{C}
\bigskip
\vfill

\clearpage

\footnotesize

\ifkorrekturansicht
  \lohead{\textsc{register}}
\fi

% theindex-Environment neu definieren ohne reledmac
\makeatletter
\renewenvironment{theindex}{%
  \ifkorrekturansicht
    \section*{\indexname}%
  \else
    \subsubsection*{Index der erwähnten Entitäten}%
  \fi
  \setlength{\parindent}{0pt}%
  \setlength{\parskip}{0pt plus 0.3pt}%
  \let\item\@idxitem
}{%
  \ifkorrekturansicht\clearpage\fi
}
\makeatother

\IfFileExists{\jobname-pw.ind}{\input{\jobname-pw.ind}}{}

% Quellenangabe nur in der Leseansicht
\ifkorrekturansicht\else
% Fallback-Definitionen, falls die .tex-Datei \titel etc. nicht gesetzt hat
\providecommand{\titel}{}
\providecommand{\editorInnen}{}
\providecommand{\dateiname}{\jobname}

\vspace{3cm}

\vfill

\footnotesize
\textsc{Quelle}: \titel. Herausgegeben von {\editorInnen}. In: \emph{Arthur Schnitzler: Briefwechsel mit Autorinnen und Autoren}.
 Digitale Edition, https://schnitzler-briefe.acdh.oeaw.ac.at/{\dateiname}.html (Stand \today)
\fi

\end{document}


