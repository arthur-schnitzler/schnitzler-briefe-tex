%% latex-korrekturansicht-vorspann.tex
%% Vorspann für die Korrekturansicht.
%% Lädt die gemeinsame Datei latex-vorspann.tex mit gesetztem Schalter.

\newif\ifkorrekturansicht
\korrekturansichttrue

\input{../tex-inputs/latex-vorspann}


\section[Arthur Schnitzler an Hermann Bahr, 19. 11. 1910]{L01982 Arthur Schnitzler an Hermann Bahr, 19. 11. 1910}
\nopagebreak\mylabel{L01982v}
\rehead{ }\normalsize\beginnumbering\briefempfaengerindex{Bahr, Hermann@\textsc{Bahr, Hermann}!zzzSchnitzler, Arthur@\emph{von Arthur Schnitzler}!1910-11-191@{19. 11. 1910}|(be}
\toendnotes[C]{\smallbreak\pagebreak[2]}\Standort{TMW, HS AM 60141 Ba.}
\physDesc{Briefkarte, 512 Zeichen
\newline{}Handschrift: schwarze Tinte, deutsche Kurrent}
\buchAbdrucke{\weitereDrucke{1) Arthur Schnitzler: \emph{The Letters of Arthur Schnitzler to Hermann Bahr}. Chapel Hill: \emph{The University of North Carolina Press} 1978, S. 108.} \weitereDrucke{2) Hermann Bahr, Arthur Schnitzler: \emph{Briefwechsel, Aufzeichnungen, Dokumente (1891–1931)}. Göttingen: \emph{Wallstein} 2018, S. 445.} }\toendnotes[C]{\smallbreak}
\pstart
           {\pb}\textcolor{gray}{\textbf{Dr. Arthur Schnitzler}}\hfill 19. 11. 910.\pend
           
\pstart
           \textcolor{gray}{\textbf{Wien XVIII. Sternwartestrasse 71\oindex{Sternwartestrasse 71@\textbf{Sternwartestraße 71}, \emph{Wohngebäude (K.WHS)}|pw}}}\pend
           \vspace{0.5em}
\pstart
           mein lieber Hermann, beim Durchſehen der Abſchrift meines letzten
               Briefes an dich merk ich daſs meine Schreiberin\pwindex{Pollak, Frieda 08.12.1881 – 13.07.1937@\textsc{Pollak, Frieda} (08.12.1881 – 13.07.1937), \emph{Sekretär/Sekretärin}|pwv} eine Stelle (»dies ganz unter uns«) irrtümlich
               unter- ſtatt durchſtrichen hat. Zur Vermeidg von Misverſtändniſſen: es iſt natürlich
               kein Geheimnis, daſs die Burg\oindex{Burgtheater@\textbf{Burgtheater}, \emph{S.THTR}|pw} heute keinen \textsc{Me{\pb}dardus\pwindex{junge Medardus. Dramatische Historie in einem Vorspiel und fuenf Aufzuegen@\emph{Der junge Medardus. Dramatische Historie in einem Vorspiel und fünf Aufzügen}|pw}} hat. Mir war nur eine Bemerkung gegen \textsc{Gerasch\pwindex{Gerasch, Alfred 17.08.1877 – 12.08.1955@\textsc{Gerasch, Alfred} (17.08.1877 – 12.08.1955), \emph{Schauspieler/Schauspielerin}|pw}} (perſönlicher Art) beim Dictiren durch den Kopf gegangen, die aber, vor der
               Aufführung auszuſprechen ich nicht richtig gefunden hätte.\pend
           
\pstart
           Pedantiſch und herzlichſt{\\[\baselineskip]}dein{\\[\baselineskip]}\spacefill\mbox{A.}\pend
           \leftskip=0em{}\selectlanguage{ngerman}\endnumbering\briefempfaengerindex{Bahr, Hermann@\textsc{Bahr, Hermann}!zzzSchnitzler, Arthur@\emph{von Arthur Schnitzler}!1910-11-191@{19. 11. 1910}|)be}\mylabel{L01982h}  \normalsize

\doendnotes{C}
\bigskip
\vfill

\clearpage

\footnotesize

\lohead{\textsc{register}}

% Definiere theindex-Environment komplett neu ohne reledmac
\makeatletter
\renewenvironment{theindex}{%
  \section*{\indexname}%
  \setlength{\parindent}{0pt}%
  \setlength{\parskip}{0pt plus 0.3pt}%
  \let\item\@idxitem
}{%
  \clearpage
}
\makeatother

\IfFileExists{\jobname-pw.ind}{\input{\jobname-pw.ind}}{}

\end{document}

      