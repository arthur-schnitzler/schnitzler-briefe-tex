%% latex-leseansicht-vorspann.tex
%% Vorspann für die Leseansicht.
%% Lädt die gemeinsame Datei latex-vorspann.tex mit nicht gesetztem Schalter.

\newif\ifkorrekturansicht
\korrekturansichtfalse

\input{../tex-inputs/latex-vorspann}


\section[Arthur Schnitzler an Hermann Bahr, 19. 11. 1910]{L01982 Arthur Schnitzler an Hermann Bahr, 19. 11. 1910}
\nopagebreak\mylabel{L01982v}
\rehead{ }\normalsize\beginnumbering\briefempfaengerindex{Bahr, Hermann@\textsc{Bahr, Hermann}!zzzSchnitzler, Arthur@\emph{von Arthur Schnitzler}!1910-11-191@{19. 11. 1910}|(be}
\toendnotes[C]{\smallbreak\pagebreak[2]}
\correspDesc{Versand  durch Arthur Schnitzler am 19. 11. 1910 in Wien
\newline{}Erhalt  durch Hermann Bahr im Zeitraum [19. 11. 1910 – 23. 11. 1910?] \textbf{Ort fehlend} }\toendnotes[C]{\smallbreak}
\Standort{TMW, HS AM 60141 Ba.}
\physDesc{Briefkarte, 512 Zeichen
\newline{}Handschrift: schwarze Tinte, deutsche Kurrent}
\buchAbdrucke{\weitereDrucke{1) \emph{19. 11. 1910, Abschrift.} In: Arthur Schnitzler: \emph{The Letters of Arthur Schnitzler to Hermann Bahr}. Edited, annotated, and with an introduction, by Donald G. Daviau. Chapel Hill: \emph{The University of North Carolina Press} 1978, S. 108 (University of North Carolina studies in the Germanic languages
                        and literatures, 89).} \weitereDrucke{2) Hermann Bahr, Arthur Schnitzler: \emph{Briefwechsel, Aufzeichnungen, Dokumente (1891–1931)}. Herausgegeben von Kurt Ifkovits und Martin Anton Müller. Göttingen: \emph{Wallstein} 2018, S. 445.} }\toendnotes[C]{\smallbreak}
\pstart
           {\pb}\textcolor{gray}{\textbf{Dr. Arthur Schnitzler}}\hfill 19. 11. 910.\pend
           
\pstart
           \textcolor{gray}{\textbf{Wien XVIII. Sternwartestrasse 71\oindex{Wien@\textbf{Wien}!XVIII., Währing@\textbf{XVIII., Währing}!Sternwartestraße 71@\textbf{Sternwartestraße 71}, \emph{Wohngebäude}|pw}}}\pend
           \vspace{0.5em}
\pstart
           mein lieber Hermann, beim Durchſehen der Abſchrift meines letzten
               Briefes an dich merk ich daſs meine Schreiberin\pwindex{Pollak, Frieda 8.\,12.\,1881 Wien – 13.\,7.\,1937 ebd.@\textsc{Pollak, Frieda} (8.\,12.\,1881 Wien – 13.\,7.\,1937 ebd.), \emph{Sekretärin}|pwv} eine Stelle (»dies ganz unter uns«) irrtümlich
               unter-{ }ſtatt durchſtrichen hat. Zur Vermeidg von Misverſtändniſſen: es iſt natürlich
               kein Geheimnis, daſs die Burg\oindex{Wien@\textbf{Wien}!I., Innere Stadt@\textbf{I., Innere Stadt}!Burgtheater@\textbf{Burgtheater}, \emph{Theater}|pw} heute keinen \textsc{Me{\pb}dardus\pwindex{Schnitzler, Arthur 15.\,5.\,1862 Wien – 21.\,10.\,1931 ebd.@\textsc{Schnitzler, Arthur} (15.\,5.\,1862 Wien – 21.\,10.\,1931 ebd.), \emph{Schriftsteller, Mediziner}!junge Medardus. Dramatische Historie in einem Vorspiel und fünf Aufzügen@\strich\emph{Der junge Medardus. Dramatische Historie in einem Vorspiel und fünf Aufzügen}|pw}} hat. Mir war nur eine Bemerkung gegen \textsc{Gerasch\pwindex{Gerasch, Alfred 17.\,8.\,1877 Berlin – 12.\,8.\,1955 Wien@\textsc{Gerasch, Alfred} (17.\,8.\,1877 Berlin – 12.\,8.\,1955 Wien), \emph{Schauspieler}|pw}} (perſönlicher Art) beim Dictiren durch den Kopf gegangen, die aber, vor der
               Aufführung auszuſprechen ich nicht richtig gefunden hätte.\pend
           
\pstart
           Pedantiſch und herzlichſt{\\[\baselineskip]}dein{\\[\baselineskip]}\spacefill\mbox{A.}\pend
           \leftskip=0em{}\selectlanguage{ngerman}\endnumbering\briefempfaengerindex{Bahr, Hermann@\textsc{Bahr, Hermann}!zzzSchnitzler, Arthur@\emph{von Arthur Schnitzler}!1910-11-191@{19. 11. 1910}|)be}\mylabel{L01982h}  \newcommand{\dateiname}{L01982}\newcommand{\titel}{Arthur Schnitzler an Hermann Bahr, 19. 11. 1910}\newcommand{\editorInnen}{Herausgegeben von Martin Anton Müller}%% latex-leseansicht-abspann.tex
%% Abspann für die Leseansicht.
%% Der Schalter \ifkorrekturansicht ist bereits durch den Vorspann gesetzt.

%% latex-abspann.tex
%% Gemeinsamer Abspann für Korrekturansicht und Leseansicht.
%% Setzt den Schalter \ifkorrekturansicht voraus (gesetzt in den
%% einbindenden Dateien latex-korrekturansicht-abspann.tex bzw.
%% latex-leseansicht-abspann.tex).
%% ---------------------------------------------------------------

\normalsize

% Das esempio-Environment wird nur in der Leseansicht benötigt
\ifkorrekturansicht\else
\newenvironment{esempio}[3]%
{
    \vspace{1.5ex}
    \rlap{\underline{#1}}
    \par
    \setlength{\parindent}{0cm}
    \nopagebreak
    \leftskip=#2cm
    \rightskip=#3cm
}
{
    \par
}
\fi

\doendnotes{C}
\bigskip
\vfill

\clearpage

\footnotesize

\ifkorrekturansicht
  \lohead{\textsc{register}}
\fi

% theindex-Environment neu definieren ohne reledmac
\makeatletter
\renewenvironment{theindex}{%
  \ifkorrekturansicht
    \section*{\indexname}%
  \else
    \subsubsection*{Index der erwähnten Entitäten}%
  \fi
  \setlength{\parindent}{0pt}%
  \setlength{\parskip}{0pt plus 0.3pt}%
  \let\item\@idxitem
}{%
  \ifkorrekturansicht\clearpage\fi
}
\makeatother

\IfFileExists{\jobname-pw.ind}{\input{\jobname-pw.ind}}{}

% Quellenangabe nur in der Leseansicht
\ifkorrekturansicht\else
% Fallback-Definitionen, falls die .tex-Datei \titel etc. nicht gesetzt hat
\providecommand{\titel}{}
\providecommand{\editorInnen}{}
\providecommand{\dateiname}{\jobname}

\vspace{3cm}

\vfill

\footnotesize
\textsc{Quelle}: \titel. Herausgegeben von {\editorInnen}. In: \emph{Arthur Schnitzler: Briefwechsel mit Autorinnen und Autoren}.
 Digitale Edition, https://schnitzler-briefe.acdh.oeaw.ac.at/{\dateiname}.html (Stand \today)
\fi

\end{document}


