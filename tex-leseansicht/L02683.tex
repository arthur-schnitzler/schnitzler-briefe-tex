%% latex-korrekturansicht-vorspann.tex
%% Vorspann für die Korrekturansicht.
%% Lädt die gemeinsame Datei latex-vorspann.tex mit gesetztem Schalter.

\newif\ifkorrekturansicht
\korrekturansichttrue

\input{../tex-inputs/latex-vorspann}


\section[Paul Goldmann an Arthur Schnitzler, 13. 10. {[}1899?{]}]{L02683 Paul Goldmann an Arthur Schnitzler, 13. 10. {[}1899?{]}}
\nopagebreak\mylabel{L02683v}
\rehead{ }\normalsize\beginnumbering\briefempfaengerindex{Schnitzler, Arthur@\textsc{Schnitzler, Arthur}!zzzGoldmann, Paul@\emph{von Paul Goldmann}!1899-10-131@{13. 10. {[}1899?{]}}|(be}
\toendnotes[C]{\smallbreak\pagebreak[2]}\Standort{DLA, A:Schnitzler, HS.NZ85.1.3169.}
\physDesc{Telegramm, 70 Zeichen
\newline{}maschinell
\newline{}Ordnung: beschnitten }\toendnotes[C]{\smallbreak}
\pstart
           \centering{}{\pb}wien\oindex{Wien@\textbf{Wien}, \emph{A.ADM2}|pw} fr pontafel\oindex{Pontebba@\textbf{Pontebba}, \emph{P.PPLA3}|pw} 55 12 13/10{ }10/15 = \pend
           \vspace{0.5em}
\pstart
           \label{K_L02683-1v}\edtext{ankomme}{\lemma{\textnormal{\emph{ankomme}}}\Cendnote{\textnormal{Goldmann\pwindex{Goldmann, Paul 31.01.1865 – 25.09.1935@\textsc{Goldmann, Paul} (31.01.1865 – 25.09.1935), \emph{Schriftsteller/Schriftstellerin, Journalist/Journalistin}|pwk} wurde von Schnitzler und Gustav
                     Schwarzkopf\pwindex{Schwarzkopf, Gustav 07.11.1853 – 13.11.1939@\textsc{Schwarzkopf, Gustav} (07.11.1853 – 13.11.1939), \emph{Schriftsteller/Schriftstellerin}|pwk} in Wien\oindex{Wien@\textbf{Wien}, \emph{A.ADM2}|pwk} empfangen. Er blieb
                  bis zum 21. 10. 1899.}}}\label{K_L02683-1}{ }heute{ }abend{ }9 uhr 45 = \spacefill\mbox{goldmann =}\pend
           \selectlanguage{ngerman}\endnumbering\briefempfaengerindex{Schnitzler, Arthur@\textsc{Schnitzler, Arthur}!zzzGoldmann, Paul@\emph{von Paul Goldmann}!1899-10-131@{13. 10. {[}1899?{]}}|)be}\mylabel{L02683h}  \normalsize

\doendnotes{C}
\bigskip
\vfill

\clearpage

\footnotesize

\lohead{\textsc{register}}

% Definiere theindex-Environment komplett neu ohne reledmac
\makeatletter
\renewenvironment{theindex}{%
  \section*{\indexname}%
  \setlength{\parindent}{0pt}%
  \setlength{\parskip}{0pt plus 0.3pt}%
  \let\item\@idxitem
}{%
  \clearpage
}
\makeatother

\IfFileExists{\jobname-pw.ind}{\input{\jobname-pw.ind}}{}

\end{document}

      