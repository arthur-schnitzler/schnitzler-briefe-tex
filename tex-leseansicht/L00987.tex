%% latex-leseansicht-vorspann.tex
%% Vorspann für die Leseansicht.
%% Lädt die gemeinsame Datei latex-vorspann.tex mit nicht gesetztem Schalter.

\newif\ifkorrekturansicht
\korrekturansichtfalse

\input{../tex-inputs/latex-vorspann}


         
         \renewcommand{\erwaehntePersonen}{Personen:  ?? [Schreibkraft der Menükarte 3.10.1899], Richard Beer-Hofmann}
         \renewcommand{\erwaehnteOrte}{Orte: Berlin, Hotel Savoy, Hôtel du Parc {\kaufmannsund}  Bristol, Sankt Michael, Wiesbaden}
         \renewcommand{\erwaehnteWerke}{Werke: Der Graf von Charolais. Ein Trauerspiel, Der Schleier der Beatrice. Schauspiel in fünf Akten}
               \section[Arthur Schnitzler an Richard Beer-Hofmann, 3. 10. 1899]{ Arthur Schnitzler an Richard Beer-Hofmann, 3. 10. 1899}\nopagebreak\mylabel{v}\rehead{ }\begin{ledgroupsized}[t]{13cm}\normalsize\beginnumbering\briefempfaengerindex{Beer-Hofmann, Richard@\textsc{Beer-Hofmann, Richard}!zzzSchnitzler, Arthur@\emph{von Arthur Schnitzler}!1899-10-031@{3. 10. 1899}|(be} \toendnotes[C]{\smallbreak\pagebreak[2]} \Standort{YCGL, MSS 31.}
\physDesc{Karte, 431 Zeichen (Klappkarte)
\newline{}Handschrift Arthur Schnitzler: 1) Bleistift, deutsche Kurrent\hspace{1em}2) Bleistift, lateinische Kurrent (\noindent{}Adresse)\hspace{1em}\newline{}Handschrift Schreibkraft: blaue Tinte, lateinische Kurrent (\noindent{}Speisenfolge)
\newline{}Versand: 1) Stempel: »\nobreak{}\oindex{Wiesbaden@\textbf{Wiesbaden}|pwk}Wiesbaden, 3. 10. 99, 3–4N\nobreak{}«.   2) Stempel: »\nobreak{}6. {[}10.{]} 99, \oindex{Sankt Michael@\textbf{Sankt Michael}|pwk}St. Michael Eppan\nobreak{}«. 
\newline{}Ordnung: mit Bleistift von unbekannter Hand datiert: »3. 10.« }\toendnotes[C]{\smallbreak}\pstart{}{\pb}Dr. Richard Beer-Hofmann\pend{}\pstart{}St. Michael in Eppan\oindex{Sankt Michael@\textbf{Sankt Michael}|pw}\pend{}{\bigskip}\pstart
           \noindent{}\centering{}\textcolor{gray}{\textbf{{\pb}Wiesbaden\oindex{Wiesbaden@\textbf{Wiesbaden}|pw}. Blick aus dem Hotel du Parc et Bristol\oindex{Hôtel du Parc {\kaufmannsund} Bristol@\textbf{Hôtel du Parc {\kaufmannsund} Bristol}|pw}}}\pend
           \pstart
           {\pb}Heute Abd fahr ich nach Berlin\oindex{Berlin@\textbf{Berlin}|pw}. – Will mein Stück\pwindex{Schnitzler, Arthur 15.05.1862 – 21.10.1931@\textsc{Schnitzler, Arthur} (15.05.1862 – 21.10.1931), \emph{Schriftsteller, Mediziner}!Schleier der Beatrice. Schauspiel in fuenf Akten1900-12-01@\strich\emph{Der Schleier der Beatrice. Schauspiel in fünf Akten} {[}1900-12-01{]}|pwv} nochmals umarbeiten. – Bleibe in Berlin\oindex{Berlin@\textbf{Berlin}|pw} wahrſcheinlich bis Sonntag. Wohne dort \textsc{Hotel Savoy}\oindex{Hotel Savoy@\textbf{Hotel Savoy}|pw}. Viele herzl Grüße. Ich freue mich über Ihre 420 Verſe\pwindex{Beer-Hofmann, Richard 1866-07-11 – 1945-09-26@\textsc{Beer-Hofmann, Richard} (1866-07-11 – 1945-09-26), \emph{Schriftsteller}!Graf von Charolais. Ein Trauerspiel1904-12-23@\strich\emph{Der Graf von Charolais. Ein Trauerspiel} {[}1904-12-23{]}|pw}.\pend
           \pstart \spacefill\mbox{A.}\pend{}\pstart
           \noindent{}{\pb}gleichfalls hiſtorisches\pend
           \pstart
           \centering{}\textcolor{gray}{\textbf{Menu.}}{ }{[}hs. ?? [Schreibkraft der Menükarte 3.10.1899]:{]} du 3. Oct. 1899\pend
           \pstart
           \noindent{}\centering{}Consommé pâtés d’Italie\pend
           \pstart
           \noindent{}\centering{}\textcolor{gray}{Canape à la meuni}ère – Pommes\pend
           \pstart
           \noindent{}\centering{}Roastbeef garni\pend
           \pstart
           \noindent{}\centering{}Haricots verts – Hareng\pend
           \pstart
           \noindent{}\centering{}Chapon rôti – Comp. – Salade\pend
           \pstart
           \noindent{}\centering{}Bavarois à la romaine\pend
           \pstart
           \noindent{}\centering{}Fruits – Dessert.\pend
           \pstart
           \noindent{}\textcolor{gray}{\textbf{\textsc{Hotel du Parc et Bristol\oindex{Hôtel du Parc {\kaufmannsund} Bristol@\textbf{Hôtel du Parc {\kaufmannsund} Bristol}|pw}}}}\pend
           
         
         \endnumbering\mylabel{h}\end{ledgroupsized}  \newcommand{\dateiname}{L00987}\newcommand{\titel}{Arthur Schnitzler an Richard Beer-Hofmann, 3. 10. 1899}\newcommand{\editorInnen}{Martin Anton Müller und Gerd-Hermann Susen}%% latex-leseansicht-abspann.tex
%% Abspann für die Leseansicht.
%% Der Schalter \ifkorrekturansicht ist bereits durch den Vorspann gesetzt.

%% latex-abspann.tex
%% Gemeinsamer Abspann für Korrekturansicht und Leseansicht.
%% Setzt den Schalter \ifkorrekturansicht voraus (gesetzt in den
%% einbindenden Dateien latex-korrekturansicht-abspann.tex bzw.
%% latex-leseansicht-abspann.tex).
%% ---------------------------------------------------------------

\normalsize

% Das esempio-Environment wird nur in der Leseansicht benötigt
\ifkorrekturansicht\else
\newenvironment{esempio}[3]%
{
    \vspace{1.5ex}
    \rlap{\underline{#1}}
    \par
    \setlength{\parindent}{0cm}
    \nopagebreak
    \leftskip=#2cm
    \rightskip=#3cm
}
{
    \par
}
\fi

\doendnotes{C}
\bigskip
\vfill

\clearpage

\footnotesize

\ifkorrekturansicht
  \lohead{\textsc{register}}
\fi

% theindex-Environment neu definieren ohne reledmac
\makeatletter
\renewenvironment{theindex}{%
  \ifkorrekturansicht
    \section*{\indexname}%
  \else
    \subsubsection*{Index der erwähnten Entitäten}%
  \fi
  \setlength{\parindent}{0pt}%
  \setlength{\parskip}{0pt plus 0.3pt}%
  \let\item\@idxitem
}{%
  \ifkorrekturansicht\clearpage\fi
}
\makeatother

\IfFileExists{\jobname-pw.ind}{\input{\jobname-pw.ind}}{}

% Quellenangabe nur in der Leseansicht
\ifkorrekturansicht\else
% Fallback-Definitionen, falls die .tex-Datei \titel etc. nicht gesetzt hat
\providecommand{\titel}{}
\providecommand{\editorInnen}{}
\providecommand{\dateiname}{\jobname}

\vspace{3cm}

\vfill

\footnotesize
\textsc{Quelle}: \titel. Herausgegeben von {\editorInnen}. In: \emph{Arthur Schnitzler: Briefwechsel mit Autorinnen und Autoren}.
 Digitale Edition, https://schnitzler-briefe.acdh.oeaw.ac.at/{\dateiname}.html (Stand \today)
\fi

\end{document}


      