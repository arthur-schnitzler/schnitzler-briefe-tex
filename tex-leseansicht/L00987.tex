\input{../tex-inputs/latex-pdf-vorspann}
\begin{center}
            \textcolor{red}{ENTWURF. ENTZIFFERUNG NOCH NICHT KORREKTURGELESEN}
                      \end{center}
            
               \section[Arthur Schnitzler an Richard Beer-Hofmann, 3. 10. 1899]{ Arthur Schnitzler an Richard Beer-Hofmann, 3. 10. 1899}\nopagebreak\mylabel{v}\rehead{ }\begin{ledgroupsized}[t]{13cm}\normalsize\beginnumbering\briefempfaengerindex{Beer-Hofmann, Richard@\textsc{Beer-Hofmann, Richard}!zzzSchnitzler, Arthur@\emph{von Arthur Schnitzler}!1899-10-031@{3. 10. 1899}|(be} \toendnotes[C]{\smallbreak\pagebreak[2]} \Standort{CUL, Schnitzler, B 8.}
\physDesc{Klappkarte
\newline{}Handschrift Arthur Schnitzler: Bleistift, deutsche Kurrent\newline{}Handschrift  : blaue Tinte, lateinische Kurrent (\noindent{}Speisenfolge)\newline{}Versand: 1) Stempel: »\nobreak{}\oindex{Wiesbaden@\textbf{Wiesbaden}|pwk}Wiesbaden, 3. 10. 99, 3–4N\nobreak{}«.  2) Stempel: »\nobreak{}6. {[}10.{]} 99, \oindex{Sankt Michael@\textbf{Sankt Michael}|pwk}St. Michael Eppan\nobreak{}«. \newline{}Ordnung: mit Bleistift von unbekannter Hand datiert: »3. 10.« }\toendnotes[C]{\smallbreak}\pstart{}{\pb}\textsc{Dr. Richard Beer-Hofmann}\pend{}\pstart{}\textsc{St. Michael in Eppan}\oindex{Sankt Michael@\textbf{Sankt Michael}|pw}\pend{}{\bigskip}\pstart
           \noindent{}\centering{}\textcolor{gray}{\textbf{{\pb}Wiesbaden\oindex{Wiesbaden@\textbf{Wiesbaden}|pw}. Blick aus dem Hotel du Parc et Bristol\oindex{Hôtel du Parc {\kaufmannsund} Bristol@\textbf{Hôtel du Parc {\kaufmannsund} Bristol}|pw}}}\pend
           \pstart
           {\pb}Heute Abd fahr ich nach Berlin\oindex{Berlin@\textbf{Berlin}|pw}. – Will mein Stück\pwindex{Schnitzler, Arthur 15.05.1862 – 21.10.1931@\textsc{Schnitzler, Arthur} (15.05.1862 – 21.10.1931), \emph{Schriftsteller, Mediziner}!Schleier der Beatrice. Schauspiel in fuenf Akten1900-12-01 – 1900-12-01@\strich\emph{Der Schleier der Beatrice. Schauspiel in fünf Akten} {[}1900-12-01 – 1900-12-01{]}|pwv} nochmals umarbeiten. – Bleibe in Berlin\oindex{Berlin@\textbf{Berlin}|pw} wahrſcheinlich bis Sonntag. Wohne dort \textsc{Hotel Savoy}\oindex{Hotel Savoy@\textbf{Hotel Savoy}|pw}. Viele herzl Grüße. Ich freue mich über Ihre 420 Verſe\pwindex{Beer-Hofmann, Richard 11.07.1866 – 26.09.1945@\textsc{Beer-Hofmann, Richard} (11.07.1866 – 26.09.1945), \emph{Schriftsteller}!Graf von Charolais. Ein Trauerspiel1904-12-23 – 1904-12-23@\strich\emph{Der Graf von Charolais. Ein Trauerspiel} {[}1904-12-23 – 1904-12-23{]}|pw}.\pend
           \pstart \spacefill\mbox{A.}\pend{}\pstart
           \noindent{}{\pb}gleichfalls hiſtorisches\pend
           \pstart
           \centering{}\textcolor{gray}{\textbf{Menu.}}{ }{[}hs. ?? [Schreibkraft der Menükarte 3.10.1899]:{]} du 3. Oct. 1899\pend
           \pstart
           \noindent{}\centering{}Consommé pâtés d’Italie\pend
           \pstart
           \noindent{}\centering{}\textcolor{gray}{Canape à la meuni}ère – Pommes\pend
           \pstart
           \noindent{}\centering{}Roastbeef garni\pend
           \pstart
           \noindent{}\centering{}Haricots verts – Hareng\pend
           \pstart
           \noindent{}\centering{}Chapon rôti – Comp. – Salade\pend
           \pstart
           \noindent{}\centering{}Bavarois à la romaine\pend
           \pstart
           \noindent{}\centering{}Fruits – Dessert.\pend
           \pstart
           \noindent{}\textcolor{gray}{\textbf{\textsc{Hotel du Parc et Bristol\oindex{Hôtel du Parc {\kaufmannsund} Bristol@\textbf{Hôtel du Parc {\kaufmannsund} Bristol}|pw}}}}\pend
           \endnumbering\briefempfaengerindex{Beer-Hofmann, Richard@\textsc{Beer-Hofmann, Richard}!zzzSchnitzler, Arthur@\emph{von Arthur Schnitzler}!1899-10-031@{3. 10. 1899}|)be}\mylabel{h}\end{ledgroupsized}  \newcommand{\dateiname}{L00987}\newcommand{\titel}{Arthur Schnitzler an Richard Beer-Hofmann, 3. 10. 1899}\newcommand{\editorInnen}{Martin Anton Müller und Gerd-Hermann Susen}\input{../tex-inputs/latex-pdf-abspann}
      