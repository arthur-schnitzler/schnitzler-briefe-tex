%% latex-korrekturansicht-vorspann.tex
%% Vorspann für die Korrekturansicht.
%% Lädt die gemeinsame Datei latex-vorspann.tex mit gesetztem Schalter.

\newif\ifkorrekturansicht
\korrekturansichttrue

\input{../tex-inputs/latex-vorspann}


\section[Arthur Schnitzler an Richard Beer-Hofmann, 3. 10. 1899]{L00987 Arthur Schnitzler an Richard Beer-Hofmann, 3. 10. 1899}
\nopagebreak\mylabel{L00987v}
\rehead{ }\normalsize\beginnumbering\briefempfaengerindex{Beer-Hofmann, Richard@\textsc{Beer-Hofmann, Richard}!zzzSchnitzler, Arthur@\emph{von Arthur Schnitzler}!1899-10-031@{3. 10. 1899}|(be}
\toendnotes[C]{\smallbreak\pagebreak[2]}\Standort{YCGL, MSS 31.}
\physDesc{Karte, 431 Zeichen (Klappkarte)
\newline{}Handschrift Arthur Schnitzler: 1) Bleistift, deutsche Kurrent\hspace{1em}2) Bleistift, lateinische Kurrent (\noindent{}Adresse)\hspace{1em}
\newline{}Handschrift Schreibkraft: blaue Tinte, lateinische Kurrent (\noindent{}Speisenfolge)
\newline{}Versand: 1) Stempel: »\nobreak{}\oindex{Wiesbaden@\textbf{Wiesbaden}, \emph{P.PPLA}|pwk}Wiesbaden, 3. 10. 99, 3–4N\nobreak{}«.   2) Stempel: »\nobreak{}6. {[}10.{]} 99, \oindex{Sankt Michael@\textbf{Sankt Michael}, \emph{Bezirk (A.BZK)}|pwk}St. Michael Eppan\nobreak{}«. 
\newline{}Ordnung: mit Bleistift von unbekannter Hand datiert: »3. 10.« }\toendnotes[C]{\smallbreak}\pstart{}{\pb}Dr. Richard Beer-Hofmann\pend{}\pstart{}St. Michael in Eppan\oindex{Sankt Michael@\textbf{Sankt Michael}, \emph{Bezirk (A.BZK)}|pw}\pend{}{\bigskip}\vspace{1em}
\pstart
           \centering{}\textcolor{gray}{\textbf{{\pb}Wiesbaden\oindex{Wiesbaden@\textbf{Wiesbaden}, \emph{P.PPLA}|pw}. Blick aus dem Hotel du Parc et Bristol\oindex{Hôtel du Parc {\kaufmannsund} Bristol@\textbf{Hôtel du Parc {\kaufmannsund} Bristol}, \emph{Hotel (K.HTL)}|pw}}}\pend
           \vspace{0.5em}
\pstart
           {\pb}Heute Abd fahr ich nach Berlin\oindex{Berlin@\textbf{Berlin}, \emph{P.PPLC}|pw}. – Will mein Stück\pwindex{Schleier der Beatrice. Schauspiel in fuenf Akten@\emph{Der Schleier der Beatrice. Schauspiel in fünf Akten}|pwv} nochmals umarbeiten. – Bleibe in Berlin\oindex{Berlin@\textbf{Berlin}, \emph{P.PPLC}|pw} wahrſcheinlich bis Sonntag. Wohne dort \textsc{Hotel Savoy}\oindex{Hotel Savoy [Berlin]@\textbf{Hotel Savoy [Berlin]}, \emph{Hotel (K.HTL)}|pw}. Viele herzl Grüße. Ich freue mich über Ihre 420 Verſe\pwindex{Graf von Charolais. Ein Trauerspiel@\emph{Der Graf von Charolais. Ein Trauerspiel}|pw}.\pend
           \pstart \spacefill\mbox{A.}\pend{}\selectlanguage{ngerman}\vspace{1em}
\pstart
           \noindent{}{\pb}gleichfalls hiſtorisches\pend
           
\pstart
           \centering{}\textcolor{gray}{\textbf{Menu.}}{ }{[}hs. :{]} du 3. Oct. 1899\pend
           
\pstart
           \centering{}Consommé pâtés d’Italie\pend
           
\pstart
           \centering{}\textcolor{gray}{Canape à la meuni}ère – Pommes\pend
           
\pstart
           \centering{}Roastbeef garni\pend
           
\pstart
           \centering{}Haricots verts – Hareng\pend
           
\pstart
           \centering{}Chapon rôti – Comp. – Salade\pend
           
\pstart
           \centering{}Bavarois à la romaine\pend
           
\pstart
           \centering{}Fruits – Dessert.\pend
           
\pstart
           \textcolor{gray}{\textbf{\textsc{Hotel du Parc et Bristol\oindex{Hôtel du Parc {\kaufmannsund} Bristol@\textbf{Hôtel du Parc {\kaufmannsund} Bristol}, \emph{Hotel (K.HTL)}|pw}}}}\pend
           \selectlanguage{ngerman}\endnumbering\briefempfaengerindex{Beer-Hofmann, Richard@\textsc{Beer-Hofmann, Richard}!zzzSchnitzler, Arthur@\emph{von Arthur Schnitzler}!1899-10-031@{3. 10. 1899}|)be}\mylabel{L00987h}  \normalsize

\doendnotes{C}
\bigskip
\vfill

\clearpage

\footnotesize

\lohead{\textsc{register}}

% Definiere theindex-Environment komplett neu ohne reledmac
\makeatletter
\renewenvironment{theindex}{%
  \section*{\indexname}%
  \setlength{\parindent}{0pt}%
  \setlength{\parskip}{0pt plus 0.3pt}%
  \let\item\@idxitem
}{%
  \clearpage
}
\makeatother

\IfFileExists{\jobname-pw.ind}{\input{\jobname-pw.ind}}{}

\end{document}

      