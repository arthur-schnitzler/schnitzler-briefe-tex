%% latex-leseansicht-vorspann.tex
%% Vorspann für die Leseansicht.
%% Lädt die gemeinsame Datei latex-vorspann.tex mit nicht gesetztem Schalter.

\newif\ifkorrekturansicht
\korrekturansichtfalse

\input{../tex-inputs/latex-vorspann}


\section[Arthur Schnitzler: Widmungsexemplar Reigen für Hermann Bahr, [2.?] 4. 1903]{L01284 Arthur Schnitzler: Widmungsexemplar Reigen für Hermann Bahr, [2.?] 4. 1903}
\nopagebreak\mylabel{L01284v}
\rehead{ }\normalsize\beginnumbering\briefempfaengerindex{Bahr, Hermann@\textsc{Bahr, Hermann}!zzzSchnitzler, Arthur@\emph{von Arthur Schnitzler}!1903-04-021@{[2.?] 4. 1903}|(be}
\toendnotes[C]{\smallbreak\pagebreak[2]}
\correspDesc{Versand  durch Arthur Schnitzler am [2.?] 4. 1903 in Wien
\newline{}Erhalt  durch Hermann Bahr im Zeitraum [2. 4. 1903
                  – 6. 4. 1903?] \textbf{Ort fehlend} }\toendnotes[C]{\smallbreak}
\Standort{Wien, Antiquariat Georg Fritsch, Katalog, September 2015.}
\physDesc{Widmung am Schmutztitel, 45 Zeichen
\newline{}Handschrift: schwarze Tinte, deutsche Kurrent
\newline{}Zusatz: Als Empfänger ist Bahr anzunehmen, da die Verwendung des
                                 Vornamens mit den Formulierungen weiterer Widmungen übereinstimmt;
                                 Provenienz und Verbleib ungeklärt }
\buchAbdrucke{\weitereDrucke{Hermann Bahr, Arthur Schnitzler: \emph{Briefwechsel, Aufzeichnungen, Dokumente (1891–1931)}. Herausgegeben von Kurt Ifkovits und Martin Anton Müller. Göttingen: \emph{Wallstein} 2018, S. 258.} }
\pstart
           \noindent{}{\pb}\textsc{Meinem lieben Hermann}\pend
           
\pstart
           Wien\oindex{Wien@\textbf{Wien}, \emph{Verwaltungsgebiet}|pw}{ }April 903.\pend
           \pstart \spacefill\mbox{Arthur}\pend{}{\vspace{1\baselineskip}}
\pstart
           \centering{}\textcolor{gray}{\textbf{REIGEN\pwindex{Schnitzler, Arthur 15.\,5.\,1862 Wien – 21.\,10.\,1931 ebd.@\textsc{Schnitzler, Arthur} (15.\,5.\,1862 Wien – 21.\,10.\,1931 ebd.), \emph{Schriftsteller, Mediziner}!Reigen. Zehn Dialoge@\strich\emph{Reigen. Zehn Dialoge}|pw}}}\pend
           \selectlanguage{ngerman}\vspace{1em}{\vspace{1\baselineskip}}
\pstart
           \centering{}{\pb}\textcolor{gray}{\textbf{ARTHUR}}\pend
           
\pstart
           \centering{}\textcolor{gray}{\textbf{SCHNITZLER}}\pend
           
\pstart
           \centering{}\textcolor{gray}{\textbf{REIGEN\pwindex{Schnitzler, Arthur 15.\,5.\,1862 Wien – 21.\,10.\,1931 ebd.@\textsc{Schnitzler, Arthur} (15.\,5.\,1862 Wien – 21.\,10.\,1931 ebd.), \emph{Schriftsteller, Mediziner}!Reigen. Zehn Dialoge@\strich\emph{Reigen. Zehn Dialoge}|pw}}}\pend
           
\pstart
           \centering{}\textcolor{gray}{\textbf{ZEHN DIALOGE}}\pend
           
\pstart
           \centering{}\textcolor{gray}{\textbf{GESCHRIEBEN Winter 1896–97}}\pend
           
\pstart
           \centering{}\textcolor{gray}{\textbf{BUCHSCHMUCK VON BERTHOLD
                        LÖFFLER\pwindex{Löffler, Bertold 28.\,9.\,1874 Liberec – 23.\,3.\,1960 Wien@\textsc{Löffler, Bertold} (28.\,9.\,1874 Liberec – 23.\,3.\,1960 Wien), \emph{Maler, Grafiker}|pw}}}\pend
           {\vspace{1\baselineskip}}
\pstart
           \centering{}\textcolor{gray}{\textbf{WIENER VERLAG\orgindex{Wiener Verlag@Wiener Verlag|pw}}}\pend
           
\pstart
           \centering{}\textcolor{gray}{\textbf{WIEN\oindex{Wien@\textbf{Wien}, \emph{Verwaltungsgebiet}|pw} UND LEIPZIG\oindex{Leipzig@\textbf{Leipzig}, \emph{Hauptstadt}|pw}}}\pend
           
\pstart
           \centering{}\textcolor{gray}{\textbf{1903}}\pend
           \selectlanguage{ngerman}\endnumbering\briefempfaengerindex{Bahr, Hermann@\textsc{Bahr, Hermann}!zzzSchnitzler, Arthur@\emph{von Arthur Schnitzler}!1903-04-021@{[2.?] 4. 1903}|)be}\mylabel{L01284h}  \newcommand{\dateiname}{L01284}\newcommand{\titel}{Arthur Schnitzler: Widmungsexemplar Reigen für Hermann Bahr, [2.?] 4. 1903}\newcommand{\editorInnen}{Martin Anton Müller und Gerd-Hermann Susen}%% latex-leseansicht-abspann.tex
%% Abspann für die Leseansicht.
%% Der Schalter \ifkorrekturansicht ist bereits durch den Vorspann gesetzt.

%% latex-abspann.tex
%% Gemeinsamer Abspann für Korrekturansicht und Leseansicht.
%% Setzt den Schalter \ifkorrekturansicht voraus (gesetzt in den
%% einbindenden Dateien latex-korrekturansicht-abspann.tex bzw.
%% latex-leseansicht-abspann.tex).
%% ---------------------------------------------------------------

\normalsize

% Das esempio-Environment wird nur in der Leseansicht benötigt
\ifkorrekturansicht\else
\newenvironment{esempio}[3]%
{
    \vspace{1.5ex}
    \rlap{\underline{#1}}
    \par
    \setlength{\parindent}{0cm}
    \nopagebreak
    \leftskip=#2cm
    \rightskip=#3cm
}
{
    \par
}
\fi

\doendnotes{C}
\bigskip
\vfill

\clearpage

\footnotesize

\ifkorrekturansicht
  \lohead{\textsc{register}}
\fi

% theindex-Environment neu definieren ohne reledmac
\makeatletter
\renewenvironment{theindex}{%
  \ifkorrekturansicht
    \section*{\indexname}%
  \else
    \subsubsection*{Index der erwähnten Entitäten}%
  \fi
  \setlength{\parindent}{0pt}%
  \setlength{\parskip}{0pt plus 0.3pt}%
  \let\item\@idxitem
}{%
  \ifkorrekturansicht\clearpage\fi
}
\makeatother

\IfFileExists{\jobname-pw.ind}{\input{\jobname-pw.ind}}{}

% Quellenangabe nur in der Leseansicht
\ifkorrekturansicht\else
% Fallback-Definitionen, falls die .tex-Datei \titel etc. nicht gesetzt hat
\providecommand{\titel}{}
\providecommand{\editorInnen}{}
\providecommand{\dateiname}{\jobname}

\vspace{3cm}

\vfill

\footnotesize
\textsc{Quelle}: \titel. Herausgegeben von {\editorInnen}. In: \emph{Arthur Schnitzler: Briefwechsel mit Autorinnen und Autoren}.
 Digitale Edition, https://schnitzler-briefe.acdh.oeaw.ac.at/{\dateiname}.html (Stand \today)
\fi

\end{document}


