%% latex-korrekturansicht-vorspann.tex
%% Vorspann für die Korrekturansicht.
%% Lädt die gemeinsame Datei latex-vorspann.tex mit gesetztem Schalter.

\newif\ifkorrekturansicht
\korrekturansichttrue

\input{../tex-inputs/latex-vorspann}


\section[Arthur Schnitzler: Widmungsexemplar Reigen für Hermann Bahr, {[}2.?{]} 4. 1903]{L01284 Arthur Schnitzler: Widmungsexemplar Reigen für Hermann Bahr,
               {[}2.?{]} 4. 1903}
\nopagebreak\mylabel{L01284v}
\rehead{ }\normalsize\beginnumbering\briefempfaengerindex{Bahr, Hermann@\textsc{Bahr, Hermann}!zzzSchnitzler, Arthur@\emph{von Arthur Schnitzler}!1903-04-021@{{[}2.?{]} 4. 1903}|(be}
\toendnotes[C]{\smallbreak\pagebreak[2]}\Standort{Wien, Antiquariat Georg Fritsch, Katalog, September 2015.}
\physDesc{Widmung am Schmutztitel, 45 Zeichen
\newline{}Handschrift: schwarze Tinte, deutsche Kurrent
\newline{}Zusatz: Als Empfänger ist Bahr anzunehmen, da die Verwendung des
                                 Vornamens mit den Formulierungen weiterer Widmungen übereinstimmt;
                                 Provenienz und Verbleib ungeklärt }
\buchAbdrucke{\weitereDrucke{Hermann Bahr, Arthur Schnitzler: \emph{Briefwechsel, Aufzeichnungen, Dokumente (1891–1931)}. Göttingen: \emph{Wallstein} 2018, S. 258.} }
\pstart
           \noindent{}{\pb}\textsc{Meinem lieben Hermann}\pend
           
\pstart
           Wien\oindex{Wien@\textbf{Wien}, \emph{A.ADM2}|pw}{ }April 903. \pend
           \pstart \spacefill\mbox{Arthur}\pend{}{\vspace{1\baselineskip}}
\pstart
           \centering{}\textcolor{gray}{\textbf{REIGEN\pwindex{Reigen. Zehn Dialoge@\emph{Reigen. Zehn Dialoge}|pw}}}\pend
           \selectlanguage{ngerman}\vspace{1em}{\vspace{1\baselineskip}}
\pstart
           \centering{}{\pb}\textcolor{gray}{\textbf{ARTHUR}}\pend
           
\pstart
           \centering{}\textcolor{gray}{\textbf{SCHNITZLER}}\pend
           
\pstart
           \centering{}\textcolor{gray}{\textbf{REIGEN\pwindex{Reigen. Zehn Dialoge@\emph{Reigen. Zehn Dialoge}|pw}}}\pend
           
\pstart
           \centering{}\textcolor{gray}{\textbf{ZEHN DIALOGE}}\pend
           
\pstart
           \centering{}\textcolor{gray}{\textbf{GESCHRIEBEN Winter 1896–97}}\pend
           
\pstart
           \centering{}\textcolor{gray}{\textbf{BUCHSCHMUCK VON BERTHOLD
                        LÖFFLER\pwindex{Loeffler, Bertold 28.09.1874 – 23.03.1960@\textsc{Löffler, Bertold} (28.09.1874 – 23.03.1960), \emph{Maler/Malerin, Grafiker/Grafikerin}|pw}}}\pend
           {\vspace{1\baselineskip}}
\pstart
           \centering{}\textcolor{gray}{\textbf{WIENER VERLAG\orgindex{Wiener Verlag@Wiener Verlag|pw}}}\pend
           
\pstart
           \centering{}\textcolor{gray}{\textbf{WIEN\oindex{Wien@\textbf{Wien}, \emph{A.ADM2}|pw} UND LEIPZIG\oindex{Leipzig@\textbf{Leipzig}, \emph{P.PPLA3}|pw}}}\pend
           
\pstart
           \centering{}\textcolor{gray}{\textbf{1903}}\pend
           \selectlanguage{ngerman}\endnumbering\briefempfaengerindex{Bahr, Hermann@\textsc{Bahr, Hermann}!zzzSchnitzler, Arthur@\emph{von Arthur Schnitzler}!1903-04-021@{{[}2.?{]} 4. 1903}|)be}\mylabel{L01284h}  \normalsize

\doendnotes{C}
\bigskip
\vfill

\clearpage

\footnotesize

\lohead{\textsc{register}}

% Definiere theindex-Environment komplett neu ohne reledmac
\makeatletter
\renewenvironment{theindex}{%
  \section*{\indexname}%
  \setlength{\parindent}{0pt}%
  \setlength{\parskip}{0pt plus 0.3pt}%
  \let\item\@idxitem
}{%
  \clearpage
}
\makeatother

\IfFileExists{\jobname-pw.ind}{\input{\jobname-pw.ind}}{}

\end{document}

      