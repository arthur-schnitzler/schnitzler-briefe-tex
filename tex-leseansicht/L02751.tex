%% latex-leseansicht-vorspann.tex
%% Vorspann für die Leseansicht.
%% Lädt die gemeinsame Datei latex-vorspann.tex mit nicht gesetztem Schalter.

\newif\ifkorrekturansicht
\korrekturansichtfalse

\input{../tex-inputs/latex-vorspann}


         
         \renewcommand{\erwaehntePersonen}{Personen: Georges Aubry, [MMe. Georges] Aubry, Émile Augier, Hermann Bahr, Richard Beer-Hofmann, Jakob Julius David, Hippolyte Fierens-Gevaert, Paul Goldmann, Clementine Goldmann, Emil Granichstaedten, Max Kalbeck, Julius Konried, Fedor Mamroth, Fritz Mauthner, Josef Rosengart, Adele Sandrock, Louise Schnitzler, Paul von Schönthan-Pernwald, Leopold Sonnemann, Adolf von Sonnenthal, Ludwig Speidel, Friedrich Uhl, Theodor Wolff}
         \renewcommand{\erwaehnteInstitutionen}{Institutionen: Berliner Tageblatt, Burgtheater, Frankfurter Zeitung, Journal des débats, La Liberté}
         \renewcommand{\erwaehnteOrte}{Orte: Berlin, Deutschland, Paris, Rue Des Prêtres Saint-Germain L'Auxerrois, Rue du Général Camou, Wien, rue Feydeau}
         \renewcommand{\erwaehnteWerke}{Werke: Berliner Lokal-Anzeiger, Berliner Tageblatt, Burgtheater. (»Liebelei«, Schauspiel in drei Aufzügen von Arthur Schnitzler. – »Rechte der Seele«, Schauspiel in einem Act von Giuseppe Giacosa, deutsch von Otto Eisenschitz.), Burgtheater. Zwei Schauspiele: »Rechte der Seele« von Giuseppe Giacosa. – »Liebelei« von Arthur Schnitzler, Burgtheater. »Liebelei«, Schauspiel in drei Acten von Arthur Schnitzler. – »Rechte der Seele«, Schauspiel in einem Acte von Guiseppe Giacosa; deutsch von Otto Eisenschitz, Courrier des Théatres [Liebelei], Das Kleine Journal, Der zerbrochene Krug im Deutschen Theater, Deutsches Theater, Die Presse, Die kleine Komödie, Frankfurter Zeitung, Journal des débats. Politiques et littéraires, La Liberté, La petite comédie. Mœurs viennois, Liebelei. Schauspiel in drei Akten, Neue Freie Presse, Neues Wiener Journal, Neues Wiener Tagblatt, Theater und Kunst. (Burgtheater.) [Liebelei, Rechte der Seele], Theater, Kunst und Literatur. (Burgtheater.) [Liebelei], Theater, Kunst und Literatur. Burgtheater [Liebelei, Rechte der Seele], Theater- und Kunstnachrichten. [Burgtheater] [Liebelei, Rechte der Seele], Thêatres. [Notre correspondant de Vienne], Wiener Brief. [Liebelei, Rechte der Seele], Wiener Tagblatt, [Aus Wien, 9. Oktober] [Liebelei], [Wien, 9. Oktober] [Liebelei], »Liebelei« [Telegramm zur Uraufführung]}
               \section[Paul Goldmann an Arthur Schnitzler, 13. 10. {[}1895{]}]{ Paul Goldmann an Arthur Schnitzler, 13. 10. {[}1895{]}}\nopagebreak\mylabel{v}\rehead{ }\begin{ledgroupsized}[t]{13cm}\normalsize\beginnumbering\briefempfaengerindex{Schnitzler, Arthur@\textsc{Schnitzler, Arthur}!zzzGoldmann, Paul@\emph{von Paul Goldmann}!1895-10-131@{13. 10. {[}1895{]}}|(be} \toendnotes[C]{\smallbreak\pagebreak[2]} \Standort{DLA, A:Schnitzler, HS.NZ85.1.3165.}
\physDesc{Brief, 3 Blätter, 11 Seiten, 4372 Zeichen
\newline{}Handschrift: blaue Tinte, deutsche Kurrent
\newline{}Schnitzler: 1) mit Bleistift das Jahr »95« vermerkt  2) mit rotem Buntstift eine seitliche Markierung und sieben Unterstreichungen}\toendnotes[C]{\smallbreak}\pstart
           \noindent{}{\pb}\textcolor{gray}{\textbf{\textbf{Frankfurter Zeitung\orgindex{Frankfurter Zeitung@Frankfurter Zeitung|pw}}}}\pend
           \pstart
           \textcolor{gray}{\textbf{(\begin{otherlanguage}{french}Gazette de Francfort\end{otherlanguage}\orgindex{Frankfurter Zeitung@Frankfurter Zeitung|pw}). }}\pend
           \pstart
           \textcolor{gray}{\textbf{\textbf{\begin{otherlanguage}{french}Fondateur M. L.
                              Sonnemann\pwindex{Sonnemann, Leopold 1831-10-29 – 1909-10-30@\textsc{Sonnemann, Leopold} (1831-10-29 – 1909-10-30), \emph{Journalist, Herausgeber}|pw}\end{otherlanguage}.}}}\pend
           \pstart
           \begin{otherlanguage}{french}\textcolor{gray}{\textbf{Journal politique, financier,}}\end{otherlanguage}\pend
           \pstart
           \begin{otherlanguage}{french}\textcolor{gray}{\textbf{commercial et littéraire.}}\end{otherlanguage}\hfill \textsc{Paris\oindex{Paris@\textbf{Paris}|pw}}, 13. October.\pend
           \pstart
           \begin{otherlanguage}{french}\textcolor{gray}{\textbf{\textbf{Paraissant trois fois par jour.}}}\end{otherlanguage}\pend
           \pstart
           \begin{otherlanguage}{french}\textcolor{gray}{\textbf{\textbf{Bureau à Paris\oindex{Paris@\textbf{Paris}|pw}}}}\end{otherlanguage}\pend
           \pstart
           \begin{otherlanguage}{french}\textcolor{gray}{\textbf{\textbf{24. Rue Feydeau\oindex{rue Feydeau@\textbf{rue Feydeau}|pw}.}}}\end{otherlanguage}\pend
           \pstart\center{}Mein lieber Freund,\pend\pstart
           Nochmals innigen Glückwunſch!\pend
           \pstart
           Jetzt, nachdem ich einige Referate geleſen, ſehe ich \strikeout{\textcolor{gray}{er}} erſt, wie groß Dein Erfolg iſt, was aus Deiner Depeſche nicht klar genug
               hervorging. Wie ich die Sache anſehe, biſt Du jetzt \label{K_L02751-1v}\edtext{lancirt}{\lemma{\textnormal{\emph{lancirt}}}\Cendnote{\textnormal{im Sinne
                  von: in der Öffentlichkeit bekannt}}}\label{K_L02751-1h}. Nach dem Wien\oindex{Wien@\textbf{Wien}|pw}er Erfolge werden die Berlin\oindex{Berlin@\textbf{Berlin}|pw}er bald
               mit dem Stücke\pwindex{Schnitzler, Arthur 15.05.1862 – 21.10.1931@\textsc{Schnitzler, Arthur} (15.05.1862 – 21.10.1931), \emph{Schriftsteller, Mediziner}!Liebelei. Schauspiel in drei Akten1895-10-09@\strich\emph{Liebelei. Schauspiel in drei Akten} {[}1895-10-09{]}|pwv} herauskommen.
               Dort wird es einen nicht minder großen Erfolg haben und eine noch intelligentere
               Kritik finden (\label{K_L02751-2v}\edtext{\textsc{Mauthner\pwindex{Mauthner, Fritz 1849-11-20 – 1923-06-29@\textsc{Mauthner, Fritz} (1849-11-20 – 1923-06-29), \emph{Schriftsteller, Journalist, Philosoph}|pw}} im »Tageblatt\orgindex{Berliner Tageblatt@Berliner Tageblatt|pw}«}{\lemma{\textnormal{\emph{Mauthner im »Tageblatt«}}}\Cendnote{\textnormal{Mauthner\pwindex{Mauthner, Fritz 1849-11-20 – 1923-06-29@\textsc{Mauthner, Fritz} (1849-11-20 – 1923-06-29), \emph{Schriftsteller, Journalist, Philosoph}|pwk} sollte dann tatsächlich schreiben: Fr. M.\pwindex{Mauthner, Fritz 1849-11-20 – 1923-06-29@\textsc{Mauthner, Fritz} (1849-11-20 – 1923-06-29), \emph{Schriftsteller, Journalist, Philosoph}|pwkv} [ = Fritz Mauthner\pwindex{Mauthner, Fritz 1849-11-20 – 1923-06-29@\textsc{Mauthner, Fritz} (1849-11-20 – 1923-06-29), \emph{Schriftsteller, Journalist, Philosoph}|pwk}]: \emph{Deutsches Theater}\pwindex{Mauthner, Fritz 1849-11-20 – 1923-06-29@\textsc{Mauthner, Fritz} (1849-11-20 – 1923-06-29), \emph{Schriftsteller, Journalist, Philosoph}!Deutsches Theater1896-02-05@\strich\emph{Deutsches Theater} {[}1896-02-05{]}|pwk}. In: \emph{Berliner Tageblatt}\pwindex{?? Werk@Nicht ermittelte Verfasserinnen und Verfasser!Berliner Tageblatt1872 – 1939@\emph{Berliner Tageblatt} {[}1872 – 1939{]}|pwk}, Jg. 25, Nr. 64, 5. 2. 1896, Morgen-Ausgabe, S. 2–3; Fritz Mauthner\pwindex{Mauthner, Fritz 1849-11-20 – 1923-06-29@\textsc{Mauthner, Fritz} (1849-11-20 – 1923-06-29), \emph{Schriftsteller, Journalist, Philosoph}|pwk}: \emph{Der zerbrochene Krug im Deutschen Theater}\pwindex{Mauthner, Fritz 1849-11-20 – 1923-06-29@\textsc{Mauthner, Fritz} (1849-11-20 – 1923-06-29), \emph{Schriftsteller, Journalist, Philosoph}!zerbrochene Krug im Deutschen Theater1896-02-05@\strich\emph{Der zerbrochene Krug im Deutschen Theater} {[}1896-02-05{]}|pwk}. In: \emph{Berliner Tageblatt}\pwindex{?? Werk@Nicht ermittelte Verfasserinnen und Verfasser!Berliner Tageblatt1872 – 1939@\emph{Berliner Tageblatt} {[}1872 – 1939{]}|pwk}, Jg. 25, Nr. 65, 5. 2. 1896, Abend-Ausgabe, S. 1–2.}}}\label{K_L02751-2h}).
               Dann wird {\pb}es über alle deutſch\oindex{Deutschland@\textbf{Deutschland}|pwv}en Bühnen gehen. Wenn Du ruhig ſo
               weiter arbeiteſt – und ich weiß, Du wirſt es thun – kann am Ende ein deutſcher \textsc{Emile Augier\pwindex{Augier, Emile 1820-09-17 – 1889-10-25@\textsc{Augier, Émile} (1820-09-17 – 1889-10-25), \emph{Schriftsteller}|pw}} daraus werden. Der erſte entſcheidende Schritt auf dieſem Wege iſt gethan, und
               ich bin recht glücklich darüber, daß Dich gleich zu Anfang der Erfolg \strikeout{\textcolor{gray}{in die Hand}} an der Hand nimmt; das iſt ein guter Führer. Wenn ich übrigens »\textsc{Émile Augier\pwindex{Augier, Emile 1820-09-17 – 1889-10-25@\textsc{Augier, Émile} (1820-09-17 – 1889-10-25), \emph{Schriftsteller}|pw}}« ſage, ſo gilt dies nur einſtweilen, und ich behalte mir vor, im Laufe der
               Zeit, je nachdem die Dinge ſich entwickeln, {\pb}noch
               viel unbeſcheidener zu werden. Immerhin bedenke nur: In ſo jungen Jahren am erſten
               deutſchen Theater\orgindex{Burgtheater@Burgtheater|pwv} mit dem
               zweiten Stücke\pwindex{Schnitzler, Arthur 15.05.1862 – 21.10.1931@\textsc{Schnitzler, Arthur} (15.05.1862 – 21.10.1931), \emph{Schriftsteller, Mediziner}!Liebelei. Schauspiel in drei Akten1895-10-09@\strich\emph{Liebelei. Schauspiel in drei Akten} {[}1895-10-09{]}|pwv} ein von allen
                  \strikeout{ernſtz\textcolor{gray}{u}} ernſtzunehmenden Leuten laut anerkannter Erfolg! Das iſt etwas, was Du in der
               deutſchen Bühnengeſchichte ſelten finden dürfteſt. Es ſcheint wirklich, daß Du zu
               ſchönen Hoffnungen für die Zukunft berechtigſt, wie \label{K_L02751-3v}\edtext{einer\pwindex{Kalbeck, Max 1850-01-04 – 1921-05-04@\textsc{Kalbeck, Max} (1850-01-04 – 1921-05-04), \emph{Journalist}|pwv} der weiſen Männer}{\lemma{\textnormal{\emph{einer der weiſen Männer}}}\Cendnote{\textnormal{Von Max
                     Kalbeck\pwindex{Kalbeck, Max 1850-01-04 – 1921-05-04@\textsc{Kalbeck, Max} (1850-01-04 – 1921-05-04), \emph{Journalist}|pwk} erschien ein Feuilleton\pwindex{Kalbeck, Max 1850-01-04 – 1921-05-04@\textsc{Kalbeck, Max} (1850-01-04 – 1921-05-04), \emph{Journalist}!Burgtheater. »Liebelei«, Schauspiel in drei Acten von Arthur Schnitzler. –
                  »Rechte der Seele«, Schauspiel in einem Acte von Guiseppe Giacosa; deutsch von
                  Otto Eisenschitz1895-10-11@\strich\emph{Burgtheater. »Liebelei«, Schauspiel in drei Acten von Arthur Schnitzler. – »Rechte der Seele«, Schauspiel in einem Acte von Guiseppe Giacosa; deutsch von Otto Eisenschitz} {[}1895-10-11{]}|pwkv} und eine Nachtkritik\pwindex{Theater, Kunst und Literatur. Burgtheater [Liebelei, Rechte der Seele]1895-10-10@\emph{Theater, Kunst und Literatur. Burgtheater [Liebelei, Rechte der Seele]} {[}1895-10-10{]}|pwkv}, wobei sich die erwähnte Aussage in der Nachtkritik\pwindex{Theater, Kunst und Literatur. Burgtheater [Liebelei, Rechte der Seele]1895-10-10@\emph{Theater, Kunst und Literatur. Burgtheater [Liebelei, Rechte der Seele]} {[}1895-10-10{]}|pwkv} findet. Max Kalbeck\pwindex{Kalbeck, Max 1850-01-04 – 1921-05-04@\textsc{Kalbeck, Max} (1850-01-04 – 1921-05-04), \emph{Journalist}|pwk}: \emph{Burgtheater. »Liebelei«, Schauspiel in drei Acten von Arthur
                        Schnitzler. – »Rechte der Seele«, Schauspiel in einem Acte von Guiseppe
                        Giacosa; deutsch von Otto Eisenschitz}\pwindex{Kalbeck, Max 1850-01-04 – 1921-05-04@\textsc{Kalbeck, Max} (1850-01-04 – 1921-05-04), \emph{Journalist}!Burgtheater. »Liebelei«, Schauspiel in drei Acten von Arthur Schnitzler. –
                  »Rechte der Seele«, Schauspiel in einem Acte von Guiseppe Giacosa; deutsch von
                  Otto Eisenschitz1895-10-11@\strich\emph{Burgtheater. »Liebelei«, Schauspiel in drei Acten von Arthur Schnitzler. – »Rechte der Seele«, Schauspiel in einem Acte von Guiseppe Giacosa; deutsch von Otto Eisenschitz} {[}1895-10-11{]}|pwk}. In: \emph{Neues Wiener Tagblatt}\pwindex{?? Werk@Nicht ermittelte Verfasserinnen und Verfasser!Neues Wiener Tagblatt1867 – 1945@\emph{Neues Wiener Tagblatt} {[}1867 – 1945{]}|pwk}, Jg. 29, Nr. 279, 11. 10. 1895, S. 1–3; M. K.\pwindex{Kalbeck, Max 1850-01-04 – 1921-05-04@\textsc{Kalbeck, Max} (1850-01-04 – 1921-05-04), \emph{Journalist}|pwk} [ = Max Kalbeck\pwindex{Kalbeck, Max 1850-01-04 – 1921-05-04@\textsc{Kalbeck, Max} (1850-01-04 – 1921-05-04), \emph{Journalist}|pwk}]: \emph{Theater,
                        Kunst und Literatur. Burgtheater}\pwindex{Theater, Kunst und Literatur. Burgtheater [Liebelei, Rechte der Seele]1895-10-10@\emph{Theater, Kunst und Literatur. Burgtheater [Liebelei, Rechte der Seele]} {[}1895-10-10{]}|pwk}. In: \emph{Neues Wiener Tagblatt}\pwindex{?? Werk@Nicht ermittelte Verfasserinnen und Verfasser!Neues Wiener Tagblatt1867 – 1945@\emph{Neues Wiener Tagblatt} {[}1867 – 1945{]}|pwk}, Jg. 29, Nr. 278, 10. 10. 1895, S. 7.}}}\label{K_L02751-3h} ſich ausdrückte, die über Dein Stück\pwindex{Schnitzler, Arthur 15.05.1862 – 21.10.1931@\textsc{Schnitzler, Arthur} (15.05.1862 – 21.10.1931), \emph{Schriftsteller, Mediziner}!Liebelei. Schauspiel in drei Akten1895-10-09@\strich\emph{Liebelei. Schauspiel in drei Akten} {[}1895-10-09{]}|pwv} geſchrieben haben.\pend
           \pstart
           {\pb}Ich habe geleſen die Referate\pwindex{Theater- und Kunstnachrichten. [Burgtheater] [Liebelei, Rechte der
                  Seele]1895-10-10@\emph{Theater- und Kunstnachrichten. [Burgtheater] [Liebelei, Rechte der Seele]} {[}1895-10-10{]}|pwv}\pwindex{Theater, Kunst und Literatur. Burgtheater [Liebelei, Rechte der Seele]1895-10-10@\emph{Theater, Kunst und Literatur. Burgtheater [Liebelei, Rechte der Seele]} {[}1895-10-10{]}|pwv}\pwindex{Kalbeck, Max 1850-01-04 – 1921-05-04@\textsc{Kalbeck, Max} (1850-01-04 – 1921-05-04), \emph{Journalist}!Burgtheater. »Liebelei«, Schauspiel in drei Acten von Arthur Schnitzler. –
                  »Rechte der Seele«, Schauspiel in einem Acte von Guiseppe Giacosa; deutsch von
                  Otto Eisenschitz1895-10-11@\strich\emph{Burgtheater. »Liebelei«, Schauspiel in drei Acten von Arthur Schnitzler. – »Rechte der Seele«, Schauspiel in einem Acte von Guiseppe Giacosa; deutsch von Otto Eisenschitz} {[}1895-10-11{]}|pwv}\pwindex{Theater, Kunst und Literatur. (Burgtheater.) [Liebelei]1895-10-10@\emph{Theater, Kunst und Literatur. (Burgtheater.) [Liebelei]} {[}1895-10-10{]}|pwv}\pwindex{Theater und Kunst. (Burgtheater.) [Liebelei, Rechte der Seele]1895-10-10@\emph{Theater und Kunst. (Burgtheater.) [Liebelei, Rechte der Seele]} {[}1895-10-10{]}|pwv}\pwindex{Granichstaedten, Emil 1847-07-08 – 1904-07-02@\textsc{Granichstaedten, Emil} (1847-07-08 – 1904-07-02), \emph{Journalist, Rechtswissenschaftler}!Burgtheater. Zwei Schauspiele: »Rechte der Seele« von Giuseppe Giacosa. –
                  »Liebelei« von Arthur Schnitzler1895-10-11@\strich\emph{Burgtheater. Zwei Schauspiele: »Rechte der Seele« von Giuseppe Giacosa. – »Liebelei« von Arthur Schnitzler} {[}1895-10-11{]}|pwv}\pwindex{Wiener Brief. [Liebelei, Rechte der Seele]1895-10-11@\emph{Wiener Brief. [Liebelei, Rechte der Seele]} {[}1895-10-11{]}|pwv}\pwindex{?? Werk@Nicht ermittelte Verfasserinnen und Verfasser!Aus Wien, 9. Oktober] [Liebelei]1895-10-10@\emph{[Aus Wien, 9. Oktober] [Liebelei]} {[}1895-10-10{]}|pwv}\pwindex{?? Werk@Nicht ermittelte Verfasserinnen und Verfasser!Liebelei« [Telegramm zur Urauffuehrung]1895-10-10@\emph{»Liebelei« [Telegramm zur Uraufführung]} {[}1895-10-10{]}|pwv}\pwindex{Wien, 9. Oktober] [Liebelei]1895-10-10@\emph{[Wien, 9. Oktober] [Liebelei]} {[}1895-10-10{]}|pwv} von: \label{K_L02751-4v}\edtext{\textsc{Speidel\pwindex{Speidel, Ludwig 1830-04-11 – 1906-02-03@\textsc{Speidel, Ludwig} (1830-04-11 – 1906-02-03), \emph{Journalist, Kritiker}|pw}\pwindex{Theater- und Kunstnachrichten. [Burgtheater] [Liebelei, Rechte der
                  Seele]1895-10-10@\emph{Theater- und Kunstnachrichten. [Burgtheater] [Liebelei, Rechte der Seele]} {[}1895-10-10{]}|pwv}}}{\lemma{\textnormal{\emph{Speidel}}}\Cendnote{\textnormal{[Ludwig Speidel\pwindex{Speidel, Ludwig 1830-04-11 – 1906-02-03@\textsc{Speidel, Ludwig} (1830-04-11 – 1906-02-03), \emph{Journalist, Kritiker}|pwk}]: \emph{Theater- und Kunstnachrichten. [Burgtheater]}\pwindex{Theater- und Kunstnachrichten. [Burgtheater] [Liebelei, Rechte der
                  Seele]1895-10-10@\emph{Theater- und Kunstnachrichten. [Burgtheater] [Liebelei, Rechte der Seele]} {[}1895-10-10{]}|pwk}. In: \emph{Neue Freie Presse}\pwindex{Neue Freie Presse1864 – 1939@\emph{Neue Freie Presse} {[}1864 – 1939{]}|pwk}, Nr. 11.181, 10. 10. 1895, S. 7. Ein weiteres Feuilleton\pwindex{Burgtheater. (»Liebelei«, Schauspiel in drei Aufzuegen von Arthur Schnitzler. –
                  »Rechte der Seele«, Schauspiel in einem Act von Giuseppe Giacosa, deutsch von Otto
                  Eisenschitz.)1895-10-13@\emph{Burgtheater. (»Liebelei«, Schauspiel in drei Aufzügen von Arthur Schnitzler. – »Rechte der Seele«, Schauspiel in einem Act von Giuseppe Giacosa, deutsch von Otto Eisenschitz.)} {[}1895-10-13{]}|pwkv} erschien am Tag dieses Briefes und war Goldmann\pwindex{Goldmann, Paul 31.01.1865 – 25.09.1935@\textsc{Goldmann, Paul} (31.01.1865 – 25.09.1935), \emph{Schriftsteller, Journalist}|pwk} zu diesem Zeitpunkt noch unbekannt: L. Sp.\pwindex{Speidel, Ludwig 1830-04-11 – 1906-02-03@\textsc{Speidel, Ludwig} (1830-04-11 – 1906-02-03), \emph{Journalist, Kritiker}|pwkv} [ = Ludwig Speidel\pwindex{Speidel, Ludwig 1830-04-11 – 1906-02-03@\textsc{Speidel, Ludwig} (1830-04-11 – 1906-02-03), \emph{Journalist, Kritiker}|pwk}]: \emph{Burgtheater. (»Liebelei«, Schauspiel in drei Aufzügen von
                        Arthur Schnitzler. – »Rechte der Seele«, Schauspiel in einem Act von
                        Giuseppe Giacosa, deutsch von Otto Eisenschitz.)}\pwindex{Burgtheater. (»Liebelei«, Schauspiel in drei Aufzuegen von Arthur Schnitzler. –
                  »Rechte der Seele«, Schauspiel in einem Act von Giuseppe Giacosa, deutsch von Otto
                  Eisenschitz.)1895-10-13@\emph{Burgtheater. (»Liebelei«, Schauspiel in drei Aufzügen von Arthur Schnitzler. – »Rechte der Seele«, Schauspiel in einem Act von Giuseppe Giacosa, deutsch von Otto Eisenschitz.)} {[}1895-10-13{]}|pwk}. In: \emph{Neue Freie Presse}\pwindex{Neue Freie Presse1864 – 1939@\emph{Neue Freie Presse} {[}1864 – 1939{]}|pwk}, Nr. 11.184, 13. 10. 1895, Morgenblatt, S. 1–3.}}}\label{K_L02751-4h} (prachtvoll), \textsc{Kalbeck\pwindex{Kalbeck, Max 1850-01-04 – 1921-05-04@\textsc{Kalbeck, Max} (1850-01-04 – 1921-05-04), \emph{Journalist}|pw}\pwindex{Theater, Kunst und Literatur. Burgtheater [Liebelei, Rechte der Seele]1895-10-10@\emph{Theater, Kunst und Literatur. Burgtheater [Liebelei, Rechte der Seele]} {[}1895-10-10{]}|pwv}\pwindex{Kalbeck, Max 1850-01-04 – 1921-05-04@\textsc{Kalbeck, Max} (1850-01-04 – 1921-05-04), \emph{Journalist}!Burgtheater. »Liebelei«, Schauspiel in drei Acten von Arthur Schnitzler. –
                  »Rechte der Seele«, Schauspiel in einem Acte von Guiseppe Giacosa; deutsch von
                  Otto Eisenschitz1895-10-11@\strich\emph{Burgtheater. »Liebelei«, Schauspiel in drei Acten von Arthur Schnitzler. – »Rechte der Seele«, Schauspiel in einem Acte von Guiseppe Giacosa; deutsch von Otto Eisenschitz} {[}1895-10-11{]}|pwv}} (die erſten sympathiſchen Zeilen, die ich von dem Manne\pwindex{Kalbeck, Max 1850-01-04 – 1921-05-04@\textsc{Kalbeck, Max} (1850-01-04 – 1921-05-04), \emph{Journalist}|pwv} leſe), \label{K_L02751-5v}\edtext{\textsc{Schoenthan\pwindex{Schoenthan-Pernwald, Paul von 19.03.1853 – 04.08.1905@\textsc{Schönthan-Pernwald, Paul von} (19.03.1853 – 04.08.1905), \emph{Schriftsteller, Journalist}|pw}\pwindex{Theater, Kunst und Literatur. (Burgtheater.) [Liebelei]1895-10-10@\emph{Theater, Kunst und Literatur. (Burgtheater.) [Liebelei]} {[}1895-10-10{]}|pwv}}}{\lemma{\textnormal{\emph{Schoenthan}}}\Cendnote{\textnormal{p. v. s.\pwindex{Schoenthan-Pernwald, Paul von 19.03.1853 – 04.08.1905@\textsc{Schönthan-Pernwald, Paul von} (19.03.1853 – 04.08.1905), \emph{Schriftsteller, Journalist}|pwkv} [ = Paul von Schönthan-Pernwald\pwindex{Schoenthan-Pernwald, Paul von 19.03.1853 – 04.08.1905@\textsc{Schönthan-Pernwald, Paul von} (19.03.1853 – 04.08.1905), \emph{Schriftsteller, Journalist}|pwk}]: \emph{Theater, Kunst und Literatur.
                        (Burgtheater.)}\pwindex{Theater, Kunst und Literatur. (Burgtheater.) [Liebelei]1895-10-10@\emph{Theater, Kunst und Literatur. (Burgtheater.) [Liebelei]} {[}1895-10-10{]}|pwk}. In: \emph{Wiener
                        Tagblatt}\pwindex{Wiener Tagblatt1886 – 1901@\emph{Wiener Tagblatt} {[}1886 – 1901{]}|pwk}, Jg. 45, Nr. 278, 10. 10. 1895,
                     S. 5–6.}}}\label{K_L02751-5h} (der vor Bühnendichter-Neid zerſpringt); ferner das
                  \label{K_L02751-6v}\edtext{Referat\pwindex{Theater und Kunst. (Burgtheater.) [Liebelei, Rechte der Seele]1895-10-10@\emph{Theater und Kunst. (Burgtheater.) [Liebelei, Rechte der Seele]} {[}1895-10-10{]}|pwv} des »Wiener Journal\pwindex{Wiener Tagblatt1886 – 1901@\emph{Wiener Tagblatt} {[}1886 – 1901{]}|pw}}{\lemma{\textnormal{\emph{Referat … Journal}}}\Cendnote{\textnormal{–v–\pwindex{David, Jakob Julius 1859-02-06 – 1906-11-20@\textsc{David, Jakob Julius} (1859-02-06 – 1906-11-20), \emph{Schriftsteller, Journalist}|pwkv} [ = Jakob Julius David\pwindex{David, Jakob Julius 1859-02-06 – 1906-11-20@\textsc{David, Jakob Julius} (1859-02-06 – 1906-11-20), \emph{Schriftsteller, Journalist}|pwk}]: \emph{Theater und Kunst. (Burgtheater.)}\pwindex{Theater und Kunst. (Burgtheater.) [Liebelei, Rechte der Seele]1895-10-10@\emph{Theater und Kunst. (Burgtheater.) [Liebelei, Rechte der Seele]} {[}1895-10-10{]}|pwk}. In: \emph{Neues Wiener Journal}\pwindex{Neues Wiener Journal1893 – 1939@\emph{Neues Wiener Journal} {[}1893 – 1939{]}|pwk}, Jg. 3, Nr. 704, 10. 10. 1895, S. 5.}}}\label{K_L02751-6h}« (verſtändnißlos,
               aber mit Einzelheiten, die ausſöhnen), endlich \label{K_L02751-7v}\edtext{\textsc{Granichstaedten\pwindex{Granichstaedten, Emil 1847-07-08 – 1904-07-02@\textsc{Granichstaedten, Emil} (1847-07-08 – 1904-07-02), \emph{Journalist, Rechtswissenschaftler}|pw}\pwindex{Granichstaedten, Emil 1847-07-08 – 1904-07-02@\textsc{Granichstaedten, Emil} (1847-07-08 – 1904-07-02), \emph{Journalist, Rechtswissenschaftler}!Burgtheater. Zwei Schauspiele: »Rechte der Seele« von Giuseppe Giacosa. –
                  »Liebelei« von Arthur Schnitzler1895-10-11@\strich\emph{Burgtheater. Zwei Schauspiele: »Rechte der Seele« von Giuseppe Giacosa. – »Liebelei« von Arthur Schnitzler} {[}1895-10-11{]}|pwv}}}{\lemma{\textnormal{\emph{Granichstaedten}}}\Cendnote{\textnormal{Emil Granichstaedten\pwindex{Granichstaedten, Emil 1847-07-08 – 1904-07-02@\textsc{Granichstaedten, Emil} (1847-07-08 – 1904-07-02), \emph{Journalist, Rechtswissenschaftler}|pwk}: \emph{Feuilleton. Burgtheater}\pwindex{Granichstaedten, Emil 1847-07-08 – 1904-07-02@\textsc{Granichstaedten, Emil} (1847-07-08 – 1904-07-02), \emph{Journalist, Rechtswissenschaftler}!Burgtheater. Zwei Schauspiele: »Rechte der Seele« von Giuseppe Giacosa. –
                  »Liebelei« von Arthur Schnitzler1895-10-11@\strich\emph{Burgtheater. Zwei Schauspiele: »Rechte der Seele« von Giuseppe Giacosa. – »Liebelei« von Arthur Schnitzler} {[}1895-10-11{]}|pwk}. In: \emph{Die Presse}\pwindex{?? Werk@Nicht ermittelte Verfasserinnen und Verfasser!Presse1848-07-03@\emph{Die Presse} {[}1848-07-03{]}|pwk}, Jg. 48, Nr. 279, 11. 10. 1895, S. 1–2.}}}\label{K_L02751-7h}, das widerliche Thier\pwindex{Granichstaedten, Emil 1847-07-08 – 1904-07-02@\textsc{Granichstaedten, Emil} (1847-07-08 – 1904-07-02), \emph{Journalist, Rechtswissenschaftler}|pwv} (Ohrfeigen!!!). \label{K_L02751-8v}\edtext{\textsc{Uhl\pwindex{Uhl, Friedrich 14.05.1825 – 20.01.1906@\textsc{Uhl, Friedrich} (14.05.1825 – 20.01.1906), \emph{Journalist}|pw}\pwindex{Wiener Brief. [Liebelei, Rechte der Seele]1895-10-11@\emph{Wiener Brief. [Liebelei, Rechte der Seele]} {[}1895-10-11{]}|pwv}\strikeout{\textcolor{gray}{×}}} in der »Frankfurter Zeitung\pwindex{?? Werk@Nicht ermittelte Verfasserinnen und Verfasser!Frankfurter Zeitung1856 – 1943@\emph{Frankfurter Zeitung} {[}1856 – 1943{]}|pw}}{\lemma{\textnormal{\emph{Uhl … Zeitung}}}\Cendnote{\textnormal{[Friedrich Uhl\pwindex{Uhl, Friedrich 14.05.1825 – 20.01.1906@\textsc{Uhl, Friedrich} (14.05.1825 – 20.01.1906), \emph{Journalist}|pwk}]: \emph{Wiener Brief}\pwindex{Wiener Brief. [Liebelei, Rechte der Seele]1895-10-11@\emph{Wiener Brief. [Liebelei, Rechte der Seele]} {[}1895-10-11{]}|pwk}. In: \emph{Frankfurter Zeitung}\pwindex{?? Werk@Nicht ermittelte Verfasserinnen und Verfasser!Frankfurter Zeitung1856 – 1943@\emph{Frankfurter Zeitung} {[}1856 – 1943{]}|pwk}, Jg. 40, Nr. 282, 11. 10. 1895,
                     Abendblatt, S. 1. }}}\label{K_L02751-8h}« hätte wärmer und ausführlicher ſein können;
               ich vermuthe, daß es ihn {\pb}verſtimmt, weil die
               Officiellen (\textsc{Speidel\pwindex{Speidel, Ludwig 1830-04-11 – 1906-02-03@\textsc{Speidel, Ludwig} (1830-04-11 – 1906-02-03), \emph{Journalist, Kritiker}|pw} etc.}) Dich loben. Auch iſt er
               wohl von denen, die Jemanden fördern, – bis er einen Erfolg hat, die aber ſofort von
               dem Erfolge ſelbſt unſympathiſch berührt werden. Eine echte Oppoſitions-Natur\pwindex{Uhl, Friedrich 14.05.1825 – 20.01.1906@\textsc{Uhl, Friedrich} (14.05.1825 – 20.01.1906), \emph{Journalist}|pwv} mit einem Worte. In \strikeout{B\textcolor{gray}{e}}{ }Berlin\oindex{Berlin@\textbf{Berlin}|pw}er Blättern las ich das kurze, aber ſehr
               freundliche \label{K_L02751-9v}\edtext{Telegramm\pwindex{?? Werk@Nicht ermittelte Verfasserinnen und Verfasser!Aus Wien, 9. Oktober] [Liebelei]1895-10-10@\emph{[Aus Wien, 9. Oktober] [Liebelei]} {[}1895-10-10{]}|pwv} des »Tageblatt\pwindex{?? Werk@Nicht ermittelte Verfasserinnen und Verfasser!Berliner Tageblatt1872 – 1939@\emph{Berliner Tageblatt} {[}1872 – 1939{]}|pw}}{\lemma{\textnormal{\emph{Telegramm des »Tageblatt}}}\Cendnote{\textnormal{[O. V.]: \emph{[Aus Wien, 9. Oktober]}\pwindex{?? Werk@Nicht ermittelte Verfasserinnen und Verfasser!Aus Wien, 9. Oktober] [Liebelei]1895-10-10@\emph{[Aus Wien, 9. Oktober] [Liebelei]} {[}1895-10-10{]}|pwk}. In:
                        \emph{Berliner Tageblatt}\pwindex{?? Werk@Nicht ermittelte Verfasserinnen und Verfasser!Berliner Tageblatt1872 – 1939@\emph{Berliner Tageblatt} {[}1872 – 1939{]}|pwk}, Jg. 24, Nr. 516,
                        10. 10. 1895, Abend-Ausgabe,
                  S. 3.}}}\label{K_L02751-9h}«, das ſehr warme \label{K_L02751-10v}\edtext{Telegramm\pwindex{?? Werk@Nicht ermittelte Verfasserinnen und Verfasser!Liebelei« [Telegramm zur Urauffuehrung]1895-10-10@\emph{»Liebelei« [Telegramm zur Uraufführung]} {[}1895-10-10{]}|pwv} des »Lokalanzeiger\pwindex{?? Werk@Nicht ermittelte Verfasserinnen und Verfasser!Berliner Lokal-Anzeiger1883@\emph{Berliner Lokal-Anzeiger} {[}1883{]}|pw}}{\lemma{\textnormal{\emph{Telegramm des »Lokalanzeiger}}}\Cendnote{\textnormal{»›Liebelei\pwindex{Schnitzler, Arthur 15.05.1862 – 21.10.1931@\textsc{Schnitzler, Arthur} (15.05.1862 – 21.10.1931), \emph{Schriftsteller, Mediziner}!Liebelei. Schauspiel in drei Akten1895-10-09@\strich\emph{Liebelei. Schauspiel in drei Akten} {[}1895-10-09{]}|pw}‹, ein Drama eines jungen Wien\oindex{Wien@\textbf{Wien}|pw}er Schriftsteller\pwindex{Schnitzler, Arthur 15.05.1862 – 21.10.1931@\textsc{Schnitzler, Arthur} (15.05.1862 – 21.10.1931), \emph{Schriftsteller, Mediziner}|pwv}s, ist gestern
                           (Mittwoch){ }Abend im Wiener Burgtheater\orgindex{Burgtheater@Burgtheater|pw}
                        zum ersten Male aufgeführt worden; wir erhalten darüber folgendes \so{Privat-Telegramm}:\textbf{Wien}\oindex{Wien@\textbf{Wien}|pw}, 9. October, 11 Uhr 50 Min. Abends (Von unserem
                        \textsc{\textcolor{gray}{\so{n}}\so{.a.}}\so{-Correspondenten.}) { / }Das bürgerliche Drama ›Liebelei\pwindex{Schnitzler, Arthur 15.05.1862 – 21.10.1931@\textsc{Schnitzler, Arthur} (15.05.1862 – 21.10.1931), \emph{Schriftsteller, Mediziner}!Liebelei. Schauspiel in drei Akten1895-10-09@\strich\emph{Liebelei. Schauspiel in drei Akten} {[}1895-10-09{]}|pw}‹ von Arthur Schnitzler\pwindex{Schnitzler, Arthur 15.05.1862 – 21.10.1931@\textsc{Schnitzler, Arthur} (15.05.1862 – 21.10.1931), \emph{Schriftsteller, Mediziner}|pw} hatte heute im Burgtheater\orgindex{Burgtheater@Burgtheater|pw} einen
                        bedeutenden Erfolg. Der Verfasser\pwindex{Schnitzler, Arthur 15.05.1862 – 21.10.1931@\textsc{Schnitzler, Arthur} (15.05.1862 – 21.10.1931), \emph{Schriftsteller, Mediziner}|pwv} wurde nach jedem Akt wiederholt gerufen, obwohl in dem
                           Stück\pwindex{Schnitzler, Arthur 15.05.1862 – 21.10.1931@\textsc{Schnitzler, Arthur} (15.05.1862 – 21.10.1931), \emph{Schriftsteller, Mediziner}!Liebelei. Schauspiel in drei Akten1895-10-09@\strich\emph{Liebelei. Schauspiel in drei Akten} {[}1895-10-09{]}|pwv} sociale
                        Verhältnisse behandelt werden, die auf dem Hoftheater\orgindex{Burgtheater@Burgtheater|pwv} sonst Befremden erregen. Das
                        Bürgermädchen, das an einer Liebelei zu Grunde geht, wurde von der Sandrock\pwindex{Sandrock, Adele 1863-08-19 – 1937-08-30@\textsc{Sandrock, Adele} (1863-08-19 – 1937-08-30), \emph{Schauspielerin}|pw} mit tragischem Nachdruck
                        gespielt, ergreifend war auch Sonnenthal\pwindex{Sonnenthal, Adolf von 1834-12-21 – 1909-04-04@\textsc{Sonnenthal, Adolf von} (1834-12-21 – 1909-04-04), \emph{Schauspieler}|pw} als ihr Vater.« (\emph{Berliner Lokal-Anzeiger}\pwindex{?? Werk@Nicht ermittelte Verfasserinnen und Verfasser!Berliner Lokal-Anzeiger1883@\emph{Berliner Lokal-Anzeiger} {[}1883{]}|pwk}, Jg. 13, Nr. 475,
                        10. 10. 1895, Morgenblatt, 1. Ausgabe,
                     S. 3)}}}\label{K_L02751-10h}« und {\pb}das
               blödſinnig-freche \label{K_L02751-11v}\edtext{Telegramm\pwindex{Wien, 9. Oktober] [Liebelei]1895-10-10@\emph{[Wien, 9. Oktober] [Liebelei]} {[}1895-10-10{]}|pwv} des »Kleinen Journal\pwindex{?? Werk@Nicht ermittelte Verfasserinnen und Verfasser!Kleine Journal1878 – 1935@\emph{Das Kleine Journal} {[}1878 – 1935{]}|pw}« }{\lemma{\textnormal{\emph{Telegramm … Journal«}}}\Cendnote{\textnormal{[Julius Konried\pwindex{Konried, Julius 27.11.1853 – 13.01.1927@\textsc{Konried, Julius} (27.11.1853 – 13.01.1927), \emph{Journalist}|pwk}]: \emph{[Wien, 9. Oktober]}\pwindex{Wien, 9. Oktober] [Liebelei]1895-10-10@\emph{[Wien, 9. Oktober] [Liebelei]} {[}1895-10-10{]}|pwk}. In: \emph{Das Kleine Journal}\pwindex{?? Werk@Nicht ermittelte Verfasserinnen und Verfasser!Kleine Journal1878 – 1935@\emph{Das Kleine Journal} {[}1878 – 1935{]}|pwk}, Jg. 17, Nr. 278, 10. 10. 1895, S. [4]. Darin ist zu lesen: »Der
                     eifrigste Anhänger der Hermann
                     Bahr\pwindex{Bahr, Hermann 19.07.1863 – 15.01.1934@\textsc{Bahr, Hermann} (19.07.1863 – 15.01.1934), \emph{Schriftsteller, Kritiker}|pw}’schen Schule, 
                     \so{Schnitzler}\pwindex{Schnitzler, Arthur 15.05.1862 – 21.10.1931@\textsc{Schnitzler, Arthur} (15.05.1862 – 21.10.1931), \emph{Schriftsteller, Mediziner}|pw}, hat heute seinen Einzug ins 
                     \so{Burgtheater}\orgindex{Burgtheater@Burgtheater|pw}
                      gehalten.«}}}\label{K_L02751-11h} (Correſpondent\pwindex{Konried, Julius 27.11.1853 – 13.01.1927@\textsc{Konried, Julius} (27.11.1853 – 13.01.1927), \emph{Journalist}|pwv} Herr \textsc{Conried\pwindex{Konried, Julius 27.11.1853 – 13.01.1927@\textsc{Konried, Julius} (27.11.1853 – 13.01.1927), \emph{Journalist}|pw}} vom »Neuen Wiener Tagblatt\pwindex{?? Werk@Nicht ermittelte Verfasserinnen und Verfasser!Neues Wiener Tagblatt1867 – 1945@\emph{Neues Wiener Tagblatt} {[}1867 – 1945{]}|pw}«), das Dich
               einen Mann aus der \textsc{Hermann Bahr\pwindex{Bahr, Hermann 19.07.1863 – 15.01.1934@\textsc{Bahr, Hermann} (19.07.1863 – 15.01.1934), \emph{Schriftsteller, Kritiker}|pw}schen} Schule nennt.\pend
           \pstart
           Den Abend der \textsc{Première} verbrachte ich mit \textsc{Th. Wolff\pwindex{Wolff, Theodor 1868-08-02 – 1943-09-23@\textsc{Wolff, Theodor} (1868-08-02 – 1943-09-23), \emph{Schriftsteller, Journalist}|pw}} (vom »Berliner Tageblatt\pwindex{?? Werk@Nicht ermittelte Verfasserinnen und Verfasser!Berliner Tageblatt1872 – 1939@\emph{Berliner Tageblatt} {[}1872 – 1939{]}|pw}«) und ſah fleißig
               auf die Uhr. Um neun Uhr meinte ich, Dein Schickfal müſſe ſich wohl
               entſchieden haben, und da ſchlug \textsc{Wolff\pwindex{Wolff, Theodor 1868-08-02 – 1943-09-23@\textsc{Wolff, Theodor} (1868-08-02 – 1943-09-23), \emph{Schriftsteller, Journalist}|pw}} vor, auf Dein {\pb}Wohl anzuſtoßen, Was
               geſchah.\pend
           \pstart
           Die Meinigen, mein Onkel\pwindex{Mamroth, Fedor 21.02.1851 – 25.06.1907@\textsc{Mamroth, Fedor} (21.02.1851 – 25.06.1907), \emph{Journalist, Kritiker}|pwv},
               meine Mutter\pwindex{Goldmann, Clementine 1842-05-15 – 1924-02-24@\textsc{Goldmann, Clementine} (1842-05-15 – 1924-02-24)|pwv}, mein Schwager\pwindex{Rosengart, Josef 1860-02-08 – 1927-08-04@\textsc{Rosengart, Josef} (1860-02-08 – 1927-08-04), \emph{Arzt}|pwv}, ſind, wie mir heut meine Mutter\pwindex{Goldmann, Clementine 1842-05-15 – 1924-02-24@\textsc{Goldmann, Clementine} (1842-05-15 – 1924-02-24)|pwv} ſchreibt, hocherfreut über Deinen Erfolg und laſſen
               Dir von Herzen gratuliren.\pend
           \pstart
           Am Tag nach der \begin{otherlanguage}{french}\textsc{Première}\end{otherlanguage}, nachdem ich Dein Telegramm erhalten, fuhr ich zur »\textsc{Liberté\orgindex{Liberte@La Liberté|pw}}« und zu den »\textsc{Débats\orgindex{Journal des debats@Journal des débats|pw}}« und bat um eine \label{K_L02751-12v}\edtext{Notiz}{\lemma{\textnormal{\emph{Notiz}}}\Cendnote{\textnormal{[Georges Aubry\pwindex{Aubry, Georges †~1923@\textsc{Aubry, Georges} (†~1923), \emph{Redakteur}|pwk}]: \emph{Thêatres. [Notre correspondant de Vienne]}\pwindex{Aubry, Georges †~1923@\textsc{Aubry, Georges} (†~1923), \emph{Redakteur}!Thêatres. [Notre correspondant de Vienne]1895-10-12@\strich\emph{Thêatres. [Notre correspondant de Vienne]} {[}1895-10-12{]}|pwk}. In: \emph{La Liberté}\pwindex{?? Werk@Nicht ermittelte Verfasserinnen und Verfasser!Liberte1865-07-16 – 1940-06-11@\emph{La Liberté} {[}1865-07-16 – 1940-06-11{]}|pwk}, Jg. 30, Nr. 11.289, 12. 10. 1895, S. 3. Siehe dazu auch Paul Goldmann an Arthur Schnitzler, 7. 10. [1895].  [Hippolyte Fierens-Gevaert\pwindex{Fierens-Gevaert, Hippolyte 1870-08-13 – 1926-12-16@\textsc{Fierens-Gevaert, Hippolyte} (1870-08-13 – 1926-12-16), \emph{Schriftsteller, Sänger, Kunstkritiker}|pwk}]: \emph{Courrier des Théatres}\pwindex{Courrier des Theatres [Liebelei]1895-10-12@\emph{Courrier des Théatres [Liebelei]} {[}1895-10-12{]}|pwk}. In: \emph{Journal des débats politiques et littéraires}\pwindex{?? Werk@Nicht ermittelte Verfasserinnen und Verfasser!Journal des debats. Politiques et litteraires1789 – 1944@\emph{Journal des débats. Politiques et littéraires} {[}1789 – 1944{]}|pwk}, Jg. 107,
                        12. 10. 1895, S. 3.}}}\label{K_L02751-12h}. Beide Blätter\orgindex{Liberte@La Liberté|pwv}\orgindex{Journal des debats@Journal des débats|pwv} haben die
               Bitte mit großer {\pb}Liebenswürdigkeit erfüllt. Ich
               ſende ſie Dir anbei; ſtoße Dich nicht an die Unrichtigkeiten, die Du in den Notizen\pwindex{Aubry, Georges †~1923@\textsc{Aubry, Georges} (†~1923), \emph{Redakteur}!Thêatres. [Notre correspondant de Vienne]1895-10-12@\strich\emph{Thêatres. [Notre correspondant de Vienne]} {[}1895-10-12{]}|pwv}\pwindex{Courrier des Theatres [Liebelei]1895-10-12@\emph{Courrier des Théatres [Liebelei]} {[}1895-10-12{]}|pwv} findeſt; ich
               habe ihnen die Geſchichte zwar genau erklärt, aber ſie haben doch geſchrieben, was
               ſie wollten; das iſt ſo Pariſ\oindex{Paris@\textbf{Paris}|pw}er Art. Jedenfalls
               aber mußt Du Dich bedanken; das iſt hier ſo Sitte. Zuerſt mußt Du \strikeout{e\textcolor{gray}{i}} Deine Viſitkarte mit der Aufſchrift: \label{K_L02751-13v}\edtext{\begin{otherlanguage}{french}\textsc{remercie bien vivement M. Fierens-Gevaert\pwindex{Fierens-Gevaert, Hippolyte 1870-08-13 – 1926-12-16@\textsc{Fierens-Gevaert, Hippolyte} (1870-08-13 – 1926-12-16), \emph{Schriftsteller, Sänger, Kunstkritiker}|pw} de son amabilité}\end{otherlanguage}}{\lemma{\textnormal{\emph{remercie … amabilité}}}\Cendnote{\textnormal{französisch: dankt sehr herzlich Herrn
                     Fierens-Gevaert\pwindex{Fierens-Gevaert, Hippolyte 1870-08-13 – 1926-12-16@\textsc{Fierens-Gevaert, Hippolyte} (1870-08-13 – 1926-12-16), \emph{Schriftsteller, Sänger, Kunstkritiker}|pwk} für seine
                  Freundlichkeit}}}\label{K_L02751-13h}{ }{\pb}ſchicken an: \begin{otherlanguage}{french}\textsc{M. Fierens-Gevaert\pwindex{Fierens-Gevaert, Hippolyte 1870-08-13 – 1926-12-16@\textsc{Fierens-Gevaert, Hippolyte} (1870-08-13 – 1926-12-16), \emph{Schriftsteller, Sänger, Kunstkritiker}|pw}, du
                        »Journal des Débats\orgindex{Journal des debats@Journal des débats|pw}«, Rue des Prêtres – St. Germain l’Auxerrois, Paris\oindex{Rue Des Prêtres Saint-Germain L'Auxerrois@\textbf{Rue Des Prêtres Saint-Germain L'Auxerrois}|pw}}\end{otherlanguage}. Eine zweite Karte ſendeſt Du an \textsc{M. Aubry\pwindex{Aubry, Georges †~1923@\textsc{Aubry, Georges} (†~1923), \emph{Redakteur}|pw}\textcolor{gray}{,}{ }\begin{otherlanguage}{french}de la »Liberté\orgindex{Liberte@La Liberté|pw}«, 10. Rue Camou, Paris\oindex{Rue du General Camou@\textbf{Rue du Général Camou}|pw}\end{otherlanguage}}. Hier mußt Du ſchon etwas wärmer ſchreiben, da Aubry\pwindex{Aubry, Georges †~1923@\textsc{Aubry, Georges} (†~1923), \emph{Redakteur}|pw} ein ſehr herzliches Intereſſe für Dich bezeigt, ſich
               eine mörderiſche Mühe {\pb}gegeben hat, um die von
               ſeiner Frau\pwindex{Aubry, [MMe. Georges] @\textsc{Aubry, [MMe. Georges]}, \emph{Übersetzerin}|pwv} überſetzte »Kleine Komödie\pwindex{Schnitzler, Arthur 15.05.1862 – 21.10.1931@\textsc{Schnitzler, Arthur} (15.05.1862 – 21.10.1931), \emph{Schriftsteller, Mediziner}!kleine Komoedie1895-08-01@\strich\emph{Die kleine Komödie} {[}1895-08-01{]}|pw}« in gutes Franzöſiſch zu bringen
               (die Überſetzung\pwindex{Aubry, [MMe. Georges] @\textsc{Aubry, [MMe. Georges]}, \emph{Übersetzerin}!petite comedie. Mœurs viennois1895-11-19 – 1895-11-28@\strich\emph{La petite comédie. Mœurs viennois} {[}Übersetzung, 1895-11-19 – 1895-11-28{]}|pwv} iſt
               infolgedeſſen vortrefflich) \textsc{et{[}c{]}}. Du ſchreibſt alſo vielleicht auf Deine Karte: \label{K_L02751-14v}\edtext{\begin{otherlanguage}{french}\textsc{remercie M. Aubry\pwindex{Aubry, Georges †~1923@\textsc{Aubry, Georges} (†~1923), \emph{Redakteur}|pw} du
                        \strikeout{bel} très-bel article\pwindex{Aubry, Georges †~1923@\textsc{Aubry, Georges} (†~1923), \emph{Redakteur}!Thêatres. [Notre correspondant de Vienne]1895-10-12@\strich\emph{Thêatres. [Notre correspondant de Vienne]} {[}1895-10-12{]}|pwv} au sujet de la »Liebelei\pwindex{Schnitzler, Arthur 15.05.1862 – 21.10.1931@\textsc{Schnitzler, Arthur} (15.05.1862 – 21.10.1931), \emph{Schriftsteller, Mediziner}!Liebelei. Schauspiel in drei Akten1895-10-09@\strich\emph{Liebelei. Schauspiel in drei Akten} {[}1895-10-09{]}|pw}«, le remercie en outre de toute la
                     peine, qu’il s’est donnée pour la traduction\pwindex{Aubry, [MMe. Georges] @\textsc{Aubry, [MMe. Georges]}, \emph{Übersetzerin}!petite comedie. Mœurs viennois1895-11-19 – 1895-11-28@\strich\emph{La petite comédie. Mœurs viennois} {[}Übersetzung, 1895-11-19 – 1895-11-28{]}|pwv} de la »Petite comédie\pwindex{Schnitzler, Arthur 15.05.1862 – 21.10.1931@\textsc{Schnitzler, Arthur} (15.05.1862 – 21.10.1931), \emph{Schriftsteller, Mediziner}!kleine Komoedie1895-08-01@\strich\emph{Die kleine Komödie} {[}1895-08-01{]}|pw}«, le remercie en un mot de toute son amabilité
                     charmante et espère {\pb}de lui serrer un jour la
                        \strikeout{main} main en ami, soit à Paris\oindex{Paris@\textbf{Paris}|pw}, soit à Vienne\oindex{Wien@\textbf{Wien}|pw}}\end{otherlanguage}}{\lemma{\textnormal{\emph{remercie … Vienne}}}\Cendnote{\textnormal{französisch: dankt Herrn Aubry\pwindex{Aubry, Georges †~1923@\textsc{Aubry, Georges} (†~1923), \emph{Redakteur}|pwk} für den sehr schönen Artikel\pwindex{Aubry, Georges †~1923@\textsc{Aubry, Georges} (†~1923), \emph{Redakteur}!Thêatres. [Notre correspondant de Vienne]1895-10-12@\strich\emph{Thêatres. [Notre correspondant de Vienne]} {[}1895-10-12{]}|pwkv} über die \emph{Liebelei}\pwindex{Schnitzler, Arthur 15.05.1862 – 21.10.1931@\textsc{Schnitzler, Arthur} (15.05.1862 – 21.10.1931), \emph{Schriftsteller, Mediziner}!Liebelei. Schauspiel in drei Akten1895-10-09@\strich\emph{Liebelei. Schauspiel in drei Akten} {[}1895-10-09{]}|pwk}, dankt auch für all die Mühen, die er sich um die
                     Übersetzung\pwindex{Aubry, [MMe. Georges] @\textsc{Aubry, [MMe. Georges]}, \emph{Übersetzerin}!petite comedie. Mœurs viennois1895-11-19 – 1895-11-28@\strich\emph{La petite comédie. Mœurs viennois} {[}Übersetzung, 1895-11-19 – 1895-11-28{]}|pwkv} der »\emph{Kleinen Komödie}\pwindex{Schnitzler, Arthur 15.05.1862 – 21.10.1931@\textsc{Schnitzler, Arthur} (15.05.1862 – 21.10.1931), \emph{Schriftsteller, Mediziner}!kleine Komoedie1895-08-01@\strich\emph{Die kleine Komödie} {[}1895-08-01{]}|pwk}« gemacht hat, dankt ihm mit
                  einem Wort für all seine liebenswürdige Freundlichkeit und hofft, ihm eines Tages
                  in Paris\oindex{Paris@\textbf{Paris}|pwk} oder in Wien\oindex{Wien@\textbf{Wien}|pwk} als Freund die Hand drücken zu dürfen}}}\label{K_L02751-14h}{\dotsfive}\pend
           \pstart
           So, da haſt Du wieder ein wenig Arbeit.\pend
           \pstart
           Nochmals, vielen Dank für Dein Telegramm! Danke auch \textsc{Richard\pwindex{Beer-Hofmann, Richard 1866-07-11 – 1945-09-26@\textsc{Beer-Hofmann, Richard} (1866-07-11 – 1945-09-26), \emph{Schriftsteller}|pw}} für das ſeinige! Und ſei von Herzen gegrüßt!\pend
           \pstart
           Dein {\\[\baselineskip]}\spacefill\mbox{Paul Goldmann.}\pend
           \leftskip=0em{}\pstart
           \noindent{}Bitte, empfiehl’ mich Deiner Frau Mama\pwindex{Schnitzler, Louise 1840-07-08 – 1911-09-09@\textsc{Schnitzler, Louise} (1840-07-08 – 1911-09-09)|pwv} und ſag’ ihr, ich laſſe ihr zu ihrem Sohne
                  gratuliren.\pend
           
         
         \endnumbering\mylabel{h}\end{ledgroupsized}  \newcommand{\dateiname}{L02751}\newcommand{\titel}{Paul Goldmann an Arthur Schnitzler, 13. 10. [1895]}\newcommand{\editorInnen}{Martin Anton Müller und Laura Untner}%% latex-leseansicht-abspann.tex
%% Abspann für die Leseansicht.
%% Der Schalter \ifkorrekturansicht ist bereits durch den Vorspann gesetzt.

%% latex-abspann.tex
%% Gemeinsamer Abspann für Korrekturansicht und Leseansicht.
%% Setzt den Schalter \ifkorrekturansicht voraus (gesetzt in den
%% einbindenden Dateien latex-korrekturansicht-abspann.tex bzw.
%% latex-leseansicht-abspann.tex).
%% ---------------------------------------------------------------

\normalsize

% Das esempio-Environment wird nur in der Leseansicht benötigt
\ifkorrekturansicht\else
\newenvironment{esempio}[3]%
{
    \vspace{1.5ex}
    \rlap{\underline{#1}}
    \par
    \setlength{\parindent}{0cm}
    \nopagebreak
    \leftskip=#2cm
    \rightskip=#3cm
}
{
    \par
}
\fi

\doendnotes{C}
\bigskip
\vfill

\clearpage

\footnotesize

\ifkorrekturansicht
  \lohead{\textsc{register}}
\fi

% theindex-Environment neu definieren ohne reledmac
\makeatletter
\renewenvironment{theindex}{%
  \ifkorrekturansicht
    \section*{\indexname}%
  \else
    \subsubsection*{Index der erwähnten Entitäten}%
  \fi
  \setlength{\parindent}{0pt}%
  \setlength{\parskip}{0pt plus 0.3pt}%
  \let\item\@idxitem
}{%
  \ifkorrekturansicht\clearpage\fi
}
\makeatother

\IfFileExists{\jobname-pw.ind}{\input{\jobname-pw.ind}}{}

% Quellenangabe nur in der Leseansicht
\ifkorrekturansicht\else
% Fallback-Definitionen, falls die .tex-Datei \titel etc. nicht gesetzt hat
\providecommand{\titel}{}
\providecommand{\editorInnen}{}
\providecommand{\dateiname}{\jobname}

\vspace{3cm}

\vfill

\footnotesize
\textsc{Quelle}: \titel. Herausgegeben von {\editorInnen}. In: \emph{Arthur Schnitzler: Briefwechsel mit Autorinnen und Autoren}.
 Digitale Edition, https://schnitzler-briefe.acdh.oeaw.ac.at/{\dateiname}.html (Stand \today)
\fi

\end{document}


      