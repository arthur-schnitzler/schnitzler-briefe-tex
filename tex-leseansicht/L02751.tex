%% latex-leseansicht-vorspann.tex
%% Vorspann für die Leseansicht.
%% Lädt die gemeinsame Datei latex-vorspann.tex mit nicht gesetztem Schalter.

\newif\ifkorrekturansicht
\korrekturansichtfalse

\input{../tex-inputs/latex-vorspann}


\section[Paul Goldmann an Arthur Schnitzler, 13. 10. {[}1895{]}]{L02751 Paul Goldmann an Arthur Schnitzler, 13. 10. [1895]}
\nopagebreak\mylabel{L02751v}
\rehead{ }\normalsize\beginnumbering\briefempfaengerindex{Schnitzler, Arthur@\textsc{Schnitzler, Arthur}!zzzGoldmann, Paul@\emph{von Paul Goldmann}!1895-10-131@{13. 10. [1895]}|(be}
\toendnotes[C]{\smallbreak\pagebreak[2]}
\correspDesc{Versand  durch Paul Goldmann am 13. 10. [1895] in Paris
\newline{}Erhalt  durch Arthur Schnitzler im Zeitraum [14. 10. 1895 – 18. 10. 1895?] in Wien}\toendnotes[C]{\smallbreak}
\Standort{DLA, A:Schnitzler, HS.NZ85.1.3165.}
\physDesc{Brief, 3 Blätter, 11 Seiten, 4371 Zeichen
\newline{}Handschrift: blaue Tinte, deutsche Kurrent
\newline{}Schnitzler: 1) mit Bleistift das Jahr »95« vermerkt  2) mit rotem Buntstift eine seitliche Markierung und sieben Unterstreichungen}\toendnotes[C]{\smallbreak}
\pstart
           {\pb}\textcolor{gray}{\textbf{\textbf{Frankfurter Zeitung\orgindex{Frankfurter Zeitung@Frankfurter Zeitung|pw}}}}\pend
           
\pstart
           \textcolor{gray}{\textbf{(\begin{otherlanguage}{french}Gazette de Francfort\end{otherlanguage}\orgindex{Frankfurter Zeitung@Frankfurter Zeitung|pw}).}}\pend
           
\pstart
           \textcolor{gray}{\textbf{\textbf{\begin{otherlanguage}{french}Fondateur M. L.
                              Sonnemann\pwindex{Sonnemann, Leopold 29.\,10.\,1831 Höchberg – 30.\,10.\,1909 Frankfurt am Main@\textsc{Sonnemann, Leopold} (29.\,10.\,1831 Höchberg – 30.\,10.\,1909 Frankfurt am Main), \emph{Journalist, Herausgeber}|pw}\end{otherlanguage}.}}}\pend
           
\pstart
           \begin{otherlanguage}{french}\textcolor{gray}{\textbf{Journal politique, financier,}}\end{otherlanguage}\pend
           
\pstart
           \begin{otherlanguage}{french}\textcolor{gray}{\textbf{commercial et littéraire.}}\end{otherlanguage}\hfill \textsc{Paris\oindex{Paris@\textbf{Paris}, \emph{Hauptstadt}|pw}}, 13. October.\pend
           
\pstart
           \begin{otherlanguage}{french}\textcolor{gray}{\textbf{\textbf{Paraissant trois fois par jour.}}}\end{otherlanguage}\pend
           
\pstart
           \begin{otherlanguage}{french}\textcolor{gray}{\textbf{\textbf{Bureau à Paris\oindex{Paris@\textbf{Paris}, \emph{Hauptstadt}|pw}}}}\end{otherlanguage}\pend
           
\pstart
           \begin{otherlanguage}{french}\textcolor{gray}{\textbf{\textbf{24. Rue Feydeau\oindex{rue Feydeau@\textbf{rue Feydeau}, \emph{Straße}|pw}.}}}\end{otherlanguage}\pend
           
\pstart\center{}Mein lieber Freund,\pend\vspace{0.5em}
\pstart
           Nochmals innigen Glückwunſch!\pend
           
\pstart
           Jetzt, nachdem ich einige Referate geleſen,{ }ſehe ich \strikeout{\textcolor{gray}{er}} erſt, wie groß Dein Erfolg iſt, was aus Deiner Depeſche nicht klar genug
               hervorging. Wie ich die Sache anſehe, biſt Du jetzt \label{K_L02751-1v}\edtext{lancirt}{\lemma{\textnormal{\emph{lancirt}}}\Cendnote{\textnormal{im Sinne
                  von: in der Öffentlichkeit bekannt}}}\label{K_L02751-1}. Nach dem Wien\oindex{Wien@\textbf{Wien}, \emph{Verwaltungsgebiet}|pw}er Erfolge werden die Berlin\oindex{Berlin@\textbf{Berlin}, \emph{Hauptstadt}|pw}er bald
               mit dem Stücke\pwindex{Schnitzler, Arthur 15.\,5.\,1862 Wien – 21.\,10.\,1931 ebd.@\textsc{Schnitzler, Arthur} (15.\,5.\,1862 Wien – 21.\,10.\,1931 ebd.), \emph{Schriftsteller, Mediziner}!Liebelei. Schauspiel in drei Akten@\strich\emph{Liebelei. Schauspiel in drei Akten}|pwv} herauskommen.
               Dort wird es einen nicht minder großen Erfolg haben und eine noch intelligentere
               Kritik finden (\label{K_L02751-2v}\edtext{\textsc{Mauthner\pwindex{Mauthner, Fritz 20.\,11.\,1849 Hořice – 29.\,6.\,1923 Meersburg@\textsc{Mauthner, Fritz} (20.\,11.\,1849 Hořice – 29.\,6.\,1923 Meersburg), \emph{Schriftsteller, Journalist, Philosoph}|pw}} im »Tageblatt\orgindex{Berliner Tageblatt@Berliner Tageblatt|pw}«}{\lemma{\textnormal{\emph{Mauthner im »Tageblatt«}}}\Cendnote{\textnormal{Mauthner\pwindex{Mauthner, Fritz 20.\,11.\,1849 Hořice – 29.\,6.\,1923 Meersburg@\textsc{Mauthner, Fritz} (20.\,11.\,1849 Hořice – 29.\,6.\,1923 Meersburg), \emph{Schriftsteller, Journalist, Philosoph}|pwk} sollte dann tatsächlich schreiben: Fr. M.\pwindex{Mauthner, Fritz 20.\,11.\,1849 Hořice – 29.\,6.\,1923 Meersburg@\textsc{Mauthner, Fritz} (20.\,11.\,1849 Hořice – 29.\,6.\,1923 Meersburg), \emph{Schriftsteller, Journalist, Philosoph}|pwkv} [ = Fritz Mauthner\pwindex{Mauthner, Fritz 20.\,11.\,1849 Hořice – 29.\,6.\,1923 Meersburg@\textsc{Mauthner, Fritz} (20.\,11.\,1849 Hořice – 29.\,6.\,1923 Meersburg), \emph{Schriftsteller, Journalist, Philosoph}|pwk}]: \emph{Deutsches Theater}\pwindex{Mauthner, Fritz 20.\,11.\,1849 Hořice – 29.\,6.\,1923 Meersburg@\textsc{Mauthner, Fritz} (20.\,11.\,1849 Hořice – 29.\,6.\,1923 Meersburg), \emph{Schriftsteller, Journalist, Philosoph}!Deutsches Theater@\strich\emph{Deutsches Theater}|pwk}. In: \emph{Berliner Tageblatt}\pwindex{Berliner Tageblatt@\emph{Berliner Tageblatt}|pwk}, Jg. 25, Nr. 64, 5. 2. 1896, Morgen-Ausgabe, S. 2–3; Fritz Mauthner\pwindex{Mauthner, Fritz 20.\,11.\,1849 Hořice – 29.\,6.\,1923 Meersburg@\textsc{Mauthner, Fritz} (20.\,11.\,1849 Hořice – 29.\,6.\,1923 Meersburg), \emph{Schriftsteller, Journalist, Philosoph}|pwk}: \emph{Der zerbrochene Krug im Deutschen Theater}\pwindex{Mauthner, Fritz 20.\,11.\,1849 Hořice – 29.\,6.\,1923 Meersburg@\textsc{Mauthner, Fritz} (20.\,11.\,1849 Hořice – 29.\,6.\,1923 Meersburg), \emph{Schriftsteller, Journalist, Philosoph}!zerbrochene Krug im Deutschen Theater@\strich\emph{Der zerbrochene Krug im Deutschen Theater}|pwk}. In: \emph{Berliner Tageblatt}\pwindex{Berliner Tageblatt@\emph{Berliner Tageblatt}|pwk}, Jg. 25, Nr. 65, 5. 2. 1896, Abend-Ausgabe, S. 1–2.}}}\label{K_L02751-2}).
               Dann wird {\pb}es über alle deutſch\oindex{Deutschland@\textbf{Deutschland}|pwv}en Bühnen gehen. Wenn Du ruhig{ }ſo
               weiter arbeiteſt – und ich weiß, Du wirſt es thun – kann am Ende ein deutſcher \textsc{Emile Augier\pwindex{Augier, Émile 17.\,9.\,1820 Valence – 25.\,10.\,1889 Croissy-sur-Seine@\textsc{Augier, Émile} (17.\,9.\,1820 Valence – 25.\,10.\,1889 Croissy-sur-Seine), \emph{Schriftsteller}|pw}} daraus werden. Der erſte entſcheidende Schritt auf dieſem Wege iſt gethan, und
               ich bin recht glücklich darüber, daß Dich gleich zu Anfang der Erfolg \strikeout{\textcolor{gray}{in die Hand}} an der Hand nimmt; das iſt ein guter Führer. Wenn ich übrigens »\textsc{Émile Augier\pwindex{Augier, Émile 17.\,9.\,1820 Valence – 25.\,10.\,1889 Croissy-sur-Seine@\textsc{Augier, Émile} (17.\,9.\,1820 Valence – 25.\,10.\,1889 Croissy-sur-Seine), \emph{Schriftsteller}|pw}}«{ }ſage,{ }ſo gilt dies nur einſtweilen, und ich behalte mir vor, im Laufe der
               Zeit, je nachdem die Dinge{ }ſich entwickeln, {\pb}noch
               viel unbeſcheidener zu werden. Immerhin bedenke nur: In{ }ſo jungen Jahren am erſten
               deutſchen Theater\orgindex{Burgtheater@Burgtheater|pwv} mit dem
               zweiten Stücke\pwindex{Schnitzler, Arthur 15.\,5.\,1862 Wien – 21.\,10.\,1931 ebd.@\textsc{Schnitzler, Arthur} (15.\,5.\,1862 Wien – 21.\,10.\,1931 ebd.), \emph{Schriftsteller, Mediziner}!Liebelei. Schauspiel in drei Akten@\strich\emph{Liebelei. Schauspiel in drei Akten}|pwv} ein von allen
                  \strikeout{ernſtz\textcolor{gray}{u}} ernſtzunehmenden Leuten laut anerkannter Erfolg! Das iſt etwas, was Du in der
               deutſchen Bühnengeſchichte{ }ſelten finden dürfteſt. Es{ }ſcheint wirklich, daß Du zu{ }ſchönen Hoffnungen für die Zukunft berechtigſt, wie \label{K_L02751-3v}\edtext{einer\pwindex{Kalbeck, Max 4.\,1.\,1850 Breslau – 4.\,5.\,1921 Wien@\textsc{Kalbeck, Max} (4.\,1.\,1850 Breslau – 4.\,5.\,1921 Wien), \emph{Journalist}|pwv} der weiſen Männer}{\lemma{\textnormal{\emph{einer der weisen Männer}}}\Cendnote{\textnormal{Von Max
                     Kalbeck\pwindex{Kalbeck, Max 4.\,1.\,1850 Breslau – 4.\,5.\,1921 Wien@\textsc{Kalbeck, Max} (4.\,1.\,1850 Breslau – 4.\,5.\,1921 Wien), \emph{Journalist}|pwk} erschienen ein Feuilleton\pwindex{Kalbeck, Max 4.\,1.\,1850 Breslau – 4.\,5.\,1921 Wien@\textsc{Kalbeck, Max} (4.\,1.\,1850 Breslau – 4.\,5.\,1921 Wien), \emph{Journalist}!Burgtheater. »Liebelei«, Schauspiel in drei Acten von Arthur Schnitzler. – »Rechte der Seele«, Schauspiel in einem Acte von Guiseppe Giacosa; deutsch von Otto Eisenschitz@\strich\emph{Burgtheater. »Liebelei«, Schauspiel in drei Acten von Arthur Schnitzler. – »Rechte der Seele«, Schauspiel in einem Acte von Guiseppe Giacosa; deutsch von Otto Eisenschitz}|pwkv} und eine Nachtkritik\pwindex{Kalbeck, Max 4.\,1.\,1850 Breslau – 4.\,5.\,1921 Wien@\textsc{Kalbeck, Max} (4.\,1.\,1850 Breslau – 4.\,5.\,1921 Wien), \emph{Journalist}!Theater, Kunst und Literatur. Burgtheater [Liebelei, Rechte der Seele]@\strich\emph{Theater, Kunst und Literatur. Burgtheater [Liebelei, Rechte der Seele]}|pwkv}, wobei sich die erwähnte Aussage in der Nachtkritik\pwindex{Kalbeck, Max 4.\,1.\,1850 Breslau – 4.\,5.\,1921 Wien@\textsc{Kalbeck, Max} (4.\,1.\,1850 Breslau – 4.\,5.\,1921 Wien), \emph{Journalist}!Theater, Kunst und Literatur. Burgtheater [Liebelei, Rechte der Seele]@\strich\emph{Theater, Kunst und Literatur. Burgtheater [Liebelei, Rechte der Seele]}|pwkv} findet. Max Kalbeck\pwindex{Kalbeck, Max 4.\,1.\,1850 Breslau – 4.\,5.\,1921 Wien@\textsc{Kalbeck, Max} (4.\,1.\,1850 Breslau – 4.\,5.\,1921 Wien), \emph{Journalist}|pwk}: \emph{Burgtheater. »Liebelei«, Schauspiel in drei Acten von Arthur
                        Schnitzler. – »Rechte der Seele«, Schauspiel in einem Acte von Guiseppe
                        Giacosa; deutsch von Otto Eisenschitz}\pwindex{Kalbeck, Max 4.\,1.\,1850 Breslau – 4.\,5.\,1921 Wien@\textsc{Kalbeck, Max} (4.\,1.\,1850 Breslau – 4.\,5.\,1921 Wien), \emph{Journalist}!Burgtheater. »Liebelei«, Schauspiel in drei Acten von Arthur Schnitzler. – »Rechte der Seele«, Schauspiel in einem Acte von Guiseppe Giacosa; deutsch von Otto Eisenschitz@\strich\emph{Burgtheater. »Liebelei«, Schauspiel in drei Acten von Arthur Schnitzler. – »Rechte der Seele«, Schauspiel in einem Acte von Guiseppe Giacosa; deutsch von Otto Eisenschitz}|pwk}. In: \emph{Neues Wiener Tagblatt}\pwindex{Neues Wiener Tagblatt@\emph{Neues Wiener Tagblatt}|pwk}, Jg. 29, Nr. 279, 11. 10. 1895, S. 1–3; M. K.\pwindex{Kalbeck, Max 4.\,1.\,1850 Breslau – 4.\,5.\,1921 Wien@\textsc{Kalbeck, Max} (4.\,1.\,1850 Breslau – 4.\,5.\,1921 Wien), \emph{Journalist}|pwk} [ = Max Kalbeck\pwindex{Kalbeck, Max 4.\,1.\,1850 Breslau – 4.\,5.\,1921 Wien@\textsc{Kalbeck, Max} (4.\,1.\,1850 Breslau – 4.\,5.\,1921 Wien), \emph{Journalist}|pwk}]: \emph{Theater,
                        Kunst und Literatur. Burgtheater}\pwindex{Kalbeck, Max 4.\,1.\,1850 Breslau – 4.\,5.\,1921 Wien@\textsc{Kalbeck, Max} (4.\,1.\,1850 Breslau – 4.\,5.\,1921 Wien), \emph{Journalist}!Theater, Kunst und Literatur. Burgtheater [Liebelei, Rechte der Seele]@\strich\emph{Theater, Kunst und Literatur. Burgtheater [Liebelei, Rechte der Seele]}|pwk}. In: \emph{Neues Wiener Tagblatt}\pwindex{Neues Wiener Tagblatt@\emph{Neues Wiener Tagblatt}|pwk}, Jg. 29, Nr. 278, 10. 10. 1895, S. 7.}}}\label{K_L02751-3}{ }ſich ausdrückte, die über Dein Stück\pwindex{Schnitzler, Arthur 15.\,5.\,1862 Wien – 21.\,10.\,1931 ebd.@\textsc{Schnitzler, Arthur} (15.\,5.\,1862 Wien – 21.\,10.\,1931 ebd.), \emph{Schriftsteller, Mediziner}!Liebelei. Schauspiel in drei Akten@\strich\emph{Liebelei. Schauspiel in drei Akten}|pwv} geſchrieben haben.\pend
           
\pstart
           {\pb}Ich habe geleſen die Referate\pwindex{Theater- und Kunstnachrichten. [Burgtheater] [Liebelei, Rechte der Seele]@\emph{Theater- und Kunstnachrichten. [Burgtheater] [Liebelei, Rechte der Seele]}|pwv}\pwindex{Kalbeck, Max 4.\,1.\,1850 Breslau – 4.\,5.\,1921 Wien@\textsc{Kalbeck, Max} (4.\,1.\,1850 Breslau – 4.\,5.\,1921 Wien), \emph{Journalist}!Theater, Kunst und Literatur. Burgtheater [Liebelei, Rechte der Seele]@\strich\emph{Theater, Kunst und Literatur. Burgtheater [Liebelei, Rechte der Seele]}|pwv}\pwindex{Kalbeck, Max 4.\,1.\,1850 Breslau – 4.\,5.\,1921 Wien@\textsc{Kalbeck, Max} (4.\,1.\,1850 Breslau – 4.\,5.\,1921 Wien), \emph{Journalist}!Burgtheater. »Liebelei«, Schauspiel in drei Acten von Arthur Schnitzler. – »Rechte der Seele«, Schauspiel in einem Acte von Guiseppe Giacosa; deutsch von Otto Eisenschitz@\strich\emph{Burgtheater. »Liebelei«, Schauspiel in drei Acten von Arthur Schnitzler. – »Rechte der Seele«, Schauspiel in einem Acte von Guiseppe Giacosa; deutsch von Otto Eisenschitz}|pwv}\pwindex{Schönthan-Pernwald, Paul von 19.\,3.\,1853 Wien – 4.\,8.\,1905 ebd.@\textsc{Schönthan-Pernwald, Paul von} (19.\,3.\,1853 Wien – 4.\,8.\,1905 ebd.), \emph{Schriftsteller, Journalist}!Theater, Kunst und Literatur. (Burgtheater.) [Liebelei]@\strich\emph{Theater, Kunst und Literatur. (Burgtheater.) [Liebelei]}|pwv}\pwindex{David, Jakob Julius 6.\,2.\,1859 Hranice – 20.\,11.\,1906 Wien@\textsc{David, Jakob Julius} (6.\,2.\,1859 Hranice – 20.\,11.\,1906 Wien), \emph{Schriftsteller, Journalist}!Theater und Kunst. (Burgtheater.) [Liebelei, Rechte der Seele]@\strich\emph{Theater und Kunst. (Burgtheater.) [Liebelei, Rechte der Seele]}|pwv}\pwindex{Granichstaedten, Emil 8.\,7.\,1847 Wien – 2.\,7.\,1904 Berlin@\textsc{Granichstaedten, Emil} (8.\,7.\,1847 Wien – 2.\,7.\,1904 Berlin), \emph{Journalist, Rechtswissenschaftler}!Burgtheater. Zwei Schauspiele: »Rechte der Seele« von Giuseppe Giacosa. – »Liebelei« von Arthur Schnitzler@\strich\emph{Burgtheater. Zwei Schauspiele: »Rechte der Seele« von Giuseppe Giacosa. – »Liebelei« von Arthur Schnitzler}|pwv}\pwindex{Wiener Brief. [Liebelei, Rechte der Seele]@\emph{Wiener Brief. [Liebelei, Rechte der Seele]}|pwv}\pwindex{Aus Wien, 9. Oktober] [Liebelei]@\emph{[Aus Wien, 9. Oktober] [Liebelei]}|pwv}\pwindex{Liebelei« [Telegramm zur Uraufführung]@\emph{»Liebelei« [Telegramm zur Uraufführung]}|pwv}\pwindex{Wien, 9. Oktober] [Liebelei]@\emph{[Wien, 9. Oktober] [Liebelei]}|pwv} von: \label{K_L02751-4v}\edtext{\textsc{Speidel\pwindex{Speidel, Ludwig 11.\,4.\,1830 Ulm – 3.\,2.\,1906 Wien@\textsc{Speidel, Ludwig} (11.\,4.\,1830 Ulm – 3.\,2.\,1906 Wien), \emph{Journalist, Kritiker}|pw}\pwindex{Theater- und Kunstnachrichten. [Burgtheater] [Liebelei, Rechte der Seele]@\emph{Theater- und Kunstnachrichten. [Burgtheater] [Liebelei, Rechte der Seele]}|pwv}}}{\lemma{\textnormal{\emph{Speidel}}}\Cendnote{\textnormal{[Ludwig Speidel\pwindex{Speidel, Ludwig 11.\,4.\,1830 Ulm – 3.\,2.\,1906 Wien@\textsc{Speidel, Ludwig} (11.\,4.\,1830 Ulm – 3.\,2.\,1906 Wien), \emph{Journalist, Kritiker}|pwk}]: \emph{Theater- und Kunstnachrichten. [Burgtheater]}\pwindex{Theater- und Kunstnachrichten. [Burgtheater] [Liebelei, Rechte der Seele]@\emph{Theater- und Kunstnachrichten. [Burgtheater] [Liebelei, Rechte der Seele]}|pwk}. In: \emph{Neue Freie Presse}\pwindex{Neue Freie Presse@\emph{Neue Freie Presse}|pwk}, Nr. 11.181, 10. 10. 1895, S. 7. Ein weiteres Feuilleton\pwindex{Speidel, Ludwig 11.\,4.\,1830 Ulm – 3.\,2.\,1906 Wien@\textsc{Speidel, Ludwig} (11.\,4.\,1830 Ulm – 3.\,2.\,1906 Wien), \emph{Journalist, Kritiker}!Burgtheater. (»Liebelei«, Schauspiel in drei Aufzügen von Arthur Schnitzler. – »Rechte der Seele«, Schauspiel in einem Act von Giuseppe Giacosa, deutsch von Otto Eisenschitz.)@\strich\emph{Burgtheater. (»Liebelei«, Schauspiel in drei Aufzügen von Arthur Schnitzler. – »Rechte der Seele«, Schauspiel in einem Act von Giuseppe Giacosa, deutsch von Otto Eisenschitz.)}|pwkv} erschien am Tag dieses Briefes und war Goldmann\pwindex{Goldmann, Paul 31.\,1.\,1865 Breslau – 25.\,9.\,1935 Wien@\textsc{Goldmann, Paul} (31.\,1.\,1865 Breslau – 25.\,9.\,1935 Wien), \emph{Schriftsteller, Journalist}|pwk} zu diesem Zeitpunkt noch unbekannt: L. Sp.\pwindex{Speidel, Ludwig 11.\,4.\,1830 Ulm – 3.\,2.\,1906 Wien@\textsc{Speidel, Ludwig} (11.\,4.\,1830 Ulm – 3.\,2.\,1906 Wien), \emph{Journalist, Kritiker}|pwkv} [ = Ludwig Speidel\pwindex{Speidel, Ludwig 11.\,4.\,1830 Ulm – 3.\,2.\,1906 Wien@\textsc{Speidel, Ludwig} (11.\,4.\,1830 Ulm – 3.\,2.\,1906 Wien), \emph{Journalist, Kritiker}|pwk}]: \emph{Burgtheater. (»Liebelei«, Schauspiel in drei Aufzügen von
                        Arthur Schnitzler. – »Rechte der Seele«, Schauspiel in einem Act von
                        Giuseppe Giacosa, deutsch von Otto Eisenschitz)}\pwindex{Speidel, Ludwig 11.\,4.\,1830 Ulm – 3.\,2.\,1906 Wien@\textsc{Speidel, Ludwig} (11.\,4.\,1830 Ulm – 3.\,2.\,1906 Wien), \emph{Journalist, Kritiker}!Burgtheater. (»Liebelei«, Schauspiel in drei Aufzügen von Arthur Schnitzler. – »Rechte der Seele«, Schauspiel in einem Act von Giuseppe Giacosa, deutsch von Otto Eisenschitz.)@\strich\emph{Burgtheater. (»Liebelei«, Schauspiel in drei Aufzügen von Arthur Schnitzler. – »Rechte der Seele«, Schauspiel in einem Act von Giuseppe Giacosa, deutsch von Otto Eisenschitz.)}|pwk}. In: \emph{Neue Freie Presse}\pwindex{Neue Freie Presse@\emph{Neue Freie Presse}|pwk}, Nr. 11.184, 13. 10. 1895, Morgenblatt, S. 1–3.}}}\label{K_L02751-4} (prachtvoll), \textsc{Kalbeck\pwindex{Kalbeck, Max 4.\,1.\,1850 Breslau – 4.\,5.\,1921 Wien@\textsc{Kalbeck, Max} (4.\,1.\,1850 Breslau – 4.\,5.\,1921 Wien), \emph{Journalist}|pw}\pwindex{Kalbeck, Max 4.\,1.\,1850 Breslau – 4.\,5.\,1921 Wien@\textsc{Kalbeck, Max} (4.\,1.\,1850 Breslau – 4.\,5.\,1921 Wien), \emph{Journalist}!Theater, Kunst und Literatur. Burgtheater [Liebelei, Rechte der Seele]@\strich\emph{Theater, Kunst und Literatur. Burgtheater [Liebelei, Rechte der Seele]}|pwv}\pwindex{Kalbeck, Max 4.\,1.\,1850 Breslau – 4.\,5.\,1921 Wien@\textsc{Kalbeck, Max} (4.\,1.\,1850 Breslau – 4.\,5.\,1921 Wien), \emph{Journalist}!Burgtheater. »Liebelei«, Schauspiel in drei Acten von Arthur Schnitzler. – »Rechte der Seele«, Schauspiel in einem Acte von Guiseppe Giacosa; deutsch von Otto Eisenschitz@\strich\emph{Burgtheater. »Liebelei«, Schauspiel in drei Acten von Arthur Schnitzler. – »Rechte der Seele«, Schauspiel in einem Acte von Guiseppe Giacosa; deutsch von Otto Eisenschitz}|pwv}} (die erſten sympathiſchen Zeilen, die ich von dem Manne\pwindex{Kalbeck, Max 4.\,1.\,1850 Breslau – 4.\,5.\,1921 Wien@\textsc{Kalbeck, Max} (4.\,1.\,1850 Breslau – 4.\,5.\,1921 Wien), \emph{Journalist}|pwv} leſe), \label{K_L02751-5v}\edtext{\textsc{Schoenthan\pwindex{Schönthan-Pernwald, Paul von 19.\,3.\,1853 Wien – 4.\,8.\,1905 ebd.@\textsc{Schönthan-Pernwald, Paul von} (19.\,3.\,1853 Wien – 4.\,8.\,1905 ebd.), \emph{Schriftsteller, Journalist}|pw}\pwindex{Schönthan-Pernwald, Paul von 19.\,3.\,1853 Wien – 4.\,8.\,1905 ebd.@\textsc{Schönthan-Pernwald, Paul von} (19.\,3.\,1853 Wien – 4.\,8.\,1905 ebd.), \emph{Schriftsteller, Journalist}!Theater, Kunst und Literatur. (Burgtheater.) [Liebelei]@\strich\emph{Theater, Kunst und Literatur. (Burgtheater.) [Liebelei]}|pwv}}}{\lemma{\textnormal{\emph{Schoenthan}}}\Cendnote{\textnormal{p. v. s.\pwindex{Schönthan-Pernwald, Paul von 19.\,3.\,1853 Wien – 4.\,8.\,1905 ebd.@\textsc{Schönthan-Pernwald, Paul von} (19.\,3.\,1853 Wien – 4.\,8.\,1905 ebd.), \emph{Schriftsteller, Journalist}|pwkv} [ = Paul von Schönthan-Pernwald\pwindex{Schönthan-Pernwald, Paul von 19.\,3.\,1853 Wien – 4.\,8.\,1905 ebd.@\textsc{Schönthan-Pernwald, Paul von} (19.\,3.\,1853 Wien – 4.\,8.\,1905 ebd.), \emph{Schriftsteller, Journalist}|pwk}]: \emph{Theater, Kunst und Literatur.
                        (Burgtheater)}\pwindex{Schönthan-Pernwald, Paul von 19.\,3.\,1853 Wien – 4.\,8.\,1905 ebd.@\textsc{Schönthan-Pernwald, Paul von} (19.\,3.\,1853 Wien – 4.\,8.\,1905 ebd.), \emph{Schriftsteller, Journalist}!Theater, Kunst und Literatur. (Burgtheater.) [Liebelei]@\strich\emph{Theater, Kunst und Literatur. (Burgtheater.) [Liebelei]}|pwk}. In: \emph{Wiener
                        Tagblatt}\pwindex{Wiener Tagblatt@\emph{Wiener Tagblatt}|pwk}, Jg. 45, Nr. 278, 10. 10. 1895,
                     S. 5–6.}}}\label{K_L02751-5} (der vor Bühnendichter-Neid zerſpringt); ferner das
                  \label{K_L02751-6v}\edtext{Referat\pwindex{David, Jakob Julius 6.\,2.\,1859 Hranice – 20.\,11.\,1906 Wien@\textsc{David, Jakob Julius} (6.\,2.\,1859 Hranice – 20.\,11.\,1906 Wien), \emph{Schriftsteller, Journalist}!Theater und Kunst. (Burgtheater.) [Liebelei, Rechte der Seele]@\strich\emph{Theater und Kunst. (Burgtheater.) [Liebelei, Rechte der Seele]}|pwv} des »Wiener Journal\pwindex{Wiener Tagblatt@\emph{Wiener Tagblatt}|pw}}{\lemma{\textnormal{\emph{Referat … Journal}}}\Cendnote{\textnormal{–v–\pwindex{David, Jakob Julius 6.\,2.\,1859 Hranice – 20.\,11.\,1906 Wien@\textsc{David, Jakob Julius} (6.\,2.\,1859 Hranice – 20.\,11.\,1906 Wien), \emph{Schriftsteller, Journalist}|pwkv} [ = Jakob Julius David\pwindex{David, Jakob Julius 6.\,2.\,1859 Hranice – 20.\,11.\,1906 Wien@\textsc{David, Jakob Julius} (6.\,2.\,1859 Hranice – 20.\,11.\,1906 Wien), \emph{Schriftsteller, Journalist}|pwk}]: \emph{Theater und Kunst. (Burgtheater)}\pwindex{David, Jakob Julius 6.\,2.\,1859 Hranice – 20.\,11.\,1906 Wien@\textsc{David, Jakob Julius} (6.\,2.\,1859 Hranice – 20.\,11.\,1906 Wien), \emph{Schriftsteller, Journalist}!Theater und Kunst. (Burgtheater.) [Liebelei, Rechte der Seele]@\strich\emph{Theater und Kunst. (Burgtheater.) [Liebelei, Rechte der Seele]}|pwk}. In: \emph{Neues Wiener Journal}\pwindex{Neues Wiener Journal@\emph{Neues Wiener Journal}|pwk}, Jg. 3, Nr. 704, 10. 10. 1895, S. 5.}}}\label{K_L02751-6}« (verſtändnißlos,
               aber mit Einzelheiten, die ausſöhnen), endlich \label{K_L02751-7v}\edtext{\textsc{Granichstaedten\pwindex{Granichstaedten, Emil 8.\,7.\,1847 Wien – 2.\,7.\,1904 Berlin@\textsc{Granichstaedten, Emil} (8.\,7.\,1847 Wien – 2.\,7.\,1904 Berlin), \emph{Journalist, Rechtswissenschaftler}|pw}\pwindex{Granichstaedten, Emil 8.\,7.\,1847 Wien – 2.\,7.\,1904 Berlin@\textsc{Granichstaedten, Emil} (8.\,7.\,1847 Wien – 2.\,7.\,1904 Berlin), \emph{Journalist, Rechtswissenschaftler}!Burgtheater. Zwei Schauspiele: »Rechte der Seele« von Giuseppe Giacosa. – »Liebelei« von Arthur Schnitzler@\strich\emph{Burgtheater. Zwei Schauspiele: »Rechte der Seele« von Giuseppe Giacosa. – »Liebelei« von Arthur Schnitzler}|pwv}}}{\lemma{\textnormal{\emph{Granichstaedten}}}\Cendnote{\textnormal{Emil Granichstaedten\pwindex{Granichstaedten, Emil 8.\,7.\,1847 Wien – 2.\,7.\,1904 Berlin@\textsc{Granichstaedten, Emil} (8.\,7.\,1847 Wien – 2.\,7.\,1904 Berlin), \emph{Journalist, Rechtswissenschaftler}|pwk}: \emph{Feuilleton. Burgtheater}\pwindex{Granichstaedten, Emil 8.\,7.\,1847 Wien – 2.\,7.\,1904 Berlin@\textsc{Granichstaedten, Emil} (8.\,7.\,1847 Wien – 2.\,7.\,1904 Berlin), \emph{Journalist, Rechtswissenschaftler}!Burgtheater. Zwei Schauspiele: »Rechte der Seele« von Giuseppe Giacosa. – »Liebelei« von Arthur Schnitzler@\strich\emph{Burgtheater. Zwei Schauspiele: »Rechte der Seele« von Giuseppe Giacosa. – »Liebelei« von Arthur Schnitzler}|pwk}. In: \emph{Die Presse}\pwindex{Presse@\emph{Die Presse}|pwk}, Jg. 48, Nr. 279, 11. 10. 1895, S. 1–2.}}}\label{K_L02751-7}, das widerliche Thier\pwindex{Granichstaedten, Emil 8.\,7.\,1847 Wien – 2.\,7.\,1904 Berlin@\textsc{Granichstaedten, Emil} (8.\,7.\,1847 Wien – 2.\,7.\,1904 Berlin), \emph{Journalist, Rechtswissenschaftler}|pwv} (Ohrfeigen!!!). \label{K_L02751-8v}\edtext{\textsc{Uhl\pwindex{Uhl, Friedrich 14.\,5.\,1825 Cieszyn – 20.\,1.\,1906 Mondsee@\textsc{Uhl, Friedrich} (14.\,5.\,1825 Cieszyn – 20.\,1.\,1906 Mondsee), \emph{Journalist}|pw}\pwindex{Wiener Brief. [Liebelei, Rechte der Seele]@\emph{Wiener Brief. [Liebelei, Rechte der Seele]}|pwv}\strikeout{\textcolor{gray}{×}}} in der »Frankfurter Zeitung\pwindex{Frankfurter Zeitung@\emph{Frankfurter Zeitung}|pw}}{\lemma{\textnormal{\emph{Uhl … Zeitung}}}\Cendnote{\textnormal{[Friedrich Uhl\pwindex{Uhl, Friedrich 14.\,5.\,1825 Cieszyn – 20.\,1.\,1906 Mondsee@\textsc{Uhl, Friedrich} (14.\,5.\,1825 Cieszyn – 20.\,1.\,1906 Mondsee), \emph{Journalist}|pwk}]: \emph{Wiener Brief}\pwindex{Wiener Brief. [Liebelei, Rechte der Seele]@\emph{Wiener Brief. [Liebelei, Rechte der Seele]}|pwk}. In: \emph{Frankfurter Zeitung}\pwindex{Frankfurter Zeitung@\emph{Frankfurter Zeitung}|pwk}, Jg. 40, Nr. 282, 11. 10. 1895,
                     Abendblatt, S. 1. }}}\label{K_L02751-8}« hätte wärmer und ausführlicher{ }ſein können;
               ich vermuthe, daß es ihn {\pb}verſtimmt, weil die
               Officiellen (\textsc{Speidel\pwindex{Speidel, Ludwig 11.\,4.\,1830 Ulm – 3.\,2.\,1906 Wien@\textsc{Speidel, Ludwig} (11.\,4.\,1830 Ulm – 3.\,2.\,1906 Wien), \emph{Journalist, Kritiker}|pw} etc.}) Dich loben. Auch iſt er
               wohl von denen, die Jemanden fördern, – bis er einen Erfolg hat, die aber{ }ſofort von
               dem Erfolge{ }ſelbſt unſympathiſch berührt werden. Eine echte Oppoſitions-Natur\pwindex{Uhl, Friedrich 14.\,5.\,1825 Cieszyn – 20.\,1.\,1906 Mondsee@\textsc{Uhl, Friedrich} (14.\,5.\,1825 Cieszyn – 20.\,1.\,1906 Mondsee), \emph{Journalist}|pwv} mit einem Worte. In \strikeout{B\textcolor{gray}{e}}{ }Berlin\oindex{Berlin@\textbf{Berlin}, \emph{Hauptstadt}|pw}er Blättern las ich das kurze, aber{ }ſehr
               freundliche \label{K_L02751-9v}\edtext{Telegramm\pwindex{Aus Wien, 9. Oktober] [Liebelei]@\emph{[Aus Wien, 9. Oktober] [Liebelei]}|pwv} des »Tageblatt\pwindex{Berliner Tageblatt@\emph{Berliner Tageblatt}|pw}}{\lemma{\textnormal{\emph{Telegramm des »Tageblatt}}}\Cendnote{\textnormal{[O. V.]: \emph{[Aus Wien, 9. Oktober]}\pwindex{Aus Wien, 9. Oktober] [Liebelei]@\emph{[Aus Wien, 9. Oktober] [Liebelei]}|pwk}. In:
                        \emph{Berliner Tageblatt}\pwindex{Berliner Tageblatt@\emph{Berliner Tageblatt}|pwk}, Jg. 24, Nr. 516,
                        10. 10. 1895, Abend-Ausgabe,
                  S. 3.}}}\label{K_L02751-9}«, das{ }ſehr warme \label{K_L02751-10v}\edtext{Telegramm\pwindex{Liebelei« [Telegramm zur Uraufführung]@\emph{»Liebelei« [Telegramm zur Uraufführung]}|pwv} des »Lokalanzeiger\pwindex{Berliner Lokal-Anzeiger@\emph{Berliner Lokal-Anzeiger}|pw}}{\lemma{\textnormal{\emph{Telegramm des »Lokalanzeiger}}}\Cendnote{\textnormal{»›Liebelei\pwindex{Schnitzler, Arthur 15.\,5.\,1862 Wien – 21.\,10.\,1931 ebd.@\textsc{Schnitzler, Arthur} (15.\,5.\,1862 Wien – 21.\,10.\,1931 ebd.), \emph{Schriftsteller, Mediziner}!Liebelei. Schauspiel in drei Akten@\strich\emph{Liebelei. Schauspiel in drei Akten}|pw}‹, ein Drama eines jungen Wien\oindex{Wien@\textbf{Wien}, \emph{Verwaltungsgebiet}|pw}er Schriftstellers, ist gestern (Mittwoch){ }Abend im Wiener Burgtheater\orgindex{Burgtheater@Burgtheater|pw}
                        zum ersten Male aufgeführt worden; wir erhalten darüber folgendes \so{Privat-Telegramm}:\textbf{Wien}\oindex{Wien@\textbf{Wien}, \emph{Verwaltungsgebiet}|pw}, 9. October, 11 Uhr 50 Min. Abends (Von unserem
                        \textsc{\textcolor{gray}{\so{n}}\so{.a.}}\so{-Correspondenten.}) { / }Das bürgerliche Drama ›Liebelei\pwindex{Schnitzler, Arthur 15.\,5.\,1862 Wien – 21.\,10.\,1931 ebd.@\textsc{Schnitzler, Arthur} (15.\,5.\,1862 Wien – 21.\,10.\,1931 ebd.), \emph{Schriftsteller, Mediziner}!Liebelei. Schauspiel in drei Akten@\strich\emph{Liebelei. Schauspiel in drei Akten}|pw}‹ von Arthur Schnitzler hatte heute im Burgtheater\orgindex{Burgtheater@Burgtheater|pw} einen
                        bedeutenden Erfolg. Der Verfasser wurde nach jedem Akt wiederholt gerufen, obwohl in dem
                           Stück\pwindex{Schnitzler, Arthur 15.\,5.\,1862 Wien – 21.\,10.\,1931 ebd.@\textsc{Schnitzler, Arthur} (15.\,5.\,1862 Wien – 21.\,10.\,1931 ebd.), \emph{Schriftsteller, Mediziner}!Liebelei. Schauspiel in drei Akten@\strich\emph{Liebelei. Schauspiel in drei Akten}|pwv} sociale
                        Verhältnisse behandelt werden, die auf dem Hoftheater\orgindex{Burgtheater@Burgtheater|pwv} sonst Befremden erregen. Das
                        Bürgermädchen, das an einer Liebelei zu Grunde geht, wurde von der Sandrock\pwindex{Sandrock, Adele 19.\,8.\,1863 Rotterdam – 30.\,8.\,1937 Berlin@\textsc{Sandrock, Adele} (19.\,8.\,1863 Rotterdam – 30.\,8.\,1937 Berlin), \emph{Schauspielerin}|pw} mit tragischem Nachdruck
                        gespielt, ergreifend war auch Sonnenthal\pwindex{Sonnenthal, Adolf von 21.\,12.\,1834 Budapest – 4.\,4.\,1909 Prag@\textsc{Sonnenthal, Adolf von} (21.\,12.\,1834 Budapest – 4.\,4.\,1909 Prag), \emph{Schauspieler}|pw} als ihr Vater.« (\emph{Berliner Lokal-Anzeiger}\pwindex{Berliner Lokal-Anzeiger@\emph{Berliner Lokal-Anzeiger}|pwk}, Jg. 13, Nr. 475,
                        10. 10. 1895, Morgenblatt, 1. Ausgabe,
                     S. 3.)}}}\label{K_L02751-10}« und {\pb}das
               blödſinnig-freche \label{K_L02751-11v}\edtext{Telegramm\pwindex{Wien, 9. Oktober] [Liebelei]@\emph{[Wien, 9. Oktober] [Liebelei]}|pwv} des »Kleinen Journal\pwindex{Kleine Journal@\emph{Das Kleine Journal}|pw}«}{\lemma{\textnormal{\emph{Telegramm … Journal«}}}\Cendnote{\textnormal{[Julius Konried\pwindex{Konried, Julius 27.\,11.\,1853 Wien – 13.\,1.\,1927 ebd.@\textsc{Konried, Julius} (27.\,11.\,1853 Wien – 13.\,1.\,1927 ebd.), \emph{Journalist}|pwk}]: \emph{[Wien, 9. Oktober]}\pwindex{Wien, 9. Oktober] [Liebelei]@\emph{[Wien, 9. Oktober] [Liebelei]}|pwk}. In: \emph{Das Kleine Journal}\pwindex{Kleine Journal@\emph{Das Kleine Journal}|pwk}, Jg. 17, Nr. 278, 10. 10. 1895, S. [4]. Darin ist zu lesen: »Der
                     eifrigste Anhänger der Hermann
                     Bahr\pwindex{Bahr, Hermann 19.\,7.\,1863 Linz – 15.\,1.\,1934 München@\textsc{Bahr, Hermann} (19.\,7.\,1863 Linz – 15.\,1.\,1934 München), \emph{Schriftsteller, Kritiker}|pw}’schen Schule, \so{Schnitzler}, hat heute seinen Einzug ins \so{Burgtheater}\orgindex{Burgtheater@Burgtheater|pw} gehalten.«}}}\label{K_L02751-11} (Correſpondent\pwindex{Konried, Julius 27.\,11.\,1853 Wien – 13.\,1.\,1927 ebd.@\textsc{Konried, Julius} (27.\,11.\,1853 Wien – 13.\,1.\,1927 ebd.), \emph{Journalist}|pwv} Herr \textsc{Conried\pwindex{Konried, Julius 27.\,11.\,1853 Wien – 13.\,1.\,1927 ebd.@\textsc{Konried, Julius} (27.\,11.\,1853 Wien – 13.\,1.\,1927 ebd.), \emph{Journalist}|pw}} vom »Neuen Wiener Tagblatt\pwindex{Neues Wiener Tagblatt@\emph{Neues Wiener Tagblatt}|pw}«), das Dich
               einen Mann aus der \textsc{Hermann Bahrschen\pwindex{Bahr, Hermann 19.\,7.\,1863 Linz – 15.\,1.\,1934 München@\textsc{Bahr, Hermann} (19.\,7.\,1863 Linz – 15.\,1.\,1934 München), \emph{Schriftsteller, Kritiker}|pw}} Schule nennt.\pend
           
\pstart
           Den Abend der \textsc{Première} verbrachte ich mit \textsc{Th. Wolff\pwindex{Wolff, Theodor 2.\,8.\,1868 Berlin – 23.\,9.\,1943 ebd.@\textsc{Wolff, Theodor} (2.\,8.\,1868 Berlin – 23.\,9.\,1943 ebd.), \emph{Schriftsteller, Journalist}|pw}} (vom »Berliner Tageblatt\pwindex{Berliner Tageblatt@\emph{Berliner Tageblatt}|pw}«) und{ }ſah fleißig
               auf die Uhr. Um neun Uhr meinte ich, Dein Schickfal müſſe{ }ſich wohl
               entſchieden haben, und da{ }ſchlug \textsc{Wolff\pwindex{Wolff, Theodor 2.\,8.\,1868 Berlin – 23.\,9.\,1943 ebd.@\textsc{Wolff, Theodor} (2.\,8.\,1868 Berlin – 23.\,9.\,1943 ebd.), \emph{Schriftsteller, Journalist}|pw}} vor, auf Dein {\pb}Wohl anzuſtoßen, Was
               geſchah.\pend
           
\pstart
           Die Meinigen, mein Onkel\pwindex{Mamroth, Fedor 21.\,2.\,1851 Breslau – 25.\,6.\,1907 Frankfurt am Main@\textsc{Mamroth, Fedor} (21.\,2.\,1851 Breslau – 25.\,6.\,1907 Frankfurt am Main), \emph{Journalist, Kritiker}|pwv},
               meine Mutter\pwindex{Goldmann, Clementine 15.\,5.\,1842 Breslau – 24.\,2.\,1924 Frankfurt am Main@\textsc{Goldmann, Clementine} (15.\,5.\,1842 Breslau – 24.\,2.\,1924 Frankfurt am Main)|pwv}, mein Schwager\pwindex{Rosengart, Josef 8.\,2.\,1860 Laupheim – 4.\,8.\,1927 Frankfurt am Main@\textsc{Rosengart, Josef} (8.\,2.\,1860 Laupheim – 4.\,8.\,1927 Frankfurt am Main), \emph{Arzt}|pwv},{ }ſind, wie mir heut meine Mutter\pwindex{Goldmann, Clementine 15.\,5.\,1842 Breslau – 24.\,2.\,1924 Frankfurt am Main@\textsc{Goldmann, Clementine} (15.\,5.\,1842 Breslau – 24.\,2.\,1924 Frankfurt am Main)|pwv}{ }ſchreibt, hocherfreut über Deinen Erfolg und laſſen
               Dir von Herzen gratuliren.\pend
           
\pstart
           Am Tag nach der \begin{otherlanguage}{french}\textsc{Première}\end{otherlanguage}, nachdem ich Dein Telegramm erhalten, fuhr ich zur »\textsc{Liberté\orgindex{Liberté@La Liberté|pw}}« und zu den »\textsc{Débats\orgindex{Journal des débats@Journal des débats|pw}}« und bat um eine \label{K_L02751-12v}\edtext{Notiz}{\lemma{\textnormal{\emph{Notiz}}}\Cendnote{\textnormal{[Georges Aubry\pwindex{Aubry, Georges †~1923@\textsc{Aubry, Georges} (†~1923), \emph{Redakteur}|pwk}]: \emph{Théâtres. [Notre correspondant de Vienne]}\pwindex{Aubry, Georges †~1923@\textsc{Aubry, Georges} (†~1923), \emph{Redakteur}!Théâtres. [Notre correspondant de Vienne]@\strich\emph{Théâtres. [Notre correspondant de Vienne]}|pwk}. In: \emph{La Liberté}\pwindex{Liberté@\emph{La Liberté}|pwk}, Jg. 30, Nr. 11.289, 12. 10. 1895, S. 3. Siehe dazu auch XXXX Auszeichnungsfehler: Dokument L02750 nicht gefunden.  [Hippolyte Fierens-Gevaert\pwindex{Fierens-Gevaert, Hippolyte 13.\,8.\,1870 Brüssel – 16.\,12.\,1926 Lüttich@\textsc{Fierens-Gevaert, Hippolyte} (13.\,8.\,1870 Brüssel – 16.\,12.\,1926 Lüttich), \emph{Schriftsteller, Sänger, Kunstkritiker}|pwk}]: \emph{Courrier des Théâtres}\pwindex{Courrier des Théâtres [Liebelei]@\emph{Courrier des Théâtres [Liebelei]}|pwk}. In: \emph{Journal des débats politiques et littéraires}\pwindex{Journal des débats. Politiques et littéraires@\emph{Journal des débats. Politiques et littéraires}|pwk}, Jg. 107,
                        12. 10. 1895, S. 3.}}}\label{K_L02751-12}. Beide Blätter\orgindex{Liberté@La Liberté|pwv}\orgindex{Journal des débats@Journal des débats|pwv} haben die
               Bitte mit großer {\pb}Liebenswürdigkeit erfüllt. Ich{ }ſende{ }ſie Dir anbei;{ }ſtoße Dich nicht an die Unrichtigkeiten, die Du in den Notizen\pwindex{Aubry, Georges †~1923@\textsc{Aubry, Georges} (†~1923), \emph{Redakteur}!Théâtres. [Notre correspondant de Vienne]@\strich\emph{Théâtres. [Notre correspondant de Vienne]}|pwv}\pwindex{Courrier des Théâtres [Liebelei]@\emph{Courrier des Théâtres [Liebelei]}|pwv} findeſt; ich
               habe ihnen die Geſchichte zwar genau erklärt, aber{ }ſie haben doch geſchrieben, was{ }ſie wollten; das iſt{ }ſo Pariſ\oindex{Paris@\textbf{Paris}, \emph{Hauptstadt}|pw}er Art. Jedenfalls
               aber mußt Du Dich bedanken; das iſt hier{ }ſo Sitte. Zuerſt mußt Du \strikeout{e\textcolor{gray}{i}} Deine Viſitkarte mit der Aufſchrift: \label{K_L02751-13v}\edtext{\begin{otherlanguage}{french}\textsc{remercie bien vivement M. Fierens-Gevaert\pwindex{Fierens-Gevaert, Hippolyte 13.\,8.\,1870 Brüssel – 16.\,12.\,1926 Lüttich@\textsc{Fierens-Gevaert, Hippolyte} (13.\,8.\,1870 Brüssel – 16.\,12.\,1926 Lüttich), \emph{Schriftsteller, Sänger, Kunstkritiker}|pw} de son amabilité}\end{otherlanguage}}{\lemma{\textnormal{\emph{remercie … amabilité}}}\Cendnote{\textnormal{französisch: dankt sehr herzlich Herrn
                     Fierens-Gevaert\pwindex{Fierens-Gevaert, Hippolyte 13.\,8.\,1870 Brüssel – 16.\,12.\,1926 Lüttich@\textsc{Fierens-Gevaert, Hippolyte} (13.\,8.\,1870 Brüssel – 16.\,12.\,1926 Lüttich), \emph{Schriftsteller, Sänger, Kunstkritiker}|pwk} für seine
                  Freundlichkeit}}}\label{K_L02751-13}{ }{\pb}ſchicken an: \begin{otherlanguage}{french}\textsc{M. Fierens-Gevaert\pwindex{Fierens-Gevaert, Hippolyte 13.\,8.\,1870 Brüssel – 16.\,12.\,1926 Lüttich@\textsc{Fierens-Gevaert, Hippolyte} (13.\,8.\,1870 Brüssel – 16.\,12.\,1926 Lüttich), \emph{Schriftsteller, Sänger, Kunstkritiker}|pw}, du
                        »Journal des Débats\orgindex{Journal des débats@Journal des débats|pw}«, Rue des Prêtres – St. Germain l’Auxerrois, Paris\oindex{Rue Des Prêtres Saint-Germain L'Auxerrois@\textbf{Rue Des Prêtres Saint-Germain L'Auxerrois}, \emph{Straße}|pw}}\end{otherlanguage}. Eine zweite Karte{ }ſendeſt Du an \textsc{M. Aubry\pwindex{Aubry, Georges †~1923@\textsc{Aubry, Georges} (†~1923), \emph{Redakteur}|pw}\textcolor{gray}{,}{ }\begin{otherlanguage}{french}de la »Liberté\orgindex{Liberté@La Liberté|pw}«, 10. Rue Camou, Paris\oindex{Rue du Général Camou@\textbf{Rue du Général Camou}, \emph{Straße}|pw}\end{otherlanguage}}. Hier mußt Du{ }ſchon etwas wärmer{ }ſchreiben, da Aubry\pwindex{Aubry, Georges †~1923@\textsc{Aubry, Georges} (†~1923), \emph{Redakteur}|pw} ein{ }ſehr herzliches Intereſſe für Dich bezeigt,{ }ſich
               eine mörderiſche Mühe {\pb}gegeben hat, um die von{ }ſeiner Frau\pwindex{Aubry, [MMe. Georges] @\textsc{Aubry, [MMe. Georges]}, \emph{Übersetzerin}|pwv} überſetzte »Kleine Komödie\pwindex{Schnitzler, Arthur 15.\,5.\,1862 Wien – 21.\,10.\,1931 ebd.@\textsc{Schnitzler, Arthur} (15.\,5.\,1862 Wien – 21.\,10.\,1931 ebd.), \emph{Schriftsteller, Mediziner}!kleine Komödie@\strich\emph{Die kleine Komödie}|pw}« in gutes Franzöſiſch zu bringen
               (die Überſetzung\pwindex{Schnitzler, Arthur 15.\,5.\,1862 Wien – 21.\,10.\,1931 ebd.@\textsc{Schnitzler, Arthur} (15.\,5.\,1862 Wien – 21.\,10.\,1931 ebd.), \emph{Schriftsteller, Mediziner}!petite comédie. Mœurs viennois@\strich\emph{La petite comédie. Mœurs viennois}|pwv} iſt
               infolgedeſſen vortrefflich) \textsc{et{[}c{]}}. Du{ }ſchreibſt alſo vielleicht auf Deine Karte: \label{K_L02751-14v}\edtext{\begin{otherlanguage}{french}\textsc{remercie M. Aubry\pwindex{Aubry, Georges †~1923@\textsc{Aubry, Georges} (†~1923), \emph{Redakteur}|pw} du
                        \strikeout{bel} très-bel article\pwindex{Aubry, Georges †~1923@\textsc{Aubry, Georges} (†~1923), \emph{Redakteur}!Théâtres. [Notre correspondant de Vienne]@\strich\emph{Théâtres. [Notre correspondant de Vienne]}|pwv} au sujet de la »Liebelei\pwindex{Schnitzler, Arthur 15.\,5.\,1862 Wien – 21.\,10.\,1931 ebd.@\textsc{Schnitzler, Arthur} (15.\,5.\,1862 Wien – 21.\,10.\,1931 ebd.), \emph{Schriftsteller, Mediziner}!Liebelei. Schauspiel in drei Akten@\strich\emph{Liebelei. Schauspiel in drei Akten}|pw}«, le remercie en outre de toute la
                     peine, qu’il s’est donnée pour la traduction\pwindex{Schnitzler, Arthur 15.\,5.\,1862 Wien – 21.\,10.\,1931 ebd.@\textsc{Schnitzler, Arthur} (15.\,5.\,1862 Wien – 21.\,10.\,1931 ebd.), \emph{Schriftsteller, Mediziner}!petite comédie. Mœurs viennois@\strich\emph{La petite comédie. Mœurs viennois}|pwv} de la »Petite comédie\pwindex{Schnitzler, Arthur 15.\,5.\,1862 Wien – 21.\,10.\,1931 ebd.@\textsc{Schnitzler, Arthur} (15.\,5.\,1862 Wien – 21.\,10.\,1931 ebd.), \emph{Schriftsteller, Mediziner}!kleine Komödie@\strich\emph{Die kleine Komödie}|pw}«, le remercie en un mot de toute son amabilité
                     charmante et espère {\pb}de lui serrer un jour la
                        \strikeout{main} main en ami, soit à Paris\oindex{Paris@\textbf{Paris}, \emph{Hauptstadt}|pw}, soit à Vienne\oindex{Wien@\textbf{Wien}, \emph{Verwaltungsgebiet}|pw}}\end{otherlanguage}}{\lemma{\textnormal{\emph{remercie … Vienne}}}\Cendnote{\textnormal{französisch: dankt Herrn Aubry\pwindex{Aubry, Georges †~1923@\textsc{Aubry, Georges} (†~1923), \emph{Redakteur}|pwk} für den sehr schönen Artikel\pwindex{Aubry, Georges †~1923@\textsc{Aubry, Georges} (†~1923), \emph{Redakteur}!Théâtres. [Notre correspondant de Vienne]@\strich\emph{Théâtres. [Notre correspondant de Vienne]}|pwkv} über \emph{Liebelei}\pwindex{Schnitzler, Arthur 15.\,5.\,1862 Wien – 21.\,10.\,1931 ebd.@\textsc{Schnitzler, Arthur} (15.\,5.\,1862 Wien – 21.\,10.\,1931 ebd.), \emph{Schriftsteller, Mediziner}!Liebelei. Schauspiel in drei Akten@\strich\emph{Liebelei. Schauspiel in drei Akten}|pwk}, dankt auch für all die Mühen, die er sich um die
                     Übersetzung\pwindex{Schnitzler, Arthur 15.\,5.\,1862 Wien – 21.\,10.\,1931 ebd.@\textsc{Schnitzler, Arthur} (15.\,5.\,1862 Wien – 21.\,10.\,1931 ebd.), \emph{Schriftsteller, Mediziner}!petite comédie. Mœurs viennois@\strich\emph{La petite comédie. Mœurs viennois}|pwkv} der »\emph{Kleinen Komödie}\pwindex{Schnitzler, Arthur 15.\,5.\,1862 Wien – 21.\,10.\,1931 ebd.@\textsc{Schnitzler, Arthur} (15.\,5.\,1862 Wien – 21.\,10.\,1931 ebd.), \emph{Schriftsteller, Mediziner}!kleine Komödie@\strich\emph{Die kleine Komödie}|pwk}« gemacht hat, dankt ihm mit
                  einem Wort für all seine liebenswürdige Freundlichkeit und hofft, ihm eines Tages
                  in Paris\oindex{Paris@\textbf{Paris}, \emph{Hauptstadt}|pwk} oder in Wien\oindex{Wien@\textbf{Wien}, \emph{Verwaltungsgebiet}|pwk} als Freund die Hand drücken zu dürfen}}}\label{K_L02751-14}{\dotsfive}\pend
           
\pstart
           So, da haſt Du wieder ein wenig Arbeit.\pend
           
\pstart
           Nochmals, vielen Dank für Dein Telegramm! Danke auch \textsc{Richard\pwindex{Beer-Hofmann, Richard 11.\,7.\,1866 Wien – 26.\,9.\,1945 New York City@\textsc{Beer-Hofmann, Richard} (11.\,7.\,1866 Wien – 26.\,9.\,1945 New York City), \emph{Schriftsteller}|pw}} für das{ }ſeinige! Und{ }ſei von Herzen gegrüßt!\pend
           
\pstart
           Dein {\\[\baselineskip]}\spacefill\mbox{Paul Goldmann.}\pend
           \leftskip=0em{}
\pstart
           \noindent{}Bitte, empfiehl’ mich Deiner Frau Mama\pwindex{Schnitzler, Louise 8.\,7.\,1840 Kőszeg – 9.\,9.\,1911 Wien@\textsc{Schnitzler, Louise} (8.\,7.\,1840 Kőszeg – 9.\,9.\,1911 Wien)|pwv} und{ }ſag’ ihr, ich laſſe ihr zu ihrem Sohne
                  gratuliren.\pend
           \selectlanguage{ngerman}\endnumbering\briefempfaengerindex{Schnitzler, Arthur@\textsc{Schnitzler, Arthur}!zzzGoldmann, Paul@\emph{von Paul Goldmann}!1895-10-131@{13. 10. [1895]}|)be}\mylabel{L02751h}  \newcommand{\dateiname}{L02751}\newcommand{\titel}{Paul Goldmann an Arthur Schnitzler, 13. 10. [1895]}\newcommand{\editorInnen}{Martin Anton Müller und Laura Untner}%% latex-leseansicht-abspann.tex
%% Abspann für die Leseansicht.
%% Der Schalter \ifkorrekturansicht ist bereits durch den Vorspann gesetzt.

%% latex-abspann.tex
%% Gemeinsamer Abspann für Korrekturansicht und Leseansicht.
%% Setzt den Schalter \ifkorrekturansicht voraus (gesetzt in den
%% einbindenden Dateien latex-korrekturansicht-abspann.tex bzw.
%% latex-leseansicht-abspann.tex).
%% ---------------------------------------------------------------

\normalsize

% Das esempio-Environment wird nur in der Leseansicht benötigt
\ifkorrekturansicht\else
\newenvironment{esempio}[3]%
{
    \vspace{1.5ex}
    \rlap{\underline{#1}}
    \par
    \setlength{\parindent}{0cm}
    \nopagebreak
    \leftskip=#2cm
    \rightskip=#3cm
}
{
    \par
}
\fi

\doendnotes{C}
\bigskip
\vfill

\clearpage

\footnotesize

\ifkorrekturansicht
  \lohead{\textsc{register}}
\fi

% theindex-Environment neu definieren ohne reledmac
\makeatletter
\renewenvironment{theindex}{%
  \ifkorrekturansicht
    \section*{\indexname}%
  \else
    \subsubsection*{Index der erwähnten Entitäten}%
  \fi
  \setlength{\parindent}{0pt}%
  \setlength{\parskip}{0pt plus 0.3pt}%
  \let\item\@idxitem
}{%
  \ifkorrekturansicht\clearpage\fi
}
\makeatother

\IfFileExists{\jobname-pw.ind}{\input{\jobname-pw.ind}}{}

% Quellenangabe nur in der Leseansicht
\ifkorrekturansicht\else
% Fallback-Definitionen, falls die .tex-Datei \titel etc. nicht gesetzt hat
\providecommand{\titel}{}
\providecommand{\editorInnen}{}
\providecommand{\dateiname}{\jobname}

\vspace{3cm}

\vfill

\footnotesize
\textsc{Quelle}: \titel. Herausgegeben von {\editorInnen}. In: \emph{Arthur Schnitzler: Briefwechsel mit Autorinnen und Autoren}.
 Digitale Edition, https://schnitzler-briefe.acdh.oeaw.ac.at/{\dateiname}.html (Stand \today)
\fi

\end{document}


