%% latex-korrekturansicht-vorspann.tex
%% Vorspann für die Korrekturansicht.
%% Lädt die gemeinsame Datei latex-vorspann.tex mit gesetztem Schalter.

\newif\ifkorrekturansicht
\korrekturansichttrue

\input{../tex-inputs/latex-vorspann}


\section[Arthur Schnitzler an Stefan Zweig, 16. 5. 1928]{L03742 Arthur Schnitzler an Stefan Zweig, 16. 5. 1928}
\nopagebreak\mylabel{L03742v}
\rehead{ }\normalsize\beginnumbering\briefempfaengerindex{Zweig, Stefan@\textsc{Zweig, Stefan}!zzzSchnitzler, Arthur@\emph{von Arthur Schnitzler}!1928-05-161@{16. 5. 1928}|(be}
\toendnotes[C]{\smallbreak\pagebreak[2]}\Standort{Jerusalem, National Library of Israel, ARC. Ms. Var. 305 1 58 Stefan Zweig Collection.}
\physDesc{Postkarte, 1 Blatt, 2 Seiten, 907 Zeichen
\newline{}Handschrift: schwarze Tinte, lateinische Kurrent
\newline{}Versand: Stempel: »\nobreak{}\oindex{XVIII., Waehring@\textbf{XVIII., Währing}, \emph{A.ADM3}|pwk}18 Wien 110, 16{[}. 5. {]}\textcolor{gray}{2}8, 13\nobreak{}«.  }\toendnotes[C]{\smallbreak}\pstart{}{\pb}\label{T_L03742-1v}\edtext{\textcolor{gray}{\textbf{A. S.}}}{\lemma{\textnormal{\emph{A. S.}}}\Cendnote{\textnormal{ovaler Absenderkleber}}}\label{T_L03742-1}\pend{}\pstart{}\textcolor{gray}{\textbf{WIEN, XVIII.}}\oindex{XVIII., Waehring@\textbf{XVIII., Währing}, \emph{A.ADM3}|pw}\pend{}\pstart{}\textcolor{gray}{\textbf{STERNWARTESTR. 71}}\oindex{Sternwartestrasse 71@\textbf{Sternwartestraße 71}, \emph{Wohngebäude (K.WHS)}|pw}\pend{}{\bigskip}\pstart{}Hr Dr. Stefan Zweig\pend{}\pstart{}Salzburg\oindex{Salzburg@\textbf{Salzburg}, \emph{A.ADM2}|pw} .\pend{}\pstart{}Kapuzinerberg 5\oindex{Paschinger Schloessl@\textbf{Paschinger Schlössl}, \emph{Wohngebäude (K.WHS)}|pw}.\pend{}{\bigskip}\vspace{1em}
\pstart
           \raggedleft{}{\pb}Wien\oindex{Wien@\textbf{Wien}, \emph{A.ADM2}|pw},
                        16. 5. 928\pend
           \vspace{0.5em}
\pstart
           lieber Stefan Zweig, Ihre nach jeder meiner Arbeiten mit so
               rührender Pünktlichkeit eintreffende Briefe, sind mir nicht nur werthvoll durch die
               klugen und herzlichen Dinge, die sie enthalten sondern als immer neuer Beweis einer
               geistigen u seelischen Anhänglichkeit, einer Treue im besten Sinn, die man im Leben
               eigentlich selten – und da{\geminationn} nicht immer dort erfährt, wo man wirkliche Freude davon
               hat. Also lassen Sie sich wieder einmal – danken, – und machen Sie doch bald Ihr
               Versprechen wahr, mir bei nächster Gelegenheit eine Stunde Ihrer, ja man darf es wohl
               sagen, kostbaren Zeit zu schenken. {\pb}Damit es nicht – gegen
               Ihre u meine Absicht – Phrase bleibe\strikeout{n}, theilen Sie
               mir vielleicht nächstens 2–3 Tage vorher mit, wa{\geminationn} Sie wieder in Wien\oindex{Wien@\textbf{Wien}, \emph{A.ADM2}|pw} sind, und wir essen zusammen. Ich möchte Sie so gern wieder
               bei mir sehen.\pend
           
\pstart
           Herzlich grüßend Ihr{\\[\baselineskip]}\spacefill\mbox{ArthurSchnitzl}\pend
           \leftskip=0em{}\selectlanguage{ngerman}\endnumbering\briefempfaengerindex{Zweig, Stefan@\textsc{Zweig, Stefan}!zzzSchnitzler, Arthur@\emph{von Arthur Schnitzler}!1928-05-161@{16. 5. 1928}|)be}\mylabel{L03742h}
\begin{anhang}
\end{anhang}\normalsize

\doendnotes{C}
\bigskip
\vfill

\clearpage

\footnotesize

\lohead{\textsc{register}}

% Definiere theindex-Environment komplett neu ohne reledmac
\makeatletter
\renewenvironment{theindex}{%
  \section*{\indexname}%
  \setlength{\parindent}{0pt}%
  \setlength{\parskip}{0pt plus 0.3pt}%
  \let\item\@idxitem
}{%
  \clearpage
}
\makeatother

\IfFileExists{\jobname-pw.ind}{\input{\jobname-pw.ind}}{}

\end{document}

      