%% latex-leseansicht-vorspann.tex
%% Vorspann für die Leseansicht.
%% Lädt die gemeinsame Datei latex-vorspann.tex mit nicht gesetztem Schalter.

\newif\ifkorrekturansicht
\korrekturansichtfalse

\input{../tex-inputs/latex-vorspann}


\section[Arthur Schnitzler an Stefan Zweig, 16. 5. 1928]{L03742 Arthur Schnitzler an Stefan Zweig, 16. 5. 1928}
\nopagebreak\mylabel{L03742v}
\rehead{ }\normalsize\beginnumbering\briefempfaengerindex{Zweig, Stefan@\textsc{Zweig, Stefan}!zzzSchnitzler, Arthur@\emph{von Arthur Schnitzler}!1928-05-161@{16. 5. 1928}|(be}
\toendnotes[C]{\smallbreak\pagebreak[2]}
\correspDesc{Versand  durch Arthur Schnitzler am 16. 5. 1928 in Wien
\newline{}Erhalt  durch Stefan Zweig im Zeitraum [17. 5. 1928 – 21. 5. 1928?] in Salzburg}\toendnotes[C]{\smallbreak}
\Standort{Jerusalem, National Library of Israel, ARC. Ms. Var. 305 1 58 Stefan Zweig Collection.}
\physDesc{Postkarte, 902 Zeichen
\newline{}Handschrift: schwarze Tinte, lateinische Kurrent
\newline{}Versand: Stempel: »\nobreak{}\oindex{XVIII., Währing@\textbf{XVIII., Währing}, \emph{Verwaltungsgebiet}|pwk}18 Wien 110, 16\textcolor{gray}{.} {[}5.{]} \textcolor{gray}{2}8, 13\nobreak{}«.  }\toendnotes[C]{\smallbreak}\pstart{}{\pb}\label{T_L03742-1v}\edtext{\textcolor{gray}{\textbf{A. S.}}}{\lemma{\textnormal{\emph{A. S.}}}\Cendnote{\textnormal{ovaler Absenderkleber}}}\label{T_L03742-1}\pend{}\pstart{}\textcolor{gray}{\textbf{WIEN, XVIII.}}\oindex{XVIII., Währing@\textbf{XVIII., Währing}, \emph{Verwaltungsgebiet}|pw}\pend{}\pstart{}\textcolor{gray}{\textbf{STERNWARTESTR. 71}}\oindex{Wien@\textbf{Wien}!XVIII., Währing@\textbf{XVIII., Währing}!Sternwartestraße 71@\textbf{Sternwartestraße 71}, \emph{Wohngebäude}|pw}\pend{}{\bigskip}\pstart{}Hr Dr. Stefan Zweig\pend{}\pstart{}Salzburg\oindex{Salzburg@\textbf{Salzburg}, \emph{Verwaltungsgebiet}|pw}.\pend{}\pstart{}Kapuzinerberg 5\oindex{Paschinger Schlössl@\textbf{Paschinger Schlössl}, \emph{Wohngebäude}|pw}.\pend{}{\bigskip}\vspace{1em}
\pstart
           \raggedleft{}{\pb}Wien\oindex{Wien@\textbf{Wien}, \emph{Verwaltungsgebiet}|pw},
                        16. 5. 928\pend
           \vspace{0.5em}
\pstart
           lieber Stefan Zweig, Ihre nach jeder meiner Arbeiten mit so
               rührender Pünktlichkeit eintreffende Briefe, sind mir nicht nur werthvoll durch die
               klugen und herzlichen Dinge, die sie enthalten sondern als immer neuer Beweis einer
               geistigen u seelischen Anhänglichkeit, einer Treue im besten Sinn, die man im Leben
               eigentlich selten – und da{\geminationn} nicht immer dort erfährt, wo man wirkliche Freude davon
               hat. Also lassen Sie sich wieder einmal – danken, – und machen Sie doch bald Ihr
               Versprechen wahr, mir bei \label{K_L03742-1v}\edtext{nächster Gelegenheit}{\lemma{\textnormal{\emph{nächster Gelegenheit}}}\Cendnote{\textnormal{Das nächste
                  belegte Treffen fand erst am 7. 4. 1930 statt.}}}\label{K_L03742-1} eine Stunde Ihrer, ja man darf es wohl
               sagen, kostbaren Zeit zu schenken. {\pb}Damit es nicht – gegen
               Ihre u meine Absicht – Phrase bleibe\strikeout{n}, theilen Sie
               mir vielleicht nächstens 2–3 Tage vorher mit, wa{\geminationn} Sie wieder in Wien\oindex{Wien@\textbf{Wien}, \emph{Verwaltungsgebiet}|pw} sind, und wir essen zusammen. Ich möchte Sie so gern wieder
               bei mir sehen.\pend
           
\pstart
           Herzlich grüßend Ihr{\\[\baselineskip]}\spacefill\mbox{ArthurSchnitz}\pend
           \leftskip=0em{}\selectlanguage{ngerman}\endnumbering\briefempfaengerindex{Zweig, Stefan@\textsc{Zweig, Stefan}!zzzSchnitzler, Arthur@\emph{von Arthur Schnitzler}!1928-05-161@{16. 5. 1928}|)be}\mylabel{L03742h}  \newcommand{\dateiname}{L03742}\newcommand{\titel}{Arthur Schnitzler an Stefan Zweig, 16. 5. 1928}\newcommand{\editorInnen}{Selma Jahnke und Martin Anton Müller}%% latex-leseansicht-abspann.tex
%% Abspann für die Leseansicht.
%% Der Schalter \ifkorrekturansicht ist bereits durch den Vorspann gesetzt.

%% latex-abspann.tex
%% Gemeinsamer Abspann für Korrekturansicht und Leseansicht.
%% Setzt den Schalter \ifkorrekturansicht voraus (gesetzt in den
%% einbindenden Dateien latex-korrekturansicht-abspann.tex bzw.
%% latex-leseansicht-abspann.tex).
%% ---------------------------------------------------------------

\normalsize

% Das esempio-Environment wird nur in der Leseansicht benötigt
\ifkorrekturansicht\else
\newenvironment{esempio}[3]%
{
    \vspace{1.5ex}
    \rlap{\underline{#1}}
    \par
    \setlength{\parindent}{0cm}
    \nopagebreak
    \leftskip=#2cm
    \rightskip=#3cm
}
{
    \par
}
\fi

\doendnotes{C}
\bigskip
\vfill

\clearpage

\footnotesize

\ifkorrekturansicht
  \lohead{\textsc{register}}
\fi

% theindex-Environment neu definieren ohne reledmac
\makeatletter
\renewenvironment{theindex}{%
  \ifkorrekturansicht
    \section*{\indexname}%
  \else
    \subsubsection*{Index der erwähnten Entitäten}%
  \fi
  \setlength{\parindent}{0pt}%
  \setlength{\parskip}{0pt plus 0.3pt}%
  \let\item\@idxitem
}{%
  \ifkorrekturansicht\clearpage\fi
}
\makeatother

\IfFileExists{\jobname-pw.ind}{\input{\jobname-pw.ind}}{}

% Quellenangabe nur in der Leseansicht
\ifkorrekturansicht\else
% Fallback-Definitionen, falls die .tex-Datei \titel etc. nicht gesetzt hat
\providecommand{\titel}{}
\providecommand{\editorInnen}{}
\providecommand{\dateiname}{\jobname}

\vspace{3cm}

\vfill

\footnotesize
\textsc{Quelle}: \titel. Herausgegeben von {\editorInnen}. In: \emph{Arthur Schnitzler: Briefwechsel mit Autorinnen und Autoren}.
 Digitale Edition, https://schnitzler-briefe.acdh.oeaw.ac.at/{\dateiname}.html (Stand \today)
\fi

\end{document}


