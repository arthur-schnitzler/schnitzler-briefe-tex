%% latex-leseansicht-vorspann.tex
%% Vorspann für die Leseansicht.
%% Lädt die gemeinsame Datei latex-vorspann.tex mit nicht gesetztem Schalter.

\newif\ifkorrekturansicht
\korrekturansichtfalse

\input{../tex-inputs/latex-vorspann}

\begin{center}
            \textcolor{red}{ENTWURF. ENTZIFFERUNG NOCH NICHT KORREKTURGELESEN}
                      \end{center}
            
               \section[Gerty von Hofmannsthal an Olga Schnitzler, 13. {[}9.{]} 1909]{ Gerty von Hofmannsthal an Olga Schnitzler, 13. {[}9.{]} 1909}\nopagebreak\mylabel{v}\rehead{ }\begin{ledgroupsized}[t]{13cm}\normalsize\beginnumbering\briefempfaengerindex{Schnitzler, Olga@\textsc{Schnitzler, Olga}!zzzHofmannsthal, Gertrude von@\emph{von Gertrude von Hofmannsthal}!1909-09-131@{13. {[}9.{]} 1909}|(be} \toendnotes[C]{\smallbreak\pagebreak[2]} \Standort{CUL, Schnitzler, B 43.}
\physDesc{Bildpostkarte
\newline{}Handschrift: schwarze Tinte, lateinische Kurrent\newline{}Versand: Stempel: »\nobreak{}\oindex{Bad Aussee@\textbf{Bad Aussee}|pwk}\textcolor{gray}{Aussee} in der
                                                  Steiermark, 13. {[}9.{]} 09\nobreak{}«.  
\newline{}Schnitzler: mit Bleistift beschriftet: »\textsc{Hofm}« \newline{}Ordnung: 1) mit Bleistift von unbekannter Hand nummeriert:
                                                »379« 2) mit Bleistift von unbekannter Hand nummeriert: »309«}\toendnotes[C]{\smallbreak}\pstart{}{\pb}Frau Olga
                        Schnitzler\pend{}\pstart{}Wien\oindex{Wien@\textbf{Wien}|pw}\pend{}\pstart{}XVIII Spöttlgasse 7\oindex{Edmund-Weiss-Gasse@\textbf{Edmund-Weiß-Gasse}|pw}\pend{}{\bigskip}\pstart
           \noindent{}\centering{}{\pb}{[}Hugo\pwindex{Hofmannsthal, Hugo von 01.02.1874 – 15.07.1929@\textsc{Hofmannsthal, Hugo von} (01.02.1874 – 15.07.1929), \emph{Schriftsteller}|pw} und Christiane von Hofmannsthal\pwindex{Hofmannsthal, Christiane von 14.05.1902 – 05.01.1987@\textsc{Hofmannsthal, Christiane von} (14.05.1902 – 05.01.1987)|pw} auf einer
                            Wiese.{]}\pend
           \pstart
           {\pb}Liebe Olga, ich danke Ihnen herzlichst für Ihren lieben Brief
                    und für die Auskunft. Die Anfälle bei der Kleinen\pwindex{Hofmannsthal, Christiane von 14.05.1902 – 05.01.1987@\textsc{Hofmannsthal, Christiane von} (14.05.1902 – 05.01.1987)|pwv} sind gottlob so dass es noch nicht
                    entschieden ist, ob es der \label{K_L01871_1v}\edtext{Keuchhusten}{\lemma{\textnormal{\emph{Keuchhusten}}}\Cendnote{\textnormal{Die Monatsangabe
                        ist am Poststempel nicht zu erkennen. Aber da Christiane\pwindex{Hofmannsthal, Christiane von 14.05.1902 – 05.01.1987@\textsc{Hofmannsthal, Christiane von} (14.05.1902 – 05.01.1987)|pwk}s Erkrankung auch in einem Brief Hugo von Hofmannsthal\pwindex{Hofmannsthal, Hugo von 01.02.1874 – 15.07.1929@\textsc{Hofmannsthal, Hugo von} (01.02.1874 – 15.07.1929), \emph{Schriftsteller}|pwk}s an Helene von Nostitz-Wallwitz\pwindex{Nostitz-Wallwitz, Helene von 18.11.1878 – 17.07.1944@\textsc{Nostitz-Wallwitz, Helene von} (18.11.1878 – 17.07.1944), \emph{Schriftstellerin}|pwk} vom
                            12. 9. 1909 Erwähnung findet, kann die Karte datiert
                        werden. (\emph{Hugo von Hofmannsthal – Helene von Nostitz.
                                Briefwechsel.} Herausgegeben von Oswalt von Nostitz.
                            Frankfurt am Main: \emph{Fischer}\orgindex{S. Fischer Verlag@S. Fischer Verlag|pwk}{ }1965, S. 87)}}}\label{K_L01871_1h} ist. Es ko{\geminationm}t einen Abend und in der Nacht, so dass sie am
                    Tag ganz frei davon ist. Ich lasse sie alle drei beisa{\geminationm}en. Ich denke jetzt {\pb}viel an Sie und wir sind sehr
                    traurig, dass wir Sie heuer \introOben{}im Sommer\introOben{} gar nicht
                    gesehen haben, vom Hugo\pwindex{Hofmannsthal, Hugo von 01.02.1874 – 15.07.1929@\textsc{Hofmannsthal, Hugo von} (01.02.1874 – 15.07.1929), \emph{Schriftsteller}|pw} viele Grüsse an Arthur\pwindex{Schnitzler, Arthur 15.05.1862 – 21.10.1931@\textsc{Schnitzler, Arthur} (15.05.1862 – 21.10.1931), \emph{Schriftsteller, Mediziner}|pw} und Sie und gute Wünsche\pend
           \pstart
           Ihre{\\[\baselineskip]}\spacefill\mbox{Gerty}\pend
           \leftskip=0em{}\endnumbering\briefempfaengerindex{Schnitzler, Olga@\textsc{Schnitzler, Olga}!zzzHofmannsthal, Gertrude von@\emph{von Gertrude von Hofmannsthal}!1909-09-131@{13. {[}9.{]} 1909}|)be}\mylabel{h}\end{ledgroupsized}  \newcommand{\dateiname}{L01871}\newcommand{\titel}{Gerty von Hofmannsthal an Olga Schnitzler, 13. [9.] 1909}\newcommand{\editorInnen}{Martin Anton Müller und Gerd-Hermann Susen}%% latex-leseansicht-abspann.tex
%% Abspann für die Leseansicht.
%% Der Schalter \ifkorrekturansicht ist bereits durch den Vorspann gesetzt.

%% latex-abspann.tex
%% Gemeinsamer Abspann für Korrekturansicht und Leseansicht.
%% Setzt den Schalter \ifkorrekturansicht voraus (gesetzt in den
%% einbindenden Dateien latex-korrekturansicht-abspann.tex bzw.
%% latex-leseansicht-abspann.tex).
%% ---------------------------------------------------------------

\normalsize

% Das esempio-Environment wird nur in der Leseansicht benötigt
\ifkorrekturansicht\else
\newenvironment{esempio}[3]%
{
    \vspace{1.5ex}
    \rlap{\underline{#1}}
    \par
    \setlength{\parindent}{0cm}
    \nopagebreak
    \leftskip=#2cm
    \rightskip=#3cm
}
{
    \par
}
\fi

\doendnotes{C}
\bigskip
\vfill

\clearpage

\footnotesize

\ifkorrekturansicht
  \lohead{\textsc{register}}
\fi

% theindex-Environment neu definieren ohne reledmac
\makeatletter
\renewenvironment{theindex}{%
  \ifkorrekturansicht
    \section*{\indexname}%
  \else
    \subsubsection*{Index der erwähnten Entitäten}%
  \fi
  \setlength{\parindent}{0pt}%
  \setlength{\parskip}{0pt plus 0.3pt}%
  \let\item\@idxitem
}{%
  \ifkorrekturansicht\clearpage\fi
}
\makeatother

\IfFileExists{\jobname-pw.ind}{\input{\jobname-pw.ind}}{}

% Quellenangabe nur in der Leseansicht
\ifkorrekturansicht\else
% Fallback-Definitionen, falls die .tex-Datei \titel etc. nicht gesetzt hat
\providecommand{\titel}{}
\providecommand{\editorInnen}{}
\providecommand{\dateiname}{\jobname}

\vspace{3cm}

\vfill

\footnotesize
\textsc{Quelle}: \titel. Herausgegeben von {\editorInnen}. In: \emph{Arthur Schnitzler: Briefwechsel mit Autorinnen und Autoren}.
 Digitale Edition, https://schnitzler-briefe.acdh.oeaw.ac.at/{\dateiname}.html (Stand \today)
\fi

\end{document}


      