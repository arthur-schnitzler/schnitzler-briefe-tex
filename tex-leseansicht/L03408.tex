%% latex-leseansicht-vorspann.tex
%% Vorspann für die Leseansicht.
%% Lädt die gemeinsame Datei latex-vorspann.tex mit nicht gesetztem Schalter.

\newif\ifkorrekturansicht
\korrekturansichtfalse

\input{../tex-inputs/latex-vorspann}


\section[ Felix Salten an Arthur Schnitzler, 6. 5. 1905]{L03408 Felix Salten an Arthur Schnitzler,  6. 5. 1905}
\nopagebreak\mylabel{L03408v}
\rehead{ }\normalsize\beginnumbering\briefempfaengerindex{Schnitzler, Arthur@\textsc{Schnitzler, Arthur}!zzzSalten, Felix@\emph{von Felix Salten}!1905-05-061@{6. 5. 1905}|(be}
\toendnotes[C]{\smallbreak\pagebreak[2]}
\correspDesc{Versand  durch Felix Salten am 6. 5. 1905 in Wien
\newline{}Erhalt  durch Arthur Schnitzler im Zeitraum [6. 5. 1905?] in Wien}\toendnotes[C]{\smallbreak}
\Standort{CUL, Schnitzler, B 89, B 1.}
\physDesc{Postkarte, 558 Zeichen
\newline{}Handschrift: Bleistift, lateinische Kurrent
\newline{}Versand: Stempel: »\nobreak{}\oindex{I., Innere Stadt@\textbf{I., Innere Stadt}, \emph{Verwaltungsgebiet}|pwk}Wien 1/1 1, 6. 5. 05, 11–12 N.\nobreak{}«.  
\newline{}Ordnung: mit Bleistift von unbekannter Hand nummeriert: »200« }\toendnotes[C]{\smallbreak}\pstart{}{\pb}Herrn D\textsuperscript{r} Arthur Schnitzler\pend{}\pstart{}Wien XVIII.\oindex{XVIII., Währing@\textbf{XVIII., Währing}, \emph{Verwaltungsgebiet}|pw}\pend{}\pstart{}Spoettelgaße 7\oindex{Wien@\textbf{Wien}!XVIII., Währing@\textbf{XVIII., Währing}!Edmund-Weiß-Gasse 7@\textbf{Edmund-Weiß-Gasse 7}, \emph{Wohngebäude}|pw}\pend{}{\bigskip}\vspace{1em}
\pstart
           \raggedleft{}{\pb}6/5 05\pend
           \vspace{0.5em}
\pstart
           Lieber – wir \label{K_L03408-1v}\edtext{wohnen
               schon Pötzleinsdorferstraße 88\oindex{Wien@\textbf{Wien}!XVIII., Währing@\textbf{XVIII., Währing}!Pötzleinsdorferstraße@\textbf{Pötzleinsdorferstraße}, \emph{Straße}|pw}}{\lemma{\textnormal{\emph{wohnen schon Pötzleinsdorferstraße 88}}}\Cendnote{\textnormal{Bei dieser Adresse – ebenso wie bei der
                     Starkfriedgasse 12\oindex{Starkfriedgassse@\textbf{Starkfriedgassse}, \emph{Straße}|pwk} im Vorjahr, die 650
                  Meter entfernt liegt – handelte es sich um Sommersitze, die nur für die warme
                  Jahreszeit angemietet wurden.}}}\label{K_L03408-1}. Spaziergänge, Sommerpläne u. s. w. können
               jetzt besprochen werden. Nach dem \label{K_L03408-2v}\edtext{Sommernachtstraum\pwindex{\textcolor{red}{\textsuperscript{XXXX indx1}}!Sommernachtstraum. Komödie in fünf Aufzügen@\strich\emph{Ein Sommernachtstraum. Komödie in fünf Aufzügen}|pw}}{\lemma{\textnormal{\emph{Sommernachtstraum}}}\Cendnote{\textnormal{Das Stück\pwindex{\textcolor{red}{\textsuperscript{XXXX indx1}}!Sommernachtstraum. Komödie in fünf Aufzügen@\strich\emph{Ein Sommernachtstraum. Komödie in fünf Aufzügen}|pwkv} – in der Inszenierung von Max Reinhardt\pwindex{Reinhardt, Max 9.\,9.\,1873 Baden bei Wien – 30.\,10.\,1943 New York City@\textsc{Reinhardt, Max} (9.\,9.\,1873 Baden bei Wien – 30.\,10.\,1943 New York City), \emph{Theaterleiter, Regisseur, Schauspieler}|pwk} – wurde in Wien\oindex{Wien@\textbf{Wien}, \emph{Verwaltungsgebiet}|pwk} erstmals am 20. 5. 1905 beim Gastspiel des \emph{Kleinen
                     Theaters}\orgindex{Kleines Theater@Kleines Theater|pwk} und des \emph{Neuen Theaters}\orgindex{Neues Theater@Neues Theater|pwk} am Theater an der Wien\oindex{Wien@\textbf{Wien}!VI., Mariahilf@\textbf{VI., Mariahilf}!Theater an der Wien@\textbf{Theater an der Wien}, \emph{Theater}|pwk} gegeben. Schnitzler besuchte die Aufführung, vgl. A. S.: \emph{Tagebuch}, 20. 5. 1905.}}}\label{K_L03408-2} wollen
               wir nach Maria Zell\oindex{Mariazell@\textbf{Mariazell}, \emph{Hauptstadt}|pw}. (Ersatz für Florenz\oindex{Florenz@\textbf{Florenz}|pw}, das aus Zeitmangel entfiel) Vielleicht machen wir die
                  \label{K_L03408-3v}\edtext{Parthie zu viert\pwindex{Schnitzler, Olga 17.\,1.\,1882 Wien – 13.\,1.\,1970 Lugano@\textsc{Schnitzler, Olga} (17.\,1.\,1882 Wien – 13.\,1.\,1970 Lugano), \emph{Schauspielerin, Sängerin}|pwv}\pwindex{Salten, Ottilie 7.\,3.\,1868 Prag – 22.\,6.\,1942 Zürich@\textsc{Salten, Ottilie} (7.\,3.\,1868 Prag – 22.\,6.\,1942 Zürich), \emph{Schauspielerin}|pwv}}{\lemma{\textnormal{\emph{Parthie zu viert}}}\Cendnote{\textnormal{Das Vorhaben verschob sich bis
                  Ende Juli 1905. Letztlich fuhr nur Salten\pwindex{Salten, Felix 6.\,9.\,1869 Budapest – 8.\,10.\,1945 Zürich@\textsc{Salten, Felix} (6.\,9.\,1869 Budapest – 8.\,10.\,1945 Zürich), \emph{Schriftsteller, Journalist, Chefredakteur}|pwk} mit seinem Schwager Richard Metzl\pwindex{Metzl, Richard 20.\,4.\,1870 Prag – 31.\,10.\,1941 Paris@\textsc{Metzl, Richard} (20.\,4.\,1870 Prag – 31.\,10.\,1941 Paris), \emph{Regisseur, Schauspieler, Theatersekretär}|pwk},
                  vgl. XXXX Auszeichnungsfehler: Dokument L03410 nicht gefunden; A. S.: \emph{Tagebuch}, 31. 7. 1905. Die Möglichkeit einer 
                  gemeinsamen Reise stand aber bis kurz vorher im Raum, vgl. XXXX Auszeichnungsfehler: Dokument L03000 nicht gefunden.}}}\label{K_L03408-3}, wie’s ja besprochen war?\pend
           
\pstart
           Schreiben Sie, wenn man Sie am besten trifft, und wann Ihre Frau\pwindex{Schnitzler, Olga 17.\,1.\,1882 Wien – 13.\,1.\,1970 Lugano@\textsc{Schnitzler, Olga} (17.\,1.\,1882 Wien – 13.\,1.\,1970 Lugano), \emph{Schauspielerin, Sängerin}|pwv} am wenigsten gestört wird. Wir wollen
                  \label{K_L03408-4v}\edtext{bald einmal Vormittag oder
               Nachmittag zu Ihnen}{\lemma{\textnormal{\emph{bald … Ihnen}}}\Cendnote{\textnormal{Ein solcher Besuch
                  ist nicht im \emph{Tagebuch}\pwindex{Schnitzler, Arthur 15.\,5.\,1862 Wien – 21.\,10.\,1931 ebd.@\textsc{Schnitzler, Arthur} (15.\,5.\,1862 Wien – 21.\,10.\,1931 ebd.), \emph{Schriftsteller, Mediziner}!Tagebuch@\strich\emph{Tagebuch}|pwk}{ }Schnitzlers belegt.}}}\label{K_L03408-4}.
               – Die gewünschten \label{K_L03408-5v}\edtext{12 Exemplare\pwindex{Zeit@\emph{Die Zeit}|pwv}\pwindex{Schnitzler, Arthur 15.\,5.\,1862 Wien – 21.\,10.\,1931 ebd.@\textsc{Schnitzler, Arthur} (15.\,5.\,1862 Wien – 21.\,10.\,1931 ebd.), \emph{Schriftsteller, Mediziner}!Zum großen Wurstel. Burleske in einem Akt@\strich\emph{Zum großen Wurstel. Burleske in einem Akt}|pwv}\pwindex{Schnitzler, Arthur 15.\,5.\,1862 Wien – 21.\,10.\,1931 ebd.@\textsc{Schnitzler, Arthur} (15.\,5.\,1862 Wien – 21.\,10.\,1931 ebd.), \emph{Schriftsteller, Mediziner}!Schiller-Feier@\strich\emph{Schiller-Feier}|pwv}}{\lemma{\textnormal{\emph{12 Exemplare}}}\Cendnote{\textnormal{Siehe XXXX Auszeichnungsfehler: Dokument L02999 nicht gefunden.
               }}}\label{K_L03408-5} haben Sie wol schon erhalten?\pend
           
\pstart
           Herzlich Ihr {\\[\baselineskip]}\spacefill\mbox{S.}\pend
           \leftskip=0em{}\selectlanguage{ngerman}\endnumbering\briefempfaengerindex{Schnitzler, Arthur@\textsc{Schnitzler, Arthur}!zzzSalten, Felix@\emph{von Felix Salten}!1905-05-061@{6. 5. 1905}|)be}\mylabel{L03408h}  \newcommand{\dateiname}{L03408}\newcommand{\titel}{Felix Salten an Arthur Schnitzler, 6. 5. 1905}\newcommand{\editorInnen}{Martin Anton Müller und Laura Untner}%% latex-leseansicht-abspann.tex
%% Abspann für die Leseansicht.
%% Der Schalter \ifkorrekturansicht ist bereits durch den Vorspann gesetzt.

%% latex-abspann.tex
%% Gemeinsamer Abspann für Korrekturansicht und Leseansicht.
%% Setzt den Schalter \ifkorrekturansicht voraus (gesetzt in den
%% einbindenden Dateien latex-korrekturansicht-abspann.tex bzw.
%% latex-leseansicht-abspann.tex).
%% ---------------------------------------------------------------

\normalsize

% Das esempio-Environment wird nur in der Leseansicht benötigt
\ifkorrekturansicht\else
\newenvironment{esempio}[3]%
{
    \vspace{1.5ex}
    \rlap{\underline{#1}}
    \par
    \setlength{\parindent}{0cm}
    \nopagebreak
    \leftskip=#2cm
    \rightskip=#3cm
}
{
    \par
}
\fi

\doendnotes{C}
\bigskip
\vfill

\clearpage

\footnotesize

\ifkorrekturansicht
  \lohead{\textsc{register}}
\fi

% theindex-Environment neu definieren ohne reledmac
\makeatletter
\renewenvironment{theindex}{%
  \ifkorrekturansicht
    \section*{\indexname}%
  \else
    \subsubsection*{Index der erwähnten Entitäten}%
  \fi
  \setlength{\parindent}{0pt}%
  \setlength{\parskip}{0pt plus 0.3pt}%
  \let\item\@idxitem
}{%
  \ifkorrekturansicht\clearpage\fi
}
\makeatother

\IfFileExists{\jobname-pw.ind}{\input{\jobname-pw.ind}}{}

% Quellenangabe nur in der Leseansicht
\ifkorrekturansicht\else
% Fallback-Definitionen, falls die .tex-Datei \titel etc. nicht gesetzt hat
\providecommand{\titel}{}
\providecommand{\editorInnen}{}
\providecommand{\dateiname}{\jobname}

\vspace{3cm}

\vfill

\footnotesize
\textsc{Quelle}: \titel. Herausgegeben von {\editorInnen}. In: \emph{Arthur Schnitzler: Briefwechsel mit Autorinnen und Autoren}.
 Digitale Edition, https://schnitzler-briefe.acdh.oeaw.ac.at/{\dateiname}.html (Stand \today)
\fi

\end{document}


