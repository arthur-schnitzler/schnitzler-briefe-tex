%% latex-korrekturansicht-vorspann.tex
%% Vorspann für die Korrekturansicht.
%% Lädt die gemeinsame Datei latex-vorspann.tex mit gesetztem Schalter.

\newif\ifkorrekturansicht
\korrekturansichttrue

\input{../tex-inputs/latex-vorspann}


\section[ Felix Salten an Arthur Schnitzler, 6. 5. 1905]{L03408 Felix Salten an Arthur Schnitzler, 6. 5. 1905}
\nopagebreak\mylabel{L03408v}
\rehead{ }\normalsize\beginnumbering\briefempfaengerindex{Schnitzler, Arthur@\textsc{Schnitzler, Arthur}!zzzSalten, Felix@\emph{von Felix Salten}!1905-05-061@{6. 5. 1905}|(be}
\toendnotes[C]{\smallbreak\pagebreak[2]}\Standort{CUL, Schnitzler, B 89, B 1.}
\physDesc{Postkarte, 558 Zeichen
\newline{}Handschrift: Bleistift, lateinische Kurrent
\newline{}Versand: Stempel: »\nobreak{}\oindex{I., Innere Stadt@\textbf{I., Innere Stadt}, \emph{A.ADM3}|pwk}Wien 1/1 1, 6. 5. 05, 11–12 N.\nobreak{}«.  
\newline{}Ordnung: mit Bleistift von unbekannter Hand nummeriert: »200« }\toendnotes[C]{\smallbreak}\pstart{}{\pb}Herrn D\textsuperscript{r} Arthur Schnitzler\pend{}\pstart{}Wien XVIII.\oindex{XVIII., Waehring@\textbf{XVIII., Währing}, \emph{A.ADM3}|pw}\pend{}\pstart{}Spoettelgaße 7\oindex{Edmund-Weiss-Gasse 7@\textbf{Edmund-Weiß-Gasse 7}, \emph{Wohngebäude (K.WHS)}|pw}\pend{}{\bigskip}\vspace{1em}
\pstart
           \raggedleft{}{\pb}6/5 05\pend
           \vspace{0.5em}
\pstart
           Lieber – wir \label{K_L03408-1v}\edtext{wohnen
               schon Pötzleinsdorferstraße 88\oindex{Poetzleinsdorferstrasse@\textbf{Pötzleinsdorferstraße}, \emph{Straße (K.STR)}|pw}}{\lemma{\textnormal{\emph{wohnen schon Pötzleinsdorferstraße 88}}}\Cendnote{\textnormal{Bei dieser Adresse – ebenso wie bei der
                     Starkfriedgasse 12\oindex{Starkfriedgassse@\textbf{Starkfriedgassse}, \emph{Straße (K.STR)}|pwk} im Vorjahr, die 650
                  Meter entfernt liegt – handelte es sich um Sommersitze, die nur für die warme
                  Jahreszeit angemietet wurden.}}}\label{K_L03408-1}. Spaziergänge, Sommerpläne u. s. w. können
               jetzt besprochen werden. Nach dem \label{K_L03408-2v}\edtext{Sommernachtstraum\pwindex{Sommernachtstraum. Komoedie in fuenf Aufzuegen@\emph{Ein Sommernachtstraum. Komödie in fünf Aufzügen}|pw}}{\lemma{\textnormal{\emph{Sommernachtstraum}}}\Cendnote{\textnormal{Das Stück\pwindex{Sommernachtstraum. Komoedie in fuenf Aufzuegen@\emph{Ein Sommernachtstraum. Komödie in fünf Aufzügen}|pwkv} – in der Inszenierung von Max Reinhardt\pwindex{Reinhardt, Max 09.09.1873 – 30.10.1943@\textsc{Reinhardt, Max} (09.09.1873 – 30.10.1943), \emph{Theaterleiter/Theaterleiterin, Regisseur/Regisseurin, Schauspieler/Schauspielerin}|pwk} – wurde in Wien\oindex{Wien@\textbf{Wien}, \emph{A.ADM2}|pwk} erstmals am 20. 5. 1905 beim Gastspiel des \emph{Kleinen
                     Theaters}\orgindex{Kleines Theater@Kleines Theater|pwk} und des \emph{Neuen Theaters}\orgindex{Neues Theater@Neues Theater|pwk} am Theater an der Wien\oindex{Theater an der Wien@\textbf{Theater an der Wien}, \emph{Theater (K.THE)}|pwk} gegeben. Schnitzler besuchte die Aufführung, vgl. A. S.: \emph{Tagebuch}, 20. 5. 1905.}}}\label{K_L03408-2} wollen
               wir nach Maria Zell\oindex{Mariazell@\textbf{Mariazell}, \emph{P.PPLA3}|pw}. (Ersatz für Florenz\oindex{Florenz@\textbf{Florenz}, \emph{P.PPLA}|pw}, das aus Zeitmangel entfiel) Vielleicht machen wir die
                  \label{K_L03408-3v}\edtext{Parthie zu viert\pwindex{Schnitzler, Olga 17.01.1882 – 13.01.1970@\textsc{Schnitzler, Olga} (17.01.1882 – 13.01.1970), \emph{Schauspieler/Schauspielerin, Sänger/Sängerin}|pwv}\pwindex{Salten, Ottilie 07.03.1868 – 22.06.1942@\textsc{Salten, Ottilie} (07.03.1868 – 22.06.1942), \emph{Schauspieler/Schauspielerin}|pwv}}{\lemma{\textnormal{\emph{Parthie zu viert}}}\Cendnote{\textnormal{Das Vorhaben verschob sich bis
                  Ende Juli 1905. Letztlich fuhr nur Salten\pwindex{Salten, Felix 06.09.1869 – 08.10.1945@\textsc{Salten, Felix} (06.09.1869 – 08.10.1945), \emph{Schriftsteller/Schriftstellerin, Journalist/Journalistin, Chefredakteur/Chefredakteurin}|pwk} mit seinem Schwager Richard Metzl\pwindex{Metzl, Richard 20.04.1870 – 31.10.1941@\textsc{Metzl, Richard} (20.04.1870 – 31.10.1941), \emph{Regisseur/Regisseurin, Schauspieler/Schauspielerin, Theatersekretär/Theatersekretärin}|pwk},
                  vgl. Felix Salten und Richard Metzl an Arthur
               Schnitzler, [30. 7. 1905?]; A. S.: \emph{Tagebuch}, 31. 7. 1905. Die Möglichkeit einer 
                  gemeinsamen Reise stand aber bis kurz vorher im Raum, vgl. Arthur Schnitzler an Felix Salten, 20. 7. 1905.}}}\label{K_L03408-3}, wie’s ja besprochen war?\pend
           
\pstart
           Schreiben Sie, wenn man Sie am besten trifft, und wann Ihre Frau\pwindex{Schnitzler, Olga 17.01.1882 – 13.01.1970@\textsc{Schnitzler, Olga} (17.01.1882 – 13.01.1970), \emph{Schauspieler/Schauspielerin, Sänger/Sängerin}|pwv} am wenigsten gestört wird. Wir wollen
                  \label{K_L03408-4v}\edtext{bald einmal Vormittag oder
               Nachmittag zu Ihnen}{\lemma{\textnormal{\emph{bald … Ihnen}}}\Cendnote{\textnormal{Ein solcher Besuch
                  ist nicht im \emph{Tagebuch}\pwindex{Tagebuch@\emph{Tagebuch}|pwk}{ }Schnitzlers belegt.}}}\label{K_L03408-4}.
               – Die gewünschten \label{K_L03408-5v}\edtext{12 Exemplare\pwindex{Zeit@\emph{Die Zeit}|pwv}\pwindex{Zum grossen Wurstel. Burleske in einem Akt@\emph{Zum großen Wurstel. Burleske in einem Akt}|pwv}\pwindex{Schiller-Feier@\emph{Schiller-Feier}|pwv}}{\lemma{\textnormal{\emph{12 Exemplare}}}\Cendnote{\textnormal{Siehe Arthur Schnitzler an Felix Salten, 29. 4. 1905.
               }}}\label{K_L03408-5} haben Sie wol schon erhalten?\pend
           
\pstart
           Herzlich Ihr {\\[\baselineskip]}\spacefill\mbox{S.}\pend
           \leftskip=0em{}\selectlanguage{ngerman}\endnumbering\briefempfaengerindex{Schnitzler, Arthur@\textsc{Schnitzler, Arthur}!zzzSalten, Felix@\emph{von Felix Salten}!1905-05-061@{6. 5. 1905}|)be}\mylabel{L03408h}  \normalsize

\doendnotes{C}
\bigskip
\vfill

\clearpage

\footnotesize

\lohead{\textsc{register}}

% Definiere theindex-Environment komplett neu ohne reledmac
\makeatletter
\renewenvironment{theindex}{%
  \section*{\indexname}%
  \setlength{\parindent}{0pt}%
  \setlength{\parskip}{0pt plus 0.3pt}%
  \let\item\@idxitem
}{%
  \clearpage
}
\makeatother

\IfFileExists{\jobname-pw.ind}{\input{\jobname-pw.ind}}{}

\end{document}

      