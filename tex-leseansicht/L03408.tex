%% latex-leseansicht-vorspann.tex
%% Vorspann für die Leseansicht.
%% Lädt die gemeinsame Datei latex-vorspann.tex mit nicht gesetztem Schalter.

\newif\ifkorrekturansicht
\korrekturansichtfalse

\input{../tex-inputs/latex-vorspann}

\begin{center}
            \textcolor{red}{ENTWURF, NICHT FERTIG KORRIGIERT}
                      \end{center}
            
         
         \renewcommand{\erwaehntePersonen}{Personen: Max Reinhardt, Ottilie Salten, Olga Schnitzler}
         \renewcommand{\erwaehnteInstitutionen}{Institutionen: Kleines Theater, Neues Theater}
         \renewcommand{\erwaehnteOrte}{Orte: Edmund-Weiß-Gasse, Florenz, I., Innere Stadt, Mariazell, Pötzleinsdorferstraße, Theater an der Wien, Wien, XVIII., Währing}
         \renewcommand{\erwaehnteWerke}{Werke: Ein Sommernachtstraum. Komödie in fünf Aufzügen}
               \section[Felix Salten an Arthur Schnitzler, 6. 5. 1905]{ Felix Salten an Arthur Schnitzler, 6. 5. 1905}\nopagebreak\mylabel{v}\rehead{ }\begin{ledgroupsized}[t]{13cm}\normalsize\beginnumbering \toendnotes[C]{\smallbreak\pagebreak[2]} \Standort{CUL, Schnitzler, B 89, B 1.}
\physDesc{Postkarte
\newline{}Handschrift: Bleistift, lateinische Kurrent\newline{}Versand: Stempel: »\nobreak{}\oindex{I., Innere Stadt@\textbf{I., Innere Stadt}|pwk}Wien 1/1, 6. 5. 05, 11–12N\nobreak{}«.  \newline{}Ordnung: mit Bleistift von unbekannter Hand nummeriert:
                                    »200« }\toendnotes[C]{\smallbreak}\pstart{}{\pb}Herrn D\textsuperscript{r} Arthur Schnitzler\pend{}\pstart{}Wien XVIII.\oindex{XVIII., Waehring@\textbf{XVIII., Währing}|pw}\pend{}\pstart{}Spoettelgaſse 7\oindex{Edmund-Weiss-Gasse@\textbf{Edmund-Weiß-Gasse}|pw}\pend{}{\bigskip}\pstart
           \raggedleft{}{\pb}6/5 05\pend
           \pstart
           Lieber – wir wohnen schon Pötzleinsdorferstraße 88\oindex{Poetzleinsdorferstrasse@\textbf{Pötzleinsdorferstraße}|pw}. Spaziergänge, Sommerpläne u. s. w.
               können jetzt besprochen werden. Nach dem \label{K_L03408-1v}\edtext{Sommernachtstraum\pwindex{\textcolor{red}{\textsuperscript{XXXX1 indx}}!Sommernachtstraum. Komoedie in fuenf Aufzuegen1843@\strich\emph{Ein Sommernachtstraum. Komödie in fünf Aufzügen} {[}1843{]}|pw}}{\lemma{\textnormal{\emph{Sommernachtstraum}}}\Cendnote{\textnormal{Das Stück\pwindex{\textcolor{red}{\textsuperscript{XXXX1 indx}}!Sommernachtstraum. Komoedie in fuenf Aufzuegen1843@\strich\emph{Ein Sommernachtstraum. Komödie in fünf Aufzügen} {[}1843{]}|pwkv} in der Inszenierung von Max Reinhardt\pwindex{Reinhardt, Max 09.09.1873 – 30.10.1943@\textsc{Reinhardt, Max} (09.09.1873 – 30.10.1943), \emph{Theaterleiter, Regisseur, Schauspieler}|pwk} wurde in Wien\oindex{Wien@\textbf{Wien}|pwk} erstmals am 20. 5. 1905 beim Gastspiel des \emph{Kleinen}\orgindex{Kleines Theater@Kleines Theater|pwk} und des \emph{Neuen Theaters}\orgindex{Neues Theater@Neues Theater|pwk} am Theater an der Wien\oindex{Theater an der Wien@\textbf{Theater an der Wien}|pwk} gegeben. Schnitzler\pwindex{Schnitzler, Arthur 15.05.1862 – 21.10.1931@\textsc{Schnitzler, Arthur} (15.05.1862 – 21.10.1931), \emph{Schriftsteller, Mediziner}|pwk} besuchte die Aufführung, vgl. A. S.: \emph{Tagebuch}, 20. 5. 1905.}}}\label{K_L03408-1h} wollen
               wir nach Maria Zell\oindex{Mariazell@\textbf{Mariazell}|pw}. (Ersatz
               für Florenz\oindex{Florenz@\textbf{Florenz}|pw}, das aus Zeitmangel entfiel)
               Vielleicht machen wir die Parthie zu viert\pwindex{Schnitzler, Olga 17.01.1882 – 13.01.1970@\textsc{Schnitzler, Olga} (17.01.1882 – 13.01.1970), \emph{Schauspielerin, Sängerin}|pwv}\pwindex{Salten, Ottilie 07.03.1868 – 22.06.1942@\textsc{Salten, Ottilie} (07.03.1868 – 22.06.1942), \emph{Schauspielerin}|pwv}, wie’s ja besprochen war? \pend
           \pstart
           Schreiben Sie, wennn man Sie am besten trifft, und wann Ihre Frau\pwindex{Schnitzler, Olga 17.01.1882 – 13.01.1970@\textsc{Schnitzler, Olga} (17.01.1882 – 13.01.1970), \emph{Schauspielerin, Sängerin}|pwv} am wenigsten gestört wird. Wir wollen
               bald einmal Vormittag oder Nachmittag zu Ihnen. – Die gewünschten \label{K_L03408-66v}\edtext{12 Exemplare}{\lemma{\textnormal{\emph{12 Exemplare}}}\Cendnote{\textnormal{vgl. Arthur Schnitzler an Felix Salten, 29. 4. 1905}}}\label{K_L03408-66h} haben Sie wol schon erhalten? \pend
           \pstart
           Herzlich Ihr
               {\\[\baselineskip]}\spacefill\mbox{S. }\pend
           \leftskip=0em{}
         
         \endnumbering\mylabel{h}\end{ledgroupsized}\begin{anhang}\end{anhang}\newcommand{\dateiname}{L03408}\newcommand{\titel}{Felix Salten an Arthur Schnitzler, 6. 5. 1905}\newcommand{\editorInnen}{Martin Anton Müller und Laura Untner}%% latex-leseansicht-abspann.tex
%% Abspann für die Leseansicht.
%% Der Schalter \ifkorrekturansicht ist bereits durch den Vorspann gesetzt.

%% latex-abspann.tex
%% Gemeinsamer Abspann für Korrekturansicht und Leseansicht.
%% Setzt den Schalter \ifkorrekturansicht voraus (gesetzt in den
%% einbindenden Dateien latex-korrekturansicht-abspann.tex bzw.
%% latex-leseansicht-abspann.tex).
%% ---------------------------------------------------------------

\normalsize

% Das esempio-Environment wird nur in der Leseansicht benötigt
\ifkorrekturansicht\else
\newenvironment{esempio}[3]%
{
    \vspace{1.5ex}
    \rlap{\underline{#1}}
    \par
    \setlength{\parindent}{0cm}
    \nopagebreak
    \leftskip=#2cm
    \rightskip=#3cm
}
{
    \par
}
\fi

\doendnotes{C}
\bigskip
\vfill

\clearpage

\footnotesize

\ifkorrekturansicht
  \lohead{\textsc{register}}
\fi

% theindex-Environment neu definieren ohne reledmac
\makeatletter
\renewenvironment{theindex}{%
  \ifkorrekturansicht
    \section*{\indexname}%
  \else
    \subsubsection*{Index der erwähnten Entitäten}%
  \fi
  \setlength{\parindent}{0pt}%
  \setlength{\parskip}{0pt plus 0.3pt}%
  \let\item\@idxitem
}{%
  \ifkorrekturansicht\clearpage\fi
}
\makeatother

\IfFileExists{\jobname-pw.ind}{\input{\jobname-pw.ind}}{}

% Quellenangabe nur in der Leseansicht
\ifkorrekturansicht\else
% Fallback-Definitionen, falls die .tex-Datei \titel etc. nicht gesetzt hat
\providecommand{\titel}{}
\providecommand{\editorInnen}{}
\providecommand{\dateiname}{\jobname}

\vspace{3cm}

\vfill

\footnotesize
\textsc{Quelle}: \titel. Herausgegeben von {\editorInnen}. In: \emph{Arthur Schnitzler: Briefwechsel mit Autorinnen und Autoren}.
 Digitale Edition, https://schnitzler-briefe.acdh.oeaw.ac.at/{\dateiname}.html (Stand \today)
\fi

\end{document}


      