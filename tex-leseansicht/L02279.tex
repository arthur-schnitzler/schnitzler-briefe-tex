%% latex-leseansicht-vorspann.tex
%% Vorspann für die Leseansicht.
%% Lädt die gemeinsame Datei latex-vorspann.tex mit nicht gesetztem Schalter.

\newif\ifkorrekturansicht
\korrekturansichtfalse

\input{../tex-inputs/latex-vorspann}


         
         \renewcommand{\erwaehntePersonen}{Personen: Titus Livius}
         \renewcommand{\erwaehnteOrte}{Orte: Griechenland, Ostia Antica, Rom, Wien}
         \renewcommand{\erwaehnteWerke}{Werke: Das Ende des Judas}
               \section[Robert Adam an Arthur Schnitzler, 5. 11. 1917]{ Robert Adam an Arthur Schnitzler, 5. 11. 1917}\nopagebreak\mylabel{v}\rehead{ }\begin{ledgroupsized}[t]{13cm}\normalsize\beginnumbering \toendnotes[C]{\smallbreak\pagebreak[2]} \Standort{CUL, Schnitzler, B 1.}
\physDesc{Brief, 1 Blatt, 4 Seiten
\newline{}Handschrift: schwarze Tinte, deutsche Kurrent
\newline{}Schnitzler: 1) mit Bleistift beschriftet: »\textsc{Adam}«  2) mit rotem Buntstift eine Unterstreichung\newline{}Ordnung: von unbekannter Hand nummeriert: »1« }\Standort{Wien, Österreichische Nationalbibliothek, Cod.ser. 52.263, 203 recto – 204 verso.}
\physDesc{Brief, maschinelle Abschrift
\newline{}Schreibmaschine}\pstart
           \raggedleft{}{\pb}Wien\oindex{Wien@\textbf{Wien}|pw}, den 5. Nov. 1917\pend
           \pstart\center{}Hochverehrter Herr Doktor!\pend\pstart
           Ich bin, von der Amtsarbeit lange aufgehalten, endlich mit den Änderungen am »Juda\pwindex{Adam, Robert 20.04.1877 – 16.10.1961@\textsc{Adam, Robert} (20.04.1877 – 16.10.1961), \emph{Schriftsteller, Richter}!Ende des JudasNone@\strich\emph{Das Ende des Judas} {[}None{]}|pw}« (dem man vielleicht auch den Titel: »Der
                    Herr naht!« geben könnte) und mit den Strichen mit mir in’s Reine gekommen. Die
                    Klarſtellung der Perſon des Juda gleich in der erſten Szene (die doch wohl die
                    erſte bleiben muß) hat ſich ohne beſondere Schwierigkeit bewerkſtelligen laſſen
                    und hat eine ziemliche Kürzung des Eingangsdialogs zur Folge, zwingt aber leider
                    auch zur Ausſcheidung mancher charakteriſtiſchen Züge. Mir will es auch
                    ſcheinen, \strikeout{daß} als ob durch dieſe frühzeitige
                    Enthüllung das über die Geſtalt gebreitete myſteriöſe Dunkel etwas lichtfleckig
                    würde und \strikeout{daß} dadurch manche Stellen folgender
                    Szenen (beſonders der Verſammlungsſzene im verfallenen Hauſe und der
                    Schlußſzene) {\pb}an Wirkung ein wenig
                    einbüßten. Vielleicht irre ich. Jedenfalls teile ich Ihre Anſicht, daß die
                    ſofort vorgenommene Feſtſtellung der Identität des Juda mit dem Judas, da ſie
                    das Verſtändnis des Publikums fördert, der Bühnenwirkſamkeit des ganzen Stückes
                    von Nutzen iſt. Ob die Änderung bei einer ſpäteren Buchausgabe beizubehalten
                    wäre, iſt eine weitere Frage, deren Beantwortung leider in abſehbarer Zeit nicht
                    dringlich werden dürfte.\pend
           \pstart
           Die zweite Szene (in Oſtia\oindex{Ostia Antica@\textbf{Ostia Antica}|pw}) und die ſechſte
                    (die Verſammlungsſzene) habe ich tüchtig zuſammengeſtrichen, indem ich alles
                    das, was ſich auf die Differenzen zwiſchen den Judenchriſten und dem
                    pauliniſchen Chriſtentum bezieht, alle Streiterei um Revier und Beſchneidung und
                    dergl., einfach eliminierte. Dadurch würde einem Leſer gewiß große Unklarheit
                    geſchaffen, aber das Theaterpublikum dürfte darüber hinwegſehen; in jedem Falle
                    wird auf dieſe Weise \substVorne{}\textsuperscript{ſind}\substDazwischen{}nicht\substHinten{} nur vieles, was langweilt, aus dem Wege geſchafft und eine größere
                    Konzentration des Intereſſes erzielt, ſondern auch – {\pb}was nicht zu verachten iſt – der
                    ſchwerſte Zenſuranſtoß beſeitigt. Damit iſt zugleich die Möglichkeit ſtarker
                    Kürzung der \textsc{Simon-Hermon-Szene} (Gaſthaus) gegeben. Nur
                    zu einer Verſtümmelung der \textsc{Hermon-Chloe}-Szene, die mir
                    ſehr an’s Herz gewachſen iſt, habe ich den Mut nicht gefunden. Dieſe
                    Schächterarbeit möchte ich, falls ſie unumgänglich nötig iſt, dem Dramaturgen
                    überlaſſen, der ja doch böſe wäre, wenn ihm nichts zu tun übrig bliebe.\pend
           \pstart
           Was die von Ihnen berührten Modernismen und Fremdworte betrifft, ſo laſſen ſich
                    manche gewiß ohne Weiteres vermeiden, und ich habe keinen Augenblick gezögert,
                    das Wort »inſipid« durch »abgeſchmackt« zu erſetzen. Andere aber müſſen, meine
                    ich, doch ſtehen bleiben; ich wüßte z. Beiſp. nicht recht, wie ich den Satz des
                    Alityr, mit dem die vorletzte Szene ſchließt: »Ich bin heut indiſponiert«
                    umändern ſollte; er iſt halt ein Schauſpieler und da muß »indiſponiert ſein« als
                        \textsc{terminus technicus} hingenommen werden; auch
                    »multiplizieren« läßt ſich ſchwer verdeutſchen. Daß ich oft abſichtlich moderne
                    Redewendungen brauche, haben Sie ja, hochverehrter Herr Doktor, be{\pb}merkt, und ich möchte nur beifügen,
                    daß ich es juſt bei einem in der römiſchen\oindex{Rom@\textbf{Rom}|pw}
                    Kaiſerzeit ſpielenden Stücke für direkt ratſam halte, damit nicht zu kargen; es
                    ſoll dadurch vermieden werden, daß die Römer\oindex{Rom@\textbf{Rom}|pw}
                    der alten Römer\oindex{Rom@\textbf{Rom}|pw}-Stücke, Livius\pwindex{Livius, Titus .0059 – 17@\textsc{Livius, Titus} (.0059 – 17), \emph{Geschichtsschreiber}|pw}-gezeugte Puppen von hartem Holz und Korn, in
                    traditioneller deutſcher Unlebendigkeit daſtehen; es ſoll gewiſſermaßen immer
                    wieder betont werden, daß dieſe Leute modern waren, wie wir modern ſind.
                    Überdies iſt der Fremdwörtergebrauch gar kein Anachronismus, da damals das
                    »gebildete« Lateiniſch mit griechiſchen\oindex{Griechenland@\textbf{Griechenland}|pw}
                    Fachausdrücken und Modewörtern und das Griechiſch\oindex{Griechenland@\textbf{Griechenland}|pw} der Orientalen mit orientaliſchen Wendungen und Floſkeln
                    durchſetzt war. Und daß ſchließlich meine alten Römer\oindex{Rom@\textbf{Rom}|pw} und Juden gute Wien\oindex{Wien@\textbf{Wien}|pw}er ſind,
                    damit halt ich gar nicht hinter dem Berge.\pend
           \pstart
           Sollten Sie, hochverehrter Herr Doktor, wirklich, ohne ſich ein Opfer
                    aufzuerlegen, Zeit finden, mit mir die Einzelheiten durchzuſprechen, ſo wäre ich
                    Ihnen außerordentlich dankbar.\pend
           \pstart
           Mit den ergebenſten Grüßen\pend
           \pstart
           Ihr{\\[\baselineskip]}\spacefill\mbox{Robert Adam}\pend
           \leftskip=0em{}
         
         \endnumbering\mylabel{h}\end{ledgroupsized}  \newcommand{\dateiname}{L02279}\newcommand{\titel}{Robert Adam an Arthur Schnitzler, 5. 11. 1917}\newcommand{\editorInnen}{Martin Anton Müller und Gerd-Hermann Susen}%% latex-leseansicht-abspann.tex
%% Abspann für die Leseansicht.
%% Der Schalter \ifkorrekturansicht ist bereits durch den Vorspann gesetzt.

%% latex-abspann.tex
%% Gemeinsamer Abspann für Korrekturansicht und Leseansicht.
%% Setzt den Schalter \ifkorrekturansicht voraus (gesetzt in den
%% einbindenden Dateien latex-korrekturansicht-abspann.tex bzw.
%% latex-leseansicht-abspann.tex).
%% ---------------------------------------------------------------

\normalsize

% Das esempio-Environment wird nur in der Leseansicht benötigt
\ifkorrekturansicht\else
\newenvironment{esempio}[3]%
{
    \vspace{1.5ex}
    \rlap{\underline{#1}}
    \par
    \setlength{\parindent}{0cm}
    \nopagebreak
    \leftskip=#2cm
    \rightskip=#3cm
}
{
    \par
}
\fi

\doendnotes{C}
\bigskip
\vfill

\clearpage

\footnotesize

\ifkorrekturansicht
  \lohead{\textsc{register}}
\fi

% theindex-Environment neu definieren ohne reledmac
\makeatletter
\renewenvironment{theindex}{%
  \ifkorrekturansicht
    \section*{\indexname}%
  \else
    \subsubsection*{Index der erwähnten Entitäten}%
  \fi
  \setlength{\parindent}{0pt}%
  \setlength{\parskip}{0pt plus 0.3pt}%
  \let\item\@idxitem
}{%
  \ifkorrekturansicht\clearpage\fi
}
\makeatother

\IfFileExists{\jobname-pw.ind}{\input{\jobname-pw.ind}}{}

% Quellenangabe nur in der Leseansicht
\ifkorrekturansicht\else
% Fallback-Definitionen, falls die .tex-Datei \titel etc. nicht gesetzt hat
\providecommand{\titel}{}
\providecommand{\editorInnen}{}
\providecommand{\dateiname}{\jobname}

\vspace{3cm}

\vfill

\footnotesize
\textsc{Quelle}: \titel. Herausgegeben von {\editorInnen}. In: \emph{Arthur Schnitzler: Briefwechsel mit Autorinnen und Autoren}.
 Digitale Edition, https://schnitzler-briefe.acdh.oeaw.ac.at/{\dateiname}.html (Stand \today)
\fi

\end{document}


      