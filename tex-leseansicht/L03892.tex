%% latex-leseansicht-vorspann.tex
%% Vorspann für die Leseansicht.
%% Lädt die gemeinsame Datei latex-vorspann.tex mit nicht gesetztem Schalter.

\newif\ifkorrekturansicht
\korrekturansichtfalse

\input{../tex-inputs/latex-vorspann}


\section[Sigmund Freud: Widmungsexemplar von Die Zukunft einer Illusion, {[}zwischen 15.? und 28. 11. 1927{]}]{L03892 Sigmund Freud: Widmungsexemplar von Die Zukunft einer Illusion, {[}zwischen 15.? und 28. 11. 1927{]}}
\nopagebreak\mylabel{L03892v}
\rehead{ }\normalsize\beginnumbering\briefempfaengerindex{Schnitzler, Arthur@\textsc{Schnitzler, Arthur}!zzzFreud, Sigmund@\emph{von Sigmund Freud}!1927-11-281@{{[}zwischen 15.? und 28. 11. 1927{]}}|(be}
\toendnotes[C]{\smallbreak\pagebreak[2]}
\correspDesc{Versand  durch Sigmund Freud im Zeitraum [zwischen 15.? und 28. 11. 1927] in Wien
\newline{}Erhalt  durch Arthur Schnitzler im Zeitraum [zwischen 15.? und 28. 11. 1927] in Wien}\toendnotes[C]{\smallbreak}
\buchAlsQuelle{Sigmund Freud: \emph{Briefe an Arthur Schnitzler.}Herausgegeben von Henry Schnitzler In: \emph{Neue deutsche Rundschau}, Jg. 66 (Januar 1955) Nr. 1, S. 106.}\toendnotes[C]{\smallbreak}
\pstart
           \noindent{}\label{K_L03891-1v}\edtext{Arthur Schnitzler als geringe Gegengabe, d. Verf. Nov. 1927.\pwindex{Freud, Sigmund 6.\,5.\,1856 Pribor – 23.\,9.\,1939 London@\textsc{Freud, Sigmund} (6.\,5.\,1856 Pribor – 23.\,9.\,1939 London), \emph{Psychoanalytiker}!Zukunft einer Illusion@\strich\emph{Die Zukunft einer Illusion}|pwv}}{\lemma{\textnormal{\emph{Arthur … 1927.}}}\Cendnote{\textnormal{Das Widmungsexemplar
                  von \emph{Die Zukunft einer Illusion}\pwindex{Freud, Sigmund 6.\,5.\,1856 Pribor – 23.\,9.\,1939 London@\textsc{Freud, Sigmund} (6.\,5.\,1856 Pribor – 23.\,9.\,1939 London), \emph{Psychoanalytiker}!Zukunft einer Illusion@\strich\emph{Die Zukunft einer Illusion}|pwk} (Leipzig, Wien, Zürich: \emph{Internationaler Psychoanalytischer Verlag}\orgindex{Internationaler Psychoanalytischer Verlag@Internationaler Psychoanalytischer Verlag|pwk}{ }1927) ist nur durch eine Anmerkung 
                  in der Edition (1955) von Heinrich Schnitzler\pwindex{Schnitzler, Heinrich 9.\,8.\,1902 Hinterbrühl – 12.\,7.\,1982 Wien@\textsc{Schnitzler, Heinrich} (9.\,8.\,1902 Hinterbrühl – 12.\,7.\,1982 Wien), \emph{Regisseur, Schauspieler}|pwk} belegt. Der
                  Verbleib ist ungeklärt. Das Buch erschien Mitte November, am 18. 11. 1927
                  lässt sich der Empfang eines Widmungsexemplars durch Max Eitingon\pwindex{Eitingon, Max 26.\,6.\,1881 Mogilev – 30.\,7.\,1943 Jerusalem@\textsc{Eitingon, Max} (26.\,6.\,1881 Mogilev – 30.\,7.\,1943 Jerusalem), \emph{Mediziner, Psychoanalytiker}|pwk} belegen.
                  Schnitzler vermerkte die Lektüre
                  für den Vorabend des 29. 11. 1927, 
                  womit auch eine zeitliche Eingrenzung nach hinten möglich ist.}}}\label{K_L03891-1}\pend
           \selectlanguage{ngerman}\endnumbering\briefempfaengerindex{Schnitzler, Arthur@\textsc{Schnitzler, Arthur}!zzzFreud, Sigmund@\emph{von Sigmund Freud}!1927-11-151@{{[}zwischen 15.? und 28. 11. 1927{]}}|)be}\mylabel{L03892h}
\begin{anhang}
\end{anhang}\newcommand{\dateiname}{L03892}\newcommand{\titel}{Sigmund Freud: Widmungsexemplar von Die Zukunft einer Illusion, [zwischen 15.? und 28. 11. 1927]}\newcommand{\editorInnen}{Selma Jahnke und Martin Anton Müller}%% latex-leseansicht-abspann.tex
%% Abspann für die Leseansicht.
%% Der Schalter \ifkorrekturansicht ist bereits durch den Vorspann gesetzt.

%% latex-abspann.tex
%% Gemeinsamer Abspann für Korrekturansicht und Leseansicht.
%% Setzt den Schalter \ifkorrekturansicht voraus (gesetzt in den
%% einbindenden Dateien latex-korrekturansicht-abspann.tex bzw.
%% latex-leseansicht-abspann.tex).
%% ---------------------------------------------------------------

\normalsize

% Das esempio-Environment wird nur in der Leseansicht benötigt
\ifkorrekturansicht\else
\newenvironment{esempio}[3]%
{
    \vspace{1.5ex}
    \rlap{\underline{#1}}
    \par
    \setlength{\parindent}{0cm}
    \nopagebreak
    \leftskip=#2cm
    \rightskip=#3cm
}
{
    \par
}
\fi

\doendnotes{C}
\bigskip
\vfill

\clearpage

\footnotesize

\ifkorrekturansicht
  \lohead{\textsc{register}}
\fi

% theindex-Environment neu definieren ohne reledmac
\makeatletter
\renewenvironment{theindex}{%
  \ifkorrekturansicht
    \section*{\indexname}%
  \else
    \subsubsection*{Index der erwähnten Entitäten}%
  \fi
  \setlength{\parindent}{0pt}%
  \setlength{\parskip}{0pt plus 0.3pt}%
  \let\item\@idxitem
}{%
  \ifkorrekturansicht\clearpage\fi
}
\makeatother

\IfFileExists{\jobname-pw.ind}{\input{\jobname-pw.ind}}{}

% Quellenangabe nur in der Leseansicht
\ifkorrekturansicht\else
% Fallback-Definitionen, falls die .tex-Datei \titel etc. nicht gesetzt hat
\providecommand{\titel}{}
\providecommand{\editorInnen}{}
\providecommand{\dateiname}{\jobname}

\vspace{3cm}

\vfill

\footnotesize
\textsc{Quelle}: \titel. Herausgegeben von {\editorInnen}. In: \emph{Arthur Schnitzler: Briefwechsel mit Autorinnen und Autoren}.
 Digitale Edition, https://schnitzler-briefe.acdh.oeaw.ac.at/{\dateiname}.html (Stand \today)
\fi

\end{document}


