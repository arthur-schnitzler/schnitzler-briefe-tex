%% latex-leseansicht-vorspann.tex
%% Vorspann für die Leseansicht.
%% Lädt die gemeinsame Datei latex-vorspann.tex mit nicht gesetztem Schalter.

\newif\ifkorrekturansicht
\korrekturansichtfalse

\input{../tex-inputs/latex-vorspann}


\section[Georg Brandes an Arthur Schnitzler, 7. 1. 1899]{L00876 Georg Brandes an Arthur Schnitzler, 7. 1. 1899}
\nopagebreak\mylabel{L00876v}
\rehead{ }\normalsize\beginnumbering\briefempfaengerindex{Schnitzler, Arthur@\textsc{Schnitzler, Arthur}!zzzBrandes, Georg@\emph{von Georg Brandes}!1899-01-071@{7. 1. 1899}|(be}
\toendnotes[C]{\smallbreak\pagebreak[2]}
\correspDesc{Versand  durch Georg Brandes am 7. 1. 1899 \textbf{Ort fehlend} 
\newline{}Erhalt  durch Arthur Schnitzler im Zeitraum [7. 1. 1899
                  – 11. 1. 1899?] in Wien}\toendnotes[C]{\smallbreak}
\Standort{CUL, Schnitzler, B 17.}
\physDesc{Brief, 1 Blatt, 4 Seiten, 2012 Zeichen
\newline{}Handschrift: Bleistift, lateinische Kurrent
\newline{}Ordnung: mit Bleistift von unbekannter Hand nummeriert:
                                    »11« }
\buchAbdrucke{\weitereDrucke{Georg Brandes, Arthur Schnitzler: \emph{Ein Briefwechsel}. Herausgegeben von Kurt Bergel. Bern: \emph{Francke} 1956, S. 69–70.} }\toendnotes[C]{\smallbreak}
\pstart
           \raggedleft{}{\pb}Kopenhagen\oindex{Kopenhagen@\textbf{Kopenhagen}, \emph{Hauptstadt}|pw}{ }7 Jan. 99\pend
           
\pstart{}Lieber Dr. Schnitzler, sehr guter Freund\pend\vspace{0.5em}
\pstart
           Haben Sie Dank für Ihre Zeilen. Was habe ich nicht alles erlebt seit ich Sie sah.
               Jetzt liege ich wieder zu Bett; die Venenentzündung ist zurückgekehrt.\pend
           
\pstart
           Ich blieb ein halbes Jahr in Italien\oindex{Italien@\textbf{Italien}|pw}, kam
               zurück, gab hier zwei Bücher aus, einen Band meiner Gedichte\pwindex{Brandes, Georg 4.\,2.\,1842 Kopenhagen – 19.\,2.\,1927 ebd.@\textsc{Brandes, Georg} (4.\,2.\,1842 Kopenhagen – 19.\,2.\,1927 ebd.)!Ungdomsvers [Jugendgedichte]@\strich\emph{Ungdomsvers [Jugendgedichte]}|pwv} (staunen Sie?) und ein Buch über einen verstorbenen Freund\pwindex{Brandes, Georg 4.\,2.\,1842 Kopenhagen – 19.\,2.\,1927 ebd.@\textsc{Brandes, Georg} (4.\,2.\,1842 Kopenhagen – 19.\,2.\,1927 ebd.)!Julius Lange@\strich\emph{Julius Lange}|pwv}, das
               hier einen sehr grossen Erfolg gehabt hat –, in 8 Tagen ausverkauft. Reiste wieder
               aus, wurde zwei Mal zurückgerufen durch Depeschen, {\pb}weil meine Mutter\pwindex{Brandes, Emilie 22.\,3.\,1818 Kopenhagen – 27.\,12.\,1898 ebd.@\textsc{Brandes, Emilie} (22.\,3.\,1818 Kopenhagen – 27.\,12.\,1898 ebd.)|pwv} krank war. Das letzte Mal war ich in
                  Polen\oindex{Polen@\textbf{Polen}|pw}, wo ich wegen meines Buches über Polen\pwindex{Brandes, Georg 4.\,2.\,1842 Kopenhagen – 19.\,2.\,1927 ebd.@\textsc{Brandes, Georg} (4.\,2.\,1842 Kopenhagen – 19.\,2.\,1927 ebd.)!Polen@\strich\emph{Polen}|pwv} (das deutsch und polnisch
               übersetzt worden) eingeladen und komisch vergöttert wurde.\pend
           
\pstart
           Zurück in einem Zug aus Lemberg\oindex{Lviv@\textbf{Lviv}|pw}. Sah meine Mutter\pwindex{Brandes, Emilie 22.\,3.\,1818 Kopenhagen – 27.\,12.\,1898 ebd.@\textsc{Brandes, Emilie} (22.\,3.\,1818 Kopenhagen – 27.\,12.\,1898 ebd.)|pwv} 14 Tage dann selbst
               krank, konnte meine Mutter\pwindex{Brandes, Emilie 22.\,3.\,1818 Kopenhagen – 27.\,12.\,1898 ebd.@\textsc{Brandes, Emilie} (22.\,3.\,1818 Kopenhagen – 27.\,12.\,1898 ebd.)|pwv}
               nicht sehen in der letzten Woche ihres Lebens und nicht an ihrer Beerdigung dasein.
               Ich habe \uline{nie einen einzigen Tag} in Kopenhagen\oindex{Kopenhagen@\textbf{Kopenhagen}, \emph{Hauptstadt}|pw} versäumt meine Mutter\pwindex{Brandes, Emilie 22.\,3.\,1818 Kopenhagen – 27.\,12.\,1898 ebd.@\textsc{Brandes, Emilie} (22.\,3.\,1818 Kopenhagen – 27.\,12.\,1898 ebd.)|pwv} zu besuchen.\pend
           
\pstart
           Und jetzt liege ich in Streit mit den Deutschen\oindex{Deutschland@\textbf{Deutschland}|pw}
               wegen der Austreibung der Dänen\oindex{Dänemark@\textbf{Dänemark}|pw} aus Schleswig\oindex{Südschleswig@\textbf{Südschleswig}, \emph{Hauptstadt}|pw}. Gibt es etwas widerlicheres als {\pb}Preussen\oindex{Preußen@\textbf{Preußen}|pw}? Nicht Frankreich\oindex{Frankreich@\textbf{Frankreich}|pw} einmal.\pend
           
\pstart
           Mit ruhiger geniessender Freude las ich Ihr \uline{Vermächtnis}\pwindex{Schnitzler, Arthur 15.\,5.\,1862 Wien – 21.\,10.\,1931 ebd.@\textsc{Schnitzler, Arthur} (15.\,5.\,1862 Wien – 21.\,10.\,1931 ebd.), \emph{Schriftsteller, Mediziner}!Vermächtnis. Schauspiel in drei Akten@\strich\emph{Das Vermächtnis. Schauspiel in drei Akten}|pw}. Es ist ein völlig originales Ding, sehr discret und vornehm, tief
               pessimistisch und human. (Kennen Sie zufällig eine kleine Erzählung von Huysmans\pwindex{Huysmans, Joris-Karl 5.\,2.\,1848 Paris – 12.\,5.\,1907 ebd.@\textsc{Huysmans, Joris-Karl} (5.\,2.\,1848 Paris – 12.\,5.\,1907 ebd.), \emph{Schriftsteller}|pw}{ }Un
                  dilemme\pwindex{Huysmans, Joris-Karl 5.\,2.\,1848 Paris – 12.\,5.\,1907 ebd.@\textsc{Huysmans, Joris-Karl} (5.\,2.\,1848 Paris – 12.\,5.\,1907 ebd.), \emph{Schriftsteller}!Dilemma@\strich\emph{Ein Dilemma}|pw} die behandelt ein ähnliches Thema, nur viel gröber oder richtiger
               ganz anders, aber es ist da ein bischen Verwandtschaft).\pend
           
\pstart
           Es ist nur Schade, dass das Stück\pwindex{Schnitzler, Arthur 15.\,5.\,1862 Wien – 21.\,10.\,1931 ebd.@\textsc{Schnitzler, Arthur} (15.\,5.\,1862 Wien – 21.\,10.\,1931 ebd.), \emph{Schriftsteller, Mediziner}!Vermächtnis. Schauspiel in drei Akten@\strich\emph{Das Vermächtnis. Schauspiel in drei Akten}|pwv} so ganz und gar traurig ist, dann wird es nicht so viel Bühnenerfolg
               haben können, {\pb}wie ich es
               wünschte. Der Vater ist wunderbar gezeichnet. Aber überhaupt ich hab Ihr Talent so
               lieb. Etwas freut mich schon, weil es von Ihnen ist.\pend
           
\pstart
           Warum lässt doch unser Freund Beer Hofmann\pwindex{Beer-Hofmann, Richard 11.\,7.\,1866 Wien – 26.\,9.\,1945 New York City@\textsc{Beer-Hofmann, Richard} (11.\,7.\,1866 Wien – 26.\,9.\,1945 New York City), \emph{Schriftsteller}|pw} nie
               von sich hören? Ist er ein bischen faul? Er ist doch ein so feiner Mensch.\pend
           
\pstart
           Denken Sie, was es heisst für einen Mann von meinem Temperament still zu liegen,
               Geduld haben zu sollen und \uline{wieder}, nachdem ich Ein
               Mal ein halbes Jahr so verlor.\pend
           
\pstart
           Behalten Sie mich lieb\pend
           
\pstart
           Ihr ergebener{\\[\baselineskip]}\spacefill\mbox{Georg Brandes}\pend
           \leftskip=0em{}\selectlanguage{ngerman}\endnumbering\briefempfaengerindex{Schnitzler, Arthur@\textsc{Schnitzler, Arthur}!zzzBrandes, Georg@\emph{von Georg Brandes}!1899-01-071@{7. 1. 1899}|)be}\mylabel{L00876h}  \newcommand{\dateiname}{L00876}\newcommand{\titel}{Georg Brandes an Arthur Schnitzler, 7. 1. 1899}\newcommand{\editorInnen}{Martin Anton Müller und Gerd-Hermann Susen}%% latex-leseansicht-abspann.tex
%% Abspann für die Leseansicht.
%% Der Schalter \ifkorrekturansicht ist bereits durch den Vorspann gesetzt.

%% latex-abspann.tex
%% Gemeinsamer Abspann für Korrekturansicht und Leseansicht.
%% Setzt den Schalter \ifkorrekturansicht voraus (gesetzt in den
%% einbindenden Dateien latex-korrekturansicht-abspann.tex bzw.
%% latex-leseansicht-abspann.tex).
%% ---------------------------------------------------------------

\normalsize

% Das esempio-Environment wird nur in der Leseansicht benötigt
\ifkorrekturansicht\else
\newenvironment{esempio}[3]%
{
    \vspace{1.5ex}
    \rlap{\underline{#1}}
    \par
    \setlength{\parindent}{0cm}
    \nopagebreak
    \leftskip=#2cm
    \rightskip=#3cm
}
{
    \par
}
\fi

\doendnotes{C}
\bigskip
\vfill

\clearpage

\footnotesize

\ifkorrekturansicht
  \lohead{\textsc{register}}
\fi

% theindex-Environment neu definieren ohne reledmac
\makeatletter
\renewenvironment{theindex}{%
  \ifkorrekturansicht
    \section*{\indexname}%
  \else
    \subsubsection*{Index der erwähnten Entitäten}%
  \fi
  \setlength{\parindent}{0pt}%
  \setlength{\parskip}{0pt plus 0.3pt}%
  \let\item\@idxitem
}{%
  \ifkorrekturansicht\clearpage\fi
}
\makeatother

\IfFileExists{\jobname-pw.ind}{\input{\jobname-pw.ind}}{}

% Quellenangabe nur in der Leseansicht
\ifkorrekturansicht\else
% Fallback-Definitionen, falls die .tex-Datei \titel etc. nicht gesetzt hat
\providecommand{\titel}{}
\providecommand{\editorInnen}{}
\providecommand{\dateiname}{\jobname}

\vspace{3cm}

\vfill

\footnotesize
\textsc{Quelle}: \titel. Herausgegeben von {\editorInnen}. In: \emph{Arthur Schnitzler: Briefwechsel mit Autorinnen und Autoren}.
 Digitale Edition, https://schnitzler-briefe.acdh.oeaw.ac.at/{\dateiname}.html (Stand \today)
\fi

\end{document}


