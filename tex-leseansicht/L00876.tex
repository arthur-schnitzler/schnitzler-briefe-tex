%% latex-korrekturansicht-vorspann.tex
%% Vorspann für die Korrekturansicht.
%% Lädt die gemeinsame Datei latex-vorspann.tex mit gesetztem Schalter.

\newif\ifkorrekturansicht
\korrekturansichttrue

\input{../tex-inputs/latex-vorspann}


\section[Georg Brandes an Arthur Schnitzler, 7. 1. 1899]{L00876 Georg Brandes an Arthur Schnitzler, 7. 1. 1899}
\nopagebreak\mylabel{L00876v}
\rehead{ }\normalsize\beginnumbering\briefempfaengerindex{Schnitzler, Arthur@\textsc{Schnitzler, Arthur}!zzzBrandes, Georg@\emph{von Georg Brandes}!1899-01-071@{7. 1. 1899}|(be}
\toendnotes[C]{\smallbreak\pagebreak[2]}\Standort{CUL, Schnitzler, B 17.}
\physDesc{Brief, 1 Blatt, 4 Seiten, 2012 Zeichen
\newline{}Handschrift: Bleistift, lateinische Kurrent
\newline{}Ordnung: mit Bleistift von unbekannter Hand nummeriert:
                                    »11« }
\buchAbdrucke{\weitereDrucke{Georg Brandes, Arthur Schnitzler: \emph{Ein Briefwechsel}. Bern: \emph{Francke} 1956, S. 69–70.} }\toendnotes[C]{\smallbreak}
\pstart
           \raggedleft{}{\pb}Kopenhagen\oindex{Kopenhagen@\textbf{Kopenhagen}, \emph{P.PPLC}|pw}{ }7 Jan. 99\pend
           
\pstart{}Lieber Dr. Schnitzler, sehr guter Freund\pend\vspace{0.5em}
\pstart
           Haben Sie Dank für Ihre Zeilen. Was habe ich nicht alles erlebt seit ich Sie sah.
               Jetzt liege ich wieder zu Bett; die Venenentzündung ist zurückgekehrt.\pend
           
\pstart
           Ich blieb ein halbes Jahr in Italien\oindex{Italien@\textbf{Italien}, \emph{A.PCLI}|pw}, kam
               zurück, gab hier zwei Bücher aus, einen Band meiner Gedichte\pwindex{Ungdomsvers [Jugendgedichte]@\emph{Ungdomsvers [Jugendgedichte]}|pwv} (staunen Sie?) und ein Buch über einen verstorbenen Freund\pwindex{Julius Lange@\emph{Julius Lange}|pwv}, das
               hier einen sehr grossen Erfolg gehabt hat –, in 8 Tagen ausverkauft. Reiste wieder
               aus, wurde zwei Mal zurückgerufen durch Depeschen, {\pb}weil meine Mutter\pwindex{Brandes, Emilie 22.03.1818 – 27.12.1898@\textsc{Brandes, Emilie} (22.03.1818 – 27.12.1898)|pwv} krank war. Das letzte Mal war ich in
                  Polen\oindex{Polen@\textbf{Polen}, \emph{A.PCLI}|pw}, wo ich wegen meines Buches über Polen\pwindex{Polen@\emph{Polen}|pwv} (das deutsch und polnisch
               übersetzt worden) eingeladen und komisch vergöttert wurde.\pend
           
\pstart
           Zurück in einem Zug aus Lemberg\oindex{Lviv@\textbf{Lviv}, \emph{P.PPLA}|pw}. Sah meine Mutter\pwindex{Brandes, Emilie 22.03.1818 – 27.12.1898@\textsc{Brandes, Emilie} (22.03.1818 – 27.12.1898)|pwv} 14 Tage dann selbst
               krank, konnte meine Mutter\pwindex{Brandes, Emilie 22.03.1818 – 27.12.1898@\textsc{Brandes, Emilie} (22.03.1818 – 27.12.1898)|pwv}
               nicht sehen in der letzten Woche ihres Lebens und nicht an ihrer Beerdigung dasein.
               Ich habe \uline{nie einen einzigen Tag} in Kopenhagen\oindex{Kopenhagen@\textbf{Kopenhagen}, \emph{P.PPLC}|pw} versäumt meine Mutter\pwindex{Brandes, Emilie 22.03.1818 – 27.12.1898@\textsc{Brandes, Emilie} (22.03.1818 – 27.12.1898)|pwv} zu besuchen.\pend
           
\pstart
           Und jetzt liege ich in Streit mit den Deutschen\oindex{Deutschland@\textbf{Deutschland}, \emph{A.PCLI}|pw}
               wegen der Austreibung der Dänen\oindex{Daenemark@\textbf{Dänemark}, \emph{A.PCLI}|pw} aus Schleswig\oindex{Suedschleswig@\textbf{Südschleswig}, \emph{P.PPLA3}|pw}. Gibt es etwas widerlicheres als {\pb}Preussen\oindex{Preussen@\textbf{Preußen}, \emph{Land (A.LND)}|pw}? Nicht Frankreich\oindex{Frankreich@\textbf{Frankreich}, \emph{A.PCLI}|pw} einmal.\pend
           
\pstart
           Mit ruhiger geniessender Freude las ich Ihr \uline{Vermächtnis}\pwindex{Vermaechtnis. Schauspiel in drei Akten@\emph{Das Vermächtnis. Schauspiel in drei Akten}|pw}. Es ist ein völlig originales Ding, sehr discret und vornehm, tief
               pessimistisch und human. (Kennen Sie zufällig eine kleine Erzählung von Huysmans\pwindex{Huysmans, Joris-Karl 05.02.1848 – 12.05.1907@\textsc{Huysmans, Joris-Karl} (05.02.1848 – 12.05.1907), \emph{Schriftsteller/Schriftstellerin}|pw}Un
                  dilemme\pwindex{Dilemma@\emph{Ein Dilemma}|pw} die behandelt ein ähnliches Thema, nur viel gröber oder richtiger
               ganz anders, aber es ist da ein bischen Verwandtschaft).\pend
           
\pstart
           Es ist nur Schade, dass das Stück\pwindex{Vermaechtnis. Schauspiel in drei Akten@\emph{Das Vermächtnis. Schauspiel in drei Akten}|pwv} so ganz und gar traurig ist, dann wird es nicht so viel Bühnenerfolg
               haben können, {\pb}wie ich es
               wünschte. Der Vater ist wunderbar gezeichnet. Aber überhaupt ich hab Ihr Talent so
               lieb. Etwas freut mich schon, weil es von Ihnen ist.\pend
           
\pstart
           Warum lässt doch unser Freund Beer Hofmann\pwindex{Beer-Hofmann, Richard 1866-07-11 – 1945-09-26@\textsc{Beer-Hofmann, Richard} (1866-07-11 – 1945-09-26), \emph{Schriftsteller/Schriftstellerin}|pw} nie
               von sich hören? Ist er ein bischen faul? Er ist doch ein so feiner Mensch.\pend
           
\pstart
           Denken Sie, was es heisst für einen Mann von meinem Temperament still zu liegen,
               Geduld haben zu sollen und \uline{wieder}, nachdem ich Ein
               Mal ein halbes Jahr so verlor.\pend
           
\pstart
           Behalten Sie mich lieb\pend
           
\pstart
           Ihr ergebener{\\[\baselineskip]}\spacefill\mbox{Georg Brandes}\pend
           \leftskip=0em{}\selectlanguage{ngerman}\endnumbering\briefempfaengerindex{Schnitzler, Arthur@\textsc{Schnitzler, Arthur}!zzzBrandes, Georg@\emph{von Georg Brandes}!1899-01-071@{7. 1. 1899}|)be}\mylabel{L00876h}  \normalsize

\doendnotes{C}
\bigskip
\vfill

\clearpage

\footnotesize

\lohead{\textsc{register}}

% Definiere theindex-Environment komplett neu ohne reledmac
\makeatletter
\renewenvironment{theindex}{%
  \section*{\indexname}%
  \setlength{\parindent}{0pt}%
  \setlength{\parskip}{0pt plus 0.3pt}%
  \let\item\@idxitem
}{%
  \clearpage
}
\makeatother

\IfFileExists{\jobname-pw.ind}{\input{\jobname-pw.ind}}{}

\end{document}

      