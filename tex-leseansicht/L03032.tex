%% latex-leseansicht-vorspann.tex
%% Vorspann für die Leseansicht.
%% Lädt die gemeinsame Datei latex-vorspann.tex mit nicht gesetztem Schalter.

\newif\ifkorrekturansicht
\korrekturansichtfalse

\input{../tex-inputs/latex-vorspann}


\section[ Arthur Schnitzler an Felix Salten, 30. 12. [1899?]]{L03032 Arthur Schnitzler an Felix Salten,  30. 12. [1899?]}
\nopagebreak\mylabel{L03032v}
\rehead{ }\normalsize\beginnumbering\briefempfaengerindex{Salten, Felix@\textsc{Salten, Felix}!zzzSchnitzler, Arthur@\emph{von Arthur Schnitzler}!1899-12-301@{30. 12. [1899?]}|(be}
\toendnotes[C]{\smallbreak\pagebreak[2]}
\correspDesc{Versand  durch Arthur Schnitzler am 30. 12. [1899?] in Wien
\newline{}Erhalt  durch Felix Salten im Zeitraum [30. 12. 1899 – 31. 12. 1899?] in Wien}\toendnotes[C]{\smallbreak}
\Standort{Wienbibliothek im Rathaus, ZPH 1681, 2.1.516.}
\physDesc{Karte, 196 Zeichen
\newline{}Handschrift: schwarze Tinte, deutsche Kurrent
\newline{}Ordnung: mit Bleistift von unbekannter Hand nummeriert: »34« }\toendnotes[C]{\smallbreak}
\pstart
           \noindent{}{\pb}Lieber Freund, thut mir{ }ſehr leid, dſs ich nicht länger warten
               konnte. Der morgige{ }Abend (Sylveſter) iſt \label{K_L03032-1v}\edtext{occupirt}{\lemma{\textnormal{\emph{occupirt}}}\Cendnote{\textnormal{Siehe A. S.: \emph{Tagebuch}, 31. 12. 1899.
               }}}\label{K_L03032-1}; \label{K_L03032-2v}\edtext{wegen 1.{ }ſchreib ich}{\lemma{\textnormal{\emph{wegen 1. schreib ich}}}\Cendnote{\textnormal{Das
                  Korrespondenzstück ist undatiert, lässt sich aber durch den Inhalt zeitlich am vorletzten Tag
                  eines Kalenderjahres verorten. Die Bestimmung des fraglichen Jahreswechsels selbst lässt
                  sich nicht mit letzter Gewissheit vornehmen. Schnitzlers Brief vom XXXX Auszeichnungsfehler: Dokument L03030 nicht gefunden 
                  dürfte aber die im vorliegenden Schreiben versprochene weitere Information enthalten, wodurch sich die Verortung im Jahr ergibt.}}}\label{K_L03032-2} Ihnen noch. Herzlichen
                  \textcolor{gray}{G}ruſs und alles gute zum neuen Jahr.\pend
           \pstart Ihr \spacefill\mbox{Arth Sch}\pend{}\selectlanguage{ngerman}\endnumbering\briefempfaengerindex{Salten, Felix@\textsc{Salten, Felix}!zzzSchnitzler, Arthur@\emph{von Arthur Schnitzler}!1899-12-301@{30. 12. [1899?]}|)be}\mylabel{L03032h}  \newcommand{\dateiname}{L03032}\newcommand{\titel}{Arthur Schnitzler an Felix Salten, 30. 12. [1899?]}\newcommand{\editorInnen}{Martin Anton Müller und Laura Untner}%% latex-leseansicht-abspann.tex
%% Abspann für die Leseansicht.
%% Der Schalter \ifkorrekturansicht ist bereits durch den Vorspann gesetzt.

%% latex-abspann.tex
%% Gemeinsamer Abspann für Korrekturansicht und Leseansicht.
%% Setzt den Schalter \ifkorrekturansicht voraus (gesetzt in den
%% einbindenden Dateien latex-korrekturansicht-abspann.tex bzw.
%% latex-leseansicht-abspann.tex).
%% ---------------------------------------------------------------

\normalsize

% Das esempio-Environment wird nur in der Leseansicht benötigt
\ifkorrekturansicht\else
\newenvironment{esempio}[3]%
{
    \vspace{1.5ex}
    \rlap{\underline{#1}}
    \par
    \setlength{\parindent}{0cm}
    \nopagebreak
    \leftskip=#2cm
    \rightskip=#3cm
}
{
    \par
}
\fi

\doendnotes{C}
\bigskip
\vfill

\clearpage

\footnotesize

\ifkorrekturansicht
  \lohead{\textsc{register}}
\fi

% theindex-Environment neu definieren ohne reledmac
\makeatletter
\renewenvironment{theindex}{%
  \ifkorrekturansicht
    \section*{\indexname}%
  \else
    \subsubsection*{Index der erwähnten Entitäten}%
  \fi
  \setlength{\parindent}{0pt}%
  \setlength{\parskip}{0pt plus 0.3pt}%
  \let\item\@idxitem
}{%
  \ifkorrekturansicht\clearpage\fi
}
\makeatother

\IfFileExists{\jobname-pw.ind}{\input{\jobname-pw.ind}}{}

% Quellenangabe nur in der Leseansicht
\ifkorrekturansicht\else
% Fallback-Definitionen, falls die .tex-Datei \titel etc. nicht gesetzt hat
\providecommand{\titel}{}
\providecommand{\editorInnen}{}
\providecommand{\dateiname}{\jobname}

\vspace{3cm}

\vfill

\footnotesize
\textsc{Quelle}: \titel. Herausgegeben von {\editorInnen}. In: \emph{Arthur Schnitzler: Briefwechsel mit Autorinnen und Autoren}.
 Digitale Edition, https://schnitzler-briefe.acdh.oeaw.ac.at/{\dateiname}.html (Stand \today)
\fi

\end{document}


