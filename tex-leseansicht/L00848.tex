%% latex-korrekturansicht-vorspann.tex
%% Vorspann für die Korrekturansicht.
%% Lädt die gemeinsame Datei latex-vorspann.tex mit gesetztem Schalter.

\newif\ifkorrekturansicht
\korrekturansichttrue

\input{../tex-inputs/latex-vorspann}


\section[Arthur Schnitzler an Georg Brandes, nicht abgesandt, 22. 9. 1898]{L00848 Arthur Schnitzler an Georg Brandes, nicht abgesandt,
               22. 9. 1898}
\nopagebreak\mylabel{L00848v}
\rehead{ }\normalsize\beginnumbering\briefempfaengerindex{Brandes, Georg@\textsc{Brandes, Georg}!zzzSchnitzler, Arthur@\emph{von Arthur Schnitzler}!1898-09-221@{22. 9. 1898}|(be}
\toendnotes[C]{\smallbreak\pagebreak[2]}\Standort{DLA, A:Schnitzler, HS.NZ85.1.440.}
\physDesc{Brief, 1 Blatt, 4 Seiten, 1988 Zeichen
\newline{}Handschrift: 1) schwarze Tinte, deutsche Kurrent\hspace{1em}2) Bleistift, deutsche Kurrent (\noindent{}Ergänzung: »(nicht abgesand{[}t{]}«)\hspace{1em}3) roter Buntstift (\noindent{}eine Unterstreichung)\hspace{1em}}\toendnotes[C]{\smallbreak}
\pstart
            (nicht abgeſand{[}t{]}\pend
           
\pstart{}{\pb}Verehrteſter Herr Brandes,\pend\vspace{0.5em}
\pstart
           ich ſchicke Ihnen heute das Stück\pwindex{Vermaechtnis. Schauspiel in drei Akten@\emph{Das Vermächtnis. Schauspiel in drei Akten}|pwv}, welches nächſtens aufgeführt wird; es iſt das Bühnenmanuscript; als
               Buch hab ich es noch nicht drucken laſſen, weil ich hoffe, daſs mir bei den Proben
               noch manches einfallen wird, um den zweiten und den Beginn des 3. Aktes höher zu
               bringen; und das erſcheint mir recht nothwendig. –\pend
           
\pstart
           – Heut hab ich eine Zeitſchrift »Das neue
                  Jahrhundert\pwindex{neue Jahrhundert@\emph{Das neue Jahrhundert}|pw}« zugeſchickt erhalten, mit Ihrem \label{K_L00848-1v}\edtext{Artikel\pwindex{Jeanne Marni@\emph{Jeanne Marni}|pwv}}{\lemma{\textnormal{\emph{Artikel}}}\Cendnote{\textnormal{Georg Brandes\pwindex{Brandes, Georg 04.02.1842 – 19.02.1927@\textsc{Brandes, Georg} (04.02.1842 – 19.02.1927)|pwk}: \emph{Jeanne Marni}\pwindex{Jeanne Marni@\emph{Jeanne Marni}|pwk}. In: \emph{Das neue
                        Jahrhundert}\pwindex{neue Jahrhundert@\emph{Das neue Jahrhundert}|pwk}, Jg. 1, H. 1, 1. 10. 1898, S. 14–19.
               }}}\label{K_L00848-1} über die \textsc{Marni}\pwindex{Marni, Jeanne 1854-01-31 – 1910-01-06@\textsc{Marni, Jeanne} (1854-01-31 – 1910-01-06), \emph{Schriftsteller/Schriftstellerin}|pw}. {\pb}Zu dieſem Artikel ſteht auch eine unendlich
               liebenswürdige Bemerkung über mein erſtes Buch\pwindex{Anatol@\emph{Anatol}|pwv}. Und doch wärs mir lieber geweſen, Sie hätten
               geſchrieben, jenes Buch iſt nicht viel werth, aber ſein Autor hat ſpäter beſſeres
               gemacht. Sie werden gleich wiſſen, warum ich das ſagen darf. Nach dem Anatol\pwindex{Anatol@\emph{Anatol}|pw} hab’ ich Ihnen das Märchen\pwindex{Maerchen. Schauspiel in drei Aufzuegen@\emph{Das Märchen. Schauspiel in drei Aufzügen}|pw} geſchickt und da haben Sie mir geſchrieben: »Sie haben
               hier eine viel höhere Stufe erreicht als in Ihrem früheren Buch\pwindex{Anatol@\emph{Anatol}|pwv}« – und ebenſo ſchienen Sie – in einem
               Brief an mich, wie in einer Bemerkung\pwindex{To Forestillinger af Henrik IV@\emph{To Forestillinger af Henrik IV}|pwv}{ }{\pb}in »\textsc{Politiken}\pwindex{Politiken@\emph{Politiken}|pw}« die »Liebelei\pwindex{Liebelei. Schauspiel in drei Akten@\emph{Liebelei. Schauspiel in drei Akten}|pw}« höher zu ſchätzen als die
               frühern Sachen. – Und heute ſteht in Ihrem Artikel\pwindex{Jeanne Marni@\emph{Jeanne Marni}|pwv} – »Sch. hat die Fähigkeit, die er hier \introOben{}(Anatol)\pwindex{Anatol@\emph{Anatol}|pw}\introOben{} bewieſen, nicht weiterentwickelt.« – Ich glaube nicht, daſs es dumme
               Empfindlichkeit iſt wenn mich dieſe Bemerkung ein bischen verſti{\geminationm}t hat – denn von Menſchen, deren Urtheil uns hoch
               ſteht, möchten wir alles hören – nur nicht; daſs ſie uns ſtehen bleiben oder gar
               herunter ſteigen ſehen. Es ist ja wirklich \substVorne{}\textsuperscript{das}\substDazwischen{}nicht\substHinten{} weſentlicher, daſs wir gelegentlich was anſtändges ſchreiben, ſondern {\pb}daſs wir uns in ſteter Entwicklung befinden – und, wie
               Sie ſehen, hatte ich nicht Urſache zu glauben, daſs Sie \uline{gerade} das bei mir zu bemerken meinen – und ich bin vielleicht ein wenig
               ſtolz darauf geweſen.\pend
           
\pstart
           Darum, mein verehrter Herr Brandes, müſſen Sie mir verzeihen, daſs ich Ihnen heute
               dieſen möglicherweiſe kindiſchen Brief ſchreibe; ich werde mich wahrſcheinlich morgen
               ſchon ſeiner ſchämen.\pend
           \pstart Seien Sie in herzlicher Ergebenheit gegrüßt von Ihrem
                  \spacefill\mbox{ArthurSchnitzler}\pend{}
\pstart
           Wien\oindex{Wien@\textbf{Wien}, \emph{A.ADM2}|pw}{ }22. 9. 98.\pend
           \selectlanguage{ngerman}\endnumbering\briefempfaengerindex{Brandes, Georg@\textsc{Brandes, Georg}!zzzSchnitzler, Arthur@\emph{von Arthur Schnitzler}!1898-09-221@{22. 9. 1898}|)be}\mylabel{L00848h}  \normalsize

\doendnotes{C}
\bigskip
\vfill

\clearpage

\footnotesize

\lohead{\textsc{register}}

% Definiere theindex-Environment komplett neu ohne reledmac
\makeatletter
\renewenvironment{theindex}{%
  \section*{\indexname}%
  \setlength{\parindent}{0pt}%
  \setlength{\parskip}{0pt plus 0.3pt}%
  \let\item\@idxitem
}{%
  \clearpage
}
\makeatother

\IfFileExists{\jobname-pw.ind}{\input{\jobname-pw.ind}}{}

\end{document}

      