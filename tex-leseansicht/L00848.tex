%% latex-leseansicht-vorspann.tex
%% Vorspann für die Leseansicht.
%% Lädt die gemeinsame Datei latex-vorspann.tex mit nicht gesetztem Schalter.

\newif\ifkorrekturansicht
\korrekturansichtfalse

\input{../tex-inputs/latex-vorspann}


\section[Arthur Schnitzler an Georg Brandes, nicht abgesandt, 22. 9. 1898]{L00848 Arthur Schnitzler an Georg Brandes, nicht abgesandt, 22. 9. 1898}
\nopagebreak\mylabel{L00848v}
\rehead{ }\normalsize\beginnumbering\briefempfaengerindex{Brandes, Georg@\textsc{Brandes, Georg}!zzzSchnitzler, Arthur@\emph{von Arthur Schnitzler}!1898-09-221@{22. 9. 1898}|(be}
\toendnotes[C]{\smallbreak\pagebreak[2]}
\correspDesc{Versand  durch Arthur Schnitzler am 22. 9. 1898 in Wien
\newline{}Erhalt  durch Georg Brandes im Zeitraum [nicht
                  abgesandt] \textbf{Ort fehlend} }\toendnotes[C]{\smallbreak}
\Standort{DLA, A:Schnitzler, HS.NZ85.1.440.}
\physDesc{Brief, 1 Blatt, 4 Seiten, 1988 Zeichen
\newline{}Handschrift: 1) schwarze Tinte, deutsche Kurrent\hspace{1em}2) Bleistift, deutsche Kurrent (\noindent{}Ergänzung: »(nicht abgesand{[}t{]}«)\hspace{1em}3) roter Buntstift (\noindent{}eine Unterstreichung)\hspace{1em}}\toendnotes[C]{\smallbreak}
\pstart
           (nicht abgeſand{[}t{]}\pend
           
\pstart{}{\pb}Verehrteſter Herr Brandes,\pend\vspace{0.5em}
\pstart
           ich{ }ſchicke Ihnen heute das Stück\pwindex{Schnitzler, Arthur 15.\,5.\,1862 Wien – 21.\,10.\,1931 ebd.@\textsc{Schnitzler, Arthur} (15.\,5.\,1862 Wien – 21.\,10.\,1931 ebd.), \emph{Schriftsteller, Mediziner}!Vermächtnis. Schauspiel in drei Akten@\strich\emph{Das Vermächtnis. Schauspiel in drei Akten}|pwv}, welches nächſtens aufgeführt wird; es iſt das Bühnenmanuscript; als
               Buch hab ich es noch nicht drucken laſſen, weil ich hoffe, daſs mir bei den Proben
               noch manches einfallen wird, um den zweiten und den Beginn des 3. Aktes höher zu
               bringen; und das erſcheint mir recht nothwendig. –\pend
           
\pstart
           – Heut hab ich eine Zeitſchrift »Das neue
                  Jahrhundert\pwindex{neue Jahrhundert@\emph{Das neue Jahrhundert}|pw}« zugeſchickt erhalten, mit Ihrem \label{K_L00848-1v}\edtext{Artikel\pwindex{Brandes, Georg 4.\,2.\,1842 Kopenhagen – 19.\,2.\,1927 ebd.@\textsc{Brandes, Georg} (4.\,2.\,1842 Kopenhagen – 19.\,2.\,1927 ebd.)!Jeanne Marni@\strich\emph{Jeanne Marni}|pwv}}{\lemma{\textnormal{\emph{Artikel}}}\Cendnote{\textnormal{Georg Brandes\pwindex{Brandes, Georg 4.\,2.\,1842 Kopenhagen – 19.\,2.\,1927 ebd.@\textsc{Brandes, Georg} (4.\,2.\,1842 Kopenhagen – 19.\,2.\,1927 ebd.)|pwk}: \emph{Jeanne Marni}\pwindex{Brandes, Georg 4.\,2.\,1842 Kopenhagen – 19.\,2.\,1927 ebd.@\textsc{Brandes, Georg} (4.\,2.\,1842 Kopenhagen – 19.\,2.\,1927 ebd.)!Jeanne Marni@\strich\emph{Jeanne Marni}|pwk}. In: \emph{Das neue
                        Jahrhundert}\pwindex{neue Jahrhundert@\emph{Das neue Jahrhundert}|pwk}, Jg. 1, H. 1, 1. 10. 1898, S. 14–19.
               }}}\label{K_L00848-1} über die \textsc{Marni}\pwindex{Marni, Jeanne 31.\,1.\,1854 Toulouse – 6.\,1.\,1910 Cannes@\textsc{Marni, Jeanne} (31.\,1.\,1854 Toulouse – 6.\,1.\,1910 Cannes), \emph{Schriftstellerin}|pw}. {\pb}Zu dieſem Artikel{ }ſteht auch eine unendlich
               liebenswürdige Bemerkung über mein erſtes Buch\pwindex{Schnitzler, Arthur 15.\,5.\,1862 Wien – 21.\,10.\,1931 ebd.@\textsc{Schnitzler, Arthur} (15.\,5.\,1862 Wien – 21.\,10.\,1931 ebd.), \emph{Schriftsteller, Mediziner}!Anatol@\strich\emph{Anatol}|pwv}. Und doch wärs mir lieber geweſen, Sie hätten
               geſchrieben, jenes Buch iſt nicht viel werth, aber{ }ſein Autor hat{ }ſpäter beſſeres
               gemacht. Sie werden gleich wiſſen, warum ich das{ }ſagen darf. Nach dem Anatol\pwindex{Schnitzler, Arthur 15.\,5.\,1862 Wien – 21.\,10.\,1931 ebd.@\textsc{Schnitzler, Arthur} (15.\,5.\,1862 Wien – 21.\,10.\,1931 ebd.), \emph{Schriftsteller, Mediziner}!Anatol@\strich\emph{Anatol}|pw} hab’ ich Ihnen das Märchen\pwindex{Schnitzler, Arthur 15.\,5.\,1862 Wien – 21.\,10.\,1931 ebd.@\textsc{Schnitzler, Arthur} (15.\,5.\,1862 Wien – 21.\,10.\,1931 ebd.), \emph{Schriftsteller, Mediziner}!Märchen. Schauspiel in drei Aufzügen@\strich\emph{Das Märchen. Schauspiel in drei Aufzügen}|pw} geſchickt und da haben Sie mir geſchrieben: »Sie haben
               hier eine viel höhere Stufe erreicht als in Ihrem früheren Buch\pwindex{Schnitzler, Arthur 15.\,5.\,1862 Wien – 21.\,10.\,1931 ebd.@\textsc{Schnitzler, Arthur} (15.\,5.\,1862 Wien – 21.\,10.\,1931 ebd.), \emph{Schriftsteller, Mediziner}!Anatol@\strich\emph{Anatol}|pwv}« – und ebenſo{ }ſchienen Sie – in einem
               Brief an mich, wie in einer Bemerkung\pwindex{Brandes, Georg 4.\,2.\,1842 Kopenhagen – 19.\,2.\,1927 ebd.@\textsc{Brandes, Georg} (4.\,2.\,1842 Kopenhagen – 19.\,2.\,1927 ebd.)!To Forestillinger af Henrik IV@\strich\emph{To Forestillinger af Henrik IV}|pwv}{ }{\pb}in »\textsc{Politiken}\pwindex{Politiken@\emph{Politiken}|pw}« die »Liebelei\pwindex{Schnitzler, Arthur 15.\,5.\,1862 Wien – 21.\,10.\,1931 ebd.@\textsc{Schnitzler, Arthur} (15.\,5.\,1862 Wien – 21.\,10.\,1931 ebd.), \emph{Schriftsteller, Mediziner}!Liebelei. Schauspiel in drei Akten@\strich\emph{Liebelei. Schauspiel in drei Akten}|pw}« höher zu{ }ſchätzen als die
               frühern Sachen. – Und heute{ }ſteht in Ihrem Artikel\pwindex{Brandes, Georg 4.\,2.\,1842 Kopenhagen – 19.\,2.\,1927 ebd.@\textsc{Brandes, Georg} (4.\,2.\,1842 Kopenhagen – 19.\,2.\,1927 ebd.)!Jeanne Marni@\strich\emph{Jeanne Marni}|pwv} – »Sch. hat die Fähigkeit, die er hier \introOben{}(Anatol)\pwindex{Schnitzler, Arthur 15.\,5.\,1862 Wien – 21.\,10.\,1931 ebd.@\textsc{Schnitzler, Arthur} (15.\,5.\,1862 Wien – 21.\,10.\,1931 ebd.), \emph{Schriftsteller, Mediziner}!Anatol@\strich\emph{Anatol}|pw}\introOben{} bewieſen, nicht weiterentwickelt.« – Ich glaube nicht, daſs es dumme
               Empfindlichkeit iſt wenn mich dieſe Bemerkung ein bischen verſti{\geminationm}t hat – denn von Menſchen, deren Urtheil uns hoch{ }ſteht, möchten wir alles hören – nur nicht; daſs{ }ſie uns{ }ſtehen bleiben oder gar
               herunter{ }ſteigen{ }ſehen. Es ist ja wirklich \substVorne{}\textsuperscript{das}\substDazwischen{}nicht\substHinten{} weſentlicher, daſs wir gelegentlich was anſtändges{ }ſchreiben,{ }ſondern {\pb}daſs wir uns in{ }ſteter Entwicklung befinden – und, wie
               Sie{ }ſehen, hatte ich nicht Urſache zu glauben, daſs Sie \uline{gerade} das bei mir zu bemerken meinen – und ich bin vielleicht ein wenig{ }ſtolz darauf geweſen.\pend
           
\pstart
           Darum, mein verehrter Herr Brandes, müſſen Sie mir verzeihen, daſs ich Ihnen heute
               dieſen möglicherweiſe kindiſchen Brief{ }ſchreibe; ich werde mich wahrſcheinlich morgen{ }ſchon{ }ſeiner{ }ſchämen.\pend
           \pstart Seien Sie in herzlicher Ergebenheit gegrüßt von Ihrem
                  \spacefill\mbox{ArthurSchnitzler}\pend{}
\pstart
           Wien\oindex{Wien@\textbf{Wien}, \emph{Verwaltungsgebiet}|pw}{ }22. 9. 98.\pend
           \selectlanguage{ngerman}\endnumbering\briefempfaengerindex{Brandes, Georg@\textsc{Brandes, Georg}!zzzSchnitzler, Arthur@\emph{von Arthur Schnitzler}!1898-09-221@{22. 9. 1898}|)be}\mylabel{L00848h}  \newcommand{\dateiname}{L00848}\newcommand{\titel}{Arthur Schnitzler an Georg Brandes, nicht abgesandt, 22. 9. 1898}\newcommand{\editorInnen}{Martin Anton Müller und Gerd-Hermann Susen}%% latex-leseansicht-abspann.tex
%% Abspann für die Leseansicht.
%% Der Schalter \ifkorrekturansicht ist bereits durch den Vorspann gesetzt.

%% latex-abspann.tex
%% Gemeinsamer Abspann für Korrekturansicht und Leseansicht.
%% Setzt den Schalter \ifkorrekturansicht voraus (gesetzt in den
%% einbindenden Dateien latex-korrekturansicht-abspann.tex bzw.
%% latex-leseansicht-abspann.tex).
%% ---------------------------------------------------------------

\normalsize

% Das esempio-Environment wird nur in der Leseansicht benötigt
\ifkorrekturansicht\else
\newenvironment{esempio}[3]%
{
    \vspace{1.5ex}
    \rlap{\underline{#1}}
    \par
    \setlength{\parindent}{0cm}
    \nopagebreak
    \leftskip=#2cm
    \rightskip=#3cm
}
{
    \par
}
\fi

\doendnotes{C}
\bigskip
\vfill

\clearpage

\footnotesize

\ifkorrekturansicht
  \lohead{\textsc{register}}
\fi

% theindex-Environment neu definieren ohne reledmac
\makeatletter
\renewenvironment{theindex}{%
  \ifkorrekturansicht
    \section*{\indexname}%
  \else
    \subsubsection*{Index der erwähnten Entitäten}%
  \fi
  \setlength{\parindent}{0pt}%
  \setlength{\parskip}{0pt plus 0.3pt}%
  \let\item\@idxitem
}{%
  \ifkorrekturansicht\clearpage\fi
}
\makeatother

\IfFileExists{\jobname-pw.ind}{\input{\jobname-pw.ind}}{}

% Quellenangabe nur in der Leseansicht
\ifkorrekturansicht\else
% Fallback-Definitionen, falls die .tex-Datei \titel etc. nicht gesetzt hat
\providecommand{\titel}{}
\providecommand{\editorInnen}{}
\providecommand{\dateiname}{\jobname}

\vspace{3cm}

\vfill

\footnotesize
\textsc{Quelle}: \titel. Herausgegeben von {\editorInnen}. In: \emph{Arthur Schnitzler: Briefwechsel mit Autorinnen und Autoren}.
 Digitale Edition, https://schnitzler-briefe.acdh.oeaw.ac.at/{\dateiname}.html (Stand \today)
\fi

\end{document}


