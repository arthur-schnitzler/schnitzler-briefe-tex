%% latex-leseansicht-vorspann.tex
%% Vorspann für die Leseansicht.
%% Lädt die gemeinsame Datei latex-vorspann.tex mit nicht gesetztem Schalter.

\newif\ifkorrekturansicht
\korrekturansichtfalse

\input{../tex-inputs/latex-vorspann}


               \section[Arthur Schnitzler an Georg Brandes, nicht abgesandt, 22. 9. 1898]{ Arthur Schnitzler an Georg Brandes, nicht abgesandt,
                    22. 9. 1898}\nopagebreak\mylabel{v}\rehead{ }\begin{ledgroupsized}[t]{13cm}\normalsize\beginnumbering\briefempfaengerindex{Brandes, Georg@\textsc{Brandes, Georg}!zzzSchnitzler, Arthur@\emph{von Arthur Schnitzler}!1898-09-221@{22. 9. 1898}|(be} \toendnotes[C]{\smallbreak\pagebreak[2]} \Standort{DLA, A:Schnitzler, HS.NZ85.1.440.}
\physDesc{Brief, 1 Blatt, 4 Seiten
\newline{}Handschrift: 1) schwarze Tinte, deutsche Kurrent\hspace{1em}2) Bleistift, deutsche Kurrent (\noindent{}Ergänzung: »(nicht abgesand{[}t{]}«)\hspace{1em}3) roter Buntstift (\noindent{}eine Unterstreichung)\hspace{1em}}\toendnotes[C]{\smallbreak}\pstart
           \noindent{} (nicht abgeſand{[}t{]}\pend
           \pstart{}{\pb}Verehrteſter Herr Brandes,\pend\pstart
           ich ſchicke Ihnen heute das Stück\pwindex{Schnitzler, Arthur 15.05.1862 – 21.10.1931@\textsc{Schnitzler, Arthur} (15.05.1862 – 21.10.1931), \emph{Schriftsteller, Mediziner}!Vermaechtnis. Schauspiel in drei Akten1898@\strich\emph{Das Vermächtnis. Schauspiel in drei Akten} {[}1898{]}|pwv}, welches nächſtens aufgeführt wird; es iſt das Bühnenmanuscript;
                    als Buch hab ich es noch nicht drucken laſſen, weil ich hoffe, daſs mir bei den
                    Proben noch manches einfallen wird, um den zweiten und den Beginn des 3. Aktes
                    höher zu bringen; und das erſcheint mir recht nothwendig. –\pend
           \pstart
           – Heut hab ich eine Zeitſchrift »Das neue
                        Jahrhundert\pwindex{neue Jahrhundert1.10.1898 – 1901@\emph{Das neue Jahrhundert}|pw}« zugeſchickt erhalten, mit Ihrem \label{K_L00848_1v}\edtext{Artikel\pwindex{Brandes, Georg 04.02.1842 – 19.02.1927@\textsc{Brandes, Georg} (04.02.1842 – 19.02.1927)!Jeanne Marni01. 10. 1898@\strich\emph{Jeanne Marni} {[}01. 10. 1898{]}|pwv}}{\lemma{\textnormal{\emph{Artikel}}}\Cendnote{\textnormal{Georg Brandes\pwindex{Brandes, Georg 04.02.1842 – 19.02.1927@\textsc{Brandes, Georg} (04.02.1842 – 19.02.1927)|pwk}: \emph{Jeanne Marni}\pwindex{Brandes, Georg 04.02.1842 – 19.02.1927@\textsc{Brandes, Georg} (04.02.1842 – 19.02.1927)!Jeanne Marni01. 10. 1898@\strich\emph{Jeanne Marni} {[}01. 10. 1898{]}|pwk}. In: \emph{Das
                                neue Jahrhundert}\pwindex{neue Jahrhundert1.10.1898 – 1901@\emph{Das neue Jahrhundert}|pwk}, Jg. 1, H. 1, 1. 10. 1898,
                            S. 14–19.}}}\label{K_L00848_1h} über die \textsc{Marni}\pwindex{Marni, Jeanne 31.01.1854 – 06.01.1910@\textsc{Marni, Jeanne} (31.01.1854 – 06.01.1910), \emph{Schriftstellerin}|pw}. {\pb}Zu dieſem Artikel ſteht auch eine
                    unendlich liebenswürdige Bemerkung über mein erſtes Buch\pwindex{Schnitzler, Arthur 15.05.1862 – 21.10.1931@\textsc{Schnitzler, Arthur} (15.05.1862 – 21.10.1931), \emph{Schriftsteller, Mediziner}!Anatol1892-10-29 – 1892-10-29@\strich\emph{Anatol} {[}1892-10-29 – 1892-10-29{]}|pwv}. Und doch wärs mir lieber geweſen, Sie hätten
                    geſchrieben, jenes Buch iſt nicht viel werth, aber ſein Autor hat ſpäter
                    beſſeres gemacht. Sie werden gleich wiſſen, warum ich das ſagen darf. Nach dem
                        Anatol\pwindex{Schnitzler, Arthur 15.05.1862 – 21.10.1931@\textsc{Schnitzler, Arthur} (15.05.1862 – 21.10.1931), \emph{Schriftsteller, Mediziner}!Anatol1892-10-29 – 1892-10-29@\strich\emph{Anatol} {[}1892-10-29 – 1892-10-29{]}|pw} hab’ ich Ihnen das Märchen\pwindex{Schnitzler, Arthur 15.05.1862 – 21.10.1931@\textsc{Schnitzler, Arthur} (15.05.1862 – 21.10.1931), \emph{Schriftsteller, Mediziner}!Maerchen. Schauspiel in drei Aufzuegen1891 – 1891@\strich\emph{Das Märchen. Schauspiel in drei Aufzügen} {[}1891 – 1891{]}|pw} geſchickt und da haben Sie mir geſchrieben: »Sie
                    haben hier eine viel höhere Stufe erreicht als in Ihrem früheren Buch\pwindex{Schnitzler, Arthur 15.05.1862 – 21.10.1931@\textsc{Schnitzler, Arthur} (15.05.1862 – 21.10.1931), \emph{Schriftsteller, Mediziner}!Anatol1892-10-29 – 1892-10-29@\strich\emph{Anatol} {[}1892-10-29 – 1892-10-29{]}|pwv}« – und ebenſo ſchienen Sie – in
                    einem Brief an mich, wie in einer Bemerkung\pwindex{Brandes, Georg 04.02.1842 – 19.02.1927@\textsc{Brandes, Georg} (04.02.1842 – 19.02.1927)!To Forestillinger af Henrik IV5.8.1896 – 5.8.1896@\strich\emph{To Forestillinger af Henrik IV} {[}5.8.1896 – 5.8.1896{]}|pwv}{ }{\pb}in »\textsc{Politiken}\pwindex{Politiken1. 1. 1884@\emph{Politiken}|pw}« die »Liebelei\pwindex{Schnitzler, Arthur 15.05.1862 – 21.10.1931@\textsc{Schnitzler, Arthur} (15.05.1862 – 21.10.1931), \emph{Schriftsteller, Mediziner}!Liebelei. Schauspiel in drei Akten9. 10. 1895@\strich\emph{Liebelei. Schauspiel in drei Akten} {[}9. 10. 1895{]}|pw}« höher zu ſchätzen als
                    die frühern Sachen. – Und heute ſteht in Ihrem Artikel\pwindex{Brandes, Georg 04.02.1842 – 19.02.1927@\textsc{Brandes, Georg} (04.02.1842 – 19.02.1927)!Jeanne Marni01. 10. 1898@\strich\emph{Jeanne Marni} {[}01. 10. 1898{]}|pwv} – »Sch. hat die Fähigkeit, die er hier \introOben{}(Anatol)\pwindex{Schnitzler, Arthur 15.05.1862 – 21.10.1931@\textsc{Schnitzler, Arthur} (15.05.1862 – 21.10.1931), \emph{Schriftsteller, Mediziner}!Anatol1892-10-29 – 1892-10-29@\strich\emph{Anatol} {[}1892-10-29 – 1892-10-29{]}|pw}\introOben{} bewieſen, nicht weiterentwickelt.« – Ich glaube nicht, daſs es dumme
                    Empfindlichkeit iſt wenn mich dieſe Bemerkung ein bischen verſti{\geminationm}t hat – denn von Menſchen, deren Urtheil uns hoch
                    ſteht, möchten wir alles hören – nur nicht; daſs ſie uns ſtehen bleiben oder gar
                    herunter ſteigen ſehen. Es ist ja wirklich \substVorne{}\textsuperscript{das}\substDazwischen{}nicht\substHinten{} weſentlicher, daſs wir gelegentlich was anſtändges ſchreiben, ſondern
                        {\pb}daſs wir uns in ſteter Entwicklung befinden –
                    und, wie Sie ſehen, hatte ich nicht Urſache zu glauben, daſs Sie \uline{gerade} das bei mir zu bemerken meinen – und ich
                    bin vielleicht ein wenig ſtolz darauf geweſen.\pend
           \pstart
           Darum, mein verehrter Herr Brandes, müſſen Sie mir verzeihen, daſs ich Ihnen
                    heute dieſen möglicherweiſe kindiſchen Brief ſchreibe; ich werde mich
                    wahrſcheinlich morgen ſchon ſeiner ſchämen.\pend
           \pstart Seien Sie in herzlicher Ergebenheit gegrüßt von Ihrem
                        \spacefill\mbox{ArthurSchnitzler}\pend{}\pstart
           Wien\oindex{Wien@\textbf{Wien}|pw}{ }22. 9. 98.\pend
           \endnumbering\briefempfaengerindex{Brandes, Georg@\textsc{Brandes, Georg}!zzzSchnitzler, Arthur@\emph{von Arthur Schnitzler}!1898-09-221@{22. 9. 1898}|)be}\mylabel{h}\end{ledgroupsized}  \newcommand{\dateiname}{L00848}\newcommand{\titel}{Arthur Schnitzler an Georg Brandes, nicht abgesandt, 22. 9. 1898}\newcommand{\editorInnen}{Martin Anton Müller und Gerd-Hermann Susen}%% latex-leseansicht-abspann.tex
%% Abspann für die Leseansicht.
%% Der Schalter \ifkorrekturansicht ist bereits durch den Vorspann gesetzt.

%% latex-abspann.tex
%% Gemeinsamer Abspann für Korrekturansicht und Leseansicht.
%% Setzt den Schalter \ifkorrekturansicht voraus (gesetzt in den
%% einbindenden Dateien latex-korrekturansicht-abspann.tex bzw.
%% latex-leseansicht-abspann.tex).
%% ---------------------------------------------------------------

\normalsize

% Das esempio-Environment wird nur in der Leseansicht benötigt
\ifkorrekturansicht\else
\newenvironment{esempio}[3]%
{
    \vspace{1.5ex}
    \rlap{\underline{#1}}
    \par
    \setlength{\parindent}{0cm}
    \nopagebreak
    \leftskip=#2cm
    \rightskip=#3cm
}
{
    \par
}
\fi

\doendnotes{C}
\bigskip
\vfill

\clearpage

\footnotesize

\ifkorrekturansicht
  \lohead{\textsc{register}}
\fi

% theindex-Environment neu definieren ohne reledmac
\makeatletter
\renewenvironment{theindex}{%
  \ifkorrekturansicht
    \section*{\indexname}%
  \else
    \subsubsection*{Index der erwähnten Entitäten}%
  \fi
  \setlength{\parindent}{0pt}%
  \setlength{\parskip}{0pt plus 0.3pt}%
  \let\item\@idxitem
}{%
  \ifkorrekturansicht\clearpage\fi
}
\makeatother

\IfFileExists{\jobname-pw.ind}{\input{\jobname-pw.ind}}{}

% Quellenangabe nur in der Leseansicht
\ifkorrekturansicht\else
% Fallback-Definitionen, falls die .tex-Datei \titel etc. nicht gesetzt hat
\providecommand{\titel}{}
\providecommand{\editorInnen}{}
\providecommand{\dateiname}{\jobname}

\vspace{3cm}

\vfill

\footnotesize
\textsc{Quelle}: \titel. Herausgegeben von {\editorInnen}. In: \emph{Arthur Schnitzler: Briefwechsel mit Autorinnen und Autoren}.
 Digitale Edition, https://schnitzler-briefe.acdh.oeaw.ac.at/{\dateiname}.html (Stand \today)
\fi

\end{document}


      