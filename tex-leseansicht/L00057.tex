%% latex-korrekturansicht-vorspann.tex
%% Vorspann für die Korrekturansicht.
%% Lädt die gemeinsame Datei latex-vorspann.tex mit gesetztem Schalter.

\newif\ifkorrekturansicht
\korrekturansichttrue

\input{../tex-inputs/latex-vorspann}


\section[Arthur Schnitzler an Richard Beer-Hofmann, {[}zwischen 1892 und Mitte 1893?{]}]{L00057 Arthur Schnitzler an Richard Beer-Hofmann, {[}zwischen 1892 und Mitte
               1893?{]}}
\nopagebreak\mylabel{L00057v}
\rehead{ }\normalsize\beginnumbering\briefempfaengerindex{Beer-Hofmann, Richard@\textsc{Beer-Hofmann, Richard}!zzzSchnitzler, Arthur@\emph{von Arthur Schnitzler}!1893-06-301@{{[}zwischen 1892 und Mitte
                  1893?{]}}|(be}
\toendnotes[C]{\smallbreak\pagebreak[2]}\Standort{YCGL, MSS 31.}
\physDesc{Briefkarte, 138 Zeichen
\newline{}Handschrift: Bleistift, deutsche Kurrent}\toendnotes[C]{\smallbreak}
\pstart
           \noindent{}{\pb}Lieber Richard;{ }\label{K_L00057-1v}\edtext{\textsc{Loris}\pwindex{Hofmannsthal, Hugo von 1874-02-01 – 1929-07-15@\textsc{Hofmannsthal, Hugo von} (1874-02-01 – 1929-07-15), \emph{Schriftsteller/Schriftstellerin}|pw}}{\lemma{\textnormal{\emph{Loris}}}\Cendnote{\textnormal{Dies ist der einzige Hinweis, der
                  erlaubt, das undatierte Korrespondenzstück zumindest irgendwie zeitlich
                  einzugrenzen, da Hofmannsthal\pwindex{Hofmannsthal, Hugo von 1874-02-01 – 1929-07-15@\textsc{Hofmannsthal, Hugo von} (1874-02-01 – 1929-07-15), \emph{Schriftsteller/Schriftstellerin}|pwk} das Pseudonym
                  nur bis Mitte 1893 regelmäßig verwendete, danach aber auch Schnitzler zunehmend dazu überging, den
                  Vornamen zu verwenden. Der erhaltene Briefwechsel Hofmannsthal\pwindex{Hofmannsthal, Hugo von 1874-02-01 – 1929-07-15@\textsc{Hofmannsthal, Hugo von} (1874-02-01 – 1929-07-15), \emph{Schriftsteller/Schriftstellerin}|pwk}/Beer-Hofmann\pwindex{Beer-Hofmann, Richard 1866-07-11 – 1945-09-26@\textsc{Beer-Hofmann, Richard} (1866-07-11 – 1945-09-26), \emph{Schriftsteller/Schriftstellerin}|pwk} legt nahe, dass erst 1892 ein vertraulicher
                  Umgang zwischen den beiden aufkam, der Mittagessen beim anderen zu Hause
                  involvierte.}}}\label{K_L00057-1} ſpeiſt nicht bei Ihnen – wir treffen uns alle um \uline{12 Uhr Mittags} im \textsc{Griensteidl}\oindex{Cafe Griensteidl@\textbf{Café Griensteidl}, \emph{Kaffeehaus (K.KAF)}|pw}; alle {\pb}ſind verſtändigt.\pend
           
\pstart
           Herzlichſt Ihr{\\[\baselineskip]}\spacefill\mbox{Arthur}\pend
           \leftskip=0em{}\selectlanguage{ngerman}\endnumbering\briefempfaengerindex{Beer-Hofmann, Richard@\textsc{Beer-Hofmann, Richard}!zzzSchnitzler, Arthur@\emph{von Arthur Schnitzler}!1892-01-011@{{[}zwischen 1892 und Mitte
                  1893?{]}}|)be}\mylabel{L00057h}  \normalsize

\doendnotes{C}
\bigskip
\vfill

\clearpage

\footnotesize

\lohead{\textsc{register}}

% Definiere theindex-Environment komplett neu ohne reledmac
\makeatletter
\renewenvironment{theindex}{%
  \section*{\indexname}%
  \setlength{\parindent}{0pt}%
  \setlength{\parskip}{0pt plus 0.3pt}%
  \let\item\@idxitem
}{%
  \clearpage
}
\makeatother

\IfFileExists{\jobname-pw.ind}{\input{\jobname-pw.ind}}{}

\end{document}

      