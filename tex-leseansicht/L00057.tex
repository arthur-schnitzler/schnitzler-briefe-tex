%% latex-leseansicht-vorspann.tex
%% Vorspann für die Leseansicht.
%% Lädt die gemeinsame Datei latex-vorspann.tex mit nicht gesetztem Schalter.

\newif\ifkorrekturansicht
\korrekturansichtfalse

\input{../tex-inputs/latex-vorspann}


               \section[Arthur Schnitzler an Richard Beer-Hofmann, {[}zwischen 1892 und Mitte 1893?{]}]{ Arthur Schnitzler an Richard Beer-Hofmann, {[}zwischen 1892 und Mitte
               1893?{]}}\nopagebreak\mylabel{v}\rehead{ }\begin{ledgroupsized}[t]{13cm}\normalsize\beginnumbering\briefempfaengerindex{Beer-Hofmann, Richard@\textsc{Beer-Hofmann, Richard}!zzzSchnitzler, Arthur@\emph{von Arthur Schnitzler}!1892-01-011@{{[}zwischen 1892 und Mitte
                  1893?{]}}|(be} \toendnotes[C]{\smallbreak\pagebreak[2]} \Standort{YCGL, MSS 31.}
\physDesc{Briefkarte
\newline{}Handschrift: Bleistift, deutsche Kurrent}\toendnotes[C]{\smallbreak}\pstart
           \noindent{}{\pb}Lieber Richard; \label{K_L00057_1v}\edtext{\textsc{Loris}\pwindex{Hofmannsthal, Hugo von 01.02.1874 – 15.07.1929@\textsc{Hofmannsthal, Hugo von} (01.02.1874 – 15.07.1929), \emph{Schriftsteller}|pw}}{\lemma{\textnormal{\emph{Loris}}}\Cendnote{\textnormal{Dies ist der einzige Hinweis, der
                  erlaubt, das undatierte Korrespondenzstück zumindest irgendwie zeitlich
                  einzugrenzen, da Hofmannsthal\pwindex{Hofmannsthal, Hugo von 01.02.1874 – 15.07.1929@\textsc{Hofmannsthal, Hugo von} (01.02.1874 – 15.07.1929), \emph{Schriftsteller}|pwk} das Pseudonym
                  nur bis Mitte 1893 regelmäßig verwendete, danach aber auch Schnitzler\pwindex{Schnitzler, Arthur 15.05.1862 – 21.10.1931@\textsc{Schnitzler, Arthur} (15.05.1862 – 21.10.1931), \emph{Schriftsteller, Mediziner}|pwk} zunehmend dazu überging, den
                  Vornamen zu verwenden. Der erhaltene Briefwechsel Hofmannsthal\pwindex{Hofmannsthal, Hugo von 01.02.1874 – 15.07.1929@\textsc{Hofmannsthal, Hugo von} (01.02.1874 – 15.07.1929), \emph{Schriftsteller}|pwk}/Beer-Hofmann\pwindex{Beer-Hofmann, Richard 11.07.1866 – 26.09.1945@\textsc{Beer-Hofmann, Richard} (11.07.1866 – 26.09.1945), \emph{Schriftsteller}|pwk} legt nahe, dass erst 1892 ein vertraulicher
                  Umgang zwischen den beiden aufkam, der Mittagessen beim anderen zu Hause
                  involvierte.}}}\label{K_L00057_1h} ſpeiſt nicht bei Ihnen – wir treffen uns alle um \uline{12 Uhr Mittags} im \textsc{Griensteidl}\oindex{Cafe Griensteidl@\textbf{Café Griensteidl}|pw}; alle {\pb}ſind verſtändigt.\pend
           \pstart
           Herzlichſt Ihr{\\[\baselineskip]}\spacefill\mbox{Arthur}\pend
           \leftskip=0em{}          \endnumbering\briefempfaengerindex{Beer-Hofmann, Richard@\textsc{Beer-Hofmann, Richard}!zzzSchnitzler, Arthur@\emph{von Arthur Schnitzler}!1892-01-011@{{[}zwischen 1892 und Mitte
                  1893?{]}}|)be}\mylabel{h}\end{ledgroupsized}  \newcommand{\dateiname}{L00057}\newcommand{\titel}{Arthur Schnitzler an Richard Beer-Hofmann, [zwischen 1892 und Mitte 1893?]}\newcommand{\editorInnen}{Martin Anton Müller und Gerd-Hermann Susen}
            \footnotesize
\begin{ledgroupsized}[t]{11.5cm}
\doendnotes{C}
\end{ledgroupsized}
         %% latex-leseansicht-abspann.tex
%% Abspann für die Leseansicht.
%% Der Schalter \ifkorrekturansicht ist bereits durch den Vorspann gesetzt.

%% latex-abspann.tex
%% Gemeinsamer Abspann für Korrekturansicht und Leseansicht.
%% Setzt den Schalter \ifkorrekturansicht voraus (gesetzt in den
%% einbindenden Dateien latex-korrekturansicht-abspann.tex bzw.
%% latex-leseansicht-abspann.tex).
%% ---------------------------------------------------------------

\normalsize

% Das esempio-Environment wird nur in der Leseansicht benötigt
\ifkorrekturansicht\else
\newenvironment{esempio}[3]%
{
    \vspace{1.5ex}
    \rlap{\underline{#1}}
    \par
    \setlength{\parindent}{0cm}
    \nopagebreak
    \leftskip=#2cm
    \rightskip=#3cm
}
{
    \par
}
\fi

\doendnotes{C}
\bigskip
\vfill

\clearpage

\footnotesize

\ifkorrekturansicht
  \lohead{\textsc{register}}
\fi

% theindex-Environment neu definieren ohne reledmac
\makeatletter
\renewenvironment{theindex}{%
  \ifkorrekturansicht
    \section*{\indexname}%
  \else
    \subsubsection*{Index der erwähnten Entitäten}%
  \fi
  \setlength{\parindent}{0pt}%
  \setlength{\parskip}{0pt plus 0.3pt}%
  \let\item\@idxitem
}{%
  \ifkorrekturansicht\clearpage\fi
}
\makeatother

\IfFileExists{\jobname-pw.ind}{\input{\jobname-pw.ind}}{}

% Quellenangabe nur in der Leseansicht
\ifkorrekturansicht\else
% Fallback-Definitionen, falls die .tex-Datei \titel etc. nicht gesetzt hat
\providecommand{\titel}{}
\providecommand{\editorInnen}{}
\providecommand{\dateiname}{\jobname}

\vspace{3cm}

\vfill

\footnotesize
\textsc{Quelle}: \titel. Herausgegeben von {\editorInnen}. In: \emph{Arthur Schnitzler: Briefwechsel mit Autorinnen und Autoren}.
 Digitale Edition, https://schnitzler-briefe.acdh.oeaw.ac.at/{\dateiname}.html (Stand \today)
\fi

\end{document}


      