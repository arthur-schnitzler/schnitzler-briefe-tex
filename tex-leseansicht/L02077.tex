%% latex-korrekturansicht-vorspann.tex
%% Vorspann für die Korrekturansicht.
%% Lädt die gemeinsame Datei latex-vorspann.tex mit gesetztem Schalter.

\newif\ifkorrekturansicht
\korrekturansichttrue

\input{../tex-inputs/latex-vorspann}


\section[Richard Beer-Hofmann an Arthur Schnitzler, 16. 7. 1912]{L02077 Richard Beer-Hofmann an Arthur Schnitzler, 16. 7. 1912}
\nopagebreak\mylabel{L02077v}
\rehead{ }\normalsize\beginnumbering\briefempfaengerindex{Schnitzler, Arthur@\textsc{Schnitzler, Arthur}!zzzBeer-Hofmann, Richard@\emph{von Richard Beer-Hofmann}!1912-07-161@{16. 7. 1912}|(be}
\toendnotes[C]{\smallbreak\pagebreak[2]}\Standort{CUL, Schnitzler, B 8.}
\physDesc{Bildpostkarte, 434 Zeichen
\newline{}Handschrift: schwarze Tinte, lateinische Kurrent
\newline{}Versand: Stempel: »\nobreak{}\oindex{St. Moritz@\textbf{St. Moritz}, \emph{P.PPL}|pwk}St. \textcolor{gray}{Moritz}
                                       Dorf, 17. VII. 12, 1\nobreak{}«.  
\newline{}Ordnung: mit Bleistift von unbekannter Hand nummeriert:
                                    »245« }
\buchAbdrucke{\weitereDrucke{Arthur Schnitzler, Richard Beer-Hofmann: \emph{Briefwechsel 1891–1931}. Wien, Zürich: \emph{Europaverlag} 1992, S. 216.} }\pstart{}{\pb}S. H.\pend{}\pstart{}Herrn\pend{}\pstart{}Arthur Schnitzler\pend{}\pstart{}Wien XVIII.\oindex{XVIII., Waehring@\textbf{XVIII., Währing}, \emph{A.ADM3}|pw}\pend{}\pstart{}Sternwartestrasse\oindex{Sternwartestrasse 71@\textbf{Sternwartestraße 71}, \emph{Wohngebäude (K.WHS)}|pw}.\pend{}{\bigskip}
\pstart
           \noindent{}\centering{}{\pb}\textcolor{gray}{\textbf{Hotel Waldhaus\oindex{Hotel Waldhaus am See@\textbf{Hotel Waldhaus am See}, \emph{Hotel (K.HTL)}|pw}{ }St. Moritz\oindex{St. Moritz@\textbf{St. Moritz}, \emph{P.PPL}|pw}, Engadin\oindex{Engadin@\textbf{Engadin}, \emph{T.VAL}|pw}}}\pend
           \vspace{1em}
\pstart
           {\pb}16. VII. 12.\pend
           \vspace{0.5em}
\pstart
           Lieber Arthur! Wir haben hier – (siehe Rückseite) Wohnung gefunden
                  \introOben{}(seit 13/VII)\introOben{} – \uline{ganz}{ }\uline{am}{ }\uline{Walde} und abseits von allem Lärm. Können aber nur bis
                  10 Aug. hier bleiben, da unsere Zi{\geminationm}er
               von da an besetzt sind. Wir wollen dann über \introOben{}den\introOben{}{ }Bodensee\oindex{Bodensee@\textbf{Bodensee}, \emph{H.LK}|pw}, (Bregenz\oindex{Bregenz@\textbf{Bregenz}, \emph{P.PPLA}|pw}? Konstanz\oindex{Konstanz@\textbf{Konstanz}, \emph{P.PPLA3}|pw}? wo wir 2–3 Tage
               bleiben wollen) nach Nordtirol\oindex{Tirol@\textbf{Tirol}, \emph{A.ADM1}|pw} oder Salzka{\geminationm}ergut\oindex{Salzkammergut@\textbf{Salzkammergut}, \emph{L.RGN}|pw}.
               Herzliche Grüsse von uns Allen an Sie Alle!\pend
           \pstart \spacefill\mbox{Richard.}\pend{}\selectlanguage{ngerman}\endnumbering\briefempfaengerindex{Schnitzler, Arthur@\textsc{Schnitzler, Arthur}!zzzBeer-Hofmann, Richard@\emph{von Richard Beer-Hofmann}!1912-07-161@{16. 7. 1912}|)be}\mylabel{L02077h}  \normalsize

\doendnotes{C}
\bigskip
\vfill

\clearpage

\footnotesize

\lohead{\textsc{register}}

% Definiere theindex-Environment komplett neu ohne reledmac
\makeatletter
\renewenvironment{theindex}{%
  \section*{\indexname}%
  \setlength{\parindent}{0pt}%
  \setlength{\parskip}{0pt plus 0.3pt}%
  \let\item\@idxitem
}{%
  \clearpage
}
\makeatother

\IfFileExists{\jobname-pw.ind}{\input{\jobname-pw.ind}}{}

\end{document}

      