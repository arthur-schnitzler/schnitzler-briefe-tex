%% latex-leseansicht-vorspann.tex
%% Vorspann für die Leseansicht.
%% Lädt die gemeinsame Datei latex-vorspann.tex mit nicht gesetztem Schalter.

\newif\ifkorrekturansicht
\korrekturansichtfalse

\input{../tex-inputs/latex-vorspann}

\begin{center}
            \textcolor{red}{ENTWURF. ENTZIFFERUNG NOCH NICHT KORREKTURGELESEN}
                      \end{center}
            
               \section[Richard Beer-Hofmann an Arthur Schnitzler, 16. 7. 1912]{ Richard Beer-Hofmann an Arthur Schnitzler, 16. 7. 1912}\nopagebreak\mylabel{v}\rehead{ }\begin{ledgroupsized}[t]{13cm}\normalsize\beginnumbering\briefempfaengerindex{Schnitzler, Arthur@\textsc{Schnitzler, Arthur}!zzzBeer-Hofmann, Richard@\emph{von Richard Beer-Hofmann}!1912-07-161@{16. 7. 1912}|(be} \toendnotes[C]{\smallbreak\pagebreak[2]} \Standort{CUL, Schnitzler, B 8.}
\physDesc{Bildpostkarte
\newline{}Handschrift: schwarze Tinte, lateinische Kurrent\newline{}Versand: Stempel: »\nobreak{}\oindex{Sankt Moritz@\textbf{Sankt Moritz}|pwk}St. \textcolor{gray}{Moritz} Dorf, 17. VII. 12, 1\nobreak{}«.  \newline{}Ordnung: mit Bleistift von unbekannter Hand nummeriert: »245« }\buchAbdrucke{\weitereDrucke{Arthur Schnitzler, Richard Beer-Hofmann: \emph{Briefwechsel 1891–1931}. Hg. Konstanze Fliedl. Wien, Zürich: \emph{Europaverlag} 1992, S. 216.} }\pstart{}{\pb}S. H.\pend{}\pstart{}Herrn\pend{}\pstart{}Arthur Schnitzler\pend{}\pstart{}Wien XVIII.\oindex{XVIII., Waehring@\textbf{XVIII., Währing}|pw}\pend{}\pstart{}Sternwartestrasse\oindex{Sternwartestrasse@\textbf{Sternwartestraße}|pw}.\pend{}{\bigskip}\pstart
           \noindent{}\centering{}{\pb}\textcolor{gray}{\textbf{Hotel Waldhaus\oindex{Hotel Waldhaus@\textbf{Hotel Waldhaus}|pw}{ }St. Moritz\oindex{Sankt Moritz@\textbf{Sankt Moritz}|pw}, Engadin\oindex{Engadin@\textbf{Engadin}|pw}}}\pend
           \pstart
           {\pb}16. VII. 12.\pend
           \pstart
           Lieber Arthur! Wir haben hier – (siehe Rückseite) Wohnung gefunden
                  \introOben{}(seit 13/VII)\introOben{} – \uline{ganz}{ }\uline{am}{ }\uline{Walde} und abseits von allem Lärm. Können aber nur bis
                  10 Aug. hier bleiben, da unsere Zi{\geminationm}er
               von da an besetzt sind. Wir wollen dann über \introOben{}den\introOben{}{ }Bodensee\oindex{Bodensee@\textbf{Bodensee}|pw}, (Bregenz\oindex{Bregenz@\textbf{Bregenz}|pw}? Konstanz\oindex{Konstanz@\textbf{Konstanz}|pw}? wo wir 2–3 Tage bleiben
               wollen) nach Nordtirol\oindex{Tirol@\textbf{Tirol}|pw} oder Salzka{\geminationm}ergut\oindex{Salzkammergut@\textbf{Salzkammergut}|pw}. Herzliche Grüsse von
               uns Allen an Sie Alle!\pend
           \pstart \spacefill\mbox{Richard.}\pend{}\endnumbering\briefempfaengerindex{Schnitzler, Arthur@\textsc{Schnitzler, Arthur}!zzzBeer-Hofmann, Richard@\emph{von Richard Beer-Hofmann}!1912-07-161@{16. 7. 1912}|)be}\mylabel{h}\end{ledgroupsized}  \newcommand{\dateiname}{L02077}\newcommand{\titel}{Richard Beer-Hofmann an Arthur Schnitzler, 16. 7. 1912}\newcommand{\editorInnen}{Martin Anton Müller und Gerd-Hermann Susen}%% latex-leseansicht-abspann.tex
%% Abspann für die Leseansicht.
%% Der Schalter \ifkorrekturansicht ist bereits durch den Vorspann gesetzt.

%% latex-abspann.tex
%% Gemeinsamer Abspann für Korrekturansicht und Leseansicht.
%% Setzt den Schalter \ifkorrekturansicht voraus (gesetzt in den
%% einbindenden Dateien latex-korrekturansicht-abspann.tex bzw.
%% latex-leseansicht-abspann.tex).
%% ---------------------------------------------------------------

\normalsize

% Das esempio-Environment wird nur in der Leseansicht benötigt
\ifkorrekturansicht\else
\newenvironment{esempio}[3]%
{
    \vspace{1.5ex}
    \rlap{\underline{#1}}
    \par
    \setlength{\parindent}{0cm}
    \nopagebreak
    \leftskip=#2cm
    \rightskip=#3cm
}
{
    \par
}
\fi

\doendnotes{C}
\bigskip
\vfill

\clearpage

\footnotesize

\ifkorrekturansicht
  \lohead{\textsc{register}}
\fi

% theindex-Environment neu definieren ohne reledmac
\makeatletter
\renewenvironment{theindex}{%
  \ifkorrekturansicht
    \section*{\indexname}%
  \else
    \subsubsection*{Index der erwähnten Entitäten}%
  \fi
  \setlength{\parindent}{0pt}%
  \setlength{\parskip}{0pt plus 0.3pt}%
  \let\item\@idxitem
}{%
  \ifkorrekturansicht\clearpage\fi
}
\makeatother

\IfFileExists{\jobname-pw.ind}{\input{\jobname-pw.ind}}{}

% Quellenangabe nur in der Leseansicht
\ifkorrekturansicht\else
% Fallback-Definitionen, falls die .tex-Datei \titel etc. nicht gesetzt hat
\providecommand{\titel}{}
\providecommand{\editorInnen}{}
\providecommand{\dateiname}{\jobname}

\vspace{3cm}

\vfill

\footnotesize
\textsc{Quelle}: \titel. Herausgegeben von {\editorInnen}. In: \emph{Arthur Schnitzler: Briefwechsel mit Autorinnen und Autoren}.
 Digitale Edition, https://schnitzler-briefe.acdh.oeaw.ac.at/{\dateiname}.html (Stand \today)
\fi

\end{document}


      