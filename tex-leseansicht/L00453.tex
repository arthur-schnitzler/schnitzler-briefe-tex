%% latex-korrekturansicht-vorspann.tex
%% Vorspann für die Korrekturansicht.
%% Lädt die gemeinsame Datei latex-vorspann.tex mit gesetztem Schalter.

\newif\ifkorrekturansicht
\korrekturansichttrue

\input{../tex-inputs/latex-vorspann}


\section[Richard Beer-Hofmann an Arthur Schnitzler, 14. 6. 1895]{L00453 Richard Beer-Hofmann an Arthur Schnitzler, 14. 6. 1895}
\nopagebreak\mylabel{L00453v}
\rehead{ }\normalsize\beginnumbering\briefempfaengerindex{Schnitzler, Arthur@\textsc{Schnitzler, Arthur}!zzzBeer-Hofmann, Richard@\emph{von Richard Beer-Hofmann}!1895-06-141@{14. 6. 1895}|(be}
\toendnotes[C]{\smallbreak\pagebreak[2]}\Standort{CUL, Schnitzler, B 8.}
\physDesc{Brief, 1 Blatt, 3 Seiten, 540 Zeichen
\newline{}Handschrift: Bleistift, lateinische Kurrent
\newline{}Schnitzler: mit Bleistift datiert: »14/6 95« und nummeriert: »60« }
\buchAbdrucke{\weitereDrucke{Arthur Schnitzler, Richard Beer-Hofmann: \emph{Briefwechsel 1891–1931}. Wien, Zürich: \emph{Europaverlag} 1992, S. 74.} }\toendnotes[C]{\smallbreak}
\pstart
           \noindent{}{\pb}Lieber Arthur! In einer halben Stunde werde ich ins Bett fallen; –
               vorher nur folgendes: Ich bin gegen \uline{Zasche\pwindex{Zasche, Theodor 18.10.1862 – 15.11.1922@\textsc{Zasche, Theodor} (18.10.1862 – 15.11.1922), \emph{Zeichner/Zeichnerin, Karikaturist/Karikaturistin}|pw}} als Illustrator – aber das wird wol nicht viel nützen. \uline{Datiren} sie jedenfalls die Novelle\pwindex{kleine Komoedie@\emph{Die kleine Komödie}|pwv}. Man {\pb}soll wissen daß sie \uline{vor}{ }Sterben\pwindex{Sterben. Novelle@\emph{Sterben. Novelle}|pw} geschrieben ist. Daß sie Fischer\pwindex{Fischer, Samuel 24.12.1859 – 15.10.1934@\textsc{Fischer, Samuel} (24.12.1859 – 15.10.1934), \emph{Verleger/Verlegerin}|pw} gefällt ist allerdings sehr unheimlich
               aber vielleicht lügt er. Keinesfalls verdient sie es, denn sie hat wirklich viel
               Grazie\pend
           
\pstart
           {\pb}Heute bin ich seelig – 14 Tage
               sind vorbei. Schreiben Sie mir mehr, und öfter, Sie wissen wie sehr ich mich damit
               freue.\pend
           
\pstart
           Gute Nacht\pend
           \pstart Ihr \spacefill\mbox{Richard}\pend{}
\pstart
           Časlau\oindex{Cáslav@\textbf{Čáslav}, \emph{P.PPL}|pw}{ }14/VI 95\pend
           \selectlanguage{ngerman}\endnumbering\briefempfaengerindex{Schnitzler, Arthur@\textsc{Schnitzler, Arthur}!zzzBeer-Hofmann, Richard@\emph{von Richard Beer-Hofmann}!1895-06-141@{14. 6. 1895}|)be}\mylabel{L00453h}  \normalsize

\doendnotes{C}
\bigskip
\vfill

\clearpage

\footnotesize

\lohead{\textsc{register}}

% Definiere theindex-Environment komplett neu ohne reledmac
\makeatletter
\renewenvironment{theindex}{%
  \section*{\indexname}%
  \setlength{\parindent}{0pt}%
  \setlength{\parskip}{0pt plus 0.3pt}%
  \let\item\@idxitem
}{%
  \clearpage
}
\makeatother

\IfFileExists{\jobname-pw.ind}{\input{\jobname-pw.ind}}{}

\end{document}

      