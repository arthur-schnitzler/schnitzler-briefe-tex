%% latex-leseansicht-vorspann.tex
%% Vorspann für die Leseansicht.
%% Lädt die gemeinsame Datei latex-vorspann.tex mit nicht gesetztem Schalter.

\newif\ifkorrekturansicht
\korrekturansichtfalse

\input{../tex-inputs/latex-vorspann}


         
         \renewcommand{\erwaehntePersonen}{Personen: Ola Hansson, Laura Marholm, Johann Schnitzler}
         \renewcommand{\erwaehnteOrte}{Orte: Oberbayern, Schliersee, Wien}
         \renewcommand{\erwaehnteWerke}{Werke: Das Buch der Frauen, Sterben. Novelle}
               \section[Laura Marholm an Arthur Schnitzler, 24. 4. 1895]{ Laura Marholm an Arthur Schnitzler, 24. 4. 1895}\nopagebreak\mylabel{v}\rehead{ }\begin{ledgroupsized}[t]{13cm}\normalsize\beginnumbering\briefempfaengerindex{Schnitzler, Arthur@\textsc{Schnitzler, Arthur}!zzzMarholm, Laura@\emph{von Laura Marholm}!1895-04-241@{24. 4. 1895}|(be} \toendnotes[C]{\smallbreak\pagebreak[2]} \Standort{TMW, HS Schn 3/65/1.}
\physDesc{Brief, 1 Blatt, 3 Seiten, 2485 Zeichen
\newline{}Handschrift: schwarze Tinte, lateinische Kurrent}\toendnotes[C]{\smallbreak}\pstart
           \raggedleft{}{\pb}Schliersee\oindex{Schliersee@\textbf{Schliersee}|pw}, Oberbaiern\oindex{Oberbayern@\textbf{Oberbayern}|pw},{\\}24 April 95.\pend
           \pstart{}Sehr geehrter Herr Doctor.\pend\pstart
           Wie ich Ihren Brief aufmachte, las ich erst: »mein Vater\pwindex{Schnitzler, Johann 10.04.1835 – 02.05.1893@\textsc{Schnitzler, Johann} (10.04.1835 – 02.05.1893), \emph{Laryngologe}|pwv} ist schon zwei Tage lang todt« und erschrak, – Sie
               hätten um ein Haar einen Condolenzbrief bekommen; da las ich ihn noch einmal, weil
               mir soviel Gutes drin gesagt wurde, was ich im Einzelnen auf seine Richtigkeit
               durchgehen wollte, – das, was Sie über die Hauptlinie sagen, machte mir eine
               besondere Freude, denn das meine ich selbst ist im Guten und Üblem der Punkt auf dem
               meine Anlage fußt. Nur beim zweiten Lesen sehe ich, daß es 2 Jahre sind und mir wurde
               ganz flau{\dots} sie haben mir so grundernsthaft geschrieben,
               Sie hätten auch ein bischen lachen können. Jetzt glaube ich, Sie thun es
               heimlich.\pend
           \pstart
           Natürlich bitte ich Sie, das häßliche Buch\pwindex{Marholm, Laura 19.04.1854 – 06.10.1928@\textsc{Marholm, Laura} (19.04.1854 – 06.10.1928), \emph{Schriftstellerin}!Buch der Frauen1894@\strich\emph{Das Buch der Frauen} {[}1894{]}|pwv} zu behalten, im Austausch von »Sterben\pwindex{Schnitzler, Arthur 15.05.1862 – 21.10.1931@\textsc{Schnitzler, Arthur} (15.05.1862 – 21.10.1931), \emph{Schriftsteller, Mediziner}!Sterben. Novelle1894-10-01 – 1894-12-01@\strich\emph{Sterben. Novelle} {[}1894-10-01 – 1894-12-01{]}|pw}« {\pb}das ich von Ihnen
               erhielt. Ich schrieb Ihnen damals über das Buch nichts – – wenn ich Ihnen den Grund
               sage, werden Sie es verstehen. Ola\pwindex{Hansson, Ola 12.11.1860 – 26.09.1925@\textsc{Hansson, Ola} (12.11.1860 – 26.09.1925), \emph{Schriftsteller}|pw} las es und
               fand es sehr gut und fein.\pend
           \pstart
           Aber ich konnte es nicht leiden – aus einem ganz subjectiven Grund {\dots} ich konnte mich damals keine Nacht zu Bett legen, ohne
               daß das kam, wovon das ganze Buch handelt. Sobald ich das Licht auslöschte und es
               ganz schwarz war, kam regelmäßig dies furchtbare Grauen vor dem Aufhören, nicht dem
               Sterben, aber dem Nichtmehrsein und nicht blos dem persönlichen Nichtmehrsein,
               sondern dem von meinen Liebsten, von dieser Weltkugel{\dotsfour} Ich
               betrachtete es gar nicht als etwas Krankhaftes, nur als einen Ausschlag von
               Vitalitätsgefühl, aber in der tiefen Schliersee\oindex{Schliersee@\textbf{Schliersee}|pw}r
               Einsamkeit, die mein Mann\pwindex{Hansson, Ola 12.11.1860 – 26.09.1925@\textsc{Hansson, Ola} (12.11.1860 – 26.09.1925), \emph{Schriftsteller}|pwv}
               liebt, war es bei mir, Tag und Nacht, immer, und steigerte sich jedesmal beim
               Einschlafen zu einem unsagbaren Angstgefühl. Darum mochte ich Ihr Buch\pwindex{Schnitzler, Arthur 15.05.1862 – 21.10.1931@\textsc{Schnitzler, Arthur} (15.05.1862 – 21.10.1931), \emph{Schriftsteller, Mediziner}!Sterben. Novelle1894-10-01 – 1894-12-01@\strich\emph{Sterben. Novelle} {[}1894-10-01 – 1894-12-01{]}|pwv} nicht, das ganz auf dieser einen Note
               gespielt wird, es potenzierte mein Eigenes zu stark{\dotsfour}\pend
           \pstart
           Jetzt ist es vorbei. Und an einem sehr schönen, duftenden, schwirrenden Tage will ich
                  »Sterben\pwindex{Schnitzler, Arthur 15.05.1862 – 21.10.1931@\textsc{Schnitzler, Arthur} (15.05.1862 – 21.10.1931), \emph{Schriftsteller, Mediziner}!Sterben. Novelle1894-10-01 – 1894-12-01@\strich\emph{Sterben. Novelle} {[}1894-10-01 – 1894-12-01{]}|pw}« wieder lesen. Wenn ich fühle, daß
                  {\pb}ich es kann.\pend
           \pstart
           Sie sind der einzige von allen Jungen, von dem ich etwas ganz Besonderes erwarten
               könnte, – dagegen bin ich nicht sicher, daß es Sie dauernd interessiren wird zu
               schreiben. Produciren ist doch auch nur eine Art von Stimulanz-Genuß {\dots} aber wieviele Stoffe können Naturen wie Sie stimuliren?
               Da Sie doch viel zu durchgebildet und von zu guter Herkunft sind als daß die
               äusserlichen Eitelkeits- und Erfolgsrücksichten viel für Sie bedeuten könnten.\pend
           \pstart
           Aber Ihr nächstes Buch schicken Sie mir wieder? nicht wahr?\pend
           \pstart
           Mit verbindlichem Gruß{\\[\baselineskip]} Ihre ergebene{\\[\baselineskip]}\spacefill\mbox{Laura Hansson-Marholm}\pend
           \leftskip=0em{}
         
         \endnumbering\mylabel{h}\end{ledgroupsized}  \newcommand{\dateiname}{L00431}\newcommand{\titel}{Laura Marholm an Arthur Schnitzler, 24. 4. 1895}\newcommand{\editorInnen}{Martin Anton Müller und Gerd-Hermann Susen}%% latex-leseansicht-abspann.tex
%% Abspann für die Leseansicht.
%% Der Schalter \ifkorrekturansicht ist bereits durch den Vorspann gesetzt.

%% latex-abspann.tex
%% Gemeinsamer Abspann für Korrekturansicht und Leseansicht.
%% Setzt den Schalter \ifkorrekturansicht voraus (gesetzt in den
%% einbindenden Dateien latex-korrekturansicht-abspann.tex bzw.
%% latex-leseansicht-abspann.tex).
%% ---------------------------------------------------------------

\normalsize

% Das esempio-Environment wird nur in der Leseansicht benötigt
\ifkorrekturansicht\else
\newenvironment{esempio}[3]%
{
    \vspace{1.5ex}
    \rlap{\underline{#1}}
    \par
    \setlength{\parindent}{0cm}
    \nopagebreak
    \leftskip=#2cm
    \rightskip=#3cm
}
{
    \par
}
\fi

\doendnotes{C}
\bigskip
\vfill

\clearpage

\footnotesize

\ifkorrekturansicht
  \lohead{\textsc{register}}
\fi

% theindex-Environment neu definieren ohne reledmac
\makeatletter
\renewenvironment{theindex}{%
  \ifkorrekturansicht
    \section*{\indexname}%
  \else
    \subsubsection*{Index der erwähnten Entitäten}%
  \fi
  \setlength{\parindent}{0pt}%
  \setlength{\parskip}{0pt plus 0.3pt}%
  \let\item\@idxitem
}{%
  \ifkorrekturansicht\clearpage\fi
}
\makeatother

\IfFileExists{\jobname-pw.ind}{\input{\jobname-pw.ind}}{}

% Quellenangabe nur in der Leseansicht
\ifkorrekturansicht\else
% Fallback-Definitionen, falls die .tex-Datei \titel etc. nicht gesetzt hat
\providecommand{\titel}{}
\providecommand{\editorInnen}{}
\providecommand{\dateiname}{\jobname}

\vspace{3cm}

\vfill

\footnotesize
\textsc{Quelle}: \titel. Herausgegeben von {\editorInnen}. In: \emph{Arthur Schnitzler: Briefwechsel mit Autorinnen und Autoren}.
 Digitale Edition, https://schnitzler-briefe.acdh.oeaw.ac.at/{\dateiname}.html (Stand \today)
\fi

\end{document}


      