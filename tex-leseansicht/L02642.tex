%% latex-korrekturansicht-vorspann.tex
%% Vorspann für die Korrekturansicht.
%% Lädt die gemeinsame Datei latex-vorspann.tex mit gesetztem Schalter.

\newif\ifkorrekturansicht
\korrekturansichttrue

\input{../tex-inputs/latex-vorspann}


\section[Paul Goldmann an Arthur Schnitzler, 4. 8. 1889]{L02642 Paul Goldmann an Arthur Schnitzler, 4. 8. 1889}
\nopagebreak\mylabel{L02642v}
\rehead{ }\normalsize\beginnumbering\briefempfaengerindex{Schnitzler, Arthur@\textsc{Schnitzler, Arthur}!zzzGoldmann, Paul@\emph{von Paul Goldmann}!1889-08-041@{4. 8. 1889}|(be}
\toendnotes[C]{\smallbreak\pagebreak[2]}\Standort{DLA, A:Schnitzler, HS.NZ85.1.3162.}
\physDesc{Brief, 1 Blatt, 3 Seiten, 1134 Zeichen
\newline{}Handschrift: schwarze Tinte, deutsche Kurrent
\newline{}Schnitzler: mit rotem Buntstift eine Unterstreichung }\toendnotes[C]{\smallbreak}
\pstart
           \centering{}{\pb}\textcolor{gray}{\textbf{\textbf{Adminiſtration: VII.
                           Seidengaſſe 7\oindex{Seidengasse@\textbf{Seidengasse}, \emph{Straße (K.STR)}|pw}} (Jos. Eberle {\kaufmannsund} Co.\orgindex{Josef Eberle Stein-, Buch und Musikaliendruckerei@Josef Eberle Stein-, Buch und Musikaliendruckerei|pw})}}\pend
           
\pstart
           \centering{}\textcolor{gray}{\textbf{An der Schönen Blauen Donau\orgindex{der schoenen blauen Donau@An der schönen blauen Donau|pw}}}\pend
           
\pstart
           \centering{}\textcolor{gray}{\textbf{Chef-Redacteur: Dr. F.
                        Mamroth\pwindex{Mamroth, Fedor 21.02.1851 – 25.06.1907@\textsc{Mamroth, Fedor} (21.02.1851 – 25.06.1907), \emph{Journalist/Journalistin, Kritiker/Kritikerin}|pw}. – Redaction: IX.,
                        Berggaſſe 31\oindex{Berggasse@\textbf{Berggasse}, \emph{Straße (K.STR)}|pw}.}}\pend
           
\pstart
           \raggedleft{}\textcolor{gray}{\textbf{Wien\oindex{Wien@\textbf{Wien}, \emph{A.ADM2}|pw}, den}}{ }4. Auguſt \textcolor{gray}{\textbf{18}}89.\pend
           
\pstart{}Verehrter Herr Doctor!\pend\vspace{0.5em}
\pstart
           Mein Onkel\pwindex{Mamroth, Fedor 21.02.1851 – 25.06.1907@\textsc{Mamroth, Fedor} (21.02.1851 – 25.06.1907), \emph{Journalist/Journalistin, Kritiker/Kritikerin}|pw}, mit dem ich geſtern beiſammen war, theilt mir mit, daß er ſich aus denſelben Gründen,
               wie ich, nämlich wegen der Düſterkeit des Süjets, ſcheut, Ihr Feuilleton\pwindex{Sohn. Aus den Papieren eines Arztes@\emph{Der Sohn. Aus den Papieren eines Arztes}|pwv} zu veröffentlichen. Im Übrigen
               hat es ihm ſehr gut gefallen und er möchte etwas \label{K_L02642-1v}\edtext{Anderes}{\lemma{\textnormal{\emph{Anderes}}}\Cendnote{\textnormal{Siehe Fedor Mamroth an Arthur Schnitzler, 2. 8. 1889.
               }}}\label{K_L02642-1} von Ihnen haben. Eine Ablehnung alſo, die Sie abſolut {\pb}nicht tragiſch nehmen dürfen. Das Nähere
               mündlich.\pend
           
\pstart
           Ich habe mich nämlich entſchloſſen, Ihre freundliche Aufforderung anzunehmen und mit
               Ihnen die \label{K_L02642-2v}\edtext{Parthie}{\lemma{\textnormal{\emph{Parthie}}}\Cendnote{\textnormal{Zwischen 10. 8. 1889 und 18. 8. 1889 wanderten Goldmann\pwindex{Goldmann, Paul 31.01.1865 – 25.09.1935@\textsc{Goldmann, Paul} (31.01.1865 – 25.09.1935), \emph{Schriftsteller/Schriftstellerin, Journalist/Journalistin}|pwk}, Schnitzler
                  und dessen Bruder Julius Schnitzler\pwindex{Schnitzler, Julius 13.07.1865 – 29.06.1939@\textsc{Schnitzler, Julius} (13.07.1865 – 29.06.1939), \emph{Chirurg/Chirurgin}|pwk} von Traunkirchen\oindex{Traunkirchen@\textbf{Traunkirchen}, \emph{P.PPL}|pwk} nach Reichenau\oindex{Reichenau [Schweiz]@\textbf{Reichenau [Schweiz]}, \emph{P.PPL}|pwk}.}}}\label{K_L02642-2} zu machen. Es fragt ſich freilich noch,
               ob ich die Fahrkarte bekomme, zur Zeit mit den redactionellen Arbeiten fertig werde
                  \textsc{etc}. Prinzipiell aber bin ich entſchloſſen, Donnerſtag{ }Abend von hier\oindex{Wien@\textbf{Wien}, \emph{A.ADM2}|pwv}
               abzureiſen und Sie Freitag{ }früh, wenn Sie inzwiſchen Ihre Entſchließungen nicht geändert haben
               ſollten, \label{K_L02642-3v}\edtext{irgendwo in der Welt}{\lemma{\textnormal{\emph{irgendwo in der Welt}}}\Cendnote{\textnormal{Sie trafen am 9. 8. 1889 auf dem Weg nach Traunkirchen\oindex{Traunkirchen@\textbf{Traunkirchen}, \emph{P.PPL}|pwk} zusammen.}}}\label{K_L02642-3} zu treffen. Ich
               bitte Sie alſo, mir umgehend mitzutheilen, wo Sie am Freitag ſind. {\pb}Vielleicht können Sie mich
               noch in \textsc{Ischl}\oindex{Bad Ischl@\textbf{Bad Ischl}, \emph{P.PPL}|pw} erwarten. Ich ſelbſt werde Ihnen am Donnerſtag
               meine mir zu beſtimmende Adreſſe \label{K_L02642-4v}\edtext{telegraphiren}{\lemma{\textnormal{\emph{telegraphiren}}}\Cendnote{\textnormal{Ein entsprechendes
                  Telegramm ist nicht überliefert.}}}\label{K_L02642-4}, ob ich mit meinen Angelegenheiten in
               Ordnung bin und kommen kann.\pend
           
\pstart
           Herzlichſten Gruß und Dank im Voraus! {\\[\baselineskip]}Ihr {\\[\baselineskip]}\spacefill\mbox{Dr. Paul Goldma{\geminationn}}\pend
           \leftskip=0em{}\selectlanguage{ngerman}\endnumbering\briefempfaengerindex{Schnitzler, Arthur@\textsc{Schnitzler, Arthur}!zzzGoldmann, Paul@\emph{von Paul Goldmann}!1889-08-041@{4. 8. 1889}|)be}\mylabel{L02642h}  \normalsize

\doendnotes{C}
\bigskip
\vfill

\clearpage

\footnotesize

\lohead{\textsc{register}}

% Definiere theindex-Environment komplett neu ohne reledmac
\makeatletter
\renewenvironment{theindex}{%
  \section*{\indexname}%
  \setlength{\parindent}{0pt}%
  \setlength{\parskip}{0pt plus 0.3pt}%
  \let\item\@idxitem
}{%
  \clearpage
}
\makeatother

\IfFileExists{\jobname-pw.ind}{\input{\jobname-pw.ind}}{}

\end{document}

      