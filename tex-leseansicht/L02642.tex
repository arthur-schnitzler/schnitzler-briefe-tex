%% latex-leseansicht-vorspann.tex
%% Vorspann für die Leseansicht.
%% Lädt die gemeinsame Datei latex-vorspann.tex mit nicht gesetztem Schalter.

\newif\ifkorrekturansicht
\korrekturansichtfalse

\input{../tex-inputs/latex-vorspann}


         
         \renewcommand{\erwaehntePersonen}{Personen: Paul Goldmann, Fedor Mamroth, Julius Schnitzler}
         \renewcommand{\erwaehnteInstitutionen}{Institutionen: An der schönen blauen Donau, Josef Eberle Stein-, Buch und Musikaliendruckerei}
         \renewcommand{\erwaehnteOrte}{Orte: Bad Ischl, Berggasse, Reichenau, Seidengasse, Traunkirchen, Wien}
         \renewcommand{\erwaehnteWerke}{Werke: Der Sohn. Aus den Papieren eines Arztes}
               \section[Paul Goldmann an Arthur Schnitzler, 4. 8. 1889]{ Paul Goldmann an Arthur Schnitzler, 4. 8. 1889}\nopagebreak\mylabel{v}\rehead{ }\begin{ledgroupsized}[t]{13cm}\normalsize\beginnumbering\briefempfaengerindex{Schnitzler, Arthur@\textsc{Schnitzler, Arthur}!zzzGoldmann, Paul@\emph{von Paul Goldmann}!1889-08-041@{4. 8. 1889}|(be} \toendnotes[C]{\smallbreak\pagebreak[2]} \Standort{DLA, A:Schnitzler, HS.NZ85.1.3162.}
\physDesc{Brief, 1 Blatt, 3 Seiten, 1134 Zeichen
\newline{}Handschrift: schwarze Tinte, deutsche Kurrent
\newline{}Schnitzler: mit rotem Buntstift eine Unterstreichung }\toendnotes[C]{\smallbreak}\pstart
           \noindent{}\centering{}{\pb}\textcolor{gray}{\textbf{\textbf{Adminiſtration: VII.
                           Seidengaſſe 7\oindex{Seidengasse@\textbf{Seidengasse}|pw}} (Jos. Eberle {\kaufmannsund} Co.\orgindex{Josef Eberle Stein-, Buch und Musikaliendruckerei@Josef Eberle Stein-, Buch und Musikaliendruckerei|pw})}}\pend
           \pstart
           \noindent{}\centering{}\textcolor{gray}{\textbf{An der Schönen Blauen Donau\orgindex{der schoenen blauen Donau@An der schönen blauen Donau|pw}}}\pend
           \pstart
           \noindent{}\centering{}\textcolor{gray}{\textbf{Chef-Redacteur: Dr. F.
                        Mamroth\pwindex{Mamroth, Fedor 21.02.1851 – 25.06.1907@\textsc{Mamroth, Fedor} (21.02.1851 – 25.06.1907), \emph{Journalist, Kritiker}|pw}. – Redaction: IX.,
                        Berggaſſe 31\oindex{Berggasse@\textbf{Berggasse}|pw}.}}\pend
           \pstart
           \raggedleft{}\textcolor{gray}{\textbf{Wien\oindex{Wien@\textbf{Wien}|pw}, den}}{ }4. Auguſt \textcolor{gray}{\textbf{18}}89.\pend
           \pstart{}Verehrter Herr Doctor!\pend\pstart
           Mein Onkel\pwindex{Mamroth, Fedor 21.02.1851 – 25.06.1907@\textsc{Mamroth, Fedor} (21.02.1851 – 25.06.1907), \emph{Journalist, Kritiker}|pw}, mit dem ich geſtern beiſammen war, theilt mir mit, daß er ſich aus denſelben Gründen,
               wie ich, nämlich wegen der Düſterkeit des Süjets, ſcheut, Ihr Feuilleton\pwindex{Schnitzler, Arthur 15.05.1862 – 21.10.1931@\textsc{Schnitzler, Arthur} (15.05.1862 – 21.10.1931), \emph{Schriftsteller, Mediziner}!Sohn. Aus den Papieren eines Arztes1. 1. 1892@\strich\emph{Der Sohn. Aus den Papieren eines Arztes} {[}1. 1. 1892{]}|pwv} zu veröffentlichen. Im Übrigen
               hat es ihm ſehr gut gefallen und er möchte etwas \label{K_L02642-1v}\edtext{Anderes}{\lemma{\textnormal{\emph{Anderes}}}\Cendnote{\textnormal{Siehe Fedor Mamroth an Arthur Schnitzler, 2. 8. 1889.
               }}}\label{K_L02642-1h} von Ihnen haben. Eine Ablehnung alſo, die Sie abſolut {\pb}nicht tragiſch nehmen dürfen. Das Nähere
               mündlich.\pend
           \pstart
           Ich habe mich nämlich entſchloſſen, Ihre freundliche Aufforderung anzunehmen und mit
               Ihnen die \label{K_L02642-2v}\edtext{Parthie}{\lemma{\textnormal{\emph{Parthie}}}\Cendnote{\textnormal{Zwischen 10. 8. 1889 und 18. 8. 1889 wanderten Goldmann\pwindex{Goldmann, Paul 31.01.1865 – 25.09.1935@\textsc{Goldmann, Paul} (31.01.1865 – 25.09.1935), \emph{Schriftsteller, Journalist}|pwk}, Schnitzler\pwindex{Schnitzler, Arthur 15.05.1862 – 21.10.1931@\textsc{Schnitzler, Arthur} (15.05.1862 – 21.10.1931), \emph{Schriftsteller, Mediziner}|pwk}
                  und dessen Bruder Julius Schnitzler\pwindex{Schnitzler, Julius 13.07.1865 – 29.06.1939@\textsc{Schnitzler, Julius} (13.07.1865 – 29.06.1939), \emph{Chirurg}|pwk} von Traunkirchen\oindex{Traunkirchen@\textbf{Traunkirchen}|pwk} nach Reichenau\oindex{Reichenau@\textbf{Reichenau}|pwk}.}}}\label{K_L02642-2h} zu machen. Es fragt ſich freilich noch,
               ob ich die Fahrkarte bekomme, zur Zeit mit den redactionellen Arbeiten fertig werde
                  \textsc{etc}. Prinzipiell aber bin ich entſchloſſen, Donnerſtag{ }Abend von hier\oindex{Wien@\textbf{Wien}|pwv}
               abzureiſen und Sie Freitag{ }früh, wenn Sie inzwiſchen Ihre Entſchließungen nicht geändert haben
               ſollten, \label{K_L02642-3v}\edtext{irgendwo in der Welt}{\lemma{\textnormal{\emph{irgendwo in der Welt}}}\Cendnote{\textnormal{Sie trafen am 9. 8. 1889 auf dem Weg nach Traunkirchen\oindex{Traunkirchen@\textbf{Traunkirchen}|pwk} zusammen.}}}\label{K_L02642-3h} zu treffen. Ich
               bitte Sie alſo, mir umgehend mitzutheilen, wo Sie am Freitag ſind. {\pb}Vielleicht können Sie mich
               noch in \textsc{Ischl}\oindex{Bad Ischl@\textbf{Bad Ischl}|pw} erwarten. Ich ſelbſt werde Ihnen am Donnerſtag
               meine mir zu beſtimmende Adreſſe \label{K_L02642-4v}\edtext{telegraphiren}{\lemma{\textnormal{\emph{telegraphiren}}}\Cendnote{\textnormal{Ein entsprechendes
                  Telegramm ist nicht überliefert.}}}\label{K_L02642-4h}, ob ich mit meinen Angelegenheiten in
               Ordnung bin und kommen kann.\pend
           \pstart
           Herzlichſten Gruß und Dank im Voraus! {\\[\baselineskip]}Ihr {\\[\baselineskip]}\spacefill\mbox{Dr. Paul Goldma{\geminationn}}\pend
           \leftskip=0em{}
         
         \endnumbering\mylabel{h}\end{ledgroupsized}  \newcommand{\dateiname}{L02642}\newcommand{\titel}{Paul Goldmann an Arthur Schnitzler, 4. 8. 1889}\newcommand{\editorInnen}{Martin Anton Müller und Laura Untner}%% latex-leseansicht-abspann.tex
%% Abspann für die Leseansicht.
%% Der Schalter \ifkorrekturansicht ist bereits durch den Vorspann gesetzt.

%% latex-abspann.tex
%% Gemeinsamer Abspann für Korrekturansicht und Leseansicht.
%% Setzt den Schalter \ifkorrekturansicht voraus (gesetzt in den
%% einbindenden Dateien latex-korrekturansicht-abspann.tex bzw.
%% latex-leseansicht-abspann.tex).
%% ---------------------------------------------------------------

\normalsize

% Das esempio-Environment wird nur in der Leseansicht benötigt
\ifkorrekturansicht\else
\newenvironment{esempio}[3]%
{
    \vspace{1.5ex}
    \rlap{\underline{#1}}
    \par
    \setlength{\parindent}{0cm}
    \nopagebreak
    \leftskip=#2cm
    \rightskip=#3cm
}
{
    \par
}
\fi

\doendnotes{C}
\bigskip
\vfill

\clearpage

\footnotesize

\ifkorrekturansicht
  \lohead{\textsc{register}}
\fi

% theindex-Environment neu definieren ohne reledmac
\makeatletter
\renewenvironment{theindex}{%
  \ifkorrekturansicht
    \section*{\indexname}%
  \else
    \subsubsection*{Index der erwähnten Entitäten}%
  \fi
  \setlength{\parindent}{0pt}%
  \setlength{\parskip}{0pt plus 0.3pt}%
  \let\item\@idxitem
}{%
  \ifkorrekturansicht\clearpage\fi
}
\makeatother

\IfFileExists{\jobname-pw.ind}{\input{\jobname-pw.ind}}{}

% Quellenangabe nur in der Leseansicht
\ifkorrekturansicht\else
% Fallback-Definitionen, falls die .tex-Datei \titel etc. nicht gesetzt hat
\providecommand{\titel}{}
\providecommand{\editorInnen}{}
\providecommand{\dateiname}{\jobname}

\vspace{3cm}

\vfill

\footnotesize
\textsc{Quelle}: \titel. Herausgegeben von {\editorInnen}. In: \emph{Arthur Schnitzler: Briefwechsel mit Autorinnen und Autoren}.
 Digitale Edition, https://schnitzler-briefe.acdh.oeaw.ac.at/{\dateiname}.html (Stand \today)
\fi

\end{document}


      