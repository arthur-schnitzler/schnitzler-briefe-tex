%% latex-leseansicht-vorspann.tex
%% Vorspann für die Leseansicht.
%% Lädt die gemeinsame Datei latex-vorspann.tex mit nicht gesetztem Schalter.

\newif\ifkorrekturansicht
\korrekturansichtfalse

\input{../tex-inputs/latex-vorspann}


\section[Georg Brandes an Arthur Schnitzler, 11. 3. 1906]{L01589 Georg Brandes an Arthur Schnitzler, 11. 3. 1906}
\nopagebreak\mylabel{L01589v}
\rehead{ }\normalsize\beginnumbering\briefempfaengerindex{Schnitzler, Arthur@\textsc{Schnitzler, Arthur}!zzzBrandes, Georg@\emph{von Georg Brandes}!1906-03-112@{11. 3. 1906}|(be}
\toendnotes[C]{\smallbreak\pagebreak[2]}
\correspDesc{Versand  durch Georg Brandes am 11. 3. 1906 \textbf{Ort fehlend} 
\newline{}Erhalt  durch Arthur Schnitzler im Zeitraum [11. 3. 1906
                  – 15. 3. 1906?] in Wien}\toendnotes[C]{\smallbreak}
\Standort{CUL, Schnitzler, B 17.}
\physDesc{Brief, 1 Blatt, 3 Seiten, 1126 Zeichen
\newline{}Handschrift: blaue Tinte, lateinische Kurrent
\newline{}Ordnung: mit Bleistift von unbekannter Hand nummeriert:
                                    »30« }
\buchAbdrucke{\weitereDrucke{Georg Brandes, Arthur Schnitzler: \emph{Ein Briefwechsel}. Herausgegeben von Kurt Bergel. Bern: \emph{Francke} 1956, S. 91.} }\toendnotes[C]{\smallbreak}
\pstart
           \raggedleft{}{\pb}Kopenhagen\oindex{Kopenhagen@\textbf{Kopenhagen}, \emph{Hauptstadt}|pw}{ }11 März 1906\pend
           
\pstart{}Verehrter und lieber Freund\pend\vspace{0.5em}
\pstart
           Haben Sie herzlichen Dank für die gute Gabe, die Sie mir schickten, Ihr letztes Schauspiel\pwindex{Schnitzler, Arthur 15.\,5.\,1862 Wien – 21.\,10.\,1931 ebd.@\textsc{Schnitzler, Arthur} (15.\,5.\,1862 Wien – 21.\,10.\,1931 ebd.), \emph{Schriftsteller, Mediziner}!Ruf des Lebens. Schauspiel in drei Akten@\strich\emph{Der Ruf des Lebens. Schauspiel in drei Akten}|pwv}. Ich habe meine
               Freude daran gehabt. Die Welt Ihrer Phantasie zieht mich immer an und erregt meine
               Bewunderung, da ich selbst wenig Phantasie besitze und erstaune, dass ein anderer all
               das erfinden kann.\pend
           
\pstart
           Seit lange beschäftigt es Sie, wie der Gedanke an den nahen Tod die Gefühle
               beeinflusst, Schleier der Beatrice\pwindex{Schnitzler, Arthur 15.\,5.\,1862 Wien – 21.\,10.\,1931 ebd.@\textsc{Schnitzler, Arthur} (15.\,5.\,1862 Wien – 21.\,10.\,1931 ebd.), \emph{Schriftsteller, Mediziner}!Schleier der Beatrice. Schauspiel in fünf Akten@\strich\emph{Der Schleier der Beatrice. Schauspiel in fünf Akten}|pw}, Lieutenant Gustl\pwindex{Schnitzler, Arthur 15.\,5.\,1862 Wien – 21.\,10.\,1931 ebd.@\textsc{Schnitzler, Arthur} (15.\,5.\,1862 Wien – 21.\,10.\,1931 ebd.), \emph{Schriftsteller, Mediziner}!Lieutenant Gustl. Novelle@\strich\emph{Lieutenant Gustl. Novelle}|pw}, usw. Hier variiren Sie das
               Thema; der Gedanke an den Tod des Liebsten wirkt ebenso. Sie sind ein Grübler über
               den Tod, wie schon Ihr »Sterben\pwindex{Schnitzler, Arthur 15.\,5.\,1862 Wien – 21.\,10.\,1931 ebd.@\textsc{Schnitzler, Arthur} (15.\,5.\,1862 Wien – 21.\,10.\,1931 ebd.), \emph{Schriftsteller, Mediziner}!Sterben. Novelle@\strich\emph{Sterben. Novelle}|pw}« zeigte. Die
               Hälfte {\pb}Ihrer Produktion ist
               Thanatos, die Hälfte Eros gewidmet. Aber dadurch haben Ihre Arbeiten eine so grosse
               Spannweite (wenn das Wort deutsch ist).\pend
           
\pstart
           Ich las eine sehr unverständige \label{K_L01589-1v}\edtext{Kritik\pwindex{Schönhoff, Leopold 1853 Vyškov – 2.\,5.\,1908 Wiesbaden@\textsc{Schönhoff, Leopold} (1853 Vyškov – 2.\,5.\,1908 Wiesbaden), \emph{Journalist, Kritiker}!Ruf des Lebens.« Schauspiel von Artur Schnitzler. Erste Aufführung im Lessingtheater@\strich\emph{»Der Ruf des Lebens.« Schauspiel von Artur Schnitzler. Erste Aufführung im Lessingtheater}|pwv}}{\lemma{\textnormal{\emph{Kritik}}}\Cendnote{\textnormal{L. Schönhoff\pwindex{Schönhoff, Leopold 1853 Vyškov – 2.\,5.\,1908 Wiesbaden@\textsc{Schönhoff, Leopold} (1853 Vyškov – 2.\,5.\,1908 Wiesbaden), \emph{Journalist, Kritiker}|pwk}: \emph{»Der Ruf des Lebens.« Schauspiel von Artur Schnitzler. Erste
                        Aufführung im Lessing-Theater}\pwindex{Schönhoff, Leopold 1853 Vyškov – 2.\,5.\,1908 Wiesbaden@\textsc{Schönhoff, Leopold} (1853 Vyškov – 2.\,5.\,1908 Wiesbaden), \emph{Journalist, Kritiker}!Ruf des Lebens.« Schauspiel von Artur Schnitzler. Erste Aufführung im Lessingtheater@\strich\emph{»Der Ruf des Lebens.« Schauspiel von Artur Schnitzler. Erste Aufführung im Lessingtheater}|pwk}. In: \emph{Der
                        Tag}\pwindex{Tag@\emph{Der Tag}|pwk}, Nr. 105, Ausgabe A, 27. 2. 1906, Erster Teil,
                     S. 1–2.}}}\label{K_L01589-1} über Ihr Werk\pwindex{Schnitzler, Arthur 15.\,5.\,1862 Wien – 21.\,10.\,1931 ebd.@\textsc{Schnitzler, Arthur} (15.\,5.\,1862 Wien – 21.\,10.\,1931 ebd.), \emph{Schriftsteller, Mediziner}!Ruf des Lebens. Schauspiel in drei Akten@\strich\emph{Der Ruf des Lebens. Schauspiel in drei Akten}|pwv} in dem \uline{Tag}\orgindex{Tag@Der Tag|pw}; es scheint mir, dass die meiste deutsche Kritik allzu viel fertige Begriffe
               und Ansprüche mitbringt; sie ist weniger geschmeidig als die unsrige.\pend
           
\pstart
           Es war mir sehr lieb, Sie jene \label{K_L01589-2v}\edtext{Stunde}{\lemma{\textnormal{\emph{Stunde}}}\Cendnote{\textnormal{am 19. 11. 1905
                  in Berlin\oindex{Berlin@\textbf{Berlin}, \emph{Hauptstadt}|pwk}}}}\label{K_L01589-2} bei Fulda\pwindex{Fulda, Ludwig 15.\,7.\,1862 Frankfurt am Main – 30.\,3.\,1939 Berlin@\textsc{Fulda, Ludwig} (15.\,7.\,1862 Frankfurt am Main – 30.\,3.\,1939 Berlin), \emph{Schriftsteller, Übersetzer}|pw} zu treffen. Ich möchte, dass
               Sie wieder einmal nach Dänemark\oindex{Dänemark@\textbf{Dänemark}|pw} kämen.\pend
           
\pstart
           Ihr dankbar verbundener{\\[\baselineskip]}\spacefill\mbox{Georg Brandes}\pend
           \leftskip=0em{}\selectlanguage{ngerman}\endnumbering\briefempfaengerindex{Schnitzler, Arthur@\textsc{Schnitzler, Arthur}!zzzBrandes, Georg@\emph{von Georg Brandes}!1906-03-112@{11. 3. 1906}|)be}\mylabel{L01589h}  \newcommand{\dateiname}{L01589}\newcommand{\titel}{Georg Brandes an Arthur Schnitzler, 11. 3. 1906}\newcommand{\editorInnen}{Martin Anton Müller und Gerd-Hermann Susen}%% latex-leseansicht-abspann.tex
%% Abspann für die Leseansicht.
%% Der Schalter \ifkorrekturansicht ist bereits durch den Vorspann gesetzt.

%% latex-abspann.tex
%% Gemeinsamer Abspann für Korrekturansicht und Leseansicht.
%% Setzt den Schalter \ifkorrekturansicht voraus (gesetzt in den
%% einbindenden Dateien latex-korrekturansicht-abspann.tex bzw.
%% latex-leseansicht-abspann.tex).
%% ---------------------------------------------------------------

\normalsize

% Das esempio-Environment wird nur in der Leseansicht benötigt
\ifkorrekturansicht\else
\newenvironment{esempio}[3]%
{
    \vspace{1.5ex}
    \rlap{\underline{#1}}
    \par
    \setlength{\parindent}{0cm}
    \nopagebreak
    \leftskip=#2cm
    \rightskip=#3cm
}
{
    \par
}
\fi

\doendnotes{C}
\bigskip
\vfill

\clearpage

\footnotesize

\ifkorrekturansicht
  \lohead{\textsc{register}}
\fi

% theindex-Environment neu definieren ohne reledmac
\makeatletter
\renewenvironment{theindex}{%
  \ifkorrekturansicht
    \section*{\indexname}%
  \else
    \subsubsection*{Index der erwähnten Entitäten}%
  \fi
  \setlength{\parindent}{0pt}%
  \setlength{\parskip}{0pt plus 0.3pt}%
  \let\item\@idxitem
}{%
  \ifkorrekturansicht\clearpage\fi
}
\makeatother

\IfFileExists{\jobname-pw.ind}{\input{\jobname-pw.ind}}{}

% Quellenangabe nur in der Leseansicht
\ifkorrekturansicht\else
% Fallback-Definitionen, falls die .tex-Datei \titel etc. nicht gesetzt hat
\providecommand{\titel}{}
\providecommand{\editorInnen}{}
\providecommand{\dateiname}{\jobname}

\vspace{3cm}

\vfill

\footnotesize
\textsc{Quelle}: \titel. Herausgegeben von {\editorInnen}. In: \emph{Arthur Schnitzler: Briefwechsel mit Autorinnen und Autoren}.
 Digitale Edition, https://schnitzler-briefe.acdh.oeaw.ac.at/{\dateiname}.html (Stand \today)
\fi

\end{document}


