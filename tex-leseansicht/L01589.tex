%% latex-korrekturansicht-vorspann.tex
%% Vorspann für die Korrekturansicht.
%% Lädt die gemeinsame Datei latex-vorspann.tex mit gesetztem Schalter.

\newif\ifkorrekturansicht
\korrekturansichttrue

\input{../tex-inputs/latex-vorspann}


\section[Georg Brandes an Arthur Schnitzler, 11. 3. 1906]{L01589 Georg Brandes an Arthur Schnitzler, 11. 3. 1906}
\nopagebreak\mylabel{L01589v}
\rehead{ }\normalsize\beginnumbering\briefempfaengerindex{Schnitzler, Arthur@\textsc{Schnitzler, Arthur}!zzzBrandes, Georg@\emph{von Georg Brandes}!1906-03-111@{11. 3. 1906}|(be}
\toendnotes[C]{\smallbreak\pagebreak[2]}\Standort{CUL, Schnitzler, B 17.}
\physDesc{Brief, 1 Blatt, 3 Seiten, 1126 Zeichen
\newline{}Handschrift: blaue Tinte, lateinische Kurrent
\newline{}Ordnung: mit Bleistift von unbekannter Hand nummeriert:
                                    »30« }
\buchAbdrucke{\weitereDrucke{Georg Brandes, Arthur Schnitzler: \emph{Ein Briefwechsel}. Bern: \emph{Francke} 1956, S. 91.} }\toendnotes[C]{\smallbreak}
\pstart
           \raggedleft{}{\pb}Kopenhagen\oindex{Kopenhagen@\textbf{Kopenhagen}, \emph{P.PPLC}|pw}{ }11 März 1906\pend
           
\pstart{}Verehrter und lieber Freund\pend\vspace{0.5em}
\pstart
           Haben Sie herzlichen Dank für die gute Gabe, die Sie mir schickten, Ihr letztes Schauspiel\pwindex{Ruf des Lebens. Schauspiel in drei Akten@\emph{Der Ruf des Lebens. Schauspiel in drei Akten}|pwv}. Ich habe meine
               Freude daran gehabt. Die Welt Ihrer Phantasie zieht mich immer an und erregt meine
               Bewunderung, da ich selbst wenig Phantasie besitze und erstaune, dass ein anderer all
               das erfinden kann.\pend
           
\pstart
           Seit lange beschäftigt es Sie, wie der Gedanke an den nahen Tod die Gefühle
               beeinflusst, Schleier der Beatrice\pwindex{Schleier der Beatrice. Schauspiel in fuenf Akten@\emph{Der Schleier der Beatrice. Schauspiel in fünf Akten}|pw}, Lieutenant Gustl\pwindex{Lieutenant Gustl. Novelle@\emph{Lieutenant Gustl. Novelle}|pw}, usw. Hier variiren Sie das
               Thema; der Gedanke an den Tod des Liebsten wirkt ebenso. Sie sind ein Grübler über
               den Tod, wie schon Ihr »Sterben\pwindex{Sterben. Novelle@\emph{Sterben. Novelle}|pw}« zeigte. Die
               Hälfte {\pb}Ihrer Produktion ist
               Thanatos, die Hälfte Eros gewidmet. Aber dadurch haben Ihre Arbeiten eine so grosse
               Spannweite (wenn das Wort deutsch ist).\pend
           
\pstart
           Ich las eine sehr unverständige \label{K_L01589-1v}\edtext{Kritik\pwindex{Ruf des Lebens.« Schauspiel von Artur Schnitzler. Erste Auffuehrung im Lessingtheater@\emph{»Der Ruf des Lebens.« Schauspiel von Artur Schnitzler. Erste Aufführung im Lessingtheater}|pwv}}{\lemma{\textnormal{\emph{Kritik}}}\Cendnote{\textnormal{L. Schönhoff\pwindex{Schoenhoff, Leopold 1853 – 02.05.1908@\textsc{Schönhoff, Leopold} (1853 – 02.05.1908), \emph{Journalist/Journalistin, Kritiker/Kritikerin}|pwk}: \emph{»Der Ruf des Lebens.« Schauspiel von Artur Schnitzler. Erste
                        Aufführung im Lessing-Theater}\pwindex{Ruf des Lebens.« Schauspiel von Artur Schnitzler. Erste Auffuehrung im Lessingtheater@\emph{»Der Ruf des Lebens.« Schauspiel von Artur Schnitzler. Erste Aufführung im Lessingtheater}|pwk}. In: \emph{Der
                        Tag}\pwindex{Tag@\emph{Der Tag}|pwk}, Nr. 105, Ausgabe A, 27. 2. 1906, Erster Teil,
                     S. 1–2.}}}\label{K_L01589-1} über Ihr Werk\pwindex{Ruf des Lebens. Schauspiel in drei Akten@\emph{Der Ruf des Lebens. Schauspiel in drei Akten}|pwv} in dem \uline{Tag}\orgindex{Tag@Der Tag|pw}; es scheint mir, dass die meiste deutsche Kritik allzu viel fertige Begriffe
               und Ansprüche mitbringt; sie ist weniger geschmeidig als die unsrige.\pend
           
\pstart
           Es war mir sehr lieb, Sie jene \label{K_L01589-2v}\edtext{Stunde}{\lemma{\textnormal{\emph{Stunde}}}\Cendnote{\textnormal{am 19. 11. 1905
                  in Berlin\oindex{Berlin@\textbf{Berlin}, \emph{P.PPLC}|pwk}}}}\label{K_L01589-2} bei Fulda\pwindex{Fulda, Ludwig 15.07.1862 – 30.03.1939@\textsc{Fulda, Ludwig} (15.07.1862 – 30.03.1939), \emph{Schriftsteller/Schriftstellerin, Übersetzer/Übersetzerin}|pw} zu treffen. Ich möchte, dass
               Sie wieder einmal nach Dänemark\oindex{Daenemark@\textbf{Dänemark}, \emph{A.PCLI}|pw} kämen.\pend
           
\pstart
           Ihr dankbar verbundener{\\[\baselineskip]}\spacefill\mbox{Georg Brandes}\pend
           \leftskip=0em{}\selectlanguage{ngerman}\endnumbering\briefempfaengerindex{Schnitzler, Arthur@\textsc{Schnitzler, Arthur}!zzzBrandes, Georg@\emph{von Georg Brandes}!1906-03-111@{11. 3. 1906}|)be}\mylabel{L01589h}  \normalsize

\doendnotes{C}
\bigskip
\vfill

\clearpage

\footnotesize

\lohead{\textsc{register}}

% Definiere theindex-Environment komplett neu ohne reledmac
\makeatletter
\renewenvironment{theindex}{%
  \section*{\indexname}%
  \setlength{\parindent}{0pt}%
  \setlength{\parskip}{0pt plus 0.3pt}%
  \let\item\@idxitem
}{%
  \clearpage
}
\makeatother

\IfFileExists{\jobname-pw.ind}{\input{\jobname-pw.ind}}{}

\end{document}

      