%% latex-leseansicht-vorspann.tex
%% Vorspann für die Leseansicht.
%% Lädt die gemeinsame Datei latex-vorspann.tex mit nicht gesetztem Schalter.

\newif\ifkorrekturansicht
\korrekturansichtfalse

\input{../tex-inputs/latex-vorspann}


\section[Felix Salten u. a. an Arthur Schnitzler, {[}November 1927 – Juni 1928?{]}]{L03580 Felix Salten u. a. an Arthur Schnitzler, {[}November 1927 – Juni 1928?{]}}
\nopagebreak\mylabel{L03580v}
\rehead{ }\normalsize\beginnumbering\briefempfaengerindex{Schnitzler, Arthur@\textsc{Schnitzler, Arthur}!zzzRehmann, Anna Katharina@\emph{von Anna Katharina Rehmann}!1928-06-302@{{[}November 1927 – Juni 1928?{]}}|(be}\briefempfaengerindex{Schnitzler, Arthur@\textsc{Schnitzler, Arthur}!zzzSalten, Paul@\emph{von Paul Salten}!1928-06-302@{{[}November 1927 – Juni 1928?{]}}|(be}\briefempfaengerindex{Schnitzler, Arthur@\textsc{Schnitzler, Arthur}!zzzSalten, Ottilie@\emph{von Ottilie Salten}!1928-06-302@{{[}November 1927 – Juni 1928?{]}}|(be}\briefempfaengerindex{Schnitzler, Arthur@\textsc{Schnitzler, Arthur}!zzzSalten, Felix@\emph{von Felix Salten}!1928-06-302@{{[}November 1927 – Juni 1928?{]}}|(be}
\toendnotes[C]{\smallbreak\pagebreak[2]}
\correspDesc{Versand  durch Felix Salten, Ottilie Salten, Paul Salten, Anna Katharina Salten im Zeitraum [November 1927 –
                  Juni 1928?] in Chemnitz
\newline{}Erhalt  durch Arthur Schnitzler im Zeitraum [November 1927 –
                  Juni 1928?] in Wien}\toendnotes[C]{\smallbreak}
\Standort{CUL, Schnitzler, B 89, B 2.}
\physDesc{Bildpostkarte, 190 Zeichen
\newline{}Handschrift Felix Salten: Bleistift, lateinische Kurrent
\newline{}Handschrift Ottilie Salten: Bleistift, lateinische Kurrent
\newline{}Handschrift Paul Salten: Bleistift
\newline{}Handschrift Anna Katharina Rehmann: Bleistift, lateinische Kurrent
\newline{}Versand: Stempel: »\nobreak{}\oindex{Chemnitz@\textbf{Chemnitz}, \emph{Hauptstadt}|pwk}Chemnitz 4, \textcolor{gray}{1}7–1\textcolor{gray}{8}\nobreak{}«.  }\toendnotes[C]{\smallbreak}\pstart{}{\pb}Herrn D\textsuperscript{r} Arthur Schnitzler\pend{}\pstart{}Wien\oindex{Wien@\textbf{Wien}, \emph{Verwaltungsgebiet}|pw}\pend{}\pstart{}XVIII. Sternwartestraße 71\oindex{Wien@\textbf{Wien}!XVIII., Währing@\textbf{XVIII., Währing}!Sternwartestraße 71@\textbf{Sternwartestraße 71}, \emph{Wohngebäude}|pw}\pend{}{\bigskip}
\pstart
           \noindent{}\centering{}{\pb}\textcolor{gray}{\textbf{Chemnitz}}\oindex{Chemnitz@\textbf{Chemnitz}, \emph{Hauptstadt}|pw}\pend
           
\pstart
           \centering{}\textcolor{gray}{\textbf{»Die Jägerin\pwindex{\textcolor{red}{\textsuperscript{XXXX indx1}}!Diana mit Bogen@\strich\emph{Diana mit Bogen}|pw}« (Kassberg-Auffahrt\oindex{Kaßbergauffahrt@\textbf{Kaßbergauffahrt}, \emph{Straße}|pw})}}\pend
           \vspace{1em}
\pstart
           \noindent{}{\pb}Viele herzlichste Grüße\pend
           \pstart Ihr \spacefill\mbox{Felix Salten}\pend{}\selectlanguage{ngerman}\vspace{1em}
\pstart
           \noindent{}{[}hs. Salten:{]} Wir sind \label{K_L03580-1v}\edtext{heute{ }hier\oindex{Chemnitz@\textbf{Chemnitz}, \emph{Hauptstadt}|pwv}}{\lemma{\textnormal{\emph{heute hier}}}\Cendnote{\textnormal{Die Karte ist undatiert und der
                  Poststempel verwischt, sodass die Datierung nur annäherungsweise mittels Indizien
                  erfolgen kann. Die verwendete Briefmarke wurde erstmals am 1. 11. 1927 ausgegeben, womit der frühestmögliche Zeitpunkt benannt
                  ist. In der Theatersaison 1927/1928 war Anna Katharina Salten\pwindex{Rehmann, Anna Katharina 18.\,8.\,1904 Wien – 27.\,3.\,1977 Zürich@\textsc{Rehmann, Anna Katharina} (18.\,8.\,1904 Wien – 27.\,3.\,1977 Zürich), \emph{Schauspielerin, Übersetzerin}|pwk}
                  am \emph{Städtischen Theater}\orgindex{Städtische Theater Chemnitz@Städtische Theater Chemnitz|pwk} in Chemnitz\oindex{Chemnitz@\textbf{Chemnitz}, \emph{Hauptstadt}|pwk} engagiert, was der Grund für die Reise der
                  Familie gewesen sein dürfte und zugleich eine Einschränkung der Datierung nach
                  hinten hin ermöglicht.}}}\label{K_L03580-1} sehr froh und denken Ihrer herzlichst {\\}\spacefill\mbox{Otti S.}{\\}{[}hs. Salten:{]} \spacefill\mbox{Paul Salten}\pend
           \selectlanguage{ngerman}\vspace{1em}
\pstart
           \noindent{}{[}hs. Rehmann:{]} Viele liebe Grüße! {\\}\spacefill\mbox{Annerl}\pend
           \selectlanguage{ngerman}\endnumbering\briefempfaengerindex{Schnitzler, Arthur@\textsc{Schnitzler, Arthur}!zzzRehmann, Anna Katharina@\emph{von Anna Katharina Rehmann}!1927-11-012@{{[}November 1927 – Juni 1928?{]}}|)be}\briefempfaengerindex{Schnitzler, Arthur@\textsc{Schnitzler, Arthur}!zzzSalten, Paul@\emph{von Paul Salten}!1927-11-012@{{[}November 1927 – Juni 1928?{]}}|)be}\briefempfaengerindex{Schnitzler, Arthur@\textsc{Schnitzler, Arthur}!zzzSalten, Ottilie@\emph{von Ottilie Salten}!1927-11-012@{{[}November 1927 – Juni 1928?{]}}|)be}\briefempfaengerindex{Schnitzler, Arthur@\textsc{Schnitzler, Arthur}!zzzSalten, Felix@\emph{von Felix Salten}!1927-11-012@{{[}November 1927 – Juni 1928?{]}}|)be}\mylabel{L03580h}  \newcommand{\dateiname}{L03580}\newcommand{\titel}{Felix Salten u. a. an Arthur Schnitzler, [November 1927 – Juni 1928?]}\newcommand{\editorInnen}{Martin Anton Müller und Laura Untner}%% latex-leseansicht-abspann.tex
%% Abspann für die Leseansicht.
%% Der Schalter \ifkorrekturansicht ist bereits durch den Vorspann gesetzt.

%% latex-abspann.tex
%% Gemeinsamer Abspann für Korrekturansicht und Leseansicht.
%% Setzt den Schalter \ifkorrekturansicht voraus (gesetzt in den
%% einbindenden Dateien latex-korrekturansicht-abspann.tex bzw.
%% latex-leseansicht-abspann.tex).
%% ---------------------------------------------------------------

\normalsize

% Das esempio-Environment wird nur in der Leseansicht benötigt
\ifkorrekturansicht\else
\newenvironment{esempio}[3]%
{
    \vspace{1.5ex}
    \rlap{\underline{#1}}
    \par
    \setlength{\parindent}{0cm}
    \nopagebreak
    \leftskip=#2cm
    \rightskip=#3cm
}
{
    \par
}
\fi

\doendnotes{C}
\bigskip
\vfill

\clearpage

\footnotesize

\ifkorrekturansicht
  \lohead{\textsc{register}}
\fi

% theindex-Environment neu definieren ohne reledmac
\makeatletter
\renewenvironment{theindex}{%
  \ifkorrekturansicht
    \section*{\indexname}%
  \else
    \subsubsection*{Index der erwähnten Entitäten}%
  \fi
  \setlength{\parindent}{0pt}%
  \setlength{\parskip}{0pt plus 0.3pt}%
  \let\item\@idxitem
}{%
  \ifkorrekturansicht\clearpage\fi
}
\makeatother

\IfFileExists{\jobname-pw.ind}{\input{\jobname-pw.ind}}{}

% Quellenangabe nur in der Leseansicht
\ifkorrekturansicht\else
% Fallback-Definitionen, falls die .tex-Datei \titel etc. nicht gesetzt hat
\providecommand{\titel}{}
\providecommand{\editorInnen}{}
\providecommand{\dateiname}{\jobname}

\vspace{3cm}

\vfill

\footnotesize
\textsc{Quelle}: \titel. Herausgegeben von {\editorInnen}. In: \emph{Arthur Schnitzler: Briefwechsel mit Autorinnen und Autoren}.
 Digitale Edition, https://schnitzler-briefe.acdh.oeaw.ac.at/{\dateiname}.html (Stand \today)
\fi

\end{document}


