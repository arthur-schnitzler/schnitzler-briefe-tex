%% latex-leseansicht-vorspann.tex
%% Vorspann für die Leseansicht.
%% Lädt die gemeinsame Datei latex-vorspann.tex mit nicht gesetztem Schalter.

\newif\ifkorrekturansicht
\korrekturansichtfalse

\input{../tex-inputs/latex-vorspann}

\begin{center}
            \textcolor{red}{ENTWURF, NICHT FERTIG KORRIGIERT}
                      \end{center}
            
         \renewcommand{\erwaehnteInstitutionen}{Institutionen: Städtische Theater Chemnitz}
         \renewcommand{\erwaehnteOrte}{Orte: Chemnitz, Kaßbergauffahrt, Sternwartestraße, Wien}
         \renewcommand{\erwaehnteWerke}{Werke: Diana mit Bogen}
               \section[Felix Salten u. a. an Arthur Schnitzler, {[}November 1927 – Juni 1928?{]}]{ Felix Salten u. a. an Arthur Schnitzler, {[}November 1927 – Juni
               1928?{]}}\nopagebreak\mylabel{v}\rehead{ }\begin{ledgroupsized}[t]{13cm}\normalsize\beginnumbering \toendnotes[C]{\smallbreak\pagebreak[2]} \Standort{CUL, Schnitzler, B 89, B 2.}
\physDesc{Bildpostkarte, 190 Zeichen
\newline{}Handschrift Felix Salten: Bleistift, lateinische Kurrent\newline{}Handschrift Ottilie Salten: Bleistift, lateinische Kurrent\newline{}Handschrift Paul Salten: Bleistift\newline{}Handschrift Anna Katharina Rehmann: Bleistift, lateinische Kurrent
\newline{}Versand: Stempel: »\nobreak{}\oindex{Chemnitz@\textbf{Chemnitz}|pwk}Chemnitz\nobreak{}«.  }\toendnotes[C]{\smallbreak}\pstart{}{\pb}Herrn D\textsuperscript{r} Arthur Schnitzler\pend{}\pstart{}Wien\oindex{Wien@\textbf{Wien}|pw}\pend{}\pstart{}XVIII. Sternwartestraße 71\oindex{Sternwartestrasse@\textbf{Sternwartestraße}|pw}\pend{}{\bigskip}\pstart
           \noindent{}{\pb}\textcolor{gray}{\textbf{Chemnitz}}\oindex{Chemnitz@\textbf{Chemnitz}|pw}\pend
           \pstart
           \textcolor{gray}{\textbf{»Die Jägerin\pwindex{\textcolor{red}{\textsuperscript{XXXX1 indx}}!Diana mit Bogen1913@\strich\emph{Diana mit Bogen} {[}1913{]}|pw}« (Kassberg-Auffahrt\oindex{Kassbergauffahrt@\textbf{Kaßbergauffahrt}|pw})}}\pend
           \pstart
           Viele herzlichste Grüße\pend
           \pstart Ihr \spacefill\mbox{Felix Salten}\pend{}\pstart
           \noindent{}{[}hs. Ottilie Salten:{]} Wir sind \label{K_L03580-1v}\edtext{heute hier}{\lemma{\textnormal{\emph{heute hier}}}\Cendnote{\textnormal{Die Karte ist undatiert und
                  der Poststempel verwischt, so dass die Datierung nur mittels Indizien und nur
                  annäherungsweise erfolgen kann. Die verwendete Briefmarke wurde erstmals am 1. 11. 1927
               ausgegeben, womit der frühest mögliche Zeitpunkt benannt ist. In der Theatersaison 1927/1928 war
                  Anna Katharina Salten\pwindex{Rehmann, Anna Katharina 18.08.1904 – 27.03.1977@\textsc{Rehmann, Anna Katharina} (18.08.1904 – 27.03.1977), \emph{Schauspielerin, Übersetzerin}|pwk} am \emph{Städtischen Theater}\orgindex{Staedtische Theater Chemnitz@Städtische Theater Chemnitz|pwk}
                  in Chemnitz\oindex{Chemnitz@\textbf{Chemnitz}|pwk} engagiert, was den Grund für die
                  Reise der Familie liefern dürfte.}}}\label{K_L03580-1h} sehr froh und denken Ihrer herzlichst\pend
           \pstart
           \spacefill\mbox{Otti S.}{\\[\baselineskip]}{[}hs. Paul Salten:{]} \spacefill\mbox{Paul Salten}\pend
           \leftskip=0em{}\pstart
           \noindent{}{[}hs. Rehmann:{]} Viele liebe Grüße! \pend
           \pstart \spacefill\mbox{Annerl}\pend{}
         
         \endnumbering\mylabel{h}\end{ledgroupsized}\begin{anhang}\end{anhang}\newcommand{\dateiname}{L03580}\newcommand{\titel}{Felix Salten u. a. an Arthur Schnitzler, [November 1927 – Juni 1928?]}\newcommand{\editorInnen}{Martin Anton Müller und Laura Untner}%% latex-leseansicht-abspann.tex
%% Abspann für die Leseansicht.
%% Der Schalter \ifkorrekturansicht ist bereits durch den Vorspann gesetzt.

%% latex-abspann.tex
%% Gemeinsamer Abspann für Korrekturansicht und Leseansicht.
%% Setzt den Schalter \ifkorrekturansicht voraus (gesetzt in den
%% einbindenden Dateien latex-korrekturansicht-abspann.tex bzw.
%% latex-leseansicht-abspann.tex).
%% ---------------------------------------------------------------

\normalsize

% Das esempio-Environment wird nur in der Leseansicht benötigt
\ifkorrekturansicht\else
\newenvironment{esempio}[3]%
{
    \vspace{1.5ex}
    \rlap{\underline{#1}}
    \par
    \setlength{\parindent}{0cm}
    \nopagebreak
    \leftskip=#2cm
    \rightskip=#3cm
}
{
    \par
}
\fi

\doendnotes{C}
\bigskip
\vfill

\clearpage

\footnotesize

\ifkorrekturansicht
  \lohead{\textsc{register}}
\fi

% theindex-Environment neu definieren ohne reledmac
\makeatletter
\renewenvironment{theindex}{%
  \ifkorrekturansicht
    \section*{\indexname}%
  \else
    \subsubsection*{Index der erwähnten Entitäten}%
  \fi
  \setlength{\parindent}{0pt}%
  \setlength{\parskip}{0pt plus 0.3pt}%
  \let\item\@idxitem
}{%
  \ifkorrekturansicht\clearpage\fi
}
\makeatother

\IfFileExists{\jobname-pw.ind}{\input{\jobname-pw.ind}}{}

% Quellenangabe nur in der Leseansicht
\ifkorrekturansicht\else
% Fallback-Definitionen, falls die .tex-Datei \titel etc. nicht gesetzt hat
\providecommand{\titel}{}
\providecommand{\editorInnen}{}
\providecommand{\dateiname}{\jobname}

\vspace{3cm}

\vfill

\footnotesize
\textsc{Quelle}: \titel. Herausgegeben von {\editorInnen}. In: \emph{Arthur Schnitzler: Briefwechsel mit Autorinnen und Autoren}.
 Digitale Edition, https://schnitzler-briefe.acdh.oeaw.ac.at/{\dateiname}.html (Stand \today)
\fi

\end{document}


      