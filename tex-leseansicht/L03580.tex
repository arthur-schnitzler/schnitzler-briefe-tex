%% latex-korrekturansicht-vorspann.tex
%% Vorspann für die Korrekturansicht.
%% Lädt die gemeinsame Datei latex-vorspann.tex mit gesetztem Schalter.

\newif\ifkorrekturansicht
\korrekturansichttrue

\input{../tex-inputs/latex-vorspann}


\section[Felix Salten u. a. an Arthur Schnitzler, {[}November 1927 – Juni 1928?{]}]{L03580 Felix Salten u. a. an Arthur Schnitzler, {[}November 1927 –
               Juni 1928?{]}}
\nopagebreak\mylabel{L03580v}
\rehead{ }\normalsize\beginnumbering\briefempfaengerindex{Schnitzler, Arthur@\textsc{Schnitzler, Arthur}!zzzRehmann, Anna Katharina@\emph{von Anna Katharina Rehmann}!1928-06-301@{{[}November 1927 –
                  Juni 1928?{]}}|(be}\briefempfaengerindex{Schnitzler, Arthur@\textsc{Schnitzler, Arthur}!zzzSalten, Paul@\emph{von Paul Salten}!1928-06-301@{{[}November 1927 –
                  Juni 1928?{]}}|(be}\briefempfaengerindex{Schnitzler, Arthur@\textsc{Schnitzler, Arthur}!zzzSalten, Ottilie@\emph{von Ottilie Salten}!1928-06-301@{{[}November 1927 –
                  Juni 1928?{]}}|(be}\briefempfaengerindex{Schnitzler, Arthur@\textsc{Schnitzler, Arthur}!zzzSalten, Felix@\emph{von Felix Salten}!1928-06-301@{{[}November 1927 –
                  Juni 1928?{]}}|(be}
\toendnotes[C]{\smallbreak\pagebreak[2]}\Standort{CUL, Schnitzler, B 89, B 2.}
\physDesc{Bildpostkarte, 190 Zeichen
\newline{}Handschrift Felix Salten: Bleistift, lateinische Kurrent
\newline{}Handschrift Ottilie Salten: Bleistift, lateinische Kurrent
\newline{}Handschrift Paul Salten: Bleistift
\newline{}Handschrift Anna Katharina Rehmann: Bleistift, lateinische Kurrent
\newline{}Versand: Stempel: »\nobreak{}\oindex{Chemnitz@\textbf{Chemnitz}, \emph{P.PPLA3}|pwk}Chemnitz 4, \textcolor{gray}{1}7–1\textcolor{gray}{8}\nobreak{}«.  }\toendnotes[C]{\smallbreak}\pstart{}{\pb}Herrn D\textsuperscript{r} Arthur Schnitzler\pend{}\pstart{}Wien\oindex{Wien@\textbf{Wien}, \emph{A.ADM2}|pw}\pend{}\pstart{}XVIII. Sternwartestraße 71\oindex{Sternwartestrasse 71@\textbf{Sternwartestraße 71}, \emph{Wohngebäude (K.WHS)}|pw}\pend{}{\bigskip}
\pstart
           \noindent{}\centering{}{\pb}\textcolor{gray}{\textbf{Chemnitz}}\oindex{Chemnitz@\textbf{Chemnitz}, \emph{P.PPLA3}|pw}\pend
           
\pstart
           \centering{}\textcolor{gray}{\textbf{»Die Jägerin\pwindex{Diana mit Bogen@\emph{Diana mit Bogen}|pw}« (Kassberg-Auffahrt\oindex{Kassbergauffahrt@\textbf{Kaßbergauffahrt}, \emph{Straße (K.STR)}|pw})}}\pend
           \vspace{1em}
\pstart
           \noindent{}{\pb}Viele herzlichste Grüße\pend
           \pstart Ihr \spacefill\mbox{Felix Salten}\pend{}\selectlanguage{ngerman}\vspace{1em}
\pstart
           \noindent{}{[}hs. :{]} Wir sind \label{K_L03580-1v}\edtext{heute{ }hier\oindex{Chemnitz@\textbf{Chemnitz}, \emph{P.PPLA3}|pwv}}{\lemma{\textnormal{\emph{heute hier}}}\Cendnote{\textnormal{Die Karte ist undatiert und der
                  Poststempel verwischt, sodass die Datierung nur annäherungsweise mittels Indizien
                  erfolgen kann. Die verwendete Briefmarke wurde erstmals am 1. 11. 1927 ausgegeben, womit der frühestmögliche Zeitpunkt benannt
                  ist. In der Theatersaison 1927/1928 war Anna Katharina Salten\pwindex{Rehmann, Anna Katharina 18.08.1904 – 27.03.1977@\textsc{Rehmann, Anna Katharina} (18.08.1904 – 27.03.1977), \emph{Schauspieler/Schauspielerin, Übersetzer/Übersetzerin}|pwk}
                  am \emph{Städtischen Theater}\orgindex{Staedtische Theater Chemnitz@Städtische Theater Chemnitz|pwk} in Chemnitz\oindex{Chemnitz@\textbf{Chemnitz}, \emph{P.PPLA3}|pwk} engagiert, was der Grund für die Reise der
                  Familie gewesen sein dürfte und zugleich eine Einschränkung der Datierung nach
                  hinten hin ermöglicht.}}}\label{K_L03580-1} sehr froh und denken Ihrer herzlichst {\\}\spacefill\mbox{Otti S.}{\\}{[}hs. :{]} \spacefill\mbox{Paul Salten}\pend
           \selectlanguage{ngerman}\vspace{1em}
\pstart
           \noindent{}{[}hs. :{]} Viele liebe Grüße! {\\}\spacefill\mbox{Annerl}\pend
           \selectlanguage{ngerman}\endnumbering\briefempfaengerindex{Schnitzler, Arthur@\textsc{Schnitzler, Arthur}!zzzRehmann, Anna Katharina@\emph{von Anna Katharina Rehmann}!1927-11-011@{{[}November 1927 –
                  Juni 1928?{]}}|)be}\briefempfaengerindex{Schnitzler, Arthur@\textsc{Schnitzler, Arthur}!zzzSalten, Paul@\emph{von Paul Salten}!1927-11-011@{{[}November 1927 –
                  Juni 1928?{]}}|)be}\briefempfaengerindex{Schnitzler, Arthur@\textsc{Schnitzler, Arthur}!zzzSalten, Ottilie@\emph{von Ottilie Salten}!1927-11-011@{{[}November 1927 –
                  Juni 1928?{]}}|)be}\briefempfaengerindex{Schnitzler, Arthur@\textsc{Schnitzler, Arthur}!zzzSalten, Felix@\emph{von Felix Salten}!1927-11-011@{{[}November 1927 –
                  Juni 1928?{]}}|)be}\mylabel{L03580h}  \normalsize

\doendnotes{C}
\bigskip
\vfill

\clearpage

\footnotesize

\lohead{\textsc{register}}

% Definiere theindex-Environment komplett neu ohne reledmac
\makeatletter
\renewenvironment{theindex}{%
  \section*{\indexname}%
  \setlength{\parindent}{0pt}%
  \setlength{\parskip}{0pt plus 0.3pt}%
  \let\item\@idxitem
}{%
  \clearpage
}
\makeatother

\IfFileExists{\jobname-pw.ind}{\input{\jobname-pw.ind}}{}

\end{document}

      