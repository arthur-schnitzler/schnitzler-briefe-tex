%% latex-leseansicht-vorspann.tex
%% Vorspann für die Leseansicht.
%% Lädt die gemeinsame Datei latex-vorspann.tex mit nicht gesetztem Schalter.

\newif\ifkorrekturansicht
\korrekturansichtfalse

\input{../tex-inputs/latex-vorspann}


               \section[Richard Beer-Hofmann an Arthur Schnitzler, 28. 8. 1929]{ Richard Beer-Hofmann an Arthur Schnitzler, 28. 8. 1929}\nopagebreak\mylabel{v}\rehead{ }\begin{ledgroupsized}[t]{13cm}\normalsize\beginnumbering\briefempfaengerindex{Schnitzler, Arthur@\textsc{Schnitzler, Arthur}!zzzBeer-Hofmann, Richard@\emph{von Richard Beer-Hofmann}!1929-08-281@{28. 8. 1929}|(be} \toendnotes[C]{\smallbreak\pagebreak[2]} \Standort{CUL, Schnitzler, B 8.}
\physDesc{Brief, 3 Blätter, 6 Seiten (paginiert)
\newline{}Handschrift: blauer Buntstift, lateinische Kurrent\newline{}Ordnung: mit Bleistift von unbekannter Hand nummeriert:
                                    »275« }\buchAbdrucke{\weitereDrucke{Arthur Schnitzler, Richard Beer-Hofmann: \emph{Briefwechsel 1891–1931}. Hg. Konstanze Fliedl. Wien, Zürich: \emph{Europaverlag} 1992, S. 231–232.} }\toendnotes[C]{\smallbreak}\pstart
           \raggedleft{}{\pb}Wien\oindex{Wien@\textbf{Wien}|pw}{ }28. VIII. 29.\pend
           \pstart
           Lieber Arthur! Ich \uuline{hoffe} am \label{K_L02521_1v}\edtext{6. VIII.}{\lemma{\textnormal{\emph{6. VIII.}}}\Cendnote{\textnormal{Salten\pwindex{Salten, Felix 06.09.1869 – 08.10.1945@\textsc{Salten, Felix} (06.09.1869 – 08.10.1945), \emph{Schriftsteller, Journalist}|pwk} hatte am 6. 9. 1929 seinen
                  60. Geburtstag.}}}\label{K_L02521_1h} schon in Marienbad\oindex{Marienbad@\textbf{Marienbad}|pw} zu
               sein. Jedenfalls werde ich F. S.\pwindex{Salten, Felix 06.09.1869 – 08.10.1945@\textsc{Salten, Felix} (06.09.1869 – 08.10.1945), \emph{Schriftsteller, Journalist}|pw} telegraphieren –
                  geschrieben\pwindex{Beer-Hofmann, Richard 11.07.1866 – 26.09.1945@\textsc{Beer-Hofmann, Richard} (11.07.1866 – 26.09.1945), \emph{Schriftsteller}!Lieber Felix Salten]1929 – 1929@\strich\emph{[Lieber Felix Salten]} {[}1929 – 1929{]}|pwv} habe ich ja für
                  Zsolnays\orgindex{Paul Zsolnay Verlag@Paul Zsolnay Verlag|pw}{ }Almanach\pwindex{Jahrbuch Paul Zsolnay Verlag@\emph{Jahrbuch Paul Zsolnay Verlag}|pw}. Blumen? – Nein! Irgend eine kleine Gabe?
               – Ich will mich nach Ihnen richten. Eigentlich: Bei einem Andern wäre all das kein
               Problem. Aber {\pb}bei F. S.\pwindex{Salten, Felix 06.09.1869 – 08.10.1945@\textsc{Salten, Felix} (06.09.1869 – 08.10.1945), \emph{Schriftsteller, Journalist}|pw}! Er ist mistrauisch, grundsätzlich leicht verletzt, i{\geminationm}er witternd, man schätze ihn nicht \strikeout{gar} genug, dabei – in seiner Eigenschaft als Kritiker –
               zu leicht der Ansicht zugeneigt, man tue etwas um ihn bei guter Laune zu erhalten –
               sogar \strikeout{gb}\uline{bei}{ }\uline{uns}, glaube ich, vielleicht von Argwohn befallen, und
               sich sagend: {\pb}»Ich habe weder
               Blumen noch sonst was geschickt als B-H. 60. wurde – na – wer weiss, was wäre, wenn
               ich \uline{nicht} Kritiker wäre – –« {\{}aber »beleidigt« wenn man ihm diese
               Argumentation unterschöbe (– schübe? – Gra{\geminationm}atik ist so
                  schwer!).{\}} Schwer mit ihm! Also: Telegra{\geminationm} – keine Blumen – irgendeine Aufmerksamkeit \uline{später}, wenn {\pb}\uline{\label{T_L02521_1v}\edtext{Sie}{\lemma{\textnormal{\emph{Sie}}}\Cendnote{\textnormal{im Original: »sie«}}}\label{T_L02521_1h} der Ansicht sind}.\pend
           \pstart
           \numberlinefalse{}\centering{}–\numberlinetrue{}\pend
           \pstart
           \noindent{}Was das Hôtel unter Ihrem Fenster anlangt – vor 31 Jahren \introOben{}waren
                  Sie\introOben{} mit Hugo\pwindex{Hofmannsthal, Hugo von 01.02.1874 – 15.07.1929@\textsc{Hofmannsthal, Hugo von} (01.02.1874 – 15.07.1929), \emph{Schriftsteller}|pw} dort – »in den nächsten
               31 Jahren \introOben{}wird es\introOben{} wol auch noch unter diesem Fenster \introOben{}sein\introOben{}« – Wäre ich der Hôtelbesitzer würde ich auf diese – Ihre
               – Äusserung hin, \uline{hoch} versichern. Bei Schnitzler
               pflegen solche Hôtels daraufhin {\pb}höhnisch abzubrennen. – \uline{Ich} bin in den Wehen des
                  IV\pwindex{Beer-Hofmann, Richard 11.07.1866 – 26.09.1945@\textsc{Beer-Hofmann, Richard} (11.07.1866 – 26.09.1945), \emph{Schriftsteller}!junge David. Sieben Bilder1933 – 1933@\strich\emph{Der junge David. Sieben Bilder} {[}1933 – 1933{]}|pwv} – dh. jetzt IV\pwindex{Beer-Hofmann, Richard 11.07.1866 – 26.09.1945@\textsc{Beer-Hofmann, Richard} (11.07.1866 – 26.09.1945), \emph{Schriftsteller}!junge David. Sieben Bilder1933 – 1933@\strich\emph{Der junge David. Sieben Bilder} {[}1933 – 1933{]}|pwv} + V. Bild\pwindex{Beer-Hofmann, Richard 11.07.1866 – 26.09.1945@\textsc{Beer-Hofmann, Richard} (11.07.1866 – 26.09.1945), \emph{Schriftsteller}!junge David. Sieben Bilder1933 – 1933@\strich\emph{Der junge David. Sieben Bilder} {[}1933 – 1933{]}|pwv}es – ich wittere, dass \strikeout{sich} aus geheimnisvollen rythmischen Gründen die VII. Bilder auf V. \strikeout{zur} sich zurückbilden werden!\pend
           \pstart
           {\pb}Gutes Wetter! Gute Laune – soviel
               ein besserer Mensch – ohne sich etwas zu vergeben – aufbringen kann, und alles Liebe
               von Paula\pwindex{Beer-Hofmann, Paula 25.02.1879 – 30.10.1939@\textsc{Beer-Hofmann, Paula} (25.02.1879 – 30.10.1939)|pw} und mir! Ihr\pend
           \pstart \spacefill\mbox{Richard}\pend{}\pstart
           Grüsse, und gute Wünsche für Frau P.\pwindex{Pollaczek, Clara Katharina 15.01.1875 – 22.07.1951@\textsc{Pollaczek, Clara Katharina} (15.01.1875 – 22.07.1951), \emph{Schriftstellerin}|pw}\pend
           \pstart
           \label{T_L02521_2v}\edtext{Format dieses Zettels}{\lemma{\textnormal{\emph{Format dieses Zettels}}}\Cendnote{\textnormal{umlaufend zuerst quer am linken Rand, dann
                  unterhalb des Textes, dann quer am linken Rand}}}\label{T_L02521_2h} nicht Geiz – sondern weil
                  \label{K_L02521_2v}\edtext{Ducki}{\lemma{\textnormal{\emph{Ducki}}}\Cendnote{\textnormal{zahme Haustaube}}}\label{K_L02521_2h} den oberen Rand meines letzten
               Brief-Kartels, während ich schrieb – besiegelte.\pend
                     \endnumbering\briefempfaengerindex{Schnitzler, Arthur@\textsc{Schnitzler, Arthur}!zzzBeer-Hofmann, Richard@\emph{von Richard Beer-Hofmann}!1929-08-281@{28. 8. 1929}|)be}\mylabel{h}\end{ledgroupsized}  \newcommand{\dateiname}{L02521}\newcommand{\titel}{Richard Beer-Hofmann an Arthur Schnitzler, 28. 8. 1929}\newcommand{\editorInnen}{Martin Anton Müller und Gerd-Hermann Susen}
            \footnotesize
\begin{ledgroupsized}[t]{11.5cm}
\doendnotes{C}
\end{ledgroupsized}
         %% latex-leseansicht-abspann.tex
%% Abspann für die Leseansicht.
%% Der Schalter \ifkorrekturansicht ist bereits durch den Vorspann gesetzt.

%% latex-abspann.tex
%% Gemeinsamer Abspann für Korrekturansicht und Leseansicht.
%% Setzt den Schalter \ifkorrekturansicht voraus (gesetzt in den
%% einbindenden Dateien latex-korrekturansicht-abspann.tex bzw.
%% latex-leseansicht-abspann.tex).
%% ---------------------------------------------------------------

\normalsize

% Das esempio-Environment wird nur in der Leseansicht benötigt
\ifkorrekturansicht\else
\newenvironment{esempio}[3]%
{
    \vspace{1.5ex}
    \rlap{\underline{#1}}
    \par
    \setlength{\parindent}{0cm}
    \nopagebreak
    \leftskip=#2cm
    \rightskip=#3cm
}
{
    \par
}
\fi

\doendnotes{C}
\bigskip
\vfill

\clearpage

\footnotesize

\ifkorrekturansicht
  \lohead{\textsc{register}}
\fi

% theindex-Environment neu definieren ohne reledmac
\makeatletter
\renewenvironment{theindex}{%
  \ifkorrekturansicht
    \section*{\indexname}%
  \else
    \subsubsection*{Index der erwähnten Entitäten}%
  \fi
  \setlength{\parindent}{0pt}%
  \setlength{\parskip}{0pt plus 0.3pt}%
  \let\item\@idxitem
}{%
  \ifkorrekturansicht\clearpage\fi
}
\makeatother

\IfFileExists{\jobname-pw.ind}{\input{\jobname-pw.ind}}{}

% Quellenangabe nur in der Leseansicht
\ifkorrekturansicht\else
% Fallback-Definitionen, falls die .tex-Datei \titel etc. nicht gesetzt hat
\providecommand{\titel}{}
\providecommand{\editorInnen}{}
\providecommand{\dateiname}{\jobname}

\vspace{3cm}

\vfill

\footnotesize
\textsc{Quelle}: \titel. Herausgegeben von {\editorInnen}. In: \emph{Arthur Schnitzler: Briefwechsel mit Autorinnen und Autoren}.
 Digitale Edition, https://schnitzler-briefe.acdh.oeaw.ac.at/{\dateiname}.html (Stand \today)
\fi

\end{document}


      