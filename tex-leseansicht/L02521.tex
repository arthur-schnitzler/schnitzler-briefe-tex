%% latex-leseansicht-vorspann.tex
%% Vorspann für die Leseansicht.
%% Lädt die gemeinsame Datei latex-vorspann.tex mit nicht gesetztem Schalter.

\newif\ifkorrekturansicht
\korrekturansichtfalse

\input{../tex-inputs/latex-vorspann}


\section[Richard Beer-Hofmann an Arthur Schnitzler, 28. 8. 1929]{L02521 Richard Beer-Hofmann an Arthur Schnitzler, 28. 8. 1929}
\nopagebreak\mylabel{L02521v}
\rehead{ }\normalsize\beginnumbering\briefempfaengerindex{Schnitzler, Arthur@\textsc{Schnitzler, Arthur}!zzzBeer-Hofmann, Richard@\emph{von Richard Beer-Hofmann}!1929-08-281@{28. 8. 1929}|(be}
\toendnotes[C]{\smallbreak\pagebreak[2]}
\correspDesc{Versand  durch Richard Beer-Hofmann am 28. 8. 1929 in Wien
\newline{}Erhalt  durch Arthur Schnitzler im Zeitraum [28. 8. 1929
                  – 1. 9. 1929?] \textbf{Ort fehlend} }\toendnotes[C]{\smallbreak}
\Standort{CUL, Schnitzler, B 8.}
\physDesc{Brief, 3 Blätter, 6 Seiten, 1730 Zeichen (paginiert)
\newline{}Handschrift: blauer Buntstift, lateinische Kurrent
\newline{}Ordnung: mit Bleistift von unbekannter Hand nummeriert:
                                    »275« }
\buchAbdrucke{\weitereDrucke{Arthur Schnitzler, Richard Beer-Hofmann: \emph{Briefwechsel 1891–1931}. Herausgegeben von Konstanze Fliedl. Wien, Zürich: \emph{Europaverlag} 1992, S. 231–232.} }\toendnotes[C]{\smallbreak}
\pstart
           \raggedleft{}{\pb}Wien\oindex{Wien@\textbf{Wien}, \emph{Verwaltungsgebiet}|pw}{ }28. VIII. 29.\pend
           \vspace{0.5em}
\pstart
           Lieber Arthur! Ich \uuline{hoffe} am \label{K_L02521-1v}\edtext{6. VIII.}{\lemma{\textnormal{\emph{6. VIII.}}}\Cendnote{\textnormal{Salten\pwindex{Salten, Felix 6.\,9.\,1869 Budapest – 8.\,10.\,1945 Zürich@\textsc{Salten, Felix} (6.\,9.\,1869 Budapest – 8.\,10.\,1945 Zürich), \emph{Schriftsteller, Journalist, Chefredakteur}|pwk} hatte am 6. 9. 1929
                  seinen 60. Geburtstag.}}}\label{K_L02521-1} schon in Marienbad\oindex{Marienbad@\textbf{Marienbad}|pw} zu sein. Jedenfalls werde ich F. S.\pwindex{Salten, Felix 6.\,9.\,1869 Budapest – 8.\,10.\,1945 Zürich@\textsc{Salten, Felix} (6.\,9.\,1869 Budapest – 8.\,10.\,1945 Zürich), \emph{Schriftsteller, Journalist, Chefredakteur}|pw} telegraphieren – geschrieben\pwindex{Beer-Hofmann, Richard 11.\,7.\,1866 Wien – 26.\,9.\,1945 New York City@\textsc{Beer-Hofmann, Richard} (11.\,7.\,1866 Wien – 26.\,9.\,1945 New York City), \emph{Schriftsteller}!Lieber Felix Salten]@\strich\emph{[Lieber Felix Salten]}|pwv} habe ich ja für \label{K_L02521-2v}\edtext{Zsolnays\orgindex{Paul Zsolnay Verlag@Paul Zsolnay Verlag|pw}{ }Almanach\pwindex{Jahrbuch Paul Zsolnay Verlag@\emph{Jahrbuch Paul Zsolnay Verlag}|pw}}{\lemma{\textnormal{\emph{Zsolnays Almanach}}}\Cendnote{\textnormal{Vgl. XXXX Auszeichnungsfehler: Dokument L02950 nicht gefunden.
               }}}\label{K_L02521-2}. Blumen? – Nein! Irgend eine kleine
               Gabe? – Ich will mich nach Ihnen richten. Eigentlich: Bei einem Andern wäre all das
               kein Problem. Aber {\pb}bei F. S.\pwindex{Salten, Felix 6.\,9.\,1869 Budapest – 8.\,10.\,1945 Zürich@\textsc{Salten, Felix} (6.\,9.\,1869 Budapest – 8.\,10.\,1945 Zürich), \emph{Schriftsteller, Journalist, Chefredakteur}|pw}! Er ist mistrauisch, grundsätzlich leicht
               verletzt, i{\geminationm}er witternd, man schätze ihn nicht \strikeout{gar} genug, dabei – in seiner Eigenschaft als Kritiker –
               zu leicht der Ansicht zugeneigt, man tue etwas um ihn bei guter Laune zu erhalten –
               sogar \strikeout{ge}{ }\uline{bei}{ }\uline{uns}, glaube ich, vielleicht von Argwohn befallen, und
               sich sagend: {\pb}»Ich habe weder
               Blumen noch sonst was geschickt als B-H. 60. wurde – na – wer weiss, was wäre, wenn
               ich \uline{nicht} Kritiker wäre – –« {\{}aber »beleidigt« wenn man ihm diese
               Argumentation unterschöbe (– schübe? – Gra{\geminationm}atik ist so
                  schwer!).{\}} Schwer mit ihm! Also: Telegra{\geminationm} – keine Blumen – irgendeine Aufmerksamkeit \uline{später}, wenn {\pb}\uline{\label{T_L02521-1v}\edtext{Sie}{\lemma{\textnormal{\emph{Sie}}}\Cendnote{\textnormal{im Original: »sie«}}}\label{T_L02521-1} der Ansicht sind}.\pend
           
\pstart
           \numberlinefalse{}\centering{}–\numberlinetrue{}\pend
           
\pstart
           Was das Hôtel unter Ihrem Fenster anlangt – vor 31 Jahren \introOben{}waren
                  Sie\introOben{} mit Hugo\pwindex{Hofmannsthal, Hugo von 1.\,2.\,1874 Wien – 15.\,7.\,1929 Rodaun@\textsc{Hofmannsthal, Hugo von} (1.\,2.\,1874 Wien – 15.\,7.\,1929 Rodaun), \emph{Schriftsteller}|pw} dort – »in den nächsten
               31 Jahren \introOben{}wird es\introOben{} wol auch noch unter diesem Fenster \introOben{}sein\introOben{}« – Wäre ich der Hôtelbesitzer würde ich auf diese – Ihre
               – Äusserung hin, \uline{hoch} versichern. Bei Schnitzler
               pflegen solche Hôtels daraufhin {\pb}höhnisch abzubrennen. – \uline{Ich} bin in den Wehen des
                  IV\pwindex{Beer-Hofmann, Richard 11.\,7.\,1866 Wien – 26.\,9.\,1945 New York City@\textsc{Beer-Hofmann, Richard} (11.\,7.\,1866 Wien – 26.\,9.\,1945 New York City), \emph{Schriftsteller}!junge David. Sieben Bilder@\strich\emph{Der junge David. Sieben Bilder}|pwv} – dh. jetzt IV\pwindex{Beer-Hofmann, Richard 11.\,7.\,1866 Wien – 26.\,9.\,1945 New York City@\textsc{Beer-Hofmann, Richard} (11.\,7.\,1866 Wien – 26.\,9.\,1945 New York City), \emph{Schriftsteller}!junge David. Sieben Bilder@\strich\emph{Der junge David. Sieben Bilder}|pwv} + V. Bild\pwindex{Beer-Hofmann, Richard 11.\,7.\,1866 Wien – 26.\,9.\,1945 New York City@\textsc{Beer-Hofmann, Richard} (11.\,7.\,1866 Wien – 26.\,9.\,1945 New York City), \emph{Schriftsteller}!junge David. Sieben Bilder@\strich\emph{Der junge David. Sieben Bilder}|pwv}es – ich wittere, dass \strikeout{sich} aus geheimnisvollen rythmischen Gründen die
               VII. Bilder auf V. \strikeout{zur} sich zurückbilden werden!\pend
           
\pstart
           {\pb}Gutes Wetter! Gute Laune – soviel
               ein besserer Mensch – ohne sich etwas zu vergeben – aufbringen kann, und alles Liebe
               von Paula\pwindex{Beer-Hofmann, Paula 25.\,2.\,1879 Wien – 30.\,10.\,1939 Zürich@\textsc{Beer-Hofmann, Paula} (25.\,2.\,1879 Wien – 30.\,10.\,1939 Zürich)|pw} und mir! Ihr\pend
           \pstart \spacefill\mbox{Richard}\pend{}
\pstart
           Grüsse, und gute Wünsche für Frau P.\pwindex{Pollaczek, Clara Katharina 15.\,1.\,1875 Wien – 22.\,7.\,1951 ebd.@\textsc{Pollaczek, Clara Katharina} (15.\,1.\,1875 Wien – 22.\,7.\,1951 ebd.), \emph{Schriftstellerin}|pw}\pend
           
\pstart
           \label{T_L02521-2v}\edtext{Format dieses Zettels}{\lemma{\textnormal{\emph{Format dieses Zettels}}}\Cendnote{\textnormal{umlaufend zuerst quer am linken Rand, dann
                  unterhalb des Textes, dann quer am linken Rand}}}\label{T_L02521-2} nicht Geiz – sondern weil
                  \label{K_L02521-3v}\edtext{Ducki}{\lemma{\textnormal{\emph{Ducki}}}\Cendnote{\textnormal{zahme Haustaube}}}\label{K_L02521-3} den oberen Rand meines letzten
               Brief-Kartels, während ich schrieb – besiegelte.\pend
           \selectlanguage{ngerman}\endnumbering\briefempfaengerindex{Schnitzler, Arthur@\textsc{Schnitzler, Arthur}!zzzBeer-Hofmann, Richard@\emph{von Richard Beer-Hofmann}!1929-08-281@{28. 8. 1929}|)be}\mylabel{L02521h}  \newcommand{\dateiname}{L02521}\newcommand{\titel}{Richard Beer-Hofmann an Arthur Schnitzler, 28. 8. 1929}\newcommand{\editorInnen}{Martin Anton Müller und Gerd-Hermann Susen}%% latex-leseansicht-abspann.tex
%% Abspann für die Leseansicht.
%% Der Schalter \ifkorrekturansicht ist bereits durch den Vorspann gesetzt.

%% latex-abspann.tex
%% Gemeinsamer Abspann für Korrekturansicht und Leseansicht.
%% Setzt den Schalter \ifkorrekturansicht voraus (gesetzt in den
%% einbindenden Dateien latex-korrekturansicht-abspann.tex bzw.
%% latex-leseansicht-abspann.tex).
%% ---------------------------------------------------------------

\normalsize

% Das esempio-Environment wird nur in der Leseansicht benötigt
\ifkorrekturansicht\else
\newenvironment{esempio}[3]%
{
    \vspace{1.5ex}
    \rlap{\underline{#1}}
    \par
    \setlength{\parindent}{0cm}
    \nopagebreak
    \leftskip=#2cm
    \rightskip=#3cm
}
{
    \par
}
\fi

\doendnotes{C}
\bigskip
\vfill

\clearpage

\footnotesize

\ifkorrekturansicht
  \lohead{\textsc{register}}
\fi

% theindex-Environment neu definieren ohne reledmac
\makeatletter
\renewenvironment{theindex}{%
  \ifkorrekturansicht
    \section*{\indexname}%
  \else
    \subsubsection*{Index der erwähnten Entitäten}%
  \fi
  \setlength{\parindent}{0pt}%
  \setlength{\parskip}{0pt plus 0.3pt}%
  \let\item\@idxitem
}{%
  \ifkorrekturansicht\clearpage\fi
}
\makeatother

\IfFileExists{\jobname-pw.ind}{\input{\jobname-pw.ind}}{}

% Quellenangabe nur in der Leseansicht
\ifkorrekturansicht\else
% Fallback-Definitionen, falls die .tex-Datei \titel etc. nicht gesetzt hat
\providecommand{\titel}{}
\providecommand{\editorInnen}{}
\providecommand{\dateiname}{\jobname}

\vspace{3cm}

\vfill

\footnotesize
\textsc{Quelle}: \titel. Herausgegeben von {\editorInnen}. In: \emph{Arthur Schnitzler: Briefwechsel mit Autorinnen und Autoren}.
 Digitale Edition, https://schnitzler-briefe.acdh.oeaw.ac.at/{\dateiname}.html (Stand \today)
\fi

\end{document}


