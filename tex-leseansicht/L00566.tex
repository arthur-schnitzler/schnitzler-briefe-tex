%% latex-leseansicht-vorspann.tex
%% Vorspann für die Leseansicht.
%% Lädt die gemeinsame Datei latex-vorspann.tex mit nicht gesetztem Schalter.

\newif\ifkorrekturansicht
\korrekturansichtfalse

\input{../tex-inputs/latex-vorspann}


\section[Richard Beer-Hofmann an Arthur Schnitzler, 22. 7. 1896]{L00566 Richard Beer-Hofmann an Arthur Schnitzler, 22. 7. 1896}
\nopagebreak\mylabel{L00566v}
\rehead{ }\normalsize\beginnumbering\briefempfaengerindex{Schnitzler, Arthur@\textsc{Schnitzler, Arthur}!zzzBeer-Hofmann, Richard@\emph{von Richard Beer-Hofmann}!1896-07-221@{22. 7. 1896}|(be}
\toendnotes[C]{\smallbreak\pagebreak[2]}
\correspDesc{Versand  durch Richard Beer-Hofmann am 22. 7. 1896 in Salzburg
\newline{}Erhalt  durch Arthur Schnitzler am 22. 7. 1896 in Trondheim}\toendnotes[C]{\smallbreak}
\Standort{CUL, Schnitzler, B 8.}
\physDesc{Telegramm, 166 Zeichen
\newline{}HandschriftX2 einer Schreibkraft: Bleistift, lateinische Kurrent
\newline{}Versand: »\noindent{}\textcolor{gray}{\textbf{\textbf{Optaget} fra}} 39 \textcolor{gray}{\textbf{den}}{ }21\textcolor{gray}{\textbf{/}}7{ }4,45 \textcolor{gray}{\textbf{midd. af}}{ }\textcolor{gray}{M}« 
\newline{}Ordnung: mit Bleistift von unbekannter Hand nummeriert:
                                    »76« }
\buchAbdrucke{\weitereDrucke{Arthur Schnitzler, Richard Beer-Hofmann: \emph{Briefwechsel 1891–1931}. Herausgegeben von Konstanze Fliedl. Wien, Zürich: \emph{Europaverlag} 1992, S. 93.} }\pstart{}{\pb}Doktor Arthur Schnitzler
                  +\pend{}\pstart{}poste restante Thiem\oindex{Trondheim@\textbf{Trondheim}, \emph{Hauptstadt}|pw}\pend{}{\bigskip}\vspace{1em}
\pstart
           {\pb}\textcolor{gray}{\textbf{Telegram fra}}{ }Salzburg\oindex{Salzburg@\textbf{Salzburg}, \emph{Verwaltungsgebiet}|pw}{ }\textcolor{gray}{\textbf{No.}} 501\textcolor{gray}{\textbf{, Ord}} 20\textcolor{gray}{\textbf{, den}}{ }22\textcolor{gray}{\textbf{/}}7{ }\textcolor{gray}{\textbf{189}}6{ }\textcolor{gray}{\textbf{Kl.}} 11,10\textcolor{gray}{\textbf{midd.}}\pend
           \vspace{0.5em}
\pstart
           Reise heute Salzburg\oindex{Salzburg@\textbf{Salzburg}, \emph{Verwaltungsgebiet}|pw} ab, über München\oindex{München@\textbf{München}|pw}{ }Berlin\oindex{Berlin@\textbf{Berlin}, \emph{Hauptstadt}|pw} bin 25{ }Kopenhagen\oindex{Kopenhagen@\textbf{Kopenhagen}, \emph{Hauptstadt}|pw} erwarte Nachricht herzlichst\pend
           \pstart \spacefill\mbox{Richard}\pend{}\selectlanguage{ngerman}\endnumbering\briefempfaengerindex{Schnitzler, Arthur@\textsc{Schnitzler, Arthur}!zzzBeer-Hofmann, Richard@\emph{von Richard Beer-Hofmann}!1896-07-221@{22. 7. 1896}|)be}\mylabel{L00566h}  \newcommand{\dateiname}{L00566}\newcommand{\titel}{Richard Beer-Hofmann an Arthur Schnitzler, 22. 7. 1896}\newcommand{\editorInnen}{Martin Anton Müller und Gerd-Hermann Susen}%% latex-leseansicht-abspann.tex
%% Abspann für die Leseansicht.
%% Der Schalter \ifkorrekturansicht ist bereits durch den Vorspann gesetzt.

%% latex-abspann.tex
%% Gemeinsamer Abspann für Korrekturansicht und Leseansicht.
%% Setzt den Schalter \ifkorrekturansicht voraus (gesetzt in den
%% einbindenden Dateien latex-korrekturansicht-abspann.tex bzw.
%% latex-leseansicht-abspann.tex).
%% ---------------------------------------------------------------

\normalsize

% Das esempio-Environment wird nur in der Leseansicht benötigt
\ifkorrekturansicht\else
\newenvironment{esempio}[3]%
{
    \vspace{1.5ex}
    \rlap{\underline{#1}}
    \par
    \setlength{\parindent}{0cm}
    \nopagebreak
    \leftskip=#2cm
    \rightskip=#3cm
}
{
    \par
}
\fi

\doendnotes{C}
\bigskip
\vfill

\clearpage

\footnotesize

\ifkorrekturansicht
  \lohead{\textsc{register}}
\fi

% theindex-Environment neu definieren ohne reledmac
\makeatletter
\renewenvironment{theindex}{%
  \ifkorrekturansicht
    \section*{\indexname}%
  \else
    \subsubsection*{Index der erwähnten Entitäten}%
  \fi
  \setlength{\parindent}{0pt}%
  \setlength{\parskip}{0pt plus 0.3pt}%
  \let\item\@idxitem
}{%
  \ifkorrekturansicht\clearpage\fi
}
\makeatother

\IfFileExists{\jobname-pw.ind}{\input{\jobname-pw.ind}}{}

% Quellenangabe nur in der Leseansicht
\ifkorrekturansicht\else
% Fallback-Definitionen, falls die .tex-Datei \titel etc. nicht gesetzt hat
\providecommand{\titel}{}
\providecommand{\editorInnen}{}
\providecommand{\dateiname}{\jobname}

\vspace{3cm}

\vfill

\footnotesize
\textsc{Quelle}: \titel. Herausgegeben von {\editorInnen}. In: \emph{Arthur Schnitzler: Briefwechsel mit Autorinnen und Autoren}.
 Digitale Edition, https://schnitzler-briefe.acdh.oeaw.ac.at/{\dateiname}.html (Stand \today)
\fi

\end{document}


