%% latex-leseansicht-vorspann.tex
%% Vorspann für die Leseansicht.
%% Lädt die gemeinsame Datei latex-vorspann.tex mit nicht gesetztem Schalter.

\newif\ifkorrekturansicht
\korrekturansichtfalse

\input{../tex-inputs/latex-vorspann}


               \section[Jakob Julius David an Arthur Schnitzler, {[}5. 10. 1895{]}]{ Jakob Julius David an Arthur Schnitzler,
                    {[}5. 10. 1895{]}}\nopagebreak\mylabel{v}\rehead{ }\begin{ledgroupsized}[t]{13cm}\normalsize\beginnumbering\briefempfaengerindex{Schnitzler, Arthur@\textsc{Schnitzler, Arthur}!zzzDavid, Jakob Julius@\emph{von Jakob Julius David}!1895-10-051@{{[}5. 10. 1895{]}}|(be} \toendnotes[C]{\smallbreak\pagebreak[2]} \Standort{CUL, Schnitzler, B 25.}
\physDesc{Briefkarte
\newline{}Handschrift: schwarze Tinte, lateinische Kurrent
\newline{}Schnitzler: mit Bleistift datiert: »5/10 95« \newline{}Ordnung: mit Bleistift von unbekannter Hand nummeriert:
                                        »2.«, von anderer Hand:
                                    »3« }\toendnotes[C]{\smallbreak}\pstart
           \noindent{}{\pb}\textcolor{gray}{\textbf{Neues Wiener
                                    Journal\orgindex{Neues Wiener Journal@Neues Wiener Journal|pw}}}\hfill \textcolor{gray}{\textbf{\textbf{Wien, IX.\oindex{IX., Alsergrund@\textbf{IX., Alsergrund}|pw},}
                                den ..........}}\pend
           \pstart
           \textcolor{gray}{\textbf{Herausgeber und
                            Chefradacteur:}}\hfill \textcolor{gray}{\textbf{Nußdorferſtraße 3\oindex{Nussdorfer Strasse@\textbf{Nussdorfer Straße}|pw}.}}\pend
           \pstart
           \textcolor{gray}{\textbf{\textbf{J. Lippowitz}\pwindex{Lippowitz, Jakob 09.10.1865 – 04.07.1934@\textsc{Lippowitz, Jakob} (09.10.1865 – 04.07.1934), \emph{Schriftsteller, Journalist}|pw}}}\hfill \textcolor{gray}{\textbf{Telegramm-Adreſſe: Neujournal,
                                Wien\oindex{Wien@\textbf{Wien}|pw}.}}\pend
           \pstart
           \textcolor{gray}{\textbf{Telephon Nr. \textbf{7920}.}}\pend
           \pstart\center{}Werther und verehrter Freund!\pend\pstart
           An Ihrem \label{K_L00500_1v}\edtext{Premièren\pwindex{Schnitzler, Arthur 15.05.1862 – 21.10.1931@\textsc{Schnitzler, Arthur} (15.05.1862 – 21.10.1931), \emph{Schriftsteller, Mediziner}!Liebelei. Schauspiel in drei Akten9. 10. 1895@\strich\emph{Liebelei. Schauspiel in drei Akten} {[}9. 10. 1895{]}|pwv}tage}{\lemma{\textnormal{\emph{Premièrentage}}}\Cendnote{\textnormal{am 9. 10. 1895}}}\label{K_L00500_1h}
                    veröffentliche ich selbst eine Studie\pwindex{Arthur Schnitzler9.10.1895 – 9.10.1895@\emph{Arthur Schnitzler} {[}9.10.1895 – 9.10.1895{]}|pwv} über Sie bei uns. Ist es \label{K_L00500_2v}\edtext{ganz unmöglich}{\lemma{\textnormal{\emph{ganz unmöglich}}}\Cendnote{\textnormal{Offensichtlich, jedenfalls erschien
                        nichts von Schnitzler\pwindex{Schnitzler, Arthur 15.05.1862 – 21.10.1931@\textsc{Schnitzler, Arthur} (15.05.1862 – 21.10.1931), \emph{Schriftsteller, Mediziner}|pwk} im Vorspann des
                        Texts.}}}\label{K_L00500_2h}, daß Sie mir, sagen wir 100 Zeilen geben,
                    autobiographisch. Sti{\geminationm}ung oder was Sie wollen, die
                    ich voranstellen könnte? Ich werde {\pb}es Ihnen immer danken und es als einen \uline{mir
                        persönlich erwiesenen Dienst} betrachten.\pend
           \pstart
           Waidmannsheil!{\\[\baselineskip]}Herzlichst Ihr{\\[\baselineskip]}\spacefill\mbox{David}\pend
           \leftskip=0em{}\endnumbering\briefempfaengerindex{Schnitzler, Arthur@\textsc{Schnitzler, Arthur}!zzzDavid, Jakob Julius@\emph{von Jakob Julius David}!1895-10-051@{{[}5. 10. 1895{]}}|)be}\mylabel{h}\end{ledgroupsized}  \newcommand{\dateiname}{L00500}\newcommand{\titel}{Jakob Julius David an Arthur Schnitzler, [5. 10. 1895]}\newcommand{\editorInnen}{Martin Anton Müller und Gerd-Hermann Susen}%% latex-leseansicht-abspann.tex
%% Abspann für die Leseansicht.
%% Der Schalter \ifkorrekturansicht ist bereits durch den Vorspann gesetzt.

%% latex-abspann.tex
%% Gemeinsamer Abspann für Korrekturansicht und Leseansicht.
%% Setzt den Schalter \ifkorrekturansicht voraus (gesetzt in den
%% einbindenden Dateien latex-korrekturansicht-abspann.tex bzw.
%% latex-leseansicht-abspann.tex).
%% ---------------------------------------------------------------

\normalsize

% Das esempio-Environment wird nur in der Leseansicht benötigt
\ifkorrekturansicht\else
\newenvironment{esempio}[3]%
{
    \vspace{1.5ex}
    \rlap{\underline{#1}}
    \par
    \setlength{\parindent}{0cm}
    \nopagebreak
    \leftskip=#2cm
    \rightskip=#3cm
}
{
    \par
}
\fi

\doendnotes{C}
\bigskip
\vfill

\clearpage

\footnotesize

\ifkorrekturansicht
  \lohead{\textsc{register}}
\fi

% theindex-Environment neu definieren ohne reledmac
\makeatletter
\renewenvironment{theindex}{%
  \ifkorrekturansicht
    \section*{\indexname}%
  \else
    \subsubsection*{Index der erwähnten Entitäten}%
  \fi
  \setlength{\parindent}{0pt}%
  \setlength{\parskip}{0pt plus 0.3pt}%
  \let\item\@idxitem
}{%
  \ifkorrekturansicht\clearpage\fi
}
\makeatother

\IfFileExists{\jobname-pw.ind}{\input{\jobname-pw.ind}}{}

% Quellenangabe nur in der Leseansicht
\ifkorrekturansicht\else
% Fallback-Definitionen, falls die .tex-Datei \titel etc. nicht gesetzt hat
\providecommand{\titel}{}
\providecommand{\editorInnen}{}
\providecommand{\dateiname}{\jobname}

\vspace{3cm}

\vfill

\footnotesize
\textsc{Quelle}: \titel. Herausgegeben von {\editorInnen}. In: \emph{Arthur Schnitzler: Briefwechsel mit Autorinnen und Autoren}.
 Digitale Edition, https://schnitzler-briefe.acdh.oeaw.ac.at/{\dateiname}.html (Stand \today)
\fi

\end{document}


      