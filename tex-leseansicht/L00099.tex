%% latex-korrekturansicht-vorspann.tex
%% Vorspann für die Korrekturansicht.
%% Lädt die gemeinsame Datei latex-vorspann.tex mit gesetztem Schalter.

\newif\ifkorrekturansicht
\korrekturansichttrue

\input{../tex-inputs/latex-vorspann}


\section[Arthur Schnitzler an Richard Beer-Hofmann, 3. 5. 1892]{L00099 Arthur Schnitzler an Richard Beer-Hofmann, 3. 5. 1892}
\nopagebreak\mylabel{L00099v}
\rehead{ }\normalsize\beginnumbering\briefempfaengerindex{Beer-Hofmann, Richard@\textsc{Beer-Hofmann, Richard}!zzzSchnitzler, Arthur@\emph{von Arthur Schnitzler}!1892-05-031@{3. 5. 1892}|(be}
\toendnotes[C]{\smallbreak\pagebreak[2]}\Standort{YCGL, MSS 31.}
\physDesc{Briefkarte, , Umschlag, 261 Zeichen
\newline{}Handschrift: Bleistift, deutsche Kurrent
\newline{}Versand: ohne postalischen Übermittlungsvermerk }
\buchAbdrucke{\weitereDrucke{Arthur Schnitzler, Richard Beer-Hofmann: \emph{Briefwechsel 1891–1931}. Wien, Zürich: \emph{Europaverlag} 1992, S. 36.} }\toendnotes[C]{\smallbreak}\pstart{}{\pb}Herrn \textsc{Dr Rich Beer-}\damage{\textcolor{gray}{H}ofmann}\pend{}{\bigskip}\vspace{1em}
\pstart
           \noindent{}{\pb}We{\geminationn} ich Ihnen
               wiederhole, lieber Richard, daß ich Ihre entzückende Pantomime\pwindex{Pierrot Hypnotiseur@\emph{Pierrot Hypnotiseur}|pwv} ungeheuer gern ſehen möchte, ſo
               will ich damit {\pb}nicht ſagen, daſs ich ſie nicht mit
               großem Vergnügen noch ein halbes Dutzend Mal leſen werde.\pend
           
\pstart
           Herzlichſt Ihr{\\[\baselineskip]}\spacefill\mbox{Arth.}\pend
           \leftskip=0em{}
\pstart
           3/5 92\pend
           \selectlanguage{ngerman}\endnumbering\briefempfaengerindex{Beer-Hofmann, Richard@\textsc{Beer-Hofmann, Richard}!zzzSchnitzler, Arthur@\emph{von Arthur Schnitzler}!1892-05-031@{3. 5. 1892}|)be}\mylabel{L00099h}  \normalsize

\doendnotes{C}
\bigskip
\vfill

\clearpage

\footnotesize

\lohead{\textsc{register}}

% Definiere theindex-Environment komplett neu ohne reledmac
\makeatletter
\renewenvironment{theindex}{%
  \section*{\indexname}%
  \setlength{\parindent}{0pt}%
  \setlength{\parskip}{0pt plus 0.3pt}%
  \let\item\@idxitem
}{%
  \clearpage
}
\makeatother

\IfFileExists{\jobname-pw.ind}{\input{\jobname-pw.ind}}{}

\end{document}

      