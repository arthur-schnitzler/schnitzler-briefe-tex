%% latex-leseansicht-vorspann.tex
%% Vorspann für die Leseansicht.
%% Lädt die gemeinsame Datei latex-vorspann.tex mit nicht gesetztem Schalter.

\newif\ifkorrekturansicht
\korrekturansichtfalse

\input{../tex-inputs/latex-vorspann}


\section[ Felix Salten an Arthur Schnitzler, {[}19?. 11. 1903{]}]{L03351 Felix Salten an Arthur Schnitzler,  [19?. 11. 1903]}
\nopagebreak\mylabel{L03351v}
\rehead{ }\normalsize\beginnumbering\briefempfaengerindex{Schnitzler, Arthur@\textsc{Schnitzler, Arthur}!zzzSalten, Felix@\emph{von Felix Salten}!1903-11-191@{{[}19?. 11. 1903{]}}|(be}
\toendnotes[C]{\smallbreak\pagebreak[2]}
\correspDesc{Versand  durch Felix Salten am [19?. 11. 1903] in Wien
\newline{}Erhalt  durch Arthur Schnitzler am [19. 11. 1903?] in Wien}\toendnotes[C]{\smallbreak}
\Standort{CUL, Schnitzler, B 89, A 2.}
\physDesc{Brief, 1 Blatt, 1 Seite, 284 Zeichen
\newline{}Handschrift: schwarze Tinte, lateinische Kurrent
\newline{}Schnitzler: mit Bleistift datiert: »18/11 903« 
\newline{}Ordnung: mit Bleistift von unbekannter Hand nummeriert: »177« }\toendnotes[C]{\smallbreak}
\pstart
           \raggedleft{}{\pb}\label{K_L03351-1v}\edtext{Donnerstag}{\lemma{\textnormal{\emph{Donnerstag}}}\Cendnote{\textnormal{Diese Wochentagsangabe und die
                        Datierung Schnitzlers, die einen
                        Mittwoch nennt, widersprechen sich. Die Angabe der Uhrzeit nach
                           Mitternacht stützt die Annahme, dass das Schreiben in der Nacht von Mittwoch auf Donnerstag um 1 Uhr früh – also
                        tatsächlich an einem Donnerstag – verfasst wurde.}}}\label{K_L03351-1}.\pend
           
\pstart
           \raggedleft{}1\textsuperscript{h} früh.\pend
           \vspace{0.5em}
\pstart
           Lieber Freund, wenn Sie Beide\pwindex{Schnitzler, Olga 17.\,1.\,1882 Wien – 13.\,1.\,1970 Lugano@\textsc{Schnitzler, Olga} (17.\,1.\,1882 Wien – 13.\,1.\,1970 Lugano), \emph{Schauspielerin, Sängerin}|pwv}{ }heute{ }Abend mit Safonoff\pwindex{Safonov, Vasilij Ilʹič 6.\,2.\,1852 Ishcherskaya – 27.\,2.\,1918 Kislovodsk@\textsc{Safonov, Vasilij Ilʹič} (6.\,2.\,1852 Ishcherskaya – 27.\,2.\,1918 Kislovodsk), \emph{Dirigent, Pianist, Musiker}|pw} bei uns\pwindex{Salten, Ottilie 7.\,3.\,1868 Prag – 22.\,6.\,1942 Zürich@\textsc{Salten, Ottilie} (7.\,3.\,1868 Prag – 22.\,6.\,1942 Zürich), \emph{Schauspielerin}|pwv} essen wollten (8\textsuperscript{h.}) würden wir uns herzlich darüber freuen. Safonoff\pwindex{Safonov, Vasilij Ilʹič 6.\,2.\,1852 Ishcherskaya – 27.\,2.\,1918 Kislovodsk@\textsc{Safonov, Vasilij Ilʹič} (6.\,2.\,1852 Ishcherskaya – 27.\,2.\,1918 Kislovodsk), \emph{Dirigent, Pianist, Musiker}|pw} ist eben angekommen, deshalb bitte ich wegen der knappen Frist um
               Entschuldigung. Sie pneumatisiren mir hoffentlich Ihre \label{K_L03351-2v}\edtext{Zusage}{\lemma{\textnormal{\emph{Zusage}}}\Cendnote{\textnormal{Schnitzler war anderweitig verpflichtet,
                     vgl. A. S.: \emph{Tagebuch}, 19. 11. 1903.}}}\label{K_L03351-2}\textcolor{gray}{.}\pend
           \pstart herzlichst Ihr \spacefill\mbox{S.}\pend{}\selectlanguage{ngerman}\endnumbering\briefempfaengerindex{Schnitzler, Arthur@\textsc{Schnitzler, Arthur}!zzzSalten, Felix@\emph{von Felix Salten}!1903-11-191@{{[}19?. 11. 1903{]}}|)be}\mylabel{L03351h}  \newcommand{\dateiname}{L03351}\newcommand{\titel}{Felix Salten an Arthur Schnitzler, [19?. 11. 1903]}\newcommand{\editorInnen}{Martin Anton Müller und Laura Untner}%% latex-leseansicht-abspann.tex
%% Abspann für die Leseansicht.
%% Der Schalter \ifkorrekturansicht ist bereits durch den Vorspann gesetzt.

%% latex-abspann.tex
%% Gemeinsamer Abspann für Korrekturansicht und Leseansicht.
%% Setzt den Schalter \ifkorrekturansicht voraus (gesetzt in den
%% einbindenden Dateien latex-korrekturansicht-abspann.tex bzw.
%% latex-leseansicht-abspann.tex).
%% ---------------------------------------------------------------

\normalsize

% Das esempio-Environment wird nur in der Leseansicht benötigt
\ifkorrekturansicht\else
\newenvironment{esempio}[3]%
{
    \vspace{1.5ex}
    \rlap{\underline{#1}}
    \par
    \setlength{\parindent}{0cm}
    \nopagebreak
    \leftskip=#2cm
    \rightskip=#3cm
}
{
    \par
}
\fi

\doendnotes{C}
\bigskip
\vfill

\clearpage

\footnotesize

\ifkorrekturansicht
  \lohead{\textsc{register}}
\fi

% theindex-Environment neu definieren ohne reledmac
\makeatletter
\renewenvironment{theindex}{%
  \ifkorrekturansicht
    \section*{\indexname}%
  \else
    \subsubsection*{Index der erwähnten Entitäten}%
  \fi
  \setlength{\parindent}{0pt}%
  \setlength{\parskip}{0pt plus 0.3pt}%
  \let\item\@idxitem
}{%
  \ifkorrekturansicht\clearpage\fi
}
\makeatother

\IfFileExists{\jobname-pw.ind}{\input{\jobname-pw.ind}}{}

% Quellenangabe nur in der Leseansicht
\ifkorrekturansicht\else
% Fallback-Definitionen, falls die .tex-Datei \titel etc. nicht gesetzt hat
\providecommand{\titel}{}
\providecommand{\editorInnen}{}
\providecommand{\dateiname}{\jobname}

\vspace{3cm}

\vfill

\footnotesize
\textsc{Quelle}: \titel. Herausgegeben von {\editorInnen}. In: \emph{Arthur Schnitzler: Briefwechsel mit Autorinnen und Autoren}.
 Digitale Edition, https://schnitzler-briefe.acdh.oeaw.ac.at/{\dateiname}.html (Stand \today)
\fi

\end{document}


