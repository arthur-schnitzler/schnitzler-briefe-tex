%% latex-korrekturansicht-vorspann.tex
%% Vorspann für die Korrekturansicht.
%% Lädt die gemeinsame Datei latex-vorspann.tex mit gesetztem Schalter.

\newif\ifkorrekturansicht
\korrekturansichttrue

\input{../tex-inputs/latex-vorspann}


\section[ Felix Salten an Arthur Schnitzler, {[}19?. 11. 1903{]}]{L03351 Felix Salten an Arthur Schnitzler, {[}19?. 11. 1903{]}}
\nopagebreak\mylabel{L03351v}
\rehead{ }\normalsize\beginnumbering\briefempfaengerindex{Schnitzler, Arthur@\textsc{Schnitzler, Arthur}!zzzSalten, Felix@\emph{von Felix Salten}!1903-11-191@{{[}19?. 11. 1903{]}}|(be}
\toendnotes[C]{\smallbreak\pagebreak[2]}\Standort{CUL, Schnitzler, B 89, A 2.}
\physDesc{Brief, 1 Blatt, 1 Seite, 284 Zeichen
\newline{}Handschrift: schwarze Tinte, lateinische Kurrent
\newline{}Schnitzler: mit Bleistift datiert: »18/11 903« 
\newline{}Ordnung: mit Bleistift von unbekannter Hand nummeriert: »177« }\toendnotes[C]{\smallbreak}
\pstart
           \raggedleft{}{\pb}\label{K_L03351-1v}\edtext{Donnerstag}{\lemma{\textnormal{\emph{Donnerstag}}}\Cendnote{\textnormal{Diese Wochentagsangabe und die
                        Datierung Schnitzlers, die einen
                        Mittwoch nennt, widersprechen sich. Die Angabe der Uhrzeit nach
                           Mitternacht stützt die Annahme, dass das Schreiben in der Nacht von Mittwoch auf Donnerstag um 1 Uhr früh – also
                        tatsächlich an einem Donnerstag – verfasst wurde.}}}\label{K_L03351-1}.\pend
           
\pstart
           \raggedleft{}1\textsuperscript{h} früh.\pend
           \vspace{0.5em}
\pstart
           Lieber Freund, wenn Sie Beide\pwindex{Schnitzler, Olga 17.01.1882 – 13.01.1970@\textsc{Schnitzler, Olga} (17.01.1882 – 13.01.1970), \emph{Schauspieler/Schauspielerin, Sänger/Sängerin}|pwv}{ }heute{ }Abend mit Safonoff\pwindex{Safonov, Vasilij Ilʹic 1852-02-06 – 1918-02-27@\textsc{Safonov, Vasilij Ilʹič} (1852-02-06 – 1918-02-27), \emph{Dirigent/Dirigentin, Pianist/Pianistin, Musiker/Musikerin}|pw} bei uns\pwindex{Salten, Ottilie 07.03.1868 – 22.06.1942@\textsc{Salten, Ottilie} (07.03.1868 – 22.06.1942), \emph{Schauspieler/Schauspielerin}|pwv} essen wollten (8\textsuperscript{h.}) würden wir uns herzlich darüber freuen. Safonoff\pwindex{Safonov, Vasilij Ilʹic 1852-02-06 – 1918-02-27@\textsc{Safonov, Vasilij Ilʹič} (1852-02-06 – 1918-02-27), \emph{Dirigent/Dirigentin, Pianist/Pianistin, Musiker/Musikerin}|pw} ist eben angekommen, deshalb bitte ich wegen der knappen Frist um
               Entschuldigung. Sie pneumatisiren mir hoffentlich Ihre \label{K_L03351-2v}\edtext{Zusage}{\lemma{\textnormal{\emph{Zusage}}}\Cendnote{\textnormal{Schnitzler war anderweitig verpflichtet,
                     vgl. A. S.: \emph{Tagebuch}, 19. 11. 1903.}}}\label{K_L03351-2}\textcolor{gray}{.}\pend
           \pstart herzlichst Ihr \spacefill\mbox{S.}\pend{}\selectlanguage{ngerman}\endnumbering\briefempfaengerindex{Schnitzler, Arthur@\textsc{Schnitzler, Arthur}!zzzSalten, Felix@\emph{von Felix Salten}!1903-11-191@{{[}19?. 11. 1903{]}}|)be}\mylabel{L03351h}  \normalsize

\doendnotes{C}
\bigskip
\vfill

\clearpage

\footnotesize

\lohead{\textsc{register}}

% Definiere theindex-Environment komplett neu ohne reledmac
\makeatletter
\renewenvironment{theindex}{%
  \section*{\indexname}%
  \setlength{\parindent}{0pt}%
  \setlength{\parskip}{0pt plus 0.3pt}%
  \let\item\@idxitem
}{%
  \clearpage
}
\makeatother

\IfFileExists{\jobname-pw.ind}{\input{\jobname-pw.ind}}{}

\end{document}

      