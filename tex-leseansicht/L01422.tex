%% latex-leseansicht-vorspann.tex
%% Vorspann für die Leseansicht.
%% Lädt die gemeinsame Datei latex-vorspann.tex mit nicht gesetztem Schalter.

\newif\ifkorrekturansicht
\korrekturansichtfalse

\input{../tex-inputs/latex-vorspann}


         
         \renewcommand{\erwaehntePersonen}{Personen: Richard Beer-Hofmann, Hugo von Hofmannsthal, Gertrude von Hofmannsthal, Franziska Mütter, Olga Schnitzler, Heinrich Schnitzler, Julius Schnitzler}
         \renewcommand{\erwaehnteOrte}{Orte: Bad Aussee, Bad Fusch, Bad Ischl, Der rothe Stadl, Kalksburg, Kalksburger Straße, Mauer, Mauerbach, Nasswald, Rax, Reichenau an der Rax, Rodaun, Sophienalpe, Taormina, Wien, Wienerwald}
         \renewcommand{\erwaehnteWerke}{Werke: Das gerettete Venedig. Trauerspiel in fünf Aufzügen, Das neue Lied, Der Weg ins Freie. Roman, Der junge Medardus. Dramatische Historie in einem Vorspiel und fünf Aufzügen, Ritterlichkeit, Zwischenspiel. Komödie in drei Akten}
               \section[Arthur Schnitzler an Hugo von Hofmannsthal, 5.–6. 8. 1904]{ Arthur Schnitzler an Hugo von Hofmannsthal, 5.–6. 8. 1904}\nopagebreak\mylabel{v}\rehead{ }\begin{ledgroupsized}[t]{13cm}\normalsize\beginnumbering\briefempfaengerindex{Hofmannsthal, Hugo von@\textsc{Hofmannsthal, Hugo von}!zzzSchnitzler, Arthur@\emph{von Arthur Schnitzler}!1904-08-061@{5.–6. 8. 1904}|(be} \toendnotes[C]{\smallbreak\pagebreak[2]} \Standort{FDH, Hs-30885,110.}
\physDesc{Brief, 2 Blätter, 8 Seiten, 3648 Zeichen
\newline{}Handschrift: schwarze Tinte, deutsche Kurrent
\newline{}Ordnung: mit Bleistift von Schnitzler mutmaßlich bei der Durchsicht der
                                 Korrespondenz 1929 das zweite Blatt nummeriert:
                                    »II« und datiert: »5/8 904« }\buchAbdrucke{\weitereDrucke{Hugo von Hofmannsthal, Arthur Schnitzler: \emph{Briefwechsel}. Hg. Therese Nickl und Heinrich Schnitzler. Frankfurt am Main: \emph{S. Fischer} 1964, S. 192–193.} }\toendnotes[C]{\smallbreak}\pstart
           \raggedleft{}{\pb}Wien\oindex{Wien@\textbf{Wien}|pw}, 5. 8. 904\pend
           \pstart
           lieber Hugo, Ihr Brief aus der Fuſch\oindex{Bad Fusch@\textbf{Bad Fusch}|pw} hat mich ſehr erfreut und ich bin begierig was Sie nun eigentlich
               alles außer dem geretteten Venedig\pwindex{Hofmannsthal, Hugo von 1874-02-01 – 1929-07-15@\textsc{Hofmannsthal, Hugo von} (1874-02-01 – 1929-07-15), \emph{Schriftsteller}!gerettete Venedig. Trauerspiel in fuenf Aufzuegen1905@\strich\emph{Das gerettete Venedig. Trauerspiel in fünf Aufzügen} {[}1905{]}|pw} von dieſem
                  So{\geminationm}er nach Hauſe bringen werden. In der Wärme die uns
               umfließt, in der Beſo{\geminationn}theit der ganzen Atmosphäre muſs
               doch etwas ſeltſam befruchtendes liegen, denn auch mir geht es ſo gut wie lange
               nicht. Es hat begonnen an einem der erſten \label{K_L01422-1v}\edtext{Tage}{\lemma{\textnormal{\emph{Tage}}}\Cendnote{\textnormal{vgl. A. S.: \emph{Tagebuch}, 3. 7. 1904}}}\label{K_L01422-1h}, da ich von meinem Unwohlſein wieder aufgeſtanden war – wo ich \introOben{}Nachmittags\introOben{} eine ganze Novellette\pwindex{Schnitzler, Arthur 15.05.1862 – 21.10.1931@\textsc{Schnitzler, Arthur} (15.05.1862 – 21.10.1931), \emph{Schriftsteller, Mediziner}!neue Lied23. 04. 1905@\strich\emph{Das neue Lied} {[}23. 04. 1905{]}|pwv}
               niederſchrieb, die mir (der Einfall beſtand ſchon ſeit {\pb}lange) Vormittags auf einem Spaziergang aufgegangen war. Dann
               arbeitete ich an dem Roman\pwindex{Schnitzler, Arthur 15.05.1862 – 21.10.1931@\textsc{Schnitzler, Arthur} (15.05.1862 – 21.10.1931), \emph{Schriftsteller, Mediziner}!Weg ins Freie. Roman1.1.1908 – 1.6.1908@\strich\emph{Der Weg ins Freie. Roman} {[}1.1.1908 – 1.6.1908{]}|pwv}
               weiter, deſſen Fülle ich nur mehr möchte beherrſchen können. Vom
                  12.–24 (ungefähr) waren wir in Reichenau\oindex{Reichenau an der Rax@\textbf{Reichenau an der Rax}|pw}, wo ich auch in guter Sti{\geminationm}ung weiterſchrieb. Ausflüge Naßwald\oindex{Nasswald@\textbf{Nasswald}|pw}, Rax\oindex{Rax@\textbf{Rax}|pw}. Rad beinah gar nicht – die vielen müheloſen
               Dahinraſer im Automobil verderben einem die naive Freude. Aber es wird ſchon
                  wiederko{\geminationm}en, in fremdem Gegenden.\pend
           \pstart
           Nun ſind wir ſeit etwa 12 Tagen wieder in Wien\oindex{Wien@\textbf{Wien}|pw} und
               in unſerer {\pb}angenehmen Wohnung gefällt es uns ſehr gut
               und wir finden uns alle Vater, Mutter\pwindex{Schnitzler, Olga 17.01.1882 – 13.01.1970@\textsc{Schnitzler, Olga} (17.01.1882 – 13.01.1970), \emph{Schauspielerin, Sängerin}|pwv} und Kind\pwindex{Schnitzler, Heinrich 09.08.1902 – 12.07.1982@\textsc{Schnitzler, Heinrich} (09.08.1902 – 12.07.1982), \emph{Regisseur, Schauspieler}|pwv}
               behaglich. Seit der Julius\pwindex{Schnitzler, Julius 13.07.1865 – 29.06.1939@\textsc{Schnitzler, Julius} (13.07.1865 – 29.06.1939), \emph{Chirurg}|pw} auf Ferien iſt
               ſteht uns ſein Fiaker zur Verfügung \strikeout{iſt}, und ſo fahr
               ich mit Olga\pwindex{Schnitzler, Olga 17.01.1882 – 13.01.1970@\textsc{Schnitzler, Olga} (17.01.1882 – 13.01.1970), \emph{Schauspielerin, Sängerin}|pw} jeden Abend aufs Land, immer aufs
               neue u immer mehr entzückt von dieſen Wiener
                  Wald\oindex{Wienerwald@\textbf{Wienerwald}|pw} Landſchaften – die mich beinah immer ſo ergreifen als käme ich nach
               langen Jahren von irgendwoher in dieſe heimatliche Wunderſamkeit zurück. Geſtern
               Abend fuhren wir an dem verwaiſten Ro{\pb}daun\oindex{Rodaun@\textbf{Rodaun}|pw} ganz nah vorüber, von Mauer\oindex{Mauer@\textbf{Mauer}|pw} über Kalksburg\oindex{Kalksburg@\textbf{Kalksburg}|pw} (eine
               Waldſtraße, Klauſenſtraße\oindex{Kalksburger Strasse@\textbf{Kalksburger Straße}|pw} glaub ich, die ich
               noch gar nicht kannte) nach dem rothen Stadel\oindex{Der rothe Stadl@\textbf{Der rothe Stadl}|pw},
               und haben Ihrer und Richard\pwindex{Beer-Hofmann, Richard 1866-07-11 – 1945-09-26@\textsc{Beer-Hofmann, Richard} (1866-07-11 – 1945-09-26), \emph{Schriftsteller}|pw}s herzlich gedacht.
               (Es war ſozuſagen eine ungeſchriebene Anſichtskarte, die ſich abſpielte) –\pend
           \pstart
           Vor ein \label{K_L01422-2v}\edtext{paar Tagen}{\lemma{\textnormal{\emph{paar Tagen}}}\Cendnote{\textnormal{vgl. A. S.: \emph{Tagebuch}, 31. 7. 1904}}}\label{K_L01422-2h}, in Mauerbach\oindex{Mauerbach@\textbf{Mauerbach}|pw}, entwickelte ſich plötzlich
               aus einer kleinen Notiz, die ich in mein Büchel eingetragen hatte, im Geſpräch mit
                  Olga\pwindex{Schnitzler, Olga 17.01.1882 – 13.01.1970@\textsc{Schnitzler, Olga} (17.01.1882 – 13.01.1970), \emph{Schauspielerin, Sängerin}|pw}, ein völliges Luſtſpielſujet\pwindex{Schnitzler, Arthur 15.05.1862 – 21.10.1931@\textsc{Schnitzler, Arthur} (15.05.1862 – 21.10.1931), \emph{Schriftsteller, Mediziner}!Zwischenspiel. Komoedie in drei Akten1905-10-12@\strich\emph{Zwischenspiel. Komödie in drei Akten} {[}1905-10-12{]}|pwv}, am nächſten Tag ent{\pb}warf ich das \textsc{Scenarium}, am
               übernächſten ſtanden die Geſtalten ſchon ſo klar vor mir, daſs ich mich berechtigt
               fühlte, die erſte ſchlamperte Niederſchrift zu beginnen, die mich wohl nicht lange in
               Anſpruch nehmen wird. Es ka{\geminationn}, we{\geminationn} die Laune bleibt, ein graziöſes Ding werden. Ein
               andres Stück, eine 5aktige
                  Komödie\pwindex{\textcolor{red}{\textsuperscript{XXXX1 indx}}!Ritterlichkeit1977@\strich\emph{Ritterlichkeit} {[}Hrsg., 1977{]}|pwv}, von der in Taormina\oindex{Taormina@\textbf{Taormina}|pw} 3 Akte ganz
               flüchtig und zum Theil blödſinnig hingeſchmiſſen wurden, die ſich aber hier,
               wenigſtens im Plan, zu etwas ſehr möglichem entwickelte, {\pb}bleibt nun bis auf weiteres liegen. Von dem phantaſtiſch hiſtoriſchen Stück\pwindex{Schnitzler, Arthur 15.05.1862 – 21.10.1931@\textsc{Schnitzler, Arthur} (15.05.1862 – 21.10.1931), \emph{Schriftsteller, Mediziner}!junge Medardus. Dramatische Historie in einem Vorspiel und fuenf Aufzuegen1910-10-26@\strich\emph{Der junge Medardus. Dramatische Historie in einem Vorspiel und fünf Aufzügen} {[}1910-10-26{]}|pwv} und manchem andern, das
               in zweiter Reihe und dritter ſteht, will ich vorläufig nicht reden; ich möchte nur
               das ſtrategiſche Talent haben, die Truppen, die ich vorläufig nicht brauche, mit der
               nöthigen Autorität in die Reſerve oder wenigſtens hinter die Schlachtlinie zu
               verweiſen (Hören Sie den ehemaligen k. u. k. Oberarzt aus dieſen Worten trompeten?)
               Außerdem {\pb}möcht ich allerdings noch manches andre: vor
               allem mehr Fleiſs{\dots}\pend
           \pstart
           \raggedleft{}6. 8\pend
           \pstart
           wurde geſtern unterbrochen und will heute nur noch viele ſchöne Grüße hinzuſetzen.
               Heute (es iſt Nachmittg) waren wir ſchon am Vormittag auf
               der Sophienalpe\oindex{Sophienalpe@\textbf{Sophienalpe}|pw}, und das iſt die Gegend, wo ich
               von den Geſtalten des Romans\pwindex{Schnitzler, Arthur 15.05.1862 – 21.10.1931@\textsc{Schnitzler, Arthur} (15.05.1862 – 21.10.1931), \emph{Schriftsteller, Mediziner}!Weg ins Freie. Roman1.1.1908 – 1.6.1908@\strich\emph{Der Weg ins Freie. Roman} {[}1.1.1908 – 1.6.1908{]}|pwv}
               am härteſten bedrängt werde. –\pend
           \pstart
           Wir bleiben nun denk ich bis Anfang September hier in Wien\oindex{Wien@\textbf{Wien}|pw}, und dann möchten wir, auf etwa 14 Tage nicht allzu weit,
                  Iſchl\oindex{Bad Ischl@\textbf{Bad Ischl}|pw} etwa. Es {\pb}wäre nicht undenkbar, daſs die Fanny Mütter\pwindex{Muetter, Franziska 25.05.1858 – 23.02.1919@\textsc{Mütter, Franziska} (25.05.1858 – 23.02.1919), \emph{Sängerin, Gesangspädagogin}|pw}
               mitkommt; aber ich halt es für unwahrſcheinlich. Kämen Sie da{\geminationn} event. auch mit \textsc{Gerty}\pwindex{Hofmannsthal, Gertrude von 16.03.1880 – 09.11.1959@\textsc{Hofmannsthal, Gertrude von} (16.03.1880 – 09.11.1959)|pw}, ſo könnten wir zwei ein paar unſrer ſchönen Radtouren vollführen? – Jedenfalls
               treffen wir uns im Herbſt, nicht wahr? –\pend
           \pstart
           Grüßen Sie was Sie in Auſſee\oindex{Bad Aussee@\textbf{Bad Aussee}|pw} von erfreulichen
               Menſchen ſehen und antworten mir raſcher als ich Ihnen diesmal geantwortet habe.\pend
           \pstart
           Herzlichſt\hspace*{1.5em}Ihr{\\[\baselineskip]}\spacefill\mbox{A.}\pend
           \leftskip=0em{}
         
         \endnumbering\mylabel{h}\end{ledgroupsized}  \newcommand{\dateiname}{L01422}\newcommand{\titel}{Arthur Schnitzler an Hugo von Hofmannsthal, 5.–6. 8. 1904}\newcommand{\editorInnen}{Martin Anton Müller und Gerd-Hermann Susen}%% latex-leseansicht-abspann.tex
%% Abspann für die Leseansicht.
%% Der Schalter \ifkorrekturansicht ist bereits durch den Vorspann gesetzt.

%% latex-abspann.tex
%% Gemeinsamer Abspann für Korrekturansicht und Leseansicht.
%% Setzt den Schalter \ifkorrekturansicht voraus (gesetzt in den
%% einbindenden Dateien latex-korrekturansicht-abspann.tex bzw.
%% latex-leseansicht-abspann.tex).
%% ---------------------------------------------------------------

\normalsize

% Das esempio-Environment wird nur in der Leseansicht benötigt
\ifkorrekturansicht\else
\newenvironment{esempio}[3]%
{
    \vspace{1.5ex}
    \rlap{\underline{#1}}
    \par
    \setlength{\parindent}{0cm}
    \nopagebreak
    \leftskip=#2cm
    \rightskip=#3cm
}
{
    \par
}
\fi

\doendnotes{C}
\bigskip
\vfill

\clearpage

\footnotesize

\ifkorrekturansicht
  \lohead{\textsc{register}}
\fi

% theindex-Environment neu definieren ohne reledmac
\makeatletter
\renewenvironment{theindex}{%
  \ifkorrekturansicht
    \section*{\indexname}%
  \else
    \subsubsection*{Index der erwähnten Entitäten}%
  \fi
  \setlength{\parindent}{0pt}%
  \setlength{\parskip}{0pt plus 0.3pt}%
  \let\item\@idxitem
}{%
  \ifkorrekturansicht\clearpage\fi
}
\makeatother

\IfFileExists{\jobname-pw.ind}{\input{\jobname-pw.ind}}{}

% Quellenangabe nur in der Leseansicht
\ifkorrekturansicht\else
% Fallback-Definitionen, falls die .tex-Datei \titel etc. nicht gesetzt hat
\providecommand{\titel}{}
\providecommand{\editorInnen}{}
\providecommand{\dateiname}{\jobname}

\vspace{3cm}

\vfill

\footnotesize
\textsc{Quelle}: \titel. Herausgegeben von {\editorInnen}. In: \emph{Arthur Schnitzler: Briefwechsel mit Autorinnen und Autoren}.
 Digitale Edition, https://schnitzler-briefe.acdh.oeaw.ac.at/{\dateiname}.html (Stand \today)
\fi

\end{document}


      