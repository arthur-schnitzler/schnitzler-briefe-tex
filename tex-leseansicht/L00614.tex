%% latex-korrekturansicht-vorspann.tex
%% Vorspann für die Korrekturansicht.
%% Lädt die gemeinsame Datei latex-vorspann.tex mit gesetztem Schalter.

\newif\ifkorrekturansicht
\korrekturansichttrue

\input{../tex-inputs/latex-vorspann}


\section[Arthur Schnitzler an Hugo von Hofmannsthal, 31. 10. 1896]{L00614 Arthur Schnitzler an Hugo von Hofmannsthal, 31. 10. 1896}
\nopagebreak\mylabel{L00614v}
\rehead{ }\normalsize\beginnumbering\briefempfaengerindex{Hofmannsthal, Hugo von@\textsc{Hofmannsthal, Hugo von}!zzzSchnitzler, Arthur@\emph{von Arthur Schnitzler}!1896-10-312@{31. 10. 1896}|(be}
\toendnotes[C]{\smallbreak\pagebreak[2]}\Standort{FDH, Hs-30885,53.}
\physDesc{Brief, 1 Blatt, 3 Seiten, 562 Zeichen
\newline{}Handschrift: Bleistift, deutsche Kurrent}
\buchAbdrucke{\weitereDrucke{Hugo von Hofmannsthal, Arthur Schnitzler: \emph{Briefwechsel}. Frankfurt am Main: \emph{S. Fischer} 1964, S. 76.} }\toendnotes[C]{\smallbreak}
\pstart
           \raggedleft{}{\pb}31. X. 96.\pend
           \vspace{0.5em}
\pstart
           Lieber Hugo, iſt das liebe \label{K_L00614-1v}\edtext{Telegramm}{\lemma{\textnormal{\emph{Telegramm}}}\Cendnote{\textnormal{Vgl. Richard Beer-Hofmann an Arthur Schnitzler, 30. [10. 1896].}}}\label{K_L00614-1} von dem »Halbwahren aus \textsc{Upsala}\oindex{Uppsala@\textbf{Uppsala}, \emph{A.ADM4}|pw}« von Ihnen –?\pend
           
\pstart
           Wie i{\geminationm}er; ich grüße Sie herzlich. Den Thor u Tod\pwindex{Thor und der Tod@\emph{Der Thor und der Tod}|pw} hat Brahm\pwindex{Brahm, Otto 05.02.1856 – 28.11.1912@\textsc{Brahm, Otto} (05.02.1856 – 28.11.1912), \emph{Theaterleiter/Theaterleiterin, Regisseur/Regisseurin}|pw} geſtern durchgeflogen u will ihn morgen \uline{leſen}. {\pb}Die Beſetzung hab ich ihm ſchon
               mitgetheilt. –\pend
           
\pstart
           Heute war Generalprobe von Freiwild\pwindex{Freiwild. Schauspiel in 3 Akten@\emph{Freiwild. Schauspiel in 3 Akten}|pw}; \textsc{Gerhart Hauptmann}\pwindex{Hauptmann, Gerhart 15.11.1862 – 06.06.1946@\textsc{Hauptmann, Gerhart} (15.11.1862 – 06.06.1946), \emph{Schriftsteller/Schriftstellerin}|pw} u \textsc{Georg Hirſchfeld}\pwindex{Hirschfeld, Georg 11.02.1873 – 17.01.1942@\textsc{Hirschfeld, Georg} (11.02.1873 – 17.01.1942), \emph{Schriftsteller/Schriftstellerin}|pw} waren dabei, und es hat offenbar auf ſie gewirkt. Mit \textsc{Hauptmann} bin ich ſchon {\pb}ein paar Mal zuſa{\geminationm}en geweſen; er iſt mir außerordentlich ſympathiſch;
               ſchon ſeine Art zu ſchauen hat mich für ihn eingenommen. –\pend
           
\pstart
           Grüßen Sie Richard\pwindex{Beer-Hofmann, Richard 1866-07-11 – 1945-09-26@\textsc{Beer-Hofmann, Richard} (1866-07-11 – 1945-09-26), \emph{Schriftsteller/Schriftstellerin}|pw} vielmals!\pend
           \pstart Ihr \spacefill\mbox{Arthur}\pend{}
\pstart
           \noindent{}Wie gehts der Novelle\pwindex{Geschichte der beiden Liebespaare@\emph{Geschichte der beiden Liebespaare}|pwv}?\pend
           \selectlanguage{ngerman}\endnumbering\briefempfaengerindex{Hofmannsthal, Hugo von@\textsc{Hofmannsthal, Hugo von}!zzzSchnitzler, Arthur@\emph{von Arthur Schnitzler}!1896-10-312@{31. 10. 1896}|)be}\mylabel{L00614h}  \normalsize

\doendnotes{C}
\bigskip
\vfill

\clearpage

\footnotesize

\lohead{\textsc{register}}

% Definiere theindex-Environment komplett neu ohne reledmac
\makeatletter
\renewenvironment{theindex}{%
  \section*{\indexname}%
  \setlength{\parindent}{0pt}%
  \setlength{\parskip}{0pt plus 0.3pt}%
  \let\item\@idxitem
}{%
  \clearpage
}
\makeatother

\IfFileExists{\jobname-pw.ind}{\input{\jobname-pw.ind}}{}

\end{document}

      