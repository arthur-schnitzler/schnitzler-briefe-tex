%% latex-korrekturansicht-vorspann.tex
%% Vorspann für die Korrekturansicht.
%% Lädt die gemeinsame Datei latex-vorspann.tex mit gesetztem Schalter.

\newif\ifkorrekturansicht
\korrekturansichttrue

\input{../tex-inputs/latex-vorspann}


\section[Hugo von Hofmannsthal an Arthur Schnitzler, 25. 3. 1897]{L00663 Hugo von Hofmannsthal an Arthur Schnitzler, 25. 3. 1897}
\nopagebreak\mylabel{L00663v}
\rehead{ }\normalsize\beginnumbering\briefempfaengerindex{Schnitzler, Arthur@\textsc{Schnitzler, Arthur}!zzzHofmannsthal, Hugo von@\emph{von Hugo von Hofmannsthal}!1897-03-251@{25. 3. 1897}|(be}
\toendnotes[C]{\smallbreak\pagebreak[2]}\buchAlsQuelle{Hugo von Hofmannsthal, Arthur Schnitzler: \emph{Briefwechsel}. Frankfurt am Main: \emph{S. Fischer} 1964, S. 79.}
\pstart
           \raggedleft{}{\pb}25. 3. 97.\pend
           
\pstart{}lieber Arthur,\pend\vspace{0.5em}
\pstart
           möchte gern manches hören, bin heute Abend fast sicher bei Richard\pwindex{Beer-Hofmann, Richard 1866-07-11 – 1945-09-26@\textsc{Beer-Hofmann, Richard} (1866-07-11 – 1945-09-26), \emph{Schriftsteller/Schriftstellerin}|pw} nachher im Pucher\oindex{Cafe Pucher@\textbf{Café Pucher}, \emph{Kaffeehaus (K.KAF)}|pw}.\pend
           \pstart \spacefill\mbox{Hugo}\pend{}
\pstart
           \noindent{}Vielleicht telephonieren Sie wenigstens um ½ 11 ins Pucher\oindex{Cafe Pucher@\textbf{Café Pucher}, \emph{Kaffeehaus (K.KAF)}|pw}.\pend
           \selectlanguage{ngerman}\endnumbering\briefempfaengerindex{Schnitzler, Arthur@\textsc{Schnitzler, Arthur}!zzzHofmannsthal, Hugo von@\emph{von Hugo von Hofmannsthal}!1897-03-251@{25. 3. 1897}|)be}\mylabel{L00663h}  \normalsize

\doendnotes{C}
\bigskip
\vfill

\clearpage

\footnotesize

\lohead{\textsc{register}}

% Definiere theindex-Environment komplett neu ohne reledmac
\makeatletter
\renewenvironment{theindex}{%
  \section*{\indexname}%
  \setlength{\parindent}{0pt}%
  \setlength{\parskip}{0pt plus 0.3pt}%
  \let\item\@idxitem
}{%
  \clearpage
}
\makeatother

\IfFileExists{\jobname-pw.ind}{\input{\jobname-pw.ind}}{}

\end{document}

      