%% latex-korrekturansicht-vorspann.tex
%% Vorspann für die Korrekturansicht.
%% Lädt die gemeinsame Datei latex-vorspann.tex mit gesetztem Schalter.

\newif\ifkorrekturansicht
\korrekturansichttrue

\input{../tex-inputs/latex-vorspann}


\section[Arthur Schnitzler an Hugo von Hofmannsthal, 17. 7. 1900]{L01057 Arthur Schnitzler an Hugo von Hofmannsthal, 17. 7. 1900}
\nopagebreak\mylabel{L01057v}
\rehead{ }\normalsize\beginnumbering\briefempfaengerindex{Hofmannsthal, Hugo von@\textsc{Hofmannsthal, Hugo von}!zzzSchnitzler, Arthur@\emph{von Arthur Schnitzler}!1900-07-171@{17. 7. 1900}|(be}
\toendnotes[C]{\smallbreak\pagebreak[2]}\Standort{FDH, Hs-30885,93.}
\physDesc{Brief, 2 Blätter, 8 Seiten, 3244 Zeichen
\newline{}Handschrift: schwarze Tinte, deutsche Kurrent
\newline{}Ordnung: mit Bleistift von Schnitzler mutmaßlich bei der Durchsicht der
                                 Korrespondenz 1929 das zweite Blatt datiert: »17/7 900« }
\buchAbdrucke{\weitereDrucke{1) Hugo von Hofmannsthal, Arthur Schnitzler: \emph{Briefwechsel}. Frankfurt am Main: \emph{S. Fischer} 1964, S. 141.} \weitereDrucke{2) Arthur Schnitzler: \emph{Briefe 1875–1912}. Frankfurt am Main: \emph{S. Fischer} 1981, S. 387–388.} }\toendnotes[C]{\smallbreak}
\pstart
           \raggedleft{}{\pb}Reichenau b/Payerbach{\\}Curhaus\oindex{Kurhaus Rudolfsbad@\textbf{Kurhaus Rudolfsbad}, \emph{Sanatorium (K.SAN)}|pw}.
                     17. 7. 900. \pend
           \vspace{0.5em}
\pstart
           mein lieber Hugo, wenn Sie dieſen Brief beko{\geminationm}en, ſind Sie ſchon wieder zurück von Ihrem kleinen
               Ausflug und haben hoffentlich \introOben{}alle\introOben{} Verdroſſenheit verloren.
                  \uline{Ich} wüßte wirklich nicht, was ich jetzt ohne
               Arbeit beginnen würde. Komme ich durch äußere Umſtände, unruhige Verhältniſſe durch
               einige Tage nicht dazu, wenigſtens ein paar kurze Stunden zu ſchreiben, ſo verſinke
               ich in eine wahre Schwermuth. Hier bin ich nun im ganzen {\pb}gut dran. Ob viel dabei herausko{\geminationm}en wird, bei dem
               nämlich was ich jetzt ſchreibe, iſt ja noch nicht ſicher, aber das weſentliche liegt
               ja wo anders. Nachher gibts ja beinah nur Aerger, ob einem was gelungen iſt oder
               nicht. Ich habe hier ein kleines Luſtſpiel\pwindex{Quellen des Nil@\emph{Die Quellen des Nil}|pwv} neu geſchrieben (deſſen erſte Faſſung \label{K_L01057-1v}\edtext{vor 2 Jahren in Tegernſee\oindex{Tegernsee@\textbf{Tegernsee}, \emph{P.PPL}|pw}}{\lemma{\textnormal{\emph{vor … Tegernſee}}}\Cendnote{\textnormal{Siehe A. S.: \emph{Tagebuch}, 2. 8. 1898.
               }}}\label{K_L01057-1} unter glücklichern Umſtänden entſtan\textcolor{gray}{d}) und bin jetzt mit
               einer ziemlich ſonderbaren Novelle\pwindex{Lieutenant Gustl. Novelle@\emph{Lieutenant Gustl. Novelle}|pwv} beſchäftigt, die mir viele Freude macht. Von dieſer {\pb}hoff ich zuverſichtlich, daſs ſie auch Ihnen andern
               Freude machen wird. Meine große Novelle\pwindex{Frau Bertha Garlan. Roman@\emph{Frau Bertha Garlan. Roman}|pwv} hab ich der \textsc{N. Dtsch. Rundschau}\orgindex{Neue Rundschau, Neue Deutsche Rundschau, Freie Buehne@Neue Rundschau, Neue Deutsche Rundschau, Freie Bühne|pw} gegeben; ſie iſt nicht übel ausgefallen; bisher kennen ſie Salten\pwindex{Salten, Felix 06.09.1869 – 08.10.1945@\textsc{Salten, Felix} (06.09.1869 – 08.10.1945), \emph{Schriftsteller/Schriftstellerin, Journalist/Journalistin, Chefredakteur/Chefredakteurin}|pw} u Schwarzkopf\pwindex{Schwarzkopf, Gustav 07.11.1853 – 13.11.1939@\textsc{Schwarzkopf, Gustav} (07.11.1853 – 13.11.1939), \emph{Schriftsteller/Schriftstellerin}|pw},
               die beide ſehr zufrieden ſcheinen. – Wie lange ich noch hier bleibe weiſs ich nicht
               genau; in etwa 8–10 Tagen dürfte ich jedenfalls in Wien\oindex{Wien@\textbf{Wien}, \emph{A.ADM2}|pw}{ }ſein; aber über die erſte Auguſthälfte
               herrſcht noch große Unklarheit. Mitte Auguſt{ }ſoll eine Fußtour bego{\geminationn}en werden, die {\pb}ich in \textsc{Altaussee}\oindex{Altaussee@\textbf{Altaussee}, \emph{A.ADM3}|pw} mit Richard\pwindex{Beer-Hofmann, Richard 1866-07-11 – 1945-09-26@\textsc{Beer-Hofmann, Richard} (1866-07-11 – 1945-09-26), \emph{Schriftsteller/Schriftstellerin}|pw} ausgeheckt habe. Paul Goldmann\pwindex{Goldmann, Paul 31.01.1865 – 25.09.1935@\textsc{Goldmann, Paul} (31.01.1865 – 25.09.1935), \emph{Schriftsteller/Schriftstellerin, Journalist/Journalistin}|pw}, Kerr\pwindex{Kerr, Alfred 25.12.1867 – 12.10.1948@\textsc{Kerr, Alfred} (25.12.1867 – 12.10.1948), \emph{Schriftsteller/Schriftstellerin, Kritiker/Kritikerin}|pw}, Oskar Meyer\pwindex{Mayer, Oskar 1876 – 15.05.1915@\textsc{Mayer, Oskar} (1876 – 15.05.1915), \emph{Schriftsteller/Schriftstellerin, Beamter/Beamte}|pw}{ }ſchließen ſich vielleicht an. Am Ende auch Georg Hirſchfeld\pwindex{Hirschfeld, Georg 11.02.1873 – 17.01.1942@\textsc{Hirschfeld, Georg} (11.02.1873 – 17.01.1942), \emph{Schriftsteller/Schriftstellerin}|pw} (Elly\pwindex{Petersen, Elly 26.02.1874 – 29.12.1965@\textsc{Petersen, Elly} (26.02.1874 – 29.12.1965), \emph{Schriftsteller/Schriftstellerin}|pw} dürfte wegen Kerr\pwindex{Kerr, Alfred 25.12.1867 – 12.10.1948@\textsc{Kerr, Alfred} (25.12.1867 – 12.10.1948), \emph{Schriftsteller/Schriftstellerin, Kritiker/Kritikerin}|pw} u
                  Goldmann\pwindex{Goldmann, Paul 31.01.1865 – 25.09.1935@\textsc{Goldmann, Paul} (31.01.1865 – 25.09.1935), \emph{Schriftsteller/Schriftstellerin, Journalist/Journalistin}|pw}{ }ſehr dafür ſein.) –\pend
           
\pstart
           Ein paar Stunden täglich plaudere ich mit einer angehenden nicht hübſchen Schauſpielerin\pwindex{Schnitzler, Olga 17.01.1882 – 13.01.1970@\textsc{Schnitzler, Olga} (17.01.1882 – 13.01.1970), \emph{Schauspieler/Schauspielerin, Sänger/Sängerin}|pwv}, die für ihre
               18 Jahre von einer unglaublichen Klugheit iſt. Sie wohnt hier mit ihrer Schweſter\pwindex{Steinrueck, Elisabeth 19.11.1885 – 07.04.1920@\textsc{Steinrück, Elisabeth} (19.11.1885 – 07.04.1920)|pwv}, die ein \label{K_L01057-2v}\edtext{16jähriges}{\lemma{\textnormal{\emph{16jähriges}}}\Cendnote{\textnormal{Elisabeth Gussmann\pwindex{Steinrueck, Elisabeth 19.11.1885 – 07.04.1920@\textsc{Steinrück, Elisabeth} (19.11.1885 – 07.04.1920)|pwk} war zu dem Zeitpunkt erst 14.}}}\label{K_L01057-2} keckes aber geſcheidtes
               Judenmädl iſt; ſtets {\pb}iſt auch ein junges blondes Ding\pwindex{Schimitschek, Bertha @\textsc{Schimitschek, Bertha}|pwv} mit ihnen, die
               wahrſcheinlich verrückt werden wird. Geſtern hab ich mit denen allen in ihrem kleinen
               Garten genachtmahlt. Die Schauſpielerin\pwindex{Schnitzler, Olga 17.01.1882 – 13.01.1970@\textsc{Schnitzler, Olga} (17.01.1882 – 13.01.1970), \emph{Schauspieler/Schauspielerin, Sänger/Sängerin}|pwv} hatte Nachmittags die \textsc{Madonna Dianora}\pwindex{Frau im Fenster@\emph{Die Frau im Fenster}|pw}{ }ſtudirt; der kleinen Schweſter\pwindex{Steinrueck, Elisabeth 19.11.1885 – 07.04.1920@\textsc{Steinrück, Elisabeth} (19.11.1885 – 07.04.1920)|pwv} hatte ein 20jähriger Verehrer\pwindex{?? [Verehrer von Elisabeth Steinrueck] @\textsc{?? [Verehrer von Elisabeth Steinrück]}|pwv} »Geſtern\pwindex{Gestern. Dramatische Studie in einem Akt in Versen@\emph{Gestern. Dramatische Studie in einem Akt in Versen}|pw}« aus Wien\oindex{Wien@\textbf{Wien}, \emph{A.ADM2}|pw} mitgebracht.
               Ich finde den Zufall hübſch, der es macht, daſs Sie das gleich erfahren können;
               nichts beruhigt mehr über die Vielheit u Verwirrtheit des Lebens, als we{\geminationn} man Fäden {\pb}irgendwo zuſa{\geminationm}en laufen ſieht. –\pend
           
\pstart
           Sonſt hab ich hier noch \textsc{Dr}\textsc{Redlich}\pwindex{Redlich, Josef 18.06.1869 – 11.11.1936@\textsc{Redlich, Josef} (18.06.1869 – 11.11.1936), \emph{Politiker/Politikerin, Rechtswissenschaftler/Rechtswissenschaftlerin}|pw} und ſeine Frau\pwindex{Leo, Alice 27.11.1876 – Juni 1966@\textsc{Leo, Alice} (27.11.1876 – Juni 1966)|pwv} (die
                  Königsberg\oindex{Kaliningrad@\textbf{Kaliningrad}, \emph{P.PPLA}|pw}erin) geſprochen; meine Mama\pwindex{Schnitzler, Louise 1840-07-08 – 1911-09-09@\textsc{Schnitzler, Louise} (1840-07-08 – 1911-09-09)|pwv} u meine Schweſter\pwindex{Hajek, Gisela 20.12.1867 – 03.02.1953@\textsc{Hajek, Gisela} (20.12.1867 – 03.02.1953)|pwv} wohnen hier, Schwägerin\pwindex{Schnitzler, Helene 16.07.1871 – September 1941@\textsc{Schnitzler, Helene} (16.07.1871 – September 1941)|pwv} u Familie in Edlach\oindex{Edlach@\textbf{Edlach}, \emph{P.PPL}|pw}. Den Vormittg verbu{\geminationm}l ich und verſpazier’ ich; nur nach Tiſch arbeite
               ich. – Wie denken Sie den Reſt des Sommers zu verbringen? Es iſt ſehr wahrſcheinlich,
               dſs ich Anfangs Auguſt in Iſchl\oindex{Bad Ischl@\textbf{Bad Ischl}, \emph{P.PPL}|pw}{ }ſein werde; ſollte man ſich nicht {\pb}irgendwo, in Salzburg\oindex{Salzburg@\textbf{Salzburg}, \emph{A.ADM2}|pw}
               z. B. begegnen können? – Richard\pwindex{Beer-Hofmann, Richard 1866-07-11 – 1945-09-26@\textsc{Beer-Hofmann, Richard} (1866-07-11 – 1945-09-26), \emph{Schriftsteller/Schriftstellerin}|pw} arbeitet. Als
               ich bei ihm war, befan\textcolor{gray}{d} ſich ſeine Frau\pwindex{Beer-Hofmann, Paula 25.02.1879 – 30.10.1939@\textsc{Beer-Hofmann, Paula} (25.02.1879 – 30.10.1939)|pwv} nicht ſehr wohl, doch ſcheint es jetzt
               viel beſſer oder ganz gut zu gehn. Schreiben Sie mir\strikeout{h}
               recht bald wieder, iſts kein Brief, ſo ſei es eine Karte. Aber verlieren wir uns
               keineswegs, auch nicht auf Tage, ganz aus den Augen.\pend
           
\pstart
           Ich hoffe Ihr Papa\pwindex{Hofmannsthal, Hugo August von 21.12.1841 – 08.12.1915@\textsc{Hofmannsthal, Hugo August von} (21.12.1841 – 08.12.1915), \emph{Bankdirektor/Bankdirektorin}|pwv} iſt ganz
               geſund. Grüßen Sie ihn, Ihre Mama\pwindex{Hofmannsthal, Anna von 27.01.1849 – 22.03.1904@\textsc{Hofmannsthal, Anna von} (27.01.1849 – 22.03.1904)|pwv}, und {\pb}die Familie Speyer\pwindex{Speyer, Nanette 05.01.1846 – 15.1.1925@\textsc{Speyer, Nanette} (05.01.1846 – 15.1.1925)|pw}\pwindex{Speyer, Albert 08.04.1836 – 25.03.1905@\textsc{Speyer, Albert} (08.04.1836 – 25.03.1905), \emph{Industrieller/Industrielle}|pw} mehr oder weniger.\pend
           
\pstart
           Herzlichſt der Ihrige{\\[\baselineskip]}\spacefill\mbox{Arthur.}\pend
           \leftskip=0em{}
\pstart
           Benützen Sie nur meine Wien\oindex{Wien@\textbf{Wien}, \emph{A.ADM2}|pw}er Adreſſe, das iſt am
               ſicherſten. Ich habe vergeſſen, daſs ich Sie von der Schauſpielerin\pwindex{Schnitzler, Olga 17.01.1882 – 13.01.1970@\textsc{Schnitzler, Olga} (17.01.1882 – 13.01.1970), \emph{Schauspieler/Schauspielerin, Sänger/Sängerin}|pwv}{ }ſehr herzlich grüßen ſoll.\pend
           \selectlanguage{ngerman}\endnumbering\briefempfaengerindex{Hofmannsthal, Hugo von@\textsc{Hofmannsthal, Hugo von}!zzzSchnitzler, Arthur@\emph{von Arthur Schnitzler}!1900-07-171@{17. 7. 1900}|)be}\mylabel{L01057h}  \normalsize

\doendnotes{C}
\bigskip
\vfill

\clearpage

\footnotesize

\lohead{\textsc{register}}

% Definiere theindex-Environment komplett neu ohne reledmac
\makeatletter
\renewenvironment{theindex}{%
  \section*{\indexname}%
  \setlength{\parindent}{0pt}%
  \setlength{\parskip}{0pt plus 0.3pt}%
  \let\item\@idxitem
}{%
  \clearpage
}
\makeatother

\IfFileExists{\jobname-pw.ind}{\input{\jobname-pw.ind}}{}

\end{document}

      