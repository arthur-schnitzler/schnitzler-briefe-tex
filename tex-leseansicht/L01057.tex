%% latex-leseansicht-vorspann.tex
%% Vorspann für die Leseansicht.
%% Lädt die gemeinsame Datei latex-vorspann.tex mit nicht gesetztem Schalter.

\newif\ifkorrekturansicht
\korrekturansichtfalse

\input{../tex-inputs/latex-vorspann}


\section[Arthur Schnitzler an Hugo von Hofmannsthal, 17. 7. 1900]{L01057 Arthur Schnitzler an Hugo von Hofmannsthal, 17. 7. 1900}
\nopagebreak\mylabel{L01057v}
\rehead{ }\normalsize\beginnumbering\briefempfaengerindex{Hofmannsthal, Hugo von@\textsc{Hofmannsthal, Hugo von}!zzzSchnitzler, Arthur@\emph{von Arthur Schnitzler}!1900-07-171@{17. 7. 1900}|(be}
\toendnotes[C]{\smallbreak\pagebreak[2]}
\correspDesc{Versand  durch Arthur Schnitzler am 17. 7. 1900 in Reichenau an der Rax
\newline{}Erhalt  durch Hugo von Hofmannsthal im Zeitraum [18. 7. 1900
                  – 22. 7. 1900?] in Wien}\toendnotes[C]{\smallbreak}
\Standort{FDH, Hs-30885,93.}
\physDesc{Brief, 2 Blätter, 8 Seiten, 3244 Zeichen
\newline{}Handschrift: schwarze Tinte, deutsche Kurrent
\newline{}Ordnung: mit Bleistift von Schnitzler mutmaßlich bei der Durchsicht der
                                 Korrespondenz 1929 das zweite Blatt datiert: »17/7 900« }
\buchAbdrucke{\weitereDrucke{1) Hugo von Hofmannsthal, Arthur Schnitzler: \emph{Briefwechsel}. Herausgegeben von Therese Nickl und Heinrich Schnitzler. Frankfurt am Main: \emph{S. Fischer} 1964, S. 141.} \weitereDrucke{2) Arthur Schnitzler: \emph{Briefe 1875–1912}. Herausgegeben von Therese Nickl und Heinrich Schnitzler. Frankfurt am Main: \emph{S. Fischer} 1981, S. 387–388.} }\toendnotes[C]{\smallbreak}
\pstart
           \raggedleft{}{\pb}Reichenau b/Payerbach{\\}Curhaus\oindex{Kurhaus Rudolfsbad@\textbf{Kurhaus Rudolfsbad}, \emph{Sanatorium}|pw}.
                     17. 7. 900.\pend
           \vspace{0.5em}
\pstart
           mein lieber Hugo, wenn Sie dieſen Brief beko{\geminationm}en,{ }ſind Sie{ }ſchon wieder zurück von Ihrem kleinen
               Ausflug und haben hoffentlich \introOben{}alle\introOben{} Verdroſſenheit verloren.
                  \uline{Ich} wüßte wirklich nicht, was ich jetzt ohne
               Arbeit beginnen würde. Komme ich durch äußere Umſtände, unruhige Verhältniſſe durch
               einige Tage nicht dazu, wenigſtens ein paar kurze Stunden zu{ }ſchreiben,{ }ſo verſinke
               ich in eine wahre Schwermuth. Hier bin ich nun im ganzen {\pb}gut dran. Ob viel dabei herausko{\geminationm}en wird, bei dem
               nämlich was ich jetzt{ }ſchreibe, iſt ja noch nicht{ }ſicher, aber das weſentliche liegt
               ja wo anders. Nachher gibts ja beinah nur Aerger, ob einem was gelungen iſt oder
               nicht. Ich habe hier ein kleines Luſtſpiel\pwindex{Schnitzler, Arthur 15.\,5.\,1862 Wien – 21.\,10.\,1931 ebd.@\textsc{Schnitzler, Arthur} (15.\,5.\,1862 Wien – 21.\,10.\,1931 ebd.), \emph{Schriftsteller, Mediziner}!Quellen des Nil@\strich\emph{Die Quellen des Nil}|pwv} neu geſchrieben (deſſen erſte Faſſung \label{K_L01057-1v}\edtext{vor 2 Jahren in Tegernſee\oindex{Tegernsee@\textbf{Tegernsee}|pw}}{\lemma{\textnormal{\emph{vor … Tegernsee}}}\Cendnote{\textnormal{Siehe A. S.: \emph{Tagebuch}, 2. 8. 1898.
               }}}\label{K_L01057-1} unter glücklichern Umſtänden entſtan\textcolor{gray}{d}) und bin jetzt mit
               einer ziemlich{ }ſonderbaren Novelle\pwindex{Schnitzler, Arthur 15.\,5.\,1862 Wien – 21.\,10.\,1931 ebd.@\textsc{Schnitzler, Arthur} (15.\,5.\,1862 Wien – 21.\,10.\,1931 ebd.), \emph{Schriftsteller, Mediziner}!Lieutenant Gustl. Novelle@\strich\emph{Lieutenant Gustl. Novelle}|pwv} beſchäftigt, die mir viele Freude macht. Von dieſer {\pb}hoff ich zuverſichtlich, daſs{ }ſie auch Ihnen andern
               Freude machen wird. Meine große Novelle\pwindex{Schnitzler, Arthur 15.\,5.\,1862 Wien – 21.\,10.\,1931 ebd.@\textsc{Schnitzler, Arthur} (15.\,5.\,1862 Wien – 21.\,10.\,1931 ebd.), \emph{Schriftsteller, Mediziner}!Frau Bertha Garlan. Roman@\strich\emph{Frau Bertha Garlan. Roman}|pwv} hab ich der \textsc{N. Dtsch. Rundschau}\orgindex{Neue Rundschau, Neue Deutsche Rundschau, Freie Bühne@Neue Rundschau, Neue Deutsche Rundschau, Freie Bühne|pw} gegeben;{ }ſie iſt nicht übel ausgefallen; bisher kennen{ }ſie Salten\pwindex{Salten, Felix 6.\,9.\,1869 Budapest – 8.\,10.\,1945 Zürich@\textsc{Salten, Felix} (6.\,9.\,1869 Budapest – 8.\,10.\,1945 Zürich), \emph{Schriftsteller, Journalist, Chefredakteur}|pw} u Schwarzkopf\pwindex{Schwarzkopf, Gustav 7.\,11.\,1853 Wien – 13.\,11.\,1939 ebd.@\textsc{Schwarzkopf, Gustav} (7.\,11.\,1853 Wien – 13.\,11.\,1939 ebd.), \emph{Schriftsteller}|pw},
               die beide{ }ſehr zufrieden{ }ſcheinen. – Wie lange ich noch hier bleibe weiſs ich nicht
               genau; in etwa 8–10 Tagen dürfte ich jedenfalls in Wien\oindex{Wien@\textbf{Wien}, \emph{Verwaltungsgebiet}|pw}{ }ſein; aber über die erſte Auguſthälfte
               herrſcht noch große Unklarheit. Mitte Auguſt{ }ſoll eine Fußtour bego{\geminationn}en werden, die {\pb}ich in \textsc{Altaussee}\oindex{Altaussee@\textbf{Altaussee}, \emph{Verwaltungsgebiet}|pw} mit Richard\pwindex{Beer-Hofmann, Richard 11.\,7.\,1866 Wien – 26.\,9.\,1945 New York City@\textsc{Beer-Hofmann, Richard} (11.\,7.\,1866 Wien – 26.\,9.\,1945 New York City), \emph{Schriftsteller}|pw} ausgeheckt habe. Paul Goldmann\pwindex{Goldmann, Paul 31.\,1.\,1865 Breslau – 25.\,9.\,1935 Wien@\textsc{Goldmann, Paul} (31.\,1.\,1865 Breslau – 25.\,9.\,1935 Wien), \emph{Schriftsteller, Journalist}|pw}, Kerr\pwindex{Kerr, Alfred 25.\,12.\,1867 Breslau – 12.\,10.\,1948 Hamburg@\textsc{Kerr, Alfred} (25.\,12.\,1867 Breslau – 12.\,10.\,1948 Hamburg), \emph{Schriftsteller, Kritiker}|pw}, Oskar Meyer\pwindex{Mayer, Oskar 1876 – 15.\,5.\,1915 München@\textsc{Mayer, Oskar} (1876 – 15.\,5.\,1915 München), \emph{Schriftsteller, Beamter}|pw}{ }ſchließen{ }ſich vielleicht an. Am Ende auch Georg Hirſchfeld\pwindex{Hirschfeld, Georg 11.\,2.\,1873 Berlin – 17.\,1.\,1942 München@\textsc{Hirschfeld, Georg} (11.\,2.\,1873 Berlin – 17.\,1.\,1942 München), \emph{Schriftsteller}|pw} (Elly\pwindex{Petersen, Elly 26.\,2.\,1874 Berlin – 29.\,12.\,1965 München@\textsc{Petersen, Elly} (26.\,2.\,1874 Berlin – 29.\,12.\,1965 München), \emph{Schriftstellerin}|pw} dürfte wegen Kerr\pwindex{Kerr, Alfred 25.\,12.\,1867 Breslau – 12.\,10.\,1948 Hamburg@\textsc{Kerr, Alfred} (25.\,12.\,1867 Breslau – 12.\,10.\,1948 Hamburg), \emph{Schriftsteller, Kritiker}|pw} u
                  Goldmann\pwindex{Goldmann, Paul 31.\,1.\,1865 Breslau – 25.\,9.\,1935 Wien@\textsc{Goldmann, Paul} (31.\,1.\,1865 Breslau – 25.\,9.\,1935 Wien), \emph{Schriftsteller, Journalist}|pw}{ }ſehr dafür{ }ſein.) –\pend
           
\pstart
           Ein paar Stunden täglich plaudere ich mit einer angehenden nicht hübſchen Schauſpielerin\pwindex{Schnitzler, Olga 17.\,1.\,1882 Wien – 13.\,1.\,1970 Lugano@\textsc{Schnitzler, Olga} (17.\,1.\,1882 Wien – 13.\,1.\,1970 Lugano), \emph{Schauspielerin, Sängerin}|pwv}, die für ihre
               18 Jahre von einer unglaublichen Klugheit iſt. Sie wohnt hier mit ihrer Schweſter\pwindex{Steinrück, Elisabeth 19.\,11.\,1885 – 7.\,4.\,1920 Partenkirchen@\textsc{Steinrück, Elisabeth} (19.\,11.\,1885 – 7.\,4.\,1920 Partenkirchen)|pwv}, die ein \label{K_L01057-2v}\edtext{16jähriges}{\lemma{\textnormal{\emph{16jähriges}}}\Cendnote{\textnormal{Elisabeth Gussmann\pwindex{Steinrück, Elisabeth 19.\,11.\,1885 – 7.\,4.\,1920 Partenkirchen@\textsc{Steinrück, Elisabeth} (19.\,11.\,1885 – 7.\,4.\,1920 Partenkirchen)|pwk} war zu dem Zeitpunkt erst 14.}}}\label{K_L01057-2} keckes aber geſcheidtes
               Judenmädl iſt;{ }ſtets {\pb}iſt auch ein junges blondes Ding\pwindex{Schimitschek, Bertha @\textsc{Schimitschek, Bertha}|pwv} mit ihnen, die
               wahrſcheinlich verrückt werden wird. Geſtern hab ich mit denen allen in ihrem kleinen
               Garten genachtmahlt. Die Schauſpielerin\pwindex{Schnitzler, Olga 17.\,1.\,1882 Wien – 13.\,1.\,1970 Lugano@\textsc{Schnitzler, Olga} (17.\,1.\,1882 Wien – 13.\,1.\,1970 Lugano), \emph{Schauspielerin, Sängerin}|pwv} hatte Nachmittags die \textsc{Madonna Dianora}\pwindex{Hofmannsthal, Hugo von 1.\,2.\,1874 Wien – 15.\,7.\,1929 Rodaun@\textsc{Hofmannsthal, Hugo von} (1.\,2.\,1874 Wien – 15.\,7.\,1929 Rodaun), \emph{Schriftsteller}!Frau im Fenster@\strich\emph{Die Frau im Fenster}|pw}{ }ſtudirt; der kleinen Schweſter\pwindex{Steinrück, Elisabeth 19.\,11.\,1885 – 7.\,4.\,1920 Partenkirchen@\textsc{Steinrück, Elisabeth} (19.\,11.\,1885 – 7.\,4.\,1920 Partenkirchen)|pwv} hatte ein 20jähriger Verehrer\pwindex{?? [Verehrer von Elisabeth Steinrück] @\textsc{?? [Verehrer von Elisabeth Steinrück]}|pwv} »Geſtern\pwindex{Hofmannsthal, Hugo von 1.\,2.\,1874 Wien – 15.\,7.\,1929 Rodaun@\textsc{Hofmannsthal, Hugo von} (1.\,2.\,1874 Wien – 15.\,7.\,1929 Rodaun), \emph{Schriftsteller}!Gestern. Dramatische Studie in einem Akt in Versen@\strich\emph{Gestern. Dramatische Studie in einem Akt in Versen}|pw}« aus Wien\oindex{Wien@\textbf{Wien}, \emph{Verwaltungsgebiet}|pw} mitgebracht.
               Ich finde den Zufall hübſch, der es macht, daſs Sie das gleich erfahren können;
               nichts beruhigt mehr über die Vielheit u Verwirrtheit des Lebens, als we{\geminationn} man Fäden {\pb}irgendwo zuſa{\geminationm}en laufen{ }ſieht. –\pend
           
\pstart
           Sonſt hab ich hier noch \textsc{Dr}{ }\textsc{Redlich}\pwindex{Redlich, Josef 18.\,6.\,1869 Hodonín – 11.\,11.\,1936 Wien@\textsc{Redlich, Josef} (18.\,6.\,1869 Hodonín – 11.\,11.\,1936 Wien), \emph{Politiker, Rechtswissenschaftler}|pw} und{ }ſeine Frau\pwindex{Leo, Alice 27.\,11.\,1876 Kaliningrad – Juni 1966 Rosenheim@\textsc{Leo, Alice} (27.\,11.\,1876 Kaliningrad – Juni 1966 Rosenheim)|pwv} (die
                  Königsberg\oindex{Kaliningrad@\textbf{Kaliningrad}|pw}erin) geſprochen; meine Mama\pwindex{Schnitzler, Louise 8.\,7.\,1840 Kőszeg – 9.\,9.\,1911 Wien@\textsc{Schnitzler, Louise} (8.\,7.\,1840 Kőszeg – 9.\,9.\,1911 Wien)|pwv} u meine Schweſter\pwindex{Hajek, Gisela 20.\,12.\,1867 Wien – 3.\,2.\,1953 Cambridge@\textsc{Hajek, Gisela} (20.\,12.\,1867 Wien – 3.\,2.\,1953 Cambridge)|pwv} wohnen hier, Schwägerin\pwindex{Schnitzler, Helene 16.\,7.\,1871 Budapest – September 1941 Atlantischer Ozean@\textsc{Schnitzler, Helene} (16.\,7.\,1871 Budapest – September 1941 Atlantischer Ozean)|pwv} u Familie in Edlach\oindex{Edlach@\textbf{Edlach}|pw}. Den Vormittg verbu{\geminationm}l ich und verſpazier’ ich; nur nach Tiſch arbeite
               ich. – Wie denken Sie den Reſt des Sommers zu verbringen? Es iſt{ }ſehr wahrſcheinlich,
               dſs ich Anfangs Auguſt in Iſchl\oindex{Bad Ischl@\textbf{Bad Ischl}|pw}{ }ſein werde;{ }ſollte man{ }ſich nicht {\pb}irgendwo, in Salzburg\oindex{Salzburg@\textbf{Salzburg}, \emph{Verwaltungsgebiet}|pw}
               z. B. begegnen können? – Richard\pwindex{Beer-Hofmann, Richard 11.\,7.\,1866 Wien – 26.\,9.\,1945 New York City@\textsc{Beer-Hofmann, Richard} (11.\,7.\,1866 Wien – 26.\,9.\,1945 New York City), \emph{Schriftsteller}|pw} arbeitet. Als
               ich bei ihm war, befan\textcolor{gray}{d}{ }ſich{ }ſeine Frau\pwindex{Beer-Hofmann, Paula 25.\,2.\,1879 Wien – 30.\,10.\,1939 Zürich@\textsc{Beer-Hofmann, Paula} (25.\,2.\,1879 Wien – 30.\,10.\,1939 Zürich)|pwv} nicht{ }ſehr wohl, doch{ }ſcheint es jetzt
               viel beſſer oder ganz gut zu gehn. Schreiben Sie mir\strikeout{h}
               recht bald wieder, iſts kein Brief,{ }ſo{ }ſei es eine Karte. Aber verlieren wir uns
               keineswegs, auch nicht auf Tage, ganz aus den Augen.\pend
           
\pstart
           Ich hoffe Ihr Papa\pwindex{Hofmannsthal, Hugo August von 21.\,12.\,1841 Wien – 8.\,12.\,1915 ebd.@\textsc{Hofmannsthal, Hugo August von} (21.\,12.\,1841 Wien – 8.\,12.\,1915 ebd.), \emph{Bankdirektor}|pwv} iſt ganz
               geſund. Grüßen Sie ihn, Ihre Mama\pwindex{Hofmannsthal, Anna von 27.\,1.\,1849 Wien – 22.\,3.\,1904 Sanatorium Fürth@\textsc{Hofmannsthal, Anna von} (27.\,1.\,1849 Wien – 22.\,3.\,1904 Sanatorium Fürth)|pwv}, und {\pb}die Familie Speyer\pwindex{Speyer, Nanette 5.\,1.\,1846 Iserlohn – 15.\,1.\,1925 Wien@\textsc{Speyer, Nanette} (5.\,1.\,1846 Iserlohn – 15.\,1.\,1925 Wien)|pw}\pwindex{Speyer, Albert 8.\,4.\,1836 Breslau – 25.\,3.\,1905 Opatija@\textsc{Speyer, Albert} (8.\,4.\,1836 Breslau – 25.\,3.\,1905 Opatija), \emph{Industrieller}|pw} mehr oder weniger.\pend
           
\pstart
           Herzlichſt der Ihrige{\\[\baselineskip]}\spacefill\mbox{Arthur.}\pend
           \leftskip=0em{}
\pstart
           Benützen Sie nur meine Wien\oindex{Wien@\textbf{Wien}, \emph{Verwaltungsgebiet}|pw}er Adreſſe, das iſt am{ }ſicherſten. Ich habe vergeſſen, daſs ich Sie von der Schauſpielerin\pwindex{Schnitzler, Olga 17.\,1.\,1882 Wien – 13.\,1.\,1970 Lugano@\textsc{Schnitzler, Olga} (17.\,1.\,1882 Wien – 13.\,1.\,1970 Lugano), \emph{Schauspielerin, Sängerin}|pwv}{ }ſehr herzlich grüßen{ }ſoll.\pend
           \selectlanguage{ngerman}\endnumbering\briefempfaengerindex{Hofmannsthal, Hugo von@\textsc{Hofmannsthal, Hugo von}!zzzSchnitzler, Arthur@\emph{von Arthur Schnitzler}!1900-07-171@{17. 7. 1900}|)be}\mylabel{L01057h}  \newcommand{\dateiname}{L01057}\newcommand{\titel}{Arthur Schnitzler an Hugo von Hofmannsthal, 17. 7. 1900}\newcommand{\editorInnen}{Martin Anton Müller und Gerd-Hermann Susen}%% latex-leseansicht-abspann.tex
%% Abspann für die Leseansicht.
%% Der Schalter \ifkorrekturansicht ist bereits durch den Vorspann gesetzt.

%% latex-abspann.tex
%% Gemeinsamer Abspann für Korrekturansicht und Leseansicht.
%% Setzt den Schalter \ifkorrekturansicht voraus (gesetzt in den
%% einbindenden Dateien latex-korrekturansicht-abspann.tex bzw.
%% latex-leseansicht-abspann.tex).
%% ---------------------------------------------------------------

\normalsize

% Das esempio-Environment wird nur in der Leseansicht benötigt
\ifkorrekturansicht\else
\newenvironment{esempio}[3]%
{
    \vspace{1.5ex}
    \rlap{\underline{#1}}
    \par
    \setlength{\parindent}{0cm}
    \nopagebreak
    \leftskip=#2cm
    \rightskip=#3cm
}
{
    \par
}
\fi

\doendnotes{C}
\bigskip
\vfill

\clearpage

\footnotesize

\ifkorrekturansicht
  \lohead{\textsc{register}}
\fi

% theindex-Environment neu definieren ohne reledmac
\makeatletter
\renewenvironment{theindex}{%
  \ifkorrekturansicht
    \section*{\indexname}%
  \else
    \subsubsection*{Index der erwähnten Entitäten}%
  \fi
  \setlength{\parindent}{0pt}%
  \setlength{\parskip}{0pt plus 0.3pt}%
  \let\item\@idxitem
}{%
  \ifkorrekturansicht\clearpage\fi
}
\makeatother

\IfFileExists{\jobname-pw.ind}{\input{\jobname-pw.ind}}{}

% Quellenangabe nur in der Leseansicht
\ifkorrekturansicht\else
% Fallback-Definitionen, falls die .tex-Datei \titel etc. nicht gesetzt hat
\providecommand{\titel}{}
\providecommand{\editorInnen}{}
\providecommand{\dateiname}{\jobname}

\vspace{3cm}

\vfill

\footnotesize
\textsc{Quelle}: \titel. Herausgegeben von {\editorInnen}. In: \emph{Arthur Schnitzler: Briefwechsel mit Autorinnen und Autoren}.
 Digitale Edition, https://schnitzler-briefe.acdh.oeaw.ac.at/{\dateiname}.html (Stand \today)
\fi

\end{document}


