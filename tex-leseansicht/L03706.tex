%% latex-leseansicht-vorspann.tex
%% Vorspann für die Leseansicht.
%% Lädt die gemeinsame Datei latex-vorspann.tex mit nicht gesetztem Schalter.

\newif\ifkorrekturansicht
\korrekturansichtfalse

\input{../tex-inputs/latex-vorspann}


\section[Elsa Plessner an Arthur Schnitzler, 20. 11. 1896]{L03706 Elsa Plessner an Arthur Schnitzler, 20. 11. 1896}
\nopagebreak\mylabel{L03706v}
\rehead{ }\normalsize\beginnumbering\briefempfaengerindex{Schnitzler, Arthur@\textsc{Schnitzler, Arthur}!zzzPlessner, Elsa@\emph{von Elsa Plessner}!1896-11-201@{20. 11. 1896}|(be}
\toendnotes[C]{\smallbreak\pagebreak[2]}
\correspDesc{Versand  durch Elsa Plessner am 20. 11. 1896 in Meran
\newline{}Erhalt  durch Arthur Schnitzler im Zeitraum [21. 11. 1896 – 25. 11. 1896?] in Wien}\toendnotes[C]{\smallbreak}
\Standort{DLA, A:Schnitzler, HS.1985.1.419.}
\physDesc{Brief, 1 Blatt, 1 Seite, 299 Zeichen
\newline{}Handschrift: schwarze Tinte, lateinische Kurrent}\toendnotes[C]{\smallbreak}
\pstart
           {\pb}Meran, Pension Wolf\oindex{Hotel Meranerhof@\textbf{Hotel Meranerhof}, \emph{Hotel}|pw}, den 20. 11. 96\pend
           
\pstart{}Verehrter Herr Doctor!\pend\vspace{0.5em}
\pstart
           Besten Dank für liebenswürdige \label{K_L03706-1v}\edtext{Zeilen}{\lemma{\textnormal{\emph{Zeilen}}}\Cendnote{\textnormal{nicht überliefert}}}\label{K_L03706-1}.
                  Aus Berlin\oindex{Berlin@\textbf{Berlin}, \emph{Hauptstadt}|pw} noch immer nichts. Werde, Ihrem Rath
               gemäß, dort anfragen.\pend
           
\pstart
           Dank für die Ihre bei Dir. B{\dotsfour}\pwindex{Brahm, Otto 5.\,2.\,1856 Hamburg – 28.\,11.\,1912 Berlin@\textsc{Brahm, Otto} (5.\,2.\,1856 Hamburg – 28.\,11.\,1912 Berlin), \emph{Theaterleiter, Regisseur}|pwv} »\label{K_L03706-2v}\edtext{Gnade, dass Ihr Euch
               dessen erinnert\pwindex{Goethe, Johann Wolfgang von 28.\,8.\,1749 Frankfurt am Main – 22.\,3.\,1832 Weimar@\textsc{Goethe, Johann Wolfgang von} (28.\,8.\,1749 Frankfurt am Main – 22.\,3.\,1832 Weimar), \emph{Schriftsteller}!Egmont. Ein Trauerspiel in fünf Aufzügen@\strich\emph{Egmont. Ein Trauerspiel in fünf Aufzügen}|pwv}}{\lemma{\textnormal{\emph{Gnade, … erinnert}}}\Cendnote{\textnormal{Zitat aus 
                  \emph{Egmont}\pwindex{Goethe, Johann Wolfgang von 28.\,8.\,1749 Frankfurt am Main – 22.\,3.\,1832 Weimar@\textsc{Goethe, Johann Wolfgang von} (28.\,8.\,1749 Frankfurt am Main – 22.\,3.\,1832 Weimar), \emph{Schriftsteller}!Egmont. Ein Trauerspiel in fünf Aufzügen@\strich\emph{Egmont. Ein Trauerspiel in fünf Aufzügen}|pwk} von Goethe\pwindex{Goethe, Johann Wolfgang von 28.\,8.\,1749 Frankfurt am Main – 22.\,3.\,1832 Weimar@\textsc{Goethe, Johann Wolfgang von} (28.\,8.\,1749 Frankfurt am Main – 22.\,3.\,1832 Weimar), \emph{Schriftsteller}|pwk}}}}\label{K_L03706-2}«. Überhaupt – Engel!! – Alte
               Sache.\pend
           
\pstart
           Hochachtung und Grüße{\\[\baselineskip]}\spacefill\mbox{Elsa Plessner.}\pend
           \leftskip=0em{}\selectlanguage{ngerman}\endnumbering\briefempfaengerindex{Schnitzler, Arthur@\textsc{Schnitzler, Arthur}!zzzPlessner, Elsa@\emph{von Elsa Plessner}!1896-11-201@{20. 11. 1896}|)be}\mylabel{L03706h}  \newcommand{\dateiname}{L03706}\newcommand{\titel}{Elsa Plessner an Arthur Schnitzler, 20. 11. 1896}\newcommand{\editorInnen}{Selma Jahnke und Martin Anton Müller}%% latex-leseansicht-abspann.tex
%% Abspann für die Leseansicht.
%% Der Schalter \ifkorrekturansicht ist bereits durch den Vorspann gesetzt.

%% latex-abspann.tex
%% Gemeinsamer Abspann für Korrekturansicht und Leseansicht.
%% Setzt den Schalter \ifkorrekturansicht voraus (gesetzt in den
%% einbindenden Dateien latex-korrekturansicht-abspann.tex bzw.
%% latex-leseansicht-abspann.tex).
%% ---------------------------------------------------------------

\normalsize

% Das esempio-Environment wird nur in der Leseansicht benötigt
\ifkorrekturansicht\else
\newenvironment{esempio}[3]%
{
    \vspace{1.5ex}
    \rlap{\underline{#1}}
    \par
    \setlength{\parindent}{0cm}
    \nopagebreak
    \leftskip=#2cm
    \rightskip=#3cm
}
{
    \par
}
\fi

\doendnotes{C}
\bigskip
\vfill

\clearpage

\footnotesize

\ifkorrekturansicht
  \lohead{\textsc{register}}
\fi

% theindex-Environment neu definieren ohne reledmac
\makeatletter
\renewenvironment{theindex}{%
  \ifkorrekturansicht
    \section*{\indexname}%
  \else
    \subsubsection*{Index der erwähnten Entitäten}%
  \fi
  \setlength{\parindent}{0pt}%
  \setlength{\parskip}{0pt plus 0.3pt}%
  \let\item\@idxitem
}{%
  \ifkorrekturansicht\clearpage\fi
}
\makeatother

\IfFileExists{\jobname-pw.ind}{\input{\jobname-pw.ind}}{}

% Quellenangabe nur in der Leseansicht
\ifkorrekturansicht\else
% Fallback-Definitionen, falls die .tex-Datei \titel etc. nicht gesetzt hat
\providecommand{\titel}{}
\providecommand{\editorInnen}{}
\providecommand{\dateiname}{\jobname}

\vspace{3cm}

\vfill

\footnotesize
\textsc{Quelle}: \titel. Herausgegeben von {\editorInnen}. In: \emph{Arthur Schnitzler: Briefwechsel mit Autorinnen und Autoren}.
 Digitale Edition, https://schnitzler-briefe.acdh.oeaw.ac.at/{\dateiname}.html (Stand \today)
\fi

\end{document}


