%% latex-leseansicht-vorspann.tex
%% Vorspann für die Leseansicht.
%% Lädt die gemeinsame Datei latex-vorspann.tex mit nicht gesetztem Schalter.

\newif\ifkorrekturansicht
\korrekturansichtfalse

\input{../tex-inputs/latex-vorspann}


\section[ Arthur Schnitzler an Felix Salten, 2. 4. 1906]{L03003 Arthur Schnitzler an Felix Salten,  2. 4. 1906}
\nopagebreak\mylabel{L03003v}
\rehead{ }\normalsize\beginnumbering\briefempfaengerindex{Salten, Felix@\textsc{Salten, Felix}!zzzSchnitzler, Arthur@\emph{von Arthur Schnitzler}!1906-04-021@{2. 4. 1906}|(be}
\toendnotes[C]{\smallbreak\pagebreak[2]}
\correspDesc{Versand  durch Arthur Schnitzler am 2. 4. 1906 in Wien
\newline{}Erhalt  durch Felix Salten im Zeitraum [3. 4. 1906
                  – 6. 4. 1906?] in Berlin}\toendnotes[C]{\smallbreak}
\Standort{Wienbibliothek im Rathaus, ZPH 1681, 2.1.516.}
\physDesc{Brief, 2 Blätter, 8 Seiten, 3570 Zeichen
\newline{}Handschrift: schwarze Tinte, deutsche Kurrent
\newline{}Ordnung: mit Bleistift von unbekannter Hand Nummerierung der Doppelseiten des
                                 Konvoluts: »20«–»23«  }\toendnotes[C]{\smallbreak}
\pstart
           {\pb}\textcolor{gray}{\textbf{Dr. Arthur Schnitzler}}\hfill 2. April 906\pend
           
\pstart
           \textcolor{gray}{\textbf{Wien, XVIII. Spoettelgasse 7\oindex{Wien@\textbf{Wien}!XVIII., Währing@\textbf{XVIII., Währing}!Edmund-Weiß-Gasse 7@\textbf{Edmund-Weiß-Gasse 7}, \emph{Wohngebäude}|pw}.}}\pend
           \vspace{0.5em}
\pstart
           lieber, vor einigen Wochen{ }ſchrieb mir Liesl\pwindex{Steinrück, Elisabeth 19.\,11.\,1885 – 7.\,4.\,1920 Partenkirchen@\textsc{Steinrück, Elisabeth} (19.\,11.\,1885 – 7.\,4.\,1920 Partenkirchen)|pw}, daſs ihr ein Bekannter, namens Engel\pwindex{Engel @\textsc{Engel}|pw}, eine ermäßigte \label{K_L03003-1v}\edtext{Seereiſe}{\lemma{\textnormal{\emph{Seereise}}}\Cendnote{\textnormal{Siehe XXXX Auszeichnungsfehler: Dokument L03416 nicht gefunden.
               }}}\label{K_L03003-1} verſchaffen werde; daſs{ }ſie{ }ſich nun in dieſer Sache an Sie zu wenden{ }ſcheint
               (wie mir Ihr letzter Brief andeutet) iſt mir\textcolor{gray}{,} wie Sie{ }ſich
               denken können,{ }ſo wenig recht als möglich. – Meinen \label{K_L03003-2v}\edtext{begeiſterten Brief}{\lemma{\textnormal{\emph{begeisterten Brief}}}\Cendnote{\textnormal{Siehe XXXX Auszeichnungsfehler: Dokument L03416 nicht gefunden.
               }}}\label{K_L03003-2} an Trebitſch\pwindex{Trebitsch, Siegfried 22.\,12.\,1868 Wien – 3.\,6.\,1956 Zürich@\textsc{Trebitsch, Siegfried} (22.\,12.\,1868 Wien – 3.\,6.\,1956 Zürich), \emph{Schriftsteller, Übersetzer}|pw} kö{\geminationn}en Sie{ }ſich ja ungefähr vorſtellen. Er{ }ſchrieb mir
               gleich nach Erſcheinen jenes Artikels\pwindex{Trebitsch, Siegfried 22.\,12.\,1868 Wien – 3.\,6.\,1956 Zürich@\textsc{Trebitsch, Siegfried} (22.\,12.\,1868 Wien – 3.\,6.\,1956 Zürich), \emph{Schriftsteller, Übersetzer}!Bühnenvertrieb@\strich\emph{Bühnenvertrieb}|pwv} in der Schb.\pwindex{Schaubühne@\emph{Die Schaubühne}|pw} ich{ }ſolle ihn
               »beruhigen«. Ich hab {\pb}ihn beruhigt. Im
               übrigen hat die Bühnenvertriebsſache{ }ſchon \substVorne{}\textsuperscript{I}\substDazwischen{}i\substHinten{}hre Bedeutung. Nur muſs{ }ſie in Zuſa{\geminationm}enhang mit
               den andern Fragen behandelt werden, die{ }ſich auf das Verhältnis des Autors zu{ }ſeiner
               geſchäftl. Umwelt beziehen. Einige dieſer Fragen hab ich \label{K_L03003-3v}\edtext{in einem Brief\pwindex{Bund der Bühnendichter. II@\emph{Bund der Bühnendichter. II}|pwv}}{\lemma{\textnormal{\emph{in einem Brief}}}\Cendnote{\textnormal{\emph{Bund der Bühnendichter. II}\pwindex{Bund der Bühnendichter. II@\emph{Bund der Bühnendichter. II}|pwk}. In: \emph{Die Schaubühne}\pwindex{Schaubühne@\emph{Die Schaubühne}|pwk}, Jg. 33, Nr. 11.176, 12. 4. 1906, S. 10. Wiedergabe in A. S.: \emph{»Das Zeitlose ist von kürzester Dauer«}, Bund der Bühnendichter, 12. 4. 1906.}}}\label{K_L03003-3} an Jacobsohn\pwindex{Jacobsohn, Siegfried 28.\,1.\,1881 Berlin – 3.\,12.\,1926 ebd.@\textsc{Jacobsohn, Siegfried} (28.\,1.\,1881 Berlin – 3.\,12.\,1926 ebd.), \emph{Journalist, Kritiker, Publizist}|pw} kurz formulirt. –\pend
           
\pstart
           Nun unſre Radreiſe »oder{ }ſo«. Wenn Sie irgendwas deutſch\oindex{Deutschland@\textbf{Deutschland}|pwv}es, Thüringer Harz\oindex{Harz@\textbf{Harz}, \emph{Gebirge}|pw}{ }\textsc{etc} vorziehen,{ }ſo möchte ich dieſe Reiſe mehr gegen den
                  So{\geminationm}er verſchieben, etwa gegen Mitte Juli, um dann gleich das Seebad\oindex{Marienlyst@\textbf{Marienlyst}, \emph{Gut}|pwv} an{\pb}ſchließen zu
               können. Ziehen Sie Tirol\oindex{Tirol@\textbf{Tirol}, \emph{Land}|pw}\oindex{Südtirol@\textbf{Südtirol}, \emph{Verwaltungsgebiet}|pw} ev. Salzka{\geminationm}ergut\oindex{Salzkammergut@\textbf{Salzkammergut}, \emph{Region}|pw}, (bayriſches Hochgebirge\oindex{Bayern@\textbf{Bayern}, \emph{Land}|pw}?) vor,{ }ſo{ }ſchlage ich erſte
               Hälfte Juni vor. Geht Ihre Frau\pwindex{Salten, Ottilie 7.\,3.\,1868 Prag – 22.\,6.\,1942 Zürich@\textsc{Salten, Ottilie} (7.\,3.\,1868 Prag – 22.\,6.\,1942 Zürich), \emph{Schauspielerin}|pwv} mit,{ }ſo käme die meine\pwindex{Schnitzler, Olga 17.\,1.\,1882 Wien – 13.\,1.\,1970 Lugano@\textsc{Schnitzler, Olga} (17.\,1.\,1882 Wien – 13.\,1.\,1970 Lugano), \emph{Schauspielerin, Sängerin}|pwv} auch, und wir würden da{\geminationn} mehr eine Radialradpartie machen, d. h. allerlei
               Fahrten, mit feſtem Stützpunkt. – Ko{\geminationm}t Otti\pwindex{Salten, Ottilie 7.\,3.\,1868 Prag – 22.\,6.\,1942 Zürich@\textsc{Salten, Ottilie} (7.\,3.\,1868 Prag – 22.\,6.\,1942 Zürich), \emph{Schauspielerin}|pw} nicht,{ }ſo{ }ſoll es eine Längspartie werden, »wie einſt im
               Mai«, (we{\geminationn} Sie uns jetzt als Julier, \textsc{resp.} Auguſtiner (Sie \introOben{}Anfang\introOben{} Julier
               und ich Endauguſtiner anſprechen.). Gar zu weite Bahnreiſe (Genf\oindex{Genf@\textbf{Genf}|pw}, Lugano\oindex{Lugano@\textbf{Lugano}, \emph{Hauptstadt}|pw}) möcht ich
               gern vermeiden, {\pb}aus 17 Gründen. – Von meiner
                  daen\oindex{Dänemark@\textbf{Dänemark}|pwv}iſchen Idee, lieber,
               werd ich{ }ſchwer abzubringen{ }ſein. Hingegen habe {[}ich{]} folgendes zu
               bemerken. Wenn Sie auf einige Wochen an die See\oindex{Ostsee@\textbf{Ostsee}|pwv} gehn, kann Ihnen doch auch die um ein paar Stunden
               verlängerte Reiſe nicht ankommen. Ko{\geminationm}en Sie aber immer
               nur auf 24 Stunden ans Ufer,{ }ſo hab ich ohnedies{ }ſehr wenig, \textsc{resp\textcolor{gray}{.}} zu wenig von Ihnen. Alles, was ich von deutſch\oindex{Deutschland@\textbf{Deutschland}|pwv}en Seebädern höre, ni{\geminationm}t
               mich dagegen ein; die bekannten {\pb}ſind in
               Hinſicht auf Publikum \textsc{etc} berüchtigt, die unbekannten{ }ſollen was Comfor\strikeout{\textcolor{gray}{×}}t \textsc{etc} anbelangt übel ausſehen. Wälder gibts nur auf
                  Rügen\oindex{Rügen@\textbf{Rügen}, \emph{Insel}|pw}. Daenemark\oindex{Dänemark@\textbf{Dänemark}|pw} ke{\geminationn} ich. Seit ich \label{K_L03003-4v}\edtext{dort geweſen}{\lemma{\textnormal{\emph{dort gewesen}}}\Cendnote{\textnormal{im Sommer 1896}}}\label{K_L03003-4} bin,{ }ſehn ich mich zurück. Die Menſchen dort (die man ja nicht kennt), der
               Himmel, die Wälder, allerlei undefinirbares iſt in der Erinnerung für mich von einem
               wahren Zauber umgeben. Auch denk ich lebhaft an einen \label{K_L03003-5v}\edtext{Abſtecher nach Schweden\oindex{Schweden@\textbf{Schweden}|pw}, ev Norwegen\oindex{Norwegen@\textbf{Norwegen}|pw}}{\lemma{\textnormal{\emph{Abstecher … Norwegen}}}\Cendnote{\textnormal{Dazu kam es nicht.}}}\label{K_L03003-5}. Wir wollen
                  \label{K_L03003-6v}\edtext{auf 2, 3 {\pb}Tage nach Kopenhagen\oindex{Kopenhagen@\textbf{Kopenhagen}, \emph{Hauptstadt}|pw}}{\lemma{\textnormal{\emph{auf 2, 3 Tage nach Kopenhagen}}}\Cendnote{\textnormal{Schnitzler war vor seinem Aufenthalt in Marienlyst\oindex{Marienlyst@\textbf{Marienlyst}, \emph{Gut}|pwk} nur am 28. 6. 1906 in Kopenhagen\oindex{Kopenhagen@\textbf{Kopenhagen}, \emph{Hauptstadt}|pwk}.}}}\label{K_L03003-6}, von dort aus inſpizire
               ich die Seeſeite\oindex{Öresund@\textbf{Öresund}|pw} nach geeignetem
               Aufenthalt. –\pend
           
\pstart
           Schönen Dank für die noch{ }ſchönern \label{K_L03003-7v}\edtext{Feu{[}i{]}lletons\pwindex{Salten, Felix 6.\,9.\,1869 Budapest – 8.\,10.\,1945 Zürich@\textsc{Salten, Felix} (6.\,9.\,1869 Budapest – 8.\,10.\,1945 Zürich), \emph{Schriftsteller, Journalist, Chefredakteur}!Kater Lampe«@\strich\emph{»Kater Lampe«}|pwv}\pwindex{Salten, Felix 6.\,9.\,1869 Budapest – 8.\,10.\,1945 Zürich@\textsc{Salten, Felix} (6.\,9.\,1869 Budapest – 8.\,10.\,1945 Zürich), \emph{Schriftsteller, Journalist, Chefredakteur}!Russisches Theater. II@\strich\emph{Russisches Theater. II}|pwv}}{\lemma{\textnormal{\emph{Feuilletons}}}\Cendnote{\textnormal{Felix Salten\pwindex{Salten, Felix 6.\,9.\,1869 Budapest – 8.\,10.\,1945 Zürich@\textsc{Salten, Felix} (6.\,9.\,1869 Budapest – 8.\,10.\,1945 Zürich), \emph{Schriftsteller, Journalist, Chefredakteur}|pwk}: \emph{Russisches Theater. II}\pwindex{Salten, Felix 6.\,9.\,1869 Budapest – 8.\,10.\,1945 Zürich@\textsc{Salten, Felix} (6.\,9.\,1869 Budapest – 8.\,10.\,1945 Zürich), \emph{Schriftsteller, Journalist, Chefredakteur}!Russisches Theater. II@\strich\emph{Russisches Theater. II}|pwk}. In: \emph{B. Z. am Mittag}\pwindex{B.Z. am Mittag@\emph{B.Z. am Mittag}|pwk}, Jg. 30, Nr. 70, 23. 3. 1906, S. 2–3; Felix Salten\pwindex{Salten, Felix 6.\,9.\,1869 Budapest – 8.\,10.\,1945 Zürich@\textsc{Salten, Felix} (6.\,9.\,1869 Budapest – 8.\,10.\,1945 Zürich), \emph{Schriftsteller, Journalist, Chefredakteur}|pwk}: \emph{»Kater Lampe«}\pwindex{Salten, Felix 6.\,9.\,1869 Budapest – 8.\,10.\,1945 Zürich@\textsc{Salten, Felix} (6.\,9.\,1869 Budapest – 8.\,10.\,1945 Zürich), \emph{Schriftsteller, Journalist, Chefredakteur}!Kater Lampe«@\strich\emph{»Kater Lampe«}|pwk}. In: \emph{B. Z. am Mittag}\pwindex{B.Z. am Mittag@\emph{B.Z. am Mittag}|pwk}, Jg. 30,
                     Nr. 72, 26. 3. 1906, S. 2.}}}\label{K_L03003-7}, Rußland\pwindex{Salten, Felix 6.\,9.\,1869 Budapest – 8.\,10.\,1945 Zürich@\textsc{Salten, Felix} (6.\,9.\,1869 Budapest – 8.\,10.\,1945 Zürich), \emph{Schriftsteller, Journalist, Chefredakteur}!Russisches Theater. II@\strich\emph{Russisches Theater. II}|pwv}\oindex{Russland@\textbf{Russland}|pw} und Lampe\pwindex{Salten, Felix 6.\,9.\,1869 Budapest – 8.\,10.\,1945 Zürich@\textsc{Salten, Felix} (6.\,9.\,1869 Budapest – 8.\,10.\,1945 Zürich), \emph{Schriftsteller, Journalist, Chefredakteur}!Kater Lampe«@\strich\emph{»Kater Lampe«}|pwv} betreffend.
               Sie \uline{haben}{ }ſich halt immer. Wenn Sie mit{ }ſich{ }ſelber
               raufen, bleiben Sie doch auf immer der Gewinner. Ich ko{\geminationm}
               zu oft gegen mich nicht auf. – Immerhin, ich arbeite jetzt. Sie{ }ſind{ }ſchon alle wieder
               da, die Geſtältchen und Geſtalten, – aber mit meiner Macht über{ }ſie{ }ſiehts noch
               ziemlich {\pb}flau aus. – Komiſch, ja{ }ſogar ein
               wenig traurig waren manche Kritiken über den Wurſtelſpaſs\pwindex{Schnitzler, Arthur 15.\,5.\,1862 Wien – 21.\,10.\,1931 ebd.@\textsc{Schnitzler, Arthur} (15.\,5.\,1862 Wien – 21.\,10.\,1931 ebd.), \emph{Schriftsteller, Mediziner}!Zum großen Wurstel. Burleske in einem Akt@\strich\emph{Zum großen Wurstel. Burleske in einem Akt}|pwv}. Es wurde mir \label{K_L03003-8v}\edtext{ſo anerkennend vermerkt, daſs mir endgiltig mies vor mir
               geworden zu{ }ſein{ }ſcheint}{\lemma{\textnormal{\emph{so … scheint}}}\Cendnote{\textnormal{Siehe A. S.: \emph{Tagebuch}, 27. 3. 1906.
               }}}\label{K_L03003-8}. Ja, »\label{K_L03003-9v}\edtext{Nordpol\oindex{Nordpol@\textbf{Nordpol}, \emph{Landzunge}|pw}fahrer müſte man{ }ſein\pwindex{Schnitzler, Arthur 15.\,5.\,1862 Wien – 21.\,10.\,1931 ebd.@\textsc{Schnitzler, Arthur} (15.\,5.\,1862 Wien – 21.\,10.\,1931 ebd.), \emph{Schriftsteller, Mediziner}!letzten Masken@\strich\emph{Die letzten Masken}|pwv}}{\lemma{\textnormal{\emph{Nordpolfahrer … sein}}}\Cendnote{\textnormal{Schnitzler paraphrasierte \emph{Die letzten Masken}\pwindex{Schnitzler, Arthur 15.\,5.\,1862 Wien – 21.\,10.\,1931 ebd.@\textsc{Schnitzler, Arthur} (15.\,5.\,1862 Wien – 21.\,10.\,1931 ebd.), \emph{Schriftsteller, Mediziner}!letzten Masken@\strich\emph{Die letzten Masken}|pwk}. Dort heißt es wörtlich: »Ein Bauer auf dem Land möcht
                        ich sein, ein Schafhirt, ein Nordpol\oindex{Nordpol@\textbf{Nordpol}, \emph{Landzunge}|pw}fahrer – ah, was du willst! –\pwindex{Schnitzler, Arthur 15.\,5.\,1862 Wien – 21.\,10.\,1931 ebd.@\textsc{Schnitzler, Arthur} (15.\,5.\,1862 Wien – 21.\,10.\,1931 ebd.), \emph{Schriftsteller, Mediziner}!letzten Masken@\strich\emph{Die letzten Masken}|pwv}«}}}\label{K_L03003-9}«{ }ſagt Weihgaſt\pwindex{Schnitzler, Arthur 15.\,5.\,1862 Wien – 21.\,10.\,1931 ebd.@\textsc{Schnitzler, Arthur} (15.\,5.\,1862 Wien – 21.\,10.\,1931 ebd.), \emph{Schriftsteller, Mediziner}!letzten Masken@\strich\emph{Die letzten Masken}|pwv}, mit dem mich{ }ſonſt nur geringe Sympathie \strikeout{\textcolor{gray}{bef}} verbindet. – \label{K_L03003-10v}\edtext{Kerr\pwindex{Kerr, Alfred 25.\,12.\,1867 Breslau – 12.\,10.\,1948 Hamburg@\textsc{Kerr, Alfred} (25.\,12.\,1867 Breslau – 12.\,10.\,1948 Hamburg), \emph{Schriftsteller, Kritiker}!Ödipus und der Ruf des Lebens@\strich\emph{Ödipus und der Ruf des Lebens}|pwv}\pwindex{Kerr, Alfred 25.\,12.\,1867 Breslau – 12.\,10.\,1948 Hamburg@\textsc{Kerr, Alfred} (25.\,12.\,1867 Breslau – 12.\,10.\,1948 Hamburg), \emph{Schriftsteller, Kritiker}|pw}}{\lemma{\textnormal{\emph{Kerr}}}\Cendnote{\textnormal{Alfred Kerr\pwindex{Kerr, Alfred 25.\,12.\,1867 Breslau – 12.\,10.\,1948 Hamburg@\textsc{Kerr, Alfred} (25.\,12.\,1867 Breslau – 12.\,10.\,1948 Hamburg), \emph{Schriftsteller, Kritiker}|pwk}: \emph{Ödipus und der Ruf des Lebens}\pwindex{Kerr, Alfred 25.\,12.\,1867 Breslau – 12.\,10.\,1948 Hamburg@\textsc{Kerr, Alfred} (25.\,12.\,1867 Breslau – 12.\,10.\,1948 Hamburg), \emph{Schriftsteller, Kritiker}!Ödipus und der Ruf des Lebens@\strich\emph{Ödipus und der Ruf des Lebens}|pwk}. In: \emph{Die neue Rundschau}\pwindex{neue Rundschau@\emph{Die neue Rundschau}|pwk}, Jg. 17, H. 5, Mai 1906, S. 492–498. Siehe A. S.: \emph{Tagebuch}, 30. 3. 1906.}}}\label{K_L03003-10} hab ich eigentlich, innerlich, (das
               innerlich bezieht{ }ſich auf ihn), charmant gefunden{\dots} Wiſſen
               Sie um wen es mir eigentlich am leideſten thut? Um die gute {\pb}Katharina\pwindex{Schnitzler, Arthur 15.\,5.\,1862 Wien – 21.\,10.\,1931 ebd.@\textsc{Schnitzler, Arthur} (15.\,5.\,1862 Wien – 21.\,10.\,1931 ebd.), \emph{Schriftsteller, Mediziner}!Ruf des Lebens. Schauspiel in drei Akten@\strich\emph{Der Ruf des Lebens. Schauspiel in drei Akten}|pwv}, die als Ophelia\pwindex{\textcolor{red}{\textsuperscript{XXXX indx1}}!Hamlet@\strich\emph{Hamlet}|pwv}{ }\substVorne{}\textsuperscript{,}\substDazwischen{}(\substHinten{}ja wär ich Julius Bauer\pwindex{Bauer, Julius 15.\,10.\,1853 Szigetvár – 11.\,6.\,1941 Wien@\textsc{Bauer, Julius} (15.\,10.\,1853 Szigetvár – 11.\,6.\,1941 Wien), \emph{Schriftsteller, Journalist, Kritiker}|pw}{ }ſo{ }ſagt ich: als
               Pophelia) behandelt wird, – weil Frl Hofmann\pwindex{Hofmann, Grete @\textsc{Hofmann, Grete}, \emph{Schauspielerin}|pw}
               im letzten Akt\pwindex{Schnitzler, Arthur 15.\,5.\,1862 Wien – 21.\,10.\,1931 ebd.@\textsc{Schnitzler, Arthur} (15.\,5.\,1862 Wien – 21.\,10.\,1931 ebd.), \emph{Schriftsteller, Mediziner}!Ruf des Lebens. Schauspiel in drei Akten@\strich\emph{Der Ruf des Lebens. Schauspiel in drei Akten}|pwv} Blumen im Haar
               hatte. Als abſichtlich von mir aus Hamlet\pwindex{\textcolor{red}{\textsuperscript{XXXX indx1}}!Hamlet@\strich\emph{Hamlet}|pw}
               herausgeſtohlene Ophelia\pwindex{\textcolor{red}{\textsuperscript{XXXX indx1}}!Hamlet@\strich\emph{Hamlet}|pwv}.
               Einer wie der andre. –\pend
           
\pstart
           \label{K_L03003-11v}\edtext{Neulich im Coloſſeum\oindex{Wien@\textbf{Wien}!XV., Rudolfsheim-Fünfhaus@\textbf{XV., Rudolfsheim-Fünfhaus}!Volkstheater in Rudolfsheim@\textbf{Volkstheater in Rudolfsheim}, \emph{Theater}|pw}}{\lemma{\textnormal{\emph{Neulich im Colosseum}}}\Cendnote{\textnormal{Siehe A. S.: \emph{Tagebuch}, 28. 3. 1906.
               }}}\label{K_L03003-11}; mit Wasserma{\geminationn}s\pwindex{Wassermann, Jakob 10.\,3.\,1873 Fürth – 1.\,1.\,1934 Altaussee@\textsc{Wassermann, Jakob} (10.\,3.\,1873 Fürth – 1.\,1.\,1934 Altaussee), \emph{Schriftsteller}|pw}\pwindex{Wassermann, Julie 5.\,12.\,1876 Wien – April 1963 Zürich@\textsc{Wassermann, Julie} (5.\,12.\,1876 Wien – April 1963 Zürich), \emph{Schriftstellerin}|pw} u. Kaufmann\pwindex{Kaufmann, Arthur 4.\,4.\,1872 Iași – 25.\,7.\,1938 Wien@\textsc{Kaufmann, Arthur} (4.\,4.\,1872 Iași – 25.\,7.\,1938 Wien), \emph{Rechtswissenschaftler, Privatgelehrte, Privatier}|pw}. Zwei Clowns als Nachtigallen den Unvergeßlichkeiten anzureihn.\pend
           
\pstart
           Grüß Sie Gott. Herzlichſt Ihr {\\[\baselineskip]}\spacefill\mbox{A.}\pend
           \leftskip=0em{}\selectlanguage{ngerman}\endnumbering\briefempfaengerindex{Salten, Felix@\textsc{Salten, Felix}!zzzSchnitzler, Arthur@\emph{von Arthur Schnitzler}!1906-04-021@{2. 4. 1906}|)be}\mylabel{L03003h}  \newcommand{\dateiname}{L03003}\newcommand{\titel}{Arthur Schnitzler an Felix Salten, 2. 4. 1906}\newcommand{\editorInnen}{Martin Anton Müller und Laura Untner}%% latex-leseansicht-abspann.tex
%% Abspann für die Leseansicht.
%% Der Schalter \ifkorrekturansicht ist bereits durch den Vorspann gesetzt.

%% latex-abspann.tex
%% Gemeinsamer Abspann für Korrekturansicht und Leseansicht.
%% Setzt den Schalter \ifkorrekturansicht voraus (gesetzt in den
%% einbindenden Dateien latex-korrekturansicht-abspann.tex bzw.
%% latex-leseansicht-abspann.tex).
%% ---------------------------------------------------------------

\normalsize

% Das esempio-Environment wird nur in der Leseansicht benötigt
\ifkorrekturansicht\else
\newenvironment{esempio}[3]%
{
    \vspace{1.5ex}
    \rlap{\underline{#1}}
    \par
    \setlength{\parindent}{0cm}
    \nopagebreak
    \leftskip=#2cm
    \rightskip=#3cm
}
{
    \par
}
\fi

\doendnotes{C}
\bigskip
\vfill

\clearpage

\footnotesize

\ifkorrekturansicht
  \lohead{\textsc{register}}
\fi

% theindex-Environment neu definieren ohne reledmac
\makeatletter
\renewenvironment{theindex}{%
  \ifkorrekturansicht
    \section*{\indexname}%
  \else
    \subsubsection*{Index der erwähnten Entitäten}%
  \fi
  \setlength{\parindent}{0pt}%
  \setlength{\parskip}{0pt plus 0.3pt}%
  \let\item\@idxitem
}{%
  \ifkorrekturansicht\clearpage\fi
}
\makeatother

\IfFileExists{\jobname-pw.ind}{\input{\jobname-pw.ind}}{}

% Quellenangabe nur in der Leseansicht
\ifkorrekturansicht\else
% Fallback-Definitionen, falls die .tex-Datei \titel etc. nicht gesetzt hat
\providecommand{\titel}{}
\providecommand{\editorInnen}{}
\providecommand{\dateiname}{\jobname}

\vspace{3cm}

\vfill

\footnotesize
\textsc{Quelle}: \titel. Herausgegeben von {\editorInnen}. In: \emph{Arthur Schnitzler: Briefwechsel mit Autorinnen und Autoren}.
 Digitale Edition, https://schnitzler-briefe.acdh.oeaw.ac.at/{\dateiname}.html (Stand \today)
\fi

\end{document}


