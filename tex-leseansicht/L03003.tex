%% latex-leseansicht-vorspann.tex
%% Vorspann für die Leseansicht.
%% Lädt die gemeinsame Datei latex-vorspann.tex mit nicht gesetztem Schalter.

\newif\ifkorrekturansicht
\korrekturansichtfalse

\input{../tex-inputs/latex-vorspann}


         
         \renewcommand{\erwaehntePersonen}{Personen: Julius Bauer,  Engel, Grete Hofmann, Siegfried Jacobsohn, Arthur Kaufmann, Alfred Kerr, Felix Salten, Ottilie Salten, Olga Schnitzler, Elisabeth Steinrück, Siegfried Trebitsch, Jakob Wassermann, Julie Wassermann}
         \renewcommand{\erwaehnteOrte}{Orte: Bayern, Berlin, Deutschland, Dänemark, Edmund-Weiß-Gasse 7, Genf, Harz, Kopenhagen, Lugano, Marienlyst, Nordpol, Norwegen, Ostsee, Russland, Rügen, Salzkammergut, Schweden, Südtirol, Tirol, Volkstheater in Rudolfsheim, Wien, Öresund}
         \renewcommand{\erwaehnteWerke}{Werke: B.Z. am Mittag, Bund der Bühnendichter. II, Bühnenvertrieb, Der Ruf des Lebens. Schauspiel in drei Akten, Die Schaubühne, Die letzten Masken, Die neue Rundschau, Hamlet, Russisches Theater. II, Zum großen Wurstel. Burleske in einem Akt, »Kater Lampe«, Ödipus und der Ruf des Lebens}
               \section[ Arthur Schnitzler an Felix Salten, 2. 4. 1906]{ Arthur Schnitzler an Felix Salten, 2. 4. 1906}\nopagebreak\mylabel{v}\rehead{ }\begin{ledgroupsized}[t]{13cm}\normalsize\beginnumbering\briefempfaengerindex{Salten, Felix@\textsc{Salten, Felix}!zzzSchnitzler, Arthur@\emph{von Arthur Schnitzler}!1906-04-021@{2. 4. 1906}|(be} \toendnotes[C]{\smallbreak\pagebreak[2]} \Standort{Wienbibliothek im Rathaus, ZPH 1681, 2.1.516.}
\physDesc{Brief, 2 Blätter, 8 Seiten, 3559 Zeichen
\newline{}Handschrift: schwarze Tinte, deutsche Kurrent
\newline{}Ordnung: mit Bleistift von unbekannter Hand Nummerierung der Doppelseiten des
                                 Konvoluts: »20«–»23«  }\toendnotes[C]{\smallbreak}\pstart
           \noindent{}{\pb}\textcolor{gray}{\textbf{Dr. Arthur Schnitzler}}\hfill 2. April 906\pend
           \pstart
           \textcolor{gray}{\textbf{Wien, XVIII. Spoettelgasse 7\oindex{Edmund-Weiss-Gasse 7@\textbf{Edmund-Weiß-Gasse 7}|pw}.}}\pend
           \pstart
           lieber, vor einigen Wochen ſchrieb mir Liesl\pwindex{Steinrueck, Elisabeth 19.11.1885 – 07.04.1920@\textsc{Steinrück, Elisabeth} (19.11.1885 – 07.04.1920)|pw}, daſs ihr ein Bekannter, namens Engel\pwindex{Engel @\textsc{Engel}|pw}, eine ermäßigte \label{K_L03003-1v}\edtext{Seereiſe}{\lemma{\textnormal{\emph{Seereiſe}}}\Cendnote{\textnormal{siehe Felix Salten an Arthur Schnitzler, 28. 3. 1906}}}\label{K_L03003-1h} verſchaffen werde; daſs ſie ſich nun in dieſer Sache an Sie zu wenden ſcheint
               (wie mir Ihr letzter Brief andeutet) iſt mir\textcolor{gray}{,} wie Sie ſich
               denken können, ſo wenig recht als möglich. – Meinen \label{K_L03003-2v}\edtext{begeiſterten Brief}{\lemma{\textnormal{\emph{begeiſterten Brief}}}\Cendnote{\textnormal{siehe Felix Salten an Arthur Schnitzler, 28. 3. 1906}}}\label{K_L03003-2h} an Trebitſch\pwindex{Trebitsch, Siegfried 22.12.1868 – 03.06.1956@\textsc{Trebitsch, Siegfried} (22.12.1868 – 03.06.1956), \emph{Schriftsteller, Übersetzer}|pw} kö{\geminationn}en Sie ſich ja ungefähr vorſtellen. Er ſchrieb mir
               gleich nach Erſcheinen jenes Artikel\pwindex{Trebitsch, Siegfried 22.12.1868 – 03.06.1956@\textsc{Trebitsch, Siegfried} (22.12.1868 – 03.06.1956), \emph{Schriftsteller, Übersetzer}!Buehnenvertrieb22. 03. 1906@\strich\emph{Bühnenvertrieb} {[}22. 03. 1906{]}|pwv}s in der Schb.\pwindex{Schaubuehne7.9.1905 – 1993@\emph{Die Schaubühne} {[}7.9.1905 – 1993{]}|pw} ich ſolle ihn
               »beruhigen«. Ich hab {\pb}ihn beruhigt. Im
               übrigen hat die Bühnenvertriebsſache ſchon \substVorne{}\textsuperscript{I}\substDazwischen{}i\substHinten{}hre Bedeutung. Nur muſs ſie in Zuſa{\geminationm}enhang mit
               den andern Fragen behandelt werden, die ſich auf das Verhältnis des Autors zu ſeiner
               geſchäftl. Umwelt beziehen. Einige dieſer Fragen hab ich \label{K_L03003-3v}\edtext{in einem Brief\pwindex{Bund der Buehnendichter. II1906-04-12@\emph{Bund der Bühnendichter. II} {[}1906-04-12{]}|pwv}}{\lemma{\textnormal{\emph{in einem Brief}}}\Cendnote{\textnormal{\emph{Bund der Bühnendichter. II}\pwindex{Bund der Buehnendichter. II1906-04-12@\emph{Bund der Bühnendichter. II} {[}1906-04-12{]}|pwk} In: \emph{Die Schaubühne}\pwindex{Schaubuehne7.9.1905 – 1993@\emph{Die Schaubühne} {[}7.9.1905 – 1993{]}|pwk}, Jg. 33, Nr. 11.176, 12. 4. 1906, S. 10. Wiedergabe in A. S.: \emph{»Das Zeitlose ist von kürzester Dauer«}, Bund der Bühnendichter, 12. 4. 1906.}}}\label{K_L03003-3h} an Jacobsohn\pwindex{Jacobsohn, Siegfried 28.01.1881 – 03.12.1926@\textsc{Jacobsohn, Siegfried} (28.01.1881 – 03.12.1926), \emph{Journalist, Kritiker, Publizist}|pw} kurz formulirt. –\pend
           \pstart
           Nun unſre Radreiſe »oder ſo«. Wenn Sie irgendwas deutſch\oindex{Deutschland@\textbf{Deutschland}|pwv}es, Thüringer Harz\oindex{Harz@\textbf{Harz}|pw}{ }\textsc{etc} vorziehen, ſo möchte ich dieſe Reiſe mehr gegen den
                  So{\geminationm}er verſchieben, etwa gegen Mitte Juli, um dann gleich das Seebad\oindex{Marienlyst@\textbf{Marienlyst}|pwv} an{\pb}ſchließen zu
               können. Ziehen Sie Tirol\oindex{Tirol@\textbf{Tirol}|pw}\oindex{Suedtirol@\textbf{Südtirol}|pw} ev. Salzka{\geminationm}ergut\oindex{Salzkammergut@\textbf{Salzkammergut}|pw}, (bayriſches Hochgebirge\oindex{Bayern@\textbf{Bayern}|pw}?) vor, ſo ſchlage ich erſte
               Hälfte Juni vor. Geht Ihre Frau\pwindex{Salten, Ottilie 07.03.1868 – 22.06.1942@\textsc{Salten, Ottilie} (07.03.1868 – 22.06.1942), \emph{Schauspielerin}|pwv} mit, ſo käme die meine\pwindex{Schnitzler, Olga 17.01.1882 – 13.01.1970@\textsc{Schnitzler, Olga} (17.01.1882 – 13.01.1970), \emph{Schauspielerin, Sängerin}|pwv} auch, und wir würden da{\geminationn} mehr eine Radialradpartie machen, d. h. allerlei
               Fahrten, mit feſtem Stützpunkt. – Ko{\geminationm}t Otti\pwindex{Salten, Ottilie 07.03.1868 – 22.06.1942@\textsc{Salten, Ottilie} (07.03.1868 – 22.06.1942), \emph{Schauspielerin}|pw} nicht, ſo ſoll es eine Längspartie werden, »wie einſt im
               Mai«, (we{\geminationn} Sie uns jetzt als Julier, \textsc{resp.} Auguſtiner (Sie \introOben{}Anfang\introOben{} Julier
               und ich Endauguſtiner anſprechen.). Gar zu weite Bahnreiſe (Genf\oindex{Genf@\textbf{Genf}|pw}, Lugano\oindex{Lugano@\textbf{Lugano}|pw}) möcht ich
               gern vermeiden, {\pb}aus 17 Gründen. – Von meiner
                  daen\oindex{Daenemark@\textbf{Dänemark}|pwv}iſchen Idee, lieber,
               werd ich ſchwer abzubringen ſein. Hingegen habe {[}ich{]} folgendes zu
               bemerken. Wenn Sie auf einige Wochen an die See\oindex{Ostsee@\textbf{Ostsee}|pwv} gehn, kann Ihnen doch auch die um ein paar Stunden
               verlängerte Reiſe nicht ankommen. Ko{\geminationm}en Sie aber immer
               nur auf 24 Stunden ans Ufer, ſo hab ich ohnedies ſehr wenig, \textsc{resp\textcolor{gray}{.}} zu wenig von Ihnen. Alles, was ich von deutſch\oindex{Deutschland@\textbf{Deutschland}|pwv}en Seebädern höre, ni{\geminationm}t
               mich dagegen ein; die bekannten {\pb}ſind in
               Hinſicht auf Publikum \textsc{etc} berüchtigt, die unbekannten
               ſollen was Comfor\strikeout{\textcolor{gray}{×}}t \textsc{etc} anbelangt übel ausſehen. Wälder gibts nur auf
                  Rügen\oindex{Ruegen@\textbf{Rügen}|pw}. Daenemark\oindex{Daenemark@\textbf{Dänemark}|pw} ke{\geminationn} ich. Seit ich \label{K_L03003-4v}\edtext{dort geweſen}{\lemma{\textnormal{\emph{dort geweſen}}}\Cendnote{\textnormal{im Sommer 1896}}}\label{K_L03003-4h} bin, ſehn ich mich zurück. Die Menſchen dort (die man ja nicht kennt), der
               Himmel, die Wälder, allerlei undefinirbares iſt in der Erinnerung für mich von einem
               wahren Zauber umgeben. Auch denk ich lebhaft an einen \label{K_L03003-5v}\edtext{Abſtecher nach Schweden\oindex{Schweden@\textbf{Schweden}|pw}, ev Norwegen\oindex{Norwegen@\textbf{Norwegen}|pw}}{\lemma{\textnormal{\emph{Abſtecher … Norwegen}}}\Cendnote{\textnormal{nicht geschehen}}}\label{K_L03003-5h}. Wir wollen
                  \label{K_L03003-6v}\edtext{auf 2, 3 {\pb}Tage nach Kopenhagen\oindex{Kopenhagen@\textbf{Kopenhagen}|pw}}{\lemma{\textnormal{\emph{auf 2, 3 Tage nach Kopenhagen}}}\Cendnote{\textnormal{Schnitzler\pwindex{Schnitzler, Arthur 15.05.1862 – 21.10.1931@\textsc{Schnitzler, Arthur} (15.05.1862 – 21.10.1931), \emph{Schriftsteller, Mediziner}|pwk} war vor seinem Aufenthalt in Marienlyst\oindex{Marienlyst@\textbf{Marienlyst}|pwk} nur am 28. 6. 1906 in Kopenhagen\oindex{Kopenhagen@\textbf{Kopenhagen}|pwk}.}}}\label{K_L03003-6h}, von dort aus inſpizire
               ich die Seeſeite\oindex{Oeresund@\textbf{Öresund}|pw} nach geeignetem
               Aufenthalt. –\pend
           \pstart
           Schönen Dank für die noch ſchönern \label{K_L03003-7v}\edtext{Feu{[}i{]}lletons\pwindex{Salten, Felix 06.09.1869 – 08.10.1945@\textsc{Salten, Felix} (06.09.1869 – 08.10.1945), \emph{Schriftsteller, Journalist}!Kater Lampe«26. 03. 1906@\strich\emph{»Kater Lampe«} {[}26. 03. 1906{]}|pwv}\pwindex{Salten, Felix 06.09.1869 – 08.10.1945@\textsc{Salten, Felix} (06.09.1869 – 08.10.1945), \emph{Schriftsteller, Journalist}!Russisches Theater. II23. 03. 1906@\strich\emph{Russisches Theater. II} {[}23. 03. 1906{]}|pwv}}{\lemma{\textnormal{\emph{Feuilletons}}}\Cendnote{\textnormal{Felix Salten\pwindex{Salten, Felix 06.09.1869 – 08.10.1945@\textsc{Salten, Felix} (06.09.1869 – 08.10.1945), \emph{Schriftsteller, Journalist}|pwk}: \emph{Russisches Theater. II}\pwindex{Salten, Felix 06.09.1869 – 08.10.1945@\textsc{Salten, Felix} (06.09.1869 – 08.10.1945), \emph{Schriftsteller, Journalist}!Russisches Theater. II23. 03. 1906@\strich\emph{Russisches Theater. II} {[}23. 03. 1906{]}|pwk}. In: \emph{B. Z. am Mittag}\pwindex{?? Werk@Nicht ermittelte Verfasserinnen und Verfasser!B.Z. am Mittag1904-10-22 – 1943@\emph{B.Z. am Mittag} {[}1904-10-22 – 1943{]}|pwk}, Jg. 30, Nr. 70, 23. 3. 1906, S. 2–3; Felix Salten\pwindex{Salten, Felix 06.09.1869 – 08.10.1945@\textsc{Salten, Felix} (06.09.1869 – 08.10.1945), \emph{Schriftsteller, Journalist}|pwk}: \emph{»Kater Lampe«}\pwindex{Salten, Felix 06.09.1869 – 08.10.1945@\textsc{Salten, Felix} (06.09.1869 – 08.10.1945), \emph{Schriftsteller, Journalist}!Kater Lampe«26. 03. 1906@\strich\emph{»Kater Lampe«} {[}26. 03. 1906{]}|pwk}. In: \emph{B. Z. am Mittag}\pwindex{?? Werk@Nicht ermittelte Verfasserinnen und Verfasser!B.Z. am Mittag1904-10-22 – 1943@\emph{B.Z. am Mittag} {[}1904-10-22 – 1943{]}|pwk}, Jg. 30,
                     Nr. 72, 26. 3. 1906, S. 2.}}}\label{K_L03003-7h}, Rußland\pwindex{Salten, Felix 06.09.1869 – 08.10.1945@\textsc{Salten, Felix} (06.09.1869 – 08.10.1945), \emph{Schriftsteller, Journalist}!Russisches Theater. II23. 03. 1906@\strich\emph{Russisches Theater. II} {[}23. 03. 1906{]}|pwv}\oindex{Russland@\textbf{Russland}|pw} und Lampe\pwindex{Salten, Felix 06.09.1869 – 08.10.1945@\textsc{Salten, Felix} (06.09.1869 – 08.10.1945), \emph{Schriftsteller, Journalist}!Kater Lampe«26. 03. 1906@\strich\emph{»Kater Lampe«} {[}26. 03. 1906{]}|pwv} betreffend.
               Sie \uline{haben} ſich halt immer. Wenn Sie mit ſich ſelber
               raufen, bleiben Sie doch auf immer der Gewinner. Ich ko{\geminationm}
               zu oft gegen mich nicht auf. – Immerhin, ich arbeite jetzt. Sie ſind ſchon alle wieder
               da, die Geſtältchen und Geſtalten, – aber mit meiner Macht über ſie ſiehts noch
               ziemlich {\pb}flau aus. – Komiſch, ja ſogar ein
               wenig traurig waren manche Kritiken über den Wurſtelſpaſs\pwindex{Schnitzler, Arthur 15.05.1862 – 21.10.1931@\textsc{Schnitzler, Arthur} (15.05.1862 – 21.10.1931), \emph{Schriftsteller, Mediziner}!Zum grossen Wurstel. Burleske in einem Akt08. 03. 1901@\strich\emph{Zum großen Wurstel. Burleske in einem Akt} {[}08. 03. 1901{]}|pwv}. Es wurde mir \label{K_L03003-8v}\edtext{ſo anerkennend vermerkt, daſs mir endgiltig mies vor mir
               geworden zu ſein ſcheint}{\lemma{\textnormal{\emph{ſo … ſcheint}}}\Cendnote{\textnormal{siehe A. S.: \emph{Tagebuch}, 27. 3. 1906}}}\label{K_L03003-8h}. Ja, »\label{K_L03003-9v}\edtext{Nordpol\oindex{Nordpol@\textbf{Nordpol}|pw}fahrer müſte man ſein\pwindex{Schnitzler, Arthur 15.05.1862 – 21.10.1931@\textsc{Schnitzler, Arthur} (15.05.1862 – 21.10.1931), \emph{Schriftsteller, Mediziner}!letzten Masken1901@\strich\emph{Die letzten Masken} {[}1901{]}|pwv}}{\lemma{\textnormal{\emph{Nordpolfahrer … ſein}}}\Cendnote{\textnormal{Schnitzler\pwindex{Schnitzler, Arthur 15.05.1862 – 21.10.1931@\textsc{Schnitzler, Arthur} (15.05.1862 – 21.10.1931), \emph{Schriftsteller, Mediziner}|pwk} paraphrasierte \emph{Die letzten Masken}\pwindex{Schnitzler, Arthur 15.05.1862 – 21.10.1931@\textsc{Schnitzler, Arthur} (15.05.1862 – 21.10.1931), \emph{Schriftsteller, Mediziner}!letzten Masken1901@\strich\emph{Die letzten Masken} {[}1901{]}|pwk}. Dort heißt es wörtlich: »Ein Bauer auf dem Land möcht
                        ich sein, ein Schafhirt, ein Nordpol\oindex{Nordpol@\textbf{Nordpol}|pw}fahrer – ah, was du willst! –\pwindex{Schnitzler, Arthur 15.05.1862 – 21.10.1931@\textsc{Schnitzler, Arthur} (15.05.1862 – 21.10.1931), \emph{Schriftsteller, Mediziner}!letzten Masken1901@\strich\emph{Die letzten Masken} {[}1901{]}|pwv}«}}}\label{K_L03003-9h}« ſagt Weihgaſt\pwindex{Schnitzler, Arthur 15.05.1862 – 21.10.1931@\textsc{Schnitzler, Arthur} (15.05.1862 – 21.10.1931), \emph{Schriftsteller, Mediziner}!letzten Masken1901@\strich\emph{Die letzten Masken} {[}1901{]}|pwv}, mit dem mich ſonſt nur geringe Sympathie \strikeout{\textcolor{gray}{bef}} verbindet. – \label{K_L03003-10v}\edtext{Kerr\pwindex{Kerr, Alfred 25.12.1867 – 12.10.1948@\textsc{Kerr, Alfred} (25.12.1867 – 12.10.1948), \emph{Schriftsteller, Kritiker}!Oedipus und der Ruf des Lebens1906-05@\strich\emph{Ödipus und der Ruf des Lebens} {[}1906-05{]}|pwv}\pwindex{Kerr, Alfred 25.12.1867 – 12.10.1948@\textsc{Kerr, Alfred} (25.12.1867 – 12.10.1948), \emph{Schriftsteller, Kritiker}|pw}}{\lemma{\textnormal{\emph{Kerr}}}\Cendnote{\textnormal{Alfred Kerr\pwindex{Kerr, Alfred 25.12.1867 – 12.10.1948@\textsc{Kerr, Alfred} (25.12.1867 – 12.10.1948), \emph{Schriftsteller, Kritiker}|pwk}: \emph{Ödipus und der Ruf des Lebens}\pwindex{Kerr, Alfred 25.12.1867 – 12.10.1948@\textsc{Kerr, Alfred} (25.12.1867 – 12.10.1948), \emph{Schriftsteller, Kritiker}!Oedipus und der Ruf des Lebens1906-05@\strich\emph{Ödipus und der Ruf des Lebens} {[}1906-05{]}|pwk}. In: \emph{Die neue Rundschau}\pwindex{?? Werk@Nicht ermittelte Verfasserinnen und Verfasser!neue Rundschau1904@\emph{Die neue Rundschau} {[}1904{]}|pwk}, Jg. 17, H. 5, Mai 1906, S. 492–498. Siehe A. S.: \emph{Tagebuch}, 30. 3. 1906.}}}\label{K_L03003-10h} hab ich eigentlich, innerlich, (das
               innerlich bezieht ſich auf ihn), charmant gefunden{\dots} Wiſſen
               Sie um wen es mir eigentlich am leideſten thut? Um die gute {\pb}Katharina\pwindex{Schnitzler, Arthur 15.05.1862 – 21.10.1931@\textsc{Schnitzler, Arthur} (15.05.1862 – 21.10.1931), \emph{Schriftsteller, Mediziner}!Ruf des Lebens. Schauspiel in drei Akten1906-02-20@\strich\emph{Der Ruf des Lebens. Schauspiel in drei Akten} {[}1906-02-20{]}|pwv}, die als Ophelia\pwindex{\textcolor{red}{\textsuperscript{XXXX1 indx}}!Hamlet1600@\strich\emph{Hamlet} {[}1600{]}|pwv}{ }\substVorne{}\textsuperscript{,}\substDazwischen{}(\substHinten{}ja wär ich Julius Bauer\pwindex{Bauer, Julius 15.10.1853 – 11.06.1941@\textsc{Bauer, Julius} (15.10.1853 – 11.06.1941), \emph{Schriftsteller, Journalist, Kritiker}|pw} ſo ſagt ich: als
               Pophelia) behandelt wird, – weil Frl Hofmann\pwindex{Hofmann, Grete @\textsc{Hofmann, Grete}, \emph{Schauspielerin}|pw}
               im letzten Akt\pwindex{Schnitzler, Arthur 15.05.1862 – 21.10.1931@\textsc{Schnitzler, Arthur} (15.05.1862 – 21.10.1931), \emph{Schriftsteller, Mediziner}!Ruf des Lebens. Schauspiel in drei Akten1906-02-20@\strich\emph{Der Ruf des Lebens. Schauspiel in drei Akten} {[}1906-02-20{]}|pwv} Blumen im Haar
               hatte. Als abſichtlich von mir aus Hamlet\pwindex{\textcolor{red}{\textsuperscript{XXXX1 indx}}!Hamlet1600@\strich\emph{Hamlet} {[}1600{]}|pw}
               herausgeſtohlene Ophelia\pwindex{\textcolor{red}{\textsuperscript{XXXX1 indx}}!Hamlet1600@\strich\emph{Hamlet} {[}1600{]}|pwv}.
               Einer wie der andre. –\pend
           \pstart
           \label{K_L03003-11v}\edtext{Neulich im Coloſſeum\oindex{Volkstheater in Rudolfsheim@\textbf{Volkstheater in Rudolfsheim}|pw}}{\lemma{\textnormal{\emph{Neulich im Coloſſeum}}}\Cendnote{\textnormal{siehe A. S.: \emph{Tagebuch}, 28. 3. 1906}}}\label{K_L03003-11h}; mit Wasserma{\geminationn}s\pwindex{Wassermann, Jakob 10.03.1873 – 01.01.1934@\textsc{Wassermann, Jakob} (10.03.1873 – 01.01.1934), \emph{Schriftsteller}|pw}\pwindex{Wassermann, Julie 05.12.1876 – April 1963@\textsc{Wassermann, Julie} (05.12.1876 – April 1963), \emph{Schriftstellerin}|pw} u. Kaufmann\pwindex{Kaufmann, Arthur 04.04.1872 – 25.07.1938@\textsc{Kaufmann, Arthur} (04.04.1872 – 25.07.1938), \emph{Rechtswissenschaftler, Privatgelehrte, Privatier}|pw}. Zwei Clowns als Nachtigallen den Unvergeßlichkeiten anzureihn.\pend
           \pstart
           Grüß Sie Gott. Herzlichſt Ihr {\\[\baselineskip]}\spacefill\mbox{A.}\pend
           \leftskip=0em{}
         
         \endnumbering\mylabel{h}\end{ledgroupsized}  \newcommand{\dateiname}{L03003}\newcommand{\titel}{Arthur Schnitzler an Felix Salten, 2. 4. 1906}\newcommand{\editorInnen}{Martin Anton Müller und Laura Untner}%% latex-leseansicht-abspann.tex
%% Abspann für die Leseansicht.
%% Der Schalter \ifkorrekturansicht ist bereits durch den Vorspann gesetzt.

%% latex-abspann.tex
%% Gemeinsamer Abspann für Korrekturansicht und Leseansicht.
%% Setzt den Schalter \ifkorrekturansicht voraus (gesetzt in den
%% einbindenden Dateien latex-korrekturansicht-abspann.tex bzw.
%% latex-leseansicht-abspann.tex).
%% ---------------------------------------------------------------

\normalsize

% Das esempio-Environment wird nur in der Leseansicht benötigt
\ifkorrekturansicht\else
\newenvironment{esempio}[3]%
{
    \vspace{1.5ex}
    \rlap{\underline{#1}}
    \par
    \setlength{\parindent}{0cm}
    \nopagebreak
    \leftskip=#2cm
    \rightskip=#3cm
}
{
    \par
}
\fi

\doendnotes{C}
\bigskip
\vfill

\clearpage

\footnotesize

\ifkorrekturansicht
  \lohead{\textsc{register}}
\fi

% theindex-Environment neu definieren ohne reledmac
\makeatletter
\renewenvironment{theindex}{%
  \ifkorrekturansicht
    \section*{\indexname}%
  \else
    \subsubsection*{Index der erwähnten Entitäten}%
  \fi
  \setlength{\parindent}{0pt}%
  \setlength{\parskip}{0pt plus 0.3pt}%
  \let\item\@idxitem
}{%
  \ifkorrekturansicht\clearpage\fi
}
\makeatother

\IfFileExists{\jobname-pw.ind}{\input{\jobname-pw.ind}}{}

% Quellenangabe nur in der Leseansicht
\ifkorrekturansicht\else
% Fallback-Definitionen, falls die .tex-Datei \titel etc. nicht gesetzt hat
\providecommand{\titel}{}
\providecommand{\editorInnen}{}
\providecommand{\dateiname}{\jobname}

\vspace{3cm}

\vfill

\footnotesize
\textsc{Quelle}: \titel. Herausgegeben von {\editorInnen}. In: \emph{Arthur Schnitzler: Briefwechsel mit Autorinnen und Autoren}.
 Digitale Edition, https://schnitzler-briefe.acdh.oeaw.ac.at/{\dateiname}.html (Stand \today)
\fi

\end{document}


      