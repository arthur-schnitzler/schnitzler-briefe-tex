%% latex-leseansicht-vorspann.tex
%% Vorspann für die Leseansicht.
%% Lädt die gemeinsame Datei latex-vorspann.tex mit nicht gesetztem Schalter.

\newif\ifkorrekturansicht
\korrekturansichtfalse

\input{../tex-inputs/latex-vorspann}

\begin{center}
            \textcolor{red}{ENTWURF, NICHT FERTIG KORRIGIERT}
                      \end{center}
            
         
         \renewcommand{\erwaehntePersonen}{Personen: Julius Bauer,  Engel, Grete Hofmann, Siegfried Jacobsohn, Arthur Kaufmann, Alfred Kerr, Felix Salten, Ottilie Salten, Elisabeth Steinrück, Siegfried Trebitsch, Jakob Wassermann}
         \renewcommand{\erwaehnteOrte}{Orte: Bayern, Dänemark, Edmund-Weiß-Gasse, Genf, Harz, Kopenhagen, Lugano, Nordpol, Russland, Rügen, Salzkammergut, Schweden, Südtirol, Tirol, Volkstheater in Rudolfsheim, Wien, Öresund}
         \renewcommand{\erwaehnteWerke}{Werke: ?? [Feuilleton zu/über Lange], ?? [Feuilleton über Lange], ?? [Feuilleton über Russland], ?? [Feuilleton über Russland], Bund der Bühnendichter. II, Bühnenvertrieb, Der Ruf des Lebens. Schauspiel in drei Akten, Die Schaubühne, Die letzten Masken, Hamlet, Zum großen Wurstel. Burleske in einem Akt}
               \section[Arthur Schnitzler an Felix Salten, 2. 4. 1906]{ Arthur Schnitzler an Felix Salten, 2. 4. 1906}\nopagebreak\mylabel{v}\rehead{ }\begin{ledgroupsized}[t]{13cm}\normalsize\beginnumbering \toendnotes[C]{\smallbreak\pagebreak[2]} \Standort{Wienbibliothek im Rathaus, ZPH 1681, 2.1.516.}
\physDesc{Brief, 2 Blätter, 8 Seiten, 3580 Zeichen
\newline{}Handschrift: schwarze Tinte, deutsche Kurrent
\newline{}Ordnung: mit Bleistift von unbekannter Hand Nummerierung der Blätter des
                                 Konvoluts: »20«–»23«  }\toendnotes[C]{\smallbreak}\pstart
           \noindent{}{\pb}\textcolor{gray}{\textbf{Dr. Arthur Schnitzler}}\hfill 2. April 906\pend
           \pstart
           \textcolor{gray}{\textbf{Wien, XVIII. Spoettelgasse 7\oindex{Edmund-Weiss-Gasse@\textbf{Edmund-Weiß-Gasse}|pw}.}}\pend
           \pstart
           lieber, vor einigen Wochen ſchrieb mir Liesl\pwindex{Steinrueck, Elisabeth 19.11.1885 – 07.04.1920@\textsc{Steinrück, Elisabeth} (19.11.1885 – 07.04.1920)|pw}, daſs ihr ein Bekannter, namens Engel\pwindex{Engel @\textsc{Engel}|pw}, eine ermäßigte Seereiſe verſchaffen werde; daſs ſie
               ſich nun in dieſer Sache an Sie zu wenden ſcheint (wie mir ihr letzter Brief
               andeutet) iſt mir wie Sie ſich denken können, ſo wenig recht als möglich.– Meinen
               begeiſterten Brief an Trebitſch\pwindex{Trebitsch, Siegfried 22.12.1868 – 03.06.1956@\textsc{Trebitsch, Siegfried} (22.12.1868 – 03.06.1956), \emph{Schriftsteller, Übersetzer}|pw} kö{\geminationn}en Sie ſich ja ungefähr vorſtellen. Er ſchrieb mir
               gleich nach Erſcheinen jenes \label{K_L03003-1v}\edtext{Artikels\pwindex{Trebitsch, Siegfried 22.12.1868 – 03.06.1956@\textsc{Trebitsch, Siegfried} (22.12.1868 – 03.06.1956), \emph{Schriftsteller, Übersetzer}!Buehnenvertrieb22. 03. 1906@\strich\emph{Bühnenvertrieb} {[}22. 03. 1906{]}|pwv}}{\lemma{\textnormal{\emph{Artikels}}}\Cendnote{\textnormal{Siegfried Trebitsch\pwindex{Trebitsch, Siegfried 22.12.1868 – 03.06.1956@\textsc{Trebitsch, Siegfried} (22.12.1868 – 03.06.1956), \emph{Schriftsteller, Übersetzer}|pwk}: \emph{Bühnenvertrieb}\pwindex{Trebitsch, Siegfried 22.12.1868 – 03.06.1956@\textsc{Trebitsch, Siegfried} (22.12.1868 – 03.06.1956), \emph{Schriftsteller, Übersetzer}!Buehnenvertrieb22. 03. 1906@\strich\emph{Bühnenvertrieb} {[}22. 03. 1906{]}|pwk}. In: \emph{Die
                        Schaubühne}\pwindex{Schaubuehne7.9.1905 – 1993@\emph{Die Schaubühne} {[}7.9.1905 – 1993{]}|pwk}, Jg. 2, Nr. 12, 22. 3. 1906,
                  S. 348–350.}}}\label{K_L03003-1h} in der Schb.\pwindex{Schaubuehne7.9.1905 – 1993@\emph{Die Schaubühne} {[}7.9.1905 – 1993{]}|pw} ich
               ſolle ihn »beruhigen«. Ich hab {\pb}ihn beruhigt.
               Im übrigen hat die Bühnenvertriebsſache ſchon \strikeout{\textcolor{gray}{i}hre} Bedeutung. Nur muſs ſie in Zuſa{\geminationm}enhang mit den andern Fragen behandelt werden, die ſich
               auf das Verhältnis des Autors zu ſeiner geſchäftl. Umwelt beziehen. Einige dieſer
               Fragen hab ich \label{K_L03003-2v}\edtext{in einem Brief\pwindex{Bund der Buehnendichter. II1906-04-12@\emph{Bund der Bühnendichter. II} {[}1906-04-12{]}|pwv}}{\lemma{\textnormal{\emph{in einem Brief}}}\Cendnote{\textnormal{\emph{Bund der Bühnendichter. II}\pwindex{Bund der Buehnendichter. II1906-04-12@\emph{Bund der Bühnendichter. II} {[}1906-04-12{]}|pwk} In: \emph{Die Schaubühne}\pwindex{Schaubuehne7.9.1905 – 1993@\emph{Die Schaubühne} {[}7.9.1905 – 1993{]}|pwk}, Jg. 33, Nr. 11.176, 12. 4. 1906, S. 10. siehe A. S.: \emph{»Das Zeitlose ist von kürzester Dauer«}, Bund der Bühnendichter, 12. 4. 1906.}}}\label{K_L03003-2h} an Jacobsohn\pwindex{Jacobsohn, Siegfried 28.01.1881 – 03.12.1926@\textsc{Jacobsohn, Siegfried} (28.01.1881 – 03.12.1926), \emph{Journalist, Kritiker, Publizist}|pw} kurz formulirt.– \pend
           \pstart
           Nun unſre Radreiſe »oder ſo«. Wenn Sie irgendwas deutſches, Thüringen Harz\oindex{Harz@\textbf{Harz}|pw}{ }\textsc{etc.} vorziehen, ſo möchte ich dieſe Reiſe mehr gegen den
                  So{\geminationm}er verſchieben, etwa gegen Mitte Juli, um dann
               gleich das Seebad an {\pb}ſchließen zu können.
               Ziehen Sie Tirol\oindex{Tirol@\textbf{Tirol}|pw}\oindex{Suedtirol@\textbf{Südtirol}|pw} ev. Salzka{\geminationm}ergut\oindex{Salzkammergut@\textbf{Salzkammergut}|pw}, (bayriſches Hochgebirge\oindex{Bayern@\textbf{Bayern}|pw}?) vor, ſo ſchlage ich erſte Hälfte Juni
               vor. Geht Ihre Frau\pwindex{Salten, Ottilie 07.03.1868 – 22.06.1942@\textsc{Salten, Ottilie} (07.03.1868 – 22.06.1942), \emph{Schauspielerin}|pwv} mit, ſo
               käme die meine auch, und wir würden da{\geminationn} mehr eine
               Radialradpartie machen, d. h. allerlei Fahrten, mit feſtem Stützpunkt.– Ko{\geminationm}t Otti\pwindex{Salten, Ottilie 07.03.1868 – 22.06.1942@\textsc{Salten, Ottilie} (07.03.1868 – 22.06.1942), \emph{Schauspielerin}|pw} nicht, ſo
               ſoll es eine Längspartie werden, »wie einſt im Mai«, (we{\geminationn} Sie uns jetzt als Julier, resp. Auguſtiner (Sie \introOben{}Anfang\introOben{}
               Julier und ich Endauguſtiner anſprechen.). Gar zu weite Bahnreiſe (Genf\oindex{Genf@\textbf{Genf}|pw}, Lugano\oindex{Lugano@\textbf{Lugano}|pw}) möcht ich
               gern vermeiden, {\pb}aus 17 Gründen.– Von meiner
                  daeniſchen\oindex{Daenemark@\textbf{Dänemark}|pw} Idee, lieber, werd ich ſchwer
               abzubringen ſein. Hingegen habe folgendes zu bemerken. Wenn Sie auf einige Wochen an
               die See gehen, kann Ihnen doch auch die um ein paar Stunden verlängerte Reiſe nicht
               ankommen. Ko{\geminationm}en Sie aber immer nur auf 24 Stunden ans
               Ufer, ſo hab ich ohnedies ſehr wenig, \textsc{resp.} zu wenig von
               Ihnen. Alles, was ich von deutſchen Seebädern höre, ni{\geminationm}t
               mich dagegen ein; die bekannten {\pb}ſind in
               Hinſicht auf Publikum \textsc{etc.} berüchtigtm die unbekannten
               ſollen was Comfor\strikeout{\textcolor{gray}{×}}t \textsc{etc} anbelangt übel ausſehen. Wälder gibts nur auf
                  Rügen\oindex{Ruegen@\textbf{Rügen}|pw}. Daenemark\oindex{Daenemark@\textbf{Dänemark}|pw} ke{\geminationn} ich. Seit ich dort geweſen bin,
               ſehn ich mich zurück. Die Menſchen dirt (die man ja nicht kennt), der Himmel, die
               Wälder, allerlei undefinirbares iſt in der Erinnerungen für mich von einem wahren
               Zauber umgeben. Auch denk ich lebhaft an einen Abſtecher nach Schweden\oindex{Schweden@\textbf{Schweden}|pw}, ev Norwegen. Wir wollen auf 2, 3 {\pb}Tage nach Kopenhagen\oindex{Kopenhagen@\textbf{Kopenhagen}|pw}, von dort aus inſpicire ich die Seeſeite\oindex{Oeresund@\textbf{Öresund}|pw} nach geeignetem Aufenthalt.– \pend
           \pstart
           Schönen Dank für die noch ſchönern \label{K_L03003-111v}\edtext{Feu{[}i{]}lletons\textcolor{red}{\textsuperscript{XXXX indx}}\textcolor{red}{\textsuperscript{XXXX indx}}}{\lemma{\textnormal{\emph{Feuilletons}}}\Cendnote{\textnormal{Felix Salten\pwindex{Salten, Felix 06.09.1869 – 08.10.1945@\textsc{Salten, Felix} (06.09.1869 – 08.10.1945), \emph{Schriftsteller, Journalist}|pwk}: \emph{»Kater Lampe«}\textcolor{red}{\textsuperscript{XXXX indx}}. In: \emph{B.
                        Z. am Mittag}\textcolor{red}{\textsuperscript{XXXX indx}}, Jg. 30, Nr. 72, 26. 3. 1906, S. 2. Felix Salten\pwindex{Salten, Felix 06.09.1869 – 08.10.1945@\textsc{Salten, Felix} (06.09.1869 – 08.10.1945), \emph{Schriftsteller, Journalist}|pwk}: \emph{Russisches Theater. II}\textcolor{red}{\textsuperscript{XXXX indx}}. In: \emph{B. Z. am Mittag}\textcolor{red}{\textsuperscript{XXXX indx}}, Jg. 30, Nr. 70,
                        23. 3. 1906, S. 2–3.}}}\label{K_L03003-111h}, Rußland\pwindex{Salten, Felix 06.09.1869 – 08.10.1945@\textsc{Salten, Felix} (06.09.1869 – 08.10.1945), \emph{Schriftsteller, Journalist}!?? [Feuilleton ueber Russland]1906-03-12@\strich\emph{?? [Feuilleton über Russland]} {[}1906-03-12{]}|pwv}\oindex{Russland@\textbf{Russland}|pw} und Lampe\textcolor{red}{\textsuperscript{XXXX indx}} betreffend.
               Sie \uline{haben} ſich halt immer. Wenn Sie mit ſich ſelber
               raufen, bleiben Sie doch auf immer der Gewinner. Ich ko{\geminationm}
               ja oft gegen mich nicht auf.– Immerhin, ich arbeite jetzt. Sie ſind ſchon alle wieder
               da, die Geſtältchen und Geſtalten, – aber mit meiner Macht über ſie ſiehts noch
               ziemlich {\pb}flau aus.– Komiſch, ja ſogar ein
               wenig traurig waren mache Kritiken über den Wurſtelſpaſs\pwindex{Schnitzler, Arthur 15.05.1862 – 21.10.1931@\textsc{Schnitzler, Arthur} (15.05.1862 – 21.10.1931), \emph{Schriftsteller, Mediziner}!Zum grossen Wurstel. Burleske in einem Akt08. 03. 1901@\strich\emph{Zum großen Wurstel. Burleske in einem Akt} {[}08. 03. 1901{]}|pwv}. Es wurde mir ſo anerkennend vermerkt, daſs
               mir endgiltig mies zu mir geworden zu ſein ſcheint. Ja, »\label{K_L03003-22v}\edtext{Nordpol\oindex{Nordpol@\textbf{Nordpol}|pw}fahrer müſte man ſein\pwindex{Schnitzler, Arthur 15.05.1862 – 21.10.1931@\textsc{Schnitzler, Arthur} (15.05.1862 – 21.10.1931), \emph{Schriftsteller, Mediziner}!letzten Masken1901@\strich\emph{Die letzten Masken} {[}1901{]}|pwv}}{\lemma{\textnormal{\emph{Nordpolfahrer … ſein}}}\Cendnote{\textnormal{Schnitzler\pwindex{Schnitzler, Arthur 15.05.1862 – 21.10.1931@\textsc{Schnitzler, Arthur} (15.05.1862 – 21.10.1931), \emph{Schriftsteller, Mediziner}|pwk} zitiert nicht, sondern
                  paraphrasiert, in \emph{Die letzten Masken}\pwindex{Schnitzler, Arthur 15.05.1862 – 21.10.1931@\textsc{Schnitzler, Arthur} (15.05.1862 – 21.10.1931), \emph{Schriftsteller, Mediziner}!letzten Masken1901@\strich\emph{Die letzten Masken} {[}1901{]}|pwk} heißt
                     es: »Ein Bauer auf dem Land möcht ich sein, ein Schafhirt, ein Nordpol\oindex{Nordpol@\textbf{Nordpol}|pw}fahrer – ah, was du
                     willst! –«}}}\label{K_L03003-22h}« ſagt Weihgaſt\pwindex{Schnitzler, Arthur 15.05.1862 – 21.10.1931@\textsc{Schnitzler, Arthur} (15.05.1862 – 21.10.1931), \emph{Schriftsteller, Mediziner}!letzten Masken1901@\strich\emph{Die letzten Masken} {[}1901{]}|pwv}, mit dem mich ſonſt nur geringe Sympathie \strikeout{\textcolor{gray}{bef}} verbindet.– Kerr\pwindex{Kerr, Alfred 25.12.1867 – 12.10.1948@\textsc{Kerr, Alfred} (25.12.1867 – 12.10.1948), \emph{Schriftsteller, Kritiker}|pw} hab ich eigentlich,
               innerlich, (das innerlich bezieht ſich auf ihn), charmant gefunden{\dots} Wiſſen Sie um wen es mir eigentlich am leideſten thut?
               Um die gute {\pb}Katharina\pwindex{Schnitzler, Arthur 15.05.1862 – 21.10.1931@\textsc{Schnitzler, Arthur} (15.05.1862 – 21.10.1931), \emph{Schriftsteller, Mediziner}!Ruf des Lebens. Schauspiel in drei Akten1906-02-20@\strich\emph{Der Ruf des Lebens. Schauspiel in drei Akten} {[}1906-02-20{]}|pwv}, die als Ophelia\pwindex{\textcolor{red}{\textsuperscript{XXXX1 indx}}!Hamlet1600@\strich\emph{Hamlet} {[}1600{]}|pwv} (ja wär ich Julius Bauer\pwindex{Bauer, Julius 15.10.1853 – 11.06.1941@\textsc{Bauer, Julius} (15.10.1853 – 11.06.1941), \emph{Schriftsteller, Journalist, Kritiker}|pw} ſo ſagt ich als Pophelia) behandelt
               wird, – weil Frl. Hofmann\pwindex{Hofmann, Grete @\textsc{Hofmann, Grete}, \emph{Schauspielerin}|pw} im letzten Akt
               Blumen im Haar hatte. Als abſichtlich von mir aus Hamlet\pwindex{\textcolor{red}{\textsuperscript{XXXX1 indx}}!Hamlet1600@\strich\emph{Hamlet} {[}1600{]}|pw} herausgeſtohlene Ophelia\pwindex{\textcolor{red}{\textsuperscript{XXXX1 indx}}!Hamlet1600@\strich\emph{Hamlet} {[}1600{]}|pwv}. Einer wie der andre.– \pend
           \pstart
           \label{K_L03003-1112v}\edtext{Neulich im Coloſſeum\oindex{XXXX Ortsangabe fehlt|pw}}{\lemma{\textnormal{\emph{Neulich im Coloſſeum}}}\Cendnote{\textnormal{siehe A. S.: \emph{Tagebuch}, 28. 3. 1906}}}\label{K_L03003-1112h}; mit Wasserma{\geminationn}s\pwindex{Wassermann, Jakob 10.03.1873 – 01.01.1934@\textsc{Wassermann, Jakob} (10.03.1873 – 01.01.1934), \emph{Schriftsteller}|pw} u. Kaufmann\pwindex{Kaufmann, Arthur 04.04.1872 – 25.07.1938@\textsc{Kaufmann, Arthur} (04.04.1872 – 25.07.1938), \emph{Wissenschaftler, Privatgelehrte, Privatier}|pw}. Zwei Clowns als
               Nachtigallen den Unvergeßlichkeiten anzureihn. \pend
           \pstart
           Grüß Sie Gott. Herzlichſt Ihr {\\[\baselineskip]}\spacefill\mbox{A.}\pend
           \leftskip=0em{}
         
         \endnumbering\mylabel{h}\end{ledgroupsized}\begin{anhang}\end{anhang}\newcommand{\dateiname}{L03003}\newcommand{\titel}{Arthur Schnitzler an Felix Salten, 2. 4. 1906}\newcommand{\editorInnen}{Martin Anton Müller und Laura Untner}%% latex-leseansicht-abspann.tex
%% Abspann für die Leseansicht.
%% Der Schalter \ifkorrekturansicht ist bereits durch den Vorspann gesetzt.

%% latex-abspann.tex
%% Gemeinsamer Abspann für Korrekturansicht und Leseansicht.
%% Setzt den Schalter \ifkorrekturansicht voraus (gesetzt in den
%% einbindenden Dateien latex-korrekturansicht-abspann.tex bzw.
%% latex-leseansicht-abspann.tex).
%% ---------------------------------------------------------------

\normalsize

% Das esempio-Environment wird nur in der Leseansicht benötigt
\ifkorrekturansicht\else
\newenvironment{esempio}[3]%
{
    \vspace{1.5ex}
    \rlap{\underline{#1}}
    \par
    \setlength{\parindent}{0cm}
    \nopagebreak
    \leftskip=#2cm
    \rightskip=#3cm
}
{
    \par
}
\fi

\doendnotes{C}
\bigskip
\vfill

\clearpage

\footnotesize

\ifkorrekturansicht
  \lohead{\textsc{register}}
\fi

% theindex-Environment neu definieren ohne reledmac
\makeatletter
\renewenvironment{theindex}{%
  \ifkorrekturansicht
    \section*{\indexname}%
  \else
    \subsubsection*{Index der erwähnten Entitäten}%
  \fi
  \setlength{\parindent}{0pt}%
  \setlength{\parskip}{0pt plus 0.3pt}%
  \let\item\@idxitem
}{%
  \ifkorrekturansicht\clearpage\fi
}
\makeatother

\IfFileExists{\jobname-pw.ind}{\input{\jobname-pw.ind}}{}

% Quellenangabe nur in der Leseansicht
\ifkorrekturansicht\else
% Fallback-Definitionen, falls die .tex-Datei \titel etc. nicht gesetzt hat
\providecommand{\titel}{}
\providecommand{\editorInnen}{}
\providecommand{\dateiname}{\jobname}

\vspace{3cm}

\vfill

\footnotesize
\textsc{Quelle}: \titel. Herausgegeben von {\editorInnen}. In: \emph{Arthur Schnitzler: Briefwechsel mit Autorinnen und Autoren}.
 Digitale Edition, https://schnitzler-briefe.acdh.oeaw.ac.at/{\dateiname}.html (Stand \today)
\fi

\end{document}


      