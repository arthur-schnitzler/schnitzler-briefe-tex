%% latex-leseansicht-vorspann.tex
%% Vorspann für die Leseansicht.
%% Lädt die gemeinsame Datei latex-vorspann.tex mit nicht gesetztem Schalter.

\newif\ifkorrekturansicht
\korrekturansichtfalse

\input{../tex-inputs/latex-vorspann}

\begin{center}
            \textcolor{red}{ENTWURF, NICHT FERTIG KORRIGIERT}
                      \end{center}
            
               \section[Paul Goldmann an Arthur Schnitzler, 18. 11. 1894]{ Paul Goldmann an Arthur Schnitzler, 18. 11. 1894}\nopagebreak\mylabel{v}\rehead{ }\begin{ledgroupsized}[t]{13cm}\normalsize\beginnumbering\briefempfaengerindex{Schnitzler, Arthur@\textsc{Schnitzler, Arthur}!zzzGoldmann, Paul@\emph{von Paul Goldmann}!1894-11-181@{18. 11. 1894}|(be} \toendnotes[C]{\smallbreak\pagebreak[2]} \Standort{DLA, A:Schnitzler, HS.NZ85.1.3164.}
\physDesc{Brief, 2 Blätter, 8 Seiten
\newline{}Handschrift: schwarze Tinte, deutsche Kurrent
\newline{}Schnitzler: 1) mit Bleistift auf dem ersten Blatt die Jahreszahl
                                       »94« vermerkt 2) mit rotem Buntstift sieben Unterstreichungen}\toendnotes[C]{\smallbreak}\pstart
           \noindent{}{\pb}\textcolor{gray}{\textbf{Frankfurter Zeitung\orgindex{Frankfurter Zeitung@Frankfurter Zeitung|pw}.}}\hfill \textsc{Paris\oindex{Paris@\textbf{Paris}|pw}}, 18. November.\pend
           \pstart
           \textcolor{gray}{\textbf{(Gazette de
                  Francfort\orgindex{Frankfurter Zeitung@Frankfurter Zeitung|pw}.)}}\pend
           \pstart
           \textcolor{gray}{\textbf{\begin{otherlanguage}{french}Fondateur\end{otherlanguage}{ }\textbf{M. L.
                  Sonnemann\pwindex{Sonnemann, Leopold 1831-10-29 – 1909-10-30@\textsc{Sonnemann, Leopold} (1831-10-29 – 1909-10-30), \emph{Journalist, Herausgeber}|pw}}.}}\pend
           \pstart
           \textcolor{gray}{\textbf{\begin{otherlanguage}{french}Journal politique,
                        financier,\end{otherlanguage}}}\pend
           \pstart
           \textcolor{gray}{\textbf{\begin{otherlanguage}{french}commercial et
                     littéraire.\end{otherlanguage}}}\pend
           \pstart
           \textcolor{gray}{\textbf{\begin{otherlanguage}{french}\textbf{Paraissant trois fois
                           par jour}\end{otherlanguage}}}.\pend
           \pstart
           \textcolor{gray}{\textbf{–}}\pend
           \pstart
           \textcolor{gray}{\textbf{\begin{otherlanguage}{french}\textbf{Bureaux à Paris\oindex{Paris@\textbf{Paris}|pw}:}\end{otherlanguage}}}\pend
           \pstart
           \textcolor{gray}{\textbf{\begin{otherlanguage}{french}\textbf{24. Rue Feydeau}\oindex{rue Feydeau@\textbf{rue Feydeau}|pw}.\end{otherlanguage}}}\pend
           \pstart{}Mein lieber Freund,\pend\pstart
           Ich will Dir täglich ſchreiben und bringe die Energie dafür nicht zuſammen. Nicht
               einmal dafür! Ich bin in einem ſchlimmen Gemütfszuſtande. Ich ſuche nach einem
               Lebensziel und finde es nicht – ſuche mich ſelbſt zu beſchränken, zu erkennen, zu
               ordnen und kann es nicht – und nach kurzen Anläufen falle ich in Zeitvergeudung,
               Außenleben und Wirrniß zurück. Dabei werde ich alle paar Tage daran erinnert, daß ich
               dreißig Jahre bin, nichts geleiſtet habe, zurückbleibe hinter allen Andern. Es iſt
               ein zerſtörendes Gefühl, und doch finde ich die {\pb}Kraft nicht zum Arbeiten. Die Zeit hätte ich jetzt, – alſo es gibt keine
               Entſchuldigung mehr. Das hindert mich an Allem, ſelbſt am Briefeſchreiben. Du
               begreifſt mich gewiß.\pend
           \pstart
           Ich raffe mich heut ein wenig zuſammen; denn ich möchte gar ſo gern hören, wie es mit
               Deinem Stücke\pwindex{Schnitzler, Arthur 15.05.1862 – 21.10.1931@\textsc{Schnitzler, Arthur} (15.05.1862 – 21.10.1931), \emph{Schriftsteller, Mediziner}!Liebelei. Schauspiel in drei Akten9. 10. 1895@\strich\emph{Liebelei. Schauspiel in drei Akten} {[}9. 10. 1895{]}|pwv} weitergeht. Was Du
               mir über Deine \label{K_L02620-1v}\edtext{erſte Unterredung mit
                  B.\pwindex{Burckhard, Max Eugen 14.07.1854 – 16.03.1912@\textsc{Burckhard, Max Eugen} (14.07.1854 – 16.03.1912), \emph{Schriftsteller, Rechtswissenschaftler, Theaterleiter}|pw}}{\lemma{\textnormal{\emph{erſte Unterredung mit
                  B.}}}\Cendnote{\textnormal{siehe A. S.: \emph{Tagebuch}, 5. 11. 1894}}}\label{K_L02620-1h}
               geſchrieben, erſcheint mir ganz und gar nicht ungünſtig. Daß es nicht ſo glatt gehen
               würde, war ſelbſtverſtändlich. Dabei geht es doch noch relativ glatt. Wenn man in
               einem Theater den Director für ſich hat, ſo iſt das, denke ich, Chance genug. Das {\pb}Übrige iſt Zopf und \textsc{chinoiserie}. Dafür ſind wir ja im guten Lande Öſterreich\oindex{Oesterreich@\textbf{Österreich}|pw}. Wüßteſt Du nur, was hier die jungen Leute dulden müſſen, ehe ſie
               aufgeführt werden. An die \textsc{Comédie
                     Française\orgindex{Comedie-Française@Comédie-Française|pw}} kommt überhaupt keiner heran, wenn ihn nicht ein Akademiker\orgindex{Academie Française@Académie Française|pwv} oder ein großer
               Komödiant protegirt, und \strikeout{\textsc{Henr}} der alte \textsc{Henri Becque\pwindex{Becque, Henri 09.04.1837 – 12.05.1899@\textsc{Becque, Henri} (09.04.1837 – 12.05.1899), \emph{Schriftsteller}|pw}} ſelbſt hat ſeinerzeit die Aufführung von »\textsc{La Parisienne\pwindex{Becque, Henri 09.04.1837 – 12.05.1899@\textsc{Becque, Henri} (09.04.1837 – 12.05.1899), \emph{Schriftsteller}!Parisienne1885@\strich\emph{La Parisienne} {[}1885{]}|pw}}« durch ein
               Machtwort des Miniſters\pwindex{Bourgeois, Leon 1851-05-29 – 1925-09-29@\textsc{Bourgeois, Léon} (1851-05-29 – 1925-09-29), \emph{Politiker, Nobelpreisträger, Jurist}|pwv}
               erzwingen müſſen. Es gibt keinen Erfolg, zu dem man nicht über Hintertreppen ſteigen
               müßte, beſonders beim Theater. Thut mir nur leid, daß ich nicht gerade jetzt um Dich
               bin, um {\pb}mit Dir über all’ die Trottelhaftigkeiten
               zu lachen, die Dir vorausſichtlich werden geſagt oder angethan werden, und vielleicht
               auch um Dir ein Paar unangenehme Wege zu erſparen. Übrigens meinſt Du es ja ſelbſt
               ironiſch, und das iſt das Beſte. Bitte, ſchreib’ mir nur raſch, wieweit die Sache
               iſt. Und möchteſt Du es nicht doch zugleich in \label{K_L02620-3v}\edtext{Berlin\oindex{Berlin@\textbf{Berlin}|pw} einreichen}{\lemma{\textnormal{\emph{Berlin einreichen}}}\Cendnote{\textnormal{XXXX}}}\label{K_L02620-3h}?\pend
           \pstart
           Geſtern habe ich die \label{K_L02620-2v}\edtext{Fortſetzung von »Sterben\pwindex{Schnitzler, Arthur 15.05.1862 – 21.10.1931@\textsc{Schnitzler, Arthur} (15.05.1862 – 21.10.1931), \emph{Schriftsteller, Mediziner}!Sterben. Novelle1.10.1894 – 1.12.1894@\strich\emph{Sterben. Novelle} {[}1.10.1894 – 1.12.1894{]}|pw}«}{\lemma{\textnormal{\emph{Fortſetzung von »Sterben«}}}\Cendnote{\textnormal{Der zweite Teil (von
                  drei) erschien Anfang November in der \emph{Neuen deutschen Rundschau}\pwindex{Neue Deutsche Rundschau1894-01-01 – 1903-12-31@\emph{Neue Deutsche Rundschau}|pwk} (H. 11,
                  S. 1073–1101).}}}\label{K_L02620-2h} geleſen. Es iſt dumm, daß man es mit
               Zwiſchenräumen \strikeout{von} von einem Monat leſen muß. Ich bin
               mir über den Eindruck infolgedeſſen jetzt weniger {\pb}klar, als am Anfang. Ich weiß nur, daß ich im Einzelnen Entzückendes und Großes
               finde. Auch iſt der Styl köſtlich in ſeiner Einfachheit, mit all’ den Tiefen
               darunter. \strikeout{Ein \textcolor{gray}{×}} Hier und da iſt es mir aber doch zu einfach. Zum Beiſpiel:
                  \textsc{Salzburg\oindex{Salzburg@\textbf{Salzburg}|pw}}, ich meine das
               Landſchaftliche und Äußerliche, iſt meiner Empfindung nach um eine \textsc{Nuance} zu blaß gerathen. Alles in Allem ein reifes und
               ernſtes Werk. Aber, wie geſagt, ich muß es als Buch im Zuſammenhange leſen. Mir ahnt
               nur, daß ich es ſchön finden werde, {\pb}aber ich habe
               noch kein klares Bewußtſein davon. Dieſe verfluchten Fortſetzungen! Eine kleine
               Äußerlichkeit: bei der Buchausgabe \label{K_mets_Goldmann_94-partII-5v}\edtext{ſtreiche}{\lemma{\textnormal{\emph{ſtreiche}}}\Cendnote{\textnormal{Schnitzler\pwindex{Schnitzler, Arthur 15.05.1862 – 21.10.1931@\textsc{Schnitzler, Arthur} (15.05.1862 – 21.10.1931), \emph{Schriftsteller, Mediziner}|pwk} veränderte die Stelle für die Buchausgabe nicht.}}}\label{K_mets_Goldmann_94-partII-5h} auf
               Seite 1077 in der 20ten Zeile \uline{von unten} hinter
               »Einwohner« die Worte »der Stadt« weg, es iſt zu viel »Stadt« in dem Abſatz.\pend
           \pstart
           Wann kriege, ich nun wohl das Stück\pwindex{Schnitzler, Arthur 15.05.1862 – 21.10.1931@\textsc{Schnitzler, Arthur} (15.05.1862 – 21.10.1931), \emph{Schriftsteller, Mediziner}!Liebelei. Schauspiel in drei Akten9. 10. 1895@\strich\emph{Liebelei. Schauspiel in drei Akten} {[}9. 10. 1895{]}|pwv} zu leſen?\pend
           \pstart
           Mein \strikeout{\textcolor{gray}{Onk}}{ }Onkel\pwindex{Mamroth, Fedor 21.02.1851 – 25.06.1907@\textsc{Mamroth, Fedor} (21.02.1851 – 25.06.1907), \emph{Journalist, Kritiker}|pwv} hat mich
               vor vier Wochen nach Deiner Adreſſe gefragt, um Dir Bücher zu ſchicken. Da ich aber
               wieder einmal mit ihm grolle, habe ich nicht geantwortet. Hätteſt Du nicht irgend
               einen Vorwand ihm zu \label{K_mets_Goldmann_94-partII-44v}\edtext{ſchreiben}{\lemma{\textnormal{\emph{ſchreiben}}}\Cendnote{\textnormal{siehe Arthur Schnitzler an Fedor Mamroth, 7. 12. 1894}}}\label{K_mets_Goldmann_94-partII-44h}{ }\strikeout{un}, damit er zugleich {\pb}Deine Adreſſe\oindex{Frankgasse@\textbf{Frankgasse}|pwv} erführe?\pend
           \pstart
           Die »Zeit\orgindex{Zeit. Wiener Wochenschrift@Die Zeit. Wiener Wochenschrift|pw}« gefällt mir ganz ausnehmend. Das iſt ein
                  Blatt\pwindex{Zeit. Wiener Wochenschrift1894 – 1904@\emph{Die Zeit. Wiener Wochenschrift}|pwv}, durchaus nach meinem
               Sinn. \textsc{Kanner\pwindex{Kanner, Heinrich 09.11.1864 – 15.02.1930@\textsc{Kanner, Heinrich} (09.11.1864 – 15.02.1930), \emph{Publizist}|pw}} übertrifft
               ſich ſelbſt, \textsc{Bahr\pwindex{Bahr, Hermann 19.07.1863 – 15.01.1934@\textsc{Bahr, Hermann} (19.07.1863 – 15.01.1934), \emph{Schriftsteller, Kritiker}|pw}} iſt
               vorzüglich als Theaterkritiker – ich meine die Art, wie er ſchreibt. Seine \label{K_L02620-11v}\edtext{Kritik\pwindex{Kunst und Leben. [Burgtheater. Minna von Barnhelm]1894-10-27@\emph{Kunst und Leben. [Burgtheater. Minna von Barnhelm]} {[}1894-10-27{]}|pwv} über die \textsc{Schratt\pwindex{Schratt, Katharina 11.09.1853 – 17.04.1940@\textsc{Schratt, Katharina} (11.09.1853 – 17.04.1940), \emph{Schauspielerin}|pw}}}{\lemma{\textnormal{\emph{Kritik über die Schratt}}}\Cendnote{\textnormal{Bahr\pwindex{Bahr, Hermann 19.07.1863 – 15.01.1934@\textsc{Bahr, Hermann} (19.07.1863 – 15.01.1934), \emph{Schriftsteller, Kritiker}|pwk} schrieb in einer Nachtkritik über die
                  Neueinstudierung von \emph{Minna von Barnhelm}\pwindex{\textcolor{red}{\textsuperscript{XXXX1 indx}}!Minna von Barnhelm oder das Soldatenglueck1767@\strich\emph{Minna von Barnhelm oder das Soldatenglück} {[}1767{]}|pwk} am \emph{Burgtheater}\orgindex{Burgtheater@Burgtheater|pwk} (erstmals
                     22. 10. 1894) unter anderem: »Die Francisca, ein
                     unverwüstliches Geschöpf der \so{Hartmann}\pwindex{Hartmann, Helene 14.09.1843 – 12.03.1898@\textsc{Hartmann, Helene} (14.09.1843 – 12.03.1898), \emph{Schauspielerin}|pw}, gibt Frau \so{Schratt}\pwindex{Schratt, Katharina 11.09.1853 – 17.04.1940@\textsc{Schratt, Katharina} (11.09.1853 – 17.04.1940), \emph{Schauspielerin}|pw}. Man heißt ja jetzt
                     unpatriotisch, wenn man für Frau Schratt\pwindex{Schratt, Katharina 11.09.1853 – 17.04.1940@\textsc{Schratt, Katharina} (11.09.1853 – 17.04.1940), \emph{Schauspielerin}|pw}
                     nicht immer schwärmt, als ob das gleich weiß Gott was für eine Beleidigung
                     wäre. Nun, ich meine, Kritik darf auch vor dem Throne nicht schweigen, den der
                     Verwöhnten Schmeichler bauen. Sie ist keine Francisca. Wenn sie schmollen will,
                     keift sie, statt neckisch wird sie zänkisch und das niedliche
                     ›Frauenzimmerchen‹ bleibt die eben zu majestätische Dame schuldig.«
                        (H. B. [=Hermann Bahr]\pwindex{Bahr, Hermann 19.07.1863 – 15.01.1934@\textsc{Bahr, Hermann} (19.07.1863 – 15.01.1934), \emph{Schriftsteller, Kritiker}|pwk}: \emph{Kunst und Leben}\pwindex{Kunst und Leben. [Burgtheater. Minna von Barnhelm]1894-10-27@\emph{Kunst und Leben. [Burgtheater. Minna von Barnhelm]} {[}1894-10-27{]}|pwk}. In: \emph{Die Zeit}\pwindex{Zeit. Wiener Wochenschrift1894 – 1904@\emph{Die Zeit. Wiener Wochenschrift}|pwk}, Bd. 1, H. 4,
                        27. 10. 1894, S. 61.)}}}\label{K_L02620-11h}, ſeine
                  \label{K_L02620-44v}\edtext{Polemik\pwindex{Kunst und Leben. [Claque am Raimundtheater]1894-11-10@\emph{Kunst und Leben. [Claque am Raimundtheater]} {[}1894-11-10{]}|pwuv} mit \textsc{Mueller-Guttenbrunn\pwindex{Mueller-Guttenbrunn, Adam 22.10.1852 – 05.01.1923@\textsc{Müller-Guttenbrunn, Adam} (22.10.1852 – 05.01.1923), \emph{Schriftsteller, Theaterleiter, Beamter}|pw}}}{\lemma{\textnormal{\emph{Polemik mit Mueller-Guttenbrunn}}}\Cendnote{\textnormal{\emph{Die Zeit}\pwindex{Zeit. Wiener Wochenschrift1894 – 1904@\emph{Die Zeit. Wiener Wochenschrift}|pwk} enthält mehrere
                  Seitenhiebe gegen den Leiter des \emph{Raimund-Theater}\orgindex{Raimund-Theater@Raimund-Theater|pwk}s, Adam
                     Müller-Guttenbrunn\pwindex{Mueller-Guttenbrunn, Adam 22.10.1852 – 05.01.1923@\textsc{Müller-Guttenbrunn, Adam} (22.10.1852 – 05.01.1923), \emph{Schriftsteller, Theaterleiter, Beamter}|pwk}. Goldmann\pwindex{Goldmann, Paul 31.01.1865 – 25.09.1935@\textsc{Goldmann, Paul} (31.01.1865 – 25.09.1935), \emph{Schriftsteller, Journalist}|pwk} dürfte
                  sich auf folgende ungezeichnete Meldung beziehen: »In der ›Wiener Allgemeinen\pwindex{Wiener Allgemeine Zeitung1.3.1880 – 11.2.1934@\emph{Wiener Allgemeine Zeitung}|pw}‹ hat neulich auch Herr \so{Müller-Guttenbrunn}\pwindex{Mueller-Guttenbrunn, Adam 22.10.1852 – 05.01.1923@\textsc{Müller-Guttenbrunn, Adam} (22.10.1852 – 05.01.1923), \emph{Schriftsteller, Theaterleiter, Beamter}|pw} gespochen
                     und mit der Sicherheit, die er stets seinen Behauptungen gibt, betheuert, dass
                     das Raimund-Theater\orgindex{Raimund-Theater@Raimund-Theater|pw} keine Claque hat. Da
                     sollte Herr \so{Salten}\pwindex{Salten, Felix 06.09.1869 – 08.10.1945@\textsc{Salten, Felix} (06.09.1869 – 08.10.1945), \emph{Schriftsteller, Journalist}|pw}, von dem die hübsche Idee dieser Antikritik ist, jetzt
                     doch auch Herrn \so{Wessely}\pwindex{Wessely @\textsc{Wessely}, \emph{Claqueur}|pw} vernehmen, den sehr intelligenten und erfahrenen Chef
                     der Claque. Er kann seine Adresse von jedem Schauspieler erfahren und ihn
                     übrigens meistens in der Kanzlei des Raimundtheater\orgindex{Raimund-Theater@Raimund-Theater|pw}s treffen, wo er sich nach den Proben, die er mit Eifer
                     hört, seine Instructionen holt.« ([O. V.=Hermann Bahr]\pwindex{Bahr, Hermann 19.07.1863 – 15.01.1934@\textsc{Bahr, Hermann} (19.07.1863 – 15.01.1934), \emph{Schriftsteller, Kritiker}|pwk}: \emph{Kunst und
                        Leben}\pwindex{Kunst und Leben. [Claque am Raimundtheater]1894-11-10@\emph{Kunst und Leben. [Claque am Raimundtheater]} {[}1894-11-10{]}|pwk}. In: \emph{Die Zeit}\pwindex{Zeit. Wiener Wochenschrift1894 – 1904@\emph{Die Zeit. Wiener Wochenschrift}|pwk}, Bd. 1,
                     H. 6, 10. 11. 1894, S. 94.)}}}\label{K_L02620-44h} und
               deſſen \label{K_L02620-v}\edtext{Regiſſeur\pwindex{Langkammer, Karl 04.08.1854 – 18.05.1936@\textsc{Langkammer, Karl} (04.08.1854 – 18.05.1936), \emph{Theaterleiter/Theaterleiterin, Regisseur/Regisseurin, Schauspieler/Schauspielerin}|pwv}}{\lemma{\textnormal{\emph{Regiſſeur}}}\Cendnote{\textnormal{Hier dürfte er sich auf die lobende und positive Nachtkritik
                        (H. B. [=Hermann Bahr]\pwindex{Bahr, Hermann 19.07.1863 – 15.01.1934@\textsc{Bahr, Hermann} (19.07.1863 – 15.01.1934), \emph{Schriftsteller, Kritiker}|pwk}: \emph{Kunst und Leben}\pwindex{Kunst und Leben. [Raimundtheater. Die Eder-Mitzi]1894-11-17@\emph{Kunst und Leben. [Raimundtheater. Die Eder-Mitzi]} {[}1894-11-17{]}|pwk}. In: \emph{Die Zeit}\pwindex{Zeit. Wiener Wochenschrift1894 – 1904@\emph{Die Zeit. Wiener Wochenschrift}|pwk}, Bd. 1, H. 7,
                        17. 11. 1894, S. 108) zur Uraufführung
                  von \emph{Die Eder-Mitzi. Wiener Volksstück in vier
                     Akten}\pwindex{\textcolor{red}{\textsuperscript{XXXX1 indx}}!Eder-Mitzi. Wiener Volksstueck in vier Akten1894-11-14@\strich\emph{Die Eder-Mitzi. Wiener Volksstück in vier Akten} {[}1894-11-14{]}|pwk} am 14. 11. 1894 am \emph{Raimund-Theater}\orgindex{Raimund-Theater@Raimund-Theater|pwk}
                  beziehen. Ob Goldmann\pwindex{Goldmann, Paul 31.01.1865 – 25.09.1935@\textsc{Goldmann, Paul} (31.01.1865 – 25.09.1935), \emph{Schriftsteller, Journalist}|pwk} das Lob ironisch las,
                  ist nicht festzustellen.}}}\label{K_L02620-h} haben mich ſehr ergötzt. Aber wenn er über Kunſt
               pontificirt, iſt er mir unerträglich. Der \label{K_L02620-23v}\edtext{Artikel über
                  Dekadenz\pwindex{Bahr, Hermann 19.07.1863 – 15.01.1934@\textsc{Bahr, Hermann} (19.07.1863 – 15.01.1934), \emph{Schriftsteller, Kritiker}!Decadence1894-11-10@\strich\emph{Décadence} {[}1894-11-10{]}|pwv}}{\lemma{\textnormal{\emph{Artikel über
                  Dekadenz}}}\Cendnote{\textnormal{Hermann Bahr\pwindex{Bahr, Hermann 19.07.1863 – 15.01.1934@\textsc{Bahr, Hermann} (19.07.1863 – 15.01.1934), \emph{Schriftsteller, Kritiker}|pwk}: \emph{Décadence}\pwindex{Bahr, Hermann 19.07.1863 – 15.01.1934@\textsc{Bahr, Hermann} (19.07.1863 – 15.01.1934), \emph{Schriftsteller, Kritiker}!Decadence1894-11-10@\strich\emph{Décadence} {[}1894-11-10{]}|pwk}. In: \emph{Die Zeit}\pwindex{Zeit. Wiener Wochenschrift1894 – 1904@\emph{Die Zeit. Wiener Wochenschrift}|pwk}, Bd. 1,
                     H. 6, 10. 11. 1894, S. 87–89.}}}\label{K_L02620-23h} im
               vorletzten Heft\pwindex{Zeit. Wiener Wochenschrift1894 – 1904@\emph{Die Zeit. Wiener Wochenschrift}|pwv} iſt vorzüglich
               gemacht, ſtrotzt aber von falſchen Angaben und Urtheilen. Die \textsc{Stefan George\pwindex{George, Stefan 17.07.1868 – 04.12.1933@\textsc{George, Stefan} (17.07.1868 – 04.12.1933), \emph{Schriftsteller, Übersetzer}|pw}}, \textsc{Hermann Bang\pwindex{Bang, Herman 20.04.1857 – 29.01.1912@\textsc{Bang, Herman} (20.04.1857 – 29.01.1912), \emph{Schriftsteller}|pw}}{ }\textsc{etc.}, die er citirt, kenne ich als \textsc{\label{K_mets_Goldmann_94-partII-88v}\edtext{Faiseurs}{\lemma{\textnormal{\emph{Faiseurs}}}\Cendnote{\textnormal{französisch:
                     Blender}}}\label{K_mets_Goldmann_94-partII-88h}}{\pb}{ }\strikeout{mit} ohne jede tiefere Begabung. Wie gefällt Dir das
                  Blatt\pwindex{Zeit. Wiener Wochenschrift1894 – 1904@\emph{Die Zeit. Wiener Wochenschrift}|pwv}? Und wir gehts damit?
               Wird es ſich halten?\pend
           \pstart
           Fräulein \textsc{Sandrock\pwindex{Sandrock, Adele 19.08.1863 – 30.08.1937@\textsc{Sandrock, Adele} (19.08.1863 – 30.08.1937), \emph{Schauspielerin}|pw}} hat
               mir einen langen, ſchönen und lieben Brief geſchrieben. Bitte ſag’ ihr einſtweilen,
               wie ſehr ich mich darüber gefreut habe, und daß ich nur nach einer Stimmung ſuche, um
               nach Gebühr zu antworten. Ich will ihr nicht aus dem erſtbeſten Wochentage heraus
               ſchreiben.\pend
           \pstart
           Und bitte, ſchreib’ mir bald und viel – von Dir, von ſonſt Allem, von Wien\oindex{Wien@\textbf{Wien}|pw} und wieder von Dir. Was ſchreibſt und lieſt Du?
               Was ſoll mit den \textsc{30 fr. 30 ct} geſchehen, die Du
               bei mit gut haſt? Viele treue Grüße! Dein\pend
           \pstart \spacefill\mbox{Paul Goldmann}\pend{}\endnumbering\briefempfaengerindex{Schnitzler, Arthur@\textsc{Schnitzler, Arthur}!zzzGoldmann, Paul@\emph{von Paul Goldmann}!1894-11-181@{18. 11. 1894}|)be}\mylabel{h}\end{ledgroupsized}\begin{anhang}\end{anhang}\newcommand{\dateiname}{L02620}\newcommand{\titel}{Paul Goldmann an Arthur Schnitzler, 18. 11. 1894}\newcommand{\editorInnen}{Martin Anton Müller und Laura Untner}
            \footnotesize
\begin{ledgroupsized}[t]{11.5cm}
\doendnotes{C}
\end{ledgroupsized}
         %% latex-leseansicht-abspann.tex
%% Abspann für die Leseansicht.
%% Der Schalter \ifkorrekturansicht ist bereits durch den Vorspann gesetzt.

%% latex-abspann.tex
%% Gemeinsamer Abspann für Korrekturansicht und Leseansicht.
%% Setzt den Schalter \ifkorrekturansicht voraus (gesetzt in den
%% einbindenden Dateien latex-korrekturansicht-abspann.tex bzw.
%% latex-leseansicht-abspann.tex).
%% ---------------------------------------------------------------

\normalsize

% Das esempio-Environment wird nur in der Leseansicht benötigt
\ifkorrekturansicht\else
\newenvironment{esempio}[3]%
{
    \vspace{1.5ex}
    \rlap{\underline{#1}}
    \par
    \setlength{\parindent}{0cm}
    \nopagebreak
    \leftskip=#2cm
    \rightskip=#3cm
}
{
    \par
}
\fi

\doendnotes{C}
\bigskip
\vfill

\clearpage

\footnotesize

\ifkorrekturansicht
  \lohead{\textsc{register}}
\fi

% theindex-Environment neu definieren ohne reledmac
\makeatletter
\renewenvironment{theindex}{%
  \ifkorrekturansicht
    \section*{\indexname}%
  \else
    \subsubsection*{Index der erwähnten Entitäten}%
  \fi
  \setlength{\parindent}{0pt}%
  \setlength{\parskip}{0pt plus 0.3pt}%
  \let\item\@idxitem
}{%
  \ifkorrekturansicht\clearpage\fi
}
\makeatother

\IfFileExists{\jobname-pw.ind}{\input{\jobname-pw.ind}}{}

% Quellenangabe nur in der Leseansicht
\ifkorrekturansicht\else
% Fallback-Definitionen, falls die .tex-Datei \titel etc. nicht gesetzt hat
\providecommand{\titel}{}
\providecommand{\editorInnen}{}
\providecommand{\dateiname}{\jobname}

\vspace{3cm}

\vfill

\footnotesize
\textsc{Quelle}: \titel. Herausgegeben von {\editorInnen}. In: \emph{Arthur Schnitzler: Briefwechsel mit Autorinnen und Autoren}.
 Digitale Edition, https://schnitzler-briefe.acdh.oeaw.ac.at/{\dateiname}.html (Stand \today)
\fi

\end{document}


      