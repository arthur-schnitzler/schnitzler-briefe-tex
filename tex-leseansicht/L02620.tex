%% latex-leseansicht-vorspann.tex
%% Vorspann für die Leseansicht.
%% Lädt die gemeinsame Datei latex-vorspann.tex mit nicht gesetztem Schalter.

\newif\ifkorrekturansicht
\korrekturansichtfalse

\input{../tex-inputs/latex-vorspann}


\section[Paul Goldmann an Arthur Schnitzler, 18. 11. 1894]{L02620 Paul Goldmann an Arthur Schnitzler, 18. 11. 1894}
\nopagebreak\mylabel{L02620v}
\rehead{ }\normalsize\beginnumbering\briefempfaengerindex{Schnitzler, Arthur@\textsc{Schnitzler, Arthur}!zzzGoldmann, Paul@\emph{von Paul Goldmann}!1894-11-181@{18. 11. 1894}|(be}
\toendnotes[C]{\smallbreak\pagebreak[2]}
\correspDesc{Versand  durch Paul Goldmann am 18. 11. 1894 in Paris
\newline{}Erhalt  durch Arthur Schnitzler im Zeitraum [19. 11. 1894 – 23. 11. 1894?] in Wien}\toendnotes[C]{\smallbreak}
\Standort{DLA, A:Schnitzler, HS.NZ85.1.3164.}
\physDesc{Brief, 2 Blätter, 8 Seiten, 4452 Zeichen
\newline{}Handschrift: schwarze Tinte, deutsche Kurrent
\newline{}Schnitzler: 1) mit Bleistift auf dem ersten Blatt die Jahreszahl »94« vermerkt  2) mit rotem Buntstift sieben Unterstreichungen}\toendnotes[C]{\smallbreak}
\pstart
           {\pb}\textcolor{gray}{\textbf{Frankfurter Zeitung\orgindex{Frankfurter Zeitung@Frankfurter Zeitung|pw}}}\pend
           
\pstart
           \textcolor{gray}{\textbf{(Gazette de
                     Francfort\orgindex{Frankfurter Zeitung@Frankfurter Zeitung|pw}).}}\pend
           
\pstart
           \textcolor{gray}{\textbf{\begin{otherlanguage}{french}Fondateur\end{otherlanguage}{ }\textbf{M. L. Sonnemann\pwindex{Sonnemann, Leopold 29.\,10.\,1831 Höchberg – 30.\,10.\,1909 Frankfurt am Main@\textsc{Sonnemann, Leopold} (29.\,10.\,1831 Höchberg – 30.\,10.\,1909 Frankfurt am Main), \emph{Journalist, Herausgeber}|pw}}.}}\pend
           
\pstart
           \textcolor{gray}{\textbf{\begin{otherlanguage}{french}Journal politique, financier,\end{otherlanguage}}}\hfill \textsc{Paris\oindex{Paris@\textbf{Paris}, \emph{Hauptstadt}|pw}}, 18. November.\pend
           
\pstart
           \textcolor{gray}{\textbf{\begin{otherlanguage}{french}commercial et littéraire.\end{otherlanguage}}}\pend
           
\pstart
           \textcolor{gray}{\textbf{\begin{otherlanguage}{french}\textbf{Paraissant trois fois par jour}\end{otherlanguage}}}.\pend
           
\pstart
           \textcolor{gray}{\textbf{\begin{otherlanguage}{french}\textbf{Bureaux à Paris\oindex{Paris@\textbf{Paris}, \emph{Hauptstadt}|pw}:}\end{otherlanguage}}}\pend
           
\pstart
           \textcolor{gray}{\textbf{\begin{otherlanguage}{french}\textbf{24. Rue Feydeau}\oindex{rue Feydeau@\textbf{rue Feydeau}, \emph{Straße}|pw}.\end{otherlanguage}}}\pend
           
\pstart\center{}Mein lieber Freund,\pend\vspace{0.5em}
\pstart
           Ich will Dir täglich{ }ſchreiben und bringe die Energie dafür nicht zuſammen. Nicht
               einmal dafür! Ich bin in einem{ }ſchlimmen Gemüthszuſtande. Ich{ }ſuche nach einem
               Lebensziel und finde es nicht –{ }ſuche mich{ }ſelbſt zu beſchränken, zu erkennen, zu
               ordnen und kann es nicht – und nach kurzen Anläufen falle ich in Zeitvergeudung,
               Außenleben und Wirrniß zurück. Dabei werde ich alle paar Tage daran erinnert, daß ich
               dreißig Jahre bin, nichts geleiſtet habe, zurückbleibe hinter allen Andern. Es iſt
               ein zerſtörendes Gefühl, und doch finde ich die {\pb}Kraft nicht zum Arbeiten. Die Zeit hätte ich jetzt, – alſo es gibt keine
               Entſchuldigung mehr. Das hindert mich an Allem,{ }ſelbſt am Briefeſchreiben. Du
               begreifſt mich gewiß.\pend
           
\pstart
           Ich raffe mich heut ein wenig zuſammen; denn ich
               möchte gar{ }ſo gern hören, wie es mit Deinem Stücke\pwindex{Schnitzler, Arthur 15.\,5.\,1862 Wien – 21.\,10.\,1931 ebd.@\textsc{Schnitzler, Arthur} (15.\,5.\,1862 Wien – 21.\,10.\,1931 ebd.), \emph{Schriftsteller, Mediziner}!Liebelei. Schauspiel in drei Akten@\strich\emph{Liebelei. Schauspiel in drei Akten}|pwv} weitergeht. Was Du mir über Deine \label{K_L02620-1v}\edtext{erſte Unterredung mit \textsc{B.\pwindex{Burckhard, Max Eugen 14.\,7.\,1854 Korneuburg – 16.\,3.\,1912 Wien@\textsc{Burckhard, Max Eugen} (14.\,7.\,1854 Korneuburg – 16.\,3.\,1912 Wien), \emph{Schriftsteller, Rechtswissenschaftler, Theaterleiter}|pw}}}{\lemma{\textnormal{\emph{erste Unterredung mit B.}}}\Cendnote{\textnormal{Siehe A. S.: \emph{Tagebuch}, 5. 11. 1894.
               }}}\label{K_L02620-1} geſchrieben, erſcheint mir ganz und gar nicht ungünſtig. Daß es nicht{ }ſo
               glatt gehen würde, war{ }ſelbſtverſtändlich. Dabei geht es doch noch relativ glatt.
               Wenn man in einem Theater den Director für{ }ſich hat,{ }ſo iſt das, denke ich, Chance
               genug. Das {\pb}Übrige iſt Zopf und \textsc{chinoiserie}. Dafür{ }ſind wir ja im guten Lande Öſterreich\oindex{Österreich@\textbf{Österreich}|pw}. Wüßteſt Du nur, was hier die jungen Leute dulden
               müſſen, ehe{ }ſie aufgeführt werden. An die \textsc{Comédie Française\orgindex{Comédie-Française@Comédie-Française|pw}} kommt überhaupt keiner heran, wenn ihn nicht ein \label{K_L02620-2v}\edtext{Akademi\orgindex{Académie Française@Académie Française|pwv}ker}{\lemma{\textnormal{\emph{Akademiker}}}\Cendnote{\textnormal{Mitglied der \emph{Académie Française}\orgindex{Académie Française@Académie Française|pwk}}}}\label{K_L02620-2} oder ein großer Komödiant protegirt, und \strikeout{\textsc{Henr}} der alte \textsc{Henri Becque\pwindex{Becque, Henry 9.\,4.\,1837 Paris – 12.\,5.\,1899 Neuilly-sur-Seine@\textsc{Becque, Henry} (9.\,4.\,1837 Paris – 12.\,5.\,1899 Neuilly-sur-Seine), \emph{Schriftsteller, Dramatiker}|pw}}{ }ſelbſt hat{ }ſeinerzeit die Aufführung von »\textsc{La Parisienne\pwindex{Becque, Henry 9.\,4.\,1837 Paris – 12.\,5.\,1899 Neuilly-sur-Seine@\textsc{Becque, Henry} (9.\,4.\,1837 Paris – 12.\,5.\,1899 Neuilly-sur-Seine), \emph{Schriftsteller, Dramatiker}!Parisienne@\strich\emph{La Parisienne}|pw}}« durch ein Machtwort des Miniſters\pwindex{Bourgeois, Léon 29.\,5.\,1851 Paris – 29.\,9.\,1925 Épernay@\textsc{Bourgeois, Léon} (29.\,5.\,1851 Paris – 29.\,9.\,1925 Épernay), \emph{Politiker, Minister, Nobelpreisträger}|pwv} erzwingen müſſen. Es gibt keinen Erfolg, zu dem man nicht über
               Hintertreppen{ }ſteigen müßte, beſonders beim Theater. Thut mir nur leid, daß ich nicht
               gerade jetzt um Dich bin, um {\pb}mit Dir über all’ die
               Trottelhaftigkeiten zu lachen, die Dir vorausſichtlich werden geſagt oder angethan
               werden, und vielleicht auch um Dir ein paar unangenehme Wege zu erſparen. Übrigens
               nimmſt Du es ja{ }ſelbſt ironiſch, und das iſt das Beſte. Bitte,{ }ſchreib’ mir nur
               raſch, wieweit die Sache\pwindex{Schnitzler, Arthur 15.\,5.\,1862 Wien – 21.\,10.\,1931 ebd.@\textsc{Schnitzler, Arthur} (15.\,5.\,1862 Wien – 21.\,10.\,1931 ebd.), \emph{Schriftsteller, Mediziner}!Liebelei. Schauspiel in drei Akten@\strich\emph{Liebelei. Schauspiel in drei Akten}|pwv} iſt.
               Und möchteſt Du es nicht doch zugleich in \label{K_L02620-3v}\edtext{Berlin\oindex{Berlin@\textbf{Berlin}, \emph{Hauptstadt}|pw} einreichen}{\lemma{\textnormal{\emph{Berlin einreichen}}}\Cendnote{\textnormal{Vgl. XXXX Auszeichnungsfehler: Dokument L02616 nicht gefunden.
               }}}\label{K_L02620-3}?\pend
           
\pstart
           Geſtern habe ich die \label{K_L02620-4v}\edtext{Fortſetzung von »Sterben\pwindex{Schnitzler, Arthur 15.\,5.\,1862 Wien – 21.\,10.\,1931 ebd.@\textsc{Schnitzler, Arthur} (15.\,5.\,1862 Wien – 21.\,10.\,1931 ebd.), \emph{Schriftsteller, Mediziner}!Sterben. Novelle@\strich\emph{Sterben. Novelle}|pw}«}{\lemma{\textnormal{\emph{Fortsetzung von »Sterben«}}}\Cendnote{\textnormal{Der zweite Teil (von
                  drei) erschien Anfang November in der \emph{Neuen deutschen Rundschau}\pwindex{Neue Deutsche Rundschau@\emph{Neue Deutsche Rundschau}|pwk} (H. 11,
                     S. 1073–1101).}}}\label{K_L02620-4} geleſen. Es iſt dumm, daß man es mit
               Zwiſchenräumen \strikeout{von} von einem Monat leſen muß. Ich bin
               mir über den Eindruck infolgedeſſen jetzt weniger {\pb}klar, als am Anfang. Ich weiß nur, daß ich im Einzelnen Entzückendes und Großes
               finde. Auch iſt der Styl köſtlich in{ }ſeiner Einfachheit, mit all’ den Tiefen
               darunter. \strikeout{Ein \textcolor{gray}{×}} Hier und da iſt es mir aber doch zu einfach. Zum Beiſpiel: \textsc{Salzburg\oindex{Salzburg@\textbf{Salzburg}, \emph{Verwaltungsgebiet}|pw}}, ich meine das Landſchaftliche und Äußerliche, iſt meiner Empfindung nach um
               eine \textsc{Nuance} zu blaß gerathen. Alles in Allem ein reifes und
               ernſtes Werk\pwindex{Schnitzler, Arthur 15.\,5.\,1862 Wien – 21.\,10.\,1931 ebd.@\textsc{Schnitzler, Arthur} (15.\,5.\,1862 Wien – 21.\,10.\,1931 ebd.), \emph{Schriftsteller, Mediziner}!Liebelei. Schauspiel in drei Akten@\strich\emph{Liebelei. Schauspiel in drei Akten}|pwv}. Aber, wie
               geſagt, ich muß es als Buch im Zuſammenhange leſen. Mir ahnt nur, daß ich es{ }ſchön
               finden werde, {\pb}aber ich habe noch kein klares
               Bewußtſein davon. Dieſe verfluchten Fortſetzungen! Eine kleine Äußerlichkeit: bei der
               Buchausgabe \label{K_L02620-5v}\edtext{ſtreiche}{\lemma{\textnormal{\emph{streiche}}}\Cendnote{\textnormal{Schnitzler veränderte die Stelle für die
                  Buchausgabe nicht.}}}\label{K_L02620-5} auf Seite 1077\pwindex{Schnitzler, Arthur 15.\,5.\,1862 Wien – 21.\,10.\,1931 ebd.@\textsc{Schnitzler, Arthur} (15.\,5.\,1862 Wien – 21.\,10.\,1931 ebd.), \emph{Schriftsteller, Mediziner}!Liebelei. Schauspiel in drei Akten@\strich\emph{Liebelei. Schauspiel in drei Akten}|pwv} in der 20ten Zeile \uline{von unten}
               hinter »Einwohner« die Worte »der Stadt« weg; es iſt zu viel »Stadt« in dem
               Abſatz.\pend
           
\pstart
           Wann kriege, ich nun wohl das Stück\pwindex{Schnitzler, Arthur 15.\,5.\,1862 Wien – 21.\,10.\,1931 ebd.@\textsc{Schnitzler, Arthur} (15.\,5.\,1862 Wien – 21.\,10.\,1931 ebd.), \emph{Schriftsteller, Mediziner}!Liebelei. Schauspiel in drei Akten@\strich\emph{Liebelei. Schauspiel in drei Akten}|pwv} zu leſen?\pend
           
\pstart
           Mein \strikeout{\textcolor{gray}{On}}{ }Onkel\pwindex{Mamroth, Fedor 21.\,2.\,1851 Breslau – 25.\,6.\,1907 Frankfurt am Main@\textsc{Mamroth, Fedor} (21.\,2.\,1851 Breslau – 25.\,6.\,1907 Frankfurt am Main), \emph{Journalist, Kritiker}|pwv} hat mich vor vier
               Wochen nach Deiner Adreſſe gefragt, um Dir \label{K_L02620-6v}\edtext{Bücher zu{ }ſchicken}{\lemma{\textnormal{\emph{Bücher zu schicken}}}\Cendnote{\textnormal{Vgl. XXXX Auszeichnungsfehler: Dokument L02614 nicht gefunden.
               }}}\label{K_L02620-6}. Da ich aber wieder einmal mit ihm grolle, habe ich nicht geantwortet.
               Hätteſt Du nicht irgend einen Vorwand ihm zu \label{K_L02620-7v}\edtext{ſchreiben}{\lemma{\textnormal{\emph{schreiben}}}\Cendnote{\textnormal{Siehe XXXX Auszeichnungsfehler: Dokument L00409 nicht gefunden.
               }}}\label{K_L02620-7}{ }\strikeout{u\textcolor{gray}{n}}, damit er zugleich {\pb}Deine Adreſſe\oindex{Wien@\textbf{Wien}!IX., Alsergrund@\textbf{IX., Alsergrund}!Frankgasse 1@\textbf{Frankgasse 1}, \emph{Wohngebäude}|pwv} erführe?\pend
           
\pstart
           Die »Zeit\orgindex{Zeit. Wiener Wochenschrift@Die Zeit. Wiener Wochenschrift|pw}« gefällt mir ganz ausnehmend. Das iſt
               ein Blatt\pwindex{Zeit. Wiener Wochenschrift@\emph{Die Zeit. Wiener Wochenschrift}|pwv}, durchaus nach
               meinem Sinn. \textsc{Kanner\pwindex{Kanner, Heinrich 9.\,11.\,1864 Galați – 15.\,2.\,1930 Wien@\textsc{Kanner, Heinrich} (9.\,11.\,1864 Galați – 15.\,2.\,1930 Wien), \emph{Herausgeber, Publizist}|pw}} übertrifft{ }ſich{ }ſelbſt, \textsc{Bahr\pwindex{Bahr, Hermann 19.\,7.\,1863 Linz – 15.\,1.\,1934 München@\textsc{Bahr, Hermann} (19.\,7.\,1863 Linz – 15.\,1.\,1934 München), \emph{Schriftsteller, Kritiker}|pw}} iſt vorzüglich als Theaterkritiker – ich meine die Art, wie er{ }ſchreibt. Seine
                  \label{K_L02620-8v}\edtext{Kritik\pwindex{Bahr, Hermann 19.\,7.\,1863 Linz – 15.\,1.\,1934 München@\textsc{Bahr, Hermann} (19.\,7.\,1863 Linz – 15.\,1.\,1934 München), \emph{Schriftsteller, Kritiker}!Kunst und Leben. [Burgtheater.] [Minna von Barnhelm]@\strich\emph{Kunst und Leben. [Burgtheater.] [Minna von Barnhelm]}|pwv} über die \textsc{Schratt\pwindex{Schratt, Katharina 11.\,9.\,1853 Baden bei Wien – 17.\,4.\,1940 Wien@\textsc{Schratt, Katharina} (11.\,9.\,1853 Baden bei Wien – 17.\,4.\,1940 Wien), \emph{Schauspielerin}|pw}}}{\lemma{\textnormal{\emph{Kritik über die Schratt}}}\Cendnote{\textnormal{Bahr\pwindex{Bahr, Hermann 19.\,7.\,1863 Linz – 15.\,1.\,1934 München@\textsc{Bahr, Hermann} (19.\,7.\,1863 Linz – 15.\,1.\,1934 München), \emph{Schriftsteller, Kritiker}|pwk} schrieb in einer Nachtkritik\pwindex{Bahr, Hermann 19.\,7.\,1863 Linz – 15.\,1.\,1934 München@\textsc{Bahr, Hermann} (19.\,7.\,1863 Linz – 15.\,1.\,1934 München), \emph{Schriftsteller, Kritiker}!Kunst und Leben. [Burgtheater.] [Minna von Barnhelm]@\strich\emph{Kunst und Leben. [Burgtheater.] [Minna von Barnhelm]}|pwkv} über die Neueinstudierung
                  von \emph{Minna von Barnhelm}\pwindex{\textcolor{red}{\textsuperscript{XXXX indx1}}!Minna von Barnhelm oder das Soldatenglück@\strich\emph{Minna von Barnhelm oder das Soldatenglück}|pwk} am \emph{Burgtheater}\orgindex{Burgtheater@Burgtheater|pwk} (erstmals 22. 10. 1894) unter
                  anderem, ziemlich unverhohlen auf die Liaison von Schratt\pwindex{Schratt, Katharina 11.\,9.\,1853 Baden bei Wien – 17.\,4.\,1940 Wien@\textsc{Schratt, Katharina} (11.\,9.\,1853 Baden bei Wien – 17.\,4.\,1940 Wien), \emph{Schauspielerin}|pwk} mit dem Kaiser Joseph I.\pwindex{Franz Joseph I. von Österreich-Ungarn 18.\,8.\,1830 Wien – 21.\,11.\,1916 ebd.@\textsc{Franz Joseph I. von Österreich-Ungarn} (18.\,8.\,1830 Wien – 21.\,11.\,1916 ebd.), \emph{Kaiser}|pwk}
                  anspielend: »Die Francisca\pwindex{\textcolor{red}{\textsuperscript{XXXX indx1}}!Minna von Barnhelm oder das Soldatenglück@\strich\emph{Minna von Barnhelm oder das Soldatenglück}|pwv}, ein unverwüstliches Geschöpf der \so{Hartmann}\pwindex{Hartmann, Helene 14.\,9.\,1843 Mannheim – 12.\,3.\,1898 Wien@\textsc{Hartmann, Helene} (14.\,9.\,1843 Mannheim – 12.\,3.\,1898 Wien), \emph{Schauspielerin}|pw}, gibt Frau \so{Schratt}\pwindex{Schratt, Katharina 11.\,9.\,1853 Baden bei Wien – 17.\,4.\,1940 Wien@\textsc{Schratt, Katharina} (11.\,9.\,1853 Baden bei Wien – 17.\,4.\,1940 Wien), \emph{Schauspielerin}|pw}. Man heißt ja jetzt unpatriotisch, wenn man für Frau Schratt\pwindex{Schratt, Katharina 11.\,9.\,1853 Baden bei Wien – 17.\,4.\,1940 Wien@\textsc{Schratt, Katharina} (11.\,9.\,1853 Baden bei Wien – 17.\,4.\,1940 Wien), \emph{Schauspielerin}|pw} nicht immer schwärmt, als ob das gleich weiß Gott
                     was für eine Beleidigung wäre. Nun, ich meine, Kritik darf auch vor dem Throne
                     nicht schweigen, den der Verwöhnten Schmeichler bauen. Sie ist keine Francisca\pwindex{\textcolor{red}{\textsuperscript{XXXX indx1}}!Minna von Barnhelm oder das Soldatenglück@\strich\emph{Minna von Barnhelm oder das Soldatenglück}|pwv}. Wenn sie
                     schmollen will, keift sie, statt neckisch wird sie zänkisch und das niedliche
                     ›Frauenzimmerchen‹ bleibt die eben zu majestätische Dame schuldig.« (H. B. [ = Hermann Bahr]\pwindex{Bahr, Hermann 19.\,7.\,1863 Linz – 15.\,1.\,1934 München@\textsc{Bahr, Hermann} (19.\,7.\,1863 Linz – 15.\,1.\,1934 München), \emph{Schriftsteller, Kritiker}|pwk}: \emph{Kunst und Leben}\pwindex{Bahr, Hermann 19.\,7.\,1863 Linz – 15.\,1.\,1934 München@\textsc{Bahr, Hermann} (19.\,7.\,1863 Linz – 15.\,1.\,1934 München), \emph{Schriftsteller, Kritiker}!Kunst und Leben. [Burgtheater.] [Minna von Barnhelm]@\strich\emph{Kunst und Leben. [Burgtheater.] [Minna von Barnhelm]}|pwk}. In: \emph{Die Zeit}\pwindex{Zeit. Wiener Wochenschrift@\emph{Die Zeit. Wiener Wochenschrift}|pwk}, Bd. 1, H. 4, 27. 10. 1894,
               S. 61.)}}}\label{K_L02620-8},{ }ſeine \label{K_L02620-9v}\edtext{Polemik\pwindex{Kunst und Leben. [Claque am Raimundtheater]@\emph{Kunst und Leben. [Claque am Raimundtheater]}|pwuv} mit \textsc{Mueller-Guttenbrunn\pwindex{Müller-Guttenbrunn, Adam 22.\,10.\,1852 Zăbrani – 5.\,1.\,1923 Wien@\textsc{Müller-Guttenbrunn, Adam} (22.\,10.\,1852 Zăbrani – 5.\,1.\,1923 Wien), \emph{Schriftsteller, Theaterleiter, Beamter}|pw}}}{\lemma{\textnormal{\emph{Polemik mit Mueller-Guttenbrunn}}}\Cendnote{\textnormal{\emph{Die Zeit}\pwindex{Zeit. Wiener Wochenschrift@\emph{Die Zeit. Wiener Wochenschrift}|pwk} enthält mehrere Seitenhiebe gegen
                  den Leiter des \emph{Raimund-Theaters}\orgindex{Raimund-Theater@Raimund-Theater|pwk}, Adam Müller-Guttenbrunn\pwindex{Müller-Guttenbrunn, Adam 22.\,10.\,1852 Zăbrani – 5.\,1.\,1923 Wien@\textsc{Müller-Guttenbrunn, Adam} (22.\,10.\,1852 Zăbrani – 5.\,1.\,1923 Wien), \emph{Schriftsteller, Theaterleiter, Beamter}|pwk}. Goldmann\pwindex{Goldmann, Paul 31.\,1.\,1865 Breslau – 25.\,9.\,1935 Wien@\textsc{Goldmann, Paul} (31.\,1.\,1865 Breslau – 25.\,9.\,1935 Wien), \emph{Schriftsteller, Journalist}|pwk} dürfte sich auf folgende ungezeichnete Meldung\pwindex{Kunst und Leben. [Claque am Raimundtheater]@\emph{Kunst und Leben. [Claque am Raimundtheater]}|pwkv} beziehen: »In
                     der ›Wiener Allgemeinen\pwindex{Wiener Allgemeine Zeitung@\emph{Wiener Allgemeine Zeitung}|pw}‹ hat neulich auch
                     Herr \so{Müller-Guttenbrunn}\pwindex{Müller-Guttenbrunn, Adam 22.\,10.\,1852 Zăbrani – 5.\,1.\,1923 Wien@\textsc{Müller-Guttenbrunn, Adam} (22.\,10.\,1852 Zăbrani – 5.\,1.\,1923 Wien), \emph{Schriftsteller, Theaterleiter, Beamter}|pw} gespochen und mit der Sicherheit, die er stets seinen Behauptungen gibt,
                     betheuert, dass das Raimund-Theater\orgindex{Raimund-Theater@Raimund-Theater|pw} keine
                     Claque hat. Da sollte Herr \so{Salten}\pwindex{Salten, Felix 6.\,9.\,1869 Budapest – 8.\,10.\,1945 Zürich@\textsc{Salten, Felix} (6.\,9.\,1869 Budapest – 8.\,10.\,1945 Zürich), \emph{Schriftsteller, Journalist, Chefredakteur}|pw}, von dem die hübsche Idee dieser Antikritik ist, jetzt doch auch Herrn
                        \so{Wessely}\pwindex{Wessely @\textsc{Wessely}, \emph{Claqueur}|pw} vernehmen, den sehr intelligenten und erfahrenen Chef der Claque. Er kann
                     seine Adresse von jedem Schauspieler erfahren und ihn übrigens meistens in der
                     Kanzlei des Raimundtheaters\orgindex{Raimund-Theater@Raimund-Theater|pw} treffen, wo er
                     sich nach den Proben, die er mit Eifer hört, seine Instructionen holt.«
                     ([Hermann Bahr\pwindex{Bahr, Hermann 19.\,7.\,1863 Linz – 15.\,1.\,1934 München@\textsc{Bahr, Hermann} (19.\,7.\,1863 Linz – 15.\,1.\,1934 München), \emph{Schriftsteller, Kritiker}|pwk}]: \emph{Kunst und Leben}\pwindex{Kunst und Leben. [Claque am Raimundtheater]@\emph{Kunst und Leben. [Claque am Raimundtheater]}|pwk}. In: \emph{Die Zeit}\pwindex{Zeit. Wiener Wochenschrift@\emph{Die Zeit. Wiener Wochenschrift}|pwk}, Bd. 1, H. 6, 10. 11. 1894,
                     S. 94.)}}}\label{K_L02620-9} und deſſen \label{K_L02620-10v}\edtext{Regiſſeur\pwindex{Langkammer, Karl 4.\,8.\,1854 Wien – 18.\,5.\,1936 ebd.@\textsc{Langkammer, Karl} (4.\,8.\,1854 Wien – 18.\,5.\,1936 ebd.), \emph{Theaterleiter, Regisseur, Schauspieler}|pwv}}{\lemma{\textnormal{\emph{Regisseur}}}\Cendnote{\textnormal{Hier dürfte er sich auf die lobende und
                  positive Nachtkritik\pwindex{Bahr, Hermann 19.\,7.\,1863 Linz – 15.\,1.\,1934 München@\textsc{Bahr, Hermann} (19.\,7.\,1863 Linz – 15.\,1.\,1934 München), \emph{Schriftsteller, Kritiker}!Kunst und Leben. [Raimundtheater.] [Die Eder-Mitzi]@\strich\emph{Kunst und Leben. [Raimundtheater.] [Die Eder-Mitzi]}|pwkv} (H. B. [ = Hermann Bahr]\pwindex{Bahr, Hermann 19.\,7.\,1863 Linz – 15.\,1.\,1934 München@\textsc{Bahr, Hermann} (19.\,7.\,1863 Linz – 15.\,1.\,1934 München), \emph{Schriftsteller, Kritiker}|pwk}: \emph{Kunst und Leben}\pwindex{Bahr, Hermann 19.\,7.\,1863 Linz – 15.\,1.\,1934 München@\textsc{Bahr, Hermann} (19.\,7.\,1863 Linz – 15.\,1.\,1934 München), \emph{Schriftsteller, Kritiker}!Kunst und Leben. [Raimundtheater.] [Die Eder-Mitzi]@\strich\emph{Kunst und Leben. [Raimundtheater.] [Die Eder-Mitzi]}|pwk}. In: \emph{Die Zeit}\pwindex{Zeit. Wiener Wochenschrift@\emph{Die Zeit. Wiener Wochenschrift}|pwk}, Bd. 1, H. 7, 17. 11. 1894, S. 108) zur
                   Uraufführung\eventindex{Raimund-Theater@\textbf{Raimund-Theater}!Uraufführung von Die Eder-Mitzi, 14.11.1894@Uraufführung von Die Eder-Mitzi, 14.11.1894|pwkv} von \emph{Die Eder-Mitzi. Wiener Volksstück
                     in vier Akten}\pwindex{\textcolor{red}{\textsuperscript{XXXX indx1}}!Eder-Mitzi. Wiener Volksstück in vier Akten@\strich\emph{Die Eder-Mitzi. Wiener Volksstück in vier Akten}|pwk} am 14. 11. 1894 am \emph{Raimund-Theater}\orgindex{Raimund-Theater@Raimund-Theater|pwk}
                  beziehen. Ob Goldmann\pwindex{Goldmann, Paul 31.\,1.\,1865 Breslau – 25.\,9.\,1935 Wien@\textsc{Goldmann, Paul} (31.\,1.\,1865 Breslau – 25.\,9.\,1935 Wien), \emph{Schriftsteller, Journalist}|pwk} das Lob ironisch las,
                  ist nicht festzustellen.}}}\label{K_L02620-10} haben mich{ }ſehr ergötzt. Aber wenn er über Kunſt
               pontificirt, iſt er mir unerträglich. Der \label{K_L02620-11v}\edtext{Artikel\pwindex{Bahr, Hermann 19.\,7.\,1863 Linz – 15.\,1.\,1934 München@\textsc{Bahr, Hermann} (19.\,7.\,1863 Linz – 15.\,1.\,1934 München), \emph{Schriftsteller, Kritiker}!Décadence@\strich\emph{Décadence}|pwv} über Dekadenz}{\lemma{\textnormal{\emph{Artikel über Dekadenz}}}\Cendnote{\textnormal{Hermann Bahr\pwindex{Bahr, Hermann 19.\,7.\,1863 Linz – 15.\,1.\,1934 München@\textsc{Bahr, Hermann} (19.\,7.\,1863 Linz – 15.\,1.\,1934 München), \emph{Schriftsteller, Kritiker}|pwk}: \emph{Décadence}\pwindex{Bahr, Hermann 19.\,7.\,1863 Linz – 15.\,1.\,1934 München@\textsc{Bahr, Hermann} (19.\,7.\,1863 Linz – 15.\,1.\,1934 München), \emph{Schriftsteller, Kritiker}!Décadence@\strich\emph{Décadence}|pwk}. In: \emph{Die
                        Zeit}\pwindex{Zeit. Wiener Wochenschrift@\emph{Die Zeit. Wiener Wochenschrift}|pwk}, Bd. 1, H. 6, 10. 11. 1894, S. 87–89.}}}\label{K_L02620-11}
               im vorletzten Heft\pwindex{Zeit. Wiener Wochenschrift@\emph{Die Zeit. Wiener Wochenschrift}|pwv} iſt
               vorzüglich gemacht,{ }ſtrotzt aber von falſchen Angaben und Urtheilen. Die \textsc{Stefan George\pwindex{George, Stefan 17.\,7.\,1868 Büdesheim – 4.\,12.\,1933 Minusio@\textsc{George, Stefan} (17.\,7.\,1868 Büdesheim – 4.\,12.\,1933 Minusio), \emph{Schriftsteller, Übersetzer}|pw}}, \textsc{Hermann Bang\pwindex{Bang, Herman 20.\,4.\,1857 Asserballe – 29.\,1.\,1912 Ogden@\textsc{Bang, Herman} (20.\,4.\,1857 Asserballe – 29.\,1.\,1912 Ogden), \emph{Schriftsteller}|pw}}{ }\textsc{etc.}, die er citirt, kenne ich als \textsc{\label{K_L02620-12v}\edtext{Faiseurs}{\lemma{\textnormal{\emph{Faiseurs}}}\Cendnote{\textnormal{französisch: Blender}}}\label{K_L02620-12}}{ }{\pb}\strikeout{mit} ohne jede tiefere Begabung. Wie gefällt Dir das
                  Blatt\pwindex{Zeit. Wiener Wochenschrift@\emph{Die Zeit. Wiener Wochenschrift}|pwv}? Und wie gehts damit?
               Wird es{ }ſich halten?\pend
           
\pstart
           Fräulein \textsc{Sandrock\pwindex{Sandrock, Adele 19.\,8.\,1863 Rotterdam – 30.\,8.\,1937 Berlin@\textsc{Sandrock, Adele} (19.\,8.\,1863 Rotterdam – 30.\,8.\,1937 Berlin), \emph{Schauspielerin}|pw}} hat mir einen langen,{ }ſchönen und lieben Brief geſchrieben. Bitte{ }ſag’ ihr
               einſtweilen, wie{ }ſehr ich mich darüber gefreut habe, und daß ich nur nach einer
               Stimmung{ }ſuche, um nach Gebühr zu antworten. Ich will ihr nicht aus dem erſtbeſten
               Wochentage heraus{ }ſchreiben.\pend
           
\pstart
           Und bitte,{ }ſchreib’ mir bald und viel – von Dir, von{ }ſonſt Allem, von Wien\oindex{Wien@\textbf{Wien}, \emph{Verwaltungsgebiet}|pw} und wieder von Dir. Was{ }ſchreibſt und lieſt
               Du? Was{ }ſoll mit den \textsc{30 fr. 30 ct} geſchehen, die Du bei mit
               gut haſt? Viele treue Grüße! Dein\pend
           \pstart \spacefill\mbox{Paul Goldmann}\pend{}\selectlanguage{ngerman}\endnumbering\briefempfaengerindex{Schnitzler, Arthur@\textsc{Schnitzler, Arthur}!zzzGoldmann, Paul@\emph{von Paul Goldmann}!1894-11-181@{18. 11. 1894}|)be}\mylabel{L02620h}  \newcommand{\dateiname}{L02620}\newcommand{\titel}{Paul Goldmann an Arthur Schnitzler, 18. 11. 1894}\newcommand{\editorInnen}{Martin Anton Müller und Laura Untner}%% latex-leseansicht-abspann.tex
%% Abspann für die Leseansicht.
%% Der Schalter \ifkorrekturansicht ist bereits durch den Vorspann gesetzt.

%% latex-abspann.tex
%% Gemeinsamer Abspann für Korrekturansicht und Leseansicht.
%% Setzt den Schalter \ifkorrekturansicht voraus (gesetzt in den
%% einbindenden Dateien latex-korrekturansicht-abspann.tex bzw.
%% latex-leseansicht-abspann.tex).
%% ---------------------------------------------------------------

\normalsize

% Das esempio-Environment wird nur in der Leseansicht benötigt
\ifkorrekturansicht\else
\newenvironment{esempio}[3]%
{
    \vspace{1.5ex}
    \rlap{\underline{#1}}
    \par
    \setlength{\parindent}{0cm}
    \nopagebreak
    \leftskip=#2cm
    \rightskip=#3cm
}
{
    \par
}
\fi

\doendnotes{C}
\bigskip
\vfill

\clearpage

\footnotesize

\ifkorrekturansicht
  \lohead{\textsc{register}}
\fi

% theindex-Environment neu definieren ohne reledmac
\makeatletter
\renewenvironment{theindex}{%
  \ifkorrekturansicht
    \section*{\indexname}%
  \else
    \subsubsection*{Index der erwähnten Entitäten}%
  \fi
  \setlength{\parindent}{0pt}%
  \setlength{\parskip}{0pt plus 0.3pt}%
  \let\item\@idxitem
}{%
  \ifkorrekturansicht\clearpage\fi
}
\makeatother

\IfFileExists{\jobname-pw.ind}{\input{\jobname-pw.ind}}{}

% Quellenangabe nur in der Leseansicht
\ifkorrekturansicht\else
% Fallback-Definitionen, falls die .tex-Datei \titel etc. nicht gesetzt hat
\providecommand{\titel}{}
\providecommand{\editorInnen}{}
\providecommand{\dateiname}{\jobname}

\vspace{3cm}

\vfill

\footnotesize
\textsc{Quelle}: \titel. Herausgegeben von {\editorInnen}. In: \emph{Arthur Schnitzler: Briefwechsel mit Autorinnen und Autoren}.
 Digitale Edition, https://schnitzler-briefe.acdh.oeaw.ac.at/{\dateiname}.html (Stand \today)
\fi

\end{document}


