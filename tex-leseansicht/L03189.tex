%% latex-leseansicht-vorspann.tex
%% Vorspann für die Leseansicht.
%% Lädt die gemeinsame Datei latex-vorspann.tex mit nicht gesetztem Schalter.

\newif\ifkorrekturansicht
\korrekturansichtfalse

\input{../tex-inputs/latex-vorspann}


         
         \renewcommand{\erwaehntePersonen}{Personen: Paul Goldmann, Hermann Mamroth}
         \renewcommand{\erwaehnteOrte}{Orte: Berlin, Bernburger Straße, Dessauer Straße, Deutsches Theater Berlin, Frankfurt am Main, Hotel Savoy}
         \renewcommand{\erwaehnteWerke}{Werke: Lebendige Stunden. Vier Einakter, Tagebuch}
               \section[ Paul Goldmann an Arthur Schnitzler, 1. 1. {[}1902{]}]{ Paul Goldmann an Arthur Schnitzler, 1. 1. {[}1902{]}}\nopagebreak\mylabel{v}\rehead{ }\begin{ledgroupsized}[t]{13cm}\normalsize\beginnumbering \toendnotes[C]{\smallbreak\pagebreak[2]} \Standort{DLA, A:Schnitzler, HS.NZ85.1.3172.}
\physDesc{Brief, 1 Blatt, 1 Seite, 424 Zeichen
\newline{}Handschrift: blaue Tinte, deutsche Kurrent
\newline{}Schnitzler: mit Bleistift das Jahr »902« vermerkt }\toendnotes[C]{\smallbreak}\pstart
           \centering{}{\pb}Frankfurt\oindex{Frankfurt am Main@\textbf{Frankfurt am Main}|pw}{ }1. Januar.\pend
           \pstart\center{}Mein lieber Freund,\pend\pstart
           Bitte, nimm’ den \label{K_L03189-1v}\edtext{Sitz}{\lemma{\textnormal{\emph{Sitz}}}\Cendnote{\textnormal{Für die Uraufführung von \emph{Lebendige Stunden}\pwindex{Schnitzler, Arthur 15.05.1862 – 21.10.1931@\textsc{Schnitzler, Arthur} (15.05.1862 – 21.10.1931), \emph{Schriftsteller, Mediziner}!Lebendige Stunden. Vier Einakter1901-12-23@\strich\emph{Lebendige Stunden. Vier Einakter} {[}1901-12-23{]}|pwk} am Deutschen Theater Berlin\oindex{Deutsches Theater Berlin@\textbf{Deutsches Theater Berlin}|pwk}. }}}\label{K_L03189-1h}, den Du neben dem meinigen (N\textsuperscript{o} 95, 10. Reihe) haſt reſerviren laſſen und ſende ihn an
               meinen Onkel\pwindex{Mamroth, Hermann @\textsc{Mamroth, Hermann}|pwv}, Herrn \textsc{Hermann Mamroth\pwindex{Mamroth, Hermann @\textsc{Mamroth, Hermann}|pw}}, \textsc{Berlin S. W.\oindex{Berlin@\textbf{Berlin}|pw}}, \textsc{Bernburgerstraße} 28\oindex{Bernburger Strasse@\textbf{Bernburger Straße}|pw}. Wir verrechnen uns nach meiner
               Rückkunft.\pend
           \pstart
           Bitte, ſchreibe mir nach meiner Berliner Wohnung\oindex{Dessauer Strasse@\textbf{Dessauer Straße}|pwv} ein Wort, wo ich Dich am Samſtag{ }\label{K_L03189-2v}\edtext{nach der Vorſtellung\pwindex{Schnitzler, Arthur 15.05.1862 – 21.10.1931@\textsc{Schnitzler, Arthur} (15.05.1862 – 21.10.1931), \emph{Schriftsteller, Mediziner}!Lebendige Stunden. Vier Einakter1901-12-23@\strich\emph{Lebendige Stunden. Vier Einakter} {[}1901-12-23{]}|pwv}}{\lemma{\textnormal{\emph{nach der Vorſtellung}}}\Cendnote{\textnormal{Hinterher war Schnitzler\pwindex{Schnitzler, Arthur 15.05.1862 – 21.10.1931@\textsc{Schnitzler, Arthur} (15.05.1862 – 21.10.1931), \emph{Schriftsteller, Mediziner}|pwk} im Hotel
                     Savoy\oindex{Hotel Savoy@\textbf{Hotel Savoy}|pwk}. Dem \emph{Tagebuch}\pwindex{\textcolor{red}{\textsuperscript{XXXX1 indx}}!Tagebuch1981 – 2000@\strich\emph{Tagebuch} {[}Hrsg., 1981 – 2000{]}|pwk} ist nicht zu
                  entnehmen, ob Goldmann\pwindex{Goldmann, Paul 31.01.1865 – 25.09.1935@\textsc{Goldmann, Paul} (31.01.1865 – 25.09.1935), \emph{Schriftsteller, Journalist}|pwk} und möglicherweise
                  auch Hermann Mamroth\pwindex{Mamroth, Hermann @\textsc{Mamroth, Hermann}|pwk} dort waren.}}}\label{K_L03189-2h}
               finde.\pend
           \pstart
           Viele treue Grüße! Und nochmals Glück zum neuen Jahr!
               {\\[\baselineskip]}Dein {\\[\baselineskip]}\spacefill\mbox{Paul Goldm}\pend
           \leftskip=0em{}
         
         \endnumbering\mylabel{h}\end{ledgroupsized}  \newcommand{\dateiname}{L03189}\newcommand{\titel}{Paul Goldmann an Arthur Schnitzler, 1. 1. [1902]}\newcommand{\editorInnen}{Martin Anton Müller und Laura Untner}%% latex-leseansicht-abspann.tex
%% Abspann für die Leseansicht.
%% Der Schalter \ifkorrekturansicht ist bereits durch den Vorspann gesetzt.

%% latex-abspann.tex
%% Gemeinsamer Abspann für Korrekturansicht und Leseansicht.
%% Setzt den Schalter \ifkorrekturansicht voraus (gesetzt in den
%% einbindenden Dateien latex-korrekturansicht-abspann.tex bzw.
%% latex-leseansicht-abspann.tex).
%% ---------------------------------------------------------------

\normalsize

% Das esempio-Environment wird nur in der Leseansicht benötigt
\ifkorrekturansicht\else
\newenvironment{esempio}[3]%
{
    \vspace{1.5ex}
    \rlap{\underline{#1}}
    \par
    \setlength{\parindent}{0cm}
    \nopagebreak
    \leftskip=#2cm
    \rightskip=#3cm
}
{
    \par
}
\fi

\doendnotes{C}
\bigskip
\vfill

\clearpage

\footnotesize

\ifkorrekturansicht
  \lohead{\textsc{register}}
\fi

% theindex-Environment neu definieren ohne reledmac
\makeatletter
\renewenvironment{theindex}{%
  \ifkorrekturansicht
    \section*{\indexname}%
  \else
    \subsubsection*{Index der erwähnten Entitäten}%
  \fi
  \setlength{\parindent}{0pt}%
  \setlength{\parskip}{0pt plus 0.3pt}%
  \let\item\@idxitem
}{%
  \ifkorrekturansicht\clearpage\fi
}
\makeatother

\IfFileExists{\jobname-pw.ind}{\input{\jobname-pw.ind}}{}

% Quellenangabe nur in der Leseansicht
\ifkorrekturansicht\else
% Fallback-Definitionen, falls die .tex-Datei \titel etc. nicht gesetzt hat
\providecommand{\titel}{}
\providecommand{\editorInnen}{}
\providecommand{\dateiname}{\jobname}

\vspace{3cm}

\vfill

\footnotesize
\textsc{Quelle}: \titel. Herausgegeben von {\editorInnen}. In: \emph{Arthur Schnitzler: Briefwechsel mit Autorinnen und Autoren}.
 Digitale Edition, https://schnitzler-briefe.acdh.oeaw.ac.at/{\dateiname}.html (Stand \today)
\fi

\end{document}


      