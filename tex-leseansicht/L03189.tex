%% latex-korrekturansicht-vorspann.tex
%% Vorspann für die Korrekturansicht.
%% Lädt die gemeinsame Datei latex-vorspann.tex mit gesetztem Schalter.

\newif\ifkorrekturansicht
\korrekturansichttrue

\input{../tex-inputs/latex-vorspann}


\section[ Paul Goldmann an Arthur Schnitzler, 1. 1. {[}1902{]}]{L03189 Paul Goldmann an Arthur Schnitzler, 1. 1. {[}1902{]}}
\nopagebreak\mylabel{L03189v}
\rehead{ }\normalsize\beginnumbering\briefempfaengerindex{Schnitzler, Arthur@\textsc{Schnitzler, Arthur}!zzzGoldmann, Paul@\emph{von Paul Goldmann}!1902-01-014@{1. 1. {[}1902{]}}|(be}
\toendnotes[C]{\smallbreak\pagebreak[2]}\Standort{DLA, A:Schnitzler, HS.NZ85.1.3172.}
\physDesc{Brief, 1 Blatt, 1 Seite, 424 Zeichen
\newline{}Handschrift: blaue Tinte, deutsche Kurrent
\newline{}Schnitzler: mit Bleistift das Jahr »902« vermerkt }\toendnotes[C]{\smallbreak}
\pstart
           \centering{}{\pb}Frankfurt\oindex{Frankfurt am Main@\textbf{Frankfurt am Main}, \emph{P.PPLA3}|pw}{ }1. Januar.\pend
           
\pstart\center{}Mein lieber Freund,\pend\vspace{0.5em}
\pstart
           Bitte, nimm’ den \label{K_L03189-1v}\edtext{Sitz}{\lemma{\textnormal{\emph{Sitz}}}\Cendnote{\textnormal{Für die Uraufführung von \emph{Lebendige Stunden}\pwindex{Lebendige Stunden. Vier Einakter@\emph{Lebendige Stunden. Vier Einakter}|pwk} am 4. 1. 1902 am \emph{Deutschen Theater Berlin}\orgindex{Deutsches Theater Berlin@Deutsches Theater Berlin|pwk}. }}}\label{K_L03189-1}, den Du neben dem meinigen (N\textsuperscript{o} 95, 10. Reihe) haſt reſerviren laſſen und ſende ihn an
               meinen Onkel\pwindex{Mamroth, Hermann @\textsc{Mamroth, Hermann}|pwv}, Herrn \textsc{Hermann Mamroth\pwindex{Mamroth, Hermann @\textsc{Mamroth, Hermann}|pw}}, \textsc{Berlin S. W.\oindex{Berlin@\textbf{Berlin}, \emph{P.PPLC}|pw}}, \textsc{Bernburgerstraße} 28\oindex{Bernburger Strasse@\textbf{Bernburger Straße}, \emph{Straße (K.STR)}|pw}. Wir verrechnen uns nach meiner
               Rückkunft.\pend
           
\pstart
           Bitte, ſchreibe mir nach meiner Berliner Wohnung\oindex{Dessauer Strasse@\textbf{Dessauer Straße}, \emph{Straße (K.STR)}|pwv} ein Wort, wo ich Dich am Samſtag{ }\label{K_L03189-2v}\edtext{nach der Vorſtellung\pwindex{Lebendige Stunden. Vier Einakter@\emph{Lebendige Stunden. Vier Einakter}|pwv}}{\lemma{\textnormal{\emph{nach der Vorſtellung}}}\Cendnote{\textnormal{Hinterher war Schnitzler im Hotel
                     Savoy\oindex{Hotel Savoy [Berlin]@\textbf{Hotel Savoy [Berlin]}, \emph{Hotel (K.HTL)}|pwk}. Dem \emph{Tagebuch}\pwindex{Tagebuch@\emph{Tagebuch}|pwk} ist nicht zu
                  entnehmen, ob Goldmann\pwindex{Goldmann, Paul 31.01.1865 – 25.09.1935@\textsc{Goldmann, Paul} (31.01.1865 – 25.09.1935), \emph{Schriftsteller/Schriftstellerin, Journalist/Journalistin}|pwk} und möglicherweise
                  auch Hermann Mamroth\pwindex{Mamroth, Hermann @\textsc{Mamroth, Hermann}|pwk} dort waren.}}}\label{K_L03189-2}
               finde.\pend
           
\pstart
           Viele treue Grüße! Und nochmals Glück zum neuen Jahr!
               {\\[\baselineskip]}Dein {\\[\baselineskip]}\spacefill\mbox{Paul Goldm}\pend
           \leftskip=0em{}\selectlanguage{ngerman}\endnumbering\briefempfaengerindex{Schnitzler, Arthur@\textsc{Schnitzler, Arthur}!zzzGoldmann, Paul@\emph{von Paul Goldmann}!1902-01-014@{1. 1. {[}1902{]}}|)be}\mylabel{L03189h}  \normalsize

\doendnotes{C}
\bigskip
\vfill

\clearpage

\footnotesize

\lohead{\textsc{register}}

% Definiere theindex-Environment komplett neu ohne reledmac
\makeatletter
\renewenvironment{theindex}{%
  \section*{\indexname}%
  \setlength{\parindent}{0pt}%
  \setlength{\parskip}{0pt plus 0.3pt}%
  \let\item\@idxitem
}{%
  \clearpage
}
\makeatother

\IfFileExists{\jobname-pw.ind}{\input{\jobname-pw.ind}}{}

\end{document}

      