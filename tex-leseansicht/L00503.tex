%% latex-leseansicht-vorspann.tex
%% Vorspann für die Leseansicht.
%% Lädt die gemeinsame Datei latex-vorspann.tex mit nicht gesetztem Schalter.

\newif\ifkorrekturansicht
\korrekturansichtfalse

\input{../tex-inputs/latex-vorspann}


         
         \newcommand{\erwaehntePersonen}{Personen: Joris-Karl Huysmans}
         \newcommand{\erwaehnteInstitutionen}{}
         \newcommand{\erwaehnteOrte}{Orte: Gmunden, Goldener Brunnen, Wien}
         \newcommand{\erwaehnteWerke}{Werke: En route, Liebelei. Schauspiel in drei Akten}
               \section[Peter Altenberg an Arthur Schnitzler, {[}10. 10. 1895{]}]{ Peter Altenberg an Arthur Schnitzler, {[}10. 10. 1895{]}}\nopagebreak\mylabel{v}\rehead{ }\begin{ledgroupsized}[t]{13cm}\normalsize\beginnumbering \toendnotes[C]{\smallbreak\pagebreak[2]} \Standort{CUL, Schnitzler, B 2.}
\physDesc{Brief, 1 Blatt, 1 Seite
\newline{}Handschrift: schwarze Tinte, deutsche Kurrent
\newline{}Schnitzler: mit Bleistift datiert: »Oct. 95« und nummeriert: »5« \newline{}Ordnung: mit Bleistift von unbekannter Hand nummeriert:
                                 »4« }\buchAbdrucke{\weitereDrucke{Peter Altenberg: \emph{Die Selbsterfindung eines Dichters. Briefe und Dokumente
                        1892–1896}. Hg. und mit einem Nachwort von Leo A. Lensing. Göttingen: \emph{Wallstein} 2009, S. 38.} }\toendnotes[C]{\smallbreak}\pstart{}{\pb}Lieber Arthur Schnitzler.\pend\pstart
           Nehme herzlich Theil an ihrem Erfolge. Habe mit Spannung die Morgenblätter von
                  \label{K_L00503_1v}\edtext{heute Donnerſtag}{\lemma{\textnormal{\emph{heute Donnerſtag}}}\Cendnote{\textnormal{Zusammen mit der Datierung Schnitzler\pwindex{Schnitzler, Arthur 15.05.1862 – 21.10.1931@\textsc{Schnitzler, Arthur} (15.05.1862 – 21.10.1931), \emph{Schriftsteller, Mediziner}|pwk}s auf »Oct. 95« lässt sich als
                  Datum für diesen Brief der 10. 10. 1895 ermitteln, der Tag nach der
                  Uraufführung der \emph{Liebelei}\pwindex{Schnitzler, Arthur 15.05.1862 – 21.10.1931@\textsc{Schnitzler, Arthur} (15.05.1862 – 21.10.1931), \emph{Schriftsteller, Mediziner}!Liebelei. Schauspiel in drei Akten1895-10-09@\strich\emph{Liebelei. Schauspiel in drei Akten} {[}1895-10-09{]}|pwk}.}}}\label{K_L00503_1h} (3 Uhr
                  Nachmittag) erwartet.\pend
           \pstart
           Hier iſt herrliche dicke Ruhe, Herbſt-Friede. Schreiben Sie mir doch einmal.\pend
           \pstart
           Ich leſe »\textsc{en route}\pwindex{Huysmans, Joris-Karl 05.02.1848 – 12.05.1907@\textsc{Huysmans, Joris-Karl} (05.02.1848 – 12.05.1907), \emph{Schriftsteller}!En route1895@\strich\emph{En route} {[}1895{]}|pw}« von \textsc{Huysmans}\pwindex{Huysmans, Joris-Karl 05.02.1848 – 12.05.1907@\textsc{Huysmans, Joris-Karl} (05.02.1848 – 12.05.1907), \emph{Schriftsteller}|pw}.\pend
           \pstart
           Sie haben hoffentlich die \label{T_L00503_1v}\edtext{\textsc{C}..........}{\lemma{\textnormal{\emph{C}}}\Cendnote{\textnormal{Von Schnitzler\pwindex{Schnitzler, Arthur 15.05.1862 – 21.10.1931@\textsc{Schnitzler, Arthur} (15.05.1862 – 21.10.1931), \emph{Schriftsteller, Mediziner}|pwk} wurden die fehlenden Buchstaben mit Bleistift in lateinischer Schreibschrift
                  ergänzt: »\textsc{igaretten}«, wobei hier die Schrift darauf
                  hindeutet, dass diese Ergänzung erst nach dem Wechsel seiner Schreibschrift und
                  mithin erst nach 1906 anzusetzen ist.}}}\label{T_L00503_1h} unter »Baumwollwaare«
               vom 16./8 erhalten?!\pend
           \pstart
           Adieu, ihr{\\[\baselineskip]}\spacefill\mbox{Richard Engländer.}\pend
           \leftskip=0em{}\pstart
           \noindent{}Goldener Brunnen\oindex{Goldener Brunnen@\textbf{Goldener Brunnen}|pw}.\pend
           
         
         \endnumbering\mylabel{h}\end{ledgroupsized}  \newcommand{\dateiname}{L00503}\newcommand{\titel}{Peter Altenberg an Arthur Schnitzler, [10. 10. 1895]}\newcommand{\editorInnen}{Martin Anton Müller und Gerd-Hermann Susen}%% latex-leseansicht-abspann.tex
%% Abspann für die Leseansicht.
%% Der Schalter \ifkorrekturansicht ist bereits durch den Vorspann gesetzt.

%% latex-abspann.tex
%% Gemeinsamer Abspann für Korrekturansicht und Leseansicht.
%% Setzt den Schalter \ifkorrekturansicht voraus (gesetzt in den
%% einbindenden Dateien latex-korrekturansicht-abspann.tex bzw.
%% latex-leseansicht-abspann.tex).
%% ---------------------------------------------------------------

\normalsize

% Das esempio-Environment wird nur in der Leseansicht benötigt
\ifkorrekturansicht\else
\newenvironment{esempio}[3]%
{
    \vspace{1.5ex}
    \rlap{\underline{#1}}
    \par
    \setlength{\parindent}{0cm}
    \nopagebreak
    \leftskip=#2cm
    \rightskip=#3cm
}
{
    \par
}
\fi

\doendnotes{C}
\bigskip
\vfill

\clearpage

\footnotesize

\ifkorrekturansicht
  \lohead{\textsc{register}}
\fi

% theindex-Environment neu definieren ohne reledmac
\makeatletter
\renewenvironment{theindex}{%
  \ifkorrekturansicht
    \section*{\indexname}%
  \else
    \subsubsection*{Index der erwähnten Entitäten}%
  \fi
  \setlength{\parindent}{0pt}%
  \setlength{\parskip}{0pt plus 0.3pt}%
  \let\item\@idxitem
}{%
  \ifkorrekturansicht\clearpage\fi
}
\makeatother

\IfFileExists{\jobname-pw.ind}{\input{\jobname-pw.ind}}{}

% Quellenangabe nur in der Leseansicht
\ifkorrekturansicht\else
% Fallback-Definitionen, falls die .tex-Datei \titel etc. nicht gesetzt hat
\providecommand{\titel}{}
\providecommand{\editorInnen}{}
\providecommand{\dateiname}{\jobname}

\vspace{3cm}

\vfill

\footnotesize
\textsc{Quelle}: \titel. Herausgegeben von {\editorInnen}. In: \emph{Arthur Schnitzler: Briefwechsel mit Autorinnen und Autoren}.
 Digitale Edition, https://schnitzler-briefe.acdh.oeaw.ac.at/{\dateiname}.html (Stand \today)
\fi

\end{document}


      