%% latex-leseansicht-vorspann.tex
%% Vorspann für die Leseansicht.
%% Lädt die gemeinsame Datei latex-vorspann.tex mit nicht gesetztem Schalter.

\newif\ifkorrekturansicht
\korrekturansichtfalse

\input{../tex-inputs/latex-vorspann}


\section[Berta Zuckerkandl an Arthur Schnitzler, {[}23. 6. 1911?{]}]{L03995 Berta Zuckerkandl an Arthur Schnitzler, {[}23. 6. 1911?{]}}
\nopagebreak\mylabel{L03995v}
\rehead{ }\normalsize\beginnumbering\briefempfaengerindex{Schnitzler, Arthur@\textsc{Schnitzler, Arthur}!zzzZuckerkandl, Berta@\emph{von Berta Zuckerkandl}!1911-06-231@{{[}23. 6. 1911?{]}}|(be}
\toendnotes[C]{\smallbreak\pagebreak[2]}
\correspDesc{Versand  durch Berta Zuckerkandl am [23. 6. 1911?] in Paris
\newline{}Erhalt  durch Arthur Schnitzler im Zeitraum [24. 6. 1911 – 28. 6. 1911?] in Wien}\toendnotes[C]{\smallbreak}
\Standort{CUL, Schnitzler, B 200.}
\physDesc{Brief, 2 Blätter, 7 Seiten, 1544 Zeichen (Absendeadresse »\textcolor{gray}{\textbf{84 RUE DE LONGCHAMP}}« auch auf dem zweiten Briefbogen)
\newline{}Handschrift: schwarze Tinte, lateinische Kurrent
\newline{}Schnitzler: mit Bleistift beschriftet: »Zuckerkan\textcolor{gray}{dl}« }\toendnotes[C]{\smallbreak}
\pstart
           {\pb}\uline{Freitag}.\pend
           
\pstart
           \raggedleft{}\textcolor{gray}{\textbf{84 RUE DE LONGCHAMP}}\oindex{84, rue de Longchamp@\textbf{84, rue de Longchamp}, \emph{Wohngebäude}|pw}\pend
           
\pstart{} Verehrter Herr Doktor!\pend\vspace{0.5em}
\pstart
           \label{K_L03995-1v}\edtext{Gestern Donnerstag}{\lemma{\textnormal{\emph{Gestern Donnerstag}}}\Cendnote{\textnormal{Schnitzler vermerkte am 20. 6. 1911 im
                            \emph{Tagebuch}\pwindex{Schnitzler, Arthur 15.\,5.\,1862 Wien – 21.\,10.\,1931 ebd.@\textsc{Schnitzler, Arthur} (15.\,5.\,1862 Wien – 21.\,10.\,1931 ebd.), \emph{Schriftsteller, Mediziner}!Tagebuch@\strich\emph{Tagebuch}|pwk} das Absenden des \emph{Medardus}\pwindex{Schnitzler, Arthur 15.\,5.\,1862 Wien – 21.\,10.\,1931 ebd.@\textsc{Schnitzler, Arthur} (15.\,5.\,1862 Wien – 21.\,10.\,1931 ebd.), \emph{Schriftsteller, Mediziner}!junge Medardus. Dramatische Historie in einem Vorspiel
                        und fünf Aufzügen@\strich\emph{Der junge Medardus. Dramatische Historie in einem Vorspiel und fünf Aufzügen}|pwk}-Szenariums an Berta Zuckerkandl\pwindex{Zuckerkandl, Berta 13.\,4.\,1864 Wien – 16.\,10.\,1945 Paris@\textsc{Zuckerkandl, Berta} (13.\,4.\,1864 Wien – 16.\,10.\,1945 Paris), \emph{Schriftstellerin, Journalistin, Übersetzerin}|pwk} nach Paris\oindex{Paris@\textbf{Paris}, \emph{Hauptstadt}|pwk}. Da zwei Tage eine übliche
                        Postlaufzeit für eine Sendung von Wien\oindex{Wien@\textbf{Wien}, \emph{Verwaltungsgebiet}|pwk}
                        nach Paris\oindex{Paris@\textbf{Paris}, \emph{Hauptstadt}|pwk} darstellen, dürfte der
                        benannte Donnerstag der Zustellung der 22. 6. 1911 gewesen sein
                        und sich der vorliegende Brief somit auf den 23. 6. 1911
                        datieren lassen.}}}\label{K_L03995-1} ist Ihr Scenario\pwindex{Schnitzler, Arthur 15.\,5.\,1862 Wien – 21.\,10.\,1931 ebd.@\textsc{Schnitzler, Arthur} (15.\,5.\,1862 Wien – 21.\,10.\,1931 ebd.), \emph{Schriftsteller, Mediziner}!junge Medardus. Dramatische Historie in einem Vorspiel
                        und fünf Aufzügen@\strich\emph{Der junge Medardus. Dramatische Historie in einem Vorspiel und fünf Aufzügen}|pwv} geko{\geminationm}en. Wofür
                    ich herzlichst danke. Es ist ausgezeichnet gemacht – u ich beginne
                        heute mit der Übersetzung.\pend
           
\pstart
           Für meine Unterre{\pb}dung mit den
                    Direktoren Herz\pwindex{Hertz, Henri 17.\,6.\,1875 Nogent-sur-Seine – 11.\,10.\,1966 16. arrondissement [Paris]@\textsc{Hertz, Henri} (17.\,6.\,1875 Nogent-sur-Seine – 11.\,10.\,1966 16. arrondissement [Paris]), \emph{Schriftsteller, Journalist, Theaterdirektor}|pw} und Coquelin\pwindex{Coquelin, Jean 1.\,12.\,1865 Paris – 1.\,10.\,1944@\textsc{Coquelin, Jean} (1.\,12.\,1865 Paris – 1.\,10.\,1944), \emph{Schauspieler}|pw} kam es zu spät. Wieder alles Erwarten – da
                    doch in Paris\oindex{Paris@\textbf{Paris}, \emph{Hauptstadt}|pw} Alles so lange dauert –
                    erhielt ich am Mitwoch{ }Früh von der Porte St. Martin\orgindex{Théâtre de la Porte Saint-Martin@Théâtre de la Porte Saint-Martin|pwv} die \label{T_L03995-1v}\edtext{Verständigung}{\lemma{\textnormal{\emph{Verständigung}}}\Cendnote{\textnormal{Sie schreibt:
                            »Verständidung«.}}}\label{T_L03995-1} dass die Direktoren\pwindex{Coquelin, Jean 1.\,12.\,1865 Paris – 1.\,10.\,1944@\textsc{Coquelin, Jean} (1.\,12.\,1865 Paris – 1.\,10.\,1944), \emph{Schauspieler}|pw}\pwindex{Hertz, Henri 17.\,6.\,1875 Nogent-sur-Seine – 11.\,10.\,1966 16. arrondissement [Paris]@\textsc{Hertz, Henri} (17.\,6.\,1875 Nogent-sur-Seine – 11.\,10.\,1966 16. arrondissement [Paris]), \emph{Schriftsteller, Journalist, Theaterdirektor}|pw} mich bitten
                        Mittwoch{ }Nachmittag sie {\pb}zu
                    besuchen. Ich hatte zum  Glück zur Vorsicht – einen langen Auszug aus dem Medardus\pwindex{Schnitzler, Arthur 15.\,5.\,1862 Wien – 21.\,10.\,1931 ebd.@\textsc{Schnitzler, Arthur} (15.\,5.\,1862 Wien – 21.\,10.\,1931 ebd.), \emph{Schriftsteller, Mediziner}!junge Medardus. Dramatische Historie in einem Vorspiel
                        und fünf Aufzügen@\strich\emph{Der junge Medardus. Dramatische Historie in einem Vorspiel und fünf Aufzügen}|pw} auf der Reise französisch\oindex{Frankreich@\textbf{Frankreich}|pw} geschrieben. Rasch diktirte ich dies in einem
                    Schreib-Maschin-Bureau ab – und um 5 Uhr war ich bei Herz\pwindex{Hertz, Henri 17.\,6.\,1875 Nogent-sur-Seine – 11.\,10.\,1966 16. arrondissement [Paris]@\textsc{Hertz, Henri} (17.\,6.\,1875 Nogent-sur-Seine – 11.\,10.\,1966 16. arrondissement [Paris]), \emph{Schriftsteller, Journalist, Theaterdirektor}|pw}{ }{\kaufmannsund}{ }Coquelin\pwindex{Coquelin, Jean 1.\,12.\,1865 Paris – 1.\,10.\,1944@\textsc{Coquelin, Jean} (1.\,12.\,1865 Paris – 1.\,10.\,1944), \emph{Schauspieler}|pw} – begleitet von D\textsuperscript{r}{ }{\pb}Frischauer\pwindex{Frischauer, Berthold 9.\,9.\,1851 Brünn – 4.\,2.\,1924 Wien@\textsc{Frischauer, Berthold} (9.\,9.\,1851 Brünn – 4.\,2.\,1924 Wien), \emph{Journalist}|pw}. Herz\pwindex{Hertz, Henri 17.\,6.\,1875 Nogent-sur-Seine – 11.\,10.\,1966 16. arrondissement [Paris]@\textsc{Hertz, Henri} (17.\,6.\,1875 Nogent-sur-Seine – 11.\,10.\,1966 16. arrondissement [Paris]), \emph{Schriftsteller, Journalist, Theaterdirektor}|pw} ist ein \label{K_L03995-2v}\edtext{\begin{otherlanguage}{french}Homme d’affaire\end{otherlanguage}}{\lemma{\textnormal{\emph{Homme d’affaire}}}\Cendnote{\textnormal{französisch: Geschäftsmann}}}\label{K_L03995-2}
                    pur – der literarisch nicht viel weiss. Coquelin\pwindex{Coquelin, Jean 1.\,12.\,1865 Paris – 1.\,10.\,1944@\textsc{Coquelin, Jean} (1.\,12.\,1865 Paris – 1.\,10.\,1944), \emph{Schauspieler}|pw} dagegen wusste Manches vom Medardus\pwindex{Schnitzler, Arthur 15.\,5.\,1862 Wien – 21.\,10.\,1931 ebd.@\textsc{Schnitzler, Arthur} (15.\,5.\,1862 Wien – 21.\,10.\,1931 ebd.), \emph{Schriftsteller, Mediziner}!junge Medardus. Dramatische Historie in einem Vorspiel
                        und fünf Aufzügen@\strich\emph{Der junge Medardus. Dramatische Historie in einem Vorspiel und fünf Aufzügen}|pw}. Ich wurde gebeten meine Inhalts-Angabe dort zu lassen. Aber
                    ich gab mündlich eine Schilderung die wie mir D\textsuperscript{r} Frischauer\pwindex{Frischauer, Berthold 9.\,9.\,1851 Brünn – 4.\,2.\,1924 Wien@\textsc{Frischauer, Berthold} (9.\,9.\,1851 Brünn – 4.\,2.\,1924 Wien), \emph{Journalist}|pw} dann sagte – riesig \label{T_L03995-2v}\edtext{inter{\pb}ressirte}{\lemma{\textnormal{\emph{interressirte}}}\Cendnote{\textnormal{Sie schreibt: »inressirte«.}}}\label{T_L03995-2}.
                    Besonders die Pretendenten-Geschichte finden die Herrn\pwindex{Coquelin, Jean 1.\,12.\,1865 Paris – 1.\,10.\,1944@\textsc{Coquelin, Jean} (1.\,12.\,1865 Paris – 1.\,10.\,1944), \emph{Schauspieler}|pw}\pwindex{Hertz, Henri 17.\,6.\,1875 Nogent-sur-Seine – 11.\,10.\,1966 16. arrondissement [Paris]@\textsc{Hertz, Henri} (17.\,6.\,1875 Nogent-sur-Seine – 11.\,10.\,1966 16. arrondissement [Paris]), \emph{Schriftsteller, Journalist, Theaterdirektor}|pw} für Paris\oindex{Paris@\textbf{Paris}, \emph{Hauptstadt}|pw} höchst erwünscht. Meine Unterredung hatte wie es sich zeigte so
                    rasch stattfinden müssen, weil Herz\pwindex{Hertz, Henri 17.\,6.\,1875 Nogent-sur-Seine – 11.\,10.\,1966 16. arrondissement [Paris]@\textsc{Hertz, Henri} (17.\,6.\,1875 Nogent-sur-Seine – 11.\,10.\,1966 16. arrondissement [Paris]), \emph{Schriftsteller, Journalist, Theaterdirektor}|pw}{ }{\kaufmannsund}{ }Coquelin\pwindex{Coquelin, Jean 1.\,12.\,1865 Paris – 1.\,10.\,1944@\textsc{Coquelin, Jean} (1.\,12.\,1865 Paris – 1.\,10.\,1944), \emph{Schauspieler}|pw}{ }heute{ }{\pb}verreisen. Sie baten mich das von
                    Ihnen zu sendende Scenarium\pwindex{Schnitzler, Arthur 15.\,5.\,1862 Wien – 21.\,10.\,1931 ebd.@\textsc{Schnitzler, Arthur} (15.\,5.\,1862 Wien – 21.\,10.\,1931 ebd.), \emph{Schriftsteller, Mediziner}!junge Medardus. Dramatische Historie in einem Vorspiel
                        und fünf Aufzügen@\strich\emph{Der junge Medardus. Dramatische Historie in einem Vorspiel und fünf Aufzügen}|pwv} möglichst rasch zu übersetzen – {\kaufmannsund} an eine mir angegebene Adresse zu senden. Dann erst werden wir etwas wissen.
                    Es ist möglich dass ich im August von {\pb}der Schweiz\oindex{Schweiz@\textbf{Schweiz}|pw} wieder zu den Herrn\pwindex{Coquelin, Jean 1.\,12.\,1865 Paris – 1.\,10.\,1944@\textsc{Coquelin, Jean} (1.\,12.\,1865 Paris – 1.\,10.\,1944), \emph{Schauspieler}|pw}\pwindex{Hertz, Henri 17.\,6.\,1875 Nogent-sur-Seine – 11.\,10.\,1966 16. arrondissement [Paris]@\textsc{Hertz, Henri} (17.\,6.\,1875 Nogent-sur-Seine – 11.\,10.\,1966 16. arrondissement [Paris]), \emph{Schriftsteller, Journalist, Theaterdirektor}|pw} ko{\geminationm}en muss. Jedenfalls war die
                    Anknüpfung günstig. Mehr kann ich nicht sagen. Einige Détails noch
                        morgen oder übermorgen – da ich heute
                    rasch schliessen muss.\pend
           
\pstart
           Ihnen u Ihrer Frau\pwindex{Schnitzler, Olga 17.\,1.\,1882 Wien – 13.\,1.\,1970 Lugano@\textsc{Schnitzler, Olga} (17.\,1.\,1882 Wien – 13.\,1.\,1970 Lugano), \emph{Schauspielerin, Sängerin}|pwv}
                    herzliche Grüsse.{\\[\baselineskip]} Ihre \spacefill\mbox{Berta Zuckerkandl}\pend
           \leftskip=0em{}\selectlanguage{ngerman}\endnumbering\briefempfaengerindex{Schnitzler, Arthur@\textsc{Schnitzler, Arthur}!zzzZuckerkandl, Berta@\emph{von Berta Zuckerkandl}!1911-06-231@{{[}23. 6. 1911?{]}}|)be}\mylabel{L03995h}
\begin{anhang}
\end{anhang}\newcommand{\dateiname}{L03995}\newcommand{\titel}{Berta Zuckerkandl an Arthur Schnitzler, [23. 6. 1911?]}\newcommand{\editorInnen}{Herausgegeben von Jahnke, SelmaMüller, Martin Anton}%% latex-leseansicht-abspann.tex
%% Abspann für die Leseansicht.
%% Der Schalter \ifkorrekturansicht ist bereits durch den Vorspann gesetzt.

%% latex-abspann.tex
%% Gemeinsamer Abspann für Korrekturansicht und Leseansicht.
%% Setzt den Schalter \ifkorrekturansicht voraus (gesetzt in den
%% einbindenden Dateien latex-korrekturansicht-abspann.tex bzw.
%% latex-leseansicht-abspann.tex).
%% ---------------------------------------------------------------

\normalsize

% Das esempio-Environment wird nur in der Leseansicht benötigt
\ifkorrekturansicht\else
\newenvironment{esempio}[3]%
{
    \vspace{1.5ex}
    \rlap{\underline{#1}}
    \par
    \setlength{\parindent}{0cm}
    \nopagebreak
    \leftskip=#2cm
    \rightskip=#3cm
}
{
    \par
}
\fi

\doendnotes{C}
\bigskip
\vfill

\clearpage

\footnotesize

\ifkorrekturansicht
  \lohead{\textsc{register}}
\fi

% theindex-Environment neu definieren ohne reledmac
\makeatletter
\renewenvironment{theindex}{%
  \ifkorrekturansicht
    \section*{\indexname}%
  \else
    \subsubsection*{Index der erwähnten Entitäten}%
  \fi
  \setlength{\parindent}{0pt}%
  \setlength{\parskip}{0pt plus 0.3pt}%
  \let\item\@idxitem
}{%
  \ifkorrekturansicht\clearpage\fi
}
\makeatother

\IfFileExists{\jobname-pw.ind}{\input{\jobname-pw.ind}}{}

% Quellenangabe nur in der Leseansicht
\ifkorrekturansicht\else
% Fallback-Definitionen, falls die .tex-Datei \titel etc. nicht gesetzt hat
\providecommand{\titel}{}
\providecommand{\editorInnen}{}
\providecommand{\dateiname}{\jobname}

\vspace{3cm}

\vfill

\footnotesize
\textsc{Quelle}: \titel. Herausgegeben von {\editorInnen}. In: \emph{Arthur Schnitzler: Briefwechsel mit Autorinnen und Autoren}.
 Digitale Edition, https://schnitzler-briefe.acdh.oeaw.ac.at/{\dateiname}.html (Stand \today)
\fi

\end{document}


