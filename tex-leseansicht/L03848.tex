%% latex-leseansicht-vorspann.tex
%% Vorspann für die Leseansicht.
%% Lädt die gemeinsame Datei latex-vorspann.tex mit nicht gesetztem Schalter.

\newif\ifkorrekturansicht
\korrekturansichtfalse

\input{../tex-inputs/latex-vorspann}


\section[Theodor Herzl an Arthur Schnitzler, 16. 2. 1895]{L03848 Theodor Herzl an Arthur Schnitzler, 16. 2. 1895}
\nopagebreak\mylabel{L03848v}
\rehead{ }\normalsize\beginnumbering\briefempfaengerindex{Schnitzler, Arthur@\textsc{Schnitzler, Arthur}!zzzHerzl, Theodor@\emph{von Theodor Herzl}!1895-02-161@{16. 2. 1895}|(be}
\toendnotes[C]{\smallbreak\pagebreak[2]}
\correspDesc{Versand  durch Theodor Herzl am 16. 2. 1895 in Paris
\newline{}Erhalt  durch Arthur Schnitzler im Zeitraum [17. 2. 1895
                  – 21. 2. 1895?] in Wien}\toendnotes[C]{\smallbreak}
\Standort{CUL, Schnitzler, B 39.}
\physDesc{Brief, 1 Blatt, 4 Seiten, 2866 Zeichen
\newline{}Handschrift: schwarze Tinte, lateinische Kurrent
\newline{}Ordnung: mit Bleistift von unbekannter Hand nummeriert: »27« }
\buchAbdrucke{\weitereDrucke{Theodor Herzl: \emph{Briefe und
                        autobiographische Notizen 1866–1895}. Bearbeitet von Johannes Wachten in Zusammenarbeit mit Chaya Harel, Daisy Tycho und Manfred Winkler. Berlin, Frankfurt am Main, Wien: \emph{Propyläen} 1983, S. 571–573 (Briefe und Tagebücher. Herausgegeben von Alex Bein, Hermann Greive, Moshe Schaerf, Julius H. Schoeps und Johannes Wachten, 1).} }\toendnotes[C]{\smallbreak}
\pstart
           \raggedleft{}{\pb}Paris\oindex{Paris@\textbf{Paris}, \emph{Hauptstadt}|pw}{ }16. II. 95\pend
           
\pstart{}Mein lieber Schnitzler,\pend\vspace{0.5em}
\pstart
           ich sehe nichts kommen. Es scheint, ich werde mir wieder einmal einen Flor um den Arm
               winden müssen. Wann ist denn die Erklärungsfrist Blumenthals\pwindex{Blumenthal, Oskar 13.\,3.\,1852 Berlin – 24.\,4.\,1917 ebd.@\textsc{Blumenthal, Oskar} (13.\,3.\,1852 Berlin – 24.\,4.\,1917 ebd.), \emph{Schriftsteller, Journalist, Theaterleiter}|pw} um? Sollten die drei Wochen schon um sein, so bitte ich Sie, ihm
               von der bekannten Hand\pwindex{?? [Schreibkraft für Arthur Schnitzler] @\textsc{?? [Schreibkraft für Arthur Schnitzler]}|pwv}
               Folgendes schreiben zu lassen:\pend
           
\pstart
           »Geehrter Herr! Die Zeit, die Ihnen zur Erklärung über die Annahme meines Schauspiels
                  D.. G.....\pwindex{Herzl, Theodor 2.\,5.\,1860 Budapest – 3.\,7.\,1904 Edlach@\textsc{Herzl, Theodor} (2.\,5.\,1860 Budapest – 3.\,7.\,1904 Edlach), \emph{Schriftsteller, Journalist}!neue Ghetto. Schauspiel in vier Acten@\strich\emph{Das neue Ghetto. Schauspiel in vier Acten}|pwv} eingeräumt war,
               ist verstrichen.\pend
           
\pstart
           Ich ersuche Sie das Manuscript\pwindex{Herzl, Theodor 2.\,5.\,1860 Budapest – 3.\,7.\,1904 Edlach@\textsc{Herzl, Theodor} (2.\,5.\,1860 Budapest – 3.\,7.\,1904 Edlach), \emph{Schriftsteller, Journalist}!neue Ghetto. Schauspiel in vier Acten@\strich\emph{Das neue Ghetto. Schauspiel in vier Acten}|pwv}
               Herrn F. Schick\pwindex{Schik, Friedrich *~6.\,9.\,1857 Wien@\textsc{Schik, Friedrich} (*~6.\,9.\,1857 Wien), \emph{Notar, Journalist, Dramaturg}|pw}{ }\strikeout{nach}{ }Wien III Reisnerstr.\oindex{Wien@\textbf{Wien}!III., Landstraße@\textbf{III., Landstraße}!Reisnerstraße 35@\textbf{Reisnerstraße 35}, \emph{Wohngebäude}|pw}  N° ? \strikeout{\textcolor{gray}{×}\-\textcolor{gray}{×}\-\textcolor{gray}{×}\-\textcolor{gray}{×}\-\textcolor{gray}{×}\-\textcolor{gray}{×}\-\textcolor{gray}{×}\-\textcolor{gray}{×}} zurückzuschicken. Achtungsvoll \label{K_L03848-1v}\edtext{Dr. A. S. – }{\lemma{\textnormal{\emph{Dr. A. S. – }}}\Cendnote{\textnormal{Für die Einreichung des
                     \emph{Neuen Ghettos}\pwindex{Herzl, Theodor 2.\,5.\,1860 Budapest – 3.\,7.\,1904 Edlach@\textsc{Herzl, Theodor} (2.\,5.\,1860 Budapest – 3.\,7.\,1904 Edlach), \emph{Schriftsteller, Journalist}!neue Ghetto. Schauspiel in vier Acten@\strich\emph{Das neue Ghetto. Schauspiel in vier Acten}|pwk} nutzte Herzl\pwindex{Herzl, Theodor 2.\,5.\,1860 Budapest – 3.\,7.\,1904 Edlach@\textsc{Herzl, Theodor} (2.\,5.\,1860 Budapest – 3.\,7.\,1904 Edlach), \emph{Schriftsteller, Journalist}|pwk} das Pseudonym Albert Schnabel.}}}\label{K_L03848-1}«\pend
           
\pstart
           Dann kommt der letzte Akt des Einreichungsdramas: Raimundtheater\orgindex{Raimund-Theater@Raimund-Theater|pw}.\pend
           
\pstart
           Wie stehen Sie mit Müller Guttenbrunn\pwindex{Müller-Guttenbrunn, Adam 22.\,10.\,1852 Zăbrani – 5.\,1.\,1923 Wien@\textsc{Müller-Guttenbrunn, Adam} (22.\,10.\,1852 Zăbrani – 5.\,1.\,1923 Wien), \emph{Schriftsteller, Theaterleiter, Beamter}|pw}? Können
               Sie zu ihm gehen und ihm sagen: Da habe ich ein Stück\pwindex{Herzl, Theodor 2.\,5.\,1860 Budapest – 3.\,7.\,1904 Edlach@\textsc{Herzl, Theodor} (2.\,5.\,1860 Budapest – 3.\,7.\,1904 Edlach), \emph{Schriftsteller, Journalist}!neue Ghetto. Schauspiel in vier Acten@\strich\emph{Das neue Ghetto. Schauspiel in vier Acten}|pwv} von Schnabel, Lesen Sie es geschwind?\pend
           
\pstart
           {\pb}Soll man ihn \substVorne{}\textsuperscript{dann}\substDazwischen{}nach der Annahme\substHinten{} eventuell ins Vertrauen ziehen? Was ist Ihre Ansicht? Mir ist Müller\pwindex{Müller-Guttenbrunn, Adam 22.\,10.\,1852 Zăbrani – 5.\,1.\,1923 Wien@\textsc{Müller-Guttenbrunn, Adam} (22.\,10.\,1852 Zăbrani – 5.\,1.\,1923 Wien), \emph{Schriftsteller, Theaterleiter, Beamter}|pw} tief zuwider, u. ich hoffe es beruht auf
               Gegenseitigkeit. Aber ich halte ihn doch für einen sehr anständigen Menschen, der
               glaube ich auch correct gegen Feinde ist. – Heraus mit Ihrer Ansicht.\pend
           
\pstart
           Was habe ich vorhergesagt! Da ist der Ekel nach dem Productionsrausch. – Basta.\pend
           
\pstart
           Warum höre ich nichts von Ihrem Stück\pwindex{Schnitzler, Arthur 15.\,5.\,1862 Wien – 21.\,10.\,1931 ebd.@\textsc{Schnitzler, Arthur} (15.\,5.\,1862 Wien – 21.\,10.\,1931 ebd.), \emph{Schriftsteller, Mediziner}!Liebelei. Schauspiel in drei Akten@\strich\emph{Liebelei. Schauspiel in drei Akten}|pwv}? Warum schicken Sie es mir nicht. Bin ich Ihnen in unserer
               Geheimnisikrämerei der letzten Monate nicht nahe genug gekommen?\pend
           
\pstart
           Ich habe ein grosses Bedürfniss nach einer guten Freundschaft. Es ist beinahe schon
               zum Annonciren!\pend
           
\pstart
           »Mann in den besten Jahren sucht einen Freund, dem er alle seine Schwächen und
               Lächerlichkeiten furchtlos anvertrauen kann.« Wie es hier in den Blättern {\pb}heisst: \label{K_L03848-2v}\edtext{\begin{otherlanguage}{french}on demande un ami désintéressé\end{otherlanguage}}{\lemma{\textnormal{\emph{on … désintéressé}}}\Cendnote{\textnormal{französisch: selbstloser Freund gesucht}}}\label{K_L03848-2}. –\pend
           
\pstart
           Ich weiss nicht, bin ich zu misstrauisch oder zu schüchtern oder \strikeout{zu} hab ich zu gute Augen – hier find ich unter meinen
               Bekannten keinen. Der Eine ist zu dumm, der Andere zu perfid, der Dritte verstimmt
               mich an der heikelsten Stelle, weil er Bekanntschaften zum Vorwärtskommen ausnutzt –
               sagen Sie mir, fühlen Sie bei dieser letzteren Beobachtung wenn Sie sie an einem
               Kameraden machen, nicht auch den Brechreiz?\pend
           
\pstart
           Ich überlese das Bisherige. Ich muss Ihnen naiv vorkommen. Gleichviel, ich bin zu
               faul den Brief von vorn anzufangen.\pend
           
\pstart
           Leben Sie wohl! Ich möchte jetzt in einem Fischerdorf in Sicilien\oindex{Sizilien@\textbf{Sizilien}, \emph{Land}|pw} sein, u. zw. im guten englischen Hôtel oben auf dem
               Berg. Tagsüber ginge ich allein {\pb}spaziren, am Meer, auf Bergen. Ich hätte dabei schöne einsame Gedanken und nicht
               den Wunsch, Jemandes Beifall durch deren Niederschreibung zu erringen. Ich würde
               Fischern zuschauen, wie sie Netze flicken, das hat mich von jeher leidenschaftlich
               interessirt. Am Abend sässe ich nach einem guten \begin{otherlanguage}{french}Diner\end{otherlanguage}{ }im Salon wo die jungen
               englischen \begin{otherlanguage}{english}Misses\end{otherlanguage} vom Tage mit anderen jungen Leuten Liebesdummheiten treiben, und
               sähe mit Wohlwollen zu.\pend
           
\pstart
           Aber ich kann nicht. Ich sitze in Paris\oindex{Paris@\textbf{Paris}, \emph{Hauptstadt}|pw}, gehe ins
                  Palais Bourbon\oindex{Palais Bourbon@\textbf{Palais Bourbon}, \emph{Regierungsgebäude}|pw}, in langweilige Theater,
               ärgere mich über Collegen und bin vielleicht nicht mehr werth als sie.\pend
           
\pstart
           Leben Sie wohl, mein lieber Freund {\\[\baselineskip]}Ihr {\\[\baselineskip]}\spacefill\mbox{Herzl}\pend
           \leftskip=0em{}\selectlanguage{ngerman}\endnumbering\briefempfaengerindex{Schnitzler, Arthur@\textsc{Schnitzler, Arthur}!zzzHerzl, Theodor@\emph{von Theodor Herzl}!1895-02-161@{16. 2. 1895}|)be}\mylabel{L03848h}
\begin{anhang}
\end{anhang}\newcommand{\dateiname}{L03848}\newcommand{\titel}{Theodor Herzl an Arthur Schnitzler, 16. 2. 1895}\newcommand{\editorInnen}{Selma Jahnke und Martin Anton Müller}%% latex-leseansicht-abspann.tex
%% Abspann für die Leseansicht.
%% Der Schalter \ifkorrekturansicht ist bereits durch den Vorspann gesetzt.

%% latex-abspann.tex
%% Gemeinsamer Abspann für Korrekturansicht und Leseansicht.
%% Setzt den Schalter \ifkorrekturansicht voraus (gesetzt in den
%% einbindenden Dateien latex-korrekturansicht-abspann.tex bzw.
%% latex-leseansicht-abspann.tex).
%% ---------------------------------------------------------------

\normalsize

% Das esempio-Environment wird nur in der Leseansicht benötigt
\ifkorrekturansicht\else
\newenvironment{esempio}[3]%
{
    \vspace{1.5ex}
    \rlap{\underline{#1}}
    \par
    \setlength{\parindent}{0cm}
    \nopagebreak
    \leftskip=#2cm
    \rightskip=#3cm
}
{
    \par
}
\fi

\doendnotes{C}
\bigskip
\vfill

\clearpage

\footnotesize

\ifkorrekturansicht
  \lohead{\textsc{register}}
\fi

% theindex-Environment neu definieren ohne reledmac
\makeatletter
\renewenvironment{theindex}{%
  \ifkorrekturansicht
    \section*{\indexname}%
  \else
    \subsubsection*{Index der erwähnten Entitäten}%
  \fi
  \setlength{\parindent}{0pt}%
  \setlength{\parskip}{0pt plus 0.3pt}%
  \let\item\@idxitem
}{%
  \ifkorrekturansicht\clearpage\fi
}
\makeatother

\IfFileExists{\jobname-pw.ind}{\input{\jobname-pw.ind}}{}

% Quellenangabe nur in der Leseansicht
\ifkorrekturansicht\else
% Fallback-Definitionen, falls die .tex-Datei \titel etc. nicht gesetzt hat
\providecommand{\titel}{}
\providecommand{\editorInnen}{}
\providecommand{\dateiname}{\jobname}

\vspace{3cm}

\vfill

\footnotesize
\textsc{Quelle}: \titel. Herausgegeben von {\editorInnen}. In: \emph{Arthur Schnitzler: Briefwechsel mit Autorinnen und Autoren}.
 Digitale Edition, https://schnitzler-briefe.acdh.oeaw.ac.at/{\dateiname}.html (Stand \today)
\fi

\end{document}


