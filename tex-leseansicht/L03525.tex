%% latex-leseansicht-vorspann.tex
%% Vorspann für die Leseansicht.
%% Lädt die gemeinsame Datei latex-vorspann.tex mit nicht gesetztem Schalter.

\newif\ifkorrekturansicht
\korrekturansichtfalse

\input{../tex-inputs/latex-vorspann}

\begin{center}
            \textcolor{red}{ENTWURF, NICHT FERTIG KORRIGIERT}
                      \end{center}
            
         
         \renewcommand{\erwaehntePersonen}{Personen: Marie Glümer, Olga Schnitzler, Elisabeth Steinrück, Ernst von Wolzogen}
         \renewcommand{\erwaehnteInstitutionen}{Institutionen: Überbrettl}
         \renewcommand{\erwaehnteOrte}{Orte: Berlin, Dessauer Straße, Wien}
         \renewcommand{\erwaehnteWerke}{}
               \section[ Paul Goldmann an Olga Gussmann, 3. 4. {[}1901{]}]{ Paul Goldmann an Olga Gussmann, 3. 4. {[}1901{]}}\nopagebreak\mylabel{v}\rehead{ }\begin{ledgroupsized}[t]{13cm}\normalsize\beginnumbering \toendnotes[C]{\smallbreak\pagebreak[2]} \Standort{DLA, A:Schnitzler, HS.NZ85.1.5247.}
\physDesc{Brief, 2 Blätter, 6 Seiten, 2388 Zeichen
\newline{}Handschrift: blaue Tinte, deutsche Kurrent
\newline{}Ordnung: mit Bleistift von Arthur
                                    Schnitzler\pwindex{Schnitzler, Arthur 15.05.1862 – 21.10.1931@\textsc{Schnitzler, Arthur} (15.05.1862 – 21.10.1931), \emph{Schriftsteller, Mediziner}|pw} das Jahr »1901.« vermerkt }\toendnotes[C]{\smallbreak}\pstart
           \noindent{}\raggedleft{}{\pb}\textcolor{gray}{\textbf{DESSAUERSTRASSE 19\oindex{Dessauer Strasse@\textbf{Dessauer Straße}|pw}}}\pend
           \pstart
           Berlin\oindex{Berlin@\textbf{Berlin}|pw}, 3. April.\pend
           \pstart\center{}Liebes Fräulein \textsc{Olga},\pend\pstart
           Schön, ſchön und ſchön! Und ich habe \uline{doch} Recht! Und
               wenn Sie werden ſo grob mit mir ſein, ſo werde ich bei Ihrem erſten Auftreten in Berlin\oindex{Berlin@\textbf{Berlin}|pw} eine ſchlechte Kritik über Sie ſchreiben!
               Oder ihnen ſonſt etwas Furchtbares anthun! Und wenn \uline{alle} Menſchen \label{K_L03525-2v}\edtext{einſam}{\lemma{\textnormal{\emph{einſam}}}\Cendnote{\textnormal{siehe Paul Goldmann an Arthur Schnitzler, 7. 7. 1907}}}\label{K_L03525-2h} ſind (was übrigens nicht wahr iſt), ſo will ich es \uline{nicht} ſein, \label{K_L03525-3v}\edtext{Himmelkreuzſchockſchwerenoth}{\lemma{\textnormal{\emph{Himmelkreuzſchockſchwerenoth}}}\Cendnote{\textnormal{umgangssprachlicher Ausruf}}}\label{K_L03525-3h}! Und wenn \uline{alle}
               Frauen eine Bagage ſind, ſo will ich doch eine haben, ſchon um auf {\pb}ſie ſchimpfen zu können! Und mein \label{K_L03525-4v}\edtext{Feuilleton}{\lemma{\textnormal{\emph{Feuilleton}}}\Cendnote{\textnormal{Bezug unklar}}}\label{K_L03525-4h} kam von Herzen und es war gut; denn es
               iſt \strikeout{\textcolor{gray}{Ar}} keine Kleinigkeit, an Gedankeninhalt eines ſo gewaltigen Werkes zu entwickeln,
               zumal wenn man gezwungen iſt, Manches zu ſagen, was der Autor\pwindex{Schnitzler, Arthur 15.05.1862 – 21.10.1931@\textsc{Schnitzler, Arthur} (15.05.1862 – 21.10.1931), \emph{Schriftsteller, Mediziner}|pwv} ſich ſelbſt nicht gedacht hat! Und
               wenn es Ihnen \strikeout{Ih\textcolor{gray}{ne}} nicht gefallen hat, ſo haben Sie mich eben nur wieder einmal unterſchätzt! Im
               Übrigen iſt es \strikeout{b\textcolor{gray}{ezeic}h} ſehr lieb
               von Ihnen, daß Sie mir geſchrieben haben, wie Sie ſchreiben. Vom Leben aber {\pb}\substVorne{}\textsuperscript{\textcolor{gray}{×}\-\textcolor{gray}{×}\-\textcolor{gray}{×}\-\textcolor{gray}{×}\-\textcolor{gray}{×}\-\textcolor{gray}{×}\-\textcolor{gray}{×}\-\textcolor{gray}{×}}\substDazwischen{}wiſſen\substHinten{} Sie lange nicht ſo viel, als Sie ſich einbilden. Und es wäre ſehr ſchön,
               wenn ich in Wien\oindex{Wien@\textbf{Wien}|pw} wäre und Sie Beide\pwindex{Schnitzler, Arthur 15.05.1862 – 21.10.1931@\textsc{Schnitzler, Arthur} (15.05.1862 – 21.10.1931), \emph{Schriftsteller, Mediziner}|pwv} öfter ſehen könnte; ich würde
               wahrſcheinlich weniger \label{K_L03525-11v}\edtext{Grillen
                  fangen}{\lemma{\textnormal{\emph{Grillen
                  fangen}}}\Cendnote{\textnormal{Anspielung auf eine Metapher
                  im vorigen Brief, vgl. Paul Goldmann an Olga Gussmann, 1. 4. [1901]}}}\label{K_L03525-11h}! Und es iſt unerhört, daß ich heut ſchon
               wieder Ihnen ſchreiben muß, ſatt Ihrem Schweſterchen\pwindex{Steinrueck, Elisabeth 19.11.1885 – 07.04.1920@\textsc{Steinrück, Elisabeth} (19.11.1885 – 07.04.1920)|pwv}, wie ich eigentlich vorhatte.\pend
           \pstart
           So, und jetzt reden wir vernünftig!\pend
           \pstart
           Dieſes kleine Fräulein \textsc{Liesl\pwindex{Steinrueck, Elisabeth 19.11.1885 – 07.04.1920@\textsc{Steinrück, Elisabeth} (19.11.1885 – 07.04.1920)|pw}} ſitzt ahnungslos in Wien\oindex{Wien@\textbf{Wien}|pw} und weiß nicht, daß
                  \label{K_L03525-12v}\edtext{hier\oindex{Berlin@\textbf{Berlin}|pwv} über ihr {\pb}Schickſal verhandelt}{\lemma{\textnormal{\emph{hier … verhandelt}}}\Cendnote{\textnormal{Siehe auch Paul Goldmann an Arthur Schnitzler, 18. 2. [1901].}}}\label{K_L03525-12h} wird. Vorgeſtern{ }Abend war ich mit \textsc{Wolzogen\pwindex{Wolzogen, Ernst von 23.04.1855 – 30.07.1934@\textsc{Wolzogen, Ernst von} (23.04.1855 – 30.07.1934), \emph{Schriftsteller}|pw}} zuſammen. Es wurde über Neuengagements für das »Überbrettl\orgindex{Ueberbrettl@Überbrettl|pw}« geſprochen, und ich ſtellte mit großer Energie die Candidatur
               Ihrer Schweſter\pwindex{Steinrueck, Elisabeth 19.11.1885 – 07.04.1920@\textsc{Steinrück, Elisabeth} (19.11.1885 – 07.04.1920)|pwv} auf. \textsc{Wolzogen\pwindex{Wolzogen, Ernst von 23.04.1855 – 30.07.1934@\textsc{Wolzogen, Ernst von} (23.04.1855 – 30.07.1934), \emph{Schriftsteller}|pw}} hat ein Vorurtheil gegen die Wien\oindex{Wien@\textbf{Wien}|pw}er Art, zu
               ſpielen, und ich weiß nicht, ob es mir gelingen wird, dieſes Vorurtheil zu
               zerſtreuen. Das beſte Mittel wäre Fräulein \textsc{Liesl\pwindex{Steinrueck, Elisabeth 19.11.1885 – 07.04.1920@\textsc{Steinrück, Elisabeth} (19.11.1885 – 07.04.1920)|pw}s} perſönliches Erſcheinen. Ich
               frage alſo: Könnte dieſe {\pb}\textcolor{gray}{mehr}verwöhnte junge Dame\pwindex{Steinrueck, Elisabeth 19.11.1885 – 07.04.1920@\textsc{Steinrück, Elisabeth} (19.11.1885 – 07.04.1920)|pwv}, falls die Sache ernſt wird, auf einige Tage nach Berlin\oindex{Berlin@\textbf{Berlin}|pw} kommen? Könnte ſie eventuell gleich ins Engagement\orgindex{Ueberbrettl@Überbrettl|pwv} gehen?\pend
           \pstart
           Ich betone: Dieſe Fragen ſind vorläufig rein akademiſch; und es iſt noch ſehr
               unſicher, ob die Sache ſich wird praktiſch verwirklichen laſſen.\pend
           \pstart
           Weitere Frage: wiſſen Sie, einen für heiteren Geſang begabten jungen Mann, {\pb}Tenor oder Baryton, ebenfalls fürs »Überbrettl\orgindex{Ueberbrettl@Überbrettl|pw}«?\pend
           \pstart
           Bitte um \uline{raſche} Antwort!\pend
           \pstart
           Die Glümer\pwindex{Gluemer, Marie 03.07.1867 – 16.11.1925@\textsc{Glümer, Marie} (03.07.1867 – 16.11.1925), \emph{Schauspielerin}|pw} iſt auf dem Wege der \label{K_L03525-17v}\edtext{Geneſung}{\lemma{\textnormal{\emph{Geneſung}}}\Cendnote{\textnormal{Marie Glümer\pwindex{Gluemer, Marie 03.07.1867 – 16.11.1925@\textsc{Glümer, Marie} (03.07.1867 – 16.11.1925), \emph{Schauspielerin}|pwk} war seit Anfang des Jahres krank. Siehe Paul Goldmann an Arthur Schnitzler, 22. 1. [1901] und folgende Briefe Goldmann\pwindex{Goldmann, Paul 31.01.1865 – 25.09.1935@\textsc{Goldmann, Paul} (31.01.1865 – 25.09.1935), \emph{Schriftsteller, Journalist}|pwk}s an Schnitzler\pwindex{Schnitzler, Arthur 15.05.1862 – 21.10.1931@\textsc{Schnitzler, Arthur} (15.05.1862 – 21.10.1931), \emph{Schriftsteller, Mediziner}|pwk}.}}}\label{K_L03525-17h}.
               Sie hat vor einigen Tagen das Sanatorium verlaſſen.\pend
           \pstart
           Und nun ſchönen Dank für Alles! Und ſeien Sie ſammt dem Schweſterlein\pwindex{Steinrueck, Elisabeth 19.11.1885 – 07.04.1920@\textsc{Steinrück, Elisabeth} (19.11.1885 – 07.04.1920)|pwv} herzlichſt gegrüßt von
               {\\[\baselineskip]}Ihrem ergebenen {\\[\baselineskip]}\spacefill\mbox{Dr. Paul Goldmann.}\pend
           \leftskip=0em{}
         
         \endnumbering\mylabel{h}\end{ledgroupsized}\begin{anhang}\end{anhang}\newcommand{\dateiname}{L03525}\newcommand{\titel}{Paul Goldmann an Olga Gussmann, 3. 4. [1901]}\newcommand{\editorInnen}{Martin Anton Müller und Laura Untner}%% latex-leseansicht-abspann.tex
%% Abspann für die Leseansicht.
%% Der Schalter \ifkorrekturansicht ist bereits durch den Vorspann gesetzt.

%% latex-abspann.tex
%% Gemeinsamer Abspann für Korrekturansicht und Leseansicht.
%% Setzt den Schalter \ifkorrekturansicht voraus (gesetzt in den
%% einbindenden Dateien latex-korrekturansicht-abspann.tex bzw.
%% latex-leseansicht-abspann.tex).
%% ---------------------------------------------------------------

\normalsize

% Das esempio-Environment wird nur in der Leseansicht benötigt
\ifkorrekturansicht\else
\newenvironment{esempio}[3]%
{
    \vspace{1.5ex}
    \rlap{\underline{#1}}
    \par
    \setlength{\parindent}{0cm}
    \nopagebreak
    \leftskip=#2cm
    \rightskip=#3cm
}
{
    \par
}
\fi

\doendnotes{C}
\bigskip
\vfill

\clearpage

\footnotesize

\ifkorrekturansicht
  \lohead{\textsc{register}}
\fi

% theindex-Environment neu definieren ohne reledmac
\makeatletter
\renewenvironment{theindex}{%
  \ifkorrekturansicht
    \section*{\indexname}%
  \else
    \subsubsection*{Index der erwähnten Entitäten}%
  \fi
  \setlength{\parindent}{0pt}%
  \setlength{\parskip}{0pt plus 0.3pt}%
  \let\item\@idxitem
}{%
  \ifkorrekturansicht\clearpage\fi
}
\makeatother

\IfFileExists{\jobname-pw.ind}{\input{\jobname-pw.ind}}{}

% Quellenangabe nur in der Leseansicht
\ifkorrekturansicht\else
% Fallback-Definitionen, falls die .tex-Datei \titel etc. nicht gesetzt hat
\providecommand{\titel}{}
\providecommand{\editorInnen}{}
\providecommand{\dateiname}{\jobname}

\vspace{3cm}

\vfill

\footnotesize
\textsc{Quelle}: \titel. Herausgegeben von {\editorInnen}. In: \emph{Arthur Schnitzler: Briefwechsel mit Autorinnen und Autoren}.
 Digitale Edition, https://schnitzler-briefe.acdh.oeaw.ac.at/{\dateiname}.html (Stand \today)
\fi

\end{document}


      