%% latex-leseansicht-vorspann.tex
%% Vorspann für die Leseansicht.
%% Lädt die gemeinsame Datei latex-vorspann.tex mit nicht gesetztem Schalter.

\newif\ifkorrekturansicht
\korrekturansichtfalse

\input{../tex-inputs/latex-vorspann}


\section[ Paul Goldmann an Olga Gussmann, 3. 4. {[}1901{]}]{L03525 Paul Goldmann an Olga Gussmann,  3. 4. [1901]}
\nopagebreak\mylabel{L03525v}
\rehead{ }\normalsize\beginnumbering\briefempfaengerindex{Schnitzler, Olga@\textsc{Schnitzler, Olga}!zzzGoldmann, Paul@\emph{von Paul Goldmann}!1901-04-032@{3. 4. [1901]}|(be}
\toendnotes[C]{\smallbreak\pagebreak[2]}
\correspDesc{Versand  durch Paul Goldmann am 3. 4. [1901] in Berlin
\newline{}Erhalt  durch Olga Gussmann im Zeitraum [4. 4. 1901
                  – 8. 4. 1901?] in Wien?}\toendnotes[C]{\smallbreak}
\Standort{DLA, A:Schnitzler, HS.NZ85.1.5247.}
\physDesc{Brief, 2 Blätter, 6 Seiten, 2393 Zeichen
\newline{}Handschrift: blaue Tinte, deutsche Kurrent
\newline{}Ordnung: mit Bleistift von Arthur Schnitzler das
                                 Jahr »1901.« vermerkt }\toendnotes[C]{\smallbreak}
\pstart
           \raggedleft{}{\pb}\textcolor{gray}{\textbf{DESSAUERSTRASSE 19\oindex{Dessauer Straße@\textbf{Dessauer Straße}, \emph{Straße}|pw}}}\pend
           
\pstart
           Berlin\oindex{Berlin@\textbf{Berlin}, \emph{Hauptstadt}|pw}, 3. April.\pend
           
\pstart\center{}Liebes Fräulein \textsc{Olga},\pend\vspace{0.5em}
\pstart
           Schön,{ }ſchön und{ }ſchön! Und ich habe \uline{doch} Recht! Und
               wenn Sie werden{ }ſo grob mit mir{ }ſein,{ }ſo werde ich bei Ihrem erſten Auftreten in Berlin\oindex{Berlin@\textbf{Berlin}, \emph{Hauptstadt}|pw} eine{ }ſchlechte Kritik über Sie{ }ſchreiben!
               Oder ihnen{ }ſonſt etwas Furchtbares anthun! Und wenn \uline{alle} Menſchen \label{K_L03525-1v}\edtext{einſam}{\lemma{\textnormal{\emph{einsam}}}\Cendnote{\textnormal{Siehe XXXX Auszeichnungsfehler: Dokument L03254 nicht gefunden.
               }}}\label{K_L03525-1}{ }ſind (was übrigens nicht wahr iſt),{ }ſo will ich es \uline{nicht}{ }ſein, \label{K_L03525-2v}\edtext{Himmelkreuzſchockſchwerenoth}{\lemma{\textnormal{\emph{Himmelkreuzschockschwerenoth}}}\Cendnote{\textnormal{umgangssprachlicher Ausruf}}}\label{K_L03525-2}! Und wenn \uline{alle}
               Frauen eine Bagage{ }ſind,{ }ſo will ich doch eine haben,{ }ſchon um auf {\pb}ſie{ }ſchimpfen zu können! Und mein \label{K_L03525-3v}\edtext{Feuilleton\pwindex{Goldmann, Paul 31.\,1.\,1865 Breslau – 25.\,9.\,1935 Wien@\textsc{Goldmann, Paul} (31.\,1.\,1865 Breslau – 25.\,9.\,1935 Wien), \emph{Schriftsteller, Journalist}!Berliner Theater. [Über unsere Kraft, zweiter Teil von Bjørnstjerne Bjørnson]@\strich\emph{Berliner Theater. [Über unsere Kraft, zweiter Teil von Bjørnstjerne Bjørnson]}|pwv}}{\lemma{\textnormal{\emph{Feuilleton}}}\Cendnote{\textnormal{Paul Goldmann\pwindex{Goldmann, Paul 31.\,1.\,1865 Breslau – 25.\,9.\,1935 Wien@\textsc{Goldmann, Paul} (31.\,1.\,1865 Breslau – 25.\,9.\,1935 Wien), \emph{Schriftsteller, Journalist}|pwk}: \emph{Berliner Theater}\pwindex{Goldmann, Paul 31.\,1.\,1865 Breslau – 25.\,9.\,1935 Wien@\textsc{Goldmann, Paul} (31.\,1.\,1865 Breslau – 25.\,9.\,1935 Wien), \emph{Schriftsteller, Journalist}!Berliner Theater. [Über unsere Kraft, zweiter Teil von Bjørnstjerne Bjørnson]@\strich\emph{Berliner Theater. [Über unsere Kraft, zweiter Teil von Bjørnstjerne Bjørnson]}|pwk}. In: \emph{Neue Freie Presse}\pwindex{Neue Freie Presse@\emph{Neue Freie Presse}|pwk}, Nr. 13.144,
                     29. 3. 1901, Morgenblatt, S. 1–4.}}}\label{K_L03525-3} kam
               von Herzen und es war gut; denn es iſt \strikeout{\textcolor{gray}{Ar}} keine Kleinigkeit, den Gedankeninhalt eines{ }ſo gewaltigen Werkes\pwindex{Bjørnson, Bjørnstjerne 8.\,12.\,1832 Kvikne – 26.\,4.\,1910 Paris@\textsc{Bjørnson, Bjørnstjerne} (8.\,12.\,1832 Kvikne – 26.\,4.\,1910 Paris), \emph{Schriftsteller, Schriftsteller}!Über unsere Kraft. Zweiter Teil@\strich\emph{Über unsere Kraft. Zweiter Teil}|pwv} zu
               entwickeln, zumal wenn man gezwungen iſt, Manches zu{ }ſagen, was der Autor\pwindex{Bjørnson, Bjørnstjerne 8.\,12.\,1832 Kvikne – 26.\,4.\,1910 Paris@\textsc{Bjørnson, Bjørnstjerne} (8.\,12.\,1832 Kvikne – 26.\,4.\,1910 Paris), \emph{Schriftsteller, Schriftsteller}|pwv}{ }ſich{ }ſelbſt nicht gedacht hat! Und
               wenn es Ihnen \strikeout{Ih\textcolor{gray}{ne}} nicht gefallen hat,{ }ſo haben Sie mich eben nur wieder einmal unterſchätzt! Im
               Übrigen iſt es \strikeout{b\textcolor{gray}{ezeic}h}{ }ſehr lieb
               von Ihnen, daß Sie mir geſchrieben haben, wie Sie{ }ſchreiben. Vom Leben aber {\pb}\substVorne{}\textsuperscript{\textcolor{gray}{geschehe}}\substDazwischen{}wiſſen\substHinten{} Sie lange nicht{ }ſo viel, als Sie{ }ſich einbilden. Und es wäre{ }ſehr{ }ſchön,
               wenn ich in Wien\oindex{Wien@\textbf{Wien}, \emph{Verwaltungsgebiet}|pw} wäre und Sie Beide öfter{ }ſehen könnte; ich würde
               wahrſcheinlich weniger \label{K_L03525-4v}\edtext{Grillen
                  fangen}{\lemma{\textnormal{\emph{Grillen
                  fangen}}}\Cendnote{\textnormal{Anspielung auf eine Metapher im
                  vorigen Brief, vgl. XXXX Auszeichnungsfehler: Dokument L03524 nicht gefunden.}}}\label{K_L03525-4}! Und es iſt unerhört, daß ich heut{ }ſchon
               wieder Ihnen{ }ſchreiben muß,{ }ſtatt Ihrem Schweſterchen\pwindex{Steinrück, Elisabeth 19.\,11.\,1885 – 7.\,4.\,1920 Partenkirchen@\textsc{Steinrück, Elisabeth} (19.\,11.\,1885 – 7.\,4.\,1920 Partenkirchen)|pwv}, wie ich eigentlich vorhatte.\pend
           
\pstart
           So, und jetzt reden wir vernünftig!\pend
           
\pstart
           Dieſes kleine Fräulein \textsc{Liesl\pwindex{Steinrück, Elisabeth 19.\,11.\,1885 – 7.\,4.\,1920 Partenkirchen@\textsc{Steinrück, Elisabeth} (19.\,11.\,1885 – 7.\,4.\,1920 Partenkirchen)|pw}}{ }ſitzt ahnungslos in Wien\oindex{Wien@\textbf{Wien}, \emph{Verwaltungsgebiet}|pw} und weiß nicht, daß
                  \label{K_L03525-5v}\edtext{hier\oindex{Berlin@\textbf{Berlin}, \emph{Hauptstadt}|pwv} über ihr {\pb}Schickſal verhandelt}{\lemma{\textnormal{\emph{hier … verhandelt}}}\Cendnote{\textnormal{Siehe auch XXXX Auszeichnungsfehler: Dokument L03059 nicht gefunden.
               }}}\label{K_L03525-5} wird. Vorgeſtern{ }Abend war ich mit \textsc{Wolzogen\pwindex{Wolzogen, Ernst von 23.\,4.\,1855 Breslau – 30.\,7.\,1934 Puppling@\textsc{Wolzogen, Ernst von} (23.\,4.\,1855 Breslau – 30.\,7.\,1934 Puppling), \emph{Schriftsteller}|pw}} zuſammen. Es wurde über Neuengagements für das »Überbrettl\orgindex{Überbrettl@Überbrettl|pw}« geſprochen, und ich{ }ſtellte mit großer Energie die Candidatur
               Ihrer Schweſter\pwindex{Steinrück, Elisabeth 19.\,11.\,1885 – 7.\,4.\,1920 Partenkirchen@\textsc{Steinrück, Elisabeth} (19.\,11.\,1885 – 7.\,4.\,1920 Partenkirchen)|pwv} auf. \textsc{Wolzogen\pwindex{Wolzogen, Ernst von 23.\,4.\,1855 Breslau – 30.\,7.\,1934 Puppling@\textsc{Wolzogen, Ernst von} (23.\,4.\,1855 Breslau – 30.\,7.\,1934 Puppling), \emph{Schriftsteller}|pw}} hat ein Vorurtheil gegen die Wien\oindex{Wien@\textbf{Wien}, \emph{Verwaltungsgebiet}|pw}er Art, zu{ }ſpielen, und ich weiß nicht, ob es mir gelingen wird, dieſes Vorurtheil zu
               zerſtreuen. Das beſte Mittel wäre Fräulein \textsc{Liesls\pwindex{Steinrück, Elisabeth 19.\,11.\,1885 – 7.\,4.\,1920 Partenkirchen@\textsc{Steinrück, Elisabeth} (19.\,11.\,1885 – 7.\,4.\,1920 Partenkirchen)|pw}} perſönliches Erſcheinen. Ich
               frage alſo: Könnte dieſe {\pb}nacherwähnte junge Dame\pwindex{Steinrück, Elisabeth 19.\,11.\,1885 – 7.\,4.\,1920 Partenkirchen@\textsc{Steinrück, Elisabeth} (19.\,11.\,1885 – 7.\,4.\,1920 Partenkirchen)|pwv}, falls die Sache ernſt
               wird, auf einige Tage nach Berlin\oindex{Berlin@\textbf{Berlin}, \emph{Hauptstadt}|pw} kommen? Könnte{ }ſie eventuell gleich ins Engagement\orgindex{Überbrettl@Überbrettl|pwv} gehen? Ich betone: Dieſe Fragen{ }ſind vorläufig rein akademiſch;
               und es iſt noch{ }ſehr unſicher, ob die Sache{ }ſich wird praktiſch verwirklichen
               laſſen.\pend
           
\pstart
           Weitere Frage: wiſſen Sie einen für heiteren Geſang begabten jungen Mann, {\pb}Tenor oder Baryton, ebenfalls fürs »Überbrettl\orgindex{Überbrettl@Überbrettl|pw}«?\pend
           
\pstart
           Bitte um \uline{raſche} Antwort!\pend
           
\pstart
           Die Glümer\pwindex{Glümer, Marie 3.\,7.\,1867 Wien – 16.\,11.\,1925 München@\textsc{Glümer, Marie} (3.\,7.\,1867 Wien – 16.\,11.\,1925 München), \emph{Schauspielerin}|pw} iſt auf dem Wege der \label{K_L03525-6v}\edtext{Geneſung}{\lemma{\textnormal{\emph{Genesung}}}\Cendnote{\textnormal{Marie Glümer\pwindex{Glümer, Marie 3.\,7.\,1867 Wien – 16.\,11.\,1925 München@\textsc{Glümer, Marie} (3.\,7.\,1867 Wien – 16.\,11.\,1925 München), \emph{Schauspielerin}|pwk} war seit Anfang des Jahres krank, vgl. XXXX Auszeichnungsfehler: Dokument L03055 nicht gefunden und folgende Briefe Goldmanns\pwindex{Goldmann, Paul 31.\,1.\,1865 Breslau – 25.\,9.\,1935 Wien@\textsc{Goldmann, Paul} (31.\,1.\,1865 Breslau – 25.\,9.\,1935 Wien), \emph{Schriftsteller, Journalist}|pwk} an Schnitzler.}}}\label{K_L03525-6}.
               Sie hat vor einigen Tagen das Sanatorium\oindex{?? [Sanatorium, Aufenthaltsort von Marie Glümer 1901]@\textbf{?? [Sanatorium, Aufenthaltsort von Marie Glümer 1901]}, \emph{Sanatorium}|pw}
               verlaſſen.\pend
           
\pstart
           Und nun{ }ſchönen Dank für Alles! Und{ }ſeien Sie{ }ſammt dem Schweſterlein\pwindex{Steinrück, Elisabeth 19.\,11.\,1885 – 7.\,4.\,1920 Partenkirchen@\textsc{Steinrück, Elisabeth} (19.\,11.\,1885 – 7.\,4.\,1920 Partenkirchen)|pwv} herzlichſt gegrüßt von
               {\\[\baselineskip]}Ihrem ergebenen {\\[\baselineskip]}\spacefill\mbox{Dr. Paul Goldmann.}\pend
           \leftskip=0em{}\selectlanguage{ngerman}\endnumbering\briefempfaengerindex{Schnitzler, Olga@\textsc{Schnitzler, Olga}!zzzGoldmann, Paul@\emph{von Paul Goldmann}!1901-04-032@{3. 4. [1901]}|)be}\mylabel{L03525h}  \newcommand{\dateiname}{L03525}\newcommand{\titel}{Paul Goldmann an Olga Gussmann, 3. 4. [1901]}\newcommand{\editorInnen}{Martin Anton Müller und Laura Untner}%% latex-leseansicht-abspann.tex
%% Abspann für die Leseansicht.
%% Der Schalter \ifkorrekturansicht ist bereits durch den Vorspann gesetzt.

%% latex-abspann.tex
%% Gemeinsamer Abspann für Korrekturansicht und Leseansicht.
%% Setzt den Schalter \ifkorrekturansicht voraus (gesetzt in den
%% einbindenden Dateien latex-korrekturansicht-abspann.tex bzw.
%% latex-leseansicht-abspann.tex).
%% ---------------------------------------------------------------

\normalsize

% Das esempio-Environment wird nur in der Leseansicht benötigt
\ifkorrekturansicht\else
\newenvironment{esempio}[3]%
{
    \vspace{1.5ex}
    \rlap{\underline{#1}}
    \par
    \setlength{\parindent}{0cm}
    \nopagebreak
    \leftskip=#2cm
    \rightskip=#3cm
}
{
    \par
}
\fi

\doendnotes{C}
\bigskip
\vfill

\clearpage

\footnotesize

\ifkorrekturansicht
  \lohead{\textsc{register}}
\fi

% theindex-Environment neu definieren ohne reledmac
\makeatletter
\renewenvironment{theindex}{%
  \ifkorrekturansicht
    \section*{\indexname}%
  \else
    \subsubsection*{Index der erwähnten Entitäten}%
  \fi
  \setlength{\parindent}{0pt}%
  \setlength{\parskip}{0pt plus 0.3pt}%
  \let\item\@idxitem
}{%
  \ifkorrekturansicht\clearpage\fi
}
\makeatother

\IfFileExists{\jobname-pw.ind}{\input{\jobname-pw.ind}}{}

% Quellenangabe nur in der Leseansicht
\ifkorrekturansicht\else
% Fallback-Definitionen, falls die .tex-Datei \titel etc. nicht gesetzt hat
\providecommand{\titel}{}
\providecommand{\editorInnen}{}
\providecommand{\dateiname}{\jobname}

\vspace{3cm}

\vfill

\footnotesize
\textsc{Quelle}: \titel. Herausgegeben von {\editorInnen}. In: \emph{Arthur Schnitzler: Briefwechsel mit Autorinnen und Autoren}.
 Digitale Edition, https://schnitzler-briefe.acdh.oeaw.ac.at/{\dateiname}.html (Stand \today)
\fi

\end{document}


