%% latex-korrekturansicht-vorspann.tex
%% Vorspann für die Korrekturansicht.
%% Lädt die gemeinsame Datei latex-vorspann.tex mit gesetztem Schalter.

\newif\ifkorrekturansicht
\korrekturansichttrue

\input{../tex-inputs/latex-vorspann}


\section[ Paul Goldmann an Olga Gussmann, 3. 4. {[}1901{]}]{L03525 Paul Goldmann an Olga Gussmann, 3. 4. {[}1901{]}}
\nopagebreak\mylabel{L03525v}
\rehead{ }\normalsize\beginnumbering\briefempfaengerindex{Schnitzler, Olga@\textsc{Schnitzler, Olga}!zzzGoldmann, Paul@\emph{von Paul Goldmann}!1901-04-032@{3. 4. {[}1901{]}}|(be}
\toendnotes[C]{\smallbreak\pagebreak[2]}\Standort{DLA, A:Schnitzler, HS.NZ85.1.5247.}
\physDesc{Brief, 2 Blätter, 6 Seiten, 2393 Zeichen
\newline{}Handschrift: blaue Tinte, deutsche Kurrent
\newline{}Ordnung: mit Bleistift von Arthur Schnitzler das
                                 Jahr »1901.« vermerkt }\toendnotes[C]{\smallbreak}
\pstart
           \raggedleft{}{\pb}\textcolor{gray}{\textbf{DESSAUERSTRASSE 19\oindex{Dessauer Strasse@\textbf{Dessauer Straße}, \emph{Straße (K.STR)}|pw}}}\pend
           
\pstart
           Berlin\oindex{Berlin@\textbf{Berlin}, \emph{P.PPLC}|pw}, 3. April.\pend
           
\pstart\center{}Liebes Fräulein \textsc{Olga},\pend\vspace{0.5em}
\pstart
           Schön, ſchön und ſchön! Und ich habe \uline{doch} Recht! Und
               wenn Sie werden ſo grob mit mir ſein, ſo werde ich bei Ihrem erſten Auftreten in Berlin\oindex{Berlin@\textbf{Berlin}, \emph{P.PPLC}|pw} eine ſchlechte Kritik über Sie ſchreiben!
               Oder ihnen ſonſt etwas Furchtbares anthun! Und wenn \uline{alle} Menſchen \label{K_L03525-1v}\edtext{einſam}{\lemma{\textnormal{\emph{einſam}}}\Cendnote{\textnormal{Siehe Paul Goldmann an Arthur Schnitzler, 7. 7. 1907.
               }}}\label{K_L03525-1} ſind (was übrigens nicht wahr iſt), ſo will ich es \uline{nicht} ſein, \label{K_L03525-2v}\edtext{Himmelkreuzſchockſchwerenoth}{\lemma{\textnormal{\emph{Himmelkreuzſchockſchwerenoth}}}\Cendnote{\textnormal{umgangssprachlicher Ausruf}}}\label{K_L03525-2}! Und wenn \uline{alle}
               Frauen eine Bagage ſind, ſo will ich doch eine haben, ſchon um auf {\pb}ſie ſchimpfen zu können! Und mein \label{K_L03525-3v}\edtext{Feuilleton\pwindex{Berliner Theater. [Ueber unsere Kraft, zweiter Teil von Bjørnstjerne Bjørnson]@\emph{Berliner Theater. [Über unsere Kraft, zweiter Teil von Bjørnstjerne Bjørnson]}|pwv}}{\lemma{\textnormal{\emph{Feuilleton}}}\Cendnote{\textnormal{Paul Goldmann\pwindex{Goldmann, Paul 31.01.1865 – 25.09.1935@\textsc{Goldmann, Paul} (31.01.1865 – 25.09.1935), \emph{Schriftsteller/Schriftstellerin, Journalist/Journalistin}|pwk}: \emph{Berliner Theater}\pwindex{Berliner Theater. [Ueber unsere Kraft, zweiter Teil von Bjørnstjerne Bjørnson]@\emph{Berliner Theater. [Über unsere Kraft, zweiter Teil von Bjørnstjerne Bjørnson]}|pwk}. In: \emph{Neue Freie Presse}\pwindex{Neue Freie Presse@\emph{Neue Freie Presse}|pwk}, Nr. 13.144,
                     29. 3. 1901, Morgenblatt, S. 1–4.}}}\label{K_L03525-3} kam
               von Herzen und es war gut; denn es iſt \strikeout{\textcolor{gray}{Ar}} keine Kleinigkeit, den Gedankeninhalt eines ſo gewaltigen Werkes\pwindex{Ueber unsere Kraft. Zweiter Teil@\emph{Über unsere Kraft. Zweiter Teil}|pwv} zu
               entwickeln, zumal wenn man gezwungen iſt, Manches zu ſagen, was der Autor\pwindex{Bjørnson, Bjørnstjerne 1832-12-08 – 1910-04-26@\textsc{Bjørnson, Bjørnstjerne} (1832-12-08 – 1910-04-26), \emph{Schriftsteller/Schriftstellerin}|pwv} ſich ſelbſt nicht gedacht hat! Und
               wenn es Ihnen \strikeout{Ih\textcolor{gray}{ne}} nicht gefallen hat, ſo haben Sie mich eben nur wieder einmal unterſchätzt! Im
               Übrigen iſt es \strikeout{b\textcolor{gray}{ezeic}h} ſehr lieb
               von Ihnen, daß Sie mir geſchrieben haben, wie Sie ſchreiben. Vom Leben aber {\pb}\substVorne{}\textsuperscript{\textcolor{gray}{geschehe}}\substDazwischen{}wiſſen\substHinten{} Sie lange nicht ſo viel, als Sie ſich einbilden. Und es wäre ſehr ſchön,
               wenn ich in Wien\oindex{Wien@\textbf{Wien}, \emph{A.ADM2}|pw} wäre und Sie Beide öfter ſehen könnte; ich würde
               wahrſcheinlich weniger \label{K_L03525-4v}\edtext{Grillen
                  fangen}{\lemma{\textnormal{\emph{Grillen
                  fangen}}}\Cendnote{\textnormal{Anspielung auf eine Metapher im
                  vorigen Brief, vgl. Paul Goldmann an Olga Gussmann, 1. 4. [1901].}}}\label{K_L03525-4}! Und es iſt unerhört, daß ich heut ſchon
               wieder Ihnen ſchreiben muß, ſtatt Ihrem Schweſterchen\pwindex{Steinrueck, Elisabeth 19.11.1885 – 07.04.1920@\textsc{Steinrück, Elisabeth} (19.11.1885 – 07.04.1920)|pwv}, wie ich eigentlich vorhatte.\pend
           
\pstart
           So, und jetzt reden wir vernünftig!\pend
           
\pstart
           Dieſes kleine Fräulein \textsc{Liesl\pwindex{Steinrueck, Elisabeth 19.11.1885 – 07.04.1920@\textsc{Steinrück, Elisabeth} (19.11.1885 – 07.04.1920)|pw}} ſitzt ahnungslos in Wien\oindex{Wien@\textbf{Wien}, \emph{A.ADM2}|pw} und weiß nicht, daß
                  \label{K_L03525-5v}\edtext{hier\oindex{Berlin@\textbf{Berlin}, \emph{P.PPLC}|pwv} über ihr {\pb}Schickſal verhandelt}{\lemma{\textnormal{\emph{hier … verhandelt}}}\Cendnote{\textnormal{Siehe auch Paul Goldmann an Arthur Schnitzler, 18. 2. [1901].
               }}}\label{K_L03525-5} wird. Vorgeſtern{ }Abend war ich mit \textsc{Wolzogen\pwindex{Wolzogen, Ernst von 23.04.1855 – 30.07.1934@\textsc{Wolzogen, Ernst von} (23.04.1855 – 30.07.1934), \emph{Schriftsteller/Schriftstellerin}|pw}} zuſammen. Es wurde über Neuengagements für das »Überbrettl\orgindex{Ueberbrettl@Überbrettl|pw}« geſprochen, und ich ſtellte mit großer Energie die Candidatur
               Ihrer Schweſter\pwindex{Steinrueck, Elisabeth 19.11.1885 – 07.04.1920@\textsc{Steinrück, Elisabeth} (19.11.1885 – 07.04.1920)|pwv} auf. \textsc{Wolzogen\pwindex{Wolzogen, Ernst von 23.04.1855 – 30.07.1934@\textsc{Wolzogen, Ernst von} (23.04.1855 – 30.07.1934), \emph{Schriftsteller/Schriftstellerin}|pw}} hat ein Vorurtheil gegen die Wien\oindex{Wien@\textbf{Wien}, \emph{A.ADM2}|pw}er Art, zu
               ſpielen, und ich weiß nicht, ob es mir gelingen wird, dieſes Vorurtheil zu
               zerſtreuen. Das beſte Mittel wäre Fräulein \textsc{Liesls\pwindex{Steinrueck, Elisabeth 19.11.1885 – 07.04.1920@\textsc{Steinrück, Elisabeth} (19.11.1885 – 07.04.1920)|pw}} perſönliches Erſcheinen. Ich
               frage alſo: Könnte dieſe {\pb}nacherwähnte junge Dame\pwindex{Steinrueck, Elisabeth 19.11.1885 – 07.04.1920@\textsc{Steinrück, Elisabeth} (19.11.1885 – 07.04.1920)|pwv}, falls die Sache ernſt
               wird, auf einige Tage nach Berlin\oindex{Berlin@\textbf{Berlin}, \emph{P.PPLC}|pw} kommen? Könnte
               ſie eventuell gleich ins Engagement\orgindex{Ueberbrettl@Überbrettl|pwv} gehen? Ich betone: Dieſe Fragen ſind vorläufig rein akademiſch;
               und es iſt noch ſehr unſicher, ob die Sache ſich wird praktiſch verwirklichen
               laſſen.\pend
           
\pstart
           Weitere Frage: wiſſen Sie einen für heiteren Geſang begabten jungen Mann, {\pb}Tenor oder Baryton, ebenfalls fürs »Überbrettl\orgindex{Ueberbrettl@Überbrettl|pw}«?\pend
           
\pstart
           Bitte um \uline{raſche} Antwort!\pend
           
\pstart
           Die Glümer\pwindex{Gluemer, Marie 03.07.1867 – 16.11.1925@\textsc{Glümer, Marie} (03.07.1867 – 16.11.1925), \emph{Schauspieler/Schauspielerin}|pw} iſt auf dem Wege der \label{K_L03525-6v}\edtext{Geneſung}{\lemma{\textnormal{\emph{Geneſung}}}\Cendnote{\textnormal{Marie Glümer\pwindex{Gluemer, Marie 03.07.1867 – 16.11.1925@\textsc{Glümer, Marie} (03.07.1867 – 16.11.1925), \emph{Schauspieler/Schauspielerin}|pwk} war seit Anfang des Jahres krank, vgl. Paul Goldmann an Arthur Schnitzler, 22. 1. [1901] und folgende Briefe Goldmanns\pwindex{Goldmann, Paul 31.01.1865 – 25.09.1935@\textsc{Goldmann, Paul} (31.01.1865 – 25.09.1935), \emph{Schriftsteller/Schriftstellerin, Journalist/Journalistin}|pwk} an Schnitzler.}}}\label{K_L03525-6}.
               Sie hat vor einigen Tagen das Sanatorium\oindex{?? [Sanatorium, Aufenthaltsort von Marie Gluemer 1901]@\textbf{?? [Sanatorium, Aufenthaltsort von Marie Glümer 1901]}, \emph{Sanatorium (K.SAN)}|pw}
               verlaſſen.\pend
           
\pstart
           Und nun ſchönen Dank für Alles! Und ſeien Sie ſammt dem Schweſterlein\pwindex{Steinrueck, Elisabeth 19.11.1885 – 07.04.1920@\textsc{Steinrück, Elisabeth} (19.11.1885 – 07.04.1920)|pwv} herzlichſt gegrüßt von
               {\\[\baselineskip]}Ihrem ergebenen {\\[\baselineskip]}\spacefill\mbox{Dr. Paul Goldmann.}\pend
           \leftskip=0em{}\selectlanguage{ngerman}\endnumbering\briefempfaengerindex{Schnitzler, Olga@\textsc{Schnitzler, Olga}!zzzGoldmann, Paul@\emph{von Paul Goldmann}!1901-04-032@{3. 4. {[}1901{]}}|)be}\mylabel{L03525h}  \normalsize

\doendnotes{C}
\bigskip
\vfill

\clearpage

\footnotesize

\lohead{\textsc{register}}

% Definiere theindex-Environment komplett neu ohne reledmac
\makeatletter
\renewenvironment{theindex}{%
  \section*{\indexname}%
  \setlength{\parindent}{0pt}%
  \setlength{\parskip}{0pt plus 0.3pt}%
  \let\item\@idxitem
}{%
  \clearpage
}
\makeatother

\IfFileExists{\jobname-pw.ind}{\input{\jobname-pw.ind}}{}

\end{document}

      