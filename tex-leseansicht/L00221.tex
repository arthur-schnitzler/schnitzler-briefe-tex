%% latex-leseansicht-vorspann.tex
%% Vorspann für die Leseansicht.
%% Lädt die gemeinsame Datei latex-vorspann.tex mit nicht gesetztem Schalter.

\newif\ifkorrekturansicht
\korrekturansichtfalse

\input{../tex-inputs/latex-vorspann}


         
         \renewcommand{\erwaehntePersonen}{Personen: Cesare Lombroso}
         \renewcommand{\erwaehnteOrte}{Orte: Berlin, Friedrichshagen, Wien}
         \renewcommand{\erwaehnteWerke}{Werke: Die Braut, Die Frau als Verbrecherin und Prostituierte}
               \section[Wilhelm Bölsche an Arthur Schnitzler, 12. 6. 1893]{ Wilhelm Bölsche an Arthur Schnitzler, 12. 6. 1893}\nopagebreak\mylabel{v}\rehead{ }\begin{ledgroupsized}[t]{13cm}\normalsize\beginnumbering \toendnotes[C]{\smallbreak\pagebreak[2]} \Standort{DLA, A:Schnitzler, HS.NZ85.1.2577,7.}
\physDesc{Brief, 1 Blatt, 3 Seiten
\newline{}Handschrift: schwarze Tinte, deutsche Kurrent
\newline{}Schnitzler: mit rotem Buntstift nummeriert: »8« und eine
                                 Unterstreichung }\buchAbdrucke{\weitereDrucke{Wilhelm Bölsche: \emph{Briefwechsel. Mit Autoren der Freien Bühne}. Hg. Gerd-Hermann Susen. Berlin: \emph{Weidler} 2010, S. 687–688 (Werke und Briefe. Wissenschaftliche Ausgabe, Briefe I).} }\toendnotes[C]{\smallbreak}\pstart
           {\pb}\textcolor{gray}{\textbf{\textit{Wilhelm Bölsche}}}\hfill 12. VI. 93\pend
           \pstart
           \textcolor{gray}{\textbf{\textit{Friedrichshagen\oindex{Friedrichshagen@\textbf{Friedrichshagen}|pw}.}}}\pend
           \pstart{}Hochgeehrter Herr Dr!\pend\pstart
           Sie haben ein Recht, ungehalten zu ſein, aber ich wünſchte Sie in meine Lage, um dann
               Ihr Urteil zu hören. Ihr Mahnbrief iſt bis jetzt unbeantwortet geblieben, weil ich
               verreiſt war, – eine äußerſt notwendige Ruhepauſe! Daß Ihre Novelle\pwindex{Schnitzler, Arthur 15.05.1862 – 21.10.1931@\textsc{Schnitzler, Arthur} (15.05.1862 – 21.10.1931), \emph{Schriftsteller, Mediziner}!Braut1932@\strich\emph{Die Braut} {[}1932{]}|pw} nicht vorher erledigt war, iſt ja eine redaktionelle
               Sünde. Bei der Maſſe der Einſendung und in Anbetracht des Umſtandes, daß ich die
               Redaktion bis in jede Couvertadreſſe hinein ganz allein zu beſorgen habe, iſt es mir
               allerdings noch nicht einmal als »Ideal« aufgetaucht, ſpäteſtens in 8 Tagen {\pb}jede Einſendung erledigen zu können, zumal da ¾ der
               Einſender ſelbſt bei dicken Romanen und Dramen nicht bloß redaktionelle, ſondern auch
               noch »wirkliche« Urteile verlangen.\pend
           \pstart
           Was Ihre Novelle\pwindex{Schnitzler, Arthur 15.05.1862 – 21.10.1931@\textsc{Schnitzler, Arthur} (15.05.1862 – 21.10.1931), \emph{Schriftsteller, Mediziner}!Braut1932@\strich\emph{Die Braut} {[}1932{]}|pw} anbetrifft, ſo iſt ſie mir
               pſychologiſch nicht recht durchdringlich: in dieſer fragmentariſchen Form lieſt ſie
               ſich bloß wie eine Umſchreibung des Lombroſo\pwindex{Lombroso, Cesare 18.11.1836 – 19.10.1909@\textsc{Lombroso, Cesare} (18.11.1836 – 19.10.1909), \emph{Mediziner, Anthropologe}|pw}’ſchen Dogma’s von der gleichſam \label{K_L00221_1v}\edtext{prädeſtinierten Dirne}{\lemma{\textnormal{\emph{prädeſtinierten Dirne}}}\Cendnote{\textnormal{In seinem Werk \emph{La donna
                     delinquente. La prostituta e la donna normale}\pwindex{Lombroso, Cesare 18.11.1836 – 19.10.1909@\textsc{Lombroso, Cesare} (18.11.1836 – 19.10.1909), \emph{Mediziner, Anthropologe}!Frau als Verbrecherin und Prostituierte1893@\strich\emph{Die Frau als Verbrecherin und Prostituierte} {[}1893{]}|pwk} (1893, deutsch
                     \emph{Das Weib als Verbrecherin und Prostituierte}\pwindex{Lombroso, Cesare 18.11.1836 – 19.10.1909@\textsc{Lombroso, Cesare} (18.11.1836 – 19.10.1909), \emph{Mediziner, Anthropologe}!Frau als Verbrecherin und Prostituierte1893@\strich\emph{Die Frau als Verbrecherin und Prostituierte} {[}1893{]}|pwk},
                     1894) vertrat Cesare Lombroso\pwindex{Lombroso, Cesare 18.11.1836 – 19.10.1909@\textsc{Lombroso, Cesare} (18.11.1836 – 19.10.1909), \emph{Mediziner, Anthropologe}|pwk}
                  die These, dass die Prostitution mancher Frauen aus ihren ›natürlichen‹ Anlagen
                  erklärbar sei und stellte eine Analogie zu den Männern her, die durch biologische
                  Anlagen zu Verbrechern würden.}}}\label{K_L00221_1h}, aber nicht wie eine Dichtung. Entſchieden
               verlangt dieſer Stoff viel mehr Fleiſch und Blut, und vielleicht bearbeiten Sie ihn
               ſo noch einmal. Die Szene\pwindex{Schnitzler, Arthur 15.05.1862 – 21.10.1931@\textsc{Schnitzler, Arthur} (15.05.1862 – 21.10.1931), \emph{Schriftsteller, Mediziner}!Braut1932@\strich\emph{Die Braut} {[}1932{]}|pwv}, {\pb}wie das Mädchen dem Bräutigam ihre Gefühle bekennt,
               halte ich für pſychologiſch ſehr unwahrſcheinlich!\pend
           \pstart
           Mit herzlichem Gruß{\\[\baselineskip]} Ihr{\\[\baselineskip]}\spacefill\mbox{W. Bölsche}\pend
           \leftskip=0em{}
         
         \endnumbering\mylabel{h}\end{ledgroupsized}  \newcommand{\dateiname}{L00221}\newcommand{\titel}{Wilhelm Bölsche an Arthur Schnitzler, 12. 6. 1893}\newcommand{\editorInnen}{Martin Anton Müller und Gerd-Hermann Susen}%% latex-leseansicht-abspann.tex
%% Abspann für die Leseansicht.
%% Der Schalter \ifkorrekturansicht ist bereits durch den Vorspann gesetzt.

%% latex-abspann.tex
%% Gemeinsamer Abspann für Korrekturansicht und Leseansicht.
%% Setzt den Schalter \ifkorrekturansicht voraus (gesetzt in den
%% einbindenden Dateien latex-korrekturansicht-abspann.tex bzw.
%% latex-leseansicht-abspann.tex).
%% ---------------------------------------------------------------

\normalsize

% Das esempio-Environment wird nur in der Leseansicht benötigt
\ifkorrekturansicht\else
\newenvironment{esempio}[3]%
{
    \vspace{1.5ex}
    \rlap{\underline{#1}}
    \par
    \setlength{\parindent}{0cm}
    \nopagebreak
    \leftskip=#2cm
    \rightskip=#3cm
}
{
    \par
}
\fi

\doendnotes{C}
\bigskip
\vfill

\clearpage

\footnotesize

\ifkorrekturansicht
  \lohead{\textsc{register}}
\fi

% theindex-Environment neu definieren ohne reledmac
\makeatletter
\renewenvironment{theindex}{%
  \ifkorrekturansicht
    \section*{\indexname}%
  \else
    \subsubsection*{Index der erwähnten Entitäten}%
  \fi
  \setlength{\parindent}{0pt}%
  \setlength{\parskip}{0pt plus 0.3pt}%
  \let\item\@idxitem
}{%
  \ifkorrekturansicht\clearpage\fi
}
\makeatother

\IfFileExists{\jobname-pw.ind}{\input{\jobname-pw.ind}}{}

% Quellenangabe nur in der Leseansicht
\ifkorrekturansicht\else
% Fallback-Definitionen, falls die .tex-Datei \titel etc. nicht gesetzt hat
\providecommand{\titel}{}
\providecommand{\editorInnen}{}
\providecommand{\dateiname}{\jobname}

\vspace{3cm}

\vfill

\footnotesize
\textsc{Quelle}: \titel. Herausgegeben von {\editorInnen}. In: \emph{Arthur Schnitzler: Briefwechsel mit Autorinnen und Autoren}.
 Digitale Edition, https://schnitzler-briefe.acdh.oeaw.ac.at/{\dateiname}.html (Stand \today)
\fi

\end{document}


      