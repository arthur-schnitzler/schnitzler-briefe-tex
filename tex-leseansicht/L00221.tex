%% latex-korrekturansicht-vorspann.tex
%% Vorspann für die Korrekturansicht.
%% Lädt die gemeinsame Datei latex-vorspann.tex mit gesetztem Schalter.

\newif\ifkorrekturansicht
\korrekturansichttrue

\input{../tex-inputs/latex-vorspann}


\section[Wilhelm Bölsche an Arthur Schnitzler, 12. 6. 1893]{L00221 Wilhelm Bölsche an Arthur Schnitzler, 12. 6. 1893}
\nopagebreak\mylabel{L00221v}
\rehead{ }\normalsize\beginnumbering\briefempfaengerindex{Schnitzler, Arthur@\textsc{Schnitzler, Arthur}!zzzBoelsche, Wilhelm@\emph{von Wilhelm Bölsche}!1893-06-121@{12. 6. 1893}|(be}
\toendnotes[C]{\smallbreak\pagebreak[2]}\Standort{DLA, A:Schnitzler, HS.NZ85.1.2577,7.}
\physDesc{Brief, 1 Blatt, 3 Seiten, 1273 Zeichen
\newline{}Handschrift: schwarze Tinte, deutsche Kurrent
\newline{}Schnitzler: mit rotem Buntstift nummeriert: »8« und eine
                                 Unterstreichung }
\buchAbdrucke{\weitereDrucke{Wilhelm Bölsche: \emph{Briefwechsel. Mit Autoren der Freien Bühne}. Berlin: \emph{Weidler} 2010, S. 687–688.} }\toendnotes[C]{\smallbreak}
\pstart
           
\pstart
           {\pb}\textcolor{gray}{\textbf{\textit{Wilhelm Bölsche}}}\pend
           
\pstart
           \raggedleft{}12. VI. 93\pend
           \pend
           
\pstart
           \textcolor{gray}{\textbf{\textit{Friedrichshagen\oindex{Friedrichshagen@\textbf{Friedrichshagen}, \emph{P.PPLX}|pw}.}}}\pend
           
\pstart{}Hochgeehrter Herr Dr!\pend\vspace{0.5em}
\pstart
           Sie haben ein Recht, ungehalten zu ſein, aber ich wünſchte Sie in meine Lage, um dann
               Ihr Urteil zu hören. Ihr Mahnbrief iſt bis jetzt unbeantwortet geblieben, weil ich
               verreiſt war, – eine äußerſt notwendige Ruhepauſe! Daß Ihre Novelle\pwindex{Braut@\emph{Die Braut}|pw} nicht vorher erledigt war, iſt ja eine redaktionelle
               Sünde. Bei der Maſſe der Einſendung und in Anbetracht des Umſtandes, daß ich die
               Redaktion bis in jede Couvertadreſſe hinein ganz allein zu beſorgen habe, iſt es mir
               allerdings noch nicht einmal als »Ideal« aufgetaucht, ſpäteſtens in 8 Tagen {\pb}jede Einſendung erledigen zu können, zumal da ¾ der
               Einſender ſelbſt bei dicken Romanen und Dramen nicht bloß redaktionelle, ſondern auch
               noch »wirkliche« Urteile verlangen.\pend
           
\pstart
           Was Ihre Novelle\pwindex{Braut@\emph{Die Braut}|pw} anbetrifft, ſo iſt ſie mir
               pſychologiſch nicht recht durchdringlich: in dieſer fragmentariſchen Form lieſt ſie
               ſich bloß wie eine Umſchreibung des Lombroſo\pwindex{Lombroso, Cesare 18.11.1836 – 19.10.1909@\textsc{Lombroso, Cesare} (18.11.1836 – 19.10.1909), \emph{Mediziner/Medizinerin, Psychologe/Psychologin, Anthropologe/Anthropologin}|pw}’ſchen Dogma’s von der gleichſam \label{K_L00221-1v}\edtext{prädeſtinierten Dirne}{\lemma{\textnormal{\emph{prädeſtinierten Dirne}}}\Cendnote{\textnormal{In seinem Werk \emph{La donna
                     delinquente. La prostituta e la donna normale}\pwindex{Weib als Verbrecherin und Prostituierte. Anthropologische Studien, gegruendet auf eine Darstellung der Biologie und Psychologie des normalen Weibes.@\emph{Das Weib als Verbrecherin und Prostituierte. Anthropologische Studien, gegründet auf eine Darstellung der Biologie und Psychologie des normalen Weibes.}|pwk} (1893, deutsch
                     \emph{Das Weib als Verbrecherin und
                     Prostituierte}\pwindex{Weib als Verbrecherin und Prostituierte. Anthropologische Studien, gegruendet auf eine Darstellung der Biologie und Psychologie des normalen Weibes.@\emph{Das Weib als Verbrecherin und Prostituierte. Anthropologische Studien, gegründet auf eine Darstellung der Biologie und Psychologie des normalen Weibes.}|pwk}, 1894) vertrat Cesare Lombroso\pwindex{Lombroso, Cesare 18.11.1836 – 19.10.1909@\textsc{Lombroso, Cesare} (18.11.1836 – 19.10.1909), \emph{Mediziner/Medizinerin, Psychologe/Psychologin, Anthropologe/Anthropologin}|pwk} die These, dass die Prostitution mancher
                  Frauen aus ihren ›natürlichen‹ Anlagen erklärbar sei, und stellte eine Analogie zu
                  den Männern her, die durch biologische Anlagen zu Verbrechern würden.}}}\label{K_L00221-1}, aber
               nicht wie eine Dichtung. Entſchieden verlangt dieſer Stoff viel mehr Fleiſch und
               Blut, und vielleicht bearbeiten Sie ihn ſo noch einmal. Die Szene\pwindex{Braut@\emph{Die Braut}|pwv}, {\pb}wie
               das Mädchen dem Bräutigam ihre Gefühle bekennt, halte ich für pſychologiſch ſehr
               unwahrſcheinlich!\pend
           
\pstart
           Mit herzlichem Gruß{\\[\baselineskip]} Ihr{\\[\baselineskip]}\spacefill\mbox{W. Bölsche}\pend
           \leftskip=0em{}\selectlanguage{ngerman}\endnumbering\briefempfaengerindex{Schnitzler, Arthur@\textsc{Schnitzler, Arthur}!zzzBoelsche, Wilhelm@\emph{von Wilhelm Bölsche}!1893-06-121@{12. 6. 1893}|)be}\mylabel{L00221h}  \normalsize

\doendnotes{C}
\bigskip
\vfill

\clearpage

\footnotesize

\lohead{\textsc{register}}

% Definiere theindex-Environment komplett neu ohne reledmac
\makeatletter
\renewenvironment{theindex}{%
  \section*{\indexname}%
  \setlength{\parindent}{0pt}%
  \setlength{\parskip}{0pt plus 0.3pt}%
  \let\item\@idxitem
}{%
  \clearpage
}
\makeatother

\IfFileExists{\jobname-pw.ind}{\input{\jobname-pw.ind}}{}

\end{document}

      