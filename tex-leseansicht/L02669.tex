%% latex-leseansicht-vorspann.tex
%% Vorspann für die Leseansicht.
%% Lädt die gemeinsame Datei latex-vorspann.tex mit nicht gesetztem Schalter.

\newif\ifkorrekturansicht
\korrekturansichtfalse

\input{../tex-inputs/latex-vorspann}


\section[Paul Goldmann an Arthur Schnitzler, 27. 10. 1891]{L02669 Paul Goldmann an Arthur Schnitzler, 27. 10. 1891}
\nopagebreak\mylabel{L02669v}
\rehead{ }\normalsize\beginnumbering\briefempfaengerindex{Schnitzler, Arthur@\textsc{Schnitzler, Arthur}!zzzGoldmann, Paul@\emph{von Paul Goldmann}!1891-10-271@{27. 10. 1891}|(be}
\toendnotes[C]{\smallbreak\pagebreak[2]}
\correspDesc{Versand  durch Paul Goldmann am 27. 10. 1891 in Brüssel
\newline{}Erhalt  durch Arthur Schnitzler im Zeitraum [28. 10. 1891 – 1. 11. 1891?] in Wien}\toendnotes[C]{\smallbreak}
\Standort{DLA, A:Schnitzler, HS.NZ85.1.3162.}
\physDesc{Brief, 3 Blätter, 10 Seiten, 6969 Zeichen
\newline{}Handschrift: blaue Tinte, deutsche Kurrent
\newline{}Schnitzler: mit rotem Buntstift zwei Unterstreichungen }\toendnotes[C]{\smallbreak}
\pstart
           \centering{}{\pb}\textcolor{gray}{\textbf{Dr. jur. Paul Goldmann}}\pend
           
\pstart
           \centering{}\textcolor{gray}{\textbf{\begin{otherlanguage}{french}Correspondant de la »Gazette de Francfort\orgindex{Frankfurter Zeitung@Frankfurter Zeitung|pw}«\end{otherlanguage}}}\pend
           
\pstart
           \centering{}\textcolor{gray}{\textbf{\begin{otherlanguage}{french}Bruxelles, 21, rue des Plantes\end{otherlanguage}\oindex{rue des Plantes@\textbf{rue des Plantes}, \emph{Straße}|pw}.}}\pend
           
\pstart
           \raggedleft{}Brüſſel\oindex{Brüssel@\textbf{Brüssel}, \emph{Hauptstadt}|pw}, 27. October 91.\pend
           
\pstart\center{}Mein lieber Arthur!\pend\vspace{0.5em}
\pstart
           Ich entſchließe mich nicht leicht zum Schreiben an Dich, offen geſtanden. Denn ich
               komme mir vor, wie \strikeout{ein\textcolor{gray}{er}} ein läſtiger Mahner, der eine Gefühlsſchuld eintreiben will, zu deren
               Honorirung nicht mehr der nöthige Beſtand vorhanden iſt. Alle Symptome{ }ſprechen mir
               dafür, daß das gekommen iſt, was kommen mußte: Daß ich für Euch ein Stück
               Vergangenheit geworden bin; und als{ }ſolches habe ich natürlich weit hinter den Sachen
               Eurer Gegenwart zurückzuſtehen. Ich bin eine Erinnerung für einſame Sonntag
               Nachmittage geworden{\dotsfive}\pend
           
\pstart
           Alſo einiges von mir. In Brüſſel\oindex{Brüssel@\textbf{Brüssel}, \emph{Hauptstadt}|pw} geht es mir
               jetzt etwas beſſer – moraliſch wenigſtens. Ich bin den Leuten hier ein klein wenig
               näher getreten, habe {\pb}manchen lieben Menſchen,
               manche{ }ſchöne Künſtlernatur gefunden und bin mit dem Einen oder dem Andern wenn auch
               nicht Freund,{ }ſo doch gut bekannt geworden. \strikeout{\textcolor{gray}{×}} Sogar ein kleines Milieu junger Künſtler und Lebemänner in meinem Alter, ein
                  \label{K_L02669-1v}\edtext{\textsc{Milieu} der \textsc{Hectors\pwindex{Augier, Émile 17.\,9.\,1820 Valence – 25.\,10.\,1889 Croissy-sur-Seine@\textsc{Augier, Émile} (17.\,9.\,1820 Valence – 25.\,10.\,1889 Croissy-sur-Seine), \emph{Schriftsteller}!Le Gendre de M. Poirier. Comédie en 4 actes en prose@\strich\emph{Le Gendre de M. Poirier. Comédie en 4 actes en prose}|pwv}\pwindex{Sandeau, Jules 19.\,2.\,1811 Aubusson – 24.\,4.\,1883 Paris@\textsc{Sandeau, Jules} (19.\,2.\,1811 Aubusson – 24.\,4.\,1883 Paris), \emph{Schriftsteller}!Le Gendre de M. Poirier. Comédie en 4 actes en prose@\strich\emph{Le Gendre de M. Poirier. Comédie en 4 actes en prose}|pwv}} und \textsc{Gastons\pwindex{Augier, Émile 17.\,9.\,1820 Valence – 25.\,10.\,1889 Croissy-sur-Seine@\textsc{Augier, Émile} (17.\,9.\,1820 Valence – 25.\,10.\,1889 Croissy-sur-Seine), \emph{Schriftsteller}!Le Gendre de M. Poirier. Comédie en 4 actes en prose@\strich\emph{Le Gendre de M. Poirier. Comédie en 4 actes en prose}|pwv}\pwindex{Sandeau, Jules 19.\,2.\,1811 Aubusson – 24.\,4.\,1883 Paris@\textsc{Sandeau, Jules} (19.\,2.\,1811 Aubusson – 24.\,4.\,1883 Paris), \emph{Schriftsteller}!Le Gendre de M. Poirier. Comédie en 4 actes en prose@\strich\emph{Le Gendre de M. Poirier. Comédie en 4 actes en prose}|pwv}}}{\lemma{\textnormal{\emph{Milieu … Gastons}}}\Cendnote{\textnormal{Goldmann\pwindex{Goldmann, Paul 31.\,1.\,1865 Breslau – 25.\,9.\,1935 Wien@\textsc{Goldmann, Paul} (31.\,1.\,1865 Breslau – 25.\,9.\,1935 Wien), \emph{Schriftsteller, Journalist}|pwk} dürfte sich auf die zwei verarmten
                  adeligen Lebemänner Hector de
                     Montmeyran\pwindex{Augier, Émile 17.\,9.\,1820 Valence – 25.\,10.\,1889 Croissy-sur-Seine@\textsc{Augier, Émile} (17.\,9.\,1820 Valence – 25.\,10.\,1889 Croissy-sur-Seine), \emph{Schriftsteller}!Le Gendre de M. Poirier. Comédie en 4 actes en prose@\strich\emph{Le Gendre de M. Poirier. Comédie en 4 actes en prose}|pwkv}\pwindex{Sandeau, Jules 19.\,2.\,1811 Aubusson – 24.\,4.\,1883 Paris@\textsc{Sandeau, Jules} (19.\,2.\,1811 Aubusson – 24.\,4.\,1883 Paris), \emph{Schriftsteller}!Le Gendre de M. Poirier. Comédie en 4 actes en prose@\strich\emph{Le Gendre de M. Poirier. Comédie en 4 actes en prose}|pwkv} und Gaston de
                     Presle\pwindex{Augier, Émile 17.\,9.\,1820 Valence – 25.\,10.\,1889 Croissy-sur-Seine@\textsc{Augier, Émile} (17.\,9.\,1820 Valence – 25.\,10.\,1889 Croissy-sur-Seine), \emph{Schriftsteller}!Le Gendre de M. Poirier. Comédie en 4 actes en prose@\strich\emph{Le Gendre de M. Poirier. Comédie en 4 actes en prose}|pwkv}\pwindex{Sandeau, Jules 19.\,2.\,1811 Aubusson – 24.\,4.\,1883 Paris@\textsc{Sandeau, Jules} (19.\,2.\,1811 Aubusson – 24.\,4.\,1883 Paris), \emph{Schriftsteller}!Le Gendre de M. Poirier. Comédie en 4 actes en prose@\strich\emph{Le Gendre de M. Poirier. Comédie en 4 actes en prose}|pwkv} aus der Komödie \emph{Le Gendre de M.
                     Poirier}\pwindex{Augier, Émile 17.\,9.\,1820 Valence – 25.\,10.\,1889 Croissy-sur-Seine@\textsc{Augier, Émile} (17.\,9.\,1820 Valence – 25.\,10.\,1889 Croissy-sur-Seine), \emph{Schriftsteller}!Le Gendre de M. Poirier. Comédie en 4 actes en prose@\strich\emph{Le Gendre de M. Poirier. Comédie en 4 actes en prose}|pwk}\pwindex{Sandeau, Jules 19.\,2.\,1811 Aubusson – 24.\,4.\,1883 Paris@\textsc{Sandeau, Jules} (19.\,2.\,1811 Aubusson – 24.\,4.\,1883 Paris), \emph{Schriftsteller}!Le Gendre de M. Poirier. Comédie en 4 actes en prose@\strich\emph{Le Gendre de M. Poirier. Comédie en 4 actes en prose}|pwk} (1854) von Émile Augier\pwindex{Augier, Émile 17.\,9.\,1820 Valence – 25.\,10.\,1889 Croissy-sur-Seine@\textsc{Augier, Émile} (17.\,9.\,1820 Valence – 25.\,10.\,1889 Croissy-sur-Seine), \emph{Schriftsteller}|pwk} und
                     Jules Sandeau\pwindex{Sandeau, Jules 19.\,2.\,1811 Aubusson – 24.\,4.\,1883 Paris@\textsc{Sandeau, Jules} (19.\,2.\,1811 Aubusson – 24.\,4.\,1883 Paris), \emph{Schriftsteller}|pwk} beziehen.}}}\label{K_L02669-1}, habe ich
               gefunden. Am meiſten verkehre ich mit \textsc{Chainaye\pwindex{Chainaye, Hector 14.\,4.\,1865 Lüttich – 4.\,9.\,1913 Ixelles@\textsc{Chainaye, Hector} (14.\,4.\,1865 Lüttich – 4.\,9.\,1913 Ixelles), \emph{Schriftsteller, Journalist}|pw}}, dem jüngſten Redacteur\pwindex{Chainaye, Hector 14.\,4.\,1865 Lüttich – 4.\,9.\,1913 Ixelles@\textsc{Chainaye, Hector} (14.\,4.\,1865 Lüttich – 4.\,9.\,1913 Ixelles), \emph{Schriftsteller, Journalist}|pwv} der \textsc{Indépendance Belge}\orgindex{Indépendance Belge@L’Indépendance Belge|pw}: enragirter Wallone\pwindex{Chainaye, Hector 14.\,4.\,1865 Lüttich – 4.\,9.\,1913 Ixelles@\textsc{Chainaye, Hector} (14.\,4.\,1865 Lüttich – 4.\,9.\,1913 Ixelles), \emph{Schriftsteller, Journalist}|pwv}
               und \label{K_L02669-2v}\edtext{Romane\pwindex{Chainaye, Hector 14.\,4.\,1865 Lüttich – 4.\,9.\,1913 Ixelles@\textsc{Chainaye, Hector} (14.\,4.\,1865 Lüttich – 4.\,9.\,1913 Ixelles), \emph{Schriftsteller, Journalist}|pwv}}{\lemma{\textnormal{\emph{Romane}}}\Cendnote{\textnormal{»Belgique romane« ist ein Überbegriff
                  für mehrere Dialekte. Der bedeutendste ist der wallonische.}}}\label{K_L02669-2}, reiches
               künſtleriſches Sentiment, Stimmungsmenſch\pwindex{Chainaye, Hector 14.\,4.\,1865 Lüttich – 4.\,9.\,1913 Ixelles@\textsc{Chainaye, Hector} (14.\,4.\,1865 Lüttich – 4.\,9.\,1913 Ixelles), \emph{Schriftsteller, Journalist}|pwv}, melancholiſches Talent, Verfaſſer\pwindex{Chainaye, Hector 14.\,4.\,1865 Lüttich – 4.\,9.\,1913 Ixelles@\textsc{Chainaye, Hector} (14.\,4.\,1865 Lüttich – 4.\,9.\,1913 Ixelles), \emph{Schriftsteller, Journalist}|pwv} myſtiſch-empfindſamer \label{K_L02669-3v}\edtext{Gedichte in Proſa}{\lemma{\textnormal{\emph{Gedichte in Prosa}}}\Cendnote{\textnormal{Prosagedichte Hector Chainayes\pwindex{Chainaye, Hector 14.\,4.\,1865 Lüttich – 4.\,9.\,1913 Ixelles@\textsc{Chainaye, Hector} (14.\,4.\,1865 Lüttich – 4.\,9.\,1913 Ixelles), \emph{Schriftsteller, Journalist}|pwk} finden sich
                  zum Beispiel in seinem Band \emph{L’Âme des choses}\pwindex{Chainaye, Hector 14.\,4.\,1865 Lüttich – 4.\,9.\,1913 Ixelles@\textsc{Chainaye, Hector} (14.\,4.\,1865 Lüttich – 4.\,9.\,1913 Ixelles), \emph{Schriftsteller, Journalist}!Âme des choses@\strich\emph{L’Âme des choses}|pwk}
                     (1935). Viele der darin enthaltenen Gedichte wurden bereits
                  zwischen 1886 und 1888 in Zeitschriften wie \emph{La Wallonie}\pwindex{Wallonie@\emph{La Wallonie}|pwk}, \emph{La Basoche}\pwindex{Basoche. Revue Littéraire et Artistique@\emph{La Basoche. Revue Littéraire et Artistique}|pwk} und \emph{La Jeune Belgique}\pwindex{Jeune Belgique@\emph{La Jeune Belgique}|pwk}
                  veröffentlicht.}}}\label{K_L02669-3}, blond, krank, \strikeout{ſ}
               geiſtſprühend und luſtig in der Converſation bei dem Allen und – was das beſte iſt –
               mit einigen \strikeout{k\textcolor{gray}{l}} Zügen, die entfernt an Dich erinnern. Nach Beſiegung des Deutſch\oindex{Deutschland@\textbf{Deutschland}|pwv}enhaſſes, der
               Verſtändigungsſchwierigkeiten, des Mißtrauens gegen den Fremden \textsc{etc. etc.} bin ich ihm näher getreten. Und in dieſe\substVorne{}\textsuperscript{m}\substDazwischen{}n\substHinten{}{ }{\pb}Tagen{ }ſtehe ich ihm rathend zur Seite bei einem
               großen Bruch mit{ }ſeiner \label{K_L02669-4v}\edtext{Maitreſſe\pwindex{?? [Partnerin von Hector Chainaye, 1891] @\textsc{?? [Partnerin von Hector Chainaye, 1891]}|pwv}}{\lemma{\textnormal{\emph{Maitresse}}}\Cendnote{\textnormal{nicht identifiziert}}}\label{K_L02669-4}, die{ }ſich zu
               tödten droht \textsc{etc. etc.} (ſiehe \label{K_L02669-5v}\edtext{\textsc{Jeannette\pwindex{Heeger, Jeanette 1.\,7.\,1865 Šternberk – 3.\,1.\,1903 Wien@\textsc{Heeger, Jeanette} (1.\,7.\,1865 Šternberk – 3.\,1.\,1903 Wien), \emph{Näherin}|pw}}}{\lemma{\textnormal{\emph{Jeannette}}}\Cendnote{\textnormal{Jeannette Heeger\pwindex{Heeger, Jeanette 1.\,7.\,1865 Šternberk – 3.\,1.\,1903 Wien@\textsc{Heeger, Jeanette} (1.\,7.\,1865 Šternberk – 3.\,1.\,1903 Wien), \emph{Näherin}|pwk}, Geliebte Schnitzlers, unternahm am 18. 12. 1889 einen
                  Suizidversuch mit einer Pistole.}}}\label{K_L02669-5}.) Ein närriſches Ding, das Leben, – nicht
               wahr? Außerdem haben{ }ſich meine Beziehungen zu den Brüſſel\oindex{Brüssel@\textbf{Brüssel}, \emph{Hauptstadt}|pw}er Journaliſten{ }ſichtlich verbeſſert. Es iſt ein geradezu enormer
               Unterſchied zwiſchen den Brüſſel\oindex{Brüssel@\textbf{Brüssel}, \emph{Hauptstadt}|pw}er und den Wien\oindex{Wien@\textbf{Wien}, \emph{Verwaltungsgebiet}|pw}er Collegen. Hier{ }ſind es – von wenigen
               Ausnahmen abgeſehen – liebe, gute Burſchen mit prächtigem Benehmen, voll Gefälligkeit
               und Liebenswürdigkeit, und manch’ eine{ }ſchöne Künftlernatur iſt auch hier darunter –
               Leute, die den Journalismus machen, um Brod zu verdienen, aber im Übrigen \label{K_L02669-6v}\edtext{\textsc{\begin{otherlanguage}{french}s’en fichent\end{otherlanguage}}}{\lemma{\textnormal{\emph{s’en fichent}}}\Cendnote{\textnormal{französisch: sich nicht kümmern}}}\label{K_L02669-6}
               und warmen Herzens der Kunſt anhängen. Ich mache hier eifrige Propaganda für die
                  \label{K_L02669-7v}\edtext{Norwege\oindex{Norwegen@\textbf{Norwegen}|pwv}r\pwindex{Ibsen, Henrik 20.\,3.\,1828 Skien – 23.\,5.\,1906 Oslo@\textsc{Ibsen, Henrik} (20.\,3.\,1828 Skien – 23.\,5.\,1906 Oslo), \emph{Schriftsteller}|pwuv}\pwindex{Hamsun, Knut 4.\,8.\,1859 Lom – 19.\,2.\,1952 Grimstad@\textsc{Hamsun, Knut} (4.\,8.\,1859 Lom – 19.\,2.\,1952 Grimstad), \emph{Schriftsteller, Nobelpreisträger}|pwuv}}{\lemma{\textnormal{\emph{Norweger}}}\Cendnote{\textnormal{Gemeint sein dürfte vor allem Henrik Ibsen\pwindex{Ibsen, Henrik 20.\,3.\,1828 Skien – 23.\,5.\,1906 Oslo@\textsc{Ibsen, Henrik} (20.\,3.\,1828 Skien – 23.\,5.\,1906 Oslo), \emph{Schriftsteller}|pwk}, eventuell auch Knut Hamsun\pwindex{Hamsun, Knut 4.\,8.\,1859 Lom – 19.\,2.\,1952 Grimstad@\textsc{Hamsun, Knut} (4.\,8.\,1859 Lom – 19.\,2.\,1952 Grimstad), \emph{Schriftsteller, Nobelpreisträger}|pwk}. In der im Folgenden erwähnten
                     Zeitungsmeldung\pwindex{Tardieu, Charles 9.\,2.\,1838 – 1909@\textsc{Tardieu, Charles} (9.\,2.\,1838 – 1909), \emph{Journalist, Chefredakteur}!Théâtres et beaux-arts@\strich\emph{Théâtres et beaux-arts}|pwkv} von Charles Tardieu\pwindex{Tardieu, Charles 9.\,2.\,1838 – 1909@\textsc{Tardieu, Charles} (9.\,2.\,1838 – 1909), \emph{Journalist, Chefredakteur}|pwk} wird allgemein von der Ibsen\pwindex{Ibsen, Henrik 20.\,3.\,1828 Skien – 23.\,5.\,1906 Oslo@\textsc{Ibsen, Henrik} (20.\,3.\,1828 Skien – 23.\,5.\,1906 Oslo), \emph{Schriftsteller}|pwk}-Schule gesprochen und vor allem der
                     Schwede\oindex{Schweden@\textbf{Schweden}|pwkv}{ }August Strindberg\pwindex{Strindberg, August 22.\,1.\,1849 Stockholm – 14.\,5.\,1912 ebd.@\textsc{Strindberg, August} (22.\,1.\,1849 Stockholm – 14.\,5.\,1912 ebd.), \emph{Schriftsteller}|pwk} behandelt.}}}\label{K_L02669-7}, und
                  \textsc{Tardieu\pwindex{Tardieu, Charles 9.\,2.\,1838 – 1909@\textsc{Tardieu, Charles} (9.\,2.\,1838 – 1909), \emph{Journalist, Chefredakteur}|pw}}, der Chefredacteur\pwindex{Tardieu, Charles 9.\,2.\,1838 – 1909@\textsc{Tardieu, Charles} (9.\,2.\,1838 – 1909), \emph{Journalist, Chefredakteur}|pwv} der
                  \textsc{Indépendance\orgindex{Indépendance Belge@L’Indépendance Belge|pw}}, der unter den intereſſanten {\pb}hieſigen \strikeout{\textcolor{gray}{S}} Collegen vielleicht der intereſſanteſte iſt, hat dieſe meine Bemühungen{ }ſammt
               Citat meines Namens in der \textsc{Indép.\pwindex{Indépendance Belge@\emph{L’Indépendance Belge}|pw}}{ }\label{K_L02669-8v}\edtext{verewigt}{\lemma{\textnormal{\emph{verewigt}}}\Cendnote{\textnormal{Charles Tardieu\pwindex{Tardieu, Charles 9.\,2.\,1838 – 1909@\textsc{Tardieu, Charles} (9.\,2.\,1838 – 1909), \emph{Journalist, Chefredakteur}|pwk}: \emph{Théâtres et beaux-arts}\pwindex{Tardieu, Charles 9.\,2.\,1838 – 1909@\textsc{Tardieu, Charles} (9.\,2.\,1838 – 1909), \emph{Journalist, Chefredakteur}!Théâtres et beaux-arts@\strich\emph{Théâtres et beaux-arts}|pwk}. In: \emph{L’Indépendance Belge}\pwindex{Tardieu, Charles 9.\,2.\,1838 – 1909@\textsc{Tardieu, Charles} (9.\,2.\,1838 – 1909), \emph{Journalist, Chefredakteur}!Théâtres et beaux-arts@\strich\emph{Théâtres et beaux-arts}|pwk}, Jg. 62, H. 281,
                        8. 10. 1891, Abendausgabe, S. 3: »\begin{otherlanguage}{french}Voilà qui nous mène en Scandinavie\oindex{Skandinavien@\textbf{Skandinavien}|pw} et de là à Berlin\oindex{Berlin@\textbf{Berlin}, \emph{Hauptstadt}|pw}
                        et Munich\oindex{München@\textbf{München}|pw}, où l’école ibsén\pwindex{Ibsen, Henrik 20.\,3.\,1828 Skien – 23.\,5.\,1906 Oslo@\textsc{Ibsen, Henrik} (20.\,3.\,1828 Skien – 23.\,5.\,1906 Oslo), \emph{Schriftsteller}|pw}ienne a un public enthousiaste.
                        Mais que parlons-nous encore d’Ibsen\pwindex{Ibsen, Henrik 20.\,3.\,1828 Skien – 23.\,5.\,1906 Oslo@\textsc{Ibsen, Henrik} (20.\,3.\,1828 Skien – 23.\,5.\,1906 Oslo), \emph{Schriftsteller}|pw}?
                           L’auteur\pwindex{Ibsen, Henrik 20.\,3.\,1828 Skien – 23.\,5.\,1906 Oslo@\textsc{Ibsen, Henrik} (20.\,3.\,1828 Skien – 23.\,5.\,1906 Oslo), \emph{Schriftsteller}|pwv} du \emph{Canard sauvage}\pwindex{Ibsen, Henrik 20.\,3.\,1828 Skien – 23.\,5.\,1906 Oslo@\textsc{Ibsen, Henrik} (20.\,3.\,1828 Skien – 23.\,5.\,1906 Oslo), \emph{Schriftsteller}!Wildente. Schauspiel in fünf Akten@\strich\emph{Die Wildente. Schauspiel in fünf Akten}|pw} est absolument distancé dans son pays\oindex{Norwegen@\textbf{Norwegen}|pwv}. Novateur\pwindex{Ibsen, Henrik 20.\,3.\,1828 Skien – 23.\,5.\,1906 Oslo@\textsc{Ibsen, Henrik} (20.\,3.\,1828 Skien – 23.\,5.\,1906 Oslo), \emph{Schriftsteller}|pwv} et réformateur\pwindex{Ibsen, Henrik 20.\,3.\,1828 Skien – 23.\,5.\,1906 Oslo@\textsc{Ibsen, Henrik} (20.\,3.\,1828 Skien – 23.\,5.\,1906 Oslo), \emph{Schriftsteller}|pwv} en Allemagne\oindex{Deutschland@\textbf{Deutschland}|pw} et en France\oindex{Frankreich@\textbf{Frankreich}|pw}, il est
                        déjà ›vieux jeu‹ dans sa Norvège\oindex{Norwegen@\textbf{Norwegen}|pw}.
                        Notre confrère\pwindex{Goldmann, Paul 31.\,1.\,1865 Breslau – 25.\,9.\,1935 Wien@\textsc{Goldmann, Paul} (31.\,1.\,1865 Breslau – 25.\,9.\,1935 Wien), \emph{Schriftsteller, Journalist}|pwv} de
                        la \emph{Gazette de Francfort}\orgindex{Frankfurter Zeitung@Frankfurter Zeitung|pw}, le docteur Goldmann\pwindex{Goldmann, Paul 31.\,1.\,1865 Breslau – 25.\,9.\,1935 Wien@\textsc{Goldmann, Paul} (31.\,1.\,1865 Breslau – 25.\,9.\,1935 Wien), \emph{Schriftsteller, Journalist}|pw}, très au
                        courant des curiosités et nouveautés littéraires, nous expliquait cela
                        dernièrement, et il nous prédisait le prochain avènement d’Auguste Strindberg\pwindex{Strindberg, August 22.\,1.\,1849 Stockholm – 14.\,5.\,1912 ebd.@\textsc{Strindberg, August} (22.\,1.\,1849 Stockholm – 14.\,5.\,1912 ebd.), \emph{Schriftsteller}|pw}, un dramaturge\pwindex{Strindberg, August 22.\,1.\,1849 Stockholm – 14.\,5.\,1912 ebd.@\textsc{Strindberg, August} (22.\,1.\,1849 Stockholm – 14.\,5.\,1912 ebd.), \emph{Schriftsteller}|pwv}{ }suéd\oindex{Schweden@\textbf{Schweden}|pwv}ois et niet{[}z{]}sch\pwindex{Nietzsche, Friedrich 15.\,10.\,1844 Röcken – 25.\,8.\,1900 Weimar@\textsc{Nietzsche, Friedrich} (15.\,10.\,1844 Röcken – 25.\,8.\,1900 Weimar), \emph{Schriftsteller, Philosoph}|pw}ien. Suéd\oindex{Schweden@\textbf{Schweden}|pw}ois? vous comprenez. Mais pour ›niet{[}z{]}sch\pwindex{Nietzsche, Friedrich 15.\,10.\,1844 Röcken – 25.\,8.\,1900 Weimar@\textsc{Nietzsche, Friedrich} (15.\,10.\,1844 Röcken – 25.\,8.\,1900 Weimar), \emph{Schriftsteller, Philosoph}|pw}ien‹
                        sachez que Frédéric
                              Niet{[}z{]}sche\pwindex{Nietzsche, Friedrich 15.\,10.\,1844 Röcken – 25.\,8.\,1900 Weimar@\textsc{Nietzsche, Friedrich} (15.\,10.\,1844 Röcken – 25.\,8.\,1900 Weimar), \emph{Schriftsteller, Philosoph}|pw} est, comme eût dit Stendhal\pwindex{Stendhal 23.\,1.\,1783 Grenoble – 23.\,3.\,1842 Paris@\textsc{Stendhal} (23.\,1.\,1783 Grenoble – 23.\,3.\,1842 Paris), \emph{Schriftsteller}|pw}, ›l’expression la plus
                        récente‹ de la philosophie allema\oindex{Deutschland@\textbf{Deutschland}|pwv}nde. Or, voici que la prédiction se vérifie. Le Théâtre Libre\orgindex{Freie Bühne@Freie Bühne|pw}\orgindex{Akademisch-dramatischer Verein München@Akademisch-dramatischer Verein München|pw} de Berlin\oindex{Berlin@\textbf{Berlin}, \emph{Hauptstadt}|pw} et celui de Munich\oindex{München@\textbf{München}|pw} monteront cet hiver \emph{Mademoiselle Julie}\pwindex{Strindberg, August 22.\,1.\,1849 Stockholm – 14.\,5.\,1912 ebd.@\textsc{Strindberg, August} (22.\,1.\,1849 Stockholm – 14.\,5.\,1912 ebd.), \emph{Schriftsteller}!Fröken Julie@\strich\emph{Fröken Julie}|pw}, de M. Auguste Strindberg\pwindex{Strindberg, August 22.\,1.\,1849 Stockholm – 14.\,5.\,1912 ebd.@\textsc{Strindberg, August} (22.\,1.\,1849 Stockholm – 14.\,5.\,1912 ebd.), \emph{Schriftsteller}|pw}, une
                           tragédie\pwindex{Strindberg, August 22.\,1.\,1849 Stockholm – 14.\,5.\,1912 ebd.@\textsc{Strindberg, August} (22.\,1.\,1849 Stockholm – 14.\,5.\,1912 ebd.), \emph{Schriftsteller}!Fröken Julie@\strich\emph{Fröken Julie}|pwv}
                        naturaliste à trois personnages, en un acte et une nuit. En deux mots Mlle Julie\pwindex{Strindberg, August 22.\,1.\,1849 Stockholm – 14.\,5.\,1912 ebd.@\textsc{Strindberg, August} (22.\,1.\,1849 Stockholm – 14.\,5.\,1912 ebd.), \emph{Schriftsteller}!Fröken Julie@\strich\emph{Fröken Julie}|pwv}, hystérique
                        par atavisme, est amoureuse du domestique de son père. Elle fait
                        littéralement le siège du valet qui lutte et-succombe. Tous deux se
                        préparent à s’enfuir. Mais la cuisinière raisonne les deux amants, les
                        rappelle au sentiment des convenances sociales, et, ma foi, réussit à les
                        calmer. La toile tombe sur une rupture, definitive, espérons-le. Il est
                        probable que l’analyse des caractères ajoute à l’intérêt de cette donnée,
                        déjà séduisante par elle même. De quoi s’agit-il après tout? D’un accident.
                        A quoi bon se troubler et déranger sa vie pour si peu de chose? Christine
                        est dans le vrai. On voit bien qu’elle sait l’art d’accommoder les
                        restes.\end{otherlanguage}«}}}\label{K_L02669-8}, worauf dann die Notiz\pwindex{Tardieu, Charles 9.\,2.\,1838 – 1909@\textsc{Tardieu, Charles} (9.\,2.\,1838 – 1909), \emph{Journalist, Chefredakteur}!Théâtres et beaux-arts@\strich\emph{Théâtres et beaux-arts}|pwv} mit »\label{K_L02669-9v}\edtext{\textsc{\begin{otherlanguage}{french}notre confrère le docteur Goldmann de le Gazette de Francfort\orgindex{Frankfurter Zeitung@Frankfurter Zeitung|pw}\end{otherlanguage}}\pwindex{Tardieu, Charles 9.\,2.\,1838 – 1909@\textsc{Tardieu, Charles} (9.\,2.\,1838 – 1909), \emph{Journalist, Chefredakteur}!Théâtres et beaux-arts@\strich\emph{Théâtres et beaux-arts}|pwv}}{\lemma{\textnormal{\emph{notre … Francfort}}}\Cendnote{\textnormal{französisch: unser Kollege\pwindex{Goldmann, Paul 31.\,1.\,1865 Breslau – 25.\,9.\,1935 Wien@\textsc{Goldmann, Paul} (31.\,1.\,1865 Breslau – 25.\,9.\,1935 Wien), \emph{Schriftsteller, Journalist}|pwkv} Dr. Goldmann\pwindex{Goldmann, Paul 31.\,1.\,1865 Breslau – 25.\,9.\,1935 Wien@\textsc{Goldmann, Paul} (31.\,1.\,1865 Breslau – 25.\,9.\,1935 Wien), \emph{Schriftsteller, Journalist}|pwk} von der \emph{Frankfurter Zeitung}\orgindex{Frankfurter Zeitung@Frankfurter Zeitung|pwk}}}}\label{K_L02669-9}« die Runde durch die Pariſ\oindex{Paris@\textbf{Paris}, \emph{Hauptstadt}|pw}er Preſſe, vom
                  \label{K_L02669-10v}\edtext{\textsc{Figaro\pwindex{Le Figaro@\emph{Le Figaro}|pw}}}{\lemma{\textnormal{\emph{Figaro}}}\Cendnote{\textnormal{Georges Boyer\pwindex{Boyer, Georges 21.\,7.\,1850 Paris – 25.\,4.\,1931 ebd.@\textsc{Boyer, Georges} (21.\,7.\,1850 Paris – 25.\,4.\,1931 ebd.), \emph{Dramaturg, Theaterkritiker}|pwk}: \emph{Courrier des Théâtres}\pwindex{Boyer, Georges 21.\,7.\,1850 Paris – 25.\,4.\,1931 ebd.@\textsc{Boyer, Georges} (21.\,7.\,1850 Paris – 25.\,4.\,1931 ebd.), \emph{Dramaturg, Theaterkritiker}!Courrier des Théâtres@\strich\emph{Courrier des Théâtres}|pwk}. In: \emph{Le Figaro}\pwindex{Le Figaro@\emph{Le Figaro}|pwk}, Jg. 37, H. 286, 13. 10. 1891,
                     S. 3.}}}\label{K_L02669-10} bis zum \label{K_L02669-11v}\edtext{\textsc{Rappel\pwindex{Le Rappel@\emph{Le Rappel}|pw}}}{\lemma{\textnormal{\emph{Rappel}}}\Cendnote{\textnormal{nicht nachgewiesen}}}\label{K_L02669-11}, gemacht
               hat. Auch d\substVorne{}\textsuperscript{\textcolor{gray}{ie}}\substDazwischen{}er\substHinten{} Verkehr \substVorne{}\textsuperscript{zu\textcolor{gray}{r}}\substDazwischen{}mit der\substHinten{} officiellen Welt iſt angenehm. Ich werde von mehreren Miniſtern mit allen
               meinem Range gebührenden Ehren empfangen \textsc{etc.} Außerdem iſt
               die Stadt\oindex{Brüssel@\textbf{Brüssel}, \emph{Hauptstadt}|pwv} mit ihrem \substVorne{}\textsuperscript{Schein}\substDazwischen{}Abglanz\substHinten{}{ }franz\oindex{Frankreich@\textbf{Frankreich}|pwv}öſiſchen Kunſtlebens recht
               intereſſant, und es gibt{ }ſchöne Abende im Theater und im Concert. Endlich das
               herrliche Hiſtoriſche. Die alte niederländ\oindex{Niederlande@\textbf{Niederlande}|pwv}iſche Malerei. Ich beginne hier langſam zu begreifen, was das für
               Dinger{ }ſind, die \textsc{Rubens\pwindex{Rubens, Peter Paul 28.\,6.\,1577 Siegen – 30.\,5.\,1640 Antwerpen@\textsc{Rubens, Peter Paul} (28.\,6.\,1577 Siegen – 30.\,5.\,1640 Antwerpen), \emph{Maler}|pw}}, \textsc{van Dyck\pwindex{Dyck, Anthonis van 22.\,3.\,1599 Antwerpen – 9.\,12.\,1641 London@\textsc{Dyck, Anthonis van} (22.\,3.\,1599 Antwerpen – 9.\,12.\,1641 London), \emph{Maler, Grafiker, Radierer}|pw}} und \textsc{Rembrandt\pwindex{Rembrandt van Rijn 15.\,7.\,1606 Leiden – 4.\,10.\,1669 Amsterdam@\textsc{Rembrandt van Rijn} (15.\,7.\,1606 Leiden – 4.\,10.\,1669 Amsterdam), \emph{Maler}|pw}}. Und das iſt ein Quell neuer und {\pb}ungeahnter
               Genüſſe.\pend
           
\pstart
           Das{ }ſind die guten Seiten. Aber die böſen{ }ſind geblieben,{ }ſind vielleicht noch
               troſtloſer als zuvor, und haben nur die Geſichter zum Theil gewechſelt. Keine
               Zukunft, keine Zukunft. Die Möglichkeit,{ }ſich ein Vermögen zu machen, exiſtirt nicht.
               Mein Gehalt iſt jämmerlich und wird nicht geſteigert. Die großen \label{K_L02669-12v}\edtext{Pflichten, die ich gegen die Meinen\pwindex{Rosengart, Vally 29.\,12.\,1866 Breslau – nach 1926@\textsc{Rosengart, Vally} (29.\,12.\,1866 Breslau – nach 1926)|pwv}\pwindex{Goldmann, Clementine 15.\,5.\,1842 Breslau – 24.\,2.\,1924 Frankfurt am Main@\textsc{Goldmann, Clementine} (15.\,5.\,1842 Breslau – 24.\,2.\,1924 Frankfurt am Main)|pwv}}{\lemma{\textnormal{\emph{Pflichten, … Meinen}}}\Cendnote{\textnormal{Siehe XXXX Auszeichnungsfehler: Dokument L02661 nicht gefunden.
               }}}\label{K_L02669-12} habe, treten immer drohender an mich heran. Und außerdem werde ich von Seiten
               des Blatt\orgindex{Frankfurter Zeitung@Frankfurter Zeitung|pwv}es genau{ }ſo gemein und
               ungerecht behandelt, wie es mir in Wien\oindex{Wien@\textbf{Wien}, \emph{Verwaltungsgebiet}|pw} geſchehen
               – H. \textsc{Sonnemann\pwindex{Sonnemann, Leopold 29.\,10.\,1831 Höchberg – 30.\,10.\,1909 Frankfurt am Main@\textsc{Sonnemann, Leopold} (29.\,10.\,1831 Höchberg – 30.\,10.\,1909 Frankfurt am Main), \emph{Journalist, Herausgeber}|pw}}, der Chef\pwindex{Sonnemann, Leopold 29.\,10.\,1831 Höchberg – 30.\,10.\,1909 Frankfurt am Main@\textsc{Sonnemann, Leopold} (29.\,10.\,1831 Höchberg – 30.\,10.\,1909 Frankfurt am Main), \emph{Journalist, Herausgeber}|pwv} und Gebieter\pwindex{Sonnemann, Leopold 29.\,10.\,1831 Höchberg – 30.\,10.\,1909 Frankfurt am Main@\textsc{Sonnemann, Leopold} (29.\,10.\,1831 Höchberg – 30.\,10.\,1909 Frankfurt am Main), \emph{Journalist, Herausgeber}|pwv}, iſt ein \strikeout{erbarmu} erbarmungsloſer Blutſauger\pwindex{Sonnemann, Leopold 29.\,10.\,1831 Höchberg – 30.\,10.\,1909 Frankfurt am Main@\textsc{Sonnemann, Leopold} (29.\,10.\,1831 Höchberg – 30.\,10.\,1909 Frankfurt am Main), \emph{Journalist, Herausgeber}|pwv}, der verlangt, daß{ }ſich{ }ſeine
               Leute zu Tode{ }ſchinden und der ihnen auch {\pb}dann noch
               beim kleinſten Verſehen heftige Vorwürfe macht. Außerdem{ }ſitzt eine \label{K_L02669-13v}\edtext{Canaille\pwindex{?? [Vorgesetzter Paul Goldmanns 1891] @\textsc{?? [Vorgesetzter Paul Goldmanns 1891]}|pwv}}{\lemma{\textnormal{\emph{Canaille}}}\Cendnote{\textnormal{Schurke, Bösewicht}}}\label{K_L02669-13} in der Redaction\orgindex{Frankfurter Zeitung@Frankfurter Zeitung|pwv}, ein Menſch\pwindex{?? [Vorgesetzter Paul Goldmanns 1891] @\textsc{?? [Vorgesetzter Paul Goldmanns 1891]}|pwv}, der mich kaum kennt, dem ich nie
               etwas gethan habe und der mich trotzdem haßt, Gott weiß warum. Er iſt zum Unglück
               mein \label{K_L02669-14v}\edtext{unmittelbarer Vorgeſetzter\pwindex{?? [Vorgesetzter Paul Goldmanns 1891] @\textsc{?? [Vorgesetzter Paul Goldmanns 1891]}|pwv}}{\lemma{\textnormal{\emph{unmittelbarer Vorgesetzter}}}\Cendnote{\textnormal{nicht identifiziert}}}\label{K_L02669-14}, und ihm
               habe ich es zu danken, daß \strikeout{\textcolor{gray}{man}} meine Ernennung für den Pariſ\oindex{Paris@\textbf{Paris}, \emph{Hauptstadt}|pw}er Poſten,
               welche im Zuge war, unterblieb, weil ich mit der \label{K_L02669-15v}\edtext{Nachricht vom Tode \textsc{Boulangers\pwindex{Boulanger, Georges 29.\,4.\,1837 Rennes – 30.\,9.\,1891 Ixelles@\textsc{Boulanger, Georges} (29.\,4.\,1837 Rennes – 30.\,9.\,1891 Ixelles), \emph{Politiker, Militär}|pw}}}{\lemma{\textnormal{\emph{Nachricht … Boulangers}}}\Cendnote{\textnormal{Georges Boulanger\pwindex{Boulanger, Georges 29.\,4.\,1837 Rennes – 30.\,9.\,1891 Ixelles@\textsc{Boulanger, Georges} (29.\,4.\,1837 Rennes – 30.\,9.\,1891 Ixelles), \emph{Politiker, Militär}|pwk} hatte am
                     30. 9. 1891 in Ixelles\oindex{Ixelles@\textbf{Ixelles}, \emph{Region}|pwk}
                  Suizid begangen.}}}\label{K_L02669-15} eine Stunde{ }ſpäter gekommen, als die officielle
               Telegraphenagentur \textcolor{gray}{–} die \textsc{Agence Havas\orgindex{Agence Havas@Agence Havas|pw}}! Und ähnliche Schurkereien. Ich leide entſetzlich darunter und{ }ſehne mich
               blutenden Herzens mehr als je nach Erlöſung. Ein kleines Capital und Rückkehr nach
                  Wien\oindex{Wien@\textbf{Wien}, \emph{Verwaltungsgebiet}|pw}. Denn das iſt nach wie vor das oberſte
               Ziel meiner Wünſche. Es vergeht nach wie vor kein Tag, {\pb}wo ich nicht zehn-, zwanzigmal an Dich und die
               theure Stadt\oindex{Wien@\textbf{Wien}, \emph{Verwaltungsgebiet}|pwv} denke. Und als das
                  Orcheſter der \textsc{Pompiers}\orgindex{Orchestre municipal des pompiers de Bruxelles@Orchestre municipal des pompiers de Bruxelles|pw}{ }Sonntag die Straßen mit dem Schrammel-Marſch\pwindex{\textcolor{red}{\textsuperscript{XXXX indx1}}!Wien bleibt Wien@\strich\emph{Wien bleibt Wien}|pwv} durchzog, lief ich
               hinterher und wiſchte mir, wie der bekannte Vater im Singſpiel\pwindex{?? [Singspiel, in dem sich ein Vater Tränen der Rührung aus den Augen wischt]@\emph{?? [Singspiel, in dem sich ein Vater Tränen der Rührung aus den Augen wischt]}|pwv}, die Thränen mit dem Rockärmel
               ab. Aber ich habe keine Hoffnung. Mein Leben wird in harter Sklaverei verfließen,
               fern von Allem, was ich lieb habe; und zu großen befreienden Werken habe ich weder
               das genügende Talent, noch die genügende Energie{\dotsfive}\pend
           
\pstart
           Wollte ich nun alle die Fragen aufſchreiben, die ich an Dich zu richten habe, es
               ginge noch ein Briefbogen darauf. Aber ich thue es nicht; denn ich weiß, daß du mir{ }ſie eh’ nicht beantworten wirſt. Der lange Brief\strikeout{,} von
               Dir, der nicht kommt,{ }ſagt mir viel mehr, als \strikeout{ein}
               einer, der gekom{\pb}men wäre. Du haſt Recht, mein
               lieber Alter; es gibt auch in der Freundſchaft »\label{K_L02669-16v}\edtext{Epiſoden\pwindex{Schnitzler, Arthur 15.\,5.\,1862 Wien – 21.\,10.\,1931 ebd.@\textsc{Schnitzler, Arthur} (15.\,5.\,1862 Wien – 21.\,10.\,1931 ebd.), \emph{Schriftsteller, Mediziner}!Episode@\strich\emph{Episode}|pwv}}{\lemma{\textnormal{\emph{Episoden}}}\Cendnote{\textnormal{Anspielung auf Schnitzlers Einakter \emph{Episode}\pwindex{Schnitzler, Arthur 15.\,5.\,1862 Wien – 21.\,10.\,1931 ebd.@\textsc{Schnitzler, Arthur} (15.\,5.\,1862 Wien – 21.\,10.\,1931 ebd.), \emph{Schriftsteller, Mediziner}!Episode@\strich\emph{Episode}|pwk}}}}\label{K_L02669-16}«. Jeder verbraucht halt in{ }ſeinem Leben eine gewiſſe Anzahl Menſchen, und von
               mir iſt nur mehr der letzte Bodenſatz vorhanden. Dir iſt kein Vorwurf zu machen. Es
               iſt die Natur, die es{ }ſo eingerichtet, daß das Vergeſſen in der{ }ſeeliſchen Welt genau{ }ſo \strikeout{meh} mechaniſch und nothwendig und mit denſelben
               Endzwecken vor{ }ſich geht, wie das Verdauen in der körperlichen{\dotsfour}\pend
           
\pstart
           Mir brennt das Gewiſſen oft, wenn ich daran denke, daß ich \textsc{Loris\pwindex{Hofmannsthal, Hugo von 1.\,2.\,1874 Wien – 15.\,7.\,1929 Rodaun@\textsc{Hofmannsthal, Hugo von} (1.\,2.\,1874 Wien – 15.\,7.\,1929 Rodaun), \emph{Schriftsteller}|pw}} und \textsc{Richard\pwindex{Beer-Hofmann, Richard 11.\,7.\,1866 Wien – 26.\,9.\,1945 New York City@\textsc{Beer-Hofmann, Richard} (11.\,7.\,1866 Wien – 26.\,9.\,1945 New York City), \emph{Schriftsteller}|pw}} noch nicht auf ihre Brieſe geantwortet habe. Aber mir lähmt der Gedanke die zum
               Schreiben angeſetzte Hand, daß{ }ſie, wenn{ }ſie meinen Brief erhalten, die Empfindung
               haben könnten\substVorne{}\textsuperscript{\textcolor{gray}{,}}\substDazwischen{}:\substHinten{} was will der Menſch eigentlich von mir? Grüße die Zwei\pwindex{Hofmannsthal, Hugo von 1.\,2.\,1874 Wien – 15.\,7.\,1929 Rodaun@\textsc{Hofmannsthal, Hugo von} (1.\,2.\,1874 Wien – 15.\,7.\,1929 Rodaun), \emph{Schriftsteller}|pwv}\pwindex{Beer-Hofmann, Richard 11.\,7.\,1866 Wien – 26.\,9.\,1945 New York City@\textsc{Beer-Hofmann, Richard} (11.\,7.\,1866 Wien – 26.\,9.\,1945 New York City), \emph{Schriftsteller}|pwv} bitte viel {\pb}tauſend Mal von mir und{ }ſage ihnen in meinem Namen
               alles Liebe und Gute, was{ }ſich finden läßt{\dots}\pend
           
\pstart
           Deinem Bruder\pwindex{Schnitzler, Julius 13.\,7.\,1865 Wien – 29.\,6.\,1939 ebd.@\textsc{Schnitzler, Julius} (13.\,7.\,1865 Wien – 29.\,6.\,1939 ebd.), \emph{Chirurg}|pwv} und \textsc{Kapper\pwindex{Kapper, Friedrich 21.\,4.\,1861 Wien – 22.\,7.\,1939 ebd.@\textsc{Kapper, Friedrich} (21.\,4.\,1861 Wien – 22.\,7.\,1939 ebd.), \emph{Mediziner}|pw}} herzlichſte Grüße. Den Deinen ergebene Empfehlungen. Dir{ }ſelbſt –{ }ſchweres
               Problem. Ich möchte Dir am Liebſten meinen Segen geben,{ }ſo abgeſchieden komme ich mir
               Dir gegenüber vor.\pend
           
\pstart
           Dein {\\[\baselineskip]}treuer {\\[\baselineskip]}\spacefill\mbox{Paul Goldmann.}\pend
           \leftskip=0em{}
\pstart
           \noindent{}Drei Bitten 1.){ }ſag’ doch dem Schuft\pwindex{Joachim, Jaques 24.\,11.\,1866 Wien – 7.\,11.\,1925 ebd.@\textsc{Joachim, Jaques} (24.\,11.\,1866 Wien – 7.\,11.\,1925 ebd.), \emph{Rechtswissenschaftler, Rechtsanwalt, Herausgeber}|pwv}, dem \textsc{Dr. Joachim\pwindex{Joachim, Jaques 24.\,11.\,1866 Wien – 7.\,11.\,1925 ebd.@\textsc{Joachim, Jaques} (24.\,11.\,1866 Wien – 7.\,11.\,1925 ebd.), \emph{Rechtswissenschaftler, Rechtsanwalt, Herausgeber}|pw}}, wenn er die ihm geſchickte kleine Arbeit\pwindex{Schnitzler, Arthur 15.\,5.\,1862 Wien – 21.\,10.\,1931 ebd.@\textsc{Schnitzler, Arthur} (15.\,5.\,1862 Wien – 21.\,10.\,1931 ebd.), \emph{Schriftsteller, Mediziner}!drei Elixire@\strich\emph{Die drei Elixire}|pwuv} nicht brauchen kann,{ }ſo{ }ſoll er mir{ }ſie augenblicklich zurückſenden, weil ich Verwendung {\pb}dafür habe; auch{ }ſoll er mir dasjenige \label{K_L02669-17v}\edtext{Heft\pwindex{Moderne Dichtung. Monatsschrift für Literatur und Kritik@\emph{Moderne Dichtung. Monatsschrift für Literatur und Kritik}|pwv} der »Modernen \uline{Dichtung}\pwindex{Moderne Dichtung. Monatsschrift für Literatur und Kritik@\emph{Moderne Dichtung. Monatsschrift für Literatur und Kritik}|pw}}{\lemma{\textnormal{\emph{Heft … Dichtung}}}\Cendnote{\textnormal{Paul Goldmann\pwindex{Goldmann, Paul 31.\,1.\,1865 Breslau – 25.\,9.\,1935 Wien@\textsc{Goldmann, Paul} (31.\,1.\,1865 Breslau – 25.\,9.\,1935 Wien), \emph{Schriftsteller, Journalist}|pwk}: \emph{Was einem so einfällt}\pwindex{Goldmann, Paul 31.\,1.\,1865 Breslau – 25.\,9.\,1935 Wien@\textsc{Goldmann, Paul} (31.\,1.\,1865 Breslau – 25.\,9.\,1935 Wien), \emph{Schriftsteller, Journalist}!Was einem so einfällt@\strich\emph{Was einem so einfällt}|pwk}. In: \emph{Moderne Dichtung}\pwindex{Moderne Dichtung. Monatsschrift für Literatur und Kritik@\emph{Moderne Dichtung. Monatsschrift für Literatur und Kritik}|pwk}, Jg. 1, Bd. 2, H. 1,
                        S. 521–522.}}}\label{K_L02669-17}« (\label{K_L02669-18v}\edtext{nicht Rundſchau\pwindex{Moderne Rundschau@\emph{Moderne Rundschau}|pw}}{\lemma{\textnormal{\emph{nicht Rundschau}}}\Cendnote{\textnormal{Paul Goldmann\pwindex{Goldmann, Paul 31.\,1.\,1865 Breslau – 25.\,9.\,1935 Wien@\textsc{Goldmann, Paul} (31.\,1.\,1865 Breslau – 25.\,9.\,1935 Wien), \emph{Schriftsteller, Journalist}|pwk}: \emph{Nämlich}\pwindex{Goldmann, Paul 31.\,1.\,1865 Breslau – 25.\,9.\,1935 Wien@\textsc{Goldmann, Paul} (31.\,1.\,1865 Breslau – 25.\,9.\,1935 Wien), \emph{Schriftsteller, Journalist}!Nämlich@\strich\emph{Nämlich}|pwk}. In: \emph{Moderne
                           Rundschau}\pwindex{Moderne Rundschau@\emph{Moderne Rundschau}|pwk}, Jg. 1, Bd. 3, H. 1, 1. 4. 1891,
                     S. 34.}}}\label{K_L02669-18}){ }ſchicken, in dem Aphorismen\pwindex{Goldmann, Paul 31.\,1.\,1865 Breslau – 25.\,9.\,1935 Wien@\textsc{Goldmann, Paul} (31.\,1.\,1865 Breslau – 25.\,9.\,1935 Wien), \emph{Schriftsteller, Journalist}!Nämlich@\strich\emph{Nämlich}|pwv} von mir erſchienen{ }ſind; ich brauche{ }ſie
                  dringend und zahle \strikeout{e\textcolor{gray}{n}} eventuell dem Buchhändler dafür 2.) haſt Du eine Ahnung, was zwiſchen \textsc{\strikeout{Herz}}{ }\textsc{Herzl\pwindex{Herzl, Theodor 2.\,5.\,1860 Budapest – 3.\,7.\,1904 Edlach@\textsc{Herzl, Theodor} (2.\,5.\,1860 Budapest – 3.\,7.\,1904 Edlach), \emph{Schriftsteller, Journalist}|pw}} und{ }ſeiner Frau\pwindex{Herzl, Julie 1.\,2.\,1868 Budapest – 10.\,11.\,1907 Wien@\textsc{Herzl, Julie} (1.\,2.\,1868 Budapest – 10.\,11.\,1907 Wien)|pwv}{ }\label{K_L02669-19v}\edtext{vorgegangen}{\lemma{\textnormal{\emph{vorgegangen}}}\Cendnote{\textnormal{Möglicherweise hatte Goldmann\pwindex{Goldmann, Paul 31.\,1.\,1865 Breslau – 25.\,9.\,1935 Wien@\textsc{Goldmann, Paul} (31.\,1.\,1865 Breslau – 25.\,9.\,1935 Wien), \emph{Schriftsteller, Journalist}|pwk} von der Ehekrise der Herzls\pwindex{Herzl, Theodor 2.\,5.\,1860 Budapest – 3.\,7.\,1904 Edlach@\textsc{Herzl, Theodor} (2.\,5.\,1860 Budapest – 3.\,7.\,1904 Edlach), \emph{Schriftsteller, Journalist}|pwk}\pwindex{Herzl, Julie 1.\,2.\,1868 Budapest – 10.\,11.\,1907 Wien@\textsc{Herzl, Julie} (1.\,2.\,1868 Budapest – 10.\,11.\,1907 Wien)|pwk} gehört. Theodor Herzl\pwindex{Herzl, Theodor 2.\,5.\,1860 Budapest – 3.\,7.\,1904 Edlach@\textsc{Herzl, Theodor} (2.\,5.\,1860 Budapest – 3.\,7.\,1904 Edlach), \emph{Schriftsteller, Journalist}|pwk} teilte seinem Schwiegervater\pwindex{Naschauer, Jacob 20.\,5.\,1837 Nagykanizsa – 1.\,1.\,1894 Budapest@\textsc{Naschauer, Jacob} (20.\,5.\,1837 Nagykanizsa – 1.\,1.\,1894 Budapest), \emph{Industrieller}|pwkv}{ }Mitte Mai 1891 mit, dass er die Scheidung wolle. Julie Herzl\pwindex{Herzl, Julie 1.\,2.\,1868 Budapest – 10.\,11.\,1907 Wien@\textsc{Herzl, Julie} (1.\,2.\,1868 Budapest – 10.\,11.\,1907 Wien)|pwk}, mit der Theodor Herzl\pwindex{Herzl, Theodor 2.\,5.\,1860 Budapest – 3.\,7.\,1904 Edlach@\textsc{Herzl, Theodor} (2.\,5.\,1860 Budapest – 3.\,7.\,1904 Edlach), \emph{Schriftsteller, Journalist}|pwk} bis zu seinem Tod verheiratet blieb, war
                      zu dieser Zeit schwanger. Vgl. Theodor Herzl\pwindex{Herzl, Theodor 2.\,5.\,1860 Budapest – 3.\,7.\,1904 Edlach@\textsc{Herzl, Theodor} (2.\,5.\,1860 Budapest – 3.\,7.\,1904 Edlach), \emph{Schriftsteller, Journalist}|pwk}:
                        \emph{Briefe und
                        Tagebücher}, herausgegeben von Alex Bein, Hermann Greive, Moshe Schaerf und
                        Julius H. Schoeps. \emph{Bd. 1: Briefe und autobiographische
                           Notizen. 1866–1895}, bearbeitet von Johannes Wachten, in
                        Zusammenarbeit mit Chaya Harel, Daisy Tycho und Manfred Winkler.
                        Berlin/Frankfurt am Main/Wien:
                           \emph{Propyläen}{ }1983, S. 439–443.}}}\label{K_L02669-19}? 3.) Weißt Du
                  vielleicht – nicht lachen, bitte! – den Namen einer \strikeout{T}{ }\uline{guten}{ }\strikeout{Tr\textcolor{gray}{u}} Truppe Tirol\oindex{Tirol@\textbf{Tirol}, \emph{Land}|pw}er Sänger, \introOben{}an\introOben{} welche man{ }ſich wenden könnte, um{ }ſie zu einer Reiſe
                  nach Brüſſel\oindex{Brüssel@\textbf{Brüssel}, \emph{Hauptstadt}|pw} zu veranlaſſen?\pend
           \selectlanguage{ngerman}\endnumbering\briefempfaengerindex{Schnitzler, Arthur@\textsc{Schnitzler, Arthur}!zzzGoldmann, Paul@\emph{von Paul Goldmann}!1891-10-271@{27. 10. 1891}|)be}\mylabel{L02669h}  \newcommand{\dateiname}{L02669}\newcommand{\titel}{Paul Goldmann an Arthur Schnitzler, 27. 10. 1891}\newcommand{\editorInnen}{Martin Anton Müller und Laura Untner}%% latex-leseansicht-abspann.tex
%% Abspann für die Leseansicht.
%% Der Schalter \ifkorrekturansicht ist bereits durch den Vorspann gesetzt.

%% latex-abspann.tex
%% Gemeinsamer Abspann für Korrekturansicht und Leseansicht.
%% Setzt den Schalter \ifkorrekturansicht voraus (gesetzt in den
%% einbindenden Dateien latex-korrekturansicht-abspann.tex bzw.
%% latex-leseansicht-abspann.tex).
%% ---------------------------------------------------------------

\normalsize

% Das esempio-Environment wird nur in der Leseansicht benötigt
\ifkorrekturansicht\else
\newenvironment{esempio}[3]%
{
    \vspace{1.5ex}
    \rlap{\underline{#1}}
    \par
    \setlength{\parindent}{0cm}
    \nopagebreak
    \leftskip=#2cm
    \rightskip=#3cm
}
{
    \par
}
\fi

\doendnotes{C}
\bigskip
\vfill

\clearpage

\footnotesize

\ifkorrekturansicht
  \lohead{\textsc{register}}
\fi

% theindex-Environment neu definieren ohne reledmac
\makeatletter
\renewenvironment{theindex}{%
  \ifkorrekturansicht
    \section*{\indexname}%
  \else
    \subsubsection*{Index der erwähnten Entitäten}%
  \fi
  \setlength{\parindent}{0pt}%
  \setlength{\parskip}{0pt plus 0.3pt}%
  \let\item\@idxitem
}{%
  \ifkorrekturansicht\clearpage\fi
}
\makeatother

\IfFileExists{\jobname-pw.ind}{\input{\jobname-pw.ind}}{}

% Quellenangabe nur in der Leseansicht
\ifkorrekturansicht\else
% Fallback-Definitionen, falls die .tex-Datei \titel etc. nicht gesetzt hat
\providecommand{\titel}{}
\providecommand{\editorInnen}{}
\providecommand{\dateiname}{\jobname}

\vspace{3cm}

\vfill

\footnotesize
\textsc{Quelle}: \titel. Herausgegeben von {\editorInnen}. In: \emph{Arthur Schnitzler: Briefwechsel mit Autorinnen und Autoren}.
 Digitale Edition, https://schnitzler-briefe.acdh.oeaw.ac.at/{\dateiname}.html (Stand \today)
\fi

\end{document}


