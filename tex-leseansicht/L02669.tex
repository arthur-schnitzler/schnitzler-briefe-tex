%% latex-leseansicht-vorspann.tex
%% Vorspann für die Leseansicht.
%% Lädt die gemeinsame Datei latex-vorspann.tex mit nicht gesetztem Schalter.

\newif\ifkorrekturansicht
\korrekturansichtfalse

\input{../tex-inputs/latex-vorspann}


         
         \renewcommand{\erwaehntePersonen}{Personen:  ?? [Partnerin von Hector Chainaye, 1891],  ?? [Vorgesetzter Paul Goldmanns 1891], Émile Augier, Richard Beer-Hofmann, Georges Boulanger, Georges Boyer, Hector Chainaye, Anthonis van Dyck, Paul Goldmann, Clementine Goldmann, Knut Hamsun, Jeanette Heeger, Theodor Herzl, Julie Herzl, Hugo von Hofmannsthal, Henrik Ibsen, Jaques Joachim, Friedrich Kapper, Jacob Naschauer, Friedrich Nietzsche,  Rembrandt van Rijn, Vally Rosengart, Peter Paul Rubens, Jules Sandeau, Julius Schnitzler, Leopold Sonnemann,  Stendhal, August Strindberg, Charles Tardieu}
         \renewcommand{\erwaehnteInstitutionen}{Institutionen: Agence Havas, Akademisch-dramatischer Verein München, Frankfurter Zeitung, Freie Bühne, L’Indépendance Belge, Orchestre municipal des pompiers de Bruxelles}
         \renewcommand{\erwaehnteOrte}{Orte: Berlin, Brüssel, Deutschland, Frankreich, Ixelles, München, Niederlande, Norwegen, Paris, Schweden, Skandinavien, Tirol, Wien, rue des Plantes}
         \renewcommand{\erwaehnteWerke}{Werke: ?? [Singspiel, in dem sich ein Vater Tränen der Rührung aus den Augen wischt], Courrier des Théâtres, Die Wildente, Die drei Elixire, Episode, Fröken Julie, La Basoche. Revue Littéraire et Artistique, La Jeune Belgique, La Wallonie, Le Figaro, Le Gendre de M. Poirier. Comédie en 4 actes en prose, Le Rappel, L’Indépendance Belge, L’Âme des choses, Moderne Dichtung. Monatsschrift für Literatur und Kritik, Moderne Rundschau, Nämlich, Théâtres et beaux-arts, Was einem so einfällt, Wien bleibt Wien}
               \section[Paul Goldmann an Arthur Schnitzler, 27. 10. 1891]{ Paul Goldmann an Arthur Schnitzler, 27. 10. 1891}\nopagebreak\mylabel{v}\rehead{ }\begin{ledgroupsized}[t]{13cm}\normalsize\beginnumbering\briefempfaengerindex{Schnitzler, Arthur@\textsc{Schnitzler, Arthur}!zzzGoldmann, Paul@\emph{von Paul Goldmann}!1891-10-271@{27. 10. 1891}|(be} \toendnotes[C]{\smallbreak\pagebreak[2]} \Standort{DLA, A:Schnitzler, HS.NZ85.1.3162.}
\physDesc{Brief, 3 Blätter, 10 Seiten, 6969 Zeichen
\newline{}Handschrift: blaue Tinte, deutsche Kurrent
\newline{}Schnitzler: mit rotem Buntstift zwei Unterstreichungen }\toendnotes[C]{\smallbreak}\pstart
           \noindent{}\centering{}{\pb}\textcolor{gray}{\textbf{Dr. jur. Paul Goldmann}}\pend
           \pstart
           \noindent{}\centering{}\textcolor{gray}{\textbf{\begin{otherlanguage}{french}Correspondant de la »Gazette de Francfort\orgindex{Frankfurter Zeitung@Frankfurter Zeitung|pw}«\end{otherlanguage}}}\pend
           \pstart
           \noindent{}\centering{}\textcolor{gray}{\textbf{\begin{otherlanguage}{french}Bruxelles, 21, rue des Plantes\end{otherlanguage}\oindex{rue des Plantes@\textbf{rue des Plantes}|pw}.}}\pend
           \pstart
           \raggedleft{}Brüſſel\oindex{Bruessel@\textbf{Brüssel}|pw}, 27.
                  October 91.\pend
           \pstart\center{}Mein lieber Arthur!\pend\pstart
           Ich entſchließe mich nicht leicht zum Schreiben an Dich, offen geſtanden. Denn ich
               komme mir vor, wie \strikeout{ein\textcolor{gray}{er}} ein läſtiger Mahner, der eine Gefühlsſchuld eintreiben will, zu deren
               Honorirung nicht mehr der nöthige Beſtand vorhanden iſt. Alle Symptome ſprechen mir
               dafür, daß das gekommen iſt, was kommen mußte: Daß ich für Euch ein Stück
               Vergangenheit geworden bin; und als ſolches habe ich natürlich weit hinter den Sachen
               Eurer Gegenwart zurückzuſtehen. Ich bin eine Erinnerung für einſame Sonntag
               Nachmittage geworden{\dotsfive}\pend
           \pstart
           Alſo einiges von mir. In Brüſſel\oindex{Bruessel@\textbf{Brüssel}|pw} geht es mir
               jetzt etwas beſſer – moraliſch wenigſtens. Ich bin den Leuten hier ein klein wenig
               näher getreten, habe {\pb}manchen lieben Menſchen,
               manche ſchöne Künſtlernatur gefunden und bin mit dem Einen oder dem Andern wenn auch
               nicht Freund, ſo doch gut bekannt geworden. \strikeout{\textcolor{gray}{×}} Sogar ein kleines Milieu junger Künſtler und Lebemänner in meinem Alter, ein
                  \label{K_L02669-1v}\edtext{\textsc{Milieu} der \textsc{Hector\pwindex{Augier, Emile 1820-09-17 – 1889-10-25@\textsc{Augier, Émile} (1820-09-17 – 1889-10-25), \emph{Schriftsteller}!Le Gendre de M. Poirier. Comedie en 4 actes en prose1854-04-08@\strich\emph{Le Gendre de M. Poirier. Comédie en 4 actes en prose} {[}1854-04-08{]}|pwv}\pwindex{Sandeau, Jules 1811-02-19 – 1883-04-24@\textsc{Sandeau, Jules} (1811-02-19 – 1883-04-24), \emph{Schriftsteller}!Le Gendre de M. Poirier. Comedie en 4 actes en prose1854-04-08@\strich\emph{Le Gendre de M. Poirier. Comédie en 4 actes en prose} {[}1854-04-08{]}|pwv}s} und \textsc{Gaston\pwindex{Augier, Emile 1820-09-17 – 1889-10-25@\textsc{Augier, Émile} (1820-09-17 – 1889-10-25), \emph{Schriftsteller}!Le Gendre de M. Poirier. Comedie en 4 actes en prose1854-04-08@\strich\emph{Le Gendre de M. Poirier. Comédie en 4 actes en prose} {[}1854-04-08{]}|pwv}\pwindex{Sandeau, Jules 1811-02-19 – 1883-04-24@\textsc{Sandeau, Jules} (1811-02-19 – 1883-04-24), \emph{Schriftsteller}!Le Gendre de M. Poirier. Comedie en 4 actes en prose1854-04-08@\strich\emph{Le Gendre de M. Poirier. Comédie en 4 actes en prose} {[}1854-04-08{]}|pwv}s}}{\lemma{\textnormal{\emph{Milieu … Gastons}}}\Cendnote{\textnormal{Er dürfte sich auf die zwei verarmten
                  adeligen Lebemänner Hector de
                     Montmeyran\pwindex{Augier, Emile 1820-09-17 – 1889-10-25@\textsc{Augier, Émile} (1820-09-17 – 1889-10-25), \emph{Schriftsteller}!Le Gendre de M. Poirier. Comedie en 4 actes en prose1854-04-08@\strich\emph{Le Gendre de M. Poirier. Comédie en 4 actes en prose} {[}1854-04-08{]}|pwkv}\pwindex{Sandeau, Jules 1811-02-19 – 1883-04-24@\textsc{Sandeau, Jules} (1811-02-19 – 1883-04-24), \emph{Schriftsteller}!Le Gendre de M. Poirier. Comedie en 4 actes en prose1854-04-08@\strich\emph{Le Gendre de M. Poirier. Comédie en 4 actes en prose} {[}1854-04-08{]}|pwkv} und Gaston de
                     Presle\pwindex{Augier, Emile 1820-09-17 – 1889-10-25@\textsc{Augier, Émile} (1820-09-17 – 1889-10-25), \emph{Schriftsteller}!Le Gendre de M. Poirier. Comedie en 4 actes en prose1854-04-08@\strich\emph{Le Gendre de M. Poirier. Comédie en 4 actes en prose} {[}1854-04-08{]}|pwkv}\pwindex{Sandeau, Jules 1811-02-19 – 1883-04-24@\textsc{Sandeau, Jules} (1811-02-19 – 1883-04-24), \emph{Schriftsteller}!Le Gendre de M. Poirier. Comedie en 4 actes en prose1854-04-08@\strich\emph{Le Gendre de M. Poirier. Comédie en 4 actes en prose} {[}1854-04-08{]}|pwkv} aus der Komödie \emph{Le Gendre de M.
                     Poirier}\pwindex{Augier, Emile 1820-09-17 – 1889-10-25@\textsc{Augier, Émile} (1820-09-17 – 1889-10-25), \emph{Schriftsteller}!Le Gendre de M. Poirier. Comedie en 4 actes en prose1854-04-08@\strich\emph{Le Gendre de M. Poirier. Comédie en 4 actes en prose} {[}1854-04-08{]}|pwk}\pwindex{Sandeau, Jules 1811-02-19 – 1883-04-24@\textsc{Sandeau, Jules} (1811-02-19 – 1883-04-24), \emph{Schriftsteller}!Le Gendre de M. Poirier. Comedie en 4 actes en prose1854-04-08@\strich\emph{Le Gendre de M. Poirier. Comédie en 4 actes en prose} {[}1854-04-08{]}|pwk} (1854) von Émile Augier\pwindex{Augier, Emile 1820-09-17 – 1889-10-25@\textsc{Augier, Émile} (1820-09-17 – 1889-10-25), \emph{Schriftsteller}|pwk} und
                     Jules Sandeau\pwindex{Sandeau, Jules 1811-02-19 – 1883-04-24@\textsc{Sandeau, Jules} (1811-02-19 – 1883-04-24), \emph{Schriftsteller}|pwk} beziehen.}}}\label{K_L02669-1h}, habe ich
               gefunden. Am meiſten verkehre ich mit \textsc{Chainaye\pwindex{Chainaye, Hector 1865-04-14 – 1913-09-04@\textsc{Chainaye, Hector} (1865-04-14 – 1913-09-04), \emph{Schriftsteller, Journalist}|pw}}, dem jüngſten Redacteur\pwindex{Chainaye, Hector 1865-04-14 – 1913-09-04@\textsc{Chainaye, Hector} (1865-04-14 – 1913-09-04), \emph{Schriftsteller, Journalist}|pwv} der \textsc{Indépendance Belge}\orgindex{Independance Belge@L’Indépendance Belge|pw}: enragirter Wallone\pwindex{Chainaye, Hector 1865-04-14 – 1913-09-04@\textsc{Chainaye, Hector} (1865-04-14 – 1913-09-04), \emph{Schriftsteller, Journalist}|pwv}
               und \label{K_L02669-2v}\edtext{Romane\pwindex{Chainaye, Hector 1865-04-14 – 1913-09-04@\textsc{Chainaye, Hector} (1865-04-14 – 1913-09-04), \emph{Schriftsteller, Journalist}|pwv}}{\lemma{\textnormal{\emph{Romane}}}\Cendnote{\textnormal{»Belgique romane« ist ein Überbegriff
                  für mehrere Dialekte. Der bedeutendste ist der wallonische.}}}\label{K_L02669-2h}, reiches
               künſtleriſches Sentiment, Stimmungsmenſch\pwindex{Chainaye, Hector 1865-04-14 – 1913-09-04@\textsc{Chainaye, Hector} (1865-04-14 – 1913-09-04), \emph{Schriftsteller, Journalist}|pwv}, melancholiſches Talent, Verfaſſer\pwindex{Chainaye, Hector 1865-04-14 – 1913-09-04@\textsc{Chainaye, Hector} (1865-04-14 – 1913-09-04), \emph{Schriftsteller, Journalist}|pwv} myſtiſch-empfindſamer \label{K_L02669-3v}\edtext{Gedichte in Proſa}{\lemma{\textnormal{\emph{Gedichte in Proſa}}}\Cendnote{\textnormal{Prosagedichte Hector Chainaye\pwindex{Chainaye, Hector 1865-04-14 – 1913-09-04@\textsc{Chainaye, Hector} (1865-04-14 – 1913-09-04), \emph{Schriftsteller, Journalist}|pwk}s finden sich
                  zum Beispiel in seinem Band \emph{L’Âme des choses}\pwindex{Chainaye, Hector 1865-04-14 – 1913-09-04@\textsc{Chainaye, Hector} (1865-04-14 – 1913-09-04), \emph{Schriftsteller, Journalist}!Âme des choses1935@\strich\emph{L’Âme des choses} {[}1935{]}|pwk}
                     (1935). Viele der darin enthaltenen Gedichte wurden bereits
                  zwischen 1886 und 1888 in Zeitschriften wie \emph{La Wallonie}\pwindex{?? Werk@Nicht ermittelte Verfasserinnen und Verfasser!Wallonie1886 – 1892@\emph{La Wallonie} {[}1886 – 1892{]}|pwk}, \emph{La Basoche}\pwindex{?? Werk@Nicht ermittelte Verfasserinnen und Verfasser!Basoche. Revue Litteraire et Artistique1884 – 1886@\emph{La Basoche. Revue Littéraire et Artistique} {[}1884 – 1886{]}|pwk} und \emph{La Jeune Belgique}\pwindex{?? Werk@Nicht ermittelte Verfasserinnen und Verfasser!Jeune Belgique1880 – 1897@\emph{La Jeune Belgique} {[}1880 – 1897{]}|pwk}
                  veröffentlicht.}}}\label{K_L02669-3h}, blond, krank, \strikeout{ſ}
               geiſtſprühend und luſtig in der Converſation bei dem Allen und – was das beſte iſt –
               mit einigen \strikeout{k\textcolor{gray}{l}} Zügen, die entfernt an Dich erinnern. Nach Beſiegung des Deutſch\oindex{Deutschland@\textbf{Deutschland}|pwv}enhaſſes, der
               Verſtändigungsſchwierigkeiten, des Mißtrauens gegen den Fremden \textsc{etc. etc.} bin ich ihm näher getreten. Und in dieſe\substVorne{}\textsuperscript{m}\substDazwischen{}n\substHinten{}{ }{\pb}Tagen ſtehe ich ihm rathend zur Seite bei einem
               großen Bruch mit ſeiner \label{K_L02669-4v}\edtext{Maitreſſe\pwindex{?? [Partnerin von Hector Chainaye, 1891] @\textsc{?? [Partnerin von Hector Chainaye, 1891]}|pwv}}{\lemma{\textnormal{\emph{Maitreſſe}}}\Cendnote{\textnormal{nicht identifiziert}}}\label{K_L02669-4h}, die ſich zu
               tödten droht \textsc{etc. etc.} (ſiehe \label{K_L02669-5v}\edtext{\textsc{Jeannette\pwindex{Heeger, Jeanette 1865-07-01 – 1903-01-03@\textsc{Heeger, Jeanette} (1865-07-01 – 1903-01-03), \emph{Näherin}|pw}}}{\lemma{\textnormal{\emph{Jeannette}}}\Cendnote{\textnormal{Jeannette Heeger\pwindex{Heeger, Jeanette 1865-07-01 – 1903-01-03@\textsc{Heeger, Jeanette} (1865-07-01 – 1903-01-03), \emph{Näherin}|pwk}, Geliebte Schnitzler\pwindex{Schnitzler, Arthur 15.05.1862 – 21.10.1931@\textsc{Schnitzler, Arthur} (15.05.1862 – 21.10.1931), \emph{Schriftsteller, Mediziner}|pwk}s, unternahm am 18. 12. 1889 einen
                  Suizidversuch mit einer Pistole.}}}\label{K_L02669-5h}.) Ein närriſches Ding, das Leben, – nicht
               wahr? Außerdem haben ſich meine Beziehungen zu den Brüſſel\oindex{Bruessel@\textbf{Brüssel}|pw}er Journaliſten ſichtlich verbeſſert. Es iſt ein geradezu enormer
               Unterſchied zwiſchen den Brüſſel\oindex{Bruessel@\textbf{Brüssel}|pw}er und den Wien\oindex{Wien@\textbf{Wien}|pw}er Collegen. Hier ſind es – von wenigen
               Ausnahmen abgeſehen – liebe, gute Burſchen mit prächtigem Benehmen, voll Gefälligkeit
               und Liebenswürdigkeit, und manch’ eine ſchöne Künftlernatur iſt auch hier darunter –
               Leute, die den Journalismus machen, um Brod zu verdienen, aber im Übrigen \label{K_L02669-6v}\edtext{\textsc{\begin{otherlanguage}{french}s’en fichent\end{otherlanguage}}}{\lemma{\textnormal{\emph{s’en fichent}}}\Cendnote{\textnormal{französisch: sich nicht kümmern}}}\label{K_L02669-6h}
               und warmen Herzens der Kunſt anhängen. Ich mache hier eifrige Propaganda für die
                  \label{K_L02669-7v}\edtext{Norwege\oindex{Norwegen@\textbf{Norwegen}|pwv}r\pwindex{Ibsen, Henrik 20.03.1828 – 23.05.1906@\textsc{Ibsen, Henrik} (20.03.1828 – 23.05.1906), \emph{Schriftsteller}|pwuv}\pwindex{Hamsun, Knut 1859-08-04 – 1952-02-19@\textsc{Hamsun, Knut} (1859-08-04 – 1952-02-19), \emph{Schriftsteller, Nobelpreisträger}|pwuv}}{\lemma{\textnormal{\emph{Norweger}}}\Cendnote{\textnormal{Gemeint sein dürfte vor allem Henrik Ibsen\pwindex{Ibsen, Henrik 20.03.1828 – 23.05.1906@\textsc{Ibsen, Henrik} (20.03.1828 – 23.05.1906), \emph{Schriftsteller}|pwk}, eventuell auch Knut Hamsun\pwindex{Hamsun, Knut 1859-08-04 – 1952-02-19@\textsc{Hamsun, Knut} (1859-08-04 – 1952-02-19), \emph{Schriftsteller, Nobelpreisträger}|pwk}. In der im Folgenden erwähnten
                     Zeitungsmeldung\pwindex{Tardieu, Charles 1838-02-09 – 1909@\textsc{Tardieu, Charles} (1838-02-09 – 1909), \emph{Journalist, Redakteur}!Theâtres et beaux-arts1891-10-08@\strich\emph{Théâtres et beaux-arts} {[}1891-10-08{]}|pwkv} von Charles Tardieu\pwindex{Tardieu, Charles 1838-02-09 – 1909@\textsc{Tardieu, Charles} (1838-02-09 – 1909), \emph{Journalist, Redakteur}|pwk} wird allgemein von der Ibsen\pwindex{Ibsen, Henrik 20.03.1828 – 23.05.1906@\textsc{Ibsen, Henrik} (20.03.1828 – 23.05.1906), \emph{Schriftsteller}|pwk}-Schule gesprochen und vor allem der
                     Schwede\oindex{Schweden@\textbf{Schweden}|pwkv}{ }August Strindberg\pwindex{Strindberg, August 22.01.1849 – 14.05.1912@\textsc{Strindberg, August} (22.01.1849 – 14.05.1912), \emph{Schriftsteller}|pwk} behandelt.}}}\label{K_L02669-7h}, und
                  \textsc{Tardieu\pwindex{Tardieu, Charles 1838-02-09 – 1909@\textsc{Tardieu, Charles} (1838-02-09 – 1909), \emph{Journalist, Redakteur}|pw}}, der Chefredacteur\pwindex{Tardieu, Charles 1838-02-09 – 1909@\textsc{Tardieu, Charles} (1838-02-09 – 1909), \emph{Journalist, Redakteur}|pwv} der
                  \textsc{Indépendance\orgindex{Independance Belge@L’Indépendance Belge|pw}}, der unter den intereſſanten {\pb}hieſigen \strikeout{\textcolor{gray}{S}} Collegen vielleicht der intereſſanteſte iſt, hat dieſe meine Bemühungen ſammt
               Citat meines Namens in der \textsc{Indép.\pwindex{?? Werk@Nicht ermittelte Verfasserinnen und Verfasser!Independance Belge1843 – 1940@\emph{L’Indépendance Belge} {[}1843 – 1940{]}|pw}}{ }\label{K_L02669-8v}\edtext{verewigt}{\lemma{\textnormal{\emph{verewigt}}}\Cendnote{\textnormal{Charles Tardieu\pwindex{Tardieu, Charles 1838-02-09 – 1909@\textsc{Tardieu, Charles} (1838-02-09 – 1909), \emph{Journalist, Redakteur}|pwk}: \emph{Théâtres et beaux-arts}\pwindex{Tardieu, Charles 1838-02-09 – 1909@\textsc{Tardieu, Charles} (1838-02-09 – 1909), \emph{Journalist, Redakteur}!Theâtres et beaux-arts1891-10-08@\strich\emph{Théâtres et beaux-arts} {[}1891-10-08{]}|pwk}. In: \emph{L’Indépendance Belge}\pwindex{Tardieu, Charles 1838-02-09 – 1909@\textsc{Tardieu, Charles} (1838-02-09 – 1909), \emph{Journalist, Redakteur}!Theâtres et beaux-arts1891-10-08@\strich\emph{Théâtres et beaux-arts} {[}1891-10-08{]}|pwk}, Jg. 62, H. 281,
                        8. 10. 1891, Abendausgabe, S. 3: »\begin{otherlanguage}{french}Voilà qui nous mène en Scandinavie\oindex{Skandinavien@\textbf{Skandinavien}|pw} et de là à Berlin\oindex{Berlin@\textbf{Berlin}|pw}
                        et Munich\oindex{Muenchen@\textbf{München}|pw}, où l’école ibsén\pwindex{Ibsen, Henrik 20.03.1828 – 23.05.1906@\textsc{Ibsen, Henrik} (20.03.1828 – 23.05.1906), \emph{Schriftsteller}|pw}ienne a un public enthousiaste.
                        Mais que parlons-nous encore d’Ibsen\pwindex{Ibsen, Henrik 20.03.1828 – 23.05.1906@\textsc{Ibsen, Henrik} (20.03.1828 – 23.05.1906), \emph{Schriftsteller}|pw}?
                           L’auteur\pwindex{Ibsen, Henrik 20.03.1828 – 23.05.1906@\textsc{Ibsen, Henrik} (20.03.1828 – 23.05.1906), \emph{Schriftsteller}|pwv} du \emph{Canard sauvage}\pwindex{Ibsen, Henrik 20.03.1828 – 23.05.1906@\textsc{Ibsen, Henrik} (20.03.1828 – 23.05.1906), \emph{Schriftsteller}!Wildente1884@\strich\emph{Die Wildente} {[}1884{]}|pw} est absolument distancé dans son pays\oindex{Norwegen@\textbf{Norwegen}|pwv}. Novateur\pwindex{Ibsen, Henrik 20.03.1828 – 23.05.1906@\textsc{Ibsen, Henrik} (20.03.1828 – 23.05.1906), \emph{Schriftsteller}|pwv} et réformateur\pwindex{Ibsen, Henrik 20.03.1828 – 23.05.1906@\textsc{Ibsen, Henrik} (20.03.1828 – 23.05.1906), \emph{Schriftsteller}|pwv} en Allemagne\oindex{Deutschland@\textbf{Deutschland}|pw} et en France\oindex{Frankreich@\textbf{Frankreich}|pw}, il est
                        déjà ›vieux jeu‹ dans sa Norvège\oindex{Norwegen@\textbf{Norwegen}|pw}.
                        Notre confrère\pwindex{Goldmann, Paul 31.01.1865 – 25.09.1935@\textsc{Goldmann, Paul} (31.01.1865 – 25.09.1935), \emph{Schriftsteller, Journalist}|pwv} de
                        la \emph{Gazette de Francfort}\orgindex{Frankfurter Zeitung@Frankfurter Zeitung|pw}, le docteur Goldmann\pwindex{Goldmann, Paul 31.01.1865 – 25.09.1935@\textsc{Goldmann, Paul} (31.01.1865 – 25.09.1935), \emph{Schriftsteller, Journalist}|pw}, très au
                        courant des curiosités et nouveautés littéraires, nous expliquait cela
                        dernièrement, et il nous prédisait le prochain avènement d’Auguste Strindberg\pwindex{Strindberg, August 22.01.1849 – 14.05.1912@\textsc{Strindberg, August} (22.01.1849 – 14.05.1912), \emph{Schriftsteller}|pw}, un dramaturge\pwindex{Strindberg, August 22.01.1849 – 14.05.1912@\textsc{Strindberg, August} (22.01.1849 – 14.05.1912), \emph{Schriftsteller}|pwv}{ }suéd\oindex{Schweden@\textbf{Schweden}|pwv}ois et niet{[}z{]}sch\pwindex{Nietzsche, Friedrich 15.10.1844 – 25.08.1900@\textsc{Nietzsche, Friedrich} (15.10.1844 – 25.08.1900), \emph{Schriftsteller, Philosoph}|pw}ien. Suéd\oindex{Schweden@\textbf{Schweden}|pw}ois? vous comprenez. Mais pour ›niet{[}z{]}sch\pwindex{Nietzsche, Friedrich 15.10.1844 – 25.08.1900@\textsc{Nietzsche, Friedrich} (15.10.1844 – 25.08.1900), \emph{Schriftsteller, Philosoph}|pw}ien‹
                        sachez que Frédéric
                              Niet{[}z{]}sche\pwindex{Nietzsche, Friedrich 15.10.1844 – 25.08.1900@\textsc{Nietzsche, Friedrich} (15.10.1844 – 25.08.1900), \emph{Schriftsteller, Philosoph}|pw} est, comme eût dit Stendhal\pwindex{Stendhal 1783-01-23 – 1842-03-23@\textsc{Stendhal} (1783-01-23 – 1842-03-23), \emph{Schriftsteller}|pw}, ›l’expression la plus
                        récente‹ de la philosophie allema\oindex{Deutschland@\textbf{Deutschland}|pwv}nde. Or, voici que la prédiction se vérifie. Le Théâtre Libre\orgindex{Freie Buehne@Freie Bühne|pw}\orgindex{Akademisch-dramatischer Verein Muenchen@Akademisch-dramatischer Verein München|pw} de Berlin\oindex{Berlin@\textbf{Berlin}|pw} et celui de Munich\oindex{Muenchen@\textbf{München}|pw} monteront cet hiver \emph{Mademoiselle Julie}\pwindex{Strindberg, August 22.01.1849 – 14.05.1912@\textsc{Strindberg, August} (22.01.1849 – 14.05.1912), \emph{Schriftsteller}!Froeken Julie1888@\strich\emph{Fröken Julie} {[}1888{]}|pw}, de M. Auguste Strindberg\pwindex{Strindberg, August 22.01.1849 – 14.05.1912@\textsc{Strindberg, August} (22.01.1849 – 14.05.1912), \emph{Schriftsteller}|pw}, une
                           tragédie\pwindex{Strindberg, August 22.01.1849 – 14.05.1912@\textsc{Strindberg, August} (22.01.1849 – 14.05.1912), \emph{Schriftsteller}!Froeken Julie1888@\strich\emph{Fröken Julie} {[}1888{]}|pwv}
                        naturaliste à trois personnages, en un acte et une nuit. En deux mots Mlle Julie\pwindex{Strindberg, August 22.01.1849 – 14.05.1912@\textsc{Strindberg, August} (22.01.1849 – 14.05.1912), \emph{Schriftsteller}!Froeken Julie1888@\strich\emph{Fröken Julie} {[}1888{]}|pwv}, hystérique
                        par atavisme, est amoureuse du domestique de son père. Elle fait
                        littéralement le siège du valet qui lutte et-succombe. Tous deux se
                        préparent à s’enfuir. Mais la cuisinière raisonne les deux amants, les
                        rappelle au sentiment des convenances sociales, et, ma foi, réussit à les
                        calmer. La toile tombe sur une rupture, definitive, espérons-le. Il est
                        probable que l’analyse des caractères ajoute à l'intérêt de cette donnée,
                        déjà séduisante par elle même. De quoi s’agit-il après tout? D'un accident.
                        A quoi bon se troubler et déranger sa vie pour si peu de chose? Christine
                        est dans le vrai. On voit bien qu'elle sait l'art d'accommoder les
                        restes.\end{otherlanguage}«}}}\label{K_L02669-8h}, worauf dann die Notiz\pwindex{Tardieu, Charles 1838-02-09 – 1909@\textsc{Tardieu, Charles} (1838-02-09 – 1909), \emph{Journalist, Redakteur}!Theâtres et beaux-arts1891-10-08@\strich\emph{Théâtres et beaux-arts} {[}1891-10-08{]}|pwv} mit »\label{K_L02669-9v}\edtext{\textsc{\begin{otherlanguage}{french}notre confrère le docteur Goldmann de le Gazette de Francfort\orgindex{Frankfurter Zeitung@Frankfurter Zeitung|pw}\end{otherlanguage}}\pwindex{Tardieu, Charles 1838-02-09 – 1909@\textsc{Tardieu, Charles} (1838-02-09 – 1909), \emph{Journalist, Redakteur}!Theâtres et beaux-arts1891-10-08@\strich\emph{Théâtres et beaux-arts} {[}1891-10-08{]}|pwv}}{\lemma{\textnormal{\emph{notre … Francfort}}}\Cendnote{\textnormal{französisch: unser Kollege\pwindex{Goldmann, Paul 31.01.1865 – 25.09.1935@\textsc{Goldmann, Paul} (31.01.1865 – 25.09.1935), \emph{Schriftsteller, Journalist}|pwkv} Dr. Goldmann\pwindex{Goldmann, Paul 31.01.1865 – 25.09.1935@\textsc{Goldmann, Paul} (31.01.1865 – 25.09.1935), \emph{Schriftsteller, Journalist}|pwk} von der \emph{Frankfurter Zeitung}\orgindex{Frankfurter Zeitung@Frankfurter Zeitung|pwk}}}}\label{K_L02669-9h}« die Runde durch die Pariſ\oindex{Paris@\textbf{Paris}|pw}er Preſſe, vom
                  \label{K_L02669-10v}\edtext{\textsc{Figaro\pwindex{Le Figaro1826-01-15@\emph{Le Figaro} {[}1826-01-15{]}|pw}}}{\lemma{\textnormal{\emph{Figaro}}}\Cendnote{\textnormal{Georges Boyer\pwindex{Boyer, Georges 1850-07-21 – 1931-04-25@\textsc{Boyer, Georges} (1850-07-21 – 1931-04-25), \emph{Dramaturg, Theaterkritiker}|pwk}: \emph{Courrier des Théâtres}\pwindex{Boyer, Georges 1850-07-21 – 1931-04-25@\textsc{Boyer, Georges} (1850-07-21 – 1931-04-25), \emph{Dramaturg, Theaterkritiker}!Courrier des Theâtres1891-10-13@\strich\emph{Courrier des Théâtres} {[}1891-10-13{]}|pwk}. In: \emph{Le Figaro}\pwindex{Le Figaro1826-01-15@\emph{Le Figaro} {[}1826-01-15{]}|pwk}, Jg. 37, H. 286, 13. 10. 1891,
                     S. 3.}}}\label{K_L02669-10h} bis zum \label{K_L02669-11v}\edtext{\textsc{Rappel\pwindex{?? Werk@Nicht ermittelte Verfasserinnen und Verfasser!Le Rappel1869-05-04 – 1933@\emph{Le Rappel} {[}1869-05-04 – 1933{]}|pw}}}{\lemma{\textnormal{\emph{Rappel}}}\Cendnote{\textnormal{nicht nachgewiesen}}}\label{K_L02669-11h}, gemacht
               hat. Auch d\substVorne{}\textsuperscript{\textcolor{gray}{ie}}\substDazwischen{}er\substHinten{} Verkehr \substVorne{}\textsuperscript{zu\textcolor{gray}{r}}\substDazwischen{}mit der\substHinten{} officiellen Welt iſt angenehm. Ich werde von mehreren Miniſtern mit allen
               meinem Range gebührenden Ehren empfangen \textsc{etc.} Außerdem iſt
               die Stadt\oindex{Bruessel@\textbf{Brüssel}|pwv} mit ihrem \substVorne{}\textsuperscript{Schein}{\allowbreak}\substDazwischen{}Abglanz\substHinten{}{ }franz\oindex{Frankreich@\textbf{Frankreich}|pwv}öſiſchen Kunſtlebens recht
               intereſſant, und es gibt ſchöne Abende im Theater und im Concert. Endlich das
               herrliche Hiſtoriſche. Die alte niederländ\oindex{Niederlande@\textbf{Niederlande}|pwv}iſche Malerei. Ich beginne hier langſam zu begreifen, was das für
               Dinger ſind, die \textsc{Rubens\pwindex{Rubens, Peter Paul 1577-06-28 – 1640-05-30@\textsc{Rubens, Peter Paul} (1577-06-28 – 1640-05-30), \emph{Bildender Künstler}|pw}}, \textsc{van Dyck\pwindex{Dyck, Anthonis van 22.03.1599 – 09.12.1641@\textsc{Dyck, Anthonis van} (22.03.1599 – 09.12.1641), \emph{Bildender Künstler}|pw}} und \textsc{Rembrandt\pwindex{Rembrandt van Rijn 15.07.1606 – 04.10.1669@\textsc{Rembrandt van Rijn} (15.07.1606 – 04.10.1669), \emph{Bildender Künstler}|pw}}. Und das iſt ein Quell neuer und {\pb}ungeahnter
               Genüſſe.\pend
           \pstart
           Das ſind die guten Seiten. Aber die böſen ſind geblieben, ſind vielleicht noch
               troſtloſer als zuvor, und haben nur die Geſichter zum Theil gewechſelt. Keine
               Zukunft, keine Zukunft. Die Möglichkeit, ſich ein Vermögen zu machen, exiſtirt nicht.
               Mein Gehalt iſt jämmerlich und wird nicht geſteigert. Die großen \label{K_L02669-12v}\edtext{Pflichten, die ich gegen die Meinen\pwindex{Rosengart, Vally *~1866-12-29@\textsc{Rosengart, Vally} (*~1866-12-29)|pwv}\pwindex{Goldmann, Clementine 1842-05-15 – 1924-02-24@\textsc{Goldmann, Clementine} (1842-05-15 – 1924-02-24)|pwv}}{\lemma{\textnormal{\emph{Pflichten, … Meinen}}}\Cendnote{\textnormal{siehe Paul Goldmann an Arthur Schnitzler, 27. 4. 1891}}}\label{K_L02669-12h} habe, treten immer drohender an mich heran. Und außerdem werde ich von Seiten
               des Blatt\orgindex{Frankfurter Zeitung@Frankfurter Zeitung|pwv}es genau ſo gemein und
               ungerecht behandelt, wie es mir in Wien\oindex{Wien@\textbf{Wien}|pw} geſchehen
               – H. \textsc{Sonnemann\pwindex{Sonnemann, Leopold 1831-10-29 – 1909-10-30@\textsc{Sonnemann, Leopold} (1831-10-29 – 1909-10-30), \emph{Journalist, Herausgeber}|pw}}, der Chef\pwindex{Sonnemann, Leopold 1831-10-29 – 1909-10-30@\textsc{Sonnemann, Leopold} (1831-10-29 – 1909-10-30), \emph{Journalist, Herausgeber}|pwv} und Gebieter\pwindex{Sonnemann, Leopold 1831-10-29 – 1909-10-30@\textsc{Sonnemann, Leopold} (1831-10-29 – 1909-10-30), \emph{Journalist, Herausgeber}|pwv}, iſt ein \strikeout{erbarmu} erbarmungsloſer Blutſauger\pwindex{Sonnemann, Leopold 1831-10-29 – 1909-10-30@\textsc{Sonnemann, Leopold} (1831-10-29 – 1909-10-30), \emph{Journalist, Herausgeber}|pwv}, der verlangt, daß ſich ſeine
               Leute zu Tode ſchinden und der ihnen auch {\pb}dann noch
               beim kleinſten Verſehen heftige Vorwürfe macht. Außerdem ſitzt eine \label{K_L02669-13v}\edtext{Canaille\pwindex{?? [Vorgesetzter Paul Goldmanns 1891] @\textsc{?? [Vorgesetzter Paul Goldmanns 1891]}|pwv}}{\lemma{\textnormal{\emph{Canaille}}}\Cendnote{\textnormal{Schurke, Bösewicht}}}\label{K_L02669-13h} in der Redaction\orgindex{Frankfurter Zeitung@Frankfurter Zeitung|pwv}, ein Menſch\pwindex{?? [Vorgesetzter Paul Goldmanns 1891] @\textsc{?? [Vorgesetzter Paul Goldmanns 1891]}|pwv}, der mich kaum kennt, dem ich nie
               etwas gethan habe und der mich trotzdem haßt, Gott weiß warum. Er iſt zum Unglück
               mein \label{K_L02669-14v}\edtext{unmittelbarer Vorgeſetzter\pwindex{?? [Vorgesetzter Paul Goldmanns 1891] @\textsc{?? [Vorgesetzter Paul Goldmanns 1891]}|pwv}}{\lemma{\textnormal{\emph{unmittelbarer Vorgeſetzter}}}\Cendnote{\textnormal{nicht identifiziert}}}\label{K_L02669-14h}, und ihm
               habe ich es zu danken, daß \strikeout{\textcolor{gray}{man}} meine Ernennung für den Pariſ\oindex{Paris@\textbf{Paris}|pw}er Poſten,
               welche im Zuge war, unterblieb, weil ich mit der \label{K_L02669-15v}\edtext{Nachricht vom Tode \textsc{Boulanger\pwindex{Boulanger, Georges 29.04.1837 – 30.09.1891@\textsc{Boulanger, Georges} (29.04.1837 – 30.09.1891), \emph{Politiker, Militär}|pw}s}}{\lemma{\textnormal{\emph{Nachricht … Boulangers}}}\Cendnote{\textnormal{Georges Boulanger\pwindex{Boulanger, Georges 29.04.1837 – 30.09.1891@\textsc{Boulanger, Georges} (29.04.1837 – 30.09.1891), \emph{Politiker, Militär}|pwk} hatte am
                     30. 9. 1891 in Ixelles\oindex{Ixelles@\textbf{Ixelles}|pwk}
                  Suizid begangen.}}}\label{K_L02669-15h} eine Stunde ſpäter gekommen, als die officielle
               Telegraphenagentur \textcolor{gray}{–} die \textsc{Agence Havas\orgindex{Agence Havas@Agence Havas|pw}}! Und ähnliche Schurkereien. Ich leide entſetzlich darunter und ſehne mich
               blutenden Herzens mehr als je nach Erlöſung. Ein kleines Capital und Rückkehr nach
                  Wien\oindex{Wien@\textbf{Wien}|pw}. Denn das iſt nach wie vor das oberſte
               Ziel meiner Wünſche. Es vergeht nach wie vor kein Tag, {\pb}wo ich nicht zehn-, zwanzigmal an Dich und die
               theure Stadt\oindex{Wien@\textbf{Wien}|pwv} denke. Und als das
                  Orcheſter der \textsc{Pompiers}\orgindex{Orchestre municipal des pompiers de Bruxelles@Orchestre municipal des pompiers de Bruxelles|pw}{ }Sonntag die Straßen mit dem Schrammel-Marſch\pwindex{\textcolor{red}{\textsuperscript{XXXX1 indx}}!Wien bleibt Wien1886@\strich\emph{Wien bleibt Wien} {[}1886{]}|pwv} durchzog, lief ich
               hinterher und wiſchte mir, wie der bekannte Vater im Singſpiel\pwindex{?? Werk@Nicht ermittelte Verfasserinnen und Verfasser!?? [Singspiel, in dem sich ein Vater Traenen der Ruehrung aus den Augen
                  wischt]vor 1891@\emph{?? [Singspiel, in dem sich ein Vater Tränen der Rührung aus den Augen wischt]} {[}vor 1891{]}|pwv}, die Thränen mit dem Rockärmel
               ab. Aber ich habe keine Hoffnung. Mein Leben wird in harter Sklaverei verfließen,
               fern von Allem, was ich lieb habe; und zu großen befreienden Werken habe ich weder
               das genügende Talent, noch die genügende Energie{\dotsfive}\pend
           \pstart
           Wollte ich nun alle die Fragen aufſchreiben, die ich an Dich zu richten habe, es
               ginge noch ein Briefbogen darauf. Aber ich thue es nicht; denn ich weiß, daß du mir
               ſie eh’ nicht beantworten wirſt. Der lange Brief\strikeout{,} von
               Dir, der nicht kommt, ſagt mir viel mehr, als \strikeout{ein}
               einer, der gekom{\pb}men wäre. Du haſt Recht, mein
               lieber Alter; es gibt auch in der Freundſchaft »\label{K_L02669-16v}\edtext{Epiſoden\pwindex{Schnitzler, Arthur 15.05.1862 – 21.10.1931@\textsc{Schnitzler, Arthur} (15.05.1862 – 21.10.1931), \emph{Schriftsteller, Mediziner}!Episode15. 9. 1889@\strich\emph{Episode} {[}15. 9. 1889{]}|pwv}}{\lemma{\textnormal{\emph{Epiſoden}}}\Cendnote{\textnormal{Anspielung auf Schnitzler\pwindex{Schnitzler, Arthur 15.05.1862 – 21.10.1931@\textsc{Schnitzler, Arthur} (15.05.1862 – 21.10.1931), \emph{Schriftsteller, Mediziner}|pwk}s Einakter Episode\pwindex{Schnitzler, Arthur 15.05.1862 – 21.10.1931@\textsc{Schnitzler, Arthur} (15.05.1862 – 21.10.1931), \emph{Schriftsteller, Mediziner}!Episode15. 9. 1889@\strich\emph{Episode} {[}15. 9. 1889{]}|pwkv}}}}\label{K_L02669-16h}«. Jeder verbraucht halt in ſeinem Leben eine gewiſſe Anzahl Menſchen, und von
               mir iſt nur mehr der letzte Bodenſatz vorhanden. Dir iſt kein Vorwurf zu machen. Es
               iſt die Natur, die es ſo eingerichtet, daß das Vergeſſen in der ſeeliſchen Welt genau
               ſo \strikeout{meh} mechaniſch und nothwendig und mit denſelben
               Endzwecken vor ſich geht, wie das Verdauen in der körperlichen{\dotsfour}\pend
           \pstart
           Mir brennt das Gewiſſen oft, wenn ich daran denke, daß ich \textsc{Loris\pwindex{Hofmannsthal, Hugo von 1874-02-01 – 1929-07-15@\textsc{Hofmannsthal, Hugo von} (1874-02-01 – 1929-07-15), \emph{Schriftsteller}|pw}} und \textsc{Richard\pwindex{Beer-Hofmann, Richard 1866-07-11 – 1945-09-26@\textsc{Beer-Hofmann, Richard} (1866-07-11 – 1945-09-26), \emph{Schriftsteller}|pw}} noch nicht auf ihre Brieſe geantwortet habe. Aber mir lähmt der Gedanke die zum
               Schreiben angeſetzte Hand, daß ſie, wenn ſie meinen Brief erhalten, die Empfindung
               haben könnten\substVorne{}\textsuperscript{\textcolor{gray}{,}}\substDazwischen{}:\substHinten{} was will der Menſch eigentlich von mir? Grüße die Zwei\pwindex{Hofmannsthal, Hugo von 1874-02-01 – 1929-07-15@\textsc{Hofmannsthal, Hugo von} (1874-02-01 – 1929-07-15), \emph{Schriftsteller}|pwv}\pwindex{Beer-Hofmann, Richard 1866-07-11 – 1945-09-26@\textsc{Beer-Hofmann, Richard} (1866-07-11 – 1945-09-26), \emph{Schriftsteller}|pwv} bitte viel {\pb}tauſend Mal von mir und ſage ihnen in meinem Namen
               alles Liebe und Gute, was ſich finden läßt{\dots}\pend
           \pstart
           Deinem Bruder\pwindex{Schnitzler, Julius 13.07.1865 – 29.06.1939@\textsc{Schnitzler, Julius} (13.07.1865 – 29.06.1939), \emph{Mediziner}|pwv} und \textsc{Kapper\pwindex{Kapper, Friedrich 21.04.1861 – 22.07.1939@\textsc{Kapper, Friedrich} (21.04.1861 – 22.07.1939), \emph{Mediziner}|pw}} herzlichſte Grüße. Den Deinen ergebene Empfehlungen. Dir ſelbſt – ſchweres
               Problem. Ich möchte Dir am Liebſten meinen Segen geben, ſo abgeſchieden komme ich mir
               Dir gegenüber vor.\pend
           \pstart
           Dein {\\[\baselineskip]}treuer {\\[\baselineskip]}\spacefill\mbox{Paul Goldmann.}\pend
           \leftskip=0em{}\pstart
           \noindent{}Drei Bitten 1.) ſag’ doch dem Schuft\pwindex{Joachim, Jaques 24.11.1866 – 07.11.1925@\textsc{Joachim, Jaques} (24.11.1866 – 07.11.1925), \emph{Wissenschaftler, Rechtsanwalt, Herausgeber}|pwv}, dem \textsc{Dr. Joachim\pwindex{Joachim, Jaques 24.11.1866 – 07.11.1925@\textsc{Joachim, Jaques} (24.11.1866 – 07.11.1925), \emph{Wissenschaftler, Rechtsanwalt, Herausgeber}|pw}}, wenn er die ihm geſchickte kleine Arbeit\pwindex{Schnitzler, Arthur 15.05.1862 – 21.10.1931@\textsc{Schnitzler, Arthur} (15.05.1862 – 21.10.1931), \emph{Schriftsteller, Mediziner}!drei Elixire1893@\strich\emph{Die drei Elixire} {[}1893{]}|pwuv} nicht brauchen kann, ſo ſoll er mir
                  ſie augenblicklich zurückſenden, weil ich Verwendung {\pb}dafür habe; auch ſoll er mir dasjenige \label{K_L02669-17v}\edtext{Heft\pwindex{Moderne Dichtung. Monatsschrift fuer Literatur und Kritik1890-01-01 – 1890-12-31@\emph{Moderne Dichtung. Monatsschrift für Literatur und Kritik} {[}1890-01-01 – 1890-12-31{]}|pwv} der »Modernen \uline{Dichtung}\pwindex{Moderne Dichtung. Monatsschrift fuer Literatur und Kritik1890-01-01 – 1890-12-31@\emph{Moderne Dichtung. Monatsschrift für Literatur und Kritik} {[}1890-01-01 – 1890-12-31{]}|pw}}{\lemma{\textnormal{\emph{Heft … Dichtung}}}\Cendnote{\textnormal{Paul Goldmann\pwindex{Goldmann, Paul 31.01.1865 – 25.09.1935@\textsc{Goldmann, Paul} (31.01.1865 – 25.09.1935), \emph{Schriftsteller, Journalist}|pwk}: \emph{Was einem so einfällt}\pwindex{Goldmann, Paul 31.01.1865 – 25.09.1935@\textsc{Goldmann, Paul} (31.01.1865 – 25.09.1935), \emph{Schriftsteller, Journalist}!Was einem so einfaellt1890-08-01@\strich\emph{Was einem so einfällt} {[}1890-08-01{]}|pwk}. In: \emph{Moderne Dichtung}\pwindex{Moderne Dichtung. Monatsschrift fuer Literatur und Kritik1890-01-01 – 1890-12-31@\emph{Moderne Dichtung. Monatsschrift für Literatur und Kritik} {[}1890-01-01 – 1890-12-31{]}|pwk}, Jg. 1, Bd. 2, H. 1,
                        S. 521–522.}}}\label{K_L02669-17h}« (\label{K_L02669-18v}\edtext{nicht Rundſchau\pwindex{Moderne Rundschau1.4.1891 – 31.12.1891@\emph{Moderne Rundschau} {[}1.4.1891 – 31.12.1891{]}|pw}}{\lemma{\textnormal{\emph{nicht Rundſchau}}}\Cendnote{\textnormal{Paul Goldmann\pwindex{Goldmann, Paul 31.01.1865 – 25.09.1935@\textsc{Goldmann, Paul} (31.01.1865 – 25.09.1935), \emph{Schriftsteller, Journalist}|pwk}: \emph{Nämlich}\pwindex{Goldmann, Paul 31.01.1865 – 25.09.1935@\textsc{Goldmann, Paul} (31.01.1865 – 25.09.1935), \emph{Schriftsteller, Journalist}!Naemlich1891-04-01@\strich\emph{Nämlich} {[}1891-04-01{]}|pwk}. In: \emph{Moderne
                           Rundschau}\pwindex{Moderne Rundschau1.4.1891 – 31.12.1891@\emph{Moderne Rundschau} {[}1.4.1891 – 31.12.1891{]}|pwk}, Jg. 1, Bd. 3, H. 1, 1. 4. 1891,
                     S. 34.}}}\label{K_L02669-18h}) ſchicken, in dem Aphorismen\pwindex{Goldmann, Paul 31.01.1865 – 25.09.1935@\textsc{Goldmann, Paul} (31.01.1865 – 25.09.1935), \emph{Schriftsteller, Journalist}!Naemlich1891-04-01@\strich\emph{Nämlich} {[}1891-04-01{]}|pwv} von mir erſchienen ſind; ich brauche ſie
                  dringend und zahle \strikeout{e\textcolor{gray}{n}} eventuell dem Buchhändler dafür 2.) haſt Du eine Ahnung, was zwiſchen \textsc{\strikeout{Herz}}{ }\textsc{Herzl\pwindex{Herzl, Theodor 1860-05-02 – 1904-07-03@\textsc{Herzl, Theodor} (1860-05-02 – 1904-07-03), \emph{Schriftsteller, Journalist}|pw}} und ſeiner Frau\pwindex{Herzl, Julie 01.02.1868 – 10.11.1907@\textsc{Herzl, Julie} (01.02.1868 – 10.11.1907)|pwv}{ }\label{K_L02669-19v}\edtext{vorgegangen}{\lemma{\textnormal{\emph{vorgegangen}}}\Cendnote{\textnormal{Möglicherweise hörte Goldmann\pwindex{Goldmann, Paul 31.01.1865 – 25.09.1935@\textsc{Goldmann, Paul} (31.01.1865 – 25.09.1935), \emph{Schriftsteller, Journalist}|pwk} von der Ehekrise der Herzl\pwindex{Herzl, Theodor 1860-05-02 – 1904-07-03@\textsc{Herzl, Theodor} (1860-05-02 – 1904-07-03), \emph{Schriftsteller, Journalist}|pwk}\pwindex{Herzl, Julie 01.02.1868 – 10.11.1907@\textsc{Herzl, Julie} (01.02.1868 – 10.11.1907)|pwk}s. Theodor Herzl\pwindex{Herzl, Theodor 1860-05-02 – 1904-07-03@\textsc{Herzl, Theodor} (1860-05-02 – 1904-07-03), \emph{Schriftsteller, Journalist}|pwk} teilte seinem Schwiegervater\pwindex{Naschauer, Jacob 1837-05-20 – 1894-01-01@\textsc{Naschauer, Jacob} (1837-05-20 – 1894-01-01), \emph{Industrieller}|pwkv}{ }Mitte Mai 1891 mit, dass er die Scheidung wolle. Julie Herzl\pwindex{Herzl, Julie 01.02.1868 – 10.11.1907@\textsc{Herzl, Julie} (01.02.1868 – 10.11.1907)|pwk}, mit der Theodor Herzl\pwindex{Herzl, Theodor 1860-05-02 – 1904-07-03@\textsc{Herzl, Theodor} (1860-05-02 – 1904-07-03), \emph{Schriftsteller, Journalist}|pwk} bis zu seinem Tod verheiratet blieb, war
                     zu dieser Zeit schwanger. Vgl. \emph{Theodor Herzl. Briefe und
                        Tagebücher}, herausgegeben von Alex Bein, Hermann Greive, Moshe Schaerf und
                        Julius H. Schoeps. \emph{Bd. 1: Briefe und autobiographische
                           Notizen. 1866–1895}, bearbeitet von Johannes Wachten, in
                        Zusammenarbeit mit Chaya Harel, Daisy Tycho und Manfred Winkler.
                        Berlin/Frankfurt a. M./Wien:
                           \emph{Propyläen}{ }1983, S. 439–443.}}}\label{K_L02669-19h}? 3.) Weißt Du
                  vielleicht – nicht lachen, bitte! – den Namen einer \strikeout{T}{ }\uline{guten}{ }\strikeout{Tr\textcolor{gray}{u}} Truppe Tirol\oindex{Tirol@\textbf{Tirol}|pw}er Sänger, \introOben{}an\introOben{} welche man ſich wenden könnte, um ſie zu einer Reiſe
                  nach Brüſſel\oindex{Bruessel@\textbf{Brüssel}|pw} zu veranlaſſen?\pend
           
         
         \endnumbering\mylabel{h}\end{ledgroupsized}  \newcommand{\dateiname}{L02669}\newcommand{\titel}{Paul Goldmann an Arthur Schnitzler, 27. 10. 1891}\newcommand{\editorInnen}{Martin Anton Müller und Laura Untner}%% latex-leseansicht-abspann.tex
%% Abspann für die Leseansicht.
%% Der Schalter \ifkorrekturansicht ist bereits durch den Vorspann gesetzt.

%% latex-abspann.tex
%% Gemeinsamer Abspann für Korrekturansicht und Leseansicht.
%% Setzt den Schalter \ifkorrekturansicht voraus (gesetzt in den
%% einbindenden Dateien latex-korrekturansicht-abspann.tex bzw.
%% latex-leseansicht-abspann.tex).
%% ---------------------------------------------------------------

\normalsize

% Das esempio-Environment wird nur in der Leseansicht benötigt
\ifkorrekturansicht\else
\newenvironment{esempio}[3]%
{
    \vspace{1.5ex}
    \rlap{\underline{#1}}
    \par
    \setlength{\parindent}{0cm}
    \nopagebreak
    \leftskip=#2cm
    \rightskip=#3cm
}
{
    \par
}
\fi

\doendnotes{C}
\bigskip
\vfill

\clearpage

\footnotesize

\ifkorrekturansicht
  \lohead{\textsc{register}}
\fi

% theindex-Environment neu definieren ohne reledmac
\makeatletter
\renewenvironment{theindex}{%
  \ifkorrekturansicht
    \section*{\indexname}%
  \else
    \subsubsection*{Index der erwähnten Entitäten}%
  \fi
  \setlength{\parindent}{0pt}%
  \setlength{\parskip}{0pt plus 0.3pt}%
  \let\item\@idxitem
}{%
  \ifkorrekturansicht\clearpage\fi
}
\makeatother

\IfFileExists{\jobname-pw.ind}{\input{\jobname-pw.ind}}{}

% Quellenangabe nur in der Leseansicht
\ifkorrekturansicht\else
% Fallback-Definitionen, falls die .tex-Datei \titel etc. nicht gesetzt hat
\providecommand{\titel}{}
\providecommand{\editorInnen}{}
\providecommand{\dateiname}{\jobname}

\vspace{3cm}

\vfill

\footnotesize
\textsc{Quelle}: \titel. Herausgegeben von {\editorInnen}. In: \emph{Arthur Schnitzler: Briefwechsel mit Autorinnen und Autoren}.
 Digitale Edition, https://schnitzler-briefe.acdh.oeaw.ac.at/{\dateiname}.html (Stand \today)
\fi

\end{document}


      