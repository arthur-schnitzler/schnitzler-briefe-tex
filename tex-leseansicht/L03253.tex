%% latex-korrekturansicht-vorspann.tex
%% Vorspann für die Korrekturansicht.
%% Lädt die gemeinsame Datei latex-vorspann.tex mit gesetztem Schalter.

\newif\ifkorrekturansicht
\korrekturansichttrue

\input{../tex-inputs/latex-vorspann}


\section[ Paul Goldmann an Arthur Schnitzler, 7. 10. 1907]{L03253 Paul Goldmann an Arthur Schnitzler, 7. 10. 1907}
\nopagebreak\mylabel{L03253v}
\rehead{ }\normalsize\beginnumbering\briefempfaengerindex{Schnitzler, Arthur@\textsc{Schnitzler, Arthur}!zzzGoldmann, Paul@\emph{von Paul Goldmann}!1907-10-071@{7. 10. 1907}|(be}
\toendnotes[C]{\smallbreak\pagebreak[2]}\Standort{DLA, A:Schnitzler, HS.NZ85.1.3175.}
\physDesc{Postkarte, 348 Zeichen
\newline{}Handschrift: 1) schwarze Tinte, deutsche Kurrent\hspace{1em}2) schwarze Tinte, lateinische Kurrent (\noindent{}Adresse)\hspace{1em}
\newline{}Versand: 1) Stempel: »\nobreak{}\oindex{I., Innere Stadt@\textbf{I., Innere Stadt}, \emph{A.ADM3}|pwk}1/1 Wien 18, 7. X. 07, XII\nobreak{}«.   2) Stempel: »\nobreak{}Wien 110, 7. X. 07, 3\nobreak{}«. 
\newline{}Schnitzler: mit Bleistift das Datum »7. 10. {[}19{]}07« vermerkt }\toendnotes[C]{\smallbreak}\pstart{}{\pb}\textcolor{gray}{\textbf{HOTEL TEGETTHOFF\oindex{Hotel Tegetthoff@\textbf{Hotel Tegetthoff}, \emph{Hotel (K.HTL)}|pw}, I. JOHANNESGASSE 23,
                  WIEN\oindex{Johannesgasse@\textbf{Johannesgasse}, \emph{Straße (K.STR)}|pw}.}}\pend{}{\bigskip}\pstart{}Herrn\pend{}\pstart{}Dr. Arthur Schnitzler\pend{}\pstart{}Wien\oindex{Wien@\textbf{Wien}, \emph{A.ADM2}|pw}\pend{}\pstart{}XVIII. Spöttelgaſse 7\oindex{Edmund-Weiss-Gasse 7@\textbf{Edmund-Weiß-Gasse 7}, \emph{Wohngebäude (K.WHS)}|pw}.\pend{}{\bigskip}\vspace{1em}
\pstart
           {\pb}\textcolor{gray}{\textbf{TELEGRAMME}}\hfill Montag{ }früh.\pend
           
\pstart
           \textcolor{gray}{\textbf{TEGETTHOFFHOTEL\oindex{Hotel Tegetthoff@\textbf{Hotel Tegetthoff}, \emph{Hotel (K.HTL)}|pw}, WIEN\oindex{Wien@\textbf{Wien}, \emph{A.ADM2}|pw}.}}\pend
           \vspace{0.5em}
\pstart
           Lieber Freund, Darf ich vielleicht morgen, Dienſtag, Abend zu Euch\pwindex{Schnitzler, Olga 17.01.1882 – 13.01.1970@\textsc{Schnitzler, Olga} (17.01.1882 – 13.01.1970), \emph{Schauspieler/Schauspielerin, Sänger/Sängerin}|pwv} kommen? Oder wollen wir uns vielleicht
               in der Stadt\oindex{Wien@\textbf{Wien}, \emph{A.ADM2}|pwv} treffen? \textsc{Beer-Hofmann\pwindex{Beer-Hofmann, Richard 1866-07-11 – 1945-09-26@\textsc{Beer-Hofmann, Richard} (1866-07-11 – 1945-09-26), \emph{Schriftsteller/Schriftstellerin}|pw}} u. \textsc{Leo Van-Jung\pwindex{Van-Jung, Leo 15.10.1866 – 02.07.1939@\textsc{Van-Jung, Leo} (15.10.1866 – 02.07.1939), \emph{Gesangspädagoge/Gesangspädagogin, Mathematiker/Mathematikerin}|pw}} würden gern \label{K_L03253-1v}\edtext{mit dabei ſein}{\lemma{\textnormal{\emph{mit dabei ſein}}}\Cendnote{\textnormal{Vgl. A. S.: \emph{Tagebuch}, 8. 10. 1907.
               }}}\label{K_L03253-1}. Ich bitte Dich um Nachricht ins \textsc{Hotel Tegethoff\oindex{Hotel Tegetthoff@\textbf{Hotel Tegetthoff}, \emph{Hotel (K.HTL)}|pw}} u. bin mit herzlichen Grüßen an Deine Frau\pwindex{Schnitzler, Olga 17.01.1882 – 13.01.1970@\textsc{Schnitzler, Olga} (17.01.1882 – 13.01.1970), \emph{Schauspieler/Schauspielerin, Sänger/Sängerin}|pwv} u. Dich\pend
           
\pstart
           Dein {\\[\baselineskip]}\spacefill\mbox{Paul Goldmann.}\pend
           \leftskip=0em{}\selectlanguage{ngerman}\endnumbering\briefempfaengerindex{Schnitzler, Arthur@\textsc{Schnitzler, Arthur}!zzzGoldmann, Paul@\emph{von Paul Goldmann}!1907-10-071@{7. 10. 1907}|)be}\mylabel{L03253h}  \normalsize

\doendnotes{C}
\bigskip
\vfill

\clearpage

\footnotesize

\lohead{\textsc{register}}

% Definiere theindex-Environment komplett neu ohne reledmac
\makeatletter
\renewenvironment{theindex}{%
  \section*{\indexname}%
  \setlength{\parindent}{0pt}%
  \setlength{\parskip}{0pt plus 0.3pt}%
  \let\item\@idxitem
}{%
  \clearpage
}
\makeatother

\IfFileExists{\jobname-pw.ind}{\input{\jobname-pw.ind}}{}

\end{document}

      