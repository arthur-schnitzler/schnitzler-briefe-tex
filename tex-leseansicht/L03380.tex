%% latex-korrekturansicht-vorspann.tex
%% Vorspann für die Korrekturansicht.
%% Lädt die gemeinsame Datei latex-vorspann.tex mit gesetztem Schalter.

\newif\ifkorrekturansicht
\korrekturansichttrue

\input{../tex-inputs/latex-vorspann}


\section[ Paul Goldmann an Arthur Schnitzler, 4. 8. {[}1903{]}]{L03380 Paul Goldmann an Arthur Schnitzler, 4. 8. {[}1903{]}}
\nopagebreak\mylabel{L03380v}
\rehead{ }\normalsize\beginnumbering\briefempfaengerindex{Schnitzler, Arthur@\textsc{Schnitzler, Arthur}!zzzGoldmann, Paul@\emph{von Paul Goldmann}!1903-08-041@{4. 8. {[}1903{]}}|(be}
\toendnotes[C]{\smallbreak\pagebreak[2]}\Standort{DLA, A:Schnitzler, HS.NZ85.1.3173.}
\physDesc{Brief, 1 Blatt, 2 Seiten, 857 Zeichen
\newline{}Handschrift: blaue Tinte, deutsche Kurrent
\newline{}Schnitzler: mit Bleistift das Jahr »903« vermerkt }\toendnotes[C]{\smallbreak}
\pstart
           \centering{}{\pb}Berlin\oindex{Berlin@\textbf{Berlin}, \emph{P.PPLC}|pw}, 4. Auguſt.\pend
           
\pstart{}Mein lieber Freund,\pend\vspace{0.5em}
\pstart
           Danke für Deinen lieben Brief!\pend
           
\pstart
           Ich habe ſchlechte Nachrichten aus Frankfurt\oindex{Frankfurt am Main@\textbf{Frankfurt am Main}, \emph{P.PPLA3}|pw}.
               Vollſtändiger Stimmungsumſchlag. Von einer \label{K_L03380-1v}\edtext{gemeinſamen Reiſe}{\lemma{\textnormal{\emph{gemeinſamen Reiſe}}}\Cendnote{\textnormal{Siehe Paul Goldmann an Arthur Schnitzler, 27. 6. [1903].
               }}}\label{K_L03380-1} keine Rede mehr.\pend
           
\pstart
           Ich bin wieder aus allen Himmeln geſtürzt. Was ich jetzt anfange, weiß ich nicht. Mit
               Dir will ich nicht reiſen, denn ich würde zu ſehr auf Deine Stimmung drücken. Mag
               auch keine ſchönen Länder ſehen. Vielleicht gehe ich nach Marienbad\oindex{Marienbad@\textbf{Marienbad}, \emph{P.PPL}|pw} zur Kur.\pend
           
\pstart
           An dieſer Geſchichte gehe ich wohl noch zu Grunde. Jede Schuld wird beſtraft. Ich
               hatte eine prachtvolle Frau\pwindex{Rottenberg, Theodore 1875-09-07 – 1945-04-05@\textsc{Rottenberg, Theodore} (1875-09-07 – 1945-04-05)|pwv},
               die {\pb}mich liebte. In meinem Wahn hielt ich ſie\pwindex{Rottenberg, Theodore 1875-09-07 – 1945-04-05@\textsc{Rottenberg, Theodore} (1875-09-07 – 1945-04-05)|pwv} für eine Dirne und trat
               ſie mit Füßen. Die Liebe iſt todt, und ich kann ſie nicht mehr erwecken. Zu ſpät bin
               ich zur Erkenntniß gekommen. Ein furchtbarer Schickſalsſpruch, dieſes: zu ſpät.\pend
           
\pstart
           Leb’ wohl, liebſter Freund, und reiſe glücklich! {\\[\baselineskip]}Dein treuer {\\[\baselineskip]}\spacefill\mbox{Paul Goldm}\pend
           \leftskip=0em{}
\pstart
           \noindent{}Viele Grüße an \textsc{Olga\pwindex{Schnitzler, Olga 17.01.1882 – 13.01.1970@\textsc{Schnitzler, Olga} (17.01.1882 – 13.01.1970), \emph{Schauspieler/Schauspielerin, Sänger/Sängerin}|pw}}!\pend
           \selectlanguage{ngerman}\endnumbering\briefempfaengerindex{Schnitzler, Arthur@\textsc{Schnitzler, Arthur}!zzzGoldmann, Paul@\emph{von Paul Goldmann}!1903-08-041@{4. 8. {[}1903{]}}|)be}\mylabel{L03380h}  \normalsize

\doendnotes{C}
\bigskip
\vfill

\clearpage

\footnotesize

\lohead{\textsc{register}}

% Definiere theindex-Environment komplett neu ohne reledmac
\makeatletter
\renewenvironment{theindex}{%
  \section*{\indexname}%
  \setlength{\parindent}{0pt}%
  \setlength{\parskip}{0pt plus 0.3pt}%
  \let\item\@idxitem
}{%
  \clearpage
}
\makeatother

\IfFileExists{\jobname-pw.ind}{\input{\jobname-pw.ind}}{}

\end{document}

      