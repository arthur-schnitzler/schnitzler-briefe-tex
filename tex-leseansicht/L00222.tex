\input{../tex-inputs/latex-pdf-vorspann}
\begin{center}
            \textcolor{red}{ENTWURF. ENTZIFFERUNG NOCH NICHT KORREKTURGELESEN}
                      \end{center}
            
               \section[Arthur Schnitzler an Wilhelm Bölsche, 14. 6. 1893]{ Arthur Schnitzler an Wilhelm Bölsche, 14. 6. 1893}\nopagebreak\mylabel{v}\rehead{ }\begin{ledgroupsized}[t]{13cm}\normalsize\beginnumbering\briefempfaengerindex{Boelsche, Wilhelm@\textsc{Bölsche, Wilhelm}!zzzSchnitzler, Arthur@\emph{von Arthur Schnitzler}!1893-06-141@{14. 6. 1893}|(be} \toendnotes[C]{\smallbreak\pagebreak[2]} \Standort{Wrocław, Biblioteka Uniwersytecka, Böl.Pis 1769.}
\physDesc{Brief, 1 Blatt (Briefpapier mit Trauerrand), 4 Seiten
\newline{}Handschrift: schwarze Tinte, deutsche Kurrent
\newline{}Bölsche: als »Erl{[}edigt{]}« gezeichnet }\buchAbdrucke{\weitereDrucke{1) Alois Woldan: \emph{Arthur Schnitzler – Briefe an Wilhelm Bölsche.} In: \emph{Germanica Wratislaviensia} (1987) Nr. 77, S. 463.} \weitereDrucke{2) Wilhelm Bölsche: \emph{Briefwechsel. Mit Autoren der Freien Bühne}. Hg. Gerd-Hermann Susen. Berlin: \emph{Weidler} 2010, S. 689 (Werke und Briefe. Wissenschaftliche Ausgabe, Briefe I).} }\toendnotes[C]{\smallbreak}\pstart
           {\pb}\uline{14. 6. 93.}\hfill \textsc{I. Grillparzerstr 7}\oindex{Grillparzerstrasse@\textbf{Grillparzerstraße}|pw}.\pend
           \pstart{}Verehrteſter Herr Doktor,\pend\pstart
           beſten Dank für die Erledigung meiner Einſendung. Leider aber haben Sie mir meine
                    andern Fragen wieder nicht beantwortet, und ich erſuche Sie neuerlich, mir
                    gütigſt mitthei{\pb}len zu wollen, ob Sie mein \uline{dreiaktiges Schauſpiel}, \uuline{Das Mährchen\pwindex{Schnitzler, Arthur 15.05.1862 – 21.10.1931@\textsc{Schnitzler, Arthur} (15.05.1862 – 21.10.1931), \emph{Schriftsteller, Mediziner}!Maerchen. Schauspiel in drei Aufzuegen1891 – 1891@\strich\emph{Das Märchen. Schauspiel in drei Aufzügen} {[}1891 – 1891{]}|pw}}, welches in der nächſten Saiſon am Leſſingtheater\orgindex{Lessing-Theater@Lessing-Theater|pw} zur Aufführung ko{\geminationm}t, im
                    Laufe dieſes So{\geminationm}ers veröffentlichen wollen. Ich war
                    ſo frei, Ihnen vor etwa {\pb}1 Jahr ein Exemplar desſelben
                    zu ſenden; wollen Sie das Stück\pwindex{Schnitzler, Arthur 15.05.1862 – 21.10.1931@\textsc{Schnitzler, Arthur} (15.05.1862 – 21.10.1931), \emph{Schriftsteller, Mediziner}!Maerchen. Schauspiel in drei Aufzuegen1891 – 1891@\strich\emph{Das Märchen. Schauspiel in drei Aufzügen} {[}1891 – 1891{]}|pwv} bringen, ſo erhalten Sie ſofort ein neues Exemplar
                    zugeſchickt.\pend
           \pstart
           Mir wäre eine Veröffentlichung in der Fr. Bühne\pwindex{Freie Buehne fuer den Entwickelungskampf der Zeit1892 – 1893@\emph{Freie Bühne für den Entwickelungskampf der Zeit}|pw}{ }ſehr werthvoll, und ich glaube, daſs das
                    Schauſpiel Ihren Leſerkreis intereſſiren {\pb}würde. –
                    Aber freilich müßte das Stück von Juli an erſcheinen. –\pend
           \pstart
           Ich hoffe, verehrteſter Herr Doktor, daſs ſich unſere Intereſſen in dieſem Fall
                    begegnen werden und ſehe Ihrer baldigen Antwort entgegen.\pend
           \pstart
           In aufrichtiger Hochachtung{\\[\baselineskip]}\spacefill\mbox{Arth Schnitzler}\pend
           \leftskip=0em{}\endnumbering\briefempfaengerindex{Boelsche, Wilhelm@\textsc{Bölsche, Wilhelm}!zzzSchnitzler, Arthur@\emph{von Arthur Schnitzler}!1893-06-141@{14. 6. 1893}|)be}\mylabel{h}\end{ledgroupsized}  \newcommand{\dateiname}{L00222}\newcommand{\titel}{Arthur Schnitzler an Wilhelm Bölsche, 14. 6. 1893}\newcommand{\editorInnen}{Martin Anton Müller und Gerd-Hermann Susen}\input{../tex-inputs/latex-pdf-abspann}
      