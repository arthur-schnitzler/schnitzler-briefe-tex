%% latex-korrekturansicht-vorspann.tex
%% Vorspann für die Korrekturansicht.
%% Lädt die gemeinsame Datei latex-vorspann.tex mit gesetztem Schalter.

\newif\ifkorrekturansicht
\korrekturansichttrue

\input{../tex-inputs/latex-vorspann}


\section[Arthur Schnitzler an Wilhelm Bölsche, 14. 6. 1893]{L00222 Arthur Schnitzler an Wilhelm Bölsche, 14. 6. 1893}
\nopagebreak\mylabel{L00222v}
\rehead{ }\normalsize\beginnumbering\briefempfaengerindex{Boelsche, Wilhelm@\textsc{Bölsche, Wilhelm}!zzzSchnitzler, Arthur@\emph{von Arthur Schnitzler}!1893-06-141@{14. 6. 1893}|(be}
\toendnotes[C]{\smallbreak\pagebreak[2]}\Standort{Wrocław, Biblioteka Uniwersytecka, Böl.Pis 1769.}
\physDesc{Brief, 1 Blatt, 4 Seiten, 930 Zeichen (Briefpapier mit Trauerrand)
\newline{}Handschrift: schwarze Tinte, deutsche Kurrent
\newline{}Bölsche: als »Erl{[}edigt{]}« gezeichnet }
\buchAbdrucke{\weitereDrucke{1) \emph{Germanica Wratislaviensia} (1987) Nr. 77, S. 463.} \weitereDrucke{2) Wilhelm Bölsche: \emph{Briefwechsel. Mit Autoren der Freien Bühne}. Berlin: \emph{Weidler} 2010, S. 689.} }\toendnotes[C]{\smallbreak}
\pstart
           
\pstart
           {\pb}\uline{14. 6. 93.}\pend
           
\pstart
           \raggedleft{}\textsc{I. Grillparzerstr 7}\oindex{Grillparzerstrasse@\textbf{Grillparzerstraße}, \emph{R.ST}|pw}.\pend
           \pend
           
\pstart{}Verehrteſter Herr Doktor,\pend\vspace{0.5em}
\pstart
           beſten Dank für die Erledigung meiner Einſendung. Leider aber haben Sie mir meine
               andern Fragen wieder nicht beantwortet, und ich erſuche Sie neuerlich, mir gütigſt
                  mitthei{\pb}len zu wollen, ob Sie mein \uline{dreiaktiges Schauſpiel}, \uuline{Das Mährchen\pwindex{Maerchen. Schauspiel in drei Aufzuegen@\emph{Das Märchen. Schauspiel in drei Aufzügen}|pw}}, welches in der nächſten Saiſon am Leſſingtheater\orgindex{Lessing-Theater@Lessing-Theater|pw} zur Aufführung ko{\geminationm}t, im Laufe
               dieſes So{\geminationm}ers veröffentlichen wollen. Ich war{ }ſo frei,
               Ihnen vor etwa {\pb}1 Jahr ein Exemplar desſelben zu{ }ſenden;
               wollen Sie das Stück\pwindex{Maerchen. Schauspiel in drei Aufzuegen@\emph{Das Märchen. Schauspiel in drei Aufzügen}|pwv} bringen,{ }ſo erhalten Sie{ }ſofort ein neues Exemplar zugeſchickt.\pend
           
\pstart
           Mir wäre eine Veröffentlichung in der Fr. Bühne\pwindex{Freie Buehne fuer den Entwickelungskampf der Zeit@\emph{Freie Bühne für den Entwickelungskampf der Zeit}|pw}{ }ſehr werthvoll, und ich glaube, daſs das Schauſpiel
               Ihren Leſerkreis intereſſiren {\pb}würde. – Aber freilich müßte
               das Stück von Juli an erſcheinen. –\pend
           
\pstart
           Ich hoffe, verehrteſter Herr Doktor, daſs{ }ſich unſere Intereſſen in dieſem Fall
               begegnen werden und{ }ſehe Ihrer baldigen Antwort entgegen.\pend
           
\pstart
           In aufrichtiger Hochachtung{\\[\baselineskip]}\spacefill\mbox{Arth Schnitzler}\pend
           \leftskip=0em{}\selectlanguage{ngerman}\endnumbering\briefempfaengerindex{Boelsche, Wilhelm@\textsc{Bölsche, Wilhelm}!zzzSchnitzler, Arthur@\emph{von Arthur Schnitzler}!1893-06-141@{14. 6. 1893}|)be}\mylabel{L00222h}  \normalsize

\doendnotes{C}
\bigskip
\vfill

\clearpage

\footnotesize

\lohead{\textsc{register}}

% Definiere theindex-Environment komplett neu ohne reledmac
\makeatletter
\renewenvironment{theindex}{%
  \section*{\indexname}%
  \setlength{\parindent}{0pt}%
  \setlength{\parskip}{0pt plus 0.3pt}%
  \let\item\@idxitem
}{%
  \clearpage
}
\makeatother

\IfFileExists{\jobname-pw.ind}{\input{\jobname-pw.ind}}{}

\end{document}

      