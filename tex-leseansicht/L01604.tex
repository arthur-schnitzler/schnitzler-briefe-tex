%% latex-korrekturansicht-vorspann.tex
%% Vorspann für die Korrekturansicht.
%% Lädt die gemeinsame Datei latex-vorspann.tex mit gesetztem Schalter.

\newif\ifkorrekturansicht
\korrekturansichttrue

\input{../tex-inputs/latex-vorspann}


\section[Arthur Schnitzler an Hermann Bahr, 24. – 25. 6. 1906]{L01604 Arthur Schnitzler an Hermann Bahr, 24. – 25. 6. 1906}
\nopagebreak\mylabel{L01604v}
\rehead{ }\normalsize\beginnumbering\briefempfaengerindex{Bahr, Hermann@\textsc{Bahr, Hermann}!zzzSchnitzler, Arthur@\emph{von Arthur Schnitzler}!1906-06-252@{24. – 25. 6. 1906}|(be}
\toendnotes[C]{\smallbreak\pagebreak[2]}\Standort{TMW, HS AM 23379 Ba.}
\physDesc{Brief, 2 Blätter, 7 Seiten, 2645 Zeichen
\newline{}Handschrift: schwarze Tinte, deutsche Kurrent
\newline{}Ordnung: Lochung }
\buchAbdrucke{\weitereDrucke{1) Arthur Schnitzler: \emph{Briefe 1875–1912}. Frankfurt am Main: \emph{S. Fischer} 1981, S. 537–538.} \weitereDrucke{2) Arthur Schnitzler: \emph{The Letters of Arthur Schnitzler to Hermann Bahr}. Chapel Hill: \emph{The University of North Carolina Press} 1978, S. 94–95.} \weitereDrucke{3) Hermann Bahr, Arthur Schnitzler: \emph{Briefwechsel, Aufzeichnungen, Dokumente (1891–1931)}. Göttingen: \emph{Wallstein} 2018, S. 379–380.} }\toendnotes[C]{\smallbreak}
\pstart
           \raggedleft{}{\pb}Wien\oindex{Wien@\textbf{Wien}, \emph{A.ADM2}|pw}, 24. 6. 90\textcolor{gray}{6}\pend
           
\pstart{}lieber Hermann, \pend\vspace{0.5em}
\pstart
           ich finde deinen neuen Einakter\pwindex{Faun. Ein Akt@\emph{Der Faun. Ein Akt}|pwv}{ }ſehr intereſſant; feſſelnd vom e\damage{rſt}en bis zum letzten Wort, und halte (we{\geminationn} es
               nicht zu einem Skandal kommt, was man bei Bahren und Faunen nie wiſſen kann) auch
               eine ſtarke Bühnenwirkung für wahrſcheinlich. (Deine 3 Einakter müſſten zuſammen
               gegeben werden; Faun\pwindex{Faun. Ein Akt@\emph{Der Faun. Ein Akt}|pw} zum Schluſs, Narr\pwindex{arme Narr. Lustspiel in einem Akt@\emph{Der arme Narr. Lustspiel in einem Akt}|pw} zu Anfang, das »du kannst ja mitkommen«, {\pb}der Helmine am Schluſs
               bekäme dann ſeine beſondre Bedeutung.)\pend
           
\pstart
           Man denkt natürlich ſo ein Stück weiter, wie man wirkliche Erlebniſſe weiter
               phantaſirt, und ſo habe ich auch einen zwei\damage{ten} u dritten Akt geſehen, die man vorläufig nicht wird ſpielen können. Der
               zweite Akt auf der ſteilen Bergwieſe. Falls du ihn ſchreiben ſollteſt, rathe ich dir,
               ihn nicht von Leſſing\pwindex{Lessing, Emil 06.05.1857 – 01.11.1921@\textsc{Lessing, Emil} (06.05.1857 – 01.11.1921), \emph{Regisseur/Regisseurin, Schauspieler/Schauspielerin}|pw} inſzeniren zu laſſen,
               der Orgien nur ein mäßiges Verſtändnis entgegenbringt, was {\pb}ſich im 4. Akt der \textsc{Beatrice}\pwindex{Schleier der Beatrice. Schauspiel in fuenf Akten@\emph{Der Schleier der Beatrice. Schauspiel in fünf Akten}|pw}{ }\label{K_L01604-1v}\edtext{ja{\geminationm}ervoll
                  erwieſen}{\lemma{\textnormal{\emph{jammervoll
                  erwieſen}}}\Cendnote{\textnormal{Die Anmerkung bezieht sich
                     auf die Inszenierung am \emph{Deutschen Theater in
                     Berlin}\orgindex{Deutsches Theater Berlin@Deutsches Theater Berlin|pwk}, die am 7. 3. 1903 Premiere hatte.}}}\label{K_L01604-1}. Dieſer
               zweite Akt, der verſchiedentlich geführt werden könnte bekäme ſeinen ganzen Sinn
               natürlich nur durch die vollendeſte Rückſichtsloſigkeit. Alſo Bedingung:
               Unaufführbarkeit. Da für mich (wenigſtens wie ich das Stück weitergedacht habe) \textsc{Helmine\pwindex{Faun. Ein Akt@\emph{Der Faun. Ein Akt}|pwv}} die Heldin iſt, brächte der 3. Akt den ſeelischen Untergang oder Sieg der \textsc{Helmine\pwindex{Faun. Ein Akt@\emph{Der Faun. Ein Akt}|pwv}}. Man wird zu irgend etwas wahrſcheinlich nur reif, wenn man eigentlich dazu
               geboren war. Man kann ein Faun ſein; man ka{\geminationn}{ }{\pb}aber kein Faun werden.
               Man kann ein Hexchen und eine Nymphe ſein, aber man ka{\geminationn}
               es nicht werden. Ich bin nicht klar darüber, ob Helmine\pwindex{Faun. Ein Akt@\emph{Der Faun. Ein Akt}|pwv} das Recht auf die Welt gebracht hat, auf die
                  ſtei\damage{le} Bergwieſe zu wandern. Jedenfalls ſie eher als Edgar, wie ja die Frauen
               überhaupt mit den Urelementen verwandter ſind als die Männer. Es wäre auch zu
               bedenken, ob \textsc{Helmine} nicht irgend was, das man nur aus {\pb}ſeiner Natur heraus
               thun darf, \label{K_L01604-2v}\edtext{\textsc{par dépit}}{\lemma{\textnormal{\emph{par dépit}}}\Cendnote{\textnormal{französisch: aus Neid}}}\label{K_L01604-2} thut – was
               vielleicht eine der häufigſten tragiſchen Verſchuldungen bedeutet. Eine andere, eher
               komoediſche Verſchuldung hinwiederum: jemand denkt auf dem Wege der \introOben{}Höher-\introOben{}Entwicklung irgendwohin gelangt \strikeout{ſei} zu ſein – und iſt nur \label{K_L01604-3v}\edtext{ataviſtiſch}{\lemma{\textnormal{\emph{ataviſtiſch}}}\Cendnote{\textnormal{neuerlich auftretende Eigenschaften früherer Generationen, die durch die
                  Entwicklung unnötig geworden sind und für überwunden gelten}}}\label{K_L01604-3} hingerathen.
               Auch auf den ſteilen Bergwieſen tanzen zumeiſt Leute, die nicht hin gehören. Dahin
               ungefähr führte mich dein fauniſch-tiefſinnig-burleskes Stückchen\pwindex{Faun. Ein Akt@\emph{Der Faun. Ein Akt}|pwv}, und ſo möchte es wahrſcheinlich
               damit {\pb}enden, daſs
               irgend welche nicht bergwieſenwürdige Geſchöpfe vom wahren Faun zu Thale geprügelt
               würden. –\pend
           \selectlanguage{ngerman}\vspace{1em}
\pstart
           \noindent{}– Heute, \introOben{}den 25.\introOben{} mein lieber Hermann, reiſen wir ab. Nach
                  Berlin\oindex{Berlin@\textbf{Berlin}, \emph{P.PPLC}|pw}. (1, 2 Tage) Kopenhagen\oindex{Kopenhagen@\textbf{Kopenhagen}, \emph{P.PPLC}|pw} (3, 4 Tage.) \textsc{Marienlyst}\oindex{Marienlyst@\textbf{Marienlyst}, \emph{S.EST}|pw}. Ein paar Wochen. Dann, Auguſt vielleicht noch irgendwohin an die Nordſee\oindex{Nordsee@\textbf{Nordsee}, \emph{H.SEA}|pw}. (\label{K_L01604-4v}\edtext{Nordvyk\oindex{Noordwijk@\textbf{Noordwijk}, \emph{A.ADM2}|pw}}{\lemma{\textnormal{\emph{Nordvyk}}}\Cendnote{\textnormal{Vgl. Felix Salten u. a. an Arthur Schnitzler, 19. 4. 1906.
               }}}\label{K_L01604-4}?). Laſs uns jedenfalls in brieflich-anſichtskartlicher Verbindung
               bleiben. – \pend
           
\pstart
           Mit guten Sommerwünſchen und {\pb}Grüßen von Olga\pwindex{Schnitzler, Olga 17.01.1882 – 13.01.1970@\textsc{Schnitzler, Olga} (17.01.1882 – 13.01.1970), \emph{Schauspieler/Schauspielerin, Sänger/Sängerin}|pw} u mir{\\[\baselineskip]}herzlichſt der Deine{\\[\baselineskip]}\spacefill\mbox{Arthur}\pend
           \leftskip=0em{}\selectlanguage{ngerman}\vspace{1em}
\pstart
           \noindent{}Das \textsc{Mscrpt\pwindex{Faun. Ein Akt@\emph{Der Faun. Ein Akt}|pwv}} ist an \textsc{Salten\pwindex{Salten, Felix 06.09.1869 – 08.10.1945@\textsc{Salten, Felix} (06.09.1869 – 08.10.1945), \emph{Schriftsteller/Schriftstellerin, Journalist/Journalistin, Chefredakteur/Chefredakteurin}|pw}} abgesandt.\pend
           \selectlanguage{ngerman}\endnumbering\briefempfaengerindex{Bahr, Hermann@\textsc{Bahr, Hermann}!zzzSchnitzler, Arthur@\emph{von Arthur Schnitzler}!2@{24. – 25. 6. 1906}|)be}\mylabel{L01604h}  \normalsize

\doendnotes{C}
\bigskip
\vfill

\clearpage

\footnotesize

\lohead{\textsc{register}}

% Definiere theindex-Environment komplett neu ohne reledmac
\makeatletter
\renewenvironment{theindex}{%
  \section*{\indexname}%
  \setlength{\parindent}{0pt}%
  \setlength{\parskip}{0pt plus 0.3pt}%
  \let\item\@idxitem
}{%
  \clearpage
}
\makeatother

\IfFileExists{\jobname-pw.ind}{\input{\jobname-pw.ind}}{}

\end{document}

      