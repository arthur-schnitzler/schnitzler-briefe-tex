%% latex-leseansicht-vorspann.tex
%% Vorspann für die Leseansicht.
%% Lädt die gemeinsame Datei latex-vorspann.tex mit nicht gesetztem Schalter.

\newif\ifkorrekturansicht
\korrekturansichtfalse

\input{../tex-inputs/latex-vorspann}


         
         \newcommand{\erwaehntePersonen}{Personen: Hermann Bahr, Emil Lessing, Felix Salten, Olga Schnitzler}
         \newcommand{\erwaehnteInstitutionen}{}
         \newcommand{\erwaehnteOrte}{Orte: Berlin, Deutsches Theater Berlin, Graz, Kopenhagen, Marienlyst, Wien}
         \newcommand{\erwaehnteWerke}{Werke: Der Faun. Ein Akt, Der Schleier der Beatrice. Schauspiel in fünf Akten, Der arme Narr. Lustspiel in einem Akt}
               \section[Arthur Schnitzler an Hermann Bahr, 24.–25. 6. 1906]{ Arthur Schnitzler an Hermann Bahr, 24.–25. 6. 1906}\nopagebreak\mylabel{v}\rehead{ }\begin{ledgroupsized}[t]{13cm}\normalsize\beginnumbering \toendnotes[C]{\smallbreak\pagebreak[2]} \Standort{TMW, HS AM 23379 Ba.}
\physDesc{Brief, 2 Blätter, 7 Seiten
\newline{}Handschrift: schwarze Tinte, deutsche Kurrent\newline{}Ordnung: Lochung }\buchAbdrucke{\weitereDrucke{1) Arthur Schnitzler: \emph{Briefe 1875–1912}. Hg. Therese Nickl und Heinrich Schnitzler. Frankfurt am Main: \emph{S. Fischer} 1981, S. 537–538.} \weitereDrucke{2) \emph{24. 6. 1906.} In: Arthur Schnitzler: \emph{The Letters of Arthur Schnitzler to Hermann Bahr}. Edited, annotated, and with an introduction, by Donald G.
                        Daviau. Chapel Hill: \emph{The University of North Carolina Press} 1978, S. 94–95 (University of North Carolina studies in the Germanic languages
                        and literatures, 89).} \weitereDrucke{3) Hermann Bahr, Arthur Schnitzler: \emph{Briefwechsel, Aufzeichnungen, Dokumente (1891–1931)}. Hg. Kurt Ifkovits und Martin Anton Müller. Göttingen: \emph{Wallstein} 2018, S. 379–380.} }\toendnotes[C]{\smallbreak}\pstart
           \raggedleft{}{\pb}Wien\oindex{Wien@\textbf{Wien}|pw}, 24. 6. 90\textcolor{gray}{6}\pend
           \pstart{}lieber Hermann, \pend\pstart
           ich finde deinen neuen Einakter\pwindex{Bahr, Hermann 19.07.1863 – 15.01.1934@\textsc{Bahr, Hermann} (19.07.1863 – 15.01.1934), \emph{Schriftsteller, Kritiker}!Faun. Ein Akt1906-10-04 – 1906-11-08@\strich\emph{Der Faun. Ein Akt} {[}1906-10-04 – 1906-11-08{]}|pwv}{ }ſehr intereſſant; feſſelnd vom e\damage{rſt}en bis zum letzten Wort, und halte (we{\geminationn} es
               nicht zu einem Skandal kommt, was man bei Bahren und Faunen nie wiſſen kann) auch
               eine ſtarke Bühnenwirkung für wahrſcheinlich. (Deine 3 Einakter müſſten zuſammen
               gegeben werden; Faun\pwindex{Bahr, Hermann 19.07.1863 – 15.01.1934@\textsc{Bahr, Hermann} (19.07.1863 – 15.01.1934), \emph{Schriftsteller, Kritiker}!Faun. Ein Akt1906-10-04 – 1906-11-08@\strich\emph{Der Faun. Ein Akt} {[}1906-10-04 – 1906-11-08{]}|pw} zum Schluſs, Narr\pwindex{Bahr, Hermann 19.07.1863 – 15.01.1934@\textsc{Bahr, Hermann} (19.07.1863 – 15.01.1934), \emph{Schriftsteller, Kritiker}!arme Narr. Lustspiel in einem Akt28.09.1905 – 12.10.1905,@\strich\emph{Der arme Narr. Lustspiel in einem Akt} {[}28.09.1905 – 12.10.1905,{]}|pw}\pwindex{Bahr, Hermann 19.07.1863 – 15.01.1934@\textsc{Bahr, Hermann} (19.07.1863 – 15.01.1934), \emph{Schriftsteller, Kritiker}!arme Narr. Lustspiel in einem Akt28.09.1905 – 12.10.1905,@\strich\emph{Der arme Narr. Lustspiel in einem Akt} {[}28.09.1905 – 12.10.1905,{]}|pw}\pwindex{Bahr, Hermann 19.07.1863 – 15.01.1934@\textsc{Bahr, Hermann} (19.07.1863 – 15.01.1934), \emph{Schriftsteller, Kritiker}!arme Narr. Lustspiel in einem Akt28.09.1905 – 12.10.1905,@\strich\emph{Der arme Narr. Lustspiel in einem Akt} {[}28.09.1905 – 12.10.1905,{]}|pw} zu Anfang, das »du kannst ja mitkommen«, {\pb}der Helmine am Schluſs
               bekäme dann ſeine beſondre Bedeutung.)\pend
           \pstart
           Man denkt natürlich ſo ein Stück weiter, wie man wirkliche Erlebniſſe weiter
               phantaſirt, und ſo habe ich auch einen zwei\damage{ten} u dritten Akt geſehen, die man vorläufig nicht wird ſpielen können. Der
               zweite Akt auf der ſteilen Bergwieſe. Falls du ihn ſchreiben ſollteſt, rathe ich dir,
               ihn nicht von Leſſing\pwindex{Lessing, Emil 06.05.1857 – 01.11.1921@\textsc{Lessing, Emil} (06.05.1857 – 01.11.1921), \emph{Regisseur, Schauspieler}|pw} inſzeniren zu laſſen, der
               Orgien nur ein mäßiges Verſtändnis entgegenbringt, was {\pb}ſich im 4. Akt der \textsc{Beatrice}\pwindex{Schnitzler, Arthur 15.05.1862 – 21.10.1931@\textsc{Schnitzler, Arthur} (15.05.1862 – 21.10.1931), \emph{Schriftsteller, Mediziner}!Schleier der Beatrice. Schauspiel in fuenf Akten1900-12-01@\strich\emph{Der Schleier der Beatrice. Schauspiel in fünf Akten} {[}1900-12-01{]}|pw}{ }\label{K_L01604_1v}\edtext{ja{\geminationm}ervoll
                  erwieſen}{\lemma{\textnormal{\emph{jammervoll
                  erwieſen}}}\Cendnote{\textnormal{Die Anmerkung bezieht sich
                  auf die Inszenierung am Deutschen Theater in
                  Berlin\oindex{Deutsches Theater Berlin@\textbf{Deutsches Theater Berlin}|pwk}, die am 7. 3. 1903 Premiere hatte.}}}\label{K_L01604_1h}. Dieſer zweite
               Akt, der verſchiedentlich geführt werden könnte bekäme ſeinen ganzen Sinn natürlich
               nur durch die vollendeſte Rückſichtsloſigkeit. Alſo Bedingung: Unaufführbarkeit. Da
               für mich (wenigſtens wie ich das Stück weitergedacht habe) \textsc{Helmine\pwindex{Bahr, Hermann 19.07.1863 – 15.01.1934@\textsc{Bahr, Hermann} (19.07.1863 – 15.01.1934), \emph{Schriftsteller, Kritiker}!Faun. Ein Akt1906-10-04 – 1906-11-08@\strich\emph{Der Faun. Ein Akt} {[}1906-10-04 – 1906-11-08{]}|pwv}} die Heldin iſt, brächte der 3. Akt den ſeelischen Untergang oder Sieg der \textsc{Helmine\pwindex{Bahr, Hermann 19.07.1863 – 15.01.1934@\textsc{Bahr, Hermann} (19.07.1863 – 15.01.1934), \emph{Schriftsteller, Kritiker}!Faun. Ein Akt1906-10-04 – 1906-11-08@\strich\emph{Der Faun. Ein Akt} {[}1906-10-04 – 1906-11-08{]}|pwv}}. Man wird zu irgend etwas wahrſcheinlich nur reif, wenn man eigentlich dazu
               geboren war. Man kann ein Faun ſein; man ka{\geminationn}{ }{\pb}aber kein Faun werden.
               Man kann ein Hexchen und eine Nymphe ſein, aber man ka{\geminationn}
               es nicht werden. Ich bin nicht klar darüber, ob Helmine\pwindex{Bahr, Hermann 19.07.1863 – 15.01.1934@\textsc{Bahr, Hermann} (19.07.1863 – 15.01.1934), \emph{Schriftsteller, Kritiker}!Faun. Ein Akt1906-10-04 – 1906-11-08@\strich\emph{Der Faun. Ein Akt} {[}1906-10-04 – 1906-11-08{]}|pwv} das Recht auf die Welt gebracht hat, auf die
                  ſtei\damage{le} Bergwieſe zu wandern. Jedenfalls ſie eher als Edgar, wie ja die Frauen
               überhaupt mit den Urelementen verwandter ſind als die Männer. Es wäre auch zu
               bedenken, ob \textsc{Helmine} nicht irgend was, das man nur aus {\pb}ſeiner Natur heraus
               thun darf, \label{K_L01604_2v}\edtext{\textsc{par dépit}}{\lemma{\textnormal{\emph{par dépit}}}\Cendnote{\textnormal{französisch: aus Neid}}}\label{K_L01604_2h} thut – was
               vielleicht eine der häufigſten tragiſchen Verſchuldungen bedeutet. Eine andere, eher
               komoediſche Verſchuldung hinwiederum: jemand denkt auf dem Wege der \introOben{}Höher-\introOben{}Entwicklung irgendwohin gelangt \strikeout{ſei} zu ſein – und iſt nur \label{K_L01604_3v}\edtext{ataviſtiſch}{\lemma{\textnormal{\emph{ataviſtiſch}}}\Cendnote{\textnormal{neuerlich auftretende Eigenschaften früherer Generationen, die durch die
                  Entwicklung unnötig geworden sind und für überwunden gelten}}}\label{K_L01604_3h} hingerathen.
               Auch auf den ſteilen Bergwieſen tanzen zumeiſt Leute, die nicht hin gehören. Dahin
               ungefähr führte mich dein fauniſch-tiefſinnig-burleskes Stückchen\pwindex{Bahr, Hermann 19.07.1863 – 15.01.1934@\textsc{Bahr, Hermann} (19.07.1863 – 15.01.1934), \emph{Schriftsteller, Kritiker}!Faun. Ein Akt1906-10-04 – 1906-11-08@\strich\emph{Der Faun. Ein Akt} {[}1906-10-04 – 1906-11-08{]}|pwv}, und ſo möchte es wahrſcheinlich damit {\pb}enden, daſs irgend
               welche nicht bergwieſenwürdige Geſchöpfe vom wahren Faun zu Thale geprügelt
               würden. –\pend
           \pstart
           \noindent{}– Heute, \introOben{}den 25.\introOben{} mein lieber Hermann, reiſen wir ab. Nach
                  Berlin\oindex{Berlin@\textbf{Berlin}|pw}. (1, 2 Tage) Kopenhagen\oindex{Kopenhagen@\textbf{Kopenhagen}|pw} (3, 4 Tage.) \textsc{Marienlyst}\oindex{Marienlyst@\textbf{Marienlyst}|pw}. Ein paar Wochen. Dann, Auguſt vielleicht noch irgendwohin an die Nordſee. (Nordyk\oindex{Graz@\textbf{Graz}|pw}?). Laſs
               uns jedenfalls in brieflich-anſichtskartlicher Verbindung bleiben. – \pend
           \pstart
           Mit guten Sommerwünſchen und {\pb}Grüßen von Olga\pwindex{Schnitzler, Olga 17.01.1882 – 13.01.1970@\textsc{Schnitzler, Olga} (17.01.1882 – 13.01.1970), \emph{Schauspielerin, Sängerin}|pw} u mir{\\[\baselineskip]}herzlichſt der Deine{\\[\baselineskip]}\spacefill\mbox{Arthur}\pend
           \leftskip=0em{}\pstart
           \noindent{}Das \textsc{Mscrpt\pwindex{Bahr, Hermann 19.07.1863 – 15.01.1934@\textsc{Bahr, Hermann} (19.07.1863 – 15.01.1934), \emph{Schriftsteller, Kritiker}!Faun. Ein Akt1906-10-04 – 1906-11-08@\strich\emph{Der Faun. Ein Akt} {[}1906-10-04 – 1906-11-08{]}|pwv}} ist an \textsc{Salten\pwindex{Salten, Felix 06.09.1869 – 08.10.1945@\textsc{Salten, Felix} (06.09.1869 – 08.10.1945), \emph{Schriftsteller, Journalist}|pw}} abgesandt.\pend
           
         
         \endnumbering\mylabel{h}\end{ledgroupsized}  \newcommand{\dateiname}{L01604}\newcommand{\titel}{Arthur Schnitzler an Hermann Bahr, 24.–25. 6. 1906}\newcommand{\editorInnen}{ Kurt Ifkovits,  Martin Anton Müller}%% latex-leseansicht-abspann.tex
%% Abspann für die Leseansicht.
%% Der Schalter \ifkorrekturansicht ist bereits durch den Vorspann gesetzt.

%% latex-abspann.tex
%% Gemeinsamer Abspann für Korrekturansicht und Leseansicht.
%% Setzt den Schalter \ifkorrekturansicht voraus (gesetzt in den
%% einbindenden Dateien latex-korrekturansicht-abspann.tex bzw.
%% latex-leseansicht-abspann.tex).
%% ---------------------------------------------------------------

\normalsize

% Das esempio-Environment wird nur in der Leseansicht benötigt
\ifkorrekturansicht\else
\newenvironment{esempio}[3]%
{
    \vspace{1.5ex}
    \rlap{\underline{#1}}
    \par
    \setlength{\parindent}{0cm}
    \nopagebreak
    \leftskip=#2cm
    \rightskip=#3cm
}
{
    \par
}
\fi

\doendnotes{C}
\bigskip
\vfill

\clearpage

\footnotesize

\ifkorrekturansicht
  \lohead{\textsc{register}}
\fi

% theindex-Environment neu definieren ohne reledmac
\makeatletter
\renewenvironment{theindex}{%
  \ifkorrekturansicht
    \section*{\indexname}%
  \else
    \subsubsection*{Index der erwähnten Entitäten}%
  \fi
  \setlength{\parindent}{0pt}%
  \setlength{\parskip}{0pt plus 0.3pt}%
  \let\item\@idxitem
}{%
  \ifkorrekturansicht\clearpage\fi
}
\makeatother

\IfFileExists{\jobname-pw.ind}{\input{\jobname-pw.ind}}{}

% Quellenangabe nur in der Leseansicht
\ifkorrekturansicht\else
% Fallback-Definitionen, falls die .tex-Datei \titel etc. nicht gesetzt hat
\providecommand{\titel}{}
\providecommand{\editorInnen}{}
\providecommand{\dateiname}{\jobname}

\vspace{3cm}

\vfill

\footnotesize
\textsc{Quelle}: \titel. Herausgegeben von {\editorInnen}. In: \emph{Arthur Schnitzler: Briefwechsel mit Autorinnen und Autoren}.
 Digitale Edition, https://schnitzler-briefe.acdh.oeaw.ac.at/{\dateiname}.html (Stand \today)
\fi

\end{document}


      