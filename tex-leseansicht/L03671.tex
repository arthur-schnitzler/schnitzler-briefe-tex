%% latex-korrekturansicht-vorspann.tex
%% Vorspann für die Korrekturansicht.
%% Lädt die gemeinsame Datei latex-vorspann.tex mit gesetztem Schalter.

\newif\ifkorrekturansicht
\korrekturansichttrue

\input{../tex-inputs/latex-vorspann}


\section[Stefan Zweig an Arthur Schnitzler, 21. 8. 1926]{L03671 Stefan Zweig an Arthur Schnitzler, 21. 8. 1926}
\nopagebreak\mylabel{L03671v}
\rehead{ }\normalsize\beginnumbering\briefempfaengerindex{Schnitzler, Arthur@\textsc{Schnitzler, Arthur}!zzzZweig, Stefan@\emph{von Stefan Zweig}!1926-08-211@{21. 8. 1926}|(be}
\toendnotes[C]{\smallbreak\pagebreak[2]}\Standort{CUL, Schnitzler, B 118.}
\physDesc{Bildpostkarte, 511 Zeichen
\newline{}Handschrift: schwarze Tinte, lateinische Kurrent
\newline{}Versand: Stempel: »\nobreak{}\oindex{Montreux@\textbf{Montreux}, \emph{Besiedelter Ort (A.BSO)}|pwk}Montreux – Bon Port, 21. VIII. 26, 17\nobreak{}«.  }
\buchAbdrucke{\weitereDrucke{Stefan Zweig: \emph{Briefwechsel mit Hermann Bahr, Sigmund Freud, Rainer Maria
                        Rilke und Arthur Schnitzler}. Frankfurt am Main: \emph{S. Fischer} 1987, S. 421–422.} }\toendnotes[C]{\smallbreak}\pstart{}{\pb}D\textsuperscript{r}
                  Arthur Schnitzler\pend{}\pstart{}Zermatt\oindex{Zermatt@\textbf{Zermatt}, \emph{A.ADM3}|pw}\pend{}\pstart{}Hotel Beau Site\oindex{Parkhotel Beau Site@\textbf{Parkhotel Beau Site}, \emph{Hotel (K.HTL)}|pw}\pend{}{\bigskip}
\pstart
           \noindent{}\centering{}{\pb}\textcolor{gray}{\textbf{Château de Chillon\oindex{Schloss Chillon@\textbf{Schloss Chillon}, \emph{Schloss (K.SLS)}|pw}}}\pend
           \vspace{1em}
\pstart
           \noindent{}{\pb}Lieber verehrter Herr
                  Doktor, ich habe nachgefragt: in Montreux\oindex{Montreux@\textbf{Montreux}, \emph{Besiedelter Ort (A.BSO)}|pw} kann man nicht Seebaden, nur in Clarens\oindex{Clarens@\textbf{Clarens}, \emph{P.PPL}|pw} und Ouchy\oindex{Ouchy@\textbf{Ouchy}, \emph{P.PPL}|pw}. Ich denke hier,
               herrlich still in glühendster Sonne im Hotel
                  Byron\oindex{Hôtel Byron@\textbf{Hôtel Byron}, \emph{Hotel (K.HTL)}|pw} in Villeneuve\oindex{Villeneuve@\textbf{Villeneuve}, \emph{P.PPL}|pw} rastend, mit viel
               Dankbarkeit unserer \label{K_L03671-1v}\edtext{Begegnung}{\lemma{\textnormal{\emph{Begegnung}}}\Cendnote{\textnormal{siehe A. S.: \emph{Tagebuch}, 20. 8. 1926.}}}\label{K_L03671-1} im Bergland!\pend
           
\pstart
           Grüssen Sie, bitte, Frau Pollaczek\pwindex{Pollaczek, Clara Katharina 15.01.1875 – 22.07.1951@\textsc{Pollaczek, Clara Katharina} (15.01.1875 – 22.07.1951), \emph{Schriftsteller/Schriftstellerin}|pw}
               ergebenst von mir und denken Sie freundlichst Ihres immer getreuen{\\[\baselineskip]}\spacefill\mbox{Stefan Zweig}\pend
           \leftskip=0em{}
\pstart
           \noindent{}{\pb}Der Blick von meinem Fenster! Ein
                  menschenleeres wunderbares Hotel\oindex{Hôtel Byron@\textbf{Hôtel Byron}, \emph{Hotel (K.HTL)}|pwv}, herrlich abseits in dem man Monate leben möchte!\pend
           \selectlanguage{ngerman}\endnumbering\briefempfaengerindex{Schnitzler, Arthur@\textsc{Schnitzler, Arthur}!zzzZweig, Stefan@\emph{von Stefan Zweig}!1926-08-211@{21. 8. 1926}|)be}\mylabel{L03671h}
\begin{anhang}
\end{anhang}\normalsize

\doendnotes{C}
\bigskip
\vfill

\clearpage

\footnotesize

\lohead{\textsc{register}}

% Definiere theindex-Environment komplett neu ohne reledmac
\makeatletter
\renewenvironment{theindex}{%
  \section*{\indexname}%
  \setlength{\parindent}{0pt}%
  \setlength{\parskip}{0pt plus 0.3pt}%
  \let\item\@idxitem
}{%
  \clearpage
}
\makeatother

\IfFileExists{\jobname-pw.ind}{\input{\jobname-pw.ind}}{}

\end{document}

      