%% latex-leseansicht-vorspann.tex
%% Vorspann für die Leseansicht.
%% Lädt die gemeinsame Datei latex-vorspann.tex mit nicht gesetztem Schalter.

\newif\ifkorrekturansicht
\korrekturansichtfalse

\input{../tex-inputs/latex-vorspann}


\section[Stefan Zweig an Arthur Schnitzler, 21. 8. 1926]{L03671 Stefan Zweig an Arthur Schnitzler, 21. 8. 1926}
\nopagebreak\mylabel{L03671v}
\rehead{ }\normalsize\beginnumbering\briefempfaengerindex{Schnitzler, Arthur@\textsc{Schnitzler, Arthur}!zzzZweig, Stefan@\emph{von Stefan Zweig}!1926-08-211@{21. 8. 1926}|(be}
\toendnotes[C]{\smallbreak\pagebreak[2]}
\correspDesc{Versand  durch Stefan Zweig am 21. 8. 1926 in Montreux
\newline{}Erhalt  durch Arthur Schnitzler im Zeitraum [22. 8. 1926 – 26. 8. 1926?] in Zermatt}\toendnotes[C]{\smallbreak}
\Standort{CUL, Schnitzler, B 118.}
\physDesc{Bildpostkarte, 511 Zeichen
\newline{}Handschrift: schwarze Tinte, lateinische Kurrent
\newline{}Versand: Stempel: »\nobreak{}\oindex{Montreux Bon-Port@\textbf{Montreux Bon-Port}, \emph{Teil eines besiedelten Ortes}|pwk}Montreux – Bon Port, 21. VIII. 26, 17\nobreak{}«.  
\newline{}Schnitzler: mit rotem Buntstift eine Unterstreichung }
\buchAbdrucke{\weitereDrucke{Stefan Zweig: \emph{Briefwechsel mit Hermann Bahr, Sigmund Freud, Rainer Maria
                        Rilke und Arthur Schnitzler}. Herausgegeben von Jeffrey B. Berlin, Hans-Ulrich Lindken und Donald A. Prater. Frankfurt am Main: \emph{S. Fischer} 1987, S. 421–422.} }\toendnotes[C]{\smallbreak}\pstart{}{\pb}D\textsuperscript{r}
                  Arthur Schnitzler\pend{}\pstart{}Zermatt\oindex{Zermatt@\textbf{Zermatt}|pw}\pend{}\pstart{}Hotel Beau Site\oindex{Parkhotel Beau Site@\textbf{Parkhotel Beau Site}, \emph{Hotel}|pw}\pend{}{\bigskip}
\pstart
           \noindent{}\centering{}{\pb}\textcolor{gray}{\textbf{Château de Chillon\oindex{Schloss Chillon@\textbf{Schloss Chillon}, \emph{Schloss}|pw}}}\pend
           \vspace{1em}
\pstart
           \noindent{}{\pb}Lieber verehrter Herr
                  Doktor, ich habe nachgefragt: in Montreux\oindex{Montreux@\textbf{Montreux}|pw} kann man nicht Seebaden, nur in Clarens\oindex{Clarens@\textbf{Clarens}|pw} und Ouchy\oindex{Ouchy@\textbf{Ouchy}|pw}. Ich denke hier,
               herrlich still in glühendster Sonne im Hotel
                  Byron\oindex{Hôtel Byron@\textbf{Hôtel Byron}, \emph{Hotel}|pw} in Villeneuve\oindex{Villeneuve@\textbf{Villeneuve}|pw} rastend, mit viel
               Dankbarkeit unserer \label{K_L03671-1v}\edtext{Begegnung im Bergland}{\lemma{\textnormal{\emph{Begegnung im Bergland}}}\Cendnote{\textnormal{Siehe A. S.: \emph{Tagebuch}, 20. 8. 1926.}}}\label{K_L03671-1}!\pend
           
\pstart
           Grüssen Sie, bitte, Frau Pollaczek\pwindex{Pollaczek, Clara Katharina 15.\,1.\,1875 Wien – 22.\,7.\,1951 ebd.@\textsc{Pollaczek, Clara Katharina} (15.\,1.\,1875 Wien – 22.\,7.\,1951 ebd.), \emph{Schriftstellerin}|pw}
               ergebenst von mir und denken Sie freundlichst Ihres immer getreuen{\\[\baselineskip]}\spacefill\mbox{Stefan Zweig}\pend
           \leftskip=0em{}
\pstart
           \noindent{}{\pb}Der Blick von meinem Fenster! Ein
                  menschenleeres wunderbares Hotel\oindex{Hôtel Byron@\textbf{Hôtel Byron}, \emph{Hotel}|pwv}, herrlich abseits in dem man Monate leben möchte!\pend
           \selectlanguage{ngerman}\endnumbering\briefempfaengerindex{Schnitzler, Arthur@\textsc{Schnitzler, Arthur}!zzzZweig, Stefan@\emph{von Stefan Zweig}!1926-08-211@{21. 8. 1926}|)be}\mylabel{L03671h}  \newcommand{\dateiname}{L03671}\newcommand{\titel}{Stefan Zweig an Arthur Schnitzler, 21. 8. 1926}\newcommand{\editorInnen}{Selma Jahnke und Martin Anton Müller}%% latex-leseansicht-abspann.tex
%% Abspann für die Leseansicht.
%% Der Schalter \ifkorrekturansicht ist bereits durch den Vorspann gesetzt.

%% latex-abspann.tex
%% Gemeinsamer Abspann für Korrekturansicht und Leseansicht.
%% Setzt den Schalter \ifkorrekturansicht voraus (gesetzt in den
%% einbindenden Dateien latex-korrekturansicht-abspann.tex bzw.
%% latex-leseansicht-abspann.tex).
%% ---------------------------------------------------------------

\normalsize

% Das esempio-Environment wird nur in der Leseansicht benötigt
\ifkorrekturansicht\else
\newenvironment{esempio}[3]%
{
    \vspace{1.5ex}
    \rlap{\underline{#1}}
    \par
    \setlength{\parindent}{0cm}
    \nopagebreak
    \leftskip=#2cm
    \rightskip=#3cm
}
{
    \par
}
\fi

\doendnotes{C}
\bigskip
\vfill

\clearpage

\footnotesize

\ifkorrekturansicht
  \lohead{\textsc{register}}
\fi

% theindex-Environment neu definieren ohne reledmac
\makeatletter
\renewenvironment{theindex}{%
  \ifkorrekturansicht
    \section*{\indexname}%
  \else
    \subsubsection*{Index der erwähnten Entitäten}%
  \fi
  \setlength{\parindent}{0pt}%
  \setlength{\parskip}{0pt plus 0.3pt}%
  \let\item\@idxitem
}{%
  \ifkorrekturansicht\clearpage\fi
}
\makeatother

\IfFileExists{\jobname-pw.ind}{\input{\jobname-pw.ind}}{}

% Quellenangabe nur in der Leseansicht
\ifkorrekturansicht\else
% Fallback-Definitionen, falls die .tex-Datei \titel etc. nicht gesetzt hat
\providecommand{\titel}{}
\providecommand{\editorInnen}{}
\providecommand{\dateiname}{\jobname}

\vspace{3cm}

\vfill

\footnotesize
\textsc{Quelle}: \titel. Herausgegeben von {\editorInnen}. In: \emph{Arthur Schnitzler: Briefwechsel mit Autorinnen und Autoren}.
 Digitale Edition, https://schnitzler-briefe.acdh.oeaw.ac.at/{\dateiname}.html (Stand \today)
\fi

\end{document}


