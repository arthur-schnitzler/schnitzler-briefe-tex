%% latex-leseansicht-vorspann.tex
%% Vorspann für die Leseansicht.
%% Lädt die gemeinsame Datei latex-vorspann.tex mit nicht gesetztem Schalter.

\newif\ifkorrekturansicht
\korrekturansichtfalse

\input{../tex-inputs/latex-vorspann}


\section[Richard Beer-Hofmann an Arthur Schnitzler, 7. 7. 1910]{L01942 Richard Beer-Hofmann an Arthur Schnitzler, 7. 7. 1910}
\nopagebreak\mylabel{L01942v}
\rehead{ }\normalsize\beginnumbering\briefempfaengerindex{Schnitzler, Arthur@\textsc{Schnitzler, Arthur}!zzzBeer-Hofmann, Richard@\emph{von Richard Beer-Hofmann}!1910-07-071@{7. 7. 1910}|(be}
\toendnotes[C]{\smallbreak\pagebreak[2]}
\correspDesc{Versand  durch Richard Beer-Hofmann am 7. 7. 1910 in Bad Ischl
\newline{}Erhalt  durch Arthur Schnitzler im Zeitraum [8. 7. 1910
                  – 12. 7. 1910?] in Wien}\toendnotes[C]{\smallbreak}
\Standort{CUL, Schnitzler, B 8.}
\physDesc{Brief, 2 Blätter, 5 Seiten, 5220 Zeichen
\newline{}Handschrift: schwarze Tinte, lateinische Kurrent
\newline{}Beilage: maschinschriftliche Abschrift, undatiert und von unbekannter
                                 Hand mit »278c« nummeriert. Die Zuordnung ist eine
                                 Mutmaßung, basierend auf der Erwähnung im Text. Die Abschrift ist
                                 um die letzte Strophe gekürzt 
\newline{}Ordnung: mit Bleistift von unbekannter Hand nummeriert:
                                    »234« }
\buchAbdrucke{\weitereDrucke{Arthur Schnitzler, Richard Beer-Hofmann: \emph{Briefwechsel 1891–1931}. Herausgegeben von Konstanze Fliedl. Wien, Zürich: \emph{Europaverlag} 1992, S. 209–210.} }\toendnotes[C]{\smallbreak}
\pstart
           \raggedleft{}{\pb}Ischl\oindex{Bad Ischl@\textbf{Bad Ischl}|pw}{ }7./VII. 10\pend
           
\pstart{}Lieber Arthur!\pend\vspace{0.5em}
\pstart
           Ihr \label{K_L01942-1v}\edtext{Brief}{\lemma{\textnormal{\emph{Brief}}}\Cendnote{\textnormal{Siehe XXXX Auszeichnungsfehler: Dokument L01941 nicht gefunden.
               }}}\label{K_L01942-1} ist mit seiner neuen Adressirung gestern angelangt. Nun weiss es der
               Briefträger – glaube ich – auch schon wo »Steinfeld 6\oindex{Steinfeld@\textbf{Steinfeld}|pw}« ist.\pend
           
\pstart
           Hier, die gewünschte Abschrift\pwindex{Beer-Hofmann, Richard 11.\,7.\,1866 Wien – 26.\,9.\,1945 New York City@\textsc{Beer-Hofmann, Richard} (11.\,7.\,1866 Wien – 26.\,9.\,1945 New York City), \emph{Schriftsteller}!Schlaflied für Mirjam@\strich\emph{Schlaflied für Mirjam}|pwv}, des »\label{K_L01942-2v}\edtext{einen schönen Verses\pwindex{\textcolor{red}{\textsuperscript{XXXX indx1}}!Welt in der man sich langweilt@\strich\emph{Die Welt in der man sich langweilt}|pwv}}{\lemma{\textnormal{\emph{einen schönen Verses}}}\Cendnote{\textnormal{In \emph{Die
                     Welt, in der man sich langweilt}\pwindex{\textcolor{red}{\textsuperscript{XXXX indx1}}!Welt in der man sich langweilt@\strich\emph{Die Welt in der man sich langweilt}|pwk} wird von einem Dichter eine Tragödie
                  vorgetragen, die »einen schönen Vers« enthält.}}}\label{K_L01942-2}« aus der »Welt i. d. m. sich langweilt\pwindex{\textcolor{red}{\textsuperscript{XXXX indx1}}!Welt in der man sich langweilt@\strich\emph{Die Welt in der man sich langweilt}|pw}«. Bitte, lassen Sie
               die Verse vielleicht von Fräulein Pollak\pwindex{Pollak, Frieda 8.\,12.\,1881 Wien – 13.\,7.\,1937 ebd.@\textsc{Pollak, Frieda} (8.\,12.\,1881 Wien – 13.\,7.\,1937 ebd.), \emph{Sekretärin}|pw}
               abtypiren, und schicken Sie mir die Abschrift zurück; ich brauche sie, um sie dem Übersetzer\pwindex{?? [Übersetzer Deutsch zu Ungarisch] 1910 – 1910@\textsc{?? [Übersetzer Deutsch zu Ungarisch]} (1910 – 1910)|pwv} ins Ungarische\oindex{Ungarn@\textbf{Ungarn}|pw} (dem ich sie vor Monaten versprach)
               zu schicken.\pend
           
\pstart
           »Das weite Land\pwindex{Schnitzler, Arthur 15.\,5.\,1862 Wien – 21.\,10.\,1931 ebd.@\textsc{Schnitzler, Arthur} (15.\,5.\,1862 Wien – 21.\,10.\,1931 ebd.), \emph{Schriftsteller, Mediziner}!weite Land. Tragikomödie in fünf Akten@\strich\emph{Das weite Land. Tragikomödie in fünf Akten}|pw}« habe ich auf der Fahrt mit
               vieler Freude gelesen. Es hat den denkbar schlanksten Aufbau, und das bewusste
               Nichtverkleiden des Constructiven wirkt am Ende – wo einem die Führung klar wird –
               wie ein neuer Reiz. Sie haben, – glaube ich – bisher noch nie so straff die Zügel
               aller Ihrer {\pb}Figuren gehalten, und
               man empfindet alles, was feinfühlige Kritiker »Beiwerk« nennen als woltuend, um
               scharf gespanntes ein wenig zu lockern. Schon im »Medardus\pwindex{Schnitzler, Arthur 15.\,5.\,1862 Wien – 21.\,10.\,1931 ebd.@\textsc{Schnitzler, Arthur} (15.\,5.\,1862 Wien – 21.\,10.\,1931 ebd.), \emph{Schriftsteller, Mediziner}!junge Medardus. Dramatische Historie in einem Vorspiel und fünf Aufzügen@\strich\emph{Der junge Medardus. Dramatische Historie in einem Vorspiel und fünf Aufzügen}|pw}« schien mir die Richtung erkennbar, nach der Sie \label{T_L01942-1v}\edtext{sich}{\lemma{\textnormal{\emph{sich}}}\Cendnote{\textnormal{Er schreibt »Sich«.}}}\label{T_L01942-1} nun wenden.\pend
           
\pstart
           Es scheint, als genügte es Ihnen nicht, und wäre nicht in Ihren Absichten gelegen,
               die stärksten Wirkungen von den einzelnen Menschen Ihres Stückes\pwindex{Schnitzler, Arthur 15.\,5.\,1862 Wien – 21.\,10.\,1931 ebd.@\textsc{Schnitzler, Arthur} (15.\,5.\,1862 Wien – 21.\,10.\,1931 ebd.), \emph{Schriftsteller, Mediziner}!weite Land. Tragikomödie in fünf Akten@\strich\emph{Das weite Land. Tragikomödie in fünf Akten}|pwv}, und ihren Schicksalen ausgehen zu
               lassen, sondern als strebten Sie, bewusst, dahin, Einzelschicksale derart miteinander
               zu verknoten, dass jedes Theilschicksal nur ein sich unterordnender Zug, eine Runzel,
               ein Grinsen, ein Blick einer einzigen Schicksalsmaske würde, deren Ausdruck, am Ende
               des Stückes\pwindex{Schnitzler, Arthur 15.\,5.\,1862 Wien – 21.\,10.\,1931 ebd.@\textsc{Schnitzler, Arthur} (15.\,5.\,1862 Wien – 21.\,10.\,1931 ebd.), \emph{Schriftsteller, Mediziner}!weite Land. Tragikomödie in fünf Akten@\strich\emph{Das weite Land. Tragikomödie in fünf Akten}|pwv},
               das wäre, was einem als Wesentlichstes haften bliebe. {\pb}Sie können freilich sagen, – dahin
               gienge endlich alles dramatische Gestalten. Nur, scheint mir jetzt – ich möchte sagen
               – die Art Ihres Vortrags, Ihre Stimme zärtlicher und liebender zu sein, wenn es um
               Verschlingungen von Schicksalen geht, als um das Fühlen der Einzelnen.\pend
           
\pstart
           Übrigens ist die Figur des »Hofreiter\pwindex{Schnitzler, Arthur 15.\,5.\,1862 Wien – 21.\,10.\,1931 ebd.@\textsc{Schnitzler, Arthur} (15.\,5.\,1862 Wien – 21.\,10.\,1931 ebd.), \emph{Schriftsteller, Mediziner}!weite Land. Tragikomödie in fünf Akten@\strich\emph{Das weite Land. Tragikomödie in fünf Akten}|pwv}« so stark herausgekommen, dass es mir wie Kindern geht, denen die
               Bösewichte des Stückes nie genug geprügelt werden. Den »Hofreiter\pwindex{Schnitzler, Arthur 15.\,5.\,1862 Wien – 21.\,10.\,1931 ebd.@\textsc{Schnitzler, Arthur} (15.\,5.\,1862 Wien – 21.\,10.\,1931 ebd.), \emph{Schriftsteller, Mediziner}!weite Land. Tragikomödie in fünf Akten@\strich\emph{Das weite Land. Tragikomödie in fünf Akten}|pwv}« an den Sie \label{K_L01942-3v}\edtext{dachten\pwindex{Friedmann, Louis Philipp 29.\,6.\,1861 Paris – 1.\,4.\,1939 Wien@\textsc{Friedmann, Louis Philipp} (29.\,6.\,1861 Paris – 1.\,4.\,1939 Wien), \emph{Industrieller, Bergsteiger}|pwv}}{\lemma{\textnormal{\emph{dachten}}}\Cendnote{\textnormal{Schnitzlers Vorlage war Louis Friedmann\pwindex{Friedmann, Louis Philipp 29.\,6.\,1861 Paris – 1.\,4.\,1939 Wien@\textsc{Friedmann, Louis Philipp} (29.\,6.\,1861 Paris – 1.\,4.\,1939 Wien), \emph{Industrieller, Bergsteiger}|pwk}, einen Fabrikanten, den er bereits im
                  \emph{Gymnasium}\orgindex{Akademisches Gymnasium Wien@Akademisches Gymnasium Wien|pwk} kennengelernt hatte, vgl. A. S.: \emph{Tagebuch}, 21. 8. 1908.
               }}}\label{K_L01942-3}, kenne ich nur sehr oberflächlich aber dieser »Charmeur« war mir in seiner
               halbfrechen, halb \label{K_L01942-4v}\edtext{minaudirenden}{\lemma{\textnormal{\emph{minaudirenden}}}\Cendnote{\textnormal{französisch: affektiert, manieriert}}}\label{K_L01942-4}
               Koketterie immer unerträglich. Alles was er sagte und tat, war ein Versuch einen zu
               beschwätzen, oder zu brutalisieren. Ich glaube i{\geminationm}er die
               Art wie er seine Liebe an die Frau bringt, muss ein Mittelding \substVorne{}\textsuperscript{von}\substDazwischen{}zwischen\substHinten{} der Energie eines {\pb}Handlungsreisenden und der eines Erpressers \strikeout{haben.}
               sein. Für Menschen dieses Schlages wäre eine Hölle leicht zu erfinden: Der Ort, wo
               Alles, um seiner selbst willen gesagt und getan wird, und wo nichts sich spiegeln
               kann. Ich begreife, dass Frauen die Existenz von Hofreiters\pwindex{Schnitzler, Arthur 15.\,5.\,1862 Wien – 21.\,10.\,1931 ebd.@\textsc{Schnitzler, Arthur} (15.\,5.\,1862 Wien – 21.\,10.\,1931 ebd.), \emph{Schriftsteller, Mediziner}!weite Land. Tragikomödie in fünf Akten@\strich\emph{Das weite Land. Tragikomödie in fünf Akten}|pwv} als eine einzige grossartige Reverenz vor ihrer
               Sexualität empfinden, aber ich verarge – Ihnen, lieber Arthur – sehr, dass Frau Genia\pwindex{Schnitzler, Arthur 15.\,5.\,1862 Wien – 21.\,10.\,1931 ebd.@\textsc{Schnitzler, Arthur} (15.\,5.\,1862 Wien – 21.\,10.\,1931 ebd.), \emph{Schriftsteller, Mediziner}!weite Land. Tragikomödie in fünf Akten@\strich\emph{Das weite Land. Tragikomödie in fünf Akten}|pwv} ihn liebt. Ich glaube
               immer, \strikeout{a} Sie haben, aus gemeinsamer Jugend her, noch
               mehr Sympathie für Herrn Fried\pwindex{Friedmann, Louis Philipp 29.\,6.\,1861 Paris – 1.\,4.\,1939 Wien@\textsc{Friedmann, Louis Philipp} (29.\,6.\,1861 Paris – 1.\,4.\,1939 Wien), \emph{Industrieller, Bergsteiger}|pwv}– – –rich
                  Hofreiter\pwindex{Schnitzler, Arthur 15.\,5.\,1862 Wien – 21.\,10.\,1931 ebd.@\textsc{Schnitzler, Arthur} (15.\,5.\,1862 Wien – 21.\,10.\,1931 ebd.), \emph{Schriftsteller, Mediziner}!weite Land. Tragikomödie in fünf Akten@\strich\emph{Das weite Land. Tragikomödie in fünf Akten}|pwv} als er verdient. Wenn schon – dann ziehe ich die Aigners\pwindex{Schnitzler, Arthur 15.\,5.\,1862 Wien – 21.\,10.\,1931 ebd.@\textsc{Schnitzler, Arthur} (15.\,5.\,1862 Wien – 21.\,10.\,1931 ebd.), \emph{Schriftsteller, Mediziner}!weite Land. Tragikomödie in fünf Akten@\strich\emph{Das weite Land. Tragikomödie in fünf Akten}|pwv} vor. Bei denen ist es animalischer,
               mehr um der Sache selbst willen, und, wie Alles Sachliche, zuletzt, nicht
               hässlich.\pend
           
\pstart
           Übrigens ist das »Und man {\pb}kann doch nicht
               Jeden – – –\pwindex{Schnitzler, Arthur 15.\,5.\,1862 Wien – 21.\,10.\,1931 ebd.@\textsc{Schnitzler, Arthur} (15.\,5.\,1862 Wien – 21.\,10.\,1931 ebd.), \emph{Schriftsteller, Mediziner}!weite Land. Tragikomödie in fünf Akten@\strich\emph{Das weite Land. Tragikomödie in fünf Akten}|pwv}« Hofreiters\pwindex{Schnitzler, Arthur 15.\,5.\,1862 Wien – 21.\,10.\,1931 ebd.@\textsc{Schnitzler, Arthur} (15.\,5.\,1862 Wien – 21.\,10.\,1931 ebd.), \emph{Schriftsteller, Mediziner}!weite Land. Tragikomödie in fünf Akten@\strich\emph{Das weite Land. Tragikomödie in fünf Akten}|pwv},
               in der letzten Scene, prachtvoll. Hier wirkt er doch grösser, und hat ein anderes
               Gesicht als die kleinlich verknitterten Züge einer lüsternen Maus (über die, von den
               klein sich kräuselnden Haaren, ein Schatten Judenthums fällt) – an die mich das
               Original immer erinnerte.\pend
           
\pstart
           Missrathenes Halbblut, das einen – nicht mich – nachdenklich machen könnte!\pend
           
\pstart
           Eine einzige Stelle im Stück würde ich gerne vermissen: Ende des III. Aktes\pwindex{Schnitzler, Arthur 15.\,5.\,1862 Wien – 21.\,10.\,1931 ebd.@\textsc{Schnitzler, Arthur} (15.\,5.\,1862 Wien – 21.\,10.\,1931 ebd.), \emph{Schriftsteller, Mediziner}!weite Land. Tragikomödie in fünf Akten@\strich\emph{Das weite Land. Tragikomödie in fünf Akten}|pwv}. Die Worte Ernas\pwindex{Schnitzler, Arthur 15.\,5.\,1862 Wien – 21.\,10.\,1931 ebd.@\textsc{Schnitzler, Arthur} (15.\,5.\,1862 Wien – 21.\,10.\,1931 ebd.), \emph{Schriftsteller, Mediziner}!weite Land. Tragikomödie in fünf Akten@\strich\emph{Das weite Land. Tragikomödie in fünf Akten}|pwv}: »Und ich ahne, es giebt noch schönre Stunden, als die dort oben
                  war auf dem Aignerturm.\pwindex{Schnitzler, Arthur 15.\,5.\,1862 Wien – 21.\,10.\,1931 ebd.@\textsc{Schnitzler, Arthur} (15.\,5.\,1862 Wien – 21.\,10.\,1931 ebd.), \emph{Schriftsteller, Mediziner}!weite Land. Tragikomödie in fünf Akten@\strich\emph{Das weite Land. Tragikomödie in fünf Akten}|pwv}«\pend
           
\pstart
           Hier – noch dazu in Association mit der \label{K_L01942-5v}\edtext{Table d’hôte}{\lemma{\textnormal{\emph{Table d’hôte}}}\Cendnote{\textnormal{französisch, wörtlich:
                  Tisch des Gastgebers; gemeint ist eine Menüfolge, bei der die Speisen vorgegeben
                  sind.}}}\label{K_L01942-5} – wirkt das nicht wie ruhige Offenheit, sondern es wird daraus ein
               komisch-pedantisches, sich an den Tisch der Liebe setzen, und auf den letzten Gang
               freuen.\pend
           
\pstart
           {\pb}Als ich hier ankam, und vor dem
                  »Hôtel Post\oindex{Hotel Post [Bad Ischl]@\textbf{Hotel Post [Bad Ischl]}, \emph{Hotel}|pw}« auf mein Gepäck wartete, war
                  \label{K_L01942-6v}\edtext{Ihr »Gustl Wahl\pwindex{Schnitzler, Arthur 15.\,5.\,1862 Wien – 21.\,10.\,1931 ebd.@\textsc{Schnitzler, Arthur} (15.\,5.\,1862 Wien – 21.\,10.\,1931 ebd.), \emph{Schriftsteller, Mediziner}!weite Land. Tragikomödie in fünf Akten@\strich\emph{Das weite Land. Tragikomödie in fünf Akten}|pwv}\pwindex{Eckstein, Friedrich 17.\,2.\,1861 Perchtoldsdorf – 10.\,11.\,1939 Wien@\textsc{Eckstein, Friedrich} (17.\,2.\,1861 Perchtoldsdorf – 10.\,11.\,1939 Wien), \emph{Schriftsteller, Chemiker, Polyhistor}|pwv}«}{\lemma{\textnormal{\emph{Ihr »Gustl Wahl«}}}\Cendnote{\textnormal{Friedrich Eckstein\pwindex{Eckstein, Friedrich 17.\,2.\,1861 Perchtoldsdorf – 10.\,11.\,1939 Wien@\textsc{Eckstein, Friedrich} (17.\,2.\,1861 Perchtoldsdorf – 10.\,11.\,1939 Wien), \emph{Schriftsteller, Chemiker, Polyhistor}|pwk} ist in der Ischler
                  Kurliste vom 6. 7. 1910 als im Hôtel
                     Post\oindex{Hotel Post [Bad Ischl]@\textbf{Hotel Post [Bad Ischl]}, \emph{Hotel}|pwk} wohnhaft gelistet.}}}\label{K_L01942-6} das erste bekannte Gesicht, das ich sah. Er
               wird meine grosse Heiterkeit bei seinem Anblick nicht verstanden haben.\pend
           
\pstart
           Lieber Arthur: Ich danke Ihnen für die schöne Nachmittagsvorstellung die Sie mir
               verschafften, bin sicher, dass Sie noch sehr viel Freude an Ihrer Tragikomödie\pwindex{Schnitzler, Arthur 15.\,5.\,1862 Wien – 21.\,10.\,1931 ebd.@\textsc{Schnitzler, Arthur} (15.\,5.\,1862 Wien – 21.\,10.\,1931 ebd.), \emph{Schriftsteller, Mediziner}!weite Land. Tragikomödie in fünf Akten@\strich\emph{Das weite Land. Tragikomödie in fünf Akten}|pwv} haben werden, habe Ihnen noch
               eine ganze Menge darüber zu sagen: (hoffentlich ko{\geminationm}en
               Sie bald hieher) – und grüsse – mit Paula\pwindex{Beer-Hofmann, Paula 25.\,2.\,1879 Wien – 30.\,10.\,1939 Zürich@\textsc{Beer-Hofmann, Paula} (25.\,2.\,1879 Wien – 30.\,10.\,1939 Zürich)|pw}
                  zusa{\geminationm}en – Sie und Ihre Frau\pwindex{Schnitzler, Olga 17.\,1.\,1882 Wien – 13.\,1.\,1970 Lugano@\textsc{Schnitzler, Olga} (17.\,1.\,1882 Wien – 13.\,1.\,1970 Lugano), \emph{Schauspielerin, Sängerin}|pwv} herzlichst\pend
           
\pstart
           Ihr{\\[\baselineskip]}\spacefill\mbox{Richard}\pend
           \leftskip=0em{}\selectlanguage{ngerman}\vspace{1em}{\vspace{1\baselineskip}}
\pstart
           \centering{}{\pb}\uline{Schlaflied für Mirjam\pwindex{Beer-Hofmann, Mirjam 4.\,9.\,1897 Wien – 24.\,12.\,1984 New York City@\textsc{Beer-Hofmann, Mirjam} (4.\,9.\,1897 Wien – 24.\,12.\,1984 New York City)|pw}.}\pend
           {\vspace{1\baselineskip}}\stanza{}Schlaf mein Kind – schlaf, es ist spät!\newverse{}Sieh, wie die Sonne zur Ruhe dort geht,\newverse{}Hinter den Bergen stirbt sie im Rot.\newverse{}Du – du weisst nichts von Sonne und Tod,\newverse{}Wendest die Augen zum Licht und zum Schein,\newverse{}Schlaf, es sind soviel Sonnen noch dein,\newverse{}Schlaf mein Kind – mein Kind, schlaf ein!\stanzaend{}\stanza{}Schlaf mein Kind – der Abendwind weht;\newverse{}Weiss man, woher er kommt, wohin er geht?\newverse{}Dunkel, verborgen die Wege hier sind,\newverse{}Dir, und auch mir, und uns allen mein Kind!\newverse{}Blinde – so gehn wir und gehen allein,\newverse{}Keiner kann Keinem Gefährte hier sein –\newverse{}Schlaf mein Kind – mein Kind, schlaf ein!\stanzaend{}\stanza{}Schlaf mein Kind und horch nicht auf mich!\newverse{}Sinn hat’s für mich nur, und Schall ist’s für dich;\newverse{}Schall nur, wie Windeswehn, Wassergerinn,\newverse{}Worte – vielleicht eines Lebens Gewinn!\newverse{}Was ich gewonnen, gräbt mit mir man ein,\newverse{}Keiner kann Keinem ein Erbe hier sein –\newverse{}Schlaf mein Kind – mein Kind, schlaf ein!\stanzaend{}
\pstart
           \spacefill\mbox{Richard Beer-Hofmann.}\pend
           \selectlanguage{ngerman}\endnumbering\briefempfaengerindex{Schnitzler, Arthur@\textsc{Schnitzler, Arthur}!zzzBeer-Hofmann, Richard@\emph{von Richard Beer-Hofmann}!1910-07-071@{7. 7. 1910}|)be}\mylabel{L01942h}  \newcommand{\dateiname}{L01942}\newcommand{\titel}{Richard Beer-Hofmann an Arthur Schnitzler, 7. 7. 1910}\newcommand{\editorInnen}{Martin Anton Müller und Gerd-Hermann Susen}%% latex-leseansicht-abspann.tex
%% Abspann für die Leseansicht.
%% Der Schalter \ifkorrekturansicht ist bereits durch den Vorspann gesetzt.

%% latex-abspann.tex
%% Gemeinsamer Abspann für Korrekturansicht und Leseansicht.
%% Setzt den Schalter \ifkorrekturansicht voraus (gesetzt in den
%% einbindenden Dateien latex-korrekturansicht-abspann.tex bzw.
%% latex-leseansicht-abspann.tex).
%% ---------------------------------------------------------------

\normalsize

% Das esempio-Environment wird nur in der Leseansicht benötigt
\ifkorrekturansicht\else
\newenvironment{esempio}[3]%
{
    \vspace{1.5ex}
    \rlap{\underline{#1}}
    \par
    \setlength{\parindent}{0cm}
    \nopagebreak
    \leftskip=#2cm
    \rightskip=#3cm
}
{
    \par
}
\fi

\doendnotes{C}
\bigskip
\vfill

\clearpage

\footnotesize

\ifkorrekturansicht
  \lohead{\textsc{register}}
\fi

% theindex-Environment neu definieren ohne reledmac
\makeatletter
\renewenvironment{theindex}{%
  \ifkorrekturansicht
    \section*{\indexname}%
  \else
    \subsubsection*{Index der erwähnten Entitäten}%
  \fi
  \setlength{\parindent}{0pt}%
  \setlength{\parskip}{0pt plus 0.3pt}%
  \let\item\@idxitem
}{%
  \ifkorrekturansicht\clearpage\fi
}
\makeatother

\IfFileExists{\jobname-pw.ind}{\input{\jobname-pw.ind}}{}

% Quellenangabe nur in der Leseansicht
\ifkorrekturansicht\else
% Fallback-Definitionen, falls die .tex-Datei \titel etc. nicht gesetzt hat
\providecommand{\titel}{}
\providecommand{\editorInnen}{}
\providecommand{\dateiname}{\jobname}

\vspace{3cm}

\vfill

\footnotesize
\textsc{Quelle}: \titel. Herausgegeben von {\editorInnen}. In: \emph{Arthur Schnitzler: Briefwechsel mit Autorinnen und Autoren}.
 Digitale Edition, https://schnitzler-briefe.acdh.oeaw.ac.at/{\dateiname}.html (Stand \today)
\fi

\end{document}


