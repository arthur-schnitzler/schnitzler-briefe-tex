%% latex-leseansicht-vorspann.tex
%% Vorspann für die Leseansicht.
%% Lädt die gemeinsame Datei latex-vorspann.tex mit nicht gesetztem Schalter.

\newif\ifkorrekturansicht
\korrekturansichtfalse

\input{../tex-inputs/latex-vorspann}


\section[Hugo und Gerty von Hofmannsthal an Arthur Schnitzler, 23. 5. 1912]{L02070 Hugo und Gerty von Hofmannsthal an Arthur Schnitzler, 23. 5. 1912}
\nopagebreak\mylabel{L02070v}
\rehead{ }\normalsize\beginnumbering\briefempfaengerindex{Schnitzler, Arthur@\textsc{Schnitzler, Arthur}!zzzHofmannsthal, Gertrude von@\emph{von Gertrude von Hofmannsthal}!1912-05-232@{23. 5. 1912}|(be}\briefempfaengerindex{Schnitzler, Arthur@\textsc{Schnitzler, Arthur}!zzzHofmannsthal, Hugo von@\emph{von Hugo von Hofmannsthal}!1912-05-232@{23. 5. 1912}|(be}
\toendnotes[C]{\smallbreak\pagebreak[2]}
\correspDesc{Versand  durch Hugo von Hofmannsthal, Gerty von Hofmannsthal am 23. 5. 1912 in Sterzing
\newline{}Erhalt  durch Arthur Schnitzler im Zeitraum [24. 5. 1912
                  – 28. 5. 1912?] in Wien}\toendnotes[C]{\smallbreak}
\Standort{CUL, Schnitzler, B 43.}
\physDesc{Bildpostkarte, 234 Zeichen
\newline{}Handschrift Hugo von Hofmannsthal: Bleistift, deutsche Kurrent
\newline{}Handschrift Gertrude von Hofmannsthal: Bleistift, lateinische Kurrent
\newline{}Versand: Stempel: »\nobreak{}\oindex{Sterzing@\textbf{Sterzing}, \emph{Hauptstadt}|pwk}\textcolor{gray}{Sterzing}\nobreak{}«.  
\newline{}Ordnung: 1) mit Bleistift von unbekannter Hand nummeriert: »\strikeout{327}«  2) mit Bleistift von unbekannter Hand nummeriert:
                                    »377«}
\buchAbdrucke{\weitereDrucke{Hugo von Hofmannsthal, Arthur Schnitzler: \emph{Briefwechsel}. Herausgegeben von Therese Nickl und Heinrich Schnitzler. Frankfurt am Main: \emph{S. Fischer} 1964, S. 265.} }\toendnotes[C]{\smallbreak}\pstart{}\textsc{{\pb}Herrn D\textsuperscript{r} A Schnitzler}\pend{}\pstart{}\textsc{Wien\oindex{Wien@\textbf{Wien}, \emph{Verwaltungsgebiet}|pw}}\pend{}\pstart{}\textsc{XVIII Sternwartestrasse 71\oindex{Wien@\textbf{Wien}!XVIII., Währing@\textbf{XVIII., Währing}!Sternwartestraße 71@\textbf{Sternwartestraße 71}, \emph{Wohngebäude}|pw}.}\pend{}{\bigskip}
\pstart
           \noindent{}\centering{}{\pb}\textcolor{gray}{\textbf{Historische Gemälde in der
                  Alten Post\oindex{Alte Post@\textbf{Alte Post}, \emph{Hotel}|pw} zu Sterzing\oindex{Sterzing@\textbf{Sterzing}, \emph{Hauptstadt}|pw}:}}\pend
           
\pstart
           \centering{}\textcolor{gray}{\textbf{Zunftfahne\pwindex{Sieß, Anton 1733 Ratschings – 1808 Mareta@\textsc{Sieß, Anton} (1733 Ratschings – 1808 Mareta), \emph{Maler}!Zunftfahne der Bäcker und Müller mit den Heiligen Elisabeth, Sebastian und Agnes@\strich\emph{Zunftfahne der Bäcker und Müller mit den Heiligen Elisabeth, Sebastian und Agnes}|pw}, gemalt von Anton Siess\pwindex{Sieß, Anton 1733 Ratschings – 1808 Mareta@\textsc{Sieß, Anton} (1733 Ratschings – 1808 Mareta), \emph{Maler}|pw}{ }1794}}\pend
           \vspace{1em}
\pstart
           \centering{}\textcolor{gray}{\textbf{{\pb}Central-Hotel Alte Post\oindex{Alte Post@\textbf{Alte Post}, \emph{Hotel}|pw}}}\pend
           
\pstart
           \centering{}\textcolor{gray}{\textbf{(erbaut 1556)}}\pend
           
\pstart
           \centering{}\textcolor{gray}{\textbf{in Sterzing a. Br.\oindex{Sterzing@\textbf{Sterzing}, \emph{Hauptstadt}|pw}
                     (950 m)}}\pend
           
\pstart
           \centering{}\textcolor{gray}{\textbf{Besitzer: F. P. KLEEWEIN\pwindex{Kleewein, Franz Paul 1867/1868 – 21.\,3.\,1914 Sterzing@\textsc{Kleewein, Franz Paul} (1867/1868 – 21.\,3.\,1914 Sterzing), \emph{Hotelbesitzer}|pw}}}\pend
           
\pstart
           \raggedleft{}23 V. 912.\pend
           \vspace{0.5em}
\pstart
           An Welsberg\oindex{Welsberg-Taisten@\textbf{Welsberg-Taisten}, \emph{Verwaltungsgebiet}|pw} vorüberfahrend gedachten wir lieber
               Tage, die wir gern erneuern möchten. Sind übermorgen \textsc{Paris}\oindex{Paris@\textbf{Paris}, \emph{Hauptstadt}|pw}, längſtens 5 VI{ }Rodaun\oindex{Wien@\textbf{Wien}!XXIII., Liesing@\textbf{XXIII., Liesing}!Rodaun@\textbf{Rodaun}, \emph{Region}|pw}.\pend
           \pstart Herzlichst Ihr\spacefill\mbox{Hugo.}\pend{}\selectlanguage{ngerman}\vspace{1em}
\pstart
           \noindent{}{[}hs. Hofmannsthal:{]} \label{T_L02070-1v}\edtext{Viele herzliche Grüsse
                  \spacefill\mbox{Gerty.}}{\lemma{\textnormal{\emph{Viele … Gerty.}}}\Cendnote{\textnormal{quer am rechten Rand}}}\label{T_L02070-1}\pend
           \selectlanguage{ngerman}\endnumbering\briefempfaengerindex{Schnitzler, Arthur@\textsc{Schnitzler, Arthur}!zzzHofmannsthal, Gertrude von@\emph{von Gertrude von Hofmannsthal}!1912-05-232@{23. 5. 1912}|)be}\briefempfaengerindex{Schnitzler, Arthur@\textsc{Schnitzler, Arthur}!zzzHofmannsthal, Hugo von@\emph{von Hugo von Hofmannsthal}!1912-05-232@{23. 5. 1912}|)be}\mylabel{L02070h}  \newcommand{\dateiname}{L02070}\newcommand{\titel}{Hugo und Gerty von Hofmannsthal an Arthur Schnitzler, 23. 5. 1912}\newcommand{\editorInnen}{Martin Anton Müller und Gerd-Hermann Susen}%% latex-leseansicht-abspann.tex
%% Abspann für die Leseansicht.
%% Der Schalter \ifkorrekturansicht ist bereits durch den Vorspann gesetzt.

%% latex-abspann.tex
%% Gemeinsamer Abspann für Korrekturansicht und Leseansicht.
%% Setzt den Schalter \ifkorrekturansicht voraus (gesetzt in den
%% einbindenden Dateien latex-korrekturansicht-abspann.tex bzw.
%% latex-leseansicht-abspann.tex).
%% ---------------------------------------------------------------

\normalsize

% Das esempio-Environment wird nur in der Leseansicht benötigt
\ifkorrekturansicht\else
\newenvironment{esempio}[3]%
{
    \vspace{1.5ex}
    \rlap{\underline{#1}}
    \par
    \setlength{\parindent}{0cm}
    \nopagebreak
    \leftskip=#2cm
    \rightskip=#3cm
}
{
    \par
}
\fi

\doendnotes{C}
\bigskip
\vfill

\clearpage

\footnotesize

\ifkorrekturansicht
  \lohead{\textsc{register}}
\fi

% theindex-Environment neu definieren ohne reledmac
\makeatletter
\renewenvironment{theindex}{%
  \ifkorrekturansicht
    \section*{\indexname}%
  \else
    \subsubsection*{Index der erwähnten Entitäten}%
  \fi
  \setlength{\parindent}{0pt}%
  \setlength{\parskip}{0pt plus 0.3pt}%
  \let\item\@idxitem
}{%
  \ifkorrekturansicht\clearpage\fi
}
\makeatother

\IfFileExists{\jobname-pw.ind}{\input{\jobname-pw.ind}}{}

% Quellenangabe nur in der Leseansicht
\ifkorrekturansicht\else
% Fallback-Definitionen, falls die .tex-Datei \titel etc. nicht gesetzt hat
\providecommand{\titel}{}
\providecommand{\editorInnen}{}
\providecommand{\dateiname}{\jobname}

\vspace{3cm}

\vfill

\footnotesize
\textsc{Quelle}: \titel. Herausgegeben von {\editorInnen}. In: \emph{Arthur Schnitzler: Briefwechsel mit Autorinnen und Autoren}.
 Digitale Edition, https://schnitzler-briefe.acdh.oeaw.ac.at/{\dateiname}.html (Stand \today)
\fi

\end{document}


