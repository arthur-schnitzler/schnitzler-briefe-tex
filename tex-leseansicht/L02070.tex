\input{../tex-inputs/latex-pdf-vorspann}
\begin{center}
            \textcolor{red}{ENTWURF. ENTZIFFERUNG NOCH NICHT KORREKTURGELESEN}
                      \end{center}
            
               \section[Hugo und Gerty von Hofmannsthal an Arthur Schnitzler, 23. 5. 1912]{ Hugo und Gerty von Hofmannsthal an Arthur Schnitzler,
               23. 5. 1912}\nopagebreak\mylabel{v}\rehead{ }\begin{ledgroupsized}[t]{13cm}\normalsize\beginnumbering\briefempfaengerindex{Schnitzler, Arthur@\textsc{Schnitzler, Arthur}!zzzHofmannsthal, Gertrude von@\emph{von Gertrude von Hofmannsthal}!1912-05-232@{23. 5. 1912}|(be}\briefempfaengerindex{Schnitzler, Arthur@\textsc{Schnitzler, Arthur}!zzzHofmannsthal, Hugo von@\emph{von Hugo von Hofmannsthal}!1912-05-232@{23. 5. 1912}|(be} \toendnotes[C]{\smallbreak\pagebreak[2]} \Standort{CUL, Schnitzler, B 43.}
\physDesc{Bildpostkarte
\newline{}Handschrift Hugo von Hofmannsthal: Bleistift, deutsche Kurrent\newline{}Handschrift Gertrude von Hofmannsthal: Bleistift, lateinische Kurrent\newline{}Versand: Stempel: »\nobreak{}\oindex{Sterzing@\textbf{Sterzing}|pwk}\textcolor{gray}{Sterzing}\nobreak{}«.  \newline{}Ordnung: 1) mit Bleistift von unbekannter Hand nummeriert: »\strikeout{327}« 2) mit Bleistift von unbekannter Hand nummeriert:
                                    »377«}\buchAbdrucke{\weitereDrucke{Hugo von Hofmannsthal, Arthur Schnitzler: \emph{Briefwechsel}. Hg. Therese Nickl und Heinrich Schnitzler. Frankfurt am Main: \emph{S. Fischer} 1964, S. 265.} }\toendnotes[C]{\smallbreak}\pstart{}{\pb}\textsc{Herrn D\textsuperscript{r} A Schnitzler}\pend{}\pstart{}\textsc{Wien}\oindex{Wien@\textbf{Wien}|pw}\pend{}\pstart{}\textsc{XVIII Sternwartestrasse 71\oindex{Sternwartestrasse@\textbf{Sternwartestraße}|pw}.}\pend{}{\bigskip}\pstart
           \noindent{}\centering{}\textcolor{gray}{\textbf{{\pb}Historische Gemälde in der
                        Alten Post\oindex{Alte Post@\textbf{Alte Post}|pw} zu Sterzing\oindex{Sterzing@\textbf{Sterzing}|pw}:}}\pend
           \pstart
           \noindent{}\centering{}\textcolor{gray}{\textbf{Zunftfahne\pwindex{Siess, Anton 1733 – 1808@\textsc{Sieß, Anton} (1733 – 1808), \emph{Maler}!Zunftfahne der Baecker und Mueller mit den Heiligen Elisabeth, Sebastian und Agnes1794 – 1794@\strich\emph{Zunftfahne der Bäcker und Müller mit den Heiligen Elisabeth, Sebastian und Agnes} {[}1794 – 1794{]}|pw}, gemalt von Anton Siess\pwindex{Siess, Anton 1733 – 1808@\textsc{Sieß, Anton} (1733 – 1808), \emph{Maler}|pw}{ }1794}}\pend
           \pstart
           \noindent{}\centering{}\textcolor{gray}{\textbf{{\pb}Central-Hotel Alte Post\oindex{Alte Post@\textbf{Alte Post}|pw}}}\pend
           \pstart
           \noindent{}\centering{}\textcolor{gray}{\textbf{(erbaut 1556)}}\pend
           \pstart
           \noindent{}\centering{}\textcolor{gray}{\textbf{in Sterzing a. Br.\oindex{Sterzing@\textbf{Sterzing}|pw}
                     (950 m)}}\pend
           \pstart
           \noindent{}\centering{}\textcolor{gray}{\textbf{Besitzer: F. P. KLEEWEIN\pwindex{Kleewein, Franz Paul 1867/1868  – 1914-03-21@\textsc{Kleewein, Franz Paul} (1867/1868  – 1914-03-21), \emph{Hotelbesitzer}|pw}}}\pend
           \pstart
           \raggedleft{}23 V. 912.\pend
           \pstart
           An Welsberg\oindex{Welsberg-Taisten@\textbf{Welsberg-Taisten}|pw} vorüberfahrend gedachten wir lieber
               Tage, die wir gern erneuern möchten. Sind übermorgen \textsc{Paris}\oindex{Paris@\textbf{Paris}|pw}, längſtens 5 VI{ }Rodaun\oindex{Rodaun@\textbf{Rodaun}|pw}.\pend
           \pstart Herzlichst Ihr\spacefill\mbox{Hugo.}\pend{}\pstart
           \noindent{}{[}hs. G. Hofmannsthal:{]} \label{T_L02070_1v}\edtext{Viele herzliche Grüsse \spacefill\mbox{Gerty.}}{\lemma{\textnormal{\emph{Viele … Gerty.}}}\Cendnote{\textnormal{quer am rechten Rand}}}\label{T_L02070_1h}\pend
           \endnumbering\briefempfaengerindex{Schnitzler, Arthur@\textsc{Schnitzler, Arthur}!zzzHofmannsthal, Gertrude von@\emph{von Gertrude von Hofmannsthal}!1912-05-232@{23. 5. 1912}|)be}\briefempfaengerindex{Schnitzler, Arthur@\textsc{Schnitzler, Arthur}!zzzHofmannsthal, Hugo von@\emph{von Hugo von Hofmannsthal}!1912-05-232@{23. 5. 1912}|)be}\mylabel{h}\end{ledgroupsized}  \newcommand{\dateiname}{L02070}\newcommand{\titel}{Hugo und Gerty von Hofmannsthal an Arthur Schnitzler, 23. 5. 1912}\newcommand{\editorInnen}{Martin Anton Müller und Gerd-Hermann Susen}\input{../tex-inputs/latex-pdf-abspann}
      