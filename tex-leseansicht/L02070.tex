%% latex-korrekturansicht-vorspann.tex
%% Vorspann für die Korrekturansicht.
%% Lädt die gemeinsame Datei latex-vorspann.tex mit gesetztem Schalter.

\newif\ifkorrekturansicht
\korrekturansichttrue

\input{../tex-inputs/latex-vorspann}


\section[Hugo und Gerty von Hofmannsthal an Arthur Schnitzler, 23. 5. 1912]{L02070 Hugo und Gerty von Hofmannsthal an Arthur Schnitzler,
               23. 5. 1912}
\nopagebreak\mylabel{L02070v}
\rehead{ }\normalsize\beginnumbering\briefempfaengerindex{Schnitzler, Arthur@\textsc{Schnitzler, Arthur}!zzzHofmannsthal, Gertrude von@\emph{von Gertrude von Hofmannsthal}!1912-05-232@{23. 5. 1912}|(be}\briefempfaengerindex{Schnitzler, Arthur@\textsc{Schnitzler, Arthur}!zzzHofmannsthal, Hugo von@\emph{von Hugo von Hofmannsthal}!1912-05-232@{23. 5. 1912}|(be}
\toendnotes[C]{\smallbreak\pagebreak[2]}\Standort{CUL, Schnitzler, B 43.}
\physDesc{Bildpostkarte, 234 Zeichen
\newline{}Handschrift Hugo von Hofmannsthal: 1) Bleistift, deutsche Kurrent\hspace{1em}2) Bleistift, lateinische Kurrent (\noindent{}Adresse)\hspace{1em}
\newline{}Handschrift Gertrude von Hofmannsthal: Bleistift, lateinische Kurrent
\newline{}Versand: Stempel: »\nobreak{}\oindex{Sterzing@\textbf{Sterzing}, \emph{P.PPLA3}|pwk}\textcolor{gray}{Sterzing}\nobreak{}«.  
\newline{}Ordnung: 1) mit Bleistift von unbekannter Hand nummeriert: »\strikeout{327}«  2) mit Bleistift von unbekannter Hand nummeriert:
                                    »377«}
\buchAbdrucke{\weitereDrucke{Hugo von Hofmannsthal, Arthur Schnitzler: \emph{Briefwechsel}. Frankfurt am Main: \emph{S. Fischer} 1964, S. 265.} }\toendnotes[C]{\smallbreak}\pstart{}{\pb}Herrn D\textsuperscript{r} A Schnitzler\pend{}\pstart{}Wien\oindex{Wien@\textbf{Wien}, \emph{A.ADM2}|pw}\pend{}\pstart{}XVIII Sternwartestrasse 71\oindex{Sternwartestrasse 71@\textbf{Sternwartestraße 71}, \emph{Wohngebäude (K.WHS)}|pw}.\pend{}{\bigskip}
\pstart
           \noindent{}\centering{}{\pb}\textcolor{gray}{\textbf{Historische Gemälde in der
                  Alten Post\oindex{Alte Post@\textbf{Alte Post}, \emph{Hotel (K.HTL)}|pw} zu Sterzing\oindex{Sterzing@\textbf{Sterzing}, \emph{P.PPLA3}|pw}:}}\pend
           
\pstart
           \centering{}\textcolor{gray}{\textbf{Zunftfahne\pwindex{Zunftfahne der Baecker und Mueller mit den Heiligen Elisabeth, Sebastian und Agnes@\emph{Zunftfahne der Bäcker und Müller mit den Heiligen Elisabeth, Sebastian und Agnes}|pw}, gemalt von Anton Siess\pwindex{Siess, Anton 1733 – 1808@\textsc{Sieß, Anton} (1733 – 1808), \emph{Maler/Malerin}|pw}{ }1794}}\pend
           \vspace{1em}
\pstart
           \centering{}\textcolor{gray}{\textbf{{\pb}Central-Hotel Alte Post\oindex{Alte Post@\textbf{Alte Post}, \emph{Hotel (K.HTL)}|pw}}}\pend
           
\pstart
           \centering{}\textcolor{gray}{\textbf{(erbaut 1556)}}\pend
           
\pstart
           \centering{}\textcolor{gray}{\textbf{in Sterzing a. Br.\oindex{Sterzing@\textbf{Sterzing}, \emph{P.PPLA3}|pw}
                     (950 m)}}\pend
           
\pstart
           \centering{}\textcolor{gray}{\textbf{Besitzer: F. P. KLEEWEIN\pwindex{Kleewein, Franz Paul 1867/1868 – 1914-03-21@\textsc{Kleewein, Franz Paul} (1867/1868 – 1914-03-21), \emph{Hotelbesitzer/Hotelbesitzerin}|pw}}}\pend
           
\pstart
           \raggedleft{}23 V. 912.\pend
           \vspace{0.5em}
\pstart
           An Welsberg\oindex{Welsberg-Taisten@\textbf{Welsberg-Taisten}, \emph{A.ADM3}|pw} vorüberfahrend gedachten wir lieber
               Tage, die wir gern erneuern möchten. Sind übermorgen \textsc{Paris}\oindex{Paris@\textbf{Paris}, \emph{P.PPLC}|pw}, längſtens 5 VI{ }Rodaun\oindex{Rodaun@\textbf{Rodaun}, \emph{A.ADM4}|pw}.\pend
           \pstart Herzlichst Ihr\spacefill\mbox{Hugo.}\pend{}\selectlanguage{ngerman}\vspace{1em}
\pstart
           \noindent{}{[}hs. :{]} \label{T_L02070-1v}\edtext{Viele herzliche Grüsse
                  \spacefill\mbox{Gerty.}}{\lemma{\textnormal{\emph{Viele … Gerty.}}}\Cendnote{\textnormal{quer am rechten Rand}}}\label{T_L02070-1}\pend
           \selectlanguage{ngerman}\endnumbering\briefempfaengerindex{Schnitzler, Arthur@\textsc{Schnitzler, Arthur}!zzzHofmannsthal, Gertrude von@\emph{von Gertrude von Hofmannsthal}!1912-05-232@{23. 5. 1912}|)be}\briefempfaengerindex{Schnitzler, Arthur@\textsc{Schnitzler, Arthur}!zzzHofmannsthal, Hugo von@\emph{von Hugo von Hofmannsthal}!1912-05-232@{23. 5. 1912}|)be}\mylabel{L02070h}  \normalsize

\doendnotes{C}
\bigskip
\vfill

\clearpage

\footnotesize

\lohead{\textsc{register}}

% Definiere theindex-Environment komplett neu ohne reledmac
\makeatletter
\renewenvironment{theindex}{%
  \section*{\indexname}%
  \setlength{\parindent}{0pt}%
  \setlength{\parskip}{0pt plus 0.3pt}%
  \let\item\@idxitem
}{%
  \clearpage
}
\makeatother

\IfFileExists{\jobname-pw.ind}{\input{\jobname-pw.ind}}{}

\end{document}

      