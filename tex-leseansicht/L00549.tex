%% latex-korrekturansicht-vorspann.tex
%% Vorspann für die Korrekturansicht.
%% Lädt die gemeinsame Datei latex-vorspann.tex mit gesetztem Schalter.

\newif\ifkorrekturansicht
\korrekturansichttrue

\input{../tex-inputs/latex-vorspann}


\section[Arthur Schnitzler an Richard Beer-Hofmann, {[}4. 6. 1896?{]}]{L00549 Arthur Schnitzler an Richard Beer-Hofmann, {[}4. 6. 1896?{]}}
\nopagebreak\mylabel{L00549v}
\rehead{ }\normalsize\beginnumbering\briefempfaengerindex{Beer-Hofmann, Richard@\textsc{Beer-Hofmann, Richard}!zzzSchnitzler, Arthur@\emph{von Arthur Schnitzler}!1896-06-041@{{[}4. 6. 1896?{]}}|(be}
\toendnotes[C]{\smallbreak\pagebreak[2]}\Standort{YCGL, MSS 31.}
\physDesc{Brief, 1 Blatt, 4 Seiten, Umschlag, 754 Zeichen
\newline{}Handschrift: Bleistift, deutsche Kurrent
\newline{}Versand: ohne postalischen Übermittlungsvermerk }
\buchAbdrucke{\weitereDrucke{Arthur Schnitzler, Richard Beer-Hofmann: \emph{Briefwechsel 1891–1931}. Wien, Zürich: \emph{Europaverlag} 1992, S. 91.} }\toendnotes[C]{\smallbreak}\pstart{}{\pb}\textsc{Herrn Dr Rich. Beer-Hofmann}\pend{}\pstart{}\textsc{Wien\oindex{Wien@\textbf{Wien}, \emph{A.ADM2}|pw}.}\pend{}\pstart{}\textsc{I. Wollzeile 15\oindex{Wollzeile@\textbf{Wollzeile}, \emph{Straße (K.STR)}|pw}}.\pend{}{\bigskip}\vspace{1em}
\pstart
           {\pb}\textcolor{gray}{\textbf{\label{T_L00549-1v}\edtext{A S}{\lemma{\textnormal{\emph{A S}}}\Cendnote{\textnormal{Prägedruck}}}\label{T_L00549-1}}}\hfill Do{\geminationn}erſ\textcolor{gray}{tg}\pend
           
\pstart{}Lieber Richard,\pend\vspace{0.5em}
\pstart
           alſo wo nachtmahl ich heute – warten Sie –\pend
           
\pstart
           Ich werde vielleicht um, resp nach 7 bei Ihnen anläuten, ja? Weiter als
               bis in den \label{K_L00549-1v}\edtext{Prater\oindex{Prater@\textbf{Prater}, \emph{Park (K.PRK)}|pw}}{\lemma{\textnormal{\emph{Prater}}}\Cendnote{\textnormal{Das Korrespondenzstück ist undatiert. Als wahrscheinlichster Tag
                  bietet sich der 4. 6. 1896 an, da an diesem Tag Schnitzler und Beer-Hofmann\pwindex{Beer-Hofmann, Richard 1866-07-11 – 1945-09-26@\textsc{Beer-Hofmann, Richard} (1866-07-11 – 1945-09-26), \emph{Schriftsteller/Schriftstellerin}|pwk} im Prater\oindex{Prater@\textbf{Prater}, \emph{Park (K.PRK)}|pwk} essen waren. Ein
                  Aufenthalt Hofmannsthals\pwindex{Hofmannsthal, Hugo von 1874-02-01 – 1929-07-15@\textsc{Hofmannsthal, Hugo von} (1874-02-01 – 1929-07-15), \emph{Schriftsteller/Schriftstellerin}|pwk} bei Christine Schönberger\pwindex{Schoenberger, Christine 1875-11-17 – 1971-02-03@\textsc{Schönberger, Christine} (1875-11-17 – 1971-02-03), \emph{Gastwirt/Gastwirtin}|pwk} lässt sich für diesen
                  Tag nicht belegen.}}}\label{K_L00549-1} wird man ſich ja doch nicht {\pb}wagen können, ſelbſt we{\geminationn}
               es ganz ſchön wird. Aber richten Sie’s ſo ein, daſs ich nicht die 5 Stöcke zu ſteigen
               brauche, ſondern daſs Sie bereit ſind herunter zu ko{\geminationm}en.
               Haben Sie keine Luſt zu warten ſo gehen Sie ruhig fort, ich verpflichte Sie zu {\pb}nichts. Ich bin \uline{jedenfalls} bis nahezu 7 zu Haus, werde arbeiten.\pend
           
\pstart
           Danke vielmals für die Bücher\pend
           
\pstart
           Sein Sie engliſch gegrüßt{\\[\baselineskip]}Ihr \spacefill\mbox{Arthur}\pend
           \leftskip=0em{}
\pstart
           Sollten Sie zu einem ſehr feſten Entschluſs gelangen, wo {\pb}wir heute Abend ſein werden, so telegrafiren Sie
               vielleicht gleich an die Tini\pwindex{Schoenberger, Christine 1875-11-17 – 1971-02-03@\textsc{Schönberger, Christine} (1875-11-17 – 1971-02-03), \emph{Gastwirt/Gastwirtin}|pw} fürn Hugo\pwindex{Hofmannsthal, Hugo von 1874-02-01 – 1929-07-15@\textsc{Hofmannsthal, Hugo von} (1874-02-01 – 1929-07-15), \emph{Schriftsteller/Schriftstellerin}|pw}. (Südbahn\oindex{Suedbahnhof@\textbf{Südbahnhof}, \emph{Bahnhofsgebäude (K.BHF)}|pw}, \label{K_L00549-2v}\edtext{z. E.}{\lemma{\textnormal{\emph{z. E.}}}\Cendnote{\textnormal{zum Exempel}}}\label{K_L00549-2})\pend
           \selectlanguage{ngerman}\endnumbering\briefempfaengerindex{Beer-Hofmann, Richard@\textsc{Beer-Hofmann, Richard}!zzzSchnitzler, Arthur@\emph{von Arthur Schnitzler}!1896-06-041@{{[}4. 6. 1896?{]}}|)be}\mylabel{L00549h}  \normalsize

\doendnotes{C}
\bigskip
\vfill

\clearpage

\footnotesize

\lohead{\textsc{register}}

% Definiere theindex-Environment komplett neu ohne reledmac
\makeatletter
\renewenvironment{theindex}{%
  \section*{\indexname}%
  \setlength{\parindent}{0pt}%
  \setlength{\parskip}{0pt plus 0.3pt}%
  \let\item\@idxitem
}{%
  \clearpage
}
\makeatother

\IfFileExists{\jobname-pw.ind}{\input{\jobname-pw.ind}}{}

\end{document}

      