%% latex-leseansicht-vorspann.tex
%% Vorspann für die Leseansicht.
%% Lädt die gemeinsame Datei latex-vorspann.tex mit nicht gesetztem Schalter.

\newif\ifkorrekturansicht
\korrekturansichtfalse

\input{../tex-inputs/latex-vorspann}


         
         \renewcommand{\erwaehntePersonen}{Personen: Richard Beer-Hofmann, Hugo von Hofmannsthal, Christine Schönberger}
         \renewcommand{\erwaehnteOrte}{Orte: Prater, Südbahnhof, Wien, Wollzeile}
         \renewcommand{\erwaehnteWerke}{}
               \section[Arthur Schnitzler an Richard Beer-Hofmann, {[}4. 6. 1896?{]}]{ Arthur Schnitzler an Richard Beer-Hofmann, {[}4. 6. 1896?{]}}\nopagebreak\mylabel{v}\rehead{ }\begin{ledgroupsized}[t]{13cm}\normalsize\beginnumbering \toendnotes[C]{\smallbreak\pagebreak[2]} \Standort{YCGL, MSS 31.}
\physDesc{Brief, 1 Blatt, 4 Seiten, Umschlag
\newline{}Handschrift: Bleistift, deutsche Kurrent\newline{}Versand: ohne postalischen Übermittlungsvermerk }\buchAbdrucke{\weitereDrucke{Arthur Schnitzler, Richard Beer-Hofmann: \emph{Briefwechsel 1891–1931}. Hg. Konstanze Fliedl. Wien, Zürich: \emph{Europaverlag} 1992, S. 91.} }\toendnotes[C]{\smallbreak}\pstart{}{\pb}\textsc{Herrn Dr Rich. Beer-Hofmann}\pend{}\pstart{}\textsc{Wien\oindex{Wien@\textbf{Wien}|pw}.}\pend{}\pstart{}\textsc{I. Wollzeile 15\oindex{Wollzeile@\textbf{Wollzeile}|pw}}.\pend{}{\bigskip}\pstart
           \noindent{}{\pb}\textcolor{gray}{\textbf{\label{T_L00549-v}\edtext{A S}{\lemma{\textnormal{\emph{A S}}}\Cendnote{\textnormal{Prägedruck}}}\label{T_L00549-h}}}\hfill Do{\geminationn}erſ\textcolor{gray}{tg}\pend
           \pstart{}Lieber Richard,\pend\pstart
           alſo wo nachtmahl ich heute – warten Sie –\pend
           \pstart
           Ich werde vielleicht um, resp nach 7 bei Ihnen anläuten, ja? Weiter als
               bis in den \label{K_L00549_1v}\edtext{Prater\oindex{Prater@\textbf{Prater}|pw}}{\lemma{\textnormal{\emph{Prater}}}\Cendnote{\textnormal{undatiert. Als ›wahrscheinlichster‹ Tag bietet sich der 4. 6. 1896 an, da an diesem Tag Schnitzler\pwindex{Schnitzler, Arthur 15.05.1862 – 21.10.1931@\textsc{Schnitzler, Arthur} (15.05.1862 – 21.10.1931), \emph{Schriftsteller, Mediziner}|pwk} und Beer-Hofmann\pwindex{Beer-Hofmann, Richard 1866-07-11 – 1945-09-26@\textsc{Beer-Hofmann, Richard} (1866-07-11 – 1945-09-26), \emph{Schriftsteller}|pwk} im Prater\oindex{Prater@\textbf{Prater}|pwk} essen. Ein
                  Aufenthalt Hofmannsthal\pwindex{Hofmannsthal, Hugo von 1874-02-01 – 1929-07-15@\textsc{Hofmannsthal, Hugo von} (1874-02-01 – 1929-07-15), \emph{Schriftsteller}|pwk}s bei Christine Schönberger\pwindex{Schoenberger, Christine 1875-11-17 – 1971-02-03@\textsc{Schönberger, Christine} (1875-11-17 – 1971-02-03), \emph{Gastwirtin}|pwk} lässt sich für diesen Tag nicht
                  belegen.}}}\label{K_L00549_1h} wird man ſich ja doch nicht {\pb}wagen
               können, ſelbſt we{\geminationn} es ganz ſchön wird. Aber richten
               Sie’s ſo ein, daſs ich nicht die 5 Stöcke zu ſteigen brauche, ſondern daſs Sie bereit
               ſind herunter zu ko{\geminationm}en. Haben Sie keine Luſt zu warten
               ſo gehen Sie ruhig fort, ich verpflichte Sie zu {\pb}nichts. Ich bin \uline{jedenfalls} bis nahezu 7 zu Haus,
               werde arbeiten.\pend
           \pstart
           Danke vielmals für die Bücher\pend
           \pstart
           Sein Sie engliſch gegrüßt{\\[\baselineskip]}Ihr \spacefill\mbox{Arthur}\pend
           \leftskip=0em{}\pstart
           Sollten Sie zu einem ſehr feſten Entschluſs gelangen, wo {\pb}wir heute Abend ſein werden, so telegrafiren Sie
               vielleicht gleich an die Tini\pwindex{Schoenberger, Christine 1875-11-17 – 1971-02-03@\textsc{Schönberger, Christine} (1875-11-17 – 1971-02-03), \emph{Gastwirtin}|pw} fürn Hugo\pwindex{Hofmannsthal, Hugo von 1874-02-01 – 1929-07-15@\textsc{Hofmannsthal, Hugo von} (1874-02-01 – 1929-07-15), \emph{Schriftsteller}|pw}. (Südbahn\oindex{Suedbahnhof@\textbf{Südbahnhof}|pw},
                  \label{K_L00549_2v}\edtext{z. E.}{\lemma{\textnormal{\emph{z. E.}}}\Cendnote{\textnormal{zum Exempel}}}\label{K_L00549_2h})\pend
           
         
         \endnumbering\mylabel{h}\end{ledgroupsized}  \newcommand{\dateiname}{L00549}\newcommand{\titel}{Arthur Schnitzler an Richard Beer-Hofmann, [4. 6. 1896?]}\newcommand{\editorInnen}{Martin Anton Müller und Gerd-Hermann Susen}%% latex-leseansicht-abspann.tex
%% Abspann für die Leseansicht.
%% Der Schalter \ifkorrekturansicht ist bereits durch den Vorspann gesetzt.

%% latex-abspann.tex
%% Gemeinsamer Abspann für Korrekturansicht und Leseansicht.
%% Setzt den Schalter \ifkorrekturansicht voraus (gesetzt in den
%% einbindenden Dateien latex-korrekturansicht-abspann.tex bzw.
%% latex-leseansicht-abspann.tex).
%% ---------------------------------------------------------------

\normalsize

% Das esempio-Environment wird nur in der Leseansicht benötigt
\ifkorrekturansicht\else
\newenvironment{esempio}[3]%
{
    \vspace{1.5ex}
    \rlap{\underline{#1}}
    \par
    \setlength{\parindent}{0cm}
    \nopagebreak
    \leftskip=#2cm
    \rightskip=#3cm
}
{
    \par
}
\fi

\doendnotes{C}
\bigskip
\vfill

\clearpage

\footnotesize

\ifkorrekturansicht
  \lohead{\textsc{register}}
\fi

% theindex-Environment neu definieren ohne reledmac
\makeatletter
\renewenvironment{theindex}{%
  \ifkorrekturansicht
    \section*{\indexname}%
  \else
    \subsubsection*{Index der erwähnten Entitäten}%
  \fi
  \setlength{\parindent}{0pt}%
  \setlength{\parskip}{0pt plus 0.3pt}%
  \let\item\@idxitem
}{%
  \ifkorrekturansicht\clearpage\fi
}
\makeatother

\IfFileExists{\jobname-pw.ind}{\input{\jobname-pw.ind}}{}

% Quellenangabe nur in der Leseansicht
\ifkorrekturansicht\else
% Fallback-Definitionen, falls die .tex-Datei \titel etc. nicht gesetzt hat
\providecommand{\titel}{}
\providecommand{\editorInnen}{}
\providecommand{\dateiname}{\jobname}

\vspace{3cm}

\vfill

\footnotesize
\textsc{Quelle}: \titel. Herausgegeben von {\editorInnen}. In: \emph{Arthur Schnitzler: Briefwechsel mit Autorinnen und Autoren}.
 Digitale Edition, https://schnitzler-briefe.acdh.oeaw.ac.at/{\dateiname}.html (Stand \today)
\fi

\end{document}


      