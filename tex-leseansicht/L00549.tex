%% latex-leseansicht-vorspann.tex
%% Vorspann für die Leseansicht.
%% Lädt die gemeinsame Datei latex-vorspann.tex mit nicht gesetztem Schalter.

\newif\ifkorrekturansicht
\korrekturansichtfalse

\input{../tex-inputs/latex-vorspann}


\section[Arthur Schnitzler an Richard Beer-Hofmann, {[}21. 3. 1895?{]}]{L00549 Arthur Schnitzler an Richard Beer-Hofmann, {[}21. 3. 1895?{]}}
\nopagebreak\mylabel{L00549v}
\rehead{ }\normalsize\beginnumbering\briefempfaengerindex{Beer-Hofmann, Richard@\textsc{Beer-Hofmann, Richard}!zzzSchnitzler, Arthur@\emph{von Arthur Schnitzler}!1895-03-211@{{[}21. 3. 1895?{]}}|(be}
\toendnotes[C]{\smallbreak\pagebreak[2]}
\correspDesc{Versand  durch Arthur Schnitzler am [21. 3. 1895?] in Wien
\newline{}Erhalt  durch Richard Beer-Hofmann am [21. 3. 1895?] in Wien}\toendnotes[C]{\smallbreak}
\Standort{YCGL, MSS 31.}
\physDesc{Brief, 1 Blatt, 4 Seiten, Kuvert, 754 Zeichen
\newline{}Handschrift: Bleistift, deutsche Kurrent
\newline{}Versand: ohne postalischen Übermittlungsvermerk }
\buchAbdrucke{\weitereDrucke{Arthur Schnitzler, Richard Beer-Hofmann: \emph{Briefwechsel 1891–1931}. Herausgegeben von Konstanze Fliedl. Wien, Zürich: \emph{Europaverlag} 1992, S. 91.} }\toendnotes[C]{\smallbreak}\pstart{}{\pb}\textsc{Herrn Dr Rich. Beer-Hofmann}\pend{}\pstart{}\textsc{Wien\oindex{Wien@\textbf{Wien}, \emph{Verwaltungsgebiet}|pw}.}\pend{}\pstart{}\textsc{I. Wollzeile 15\oindex{Wien@\textbf{Wien}!I., Innere Stadt@\textbf{I., Innere Stadt}!Wollzeile 15 (»Berthahof«)@\textbf{Wollzeile 15 (»Berthahof«)}, \emph{Wohngebäude}|pw}}.\pend{}{\bigskip}\vspace{1em}
\pstart
           {\pb}\textcolor{gray}{\textbf{\label{T_L00549-1v}\edtext{A S}{\lemma{\textnormal{\emph{A S}}}\Cendnote{\textnormal{Prägedruck}}}\label{T_L00549-1}}}\hfill \label{K_L00549-1v}\edtext{Do{\geminationn}erſ\textcolor{gray}{tg}}{\lemma{\textnormal{\emph{Donnerstg}}}\Cendnote{\textnormal{Das Korrespondenzstück ist undatiert. Das verwendete 
                              Briefpapier mit dem Prägedruck lässt sich für den Zeitraum Jänner bis Mai 1895 belegen. Innerhalb dieses 
                              Zeitraums gibt es nur einen Donnerstag, der für das Treffen in Frage kommt. Das schließt aber nicht aus, dass das Treffen eben
                              nicht zustande kam. Ein
                              Aufenthalt Hofmannsthals\pwindex{Hofmannsthal, Hugo von 1.\,2.\,1874 Wien – 15.\,7.\,1929 Rodaun@\textsc{Hofmannsthal, Hugo von} (1.\,2.\,1874 Wien – 15.\,7.\,1929 Rodaun), \emph{Schriftsteller}|pwk} bei Christine Schönberger\pwindex{Kepert, Christine 17.\,11.\,1875 – 3.\,2.\,1971 Wien@\textsc{Kepert, Christine} (17.\,11.\,1875 – 3.\,2.\,1971 Wien), \emph{Gastwirtin}|pwk} lässt sich für diesen
                              Tag nicht belegen.}}}\label{K_L00549-1}\pend
           
\pstart{}Lieber Richard,\pend\vspace{0.5em}
\pstart
           alſo wo nachtmahl ich heute – warten Sie –\pend
           
\pstart
           Ich werde vielleicht um, resp nach 7 bei Ihnen anläuten, ja? Weiter als
               bis in den 
               Prater\oindex{Wien@\textbf{Wien}!II., Leopoldstadt@\textbf{II., Leopoldstadt}!Prater@\textbf{Prater}, \emph{Park}|pw}
                wird man{ }ſich ja doch nicht {\pb}wagen können,{ }ſelbſt we{\geminationn}
               es ganz{ }ſchön wird. Aber richten Sie’s{ }ſo ein, daſs ich nicht die 5 Stöcke zu{ }ſteigen
               brauche,{ }ſondern daſs Sie bereit{ }ſind herunter zu ko{\geminationm}en.
               Haben Sie keine Luſt zu warten{ }ſo gehen Sie ruhig fort, ich verpflichte Sie zu {\pb}nichts. Ich bin \uline{jedenfalls} bis nahezu 7 zu Haus, werde arbeiten.\pend
           
\pstart
           Danke vielmals für die Bücher\pend
           
\pstart
           Sein Sie engliſch gegrüßt{\\[\baselineskip]}Ihr \spacefill\mbox{Arthur}\pend
           \leftskip=0em{}
\pstart
           Sollten Sie zu einem{ }ſehr feſten Entschluſs gelangen, wo {\pb}wir heute Abend{ }ſein werden, so telegrafiren Sie
               vielleicht gleich an die Tini\pwindex{Kepert, Christine 17.\,11.\,1875 – 3.\,2.\,1971 Wien@\textsc{Kepert, Christine} (17.\,11.\,1875 – 3.\,2.\,1971 Wien), \emph{Gastwirtin}|pw} fürn Hugo\pwindex{Hofmannsthal, Hugo von 1.\,2.\,1874 Wien – 15.\,7.\,1929 Rodaun@\textsc{Hofmannsthal, Hugo von} (1.\,2.\,1874 Wien – 15.\,7.\,1929 Rodaun), \emph{Schriftsteller}|pw}. (Südbahn\oindex{Wien@\textbf{Wien}!X., Favoriten@\textbf{X., Favoriten}!Südbahnhof@\textbf{Südbahnhof}, \emph{Bahnhofsgebäude}|pw}, \label{K_L00549-2v}\edtext{z. E.}{\lemma{\textnormal{\emph{z. E.}}}\Cendnote{\textnormal{zum Exempel}}}\label{K_L00549-2})\pend
           \selectlanguage{ngerman}\endnumbering\briefempfaengerindex{Beer-Hofmann, Richard@\textsc{Beer-Hofmann, Richard}!zzzSchnitzler, Arthur@\emph{von Arthur Schnitzler}!1895-03-211@{{[}21. 3. 1895?{]}}|)be}\mylabel{L00549h}  \newcommand{\dateiname}{L00549}\newcommand{\titel}{Arthur Schnitzler an Richard Beer-Hofmann, [21. 3. 1895?]}\newcommand{\editorInnen}{Martin Anton Müller und Gerd-Hermann Susen}%% latex-leseansicht-abspann.tex
%% Abspann für die Leseansicht.
%% Der Schalter \ifkorrekturansicht ist bereits durch den Vorspann gesetzt.

%% latex-abspann.tex
%% Gemeinsamer Abspann für Korrekturansicht und Leseansicht.
%% Setzt den Schalter \ifkorrekturansicht voraus (gesetzt in den
%% einbindenden Dateien latex-korrekturansicht-abspann.tex bzw.
%% latex-leseansicht-abspann.tex).
%% ---------------------------------------------------------------

\normalsize

% Das esempio-Environment wird nur in der Leseansicht benötigt
\ifkorrekturansicht\else
\newenvironment{esempio}[3]%
{
    \vspace{1.5ex}
    \rlap{\underline{#1}}
    \par
    \setlength{\parindent}{0cm}
    \nopagebreak
    \leftskip=#2cm
    \rightskip=#3cm
}
{
    \par
}
\fi

\doendnotes{C}
\bigskip
\vfill

\clearpage

\footnotesize

\ifkorrekturansicht
  \lohead{\textsc{register}}
\fi

% theindex-Environment neu definieren ohne reledmac
\makeatletter
\renewenvironment{theindex}{%
  \ifkorrekturansicht
    \section*{\indexname}%
  \else
    \subsubsection*{Index der erwähnten Entitäten}%
  \fi
  \setlength{\parindent}{0pt}%
  \setlength{\parskip}{0pt plus 0.3pt}%
  \let\item\@idxitem
}{%
  \ifkorrekturansicht\clearpage\fi
}
\makeatother

\IfFileExists{\jobname-pw.ind}{\input{\jobname-pw.ind}}{}

% Quellenangabe nur in der Leseansicht
\ifkorrekturansicht\else
% Fallback-Definitionen, falls die .tex-Datei \titel etc. nicht gesetzt hat
\providecommand{\titel}{}
\providecommand{\editorInnen}{}
\providecommand{\dateiname}{\jobname}

\vspace{3cm}

\vfill

\footnotesize
\textsc{Quelle}: \titel. Herausgegeben von {\editorInnen}. In: \emph{Arthur Schnitzler: Briefwechsel mit Autorinnen und Autoren}.
 Digitale Edition, https://schnitzler-briefe.acdh.oeaw.ac.at/{\dateiname}.html (Stand \today)
\fi

\end{document}


