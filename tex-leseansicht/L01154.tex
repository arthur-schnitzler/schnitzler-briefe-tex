%% latex-leseansicht-vorspann.tex
%% Vorspann für die Leseansicht.
%% Lädt die gemeinsame Datei latex-vorspann.tex mit nicht gesetztem Schalter.

\newif\ifkorrekturansicht
\korrekturansichtfalse

\input{../tex-inputs/latex-vorspann}

\begin{center}
            \textcolor{red}{ENTWURF. ENTZIFFERUNG NOCH NICHT KORREKTURGELESEN}
                      \end{center}
            
               \section[Arthur Schnitzler an Richard Beer-Hofmann, 30. 7. 1901]{ Arthur Schnitzler an Richard Beer-Hofmann, 30. 7. 1901}\nopagebreak\mylabel{v}\rehead{ }\begin{ledgroupsized}[t]{13cm}\normalsize\beginnumbering\briefempfaengerindex{Beer-Hofmann, Richard@\textsc{Beer-Hofmann, Richard}!zzzSchnitzler, Arthur@\emph{von Arthur Schnitzler}!1901-07-301@{30. 7. 1901}|(be} \toendnotes[C]{\smallbreak\pagebreak[2]} \Standort{YCGL, MSS 31.}
\physDesc{Bildpostkarte
\newline{}Handschrift: Bleistift, deutsche Kurrent\newline{}Versand: 1) Stempel: »\nobreak{}\oindex{Sand in Taufers@\textbf{Sand in Taufers}|pwk}Sand im
                                                  Tau\textcolor{gray}{fer}{[}ertal{]}, 30 \textcolor{gray}{7} 01\nobreak{}«.  2) Stempel: »\nobreak{}\oindex{Poertschach@\textbf{Pörtschach}|pwk}Pörtschach
                                                  {[}a{]}m See, {[}31. 07.{]} 01\nobreak{}«. \newline{}Ordnung: mit Bleistift von unbekannter Hand datiert: »30. 7.« }\toendnotes[C]{\smallbreak}\pstart{}{\pb}Hrn Dr. \textsc{Rich.
                            Beer-Hofmann}\pend{}\pstart{}\textsc{Pörtschach \introOben{}am Wörthersee\introOben{}}\oindex{Poertschach@\textbf{Pörtschach}|pw}\pend{}\pstart{}\textsc{Villa Arnstein}\oindex{Villa Arnstein@\textbf{Villa Arnstein}|pw}\pend{}{\bigskip}\pstart
           \noindent{}\centering{}{\pb}\textcolor{gray}{\textbf{Sand in Taufers\oindex{Sand in Taufers@\textbf{Sand in Taufers}|pw}}}\pend
           \pstart
           Vielen Dank und vielen Gruſs. Ich bleibe vorläufig in \textsc{Vahrn}\oindex{Vahrn@\textbf{Vahrn}|pw}, wo es mir ſehr behagt. Hoffentlich dem Paul\pwindex{Goldmann, Paul 31.01.1865 – 25.09.1935@\textsc{Goldmann, Paul} (31.01.1865 – 25.09.1935), \emph{Schriftsteller, Journalist}|pw} auch. Sehe mir eben \textsc{Taufers}\oindex{Sand in Taufers@\textbf{Sand in Taufers}|pw} an. \label{K_L01154_1v}\edtext{Sie auch}{\lemma{\textnormal{\emph{Sie auch}}}\Cendnote{\textnormal{Das bezieht sich auf das
                        Postkartenmotiv.}}}\label{K_L01154_1h}, aber es ſieht natürlich anders aus.\pend
           \pstart
           30. 7. 901.\pend
           \pstart \spacefill\mbox{A.}\pend{}\endnumbering\briefempfaengerindex{Beer-Hofmann, Richard@\textsc{Beer-Hofmann, Richard}!zzzSchnitzler, Arthur@\emph{von Arthur Schnitzler}!1901-07-301@{30. 7. 1901}|)be}\mylabel{h}\end{ledgroupsized}  \newcommand{\dateiname}{L01154}\newcommand{\titel}{Arthur Schnitzler an Richard Beer-Hofmann, 30. 7. 1901}\newcommand{\editorInnen}{Martin Anton Müller und Gerd-Hermann Susen}%% latex-leseansicht-abspann.tex
%% Abspann für die Leseansicht.
%% Der Schalter \ifkorrekturansicht ist bereits durch den Vorspann gesetzt.

%% latex-abspann.tex
%% Gemeinsamer Abspann für Korrekturansicht und Leseansicht.
%% Setzt den Schalter \ifkorrekturansicht voraus (gesetzt in den
%% einbindenden Dateien latex-korrekturansicht-abspann.tex bzw.
%% latex-leseansicht-abspann.tex).
%% ---------------------------------------------------------------

\normalsize

% Das esempio-Environment wird nur in der Leseansicht benötigt
\ifkorrekturansicht\else
\newenvironment{esempio}[3]%
{
    \vspace{1.5ex}
    \rlap{\underline{#1}}
    \par
    \setlength{\parindent}{0cm}
    \nopagebreak
    \leftskip=#2cm
    \rightskip=#3cm
}
{
    \par
}
\fi

\doendnotes{C}
\bigskip
\vfill

\clearpage

\footnotesize

\ifkorrekturansicht
  \lohead{\textsc{register}}
\fi

% theindex-Environment neu definieren ohne reledmac
\makeatletter
\renewenvironment{theindex}{%
  \ifkorrekturansicht
    \section*{\indexname}%
  \else
    \subsubsection*{Index der erwähnten Entitäten}%
  \fi
  \setlength{\parindent}{0pt}%
  \setlength{\parskip}{0pt plus 0.3pt}%
  \let\item\@idxitem
}{%
  \ifkorrekturansicht\clearpage\fi
}
\makeatother

\IfFileExists{\jobname-pw.ind}{\input{\jobname-pw.ind}}{}

% Quellenangabe nur in der Leseansicht
\ifkorrekturansicht\else
% Fallback-Definitionen, falls die .tex-Datei \titel etc. nicht gesetzt hat
\providecommand{\titel}{}
\providecommand{\editorInnen}{}
\providecommand{\dateiname}{\jobname}

\vspace{3cm}

\vfill

\footnotesize
\textsc{Quelle}: \titel. Herausgegeben von {\editorInnen}. In: \emph{Arthur Schnitzler: Briefwechsel mit Autorinnen und Autoren}.
 Digitale Edition, https://schnitzler-briefe.acdh.oeaw.ac.at/{\dateiname}.html (Stand \today)
\fi

\end{document}


      