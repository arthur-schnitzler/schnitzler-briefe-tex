%% latex-leseansicht-vorspann.tex
%% Vorspann für die Leseansicht.
%% Lädt die gemeinsame Datei latex-vorspann.tex mit nicht gesetztem Schalter.

\newif\ifkorrekturansicht
\korrekturansichtfalse

\input{../tex-inputs/latex-vorspann}


\section[Arthur Schnitzler an Gerhart Hauptmann, 14. 11. 1922]{L02393 Arthur Schnitzler an Gerhart Hauptmann, 14. 11. 1922}
\nopagebreak\mylabel{L02393v}
\rehead{ }\normalsize\beginnumbering\briefempfaengerindex{Hauptmann, Gerhart@\textsc{Hauptmann, Gerhart}!zzzSchnitzler, Arthur@\emph{von Arthur Schnitzler}!1922-11-141@{14. 11. 1922}|(be}
\toendnotes[C]{\smallbreak\pagebreak[2]}
\correspDesc{Versand  durch Arthur Schnitzler am 14. 11. 1922 in Wien
\newline{}Erhalt  durch Gerhart Hauptmann am 14. 11. 1922 in Agnetendorf}\toendnotes[C]{\smallbreak}
\Standort{Staatsbibliothek Berlin – Preußischer Kulturbesitz, GHBrBl A:Schnitzler (15).}
\physDesc{Telegramm, 122 Zeichen
\newline{}HandschriftX2 einer Schreibkraft: Bleistift, deutsche Kurrent
\newline{}Versand: »\noindent{}\textcolor{gray}{\textbf{\textbf{Aufgenommen} den}}{ }14\textcolor{gray}{\textbf{/}}11 \textcolor{gray}{\textbf{192}}2{ / }\textcolor{gray}{\textbf{um}} 5 \textcolor{gray}{\textbf{Uhr}}
                                          30 \textcolor{gray}{\textbf{Min. vorm./nachm.}}{ / }\textcolor{gray}{\textbf{von}}{ }\textcolor{gray}{Hlr}{ / }\textcolor{gray}{\textbf{durch}}{ }\textcolor{gray}{×}\-\textcolor{gray}{×}« }\toendnotes[C]{\smallbreak}\pstart{}{\pb}Gerhart Hauptmann\pend{}\pstart{}Agnetendorf\oindex{Jagniątków@\textbf{Jagniątków}|pw}\pend{}{\bigskip}\vspace{1em}
\pstart
           {\pb}\textcolor{gray}{\textbf{Telegramm aus}}{ }Wien\oindex{Wien@\textbf{Wien}, \emph{Verwaltungsgebiet}|pw} 72\hfill 15\textcolor{gray}{\textbf{W. den}}{ }14\textcolor{gray}{\textbf{/}}11{ }\textcolor{gray}{\textbf{um}}{ }11\textcolor{gray}{\textbf{Uhr}}\pend
           \vspace{0.5em}
\pstart
           Seien Sie \label{K_L02393-1v}\edtext{gegrüßt}{\lemma{\textnormal{\emph{gegrüßt}}}\Cendnote{\textnormal{Es handelt sich um einen Glückwunsch zum
                  60. Geburtstag.}}}\label{K_L02393-1} bedankt geſegnet in Liebe und Verehrung\hspace*{1.5em}Ihr\pend
           \pstart \spacefill\mbox{Arthur Schnitzler}\pend{}\selectlanguage{ngerman}\endnumbering\briefempfaengerindex{Hauptmann, Gerhart@\textsc{Hauptmann, Gerhart}!zzzSchnitzler, Arthur@\emph{von Arthur Schnitzler}!1922-11-141@{14. 11. 1922}|)be}\mylabel{L02393h}  \newcommand{\dateiname}{L02393}\newcommand{\titel}{Arthur Schnitzler an Gerhart Hauptmann, 14. 11. 1922}\newcommand{\editorInnen}{Herausgegeben von Martin Anton Müller}%% latex-leseansicht-abspann.tex
%% Abspann für die Leseansicht.
%% Der Schalter \ifkorrekturansicht ist bereits durch den Vorspann gesetzt.

%% latex-abspann.tex
%% Gemeinsamer Abspann für Korrekturansicht und Leseansicht.
%% Setzt den Schalter \ifkorrekturansicht voraus (gesetzt in den
%% einbindenden Dateien latex-korrekturansicht-abspann.tex bzw.
%% latex-leseansicht-abspann.tex).
%% ---------------------------------------------------------------

\normalsize

% Das esempio-Environment wird nur in der Leseansicht benötigt
\ifkorrekturansicht\else
\newenvironment{esempio}[3]%
{
    \vspace{1.5ex}
    \rlap{\underline{#1}}
    \par
    \setlength{\parindent}{0cm}
    \nopagebreak
    \leftskip=#2cm
    \rightskip=#3cm
}
{
    \par
}
\fi

\doendnotes{C}
\bigskip
\vfill

\clearpage

\footnotesize

\ifkorrekturansicht
  \lohead{\textsc{register}}
\fi

% theindex-Environment neu definieren ohne reledmac
\makeatletter
\renewenvironment{theindex}{%
  \ifkorrekturansicht
    \section*{\indexname}%
  \else
    \subsubsection*{Index der erwähnten Entitäten}%
  \fi
  \setlength{\parindent}{0pt}%
  \setlength{\parskip}{0pt plus 0.3pt}%
  \let\item\@idxitem
}{%
  \ifkorrekturansicht\clearpage\fi
}
\makeatother

\IfFileExists{\jobname-pw.ind}{\input{\jobname-pw.ind}}{}

% Quellenangabe nur in der Leseansicht
\ifkorrekturansicht\else
% Fallback-Definitionen, falls die .tex-Datei \titel etc. nicht gesetzt hat
\providecommand{\titel}{}
\providecommand{\editorInnen}{}
\providecommand{\dateiname}{\jobname}

\vspace{3cm}

\vfill

\footnotesize
\textsc{Quelle}: \titel. Herausgegeben von {\editorInnen}. In: \emph{Arthur Schnitzler: Briefwechsel mit Autorinnen und Autoren}.
 Digitale Edition, https://schnitzler-briefe.acdh.oeaw.ac.at/{\dateiname}.html (Stand \today)
\fi

\end{document}


