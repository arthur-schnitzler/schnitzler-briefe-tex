%% latex-leseansicht-vorspann.tex
%% Vorspann für die Leseansicht.
%% Lädt die gemeinsame Datei latex-vorspann.tex mit nicht gesetztem Schalter.

\newif\ifkorrekturansicht
\korrekturansichtfalse

\input{../tex-inputs/latex-vorspann}


\section[Arthur Schnitzler an Berta Zuckerkandl, 8. 8. 1929]{L03977 Arthur Schnitzler an Berta Zuckerkandl, 8. 8. 1929}
\nopagebreak\mylabel{L03977v}
\rehead{ }\normalsize\beginnumbering\briefempfaengerindex{Zuckerkandl, Berta@\textsc{Zuckerkandl, Berta}!zzzSchnitzler, Arthur@\emph{von Arthur Schnitzler}!1929-08-081@{8. 8. 1929}|(be}
\toendnotes[C]{\smallbreak\pagebreak[2]}
\correspDesc{Versand  durch Arthur Schnitzler am 8. 8. 1929 in Wien
\newline{}Erhalt  durch Berta Zuckerkandl im Zeitraum [8. 8. 1929
                  – 11. 8. 1929?] in Wien}\toendnotes[C]{\smallbreak}
\Standort{DLA, HS.1985.1.2282.}
\physDesc{Brief, Durchschlag, 1 Blatt, 2 Seiten, 2141 Zeichen
\newline{}Schreibmaschine
\newline{}Handschrift: roter Buntstift, lateinische Kurrent (\noindent{}beschriftet: »\uline{Zuckerkandl}«, »Reigen« und »Frkr«, vier
                                 Unterstreichungen)}\toendnotes[C]{\smallbreak}
\pstart
           \raggedleft{}{\pb}8. 8. 1929.\pend
           
\pstart{}Liebe und verehrte Frau Hofrätin.\pend\vspace{0.5em}
\pstart
           Herr Maurice Rémon\pwindex{Rémon, Maurice 27.\,11.\,1861 Paris – 20.\,6.\,1945 Mérignac@\textsc{Rémon, Maurice} (27.\,11.\,1861 Paris – 20.\,6.\,1945 Mérignac), \emph{Übersetzer}|pw} hat sich neuerdings wegen
               der »Ronde\pwindex{Schnitzler, Arthur 15. 5. 1862 Wien – 21. 10. 1931 ebd.@\textsc{Schnitzler, Arthur} (15. 5. 1862 Wien – 21. 10. 1931 ebd.), \emph{Schriftsteller, Mediziner}!ronde. Dix scènes dialoguées@\strich\emph{La ronde. Dix scènes dialoguées}|pw}\pwindex{Schnitzler, Arthur 15. 5. 1862 Wien – 21. 10. 1931 ebd.@\textsc{Schnitzler, Arthur} (15. 5. 1862 Wien – 21. 10. 1931 ebd.), \emph{Schriftsteller, Mediziner}!Reigen. Zehn Dialoge@\strich\emph{Reigen. Zehn Dialoge}|pw}« resp. einer Aufführung im
                  Theater Albert Premier\orgindex{Théâtre Tristan-Bernard@Théâtre Tristan-Bernard|pw} an mich gewandt unter
                  \label{K_L03977-1v}\edtext{Beischluss eines Briefes}{\lemma{\textnormal{\emph{Beischluss eines Briefes}}}\Cendnote{\textnormal{Weder der Brief von Maurice Rémon\pwindex{Rémon, Maurice 27.\,11.\,1861 Paris – 20.\,6.\,1945 Mérignac@\textsc{Rémon, Maurice} (27.\,11.\,1861 Paris – 20.\,6.\,1945 Mérignac), \emph{Übersetzer}|pwk} noch derjenige von Irénée Mauget\pwindex{Mauget, Irénée 1881 Angoulême – 1976@\textsc{Mauget, Irénée} (1881 Angoulême – 1976), \emph{Herausgeber, Theaterdirektor, Schriftsteller}|pwk} sind überliefert.}}}\label{K_L03977-1} von Herrn Mauget\pwindex{Mauget, Irénée 1881 Angoulême – 1976@\textsc{Mauget, Irénée} (1881 Angoulême – 1976), \emph{Herausgeber, Theaterdirektor, Schriftsteller}|pw}, den ich wieder an ihn wieder
               zurücksenden musste. In diesem Brief ist sogar schon die eventuelle Generalprobe auf
               den 16. November angesetzt. Herr Rémon\pwindex{Rémon, Maurice 27.\,11.\,1861 Paris – 20.\,6.\,1945 Mérignac@\textsc{Rémon, Maurice} (27.\,11.\,1861 Paris – 20.\,6.\,1945 Mérignac), \emph{Übersetzer}|pw} hat mich um Ihre Adresse gebeten, liebe Freundin, und wird sich
               jedesfalls direkt mit Ihnen in Verbindung setzen. Meiner Ansicht nach könnte man der
               Angelegenheit nur unter der Voraussetzung einer angemessenen Vorschusssumme näher
               treten; aber wenn Sie es für richtig finden und weiter auf Antoine\pwindex{Antoine, André 31.\,1.\,1858 Limoges – 23.\,10.\,1943 Le Pouliguen@\textsc{Antoine, André} (31.\,1.\,1858 Limoges – 23.\,10.\,1943 Le Pouliguen), \emph{Theaterleiter, Schauspieler}|pw} bauen, so bin ich auch einverstanden den Antrag Mauget\pwindex{Mauget, Irénée 1881 Angoulême – 1976@\textsc{Mauget, Irénée} (1881 Angoulême – 1976), \emph{Herausgeber, Theaterdirektor, Schriftsteller}|pw}-Rémon\pwindex{Rémon, Maurice 27.\,11.\,1861 Paris – 20.\,6.\,1945 Mérignac@\textsc{Rémon, Maurice} (27.\,11.\,1861 Paris – 20.\,6.\,1945 Mérignac), \emph{Übersetzer}|pw} endgültig abzulehnen.. Für alle Fälle aber sollte man, denke ich, den
                  Rémon’schen\pwindex{Rémon, Maurice 27.\,11.\,1861 Paris – 20.\,6.\,1945 Mérignac@\textsc{Rémon, Maurice} (27.\,11.\,1861 Paris – 20.\,6.\,1945 Mérignac), \emph{Übersetzer}|pw} Antrag zum Zwecke einer hier
               doch wohl erlaubten höflichen Pression auf die anderen Bewerber benützen. \label{K_L03977-2v}\edtext{Ich schreibe Rémon\pwindex{Rémon, Maurice 27.\,11.\,1861 Paris – 20.\,6.\,1945 Mérignac@\textsc{Rémon, Maurice} (27.\,11.\,1861 Paris – 20.\,6.\,1945 Mérignac), \emph{Übersetzer}|pw}}{\lemma{\textnormal{\emph{Ich schreibe Rémon}}}\Cendnote{\textnormal{Arthur Schnitzler an Maurice Rémon\pwindex{Rémon, Maurice 27.\,11.\,1861 Paris – 20.\,6.\,1945 Mérignac@\textsc{Rémon, Maurice} (27.\,11.\,1861 Paris – 20.\,6.\,1945 Mérignac), \emph{Übersetzer}|pwk}, 8. 8. 1929,
                        \emph{Deutsches Literaturarchiv Marbach},
                     HS.1985.1.1686.}}}\label{K_L03977-2}, dass ich mich in dieser Sache, wie
               selbstverständlich vollkommen Ihrer Führung und Ihrem Rate anvertraue. Auf welche
               Weise könnten wir denn in den Besitz einer Abschrift des »Weiten Land\pwindex{Schnitzler, Arthur 15. 5. 1862 Wien – 21. 10. 1931 ebd.@\textsc{Schnitzler, Arthur} (15. 5. 1862 Wien – 21. 10. 1931 ebd.), \emph{Schriftsteller, Mediziner}!weite Land. Tragikomödie in fünf Akten@\strich\emph{Das weite Land. Tragikomödie in fünf Akten}|pw}\pwindex{Schnitzler, Arthur 15. 5. 1862 Wien – 21. 10. 1931 ebd.@\textsc{Schnitzler, Arthur} (15. 5. 1862 Wien – 21. 10. 1931 ebd.), \emph{Schriftsteller, Mediziner}!Le Pays de l’âme. Drame en 5 actes@\strich\emph{Le Pays de l’âme. Drame en 5 actes}|pw}« in französischer\oindex{Frankreich@\textbf{Frankreich}|pw} Sprache gelangen? Wo ist ein, resp. das Exemplar\pwindex{Schnitzler, Arthur 15. 5. 1862 Wien – 21. 10. 1931 ebd.@\textsc{Schnitzler, Arthur} (15. 5. 1862 Wien – 21. 10. 1931 ebd.), \emph{Schriftsteller, Mediziner}!Le Pays de l’âme. Drame en 5 actes@\strich\emph{Le Pays de l’âme. Drame en 5 actes}|pwv}? Mme. Clauser\pwindex{Clauser, Suzanne 16.\,5.\,1898 Wien – 11.\,9.\,1981 Paris@\textsc{Clauser, Suzanne} (16.\,5.\,1898 Wien – 11.\,9.\,1981 Paris), \emph{Schriftstellerin, Übersetzerin}|pw} möchte sich, wie Sie ja wohl wissen, bei Mme. Marnac\pwindex{Marnac, Jeanne 8.\,2.\,1892 Brüssel – 2.\,12.\,1976 Paris@\textsc{Marnac, Jeanne} (8.\,2.\,1892 Brüssel – 2.\,12.\,1976 Paris), \emph{Theaterleiterin, Schauspielerin}|pw} entweder für »Zwischenspiel\pwindex{Schnitzler, Arthur 15. 5. 1862 Wien – 21. 10. 1931 ebd.@\textsc{Schnitzler, Arthur} (15. 5. 1862 Wien – 21. 10. 1931 ebd.), \emph{Schriftsteller, Mediziner}!Zwischenspiel. Komödie in drei Akten@\strich\emph{Zwischenspiel. Komödie in drei Akten}|pw}\pwindex{Schnitzler, Arthur 15. 5. 1862 Wien – 21. 10. 1931 ebd.@\textsc{Schnitzler, Arthur} (15. 5. 1862 Wien – 21. 10. 1931 ebd.), \emph{Schriftsteller, Mediziner}!?? [französische Übersetzung von Zwischenspiel]@\strich\emph{?? [französische Übersetzung von Zwischenspiel]}|pw}« (Rémon\pwindex{Rémon, Maurice 27.\,11.\,1861 Paris – 20.\,6.\,1945 Mérignac@\textsc{Rémon, Maurice} (27.\,11.\,1861 Paris – 20.\,6.\,1945 Mérignac), \emph{Übersetzer}|pw})
               oder »Weites Land\pwindex{Schnitzler, Arthur 15. 5. 1862 Wien – 21. 10. 1931 ebd.@\textsc{Schnitzler, Arthur} (15. 5. 1862 Wien – 21. 10. 1931 ebd.), \emph{Schriftsteller, Mediziner}!weite Land. Tragikomödie in fünf Akten@\strich\emph{Das weite Land. Tragikomödie in fünf Akten}|pw}« bemühen.\pend
           
\pstart
           Von Alzir Hélla\pwindex{Hella, Alzir 30.\,12.\,1881 Vieux Condé – 14.\,7.\,1953 Paris@\textsc{Hella, Alzir} (30.\,12.\,1881 Vieux Condé – 14.\,7.\,1953 Paris), \emph{Übersetzer}|pw} habe ich wohl endlich \label{K_L03977-3v}\edtext{die Revue des vivants\pwindex{Revue des Vivants@\emph{La Revue des Vivants}|pw}}{\lemma{\textnormal{\emph{die Revue des vivants}}}\Cendnote{\textnormal{\emph{Le Célibataire}\pwindex{Schnitzler, Arthur 15. 5. 1862 Wien – 21. 10. 1931 ebd.@\textsc{Schnitzler, Arthur} (15. 5. 1862 Wien – 21. 10. 1931 ebd.), \emph{Schriftsteller, Mediziner}!Le Célibataire@\strich\emph{Le Célibataire}|pwk}. In: \emph{La Revue des Vivants}\pwindex{Revue des Vivants@\emph{La Revue des Vivants}|pwk}, Jg. 3, Nr. 3, März 1929, S. 488–498.}}}\label{K_L03977-3} mit der Uebersetzung\pwindex{Schnitzler, Arthur 15. 5. 1862 Wien – 21. 10. 1931 ebd.@\textsc{Schnitzler, Arthur} (15. 5. 1862 Wien – 21. 10. 1931 ebd.), \emph{Schriftsteller, Mediziner}!Le Célibataire@\strich\emph{Le Célibataire}|pwv} des »Tod des Junggesellen\pwindex{Schnitzler, Arthur 15. 5. 1862 Wien – 21. 10. 1931 ebd.@\textsc{Schnitzler, Arthur} (15. 5. 1862 Wien – 21. 10. 1931 ebd.), \emph{Schriftsteller, Mediziner}!Tod des Junggesellen. Novelle@\strich\emph{Der Tod des Junggesellen. Novelle}|pw}«, sowie die Mitteilung vom Erscheinen
               der »Beate\pwindex{Schnitzler, Arthur 15. 5. 1862 Wien – 21. 10. 1931 ebd.@\textsc{Schnitzler, Arthur} (15. 5. 1862 Wien – 21. 10. 1931 ebd.), \emph{Schriftsteller, Mediziner}!Frau Beate und ihr Sohn. Novelle@\strich\emph{Frau Beate und ihr Sohn. Novelle}|pw}\pwindex{Schnitzler, Arthur 15. 5. 1862 Wien – 21. 10. 1931 ebd.@\textsc{Schnitzler, Arthur} (15. 5. 1862 Wien – 21. 10. 1931 ebd.), \emph{Schriftsteller, Mediziner}!Madame Beate et son fils@\strich\emph{Madame Beate et son fils}|pw}« \label{K_L03977-4v}\edtext{in Buchform}{\lemma{\textnormal{\emph{in Buchform}}}\Cendnote{\textnormal{Arthur Schnitzler: \emph{Madame Béate et son fils. Roman}\pwindex{Schnitzler, Arthur 15. 5. 1862 Wien – 21. 10. 1931 ebd.@\textsc{Schnitzler, Arthur} (15. 5. 1862 Wien – 21. 10. 1931 ebd.), \emph{Schriftsteller, Mediziner}!Madame Beate et son fils@\strich\emph{Madame Beate et son fils}|pwk}. Traduit de l'Allemand par A. Hella\pwindex{Hella, Alzir 30.\,12.\,1881 Vieux Condé – 14.\,7.\,1953 Paris@\textsc{Hella, Alzir} (30.\,12.\,1881 Vieux Condé – 14.\,7.\,1953 Paris), \emph{Übersetzer}|pwk} et O. Bournac\pwindex{Bournac, Olivier 13.\,8.\,1885 Saint-Amans-du-Pech – Anfang Januar 1931 Toulon@\textsc{Bournac, Olivier} (13.\,8.\,1885 Saint-Amans-du-Pech – Anfang Januar 1931 Toulon), \emph{Schriftsteller, Übersetzer}|pwk}, Paris: \emph{Victor Attinger}{ }1929.}}}\label{K_L03977-4} erhalten, aber kein
               Honorar. Wir müssen also (85:15) weiter {\pb}darben.\pend
           
\pstart
           Ich hoffe, Gastein\oindex{Bad Gastein@\textbf{Bad Gastein}, \emph{Hauptstadt}|pw} tut Ihnen gut und Sie kommen
               so gesund als möglich nach Wien\oindex{Wien@\textbf{Wien}, \emph{Verwaltungsgebiet}|pw} zurück. Ich selbst
               bin, wie Sie merken, noch immer da, vielleicht kommt \label{K_L03977-5v}\edtext{mein Sohn\pwindex{Schnitzler, Heinrich 9.\,8.\,1902 Hinterbrühl – 12.\,7.\,1982 Wien@\textsc{Schnitzler, Heinrich} (9.\,8.\,1902 Hinterbrühl – 12.\,7.\,1982 Wien), \emph{Regisseur, Schauspieler}|pwv}}{\lemma{\textnormal{\emph{mein Sohn}}}\Cendnote{\textnormal{Heinrich Schnitzler\pwindex{Schnitzler, Heinrich 9.\,8.\,1902 Hinterbrühl – 12.\,7.\,1982 Wien@\textsc{Schnitzler, Heinrich} (9.\,8.\,1902 Hinterbrühl – 12.\,7.\,1982 Wien), \emph{Regisseur, Schauspieler}|pwk} kam laut \emph{Tagebuch}\pwindex{Schnitzler, Arthur 15. 5. 1862 Wien – 21. 10. 1931 ebd.@\textsc{Schnitzler, Arthur} (15. 5. 1862 Wien – 21. 10. 1931 ebd.), \emph{Schriftsteller, Mediziner}!Tagebuch@\strich\emph{Tagebuch}|pwk} am 13. 8. 1929 in Wien\oindex{Wien@\textbf{Wien}, \emph{Verwaltungsgebiet}|pwk} an.}}}\label{K_L03977-5} noch einmal und wahrscheinlich werde ich erst gegen den
                  20. d.\label{K_L03977-6v}\edtext{in die französische Schweiz\oindex{Romandie@\textbf{Romandie}|pw}}{\lemma{\textnormal{\emph{in … Schweiz}}}\Cendnote{\textnormal{Schnitzler verließ Wien\oindex{Wien@\textbf{Wien}, \emph{Verwaltungsgebiet}|pwk} am 19. 8. 1929, um mit Clara Katharina Pollaczek\pwindex{Pollaczek, Clara Katharina 15.\,1.\,1875 Wien – 22.\,7.\,1951 ebd.@\textsc{Pollaczek, Clara Katharina} (15.\,1.\,1875 Wien – 22.\,7.\,1951 ebd.), \emph{Schriftstellerin}|pwk} Ferien in Caux\oindex{Caux@\textbf{Caux}|pwk} und Terriet\oindex{Territet@\textbf{Territet}|pwk} zu
                  machen. Im Anschluss reiste er allein nach Marienbad\oindex{Marienbad@\textbf{Marienbad}|pwk}, um seine Exfrau Olga
                     Schnitzler\pwindex{Schnitzler, Olga 17.\,1.\,1882 Wien – 13.\,1.\,1970 Lugano@\textsc{Schnitzler, Olga} (17.\,1.\,1882 Wien – 13.\,1.\,1970 Lugano), \emph{Schauspielerin, Sängerin}|pwk} in Franzensbad\oindex{Franzensbad@\textbf{Franzensbad}|pwk} zu treffen,
                  und kehrte am 23. 9. 1929 nach Wien\oindex{Wien@\textbf{Wien}, \emph{Verwaltungsgebiet}|pwk}
                  zurück.}}}\label{K_L03977-6} fahren; vollkommen feststehend ist es noch nicht. Von Frau Trude\pwindex{Zuckerkandl, Gertrude 18.\,9.\,1895 Wien – 13.\,7.\,1981 Paris@\textsc{Zuckerkandl, Gertrude} (18.\,9.\,1895 Wien – 13.\,7.\,1981 Paris), \emph{Malerin}|pw}, mit der ich heute morgens
               telefoniert habe{[},{]} habe ich das Allerbeste über Ihr und der
               Ihrigen Befinden gehört. Frau Trude\pwindex{Zuckerkandl, Gertrude 18.\,9.\,1895 Wien – 13.\,7.\,1981 Paris@\textsc{Zuckerkandl, Gertrude} (18.\,9.\,1895 Wien – 13.\,7.\,1981 Paris), \emph{Malerin}|pw} verdanke
               ich auch Ihre Adresse\oindex{Conrad Strochner-Straße 3@\textbf{Conrad Strochner-Straße 3}, \emph{Wohngebäude}|pwv}, die
               mir entfallen war.\pend
           \pstart Herzlichste Grüsse und die allerbesten Sommerwünsche, freundschaftlichst wie
               immer Der Ihre\pend{}{\vspace{1\baselineskip}}
\pstart
           \noindent{}Frau Hofrätin Bertha Zuckerkandl,{\\}Bad Gastein\oindex{Bad Gastein@\textbf{Bad Gastein}, \emph{Hauptstadt}|pw}{\\}Gruber-Haus\oindex{Conrad Strochner-Straße 3@\textbf{Conrad Strochner-Straße 3}, \emph{Wohngebäude}|pw}.\pend
           \selectlanguage{ngerman}\endnumbering\briefempfaengerindex{Zuckerkandl, Berta@\textsc{Zuckerkandl, Berta}!zzzSchnitzler, Arthur@\emph{von Arthur Schnitzler}!1929-08-081@{8. 8. 1929}|)be}\mylabel{L03977h}
\begin{anhang}
\end{anhang}\newcommand{\dateiname}{L03977}\newcommand{\titel}{Arthur Schnitzler an Berta Zuckerkandl, 8. 8. 1929}\newcommand{\editorInnen}{Herausgegeben von Jahnke, SelmaMüller, Martin Anton}%% latex-leseansicht-abspann.tex
%% Abspann für die Leseansicht.
%% Der Schalter \ifkorrekturansicht ist bereits durch den Vorspann gesetzt.

%% latex-abspann.tex
%% Gemeinsamer Abspann für Korrekturansicht und Leseansicht.
%% Setzt den Schalter \ifkorrekturansicht voraus (gesetzt in den
%% einbindenden Dateien latex-korrekturansicht-abspann.tex bzw.
%% latex-leseansicht-abspann.tex).
%% ---------------------------------------------------------------

\normalsize

% Das esempio-Environment wird nur in der Leseansicht benötigt
\ifkorrekturansicht\else
\newenvironment{esempio}[3]%
{
    \vspace{1.5ex}
    \rlap{\underline{#1}}
    \par
    \setlength{\parindent}{0cm}
    \nopagebreak
    \leftskip=#2cm
    \rightskip=#3cm
}
{
    \par
}
\fi

\doendnotes{C}
\bigskip
\vfill

\clearpage

\footnotesize

\ifkorrekturansicht
  \lohead{\textsc{register}}
\fi

% theindex-Environment neu definieren ohne reledmac
\makeatletter
\renewenvironment{theindex}{%
  \ifkorrekturansicht
    \section*{\indexname}%
  \else
    \subsubsection*{Index der erwähnten Entitäten}%
  \fi
  \setlength{\parindent}{0pt}%
  \setlength{\parskip}{0pt plus 0.3pt}%
  \let\item\@idxitem
}{%
  \ifkorrekturansicht\clearpage\fi
}
\makeatother

\IfFileExists{\jobname-pw.ind}{\input{\jobname-pw.ind}}{}

% Quellenangabe nur in der Leseansicht
\ifkorrekturansicht\else
% Fallback-Definitionen, falls die .tex-Datei \titel etc. nicht gesetzt hat
\providecommand{\titel}{}
\providecommand{\editorInnen}{}
\providecommand{\dateiname}{\jobname}

\vspace{3cm}

\vfill

\footnotesize
\textsc{Quelle}: \titel. Herausgegeben von {\editorInnen}. In: \emph{Arthur Schnitzler: Briefwechsel mit Autorinnen und Autoren}.
 Digitale Edition, https://schnitzler-briefe.acdh.oeaw.ac.at/{\dateiname}.html (Stand \today)
\fi

\end{document}


