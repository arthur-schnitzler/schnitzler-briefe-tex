%% latex-korrekturansicht-vorspann.tex
%% Vorspann für die Korrekturansicht.
%% Lädt die gemeinsame Datei latex-vorspann.tex mit gesetztem Schalter.

\newif\ifkorrekturansicht
\korrekturansichttrue

\input{../tex-inputs/latex-vorspann}


\section[Arthur Schnitzler an Georg Brandes, 14. 12. 1924]{L02423 Arthur Schnitzler an Georg Brandes, 14. 12. 1924}
\nopagebreak\mylabel{L02423v}
\rehead{ }\normalsize\beginnumbering\briefempfaengerindex{Brandes, Georg@\textsc{Brandes, Georg}!zzzSchnitzler, Arthur@\emph{von Arthur Schnitzler}!1924-12-141@{14. 12. 1924}|(be}
\toendnotes[C]{\smallbreak\pagebreak[2]}\Standort{Kopenhagen, Det Kongelige Bibliotek, Georg Brandes Arkiv, box 125.}
\physDesc{Brief, 4 Blätter, 7 Seiten, 4975 Zeichen (die weiteren Blätter von Schnitzler nummeriert:
                                    »II«–»IV«, die Rückseite des
                                 zweiten Blattes blieb unbeschrieben)
\newline{}Handschrift: schwarze Tinte, lateinische Kurrent
\newline{}Ordnung: 1) mit Bleistift von unbekannter Hand nummeriert:
                                    »49.«  2) mit Bleistift von unbekannter Hand die weiteren Blätter datiert: »14/12 24«}
\buchAbdrucke{\weitereDrucke{Georg Brandes, Arthur Schnitzler: \emph{Ein Briefwechsel}. Bern: \emph{Francke} 1956, S. 141–143.} }\toendnotes[C]{\smallbreak}
\pstart
           \raggedleft{}{\pb}Wien\oindex{Wien@\textbf{Wien}, \emph{A.ADM2}|pw}{ }14. 12. 924\pend
           \vspace{0.5em}
\pstart
           mein lieber und verehrter Freund, den Empfang Ihres Briefes vom
                  10. Dezember will ich gleich mit dem herzlichsten Dank bestätigen.
               Denken Sie, \uline{mit der gleichen Post} kam Ihr Julius Caesar\pwindex{Gaius Julius Cæsar@\emph{Gaius Julius Cæsar}|pw} – vom Verleger \introOben{}(Reiss\orgindex{Erich-Reiss-Verlag@Erich-Reiss-Verlag|pw})\introOben{} übersandt, zugleich mit dem
               dritten Band der neuen Ausgabe der Hauptströmungen\pwindex{Hauptstroemungen der Literatur des neunzehnten Jahrhunderts@\emph{Hauptströmungen der Literatur des neunzehnten Jahrhunderts}|pw}. Also – dieser Caesar\pwindex{Gaius Julius Cæsar@\emph{Gaius Julius Cæsar}|pw}
               ist ohne Ihre Autorisation in Deutschland\oindex{Deutschland@\textbf{Deutschland}, \emph{A.PCLI}|pw}
               erschienen? Aber Voltaire\pwindex{Voltaire und sein Jahrhundert@\emph{Voltaire und sein Jahrhundert}|pw}, Michel Angelo\pwindex{Michelangelo Buonarotti@\emph{Michelangelo Buonarotti}|pw}, Goethe\pwindex{Wolfgang Goethe@\emph{Wolfgang Goethe}|pw} – das
               sind doch autorisirte deutsche Ausgaben? Bitte sagen Sie mir ein Wort darüber. Ich
               erkundigte mich im vergangenen Frühjahr – anläßlich meiner Bestätigung der
               eingetroffenen anderen Brandes Bände, – \introOben{}bei Reiss\orgindex{Erich-Reiss-Verlag@Erich-Reiss-Verlag|pw}\introOben{} für wann der Caesar\pwindex{Gaius Julius Cæsar@\emph{Gaius Julius Cæsar}|pw} zu erwarten sei – er
               erwiderte, dſs er ihn gleich nach Erscheinen an mich senden werde – das hat er {\pb}nun gethan – und \uline{Sie} sollten erst durch mich authentisches von diesem deutschen Caesar\pwindex{Gaius Julius Cæsar@\emph{Gaius Julius Cæsar}|pw} erfahren – u hatten nicht einmal Honorar
               erhalten??\pend
           
\pstart
           – Die Angelegenheit irritirt mich vielleicht darum ein bißchen mehr, weil ich immer
               wieder so arge und ärgerliche Dinge mit meinen Büchern im Ausland erlebe. Noch nie
               ist der Diebstahl, jeder Raub am geistigen Eigenthum so schamlos betrieben worden als
               jetzt! Man muſs Mahnbriefe schreiben, Prozeſſe führen – oh nicht nur ins Ausland; –
               auch in nächste Nähe, – verschwendet Zeit und Geisteskraft an geschäftliche
               Correspondenzen – und erreicht so wenig! – Aber genug davon. –\pend
           
\pstart
           Es freut mich, dſs Ihnen die Kom. der Verführung\pwindex{Komoedie der Verfuehrung. In drei Akten@\emph{Komödie der Verführung. In drei Akten}|pw}
               einigen Spaſs gemacht hat und dſs Sie mir die Palmen, die ich in Gilleleje\oindex{Gilleleje@\textbf{Gilleleje}, \emph{P.PPL}|pw} wachsen {\pb}lieſs,
               nicht übel genommen haben – \label{T_L02423-1v}\edtext{(}{\lemma{\textnormal{\emph{(}}}\Cendnote{\textnormal{die öffnende Klammer doppelt platziert,
                  vermutlich eine versuchte Verdeutlichung, da bei einer Tinte fehlt}}}\label{T_L02423-1}im
               Gegensatz zu einer Landsmännin (und entfernten Verwandten) von Ihnen glaub ich), der
               Frau Karen Stampe Bendix\pwindex{Stampe-Bendix, Karen 08.02.1881 – 1963-12-17@\textsc{Stampe-Bendix, Karen} (08.02.1881 – 1963-12-17), \emph{Schriftsteller/Schriftstellerin}|pw}, die ich manchmal sehe
               – und die eine reizende kleine Tochter\pwindex{Ellis, Lillian 25.05.1907 – 21.02.1951@\textsc{Ellis, Lillian} (25.05.1907 – 21.02.1951), \emph{Schauspieler/Schauspielerin, Tänzer/Tänzerin}|pwv} – Tänzerin hat.) Das Stück\pwindex{Komoedie der Verfuehrung. In drei Akten@\emph{Komödie der Verführung. In drei Akten}|pwv} hat es ziemlich schwer und wird sich – wie es mit
               meinen meisten Stücken geht – von meinen allerersten abgesehen, – nur allmälig
               durchsetzen. Die Verlogenheit der Kritik in »moralischen« Dingen ist seltsamerweise –
               je freier die Existenz gerade in dieser Hinsicht sich gestaltet hat – ungeheuerlicher
               als je. Für mich hat jetzt das Völkchen eine neue Formel gefunden: dſs ich nemlich
               eine »versunkene Welt« gestalte, für die sich kein Mensch mehr interessire. (Man darf
               nur Dramen von 1924 schreiben – haben Sie das gewußt?) Auch sind Tod und
               Liebe unwürdige Sujets; – nur Grenzregulirungen, Valutenaenderungen, Steuerfragen,
               Diebstähle und Hungerrevolten interessiren den {\pb}ernsten (insbesondere ernsten deutschen) Mann. –\pend
           
\pstart
           Hab ich Ihnen schon einmal geschrieben, dſs mein Sohn Heinrich\pwindex{Schnitzler, Heinrich 09.08.1902 – 12.07.1982@\textsc{Schnitzler, Heinrich} (09.08.1902 – 12.07.1982), \emph{Regisseur/Regisseurin, Schauspieler/Schauspielerin}|pw} in Berlin
                  Staatstheater\oindex{Schauspielhaus Berlin@\textbf{Schauspielhaus Berlin}, \emph{Theater (K.THE)}|pw} engagirt ist? Er fühlt sich dort sehr wohl; er wird wohl
               allmälig nach dem Regisseur und Theaterdirector zu sich entwickeln. Anfangs sah's
               aus, als würd er Kapellmeister werden.\pend
           
\pstart
           Meine Frau\pwindex{Schnitzler, Olga 17.01.1882 – 13.01.1970@\textsc{Schnitzler, Olga} (17.01.1882 – 13.01.1970), \emph{Schauspieler/Schauspielerin, Sänger/Sängerin}|pwv} lebt in Baden-Baden\oindex{Baden-Baden@\textbf{Baden-Baden}, \emph{P.PPL}|pw}; – so bin ich jetzt hier mit meiner
               fünfzehnjährigen aber sehr erwachsenen Tochter Lili\pwindex{Cappellini, Lili 13.09.1909 – 26.07.1928@\textsc{Cappellini, Lili} (13.09.1909 – 26.07.1928)|pw} (Interesse: Sprachen, – Theater, – Geschichte (vor allem Friedrich II\pwindex{Friedrich II. von Preussen 24.01.1712 – 17.08.1786@\textsc{Friedrich II. von Preußen} (24.01.1712 – 17.08.1786), \emph{König/Königin}|pw} und Napoleon\pwindex{Bonaparte, Napoleon 15.08.1769 – 21.05.1821@\textsc{Bonaparte, Napoleon} (15.08.1769 – 21.05.1821), \emph{Kaiser/Kaiserin, Musiker/Musikerin}|pw}) – Eisläufen und Tanzen) allein, sehe aber ziemlich
               viele Menschen – die Hälfte davon \substVorne{}\textsuperscript{selten}\substDazwischen{}kaum\substHinten{} öfter als 1–2 Mal. Auch so liebe Freunde wie Richard Beer Hofmann\pwindex{Beer-Hofmann, Richard 1866-07-11 – 1945-09-26@\textsc{Beer-Hofmann, Richard} (1866-07-11 – 1945-09-26), \emph{Schriftsteller/Schriftstellerin}|pw} seh ich eigentlich selten; – und Hofmannsthal\pwindex{Hofmannsthal, Hugo von 1874-02-01 – 1929-07-15@\textsc{Hofmannsthal, Hugo von} (1874-02-01 – 1929-07-15), \emph{Schriftsteller/Schriftstellerin}|pw} – da gibt es Pausen bis zu einem
               Jahr! B.-H\pwindex{Beer-Hofmann, Richard 1866-07-11 – 1945-09-26@\textsc{Beer-Hofmann, Richard} (1866-07-11 – 1945-09-26), \emph{Schriftsteller/Schriftstellerin}|pw} hat jetzt einen erheblichen Erfolg
               als Regisseur gehabt; er hat ein englisches\oindex{England@\textbf{England}, \emph{A.ADM1}|pw}{ }Stück\pwindex{Ueberfahrt. Schauspiel in drei Akten@\emph{Überfahrt. Schauspiel in drei Akten}|pwv} umgearbeitet {\pb}und \label{K_L02423-1v}\edtext{inszenirt}{\lemma{\textnormal{\emph{inszenirt}}}\Cendnote{\textnormal{Sutton Vanes\pwindex{Vane, Sutton 09.11.1888 – 15.06.1963@\textsc{Vane, Sutton} (09.11.1888 – 15.06.1963), \emph{Schriftsteller/Schriftstellerin}|pwk}{ }\emph{Outward Bound}\pwindex{Ueberfahrt. Schauspiel in drei Akten@\emph{Überfahrt. Schauspiel in drei Akten}|pwk} wurde am 14. 11. 1924 im Theater in der Josefstadt\oindex{Theater in der Josefstadt@\textbf{Theater in der Josefstadt}, \emph{Theater (K.THE)}|pwk} in der Übersetzung von Otto Klement\pwindex{Beer-Hofmann, Richard 1866-07-11 – 1945-09-26@\textsc{Beer-Hofmann, Richard} (1866-07-11 – 1945-09-26), \emph{Schriftsteller/Schriftstellerin}|pwk} – also unter Pseudonym – und in
                  Regie von Beer-Hofmann\pwindex{Beer-Hofmann, Richard 1866-07-11 – 1945-09-26@\textsc{Beer-Hofmann, Richard} (1866-07-11 – 1945-09-26), \emph{Schriftsteller/Schriftstellerin}|pwk} gegeben.}}}\label{K_L02423-1}.
               Seine Tochter Mirjam\pwindex{Beer-Hofmann, Mirjam 04.09.1897 – 24.12.1984@\textsc{Beer-Hofmann, Mirjam} (04.09.1897 – 24.12.1984)|pw} hat geheiratet, und wird
               mit ihrem Gatten\pwindex{Lens, Ernst 19.11.1890 – 1962@\textsc{Lens, Ernst} (19.11.1890 – 1962), \emph{Unternehmer/Unternehmerin}|pwv}
               wahrscheinlich bald nach Kopenhagen\oindex{Kopenhagen@\textbf{Kopenhagen}, \emph{P.PPLC}|pw}
               übersiedeln. –\pend
           
\pstart
           Es erscheinen bald wieder Novellen\pwindex{Frau des Richters. Novelle@\emph{Die Frau des Richters. Novelle}|pwv}\pwindex{Traumnovelle@\emph{Traumnovelle}|pwv} von mir, – und ein Versstück\pwindex{Gang zum Weiher. Dramatische Dichtung@\emph{Der Gang zum Weiher. Dramatische Dichtung}|pwv} wird vielleicht auch bald fertig sein; –
               besonders viel aber feil ich an aphoristisch-fragmentistischem herum – mein
               Bedürfnis, in möglichst \strikeout{kurz} praeciser u conciser
               Form gewisse Lebenswahrheiten auszusprechen – die natürlich an sich nicht neu sind –
               zu denen ich aber meinen eigenen Weg gegangen bin – dieses Bedürfnis wird mit den
               Jahren immer stärker. Es ist auch etwas Pedanterie und etwas Verspieltheit dabei.\pend
           
\pstart
           Ich bin sehr glücklich, dſs Sie immer in gleicher Herzlichkeit meiner gedenken – was
               Sie mir bedeuten, – muſs ich Ihnen das noch sagen? Ich hoffe Sie sind schon ganz wohl
               und der Jesus\pwindex{Jesus 7–4 v. u. Z. – 30/31@\textsc{Jesus} (7–4 v. u. Z. – 30/31), \emph{Wanderprediger/Wanderpredigerin}|pw}\pwindex{Urkristendom@\emph{Urkristendom}|pwv} ist bald vollendet. Was Sie, Georg Brandes, {\pb}in diesem letzten Jahrzehnt gemacht haben – und \uline{wie}
               Sie es gemacht haben –; gibt es dafür in der Geschichte menschlicher Geistesarbeit
               ein Analogon? Und wie menschlich nah sind Sie einem \introOben{}in\introOben{} jedem
               Ihrer Bücher, wie liebt man Sie in jedem! – Und ob Jesus\pwindex{Jesus 7–4 v. u. Z. – 30/31@\textsc{Jesus} (7–4 v. u. Z. – 30/31), \emph{Wanderprediger/Wanderpredigerin}|pw} ein Lebendiger oder ein Mythos war; – in Ihrem Buch\pwindex{Urkristendom@\emph{Urkristendom}|pwv} wird er beides zu sein verstehn. –\pend
           
\pstart
           Im Januar werd ich wahrscheinlich eine Vortragsreise in der Schweiz\oindex{Schweiz@\textbf{Schweiz}, \emph{A.PCLI}|pw} machen. Und wann sieht man einander
               wieder? Sie haben's ja in der Komoedie der Verf.\pwindex{Komoedie der Verfuehrung. In drei Akten@\emph{Komödie der Verführung. In drei Akten}|pw}
               gelesen: das Alter ist nur eine Intrigue, die die Jugend gegen uns einfädelt. In
               meinem nächsten Stück\pwindex{Gang zum Weiher. Dramatische Dichtung@\emph{Der Gang zum Weiher. Dramatische Dichtung}|pwv} soll der
               Neunzigjährige als Sieger übrig bleiben.\pend
           
\pstart
           {\pb}– Schreiben Sie mir bald wieder einen Brief, mein
               verehrter Freund – oder we{\geminationn}s Ihnen leichter von der Hand
               gehen sollte, ein Buch. Es darf ja auch eins über Brandes sein.\pend
           
\pstart
           Seien Sie herzlichst gegrüßt von{\\[\baselineskip]}Ihrem getreuen{\\[\baselineskip]}\spacefill\mbox{Arthur Schnitzler}\pend
           \leftskip=0em{}
\pstart
           \noindent{}Verzeihen Sie die Klexe! Fließende Tinte – neue englische\oindex{England@\textbf{England}, \emph{A.ADM1}|pw} Feder, – Ungeschicklichkeit. –\pend
           \selectlanguage{ngerman}\endnumbering\briefempfaengerindex{Brandes, Georg@\textsc{Brandes, Georg}!zzzSchnitzler, Arthur@\emph{von Arthur Schnitzler}!1924-12-141@{14. 12. 1924}|)be}\mylabel{L02423h}  \normalsize

\doendnotes{C}
\bigskip
\vfill

\clearpage

\footnotesize

\lohead{\textsc{register}}

% Definiere theindex-Environment komplett neu ohne reledmac
\makeatletter
\renewenvironment{theindex}{%
  \section*{\indexname}%
  \setlength{\parindent}{0pt}%
  \setlength{\parskip}{0pt plus 0.3pt}%
  \let\item\@idxitem
}{%
  \clearpage
}
\makeatother

\IfFileExists{\jobname-pw.ind}{\input{\jobname-pw.ind}}{}

\end{document}

      