%% latex-korrekturansicht-vorspann.tex
%% Vorspann für die Korrekturansicht.
%% Lädt die gemeinsame Datei latex-vorspann.tex mit gesetztem Schalter.

\newif\ifkorrekturansicht
\korrekturansichttrue

\input{../tex-inputs/latex-vorspann}


\section[Richard Beer-Hofmann an Arthur Schnitzler, 27. 7. 1901]{L01153 Richard Beer-Hofmann an Arthur Schnitzler, 27. 7. 1901}
\nopagebreak\mylabel{L01153v}
\rehead{ }\normalsize\beginnumbering\briefempfaengerindex{Schnitzler, Arthur@\textsc{Schnitzler, Arthur}!zzzBeer-Hofmann, Richard@\emph{von Richard Beer-Hofmann}!1901-07-272@{27.  7. 1901}|(be}
\toendnotes[C]{\smallbreak\pagebreak[2]}\Standort{CUL, Schnitzler, B 8.}
\physDesc{Brief, 1 Blatt, 2 Seiten, 356 Zeichen
\newline{}Handschrift: blauer Buntstift, lateinische Kurrent
\newline{}Ordnung: mit Bleistift von unbekannter Hand nummeriert:
                                    »165« }\toendnotes[C]{\smallbreak}
\pstart
           \raggedleft{}{\pb}Pörtschach\oindex{Poertschach am Woerthersee@\textbf{Pörtschach am Wörthersee}, \emph{P.PPL}|pw}{ }27/VII 01\pend
           \vspace{0.5em}
\pstart
           Lieber Arthur! Wir haben zusammen dort\oindex{Hotel Stiegl@\textbf{Hotel Stiegl}, \emph{Hotel (K.HTL)}|pwv} gegessen. Ich hatte dort gewohnt. Die Zi{\geminationm}er in der Dépendance \uline{sehr} zu empfehlen. Electr Licht neu eingerichtet.\pend
           
\pstart
           \uline{Stiegl\oindex{Hotel Stiegl@\textbf{Hotel Stiegl}, \emph{Hotel (K.HTL)}|pw}} nicht \uline{Stingl}\pend
           
\pstart
           An Paul\pwindex{Goldmann, Paul 31.01.1865 – 25.09.1935@\textsc{Goldmann, Paul} (31.01.1865 – 25.09.1935), \emph{Schriftsteller/Schriftstellerin, Journalist/Journalistin}|pw} habe ich nach Erhalt Ihres letzten
               Briefes sofort nach Berlin\oindex{Berlin@\textbf{Berlin}, \emph{P.PPLC}|pw} geschrieben, {\pb}keine Nachricht bisher.\pend
           
\pstart
           Bitte schreiben Sie mir so oft \label{T_L01153-1v}\edtext{Sie}{\lemma{\textnormal{\emph{Sie}}}\Cendnote{\textnormal{geschrieben »sie«}}}\label{T_L01153-1} Aufenthalt
               wechseln.\pend
           
\pstart
           Von Herzen{\\[\baselineskip]}Ihr{\\[\baselineskip]}\spacefill\mbox{Richard}\pend
           \leftskip=0em{}\selectlanguage{ngerman}\endnumbering\briefempfaengerindex{Schnitzler, Arthur@\textsc{Schnitzler, Arthur}!zzzBeer-Hofmann, Richard@\emph{von Richard Beer-Hofmann}!1901-07-272@{27.  7. 1901}|)be}\mylabel{L01153h}  \normalsize

\doendnotes{C}
\bigskip
\vfill

\clearpage

\footnotesize

\lohead{\textsc{register}}

% Definiere theindex-Environment komplett neu ohne reledmac
\makeatletter
\renewenvironment{theindex}{%
  \section*{\indexname}%
  \setlength{\parindent}{0pt}%
  \setlength{\parskip}{0pt plus 0.3pt}%
  \let\item\@idxitem
}{%
  \clearpage
}
\makeatother

\IfFileExists{\jobname-pw.ind}{\input{\jobname-pw.ind}}{}

\end{document}

      