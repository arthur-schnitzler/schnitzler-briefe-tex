\input{../tex-inputs/latex-pdf-vorspann}
\begin{center}
            \textcolor{red}{ENTWURF. ENTZIFFERUNG NOCH NICHT KORREKTURGELESEN}
                      \end{center}
            
               \section[Arthur Schnitzler an Richard Beer-Hofmann, 1. 9. 1917]{ Arthur Schnitzler an Richard Beer-Hofmann, 1. 9. 1917}\nopagebreak\mylabel{v}\rehead{ }\begin{ledgroupsized}[t]{13cm}\normalsize\beginnumbering\briefempfaengerindex{Beer-Hofmann, Richard@\textsc{Beer-Hofmann, Richard}!zzzSchnitzler, Arthur@\emph{von Arthur Schnitzler}!1917-09-011@{1. 9. 1917}|(be} \toendnotes[C]{\smallbreak\pagebreak[2]} \Standort{YCGL, MSS 31.}
\physDesc{Postkarte
\newline{}Handschrift: Bleistift, deutsche Kurrent\newline{}Versand: Stempel: »\nobreak{}Wien, 2. IX. 17, X\nobreak{}«.  
\newline{}Beer-Hofmann: mit blauem Buntstift das Datum der Beantwortung
                                    vermerkt: »B. 4./IX 17« }\toendnotes[C]{\smallbreak}\pstart{}{\pb}\textsc{Dr Schnitzler Wien XVIII\oindex{XVIII., Waehring@\textbf{XVIII., Währing}|pw}}\pend{}\pstart{}\textsc{Sternwartestr 71\oindex{Sternwartestrasse@\textbf{Sternwartestraße}|pw}.}\pend{}{\bigskip}\pstart{}\textsc{Herrn Dr. Richard Beer-Hofmann}\pend{}\pstart{}\textsc{Bad Ischl\oindex{Bad Ischl@\textbf{Bad Ischl}|pw}}\pend{}\pstart{}\textsc{Grazerstr 56\oindex{Grazer Strasse@\textbf{Grazer Straße}|pw}}\pend{}{\bigskip}\pstart
           \raggedleft{}{\pb}1. 9. 1917\pend
           \pstart
           lieber Richard, Gustav\pwindex{Schwarzkopf, Gustav 07.11.1853 – 13.11.1939@\textsc{Schwarzkopf, Gustav} (07.11.1853 – 13.11.1939), \emph{Schriftsteller}|pw} erzählt mir dſs Sie vielleicht nach Berlin\oindex{Berlin@\textbf{Berlin}|pw} fahren. Bitte ſchreiben Sie mir nach Partenkirchen\oindex{Partenkirchen@\textbf{Partenkirchen}|pw} (Haus
                  Tannenberg\oindex{Haus Tannenberg@\textbf{Haus Tannenberg}|pw}) wohin ich Montag meiner Frau\pwindex{Schnitzler, Olga 17.01.1882 – 13.01.1970@\textsc{Schnitzler, Olga} (17.01.1882 – 13.01.1970), \emph{Schauspielerin, Sängerin}|pwv} nachreiſe, wann Sie etwa in Berlin\oindex{Berlin@\textbf{Berlin}|pw} sein dürften.\pend
           \pstart
           Haben Sie meinen letzten (zweiten) {\pb}\label{K_L02271-1v}\edtext{Brief}{\lemma{\textnormal{\emph{Brief}}}\Cendnote{\textnormal{siehe Arthur Schnitzler an Richard Beer-Hofmann, 23. 7. 1917}}}\label{K_L02271-1h} erhalten?\pend
           \pstart
           Herzliche Grüße Ihnen und den Ihrigen{\\[\baselineskip]}Ihr{\\[\baselineskip]}\spacefill\mbox{Arth}\pend
           \leftskip=0em{}\endnumbering\briefempfaengerindex{Beer-Hofmann, Richard@\textsc{Beer-Hofmann, Richard}!zzzSchnitzler, Arthur@\emph{von Arthur Schnitzler}!1917-09-011@{1. 9. 1917}|)be}\mylabel{h}\end{ledgroupsized}  \newcommand{\dateiname}{L02271}\newcommand{\titel}{Arthur Schnitzler an Richard Beer-Hofmann, 1. 9. 1917}\newcommand{\editorInnen}{Martin Anton Müller und Gerd-Hermann Susen}\input{../tex-inputs/latex-pdf-abspann}
      