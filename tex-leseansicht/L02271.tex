%% latex-korrekturansicht-vorspann.tex
%% Vorspann für die Korrekturansicht.
%% Lädt die gemeinsame Datei latex-vorspann.tex mit gesetztem Schalter.

\newif\ifkorrekturansicht
\korrekturansichttrue

\input{../tex-inputs/latex-vorspann}


\section[Arthur Schnitzler an Richard Beer-Hofmann, 1. 9. 1917]{L02271 Arthur Schnitzler an Richard Beer-Hofmann, 1. 9. 1917}
\nopagebreak\mylabel{L02271v}
\rehead{ }\normalsize\beginnumbering\briefempfaengerindex{Beer-Hofmann, Richard@\textsc{Beer-Hofmann, Richard}!zzzSchnitzler, Arthur@\emph{von Arthur Schnitzler}!1917-09-011@{1. 9. 1917}|(be}
\toendnotes[C]{\smallbreak\pagebreak[2]}\Standort{YCGL, MSS 31.}
\physDesc{Postkarte, 394 Zeichen
\newline{}Handschrift: 1) Bleistift, deutsche Kurrent\hspace{1em}2) Bleistift, lateinische Kurrent (\noindent{}Adresse)\hspace{1em}
\newline{}Versand: Stempel: »\nobreak{}Wien, 2. IX. 17, X\nobreak{}«.  
\newline{}Beer-Hofmann: mit blauem Buntstift das Datum der Beantwortung
                                    vermerkt: »B. 4./IX 17« }\toendnotes[C]{\smallbreak}\pstart{}{\pb}Dr Schnitzler Wien XVIII\oindex{XVIII., Waehring@\textbf{XVIII., Währing}, \emph{A.ADM3}|pw}\pend{}\pstart{}Sternwartestr 71\oindex{Sternwartestrasse 71@\textbf{Sternwartestraße 71}, \emph{Wohngebäude (K.WHS)}|pw}.\pend{}{\bigskip}\pstart{}Herrn Dr. Richard Beer-Hofmann\pend{}\pstart{}Bad Ischl\oindex{Bad Ischl@\textbf{Bad Ischl}, \emph{P.PPL}|pw}\pend{}\pstart{}Grazerstr 56\oindex{Grazer Strasse [Bad Ischl]@\textbf{Grazer Straße [Bad Ischl]}, \emph{Straße (K.STR)}|pw}\pend{}{\bigskip}\vspace{1em}
\pstart
           \raggedleft{}{\pb}1. 9. 1917\pend
           \vspace{0.5em}
\pstart
           lieber Richard,{ }Gustav\pwindex{Schwarzkopf, Gustav 07.11.1853 – 13.11.1939@\textsc{Schwarzkopf, Gustav} (07.11.1853 – 13.11.1939), \emph{Schriftsteller/Schriftstellerin}|pw} erzählt mir dſs Sie vielleicht nach Berlin\oindex{Berlin@\textbf{Berlin}, \emph{P.PPLC}|pw} fahren. Bitte ſchreiben Sie mir nach Partenkirchen\oindex{Partenkirchen@\textbf{Partenkirchen}, \emph{Teil eines besiedelten Ortes (A.BSOX)}|pw} (Haus Tannenberg\oindex{Haus Tannenberg@\textbf{Haus Tannenberg}, \emph{Beherbergungsgebäude (K.BHB)}|pw}) wohin ich Montag meiner Frau\pwindex{Schnitzler, Olga 17.01.1882 – 13.01.1970@\textsc{Schnitzler, Olga} (17.01.1882 – 13.01.1970), \emph{Schauspieler/Schauspielerin, Sänger/Sängerin}|pwv} nachreiſe, wann Sie etwa in Berlin\oindex{Berlin@\textbf{Berlin}, \emph{P.PPLC}|pw} sein dürften.\pend
           
\pstart
           Haben Sie meinen letzten (zweiten) {\pb}\label{K_L02271-1v}\edtext{Brief}{\lemma{\textnormal{\emph{Brief}}}\Cendnote{\textnormal{Siehe Arthur Schnitzler an Richard Beer-Hofmann, 23. 7. 1917.
               }}}\label{K_L02271-1} erhalten?\pend
           
\pstart
           Herzliche Grüße Ihnen und den Ihrigen{\\[\baselineskip]}Ihr{\\[\baselineskip]}\spacefill\mbox{Arth}\pend
           \leftskip=0em{}\selectlanguage{ngerman}\endnumbering\briefempfaengerindex{Beer-Hofmann, Richard@\textsc{Beer-Hofmann, Richard}!zzzSchnitzler, Arthur@\emph{von Arthur Schnitzler}!1917-09-011@{1. 9. 1917}|)be}\mylabel{L02271h}  \normalsize

\doendnotes{C}
\bigskip
\vfill

\clearpage

\footnotesize

\lohead{\textsc{register}}

% Definiere theindex-Environment komplett neu ohne reledmac
\makeatletter
\renewenvironment{theindex}{%
  \section*{\indexname}%
  \setlength{\parindent}{0pt}%
  \setlength{\parskip}{0pt plus 0.3pt}%
  \let\item\@idxitem
}{%
  \clearpage
}
\makeatother

\IfFileExists{\jobname-pw.ind}{\input{\jobname-pw.ind}}{}

\end{document}

      