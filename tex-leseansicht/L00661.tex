%% latex-korrekturansicht-vorspann.tex
%% Vorspann für die Korrekturansicht.
%% Lädt die gemeinsame Datei latex-vorspann.tex mit gesetztem Schalter.

\newif\ifkorrekturansicht
\korrekturansichttrue

\input{../tex-inputs/latex-vorspann}


\section[Peter Altenberg an Arthur Schnitzler, {[}24. 3. 1897{]}]{L00661 Peter Altenberg an Arthur Schnitzler, {[}24. 3. 1897{]}}
\nopagebreak\mylabel{L00661v}
\rehead{ }\normalsize\beginnumbering\briefempfaengerindex{Schnitzler, Arthur@\textsc{Schnitzler, Arthur}!zzzAltenberg, Peter@\emph{von Peter Altenberg}!1897-03-241@{{[}24. 3. 1897{]}}|(be}
\toendnotes[C]{\smallbreak\pagebreak[2]}\Standort{CUL, Schnitzler, B 2.}
\physDesc{Brief, 1 Blatt, 1 Seite, 229 Zeichen
\newline{}Handschrift: schwarze Tinte, deutsche Kurrent
\newline{}Schnitzler: mit Bleistift datiert: »24/3 97« 
\newline{}Ordnung: mit Bleistift von unbekannter Hand nummeriert:
                                 »5« }\toendnotes[C]{\smallbreak}
\pstart{}{\pb}Lieber \textsc{D\textsuperscript{r}} Arthur Schnitzler:\pend\vspace{0.5em}
\pstart
           danke ſehr für Brief. Konnte nicht kommen, da \textsc{Souper-Rendezvous} hatte. \label{K_L00661-1v}\edtext{Leſe
               natürlich nicht}{\lemma{\textnormal{\emph{Leſe
               natürlich nicht}}}\Cendnote{\textnormal{Am
                     24. 3. 1897 fand die Probe zur öffentlichen Lesung von Hermann Bahr\pwindex{Bahr, Hermann 19.07.1863 – 15.01.1934@\textsc{Bahr, Hermann} (19.07.1863 – 15.01.1934), \emph{Schriftsteller/Schriftstellerin, Kritiker/Kritikerin}|pwk}, Georg Hirschfeld\pwindex{Hirschfeld, Georg 11.02.1873 – 17.01.1942@\textsc{Hirschfeld, Georg} (11.02.1873 – 17.01.1942), \emph{Schriftsteller/Schriftstellerin}|pwk} und Hugo
                     von Hofmannsthal\pwindex{Hofmannsthal, Hugo von 1874-02-01 – 1929-07-15@\textsc{Hofmannsthal, Hugo von} (1874-02-01 – 1929-07-15), \emph{Schriftsteller/Schriftstellerin}|pwk} statt. Auch Altenberg\pwindex{Altenberg, Peter 09.03.1859 – 08.01.1919@\textsc{Altenberg, Peter} (09.03.1859 – 08.01.1919), \emph{Schriftsteller/Schriftstellerin}|pwk} war eingeladen, sagte aber mit diesem Brief einen etwaigen
                  Auftritt bei der Veranstaltung ab. Die Lesung selbst fand am
                     28. 3. 1897 statt.}}}\label{K_L00661-1}. Aber könnte man eine Umſonſt-Karte
               erhalten?! Bitte, laſſen Sie es mir in das \textsc{Pucher-Café}\oindex{Cafe Pucher@\textbf{Café Pucher}, \emph{Kaffeehaus (K.KAF)}|pw}{ }ſagen.\pend
           
\pstart
           Ihr{\\[\baselineskip]}\spacefill\mbox{Peter Altenberg}\pend
           \leftskip=0em{}\selectlanguage{ngerman}\endnumbering\briefempfaengerindex{Schnitzler, Arthur@\textsc{Schnitzler, Arthur}!zzzAltenberg, Peter@\emph{von Peter Altenberg}!1897-03-241@{{[}24. 3. 1897{]}}|)be}\mylabel{L00661h}  \normalsize

\doendnotes{C}
\bigskip
\vfill

\clearpage

\footnotesize

\lohead{\textsc{register}}

% Definiere theindex-Environment komplett neu ohne reledmac
\makeatletter
\renewenvironment{theindex}{%
  \section*{\indexname}%
  \setlength{\parindent}{0pt}%
  \setlength{\parskip}{0pt plus 0.3pt}%
  \let\item\@idxitem
}{%
  \clearpage
}
\makeatother

\IfFileExists{\jobname-pw.ind}{\input{\jobname-pw.ind}}{}

\end{document}

      