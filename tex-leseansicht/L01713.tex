%% latex-korrekturansicht-vorspann.tex
%% Vorspann für die Korrekturansicht.
%% Lädt die gemeinsame Datei latex-vorspann.tex mit gesetztem Schalter.

\newif\ifkorrekturansicht
\korrekturansichttrue

\input{../tex-inputs/latex-vorspann}


\section[Hermann Bahr an Arthur Schnitzler, 29. 9. 1907]{L01713 Hermann Bahr an Arthur Schnitzler, 29. 9. 1907}
\nopagebreak\mylabel{L01713v}
\rehead{ }\normalsize\beginnumbering\briefempfaengerindex{Schnitzler, Arthur@\textsc{Schnitzler, Arthur}!zzzBahr, Hermann@\emph{von Hermann Bahr}!1907-09-291@{29. 9. 1907}|(be}
\toendnotes[C]{\smallbreak\pagebreak[2]}\Standort{CUL, Schnitzler, B 5b.}
\physDesc{Brief, 1 Blatt, 2 Seiten, 672 Zeichen
\newline{}Handschrift Lisa Clarus: blaue Tinte, lateinische Kurrent
\newline{}Handschrift Hermann Bahr: blaue Tinte (\noindent{}Unterschrift)
\newline{}Ordnung: mit Bleistift von unbekannter Hand nummeriert:
                                    »151« }
\buchAbdrucke{\weitereDrucke{Hermann Bahr, Arthur Schnitzler: \emph{Briefwechsel, Aufzeichnungen, Dokumente (1891–1931)}. Göttingen: \emph{Wallstein} 2018, S. 395.} }\toendnotes[C]{\smallbreak}
\pstart
           \raggedleft{}{\pb}29. 9. 07.\pend
           
\pstart\center{}Lieber Arthur!\pend\vspace{0.5em}
\pstart
           Ich habe, seit ich \label{K_L01713-1v}\edtext{zurück}{\lemma{\textnormal{\emph{zurück}}}\Cendnote{\textnormal{Ab dem 4. 9. 1907 verbrachte
                     Bahr\pwindex{Bahr, Hermann 19.07.1863 – 15.01.1934@\textsc{Bahr, Hermann} (19.07.1863 – 15.01.1934), \emph{Schriftsteller/Schriftstellerin, Kritiker/Kritikerin}|pwk} ein paar Tage am Semmering\oindex{Semmering@\textbf{Semmering}, \emph{A.ADM3}|pwk}. Möglicherweise ist das auch auf den Sommerurlaub
                  zu beziehen, von dem er spätestens am 21. 8. 1907 zurückgekehrt
                  war.}}}\label{K_L01713-1} bin, jeden Tag zu Dir wollen, jeden Tag kam was anderes dazwischen und
               ich war so gehetzt, dass es leider wirklich nicht gieng. Nun wieder nach Berlin\oindex{Berlin@\textbf{Berlin}, \emph{P.PPLC}|pw} abreisend, kann ich Dir und Deiner lieben
                  Frau\pwindex{Schnitzler, Olga 17.01.1882 – 13.01.1970@\textsc{Schnitzler, Olga} (17.01.1882 – 13.01.1970), \emph{Schauspieler/Schauspielerin, Sänger/Sängerin}|pwv} nur noch die
               herzlichsten Grüsse und alle guten Wünsche für den Winter schicken. Ich möchte Dir
               noch sagen, dass uns\pwindex{Bahr-Mildenburg, Anna 29.11.1872 – 27.01.1947@\textsc{Bahr-Mildenburg, Anna} (29.11.1872 – 27.01.1947), \emph{Sänger/Sängerin}|pwv} im
               Sommer Dein neues Buch, »Dämmerseelen\pwindex{Daemmerseelen. Novellen@\emph{Dämmerseelen. Novellen}|pw}«, ein sehr
               lieber Gefährte war, und möchte Dich bitten, Dir von Salten\pwindex{Salten, Felix 06.09.1869 – 08.10.1945@\textsc{Salten, Felix} (06.09.1869 – 08.10.1945), \emph{Schriftsteller/Schriftstellerin, Journalist/Journalistin, Chefredakteur/Chefredakteurin}|pw}\strikeout{,} mein neues Stück\pwindex{gelbe Nachtigall@\emph{Die gelbe Nachtigall}|pwv}{ }{\pb}geben zu lassen und es dann an Richard\pwindex{Beer-Hofmann, Richard 1866-07-11 – 1945-09-26@\textsc{Beer-Hofmann, Richard} (1866-07-11 – 1945-09-26), \emph{Schriftsteller/Schriftstellerin}|pw} weiter zu geben; ich habe leider jetzt kein anderes
               Exemplar frei und wünsche sehr, dass Du den Scherz kennen lernen mögest.\pend
           
\pstart
           Herzlichst{\\[\baselineskip]}Dein alter{\\[\baselineskip]}\spacefill\mbox{{[}hs. :{]} Hermann}\pend
           \leftskip=0em{}\selectlanguage{ngerman}\endnumbering\briefempfaengerindex{Schnitzler, Arthur@\textsc{Schnitzler, Arthur}!zzzBahr, Hermann@\emph{von Hermann Bahr}!1907-09-291@{29. 9. 1907}|)be}\mylabel{L01713h}  \normalsize

\doendnotes{C}
\bigskip
\vfill

\clearpage

\footnotesize

\lohead{\textsc{register}}

% Definiere theindex-Environment komplett neu ohne reledmac
\makeatletter
\renewenvironment{theindex}{%
  \section*{\indexname}%
  \setlength{\parindent}{0pt}%
  \setlength{\parskip}{0pt plus 0.3pt}%
  \let\item\@idxitem
}{%
  \clearpage
}
\makeatother

\IfFileExists{\jobname-pw.ind}{\input{\jobname-pw.ind}}{}

\end{document}

      