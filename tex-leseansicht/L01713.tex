\input{../tex-inputs/latex-pdf-vorspann}
\begin{center}
            \textcolor{red}{ENTWURF. ENTZIFFERUNG NOCH NICHT KORREKTURGELESEN}
                      \end{center}
            
               \section[Hermann Bahr an Arthur Schnitzler, 29. 9. 1907]{ Hermann Bahr an Arthur Schnitzler, 29. 9. 1907}\nopagebreak\mylabel{v}\rehead{ }\begin{ledgroupsized}[t]{13cm}\normalsize\beginnumbering\briefempfaengerindex{Schnitzler, Arthur@\textsc{Schnitzler, Arthur}!zzzBahr, Hermann@\emph{von Hermann Bahr}!1907-09-291@{29. 9. 1907}|(be} \toendnotes[C]{\smallbreak\pagebreak[2]} \Standort{CUL, Schnitzler, B 5b.}
\physDesc{Brief, 1 Blatt, 2 Seiten
\newline{}Handschrift Lisa Clarus: blaue Tinte, lateinische Kurrent\newline{}Handschrift Hermann Bahr: blaue Tinte (\noindent{}Unterschrift)\newline{}Ordnung: mit Bleistift von unbekannter Hand nummeriert:
                                    »151« }\buchAbdrucke{\weitereDrucke{Hermann Bahr, Arthur Schnitzler: \emph{Briefwechsel, Aufzeichnungen, Dokumente (1891–1931)}. Hg. Kurt Ifkovits und Martin Anton Müller. Göttingen: \emph{Wallstein} 2018, S. 395.} }\toendnotes[C]{\smallbreak}\pstart
           \raggedleft{}{\pb}29. 9. 07.\pend
           \pstart\center{}Lieber Arthur!\pend\pstart
           Ich habe, seit ich \label{K_L01713_1v}\edtext{zurück}{\lemma{\textnormal{\emph{zurück}}}\Cendnote{\textnormal{Ab dem 4. 9. 1907 verbrachte Bahr\pwindex{Bahr, Hermann 19.07.1863 – 15.01.1934@\textsc{Bahr, Hermann} (19.07.1863 – 15.01.1934), \emph{Schriftsteller, Kritiker}|pwk} ein paar Tage am Semmering\oindex{Semmering@\textbf{Semmering}|pwk}. Möglicherweise ist das auch auf den Sommerurlaub
                  zu beziehen, von dem er spätestens am 21. 8. 1907 zurückgekehrt
                  war.}}}\label{K_L01713_1h} bin, jeden Tag zu Dir wollen, jeden Tag kam was anderes dazwischen und
               ich war so gehetzt, dass es leider wirklich nicht gieng. Nun wieder nach Berlin\oindex{Berlin@\textbf{Berlin}|pw} abreisend, kann ich Dir und Deiner lieben Frau\pwindex{Schnitzler, Olga 17.01.1882 – 13.01.1970@\textsc{Schnitzler, Olga} (17.01.1882 – 13.01.1970), \emph{Schauspielerin, Sängerin}|pwv} nur noch die herzlichsten
               Grüsse und alle guten Wünsche für den Winter schicken. Ich möchte Dir noch sagen,
               dass uns\pwindex{Bahr-Mildenburg, Anna 29.11.1872 – 27.01.1947@\textsc{Bahr-Mildenburg, Anna} (29.11.1872 – 27.01.1947), \emph{Sängerin}|pwv} im Sommer Dein neues
               Buch, »Dämmerseelen\pwindex{Schnitzler, Arthur 15.05.1862 – 21.10.1931@\textsc{Schnitzler, Arthur} (15.05.1862 – 21.10.1931), \emph{Schriftsteller, Mediziner}!Daemmerseelen. Novellen1907@\strich\emph{Dämmerseelen. Novellen} {[}1907{]}|pw}«, ein sehr lieber Gefährte war,
               und möchte Dich bitten, Dir von Salten\pwindex{Salten, Felix 06.09.1869 – 08.10.1945@\textsc{Salten, Felix} (06.09.1869 – 08.10.1945), \emph{Schriftsteller, Journalist}|pw}\strikeout{,} mein neues Stück\pwindex{Bahr, Hermann 19.07.1863 – 15.01.1934@\textsc{Bahr, Hermann} (19.07.1863 – 15.01.1934), \emph{Schriftsteller, Kritiker}!gelbe Nachtigall1907@\strich\emph{Die gelbe Nachtigall} {[}1907{]}|pwv}{ }{\pb}geben zu lassen und es dann an Richard\pwindex{Beer-Hofmann, Richard 11.07.1866 – 26.09.1945@\textsc{Beer-Hofmann, Richard} (11.07.1866 – 26.09.1945), \emph{Schriftsteller}|pw} weiter zu geben; ich habe leider jetzt kein anderes
               Exemplar frei und wünsche sehr, dass Du den Scherz kennen lernen mögest.\pend
           \pstart
           Herzlichst{\\[\baselineskip]}Dein alter{\\[\baselineskip]}\spacefill\mbox{{[}hs. Bahr:{]} Hermann}\pend
           \leftskip=0em{}\endnumbering\briefempfaengerindex{Schnitzler, Arthur@\textsc{Schnitzler, Arthur}!zzzBahr, Hermann@\emph{von Hermann Bahr}!1907-09-291@{29. 9. 1907}|)be}\mylabel{h}\end{ledgroupsized}  \newcommand{\dateiname}{L01713}\newcommand{\titel}{Hermann Bahr an Arthur Schnitzler, 29. 9. 1907}\newcommand{\editorInnen}{ Kurt Ifkovits,  Martin Anton Müller}\input{../tex-inputs/latex-pdf-abspann}
      