%% latex-leseansicht-vorspann.tex
%% Vorspann für die Leseansicht.
%% Lädt die gemeinsame Datei latex-vorspann.tex mit nicht gesetztem Schalter.

\newif\ifkorrekturansicht
\korrekturansichtfalse

\input{../tex-inputs/latex-vorspann}


\section[ Paul Goldmann an Arthur Schnitzler, 6. 4. {[}1901{]}]{L03063 Paul Goldmann an Arthur Schnitzler,  6. 4. [1901]}
\nopagebreak\mylabel{L03063v}
\rehead{ }\normalsize\beginnumbering\briefempfaengerindex{Schnitzler, Arthur@\textsc{Schnitzler, Arthur}!zzzGoldmann, Paul@\emph{von Paul Goldmann}!1901-04-062@{6. 4. [1901]}|(be}
\toendnotes[C]{\smallbreak\pagebreak[2]}
\correspDesc{Versand  durch Paul Goldmann am 6. 4. [1901] in Berlin
\newline{}Erhalt  durch Arthur Schnitzler im Zeitraum [7. 4. 1901
                  – 11. 4. 1901?] in Rom}\toendnotes[C]{\smallbreak}
\Standort{DLA, A:Schnitzler, HS.NZ85.1.3171.}
\physDesc{Brief, 1 Blatt, 4 Seiten, 1404 Zeichen
\newline{}Handschrift: blaue Tinte, deutsche Kurrent
\newline{}Schnitzler: 1) mit Bleistift das Jahr »901« vermerkt  2) mit rotem Buntstift elf Unterstreichungen}\toendnotes[C]{\smallbreak}
\pstart
           \raggedleft{}{\pb}\textcolor{gray}{\textbf{DESSAUERSTRASSE 19}}\oindex{Dessauer Straße@\textbf{Dessauer Straße}, \emph{Straße}|pw}\pend
           
\pstart
           Berlin\oindex{Berlin@\textbf{Berlin}, \emph{Hauptstadt}|pw}, 6. April.\pend
           
\pstart\center{}Mein lieber Freund,\pend\vspace{0.5em}
\pstart
           Alſo Du biſt jetzt in \label{K_L03063-1v}\edtext{Rom\oindex{Rom@\textbf{Rom}, \emph{Hauptstadt}|pw}}{\lemma{\textnormal{\emph{Rom}}}\Cendnote{\textnormal{Schnitzler hielt sich vom 31. 3. 1901 bis zum 11. 4. 1901 in Rom\oindex{Rom@\textbf{Rom}, \emph{Hauptstadt}|pwk} auf.}}}\label{K_L03063-1}, und es iſt gewiß{ }ſehr
               herrlich.\pend
           
\pstart
           Daß \label{K_L03063-2v}\edtext{\textsc{Antoine\pwindex{Antoine, André 31.\,1.\,1858 Limoges – 23.\,10.\,1943 Le Pouliguen@\textsc{Antoine, André} (31.\,1.\,1858 Limoges – 23.\,10.\,1943 Le Pouliguen), \emph{Theaterleiter, Schauspieler}|pw}} die »Gefährtin\pwindex{Schnitzler, Arthur 15.\,5.\,1862 Wien – 21.\,10.\,1931 ebd.@\textsc{Schnitzler, Arthur} (15.\,5.\,1862 Wien – 21.\,10.\,1931 ebd.), \emph{Schriftsteller, Mediziner}!Gefährtin. Schauspiel in einem Akt@\strich\emph{Die Gefährtin. Schauspiel in einem Akt}|pw}« aufführt}{\lemma{\textnormal{\emph{Antoine … aufführt}}}\Cendnote{\textnormal{Schnitzlers Einakter \emph{Die Gefährtin}\pwindex{Schnitzler, Arthur 15.\,5.\,1862 Wien – 21.\,10.\,1931 ebd.@\textsc{Schnitzler, Arthur} (15.\,5.\,1862 Wien – 21.\,10.\,1931 ebd.), \emph{Schriftsteller, Mediziner}!Gefährtin. Schauspiel in einem Akt@\strich\emph{Die Gefährtin. Schauspiel in einem Akt}|pwk} wurde als \emph{La
                     Compagne}\pwindex{Schnitzler, Arthur 15.\,5.\,1862 Wien – 21.\,10.\,1931 ebd.@\textsc{Schnitzler, Arthur} (15.\,5.\,1862 Wien – 21.\,10.\,1931 ebd.), \emph{Schriftsteller, Mediziner}!Compagne. Comédie en une acte@\strich\emph{La Compagne. Comédie en une acte}|pwk} zwischen 29. 4. 1902 und 4. 5. 1902 viermal im Théatre Antoine\oindex{Théâtre Antoine-Simone Berriau@\textbf{Théâtre Antoine-Simone Berriau}, \emph{Theater}|pwk} aufgeführt. Schon im Jahr davor war die Annahme des Stücks\pwindex{Schnitzler, Arthur 15.\,5.\,1862 Wien – 21.\,10.\,1931 ebd.@\textsc{Schnitzler, Arthur} (15.\,5.\,1862 Wien – 21.\,10.\,1931 ebd.), \emph{Schriftsteller, Mediziner}!Gefährtin. Schauspiel in einem Akt@\strich\emph{Die Gefährtin. Schauspiel in einem Akt}|pwkv} in Zeitungen gemeldet
                  worden.}}}\label{K_L03063-2}, haſt Du wohl geleſen.\pend
           
\pstart
           Die kleine \textsc{Dora Speyer\pwindex{Michaelis, Dora 23.\,5.\,1881 Wien – 22.\,1.\,1946 New York City@\textsc{Michaelis, Dora} (23.\,5.\,1881 Wien – 22.\,1.\,1946 New York City)|pw}}{ }ſprach mit mir über ihre Liebe zu Dir. Ich{ }ſagte ihr, Du würdeſt wohl kaum
               heirathen, wenigſtens jetzt nicht{ }ſo bald, und{ }ſie{ }ſolle mit der {\pb}\label{K_L03063-3v}\edtext{Geſchichte}{\lemma{\textnormal{\emph{Geschichte}}}\Cendnote{\textnormal{Siehe XXXX Auszeichnungsfehler: Dokument L03062 nicht gefunden.
               }}}\label{K_L03063-3} fertigzuwerden{ }ſuchen. Das war wohl auch in Deinem Sinne? Hier hat{ }ſich ein
                  Cousin\pwindex{Michaelis, Karl 5.\,1.\,1872 Berlin – 4.\,11.\,1958 Nyon@\textsc{Michaelis, Karl} (5.\,1.\,1872 Berlin – 4.\,11.\,1958 Nyon), \emph{Rechtsanwalt, Chemiker, Patentanwalt}|pwv}, ein \label{K_L03063-4v}\edtext{\textsc{Dr. Michaelis\pwindex{Michaelis, Karl 5.\,1.\,1872 Berlin – 4.\,11.\,1958 Nyon@\textsc{Michaelis, Karl} (5.\,1.\,1872 Berlin – 4.\,11.\,1958 Nyon), \emph{Rechtsanwalt, Chemiker, Patentanwalt}|pw}}}{\lemma{\textnormal{\emph{Dr. Michaelis}}}\Cendnote{\textnormal{Karl Michaelis\pwindex{Michaelis, Karl 5.\,1.\,1872 Berlin – 4.\,11.\,1958 Nyon@\textsc{Michaelis, Karl} (5.\,1.\,1872 Berlin – 4.\,11.\,1958 Nyon), \emph{Rechtsanwalt, Chemiker, Patentanwalt}|pwk}, der spätere Ehemann}}}\label{K_L03063-4},
               wohlhabender Chemiker, in die Kleine\pwindex{Michaelis, Dora 23.\,5.\,1881 Wien – 22.\,1.\,1946 New York City@\textsc{Michaelis, Dora} (23.\,5.\,1881 Wien – 22.\,1.\,1946 New York City)|pwv} verliebt. Sie findet ihn auch{ }ſympathiſch. Ich denke, die
               Conſequenzen w\substVorne{}\textsuperscript{u}\substDazwischen{}e\substHinten{}rden \strikeout{\textcolor{gray}{end}} gezogen werden.\pend
           
\pstart
           Frau \textsc{Frida Strindberg\pwindex{Strindberg, Frida 4.\,4.\,1872 Mondsee – 28.\,6.\,1943 Salzburg@\textsc{Strindberg, Frida} (4.\,4.\,1872 Mondsee – 28.\,6.\,1943 Salzburg)|pw}} hat thatſächlich ein Verhältniß mit dem jungen \textsc{Dr. Evers\pwindex{Ewers, Hanns Heinz 3.\,11.\,1871 Düsseldorf – 12.\,6.\,1943 Berlin@\textsc{Ewers, Hanns Heinz} (3.\,11.\,1871 Düsseldorf – 12.\,6.\,1943 Berlin), \emph{Schriftsteller}|pw}} und wird wohl deswegen \strikeout{in} in Berlin\oindex{Berlin@\textbf{Berlin}, \emph{Hauptstadt}|pw} bleiben.\pend
           
\pstart
           Der Direktor \textsc{Martin\pwindex{Marton, Paul Martin @\textsc{Marton, Paul Martin}, \emph{Schriftsteller, Theaterleiter}|pw}} von der {\pb}Seceſſionsbühne\orgindex{Secessionsbühne@Secessionsbühne|pw}, den wir Beide für einen{ }ſo
               braven Menſchen hielten,{ }ſcheint ein Lump zu{ }ſein. \textsc{Christians\pwindex{Christians, Rudolf 15.\,1.\,1869 Middoge – 7.\,2.\,1921 Pasadena@\textsc{Christians, Rudolf} (15.\,1.\,1869 Middoge – 7.\,2.\,1921 Pasadena), \emph{Schauspieler}|pw}} erzählte mir einige Schweinereien, die er\pwindex{Marton, Paul Martin @\textsc{Marton, Paul Martin}, \emph{Schriftsteller, Theaterleiter}|pwv} gemacht, und{ }ſprach von ihm in Ausdrücken, von denen
               »Zuchthäusler« noch der gelindeſte war.\pend
           
\pstart
           \textsc{Wolzogen}\pwindex{Wolzogen, Ernst von 23.\,4.\,1855 Breslau – 30.\,7.\,1934 Puppling@\textsc{Wolzogen, Ernst von} (23.\,4.\,1855 Breslau – 30.\,7.\,1934 Puppling), \emph{Schriftsteller}|pw} bekommt nächſte Saiſon ein \label{K_L03063-5v}\edtext{eigenes Theater\orgindex{Überbrettl@Überbrettl|pwv}}{\lemma{\textnormal{\emph{eigenes Theater}}}\Cendnote{\textnormal{Gemeint war der Umzug des seit
                  Jahresbeginn 1901 aktiven \emph{Überbrettl}\orgindex{Überbrettl@Überbrettl|pwk} in ein Gebäude in der Köpenicker Straße 68\oindex{Köpenicker Straße@\textbf{Köpenicker Straße}, \emph{Straße}|pwk}.}}}\label{K_L03063-5}. Geldgeber iſt der \textsc{Prof. Stein\pwindex{Stein, Ludwig 12.\,11.\,1859 Benye – 15.\,7.\,1930 Salzburg@\textsc{Stein, Ludwig} (12.\,11.\,1859 Benye – 15.\,7.\,1930 Salzburg), \emph{Philosoph, Soziologe, Publizist}|pw}} aus \textsc{Bern\oindex{Bern@\textbf{Bern}, \emph{Hauptstadt}|pw}}, jener{ }ſeichte philoſophiſche Schwätzer, den Du wohl in {\pb}der N. Fr. Pr.\pwindex{Neue Freie Presse@\emph{Neue Freie Presse}|pw}
               häufig – nicht geleſen haſt. Ich bin gegenwärtig{ }ſehr bemüht, das \label{K_L03063-6v}\edtext{Engagement von Frl. \textsc{Liesl\pwindex{Steinrück, Elisabeth 19.\,11.\,1885 – 7.\,4.\,1920 Partenkirchen@\textsc{Steinrück, Elisabeth} (19.\,11.\,1885 – 7.\,4.\,1920 Partenkirchen)|pw}}}{\lemma{\textnormal{\emph{Engagement … Liesl}}}\Cendnote{\textnormal{Siehe XXXX Auszeichnungsfehler: Dokument L03059 nicht gefunden.
               }}}\label{K_L03063-6} durchzuſetzen, weiß aber nicht, ob es mir gelingen wird.\pend
           
\pstart
           \textsc{Kerr\pwindex{Kerr, Alfred 25.\,12.\,1867 Breslau – 12.\,10.\,1948 Hamburg@\textsc{Kerr, Alfred} (25.\,12.\,1867 Breslau – 12.\,10.\,1948 Hamburg), \emph{Schriftsteller, Kritiker}|pw}} geht Dienſtag nach \textsc{Paris}\oindex{Paris@\textbf{Paris}, \emph{Hauptstadt}|pw}, auf einige Monate. Er möchte rieſig gern \label{K_L03063-7v}\edtext{im Sommer mit uns{ }ſein}{\lemma{\textnormal{\emph{im Sommer mit uns sein}}}\Cendnote{\textnormal{Kerr\pwindex{Kerr, Alfred 25.\,12.\,1867 Breslau – 12.\,10.\,1948 Hamburg@\textsc{Kerr, Alfred} (25.\,12.\,1867 Breslau – 12.\,10.\,1948 Hamburg), \emph{Schriftsteller, Kritiker}|pwk} und Schnitzler sahen sich im Sommer 1901
                  nicht.}}}\label{K_L03063-7}. Das wird{ }ſich ja wohl machen laſſen.\pend
           
\pstart
           Glückliche \label{K_L03063-8v}\edtext{Oſtern}{\lemma{\textnormal{\emph{Ostern}}}\Cendnote{\textnormal{In diesem Jahr wurde Ostern am 7. 4. 1901 gefeiert. Als impliziter Hinweis kann die
                  Stelle so gelesen werden, dass Goldmann\pwindex{Goldmann, Paul 31.\,1.\,1865 Breslau – 25.\,9.\,1935 Wien@\textsc{Goldmann, Paul} (31.\,1.\,1865 Breslau – 25.\,9.\,1935 Wien), \emph{Schriftsteller, Journalist}|pwk}
                  davon ausging, dass Schnitzler den
                  vorliegenden Brief während seiner Reise in Rom\oindex{Rom@\textbf{Rom}, \emph{Hauptstadt}|pwk}
                  erhalten würde und nicht erst nach seiner Rückkehr in Wien\oindex{Wien@\textbf{Wien}, \emph{Verwaltungsgebiet}|pwk}.}}}\label{K_L03063-8}! Viele treue Grüße! {\\[\baselineskip]}Dein \spacefill\mbox{Paul
                  Goldmann.}\pend
           \leftskip=0em{}\selectlanguage{ngerman}\endnumbering\briefempfaengerindex{Schnitzler, Arthur@\textsc{Schnitzler, Arthur}!zzzGoldmann, Paul@\emph{von Paul Goldmann}!1901-04-062@{6. 4. [1901]}|)be}\mylabel{L03063h}  \newcommand{\dateiname}{L03063}\newcommand{\titel}{Paul Goldmann an Arthur Schnitzler, 6. 4. [1901]}\newcommand{\editorInnen}{Martin Anton Müller und Laura Untner}%% latex-leseansicht-abspann.tex
%% Abspann für die Leseansicht.
%% Der Schalter \ifkorrekturansicht ist bereits durch den Vorspann gesetzt.

%% latex-abspann.tex
%% Gemeinsamer Abspann für Korrekturansicht und Leseansicht.
%% Setzt den Schalter \ifkorrekturansicht voraus (gesetzt in den
%% einbindenden Dateien latex-korrekturansicht-abspann.tex bzw.
%% latex-leseansicht-abspann.tex).
%% ---------------------------------------------------------------

\normalsize

% Das esempio-Environment wird nur in der Leseansicht benötigt
\ifkorrekturansicht\else
\newenvironment{esempio}[3]%
{
    \vspace{1.5ex}
    \rlap{\underline{#1}}
    \par
    \setlength{\parindent}{0cm}
    \nopagebreak
    \leftskip=#2cm
    \rightskip=#3cm
}
{
    \par
}
\fi

\doendnotes{C}
\bigskip
\vfill

\clearpage

\footnotesize

\ifkorrekturansicht
  \lohead{\textsc{register}}
\fi

% theindex-Environment neu definieren ohne reledmac
\makeatletter
\renewenvironment{theindex}{%
  \ifkorrekturansicht
    \section*{\indexname}%
  \else
    \subsubsection*{Index der erwähnten Entitäten}%
  \fi
  \setlength{\parindent}{0pt}%
  \setlength{\parskip}{0pt plus 0.3pt}%
  \let\item\@idxitem
}{%
  \ifkorrekturansicht\clearpage\fi
}
\makeatother

\IfFileExists{\jobname-pw.ind}{\input{\jobname-pw.ind}}{}

% Quellenangabe nur in der Leseansicht
\ifkorrekturansicht\else
% Fallback-Definitionen, falls die .tex-Datei \titel etc. nicht gesetzt hat
\providecommand{\titel}{}
\providecommand{\editorInnen}{}
\providecommand{\dateiname}{\jobname}

\vspace{3cm}

\vfill

\footnotesize
\textsc{Quelle}: \titel. Herausgegeben von {\editorInnen}. In: \emph{Arthur Schnitzler: Briefwechsel mit Autorinnen und Autoren}.
 Digitale Edition, https://schnitzler-briefe.acdh.oeaw.ac.at/{\dateiname}.html (Stand \today)
\fi

\end{document}


