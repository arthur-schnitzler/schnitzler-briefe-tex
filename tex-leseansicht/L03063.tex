%% latex-korrekturansicht-vorspann.tex
%% Vorspann für die Korrekturansicht.
%% Lädt die gemeinsame Datei latex-vorspann.tex mit gesetztem Schalter.

\newif\ifkorrekturansicht
\korrekturansichttrue

\input{../tex-inputs/latex-vorspann}


\section[ Paul Goldmann an Arthur Schnitzler, 6. 4. {[}1901{]}]{L03063 Paul Goldmann an Arthur Schnitzler, 6. 4. {[}1901{]}}
\nopagebreak\mylabel{L03063v}
\rehead{ }\normalsize\beginnumbering\briefempfaengerindex{Schnitzler, Arthur@\textsc{Schnitzler, Arthur}!zzzGoldmann, Paul@\emph{von Paul Goldmann}!1901-04-062@{6. 4. {[}1901{]}}|(be}
\toendnotes[C]{\smallbreak\pagebreak[2]}\Standort{DLA, A:Schnitzler, HS.NZ85.1.3171.}
\physDesc{Brief, 1 Blatt, 4 Seiten, 1404 Zeichen
\newline{}Handschrift: blaue Tinte, deutsche Kurrent
\newline{}Schnitzler: 1) mit Bleistift das Jahr »901« vermerkt  2) mit rotem Buntstift elf Unterstreichungen}\toendnotes[C]{\smallbreak}
\pstart
           \raggedleft{}{\pb}\textcolor{gray}{\textbf{DESSAUERSTRASSE 19}}\oindex{Dessauer Strasse@\textbf{Dessauer Straße}, \emph{Straße (K.STR)}|pw}\pend
           
\pstart
           Berlin\oindex{Berlin@\textbf{Berlin}, \emph{P.PPLC}|pw}, 6. April.\pend
           
\pstart\center{}Mein lieber Freund,\pend\vspace{0.5em}
\pstart
           Alſo Du biſt jetzt in \label{K_L03063-1v}\edtext{Rom\oindex{Rom@\textbf{Rom}, \emph{P.PPLC}|pw}}{\lemma{\textnormal{\emph{Rom}}}\Cendnote{\textnormal{Schnitzler hielt sich vom 31. 3. 1901 bis zum 11. 4. 1901 in Rom\oindex{Rom@\textbf{Rom}, \emph{P.PPLC}|pwk} auf.}}}\label{K_L03063-1}, und es iſt gewiß ſehr
               herrlich.\pend
           
\pstart
           Daß \label{K_L03063-2v}\edtext{\textsc{Antoine\pwindex{Antoine, Andre 1858-01-31 – 1943-10-23@\textsc{Antoine, André} (1858-01-31 – 1943-10-23), \emph{Theaterleiter/Theaterleiterin, Schauspieler/Schauspielerin}|pw}} die »Gefährtin\pwindex{Gefaehrtin. Schauspiel in einem Akt@\emph{Die Gefährtin. Schauspiel in einem Akt}|pw}« aufführt}{\lemma{\textnormal{\emph{Antoine … aufführt}}}\Cendnote{\textnormal{Schnitzlers Einakter \emph{Die Gefährtin}\pwindex{Gefaehrtin. Schauspiel in einem Akt@\emph{Die Gefährtin. Schauspiel in einem Akt}|pwk} wurde als \emph{La
                     Compagne}\pwindex{Compagne. Comedie en une acte@\emph{La Compagne. Comédie en une acte}|pwk} zwischen 29. 4. 1902 und 4. 5. 1902 viermal im Théatre Antoine\oindex{Theâtre Antoine-Simone Berriau@\textbf{Théâtre Antoine-Simone Berriau}, \emph{Theater (K.THE)}|pwk} aufgeführt. Schon im Jahr davor war die Annahme des Stücks\pwindex{Gefaehrtin. Schauspiel in einem Akt@\emph{Die Gefährtin. Schauspiel in einem Akt}|pwkv} in Zeitungen gemeldet
                  worden.}}}\label{K_L03063-2}, haſt Du wohl geleſen.\pend
           
\pstart
           Die kleine \textsc{Dora Speyer\pwindex{Michaelis, Dora 23.05.1881 – 22.01.1946@\textsc{Michaelis, Dora} (23.05.1881 – 22.01.1946)|pw}} ſprach mit mir über ihre Liebe zu Dir. Ich ſagte ihr, Du würdeſt wohl kaum
               heirathen, wenigſtens jetzt nicht ſo bald, und ſie ſolle mit der {\pb}\label{K_L03063-3v}\edtext{Geſchichte}{\lemma{\textnormal{\emph{Geſchichte}}}\Cendnote{\textnormal{Siehe Paul Goldmann an Arthur Schnitzler, 21. 3. [1901].
               }}}\label{K_L03063-3} fertigzuwerden ſuchen. Das war wohl auch in Deinem Sinne? Hier hat ſich ein
                  Cousin\pwindex{Michaelis, Karl 05.01.1872 – 04.11.1958@\textsc{Michaelis, Karl} (05.01.1872 – 04.11.1958), \emph{Rechtsanwalt/Rechtsanwältin, Chemiker/Chemikerin, Patentanwalt/Patentanwältin}|pwv}, ein \label{K_L03063-4v}\edtext{\textsc{Dr. Michaelis\pwindex{Michaelis, Karl 05.01.1872 – 04.11.1958@\textsc{Michaelis, Karl} (05.01.1872 – 04.11.1958), \emph{Rechtsanwalt/Rechtsanwältin, Chemiker/Chemikerin, Patentanwalt/Patentanwältin}|pw}}}{\lemma{\textnormal{\emph{Dr. Michaelis}}}\Cendnote{\textnormal{Karl Michaelis\pwindex{Michaelis, Karl 05.01.1872 – 04.11.1958@\textsc{Michaelis, Karl} (05.01.1872 – 04.11.1958), \emph{Rechtsanwalt/Rechtsanwältin, Chemiker/Chemikerin, Patentanwalt/Patentanwältin}|pwk}, der spätere Ehemann}}}\label{K_L03063-4},
               wohlhabender Chemiker, in die Kleine\pwindex{Michaelis, Dora 23.05.1881 – 22.01.1946@\textsc{Michaelis, Dora} (23.05.1881 – 22.01.1946)|pwv} verliebt. Sie findet ihn auch ſympathiſch. Ich denke, die
               Conſequenzen w\substVorne{}\textsuperscript{u}\substDazwischen{}e\substHinten{}rden \strikeout{\textcolor{gray}{end}} gezogen werden.\pend
           
\pstart
           Frau \textsc{Frida Strindberg\pwindex{Strindberg, Frida 04.04.1872 – 28.06.1943@\textsc{Strindberg, Frida} (04.04.1872 – 28.06.1943)|pw}} hat thatſächlich ein Verhältniß mit dem jungen \textsc{Dr. Evers\pwindex{Ewers, Hanns Heinz 03.11.1871 – 12.06.1943@\textsc{Ewers, Hanns Heinz} (03.11.1871 – 12.06.1943), \emph{Schriftsteller/Schriftstellerin}|pw}} und wird wohl deswegen \strikeout{in} in Berlin\oindex{Berlin@\textbf{Berlin}, \emph{P.PPLC}|pw} bleiben.\pend
           
\pstart
           Der Direktor \textsc{Martin\pwindex{Marton, Paul Martin @\textsc{Marton, Paul Martin}, \emph{Schriftsteller/Schriftstellerin, Theaterleiter/Theaterleiterin}|pw}} von der {\pb}Seceſſionsbühne\orgindex{Secessionsbuehne@Secessionsbühne|pw}, den wir Beide für einen ſo
               braven Menſchen hielten, ſcheint ein Lump zu ſein. \textsc{Christians\pwindex{Christians, Rudolf 15.01.1869 – 07.02.1921@\textsc{Christians, Rudolf} (15.01.1869 – 07.02.1921), \emph{Schauspieler/Schauspielerin}|pw}} erzählte mir einige Schweinereien, die er\pwindex{Marton, Paul Martin @\textsc{Marton, Paul Martin}, \emph{Schriftsteller/Schriftstellerin, Theaterleiter/Theaterleiterin}|pwv} gemacht, und ſprach von ihm in Ausdrücken, von denen
               »Zuchthäusler« noch der gelindeſte war.\pend
           
\pstart
           \textsc{Wolzogen}\pwindex{Wolzogen, Ernst von 23.04.1855 – 30.07.1934@\textsc{Wolzogen, Ernst von} (23.04.1855 – 30.07.1934), \emph{Schriftsteller/Schriftstellerin}|pw} bekommt nächſte Saiſon ein \label{K_L03063-5v}\edtext{eigenes Theater\orgindex{Ueberbrettl@Überbrettl|pwv}}{\lemma{\textnormal{\emph{eigenes Theater}}}\Cendnote{\textnormal{Gemeint war der Umzug des seit
                  Jahresbeginn 1901 aktiven \emph{Überbrettl}\orgindex{Ueberbrettl@Überbrettl|pwk} in ein Gebäude in der Köpenicker Straße 68\oindex{Koepenicker Strasse@\textbf{Köpenicker Straße}, \emph{Straße (K.STR)}|pwk}.}}}\label{K_L03063-5}. Geldgeber iſt der \textsc{Prof. Stein\pwindex{Stein, Ludwig 12.11.1859 – 15.07.1930@\textsc{Stein, Ludwig} (12.11.1859 – 15.07.1930), \emph{Philosoph/Philosophin, Soziologe/Soziologin, Publizist/Publizistin}|pw}} aus \textsc{Bern\oindex{Bern@\textbf{Bern}, \emph{P.PPLC}|pw}}, jener ſeichte philoſophiſche Schwätzer, den Du wohl in {\pb}der N. Fr. Pr.\pwindex{Neue Freie Presse@\emph{Neue Freie Presse}|pw}
               häufig – nicht geleſen haſt. Ich bin gegenwärtig ſehr bemüht, das \label{K_L03063-6v}\edtext{Engagement von Frl. \textsc{Liesl\pwindex{Steinrueck, Elisabeth 19.11.1885 – 07.04.1920@\textsc{Steinrück, Elisabeth} (19.11.1885 – 07.04.1920)|pw}}}{\lemma{\textnormal{\emph{Engagement … Liesl}}}\Cendnote{\textnormal{Siehe Paul Goldmann an Arthur Schnitzler, 18. 2. [1901].
               }}}\label{K_L03063-6} durchzuſetzen, weiß aber nicht, ob es mir gelingen wird.\pend
           
\pstart
           \textsc{Kerr\pwindex{Kerr, Alfred 25.12.1867 – 12.10.1948@\textsc{Kerr, Alfred} (25.12.1867 – 12.10.1948), \emph{Schriftsteller/Schriftstellerin, Kritiker/Kritikerin}|pw}} geht Dienſtag nach \textsc{Paris}\oindex{Paris@\textbf{Paris}, \emph{P.PPLC}|pw}, auf einige Monate. Er möchte rieſig gern \label{K_L03063-7v}\edtext{im Sommer mit uns ſein}{\lemma{\textnormal{\emph{im Sommer mit uns ſein}}}\Cendnote{\textnormal{Kerr\pwindex{Kerr, Alfred 25.12.1867 – 12.10.1948@\textsc{Kerr, Alfred} (25.12.1867 – 12.10.1948), \emph{Schriftsteller/Schriftstellerin, Kritiker/Kritikerin}|pwk} und Schnitzler sahen sich im Sommer 1901
                  nicht.}}}\label{K_L03063-7}. Das wird ſich ja wohl machen laſſen.\pend
           
\pstart
           Glückliche \label{K_L03063-8v}\edtext{Oſtern}{\lemma{\textnormal{\emph{Oſtern}}}\Cendnote{\textnormal{In diesem Jahr wurde Ostern am 7. 4. 1901 gefeiert. Als impliziter Hinweis kann die
                  Stelle so gelesen werden, dass Goldmann\pwindex{Goldmann, Paul 31.01.1865 – 25.09.1935@\textsc{Goldmann, Paul} (31.01.1865 – 25.09.1935), \emph{Schriftsteller/Schriftstellerin, Journalist/Journalistin}|pwk}
                  davon ausging, dass Schnitzler den
                  vorliegenden Brief während seiner Reise in Rom\oindex{Rom@\textbf{Rom}, \emph{P.PPLC}|pwk}
                  erhalten würde und nicht erst nach seiner Rückkehr in Wien\oindex{Wien@\textbf{Wien}, \emph{A.ADM2}|pwk}.}}}\label{K_L03063-8}! Viele treue Grüße! {\\[\baselineskip]}Dein \spacefill\mbox{Paul
                  Goldmann.}\pend
           \leftskip=0em{}\selectlanguage{ngerman}\endnumbering\briefempfaengerindex{Schnitzler, Arthur@\textsc{Schnitzler, Arthur}!zzzGoldmann, Paul@\emph{von Paul Goldmann}!1901-04-062@{6. 4. {[}1901{]}}|)be}\mylabel{L03063h}  \normalsize

\doendnotes{C}
\bigskip
\vfill

\clearpage

\footnotesize

\lohead{\textsc{register}}

% Definiere theindex-Environment komplett neu ohne reledmac
\makeatletter
\renewenvironment{theindex}{%
  \section*{\indexname}%
  \setlength{\parindent}{0pt}%
  \setlength{\parskip}{0pt plus 0.3pt}%
  \let\item\@idxitem
}{%
  \clearpage
}
\makeatother

\IfFileExists{\jobname-pw.ind}{\input{\jobname-pw.ind}}{}

\end{document}

      