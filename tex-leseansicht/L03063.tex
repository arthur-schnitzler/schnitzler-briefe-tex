%% latex-leseansicht-vorspann.tex
%% Vorspann für die Leseansicht.
%% Lädt die gemeinsame Datei latex-vorspann.tex mit nicht gesetztem Schalter.

\newif\ifkorrekturansicht
\korrekturansichtfalse

\input{../tex-inputs/latex-vorspann}


         
         \renewcommand{\erwaehntePersonen}{Personen: André Antoine, Rudolf Christians, Hanns Heinz Ewers, Paul Goldmann, Alfred Kerr, Paul Martin Marton, Dora Michaelis, Karl Michaelis, Ludwig Stein, Elisabeth Steinrück, Frida Strindberg, Ernst von Wolzogen}
         \renewcommand{\erwaehnteInstitutionen}{Institutionen: Secessionsbühne, Überbrettl}
         \renewcommand{\erwaehnteOrte}{Orte: Berlin, Bern, Dessauer Straße, Köpenicker Straße, Paris, Rom, Théâtre Antoine-Simone Berriau, Wien}
         \renewcommand{\erwaehnteWerke}{Werke: Die Gefährtin. Schauspiel in einem Akt, La Compagne. Comédie en une acte, Neue Freie Presse}
               \section[ Paul Goldmann an Arthur Schnitzler, 6. 4. {[}1901{]}]{ Paul Goldmann an Arthur Schnitzler, 6. 4. {[}1901{]}}\nopagebreak\mylabel{v}\rehead{ }\begin{ledgroupsized}[t]{13cm}\normalsize\beginnumbering\briefempfaengerindex{Schnitzler, Arthur@\textsc{Schnitzler, Arthur}!zzzGoldmann, Paul@\emph{von Paul Goldmann}!1901-04-062@{6. 4. {[}1901{]}}|(be} \toendnotes[C]{\smallbreak\pagebreak[2]} \Standort{DLA, A:Schnitzler, HS.NZ85.1.3171.}
\physDesc{Brief, 1 Blatt, 4 Seiten, 1404 Zeichen
\newline{}Handschrift: blaue Tinte, deutsche Kurrent
\newline{}Schnitzler: 1) mit Bleistift das Jahr »901« vermerkt  2) mit rotem Buntstift elf Unterstreichungen}\toendnotes[C]{\smallbreak}\pstart
           \noindent{}\raggedleft{}{\pb}\textcolor{gray}{\textbf{DESSAUERSTRASSE 19}}\oindex{Dessauer Strasse@\textbf{Dessauer Straße}|pw}\pend
           \pstart
           Berlin\oindex{Berlin@\textbf{Berlin}|pw}, 6. April.\pend
           \pstart\center{}Mein lieber Freund,\pend\pstart
           Alſo Du biſt jetzt in \label{K_L03063-1v}\edtext{Rom\oindex{Rom@\textbf{Rom}|pw}}{\lemma{\textnormal{\emph{Rom}}}\Cendnote{\textnormal{Schnitzler\pwindex{Schnitzler, Arthur 15.05.1862 – 21.10.1931@\textsc{Schnitzler, Arthur} (15.05.1862 – 21.10.1931), \emph{Schriftsteller, Mediziner}|pwk} hielt sich von 31. 3. 1901 bis 11. 4. 1901 in Rom\oindex{Rom@\textbf{Rom}|pwk} auf.}}}\label{K_L03063-1h}, und es iſt gewiß ſehr
               herrlich.\pend
           \pstart
           Daß \label{K_L03063-2v}\edtext{\textsc{Antoine\pwindex{Antoine, Andre 1858-01-31 – 1943-10-23@\textsc{Antoine, André} (1858-01-31 – 1943-10-23), \emph{Theaterleiter, Schauspieler}|pw}} die »Gefährtin\pwindex{Schnitzler, Arthur 15.05.1862 – 21.10.1931@\textsc{Schnitzler, Arthur} (15.05.1862 – 21.10.1931), \emph{Schriftsteller, Mediziner}!Gefaehrtin. Schauspiel in einem Akt1899-03-01@\strich\emph{Die Gefährtin. Schauspiel in einem Akt} {[}1899-03-01{]}|pw}« aufführt}{\lemma{\textnormal{\emph{Antoine … aufführt}}}\Cendnote{\textnormal{Schnitzler\pwindex{Schnitzler, Arthur 15.05.1862 – 21.10.1931@\textsc{Schnitzler, Arthur} (15.05.1862 – 21.10.1931), \emph{Schriftsteller, Mediziner}|pwk}s Einakter \emph{Die Gefährtin}\pwindex{Schnitzler, Arthur 15.05.1862 – 21.10.1931@\textsc{Schnitzler, Arthur} (15.05.1862 – 21.10.1931), \emph{Schriftsteller, Mediziner}!Gefaehrtin. Schauspiel in einem Akt1899-03-01@\strich\emph{Die Gefährtin. Schauspiel in einem Akt} {[}1899-03-01{]}|pwk} wurde als \emph{La
                     Compagne}\pwindex{\textcolor{red}{\textsuperscript{XXXX1 indx}}!Compagne. Comedie en une acte1902-04-29@\strich\emph{La Compagne. Comédie en une acte} {[}Übersetzung, 1902-04-29{]}|pwk} zwischen 29. 4. 1902 und 4. 5. 1902 vier Mal im Théatre Antoine\oindex{Theâtre Antoine-Simone Berriau@\textbf{Théâtre Antoine-Simone Berriau}|pwk} aufgeführt. Schon im Jahr davor war die Annahme des Stück\pwindex{Schnitzler, Arthur 15.05.1862 – 21.10.1931@\textsc{Schnitzler, Arthur} (15.05.1862 – 21.10.1931), \emph{Schriftsteller, Mediziner}!Gefaehrtin. Schauspiel in einem Akt1899-03-01@\strich\emph{Die Gefährtin. Schauspiel in einem Akt} {[}1899-03-01{]}|pwkv}s in Zeitungen gemeldet
                  worden.}}}\label{K_L03063-2h}, haſt Du wohl geleſen.\pend
           \pstart
           Die kleine \textsc{Dora Speyer\pwindex{Michaelis, Dora 23.05.1881 – 22.01.1946@\textsc{Michaelis, Dora} (23.05.1881 – 22.01.1946)|pw}} ſprach mit mir über ihre Liebe zu Dir. Ich ſagte ihr, Du würdeſt wohl kaum
               heirathen, wenigſtens jetzt nicht ſo bald, und ſie ſolle mit der {\pb}\label{K_L03063-3v}\edtext{Geſchichte}{\lemma{\textnormal{\emph{Geſchichte}}}\Cendnote{\textnormal{siehe Paul Goldmann an Arthur Schnitzler, 21. 3. [1901]}}}\label{K_L03063-3h} fertigzuwerden ſuchen. Das war wohl auch in Deinem Sinne? Hier hat ſich ein
                  Cousin\pwindex{Michaelis, Karl 05.01.1872 – 04.11.1958@\textsc{Michaelis, Karl} (05.01.1872 – 04.11.1958), \emph{Rechtsanwalt, Chemiker, Patentanwalt}|pwv}, ein \label{K_L03063-4v}\edtext{\textsc{Dr. Michaelis\pwindex{Michaelis, Karl 05.01.1872 – 04.11.1958@\textsc{Michaelis, Karl} (05.01.1872 – 04.11.1958), \emph{Rechtsanwalt, Chemiker, Patentanwalt}|pw}}}{\lemma{\textnormal{\emph{Dr. Michaelis}}}\Cendnote{\textnormal{Karl Michaelis\pwindex{Michaelis, Karl 05.01.1872 – 04.11.1958@\textsc{Michaelis, Karl} (05.01.1872 – 04.11.1958), \emph{Rechtsanwalt, Chemiker, Patentanwalt}|pwk}, der spätere Ehemann}}}\label{K_L03063-4h},
               wohlhabender Chemiker, in die Kleine\pwindex{Michaelis, Dora 23.05.1881 – 22.01.1946@\textsc{Michaelis, Dora} (23.05.1881 – 22.01.1946)|pwv} verliebt. Sie findet ihn auch ſympathiſch. Ich denke, die
               Conſequenzen w\substVorne{}\textsuperscript{u}\substDazwischen{}e\substHinten{}rden \strikeout{\textcolor{gray}{end}} gezogen werden.\pend
           \pstart
           Frau \textsc{Frida Strindberg\pwindex{Strindberg, Frida 04.04.1872 – 28.06.1943@\textsc{Strindberg, Frida} (04.04.1872 – 28.06.1943)|pw}} hat thatſächlich ein Verhältniß mit dem jungen \textsc{Dr. Evers\pwindex{Ewers, Hanns Heinz 03.11.1871 – 12.06.1943@\textsc{Ewers, Hanns Heinz} (03.11.1871 – 12.06.1943), \emph{Schriftsteller}|pw}} und wird wohl deswegen \strikeout{in} in Berlin\oindex{Berlin@\textbf{Berlin}|pw} bleiben.\pend
           \pstart
           Der Direktor \textsc{Martin\pwindex{Marton, Paul Martin @\textsc{Marton, Paul Martin}, \emph{Schriftsteller, Theaterleiter}|pw}} von der {\pb}Seceſſionsbühne\orgindex{Secessionsbuehne@Secessionsbühne|pw}, den wir Beide für einen ſo
               braven Menſchen hielten, ſcheint ein Lump zu ſein. \textsc{Christians\pwindex{Christians, Rudolf 15.01.1869 – 07.02.1921@\textsc{Christians, Rudolf} (15.01.1869 – 07.02.1921), \emph{Schauspieler}|pw}} erzählte mir einige Schweinereien, die er\pwindex{Marton, Paul Martin @\textsc{Marton, Paul Martin}, \emph{Schriftsteller, Theaterleiter}|pwv} gemacht, und ſprach von ihm in Ausdrücken, von denen
               »Zuchthäusler« noch der gelindeſte war.\pend
           \pstart
           \textsc{Wolzogen}\pwindex{Wolzogen, Ernst von 23.04.1855 – 30.07.1934@\textsc{Wolzogen, Ernst von} (23.04.1855 – 30.07.1934), \emph{Schriftsteller}|pw} bekommt nächſte Saiſon ein \label{K_L03063-5v}\edtext{eigenes Theater\orgindex{Ueberbrettl@Überbrettl|pwv}}{\lemma{\textnormal{\emph{eigenes Theater}}}\Cendnote{\textnormal{Gemeint war der Umzug des seit
                  Jahresbeginn 1901 aktiven \emph{Überbrettl}\orgindex{Ueberbrettl@Überbrettl|pwk} in ein Gebäude in der Köpenicker Straße 68\oindex{Koepenicker Strasse@\textbf{Köpenicker Straße}|pwk}.}}}\label{K_L03063-5h}. Geldgeber iſt der \textsc{Prof. Stein\pwindex{Stein, Ludwig 12.11.1859 – 15.07.1930@\textsc{Stein, Ludwig} (12.11.1859 – 15.07.1930), \emph{Philosoph, Soziologe, Publizist}|pw}} aus \textsc{Bern\oindex{Bern@\textbf{Bern}|pw}}, jener ſeichte philoſophiſche Schwätzer, den Du wohl in {\pb}der N. Fr. Pr.\pwindex{Neue Freie Presse1864 – 1939@\emph{Neue Freie Presse} {[}1864 – 1939{]}|pw}
               häufig – nicht geleſen haſt. Ich bin gegenwärtig ſehr bemüht, das \label{K_L03063-6v}\edtext{Engagement von Frl. \textsc{Liesl\pwindex{Steinrueck, Elisabeth 19.11.1885 – 07.04.1920@\textsc{Steinrück, Elisabeth} (19.11.1885 – 07.04.1920)|pw}}}{\lemma{\textnormal{\emph{Engagement … Liesl}}}\Cendnote{\textnormal{siehe Paul Goldmann an Arthur Schnitzler, 18. 2. [1901]}}}\label{K_L03063-6h} durchzuſetzen, weiß aber nicht, ob es mir gelingen wird.\pend
           \pstart
           \textsc{Kerr\pwindex{Kerr, Alfred 25.12.1867 – 12.10.1948@\textsc{Kerr, Alfred} (25.12.1867 – 12.10.1948), \emph{Schriftsteller, Kritiker}|pw}} geht Dienſtag nach \textsc{Paris}\oindex{Paris@\textbf{Paris}|pw}, auf einige Monate. Er möchte rieſig gern \label{K_L03063-7v}\edtext{im Sommer mit uns ſein}{\lemma{\textnormal{\emph{im Sommer mit uns ſein}}}\Cendnote{\textnormal{Kerr\pwindex{Kerr, Alfred 25.12.1867 – 12.10.1948@\textsc{Kerr, Alfred} (25.12.1867 – 12.10.1948), \emph{Schriftsteller, Kritiker}|pwk} und Schnitzler\pwindex{Schnitzler, Arthur 15.05.1862 – 21.10.1931@\textsc{Schnitzler, Arthur} (15.05.1862 – 21.10.1931), \emph{Schriftsteller, Mediziner}|pwk} sahen sich im Sommer 1901
                  nicht.}}}\label{K_L03063-7h}. Das wird ſich ja wohl machen laſſen.\pend
           \pstart
           Glückliche \label{K_L03063-8v}\edtext{Oſtern}{\lemma{\textnormal{\emph{Oſtern}}}\Cendnote{\textnormal{In diesem Jahr wurde Ostern am 7. 4. 1901 gefeiert. Als impliziter Hinweis kann die
                  Stelle so gelesen werden, dass Goldmann\pwindex{Goldmann, Paul 31.01.1865 – 25.09.1935@\textsc{Goldmann, Paul} (31.01.1865 – 25.09.1935), \emph{Schriftsteller, Journalist}|pwk}
                  davon ausging, dass Schnitzler\pwindex{Schnitzler, Arthur 15.05.1862 – 21.10.1931@\textsc{Schnitzler, Arthur} (15.05.1862 – 21.10.1931), \emph{Schriftsteller, Mediziner}|pwk} den
                  vorliegenden Brief während seiner Reise in Rom\oindex{Rom@\textbf{Rom}|pwk}
                  erhalten würde und nicht erst nach seiner Rückkehr in Wien\oindex{Wien@\textbf{Wien}|pwk}.}}}\label{K_L03063-8h}! Viele treue Grüße! {\\[\baselineskip]}Dein \spacefill\mbox{Paul
                  Goldmann.}\pend
           \leftskip=0em{}
         
         \endnumbering\mylabel{h}\end{ledgroupsized}  \newcommand{\dateiname}{L03063}\newcommand{\titel}{Paul Goldmann an Arthur Schnitzler, 6. 4. [1901]}\newcommand{\editorInnen}{Martin Anton Müller und Laura Untner}%% latex-leseansicht-abspann.tex
%% Abspann für die Leseansicht.
%% Der Schalter \ifkorrekturansicht ist bereits durch den Vorspann gesetzt.

%% latex-abspann.tex
%% Gemeinsamer Abspann für Korrekturansicht und Leseansicht.
%% Setzt den Schalter \ifkorrekturansicht voraus (gesetzt in den
%% einbindenden Dateien latex-korrekturansicht-abspann.tex bzw.
%% latex-leseansicht-abspann.tex).
%% ---------------------------------------------------------------

\normalsize

% Das esempio-Environment wird nur in der Leseansicht benötigt
\ifkorrekturansicht\else
\newenvironment{esempio}[3]%
{
    \vspace{1.5ex}
    \rlap{\underline{#1}}
    \par
    \setlength{\parindent}{0cm}
    \nopagebreak
    \leftskip=#2cm
    \rightskip=#3cm
}
{
    \par
}
\fi

\doendnotes{C}
\bigskip
\vfill

\clearpage

\footnotesize

\ifkorrekturansicht
  \lohead{\textsc{register}}
\fi

% theindex-Environment neu definieren ohne reledmac
\makeatletter
\renewenvironment{theindex}{%
  \ifkorrekturansicht
    \section*{\indexname}%
  \else
    \subsubsection*{Index der erwähnten Entitäten}%
  \fi
  \setlength{\parindent}{0pt}%
  \setlength{\parskip}{0pt plus 0.3pt}%
  \let\item\@idxitem
}{%
  \ifkorrekturansicht\clearpage\fi
}
\makeatother

\IfFileExists{\jobname-pw.ind}{\input{\jobname-pw.ind}}{}

% Quellenangabe nur in der Leseansicht
\ifkorrekturansicht\else
% Fallback-Definitionen, falls die .tex-Datei \titel etc. nicht gesetzt hat
\providecommand{\titel}{}
\providecommand{\editorInnen}{}
\providecommand{\dateiname}{\jobname}

\vspace{3cm}

\vfill

\footnotesize
\textsc{Quelle}: \titel. Herausgegeben von {\editorInnen}. In: \emph{Arthur Schnitzler: Briefwechsel mit Autorinnen und Autoren}.
 Digitale Edition, https://schnitzler-briefe.acdh.oeaw.ac.at/{\dateiname}.html (Stand \today)
\fi

\end{document}


      