%% latex-leseansicht-vorspann.tex
%% Vorspann für die Leseansicht.
%% Lädt die gemeinsame Datei latex-vorspann.tex mit nicht gesetztem Schalter.

\newif\ifkorrekturansicht
\korrekturansichtfalse

\input{../tex-inputs/latex-vorspann}


         
         \renewcommand{\erwaehntePersonen}{Personen: Hugo von Hofmannsthal, Frieda Pollak}
         \renewcommand{\erwaehnteOrte}{Orte: Gürtel, Hofmannsthal-Schlössl, Mauer, Maurer Berg, Rodaun, Stallburggasse, Wien}
         \renewcommand{\erwaehnteWerke}{}
               \section[Hugo Hofmannsthal an Arthur Schnitzler, {[}19. 3. 1921{]}]{ Hugo Hofmannsthal an Arthur Schnitzler, {[}19. 3. 1921{]}}\nopagebreak\mylabel{v}\rehead{ }\begin{ledgroupsized}[t]{13cm}\normalsize\beginnumbering \toendnotes[C]{\smallbreak\pagebreak[2]} \Standort{CUL, Schnitzler, B 43.}
\physDesc{Brief, 1 Blatt, 4 Seiten, 1377 Zeichen
\newline{}Handschrift: Bleistift, deutsche Kurrent
\newline{}Schnitzler: mit Bleistift datiert: »19/3 21« und beschriftet: »\textsc{Hugo}« 
\newline{}Ordnung: 1) mit Bleistift von Frieda
                                    Pollak\pwindex{Pollak, Frieda 08.12.1881 – 13.07.1937@\textsc{Pollak, Frieda} (08.12.1881 – 13.07.1937), \emph{Sekretärin}|pw} (?) mit dem Buchstaben »A«
                                 (Abgeschrieben/Abschrift) gekennzeichnet  2) mit Bleistift von unbekannter Hand nummeriert: »\strikeout{366}« 3) mit Bleistift von unbekannter Hand nummeriert:
                                    »370«}\buchAbdrucke{\weitereDrucke{Hugo von Hofmannsthal, Arthur Schnitzler: \emph{Briefwechsel}. Hg. Therese Nickl und Heinrich Schnitzler. Frankfurt am Main: \emph{S. Fischer} 1964, S. 294–295.} }\toendnotes[C]{\smallbreak}\pstart
           \raggedleft{}{\pb}Stallburggaſſe 2\oindex{Stallburggasse@\textbf{Stallburggasse}|pw}, \strikeout{Frei}{ }Samstg\pend
           \pstart{}mein lieber Arthur\pend\pstart
           es iſt mir traurig, Sie immer nur wie einen Schatten von weitem zu ſehen oder ein
               paar Worte miteinander zu wechſeln.\hspace*{1.5em}Ich möchte ſo
               gerne wieder einmal eine Stunde im Freien mit Ihnen herumgehen – geht es nicht?\hspace*{1.5em}Ich denke oft und herzlich an Sie, Sie ſind doch ein
               Stück von meinem Leben. Ob man die Lebensdinge im Geſpräch berührt oder nicht – ſie
               sind einmal {\pb}da, und müſſen
               irgendwie getragen werden, und von den Freunden mitgetragen werden.\pend
           \pstart
           Verſtehen freilich – ganz verſtehen tut man ja auch die Zuſa{\geminationm}enhänge des eigenen Lebens nicht, viel weniger die der
               Andern.\pend
           \pstart
           Könnten Sie nicht ſich entſchließen in der Oſterwoche doch einmal für
               das Mittageſſen und ein paar Nachmittagsſtunden nach Rodaun\oindex{Rodaun@\textbf{Rodaun}|pw} zu ko{\geminationm}en? {\pb}Sie führen etwa vormittag übern
                  Gürtel\oindex{Guertel@\textbf{Gürtel}|pw} herüber bis Mauer\oindex{Mauer@\textbf{Mauer}|pw} (keine 1¼ Stunden) gingen übern Maurer Berg\oindex{Maurer Berg@\textbf{Maurer Berg}|pw} zu uns – und beträten nach ſo viel Jahren das Haus\oindex{Hofmannsthal-Schloessl@\textbf{Hofmannsthal-Schlössl}|pwv} wieder in dem ich nun
               20 Jahre wohne und um das ich – um es weiter behalten zu können – jetzt einen harten
               Kampf kämpfe, weil ja eben eine {\pb}allgemeine Schwierigkeit und misère auch jedes einzelne Individuum in irgend einem
               Punkt ergreift, wie ein um ſich freſſendes Feuer.\pend
           \pstart
           Ko{\geminationm}en Sie doch Mittwoch herüber, ja?\pend
           \pstart
           Wenn das nicht geht, ſo ko{\geminationm}en Sie doch \label{K_L02364-1v}\edtext{Freitag}{\lemma{\textnormal{\emph{Freitag}}}\Cendnote{\textnormal{siehe A. S.: \emph{Tagebuch}, 25. 3. 1921}}}\label{K_L02364-1h}{ }\uline{vormittag}, etwa um 10 oder ½ 11 zu mir in die Stallburggaſſe\oindex{Stallburggasse@\textbf{Stallburggasse}|pw}. – Aber das iſt weniger! – Bitte
               ſchicken Sie ein telegram, ob Sie ko{\geminationm}en.\pend
           \pstart Ihr \spacefill\mbox{Hugo.}\pend{}
         
         \endnumbering\mylabel{h}\end{ledgroupsized}  \newcommand{\dateiname}{L02364}\newcommand{\titel}{Hugo Hofmannsthal an Arthur Schnitzler, [19. 3. 1921]}\newcommand{\editorInnen}{Martin Anton Müller und Gerd-Hermann Susen}%% latex-leseansicht-abspann.tex
%% Abspann für die Leseansicht.
%% Der Schalter \ifkorrekturansicht ist bereits durch den Vorspann gesetzt.

%% latex-abspann.tex
%% Gemeinsamer Abspann für Korrekturansicht und Leseansicht.
%% Setzt den Schalter \ifkorrekturansicht voraus (gesetzt in den
%% einbindenden Dateien latex-korrekturansicht-abspann.tex bzw.
%% latex-leseansicht-abspann.tex).
%% ---------------------------------------------------------------

\normalsize

% Das esempio-Environment wird nur in der Leseansicht benötigt
\ifkorrekturansicht\else
\newenvironment{esempio}[3]%
{
    \vspace{1.5ex}
    \rlap{\underline{#1}}
    \par
    \setlength{\parindent}{0cm}
    \nopagebreak
    \leftskip=#2cm
    \rightskip=#3cm
}
{
    \par
}
\fi

\doendnotes{C}
\bigskip
\vfill

\clearpage

\footnotesize

\ifkorrekturansicht
  \lohead{\textsc{register}}
\fi

% theindex-Environment neu definieren ohne reledmac
\makeatletter
\renewenvironment{theindex}{%
  \ifkorrekturansicht
    \section*{\indexname}%
  \else
    \subsubsection*{Index der erwähnten Entitäten}%
  \fi
  \setlength{\parindent}{0pt}%
  \setlength{\parskip}{0pt plus 0.3pt}%
  \let\item\@idxitem
}{%
  \ifkorrekturansicht\clearpage\fi
}
\makeatother

\IfFileExists{\jobname-pw.ind}{\input{\jobname-pw.ind}}{}

% Quellenangabe nur in der Leseansicht
\ifkorrekturansicht\else
% Fallback-Definitionen, falls die .tex-Datei \titel etc. nicht gesetzt hat
\providecommand{\titel}{}
\providecommand{\editorInnen}{}
\providecommand{\dateiname}{\jobname}

\vspace{3cm}

\vfill

\footnotesize
\textsc{Quelle}: \titel. Herausgegeben von {\editorInnen}. In: \emph{Arthur Schnitzler: Briefwechsel mit Autorinnen und Autoren}.
 Digitale Edition, https://schnitzler-briefe.acdh.oeaw.ac.at/{\dateiname}.html (Stand \today)
\fi

\end{document}


      