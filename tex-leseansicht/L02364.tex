%% latex-korrekturansicht-vorspann.tex
%% Vorspann für die Korrekturansicht.
%% Lädt die gemeinsame Datei latex-vorspann.tex mit gesetztem Schalter.

\newif\ifkorrekturansicht
\korrekturansichttrue

\input{../tex-inputs/latex-vorspann}


\section[Hugo Hofmannsthal an Arthur Schnitzler, {[}19. 3. 1921{]}]{L02364 Hugo Hofmannsthal an Arthur Schnitzler, {[}19. 3. 1921{]}}
\nopagebreak\mylabel{L02364v}
\rehead{ }\normalsize\beginnumbering\briefempfaengerindex{Schnitzler, Arthur@\textsc{Schnitzler, Arthur}!zzzHofmannsthal, Hugo von@\emph{von Hugo von Hofmannsthal}!1921-03-191@{{[}19. 3. 1921{]}}|(be}
\toendnotes[C]{\smallbreak\pagebreak[2]}\Standort{CUL, Schnitzler, B 43.}
\physDesc{Brief, 1 Blatt, 4 Seiten, 1377 Zeichen
\newline{}Handschrift: Bleistift, deutsche Kurrent
\newline{}Schnitzler: mit Bleistift datiert: »19/3 21« und beschriftet: »\textsc{Hugo}« 
\newline{}Ordnung: 1) mit Bleistift von Frieda
                                    Pollak\pwindex{Pollak, Frieda 08.12.1881 – 13.07.1937@\textsc{Pollak, Frieda} (08.12.1881 – 13.07.1937), \emph{Sekretär/Sekretärin}|pw} (?) mit dem Buchstaben »A«
                                 (Abgeschrieben/Abschrift) gekennzeichnet  2) mit Bleistift von unbekannter Hand nummeriert: »\strikeout{366}« 3) mit Bleistift von unbekannter Hand nummeriert:
                                    »370«}
\buchAbdrucke{\weitereDrucke{Hugo von Hofmannsthal, Arthur Schnitzler: \emph{Briefwechsel}. Frankfurt am Main: \emph{S. Fischer} 1964, S. 294–295.} }\toendnotes[C]{\smallbreak}
\pstart
           \raggedleft{}{\pb}Stallburggaſſe 2\oindex{Stallburggasse@\textbf{Stallburggasse}, \emph{Straße (K.STR)}|pw}, \strikeout{Frei}{ }Samstg\pend
           
\pstart{}mein lieber Arthur\pend\vspace{0.5em}
\pstart
           es iſt mir traurig, Sie immer nur wie einen Schatten von weitem zu ſehen oder ein
               paar Worte miteinander zu wechſeln.\hspace*{1.5em}Ich möchte ſo
               gerne wieder einmal eine Stunde im Freien mit Ihnen herumgehen – geht es nicht?\hspace*{1.5em}Ich denke oft und herzlich an Sie, Sie ſind doch ein
               Stück von meinem Leben. Ob man die Lebensdinge im Geſpräch berührt oder nicht – ſie
               sind einmal {\pb}da, und müſſen
               irgendwie getragen werden, und von den Freunden mitgetragen werden.\pend
           
\pstart
           Verſtehen freilich – ganz verſtehen tut man ja auch die Zuſa{\geminationm}enhänge des eigenen Lebens nicht, viel weniger die der
               Andern.\pend
           
\pstart
           Könnten Sie nicht ſich entſchließen in der Oſterwoche doch einmal für
               das Mittageſſen und ein paar Nachmittagsſtunden nach Rodaun\oindex{Rodaun@\textbf{Rodaun}, \emph{A.ADM4}|pw} zu ko{\geminationm}en? {\pb}Sie führen etwa vormittag übern
                  Gürtel\oindex{Guertel@\textbf{Gürtel}, \emph{Straße (K.STR)}|pw} herüber bis Mauer\oindex{Mauer@\textbf{Mauer}, \emph{eingemeindeter Ort (A.VOO)}|pw} (keine 1¼ Stunden) gingen übern Maurer Berg\oindex{Maurer Berg@\textbf{Maurer Berg}, \emph{Berg (N.BRG)}|pw} zu uns – und beträten nach ſo viel Jahren das Haus\oindex{Hofmannsthal-Schloessl@\textbf{Hofmannsthal-Schlössl}, \emph{Schloss (K.SLS)}|pwv} wieder in dem ich nun
               20 Jahre wohne und um das ich – um es weiter behalten zu können – jetzt einen harten
               Kampf kämpfe, weil ja eben eine {\pb}allgemeine Schwierigkeit und misère auch jedes einzelne Individuum in irgend einem
               Punkt ergreift, wie ein um ſich freſſendes Feuer.\pend
           
\pstart
           Ko{\geminationm}en Sie doch Mittwoch herüber, ja?\pend
           
\pstart
           Wenn das nicht geht, ſo ko{\geminationm}en Sie doch \label{K_L02364-1v}\edtext{Freitag}{\lemma{\textnormal{\emph{Freitag}}}\Cendnote{\textnormal{Siehe A. S.: \emph{Tagebuch}, 25. 3. 1921.
               }}}\label{K_L02364-1}{ }\uline{vormittag}, etwa um 10 oder ½ 11 zu mir in die Stallburggaſſe\oindex{Stallburggasse@\textbf{Stallburggasse}, \emph{Straße (K.STR)}|pw}. – Aber das iſt weniger! – Bitte
               ſchicken Sie ein telegram, ob Sie ko{\geminationm}en.\pend
           \pstart Ihr \spacefill\mbox{Hugo.}\pend{}\selectlanguage{ngerman}\endnumbering\briefempfaengerindex{Schnitzler, Arthur@\textsc{Schnitzler, Arthur}!zzzHofmannsthal, Hugo von@\emph{von Hugo von Hofmannsthal}!1921-03-191@{{[}19. 3. 1921{]}}|)be}\mylabel{L02364h}  \normalsize

\doendnotes{C}
\bigskip
\vfill

\clearpage

\footnotesize

\lohead{\textsc{register}}

% Definiere theindex-Environment komplett neu ohne reledmac
\makeatletter
\renewenvironment{theindex}{%
  \section*{\indexname}%
  \setlength{\parindent}{0pt}%
  \setlength{\parskip}{0pt plus 0.3pt}%
  \let\item\@idxitem
}{%
  \clearpage
}
\makeatother

\IfFileExists{\jobname-pw.ind}{\input{\jobname-pw.ind}}{}

\end{document}

      