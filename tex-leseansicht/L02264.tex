%% latex-leseansicht-vorspann.tex
%% Vorspann für die Leseansicht.
%% Lädt die gemeinsame Datei latex-vorspann.tex mit nicht gesetztem Schalter.

\newif\ifkorrekturansicht
\korrekturansichtfalse

\input{../tex-inputs/latex-vorspann}


         
         \renewcommand{\erwaehntePersonen}{Personen: Robert Adam, Richard Beer, Alexandre père Dumas, Aemilius Hacker, Viktor Franz Patzner, Maria Pollak}
         \renewcommand{\erwaehnteInstitutionen}{Institutionen: Bezirksgericht Wien Josefstadt}
         \renewcommand{\erwaehnteOrte}{Orte: Steiermark, VIII., Josefstadt, Wien, XXI., Floridsdorf, Zistersdorf}
         \renewcommand{\erwaehnteWerke}{Werke: Das Ende des Judas, Meine Memoiren}
               \section[Robert Adam an Arthur Schnitzler, 19. 6. 1917]{ Robert Adam an Arthur Schnitzler, 19. 6. 1917}\nopagebreak\mylabel{v}\rehead{ }\begin{ledgroupsized}[t]{13cm}\normalsize\beginnumbering\briefempfaengerindex{Schnitzler, Arthur@\textsc{Schnitzler, Arthur}!zzzAdam, Robert@\emph{von Robert Adam}!1917-06-191@{19. 6. 1917}|(be} \toendnotes[C]{\smallbreak\pagebreak[2]} \Standort{DLA, A:Schnitzler, HS.NZ85.1.4230,19.}
\physDesc{Brief, 1 Blatt, 4 Seiten, 2234 Zeichen
\newline{}Handschrift: schwarze Tinte, deutsche Kurrent
\newline{}Schnitzler: 1) mit Bleistift beschriftet: »\textsc{Adam}«  2) mit rotem Buntstift mehrere Unterstreichungen}\Standort{Wien, Österreichische Nationalbibliothek, Cod.ser. 52.263, 197.}
\physDesc{Brief, maschinenschriftliche Abschrift, 1 Blatt, 1 Seite, 2234 Zeichen
\newline{}Schreibmaschine}\toendnotes[C]{\smallbreak}\pstart
           \raggedleft{}{\pb}Wien\oindex{Wien@\textbf{Wien}|pw}, am 19. Juni 1917. \pend
           \pstart{}Hochverehrter Herr Doktor!\pend\pstart
           Ich danke Ihnen herzlich für Ihren Glückwunſch. Die Verſetzung von Floridsdorf\oindex{XXI., Floridsdorf@\textbf{XXI., Floridsdorf}|pw} zum Bezirksgericht Joſefſtadt\oindex{VIII., Josefstadt@\textbf{VIII., Josefstadt}|pw}\orgindex{Bezirksgericht Wien Josefstadt@Bezirksgericht Wien Josefstadt|pwv} empfand und empfinde ich noch als eine Befreiung aus dem unleidlichſten
               Zuſtande, dem Zwang zur Zeitvergeudung. Denn mochte ich mich auch bemühen, die
               endloſen täglichen Tramwayfahrten zu irgendeinem Studium auszunützen, es gelang
               höchſtens bei der Morgenfahrt, während mir die Rückreiſe, die ich ermüdet und hungrig
               zurücklegen mußte, nur gerade noch eine Zeitungslektüre {\pb}verſtattete. Auch die Amtsbeſchäftigung – die
               Säuberung einer von meinem verſtorbenen Vorgänger\pwindex{Hacker, Aemilius Maerz 1870 – 25.03.1912@\textsc{Hacker, Aemilius} (März 1870 – 25.03.1912), \emph{Richter, Alpinist}|pwuv} arg verwahrloſten außerſtreitigen Abteilung – bot
               nur wenig Befriedigung.\pend
           \pstart
           Durch die Verſetzung bin ich allerdings wieder, und zwar aller Wahrſcheinlichkeit
               nach auf längere Zeit, in die Nachrichtertätigkeit zurückgeworfen; da ich aber nur in
               Preistreibereiſachen zu judizieren habe, bleibt mir das Peinliche fern, das in jeder
               andern Nachjudikatur in Zeiten allgemeiner Not liegt. Ich brauche nicht Leute zu
               verurteilen, deren Vergehen durch die Hungersnot kauſal begründet iſt, ſondern habe
               vor allem gegen ſolche einzuſchreiten, deren Vergehen {\pb}eben die Mitverurſachung der Hungersnot bildet. Und ſo arbeite ich ohne böſes
               Gewiſſen.\pend
           \pstart
           Auch literariſch bin ich nicht ganz untätig. Von einer ſeltſamen Urchriſtenkomödie\pwindex{Adam, Robert 20.04.1877 – 16.10.1961@\textsc{Adam, Robert} (20.04.1877 – 16.10.1961), \emph{Schriftsteller, Richter}!Ende des Judas@\strich\emph{Das Ende des Judas}|pwv} (oder Tragödie?) habe ich
               faſt drei Akte im Rohen fertig entworfen und hoffe, die reſtlichen zwei Akte, die mir
               beſonders am Herzen liegen, während des Urlaubs zu Papier zu bringen. Diesen trete
               ich Ende Juni an und will ihn zur Hälfte bei Frau\pwindex{Pollak, Maria 06.10.1889 – 27.03.1948@\textsc{Pollak, Maria} (06.10.1889 – 27.03.1948)|pwv} und Kind\pwindex{Patzner, Viktor Franz 13.09.1916 – 21.12.1982@\textsc{Patzner, Viktor Franz} (13.09.1916 – 21.12.1982), \emph{Rechtsanwalt}|pwv} verbringen, die ich günſtigerer Ernährungsverhältniſſe
               wegen in meinem früheren Dienſtorte, in Ziſtersdorf\oindex{Zistersdorf@\textbf{Zistersdorf}|pw}, angesiedelt habe; während der reſtlichen Zeit gedenke ich mit
                  D\textsuperscript{r}{ }\textsc{Beer}\pwindex{Beer, Richard 04.03.1876 – 06.04.1934@\textsc{Beer, Richard} (04.03.1876 – 06.04.1934), \emph{Rechtsanwalt}|pw} irgendwo in Steiermark\oindex{Steiermark@\textbf{Steiermark}|pw}, bewaffnet mit einer
               Salami, das dazu gehörige tägliche {\pb}Brot zu
               ſuchen.\pend
           \pstart
           Da ich nicht weiß, wann Sie, hochverehrter Herr Doktor, nach Wien\oindex{Wien@\textbf{Wien}|pw} zurückkehren – das herrliche Wetter dürfte Ihre Rückkehr
               wohl verzögern –, will ich im Laufe der nächſten Woche bei Ihnen anklopfen, auf die
               Gefahr hin, Sie nicht anzutreffen.\pend
           \pstart
           Indem ich ſchließlich den Rückerhalt des \textsc{Dumas}\pwindex{Dumas, Alexandre pere 24.07.1802 – 05.12.1870@\textsc{Dumas, Alexandre père} (24.07.1802 – 05.12.1870), \emph{Schriftsteller}|pw}\pwindex{Dumas, Alexandre pere 24.07.1802 – 05.12.1870@\textsc{Dumas, Alexandre père} (24.07.1802 – 05.12.1870), \emph{Schriftsteller}!Meine Memoiren1852 – 1856@\strich\emph{Meine Memoiren} {[}1852 – 1856{]}|pwv} mit beſtem Dank beſtätige, verbleibe ich mit beſten Grüßen und Empfehlungen
               Ihr\pend
           \pstart
           ſehr ergebener{\\[\baselineskip]}\spacefill\mbox{Robert Adam}\pend
           \leftskip=0em{}
         
         \endnumbering\mylabel{h}\end{ledgroupsized}  \newcommand{\dateiname}{L02264}\newcommand{\titel}{Robert Adam an Arthur Schnitzler, 19. 6. 1917}\newcommand{\editorInnen}{Martin Anton Müller und Gerd-Hermann Susen}%% latex-leseansicht-abspann.tex
%% Abspann für die Leseansicht.
%% Der Schalter \ifkorrekturansicht ist bereits durch den Vorspann gesetzt.

%% latex-abspann.tex
%% Gemeinsamer Abspann für Korrekturansicht und Leseansicht.
%% Setzt den Schalter \ifkorrekturansicht voraus (gesetzt in den
%% einbindenden Dateien latex-korrekturansicht-abspann.tex bzw.
%% latex-leseansicht-abspann.tex).
%% ---------------------------------------------------------------

\normalsize

% Das esempio-Environment wird nur in der Leseansicht benötigt
\ifkorrekturansicht\else
\newenvironment{esempio}[3]%
{
    \vspace{1.5ex}
    \rlap{\underline{#1}}
    \par
    \setlength{\parindent}{0cm}
    \nopagebreak
    \leftskip=#2cm
    \rightskip=#3cm
}
{
    \par
}
\fi

\doendnotes{C}
\bigskip
\vfill

\clearpage

\footnotesize

\ifkorrekturansicht
  \lohead{\textsc{register}}
\fi

% theindex-Environment neu definieren ohne reledmac
\makeatletter
\renewenvironment{theindex}{%
  \ifkorrekturansicht
    \section*{\indexname}%
  \else
    \subsubsection*{Index der erwähnten Entitäten}%
  \fi
  \setlength{\parindent}{0pt}%
  \setlength{\parskip}{0pt plus 0.3pt}%
  \let\item\@idxitem
}{%
  \ifkorrekturansicht\clearpage\fi
}
\makeatother

\IfFileExists{\jobname-pw.ind}{\input{\jobname-pw.ind}}{}

% Quellenangabe nur in der Leseansicht
\ifkorrekturansicht\else
% Fallback-Definitionen, falls die .tex-Datei \titel etc. nicht gesetzt hat
\providecommand{\titel}{}
\providecommand{\editorInnen}{}
\providecommand{\dateiname}{\jobname}

\vspace{3cm}

\vfill

\footnotesize
\textsc{Quelle}: \titel. Herausgegeben von {\editorInnen}. In: \emph{Arthur Schnitzler: Briefwechsel mit Autorinnen und Autoren}.
 Digitale Edition, https://schnitzler-briefe.acdh.oeaw.ac.at/{\dateiname}.html (Stand \today)
\fi

\end{document}


      