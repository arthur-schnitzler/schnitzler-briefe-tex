%% latex-leseansicht-vorspann.tex
%% Vorspann für die Leseansicht.
%% Lädt die gemeinsame Datei latex-vorspann.tex mit nicht gesetztem Schalter.

\newif\ifkorrekturansicht
\korrekturansichtfalse

\input{../tex-inputs/latex-vorspann}


\section[Arthur Schnitzler an Gustav Schwarzkopf, 28. 7. 1895]{L04110 Arthur Schnitzler an Gustav Schwarzkopf, 28. 7. 1895}
\nopagebreak\mylabel{L04110v}
\rehead{ }\normalsize\beginnumbering\briefempfaengerindex{Schwarzkopf, Gustav@\textsc{Schwarzkopf, Gustav}!zzzSchnitzler, Arthur@\emph{von Arthur Schnitzler}!1895-07-282@{28. 7. 1895}|(be}
\toendnotes[C]{\smallbreak\pagebreak[2]}
\correspDesc{Versand  durch Arthur Schnitzler am 28. 7. 1895 in Wien
\newline{}Erhalt  durch Gustav Schwarzkopf im Zeitraum [28. 7. 1895 – 31. 7. 1895?] in Wien}\toendnotes[C]{\smallbreak}
\Standort{CUL, Schnitzler, B 96.}
\physDesc{Brief, 1 Blatt, 4 Seiten, 1309 Zeichen
\newline{}Handschrift: schwarze Tinte, deutsche Kurrent}
\buchAbdrucke{\weitereDrucke{Arthur Schnitzler: \emph{Briefe 1875–1912}. Herausgegeben von Therese Nickl und Heinrich Schnitzler. Frankfurt am Main: \emph{S. Fischer} 1981, S. 273.} }\toendnotes[C]{\smallbreak}
\pstart{}{\pb}Lieber Freund!\pend\vspace{0.5em}
\pstart
           Ihre Gründe gegen Iſchl\oindex{Bad Ischl@\textbf{Bad Ischl}|pw} ſind wenig{ }ſtichhältig –
               die Laune werden Sie niemandem verderben, das beſorgt jeder{ }ſelber. Auch können
               Sie ſich in ganz gefährlichen Momenten an entlegene Waldpartien zurückziehen. Welche
                  Sti{\geminationm}ung iſt de{\geminationn}
               eigentlich für Ischl\oindex{Bad Ischl@\textbf{Bad Ischl}|pw} erforderlich? \strikeout{Daſs Sie} Ich wiederhole alſo meinen Rath Ihr Alibi {\pb}hieher zu verlegen, wo Sie wenigſtens,
               Zeugen finden. Nur, um Hi{\geminationm}els willen, ſtören Sie Richard\pwindex{Beer-Hofmann, Richard 11.\,7.\,1866 Wien – 26.\,9.\,1945 New York City@\textsc{Beer-Hofmann, Richard} (11.\,7.\,1866 Wien – 26.\,9.\,1945 New York City), \emph{Schriftsteller}|pw} nicht im Arbeiten! Ich könnte das nicht
               verantworten. –\pend
           
\pstart
           \textsc{Hans von Kahlenberg\pwindex{Keßler, Helene 23.\,2.\,1870 Heilbad Heiligenstadt – 8.\,8.\,1957 Baden-Baden@\textsc{Keßler, Helene} (23.\,2.\,1870 Heilbad Heiligenstadt – 8.\,8.\,1957 Baden-Baden), \emph{Schriftstellerin}|pw}} ke{\geminationn} ich nicht. Was ist das für ein Buch\pwindex{Keßler, Helene 23.\,2.\,1870 Heilbad Heiligenstadt – 8.\,8.\,1957 Baden-Baden@\textsc{Keßler, Helene} (23.\,2.\,1870 Heilbad Heiligenstadt – 8.\,8.\,1957 Baden-Baden), \emph{Schriftstellerin}!Narr. Roman@\strich\emph{Ein Narr. Roman}|pwv}? – Hier läuft ein Autor\pwindex{Goldscheid, Rudolf 12.\,8.\,1870 Wien – 31.\,10.\,1931 ebd.@\textsc{Goldscheid, Rudolf} (12.\,8.\,1870 Wien – 31.\,10.\,1931 ebd.), \emph{Philosoph, Publizist}|pwv} herum, der auf »Beobachter« \textsc{posirt} und hämiſch nach der \textsc{Table des
                  hôtes} von einer Bank aus das Treiben der {\pb}Menge betrachtet, in widerlichem
               Stolz. Es ist \textsc{Rudolf Golm\pwindex{Goldscheid, Rudolf 12.\,8.\,1870 Wien – 31.\,10.\,1931 ebd.@\textsc{Goldscheid, Rudolf} (12.\,8.\,1870 Wien – 31.\,10.\,1931 ebd.), \emph{Philosoph, Publizist}|pw}}. Haben Sie von dem was geleſen? – Mein Couſin\pwindex{Mandl, Alfred 28.\,2.\,1878 Wien – 11.\,2.\,1926 Prag@\textsc{Mandl, Alfred} (28.\,2.\,1878 Wien – 11.\,2.\,1926 Prag), \emph{Ingenieur}|pwuv} hat ſich ſeinen Roman\pwindex{Goldscheid, Rudolf 12.\,8.\,1870 Wien – 31.\,10.\,1931 ebd.@\textsc{Goldscheid, Rudolf} (12.\,8.\,1870 Wien – 31.\,10.\,1931 ebd.), \emph{Philosoph, Publizist}!alte Adam und die neue Eva. Ein Roman unserer Übergangszeit@\strich\emph{Der alte Adam und die neue Eva. Ein Roman unserer Übergangszeit}|pwv} gekauft, um den Autor dann mit mehr Recht faſſen zu
               können.\pend
           
\pstart
           – Ich werde hier von böſen Träumen geplagt; geſtern erſchien mir der Verleger \textsc{Fischer\pwindex{Fischer, Samuel 24.\,12.\,1859 Liptovský Mikuláš – 15.\,10.\,1934 Berlin@\textsc{Fischer, Samuel} (24.\,12.\,1859 Liptovský Mikuláš – 15.\,10.\,1934 Berlin), \emph{Verleger}|pw}}; er ſtreichelte mir die Wangen, u. ich ſagte ihm: Zärtlich{ }ſein können Sie –
               aber zahlen nicht!– {\pb}Und heute Nachts
               träumte ich von \textsc{Blumenthal\pwindex{Blumenthal, Oskar 13.\,3.\,1852 Berlin – 24.\,4.\,1917 ebd.@\textsc{Blumenthal, Oskar} (13.\,3.\,1852 Berlin – 24.\,4.\,1917 ebd.), \emph{Schriftsteller, Journalist, Theaterleiter}|pw}}; ich hab ihn nur geſehn, geſchehen iſt nichts beſonders. Ich nehme an, er
               hat ſein Wort gebrochen, was ja ganz geräuſchlos paſſiren kann. –\pend
           
\pstart
           Seien Sie herzlich gegrüßt und laſſen{\\[\baselineskip]} bald ein Wort von ſich hören.{\\[\baselineskip]}Richard\pwindex{Beer-Hofmann, Richard 11.\,7.\,1866 Wien – 26.\,9.\,1945 New York City@\textsc{Beer-Hofmann, Richard} (11.\,7.\,1866 Wien – 26.\,9.\,1945 New York City), \emph{Schriftsteller}|pw} grüßt vielmals.{\\[\baselineskip]} Ihr
                  \spacefill\mbox{ArthSch}\pend
           \leftskip=0em{}
\pstart
           28/7 95\pend
           
\pstart
           \textsc{Ischl\oindex{Bad Ischl@\textbf{Bad Ischl}|pw}}.\pend
           \selectlanguage{ngerman}\endnumbering\briefempfaengerindex{Schwarzkopf, Gustav@\textsc{Schwarzkopf, Gustav}!zzzSchnitzler, Arthur@\emph{von Arthur Schnitzler}!1895-07-282@{28. 7. 1895}|)be}\mylabel{L04110h}
\begin{anhang}
\end{anhang}\newcommand{\dateiname}{L04110}\newcommand{\titel}{Arthur Schnitzler an Gustav Schwarzkopf, 28. 7. 1895}\newcommand{\editorInnen}{Herausgegeben von Jahnke, SelmaMüller, Martin Anton}%% latex-leseansicht-abspann.tex
%% Abspann für die Leseansicht.
%% Der Schalter \ifkorrekturansicht ist bereits durch den Vorspann gesetzt.

%% latex-abspann.tex
%% Gemeinsamer Abspann für Korrekturansicht und Leseansicht.
%% Setzt den Schalter \ifkorrekturansicht voraus (gesetzt in den
%% einbindenden Dateien latex-korrekturansicht-abspann.tex bzw.
%% latex-leseansicht-abspann.tex).
%% ---------------------------------------------------------------

\normalsize

% Das esempio-Environment wird nur in der Leseansicht benötigt
\ifkorrekturansicht\else
\newenvironment{esempio}[3]%
{
    \vspace{1.5ex}
    \rlap{\underline{#1}}
    \par
    \setlength{\parindent}{0cm}
    \nopagebreak
    \leftskip=#2cm
    \rightskip=#3cm
}
{
    \par
}
\fi

\doendnotes{C}
\bigskip
\vfill

\clearpage

\footnotesize

\ifkorrekturansicht
  \lohead{\textsc{register}}
\fi

% theindex-Environment neu definieren ohne reledmac
\makeatletter
\renewenvironment{theindex}{%
  \ifkorrekturansicht
    \section*{\indexname}%
  \else
    \subsubsection*{Index der erwähnten Entitäten}%
  \fi
  \setlength{\parindent}{0pt}%
  \setlength{\parskip}{0pt plus 0.3pt}%
  \let\item\@idxitem
}{%
  \ifkorrekturansicht\clearpage\fi
}
\makeatother

\IfFileExists{\jobname-pw.ind}{\input{\jobname-pw.ind}}{}

% Quellenangabe nur in der Leseansicht
\ifkorrekturansicht\else
% Fallback-Definitionen, falls die .tex-Datei \titel etc. nicht gesetzt hat
\providecommand{\titel}{}
\providecommand{\editorInnen}{}
\providecommand{\dateiname}{\jobname}

\vspace{3cm}

\vfill

\footnotesize
\textsc{Quelle}: \titel. Herausgegeben von {\editorInnen}. In: \emph{Arthur Schnitzler: Briefwechsel mit Autorinnen und Autoren}.
 Digitale Edition, https://schnitzler-briefe.acdh.oeaw.ac.at/{\dateiname}.html (Stand \today)
\fi

\end{document}


