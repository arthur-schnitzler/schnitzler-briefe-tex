\input{../tex-inputs/latex-pdf-vorspann}
\begin{center}
            \textcolor{red}{ENTWURF. ENTZIFFERUNG NOCH NICHT KORREKTURGELESEN}
                      \end{center}
            
               \section[Arthur Schnitzler an Richard Beer-Hofmann, 5. 3. 1893]{ Arthur Schnitzler an Richard Beer-Hofmann, 5. 3. 1893}\nopagebreak\mylabel{v}\rehead{ }\begin{ledgroupsized}[t]{13cm}\normalsize\beginnumbering\briefempfaengerindex{Beer-Hofmann, Richard@\textsc{Beer-Hofmann, Richard}!zzzSchnitzler, Arthur@\emph{von Arthur Schnitzler}!1893-03-051@{5. 3. 1893}|(be} \toendnotes[C]{\smallbreak\pagebreak[2]} \Standort{YCGL, MSS 31.}
\physDesc{Brief, 1 Blatt, 4 Seiten, Umschlag
\newline{}Handschrift: blaue Tinte, deutsche Kurrent\newline{}Versand: 1) Stempel: »\nobreak{}\oindex{Pension Quisisana@\textbf{Pension Quisisana}|pwk}Pension »Quisisana« Abbazia\nobreak{}«.  2) Stempel: »\nobreak{}\oindex{Opatija@\textbf{Opatija}|pwk}Abbazia, 5/3 9\textcolor{gray}{3}\nobreak{}«. 3) Stempel: »\nobreak{}\oindex{I., Innere Stadt@\textbf{I., Innere Stadt}|pwk}Wien 1/1, 6/3. 93, 11½V–1N, Bestellt\nobreak{}«. }\buchAbdrucke{\weitereDrucke{Arthur Schnitzler, Richard Beer-Hofmann: \emph{Briefwechsel 1891–1931}. Hg. Konstanze Fliedl. Wien, Zürich: \emph{Europaverlag} 1992, S. 42.} }\toendnotes[C]{\smallbreak}\pstart{}{\pb}\textsc{Herrn Doctor Richard Beer-Hofmann}\pend{}\pstart{}\textsc{Wien\oindex{Wien@\textbf{Wien}|pw}}\pend{}\pstart{}\textsc{I Wollzeile 15\oindex{Wollzeile@\textbf{Wollzeile}|pw}}.\pend{}{\bigskip}\pstart{}{\pb}Lieber Richard,\pend\pstart
           für die Anempfehlung von \textsc{Quisisana}\oindex{Pension Quisisana@\textbf{Pension Quisisana}|pw} meinen beſten Dank! Ich fühle mich hier ſehr wohl, und habe überdies ein sehr
               hübſches Parterrezi{\geminationm}er mit Ausblick aufs weite Meer, das
               herrlichſte Wetter (ke{\geminationn}e keinen Ueberzieher mehr) und
               ſehr ſympathiſche Geſellschaft (die malende Schweſter\pwindex{Rosenthal, Marie 28.03.1869 – 1942@\textsc{Rosenthal, Marie} (28.03.1869 – 1942), \emph{Malerin}|pwv}{ }\textsc{Rosenthal}\pwindex{Rosenthal, Moritz 17.12.1862 – 03.09.1946@\textsc{Rosenthal, Moritz} (17.12.1862 – 03.09.1946), \emph{Pianist}|pw}’s und die \textsc{Sophie Link}\pwindex{Link, Sophie 1860 – 01.10.1900@\textsc{Link, Sophie} (1860 – 01.10.1900), \emph{Sängerin}|pw}, ſeit 6 Wochen in Berlin\oindex{Berlin@\textbf{Berlin}|pw}{ }verheiratet\pwindex{Loewenstein, Harry 29.1.1851 – 8.8.1907@\textsc{Löwenstein, Harry} (29.1.1851 – 8.8.1907), \emph{Börsenmakler}|pwv}). – Ich bin meiſt
               im Freien, und pendle zwiſchen \textsc{Lovrana}\oindex{Lovran@\textbf{Lovran}|pw} und \textsc{Voloska}\oindex{Volosko@\textbf{Volosko}|pw}{ }{\pb}hin u her. – Gearbeitet – wenig; i{\geminationm}erhin ein Stück der Novellette\pwindex{Schnitzler, Arthur 15.05.1862 – 21.10.1931@\textsc{Schnitzler, Arthur} (15.05.1862 – 21.10.1931), \emph{Schriftsteller, Mediziner}!kleine Komoedie01.08.1895 – 01.08.1895@\strich\emph{Die kleine Komödie} {[}01.08.1895 – 01.08.1895{]}|pwv}. – Die »Familie\pwindex{Schnitzler, Arthur 15.05.1862 – 21.10.1931@\textsc{Schnitzler, Arthur} (15.05.1862 – 21.10.1931), \emph{Schriftsteller, Mediziner}!Familie1977@\strich\emph{Familie} {[}1977{]}|pw}« durchgeleſen, merke, daſs was fehlt, und bin nicht recht klar was.
               Ich werde es auch jedenfalls in 2–3 Wochen vorleſen, aber um Rathschläge erſuchen
               müſſen. Keineswegs leſe ich, bevor wir Ihre Novelle\pwindex{Beer-Hofmann, Richard 11.07.1866 – 26.09.1945@\textsc{Beer-Hofmann, Richard} (11.07.1866 – 26.09.1945), \emph{Schriftsteller}!Kind1893@\strich\emph{Das Kind} {[}1893{]}|pwv} zu hören beko{\geminationm}en, was
               hoffentlich kurz nach meiner Ankunft möglich ſein wird! –\pend
           \pstart
           – Ich denke nicht gern ans Fortreiſen; die Ruhe hier thut mir ganz unbeſchreiblich
               wohl; wäre ich mein eigner Herr, ſo blieb’ ich zwei Monate da. We{\geminationn} man auch nicht {\pb}arbeitet, – man hat die Empfindung, daſs man es jeden Augenblick könnte, was faſt
               noch mehr werth ist. – Hübſch wär’s, we{\geminationn} wir das nächſte
               Frühjahr die ganze \textsc{Quisisana}\oindex{Pension Quisisana@\textbf{Pension Quisisana}|pw} miethen könnten! – Ah, diese Luft – einfach entzückend! – Es iſt doch recht
               traurig zu den »Müſſenden« zu gehören! –\pend
           \pstart
           Grüßen Sie \textsc{Loris}\pwindex{Hofmannsthal, Hugo von 01.02.1874 – 15.07.1929@\textsc{Hofmannsthal, Hugo von} (01.02.1874 – 15.07.1929), \emph{Schriftsteller}|pw} und \textsc{Salten}\pwindex{Salten, Felix 06.09.1869 – 08.10.1945@\textsc{Salten, Felix} (06.09.1869 – 08.10.1945), \emph{Schriftsteller, Journalist}|pw} aufs allerherzlichſte, desgleichen \textsc{Schwarzkopf}\pwindex{Schwarzkopf, Gustav 07.11.1853 – 13.11.1939@\textsc{Schwarzkopf, Gustav} (07.11.1853 – 13.11.1939), \emph{Schriftsteller}|pw}, der mir doch zwei Zeilen über das Befinden seines Bruders\pwindex{Schwarzkopf, Rudolf 25.05.1861 – 13.10.1893@\textsc{Schwarzkopf, Rudolf} (25.05.1861 – 13.10.1893), \emph{Schriftsteller}|pwv}{ }ſchreiben möchte; und grüßen Sie nebſtbei
               jedermann, der die Freundlichkeit hat nach mir zu fragen. – Schade, daſs {\pb}Sie nicht auch da ſind! Hoffentlich find ich Sie aber
               in geſegneterer Sti{\geminationm}ung als ich Sie verlaſſen!\pend
           \pstart
           Stets der Ihre{\\[\baselineskip]}\spacefill\mbox{Arthur.}\pend
           \leftskip=0em{}\pstart
           \textsc{Abbazia}\oindex{Opatija@\textbf{Opatija}|pw}5. 3. 9\textcolor{gray}{3}. So{\geminationn}tag. –\pend
           \endnumbering\briefempfaengerindex{Beer-Hofmann, Richard@\textsc{Beer-Hofmann, Richard}!zzzSchnitzler, Arthur@\emph{von Arthur Schnitzler}!1893-03-051@{5. 3. 1893}|)be}\mylabel{h}\end{ledgroupsized}  \newcommand{\dateiname}{L00185}\newcommand{\titel}{Arthur Schnitzler an Richard Beer-Hofmann, 5. 3. 1893}\newcommand{\editorInnen}{Martin Anton Müller und Gerd-Hermann Susen}\input{../tex-inputs/latex-pdf-abspann}
      