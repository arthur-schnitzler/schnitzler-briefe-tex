%% latex-leseansicht-vorspann.tex
%% Vorspann für die Leseansicht.
%% Lädt die gemeinsame Datei latex-vorspann.tex mit nicht gesetztem Schalter.

\newif\ifkorrekturansicht
\korrekturansichtfalse

\input{../tex-inputs/latex-vorspann}


\section[Arthur Schnitzler an Richard Beer-Hofmann, 5. 3. 1893]{L00185 Arthur Schnitzler an Richard Beer-Hofmann, 5. 3. 1893}
\nopagebreak\mylabel{L00185v}
\rehead{ }\normalsize\beginnumbering\briefempfaengerindex{Beer-Hofmann, Richard@\textsc{Beer-Hofmann, Richard}!zzzSchnitzler, Arthur@\emph{von Arthur Schnitzler}!1893-03-051@{5. 3. 1893}|(be}
\toendnotes[C]{\smallbreak\pagebreak[2]}
\correspDesc{Versand  durch Arthur Schnitzler am 5. 3. 1893 in Opatija
\newline{}Erhalt  durch Richard Beer-Hofmann am 6. 3. 1893 in Wien}\toendnotes[C]{\smallbreak}
\Standort{YCGL, MSS 31.}
\physDesc{Brief, 1 Blatt, 4 Seiten, Kuvert, 1667 Zeichen
\newline{}Handschrift: lila Tinte, deutsche Kurrent
\newline{}Versand: 1) Stempel: »\nobreak{}\oindex{Pension Quisisana@\textbf{Pension Quisisana}, \emph{Hotel}|pwk}Pension »Quisisana«
                                                  Abbazia\nobreak{}«.   2) Stempel: »\nobreak{}\oindex{Opatija@\textbf{Opatija}, \emph{Hauptstadt}|pwk}Abbazia, 5/3 9 \textcolor{gray}{3}\nobreak{}«.  3) Stempel: »\nobreak{}\oindex{I., Innere Stadt@\textbf{I., Innere Stadt}, \emph{Verwaltungsgebiet}|pwk}Wien 1/1, 6/3. 93, 11½V–1N, Bestellt\nobreak{}«. }
\buchAbdrucke{\weitereDrucke{Arthur Schnitzler, Richard Beer-Hofmann: \emph{Briefwechsel 1891–1931}. Herausgegeben von Konstanze Fliedl. Wien, Zürich: \emph{Europaverlag} 1992, S. 42.} }\toendnotes[C]{\smallbreak}\pstart{}{\pb}\textsc{Herrn Doctor Richard Beer-Hofmann}\pend{}\pstart{}\textsc{Wien\oindex{Wien@\textbf{Wien}, \emph{Verwaltungsgebiet}|pw}}\pend{}\pstart{}\textsc{I Wollzeile 15\oindex{Wien@\textbf{Wien}!I., Innere Stadt@\textbf{I., Innere Stadt}!Wollzeile 15 (»Berthahof«)@\textbf{Wollzeile 15 (»Berthahof«)}, \emph{Wohngebäude}|pw}}. \pend{}{\bigskip}\vspace{1em}
\pstart{}{\pb}Lieber Richard,\pend\vspace{0.5em}
\pstart
           für die Anempfehlung von \textsc{Quisisana}\oindex{Pension Quisisana@\textbf{Pension Quisisana}, \emph{Hotel}|pw} meinen beſten Dank! Ich fühle mich hier{ }ſehr wohl, und habe überdies ein
                    sehr hübſches Parterrezi{\geminationm}er mit Ausblick aufs weite
                    Meer, das herrlichſte Wetter (ke{\geminationn}e keinen
                    Ueberzieher mehr) und{ }ſehr{ }ſympathiſche Geſellschaft (die malende Schweſter\pwindex{Rosenthal, Marie 28.\,3.\,1869 Lviv – 1942 Belgrad@\textsc{Rosenthal, Marie} (28.\,3.\,1869 Lviv – 1942 Belgrad), \emph{Malerin}|pwv}{ }\textsc{Rosenthal}\pwindex{Rosenthal, Moriz 17.\,12.\,1862 Lviv – 3.\,9.\,1946 New York City@\textsc{Rosenthal, Moriz} (17.\,12.\,1862 Lviv – 3.\,9.\,1946 New York City), \emph{Komponist, Pianist}|pw}’s und die \textsc{Sophie Link}\pwindex{Link, Sophie 1860 Budapest – 1.\,10.\,1900 New York City@\textsc{Link, Sophie} (1860 Budapest – 1.\,10.\,1900 New York City), \emph{Sängerin}|pw},{ }ſeit 6 Wochen in Berlin\oindex{Berlin@\textbf{Berlin}, \emph{Hauptstadt}|pw}{ }verheiratet\pwindex{Löwenstein, Harry 29.\,1.\,1851 Göttingen – 8.\,8.\,1907 Berlin@\textsc{Löwenstein, Harry} (29.\,1.\,1851 Göttingen – 8.\,8.\,1907 Berlin), \emph{Börsenmakler}|pwv}). – Ich
                    bin meiſt im Freien, und pendle zwiſchen \textsc{Lovrana}\oindex{Lovran@\textbf{Lovran}, \emph{Hauptstadt}|pw} und \textsc{Voloska}\oindex{Volosko@\textbf{Volosko}|pw}{ }{\pb}hin u her. – Gearbeitet – wenig; i{\geminationm}erhin ein Stück der Novellette\pwindex{Schnitzler, Arthur 15.\,5.\,1862 Wien – 21.\,10.\,1931 ebd.@\textsc{Schnitzler, Arthur} (15.\,5.\,1862 Wien – 21.\,10.\,1931 ebd.), \emph{Schriftsteller, Mediziner}!kleine Komödie@\strich\emph{Die kleine Komödie}|pwv}. – Die »Familie\pwindex{Schnitzler, Arthur 15.\,5.\,1862 Wien – 21.\,10.\,1931 ebd.@\textsc{Schnitzler, Arthur} (15.\,5.\,1862 Wien – 21.\,10.\,1931 ebd.), \emph{Schriftsteller, Mediziner}!Familie@\strich\emph{Familie}|pw}« durchgeleſen, merke, daſs was
                    fehlt, und bin nicht recht klar was. Ich werde es auch jedenfalls in 2–3 Wochen
                    vorleſen, aber um Rathschläge erſuchen müſſen. Keineswegs leſe ich, bevor wir
                    Ihre Novelle\pwindex{Beer-Hofmann, Richard 11.\,7.\,1866 Wien – 26.\,9.\,1945 New York City@\textsc{Beer-Hofmann, Richard} (11.\,7.\,1866 Wien – 26.\,9.\,1945 New York City), \emph{Schriftsteller}!Kind@\strich\emph{Das Kind}|pwv} zu hören
                    beko{\geminationm}en, was hoffentlich kurz nach meiner Ankunft
                    möglich{ }ſein wird! –\pend
           
\pstart
           – Ich denke nicht gern ans Fortreiſen; die Ruhe hier thut mir ganz
                    unbeſchreiblich wohl; wäre ich mein eigner Herr,{ }ſo blieb’ ich zwei Monate da.
                    We{\geminationn} man auch nicht {\pb}arbeitet, – man hat die Empfindung, daſs man es jeden Augenblick könnte, was
                    faſt noch mehr werth ist. – Hübſch wär’s, we{\geminationn} wir
                    das nächſte Frühjahr die ganze \textsc{Quisisana}\oindex{Pension Quisisana@\textbf{Pension Quisisana}, \emph{Hotel}|pw} miethen könnten! – Ah, diese Luft – einfach entzückend! – Es iſt doch
                    recht traurig zu den »Müſſenden« zu gehören! –\pend
           
\pstart
           Grüßen Sie \textsc{Loris}\pwindex{Hofmannsthal, Hugo von 1.\,2.\,1874 Wien – 15.\,7.\,1929 Rodaun@\textsc{Hofmannsthal, Hugo von} (1.\,2.\,1874 Wien – 15.\,7.\,1929 Rodaun), \emph{Schriftsteller}|pw} und \textsc{Salten}\pwindex{Salten, Felix 6.\,9.\,1869 Budapest – 8.\,10.\,1945 Zürich@\textsc{Salten, Felix} (6.\,9.\,1869 Budapest – 8.\,10.\,1945 Zürich), \emph{Schriftsteller, Journalist, Chefredakteur}|pw} aufs allerherzlichſte, desgleichen \textsc{Schwarzkopf}\pwindex{Schwarzkopf, Gustav 7.\,11.\,1853 Wien – 13.\,11.\,1939 ebd.@\textsc{Schwarzkopf, Gustav} (7.\,11.\,1853 Wien – 13.\,11.\,1939 ebd.), \emph{Schriftsteller}|pw}, der mir doch zwei Zeilen über das Befinden seines Bruders\pwindex{Schwarzkopf, Rudolf 25.\,5.\,1861 Wien – 13.\,10.\,1893 Meran@\textsc{Schwarzkopf, Rudolf} (25.\,5.\,1861 Wien – 13.\,10.\,1893 Meran), \emph{Schriftsteller}|pwv}{ }{ }ſchreiben möchte; und grüßen Sie nebſtbei
                    jedermann, der die Freundlichkeit hat nach mir zu fragen. – Schade, daſs {\pb}Sie nicht auch da{ }ſind! Hoffentlich find ich
                    Sie aber in geſegneterer Sti{\geminationm}ung als ich Sie
                    verlaſſen!\pend
           
\pstart
           Stets der Ihre {\\[\baselineskip]}\spacefill\mbox{Arthur.}\pend
           \leftskip=0em{}
\pstart
           \textsc{Abbazia}\oindex{Opatija@\textbf{Opatija}, \emph{Hauptstadt}|pw}{ }5. 3. 9\textcolor{gray}{3}. So{\geminationn}tag. –
                    \pend
           \selectlanguage{ngerman}\endnumbering\briefempfaengerindex{Beer-Hofmann, Richard@\textsc{Beer-Hofmann, Richard}!zzzSchnitzler, Arthur@\emph{von Arthur Schnitzler}!1893-03-051@{5. 3. 1893}|)be}\mylabel{L00185h}  \newcommand{\dateiname}{L00185}\newcommand{\titel}{Arthur Schnitzler an Richard Beer-Hofmann, 5. 3. 1893}\newcommand{\editorInnen}{Martin Anton Müller und Gerd-Hermann Susen}%% latex-leseansicht-abspann.tex
%% Abspann für die Leseansicht.
%% Der Schalter \ifkorrekturansicht ist bereits durch den Vorspann gesetzt.

%% latex-abspann.tex
%% Gemeinsamer Abspann für Korrekturansicht und Leseansicht.
%% Setzt den Schalter \ifkorrekturansicht voraus (gesetzt in den
%% einbindenden Dateien latex-korrekturansicht-abspann.tex bzw.
%% latex-leseansicht-abspann.tex).
%% ---------------------------------------------------------------

\normalsize

% Das esempio-Environment wird nur in der Leseansicht benötigt
\ifkorrekturansicht\else
\newenvironment{esempio}[3]%
{
    \vspace{1.5ex}
    \rlap{\underline{#1}}
    \par
    \setlength{\parindent}{0cm}
    \nopagebreak
    \leftskip=#2cm
    \rightskip=#3cm
}
{
    \par
}
\fi

\doendnotes{C}
\bigskip
\vfill

\clearpage

\footnotesize

\ifkorrekturansicht
  \lohead{\textsc{register}}
\fi

% theindex-Environment neu definieren ohne reledmac
\makeatletter
\renewenvironment{theindex}{%
  \ifkorrekturansicht
    \section*{\indexname}%
  \else
    \subsubsection*{Index der erwähnten Entitäten}%
  \fi
  \setlength{\parindent}{0pt}%
  \setlength{\parskip}{0pt plus 0.3pt}%
  \let\item\@idxitem
}{%
  \ifkorrekturansicht\clearpage\fi
}
\makeatother

\IfFileExists{\jobname-pw.ind}{\input{\jobname-pw.ind}}{}

% Quellenangabe nur in der Leseansicht
\ifkorrekturansicht\else
% Fallback-Definitionen, falls die .tex-Datei \titel etc. nicht gesetzt hat
\providecommand{\titel}{}
\providecommand{\editorInnen}{}
\providecommand{\dateiname}{\jobname}

\vspace{3cm}

\vfill

\footnotesize
\textsc{Quelle}: \titel. Herausgegeben von {\editorInnen}. In: \emph{Arthur Schnitzler: Briefwechsel mit Autorinnen und Autoren}.
 Digitale Edition, https://schnitzler-briefe.acdh.oeaw.ac.at/{\dateiname}.html (Stand \today)
\fi

\end{document}


