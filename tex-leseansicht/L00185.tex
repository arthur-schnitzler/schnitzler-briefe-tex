%% latex-korrekturansicht-vorspann.tex
%% Vorspann für die Korrekturansicht.
%% Lädt die gemeinsame Datei latex-vorspann.tex mit gesetztem Schalter.

\newif\ifkorrekturansicht
\korrekturansichttrue

\input{../tex-inputs/latex-vorspann}


\section[Arthur Schnitzler an Richard Beer-Hofmann, 5. 3. 1893]{L00185 Arthur Schnitzler an Richard Beer-Hofmann, 5. 3. 1893}
\nopagebreak\mylabel{L00185v}
\rehead{ }\normalsize\beginnumbering\briefempfaengerindex{Beer-Hofmann, Richard@\textsc{Beer-Hofmann, Richard}!zzzSchnitzler, Arthur@\emph{von Arthur Schnitzler}!1893-03-051@{5. 3. 1893}|(be}
\toendnotes[C]{\smallbreak\pagebreak[2]}\Standort{YCGL, MSS 31.}
\physDesc{Brief, 1 Blatt, 4 Seiten, Umschlag, 1667 Zeichen
\newline{}Handschrift: blaue Tinte, deutsche Kurrent
\newline{}Versand: 1) Stempel: »\nobreak{}\oindex{Pension Quisisana@\textbf{Pension Quisisana}, \emph{Hotel (K.HTL)}|pwk}Pension »Quisisana« Abbazia\nobreak{}«.   2) Stempel: »\nobreak{}\oindex{Opatija@\textbf{Opatija}, \emph{P.PPLA2}|pwk}Abbazia, 
                                       5/3 9
                                       \textcolor{gray}{3}\nobreak{}«.  3) Stempel: »\nobreak{}\oindex{I., Innere Stadt@\textbf{I., Innere Stadt}, \emph{A.ADM3}|pwk}Wien 1/1, 6/3. 93, 11½V–1N, Bestellt\nobreak{}«. }
\buchAbdrucke{\weitereDrucke{Arthur Schnitzler, Richard Beer-Hofmann: \emph{Briefwechsel 1891–1931}. Wien, Zürich: \emph{Europaverlag} 1992, S. 42.} }\toendnotes[C]{\smallbreak}\pstart{}{\pb}\textsc{Herrn Doctor Richard Beer-Hofmann}\pend{}\pstart{}\textsc{Wien\oindex{Wien@\textbf{Wien}, \emph{A.ADM2}|pw}}\pend{}\pstart{}\textsc{I Wollzeile 15\oindex{Wollzeile@\textbf{Wollzeile}, \emph{Straße (K.STR)}|pw}}
                  .
               \pend{}{\bigskip}\vspace{1em}
\pstart{}{\pb}
                  Lieber Richard,
               \pend\vspace{0.5em}
\pstart
           
               für die Anempfehlung von 
               \textsc{Quisisana}\oindex{Pension Quisisana@\textbf{Pension Quisisana}, \emph{Hotel (K.HTL)}|pw}
                meinen beſten Dank! Ich fühle mich hier ſehr wohl, und habe überdies ein sehr
               hübſches Parterrezi
               {\geminationm}
               er mit Ausblick aufs weite Meer, das
               herrlichſte Wetter (ke
               {\geminationn}
               e keinen Ueberzieher mehr) und
               ſehr ſympathiſche Geſellschaft (die malende 
               Schweſter\pwindex{Rosenthal, Marie 28.03.1869 – 1942@\textsc{Rosenthal, Marie} (28.03.1869 – 1942), \emph{Maler/Malerin}|pwv}{ }\textsc{Rosenthal}\pwindex{Rosenthal, Moriz 17.12.1862 – 03.09.1946@\textsc{Rosenthal, Moriz} (17.12.1862 – 03.09.1946), \emph{Komponist/Komponistin, Pianist/Pianistin}|pw}
               ’s und die 
               \textsc{Sophie Link}\pwindex{Link, Sophie 1860 – 01.10.1900@\textsc{Link, Sophie} (1860 – 01.10.1900), \emph{Sänger/Sängerin}|pw}
               , ſeit 6 Wochen in 
               Berlin\oindex{Berlin@\textbf{Berlin}, \emph{P.PPLC}|pw}{ }verheiratet\pwindex{Loewenstein, Harry 29.1.1851 – 8.8.1907@\textsc{Löwenstein, Harry} (29.1.1851 – 8.8.1907), \emph{Börsenmakler/Börsenmaklerin}|pwv}
               ). – Ich bin
               meiſt im Freien, und pendle zwiſchen 
               \textsc{Lovrana}\oindex{Lovran@\textbf{Lovran}, \emph{P.PPLA2}|pw}
                und 
               \textsc{Voloska}\oindex{Volosko@\textbf{Volosko}, \emph{P.PPL}|pw}{ }{\pb}
               hin u her. – Gearbeitet – wenig; i
               {\geminationm}
               erhin ein Stück der 
               Novellette\pwindex{kleine Komoedie@\emph{Die kleine Komödie}|pwv}
               . – Die »
               Familie\pwindex{Familie@\emph{Familie}|pw}
               « durchgeleſen, merke, daſs was fehlt, und bin nicht recht klar was.
               Ich werde es auch jedenfalls in 2–3 Wochen vorleſen, aber um Rathschläge erſuchen
               müſſen. Keineswegs leſe ich, bevor wir Ihre 
               Novelle\pwindex{Kind@\emph{Das Kind}|pwv}
                zu hören beko
               {\geminationm}
               en, was
               hoffentlich kurz nach meiner Ankunft möglich ſein wird! –
            \pend
           
\pstart
           
               – Ich denke nicht gern ans Fortreiſen; die Ruhe hier thut mir ganz unbeſchreiblich
               wohl; wäre ich mein eigner Herr, ſo blieb’ ich zwei Monate da. We
               {\geminationn}
                man auch nicht 
               {\pb}
               arbeitet, – man hat die Empfindung, daſs man es jeden Augenblick könnte, was faſt
               noch mehr werth ist. – Hübſch wär’s, we
               {\geminationn}
                wir das nächſte
               Frühjahr die ganze 
               \textsc{Quisisana}\oindex{Pension Quisisana@\textbf{Pension Quisisana}, \emph{Hotel (K.HTL)}|pw}
                miethen könnten! – Ah, diese Luft – einfach entzückend! – Es iſt doch recht
               traurig zu den »Müſſenden« zu gehören! –
            \pend
           
\pstart
           
               Grüßen Sie 
               \textsc{Loris}\pwindex{Hofmannsthal, Hugo von 1874-02-01 – 1929-07-15@\textsc{Hofmannsthal, Hugo von} (1874-02-01 – 1929-07-15), \emph{Schriftsteller/Schriftstellerin}|pw}
                und 
               \textsc{Salten}\pwindex{Salten, Felix 06.09.1869 – 08.10.1945@\textsc{Salten, Felix} (06.09.1869 – 08.10.1945), \emph{Schriftsteller/Schriftstellerin, Journalist/Journalistin, Chefredakteur/Chefredakteurin}|pw}
                aufs allerherzlichſte, desgleichen 
               \textsc{Schwarzkopf}\pwindex{Schwarzkopf, Gustav 07.11.1853 – 13.11.1939@\textsc{Schwarzkopf, Gustav} (07.11.1853 – 13.11.1939), \emph{Schriftsteller/Schriftstellerin}|pw}
               , der mir doch zwei Zeilen über das Befinden seines 
               Bruders\pwindex{Schwarzkopf, Rudolf 25.05.1861 – 13.10.1893@\textsc{Schwarzkopf, Rudolf} (25.05.1861 – 13.10.1893), \emph{Schriftsteller/Schriftstellerin}|pwv}{ }
               ſchreiben möchte; und grüßen Sie nebſtbei
               jedermann, der die Freundlichkeit hat nach mir zu fragen. – Schade, daſs 
               {\pb}
               Sie nicht auch da ſind! Hoffentlich find ich Sie aber
               in geſegneterer Sti
               {\geminationm}
               ung als ich Sie verlaſſen!
            \pend
           
\pstart
           
               Stets der Ihre
               {\\[\baselineskip]}\spacefill\mbox{Arthur.}\pend
           \leftskip=0em{}
\pstart
           \textsc{Abbazia}\oindex{Opatija@\textbf{Opatija}, \emph{P.PPLA2}|pw}
                     5. 3. 9
                     \textcolor{gray}{3}
                  . 
                  
                     So
                     {\geminationn}
                     tag
                  
                  . –
               \pend
           \selectlanguage{ngerman}\endnumbering\briefempfaengerindex{Beer-Hofmann, Richard@\textsc{Beer-Hofmann, Richard}!zzzSchnitzler, Arthur@\emph{von Arthur Schnitzler}!1893-03-051@{5. 3. 1893}|)be}\mylabel{L00185h}  \normalsize

\doendnotes{C}
\bigskip
\vfill

\clearpage

\footnotesize

\lohead{\textsc{register}}

% Definiere theindex-Environment komplett neu ohne reledmac
\makeatletter
\renewenvironment{theindex}{%
  \section*{\indexname}%
  \setlength{\parindent}{0pt}%
  \setlength{\parskip}{0pt plus 0.3pt}%
  \let\item\@idxitem
}{%
  \clearpage
}
\makeatother

\IfFileExists{\jobname-pw.ind}{\input{\jobname-pw.ind}}{}

\end{document}

      