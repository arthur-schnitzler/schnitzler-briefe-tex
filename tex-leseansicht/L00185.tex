%% latex-leseansicht-vorspann.tex
%% Vorspann für die Leseansicht.
%% Lädt die gemeinsame Datei latex-vorspann.tex mit nicht gesetztem Schalter.

\newif\ifkorrekturansicht
\korrekturansichtfalse

\input{../tex-inputs/latex-vorspann}


         
         \newcommand{\erwaehntePersonen}{Personen: Richard Beer-Hofmann, Hugo von Hofmannsthal, Sophie Link, Harry Löwenstein, Marie Rosenthal, Moritz Rosenthal, Felix Salten, Gustav Schwarzkopf, Rudolf Schwarzkopf}
         \newcommand{\erwaehnteOrte}{Orte: Berlin, I., Innere Stadt, Lovran, Opatija, Pension Quisisana, Volosko, Wien, Wollzeile}
         \newcommand{\erwaehnteWerke}{Werke: Das Kind, Die kleine Komödie, Familie}
               \section[Arthur Schnitzler an Richard Beer-Hofmann, 5. 3. 1893]{ Arthur Schnitzler an Richard Beer-Hofmann, 5. 3. 1893}\nopagebreak\mylabel{v}\rehead{ }\begin{ledgroupsized}[t]{13cm}\normalsize\beginnumbering \toendnotes[C]{\smallbreak\pagebreak[2]} \Standort{YCGL, MSS 31.}
\physDesc{Brief, 1 Blatt, 4 Seiten, Umschlag
\newline{}Handschrift: blaue Tinte, deutsche Kurrent\newline{}Versand: 1) Stempel: »\nobreak{}\oindex{Pension Quisisana@\textbf{Pension Quisisana}|pwk}Pension »Quisisana« Abbazia\nobreak{}«.   2) Stempel: »\nobreak{}\oindex{Opatija@\textbf{Opatija}|pwk}Abbazia, 5/3 9\textcolor{gray}{3}\nobreak{}«.  3) Stempel: »\nobreak{}\oindex{I., Innere Stadt@\textbf{I., Innere Stadt}|pwk}Wien 1/1, 6/3. 93, 11½V–1N, Bestellt\nobreak{}«. }\buchAbdrucke{\weitereDrucke{Arthur Schnitzler, Richard Beer-Hofmann: \emph{Briefwechsel 1891–1931}. Hg. Konstanze Fliedl. Wien, Zürich: \emph{Europaverlag} 1992, S. 42.} }\toendnotes[C]{\smallbreak}\pstart{}{\pb}\textsc{Herrn Doctor Richard Beer-Hofmann}\pend{}\pstart{}\textsc{Wien\oindex{Wien@\textbf{Wien}|pw}}\pend{}\pstart{}\textsc{I Wollzeile 15\oindex{Wollzeile@\textbf{Wollzeile}|pw}}.\pend{}{\bigskip}\pstart{}{\pb}Lieber Richard,\pend\pstart
           für die Anempfehlung von \textsc{Quisisana}\oindex{Pension Quisisana@\textbf{Pension Quisisana}|pw} meinen beſten Dank! Ich fühle mich hier ſehr wohl, und habe überdies ein sehr
               hübſches Parterrezi{\geminationm}er mit Ausblick aufs weite Meer, das
               herrlichſte Wetter (ke{\geminationn}e keinen Ueberzieher mehr) und
               ſehr ſympathiſche Geſellschaft (die malende Schweſter\pwindex{Rosenthal, Marie 28.03.1869 – 1942@\textsc{Rosenthal, Marie} (28.03.1869 – 1942), \emph{Malerin}|pwv}{ }\textsc{Rosenthal}\pwindex{Rosenthal, Moritz 17.12.1862 – 03.09.1946@\textsc{Rosenthal, Moritz} (17.12.1862 – 03.09.1946), \emph{Komponist, Pianist}|pw}’s und die \textsc{Sophie Link}\pwindex{Link, Sophie 1860 – 01.10.1900@\textsc{Link, Sophie} (1860 – 01.10.1900), \emph{Sängerin}|pw}, ſeit 6 Wochen in Berlin\oindex{Berlin@\textbf{Berlin}|pw}{ }verheiratet\pwindex{Loewenstein, Harry 29.1.1851 – 8.8.1907@\textsc{Löwenstein, Harry} (29.1.1851 – 8.8.1907), \emph{Börsenmakler}|pwv}). – Ich bin meiſt
               im Freien, und pendle zwiſchen \textsc{Lovrana}\oindex{Lovran@\textbf{Lovran}|pw} und \textsc{Voloska}\oindex{Volosko@\textbf{Volosko}|pw}{ }{\pb}hin u her. – Gearbeitet – wenig; i{\geminationm}erhin ein Stück der Novellette\pwindex{Schnitzler, Arthur 15.05.1862 – 21.10.1931@\textsc{Schnitzler, Arthur} (15.05.1862 – 21.10.1931), \emph{Schriftsteller, Mediziner}!kleine Komoedie1895-08-01@\strich\emph{Die kleine Komödie} {[}1895-08-01{]}|pwv}. – Die »Familie\pwindex{Schnitzler, Arthur 15.05.1862 – 21.10.1931@\textsc{Schnitzler, Arthur} (15.05.1862 – 21.10.1931), \emph{Schriftsteller, Mediziner}!Familie1977@\strich\emph{Familie} {[}1977{]}|pw}« durchgeleſen, merke, daſs was fehlt, und bin nicht recht klar was.
               Ich werde es auch jedenfalls in 2–3 Wochen vorleſen, aber um Rathschläge erſuchen
               müſſen. Keineswegs leſe ich, bevor wir Ihre Novelle\pwindex{Beer-Hofmann, Richard 1866-07-11 – 1945-09-26@\textsc{Beer-Hofmann, Richard} (1866-07-11 – 1945-09-26), \emph{Schriftsteller}!Kind1893@\strich\emph{Das Kind} {[}1893{]}|pwv} zu hören beko{\geminationm}en, was
               hoffentlich kurz nach meiner Ankunft möglich ſein wird! –\pend
           \pstart
           – Ich denke nicht gern ans Fortreiſen; die Ruhe hier thut mir ganz unbeſchreiblich
               wohl; wäre ich mein eigner Herr, ſo blieb’ ich zwei Monate da. We{\geminationn} man auch nicht {\pb}arbeitet, – man hat die Empfindung, daſs man es jeden Augenblick könnte, was faſt
               noch mehr werth ist. – Hübſch wär’s, we{\geminationn} wir das nächſte
               Frühjahr die ganze \textsc{Quisisana}\oindex{Pension Quisisana@\textbf{Pension Quisisana}|pw} miethen könnten! – Ah, diese Luft – einfach entzückend! – Es iſt doch recht
               traurig zu den »Müſſenden« zu gehören! –\pend
           \pstart
           Grüßen Sie \textsc{Loris}\pwindex{Hofmannsthal, Hugo von 1874-02-01 – 1929-07-15@\textsc{Hofmannsthal, Hugo von} (1874-02-01 – 1929-07-15), \emph{Schriftsteller}|pw} und \textsc{Salten}\pwindex{Salten, Felix 06.09.1869 – 08.10.1945@\textsc{Salten, Felix} (06.09.1869 – 08.10.1945), \emph{Schriftsteller, Journalist}|pw} aufs allerherzlichſte, desgleichen \textsc{Schwarzkopf}\pwindex{Schwarzkopf, Gustav 07.11.1853 – 13.11.1939@\textsc{Schwarzkopf, Gustav} (07.11.1853 – 13.11.1939), \emph{Schriftsteller}|pw}, der mir doch zwei Zeilen über das Befinden seines Bruders\pwindex{Schwarzkopf, Rudolf 25.05.1861 – 13.10.1893@\textsc{Schwarzkopf, Rudolf} (25.05.1861 – 13.10.1893), \emph{Schriftsteller}|pwv}{ }ſchreiben möchte; und grüßen Sie nebſtbei
               jedermann, der die Freundlichkeit hat nach mir zu fragen. – Schade, daſs {\pb}Sie nicht auch da ſind! Hoffentlich find ich Sie aber
               in geſegneterer Sti{\geminationm}ung als ich Sie verlaſſen!\pend
           \pstart
           Stets der Ihre{\\[\baselineskip]}\spacefill\mbox{Arthur.}\pend
           \leftskip=0em{}\pstart
           \textsc{Abbazia}\oindex{Opatija@\textbf{Opatija}|pw}5. 3. 9\textcolor{gray}{3}. So{\geminationn}tag. –\pend
           
         
         \endnumbering\mylabel{h}\end{ledgroupsized}  \newcommand{\dateiname}{L00185}\newcommand{\titel}{Arthur Schnitzler an Richard Beer-Hofmann, 5. 3. 1893}\newcommand{\editorInnen}{Martin Anton Müller und Gerd-Hermann Susen}%% latex-leseansicht-abspann.tex
%% Abspann für die Leseansicht.
%% Der Schalter \ifkorrekturansicht ist bereits durch den Vorspann gesetzt.

%% latex-abspann.tex
%% Gemeinsamer Abspann für Korrekturansicht und Leseansicht.
%% Setzt den Schalter \ifkorrekturansicht voraus (gesetzt in den
%% einbindenden Dateien latex-korrekturansicht-abspann.tex bzw.
%% latex-leseansicht-abspann.tex).
%% ---------------------------------------------------------------

\normalsize

% Das esempio-Environment wird nur in der Leseansicht benötigt
\ifkorrekturansicht\else
\newenvironment{esempio}[3]%
{
    \vspace{1.5ex}
    \rlap{\underline{#1}}
    \par
    \setlength{\parindent}{0cm}
    \nopagebreak
    \leftskip=#2cm
    \rightskip=#3cm
}
{
    \par
}
\fi

\doendnotes{C}
\bigskip
\vfill

\clearpage

\footnotesize

\ifkorrekturansicht
  \lohead{\textsc{register}}
\fi

% theindex-Environment neu definieren ohne reledmac
\makeatletter
\renewenvironment{theindex}{%
  \ifkorrekturansicht
    \section*{\indexname}%
  \else
    \subsubsection*{Index der erwähnten Entitäten}%
  \fi
  \setlength{\parindent}{0pt}%
  \setlength{\parskip}{0pt plus 0.3pt}%
  \let\item\@idxitem
}{%
  \ifkorrekturansicht\clearpage\fi
}
\makeatother

\IfFileExists{\jobname-pw.ind}{\input{\jobname-pw.ind}}{}

% Quellenangabe nur in der Leseansicht
\ifkorrekturansicht\else
% Fallback-Definitionen, falls die .tex-Datei \titel etc. nicht gesetzt hat
\providecommand{\titel}{}
\providecommand{\editorInnen}{}
\providecommand{\dateiname}{\jobname}

\vspace{3cm}

\vfill

\footnotesize
\textsc{Quelle}: \titel. Herausgegeben von {\editorInnen}. In: \emph{Arthur Schnitzler: Briefwechsel mit Autorinnen und Autoren}.
 Digitale Edition, https://schnitzler-briefe.acdh.oeaw.ac.at/{\dateiname}.html (Stand \today)
\fi

\end{document}


      