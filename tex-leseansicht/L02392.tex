%% latex-leseansicht-vorspann.tex
%% Vorspann für die Leseansicht.
%% Lädt die gemeinsame Datei latex-vorspann.tex mit nicht gesetztem Schalter.

\newif\ifkorrekturansicht
\korrekturansichtfalse

\input{../tex-inputs/latex-vorspann}


         
         \renewcommand{\erwaehntePersonen}{Personen: Thomas Mann, Scofield Thayer}
         \renewcommand{\erwaehnteOrte}{Orte: München, Niederlande, Wien}
         \renewcommand{\erwaehnteWerke}{Werke: Arthur Schnitzler. Zu seinem sechzigsten Geburtstag (15. Mai 1922), Betrachtungen eines Unpolitischen, Casanovas Heimfahrt, Der Zauberberg. Roman, Die neue Rundschau, Von deutscher Republik. Gerhart Hauptmann zum sechzigsten Geburtstag, Wilhelm Meister}
               \section[Thomas Mann an Arthur Schnitzler, 4. 9. 1922]{ Thomas Mann an Arthur Schnitzler, 4. 9. 1922}\nopagebreak\mylabel{v}\rehead{ }\begin{ledgroupsized}[t]{13cm}\normalsize\beginnumbering \toendnotes[C]{\smallbreak\pagebreak[2]} \Standort{CUL, Schnitzler, B 67.}
\physDesc{Brief, 1 Blatt, 3 Seiten, 1931 Zeichen
\newline{}Handschrift: schwarze Tinte, deutsche Kurrent
\newline{}Schnitzler: 1) mit Bleistift beschriftet: »\textsc{Thomas Mann}«, von unbekannter Hand »abg.« (für:
                                 abgeschrieben)  2) mit rotem Buntstift mehrere Unterstreichungen}\buchAbdrucke{\weitereDrucke{1) Thomas Mann: \emph{Briefe 1889–1936}. Mann, Erika. Frankfurt am Main: \emph{S. Fischer} 1961, S. 199.} \weitereDrucke{2) Hertha Krotkoff: \emph{Arthur Schnitzler – Thomas Mann: Briefe.} In: \emph{Modern Austrian Literature}, Jg. 7 (1974) Nr. 1/2, S. 18–19.} }\toendnotes[C]{\smallbreak}\pstart
           \raggedleft{}{\pb}München\oindex{Muenchen@\textbf{München}|pw} den
                  4. IX. 22.\pend
           \pstart{}Verehrter Herr Dr. Schnitzler,\pend\pstart
           ich habe Ihnen noch zu danken für die gütigen Zeilen, die mir Mr. Thayer\pwindex{Thayer, Scofield 12.12.1889 – 09.07.1982@\textsc{Thayer, Scofield} (12.12.1889 – 09.07.1982), \emph{Journalist, Herausgeber}|pw}, ein wirklich ſehr ſympathiſcher junger Mann, von Ihnen
               überbrachte. Es haben ſich aus dieſer Bekanntſchaft geſchäftliche Abmachungen
               ergeben, die mir als hochgradigem Familienvater höchſt angenehm ſein müſſen.\pend
           \pstart
           Eine große Freude war es mir, bei Gelegenheit Ihres 60. Geburtstags von der Liebe zu
                  zeugen\pwindex{Arthur Schnitzler. Zu seinem sechzigsten Geburtstag (15. Mai 1922)01. 05. 1922@\emph{Arthur Schnitzler. Zu seinem sechzigsten Geburtstag (15. Mai 1922)} {[}01. 05. 1922{]}|pwv}, mit der ich Ihrem
               bezaubernden Lebenswerk anhänge. Eben leſe ich Caſanovas Heimkehr\pwindex{Schnitzler, Arthur 15.05.1862 – 21.10.1931@\textsc{Schnitzler, Arthur} (15.05.1862 – 21.10.1931), \emph{Schriftsteller, Mediziner}!Casanovas Heimfahrt1.7.1918 – 1.9.1918@\strich\emph{Casanovas Heimfahrt} {[}1.7.1918 – 1.9.1918{]}|pw} – die Novelle war mir ſonderbarer Weiſe bisher unbekannt
               geblieben – und kann die tiefe Zufriedenheit nicht ſchildern, mit der ich {\pb}mich von Ihrer Erzählungskunſt tragen
               laſſe.\pend
           \pstart
           Im \label{K_L02392-1v}\edtext{Oktober-Heft}{\lemma{\textnormal{\emph{Oktober-Heft}}}\Cendnote{\textnormal{Es wurde November
                     (S. 1072–1106).}}}\label{K_L02392-1h} der Neuen
                  Rundſchau\pwindex{?? Werk@Nicht ermittelte Verfasserinnen und Verfasser!neue Rundschau1904@\emph{Die neue Rundschau} {[}1904{]}|pw} werden Sie einen größeren Beitrag von mir finden, einen Aufſatz,
               betitelt »Von deutſcher Republik\pwindex{Mann, Thomas 06.06.1875 – 12.08.1955@\textsc{Mann, Thomas} (06.06.1875 – 12.08.1955), \emph{Schriftsteller}!Von deutscher Republik. Gerhart Hauptmann zum sechzigsten Geburtstag01. 11. 1922@\strich\emph{Von deutscher Republik. Gerhart Hauptmann zum sechzigsten Geburtstag} {[}01. 11. 1922{]}|pw}«, der
               vielleicht gar durch zwei Hefte wird fortgeſetzt werden müſſen. Ich ermahne darin die
               renitenten Teile unſerer Jugend und unſeres Bürgertums ſich endlich vorbehaltlos in
               den Dienſt der Republik und der Humanität zu ſtellen, – eine Tendenz, über die Sie
               vielleicht erſtaunt ſein werden. Aber gerade als Verfaſſer der »Betrachtungen eines Unpolitiſchen\pwindex{Mann, Thomas 06.06.1875 – 12.08.1955@\textsc{Mann, Thomas} (06.06.1875 – 12.08.1955), \emph{Schriftsteller}!Betrachtungen eines Unpolitischen1918@\strich\emph{Betrachtungen eines Unpolitischen} {[}1918{]}|pw}« glaubte ich meinem Lande ein
               ſolches Manifeſt in dieſem Augenblick ſchuldig zu ſein. Und was die Verliebtheit in
               den Gedanken der Humanität betrifft, die ich ſeit einiger Zeit bei mir feſtstelle, ſo
               mag ſie mit dem Roman\pwindex{Mann, Thomas 06.06.1875 – 12.08.1955@\textsc{Mann, Thomas} (06.06.1875 – 12.08.1955), \emph{Schriftsteller}!Zauberberg. Roman1924@\strich\emph{Der Zauberberg. Roman} {[}1924{]}|pwv}
               zuſammenhängen, an dem ich ſchon {\pb}allzu
               lange ſchreibe, einer Art von Bildungsgeſchichte und Wilhelm Meiſter\pwindex{\textcolor{red}{\textsuperscript{XXXX1 indx}}!Wilhelm Meister1795 – 1821@\strich\emph{Wilhelm Meister} {[}1795 – 1821{]}|pwv}iade, worin ein junger Menſch (vor dem
               Kriege) durch das Erlebnis der Krankheit und des Todes zur Idee des Menſchen und des
               Staates geführt wird. – Verzeihen Sie die unerbetene Vertraulichkeit! – –\pend
           \pstart
           Im Oktober werde ich Ihren Spuren in Holland\oindex{Niederlande@\textbf{Niederlande}|pw} folgen. Im Januar{ }ſoll ich Wien\oindex{Wien@\textbf{Wien}|pw}
               wiederſehen und damit, ſo hoffe ich, Sie. Ich freue mich ſehr darauf.\pend
           \pstart
           In herzlicher Ehrerbietung Sie grüßend bin ich, lieber Herr Dr.
               Schnitzler,{\\[\baselineskip]}Ihr ergebenſter{\\[\baselineskip]}\spacefill\mbox{Thomas Mann.}\pend
           \leftskip=0em{}
         
         \endnumbering\mylabel{h}\end{ledgroupsized}  \newcommand{\dateiname}{L02392}\newcommand{\titel}{Thomas Mann an Arthur Schnitzler, 4. 9. 1922}\newcommand{\editorInnen}{Martin Anton Müller und Gerd-Hermann Susen}%% latex-leseansicht-abspann.tex
%% Abspann für die Leseansicht.
%% Der Schalter \ifkorrekturansicht ist bereits durch den Vorspann gesetzt.

%% latex-abspann.tex
%% Gemeinsamer Abspann für Korrekturansicht und Leseansicht.
%% Setzt den Schalter \ifkorrekturansicht voraus (gesetzt in den
%% einbindenden Dateien latex-korrekturansicht-abspann.tex bzw.
%% latex-leseansicht-abspann.tex).
%% ---------------------------------------------------------------

\normalsize

% Das esempio-Environment wird nur in der Leseansicht benötigt
\ifkorrekturansicht\else
\newenvironment{esempio}[3]%
{
    \vspace{1.5ex}
    \rlap{\underline{#1}}
    \par
    \setlength{\parindent}{0cm}
    \nopagebreak
    \leftskip=#2cm
    \rightskip=#3cm
}
{
    \par
}
\fi

\doendnotes{C}
\bigskip
\vfill

\clearpage

\footnotesize

\ifkorrekturansicht
  \lohead{\textsc{register}}
\fi

% theindex-Environment neu definieren ohne reledmac
\makeatletter
\renewenvironment{theindex}{%
  \ifkorrekturansicht
    \section*{\indexname}%
  \else
    \subsubsection*{Index der erwähnten Entitäten}%
  \fi
  \setlength{\parindent}{0pt}%
  \setlength{\parskip}{0pt plus 0.3pt}%
  \let\item\@idxitem
}{%
  \ifkorrekturansicht\clearpage\fi
}
\makeatother

\IfFileExists{\jobname-pw.ind}{\input{\jobname-pw.ind}}{}

% Quellenangabe nur in der Leseansicht
\ifkorrekturansicht\else
% Fallback-Definitionen, falls die .tex-Datei \titel etc. nicht gesetzt hat
\providecommand{\titel}{}
\providecommand{\editorInnen}{}
\providecommand{\dateiname}{\jobname}

\vspace{3cm}

\vfill

\footnotesize
\textsc{Quelle}: \titel. Herausgegeben von {\editorInnen}. In: \emph{Arthur Schnitzler: Briefwechsel mit Autorinnen und Autoren}.
 Digitale Edition, https://schnitzler-briefe.acdh.oeaw.ac.at/{\dateiname}.html (Stand \today)
\fi

\end{document}


      