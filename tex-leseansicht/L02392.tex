%% latex-leseansicht-vorspann.tex
%% Vorspann für die Leseansicht.
%% Lädt die gemeinsame Datei latex-vorspann.tex mit nicht gesetztem Schalter.

\newif\ifkorrekturansicht
\korrekturansichtfalse

\input{../tex-inputs/latex-vorspann}


\section[Thomas Mann an Arthur Schnitzler, 4. 9. 1922]{L02392 Thomas Mann an Arthur Schnitzler, 4. 9. 1922}
\nopagebreak\mylabel{L02392v}
\rehead{ }\normalsize\beginnumbering\briefempfaengerindex{Schnitzler, Arthur@\textsc{Schnitzler, Arthur}!zzzMann, Thomas@\emph{von Thomas Mann}!1922-09-041@{4. 9. 1922}|(be}
\toendnotes[C]{\smallbreak\pagebreak[2]}
\correspDesc{Versand  durch Thomas Mann am 4. 9. 1922 in München
\newline{}Erhalt  durch Arthur Schnitzler im Zeitraum [5. 9. 1922
                  – 9. 9. 1922?] in Wien}\toendnotes[C]{\smallbreak}
\Standort{CUL, Schnitzler, B 67.}
\physDesc{Brief, 1 Blatt, 3 Seiten, 1931 Zeichen
\newline{}Handschrift: schwarze Tinte, deutsche Kurrent
\newline{}Schnitzler: 1) mit Bleistift beschriftet: »\textsc{Thomas Mann}«, von unbekannter Hand »abg.« (für:
                                 abgeschrieben)  2) mit rotem Buntstift mehrere Unterstreichungen}
\buchAbdrucke{\weitereDrucke{1) Thomas Mann: \emph{Briefe 1889–1936}. Herausgegeben von Erika Mann. Frankfurt am Main: \emph{S. Fischer} 1961, S. 199.} \weitereDrucke{2) Hertha Krotkoff: \emph{Arthur Schnitzler – Thomas Mann: Briefe.} In: \emph{Modern Austrian Literature}, Jg. 7 (1974) Nr. 1/2, S. 18–19.} }\toendnotes[C]{\smallbreak}
\pstart
           \raggedleft{}{\pb}München\oindex{München@\textbf{München}|pw} den
                  4. IX. 22.\pend
           
\pstart{}Verehrter Herr Dr. Schnitzler,\pend\vspace{0.5em}
\pstart
           ich habe Ihnen noch zu danken für die gütigen Zeilen, die mir Mr. Thayer\pwindex{Thayer, Scofield 12.\,12.\,1889 Worcester – 9.\,7.\,1982 Edgartown@\textsc{Thayer, Scofield} (12.\,12.\,1889 Worcester – 9.\,7.\,1982 Edgartown), \emph{Journalist, Herausgeber}|pw}, ein wirklich{ }ſehr{ }ſympathiſcher junger Mann, von Ihnen
               überbrachte. Es haben{ }ſich aus dieſer Bekanntſchaft geſchäftliche Abmachungen
               ergeben, die mir als hochgradigem Familienvater höchſt angenehm{ }ſein müſſen.\pend
           
\pstart
           Eine große Freude war es mir, bei Gelegenheit Ihres 60. Geburtstags von der Liebe zu
                  zeugen\pwindex{Arthur Schnitzler. Zu seinem sechzigsten Geburtstag (15. Mai 1922)@\emph{Arthur Schnitzler. Zu seinem sechzigsten Geburtstag (15. Mai 1922)}|pwv}, mit der ich Ihrem
               bezaubernden Lebenswerk anhänge. Eben leſe ich Caſanovas Heimkehr\pwindex{Schnitzler, Arthur 15.\,5.\,1862 Wien – 21.\,10.\,1931 ebd.@\textsc{Schnitzler, Arthur} (15.\,5.\,1862 Wien – 21.\,10.\,1931 ebd.), \emph{Schriftsteller, Mediziner}!Casanovas Heimfahrt@\strich\emph{Casanovas Heimfahrt}|pw} – die Novelle war mir{ }ſonderbarer Weiſe bisher unbekannt
               geblieben – und kann die tiefe Zufriedenheit nicht{ }ſchildern, mit der ich {\pb}mich von Ihrer Erzählungskunſt tragen
               laſſe.\pend
           
\pstart
           Im \label{K_L02392-1v}\edtext{Oktober-Heft}{\lemma{\textnormal{\emph{Oktober-Heft}}}\Cendnote{\textnormal{Es wurde November: Thomas Mann\pwindex{Mann, Thomas 6.\,6.\,1875 Lübeck – 12.\,8.\,1955 Zürich@\textsc{Mann, Thomas} (6.\,6.\,1875 Lübeck – 12.\,8.\,1955 Zürich), \emph{Schriftsteller}|pwk}: \emph{Von deutscher Republik. Gerhart Hauptmann zum sechzigsten
                        Geburtstag}\pwindex{Mann, Thomas 6.\,6.\,1875 Lübeck – 12.\,8.\,1955 Zürich@\textsc{Mann, Thomas} (6.\,6.\,1875 Lübeck – 12.\,8.\,1955 Zürich), \emph{Schriftsteller}!Von deutscher Republik. Gerhart Hauptmann zum sechzigsten Geburtstag@\strich\emph{Von deutscher Republik. Gerhart Hauptmann zum sechzigsten Geburtstag}|pwk}. In: \emph{Die neue Rundschau}\pwindex{neue Rundschau@\emph{Die neue Rundschau}|pwk},
                     Jg. 33, H. 11, November 1922, S. 1072–1106.}}}\label{K_L02392-1} der Neuen Rundſchau\pwindex{neue Rundschau@\emph{Die neue Rundschau}|pw} werden Sie einen größeren Beitrag
               von mir finden, einen Aufſatz, betitelt »Von
                  deutſcher Republik\pwindex{Mann, Thomas 6.\,6.\,1875 Lübeck – 12.\,8.\,1955 Zürich@\textsc{Mann, Thomas} (6.\,6.\,1875 Lübeck – 12.\,8.\,1955 Zürich), \emph{Schriftsteller}!Von deutscher Republik. Gerhart Hauptmann zum sechzigsten Geburtstag@\strich\emph{Von deutscher Republik. Gerhart Hauptmann zum sechzigsten Geburtstag}|pw}«, der vielleicht gar durch zwei Hefte wird fortgeſetzt
               werden müſſen. Ich ermahne darin die renitenten Teile unſerer Jugend und unſeres
               Bürgertums{ }ſich endlich vorbehaltlos in den Dienſt der Republik und der Humanität zu{ }ſtellen, – eine Tendenz, über die Sie vielleicht erſtaunt{ }ſein werden. Aber gerade
               als Verfaſſer der »Betrachtungen eines
                  Unpolitiſchen\pwindex{Mann, Thomas 6.\,6.\,1875 Lübeck – 12.\,8.\,1955 Zürich@\textsc{Mann, Thomas} (6.\,6.\,1875 Lübeck – 12.\,8.\,1955 Zürich), \emph{Schriftsteller}!Betrachtungen eines Unpolitischen@\strich\emph{Betrachtungen eines Unpolitischen}|pw}« glaubte ich meinem Lande ein{ }ſolches Manifeſt in dieſem
               Augenblick{ }ſchuldig zu{ }ſein. Und was die Verliebtheit in den Gedanken der Humanität
               betrifft, die ich{ }ſeit einiger Zeit bei mir feſtstelle,{ }ſo mag{ }ſie mit dem Roman\pwindex{Mann, Thomas 6.\,6.\,1875 Lübeck – 12.\,8.\,1955 Zürich@\textsc{Mann, Thomas} (6.\,6.\,1875 Lübeck – 12.\,8.\,1955 Zürich), \emph{Schriftsteller}!Zauberberg. Roman@\strich\emph{Der Zauberberg. Roman}|pwv} zuſammenhängen, an dem
               ich{ }ſchon {\pb}allzu lange{ }ſchreibe, einer Art
               von Bildungsgeſchichte und Wilhelm
                  Meiſter\pwindex{\textcolor{red}{\textsuperscript{XXXX indx1}}!Wilhelm Meister@\strich\emph{Wilhelm Meister}|pwv}iade, worin ein junger Menſch (vor dem Kriege) durch das Erlebnis der
               Krankheit und des Todes zur Idee des Menſchen und des Staates geführt wird. –
               Verzeihen Sie die unerbetene Vertraulichkeit! – –\pend
           
\pstart
           Im Oktober werde ich Ihren Spuren in Holland\oindex{Niederlande@\textbf{Niederlande}|pw} folgen. Im Januar{ }ſoll ich Wien\oindex{Wien@\textbf{Wien}, \emph{Verwaltungsgebiet}|pw}
               wiederſehen und damit,{ }ſo hoffe ich, Sie. Ich freue mich{ }ſehr darauf.\pend
           
\pstart
           In herzlicher Ehrerbietung Sie grüßend bin ich, lieber Herr Dr.
               Schnitzler,{\\[\baselineskip]}Ihr ergebenſter{\\[\baselineskip]}\spacefill\mbox{Thomas Mann.}\pend
           \leftskip=0em{}\selectlanguage{ngerman}\endnumbering\briefempfaengerindex{Schnitzler, Arthur@\textsc{Schnitzler, Arthur}!zzzMann, Thomas@\emph{von Thomas Mann}!1922-09-041@{4. 9. 1922}|)be}\mylabel{L02392h}  \newcommand{\dateiname}{L02392}\newcommand{\titel}{Thomas Mann an Arthur Schnitzler, 4. 9. 1922}\newcommand{\editorInnen}{Martin Anton Müller und Gerd-Hermann Susen}%% latex-leseansicht-abspann.tex
%% Abspann für die Leseansicht.
%% Der Schalter \ifkorrekturansicht ist bereits durch den Vorspann gesetzt.

%% latex-abspann.tex
%% Gemeinsamer Abspann für Korrekturansicht und Leseansicht.
%% Setzt den Schalter \ifkorrekturansicht voraus (gesetzt in den
%% einbindenden Dateien latex-korrekturansicht-abspann.tex bzw.
%% latex-leseansicht-abspann.tex).
%% ---------------------------------------------------------------

\normalsize

% Das esempio-Environment wird nur in der Leseansicht benötigt
\ifkorrekturansicht\else
\newenvironment{esempio}[3]%
{
    \vspace{1.5ex}
    \rlap{\underline{#1}}
    \par
    \setlength{\parindent}{0cm}
    \nopagebreak
    \leftskip=#2cm
    \rightskip=#3cm
}
{
    \par
}
\fi

\doendnotes{C}
\bigskip
\vfill

\clearpage

\footnotesize

\ifkorrekturansicht
  \lohead{\textsc{register}}
\fi

% theindex-Environment neu definieren ohne reledmac
\makeatletter
\renewenvironment{theindex}{%
  \ifkorrekturansicht
    \section*{\indexname}%
  \else
    \subsubsection*{Index der erwähnten Entitäten}%
  \fi
  \setlength{\parindent}{0pt}%
  \setlength{\parskip}{0pt plus 0.3pt}%
  \let\item\@idxitem
}{%
  \ifkorrekturansicht\clearpage\fi
}
\makeatother

\IfFileExists{\jobname-pw.ind}{\input{\jobname-pw.ind}}{}

% Quellenangabe nur in der Leseansicht
\ifkorrekturansicht\else
% Fallback-Definitionen, falls die .tex-Datei \titel etc. nicht gesetzt hat
\providecommand{\titel}{}
\providecommand{\editorInnen}{}
\providecommand{\dateiname}{\jobname}

\vspace{3cm}

\vfill

\footnotesize
\textsc{Quelle}: \titel. Herausgegeben von {\editorInnen}. In: \emph{Arthur Schnitzler: Briefwechsel mit Autorinnen und Autoren}.
 Digitale Edition, https://schnitzler-briefe.acdh.oeaw.ac.at/{\dateiname}.html (Stand \today)
\fi

\end{document}


