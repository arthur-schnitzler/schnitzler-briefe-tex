%% latex-korrekturansicht-vorspann.tex
%% Vorspann für die Korrekturansicht.
%% Lädt die gemeinsame Datei latex-vorspann.tex mit gesetztem Schalter.

\newif\ifkorrekturansicht
\korrekturansichttrue

\input{../tex-inputs/latex-vorspann}


\section[Thomas Mann an Arthur Schnitzler, 4. 9. 1922]{L02392 Thomas Mann an Arthur Schnitzler, 4. 9. 1922}
\nopagebreak\mylabel{L02392v}
\rehead{ }\normalsize\beginnumbering\briefempfaengerindex{Schnitzler, Arthur@\textsc{Schnitzler, Arthur}!zzzMann, Thomas@\emph{von Thomas Mann}!1922-09-041@{4. 9. 1922}|(be}
\toendnotes[C]{\smallbreak\pagebreak[2]}\Standort{CUL, Schnitzler, B 67.}
\physDesc{Brief, 1 Blatt, 3 Seiten, 1931 Zeichen
\newline{}Handschrift: schwarze Tinte, deutsche Kurrent
\newline{}Schnitzler: 1) mit Bleistift beschriftet: »\textsc{Thomas Mann}«, von unbekannter Hand »abg.« (für:
                                 abgeschrieben)  2) mit rotem Buntstift mehrere Unterstreichungen}
\buchAbdrucke{\weitereDrucke{1) Thomas Mann: \emph{Briefe 1889–1936}. Frankfurt am Main: \emph{S. Fischer} 1961, S. 199.} \weitereDrucke{2) \emph{Modern Austrian Literature}, Jg. 7 (1974) Nr. 1/2, S. 18–19.} }\toendnotes[C]{\smallbreak}
\pstart
           \raggedleft{}{\pb}München\oindex{Muenchen@\textbf{München}, \emph{P.PPLA}|pw} den
                  4. IX. 22.\pend
           
\pstart{}Verehrter Herr Dr. Schnitzler,\pend\vspace{0.5em}
\pstart
           ich habe Ihnen noch zu danken für die gütigen Zeilen, die mir Mr. Thayer\pwindex{Thayer, Scofield 12.12.1889 – 09.07.1982@\textsc{Thayer, Scofield} (12.12.1889 – 09.07.1982), \emph{Journalist/Journalistin, Herausgeber/Herausgeberin}|pw}, ein wirklich ſehr ſympathiſcher junger Mann, von Ihnen
               überbrachte. Es haben ſich aus dieſer Bekanntſchaft geſchäftliche Abmachungen
               ergeben, die mir als hochgradigem Familienvater höchſt angenehm ſein müſſen.\pend
           
\pstart
           Eine große Freude war es mir, bei Gelegenheit Ihres 60. Geburtstags von der Liebe zu
                  zeugen\pwindex{Arthur Schnitzler. Zu seinem sechzigsten Geburtstag (15. Mai 1922)@\emph{Arthur Schnitzler. Zu seinem sechzigsten Geburtstag (15. Mai 1922)}|pwv}, mit der ich Ihrem
               bezaubernden Lebenswerk anhänge. Eben leſe ich Caſanovas Heimkehr\pwindex{Casanovas Heimfahrt@\emph{Casanovas Heimfahrt}|pw} – die Novelle war mir ſonderbarer Weiſe bisher unbekannt
               geblieben – und kann die tiefe Zufriedenheit nicht ſchildern, mit der ich {\pb}mich von Ihrer Erzählungskunſt tragen
               laſſe.\pend
           
\pstart
           Im \label{K_L02392-1v}\edtext{Oktober-Heft}{\lemma{\textnormal{\emph{Oktober-Heft}}}\Cendnote{\textnormal{Es wurde November: Thomas Mann\pwindex{Mann, Thomas 06.06.1875 – 12.08.1955@\textsc{Mann, Thomas} (06.06.1875 – 12.08.1955), \emph{Schriftsteller/Schriftstellerin}|pwk}: \emph{Von deutscher Republik. Gerhart Hauptmann zum sechzigsten
                        Geburtstag}\pwindex{Von deutscher Republik. Gerhart Hauptmann zum sechzigsten Geburtstag@\emph{Von deutscher Republik. Gerhart Hauptmann zum sechzigsten Geburtstag}|pwk}. In: \emph{Die neue Rundschau}\pwindex{neue Rundschau@\emph{Die neue Rundschau}|pwk},
                     Jg. 33, H. 11, November 1922, S. 1072–1106.}}}\label{K_L02392-1} der Neuen Rundſchau\pwindex{neue Rundschau@\emph{Die neue Rundschau}|pw} werden Sie einen größeren Beitrag
               von mir finden, einen Aufſatz, betitelt »Von
                  deutſcher Republik\pwindex{Von deutscher Republik. Gerhart Hauptmann zum sechzigsten Geburtstag@\emph{Von deutscher Republik. Gerhart Hauptmann zum sechzigsten Geburtstag}|pw}«, der vielleicht gar durch zwei Hefte wird fortgeſetzt
               werden müſſen. Ich ermahne darin die renitenten Teile unſerer Jugend und unſeres
               Bürgertums ſich endlich vorbehaltlos in den Dienſt der Republik und der Humanität zu
               ſtellen, – eine Tendenz, über die Sie vielleicht erſtaunt ſein werden. Aber gerade
               als Verfaſſer der »Betrachtungen eines
                  Unpolitiſchen\pwindex{Betrachtungen eines Unpolitischen@\emph{Betrachtungen eines Unpolitischen}|pw}« glaubte ich meinem Lande ein ſolches Manifeſt in dieſem
               Augenblick ſchuldig zu ſein. Und was die Verliebtheit in den Gedanken der Humanität
               betrifft, die ich ſeit einiger Zeit bei mir feſtstelle, ſo mag ſie mit dem Roman\pwindex{Zauberberg. Roman@\emph{Der Zauberberg. Roman}|pwv} zuſammenhängen, an dem
               ich ſchon {\pb}allzu lange ſchreibe, einer Art
               von Bildungsgeſchichte und Wilhelm
                  Meiſter\pwindex{Wilhelm Meister@\emph{Wilhelm Meister}|pwv}iade, worin ein junger Menſch (vor dem Kriege) durch das Erlebnis der
               Krankheit und des Todes zur Idee des Menſchen und des Staates geführt wird. –
               Verzeihen Sie die unerbetene Vertraulichkeit! – –\pend
           
\pstart
           Im Oktober werde ich Ihren Spuren in Holland\oindex{Niederlande@\textbf{Niederlande}, \emph{A.PCLI}|pw} folgen. Im Januar{ }ſoll ich Wien\oindex{Wien@\textbf{Wien}, \emph{A.ADM2}|pw}
               wiederſehen und damit, ſo hoffe ich, Sie. Ich freue mich ſehr darauf.\pend
           
\pstart
           In herzlicher Ehrerbietung Sie grüßend bin ich, lieber Herr Dr.
               Schnitzler,{\\[\baselineskip]}Ihr ergebenſter{\\[\baselineskip]}\spacefill\mbox{Thomas Mann.}\pend
           \leftskip=0em{}\selectlanguage{ngerman}\endnumbering\briefempfaengerindex{Schnitzler, Arthur@\textsc{Schnitzler, Arthur}!zzzMann, Thomas@\emph{von Thomas Mann}!1922-09-041@{4. 9. 1922}|)be}\mylabel{L02392h}  \normalsize

\doendnotes{C}
\bigskip
\vfill

\clearpage

\footnotesize

\lohead{\textsc{register}}

% Definiere theindex-Environment komplett neu ohne reledmac
\makeatletter
\renewenvironment{theindex}{%
  \section*{\indexname}%
  \setlength{\parindent}{0pt}%
  \setlength{\parskip}{0pt plus 0.3pt}%
  \let\item\@idxitem
}{%
  \clearpage
}
\makeatother

\IfFileExists{\jobname-pw.ind}{\input{\jobname-pw.ind}}{}

\end{document}

      