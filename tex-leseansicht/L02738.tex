%% latex-leseansicht-vorspann.tex
%% Vorspann für die Leseansicht.
%% Lädt die gemeinsame Datei latex-vorspann.tex mit nicht gesetztem Schalter.

\newif\ifkorrekturansicht
\korrekturansichtfalse

\input{../tex-inputs/latex-vorspann}


\section[Paul Goldmann an Arthur Schnitzler, 29. 6. [1895]]{L02738 Paul Goldmann an Arthur Schnitzler, 29. 6. [1895]}
\nopagebreak\mylabel{L02738v}
\rehead{ }\normalsize\beginnumbering\briefempfaengerindex{Schnitzler, Arthur@\textsc{Schnitzler, Arthur}!zzzGoldmann, Paul@\emph{von Paul Goldmann}!1895-06-291@{29. 6. [1895]}|(be}
\toendnotes[C]{\smallbreak\pagebreak[2]}
\correspDesc{Versand  durch Paul Goldmann am 29. 6. [1895] in Paris
\newline{}Erhalt  durch Arthur Schnitzler im Zeitraum [30. 6. 1895
                  – 4. 7. 1895?] in Wien}\toendnotes[C]{\smallbreak}
\Standort{DLA, A:Schnitzler, HS.NZ85.1.3165.}
\physDesc{Brief, 2 Blätter, 8 Seiten, 2396 Zeichen
\newline{}Handschrift: schwarze Tinte, deutsche Kurrent
\newline{}Schnitzler: 1) mit schwarzer Tinte das Jahr »\textcolor{gray}{9}5« vermerkt  2) mit rotem Buntstift zwei Unterstreichungen}\toendnotes[C]{\smallbreak}
\pstart
           {\pb}\textcolor{gray}{\textbf{\textbf{Frankfurter Zeitung\orgindex{Frankfurter Zeitung@Frankfurter Zeitung|pw}}}}\hfill \textsc{Paris\oindex{Paris@\textbf{Paris}, \emph{Hauptstadt}|pw}}, 29. Juni.\pend
           
\pstart
           \textcolor{gray}{\textbf{(\begin{otherlanguage}{french}Gazette de Francfort\end{otherlanguage}\orgindex{Frankfurter Zeitung@Frankfurter Zeitung|pw}).}}\pend
           
\pstart
           \textcolor{gray}{\textbf{\textbf{\begin{otherlanguage}{french}Fondateur M. L.
                              Sonnemann\pwindex{Sonnemann, Leopold 29.\,10.\,1831 Höchberg – 30.\,10.\,1909 Frankfurt am Main@\textsc{Sonnemann, Leopold} (29.\,10.\,1831 Höchberg – 30.\,10.\,1909 Frankfurt am Main), \emph{Journalist, Herausgeber}|pw}\end{otherlanguage}.}}}\pend
           
\pstart
           \begin{otherlanguage}{french}\textcolor{gray}{\textbf{Journal politique, financier,}}\end{otherlanguage}\pend
           
\pstart
           \begin{otherlanguage}{french}\textcolor{gray}{\textbf{commercial et littéraire.}}\end{otherlanguage}\pend
           
\pstart
           \begin{otherlanguage}{french}\textcolor{gray}{\textbf{\textbf{Paraissant trois fois par jour.}}}\end{otherlanguage}\pend
           
\pstart
           \begin{otherlanguage}{french}\textcolor{gray}{\textbf{\textbf{Bureau à Paris\oindex{Paris@\textbf{Paris}, \emph{Hauptstadt}|pw}}}}\end{otherlanguage}\pend
           
\pstart
           \begin{otherlanguage}{french}\textcolor{gray}{\textbf{\textbf{24. Rue Feydeau\oindex{rue Feydeau@\textbf{rue Feydeau}, \emph{Straße}|pw}.}}}\end{otherlanguage}\pend
           
\pstart\center{}Mein lieber Freund,\pend\vspace{0.5em}
\pstart
           Noch weiß ich nichts ganz Genaues über meinen Urlaub; aber die Sache wird ungefähr{ }ſo{ }ſein: zwiſchen dem 10. und 15.
                  Auguſt gehe ich nach \textsc{Toelz\oindex{Bad Tölz@\textbf{Bad Tölz}, \emph{Hauptstadt}|pw}}, das 2 Stunden Bahnfahrt von \textsc{Muenchen\oindex{München@\textbf{München}|pw}} entfernt iſt, u. gebrauche dort die Kur, drei oder vier Wochen, je nach
               ärztlicher Vorſchrift. {\pb}Dann wird mein Urlaub wohl
               zu Ende{ }ſein. Immerhin hoffe ich doch{ }ſo um den 5. September herum acht Tage in München\oindex{München@\textbf{München}|pw} verbringen zu können. Du kannſt Dir denken, wie leid es mir thut,
               Dir diesmal nicht mehr entgegenkommen zu können; denn auch mein liebſter Wunſch für
               dieſen Sommer wäre, dich zu treffen. Aber ich muß {\pb}etwas für die Geſundheit (?!) thun, denn ich bin gar{ }ſehr elend: Wie alſo, wenn Du
               Deine \label{K_L02738-1v}\edtext{Bicycle-\textsc{Tour}}{\lemma{\textnormal{\emph{Bicycle-Tour}}}\Cendnote{\textnormal{Am 24. 8. 1895 startete Schnitzler mit Felix
                     Salten\pwindex{Salten, Felix 6.\,9.\,1869 Budapest – 8.\,10.\,1945 Zürich@\textsc{Salten, Felix} (6.\,9.\,1869 Budapest – 8.\,10.\,1945 Zürich), \emph{Schriftsteller, Journalist, Chefredakteur}|pwk} eine Radtour in Salzburg\oindex{Salzburg@\textbf{Salzburg}, \emph{Verwaltungsgebiet}|pwk}. Am
                     25. 8. 1895 kam
                     Schnitzler in Bad Tölz\oindex{Bad Tölz@\textbf{Bad Tölz}, \emph{Hauptstadt}|pwk} an, wo er den nächsten Tag mit Goldmann\pwindex{Goldmann, Paul 31.\,1.\,1865 Breslau – 25.\,9.\,1935 Wien@\textsc{Goldmann, Paul} (31.\,1.\,1865 Breslau – 25.\,9.\,1935 Wien), \emph{Schriftsteller, Journalist}|pwk} verbrachte. Am 27. 8. 1895 fuhren Schnitzler und Salten\pwindex{Salten, Felix 6.\,9.\,1869 Budapest – 8.\,10.\,1945 Zürich@\textsc{Salten, Felix} (6.\,9.\,1869 Budapest – 8.\,10.\,1945 Zürich), \emph{Schriftsteller, Journalist, Chefredakteur}|pwk}
                  weiter nach München\oindex{München@\textbf{München}|pwk}, wohin auch Goldmann\pwindex{Goldmann, Paul 31.\,1.\,1865 Breslau – 25.\,9.\,1935 Wien@\textsc{Goldmann, Paul} (31.\,1.\,1865 Breslau – 25.\,9.\,1935 Wien), \emph{Schriftsteller, Journalist}|pwk} nachreiste.}}}\label{K_L02738-1} nach \textsc{Muenchen}\oindex{München@\textbf{München}|pw} auf den \substVorne{}\textsuperscript{December}\substDazwischen{}September\substHinten{} ließeſt, etwa \strikeout{z\textcolor{gray}{u}} nach Rückkehr \strikeout{v} von Kopenhagen\oindex{Kopenhagen@\textbf{Kopenhagen}, \emph{Hauptstadt}|pw}? Oder{ }ſonſt, wie Du willſt. Beſtimme, und ich
               werde{ }ſuchen, mich nach Dir zu richten.\pend
           
\pstart
           Von der Frau \textsc{Andreas\pwindex{Andreas-Salomé, Lou 12.\,2.\,1861 Sankt Petersburg – 5.\,2.\,1937 Göttingen@\textsc{Andreas-Salomé, Lou} (12.\,2.\,1861 Sankt Petersburg – 5.\,2.\,1937 Göttingen), \emph{Schriftstellerin}|pw}} hatte ich {\pb}\label{K_L02738-2v}\edtext{folgende kurzen Zeilen}{\lemma{\textnormal{\emph{folgende kurzen Zeilen}}}\Cendnote{\textnormal{Siehe XXXX Auszeichnungsfehler: Dokument L00445 nicht gefunden.
               }}}\label{K_L02738-2}, die ich Dir{ }ſende. Liebenswürdig, aber unnatürlich und gekünſtelt. Die \strikeout{Doppel} Doppel-Adjektive »tief und deutlich empfand ich«{ }ſind das beſte Zeichen dafür, daß man gar nichts empfindet. Oder nein? {\dotsfour}\pend
           
\pstart
           Nochmals von Herzen glückliche Reiſe, liebſter Freund! Ich freue mich, daß Dir {\pb}der Sommer diesmal ein{ }ſo reiches Programm bringt.
               Wie denkſt Du über eine Rückreiſe von \textsc{Kopenhagen\oindex{Kopenhagen@\textbf{Kopenhagen}, \emph{Hauptstadt}|pw} via Paris\oindex{Paris@\textbf{Paris}, \emph{Hauptstadt}|pw}}?\pend
           
\pstart
           Die  Aufführungs\pwindex{Schnitzler, Arthur 15.\,5.\,1862 Wien – 21.\,10.\,1931 ebd.@\textsc{Schnitzler, Arthur} (15.\,5.\,1862 Wien – 21.\,10.\,1931 ebd.), \emph{Schriftsteller, Mediziner}!Liebelei. Schauspiel in drei Akten@\strich\emph{Liebelei. Schauspiel in drei Akten}|pwv}-Chancen machen
               mir doch jetzt einen recht ernſten Eindruck. \label{K_L02738-3v}\edtext{\textsc{Sonnenthal\pwindex{Sonnenthal, Adolf von 21.\,12.\,1834 Budapest – 4.\,4.\,1909 Prag@\textsc{Sonnenthal, Adolf von} (21.\,12.\,1834 Budapest – 4.\,4.\,1909 Prag), \emph{Schauspieler}|pw}}, \textsc{Mitterwurzer\pwindex{Mitterwurzer, Friedrich 16.\,10.\,1844 Dresden – 13.\,2.\,1897 Wien@\textsc{Mitterwurzer, Friedrich} (16.\,10.\,1844 Dresden – 13.\,2.\,1897 Wien), \emph{Schauspieler}|pw}}}{\lemma{\textnormal{\emph{Sonnenthal, Mitterwurzer}}}\Cendnote{\textnormal{Bei der Uraufführung\eventindex{Burgtheater@\textbf{Burgtheater}!Uraufführung von Liebelei, Premiere von Rechte der Seele, 9.10.1895@Uraufführung von Liebelei, Premiere von Rechte der Seele, 9.10.1895|pwkv} von \emph{Liebelei}\pwindex{Schnitzler, Arthur 15.\,5.\,1862 Wien – 21.\,10.\,1931 ebd.@\textsc{Schnitzler, Arthur} (15.\,5.\,1862 Wien – 21.\,10.\,1931 ebd.), \emph{Schriftsteller, Mediziner}!Liebelei. Schauspiel in drei Akten@\strich\emph{Liebelei. Schauspiel in drei Akten}|pwk} am 9. 10. 1895 im Burgtheater\oindex{Wien@\textbf{Wien}!I., Innere Stadt@\textbf{I., Innere Stadt}!Burgtheater@\textbf{Burgtheater}, \emph{Theater}|pwk} spielte Adolf von
                     Sonnenthal\pwindex{Sonnenthal, Adolf von 21.\,12.\,1834 Budapest – 4.\,4.\,1909 Prag@\textsc{Sonnenthal, Adolf von} (21.\,12.\,1834 Budapest – 4.\,4.\,1909 Prag), \emph{Schauspieler}|pwk} den alten Weiring\pwindex{Schnitzler, Arthur 15.\,5.\,1862 Wien – 21.\,10.\,1931 ebd.@\textsc{Schnitzler, Arthur} (15.\,5.\,1862 Wien – 21.\,10.\,1931 ebd.), \emph{Schriftsteller, Mediziner}!Liebelei. Schauspiel in drei Akten@\strich\emph{Liebelei. Schauspiel in drei Akten}|pwkv}, Friedrich Mitterwurzer\pwindex{Mitterwurzer, Friedrich 16.\,10.\,1844 Dresden – 13.\,2.\,1897 Wien@\textsc{Mitterwurzer, Friedrich} (16.\,10.\,1844 Dresden – 13.\,2.\,1897 Wien), \emph{Schauspieler}|pwk}
                  den Herrn\pwindex{Schnitzler, Arthur 15.\,5.\,1862 Wien – 21.\,10.\,1931 ebd.@\textsc{Schnitzler, Arthur} (15.\,5.\,1862 Wien – 21.\,10.\,1931 ebd.), \emph{Schriftsteller, Mediziner}!Liebelei. Schauspiel in drei Akten@\strich\emph{Liebelei. Schauspiel in drei Akten}|pwkv} und Adele Sandrock\pwindex{Sandrock, Adele 19.\,8.\,1863 Rotterdam – 30.\,8.\,1937 Berlin@\textsc{Sandrock, Adele} (19.\,8.\,1863 Rotterdam – 30.\,8.\,1937 Berlin), \emph{Schauspielerin}|pwk} die Christine\pwindex{Schnitzler, Arthur 15.\,5.\,1862 Wien – 21.\,10.\,1931 ebd.@\textsc{Schnitzler, Arthur} (15.\,5.\,1862 Wien – 21.\,10.\,1931 ebd.), \emph{Schriftsteller, Mediziner}!Liebelei. Schauspiel in drei Akten@\strich\emph{Liebelei. Schauspiel in drei Akten}|pwkv}.}}}\label{K_L02738-3}, das wäre herrlich.
               Aber \strikeout{w\textcolor{gray}{e}} wer gibt das Mädel\pwindex{Schnitzler, Arthur 15.\,5.\,1862 Wien – 21.\,10.\,1931 ebd.@\textsc{Schnitzler, Arthur} (15.\,5.\,1862 Wien – 21.\,10.\,1931 ebd.), \emph{Schriftsteller, Mediziner}!Liebelei. Schauspiel in drei Akten@\strich\emph{Liebelei. Schauspiel in drei Akten}|pwv}? Und
                  {\pb}was hörſt Du aus \textsc{Berlin\oindex{Berlin@\textbf{Berlin}, \emph{Hauptstadt}|pw}}?\pend
           
\pstart
           Auch dieſe \label{K_L02738-4v}\edtext{reichliche Production}{\lemma{\textnormal{\emph{reichliche Production}}}\Cendnote{\textnormal{Zuletzt arbeitete Schnitzler an \emph{Freiwild}\pwindex{Schnitzler, Arthur 15.\,5.\,1862 Wien – 21.\,10.\,1931 ebd.@\textsc{Schnitzler, Arthur} (15.\,5.\,1862 Wien – 21.\,10.\,1931 ebd.), \emph{Schriftsteller, Mediziner}!Freiwild. Schauspiel in 3 Akten@\strich\emph{Freiwild. Schauspiel in 3 Akten}|pwk}, \emph{Die Frau des Weisen}\pwindex{Schnitzler, Arthur 15.\,5.\,1862 Wien – 21.\,10.\,1931 ebd.@\textsc{Schnitzler, Arthur} (15.\,5.\,1862 Wien – 21.\,10.\,1931 ebd.), \emph{Schriftsteller, Mediziner}!Frau des Weisen. Erzählung@\strich\emph{Die Frau des Weisen. Erzählung}|pwk} und \emph{Der Empfindsame}\pwindex{Schnitzler, Arthur 15.\,5.\,1862 Wien – 21.\,10.\,1931 ebd.@\textsc{Schnitzler, Arthur} (15.\,5.\,1862 Wien – 21.\,10.\,1931 ebd.), \emph{Schriftsteller, Mediziner}!Empfindsame. Eine Burleske@\strich\emph{Der Empfindsame. Eine Burleske}|pwk}.}}}\label{K_L02738-4} iſt{ }ſchön. Man{ }ſoll
               aber gar nicht darüber reden, ums nicht zu berufen. Ich{ }ſage eben nur, daß es{ }ſchön
               iſt.\pend
           
\pstart
           Verleger? Schreib’ ruhig an den \label{K_L02738-5v}\edtext{Mann\pwindex{Debarge, Louis 14.\,12.\,1859 Genf – 7.\,9.\,1937 ebd.@\textsc{Debarge, Louis} (14.\,12.\,1859 Genf – 7.\,9.\,1937 ebd.), \emph{Herausgeber}|pwuv}}{\lemma{\textnormal{\emph{Mann}}}\Cendnote{\textnormal{Louis Debarge\pwindex{Debarge, Louis 14.\,12.\,1859 Genf – 7.\,9.\,1937 ebd.@\textsc{Debarge, Louis} (14.\,12.\,1859 Genf – 7.\,9.\,1937 ebd.), \emph{Herausgeber}|pwk}, der Gründer und Herausgeber
                  der \emph{Semaine Littéraire}\orgindex{Semaine Littéraire@La Semaine Littéraire|pwk}. Seine Briefe an Schnitzler liegen heute im \emph{Deutschen Literaturarchiv Marbach},
                  HS.1985.1.2728.}}}\label{K_L02738-5} von der »\textsc{Semaine littéraire\orgindex{Semaine Littéraire@La Semaine Littéraire|pw}}.« Du brauchſt ja von der \label{K_L02738-6v}\edtext{\textsc{Mercure\pwindex{Mercure de France@\emph{Mercure de France}|pw}}-Notiz\pwindex{Albert, Henri 16.\,11.\,1869 Niederbronn-les-Bains – 3.\,8.\,1921 Straßburg@\textsc{Albert, Henri} (16.\,11.\,1869 Niederbronn-les-Bains – 3.\,8.\,1921 Straßburg), \emph{Journalist, Kritiker, Übersetzer}!Journaux et Revues. [Le dernier numéro]@\strich\emph{Journaux et Revues. [Le dernier numéro]}|pwv}}{\lemma{\textnormal{\emph{Mercure-Notiz}}}\Cendnote{\textnormal{Henri Albert\pwindex{Albert, Henri 16.\,11.\,1869 Niederbronn-les-Bains – 3.\,8.\,1921 Straßburg@\textsc{Albert, Henri} (16.\,11.\,1869 Niederbronn-les-Bains – 3.\,8.\,1921 Straßburg), \emph{Journalist, Kritiker, Übersetzer}|pwk}: \emph{Journaux et Revues. [Le dernier numéro]}\pwindex{Albert, Henri 16.\,11.\,1869 Niederbronn-les-Bains – 3.\,8.\,1921 Straßburg@\textsc{Albert, Henri} (16.\,11.\,1869 Niederbronn-les-Bains – 3.\,8.\,1921 Straßburg), \emph{Journalist, Kritiker, Übersetzer}!Journaux et Revues. [Le dernier numéro]@\strich\emph{Journaux et Revues. [Le dernier numéro]}|pwk}. In: \emph{Mercure de France}\pwindex{Mercure de France@\emph{Mercure de France}|pwk}, Jg. 12, Nr. 66, 1. 6. 1895, S. 371–372, hier: S. 372. Darin
                  berichtet Albert\pwindex{Albert, Henri 16.\,11.\,1869 Niederbronn-les-Bains – 3.\,8.\,1921 Straßburg@\textsc{Albert, Henri} (16.\,11.\,1869 Niederbronn-les-Bains – 3.\,8.\,1921 Straßburg), \emph{Journalist, Kritiker, Übersetzer}|pwk}, von Schnitzler um ein paar Worte anlässlich des Abdrucks von
                     \emph{Mourir}\pwindex{Schnitzler, Arthur 15.\,5.\,1862 Wien – 21.\,10.\,1931 ebd.@\textsc{Schnitzler, Arthur} (15.\,5.\,1862 Wien – 21.\,10.\,1931 ebd.), \emph{Schriftsteller, Mediziner}!Mourir. Roman@\strich\emph{Mourir. Roman}|pwk} in der \emph{Semaine littéraire}\orgindex{Semaine Littéraire@La Semaine Littéraire|pwk} gebeten worden zu sein. Da ihm der Leiter\pwindex{Debarge, Louis 14.\,12.\,1859 Genf – 7.\,9.\,1937 ebd.@\textsc{Debarge, Louis} (14.\,12.\,1859 Genf – 7.\,9.\,1937 ebd.), \emph{Herausgeber}|pwkv} der \emph{Semaine littéraire}\orgindex{Semaine Littéraire@La Semaine Littéraire|pwk} aber geschrieben habe, er dürfe nicht
                  erwähnen, dass das Liebespaar in \emph{Sterben}\pwindex{Schnitzler, Arthur 15.\,5.\,1862 Wien – 21.\,10.\,1931 ebd.@\textsc{Schnitzler, Arthur} (15.\,5.\,1862 Wien – 21.\,10.\,1931 ebd.), \emph{Schriftsteller, Mediziner}!Sterben. Novelle@\strich\emph{Sterben. Novelle}|pwk}
                  nicht verheiratet sei, habe er dankend abgelehnt.}}}\label{K_L02738-6} gar nichts zu wiſſen. Ich
               hab’{ }ſie {\pb}übrigens auch recht überflüſſig gefunden.
               Aber das iſt{ }ſo Pariſ\oindex{Paris@\textbf{Paris}, \emph{Hauptstadt}|pw}er Art: immer nur von{ }ſich
               reden. Alle haben{ }ſie hier was von \textsc{Hermann Bahr\pwindex{Bahr, Hermann 19.\,7.\,1863 Linz – 15.\,1.\,1934 München@\textsc{Bahr, Hermann} (19.\,7.\,1863 Linz – 15.\,1.\,1934 München), \emph{Schriftsteller, Kritiker}|pw}} an{ }ſich.\pend
           
\pstart
           Mit \label{K_L02738-7v}\edtext{\textsc{Langen\pwindex{Langen, Albert 8.\,7.\,1869 Antwerpen – 30.\,4.\,1909 München@\textsc{Langen, Albert} (8.\,7.\,1869 Antwerpen – 30.\,4.\,1909 München), \emph{Verleger}|pw}\orgindex{Albert Langen@Albert Langen|pw}}}{\lemma{\textnormal{\emph{Langen}}}\Cendnote{\textnormal{Siehe XXXX Auszeichnungsfehler: Dokument L02733 nicht gefunden.
               }}}\label{K_L02738-7} wird nichts zu machen{ }ſein. Er iſt ein blödſinniger Idiot. Er haßt mich, weil
               er weiß, daß ich weiß, daß er ein Idiot iſt; und er {\pb}haßt Dich, weil Du mein Freund biſt. Auch gibt er keine franzöſiſchen Bücher mehr
               heraus. Aber ich will einmal etwas Anderes durch \textsc{Henri Becque\pwindex{Becque, Henry 9.\,4.\,1837 Paris – 12.\,5.\,1899 Neuilly-sur-Seine@\textsc{Becque, Henry} (9.\,4.\,1837 Paris – 12.\,5.\,1899 Neuilly-sur-Seine), \emph{Schriftsteller, Dramatiker}|pw}} verſuchen.\pend
           
\pstart
           Soll’ ich Dir die fran\oindex{Frankreich@\textbf{Frankreich}|pwv}zöſiſchen
               Blätter, die ich für Dich{ }ſammle, auch nach unterwegs{ }ſchicken? Es macht mir gar
               nichts, denn ich{ }ſammle{ }ſo wie{ }ſo.\pend
           
\pstart
           Viele treue Grüße Dir und \textsc{Richard\pwindex{Beer-Hofmann, Richard 11.\,7.\,1866 Wien – 26.\,9.\,1945 New York City@\textsc{Beer-Hofmann, Richard} (11.\,7.\,1866 Wien – 26.\,9.\,1945 New York City), \emph{Schriftsteller}|pw}}. Von Herzen {\\[\baselineskip]}Dein {\\[\baselineskip]}\spacefill\mbox{Paul Goldmann\textcolor{gray}{.}}\pend
           \leftskip=0em{}\selectlanguage{ngerman}\endnumbering\briefempfaengerindex{Schnitzler, Arthur@\textsc{Schnitzler, Arthur}!zzzGoldmann, Paul@\emph{von Paul Goldmann}!1895-06-291@{29. 6. [1895]}|)be}\mylabel{L02738h}  \newcommand{\dateiname}{L02738}\newcommand{\titel}{Paul Goldmann an Arthur Schnitzler, 29. 6. [1895]}\newcommand{\editorInnen}{Martin Anton Müller und Laura Untner}%% latex-leseansicht-abspann.tex
%% Abspann für die Leseansicht.
%% Der Schalter \ifkorrekturansicht ist bereits durch den Vorspann gesetzt.

%% latex-abspann.tex
%% Gemeinsamer Abspann für Korrekturansicht und Leseansicht.
%% Setzt den Schalter \ifkorrekturansicht voraus (gesetzt in den
%% einbindenden Dateien latex-korrekturansicht-abspann.tex bzw.
%% latex-leseansicht-abspann.tex).
%% ---------------------------------------------------------------

\normalsize

% Das esempio-Environment wird nur in der Leseansicht benötigt
\ifkorrekturansicht\else
\newenvironment{esempio}[3]%
{
    \vspace{1.5ex}
    \rlap{\underline{#1}}
    \par
    \setlength{\parindent}{0cm}
    \nopagebreak
    \leftskip=#2cm
    \rightskip=#3cm
}
{
    \par
}
\fi

\doendnotes{C}
\bigskip
\vfill

\clearpage

\footnotesize

\ifkorrekturansicht
  \lohead{\textsc{register}}
\fi

% theindex-Environment neu definieren ohne reledmac
\makeatletter
\renewenvironment{theindex}{%
  \ifkorrekturansicht
    \section*{\indexname}%
  \else
    \subsubsection*{Index der erwähnten Entitäten}%
  \fi
  \setlength{\parindent}{0pt}%
  \setlength{\parskip}{0pt plus 0.3pt}%
  \let\item\@idxitem
}{%
  \ifkorrekturansicht\clearpage\fi
}
\makeatother

\IfFileExists{\jobname-pw.ind}{\input{\jobname-pw.ind}}{}

% Quellenangabe nur in der Leseansicht
\ifkorrekturansicht\else
% Fallback-Definitionen, falls die .tex-Datei \titel etc. nicht gesetzt hat
\providecommand{\titel}{}
\providecommand{\editorInnen}{}
\providecommand{\dateiname}{\jobname}

\vspace{3cm}

\vfill

\footnotesize
\textsc{Quelle}: \titel. Herausgegeben von {\editorInnen}. In: \emph{Arthur Schnitzler: Briefwechsel mit Autorinnen und Autoren}.
 Digitale Edition, https://schnitzler-briefe.acdh.oeaw.ac.at/{\dateiname}.html (Stand \today)
\fi

\end{document}


