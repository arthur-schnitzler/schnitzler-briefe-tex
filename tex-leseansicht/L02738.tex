%% latex-korrekturansicht-vorspann.tex
%% Vorspann für die Korrekturansicht.
%% Lädt die gemeinsame Datei latex-vorspann.tex mit gesetztem Schalter.

\newif\ifkorrekturansicht
\korrekturansichttrue

\input{../tex-inputs/latex-vorspann}


\section[Paul Goldmann an Arthur Schnitzler, 29. 6. {[}1895{]}]{L02738 Paul Goldmann an Arthur Schnitzler, 29. 6. {[}1895{]}}
\nopagebreak\mylabel{L02738v}
\rehead{ }\normalsize\beginnumbering\briefempfaengerindex{Schnitzler, Arthur@\textsc{Schnitzler, Arthur}!zzzGoldmann, Paul@\emph{von Paul Goldmann}!1895-06-291@{29. 6. {[}1895{]}}|(be}
\toendnotes[C]{\smallbreak\pagebreak[2]}\Standort{DLA, A:Schnitzler, HS.NZ85.1.3165.}
\physDesc{Brief, 2 Blätter, 8 Seiten, 2396 Zeichen
\newline{}Handschrift: schwarze Tinte, deutsche Kurrent
\newline{}Schnitzler: 1) mit schwarzer Tinte das Jahr »\textcolor{gray}{9}5« vermerkt  2) mit rotem Buntstift zwei Unterstreichungen}\toendnotes[C]{\smallbreak}
\pstart
           {\pb}\textcolor{gray}{\textbf{\textbf{Frankfurter Zeitung\orgindex{Frankfurter Zeitung@Frankfurter Zeitung|pw}}}}\hfill \textsc{Paris\oindex{Paris@\textbf{Paris}, \emph{P.PPLC}|pw}}, 29. Juni.\pend
           
\pstart
           \textcolor{gray}{\textbf{(\begin{otherlanguage}{french}Gazette de Francfort\end{otherlanguage}\orgindex{Frankfurter Zeitung@Frankfurter Zeitung|pw}). }}\pend
           
\pstart
           \textcolor{gray}{\textbf{\textbf{\begin{otherlanguage}{french}Fondateur M. L.
                              Sonnemann\pwindex{Sonnemann, Leopold 1831-10-29 – 1909-10-30@\textsc{Sonnemann, Leopold} (1831-10-29 – 1909-10-30), \emph{Journalist/Journalistin, Herausgeber/Herausgeberin}|pw}\end{otherlanguage}.}}}\pend
           
\pstart
           \begin{otherlanguage}{french}\textcolor{gray}{\textbf{Journal politique, financier,}}\end{otherlanguage}\pend
           
\pstart
           \begin{otherlanguage}{french}\textcolor{gray}{\textbf{commercial et littéraire.}}\end{otherlanguage}\pend
           
\pstart
           \begin{otherlanguage}{french}\textcolor{gray}{\textbf{\textbf{Paraissant trois fois par jour.}}}\end{otherlanguage}\pend
           
\pstart
           \begin{otherlanguage}{french}\textcolor{gray}{\textbf{\textbf{Bureau à Paris\oindex{Paris@\textbf{Paris}, \emph{P.PPLC}|pw}}}}\end{otherlanguage}\pend
           
\pstart
           \begin{otherlanguage}{french}\textcolor{gray}{\textbf{\textbf{24. Rue Feydeau\oindex{rue Feydeau@\textbf{rue Feydeau}, \emph{Straße (K.STR)}|pw}.}}}\end{otherlanguage}\pend
           
\pstart\center{}Mein lieber Freund,\pend\vspace{0.5em}
\pstart
           Noch weiß ich nichts ganz Genaues über meinen Urlaub; aber die Sache wird ungefähr ſo
               ſein: zwiſchen dem 10. und 15.
                  Auguſt gehe ich nach \textsc{Toelz\oindex{Bad Toelz@\textbf{Bad Tölz}, \emph{P.PPLA3}|pw}}, das 2 Stunden Bahnfahrt von \textsc{Muenchen\oindex{Muenchen@\textbf{München}, \emph{P.PPLA}|pw}} entfernt iſt, u. gebrauche dort die Kur, drei oder vier Wochen, je nach
               ärztlicher Vorſchrift. {\pb}Dann wird mein Urlaub wohl
               zu Ende ſein. Immerhin hoffe ich doch ſo um den 5. September herum acht Tage in München\oindex{Muenchen@\textbf{München}, \emph{P.PPLA}|pw} verbringen zu können. Du kannſt Dir denken, wie leid es mir thut,
               Dir diesmal nicht mehr entgegenkommen zu können; denn auch mein liebſter Wunſch für
               dieſen Sommer wäre, dich zu treffen. Aber ich muß {\pb}etwas für die Geſundheit (?!) thun, denn ich bin gar ſehr elend: Wie alſo, wenn Du
               Deine \label{K_L02738-1v}\edtext{Bicycle-\textsc{Tour}}{\lemma{\textnormal{\emph{Bicycle-Tour}}}\Cendnote{\textnormal{Am 24. 8. 1895 startete Schnitzler mit Felix
                     Salten\pwindex{Salten, Felix 06.09.1869 – 08.10.1945@\textsc{Salten, Felix} (06.09.1869 – 08.10.1945), \emph{Schriftsteller/Schriftstellerin, Journalist/Journalistin, Chefredakteur/Chefredakteurin}|pwk} eine Radtour in Salzburg\oindex{Salzburg@\textbf{Salzburg}, \emph{A.ADM2}|pwk}. Am
                     25. 8. 1895 kam
                     Schnitzler in Bad Tölz\oindex{Bad Toelz@\textbf{Bad Tölz}, \emph{P.PPLA3}|pwk} an, wo er den nächsten Tag mit Goldmann\pwindex{Goldmann, Paul 31.01.1865 – 25.09.1935@\textsc{Goldmann, Paul} (31.01.1865 – 25.09.1935), \emph{Schriftsteller/Schriftstellerin, Journalist/Journalistin}|pwk} verbrachte. Am 27. 8. 1895 fuhren Schnitzler und Salten\pwindex{Salten, Felix 06.09.1869 – 08.10.1945@\textsc{Salten, Felix} (06.09.1869 – 08.10.1945), \emph{Schriftsteller/Schriftstellerin, Journalist/Journalistin, Chefredakteur/Chefredakteurin}|pwk}
                  weiter nach München\oindex{Muenchen@\textbf{München}, \emph{P.PPLA}|pwk}, wohin auch Goldmann\pwindex{Goldmann, Paul 31.01.1865 – 25.09.1935@\textsc{Goldmann, Paul} (31.01.1865 – 25.09.1935), \emph{Schriftsteller/Schriftstellerin, Journalist/Journalistin}|pwk} nachreiste.}}}\label{K_L02738-1} nach \textsc{Muenchen}\oindex{Muenchen@\textbf{München}, \emph{P.PPLA}|pw} auf den \substVorne{}\textsuperscript{December}\substDazwischen{}September\substHinten{} ließeſt, etwa \strikeout{z\textcolor{gray}{u}} nach Rückkehr \strikeout{v} von Kopenhagen\oindex{Kopenhagen@\textbf{Kopenhagen}, \emph{P.PPLC}|pw}? Oder ſonſt, wie Du willſt. Beſtimme, und ich
               werde ſuchen, mich nach Dir zu richten.\pend
           
\pstart
           Von der Frau \textsc{Andreas\pwindex{Andreas-Salome, Lou 12.02.1861 – 05.02.1937@\textsc{Andreas-Salomé, Lou} (12.02.1861 – 05.02.1937), \emph{Schriftsteller/Schriftstellerin}|pw}} hatte ich {\pb}\label{K_L02738-2v}\edtext{folgende kurzen Zeilen}{\lemma{\textnormal{\emph{folgende kurzen Zeilen}}}\Cendnote{\textnormal{Siehe Lou Andreas-Salomé an Arthur Schnitzler, 25. 5. 1895.
               }}}\label{K_L02738-2}, die ich Dir ſende. Liebenswürdig, aber unnatürlich und gekünſtelt. Die \strikeout{Doppel} Doppel-Adjektive »tief und deutlich empfand ich«
               ſind das beſte Zeichen dafür, daß man gar nichts empfindet. Oder nein? {\dotsfour}\pend
           
\pstart
           Nochmals von Herzen glückliche Reiſe, liebſter Freund! Ich freue mich, daß Dir {\pb}der Sommer diesmal ein ſo reiches Programm bringt.
               Wie denkſt Du über eine Rückreiſe von \textsc{Kopenhagen\oindex{Kopenhagen@\textbf{Kopenhagen}, \emph{P.PPLC}|pw} via Paris\oindex{Paris@\textbf{Paris}, \emph{P.PPLC}|pw}}?\pend
           
\pstart
           Die  Aufführungs\pwindex{Liebelei. Schauspiel in drei Akten@\emph{Liebelei. Schauspiel in drei Akten}|pwv}-Chancen machen
               mir doch jetzt einen recht ernſten Eindruck. \label{K_L02738-3v}\edtext{\textsc{Sonnenthal\pwindex{Sonnenthal, Adolf von 1834-12-21 – 1909-04-04@\textsc{Sonnenthal, Adolf von} (1834-12-21 – 1909-04-04), \emph{Schauspieler/Schauspielerin}|pw}}, \textsc{Mitterwurzer\pwindex{Mitterwurzer, Friedrich 16.10.1844 – 13.02.1897@\textsc{Mitterwurzer, Friedrich} (16.10.1844 – 13.02.1897), \emph{Schauspieler/Schauspielerin}|pw}}}{\lemma{\textnormal{\emph{Sonnenthal, Mitterwurzer}}}\Cendnote{\textnormal{Bei der Uraufführung von \emph{Liebelei}\pwindex{Liebelei. Schauspiel in drei Akten@\emph{Liebelei. Schauspiel in drei Akten}|pwk} am 9. 10. 1895 im Burgtheater\oindex{Burgtheater@\textbf{Burgtheater}, \emph{S.THTR}|pwk} spielte Adolf von
                     Sonnenthal\pwindex{Sonnenthal, Adolf von 1834-12-21 – 1909-04-04@\textsc{Sonnenthal, Adolf von} (1834-12-21 – 1909-04-04), \emph{Schauspieler/Schauspielerin}|pwk} den alten Weiring\pwindex{Liebelei. Schauspiel in drei Akten@\emph{Liebelei. Schauspiel in drei Akten}|pwkv}, Friedrich Mitterwurzer\pwindex{Mitterwurzer, Friedrich 16.10.1844 – 13.02.1897@\textsc{Mitterwurzer, Friedrich} (16.10.1844 – 13.02.1897), \emph{Schauspieler/Schauspielerin}|pwk}
                  den Herrn\pwindex{Liebelei. Schauspiel in drei Akten@\emph{Liebelei. Schauspiel in drei Akten}|pwkv} und Adele Sandrock\pwindex{Sandrock, Adele 1863-08-19 – 1937-08-30@\textsc{Sandrock, Adele} (1863-08-19 – 1937-08-30), \emph{Schauspieler/Schauspielerin}|pwk} die Christine\pwindex{Liebelei. Schauspiel in drei Akten@\emph{Liebelei. Schauspiel in drei Akten}|pwkv}.}}}\label{K_L02738-3}, das wäre herrlich.
               Aber \strikeout{w\textcolor{gray}{e}} wer gibt das Mädel\pwindex{Liebelei. Schauspiel in drei Akten@\emph{Liebelei. Schauspiel in drei Akten}|pwv}? Und
                  {\pb}was hörſt Du aus \textsc{Berlin\oindex{Berlin@\textbf{Berlin}, \emph{P.PPLC}|pw}}?\pend
           
\pstart
           Auch dieſe \label{K_L02738-4v}\edtext{reichliche Production}{\lemma{\textnormal{\emph{reichliche Production}}}\Cendnote{\textnormal{Zuletzt arbeitete Schnitzler an \emph{Freiwild}\pwindex{Freiwild. Schauspiel in 3 Akten@\emph{Freiwild. Schauspiel in 3 Akten}|pwk}, \emph{Die Frau des Weisen}\pwindex{Frau des Weisen. Erzaehlung@\emph{Die Frau des Weisen. Erzählung}|pwk} und \emph{Der Empfindsame}\pwindex{Empfindsame. Eine Burleske@\emph{Der Empfindsame. Eine Burleske}|pwk}.}}}\label{K_L02738-4} iſt ſchön. Man ſoll
               aber gar nicht darüber reden, ums nicht zu berufen. Ich ſage eben nur, daß es ſchön
               iſt.\pend
           
\pstart
           Verleger? Schreib’ ruhig an den \label{K_L02738-5v}\edtext{Mann\pwindex{Debarge, Louis 1859-12-14 – 1937-09-07@\textsc{Debarge, Louis} (1859-12-14 – 1937-09-07), \emph{Herausgeber/Herausgeberin}|pwuv}}{\lemma{\textnormal{\emph{Mann}}}\Cendnote{\textnormal{Louis Debarge\pwindex{Debarge, Louis 1859-12-14 – 1937-09-07@\textsc{Debarge, Louis} (1859-12-14 – 1937-09-07), \emph{Herausgeber/Herausgeberin}|pwk}, der Gründer und Herausgeber
                  der \emph{Semaine Littéraire}\orgindex{Semaine Litteraire@La Semaine Littéraire|pwk}. Seine Briefe an Schnitzler liegen heute im \emph{Deutschen Literaturarchiv Marbach},
                  HS.1985.1.2728.}}}\label{K_L02738-5} von der »\textsc{Semaine littéraire\orgindex{Semaine Litteraire@La Semaine Littéraire|pw}}.« Du brauchſt ja von der \label{K_L02738-6v}\edtext{\textsc{Mercure\pwindex{Mercure de France@\emph{Mercure de France}|pw}}-Notiz\pwindex{Journaux et Revues. [Le dernier numero]@\emph{Journaux et Revues. [Le dernier numéro]}|pwv}}{\lemma{\textnormal{\emph{Mercure-Notiz}}}\Cendnote{\textnormal{Henri Albert\pwindex{Albert, Henri 1869-11-16 – 1921-08-03@\textsc{Albert, Henri} (1869-11-16 – 1921-08-03), \emph{Journalist/Journalistin, Kritiker/Kritikerin, Übersetzer/Übersetzerin}|pwk}: \emph{Journaux et Revues. [Le dernier numéro]}\pwindex{Journaux et Revues. [Le dernier numero]@\emph{Journaux et Revues. [Le dernier numéro]}|pwk}. In: \emph{Mercure de France}\pwindex{Mercure de France@\emph{Mercure de France}|pwk}, Jg. 12, Nr. 66, 1. 6. 1895, S. 371–372, hier: S. 372. Darin
                  berichtet Albert\pwindex{Albert, Henri 1869-11-16 – 1921-08-03@\textsc{Albert, Henri} (1869-11-16 – 1921-08-03), \emph{Journalist/Journalistin, Kritiker/Kritikerin, Übersetzer/Übersetzerin}|pwk}, von Schnitzler um ein paar Worte anlässlich des Abdrucks von
                     \emph{Mourir}\pwindex{Mourir. Roman@\emph{Mourir. Roman}|pwk} in der \emph{Semaine littéraire}\orgindex{Semaine Litteraire@La Semaine Littéraire|pwk} gebeten worden zu sein. Da ihm der Leiter\pwindex{Debarge, Louis 1859-12-14 – 1937-09-07@\textsc{Debarge, Louis} (1859-12-14 – 1937-09-07), \emph{Herausgeber/Herausgeberin}|pwkv} der \emph{Semaine littéraire}\orgindex{Semaine Litteraire@La Semaine Littéraire|pwk} aber geschrieben habe, er dürfe nicht
                  erwähnen, dass das Liebespaar in \emph{Sterben}\pwindex{Sterben. Novelle@\emph{Sterben. Novelle}|pwk}
                  nicht verheiratet sei, habe er dankend abgelehnt.}}}\label{K_L02738-6} gar nichts zu wiſſen. Ich
               hab’ ſie {\pb}übrigens auch recht überflüſſig gefunden.
               Aber das iſt ſo Pariſ\oindex{Paris@\textbf{Paris}, \emph{P.PPLC}|pw}er Art: immer nur von ſich
               reden. Alle haben ſie hier was von \textsc{Hermann Bahr\pwindex{Bahr, Hermann 19.07.1863 – 15.01.1934@\textsc{Bahr, Hermann} (19.07.1863 – 15.01.1934), \emph{Schriftsteller/Schriftstellerin, Kritiker/Kritikerin}|pw}} an ſich.\pend
           
\pstart
           Mit \label{K_L02738-7v}\edtext{\textsc{Langen\pwindex{Langen, Albert 1869-07-08 – 1909-04-30@\textsc{Langen, Albert} (1869-07-08 – 1909-04-30), \emph{Verleger/Verlegerin}|pw}\orgindex{Albert Langen@Albert Langen|pw}}}{\lemma{\textnormal{\emph{Langen}}}\Cendnote{\textnormal{Siehe Paul Goldmann an Arthur Schnitzler, 3. 4. [1895].
               }}}\label{K_L02738-7} wird nichts zu machen ſein. Er iſt ein blödſinniger Idiot. Er haßt mich, weil
               er weiß, daß ich weiß, daß er ein Idiot iſt; und er {\pb}haßt Dich, weil Du mein Freund biſt. Auch gibt er keine franzöſiſchen Bücher mehr
               heraus. Aber ich will einmal etwas Anderes durch \textsc{Henri Becque\pwindex{Becque, Henry 1837-04-09 – 1899-05-12@\textsc{Becque, Henry} (1837-04-09 – 1899-05-12), \emph{Schriftsteller/Schriftstellerin, Dramatiker/Dramatikerin}|pw}} verſuchen.\pend
           
\pstart
           Soll’ ich Dir die fran\oindex{Frankreich@\textbf{Frankreich}, \emph{A.PCLI}|pwv}zöſiſchen
               Blätter, die ich für Dich ſammle, auch nach unterwegs ſchicken? Es macht mir gar
               nichts, denn ich ſammle ſo wie ſo.\pend
           
\pstart
           Viele treue Grüße Dir und \textsc{Richard\pwindex{Beer-Hofmann, Richard 1866-07-11 – 1945-09-26@\textsc{Beer-Hofmann, Richard} (1866-07-11 – 1945-09-26), \emph{Schriftsteller/Schriftstellerin}|pw}}. Von Herzen {\\[\baselineskip]}Dein {\\[\baselineskip]}\spacefill\mbox{Paul Goldmann\textcolor{gray}{.}}\pend
           \leftskip=0em{}\selectlanguage{ngerman}\endnumbering\briefempfaengerindex{Schnitzler, Arthur@\textsc{Schnitzler, Arthur}!zzzGoldmann, Paul@\emph{von Paul Goldmann}!1895-06-291@{29. 6. {[}1895{]}}|)be}\mylabel{L02738h}  \normalsize

\doendnotes{C}
\bigskip
\vfill

\clearpage

\footnotesize

\lohead{\textsc{register}}

% Definiere theindex-Environment komplett neu ohne reledmac
\makeatletter
\renewenvironment{theindex}{%
  \section*{\indexname}%
  \setlength{\parindent}{0pt}%
  \setlength{\parskip}{0pt plus 0.3pt}%
  \let\item\@idxitem
}{%
  \clearpage
}
\makeatother

\IfFileExists{\jobname-pw.ind}{\input{\jobname-pw.ind}}{}

\end{document}

      