%% latex-leseansicht-vorspann.tex
%% Vorspann für die Leseansicht.
%% Lädt die gemeinsame Datei latex-vorspann.tex mit nicht gesetztem Schalter.

\newif\ifkorrekturansicht
\korrekturansichtfalse

\input{../tex-inputs/latex-vorspann}


\section[ Felix Salten an Arthur Schnitzler, {[}8. 2. 1896{]}]{L03169 Felix Salten an Arthur Schnitzler,  [8. 2. 1896]}
\nopagebreak\mylabel{L03169v}
\rehead{ }\normalsize\beginnumbering\briefempfaengerindex{Schnitzler, Arthur@\textsc{Schnitzler, Arthur}!zzzSalten, Felix@\emph{von Felix Salten}!1896-02-081@{{[}8. 2. 1896{]}}|(be}
\toendnotes[C]{\smallbreak\pagebreak[2]}
\correspDesc{Versand  durch Felix Salten am [8. 2. 1896] in Wien
\newline{}Erhalt  durch Arthur Schnitzler im Zeitraum [9. 2. 1896
                  – 11. 2. 1896?] in Berlin}\toendnotes[C]{\smallbreak}
\Standort{CUL, Schnitzler, B 89, A 1.}
\physDesc{Brief, 1 Blatt, 4 Seiten, 2241 Zeichen
\newline{}Handschrift: Bleistift, lateinische Kurrent
\newline{}Schnitzler: mit Bleistift datiert: »8/2 96« 
\newline{}Ordnung: mit Bleistift von unbekannter Hand nummeriert: »68.« }\toendnotes[C]{\smallbreak}
\pstart
           \raggedleft{}{\pb}Samstag.\pend
           \vspace{0.5em}
\pstart
           Lieber Freund, Nachtredacteur beim Neuen Wiener Tagblatt\orgindex{Neues Wiener Tagblatt@Neues Wiener Tagblatt|pw} ist ein Herr \label{K_L03169-1v}\edtext{Sigmund Hahn\pwindex{Hahn, Sigmund 1844 Dubá – 21.\,2.\,1929 Wien@\textsc{Hahn, Sigmund} (1844 Dubá – 21.\,2.\,1929 Wien), \emph{Redakteur}|pw}}{\lemma{\textnormal{\emph{Sigmund Hahn}}}\Cendnote{\textnormal{Schnitzler hielt sich in Berlin\oindex{Berlin@\textbf{Berlin}, \emph{Hauptstadt}|pwk} auf, wo am 4. 2. 1896 am \emph{Deutschen Theater}\orgindex{Deutsches Theater Berlin@Deutsches Theater Berlin|pwk} die gemeinsame Premiere von \emph{Liebelei}\pwindex{Schnitzler, Arthur 15. 5. 1862 Wien – 21. 10. 1931 ebd.@\textsc{Schnitzler, Arthur} (15. 5. 1862 Wien – 21. 10. 1931 ebd.), \emph{Schriftsteller, Mediziner}!Liebelei. Schauspiel in drei Akten@\strich\emph{Liebelei. Schauspiel in drei Akten}|pwk} und \emph{Der
                     zerbrochene Krug}\pwindex{Kleist, Heinrich von 18.\,10.\,1777 Frankfurt (Oder) – 21.\,11.\,1811 Kleiner Wannsee@\textsc{Kleist, Heinrich von} (18.\,10.\,1777 Frankfurt (Oder) – 21.\,11.\,1811 Kleiner Wannsee), \emph{Schriftsteller}!zerbrochene Krug. Ein Lustspiel in drei Aufzügen@\strich\emph{Der zerbrochene Krug. Ein Lustspiel in drei Aufzügen}|pwk} stattfand. Schnitzler erwähnt sowohl das Studium der Nachtkritiken (5. 2. 1896) wie auch
                  die Feuilletons (6. 2. 1896) in seinem \emph{Tagebuch}\pwindex{Schnitzler, Arthur 15. 5. 1862 Wien – 21. 10. 1931 ebd.@\textsc{Schnitzler, Arthur} (15. 5. 1862 Wien – 21. 10. 1931 ebd.), \emph{Schriftsteller, Mediziner}!Tagebuch@\strich\emph{Tagebuch}|pwk}.
                  Hier dürfte er der Notiz\pwindex{Theater und Kunst [Liebelei am Deutschen Theater]@\emph{Theater und Kunst [Liebelei am Deutschen Theater]}|pwkv}
                  im Abendblatt des \emph{Neuen Wiener Tagblatts}\pwindex{Neues Wiener Tagblatt@\emph{Neues Wiener Tagblatt}|pwk}:
                     [O. V.]: \emph{Theater und Kunst}\pwindex{Theater und Kunst [Liebelei am Deutschen Theater]@\emph{Theater und Kunst [Liebelei am Deutschen Theater]}|pwk}. In: \emph{Neues Wiener Abendblatt. Abend-Ausgabe des
                        »Neuen Wiener Tagblatt«}\pwindex{Neues Wiener Tagblatt@\emph{Neues Wiener Tagblatt}|pwk}, Jg. 30, Nr. 35, 5. 2 1896, S. 3 nachgeforscht haben.}}}\label{K_L03169-1}, von dem ich aber
               garnichts weiss. \label{K_L03169-2v}\edtext{Berlin\oindex{Berlin@\textbf{Berlin}, \emph{Hauptstadt}|pw} hat mir viele Freude}{\lemma{\textnormal{\emph{Berlin … Freude}}}\Cendnote{\textnormal{Salten\pwindex{Salten, Felix 6.\,9.\,1869 Budapest – 8.\,10.\,1945 Zürich@\textsc{Salten, Felix} (6.\,9.\,1869 Budapest – 8.\,10.\,1945 Zürich), \emph{Schriftsteller, Journalist, Chefredakteur}|pwk} zeigt sich erfreut darüber, dass die
                     Berlin\oindex{Berlin@\textbf{Berlin}, \emph{Hauptstadt}|pwk}er Inszenierung von \emph{Liebelei}\pwindex{Schnitzler, Arthur 15. 5. 1862 Wien – 21. 10. 1931 ebd.@\textsc{Schnitzler, Arthur} (15. 5. 1862 Wien – 21. 10. 1931 ebd.), \emph{Schriftsteller, Mediziner}!Liebelei. Schauspiel in drei Akten@\strich\emph{Liebelei. Schauspiel in drei Akten}|pwk} in der (Wien\oindex{Wien@\textbf{Wien}, \emph{Verwaltungsgebiet}|pwk}er) Presse viel und positiv besprochen wurde.}}}\label{K_L03169-2} gemacht,
                  \textcolor{gray}{–} das war sehr hübsch und hat hier\oindex{Wien@\textbf{Wien}, \emph{Verwaltungsgebiet}|pwv} gut gewirkt. Ludaßy\pwindex{Gans-Ludassy, Julius von 13.\,4.\,1858 Wien – 30.\,9.\,1922 ebd.@\textsc{Gans-Ludassy, Julius von} (13.\,4.\,1858 Wien – 30.\,9.\,1922 ebd.), \emph{Schriftsteller, Journalist, Herausgeber}|pw} verhält mich zu einer Revue über Ihre Berlin\oindex{Berlin@\textbf{Berlin}, \emph{Hauptstadt}|pw}er u. \label{K_L03169-3v}\edtext{Frankfurt\oindex{Frankfurt am Main@\textbf{Frankfurt am Main}, \emph{Hauptstadt}|pw}er Erfolge}{\lemma{\textnormal{\emph{Frankfurter Erfolge}}}\Cendnote{\textnormal{\emph{Liebelei}\pwindex{Schnitzler, Arthur 15. 5. 1862 Wien – 21. 10. 1931 ebd.@\textsc{Schnitzler, Arthur} (15. 5. 1862 Wien – 21. 10. 1931 ebd.), \emph{Schriftsteller, Mediziner}!Liebelei. Schauspiel in drei Akten@\strich\emph{Liebelei. Schauspiel in drei Akten}|pwk} wurde seit
                     11. 1. 1896 auch
                  in Frankfurt am Main\oindex{Frankfurt am Main@\textbf{Frankfurt am Main}, \emph{Hauptstadt}|pwk} am \emph{Städtischen Schauspielhaus}\orgindex{Frankfurter Stadttheater@Frankfurter Stadttheater|pwk} gegeben.}}}\label{K_L03169-3}, – wenn die
               Leute was reden, schieb ich es ihm auch zu. Trotzdem sind wir eine Clique. \label{K_L03169-4v}\edtext{Glauben Sie bei Fritz Mauthner\pwindex{Mauthner, Fritz 20.\,11.\,1849 Hořice – 29.\,6.\,1923 Meersburg@\textsc{Mauthner, Fritz} (20.\,11.\,1849 Hořice – 29.\,6.\,1923 Meersburg), \emph{Schriftsteller, Journalist, Philosoph}|pw}\pwindex{Mauthner, Fritz 20.\,11.\,1849 Hořice – 29.\,6.\,1923 Meersburg@\textsc{Mauthner, Fritz} (20.\,11.\,1849 Hořice – 29.\,6.\,1923 Meersburg), \emph{Schriftsteller, Journalist, Philosoph}!Deutsches Theater@\strich\emph{Deutsches Theater}|pwv} wirklich an Lothar\pwindex{Lothar, Rudolf 23.\,2.\,1865 Budapest – 2.\,10.\,1943 ebd.@\textsc{Lothar, Rudolf} (23.\,2.\,1865 Budapest – 2.\,10.\,1943 ebd.), \emph{Schriftsteller, Journalist, Theaterdirektor}|pw}}{\lemma{\textnormal{\emph{Glauben … Lothar}}}\Cendnote{\textnormal{Er fragt nach, ob er wirklich Rudolf Lothar\pwindex{Lothar, Rudolf 23.\,2.\,1865 Budapest – 2.\,10.\,1943 ebd.@\textsc{Lothar, Rudolf} (23.\,2.\,1865 Budapest – 2.\,10.\,1943 ebd.), \emph{Schriftsteller, Journalist, Theaterdirektor}|pwk} für den Stichwortgeber für die Besprechungen von 
                  { }Fritz Mauthner\pwindex{Mauthner, Fritz 20.\,11.\,1849 Hořice – 29.\,6.\,1923 Meersburg@\textsc{Mauthner, Fritz} (20.\,11.\,1849 Hořice – 29.\,6.\,1923 Meersburg), \emph{Schriftsteller, Journalist, Philosoph}|pwk} halte. Von Mauthner\pwindex{Mauthner, Fritz 20.\,11.\,1849 Hořice – 29.\,6.\,1923 Meersburg@\textsc{Mauthner, Fritz} (20.\,11.\,1849 Hořice – 29.\,6.\,1923 Meersburg), \emph{Schriftsteller, Journalist, Philosoph}|pwk} erschienen zwei Texte im
                     \emph{Berliner Tageblatt}\pwindex{Berliner Tageblatt@\emph{Berliner Tageblatt}|pwk}: Fr. M.\pwindex{Mauthner, Fritz 20.\,11.\,1849 Hořice – 29.\,6.\,1923 Meersburg@\textsc{Mauthner, Fritz} (20.\,11.\,1849 Hořice – 29.\,6.\,1923 Meersburg), \emph{Schriftsteller, Journalist, Philosoph}|pwkv} [ = Fritz Mauthner\pwindex{Mauthner, Fritz 20.\,11.\,1849 Hořice – 29.\,6.\,1923 Meersburg@\textsc{Mauthner, Fritz} (20.\,11.\,1849 Hořice – 29.\,6.\,1923 Meersburg), \emph{Schriftsteller, Journalist, Philosoph}|pwk}]: \emph{Deutsches Theater}\pwindex{Mauthner, Fritz 20.\,11.\,1849 Hořice – 29.\,6.\,1923 Meersburg@\textsc{Mauthner, Fritz} (20.\,11.\,1849 Hořice – 29.\,6.\,1923 Meersburg), \emph{Schriftsteller, Journalist, Philosoph}!Deutsches Theater@\strich\emph{Deutsches Theater}|pwk}. In: \emph{Berliner Tageblatt}\pwindex{Berliner Tageblatt@\emph{Berliner Tageblatt}|pwk}, Jg. 25, Nr. 64, 5. 2. 1896, Morgen-Ausgabe, S. 2–3; Fr. M.\pwindex{Mauthner, Fritz 20.\,11.\,1849 Hořice – 29.\,6.\,1923 Meersburg@\textsc{Mauthner, Fritz} (20.\,11.\,1849 Hořice – 29.\,6.\,1923 Meersburg), \emph{Schriftsteller, Journalist, Philosoph}|pwkv} [ = Fritz Mauthner\pwindex{Mauthner, Fritz 20.\,11.\,1849 Hořice – 29.\,6.\,1923 Meersburg@\textsc{Mauthner, Fritz} (20.\,11.\,1849 Hořice – 29.\,6.\,1923 Meersburg), \emph{Schriftsteller, Journalist, Philosoph}|pwk}]: \emph{Der zerbrochene Krug im Deutschen Theater}\pwindex{Mauthner, Fritz 20.\,11.\,1849 Hořice – 29.\,6.\,1923 Meersburg@\textsc{Mauthner, Fritz} (20.\,11.\,1849 Hořice – 29.\,6.\,1923 Meersburg), \emph{Schriftsteller, Journalist, Philosoph}!zerbrochene Krug im Deutschen Theater@\strich\emph{Der zerbrochene Krug im Deutschen Theater}|pwk}. In: \emph{Berliner Tageblatt}\pwindex{Berliner Tageblatt@\emph{Berliner Tageblatt}|pwk}, Jg. 25, Nr. 65, 5. 2. 1896, Abend-Ausgabe, S. 1–2.}}}\label{K_L03169-4}?
               In \label{K_L03169-5v}\edtext{Olmütz\oindex{Olomouc@\textbf{Olomouc}|pw} haben Sie einen großen Erfolg}{\lemma{\textnormal{\emph{Olmütz … Erfolg}}}\Cendnote{\textnormal{Am 30. 1. 1896 hatte am \emph{Königlich-Städtischem Theater zu Olmütz}\orgindex{Mährisches Theater Olmütz@Mährisches Theater Olmütz|pwk} die Premiere von \emph{Liebelei}\pwindex{Schnitzler, Arthur 15. 5. 1862 Wien – 21. 10. 1931 ebd.@\textsc{Schnitzler, Arthur} (15. 5. 1862 Wien – 21. 10. 1931 ebd.), \emph{Schriftsteller, Mediziner}!Liebelei. Schauspiel in drei Akten@\strich\emph{Liebelei. Schauspiel in drei Akten}|pwk} stattgefunden.}}}\label{K_L03169-5} gehabt, – sonst sind Sie
               weder in Brünn\oindex{Brünn@\textbf{Brünn}|pw} noch in Prag\oindex{Prag@\textbf{Prag}, \emph{Land}|pw} gewesen, das Mährische
                  Tagblatt\pwindex{Mährisches Tagblatt@\emph{Mährisches Tagblatt}|pw} heb’ ich Ihnen auf, – die \label{K_L03169-6v}\edtext{Kritik\pwindex{Liebelei«. Schauspiel in 3 Acten von Arthur Schnitzler@\emph{»Liebelei«. Schauspiel in 3 Acten von Arthur Schnitzler}|pwv}}{\lemma{\textnormal{\emph{Kritik}}}\Cendnote{\textnormal{[O. V.]: \emph{»Liebelei«. Schauspiel in 3 Acten
                        von Arthur Schnitzler}\pwindex{Liebelei«. Schauspiel in 3 Acten von Arthur Schnitzler@\emph{»Liebelei«. Schauspiel in 3 Acten von Arthur Schnitzler}|pwk}. In: \emph{Mährisches
                        Tagblatt}\pwindex{Mährisches Tagblatt@\emph{Mährisches Tagblatt}|pwk}, Jg. 17, Nr. 25, 31. 1. 1896,
                     S. 5–6.}}}\label{K_L03169-6} ist köstlich.\pend
           
\pstart
           Hier\oindex{Wien@\textbf{Wien}, \emph{Verwaltungsgebiet}|pwv} ist ein wunderschönes
               Frühlingswetter, das alle guten Vorsätze hervor{\pb}treibt und gute Laune
               schafft. Zudem habe ich noch \label{K_L03169-7v}\edtext{Frl. M.\pwindex{Salten, Ottilie 7.\,3.\,1868 Prag – 22.\,6.\,1942 Zürich@\textsc{Salten, Ottilie} (7.\,3.\,1868 Prag – 22.\,6.\,1942 Zürich), \emph{Schauspielerin}|pw}}{\lemma{\textnormal{\emph{Frl. M.}}}\Cendnote{\textnormal{Ottilie Metzl\pwindex{Salten, Ottilie 7.\,3.\,1868 Prag – 22.\,6.\,1942 Zürich@\textsc{Salten, Ottilie} (7.\,3.\,1868 Prag – 22.\,6.\,1942 Zürich), \emph{Schauspielerin}|pwk}, Saltens\pwindex{Salten, Felix 6.\,9.\,1869 Budapest – 8.\,10.\,1945 Zürich@\textsc{Salten, Felix} (6.\,9.\,1869 Budapest – 8.\,10.\,1945 Zürich), \emph{Schriftsteller, Journalist, Chefredakteur}|pwk} spätere Ehefrau}}}\label{K_L03169-7} – Neulich, es war Dienstag, erzählt sie mir, sie habe Alles der Frau Mitterwurzer\pwindex{Mitterwurzer, Wilhelmine 27.\,3.\,1848 Freiburg im Breisgau – 3.\,8.\,1909 Wien@\textsc{Mitterwurzer, Wilhelmine} (27.\,3.\,1848 Freiburg im Breisgau – 3.\,8.\,1909 Wien), \emph{Schauspielerin}|pw} gesagt. Diese sei sehr erschrocken
               und habe ihr dringend gerathen, den Verkehr mit mir aufzugeben. Darauf entgegnete
               Frl. M.\pwindex{Salten, Ottilie 7.\,3.\,1868 Prag – 22.\,6.\,1942 Zürich@\textsc{Salten, Ottilie} (7.\,3.\,1868 Prag – 22.\,6.\,1942 Zürich), \emph{Schauspielerin}|pw} sie könne das nicht, und Frau Mitterw.\pwindex{Mitterwurzer, Wilhelmine 27.\,3.\,1848 Freiburg im Breisgau – 3.\,8.\,1909 Wien@\textsc{Mitterwurzer, Wilhelmine} (27.\,3.\,1848 Freiburg im Breisgau – 3.\,8.\,1909 Wien), \emph{Schauspielerin}|pw} wünschte dann mich wenigstens kennen
               zu lernen. »\uline{Sie} wird mich gleich durch und durch
               schauen?« Natürlich. Sie will mich auch einladen und wir wollen uns bei ihr oben
               sehen. Tags darauf komme ich in die Redaction\orgindex{Wiener Allgemeine Zeitung@Wiener Allgemeine Zeitung|pwv} und erfahre, dass ich sogleich
               ein \label{K_L03169-8v}\edtext{Feuilleton\pwindex{Salten, Felix 6.\,9.\,1869 Budapest – 8.\,10.\,1945 Zürich@\textsc{Salten, Felix} (6.\,9.\,1869 Budapest – 8.\,10.\,1945 Zürich), \emph{Schriftsteller, Journalist, Chefredakteur}!Wilhelmine Mitterwurzer@\strich\emph{Wilhelmine Mitterwurzer}|pwv}}{\lemma{\textnormal{\emph{Feuilleton}}}\Cendnote{\textnormal{f. s.\pwindex{Salten, Felix 6.\,9.\,1869 Budapest – 8.\,10.\,1945 Zürich@\textsc{Salten, Felix} (6.\,9.\,1869 Budapest – 8.\,10.\,1945 Zürich), \emph{Schriftsteller, Journalist, Chefredakteur}|pwk} [ = Felix Salten\pwindex{Salten, Felix 6.\,9.\,1869 Budapest – 8.\,10.\,1945 Zürich@\textsc{Salten, Felix} (6.\,9.\,1869 Budapest – 8.\,10.\,1945 Zürich), \emph{Schriftsteller, Journalist, Chefredakteur}|pwk}]: \emph{Wilhelmine Mitterwurzer}\pwindex{Salten, Felix 6.\,9.\,1869 Budapest – 8.\,10.\,1945 Zürich@\textsc{Salten, Felix} (6.\,9.\,1869 Budapest – 8.\,10.\,1945 Zürich), \emph{Schriftsteller, Journalist, Chefredakteur}!Wilhelmine Mitterwurzer@\strich\emph{Wilhelmine Mitterwurzer}|pwk}. In: \emph{Wiener
                        Allgemeine Zeitung}\pwindex{Wiener Allgemeine Zeitung@\emph{Wiener Allgemeine Zeitung}|pwk}, Nr. 5382, 6. 2. 1896, S. 3.}}}\label{K_L03169-8} schreiben muss – über Frau Mitterwurzer\pwindex{Mitterwurzer, Wilhelmine 27.\,3.\,1848 Freiburg im Breisgau – 3.\,8.\,1909 Wien@\textsc{Mitterwurzer, Wilhelmine} (27.\,3.\,1848 Freiburg im Breisgau – 3.\,8.\,1909 Wien), \emph{Schauspielerin}|pw} – das Leben, – \substVorne{}\textsuperscript{s}\substDazwischen{}S\substHinten{}ie wissen schon.\pend
           
\pstart
           Richard\pwindex{Beer-Hofmann, Richard 11.\,7.\,1866 Wien – 26.\,9.\,1945 New York City@\textsc{Beer-Hofmann, Richard} (11.\,7.\,1866 Wien – 26.\,9.\,1945 New York City), \emph{Schriftsteller}|pw} ist sehr lieb, war neu{\pb}lich mit seinem \label{K_L03169-9v}\edtext{Mäderl\pwindex{Beer-Hofmann, Paula 25.\,2.\,1879 Wien – 30.\,10.\,1939 Zürich@\textsc{Beer-Hofmann, Paula} (25.\,2.\,1879 Wien – 30.\,10.\,1939 Zürich)|pwv}}{\lemma{\textnormal{\emph{Mäderl}}}\Cendnote{\textnormal{Paula Lissy\pwindex{Beer-Hofmann, Paula 25.\,2.\,1879 Wien – 30.\,10.\,1939 Zürich@\textsc{Beer-Hofmann, Paula} (25.\,2.\,1879 Wien – 30.\,10.\,1939 Zürich)|pwk}, Beer-Hofmanns\pwindex{Beer-Hofmann, Richard 11.\,7.\,1866 Wien – 26.\,9.\,1945 New York City@\textsc{Beer-Hofmann, Richard} (11.\,7.\,1866 Wien – 26.\,9.\,1945 New York City), \emph{Schriftsteller}|pwk} spätere Ehefrau. Die Geringschätzung, die in
                     Saltens\pwindex{Salten, Felix 6.\,9.\,1869 Budapest – 8.\,10.\,1945 Zürich@\textsc{Salten, Felix} (6.\,9.\,1869 Budapest – 8.\,10.\,1945 Zürich), \emph{Schriftsteller, Journalist, Chefredakteur}|pwk} Ausdrucksweise spürbar ist,
                  dürfte ein Ausdruck dessen sein, dass sie aus dem Kleinbürgertum stammte.}}}\label{K_L03169-9}
               im Josefstädter Theater\oindex{Wien@\textbf{Wien}!VIII., Josefstadt@\textbf{VIII., Josefstadt}!Theater in der Josefstadt@\textbf{Theater in der Josefstadt}, \emph{Theater}|pw}, und ist stolz darauf. Engländer\pwindex{Altenberg, Peter 9.\,3.\,1859 Wien – 8.\,1.\,1919 ebd.@\textsc{Altenberg, Peter} (9.\,3.\,1859 Wien – 8.\,1.\,1919 ebd.), \emph{Schriftsteller}|pw} war dabei, und erklärt sie natürlich
               für das Höchste.\pend
           
\pstart
           Sonntag war ich bei der \label{K_L03169-10v}\edtext{Matinée}{\lemma{\textnormal{\emph{Matinée}}}\Cendnote{\textnormal{Salten\pwindex{Salten, Felix 6.\,9.\,1869 Budapest – 8.\,10.\,1945 Zürich@\textsc{Salten, Felix} (6.\,9.\,1869 Budapest – 8.\,10.\,1945 Zürich), \emph{Schriftsteller, Journalist, Chefredakteur}|pwk} hatte eine kurze Rezension verfasst: f.\pwindex{Salten, Felix 6.\,9.\,1869 Budapest – 8.\,10.\,1945 Zürich@\textsc{Salten, Felix} (6.\,9.\,1869 Budapest – 8.\,10.\,1945 Zürich), \emph{Schriftsteller, Journalist, Chefredakteur}|pwk} [ = Felix Salten\pwindex{Salten, Felix 6.\,9.\,1869 Budapest – 8.\,10.\,1945 Zürich@\textsc{Salten, Felix} (6.\,9.\,1869 Budapest – 8.\,10.\,1945 Zürich), \emph{Schriftsteller, Journalist, Chefredakteur}|pwk}]: \emph{Matinée}\pwindex{Salten, Felix 6.\,9.\,1869 Budapest – 8.\,10.\,1945 Zürich@\textsc{Salten, Felix} (6.\,9.\,1869 Budapest – 8.\,10.\,1945 Zürich), \emph{Schriftsteller, Journalist, Chefredakteur}!Matinée@\strich\emph{Matinée}|pwk}. In: \emph{Wiener Allgemeine
                        Zeitung}\pwindex{Wiener Allgemeine Zeitung@\emph{Wiener Allgemeine Zeitung}|pwk}, Nr. 5380, 4. 2. 1896,
                     S. 4.}}}\label{K_L03169-10} im Theater auf der Wien\oindex{Wien@\textbf{Wien}!VI., Mariahilf@\textbf{VI., Mariahilf}!Theater an der Wien@\textbf{Theater an der Wien}, \emph{Theater}|pw}
               fortwährend auf der Bühne. Mitterwurzer\pwindex{Mitterwurzer, Friedrich 16.\,10.\,1844 Dresden – 13.\,2.\,1897 Wien@\textsc{Mitterwurzer, Friedrich} (16.\,10.\,1844 Dresden – 13.\,2.\,1897 Wien), \emph{Schauspieler}|pw} rief
               nach Aktschluss\pwindex{Schönthan-Pernwald, Paul von 19.\,3.\,1853 Wien – 4.\,8.\,1905 ebd.@\textsc{Schönthan-Pernwald, Paul von} (19.\,3.\,1853 Wien – 4.\,8.\,1905 ebd.), \emph{Schriftsteller, Journalist}!Gelegenheitskauf@\strich\emph{Gelegenheitskauf}|pwv} das Frl. M.\pwindex{Salten, Ottilie 7.\,3.\,1868 Prag – 22.\,6.\,1942 Zürich@\textsc{Salten, Ottilie} (7.\,3.\,1868 Prag – 22.\,6.\,1942 Zürich), \emph{Schauspielerin}|pw} sie solle mit ihm herauskommen, sich
               verbeugen, – sie wollte nicht, der schrie ihr nach: »Frl. Sandrock\pwindex{Sandrock, Adele 19.\,8.\,1863 Rotterdam – 30.\,8.\,1937 Berlin@\textsc{Sandrock, Adele} (19.\,8.\,1863 Rotterdam – 30.\,8.\,1937 Berlin), \emph{Schauspielerin}|pw}{ }Frl. Sandrock\pwindex{Sandrock, Adele 19.\,8.\,1863 Rotterdam – 30.\,8.\,1937 Berlin@\textsc{Sandrock, Adele} (19.\,8.\,1863 Rotterdam – 30.\,8.\,1937 Berlin), \emph{Schauspielerin}|pw}!« und als \label{K_L03169-11v}\edtext{sie}{\lemma{\textnormal{\emph{sie}}}\Cendnote{\textnormal{Salten\pwindex{Salten, Felix 6.\,9.\,1869 Budapest – 8.\,10.\,1945 Zürich@\textsc{Salten, Felix} (6.\,9.\,1869 Budapest – 8.\,10.\,1945 Zürich), \emph{Schriftsteller, Journalist, Chefredakteur}|pwk} schrieb »Sie«.}}}\label{K_L03169-11}
               ihn darauf aufmerksam machte, wurde er tobsüchtig. Von Frl. S.\pwindex{Sandrock, Adele 19.\,8.\,1863 Rotterdam – 30.\,8.\,1937 Berlin@\textsc{Sandrock, Adele} (19.\,8.\,1863 Rotterdam – 30.\,8.\,1937 Berlin), \emph{Schauspielerin}|pw} sind Kleinigkeiten zu berichten{\dotstwo} Ich befand mich ungeheuer wol und daheim auf der Bühne, und hab an Sie gedacht.
                  P. v. Schönthan\pwindex{Schönthan-Pernwald, Paul von 19.\,3.\,1853 Wien – 4.\,8.\,1905 ebd.@\textsc{Schönthan-Pernwald, Paul von} (19.\,3.\,1853 Wien – 4.\,8.\,1905 ebd.), \emph{Schriftsteller, Journalist}|pw} ging umher, und erzählte
               den Schauspielern, dass er dieses Stück\pwindex{Schönthan-Pernwald, Paul von 19.\,3.\,1853 Wien – 4.\,8.\,1905 ebd.@\textsc{Schönthan-Pernwald, Paul von} (19.\,3.\,1853 Wien – 4.\,8.\,1905 ebd.), \emph{Schriftsteller, Journalist}!Gelegenheitskauf@\strich\emph{Gelegenheitskauf}|pwv} mit seinem Herzblut geschrieben, – man überschätzt die Leute noch
               immer. Der Gelegenheits{\pb}kauf\pwindex{Schönthan-Pernwald, Paul von 19.\,3.\,1853 Wien – 4.\,8.\,1905 ebd.@\textsc{Schönthan-Pernwald, Paul von} (19.\,3.\,1853 Wien – 4.\,8.\,1905 ebd.), \emph{Schriftsteller, Journalist}!Gelegenheitskauf@\strich\emph{Gelegenheitskauf}|pw}{ }ist übrigens im Burgtheater\orgindex{Burgtheater@Burgtheater|pw} und im Lessingtheater\orgindex{Lessing-Theater@Lessing-Theater|pw}
               angenommen.\pend
           
\pstart
           Eben kommt das Repertoire. Sie sind in dieser Woche nicht drauf, was auch erklärlich
                  ist{[}.{]}{ }Dienstag kommt der Dornenweg\pwindex{Philippi, Felix 5.\,8.\,1851 Berlin – 23.\,11.\,1921 ebd.@\textsc{Philippi, Felix} (5.\,8.\,1851 Berlin – 23.\,11.\,1921 ebd.), \emph{Schriftsteller}!Dornenweg. Schauspiel in drei Aufzügen@\strich\emph{Der Dornenweg. Schauspiel in drei Aufzügen}|pw}. Da sind Sie ja bis Abends da, und \label{K_L03169-12v}\edtext{im Theater\oindex{Wien@\textbf{Wien}!I., Innere Stadt@\textbf{I., Innere Stadt}!Burgtheater@\textbf{Burgtheater}, \emph{Theater}|pwv}}{\lemma{\textnormal{\emph{im Theater}}}\Cendnote{\textnormal{Bei der Premiere von \emph{Der Dornenweg}\pwindex{Philippi, Felix 5.\,8.\,1851 Berlin – 23.\,11.\,1921 ebd.@\textsc{Philippi, Felix} (5.\,8.\,1851 Berlin – 23.\,11.\,1921 ebd.), \emph{Schriftsteller}!Dornenweg. Schauspiel in drei Aufzügen@\strich\emph{Der Dornenweg. Schauspiel in drei Aufzügen}|pwk} im Burgtheater\oindex{Wien@\textbf{Wien}!I., Innere Stadt@\textbf{I., Innere Stadt}!Burgtheater@\textbf{Burgtheater}, \emph{Theater}|pwk}, siehe A. S.: \emph{Tagebuch}, 11. 2. 1896.}}}\label{K_L03169-12}.\pend
           
\pstart
           Herzlichst Ihr {\\[\baselineskip]}\spacefill\mbox{Salten}\pend
           \leftskip=0em{}\selectlanguage{ngerman}\endnumbering\briefempfaengerindex{Schnitzler, Arthur@\textsc{Schnitzler, Arthur}!zzzSalten, Felix@\emph{von Felix Salten}!1896-02-081@{{[}8. 2. 1896{]}}|)be}\mylabel{L03169h}  \newcommand{\dateiname}{L03169}\newcommand{\titel}{Felix Salten an Arthur Schnitzler, [8. 2. 1896]}\newcommand{\editorInnen}{Martin Anton Müller und Laura Untner}%% latex-leseansicht-abspann.tex
%% Abspann für die Leseansicht.
%% Der Schalter \ifkorrekturansicht ist bereits durch den Vorspann gesetzt.

%% latex-abspann.tex
%% Gemeinsamer Abspann für Korrekturansicht und Leseansicht.
%% Setzt den Schalter \ifkorrekturansicht voraus (gesetzt in den
%% einbindenden Dateien latex-korrekturansicht-abspann.tex bzw.
%% latex-leseansicht-abspann.tex).
%% ---------------------------------------------------------------

\normalsize

% Das esempio-Environment wird nur in der Leseansicht benötigt
\ifkorrekturansicht\else
\newenvironment{esempio}[3]%
{
    \vspace{1.5ex}
    \rlap{\underline{#1}}
    \par
    \setlength{\parindent}{0cm}
    \nopagebreak
    \leftskip=#2cm
    \rightskip=#3cm
}
{
    \par
}
\fi

\doendnotes{C}
\bigskip
\vfill

\clearpage

\footnotesize

\ifkorrekturansicht
  \lohead{\textsc{register}}
\fi

% theindex-Environment neu definieren ohne reledmac
\makeatletter
\renewenvironment{theindex}{%
  \ifkorrekturansicht
    \section*{\indexname}%
  \else
    \subsubsection*{Index der erwähnten Entitäten}%
  \fi
  \setlength{\parindent}{0pt}%
  \setlength{\parskip}{0pt plus 0.3pt}%
  \let\item\@idxitem
}{%
  \ifkorrekturansicht\clearpage\fi
}
\makeatother

\IfFileExists{\jobname-pw.ind}{\input{\jobname-pw.ind}}{}

% Quellenangabe nur in der Leseansicht
\ifkorrekturansicht\else
% Fallback-Definitionen, falls die .tex-Datei \titel etc. nicht gesetzt hat
\providecommand{\titel}{}
\providecommand{\editorInnen}{}
\providecommand{\dateiname}{\jobname}

\vspace{3cm}

\vfill

\footnotesize
\textsc{Quelle}: \titel. Herausgegeben von {\editorInnen}. In: \emph{Arthur Schnitzler: Briefwechsel mit Autorinnen und Autoren}.
 Digitale Edition, https://schnitzler-briefe.acdh.oeaw.ac.at/{\dateiname}.html (Stand \today)
\fi

\end{document}


