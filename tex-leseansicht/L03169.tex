%% latex-korrekturansicht-vorspann.tex
%% Vorspann für die Korrekturansicht.
%% Lädt die gemeinsame Datei latex-vorspann.tex mit gesetztem Schalter.

\newif\ifkorrekturansicht
\korrekturansichttrue

\input{../tex-inputs/latex-vorspann}


\section[ Felix Salten an Arthur Schnitzler, {[}8. 2. 1896{]}]{L03169 Felix Salten an Arthur Schnitzler, {[}8. 2. 1896{]}}
\nopagebreak\mylabel{L03169v}
\rehead{ }\normalsize\beginnumbering\briefempfaengerindex{Schnitzler, Arthur@\textsc{Schnitzler, Arthur}!zzzSalten, Felix@\emph{von Felix Salten}!1896-02-081@{{[}8. 2. 1896{]}}|(be}
\toendnotes[C]{\smallbreak\pagebreak[2]}\Standort{CUL, Schnitzler, B 89, A 1.}
\physDesc{Brief, 1 Blatt, 4 Seiten, 2241 Zeichen
\newline{}Handschrift: Bleistift, lateinische Kurrent
\newline{}Schnitzler: mit Bleistift datiert: »8/2 96« 
\newline{}Ordnung: mit Bleistift von unbekannter Hand nummeriert: »68.« }\toendnotes[C]{\smallbreak}
\pstart
           \raggedleft{}{\pb}Samstag.\pend
           \vspace{0.5em}
\pstart
           Lieber Freund, Nachtredacteur beim Neuen Wiener Tagblatt\orgindex{Neues Wiener Tagblatt@Neues Wiener Tagblatt|pw} ist ein Herr \label{K_L03169-1v}\edtext{Sigmund Hahn\pwindex{Hahn, Sigmund 1844 – 1929-02-21@\textsc{Hahn, Sigmund} (1844 – 1929-02-21), \emph{Redakteur/Redakteurin}|pw}}{\lemma{\textnormal{\emph{Sigmund Hahn}}}\Cendnote{\textnormal{Schnitzler hielt sich in Berlin\oindex{Berlin@\textbf{Berlin}, \emph{P.PPLC}|pwk} auf, wo am 4. 2. 1896 am \emph{Deutschen Theater}\orgindex{Deutsches Theater Berlin@Deutsches Theater Berlin|pwk} die gemeinsame Premiere von \emph{Liebelei}\pwindex{Liebelei. Schauspiel in drei Akten@\emph{Liebelei. Schauspiel in drei Akten}|pwk} und \emph{Der
                     zerbrochene Krug}\pwindex{zerbrochene Krug. Ein Lustspiel in drei Aufzuegen@\emph{Der zerbrochene Krug. Ein Lustspiel in drei Aufzügen}|pwk} stattfand. Schnitzler erwähnt sowohl das Studium der Nachtkritiken (5. 2. 1896) wie auch
                  die Feuilletons (6. 2. 1896) in seinem \emph{Tagebuch}\pwindex{Tagebuch@\emph{Tagebuch}|pwk}.
                  Hier dürfte er der Notiz\pwindex{Theater und Kunst [Liebelei am Deutschen Theater]@\emph{Theater und Kunst [Liebelei am Deutschen Theater]}|pwkv}
                  im Abendblatt des \emph{Neuen Wiener Tagblatts}\pwindex{Neues Wiener Tagblatt@\emph{Neues Wiener Tagblatt}|pwk}:
                     [O. V.]: \emph{Theater und Kunst}\pwindex{Theater und Kunst [Liebelei am Deutschen Theater]@\emph{Theater und Kunst [Liebelei am Deutschen Theater]}|pwk}. In: \emph{Neues Wiener Abendblatt. Abend-Ausgabe des
                        »Neuen Wiener Tagblatt«}\pwindex{Neues Wiener Tagblatt@\emph{Neues Wiener Tagblatt}|pwk}, Jg. 30, Nr. 35, 5. 2 1896, S. 3 nachgeforscht haben.}}}\label{K_L03169-1}, von dem ich aber
               garnichts weiss. \label{K_L03169-2v}\edtext{Berlin\oindex{Berlin@\textbf{Berlin}, \emph{P.PPLC}|pw} hat mir viele Freude}{\lemma{\textnormal{\emph{Berlin … Freude}}}\Cendnote{\textnormal{Salten\pwindex{Salten, Felix 06.09.1869 – 08.10.1945@\textsc{Salten, Felix} (06.09.1869 – 08.10.1945), \emph{Schriftsteller/Schriftstellerin, Journalist/Journalistin, Chefredakteur/Chefredakteurin}|pwk} zeigt sich erfreut darüber, dass die
                     Berlin\oindex{Berlin@\textbf{Berlin}, \emph{P.PPLC}|pwk}er Inszenierung von \emph{Liebelei}\pwindex{Liebelei. Schauspiel in drei Akten@\emph{Liebelei. Schauspiel in drei Akten}|pwk} in der (Wien\oindex{Wien@\textbf{Wien}, \emph{A.ADM2}|pwk}er) Presse viel und positiv besprochen wurde.}}}\label{K_L03169-2} gemacht,
                  \textcolor{gray}{–} das war sehr hübsch und hat hier\oindex{Wien@\textbf{Wien}, \emph{A.ADM2}|pwv} gut gewirkt. Ludaßy\pwindex{Gans-Ludassy, Julius von 13.04.1858 – 30.09.1922@\textsc{Gans-Ludassy, Julius von} (13.04.1858 – 30.09.1922), \emph{Schriftsteller/Schriftstellerin, Journalist/Journalistin, Herausgeber/Herausgeberin}|pw} verhält mich zu einer Revue über Ihre Berlin\oindex{Berlin@\textbf{Berlin}, \emph{P.PPLC}|pw}er u. \label{K_L03169-3v}\edtext{Frankfurt\oindex{Frankfurt am Main@\textbf{Frankfurt am Main}, \emph{P.PPLA3}|pw}er Erfolge}{\lemma{\textnormal{\emph{Frankfurter Erfolge}}}\Cendnote{\textnormal{\emph{Liebelei}\pwindex{Liebelei. Schauspiel in drei Akten@\emph{Liebelei. Schauspiel in drei Akten}|pwk} wurde seit
                     11. 1. 1896 auch
                  in Frankfurt am Main\oindex{Frankfurt am Main@\textbf{Frankfurt am Main}, \emph{P.PPLA3}|pwk} am \emph{Städtischen Schauspielhaus}XXXX ORGangabe fehlt gegeben.}}}\label{K_L03169-3}, – wenn die
               Leute was reden, schieb ich es ihm auch zu. Trotzdem sind wir eine Clique. \label{K_L03169-4v}\edtext{Glauben Sie bei Fritz Mauthner\pwindex{Mauthner, Fritz 1849-11-20 – 1923-06-29@\textsc{Mauthner, Fritz} (1849-11-20 – 1923-06-29), \emph{Schriftsteller/Schriftstellerin, Journalist/Journalistin, Philosoph/Philosophin}|pw}\pwindex{Deutsches Theater@\emph{Deutsches Theater}|pwv} wirklich an Lothar\pwindex{Lothar, Rudolf 23.2.1865 – 2.10.1943@\textsc{Lothar, Rudolf} (23.2.1865 – 2.10.1943), \emph{Schriftsteller/Schriftstellerin, Journalist/Journalistin, Theaterdirektor/Theaterdirektorin}|pw}}{\lemma{\textnormal{\emph{Glauben … Lothar}}}\Cendnote{\textnormal{Er fragt nach, ob er wirklich Rudolf Lothar\pwindex{Lothar, Rudolf 23.2.1865 – 2.10.1943@\textsc{Lothar, Rudolf} (23.2.1865 – 2.10.1943), \emph{Schriftsteller/Schriftstellerin, Journalist/Journalistin, Theaterdirektor/Theaterdirektorin}|pwk} für den Stichwortgeber für die Besprechungen von 
                  { }Fritz Mauthner\pwindex{Mauthner, Fritz 1849-11-20 – 1923-06-29@\textsc{Mauthner, Fritz} (1849-11-20 – 1923-06-29), \emph{Schriftsteller/Schriftstellerin, Journalist/Journalistin, Philosoph/Philosophin}|pwk} halte. Von Mauthner\pwindex{Mauthner, Fritz 1849-11-20 – 1923-06-29@\textsc{Mauthner, Fritz} (1849-11-20 – 1923-06-29), \emph{Schriftsteller/Schriftstellerin, Journalist/Journalistin, Philosoph/Philosophin}|pwk} erschienen zwei Texte im
                     \emph{Berliner Tageblatt}\pwindex{Berliner Tageblatt@\emph{Berliner Tageblatt}|pwk}: Fr. M.\pwindex{Mauthner, Fritz 1849-11-20 – 1923-06-29@\textsc{Mauthner, Fritz} (1849-11-20 – 1923-06-29), \emph{Schriftsteller/Schriftstellerin, Journalist/Journalistin, Philosoph/Philosophin}|pwkv} [ = Fritz Mauthner\pwindex{Mauthner, Fritz 1849-11-20 – 1923-06-29@\textsc{Mauthner, Fritz} (1849-11-20 – 1923-06-29), \emph{Schriftsteller/Schriftstellerin, Journalist/Journalistin, Philosoph/Philosophin}|pwk}]: \emph{Deutsches Theater}\pwindex{Deutsches Theater@\emph{Deutsches Theater}|pwk}. In: \emph{Berliner Tageblatt}\pwindex{Berliner Tageblatt@\emph{Berliner Tageblatt}|pwk}, Jg. 25, Nr. 64, 5. 2. 1896, Morgen-Ausgabe, S. 2–3; Fr. M.\pwindex{Mauthner, Fritz 1849-11-20 – 1923-06-29@\textsc{Mauthner, Fritz} (1849-11-20 – 1923-06-29), \emph{Schriftsteller/Schriftstellerin, Journalist/Journalistin, Philosoph/Philosophin}|pwkv} [ = Fritz Mauthner\pwindex{Mauthner, Fritz 1849-11-20 – 1923-06-29@\textsc{Mauthner, Fritz} (1849-11-20 – 1923-06-29), \emph{Schriftsteller/Schriftstellerin, Journalist/Journalistin, Philosoph/Philosophin}|pwk}]: \emph{Der zerbrochene Krug im Deutschen Theater}\pwindex{zerbrochene Krug im Deutschen Theater@\emph{Der zerbrochene Krug im Deutschen Theater}|pwk}. In: \emph{Berliner Tageblatt}\pwindex{Berliner Tageblatt@\emph{Berliner Tageblatt}|pwk}, Jg. 25, Nr. 65, 5. 2. 1896, Abend-Ausgabe, S. 1–2.}}}\label{K_L03169-4}?
               In \label{K_L03169-5v}\edtext{Olmütz\oindex{Olomouc@\textbf{Olomouc}, \emph{P.PPLA}|pw} haben Sie einen großen Erfolg}{\lemma{\textnormal{\emph{Olmütz … Erfolg}}}\Cendnote{\textnormal{Am 30. 1. 1896 hatte am \emph{Königlich-Städtischem Theater zu Olmütz}\orgindex{Maehrisches Theater Olmuetz@Mährisches Theater Olmütz|pwk} die Premiere von \emph{Liebelei}\pwindex{Liebelei. Schauspiel in drei Akten@\emph{Liebelei. Schauspiel in drei Akten}|pwk} stattgefunden.}}}\label{K_L03169-5} gehabt, – sonst sind Sie
               weder in Brünn\oindex{Bruenn@\textbf{Brünn}, \emph{P.PPLA}|pw} noch in Prag\oindex{Prag@\textbf{Prag}, \emph{A.ADM1}|pw} gewesen, das Mährische
                  Tagblatt\pwindex{Maehrisches Tagblatt@\emph{Mährisches Tagblatt}|pw} heb’ ich Ihnen auf, – die \label{K_L03169-6v}\edtext{Kritik\pwindex{Liebelei«. Schauspiel in 3 Acten von Arthur Schnitzler@\emph{»Liebelei«. Schauspiel in 3 Acten von Arthur Schnitzler}|pwv}}{\lemma{\textnormal{\emph{Kritik}}}\Cendnote{\textnormal{[O. V.]: \emph{»Liebelei«. Schauspiel in 3 Acten
                        von Arthur Schnitzler}\pwindex{Liebelei«. Schauspiel in 3 Acten von Arthur Schnitzler@\emph{»Liebelei«. Schauspiel in 3 Acten von Arthur Schnitzler}|pwk}. In: \emph{Mährisches
                        Tagblatt}\pwindex{Maehrisches Tagblatt@\emph{Mährisches Tagblatt}|pwk}, Jg. 17, Nr. 25, 31. 1. 1896,
                     S. 5–6.}}}\label{K_L03169-6} ist köstlich.\pend
           
\pstart
           Hier\oindex{Wien@\textbf{Wien}, \emph{A.ADM2}|pwv} ist ein wunderschönes
               Frühlingswetter, das alle guten Vorsätze hervor{\pb}treibt und gute Laune
               schafft. Zudem habe ich noch \label{K_L03169-7v}\edtext{Frl. M.\pwindex{Salten, Ottilie 07.03.1868 – 22.06.1942@\textsc{Salten, Ottilie} (07.03.1868 – 22.06.1942), \emph{Schauspieler/Schauspielerin}|pw}}{\lemma{\textnormal{\emph{Frl. M.}}}\Cendnote{\textnormal{Ottilie Metzl\pwindex{Salten, Ottilie 07.03.1868 – 22.06.1942@\textsc{Salten, Ottilie} (07.03.1868 – 22.06.1942), \emph{Schauspieler/Schauspielerin}|pwk}, Saltens\pwindex{Salten, Felix 06.09.1869 – 08.10.1945@\textsc{Salten, Felix} (06.09.1869 – 08.10.1945), \emph{Schriftsteller/Schriftstellerin, Journalist/Journalistin, Chefredakteur/Chefredakteurin}|pwk} spätere Ehefrau}}}\label{K_L03169-7} – Neulich, es war Dienstag, erzählt sie mir, sie habe Alles der Frau Mitterwurzer\pwindex{Mitterwurzer, Wilhelmine 27.03.1848 – 03.08.1909@\textsc{Mitterwurzer, Wilhelmine} (27.03.1848 – 03.08.1909), \emph{Schauspieler/Schauspielerin}|pw} gesagt. Diese sei sehr erschrocken
               und habe ihr dringend gerathen, den Verkehr mit mir aufzugeben. Darauf entgegnete
               Frl. M.\pwindex{Salten, Ottilie 07.03.1868 – 22.06.1942@\textsc{Salten, Ottilie} (07.03.1868 – 22.06.1942), \emph{Schauspieler/Schauspielerin}|pw} sie könne das nicht, und Frau Mitterw.\pwindex{Mitterwurzer, Wilhelmine 27.03.1848 – 03.08.1909@\textsc{Mitterwurzer, Wilhelmine} (27.03.1848 – 03.08.1909), \emph{Schauspieler/Schauspielerin}|pw} wünschte dann mich wenigstens kennen
               zu lernen. »\uline{Sie} wird mich gleich durch und durch
               schauen?« Natürlich. Sie will mich auch einladen und wir wollen uns bei ihr oben
               sehen. Tags darauf komme ich in die Redaction\orgindex{Wiener Allgemeine Zeitung@Wiener Allgemeine Zeitung|pwv} und erfahre, dass ich sogleich
               ein \label{K_L03169-8v}\edtext{Feuilleton\pwindex{Wilhelmine Mitterwurzer@\emph{Wilhelmine Mitterwurzer}|pwv}}{\lemma{\textnormal{\emph{Feuilleton}}}\Cendnote{\textnormal{f. s.\pwindex{Salten, Felix 06.09.1869 – 08.10.1945@\textsc{Salten, Felix} (06.09.1869 – 08.10.1945), \emph{Schriftsteller/Schriftstellerin, Journalist/Journalistin, Chefredakteur/Chefredakteurin}|pwk} [ = Felix Salten\pwindex{Salten, Felix 06.09.1869 – 08.10.1945@\textsc{Salten, Felix} (06.09.1869 – 08.10.1945), \emph{Schriftsteller/Schriftstellerin, Journalist/Journalistin, Chefredakteur/Chefredakteurin}|pwk}]: \emph{Wilhelmine Mitterwurzer}\pwindex{Wilhelmine Mitterwurzer@\emph{Wilhelmine Mitterwurzer}|pwk}. In: \emph{Wiener
                        Allgemeine Zeitung}\pwindex{Wiener Allgemeine Zeitung@\emph{Wiener Allgemeine Zeitung}|pwk}, Nr. 5382, 6. 2. 1896, S. 3.}}}\label{K_L03169-8} schreiben muss – über Frau Mitterwurzer\pwindex{Mitterwurzer, Wilhelmine 27.03.1848 – 03.08.1909@\textsc{Mitterwurzer, Wilhelmine} (27.03.1848 – 03.08.1909), \emph{Schauspieler/Schauspielerin}|pw} – das Leben, – \substVorne{}\textsuperscript{s}\substDazwischen{}S\substHinten{}ie wissen schon.\pend
           
\pstart
           Richard\pwindex{Beer-Hofmann, Richard 1866-07-11 – 1945-09-26@\textsc{Beer-Hofmann, Richard} (1866-07-11 – 1945-09-26), \emph{Schriftsteller/Schriftstellerin}|pw} ist sehr lieb, war neu{\pb}lich mit seinem \label{K_L03169-9v}\edtext{Mäderl\pwindex{Beer-Hofmann, Paula 25.02.1879 – 30.10.1939@\textsc{Beer-Hofmann, Paula} (25.02.1879 – 30.10.1939)|pwv}}{\lemma{\textnormal{\emph{Mäderl}}}\Cendnote{\textnormal{Paula Lissy\pwindex{Beer-Hofmann, Paula 25.02.1879 – 30.10.1939@\textsc{Beer-Hofmann, Paula} (25.02.1879 – 30.10.1939)|pwk}, Beer-Hofmanns\pwindex{Beer-Hofmann, Richard 1866-07-11 – 1945-09-26@\textsc{Beer-Hofmann, Richard} (1866-07-11 – 1945-09-26), \emph{Schriftsteller/Schriftstellerin}|pwk} spätere Ehefrau. Die Geringschätzung, die in
                     Saltens\pwindex{Salten, Felix 06.09.1869 – 08.10.1945@\textsc{Salten, Felix} (06.09.1869 – 08.10.1945), \emph{Schriftsteller/Schriftstellerin, Journalist/Journalistin, Chefredakteur/Chefredakteurin}|pwk} Ausdrucksweise spürbar ist,
                  dürfte ein Ausdruck dessen sein, dass sie aus dem Kleinbürgertum stammte.}}}\label{K_L03169-9}
               im Josefstädter Theater\oindex{Theater in der Josefstadt@\textbf{Theater in der Josefstadt}, \emph{Theater (K.THE)}|pw}, und ist stolz darauf. Engländer\pwindex{Altenberg, Peter 09.03.1859 – 08.01.1919@\textsc{Altenberg, Peter} (09.03.1859 – 08.01.1919), \emph{Schriftsteller/Schriftstellerin}|pw} war dabei, und erklärt sie natürlich
               für das Höchste.\pend
           
\pstart
           Sonntag war ich bei der \label{K_L03169-10v}\edtext{Matinée}{\lemma{\textnormal{\emph{Matinée}}}\Cendnote{\textnormal{Salten\pwindex{Salten, Felix 06.09.1869 – 08.10.1945@\textsc{Salten, Felix} (06.09.1869 – 08.10.1945), \emph{Schriftsteller/Schriftstellerin, Journalist/Journalistin, Chefredakteur/Chefredakteurin}|pwk} hatte eine kurze Rezension verfasst: f.\pwindex{Salten, Felix 06.09.1869 – 08.10.1945@\textsc{Salten, Felix} (06.09.1869 – 08.10.1945), \emph{Schriftsteller/Schriftstellerin, Journalist/Journalistin, Chefredakteur/Chefredakteurin}|pwk} [ = Felix Salten\pwindex{Salten, Felix 06.09.1869 – 08.10.1945@\textsc{Salten, Felix} (06.09.1869 – 08.10.1945), \emph{Schriftsteller/Schriftstellerin, Journalist/Journalistin, Chefredakteur/Chefredakteurin}|pwk}]: \emph{Matinée}\pwindex{Matinee@\emph{Matinée}|pwk}. In: \emph{Wiener Allgemeine
                        Zeitung}\pwindex{Wiener Allgemeine Zeitung@\emph{Wiener Allgemeine Zeitung}|pwk}, Nr. 5380, 4. 2. 1896,
                     S. 4.}}}\label{K_L03169-10} im Theater auf der Wien\oindex{Theater an der Wien@\textbf{Theater an der Wien}, \emph{Theater (K.THE)}|pw}
               fortwährend auf der Bühne. Mitterwurzer\pwindex{Mitterwurzer, Friedrich 16.10.1844 – 13.02.1897@\textsc{Mitterwurzer, Friedrich} (16.10.1844 – 13.02.1897), \emph{Schauspieler/Schauspielerin}|pw} rief
               nach Aktschluss\pwindex{Gelegenheitskauf@\emph{Gelegenheitskauf}|pwv} das Frl. M.\pwindex{Salten, Ottilie 07.03.1868 – 22.06.1942@\textsc{Salten, Ottilie} (07.03.1868 – 22.06.1942), \emph{Schauspieler/Schauspielerin}|pw} sie solle mit ihm herauskommen, sich
               verbeugen, – sie wollte nicht, der schrie ihr nach: »Frl. Sandrock\pwindex{Sandrock, Adele 1863-08-19 – 1937-08-30@\textsc{Sandrock, Adele} (1863-08-19 – 1937-08-30), \emph{Schauspieler/Schauspielerin}|pw}{ }Frl. Sandrock\pwindex{Sandrock, Adele 1863-08-19 – 1937-08-30@\textsc{Sandrock, Adele} (1863-08-19 – 1937-08-30), \emph{Schauspieler/Schauspielerin}|pw}!« und als \label{K_L03169-11v}\edtext{sie}{\lemma{\textnormal{\emph{sie}}}\Cendnote{\textnormal{Salten\pwindex{Salten, Felix 06.09.1869 – 08.10.1945@\textsc{Salten, Felix} (06.09.1869 – 08.10.1945), \emph{Schriftsteller/Schriftstellerin, Journalist/Journalistin, Chefredakteur/Chefredakteurin}|pwk} schrieb »Sie«.}}}\label{K_L03169-11}
               ihn darauf aufmerksam machte, wurde er tobsüchtig. Von Frl. S.\pwindex{Sandrock, Adele 1863-08-19 – 1937-08-30@\textsc{Sandrock, Adele} (1863-08-19 – 1937-08-30), \emph{Schauspieler/Schauspielerin}|pw} sind Kleinigkeiten zu berichten{\dotstwo} Ich befand mich ungeheuer wol und daheim auf der Bühne, und hab an Sie gedacht.
                  P. v. Schönthan\pwindex{Schoenthan-Pernwald, Paul von 19.03.1853 – 04.08.1905@\textsc{Schönthan-Pernwald, Paul von} (19.03.1853 – 04.08.1905), \emph{Schriftsteller/Schriftstellerin, Journalist/Journalistin}|pw} ging umher, und erzählte
               den Schauspielern, dass er dieses Stück\pwindex{Gelegenheitskauf@\emph{Gelegenheitskauf}|pwv} mit seinem Herzblut geschrieben, – man überschätzt die Leute noch
               immer. Der Gelegenheits{\pb}kauf\pwindex{Gelegenheitskauf@\emph{Gelegenheitskauf}|pw}{ }ist übrigens im Burgtheater\orgindex{Burgtheater@Burgtheater|pw} und im Lessingtheater\orgindex{Lessing-Theater@Lessing-Theater|pw}
               angenommen.\pend
           
\pstart
           Eben kommt das Repertoire. Sie sind in dieser Woche nicht drauf, was auch erklärlich
                  ist{[}.{]}{ }Dienstag kommt der Dornenweg\pwindex{Dornenweg. Schauspiel in drei Aufzuegen@\emph{Der Dornenweg. Schauspiel in drei Aufzügen}|pw}. Da sind Sie ja bis Abends da, und \label{K_L03169-12v}\edtext{im Theater\oindex{Burgtheater@\textbf{Burgtheater}, \emph{S.THTR}|pwv}}{\lemma{\textnormal{\emph{im Theater}}}\Cendnote{\textnormal{Bei der Premiere von \emph{Der Dornenweg}\pwindex{Dornenweg. Schauspiel in drei Aufzuegen@\emph{Der Dornenweg. Schauspiel in drei Aufzügen}|pwk} im Burgtheater\oindex{Burgtheater@\textbf{Burgtheater}, \emph{S.THTR}|pwk}, siehe A. S.: \emph{Tagebuch}, 11. 2. 1896.}}}\label{K_L03169-12}.\pend
           
\pstart
           Herzlichst Ihr {\\[\baselineskip]}\spacefill\mbox{Salten}\pend
           \leftskip=0em{}\selectlanguage{ngerman}\endnumbering\briefempfaengerindex{Schnitzler, Arthur@\textsc{Schnitzler, Arthur}!zzzSalten, Felix@\emph{von Felix Salten}!1896-02-081@{{[}8. 2. 1896{]}}|)be}\mylabel{L03169h}  \normalsize

\doendnotes{C}
\bigskip
\vfill

\clearpage

\footnotesize

\lohead{\textsc{register}}

% Definiere theindex-Environment komplett neu ohne reledmac
\makeatletter
\renewenvironment{theindex}{%
  \section*{\indexname}%
  \setlength{\parindent}{0pt}%
  \setlength{\parskip}{0pt plus 0.3pt}%
  \let\item\@idxitem
}{%
  \clearpage
}
\makeatother

\IfFileExists{\jobname-pw.ind}{\input{\jobname-pw.ind}}{}

\end{document}

      