%% latex-leseansicht-vorspann.tex
%% Vorspann für die Leseansicht.
%% Lädt die gemeinsame Datei latex-vorspann.tex mit nicht gesetztem Schalter.

\newif\ifkorrekturansicht
\korrekturansichtfalse

\input{../tex-inputs/latex-vorspann}

\begin{center}
            \textcolor{red}{ENTWURF, NICHT FERTIG KORRIGIERT}
                      \end{center}
            
         
         \renewcommand{\erwaehntePersonen}{Personen: Peter Altenberg, Richard Beer-Hofmann, Paula Beer-Hofmann, Julius von Gans-Ludassy, Sigmund Hahn, Rudolf Lothar, Fritz Mauthner, Wilhelmine Mitterwurzer, Friedrich Mitterwurzer, Ottilie Salten, Adele Sandrock, Paul von Schönthan-Pernwald}
         \renewcommand{\erwaehnteInstitutionen}{Institutionen: Burgtheater, Lessing-Theater, Mährisches Theater Olmütz, Neues Wiener Tagblatt}
         \renewcommand{\erwaehnteOrte}{Orte: Berlin, Brünn, Burgtheater, Frankfurt am Main, Olomouc, Prag, Theater an der Wien, Theater in der Josefstadt, Wien}
         \renewcommand{\erwaehnteWerke}{Werke: Berliner Tageblatt, Der Dornenweg, Der zerbrochene Krug im Deutschen Theater, Deutsches Theater, Gelegenheitskauf, Liebelei. Schauspiel in drei Akten, Matinée, Mährisches Tagblatt, Wiener Allgemeine Zeitung, Wilhelmine Mitterwurzer, »Liebelei«. Schauspiel in 3 Acten von Arthur Schnitzler}
               \section[Felix Salten an Arthur Schnitzler, {[}8. 2. 1896{]}]{ Felix Salten an Arthur Schnitzler, {[}8. 2. 1896{]}}\nopagebreak\mylabel{v}\rehead{ }\begin{ledgroupsized}[t]{13cm}\normalsize\beginnumbering \toendnotes[C]{\smallbreak\pagebreak[2]} \Standort{CUL, Schnitzler, B 89, A 1.}
\physDesc{Brief, 1 Blatt, 4 Seiten
\newline{}Handschrift: Bleistift, lateinische Kurrent
\newline{}Schnitzler: mit Bleistift datiert: »8/2 96« \newline{}Ordnung: mit Bleistift von unbekannter Hand nummeriert: »68« }\toendnotes[C]{\smallbreak}\pstart
           \raggedleft{}{\pb}Samstag. \pend
           \pstart
           Lieber Freund, Nachtredakteur beim Neuen Wiener Tagblatt\orgindex{Neues Wiener Tagblatt@Neues Wiener Tagblatt|pw} ist ein Herr Sigmund Hahn\pwindex{Hahn, Sigmund 1844 – 1929-02-21@\textsc{Hahn, Sigmund} (1844 – 1929-02-21), \emph{Redakteur}|pw}, von dem ich aber garnichts weiss.\hspace*{2.5em}\label{K_L03169-1v}\edtext{Berlin\oindex{Berlin@\textbf{Berlin}|pw} hat mir viele Freude}{\lemma{\textnormal{\emph{Berlin … Freude}}}\Cendnote{\textnormal{Schnitzler\pwindex{Schnitzler, Arthur 15.05.1862 – 21.10.1931@\textsc{Schnitzler, Arthur} (15.05.1862 – 21.10.1931), \emph{Schriftsteller, Mediziner}|pwk} hatte am
                     4. 2. 1896 an
                  der Premiere von \emph{Liebelei}\pwindex{Schnitzler, Arthur 15.05.1862 – 21.10.1931@\textsc{Schnitzler, Arthur} (15.05.1862 – 21.10.1931), \emph{Schriftsteller, Mediziner}!Liebelei. Schauspiel in drei Akten1895-10-09@\strich\emph{Liebelei. Schauspiel in drei Akten} {[}1895-10-09{]}|pwk} in Berlin\oindex{Berlin@\textbf{Berlin}|pwk} teilgenommen. Die Inszenierung wurde in der Wien\oindex{Wien@\textbf{Wien}|pwk}er Presse viel besprochen.}}}\label{K_L03169-1h} gemacht,
               das war sehr hübsch und hat hier gut gewirkt. Ludassy\pwindex{Gans-Ludassy, Julius von 13.04.1858 – 30.09.1922@\textsc{Gans-Ludassy, Julius von} (13.04.1858 – 30.09.1922), \emph{Schriftsteller, Journalist, Herausgeber}|pw} verhält mich zu einer Revue über Ihre Berlin\oindex{Berlin@\textbf{Berlin}|pw}er und Frankfurt\oindex{Frankfurt am Main@\textbf{Frankfurt am Main}|pw}er Erfolge, — wenn
               die Leute was reden, schrieb ich es ihm auch zu. Trotzdem sind wir eine Clique.
                  \label{K_L03169-3v}\edtext{Glauben Sie, bei \label{K_L03169-2v}\edtext{Fritz Mauthner\pwindex{Mauthner, Fritz 1849-11-20 – 1923-06-29@\textsc{Mauthner, Fritz} (1849-11-20 – 1923-06-29), \emph{Schriftsteller, Journalist, Philosoph}|pw}\pwindex{Mauthner, Fritz 1849-11-20 – 1923-06-29@\textsc{Mauthner, Fritz} (1849-11-20 – 1923-06-29), \emph{Schriftsteller, Journalist, Philosoph}!Deutsches Theater1896-02-05@\strich\emph{Deutsches Theater} {[}1896-02-05{]}|pwv}}{\lemma{\textnormal{\emph{Fritz Mauthner}}}\Cendnote{\textnormal{Fr. M.\pwindex{Mauthner, Fritz 1849-11-20 – 1923-06-29@\textsc{Mauthner, Fritz} (1849-11-20 – 1923-06-29), \emph{Schriftsteller, Journalist, Philosoph}|pwkv} [=Fritz Mauthner\pwindex{Mauthner, Fritz 1849-11-20 – 1923-06-29@\textsc{Mauthner, Fritz} (1849-11-20 – 1923-06-29), \emph{Schriftsteller, Journalist, Philosoph}|pwk}]: \emph{Deutsches Theater}\pwindex{Mauthner, Fritz 1849-11-20 – 1923-06-29@\textsc{Mauthner, Fritz} (1849-11-20 – 1923-06-29), \emph{Schriftsteller, Journalist, Philosoph}!Deutsches Theater1896-02-05@\strich\emph{Deutsches Theater} {[}1896-02-05{]}|pwk}. In: \emph{Berliner Tageblatt}\pwindex{?? Werk@Nicht ermittelte Verfasserinnen und Verfasser!Berliner Tageblatt1872 – 1939@\emph{Berliner Tageblatt} {[}1872 – 1939{]}|pwk}, Jg. 25, Nr. 64, 5. 2. 1896, Morgen-Ausgabe, S. 2–3; Fritz Mauthner\pwindex{Mauthner, Fritz 1849-11-20 – 1923-06-29@\textsc{Mauthner, Fritz} (1849-11-20 – 1923-06-29), \emph{Schriftsteller, Journalist, Philosoph}|pwk}: \emph{Der zerbrochene Krug im Deutschen Theater}\pwindex{Mauthner, Fritz 1849-11-20 – 1923-06-29@\textsc{Mauthner, Fritz} (1849-11-20 – 1923-06-29), \emph{Schriftsteller, Journalist, Philosoph}!zerbrochene Krug im Deutschen Theater1896-02-05@\strich\emph{Der zerbrochene Krug im Deutschen Theater} {[}1896-02-05{]}|pwk}. In: \emph{Berliner Tageblatt}\pwindex{?? Werk@Nicht ermittelte Verfasserinnen und Verfasser!Berliner Tageblatt1872 – 1939@\emph{Berliner Tageblatt} {[}1872 – 1939{]}|pwk}, Jg. 25, Nr. 65, 5. 2. 1896, Abend-Ausgabe, S. 1–2.}}}\label{K_L03169-2h}
               wirklich an Lothar\pwindex{Lothar, Rudolf 23.2.1865 – 2.10.1943@\textsc{Lothar, Rudolf} (23.2.1865 – 2.10.1943), \emph{Schriftsteller, Journalist, Theaterdirektor}|pw}}{\lemma{\textnormal{\emph{Glauben … Lothar}}}\Cendnote{\textnormal{Also ob Rudolf
                     Lothar\pwindex{Lothar, Rudolf 23.2.1865 – 2.10.1943@\textsc{Lothar, Rudolf} (23.2.1865 – 2.10.1943), \emph{Schriftsteller, Journalist, Theaterdirektor}|pwk}{ }Fritz Mauthner\pwindex{Mauthner, Fritz 1849-11-20 – 1923-06-29@\textsc{Mauthner, Fritz} (1849-11-20 – 1923-06-29), \emph{Schriftsteller, Journalist, Philosoph}|pwk} mit Stichworten versorgt hat.}}}\label{K_L03169-3h}? \pend
           \pstart
           In \label{K_L03169-6v}\edtext{Olmütz\oindex{Olomouc@\textbf{Olomouc}|pw} haben Sie einen großen Erfolg}{\lemma{\textnormal{\emph{Olmütz … Erfolg}}}\Cendnote{\textnormal{Premiere von \emph{Liebelei}\pwindex{Schnitzler, Arthur 15.05.1862 – 21.10.1931@\textsc{Schnitzler, Arthur} (15.05.1862 – 21.10.1931), \emph{Schriftsteller, Mediziner}!Liebelei. Schauspiel in drei Akten1895-10-09@\strich\emph{Liebelei. Schauspiel in drei Akten} {[}1895-10-09{]}|pwk} am
                     30. 1. 1896 am \emph{Königlich-Städtischem Theater zu Olmütz}\orgindex{Maehrisches Theater Olmuetz@Mährisches Theater Olmütz|pwk}}}}\label{K_L03169-6h} gehabt, — sonst sind Sie
               weder in Brünn\oindex{Bruenn@\textbf{Brünn}|pw} noch in Prag\oindex{Prag@\textbf{Prag}|pw} gewesen. \pend
           \pstart
           Das Mährische Tagblatt\pwindex{?? Werk@Nicht ermittelte Verfasserinnen und Verfasser!Maehrisches Tagblatt1880 – 1945@\emph{Mährisches Tagblatt} {[}1880 – 1945{]}|pw} heb’ ich Ihnen auf, — die
                  \label{K_L03169-4v}\edtext{Kritik\pwindex{?? Werk@Nicht ermittelte Verfasserinnen und Verfasser!Liebelei«. Schauspiel in 3 Acten von Arthur Schnitzler1896-01-31@\emph{»Liebelei«. Schauspiel in 3 Acten von Arthur Schnitzler} {[}1896-01-31{]}|pwv}}{\lemma{\textnormal{\emph{Kritik}}}\Cendnote{\textnormal{[O. V.:] \emph{»Liebelei«. Schauspiel in
                        3 Acten von Arthur Schnitzler}\pwindex{?? Werk@Nicht ermittelte Verfasserinnen und Verfasser!Liebelei«. Schauspiel in 3 Acten von Arthur Schnitzler1896-01-31@\emph{»Liebelei«. Schauspiel in 3 Acten von Arthur Schnitzler} {[}1896-01-31{]}|pwk}. In: \emph{Mährisches Tagblatt}\pwindex{?? Werk@Nicht ermittelte Verfasserinnen und Verfasser!Maehrisches Tagblatt1880 – 1945@\emph{Mährisches Tagblatt} {[}1880 – 1945{]}|pwk}, Jg. 17, Nr. 25,
                        31. 1. 1896, S. 5–6.}}}\label{K_L03169-4h} ist köstlich. \pend
           \pstart
           Hier ist ein wunderschönes Frühlingswetter, das alle guten Vorsätze hervor{\pb}treibt und gute Laune
               schafft. Zudem habe ich noch Frl. M.\pwindex{Salten, Ottilie 07.03.1868 – 22.06.1942@\textsc{Salten, Ottilie} (07.03.1868 – 22.06.1942), \emph{Schauspielerin}|pw} – Neulich
               es war Dienstag, erzählt sie mir, sie habe alles der Frau Mitterwurzer\pwindex{Mitterwurzer, Wilhelmine 27.03.1848 – 03.08.1909@\textsc{Mitterwurzer, Wilhelmine} (27.03.1848 – 03.08.1909), \emph{Schauspielerin}|pw} gesagt. Diese sei sehr erschrocken und habe ihr
               dringend gerathen, den Verkehr mit mir aufzugeben. Darauf entgegnete Frl. M\pwindex{Salten, Ottilie 07.03.1868 – 22.06.1942@\textsc{Salten, Ottilie} (07.03.1868 – 22.06.1942), \emph{Schauspielerin}|pw}, sie könne das nicht, und Frau Mitterw.\pwindex{Mitterwurzer, Wilhelmine 27.03.1848 – 03.08.1909@\textsc{Mitterwurzer, Wilhelmine} (27.03.1848 – 03.08.1909), \emph{Schauspielerin}|pw} wünschte dann mich wenigstens kennen
               zu lernen. »\uline{Sie} wird mich gleich durch und
               durchschauen?« Natürlich. Sie will mich auch einladen und wir wollen uns bei ihr oben
               sehen. Tags darauf komme ich in die Redaction und erfahre, dass ich sogleich \uline{ein}{ }\label{K_L03169-8v}\edtext{Feuilleton\pwindex{Wilhelmine Mitterwurzer1896-02-06@\emph{Wilhelmine Mitterwurzer} {[}1896-02-06{]}|pwv}}{\lemma{\textnormal{\emph{Feuilleton}}}\Cendnote{\textnormal{f. s.\pwindex{Salten, Felix 06.09.1869 – 08.10.1945@\textsc{Salten, Felix} (06.09.1869 – 08.10.1945), \emph{Schriftsteller, Journalist}|pwk} [ = Felix Salten\pwindex{Salten, Felix 06.09.1869 – 08.10.1945@\textsc{Salten, Felix} (06.09.1869 – 08.10.1945), \emph{Schriftsteller, Journalist}|pwk}]: \emph{Wilhelmine Mitterwurzer}\pwindex{Wilhelmine Mitterwurzer1896-02-06@\emph{Wilhelmine Mitterwurzer} {[}1896-02-06{]}|pwk}. In: \emph{Wiener Allgemeine Zeitung}\pwindex{?? Werk@Nicht ermittelte Verfasserinnen und Verfasser!Wiener Allgemeine Zeitung1.3.1880 – 11.2.1934@\emph{Wiener Allgemeine Zeitung} {[}1.3.1880 – 11.2.1934{]}|pwk}, Nr. 5.382,
                        6. 2. 1896, S. 3.}}}\label{K_L03169-8h} schreiben muss –
               über Frau Mitterwurzer\pwindex{Mitterwurzer, Wilhelmine 27.03.1848 – 03.08.1909@\textsc{Mitterwurzer, Wilhelmine} (27.03.1848 – 03.08.1909), \emph{Schauspielerin}|pw} – Das Leben, – Sie
               wissen schon.\pend
           \pstart
           Richard\pwindex{Beer-Hofmann, Richard 1866-07-11 – 1945-09-26@\textsc{Beer-Hofmann, Richard} (1866-07-11 – 1945-09-26), \emph{Schriftsteller}|pw} ist sehr lieb, war neu{\pb}lich mit seinem Mädel\pwindex{Beer-Hofmann, Paula 25.02.1879 – 30.10.1939@\textsc{Beer-Hofmann, Paula} (25.02.1879 – 30.10.1939)|pwv} im Josefstädter Theater\oindex{Theater in der Josefstadt@\textbf{Theater in der Josefstadt}|pw}, und ist stolz darauf. Engländer\pwindex{Altenberg, Peter 09.03.1859 – 08.01.1919@\textsc{Altenberg, Peter} (09.03.1859 – 08.01.1919), \emph{Schriftsteller}|pw} war dabei, und erklärt sie natürlich für das
               Höchste.\pend
           \pstart
           Sonntag war ich bei der \label{K_L03169-111v}\edtext{Matinée}{\lemma{\textnormal{\emph{Matinée}}}\Cendnote{\textnormal{Salten\pwindex{Salten, Felix 06.09.1869 – 08.10.1945@\textsc{Salten, Felix} (06.09.1869 – 08.10.1945), \emph{Schriftsteller, Journalist}|pwk} verfasste eine kurze Rezension: f.\pwindex{Salten, Felix 06.09.1869 – 08.10.1945@\textsc{Salten, Felix} (06.09.1869 – 08.10.1945), \emph{Schriftsteller, Journalist}|pwk} [ = Felix Salten\pwindex{Salten, Felix 06.09.1869 – 08.10.1945@\textsc{Salten, Felix} (06.09.1869 – 08.10.1945), \emph{Schriftsteller, Journalist}|pwk}]: \emph{Matinée}\pwindex{Matinee1896-02-04@\emph{Matinée} {[}1896-02-04{]}|pwk}. In: \emph{Wiener Allgemeine Zeitung}\pwindex{?? Werk@Nicht ermittelte Verfasserinnen und Verfasser!Wiener Allgemeine Zeitung1.3.1880 – 11.2.1934@\emph{Wiener Allgemeine Zeitung} {[}1.3.1880 – 11.2.1934{]}|pwk}, Nr. 5.380, 4. 2. 1896, S. 4.}}}\label{K_L03169-111h} im Theater auf der
                  Wien\oindex{Theater an der Wien@\textbf{Theater an der Wien}|pw} fortwährend auf der Bühne. Mitterwurzer\pwindex{Mitterwurzer, Friedrich 16.10.1844 – 13.02.1897@\textsc{Mitterwurzer, Friedrich} (16.10.1844 – 13.02.1897), \emph{Schauspieler}|pw} rief nach Aktschluss das Frl. M.\pwindex{Salten, Ottilie 07.03.1868 – 22.06.1942@\textsc{Salten, Ottilie} (07.03.1868 – 22.06.1942), \emph{Schauspielerin}|pw} sie solle mit ihm herauskommen, sich verbeugen, – sie wollte nicht, der
               schrie ihr nach: »Frl. Sandrock\pwindex{Sandrock, Adele 1863-08-19 – 1937-08-30@\textsc{Sandrock, Adele} (1863-08-19 – 1937-08-30), \emph{Schauspielerin}|pw}{ }Frl. Sandrock\pwindex{Sandrock, Adele 1863-08-19 – 1937-08-30@\textsc{Sandrock, Adele} (1863-08-19 – 1937-08-30), \emph{Schauspielerin}|pw}!« und als sie ihn darauf aufmerksam machte, wurde er tobsüchtig.
               Von Frl. S.\pwindex{Sandrock, Adele 1863-08-19 – 1937-08-30@\textsc{Sandrock, Adele} (1863-08-19 – 1937-08-30), \emph{Schauspielerin}|pw} sind Kleinigkeiten zu berichten{\dotstwo} Ich befand mich ungeheuer wol und daheim auf der Bühne,
               und hab an Sie gedacht. P. v. Schönthan\pwindex{Schoenthan-Pernwald, Paul von 19.03.1853 – 04.08.1905@\textsc{Schönthan-Pernwald, Paul von} (19.03.1853 – 04.08.1905), \emph{Schriftsteller, Journalist, Schriftsteller}|pw} ging
               umher, und erzählte den Schauspielern, dass er dieses Stück\pwindex{Schoenthan-Pernwald, Paul von 19.03.1853 – 04.08.1905@\textsc{Schönthan-Pernwald, Paul von} (19.03.1853 – 04.08.1905), \emph{Schriftsteller, Journalist, Schriftsteller}!Gelegenheitskauf1896-02-03@\strich\emph{Gelegenheitskauf} {[}1896-02-03{]}|pwv} mit seinem \uline{Herzblut}
               geschrieben, – man überschätzt die Leute noch immer. Der Gelegenheits{\pb}kauf\pwindex{Schoenthan-Pernwald, Paul von 19.03.1853 – 04.08.1905@\textsc{Schönthan-Pernwald, Paul von} (19.03.1853 – 04.08.1905), \emph{Schriftsteller, Journalist, Schriftsteller}!Gelegenheitskauf1896-02-03@\strich\emph{Gelegenheitskauf} {[}1896-02-03{]}|pw}
               ist übrigens im Burgtheater\orgindex{Burgtheater@Burgtheater|pw} und im Lessingtheater\orgindex{Lessing-Theater@Lessing-Theater|pw} angenommen. \pend
           \pstart
           Eben kommt das Repertoire. Sie sind in dieser Woche nicht drauf, was auch erklärlich
               ist. Dienstag kommt der Dornenweg\pwindex{\textcolor{red}{\textsuperscript{XXXX1 indx}}!Dornenweg1898@\strich\emph{Der Dornenweg} {[}1898{]}|pw}. Da sind Sie
               ja bis abends da, und im Theater\oindex{Burgtheater@\textbf{Burgtheater}|pwv}. \pend
           \pstart
           Herzlichst Ihr {\\[\baselineskip]}\spacefill\mbox{Salten}\pend
           \leftskip=0em{}
         
         \endnumbering\mylabel{h}\end{ledgroupsized}\begin{anhang}\end{anhang}\newcommand{\dateiname}{L03169}\newcommand{\titel}{Felix Salten an Arthur Schnitzler, [8. 2. 1896]}\newcommand{\editorInnen}{Martin Anton Müller und Laura Untner}%% latex-leseansicht-abspann.tex
%% Abspann für die Leseansicht.
%% Der Schalter \ifkorrekturansicht ist bereits durch den Vorspann gesetzt.

%% latex-abspann.tex
%% Gemeinsamer Abspann für Korrekturansicht und Leseansicht.
%% Setzt den Schalter \ifkorrekturansicht voraus (gesetzt in den
%% einbindenden Dateien latex-korrekturansicht-abspann.tex bzw.
%% latex-leseansicht-abspann.tex).
%% ---------------------------------------------------------------

\normalsize

% Das esempio-Environment wird nur in der Leseansicht benötigt
\ifkorrekturansicht\else
\newenvironment{esempio}[3]%
{
    \vspace{1.5ex}
    \rlap{\underline{#1}}
    \par
    \setlength{\parindent}{0cm}
    \nopagebreak
    \leftskip=#2cm
    \rightskip=#3cm
}
{
    \par
}
\fi

\doendnotes{C}
\bigskip
\vfill

\clearpage

\footnotesize

\ifkorrekturansicht
  \lohead{\textsc{register}}
\fi

% theindex-Environment neu definieren ohne reledmac
\makeatletter
\renewenvironment{theindex}{%
  \ifkorrekturansicht
    \section*{\indexname}%
  \else
    \subsubsection*{Index der erwähnten Entitäten}%
  \fi
  \setlength{\parindent}{0pt}%
  \setlength{\parskip}{0pt plus 0.3pt}%
  \let\item\@idxitem
}{%
  \ifkorrekturansicht\clearpage\fi
}
\makeatother

\IfFileExists{\jobname-pw.ind}{\input{\jobname-pw.ind}}{}

% Quellenangabe nur in der Leseansicht
\ifkorrekturansicht\else
% Fallback-Definitionen, falls die .tex-Datei \titel etc. nicht gesetzt hat
\providecommand{\titel}{}
\providecommand{\editorInnen}{}
\providecommand{\dateiname}{\jobname}

\vspace{3cm}

\vfill

\footnotesize
\textsc{Quelle}: \titel. Herausgegeben von {\editorInnen}. In: \emph{Arthur Schnitzler: Briefwechsel mit Autorinnen und Autoren}.
 Digitale Edition, https://schnitzler-briefe.acdh.oeaw.ac.at/{\dateiname}.html (Stand \today)
\fi

\end{document}


      