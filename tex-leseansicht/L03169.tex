%% latex-leseansicht-vorspann.tex
%% Vorspann für die Leseansicht.
%% Lädt die gemeinsame Datei latex-vorspann.tex mit nicht gesetztem Schalter.

\newif\ifkorrekturansicht
\korrekturansichtfalse

\input{../tex-inputs/latex-vorspann}


         
         \renewcommand{\erwaehntePersonen}{Personen: Peter Altenberg, Richard Beer-Hofmann, Paula Beer-Hofmann, Julius von Gans-Ludassy, Sigmund Hahn, Rudolf Lothar, Fritz Mauthner, Wilhelmine Mitterwurzer, Friedrich Mitterwurzer, Felix Salten, Ottilie Salten, Adele Sandrock, Paul von Schönthan-Pernwald}
         \renewcommand{\erwaehnteInstitutionen}{Institutionen: Burgtheater, Deutsches Theater Berlin, Frankfurter Städtisches Schauspielhaus, Lessing-Theater, Mährisches Theater Olmütz, Neues Wiener Tagblatt, Wiener Allgemeine Zeitung}
         \renewcommand{\erwaehnteOrte}{Orte: Berlin, Brünn, Burgtheater, Frankfurt am Main, Olomouc, Prag, Theater an der Wien, Theater in der Josefstadt, Wien}
         \renewcommand{\erwaehnteWerke}{Werke: Berliner Tageblatt, Der Dornenweg. Schauspiel in drei Aufzügen, Der zerbrochene Krug im Deutschen Theater, Der zerbrochene Krug. Ein Lustspiel in drei Aufzügen, Deutsches Theater, Gelegenheitskauf, Liebelei. Schauspiel in drei Akten, Matinée, Mährisches Tagblatt, Neues Wiener Tagblatt, Tagebuch, Theater und Kunst [Liebelei am Deutschen Theater], Wiener Allgemeine Zeitung, Wilhelmine Mitterwurzer, »Liebelei«. Schauspiel in 3 Acten von Arthur Schnitzler}
               \section[ Felix Salten an Arthur Schnitzler, {[}8. 2. 1896{]}]{ Felix Salten an Arthur Schnitzler, {[}8. 2. 1896{]}}\nopagebreak\mylabel{v}\rehead{ }\begin{ledgroupsized}[t]{13cm}\normalsize\beginnumbering\briefempfaengerindex{Schnitzler, Arthur@\textsc{Schnitzler, Arthur}!zzzSalten, Felix@\emph{von Felix Salten}!1896-02-081@{{[}8. 2. 1896{]}}|(be} \toendnotes[C]{\smallbreak\pagebreak[2]} \Standort{CUL, Schnitzler, B 89, A 1.}
\physDesc{Brief, 1 Blatt, 4 Seiten, 2241 Zeichen
\newline{}Handschrift: Bleistift, lateinische Kurrent
\newline{}Schnitzler: mit Bleistift datiert: »8/2 96« 
\newline{}Ordnung: mit Bleistift von unbekannter Hand nummeriert: »68.« }\toendnotes[C]{\smallbreak}\pstart
           \raggedleft{}{\pb}Samstag.\pend
           \pstart
           Lieber Freund, Nachtredacteur beim Neuen Wiener Tagblatt\orgindex{Neues Wiener Tagblatt@Neues Wiener Tagblatt|pw} ist ein Herr \label{K_L03169-1v}\edtext{Sigmund Hahn\pwindex{Hahn, Sigmund 1844 – 1929-02-21@\textsc{Hahn, Sigmund} (1844 – 1929-02-21), \emph{Redakteur}|pw}}{\lemma{\textnormal{\emph{Sigmund Hahn}}}\Cendnote{\textnormal{Schnitzler\pwindex{Schnitzler, Arthur 15.05.1862 – 21.10.1931@\textsc{Schnitzler, Arthur} (15.05.1862 – 21.10.1931), \emph{Schriftsteller, Mediziner}|pwk} hielt sich in Berlin\oindex{Berlin@\textbf{Berlin}|pwk} auf, wo am 4. 2. 1896 am \emph{Deutschen Theater}\orgindex{Deutsches Theater Berlin@Deutsches Theater Berlin|pwk} die gemeinsame Premiere von \emph{Liebelei}\pwindex{Schnitzler, Arthur 15.05.1862 – 21.10.1931@\textsc{Schnitzler, Arthur} (15.05.1862 – 21.10.1931), \emph{Schriftsteller, Mediziner}!Liebelei. Schauspiel in drei Akten1895-10-09@\strich\emph{Liebelei. Schauspiel in drei Akten} {[}1895-10-09{]}|pwk} und \emph{Der
                     zerbrochene Krug}\pwindex{\textcolor{red}{\textsuperscript{XXXX1 indx}}!zerbrochene Krug. Ein Lustspiel in drei Aufzuegen1808@\strich\emph{Der zerbrochene Krug. Ein Lustspiel in drei Aufzügen} {[}1808{]}|pwk} stattfand. Schnitzler\pwindex{Schnitzler, Arthur 15.05.1862 – 21.10.1931@\textsc{Schnitzler, Arthur} (15.05.1862 – 21.10.1931), \emph{Schriftsteller, Mediziner}|pwk} erwähnt sowohl das Studium der Nachtkritiken (5. 2. 1896) wie auch
                  die Feuilletons (6. 2. 1896) in seinem \emph{Tagebuch}\pwindex{\textcolor{red}{\textsuperscript{XXXX1 indx}}!Tagebuch1981 – 2000@\strich\emph{Tagebuch} {[}Hrsg., 1981 – 2000{]}|pwk}.
                  Hier dürfte er der Notiz\pwindex{?? Werk@Nicht ermittelte Verfasserinnen und Verfasser!Theater und Kunst [Liebelei am Deutschen Theater]1896-02-05@\emph{Theater und Kunst [Liebelei am Deutschen Theater]} {[}1896-02-05{]}|pwkv}
                  im Abendblatt des \emph{Neuen Wiener Tagblatts}\pwindex{?? Werk@Nicht ermittelte Verfasserinnen und Verfasser!Neues Wiener Tagblatt1867 – 1945@\emph{Neues Wiener Tagblatt} {[}1867 – 1945{]}|pwk}:
                     [O. V.]: \emph{Theater und Kunst}\pwindex{?? Werk@Nicht ermittelte Verfasserinnen und Verfasser!Theater und Kunst [Liebelei am Deutschen Theater]1896-02-05@\emph{Theater und Kunst [Liebelei am Deutschen Theater]} {[}1896-02-05{]}|pwk}. In: \emph{Neues Wiener Abendblatt. Abend-Ausgabe des
                        »Neuen Wiener Tagblatt«}\pwindex{?? Werk@Nicht ermittelte Verfasserinnen und Verfasser!Neues Wiener Tagblatt1867 – 1945@\emph{Neues Wiener Tagblatt} {[}1867 – 1945{]}|pwk}, Jg. 30, Nr. 35, 5. 2 1896, S. 3 nachgeforscht haben.}}}\label{K_L03169-1h}, von dem ich aber
               garnichts weiss. \label{K_L03169-2v}\edtext{Berlin\oindex{Berlin@\textbf{Berlin}|pw} hat mir viele Freude}{\lemma{\textnormal{\emph{Berlin … Freude}}}\Cendnote{\textnormal{Salten\pwindex{Salten, Felix 06.09.1869 – 08.10.1945@\textsc{Salten, Felix} (06.09.1869 – 08.10.1945), \emph{Schriftsteller, Journalist, Chefredakteur}|pwk} zeigt sich erfreut darüber, dass die
                     Berlin\oindex{Berlin@\textbf{Berlin}|pwk}er Inszenierung von \emph{Liebelei}\pwindex{Schnitzler, Arthur 15.05.1862 – 21.10.1931@\textsc{Schnitzler, Arthur} (15.05.1862 – 21.10.1931), \emph{Schriftsteller, Mediziner}!Liebelei. Schauspiel in drei Akten1895-10-09@\strich\emph{Liebelei. Schauspiel in drei Akten} {[}1895-10-09{]}|pwk} in der (Wien\oindex{Wien@\textbf{Wien}|pwk}er) Presse viel und positiv besprochen wurde.}}}\label{K_L03169-2h} gemacht,
                  \textcolor{gray}{–} das war sehr hübsch und hat hier\oindex{Wien@\textbf{Wien}|pwv} gut gewirkt. Ludaßy\pwindex{Gans-Ludassy, Julius von 13.04.1858 – 30.09.1922@\textsc{Gans-Ludassy, Julius von} (13.04.1858 – 30.09.1922), \emph{Schriftsteller, Journalist, Herausgeber}|pw} verhält mich zu einer Revue über Ihre Berlin\oindex{Berlin@\textbf{Berlin}|pw}er u. \label{K_L03169-3v}\edtext{Frankfurt\oindex{Frankfurt am Main@\textbf{Frankfurt am Main}|pw}er Erfolge}{\lemma{\textnormal{\emph{Frankfurter Erfolge}}}\Cendnote{\textnormal{\emph{Liebelei}\pwindex{Schnitzler, Arthur 15.05.1862 – 21.10.1931@\textsc{Schnitzler, Arthur} (15.05.1862 – 21.10.1931), \emph{Schriftsteller, Mediziner}!Liebelei. Schauspiel in drei Akten1895-10-09@\strich\emph{Liebelei. Schauspiel in drei Akten} {[}1895-10-09{]}|pwk} wurde seit
                     11. 1. 1896 auch
                  in Frankfurt am Main\oindex{Frankfurt am Main@\textbf{Frankfurt am Main}|pwk} am \emph{Städtischen Schauspielhaus}\orgindex{Frankfurter Staedtisches Schauspielhaus@Frankfurter Städtisches Schauspielhaus|pwk} gegeben.}}}\label{K_L03169-3h}, – wenn die
               Leute was reden, schieb ich es ihm auch zu. Trotzdem sind wir eine Clique. \label{K_L03169-4v}\edtext{Glauben Sie bei Fritz Mauthner\pwindex{Mauthner, Fritz 1849-11-20 – 1923-06-29@\textsc{Mauthner, Fritz} (1849-11-20 – 1923-06-29), \emph{Schriftsteller, Journalist, Philosoph}|pw}\pwindex{Mauthner, Fritz 1849-11-20 – 1923-06-29@\textsc{Mauthner, Fritz} (1849-11-20 – 1923-06-29), \emph{Schriftsteller, Journalist, Philosoph}!Deutsches Theater1896-02-05@\strich\emph{Deutsches Theater} {[}1896-02-05{]}|pwv} wirklich an Lothar\pwindex{Lothar, Rudolf 23.2.1865 – 2.10.1943@\textsc{Lothar, Rudolf} (23.2.1865 – 2.10.1943), \emph{Schriftsteller, Journalist, Theaterdirektor}|pw}}{\lemma{\textnormal{\emph{Glauben … Lothar}}}\Cendnote{\textnormal{Er fragt nach, ob er wirklich Rudolf Lothar\pwindex{Lothar, Rudolf 23.2.1865 – 2.10.1943@\textsc{Lothar, Rudolf} (23.2.1865 – 2.10.1943), \emph{Schriftsteller, Journalist, Theaterdirektor}|pwk} für den Stichwortgeber für die Besprechungen von 
                  { }Fritz Mauthner\pwindex{Mauthner, Fritz 1849-11-20 – 1923-06-29@\textsc{Mauthner, Fritz} (1849-11-20 – 1923-06-29), \emph{Schriftsteller, Journalist, Philosoph}|pwk} halte. Von Mauthner\pwindex{Mauthner, Fritz 1849-11-20 – 1923-06-29@\textsc{Mauthner, Fritz} (1849-11-20 – 1923-06-29), \emph{Schriftsteller, Journalist, Philosoph}|pwk} erschienen zwei Texte im
                     \emph{Berliner Tageblatt}\pwindex{?? Werk@Nicht ermittelte Verfasserinnen und Verfasser!Berliner Tageblatt1872 – 1939@\emph{Berliner Tageblatt} {[}1872 – 1939{]}|pwk}: Fr. M.\pwindex{Mauthner, Fritz 1849-11-20 – 1923-06-29@\textsc{Mauthner, Fritz} (1849-11-20 – 1923-06-29), \emph{Schriftsteller, Journalist, Philosoph}|pwkv} [ = Fritz Mauthner\pwindex{Mauthner, Fritz 1849-11-20 – 1923-06-29@\textsc{Mauthner, Fritz} (1849-11-20 – 1923-06-29), \emph{Schriftsteller, Journalist, Philosoph}|pwk}]: \emph{Deutsches Theater}\pwindex{Mauthner, Fritz 1849-11-20 – 1923-06-29@\textsc{Mauthner, Fritz} (1849-11-20 – 1923-06-29), \emph{Schriftsteller, Journalist, Philosoph}!Deutsches Theater1896-02-05@\strich\emph{Deutsches Theater} {[}1896-02-05{]}|pwk}. In: \emph{Berliner Tageblatt}\pwindex{?? Werk@Nicht ermittelte Verfasserinnen und Verfasser!Berliner Tageblatt1872 – 1939@\emph{Berliner Tageblatt} {[}1872 – 1939{]}|pwk}, Jg. 25, Nr. 64, 5. 2. 1896, Morgen-Ausgabe, S. 2–3; Fr. M.\pwindex{Mauthner, Fritz 1849-11-20 – 1923-06-29@\textsc{Mauthner, Fritz} (1849-11-20 – 1923-06-29), \emph{Schriftsteller, Journalist, Philosoph}|pwkv} [ = Fritz Mauthner\pwindex{Mauthner, Fritz 1849-11-20 – 1923-06-29@\textsc{Mauthner, Fritz} (1849-11-20 – 1923-06-29), \emph{Schriftsteller, Journalist, Philosoph}|pwk}]: \emph{Der zerbrochene Krug im Deutschen Theater}\pwindex{Mauthner, Fritz 1849-11-20 – 1923-06-29@\textsc{Mauthner, Fritz} (1849-11-20 – 1923-06-29), \emph{Schriftsteller, Journalist, Philosoph}!zerbrochene Krug im Deutschen Theater1896-02-05@\strich\emph{Der zerbrochene Krug im Deutschen Theater} {[}1896-02-05{]}|pwk}. In: \emph{Berliner Tageblatt}\pwindex{?? Werk@Nicht ermittelte Verfasserinnen und Verfasser!Berliner Tageblatt1872 – 1939@\emph{Berliner Tageblatt} {[}1872 – 1939{]}|pwk}, Jg. 25, Nr. 65, 5. 2. 1896, Abend-Ausgabe, S. 1–2.}}}\label{K_L03169-4h}?
               In \label{K_L03169-5v}\edtext{Olmütz\oindex{Olomouc@\textbf{Olomouc}|pw} haben Sie einen großen Erfolg}{\lemma{\textnormal{\emph{Olmütz … Erfolg}}}\Cendnote{\textnormal{Am 30. 1. 1896 hatte am \emph{Königlich-Städtischem Theater zu Olmütz}\orgindex{Maehrisches Theater Olmuetz@Mährisches Theater Olmütz|pwk} die Premiere von \emph{Liebelei}\pwindex{Schnitzler, Arthur 15.05.1862 – 21.10.1931@\textsc{Schnitzler, Arthur} (15.05.1862 – 21.10.1931), \emph{Schriftsteller, Mediziner}!Liebelei. Schauspiel in drei Akten1895-10-09@\strich\emph{Liebelei. Schauspiel in drei Akten} {[}1895-10-09{]}|pwk} stattgefunden.}}}\label{K_L03169-5h} gehabt, – sonst sind Sie
               weder in Brünn\oindex{Bruenn@\textbf{Brünn}|pw} noch in Prag\oindex{Prag@\textbf{Prag}|pw} gewesen, das Mährische
                  Tagblatt\pwindex{?? Werk@Nicht ermittelte Verfasserinnen und Verfasser!Maehrisches Tagblatt1880 – 1945@\emph{Mährisches Tagblatt} {[}1880 – 1945{]}|pw} heb’ ich Ihnen auf, – die \label{K_L03169-6v}\edtext{Kritik\pwindex{?? Werk@Nicht ermittelte Verfasserinnen und Verfasser!Liebelei«. Schauspiel in 3 Acten von Arthur Schnitzler1896-01-31@\emph{»Liebelei«. Schauspiel in 3 Acten von Arthur Schnitzler} {[}1896-01-31{]}|pwv}}{\lemma{\textnormal{\emph{Kritik}}}\Cendnote{\textnormal{[O. V.]: \emph{»Liebelei«. Schauspiel in 3 Acten
                        von Arthur Schnitzler}\pwindex{?? Werk@Nicht ermittelte Verfasserinnen und Verfasser!Liebelei«. Schauspiel in 3 Acten von Arthur Schnitzler1896-01-31@\emph{»Liebelei«. Schauspiel in 3 Acten von Arthur Schnitzler} {[}1896-01-31{]}|pwk}. In: \emph{Mährisches
                        Tagblatt}\pwindex{?? Werk@Nicht ermittelte Verfasserinnen und Verfasser!Maehrisches Tagblatt1880 – 1945@\emph{Mährisches Tagblatt} {[}1880 – 1945{]}|pwk}, Jg. 17, Nr. 25, 31. 1. 1896,
                     S. 5–6.}}}\label{K_L03169-6h} ist köstlich.\pend
           \pstart
           Hier\oindex{Wien@\textbf{Wien}|pwv} ist ein wunderschönes
               Frühlingswetter, das alle guten Vorsätze hervor{\pb}treibt und gute Laune
               schafft. Zudem habe ich noch \label{K_L03169-7v}\edtext{Frl. M.\pwindex{Salten, Ottilie 07.03.1868 – 22.06.1942@\textsc{Salten, Ottilie} (07.03.1868 – 22.06.1942), \emph{Schauspielerin}|pw}}{\lemma{\textnormal{\emph{Frl. M.}}}\Cendnote{\textnormal{Ottilie Metzl\pwindex{Salten, Ottilie 07.03.1868 – 22.06.1942@\textsc{Salten, Ottilie} (07.03.1868 – 22.06.1942), \emph{Schauspielerin}|pwk}, Saltens\pwindex{Salten, Felix 06.09.1869 – 08.10.1945@\textsc{Salten, Felix} (06.09.1869 – 08.10.1945), \emph{Schriftsteller, Journalist, Chefredakteur}|pwk} spätere Ehefrau}}}\label{K_L03169-7h} – Neulich, es war Dienstag, erzählt sie mir, sie habe Alles der Frau Mitterwurzer\pwindex{Mitterwurzer, Wilhelmine 27.03.1848 – 03.08.1909@\textsc{Mitterwurzer, Wilhelmine} (27.03.1848 – 03.08.1909), \emph{Schauspielerin}|pw} gesagt. Diese sei sehr erschrocken
               und habe ihr dringend gerathen, den Verkehr mit mir aufzugeben. Darauf entgegnete
               Frl. M.\pwindex{Salten, Ottilie 07.03.1868 – 22.06.1942@\textsc{Salten, Ottilie} (07.03.1868 – 22.06.1942), \emph{Schauspielerin}|pw} sie könne das nicht, und Frau Mitterw.\pwindex{Mitterwurzer, Wilhelmine 27.03.1848 – 03.08.1909@\textsc{Mitterwurzer, Wilhelmine} (27.03.1848 – 03.08.1909), \emph{Schauspielerin}|pw} wünschte dann mich wenigstens kennen
               zu lernen. »\uline{Sie} wird mich gleich durch und durch
               schauen?« Natürlich. Sie will mich auch einladen und wir wollen uns bei ihr oben
               sehen. Tags darauf komme ich in die Redaction\orgindex{Wiener Allgemeine Zeitung@Wiener Allgemeine Zeitung|pwv} und erfahre, dass ich sogleich
               ein \label{K_L03169-8v}\edtext{Feuilleton\pwindex{Wilhelmine Mitterwurzer1896-02-06@\emph{Wilhelmine Mitterwurzer} {[}1896-02-06{]}|pwv}}{\lemma{\textnormal{\emph{Feuilleton}}}\Cendnote{\textnormal{f. s.\pwindex{Salten, Felix 06.09.1869 – 08.10.1945@\textsc{Salten, Felix} (06.09.1869 – 08.10.1945), \emph{Schriftsteller, Journalist, Chefredakteur}|pwk} [ = Felix Salten\pwindex{Salten, Felix 06.09.1869 – 08.10.1945@\textsc{Salten, Felix} (06.09.1869 – 08.10.1945), \emph{Schriftsteller, Journalist, Chefredakteur}|pwk}]: \emph{Wilhelmine Mitterwurzer}\pwindex{Wilhelmine Mitterwurzer1896-02-06@\emph{Wilhelmine Mitterwurzer} {[}1896-02-06{]}|pwk}. In: \emph{Wiener
                        Allgemeine Zeitung}\pwindex{Wiener Allgemeine Zeitung1.3.1880 – 11.2.1934@\emph{Wiener Allgemeine Zeitung} {[}1.3.1880 – 11.2.1934{]}|pwk}, Nr. 5382, 6. 2. 1896, S. 3.}}}\label{K_L03169-8h} schreiben muss – über Frau Mitterwurzer\pwindex{Mitterwurzer, Wilhelmine 27.03.1848 – 03.08.1909@\textsc{Mitterwurzer, Wilhelmine} (27.03.1848 – 03.08.1909), \emph{Schauspielerin}|pw} – das Leben, – \substVorne{}\textsuperscript{s}\substDazwischen{}S\substHinten{}ie wissen schon.\pend
           \pstart
           Richard\pwindex{Beer-Hofmann, Richard 1866-07-11 – 1945-09-26@\textsc{Beer-Hofmann, Richard} (1866-07-11 – 1945-09-26), \emph{Schriftsteller}|pw} ist sehr lieb, war neu{\pb}lich mit seinem \label{K_L03169-9v}\edtext{Mäderl\pwindex{Beer-Hofmann, Paula 25.02.1879 – 30.10.1939@\textsc{Beer-Hofmann, Paula} (25.02.1879 – 30.10.1939)|pwv}}{\lemma{\textnormal{\emph{Mäderl}}}\Cendnote{\textnormal{Paula Lissy\pwindex{Beer-Hofmann, Paula 25.02.1879 – 30.10.1939@\textsc{Beer-Hofmann, Paula} (25.02.1879 – 30.10.1939)|pwk}, Beer-Hofmanns\pwindex{Beer-Hofmann, Richard 1866-07-11 – 1945-09-26@\textsc{Beer-Hofmann, Richard} (1866-07-11 – 1945-09-26), \emph{Schriftsteller}|pwk} spätere Ehefrau. Die Geringschätzung, die in
                     Saltens\pwindex{Salten, Felix 06.09.1869 – 08.10.1945@\textsc{Salten, Felix} (06.09.1869 – 08.10.1945), \emph{Schriftsteller, Journalist, Chefredakteur}|pwk} Ausdrucksweise spürbar ist,
                  dürfte ein Ausdruck dessen sein, dass sie aus dem Kleinbürgertum stammte.}}}\label{K_L03169-9h}
               im Josefstädter Theater\oindex{Theater in der Josefstadt@\textbf{Theater in der Josefstadt}|pw}, und ist stolz darauf. Engländer\pwindex{Altenberg, Peter 09.03.1859 – 08.01.1919@\textsc{Altenberg, Peter} (09.03.1859 – 08.01.1919), \emph{Schriftsteller}|pw} war dabei, und erklärt sie natürlich
               für das Höchste.\pend
           \pstart
           Sonntag war ich bei der \label{K_L03169-10v}\edtext{Matinée}{\lemma{\textnormal{\emph{Matinée}}}\Cendnote{\textnormal{Salten\pwindex{Salten, Felix 06.09.1869 – 08.10.1945@\textsc{Salten, Felix} (06.09.1869 – 08.10.1945), \emph{Schriftsteller, Journalist, Chefredakteur}|pwk} hatte eine kurze Rezension verfasst: f.\pwindex{Salten, Felix 06.09.1869 – 08.10.1945@\textsc{Salten, Felix} (06.09.1869 – 08.10.1945), \emph{Schriftsteller, Journalist, Chefredakteur}|pwk} [ = Felix Salten\pwindex{Salten, Felix 06.09.1869 – 08.10.1945@\textsc{Salten, Felix} (06.09.1869 – 08.10.1945), \emph{Schriftsteller, Journalist, Chefredakteur}|pwk}]: \emph{Matinée}\pwindex{Matinee1896-02-04@\emph{Matinée} {[}1896-02-04{]}|pwk}. In: \emph{Wiener Allgemeine
                        Zeitung}\pwindex{Wiener Allgemeine Zeitung1.3.1880 – 11.2.1934@\emph{Wiener Allgemeine Zeitung} {[}1.3.1880 – 11.2.1934{]}|pwk}, Nr. 5380, 4. 2. 1896,
                     S. 4.}}}\label{K_L03169-10h} im Theater auf der Wien\oindex{Theater an der Wien@\textbf{Theater an der Wien}|pw}
               fortwährend auf der Bühne. Mitterwurzer\pwindex{Mitterwurzer, Friedrich 16.10.1844 – 13.02.1897@\textsc{Mitterwurzer, Friedrich} (16.10.1844 – 13.02.1897), \emph{Schauspieler}|pw} rief
               nach Aktschluss\pwindex{Schoenthan-Pernwald, Paul von 19.03.1853 – 04.08.1905@\textsc{Schönthan-Pernwald, Paul von} (19.03.1853 – 04.08.1905), \emph{Schriftsteller, Journalist}!Gelegenheitskauf1896-02-03@\strich\emph{Gelegenheitskauf} {[}1896-02-03{]}|pwv} das Frl. M.\pwindex{Salten, Ottilie 07.03.1868 – 22.06.1942@\textsc{Salten, Ottilie} (07.03.1868 – 22.06.1942), \emph{Schauspielerin}|pw} sie solle mit ihm herauskommen, sich
               verbeugen, – sie wollte nicht, der schrie ihr nach: »Frl. Sandrock\pwindex{Sandrock, Adele 1863-08-19 – 1937-08-30@\textsc{Sandrock, Adele} (1863-08-19 – 1937-08-30), \emph{Schauspielerin}|pw}{ }Frl. Sandrock\pwindex{Sandrock, Adele 1863-08-19 – 1937-08-30@\textsc{Sandrock, Adele} (1863-08-19 – 1937-08-30), \emph{Schauspielerin}|pw}!« und als \label{K_L03169-11v}\edtext{sie}{\lemma{\textnormal{\emph{sie}}}\Cendnote{\textnormal{Salten\pwindex{Salten, Felix 06.09.1869 – 08.10.1945@\textsc{Salten, Felix} (06.09.1869 – 08.10.1945), \emph{Schriftsteller, Journalist, Chefredakteur}|pwk} schrieb »Sie«.}}}\label{K_L03169-11h}
               ihn darauf aufmerksam machte, wurde er tobsüchtig. Von Frl. S.\pwindex{Sandrock, Adele 1863-08-19 – 1937-08-30@\textsc{Sandrock, Adele} (1863-08-19 – 1937-08-30), \emph{Schauspielerin}|pw} sind Kleinigkeiten zu berichten{\dotstwo} Ich befand mich ungeheuer wol und daheim auf der Bühne, und hab an Sie gedacht.
                  P. v. Schönthan\pwindex{Schoenthan-Pernwald, Paul von 19.03.1853 – 04.08.1905@\textsc{Schönthan-Pernwald, Paul von} (19.03.1853 – 04.08.1905), \emph{Schriftsteller, Journalist}|pw} ging umher, und erzählte
               den Schauspielern, dass er dieses Stück\pwindex{Schoenthan-Pernwald, Paul von 19.03.1853 – 04.08.1905@\textsc{Schönthan-Pernwald, Paul von} (19.03.1853 – 04.08.1905), \emph{Schriftsteller, Journalist}!Gelegenheitskauf1896-02-03@\strich\emph{Gelegenheitskauf} {[}1896-02-03{]}|pwv} mit seinem Herzblut geschrieben, – man überschätzt die Leute noch
               immer. Der Gelegenheits{\pb}kauf\pwindex{Schoenthan-Pernwald, Paul von 19.03.1853 – 04.08.1905@\textsc{Schönthan-Pernwald, Paul von} (19.03.1853 – 04.08.1905), \emph{Schriftsteller, Journalist}!Gelegenheitskauf1896-02-03@\strich\emph{Gelegenheitskauf} {[}1896-02-03{]}|pw}{ }ist übrigens im Burgtheater\orgindex{Burgtheater@Burgtheater|pw} und im Lessingtheater\orgindex{Lessing-Theater@Lessing-Theater|pw}
               angenommen.\pend
           \pstart
           Eben kommt das Repertoire. Sie sind in dieser Woche nicht drauf, was auch erklärlich
                  ist{[}.{]}{ }Dienstag kommt der Dornenweg\pwindex{\textcolor{red}{\textsuperscript{XXXX1 indx}}!Dornenweg. Schauspiel in drei Aufzuegen1895@\strich\emph{Der Dornenweg. Schauspiel in drei Aufzügen} {[}1895{]}|pw}. Da sind Sie ja bis Abends da, und \label{K_L03169-12v}\edtext{im Theater\oindex{Burgtheater@\textbf{Burgtheater}|pwv}}{\lemma{\textnormal{\emph{im Theater}}}\Cendnote{\textnormal{Bei der Premiere von \emph{Der Dornenweg}\pwindex{\textcolor{red}{\textsuperscript{XXXX1 indx}}!Dornenweg. Schauspiel in drei Aufzuegen1895@\strich\emph{Der Dornenweg. Schauspiel in drei Aufzügen} {[}1895{]}|pwk} im Burgtheater\oindex{Burgtheater@\textbf{Burgtheater}|pwk}, siehe A. S.: \emph{Tagebuch}, 11. 2. 1896.}}}\label{K_L03169-12h}.\pend
           \pstart
           Herzlichst Ihr {\\[\baselineskip]}\spacefill\mbox{Salten}\pend
           \leftskip=0em{}
         
         \endnumbering\mylabel{h}\end{ledgroupsized}  \newcommand{\dateiname}{L03169}\newcommand{\titel}{Felix Salten an Arthur Schnitzler, [8. 2. 1896]}\newcommand{\editorInnen}{Martin Anton Müller und Laura Untner}%% latex-leseansicht-abspann.tex
%% Abspann für die Leseansicht.
%% Der Schalter \ifkorrekturansicht ist bereits durch den Vorspann gesetzt.

%% latex-abspann.tex
%% Gemeinsamer Abspann für Korrekturansicht und Leseansicht.
%% Setzt den Schalter \ifkorrekturansicht voraus (gesetzt in den
%% einbindenden Dateien latex-korrekturansicht-abspann.tex bzw.
%% latex-leseansicht-abspann.tex).
%% ---------------------------------------------------------------

\normalsize

% Das esempio-Environment wird nur in der Leseansicht benötigt
\ifkorrekturansicht\else
\newenvironment{esempio}[3]%
{
    \vspace{1.5ex}
    \rlap{\underline{#1}}
    \par
    \setlength{\parindent}{0cm}
    \nopagebreak
    \leftskip=#2cm
    \rightskip=#3cm
}
{
    \par
}
\fi

\doendnotes{C}
\bigskip
\vfill

\clearpage

\footnotesize

\ifkorrekturansicht
  \lohead{\textsc{register}}
\fi

% theindex-Environment neu definieren ohne reledmac
\makeatletter
\renewenvironment{theindex}{%
  \ifkorrekturansicht
    \section*{\indexname}%
  \else
    \subsubsection*{Index der erwähnten Entitäten}%
  \fi
  \setlength{\parindent}{0pt}%
  \setlength{\parskip}{0pt plus 0.3pt}%
  \let\item\@idxitem
}{%
  \ifkorrekturansicht\clearpage\fi
}
\makeatother

\IfFileExists{\jobname-pw.ind}{\input{\jobname-pw.ind}}{}

% Quellenangabe nur in der Leseansicht
\ifkorrekturansicht\else
% Fallback-Definitionen, falls die .tex-Datei \titel etc. nicht gesetzt hat
\providecommand{\titel}{}
\providecommand{\editorInnen}{}
\providecommand{\dateiname}{\jobname}

\vspace{3cm}

\vfill

\footnotesize
\textsc{Quelle}: \titel. Herausgegeben von {\editorInnen}. In: \emph{Arthur Schnitzler: Briefwechsel mit Autorinnen und Autoren}.
 Digitale Edition, https://schnitzler-briefe.acdh.oeaw.ac.at/{\dateiname}.html (Stand \today)
\fi

\end{document}


      