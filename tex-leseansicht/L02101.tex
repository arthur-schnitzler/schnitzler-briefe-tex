%% latex-korrekturansicht-vorspann.tex
%% Vorspann für die Korrekturansicht.
%% Lädt die gemeinsame Datei latex-vorspann.tex mit gesetztem Schalter.

\newif\ifkorrekturansicht
\korrekturansichttrue

\input{../tex-inputs/latex-vorspann}


\section[Arthur Schnitzler an Georg Brandes, 20. 11. 1912]{L02101 Arthur Schnitzler an Georg Brandes, 20. 11. 1912}
\nopagebreak\mylabel{L02101v}
\rehead{ }\normalsize\beginnumbering\briefempfaengerindex{Brandes, Georg@\textsc{Brandes, Georg}!zzzSchnitzler, Arthur@\emph{von Arthur Schnitzler}!1912-11-201@{20. 11. 1912}|(be}
\toendnotes[C]{\smallbreak\pagebreak[2]}\Standort{Kopenhagen, Det Kongelige Bibliotek, Georg Brandes Arkiv, box 125.}
\physDesc{Brief, 1 Blatt, 1 Seite, 817 Zeichen
\newline{}Schreibmaschine
\newline{}Handschrift: schwarze Tinte, deutsche Kurrent (\noindent{}eine Korrektur, Unterschrift, Nachschrift)
\newline{}Ordnung: mit Bleistift von unbekannter Hand nummeriert:
                                    »33.« }
\buchAbdrucke{\weitereDrucke{Georg Brandes, Arthur Schnitzler: \emph{Ein Briefwechsel}. Bern: \emph{Francke} 1956, S. 105.} }\toendnotes[C]{\smallbreak}
\pstart
           {\pb}\textcolor{gray}{\textbf{Dr. Arthur Schnitzler}}{\\}\textcolor{gray}{\textbf{Wien XVIII. Sternwartestrasse 71\oindex{Sternwartestrasse 71@\textbf{Sternwartestraße 71}, \emph{Wohngebäude (K.WHS)}|pw}}}\pend
           
\pstart
           \raggedleft{}20. 11. 1912. \pend
           
\pstart\center{}Lieber und verehrter Herr Brandes.\pend\vspace{0.5em}
\pstart
           Da ich leider nicht weiss, wo Sie abgestiegen sind, sende ich Ihnen diesen Brief in
               die Urania\oindex{Urania@\textbf{Urania}, \emph{Volksbildungsanstalt (K.VBA)}|pw}. Ich frage vor allem bei Ihnen an, ob
               Sie uns das Vergnügen machen wollen am Freitag Abend gegen acht bei uns
               zu essen. Es wäre sehr liebenswürdig von Ihnen mir gleich nach Empfang dieser Zeilen
               pneumatisch eine zusagende Antwort zu senden. Morgen Abend, Donnerstag,
               werde ich Ihnen nach Ihrer \label{K_L02101-1v}\edtext{Vorlesung}{\lemma{\textnormal{\emph{Vorlesung}}}\Cendnote{\textnormal{In seinem zweiten Vortrag
                  sprach Brandes\pwindex{Brandes, Georg 04.02.1842 – 19.02.1927@\textsc{Brandes, Georg} (04.02.1842 – 19.02.1927)|pwk} am 21. 11. 1912 um ½ 8 im Volksbildungshaus
                     Urania\oindex{Urania@\textbf{Urania}, \emph{Volksbildungsanstalt (K.VBA)}|pwk} über »Goethe\pwindex{Goethe, Johann Wolfgang von 1749-08-28 – 1832-03-22@\textsc{Goethe, Johann Wolfgang von} (1749-08-28 – 1832-03-22), \emph{Schriftsteller/Schriftstellerin}|pwk} und die Zeitalter«. Am Vortag hatte er bereits über »Jeanne d’Arc\pwindex{Arc, Jeanne D um 1412 – 1431-05-30@\textsc{Arc, Jeanne d’} (um 1412 – 1431-05-30), \emph{Nationalheiliger/Nationalheilige, Militär/Militärin}|pw}« gesprochen, die dritte und letzte Vorlesung war Strindberg\pwindex{Strindberg, August 22.01.1849 – 14.05.1912@\textsc{Strindberg, August} (22.01.1849 – 14.05.1912), \emph{Schriftsteller/Schriftstellerin}|pwk} gewidmet.}}}\label{K_L02101-1} endlich wieder \label{K_L02101-2v}\edtext{die Hand drücken}{\lemma{\textnormal{\emph{die Hand drücken}}}\Cendnote{\textnormal{Vgl. A. S.: \emph{Tagebuch}, 21. 11. 1912.
               }}}\label{K_L02101-2} können. Seien Sie willkommen in Wien\oindex{Wien@\textbf{Wien}, \emph{A.ADM2}|pw} und
               herzliche Grüsse.\pend
           
\pstart
           Ihr sehr ergebener{\\[\baselineskip]}\spacefill\mbox{{[}hs.:{]} ArthurSchnitzler}\pend
           \leftskip=0em{}
\pstart
           {[}ms.:{]} Samstag{ }Abend fahre ich nach Berlin\oindex{Berlin@\textbf{Berlin}, \emph{P.PPLC}|pw} zu
                  den Proben meines neuen Stückes\pwindex{Professor Bernhardi. Komoedie in fuenf Akten@\emph{Professor Bernhardi. Komödie in fünf Akten}|pwv}. Sollten Sie den Freitag Abend schon vergeben haben, so schenken
                  Sie uns den \introOben{}Freitag\introOben{} Mittag gegen ½ 2.\pend
           
\pstart
           Herrn Georg Brandes, Wien\oindex{Wien@\textbf{Wien}, \emph{A.ADM2}|pw}.\pend
           
\pstart
           {[}hs.:{]} Erfahre eben Ihre Adreſſe – ſchicke alſo den Brief ans Continental\oindex{Hotel Continental [Wien]@\textbf{Hotel Continental [Wien]}, \emph{Hotel (K.HTL)}|pw}.\pend
           \selectlanguage{ngerman}\endnumbering\briefempfaengerindex{Brandes, Georg@\textsc{Brandes, Georg}!zzzSchnitzler, Arthur@\emph{von Arthur Schnitzler}!1912-11-201@{20. 11. 1912}|)be}\mylabel{L02101h}  \normalsize

\doendnotes{C}
\bigskip
\vfill

\clearpage

\footnotesize

\lohead{\textsc{register}}

% Definiere theindex-Environment komplett neu ohne reledmac
\makeatletter
\renewenvironment{theindex}{%
  \section*{\indexname}%
  \setlength{\parindent}{0pt}%
  \setlength{\parskip}{0pt plus 0.3pt}%
  \let\item\@idxitem
}{%
  \clearpage
}
\makeatother

\IfFileExists{\jobname-pw.ind}{\input{\jobname-pw.ind}}{}

\end{document}

      