%% latex-leseansicht-vorspann.tex
%% Vorspann für die Leseansicht.
%% Lädt die gemeinsame Datei latex-vorspann.tex mit nicht gesetztem Schalter.

\newif\ifkorrekturansicht
\korrekturansichtfalse

\input{../tex-inputs/latex-vorspann}


\section[Arthur Schnitzler an Georg Brandes, 20. 11. 1912]{L02101 Arthur Schnitzler an Georg Brandes, 20. 11. 1912}
\nopagebreak\mylabel{L02101v}
\rehead{ }\normalsize\beginnumbering\briefempfaengerindex{Brandes, Georg@\textsc{Brandes, Georg}!zzzSchnitzler, Arthur@\emph{von Arthur Schnitzler}!1912-11-201@{20. 11. 1912}|(be}
\toendnotes[C]{\smallbreak\pagebreak[2]}
\correspDesc{Versand  durch Arthur Schnitzler am 20. 11. 1912 in Wien
\newline{}Erhalt  durch Georg Brandes im Zeitraum [20. 11. 1912 – 24. 11. 1912?] in Wien}\toendnotes[C]{\smallbreak}
\Standort{Kopenhagen, Det Kongelige Bibliotek, Georg Brandes Arkiv, box 125.}
\physDesc{Brief, 1 Blatt, 1 Seite, 817 Zeichen
\newline{}Schreibmaschine
\newline{}Handschrift: schwarze Tinte, deutsche Kurrent (\noindent{}eine Korrektur, Unterschrift, Nachschrift)
\newline{}Ordnung: mit Bleistift von unbekannter Hand nummeriert:
                                    »33.« }
\buchAbdrucke{\weitereDrucke{Georg Brandes, Arthur Schnitzler: \emph{Ein Briefwechsel}. Herausgegeben von Kurt Bergel. Bern: \emph{Francke} 1956, S. 105.} }\toendnotes[C]{\smallbreak}
\pstart
           {\pb}\textcolor{gray}{\textbf{Dr. Arthur Schnitzler}}{\\}\textcolor{gray}{\textbf{Wien XVIII. Sternwartestrasse 71\oindex{Wien@\textbf{Wien}!XVIII., Währing@\textbf{XVIII., Währing}!Sternwartestraße 71@\textbf{Sternwartestraße 71}, \emph{Wohngebäude}|pw}}}\pend
           
\pstart
           \raggedleft{}20. 11. 1912.\pend
           
\pstart\center{}Lieber und verehrter Herr Brandes.\pend\vspace{0.5em}
\pstart
           Da ich leider nicht weiss, wo Sie abgestiegen sind, sende ich Ihnen diesen Brief in
               die Urania\oindex{Wien@\textbf{Wien}!I., Innere Stadt@\textbf{I., Innere Stadt}!Urania@\textbf{Urania}, \emph{Volksbildungsanstalt}|pw}. Ich frage vor allem bei Ihnen an, ob
               Sie uns das Vergnügen machen wollen am Freitag Abend gegen acht bei uns
               zu essen. Es wäre sehr liebenswürdig von Ihnen mir gleich nach Empfang dieser Zeilen
               pneumatisch eine zusagende Antwort zu senden. Morgen Abend, Donnerstag,
               werde ich Ihnen nach Ihrer \label{K_L02101-1v}\edtext{Vorlesung}{\lemma{\textnormal{\emph{Vorlesung}}}\Cendnote{\textnormal{In seinem zweiten Vortrag
                  sprach Brandes\pwindex{Brandes, Georg 4.\,2.\,1842 Kopenhagen – 19.\,2.\,1927 ebd.@\textsc{Brandes, Georg} (4.\,2.\,1842 Kopenhagen – 19.\,2.\,1927 ebd.)|pwk} am 21. 11. 1912 um ½ 8 im Volksbildungshaus
                     Urania\oindex{Wien@\textbf{Wien}!I., Innere Stadt@\textbf{I., Innere Stadt}!Urania@\textbf{Urania}, \emph{Volksbildungsanstalt}|pwk} über »Goethe\pwindex{Goethe, Johann Wolfgang von 28.\,8.\,1749 Frankfurt am Main – 22.\,3.\,1832 Weimar@\textsc{Goethe, Johann Wolfgang von} (28.\,8.\,1749 Frankfurt am Main – 22.\,3.\,1832 Weimar), \emph{Schriftsteller}|pwk} und die Zeitalter«. Am Vortag hatte er bereits über »Jeanne d’Arc\pwindex{Arc, Jeanne d’ um 1412 Domrémy-la-Pucelle – 30.\,5.\,1431 Rouen@\textsc{Arc, Jeanne d’} (um 1412 Domrémy-la-Pucelle – 30.\,5.\,1431 Rouen), \emph{Nationalheilige, Militärin}|pw}« gesprochen, die dritte und letzte Vorlesung war Strindberg\pwindex{Strindberg, August 22.\,1.\,1849 Stockholm – 14.\,5.\,1912 ebd.@\textsc{Strindberg, August} (22.\,1.\,1849 Stockholm – 14.\,5.\,1912 ebd.), \emph{Schriftsteller}|pwk} gewidmet.}}}\label{K_L02101-1} endlich wieder \label{K_L02101-2v}\edtext{die Hand drücken}{\lemma{\textnormal{\emph{die Hand drücken}}}\Cendnote{\textnormal{Vgl. A. S.: \emph{Tagebuch}, 21. 11. 1912.
               }}}\label{K_L02101-2} können. Seien Sie willkommen in Wien\oindex{Wien@\textbf{Wien}, \emph{Verwaltungsgebiet}|pw} und
               herzliche Grüsse.\pend
           
\pstart
           Ihr sehr ergebener{\\[\baselineskip]}\spacefill\mbox{{[}hs.:{]} ArthurSchnitzler}\pend
           \leftskip=0em{}
\pstart
           {[}ms.:{]} Samstag{ }Abend fahre ich nach Berlin\oindex{Berlin@\textbf{Berlin}, \emph{Hauptstadt}|pw} zu
                  den Proben meines neuen Stückes\pwindex{Schnitzler, Arthur 15.\,5.\,1862 Wien – 21.\,10.\,1931 ebd.@\textsc{Schnitzler, Arthur} (15.\,5.\,1862 Wien – 21.\,10.\,1931 ebd.), \emph{Schriftsteller, Mediziner}!Professor Bernhardi. Komödie in fünf Akten@\strich\emph{Professor Bernhardi. Komödie in fünf Akten}|pwv}. Sollten Sie den Freitag Abend schon vergeben haben, so schenken
                  Sie uns den \introOben{}Freitag\introOben{} Mittag gegen ½ 2.\pend
           
\pstart
           Herrn Georg Brandes, Wien\oindex{Wien@\textbf{Wien}, \emph{Verwaltungsgebiet}|pw}.\pend
           
\pstart
           {[}hs.:{]} Erfahre eben Ihre Adreſſe –{ }ſchicke alſo den Brief ans Continental\oindex{Wien@\textbf{Wien}!II., Leopoldstadt@\textbf{II., Leopoldstadt}!Hotel Continental [Wien]@\textbf{Hotel Continental [Wien]}, \emph{Hotel}|pw}.\pend
           \selectlanguage{ngerman}\endnumbering\briefempfaengerindex{Brandes, Georg@\textsc{Brandes, Georg}!zzzSchnitzler, Arthur@\emph{von Arthur Schnitzler}!1912-11-201@{20. 11. 1912}|)be}\mylabel{L02101h}  \newcommand{\dateiname}{L02101}\newcommand{\titel}{Arthur Schnitzler an Georg Brandes, 20. 11. 1912}\newcommand{\editorInnen}{Martin Anton Müller und Gerd-Hermann Susen}%% latex-leseansicht-abspann.tex
%% Abspann für die Leseansicht.
%% Der Schalter \ifkorrekturansicht ist bereits durch den Vorspann gesetzt.

%% latex-abspann.tex
%% Gemeinsamer Abspann für Korrekturansicht und Leseansicht.
%% Setzt den Schalter \ifkorrekturansicht voraus (gesetzt in den
%% einbindenden Dateien latex-korrekturansicht-abspann.tex bzw.
%% latex-leseansicht-abspann.tex).
%% ---------------------------------------------------------------

\normalsize

% Das esempio-Environment wird nur in der Leseansicht benötigt
\ifkorrekturansicht\else
\newenvironment{esempio}[3]%
{
    \vspace{1.5ex}
    \rlap{\underline{#1}}
    \par
    \setlength{\parindent}{0cm}
    \nopagebreak
    \leftskip=#2cm
    \rightskip=#3cm
}
{
    \par
}
\fi

\doendnotes{C}
\bigskip
\vfill

\clearpage

\footnotesize

\ifkorrekturansicht
  \lohead{\textsc{register}}
\fi

% theindex-Environment neu definieren ohne reledmac
\makeatletter
\renewenvironment{theindex}{%
  \ifkorrekturansicht
    \section*{\indexname}%
  \else
    \subsubsection*{Index der erwähnten Entitäten}%
  \fi
  \setlength{\parindent}{0pt}%
  \setlength{\parskip}{0pt plus 0.3pt}%
  \let\item\@idxitem
}{%
  \ifkorrekturansicht\clearpage\fi
}
\makeatother

\IfFileExists{\jobname-pw.ind}{\input{\jobname-pw.ind}}{}

% Quellenangabe nur in der Leseansicht
\ifkorrekturansicht\else
% Fallback-Definitionen, falls die .tex-Datei \titel etc. nicht gesetzt hat
\providecommand{\titel}{}
\providecommand{\editorInnen}{}
\providecommand{\dateiname}{\jobname}

\vspace{3cm}

\vfill

\footnotesize
\textsc{Quelle}: \titel. Herausgegeben von {\editorInnen}. In: \emph{Arthur Schnitzler: Briefwechsel mit Autorinnen und Autoren}.
 Digitale Edition, https://schnitzler-briefe.acdh.oeaw.ac.at/{\dateiname}.html (Stand \today)
\fi

\end{document}


