%% latex-korrekturansicht-vorspann.tex
%% Vorspann für die Korrekturansicht.
%% Lädt die gemeinsame Datei latex-vorspann.tex mit gesetztem Schalter.

\newif\ifkorrekturansicht
\korrekturansichttrue

\input{../tex-inputs/latex-vorspann}


\section[Arthur Schnitzler an Hermann Bahr, 26. 6. 1901]{L01134 Arthur Schnitzler an Hermann Bahr, 26. 6. 1901}
\nopagebreak\mylabel{L01134v}
\rehead{ }\normalsize\beginnumbering\briefempfaengerindex{Bahr, Hermann@\textsc{Bahr, Hermann}!zzzSchnitzler, Arthur@\emph{von Arthur Schnitzler}!1901-06-261@{26. 6. 1901}|(be}
\toendnotes[C]{\smallbreak\pagebreak[2]}\Standort{TMW, HS AM 23344 Ba.}
\physDesc{Brief, 1 Blatt, 4 Seiten, 1452 Zeichen
\newline{}Handschrift: schwarze Tinte, deutsche Kurrent
\newline{}Ordnung: 1) Lochung  2) mit Bleistift von unbekannter Hand datiert: »26. 6. 01«}
\buchAbdrucke{\weitereDrucke{1) Arthur Schnitzler: \emph{Briefe 1875–1912}. Frankfurt am Main: \emph{S. Fischer} 1981, S. 347–348.} \weitereDrucke{2) Arthur Schnitzler: \emph{The Letters of Arthur Schnitzler to Hermann Bahr}. Chapel Hill: \emph{The University of North Carolina Press} 1978, S. 68.} \weitereDrucke{3) \emph{Die Neue Rundschau}, Bd. 68 (1957) Nr. 1, S. 91–92.} \weitereDrucke{4) Hermann Bahr, Arthur Schnitzler: \emph{Briefwechsel, Aufzeichnungen, Dokumente (1891–1931)}. Göttingen: \emph{Wallstein} 2018, S. 211.} }\toendnotes[C]{\smallbreak}
\pstart
           \noindent{}{\pb}mein lieber
                  Hermann, ich danke dir herzlich für den neuen Beweis von Sympathie, den
               du mir mit deinem lieben Brief vom 22. gegeben haſt. Über die Sache
               ſelbſt iſt ja kaum was zu ſagen – ſelten lag ein Fall klarer zu Tage. Wahrhaftig –
               ſie haben meinen Lieutenant Guſtl\pwindex{Lieutenant Gustl. Novelle@\emph{Lieutenant Gustl. Novelle}|pw} nicht
               verdient! Ich ſeh es ein. Haſt du vielleicht neulich den  \label{K_L01134-1v}\edtext{Artikel\pwindex{Lieutenant Gustl«@\emph{»Lieutenant Gustl«}|pwv}}{\lemma{\textnormal{\emph{Artikel}}}\Cendnote{\textnormal{Obwohl ohne Verfasserangabe erschienen,
                  ist \emph{»Lieutenant Gustl«}\pwindex{Lieutenant Gustl«@\emph{»Lieutenant Gustl«}|pwk} (\emph{Reichswehr}\orgindex{Reichswehr@Reichswehr|pwk}, Jg. 14, Nr. 2645, Morgenblatt,
                     S. 1–2) durch die Position als Editorial dem Herausgeber Davis\pwindex{Davis, Gustav 03.03.1856 – 21.08.1951@\textsc{Davis, Gustav} (03.03.1856 – 21.08.1951), \emph{Journalist/Journalistin, Herausgeber/Herausgeberin}|pwk} zuzuschreiben, was Bahr\pwindex{Bahr, Hermann 19.07.1863 – 15.01.1934@\textsc{Bahr, Hermann} (19.07.1863 – 15.01.1934), \emph{Schriftsteller/Schriftstellerin, Kritiker/Kritikerin}|pwk} auch in der Folge tut (Hermann Bahr an Arthur Schnitzler, 5. 7. 1901).}}}\label{K_L01134-1} in der Reichswehr\orgindex{Reichswehr@Reichswehr|pw} geleſen? Ich glaube, in dem ſteht das großartigſte an Dummheit,
               was in dieser Affaire geleiſtet wurde. Nemlich: ich hätte meine {\pb}Charge nur deshalb
               nicht vor 5 Jahren (wie es mein Recht geweſen) nieder gelegt – \substVorne{}\textsuperscript{,}\substDazwischen{}weil\substHinten{} ich eben doch gern gelegentlich in Uniform \introOben{}u\introOben{} mit
               dem \label{K_L01134-2v}\edtext{Stürmer}{\lemma{\textnormal{\emph{Stürmer}}}\Cendnote{\textnormal{Mütze, die durch unterschiedliche Ausprägungen über die
                  Zugehörigkeit zu einer bestimmten Burschenschaft Auskunft gab; Schnitzler gibt den Artikel\pwindex{Lieutenant Gustl«@\emph{»Lieutenant Gustl«}|pwkv} jedoch in diesem Detail falsch wieder, darin
                  wird von »Federhut, Säbel und Porte-\textsc{épée}« gesprochen.}}}\label{K_L01134-2} paradirt! – Ich wollte einen Preis von einer Million
               ausſchreiben für den, der mich ſeit meinem letzten Hauptrapport in Uniform geſehen –
               aber wer weiſs – unter dieſen Leuten findet ſich am Ende auch einer, der es
               beſchwören kann.\pend
           
\pstart
           Laß mich bei dieſer Gelegenheit auch einmal ſagen, wie ſehr es {\pb}mich freut, daſs wir
               nun beide über die \label{LL036-1v}zeitweiligen
                  Entfremdungen\label{LL036-1h} hinaus ſind, die ja wahrſcheinlich bei Naturen wie den
               unſern entwicklungsphyſiologiſch bedingt und daher nothwendg ſind (du ſiehſt ich bin
               immer »wiſſenschaftlich«.) Nun iſt das Alter der M\damage{is}verständniſſe wohl endgiltig für uns vorbei und wir ſind ſo weit, daſs wir
               einander – vielleicht auch ein bischen um unſerer Fehler willen – Freunde ſein und
                  {\pb}bleiben dürfen.\pend
           
\pstart
           In dieſer Vorausſicht drücke ich dir von Herzen die Hand und grüße dich
               vielmals{\\[\baselineskip]} dein \spacefill\mbox{Arthur}\pend
           \leftskip=0em{}
\pstart
           \textsc{Innsbruck\oindex{Innsbruck@\textbf{Innsbruck}, \emph{A.ADM2}|pw}}, 26. 6. 901\pend
           \selectlanguage{ngerman}\endnumbering\briefempfaengerindex{Bahr, Hermann@\textsc{Bahr, Hermann}!zzzSchnitzler, Arthur@\emph{von Arthur Schnitzler}!1901-06-261@{26. 6. 1901}|)be}\mylabel{L01134h}  \normalsize

\doendnotes{C}
\bigskip
\vfill

\clearpage

\footnotesize

\lohead{\textsc{register}}

% Definiere theindex-Environment komplett neu ohne reledmac
\makeatletter
\renewenvironment{theindex}{%
  \section*{\indexname}%
  \setlength{\parindent}{0pt}%
  \setlength{\parskip}{0pt plus 0.3pt}%
  \let\item\@idxitem
}{%
  \clearpage
}
\makeatother

\IfFileExists{\jobname-pw.ind}{\input{\jobname-pw.ind}}{}

\end{document}

      