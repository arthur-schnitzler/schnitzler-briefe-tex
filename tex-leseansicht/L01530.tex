%% latex-leseansicht-vorspann.tex
%% Vorspann für die Leseansicht.
%% Lädt die gemeinsame Datei latex-vorspann.tex mit nicht gesetztem Schalter.

\newif\ifkorrekturansicht
\korrekturansichtfalse

\input{../tex-inputs/latex-vorspann}


\section[Hugo und Gerty von Hofmannsthal an Arthur Schnitzler, {{[}}16. 7. 1905?{{]}}]{L01530 Hugo und Gerty von Hofmannsthal an Arthur Schnitzler, {[}16. 7. 1905?{]}}
\nopagebreak\mylabel{L01530v}
\rehead{ }\normalsize\beginnumbering\briefempfaengerindex{Schnitzler, Arthur@\textsc{Schnitzler, Arthur}!zzzHofmannsthal, Gertrude von@\emph{von Gertrude von Hofmannsthal}!1905-07-162@{{[}16. 7. 1905?{]}}|(be}\briefempfaengerindex{Schnitzler, Arthur@\textsc{Schnitzler, Arthur}!zzzHofmannsthal, Hugo von@\emph{von Hugo von Hofmannsthal}!1905-07-162@{{[}16. 7. 1905?{]}}|(be}
\toendnotes[C]{\smallbreak\pagebreak[2]}
\correspDesc{Versand  durch Hugo von Hofmannsthal, Gerty von Hofmannsthal am [16. 7. 1905?] in Kremsmünster
\newline{}Erhalt  durch Arthur Schnitzler im Zeitraum [17. 7. 1905
                  – 21. 7. 1905?] in Wien}\toendnotes[C]{\smallbreak}
\buchAlsQuelle{Hugo von Hofmannsthal, Arthur Schnitzler: \emph{Briefwechsel}. Herausgegeben von Therese Nickl und Heinrich Schnitzler. Frankfurt am Main: \emph{S. Fischer} 1964, S. 214.}\toendnotes[C]{\smallbreak}
\pstart
           {\pb}{[}Ansichtskarte{]}\hfill {[}\label{K_L01530-1v}\edtext{Kremsmünster}{\lemma{\textnormal{\emph{Kremsmünster}}}\Cendnote{\textnormal{Dieser Besuch fand, als
                              Tagesausflug von Wels\oindex{Wels@\textbf{Wels}, \emph{Hauptstadt}|pwk}, am
                                 16. 7. 1905 statt.}}}\label{K_L01530-1}\oindex{Kremsmünster@\textbf{Kremsmünster}, \emph{Verwaltungsgebiet}|pw}, 1905{]}\pend
           \vspace{0.5em}
\pstart
           Einen solchen \label{K_L01530-2v}\edtext{Gang}{\lemma{\textnormal{\emph{Gang}}}\Cendnote{\textnormal{Gemeint dürfte der Arkadengang des Stiftes Kremsmünster\oindex{Stift Kremsmünster@\textbf{Stift Kremsmünster}, \emph{Kloster}|pwk} sein.}}}\label{K_L01530-2} wünschen
               Ihnen \spacefill\mbox{Hugo}{ }\spacefill\mbox{{[}hs. Hofmannsthal:{]} – Gerty.}\pend
           \selectlanguage{ngerman}\endnumbering\briefempfaengerindex{Schnitzler, Arthur@\textsc{Schnitzler, Arthur}!zzzHofmannsthal, Gertrude von@\emph{von Gertrude von Hofmannsthal}!1905-07-162@{{[}16. 7. 1905?{]}}|)be}\briefempfaengerindex{Schnitzler, Arthur@\textsc{Schnitzler, Arthur}!zzzHofmannsthal, Hugo von@\emph{von Hugo von Hofmannsthal}!1905-07-162@{{[}16. 7. 1905?{]}}|)be}\mylabel{L01530h}  \newcommand{\dateiname}{L01530}\newcommand{\titel}{Hugo und Gerty von Hofmannsthal an Arthur Schnitzler, [16. 7. 1905?]}\newcommand{\editorInnen}{Martin Anton Müller und Gerd-Hermann Susen}%% latex-leseansicht-abspann.tex
%% Abspann für die Leseansicht.
%% Der Schalter \ifkorrekturansicht ist bereits durch den Vorspann gesetzt.

%% latex-abspann.tex
%% Gemeinsamer Abspann für Korrekturansicht und Leseansicht.
%% Setzt den Schalter \ifkorrekturansicht voraus (gesetzt in den
%% einbindenden Dateien latex-korrekturansicht-abspann.tex bzw.
%% latex-leseansicht-abspann.tex).
%% ---------------------------------------------------------------

\normalsize

% Das esempio-Environment wird nur in der Leseansicht benötigt
\ifkorrekturansicht\else
\newenvironment{esempio}[3]%
{
    \vspace{1.5ex}
    \rlap{\underline{#1}}
    \par
    \setlength{\parindent}{0cm}
    \nopagebreak
    \leftskip=#2cm
    \rightskip=#3cm
}
{
    \par
}
\fi

\doendnotes{C}
\bigskip
\vfill

\clearpage

\footnotesize

\ifkorrekturansicht
  \lohead{\textsc{register}}
\fi

% theindex-Environment neu definieren ohne reledmac
\makeatletter
\renewenvironment{theindex}{%
  \ifkorrekturansicht
    \section*{\indexname}%
  \else
    \subsubsection*{Index der erwähnten Entitäten}%
  \fi
  \setlength{\parindent}{0pt}%
  \setlength{\parskip}{0pt plus 0.3pt}%
  \let\item\@idxitem
}{%
  \ifkorrekturansicht\clearpage\fi
}
\makeatother

\IfFileExists{\jobname-pw.ind}{\input{\jobname-pw.ind}}{}

% Quellenangabe nur in der Leseansicht
\ifkorrekturansicht\else
% Fallback-Definitionen, falls die .tex-Datei \titel etc. nicht gesetzt hat
\providecommand{\titel}{}
\providecommand{\editorInnen}{}
\providecommand{\dateiname}{\jobname}

\vspace{3cm}

\vfill

\footnotesize
\textsc{Quelle}: \titel. Herausgegeben von {\editorInnen}. In: \emph{Arthur Schnitzler: Briefwechsel mit Autorinnen und Autoren}.
 Digitale Edition, https://schnitzler-briefe.acdh.oeaw.ac.at/{\dateiname}.html (Stand \today)
\fi

\end{document}


