%% latex-leseansicht-vorspann.tex
%% Vorspann für die Leseansicht.
%% Lädt die gemeinsame Datei latex-vorspann.tex mit nicht gesetztem Schalter.

\newif\ifkorrekturansicht
\korrekturansichtfalse

\input{../tex-inputs/latex-vorspann}


         
         \renewcommand{\erwaehntePersonen}{Personen: Richard Beer-Hofmann, Paul Goldmann, Hugo von Hofmannsthal, Victor Carl Wolf, Emil Wolf, Charlotte Wolf}
         \renewcommand{\erwaehnteOrte}{Orte: Brünn, Bösendorferstraße, Kärntnerring, Kärntnerring 12/Bösendorferstraße 11, Ordination Dr. Arthur Schnitzler Giselastraße 11, Wien}
         \renewcommand{\erwaehnteWerke}{Werke: Alkandi’s Lied, An der schönen blauen Donau, Bleisoldaten. Novellette, Episode, Lieder eines Nervösen}
               \section[Richard Beer-Hofmann an Arthur Schnitzler, 30. 5. 1891]{ Richard Beer-Hofmann an Arthur Schnitzler, 30. 5. 1891}\nopagebreak\mylabel{v}\rehead{ }\begin{ledgroupsized}[t]{13cm}\normalsize\beginnumbering\briefempfaengerindex{Schnitzler, Arthur@\textsc{Schnitzler, Arthur}!zzzBeer-Hofmann, Richard@\emph{von Richard Beer-Hofmann}!1891-05-301@{30. 5. 1891}|(be} \toendnotes[C]{\smallbreak\pagebreak[2]} \Standort{CUL, Schnitzler, B 8.}
\physDesc{Briefkarte, 1195 Zeichen
\newline{}Handschrift: schwarze Tinte, lateinische Kurrent
\newline{}Schnitzler: mit Bleistift datiert: »30/5 91« und nummeriert: »2.« }\buchAbdrucke{\weitereDrucke{Arthur Schnitzler, Richard Beer-Hofmann: \emph{Briefwechsel 1891–1931}. Hg. Konstanze Fliedl. Wien, Zürich: \emph{Europaverlag} 1992, S. 30.} }\toendnotes[C]{\smallbreak}\pstart\center{}{\pb}Lieber Arthur!\pend\pstart
           Denken Sie mein Cousin\pwindex{Wolf, Victor Carl 1866 – 10.3.1903@\textsc{Wolf, Victor Carl} (1866 – 10.3.1903), \emph{Unternehmer}|pwuv}\pwindex{Wolf, Emil 24.06.1864 – 31.10.1942@\textsc{Wolf, Emil} (24.06.1864 – 31.10.1942), \emph{Rechtsanwalt}|pwuv} hat auf mein Anrathen \label{K_L00016-1v}\edtext{die alten Jahrgänge}{\lemma{\textnormal{\emph{die alten Jahrgänge}}}\Cendnote{\textnormal{\emph{An der schönen blauen Donau}\pwindex{der schoenen blauen Donau1886-01-15 – 1896@\emph{An der schönen blauen Donau} {[}1886-01-15 – 1896{]}|pwk}, ein
                  »Unterhaltungsblatt für die Familie«, erschien seit dem 15. 1. 1886
                  alle 14 Tage. Die von Beer-Hofmann\pwindex{Beer-Hofmann, Richard 1866-07-11 – 1945-09-26@\textsc{Beer-Hofmann, Richard} (1866-07-11 – 1945-09-26), \emph{Schriftsteller}|pwk}
                  angesprochenen Texte finden sich in den Jahrgängen 1888 bis 1890.}}}\label{K_L00016-1h} der »blauen Donau\pwindex{der schoenen blauen Donau1886-01-15 – 1896@\emph{An der schönen blauen Donau} {[}1886-01-15 – 1896{]}|pw}« gekauft und an Sonntag
               Nachmittagen, wenn ich frei bin lese ich Einzelnes daraus vor; Philisterpublikum zum
               größten Theil aber Publikum. Loris\pwindex{Hofmannsthal, Hugo von 1874-02-01 – 1929-07-15@\textsc{Hofmannsthal, Hugo von} (1874-02-01 – 1929-07-15), \emph{Schriftsteller}|pw}{ }Gedichte, von Paul\pwindex{Goldmann, Paul 31.01.1865 – 25.09.1935@\textsc{Goldmann, Paul} (31.01.1865 – 25.09.1935), \emph{Schriftsteller, Journalist}|pw} die Bleisoldaten\pwindex{Goldmann, Paul 31.01.1865 – 25.09.1935@\textsc{Goldmann, Paul} (31.01.1865 – 25.09.1935), \emph{Schriftsteller, Journalist}!Bleisoldaten. Novellette15. 12. 1888@\strich\emph{Bleisoldaten. Novellette} {[}15. 12. 1888{]}|pw} und noch einige
               Kleinigkeiten, von Ihnen Gedichte, »Episode\pwindex{Schnitzler, Arthur 15.05.1862 – 21.10.1931@\textsc{Schnitzler, Arthur} (15.05.1862 – 21.10.1931), \emph{Schriftsteller, Mediziner}!Episode15. 9. 1889@\strich\emph{Episode} {[}15. 9. 1889{]}|pw}« und
                  »Alkandi\pwindex{Schnitzler, Arthur 15.05.1862 – 21.10.1931@\textsc{Schnitzler, Arthur} (15.05.1862 – 21.10.1931), \emph{Schriftsteller, Mediziner}!Alkandi s Lied15.8.1890 – 1.9.1890@\strich\emph{Alkandi’s Lied} {[}15.8.1890 – 1.9.1890{]}|pw}«. Die »Lieder eines Nervösen\pwindex{Lieder eines Nervoesen1889-07-01@\emph{Lieder eines Nervösen} {[}1889-07-01{]}|pw}« kannte ich nicht{[}.{]} sie
               haben mir nie was von ihnen gesagt, und sie stehen auch nicht auf der Höhe der
               anderen. Episode\pwindex{Schnitzler, Arthur 15.05.1862 – 21.10.1931@\textsc{Schnitzler, Arthur} (15.05.1862 – 21.10.1931), \emph{Schriftsteller, Mediziner}!Episode15. 9. 1889@\strich\emph{Episode} {[}15. 9. 1889{]}|pw} ist merkwürdigerweise begriffen
               worden und hat gefallen {\pb}was ich
               zwei Cousins\pwindex{Wolf, Victor Carl 1866 – 10.3.1903@\textsc{Wolf, Victor Carl} (1866 – 10.3.1903), \emph{Unternehmer}|pwv}\pwindex{Wolf, Emil 24.06.1864 – 31.10.1942@\textsc{Wolf, Emil} (24.06.1864 – 31.10.1942), \emph{Rechtsanwalt}|pwv} die
               Publicum waren nicht zugetraut hätte. Alkandi\pwindex{Schnitzler, Arthur 15.05.1862 – 21.10.1931@\textsc{Schnitzler, Arthur} (15.05.1862 – 21.10.1931), \emph{Schriftsteller, Mediziner}!Alkandi s Lied15.8.1890 – 1.9.1890@\strich\emph{Alkandi’s Lied} {[}15.8.1890 – 1.9.1890{]}|pw}
               las ich spät Abends, und als meine Tante\pwindex{Wolf, Charlotte †~8.5.1899@\textsc{Wolf, Charlotte} (†~8.5.1899)|pwv} mich erinnerte daß es spät sei war mein Cousin\pwindex{Wolf, Victor Carl 1866 – 10.3.1903@\textsc{Wolf, Victor Carl} (1866 – 10.3.1903), \emph{Unternehmer}|pwuv}\pwindex{Wolf, Emil 24.06.1864 – 31.10.1942@\textsc{Wolf, Emil} (24.06.1864 – 31.10.1942), \emph{Rechtsanwalt}|pwuv}
               derart wüthend über die Störung daß er einen halben Jahrgang »blaue Donau\pwindex{der schoenen blauen Donau1886-01-15 – 1896@\emph{An der schönen blauen Donau} {[}1886-01-15 – 1896{]}|pw}« zu Boden warf! »\uline{Die Macht
                  der Poesie}«. Wenn Sie glauben ich hätte viel Zeit zum Schreiben irren Sie;
               heute habe ich Kaserninspection und muß hier in der Kaserne sitzen, und übernachten,
               sonst käme ich nicht zum Schreiben. Wenn sie Lust haben schreiben Sie
               Ihrem \spacefill\mbox{Richard}\pend
           \pstart
           30 Mai 91\pend
           \pstart
           \label{T_L00016-1v}\edtext{Daß Sie mir als Adresse}{\lemma{\textnormal{\emph{Daß Sie mir als Adresse}}}\Cendnote{\textnormal{weiter quer am linken Rand}}}\label{T_L00016-1h}{ }\label{K_L00016-2v}\edtext{Giselastrasse\oindex{Ordination Dr. Arthur Schnitzler Giselastrasse 11@\textbf{Ordination Dr. Arthur Schnitzler Giselastraße 11}|pw} und nicht Ring\oindex{Kaerntnerring@\textbf{Kärntnerring}|pw}}{\lemma{\textnormal{\emph{Giselastrasse … Ring}}}\Cendnote{\textnormal{Das Haus\oindex{Kaerntnerring 12/Boesendorferstrasse 11@\textbf{Kärntnerring 12/Bösendorferstraße 11}|pwkv} ist durchgängig und hat zwei 
                     Adressen, 
                     in der Giselastraße\oindex{Boesendorferstrasse@\textbf{Bösendorferstraße}|pwk} (heute: Bösendorferstraße\oindex{Boesendorferstrasse@\textbf{Bösendorferstraße}|pwk}) und
                     am repräsentativeren Kärntnerring\oindex{Kaerntnerring@\textbf{Kärntnerring}|pwk}.}}}\label{K_L00016-2h} angaben ist sehr
                  hübsch von Ihnen; ich danke. Mein Brief \label{T_L00016-2v}\edtext{und »Sie« werden sich auf der Stiege treffen.}{\lemma{\textnormal{\emph{und … treffen.}}}\Cendnote{\textnormal{am oberen Rand auf dem Kopf}}}\label{T_L00016-2h}\pend
           
         
         \endnumbering\mylabel{h}\end{ledgroupsized}  \newcommand{\dateiname}{L00016}\newcommand{\titel}{Richard Beer-Hofmann an Arthur Schnitzler, 30. 5. 1891}\newcommand{\editorInnen}{Martin Anton Müller und Gerd-Hermann Susen}%% latex-leseansicht-abspann.tex
%% Abspann für die Leseansicht.
%% Der Schalter \ifkorrekturansicht ist bereits durch den Vorspann gesetzt.

%% latex-abspann.tex
%% Gemeinsamer Abspann für Korrekturansicht und Leseansicht.
%% Setzt den Schalter \ifkorrekturansicht voraus (gesetzt in den
%% einbindenden Dateien latex-korrekturansicht-abspann.tex bzw.
%% latex-leseansicht-abspann.tex).
%% ---------------------------------------------------------------

\normalsize

% Das esempio-Environment wird nur in der Leseansicht benötigt
\ifkorrekturansicht\else
\newenvironment{esempio}[3]%
{
    \vspace{1.5ex}
    \rlap{\underline{#1}}
    \par
    \setlength{\parindent}{0cm}
    \nopagebreak
    \leftskip=#2cm
    \rightskip=#3cm
}
{
    \par
}
\fi

\doendnotes{C}
\bigskip
\vfill

\clearpage

\footnotesize

\ifkorrekturansicht
  \lohead{\textsc{register}}
\fi

% theindex-Environment neu definieren ohne reledmac
\makeatletter
\renewenvironment{theindex}{%
  \ifkorrekturansicht
    \section*{\indexname}%
  \else
    \subsubsection*{Index der erwähnten Entitäten}%
  \fi
  \setlength{\parindent}{0pt}%
  \setlength{\parskip}{0pt plus 0.3pt}%
  \let\item\@idxitem
}{%
  \ifkorrekturansicht\clearpage\fi
}
\makeatother

\IfFileExists{\jobname-pw.ind}{\input{\jobname-pw.ind}}{}

% Quellenangabe nur in der Leseansicht
\ifkorrekturansicht\else
% Fallback-Definitionen, falls die .tex-Datei \titel etc. nicht gesetzt hat
\providecommand{\titel}{}
\providecommand{\editorInnen}{}
\providecommand{\dateiname}{\jobname}

\vspace{3cm}

\vfill

\footnotesize
\textsc{Quelle}: \titel. Herausgegeben von {\editorInnen}. In: \emph{Arthur Schnitzler: Briefwechsel mit Autorinnen und Autoren}.
 Digitale Edition, https://schnitzler-briefe.acdh.oeaw.ac.at/{\dateiname}.html (Stand \today)
\fi

\end{document}


      