%% latex-korrekturansicht-vorspann.tex
%% Vorspann für die Korrekturansicht.
%% Lädt die gemeinsame Datei latex-vorspann.tex mit gesetztem Schalter.

\newif\ifkorrekturansicht
\korrekturansichttrue

\input{../tex-inputs/latex-vorspann}


\section[Richard Beer-Hofmann an Arthur Schnitzler, 30. 5. 1891]{L00016 Richard Beer-Hofmann an Arthur Schnitzler,30. 5. 1891}
\nopagebreak\mylabel{L00016v}
\rehead{ }\normalsize\beginnumbering\briefempfaengerindex{Schnitzler, Arthur@\textsc{Schnitzler, Arthur}!zzzBeer-Hofmann, Richard@\emph{von Richard Beer-Hofmann}!1891-05-301@{30. 5. 1891}|(be}
\toendnotes[C]{\smallbreak\pagebreak[2]}\Standort{CUL, Schnitzler, B 8.}
\physDesc{Briefkarte, 1195 Zeichen
\newline{}Handschrift: schwarze Tinte, lateinische Kurrent
\newline{}Schnitzler: mit Bleistift datiert: »30/5 91« und nummeriert: »2.« }
\buchAbdrucke{\weitereDrucke{Arthur Schnitzler, Richard Beer-Hofmann: \emph{Briefwechsel 1891–1931}. Wien, Zürich: \emph{Europaverlag} 1992, S. 30.} }\toendnotes[C]{\smallbreak}
\pstart\center{}{\pb}Lieber Arthur!\pend\vspace{0.5em}
\pstart
           Denken Sie mein Cousin\pwindex{Wolf, Victor Carl 1866 – 10.\,3.\,1903 Wien@\textsc{Wolf, Victor Carl} (1866 – 10.\,3.\,1903 Wien), \emph{Unternehmer}|pwuv}\pwindex{Wolf, Emil 24.\,6.\,1864 Bruenn – 31.\,10.\,1942 Konzentrationslager Theresienstadt@\textsc{Wolf, Emil} (24.\,6.\,1864 Brünn – 31.\,10.\,1942 Konzentrationslager Theresienstadt), \emph{Rechtsanwalt}|pwuv} hat auf mein Anrathen \label{K_L00016-1v}\edtext{die alten Jahrgänge}{\lemma{\textnormal{\emph{die alten Jahrgänge}}}\Cendnote{\textnormal{\emph{An der schönen blauen Donau}\pwindex{der schoenen blauen Donau@\emph{An der schönen blauen Donau}|pwk}, ein
                  »Unterhaltungsblatt für die Familie«, erschien seit dem 15. 1. 1886
                  alle 14 Tage. Die von Beer-Hofmann\pwindex{Beer-Hofmann, Richard 11.\,7.\,1866 Wien – 26.\,9.\,1945 New York City@\textsc{Beer-Hofmann, Richard} (11.\,7.\,1866 Wien – 26.\,9.\,1945 New York City), \emph{Schriftsteller}|pwk}
                  angesprochenen Texte finden sich in den Jahrgängen 1888 bis 1890.}}}\label{K_L00016-1} der »blauen Donau\pwindex{der schoenen blauen Donau@\emph{An der schönen blauen Donau}|pw}« gekauft und an Sonntag
               Nachmittagen, wenn ich frei bin lese ich Einzelnes daraus vor; Philisterpublikum zum
               größten Theil aber Publikum. Loris\pwindex{Hofmannsthal, Hugo von 1.\,2.\,1874 Wien – 15.\,7.\,1929 Rodaun@\textsc{Hofmannsthal, Hugo von} (1.\,2.\,1874 Wien – 15.\,7.\,1929 Rodaun), \emph{Schriftsteller}|pw}{ }Gedichte, von Paul\pwindex{Goldmann, Paul 31.\,1.\,1865 Breslau – 25.\,9.\,1935 Wien@\textsc{Goldmann, Paul} (31.\,1.\,1865 Breslau – 25.\,9.\,1935 Wien), \emph{Schriftsteller, Journalist}|pw} die BleisoldatenSEXref\pwindex{Goldmann, Paul 31.\,1.\,1865 Breslau – 25.\,9.\,1935 Wien@\textsc{Goldmann, Paul} (31.\,1.\,1865 Breslau – 25.\,9.\,1935 Wien), \emph{Schriftsteller, Journalist}!Bleisoldaten. Novellette@\strich\emph{Bleisoldaten. Novellette}|pw} und noch einige
               Kleinigkeiten, von Ihnen Gedichte, »EpisodeSEXref\pwindex{Schnitzler, Arthur 15.\,5.\,1862 Wien – 21.\,10.\,1931 ebd.@\textsc{Schnitzler, Arthur} (15.\,5.\,1862 Wien – 21.\,10.\,1931 ebd.), \emph{Schriftsteller*in, Mediziner*in}!Episode@\strich\emph{Episode}|pw}« und
                  »AlkandiSEXref\pwindex{Schnitzler, Arthur 15.\,5.\,1862 Wien – 21.\,10.\,1931 ebd.@\textsc{Schnitzler, Arthur} (15.\,5.\,1862 Wien – 21.\,10.\,1931 ebd.), \emph{Schriftsteller*in, Mediziner*in}!Alkandi s Lied@\strich\emph{Alkandi’s Lied}|pw}«. Die »Lieder eines NervösenSEXref\pwindex{Schnitzler, Arthur 15.\,5.\,1862 Wien – 21.\,10.\,1931 ebd.@\textsc{Schnitzler, Arthur} (15.\,5.\,1862 Wien – 21.\,10.\,1931 ebd.), \emph{Schriftsteller*in, Mediziner*in}!Lieder eines Nervoesen@\strich\emph{Lieder eines Nervösen}|pw}« kannte ich nicht{[}.{]} sie
               haben mir nie was von ihnen gesagt, und sie stehen auch nicht auf der Höhe der
               anderen. EpisodeSEXref\pwindex{Schnitzler, Arthur 15.\,5.\,1862 Wien – 21.\,10.\,1931 ebd.@\textsc{Schnitzler, Arthur} (15.\,5.\,1862 Wien – 21.\,10.\,1931 ebd.), \emph{Schriftsteller*in, Mediziner*in}!Episode@\strich\emph{Episode}|pw} ist merkwürdigerweise begriffen
               worden und hat gefallen {\pb}was ich
               zwei Cousins\pwindex{Wolf, Victor Carl 1866 – 10.\,3.\,1903 Wien@\textsc{Wolf, Victor Carl} (1866 – 10.\,3.\,1903 Wien), \emph{Unternehmer}|pwv}\pwindex{Wolf, Emil 24.\,6.\,1864 Bruenn – 31.\,10.\,1942 Konzentrationslager Theresienstadt@\textsc{Wolf, Emil} (24.\,6.\,1864 Brünn – 31.\,10.\,1942 Konzentrationslager Theresienstadt), \emph{Rechtsanwalt}|pwv} die
               Publicum waren nicht zugetraut hätte. AlkandiSEXref\pwindex{Schnitzler, Arthur 15.\,5.\,1862 Wien – 21.\,10.\,1931 ebd.@\textsc{Schnitzler, Arthur} (15.\,5.\,1862 Wien – 21.\,10.\,1931 ebd.), \emph{Schriftsteller*in, Mediziner*in}!Alkandi s Lied@\strich\emph{Alkandi’s Lied}|pw}
               las ich spät Abends, und als meine Tante\pwindex{Wolf, Charlotte †~8.\,5.\,1899 Wien@\textsc{Wolf, Charlotte} (†~8.\,5.\,1899 Wien)|pwv} mich erinnerte daß es spät sei war mein Cousin\pwindex{Wolf, Victor Carl 1866 – 10.\,3.\,1903 Wien@\textsc{Wolf, Victor Carl} (1866 – 10.\,3.\,1903 Wien), \emph{Unternehmer}|pwuv}\pwindex{Wolf, Emil 24.\,6.\,1864 Bruenn – 31.\,10.\,1942 Konzentrationslager Theresienstadt@\textsc{Wolf, Emil} (24.\,6.\,1864 Brünn – 31.\,10.\,1942 Konzentrationslager Theresienstadt), \emph{Rechtsanwalt}|pwuv}
               derart wüthend über die Störung daß er einen halben Jahrgang »blaue Donau\pwindex{der schoenen blauen Donau@\emph{An der schönen blauen Donau}|pw}« zu Boden warf! »\uline{Die Macht
                  der Poesie}«. Wenn Sie glauben ich hätte viel Zeit zum Schreiben irren Sie;
               heute habe ich Kaserninspection und muß hier in der Kaserne sitzen, und übernachten,
               sonst käme ich nicht zum Schreiben. Wenn sie Lust haben schreiben Sie
               Ihrem \spacefill\mbox{Richard}\pend
           
\pstart
           30 Mai 91\pend
           
\pstart
           \label{T_L00016-1v}\edtext{Daß Sie mir als Adresse}{\lemma{\textnormal{\emph{Daß Sie mir als Adresse}}}\Cendnote{\textnormal{weiter quer am linken Rand}}}\label{T_L00016-1}{ }\label{K_L00016-2v}\edtext{Giselastrasse\oindex{Ordination Arthur Schnitzler [Boesendorferstrasse 11]@\textbf{Ordination Arthur Schnitzler [Bösendorferstraße 11]}|pw} und nicht Ring\oindex{Kaerntnerring@\textbf{Kärntnerring}|pw}}{\lemma{\textnormal{\emph{Giselastrasse … Ring}}}\Cendnote{\textnormal{Das Haus\oindex{Kaerntnerring 12/Boesendorferstrasse 11@\textbf{Kärntnerring 12/Bösendorferstraße 11}|pwkv} ist durchgängig und hat zwei 
                     Adressen, 
                     in der Giselastraße\oindex{Boesendorferstrasse@\textbf{Bösendorferstraße}|pwk} (heute: Bösendorferstraße\oindex{Boesendorferstrasse@\textbf{Bösendorferstraße}|pwk}) und
                     am repräsentativeren Kärntnerring\oindex{Kaerntnerring@\textbf{Kärntnerring}|pwk}.}}}\label{K_L00016-2} angaben ist sehr
                  hübsch von Ihnen; ich danke. Mein Brief \label{T_L00016-2v}\edtext{und »Sie« werden sich auf der Stiege treffen.}{\lemma{\textnormal{\emph{und … treffen.}}}\Cendnote{\textnormal{am oberen Rand auf dem Kopf}}}\label{T_L00016-2}\pend
           \selectlanguage{ngerman}\endnumbering\briefempfaengerindex{Schnitzler, Arthur@\textsc{Schnitzler, Arthur}!zzzBeer-Hofmann, Richard@\emph{von Richard Beer-Hofmann}!1891-05-301@{30. 5. 1891}|)be}\mylabel{L00016h}  \normalsize

\doendnotes{C}
\bigskip
\vfill

\clearpage

\footnotesize

\lohead{\textsc{register}}

% Definiere theindex-Environment komplett neu ohne reledmac
\makeatletter
\renewenvironment{theindex}{%
  \section*{\indexname}%
  \setlength{\parindent}{0pt}%
  \setlength{\parskip}{0pt plus 0.3pt}%
  \let\item\@idxitem
}{%
  \clearpage
}
\makeatother

\IfFileExists{\jobname-pw.ind}{\input{\jobname-pw.ind}}{}

\end{document}

      