\input{../tex-inputs/latex-pdf-vorspann}
\begin{center}
            \textcolor{red}{ENTWURF. ENTZIFFERUNG NOCH NICHT KORREKTURGELESEN}
                      \end{center}
            
               \section[Richard Beer-Hofmann an Arthur Schnitzler, 30. 5. 1891]{ Richard Beer-Hofmann an Arthur Schnitzler, 30. 5. 1891}\nopagebreak\mylabel{v}\rehead{ }\begin{ledgroupsized}[t]{13cm}\normalsize\beginnumbering\briefempfaengerindex{Schnitzler, Arthur@\textsc{Schnitzler, Arthur}!zzzBeer-Hofmann, Richard@\emph{von Richard Beer-Hofmann}!1891-05-301@{30. 5. 1891}|(be} \toendnotes[C]{\smallbreak\pagebreak[2]} \Standort{CUL, Schnitzler, B 8.}
\physDesc{Briefkarte
\newline{}Handschrift: schwarze Tinte, lateinische Kurrent
\newline{}Schnitzler: mit Bleistift datiert: »30/5 91« und nummeriert: »2.« }\buchAbdrucke{\weitereDrucke{Arthur Schnitzler, Richard Beer-Hofmann: \emph{Briefwechsel 1891–1931}. Hg. Konstanze Fliedl. Wien, Zürich: \emph{Europaverlag} 1992, S. 30.} }\toendnotes[C]{\smallbreak}\pstart\center{}{\pb}Lieber Arthur!\pend\pstart
           Denken Sie mein Cousin\pwindex{Wolf, Victor Carl 1866 – 10.3.1903@\textsc{Wolf, Victor Carl} (1866 – 10.3.1903), \emph{Unternehmer}|pwuv}\pwindex{Wolf, Emil 24.06.1864 – 31.10.1942@\textsc{Wolf, Emil} (24.06.1864 – 31.10.1942), \emph{Rechtsanwalt}|pwuv} hat auf mein Anrathen \label{K_L00016_1v}\edtext{die alten Jahrgänge}{\lemma{\textnormal{\emph{die alten Jahrgänge}}}\Cendnote{\textnormal{\emph{An der schönen blauen Donau}\pwindex{der schoenen blauen Donau1886 – 1896@\emph{An der schönen blauen Donau}|pwk}, ein
                  »Unterhaltungsblatt für die Familie«, erschien seit dem 15. 1. 1886
                  alle 14 Tage. Die von Beer-Hofmann\pwindex{Beer-Hofmann, Richard 11.07.1866 – 26.09.1945@\textsc{Beer-Hofmann, Richard} (11.07.1866 – 26.09.1945), \emph{Schriftsteller}|pwk}
                  angesprochenen Texte finden sich in den Jahrgängen 1888 bis 1890.}}}\label{K_L00016_1h} der »blauen Donau\pwindex{der schoenen blauen Donau1886 – 1896@\emph{An der schönen blauen Donau}|pw}« gekauft und an Sonntag Nachmittagen,
               wenn ich frei bin lese ich Einzelnes daraus vor; Philisterpublikum zum größten Theil
               aber Publikum. Loris\pwindex{Hofmannsthal, Hugo von 01.02.1874 – 15.07.1929@\textsc{Hofmannsthal, Hugo von} (01.02.1874 – 15.07.1929), \emph{Schriftsteller}|pw}{ }Gedichte, von Paul\pwindex{Goldmann, Paul 31.01.1865 – 25.09.1935@\textsc{Goldmann, Paul} (31.01.1865 – 25.09.1935), \emph{Schriftsteller, Journalist}|pw} die Bleisoldaten\pwindex{Goldmann, Paul 31.01.1865 – 25.09.1935@\textsc{Goldmann, Paul} (31.01.1865 – 25.09.1935), \emph{Schriftsteller, Journalist}!Bleisoldaten. Novellette15.12.1888 – 15.12.1888@\strich\emph{Bleisoldaten. Novellette} {[}15.12.1888 – 15.12.1888{]}|pw} und noch einige
               Kleinigkeiten, von Ihnen Gedichte, »Episode\pwindex{Schnitzler, Arthur 15.05.1862 – 21.10.1931@\textsc{Schnitzler, Arthur} (15.05.1862 – 21.10.1931), \emph{Schriftsteller, Mediziner}!Episode8. 09. 1889@\strich\emph{Episode} {[}8. 09. 1889{]}|pw}« und
                  »Alkandi\pwindex{Schnitzler, Arthur 15.05.1862 – 21.10.1931@\textsc{Schnitzler, Arthur} (15.05.1862 – 21.10.1931), \emph{Schriftsteller, Mediziner}!Alkandi s Lied15.8.1890 – 1.9.1890@\strich\emph{Alkandi’s Lied} {[}15.8.1890 – 1.9.1890{]}|pw}«. Die »Lieder eines Nervösen\pwindex{Lieder eines Nervoesen1889-07-01@\emph{Lieder eines Nervösen} {[}1889-07-01{]}|pw}« kannte ich nicht{[}.{]} sie haben mir
               nie was von ihnen gesagt, und sie stehen auch nicht auf der Höhe der anderen. Episode\pwindex{Schnitzler, Arthur 15.05.1862 – 21.10.1931@\textsc{Schnitzler, Arthur} (15.05.1862 – 21.10.1931), \emph{Schriftsteller, Mediziner}!Episode8. 09. 1889@\strich\emph{Episode} {[}8. 09. 1889{]}|pw} ist merkwürdigerweise begriffen worden und
               hat gefallen {\pb}was ich zwei Cousins\pwindex{Wolf, Victor Carl 1866 – 10.3.1903@\textsc{Wolf, Victor Carl} (1866 – 10.3.1903), \emph{Unternehmer}|pwv}\pwindex{Wolf, Emil 24.06.1864 – 31.10.1942@\textsc{Wolf, Emil} (24.06.1864 – 31.10.1942), \emph{Rechtsanwalt}|pwv} die Publicum
               waren nicht zugetraut hätte. Alkandi\pwindex{Schnitzler, Arthur 15.05.1862 – 21.10.1931@\textsc{Schnitzler, Arthur} (15.05.1862 – 21.10.1931), \emph{Schriftsteller, Mediziner}!Alkandi s Lied15.8.1890 – 1.9.1890@\strich\emph{Alkandi’s Lied} {[}15.8.1890 – 1.9.1890{]}|pw} las ich spät
               Abends, und als meine Tante\pwindex{Wolf, Charlotte †~8.5.1899@\textsc{Wolf, Charlotte} (†~8.5.1899)|pwv}
               mich erinnerte daß es spät sei war mein Cousin\pwindex{Wolf, Victor Carl 1866 – 10.3.1903@\textsc{Wolf, Victor Carl} (1866 – 10.3.1903), \emph{Unternehmer}|pwuv}\pwindex{Wolf, Emil 24.06.1864 – 31.10.1942@\textsc{Wolf, Emil} (24.06.1864 – 31.10.1942), \emph{Rechtsanwalt}|pwuv} derart wüthend über die Störung daß er
               einen halben Jahrgang »blaue Donau\pwindex{der schoenen blauen Donau1886 – 1896@\emph{An der schönen blauen Donau}|pw}« zu Boden warf!
                  »\uline{Die Macht der Poesie}«. Wenn Sie glauben ich hätte
               viel Zeit zum Schreiben irren Sie; heute habe ich Kaserninspection und muß hier in
               der Kaserne sitzen, und übernachten, sonst käme ich nicht zum Schreiben. Wenn sie
               Lust haben schreiben Sie Ihrem \spacefill\mbox{Richard}\pend
           \pstart
           30 Mai 91\pend
           \pstart
           \label{T_L00016_1v}\edtext{Daß Sie mir als Adresse}{\lemma{\textnormal{\emph{Daß Sie mir als Adresse}}}\Cendnote{\textnormal{weiter quer am linken Rand}}}\label{T_L00016_1h}{ }\label{K_L00016_2v}\edtext{Giselastrasse\oindex{Boesendorferstrasse@\textbf{Bösendorferstraße}|pw} und nicht Ring\oindex{Burgring@\textbf{Burgring}|pw}}{\lemma{\textnormal{\emph{Giselastrasse … Ring}}}\Cendnote{\textnormal{Das Haus hatte zwei Eingänge, wobei
                     die letztere Adresse die repräsentativere darstellt.}}}\label{K_L00016_2h} angaben ist sehr
                  hübsch von Ihnen; ich danke. Mein Brief \label{T_L00016_2v}\edtext{und »Sie« werden sich auf der Stiege treffen.}{\lemma{\textnormal{\emph{und … treffen.}}}\Cendnote{\textnormal{am oberen Rand auf dem Kopf}}}\label{T_L00016_2h}\pend
           \endnumbering\briefempfaengerindex{Schnitzler, Arthur@\textsc{Schnitzler, Arthur}!zzzBeer-Hofmann, Richard@\emph{von Richard Beer-Hofmann}!1891-05-301@{30. 5. 1891}|)be}\mylabel{h}\end{ledgroupsized}  \newcommand{\dateiname}{L00016}\newcommand{\titel}{Richard Beer-Hofmann an Arthur Schnitzler, 30. 5. 1891}\newcommand{\editorInnen}{Martin Anton Müller und Gerd-Hermann Susen}\input{../tex-inputs/latex-pdf-abspann}
      