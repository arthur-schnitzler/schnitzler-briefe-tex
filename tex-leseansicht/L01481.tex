%% latex-korrekturansicht-vorspann.tex
%% Vorspann für die Korrekturansicht.
%% Lädt die gemeinsame Datei latex-vorspann.tex mit gesetztem Schalter.

\newif\ifkorrekturansicht
\korrekturansichttrue

\input{../tex-inputs/latex-vorspann}


\section[Arthur Schnitzler an Richard Beer-Hofmann, 23. 12. 1904]{L01481 Arthur Schnitzler an Richard Beer-Hofmann, 23. 12. 1904}
\nopagebreak\mylabel{L01481v}
\rehead{ }\normalsize\beginnumbering\briefempfaengerindex{Beer-Hofmann, Richard@\textsc{Beer-Hofmann, Richard}!zzzSchnitzler, Arthur@\emph{von Arthur Schnitzler}!1904-12-231@{23. 12. 1904}|(be}
\toendnotes[C]{\smallbreak\pagebreak[2]}\Standort{YCGL, MSS 31.}
\physDesc{Telegramm, 255 Zeichen
\newline{}maschinell
\newline{}Versand: 1) Stempel: »\nobreak{}\oindex{Berlin@\textbf{Berlin}, \emph{P.PPLC}|pwk}Berlin N. W. 6, 23. 12. 04, 11–V\nobreak{}«.   2) »\textcolor{gray}{\textbf{\textbf{Aufgenommen} von}}{ }W{ }\textcolor{gray}{\textbf{den}}{ }23\textcolor{gray}{\textbf{/}}12{ }\textcolor{gray}{\textbf{um}}{ }10 \textcolor{gray}{\textbf{Uhr}} 30 \textcolor{gray}{\textbf{M.}}n{ }\textcolor{gray}{\textbf{durch}}{ }\textcolor{gray}{Hw}«}
\buchAbdrucke{\weitereDrucke{Arthur Schnitzler, Richard Beer-Hofmann: \emph{Briefwechsel 1891–1931}. Wien, Zürich: \emph{Europaverlag} 1992, S. 171.} }\toendnotes[C]{\smallbreak}\pstart{}{\pb}richard beerhofmann berlin\pend{}\pstart{}neues theater.=\oindex{Neues Theater@\textbf{Neues Theater}, \emph{Theater (K.THE)}|pw}\pend{}{\bigskip}\vspace{1em}
\pstart
           {\pb}\textcolor{gray}{\textbf{Telegramm aus}} de wien\oindex{Wien@\textbf{Wien}, \emph{A.ADM2}|pw}
                  lll.-580 3l 239 40–m= \pend
           \vspace{0.5em}
\pstart
           dieser wunsch sei meinem freund geweiht dass in seinem sehr geliebten werke\pwindex{Graf von Charolais. Ein Trauerspiel@\emph{Der Graf von Charolais. Ein Trauerspiel}|pwv} jeder alle weichheit alle staerke
               einer ungebrochenen menschlichkeit keiner den beruehmten \label{K_L01481-1v}\edtext{bruch}{\lemma{\textnormal{\emph{bruch}}}\Cendnote{\textnormal{Zwischen dem dritten
                  und vierten Akt sind Psychologie und Motivierung der Figuren nicht völlig stringent,
                  was auch von der Kritik wahrgenommen wurde.}}}\label{K_L01481-1} bemerke =
                  \spacefill\mbox{= arthur +}\pend
           \selectlanguage{ngerman}\endnumbering\briefempfaengerindex{Beer-Hofmann, Richard@\textsc{Beer-Hofmann, Richard}!zzzSchnitzler, Arthur@\emph{von Arthur Schnitzler}!1904-12-231@{23. 12. 1904}|)be}\mylabel{L01481h}  \normalsize

\doendnotes{C}
\bigskip
\vfill

\clearpage

\footnotesize

\lohead{\textsc{register}}

% Definiere theindex-Environment komplett neu ohne reledmac
\makeatletter
\renewenvironment{theindex}{%
  \section*{\indexname}%
  \setlength{\parindent}{0pt}%
  \setlength{\parskip}{0pt plus 0.3pt}%
  \let\item\@idxitem
}{%
  \clearpage
}
\makeatother

\IfFileExists{\jobname-pw.ind}{\input{\jobname-pw.ind}}{}

\end{document}

      