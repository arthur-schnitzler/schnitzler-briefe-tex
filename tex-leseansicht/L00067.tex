%% latex-korrekturansicht-vorspann.tex
%% Vorspann für die Korrekturansicht.
%% Lädt die gemeinsame Datei latex-vorspann.tex mit gesetztem Schalter.

\newif\ifkorrekturansicht
\korrekturansichttrue

\input{../tex-inputs/latex-vorspann}


\section[Hugo von Hofmannsthal an Arthur Schnitzler, {[}3.? 2. 1892{]}]{L00067 Hugo von Hofmannsthal an Arthur Schnitzler, {[}3.? 2. 1892{]}}
\nopagebreak\mylabel{L00067v}
\rehead{ }\normalsize\beginnumbering\briefempfaengerindex{Schnitzler, Arthur@\textsc{Schnitzler, Arthur}!zzzHofmannsthal, Hugo von@\emph{von Hugo von Hofmannsthal}!1892-02-031@{{[}3.? 2. 1892{]}}|(be}
\toendnotes[C]{\smallbreak\pagebreak[2]}\Standort{CUL, Schnitzler, B 43.}
\physDesc{Brief, 1 Blatt, 1 Seite, 141 Zeichen
\newline{}Handschrift: schwarze Tinte, deutsche Kurrent
\newline{}Schnitzler: mit Bleistift nummeriert: »14« }
\buchAbdrucke{\weitereDrucke{1) Hugo von Hofmannsthal, Arthur Schnitzler: \emph{Briefwechsel}. Frankfurt am Main: \emph{S. Fischer} 1964, S. 15.} \weitereDrucke{2) Hermann Bahr, Arthur Schnitzler: \emph{Briefwechsel, Aufzeichnungen, Dokumente (1891–1931)}. Göttingen: \emph{Wallstein} 2018, S. 21.} }\toendnotes[C]{\smallbreak}
\pstart{}{\pb}Lieber Freund.\pend\vspace{0.5em}
\pstart
           Ich bitte um die \label{K_L00067-1v}\edtext{geſtern}{\lemma{\textnormal{\emph{geſtern}}}\Cendnote{\textnormal{Vgl. A. S.: \emph{Tagebuch}, 31. 1. 1892. Gegen die
                  Datierung spricht, dass am 2. 2. 1892 noch ein Treffen stattfindet, das hier nicht thematisiert
                  wird.}}}\label{K_L00067-1} vergeſſenen \label{K_L00067-2v}\edtext{\textsc{Aveugles}\pwindex{Blinden@\emph{Die Blinden}|pw}}{\lemma{\textnormal{\emph{Aveugles}}}\Cendnote{\textnormal{In der Folge übersetzte Hofmannsthal\pwindex{Hofmannsthal, Hugo von 1874-02-01 – 1929-07-15@\textsc{Hofmannsthal, Hugo von} (1874-02-01 – 1929-07-15), \emph{Schriftsteller/Schriftstellerin}|pwk} ausschließlich diesen Einakter
                  von Maeterlinck\pwindex{Maeterlinck, Maurice 29.08.1862 – 06.05.1949@\textsc{Maeterlinck, Maurice} (29.08.1862 – 06.05.1949), \emph{Schriftsteller/Schriftstellerin}|pwk} (vgl. Brief an Marie Herzfeld\pwindex{Herzfeld, Marie 20.03.1855 – 22.09.1940@\textsc{Herzfeld, Marie} (20.03.1855 – 22.09.1940), \emph{Schriftsteller/Schriftstellerin, Übersetzer/Übersetzerin}|pwk}, 9. 3. 1892,
                     in: Hugo von Hofmannsthal\pwindex{Hofmannsthal, Hugo von 1874-02-01 – 1929-07-15@\textsc{Hofmannsthal, Hugo von} (1874-02-01 – 1929-07-15), \emph{Schriftsteller/Schriftstellerin}|pwk}: \emph{Briefe an Marie Herzfeld}. Herausgegeben von Horst Weber. Heidelberg:
                        \emph{Lothar Stiehm}{ }1967, S. 23).}}}\label{K_L00067-2}{ }\textsc{Bérénice}\pwindex{Garten der Berenice@\emph{Der Garten der Bérenice}|pw} u. \textsc{Sept Princesses}\pwindex{sieben Prinzessinnen@\emph{Die sieben Prinzessinnen}|pw}.\pend
           
\pstart
           Es bleibt bei Sonntag?\pend
           \pstart \spacefill\mbox{Loris.}\pend{}
\pstart
           \noindent{}Die \label{K_L00067-3v}\edtext{Überwindung\pwindex{Ueberwindung des Naturalismus. Als zweite Reihe von »Die Kritik der Moderne«@\emph{Die Überwindung des Naturalismus. Als zweite Reihe von »Die Kritik der Moderne«}|pw}}{\lemma{\textnormal{\emph{Überwindung}}}\Cendnote{\textnormal{Wohl wegen des Artikels \emph{Maurice Maeterlinck}\pwindex{Maurice Maeterlinck@\emph{Maurice Maeterlinck}|pwk}. In: Hermann Bahr\pwindex{Bahr, Hermann 19.07.1863 – 15.01.1934@\textsc{Bahr, Hermann} (19.07.1863 – 15.01.1934), \emph{Schriftsteller/Schriftstellerin, Kritiker/Kritikerin}|pwk}: \emph{Die Überwindung des Naturalismus}\pwindex{Ueberwindung des Naturalismus. Als zweite Reihe von »Die Kritik der Moderne«@\emph{Die Überwindung des Naturalismus. Als zweite Reihe von »Die Kritik der Moderne«}|pwk}. Dresden, Leipzig:
                           \emph{E. Pierson}\orgindex{E. Pierson s Verlag@E. Pierson’s Verlag|pwk}{ }1891, S. 189–198 (Als zweite Reihe von »Zur Kritik der
                        Moderne«). Erstdruck: \emph{Magazin
                           für Litteratur}\pwindex{Magazin fuer die Literatur des Auslandes@\emph{Magazin für die Literatur des Auslandes}|pwk}, Jg. 60, Nr. 2, 10. 1. 1891,
                        S. 25–27.}}}\label{K_L00067-3} habe ich zuhauſe\pend
           \selectlanguage{ngerman}\endnumbering\briefempfaengerindex{Schnitzler, Arthur@\textsc{Schnitzler, Arthur}!zzzHofmannsthal, Hugo von@\emph{von Hugo von Hofmannsthal}!1892-02-031@{{[}3.? 2. 1892{]}}|)be}\mylabel{L00067h}  \normalsize

\doendnotes{C}
\bigskip
\vfill

\clearpage

\footnotesize

\lohead{\textsc{register}}

% Definiere theindex-Environment komplett neu ohne reledmac
\makeatletter
\renewenvironment{theindex}{%
  \section*{\indexname}%
  \setlength{\parindent}{0pt}%
  \setlength{\parskip}{0pt plus 0.3pt}%
  \let\item\@idxitem
}{%
  \clearpage
}
\makeatother

\IfFileExists{\jobname-pw.ind}{\input{\jobname-pw.ind}}{}

\end{document}

      