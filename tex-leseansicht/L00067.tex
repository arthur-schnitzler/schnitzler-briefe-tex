%% latex-leseansicht-vorspann.tex
%% Vorspann für die Leseansicht.
%% Lädt die gemeinsame Datei latex-vorspann.tex mit nicht gesetztem Schalter.

\newif\ifkorrekturansicht
\korrekturansichtfalse

\input{../tex-inputs/latex-vorspann}

\begin{center}
            \textcolor{red}{ENTWURF. ENTZIFFERUNG NOCH NICHT KORREKTURGELESEN}
                      \end{center}
            
               \section[Hugo von Hofmannsthal an Arthur Schnitzler, {[}3.? 2. 1892{]}]{ Hugo von Hofmannsthal an Arthur Schnitzler, {[}3.? 2. 1892{]}}\nopagebreak\mylabel{v}\rehead{ }\begin{ledgroupsized}[t]{13cm}\normalsize\beginnumbering\briefempfaengerindex{Schnitzler, Arthur@\textsc{Schnitzler, Arthur}!zzzHofmannsthal, Hugo von@\emph{von Hugo von Hofmannsthal}!1892-02-031@{{[}3.? 2. 1892{]}}|(be} \toendnotes[C]{\smallbreak\pagebreak[2]} \Standort{CUL, Schnitzler, B 43.}
\physDesc{Brief, 1 Blatt, 1 Seite
\newline{}Handschrift: schwarze Tinte, deutsche Kurrent
\newline{}Schnitzler: mit Bleistift nummeriert: »14« }\buchAbdrucke{\weitereDrucke{1) Hugo von Hofmannsthal, Arthur Schnitzler: \emph{Briefwechsel}. Hg. Therese Nickl und Heinrich Schnitzler. Frankfurt am Main: \emph{S. Fischer} 1964, S. 15.} \weitereDrucke{2) Hermann Bahr, Arthur Schnitzler: \emph{Briefwechsel, Aufzeichnungen, Dokumente (1891–1931)}. Hg. Kurt Ifkovits und Martin Anton Müller. Göttingen: \emph{Wallstein} 2018, S. 21.} }\toendnotes[C]{\smallbreak}\pstart{}{\pb}Lieber Freund.\pend\pstart
           Ich bitte um die \label{K_L00067_1v}\edtext{geſtern}{\lemma{\textnormal{\emph{geſtern}}}\Cendnote{\textnormal{vgl. A. S.: \emph{Tagebuch}, 31. 1. 1892. Gegen die
                  Datierung spricht, dass am 2. 2. 1892 noch ein Treffen stattfindet, das hier nicht thematisiert
                  wird.}}}\label{K_L00067_1h} vergeſſenen \label{K_L00067_2v}\edtext{\textsc{Aveugles}\pwindex{Maeterlinck, Maurice 29.08.1862 – 06.05.1949@\textsc{Maeterlinck, Maurice} (29.08.1862 – 06.05.1949), \emph{Schriftsteller}!Blinden1891@\strich\emph{Die Blinden} {[}1891{]}|pw}}{\lemma{\textnormal{\emph{Aveugles}}}\Cendnote{\textnormal{In der Folge übersetzte Hofmannsthal\pwindex{Hofmannsthal, Hugo von 01.02.1874 – 15.07.1929@\textsc{Hofmannsthal, Hugo von} (01.02.1874 – 15.07.1929), \emph{Schriftsteller}|pwk} ausschließlich diesen Einakter von Maeterlinck\pwindex{Maeterlinck, Maurice 29.08.1862 – 06.05.1949@\textsc{Maeterlinck, Maurice} (29.08.1862 – 06.05.1949), \emph{Schriftsteller}|pwk} (vgl. Brief an Marie Herzfeld\pwindex{Herzfeld, Marie 20.03.1855 – 22.09.1940@\textsc{Herzfeld, Marie} (20.03.1855 – 22.09.1940), \emph{Schriftstellerin, Übersetzerin}|pwk}, 9. 3. 1892, in:
                        Hugo von Hofmannsthal\pwindex{Hofmannsthal, Hugo von 01.02.1874 – 15.07.1929@\textsc{Hofmannsthal, Hugo von} (01.02.1874 – 15.07.1929), \emph{Schriftsteller}|pwk}: \emph{Briefe an Marie Herzfeld}. Hg. Horst Weber. Heidelberg:
                        \emph{Lothar Stiehm}{ }1967, S. 23).}}}\label{K_L00067_2h}{ }\textsc{Bérénice}\pwindex{\textcolor{red}{\textsuperscript{XXXX1 indx}}!Garten der Berenice1891@\strich\emph{Der Garten der Bérenice} {[}1891{]}|pw} u. \textsc{Sept Princesses}\pwindex{Maeterlinck, Maurice 29.08.1862 – 06.05.1949@\textsc{Maeterlinck, Maurice} (29.08.1862 – 06.05.1949), \emph{Schriftsteller}!sieben Prinzessinnen1891@\strich\emph{Die sieben Prinzessinnen} {[}1891{]}|pw}.\pend
           \pstart
           Es bleibt bei Sonntag?\pend
           \pstart \spacefill\mbox{Loris.}\pend{}\pstart
           \noindent{}Die \label{K_L00067_3v}\edtext{Überwindung\pwindex{Bahr, Hermann 19.07.1863 – 15.01.1934@\textsc{Bahr, Hermann} (19.07.1863 – 15.01.1934), \emph{Schriftsteller, Kritiker}!Ueberwindung des Naturalismus1891@\strich\emph{Die Überwindung des Naturalismus} {[}1891{]}|pw}}{\lemma{\textnormal{\emph{Überwindung}}}\Cendnote{\textnormal{Wohl wegen des Artikels \emph{Maurice Maeterlinck}\pwindex{Bahr, Hermann 19.07.1863 – 15.01.1934@\textsc{Bahr, Hermann} (19.07.1863 – 15.01.1934), \emph{Schriftsteller, Kritiker}!Maurice Maeterlinck10. 01. 1891@\strich\emph{Maurice Maeterlinck} {[}10. 01. 1891{]}|pwk}. In: Hermann Bahr\pwindex{Bahr, Hermann 19.07.1863 – 15.01.1934@\textsc{Bahr, Hermann} (19.07.1863 – 15.01.1934), \emph{Schriftsteller, Kritiker}|pwk}: \emph{Die
                        Überwindung des Naturalismus}\pwindex{Bahr, Hermann 19.07.1863 – 15.01.1934@\textsc{Bahr, Hermann} (19.07.1863 – 15.01.1934), \emph{Schriftsteller, Kritiker}!Ueberwindung des Naturalismus1891@\strich\emph{Die Überwindung des Naturalismus} {[}1891{]}|pwk}. Dresden, Leipzig: \emph{E. Pierson}\orgindex{E. Pierson s Verlag@E. Pierson’s Verlag|pwk}{ }1891, S. 189–198 (Als zweite Reihe von »Zur Kritik der
                        Moderne«). Erstdruck: \emph{Magazin für
                           Litteratur}\pwindex{Magazin fuer die Literatur des Auslandes1832 – 1915@\emph{Magazin für die Literatur des Auslandes}|pwk}, Jg. 60, Nr. 2, 10. 1. 1891,
                     S. 25–27.}}}\label{K_L00067_3h} habe ich zuhauſe\pend
           \endnumbering\briefempfaengerindex{Schnitzler, Arthur@\textsc{Schnitzler, Arthur}!zzzHofmannsthal, Hugo von@\emph{von Hugo von Hofmannsthal}!1892-02-031@{{[}3.? 2. 1892{]}}|)be}\mylabel{h}\end{ledgroupsized}  \newcommand{\dateiname}{L00067}\newcommand{\titel}{Hugo von Hofmannsthal an Arthur Schnitzler, [3.? 2. 1892]}\newcommand{\editorInnen}{ Martin Anton Müller und Gerd-Hermann Susen}%% latex-leseansicht-abspann.tex
%% Abspann für die Leseansicht.
%% Der Schalter \ifkorrekturansicht ist bereits durch den Vorspann gesetzt.

%% latex-abspann.tex
%% Gemeinsamer Abspann für Korrekturansicht und Leseansicht.
%% Setzt den Schalter \ifkorrekturansicht voraus (gesetzt in den
%% einbindenden Dateien latex-korrekturansicht-abspann.tex bzw.
%% latex-leseansicht-abspann.tex).
%% ---------------------------------------------------------------

\normalsize

% Das esempio-Environment wird nur in der Leseansicht benötigt
\ifkorrekturansicht\else
\newenvironment{esempio}[3]%
{
    \vspace{1.5ex}
    \rlap{\underline{#1}}
    \par
    \setlength{\parindent}{0cm}
    \nopagebreak
    \leftskip=#2cm
    \rightskip=#3cm
}
{
    \par
}
\fi

\doendnotes{C}
\bigskip
\vfill

\clearpage

\footnotesize

\ifkorrekturansicht
  \lohead{\textsc{register}}
\fi

% theindex-Environment neu definieren ohne reledmac
\makeatletter
\renewenvironment{theindex}{%
  \ifkorrekturansicht
    \section*{\indexname}%
  \else
    \subsubsection*{Index der erwähnten Entitäten}%
  \fi
  \setlength{\parindent}{0pt}%
  \setlength{\parskip}{0pt plus 0.3pt}%
  \let\item\@idxitem
}{%
  \ifkorrekturansicht\clearpage\fi
}
\makeatother

\IfFileExists{\jobname-pw.ind}{\input{\jobname-pw.ind}}{}

% Quellenangabe nur in der Leseansicht
\ifkorrekturansicht\else
% Fallback-Definitionen, falls die .tex-Datei \titel etc. nicht gesetzt hat
\providecommand{\titel}{}
\providecommand{\editorInnen}{}
\providecommand{\dateiname}{\jobname}

\vspace{3cm}

\vfill

\footnotesize
\textsc{Quelle}: \titel. Herausgegeben von {\editorInnen}. In: \emph{Arthur Schnitzler: Briefwechsel mit Autorinnen und Autoren}.
 Digitale Edition, https://schnitzler-briefe.acdh.oeaw.ac.at/{\dateiname}.html (Stand \today)
\fi

\end{document}


      