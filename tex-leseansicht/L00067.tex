%% latex-leseansicht-vorspann.tex
%% Vorspann für die Leseansicht.
%% Lädt die gemeinsame Datei latex-vorspann.tex mit nicht gesetztem Schalter.

\newif\ifkorrekturansicht
\korrekturansichtfalse

\input{../tex-inputs/latex-vorspann}


\section[Hugo von Hofmannsthal an Arthur Schnitzler, {[}3.? 2. 1892{]}]{L00067 Hugo von Hofmannsthal an Arthur Schnitzler, {[}3.? 2. 1892{]}}
\nopagebreak\mylabel{L00067v}
\rehead{ }\normalsize\beginnumbering\briefempfaengerindex{Schnitzler, Arthur@\textsc{Schnitzler, Arthur}!zzzHofmannsthal, Hugo von@\emph{von Hugo von Hofmannsthal}!1892-02-031@{{[}3.? 2. 1892{]}}|(be}
\toendnotes[C]{\smallbreak\pagebreak[2]}
\correspDesc{Versand  durch Hugo von Hofmannsthal am [3.? 2. 1892] in Wien
\newline{}Erhalt  durch Arthur Schnitzler im Zeitraum [3. 2. 1892
                  – 7. 2. 1892?] in Wien}\toendnotes[C]{\smallbreak}
\Standort{CUL, Schnitzler, B 43.}
\physDesc{Brief, 1 Blatt, 1 Seite, 141 Zeichen
\newline{}Handschrift: schwarze Tinte, deutsche Kurrent
\newline{}Schnitzler: mit Bleistift nummeriert: »14« }
\buchAbdrucke{\weitereDrucke{1) Hugo von Hofmannsthal, Arthur Schnitzler: \emph{Briefwechsel}. Herausgegeben von Therese Nickl und Heinrich Schnitzler. Frankfurt am Main: \emph{S. Fischer} 1964, S. 15.} \weitereDrucke{2) Hermann Bahr, Arthur Schnitzler: \emph{Briefwechsel, Aufzeichnungen, Dokumente (1891–1931)}. Herausgegeben von Kurt Ifkovits und Martin Anton Müller. Göttingen: \emph{Wallstein} 2018, S. 21.} }\toendnotes[C]{\smallbreak}
\pstart{}{\pb}Lieber Freund.\pend\vspace{0.5em}
\pstart
           Ich bitte um die \label{K_L00067-1v}\edtext{geſtern}{\lemma{\textnormal{\emph{gestern}}}\Cendnote{\textnormal{Vgl. A. S.: \emph{Tagebuch}, 31. 1. 1892. Gegen die
                  Datierung spricht, dass am 2. 2. 1892 noch ein Treffen stattfindet, das hier nicht thematisiert
                  wird.}}}\label{K_L00067-1} vergeſſenen \label{K_L00067-2v}\edtext{\textsc{Aveugles}\pwindex{Maeterlinck, Maurice 29.\,8.\,1862 Gent – 6.\,5.\,1949 Nizza@\textsc{Maeterlinck, Maurice} (29.\,8.\,1862 Gent – 6.\,5.\,1949 Nizza), \emph{Schriftsteller}!Blinden@\strich\emph{Die Blinden}|pw}}{\lemma{\textnormal{\emph{Aveugles}}}\Cendnote{\textnormal{In der Folge übersetzte Hofmannsthal\pwindex{Hofmannsthal, Hugo von 1.\,2.\,1874 Wien – 15.\,7.\,1929 Rodaun@\textsc{Hofmannsthal, Hugo von} (1.\,2.\,1874 Wien – 15.\,7.\,1929 Rodaun), \emph{Schriftsteller}|pwk} ausschließlich diesen Einakter
                  von Maeterlinck\pwindex{Maeterlinck, Maurice 29.\,8.\,1862 Gent – 6.\,5.\,1949 Nizza@\textsc{Maeterlinck, Maurice} (29.\,8.\,1862 Gent – 6.\,5.\,1949 Nizza), \emph{Schriftsteller}|pwk} (vgl. Brief an Marie Herzfeld\pwindex{Herzfeld, Marie 20.\,3.\,1855 Kőszeg – 22.\,9.\,1940 Mining@\textsc{Herzfeld, Marie} (20.\,3.\,1855 Kőszeg – 22.\,9.\,1940 Mining), \emph{Schriftstellerin, Übersetzerin}|pwk}, 9. 3. 1892,
                     in: Hugo von Hofmannsthal\pwindex{Hofmannsthal, Hugo von 1.\,2.\,1874 Wien – 15.\,7.\,1929 Rodaun@\textsc{Hofmannsthal, Hugo von} (1.\,2.\,1874 Wien – 15.\,7.\,1929 Rodaun), \emph{Schriftsteller}|pwk}: \emph{Briefe an Marie Herzfeld}. Herausgegeben von Horst Weber. Heidelberg:
                        \emph{Lothar Stiehm}{ }1967, S. 23).}}}\label{K_L00067-2}{ }\textsc{Bérénice}\pwindex{\textcolor{red}{\textsuperscript{XXXX indx1}}!Garten der Bérenice@\strich\emph{Der Garten der Bérenice}|pw} u. \textsc{Sept Princesses}\pwindex{Maeterlinck, Maurice 29.\,8.\,1862 Gent – 6.\,5.\,1949 Nizza@\textsc{Maeterlinck, Maurice} (29.\,8.\,1862 Gent – 6.\,5.\,1949 Nizza), \emph{Schriftsteller}!sieben Prinzessinnen@\strich\emph{Die sieben Prinzessinnen}|pw}.\pend
           
\pstart
           Es bleibt bei Sonntag?\pend
           \pstart \spacefill\mbox{Loris.}\pend{}
\pstart
           \noindent{}Die \label{K_L00067-3v}\edtext{Überwindung\pwindex{Bahr, Hermann 19.\,7.\,1863 Linz – 15.\,1.\,1934 München@\textsc{Bahr, Hermann} (19.\,7.\,1863 Linz – 15.\,1.\,1934 München), \emph{Schriftsteller, Kritiker}!Überwindung des Naturalismus. Als zweite Reihe von »Die Kritik der Moderne«@\strich\emph{Die Überwindung des Naturalismus. Als zweite Reihe von »Die Kritik der Moderne«}|pw}}{\lemma{\textnormal{\emph{Überwindung}}}\Cendnote{\textnormal{Wohl wegen des Artikels \emph{Maurice Maeterlinck}\pwindex{Bahr, Hermann 19.\,7.\,1863 Linz – 15.\,1.\,1934 München@\textsc{Bahr, Hermann} (19.\,7.\,1863 Linz – 15.\,1.\,1934 München), \emph{Schriftsteller, Kritiker}!Maurice Maeterlinck@\strich\emph{Maurice Maeterlinck}|pwk}. In: Hermann Bahr\pwindex{Bahr, Hermann 19.\,7.\,1863 Linz – 15.\,1.\,1934 München@\textsc{Bahr, Hermann} (19.\,7.\,1863 Linz – 15.\,1.\,1934 München), \emph{Schriftsteller, Kritiker}|pwk}: \emph{Die Überwindung des Naturalismus}\pwindex{Bahr, Hermann 19.\,7.\,1863 Linz – 15.\,1.\,1934 München@\textsc{Bahr, Hermann} (19.\,7.\,1863 Linz – 15.\,1.\,1934 München), \emph{Schriftsteller, Kritiker}!Überwindung des Naturalismus. Als zweite Reihe von »Die Kritik der Moderne«@\strich\emph{Die Überwindung des Naturalismus. Als zweite Reihe von »Die Kritik der Moderne«}|pwk}. Dresden, Leipzig:
                           \emph{E. Pierson}\orgindex{E. Pierson’s Verlag@E. Pierson’s Verlag|pwk}{ }1891, S. 189–198 (Als zweite Reihe von »Zur Kritik der
                        Moderne«). Erstdruck: \emph{Magazin
                           für Litteratur}\pwindex{Magazin für die Literatur des Auslandes@\emph{Magazin für die Literatur des Auslandes}|pwk}, Jg. 60, Nr. 2, 10. 1. 1891,
                        S. 25–27.}}}\label{K_L00067-3} habe ich zuhauſe\pend
           \selectlanguage{ngerman}\endnumbering\briefempfaengerindex{Schnitzler, Arthur@\textsc{Schnitzler, Arthur}!zzzHofmannsthal, Hugo von@\emph{von Hugo von Hofmannsthal}!1892-02-031@{{[}3.? 2. 1892{]}}|)be}\mylabel{L00067h}  \newcommand{\dateiname}{L00067}\newcommand{\titel}{Hugo von Hofmannsthal an Arthur Schnitzler, [3.? 2. 1892]}\newcommand{\editorInnen}{Herausgegeben von Martin Anton Müller}%% latex-leseansicht-abspann.tex
%% Abspann für die Leseansicht.
%% Der Schalter \ifkorrekturansicht ist bereits durch den Vorspann gesetzt.

%% latex-abspann.tex
%% Gemeinsamer Abspann für Korrekturansicht und Leseansicht.
%% Setzt den Schalter \ifkorrekturansicht voraus (gesetzt in den
%% einbindenden Dateien latex-korrekturansicht-abspann.tex bzw.
%% latex-leseansicht-abspann.tex).
%% ---------------------------------------------------------------

\normalsize

% Das esempio-Environment wird nur in der Leseansicht benötigt
\ifkorrekturansicht\else
\newenvironment{esempio}[3]%
{
    \vspace{1.5ex}
    \rlap{\underline{#1}}
    \par
    \setlength{\parindent}{0cm}
    \nopagebreak
    \leftskip=#2cm
    \rightskip=#3cm
}
{
    \par
}
\fi

\doendnotes{C}
\bigskip
\vfill

\clearpage

\footnotesize

\ifkorrekturansicht
  \lohead{\textsc{register}}
\fi

% theindex-Environment neu definieren ohne reledmac
\makeatletter
\renewenvironment{theindex}{%
  \ifkorrekturansicht
    \section*{\indexname}%
  \else
    \subsubsection*{Index der erwähnten Entitäten}%
  \fi
  \setlength{\parindent}{0pt}%
  \setlength{\parskip}{0pt plus 0.3pt}%
  \let\item\@idxitem
}{%
  \ifkorrekturansicht\clearpage\fi
}
\makeatother

\IfFileExists{\jobname-pw.ind}{\input{\jobname-pw.ind}}{}

% Quellenangabe nur in der Leseansicht
\ifkorrekturansicht\else
% Fallback-Definitionen, falls die .tex-Datei \titel etc. nicht gesetzt hat
\providecommand{\titel}{}
\providecommand{\editorInnen}{}
\providecommand{\dateiname}{\jobname}

\vspace{3cm}

\vfill

\footnotesize
\textsc{Quelle}: \titel. Herausgegeben von {\editorInnen}. In: \emph{Arthur Schnitzler: Briefwechsel mit Autorinnen und Autoren}.
 Digitale Edition, https://schnitzler-briefe.acdh.oeaw.ac.at/{\dateiname}.html (Stand \today)
\fi

\end{document}


