%% latex-leseansicht-vorspann.tex
%% Vorspann für die Leseansicht.
%% Lädt die gemeinsame Datei latex-vorspann.tex mit nicht gesetztem Schalter.

\newif\ifkorrekturansicht
\korrekturansichtfalse

\input{../tex-inputs/latex-vorspann}


\section[Arthur Schnitzler an Gustav Schwarzkopf, 7. 8. 1914]{L04160 Arthur Schnitzler an Gustav Schwarzkopf, 7. 8. 1914}
\nopagebreak\mylabel{L04160v}
\rehead{ }\normalsize\beginnumbering\briefempfaengerindex{Schwarzkopf, Gustav@\textsc{Schwarzkopf, Gustav}!zzzSchnitzler, Arthur@\emph{von Arthur Schnitzler}!1914-08-071@{7. 8. 1914}|(be}
\toendnotes[C]{\smallbreak\pagebreak[2]}
\correspDesc{Versand  durch Arthur Schnitzler am 7. 8. 1914 in Brijuni
\newline{}Erhalt  durch Gustav Schwarzkopf im Zeitraum [8. 8. 1914 – 12. 8. 1914?] in Wien}\toendnotes[C]{\smallbreak}
\Standort{CUL, Schnitzler, B 96.}
\physDesc{Kartenbrief, 659 Zeichen
\newline{}Handschrift: schwarze Tinte, deutsche Kurrent
\newline{}Versand: Stempel: »\nobreak{}\oindex{Brijuni@\textbf{Brijuni}|pwk}B{[}rioni{]}, 7. \textcolor{gray}{8.} {[}14{]}\nobreak{}«.  }\pstart{}{\pb}Herrn Guſtav Schwarzkopf\pend{}\pstart{}Wien I\oindex{I., Innere Stadt@\textbf{I., Innere Stadt}, \emph{Verwaltungsgebiet}|pw}\pend{}\pstart{}Tiefer Graben \substVorne{}\textsuperscript{23.}\substDazwischen{}17.\substHinten{}\oindex{Wien@\textbf{Wien}!I., Innere Stadt@\textbf{I., Innere Stadt}!Tiefer Graben 17@\textbf{Tiefer Graben 17}, \emph{Wohngebäude}|pw}.\pend{}{\bigskip}\vspace{1em}
\pstart
           \raggedleft{}{\pb}Brioni\oindex{Brijuni@\textbf{Brijuni}|pw},
                  7. 8. 914.\pend
           \vspace{0.5em}
\pstart
           lieber Guſtav, we{\geminationn} ich auch über das
               Ausbleiben einer Nachricht keineswegs verwundert war – die Aufklärung hat mich
               begreiflicherweiſe{ }ſehr überraſcht; und ich bin nur froh, daſs die Sache ſo gut
               ausgefallen iſt. We{\geminationn} Sie Zeit u Luſt haben, ſchreiben
               Sie mir vielleicht etwas mehr – insbeſondere wie es Ihnen jetzt geht, wer Sie
               behandelt u. ſ. w. An Ende hätten Sie hier die Reizung gar nicht beko{\geminationm}en–? Diätfehler auf der Reiſe–? Bei
               uns gibts nichts neues; alles befindet ſich wohl, es wird gebadet, geſegelt u
               muſiziert, heut erwarten wir Stephi\pwindex{Bachrach, Stefanie 22.\,5.\,1887 Wien – 16.\,5.\,1917 ebd.@\textsc{Bachrach, Stefanie} (22.\,5.\,1887 Wien – 16.\,5.\,1917 ebd.), \emph{Krankenpflegerin}|pw}. Wir
               grüßen Sie aufs allerherzlichſte!\pend
           \pstart Ihr \spacefill\mbox{Arthur}\pend{}\selectlanguage{ngerman}\endnumbering\briefempfaengerindex{Schwarzkopf, Gustav@\textsc{Schwarzkopf, Gustav}!zzzSchnitzler, Arthur@\emph{von Arthur Schnitzler}!1914-08-071@{7. 8. 1914}|)be}\mylabel{L04160h}
\begin{anhang}
\end{anhang}\newcommand{\dateiname}{L04160}\newcommand{\titel}{Arthur Schnitzler an Gustav Schwarzkopf, 7. 8. 1914}\newcommand{\editorInnen}{Herausgegeben von Jahnke, SelmaMüller, Martin Anton}%% latex-leseansicht-abspann.tex
%% Abspann für die Leseansicht.
%% Der Schalter \ifkorrekturansicht ist bereits durch den Vorspann gesetzt.

%% latex-abspann.tex
%% Gemeinsamer Abspann für Korrekturansicht und Leseansicht.
%% Setzt den Schalter \ifkorrekturansicht voraus (gesetzt in den
%% einbindenden Dateien latex-korrekturansicht-abspann.tex bzw.
%% latex-leseansicht-abspann.tex).
%% ---------------------------------------------------------------

\normalsize

% Das esempio-Environment wird nur in der Leseansicht benötigt
\ifkorrekturansicht\else
\newenvironment{esempio}[3]%
{
    \vspace{1.5ex}
    \rlap{\underline{#1}}
    \par
    \setlength{\parindent}{0cm}
    \nopagebreak
    \leftskip=#2cm
    \rightskip=#3cm
}
{
    \par
}
\fi

\doendnotes{C}
\bigskip
\vfill

\clearpage

\footnotesize

\ifkorrekturansicht
  \lohead{\textsc{register}}
\fi

% theindex-Environment neu definieren ohne reledmac
\makeatletter
\renewenvironment{theindex}{%
  \ifkorrekturansicht
    \section*{\indexname}%
  \else
    \subsubsection*{Index der erwähnten Entitäten}%
  \fi
  \setlength{\parindent}{0pt}%
  \setlength{\parskip}{0pt plus 0.3pt}%
  \let\item\@idxitem
}{%
  \ifkorrekturansicht\clearpage\fi
}
\makeatother

\IfFileExists{\jobname-pw.ind}{\input{\jobname-pw.ind}}{}

% Quellenangabe nur in der Leseansicht
\ifkorrekturansicht\else
% Fallback-Definitionen, falls die .tex-Datei \titel etc. nicht gesetzt hat
\providecommand{\titel}{}
\providecommand{\editorInnen}{}
\providecommand{\dateiname}{\jobname}

\vspace{3cm}

\vfill

\footnotesize
\textsc{Quelle}: \titel. Herausgegeben von {\editorInnen}. In: \emph{Arthur Schnitzler: Briefwechsel mit Autorinnen und Autoren}.
 Digitale Edition, https://schnitzler-briefe.acdh.oeaw.ac.at/{\dateiname}.html (Stand \today)
\fi

\end{document}


