%% latex-leseansicht-vorspann.tex
%% Vorspann für die Leseansicht.
%% Lädt die gemeinsame Datei latex-vorspann.tex mit nicht gesetztem Schalter.

\newif\ifkorrekturansicht
\korrekturansichtfalse

\input{../tex-inputs/latex-vorspann}


\section[ Paul Goldmann an Arthur Schnitzler, 11. 6. [1901]]{L03069 Paul Goldmann an Arthur Schnitzler,  11. 6. [1901]}
\nopagebreak\mylabel{L03069v}
\rehead{ }\normalsize\beginnumbering\briefempfaengerindex{Schnitzler, Arthur@\textsc{Schnitzler, Arthur}!zzzGoldmann, Paul@\emph{von Paul Goldmann}!1901-06-112@{11. 6. [1901]}|(be}
\toendnotes[C]{\smallbreak\pagebreak[2]}
\correspDesc{Versand  durch Paul Goldmann am 11. 6. [1901] in Berlin
\newline{}Erhalt  durch Arthur Schnitzler im Zeitraum [12. 6. 1901
                  – 16. 6. 1901?] in Salzburg}\toendnotes[C]{\smallbreak}
\Standort{DLA, A:Schnitzler, HS.NZ85.1.3171.}
\physDesc{Brief, 1 Blatt, 4 Seiten, 1364 Zeichen
\newline{}Handschrift: blaue Tinte, deutsche Kurrent
\newline{}Schnitzler: mit rotem Buntstift drei Unterstreichungen }\toendnotes[C]{\smallbreak}
\pstart
           \raggedleft{}{\pb}\textcolor{gray}{\textbf{DESSAUERSTRASSE 19}}\oindex{Dessauer Straße@\textbf{Dessauer Straße}, \emph{Straße}|pw}\pend
           
\pstart
           Berlin\oindex{Berlin@\textbf{Berlin}, \emph{Hauptstadt}|pw}, 11. Juni.\pend
           
\pstart\center{}Mein lieber Freund,\pend\vspace{0.5em}
\pstart
           Endlich ein Brief! Ich war{ }ſchon in Sorge. Jetzt alſo kann ich Dir glückliche Reiſe
               wünſchen, – eine frohe \label{K_L03069-1v}\edtext{Sommerfahrt}{\lemma{\textnormal{\emph{Sommerfahrt}}}\Cendnote{\textnormal{Schnitzler und Olga Gussmann\pwindex{Schnitzler, Olga 17.\,1.\,1882 Wien – 13.\,1.\,1970 Lugano@\textsc{Schnitzler, Olga} (17.\,1.\,1882 Wien – 13.\,1.\,1970 Lugano), \emph{Schauspielerin, Sängerin}|pwk} verbrachten den Sommer zwischen 12. 6. 1901 und 27. 8. 1901 in Salzburg\oindex{Salzburg@\textbf{Salzburg}, \emph{Verwaltungsgebiet}|pwk}, Tirol\oindex{Tirol@\textbf{Tirol}, \emph{Land}|pwk} und Südtirol\oindex{Südtirol@\textbf{Südtirol}, \emph{Verwaltungsgebiet}|pwk}. }}}\label{K_L03069-1} Dir und
               der lieben Gefährtin\pwindex{Schnitzler, Olga 17.\,1.\,1882 Wien – 13.\,1.\,1970 Lugano@\textsc{Schnitzler, Olga} (17.\,1.\,1882 Wien – 13.\,1.\,1970 Lugano), \emph{Schauspielerin, Sängerin}|pwv}. Eine
               oder die andere Andeutung in Deinem Briefe verſtehe ich nicht. Du wirſt mir{ }ſie wohl
               mündlich aufklären. \label{K_L03069-2v}\edtext{Schlimme Nachricht
               von \textsc{Mizzi Gl.\pwindex{Glümer, Marie 3.\,7.\,1867 Wien – 16.\,11.\,1925 München@\textsc{Glümer, Marie} (3.\,7.\,1867 Wien – 16.\,11.\,1925 München), \emph{Schauspielerin}|pw}}}{\lemma{\textnormal{\emph{Schlimme … Gl.}}}\Cendnote{\textnormal{Marie Glümer\pwindex{Glümer, Marie 3.\,7.\,1867 Wien – 16.\,11.\,1925 München@\textsc{Glümer, Marie} (3.\,7.\,1867 Wien – 16.\,11.\,1925 München), \emph{Schauspielerin}|pwk} war neuerdings erkrankt, vgl. A. S.: \emph{Tagebuch}, 6. 6. 1901.}}}\label{K_L03069-2} Die
               Ärmste!\pend
           
\pstart
           Hoffentlich \label{K_L03069-3v}\edtext{ſehen wir uns}{\lemma{\textnormal{\emph{sehen wir uns}}}\Cendnote{\textnormal{Siehe XXXX Auszeichnungsfehler: Dokument L03064 nicht gefunden.
               }}}\label{K_L03069-3} in einigen {\pb}Wochen. Ich möchte diesmal{ }ſchon
               Ende Juli fort, – mit Rückſicht darauf, daß ich kaput
               bin, wie{ }ſchon lange nicht. Zur Stärkung der erſchlafften Nerven brauchte ich
               allerdings Höhenluft. Darum bin ich wieder unſchlüſſig geworden bezüglich des \label{K_L03069-4v}\edtext{Wörther Sees\oindex{Wörthersee@\textbf{Wörthersee}, \emph{See}|pw}}{\lemma{\textnormal{\emph{Wörther Sees}}}\Cendnote{\textnormal{Siehe XXXX Auszeichnungsfehler: Dokument L03066 nicht gefunden.
               }}}\label{K_L03069-4}. An hohen Orten anderſeits fürchte ich die Einſamkeit. Weiß alſo nicht, was
               werden wird.\pend
           
\pstart
           Nun wirſt Du wohl auch zum Arbeiten kommen, und ich freue mich, daß \label{K_L03069-5v}\edtext{der dramatiſche Stoff\pwindex{Schnitzler, Arthur 15.\,5.\,1862 Wien – 21.\,10.\,1931 ebd.@\textsc{Schnitzler, Arthur} (15.\,5.\,1862 Wien – 21.\,10.\,1931 ebd.), \emph{Schriftsteller, Mediziner}!einsame Weg. Schauspiel in fünf Akten@\strich\emph{Der einsame Weg. Schauspiel in fünf Akten}|pwv} vom {\pb}vorigen Jahr}{\lemma{\textnormal{\emph{der … Jahr}}}\Cendnote{\textnormal{Bezug auf \emph{Der einsame Weg}\pwindex{Schnitzler, Arthur 15.\,5.\,1862 Wien – 21.\,10.\,1931 ebd.@\textsc{Schnitzler, Arthur} (15.\,5.\,1862 Wien – 21.\,10.\,1931 ebd.), \emph{Schriftsteller, Mediziner}!einsame Weg. Schauspiel in fünf Akten@\strich\emph{Der einsame Weg. Schauspiel in fünf Akten}|pwk}, den Schnitzler am 25. 8. 1900, während er mit Goldmann\pwindex{Goldmann, Paul 31.\,1.\,1865 Breslau – 25.\,9.\,1935 Wien@\textsc{Goldmann, Paul} (31.\,1.\,1865 Breslau – 25.\,9.\,1935 Wien), \emph{Schriftsteller, Journalist}|pwk} verreist war, entworfen hatte.}}}\label{K_L03069-5} ausgereift iſt und zum
               Greifen fertig daliegt. Ich denke, es wird eines Deiner beſten Stücke werden.\pend
           
\pstart
           Viele treue Grüße an Dich und Fräulein \textsc{Olga\pwindex{Schnitzler, Olga 17.\,1.\,1882 Wien – 13.\,1.\,1970 Lugano@\textsc{Schnitzler, Olga} (17.\,1.\,1882 Wien – 13.\,1.\,1970 Lugano), \emph{Schauspielerin, Sängerin}|pw}}! {\\[\baselineskip]}Dein {\\[\baselineskip]}\spacefill\mbox{Paul Goldmann}\pend
           \leftskip=0em{}
\pstart
           \noindent{}\textsc{Dr. Montij Jacobs\pwindex{Jacobs, Monty 5.\,1.\,1875 Szczecin – 29.\,12.\,1945 London@\textsc{Jacobs, Monty} (5.\,1.\,1875 Szczecin – 29.\,12.\,1945 London), \emph{Schriftsteller, Journalist, Kritiker}|pw}}, der im Börſencourier\pwindex{Berliner Börsen-Courier@\emph{Berliner Börsen-Courier}|pw} über Dich \label{K_L03069-6v}\edtext{geſchrieben\pwindex{Jacobs, Monty 5.\,1.\,1875 Szczecin – 29.\,12.\,1945 London@\textsc{Jacobs, Monty} (5.\,1.\,1875 Szczecin – 29.\,12.\,1945 London), \emph{Schriftsteller, Journalist, Kritiker}!Arthur Schnitzler’s neue Werke@\strich\emph{Arthur Schnitzler’s neue Werke}|pwv}}{\lemma{\textnormal{\emph{geschrieben}}}\Cendnote{\textnormal{nicht nachgewiesen. In Schnitzlers Zeitungsausschnittsammlung
                     findet sich jedoch ein Abzug (University of Exeter, \emph{Schnitzler
                           Press Cuttings Archive}, Box 2/1).}}}\label{K_L03069-6}, iſt ein
                  junger Germaniſt, der \label{K_L03069-7v}\edtext{in wenigen
                     Wochen}{\lemma{\textnormal{\emph{in wenigen
                     Wochen}}}\Cendnote{\textnormal{Monty Jacobs\pwindex{Jacobs, Monty 5.\,1.\,1875 Szczecin – 29.\,12.\,1945 London@\textsc{Jacobs, Monty} (5.\,1.\,1875 Szczecin – 29.\,12.\,1945 London), \emph{Schriftsteller, Journalist, Kritiker}|pwk} und Dora Levysohn\pwindex{Jacobs, Dora 1880 – 1956 London@\textsc{Jacobs, Dora} (1880 – 1956 London)|pwk} (dann Jacobs) heirateten am 25. 6. 1901. }}}\label{K_L03069-7} die Tochter\pwindex{Jacobs, Dora 1880 – 1956 London@\textsc{Jacobs, Dora} (1880 – 1956 London)|pwv} des Herrn {\pb}\textsc{Levysohn\pwindex{Levysohn, Ulrich 1846 – 17.\,10.\,1908@\textsc{Levysohn, Ulrich} (1846 – 17.\,10.\,1908), \emph{Buchhändler, Zeitungsverleger}|pw}}, des Direktors des »Börſencourier\orgindex{Berliner Börsen-Courier@Berliner Börsen-Courier|pw}«
                  heirathen wird.\pend
           
\pstart
           Lies die reizenden \label{K_L03069-8v}\edtext{Memoiren\pwindex{Thiébault, Dieudonné 16.\,12.\,1733 Rupt-sur-Moselle – 5.\,12.\,1807 Versailles@\textsc{Thiébault, Dieudonné} (16.\,12.\,1733 Rupt-sur-Moselle – 5.\,12.\,1807 Versailles), \emph{Professor}!Friedrich der Große und sein Hof. Persönliche Erinnerungen an einen zwanzigjährigen Aufenthalt in Berlin@\strich\emph{Friedrich der Große und sein Hof. Persönliche Erinnerungen an einen zwanzigjährigen Aufenthalt in Berlin}|pwv}{ }\textsc{Thielbauts\pwindex{Thiébault, Dieudonné 16.\,12.\,1733 Rupt-sur-Moselle – 5.\,12.\,1807 Versailles@\textsc{Thiébault, Dieudonné} (16.\,12.\,1733 Rupt-sur-Moselle – 5.\,12.\,1807 Versailles), \emph{Professor}|pw}}}{\lemma{\textnormal{\emph{Memoiren Thielbauts}}}\Cendnote{\textnormal{Dieudonné Thiébault\pwindex{Thiébault, Dieudonné 16.\,12.\,1733 Rupt-sur-Moselle – 5.\,12.\,1807 Versailles@\textsc{Thiébault, Dieudonné} (16.\,12.\,1733 Rupt-sur-Moselle – 5.\,12.\,1807 Versailles), \emph{Professor}|pwk}: \emph{Friedrich der Große und sein Hof. Persönliche
                           Erinnerungen an einen zwanzigjährigen Aufenthalt in Berlin}\pwindex{Thiébault, Dieudonné 16.\,12.\,1733 Rupt-sur-Moselle – 5.\,12.\,1807 Versailles@\textsc{Thiébault, Dieudonné} (16.\,12.\,1733 Rupt-sur-Moselle – 5.\,12.\,1807 Versailles), \emph{Professor}!Friedrich der Große und sein Hof. Persönliche Erinnerungen an einen zwanzigjährigen Aufenthalt in Berlin@\strich\emph{Friedrich der Große und sein Hof. Persönliche Erinnerungen an einen zwanzigjährigen Aufenthalt in Berlin}|pwk}. Erste
                        deutsche Bearbeitung von Heinrich
                           Conrad\pwindex{Conrad, Heinrich 19.\,10.\,1866 Hamburg – 20.\,12.\,1918 München@\textsc{Conrad, Heinrich} (19.\,10.\,1866 Hamburg – 20.\,12.\,1918 München), \emph{Übersetzer, Romanist}|pwk}. Stuttgart: \emph{Verlag von Robert Lutz}\orgindex{Robert Lutz@Robert Lutz|pwk}{ }1901. Nachweisbar ist die Lektüre durch Schnitzler erst Jahre später, am 15. 4. 1909.}}}\label{K_L03069-8} vom Hofe Friedrichs des Großen\pwindex{Friedrich II. von Preußen 24.\,1.\,1712 Berlin – 17.\,8.\,1786 Potsdam@\textsc{Friedrich II. von Preußen} (24.\,1.\,1712 Berlin – 17.\,8.\,1786 Potsdam), \emph{König}|pw}, die{ }ſoeben in guter deutſcher
                  Ausgabe erſchienen{ }ſind.\pend
           
\pstart
           Über die \label{K_L03069-9v}\edtext{Hochzeit}{\lemma{\textnormal{\emph{Hochzeit}}}\Cendnote{\textnormal{Hugo von Hofmannsthal\pwindex{Hofmannsthal, Hugo von 1.\,2.\,1874 Wien – 15.\,7.\,1929 Rodaun@\textsc{Hofmannsthal, Hugo von} (1.\,2.\,1874 Wien – 15.\,7.\,1929 Rodaun), \emph{Schriftsteller}|pwk} und Gertrude Schlesinger\pwindex{Hofmannsthal, Gertrude von 16.\,3.\,1880 Wien – 9.\,11.\,1959 Paddington@\textsc{Hofmannsthal, Gertrude von} (16.\,3.\,1880 Wien – 9.\,11.\,1959 Paddington)|pwk} hatten am 1. 6. 1901 geheiratet.}}}\label{K_L03069-9}
                  Deines Freundes \textsc{Hoffmannsthal\pwindex{Hofmannsthal, Hugo von 1.\,2.\,1874 Wien – 15.\,7.\,1929 Rodaun@\textsc{Hofmannsthal, Hugo von} (1.\,2.\,1874 Wien – 15.\,7.\,1929 Rodaun), \emph{Schriftsteller}|pw}} hätteſt Du mir auch ein Wort{ }ſchreiben können.\pend
           \selectlanguage{ngerman}\endnumbering\briefempfaengerindex{Schnitzler, Arthur@\textsc{Schnitzler, Arthur}!zzzGoldmann, Paul@\emph{von Paul Goldmann}!1901-06-112@{11. 6. [1901]}|)be}\mylabel{L03069h}  \newcommand{\dateiname}{L03069}\newcommand{\titel}{Paul Goldmann an Arthur Schnitzler, 11. 6. [1901]}\newcommand{\editorInnen}{Martin Anton Müller und Laura Untner}%% latex-leseansicht-abspann.tex
%% Abspann für die Leseansicht.
%% Der Schalter \ifkorrekturansicht ist bereits durch den Vorspann gesetzt.

%% latex-abspann.tex
%% Gemeinsamer Abspann für Korrekturansicht und Leseansicht.
%% Setzt den Schalter \ifkorrekturansicht voraus (gesetzt in den
%% einbindenden Dateien latex-korrekturansicht-abspann.tex bzw.
%% latex-leseansicht-abspann.tex).
%% ---------------------------------------------------------------

\normalsize

% Das esempio-Environment wird nur in der Leseansicht benötigt
\ifkorrekturansicht\else
\newenvironment{esempio}[3]%
{
    \vspace{1.5ex}
    \rlap{\underline{#1}}
    \par
    \setlength{\parindent}{0cm}
    \nopagebreak
    \leftskip=#2cm
    \rightskip=#3cm
}
{
    \par
}
\fi

\doendnotes{C}
\bigskip
\vfill

\clearpage

\footnotesize

\ifkorrekturansicht
  \lohead{\textsc{register}}
\fi

% theindex-Environment neu definieren ohne reledmac
\makeatletter
\renewenvironment{theindex}{%
  \ifkorrekturansicht
    \section*{\indexname}%
  \else
    \subsubsection*{Index der erwähnten Entitäten}%
  \fi
  \setlength{\parindent}{0pt}%
  \setlength{\parskip}{0pt plus 0.3pt}%
  \let\item\@idxitem
}{%
  \ifkorrekturansicht\clearpage\fi
}
\makeatother

\IfFileExists{\jobname-pw.ind}{\input{\jobname-pw.ind}}{}

% Quellenangabe nur in der Leseansicht
\ifkorrekturansicht\else
% Fallback-Definitionen, falls die .tex-Datei \titel etc. nicht gesetzt hat
\providecommand{\titel}{}
\providecommand{\editorInnen}{}
\providecommand{\dateiname}{\jobname}

\vspace{3cm}

\vfill

\footnotesize
\textsc{Quelle}: \titel. Herausgegeben von {\editorInnen}. In: \emph{Arthur Schnitzler: Briefwechsel mit Autorinnen und Autoren}.
 Digitale Edition, https://schnitzler-briefe.acdh.oeaw.ac.at/{\dateiname}.html (Stand \today)
\fi

\end{document}


