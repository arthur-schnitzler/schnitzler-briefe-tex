%% latex-leseansicht-vorspann.tex
%% Vorspann für die Leseansicht.
%% Lädt die gemeinsame Datei latex-vorspann.tex mit nicht gesetztem Schalter.

\newif\ifkorrekturansicht
\korrekturansichtfalse

\input{../tex-inputs/latex-vorspann}


         
         \renewcommand{\erwaehntePersonen}{Personen: Heinrich Conrad,  Friedrich II. von Preußen, Marie Glümer, Paul Goldmann, Hugo von Hofmannsthal, Gertrude von Hofmannsthal, Monty Jacobs, Dora Jacobs, Ulrich Levysohn, Olga Schnitzler, Dieudonné Thiébault}
         \renewcommand{\erwaehnteInstitutionen}{Institutionen: Berliner Börsen-Courier, Robert Lutz}
         \renewcommand{\erwaehnteOrte}{Orte: Berlin, Dessauer Straße, Salzburg, Südtirol, Tirol, Wörthersee}
         \renewcommand{\erwaehnteWerke}{Werke: Arthur Schnitzler’s neue Werke, Berliner Börsen-Courier, Der einsame Weg. Schauspiel in fünf Akten, Friedrich der Große und sein Hof. Persönliche Erinnerungen an einen zwanzigjährigen Aufenthalt in Berlin}
               \section[ Paul Goldmann an Arthur Schnitzler, 11. 6. {[}1901{]}]{ Paul Goldmann an Arthur Schnitzler, 11. 6. {[}1901{]}}\nopagebreak\mylabel{v}\rehead{ }\begin{ledgroupsized}[t]{13cm}\normalsize\beginnumbering \toendnotes[C]{\smallbreak\pagebreak[2]} \Standort{DLA, A:Schnitzler, HS.NZ85.1.3171.}
\physDesc{Brief, 1 Blatt, 4 Seiten, 1364 Zeichen
\newline{}Handschrift: blaue Tinte, deutsche Kurrent
\newline{}Schnitzler: mit rotem Buntstift drei Unterstreichungen }\toendnotes[C]{\smallbreak}\pstart
           \noindent{}\raggedleft{}{\pb}\textcolor{gray}{\textbf{DESSAUERSTRASSE 19}}\oindex{Dessauer Strasse@\textbf{Dessauer Straße}|pw}\pend
           \pstart
           Berlin\oindex{Berlin@\textbf{Berlin}|pw}, 11. Juni.\pend
           \pstart\center{}Mein lieber Freund,\pend\pstart
           Endlich ein Brief! Ich war ſchon in Sorge. Jetzt alſo kann ich Dir glückliche Reiſe
               wünſchen, – eine frohe \label{K_L03069-1v}\edtext{Sommerfahrt}{\lemma{\textnormal{\emph{Sommerfahrt}}}\Cendnote{\textnormal{Schnitzler\pwindex{Schnitzler, Arthur 15.05.1862 – 21.10.1931@\textsc{Schnitzler, Arthur} (15.05.1862 – 21.10.1931), \emph{Schriftsteller, Mediziner}|pwk} und Olga Gussmann\pwindex{Schnitzler, Olga 17.01.1882 – 13.01.1970@\textsc{Schnitzler, Olga} (17.01.1882 – 13.01.1970), \emph{Schauspielerin, Sängerin}|pwk} verbrachten den Sommer zwischen 12. 6. 1901 und 27. 8. 1901 in Salzburg\oindex{Salzburg@\textbf{Salzburg}|pwk}, Tirol\oindex{Tirol@\textbf{Tirol}|pwk} und Südtirol\oindex{Suedtirol@\textbf{Südtirol}|pwk}. }}}\label{K_L03069-1h} Dir und
               der lieben Gefährtin\pwindex{Schnitzler, Olga 17.01.1882 – 13.01.1970@\textsc{Schnitzler, Olga} (17.01.1882 – 13.01.1970), \emph{Schauspielerin, Sängerin}|pwv}. Eine
               oder die andere Andeutung in Deinem Briefe verſtehe ich nicht. Du wirſt mir ſie wohl
               mündlich aufklären. \label{K_L03069-2v}\edtext{Schlimme Nachricht
               von \textsc{Mizzi Gl.\pwindex{Gluemer, Marie 03.07.1867 – 16.11.1925@\textsc{Glümer, Marie} (03.07.1867 – 16.11.1925), \emph{Schauspielerin}|pw}}}{\lemma{\textnormal{\emph{Schlimme … Gl.}}}\Cendnote{\textnormal{Marie Glümer\pwindex{Gluemer, Marie 03.07.1867 – 16.11.1925@\textsc{Glümer, Marie} (03.07.1867 – 16.11.1925), \emph{Schauspielerin}|pwk} war neuerdings erkrankt, vgl. A. S.: \emph{Tagebuch}, 6. 6. 1901.}}}\label{K_L03069-2h} Die
               Ärmste!\pend
           \pstart
           Hoffentlich \label{K_L03069-3v}\edtext{ſehen wir uns}{\lemma{\textnormal{\emph{ſehen wir uns}}}\Cendnote{\textnormal{siehe Paul Goldmann an Arthur Schnitzler, 26. 4. [1901]}}}\label{K_L03069-3h} in einigen {\pb}Wochen. Ich möchte diesmal ſchon
               Ende Juli fort, – mit Rückſicht darauf, daß ich kaput
               bin, wie ſchon lange nicht. Zur Stärkung der erſchlafften Nerven brauchte ich
               allerdings Höhenluft. Darum bin ich wieder unſchlüſſig geworden bezüglich des \label{K_L03069-4v}\edtext{Wörther See\oindex{Woerthersee@\textbf{Wörthersee}|pw}}{\lemma{\textnormal{\emph{Wörther See}}}\Cendnote{\textnormal{siehe Paul Goldmann an Arthur Schnitzler, 13. 5. [1901]}}}\label{K_L03069-4h}s. An hohen Orten anderſeits fürchte ich die Einſamkeit. Weiß alſo nicht, was
               werden wird.\pend
           \pstart
           Nun wirſt Du wohl auch zum Arbeiten kommen, und ich freue mich, daß \label{K_L03069-5v}\edtext{der dramatiſche Stoff\pwindex{Schnitzler, Arthur 15.05.1862 – 21.10.1931@\textsc{Schnitzler, Arthur} (15.05.1862 – 21.10.1931), \emph{Schriftsteller, Mediziner}!einsame Weg. Schauspiel in fuenf Akten1904@\strich\emph{Der einsame Weg. Schauspiel in fünf Akten} {[}1904{]}|pwv} vom {\pb}vorigen Jahr}{\lemma{\textnormal{\emph{der … Jahr}}}\Cendnote{\textnormal{Bezug auf \emph{Der einsame Weg}\pwindex{Schnitzler, Arthur 15.05.1862 – 21.10.1931@\textsc{Schnitzler, Arthur} (15.05.1862 – 21.10.1931), \emph{Schriftsteller, Mediziner}!einsame Weg. Schauspiel in fuenf Akten1904@\strich\emph{Der einsame Weg. Schauspiel in fünf Akten} {[}1904{]}|pwk}, den Schnitzler\pwindex{Schnitzler, Arthur 15.05.1862 – 21.10.1931@\textsc{Schnitzler, Arthur} (15.05.1862 – 21.10.1931), \emph{Schriftsteller, Mediziner}|pwk} am 25. 8. 1900, während er mit Goldmann\pwindex{Goldmann, Paul 31.01.1865 – 25.09.1935@\textsc{Goldmann, Paul} (31.01.1865 – 25.09.1935), \emph{Schriftsteller, Journalist}|pwk} verreist war, entworfen hatte.}}}\label{K_L03069-5h} ausgereift iſt und zum
               Greifen fertig daliegt. Ich denke, es wird eines Deiner beſten Stücke werden.\pend
           \pstart
           Viele treue Grüße an Dich und Fräulein \textsc{Olga\pwindex{Schnitzler, Olga 17.01.1882 – 13.01.1970@\textsc{Schnitzler, Olga} (17.01.1882 – 13.01.1970), \emph{Schauspielerin, Sängerin}|pw}}! {\\[\baselineskip]}Dein {\\[\baselineskip]}\spacefill\mbox{Paul Goldmann}\pend
           \leftskip=0em{}\pstart
           \noindent{}\textsc{Dr. Montij Jacobs\pwindex{Jacobs, Monty 05.01.1875 – 29.12.1945@\textsc{Jacobs, Monty} (05.01.1875 – 29.12.1945), \emph{Schriftsteller, Journalist, Kritiker}|pw}}, der im Börſencourier\pwindex{?? Werk@Nicht ermittelte Verfasserinnen und Verfasser!Berliner Boersen-Courier1868 – 1933@\emph{Berliner Börsen-Courier} {[}1868 – 1933{]}|pw} über Dich \label{K_L03069-6v}\edtext{geſchrieben\pwindex{Jacobs, Monty 05.01.1875 – 29.12.1945@\textsc{Jacobs, Monty} (05.01.1875 – 29.12.1945), \emph{Schriftsteller, Journalist, Kritiker}!Arthur Schnitzler s neue Werke1901-06@\strich\emph{Arthur Schnitzler’s neue Werke} {[}1901-06{]}|pwv}}{\lemma{\textnormal{\emph{geſchrieben}}}\Cendnote{\textnormal{Nicht nachgewiesen. In Schnitzler\pwindex{Schnitzler, Arthur 15.05.1862 – 21.10.1931@\textsc{Schnitzler, Arthur} (15.05.1862 – 21.10.1931), \emph{Schriftsteller, Mediziner}|pwk}s Zeitungsausschnittssammlung
                     findet sich jedoch ein Abzug (University of Exeter, \emph{Schnitzler
                           Press Cuttings Archive}, Box 2/1).}}}\label{K_L03069-6h}, iſt ein
                  junger Germaniſt, der \label{K_L03069-7v}\edtext{in wenigen
                     Wochen}{\lemma{\textnormal{\emph{in wenigen
                     Wochen}}}\Cendnote{\textnormal{Monty Jacobs\pwindex{Jacobs, Monty 05.01.1875 – 29.12.1945@\textsc{Jacobs, Monty} (05.01.1875 – 29.12.1945), \emph{Schriftsteller, Journalist, Kritiker}|pwk} und Dora Levysohn\pwindex{Jacobs, Dora 1880 – 1956@\textsc{Jacobs, Dora} (1880 – 1956)|pwk} (dann Jacobs) heirateten am 25. 6. 1901. }}}\label{K_L03069-7h} die Tochter\pwindex{Jacobs, Dora 1880 – 1956@\textsc{Jacobs, Dora} (1880 – 1956)|pwv} des Herrn {\pb}\textsc{Levysohn\pwindex{Levysohn, Ulrich 1846 – 1908-10-17@\textsc{Levysohn, Ulrich} (1846 – 1908-10-17), \emph{Buchhändler, Zeitungsverleger}|pw}}, des Direktors des »Börſencourier\orgindex{Berliner Boersen-Courier@Berliner Börsen-Courier|pw}«
                  heirathen wird.\pend
           \pstart
           Lies die reizenden \label{K_L03069-8v}\edtext{Memoiren\pwindex{Thiebault, Dieudonne 1733-12-16 – 1807-12-05@\textsc{Thiébault, Dieudonné} (1733-12-16 – 1807-12-05), \emph{Professor}!Friedrich der Grosse und sein Hof. Persoenliche Erinnerungen an einen zwanzigjaehrigen Aufenthalt in Berlin1901@\strich\emph{Friedrich der Große und sein Hof. Persönliche Erinnerungen an einen zwanzigjährigen Aufenthalt in Berlin} {[}1901{]}|pwv}{ }\textsc{Thielbaut\pwindex{Thiebault, Dieudonne 1733-12-16 – 1807-12-05@\textsc{Thiébault, Dieudonné} (1733-12-16 – 1807-12-05), \emph{Professor}|pw}s}}{\lemma{\textnormal{\emph{Memoiren Thielbauts}}}\Cendnote{\textnormal{Dieudonné Thiébault\pwindex{Thiebault, Dieudonne 1733-12-16 – 1807-12-05@\textsc{Thiébault, Dieudonné} (1733-12-16 – 1807-12-05), \emph{Professor}|pwk}: \emph{Friedrich der Große und sein Hof. Persönliche
                           Erinnerungen an einen zwanzigjährigen Aufenthalt in Berlin}\pwindex{Thiebault, Dieudonne 1733-12-16 – 1807-12-05@\textsc{Thiébault, Dieudonné} (1733-12-16 – 1807-12-05), \emph{Professor}!Friedrich der Grosse und sein Hof. Persoenliche Erinnerungen an einen zwanzigjaehrigen Aufenthalt in Berlin1901@\strich\emph{Friedrich der Große und sein Hof. Persönliche Erinnerungen an einen zwanzigjährigen Aufenthalt in Berlin} {[}1901{]}|pwk}. Erste
                        deutsche Bearbeitung von Heinrich
                           Conrad\pwindex{Conrad, Heinrich 19.10.1866 – 1918-12-20@\textsc{Conrad, Heinrich} (19.10.1866 – 1918-12-20), \emph{Übersetzer, Romanist}|pwk}. Stuttgart: \emph{Verlag von Robert Lutz}\orgindex{Robert Lutz@Robert Lutz|pwk}{ }1901. Nachweisbar ist die Lektüre durch Schnitzler\pwindex{Schnitzler, Arthur 15.05.1862 – 21.10.1931@\textsc{Schnitzler, Arthur} (15.05.1862 – 21.10.1931), \emph{Schriftsteller, Mediziner}|pwk} erst Jahre später, am 15. 4. 1909.}}}\label{K_L03069-8h} vom Hofe Friedrichs des Großen\pwindex{Friedrich II. von Preussen 24.01.1712 – 17.08.1786@\textsc{Friedrich II. von Preußen} (24.01.1712 – 17.08.1786), \emph{König}|pw}, die ſoeben in guter deutſcher
                  Ausgabe erſchienen ſind.\pend
           \pstart
           Über die \label{K_L03069-9v}\edtext{Hochzeit}{\lemma{\textnormal{\emph{Hochzeit}}}\Cendnote{\textnormal{Hugo von Hofmannsthal\pwindex{Hofmannsthal, Hugo von 1874-02-01 – 1929-07-15@\textsc{Hofmannsthal, Hugo von} (1874-02-01 – 1929-07-15), \emph{Schriftsteller}|pwk} und Gertrude Schlesinger\pwindex{Hofmannsthal, Gertrude von 16.03.1880 – 09.11.1959@\textsc{Hofmannsthal, Gertrude von} (16.03.1880 – 09.11.1959)|pwk} (dann von
                     Hofmannsthal) heirateten am 1. 6. 1901.}}}\label{K_L03069-9h}
                  Deines Freundes \textsc{Hoffmannsthal\pwindex{Hofmannsthal, Hugo von 1874-02-01 – 1929-07-15@\textsc{Hofmannsthal, Hugo von} (1874-02-01 – 1929-07-15), \emph{Schriftsteller}|pw}} hätteſt Du mir auch ein Wort ſchreiben können.\pend
           
         
         \endnumbering\mylabel{h}\end{ledgroupsized}  \newcommand{\dateiname}{L03069}\newcommand{\titel}{Paul Goldmann an Arthur Schnitzler, 11. 6. [1901]}\newcommand{\editorInnen}{Martin Anton Müller und Laura Untner}%% latex-leseansicht-abspann.tex
%% Abspann für die Leseansicht.
%% Der Schalter \ifkorrekturansicht ist bereits durch den Vorspann gesetzt.

%% latex-abspann.tex
%% Gemeinsamer Abspann für Korrekturansicht und Leseansicht.
%% Setzt den Schalter \ifkorrekturansicht voraus (gesetzt in den
%% einbindenden Dateien latex-korrekturansicht-abspann.tex bzw.
%% latex-leseansicht-abspann.tex).
%% ---------------------------------------------------------------

\normalsize

% Das esempio-Environment wird nur in der Leseansicht benötigt
\ifkorrekturansicht\else
\newenvironment{esempio}[3]%
{
    \vspace{1.5ex}
    \rlap{\underline{#1}}
    \par
    \setlength{\parindent}{0cm}
    \nopagebreak
    \leftskip=#2cm
    \rightskip=#3cm
}
{
    \par
}
\fi

\doendnotes{C}
\bigskip
\vfill

\clearpage

\footnotesize

\ifkorrekturansicht
  \lohead{\textsc{register}}
\fi

% theindex-Environment neu definieren ohne reledmac
\makeatletter
\renewenvironment{theindex}{%
  \ifkorrekturansicht
    \section*{\indexname}%
  \else
    \subsubsection*{Index der erwähnten Entitäten}%
  \fi
  \setlength{\parindent}{0pt}%
  \setlength{\parskip}{0pt plus 0.3pt}%
  \let\item\@idxitem
}{%
  \ifkorrekturansicht\clearpage\fi
}
\makeatother

\IfFileExists{\jobname-pw.ind}{\input{\jobname-pw.ind}}{}

% Quellenangabe nur in der Leseansicht
\ifkorrekturansicht\else
% Fallback-Definitionen, falls die .tex-Datei \titel etc. nicht gesetzt hat
\providecommand{\titel}{}
\providecommand{\editorInnen}{}
\providecommand{\dateiname}{\jobname}

\vspace{3cm}

\vfill

\footnotesize
\textsc{Quelle}: \titel. Herausgegeben von {\editorInnen}. In: \emph{Arthur Schnitzler: Briefwechsel mit Autorinnen und Autoren}.
 Digitale Edition, https://schnitzler-briefe.acdh.oeaw.ac.at/{\dateiname}.html (Stand \today)
\fi

\end{document}


      