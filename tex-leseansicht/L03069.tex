%% latex-korrekturansicht-vorspann.tex
%% Vorspann für die Korrekturansicht.
%% Lädt die gemeinsame Datei latex-vorspann.tex mit gesetztem Schalter.

\newif\ifkorrekturansicht
\korrekturansichttrue

\input{../tex-inputs/latex-vorspann}


\section[ Paul Goldmann an Arthur Schnitzler, 11. 6. {[}1901{]}]{L03069 Paul Goldmann an Arthur Schnitzler, 11. 6. {[}1901{]}}
\nopagebreak\mylabel{L03069v}
\rehead{ }\normalsize\beginnumbering\briefempfaengerindex{Schnitzler, Arthur@\textsc{Schnitzler, Arthur}!zzzGoldmann, Paul@\emph{von Paul Goldmann}!1901-06-112@{11. 6. {[}1901{]}}|(be}
\toendnotes[C]{\smallbreak\pagebreak[2]}\Standort{DLA, A:Schnitzler, HS.NZ85.1.3171.}
\physDesc{Brief, 1 Blatt, 4 Seiten, 1364 Zeichen
\newline{}Handschrift: blaue Tinte, deutsche Kurrent
\newline{}Schnitzler: mit rotem Buntstift drei Unterstreichungen }\toendnotes[C]{\smallbreak}
\pstart
           \raggedleft{}{\pb}\textcolor{gray}{\textbf{DESSAUERSTRASSE 19}}\oindex{Dessauer Strasse@\textbf{Dessauer Straße}, \emph{Straße (K.STR)}|pw}\pend
           
\pstart
           Berlin\oindex{Berlin@\textbf{Berlin}, \emph{P.PPLC}|pw}, 11. Juni.\pend
           
\pstart\center{}Mein lieber Freund,\pend\vspace{0.5em}
\pstart
           Endlich ein Brief! Ich war ſchon in Sorge. Jetzt alſo kann ich Dir glückliche Reiſe
               wünſchen, – eine frohe \label{K_L03069-1v}\edtext{Sommerfahrt}{\lemma{\textnormal{\emph{Sommerfahrt}}}\Cendnote{\textnormal{Schnitzler und Olga Gussmann\pwindex{Schnitzler, Olga 17.01.1882 – 13.01.1970@\textsc{Schnitzler, Olga} (17.01.1882 – 13.01.1970), \emph{Schauspieler/Schauspielerin, Sänger/Sängerin}|pwk} verbrachten den Sommer zwischen 12. 6. 1901 und 27. 8. 1901 in Salzburg\oindex{Salzburg@\textbf{Salzburg}, \emph{A.ADM2}|pwk}, Tirol\oindex{Tirol@\textbf{Tirol}, \emph{A.ADM1}|pwk} und Südtirol\oindex{Suedtirol@\textbf{Südtirol}, \emph{A.ADM2}|pwk}. }}}\label{K_L03069-1} Dir und
               der lieben Gefährtin\pwindex{Schnitzler, Olga 17.01.1882 – 13.01.1970@\textsc{Schnitzler, Olga} (17.01.1882 – 13.01.1970), \emph{Schauspieler/Schauspielerin, Sänger/Sängerin}|pwv}. Eine
               oder die andere Andeutung in Deinem Briefe verſtehe ich nicht. Du wirſt mir ſie wohl
               mündlich aufklären. \label{K_L03069-2v}\edtext{Schlimme Nachricht
               von \textsc{Mizzi Gl.\pwindex{Gluemer, Marie 03.07.1867 – 16.11.1925@\textsc{Glümer, Marie} (03.07.1867 – 16.11.1925), \emph{Schauspieler/Schauspielerin}|pw}}}{\lemma{\textnormal{\emph{Schlimme … Gl.}}}\Cendnote{\textnormal{Marie Glümer\pwindex{Gluemer, Marie 03.07.1867 – 16.11.1925@\textsc{Glümer, Marie} (03.07.1867 – 16.11.1925), \emph{Schauspieler/Schauspielerin}|pwk} war neuerdings erkrankt, vgl. A. S.: \emph{Tagebuch}, 6. 6. 1901.}}}\label{K_L03069-2} Die
               Ärmste!\pend
           
\pstart
           Hoffentlich \label{K_L03069-3v}\edtext{ſehen wir uns}{\lemma{\textnormal{\emph{ſehen wir uns}}}\Cendnote{\textnormal{Siehe Paul Goldmann an Arthur Schnitzler, 26. 4. [1901].
               }}}\label{K_L03069-3} in einigen {\pb}Wochen. Ich möchte diesmal ſchon
               Ende Juli fort, – mit Rückſicht darauf, daß ich kaput
               bin, wie ſchon lange nicht. Zur Stärkung der erſchlafften Nerven brauchte ich
               allerdings Höhenluft. Darum bin ich wieder unſchlüſſig geworden bezüglich des \label{K_L03069-4v}\edtext{Wörther Sees\oindex{Woerthersee@\textbf{Wörthersee}, \emph{H.LK}|pw}}{\lemma{\textnormal{\emph{Wörther Sees}}}\Cendnote{\textnormal{Siehe Paul Goldmann an Arthur Schnitzler, 13. 5. [1901].
               }}}\label{K_L03069-4}. An hohen Orten anderſeits fürchte ich die Einſamkeit. Weiß alſo nicht, was
               werden wird.\pend
           
\pstart
           Nun wirſt Du wohl auch zum Arbeiten kommen, und ich freue mich, daß \label{K_L03069-5v}\edtext{der dramatiſche Stoff\pwindex{einsame Weg. Schauspiel in fuenf Akten@\emph{Der einsame Weg. Schauspiel in fünf Akten}|pwv} vom {\pb}vorigen Jahr}{\lemma{\textnormal{\emph{der … Jahr}}}\Cendnote{\textnormal{Bezug auf \emph{Der einsame Weg}\pwindex{einsame Weg. Schauspiel in fuenf Akten@\emph{Der einsame Weg. Schauspiel in fünf Akten}|pwk}, den Schnitzler am 25. 8. 1900, während er mit Goldmann\pwindex{Goldmann, Paul 31.01.1865 – 25.09.1935@\textsc{Goldmann, Paul} (31.01.1865 – 25.09.1935), \emph{Schriftsteller/Schriftstellerin, Journalist/Journalistin}|pwk} verreist war, entworfen hatte.}}}\label{K_L03069-5} ausgereift iſt und zum
               Greifen fertig daliegt. Ich denke, es wird eines Deiner beſten Stücke werden.\pend
           
\pstart
           Viele treue Grüße an Dich und Fräulein \textsc{Olga\pwindex{Schnitzler, Olga 17.01.1882 – 13.01.1970@\textsc{Schnitzler, Olga} (17.01.1882 – 13.01.1970), \emph{Schauspieler/Schauspielerin, Sänger/Sängerin}|pw}}! {\\[\baselineskip]}Dein {\\[\baselineskip]}\spacefill\mbox{Paul Goldmann}\pend
           \leftskip=0em{}
\pstart
           \noindent{}\textsc{Dr. Montij Jacobs\pwindex{Jacobs, Monty 05.01.1875 – 29.12.1945@\textsc{Jacobs, Monty} (05.01.1875 – 29.12.1945), \emph{Schriftsteller/Schriftstellerin, Journalist/Journalistin, Kritiker/Kritikerin}|pw}}, der im Börſencourier\pwindex{Berliner Boersen-Courier@\emph{Berliner Börsen-Courier}|pw} über Dich \label{K_L03069-6v}\edtext{geſchrieben\pwindex{Arthur Schnitzler s neue Werke@\emph{Arthur Schnitzler’s neue Werke}|pwv}}{\lemma{\textnormal{\emph{geſchrieben}}}\Cendnote{\textnormal{nicht nachgewiesen. In Schnitzlers Zeitungsausschnittsammlung
                     findet sich jedoch ein Abzug (University of Exeter, \emph{Schnitzler
                           Press Cuttings Archive}, Box 2/1).}}}\label{K_L03069-6}, iſt ein
                  junger Germaniſt, der \label{K_L03069-7v}\edtext{in wenigen
                     Wochen}{\lemma{\textnormal{\emph{in wenigen
                     Wochen}}}\Cendnote{\textnormal{Monty Jacobs\pwindex{Jacobs, Monty 05.01.1875 – 29.12.1945@\textsc{Jacobs, Monty} (05.01.1875 – 29.12.1945), \emph{Schriftsteller/Schriftstellerin, Journalist/Journalistin, Kritiker/Kritikerin}|pwk} und Dora Levysohn\pwindex{Jacobs, Dora 1880 – 1956@\textsc{Jacobs, Dora} (1880 – 1956)|pwk} (dann Jacobs) heirateten am 25. 6. 1901. }}}\label{K_L03069-7} die Tochter\pwindex{Jacobs, Dora 1880 – 1956@\textsc{Jacobs, Dora} (1880 – 1956)|pwv} des Herrn {\pb}\textsc{Levysohn\pwindex{Levysohn, Ulrich 1846 – 1908-10-17@\textsc{Levysohn, Ulrich} (1846 – 1908-10-17), \emph{Buchhändler/Buchhändlerin, Zeitungsverleger/Zeitungsverlegerin}|pw}}, des Direktors des »Börſencourier\orgindex{Berliner Boersen-Courier@Berliner Börsen-Courier|pw}«
                  heirathen wird.\pend
           
\pstart
           Lies die reizenden \label{K_L03069-8v}\edtext{Memoiren\pwindex{Friedrich der Grosse und sein Hof. Persoenliche Erinnerungen an einen zwanzigjaehrigen Aufenthalt in Berlin@\emph{Friedrich der Große und sein Hof. Persönliche Erinnerungen an einen zwanzigjährigen Aufenthalt in Berlin}|pwv}{ }\textsc{Thielbauts\pwindex{Thiebault, Dieudonne 1733-12-16 – 1807-12-05@\textsc{Thiébault, Dieudonné} (1733-12-16 – 1807-12-05), \emph{Professor/Professorin}|pw}}}{\lemma{\textnormal{\emph{Memoiren Thielbauts}}}\Cendnote{\textnormal{Dieudonné Thiébault\pwindex{Thiebault, Dieudonne 1733-12-16 – 1807-12-05@\textsc{Thiébault, Dieudonné} (1733-12-16 – 1807-12-05), \emph{Professor/Professorin}|pwk}: \emph{Friedrich der Große und sein Hof. Persönliche
                           Erinnerungen an einen zwanzigjährigen Aufenthalt in Berlin}\pwindex{Friedrich der Grosse und sein Hof. Persoenliche Erinnerungen an einen zwanzigjaehrigen Aufenthalt in Berlin@\emph{Friedrich der Große und sein Hof. Persönliche Erinnerungen an einen zwanzigjährigen Aufenthalt in Berlin}|pwk}. Erste
                        deutsche Bearbeitung von Heinrich
                           Conrad\pwindex{Conrad, Heinrich 19.10.1866 – 1918-12-20@\textsc{Conrad, Heinrich} (19.10.1866 – 1918-12-20), \emph{Übersetzer/Übersetzerin, Romanist/Romanistin}|pwk}. Stuttgart: \emph{Verlag von Robert Lutz}\orgindex{Robert Lutz@Robert Lutz|pwk}{ }1901. Nachweisbar ist die Lektüre durch Schnitzler erst Jahre später, am 15. 4. 1909.}}}\label{K_L03069-8} vom Hofe Friedrichs des Großen\pwindex{Friedrich II. von Preussen 24.01.1712 – 17.08.1786@\textsc{Friedrich II. von Preußen} (24.01.1712 – 17.08.1786), \emph{König/Königin}|pw}, die ſoeben in guter deutſcher
                  Ausgabe erſchienen ſind.\pend
           
\pstart
           Über die \label{K_L03069-9v}\edtext{Hochzeit}{\lemma{\textnormal{\emph{Hochzeit}}}\Cendnote{\textnormal{Hugo von Hofmannsthal\pwindex{Hofmannsthal, Hugo von 1874-02-01 – 1929-07-15@\textsc{Hofmannsthal, Hugo von} (1874-02-01 – 1929-07-15), \emph{Schriftsteller/Schriftstellerin}|pwk} und Gertrude Schlesinger\pwindex{Hofmannsthal, Gertrude von 16.03.1880 – 09.11.1959@\textsc{Hofmannsthal, Gertrude von} (16.03.1880 – 09.11.1959)|pwk} hatten am 1. 6. 1901 geheiratet.}}}\label{K_L03069-9}
                  Deines Freundes \textsc{Hoffmannsthal\pwindex{Hofmannsthal, Hugo von 1874-02-01 – 1929-07-15@\textsc{Hofmannsthal, Hugo von} (1874-02-01 – 1929-07-15), \emph{Schriftsteller/Schriftstellerin}|pw}} hätteſt Du mir auch ein Wort ſchreiben können.\pend
           \selectlanguage{ngerman}\endnumbering\briefempfaengerindex{Schnitzler, Arthur@\textsc{Schnitzler, Arthur}!zzzGoldmann, Paul@\emph{von Paul Goldmann}!1901-06-112@{11. 6. {[}1901{]}}|)be}\mylabel{L03069h}  \normalsize

\doendnotes{C}
\bigskip
\vfill

\clearpage

\footnotesize

\lohead{\textsc{register}}

% Definiere theindex-Environment komplett neu ohne reledmac
\makeatletter
\renewenvironment{theindex}{%
  \section*{\indexname}%
  \setlength{\parindent}{0pt}%
  \setlength{\parskip}{0pt plus 0.3pt}%
  \let\item\@idxitem
}{%
  \clearpage
}
\makeatother

\IfFileExists{\jobname-pw.ind}{\input{\jobname-pw.ind}}{}

\end{document}

      