%% latex-leseansicht-vorspann.tex
%% Vorspann für die Leseansicht.
%% Lädt die gemeinsame Datei latex-vorspann.tex mit nicht gesetztem Schalter.

\newif\ifkorrekturansicht
\korrekturansichtfalse

\input{../tex-inputs/latex-vorspann}


\section[Arthur Schnitzler: Widmungsexemplar Reichtum für Hugo von Hofmannsthal, {[}nach Mitte Oktober 1891?{]}]{L00044 Arthur Schnitzler: Widmungsexemplar Reichtum für Hugo von Hofmannsthal, {[}nach Mitte Oktober 1891?{]}}
\nopagebreak\mylabel{L00044v}
\rehead{ }\normalsize\beginnumbering\briefempfaengerindex{Hofmannsthal, Hugo von@\textsc{Hofmannsthal, Hugo von}!zzzSchnitzler, Arthur@\emph{von Arthur Schnitzler}!1891-12-311@{{[}nach Mitte Oktober 1891?{]}}|(be}
\toendnotes[C]{\smallbreak\pagebreak[2]}
\correspDesc{Versand  durch Arthur Schnitzler im Zeitraum [nach Mitte Oktober 1891?] \textbf{Ort fehlend} 
\newline{}Erhalt  durch Hugo von Hofmannsthal im Zeitraum [nach Mitte Oktober 1891?] \textbf{Ort fehlend} }\toendnotes[C]{\smallbreak}
\Standort{FDH, FDH 3239.}
\physDesc{Widmung am Umschlag, 31 Zeichen
\newline{}Handschrift: schwarze Tinte, deutsche Kurrent}
\buchAbdrucke{\weitereDrucke{Hugo von Hofmannsthal: \emph{Bibliothek}. Herausgegeben von Ellen Ritter † in Zusammenarbeit mit Dalia Bukauskaité und Konrad Heumann. Frankfurt am Main: \emph{S. Fischer} 2011, S. 605 (Sämtliche Werke. Kritische Ausgabe, XL).} }\toendnotes[C]{\smallbreak}
\pstart
           \noindent{}{\pb}Meinem \damage{li}eben Freunde \textcolor{gray}{L}\damage{oris}\pend
           \pstart \spacefill\mbox{Arth}\pend{}{\vspace{1\baselineskip}}
\pstart
           \centering{}\textcolor{gray}{\textbf{\so{Reichtum}\pwindex{Schnitzler, Arthur 15.\,5.\,1862 Wien – 21.\,10.\,1931 ebd.@\textsc{Schnitzler, Arthur} (15.\,5.\,1862 Wien – 21.\,10.\,1931 ebd.), \emph{Schriftsteller, Mediziner}!Reichtum. Erzählung@\strich\emph{Reichtum. Erzählung}|pw}}}\pend
           
\pstart
           \centering{}\textcolor{gray}{\textbf{Erzählung}}{\\}\textcolor{gray}{\textbf{von}}{\\}\textcolor{gray}{\textbf{Arthur Schnitzler.}}\pend
           {\vspace{1\baselineskip}}
\pstart
           \centering{}\textcolor{gray}{\textbf{\label{K_L00044-1v}\edtext{Separat-Abdruck}{\lemma{\textnormal{\emph{Separat-Abdruck}}}\Cendnote{\textnormal{In seinem Brief vom XXXX Auszeichnungsfehler: Dokument L00039 nicht gefunden schreibt Schnitzler, noch mehrere Änderungen an der
                     Zeitschriftenfassung für den Separatabdruck vornehmen zu wollen. Es ist
                     anzunehmen, dass dieser Druck zeitnah zum Abdruck des 4. Teils am
                        15. 10. 1891 fertiggestellt wurde.}}}\label{K_L00044-1} aus der »Modernen Rundſchau\pwindex{Moderne Rundschau@\emph{Moderne Rundschau}|pw}«.}}\pend
           
\pstart
           \centering{}\textcolor{gray}{\textbf{\so{Halbmonatſchrift.}}}\pend
           
\pstart
           \centering{}\textcolor{gray}{\textbf{Herausgegeben von \textbf{J. Joachim}\pwindex{Joachim, Jaques 24.\,11.\,1866 Wien – 7.\,11.\,1925 ebd.@\textsc{Joachim, Jaques} (24.\,11.\,1866 Wien – 7.\,11.\,1925 ebd.), \emph{Rechtswissenschaftler, Rechtsanwalt, Herausgeber}|pw} und \textbf{E. M. Kafka}\pwindex{Kafka, Eduard Michael 11.\,3.\,1869 Wien – 6.\,8.\,1893 Brünn@\textsc{Kafka, Eduard Michael} (11.\,3.\,1869 Wien – 6.\,8.\,1893 Brünn), \emph{Redakteur}|pw}.}}\pend
           
\pstart
           \centering{}\textcolor{gray}{\textbf{Druck von Carl Steinhardt {\kaufmannsund} Cie.\orgindex{Carl Steinhardt und Co.@Carl Steinhardt {\kaufmannsund}  Co.|pw} (verantw. Leiter Guſtav Röttig\pwindex{Röttig, Gustav 8.\,12.\,1855 Wien – nach 1918@\textsc{Röttig, Gustav} (8.\,12.\,1855 Wien – nach 1918), \emph{Redakteur, Drucker}|pw}), Wien\oindex{Wien@\textbf{Wien}, \emph{Verwaltungsgebiet}|pw},
                     IX., Hahngaſſe 12\oindex{Wien@\textbf{Wien}!IX., Alsergrund@\textbf{IX., Alsergrund}!Hahngasse@\textbf{Hahngasse}, \emph{Straße}|pw}.}}\pend
           \selectlanguage{ngerman}\endnumbering\briefempfaengerindex{Hofmannsthal, Hugo von@\textsc{Hofmannsthal, Hugo von}!zzzSchnitzler, Arthur@\emph{von Arthur Schnitzler}!1891-10-151@{{[}nach Mitte Oktober 1891?{]}}|)be}\mylabel{L00044h}  \newcommand{\dateiname}{L00044}\newcommand{\titel}{Arthur Schnitzler: Widmungsexemplar Reichtum für Hugo von Hofmannsthal, [nach Mitte Oktober 1891?]}\newcommand{\editorInnen}{Martin Anton Müller und Gerd-Hermann Susen}%% latex-leseansicht-abspann.tex
%% Abspann für die Leseansicht.
%% Der Schalter \ifkorrekturansicht ist bereits durch den Vorspann gesetzt.

%% latex-abspann.tex
%% Gemeinsamer Abspann für Korrekturansicht und Leseansicht.
%% Setzt den Schalter \ifkorrekturansicht voraus (gesetzt in den
%% einbindenden Dateien latex-korrekturansicht-abspann.tex bzw.
%% latex-leseansicht-abspann.tex).
%% ---------------------------------------------------------------

\normalsize

% Das esempio-Environment wird nur in der Leseansicht benötigt
\ifkorrekturansicht\else
\newenvironment{esempio}[3]%
{
    \vspace{1.5ex}
    \rlap{\underline{#1}}
    \par
    \setlength{\parindent}{0cm}
    \nopagebreak
    \leftskip=#2cm
    \rightskip=#3cm
}
{
    \par
}
\fi

\doendnotes{C}
\bigskip
\vfill

\clearpage

\footnotesize

\ifkorrekturansicht
  \lohead{\textsc{register}}
\fi

% theindex-Environment neu definieren ohne reledmac
\makeatletter
\renewenvironment{theindex}{%
  \ifkorrekturansicht
    \section*{\indexname}%
  \else
    \subsubsection*{Index der erwähnten Entitäten}%
  \fi
  \setlength{\parindent}{0pt}%
  \setlength{\parskip}{0pt plus 0.3pt}%
  \let\item\@idxitem
}{%
  \ifkorrekturansicht\clearpage\fi
}
\makeatother

\IfFileExists{\jobname-pw.ind}{\input{\jobname-pw.ind}}{}

% Quellenangabe nur in der Leseansicht
\ifkorrekturansicht\else
% Fallback-Definitionen, falls die .tex-Datei \titel etc. nicht gesetzt hat
\providecommand{\titel}{}
\providecommand{\editorInnen}{}
\providecommand{\dateiname}{\jobname}

\vspace{3cm}

\vfill

\footnotesize
\textsc{Quelle}: \titel. Herausgegeben von {\editorInnen}. In: \emph{Arthur Schnitzler: Briefwechsel mit Autorinnen und Autoren}.
 Digitale Edition, https://schnitzler-briefe.acdh.oeaw.ac.at/{\dateiname}.html (Stand \today)
\fi

\end{document}


