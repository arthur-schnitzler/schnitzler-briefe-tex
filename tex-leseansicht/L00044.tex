%% latex-korrekturansicht-vorspann.tex
%% Vorspann für die Korrekturansicht.
%% Lädt die gemeinsame Datei latex-vorspann.tex mit gesetztem Schalter.

\newif\ifkorrekturansicht
\korrekturansichttrue

\input{../tex-inputs/latex-vorspann}


\section[Arthur Schnitzler: Widmungsexemplar Reichtum für Hugo von Hofmannsthal, {[}nach Mitte Oktober 1891?{]}]{L00044 Arthur Schnitzler: Widmungsexemplar Reichtum für Hugo von Hofmannsthal,
               {[}nach Mitte Oktober 1891?{]}}
\nopagebreak\mylabel{L00044v}
\rehead{ }\normalsize\beginnumbering\briefempfaengerindex{Hofmannsthal, Hugo von@\textsc{Hofmannsthal, Hugo von}!zzzSchnitzler, Arthur@\emph{von Arthur Schnitzler}!1891-12-311@{{[}nach Mitte Oktober 1891?{]}}|(be}
\toendnotes[C]{\smallbreak\pagebreak[2]}\Standort{FDH, FDH 3239.}
\physDesc{Widmung am Umschlag, 31 Zeichen
\newline{}Handschrift: schwarze Tinte, deutsche Kurrent}
\buchAbdrucke{\weitereDrucke{Hugo von Hofmannsthal: \emph{Bibliothek}. Frankfurt am Main: \emph{S. Fischer} 2011, S. 605.} }\toendnotes[C]{\smallbreak}
\pstart
           \noindent{}{\pb}Meinem \damage{li}eben Freunde \textcolor{gray}{L}\damage{oris}\pend
           \pstart \spacefill\mbox{Arth}\pend{}{\vspace{1\baselineskip}}
\pstart
           \centering{}\textcolor{gray}{\textbf{\so{Reichtum}\pwindex{Reichtum. Erzaehlung@\emph{Reichtum. Erzählung}|pw}}}\pend
           
\pstart
           \centering{}\textcolor{gray}{\textbf{Erzählung}}{\\}\textcolor{gray}{\textbf{von}}{\\}\textcolor{gray}{\textbf{Arthur Schnitzler.}}\pend
           {\vspace{1\baselineskip}}
\pstart
           \centering{}\textcolor{gray}{\textbf{\label{K_L00044-1v}\edtext{Separat-Abdruck}{\lemma{\textnormal{\emph{Separat-Abdruck}}}\Cendnote{\textnormal{In seinem Brief vom 11. 9. 1891 schreibt Schnitzler, noch mehrere Änderungen an der
                     Zeitschriftenfassung für den Separatabdruck vornehmen zu wollen. Es ist
                     anzunehmen, dass dieser Druck zeitnah zum Abdruck des 4. Teils am
                        15. 10. 1891 fertiggestellt wurde.}}}\label{K_L00044-1} aus der »Modernen Rundſchau\pwindex{Moderne Rundschau@\emph{Moderne Rundschau}|pw}«.}}\pend
           
\pstart
           \centering{}\textcolor{gray}{\textbf{\so{Halbmonatſchrift.}}}\pend
           
\pstart
           \centering{}\textcolor{gray}{\textbf{Herausgegeben von \textbf{J. Joachim}\pwindex{Joachim, Jaques 24.11.1866 – 07.11.1925@\textsc{Joachim, Jaques} (24.11.1866 – 07.11.1925), \emph{Rechtswissenschaftler/Rechtswissenschaftlerin, Rechtsanwalt/Rechtsanwältin, Herausgeber/Herausgeberin}|pw} und \textbf{E. M. Kafka}\pwindex{Kafka, Eduard Michael 11.03.1869 – 06.08.1893@\textsc{Kafka, Eduard Michael} (11.03.1869 – 06.08.1893), \emph{Redakteur/Redakteurin}|pw}.}}\pend
           
\pstart
           \centering{}\textcolor{gray}{\textbf{Druck von Carl Steinhardt {\kaufmannsund} Cie.\orgindex{Carl Steinhardt und Co.@Carl Steinhardt {\kaufmannsund}  Co.|pw} (verantw. Leiter Guſtav Röttig\pwindex{Roettig, Gustav 1855-12-08 – nach 1918@\textsc{Röttig, Gustav} (1855-12-08 – nach 1918), \emph{Redakteur/Redakteurin, Drucker/Druckerin}|pw}), Wien\oindex{Wien@\textbf{Wien}, \emph{A.ADM2}|pw},
                     IX., Hahngaſſe 12\oindex{Hahngasse@\textbf{Hahngasse}, \emph{Straße (K.STR)}|pw}.}}\pend
           \selectlanguage{ngerman}\endnumbering\briefempfaengerindex{Hofmannsthal, Hugo von@\textsc{Hofmannsthal, Hugo von}!zzzSchnitzler, Arthur@\emph{von Arthur Schnitzler}!1891-10-151@{{[}nach Mitte Oktober 1891?{]}}|)be}\mylabel{L00044h}  \normalsize

\doendnotes{C}
\bigskip
\vfill

\clearpage

\footnotesize

\lohead{\textsc{register}}

% Definiere theindex-Environment komplett neu ohne reledmac
\makeatletter
\renewenvironment{theindex}{%
  \section*{\indexname}%
  \setlength{\parindent}{0pt}%
  \setlength{\parskip}{0pt plus 0.3pt}%
  \let\item\@idxitem
}{%
  \clearpage
}
\makeatother

\IfFileExists{\jobname-pw.ind}{\input{\jobname-pw.ind}}{}

\end{document}

      