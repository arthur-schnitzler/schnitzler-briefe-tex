%% latex-leseansicht-vorspann.tex
%% Vorspann für die Leseansicht.
%% Lädt die gemeinsame Datei latex-vorspann.tex mit nicht gesetztem Schalter.

\newif\ifkorrekturansicht
\korrekturansichtfalse

\input{../tex-inputs/latex-vorspann}


         
         \newcommand{\erwaehntePersonen}{Personen: Hugo von Hofmannsthal, Jaques Joachim, Eduard Michael Kafka, Gustav Röttig}
         \newcommand{\erwaehnteInstitutionen}{Institutionen: Carl Steinhardt & Co.}
         \newcommand{\erwaehnteOrte}{Orte: Hahngasse, Wien}
         \newcommand{\erwaehnteWerke}{Werke: Moderne Rundschau, Reichtum. Erzählung}
               \section[Arthur Schnitzler: Widmungsexemplar Reichtum für Hugo von Hofmannsthal, {[}nach Mitte Oktober 1891?{]}]{ Arthur Schnitzler: Widmungsexemplar Reichtum für Hugo von
                    Hofmannsthal, {[}nach Mitte Oktober 1891?{]}}\nopagebreak\mylabel{v}\rehead{ }\begin{ledgroupsized}[t]{13cm}\normalsize\beginnumbering \toendnotes[C]{\smallbreak\pagebreak[2]} \Standort{FDH, FDH 3239.}
\physDesc{Widmung am Umschlag
\newline{}Handschrift: schwarze Tinte, deutsche Kurrent}\buchAbdrucke{\weitereDrucke{Hugo von Hofmannsthal: \emph{Bibliothek}. Hg. Ellen Ritter † in Zusammenarbeit mit Dalia Bukauskaité und
                                Konrad Heumann. Frankfurt am Main: \emph{S. Fischer} 2011, S. 605 (Sämtliche Werke. Kritische Ausgabe, XL).} }\toendnotes[C]{\smallbreak}\pstart
           \noindent{}{\pb}Meinem \damage{li}eben Freunde \textcolor{gray}{L}\damage{oris}\pend
           \pstart \spacefill\mbox{Arth}\pend{}{\bigskip}\pstart
           \noindent{}\centering{}\textcolor{gray}{\textbf{\so{Reichtum}\pwindex{Schnitzler, Arthur 15.05.1862 – 21.10.1931@\textsc{Schnitzler, Arthur} (15.05.1862 – 21.10.1931), \emph{Schriftsteller, Mediziner}!Reichtum. Erzaehlung1.9.1891 – 15.10.1891@\strich\emph{Reichtum. Erzählung} {[}1.9.1891 – 15.10.1891{]}|pw}}}\pend
           \pstart
           \noindent{}\centering{}\textcolor{gray}{\textbf{Erzählung}}{\\}\textcolor{gray}{\textbf{von}}{\\}\textcolor{gray}{\textbf{Arthur Schnitzler.}}\pend
           {\bigskip}\pstart
           \noindent{}\centering{}\textcolor{gray}{\textbf{\label{K_L00044_1v}\edtext{Separat-Abdruck}{\lemma{\textnormal{\emph{Separat-Abdruck}}}\Cendnote{\textnormal{In seinem Brief vom 11. 9. 1891 schreibt Schnitzler\pwindex{Schnitzler, Arthur 15.05.1862 – 21.10.1931@\textsc{Schnitzler, Arthur} (15.05.1862 – 21.10.1931), \emph{Schriftsteller, Mediziner}|pwk}, noch mehrere Änderungen an der
                            Zeitschriftenfassung für den Separatabdruck vornehmen zu wollen. Es ist
                            anzunehmen, dass dieser Druck zeitnah zum Abdruck des 4. Teils am
                                15. 10. 1891 fertiggestellt wurde.}}}\label{K_L00044_1h} aus der »Modernen Rundſchau\pwindex{Moderne Rundschau1.4.1891 – 31.12.1891@\emph{Moderne Rundschau} {[}1.4.1891 – 31.12.1891{]}|pw}«.}}\pend
           \pstart
           \noindent{}\centering{}\textcolor{gray}{\textbf{\so{Halbmonatſchrift.}}}\pend
           \pstart
           \noindent{}\centering{}\textcolor{gray}{\textbf{Herausgegeben von \textbf{J. Joachim}\pwindex{Joachim, Jaques 24.11.1866 – 07.11.1925@\textsc{Joachim, Jaques} (24.11.1866 – 07.11.1925), \emph{Rechtswissenschaftler, Rechtsanwalt, Herausgeber}|pw} und \textbf{E. M. Kafka}\pwindex{Kafka, Eduard Michael 11.03.1869 – 06.08.1893@\textsc{Kafka, Eduard Michael} (11.03.1869 – 06.08.1893), \emph{Redakteur}|pw}.}}\pend
           \pstart
           \noindent{}\centering{}\textcolor{gray}{\textbf{Druck von Carl Steinhardt {\kaufmannsund} Cie.\orgindex{Carl Steinhardt und Co.@Carl Steinhardt {\kaufmannsund}  Co.|pw} (verantw. Leiter Guſtav Röttig\pwindex{Roettig, Gustav 1855-12-08 – nach 1918@\textsc{Röttig, Gustav} (1855-12-08 – nach 1918), \emph{Redakteur, Drucker}|pw}), Wien\oindex{Wien@\textbf{Wien}|pw}, IX.,
                            Hahngaſſe 12\oindex{Hahngasse@\textbf{Hahngasse}|pw}.}}\pend
           
         
         \endnumbering\mylabel{h}\end{ledgroupsized}  \newcommand{\dateiname}{L00044}\newcommand{\titel}{Arthur Schnitzler: Widmungsexemplar Reichtum für Hugo von Hofmannsthal, [nach Mitte Oktober 1891?]}\newcommand{\editorInnen}{Martin Anton Müller und Gerd-Hermann Susen}%% latex-leseansicht-abspann.tex
%% Abspann für die Leseansicht.
%% Der Schalter \ifkorrekturansicht ist bereits durch den Vorspann gesetzt.

%% latex-abspann.tex
%% Gemeinsamer Abspann für Korrekturansicht und Leseansicht.
%% Setzt den Schalter \ifkorrekturansicht voraus (gesetzt in den
%% einbindenden Dateien latex-korrekturansicht-abspann.tex bzw.
%% latex-leseansicht-abspann.tex).
%% ---------------------------------------------------------------

\normalsize

% Das esempio-Environment wird nur in der Leseansicht benötigt
\ifkorrekturansicht\else
\newenvironment{esempio}[3]%
{
    \vspace{1.5ex}
    \rlap{\underline{#1}}
    \par
    \setlength{\parindent}{0cm}
    \nopagebreak
    \leftskip=#2cm
    \rightskip=#3cm
}
{
    \par
}
\fi

\doendnotes{C}
\bigskip
\vfill

\clearpage

\footnotesize

\ifkorrekturansicht
  \lohead{\textsc{register}}
\fi

% theindex-Environment neu definieren ohne reledmac
\makeatletter
\renewenvironment{theindex}{%
  \ifkorrekturansicht
    \section*{\indexname}%
  \else
    \subsubsection*{Index der erwähnten Entitäten}%
  \fi
  \setlength{\parindent}{0pt}%
  \setlength{\parskip}{0pt plus 0.3pt}%
  \let\item\@idxitem
}{%
  \ifkorrekturansicht\clearpage\fi
}
\makeatother

\IfFileExists{\jobname-pw.ind}{\input{\jobname-pw.ind}}{}

% Quellenangabe nur in der Leseansicht
\ifkorrekturansicht\else
% Fallback-Definitionen, falls die .tex-Datei \titel etc. nicht gesetzt hat
\providecommand{\titel}{}
\providecommand{\editorInnen}{}
\providecommand{\dateiname}{\jobname}

\vspace{3cm}

\vfill

\footnotesize
\textsc{Quelle}: \titel. Herausgegeben von {\editorInnen}. In: \emph{Arthur Schnitzler: Briefwechsel mit Autorinnen und Autoren}.
 Digitale Edition, https://schnitzler-briefe.acdh.oeaw.ac.at/{\dateiname}.html (Stand \today)
\fi

\end{document}


      