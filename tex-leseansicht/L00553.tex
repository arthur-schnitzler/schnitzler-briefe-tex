%% latex-leseansicht-vorspann.tex
%% Vorspann für die Leseansicht.
%% Lädt die gemeinsame Datei latex-vorspann.tex mit nicht gesetztem Schalter.

\newif\ifkorrekturansicht
\korrekturansichtfalse

\input{../tex-inputs/latex-vorspann}


\section[Max Burckhard an Arthur Schnitzler, 25. 6. 1896]{L00553 Max Burckhard an Arthur Schnitzler, 25. 6. 1896}
\nopagebreak\mylabel{L00553v}
\rehead{ }\normalsize\beginnumbering\briefempfaengerindex{Schnitzler, Arthur@\textsc{Schnitzler, Arthur}!zzzBurckhard, Max Eugen@\emph{von Max Eugen Burckhard}!1896-06-252@{25. 6. 1896}|(be}
\toendnotes[C]{\smallbreak\pagebreak[2]}
\correspDesc{Versand  durch Max Burckhard am 25. 6. 1896 in Wien
\newline{}Erhalt  durch Arthur Schnitzler im Zeitraum [25. 6. 1896
                  – 29. 6. 1896?] in Wien}\toendnotes[C]{\smallbreak}
\Standort{CUL, Schnitzler, B 20.}
\physDesc{Brief, 1 Blatt, 1 Seite, 246 Zeichen
\newline{}Handschrift: schwarze Tinte, deutsche Kurrent
\newline{}Schnitzler: mit Bleistift nummeriert: »8« }\toendnotes[C]{\smallbreak}
\pstart
           {\pb}\textcolor{gray}{\textbf{k. k. Hofburgtheater\oindex{Wien@\textbf{Wien}!I., Innere Stadt@\textbf{I., Innere Stadt}!Burgtheater@\textbf{Burgtheater}, \emph{Theater}|pw} Direction}}\hfill Wien\oindex{Wien@\textbf{Wien}, \emph{Verwaltungsgebiet}|pw}{ }25/6 96\pend
           
\pstart{}Sehr verehrter Herr Doctor!\pend\vspace{0.5em}
\pstart
           Bezugnehmend auf unſere mündliche Rückſprache bin ich{ }ſo frei, Ihnen die Komödie
                  \label{K_L00553-1v}\edtext{Der Glückspilz\pwindex{Capus, Alfred 25.\,11.\,1858 Aix-en-Provence – 1.\,11.\,1922 Neuilly-sur-Seine@\textsc{Capus, Alfred} (25.\,11.\,1858 Aix-en-Provence – 1.\,11.\,1922 Neuilly-sur-Seine), \emph{Schriftsteller, Journalist}!Brignol et sa fille@\strich\emph{Brignol et sa fille}|pwu}\pwindex{Capus, Alfred 25.\,11.\,1858 Aix-en-Provence – 1.\,11.\,1922 Neuilly-sur-Seine@\textsc{Capus, Alfred} (25.\,11.\,1858 Aix-en-Provence – 1.\,11.\,1922 Neuilly-sur-Seine), \emph{Schriftsteller, Journalist}!Innocent@\strich\emph{Innocent}|pwu}\pwindex{\textcolor{red}{\textsuperscript{XXXX indx1}}!Innocent@\strich\emph{Innocent}|pwu}}{\lemma{\textnormal{\emph{Der Glückspilz}}}\Cendnote{\textnormal{Unklar. Von Alfred Capus\pwindex{Capus, Alfred 25.\,11.\,1858 Aix-en-Provence – 1.\,11.\,1922 Neuilly-sur-Seine@\textsc{Capus, Alfred} (25.\,11.\,1858 Aix-en-Provence – 1.\,11.\,1922 Neuilly-sur-Seine), \emph{Schriftsteller, Journalist}|pwk} kommen die zwei Stücke \emph{Brignolle et sa fille}\pwindex{Capus, Alfred 25.\,11.\,1858 Aix-en-Provence – 1.\,11.\,1922 Neuilly-sur-Seine@\textsc{Capus, Alfred} (25.\,11.\,1858 Aix-en-Provence – 1.\,11.\,1922 Neuilly-sur-Seine), \emph{Schriftsteller, Journalist}!Brignol et sa fille@\strich\emph{Brignol et sa fille}|pwk} (Uraufführung\eventindex{Uraufführung von Brignol et sa fille, 23.11.1894@Uraufführung von Brignol et sa fille, 23.11.1894|pwkv}{ }23. 11. 1894)
                und \emph{Innocent}\pwindex{Capus, Alfred 25.\,11.\,1858 Aix-en-Provence – 1.\,11.\,1922 Neuilly-sur-Seine@\textsc{Capus, Alfred} (25.\,11.\,1858 Aix-en-Provence – 1.\,11.\,1922 Neuilly-sur-Seine), \emph{Schriftsteller, Journalist}!Innocent@\strich\emph{Innocent}|pwk}\pwindex{\textcolor{red}{\textsuperscript{XXXX indx1}}!Innocent@\strich\emph{Innocent}|pwk} (Uraufführung\eventindex{Uraufführung von Innocent, 7.2.1896@Uraufführung von Innocent, 7.2.1896|pwkv}{ }7. 2. 1896) infrage.}}}\label{K_L00553-1} von \textsc{Capus}\pwindex{Capus, Alfred 25.\,11.\,1858 Aix-en-Provence – 1.\,11.\,1922 Neuilly-sur-Seine@\textsc{Capus, Alfred} (25.\,11.\,1858 Aix-en-Provence – 1.\,11.\,1922 Neuilly-sur-Seine), \emph{Schriftsteller, Journalist}|pw} mit verbindlichem Danke für Ihre freundliche Bemühung zurückzuſenden.\pend
           
\pstart
           In aufrichtiger Verehrung{\\[\baselineskip]}\spacefill\mbox{D\textsuperscript{r}Burckhard}\pend
           \leftskip=0em{}\selectlanguage{ngerman}\endnumbering\briefempfaengerindex{Schnitzler, Arthur@\textsc{Schnitzler, Arthur}!zzzBurckhard, Max Eugen@\emph{von Max Eugen Burckhard}!1896-06-252@{25. 6. 1896}|)be}\mylabel{L00553h}  \newcommand{\dateiname}{L00553}\newcommand{\titel}{Max Burckhard an Arthur Schnitzler, 25. 6. 1896}\newcommand{\editorInnen}{Martin Anton Müller und Gerd-Hermann Susen}%% latex-leseansicht-abspann.tex
%% Abspann für die Leseansicht.
%% Der Schalter \ifkorrekturansicht ist bereits durch den Vorspann gesetzt.

%% latex-abspann.tex
%% Gemeinsamer Abspann für Korrekturansicht und Leseansicht.
%% Setzt den Schalter \ifkorrekturansicht voraus (gesetzt in den
%% einbindenden Dateien latex-korrekturansicht-abspann.tex bzw.
%% latex-leseansicht-abspann.tex).
%% ---------------------------------------------------------------

\normalsize

% Das esempio-Environment wird nur in der Leseansicht benötigt
\ifkorrekturansicht\else
\newenvironment{esempio}[3]%
{
    \vspace{1.5ex}
    \rlap{\underline{#1}}
    \par
    \setlength{\parindent}{0cm}
    \nopagebreak
    \leftskip=#2cm
    \rightskip=#3cm
}
{
    \par
}
\fi

\doendnotes{C}
\bigskip
\vfill

\clearpage

\footnotesize

\ifkorrekturansicht
  \lohead{\textsc{register}}
\fi

% theindex-Environment neu definieren ohne reledmac
\makeatletter
\renewenvironment{theindex}{%
  \ifkorrekturansicht
    \section*{\indexname}%
  \else
    \subsubsection*{Index der erwähnten Entitäten}%
  \fi
  \setlength{\parindent}{0pt}%
  \setlength{\parskip}{0pt plus 0.3pt}%
  \let\item\@idxitem
}{%
  \ifkorrekturansicht\clearpage\fi
}
\makeatother

\IfFileExists{\jobname-pw.ind}{\input{\jobname-pw.ind}}{}

% Quellenangabe nur in der Leseansicht
\ifkorrekturansicht\else
% Fallback-Definitionen, falls die .tex-Datei \titel etc. nicht gesetzt hat
\providecommand{\titel}{}
\providecommand{\editorInnen}{}
\providecommand{\dateiname}{\jobname}

\vspace{3cm}

\vfill

\footnotesize
\textsc{Quelle}: \titel. Herausgegeben von {\editorInnen}. In: \emph{Arthur Schnitzler: Briefwechsel mit Autorinnen und Autoren}.
 Digitale Edition, https://schnitzler-briefe.acdh.oeaw.ac.at/{\dateiname}.html (Stand \today)
\fi

\end{document}


