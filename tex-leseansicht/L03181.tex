%% latex-leseansicht-vorspann.tex
%% Vorspann für die Leseansicht.
%% Lädt die gemeinsame Datei latex-vorspann.tex mit nicht gesetztem Schalter.

\newif\ifkorrekturansicht
\korrekturansichtfalse

\input{../tex-inputs/latex-vorspann}

\begin{center}
            \textcolor{red}{ENTWURF, NICHT FERTIG KORRIGIERT}
                      \end{center}
            
         
         \renewcommand{\erwaehntePersonen}{Personen: Jenny Bally, Samuel Fischer, Wilhelmine Mitterwurzer, Friedrich Mitterwurzer, Max Nordau, Marie Reinhard}
         \renewcommand{\erwaehnteInstitutionen}{Institutionen: Wiener Allgemeine Zeitung, Wiener Verlag}
         \renewcommand{\erwaehnteOrte}{Orte: Berlin, Burgtheater, Wien}
         \renewcommand{\erwaehnteWerke}{Werke: Begräbnis, Cocotte und Kellner, Der Hinterbliebene, Der Hinterbliebene. Kurze Novellen, Der Hund, Die Hochzeit auf dem Lande, Die kleine Veronika. Novelle, Don Karlos, Infant von Spanien, Einiges über Schiller’s »Don Carlos«, Flucht, Freiwild. Schauspiel in 3 Akten, Heldentod. Novelle, Judith. Eine Tragödie in fünf Aufzügen, Neue Freie Presse}
               \section[Felix Salten an Arthur Schnitzler, {[}Ende Oktober 1896{]}]{ Felix Salten an Arthur Schnitzler, {[}Ende Oktober 1896{]}}\nopagebreak\mylabel{v}\rehead{ }\begin{ledgroupsized}[t]{13cm}\normalsize\beginnumbering \toendnotes[C]{\smallbreak\pagebreak[2]} \Standort{CUL, Schnitzler, B 89, A 1.}
\physDesc{Brief, 1 Blatt, 4 Seiten, 2041 Zeichen
\newline{}Handschrift: Bleistift, lateinische Kurrent
\newline{}Schnitzler: mit Bleistift datiert: »Ende Oct 96« 
\newline{}Ordnung: mit Bleistift von unbekannter Hand nummeriert:
                                    »80« }\toendnotes[C]{\smallbreak}\pstart
           \noindent{}{\pb}Lieber Arthur, ob die Versti{\geminationm}ung über
               das Stück\pwindex{Schnitzler, Arthur 15.05.1862 – 21.10.1931@\textsc{Schnitzler, Arthur} (15.05.1862 – 21.10.1931), \emph{Schriftsteller, Mediziner}!Freiwild. Schauspiel in 3 Akten1896@\strich\emph{Freiwild. Schauspiel in 3 Akten} {[}1896{]}|pwv} nicht jenes
               unangenehme Gefühl ist, das man i{\geminationm}er hat, wenn man
               fremde Leute zum ersten Mal eigene Worte aussprechen hört? Ich \label{K_L03181-1v}\edtext{fahre Montag Abend von hier ab
               und bin also Dienstag Mittag bei Ihnen}{\lemma{\textnormal{\emph{fahre … Ihnen}}}\Cendnote{\textnormal{vgl. A. S.: \emph{Tagebuch}, 3. 11. 1896}}}\label{K_L03181-1h}. Wenn es Ihre sonstigen Umstände zulaßen und Sie es leicht können, möchte ich
               Sie um etwas bitten. Sprechen Sie vielleicht mit dem Verleger Fischer\pwindex{Fischer, Samuel 24.12.1859 – 15.10.1934@\textsc{Fischer, Samuel} (24.12.1859 – 15.10.1934), \emph{Verleger}|pw} von mir. Ich will endlich mein \label{K_L03181-56v}\edtext{Buch\pwindex{Salten, Felix 06.09.1869 – 08.10.1945@\textsc{Salten, Felix} (06.09.1869 – 08.10.1945), \emph{Schriftsteller, Journalist}!Hinterbliebene. Kurze Novellen1900@\strich\emph{Der Hinterbliebene. Kurze Novellen} {[}1900{]}|pwv}}{\lemma{\textnormal{\emph{Buch}}}\Cendnote{\textnormal{Die Novellensammlung \emph{Der Hinterbliebene}\pwindex{Salten, Felix 06.09.1869 – 08.10.1945@\textsc{Salten, Felix} (06.09.1869 – 08.10.1945), \emph{Schriftsteller, Journalist}!Hinterbliebene. Kurze Novellen1900@\strich\emph{Der Hinterbliebene. Kurze Novellen} {[}1900{]}|pwk} erschien erst 1900 im \emph{Wiener Verlag}\orgindex{Wiener Verlag@Wiener Verlag|pwk}, aus der hier projektierten
                  Abmachung wurde also nichts.}}}\label{K_L03181-56h} herausgeben. {\pb}Sie wissen, dass mich
               nicht innerliche Gründe dazu bestimmen, denn in der Stimmung, in der ich jetzt seit
               längerer Zeit lebe, möchte ich am liebsten Alles verbrennen. Aber ganz äußerlich
               brauche ich dieses Buch\pwindex{Salten, Felix 06.09.1869 – 08.10.1945@\textsc{Salten, Felix} (06.09.1869 – 08.10.1945), \emph{Schriftsteller, Journalist}!Hinterbliebene. Kurze Novellen1900@\strich\emph{Der Hinterbliebene. Kurze Novellen} {[}1900{]}|pwv} gerade
               jetzt, aus vielen Gründen, vor mir selbst und vor den Anderen. Ich habe meine
               Novellen fertig. Heldentod\pwindex{Salten, Felix 06.09.1869 – 08.10.1945@\textsc{Salten, Felix} (06.09.1869 – 08.10.1945), \emph{Schriftsteller, Journalist}!Heldentod. Novelle1895-01-01@\strich\emph{Heldentod. Novelle} {[}1895-01-01{]}|pw} – Hinterbliebener\pwindex{Salten, Felix 06.09.1869 – 08.10.1945@\textsc{Salten, Felix} (06.09.1869 – 08.10.1945), \emph{Schriftsteller, Journalist}!Hinterbliebene1899-03-04 – 1899-03-11@\strich\emph{Der Hinterbliebene} {[}1899-03-04 – 1899-03-11{]}|pw} – Flucht\pwindex{Salten, Felix 06.09.1869 – 08.10.1945@\textsc{Salten, Felix} (06.09.1869 – 08.10.1945), \emph{Schriftsteller, Journalist}!Flucht1900@\strich\emph{Flucht} {[}1900{]}|pw}
               – Cocotte u. Kellner\pwindex{Salten, Felix 06.09.1869 – 08.10.1945@\textsc{Salten, Felix} (06.09.1869 – 08.10.1945), \emph{Schriftsteller, Journalist}!Cocotte und KellnerNone@\strich\emph{Cocotte und Kellner} {[}None{]}|pw} – Begräbnis\pwindex{Salten, Felix 06.09.1869 – 08.10.1945@\textsc{Salten, Felix} (06.09.1869 – 08.10.1945), \emph{Schriftsteller, Journalist}!Begraebnis17. 7. 1893@\strich\emph{Begräbnis} {[}17. 7. 1893{]}|pw} – Der Hund\pwindex{Salten, Felix 06.09.1869 – 08.10.1945@\textsc{Salten, Felix} (06.09.1869 – 08.10.1945), \emph{Schriftsteller, Journalist}!HundNone@\strich\emph{Der Hund} {[}None{]}|pw} –
                  Die Hochzeit auf dem Lande\pwindex{Salten, Felix 06.09.1869 – 08.10.1945@\textsc{Salten, Felix} (06.09.1869 – 08.10.1945), \emph{Schriftsteller, Journalist}!Hochzeit auf dem LandeNone@\strich\emph{Die Hochzeit auf dem Lande} {[}None{]}|pw} – Die Confirmandin\pwindex{Salten, Felix 06.09.1869 – 08.10.1945@\textsc{Salten, Felix} (06.09.1869 – 08.10.1945), \emph{Schriftsteller, Journalist}!kleine Veronika. Novelle1902-12-01@\strich\emph{Die kleine Veronika. Novelle} {[}1902-12-01{]}|pw}. Wenn wir wieder in Wien\oindex{Wien@\textbf{Wien}|pw} sind, werde ich Ihnen, was Sie noch nicht kennen, vorlesen.
               Für {\pb}jetzt wäre es mir nur
               von Werth, wenn ich mit Fischer\pwindex{Fischer, Samuel 24.12.1859 – 15.10.1934@\textsc{Fischer, Samuel} (24.12.1859 – 15.10.1934), \emph{Verleger}|pw} principiell
               ins Reine komme, die Manuscripte schickte ich ihm dann von hier aus. Ich will nur,
               wenn ich einmal dort bin, die Sache persönlich betreiben können.\pend
           \pstart
           Wenn Sie glauben, dass ich recht habe, und wenn Sie soweit Sie sich meiner Novellen
               entsinnen, denken, dass ich es wagen kann, dann, bitte, sprechen sie mit Fischer\pwindex{Fischer, Samuel 24.12.1859 – 15.10.1934@\textsc{Fischer, Samuel} (24.12.1859 – 15.10.1934), \emph{Verleger}|pw},– natürlich nur, wenn es ihnen sonst
               nicht unbequem ist, mit ihn zu reden. In der Allg.
                  Zeitg\orgindex{Wiener Allgemeine Zeitung@Wiener Allgemeine Zeitung|pw} scheinen sich {\pb}Veränderungen vorzubereiten, nach welchen es fraglich wird, ob ich meine Stellung
               behalte. Doch davon mündlich. Haben Sie heute Max
                  Nordau\pwindex{Nordau, Max 29.07.1849 – 22.01.1923@\textsc{Nordau, Max} (29.07.1849 – 22.01.1923), \emph{Schriftsteller, Mediziner}|pw}{ }\label{K_L03181-876v}\edtext{über den Don Carlos\pwindex{\textcolor{red}{\textsuperscript{XXXX1 indx}}!Don Karlos, Infant von Spanien1787@\strich\emph{Don Karlos, Infant von Spanien} {[}1787{]}|pw}\pwindex{Nordau, Max 29.07.1849 – 22.01.1923@\textsc{Nordau, Max} (29.07.1849 – 22.01.1923), \emph{Schriftsteller, Mediziner}!Einiges ueber Schiller s »Don Carlos«1896-11-30@\strich\emph{Einiges über Schiller’s »Don Carlos«} {[}1896-11-30{]}|pwv}}{\lemma{\textnormal{\emph{über den Don Carlos}}}\Cendnote{\textnormal{Max Nordau\pwindex{Nordau, Max 29.07.1849 – 22.01.1923@\textsc{Nordau, Max} (29.07.1849 – 22.01.1923), \emph{Schriftsteller, Mediziner}|pwk}: \emph{Einiges über Schiller’s »Don Carlos«}\pwindex{Nordau, Max 29.07.1849 – 22.01.1923@\textsc{Nordau, Max} (29.07.1849 – 22.01.1923), \emph{Schriftsteller, Mediziner}!Einiges ueber Schiller s »Don Carlos«1896-11-30@\strich\emph{Einiges über Schiller’s »Don Carlos«} {[}1896-11-30{]}|pwk}. In: \emph{Neue Freie Presse}\pwindex{Neue Freie Presse1864 – 1939@\emph{Neue Freie Presse} {[}1864 – 1939{]}|pwk}, Nr. 11.561,
                        30. 10. 1896, Morgenblatt, S. 1–3.}}}\label{K_L03181-876h} gelesen? Er
               kommt sich riesig bahnbrechend vor. \pend
           \pstart
           Frl. M. II\pwindex{Reinhard, Marie 1871-03-13 – 1899-03-18@\textsc{Reinhard, Marie} (1871-03-13 – 1899-03-18), \emph{Gesangspädagogin}|pw} saß neulich im Burgtheater\oindex{Burgtheater@\textbf{Burgtheater}|pw} einige Reihen von mir, Mittelgang Ecke – fein!
               elejant! Und Jenny Singer\pwindex{Bally, Jenny *~22.03.1874@\textsc{Bally, Jenny} (*~22.03.1874)|pw} hat sich wieder
               einmal verlobt -\pend
           \pstart
           \label{T_L03181-1v}\edtext{Geheim:}{\lemma{\textnormal{\emph{Geheim:}}}\Cendnote{\textnormal{ohne Doppelpunkt, dafür mit Markierung durch einen Strich
                  seitlich am linken Rand des folgenden Absatzes}}}\label{T_L03181-1h}\pend
           \pstart
           Judith\pwindex{\textcolor{red}{\textsuperscript{XXXX1 indx}}!Judith. Eine Tragoedie in fuenf Aufzuegen1840-07-06@\strich\emph{Judith. Eine Tragödie in fünf Aufzügen} {[}1840-07-06{]}|pw} soll nicht aufgeführt werden, weil Frau
                  Mittwz.\pwindex{Mitterwurzer, Wilhelmine 27.03.1848 – 03.08.1909@\textsc{Mitterwurzer, Wilhelmine} (27.03.1848 – 03.08.1909), \emph{Schauspielerin}|pw} fürchtet, der Erfolg wird nicht
               gross genug sein, und Herr Mitterwurzer\pwindex{Mitterwurzer, Friedrich 16.10.1844 – 13.02.1897@\textsc{Mitterwurzer, Friedrich} (16.10.1844 – 13.02.1897), \emph{Schauspieler}|pw} trägt
               einen Revolver mit sich, mit dem er sich erschiessen will, weil er in seine Frau
               verliebt und auf den Cadetten eifersüchtig ist. \pend
           \pstart Herzlichst \spacefill\mbox{Salten.}\pend{}
         
         \endnumbering\mylabel{h}\end{ledgroupsized}\begin{anhang}\end{anhang}\newcommand{\dateiname}{L03181}\newcommand{\titel}{Felix Salten an Arthur Schnitzler, [Ende Oktober 1896]}\newcommand{\editorInnen}{Martin Anton Müller und Laura Untner}%% latex-leseansicht-abspann.tex
%% Abspann für die Leseansicht.
%% Der Schalter \ifkorrekturansicht ist bereits durch den Vorspann gesetzt.

%% latex-abspann.tex
%% Gemeinsamer Abspann für Korrekturansicht und Leseansicht.
%% Setzt den Schalter \ifkorrekturansicht voraus (gesetzt in den
%% einbindenden Dateien latex-korrekturansicht-abspann.tex bzw.
%% latex-leseansicht-abspann.tex).
%% ---------------------------------------------------------------

\normalsize

% Das esempio-Environment wird nur in der Leseansicht benötigt
\ifkorrekturansicht\else
\newenvironment{esempio}[3]%
{
    \vspace{1.5ex}
    \rlap{\underline{#1}}
    \par
    \setlength{\parindent}{0cm}
    \nopagebreak
    \leftskip=#2cm
    \rightskip=#3cm
}
{
    \par
}
\fi

\doendnotes{C}
\bigskip
\vfill

\clearpage

\footnotesize

\ifkorrekturansicht
  \lohead{\textsc{register}}
\fi

% theindex-Environment neu definieren ohne reledmac
\makeatletter
\renewenvironment{theindex}{%
  \ifkorrekturansicht
    \section*{\indexname}%
  \else
    \subsubsection*{Index der erwähnten Entitäten}%
  \fi
  \setlength{\parindent}{0pt}%
  \setlength{\parskip}{0pt plus 0.3pt}%
  \let\item\@idxitem
}{%
  \ifkorrekturansicht\clearpage\fi
}
\makeatother

\IfFileExists{\jobname-pw.ind}{\input{\jobname-pw.ind}}{}

% Quellenangabe nur in der Leseansicht
\ifkorrekturansicht\else
% Fallback-Definitionen, falls die .tex-Datei \titel etc. nicht gesetzt hat
\providecommand{\titel}{}
\providecommand{\editorInnen}{}
\providecommand{\dateiname}{\jobname}

\vspace{3cm}

\vfill

\footnotesize
\textsc{Quelle}: \titel. Herausgegeben von {\editorInnen}. In: \emph{Arthur Schnitzler: Briefwechsel mit Autorinnen und Autoren}.
 Digitale Edition, https://schnitzler-briefe.acdh.oeaw.ac.at/{\dateiname}.html (Stand \today)
\fi

\end{document}


      