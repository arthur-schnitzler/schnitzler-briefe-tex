%% latex-leseansicht-vorspann.tex
%% Vorspann für die Leseansicht.
%% Lädt die gemeinsame Datei latex-vorspann.tex mit nicht gesetztem Schalter.

\newif\ifkorrekturansicht
\korrekturansichtfalse

\input{../tex-inputs/latex-vorspann}


\section[Felix Salten an Arthur Schnitzler, {[}30. 10. 1896{]}]{L03181 Felix Salten an Arthur Schnitzler, {[}30. 10. 1896{]}}
\nopagebreak\mylabel{L03181v}
\rehead{ }\normalsize\beginnumbering\briefempfaengerindex{Schnitzler, Arthur@\textsc{Schnitzler, Arthur}!zzzSalten, Felix@\emph{von Felix Salten}!1896-10-303@{{[}30. 10. 1896{]}}|(be}
\toendnotes[C]{\smallbreak\pagebreak[2]}
\correspDesc{Versand  durch Felix Salten am [30. 10. 1896] in Wien
\newline{}Erhalt  durch Arthur Schnitzler im Zeitraum [31. 10. 1896 – 4. 11. 1896?] in Berlin}\toendnotes[C]{\smallbreak}
\Standort{CUL, Schnitzler, B 89, A 1.}
\physDesc{Brief, 1 Blatt, 4 Seiten, 2018 Zeichen
\newline{}Handschrift: Bleistift, lateinische Kurrent
\newline{}Schnitzler: mit Bleistift datiert: »Ende Oct 96« 
\newline{}Ordnung: mit Bleistift von unbekannter Hand nummeriert: »80« }\toendnotes[C]{\smallbreak}
\pstart
           \noindent{}{\pb}Lieber Arthur, ob die \label{K_L03181-1v}\edtext{Versti{\geminationm}ung über das Stück\pwindex{Schnitzler, Arthur 15.\,5.\,1862 Wien – 21.\,10.\,1931 ebd.@\textsc{Schnitzler, Arthur} (15.\,5.\,1862 Wien – 21.\,10.\,1931 ebd.), \emph{Schriftsteller, Mediziner}!Freiwild. Schauspiel in 3 Akten@\strich\emph{Freiwild. Schauspiel in 3 Akten}|pwv}}{\lemma{\textnormal{\emph{Verstimmung … Stück}}}\Cendnote{\textnormal{Vgl. A. S.: \emph{Tagebuch}, 28. 10. 1896.
               }}}\label{K_L03181-1} nicht jenes unangenehme Gefühl ist, das man i{\geminationm}er
               hat, wenn man fremde Leute zum ersten Mal eigene Worte aussprechen hört? Ich \label{K_L03181-2v}\edtext{fahre Montag{ }Abend von hier\oindex{Wien@\textbf{Wien}, \emph{Verwaltungsgebiet}|pwv} ab
               und bin also Dienstag{ }Mittag bei Ihnen}{\lemma{\textnormal{\emph{fahre … Ihnen}}}\Cendnote{\textnormal{Siehe A. S.: \emph{Tagebuch}, 3. 11. 1896.
               }}}\label{K_L03181-2}. Wenn es Ihre sonstigen Umstände zulaßen, und Sie es leicht können, möchte
               ich Sie um etwas bitten. Sprechen Sie vielleicht mit dem Verleger Fischer\pwindex{Fischer, Samuel 24.\,12.\,1859 Liptovský Mikuláš – 15.\,10.\,1934 Berlin@\textsc{Fischer, Samuel} (24.\,12.\,1859 Liptovský Mikuláš – 15.\,10.\,1934 Berlin), \emph{Verleger}|pw} von mir. Ich will endlich mein \label{K_L03181-3v}\edtext{Buch\pwindex{Salten, Felix 6.\,9.\,1869 Budapest – 8.\,10.\,1945 Zürich@\textsc{Salten, Felix} (6.\,9.\,1869 Budapest – 8.\,10.\,1945 Zürich), \emph{Schriftsteller, Journalist, Chefredakteur}!Hinterbliebene. Kurze Novellen@\strich\emph{Der Hinterbliebene. Kurze Novellen}|pwv}}{\lemma{\textnormal{\emph{Buch}}}\Cendnote{\textnormal{Die Novellensammlung \emph{Der Hinterbliebene}\pwindex{Salten, Felix 6.\,9.\,1869 Budapest – 8.\,10.\,1945 Zürich@\textsc{Salten, Felix} (6.\,9.\,1869 Budapest – 8.\,10.\,1945 Zürich), \emph{Schriftsteller, Journalist, Chefredakteur}!Hinterbliebene. Kurze Novellen@\strich\emph{Der Hinterbliebene. Kurze Novellen}|pwk} erschien erst 1900 im \emph{Wiener Verlag}\orgindex{Wiener Verlag@Wiener Verlag|pwk}. Aus der hier
                  projektierten Abmachung wurde also nichts.}}}\label{K_L03181-3} herausgeben. {\pb}Sie wissen, dass mich
               nicht innerliche Gründe dazu bestimmen, denn in der Stimmung, in der ich jetzt seit
               längerer Zeit lebe, möchte ich am liebsten Alles verbrennen. Aber ganz äußerlich
               brauche ich dieses Buch\pwindex{Salten, Felix 6.\,9.\,1869 Budapest – 8.\,10.\,1945 Zürich@\textsc{Salten, Felix} (6.\,9.\,1869 Budapest – 8.\,10.\,1945 Zürich), \emph{Schriftsteller, Journalist, Chefredakteur}!Hinterbliebene. Kurze Novellen@\strich\emph{Der Hinterbliebene. Kurze Novellen}|pwv} gerade
               jetzt, aus vielen Gründen, vor mir selbst und vor den Anderen. Ich habe meine Novellen\pwindex{Salten, Felix 6.\,9.\,1869 Budapest – 8.\,10.\,1945 Zürich@\textsc{Salten, Felix} (6.\,9.\,1869 Budapest – 8.\,10.\,1945 Zürich), \emph{Schriftsteller, Journalist, Chefredakteur}!Hinterbliebene. Kurze Novellen@\strich\emph{Der Hinterbliebene. Kurze Novellen}|pwv} fertig. Heldentod\pwindex{Salten, Felix 6.\,9.\,1869 Budapest – 8.\,10.\,1945 Zürich@\textsc{Salten, Felix} (6.\,9.\,1869 Budapest – 8.\,10.\,1945 Zürich), \emph{Schriftsteller, Journalist, Chefredakteur}!Heldentod@\strich\emph{Heldentod}|pw}, Hinterbliebener\pwindex{Salten, Felix 6.\,9.\,1869 Budapest – 8.\,10.\,1945 Zürich@\textsc{Salten, Felix} (6.\,9.\,1869 Budapest – 8.\,10.\,1945 Zürich), \emph{Schriftsteller, Journalist, Chefredakteur}!Hinterbliebene@\strich\emph{Der Hinterbliebene}|pw} – Flucht\pwindex{Salten, Felix 6.\,9.\,1869 Budapest – 8.\,10.\,1945 Zürich@\textsc{Salten, Felix} (6.\,9.\,1869 Budapest – 8.\,10.\,1945 Zürich), \emph{Schriftsteller, Journalist, Chefredakteur}!Flucht@\strich\emph{Flucht}|pw} – Cocotte u. Kellner\pwindex{Salten, Felix 6.\,9.\,1869 Budapest – 8.\,10.\,1945 Zürich@\textsc{Salten, Felix} (6.\,9.\,1869 Budapest – 8.\,10.\,1945 Zürich), \emph{Schriftsteller, Journalist, Chefredakteur}!Cocotte und Kellner@\strich\emph{Cocotte und Kellner}|pw} – Begräbnis\pwindex{Salten, Felix 6.\,9.\,1869 Budapest – 8.\,10.\,1945 Zürich@\textsc{Salten, Felix} (6.\,9.\,1869 Budapest – 8.\,10.\,1945 Zürich), \emph{Schriftsteller, Journalist, Chefredakteur}!Begräbnis@\strich\emph{Begräbnis}|pw} – Der Hund\pwindex{Salten, Felix 6.\,9.\,1869 Budapest – 8.\,10.\,1945 Zürich@\textsc{Salten, Felix} (6.\,9.\,1869 Budapest – 8.\,10.\,1945 Zürich), \emph{Schriftsteller, Journalist, Chefredakteur}!Hund@\strich\emph{Der Hund}|pw} –
                  Die Hochzeit auf dem Lande\pwindex{Salten, Felix 6.\,9.\,1869 Budapest – 8.\,10.\,1945 Zürich@\textsc{Salten, Felix} (6.\,9.\,1869 Budapest – 8.\,10.\,1945 Zürich), \emph{Schriftsteller, Journalist, Chefredakteur}!Hochzeit auf dem Lande@\strich\emph{Die Hochzeit auf dem Lande}|pw} – Die Confirmandin\pwindex{Salten, Felix 6.\,9.\,1869 Budapest – 8.\,10.\,1945 Zürich@\textsc{Salten, Felix} (6.\,9.\,1869 Budapest – 8.\,10.\,1945 Zürich), \emph{Schriftsteller, Journalist, Chefredakteur}!kleine Veronika@\strich\emph{Die kleine Veronika}|pw}. Wenn wir wieder in Wien\oindex{Wien@\textbf{Wien}, \emph{Verwaltungsgebiet}|pw} sind, werde ich Ihnen, \label{K_L03181-4v}\edtext{was Sie noch nicht kennen}{\lemma{\textnormal{\emph{was … kennen}}}\Cendnote{\textnormal{Nachweislich hatte Salten\pwindex{Salten, Felix 6.\,9.\,1869 Budapest – 8.\,10.\,1945 Zürich@\textsc{Salten, Felix} (6.\,9.\,1869 Budapest – 8.\,10.\,1945 Zürich), \emph{Schriftsteller, Journalist, Chefredakteur}|pwk}{ }Schnitzler bereits \emph{Begräbnis}\pwindex{Salten, Felix 6.\,9.\,1869 Budapest – 8.\,10.\,1945 Zürich@\textsc{Salten, Felix} (6.\,9.\,1869 Budapest – 8.\,10.\,1945 Zürich), \emph{Schriftsteller, Journalist, Chefredakteur}!Begräbnis@\strich\emph{Begräbnis}|pwk} (18. 5. 1893), \emph{Der
                     Hinterbliebene}\pwindex{Salten, Felix 6.\,9.\,1869 Budapest – 8.\,10.\,1945 Zürich@\textsc{Salten, Felix} (6.\,9.\,1869 Budapest – 8.\,10.\,1945 Zürich), \emph{Schriftsteller, Journalist, Chefredakteur}!Hinterbliebene@\strich\emph{Der Hinterbliebene}|pwk} (19. 4. 1894) und \emph{Heldentod}\pwindex{Salten, Felix 6.\,9.\,1869 Budapest – 8.\,10.\,1945 Zürich@\textsc{Salten, Felix} (6.\,9.\,1869 Budapest – 8.\,10.\,1945 Zürich), \emph{Schriftsteller, Journalist, Chefredakteur}!Heldentod@\strich\emph{Heldentod}|pwk} (31. 7. 1894)
                  vorgelesen.}}}\label{K_L03181-4}, vorlesen. Für {\pb}jetzt wäre es mir nur von
               Werth, wenn ich mit Fischer\pwindex{Fischer, Samuel 24.\,12.\,1859 Liptovský Mikuláš – 15.\,10.\,1934 Berlin@\textsc{Fischer, Samuel} (24.\,12.\,1859 Liptovský Mikuláš – 15.\,10.\,1934 Berlin), \emph{Verleger}|pw} principiell ins
               Reine komme, die Manuscripte\pwindex{Salten, Felix 6.\,9.\,1869 Budapest – 8.\,10.\,1945 Zürich@\textsc{Salten, Felix} (6.\,9.\,1869 Budapest – 8.\,10.\,1945 Zürich), \emph{Schriftsteller, Journalist, Chefredakteur}!Hinterbliebene. Kurze Novellen@\strich\emph{Der Hinterbliebene. Kurze Novellen}|pwv}
               schickte ich ihm dann von hier\oindex{Wien@\textbf{Wien}, \emph{Verwaltungsgebiet}|pwv}
               aus. Ich will nur, wenn ich einmal dort\oindex{Berlin@\textbf{Berlin}, \emph{Hauptstadt}|pwv} bin, die Sache persönlich betreiben können.\pend
           
\pstart
           Wenn Sie glauben, dass ich recht habe, und wenn Sie soweit Sie sich meiner Novellen\pwindex{Salten, Felix 6.\,9.\,1869 Budapest – 8.\,10.\,1945 Zürich@\textsc{Salten, Felix} (6.\,9.\,1869 Budapest – 8.\,10.\,1945 Zürich), \emph{Schriftsteller, Journalist, Chefredakteur}!Hinterbliebene. Kurze Novellen@\strich\emph{Der Hinterbliebene. Kurze Novellen}|pwv} entsinnen, denken,
               dass ich es wagen kann, dann, bitte, sprechen Sie mit Fischer\pwindex{Fischer, Samuel 24.\,12.\,1859 Liptovský Mikuláš – 15.\,10.\,1934 Berlin@\textsc{Fischer, Samuel} (24.\,12.\,1859 Liptovský Mikuláš – 15.\,10.\,1934 Berlin), \emph{Verleger}|pw}, – natürlich nur, wenn es Ihnen sonst nicht unbequem
               ist, mit ihm zu reden. In der Allg. Zeitg\orgindex{Wiener Allgemeine Zeitung@Wiener Allgemeine Zeitung|pw}
               scheinen sich {\pb}Veränderungen
               vorzubereiten, nach welchen es fraglich wird, ob ich meine \label{K_L03181-5v}\edtext{Stellung behalte}{\lemma{\textnormal{\emph{Stellung behalte}}}\Cendnote{\textnormal{Salten\pwindex{Salten, Felix 6.\,9.\,1869 Budapest – 8.\,10.\,1945 Zürich@\textsc{Salten, Felix} (6.\,9.\,1869 Budapest – 8.\,10.\,1945 Zürich), \emph{Schriftsteller, Journalist, Chefredakteur}|pwk} blieb bis Ende Juni 1902 in der Redaktion der \emph{Wiener Allgemeinen Zeitung}\orgindex{Wiener Allgemeine Zeitung@Wiener Allgemeine Zeitung|pwk}.}}}\label{K_L03181-5}. Doch davon mündlich. Haben Sie heute{ }Max Nordau\pwindex{Nordau, Max 29.\,7.\,1849 Budapest – 22.\,1.\,1923 Paris@\textsc{Nordau, Max} (29.\,7.\,1849 Budapest – 22.\,1.\,1923 Paris), \emph{Schriftsteller, Mediziner}|pw}{ }\label{K_L03181-6v}\edtext{über den Don Carlos\pwindex{\textcolor{red}{\textsuperscript{XXXX indx1}}!Don Karlos, Infant von Spanien@\strich\emph{Don Karlos, Infant von Spanien}|pw}\pwindex{Nordau, Max 29.\,7.\,1849 Budapest – 22.\,1.\,1923 Paris@\textsc{Nordau, Max} (29.\,7.\,1849 Budapest – 22.\,1.\,1923 Paris), \emph{Schriftsteller, Mediziner}!Einiges über Schiller’s »Don Carlos«@\strich\emph{Einiges über Schiller’s »Don Carlos«}|pwv}}{\lemma{\textnormal{\emph{über den Don Carlos}}}\Cendnote{\textnormal{Max Nordau\pwindex{Nordau, Max 29.\,7.\,1849 Budapest – 22.\,1.\,1923 Paris@\textsc{Nordau, Max} (29.\,7.\,1849 Budapest – 22.\,1.\,1923 Paris), \emph{Schriftsteller, Mediziner}|pwk}: \emph{Einiges über Schiller’s »Don Carlos«}\pwindex{Nordau, Max 29.\,7.\,1849 Budapest – 22.\,1.\,1923 Paris@\textsc{Nordau, Max} (29.\,7.\,1849 Budapest – 22.\,1.\,1923 Paris), \emph{Schriftsteller, Mediziner}!Einiges über Schiller’s »Don Carlos«@\strich\emph{Einiges über Schiller’s »Don Carlos«}|pwk}. In: \emph{Neue Freie Presse}\pwindex{Neue Freie Presse@\emph{Neue Freie Presse}|pwk}, Nr. 11.561, 30. 10. 1896, Morgenblatt, S. 1–3. Durch
                  diesen Verweis ist der Brief datierbar.}}}\label{K_L03181-6} gelesen? Er ko\textcolor{gray}{{\geminationm}}t sich riesig
               bahnbrechend vor. Frl. \label{K_L03181-7v}\edtext{M. II.\pwindex{Reinhard, Marie 13.\,3.\,1871 Wien – 18.\,3.\,1899 ebd.@\textsc{Reinhard, Marie} (13.\,3.\,1871 Wien – 18.\,3.\,1899 ebd.), \emph{Gesangspädagogin}|pw}}{\lemma{\textnormal{\emph{M. II.}}}\Cendnote{\textnormal{Marie Reinhard\pwindex{Reinhard, Marie 13.\,3.\,1871 Wien – 18.\,3.\,1899 ebd.@\textsc{Reinhard, Marie} (13.\,3.\,1871 Wien – 18.\,3.\,1899 ebd.), \emph{Gesangspädagogin}|pwk}}}}\label{K_L03181-7} saß neulich im Burgtheater\oindex{Wien@\textbf{Wien}!I., Innere Stadt@\textbf{I., Innere Stadt}!Burgtheater@\textbf{Burgtheater}, \emph{Theater}|pw} einige Reihen
               von mir, Mittelgang Ecke – fein! elejant! und Jenny
                  Singer\pwindex{Bally, Jenny *~22.\,3.\,1874 Wien@\textsc{Bally, Jenny} (*~22.\,3.\,1874 Wien)|pw} hat sich wieder einmal \label{K_L03181-8v}\edtext{verlobt\pwindex{Bally, Isidore S. *~24.\,4.\,1864 Bukarest@\textsc{Bally, Isidore S.} (*~24.\,4.\,1864 Bukarest), \emph{Journalist, Rentier, Bankier}|pwv}}{\lemma{\textnormal{\emph{verlobt}}}\Cendnote{\textnormal{Jenny Singer\pwindex{Bally, Jenny *~22.\,3.\,1874 Wien@\textsc{Bally, Jenny} (*~22.\,3.\,1874 Wien)|pwk} hatte sich mit Isidor S. Bally\pwindex{Bally, Isidore S. *~24.\,4.\,1864 Bukarest@\textsc{Bally, Isidore S.} (*~24.\,4.\,1864 Bukarest), \emph{Journalist, Rentier, Bankier}|pwk} verlobt. Sie heirateten am
                     27. 12. 1896.}}}\label{K_L03181-8}\textcolor{gray}{.}\pend
           
\pstart
           \label{T_L03181-1v}\edtext{Geheim:}{\lemma{\textnormal{\emph{Geheim:}}}\Cendnote{\textnormal{ohne Doppelpunkt, dafür mit Markierung durch eine Klammer
                  seitlich am linken Rand des folgenden Absatzes}}}\label{T_L03181-1}\pend
           
\pstart
           Judith\pwindex{Hebbel, Friedrich 18.\,3.\,1813 Wesselburen – 13.\,12.\,1863 Wien@\textsc{Hebbel, Friedrich} (18.\,3.\,1813 Wesselburen – 13.\,12.\,1863 Wien), \emph{Schriftsteller}!Judith. Eine Tragödie in fünf Aufzügen@\strich\emph{Judith. Eine Tragödie in fünf Aufzügen}|pw} soll \label{K_L03181-9v}\edtext{nicht aufgeführt}{\lemma{\textnormal{\emph{nicht aufgeführt}}}\Cendnote{\textnormal{Friedrich Hebbels\pwindex{Hebbel, Friedrich 18.\,3.\,1813 Wesselburen – 13.\,12.\,1863 Wien@\textsc{Hebbel, Friedrich} (18.\,3.\,1813 Wesselburen – 13.\,12.\,1863 Wien), \emph{Schriftsteller}|pwk} Fünfakter \emph{Judith}\pwindex{Hebbel, Friedrich 18.\,3.\,1813 Wesselburen – 13.\,12.\,1863 Wien@\textsc{Hebbel, Friedrich} (18.\,3.\,1813 Wesselburen – 13.\,12.\,1863 Wien), \emph{Schriftsteller}!Judith. Eine Tragödie in fünf Aufzügen@\strich\emph{Judith. Eine Tragödie in fünf Aufzügen}|pwk} wurde trotzdem aufgeführt. Schnitzler besuchte die Vorstellung am 13. 11. 1896.}}}\label{K_L03181-9}
               werden, weil Frau Mittwz.\pwindex{Mitterwurzer, Wilhelmine 27.\,3.\,1848 Freiburg im Breisgau – 3.\,8.\,1909 Wien@\textsc{Mitterwurzer, Wilhelmine} (27.\,3.\,1848 Freiburg im Breisgau – 3.\,8.\,1909 Wien), \emph{Schauspielerin}|pw} fürchtet, der Erfolg
               wird nicht gross genug sein, und Herr Mitterwurzer\pwindex{Mitterwurzer, Friedrich 16.\,10.\,1844 Dresden – 13.\,2.\,1897 Wien@\textsc{Mitterwurzer, Friedrich} (16.\,10.\,1844 Dresden – 13.\,2.\,1897 Wien), \emph{Schauspieler}|pw} trägt einen Revolver mit sich, mit dem er sich erschiessen
               will, weil er in seine Frau\pwindex{Mitterwurzer, Wilhelmine 27.\,3.\,1848 Freiburg im Breisgau – 3.\,8.\,1909 Wien@\textsc{Mitterwurzer, Wilhelmine} (27.\,3.\,1848 Freiburg im Breisgau – 3.\,8.\,1909 Wien), \emph{Schauspielerin}|pwv}
               verliebt und auf den \label{K_L03181-10v}\edtext{Cadetten\pwindex{?? [Kadett] @\textsc{?? [Kadett]}|pwv}}{\lemma{\textnormal{\emph{Cadetten}}}\Cendnote{\textnormal{nicht ermittelt}}}\label{K_L03181-10} eifersüchtig
               ist.\pend
           \pstart herzlichst \spacefill\mbox{Salten}\pend{}\selectlanguage{ngerman}\endnumbering\briefempfaengerindex{Schnitzler, Arthur@\textsc{Schnitzler, Arthur}!zzzSalten, Felix@\emph{von Felix Salten}!1896-10-303@{{[}30. 10. 1896{]}}|)be}\mylabel{L03181h}  \newcommand{\dateiname}{L03181}\newcommand{\titel}{Felix Salten an Arthur Schnitzler, [30. 10. 1896]}\newcommand{\editorInnen}{Martin Anton Müller und Laura Untner}%% latex-leseansicht-abspann.tex
%% Abspann für die Leseansicht.
%% Der Schalter \ifkorrekturansicht ist bereits durch den Vorspann gesetzt.

%% latex-abspann.tex
%% Gemeinsamer Abspann für Korrekturansicht und Leseansicht.
%% Setzt den Schalter \ifkorrekturansicht voraus (gesetzt in den
%% einbindenden Dateien latex-korrekturansicht-abspann.tex bzw.
%% latex-leseansicht-abspann.tex).
%% ---------------------------------------------------------------

\normalsize

% Das esempio-Environment wird nur in der Leseansicht benötigt
\ifkorrekturansicht\else
\newenvironment{esempio}[3]%
{
    \vspace{1.5ex}
    \rlap{\underline{#1}}
    \par
    \setlength{\parindent}{0cm}
    \nopagebreak
    \leftskip=#2cm
    \rightskip=#3cm
}
{
    \par
}
\fi

\doendnotes{C}
\bigskip
\vfill

\clearpage

\footnotesize

\ifkorrekturansicht
  \lohead{\textsc{register}}
\fi

% theindex-Environment neu definieren ohne reledmac
\makeatletter
\renewenvironment{theindex}{%
  \ifkorrekturansicht
    \section*{\indexname}%
  \else
    \subsubsection*{Index der erwähnten Entitäten}%
  \fi
  \setlength{\parindent}{0pt}%
  \setlength{\parskip}{0pt plus 0.3pt}%
  \let\item\@idxitem
}{%
  \ifkorrekturansicht\clearpage\fi
}
\makeatother

\IfFileExists{\jobname-pw.ind}{\input{\jobname-pw.ind}}{}

% Quellenangabe nur in der Leseansicht
\ifkorrekturansicht\else
% Fallback-Definitionen, falls die .tex-Datei \titel etc. nicht gesetzt hat
\providecommand{\titel}{}
\providecommand{\editorInnen}{}
\providecommand{\dateiname}{\jobname}

\vspace{3cm}

\vfill

\footnotesize
\textsc{Quelle}: \titel. Herausgegeben von {\editorInnen}. In: \emph{Arthur Schnitzler: Briefwechsel mit Autorinnen und Autoren}.
 Digitale Edition, https://schnitzler-briefe.acdh.oeaw.ac.at/{\dateiname}.html (Stand \today)
\fi

\end{document}


