%% latex-leseansicht-vorspann.tex
%% Vorspann für die Leseansicht.
%% Lädt die gemeinsame Datei latex-vorspann.tex mit nicht gesetztem Schalter.

\newif\ifkorrekturansicht
\korrekturansichtfalse

\input{../tex-inputs/latex-vorspann}


         
         \renewcommand{\erwaehntePersonen}{Personen:  ?? [Kadett], Jenny Bally, Isidor S. Bally, Samuel Fischer, Friedrich Hebbel, Wilhelmine Mitterwurzer, Friedrich Mitterwurzer, Max Nordau, Marie Reinhard, Felix Salten}
         \renewcommand{\erwaehnteInstitutionen}{Institutionen: Wiener Allgemeine Zeitung, Wiener Verlag}
         \renewcommand{\erwaehnteOrte}{Orte: Berlin, Burgtheater, Wien}
         \renewcommand{\erwaehnteWerke}{Werke: Begräbnis, Cocotte und Kellner, Der Hinterbliebene, Der Hinterbliebene. Kurze Novellen, Der Hund, Die Hochzeit auf dem Lande, Die kleine Veronika, Don Karlos, Infant von Spanien, Einiges über Schiller’s »Don Carlos«, Flucht, Freiwild. Schauspiel in 3 Akten, Heldentod, Judith. Eine Tragödie in fünf Aufzügen, Neue Freie Presse}
               \section[Felix Salten an Arthur Schnitzler, {[}30. 10. 1896{]}]{ Felix Salten an Arthur Schnitzler, {[}30. 10. 1896{]}}\nopagebreak\mylabel{v}\rehead{ }\begin{ledgroupsized}[t]{13cm}\normalsize\beginnumbering\briefempfaengerindex{Schnitzler, Arthur@\textsc{Schnitzler, Arthur}!zzzSalten, Felix@\emph{von Felix Salten}!1896-10-303@{{[}30. 10. 1896{]}}|(be} \toendnotes[C]{\smallbreak\pagebreak[2]} \Standort{CUL, Schnitzler, B 89, A 1.}
\physDesc{Brief, 1 Blatt, 4 Seiten, 2018 Zeichen
\newline{}Handschrift: Bleistift, lateinische Kurrent
\newline{}Schnitzler: mit Bleistift datiert: »Ende Oct 96« 
\newline{}Ordnung: mit Bleistift von unbekannter Hand nummeriert: »80« }\toendnotes[C]{\smallbreak}\pstart
           \noindent{}{\pb}Lieber Arthur, ob die \label{K_L03181-1v}\edtext{Versti{\geminationm}ung über das Stück\pwindex{Schnitzler, Arthur 15.05.1862 – 21.10.1931@\textsc{Schnitzler, Arthur} (15.05.1862 – 21.10.1931), \emph{Schriftsteller, Mediziner}!Freiwild. Schauspiel in 3 Akten1896@\strich\emph{Freiwild. Schauspiel in 3 Akten} {[}1896{]}|pwv}}{\lemma{\textnormal{\emph{Verstimmung … Stück}}}\Cendnote{\textnormal{Vgl. A. S.: \emph{Tagebuch}, 28. 10. 1896.
               }}}\label{K_L03181-1h} nicht jenes unangenehme Gefühl ist, das man i{\geminationm}er
               hat, wenn man fremde Leute zum ersten Mal eigene Worte aussprechen hört? Ich \label{K_L03181-2v}\edtext{fahre Montag{ }Abend von hier\oindex{Wien@\textbf{Wien}|pwv} ab
               und bin also Dienstag{ }Mittag bei Ihnen}{\lemma{\textnormal{\emph{fahre … Ihnen}}}\Cendnote{\textnormal{Siehe A. S.: \emph{Tagebuch}, 3. 11. 1896.
               }}}\label{K_L03181-2h}. Wenn es Ihre sonstigen Umstände zulaßen, und Sie es leicht können, möchte
               ich Sie um etwas bitten. Sprechen Sie vielleicht mit dem Verleger Fischer\pwindex{Fischer, Samuel 24.12.1859 – 15.10.1934@\textsc{Fischer, Samuel} (24.12.1859 – 15.10.1934), \emph{Verleger}|pw} von mir. Ich will endlich mein \label{K_L03181-3v}\edtext{Buch\pwindex{Salten, Felix 06.09.1869 – 08.10.1945@\textsc{Salten, Felix} (06.09.1869 – 08.10.1945), \emph{Schriftsteller, Journalist, Chefredakteur}!Hinterbliebene. Kurze Novellen1900@\strich\emph{Der Hinterbliebene. Kurze Novellen} {[}1900{]}|pwv}}{\lemma{\textnormal{\emph{Buch}}}\Cendnote{\textnormal{Die Novellensammlung \emph{Der Hinterbliebene}\pwindex{Salten, Felix 06.09.1869 – 08.10.1945@\textsc{Salten, Felix} (06.09.1869 – 08.10.1945), \emph{Schriftsteller, Journalist, Chefredakteur}!Hinterbliebene. Kurze Novellen1900@\strich\emph{Der Hinterbliebene. Kurze Novellen} {[}1900{]}|pwk} erschien erst 1900 im \emph{Wiener Verlag}\orgindex{Wiener Verlag@Wiener Verlag|pwk}. Aus der hier
                  projektierten Abmachung wurde also nichts.}}}\label{K_L03181-3h} herausgeben. {\pb}Sie wissen, dass mich
               nicht innerliche Gründe dazu bestimmen, denn in der Stimmung, in der ich jetzt seit
               längerer Zeit lebe, möchte ich am liebsten Alles verbrennen. Aber ganz äußerlich
               brauche ich dieses Buch\pwindex{Salten, Felix 06.09.1869 – 08.10.1945@\textsc{Salten, Felix} (06.09.1869 – 08.10.1945), \emph{Schriftsteller, Journalist, Chefredakteur}!Hinterbliebene. Kurze Novellen1900@\strich\emph{Der Hinterbliebene. Kurze Novellen} {[}1900{]}|pwv} gerade
               jetzt, aus vielen Gründen, vor mir selbst und vor den Anderen. Ich habe meine Novellen\pwindex{Salten, Felix 06.09.1869 – 08.10.1945@\textsc{Salten, Felix} (06.09.1869 – 08.10.1945), \emph{Schriftsteller, Journalist, Chefredakteur}!Hinterbliebene. Kurze Novellen1900@\strich\emph{Der Hinterbliebene. Kurze Novellen} {[}1900{]}|pwv} fertig. Heldentod\pwindex{Salten, Felix 06.09.1869 – 08.10.1945@\textsc{Salten, Felix} (06.09.1869 – 08.10.1945), \emph{Schriftsteller, Journalist, Chefredakteur}!Heldentod1895-01-01@\strich\emph{Heldentod} {[}1895-01-01{]}|pw}, Hinterbliebener\pwindex{Salten, Felix 06.09.1869 – 08.10.1945@\textsc{Salten, Felix} (06.09.1869 – 08.10.1945), \emph{Schriftsteller, Journalist, Chefredakteur}!Hinterbliebene1899-03-04 – 1899-03-11@\strich\emph{Der Hinterbliebene} {[}1899-03-04 – 1899-03-11{]}|pw} – Flucht\pwindex{Salten, Felix 06.09.1869 – 08.10.1945@\textsc{Salten, Felix} (06.09.1869 – 08.10.1945), \emph{Schriftsteller, Journalist, Chefredakteur}!Flucht1899-07-31@\strich\emph{Flucht} {[}1899-07-31{]}|pw} – Cocotte u. Kellner\pwindex{Salten, Felix 06.09.1869 – 08.10.1945@\textsc{Salten, Felix} (06.09.1869 – 08.10.1945), \emph{Schriftsteller, Journalist, Chefredakteur}!Cocotte und Kellner@\strich\emph{Cocotte und Kellner}|pw} – Begräbnis\pwindex{Salten, Felix 06.09.1869 – 08.10.1945@\textsc{Salten, Felix} (06.09.1869 – 08.10.1945), \emph{Schriftsteller, Journalist, Chefredakteur}!Begraebnis17. 7. 1893@\strich\emph{Begräbnis} {[}17. 7. 1893{]}|pw} – Der Hund\pwindex{Salten, Felix 06.09.1869 – 08.10.1945@\textsc{Salten, Felix} (06.09.1869 – 08.10.1945), \emph{Schriftsteller, Journalist, Chefredakteur}!Hund@\strich\emph{Der Hund}|pw} –
                  Die Hochzeit auf dem Lande\pwindex{Salten, Felix 06.09.1869 – 08.10.1945@\textsc{Salten, Felix} (06.09.1869 – 08.10.1945), \emph{Schriftsteller, Journalist, Chefredakteur}!Hochzeit auf dem Lande@\strich\emph{Die Hochzeit auf dem Lande}|pw} – Die Confirmandin\pwindex{Salten, Felix 06.09.1869 – 08.10.1945@\textsc{Salten, Felix} (06.09.1869 – 08.10.1945), \emph{Schriftsteller, Journalist, Chefredakteur}!kleine Veronika1902-12-01@\strich\emph{Die kleine Veronika} {[}1902-12-01{]}|pw}. Wenn wir wieder in Wien\oindex{Wien@\textbf{Wien}|pw} sind, werde ich Ihnen, \label{K_L03181-4v}\edtext{was Sie noch nicht kennen}{\lemma{\textnormal{\emph{was … kennen}}}\Cendnote{\textnormal{Nachweislich hatte Salten\pwindex{Salten, Felix 06.09.1869 – 08.10.1945@\textsc{Salten, Felix} (06.09.1869 – 08.10.1945), \emph{Schriftsteller, Journalist, Chefredakteur}|pwk}{ }Schnitzler\pwindex{Schnitzler, Arthur 15.05.1862 – 21.10.1931@\textsc{Schnitzler, Arthur} (15.05.1862 – 21.10.1931), \emph{Schriftsteller, Mediziner}|pwk} bereits \emph{Begräbnis}\pwindex{Salten, Felix 06.09.1869 – 08.10.1945@\textsc{Salten, Felix} (06.09.1869 – 08.10.1945), \emph{Schriftsteller, Journalist, Chefredakteur}!Begraebnis17. 7. 1893@\strich\emph{Begräbnis} {[}17. 7. 1893{]}|pwk} (18. 5. 1893), \emph{Der
                     Hinterbliebene}\pwindex{Salten, Felix 06.09.1869 – 08.10.1945@\textsc{Salten, Felix} (06.09.1869 – 08.10.1945), \emph{Schriftsteller, Journalist, Chefredakteur}!Hinterbliebene1899-03-04 – 1899-03-11@\strich\emph{Der Hinterbliebene} {[}1899-03-04 – 1899-03-11{]}|pwk} (19. 4. 1894) und \emph{Heldentod}\pwindex{Salten, Felix 06.09.1869 – 08.10.1945@\textsc{Salten, Felix} (06.09.1869 – 08.10.1945), \emph{Schriftsteller, Journalist, Chefredakteur}!Heldentod1895-01-01@\strich\emph{Heldentod} {[}1895-01-01{]}|pwk} (31. 7. 1894)
                  vorgelesen.}}}\label{K_L03181-4h}, vorlesen. Für {\pb}jetzt wäre es mir nur von
               Werth, wenn ich mit Fischer\pwindex{Fischer, Samuel 24.12.1859 – 15.10.1934@\textsc{Fischer, Samuel} (24.12.1859 – 15.10.1934), \emph{Verleger}|pw} principiell ins
               Reine komme, die Manuscripte\pwindex{Salten, Felix 06.09.1869 – 08.10.1945@\textsc{Salten, Felix} (06.09.1869 – 08.10.1945), \emph{Schriftsteller, Journalist, Chefredakteur}!Hinterbliebene. Kurze Novellen1900@\strich\emph{Der Hinterbliebene. Kurze Novellen} {[}1900{]}|pwv}
               schickte ich ihm dann von hier\oindex{Wien@\textbf{Wien}|pwv}
               aus. Ich will nur, wenn ich einmal dort\oindex{Berlin@\textbf{Berlin}|pwv} bin, die Sache persönlich betreiben können.\pend
           \pstart
           Wenn Sie glauben, dass ich recht habe, und wenn Sie soweit Sie sich meiner Novellen\pwindex{Salten, Felix 06.09.1869 – 08.10.1945@\textsc{Salten, Felix} (06.09.1869 – 08.10.1945), \emph{Schriftsteller, Journalist, Chefredakteur}!Hinterbliebene. Kurze Novellen1900@\strich\emph{Der Hinterbliebene. Kurze Novellen} {[}1900{]}|pwv} entsinnen, denken,
               dass ich es wagen kann, dann, bitte, sprechen Sie mit Fischer\pwindex{Fischer, Samuel 24.12.1859 – 15.10.1934@\textsc{Fischer, Samuel} (24.12.1859 – 15.10.1934), \emph{Verleger}|pw}, – natürlich nur, wenn es Ihnen sonst nicht unbequem
               ist, mit ihm zu reden. In der Allg. Zeitg\orgindex{Wiener Allgemeine Zeitung@Wiener Allgemeine Zeitung|pw}
               scheinen sich {\pb}Veränderungen
               vorzubereiten, nach welchen es fraglich wird, ob ich meine \label{K_L03181-5v}\edtext{Stellung behalte}{\lemma{\textnormal{\emph{Stellung behalte}}}\Cendnote{\textnormal{Salten\pwindex{Salten, Felix 06.09.1869 – 08.10.1945@\textsc{Salten, Felix} (06.09.1869 – 08.10.1945), \emph{Schriftsteller, Journalist, Chefredakteur}|pwk} blieb bis Ende Juni 1902 in der Redaktion der \emph{Wiener Allgemeinen Zeitung}\orgindex{Wiener Allgemeine Zeitung@Wiener Allgemeine Zeitung|pwk}.}}}\label{K_L03181-5h}. Doch davon mündlich. Haben Sie heute{ }Max Nordau\pwindex{Nordau, Max 29.07.1849 – 22.01.1923@\textsc{Nordau, Max} (29.07.1849 – 22.01.1923), \emph{Schriftsteller, Mediziner}|pw}{ }\label{K_L03181-6v}\edtext{über den Don Carlos\pwindex{\textcolor{red}{\textsuperscript{XXXX1 indx}}!Don Karlos, Infant von Spanien1787@\strich\emph{Don Karlos, Infant von Spanien} {[}1787{]}|pw}\pwindex{Nordau, Max 29.07.1849 – 22.01.1923@\textsc{Nordau, Max} (29.07.1849 – 22.01.1923), \emph{Schriftsteller, Mediziner}!Einiges ueber Schiller s »Don Carlos«1896-11-30@\strich\emph{Einiges über Schiller’s »Don Carlos«} {[}1896-11-30{]}|pwv}}{\lemma{\textnormal{\emph{über den Don Carlos}}}\Cendnote{\textnormal{Max Nordau\pwindex{Nordau, Max 29.07.1849 – 22.01.1923@\textsc{Nordau, Max} (29.07.1849 – 22.01.1923), \emph{Schriftsteller, Mediziner}|pwk}: \emph{Einiges über Schiller’s »Don Carlos«}\pwindex{Nordau, Max 29.07.1849 – 22.01.1923@\textsc{Nordau, Max} (29.07.1849 – 22.01.1923), \emph{Schriftsteller, Mediziner}!Einiges ueber Schiller s »Don Carlos«1896-11-30@\strich\emph{Einiges über Schiller’s »Don Carlos«} {[}1896-11-30{]}|pwk}. In: \emph{Neue Freie Presse}\pwindex{Neue Freie Presse1864 – 1939@\emph{Neue Freie Presse} {[}1864 – 1939{]}|pwk}, Nr. 11.561, 30. 10. 1896, Morgenblatt, S. 1–3. Durch
                  diesen Verweis ist der Brief datierbar.}}}\label{K_L03181-6h} gelesen? Er ko\textcolor{gray}{{\geminationm}}t sich riesig
               bahnbrechend vor. Frl. \label{K_L03181-7v}\edtext{M. II.\pwindex{Reinhard, Marie 1871-03-13 – 1899-03-18@\textsc{Reinhard, Marie} (1871-03-13 – 1899-03-18), \emph{Gesangspädagogin}|pw}}{\lemma{\textnormal{\emph{M. II.}}}\Cendnote{\textnormal{Marie Reinhard\pwindex{Reinhard, Marie 1871-03-13 – 1899-03-18@\textsc{Reinhard, Marie} (1871-03-13 – 1899-03-18), \emph{Gesangspädagogin}|pwk}}}}\label{K_L03181-7h} saß neulich im Burgtheater\oindex{Burgtheater@\textbf{Burgtheater}|pw} einige Reihen
               von mir, Mittelgang Ecke – fein! elejant! und Jenny
                  Singer\pwindex{Bally, Jenny *~22.03.1874@\textsc{Bally, Jenny} (*~22.03.1874)|pw} hat sich wieder einmal \label{K_L03181-8v}\edtext{verlobt\pwindex{Bally, Isidor S. *~1864@\textsc{Bally, Isidor S.} (*~1864), \emph{Journalist}|pwv}}{\lemma{\textnormal{\emph{verlobt}}}\Cendnote{\textnormal{Jenny Singer\pwindex{Bally, Jenny *~22.03.1874@\textsc{Bally, Jenny} (*~22.03.1874)|pwk} hatte sich mit Isidor S. Bally\pwindex{Bally, Isidor S. *~1864@\textsc{Bally, Isidor S.} (*~1864), \emph{Journalist}|pwk} verlobt. Sie heirateten am
                     27. 12. 1896.}}}\label{K_L03181-8h}\textcolor{gray}{.}\pend
           \pstart
           \label{T_L03181-1v}\edtext{Geheim:}{\lemma{\textnormal{\emph{Geheim:}}}\Cendnote{\textnormal{ohne Doppelpunkt, dafür mit Markierung durch eine Klammer
                  seitlich am linken Rand des folgenden Absatzes}}}\label{T_L03181-1h}\pend
           \pstart
           Judith\pwindex{Hebbel, Friedrich 18.03.1813 – 13.12.1863@\textsc{Hebbel, Friedrich} (18.03.1813 – 13.12.1863), \emph{Schriftsteller}!Judith. Eine Tragoedie in fuenf Aufzuegen1840-07-06@\strich\emph{Judith. Eine Tragödie in fünf Aufzügen} {[}1840-07-06{]}|pw} soll \label{K_L03181-9v}\edtext{nicht aufgeführt}{\lemma{\textnormal{\emph{nicht aufgeführt}}}\Cendnote{\textnormal{Friedrich Hebbels\pwindex{Hebbel, Friedrich 18.03.1813 – 13.12.1863@\textsc{Hebbel, Friedrich} (18.03.1813 – 13.12.1863), \emph{Schriftsteller}|pwk} Fünfakter \emph{Judith}\pwindex{Hebbel, Friedrich 18.03.1813 – 13.12.1863@\textsc{Hebbel, Friedrich} (18.03.1813 – 13.12.1863), \emph{Schriftsteller}!Judith. Eine Tragoedie in fuenf Aufzuegen1840-07-06@\strich\emph{Judith. Eine Tragödie in fünf Aufzügen} {[}1840-07-06{]}|pwk} wurde trotzdem aufgeführt. Schnitzler\pwindex{Schnitzler, Arthur 15.05.1862 – 21.10.1931@\textsc{Schnitzler, Arthur} (15.05.1862 – 21.10.1931), \emph{Schriftsteller, Mediziner}|pwk} besuchte die Vorstellung am 13. 11. 1896.}}}\label{K_L03181-9h}
               werden, weil Frau Mittwz.\pwindex{Mitterwurzer, Wilhelmine 27.03.1848 – 03.08.1909@\textsc{Mitterwurzer, Wilhelmine} (27.03.1848 – 03.08.1909), \emph{Schauspielerin}|pw} fürchtet, der Erfolg
               wird nicht gross genug sein, und Herr Mitterwurzer\pwindex{Mitterwurzer, Friedrich 16.10.1844 – 13.02.1897@\textsc{Mitterwurzer, Friedrich} (16.10.1844 – 13.02.1897), \emph{Schauspieler}|pw} trägt einen Revolver mit sich, mit dem er sich erschiessen
               will, weil er in seine Frau\pwindex{Mitterwurzer, Wilhelmine 27.03.1848 – 03.08.1909@\textsc{Mitterwurzer, Wilhelmine} (27.03.1848 – 03.08.1909), \emph{Schauspielerin}|pwv}
               verliebt und auf den \label{K_L03181-10v}\edtext{Cadetten\pwindex{?? [Kadett] @\textsc{?? [Kadett]}|pwv}}{\lemma{\textnormal{\emph{Cadetten}}}\Cendnote{\textnormal{nicht ermittelt}}}\label{K_L03181-10h} eifersüchtig
               ist.\pend
           \pstart herzlichst \spacefill\mbox{Salten}\pend{}
         
         \endnumbering\mylabel{h}\end{ledgroupsized}  \newcommand{\dateiname}{L03181}\newcommand{\titel}{Felix Salten an Arthur Schnitzler, [30. 10. 1896]}\newcommand{\editorInnen}{Martin Anton Müller und Laura Untner}%% latex-leseansicht-abspann.tex
%% Abspann für die Leseansicht.
%% Der Schalter \ifkorrekturansicht ist bereits durch den Vorspann gesetzt.

%% latex-abspann.tex
%% Gemeinsamer Abspann für Korrekturansicht und Leseansicht.
%% Setzt den Schalter \ifkorrekturansicht voraus (gesetzt in den
%% einbindenden Dateien latex-korrekturansicht-abspann.tex bzw.
%% latex-leseansicht-abspann.tex).
%% ---------------------------------------------------------------

\normalsize

% Das esempio-Environment wird nur in der Leseansicht benötigt
\ifkorrekturansicht\else
\newenvironment{esempio}[3]%
{
    \vspace{1.5ex}
    \rlap{\underline{#1}}
    \par
    \setlength{\parindent}{0cm}
    \nopagebreak
    \leftskip=#2cm
    \rightskip=#3cm
}
{
    \par
}
\fi

\doendnotes{C}
\bigskip
\vfill

\clearpage

\footnotesize

\ifkorrekturansicht
  \lohead{\textsc{register}}
\fi

% theindex-Environment neu definieren ohne reledmac
\makeatletter
\renewenvironment{theindex}{%
  \ifkorrekturansicht
    \section*{\indexname}%
  \else
    \subsubsection*{Index der erwähnten Entitäten}%
  \fi
  \setlength{\parindent}{0pt}%
  \setlength{\parskip}{0pt plus 0.3pt}%
  \let\item\@idxitem
}{%
  \ifkorrekturansicht\clearpage\fi
}
\makeatother

\IfFileExists{\jobname-pw.ind}{\input{\jobname-pw.ind}}{}

% Quellenangabe nur in der Leseansicht
\ifkorrekturansicht\else
% Fallback-Definitionen, falls die .tex-Datei \titel etc. nicht gesetzt hat
\providecommand{\titel}{}
\providecommand{\editorInnen}{}
\providecommand{\dateiname}{\jobname}

\vspace{3cm}

\vfill

\footnotesize
\textsc{Quelle}: \titel. Herausgegeben von {\editorInnen}. In: \emph{Arthur Schnitzler: Briefwechsel mit Autorinnen und Autoren}.
 Digitale Edition, https://schnitzler-briefe.acdh.oeaw.ac.at/{\dateiname}.html (Stand \today)
\fi

\end{document}


      