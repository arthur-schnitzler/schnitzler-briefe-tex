%% latex-korrekturansicht-vorspann.tex
%% Vorspann für die Korrekturansicht.
%% Lädt die gemeinsame Datei latex-vorspann.tex mit gesetztem Schalter.

\newif\ifkorrekturansicht
\korrekturansichttrue

\input{../tex-inputs/latex-vorspann}


\section[Elsa Plessner an Arthur Schnitzler, 1. 12. 1896]{L03708 Elsa Plessner an Arthur Schnitzler, 1. 12. 1896}
\nopagebreak\mylabel{L03708v}
\rehead{ }\normalsize\beginnumbering\briefempfaengerindex{Schnitzler, Arthur@\textsc{Schnitzler, Arthur}!zzzPlessner, Elsa@\emph{von Elsa Plessner}!1896-12-011@{1. 12. 1896}|(be}
\toendnotes[C]{\smallbreak\pagebreak[2]}\Standort{DLA, A:Schnitzler, HS.1985.1.419.}
\physDesc{Brief,  Blätter, 3 Seiten, 2234 Zeichen
\newline{}Handschrift: , lateinische Kurrent}\toendnotes[C]{\smallbreak}
\pstart
           {\pb}Meran, Pension Wolf\oindex{Hotel Meranerhof@\textbf{Hotel Meranerhof}, \emph{Hotel (K.HTL)}|pw}, den 1. December
                     1896\pend
           
\pstart
           Motto: »Unverschämt! – Was? –« (?)\pend
           
\pstart{}Hochverehrter Herr Doctor!\pend\vspace{0.5em}
\pstart
           \label{K_L03708-1v}\edtext{Beifolgenden Brief}{\lemma{\textnormal{\emph{Beifolgenden Brief}}}\Cendnote{\textnormal{Der abschlägige Brief von Otto Brahm\pwindex{Brahm, Otto 05.02.1856 – 28.11.1912@\textsc{Brahm, Otto} (05.02.1856 – 28.11.1912), \emph{Theaterleiter/Theaterleiterin, Regisseur/Regisseurin}|pwk}, Leiter des
                     \emph{Deutschen Theaters}\orgindex{Deutsches Theater Berlin@Deutsches Theater Berlin|pwk} in Berlin\oindex{Berlin@\textbf{Berlin}, \emph{P.PPLC}|pwk} ist nicht überliefert. Plessner\pwindex{Plessner, Elsa 22.08.1875 – 01.05.1932@\textsc{Plessner, Elsa} (22.08.1875 – 01.05.1932), \emph{Schriftsteller/Schriftstellerin}|pwk}
                 hatte ihm in Absprache mit Schnitzler ihr Theaterstück
                     \emph{Heimweh}\pwindex{Heimweh [dreiaktige Tragikomoedie]@\emph{Heimweh [dreiaktige Tragikomödie]}|pwk} eingesendet und zur Aufführung
                  angeboten, vgl. Elsa Plessner an Arthur Schnitzler, 28. 9. 1896.}}}\label{K_L03708-1} erhielt ich – gestern von meiner Mama\pwindex{Plessner, Clementine 1855-12-07 – 1943-02-27@\textsc{Plessner, Clementine} (1855-12-07 – 1943-02-27), \emph{Schauspieler/Schauspielerin, Filmschauspieler/Filmschauspielerin}|pw} zugesendet, nachdem sie ihn vierzehn Tage
               lang aus »Rücksicht für meinen Gesundheitszustand« und zu dessen Schonung
               zurückbehalten hat und erst \strikeout{auf} die Erwähnung
               meinerseits dort (B\pwindex{Brahm, Otto 05.02.1856 – 28.11.1912@\textsc{Brahm, Otto} (05.02.1856 – 28.11.1912), \emph{Theaterleiter/Theaterleiterin, Regisseur/Regisseurin}|pwv})
               nochmals angefragt zu haben hat sie veranlasst, ihn herauszugeben. Die »Schonung«, an
               und für sich überflüssig, ist in diesem Fall gar nicht angebracht, denn ich habe ja
               dieses Resultat täglich erwartet und das sage ich ganz ehrlich!! – Sie wissen ja! –
               Die Pille, so {\pb}liebenswürdig in einer verbindlichen Oblate, (medicinisch
               richtig! – was?) hat mich durchaus nicht niedergeschmettert. \uline{Vergleich ausgeschlossen}, kam die »Athenerin\pwindex{Athenerin. Drama in drei Aufzuegen@\emph{Die Athenerin. Drama in drei Aufzügen}|pw}« 4 mal von dort zurück!! \introOben{}\uuline{\edtext{sagt}{\Cendnote{vierfach unterstrichen}}} man!!\introOben{} Ganz \begin{otherlanguage}{french}\label{K_L03708-2v}\edtext{entre nous}{\lemma{\textnormal{\emph{entre nous}}}\Cendnote{\textnormal{französisch: unter uns}}}\label{K_L03708-2}{ }\end{otherlanguage} gesagt; sah ich bei der letzten Lecture meines Opus\pwindex{Heimweh [dreiaktige Tragikomoedie]@\emph{Heimweh [dreiaktige Tragikomödie]}|pwv} Schwächen die ich früher nie gesehen
               habe! \begin{otherlanguage}{french}\label{K_L03708-3v}\edtext{Chose agreable}{\lemma{\textnormal{\emph{Chose agreable}}}\Cendnote{\textnormal{französisch: angenehme Sache}}}\label{K_L03708-3}{ }\end{otherlanguage} – d. h. ich bin drüber hinaus gewachsen. Um so angenehmer, da neues Stück\pwindex{Orchideen [Schauspiel in drei Akten]@\emph{Orchideen [Schauspiel in drei Akten]}|pwv} vor mir! – Hoffe gut! –
                  \begin{otherlanguage}{italian}\label{K_L03708-4v}\edtext{Vederemo}{\lemma{\textnormal{\emph{Vederemo}}}\Cendnote{\textnormal{italienisch vedremo: wir werden sehen}}}\label{K_L03708-4}{ }\end{otherlanguage}! – – – – Hauptsache – was mache ich jetzt mit »Heimweh\pwindex{Heimweh [dreiaktige Tragikomoedie]@\emph{Heimweh [dreiaktige Tragikomödie]}|pw}«.  – Bitte, bitte, guten Rath!! – Bühne? – Keine
               Lust {\pb}glaube auch aussichtslos. – Was nun? – Wenn Sie so gut sein
               wollten, mir einen guten Rath zu geben – – – Verlag? – S. Fischer\orgindex{S. Fischer Verlag@S. Fischer Verlag|pw}?{\dots}{ }A. Langen\orgindex{Albert Langen@Albert Langen|pw} hat es im Vorjahr der Marholm\pwindex{Marholm, Laura 19.04.1854 – 06.10.1928@\textsc{Marholm, Laura} (19.04.1854 – 06.10.1928), \emph{Schriftsteller/Schriftstellerin}|pw}{ }\label{K_L03708-5v}\edtext{refusirt}{\lemma{\textnormal{\emph{refusirt}}}\Cendnote{\textnormal{Tatsächlich war Laura
                     Marholms\pwindex{Marholm, Laura 19.04.1854 – 06.10.1928@\textsc{Marholm, Laura} (19.04.1854 – 06.10.1928), \emph{Schriftsteller/Schriftstellerin}|pwk} Drama \emph{Karla Bühring}\pwindex{Karla Buehring. Ein Frauendrama in vier Acten@\emph{Karla Bühring. Ein Frauendrama in vier Acten}|pwk}{ }1895 bei \emph{A. Langen}\orgindex{Albert Langen@Albert Langen|pwk}
                  erschienen.}}}\label{K_L03708-5}!! Möchte doch \uline{so} gern hinaus! –
               Vielleicht kindisch – »ein Buch!« Wirklich und wahrhaftig ein gedrucktes Buch!!  –
               Alte Leidenschaft von mir! – Drum – aber wahr! – Lachen Sie so herzlich Sie wollen,
               verehrter Herr Doctor, ich lache auch mit – da liegt mir gar nichts dran – aber
               rathen Sie mir!! – – – – – – Richtig! – Nochmals herzlichsten Dank für Ihre gütige
               Intervention bei Dir. B.\pwindex{Brahm, Otto 05.02.1856 – 28.11.1912@\textsc{Brahm, Otto} (05.02.1856 – 28.11.1912), \emph{Theaterleiter/Theaterleiterin, Regisseur/Regisseurin}|pw}! – Wenn Sie jetzt, wo
               die schöne Wiener\oindex{Wien@\textbf{Wien}, \emph{A.ADM2}|pw} Saison, aus der ich mich bis zum
                  Frühjahr selber verbannt habe, so prächtig im Gange ist, ein paar
               Augenblicke für mich Zeit finden, so packen Sie sie beim Schopf und senden ein paar
               Zeilen als Strahlen der Literatursonne an einer armen, bleichsüchtigen Blaustrumpf
               und die werden mir mehr Freude bereiten, als die \introOben{}der\introOben{}{ }Meraner\oindex{Meran@\textbf{Meran}, \emph{P.PPLA3}|pw} Sonne, die auf so viel krankes
               Menschenzeug herabstrahlen müssen. – – Bitte! – Ja? – Was mach ich also?! – \pend
           
\pstart
           Voraus Dank mit zwei Dutzend Ausrufungszeichen – ergebenste Grüße{\\[\baselineskip]}\spacefill\mbox{Elsa Plessner}\pend
           \leftskip=0em{}\selectlanguage{ngerman}\endnumbering\briefempfaengerindex{Schnitzler, Arthur@\textsc{Schnitzler, Arthur}!zzzPlessner, Elsa@\emph{von Elsa Plessner}!1896-12-011@{1. 12. 1896}|)be}\mylabel{L03708h}
\begin{anhang}
\end{anhang}\normalsize

\doendnotes{C}
\bigskip
\vfill

\clearpage

\footnotesize

\lohead{\textsc{register}}

% Definiere theindex-Environment komplett neu ohne reledmac
\makeatletter
\renewenvironment{theindex}{%
  \section*{\indexname}%
  \setlength{\parindent}{0pt}%
  \setlength{\parskip}{0pt plus 0.3pt}%
  \let\item\@idxitem
}{%
  \clearpage
}
\makeatother

\IfFileExists{\jobname-pw.ind}{\input{\jobname-pw.ind}}{}

\end{document}

      