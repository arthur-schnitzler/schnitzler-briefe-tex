%% latex-korrekturansicht-vorspann.tex
%% Vorspann für die Korrekturansicht.
%% Lädt die gemeinsame Datei latex-vorspann.tex mit gesetztem Schalter.

\newif\ifkorrekturansicht
\korrekturansichttrue

\input{../tex-inputs/latex-vorspann}


\section[Hugo von Hofmannsthal an Arthur Schnitzler, 1. 11. {[}1907{]}]{L01727 Hugo von Hofmannsthal an Arthur Schnitzler, 1. 11. {[}1907{]}}
\nopagebreak\mylabel{L01727v}
\rehead{ }\normalsize\beginnumbering\briefempfaengerindex{Schnitzler, Arthur@\textsc{Schnitzler, Arthur}!zzzHofmannsthal, Hugo von@\emph{von Hugo von Hofmannsthal}!1907-11-011@{1. 11. {[}1907{]}}|(be}
\toendnotes[C]{\smallbreak\pagebreak[2]}\Standort{CUL, Schnitzler, B 43.}
\physDesc{Brief, 1 Blatt, 4 Seiten, 1769 Zeichen
\newline{}Handschrift: schwarze Tinte, deutsche Kurrent
\newline{}Schnitzler: mit Bleistift die Jahreszahl ergänzt: »90\textcolor{gray}{7}« 
\newline{}Ordnung: mit Bleistift von unbekannter Hand nummeriert:
                                    »187a« und beschriftet:
                                 »?Date?« }
\buchAbdrucke{\weitereDrucke{Hugo von Hofmannsthal, Arthur Schnitzler: \emph{Briefwechsel}. Frankfurt am Main: \emph{S. Fischer} 1964, S. 232–233.} }\toendnotes[C]{\smallbreak}
\pstart
           \raggedleft{}{\pb}R.\oindex{Rodaun@\textbf{Rodaun}, \emph{A.ADM4}|pw}, 1. XI.\pend
           
\pstart{}mein guter Arthur,\pend\vspace{0.5em}
\pstart
           wir kämen ja ſehr gern – aber ich arbeite jetzt (ungefähr ſeit 2 Wochen) jeden
               Vormittag jeden Abend. Durch einen Abend bei Euch verlöre ich einen Abend und den
               nächſten Vormittag (und vielleicht durch Nervoſität mehr als das) alſo muſs ich
               leider verzichten.\pend
           
\pstart
           Nicht wahr Sie bringen das Geſpräch dann mit Auernheimer\pwindex{Auernheimer, Raoul 15.04.1876 – 06.01.1948@\textsc{Auernheimer, Raoul} (15.04.1876 – 06.01.1948), \emph{Schriftsteller/Schriftstellerin, Journalist/Journalistin, Kritiker/Kritikerin}|pw} auf mich und ſpeciell darauf, daſs er den »\label{K_L01727-1v}\edtext{Rodauner Aeſtheten\pwindex{Herr von Balthesser@\emph{Der Herr von Balthesser}|pwv}}{\lemma{\textnormal{\emph{Rodauner Aeſtheten}}}\Cendnote{\textnormal{Auernheimer\pwindex{Auernheimer, Raoul 15.04.1876 – 06.01.1948@\textsc{Auernheimer, Raoul} (15.04.1876 – 06.01.1948), \emph{Schriftsteller/Schriftstellerin, Journalist/Journalistin, Kritiker/Kritikerin}|pwk} schreibt in seiner Rezension
                  von Richard von Schaukals\pwindex{Schaukal, Richard von 27.05.1874 – 10.10.1942@\textsc{Schaukal, Richard von} (27.05.1874 – 10.10.1942), \emph{Schriftsteller/Schriftstellerin, Kunstkritiker/Kunstkritikerin}|pwk}{ }\emph{Leben und Meinungen des Herrn Andreas von
                     Balthesser}\pwindex{Leben und Meinungen des Herrn Andreas von Balthesser@\emph{Leben und Meinungen des Herrn Andreas von Balthesser}|pwk}: »Der Rodauner Ästhet\pwindex{Hofmannsthal, Hugo von 1874-02-01 – 1929-07-15@\textsc{Hofmannsthal, Hugo von} (1874-02-01 – 1929-07-15), \emph{Schriftsteller/Schriftstellerin}|pwv} geht ihm sogar entgegen und macht
                     dem neu Angekommenen ein Kompliment über sein jüngstes Buch.«. Raoul Auernheimer\pwindex{Auernheimer, Raoul 15.04.1876 – 06.01.1948@\textsc{Auernheimer, Raoul} (15.04.1876 – 06.01.1948), \emph{Schriftsteller/Schriftstellerin, Journalist/Journalistin, Kritiker/Kritikerin}|pwk}: \emph{Der Herr von Balthesser}\pwindex{Herr von Balthesser@\emph{Der Herr von Balthesser}|pwk}. In: \emph{Neue Freie Presse}\pwindex{Neue Freie Presse@\emph{Neue Freie Presse}|pwk}, Nr. 15.462, 8. 9. 1907,
                     Morgenblatt, S. 1–3, hier S. 1.
               }}}\label{K_L01727-1}« {\pb}anführte als eine Figur
               die von Schaukal\pwindex{Schaukal, Richard von 27.05.1874 – 10.10.1942@\textsc{Schaukal, Richard von} (27.05.1874 – 10.10.1942), \emph{Schriftsteller/Schriftstellerin, Kunstkritiker/Kunstkritikerin}|pw} entzückt iſt und der Schaukal\pwindex{Schaukal, Richard von 27.05.1874 – 10.10.1942@\textsc{Schaukal, Richard von} (27.05.1874 – 10.10.1942), \emph{Schriftsteller/Schriftstellerin, Kunstkritiker/Kunstkritikerin}|pw} für ſeinen Dreck (um den ſich das Feuilleton\pwindex{Herr von Balthesser@\emph{Der Herr von Balthesser}|pwv} dreht)
               becomplimentiert. Fragen Sie ihn bitte welche meiner Arbeiten einer ähnlichen
               Characterisierung die Handhabe bietet.\pend
           
\pstart
           Ich habe es ſo ſatt, nach 17 Jahren ziemlich ernſthaften Arbeitens in dieſer Weiſe
               »ironiſiert« zu werden – und in dieſem Fall iſt es ja kein Lausbub\pwindex{Auernheimer, Raoul 15.04.1876 – 06.01.1948@\textsc{Auernheimer, Raoul} (15.04.1876 – 06.01.1948), \emph{Schriftsteller/Schriftstellerin, Journalist/Journalistin, Kritiker/Kritikerin}|pwv}, ſondern jemand anſcheinend
               Anſtändiger. Also wozu?\pend
           
\pstart
           {\pb}Mein Stück\pwindex{Silvia im »Stern«@\emph{Silvia im »Stern«}|pwv} iſt ein recht ſonderbares Ding. Wenns
               nicht miſslingt – iſt es viel wert, für mich meine ich. Jedenfalls gehen mir hie und
               da einige Ahnungen auf darüber wie das was man die Leute reden läſst wieder
               zurückwirkt auf die ſogenannte Handlung (das Scenarium) u. ſ. f. u. ſ. f.\hspace*{1.5em}Sehr einſam iſt man in ſolchen Momenten, wie tief in
               einem Bergwerk nur im Finſtern {\pb}irgendwo neben ſich, aber weit, glaubt man einen andern hämmern zu hören.\hspace*{1.5em}Sie z. B.\hspace*{1.5em}So habe ich
               neulich den erſten Act vom »Ruf des Lebens\pwindex{Ruf des Lebens. Schauspiel in drei Akten@\emph{Der Ruf des Lebens. Schauspiel in drei Akten}|pw}« ſehr
               aufmerkſam geleſen, mit viel Gewinn (vielleicht auch für Sie.) Ich glaube das
               notwendige organiſche Stück ſteckt hier (wie natürlich){[}.{]} Sie
               ſind aber wie mit geſchloſſenen Augen darüber hinweggegangen. (In der Scene Marie\pwindex{Ruf des Lebens. Schauspiel in drei Akten@\emph{Der Ruf des Lebens. Schauspiel in drei Akten}|pwv}–Adjunct\pwindex{Ruf des Lebens. Schauspiel in drei Akten@\emph{Der Ruf des Lebens. Schauspiel in drei Akten}|pwv}{ }ſteckt die \uline{Idee} des
               Stückes.) Davon nächſtens.\pend
           
\pstart
           Ich glaube ich werde Sie plötzlich \uline{brauchen}, zu
               Hilfe.\pend
           
\pstart
           Adieu.{\\[\baselineskip]}Ihr\spacefill\mbox{Hugo.}\pend
           \leftskip=0em{}
\pstart
           \noindent{}\label{T_L01727-1v}\edtext{Ich wüſste gern, wie denn überhaupt
                     A.\pwindex{Auernheimer, Raoul 15.04.1876 – 06.01.1948@\textsc{Auernheimer, Raoul} (15.04.1876 – 06.01.1948), \emph{Schriftsteller/Schriftstellerin, Journalist/Journalistin, Kritiker/Kritikerin}|pw} zu meinen Arbeiten ſteht, z. B. den
                     proſaiſchen.}{\lemma{\textnormal{\emph{Ich … proſaiſchen.}}}\Cendnote{\textnormal{quer am linken Rand der
                     zweiten Seite}}}\label{T_L01727-1}\pend
           \selectlanguage{ngerman}\endnumbering\briefempfaengerindex{Schnitzler, Arthur@\textsc{Schnitzler, Arthur}!zzzHofmannsthal, Hugo von@\emph{von Hugo von Hofmannsthal}!1907-11-011@{1. 11. {[}1907{]}}|)be}\mylabel{L01727h}  \normalsize

\doendnotes{C}
\bigskip
\vfill

\clearpage

\footnotesize

\lohead{\textsc{register}}

% Definiere theindex-Environment komplett neu ohne reledmac
\makeatletter
\renewenvironment{theindex}{%
  \section*{\indexname}%
  \setlength{\parindent}{0pt}%
  \setlength{\parskip}{0pt plus 0.3pt}%
  \let\item\@idxitem
}{%
  \clearpage
}
\makeatother

\IfFileExists{\jobname-pw.ind}{\input{\jobname-pw.ind}}{}

\end{document}

      