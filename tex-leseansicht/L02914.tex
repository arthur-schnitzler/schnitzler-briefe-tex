%% latex-leseansicht-vorspann.tex
%% Vorspann für die Leseansicht.
%% Lädt die gemeinsame Datei latex-vorspann.tex mit nicht gesetztem Schalter.

\newif\ifkorrekturansicht
\korrekturansichtfalse

\input{../tex-inputs/latex-vorspann}

\begin{center}
            \textcolor{red}{ENTWURF, NICHT FERTIG KORRIGIERT}
                      \end{center}
            
         
         \renewcommand{\erwaehntePersonen}{Personen: A. Beit, Georg von Bleichröder, Rosa Freudenthal, Lydia von Fulmen, Johan Cesar Godeffroy, Carl von Lang-Puchhof, Karl August von Schmieder, W. Warne}
         \renewcommand{\erwaehnteInstitutionen}{Institutionen: Rennstall Lang-Puchhof und Schmieder}
         \renewcommand{\erwaehnteOrte}{Orte: Berlin, Cadore, Dessauer Straße, Deutschland, Hamburg, Hamburg-Groß Borstel, Hoppegarten, Köln, Over Norton, Rheinland, Wien}
         \renewcommand{\erwaehnteWerke}{Werke: Liebelei. Schauspiel in drei Akten, Nemzeti dal, Reigen. Zehn Dialoge, Ritterlichkeit, [Rennbericht, Pferd Liebelei]}
               \section[ Paul Goldmann an Arthur Schnitzler, 2. 5. {[}1900{]}]{ Paul Goldmann an Arthur Schnitzler, 2. 5. {[}1900{]}}\nopagebreak\mylabel{v}\rehead{ }\begin{ledgroupsized}[t]{13cm}\normalsize\beginnumbering \toendnotes[C]{\smallbreak\pagebreak[2]} \Standort{DLA, A:Schnitzler, HS.NZ85.1.3170.}
\physDesc{Brief, 1 Blatt, 3 Seiten
\newline{}Handschrift: blaue Tinte, deutsche Kurrent\newline{}Beilage: ein Zeitungsausschnitt, beschnitten }\toendnotes[C]{\smallbreak}\pstart{}{\pb}\textcolor{gray}{\textbf{DESSAUERSTRASSE 19}}\oindex{Dessauer Strasse@\textbf{Dessauer Straße}|pw}\pend{}{\bigskip}\pstart
           \raggedleft{}Berlin\oindex{Berlin@\textbf{Berlin}|pw}, 2. Mai.\pend
           \pstart{}Mein lieber Freund,\pend\pstart
           In aller Eile Dank für Deinen lieben Brief! Mich hat die \label{K_L02914-1v}\edtext{Frau
                  Rechtsanwalt\pwindex{Freudenthal, Rosa 1862 – 18.06.1905@\textsc{Freudenthal, Rosa} (1862 – 18.06.1905)|pwv}}{\lemma{\textnormal{\emph{Frau
                  Rechtsanwalt}}}\Cendnote{\textnormal{siehe Paul Goldmann an Arthur Schnitzler, 20. 2. 1900}}}\label{K_L02914-1h} um den »Reigen\pwindex{Schnitzler, Arthur 15.05.1862 – 21.10.1931@\textsc{Schnitzler, Arthur} (15.05.1862 – 21.10.1931), \emph{Schriftsteller, Mediziner}!Reigen. Zehn Dialoge1900@\strich\emph{Reigen. Zehn Dialoge} {[}1900{]}|pw}« erſucht. Ich hielt mich
               aber nicht für berechtigt, der Frau\pwindex{Freudenthal, Rosa 1862 – 18.06.1905@\textsc{Freudenthal, Rosa} (1862 – 18.06.1905)|pwv} das Buch\pwindex{Schnitzler, Arthur 15.05.1862 – 21.10.1931@\textsc{Schnitzler, Arthur} (15.05.1862 – 21.10.1931), \emph{Schriftsteller, Mediziner}!Reigen. Zehn Dialoge1900@\strich\emph{Reigen. Zehn Dialoge} {[}1900{]}|pw} zu geben, und habe mich
               damit ausgeredet, ich hätte es verborgt.\pend
           \pstart
           Wie Du aus beifolgendem \label{K_L02914-3v}\edtext{Rennbericht\pwindex{?? Werk@Nicht ermittelte Verfasserinnen und Verfasser!Rennbericht, Pferd Liebelei]1900-05@\emph{[Rennbericht, Pferd Liebelei]} {[}1900-05{]}|pwv}}{\lemma{\textnormal{\emph{Rennbericht}}}\Cendnote{\textnormal{es ist unklar, aus welcher Zeitung der
                     Ausschnitt\pwindex{?? Werk@Nicht ermittelte Verfasserinnen und Verfasser!Rennbericht, Pferd Liebelei]1900-05@\emph{[Rennbericht, Pferd Liebelei]} {[}1900-05{]}|pwkv}
                  stammt}}}\label{K_L02914-3h} ſiehſt, iſt hier beim letzten Rennen ein Pferd »Liebelei\pwindex{Schnitzler, Arthur 15.05.1862 – 21.10.1931@\textsc{Schnitzler, Arthur} (15.05.1862 – 21.10.1931), \emph{Schriftsteller, Mediziner}!Liebelei. Schauspiel in drei Akten1895-10-09@\strich\emph{Liebelei. Schauspiel in drei Akten} {[}1895-10-09{]}|pw}« gelaufen. Es {\pb}gehört \label{K_L02914-11v}\edtext{einem ſüddeutſch\oindex{Deutschland@\textbf{Deutschland}|pwv}en Beſitzer\pwindex{Lang-Puchhof, Carl von 1854-09-18 – 1916-04-07@\textsc{Lang-Puchhof, Carl von} (1854-09-18 – 1916-04-07), \emph{Gutsbesitzer, Rennstallbesitzer, Jurist}|pwv}\pwindex{Schmieder, Karl August von 1867-05-29 – 1941-03-06@\textsc{Schmieder, Karl August von} (1867-05-29 – 1941-03-06), \emph{Rennstallbesitzer}|pwv}}{\lemma{\textnormal{\emph{einem … Beſitzer}}}\Cendnote{\textnormal{Das Pferd »\emph{Liebelei}\pwindex{Schnitzler, Arthur 15.05.1862 – 21.10.1931@\textsc{Schnitzler, Arthur} (15.05.1862 – 21.10.1931), \emph{Schriftsteller, Mediziner}!Liebelei. Schauspiel in drei Akten1895-10-09@\strich\emph{Liebelei. Schauspiel in drei Akten} {[}1895-10-09{]}|pwk}« gehörte Carl
                     von Lang-Puchhof\pwindex{Lang-Puchhof, Carl von 1854-09-18 – 1916-04-07@\textsc{Lang-Puchhof, Carl von} (1854-09-18 – 1916-04-07), \emph{Gutsbesitzer, Rennstallbesitzer, Jurist}|pwk} und Karl August von
                     Schmieder\pwindex{Schmieder, Karl August von 1867-05-29 – 1941-03-06@\textsc{Schmieder, Karl August von} (1867-05-29 – 1941-03-06), \emph{Rennstallbesitzer}|pwk}, die von 1898 bis 1907 einen Pferderennstall\orgindex{Rennstall Lang-Puchhof und Schmieder@Rennstall Lang-Puchhof und Schmieder|pwkv} in Hoppegarten\oindex{Hoppegarten@\textbf{Hoppegarten}|pwk}
                  betrieben. Goldmann\pwindex{Goldmann, Paul 31.01.1865 – 25.09.1935@\textsc{Goldmann, Paul} (31.01.1865 – 25.09.1935), \emph{Schriftsteller, Journalist}|pwk} bezog sich vermutlich
                  auf den Rheinländ\oindex{Rheinland@\textbf{Rheinland}|pwkv}er Lang-Puchhof\pwindex{Lang-Puchhof, Carl von 1854-09-18 – 1916-04-07@\textsc{Lang-Puchhof, Carl von} (1854-09-18 – 1916-04-07), \emph{Gutsbesitzer, Rennstallbesitzer, Jurist}|pwk}.}}}\label{K_L02914-11h} und heißt offenbar
               nach Deinem Stück\pwindex{Schnitzler, Arthur 15.05.1862 – 21.10.1931@\textsc{Schnitzler, Arthur} (15.05.1862 – 21.10.1931), \emph{Schriftsteller, Mediziner}!Liebelei. Schauspiel in drei Akten1895-10-09@\strich\emph{Liebelei. Schauspiel in drei Akten} {[}1895-10-09{]}|pwv}. Dies iſt
               der Ruhm, mein lieber Freund!\pend
           \pstart
           Es freut mich ſehr, zu hören, daß Du eine \label{K_L02914-5v}\edtext{Poſſe\pwindex{Schnitzler, Arthur 15.05.1862 – 21.10.1931@\textsc{Schnitzler, Arthur} (15.05.1862 – 21.10.1931), \emph{Schriftsteller, Mediziner}!Ritterlichkeit1977@\strich\emph{Ritterlichkeit} {[}1977{]}|pwuv}}{\lemma{\textnormal{\emph{Poſſe}}}\Cendnote{\textnormal{womöglich Bezug auf das Fragment
                  gebliebene und erst posthum veröffentlichte Drama \emph{Ritterlichkeit}\pwindex{Schnitzler, Arthur 15.05.1862 – 21.10.1931@\textsc{Schnitzler, Arthur} (15.05.1862 – 21.10.1931), \emph{Schriftsteller, Mediziner}!Ritterlichkeit1977@\strich\emph{Ritterlichkeit} {[}1977{]}|pwk}, das Schnitzler\pwindex{Schnitzler, Arthur 15.05.1862 – 21.10.1931@\textsc{Schnitzler, Arthur} (15.05.1862 – 21.10.1931), \emph{Schriftsteller, Mediziner}|pwk} am
                     23. 4. 1900
                  vorläufig unter dem Titel »Drama\pwindex{Schnitzler, Arthur 15.05.1862 – 21.10.1931@\textsc{Schnitzler, Arthur} (15.05.1862 – 21.10.1931), \emph{Schriftsteller, Mediziner}!Ritterlichkeit1977@\strich\emph{Ritterlichkeit} {[}1977{]}|pwkv}« beendet hatte}}}\label{K_L02914-5h} geſchrieben haſt. So biſt Du \strikeout{\textcolor{gray}{×}} auf halbem Wege zu dem \label{K_L02914-7v}\edtext{Luſtſpiel}{\lemma{\textnormal{\emph{Luſtſpiel}}}\Cendnote{\textnormal{vgl. Paul Goldmann an Arthur Schnitzler, 8. 12. [1893], Paul Goldmann an Arthur Schnitzler, 23. 12. [1893] und Paul Goldmann an Arthur Schnitzler, 2. [1.? 1897]}}}\label{K_L02914-7h}, das ich nicht ablaſſen werde, von Dir zu verlangen.\pend
           \pstart
           Nächſtens mehr! Heut habe ich nur zwei Minuten.\pend
           \pstart
           {\pb}Viele treue Grüße! {\\[\baselineskip]}Dein {\\[\baselineskip]}\spacefill\mbox{Paul Goldmann}\pend
           \leftskip=0em{}\pstart
           \textcolor{gray}{\textbf{{\pb}Unter den Pferden, die bereits »was gezeigt haben«
                  fallen ganz besonders \so{Liebelei\pwindex{Schnitzler, Arthur 15.05.1862 – 21.10.1931@\textsc{Schnitzler, Arthur} (15.05.1862 – 21.10.1931), \emph{Schriftsteller, Mediziner}!Liebelei. Schauspiel in drei Akten1895-10-09@\strich\emph{Liebelei. Schauspiel in drei Akten} {[}1895-10-09{]}|pw}}, die Dritte zu Over Norton\oindex{Over Norton@\textbf{Over Norton}|pw} und
                  Seraphine im Großen Köln\oindex{Koeln@\textbf{Köln}|pw}ischen Handicap und
                     \so{Cadore\oindex{Cadore@\textbf{Cadore}|pw}}, der mit frischem Lorbeer gekrönte Sieger des Hamburg\oindex{Hamburg@\textbf{Hamburg}|pw}er Godeffroy\pwindex{Godeffroy, Johan Cesar 1813-07-01 – 1885-02-09@\textsc{Godeffroy, Johan Cesar} (1813-07-01 – 1885-02-09), \emph{Kaufmann}|pw}-Rennens, auf.
                  Für die Hamburg\oindex{Hamburg@\textbf{Hamburg}|pw}er Ueberraschung muß der Bleichröder\pwindex{Bleichroeder, Georg von 1857-10-27 – 1902-06-11@\textsc{Bleichröder, Georg von} (1857-10-27 – 1902-06-11), \emph{Bankier}|pw}’sche Wallach volle zehn Pfund
                  mehr aufnehmen und wir glauben offen gestanden nicht, daß es dem Dreijährigen mit
                  dem hohen Gewicht von 55½ kg gelingen wird, die Situation zu beherrschen. \so{Liebelei\pwindex{Schnitzler, Arthur 15.05.1862 – 21.10.1931@\textsc{Schnitzler, Arthur} (15.05.1862 – 21.10.1931), \emph{Schriftsteller, Mediziner}!Liebelei. Schauspiel in drei Akten1895-10-09@\strich\emph{Liebelei. Schauspiel in drei Akten} {[}1895-10-09{]}|pw}} ist viel besser daran. Zwar drücken 64½ kg auch, aber die Talpra-Magyar\pwindex{\textcolor{red}{\textsuperscript{XXXX1 indx}}!Nemzeti dal1848@\strich\emph{Nemzeti dal} {[}1848{]}|pwv}-Tochter ist ein Pferd mit
                  reellen Fähigkeiten – ein »Frühjahrspferd«, – das auch in Köln\oindex{Koeln@\textbf{Köln}|pw} eine gute Leistung vollbrachte. Seitdem soll sie sich
                  ganz wesentlich verbessert haben. Wir würden ihr auch ohne Bedenken unsere
                  Sympathien zuwenden, wenn der \so{Borstel\oindex{Hamburg-Gross Borstel@\textbf{Hamburg-Groß Borstel}|pwuv}er
                     Stall}, der augenblicklich auf der Höhe steht, nicht \so{Heroine}, die im Gewicht außerordentlich begünstigt ist, im Rennen hätte.
                  Wie aus guter Quelle verlautet, ist Heroine in ausgezeichneter Verfassung und soll
                  ihren Trainer\pwindex{Beit, A. @\textsc{Beit, A.}, \emph{Sportler}|pwv} in der Arbeit sehr befriedigt haben. Man wird gut thun, der Fulmen-Tochter\pwindex{Fulmen, Lydia von @\textsc{Fulmen, Lydia von}, \emph{Sportlerin}|pwuv} für das große Rennen die gebührende Beachtung zu schenken.
                     \so{Nicolo} ist ebenfalls nicht schlecht im Handicap,
                  jedoch nicht in Form. Sein Laufen in Köln\oindex{Koeln@\textbf{Köln}|pw} war
                  durchaus nicht berühmt und wir glauben kaum, daß von ihm eine Ueberraschung zu
                  erwarten ist. Eher von X, der von Warne\pwindex{Warne, W. @\textsc{Warne, W.}, \emph{Sportler/Sportlerin}|pw}
                  gesteuert, bei der günstigen Distanz durchaus nicht ohne Chancen ist. \so{Connex} und \so{Radler} Schluß
                  dürfte aber doch \textbf{Heroine} das bessere Ende vor \so{Liebelei\pwindex{Schnitzler, Arthur 15.05.1862 – 21.10.1931@\textsc{Schnitzler, Arthur} (15.05.1862 – 21.10.1931), \emph{Schriftsteller, Mediziner}!Liebelei. Schauspiel in drei Akten1895-10-09@\strich\emph{Liebelei. Schauspiel in drei Akten} {[}1895-10-09{]}|pw}} und X behalten.}}\pend
           
         
         \endnumbering\mylabel{h}\end{ledgroupsized}\begin{anhang}\end{anhang}\newcommand{\dateiname}{L02914}\newcommand{\titel}{Paul Goldmann an Arthur Schnitzler, 2. 5. [1900]}\newcommand{\editorInnen}{Martin Anton Müller und Laura Untner}%% latex-leseansicht-abspann.tex
%% Abspann für die Leseansicht.
%% Der Schalter \ifkorrekturansicht ist bereits durch den Vorspann gesetzt.

%% latex-abspann.tex
%% Gemeinsamer Abspann für Korrekturansicht und Leseansicht.
%% Setzt den Schalter \ifkorrekturansicht voraus (gesetzt in den
%% einbindenden Dateien latex-korrekturansicht-abspann.tex bzw.
%% latex-leseansicht-abspann.tex).
%% ---------------------------------------------------------------

\normalsize

% Das esempio-Environment wird nur in der Leseansicht benötigt
\ifkorrekturansicht\else
\newenvironment{esempio}[3]%
{
    \vspace{1.5ex}
    \rlap{\underline{#1}}
    \par
    \setlength{\parindent}{0cm}
    \nopagebreak
    \leftskip=#2cm
    \rightskip=#3cm
}
{
    \par
}
\fi

\doendnotes{C}
\bigskip
\vfill

\clearpage

\footnotesize

\ifkorrekturansicht
  \lohead{\textsc{register}}
\fi

% theindex-Environment neu definieren ohne reledmac
\makeatletter
\renewenvironment{theindex}{%
  \ifkorrekturansicht
    \section*{\indexname}%
  \else
    \subsubsection*{Index der erwähnten Entitäten}%
  \fi
  \setlength{\parindent}{0pt}%
  \setlength{\parskip}{0pt plus 0.3pt}%
  \let\item\@idxitem
}{%
  \ifkorrekturansicht\clearpage\fi
}
\makeatother

\IfFileExists{\jobname-pw.ind}{\input{\jobname-pw.ind}}{}

% Quellenangabe nur in der Leseansicht
\ifkorrekturansicht\else
% Fallback-Definitionen, falls die .tex-Datei \titel etc. nicht gesetzt hat
\providecommand{\titel}{}
\providecommand{\editorInnen}{}
\providecommand{\dateiname}{\jobname}

\vspace{3cm}

\vfill

\footnotesize
\textsc{Quelle}: \titel. Herausgegeben von {\editorInnen}. In: \emph{Arthur Schnitzler: Briefwechsel mit Autorinnen und Autoren}.
 Digitale Edition, https://schnitzler-briefe.acdh.oeaw.ac.at/{\dateiname}.html (Stand \today)
\fi

\end{document}


      