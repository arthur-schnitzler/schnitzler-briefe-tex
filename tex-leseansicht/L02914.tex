%% latex-korrekturansicht-vorspann.tex
%% Vorspann für die Korrekturansicht.
%% Lädt die gemeinsame Datei latex-vorspann.tex mit gesetztem Schalter.

\newif\ifkorrekturansicht
\korrekturansichttrue

\input{../tex-inputs/latex-vorspann}


\section[ Paul Goldmann an Arthur Schnitzler, 2. 5. {[}1900{]}]{L02914 Paul Goldmann an Arthur Schnitzler, 2. 5. {[}1900{]}}
\nopagebreak\mylabel{L02914v}
\rehead{ }\normalsize\beginnumbering\briefempfaengerindex{Schnitzler, Arthur@\textsc{Schnitzler, Arthur}!zzzGoldmann, Paul@\emph{von Paul Goldmann}!1900-05-021@{2. 5. {[}1900{]}}|(be}
\toendnotes[C]{\smallbreak\pagebreak[2]}\Standort{DLA, A:Schnitzler, HS.NZ85.1.3170.}
\physDesc{Brief, 1 Blatt, 3 Seiten, 702 Zeichen
\newline{}Handschrift: blaue Tinte, deutsche Kurrent
\newline{}Beilage: ein Zeitungsausschnitt, beschnitten }\toendnotes[C]{\smallbreak}
\pstart
           {\pb}\textcolor{gray}{\textbf{DESSAUERSTRASSE 19}}\oindex{Dessauer Strasse@\textbf{Dessauer Straße}, \emph{Straße (K.STR)}|pw}\pend
           
\pstart
           \raggedleft{}Berlin\oindex{Berlin@\textbf{Berlin}, \emph{P.PPLC}|pw}, 2. Mai.\pend
           
\pstart{}Mein lieber Freund,\pend\vspace{0.5em}
\pstart
           In aller Eile Dank für Deinen lieben Brief!\pend
           
\pstart
           Mich hat die \label{K_L02914-1v}\edtext{Frau
                  Rechtsanwalt\pwindex{Freudenthal, Rosa 1862 – 18.06.1905@\textsc{Freudenthal, Rosa} (1862 – 18.06.1905)|pwv}}{\lemma{\textnormal{\emph{Frau
                  Rechtsanwalt}}}\Cendnote{\textnormal{Siehe Paul Goldmann an Arthur Schnitzler, 20. 2. 1900.
               }}}\label{K_L02914-1} um den »Reigen\pwindex{Reigen. Zehn Dialoge@\emph{Reigen. Zehn Dialoge}|pw}« erſucht. Ich hielt mich
               aber nicht für berechtigt, der Frau\pwindex{Freudenthal, Rosa 1862 – 18.06.1905@\textsc{Freudenthal, Rosa} (1862 – 18.06.1905)|pwv} das Buch\pwindex{Reigen. Zehn Dialoge@\emph{Reigen. Zehn Dialoge}|pw} zu geben, und habe mich
               damit ausgeredet, ich hätte es verborgt.\pend
           
\pstart
           Wie Du aus beifolgendem \label{K_L02914-2v}\edtext{Rennbericht\pwindex{Rennbericht, Pferd Liebelei]@\emph{[Rennbericht, Pferd Liebelei]}|pwv}}{\lemma{\textnormal{\emph{Rennbericht}}}\Cendnote{\textnormal{Es ist unklar, aus welcher Zeitung der
                     Ausschnitt\pwindex{Rennbericht, Pferd Liebelei]@\emph{[Rennbericht, Pferd Liebelei]}|pwkv}
                  stammt.}}}\label{K_L02914-2} ſiehſt, iſt hier beim letzten Rennen ein Pferd »Liebelei\pwindex{Liebelei. Schauspiel in drei Akten@\emph{Liebelei. Schauspiel in drei Akten}|pw}« gelaufen. Es {\pb}gehört \label{K_L02914-3v}\edtext{einem ſüddeutſch\oindex{Deutschland@\textbf{Deutschland}, \emph{A.PCLI}|pwv}en Beſitzer\pwindex{Lang-Puchhof, Carl von 1854-09-18 – 1916-04-07@\textsc{Lang-Puchhof, Carl von} (1854-09-18 – 1916-04-07), \emph{Gutsbesitzer/Gutsbesitzerin, Rennstallbesitzer/Rennstallbesitzerin, Jurist/Juristin}|pwv}\pwindex{Schmieder, Karl August von 1867-05-29 – 1941-03-06@\textsc{Schmieder, Karl August von} (1867-05-29 – 1941-03-06), \emph{Rennstallbesitzer/Rennstallbesitzerin}|pwv}}{\lemma{\textnormal{\emph{einem … Beſitzer}}}\Cendnote{\textnormal{Das Pferd »Liebelei\pwindex{Liebelei. Schauspiel in drei Akten@\emph{Liebelei. Schauspiel in drei Akten}|pwkv}« gehörte Carl
                     von Lang-Puchhof\pwindex{Lang-Puchhof, Carl von 1854-09-18 – 1916-04-07@\textsc{Lang-Puchhof, Carl von} (1854-09-18 – 1916-04-07), \emph{Gutsbesitzer/Gutsbesitzerin, Rennstallbesitzer/Rennstallbesitzerin, Jurist/Juristin}|pwk} und Karl August von
                     Schmieder\pwindex{Schmieder, Karl August von 1867-05-29 – 1941-03-06@\textsc{Schmieder, Karl August von} (1867-05-29 – 1941-03-06), \emph{Rennstallbesitzer/Rennstallbesitzerin}|pwk}, die von 1898 bis 1907 einen Pferderennstall\orgindex{Rennstall Lang-Puchhof und Schmieder@Rennstall Lang-Puchhof und Schmieder|pwkv} in Hoppegarten\oindex{Hoppegarten@\textbf{Hoppegarten}, \emph{P.PPLA4}|pwk}
                  betrieben. Goldmann\pwindex{Goldmann, Paul 31.01.1865 – 25.09.1935@\textsc{Goldmann, Paul} (31.01.1865 – 25.09.1935), \emph{Schriftsteller/Schriftstellerin, Journalist/Journalistin}|pwk} bezog sich vermutlich
                  auf den Rheinländ\oindex{Rheinland@\textbf{Rheinland}, \emph{Teil eines Landes (A.LNDX)}|pwkv}er Lang-Puchhof\pwindex{Lang-Puchhof, Carl von 1854-09-18 – 1916-04-07@\textsc{Lang-Puchhof, Carl von} (1854-09-18 – 1916-04-07), \emph{Gutsbesitzer/Gutsbesitzerin, Rennstallbesitzer/Rennstallbesitzerin, Jurist/Juristin}|pwk}.}}}\label{K_L02914-3} und heißt offenbar
               nach Deinem Stück\pwindex{Liebelei. Schauspiel in drei Akten@\emph{Liebelei. Schauspiel in drei Akten}|pwv}. Dies iſt
               der Ruhm, mein lieber Freund!\pend
           
\pstart
           Es freut mich ſehr, zu hören, daß Du eine \label{K_L02914-4v}\edtext{Poſſe\pwindex{Ritterlichkeit@\emph{Ritterlichkeit}|pwuv}}{\lemma{\textnormal{\emph{Poſſe}}}\Cendnote{\textnormal{Wahrscheinlich handelt es sich um eine
                  Bezugnahme auf das Fragment gebliebene und erst postum veröffentlichte Drama \emph{Ritterlichkeit}\pwindex{Ritterlichkeit@\emph{Ritterlichkeit}|pwk}, das Schnitzler am 23. 4. 1900 vorläufig unter dem Titel »Drama\pwindex{Ritterlichkeit@\emph{Ritterlichkeit}|pwkv}« beendet hatte.}}}\label{K_L02914-4}
               geſchrieben haſt. So biſt Du \strikeout{\textcolor{gray}{×}} auf halbem Wege zu dem \label{K_L02914-5v}\edtext{Luſtſpiel}{\lemma{\textnormal{\emph{Luſtſpiel}}}\Cendnote{\textnormal{In der Korrespondenz mit
                     Goldmann\pwindex{Goldmann, Paul 31.01.1865 – 25.09.1935@\textsc{Goldmann, Paul} (31.01.1865 – 25.09.1935), \emph{Schriftsteller/Schriftstellerin, Journalist/Journalistin}|pwk} ist davon mehrfach die Rede:
                     vgl. Paul Goldmann an Arthur Schnitzler, 8. 12. [1893], 23. 12. [1893], 2. [1.? 1897] und 17. 4. [1902]. Im Sommer 1900 arbeitete Schnitzler an \emph{Die Quellen des Nil}\pwindex{Quellen des Nil@\emph{Die Quellen des Nil}|pwk}
                  weiter (vgl. Arthur Schnitzler an Hugo von Hofmannsthal, 17. 7. 1900). }}}\label{K_L02914-5},
               das ich nicht ablaſſen werde, von Dir zu verlangen.\pend
           
\pstart
           Nächſtens mehr! Heut habe ich nur zwei Minuten.\pend
           
\pstart
           {\pb}Viele treue Grüße! {\\[\baselineskip]}Dein {\\[\baselineskip]}\spacefill\mbox{Paul Goldmann}\pend
           \leftskip=0em{}\selectlanguage{ngerman}\vspace{1em}{\vspace{1\baselineskip}}
\pstart
           \textcolor{gray}{\textbf{{\pb}Unter den Pferden, die bereits »was gezeigt haben«
                  fallen ganz beſonders }}\textcolor{gray}{\textbf{\so{Liebelei}}}\pwindex{Liebelei. Schauspiel in drei Akten@\emph{Liebelei. Schauspiel in drei Akten}|pwv}\textcolor{gray}{\textbf{, die Dritte zu Over Norton und Seraphine im Großen Köln\oindex{Koeln@\textbf{Köln}, \emph{P.PPLA2}|pw}iſchen Handicap und \so{Cadore}\oindex{Cadore@\textbf{Cadore}, \emph{Tal (N.TAL)}|pw}, der mit friſchem Lorbeer gekrönte Sieger des Hamburg\oindex{Hamburg@\textbf{Hamburg}, \emph{P.PPLA}|pw}er Godeffroy-Rennens, auf. Für die Hamburg\oindex{Hamburg@\textbf{Hamburg}, \emph{P.PPLA}|pw}er Ueberraſchung muß der Bleichröder\pwindex{Bleichroeder, Georg von 1857-10-27 – 1902-06-11@\textsc{Bleichröder, Georg von} (1857-10-27 – 1902-06-11), \emph{Bankier/Bankierin}|pw}’ſche Wallach volle zehn Pfund mehr aufnehmen und wir glauben
                  offen geſtanden nicht, daß es dem Dreijährigen mit dem hohen Gewicht von 55½\textsc{kg} gelingen wird, die Situation zu beherrſchen.}}{ }\textcolor{gray}{\textbf{\so{Liebelei}}}\pwindex{Liebelei. Schauspiel in drei Akten@\emph{Liebelei. Schauspiel in drei Akten}|pwv}\textcolor{gray}{\textbf{{ }iſt viel beſſer daran. Zwar drücken 64½ \textsc{kg} auch, aber die \label{K_L02914-6v}\edtext{Talpra-Magyar\pwindex{Nemzeti dal@\emph{Nemzeti dal}|pwv}-Tochter}{\lemma{\textnormal{\emph{Talpra-Magyar-Tochter}}}\Cendnote{\textnormal{»Talpra-Magyar« war eines der
                     begehrtesten Zuchtpferde der Zeit, benannt nach den ersten beiden Worten des
                     revolutionären Gedichts \emph{Nemzeti dal}\pwindex{Nemzeti dal@\emph{Nemzeti dal}|pwk}
                        (1848) von Sándor
                     Petőfi\pwindex{Petőfi, Sándor 01.01.1823 – 1849-07-31@\textsc{Petőfi, Sándor} (01.01.1823 – 1849-07-31), \emph{Schriftsteller/Schriftstellerin, Revolutionär/Revolutionärin}|pwk}.}}}\label{K_L02914-6} iſt ein Pferd mit reellen Fähigkeiten – ein
                  »Frühjahrspferd« –{[},{]} das auch in Köln\oindex{Koeln@\textbf{Köln}, \emph{P.PPLA2}|pw} eine gute Leiſtung vollbrachte. Seitdem ſoll ſie ſich
                  ganz weſentlich verbeſſert haben. Wir würden ihr auch ohne Bedenken unſere
                  Sympathien zuwenden, wenn der }}\textcolor{gray}{\textbf{\so{Borſtel}}}\oindex{Hamburg-Gross Borstel@\textbf{Hamburg-Groß Borstel}, \emph{Bezirk (A.BZK)}|pwuv}\textcolor{gray}{\textbf{\so{er Stall}, der augenblicklich auf der Höhe ſteht, nicht
                     \so{Heroine}, die im Gewicht außerordentlich begünſtigt
                  iſt, im Rennen hätte. Wie aus guter Quelle verlautet, iſt Heroine in
                  ausgezeichneter Verfaſſung und ſoll ihren Trainer\pwindex{Beit, A. @\textsc{Beit, A.}, \emph{Sportler/Sportlerin}|pwv} in der Arbeit ſehr befriedigt
                  haben. Man wird gut thun, der Fulmen-Tochter\pwindex{Fulmen, Lydia von @\textsc{Fulmen, Lydia von}, \emph{Sportler/Sportlerin}|pwuv} für das große Rennen die gebührende
                  Beachtung zu ſchenken. \so{Nicolo} iſt ebenfalls nicht
                  ſchlecht im Handicap, jedoch nicht in Form. Sein Laufen in Köln\oindex{Koeln@\textbf{Köln}, \emph{P.PPLA2}|pw} war durchaus nicht berühmt und wir glauben kaum, daß
                  von ihm eine Ueberraſchung zu erwarten iſt. Eher von \textsc{X},
                  der von Warne\pwindex{Warne, W. @\textsc{Warne, W.}, \emph{Sportler/Sportlerin}|pw} geſteuert, bei der günſtigen
                  Diſtanz durchaus nicht ohne Chancen iſt. \so{Connex} und \so{Radler} erſcheinen aus dem Lot zunächſt für die Plätze in
                  Betracht zu kommen. Zum Schluß dürfte aber doch}}\pend
           
\pstart
           \centering{}\textcolor{gray}{\textbf{\textbf{Heroine}}}\pend
           
\pstart
           \textcolor{gray}{\textbf{das beſſere Ende vor}}{ }\textcolor{gray}{\textbf{\so{Liebelei}}}\pwindex{Liebelei. Schauspiel in drei Akten@\emph{Liebelei. Schauspiel in drei Akten}|pw}\textcolor{gray}{\textbf{{ }und \textsc{X}
                     behalten{[}.{]}}}\pend
           \selectlanguage{ngerman}\endnumbering\briefempfaengerindex{Schnitzler, Arthur@\textsc{Schnitzler, Arthur}!zzzGoldmann, Paul@\emph{von Paul Goldmann}!1900-05-021@{2. 5. {[}1900{]}}|)be}\mylabel{L02914h}  \normalsize

\doendnotes{C}
\bigskip
\vfill

\clearpage

\footnotesize

\lohead{\textsc{register}}

% Definiere theindex-Environment komplett neu ohne reledmac
\makeatletter
\renewenvironment{theindex}{%
  \section*{\indexname}%
  \setlength{\parindent}{0pt}%
  \setlength{\parskip}{0pt plus 0.3pt}%
  \let\item\@idxitem
}{%
  \clearpage
}
\makeatother

\IfFileExists{\jobname-pw.ind}{\input{\jobname-pw.ind}}{}

\end{document}

      