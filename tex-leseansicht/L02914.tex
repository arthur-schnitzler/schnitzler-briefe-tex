%% latex-leseansicht-vorspann.tex
%% Vorspann für die Leseansicht.
%% Lädt die gemeinsame Datei latex-vorspann.tex mit nicht gesetztem Schalter.

\newif\ifkorrekturansicht
\korrekturansichtfalse

\input{../tex-inputs/latex-vorspann}


\section[ Paul Goldmann an Arthur Schnitzler, 2. 5. {[}1900{]}]{L02914 Paul Goldmann an Arthur Schnitzler,  2. 5. [1900]}
\nopagebreak\mylabel{L02914v}
\rehead{ }\normalsize\beginnumbering\briefempfaengerindex{Schnitzler, Arthur@\textsc{Schnitzler, Arthur}!zzzGoldmann, Paul@\emph{von Paul Goldmann}!1900-05-021@{2. 5. [1900]}|(be}
\toendnotes[C]{\smallbreak\pagebreak[2]}
\correspDesc{Versand  durch Paul Goldmann am 2. 5. [1900] in Berlin
\newline{}Erhalt  durch Arthur Schnitzler im Zeitraum [3. 5. 1900
                  – 7. 5. 1900?] in Wien}\toendnotes[C]{\smallbreak}
\Standort{DLA, A:Schnitzler, HS.NZ85.1.3170.}
\physDesc{Brief, 1 Blatt, 3 Seiten, 702 Zeichen
\newline{}Handschrift: blaue Tinte, deutsche Kurrent
\newline{}Beilage: ein Zeitungsausschnitt, beschnitten }\toendnotes[C]{\smallbreak}
\pstart
           {\pb}\textcolor{gray}{\textbf{DESSAUERSTRASSE 19}}\oindex{Dessauer Straße@\textbf{Dessauer Straße}, \emph{Straße}|pw}\pend
           
\pstart
           \raggedleft{}Berlin\oindex{Berlin@\textbf{Berlin}, \emph{Hauptstadt}|pw}, 2. Mai.\pend
           
\pstart{}Mein lieber Freund,\pend\vspace{0.5em}
\pstart
           In aller Eile Dank für Deinen lieben Brief!\pend
           
\pstart
           Mich hat die \label{K_L02914-1v}\edtext{Frau
                  Rechtsanwalt\pwindex{Freudenthal, Rosa 1862 – 18.\,6.\,1905 Berlin@\textsc{Freudenthal, Rosa} (1862 – 18.\,6.\,1905 Berlin)|pwv}}{\lemma{\textnormal{\emph{Frau
                  Rechtsanwalt}}}\Cendnote{\textnormal{Siehe XXXX Auszeichnungsfehler: Dokument L02905 nicht gefunden.
               }}}\label{K_L02914-1} um den »Reigen\pwindex{Schnitzler, Arthur 15.\,5.\,1862 Wien – 21.\,10.\,1931 ebd.@\textsc{Schnitzler, Arthur} (15.\,5.\,1862 Wien – 21.\,10.\,1931 ebd.), \emph{Schriftsteller, Mediziner}!Reigen. Zehn Dialoge@\strich\emph{Reigen. Zehn Dialoge}|pw}« erſucht. Ich hielt mich
               aber nicht für berechtigt, der Frau\pwindex{Freudenthal, Rosa 1862 – 18.\,6.\,1905 Berlin@\textsc{Freudenthal, Rosa} (1862 – 18.\,6.\,1905 Berlin)|pwv} das Buch\pwindex{Schnitzler, Arthur 15.\,5.\,1862 Wien – 21.\,10.\,1931 ebd.@\textsc{Schnitzler, Arthur} (15.\,5.\,1862 Wien – 21.\,10.\,1931 ebd.), \emph{Schriftsteller, Mediziner}!Reigen. Zehn Dialoge@\strich\emph{Reigen. Zehn Dialoge}|pw} zu geben, und habe mich
               damit ausgeredet, ich hätte es verborgt.\pend
           
\pstart
           Wie Du aus beifolgendem \label{K_L02914-2v}\edtext{Rennbericht\pwindex{Rennbericht, Pferd Liebelei]@\emph{[Rennbericht, Pferd Liebelei]}|pwv}}{\lemma{\textnormal{\emph{Rennbericht}}}\Cendnote{\textnormal{Es ist unklar, aus welcher Zeitung der
                     Ausschnitt\pwindex{Rennbericht, Pferd Liebelei]@\emph{[Rennbericht, Pferd Liebelei]}|pwkv}
                  stammt.}}}\label{K_L02914-2}{ }ſiehſt, iſt hier beim letzten Rennen ein Pferd »Liebelei\pwindex{Schnitzler, Arthur 15.\,5.\,1862 Wien – 21.\,10.\,1931 ebd.@\textsc{Schnitzler, Arthur} (15.\,5.\,1862 Wien – 21.\,10.\,1931 ebd.), \emph{Schriftsteller, Mediziner}!Liebelei. Schauspiel in drei Akten@\strich\emph{Liebelei. Schauspiel in drei Akten}|pw}« gelaufen. Es {\pb}gehört \label{K_L02914-3v}\edtext{einem{ }ſüddeutſch\oindex{Deutschland@\textbf{Deutschland}|pwv}en Beſitzer\pwindex{Lang-Puchhof, Carl von 18.\,9.\,1854 Neustadt an der Weinstraße – 7.\,4.\,1916 München@\textsc{Lang-Puchhof, Carl von} (18.\,9.\,1854 Neustadt an der Weinstraße – 7.\,4.\,1916 München), \emph{Gutsbesitzer, Rennstallbesitzer, Jurist}|pwv}\pwindex{Schmieder, Karl August von 29.\,5.\,1867 Breslau – 6.\,3.\,1941 Steinach@\textsc{Schmieder, Karl August von} (29.\,5.\,1867 Breslau – 6.\,3.\,1941 Steinach), \emph{Rennstallbesitzer}|pwv}}{\lemma{\textnormal{\emph{einem … Besitzer}}}\Cendnote{\textnormal{Das Pferd »Liebelei\pwindex{Schnitzler, Arthur 15.\,5.\,1862 Wien – 21.\,10.\,1931 ebd.@\textsc{Schnitzler, Arthur} (15.\,5.\,1862 Wien – 21.\,10.\,1931 ebd.), \emph{Schriftsteller, Mediziner}!Liebelei. Schauspiel in drei Akten@\strich\emph{Liebelei. Schauspiel in drei Akten}|pwkv}« gehörte Carl
                     von Lang-Puchhof\pwindex{Lang-Puchhof, Carl von 18.\,9.\,1854 Neustadt an der Weinstraße – 7.\,4.\,1916 München@\textsc{Lang-Puchhof, Carl von} (18.\,9.\,1854 Neustadt an der Weinstraße – 7.\,4.\,1916 München), \emph{Gutsbesitzer, Rennstallbesitzer, Jurist}|pwk} und Karl August von
                     Schmieder\pwindex{Schmieder, Karl August von 29.\,5.\,1867 Breslau – 6.\,3.\,1941 Steinach@\textsc{Schmieder, Karl August von} (29.\,5.\,1867 Breslau – 6.\,3.\,1941 Steinach), \emph{Rennstallbesitzer}|pwk}, die von 1898 bis 1907 einen Pferderennstall\orgindex{Rennstall Lang-Puchhof und Schmieder@Rennstall Lang-Puchhof und Schmieder|pwkv} in Hoppegarten\oindex{Hoppegarten@\textbf{Hoppegarten}, \emph{Hauptstadt}|pwk}
                  betrieben. Goldmann\pwindex{Goldmann, Paul 31.\,1.\,1865 Breslau – 25.\,9.\,1935 Wien@\textsc{Goldmann, Paul} (31.\,1.\,1865 Breslau – 25.\,9.\,1935 Wien), \emph{Schriftsteller, Journalist}|pwk} bezog sich vermutlich
                  auf den Rheinländ\oindex{Rheinland@\textbf{Rheinland}|pwkv}er Lang-Puchhof\pwindex{Lang-Puchhof, Carl von 18.\,9.\,1854 Neustadt an der Weinstraße – 7.\,4.\,1916 München@\textsc{Lang-Puchhof, Carl von} (18.\,9.\,1854 Neustadt an der Weinstraße – 7.\,4.\,1916 München), \emph{Gutsbesitzer, Rennstallbesitzer, Jurist}|pwk}.}}}\label{K_L02914-3} und heißt offenbar
               nach Deinem Stück\pwindex{Schnitzler, Arthur 15.\,5.\,1862 Wien – 21.\,10.\,1931 ebd.@\textsc{Schnitzler, Arthur} (15.\,5.\,1862 Wien – 21.\,10.\,1931 ebd.), \emph{Schriftsteller, Mediziner}!Liebelei. Schauspiel in drei Akten@\strich\emph{Liebelei. Schauspiel in drei Akten}|pwv}. Dies iſt
               der Ruhm, mein lieber Freund!\pend
           
\pstart
           Es freut mich{ }ſehr, zu hören, daß Du eine \label{K_L02914-4v}\edtext{Poſſe\pwindex{Schnitzler, Arthur 15.\,5.\,1862 Wien – 21.\,10.\,1931 ebd.@\textsc{Schnitzler, Arthur} (15.\,5.\,1862 Wien – 21.\,10.\,1931 ebd.), \emph{Schriftsteller, Mediziner}!Ritterlichkeit@\strich\emph{Ritterlichkeit}|pwuv}}{\lemma{\textnormal{\emph{Posse}}}\Cendnote{\textnormal{Wahrscheinlich handelt es sich um eine
                  Bezugnahme auf das Fragment gebliebene und erst postum veröffentlichte Drama \emph{Ritterlichkeit}\pwindex{Schnitzler, Arthur 15.\,5.\,1862 Wien – 21.\,10.\,1931 ebd.@\textsc{Schnitzler, Arthur} (15.\,5.\,1862 Wien – 21.\,10.\,1931 ebd.), \emph{Schriftsteller, Mediziner}!Ritterlichkeit@\strich\emph{Ritterlichkeit}|pwk}, das Schnitzler am 23. 4. 1900 vorläufig unter dem Titel »Drama\pwindex{Schnitzler, Arthur 15.\,5.\,1862 Wien – 21.\,10.\,1931 ebd.@\textsc{Schnitzler, Arthur} (15.\,5.\,1862 Wien – 21.\,10.\,1931 ebd.), \emph{Schriftsteller, Mediziner}!Ritterlichkeit@\strich\emph{Ritterlichkeit}|pwkv}« beendet hatte.}}}\label{K_L02914-4}
               geſchrieben haſt. So biſt Du \strikeout{\textcolor{gray}{×}} auf halbem Wege zu dem \label{K_L02914-5v}\edtext{Luſtſpiel}{\lemma{\textnormal{\emph{Lustspiel}}}\Cendnote{\textnormal{In der Korrespondenz mit
                     Goldmann\pwindex{Goldmann, Paul 31.\,1.\,1865 Breslau – 25.\,9.\,1935 Wien@\textsc{Goldmann, Paul} (31.\,1.\,1865 Breslau – 25.\,9.\,1935 Wien), \emph{Schriftsteller, Journalist}|pwk} ist davon mehrfach die Rede:
                     vgl. XXXX Auszeichnungsfehler: Dokument L02723 nicht gefunden, XXXX Auszeichnungsfehler: Dokument L02724 nicht gefunden, XXXX Auszeichnungsfehler: Dokument L02792 nicht gefunden und XXXX Auszeichnungsfehler: Dokument L03204 nicht gefunden. Im Sommer 1900 arbeitete Schnitzler an \emph{Die Quellen des Nil}\pwindex{Schnitzler, Arthur 15.\,5.\,1862 Wien – 21.\,10.\,1931 ebd.@\textsc{Schnitzler, Arthur} (15.\,5.\,1862 Wien – 21.\,10.\,1931 ebd.), \emph{Schriftsteller, Mediziner}!Quellen des Nil@\strich\emph{Die Quellen des Nil}|pwk}
                  weiter (vgl. XXXX Auszeichnungsfehler: Dokument L01057 nicht gefunden). }}}\label{K_L02914-5},
               das ich nicht ablaſſen werde, von Dir zu verlangen.\pend
           
\pstart
           Nächſtens mehr! Heut habe ich nur zwei Minuten.\pend
           
\pstart
           {\pb}Viele treue Grüße! {\\[\baselineskip]}Dein {\\[\baselineskip]}\spacefill\mbox{Paul Goldmann}\pend
           \leftskip=0em{}\selectlanguage{ngerman}\vspace{1em}{\vspace{1\baselineskip}}
\pstart
           \textcolor{gray}{\textbf{{\pb}Unter den Pferden, die bereits »was gezeigt haben«
                  fallen ganz beſonders}}{ }\textcolor{gray}{\textbf{\so{Liebelei}}}\pwindex{Schnitzler, Arthur 15.\,5.\,1862 Wien – 21.\,10.\,1931 ebd.@\textsc{Schnitzler, Arthur} (15.\,5.\,1862 Wien – 21.\,10.\,1931 ebd.), \emph{Schriftsteller, Mediziner}!Liebelei. Schauspiel in drei Akten@\strich\emph{Liebelei. Schauspiel in drei Akten}|pwv}\textcolor{gray}{\textbf{, die Dritte zu Over Norton und Seraphine im Großen Köln\oindex{Köln@\textbf{Köln}, \emph{Hauptstadt}|pw}iſchen Handicap und \so{Cadore}\oindex{Cadore@\textbf{Cadore}, \emph{Tal}|pwv}, der mit friſchem Lorbeer gekrönte Sieger des Hamburg\oindex{Hamburg@\textbf{Hamburg}|pw}er Godeffroy-Rennens, auf. Für die Hamburg\oindex{Hamburg@\textbf{Hamburg}|pw}er Ueberraſchung muß der Bleichröder\pwindex{Bleichröder, Georg von 27.\,10.\,1857 Berlin – 11.\,6.\,1902 Lechenich@\textsc{Bleichröder, Georg von} (27.\,10.\,1857 Berlin – 11.\,6.\,1902 Lechenich), \emph{Bankier}|pw}’ſche Wallach volle zehn Pfund mehr aufnehmen und wir glauben
                  offen geſtanden nicht, daß es dem Dreijährigen mit dem hohen Gewicht von 55½\textsc{kg} gelingen wird, die Situation zu beherrſchen.}}{ }\textcolor{gray}{\textbf{\so{Liebelei}}}\pwindex{Schnitzler, Arthur 15.\,5.\,1862 Wien – 21.\,10.\,1931 ebd.@\textsc{Schnitzler, Arthur} (15.\,5.\,1862 Wien – 21.\,10.\,1931 ebd.), \emph{Schriftsteller, Mediziner}!Liebelei. Schauspiel in drei Akten@\strich\emph{Liebelei. Schauspiel in drei Akten}|pwv}\textcolor{gray}{\textbf{{ }iſt viel beſſer daran. Zwar drücken 64½ \textsc{kg} auch, aber die \label{K_L02914-6v}\edtext{Talpra-Magyar\pwindex{Petőfi, Sándor 1.\,1.\,1823 Kiskőrös – 31.\,7.\,1849 Sighișoara@\textsc{Petőfi, Sándor} (1.\,1.\,1823 Kiskőrös – 31.\,7.\,1849 Sighișoara), \emph{Schriftsteller, Revolutionär}!Nemzeti dal@\strich\emph{Nemzeti dal}|pwv}-Tochter}{\lemma{\textnormal{\emph{Talpra-Magyar-Tochter}}}\Cendnote{\textnormal{»Talpra-Magyar« war eines der
                     begehrtesten Zuchtpferde der Zeit, benannt nach den ersten beiden Worten des
                     revolutionären Gedichts \emph{Nemzeti dal}\pwindex{Petőfi, Sándor 1.\,1.\,1823 Kiskőrös – 31.\,7.\,1849 Sighișoara@\textsc{Petőfi, Sándor} (1.\,1.\,1823 Kiskőrös – 31.\,7.\,1849 Sighișoara), \emph{Schriftsteller, Revolutionär}!Nemzeti dal@\strich\emph{Nemzeti dal}|pwk}
                        (1848) von Sándor
                     Petőfi\pwindex{Petőfi, Sándor 1.\,1.\,1823 Kiskőrös – 31.\,7.\,1849 Sighișoara@\textsc{Petőfi, Sándor} (1.\,1.\,1823 Kiskőrös – 31.\,7.\,1849 Sighișoara), \emph{Schriftsteller, Revolutionär}|pwk}.}}}\label{K_L02914-6} iſt ein Pferd mit reellen Fähigkeiten – ein
                  »Frühjahrspferd« –{[},{]} das auch in Köln\oindex{Köln@\textbf{Köln}, \emph{Hauptstadt}|pw} eine gute Leiſtung vollbrachte. Seitdem{ }ſoll{ }ſie{ }ſich
                  ganz weſentlich verbeſſert haben. Wir würden ihr auch ohne Bedenken unſere
                  Sympathien zuwenden, wenn der}}{ }\textcolor{gray}{\textbf{\so{Borſtel}}}\oindex{Hamburg-Groß Borstel@\textbf{Hamburg-Groß Borstel}, \emph{Bezirk}|pwuv}\textcolor{gray}{\textbf{\so{er Stall}, der augenblicklich auf der Höhe{ }ſteht, nicht
                     \so{Heroine}, die im Gewicht außerordentlich begünſtigt
                  iſt, im Rennen hätte. Wie aus guter Quelle verlautet, iſt Heroine in
                  ausgezeichneter Verfaſſung und{ }ſoll ihren Trainer\pwindex{Beit, A. @\textsc{Beit, A.}, \emph{Sportler}|pwv} in der Arbeit{ }ſehr befriedigt
                  haben. Man wird gut thun, der Fulmen-Tochter\pwindex{Fulmen, Lydia von @\textsc{Fulmen, Lydia von}, \emph{Sportlerin}|pwuv} für das große Rennen die gebührende
                  Beachtung zu{ }ſchenken. \so{Nicolo} iſt ebenfalls nicht{ }ſchlecht im Handicap, jedoch nicht in Form. Sein Laufen in Köln\oindex{Köln@\textbf{Köln}, \emph{Hauptstadt}|pw} war durchaus nicht berühmt und wir glauben kaum, daß
                  von ihm eine Ueberraſchung zu erwarten iſt. Eher von \textsc{X},
                  der von Warne\pwindex{Warne, W. @\textsc{Warne, W.}, \emph{Sportler/Sportlerin}|pw} geſteuert, bei der günſtigen
                  Diſtanz durchaus nicht ohne Chancen iſt. \so{Connex} und \so{Radler} erſcheinen aus dem Lot zunächſt für die Plätze in
                  Betracht zu kommen. Zum Schluß dürfte aber doch}}\pend
           
\pstart
           \centering{}\textcolor{gray}{\textbf{\textbf{Heroine}}}\pend
           
\pstart
           \textcolor{gray}{\textbf{das beſſere Ende vor}}{ }\textcolor{gray}{\textbf{\so{Liebelei}}}\pwindex{Schnitzler, Arthur 15.\,5.\,1862 Wien – 21.\,10.\,1931 ebd.@\textsc{Schnitzler, Arthur} (15.\,5.\,1862 Wien – 21.\,10.\,1931 ebd.), \emph{Schriftsteller, Mediziner}!Liebelei. Schauspiel in drei Akten@\strich\emph{Liebelei. Schauspiel in drei Akten}|pw}\textcolor{gray}{\textbf{{ }und \textsc{X}
                     behalten{[}.{]}}}\pend
           \selectlanguage{ngerman}\endnumbering\briefempfaengerindex{Schnitzler, Arthur@\textsc{Schnitzler, Arthur}!zzzGoldmann, Paul@\emph{von Paul Goldmann}!1900-05-021@{2. 5. [1900]}|)be}\mylabel{L02914h}  \newcommand{\dateiname}{L02914}\newcommand{\titel}{Paul Goldmann an Arthur Schnitzler, 2. 5. [1900]}\newcommand{\editorInnen}{Martin Anton Müller und Laura Untner}%% latex-leseansicht-abspann.tex
%% Abspann für die Leseansicht.
%% Der Schalter \ifkorrekturansicht ist bereits durch den Vorspann gesetzt.

%% latex-abspann.tex
%% Gemeinsamer Abspann für Korrekturansicht und Leseansicht.
%% Setzt den Schalter \ifkorrekturansicht voraus (gesetzt in den
%% einbindenden Dateien latex-korrekturansicht-abspann.tex bzw.
%% latex-leseansicht-abspann.tex).
%% ---------------------------------------------------------------

\normalsize

% Das esempio-Environment wird nur in der Leseansicht benötigt
\ifkorrekturansicht\else
\newenvironment{esempio}[3]%
{
    \vspace{1.5ex}
    \rlap{\underline{#1}}
    \par
    \setlength{\parindent}{0cm}
    \nopagebreak
    \leftskip=#2cm
    \rightskip=#3cm
}
{
    \par
}
\fi

\doendnotes{C}
\bigskip
\vfill

\clearpage

\footnotesize

\ifkorrekturansicht
  \lohead{\textsc{register}}
\fi

% theindex-Environment neu definieren ohne reledmac
\makeatletter
\renewenvironment{theindex}{%
  \ifkorrekturansicht
    \section*{\indexname}%
  \else
    \subsubsection*{Index der erwähnten Entitäten}%
  \fi
  \setlength{\parindent}{0pt}%
  \setlength{\parskip}{0pt plus 0.3pt}%
  \let\item\@idxitem
}{%
  \ifkorrekturansicht\clearpage\fi
}
\makeatother

\IfFileExists{\jobname-pw.ind}{\input{\jobname-pw.ind}}{}

% Quellenangabe nur in der Leseansicht
\ifkorrekturansicht\else
% Fallback-Definitionen, falls die .tex-Datei \titel etc. nicht gesetzt hat
\providecommand{\titel}{}
\providecommand{\editorInnen}{}
\providecommand{\dateiname}{\jobname}

\vspace{3cm}

\vfill

\footnotesize
\textsc{Quelle}: \titel. Herausgegeben von {\editorInnen}. In: \emph{Arthur Schnitzler: Briefwechsel mit Autorinnen und Autoren}.
 Digitale Edition, https://schnitzler-briefe.acdh.oeaw.ac.at/{\dateiname}.html (Stand \today)
\fi

\end{document}


