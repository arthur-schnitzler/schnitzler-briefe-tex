%% latex-korrekturansicht-vorspann.tex
%% Vorspann für die Korrekturansicht.
%% Lädt die gemeinsame Datei latex-vorspann.tex mit gesetztem Schalter.

\newif\ifkorrekturansicht
\korrekturansichttrue

\input{../tex-inputs/latex-vorspann}


\section[Jakob Julius David an Arthur Schnitzler, 27. 2. 1899]{L00894 Jakob Julius David an Arthur Schnitzler, 27. 2. 1899}
\nopagebreak\mylabel{L00894v}
\rehead{ }\normalsize\beginnumbering\briefempfaengerindex{Schnitzler, Arthur@\textsc{Schnitzler, Arthur}!zzzDavid, Jakob Julius@\emph{von Jakob Julius David}!1899-02-271@{27. 2. 1899}|(be}
\toendnotes[C]{\smallbreak\pagebreak[2]}\Standort{TMW, HS Schn 1/93/1.}
\physDesc{Postkarte, 466 Zeichen
\newline{}Handschrift: schwarze Tinte, lateinische Kurrent
\newline{}Versand: 1) Rohrpost  2) Stempel: »\nobreak{}\oindex{I., Innere Stadt@\textbf{I., Innere Stadt}, \emph{A.ADM3}|pwk}Wien 1/1, 27 II 99, 1 20V\nobreak{}«.  3) Stempel: »\nobreak{}\oindex{IX., Alsergrund@\textbf{IX., Alsergrund}, \emph{A.ADM3}|pwk}Wien 9/2, 27 II 99, 1 50N\nobreak{}«. }\toendnotes[C]{\smallbreak}\pstart{}{\pb}Herrn D\textsuperscript{r}. Arthur Schnitzler\pend{}\pstart{}IX.\oindex{IX., Alsergrund@\textbf{IX., Alsergrund}, \emph{A.ADM3}|pw}\pend{}\pstart{}Franckgaße 1\oindex{Frankgasse 1@\textbf{Frankgasse 1}, \emph{Wohngebäude (K.WHS)}|pw}\pend{}{\bigskip}\vspace{1em}
\pstart\center{}{\pb}Werther
                  Herr!\pend\vspace{0.5em}
\pstart
           Ich habe heute im Theater vergeblich versucht, mir Ihre drei Einacter\pwindex{gruene Kakadu – Paracelsus – Die Gefaehrtin. Drei Einakter@\emph{Der grüne Kakadu – Paracelsus – Die Gefährtin. Drei Einakter}|pwv} zu verschaffen. Ohne Ansicht des
               Buches ka{\geminationn}{ }\uline{ich} nicht \label{K_L00894-1v}\edtext{schreiben}{\lemma{\textnormal{\emph{schreiben}}}\Cendnote{\textnormal{In Folge
                  entstand: J. J. David\pwindex{David, Jakob Julius 1859-02-06 – 1906-11-20@\textsc{David, Jakob Julius} (1859-02-06 – 1906-11-20), \emph{Schriftsteller/Schriftstellerin, Journalist/Journalistin}|pwk}: \emph{Aus ungleichen Tagen}\pwindex{Aus ungleichen Tagen@\emph{Aus ungleichen Tagen}|pwk}. In: \emph{Neues Wiener Journal}\pwindex{Neues Wiener Journal@\emph{Neues Wiener Journal}|pwk}, Jg. 7, Nr. 1925,
                        2. 3. 1899, S. 1–2.}}}\label{K_L00894-1}; ich bitte Sie also, mir die
                  Stücke\pwindex{gruene Kakadu – Paracelsus – Die Gefaehrtin. Drei Einakter@\emph{Der grüne Kakadu – Paracelsus – Die Gefährtin. Drei Einakter}|pwv} auf einige Stunden,
               nur über Nacht, es sei von heute oder morgen zu leihen. Sie sollen sie
                  Dienstag oder Mitwoch zu Ihrer paßenden Stunde dort
               finden, wo Sie wollen. Unter allen Umständen erbitte ich um Nachricht.\pend
           
\pstart
           Bestens Ihr{\\[\baselineskip]}\spacefill\mbox{David}\pend
           \leftskip=0em{}
\pstart
           \noindent{}II. Ob. Donaustraße 59\oindex{Obere Donaustrasse@\textbf{Obere Donaustraße}, \emph{Straße (K.STR)}|pw}\pend
           \selectlanguage{ngerman}\endnumbering\briefempfaengerindex{Schnitzler, Arthur@\textsc{Schnitzler, Arthur}!zzzDavid, Jakob Julius@\emph{von Jakob Julius David}!1899-02-271@{27. 2. 1899}|)be}\mylabel{L00894h}  \normalsize

\doendnotes{C}
\bigskip
\vfill

\clearpage

\footnotesize

\lohead{\textsc{register}}

% Definiere theindex-Environment komplett neu ohne reledmac
\makeatletter
\renewenvironment{theindex}{%
  \section*{\indexname}%
  \setlength{\parindent}{0pt}%
  \setlength{\parskip}{0pt plus 0.3pt}%
  \let\item\@idxitem
}{%
  \clearpage
}
\makeatother

\IfFileExists{\jobname-pw.ind}{\input{\jobname-pw.ind}}{}

\end{document}

      