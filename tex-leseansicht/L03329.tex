%% latex-korrekturansicht-vorspann.tex
%% Vorspann für die Korrekturansicht.
%% Lädt die gemeinsame Datei latex-vorspann.tex mit gesetztem Schalter.

\newif\ifkorrekturansicht
\korrekturansichttrue

\input{../tex-inputs/latex-vorspann}


\section[ Felix Salten an Arthur Schnitzler, 13. {[}5.{]} 1902]{L03329 Felix Salten an Arthur Schnitzler, 13. {[}5.{]} 1902}
\nopagebreak\mylabel{L03329v}
\rehead{ }\normalsize\beginnumbering\briefempfaengerindex{Schnitzler, Arthur@\textsc{Schnitzler, Arthur}!zzzSalten, Felix@\emph{von Felix Salten}!1902-05-131@{13. {[}5.{]} 1902}|(be}
\toendnotes[C]{\smallbreak\pagebreak[2]}\Standort{CUL, Schnitzler, B 89, A 2.}
\physDesc{Bildpostkarte, 90 Zeichen
\newline{}Handschrift: Bleistift, lateinische Kurrent
\newline{}Versand: 1) Stempel: »\nobreak{}\oindex{Trient@\textbf{Trient}|pwk}Trient 2 Trento 2, 13\textcolor{gray}{/5} 0{[}2{]}, 6\nobreak{}«.   2) Stempel: »\nobreak{}\oindex{IX., Alsergrund@\textbf{IX., Alsergrund}|pwk}9/3 W\textcolor{gray}{ien 72}, 20. {[}5. 02{]}, 8, B{[}estellt{]}\nobreak{}«. datiert: »13/5«
\newline{}Ordnung: mit Bleistift von unbekannter Hand nummeriert: »153« }\pstart{}{\pb}Herrn D\textsuperscript{r} Arthur Schnitzler\pend{}\pstart{}Wien IX.\oindex{IX., Alsergrund@\textbf{IX., Alsergrund}|pw}\pend{}\pstart{}Frankgaße N\textsuperscript{o} 1\oindex{Frankgasse 1@\textbf{Frankgasse 1}|pw}\pend{}{\bigskip}
\pstart
           {\pb}\textcolor{gray}{\textbf{Trento\oindex{Trient@\textbf{Trient}|pw}}}\hfill \textcolor{gray}{\textbf{Caserma alla Ca’ di Dio\oindex{Kaserne Ca di Dio@\textbf{Kaserne Ca’ di Dio}|pw}}}\pend
           \vspace{1em}
\pstart
           \noindent{}{\pb}Bin endlich unterwegs.\pend
           \pstart herzlichst Ihr \spacefill\mbox{Salten\textcolor{gray}{.}}\pend{}\selectlanguage{ngerman}\endnumbering\briefempfaengerindex{Schnitzler, Arthur@\textsc{Schnitzler, Arthur}!zzzSalten, Felix@\emph{von Felix Salten}!1902-05-131@{13. {[}5.{]} 1902}|)be}\mylabel{L03329h}  \normalsize

\doendnotes{C}
\bigskip
\vfill

\clearpage

\footnotesize

\lohead{\textsc{register}}

% Definiere theindex-Environment komplett neu ohne reledmac
\makeatletter
\renewenvironment{theindex}{%
  \section*{\indexname}%
  \setlength{\parindent}{0pt}%
  \setlength{\parskip}{0pt plus 0.3pt}%
  \let\item\@idxitem
}{%
  \clearpage
}
\makeatother

\IfFileExists{\jobname-pw.ind}{\input{\jobname-pw.ind}}{}

\end{document}

      