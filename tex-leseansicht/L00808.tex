%% latex-leseansicht-vorspann.tex
%% Vorspann für die Leseansicht.
%% Lädt die gemeinsame Datei latex-vorspann.tex mit nicht gesetztem Schalter.

\newif\ifkorrekturansicht
\korrekturansichtfalse

\input{../tex-inputs/latex-vorspann}


\section[Hugo von Hofmannsthal an Arthur Schnitzler, {{[}}21. 6. 1898{{]}}]{L00808 Hugo von Hofmannsthal an Arthur Schnitzler, {[}21. 6. 1898{]}}
\nopagebreak\mylabel{L00808v}
\rehead{ }\normalsize\beginnumbering\briefempfaengerindex{Schnitzler, Arthur@\textsc{Schnitzler, Arthur}!zzzHofmannsthal, Hugo von@\emph{von Hugo von Hofmannsthal}!1898-06-211@{21. 6. 1898}|(be}
\toendnotes[C]{\smallbreak\pagebreak[2]}
\correspDesc{Versand  durch Hugo von Hofmannsthal am 21. 6. 1898 in Hinterbrühl
\newline{}Erhalt  durch Arthur Schnitzler im Zeitraum [22. 6. 1898
                  – 26. 6. 1898?] in Wien}\toendnotes[C]{\smallbreak}
\Standort{CUL, Schnitzler, B 43.}
\physDesc{Brief, 1 Blatt, 4 Seiten, 806 Zeichen
\newline{}Handschrift: schwarze Tinte, deutsche Kurrent
\newline{}Schnitzler: mit schwarzer Tinte datiert: »21/6 98« 
\newline{}Ordnung: mit Bleistift von unbekannter Hand nummeriert:
                                    »115« }
\buchAbdrucke{\weitereDrucke{Hugo von Hofmannsthal, Arthur Schnitzler: \emph{Briefwechsel}. Herausgegeben von Therese Nickl und Heinrich Schnitzler. Frankfurt am Main: \emph{S. Fischer} 1964, S. 103.} }\toendnotes[C]{\smallbreak}
\pstart
           \raggedleft{}{\pb}Dienstag.\pend
           
\pstart{}mein lieber Arthur\pend\vspace{0.5em}
\pstart
           es war mir{ }ſehr leid, daſs Sie{ }ſich für \label{K_L00808-1v}\edtext{einen Tag}{\lemma{\textnormal{\emph{einen Tag}}}\Cendnote{\textnormal{Schnitzler wollte am
                     16. 6. 1898 nach Hinterbrühl\oindex{Hinterbrühl@\textbf{Hinterbrühl}, \emph{Hauptstadt}|pwk}
                  radeln, wurde aber von einem Regenguss abgehalten.}}}\label{K_L00808-1} angeſagt haben und dann
               doch nicht an einem andern geko{\geminationm}en{ }ſind, \strikeout{es} ich verlang mir{ }ſehr, mit Ihnen zuſa{\geminationm}enzuſein.\pend
           
\pstart
           Jetzt hab ich nur wenige {\pb}Tage
               mehr und die möcht ich mir{ }ſehr{ }ſparſam einteilen, bitte alſo wenn es geht, theilen
               Sie{ }ſich’s auch{ }ſo ein, wie ich Sie dann bitten werde.\pend
           
\pstart
           Übermorgen Donnerstag iſt meine \label{K_L00808-2v}\edtext{Prüfung}{\lemma{\textnormal{\emph{Prüfung}}}\Cendnote{\textnormal{Am
               23. 6. 1898 hatte Hofmannsthal\pwindex{Hofmannsthal, Hugo von 1.\,2.\,1874 Wien – 15.\,7.\,1929 Rodaun@\textsc{Hofmannsthal, Hugo von} (1.\,2.\,1874 Wien – 15.\,7.\,1929 Rodaun), \emph{Schriftsteller}|pwk} sein Hauptrigorosum in Romanischer
                  Philologie.}}}\label{K_L00808-2}, dann werde {\pb}ich Ihnen gleich{ }ſchreiben. Mittwoch den 29\textsuperscript{ten} um mittag muſs ich{ }ſchon abreiſen.\pend
           
\pstart
           Vor der Prüfung geh ich abends nicht ins Café weil ich zu müd werd.\pend
           
\pstart
           Herzlich Ihr{\\[\baselineskip]}\spacefill\mbox{Hugo.}\pend
           \leftskip=0em{}
\pstart
           \noindent{}Bitte lieber Arthur richten Sie {\pb}mir \uline{viele} Bücher die{ }ſchön zum leſen{ }ſind für
                  die Waffenübung ich hab gar nichts. Womöglich wenn Sie’s haben möcht ich auch eine
                     Novellenſa{\geminationm}lung oder{ }ſonſt etwas wo ältere
                  allenfalls phantaſtiſche Stoffe drin{ }ſind.\pend
           \selectlanguage{ngerman}\endnumbering\briefempfaengerindex{Schnitzler, Arthur@\textsc{Schnitzler, Arthur}!zzzHofmannsthal, Hugo von@\emph{von Hugo von Hofmannsthal}!1898-06-211@{21. 6. 1898}|)be}\mylabel{L00808h}  \newcommand{\dateiname}{L00808}\newcommand{\titel}{Hugo von Hofmannsthal an Arthur Schnitzler, [21. 6. 1898]}\newcommand{\editorInnen}{Martin Anton Müller und Gerd-Hermann Susen}%% latex-leseansicht-abspann.tex
%% Abspann für die Leseansicht.
%% Der Schalter \ifkorrekturansicht ist bereits durch den Vorspann gesetzt.

%% latex-abspann.tex
%% Gemeinsamer Abspann für Korrekturansicht und Leseansicht.
%% Setzt den Schalter \ifkorrekturansicht voraus (gesetzt in den
%% einbindenden Dateien latex-korrekturansicht-abspann.tex bzw.
%% latex-leseansicht-abspann.tex).
%% ---------------------------------------------------------------

\normalsize

% Das esempio-Environment wird nur in der Leseansicht benötigt
\ifkorrekturansicht\else
\newenvironment{esempio}[3]%
{
    \vspace{1.5ex}
    \rlap{\underline{#1}}
    \par
    \setlength{\parindent}{0cm}
    \nopagebreak
    \leftskip=#2cm
    \rightskip=#3cm
}
{
    \par
}
\fi

\doendnotes{C}
\bigskip
\vfill

\clearpage

\footnotesize

\ifkorrekturansicht
  \lohead{\textsc{register}}
\fi

% theindex-Environment neu definieren ohne reledmac
\makeatletter
\renewenvironment{theindex}{%
  \ifkorrekturansicht
    \section*{\indexname}%
  \else
    \subsubsection*{Index der erwähnten Entitäten}%
  \fi
  \setlength{\parindent}{0pt}%
  \setlength{\parskip}{0pt plus 0.3pt}%
  \let\item\@idxitem
}{%
  \ifkorrekturansicht\clearpage\fi
}
\makeatother

\IfFileExists{\jobname-pw.ind}{\input{\jobname-pw.ind}}{}

% Quellenangabe nur in der Leseansicht
\ifkorrekturansicht\else
% Fallback-Definitionen, falls die .tex-Datei \titel etc. nicht gesetzt hat
\providecommand{\titel}{}
\providecommand{\editorInnen}{}
\providecommand{\dateiname}{\jobname}

\vspace{3cm}

\vfill

\footnotesize
\textsc{Quelle}: \titel. Herausgegeben von {\editorInnen}. In: \emph{Arthur Schnitzler: Briefwechsel mit Autorinnen und Autoren}.
 Digitale Edition, https://schnitzler-briefe.acdh.oeaw.ac.at/{\dateiname}.html (Stand \today)
\fi

\end{document}


