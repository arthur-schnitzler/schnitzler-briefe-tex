%% latex-korrekturansicht-vorspann.tex
%% Vorspann für die Korrekturansicht.
%% Lädt die gemeinsame Datei latex-vorspann.tex mit gesetztem Schalter.

\newif\ifkorrekturansicht
\korrekturansichttrue

\input{../tex-inputs/latex-vorspann}


\section[Arthur Schnitzler an Hermann Bahr, 19. 5. 1900]{L01038 Arthur Schnitzler an Hermann Bahr, 19. 5. 1900}
\nopagebreak\mylabel{L01038v}
\rehead{ }\normalsize\beginnumbering\briefempfaengerindex{Bahr, Hermann@\textsc{Bahr, Hermann}!zzzSchnitzler, Arthur@\emph{von Arthur Schnitzler}!1900-05-192@{19. 5. 1900}|(be}
\toendnotes[C]{\smallbreak\pagebreak[2]}\Standort{TMW, HS AM 23337 Ba.}
\physDesc{Brief, 1 Blatt, 2 Seiten, 311 Zeichen
\newline{}Handschrift: schwarze Tinte, deutsche Kurrent
\newline{}Ordnung: Lochung }
\buchAbdrucke{\weitereDrucke{1) Arthur Schnitzler: \emph{The Letters of Arthur Schnitzler to Hermann Bahr}. Chapel Hill: \emph{The University of North Carolina Press} 1978, S. 66.} \weitereDrucke{2) Hermann Bahr, Arthur Schnitzler: \emph{Briefwechsel, Aufzeichnungen, Dokumente (1891–1931)}. Göttingen: \emph{Wallstein} 2018, S. 176.} }\toendnotes[C]{\smallbreak}
\pstart{}{\pb}lieber Hermann,
               \pend\vspace{0.5em}
\pstart
           ich habe gar nichts dagegen, we{\geminationn} du Herrn Doctor Geiringer\pwindex{Geiringer, Friedrich 22.01.1859 – 19.10.1923@\textsc{Geiringer, Friedrich} (22.01.1859 – 19.10.1923), \emph{Rechtsanwalt/Rechtsanwältin}|pw} dein Exemplar des »Reigen\pwindex{Reigen. Zehn Dialoge@\emph{Reigen. Zehn Dialoge}|pw}« leihweiſe zur Verfügung ſtellſt. Ich ſelbſt will u ka{\geminationn} eigentlich ein Buch von mir nicht \label{T_L01038-1v}\edtext{\uline{herleihen}}{\lemma{\textnormal{\emph{herleihen}}}\Cendnote{\textnormal{Unterstreichung am Papier erkennbar, aber
                  teilweise ohne Tinte; wohl zur Verdeutlichung »leihen« über dem Text
                  wiederholt}}}\label{T_L01038-1}; mü\textcolor{gray}{ß}t es {\pb}gleich herſchenken, nur
               dazu reichen mir die Exemplare nicht mehr.\pend
           
\pstart
           Herzlich grüßend{\\[\baselineskip]}dein \spacefill\mbox{Arthur Schn}\pend
           \leftskip=0em{}
\pstart
           19. 5. 900.\pend
           \selectlanguage{ngerman}\endnumbering\briefempfaengerindex{Bahr, Hermann@\textsc{Bahr, Hermann}!zzzSchnitzler, Arthur@\emph{von Arthur Schnitzler}!1900-05-192@{19. 5. 1900}|)be}\mylabel{L01038h}  \normalsize

\doendnotes{C}
\bigskip
\vfill

\clearpage

\footnotesize

\lohead{\textsc{register}}

% Definiere theindex-Environment komplett neu ohne reledmac
\makeatletter
\renewenvironment{theindex}{%
  \section*{\indexname}%
  \setlength{\parindent}{0pt}%
  \setlength{\parskip}{0pt plus 0.3pt}%
  \let\item\@idxitem
}{%
  \clearpage
}
\makeatother

\IfFileExists{\jobname-pw.ind}{\input{\jobname-pw.ind}}{}

\end{document}

      