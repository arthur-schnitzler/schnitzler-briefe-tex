%% latex-leseansicht-vorspann.tex
%% Vorspann für die Leseansicht.
%% Lädt die gemeinsame Datei latex-vorspann.tex mit nicht gesetztem Schalter.

\newif\ifkorrekturansicht
\korrekturansichtfalse

\input{../tex-inputs/latex-vorspann}


\section[Sigmund Freud an Arthur Schnitzler, 8. 6. 1922]{L03888 Sigmund Freud an Arthur Schnitzler, 8. 6. 1922}
\nopagebreak\mylabel{L03888v}
\rehead{ }\normalsize\beginnumbering\briefempfaengerindex{Schnitzler, Arthur@\textsc{Schnitzler, Arthur}!zzzFreud, Sigmund@\emph{von Sigmund Freud}!1922-06-081@{8. 6. 1922}|(be}
\toendnotes[C]{\smallbreak\pagebreak[2]}
\correspDesc{Versand  durch Sigmund Freud am 8. 6. 1922 in Wien
\newline{}Erhalt  durch Arthur Schnitzler im Zeitraum [9. 6. 1922 – 13. 6. 1922?] \textbf{Ort fehlend} }\toendnotes[C]{\smallbreak}
\Standort{Washington, DC, Library of Congress, Freud Archives, C41F8.}
\physDesc{Brief, Fotokopie, 2 Blätter, 2 Seiten, 986 Zeichen
\newline{}Handschrift: schwarze Tinte, deutsche Kurrent
\newline{}Zusatz: Der Verbleib des Originals ist ungeklärt. Zum Zeitpunkt
                              der ersten Edition 1955 befand es sich
                                 im Besitz von Heinrich Schnitzler\pwindex{Schnitzler, Heinrich 9.\,8.\,1902 Hinterbrühl – 12.\,7.\,1982 Wien@\textsc{Schnitzler, Heinrich} (9.\,8.\,1902 Hinterbrühl – 12.\,7.\,1982 Wien), \emph{Regisseur, Schauspieler}|pw}. }
\buchAbdrucke{\weitereDrucke{1) Sigmund Freud: \emph{Briefe an Arthur Schnitzler.}Herausgegeben von Henry Schnitzler In: \emph{Neue deutsche Rundschau}, Jg. 66 (Januar 1955) Nr. 1, S. 98.} \weitereDrucke{2) Sigmund Freud: \emph{Sigmund Freud Edition. Digitale historisch-kritische Gesamtausgabe}. Herausgegeben von Christine Diercks, Arkadi Blatow und Elisabeth Skale. (2014–2025) \url{https://www.freudedition.net/briefe/freud-sigmund/schnitzler-arthur/1922/06/08}.} }\toendnotes[C]{\smallbreak}
\pstart
           \raggedleft{}{\pb}8. Juni 1922\pend
           
\pstart
           \textcolor{gray}{\textbf{PROF. D\textsuperscript{R.} FREUD}}\hfill \textcolor{gray}{\textbf{WIEN IX., BERGGASSE 19\oindex{Wien@\textbf{Wien}!IX., Alsergrund@\textbf{IX., Alsergrund}!Berggasse 19@\textbf{Berggasse 19}, \emph{Wohngebäude}|pw}. }}\pend
           
\pstart\center{}Verehrter Herr Doktor\pend\vspace{0.5em}
\pstart
           Sie{ }ſtellen mir in Ihrem liebenswürdigen
               \label{K_L03887-1v}\edtext{Schreiben}{\lemma{\textnormal{\emph{Schreiben}}}\Cendnote{\textnormal{nicht überliefert}}}\label{K_L03887-1} eine Zuſammenkunft oder einen –
               Beſuch in Ausſicht{ }ſo daß wir einmal mit
               einander plaudern können,{ }ſo lange es noch
               Zeit iſt, wie Sie andeuten. Ich freue mich darauf,
               ohne mir ein Programm für dieſe Stunden
               zu machen.\pend
           
\pstart
           Darf ich Ihnen nun vorſchlagen, einfach an
               einem \label{K_L03887-2v}\edtext{Abend der nächſten Woche}{\lemma{\textnormal{\emph{Abend der nächsten Woche}}}\Cendnote{\textnormal{Vgl. A. S.: \emph{Tagebuch}, 16. 6. 1922.}}}\label{K_L03887-2} ein Abendeſſen mit uns zu teilen? Wir{ }ſind: meine
               Frau\pwindex{Freud, Martha 26.\,7.\,1861 Hamburg – 2.\,11.\,1951 London@\textsc{Freud, Martha} (26.\,7.\,1861 Hamburg – 2.\,11.\,1951 London)|pwv} und die Ihnen \label{K_L03887-3v}\edtext{bereits bekannte Tochter\pwindex{Freud, Anna 3.\,12.\,1895 Wien – 9.\,10.\,1982 London@\textsc{Freud, Anna} (3.\,12.\,1895 Wien – 9.\,10.\,1982 London), \emph{Psychoanalytikerin}|pwv}}{\lemma{\textnormal{\emph{bereits bekannte Tochter}}}\Cendnote{\textnormal{Schnitzlers
                  Tochter Lili\pwindex{Cappellini, Lili 13.\,9.\,1909 Wien – 26.\,7.\,1928 Venedig@\textsc{Cappellini, Lili} (13.\,9.\,1909 Wien – 26.\,7.\,1928 Venedig)|pwk} hatte 1921 ein paar Stunden bei Anna Freud\pwindex{Freud, Anna 3.\,12.\,1895 Wien – 9.\,10.\,1982 London@\textsc{Freud, Anna} (3.\,12.\,1895 Wien – 9.\,10.\,1982 London), \emph{Psychoanalytikerin}|pwk} genommen.
               }}}\label{K_L03887-3}
               außer meiner Perſon. Es wird kein anderer
               mit dabei{ }ſein.  Da ich tagsüber bis 8\textsuperscript{h} in
               der Arbeit bin und einige Abende regelmäßig beſetzt habe, muß ich mich beſtimmterer
               Vorſchläge getrauen. Ich lege Ihnen den
               12\textsuperscript{t} (Montag), 13\textsuperscript{t} (Dienstag), 16\textsuperscript{ten} (Freitag) zur
               Auswal vor, wenn Ihnen dieſe Woche und
               Art des Beiſa{\geminationm}enſeins überhaupt recht iſt.
               Da ich zufällig gehört habe, daß Sie in Wien\oindex{Wien@\textbf{Wien}, \emph{Verwaltungsgebiet}|pw}
               geblieben{ }ſind, und ich{ }ſelbſt am 29 d. M.
               die Stadt verlaſſe,{ }ſchreibe ich Ihnen früher,
               als mich Ihr Brief berechtigt hätte.\pend
           
\pstart
           Ihr herzlich ergebener{\\[\baselineskip]}\spacefill\mbox{Freud}\pend
           \leftskip=0em{}\selectlanguage{ngerman}\endnumbering\briefempfaengerindex{Schnitzler, Arthur@\textsc{Schnitzler, Arthur}!zzzFreud, Sigmund@\emph{von Sigmund Freud}!1922-06-081@{8. 6. 1922}|)be}\mylabel{L03888h}
\begin{anhang}
\end{anhang}\newcommand{\dateiname}{L03888}\newcommand{\titel}{Sigmund Freud an Arthur Schnitzler, 8. 6. 1922}\newcommand{\editorInnen}{Selma Jahnke und Martin Anton Müller}%% latex-leseansicht-abspann.tex
%% Abspann für die Leseansicht.
%% Der Schalter \ifkorrekturansicht ist bereits durch den Vorspann gesetzt.

%% latex-abspann.tex
%% Gemeinsamer Abspann für Korrekturansicht und Leseansicht.
%% Setzt den Schalter \ifkorrekturansicht voraus (gesetzt in den
%% einbindenden Dateien latex-korrekturansicht-abspann.tex bzw.
%% latex-leseansicht-abspann.tex).
%% ---------------------------------------------------------------

\normalsize

% Das esempio-Environment wird nur in der Leseansicht benötigt
\ifkorrekturansicht\else
\newenvironment{esempio}[3]%
{
    \vspace{1.5ex}
    \rlap{\underline{#1}}
    \par
    \setlength{\parindent}{0cm}
    \nopagebreak
    \leftskip=#2cm
    \rightskip=#3cm
}
{
    \par
}
\fi

\doendnotes{C}
\bigskip
\vfill

\clearpage

\footnotesize

\ifkorrekturansicht
  \lohead{\textsc{register}}
\fi

% theindex-Environment neu definieren ohne reledmac
\makeatletter
\renewenvironment{theindex}{%
  \ifkorrekturansicht
    \section*{\indexname}%
  \else
    \subsubsection*{Index der erwähnten Entitäten}%
  \fi
  \setlength{\parindent}{0pt}%
  \setlength{\parskip}{0pt plus 0.3pt}%
  \let\item\@idxitem
}{%
  \ifkorrekturansicht\clearpage\fi
}
\makeatother

\IfFileExists{\jobname-pw.ind}{\input{\jobname-pw.ind}}{}

% Quellenangabe nur in der Leseansicht
\ifkorrekturansicht\else
% Fallback-Definitionen, falls die .tex-Datei \titel etc. nicht gesetzt hat
\providecommand{\titel}{}
\providecommand{\editorInnen}{}
\providecommand{\dateiname}{\jobname}

\vspace{3cm}

\vfill

\footnotesize
\textsc{Quelle}: \titel. Herausgegeben von {\editorInnen}. In: \emph{Arthur Schnitzler: Briefwechsel mit Autorinnen und Autoren}.
 Digitale Edition, https://schnitzler-briefe.acdh.oeaw.ac.at/{\dateiname}.html (Stand \today)
\fi

\end{document}


