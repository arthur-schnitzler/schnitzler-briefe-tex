\input{../tex-inputs/latex-pdf-vorspann}
\begin{center}
            \textcolor{red}{ENTWURF. ENTZIFFERUNG NOCH NICHT KORREKTURGELESEN}
                      \end{center}
            
               \section[Richard Beer-Hofmann an Arthur Schnitzler, 26. 6. 1905]{ Richard Beer-Hofmann an Arthur Schnitzler, 26. 6. 1905}\nopagebreak\mylabel{v}\rehead{ }\begin{ledgroupsized}[t]{13cm}\normalsize\beginnumbering\briefempfaengerindex{Schnitzler, Arthur@\textsc{Schnitzler, Arthur}!zzzBeer-Hofmann, Richard@\emph{von Richard Beer-Hofmann}!1905-06-261@{26. 6. 1905}|(be} \toendnotes[C]{\smallbreak\pagebreak[2]} \Standort{CUL, Schnitzler, B 8.}
\physDesc{Kartenbrief
\newline{}Handschrift: Bleistift, lateinische Kurrent\newline{}Versand: 1) Rohrpost 2) Stempel: »\nobreak{}\oindex{I., Innere Stadt@\textbf{I., Innere Stadt}|pwk}Wien 1/\textcolor{gray}{1}, 26 VI 05, 7–N\nobreak{}«. 3) Stempel: »\nobreak{}\oindex{XVIII., Waehring@\textbf{XVIII., Währing}|pwk}18/1 Wien 111, 26 VI 05, 7\textsuperscript{50}\nobreak{}«. \newline{}Ordnung: mit Bleistift von unbekannter Hand nummeriert:
                                    »201« }\toendnotes[C]{\smallbreak}\pstart{}{\pb}Beer-Hofmann\pend{}\pstart{}Richard\pend{}{\bigskip}\pstart{}Herrn\pend{}\pstart{}Arthur Schnitzler\pend{}\pstart{}Wien\oindex{Wien@\textbf{Wien}|pw}\pend{}\pstart{}VIII Spöttelgasse 7\oindex{Edmund-Weiss-Gasse@\textbf{Edmund-Weiß-Gasse}|pw}\pend{}{\bigskip}\pstart
           \noindent{}{\pb}Lieber Arthur! Welchen Tag dieser Woche – \label{K_L01527_1v}\edtext{Feiertag}{\lemma{\textnormal{\emph{Feiertag}}}\Cendnote{\textnormal{am 29. Juni war Peter- und Paulstag}}}\label{K_L01527_1h}
                  ausgeno{\geminationm}en, wollen Sie (Plural) – endlich – \strikeout{mit} zu uns am Nachmittag ko{\geminationm}en und bis zum 10.13 bleiben?
               Telegraphiren Sie mir womöglich morgen – oder telephoniren Sie.
               Herzlich\pend
           \pstart
           \spacefill\mbox{Richard}\pend
           \endnumbering\briefempfaengerindex{Schnitzler, Arthur@\textsc{Schnitzler, Arthur}!zzzBeer-Hofmann, Richard@\emph{von Richard Beer-Hofmann}!1905-06-261@{26. 6. 1905}|)be}\mylabel{h}\end{ledgroupsized}  \newcommand{\dateiname}{L01527}\newcommand{\titel}{Richard Beer-Hofmann an Arthur Schnitzler, 26. 6. 1905}\newcommand{\editorInnen}{Martin Anton Müller und Gerd-Hermann Susen}\input{../tex-inputs/latex-pdf-abspann}
      