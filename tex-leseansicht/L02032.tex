%% latex-leseansicht-vorspann.tex
%% Vorspann für die Leseansicht.
%% Lädt die gemeinsame Datei latex-vorspann.tex mit nicht gesetztem Schalter.

\newif\ifkorrekturansicht
\korrekturansichtfalse

\input{../tex-inputs/latex-vorspann}


\section[Thomas Mann an Arthur Schnitzler, 25. 9. 1911]{L02032 Thomas Mann an Arthur Schnitzler, 25. 9. 1911}
\nopagebreak\mylabel{L02032v}
\rehead{ }\normalsize\beginnumbering\briefempfaengerindex{Schnitzler, Arthur@\textsc{Schnitzler, Arthur}!zzzMann, Thomas@\emph{von Thomas Mann}!1911-09-251@{25. 9.  1911}|(be}
\toendnotes[C]{\smallbreak\pagebreak[2]}
\correspDesc{Versand  durch Thomas Mann am 25. 9.  1911 in Bad Tölz
\newline{}Erhalt  durch Arthur Schnitzler im Zeitraum [26. 9. 1911
                  – 30. 9. 1911?] in Wien}\toendnotes[C]{\smallbreak}
\Standort{CUL, Schnitzler, B 67.}
\physDesc{Brief, 1 Blatt, 4 Seiten, 1540 Zeichen
\newline{}Handschrift: schwarze Tinte, deutsche Kurrent
\newline{}Schnitzler: 1) mit Bleistift beschriftet: »Ma{\geminationn}«  2) mit rotem Buntstift zwei Unterstreichungen}
\buchAbdrucke{\weitereDrucke{1) Hertha Krotkoff: \emph{Arthur Schnitzler – Thomas Mann: Briefe.} In: \emph{Modern Austrian Literature}, Jg. 7 (1974) Nr. 1/2, S. 14–15.} \weitereDrucke{2) Hans-Ulrich Lindken: \emph{Arthur Schnitzler. Aspekte und Akzente. Materialien zu Leben
                        und Werk}. Frankfurt am Main, Bern, Göttingen: \emph{Peter Lang} 1984, S. 196–197 (Europäische Hochschulschriften, Reihe 1, Deutsche Sprache und
                        Literatur, 754).} }\toendnotes[C]{\smallbreak}
\pstart
           \raggedleft{}{\pb}\textcolor{gray}{\textbf{\textsc{Bad Tölz}\oindex{Bad Tölz@\textbf{Bad Tölz}, \emph{Hauptstadt}|pw}\textsc{, den}}}{ }25. IX. 1911.\pend
           
\pstart
           \raggedleft{}\textcolor{gray}{\textbf{LANDHAUS THOMAS MANN\oindex{Thomas Mann Villa@\textbf{Thomas Mann Villa}, \emph{Wohngebäude}|pw}.}}\pend
           
\pstart{}Sehr verehrter Herr Doctor:\pend\vspace{0.5em}
\pstart
           Durch meinen Bruder\pwindex{Mann, Heinrich 27.\,3.\,1871 Lübeck – 11.\,3.\,1950 Santa Monica@\textsc{Mann, Heinrich} (27.\,3.\,1871 Lübeck – 11.\,3.\,1950 Santa Monica), \emph{Schriftsteller}|pwv}, der zur
               Zeit bei uns wohnt, erfahre ich von dem Hinſcheiden Ihrer Mutter\pwindex{Schnitzler, Louise 8.\,7.\,1840 Kőszeg – 9.\,9.\,1911 Wien@\textsc{Schnitzler, Louise} (8.\,7.\,1840 Kőszeg – 9.\,9.\,1911 Wien)|pwv} und möchte Sie bitten, den Ausdruck
               auch meiner herzlichen Teilnahme freundlichſt entgegenzunehmen.\pend
           
\pstart
           Ich las mit großer Bewunderung Ihre{ }ſo wunderbar gehobene \label{K_L02032-111v}\edtext{Dichtung\pwindex{Schnitzler, Arthur 15.\,5.\,1862 Wien – 21.\,10.\,1931 ebd.@\textsc{Schnitzler, Arthur} (15.\,5.\,1862 Wien – 21.\,10.\,1931 ebd.), \emph{Schriftsteller, Mediziner}!Hirtenflöte. Novelle@\strich\emph{Die Hirtenflöte. Novelle}|pwv}}{\lemma{\textnormal{\emph{Dichtung}}}\Cendnote{\textnormal{Arthur Schnitzler: \emph{Die Hirtenflöte.
                             Novelle}\pwindex{Schnitzler, Arthur 15.\,5.\,1862 Wien – 21.\,10.\,1931 ebd.@\textsc{Schnitzler, Arthur} (15.\,5.\,1862 Wien – 21.\,10.\,1931 ebd.), \emph{Schriftsteller, Mediziner}!Hirtenflöte. Novelle@\strich\emph{Die Hirtenflöte. Novelle}|pwk}. In: \emph{Die neue
                                 Rundschau}\pwindex{Neue Deutsche Rundschau@\emph{Neue Deutsche Rundschau}|pwk}, Jg. 22, H. 9, September 1911,
                         1249–1273.}}}\label{K_L02032-111} in der »Rundſchau\pwindex{neue Rundschau@\emph{Die neue Rundschau}|pw}« und erwarte mit freudiger Ungeduld die Münchner\oindex{München@\textbf{München}|pw}{ }{\pb}\label{K_L02032-1v}\edtext{Erſtaufführung\eventindex{Residenztheater München@\textbf{Residenztheater München}!Premiere von Das weite Land, 14.10.1911 [IV.]@Premiere von Das weite Land, 14.10.1911 [IV.]|pwv}}{\lemma{\textnormal{\emph{Erstaufführung}}}\Cendnote{\textnormal{Am 14. 10. 1911 fand die
                    Uraufführung\eventindex{Residenztheater München@\textbf{Residenztheater München}!Premiere von Das weite Land, 14.10.1911 [IV.]@Premiere von Das weite Land, 14.10.1911 [IV.]|pwkv}\eventindex{Burgtheater@\textbf{Burgtheater}!Premiere von Das weite Land, 14.10.1911 [I.]@Premiere von Das weite Land, 14.10.1911 [I.]|pwkv}\eventindex{Lobe-Theater@\textbf{Lobe-Theater}!Premiere von Das weite Land, 14.10.1911 [II.]@Premiere von Das weite Land, 14.10.1911 [II.]|pwkv}\eventindex{Lessing-Theater@\textbf{Lessing-Theater}!Premiere von Das weite Land, 14.10.1911 [III.]@Premiere von Das weite Land, 14.10.1911 [III.]|pwkv}\eventindex{Neues Deutsches Theater@\textbf{Neues Deutsches Theater}!Premiere von Das weite Land, 14.10.1911 [V.]@Premiere von Das weite Land, 14.10.1911 [V.]|pwkv}\eventindex{Leipzig@\textbf{Leipzig}!Premiere von Das weite Land, 14.10.1911 [VI.]@Premiere von Das weite Land, 14.10.1911 [VI.]|pwkv}\eventindex{Hannover@\textbf{Hannover}!Premiere von Das weite Land, 14.10.1911 [VII.]@Premiere von Das weite Land, 14.10.1911 [VII.]|pwkv} in mehreren Städten gleichzeitig statt.}}}\label{K_L02032-1} Ihres neuen Stückes\pwindex{Schnitzler, Arthur 15.\,5.\,1862 Wien – 21.\,10.\,1931 ebd.@\textsc{Schnitzler, Arthur} (15.\,5.\,1862 Wien – 21.\,10.\,1931 ebd.), \emph{Schriftsteller, Mediziner}!weite Land. Tragikomödie in fünf Akten@\strich\emph{Das weite Land. Tragikomödie in fünf Akten}|pwv}. Meinen Bruder\pwindex{Mann, Heinrich 27.\,3.\,1871 Lübeck – 11.\,3.\,1950 Santa Monica@\textsc{Mann, Heinrich} (27.\,3.\,1871 Lübeck – 11.\,3.\,1950 Santa Monica), \emph{Schriftsteller}|pwv}{ }ſehe ich{ }ſchwer verſtimmt – und bin es mit ihm –
               über das Fehlſchlagen der Hoffnungen, die er auf sein Drama\pwindex{Mann, Heinrich 27.\,3.\,1871 Lübeck – 11.\,3.\,1950 Santa Monica@\textsc{Mann, Heinrich} (27.\,3.\,1871 Lübeck – 11.\,3.\,1950 Santa Monica), \emph{Schriftsteller}!Schauspielerin@\strich\emph{Schauspielerin}|pwv} geſetzt hatte. Ich habe es erſt jetzt hier in der
               Korrektur geleſen und muß zum Mindeſten die Energie bewundern, mit der ein an weit
               ausladender Breite gewöhnter Romancier\pwindex{Mann, Heinrich 27.\,3.\,1871 Lübeck – 11.\,3.\,1950 Santa Monica@\textsc{Mann, Heinrich} (27.\,3.\,1871 Lübeck – 11.\,3.\,1950 Santa Monica), \emph{Schriftsteller}|pwv}{ }ſo viel Leidenſchaft und Schickſal in ein paar
               knappe Dialoge zuſammenzupreſſen vermochte. Gewiß, die Theaterdirektoren thun {\pb}höchſt Unrecht, das Stück\pwindex{Mann, Heinrich 27.\,3.\,1871 Lübeck – 11.\,3.\,1950 Santa Monica@\textsc{Mann, Heinrich} (27.\,3.\,1871 Lübeck – 11.\,3.\,1950 Santa Monica), \emph{Schriftsteller}!Schauspielerin@\strich\emph{Schauspielerin}|pwv} zurückzuweiſen! Es mag{ }ſein, daß die
               beiden{ }ſpäteren Akte gegen den erſten an Bühnenwirkſamkeit zurückſtehen, aber
               dichteriſch genommen bringen{ }ſie die eindringlichſten Dinge, und die{ }ſchönſten
               Repliken{ }ſind in ihnen enthalten. Und iſt es nicht{ }ſchließlich{ }ſo, daß eine
               dramatiſche Arbeit dieſes Autors\pwindex{Mann, Heinrich 27.\,3.\,1871 Lübeck – 11.\,3.\,1950 Santa Monica@\textsc{Mann, Heinrich} (27.\,3.\,1871 Lübeck – 11.\,3.\,1950 Santa Monica), \emph{Schriftsteller}|pwv} ohne Weiteres aufgeführt werden müßte? Wäre das nicht eine{ }ſelbſtverſtändliche Aufmerkſamkeit des Theaters gegen den Dichter\pwindex{Mann, Heinrich 27.\,3.\,1871 Lübeck – 11.\,3.\,1950 Santa Monica@\textsc{Mann, Heinrich} (27.\,3.\,1871 Lübeck – 11.\,3.\,1950 Santa Monica), \emph{Schriftsteller}|pwv} der »Kleinen Stadt\pwindex{Mann, Heinrich 27.\,3.\,1871 Lübeck – 11.\,3.\,1950 Santa Monica@\textsc{Mann, Heinrich} (27.\,3.\,1871 Lübeck – 11.\,3.\,1950 Santa Monica), \emph{Schriftsteller}!kleine Stadt@\strich\emph{Die kleine Stadt}|pw}«? Entfällt da{\pb}bei für die Direktoren nicht jede
               künſtleriſche Verantwortung? Hoffentlich erkennt nun wenigſtens Frau Durieux\pwindex{Durieux, Tilla 18.\,8.\,1880 Wien – 20.\,2.\,1971 Berlin@\textsc{Durieux, Tilla} (18.\,8.\,1880 Wien – 20.\,2.\,1971 Berlin), \emph{Schauspielerin}|pw} in Berlin\oindex{Berlin@\textbf{Berlin}, \emph{Hauptstadt}|pw} in der Leonie\pwindex{Mann, Heinrich 27.\,3.\,1871 Lübeck – 11.\,3.\,1950 Santa Monica@\textsc{Mann, Heinrich} (27.\,3.\,1871 Lübeck – 11.\,3.\,1950 Santa Monica), \emph{Schriftsteller}!Schauspielerin@\strich\emph{Schauspielerin}|pwv}
               eine gute Rolle.\pend
           
\pstart
           Mit den beſten Empfehlungen an Sie und Ihre Gattin\pwindex{Schnitzler, Olga 17.\,1.\,1882 Wien – 13.\,1.\,1970 Lugano@\textsc{Schnitzler, Olga} (17.\,1.\,1882 Wien – 13.\,1.\,1970 Lugano), \emph{Schauspielerin, Sängerin}|pwv}, {\\}ſehr verehrter Herr Doctor,\pend
           
\pstart
           Ihr ergebenſter{\\[\baselineskip]}\spacefill\mbox{Thomas Mann.}\pend
           \leftskip=0em{}\selectlanguage{ngerman}\endnumbering\briefempfaengerindex{Schnitzler, Arthur@\textsc{Schnitzler, Arthur}!zzzMann, Thomas@\emph{von Thomas Mann}!1911-09-251@{25. 9.  1911}|)be}\mylabel{L02032h}  \newcommand{\dateiname}{L02032}\newcommand{\titel}{Thomas Mann an Arthur Schnitzler, 25. 9. 1911}\newcommand{\editorInnen}{Martin Anton Müller und Gerd-Hermann Susen}%% latex-leseansicht-abspann.tex
%% Abspann für die Leseansicht.
%% Der Schalter \ifkorrekturansicht ist bereits durch den Vorspann gesetzt.

%% latex-abspann.tex
%% Gemeinsamer Abspann für Korrekturansicht und Leseansicht.
%% Setzt den Schalter \ifkorrekturansicht voraus (gesetzt in den
%% einbindenden Dateien latex-korrekturansicht-abspann.tex bzw.
%% latex-leseansicht-abspann.tex).
%% ---------------------------------------------------------------

\normalsize

% Das esempio-Environment wird nur in der Leseansicht benötigt
\ifkorrekturansicht\else
\newenvironment{esempio}[3]%
{
    \vspace{1.5ex}
    \rlap{\underline{#1}}
    \par
    \setlength{\parindent}{0cm}
    \nopagebreak
    \leftskip=#2cm
    \rightskip=#3cm
}
{
    \par
}
\fi

\doendnotes{C}
\bigskip
\vfill

\clearpage

\footnotesize

\ifkorrekturansicht
  \lohead{\textsc{register}}
\fi

% theindex-Environment neu definieren ohne reledmac
\makeatletter
\renewenvironment{theindex}{%
  \ifkorrekturansicht
    \section*{\indexname}%
  \else
    \subsubsection*{Index der erwähnten Entitäten}%
  \fi
  \setlength{\parindent}{0pt}%
  \setlength{\parskip}{0pt plus 0.3pt}%
  \let\item\@idxitem
}{%
  \ifkorrekturansicht\clearpage\fi
}
\makeatother

\IfFileExists{\jobname-pw.ind}{\input{\jobname-pw.ind}}{}

% Quellenangabe nur in der Leseansicht
\ifkorrekturansicht\else
% Fallback-Definitionen, falls die .tex-Datei \titel etc. nicht gesetzt hat
\providecommand{\titel}{}
\providecommand{\editorInnen}{}
\providecommand{\dateiname}{\jobname}

\vspace{3cm}

\vfill

\footnotesize
\textsc{Quelle}: \titel. Herausgegeben von {\editorInnen}. In: \emph{Arthur Schnitzler: Briefwechsel mit Autorinnen und Autoren}.
 Digitale Edition, https://schnitzler-briefe.acdh.oeaw.ac.at/{\dateiname}.html (Stand \today)
\fi

\end{document}


