%% latex-leseansicht-vorspann.tex
%% Vorspann für die Leseansicht.
%% Lädt die gemeinsame Datei latex-vorspann.tex mit nicht gesetztem Schalter.

\newif\ifkorrekturansicht
\korrekturansichtfalse

\input{../tex-inputs/latex-vorspann}


         
         \renewcommand{\erwaehntePersonen}{Personen: Tilla Durieux, Heinrich Mann, Louise Schnitzler, Olga Schnitzler}
         \renewcommand{\erwaehnteOrte}{Orte: Bad Tölz, Berlin, München, Thomas Mann Villa, Wien}
         \renewcommand{\erwaehnteWerke}{Werke: Das weite Land. Tragikomödie in fünf Akten, Die Hirtenflöte. Novelle, Die kleine Stadt, Die neue Rundschau, Schauspielerin}
               \section[Thomas Mann an Arthur Schnitzler, 25. 9. 1911]{ Thomas Mann an Arthur Schnitzler, 25. 9. 1911}\nopagebreak\mylabel{v}\rehead{ }\begin{ledgroupsized}[t]{13cm}\normalsize\beginnumbering \toendnotes[C]{\smallbreak\pagebreak[2]} \Standort{CUL, Schnitzler, B 67.}
\physDesc{Brief, 1 Blatt, 4 Seiten, 1540 Zeichen
\newline{}Handschrift: schwarze Tinte, deutsche Kurrent
\newline{}Schnitzler: 1) mit Bleistift beschriftet: »Ma{\geminationn}«  2) mit rotem Buntstift zwei Unterstreichungen}\buchAbdrucke{\weitereDrucke{1) Hertha Krotkoff: \emph{Arthur Schnitzler – Thomas Mann: Briefe.} In: \emph{Modern Austrian Literature}, Jg. 7 (1974) Nr. 1/2, S. 14–15.} \weitereDrucke{2) Hans-Ulrich Lindken: \emph{Arthur Schnitzler. Aspekte und Akzente. Materialien zu Leben
                        und Werk}. Frankfurt am Main, Bern, Göttingen: \emph{Peter Lang} 1984, S. 196–197 (Europäische Hochschulschriften, Reihe 1, Deutsche Sprache und
                        Literatur, 754).} }\toendnotes[C]{\smallbreak}\pstart
           \noindent{}\raggedleft{}{\pb}\textcolor{gray}{\textbf{\textsc{Bad Tölz}\oindex{Bad Toelz@\textbf{Bad Tölz}|pw}\textsc{, den}}}{ }25. IX. 1911.\pend
           \pstart
           \noindent{}\raggedleft{}\textcolor{gray}{\textbf{LANDHAUS THOMAS MANN.\oindex{Thomas Mann Villa@\textbf{Thomas Mann Villa}|pw}}}\pend
           \pstart{}Sehr verehrter Herr Doctor:\pend\pstart
           Durch meinen Bruder\pwindex{Mann, Heinrich 27.03.1871 – 11.03.1950@\textsc{Mann, Heinrich} (27.03.1871 – 11.03.1950), \emph{Schriftsteller}|pwv}, der zur
               Zeit bei uns wohnt, erfahre ich von dem Hinſcheiden Ihrer Mutter\pwindex{Schnitzler, Louise 1840-07-08 – 1911-09-09@\textsc{Schnitzler, Louise} (1840-07-08 – 1911-09-09)|pwv} und möchte Sie bitten, den Ausdruck
               auch meiner herzlichen Teilnahme freundlichſt entgegenzunehmen.\pend
           \pstart
           Ich las mit großer Bewunderung Ihre ſo wunderbar gehobene Dichtung\pwindex{Schnitzler, Arthur 15.05.1862 – 21.10.1931@\textsc{Schnitzler, Arthur} (15.05.1862 – 21.10.1931), \emph{Schriftsteller, Mediziner}!Hirtenfloete. NovelleSeptember 1911@\strich\emph{Die Hirtenflöte. Novelle} {[}September 1911{]}|pwv} in der »Rundſchau\pwindex{?? Werk@Nicht ermittelte Verfasserinnen und Verfasser!neue Rundschau1904@\emph{Die neue Rundschau} {[}1904{]}|pw}« und erwarte mit freudiger Ungeduld die Münchner\oindex{Muenchen@\textbf{München}|pw}{ }{\pb}\label{K_L02032-1v}\edtext{Erſtaufführung}{\lemma{\textnormal{\emph{Erſtaufführung}}}\Cendnote{\textnormal{Am 14. 10. 1911 fand die
                  Uraufführung in mehreren Städten gleichzeitig statt.}}}\label{K_L02032-1h} Ihres neuen Stückes\pwindex{Schnitzler, Arthur 15.05.1862 – 21.10.1931@\textsc{Schnitzler, Arthur} (15.05.1862 – 21.10.1931), \emph{Schriftsteller, Mediziner}!weite Land. Tragikomoedie in fuenf Akten1910-10-20@\strich\emph{Das weite Land. Tragikomödie in fünf Akten} {[}1910-10-20{]}|pwv}. Meinen Bruder\pwindex{Mann, Heinrich 27.03.1871 – 11.03.1950@\textsc{Mann, Heinrich} (27.03.1871 – 11.03.1950), \emph{Schriftsteller}|pwv}{ }ſehe ich ſchwer verſtimmt – und bin es mit ihm –
               über das Fehlſchlagen der Hoffnungen, die er auf sein Drama\pwindex{Mann, Heinrich 27.03.1871 – 11.03.1950@\textsc{Mann, Heinrich} (27.03.1871 – 11.03.1950), \emph{Schriftsteller}!Schauspielerin1911@\strich\emph{Schauspielerin} {[}1911{]}|pwv} geſetzt hatte. Ich habe es erſt jetzt hier in der
               Korrektur geleſen und muß zum Mindeſten die Energie bewundern, mit der ein an weit
               ausladender Breite gewöhnter Romancier\pwindex{Mann, Heinrich 27.03.1871 – 11.03.1950@\textsc{Mann, Heinrich} (27.03.1871 – 11.03.1950), \emph{Schriftsteller}|pwv}{ }ſo viel Leidenſchaft und Schickſal in ein paar
               knappe Dialoge zuſammenzupreſſen vermochte. Gewiß, die Theaterdirektoren thun {\pb}höchſt Unrecht, das Stück\pwindex{Mann, Heinrich 27.03.1871 – 11.03.1950@\textsc{Mann, Heinrich} (27.03.1871 – 11.03.1950), \emph{Schriftsteller}!Schauspielerin1911@\strich\emph{Schauspielerin} {[}1911{]}|pwv} zurückzuweiſen! Es mag ſein, daß die
               beiden ſpäteren Akte gegen den erſten an Bühnenwirkſamkeit zurückſtehen, aber
               dichteriſch genommen bringen ſie die eindringlichſten Dinge, und die ſchönſten
               Repliken ſind in ihnen enthalten. Und iſt es nicht ſchließlich ſo, daß eine
               dramatiſche Arbeit dieſes Autors\pwindex{Mann, Heinrich 27.03.1871 – 11.03.1950@\textsc{Mann, Heinrich} (27.03.1871 – 11.03.1950), \emph{Schriftsteller}|pwv} ohne Weiteres aufgeführt werden müßte? Wäre das nicht eine
               ſelbſtverſtändliche Aufmerkſamkeit des Theaters gegen den Dichter\pwindex{Mann, Heinrich 27.03.1871 – 11.03.1950@\textsc{Mann, Heinrich} (27.03.1871 – 11.03.1950), \emph{Schriftsteller}|pwv} der »Kleinen Stadt\pwindex{Mann, Heinrich 27.03.1871 – 11.03.1950@\textsc{Mann, Heinrich} (27.03.1871 – 11.03.1950), \emph{Schriftsteller}!kleine Stadt1909@\strich\emph{Die kleine Stadt} {[}1909{]}|pw}«? Entfällt da{\pb}bei für die Direktoren nicht jede
               künſtleriſche Verantwortung? Hoffentlich erkennt nun wenigſtens Frau Durieux\pwindex{Durieux, Tilla 18.08.1880 – 20.02.1971@\textsc{Durieux, Tilla} (18.08.1880 – 20.02.1971), \emph{Schauspielerin}|pw} in Berlin\oindex{Berlin@\textbf{Berlin}|pw} in der Leonie\pwindex{Mann, Heinrich 27.03.1871 – 11.03.1950@\textsc{Mann, Heinrich} (27.03.1871 – 11.03.1950), \emph{Schriftsteller}!Schauspielerin1911@\strich\emph{Schauspielerin} {[}1911{]}|pwv}
               eine gute Rolle.\pend
           \pstart
           Mit den beſten Empfehlungen an Sie und Ihre Gattin\pwindex{Schnitzler, Olga 17.01.1882 – 13.01.1970@\textsc{Schnitzler, Olga} (17.01.1882 – 13.01.1970), \emph{Schauspielerin, Sängerin}|pwv}, {\\}ſehr verehrter Herr Doctor,\pend
           \pstart
           Ihr ergebenſter{\\[\baselineskip]}\spacefill\mbox{Thomas Mann.}\pend
           \leftskip=0em{}
         
         \endnumbering\mylabel{h}\end{ledgroupsized}  \newcommand{\dateiname}{L02032}\newcommand{\titel}{Thomas Mann an Arthur Schnitzler, 25. 9. 1911}\newcommand{\editorInnen}{Martin Anton Müller und Gerd-Hermann Susen}%% latex-leseansicht-abspann.tex
%% Abspann für die Leseansicht.
%% Der Schalter \ifkorrekturansicht ist bereits durch den Vorspann gesetzt.

%% latex-abspann.tex
%% Gemeinsamer Abspann für Korrekturansicht und Leseansicht.
%% Setzt den Schalter \ifkorrekturansicht voraus (gesetzt in den
%% einbindenden Dateien latex-korrekturansicht-abspann.tex bzw.
%% latex-leseansicht-abspann.tex).
%% ---------------------------------------------------------------

\normalsize

% Das esempio-Environment wird nur in der Leseansicht benötigt
\ifkorrekturansicht\else
\newenvironment{esempio}[3]%
{
    \vspace{1.5ex}
    \rlap{\underline{#1}}
    \par
    \setlength{\parindent}{0cm}
    \nopagebreak
    \leftskip=#2cm
    \rightskip=#3cm
}
{
    \par
}
\fi

\doendnotes{C}
\bigskip
\vfill

\clearpage

\footnotesize

\ifkorrekturansicht
  \lohead{\textsc{register}}
\fi

% theindex-Environment neu definieren ohne reledmac
\makeatletter
\renewenvironment{theindex}{%
  \ifkorrekturansicht
    \section*{\indexname}%
  \else
    \subsubsection*{Index der erwähnten Entitäten}%
  \fi
  \setlength{\parindent}{0pt}%
  \setlength{\parskip}{0pt plus 0.3pt}%
  \let\item\@idxitem
}{%
  \ifkorrekturansicht\clearpage\fi
}
\makeatother

\IfFileExists{\jobname-pw.ind}{\input{\jobname-pw.ind}}{}

% Quellenangabe nur in der Leseansicht
\ifkorrekturansicht\else
% Fallback-Definitionen, falls die .tex-Datei \titel etc. nicht gesetzt hat
\providecommand{\titel}{}
\providecommand{\editorInnen}{}
\providecommand{\dateiname}{\jobname}

\vspace{3cm}

\vfill

\footnotesize
\textsc{Quelle}: \titel. Herausgegeben von {\editorInnen}. In: \emph{Arthur Schnitzler: Briefwechsel mit Autorinnen und Autoren}.
 Digitale Edition, https://schnitzler-briefe.acdh.oeaw.ac.at/{\dateiname}.html (Stand \today)
\fi

\end{document}


      