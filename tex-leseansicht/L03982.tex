%% latex-leseansicht-vorspann.tex
%% Vorspann für die Leseansicht.
%% Lädt die gemeinsame Datei latex-vorspann.tex mit nicht gesetztem Schalter.

\newif\ifkorrekturansicht
\korrekturansichtfalse

\input{../tex-inputs/latex-vorspann}


\section[Arthur Schnitzler an Berta Zuckerkandl, 31. 7. 1920]{L03982 Arthur Schnitzler an Berta Zuckerkandl, 31. 7. 1920}
\nopagebreak\mylabel{L03982v}
\rehead{ }\normalsize\beginnumbering\briefempfaengerindex{Zuckerkandl, Berta@\textsc{Zuckerkandl, Berta}!zzzSchnitzler, Arthur@\emph{von Arthur Schnitzler}!1920-07-311@{31. 7. 1920}|(be}
\toendnotes[C]{\smallbreak\pagebreak[2]}
\correspDesc{Versand  durch Arthur Schnitzler am 31. 7. 1920 in Wien
\newline{}Erhalt  durch Berta Zuckerkandl im Zeitraum [1. 8. 1920
                  – 5. 8. 1920?] in Kreuzberg [Breitenstein]}\toendnotes[C]{\smallbreak}
\Standort{Wien, Österreichische Nationalbibliothek, 405/B78/3 LIT MAG.}
\physDesc{Brief, 1 Blatt, 2 Seiten, 1362 Zeichen
\newline{}Handschrift: Bleistift, lateinische Kurrent}\toendnotes[C]{\smallbreak}
\pstart
           \raggedleft{}{\pb}Wien\oindex{Wien@\textbf{Wien}, \emph{Verwaltungsgebiet}|pw}, 31. 7. 20\pend
           
\pstart{}liebe und verehrte gnädige Frau,\pend\vspace{0.5em}
\pstart
           neulich war ich schon \label{K_L03982-1v}\edtext{ganz in Ihrer
                  Nähe}{\lemma{\textnormal{\emph{ganz in Ihrer
                  Nähe}}}\Cendnote{\textnormal{Berta Zuckerkandl\pwindex{Zuckerkandl, Berta 13.\,4.\,1864 Wien – 16.\,10.\,1945 Paris@\textsc{Zuckerkandl, Berta} (13.\,4.\,1864 Wien – 16.\,10.\,1945 Paris), \emph{Schriftstellerin, Journalistin, Übersetzerin}|pwk} hielt sich bei Alma Mahler\pwindex{Mahler-Werfel, Alma Maria 31.\,8.\,1879 Wien – 11.\,12.\,1964 New York City@\textsc{Mahler-Werfel, Alma Maria} (31.\,8.\,1879 Wien – 11.\,12.\,1964 New York City)|pwk} in Breitenstein am Semmering\oindex{Breitenstein am Semmering@\textbf{Breitenstein am Semmering}|pwk} auf. Schnitzler unternahm vom 26. 7. 1920 bis zum 28. 7. 1920 einen Ausflug nach Reichenau an der Rax\oindex{Reichenau an der Rax@\textbf{Reichenau an der Rax}, \emph{Verwaltungsgebiet}|pwk}.}}}\label{K_L03982-1} – und hab es vorgezogen, \label{K_L03982-2v}\edtext{aus Gründen}{\lemma{\textnormal{\emph{aus Gründen}}}\Cendnote{\textnormal{Im \emph{Tagebuch}\pwindex{Schnitzler, Arthur 15. 5. 1862 Wien – 21. 10. 1931 ebd.@\textsc{Schnitzler, Arthur} (15. 5. 1862 Wien – 21. 10. 1931 ebd.), \emph{Schriftsteller, Mediziner}!Tagebuch@\strich\emph{Tagebuch}|pwk} vermerkte
                     Schnitzler, dessen Ehe mit Olga Schnitzler\pwindex{Schnitzler, Olga 17.\,1.\,1882 Wien – 13.\,1.\,1970 Lugano@\textsc{Schnitzler, Olga} (17.\,1.\,1882 Wien – 13.\,1.\,1970 Lugano), \emph{Schauspielerin, Sängerin}|pwk} auf die Trennung zusteuerte:
                     »Nm. fuhr ich nach Reichenau\oindex{Reichenau an der Rax@\textbf{Reichenau an der Rax}, \emph{Verwaltungsgebiet}|pw}; vorher
                     qualvolle Scene; wegen Kreuzberg\oindex{Kreuzberg [Breitenstein]@\textbf{Kreuzberg [Breitenstein]}, \emph{Berg}|pw}«, siehe A. S.: \emph{Tagebuch}, 26. 7. 1920. Am
                     Kreuzberg\oindex{Kreuzberg [Breitenstein]@\textbf{Kreuzberg [Breitenstein]}, \emph{Berg}|pwk}, der zu Breitenstein am Semmering\oindex{Breitenstein am Semmering@\textbf{Breitenstein am Semmering}|pwk} gehört, lag das Haus\oindex{Haus Mahler@\textbf{Haus Mahler}, \emph{Gebäude}|pwkv} von Alma Mahler\pwindex{Mahler-Werfel, Alma Maria 31.\,8.\,1879 Wien – 11.\,12.\,1964 New York City@\textsc{Mahler-Werfel, Alma Maria} (31.\,8.\,1879 Wien – 11.\,12.\,1964 New York City)|pwk}, in dem sie sich mit ihrem Partner Franz Werfel\pwindex{Werfel, Franz 10.\,9.\,1890 Prag – 26.\,8.\,1945 Beverly Hills@\textsc{Werfel, Franz} (10.\,9.\,1890 Prag – 26.\,8.\,1945 Beverly Hills), \emph{Schriftsteller}|pwk} aufhielt.}}}\label{K_L03982-2}, die ich bei
               nächster mündlicher Gelegenheit mittheilen werde, statt auf den Kreuzberg\oindex{Kreuzberg [Breitenstein]@\textbf{Kreuzberg [Breitenstein]}, \emph{Berg}|pw}\pwindex{Mahler-Werfel, Alma Maria 31.\,8.\,1879 Wien – 11.\,12.\,1964 New York City@\textsc{Mahler-Werfel, Alma Maria} (31.\,8.\,1879 Wien – 11.\,12.\,1964 New York City)|pwv}\pwindex{Werfel, Franz 10.\,9.\,1890 Prag – 26.\,8.\,1945 Beverly Hills@\textsc{Werfel, Franz} (10.\,9.\,1890 Prag – 26.\,8.\,1945 Beverly Hills), \emph{Schriftsteller}|pwv} auf die Rax\oindex{Rax@\textbf{Rax}, \emph{Berg}|pw} zu gehen. Die Escapade endete
               – \label{K_L03982-3v}\edtext{Nebel u Regen}{\lemma{\textnormal{\emph{Nebel u Regen}}}\Cendnote{\textnormal{Vgl. A. S.: \emph{Tagebuch}, 28. 7. 1920.}}}\label{K_L03982-3}. Sollte
               das Wetter sich ändern, so wäre es wohl möglich, daß ich im Lauf der nächsten Woche
               bei Ihnen erscheine – we{\geminationn} nicht, darf ich Sie
               hoffentlich gleich nach Ihrem Wiedereintreffen in Wien\oindex{Wien@\textbf{Wien}, \emph{Verwaltungsgebiet}|pw} sehen. Ich hätte ein rechtes Bedürfnis mit Ihnen zu reden – schriftlich
               lassen sich all die Dinge in ihren Auf- u abſchüben, all dies unfassbare, und das
               fassbare fast noch weniger – kaum behandeln. Vor 10. fährt Olga\pwindex{Schnitzler, Olga 17.\,1.\,1882 Wien – 13.\,1.\,1970 Lugano@\textsc{Schnitzler, Olga} (17.\,1.\,1882 Wien – 13.\,1.\,1970 Lugano), \emph{Schauspielerin, Sängerin}|pw} nach Gastein\oindex{Bad Gastein@\textbf{Bad Gastein}, \emph{Hauptstadt}|pw}, Lili\pwindex{Cappellini, Lili 13.\,9.\,1909 Wien – 26.\,7.\,1928 Venedig@\textsc{Cappellini, Lili} (13.\,9.\,1909 Wien – 26.\,7.\,1928 Venedig)|pw} fährt mit, – aber nicht
               bis Gastein\oindex{Bad Gastein@\textbf{Bad Gastein}, \emph{Hauptstadt}|pw}, sondern nach Attnang\oindex{Attnang-Puchheim@\textbf{Attnang-Puchheim}, \emph{Hauptstadt}|pw} od. Salzburg\oindex{Salzburg@\textbf{Salzburg}, \emph{Verwaltungsgebiet}|pw}, wo
               sie von Frl Pollak\pwindex{Pollak, Frieda 8.\,12.\,1881 Wien – 13.\,7.\,1937 ebd.@\textsc{Pollak, Frieda} (8.\,12.\,1881 Wien – 13.\,7.\,1937 ebd.), \emph{Sekretärin}|pw} in Empfang geno{\geminationm}en wird. {\pb}um einer Einladg meiner Schwester\pwindex{Hajek, Gisela 20.\,12.\,1867 Wien – 3.\,2.\,1953 Cambridge@\textsc{Hajek, Gisela} (20.\,12.\,1867 Wien – 3.\,2.\,1953 Cambridge)|pwv} nach Altaussee\oindex{Altaussee@\textbf{Altaussee}, \emph{Verwaltungsgebiet}|pw} zu folgen.
               Ich selbst will da{\geminationn}\label{K_L03982-4v}\edtext{zwischen 20. u
               25. August}{\lemma{\textnormal{\emph{zwischen … August}}}\Cendnote{\textnormal{Schnitzler verließ Wien am 24. 8. 1920 und erreichte Aussee, wo er bis zum 14. 9. 1920 blieb, am 27. 8. 1920.}}}\label{K_L03982-4} auch hin; Heini\pwindex{Schnitzler, Heinrich 9.\,8.\,1902 Hinterbrühl – 12.\,7.\,1982 Wien@\textsc{Schnitzler, Heinrich} (9.\,8.\,1902 Hinterbrühl – 12.\,7.\,1982 Wien), \emph{Regisseur, Schauspieler}|pw} der
               am 17. etwa nach München\oindex{München@\textbf{München}|pw} fährt,
               trifft Anfg. September in Altaussee\oindex{Altaussee@\textbf{Altaussee}, \emph{Verwaltungsgebiet}|pw} mit mir zusa{\geminationm}en, und, nach dem
               augenblicklichen Stand der Angelegenheiten, wäre es nicht undenkbar, daß Olga\pwindex{Schnitzler, Olga 17.\,1.\,1882 Wien – 13.\,1.\,1970 Lugano@\textsc{Schnitzler, Olga} (17.\,1.\,1882 Wien – 13.\,1.\,1970 Lugano), \emph{Schauspielerin, Sängerin}|pw} von Gastein\oindex{Bad Gastein@\textbf{Bad Gastein}, \emph{Hauptstadt}|pw} aus auch nach Altaussee\oindex{Altaussee@\textbf{Altaussee}, \emph{Verwaltungsgebiet}|pw}
               käme, – ein veritabler Familientag wie Sie sehen, in Conjunctivo vorläufig. –\pend
           
\pstart
           Für Ihren lieben Brief, und für Ihre freundschaflichen Gefühle überhaupt bin ich
               Ihnen dankbarer als ich Ihnen sagen kann! Alles herzlich an Frau Alma\pwindex{Mahler-Werfel, Alma Maria 31.\,8.\,1879 Wien – 11.\,12.\,1964 New York City@\textsc{Mahler-Werfel, Alma Maria} (31.\,8.\,1879 Wien – 11.\,12.\,1964 New York City)|pw} und Werfel\pwindex{Werfel, Franz 10.\,9.\,1890 Prag – 26.\,8.\,1945 Beverly Hills@\textsc{Werfel, Franz} (10.\,9.\,1890 Prag – 26.\,8.\,1945 Beverly Hills), \emph{Schriftsteller}|pw}.\pend
           
\pstart
           Ihr getreuer{\\[\baselineskip]}\spacefill\mbox{ArthSch}\pend
           \leftskip=0em{}\selectlanguage{ngerman}\endnumbering\briefempfaengerindex{Zuckerkandl, Berta@\textsc{Zuckerkandl, Berta}!zzzSchnitzler, Arthur@\emph{von Arthur Schnitzler}!1920-07-311@{31. 7. 1920}|)be}\mylabel{L03982h}
\begin{anhang}
\end{anhang}\newcommand{\dateiname}{L03982}\newcommand{\titel}{Arthur Schnitzler an Berta Zuckerkandl, 31. 7. 1920}\newcommand{\editorInnen}{Herausgegeben von Jahnke, SelmaMüller, Martin Anton}%% latex-leseansicht-abspann.tex
%% Abspann für die Leseansicht.
%% Der Schalter \ifkorrekturansicht ist bereits durch den Vorspann gesetzt.

%% latex-abspann.tex
%% Gemeinsamer Abspann für Korrekturansicht und Leseansicht.
%% Setzt den Schalter \ifkorrekturansicht voraus (gesetzt in den
%% einbindenden Dateien latex-korrekturansicht-abspann.tex bzw.
%% latex-leseansicht-abspann.tex).
%% ---------------------------------------------------------------

\normalsize

% Das esempio-Environment wird nur in der Leseansicht benötigt
\ifkorrekturansicht\else
\newenvironment{esempio}[3]%
{
    \vspace{1.5ex}
    \rlap{\underline{#1}}
    \par
    \setlength{\parindent}{0cm}
    \nopagebreak
    \leftskip=#2cm
    \rightskip=#3cm
}
{
    \par
}
\fi

\doendnotes{C}
\bigskip
\vfill

\clearpage

\footnotesize

\ifkorrekturansicht
  \lohead{\textsc{register}}
\fi

% theindex-Environment neu definieren ohne reledmac
\makeatletter
\renewenvironment{theindex}{%
  \ifkorrekturansicht
    \section*{\indexname}%
  \else
    \subsubsection*{Index der erwähnten Entitäten}%
  \fi
  \setlength{\parindent}{0pt}%
  \setlength{\parskip}{0pt plus 0.3pt}%
  \let\item\@idxitem
}{%
  \ifkorrekturansicht\clearpage\fi
}
\makeatother

\IfFileExists{\jobname-pw.ind}{\input{\jobname-pw.ind}}{}

% Quellenangabe nur in der Leseansicht
\ifkorrekturansicht\else
% Fallback-Definitionen, falls die .tex-Datei \titel etc. nicht gesetzt hat
\providecommand{\titel}{}
\providecommand{\editorInnen}{}
\providecommand{\dateiname}{\jobname}

\vspace{3cm}

\vfill

\footnotesize
\textsc{Quelle}: \titel. Herausgegeben von {\editorInnen}. In: \emph{Arthur Schnitzler: Briefwechsel mit Autorinnen und Autoren}.
 Digitale Edition, https://schnitzler-briefe.acdh.oeaw.ac.at/{\dateiname}.html (Stand \today)
\fi

\end{document}


