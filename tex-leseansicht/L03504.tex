%% latex-leseansicht-vorspann.tex
%% Vorspann für die Leseansicht.
%% Lädt die gemeinsame Datei latex-vorspann.tex mit nicht gesetztem Schalter.

\newif\ifkorrekturansicht
\korrekturansichtfalse

\input{../tex-inputs/latex-vorspann}


\section[Felix Salten u. a. an Arthur Schnitzler, {[}zwischen 19. und 30. 7.? 1909{]}]{L03504 Felix Salten u. a. an Arthur Schnitzler, {[}zwischen 19. und 30. 7.? 1909{]}}
\nopagebreak\mylabel{L03504v}
\rehead{ }\normalsize\beginnumbering\briefempfaengerindex{Schnitzler, Arthur@\textsc{Schnitzler, Arthur}!zzzFischer, Hedwig@\emph{von Hedwig Fischer}!1909-07-301@{{[}zwischen 19. und 30. 7.? 1909{]}}|(be}\briefempfaengerindex{Schnitzler, Arthur@\textsc{Schnitzler, Arthur}!zzzFischer, Samuel@\emph{von Samuel Fischer}!1909-07-301@{{[}zwischen 19. und 30. 7.? 1909{]}}|(be}\briefempfaengerindex{Schnitzler, Arthur@\textsc{Schnitzler, Arthur}!zzzWassermann, Jakob@\emph{von Jakob Wassermann}!1909-07-301@{{[}zwischen 19. und 30. 7.? 1909{]}}|(be}\briefempfaengerindex{Schnitzler, Arthur@\textsc{Schnitzler, Arthur}!zzzSalten, Ottilie@\emph{von Ottilie Salten}!1909-07-301@{{[}zwischen 19. und 30. 7.? 1909{]}}|(be}\briefempfaengerindex{Schnitzler, Arthur@\textsc{Schnitzler, Arthur}!zzzSalten, Felix@\emph{von Felix Salten}!1909-07-301@{{[}zwischen 19. und 30. 7.? 1909{]}}|(be}
\toendnotes[C]{\smallbreak\pagebreak[2]}
\correspDesc{Versand  durch Felix Salten, Ottilie Salten, Jakob Wassermann, Samuel Fischer, Hedwig Fischer im Zeitraum [zwischen 19. und 30. 7.? 1909] in Prato Piazza
\newline{}Übermittlung  in Landro
\newline{}Erhalt  durch Arthur Schnitzler im Zeitraum [zwischen 20. und 31. 7.? 1909] in Edlach}\toendnotes[C]{\smallbreak}
\Standort{CUL, Schnitzler, B 89, B 1.}
\physDesc{Bildpostkarte, 396 Zeichen
\newline{}Handschrift Felix Salten: Bleistift, lateinische Kurrent
\newline{}Handschrift Ottilie Salten: Bleistift, deutsche Kurrent
\newline{}Handschrift Hedwig Fischer: Bleistift, deutsche Kurrent
\newline{}Handschrift Jakob Wassermann: Bleistift, lateinische Kurrent
\newline{}Handschrift Samuel Fischer: Bleistift, lateinische Kurrent
\newline{}Versand: 1) Stempel: »\nobreak{}\oindex{Hôtel Dürrenstein@\textbf{Hôtel Dürrenstein}, \emph{Hotel}|pwk}Hôtel Dürrenstein, 2000 M. Plätzwiese\oindex{Prato Piazza@\textbf{Prato Piazza}, \emph{Berg}|pw} 2000 M.Alois Pahler\pwindex{Pahler, Alois @\textsc{Pahler, Alois}, \emph{Hotelbesitzer}|pw}\nobreak{}«.   2) Stempel: »\nobreak{}\oindex{Höhlenstein@\textbf{Höhlenstein}|pwk}{[}L{]}andro, 8\nobreak{}«. 
\newline{}Schnitzler: mit Bleistift Vermerk: »\textsc{Salten}« 
\newline{}Ordnung: mit Bleistift von unbekannter Hand nummeriert: »254« }\toendnotes[C]{\smallbreak}\pstart{}{\pb}Herrn\pend{}\pstart{}D\textsuperscript{r} Arthur Schnitzler\pend{}\pstart{}Edlach \textsuperscript{b}/Reichenau\oindex{Edlach@\textbf{Edlach}|pw}\pend{}\pstart{}Südbahn\orgindex{Südbahnstrecke@Südbahnstrecke|pw}\pend{}\pstart{}Nied\textcolor{gray}{.} Öst\oindex{Niederösterreich@\textbf{Niederösterreich}, \emph{Land}|pw}\pend{}{\bigskip}
\pstart
           {\pb}\textcolor{gray}{\textbf{Plätzwiesen\oindex{Prato Piazza@\textbf{Prato Piazza}, \emph{Berg}|pw} (2003 m) mit Hoher Gaisl\oindex{Croda Rossa@\textbf{Croda Rossa}, \emph{Berg}|pw} (3148 m).}}\hfill \textcolor{gray}{\textbf{Tirol\oindex{Südtirol@\textbf{Südtirol}, \emph{Verwaltungsgebiet}|pw}.}}\pend
           \vspace{1em}
\pstart
           \noindent{}{\pb}Schöner Weg – schönes Ausruhen und herzliches Gedenken der Entfernten. Hoffentlich
               geht es Ihrer \label{K_L03504-1v}\edtext{Frau\pwindex{Schnitzler, Olga 17.\,1.\,1882 Wien – 13.\,1.\,1970 Lugano@\textsc{Schnitzler, Olga} (17.\,1.\,1882 Wien – 13.\,1.\,1970 Lugano), \emph{Schauspielerin, Sängerin}|pwv} dauernd gut u. Heini\pwindex{Schnitzler, Heinrich 9.\,8.\,1902 Hinterbrühl – 12.\,7.\,1982 Wien@\textsc{Schnitzler, Heinrich} (9.\,8.\,1902 Hinterbrühl – 12.\,7.\,1982 Wien), \emph{Regisseur, Schauspieler}|pw} ist ganz gesund}{\lemma{\textnormal{\emph{Frau … gesund}}}\Cendnote{\textnormal{Die Karte ist undatiert und lässt sich nur anhand
                  einiger Indizien einem Zeitraum zuordnen: Olga
                     Schnitzler\pwindex{Schnitzler, Olga 17.\,1.\,1882 Wien – 13.\,1.\,1970 Lugano@\textsc{Schnitzler, Olga} (17.\,1.\,1882 Wien – 13.\,1.\,1970 Lugano), \emph{Schauspielerin, Sängerin}|pwk} war schwanger und hatte zeitweise Beschwerden, vgl. A. S.: \emph{Tagebuch}, 26. 6. 1909. Heinrichs\pwindex{Schnitzler, Heinrich 9.\,8.\,1902 Hinterbrühl – 12.\,7.\,1982 Wien@\textsc{Schnitzler, Heinrich} (9.\,8.\,1902 Hinterbrühl – 12.\,7.\,1982 Wien), \emph{Regisseur, Schauspieler}|pwk} Keuchhusten heilte Anfang Juli 1909 aus. Samuel Fischer\pwindex{Fischer, Samuel 24.\,12.\,1859 Liptovský Mikuláš – 15.\,10.\,1934 Berlin@\textsc{Fischer, Samuel} (24.\,12.\,1859 Liptovský Mikuláš – 15.\,10.\,1934 Berlin), \emph{Verleger}|pwk} schrieb am 20. 7. 1909 aus Landro\oindex{Höhlenstein@\textbf{Höhlenstein}|pwk} an Schnitzler (vgl. 
                           Samuel Fischer\pwindex{Fischer, Samuel 24.\,12.\,1859 Liptovský Mikuláš – 15.\,10.\,1934 Berlin@\textsc{Fischer, Samuel} (24.\,12.\,1859 Liptovský Mikuláš – 15.\,10.\,1934 Berlin), \emph{Verleger}|pwk}, Hedwig Fischer\pwindex{Fischer, Hedwig 8.\,9.\,1871 Szczecin – 11.\,4.\,1952 Königstein im Taunus@\textsc{Fischer, Hedwig} (8.\,9.\,1871 Szczecin – 11.\,4.\,1952 Königstein im Taunus)|pwk}: \emph{Briefwechsel mit
                              Autoren}. Herausgegeben von Dierk Rodewald und Corinna Fiedler. Mit
                           einer Einführung von Bernhard Zeller. Frankfurt am Main:
                           \emph{S. Fischer}{ }1989, S. 84). Da auf der Karte Saltens\pwindex{Salten, Felix 6.\,9.\,1869 Budapest – 8.\,10.\,1945 Zürich@\textsc{Salten, Felix} (6.\,9.\,1869 Budapest – 8.\,10.\,1945 Zürich), \emph{Schriftsteller, Journalist, Chefredakteur}|pwk} vom XXXX Auszeichnungsfehler: Dokument L03503 nicht gefunden die Anwesenheit Fischers\pwindex{Fischer, Samuel 24.\,12.\,1859 Liptovský Mikuláš – 15.\,10.\,1934 Berlin@\textsc{Fischer, Samuel} (24.\,12.\,1859 Liptovský Mikuláš – 15.\,10.\,1934 Berlin), \emph{Verleger}|pwk} nicht erwähnt wird und Heinrichs\pwindex{Schnitzler, Heinrich 9.\,8.\,1902 Hinterbrühl – 12.\,7.\,1982 Wien@\textsc{Schnitzler, Heinrich} (9.\,8.\,1902 Hinterbrühl – 12.\,7.\,1982 Wien), \emph{Regisseur, Schauspieler}|pwk} Keuchhusten erst »besser« geworden ist, dürfte die vorliegende Karte
                  danach abgefasst worden sein – und vor dem Monatsende, da auf der Karte vom XXXX Auszeichnungsfehler: Dokument L03505 nicht gefunden nicht mehr nach
                  dem Befinden Heinrichs\pwindex{Schnitzler, Heinrich 9.\,8.\,1902 Hinterbrühl – 12.\,7.\,1982 Wien@\textsc{Schnitzler, Heinrich} (9.\,8.\,1902 Hinterbrühl – 12.\,7.\,1982 Wien), \emph{Regisseur, Schauspieler}|pwk} gefragt
               wird.}}}\label{K_L03504-1}. Alles Herzliche von uns zu Ihnen\pend
           \pstart Ihr \spacefill\mbox{Salten}\pend{}\selectlanguage{ngerman}\vspace{1em}
\pstart
           \noindent{}{[}hs. Salten:{]} Viele{ }ſchöne Grüße {\\}\spacefill\mbox{Otti}\pend
           \selectlanguage{ngerman}\vspace{1em}
\pstart
           \noindent{}{[}hs. Fischer:{]} herzliche Grüße und viele gute Wünsche für Frau \textsc{Schnitzler}\pwindex{Schnitzler, Olga 17.\,1.\,1882 Wien – 13.\,1.\,1970 Lugano@\textsc{Schnitzler, Olga} (17.\,1.\,1882 Wien – 13.\,1.\,1970 Lugano), \emph{Schauspielerin, Sängerin}|pw} u. \textsc{Heini}\pwindex{Schnitzler, Heinrich 9.\,8.\,1902 Hinterbrühl – 12.\,7.\,1982 Wien@\textsc{Schnitzler, Heinrich} (9.\,8.\,1902 Hinterbrühl – 12.\,7.\,1982 Wien), \emph{Regisseur, Schauspieler}|pw}{ }{\\}\spacefill\mbox{Hedwig Fischer.}\pend
           \selectlanguage{ngerman}\vspace{1em}
\pstart
           \noindent{}{[}hs. Wassermann:{]} Herzlich grüsst Ihr \spacefill\mbox{JakobWassermann}\pend
           \selectlanguage{ngerman}\vspace{1em}
\pstart
           \noindent{}{[}hs. Fischer:{]} Herzliche Grüße Ihr \spacefill\mbox{SFischer}\pend
           \selectlanguage{ngerman}\endnumbering\briefempfaengerindex{Schnitzler, Arthur@\textsc{Schnitzler, Arthur}!zzzFischer, Hedwig@\emph{von Hedwig Fischer}!1909-07-191@{{[}zwischen 19. und 30. 7.? 1909{]}}|)be}\briefempfaengerindex{Schnitzler, Arthur@\textsc{Schnitzler, Arthur}!zzzFischer, Samuel@\emph{von Samuel Fischer}!1909-07-191@{{[}zwischen 19. und 30. 7.? 1909{]}}|)be}\briefempfaengerindex{Schnitzler, Arthur@\textsc{Schnitzler, Arthur}!zzzWassermann, Jakob@\emph{von Jakob Wassermann}!1909-07-191@{{[}zwischen 19. und 30. 7.? 1909{]}}|)be}\briefempfaengerindex{Schnitzler, Arthur@\textsc{Schnitzler, Arthur}!zzzSalten, Ottilie@\emph{von Ottilie Salten}!1909-07-191@{{[}zwischen 19. und 30. 7.? 1909{]}}|)be}\briefempfaengerindex{Schnitzler, Arthur@\textsc{Schnitzler, Arthur}!zzzSalten, Felix@\emph{von Felix Salten}!1909-07-191@{{[}zwischen 19. und 30. 7.? 1909{]}}|)be}\mylabel{L03504h}  \newcommand{\dateiname}{L03504}\newcommand{\titel}{Felix Salten u. a. an Arthur Schnitzler, [zwischen 19. und 30. 7.? 1909]}\newcommand{\editorInnen}{Martin Anton Müller und Laura Untner}%% latex-leseansicht-abspann.tex
%% Abspann für die Leseansicht.
%% Der Schalter \ifkorrekturansicht ist bereits durch den Vorspann gesetzt.

%% latex-abspann.tex
%% Gemeinsamer Abspann für Korrekturansicht und Leseansicht.
%% Setzt den Schalter \ifkorrekturansicht voraus (gesetzt in den
%% einbindenden Dateien latex-korrekturansicht-abspann.tex bzw.
%% latex-leseansicht-abspann.tex).
%% ---------------------------------------------------------------

\normalsize

% Das esempio-Environment wird nur in der Leseansicht benötigt
\ifkorrekturansicht\else
\newenvironment{esempio}[3]%
{
    \vspace{1.5ex}
    \rlap{\underline{#1}}
    \par
    \setlength{\parindent}{0cm}
    \nopagebreak
    \leftskip=#2cm
    \rightskip=#3cm
}
{
    \par
}
\fi

\doendnotes{C}
\bigskip
\vfill

\clearpage

\footnotesize

\ifkorrekturansicht
  \lohead{\textsc{register}}
\fi

% theindex-Environment neu definieren ohne reledmac
\makeatletter
\renewenvironment{theindex}{%
  \ifkorrekturansicht
    \section*{\indexname}%
  \else
    \subsubsection*{Index der erwähnten Entitäten}%
  \fi
  \setlength{\parindent}{0pt}%
  \setlength{\parskip}{0pt plus 0.3pt}%
  \let\item\@idxitem
}{%
  \ifkorrekturansicht\clearpage\fi
}
\makeatother

\IfFileExists{\jobname-pw.ind}{\input{\jobname-pw.ind}}{}

% Quellenangabe nur in der Leseansicht
\ifkorrekturansicht\else
% Fallback-Definitionen, falls die .tex-Datei \titel etc. nicht gesetzt hat
\providecommand{\titel}{}
\providecommand{\editorInnen}{}
\providecommand{\dateiname}{\jobname}

\vspace{3cm}

\vfill

\footnotesize
\textsc{Quelle}: \titel. Herausgegeben von {\editorInnen}. In: \emph{Arthur Schnitzler: Briefwechsel mit Autorinnen und Autoren}.
 Digitale Edition, https://schnitzler-briefe.acdh.oeaw.ac.at/{\dateiname}.html (Stand \today)
\fi

\end{document}


