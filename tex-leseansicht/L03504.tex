%% latex-leseansicht-vorspann.tex
%% Vorspann für die Leseansicht.
%% Lädt die gemeinsame Datei latex-vorspann.tex mit nicht gesetztem Schalter.

\newif\ifkorrekturansicht
\korrekturansichtfalse

\input{../tex-inputs/latex-vorspann}

\begin{center}
            \textcolor{red}{ENTWURF, NICHT FERTIG KORRIGIERT}
                      \end{center}
            
         
         \renewcommand{\erwaehntePersonen}{Personen: Alois Pahler, Olga Schnitzler, Heinrich Schnitzler}
         \renewcommand{\erwaehnteInstitutionen}{Institutionen: Südbahnstrecke}
         \renewcommand{\erwaehnteOrte}{Orte: Croda Rossa, Edlach, Hôtel Dürrenstein, Höhlenstein, Niederösterreich, Prato Piazza, Südtirol}
         \renewcommand{\erwaehnteWerke}{}
               \section[Felix Salten u. a. an Arthur Schnitzler, {[}zwischen 19. und 30. 7.? 1909{]}]{ Felix Salten u. a. an Arthur Schnitzler, {[}zwischen 19. und
               30. 7.? 1909{]}}\nopagebreak\mylabel{v}\rehead{ }\begin{ledgroupsized}[t]{13cm}\normalsize\beginnumbering \toendnotes[C]{\smallbreak\pagebreak[2]} \Standort{CUL, Schnitzler, B 89, B 1.}
\physDesc{Bildpostkarte, 448 Zeichen
\newline{}Handschrift Felix Salten: Bleistift, lateinische Kurrent\newline{}Handschrift Ottilie Salten: Bleistift, deutsche Kurrent\newline{}Handschrift Hedwig Fischer: Bleistift, deutsche Kurrent\newline{}Handschrift Jakob Wassermann: Bleistift, lateinische Kurrent\newline{}Handschrift Samuel Fischer: Bleistift, lateinische Kurrent
\newline{}Versand: Stempel: »\nobreak{}\oindex{Hoehlenstein@\textbf{Höhlenstein}|pwk}\textcolor{gray}{L}andro, 8\nobreak{}«.  
\newline{}Schnitzler: mit Bleistift Vermerk: »\textsc{Salten}« 
\newline{}Ordnung: mit Bleistift von unbekannter Hand nummeriert:
                                    »254« }\toendnotes[C]{\smallbreak}\pstart{}{\pb}Herrn\pend{}\pstart{}D\textsuperscript{r} Arthur Schnitzler\pend{}\pstart{}Edlach \textcolor{gray}{b/} Reichenau\oindex{Edlach@\textbf{Edlach}|pw}\pend{}\pstart{}Südbahn\orgindex{Suedbahnstrecke@Südbahnstrecke|pw}\pend{}\pstart{}Nied\textcolor{gray}{.} Öst\textcolor{gray}{.}\oindex{Niederoesterreich@\textbf{Niederösterreich}|pw}\pend{}{\bigskip}\pstart
           \noindent{}\centering{}{\pb}\textcolor{gray}{\textbf{Plätzwiesen\oindex{Prato Piazza@\textbf{Prato Piazza}|pw} (2003 m) mit Hoher Gaisl\oindex{Croda Rossa@\textbf{Croda Rossa}|pw} (3148 m). Tirol\oindex{Suedtirol@\textbf{Südtirol}|pw}.}}\pend
           \pstart
           \noindent{}\centering{}\textcolor{gray}{\textbf{\textit{Hôtel Dürrenstein\oindex{Hôtel Duerrenstein@\textbf{Hôtel Dürrenstein}|pw}}}}\pend
           \pstart
           \noindent{}\centering{}\textcolor{gray}{\textbf{\textit{2000 M. PLÄTZWIESE\oindex{Prato Piazza@\textbf{Prato Piazza}|pw}
                        2000 M.}}}\pend
           \pstart
           \noindent{}\centering{}\textcolor{gray}{\textbf{\textit{Alois Pahler\pwindex{Pahler, Alois @\textsc{Pahler, Alois}, \emph{Hotelbesitzer}|pw}}}}\pend
           \pstart
           Schöner Weg – schönes Ausruhen und herzliches Gedenken der Entfernten. Hoffentlich
               geht es Ihrer \label{K_L03504-1v}\edtext{Frau\pwindex{Schnitzler, Olga 17.01.1882 – 13.01.1970@\textsc{Schnitzler, Olga} (17.01.1882 – 13.01.1970), \emph{Schauspielerin, Sängerin}|pwv} dauernd gut u. Heini\pwindex{Schnitzler, Heinrich 09.08.1902 – 12.07.1982@\textsc{Schnitzler, Heinrich} (09.08.1902 – 12.07.1982), \emph{Regisseur, Schauspieler}|pw} ist ganz gesund}{\lemma{\textnormal{\emph{Frau … gesund}}}\Cendnote{\textnormal{Die Karte ist undatiert und lässt sich zeitlich nur Anhand
                  verschiedener Indizien einem Zeitraum zuordnen. Olga Schnitzler\pwindex{Schnitzler, Olga 17.01.1882 – 13.01.1970@\textsc{Schnitzler, Olga} (17.01.1882 – 13.01.1970), \emph{Schauspielerin, Sängerin}|pwk} war schwanger und hatte zeitweise Beschwerden, vgl. A. S.: \emph{Tagebuch}, 26. 6. 1909. Heinrichs\pwindex{Schnitzler, Heinrich 09.08.1902 – 12.07.1982@\textsc{Schnitzler, Heinrich} (09.08.1902 – 12.07.1982), \emph{Regisseur, Schauspieler}|pwk} Keuchhusten wiederum heilte
                     Anfang Juli 1909 aus. S. Fischer\pwindex{Fischer, Samuel 24.12.1859 – 15.10.1934@\textsc{Fischer, Samuel} (24.12.1859 – 15.10.1934), \emph{Verleger}|pwk} schrieb am 20. 7. 1909 aus Landro\oindex{Hoehlenstein@\textbf{Höhlenstein}|pwk} an Schnitzler\pwindex{Schnitzler, Arthur 15.05.1862 – 21.10.1931@\textsc{Schnitzler, Arthur} (15.05.1862 – 21.10.1931), \emph{Schriftsteller, Mediziner}|pwk} (\emph{Briefwechsel mit Autoren}, S. 84). Nachdem in
                  der Karte Salten\pwindex{Salten, Felix 06.09.1869 – 08.10.1945@\textsc{Salten, Felix} (06.09.1869 – 08.10.1945), \emph{Schriftsteller, Journalist}|pwk}s vom [zwischen 19. und
                  30. 7.? 1909] die Anwesenheit
                     Fischers\pwindex{Fischer, Samuel 24.12.1859 – 15.10.1934@\textsc{Fischer, Samuel} (24.12.1859 – 15.10.1934), \emph{Verleger}|pwk} nicht erwähnt, dürfte die
                  vorliegende Karte danach abgefasst sein – und vor dem Monatsende, da in der Karte
                  vom 31. 7. 1909 nicht mehr
                  nach dem Befinden Heinrichs\pwindex{Schnitzler, Heinrich 09.08.1902 – 12.07.1982@\textsc{Schnitzler, Heinrich} (09.08.1902 – 12.07.1982), \emph{Regisseur, Schauspieler}|pwk} gefragt
                  wird.}}}\label{K_L03504-1h}. Alles herzliche von uns zu Ihnen\pend
           \pstart Ihr \spacefill\mbox{Salten}\pend{}\pstart
           \noindent{}{[}hs. Ottilie Salten:{]} Viele ſchöne Grüße\pend
           \pstart \spacefill\mbox{Otti}\pend{}\pstart
           \noindent{}{[}hs. Hedwig Fischer:{]} herzliche Grüße und viele gute Wünsche für Frau Schnitzler\pwindex{Schnitzler, Olga 17.01.1882 – 13.01.1970@\textsc{Schnitzler, Olga} (17.01.1882 – 13.01.1970), \emph{Schauspielerin, Sängerin}|pw} u. Heini\pwindex{Schnitzler, Heinrich 09.08.1902 – 12.07.1982@\textsc{Schnitzler, Heinrich} (09.08.1902 – 12.07.1982), \emph{Regisseur, Schauspieler}|pw}\pend
           \pstart \spacefill\mbox{Hedwig Fischer.}\pend{}\pstart
           \noindent{}{[}hs. Wassermann:{]} Herzlich grüsst\pend
           \pstart Ihr \spacefill\mbox{JakobWassermann}\pend{}\pstart
           \noindent{}{[}hs. Samuel Fischer:{]} Herzliche Grüße Ihr\pend
           \pstart \spacefill\mbox{SFischer}\pend{}
         
         \endnumbering\mylabel{h}\end{ledgroupsized}\begin{anhang}\end{anhang}\newcommand{\dateiname}{L03504}\newcommand{\titel}{Felix Salten u. a. an Arthur Schnitzler, [zwischen 19. und 30. 7.? 1909]}\newcommand{\editorInnen}{Martin Anton Müller und Laura Untner}%% latex-leseansicht-abspann.tex
%% Abspann für die Leseansicht.
%% Der Schalter \ifkorrekturansicht ist bereits durch den Vorspann gesetzt.

%% latex-abspann.tex
%% Gemeinsamer Abspann für Korrekturansicht und Leseansicht.
%% Setzt den Schalter \ifkorrekturansicht voraus (gesetzt in den
%% einbindenden Dateien latex-korrekturansicht-abspann.tex bzw.
%% latex-leseansicht-abspann.tex).
%% ---------------------------------------------------------------

\normalsize

% Das esempio-Environment wird nur in der Leseansicht benötigt
\ifkorrekturansicht\else
\newenvironment{esempio}[3]%
{
    \vspace{1.5ex}
    \rlap{\underline{#1}}
    \par
    \setlength{\parindent}{0cm}
    \nopagebreak
    \leftskip=#2cm
    \rightskip=#3cm
}
{
    \par
}
\fi

\doendnotes{C}
\bigskip
\vfill

\clearpage

\footnotesize

\ifkorrekturansicht
  \lohead{\textsc{register}}
\fi

% theindex-Environment neu definieren ohne reledmac
\makeatletter
\renewenvironment{theindex}{%
  \ifkorrekturansicht
    \section*{\indexname}%
  \else
    \subsubsection*{Index der erwähnten Entitäten}%
  \fi
  \setlength{\parindent}{0pt}%
  \setlength{\parskip}{0pt plus 0.3pt}%
  \let\item\@idxitem
}{%
  \ifkorrekturansicht\clearpage\fi
}
\makeatother

\IfFileExists{\jobname-pw.ind}{\input{\jobname-pw.ind}}{}

% Quellenangabe nur in der Leseansicht
\ifkorrekturansicht\else
% Fallback-Definitionen, falls die .tex-Datei \titel etc. nicht gesetzt hat
\providecommand{\titel}{}
\providecommand{\editorInnen}{}
\providecommand{\dateiname}{\jobname}

\vspace{3cm}

\vfill

\footnotesize
\textsc{Quelle}: \titel. Herausgegeben von {\editorInnen}. In: \emph{Arthur Schnitzler: Briefwechsel mit Autorinnen und Autoren}.
 Digitale Edition, https://schnitzler-briefe.acdh.oeaw.ac.at/{\dateiname}.html (Stand \today)
\fi

\end{document}


      