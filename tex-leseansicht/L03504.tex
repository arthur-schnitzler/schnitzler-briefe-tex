%% latex-korrekturansicht-vorspann.tex
%% Vorspann für die Korrekturansicht.
%% Lädt die gemeinsame Datei latex-vorspann.tex mit gesetztem Schalter.

\newif\ifkorrekturansicht
\korrekturansichttrue

\input{../tex-inputs/latex-vorspann}


\section[Felix Salten u. a. an Arthur Schnitzler, {[}zwischen 19. und 30. 7.? 1909{]}]{L03504 Felix Salten u. a. an Arthur Schnitzler,
               {[}zwischen 19. und 30. 7.? 1909{]}}
\nopagebreak\mylabel{L03504v}
\rehead{ }\normalsize\beginnumbering\briefempfaengerindex{Schnitzler, Arthur@\textsc{Schnitzler, Arthur}!zzzFischer, Hedwig@\emph{von Hedwig Fischer}!1909-07-301@{{[}zwischen 19. und 30. 7.? 1909{]}}|(be}\briefempfaengerindex{Schnitzler, Arthur@\textsc{Schnitzler, Arthur}!zzzFischer, Samuel@\emph{von Samuel Fischer}!1909-07-301@{{[}zwischen 19. und 30. 7.? 1909{]}}|(be}\briefempfaengerindex{Schnitzler, Arthur@\textsc{Schnitzler, Arthur}!zzzWassermann, Jakob@\emph{von Jakob Wassermann}!1909-07-301@{{[}zwischen 19. und 30. 7.? 1909{]}}|(be}\briefempfaengerindex{Schnitzler, Arthur@\textsc{Schnitzler, Arthur}!zzzSalten, Ottilie@\emph{von Ottilie Salten}!1909-07-301@{{[}zwischen 19. und 30. 7.? 1909{]}}|(be}\briefempfaengerindex{Schnitzler, Arthur@\textsc{Schnitzler, Arthur}!zzzSalten, Felix@\emph{von Felix Salten}!1909-07-301@{{[}zwischen 19. und 30. 7.? 1909{]}}|(be}
\toendnotes[C]{\smallbreak\pagebreak[2]}\Standort{CUL, Schnitzler, B 89, B 1.}
\physDesc{Bildpostkarte, 396 Zeichen
\newline{}Handschrift Felix Salten: Bleistift, lateinische Kurrent
\newline{}Handschrift Ottilie Salten: Bleistift, deutsche Kurrent
\newline{}Handschrift Hedwig Fischer: Bleistift, deutsche Kurrent
\newline{}Handschrift Jakob Wassermann: Bleistift, lateinische Kurrent
\newline{}Handschrift Samuel Fischer: Bleistift, lateinische Kurrent
\newline{}Versand: 1) Stempel: »\nobreak{}\oindex{Hôtel Duerrenstein@\textbf{Hôtel Dürrenstein}, \emph{Hotel (K.HTL)}|pwk}Hôtel Dürrenstein, 2000 M. Plätzwiese\oindex{Prato Piazza@\textbf{Prato Piazza}, \emph{T.MT}|pw} 2000 M.Alois Pahler\pwindex{Pahler, Alois @\textsc{Pahler, Alois}, \emph{Hotelbesitzer/Hotelbesitzerin}|pw}\nobreak{}«.   2) Stempel: »\nobreak{}\oindex{Hoehlenstein@\textbf{Höhlenstein}, \emph{P.PPLQ}|pwk}{[}L{]}andro, 8\nobreak{}«. 
\newline{}Schnitzler: mit Bleistift Vermerk: »\textsc{Salten}« 
\newline{}Ordnung: mit Bleistift von unbekannter Hand nummeriert: »254« }\toendnotes[C]{\smallbreak}\pstart{}{\pb}Herrn\pend{}\pstart{}D\textsuperscript{r} Arthur Schnitzler\pend{}\pstart{}Edlach \textsuperscript{b}/Reichenau\oindex{Edlach@\textbf{Edlach}, \emph{P.PPL}|pw}\pend{}\pstart{}Südbahn\orgindex{Suedbahnstrecke@Südbahnstrecke|pw}\pend{}\pstart{}Nied\textcolor{gray}{.} Öst\oindex{Niederoesterreich@\textbf{Niederösterreich}, \emph{A.ADM1}|pw}\pend{}{\bigskip}
\pstart
           {\pb}\textcolor{gray}{\textbf{Plätzwiesen\oindex{Prato Piazza@\textbf{Prato Piazza}, \emph{T.MT}|pw} (2003 m) mit Hoher Gaisl\oindex{Croda Rossa@\textbf{Croda Rossa}, \emph{T.MT}|pw} (3148 m).}}\hfill \textcolor{gray}{\textbf{Tirol\oindex{Suedtirol@\textbf{Südtirol}, \emph{A.ADM2}|pw}.}}\pend
           \vspace{1em}
\pstart
           \noindent{}{\pb}Schöner Weg – schönes Ausruhen und herzliches Gedenken der Entfernten. Hoffentlich
               geht es Ihrer \label{K_L03504-1v}\edtext{Frau\pwindex{Schnitzler, Olga 17.01.1882 – 13.01.1970@\textsc{Schnitzler, Olga} (17.01.1882 – 13.01.1970), \emph{Schauspieler/Schauspielerin, Sänger/Sängerin}|pwv} dauernd gut u. Heini\pwindex{Schnitzler, Heinrich 09.08.1902 – 12.07.1982@\textsc{Schnitzler, Heinrich} (09.08.1902 – 12.07.1982), \emph{Regisseur/Regisseurin, Schauspieler/Schauspielerin}|pw} ist ganz gesund}{\lemma{\textnormal{\emph{Frau … gesund}}}\Cendnote{\textnormal{Die Karte ist undatiert und lässt sich nur anhand
                  einiger Indizien einem Zeitraum zuordnen: Olga
                     Schnitzler\pwindex{Schnitzler, Olga 17.01.1882 – 13.01.1970@\textsc{Schnitzler, Olga} (17.01.1882 – 13.01.1970), \emph{Schauspieler/Schauspielerin, Sänger/Sängerin}|pwk} war schwanger und hatte zeitweise Beschwerden, vgl. A. S.: \emph{Tagebuch}, 26. 6. 1909. Heinrichs\pwindex{Schnitzler, Heinrich 09.08.1902 – 12.07.1982@\textsc{Schnitzler, Heinrich} (09.08.1902 – 12.07.1982), \emph{Regisseur/Regisseurin, Schauspieler/Schauspielerin}|pwk} Keuchhusten heilte Anfang Juli 1909 aus. Samuel Fischer\pwindex{Fischer, Samuel 24.12.1859 – 15.10.1934@\textsc{Fischer, Samuel} (24.12.1859 – 15.10.1934), \emph{Verleger/Verlegerin}|pwk} schrieb am 20. 7. 1909 aus Landro\oindex{Hoehlenstein@\textbf{Höhlenstein}, \emph{P.PPLQ}|pwk} an Schnitzler (vgl. 
                           Samuel Fischer\pwindex{Fischer, Samuel 24.12.1859 – 15.10.1934@\textsc{Fischer, Samuel} (24.12.1859 – 15.10.1934), \emph{Verleger/Verlegerin}|pwk}, Hedwig Fischer\pwindex{Fischer, Hedwig 08.09.1871 – 11.04.1952@\textsc{Fischer, Hedwig} (08.09.1871 – 11.04.1952)|pwk}: \emph{Briefwechsel mit
                              Autoren}. Herausgegeben von Dierk Rodewald und Corinna Fiedler. Mit
                           einer Einführung von Bernhard Zeller. Frankfurt am Main:
                           \emph{S. Fischer}{ }1989, S. 84). Da auf der Karte Saltens\pwindex{Salten, Felix 06.09.1869 – 08.10.1945@\textsc{Salten, Felix} (06.09.1869 – 08.10.1945), \emph{Schriftsteller/Schriftstellerin, Journalist/Journalistin, Chefredakteur/Chefredakteurin}|pwk} vom 18. 7. 1909 die Anwesenheit Fischers\pwindex{Fischer, Samuel 24.12.1859 – 15.10.1934@\textsc{Fischer, Samuel} (24.12.1859 – 15.10.1934), \emph{Verleger/Verlegerin}|pwk} nicht erwähnt wird und Heinrichs\pwindex{Schnitzler, Heinrich 09.08.1902 – 12.07.1982@\textsc{Schnitzler, Heinrich} (09.08.1902 – 12.07.1982), \emph{Regisseur/Regisseurin, Schauspieler/Schauspielerin}|pwk} Keuchhusten erst »besser« geworden ist, dürfte die vorliegende Karte
                  danach abgefasst worden sein – und vor dem Monatsende, da auf der Karte vom 31. 7. 1909 nicht mehr nach
                  dem Befinden Heinrichs\pwindex{Schnitzler, Heinrich 09.08.1902 – 12.07.1982@\textsc{Schnitzler, Heinrich} (09.08.1902 – 12.07.1982), \emph{Regisseur/Regisseurin, Schauspieler/Schauspielerin}|pwk} gefragt
               wird.}}}\label{K_L03504-1}. Alles Herzliche von uns zu Ihnen\pend
           \pstart Ihr \spacefill\mbox{Salten}\pend{}\selectlanguage{ngerman}\vspace{1em}
\pstart
           \noindent{}{[}hs. :{]} Viele ſchöne Grüße {\\}\spacefill\mbox{Otti}\pend
           \selectlanguage{ngerman}\vspace{1em}
\pstart
           \noindent{}{[}hs. :{]} herzliche Grüße und viele gute Wünsche für Frau \textsc{Schnitzler}\pwindex{Schnitzler, Olga 17.01.1882 – 13.01.1970@\textsc{Schnitzler, Olga} (17.01.1882 – 13.01.1970), \emph{Schauspieler/Schauspielerin, Sänger/Sängerin}|pw} u. \textsc{Heini}\pwindex{Schnitzler, Heinrich 09.08.1902 – 12.07.1982@\textsc{Schnitzler, Heinrich} (09.08.1902 – 12.07.1982), \emph{Regisseur/Regisseurin, Schauspieler/Schauspielerin}|pw}{ }{\\}\spacefill\mbox{Hedwig Fischer.}\pend
           \selectlanguage{ngerman}\vspace{1em}
\pstart
           \noindent{}{[}hs. :{]} Herzlich grüsst Ihr \spacefill\mbox{JakobWassermann}\pend
           \selectlanguage{ngerman}\vspace{1em}
\pstart
           \noindent{}{[}hs. :{]} Herzliche Grüße Ihr \spacefill\mbox{SFischer}\pend
           \selectlanguage{ngerman}\endnumbering\briefempfaengerindex{Schnitzler, Arthur@\textsc{Schnitzler, Arthur}!zzzFischer, Hedwig@\emph{von Hedwig Fischer}!1909-07-191@{{[}zwischen 19. und 30. 7.? 1909{]}}|)be}\briefempfaengerindex{Schnitzler, Arthur@\textsc{Schnitzler, Arthur}!zzzFischer, Samuel@\emph{von Samuel Fischer}!1909-07-191@{{[}zwischen 19. und 30. 7.? 1909{]}}|)be}\briefempfaengerindex{Schnitzler, Arthur@\textsc{Schnitzler, Arthur}!zzzWassermann, Jakob@\emph{von Jakob Wassermann}!1909-07-191@{{[}zwischen 19. und 30. 7.? 1909{]}}|)be}\briefempfaengerindex{Schnitzler, Arthur@\textsc{Schnitzler, Arthur}!zzzSalten, Ottilie@\emph{von Ottilie Salten}!1909-07-191@{{[}zwischen 19. und 30. 7.? 1909{]}}|)be}\briefempfaengerindex{Schnitzler, Arthur@\textsc{Schnitzler, Arthur}!zzzSalten, Felix@\emph{von Felix Salten}!1909-07-191@{{[}zwischen 19. und 30. 7.? 1909{]}}|)be}\mylabel{L03504h}  \normalsize

\doendnotes{C}
\bigskip
\vfill

\clearpage

\footnotesize

\lohead{\textsc{register}}

% Definiere theindex-Environment komplett neu ohne reledmac
\makeatletter
\renewenvironment{theindex}{%
  \section*{\indexname}%
  \setlength{\parindent}{0pt}%
  \setlength{\parskip}{0pt plus 0.3pt}%
  \let\item\@idxitem
}{%
  \clearpage
}
\makeatother

\IfFileExists{\jobname-pw.ind}{\input{\jobname-pw.ind}}{}

\end{document}

      