%% latex-leseansicht-vorspann.tex
%% Vorspann für die Leseansicht.
%% Lädt die gemeinsame Datei latex-vorspann.tex mit nicht gesetztem Schalter.

\newif\ifkorrekturansicht
\korrekturansichtfalse

\input{../tex-inputs/latex-vorspann}


\section[Paul Goldmann an Arthur Schnitzler, 1. 4. {[}1896{]}]{L02769 Paul Goldmann an Arthur Schnitzler, 1. 4. [1896]}
\nopagebreak\mylabel{L02769v}
\rehead{ }\normalsize\beginnumbering\briefempfaengerindex{Schnitzler, Arthur@\textsc{Schnitzler, Arthur}!zzzGoldmann, Paul@\emph{von Paul Goldmann}!1896-04-011@{1. 4. [1896]}|(be}
\toendnotes[C]{\smallbreak\pagebreak[2]}
\correspDesc{Versand  durch Paul Goldmann am 1. 4. [1896] in Paris
\newline{}Erhalt  durch Arthur Schnitzler im Zeitraum [2. 4. 1896
                  – 6. 4. 1896?] in Wien}\toendnotes[C]{\smallbreak}
\Standort{DLA, A:Schnitzler, HS.NZ85.1.3166.}
\physDesc{Brief, 1 Blatt, 4 Seiten, 2222 Zeichen
\newline{}Handschrift: blaue Tinte, deutsche Kurrent
\newline{}Schnitzler: 1) mit Bleistift das Jahr »96« vermerkt  2) mit rotem Buntstift vier Unterstreichungen}\toendnotes[C]{\smallbreak}
\pstart
           {\pb}\textcolor{gray}{\textbf{\textbf{Frankfurter Zeitung\orgindex{Frankfurter Zeitung@Frankfurter Zeitung|pw}}}}\pend
           
\pstart
           \textcolor{gray}{\textbf{(\begin{otherlanguage}{french}Gazette de Francfort\end{otherlanguage}\orgindex{Frankfurter Zeitung@Frankfurter Zeitung|pw}).}}\pend
           
\pstart
           \textcolor{gray}{\textbf{\textbf{\begin{otherlanguage}{french}Fondateur M.\end{otherlanguage}{ }L. Sonnemann\pwindex{Sonnemann, Leopold 29.\,10.\,1831 Höchberg – 30.\,10.\,1909 Frankfurt am Main@\textsc{Sonnemann, Leopold} (29.\,10.\,1831 Höchberg – 30.\,10.\,1909 Frankfurt am Main), \emph{Journalist, Herausgeber}|pw}.}}}\pend
           
\pstart
           \begin{otherlanguage}{french}\textcolor{gray}{\textbf{Journal\pwindex{Frankfurter Zeitung@\emph{Frankfurter Zeitung}|pwv} politique,
                        financier,}}\end{otherlanguage}\pend
           
\pstart
           \begin{otherlanguage}{french}\textcolor{gray}{\textbf{commercial et littéraire.}}\end{otherlanguage}\pend
           
\pstart
           \begin{otherlanguage}{french}\textcolor{gray}{\textbf{\textbf{Paraissant trois fois par jour.}}}\end{otherlanguage}\hfill \textsc{Paris\oindex{Paris@\textbf{Paris}, \emph{Hauptstadt}|pw}}, 1. April.\pend
           
\pstart
           \begin{otherlanguage}{french}\textcolor{gray}{\textbf{\textbf{Bureau à Paris\oindex{Paris@\textbf{Paris}, \emph{Hauptstadt}|pw}:}}}\end{otherlanguage}\pend
           
\pstart
           \begin{otherlanguage}{french}\textcolor{gray}{\textbf{\textbf{24. Rue Feydeau\oindex{rue Feydeau@\textbf{rue Feydeau}, \emph{Straße}|pw}.}}}\end{otherlanguage}\pend
           
\pstart\center{}Mein lieber Freund,\pend\vspace{0.5em}
\pstart
           Du{ }ſiehſt wohl, was Alles in der fran\oindex{Frankreich@\textbf{Frankreich}|pwv}zöſiſchen Politik vorgeht. Der Teufel iſt los, und ich komme noch immer
               nicht dazu, Dir zu{ }ſchreiben. Ich will Dir nur in der Eile für Deinen letzten lieben
               Brief danken. Auch für Deine Photographie, die mich unendlich erfreut hat, habe ich
               Dir wohl noch nicht gedankt. \textsc{Richard Specht\pwindex{Specht, Richard 7.\,12.\,1870 Wien – 18.\,3.\,1932 ebd.@\textsc{Specht, Richard} (7.\,12.\,1870 Wien – 18.\,3.\,1932 ebd.), \emph{Schriftsteller, Journalist, Kritiker}|pw}} iſt hier und macht mir viel Vergnügen; er iſt ein lieber,{ }ſanfter Menſch
               geworden; aber Talent hat er wohl nicht; er las uns ein \label{K_L02769-1v}\edtext{Vers-Drama\pwindex{Specht, Richard 7.\,12.\,1870 Wien – 18.\,3.\,1932 ebd.@\textsc{Specht, Richard} (7.\,12.\,1870 Wien – 18.\,3.\,1932 ebd.), \emph{Schriftsteller, Journalist, Kritiker}!Pierrot bossu. Eine Commedia dell’Arte zur Fastnacht in gar zierlichen Reimen@\strich\emph{Pierrot bossu. Eine Commedia dell’Arte zur Fastnacht in gar zierlichen Reimen}|pwv}}{\lemma{\textnormal{\emph{Vers-Drama}}}\Cendnote{\textnormal{\emph{Pierrot bossu. Eine Commedia dell’Arte zur
                     Fastnacht in gar zierlichen Reimen}\pwindex{Specht, Richard 7.\,12.\,1870 Wien – 18.\,3.\,1932 ebd.@\textsc{Specht, Richard} (7.\,12.\,1870 Wien – 18.\,3.\,1932 ebd.), \emph{Schriftsteller, Journalist, Kritiker}!Pierrot bossu. Eine Commedia dell’Arte zur Fastnacht in gar zierlichen Reimen@\strich\emph{Pierrot bossu. Eine Commedia dell’Arte zur Fastnacht in gar zierlichen Reimen}|pwk}, verfertigt von Richard Specht\pwindex{Specht, Richard 7.\,12.\,1870 Wien – 18.\,3.\,1932 ebd.@\textsc{Specht, Richard} (7.\,12.\,1870 Wien – 18.\,3.\,1932 ebd.), \emph{Schriftsteller, Journalist, Kritiker}|pwk}, war Mitte Februar 1896 bei \emph{E. Pierson}\orgindex{E. Pierson’s Verlag@E. Pierson’s Verlag|pwk} erschienen.}}}\label{K_L02769-1}: Verſe, aber
               keine Poeſie. Armer Burſch\pwindex{Specht, Richard 7.\,12.\,1870 Wien – 18.\,3.\,1932 ebd.@\textsc{Specht, Richard} (7.\,12.\,1870 Wien – 18.\,3.\,1932 ebd.), \emph{Schriftsteller, Journalist, Kritiker}|pwv}!
               Er möchte{ }ſo gern!\pend
           
\pstart
           {\pb}Was Du über die Judenfrage im Zuſammenhang mit \textsc{Herzls\pwindex{Herzl, Theodor 2.\,5.\,1860 Budapest – 3.\,7.\,1904 Edlach@\textsc{Herzl, Theodor} (2.\,5.\,1860 Budapest – 3.\,7.\,1904 Edlach), \emph{Schriftsteller, Journalist}|pw}}{ }Buch\pwindex{Herzl, Theodor 2.\,5.\,1860 Budapest – 3.\,7.\,1904 Edlach@\textsc{Herzl, Theodor} (2.\,5.\,1860 Budapest – 3.\,7.\,1904 Edlach), \emph{Schriftsteller, Journalist}!Judenstaat. Versuch einer modernen Lösung der Judenfrage@\strich\emph{Der Judenstaat. Versuch einer modernen Lösung der Judenfrage}|pwv}{ }ſchreibſt, iſt prächtig
               und mir ganz aus der Seele geſprochen. Aber das Buch\pwindex{Herzl, Theodor 2.\,5.\,1860 Budapest – 3.\,7.\,1904 Edlach@\textsc{Herzl, Theodor} (2.\,5.\,1860 Budapest – 3.\,7.\,1904 Edlach), \emph{Schriftsteller, Journalist}!Judenstaat. Versuch einer modernen Lösung der Judenfrage@\strich\emph{Der Judenstaat. Versuch einer modernen Lösung der Judenfrage}|pwv} iſt wirklich albern, – oberflächlich noch dazu und
               falſch{ }ſentimental. Echte{ }ſchlechte Feuilletoniſten-Literatur. Aber wie verbohrt, wie
               falſch beobachtend muß ein Menſch{ }ſein, der heut noch behauptet, die Juden{ }ſeien ein
               Volk. Du und ich, der Rabbi{ }\strikeout{\textcolor{gray}{Blo}{ }\textsc{Bloc\textcolor{gray}{h}}}{ }\textsc{\label{K_L02769-2v}\edtext{Bloch\pwindex{Bloch, Joseph Samuel 20.\,11.\,1850 Dukla – 1.\,10.\,1923 Wien@\textsc{Bloch, Joseph Samuel} (20.\,11.\,1850 Dukla – 1.\,10.\,1923 Wien), \emph{Politiker, Publizist, Rabbiner}|pw}}{\lemma{\textnormal{\emph{Bloch}}}\Cendnote{\textnormal{Joseph Samuel Bloch\pwindex{Bloch, Joseph Samuel 20.\,11.\,1850 Dukla – 1.\,10.\,1923 Wien@\textsc{Bloch, Joseph Samuel} (20.\,11.\,1850 Dukla – 1.\,10.\,1923 Wien), \emph{Politiker, Publizist, Rabbiner}|pwk} trat als
                     Abgeordneter im \emph{Reichsrat}\orgindex{Reichsrat@Reichsrat|pwk} engagiert gegen
                     antisemitische Verleumdungen auf.}}}\label{K_L02769-2}} und der Jud’, der unten »handel\textcolor{gray}{n}«{ }ſchreit – ein Volk! Das
               iſt echt \textsc{Herzl\pwindex{Herzl, Theodor 2.\,5.\,1860 Budapest – 3.\,7.\,1904 Edlach@\textsc{Herzl, Theodor} (2.\,5.\,1860 Budapest – 3.\,7.\,1904 Edlach), \emph{Schriftsteller, Journalist}|pw}}. So hat er auch die fran\oindex{Frankreich@\textbf{Frankreich}|pwv}zöſiſchen Dinge angeſchaut u. immer unrichtig geſehen. Für mich gibt es
               eben nur eine Löſung der Judenfrage: daß die Juden{ }ſchließlich {\pb}Alle Chriſten werden. Jeſus\pwindex{Jesus 7–4 v.\,u.\,Z. Nazareth – 30/31 Jerusalem@\textsc{Jesus} (7–4 v.\,u.\,Z. Nazareth – 30/31 Jerusalem), \emph{Wanderprediger}|pw} iſt mir doch der{ }ſympathiſcheſte Jude und ich will gern
               zu{ }ſeinen Jüngern zählen{\dotsfive}\pend
           
\pstart
           Mein Onkel\pwindex{Mamroth, Fedor 21.\,2.\,1851 Breslau – 25.\,6.\,1907 Frankfurt am Main@\textsc{Mamroth, Fedor} (21.\,2.\,1851 Breslau – 25.\,6.\,1907 Frankfurt am Main), \emph{Journalist, Kritiker}|pwv} hat nett über »\textsc{Anatol\pwindex{Schnitzler, Arthur 15. 5. 1862 Wien – 21. 10. 1931 ebd.@\textsc{Schnitzler, Arthur} (15. 5. 1862 Wien – 21. 10. 1931 ebd.), \emph{Schriftsteller, Mediziner}!Anatol@\strich\emph{Anatol}|pw}}« \label{K_L02769-3v}\edtext{geſchrieben\pwindex{Mamroth, Fedor 21.\,2.\,1851 Breslau – 25.\,6.\,1907 Frankfurt am Main@\textsc{Mamroth, Fedor} (21.\,2.\,1851 Breslau – 25.\,6.\,1907 Frankfurt am Main), \emph{Journalist, Kritiker}!Schauspielhaus. [Untreu und Abschiedssouper]@\strich\emph{Schauspielhaus. [Untreu und Abschiedssouper]}|pwv}}{\lemma{\textnormal{\emph{geschrieben}}}\Cendnote{\textnormal{m.\pwindex{Mamroth, Fedor 21.\,2.\,1851 Breslau – 25.\,6.\,1907 Frankfurt am Main@\textsc{Mamroth, Fedor} (21.\,2.\,1851 Breslau – 25.\,6.\,1907 Frankfurt am Main), \emph{Journalist, Kritiker}|pwk} [ = Fedor Mamroth\pwindex{Mamroth, Fedor 21.\,2.\,1851 Breslau – 25.\,6.\,1907 Frankfurt am Main@\textsc{Mamroth, Fedor} (21.\,2.\,1851 Breslau – 25.\,6.\,1907 Frankfurt am Main), \emph{Journalist, Kritiker}|pwk}]: \emph{Schauspielhaus}\pwindex{Mamroth, Fedor 21.\,2.\,1851 Breslau – 25.\,6.\,1907 Frankfurt am Main@\textsc{Mamroth, Fedor} (21.\,2.\,1851 Breslau – 25.\,6.\,1907 Frankfurt am Main), \emph{Journalist, Kritiker}!Schauspielhaus. [Untreu und Abschiedssouper]@\strich\emph{Schauspielhaus. [Untreu und Abschiedssouper]}|pwk}.
                     In: \emph{Frankfurter Zeitung}\pwindex{Frankfurter Zeitung@\emph{Frankfurter Zeitung}|pwk}, Jg. 40, Nr. 89,
                     29. 3. 1896, Zweites Morgenblatt, S. 1. Mamroth\pwindex{Mamroth, Fedor 21.\,2.\,1851 Breslau – 25.\,6.\,1907 Frankfurt am Main@\textsc{Mamroth, Fedor} (21.\,2.\,1851 Breslau – 25.\,6.\,1907 Frankfurt am Main), \emph{Journalist, Kritiker}|pwk} besprach die
                  gemeinsame Aufführung von \emph{Untreu}\pwindex{Bracco, Roberto 10.\,11.\,1861 Neapel – 20.\,4.\,1943 Sorrent@\textsc{Bracco, Roberto} (10.\,11.\,1861 Neapel – 20.\,4.\,1943 Sorrent), \emph{Schriftsteller}!Untreu. Komödie in 3 Acten@\strich\emph{Untreu. Komödie in 3 Acten}|pwk} von Roberto Bracco\pwindex{Bracco, Roberto 10.\,11.\,1861 Neapel – 20.\,4.\,1943 Sorrent@\textsc{Bracco, Roberto} (10.\,11.\,1861 Neapel – 20.\,4.\,1943 Sorrent), \emph{Schriftsteller}|pwk} und Schnitzlers{ }\emph{Abschiedssouper}\pwindex{Schnitzler, Arthur 15. 5. 1862 Wien – 21. 10. 1931 ebd.@\textsc{Schnitzler, Arthur} (15. 5. 1862 Wien – 21. 10. 1931 ebd.), \emph{Schriftsteller, Mediziner}!Abschiedssouper@\strich\emph{Abschiedssouper}|pwk} am \emph{Frankfurter
                     Schauspielhaus}\orgindex{Frankfurter Stadttheater@Frankfurter Stadttheater|pwk} am 26. 3. 1896.}}}\label{K_L02769-3}. Meine Mutter\pwindex{Goldmann, Clementine 15.\,5.\,1842 Breslau – 24.\,2.\,1924 Frankfurt am Main@\textsc{Goldmann, Clementine} (15.\,5.\,1842 Breslau – 24.\,2.\,1924 Frankfurt am Main)|pwv}{ }ſendet noch folgende
               Ergänzungs-Kritik:\pend
           {\vspace{1\baselineskip}}
\pstart
           {[}hs. Goldmann:{]} \label{T_L02769-1v}\edtext{Das »Abschieds« Souper\pwindex{Schnitzler, Arthur 15. 5. 1862 Wien – 21. 10. 1931 ebd.@\textsc{Schnitzler, Arthur} (15. 5. 1862 Wien – 21. 10. 1931 ebd.), \emph{Schriftsteller, Mediziner}!Abschiedssouper@\strich\emph{Abschiedssouper}|pw} von deinem Freunde hat uns{ }ſehr gefallen – we{\geminationn} es auch für die{ }ſtupiden Frankfurt\oindex{Frankfurt am Main@\textbf{Frankfurt am Main}, \emph{Hauptstadt}|pw}er – viel zu fein war\textcolor{gray}{.}}{\lemma{\textnormal{\emph{Das … war.}}}\Cendnote{\textnormal{Ausschnitt aus einem Brief von Clementine Goldmann\pwindex{Goldmann, Clementine 15.\,5.\,1842 Breslau – 24.\,2.\,1924 Frankfurt am Main@\textsc{Goldmann, Clementine} (15.\,5.\,1842 Breslau – 24.\,2.\,1924 Frankfurt am Main)|pwk} auf einem eingeklebten
                  Zettel (blaue Tinte, deutsche Kurrentschrift)}}}\label{T_L02769-1}\pend
           {\vspace{1\baselineskip}}
\pstart
           {[}hs. Goldmann:{]} Oſtern möchte ich nach Frankfurt\oindex{Frankfurt am Main@\textbf{Frankfurt am Main}, \emph{Hauptstadt}|pw} fahren, weiß aber noch nicht, woher ich das Geld nehmen werde.
               Aber ich bin todt gearbeitet und habe ein {\pb}heftiges
               Bedürfniß nach ein paar Ruhetagen. Mit meinen Augen geht es{ }ſchlecht,{ }ſie wollen
               nicht mehr mit, und ich habe große Sorgen.\pend
           
\pstart
           Vielleicht{ }ſchreibe ich Dir den langen Brief doch noch vor den Feiertagen. Wenn
               nicht: fröhliche Oſtern.\pend
           
\pstart
           Grüß’ Dich Gott, mein lieber Freund{\\[\baselineskip]}Dein {\\[\baselineskip]}\spacefill\mbox{Paul Goldmann.}\pend
           \leftskip=0em{}
\pstart
           \noindent{}Der \label{K_L02769-4v}\edtext{Artikel\pwindex{Hofmannsthal, Hugo von 1.\,2.\,1874 Wien – 15.\,7.\,1929 Rodaun@\textsc{Hofmannsthal, Hugo von} (1.\,2.\,1874 Wien – 15.\,7.\,1929 Rodaun), \emph{Schriftsteller}!Gedichte von Stefan George@\strich\emph{Gedichte von Stefan George}|pwv}}{\lemma{\textnormal{\emph{Artikel}}}\Cendnote{\textnormal{Hugo von Hofmannsthal\pwindex{Hofmannsthal, Hugo von 1.\,2.\,1874 Wien – 15.\,7.\,1929 Rodaun@\textsc{Hofmannsthal, Hugo von} (1.\,2.\,1874 Wien – 15.\,7.\,1929 Rodaun), \emph{Schriftsteller}|pwk}: \emph{Gedichte von Stefan George}\pwindex{Hofmannsthal, Hugo von 1.\,2.\,1874 Wien – 15.\,7.\,1929 Rodaun@\textsc{Hofmannsthal, Hugo von} (1.\,2.\,1874 Wien – 15.\,7.\,1929 Rodaun), \emph{Schriftsteller}!Gedichte von Stefan George@\strich\emph{Gedichte von Stefan George}|pwk}. In: \emph{Die Zeit}\pwindex{Zeit. Wiener Wochenschrift@\emph{Die Zeit. Wiener Wochenschrift}|pwk}, Bd. 6, Nr. 77, 21. 3. 1896, S. 189–191.}}}\label{K_L02769-4} des
                  kleinen \textsc{Loris\pwindex{Hofmannsthal, Hugo von 1.\,2.\,1874 Wien – 15.\,7.\,1929 Rodaun@\textsc{Hofmannsthal, Hugo von} (1.\,2.\,1874 Wien – 15.\,7.\,1929 Rodaun), \emph{Schriftsteller}|pw}} in der »Zeit\pwindex{Zeit. Wiener Wochenschrift@\emph{Die Zeit. Wiener Wochenschrift}|pw}« über \textsc{Stefan Georges\pwindex{George, Stefan 17.\,7.\,1868 Büdesheim – 4.\,12.\,1933 Minusio@\textsc{George, Stefan} (17.\,7.\,1868 Büdesheim – 4.\,12.\,1933 Minusio), \emph{Schriftsteller, Übersetzer}|pw}} hat mich einfach empört. \textsc{Stefan Georges\pwindex{George, Stefan 17.\,7.\,1868 Büdesheim – 4.\,12.\,1933 Minusio@\textsc{George, Stefan} (17.\,7.\,1868 Büdesheim – 4.\,12.\,1933 Minusio), \emph{Schriftsteller, Übersetzer}|pw}} iſt eine prätentiöſe Talentloſigkeit, und der Artikel\pwindex{Hofmannsthal, Hugo von 1.\,2.\,1874 Wien – 15.\,7.\,1929 Rodaun@\textsc{Hofmannsthal, Hugo von} (1.\,2.\,1874 Wien – 15.\,7.\,1929 Rodaun), \emph{Schriftsteller}!Gedichte von Stefan George@\strich\emph{Gedichte von Stefan George}|pwv}, abgeſehen von dem falſchen
                  Urtheil, iſt in einem unerhört{ }ſchwülſtigen u. manierirten Styl geſchrieben. Ein
                  zweiter \textsc{Hermann Bahr\pwindex{Bahr, Hermann 19.\,7.\,1863 Linz – 15.\,1.\,1934 München@\textsc{Bahr, Hermann} (19.\,7.\,1863 Linz – 15.\,1.\,1934 München), \emph{Schriftsteller, Kritiker}|pw}}!\pend
           
\pstart
           {\pb}\label{T_L02769-2v}\edtext{\uline{Gruß an \textsc{Richard\pwindex{Beer-Hofmann, Richard 11.\,7.\,1866 Wien – 26.\,9.\,1945 New York City@\textsc{Beer-Hofmann, Richard} (11.\,7.\,1866 Wien – 26.\,9.\,1945 New York City), \emph{Schriftsteller}|pw}}!}}{\lemma{\textnormal{\emph{Gruß an Richard!}}}\Cendnote{\textnormal{kopfüber am oberen Rand der ersten
                     Seite}}}\label{T_L02769-2}\pend
           \selectlanguage{ngerman}\endnumbering\briefempfaengerindex{Schnitzler, Arthur@\textsc{Schnitzler, Arthur}!zzzGoldmann, Paul@\emph{von Paul Goldmann}!1896-04-011@{1. 4. [1896]}|)be}\mylabel{L02769h}  \newcommand{\dateiname}{L02769}\newcommand{\titel}{Paul Goldmann an Arthur Schnitzler, 1. 4. [1896]}\newcommand{\editorInnen}{Martin Anton Müller und Laura Untner}%% latex-leseansicht-abspann.tex
%% Abspann für die Leseansicht.
%% Der Schalter \ifkorrekturansicht ist bereits durch den Vorspann gesetzt.

%% latex-abspann.tex
%% Gemeinsamer Abspann für Korrekturansicht und Leseansicht.
%% Setzt den Schalter \ifkorrekturansicht voraus (gesetzt in den
%% einbindenden Dateien latex-korrekturansicht-abspann.tex bzw.
%% latex-leseansicht-abspann.tex).
%% ---------------------------------------------------------------

\normalsize

% Das esempio-Environment wird nur in der Leseansicht benötigt
\ifkorrekturansicht\else
\newenvironment{esempio}[3]%
{
    \vspace{1.5ex}
    \rlap{\underline{#1}}
    \par
    \setlength{\parindent}{0cm}
    \nopagebreak
    \leftskip=#2cm
    \rightskip=#3cm
}
{
    \par
}
\fi

\doendnotes{C}
\bigskip
\vfill

\clearpage

\footnotesize

\ifkorrekturansicht
  \lohead{\textsc{register}}
\fi

% theindex-Environment neu definieren ohne reledmac
\makeatletter
\renewenvironment{theindex}{%
  \ifkorrekturansicht
    \section*{\indexname}%
  \else
    \subsubsection*{Index der erwähnten Entitäten}%
  \fi
  \setlength{\parindent}{0pt}%
  \setlength{\parskip}{0pt plus 0.3pt}%
  \let\item\@idxitem
}{%
  \ifkorrekturansicht\clearpage\fi
}
\makeatother

\IfFileExists{\jobname-pw.ind}{\input{\jobname-pw.ind}}{}

% Quellenangabe nur in der Leseansicht
\ifkorrekturansicht\else
% Fallback-Definitionen, falls die .tex-Datei \titel etc. nicht gesetzt hat
\providecommand{\titel}{}
\providecommand{\editorInnen}{}
\providecommand{\dateiname}{\jobname}

\vspace{3cm}

\vfill

\footnotesize
\textsc{Quelle}: \titel. Herausgegeben von {\editorInnen}. In: \emph{Arthur Schnitzler: Briefwechsel mit Autorinnen und Autoren}.
 Digitale Edition, https://schnitzler-briefe.acdh.oeaw.ac.at/{\dateiname}.html (Stand \today)
\fi

\end{document}


