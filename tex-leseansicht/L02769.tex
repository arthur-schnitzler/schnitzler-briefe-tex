%% latex-leseansicht-vorspann.tex
%% Vorspann für die Leseansicht.
%% Lädt die gemeinsame Datei latex-vorspann.tex mit nicht gesetztem Schalter.

\newif\ifkorrekturansicht
\korrekturansichtfalse

\input{../tex-inputs/latex-vorspann}


         
         \renewcommand{\erwaehntePersonen}{Personen: Hermann Bahr, Richard Beer-Hofmann, Joseph Samuel Bloch, Roberto Bracco, Stefan George, Paul Goldmann, Clementine Goldmann, Theodor Herzl, Hugo von Hofmannsthal,  Jesus, Fedor Mamroth, Leopold Sonnemann, Richard Specht}
         \renewcommand{\erwaehnteInstitutionen}{Institutionen: E. Pierson’s Verlag, Frankfurter Städtisches Schauspielhaus, Frankfurter Zeitung, Reichsrat}
         \renewcommand{\erwaehnteOrte}{Orte: Frankfurt am Main, Frankreich, Paris, Wien, rue Feydeau}
         \renewcommand{\erwaehnteWerke}{Werke: Abschiedssouper, Anatol, Der Judenstaat. Versuch einer modernen Lösung der Judenfrage, Die Zeit. Wiener Wochenschrift, Frankfurter Zeitung, Gedichte von Stefan George, Pierrot bossu. Eine Commedia dell’Arte zur Fastnacht in gar zierlichen Reimen, Schauspielhaus. [Untreu und Abschiedssouper], Untreu. Komödie in 3 Acten}
               \section[Paul Goldmann an Arthur Schnitzler, 1. 4. {[}1896{]}]{ Paul Goldmann an Arthur Schnitzler, 1. 4. {[}1896{]}}\nopagebreak\mylabel{v}\rehead{ }\begin{ledgroupsized}[t]{13cm}\normalsize\beginnumbering\briefempfaengerindex{Schnitzler, Arthur@\textsc{Schnitzler, Arthur}!zzzGoldmann, Paul@\emph{von Paul Goldmann}!1896-04-011@{1. 4. {[}1896{]}}|(be} \toendnotes[C]{\smallbreak\pagebreak[2]} \Standort{DLA, A:Schnitzler, HS.NZ85.1.3166.}
\physDesc{Brief, 1 Blatt, 4 Seiten, 2222 Zeichen
\newline{}Handschrift: blaue Tinte, deutsche Kurrent
\newline{}Schnitzler: 1) mit Bleistift das Jahr »96« vermerkt  2) mit rotem Buntstift vier Unterstreichungen}\toendnotes[C]{\smallbreak}\pstart
           \noindent{}{\pb}\textcolor{gray}{\textbf{\textbf{Frankfurter Zeitung\orgindex{Frankfurter Zeitung@Frankfurter Zeitung|pw}}}}\pend
           \pstart
           \textcolor{gray}{\textbf{(\begin{otherlanguage}{french}Gazette de Francfort\end{otherlanguage}\orgindex{Frankfurter Zeitung@Frankfurter Zeitung|pw}).}}\pend
           \pstart
           \textcolor{gray}{\textbf{\textbf{\begin{otherlanguage}{french}Fondateur M.\end{otherlanguage}{ }L. Sonnemann\pwindex{Sonnemann, Leopold 1831-10-29 – 1909-10-30@\textsc{Sonnemann, Leopold} (1831-10-29 – 1909-10-30), \emph{Journalist, Herausgeber}|pw}.}}}\pend
           \pstart
           \begin{otherlanguage}{french}\textcolor{gray}{\textbf{Journal\pwindex{?? Werk@Nicht ermittelte Verfasserinnen und Verfasser!Frankfurter Zeitung1856 – 1943@\emph{Frankfurter Zeitung} {[}1856 – 1943{]}|pwv} politique,
                        financier,}}\end{otherlanguage}\pend
           \pstart
           \begin{otherlanguage}{french}\textcolor{gray}{\textbf{commercial et littéraire.}}\end{otherlanguage}\pend
           \pstart
           \begin{otherlanguage}{french}\textcolor{gray}{\textbf{\textbf{Paraissant trois fois par jour.}}}\end{otherlanguage}\hfill \textsc{Paris\oindex{Paris@\textbf{Paris}|pw}}, 1. April.\pend
           \pstart
           \begin{otherlanguage}{french}\textcolor{gray}{\textbf{\textbf{Bureau à Paris\oindex{Paris@\textbf{Paris}|pw}:}}}\end{otherlanguage}\pend
           \pstart
           \begin{otherlanguage}{french}\textcolor{gray}{\textbf{\textbf{24. Rue Feydeau\oindex{rue Feydeau@\textbf{rue Feydeau}|pw}.}}}\end{otherlanguage}\pend
           \pstart\center{}Mein lieber Freund,\pend\pstart
           Du ſiehſt wohl, was Alles in der fran\oindex{Frankreich@\textbf{Frankreich}|pwv}zöſiſchen Politik vorgeht. Der Teufel iſt los, und ich komme noch immer
               nicht dazu, Dir zu ſchreiben. Ich will Dir nur in der Eile für Deinen letzten lieben
               Brief danken. Auch für Deine Photographie, die mich unendlich erfreut hat, habe ich
               Dir wohl noch nicht gedankt. \textsc{Richard Specht\pwindex{Specht, Richard 07.12.1870 – 18.03.1932@\textsc{Specht, Richard} (07.12.1870 – 18.03.1932), \emph{Schriftsteller, Journalist, Kritiker}|pw}} iſt hier und macht mir viel Vergnügen; er iſt ein lieber, ſanfter Menſch
               geworden; aber Talent hat er wohl nicht; er las uns ein \label{K_L02769-1v}\edtext{Vers-Drama\pwindex{Specht, Richard 07.12.1870 – 18.03.1932@\textsc{Specht, Richard} (07.12.1870 – 18.03.1932), \emph{Schriftsteller, Journalist, Kritiker}!Pierrot bossu. Eine Commedia dell Arte zur Fastnacht in gar zierlichen
                  Reimen1895@\strich\emph{Pierrot bossu. Eine Commedia dell’Arte zur Fastnacht in gar zierlichen Reimen} {[}1895{]}|pwv}}{\lemma{\textnormal{\emph{Vers-Drama}}}\Cendnote{\textnormal{\emph{Pierrot bossu. Eine Commedia dell’Arte zur
                     Fastnacht in gar zierlichen Reimen}\pwindex{Specht, Richard 07.12.1870 – 18.03.1932@\textsc{Specht, Richard} (07.12.1870 – 18.03.1932), \emph{Schriftsteller, Journalist, Kritiker}!Pierrot bossu. Eine Commedia dell Arte zur Fastnacht in gar zierlichen
                  Reimen1895@\strich\emph{Pierrot bossu. Eine Commedia dell’Arte zur Fastnacht in gar zierlichen Reimen} {[}1895{]}|pwk}, verfertigt von Richard Specht\pwindex{Specht, Richard 07.12.1870 – 18.03.1932@\textsc{Specht, Richard} (07.12.1870 – 18.03.1932), \emph{Schriftsteller, Journalist, Kritiker}|pwk}, war Mitte Februar 1896 bei \emph{E. Pierson}\orgindex{E. Pierson s Verlag@E. Pierson’s Verlag|pwk} erschienen.}}}\label{K_L02769-1h}: Verſe, aber
               keine Poeſie. Armer Burſch\pwindex{Specht, Richard 07.12.1870 – 18.03.1932@\textsc{Specht, Richard} (07.12.1870 – 18.03.1932), \emph{Schriftsteller, Journalist, Kritiker}|pwv}!
               Er möchte ſo gern!\pend
           \pstart
           {\pb}Was Du über die Judenfrage im Zuſammenhang mit \textsc{Herzls\pwindex{Herzl, Theodor 1860-05-02 – 1904-07-03@\textsc{Herzl, Theodor} (1860-05-02 – 1904-07-03), \emph{Schriftsteller, Journalist}|pw}}{ }Buch\pwindex{Herzl, Theodor 1860-05-02 – 1904-07-03@\textsc{Herzl, Theodor} (1860-05-02 – 1904-07-03), \emph{Schriftsteller, Journalist}!Judenstaat. Versuch einer modernen Loesung der Judenfrage1896-02-20@\strich\emph{Der Judenstaat. Versuch einer modernen Lösung der Judenfrage} {[}1896-02-20{]}|pwv} ſchreibſt, iſt prächtig
               und mir ganz aus der Seele geſprochen. Aber das Buch\pwindex{Herzl, Theodor 1860-05-02 – 1904-07-03@\textsc{Herzl, Theodor} (1860-05-02 – 1904-07-03), \emph{Schriftsteller, Journalist}!Judenstaat. Versuch einer modernen Loesung der Judenfrage1896-02-20@\strich\emph{Der Judenstaat. Versuch einer modernen Lösung der Judenfrage} {[}1896-02-20{]}|pwv} iſt wirklich albern, – oberflächlich noch dazu und
               falſch ſentimental. Echte ſchlechte Feuilletoniſten-Literatur. Aber wie verbohrt, wie
               falſch beobachtend muß ein Menſch ſein, der heut noch behauptet, die Juden ſeien ein
               Volk. Du und ich, der Rabbi{ }\strikeout{\textcolor{gray}{Blo}{ }\textsc{Bloc\textcolor{gray}{h}}}{ }\textsc{\label{K_L02769-2v}\edtext{Bloch\pwindex{Bloch, Joseph Samuel 1850-11-20 – 1923-10-01@\textsc{Bloch, Joseph Samuel} (1850-11-20 – 1923-10-01), \emph{Politiker, Publizist, Rabbiner}|pw}}{\lemma{\textnormal{\emph{Bloch}}}\Cendnote{\textnormal{Joseph Samuel Bloch\pwindex{Bloch, Joseph Samuel 1850-11-20 – 1923-10-01@\textsc{Bloch, Joseph Samuel} (1850-11-20 – 1923-10-01), \emph{Politiker, Publizist, Rabbiner}|pwk} trat als
                     Abgeordneter im \emph{Reichsrat}\orgindex{Reichsrat@Reichsrat|pwk} engagiert gegen
                     antisemitische Verleumdungen auf.}}}\label{K_L02769-2h}} und der Jud’, der unten »handel\textcolor{gray}{n}« ſchreit – ein Volk! Das
               iſt echt \textsc{Herzl\pwindex{Herzl, Theodor 1860-05-02 – 1904-07-03@\textsc{Herzl, Theodor} (1860-05-02 – 1904-07-03), \emph{Schriftsteller, Journalist}|pw}}. So hat er auch die fran\oindex{Frankreich@\textbf{Frankreich}|pwv}zöſiſchen Dinge angeſchaut u. immer unrichtig geſehen. Für mich gibt es
               eben nur eine Löſung der Judenfrage: daß die Juden ſchließlich {\pb}Alle Chriſten werden. Jeſus\pwindex{Jesus 7–4 v. u. Z. – 30/31@\textsc{Jesus} (7–4 v. u. Z. – 30/31), \emph{Prediger}|pw} iſt mir doch der ſympathiſcheſte Jude und ich will gern
               zu ſeinen Jüngern zählen{\dotsfive}\pend
           \pstart
           Mein Onkel\pwindex{Mamroth, Fedor 21.02.1851 – 25.06.1907@\textsc{Mamroth, Fedor} (21.02.1851 – 25.06.1907), \emph{Journalist, Kritiker}|pwv} hat nett über »\textsc{Anatol\pwindex{Schnitzler, Arthur 15.05.1862 – 21.10.1931@\textsc{Schnitzler, Arthur} (15.05.1862 – 21.10.1931), \emph{Schriftsteller, Mediziner}!Anatol1892-10-29@\strich\emph{Anatol} {[}1892-10-29{]}|pw}}« \label{K_L02769-3v}\edtext{geſchrieben\pwindex{Schauspielhaus. [Untreu und Abschiedssouper]1896-03-29@\emph{Schauspielhaus. [Untreu und Abschiedssouper]} {[}1896-03-29{]}|pwv}}{\lemma{\textnormal{\emph{geſchrieben}}}\Cendnote{\textnormal{m.\pwindex{Mamroth, Fedor 21.02.1851 – 25.06.1907@\textsc{Mamroth, Fedor} (21.02.1851 – 25.06.1907), \emph{Journalist, Kritiker}|pwk} [ = Fedor Mamroth\pwindex{Mamroth, Fedor 21.02.1851 – 25.06.1907@\textsc{Mamroth, Fedor} (21.02.1851 – 25.06.1907), \emph{Journalist, Kritiker}|pwk}]: \emph{Schauspielhaus}\pwindex{Schauspielhaus. [Untreu und Abschiedssouper]1896-03-29@\emph{Schauspielhaus. [Untreu und Abschiedssouper]} {[}1896-03-29{]}|pwk}.
                     In: \emph{Frankfurter Zeitung}\pwindex{?? Werk@Nicht ermittelte Verfasserinnen und Verfasser!Frankfurter Zeitung1856 – 1943@\emph{Frankfurter Zeitung} {[}1856 – 1943{]}|pwk}, Jg. 40, Nr. 89,
                     29. 3. 1896, Zweites Morgenblatt, S. 1. Mamroth\pwindex{Mamroth, Fedor 21.02.1851 – 25.06.1907@\textsc{Mamroth, Fedor} (21.02.1851 – 25.06.1907), \emph{Journalist, Kritiker}|pwk} besprach die
                  gemeinsame Aufführung von \emph{Untreu}\pwindex{\textcolor{red}{\textsuperscript{XXXX1 indx}}!Untreu. Komoedie in 3 Acten1895@\strich\emph{Untreu. Komödie in 3 Acten} {[}Übersetzung, 1895{]}|pwk} von Roberto Bracco\pwindex{Bracco, Roberto 10.11.1861 – 20.04.1943@\textsc{Bracco, Roberto} (10.11.1861 – 20.04.1943), \emph{Schriftsteller}|pwk} und Schnitzlers\pwindex{Schnitzler, Arthur 15.05.1862 – 21.10.1931@\textsc{Schnitzler, Arthur} (15.05.1862 – 21.10.1931), \emph{Schriftsteller, Mediziner}|pwk}{ }\emph{Abschiedssouper}\pwindex{Schnitzler, Arthur 15.05.1862 – 21.10.1931@\textsc{Schnitzler, Arthur} (15.05.1862 – 21.10.1931), \emph{Schriftsteller, Mediziner}!Abschiedssouper1892@\strich\emph{Abschiedssouper} {[}1892{]}|pwk} am \emph{Frankfurter
                     Schauspielhaus}\orgindex{Frankfurter Staedtisches Schauspielhaus@Frankfurter Städtisches Schauspielhaus|pwk} am 26. 3. 1896.}}}\label{K_L02769-3h}. Meine Mutter\pwindex{Goldmann, Clementine 1842-05-15 – 1924-02-24@\textsc{Goldmann, Clementine} (1842-05-15 – 1924-02-24)|pwv} ſendet noch folgende
               Ergänzungs-Kritik:\pend
           {\bigskip}\pstart
           \noindent{}{[}hs. Clementine Goldmann:{]} \label{T_L02769-1v}\edtext{Das »Abschieds« Souper\pwindex{Schnitzler, Arthur 15.05.1862 – 21.10.1931@\textsc{Schnitzler, Arthur} (15.05.1862 – 21.10.1931), \emph{Schriftsteller, Mediziner}!Abschiedssouper1892@\strich\emph{Abschiedssouper} {[}1892{]}|pw} von deinem Freunde hat uns ſehr gefallen – we{\geminationn} es auch für die ſtupiden Frankfurt\oindex{Frankfurt am Main@\textbf{Frankfurt am Main}|pw}er – viel zu fein war\textcolor{gray}{.}}{\lemma{\textnormal{\emph{Das … war.}}}\Cendnote{\textnormal{Ausschnitt aus einem Brief von Clementine Goldmann\pwindex{Goldmann, Clementine 1842-05-15 – 1924-02-24@\textsc{Goldmann, Clementine} (1842-05-15 – 1924-02-24)|pwk} auf einem eingeklebten
                  Zettel (blaue Tinte, deutsche Kurrentschrift)}}}\label{T_L02769-1h}\pend
           {\bigskip}\pstart
           \noindent{}{[}hs. Paul Goldmann:{]} Oſtern möchte ich nach Frankfurt\oindex{Frankfurt am Main@\textbf{Frankfurt am Main}|pw} fahren, weiß aber noch nicht, woher ich das Geld nehmen werde.
               Aber ich bin todt gearbeitet und habe ein {\pb}heftiges
               Bedürfniß nach ein paar Ruhetagen. Mit meinen Augen geht es ſchlecht, ſie wollen
               nicht mehr mit, und ich habe große Sorgen.\pend
           \pstart
           Vielleicht ſchreibe ich Dir den langen Brief doch noch vor den Feiertagen. Wenn
               nicht: fröhliche Oſtern.\pend
           \pstart
           Grüß’ Dich Gott, mein lieber Freund{\\[\baselineskip]}Dein {\\[\baselineskip]}\spacefill\mbox{Paul Goldmann.}\pend
           \leftskip=0em{}\pstart
           \noindent{}Der \label{K_L02769-4v}\edtext{Artikel\pwindex{Hofmannsthal, Hugo von 1874-02-01 – 1929-07-15@\textsc{Hofmannsthal, Hugo von} (1874-02-01 – 1929-07-15), \emph{Schriftsteller}!Gedichte von Stefan George1896-03-21@\strich\emph{Gedichte von Stefan George} {[}1896-03-21{]}|pwv}}{\lemma{\textnormal{\emph{Artikel}}}\Cendnote{\textnormal{Hugo von Hofmannsthal\pwindex{Hofmannsthal, Hugo von 1874-02-01 – 1929-07-15@\textsc{Hofmannsthal, Hugo von} (1874-02-01 – 1929-07-15), \emph{Schriftsteller}|pwk}: \emph{Gedichte von Stefan George}\pwindex{Hofmannsthal, Hugo von 1874-02-01 – 1929-07-15@\textsc{Hofmannsthal, Hugo von} (1874-02-01 – 1929-07-15), \emph{Schriftsteller}!Gedichte von Stefan George1896-03-21@\strich\emph{Gedichte von Stefan George} {[}1896-03-21{]}|pwk}. In: \emph{Die Zeit}\pwindex{Zeit. Wiener Wochenschrift1894 – 1904@\emph{Die Zeit. Wiener Wochenschrift} {[}1894 – 1904{]}|pwk}, Bd. 6, Nr. 77, 21. 3. 1896, S. 189–191.}}}\label{K_L02769-4h} des
                  kleinen \textsc{Loris\pwindex{Hofmannsthal, Hugo von 1874-02-01 – 1929-07-15@\textsc{Hofmannsthal, Hugo von} (1874-02-01 – 1929-07-15), \emph{Schriftsteller}|pw}} in der »Zeit\pwindex{Zeit. Wiener Wochenschrift1894 – 1904@\emph{Die Zeit. Wiener Wochenschrift} {[}1894 – 1904{]}|pw}« über \textsc{Stefan Georges\pwindex{George, Stefan 17.07.1868 – 04.12.1933@\textsc{George, Stefan} (17.07.1868 – 04.12.1933), \emph{Schriftsteller, Übersetzer}|pw}} hat mich einfach empört. \textsc{Stefan Georges\pwindex{George, Stefan 17.07.1868 – 04.12.1933@\textsc{George, Stefan} (17.07.1868 – 04.12.1933), \emph{Schriftsteller, Übersetzer}|pw}} iſt eine prätentiöſe Talentloſigkeit, und der Artikel\pwindex{Hofmannsthal, Hugo von 1874-02-01 – 1929-07-15@\textsc{Hofmannsthal, Hugo von} (1874-02-01 – 1929-07-15), \emph{Schriftsteller}!Gedichte von Stefan George1896-03-21@\strich\emph{Gedichte von Stefan George} {[}1896-03-21{]}|pwv}, abgeſehen von dem falſchen
                  Urtheil, iſt in einem unerhört ſchwülſtigen u. manierirten Styl geſchrieben. Ein
                  zweiter \textsc{Hermann Bahr\pwindex{Bahr, Hermann 19.07.1863 – 15.01.1934@\textsc{Bahr, Hermann} (19.07.1863 – 15.01.1934), \emph{Schriftsteller, Kritiker}|pw}}!\pend
           \pstart
           {\pb}\label{T_L02769-2v}\edtext{\uline{Gruß an \textsc{Richard\pwindex{Beer-Hofmann, Richard 1866-07-11 – 1945-09-26@\textsc{Beer-Hofmann, Richard} (1866-07-11 – 1945-09-26), \emph{Schriftsteller}|pw}}!}}{\lemma{\textnormal{\emph{Gruß an Richard!}}}\Cendnote{\textnormal{kopfüber am oberen Rand der ersten
                     Seite}}}\label{T_L02769-2h}\pend
           
         
         \endnumbering\mylabel{h}\end{ledgroupsized}  \newcommand{\dateiname}{L02769}\newcommand{\titel}{Paul Goldmann an Arthur Schnitzler, 1. 4. [1896]}\newcommand{\editorInnen}{Martin Anton Müller und Laura Untner}%% latex-leseansicht-abspann.tex
%% Abspann für die Leseansicht.
%% Der Schalter \ifkorrekturansicht ist bereits durch den Vorspann gesetzt.

%% latex-abspann.tex
%% Gemeinsamer Abspann für Korrekturansicht und Leseansicht.
%% Setzt den Schalter \ifkorrekturansicht voraus (gesetzt in den
%% einbindenden Dateien latex-korrekturansicht-abspann.tex bzw.
%% latex-leseansicht-abspann.tex).
%% ---------------------------------------------------------------

\normalsize

% Das esempio-Environment wird nur in der Leseansicht benötigt
\ifkorrekturansicht\else
\newenvironment{esempio}[3]%
{
    \vspace{1.5ex}
    \rlap{\underline{#1}}
    \par
    \setlength{\parindent}{0cm}
    \nopagebreak
    \leftskip=#2cm
    \rightskip=#3cm
}
{
    \par
}
\fi

\doendnotes{C}
\bigskip
\vfill

\clearpage

\footnotesize

\ifkorrekturansicht
  \lohead{\textsc{register}}
\fi

% theindex-Environment neu definieren ohne reledmac
\makeatletter
\renewenvironment{theindex}{%
  \ifkorrekturansicht
    \section*{\indexname}%
  \else
    \subsubsection*{Index der erwähnten Entitäten}%
  \fi
  \setlength{\parindent}{0pt}%
  \setlength{\parskip}{0pt plus 0.3pt}%
  \let\item\@idxitem
}{%
  \ifkorrekturansicht\clearpage\fi
}
\makeatother

\IfFileExists{\jobname-pw.ind}{\input{\jobname-pw.ind}}{}

% Quellenangabe nur in der Leseansicht
\ifkorrekturansicht\else
% Fallback-Definitionen, falls die .tex-Datei \titel etc. nicht gesetzt hat
\providecommand{\titel}{}
\providecommand{\editorInnen}{}
\providecommand{\dateiname}{\jobname}

\vspace{3cm}

\vfill

\footnotesize
\textsc{Quelle}: \titel. Herausgegeben von {\editorInnen}. In: \emph{Arthur Schnitzler: Briefwechsel mit Autorinnen und Autoren}.
 Digitale Edition, https://schnitzler-briefe.acdh.oeaw.ac.at/{\dateiname}.html (Stand \today)
\fi

\end{document}


      