%% latex-korrekturansicht-vorspann.tex
%% Vorspann für die Korrekturansicht.
%% Lädt die gemeinsame Datei latex-vorspann.tex mit gesetztem Schalter.

\newif\ifkorrekturansicht
\korrekturansichttrue

\input{../tex-inputs/latex-vorspann}


\section[Arthur Schnitzler an Richard Beer-Hofmann, 29. 1. 1921]{L02359 Arthur Schnitzler an Richard Beer-Hofmann, 29. 1. 1921}
\nopagebreak\mylabel{L02359v}
\rehead{ }\normalsize\beginnumbering\briefempfaengerindex{Beer-Hofmann, Richard@\textsc{Beer-Hofmann, Richard}!zzzSchnitzler, Arthur@\emph{von Arthur Schnitzler}!1921-01-291@{29. 1. 1921}|(be}
\toendnotes[C]{\smallbreak\pagebreak[2]}\Standort{CUL, Schnitzler, B 8.1, S. 155.}
\physDesc{Visitenkarte, maschinenschriftliche Abschrift1 Blatt, 1 Seite, 48 Zeichen
\newline{}Schreibmaschine
\newline{}Ordnung: von unbekannter Hand als Briefnummer »349«
                                 gekennzeichnet }
\buchAbdrucke{\weitereDrucke{Arthur Schnitzler, Richard Beer-Hofmann: \emph{Briefwechsel 1891–1931}. Wien, Zürich: \emph{Europaverlag} 1992, S. 228.} }\toendnotes[C]{\smallbreak}
\pstart
           \raggedleft{}{\pb}29. 1. 1921. \pend
           \vspace{0.5em}
\pstart
           (\strikeout{K} Visitekarte: \label{K_L02359-1v}\edtext{General-Probe}{\lemma{\textnormal{\emph{General-Probe}}}\Cendnote{\textnormal{am selben Tag, am 29. 1. 1921}}}\label{K_L02359-1}{ }Reigen\pwindex{Reigen. Zehn Dialoge@\emph{Reigen. Zehn Dialoge}|pw}.)\pend
           \selectlanguage{ngerman}\endnumbering\briefempfaengerindex{Beer-Hofmann, Richard@\textsc{Beer-Hofmann, Richard}!zzzSchnitzler, Arthur@\emph{von Arthur Schnitzler}!1921-01-291@{29. 1. 1921}|)be}\mylabel{L02359h}  \normalsize

\doendnotes{C}
\bigskip
\vfill

\clearpage

\footnotesize

\lohead{\textsc{register}}

% Definiere theindex-Environment komplett neu ohne reledmac
\makeatletter
\renewenvironment{theindex}{%
  \section*{\indexname}%
  \setlength{\parindent}{0pt}%
  \setlength{\parskip}{0pt plus 0.3pt}%
  \let\item\@idxitem
}{%
  \clearpage
}
\makeatother

\IfFileExists{\jobname-pw.ind}{\input{\jobname-pw.ind}}{}

\end{document}

      