%% latex-leseansicht-vorspann.tex
%% Vorspann für die Leseansicht.
%% Lädt die gemeinsame Datei latex-vorspann.tex mit nicht gesetztem Schalter.

\newif\ifkorrekturansicht
\korrekturansichtfalse

\input{../tex-inputs/latex-vorspann}


\section[Hugo von Hofmannsthal an Arthur Schnitzler, 21. [8. 1895]]{L00476 Hugo von Hofmannsthal an Arthur Schnitzler, 21. [8. 1895]}
\nopagebreak\mylabel{L00476v}
\rehead{ }\normalsize\beginnumbering\briefempfaengerindex{Schnitzler, Arthur@\textsc{Schnitzler, Arthur}!zzzHofmannsthal, Hugo von@\emph{von Hugo von Hofmannsthal}!1895-08-211@{21. [8. 1895]}|(be}
\toendnotes[C]{\smallbreak\pagebreak[2]}
\correspDesc{Versand  durch Hugo von Hofmannsthal am 21. [8. 1895] in Znojmo
\newline{}Erhalt  durch Arthur Schnitzler im Zeitraum [22. 8. 1895
                  – 26. 8. 1895?] in Wien}\toendnotes[C]{\smallbreak}
\Standort{CUL, Schnitzler, B 43.}
\physDesc{Brief, 1 Blatt, 4 Seiten, 1578 Zeichen
\newline{}Handschrift: Bleistift, deutsche Kurrent
\newline{}Schnitzler: mit Bleistift das Datum vervollständigt: »8. 95« und nummeriert: »75« }
\buchAbdrucke{\weitereDrucke{1) Hugo von Hofmannsthal: \emph{Briefe. 1890–1901}. Berlin: \emph{S. Fischer} 1935, S. 174–175.} \weitereDrucke{2) Hugo von Hofmannsthal, Arthur Schnitzler: \emph{Briefwechsel}. Herausgegeben von Therese Nickl und Heinrich Schnitzler. Frankfurt am Main: \emph{S. Fischer} 1964, S. 60–61.} }\toendnotes[C]{\smallbreak}
\pstart
           \raggedleft{}{\pb}Quartier zu Klein Teſſwitz\oindex{Dobšice u Znojma@\textbf{Dobšice u Znojma}, \emph{Bezirk}|pw} bei Znaim\oindex{Znaim@\textbf{Znaim}, \emph{Hauptstadt}|pw},{\\}Mittwoch 21\textsuperscript{ten}\pend
           \vspace{0.5em}
\pstart
           Es freut mich herzlich, Sie zufrieden zu wiſſen und von guten und geſcheiten Menſchen
               umgeben zu denken. Unſer Goldmann\pwindex{Goldmann, Paul 31.\,1.\,1865 Breslau – 25.\,9.\,1935 Wien@\textsc{Goldmann, Paul} (31.\,1.\,1865 Breslau – 25.\,9.\,1935 Wien), \emph{Schriftsteller, Journalist}|pw}, der im
               Journalismus lebt und{ }ſich{ }ſo völlig vor \textsc{\label{K_L00476-1v}\edtext{mesquinerie}{\lemma{\textnormal{\emph{mesquinerie}}}\Cendnote{\textnormal{Knausrigkeit}}}\label{K_L00476-1}} bewahrt hat, und Frau \textsc{D\textsuperscript{r}{ }Salomé\pwindex{Andreas-Salomé, Lou 12.\,2.\,1861 Sankt Petersburg – 5.\,2.\,1937 Göttingen@\textsc{Andreas-Salomé, Lou} (12.\,2.\,1861 Sankt Petersburg – 5.\,2.\,1937 Göttingen), \emph{Schriftstellerin}|pw}}{ }ſind ganz die Atmoſphäre, worin einem die
               Vermuthung von der Jugend der Seele glaubhaft wird. Ich bin, in gewiſſem Sinn,
               mutterſeelenallein, und {\pb}doch{ }ſo
               montiert, daſs ich mich manchmal gewaltſam zwingen muſs, an die Realität zu glauben.
               Mir iſt, wie einem der in der tiefen{ }ſtillen Kajüte eines Schiffes dem{ }ſchönſten Land
               langſam zufährt.\pend
           
\pstart
           Es{ }ſind wundervolle Sommertage. Ich wohne in einem kühlen niedrigen Bauernzimmer,
               hinter einem großen Birnbaum. Gegenüber iſt ein zehnjähriges Mädel, die doch eine
               Frau iſt, und ihr eigenes Kind, ihre eigene Mutter iſt. Ich habe den »Faust\pwindex{\textcolor{red}{\textsuperscript{XXXX indx1}}!Faust. Eine Tragödie@\strich\emph{Faust. Eine Tragödie}|pw}« mit und die Wanderjahre\pwindex{\textcolor{red}{\textsuperscript{XXXX indx1}}!Wilhelm Meisters Wanderjahre@\strich\emph{Wilhelm Meisters Wanderjahre}|pw}. Ich weiß von meinem {\pb}wirklichen Leben und bin doch
               unendlich weit davon.\pend
           
\pstart
           Die friſchen Birnen{ }ſind ganz warm von der gedämpften Sonne, die im Wipfel des
               Birnbaums iſt. Von der Helena\pwindex{\textcolor{red}{\textsuperscript{XXXX indx1}}!Faust. Eine Tragödie@\strich\emph{Faust. Eine Tragödie}|pw} les’ ich dieſen
               Vers: »\label{K_L00476-2v}\edtext{Wer{ }ſie verſteht, der darf{ }ſie
                  nicht entbehren!\pwindex{\textcolor{red}{\textsuperscript{XXXX indx1}}!Faust. Eine Tragödie@\strich\emph{Faust. Eine Tragödie}|pwv}}{\lemma{\textnormal{\emph{Wer … entbehren!}}}\Cendnote{\textnormal{richtig: »Wer sie erkennt der
                     darf sie nicht entbehren.« (II. Teil, Ende des 1.
                  Akts).}}}\label{K_L00476-2}« Heute abend werd ich nach Znaim\oindex{Znaim@\textbf{Znaim}, \emph{Hauptstadt}|pw} hineinfahren, wo Muſik von den Deutſchmeiſtern\orgindex{Hoch- und Deutschmeisterkapelle@Hoch- und Deutschmeisterkapelle|pw} iſt und in der kühlen{ }ſternhellen Nacht zurückfahren, ein
               biſſel vom weißen Wein montiert, auf einem hohen Wagen, der{ }ſehr {\pb}unſicher fährt, mit meinem
               Rittmeiſter und meinem hübſchen und indolent-graciöſen Lieutenant, die in der Nacht{ }ſehr wenig und{ }ſehr lieb reden werden. Begreifen Sie daſs ich zufrieden bin?\pend
           
\pstart
           Leben Sie wohl und denken mit Ihren Freunden freundlich an mich. Adieu.\pend
           
\pstart
           Der Ihre{\\[\baselineskip]}\spacefill\mbox{Hugo.}\pend
           \leftskip=0em{}\selectlanguage{ngerman}\endnumbering\briefempfaengerindex{Schnitzler, Arthur@\textsc{Schnitzler, Arthur}!zzzHofmannsthal, Hugo von@\emph{von Hugo von Hofmannsthal}!1895-08-211@{21. [8. 1895]}|)be}\mylabel{L00476h}  \newcommand{\dateiname}{L00476}\newcommand{\titel}{Hugo von Hofmannsthal an Arthur Schnitzler, 21. [8. 1895]}\newcommand{\editorInnen}{Martin Anton Müller und Gerd-Hermann Susen}%% latex-leseansicht-abspann.tex
%% Abspann für die Leseansicht.
%% Der Schalter \ifkorrekturansicht ist bereits durch den Vorspann gesetzt.

%% latex-abspann.tex
%% Gemeinsamer Abspann für Korrekturansicht und Leseansicht.
%% Setzt den Schalter \ifkorrekturansicht voraus (gesetzt in den
%% einbindenden Dateien latex-korrekturansicht-abspann.tex bzw.
%% latex-leseansicht-abspann.tex).
%% ---------------------------------------------------------------

\normalsize

% Das esempio-Environment wird nur in der Leseansicht benötigt
\ifkorrekturansicht\else
\newenvironment{esempio}[3]%
{
    \vspace{1.5ex}
    \rlap{\underline{#1}}
    \par
    \setlength{\parindent}{0cm}
    \nopagebreak
    \leftskip=#2cm
    \rightskip=#3cm
}
{
    \par
}
\fi

\doendnotes{C}
\bigskip
\vfill

\clearpage

\footnotesize

\ifkorrekturansicht
  \lohead{\textsc{register}}
\fi

% theindex-Environment neu definieren ohne reledmac
\makeatletter
\renewenvironment{theindex}{%
  \ifkorrekturansicht
    \section*{\indexname}%
  \else
    \subsubsection*{Index der erwähnten Entitäten}%
  \fi
  \setlength{\parindent}{0pt}%
  \setlength{\parskip}{0pt plus 0.3pt}%
  \let\item\@idxitem
}{%
  \ifkorrekturansicht\clearpage\fi
}
\makeatother

\IfFileExists{\jobname-pw.ind}{\input{\jobname-pw.ind}}{}

% Quellenangabe nur in der Leseansicht
\ifkorrekturansicht\else
% Fallback-Definitionen, falls die .tex-Datei \titel etc. nicht gesetzt hat
\providecommand{\titel}{}
\providecommand{\editorInnen}{}
\providecommand{\dateiname}{\jobname}

\vspace{3cm}

\vfill

\footnotesize
\textsc{Quelle}: \titel. Herausgegeben von {\editorInnen}. In: \emph{Arthur Schnitzler: Briefwechsel mit Autorinnen und Autoren}.
 Digitale Edition, https://schnitzler-briefe.acdh.oeaw.ac.at/{\dateiname}.html (Stand \today)
\fi

\end{document}


