\input{../tex-inputs/latex-pdf-vorspann}
\begin{center}
            \textcolor{red}{ENTWURF. ENTZIFFERUNG NOCH NICHT KORREKTURGELESEN}
                      \end{center}
            
               \section[Arthur Schnitzler an Paul Goldmann, 19. 3. 1899]{ Arthur Schnitzler an Paul Goldmann, 19. 3. 1899}\nopagebreak\mylabel{v}\rehead{ }\begin{ledgroupsized}[t]{13cm}\normalsize\beginnumbering\briefempfaengerindex{Goldmann, Paul@\textsc{Goldmann, Paul}!zzzSchnitzler, Arthur@\emph{von Arthur Schnitzler}!1899-03-191@{19. 3. 1899}|(be} \toendnotes[C]{\smallbreak\pagebreak[2]} \Standort{Wienbibliothek im Rathaus, ZPH 1681, Archivbox 11, 2.4.15.}
\physDesc{Telegramm, Entwurf
\newline{}Handschrift: 1) Bleistift, lateinische Kurrent\hspace{1em}2) Bleistift, deutsche Kurrent (\noindent{}Fußnote)\hspace{1em}}\toendnotes[C]{\smallbreak}\pstart
           \noindent{}{\pb}Doctor Paul Goldmann\hfill Frankfurt am Main\oindex{Frankfurt am Main@\textbf{Frankfurt am Main}|pw}\pend
           \pstart
           Frankfurt\oindex{Frankfurt am Main@\textbf{Frankfurt am Main}|pw}er \substVorne{}\textsuperscript{Rossert\oindex{Rossertstrasse@\textbf{Rossertstraße}|pw}}{\allowbreak}\substDazwischen{}Zeitung\orgindex{Frankfurter Zeitung@Frankfurter Zeitung|pw}\substHinten{}{ }\strikeout{Hotel Central\oindex{Central-Hotel@\textbf{Central-Hotel}|pw}}\footnote{\noindent{}ſelbſt geſtrichen. Schnitzler}\pend
           \pstart
           Mizi\pwindex{Reinhard, Marie 13.03.1871 – 18.03.1899@\textsc{Reinhard, Marie} (13.03.1871 – 18.03.1899), \emph{Gesangspädagogin}|pw} nach zweitägigem Krankenlager gestern{ }Abend an \label{K_L02635-1v}\edtext{Perforationsperitonitis}{\lemma{\textnormal{\emph{Perforationsperitonitis}}}\Cendnote{\textnormal{Bauchfellentzündung, ausgelöst durch eine Durchlöcherung, in Folge derer
                  Flüssigkeit in die Bauchdecke kommt}}}\label{K_L02635-1h} gestorben.\pend
           \pstart
           Kann heut \label{K_L02635-11v}\edtext{nicht mehr schreiben}{\lemma{\textnormal{\emph{nicht mehr schreiben}}}\Cendnote{\textnormal{Die Überlieferung im Nachlass Salten\pwindex{Salten, Felix 06.09.1869 – 08.10.1945@\textsc{Salten, Felix} (06.09.1869 – 08.10.1945), \emph{Schriftsteller, Journalist}|pwk} deutet darauf hin, dass dieser von Schnitzler\pwindex{Schnitzler, Arthur 15.05.1862 – 21.10.1931@\textsc{Schnitzler, Arthur} (15.05.1862 – 21.10.1931), \emph{Schriftsteller, Mediziner}|pwk} beauftragt war, das Telegramm
                  abzusenden.}}}\label{K_L02635-11h}. Alles alles scheint zu Ende\pend
           \pstart \spacefill\mbox{Arthur}\pend{}\endnumbering\briefempfaengerindex{Goldmann, Paul@\textsc{Goldmann, Paul}!zzzSchnitzler, Arthur@\emph{von Arthur Schnitzler}!1899-03-191@{19. 3. 1899}|)be}\mylabel{h}\end{ledgroupsized}  \newcommand{\dateiname}{L02635}\newcommand{\titel}{Arthur Schnitzler an Paul Goldmann, 19. 3. 1899}\newcommand{\editorInnen}{Martin Anton Müller und Laura Untner}\input{../tex-inputs/latex-pdf-abspann}
      