%% latex-korrekturansicht-vorspann.tex
%% Vorspann für die Korrekturansicht.
%% Lädt die gemeinsame Datei latex-vorspann.tex mit gesetztem Schalter.

\newif\ifkorrekturansicht
\korrekturansichttrue

\input{../tex-inputs/latex-vorspann}


\section[Arthur Schnitzler an Paul Goldmann, 19. 3. 1899]{L02635 Arthur Schnitzler an Paul Goldmann, 19. 3. 1899}
\nopagebreak\mylabel{L02635v}
\rehead{ }\normalsize\beginnumbering\briefempfaengerindex{Goldmann, Paul@\textsc{Goldmann, Paul}!zzzSchnitzler, Arthur@\emph{von Arthur Schnitzler}!1899-03-191@{19. 3. 1899}|(be}
\toendnotes[C]{\smallbreak\pagebreak[2]}\Standort{Wienbibliothek im Rathaus, ZPH 1681, Archivbox 11, 2.4.15.}
\physDesc{Telegrammentwurf, 217 Zeichen
\newline{}Handschrift: 1) Bleistift, lateinische Kurrent\hspace{1em}2) Bleistift, deutsche Kurrent (\noindent{}Fußnote)\hspace{1em}}\toendnotes[C]{\smallbreak}
\pstart
           {\pb}Doctor Paul Goldmann\hfill Frankfurt a/M\oindex{Frankfurt am Main@\textbf{Frankfurt am Main}, \emph{P.PPLA3}|pw}\pend
           
\pstart
           Frankfurt\oindex{Frankfurt am Main@\textbf{Frankfurt am Main}, \emph{P.PPLA3}|pw}er \substVorne{}\textsuperscript{Rossert\oindex{Rossertstrasse@\textbf{Rossertstraße}, \emph{Straße (K.STR)}|pw}}\substDazwischen{}Zeitung\orgindex{Frankfurter Zeitung@Frankfurter Zeitung|pw}\substHinten{} – \strikeout{Hotel Central\oindex{Central-Hotel@\textbf{Central-Hotel}, \emph{Hotel (K.HTL)}|pw}}\noindent{}ſelbſt geſtrichen. Schnitzler\pend
           \vspace{0.5em}
\pstart
           Mizi\pwindex{Reinhard, Marie 1871-03-13 – 1899-03-18@\textsc{Reinhard, Marie} (1871-03-13 – 1899-03-18), \emph{Gesangspädagoge/Gesangspädagogin}|pw} nach zweitägigem Krankenlager gestern{ }Abend an \label{K_L02635-1v}\edtext{Perforationsperitonitis}{\lemma{\textnormal{\emph{Perforationsperitonitis}}}\Cendnote{\textnormal{Bauchfellentzündung, ausgelöst durch eine Durchlöcherung, in Folge derer
                  Flüssigkeit in die Bauchdecke kommt}}}\label{K_L02635-1} gestorben.\pend
           
\pstart
           Kann heut \label{K_L02635-2v}\edtext{nicht mehr schreiben}{\lemma{\textnormal{\emph{nicht mehr schreiben}}}\Cendnote{\textnormal{Die Überlieferung im Nachlass Salten\pwindex{Salten, Felix 06.09.1869 – 08.10.1945@\textsc{Salten, Felix} (06.09.1869 – 08.10.1945), \emph{Schriftsteller/Schriftstellerin, Journalist/Journalistin, Chefredakteur/Chefredakteurin}|pwk} deutet darauf hin, dass dieser von Schnitzler beauftragt war, das Telegramm
                  abzusenden.}}}\label{K_L02635-2}. Alles alles scheint zu Ende\pend
           \pstart \spacefill\mbox{Arthur}\pend{}\selectlanguage{ngerman}\endnumbering\briefempfaengerindex{Goldmann, Paul@\textsc{Goldmann, Paul}!zzzSchnitzler, Arthur@\emph{von Arthur Schnitzler}!1899-03-191@{19. 3. 1899}|)be}\mylabel{L02635h}  \normalsize

\doendnotes{C}
\bigskip
\vfill

\clearpage

\footnotesize

\lohead{\textsc{register}}

% Definiere theindex-Environment komplett neu ohne reledmac
\makeatletter
\renewenvironment{theindex}{%
  \section*{\indexname}%
  \setlength{\parindent}{0pt}%
  \setlength{\parskip}{0pt plus 0.3pt}%
  \let\item\@idxitem
}{%
  \clearpage
}
\makeatother

\IfFileExists{\jobname-pw.ind}{\input{\jobname-pw.ind}}{}

\end{document}

      