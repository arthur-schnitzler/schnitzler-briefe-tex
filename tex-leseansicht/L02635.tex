%% latex-leseansicht-vorspann.tex
%% Vorspann für die Leseansicht.
%% Lädt die gemeinsame Datei latex-vorspann.tex mit nicht gesetztem Schalter.

\newif\ifkorrekturansicht
\korrekturansichtfalse

\input{../tex-inputs/latex-vorspann}


\section[Arthur Schnitzler an Paul Goldmann, 19. 3. 1899]{L02635 Arthur Schnitzler an Paul Goldmann, 19. 3. 1899}
\nopagebreak\mylabel{L02635v}
\rehead{ }\normalsize\beginnumbering\briefempfaengerindex{Goldmann, Paul@\textsc{Goldmann, Paul}!zzzSchnitzler, Arthur@\emph{von Arthur Schnitzler}!1899-03-191@{19. 3. 1899}|(be}
\toendnotes[C]{\smallbreak\pagebreak[2]}
\correspDesc{Versand  durch Arthur Schnitzler am 19. 3. 1899 in Wien
\newline{}Erhalt  durch Paul Goldmann am 19. 3. 1899 in Frankfurt am Main}\toendnotes[C]{\smallbreak}
\Standort{Wienbibliothek im Rathaus, ZPH 1681, Archivbox 11, 2.4.15.}
\physDesc{Telegrammentwurf, 217 Zeichen
\newline{}Handschrift: 1) Bleistift, lateinische Kurrent\hspace{1em}2) Bleistift, deutsche Kurrent (\noindent{}Fußnote)\hspace{1em}}\Standort{DLA, A:Schnitzler, HS.NZ85.1.1897.}
\physDesc{Telegrammentwurf, fotografische Vervielfältigung, 1 Blatt, 1 Seite
\newline{}Handschrift: 1) Bleistift, lateinische Kurrent\hspace{1em}2) Bleistift, deutsche Kurrent (\noindent{}Fußnote)\hspace{1em}
\newline{}Ordnung: mit roter Tinte Vermerk: »ohne Datum (1899?)«
                               }\toendnotes[C]{\smallbreak}
\pstart
           {\pb}Doctor Paul Goldmann\hfill Frankfurt a/M\oindex{Frankfurt am Main@\textbf{Frankfurt am Main}, \emph{Hauptstadt}|pw}\pend
           
\pstart
           Frankfurt\oindex{Frankfurt am Main@\textbf{Frankfurt am Main}, \emph{Hauptstadt}|pw}er \substVorne{}\textsuperscript{Rossert\oindex{Rossertstraße@\textbf{Rossertstraße}, \emph{Straße}|pw}}\substDazwischen{}Zeitung\orgindex{Frankfurter Zeitung@Frankfurter Zeitung|pw}\substHinten{} – \strikeout{Hotel Central\oindex{Central-Hotel@\textbf{Central-Hotel}, \emph{Hotel}|pw}}\footnote{\noindent{}ſelbſt geſtrichen. Schnitzler}\pend
           \vspace{0.5em}
\pstart
           Mizi\pwindex{Reinhard, Marie 13.\,3.\,1871 Wien – 18.\,3.\,1899 ebd.@\textsc{Reinhard, Marie} (13.\,3.\,1871 Wien – 18.\,3.\,1899 ebd.), \emph{Gesangspädagogin}|pw} nach zweitägigem Krankenlager gestern{ }Abend an \label{K_L02635-1v}\edtext{Perforationsperitonitis}{\lemma{\textnormal{\emph{Perforationsperitonitis}}}\Cendnote{\textnormal{Bauchfellentzündung, ausgelöst durch eine Durchlöcherung, in Folge derer
                  Flüssigkeit in die Bauchdecke kommt}}}\label{K_L02635-1} gestorben.\pend
           
\pstart
           Kann heut \label{K_L02635-2v}\edtext{nicht mehr schreiben}{\lemma{\textnormal{\emph{nicht mehr schreiben}}}\Cendnote{\textnormal{Die Überlieferung im Nachlass Salten\pwindex{Salten, Felix 6.\,9.\,1869 Budapest – 8.\,10.\,1945 Zürich@\textsc{Salten, Felix} (6.\,9.\,1869 Budapest – 8.\,10.\,1945 Zürich), \emph{Schriftsteller, Journalist, Chefredakteur}|pwk} deutet darauf hin, dass dieser von Schnitzler beauftragt war, das Telegramm
                  abzusenden.}}}\label{K_L02635-2}. Alles alles scheint zu Ende\pend
           \pstart \spacefill\mbox{Arthur}\pend{}\selectlanguage{ngerman}\endnumbering\briefempfaengerindex{Goldmann, Paul@\textsc{Goldmann, Paul}!zzzSchnitzler, Arthur@\emph{von Arthur Schnitzler}!1899-03-191@{19. 3. 1899}|)be}\mylabel{L02635h}  \newcommand{\dateiname}{L02635}\newcommand{\titel}{Arthur Schnitzler an Paul Goldmann, 19. 3. 1899}\newcommand{\editorInnen}{Martin Anton Müller und Laura Untner}%% latex-leseansicht-abspann.tex
%% Abspann für die Leseansicht.
%% Der Schalter \ifkorrekturansicht ist bereits durch den Vorspann gesetzt.

%% latex-abspann.tex
%% Gemeinsamer Abspann für Korrekturansicht und Leseansicht.
%% Setzt den Schalter \ifkorrekturansicht voraus (gesetzt in den
%% einbindenden Dateien latex-korrekturansicht-abspann.tex bzw.
%% latex-leseansicht-abspann.tex).
%% ---------------------------------------------------------------

\normalsize

% Das esempio-Environment wird nur in der Leseansicht benötigt
\ifkorrekturansicht\else
\newenvironment{esempio}[3]%
{
    \vspace{1.5ex}
    \rlap{\underline{#1}}
    \par
    \setlength{\parindent}{0cm}
    \nopagebreak
    \leftskip=#2cm
    \rightskip=#3cm
}
{
    \par
}
\fi

\doendnotes{C}
\bigskip
\vfill

\clearpage

\footnotesize

\ifkorrekturansicht
  \lohead{\textsc{register}}
\fi

% theindex-Environment neu definieren ohne reledmac
\makeatletter
\renewenvironment{theindex}{%
  \ifkorrekturansicht
    \section*{\indexname}%
  \else
    \subsubsection*{Index der erwähnten Entitäten}%
  \fi
  \setlength{\parindent}{0pt}%
  \setlength{\parskip}{0pt plus 0.3pt}%
  \let\item\@idxitem
}{%
  \ifkorrekturansicht\clearpage\fi
}
\makeatother

\IfFileExists{\jobname-pw.ind}{\input{\jobname-pw.ind}}{}

% Quellenangabe nur in der Leseansicht
\ifkorrekturansicht\else
% Fallback-Definitionen, falls die .tex-Datei \titel etc. nicht gesetzt hat
\providecommand{\titel}{}
\providecommand{\editorInnen}{}
\providecommand{\dateiname}{\jobname}

\vspace{3cm}

\vfill

\footnotesize
\textsc{Quelle}: \titel. Herausgegeben von {\editorInnen}. In: \emph{Arthur Schnitzler: Briefwechsel mit Autorinnen und Autoren}.
 Digitale Edition, https://schnitzler-briefe.acdh.oeaw.ac.at/{\dateiname}.html (Stand \today)
\fi

\end{document}


