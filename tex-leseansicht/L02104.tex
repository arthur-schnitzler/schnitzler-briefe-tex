%% latex-korrekturansicht-vorspann.tex
%% Vorspann für die Korrekturansicht.
%% Lädt die gemeinsame Datei latex-vorspann.tex mit gesetztem Schalter.

\newif\ifkorrekturansicht
\korrekturansichttrue

\input{../tex-inputs/latex-vorspann}


\section[Richard und Paula Beer-Hofmann an Arthur Schnitzler, 29. 11. 1912]{L02104 Richard und Paula Beer-Hofmann an Arthur Schnitzler,
               29. 11. 1912}
\nopagebreak\mylabel{L02104v}
\rehead{ }\normalsize\beginnumbering\briefempfaengerindex{Schnitzler, Arthur@\textsc{Schnitzler, Arthur}!zzzBeer-Hofmann, Paula@\emph{von Paula Beer-Hofmann}!1912-11-291@{29. 11. 1912}|(be}\briefempfaengerindex{Schnitzler, Arthur@\textsc{Schnitzler, Arthur}!zzzBeer-Hofmann, Richard@\emph{von Richard Beer-Hofmann}!1912-11-291@{29. 11. 1912}|(be}
\toendnotes[C]{\smallbreak\pagebreak[2]}\Standort{CUL, Schnitzler, B 8.}
\physDesc{Telegramm, 142 Zeichen
\newline{}maschinell
\newline{}Schnitzler: mit Bleistift datiert: »29. 11.{ }8\textsuperscript{n}« 
\newline{}Ordnung: mit Bleistift von unbekannter Hand nummeriert:
                                    »250« }
\buchAbdrucke{\weitereDrucke{Arthur Schnitzler, Richard Beer-Hofmann: \emph{Briefwechsel 1891–1931}. Wien, Zürich: \emph{Europaverlag} 1992, S. 217.} }
\pstart
           \noindent{}{\pb}freuen uns herzlich mit ihnen des
               grossen erfolges hoffen dass bernhardi\pwindex{Professor Bernhardi. Komoedie in fuenf Akten@\emph{Professor Bernhardi. Komödie in fünf Akten}|pw}
               mindestens drei monate haft im kleinen theater\oindex{Kleines Theater@\textbf{Kleines Theater}, \emph{Theater (K.THE)}|pw}
               erhaelt –\pend
           \pstart \spacefill\mbox{richard und paula}\pend{}\selectlanguage{ngerman}\endnumbering\briefempfaengerindex{Schnitzler, Arthur@\textsc{Schnitzler, Arthur}!zzzBeer-Hofmann, Paula@\emph{von Paula Beer-Hofmann}!1912-11-291@{29. 11. 1912}|)be}\briefempfaengerindex{Schnitzler, Arthur@\textsc{Schnitzler, Arthur}!zzzBeer-Hofmann, Richard@\emph{von Richard Beer-Hofmann}!1912-11-291@{29. 11. 1912}|)be}\mylabel{L02104h}  \normalsize

\doendnotes{C}
\bigskip
\vfill

\clearpage

\footnotesize

\lohead{\textsc{register}}

% Definiere theindex-Environment komplett neu ohne reledmac
\makeatletter
\renewenvironment{theindex}{%
  \section*{\indexname}%
  \setlength{\parindent}{0pt}%
  \setlength{\parskip}{0pt plus 0.3pt}%
  \let\item\@idxitem
}{%
  \clearpage
}
\makeatother

\IfFileExists{\jobname-pw.ind}{\input{\jobname-pw.ind}}{}

\end{document}

      