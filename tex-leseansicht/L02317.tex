%% latex-korrekturansicht-vorspann.tex
%% Vorspann für die Korrekturansicht.
%% Lädt die gemeinsame Datei latex-vorspann.tex mit gesetztem Schalter.

\newif\ifkorrekturansicht
\korrekturansichttrue

\input{../tex-inputs/latex-vorspann}


\section[Hugo von Hofmannsthal an Arthur Schnitzler, 21. 12. {[}1918?{]}]{L02317 Hugo von Hofmannsthal an Arthur Schnitzler, 21. 12. {[}1918?{]}}
\nopagebreak\mylabel{L02317v}
\rehead{ }\normalsize\beginnumbering\briefempfaengerindex{Schnitzler, Arthur@\textsc{Schnitzler, Arthur}!zzzHofmannsthal, Hugo von@\emph{von Hugo von Hofmannsthal}!1918-12-211@{21. 12. {[}1918?{]}}|(be}
\toendnotes[C]{\smallbreak\pagebreak[2]}\Standort{CUL, Schnitzler, B 43.}
\physDesc{Brief, 1 Blatt, 4 Seiten, 1098 Zeichen
\newline{}Handschrift: schwarze Tinte, deutsche Kurrent
\newline{}Schnitzler: 1) mit Bleistift die Jahreszahl ergänzt: »19«  2) mit rotem Buntstift zwei Unterstreichungen
\newline{}Ordnung: 1) mit Bleistift von Frieda
                                    Pollak\pwindex{Pollak, Frieda 08.12.1881 – 13.07.1937@\textsc{Pollak, Frieda} (08.12.1881 – 13.07.1937), \emph{Sekretär/Sekretärin}|pw} (?) mit dem Buchstaben »A«
                                 (Abgeschrieben/Abschrift) gekennzeichnet  2) mit Bleistift von unbekannter Hand nummeriert: »\strikeout{357}« 3) mit Bleistift von unbekannter Hand nummeriert:
                                    »385«}
\buchAbdrucke{\weitereDrucke{Hugo von Hofmannsthal, Arthur Schnitzler: \emph{Briefwechsel}. Frankfurt am Main: \emph{S. Fischer} 1964, S. 228–229.} }\toendnotes[C]{\smallbreak}
\pstart
           \raggedleft{}{\pb}21. XII.\pend
           
\pstart{}mein lieber Arthur\pend\vspace{0.5em}
\pstart
           recht ſehr freu ich mich \label{K_L02317-1v}\edtext{heute
                  abend}{\lemma{\textnormal{\emph{heute
                  abend}}}\Cendnote{\textnormal{Obzwar von Schnitzler mit der Jahreszahl »19« versehen, dürfte der Brief bereits 1918 gelaufen sein. Am
                     21. 12. 1918 fand die Wien\oindex{Wien@\textbf{Wien}, \emph{A.ADM2}|pwk}er
                  Erstaufführung statt. Zwar wurde \emph{Professor
                     Bernhardi}\pwindex{Professor Bernhardi. Komoedie in fuenf Akten@\emph{Professor Bernhardi. Komödie in fünf Akten}|pwk} auch am 21. 12. 1919 gespielt, doch war der
                  Verfasser nur vor der Premiere an den Proben beteiligt.}}}\label{K_L02317-1} ein Stück\pwindex{Professor Bernhardi. Komoedie in fuenf Akten@\emph{Professor Bernhardi. Komödie in fünf Akten}|pwv} von Ihnen, eine Ihrer ſtärkſten u.
               glücklichſten Arbeiten wie ich glaube, ſpielen zu ſehen. Ein ſolcher Abend bindet,
               über den Abgrund des Geſchehens hinweg, die Jahre an die Jahre und erweckt ein kaum
               definierbares Gefühl: daſs ein Teil von uns doch all dieſem Geſchehen entrückt und
               von all dem {\pb}unberührbar iſt.\pend
           
\pstart
           Sehr lieb war’s mir auch den »\label{K_L02317-2v}\edtext{\textsc{Casanova}\pwindex{Schwestern oder Casanova in Spa. Lustspiel in Versen@\emph{Die Schwestern oder Casanova in Spa. Lustspiel in Versen}|pw}}{\lemma{\textnormal{\emph{Casanova}}}\Cendnote{\textnormal{\emph{Casanovas Heimkehr}\pwindex{Schwestern oder Casanova in Spa. Lustspiel in Versen@\emph{Die Schwestern oder Casanova in Spa. Lustspiel in Versen}|pwk} erschien im Dezember
                     1918. Es ist anzunehmen, dass Schnitzler{ }Hofmannsthal\pwindex{Hofmannsthal, Hugo von 1874-02-01 – 1929-07-15@\textsc{Hofmannsthal, Hugo von} (1874-02-01 – 1929-07-15), \emph{Schriftsteller/Schriftstellerin}|pwk} ein weiteres Exemplar widmete,
                  da das erste direkt aus dem Verlag kam, vgl. Hugo von Hofmannsthal an Arthur Schnitzler, [Anfang Dezember
               1918]. Das Exemplar ist nicht überliefert.}}}\label{K_L02317-2}« von Ihrer
               eigenen Hand und mit Ihrem Namenszug zu empfangen – ſo gibt es doch Dinge u. Bezüge
               die ſich nicht verändern.\pend
           
\pstart
           Sehr gern, lieber Arthur, möchte ich Sie aber doch wiederſehen. So unbequem es iſt,
               ich komme gerne {\pb}hinaus.
               Vormittags einmal – ich glaube, aus früheren Zeiten, das ſtört Sie nicht in der
               Arbeit.\pend
           
\pstart
           Ich ſchrieb Ihnen das vor ein oder zwei Wochen, damals waren aber noch die Proben\pwindex{Professor Bernhardi. Komoedie in fuenf Akten@\emph{Professor Bernhardi. Komödie in fünf Akten}|pwv} vor Ihnen ſo haben Sie
               mir wahrſcheinlich deswegen nicht geantwortet.\pend
           
\pstart
           Ich bin die Tage 28{ }29{ }30{ }31 in Wien\oindex{Wien@\textbf{Wien}, \emph{A.ADM2}|pw} zur Verfügung. Bitte
               ſchreiben {\pb}Sie auf einer Karte in
               die Stallburggaſſe\oindex{Stallburggasse@\textbf{Stallburggasse}, \emph{Straße (K.STR)}|pw}, an welchem von dieſen Tagen
               Sie mich ſehen wollen.\pend
           
\pstart
           Ich würde dann trachten 10 ½ draußen zu ſein.\pend
           
\pstart
           Herzlich Ihr{\\[\baselineskip]}\spacefill\mbox{Hugo.}\pend
           \leftskip=0em{}\selectlanguage{ngerman}\endnumbering\briefempfaengerindex{Schnitzler, Arthur@\textsc{Schnitzler, Arthur}!zzzHofmannsthal, Hugo von@\emph{von Hugo von Hofmannsthal}!1918-12-211@{21. 12. {[}1918?{]}}|)be}\mylabel{L02317h}  \normalsize

\doendnotes{C}
\bigskip
\vfill

\clearpage

\footnotesize

\lohead{\textsc{register}}

% Definiere theindex-Environment komplett neu ohne reledmac
\makeatletter
\renewenvironment{theindex}{%
  \section*{\indexname}%
  \setlength{\parindent}{0pt}%
  \setlength{\parskip}{0pt plus 0.3pt}%
  \let\item\@idxitem
}{%
  \clearpage
}
\makeatother

\IfFileExists{\jobname-pw.ind}{\input{\jobname-pw.ind}}{}

\end{document}

      