%% latex-leseansicht-vorspann.tex
%% Vorspann für die Leseansicht.
%% Lädt die gemeinsame Datei latex-vorspann.tex mit nicht gesetztem Schalter.

\newif\ifkorrekturansicht
\korrekturansichtfalse

\input{../tex-inputs/latex-vorspann}


         
         \renewcommand{\erwaehntePersonen}{Personen: Hugo von Hofmannsthal, Frieda Pollak}
         \renewcommand{\erwaehnteOrte}{Orte: Stallburggasse, Wien}
         \renewcommand{\erwaehnteWerke}{Werke: Die Schwestern oder Casanova in Spa. Lustspiel in Versen, Professor Bernhardi. Komödie in fünf Akten}
               \section[Hugo von Hofmannsthal an Arthur Schnitzler, 21. 12. {[}1918?{]}]{ Hugo von Hofmannsthal an Arthur Schnitzler, 21. 12. {[}1918?{]}}\nopagebreak\mylabel{v}\rehead{ }\begin{ledgroupsized}[t]{13cm}\normalsize\beginnumbering \toendnotes[C]{\smallbreak\pagebreak[2]} \Standort{CUL, Schnitzler, B 43.}
\physDesc{Brief, 1 Blatt, 4 Seiten, 1098 Zeichen
\newline{}Handschrift: schwarze Tinte, deutsche Kurrent
\newline{}Schnitzler: 1) mit Bleistift die Jahreszahl ergänzt: »19«  2) mit rotem Buntstift zwei Unterstreichungen
\newline{}Ordnung: 1) mit Bleistift von Frieda
                                    Pollak\pwindex{Pollak, Frieda 08.12.1881 – 13.07.1937@\textsc{Pollak, Frieda} (08.12.1881 – 13.07.1937), \emph{Sekretärin}|pw} (?) mit dem Buchstaben »A«
                                 (Abgeschrieben/Abschrift) gekennzeichnet  2) mit Bleistift von unbekannter Hand nummeriert: »\strikeout{357}« 3) mit Bleistift von unbekannter Hand nummeriert:
                                    »385«}\buchAbdrucke{\weitereDrucke{Hugo von Hofmannsthal, Arthur Schnitzler: \emph{Briefwechsel}. Hg. Therese Nickl und Heinrich Schnitzler. Frankfurt am Main: \emph{S. Fischer} 1964, S. 228–229.} }\toendnotes[C]{\smallbreak}\pstart
           \raggedleft{}{\pb}21. XII.\pend
           \pstart{}mein lieber Arthur\pend\pstart
           recht ſehr freu ich mich \label{K_L02317-1v}\edtext{heute
                  abend}{\lemma{\textnormal{\emph{heute
                  abend}}}\Cendnote{\textnormal{Obzwar von Schnitzler\pwindex{Schnitzler, Arthur 15.05.1862 – 21.10.1931@\textsc{Schnitzler, Arthur} (15.05.1862 – 21.10.1931), \emph{Schriftsteller, Mediziner}|pwk} mit der Jahreszahl »19« versehen, dürfte der Brief bereits 1918 gelaufen sein. Am
                     21. 12. 1918 fand die Wien\oindex{Wien@\textbf{Wien}|pwk}er
                  Erstaufführung statt. Zwar wurde \emph{Professor
                     Bernhardi}\pwindex{Schnitzler, Arthur 15.05.1862 – 21.10.1931@\textsc{Schnitzler, Arthur} (15.05.1862 – 21.10.1931), \emph{Schriftsteller, Mediziner}!Professor Bernhardi. Komoedie in fuenf Akten1912@\strich\emph{Professor Bernhardi. Komödie in fünf Akten} {[}1912{]}|pwk} auch am 21. 12. 1919 gespielt, doch war der
                  Verfasser nur vor der Premiere an den Proben beteiligt.}}}\label{K_L02317-1h} ein Stück\pwindex{Schnitzler, Arthur 15.05.1862 – 21.10.1931@\textsc{Schnitzler, Arthur} (15.05.1862 – 21.10.1931), \emph{Schriftsteller, Mediziner}!Professor Bernhardi. Komoedie in fuenf Akten1912@\strich\emph{Professor Bernhardi. Komödie in fünf Akten} {[}1912{]}|pwv} von Ihnen, eine Ihrer ſtärkſten u.
               glücklichſten Arbeiten wie ich glaube, ſpielen zu ſehen. Ein ſolcher Abend bindet,
               über den Abgrund des Geſchehens hinweg, die Jahre an die Jahre und erweckt ein kaum
               definierbares Gefühl: daſs ein Teil von uns doch all dieſem Geſchehen entrückt und
               von all dem {\pb}unberührbar iſt.\pend
           \pstart
           Sehr lieb war’s mir auch den »\label{K_L02317-2v}\edtext{\textsc{Casanova}\pwindex{Schnitzler, Arthur 15.05.1862 – 21.10.1931@\textsc{Schnitzler, Arthur} (15.05.1862 – 21.10.1931), \emph{Schriftsteller, Mediziner}!Schwestern oder Casanova in Spa. Lustspiel in Versen01. 10. 1919@\strich\emph{Die Schwestern oder Casanova in Spa. Lustspiel in Versen} {[}01. 10. 1919{]}|pw}}{\lemma{\textnormal{\emph{Casanova}}}\Cendnote{\textnormal{\emph{Casanovas Heimkehr}\pwindex{Schnitzler, Arthur 15.05.1862 – 21.10.1931@\textsc{Schnitzler, Arthur} (15.05.1862 – 21.10.1931), \emph{Schriftsteller, Mediziner}!Schwestern oder Casanova in Spa. Lustspiel in Versen01. 10. 1919@\strich\emph{Die Schwestern oder Casanova in Spa. Lustspiel in Versen} {[}01. 10. 1919{]}|pwk} erschien im Dezember
                     1918. Es ist anzunehmen, dass Schnitzler\pwindex{Schnitzler, Arthur 15.05.1862 – 21.10.1931@\textsc{Schnitzler, Arthur} (15.05.1862 – 21.10.1931), \emph{Schriftsteller, Mediziner}|pwk}{ }Hofmannsthal\pwindex{Hofmannsthal, Hugo von 1874-02-01 – 1929-07-15@\textsc{Hofmannsthal, Hugo von} (1874-02-01 – 1929-07-15), \emph{Schriftsteller}|pwk} ein weiteres Exemplar widmete,
                  da das erste direkt aus dem Verlag kam (Vgl. Hugo von Hofmannsthal an Arthur Schnitzler, [Anfang Dezember
               1918]). Das Exemplar ist nicht überliefert.}}}\label{K_L02317-2h}« von Ihrer
               eigenen Hand und mit Ihrem Namenszug zu empfangen – ſo gibt es doch Dinge u. Bezüge
               die ſich nicht verändern.\pend
           \pstart
           Sehr gern, lieber Arthur, möchte ich Sie aber doch wiederſehen. So unbequem es iſt,
               ich komme gerne {\pb}hinaus.
               Vormittags einmal – ich glaube, aus früheren Zeiten, das ſtört Sie nicht in der
               Arbeit.\pend
           \pstart
           Ich ſchrieb Ihnen das vor ein oder zwei Wochen, damals waren aber noch die Proben\pwindex{Schnitzler, Arthur 15.05.1862 – 21.10.1931@\textsc{Schnitzler, Arthur} (15.05.1862 – 21.10.1931), \emph{Schriftsteller, Mediziner}!Professor Bernhardi. Komoedie in fuenf Akten1912@\strich\emph{Professor Bernhardi. Komödie in fünf Akten} {[}1912{]}|pwv} vor Ihnen ſo haben Sie
               mir wahrſcheinlich deswegen nicht geantwortet.\pend
           \pstart
           Ich bin die Tage 28{ }29{ }30{ }31 in Wien\oindex{Wien@\textbf{Wien}|pw} zur Verfügung. Bitte
               ſchreiben {\pb}Sie auf einer Karte in
               die Stallburggaſſe\oindex{Stallburggasse@\textbf{Stallburggasse}|pw}, an welchem von dieſen Tagen
               Sie mich ſehen wollen.\pend
           \pstart
           Ich würde dann trachten 10 ½ draußen zu ſein.\pend
           \pstart
           Herzlich Ihr{\\[\baselineskip]}\spacefill\mbox{Hugo.}\pend
           \leftskip=0em{}
         
         \endnumbering\mylabel{h}\end{ledgroupsized}  \newcommand{\dateiname}{L02317}\newcommand{\titel}{Hugo von Hofmannsthal an Arthur Schnitzler, 21. 12. [1918?]}\newcommand{\editorInnen}{Martin Anton Müller und Gerd-Hermann Susen}%% latex-leseansicht-abspann.tex
%% Abspann für die Leseansicht.
%% Der Schalter \ifkorrekturansicht ist bereits durch den Vorspann gesetzt.

%% latex-abspann.tex
%% Gemeinsamer Abspann für Korrekturansicht und Leseansicht.
%% Setzt den Schalter \ifkorrekturansicht voraus (gesetzt in den
%% einbindenden Dateien latex-korrekturansicht-abspann.tex bzw.
%% latex-leseansicht-abspann.tex).
%% ---------------------------------------------------------------

\normalsize

% Das esempio-Environment wird nur in der Leseansicht benötigt
\ifkorrekturansicht\else
\newenvironment{esempio}[3]%
{
    \vspace{1.5ex}
    \rlap{\underline{#1}}
    \par
    \setlength{\parindent}{0cm}
    \nopagebreak
    \leftskip=#2cm
    \rightskip=#3cm
}
{
    \par
}
\fi

\doendnotes{C}
\bigskip
\vfill

\clearpage

\footnotesize

\ifkorrekturansicht
  \lohead{\textsc{register}}
\fi

% theindex-Environment neu definieren ohne reledmac
\makeatletter
\renewenvironment{theindex}{%
  \ifkorrekturansicht
    \section*{\indexname}%
  \else
    \subsubsection*{Index der erwähnten Entitäten}%
  \fi
  \setlength{\parindent}{0pt}%
  \setlength{\parskip}{0pt plus 0.3pt}%
  \let\item\@idxitem
}{%
  \ifkorrekturansicht\clearpage\fi
}
\makeatother

\IfFileExists{\jobname-pw.ind}{\input{\jobname-pw.ind}}{}

% Quellenangabe nur in der Leseansicht
\ifkorrekturansicht\else
% Fallback-Definitionen, falls die .tex-Datei \titel etc. nicht gesetzt hat
\providecommand{\titel}{}
\providecommand{\editorInnen}{}
\providecommand{\dateiname}{\jobname}

\vspace{3cm}

\vfill

\footnotesize
\textsc{Quelle}: \titel. Herausgegeben von {\editorInnen}. In: \emph{Arthur Schnitzler: Briefwechsel mit Autorinnen und Autoren}.
 Digitale Edition, https://schnitzler-briefe.acdh.oeaw.ac.at/{\dateiname}.html (Stand \today)
\fi

\end{document}


      