%% latex-korrekturansicht-vorspann.tex
%% Vorspann für die Korrekturansicht.
%% Lädt die gemeinsame Datei latex-vorspann.tex mit gesetztem Schalter.

\newif\ifkorrekturansicht
\korrekturansichttrue

\input{../tex-inputs/latex-vorspann}


\section[Paul Goldmann und Fedor Mamroth an Arthur Schnitzler, 4. 6. 1893]{L02710 Paul Goldmann und Fedor Mamroth an Arthur Schnitzler,
               4. 6. 1893}
\nopagebreak\mylabel{L02710v}
\rehead{ }\normalsize\beginnumbering\briefempfaengerindex{Schnitzler, Arthur@\textsc{Schnitzler, Arthur}!zzzMamroth, Fedor@\emph{von Fedor Mamroth}!1893-06-042@{4. 6. 1893}|(be}\briefempfaengerindex{Schnitzler, Arthur@\textsc{Schnitzler, Arthur}!zzzGoldmann, Paul@\emph{von Paul Goldmann}!1893-06-042@{4. 6. 1893}|(be}
\toendnotes[C]{\smallbreak\pagebreak[2]}\Standort{DLA, A:Schnitzler, HS.NZ85.1.3163.}
\physDesc{Brief, 1 Blatt, 2 Seiten, 325 Zeichen
\newline{}Handschrift Paul Goldmann: blaue Tinte, deutsche Kurrent}\toendnotes[C]{\smallbreak}
\pstart
           {\pb}\textcolor{gray}{\textbf{\textbf{Frankfurter Zeitung}}}\orgindex{Frankfurter Zeitung@Frankfurter Zeitung|pw}\pend
           
\pstart
           \textcolor{gray}{\textbf{und}}\pend
           
\pstart
           \textcolor{gray}{\textbf{\textbf{Handelsblatt}.}}\hfill \textcolor{gray}{\textbf{Frankfurt a. M.\oindex{Frankfurt am Main@\textbf{Frankfurt am Main}, \emph{P.PPLA3}|pw},}}{ }4. Juni \textcolor{gray}{\textbf{189}}3.\pend
           
\pstart
           \textcolor{gray}{\textbf{\textbf{Redaktion.}\noindent{}\textcolor{gray}{\textbf{Für die Redaktion\orgindex{Frankfurter Zeitung@Frankfurter Zeitung|pwv} bestimmte Briefe und Sendungen wolle
                              man \so{nicht} an die Person eines Redakteurs,
                              sondern stets \textbf{an die Redaktion\orgindex{Frankfurter Zeitung@Frankfurter Zeitung|pwv} der Frankfurter Zeitung\pwindex{Frankfurter Zeitung@\emph{Frankfurter Zeitung}|pw}} adressiren.}}}}\pend
           
\pstart
           \textcolor{gray}{\textbf{\textbf{Telegramm-Adresse:}}}\pend
           
\pstart
           \textcolor{gray}{\textbf{\textbf{Zeitung Frankfurt
                        Main\oindex{Frankfurt am Main@\textbf{Frankfurt am Main}, \emph{P.PPLA3}|pw}.}}}\pend
           \vspace{0.5em}
\pstart
           Adreſſen von Verlegern\orgindex{Verlag Wilhelm Friedrich@Verlag Wilhelm Friedrich|pwv}\orgindex{S. Schottlaender@S. Schottländer|pwv}\orgindex{E. Pierson s Verlag@E. Pierson’s Verlag|pwv}\orgindex{S. Fischer Verlag@S. Fischer Verlag|pwv}\orgindex{Freund und Jeckel@Freund {\kaufmannsund}  Jeckel|pwv}, an die
                  \label{K_L02710-1v}\edtext{wir}{\lemma{\textnormal{\emph{wir}}}\Cendnote{\textnormal{Das »wir« macht, in Fortführung der
                  Überlegungen, die im Brief vom Vortag (Paul Goldmann an Arthur Schnitzler, 3. 6. 1893) dargelegt sind, auch Fedor
                     Mamroth\pwindex{Mamroth, Fedor 21.02.1851 – 25.06.1907@\textsc{Mamroth, Fedor} (21.02.1851 – 25.06.1907), \emph{Journalist/Journalistin, Kritiker/Kritikerin}|pwk} zum Verfasser des Briefes.}}}\label{K_L02710-1} Dir rathen, Dich zu wenden
               (zuerſt an \textsc{Fischer\pwindex{Fischer, Samuel 24.12.1859 – 15.10.1934@\textsc{Fischer, Samuel} (24.12.1859 – 15.10.1934), \emph{Verleger/Verlegerin}|pw}\orgindex{S. Fischer Verlag@S. Fischer Verlag|pwv}}.)\pend
           
\pstart
           \textsc{Wilhelm Friedrich\pwindex{Friedrich, Wilhelm 1851-11-15 – 1925-10-09@\textsc{Friedrich, Wilhelm} (1851-11-15 – 1925-10-09), \emph{Verleger/Verlegerin}|pw}\orgindex{Verlag Wilhelm Friedrich@Verlag Wilhelm Friedrich|pwv}}{ }\textsc{Leipzig\oindex{Leipzig@\textbf{Leipzig}, \emph{P.PPLA3}|pw}}.\pend
           
\pstart
           \textsc{Schlesische Buchdruckerei Kunst- und
                     Verlags-Anstalt vorm. S. Schottlaender\pwindex{Schottlaender, Salo 1844-06-19 – 1920-04-02@\textsc{Schottlaender, Salo} (1844-06-19 – 1920-04-02), \emph{Verleger/Verlegerin}|pw}\orgindex{S. Schottlaender@S. Schottländer|pw}, Breslau\oindex{Breslau@\textbf{Breslau}, \emph{P.PPLA}|pw}}.\pend
           
\pstart
           \textsc{E. Piersons\pwindex{Pierson, Edgar 1849-03-14 – 1919@\textsc{Pierson, Edgar} (1849-03-14 – 1919), \emph{Verleger/Verlegerin}|pw}s Verlag\orgindex{E. Pierson s Verlag@E. Pierson’s Verlag|pw}, Dresden, Altstadt\oindex{Dresden@\textbf{Dresden}, \emph{P.PPLA}|pw}}.\pend
           
\pstart
           \textsc{S. Fischer\pwindex{Fischer, Samuel 24.12.1859 – 15.10.1934@\textsc{Fischer, Samuel} (24.12.1859 – 15.10.1934), \emph{Verleger/Verlegerin}|pw}\orgindex{S. Fischer Verlag@S. Fischer Verlag|pw}, Berlin\oindex{Berlin@\textbf{Berlin}, \emph{P.PPLC}|pw}}{ }\textsc{Koethenerstraſse} 44\oindex{Koethenerstrasse@\textbf{Köthenerstraße}, \emph{Straße (K.STR)}|pw}.\pend
           
\pstart
           \textsc{Freund\pwindex{Freund, Carl @\textsc{Freund, Carl}, \emph{Verleger/Verlegerin}|pw} und Jeckel\pwindex{Jeckel, Max *~1879-04-18@\textsc{Jeckel, Max} (*~1879-04-18), \emph{Verleger/Verlegerin, Buchhändler/Buchhändlerin}|pw}\orgindex{Freund und Jeckel@Freund {\kaufmannsund}  Jeckel|pw}}, {\pb}\textsc{Berlin N. W. 23, Altonaerstraſse 37a\oindex{Altonaer Strasse@\textbf{Altonaer Straße}, \emph{Straße (K.STR)}|pw}}.\pend
           \selectlanguage{ngerman}\endnumbering\briefempfaengerindex{Schnitzler, Arthur@\textsc{Schnitzler, Arthur}!zzzMamroth, Fedor@\emph{von Fedor Mamroth}!1893-06-042@{4. 6. 1893}|)be}\briefempfaengerindex{Schnitzler, Arthur@\textsc{Schnitzler, Arthur}!zzzGoldmann, Paul@\emph{von Paul Goldmann}!1893-06-042@{4. 6. 1893}|)be}\mylabel{L02710h}  \normalsize

\doendnotes{C}
\bigskip
\vfill

\clearpage

\footnotesize

\lohead{\textsc{register}}

% Definiere theindex-Environment komplett neu ohne reledmac
\makeatletter
\renewenvironment{theindex}{%
  \section*{\indexname}%
  \setlength{\parindent}{0pt}%
  \setlength{\parskip}{0pt plus 0.3pt}%
  \let\item\@idxitem
}{%
  \clearpage
}
\makeatother

\IfFileExists{\jobname-pw.ind}{\input{\jobname-pw.ind}}{}

\end{document}

      