%% latex-leseansicht-vorspann.tex
%% Vorspann für die Leseansicht.
%% Lädt die gemeinsame Datei latex-vorspann.tex mit nicht gesetztem Schalter.

\newif\ifkorrekturansicht
\korrekturansichtfalse

\input{../tex-inputs/latex-vorspann}


\section[Paul Goldmann und Fedor Mamroth an Arthur Schnitzler, 4. 6. 1893]{L02710 Paul Goldmann und Fedor Mamroth an Arthur Schnitzler, 4. 6. 1893}
\nopagebreak\mylabel{L02710v}
\rehead{ }\normalsize\beginnumbering\briefempfaengerindex{Schnitzler, Arthur@\textsc{Schnitzler, Arthur}!zzzMamroth, Fedor@\emph{von Fedor Mamroth}!1893-06-042@{4. 6. 1893}|(be}\briefempfaengerindex{Schnitzler, Arthur@\textsc{Schnitzler, Arthur}!zzzGoldmann, Paul@\emph{von Paul Goldmann}!1893-06-042@{4. 6. 1893}|(be}
\toendnotes[C]{\smallbreak\pagebreak[2]}
\correspDesc{Versand  durch Paul Goldmann, Fedor Mamroth am 4. 6. 1893 in Frankfurt am Main
\newline{}Erhalt  durch Arthur Schnitzler im Zeitraum [5. 6. 1893
                  – 9. 6. 1893?] in Wien}\toendnotes[C]{\smallbreak}
\Standort{DLA, A:Schnitzler, HS.NZ85.1.3163.}
\physDesc{Brief, 1 Blatt, 2 Seiten, 325 Zeichen
\newline{}HandschriftX2  : blaue Tinte, deutsche Kurrent}\toendnotes[C]{\smallbreak}
\pstart
           {\pb}\textcolor{gray}{\textbf{\textbf{Frankfurter Zeitung}}}\orgindex{Frankfurter Zeitung@Frankfurter Zeitung|pw}\pend
           
\pstart
           \textcolor{gray}{\textbf{und}}\pend
           
\pstart
           \textcolor{gray}{\textbf{\textbf{Handelsblatt}.}}\hfill \textcolor{gray}{\textbf{Frankfurt a. M.\oindex{Frankfurt am Main@\textbf{Frankfurt am Main}, \emph{Hauptstadt}|pw},}}{ }4. Juni \textcolor{gray}{\textbf{189}}3.\pend
           
\pstart
           \textcolor{gray}{\textbf{\textbf{Redaktion.}\footnote{\noindent{}\textcolor{gray}{\textbf{Für die Redaktion\orgindex{Frankfurter Zeitung@Frankfurter Zeitung|pwv} bestimmte Briefe und Sendungen wolle
                              man \so{nicht} an die Person eines Redakteurs,
                              sondern stets \textbf{an die Redaktion\orgindex{Frankfurter Zeitung@Frankfurter Zeitung|pwv} der Frankfurter Zeitung\pwindex{Frankfurter Zeitung@\emph{Frankfurter Zeitung}|pw}} adressiren.}}}}}\pend
           
\pstart
           \textcolor{gray}{\textbf{\textbf{Telegramm-Adresse:}}}\pend
           
\pstart
           \textcolor{gray}{\textbf{\textbf{Zeitung Frankfurt
                        Main\oindex{Frankfurt am Main@\textbf{Frankfurt am Main}, \emph{Hauptstadt}|pw}.}}}\pend
           \vspace{0.5em}
\pstart
           Adreſſen von Verlegern\orgindex{Verlag Wilhelm Friedrich@Verlag Wilhelm Friedrich|pwv}\orgindex{S. Schottländer@S. Schottländer|pwv}\orgindex{E. Pierson’s Verlag@E. Pierson’s Verlag|pwv}\orgindex{S. Fischer Verlag@S. Fischer Verlag|pwv}\orgindex{Freund und Jeckel@Freund {\kaufmannsund}  Jeckel|pwv}, an die
                  \label{K_L02710-1v}\edtext{wir}{\lemma{\textnormal{\emph{wir}}}\Cendnote{\textnormal{Das »wir« macht, in Fortführung der
                  Überlegungen, die im Brief vom Vortag (XXXX Auszeichnungsfehler: Dokument L02709 nicht gefunden) dargelegt sind, auch Fedor
                     Mamroth\pwindex{Mamroth, Fedor 21.\,2.\,1851 Breslau – 25.\,6.\,1907 Frankfurt am Main@\textsc{Mamroth, Fedor} (21.\,2.\,1851 Breslau – 25.\,6.\,1907 Frankfurt am Main), \emph{Journalist, Kritiker}|pwk} zum Verfasser des Briefes.}}}\label{K_L02710-1} Dir rathen, Dich zu wenden
               (zuerſt an \textsc{Fischer\pwindex{Fischer, Samuel 24.\,12.\,1859 Liptovský Mikuláš – 15.\,10.\,1934 Berlin@\textsc{Fischer, Samuel} (24.\,12.\,1859 Liptovský Mikuláš – 15.\,10.\,1934 Berlin), \emph{Verleger}|pw}\orgindex{S. Fischer Verlag@S. Fischer Verlag|pwv}}.)\pend
           
\pstart
           \textsc{Wilhelm Friedrich\pwindex{Friedrich, Wilhelm 15.\,11.\,1851 Anklam – 9.\,10.\,1925 Magugnano@\textsc{Friedrich, Wilhelm} (15.\,11.\,1851 Anklam – 9.\,10.\,1925 Magugnano), \emph{Verleger}|pw}\orgindex{Verlag Wilhelm Friedrich@Verlag Wilhelm Friedrich|pwv}}{ }\textsc{Leipzig\oindex{Leipzig@\textbf{Leipzig}, \emph{Hauptstadt}|pw}}.\pend
           
\pstart
           \textsc{Schlesische Buchdruckerei Kunst- und
                     Verlags-Anstalt vorm. S. Schottlaender\pwindex{Schottlaender, Salo 19.\,6.\,1844 Ziębice – 2.\,4.\,1920 Breslau@\textsc{Schottlaender, Salo} (19.\,6.\,1844 Ziębice – 2.\,4.\,1920 Breslau), \emph{Verleger}|pw}\orgindex{S. Schottländer@S. Schottländer|pw}, Breslau\oindex{Breslau@\textbf{Breslau}|pw}}.\pend
           
\pstart
           \textsc{E. Piersons\pwindex{Pierson, Edgar 14.\,3.\,1849 Hamburg – 1919 Dresden@\textsc{Pierson, Edgar} (14.\,3.\,1849 Hamburg – 1919 Dresden), \emph{Verleger}|pw}s Verlag\orgindex{E. Pierson’s Verlag@E. Pierson’s Verlag|pw}, Dresden, Altstadt\oindex{Dresden@\textbf{Dresden}|pw}}.\pend
           
\pstart
           \textsc{S. Fischer\pwindex{Fischer, Samuel 24.\,12.\,1859 Liptovský Mikuláš – 15.\,10.\,1934 Berlin@\textsc{Fischer, Samuel} (24.\,12.\,1859 Liptovský Mikuláš – 15.\,10.\,1934 Berlin), \emph{Verleger}|pw}\orgindex{S. Fischer Verlag@S. Fischer Verlag|pw}, Berlin\oindex{Berlin@\textbf{Berlin}, \emph{Hauptstadt}|pw}}{ }\textsc{Koethenerstraſse} 44\oindex{Köthenerstraße@\textbf{Köthenerstraße}, \emph{Straße}|pw}.\pend
           
\pstart
           \textsc{Freund\pwindex{Freund, Carl @\textsc{Freund, Carl}, \emph{Verleger}|pw} und Jeckel\pwindex{Jeckel, Max *~18.\,4.\,1879@\textsc{Jeckel, Max} (*~18.\,4.\,1879), \emph{Verleger, Buchhändler}|pw}\orgindex{Freund und Jeckel@Freund {\kaufmannsund}  Jeckel|pw}}, {\pb}\textsc{Berlin N. W. 23, Altonaerstraſse 37a\oindex{Altonaer Straße@\textbf{Altonaer Straße}, \emph{Straße}|pw}}.\pend
           \selectlanguage{ngerman}\endnumbering\briefempfaengerindex{Schnitzler, Arthur@\textsc{Schnitzler, Arthur}!zzzMamroth, Fedor@\emph{von Fedor Mamroth}!1893-06-042@{4. 6. 1893}|)be}\briefempfaengerindex{Schnitzler, Arthur@\textsc{Schnitzler, Arthur}!zzzGoldmann, Paul@\emph{von Paul Goldmann}!1893-06-042@{4. 6. 1893}|)be}\mylabel{L02710h}  \newcommand{\dateiname}{L02710}\newcommand{\titel}{Paul Goldmann und Fedor Mamroth an Arthur Schnitzler, 4. 6. 1893}\newcommand{\editorInnen}{Martin Anton Müller und Laura Untner}%% latex-leseansicht-abspann.tex
%% Abspann für die Leseansicht.
%% Der Schalter \ifkorrekturansicht ist bereits durch den Vorspann gesetzt.

%% latex-abspann.tex
%% Gemeinsamer Abspann für Korrekturansicht und Leseansicht.
%% Setzt den Schalter \ifkorrekturansicht voraus (gesetzt in den
%% einbindenden Dateien latex-korrekturansicht-abspann.tex bzw.
%% latex-leseansicht-abspann.tex).
%% ---------------------------------------------------------------

\normalsize

% Das esempio-Environment wird nur in der Leseansicht benötigt
\ifkorrekturansicht\else
\newenvironment{esempio}[3]%
{
    \vspace{1.5ex}
    \rlap{\underline{#1}}
    \par
    \setlength{\parindent}{0cm}
    \nopagebreak
    \leftskip=#2cm
    \rightskip=#3cm
}
{
    \par
}
\fi

\doendnotes{C}
\bigskip
\vfill

\clearpage

\footnotesize

\ifkorrekturansicht
  \lohead{\textsc{register}}
\fi

% theindex-Environment neu definieren ohne reledmac
\makeatletter
\renewenvironment{theindex}{%
  \ifkorrekturansicht
    \section*{\indexname}%
  \else
    \subsubsection*{Index der erwähnten Entitäten}%
  \fi
  \setlength{\parindent}{0pt}%
  \setlength{\parskip}{0pt plus 0.3pt}%
  \let\item\@idxitem
}{%
  \ifkorrekturansicht\clearpage\fi
}
\makeatother

\IfFileExists{\jobname-pw.ind}{\input{\jobname-pw.ind}}{}

% Quellenangabe nur in der Leseansicht
\ifkorrekturansicht\else
% Fallback-Definitionen, falls die .tex-Datei \titel etc. nicht gesetzt hat
\providecommand{\titel}{}
\providecommand{\editorInnen}{}
\providecommand{\dateiname}{\jobname}

\vspace{3cm}

\vfill

\footnotesize
\textsc{Quelle}: \titel. Herausgegeben von {\editorInnen}. In: \emph{Arthur Schnitzler: Briefwechsel mit Autorinnen und Autoren}.
 Digitale Edition, https://schnitzler-briefe.acdh.oeaw.ac.at/{\dateiname}.html (Stand \today)
\fi

\end{document}


