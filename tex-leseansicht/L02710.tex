%% latex-leseansicht-vorspann.tex
%% Vorspann für die Leseansicht.
%% Lädt die gemeinsame Datei latex-vorspann.tex mit nicht gesetztem Schalter.

\newif\ifkorrekturansicht
\korrekturansichtfalse

\input{../tex-inputs/latex-vorspann}


         
         \newcommand{\erwaehntePersonen}{Personen: Samuel Fischer, Carl Freund, Wilhelm Friedrich, Max Jeckel, Edgar Pierson, Salo Schottlaender}
         \newcommand{\erwaehnteInstitutionen}{Institutionen: E. Pierson’s Verlag, Frankfurter Zeitung, Freund & Jeckel, S. Fischer Verlag, S. Schottländer, Verlag Wilhelm Friedrich}
         \newcommand{\erwaehnteOrte}{Orte: Altonaer Straße, Berlin, Breslau, Dresden, Frankfurt am Main, Köthenerstraße, Leipzig, Wien}
         \newcommand{\erwaehnteWerke}{Werke: Frankfurter Zeitung}
               \section[Paul Goldmann und Fedor Mamroth an Arthur Schnitzler, 4. 6. 1893]{ Paul Goldmann und Fedor Mamroth an Arthur Schnitzler,
               4. 6. 1893}\nopagebreak\mylabel{v}\rehead{ }\begin{ledgroupsized}[t]{13cm}\normalsize\beginnumbering \toendnotes[C]{\smallbreak\pagebreak[2]} \Standort{DLA, A:Schnitzler, HS.NZ85.1.3163.}
\physDesc{Brief, 1 Blatt, 2 Seiten
\newline{}Handschrift Paul Goldmann: blaue Tinte, deutsche Kurrent}\toendnotes[C]{\smallbreak}\pstart
           \noindent{}{\pb}\textcolor{gray}{\textbf{\textbf{Frankfurter Zeitung}}}\orgindex{Frankfurter Zeitung@Frankfurter Zeitung|pw}\pend
           \pstart
           \textcolor{gray}{\textbf{und}}\pend
           \pstart
           \textcolor{gray}{\textbf{\textbf{Handelsblatt}.}}\hfill \textcolor{gray}{\textbf{Frankfurt a. M.\oindex{Frankfurt am Main@\textbf{Frankfurt am Main}|pw},}}{ }4. Juni \textcolor{gray}{\textbf{189}}3.\pend
           \pstart
           \textcolor{gray}{\textbf{\textbf{Redaktion.}\footnote{\noindent{}\textcolor{gray}{\textbf{Für die Redaktion\orgindex{Frankfurter Zeitung@Frankfurter Zeitung|pwv} bestimmte Briefe und Sendungen wolle
                              man \so{nicht} an die Person eines Redakteurs,
                              sondern stets \textbf{an die Redaktion\orgindex{Frankfurter Zeitung@Frankfurter Zeitung|pwv} der Frankfurter Zeitung\pwindex{?? Werk@Nicht ermittelte Verfasserinnen und Verfasser!Frankfurter Zeitung1856 – 1943@\emph{Frankfurter Zeitung} {[}1856 – 1943{]}|pw}} adressiren.}}}}}\pend
           \pstart
           \textcolor{gray}{\textbf{\textbf{Telegramm-Adresse:}}}\pend
           \pstart
           \textcolor{gray}{\textbf{\textbf{Zeitung Frankfurt
                        Main\oindex{Frankfurt am Main@\textbf{Frankfurt am Main}|pw}.}}}\pend
           \pstart
           Adreſſen von Verlegern\orgindex{Verlag Wilhelm Friedrich@Verlag Wilhelm Friedrich|pwv}\orgindex{S. Schottlaender@S. Schottländer|pwv}\orgindex{E. Pierson s Verlag@E. Pierson’s Verlag|pwv}\orgindex{S. Fischer Verlag@S. Fischer Verlag|pwv}\orgindex{Freund und Jeckel@Freund {\kaufmannsund}  Jeckel|pwv}, an die
                  \label{K_L02710-1v}\edtext{wir}{\lemma{\textnormal{\emph{wir}}}\Cendnote{\textnormal{Das »wir« macht, in Fortführung der
                  Überlegungen, die im Brief vom Vortag (Paul Goldmann an Arthur Schnitzler, 3. 6. 1893) dargelegt sind, auch Fedor
                     Mamroth\pwindex{Mamroth, Fedor 21.02.1851 – 25.06.1907@\textsc{Mamroth, Fedor} (21.02.1851 – 25.06.1907), \emph{Journalist, Kritiker}|pwk} zum Verfasser des Briefes.}}}\label{K_L02710-1h} Dir rathen, Dich zu wenden
               (zuerſt an \textsc{Fischer\pwindex{Fischer, Samuel 24.12.1859 – 15.10.1934@\textsc{Fischer, Samuel} (24.12.1859 – 15.10.1934), \emph{Verleger}|pw}\orgindex{S. Fischer Verlag@S. Fischer Verlag|pwv}}.)\pend
           \pstart
           \textsc{Wilhelm Friedrich\pwindex{Friedrich, Wilhelm 1851-11-15 – 1925-10-09@\textsc{Friedrich, Wilhelm} (1851-11-15 – 1925-10-09), \emph{Verleger}|pw}\orgindex{Verlag Wilhelm Friedrich@Verlag Wilhelm Friedrich|pwv}}{ }\textsc{Leipzig\oindex{Leipzig@\textbf{Leipzig}|pw}}.\pend
           \pstart
           \textsc{Schlesische Buchdruckerei Kunst- und
                     Verlags-Anstalt vorm. S. Schottlaender\pwindex{Schottlaender, Salo 1844-06-19 – 1920-04-02@\textsc{Schottlaender, Salo} (1844-06-19 – 1920-04-02), \emph{Verleger}|pw}\orgindex{S. Schottlaender@S. Schottländer|pw}, Breslau\oindex{Breslau@\textbf{Breslau}|pw}}.\pend
           \pstart
           \textsc{E. Pierson\pwindex{Pierson, Edgar 1849-03-14 – 1919@\textsc{Pierson, Edgar} (1849-03-14 – 1919), \emph{Verleger}|pw}s Verlag\orgindex{E. Pierson s Verlag@E. Pierson’s Verlag|pw}, Dresden, Altstadt\oindex{Dresden@\textbf{Dresden}|pw}}.\pend
           \pstart
           \textsc{S. Fischer\pwindex{Fischer, Samuel 24.12.1859 – 15.10.1934@\textsc{Fischer, Samuel} (24.12.1859 – 15.10.1934), \emph{Verleger}|pw}\orgindex{S. Fischer Verlag@S. Fischer Verlag|pw}, Berlin\oindex{Berlin@\textbf{Berlin}|pw}}{ }\textsc{Koethenerstraße} 44\oindex{Koethenerstrasse@\textbf{Köthenerstraße}|pw}.\pend
           \pstart
           \textsc{Freund\pwindex{Freund, Carl @\textsc{Freund, Carl}, \emph{Verleger}|pw} und Jeckel\pwindex{Jeckel, Max *~1879-04-18@\textsc{Jeckel, Max} (*~1879-04-18), \emph{Verleger, Buchhändler}|pw}\orgindex{Freund und Jeckel@Freund {\kaufmannsund}  Jeckel|pw}}, {\pb}\textsc{Berlin N. W. 23, Altonaerstraße 37a\oindex{Altonaer Strasse@\textbf{Altonaer Straße}|pw}}.\pend
           
         
         \endnumbering\mylabel{h}\end{ledgroupsized}  \newcommand{\dateiname}{L02710}\newcommand{\titel}{Paul Goldmann und Fedor Mamroth an Arthur Schnitzler, 4. 6. 1893}\newcommand{\editorInnen}{Martin Anton Müller und Laura Untner}%% latex-leseansicht-abspann.tex
%% Abspann für die Leseansicht.
%% Der Schalter \ifkorrekturansicht ist bereits durch den Vorspann gesetzt.

%% latex-abspann.tex
%% Gemeinsamer Abspann für Korrekturansicht und Leseansicht.
%% Setzt den Schalter \ifkorrekturansicht voraus (gesetzt in den
%% einbindenden Dateien latex-korrekturansicht-abspann.tex bzw.
%% latex-leseansicht-abspann.tex).
%% ---------------------------------------------------------------

\normalsize

% Das esempio-Environment wird nur in der Leseansicht benötigt
\ifkorrekturansicht\else
\newenvironment{esempio}[3]%
{
    \vspace{1.5ex}
    \rlap{\underline{#1}}
    \par
    \setlength{\parindent}{0cm}
    \nopagebreak
    \leftskip=#2cm
    \rightskip=#3cm
}
{
    \par
}
\fi

\doendnotes{C}
\bigskip
\vfill

\clearpage

\footnotesize

\ifkorrekturansicht
  \lohead{\textsc{register}}
\fi

% theindex-Environment neu definieren ohne reledmac
\makeatletter
\renewenvironment{theindex}{%
  \ifkorrekturansicht
    \section*{\indexname}%
  \else
    \subsubsection*{Index der erwähnten Entitäten}%
  \fi
  \setlength{\parindent}{0pt}%
  \setlength{\parskip}{0pt plus 0.3pt}%
  \let\item\@idxitem
}{%
  \ifkorrekturansicht\clearpage\fi
}
\makeatother

\IfFileExists{\jobname-pw.ind}{\input{\jobname-pw.ind}}{}

% Quellenangabe nur in der Leseansicht
\ifkorrekturansicht\else
% Fallback-Definitionen, falls die .tex-Datei \titel etc. nicht gesetzt hat
\providecommand{\titel}{}
\providecommand{\editorInnen}{}
\providecommand{\dateiname}{\jobname}

\vspace{3cm}

\vfill

\footnotesize
\textsc{Quelle}: \titel. Herausgegeben von {\editorInnen}. In: \emph{Arthur Schnitzler: Briefwechsel mit Autorinnen und Autoren}.
 Digitale Edition, https://schnitzler-briefe.acdh.oeaw.ac.at/{\dateiname}.html (Stand \today)
\fi

\end{document}


      