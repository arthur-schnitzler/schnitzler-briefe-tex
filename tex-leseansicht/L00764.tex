%% latex-korrekturansicht-vorspann.tex
%% Vorspann für die Korrekturansicht.
%% Lädt die gemeinsame Datei latex-vorspann.tex mit gesetztem Schalter.

\newif\ifkorrekturansicht
\korrekturansichttrue

\input{../tex-inputs/latex-vorspann}


\section[Max Burckhard an Arthur Schnitzler, {[}nach dem 18. 1. 1898{]}]{L00764 Max Burckhard an Arthur Schnitzler, {[}nach dem 18. 1. 1898{]}}
\nopagebreak\mylabel{L00764v}
\rehead{ }\normalsize\beginnumbering\briefempfaengerindex{Schnitzler, Arthur@\textsc{Schnitzler, Arthur}!zzzBurckhard, Max Eugen@\emph{von Max Eugen Burckhard}!1898-12-311@{{[}nach dem
                  18. 1. 1898{]}}|(be}
\toendnotes[C]{\smallbreak\pagebreak[2]}\Standort{CUL, Schnitzler, B 20.}
\physDesc{Visitenkarte, 39 Zeichen
\newline{}Handschrift: Bleistift, deutsche Kurrent}\toendnotes[C]{\smallbreak}
\pstart
           \noindent{}\centering{}{\pb}\textcolor{gray}{\textbf{\textsc{D\textsuperscript{r.} Max Eugen Burckhard}}}\pend
           
\pstart
           \centering{}\textcolor{gray}{\textbf{\textsc{\strikeout{K. u. K. \label{K_L00764-1v}\edtext{Director}{\lemma{\textnormal{\emph{Director}}}\Cendnote{\textnormal{Burckhard\pwindex{Burckhard, Max Eugen 14.07.1854 – 16.03.1912@\textsc{Burckhard, Max Eugen} (14.07.1854 – 16.03.1912), \emph{Schriftsteller/Schriftstellerin, Rechtswissenschaftler/Rechtswissenschaftlerin, Theaterleiter/Theaterleiterin}|pwk} legte am
                              18. 1. 1898 die Leitung des Burgtheaters\oindex{Burgtheater@\textbf{Burgtheater}, \emph{S.THTR}|pwk} nieder. Die handschriftliche Streichung dürfte
                           diese Karte also unmittelbar in zeitlicher Nähe verorten.}}}\label{K_L00764-1} des K. K.
                           Hofburgtheaters\oindex{Burgtheater@\textbf{Burgtheater}, \emph{S.THTR}|pw}}}}}\pend
           
\pstart
           mit herzlichſten Grüßen und bestem Dank\pend
           \selectlanguage{ngerman}\endnumbering\briefempfaengerindex{Schnitzler, Arthur@\textsc{Schnitzler, Arthur}!zzzBurckhard, Max Eugen@\emph{von Max Eugen Burckhard}!1898-01-181@{{[}nach dem
                  18. 1. 1898{]}}|)be}\mylabel{L00764h}  \normalsize

\doendnotes{C}
\bigskip
\vfill

\clearpage

\footnotesize

\lohead{\textsc{register}}

% Definiere theindex-Environment komplett neu ohne reledmac
\makeatletter
\renewenvironment{theindex}{%
  \section*{\indexname}%
  \setlength{\parindent}{0pt}%
  \setlength{\parskip}{0pt plus 0.3pt}%
  \let\item\@idxitem
}{%
  \clearpage
}
\makeatother

\IfFileExists{\jobname-pw.ind}{\input{\jobname-pw.ind}}{}

\end{document}

      