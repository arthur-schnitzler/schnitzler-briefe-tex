%% latex-leseansicht-vorspann.tex
%% Vorspann für die Leseansicht.
%% Lädt die gemeinsame Datei latex-vorspann.tex mit nicht gesetztem Schalter.

\newif\ifkorrekturansicht
\korrekturansichtfalse

\input{../tex-inputs/latex-vorspann}


\section[ Felix Salten an Arthur Schnitzler, {[}29. 8. 1899{]}]{L03299 Felix Salten an Arthur Schnitzler,  [29. 8. 1899]}
\nopagebreak\mylabel{L03299v}
\rehead{ }\normalsize\beginnumbering\briefempfaengerindex{Schnitzler, Arthur@\textsc{Schnitzler, Arthur}!zzzSalten, Felix@\emph{von Felix Salten}!1899-08-291@{{[}29. 8. 1899{]}}|(be}
\toendnotes[C]{\smallbreak\pagebreak[2]}
\correspDesc{Versand  durch Felix Salten am [29. 8. 1899] in Wien
\newline{}Erhalt  durch Arthur Schnitzler am 29. 8. 1899 in Ischl}\toendnotes[C]{\smallbreak}
\Standort{CUL, Schnitzler, B 89, A 2.}
\physDesc{Brief, 1 Blatt, 2 Seiten, 783 Zeichen
\newline{}Handschrift: schwarze Tinte, lateinische Kurrent
\newline{}Schnitzler: mit Bleistift datiert: »29/8 9\textcolor{gray}{9}« 
\newline{}Ordnung: mit Bleistift von unbekannter Hand nummeriert: »123« }\toendnotes[C]{\smallbreak}
\pstart
           \raggedleft{}{\pb}Dienstag.\pend
           \vspace{0.5em}
\pstart
           Lieber, ich sende Ihnen gleichzeitig die \label{K_L03299-1v}\edtext{versprochenen Zeitungen\pwindex{Wiener Allgemeine Montags-Zeitung@\emph{Wiener Allgemeine Montags-Zeitung}|pwv}}{\lemma{\textnormal{\emph{versprochenen Zeitungen}}}\Cendnote{\textnormal{Es dürfte sich um drei Novellen Saltens\pwindex{Salten, Felix 6.\,9.\,1869 Budapest – 8.\,10.\,1945 Zürich@\textsc{Salten, Felix} (6.\,9.\,1869 Budapest – 8.\,10.\,1945 Zürich), \emph{Schriftsteller, Journalist, Chefredakteur}|pwk} handeln, die seit dem ersten Heft vom
                     3. 7. 1899 in der \emph{Wiener Allgemeinen Montags-Zeitung}\pwindex{Wiener Allgemeine Montags-Zeitung@\emph{Wiener Allgemeine Montags-Zeitung}|pwk} erschienen waren, da
                     Schnitzler in seinem Antwortschreiben vom
                     XXXX Auszeichnungsfehler: Dokument L02967 nicht gefunden zwei davon
                  direkt anspricht: \emph{Flucht}\pwindex{Salten, Felix 6.\,9.\,1869 Budapest – 8.\,10.\,1945 Zürich@\textsc{Salten, Felix} (6.\,9.\,1869 Budapest – 8.\,10.\,1945 Zürich), \emph{Schriftsteller, Journalist, Chefredakteur}!Flucht@\strich\emph{Flucht}|pwk} (31. 7. 1899, S. 2–3 und 7. 8. 1899, S. 3–4) und \emph{Das
                     Manhard-Zimmer}\pwindex{Salten, Felix 6.\,9.\,1869 Budapest – 8.\,10.\,1945 Zürich@\textsc{Salten, Felix} (6.\,9.\,1869 Budapest – 8.\,10.\,1945 Zürich), \emph{Schriftsteller, Journalist, Chefredakteur}!Manhard-Zimmer@\strich\emph{Das Manhard-Zimmer}|pwk} (21. 8. 1899, S. 3–4). Zusätzlich war in
                  dem Blatt\pwindex{Wiener Allgemeine Montags-Zeitung@\emph{Wiener Allgemeine Montags-Zeitung}|pwkv}{ }\emph{Sedan}\pwindex{Salten, Felix 6.\,9.\,1869 Budapest – 8.\,10.\,1945 Zürich@\textsc{Salten, Felix} (6.\,9.\,1869 Budapest – 8.\,10.\,1945 Zürich), \emph{Schriftsteller, Journalist, Chefredakteur}!Sedan@\strich\emph{Sedan}|pwk} (3. 7. 1899, S. 2) erschienen.}}}\label{K_L03299-1}, und
               bitte Sie, mir gelegentlich zu sagen, was Sie drüber denken, und wie Sie glauben,
               dass mans besser machen könnte. Haben Sie sich über die \label{K_L03299-2v}\edtext{Pneumatik}{\lemma{\textnormal{\emph{Pneumatik}}}\Cendnote{\textnormal{
                     Am 24. 8. 1899 
                     hatte Schnitzler{ }Salten\pwindex{Salten, Felix 6.\,9.\,1869 Budapest – 8.\,10.\,1945 Zürich@\textsc{Salten, Felix} (6.\,9.\,1869 Budapest – 8.\,10.\,1945 Zürich), \emph{Schriftsteller, Journalist, Chefredakteur}|pwk}
                     mit dem Rad nach Traunkirchen\oindex{Traunkirchen@\textbf{Traunkirchen}|pwk} begleitet. Bei dieser
                     Fahrt dürfte er Probleme mit dem Reifendruck gehabt haben.}}}\label{K_L03299-2} sehr geärgert? Ich habe mit der Zeitung\pwindex{Wiener Allgemeine Montags-Zeitung@\emph{Wiener Allgemeine Montags-Zeitung}|pwv} sehr viel zu thun, arbeite aber
               gleichwol ziemlich viel. Ich denke ernsthaft daran, die \label{K_L03299-3v}\edtext{Novellen\pwindex{Salten, Felix 6.\,9.\,1869 Budapest – 8.\,10.\,1945 Zürich@\textsc{Salten, Felix} (6.\,9.\,1869 Budapest – 8.\,10.\,1945 Zürich), \emph{Schriftsteller, Journalist, Chefredakteur}!Hinterbliebene. Kurze Novellen@\strich\emph{Der Hinterbliebene. Kurze Novellen}|pwv} herauszugeben}{\lemma{\textnormal{\emph{Novellen herauszugeben}}}\Cendnote{\textnormal{Zu den bereits in der \emph{Wiener Allgemeinen Montags-Zeitung}\pwindex{Wiener Allgemeine Montags-Zeitung@\emph{Wiener Allgemeine Montags-Zeitung}|pwk} erschienenen drei Novellen\pwindex{Salten, Felix 6.\,9.\,1869 Budapest – 8.\,10.\,1945 Zürich@\textsc{Salten, Felix} (6.\,9.\,1869 Budapest – 8.\,10.\,1945 Zürich), \emph{Schriftsteller, Journalist, Chefredakteur}!Manhard-Zimmer@\strich\emph{Das Manhard-Zimmer}|pwkv}\pwindex{Salten, Felix 6.\,9.\,1869 Budapest – 8.\,10.\,1945 Zürich@\textsc{Salten, Felix} (6.\,9.\,1869 Budapest – 8.\,10.\,1945 Zürich), \emph{Schriftsteller, Journalist, Chefredakteur}!Sedan@\strich\emph{Sedan}|pwkv}\pwindex{Salten, Felix 6.\,9.\,1869 Budapest – 8.\,10.\,1945 Zürich@\textsc{Salten, Felix} (6.\,9.\,1869 Budapest – 8.\,10.\,1945 Zürich), \emph{Schriftsteller, Journalist, Chefredakteur}!Flucht@\strich\emph{Flucht}|pwkv} fügte Salten\pwindex{Salten, Felix 6.\,9.\,1869 Budapest – 8.\,10.\,1945 Zürich@\textsc{Salten, Felix} (6.\,9.\,1869 Budapest – 8.\,10.\,1945 Zürich), \emph{Schriftsteller, Journalist, Chefredakteur}|pwk} fünf weitere
                  hinzu und vereinigte sie zum Novellenband \emph{Der
                     Hinterbliebene. Kurze Novellen}\pwindex{Salten, Felix 6.\,9.\,1869 Budapest – 8.\,10.\,1945 Zürich@\textsc{Salten, Felix} (6.\,9.\,1869 Budapest – 8.\,10.\,1945 Zürich), \emph{Schriftsteller, Journalist, Chefredakteur}!Hinterbliebene. Kurze Novellen@\strich\emph{Der Hinterbliebene. Kurze Novellen}|pwk}, der 1900 im \emph{Wiener Verlag}\orgindex{Wiener Verlag@Wiener Verlag|pwk} erschien.}}}\label{K_L03299-3}: \label{K_L03299-4v}\edtext{Der Hinterbliebene\pwindex{Salten, Felix 6.\,9.\,1869 Budapest – 8.\,10.\,1945 Zürich@\textsc{Salten, Felix} (6.\,9.\,1869 Budapest – 8.\,10.\,1945 Zürich), \emph{Schriftsteller, Journalist, Chefredakteur}!Hinterbliebene@\strich\emph{Der Hinterbliebene}|pw}}{\lemma{\textnormal{\emph{Der Hinterbliebene}}}\Cendnote{\textnormal{\emph{Die Zeit}\pwindex{Zeit. Wiener Wochenschrift@\emph{Die Zeit. Wiener Wochenschrift}|pwk}, Bd. 18, Nr. 231, 4. 3. 1899 – Nr. 232, 11. 3. 1899.}}}\label{K_L03299-4}, Flucht\pwindex{Salten, Felix 6.\,9.\,1869 Budapest – 8.\,10.\,1945 Zürich@\textsc{Salten, Felix} (6.\,9.\,1869 Budapest – 8.\,10.\,1945 Zürich), \emph{Schriftsteller, Journalist, Chefredakteur}!Flucht@\strich\emph{Flucht}|pw}, \label{K_L03299-5v}\edtext{Begräbnis\pwindex{Salten, Felix 6.\,9.\,1869 Budapest – 8.\,10.\,1945 Zürich@\textsc{Salten, Felix} (6.\,9.\,1869 Budapest – 8.\,10.\,1945 Zürich), \emph{Schriftsteller, Journalist, Chefredakteur}!Begräbnis@\strich\emph{Begräbnis}|pw}}{\lemma{\textnormal{\emph{Begräbnis}}}\Cendnote{\textnormal{\emph{Wiener Allgemeine Montags-Zeitung}\pwindex{Wiener Allgemeine Montags-Zeitung@\emph{Wiener Allgemeine Montags-Zeitung}|pwk}, 6. 10. 1899; Erstdruck: \emph{Mährisches Tagblatt}\pwindex{Mährisches Tagblatt@\emph{Mährisches Tagblatt}|pwk}, Jg. 14, Nr. 160,
                        17. 7. 1893, S. 1–2.}}}\label{K_L03299-5}, \label{K_L03299-6v}\edtext{Heldentod\pwindex{Salten, Felix 6.\,9.\,1869 Budapest – 8.\,10.\,1945 Zürich@\textsc{Salten, Felix} (6.\,9.\,1869 Budapest – 8.\,10.\,1945 Zürich), \emph{Schriftsteller, Journalist, Chefredakteur}!Heldentod@\strich\emph{Heldentod}|pw}}{\lemma{\textnormal{\emph{Heldentod}}}\Cendnote{\textnormal{\emph{Wiener Allgemeine Zeitung}\pwindex{Wiener Allgemeine Zeitung@\emph{Wiener Allgemeine Zeitung}|pwk}, Nr. 5044,
                        1. 1. 1895, Neujahrs-Beilage,
                  S. 3–4.}}}\label{K_L03299-6}, \label{K_L03299-7v}\edtext{Fernen\pwindex{Salten, Felix 6.\,9.\,1869 Budapest – 8.\,10.\,1945 Zürich@\textsc{Salten, Felix} (6.\,9.\,1869 Budapest – 8.\,10.\,1945 Zürich), \emph{Schriftsteller, Journalist, Chefredakteur}!Fernen@\strich\emph{Fernen}|pw}}{\lemma{\textnormal{\emph{Fernen}}}\Cendnote{\textnormal{\emph{Wiener Allgemeine Zeitung}\pwindex{Wiener Allgemeine Zeitung@\emph{Wiener Allgemeine Zeitung}|pwk}, Nr. 5947,
                        25. 12. 1897, Weihnachts-Beilage,
                     S. [3–4].}}}\label{K_L03299-7}, Sedan\pwindex{Salten, Felix 6.\,9.\,1869 Budapest – 8.\,10.\,1945 Zürich@\textsc{Salten, Felix} (6.\,9.\,1869 Budapest – 8.\,10.\,1945 Zürich), \emph{Schriftsteller, Journalist, Chefredakteur}!Sedan@\strich\emph{Sedan}|pw}, \label{K_L03299-8v}\edtext{Lebenszeit\pwindex{Salten, Felix 6.\,9.\,1869 Budapest – 8.\,10.\,1945 Zürich@\textsc{Salten, Felix} (6.\,9.\,1869 Budapest – 8.\,10.\,1945 Zürich), \emph{Schriftsteller, Journalist, Chefredakteur}!Lebenszeit@\strich\emph{Lebenszeit}|pw}}{\lemma{\textnormal{\emph{Lebenszeit}}}\Cendnote{\textnormal{Erstdruck vor der Buchausgabe\pwindex{Salten, Felix 6.\,9.\,1869 Budapest – 8.\,10.\,1945 Zürich@\textsc{Salten, Felix} (6.\,9.\,1869 Budapest – 8.\,10.\,1945 Zürich), \emph{Schriftsteller, Journalist, Chefredakteur}!Hinterbliebene. Kurze Novellen@\strich\emph{Der Hinterbliebene. Kurze Novellen}|pwkv} unbekannt}}}\label{K_L03299-8}. Bitte,
               sagen Sie mir, was Sie davon halten, ob nämlich all diese Dinge nicht doch zu
               werthlos sind. (Nicht Affectation) Aber ich glaube, \uline{wenn} ich sie überhaupt als Buch erscheinen laße, dann will ichs jetzt thun,
               denn später, wenn Anderes fertig ist, {\pb}werde ichs gewiss nicht mehr
               wollen.\pend
           
\pstart
           Wann kommen Sie nach \label{K_L03299-9v}\edtext{Wien\oindex{Wien@\textbf{Wien}, \emph{Verwaltungsgebiet}|pw}}{\lemma{\textnormal{\emph{Wien}}}\Cendnote{\textnormal{Schnitzler kam erst am 12. 10. 1899 wieder
                  nach Wien\oindex{Wien@\textbf{Wien}, \emph{Verwaltungsgebiet}|pwk} zurück.}}}\label{K_L03299-9}?\pend
           
\pstart
           Herzlichst {\\[\baselineskip]}Ihr {\\[\baselineskip]}\spacefill\mbox{Salten}\pend
           \leftskip=0em{}
\pstart
           \noindent{}Grüßen Sie \label{K_L03299-10v}\edtext{Hugo\pwindex{Hofmannsthal, Hugo von 1.\,2.\,1874 Wien – 15.\,7.\,1929 Rodaun@\textsc{Hofmannsthal, Hugo von} (1.\,2.\,1874 Wien – 15.\,7.\,1929 Rodaun), \emph{Schriftsteller}|pw}}{\lemma{\textnormal{\emph{Hugo}}}\Cendnote{\textnormal{Hugo von Hofmannsthal\pwindex{Hofmannsthal, Hugo von 1.\,2.\,1874 Wien – 15.\,7.\,1929 Rodaun@\textsc{Hofmannsthal, Hugo von} (1.\,2.\,1874 Wien – 15.\,7.\,1929 Rodaun), \emph{Schriftsteller}|pwk} war am 22. 8. 1899 in Ischl\oindex{Bad Ischl@\textbf{Bad Ischl}|pwk} angekommen.}}}\label{K_L03299-10}.\pend
           \selectlanguage{ngerman}\endnumbering\briefempfaengerindex{Schnitzler, Arthur@\textsc{Schnitzler, Arthur}!zzzSalten, Felix@\emph{von Felix Salten}!1899-08-291@{{[}29. 8. 1899{]}}|)be}\mylabel{L03299h}  \newcommand{\dateiname}{L03299}\newcommand{\titel}{Felix Salten an Arthur Schnitzler, [29. 8. 1899]}\newcommand{\editorInnen}{Martin Anton Müller und Laura Untner}%% latex-leseansicht-abspann.tex
%% Abspann für die Leseansicht.
%% Der Schalter \ifkorrekturansicht ist bereits durch den Vorspann gesetzt.

%% latex-abspann.tex
%% Gemeinsamer Abspann für Korrekturansicht und Leseansicht.
%% Setzt den Schalter \ifkorrekturansicht voraus (gesetzt in den
%% einbindenden Dateien latex-korrekturansicht-abspann.tex bzw.
%% latex-leseansicht-abspann.tex).
%% ---------------------------------------------------------------

\normalsize

% Das esempio-Environment wird nur in der Leseansicht benötigt
\ifkorrekturansicht\else
\newenvironment{esempio}[3]%
{
    \vspace{1.5ex}
    \rlap{\underline{#1}}
    \par
    \setlength{\parindent}{0cm}
    \nopagebreak
    \leftskip=#2cm
    \rightskip=#3cm
}
{
    \par
}
\fi

\doendnotes{C}
\bigskip
\vfill

\clearpage

\footnotesize

\ifkorrekturansicht
  \lohead{\textsc{register}}
\fi

% theindex-Environment neu definieren ohne reledmac
\makeatletter
\renewenvironment{theindex}{%
  \ifkorrekturansicht
    \section*{\indexname}%
  \else
    \subsubsection*{Index der erwähnten Entitäten}%
  \fi
  \setlength{\parindent}{0pt}%
  \setlength{\parskip}{0pt plus 0.3pt}%
  \let\item\@idxitem
}{%
  \ifkorrekturansicht\clearpage\fi
}
\makeatother

\IfFileExists{\jobname-pw.ind}{\input{\jobname-pw.ind}}{}

% Quellenangabe nur in der Leseansicht
\ifkorrekturansicht\else
% Fallback-Definitionen, falls die .tex-Datei \titel etc. nicht gesetzt hat
\providecommand{\titel}{}
\providecommand{\editorInnen}{}
\providecommand{\dateiname}{\jobname}

\vspace{3cm}

\vfill

\footnotesize
\textsc{Quelle}: \titel. Herausgegeben von {\editorInnen}. In: \emph{Arthur Schnitzler: Briefwechsel mit Autorinnen und Autoren}.
 Digitale Edition, https://schnitzler-briefe.acdh.oeaw.ac.at/{\dateiname}.html (Stand \today)
\fi

\end{document}


