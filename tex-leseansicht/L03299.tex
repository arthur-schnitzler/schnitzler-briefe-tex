%% latex-leseansicht-vorspann.tex
%% Vorspann für die Leseansicht.
%% Lädt die gemeinsame Datei latex-vorspann.tex mit nicht gesetztem Schalter.

\newif\ifkorrekturansicht
\korrekturansichtfalse

\input{../tex-inputs/latex-vorspann}

\begin{center}
            \textcolor{red}{ENTWURF, NICHT FERTIG KORRIGIERT}
                      \end{center}
            
         
         \renewcommand{\erwaehntePersonen}{Personen: Hugo von Hofmannsthal}
         \renewcommand{\erwaehnteInstitutionen}{Institutionen: Wiener Allgemeine Zeitung, Wiener Verlag}
         \renewcommand{\erwaehnteOrte}{Orte: Wien}
         \renewcommand{\erwaehnteWerke}{Werke: Begräbnis, Das Manhard-Zimmer, Der Hinterbliebene, Der Hinterbliebene. Kurze Novellen, Fernen, Flucht, Heldentod. Novelle, Lebenszeit, Sedan}
               \section[Felix Salten an Arthur Schnitzler, {[}29. 8. 1899{]}]{ Felix Salten an Arthur Schnitzler, {[}29. 8. 1899{]}}\nopagebreak\mylabel{v}\rehead{ }\begin{ledgroupsized}[t]{13cm}\normalsize\beginnumbering \toendnotes[C]{\smallbreak\pagebreak[2]} \Standort{CUL, Schnitzler, B 89, A 2.}
\physDesc{Brief, 1 Blatt, 2 Seiten, 788 Zeichen
\newline{}Handschrift: schwarze Tinte, lateinische Kurrent
\newline{}Schnitzler: mit Bleistift datiert: »29/8 9\textcolor{gray}{9}« 
\newline{}Ordnung: mit Bleistift von unbekannter Hand nummeriert:
                                    »123« }\toendnotes[C]{\smallbreak}\pstart
           \raggedleft{}{\pb}Dienstag. \pend
           \pstart
           Lieber, ich sende Ihnen gleichzeitig die versprochenen Zeitungen,
               und bitte Sie, mir gelegentlich zu sagen, was Sie drüber denken, und wie Sie glauben,
               dass mans besser machen könnte. Haben Sie sich über die Pneumatik sehr geärgert? Ich
               habe mit der Zeitung\orgindex{Wiener Allgemeine Zeitung@Wiener Allgemeine Zeitung|pwv} sehr viel
               zu thun, arbeite aber gleichwol ziemlich viel. Ich denke ernsthaft daran, die \label{K_L03299-1v}\edtext{Novellen herauszugeben}{\lemma{\textnormal{\emph{Novellen herauszugeben}}}\Cendnote{\textnormal{Die im Folgenden aufgezählten sieben
                  Novellen wurden zusammen mit einer achten – \emph{Das
                     Manhard-Zimmer}\pwindex{Salten, Felix 06.09.1869 – 08.10.1945@\textsc{Salten, Felix} (06.09.1869 – 08.10.1945), \emph{Schriftsteller, Journalist}!Manhard-ZimmerNone@\strich\emph{Das Manhard-Zimmer} {[}None{]}|pwk} – zum Novellenband \emph{Der
                     Hinterbliebene. Kurze Novellen}\pwindex{Salten, Felix 06.09.1869 – 08.10.1945@\textsc{Salten, Felix} (06.09.1869 – 08.10.1945), \emph{Schriftsteller, Journalist}!Hinterbliebene. Kurze Novellen1900@\strich\emph{Der Hinterbliebene. Kurze Novellen} {[}1900{]}|pwk} vereinigt, der 1900 im \emph{Wiener Verlag}\orgindex{Wiener Verlag@Wiener Verlag|pwk} erschien. Auch das \emph{Das Manhard-Zimmer}\pwindex{Salten, Felix 06.09.1869 – 08.10.1945@\textsc{Salten, Felix} (06.09.1869 – 08.10.1945), \emph{Schriftsteller, Journalist}!Manhard-ZimmerNone@\strich\emph{Das Manhard-Zimmer} {[}None{]}|pwk} dürfte Salten\pwindex{Salten, Felix 06.09.1869 – 08.10.1945@\textsc{Salten, Felix} (06.09.1869 – 08.10.1945), \emph{Schriftsteller, Journalist}|pwk}s Sendung beigelegen haben, da Schnitzler\pwindex{Schnitzler, Arthur 15.05.1862 – 21.10.1931@\textsc{Schnitzler, Arthur} (15.05.1862 – 21.10.1931), \emph{Schriftsteller, Mediziner}|pwk} es in seiner Antwort anspricht. Für die meisten
                  Novellen sind Erstdrucke nachgewiesen, aber es ist sehr wahrscheinlich, dass auch
                  die anderen bereits publiziert waren.}}}\label{K_L03299-1h}: Der
                  Hinterbliebene\pwindex{Salten, Felix 06.09.1869 – 08.10.1945@\textsc{Salten, Felix} (06.09.1869 – 08.10.1945), \emph{Schriftsteller, Journalist}!Hinterbliebene1899-03-04 – 1899-03-11@\strich\emph{Der Hinterbliebene} {[}1899-03-04 – 1899-03-11{]}|pw}, Flucht\pwindex{Salten, Felix 06.09.1869 – 08.10.1945@\textsc{Salten, Felix} (06.09.1869 – 08.10.1945), \emph{Schriftsteller, Journalist}!Flucht1900@\strich\emph{Flucht} {[}1900{]}|pw}, Begräbnis\pwindex{Salten, Felix 06.09.1869 – 08.10.1945@\textsc{Salten, Felix} (06.09.1869 – 08.10.1945), \emph{Schriftsteller, Journalist}!Begraebnis17. 7. 1893@\strich\emph{Begräbnis} {[}17. 7. 1893{]}|pw}, Heldentod\pwindex{Salten, Felix 06.09.1869 – 08.10.1945@\textsc{Salten, Felix} (06.09.1869 – 08.10.1945), \emph{Schriftsteller, Journalist}!Heldentod. Novelle1895-01-01@\strich\emph{Heldentod. Novelle} {[}1895-01-01{]}|pw},
                  Fernen\pwindex{Salten, Felix 06.09.1869 – 08.10.1945@\textsc{Salten, Felix} (06.09.1869 – 08.10.1945), \emph{Schriftsteller, Journalist}!Fernen1897-12-25@\strich\emph{Fernen} {[}1897-12-25{]}|pw}, Sedan\pwindex{Salten, Felix 06.09.1869 – 08.10.1945@\textsc{Salten, Felix} (06.09.1869 – 08.10.1945), \emph{Schriftsteller, Journalist}!Sedan1900@\strich\emph{Sedan} {[}1900{]}|pw}, Lebenszeit\pwindex{Salten, Felix 06.09.1869 – 08.10.1945@\textsc{Salten, Felix} (06.09.1869 – 08.10.1945), \emph{Schriftsteller, Journalist}!Lebenszeit1900@\strich\emph{Lebenszeit} {[}1900{]}|pw}. Bitte, sagen Sie
               mir, was Sie davon halten, ob nämlich all diese Dinge nicht doch zu werthlos sind.
               (Nicht Affectation) Aber ich glaube, \uline{wenn} ich sie
               überhaupt als Buch erscheinen laße, dann will ichs jetzt thun, denn später, wenn
               anderes fertig ist, {\pb}werde ichs
               gewiss nicht mehr wollen. \pend
           \pstart
           Wann kommen Sie nach Wien\oindex{Wien@\textbf{Wien}|pw}? \pend
           \pstart
           Herzlichst {\\[\baselineskip]}Ihr {\\[\baselineskip]}\spacefill\mbox{Salten}\pend
           \leftskip=0em{}\pstart
           \noindent{}Grüßen Sie Hugo\pwindex{Hofmannsthal, Hugo von 1874-02-01 – 1929-07-15@\textsc{Hofmannsthal, Hugo von} (1874-02-01 – 1929-07-15), \emph{Schriftsteller}|pw}.\pend
           
         
         \endnumbering\mylabel{h}\end{ledgroupsized}\begin{anhang}\end{anhang}\newcommand{\dateiname}{L03299}\newcommand{\titel}{Felix Salten an Arthur Schnitzler, [29. 8. 1899]}\newcommand{\editorInnen}{Martin Anton Müller und Laura Untner}%% latex-leseansicht-abspann.tex
%% Abspann für die Leseansicht.
%% Der Schalter \ifkorrekturansicht ist bereits durch den Vorspann gesetzt.

%% latex-abspann.tex
%% Gemeinsamer Abspann für Korrekturansicht und Leseansicht.
%% Setzt den Schalter \ifkorrekturansicht voraus (gesetzt in den
%% einbindenden Dateien latex-korrekturansicht-abspann.tex bzw.
%% latex-leseansicht-abspann.tex).
%% ---------------------------------------------------------------

\normalsize

% Das esempio-Environment wird nur in der Leseansicht benötigt
\ifkorrekturansicht\else
\newenvironment{esempio}[3]%
{
    \vspace{1.5ex}
    \rlap{\underline{#1}}
    \par
    \setlength{\parindent}{0cm}
    \nopagebreak
    \leftskip=#2cm
    \rightskip=#3cm
}
{
    \par
}
\fi

\doendnotes{C}
\bigskip
\vfill

\clearpage

\footnotesize

\ifkorrekturansicht
  \lohead{\textsc{register}}
\fi

% theindex-Environment neu definieren ohne reledmac
\makeatletter
\renewenvironment{theindex}{%
  \ifkorrekturansicht
    \section*{\indexname}%
  \else
    \subsubsection*{Index der erwähnten Entitäten}%
  \fi
  \setlength{\parindent}{0pt}%
  \setlength{\parskip}{0pt plus 0.3pt}%
  \let\item\@idxitem
}{%
  \ifkorrekturansicht\clearpage\fi
}
\makeatother

\IfFileExists{\jobname-pw.ind}{\input{\jobname-pw.ind}}{}

% Quellenangabe nur in der Leseansicht
\ifkorrekturansicht\else
% Fallback-Definitionen, falls die .tex-Datei \titel etc. nicht gesetzt hat
\providecommand{\titel}{}
\providecommand{\editorInnen}{}
\providecommand{\dateiname}{\jobname}

\vspace{3cm}

\vfill

\footnotesize
\textsc{Quelle}: \titel. Herausgegeben von {\editorInnen}. In: \emph{Arthur Schnitzler: Briefwechsel mit Autorinnen und Autoren}.
 Digitale Edition, https://schnitzler-briefe.acdh.oeaw.ac.at/{\dateiname}.html (Stand \today)
\fi

\end{document}


      