%% latex-leseansicht-vorspann.tex
%% Vorspann für die Leseansicht.
%% Lädt die gemeinsame Datei latex-vorspann.tex mit nicht gesetztem Schalter.

\newif\ifkorrekturansicht
\korrekturansichtfalse

\input{../tex-inputs/latex-vorspann}


\section[Arthur Schnitzler an Theodor Herzl, 5. 8. 1892]{L03901 Arthur Schnitzler an Theodor Herzl, 5. 8. 1892}
\nopagebreak\mylabel{L03901v}
\rehead{ }\normalsize\beginnumbering\briefempfaengerindex{Herzl, Theodor@\textsc{Herzl, Theodor}!zzzSchnitzler, Arthur@\emph{von Arthur Schnitzler}!1892-08-051@{5. 8. 1892}|(be}
\toendnotes[C]{\smallbreak\pagebreak[2]}
\correspDesc{Versand  durch Arthur Schnitzler am 5. 8. 1892 in Wien
\newline{}Erhalt  durch Theodor Herzl in Wien}\toendnotes[C]{\smallbreak}
\Standort{Jerusalem, Central Zionist Archives, H1:1924-2.}
\physDesc{Brief, 4 Blätter, 16 Seiten, 5586 Zeichen
\newline{}Handschrift: schwarze Tinte, deutsche Kurrent
\newline{}Ordnung: mit Bleistift von unbekannter Hand nummeriert: »(2)« und innerhalb das Konvoluts paginiert: »3«–»19« }
\buchAbdrucke{\weitereDrucke{1) \pwindex{Kellner, Leon 17.\,4.\,1859 Tarnów – 5.\,12.\,1928 Wien@\textsc{Kellner, Leon} (17.\,4.\,1859 Tarnów – 5.\,12.\,1928 Wien), \emph{Zionist, Literaturhistoriker, Anglist}!Theodor Herzls Lehrjahre (1860–1895). Nach den handschriftlichen Quellen@\strich\emph{Theodor Herzls Lehrjahre (1860–1895). Nach den handschriftlichen Quellen}|pwk}Leon Kellner: \emph{Theodor Herzls Lehrjahre (1860–1895). Nach den
                        handschriftlichen Quellen}. Wien, Berlin: \emph{R. Löwit-Verlag} 1920, S. 108–111.} \weitereDrucke{2) H. M. [=Hermann Menkes]: \emph{Briefwechsel zwischen Theodor Herzl und Artur Schnitzler. Die Lehrjahre des berühmten Zionisten.} In: \emph{Neues Wiener Journal}, Jg. 28, Nr. 9540, 29. 5. 1920, S. 3–4.} \weitereDrucke{3) Arthur Schnitzler: \emph{Briefe 1875–1912}. Herausgegeben von Therese Nickl und Heinrich Schnitzler. Frankfurt am Main: \emph{S. Fischer} 1981, S. 124–126.} }\toendnotes[C]{\smallbreak}
\pstart
           \noindent{}{\pb}Verehrteſter Freund, eine Ahnung muſs ich doch
                  i{\geminationm}er gehabt haben, daſs wir einander einmal näher
               kommen, daß Sie mich ſogar »loben« werden; denn es iſt merkwürdig, mit welcher
               Praeciſion mir die einzelnen Etappen unſrer kurzathmigen Beka{\geminationn}tſchaft im Gedächtnis geblieben{ }ſind. Ich weiſs noch –
                  »\label{K_L03901-1v}\edtext{als ich zum erſten Mal dich ſah\pwindex{Scheffel, Joseph Victor von 16.\,2.\,1826 Karlsruhe – 9.\,4.\,1886@\textsc{Scheffel, Joseph Victor von} (16.\,2.\,1826 Karlsruhe – 9.\,4.\,1886), \emph{Schriftsteller, Zeichner, Volkskundler}!Trompeter von Säckingen. Ein Sang vom Oberrhein@\strich\emph{Der Trompeter von Säckingen. Ein Sang vom Oberrhein}|pwv}}{\lemma{\textnormal{\emph{als … dich sah}}}\Cendnote{\textnormal{Vers aus einem Lied aus \emph{Der
                     Trompeter von Säckingen}\pwindex{Scheffel, Joseph Victor von 16.\,2.\,1826 Karlsruhe – 9.\,4.\,1886@\textsc{Scheffel, Joseph Victor von} (16.\,2.\,1826 Karlsruhe – 9.\,4.\,1886), \emph{Schriftsteller, Zeichner, Volkskundler}!Trompeter von Säckingen. Ein Sang vom Oberrhein@\strich\emph{Der Trompeter von Säckingen. Ein Sang vom Oberrhein}|pwk} von Joseph
                     Victor von Scheffel\pwindex{Scheffel, Joseph Victor von 16.\,2.\,1826 Karlsruhe – 9.\,4.\,1886@\textsc{Scheffel, Joseph Victor von} (16.\,2.\,1826 Karlsruhe – 9.\,4.\,1886), \emph{Schriftsteller, Zeichner, Volkskundler}|pwk}, eventuell hier zitiert nach einer der vielen
                  Vertonungen.}}}\label{K_L03901-1}« – das war in der akad.
                  Leſehalle\orgindex{Akademische Lesehalle@Akademische Lesehalle|pw}. Sie hielten eine Rede {\pb}und waren »ſcharf«
               – in einer Weiſe ſcharf! Ich befand mich in Ihrer Nähe und hatte die Empfindung als
                  we{\geminationn} Sie mich mit einem gewiſſen milden Sarkasmus
               betrachteten; Sie lächelten ironiſch – und ich bega{\geminationn} Sie
               zu beneiden. »Wer{ }ſo reden und ſo lächeln könnte« dachte ich
                  mir{[}.{]} Bald darauf hörte ich noch mehr von Ihnen: im Kaffehaus,
               das ich viel eifriger beſuchte als die politiſchen Discuſſions- und Wahlabende {\pb}der Leſehalle\orgindex{Akademische Lesehalle@Akademische Lesehalle|pw}, und wo
               Sie eines beträchtlichen Rufes als Dominoſpieler (hauptſächlich im blinden Domino,
               wie ich leider hinzuſetzen muß) genoſſen. Einige Schöngeiſter ſprachen übrigens
               bereits von Ihrer Bedeutung als dreiaktiger Luſtſpieldichter. Wollen Sie einer Beweis
               für meine literarhiſtoriſche Begabung? Ich weiſs noch genau, daß Siegfried Wertheimer\pwindex{Wertheimer, Siegfried 15.\,7.\,1860 Wien – 11.\,3.\,1919 ebd.@\textsc{Wertheimer, Siegfried} (15.\,7.\,1860 Wien – 11.\,3.\,1919 ebd.)|pw} der erſte war, der mir von dem \uline{Dichter} Herzl ſprach. {\pb}Bald
               darauf lernte ich sie perſönlich ke{\geminationn}en und las zwei
               Ihrer Stücke im Manuscript: \textsc{\label{K_L03901-2v}\edtext{Tabarin\pwindex{Herzl, Theodor 2.\,5.\,1860 Budapest – 3.\,7.\,1904 Edlach@\textsc{Herzl, Theodor} (2.\,5.\,1860 Budapest – 3.\,7.\,1904 Edlach), \emph{Schriftsteller, Journalist}!Tabarin. Schauspiel in einem Act. Frei nach Catulle Mendès@\strich\emph{Tabarin. Schauspiel in einem Act. Frei nach Catulle Mendès}|pw}}{\lemma{\textnormal{\emph{Tabarin}}}\Cendnote{\textnormal{Vgl. A. S.: \emph{Tagebuch}, 1. 2. 1886.}}}\label{K_L03901-2}} und ein zweites – hieſs es nicht »\label{K_L03901-3v}\edtext{die Aufgeregten\pwindex{Herzl, Theodor 2.\,5.\,1860 Budapest – 3.\,7.\,1904 Edlach@\textsc{Herzl, Theodor} (2.\,5.\,1860 Budapest – 3.\,7.\,1904 Edlach), \emph{Schriftsteller, Journalist}!Enttäuschten. Komödie in vier Acten@\strich\emph{Die Enttäuschten. Komödie in vier Acten}|pwu}}{\lemma{\textnormal{\emph{die Aufgeregten}}}\Cendnote{\textnormal{Es dürfte
                  sich um die vieraktige Komödie \emph{Die Enttäuschten}\pwindex{Herzl, Theodor 2.\,5.\,1860 Budapest – 3.\,7.\,1904 Edlach@\textsc{Herzl, Theodor} (2.\,5.\,1860 Budapest – 3.\,7.\,1904 Edlach), \emph{Schriftsteller, Journalist}!Enttäuschten. Komödie in vier Acten@\strich\emph{Die Enttäuschten. Komödie in vier Acten}|pwk} handeln.}}}\label{K_L03901-3}«?
               Und wieder beneidete ich Sie – »wer ſolche Stücke ſchreiben könnte« – (damals ſchrieb
               ich \introOben{}nemlich\introOben{} ganz besti{\geminationm}t ſchlechtere Stücke
               als Sie!–) Aber die ganze Studentenzeit verſtrich, ohne daſs wir ein Verhältnis zu
               einander finden k\substVorne{}\textsuperscript{ö }\substDazwischen{}o\substHinten{}nnten, – offenbar {\pb}wie mir Ihre letzten Zeilen
               beweiſen – weil ich – für Sie zu arrogant war! –\pend
           
\pstart
           – \label{K_L03901-4v}\edtext{In Kammer\oindex{Kammer@\textbf{Kammer}|pw}}{\lemma{\textnormal{\emph{In Kammer}}}\Cendnote{\textnormal{Vgl. A. S.: \emph{Tagebuch}, 8. 8. 1885.}}}\label{K_L03901-4} habe ich
               Sie dann geſprochen, als wir ſchon beide Doktoren waren; Sie waren von einem Kreis
               hübſcher junger Frauen umgeben – und wieder habe ich Sie – hoffentlich nicht ganz
               ohne Grund – »beneidet«. Und auch damals lächelten Sie ironiſch! – Und wieder
               verlieſs ich {\pb}Sie mit jener gedrückten Sti{\geminationm}ung, die man Leuten gegenüber hat, die einem auf
               derſelben Straße zwanzig Schritte weit vorauslaufen. An diese Erinnerung aber reiht{ }ſich eine von denen, die über das perſönliche weit hinaus gehend, in einer Geſchichte
               der modernen Literatur als kleingedruckte Anmerkung einen ſichern Platz \substVorne{}\textsuperscript{findet}\substDazwischen{}verdiente\substHinten{}. Das neue Burgtheater\oindex{Wien@\textbf{Wien}!I., Innere Stadt@\textbf{I., Innere Stadt}!Burgtheater@\textbf{Burgtheater}, \emph{Theater}|pw}{ }{\pb}war noch im Bau; wie ſpazierten an einem Spätherbſtabende
               vor dem Bretterzaun auf u. ab. Natürlich hatten wir uns zufällig getroffen – da es
               uns ja \substVorne{}\textsuperscript{bisher}\substDazwischen{}bis heute\substHinten{} noch nicht gegönnt war, uns je abſichtlich zu begegnen. Da ſagten Sie, mit
               einem beſcheiden erobernden Blick, der auf den emporſteigenden Mauern ruhen blieb: da
               komm’ ich einmal hinein! {\pb}Ja, mein lieber Freund, damals
               wäre der Moment geweſen, mich für Ihr vielfaches ironiſches Lächeln einmal \textsc{pauschaliter} mittelſt eines grauſen Hohnlachens zu
               revanchiren – ich blieb jedoch stumm; ich ka{\geminationn} es
                  \introOben{}nicht\introOben{} läugnen, Sie haben mir damals mehr imponirt als je. Sie werden
               begreifen, daſs ich dieſe kleine Geſchichte, welche \substVorne{}\textsuperscript{ich die}\substDazwischen{}von den\substHinten{} Thatſachen zum Rang {\pb}einer Anekdote emporgehoben
               wurde, jedem Menschen erzähle, der den Namen »Theoder Herzl« ausſpricht. Sie iſt
               aber ſo wahrſcheinlich, daſs Sie alle Welt für erfunden hält. – Ich erinnere mich
               auch eines letzten Zusa{\geminationm}entreffens mit Ihnen – auf
               irgend einem Ball, in einer Nacht, wie Sie{ }ſchon lange, aber ſchon ſehr lang ein
               berühmter Ma{\geminationn} waren, {\pb}während
               ich, an mir, an meinem Beruf – an beiden! – verzweifelnd, von niemand eigentlich
                  ernſt geno{\geminationm}en, meinen Ehrgeiz als »guter
               Gesellschafter« und \textsc{demi mondainer}{ }\introOben{}(im \textsc{Bourget}’schen\pwindex{Bourget, Paul 2.\,9.\,1852 Amiens – 25.\,12.\,1935 Paris@\textsc{Bourget, Paul} (2.\,9.\,1852 Amiens – 25.\,12.\,1935 Paris), \emph{Schriftsteller}|pw} Sinn)\introOben{} befriedigen ſuchte.
               Ich war an jenem Abend beſonders gut gelaunt und, wie ich glaubte, namenlos elegant.
               Da – erſchienen Sie. Mit ruhigen überlegenen Augen prüften Sie meine Cravate – {\pb}und – vernichteten mich. Wiſſen Sie was Sie ſagten –?
                  »\label{K_L03901-5v}\edtext{Und ich hielt Sie für einen – \textsc{Brummel\pwindex{Brummel, George Bryan 7.\,6.\,1778 London – 30.\,3.\,1840 Caen@\textsc{Brummel, George Bryan} (7.\,6.\,1778 London – 30.\,3.\,1840 Caen), \emph{Dandy}|pw}}}{\lemma{\textnormal{\emph{Und … Brummel}}}\Cendnote{\textnormal{In einem \emph{Tagebuch}\pwindex{Schnitzler, Arthur 15.\,5.\,1862 Wien – 21.\,10.\,1931 ebd.@\textsc{Schnitzler, Arthur} (15.\,5.\,1862 Wien – 21.\,10.\,1931 ebd.), \emph{Schriftsteller, Mediziner}!Tagebuch@\strich\emph{Tagebuch}|pwk}-Eintrag zum 10. 12. 1916 nennt Schnitzler die Aussage als 30 Jahre zurückliegend. Eine
                  genauere zeitliche Verortung des Ereignisses ist nicht möglich. In einer
                  autobiografischen Aufzeichnung (\emph{Deutsches Literaturarchiv
                        Marbach}, HS.1985.1.198) wird nur das Objekt der Aufregung
                  näher identifiziert: »eine gerippte weisse
               Kravatte«.}}}\label{K_L03901-5}!!! –« Ich hatte die deutliche Empfindung in Ungemach
               gefallen zu{ }ſein. Es war klar, daſs ich lernen mußte, meine Cravate beſſer zu knüpfen
               oder doch wenigſtens auf einem andern Gebiet etwas hervorrragendes zu leiſten. In
               kühnen Momenten vermaſs ich mich, beiden Zielen zuzuſtreben; – {\pb}vielleicht werde ich Sie auch einmal von meiner
               Cravatenknüpfbegabung zu überzeugen Gelegenheit haben? – Und we{\geminationn} ich nun heute bedenke, daß Sie offenbar darum mit mir
               nicht verkehren ko{\geminationn}ten – weil ich Ihnen dünkelhaft\strikeout{en} vorkam! Und gar Ihnen gegenüber! Ich, der{ }ſich die
                  \textsc{causa Hirschkron\pwindex{Herzl, Theodor 2.\,5.\,1860 Budapest – 3.\,7.\,1904 Edlach@\textsc{Herzl, Theodor} (2.\,5.\,1860 Budapest – 3.\,7.\,1904 Edlach), \emph{Schriftsteller, Journalist}!causa Hirschkorn. Lustspiel in einem Act@\strich\emph{Die causa Hirschkorn. Lustspiel in einem Act}|pw}} aus der Leihbibliothek, das Neue von der
                  Venus\pwindex{Herzl, Theodor 2.\,5.\,1860 Budapest – 3.\,7.\,1904 Edlach@\textsc{Herzl, Theodor} (2.\,5.\,1860 Budapest – 3.\,7.\,1904 Edlach), \emph{Schriftsteller, Journalist}!Neues von der Venus@\strich\emph{Neues von der Venus}|pw} von einem guten Beka{\geminationn}ten\pwindex{?? [Bekannter von Schnitzler, der frühe Werke von Herzl besitzt] @\textsc{?? [Bekannter von Schnitzler, der frühe Werke von Herzl besitzt]}|pwv} ausge{\pb}liehen – und
               der{ }ſich das »Buch der Narrheit\pwindex{Herzl, Theodor 2.\,5.\,1860 Budapest – 3.\,7.\,1904 Edlach@\textsc{Herzl, Theodor} (2.\,5.\,1860 Budapest – 3.\,7.\,1904 Edlach), \emph{Schriftsteller, Journalist}!Buch der Narrheit@\strich\emph{Buch der Narrheit}|pw}« ſogar gekauft
               hat – als es einen Tages in einer Auslage um 15 Xr.{ }ſichtbar wurde. Ich, der zwar vom
                  »Flüchtling\pwindex{Herzl, Theodor 2.\,5.\,1860 Budapest – 3.\,7.\,1904 Edlach@\textsc{Herzl, Theodor} (2.\,5.\,1860 Budapest – 3.\,7.\,1904 Edlach), \emph{Schriftsteller, Journalist}!Flüchtling. Lustspiel in einem Aufzug@\strich\emph{Der Flüchtling. Lustspiel in einem Aufzug}|pw}« behauptete, er könne nur durch
               die Burgtheater\orgindex{Burgtheater@Burgtheater|pw}beſetzung gehalten werden, der
               aber \introOben{}bei\introOben{} dem »Prinzen aus Genieland\pwindex{Herzl, Theodor 2.\,5.\,1860 Budapest – 3.\,7.\,1904 Edlach@\textsc{Herzl, Theodor} (2.\,5.\,1860 Budapest – 3.\,7.\,1904 Edlach), \emph{Schriftsteller, Journalist}!Prinzen aus Genieland. Lustspiel in 4 Akten@\strich\emph{Prinzen aus Genieland. Lustspiel in 4 Akten}|pw}«
               die Anſicht verfocht, daſs ſie in Carltheater\orgindex{Carl-Theater@Carl-Theater|pw} zu
               Grund geſpielt {\pb}würde! – Ich weiſs nicht, ob es mir mit dem
               bisherigen gelungen iſt, Ihnen gerade das zu ſagen, was ich Ihnen ſagen will: daſs es
               wahrhaftig nicht viel Menſchen auf der Welt gibt, auf deren Urtheil ich den gleichen
               Werth legen möchte wie auf das Ihre. Ermeſſen Sie daraus, wie ſehr mich Ihre
               freundliche Anerke{\geminationn}ung gefreut, und wie wohlthuend mich
               beſonders {\pb}der warme und reiche Ton berührt hat, mit welchem
               Sie zu mir ſprechen. Daſs ich Ihnen aber auch perſönlich ſympathisch geworden bin,
               kann ich unmöglich der Beka{\geminationn}tſchaft mit meinem Stück\pwindex{Schnitzler, Arthur 15.\,5.\,1862 Wien – 21.\,10.\,1931 ebd.@\textsc{Schnitzler, Arthur} (15.\,5.\,1862 Wien – 21.\,10.\,1931 ebd.), \emph{Schriftsteller, Mediziner}!Märchen. Schauspiel in drei Aufzügen@\strich\emph{Das Märchen. Schauspiel in drei Aufzügen}|pwv} allein zuſchreiben: da
               hat gewiſs mein Freund Paul\pwindex{Goldmann, Paul 31.\,1.\,1865 Breslau – 25.\,9.\,1935 Wien@\textsc{Goldmann, Paul} (31.\,1.\,1865 Breslau – 25.\,9.\,1935 Wien), \emph{Schriftsteller, Journalist}|pw}, der beſte und
               liebeswürdigſte der Menschen, das ſeinige dazugethan. Ich ſage Ihnen für heute Adieu,
                  {\pb}verehrter Freund, und bitte Sie, meiner herzlichen
               Ergebenheit für alle Zeit verſichert zu ſein.\pend
           \pstart Ihr \spacefill\mbox{Arthur Schnitzler}\pend{}
\pstart
           Wien\oindex{Wien@\textbf{Wien}, \emph{Verwaltungsgebiet}|pw}{ }5. August 92.\pend
           \selectlanguage{ngerman}\endnumbering\briefempfaengerindex{Herzl, Theodor@\textsc{Herzl, Theodor}!zzzSchnitzler, Arthur@\emph{von Arthur Schnitzler}!1892-08-051@{5. 8. 1892}|)be}\mylabel{L03901h}
\begin{anhang}
\end{anhang}\newcommand{\dateiname}{L03901}\newcommand{\titel}{Arthur Schnitzler an Theodor Herzl, 5. 8. 1892}\newcommand{\editorInnen}{Herausgegeben von Jahnke, SelmaMüller, Martin Anton}%% latex-leseansicht-abspann.tex
%% Abspann für die Leseansicht.
%% Der Schalter \ifkorrekturansicht ist bereits durch den Vorspann gesetzt.

%% latex-abspann.tex
%% Gemeinsamer Abspann für Korrekturansicht und Leseansicht.
%% Setzt den Schalter \ifkorrekturansicht voraus (gesetzt in den
%% einbindenden Dateien latex-korrekturansicht-abspann.tex bzw.
%% latex-leseansicht-abspann.tex).
%% ---------------------------------------------------------------

\normalsize

% Das esempio-Environment wird nur in der Leseansicht benötigt
\ifkorrekturansicht\else
\newenvironment{esempio}[3]%
{
    \vspace{1.5ex}
    \rlap{\underline{#1}}
    \par
    \setlength{\parindent}{0cm}
    \nopagebreak
    \leftskip=#2cm
    \rightskip=#3cm
}
{
    \par
}
\fi

\doendnotes{C}
\bigskip
\vfill

\clearpage

\footnotesize

\ifkorrekturansicht
  \lohead{\textsc{register}}
\fi

% theindex-Environment neu definieren ohne reledmac
\makeatletter
\renewenvironment{theindex}{%
  \ifkorrekturansicht
    \section*{\indexname}%
  \else
    \subsubsection*{Index der erwähnten Entitäten}%
  \fi
  \setlength{\parindent}{0pt}%
  \setlength{\parskip}{0pt plus 0.3pt}%
  \let\item\@idxitem
}{%
  \ifkorrekturansicht\clearpage\fi
}
\makeatother

\IfFileExists{\jobname-pw.ind}{\input{\jobname-pw.ind}}{}

% Quellenangabe nur in der Leseansicht
\ifkorrekturansicht\else
% Fallback-Definitionen, falls die .tex-Datei \titel etc. nicht gesetzt hat
\providecommand{\titel}{}
\providecommand{\editorInnen}{}
\providecommand{\dateiname}{\jobname}

\vspace{3cm}

\vfill

\footnotesize
\textsc{Quelle}: \titel. Herausgegeben von {\editorInnen}. In: \emph{Arthur Schnitzler: Briefwechsel mit Autorinnen und Autoren}.
 Digitale Edition, https://schnitzler-briefe.acdh.oeaw.ac.at/{\dateiname}.html (Stand \today)
\fi

\end{document}


