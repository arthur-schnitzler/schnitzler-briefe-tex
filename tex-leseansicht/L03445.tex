%% latex-leseansicht-vorspann.tex
%% Vorspann für die Leseansicht.
%% Lädt die gemeinsame Datei latex-vorspann.tex mit nicht gesetztem Schalter.

\newif\ifkorrekturansicht
\korrekturansichtfalse

\input{../tex-inputs/latex-vorspann}

\begin{center}
            \textcolor{red}{ENTWURF, NICHT FERTIG KORRIGIERT}
                      \end{center}
            
         
         \renewcommand{\erwaehntePersonen}{Personen:  Eduard VII., Hugo von Hofmannsthal, Detlev von Liliencron, Max Reinhardt, Theodore Rottenberg, Ludwig Rottenberg, Olga Schnitzler, Heinrich Schnitzler,  Wilhelm II. von Preußen}
         \renewcommand{\erwaehnteInstitutionen}{Institutionen: Deutsches Theater Berlin}
         \renewcommand{\erwaehnteOrte}{Orte: Berlin, Burgtheater, Dessauer Straße, Deutschland, Edmund-Weiß-Gasse, England, Frankfurt am Main, Kiel, Marienbad, Pompei, Taormina, Wien}
         \renewcommand{\erwaehnteWerke}{Werke: Zwischenspiel. Komödie in drei Akten}
               \section[ Paul Goldmann an Arthur Schnitzler, 23. 6. {[}1904{]}]{ Paul Goldmann an Arthur Schnitzler, 23. 6. {[}1904{]}}\nopagebreak\mylabel{v}\rehead{ }\begin{ledgroupsized}[t]{13cm}\normalsize\beginnumbering \toendnotes[C]{\smallbreak\pagebreak[2]} \Standort{DLA, A:Schnitzler, HS.NZ85.1.3174.}
\physDesc{Brief, 1 Blatt, 4 Seiten
\newline{}Handschrift: blaue Tinte, deutsche Kurrent
\newline{}Schnitzler: 1) mit Bleistift das Jahr »{[}1{]}904« vermerkt  2) mit rotem Buntstift zwei Unterstreichungen}\toendnotes[C]{\smallbreak}\pstart
           \noindent{}\raggedleft{}{\pb}\textcolor{gray}{\textbf{DESSAUERSTRASSE 19\oindex{Dessauer Strasse@\textbf{Dessauer Straße}|pw}}}\pend
           \pstart
           Berlin\oindex{Berlin@\textbf{Berlin}|pw}, 23. Juni.\pend
           \pstart\center{}Mein lieber Freund,\pend\pstart
           Ich habe mich ſehr gefreut, zu erſehen, daß Ihr, Du und Deine Frau\pwindex{Schnitzler, Olga 17.01.1882 – 13.01.1970@\textsc{Schnitzler, Olga} (17.01.1882 – 13.01.1970), \emph{Schauspielerin, Sängerin}|pwv}, wohlbehalten zurückgekommen ſeid und
               daß Eure \label{K_L03445-1v}\edtext{Reiſe}{\lemma{\textnormal{\emph{Reiſe}}}\Cendnote{\textnormal{siehe Paul Goldmann an Arthur Schnitzler, 14. 3. [1904]}}}\label{K_L03445-1h} ſo ſchön verlaufen iſt. Und bei der Rückkehr aus \textsc{Taormina\oindex{Taormina@\textbf{Taormina}|pw}} und \textsc{Pompeji\oindex{Pompei@\textbf{Pompei}|pw}} zu Hauſe einen blondlockigen Sohn\pwindex{Schnitzler, Heinrich 09.08.1902 – 12.07.1982@\textsc{Schnitzler, Heinrich} (09.08.1902 – 12.07.1982), \emph{Regisseur, Schauspieler}|pwv} vorzufinden, iſt auch nicht übel.\pend
           \pstart
           Ob mich mein Weg dieſes Jahr nach \label{K_L03445-2v}\edtext{Wien\oindex{Wien@\textbf{Wien}|pw}}{\lemma{\textnormal{\emph{Wien}}}\Cendnote{\textnormal{Goldmann\pwindex{Goldmann, Paul 31.01.1865 – 25.09.1935@\textsc{Goldmann, Paul} (31.01.1865 – 25.09.1935), \emph{Schriftsteller, Journalist}|pwk} war jedenfalls am 10. 8. 1904 und am 11. 8. 1904 in Wien\oindex{Wien@\textbf{Wien}|pwk}. Am 11. 8. 1904 besuchte er Arthur und Olga Schnitzler\pwindex{Schnitzler, Arthur 15.05.1862 – 21.10.1931@\textsc{Schnitzler, Arthur} (15.05.1862 – 21.10.1931), \emph{Schriftsteller, Mediziner}|pwk}\pwindex{Schnitzler, Olga 17.01.1882 – 13.01.1970@\textsc{Schnitzler, Olga} (17.01.1882 – 13.01.1970), \emph{Schauspielerin, Sängerin}|pwk}. Im September war
                  er noch einmal in Wien, vgl. A. S.: \emph{Tagebuch}, 21. 9. 1904.}}}\label{K_L03445-2h} führen wird, iſt fraglich. Sollte es der Fall ſein, ſo
               wird es mir natürlich eine große Freude ſein, Dich dort wiederzuſehen. {\pb}Bei Marienbad\oindex{Marienbad@\textbf{Marienbad}|pw}
               bleibt es wahrſcheinlich. Was hinterher noch geſchechen wird, iſt ganz ungewiß.
               Sobald ich Genaueres weiß, theile ich es Dir mit; und es wäre ſehr ſchön, wenn ſich
               eine Möglichkeit finden ließe, Dich unterwegs zu treffen.\pend
           \pstart
           Jetzt im Sommer werden ſich wohl wieder alle Vorzüge Eurer prachtvoll gelegenen
                  \label{K_L03445-3v}\edtext{Wohnung\oindex{Edmund-Weiss-Gasse@\textbf{Edmund-Weiß-Gasse}|pwv}}{\lemma{\textnormal{\emph{Wohnung}}}\Cendnote{\textnormal{siehe Felix Salten an Arthur Schnitzler, 17. 9. 1903}}}\label{K_L03445-3h} entfalten, und ich wünſche Dir eine Reihe guter Arbeitsſtunden auf Deiner
               Veranda mit dem Blick ins {\pb}Grüne. Schreibſt Du ein
                  \label{K_L03445-4v}\edtext{neues Stück}{\lemma{\textnormal{\emph{neues Stück}}}\Cendnote{\textnormal{Das nächste große dramatische Werk, an dem Schnitzler\pwindex{Schnitzler, Arthur 15.05.1862 – 21.10.1931@\textsc{Schnitzler, Arthur} (15.05.1862 – 21.10.1931), \emph{Schriftsteller, Mediziner}|pwk} arbeitete, war die Komödie \emph{Zwischenspiel}\pwindex{Schnitzler, Arthur 15.05.1862 – 21.10.1931@\textsc{Schnitzler, Arthur} (15.05.1862 – 21.10.1931), \emph{Schriftsteller, Mediziner}!Zwischenspiel. Komoedie in drei Akten1905-10-12@\strich\emph{Zwischenspiel. Komödie in drei Akten} {[}1905-10-12{]}|pwk}. Sie wurde am 12. 10. 1905 am Burgtheater\oindex{Burgtheater@\textbf{Burgtheater}|pwk} uraufgeführt.}}}\label{K_L03445-4h}? Und gedenkſt
               Du Dich \substVorne{}\textsuperscript{damit}\substDazwischen{}damit\substHinten{} an dem \label{K_L03445-5v}\edtext{Wet\strikeout{t}kampf der Theater}{\lemma{\textnormal{\emph{Wettkampf der Theater}}}\Cendnote{\textnormal{womöglich Bezug auf die Übernahme des \emph{Deutschen Theater}\orgindex{Deutsches Theater Berlin@Deutsches Theater Berlin|pwk}s durch Max Reinhardt\pwindex{Reinhardt, Max 09.09.1873 – 30.10.1943@\textsc{Reinhardt, Max} (09.09.1873 – 30.10.1943), \emph{Theaterleiter, Regisseur, Schauspieler}|pwk} im Oktober 1905}}}\label{K_L03445-5h} zu betheligen, der im kommenden Winter in Berlin\oindex{Berlin@\textbf{Berlin}|pw} mit noch nicht dageweſener Heftigkeit entbrennen wird?\pend
           \pstart
           Meine Freundin\pwindex{Rottenberg, Theodore 1875-09-07 – 1945-04-05@\textsc{Rottenberg, Theodore} (1875-09-07 – 1945-04-05)|pwv} erwidert
               herzlich Deinen Gruß. Es geht \strikeout{ih} ihr, wie es ihr
               ging. Sie leidet ſchwer unter den unerträglichen Verhältniſſen ihrer Ehe\pwindex{Rottenberg, Ludwig 11.10.1864 – 6.5.1932@\textsc{Rottenberg, Ludwig} (11.10.1864 – 6.5.1932), \emph{Kapellmeister}|pwv} und der Enge und gemeinen
               Klatſchſucht der Kleinſtadt\oindex{Frankfurt am Main@\textbf{Frankfurt am Main}|pwv}.
               Sie ſehnt ſich danach, ſich mit mir zu vereinigen; ich ſehne mich nach ihr. Aber die
                  {\pb}materiellen Verhältniſſe erlauben es nicht,
               dieſe beiderfeitige Sehnſucht endgiltig zu befriedigen. Und die
               L\textcolor{gray}{ö}ſung iſt nach wie vor: Fortwurſteln{\dotsfour}\pend
           \pstart
           Daß Ihr \textsc{Hoffmannsthal\pwindex{Hofmannsthal, Hugo von 1874-02-01 – 1929-07-15@\textsc{Hofmannsthal, Hugo von} (1874-02-01 – 1929-07-15), \emph{Schriftsteller}|pw}} in der \label{K_L03445-6v}\edtext{\textsc{Liliencron\pwindex{Liliencron, Detlev von 03.06.1844 – 22.07.1909@\textsc{Liliencron, Detlev von} (03.06.1844 – 22.07.1909)|pw}}-Affaire}{\lemma{\textnormal{\emph{Liliencron-Affaire}}}\Cendnote{\textnormal{siehe Hugo von Hofmannsthal an Arthur Schnitzler, 1[9?]. 6. [1904] und A. S.: \emph{Tagebuch}, 2. 6. 1904}}}\label{K_L03445-6h} Unrecht gebt, erfreut mich ebenſoſehr, wie es mich überraſcht.\pend
           \pstart
           Ich fahre heut{ }Mittag nach \textsc{Kiel\oindex{Kiel@\textbf{Kiel}|pw}}, um über die \label{K_L03445-7v}\edtext{Monarchen\pwindex{Eduard VII. 09.09.1841 – 06.05.1910@\textsc{Eduard VII.} (09.09.1841 – 06.05.1910), \emph{Regent/Regentin >> König/Königin}|pwv}\pwindex{Wilhelm II. von Preussen 27.1.1859 – 4.6.1941@\textsc{Wilhelm II. von Preußen} (27.1.1859 – 4.6.1941), \emph{Kaiser}|pwv}-Zuſammenkunft}{\lemma{\textnormal{\emph{Monarchen-Zuſammenkunft}}}\Cendnote{\textnormal{Bezug
                  auf den Besuch des engl\oindex{England@\textbf{England}|pwkv}ischen Königs Eduard VII.\pwindex{Eduard VII. 09.09.1841 – 06.05.1910@\textsc{Eduard VII.} (09.09.1841 – 06.05.1910), \emph{Regent/Regentin >> König/Königin}|pwk} im
                  Rahmen der Kiel\oindex{Kiel@\textbf{Kiel}|pwk}er Woche und dem
                  Zusammentreffen mit dessen Neffen, dem deutsch\oindex{Deutschland@\textbf{Deutschland}|pwkv}en Kaiser Wilhelm
                     II.\pwindex{Wilhelm II. von Preussen 27.1.1859 – 4.6.1941@\textsc{Wilhelm II. von Preußen} (27.1.1859 – 4.6.1941), \emph{Kaiser}|pwk}}}}\label{K_L03445-7h} zu berichten.\pend
           \pstart
           Herzliche Grüße an Dich {\\[\baselineskip]}und Deine Frau\pwindex{Schnitzler, Olga 17.01.1882 – 13.01.1970@\textsc{Schnitzler, Olga} (17.01.1882 – 13.01.1970), \emph{Schauspielerin, Sängerin}|pwv} von Deinem getreuen {\\[\baselineskip]}\spacefill\mbox{Paul Goldmann}\pend
           \leftskip=0em{}
         
         \endnumbering\mylabel{h}\end{ledgroupsized}\begin{anhang}\end{anhang}\newcommand{\dateiname}{L03445}\newcommand{\titel}{Paul Goldmann an Arthur Schnitzler, 23. 6. [1904]}\newcommand{\editorInnen}{Martin Anton Müller und Laura Untner}%% latex-leseansicht-abspann.tex
%% Abspann für die Leseansicht.
%% Der Schalter \ifkorrekturansicht ist bereits durch den Vorspann gesetzt.

%% latex-abspann.tex
%% Gemeinsamer Abspann für Korrekturansicht und Leseansicht.
%% Setzt den Schalter \ifkorrekturansicht voraus (gesetzt in den
%% einbindenden Dateien latex-korrekturansicht-abspann.tex bzw.
%% latex-leseansicht-abspann.tex).
%% ---------------------------------------------------------------

\normalsize

% Das esempio-Environment wird nur in der Leseansicht benötigt
\ifkorrekturansicht\else
\newenvironment{esempio}[3]%
{
    \vspace{1.5ex}
    \rlap{\underline{#1}}
    \par
    \setlength{\parindent}{0cm}
    \nopagebreak
    \leftskip=#2cm
    \rightskip=#3cm
}
{
    \par
}
\fi

\doendnotes{C}
\bigskip
\vfill

\clearpage

\footnotesize

\ifkorrekturansicht
  \lohead{\textsc{register}}
\fi

% theindex-Environment neu definieren ohne reledmac
\makeatletter
\renewenvironment{theindex}{%
  \ifkorrekturansicht
    \section*{\indexname}%
  \else
    \subsubsection*{Index der erwähnten Entitäten}%
  \fi
  \setlength{\parindent}{0pt}%
  \setlength{\parskip}{0pt plus 0.3pt}%
  \let\item\@idxitem
}{%
  \ifkorrekturansicht\clearpage\fi
}
\makeatother

\IfFileExists{\jobname-pw.ind}{\input{\jobname-pw.ind}}{}

% Quellenangabe nur in der Leseansicht
\ifkorrekturansicht\else
% Fallback-Definitionen, falls die .tex-Datei \titel etc. nicht gesetzt hat
\providecommand{\titel}{}
\providecommand{\editorInnen}{}
\providecommand{\dateiname}{\jobname}

\vspace{3cm}

\vfill

\footnotesize
\textsc{Quelle}: \titel. Herausgegeben von {\editorInnen}. In: \emph{Arthur Schnitzler: Briefwechsel mit Autorinnen und Autoren}.
 Digitale Edition, https://schnitzler-briefe.acdh.oeaw.ac.at/{\dateiname}.html (Stand \today)
\fi

\end{document}


      