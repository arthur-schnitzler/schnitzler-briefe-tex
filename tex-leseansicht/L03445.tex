%% latex-korrekturansicht-vorspann.tex
%% Vorspann für die Korrekturansicht.
%% Lädt die gemeinsame Datei latex-vorspann.tex mit gesetztem Schalter.

\newif\ifkorrekturansicht
\korrekturansichttrue

\input{../tex-inputs/latex-vorspann}


\section[ Paul Goldmann an Arthur Schnitzler, 23. 6. {[}1904{]}]{L03445 Paul Goldmann an Arthur Schnitzler, 23. 6. {[}1904{]}}
\nopagebreak\mylabel{L03445v}
\rehead{ }\normalsize\beginnumbering\briefempfaengerindex{Schnitzler, Arthur@\textsc{Schnitzler, Arthur}!zzzGoldmann, Paul@\emph{von Paul Goldmann}!1904-06-232@{23. 6. {[}1904{]}}|(be}
\toendnotes[C]{\smallbreak\pagebreak[2]}\Standort{DLA, A:Schnitzler, HS.NZ85.1.3174.}
\physDesc{Brief, 1 Blatt, 4 Seiten, 1736 Zeichen
\newline{}Handschrift: blaue Tinte, deutsche Kurrent
\newline{}Schnitzler: 1) mit Bleistift das Jahr »904« vermerkt  2) mit rotem Buntstift zwei Unterstreichungen}\toendnotes[C]{\smallbreak}
\pstart
           \raggedleft{}{\pb}\textcolor{gray}{\textbf{DESSAUERSTRASSE 19\oindex{Dessauer Strasse@\textbf{Dessauer Straße}, \emph{Straße (K.STR)}|pw}}}\pend
           
\pstart
           Berlin\oindex{Berlin@\textbf{Berlin}, \emph{P.PPLC}|pw}, 23. Juni.\pend
           
\pstart\center{}Mein lieber Freund,\pend\vspace{0.5em}
\pstart
           Ich habe mich ſehr gefreut, zu erſehen, daß Ihr, Du und Deine Frau\pwindex{Schnitzler, Olga 17.01.1882 – 13.01.1970@\textsc{Schnitzler, Olga} (17.01.1882 – 13.01.1970), \emph{Schauspieler/Schauspielerin, Sänger/Sängerin}|pwv}, wohlbehalten zurückgekommen ſeid und
               daß Eure \label{K_L03445-1v}\edtext{Reiſe}{\lemma{\textnormal{\emph{Reiſe}}}\Cendnote{\textnormal{Siehe Paul Goldmann an Arthur Schnitzler, 14. 3. [1904].
               }}}\label{K_L03445-1} ſo ſchön verlaufen iſt. Und bei der Rückkehr aus \textsc{Taormina\oindex{Taormina@\textbf{Taormina}, \emph{P.PPLA3}|pw}} und \textsc{Pompeji\oindex{Pompeji@\textbf{Pompeji}, \emph{S.ANS}|pw}} zu Hauſe einen blondlockigen Sohn\pwindex{Schnitzler, Heinrich 09.08.1902 – 12.07.1982@\textsc{Schnitzler, Heinrich} (09.08.1902 – 12.07.1982), \emph{Regisseur/Regisseurin, Schauspieler/Schauspielerin}|pwv} vorzufinden, iſt auch nicht übel.\pend
           
\pstart
           Ob mich mein Weg dieſes Jahr nach \label{K_L03445-2v}\edtext{Wien\oindex{Wien@\textbf{Wien}, \emph{A.ADM2}|pw}}{\lemma{\textnormal{\emph{Wien}}}\Cendnote{\textnormal{Goldmann\pwindex{Goldmann, Paul 31.01.1865 – 25.09.1935@\textsc{Goldmann, Paul} (31.01.1865 – 25.09.1935), \emph{Schriftsteller/Schriftstellerin, Journalist/Journalistin}|pwk} war jedenfalls am 10. 8. 1904 und am 11. 8. 1904 in Wien\oindex{Wien@\textbf{Wien}, \emph{A.ADM2}|pwk}. Am 11. 8. 1904 besuchte er Arthur
                  und Olga Schnitzler\pwindex{Schnitzler, Olga 17.01.1882 – 13.01.1970@\textsc{Schnitzler, Olga} (17.01.1882 – 13.01.1970), \emph{Schauspieler/Schauspielerin, Sänger/Sängerin}|pwk}. Im September war er noch einmal in Wien\oindex{Wien@\textbf{Wien}, \emph{A.ADM2}|pwk}, vgl. A. S.: \emph{Tagebuch}, 21. 9. 1904.}}}\label{K_L03445-2} führen wird, iſt fraglich. Sollte es der Fall ſein, ſo
               wird es mir natürlich eine große Freude ſein, Dich dort wiederzuſehen. {\pb}Bei Marienbad\oindex{Marienbad@\textbf{Marienbad}, \emph{P.PPL}|pw}
               bleibt es wahrſcheinlich. Was hinterher noch geſchehen wird, iſt ganz ungewiß. Sobald
               ich Genaueres weiß, theile ich es Dir mit; und es wäre ſehr ſchön, wenn ſich eine
               Möglichkeit finden ließe, Dich unterwegs zu treffen.\pend
           
\pstart
           Jetzt im Sommer werden ſich wohl wieder alle Vorzüge Eurer prachtvoll gelegenen Wohnung\oindex{Edmund-Weiss-Gasse 7@\textbf{Edmund-Weiß-Gasse 7}, \emph{Wohngebäude (K.WHS)}|pwv} entfalten, und ich
               wünſche Dir eine Reihe guter Arbeitsſtunden auf Deiner Veranda mit dem Blick ins {\pb}Grüne. Schreibſt Du ein \label{K_L03445-3v}\edtext{neues Stück}{\lemma{\textnormal{\emph{neues Stück}}}\Cendnote{\textnormal{Das
                  nächste große dramatische Werk, an dem Schnitzler arbeitete, war die Komödie \emph{Zwischenspiel}\pwindex{Zwischenspiel. Komoedie in drei Akten@\emph{Zwischenspiel. Komödie in drei Akten}|pwk}. Im \emph{Tagebuch}\pwindex{Tagebuch@\emph{Tagebuch}|pwk} ist die
                  Arbeit daran aber erst ab dem 1. 8. 1904 vermerkt, die Idee datierte er auf den 31. 7. 1904.}}}\label{K_L03445-3}? Und gedenkſt Du Dich \substVorne{}\textsuperscript{damit}\substDazwischen{}damit\substHinten{} an dem \label{K_L03445-4v}\edtext{Wettkampf der
                  Theater}{\lemma{\textnormal{\emph{Wettkampf der
                  Theater}}}\Cendnote{\textnormal{Mit der kommenden
                  Theatersaison übernahm Otto Brahm\pwindex{Brahm, Otto 05.02.1856 – 28.11.1912@\textsc{Brahm, Otto} (05.02.1856 – 28.11.1912), \emph{Theaterleiter/Theaterleiterin, Regisseur/Regisseurin}|pwk} die
                     Leitung des \emph{Lessing-Theaters}\orgindex{Lessing-Theater@Lessing-Theater|pwk}. Das \emph{Deutsche Theater}\orgindex{Deutsches Theater Berlin@Deutsches Theater Berlin|pwk}, das er bisher geleitet hatte,
                  wurde von Paul Lindau\pwindex{Lindau, Paul 03.06.1839 – 31.01.1919@\textsc{Lindau, Paul} (03.06.1839 – 31.01.1919), \emph{Schriftsteller/Schriftstellerin, Kritiker/Kritikerin, Theaterleiter/Theaterleiterin}|pwk} übernommen. In Folge
                  kam es zu einem Wettstreit, ob Brahm\pwindex{Brahm, Otto 05.02.1856 – 28.11.1912@\textsc{Brahm, Otto} (05.02.1856 – 28.11.1912), \emph{Theaterleiter/Theaterleiterin, Regisseur/Regisseurin}|pwk} das
                  neue Haus\orgindex{Lessing-Theater@Lessing-Theater|pwkv} auf das Niveau des
                     alten\orgindex{Deutsches Theater Berlin@Deutsches Theater Berlin|pwkv} bringen konnte und
                  ob das alte seine Qualität zu halten in der Lage war. Lindau\pwindex{Lindau, Paul 03.06.1839 – 31.01.1919@\textsc{Lindau, Paul} (03.06.1839 – 31.01.1919), \emph{Schriftsteller/Schriftstellerin, Kritiker/Kritikerin, Theaterleiter/Theaterleiterin}|pwk} verlor, er konnte das Theater\orgindex{Deutsches Theater Berlin@Deutsches Theater Berlin|pwkv} nur eine Saison lang führen. Ab
                  der Theatersaison 1905/1906
                  übernahm es Max Reinhardt\pwindex{Reinhardt, Max 09.09.1873 – 30.10.1943@\textsc{Reinhardt, Max} (09.09.1873 – 30.10.1943), \emph{Theaterleiter/Theaterleiterin, Regisseur/Regisseurin, Schauspieler/Schauspielerin}|pwk}.}}}\label{K_L03445-4} zu
               betheiligen, der im kommenden Winter in Berlin\oindex{Berlin@\textbf{Berlin}, \emph{P.PPLC}|pw}
               mit noch nicht dageweſener Heftigkeit entbrennen wird?\pend
           
\pstart
           Meine Freundin\pwindex{Rottenberg, Theodore 1875-09-07 – 1945-04-05@\textsc{Rottenberg, Theodore} (1875-09-07 – 1945-04-05)|pwv} erwidert
               herzlich Deinen Gruß. Es geht \strikeout{ih} ihr, wie es ihr
               ging. Sie leidet ſchwer unter den unerträglichen Verhältniſſen ihrer Ehe\pwindex{Rottenberg, Ludwig 11.10.1864 – 6.5.1932@\textsc{Rottenberg, Ludwig} (11.10.1864 – 6.5.1932), \emph{Kapellmeister/Kapellmeisterin}|pwv} und der Enge und gemeinen
               Klatſchſucht der Kleinſtadt\oindex{Frankfurt am Main@\textbf{Frankfurt am Main}, \emph{P.PPLA3}|pwv}.
               Sie ſehnt ſich danach, \strikeout{\textcolor{gray}{n}} ſich mit mir zu vereinigen; ich ſehne mich nach ihr. Aber die {\pb}materiellen Verhältniſſe erlauben es nicht, dieſe
               beiderſeitige Sehnſucht endgiltig zu befriedigen. Und die Löſung iſt nach wie vor:
                  Fortwurſteln{\dotsfour}\pend
           
\pstart
           Daß Ihr \textsc{Hoffmannsthal\pwindex{Hofmannsthal, Hugo von 1874-02-01 – 1929-07-15@\textsc{Hofmannsthal, Hugo von} (1874-02-01 – 1929-07-15), \emph{Schriftsteller/Schriftstellerin}|pw}} in der \label{K_L03445-5v}\edtext{\textsc{Liliencron\pwindex{Liliencron, Detlev von 03.06.1844 – 22.07.1909@\textsc{Liliencron, Detlev von} (03.06.1844 – 22.07.1909), \emph{Schriftsteller/Schriftstellerin, Dichter/Dichterin, Dramatiker/Dramatikerin}|pw}}-Affaire}{\lemma{\textnormal{\emph{Liliencron-Affaire}}}\Cendnote{\textnormal{Siehe Hugo von Hofmannsthal an Arthur Schnitzler, 1[9?]. 6. [1904] und A. S.: \emph{Tagebuch}, 2. 6. 1904.
               }}}\label{K_L03445-5} Unrecht gebt, erfreut mich ebenſoſehr, wie es mich überraſcht.\pend
           
\pstart
           Ich fahre heut{ }Mittag nach \textsc{Kiel\oindex{Kiel@\textbf{Kiel}, \emph{P.PPLA}|pw}}, um über die \label{K_L03445-6v}\edtext{Monarchen\pwindex{Eduard VII. 09.09.1841 – 06.05.1910@\textsc{Eduard VII.} (09.09.1841 – 06.05.1910), \emph{König/Königin}|pwv}\pwindex{Wilhelm II. von Preussen 27.1.1859 – 4.6.1941@\textsc{Wilhelm II. von Preußen} (27.1.1859 – 4.6.1941), \emph{Kaiser/Kaiserin}|pwv}-Zuſammenkunft}{\lemma{\textnormal{\emph{Monarchen-Zuſammenkunft}}}\Cendnote{\textnormal{Anlässlich der »Kiel\oindex{Kiel@\textbf{Kiel}, \emph{P.PPLA}|pwk}er Woche« trafen der engl\oindex{England@\textbf{England}, \emph{A.ADM1}|pwkv}ische König Eduard VII.\pwindex{Eduard VII. 09.09.1841 – 06.05.1910@\textsc{Eduard VII.} (09.09.1841 – 06.05.1910), \emph{König/Königin}|pwk} und sein Neffe, der deutsch\oindex{Deutschland@\textbf{Deutschland}, \emph{A.PCLI}|pwkv}e Kaiser Wilhelm II.\pwindex{Wilhelm II. von Preussen 27.1.1859 – 4.6.1941@\textsc{Wilhelm II. von Preußen} (27.1.1859 – 4.6.1941), \emph{Kaiser/Kaiserin}|pwk}, aufeinander.}}}\label{K_L03445-6} zu
               berichten.\pend
           
\pstart
           Herzliche Grüße an Dich und Deine Frau\pwindex{Schnitzler, Olga 17.01.1882 – 13.01.1970@\textsc{Schnitzler, Olga} (17.01.1882 – 13.01.1970), \emph{Schauspieler/Schauspielerin, Sänger/Sängerin}|pwv} von Deinem getreuen {\\[\baselineskip]}\spacefill\mbox{Paul Goldmann}\pend
           \leftskip=0em{}\selectlanguage{ngerman}\endnumbering\briefempfaengerindex{Schnitzler, Arthur@\textsc{Schnitzler, Arthur}!zzzGoldmann, Paul@\emph{von Paul Goldmann}!1904-06-232@{23. 6. {[}1904{]}}|)be}\mylabel{L03445h}  \normalsize

\doendnotes{C}
\bigskip
\vfill

\clearpage

\footnotesize

\lohead{\textsc{register}}

% Definiere theindex-Environment komplett neu ohne reledmac
\makeatletter
\renewenvironment{theindex}{%
  \section*{\indexname}%
  \setlength{\parindent}{0pt}%
  \setlength{\parskip}{0pt plus 0.3pt}%
  \let\item\@idxitem
}{%
  \clearpage
}
\makeatother

\IfFileExists{\jobname-pw.ind}{\input{\jobname-pw.ind}}{}

\end{document}

      