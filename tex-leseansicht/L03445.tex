%% latex-leseansicht-vorspann.tex
%% Vorspann für die Leseansicht.
%% Lädt die gemeinsame Datei latex-vorspann.tex mit nicht gesetztem Schalter.

\newif\ifkorrekturansicht
\korrekturansichtfalse

\input{../tex-inputs/latex-vorspann}


\section[ Paul Goldmann an Arthur Schnitzler, 23. 6. [1904]]{L03445 Paul Goldmann an Arthur Schnitzler,  23. 6. [1904]}
\nopagebreak\mylabel{L03445v}
\rehead{ }\normalsize\beginnumbering\briefempfaengerindex{Schnitzler, Arthur@\textsc{Schnitzler, Arthur}!zzzGoldmann, Paul@\emph{von Paul Goldmann}!1904-06-232@{23. 6. [1904]}|(be}
\toendnotes[C]{\smallbreak\pagebreak[2]}
\correspDesc{Versand  durch Paul Goldmann am 23. 6. [1904] in Berlin
\newline{}Erhalt  durch Arthur Schnitzler im Zeitraum [24. 6. 1904
                  – 28. 6. 1904?] in Wien}\toendnotes[C]{\smallbreak}
\Standort{DLA, A:Schnitzler, HS.NZ85.1.3174.}
\physDesc{Brief, 1 Blatt, 4 Seiten, 1736 Zeichen
\newline{}Handschrift: blaue Tinte, deutsche Kurrent
\newline{}Schnitzler: 1) mit Bleistift das Jahr »904« vermerkt  2) mit rotem Buntstift zwei Unterstreichungen}\toendnotes[C]{\smallbreak}
\pstart
           \raggedleft{}{\pb}\textcolor{gray}{\textbf{DESSAUERSTRASSE 19\oindex{Dessauer Straße@\textbf{Dessauer Straße}, \emph{Straße}|pw}}}\pend
           
\pstart
           Berlin\oindex{Berlin@\textbf{Berlin}, \emph{Hauptstadt}|pw}, 23. Juni.\pend
           
\pstart\center{}Mein lieber Freund,\pend\vspace{0.5em}
\pstart
           Ich habe mich{ }ſehr gefreut, zu erſehen, daß Ihr, Du und Deine Frau\pwindex{Schnitzler, Olga 17.\,1.\,1882 Wien – 13.\,1.\,1970 Lugano@\textsc{Schnitzler, Olga} (17.\,1.\,1882 Wien – 13.\,1.\,1970 Lugano), \emph{Schauspielerin, Sängerin}|pwv}, wohlbehalten zurückgekommen{ }ſeid und
               daß Eure \label{K_L03445-1v}\edtext{Reiſe}{\lemma{\textnormal{\emph{Reise}}}\Cendnote{\textnormal{Siehe XXXX Auszeichnungsfehler: Dokument L03440 nicht gefunden.
               }}}\label{K_L03445-1}{ }ſo{ }ſchön verlaufen iſt. Und bei der Rückkehr aus \textsc{Taormina\oindex{Taormina@\textbf{Taormina}, \emph{Hauptstadt}|pw}} und \textsc{Pompeji\oindex{Pompeji@\textbf{Pompeji}, \emph{Ausgrabung}|pw}} zu Hauſe einen blondlockigen Sohn\pwindex{Schnitzler, Heinrich 9.\,8.\,1902 Hinterbrühl – 12.\,7.\,1982 Wien@\textsc{Schnitzler, Heinrich} (9.\,8.\,1902 Hinterbrühl – 12.\,7.\,1982 Wien), \emph{Regisseur, Schauspieler}|pwv} vorzufinden, iſt auch nicht übel.\pend
           
\pstart
           Ob mich mein Weg dieſes Jahr nach \label{K_L03445-2v}\edtext{Wien\oindex{Wien@\textbf{Wien}, \emph{Verwaltungsgebiet}|pw}}{\lemma{\textnormal{\emph{Wien}}}\Cendnote{\textnormal{Goldmann\pwindex{Goldmann, Paul 31.\,1.\,1865 Breslau – 25.\,9.\,1935 Wien@\textsc{Goldmann, Paul} (31.\,1.\,1865 Breslau – 25.\,9.\,1935 Wien), \emph{Schriftsteller, Journalist}|pwk} war jedenfalls am 10. 8. 1904 und am 11. 8. 1904 in Wien\oindex{Wien@\textbf{Wien}, \emph{Verwaltungsgebiet}|pwk}. Am 11. 8. 1904 besuchte er Arthur
                  und Olga Schnitzler\pwindex{Schnitzler, Olga 17.\,1.\,1882 Wien – 13.\,1.\,1970 Lugano@\textsc{Schnitzler, Olga} (17.\,1.\,1882 Wien – 13.\,1.\,1970 Lugano), \emph{Schauspielerin, Sängerin}|pwk}. Im September war er noch einmal in Wien\oindex{Wien@\textbf{Wien}, \emph{Verwaltungsgebiet}|pwk}, vgl. A. S.: \emph{Tagebuch}, 21. 9. 1904.}}}\label{K_L03445-2} führen wird, iſt fraglich. Sollte es der Fall{ }ſein,{ }ſo
               wird es mir natürlich eine große Freude{ }ſein, Dich dort wiederzuſehen. {\pb}Bei Marienbad\oindex{Marienbad@\textbf{Marienbad}|pw}
               bleibt es wahrſcheinlich. Was hinterher noch geſchehen wird, iſt ganz ungewiß. Sobald
               ich Genaueres weiß, theile ich es Dir mit; und es wäre{ }ſehr{ }ſchön, wenn{ }ſich eine
               Möglichkeit finden ließe, Dich unterwegs zu treffen.\pend
           
\pstart
           Jetzt im Sommer werden{ }ſich wohl wieder alle Vorzüge Eurer prachtvoll gelegenen Wohnung\oindex{Wien@\textbf{Wien}!XVIII., Währing@\textbf{XVIII., Währing}!Edmund-Weiß-Gasse 7@\textbf{Edmund-Weiß-Gasse 7}, \emph{Wohngebäude}|pwv} entfalten, und ich
               wünſche Dir eine Reihe guter Arbeitsſtunden auf Deiner Veranda mit dem Blick ins {\pb}Grüne. Schreibſt Du ein \label{K_L03445-3v}\edtext{neues Stück}{\lemma{\textnormal{\emph{neues Stück}}}\Cendnote{\textnormal{Das
                  nächste große dramatische Werk, an dem Schnitzler arbeitete, war die Komödie \emph{Zwischenspiel}\pwindex{Schnitzler, Arthur 15.\,5.\,1862 Wien – 21.\,10.\,1931 ebd.@\textsc{Schnitzler, Arthur} (15.\,5.\,1862 Wien – 21.\,10.\,1931 ebd.), \emph{Schriftsteller, Mediziner}!Zwischenspiel. Komödie in drei Akten@\strich\emph{Zwischenspiel. Komödie in drei Akten}|pwk}. Im \emph{Tagebuch}\pwindex{Schnitzler, Arthur 15.\,5.\,1862 Wien – 21.\,10.\,1931 ebd.@\textsc{Schnitzler, Arthur} (15.\,5.\,1862 Wien – 21.\,10.\,1931 ebd.), \emph{Schriftsteller, Mediziner}!Tagebuch@\strich\emph{Tagebuch}|pwk} ist die
                  Arbeit daran aber erst ab dem 1. 8. 1904 vermerkt, die Idee datierte er auf den 31. 7. 1904.}}}\label{K_L03445-3}? Und gedenkſt Du Dich \substVorne{}\textsuperscript{damit}\substDazwischen{}damit\substHinten{} an dem \label{K_L03445-4v}\edtext{Wettkampf der
                  Theater}{\lemma{\textnormal{\emph{Wettkampf der
                  Theater}}}\Cendnote{\textnormal{Mit der kommenden
                  Theatersaison übernahm Otto Brahm\pwindex{Brahm, Otto 5.\,2.\,1856 Hamburg – 28.\,11.\,1912 Berlin@\textsc{Brahm, Otto} (5.\,2.\,1856 Hamburg – 28.\,11.\,1912 Berlin), \emph{Theaterleiter, Regisseur}|pwk} die
                     Leitung des \emph{Lessing-Theaters}\orgindex{Lessing-Theater@Lessing-Theater|pwk}. Das \emph{Deutsche Theater}\orgindex{Deutsches Theater Berlin@Deutsches Theater Berlin|pwk}, das er bisher geleitet hatte,
                  wurde von Paul Lindau\pwindex{Lindau, Paul 3.\,6.\,1839 Magdeburg – 31.\,1.\,1919 Berlin@\textsc{Lindau, Paul} (3.\,6.\,1839 Magdeburg – 31.\,1.\,1919 Berlin), \emph{Schriftsteller, Kritiker, Theaterleiter}|pwk} übernommen. In Folge
                  kam es zu einem Wettstreit, ob Brahm\pwindex{Brahm, Otto 5.\,2.\,1856 Hamburg – 28.\,11.\,1912 Berlin@\textsc{Brahm, Otto} (5.\,2.\,1856 Hamburg – 28.\,11.\,1912 Berlin), \emph{Theaterleiter, Regisseur}|pwk} das
                  neue Haus\orgindex{Lessing-Theater@Lessing-Theater|pwkv} auf das Niveau des
                     alten\orgindex{Deutsches Theater Berlin@Deutsches Theater Berlin|pwkv} bringen konnte und
                  ob das alte seine Qualität zu halten in der Lage war. Lindau\pwindex{Lindau, Paul 3.\,6.\,1839 Magdeburg – 31.\,1.\,1919 Berlin@\textsc{Lindau, Paul} (3.\,6.\,1839 Magdeburg – 31.\,1.\,1919 Berlin), \emph{Schriftsteller, Kritiker, Theaterleiter}|pwk} verlor, er konnte das Theater\orgindex{Deutsches Theater Berlin@Deutsches Theater Berlin|pwkv} nur eine Saison lang führen. Ab
                  der Theatersaison 1905/1906
                  übernahm es Max Reinhardt\pwindex{Reinhardt, Max 9.\,9.\,1873 Baden bei Wien – 30.\,10.\,1943 New York City@\textsc{Reinhardt, Max} (9.\,9.\,1873 Baden bei Wien – 30.\,10.\,1943 New York City), \emph{Theaterleiter, Regisseur, Schauspieler}|pwk}.}}}\label{K_L03445-4} zu
               betheiligen, der im kommenden Winter in Berlin\oindex{Berlin@\textbf{Berlin}, \emph{Hauptstadt}|pw}
               mit noch nicht dageweſener Heftigkeit entbrennen wird?\pend
           
\pstart
           Meine Freundin\pwindex{Rottenberg, Theodore 7.\,9.\,1875 – 5.\,4.\,1945 Limburg an der Lahn@\textsc{Rottenberg, Theodore} (7.\,9.\,1875 – 5.\,4.\,1945 Limburg an der Lahn)|pwv} erwidert
               herzlich Deinen Gruß. Es geht \strikeout{ih} ihr, wie es ihr
               ging. Sie leidet{ }ſchwer unter den unerträglichen Verhältniſſen ihrer Ehe\pwindex{Rottenberg, Ludwig 11.\,10.\,1864 Czernowitz – 6.\,5.\,1932 Frankfurt am Main@\textsc{Rottenberg, Ludwig} (11.\,10.\,1864 Czernowitz – 6.\,5.\,1932 Frankfurt am Main), \emph{Kapellmeister}|pwv} und der Enge und gemeinen
               Klatſchſucht der Kleinſtadt\oindex{Frankfurt am Main@\textbf{Frankfurt am Main}, \emph{Hauptstadt}|pwv}.
               Sie{ }ſehnt{ }ſich danach, \strikeout{\textcolor{gray}{n}}{ }ſich mit mir zu vereinigen; ich{ }ſehne mich nach ihr. Aber die {\pb}materiellen Verhältniſſe erlauben es nicht, dieſe
               beiderſeitige Sehnſucht endgiltig zu befriedigen. Und die Löſung iſt nach wie vor:
                  Fortwurſteln{\dotsfour}\pend
           
\pstart
           Daß Ihr \textsc{Hoffmannsthal\pwindex{Hofmannsthal, Hugo von 1.\,2.\,1874 Wien – 15.\,7.\,1929 Rodaun@\textsc{Hofmannsthal, Hugo von} (1.\,2.\,1874 Wien – 15.\,7.\,1929 Rodaun), \emph{Schriftsteller}|pw}} in der \label{K_L03445-5v}\edtext{\textsc{Liliencron\pwindex{Liliencron, Detlev von 3.\,6.\,1844 Kiel – 22.\,7.\,1909 Rahlstedt@\textsc{Liliencron, Detlev von} (3.\,6.\,1844 Kiel – 22.\,7.\,1909 Rahlstedt), \emph{Schriftsteller, Dichter, Dramatiker}|pw}}-Affaire}{\lemma{\textnormal{\emph{Liliencron-Affaire}}}\Cendnote{\textnormal{Siehe XXXX Auszeichnungsfehler: Dokument L01406 nicht gefunden und A. S.: \emph{Tagebuch}, 2. 6. 1904.
               }}}\label{K_L03445-5} Unrecht gebt, erfreut mich ebenſoſehr, wie es mich überraſcht.\pend
           
\pstart
           Ich fahre heut{ }Mittag nach \textsc{Kiel\oindex{Kiel@\textbf{Kiel}|pw}}, um über die \label{K_L03445-6v}\edtext{Monarchen\pwindex{Eduard VII. 9.\,9.\,1841 London – 6.\,5.\,1910 ebd.@\textsc{Eduard VII.} (9.\,9.\,1841 London – 6.\,5.\,1910 ebd.), \emph{König}|pwv}\pwindex{Wilhelm II. von Preußen 27.\,1.\,1859 Potsdam – 4.\,6.\,1941 Gemeente Utrechtse Heuvelrug@\textsc{Wilhelm II. von Preußen} (27.\,1.\,1859 Potsdam – 4.\,6.\,1941 Gemeente Utrechtse Heuvelrug), \emph{Kaiser}|pwv}-Zuſammenkunft}{\lemma{\textnormal{\emph{Monarchen-Zusammenkunft}}}\Cendnote{\textnormal{Anlässlich der »Kiel\oindex{Kiel@\textbf{Kiel}|pwk}er Woche« trafen der engl\oindex{England@\textbf{England}, \emph{Land}|pwkv}ische König Eduard VII.\pwindex{Eduard VII. 9.\,9.\,1841 London – 6.\,5.\,1910 ebd.@\textsc{Eduard VII.} (9.\,9.\,1841 London – 6.\,5.\,1910 ebd.), \emph{König}|pwk} und sein Neffe, der deutsch\oindex{Deutschland@\textbf{Deutschland}|pwkv}e Kaiser Wilhelm II.\pwindex{Wilhelm II. von Preußen 27.\,1.\,1859 Potsdam – 4.\,6.\,1941 Gemeente Utrechtse Heuvelrug@\textsc{Wilhelm II. von Preußen} (27.\,1.\,1859 Potsdam – 4.\,6.\,1941 Gemeente Utrechtse Heuvelrug), \emph{Kaiser}|pwk}, aufeinander.}}}\label{K_L03445-6} zu
               berichten.\pend
           
\pstart
           Herzliche Grüße an Dich und Deine Frau\pwindex{Schnitzler, Olga 17.\,1.\,1882 Wien – 13.\,1.\,1970 Lugano@\textsc{Schnitzler, Olga} (17.\,1.\,1882 Wien – 13.\,1.\,1970 Lugano), \emph{Schauspielerin, Sängerin}|pwv} von Deinem getreuen {\\[\baselineskip]}\spacefill\mbox{Paul Goldmann}\pend
           \leftskip=0em{}\selectlanguage{ngerman}\endnumbering\briefempfaengerindex{Schnitzler, Arthur@\textsc{Schnitzler, Arthur}!zzzGoldmann, Paul@\emph{von Paul Goldmann}!1904-06-232@{23. 6. [1904]}|)be}\mylabel{L03445h}  \newcommand{\dateiname}{L03445}\newcommand{\titel}{Paul Goldmann an Arthur Schnitzler, 23. 6. [1904]}\newcommand{\editorInnen}{Martin Anton Müller und Laura Untner}%% latex-leseansicht-abspann.tex
%% Abspann für die Leseansicht.
%% Der Schalter \ifkorrekturansicht ist bereits durch den Vorspann gesetzt.

%% latex-abspann.tex
%% Gemeinsamer Abspann für Korrekturansicht und Leseansicht.
%% Setzt den Schalter \ifkorrekturansicht voraus (gesetzt in den
%% einbindenden Dateien latex-korrekturansicht-abspann.tex bzw.
%% latex-leseansicht-abspann.tex).
%% ---------------------------------------------------------------

\normalsize

% Das esempio-Environment wird nur in der Leseansicht benötigt
\ifkorrekturansicht\else
\newenvironment{esempio}[3]%
{
    \vspace{1.5ex}
    \rlap{\underline{#1}}
    \par
    \setlength{\parindent}{0cm}
    \nopagebreak
    \leftskip=#2cm
    \rightskip=#3cm
}
{
    \par
}
\fi

\doendnotes{C}
\bigskip
\vfill

\clearpage

\footnotesize

\ifkorrekturansicht
  \lohead{\textsc{register}}
\fi

% theindex-Environment neu definieren ohne reledmac
\makeatletter
\renewenvironment{theindex}{%
  \ifkorrekturansicht
    \section*{\indexname}%
  \else
    \subsubsection*{Index der erwähnten Entitäten}%
  \fi
  \setlength{\parindent}{0pt}%
  \setlength{\parskip}{0pt plus 0.3pt}%
  \let\item\@idxitem
}{%
  \ifkorrekturansicht\clearpage\fi
}
\makeatother

\IfFileExists{\jobname-pw.ind}{\input{\jobname-pw.ind}}{}

% Quellenangabe nur in der Leseansicht
\ifkorrekturansicht\else
% Fallback-Definitionen, falls die .tex-Datei \titel etc. nicht gesetzt hat
\providecommand{\titel}{}
\providecommand{\editorInnen}{}
\providecommand{\dateiname}{\jobname}

\vspace{3cm}

\vfill

\footnotesize
\textsc{Quelle}: \titel. Herausgegeben von {\editorInnen}. In: \emph{Arthur Schnitzler: Briefwechsel mit Autorinnen und Autoren}.
 Digitale Edition, https://schnitzler-briefe.acdh.oeaw.ac.at/{\dateiname}.html (Stand \today)
\fi

\end{document}


