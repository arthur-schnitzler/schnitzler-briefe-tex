%% latex-leseansicht-vorspann.tex
%% Vorspann für die Leseansicht.
%% Lädt die gemeinsame Datei latex-vorspann.tex mit nicht gesetztem Schalter.

\newif\ifkorrekturansicht
\korrekturansichtfalse

\input{../tex-inputs/latex-vorspann}


         
         \renewcommand{\erwaehntePersonen}{Personen: Otto Brahm,  Eduard VII., Paul Goldmann, Hugo von Hofmannsthal, Detlev von Liliencron, Paul Lindau, Max Reinhardt, Theodore Rottenberg, Ludwig Rottenberg, Olga Schnitzler, Heinrich Schnitzler,  Wilhelm II. von Preußen}
         \renewcommand{\erwaehnteInstitutionen}{Institutionen: Deutsches Theater Berlin, Lessing-Theater}
         \renewcommand{\erwaehnteOrte}{Orte: Berlin, Dessauer Straße, Deutschland, Edmund-Weiß-Gasse 7, England, Frankfurt am Main, Kiel, Marienbad, Pompei, Taormina, Wien}
         \renewcommand{\erwaehnteWerke}{Werke: Tagebuch, Zwischenspiel. Komödie in drei Akten}
               \section[ Paul Goldmann an Arthur Schnitzler, 23. 6. {[}1904{]}]{ Paul Goldmann an Arthur Schnitzler, 23. 6. {[}1904{]}}\nopagebreak\mylabel{v}\rehead{ }\begin{ledgroupsized}[t]{13cm}\normalsize\beginnumbering\briefempfaengerindex{Schnitzler, Arthur@\textsc{Schnitzler, Arthur}!zzzGoldmann, Paul@\emph{von Paul Goldmann}!1904-06-232@{23. 6. {[}1904{]}}|(be} \toendnotes[C]{\smallbreak\pagebreak[2]} \Standort{DLA, A:Schnitzler, HS.NZ85.1.3174.}
\physDesc{Brief, 1 Blatt, 4 Seiten, 1736 Zeichen
\newline{}Handschrift: blaue Tinte, deutsche Kurrent
\newline{}Schnitzler: 1) mit Bleistift das Jahr »904« vermerkt  2) mit rotem Buntstift zwei Unterstreichungen}\toendnotes[C]{\smallbreak}\pstart
           \noindent{}\raggedleft{}{\pb}\textcolor{gray}{\textbf{DESSAUERSTRASSE 19\oindex{Dessauer Strasse@\textbf{Dessauer Straße}|pw}}}\pend
           \pstart
           Berlin\oindex{Berlin@\textbf{Berlin}|pw}, 23. Juni.\pend
           \pstart\center{}Mein lieber Freund,\pend\pstart
           Ich habe mich ſehr gefreut, zu erſehen, daß Ihr, Du und Deine Frau\pwindex{Schnitzler, Olga 17.01.1882 – 13.01.1970@\textsc{Schnitzler, Olga} (17.01.1882 – 13.01.1970), \emph{Schauspielerin, Sängerin}|pwv}, wohlbehalten zurückgekommen ſeid und
               daß Eure \label{K_L03445-1v}\edtext{Reiſe}{\lemma{\textnormal{\emph{Reiſe}}}\Cendnote{\textnormal{Siehe Paul Goldmann an Arthur Schnitzler, 14. 3. [1904].
               }}}\label{K_L03445-1h} ſo ſchön verlaufen iſt. Und bei der Rückkehr aus \textsc{Taormina\oindex{Taormina@\textbf{Taormina}|pw}} und \textsc{Pompeji\oindex{Pompei@\textbf{Pompei}|pw}} zu Hauſe einen blondlockigen Sohn\pwindex{Schnitzler, Heinrich 09.08.1902 – 12.07.1982@\textsc{Schnitzler, Heinrich} (09.08.1902 – 12.07.1982), \emph{Regisseur, Schauspieler}|pwv} vorzufinden, iſt auch nicht übel.\pend
           \pstart
           Ob mich mein Weg dieſes Jahr nach \label{K_L03445-2v}\edtext{Wien\oindex{Wien@\textbf{Wien}|pw}}{\lemma{\textnormal{\emph{Wien}}}\Cendnote{\textnormal{Goldmann\pwindex{Goldmann, Paul 31.01.1865 – 25.09.1935@\textsc{Goldmann, Paul} (31.01.1865 – 25.09.1935), \emph{Schriftsteller, Journalist}|pwk} war jedenfalls am 10. 8. 1904 und am 11. 8. 1904 in Wien\oindex{Wien@\textbf{Wien}|pwk}. Am 11. 8. 1904 besuchte er Arthur\pwindex{Schnitzler, Arthur 15.05.1862 – 21.10.1931@\textsc{Schnitzler, Arthur} (15.05.1862 – 21.10.1931), \emph{Schriftsteller, Mediziner}|pwk}
                  und Olga Schnitzler\textcolor{red}{\textsuperscript{\textbf{KEY}}}. Im September war er noch einmal in Wien\oindex{Wien@\textbf{Wien}|pwk}, vgl. A. S.: \emph{Tagebuch}, 21. 9. 1904.}}}\label{K_L03445-2h} führen wird, iſt fraglich. Sollte es der Fall ſein, ſo
               wird es mir natürlich eine große Freude ſein, Dich dort wiederzuſehen. {\pb}Bei Marienbad\oindex{Marienbad@\textbf{Marienbad}|pw}
               bleibt es wahrſcheinlich. Was hinterher noch geſchehen wird, iſt ganz ungewiß. Sobald
               ich Genaueres weiß, theile ich es Dir mit; und es wäre ſehr ſchön, wenn ſich eine
               Möglichkeit finden ließe, Dich unterwegs zu treffen.\pend
           \pstart
           Jetzt im Sommer werden ſich wohl wieder alle Vorzüge Eurer prachtvoll gelegenen Wohnung\oindex{Edmund-Weiss-Gasse 7@\textbf{Edmund-Weiß-Gasse 7}|pwv} entfalten, und ich
               wünſche Dir eine Reihe guter Arbeitsſtunden auf Deiner Veranda mit dem Blick ins {\pb}Grüne. Schreibſt Du ein \label{K_L03445-3v}\edtext{neues Stück}{\lemma{\textnormal{\emph{neues Stück}}}\Cendnote{\textnormal{Das
                  nächste große dramatische Werk, an dem Schnitzler\pwindex{Schnitzler, Arthur 15.05.1862 – 21.10.1931@\textsc{Schnitzler, Arthur} (15.05.1862 – 21.10.1931), \emph{Schriftsteller, Mediziner}|pwk} arbeitete, war die Komödie \emph{Zwischenspiel}\pwindex{Schnitzler, Arthur 15.05.1862 – 21.10.1931@\textsc{Schnitzler, Arthur} (15.05.1862 – 21.10.1931), \emph{Schriftsteller, Mediziner}!Zwischenspiel. Komoedie in drei Akten1905-10-12@\strich\emph{Zwischenspiel. Komödie in drei Akten} {[}1905-10-12{]}|pwk}. Im \emph{Tagebuch}\pwindex{\textcolor{red}{\textsuperscript{XXXX1 indx}}!Tagebuch1981 – 2000@\strich\emph{Tagebuch} {[}Hrsg., 1981 – 2000{]}|pwk} ist die
                  Arbeit daran aber erst ab dem 1. 8. 1904 vermerkt, die Idee datierte er auf den 31. 7. 1904.}}}\label{K_L03445-3h}? Und gedenkſt Du Dich \substVorne{}\textsuperscript{damit}\substDazwischen{}damit\substHinten{} an dem \label{K_L03445-4v}\edtext{Wettkampf der
                  Theater}{\lemma{\textnormal{\emph{Wettkampf der
                  Theater}}}\Cendnote{\textnormal{Mit der kommenden
                  Theatersaison übernahm Otto Brahm\pwindex{Brahm, Otto 05.02.1856 – 28.11.1912@\textsc{Brahm, Otto} (05.02.1856 – 28.11.1912), \emph{Theaterleiter, Regisseur}|pwk} die
                     Leitung des \emph{Lessing-Theaters}\orgindex{Lessing-Theater@Lessing-Theater|pwk}. Das \emph{Deutsche Theater}\orgindex{Deutsches Theater Berlin@Deutsches Theater Berlin|pwk}, das er bisher geleitet hatte,
                  wurde von Paul Lindau\pwindex{Lindau, Paul 03.06.1839 – 31.01.1919@\textsc{Lindau, Paul} (03.06.1839 – 31.01.1919), \emph{Schriftsteller, Kritiker, Theaterleiter}|pwk} übernommen. In Folge
                  kam es zu einem Wettstreit, ob Brahm\pwindex{Brahm, Otto 05.02.1856 – 28.11.1912@\textsc{Brahm, Otto} (05.02.1856 – 28.11.1912), \emph{Theaterleiter, Regisseur}|pwk} das
                  neue Haus\orgindex{Lessing-Theater@Lessing-Theater|pwkv} auf das Niveau des
                     alten\orgindex{Deutsches Theater Berlin@Deutsches Theater Berlin|pwkv} bringen konnte und
                  ob das alte seine Qualität zu halten in der Lage war. Lindau\pwindex{Lindau, Paul 03.06.1839 – 31.01.1919@\textsc{Lindau, Paul} (03.06.1839 – 31.01.1919), \emph{Schriftsteller, Kritiker, Theaterleiter}|pwk} verlor, er konnte das Theater\orgindex{Deutsches Theater Berlin@Deutsches Theater Berlin|pwkv} nur eine Saison lang führen. Ab
                  der Theatersaison 1905/1906
                  übernahm es Max Reinhardt\pwindex{Reinhardt, Max 09.09.1873 – 30.10.1943@\textsc{Reinhardt, Max} (09.09.1873 – 30.10.1943), \emph{Theaterleiter, Regisseur, Schauspieler}|pwk}.}}}\label{K_L03445-4h} zu
               betheiligen, der im kommenden Winter in Berlin\oindex{Berlin@\textbf{Berlin}|pw}
               mit noch nicht dageweſener Heftigkeit entbrennen wird?\pend
           \pstart
           Meine Freundin\pwindex{Rottenberg, Theodore 1875-09-07 – 1945-04-05@\textsc{Rottenberg, Theodore} (1875-09-07 – 1945-04-05)|pwv} erwidert
               herzlich Deinen Gruß. Es geht \strikeout{ih} ihr, wie es ihr
               ging. Sie leidet ſchwer unter den unerträglichen Verhältniſſen ihrer Ehe\pwindex{Rottenberg, Ludwig 11.10.1864 – 6.5.1932@\textsc{Rottenberg, Ludwig} (11.10.1864 – 6.5.1932), \emph{Kapellmeister}|pwv} und der Enge und gemeinen
               Klatſchſucht der Kleinſtadt\oindex{Frankfurt am Main@\textbf{Frankfurt am Main}|pwv}.
               Sie ſehnt ſich danach, \strikeout{\textcolor{gray}{n}} ſich mit mir zu vereinigen; ich ſehne mich nach ihr. Aber die {\pb}materiellen Verhältniſſe erlauben es nicht, dieſe
               beiderfeitige Sehnſucht endgiltig zu befriedigen. Und die Löſung iſt nach wie vor:
                  Fortwurſteln{\dotsfour}\pend
           \pstart
           Daß Ihr \textsc{Hoffmannsthal\pwindex{Hofmannsthal, Hugo von 1874-02-01 – 1929-07-15@\textsc{Hofmannsthal, Hugo von} (1874-02-01 – 1929-07-15), \emph{Schriftsteller}|pw}} in der \label{K_L03445-5v}\edtext{\textsc{Liliencron\pwindex{Liliencron, Detlev von 03.06.1844 – 22.07.1909@\textsc{Liliencron, Detlev von} (03.06.1844 – 22.07.1909), \emph{Schriftsteller, Dichter, Dramatiker}|pw}}-Affaire}{\lemma{\textnormal{\emph{Liliencron-Affaire}}}\Cendnote{\textnormal{Siehe Hugo von Hofmannsthal an Arthur Schnitzler, 1[9?]. 6. [1904] und A. S.: \emph{Tagebuch}, 2. 6. 1904.
               }}}\label{K_L03445-5h} Unrecht gebt, erfreut mich ebenſoſehr, wie es mich überraſcht.\pend
           \pstart
           Ich fahre heut{ }Mittag nach \textsc{Kiel\oindex{Kiel@\textbf{Kiel}|pw}}, um über die \label{K_L03445-6v}\edtext{Monarchen\pwindex{Eduard VII. 09.09.1841 – 06.05.1910@\textsc{Eduard VII.} (09.09.1841 – 06.05.1910), \emph{König/Königin}|pwv}\pwindex{Wilhelm II. von Preussen 27.1.1859 – 4.6.1941@\textsc{Wilhelm II. von Preußen} (27.1.1859 – 4.6.1941), \emph{Kaiser}|pwv}-Zuſammenkunft}{\lemma{\textnormal{\emph{Monarchen-Zuſammenkunft}}}\Cendnote{\textnormal{Anlässlich der »Kiel\oindex{Kiel@\textbf{Kiel}|pwk}er Woche« trafen der engl\oindex{England@\textbf{England}|pwkv}ische König Eduard VII.\pwindex{Eduard VII. 09.09.1841 – 06.05.1910@\textsc{Eduard VII.} (09.09.1841 – 06.05.1910), \emph{König/Königin}|pwk} und sein Neffe, der deutsch\oindex{Deutschland@\textbf{Deutschland}|pwkv}e Kaiser Wilhelm II.\pwindex{Wilhelm II. von Preussen 27.1.1859 – 4.6.1941@\textsc{Wilhelm II. von Preußen} (27.1.1859 – 4.6.1941), \emph{Kaiser}|pwk}, aufeinander.}}}\label{K_L03445-6h} zu
               berichten.\pend
           \pstart
           Herzliche Grüße an Dich und Deine Frau\pwindex{Schnitzler, Olga 17.01.1882 – 13.01.1970@\textsc{Schnitzler, Olga} (17.01.1882 – 13.01.1970), \emph{Schauspielerin, Sängerin}|pwv} von Deinem getreuen {\\[\baselineskip]}\spacefill\mbox{Paul Goldmann}\pend
           \leftskip=0em{}
         
         \endnumbering\mylabel{h}\end{ledgroupsized}  \newcommand{\dateiname}{L03445}\newcommand{\titel}{Paul Goldmann an Arthur Schnitzler, 23. 6. [1904]}\newcommand{\editorInnen}{Martin Anton Müller und Laura Untner}%% latex-leseansicht-abspann.tex
%% Abspann für die Leseansicht.
%% Der Schalter \ifkorrekturansicht ist bereits durch den Vorspann gesetzt.

%% latex-abspann.tex
%% Gemeinsamer Abspann für Korrekturansicht und Leseansicht.
%% Setzt den Schalter \ifkorrekturansicht voraus (gesetzt in den
%% einbindenden Dateien latex-korrekturansicht-abspann.tex bzw.
%% latex-leseansicht-abspann.tex).
%% ---------------------------------------------------------------

\normalsize

% Das esempio-Environment wird nur in der Leseansicht benötigt
\ifkorrekturansicht\else
\newenvironment{esempio}[3]%
{
    \vspace{1.5ex}
    \rlap{\underline{#1}}
    \par
    \setlength{\parindent}{0cm}
    \nopagebreak
    \leftskip=#2cm
    \rightskip=#3cm
}
{
    \par
}
\fi

\doendnotes{C}
\bigskip
\vfill

\clearpage

\footnotesize

\ifkorrekturansicht
  \lohead{\textsc{register}}
\fi

% theindex-Environment neu definieren ohne reledmac
\makeatletter
\renewenvironment{theindex}{%
  \ifkorrekturansicht
    \section*{\indexname}%
  \else
    \subsubsection*{Index der erwähnten Entitäten}%
  \fi
  \setlength{\parindent}{0pt}%
  \setlength{\parskip}{0pt plus 0.3pt}%
  \let\item\@idxitem
}{%
  \ifkorrekturansicht\clearpage\fi
}
\makeatother

\IfFileExists{\jobname-pw.ind}{\input{\jobname-pw.ind}}{}

% Quellenangabe nur in der Leseansicht
\ifkorrekturansicht\else
% Fallback-Definitionen, falls die .tex-Datei \titel etc. nicht gesetzt hat
\providecommand{\titel}{}
\providecommand{\editorInnen}{}
\providecommand{\dateiname}{\jobname}

\vspace{3cm}

\vfill

\footnotesize
\textsc{Quelle}: \titel. Herausgegeben von {\editorInnen}. In: \emph{Arthur Schnitzler: Briefwechsel mit Autorinnen und Autoren}.
 Digitale Edition, https://schnitzler-briefe.acdh.oeaw.ac.at/{\dateiname}.html (Stand \today)
\fi

\end{document}


      