%% latex-leseansicht-vorspann.tex
%% Vorspann für die Leseansicht.
%% Lädt die gemeinsame Datei latex-vorspann.tex mit nicht gesetztem Schalter.

\newif\ifkorrekturansicht
\korrekturansichtfalse

\input{../tex-inputs/latex-vorspann}

\begin{center}
            \textcolor{red}{ENTWURF, NICHT FERTIG KORRIGIERT}
                      \end{center}
            
         
         \renewcommand{\erwaehntePersonen}{Personen: Samuel Fischer, Hans Jacob, Felix Salten, Franz Werfel, Paul Zsolnay}
         \renewcommand{\erwaehnteInstitutionen}{Institutionen: Paul Zsolnay Verlag, S. Fischer Verlag}
         \renewcommand{\erwaehnteOrte}{Orte: Berlin, Paris, Wien, XVIII., Währing}
         \renewcommand{\erwaehnteWerke}{Werke: Verdi}
               \section[ Arthur Schnitzler an Felix Salten, 10. 12. 1923]{ Arthur Schnitzler an Felix Salten, 10. 12. 1923}\nopagebreak\mylabel{v}\rehead{ }\begin{ledgroupsized}[t]{13cm}\normalsize\beginnumbering \toendnotes[C]{\smallbreak\pagebreak[2]} \Standort{DLA, A:Schnitzler, HS.NZ85.1.1751.}
\physDesc{Brief, Durchschlag, 1 Blatt, 1 Seite, 918 Zeichen
\newline{}maschinell
\newline{}Handschrift: roter Buntstift, lateinische Kurrent (\noindent{}drei Unterstreichungen und in der linken oberen Ecke Vermerk: »Salten\pwindex{Salten, Felix 06.09.1869 – 08.10.1945@\textsc{Salten, Felix} (06.09.1869 – 08.10.1945), \emph{Schriftsteller, Journalist}|pw}«)}\toendnotes[C]{\smallbreak}\pstart
           \raggedleft{}{\pb}10. 12. 1923.\pend
           \pstart{}Lieber,\pend\pstart
           \label{K_L02948-1v}\edtext{gestern war Hans
                  Jacob\pwindex{Jacob, Hans 20.11.1896 – 06.03.1961@\textsc{Jacob, Hans} (20.11.1896 – 06.03.1961), \emph{Schriftsteller, Übersetzer}|pw} bei mir}{\lemma{\textnormal{\emph{gestern … mir}}}\Cendnote{\textnormal{siehe A. S.: \emph{Tagebuch}, 9. 12. 1923}}}\label{K_L02948-1h}, von dem ich Ihnen neulich sprach und der mir in meinen Verhandlungen mit S. Fischer\orgindex{S. Fischer Verlag@S. Fischer Verlag|pw}\pwindex{Fischer, Samuel 24.12.1859 – 15.10.1934@\textsc{Fischer, Samuel} (24.12.1859 – 15.10.1934), \emph{Verleger}|pw} in der letzten Zeit ganz unschätzbare Dienste geleistet hat. Das Gespräch kam
               begreiflicherweise auch auf hiesige Verlagsgründungen, eine Frage, die mich momentan
               aus in Ihnen bekannten Gründen besonders interessiert, ist insbesondere die
               Angliederung eines Theatervertriebs an den \label{K_L02948-2v}\edtext{Buchverlag, den Zsolnay\pwindex{Zsolnay, Paul 1895-06-12 – 13.05.1961@\textsc{Zsolnay, Paul} (1895-06-12 – 13.05.1961), \emph{Verleger}|pw}
               zu gründen gedenkt}{\lemma{\textnormal{\emph{Buchverlag, … gedenkt}}}\Cendnote{\textnormal{Paul Zsolnay\pwindex{Zsolnay, Paul 1895-06-12 – 13.05.1961@\textsc{Zsolnay, Paul} (1895-06-12 – 13.05.1961), \emph{Verleger}|pwk}s Bemühungen um die Gründung
                  eines eigenen Verlags manifestierten sich in den kommenden Wochen. Im April 1924 erschien das erste Buch im \emph{Paul Zsolnay Verlag}\orgindex{Paul Zsolnay Verlag@Paul Zsolnay Verlag|pwk}: Franz Werfel\pwindex{Werfel, Franz 10.09.1890 – 26.08.1945@\textsc{Werfel, Franz} (10.09.1890 – 26.08.1945), \emph{Schriftsteller}|pwk}s \emph{Verdi}\pwindex{Werfel, Franz 10.09.1890 – 26.08.1945@\textsc{Werfel, Franz} (10.09.1890 – 26.08.1945), \emph{Schriftsteller}!Verdi@\strich\emph{Verdi}|pwk}.}}}\label{K_L02948-2h}. Aber
               auch allerlei anderes kam zur Sprache und Hans
                  Jacob\pwindex{Jacob, Hans 20.11.1896 – 06.03.1961@\textsc{Jacob, Hans} (20.11.1896 – 06.03.1961), \emph{Schriftsteller, Übersetzer}|pw} berichtete mir viel, was, wie ich glaube, auch für Z.\pwindex{Zsolnay, Paul 1895-06-12 – 13.05.1961@\textsc{Zsolnay, Paul} (1895-06-12 – 13.05.1961), \emph{Verleger}|pw} mancherlei Interesse haben könnte. Ich will Sie heute nur fragen, lieber, ob Sie einmal für Hans Jacob\pwindex{Jacob, Hans 20.11.1896 – 06.03.1961@\textsc{Jacob, Hans} (20.11.1896 – 06.03.1961), \emph{Schriftsteller, Übersetzer}|pw} (der für einige, vielleicht längere
               Zeit aus Berlin\oindex{Berlin@\textbf{Berlin}|pw} hier ist) eine halbe Stunde Zeit
               haben. Er würde besonderen Wert darauf legen Sie zu sprechen. Darf ich ihm eine
               günstige Botschaft bestellen?\pend
           \pstart Auf bald und sehr herzliche Grüsse\pend{}{\bigskip}\pstart
           \noindent{}Herrn Felix Salten,\pend
           \pstart
           Wien XVIII.\oindex{XVIII., Waehring@\textbf{XVIII., Währing}|pw}\pend
           
         
         \endnumbering\mylabel{h}\end{ledgroupsized}  \newcommand{\dateiname}{L02948}\newcommand{\titel}{Arthur Schnitzler an Felix Salten, 10. 12. 1923}\newcommand{\editorInnen}{Martin Anton Müller und Laura Untner}%% latex-leseansicht-abspann.tex
%% Abspann für die Leseansicht.
%% Der Schalter \ifkorrekturansicht ist bereits durch den Vorspann gesetzt.

%% latex-abspann.tex
%% Gemeinsamer Abspann für Korrekturansicht und Leseansicht.
%% Setzt den Schalter \ifkorrekturansicht voraus (gesetzt in den
%% einbindenden Dateien latex-korrekturansicht-abspann.tex bzw.
%% latex-leseansicht-abspann.tex).
%% ---------------------------------------------------------------

\normalsize

% Das esempio-Environment wird nur in der Leseansicht benötigt
\ifkorrekturansicht\else
\newenvironment{esempio}[3]%
{
    \vspace{1.5ex}
    \rlap{\underline{#1}}
    \par
    \setlength{\parindent}{0cm}
    \nopagebreak
    \leftskip=#2cm
    \rightskip=#3cm
}
{
    \par
}
\fi

\doendnotes{C}
\bigskip
\vfill

\clearpage

\footnotesize

\ifkorrekturansicht
  \lohead{\textsc{register}}
\fi

% theindex-Environment neu definieren ohne reledmac
\makeatletter
\renewenvironment{theindex}{%
  \ifkorrekturansicht
    \section*{\indexname}%
  \else
    \subsubsection*{Index der erwähnten Entitäten}%
  \fi
  \setlength{\parindent}{0pt}%
  \setlength{\parskip}{0pt plus 0.3pt}%
  \let\item\@idxitem
}{%
  \ifkorrekturansicht\clearpage\fi
}
\makeatother

\IfFileExists{\jobname-pw.ind}{\input{\jobname-pw.ind}}{}

% Quellenangabe nur in der Leseansicht
\ifkorrekturansicht\else
% Fallback-Definitionen, falls die .tex-Datei \titel etc. nicht gesetzt hat
\providecommand{\titel}{}
\providecommand{\editorInnen}{}
\providecommand{\dateiname}{\jobname}

\vspace{3cm}

\vfill

\footnotesize
\textsc{Quelle}: \titel. Herausgegeben von {\editorInnen}. In: \emph{Arthur Schnitzler: Briefwechsel mit Autorinnen und Autoren}.
 Digitale Edition, https://schnitzler-briefe.acdh.oeaw.ac.at/{\dateiname}.html (Stand \today)
\fi

\end{document}


      