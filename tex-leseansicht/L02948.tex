%% latex-korrekturansicht-vorspann.tex
%% Vorspann für die Korrekturansicht.
%% Lädt die gemeinsame Datei latex-vorspann.tex mit gesetztem Schalter.

\newif\ifkorrekturansicht
\korrekturansichttrue

\input{../tex-inputs/latex-vorspann}


\section[ Arthur Schnitzler an Felix Salten, 10. 12. 1923]{L02948 Arthur Schnitzler an Felix Salten, 10. 12. 1923}
\nopagebreak\mylabel{L02948v}
\rehead{ }\normalsize\beginnumbering\briefempfaengerindex{Salten, Felix@\textsc{Salten, Felix}!zzzSchnitzler, Arthur@\emph{von Arthur Schnitzler}!1923-12-101@{10. 12. 1923}|(be}
\toendnotes[C]{\smallbreak\pagebreak[2]}\Standort{DLA, A:Schnitzler, HS.NZ85.1.1751.}
\physDesc{Brief, Durchschlag1 Blatt, 1 Seite, 918 Zeichen
\newline{}maschinell
\newline{}Handschrift: roter Buntstift, lateinische Kurrent (\noindent{}in der linken oberen Ecke Vermerk: »Salten\pwindex{Salten, Felix 06.09.1869 – 08.10.1945@\textsc{Salten, Felix} (06.09.1869 – 08.10.1945), \emph{Schriftsteller/Schriftstellerin, Journalist/Journalistin, Chefredakteur/Chefredakteurin}|pw}« und drei Unterstreichungen)}\toendnotes[C]{\smallbreak}
\pstart
           \raggedleft{}{\pb}10. 12. 1923.\pend
           
\pstart{}Lieber,\pend\vspace{0.5em}
\pstart
           \label{K_L02948-1v}\edtext{gestern war Hans
                  Jacob\pwindex{Jacob, Hans 20.11.1896 – 06.03.1961@\textsc{Jacob, Hans} (20.11.1896 – 06.03.1961), \emph{Schriftsteller/Schriftstellerin, Übersetzer/Übersetzerin}|pw} bei mir}{\lemma{\textnormal{\emph{gestern … mir}}}\Cendnote{\textnormal{Siehe A. S.: \emph{Tagebuch}, 9. 12. 1923.
               }}}\label{K_L02948-1}, von dem ich Ihnen neulich sprach und der mir in meinen Verhandlungen mit S. Fischer\orgindex{S. Fischer Verlag@S. Fischer Verlag|pw}\pwindex{Fischer, Samuel 24.12.1859 – 15.10.1934@\textsc{Fischer, Samuel} (24.12.1859 – 15.10.1934), \emph{Verleger/Verlegerin}|pw} in der letzten Zeit ganz unschätzbare Dienste geleistet hat. Das Gespräch kam
               begreiflicherweise auch auf hiesige Verlagsgründungen, eine Frage, die mich momentan
               aus in Ihnen bekannten Gründen besonders interessiert, ist insbesondere die
               Angliederung eines Theatervertriebs an den \label{K_L02948-2v}\edtext{Buchverlag, den Zsolnay\pwindex{Zsolnay, Paul 1895-06-12 – 13.05.1961@\textsc{Zsolnay, Paul} (1895-06-12 – 13.05.1961), \emph{Verleger/Verlegerin}|pw}
               zu gründen gedenkt}{\lemma{\textnormal{\emph{Buchverlag, … gedenkt}}}\Cendnote{\textnormal{Paul Zsolnays\pwindex{Zsolnay, Paul 1895-06-12 – 13.05.1961@\textsc{Zsolnay, Paul} (1895-06-12 – 13.05.1961), \emph{Verleger/Verlegerin}|pwk} Bemühungen um die Gründung
                  eines eigenen Verlags manifestierten sich in den kommenden Wochen. Im April 1924 erschien das erste Buch im \emph{Paul Zsolnay Verlag}\orgindex{Paul Zsolnay Verlag@Paul Zsolnay Verlag|pwk}: Franz Werfels\pwindex{Werfel, Franz 10.09.1890 – 26.08.1945@\textsc{Werfel, Franz} (10.09.1890 – 26.08.1945), \emph{Schriftsteller/Schriftstellerin}|pwk}{ }\emph{Verdi}\pwindex{Verdi. Roman der Oper@\emph{Verdi. Roman der Oper}|pwk}.}}}\label{K_L02948-2}. Aber
               auch allerlei anderes kam zur Sprache und Hans
                  Jacob\pwindex{Jacob, Hans 20.11.1896 – 06.03.1961@\textsc{Jacob, Hans} (20.11.1896 – 06.03.1961), \emph{Schriftsteller/Schriftstellerin, Übersetzer/Übersetzerin}|pw} berichtete mir viel, was, wie ich glaube, auch für Z.\pwindex{Zsolnay, Paul 1895-06-12 – 13.05.1961@\textsc{Zsolnay, Paul} (1895-06-12 – 13.05.1961), \emph{Verleger/Verlegerin}|pw} mancherlei Interesse haben könnte. Ich will Sie heute nur fragen, lieber, ob Sie einmal für Hans Jacob\pwindex{Jacob, Hans 20.11.1896 – 06.03.1961@\textsc{Jacob, Hans} (20.11.1896 – 06.03.1961), \emph{Schriftsteller/Schriftstellerin, Übersetzer/Übersetzerin}|pw} (der für einige, vielleicht längere
               Zeit aus Berlin\oindex{Berlin@\textbf{Berlin}, \emph{P.PPLC}|pw} hier ist) eine halbe Stunde Zeit
               haben. Er würde besonderen Wert darauf legen Sie zu sprechen. Darf ich ihm eine
               günstige Botschaft bestellen?\pend
           \pstart Auf bald und sehr herzliche Grüsse\pend{}{\vspace{1\baselineskip}}
\pstart
           \noindent{}Herrn Felix Salten,\pend
           
\pstart
           Wien XVIII.\oindex{XVIII., Waehring@\textbf{XVIII., Währing}, \emph{A.ADM3}|pw}\pend
           \selectlanguage{ngerman}\endnumbering\briefempfaengerindex{Salten, Felix@\textsc{Salten, Felix}!zzzSchnitzler, Arthur@\emph{von Arthur Schnitzler}!1923-12-101@{10. 12. 1923}|)be}\mylabel{L02948h}  \normalsize

\doendnotes{C}
\bigskip
\vfill

\clearpage

\footnotesize

\lohead{\textsc{register}}

% Definiere theindex-Environment komplett neu ohne reledmac
\makeatletter
\renewenvironment{theindex}{%
  \section*{\indexname}%
  \setlength{\parindent}{0pt}%
  \setlength{\parskip}{0pt plus 0.3pt}%
  \let\item\@idxitem
}{%
  \clearpage
}
\makeatother

\IfFileExists{\jobname-pw.ind}{\input{\jobname-pw.ind}}{}

\end{document}

      