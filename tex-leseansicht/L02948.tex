%% latex-leseansicht-vorspann.tex
%% Vorspann für die Leseansicht.
%% Lädt die gemeinsame Datei latex-vorspann.tex mit nicht gesetztem Schalter.

\newif\ifkorrekturansicht
\korrekturansichtfalse

\input{../tex-inputs/latex-vorspann}


\section[ Arthur Schnitzler an Felix Salten, 10. 12. 1923]{L02948 Arthur Schnitzler an Felix Salten,  10. 12. 1923}
\nopagebreak\mylabel{L02948v}
\rehead{ }\normalsize\beginnumbering\briefempfaengerindex{Salten, Felix@\textsc{Salten, Felix}!zzzSchnitzler, Arthur@\emph{von Arthur Schnitzler}!1923-12-101@{10. 12. 1923}|(be}
\toendnotes[C]{\smallbreak\pagebreak[2]}
\correspDesc{Versand  durch Arthur Schnitzler am 10. 12. 1923 in Paris
\newline{}Erhalt  durch Felix Salten im Zeitraum [11. 12. 1923 – 15. 12. 1923?] in Wien}\toendnotes[C]{\smallbreak}
\Standort{DLA, A:Schnitzler, HS.NZ85.1.1751.}
\physDesc{Brief, Durchschlag, 1 Blatt, 1 Seite, 918 Zeichen
\newline{}maschinell
\newline{}Handschrift: roter Buntstift, lateinische Kurrent (\noindent{}in der linken oberen Ecke Vermerk: »Salten\pwindex{Salten, Felix 6.\,9.\,1869 Budapest – 8.\,10.\,1945 Zürich@\textsc{Salten, Felix} (6.\,9.\,1869 Budapest – 8.\,10.\,1945 Zürich), \emph{Schriftsteller, Journalist, Chefredakteur}|pw}« und drei Unterstreichungen)}\toendnotes[C]{\smallbreak}
\pstart
           \raggedleft{}{\pb}10. 12. 1923.\pend
           
\pstart{}Lieber,\pend\vspace{0.5em}
\pstart
           \label{K_L02948-1v}\edtext{gestern war Hans
                  Jacob\pwindex{Jacob, Hans 20.\,11.\,1896 Berlin – 6.\,3.\,1961 Paris@\textsc{Jacob, Hans} (20.\,11.\,1896 Berlin – 6.\,3.\,1961 Paris), \emph{Schriftsteller, Übersetzer}|pw} bei mir}{\lemma{\textnormal{\emph{gestern … mir}}}\Cendnote{\textnormal{Siehe A. S.: \emph{Tagebuch}, 9. 12. 1923.
               }}}\label{K_L02948-1}, von dem ich Ihnen neulich sprach und der mir in meinen Verhandlungen mit S. Fischer\orgindex{S. Fischer Verlag@S. Fischer Verlag|pw}\pwindex{Fischer, Samuel 24.\,12.\,1859 Liptovský Mikuláš – 15.\,10.\,1934 Berlin@\textsc{Fischer, Samuel} (24.\,12.\,1859 Liptovský Mikuláš – 15.\,10.\,1934 Berlin), \emph{Verleger}|pw} in der letzten Zeit ganz unschätzbare Dienste geleistet hat. Das Gespräch kam
               begreiflicherweise auch auf hiesige Verlagsgründungen, eine Frage, die mich momentan
               aus in Ihnen bekannten Gründen besonders interessiert, ist insbesondere die
               Angliederung eines Theatervertriebs an den \label{K_L02948-2v}\edtext{Buchverlag, den Zsolnay\pwindex{Zsolnay, Paul 12.\,6.\,1895 Budapest – 13.\,5.\,1961 Wien@\textsc{Zsolnay, Paul} (12.\,6.\,1895 Budapest – 13.\,5.\,1961 Wien), \emph{Verleger}|pw}
               zu gründen gedenkt}{\lemma{\textnormal{\emph{Buchverlag, … gedenkt}}}\Cendnote{\textnormal{Paul Zsolnays\pwindex{Zsolnay, Paul 12.\,6.\,1895 Budapest – 13.\,5.\,1961 Wien@\textsc{Zsolnay, Paul} (12.\,6.\,1895 Budapest – 13.\,5.\,1961 Wien), \emph{Verleger}|pwk} Bemühungen um die Gründung
                  eines eigenen Verlags manifestierten sich in den kommenden Wochen. Im April 1924 erschien das erste Buch im \emph{Paul Zsolnay Verlag}\orgindex{Paul Zsolnay Verlag@Paul Zsolnay Verlag|pwk}: Franz Werfels\pwindex{Werfel, Franz 10.\,9.\,1890 Prag – 26.\,8.\,1945 Beverly Hills@\textsc{Werfel, Franz} (10.\,9.\,1890 Prag – 26.\,8.\,1945 Beverly Hills), \emph{Schriftsteller}|pwk}{ }\emph{Verdi}\pwindex{Werfel, Franz 10.\,9.\,1890 Prag – 26.\,8.\,1945 Beverly Hills@\textsc{Werfel, Franz} (10.\,9.\,1890 Prag – 26.\,8.\,1945 Beverly Hills), \emph{Schriftsteller}!Verdi. Roman der Oper@\strich\emph{Verdi. Roman der Oper}|pwk}.}}}\label{K_L02948-2}. Aber
               auch allerlei anderes kam zur Sprache und Hans
                  Jacob\pwindex{Jacob, Hans 20.\,11.\,1896 Berlin – 6.\,3.\,1961 Paris@\textsc{Jacob, Hans} (20.\,11.\,1896 Berlin – 6.\,3.\,1961 Paris), \emph{Schriftsteller, Übersetzer}|pw} berichtete mir viel, was, wie ich glaube, auch für Z.\pwindex{Zsolnay, Paul 12.\,6.\,1895 Budapest – 13.\,5.\,1961 Wien@\textsc{Zsolnay, Paul} (12.\,6.\,1895 Budapest – 13.\,5.\,1961 Wien), \emph{Verleger}|pw} mancherlei Interesse haben könnte. Ich will Sie heute nur fragen, lieber, ob Sie einmal für Hans Jacob\pwindex{Jacob, Hans 20.\,11.\,1896 Berlin – 6.\,3.\,1961 Paris@\textsc{Jacob, Hans} (20.\,11.\,1896 Berlin – 6.\,3.\,1961 Paris), \emph{Schriftsteller, Übersetzer}|pw} (der für einige, vielleicht längere
               Zeit aus Berlin\oindex{Berlin@\textbf{Berlin}, \emph{Hauptstadt}|pw} hier ist) eine halbe Stunde Zeit
               haben. Er würde besonderen Wert darauf legen Sie zu sprechen. Darf ich ihm eine
               günstige Botschaft bestellen?\pend
           \pstart Auf bald und sehr herzliche Grüsse\pend{}{\vspace{1\baselineskip}}
\pstart
           \noindent{}Herrn Felix Salten,\pend
           
\pstart
           Wien XVIII.\oindex{XVIII., Währing@\textbf{XVIII., Währing}, \emph{Verwaltungsgebiet}|pw}\pend
           \selectlanguage{ngerman}\endnumbering\briefempfaengerindex{Salten, Felix@\textsc{Salten, Felix}!zzzSchnitzler, Arthur@\emph{von Arthur Schnitzler}!1923-12-101@{10. 12. 1923}|)be}\mylabel{L02948h}  \newcommand{\dateiname}{L02948}\newcommand{\titel}{Arthur Schnitzler an Felix Salten, 10. 12. 1923}\newcommand{\editorInnen}{Martin Anton Müller und Laura Untner}%% latex-leseansicht-abspann.tex
%% Abspann für die Leseansicht.
%% Der Schalter \ifkorrekturansicht ist bereits durch den Vorspann gesetzt.

%% latex-abspann.tex
%% Gemeinsamer Abspann für Korrekturansicht und Leseansicht.
%% Setzt den Schalter \ifkorrekturansicht voraus (gesetzt in den
%% einbindenden Dateien latex-korrekturansicht-abspann.tex bzw.
%% latex-leseansicht-abspann.tex).
%% ---------------------------------------------------------------

\normalsize

% Das esempio-Environment wird nur in der Leseansicht benötigt
\ifkorrekturansicht\else
\newenvironment{esempio}[3]%
{
    \vspace{1.5ex}
    \rlap{\underline{#1}}
    \par
    \setlength{\parindent}{0cm}
    \nopagebreak
    \leftskip=#2cm
    \rightskip=#3cm
}
{
    \par
}
\fi

\doendnotes{C}
\bigskip
\vfill

\clearpage

\footnotesize

\ifkorrekturansicht
  \lohead{\textsc{register}}
\fi

% theindex-Environment neu definieren ohne reledmac
\makeatletter
\renewenvironment{theindex}{%
  \ifkorrekturansicht
    \section*{\indexname}%
  \else
    \subsubsection*{Index der erwähnten Entitäten}%
  \fi
  \setlength{\parindent}{0pt}%
  \setlength{\parskip}{0pt plus 0.3pt}%
  \let\item\@idxitem
}{%
  \ifkorrekturansicht\clearpage\fi
}
\makeatother

\IfFileExists{\jobname-pw.ind}{\input{\jobname-pw.ind}}{}

% Quellenangabe nur in der Leseansicht
\ifkorrekturansicht\else
% Fallback-Definitionen, falls die .tex-Datei \titel etc. nicht gesetzt hat
\providecommand{\titel}{}
\providecommand{\editorInnen}{}
\providecommand{\dateiname}{\jobname}

\vspace{3cm}

\vfill

\footnotesize
\textsc{Quelle}: \titel. Herausgegeben von {\editorInnen}. In: \emph{Arthur Schnitzler: Briefwechsel mit Autorinnen und Autoren}.
 Digitale Edition, https://schnitzler-briefe.acdh.oeaw.ac.at/{\dateiname}.html (Stand \today)
\fi

\end{document}


