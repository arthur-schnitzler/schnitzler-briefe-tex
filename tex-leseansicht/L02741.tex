%% latex-leseansicht-vorspann.tex
%% Vorspann für die Leseansicht.
%% Lädt die gemeinsame Datei latex-vorspann.tex mit nicht gesetztem Schalter.

\newif\ifkorrekturansicht
\korrekturansichtfalse

\input{../tex-inputs/latex-vorspann}


\section[Paul Goldmann an Arthur Schnitzler, 25. 7. {[}1895{]}]{L02741 Paul Goldmann an Arthur Schnitzler, 25. 7. [1895]}
\nopagebreak\mylabel{L02741v}
\rehead{ }\normalsize\beginnumbering\briefempfaengerindex{Schnitzler, Arthur@\textsc{Schnitzler, Arthur}!zzzGoldmann, Paul@\emph{von Paul Goldmann}!1895-07-252@{25. 7. [1895]}|(be}
\toendnotes[C]{\smallbreak\pagebreak[2]}
\correspDesc{Versand  durch Paul Goldmann am 25. 7. [1895] in Paris
\newline{}Erhalt  durch Arthur Schnitzler im Zeitraum [26. 7. 1895
                  – 30. 7. 1895?] in Bad Ischl}\toendnotes[C]{\smallbreak}
\Standort{DLA, A:Schnitzler, HS.NZ85.1.3165.}
\physDesc{Brief, 2 Blätter, 8 Seiten, 3752 Zeichen
\newline{}Handschrift: schwarze Tinte, deutsche Kurrent
\newline{}Schnitzler: 1) mit Bleistift das Jahr »95« vermerkt  2) mit rotem Buntstift zwei Unterstreichungen}\toendnotes[C]{\smallbreak}
\pstart
           {\pb}\textcolor{gray}{\textbf{\textbf{Frankfurter Zeitung\orgindex{Frankfurter Zeitung@Frankfurter Zeitung|pw}}}}\pend
           
\pstart
           \textcolor{gray}{\textbf{(\begin{otherlanguage}{french}Gazette de Francfort\end{otherlanguage}\orgindex{Frankfurter Zeitung@Frankfurter Zeitung|pw}).}}\pend
           
\pstart
           \textcolor{gray}{\textbf{\textbf{\begin{otherlanguage}{french}Fondateur M. L.
                                 Sonnemann\pwindex{Sonnemann, Leopold 29.\,10.\,1831 Höchberg – 30.\,10.\,1909 Frankfurt am Main@\textsc{Sonnemann, Leopold} (29.\,10.\,1831 Höchberg – 30.\,10.\,1909 Frankfurt am Main), \emph{Journalist, Herausgeber}|pw}\end{otherlanguage}.}}}\hfill \textsc{Paris\oindex{Paris@\textbf{Paris}, \emph{Hauptstadt}|pw}}, 25. Juli.\pend
           
\pstart
           \begin{otherlanguage}{french}\textcolor{gray}{\textbf{Journal politique, financier,}}\end{otherlanguage}\pend
           
\pstart
           \begin{otherlanguage}{french}\textcolor{gray}{\textbf{commercial et littéraire.}}\end{otherlanguage}\pend
           
\pstart
           \begin{otherlanguage}{french}\textcolor{gray}{\textbf{\textbf{Paraissant trois fois par jour.}}}\end{otherlanguage}\pend
           
\pstart
           \begin{otherlanguage}{french}\textcolor{gray}{\textbf{\textbf{Bureau à Paris\oindex{Paris@\textbf{Paris}, \emph{Hauptstadt}|pw}}}}\end{otherlanguage}\pend
           
\pstart
           \begin{otherlanguage}{french}\textcolor{gray}{\textbf{\textbf{24. Rue Feydeau\oindex{rue Feydeau@\textbf{rue Feydeau}, \emph{Straße}|pw}.}}}\end{otherlanguage}\pend
           
\pstart\center{}Mein lieber Freund,\pend\vspace{0.5em}
\pstart
           Gern hätte ich Dir Deinen lieben Brief von neulich gleich beantwortet. Aber es gab
               gar{ }ſoviel zu thun.\pend
           
\pstart
           Alſo Ihr geht doch noch nach \textsc{Kopenhagen\oindex{Kopenhagen@\textbf{Kopenhagen}, \emph{Hauptstadt}|pw}}? Habt Ihr Nachrichten von Frau \textsc{Andreas\pwindex{Andreas-Salomé, Lou 12.\,2.\,1861 Sankt Petersburg – 5.\,2.\,1937 Göttingen@\textsc{Andreas-Salomé, Lou} (12.\,2.\,1861 Sankt Petersburg – 5.\,2.\,1937 Göttingen), \emph{Schriftstellerin}|pw}}?\pend
           
\pstart
           Was mich anlangt,{ }ſo gedenke ich am 1. Auguſt hier
               abzureiſen. Ich gehe nach \textsc{Toelz\oindex{Bad Tölz@\textbf{Bad Tölz}, \emph{Hauptstadt}|pw}} zum Kur-Gebrauche. Ich bin{ }ſehr krank. Seit faſt einem Jahre leide ich an einer
               unerklärlichen Affection des rechten Auges: \textsc{Pupillen}-Ungleichheit. Schmerzen, {\pb}Sehſtörungen \textsc{etc}. Die \strikeout{Ä\textcolor{gray}{r}zte} Ärzte{ }ſagen mir nichts u. drängen nur zur Kur.
               Ich fürchte \label{K_L02741-1v}\edtext{\textsc{tumor cerebri}}{\lemma{\textnormal{\emph{tumor cerebri}}}\Cendnote{\textnormal{lateinisch: Hirntumor}}}\label{K_L02741-1}.\pend
           
\pstart
           So bleibe ich alſo in \textsc{Toelz\oindex{Bad Tölz@\textbf{Bad Tölz}, \emph{Hauptstadt}|pw}} vorausſichtlich vier Wochen. \textsc{Toelz\oindex{Bad Tölz@\textbf{Bad Tölz}, \emph{Hauptstadt}|pw}} liegt etwa zwei Bahnſtunden von \textsc{Muenchen\oindex{München@\textbf{München}|pw}} entfernt. Zwiſchen dem 23. u. 30. Auguſt bin ich jedenfalls noch dort. Vielleicht
               treffen wir uns alſo in \textsc{Muenchen\oindex{München@\textbf{München}|pw}} (wenn ich die Kur unterbrechen darf). Oder auch{ }ſonſtwo – ich erwarte Deine
               Dispoſitionen. Wenn Du mir{ }ſofort antworteſt,{ }ſo erreicht mich ein Brief von Dir noch
               hier. Jedenfalls theile ich Dir {\pb}ſofort meine \strikeout{U\textcolor{gray}{n}} Unterwegs-Adreſſe mit, und wir bleiben dann wohl in Verbindung. Wie innig ich
               mich darauf freue, Dich wiederzuſehen, brauche ich kaum zu{ }ſagen. Und \label{K_L02741-2v}\edtext{\textsc{Richard\pwindex{Beer-Hofmann, Richard 11.\,7.\,1866 Wien – 26.\,9.\,1945 New York City@\textsc{Beer-Hofmann, Richard} (11.\,7.\,1866 Wien – 26.\,9.\,1945 New York City), \emph{Schriftsteller}|pw}}}{\lemma{\textnormal{\emph{Richard}}}\Cendnote{\textnormal{Goldmann\pwindex{Goldmann, Paul 31.\,1.\,1865 Breslau – 25.\,9.\,1935 Wien@\textsc{Goldmann, Paul} (31.\,1.\,1865 Breslau – 25.\,9.\,1935 Wien), \emph{Schriftsteller, Journalist}|pwk}, Schnitzler und Richard
                     Beer-Hofmann\pwindex{Beer-Hofmann, Richard 11.\,7.\,1866 Wien – 26.\,9.\,1945 New York City@\textsc{Beer-Hofmann, Richard} (11.\,7.\,1866 Wien – 26.\,9.\,1945 New York City), \emph{Schriftsteller}|pwk} sahen sich zwischen 31. 8. 1895 und 6. 9. 1895 mehrfach in und um München\oindex{München@\textbf{München}|pwk}.}}}\label{K_L02741-2}, werde ich den auch{ }ſehen?\pend
           
\pstart
           Ich habe oft in dieſen Wochen der{ }ſchönen Tage im \label{K_L02741-3v}\edtext{vorigen Jahre}{\lemma{\textnormal{\emph{vorigen Jahre}}}\Cendnote{\textnormal{Siehe A. S.: \emph{Tagebuch}, 23. 8. 1894.
               }}}\label{K_L02741-3} gedacht. Ich wünſchte, ich könnte wieder hin, nach \textsc{Ischl\oindex{Bad Ischl@\textbf{Bad Ischl}|pw}}{ }\strikeout{z} und zu Euch. Ich habe Heimweh nach dem Allen. Du
               ahnſt nicht, mein lieber Freund, wie verzweifelt und troſtlos ich bin. Manchmal{ }ſtaune ich über mich{ }ſelber, daß ich {\pb}noch
                  aufrechtſtehe{\dotssix}\pend
           
\pstart
           Ich{ }ſende Dir anbei die geſammelten Artikel\pwindex{Becque, Henry 9.\,4.\,1837 Paris – 12.\,5.\,1899 Neuilly-sur-Seine@\textsc{Becque, Henry} (9.\,4.\,1837 Paris – 12.\,5.\,1899 Neuilly-sur-Seine), \emph{Schriftsteller, Dramatiker}!Querelles littéraires@\strich\emph{Querelles littéraires}|pwv} von \textsc{Henry Becque\pwindex{Becque, Henry 9.\,4.\,1837 Paris – 12.\,5.\,1899 Neuilly-sur-Seine@\textsc{Becque, Henry} (9.\,4.\,1837 Paris – 12.\,5.\,1899 Neuilly-sur-Seine), \emph{Schriftsteller, Dramatiker}|pw}}, mit der Bitte, mir das \label{K_L02741-4v}\edtext{Buch\pwindex{Becque, Henry 9.\,4.\,1837 Paris – 12.\,5.\,1899 Neuilly-sur-Seine@\textsc{Becque, Henry} (9.\,4.\,1837 Paris – 12.\,5.\,1899 Neuilly-sur-Seine), \emph{Schriftsteller, Dramatiker}!Querelles littéraires@\strich\emph{Querelles littéraires}|pwv}}{\lemma{\textnormal{\emph{Buch}}}\Cendnote{\textnormal{Henry Becque\pwindex{Becque, Henry 9.\,4.\,1837 Paris – 12.\,5.\,1899 Neuilly-sur-Seine@\textsc{Becque, Henry} (9.\,4.\,1837 Paris – 12.\,5.\,1899 Neuilly-sur-Seine), \emph{Schriftsteller, Dramatiker}|pwk}: \emph{Querelles Littéraires}\pwindex{Becque, Henry 9.\,4.\,1837 Paris – 12.\,5.\,1899 Neuilly-sur-Seine@\textsc{Becque, Henry} (9.\,4.\,1837 Paris – 12.\,5.\,1899 Neuilly-sur-Seine), \emph{Schriftsteller, Dramatiker}!Querelles littéraires@\strich\emph{Querelles littéraires}|pwk}. Avec un portrait hors texte.
                     Paris: \emph{Les éditions G. Crès}{ }1890.}}}\label{K_L02741-4} gelegentlich zurückzuſchicken. Es iſt Alles perſönliche Polemik,
               recht dürr und wenig erfreulich. Aber ich denke mir, wenn Dich die Theater-Canaillen
               kränken, wirſt Du vielleicht ein wenig Troſt darin finden, daß es Anderen noch{ }ſchlimmer geht. Auch iſt doch der Haß des \strikeout{Mann\pwindex{Becque, Henry 9.\,4.\,1837 Paris – 12.\,5.\,1899 Neuilly-sur-Seine@\textsc{Becque, Henry} (9.\,4.\,1837 Paris – 12.\,5.\,1899 Neuilly-sur-Seine), \emph{Schriftsteller, Dramatiker}|pwv}\textcolor{gray}{e}}{ }Mann\pwindex{Becque, Henry 9.\,4.\,1837 Paris – 12.\,5.\,1899 Neuilly-sur-Seine@\textsc{Becque, Henry} (9.\,4.\,1837 Paris – 12.\,5.\,1899 Neuilly-sur-Seine), \emph{Schriftsteller, Dramatiker}|pwv}es (\textsc{Becque\pwindex{Becque, Henry 9.\,4.\,1837 Paris – 12.\,5.\,1899 Neuilly-sur-Seine@\textsc{Becque, Henry} (9.\,4.\,1837 Paris – 12.\,5.\,1899 Neuilly-sur-Seine), \emph{Schriftsteller, Dramatiker}|pw}}) mit all’ dem Klatſch, den er aufrührt, manchmal recht amüſant. In den
               Druckſachen, die ich Dir dieſer Tage {\pb}ſandte, iſt
               diesmal wenig Beſonderes. Ich empfehle Dir nur in der »\textsc{Revue Blanche\pwindex{Revue blanche@\emph{La Revue blanche}|pw}}« \strikeout{die Geſchi} die recht nette \label{K_L02741-5v}\edtext{Geſchichte\pwindex{Muhlfeld, Lucien 4.\,8.\,1870 Paris – 1.\,12.\,1902 ebd.@\textsc{Muhlfeld, Lucien} (4.\,8.\,1870 Paris – 1.\,12.\,1902 ebd.), \emph{Schriftsteller, Literaturkritiker}!Pour le Cœur gros de la Poupée@\strich\emph{Pour le Cœur gros de la Poupée}|pwv}}{\lemma{\textnormal{\emph{Geschichte}}}\Cendnote{\textnormal{Lucien Muhlfeld\pwindex{Muhlfeld, Lucien 4.\,8.\,1870 Paris – 1.\,12.\,1902 ebd.@\textsc{Muhlfeld, Lucien} (4.\,8.\,1870 Paris – 1.\,12.\,1902 ebd.), \emph{Schriftsteller, Literaturkritiker}|pwk}: \emph{Pour le Cœur gros de la Poupée}\pwindex{Muhlfeld, Lucien 4.\,8.\,1870 Paris – 1.\,12.\,1902 ebd.@\textsc{Muhlfeld, Lucien} (4.\,8.\,1870 Paris – 1.\,12.\,1902 ebd.), \emph{Schriftsteller, Literaturkritiker}!Pour le Cœur gros de la Poupée@\strich\emph{Pour le Cœur gros de la Poupée}|pwk}. In: \emph{La revue blanche}\pwindex{Revue blanche@\emph{La Revue blanche}|pwk}, Jg. 9, Nr. 50, 1. 7. 1895, S. 14–18.}}}\label{K_L02741-5} von \textsc{Muhlfeld\pwindex{Muhlfeld, Lucien 4.\,8.\,1870 Paris – 1.\,12.\,1902 ebd.@\textsc{Muhlfeld, Lucien} (4.\,8.\,1870 Paris – 1.\,12.\,1902 ebd.), \emph{Schriftsteller, Literaturkritiker}|pw}}.\pend
           
\pstart
           Ob ich durch \textsc{Becque\pwindex{Becque, Henry 9.\,4.\,1837 Paris – 12.\,5.\,1899 Neuilly-sur-Seine@\textsc{Becque, Henry} (9.\,4.\,1837 Paris – 12.\,5.\,1899 Neuilly-sur-Seine), \emph{Schriftsteller, Dramatiker}|pw}} etwas für Deinen Verlag durchſetzen werde, weiß ich nicht. Er iſt{ }ſo{ }ſehr mit{ }ſich beſchäftigt, daß es{ }ſchwer iſt, ihn für einen Anderen dauernd zu
               intereſſiren.\pend
           
\pstart
           Daß dein Bruder\pwindex{Schnitzler, Julius 13.\,7.\,1865 Wien – 29.\,6.\,1939 ebd.@\textsc{Schnitzler, Julius} (13.\,7.\,1865 Wien – 29.\,6.\,1939 ebd.), \emph{Chirurg}|pwv} und Deine
                  Schwägerin\pwindex{Schnitzler, Helene 16.\,7.\,1871 Budapest – September 1941 Atlantischer Ozean@\textsc{Schnitzler, Helene} (16.\,7.\,1871 Budapest – September 1941 Atlantischer Ozean)|pwv} einen \label{K_L02741-6v}\edtext{Sohn\pwindex{Schnitzler, Hans 11.\,7.\,1895 Wien – 26.\,3.\,1967 Chicago@\textsc{Schnitzler, Hans} (11.\,7.\,1895 Wien – 26.\,3.\,1967 Chicago), \emph{Chirurg}|pwv}}{\lemma{\textnormal{\emph{Sohn}}}\Cendnote{\textnormal{Hans Schnitzler\pwindex{Schnitzler, Hans 11.\,7.\,1895 Wien – 26.\,3.\,1967 Chicago@\textsc{Schnitzler, Hans} (11.\,7.\,1895 Wien – 26.\,3.\,1967 Chicago), \emph{Chirurg}|pwk} wurde am 11. 7. 1895 geboren.}}}\label{K_L02741-6} haben, habe ich mit Freude
                  {\pb}vernommen. Ich glaube,{ }ſie konnten nichts
               Anderes haben als einen Sohn. Der wird ein geſcheiter und lieber Burſch\pwindex{Schnitzler, Hans 11.\,7.\,1895 Wien – 26.\,3.\,1967 Chicago@\textsc{Schnitzler, Hans} (11.\,7.\,1895 Wien – 26.\,3.\,1967 Chicago), \emph{Chirurg}|pwv} werden. Ich möchte ihnen
                  g\textcolor{gray}{er}n direct{ }ſchreiben und gratuliren, aber ich wags nicht. Denn
               ich habe mich noch immer nicht für das reizende \label{K_L02741-7v}\edtext{Bild}{\lemma{\textnormal{\emph{Bild}}}\Cendnote{\textnormal{Siehe XXXX Auszeichnungsfehler: Dokument L02726 nicht gefunden.
               }}}\label{K_L02741-7} bedankt, das{ }ſie mir zu Neujahr geſchenkt. Ich
               wollte die Antwort bis zum Gegengeſchenk aufſchieben und habe bis heut nichts Paſſendes gefunden. Was müſſen die{ }ſich von
               mir denken!\pend
           
\pstart
           {\pb}Deine Frau \label{K_L02741-8v}\edtext{Mutter\pwindex{Schnitzler, Louise 8.\,7.\,1840 Kőszeg – 9.\,9.\,1911 Wien@\textsc{Schnitzler, Louise} (8.\,7.\,1840 Kőszeg – 9.\,9.\,1911 Wien)|pwv}}{\lemma{\textnormal{\emph{Mutter}}}\Cendnote{\textnormal{Siehe A. S.: \emph{Tagebuch}, 18. 7. 1895.
               }}}\label{K_L02741-8} dürſte mit Dir{ }ſein. Bitte empfiehl’ mich ihr recht angelegentlich.\pend
           
\pstart
           Meine Mutter\pwindex{Goldmann, Clementine 15.\,5.\,1842 Breslau – 24.\,2.\,1924 Frankfurt am Main@\textsc{Goldmann, Clementine} (15.\,5.\,1842 Breslau – 24.\,2.\,1924 Frankfurt am Main)|pwv} iſt{ }ſeit zwei
               Monaten zu Beſuch bei mir \strikeout{und}. Wir{ }ſprechen oft von
               Dir, und{ }ſie dankt Dir die Freundſchaft, die Du mir bezeigſt, nicht minder, wie ich{ }ſelbſt. Sie iſt krank, die Ärmſte\pwindex{Goldmann, Clementine 15.\,5.\,1842 Breslau – 24.\,2.\,1924 Frankfurt am Main@\textsc{Goldmann, Clementine} (15.\,5.\,1842 Breslau – 24.\,2.\,1924 Frankfurt am Main)|pwv}, ohne es zu ahnen (\textsc{Diabetes}). Jetzt erſt,
               wo ich denken muß,{ }ſie zu verlieren,{ }ſehe ich, was{ }ſie mir iſt. Die Einzige\pwindex{Goldmann, Clementine 15.\,5.\,1842 Breslau – 24.\,2.\,1924 Frankfurt am Main@\textsc{Goldmann, Clementine} (15.\,5.\,1842 Breslau – 24.\,2.\,1924 Frankfurt am Main)|pwv} auf der Welt, die mich noch \strikeout{f\textcolor{gray}{ü}r} mit den alten {\pb}Augen anſieht, für die{ }ſich nichts geändert, für die
               ich noch der hoffnungsreiche und wohlgeſtalte Sohn bin! Und dieſe rührende,
               geräuſchloſe Liebe, die immer um Einen iſt, wie ein{ }ſtiller Segen, und nie etwas für{ }ſich verlangt! Manchmal gehen wir mitſammen über die Straße, und da denke ich, wie
                  \strikeout{trotz} ich{ }ſie mir{ }ſo nahe und{ }ſo unentbehrlich
               fühle und wie trotzdem bereits in jedem von uns das Grauenhafte lebendig iſt, das uns
               auseinanderreißen wird.\pend
           
\pstart
           Sie hat Dich{ }ſchon oft grüßen laſſen, ich habs aber immer vergeſſen.\pend
           
\pstart
           Leb’ wohl, liebſter Freund! {\\[\baselineskip]}Dein \spacefill\mbox{Paul Goldmn}\pend
           \leftskip=0em{}
\pstart
           \noindent{}Viele Grüße an \textsc{Richard\pwindex{Beer-Hofmann, Richard 11.\,7.\,1866 Wien – 26.\,9.\,1945 New York City@\textsc{Beer-Hofmann, Richard} (11.\,7.\,1866 Wien – 26.\,9.\,1945 New York City), \emph{Schriftsteller}|pw}}!\pend
           \selectlanguage{ngerman}\endnumbering\briefempfaengerindex{Schnitzler, Arthur@\textsc{Schnitzler, Arthur}!zzzGoldmann, Paul@\emph{von Paul Goldmann}!1895-07-252@{25. 7. [1895]}|)be}\mylabel{L02741h}  \newcommand{\dateiname}{L02741}\newcommand{\titel}{Paul Goldmann an Arthur Schnitzler, 25. 7. [1895]}\newcommand{\editorInnen}{Martin Anton Müller und Laura Untner}%% latex-leseansicht-abspann.tex
%% Abspann für die Leseansicht.
%% Der Schalter \ifkorrekturansicht ist bereits durch den Vorspann gesetzt.

%% latex-abspann.tex
%% Gemeinsamer Abspann für Korrekturansicht und Leseansicht.
%% Setzt den Schalter \ifkorrekturansicht voraus (gesetzt in den
%% einbindenden Dateien latex-korrekturansicht-abspann.tex bzw.
%% latex-leseansicht-abspann.tex).
%% ---------------------------------------------------------------

\normalsize

% Das esempio-Environment wird nur in der Leseansicht benötigt
\ifkorrekturansicht\else
\newenvironment{esempio}[3]%
{
    \vspace{1.5ex}
    \rlap{\underline{#1}}
    \par
    \setlength{\parindent}{0cm}
    \nopagebreak
    \leftskip=#2cm
    \rightskip=#3cm
}
{
    \par
}
\fi

\doendnotes{C}
\bigskip
\vfill

\clearpage

\footnotesize

\ifkorrekturansicht
  \lohead{\textsc{register}}
\fi

% theindex-Environment neu definieren ohne reledmac
\makeatletter
\renewenvironment{theindex}{%
  \ifkorrekturansicht
    \section*{\indexname}%
  \else
    \subsubsection*{Index der erwähnten Entitäten}%
  \fi
  \setlength{\parindent}{0pt}%
  \setlength{\parskip}{0pt plus 0.3pt}%
  \let\item\@idxitem
}{%
  \ifkorrekturansicht\clearpage\fi
}
\makeatother

\IfFileExists{\jobname-pw.ind}{\input{\jobname-pw.ind}}{}

% Quellenangabe nur in der Leseansicht
\ifkorrekturansicht\else
% Fallback-Definitionen, falls die .tex-Datei \titel etc. nicht gesetzt hat
\providecommand{\titel}{}
\providecommand{\editorInnen}{}
\providecommand{\dateiname}{\jobname}

\vspace{3cm}

\vfill

\footnotesize
\textsc{Quelle}: \titel. Herausgegeben von {\editorInnen}. In: \emph{Arthur Schnitzler: Briefwechsel mit Autorinnen und Autoren}.
 Digitale Edition, https://schnitzler-briefe.acdh.oeaw.ac.at/{\dateiname}.html (Stand \today)
\fi

\end{document}


