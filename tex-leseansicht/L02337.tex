%% latex-leseansicht-vorspann.tex
%% Vorspann für die Leseansicht.
%% Lädt die gemeinsame Datei latex-vorspann.tex mit nicht gesetztem Schalter.

\newif\ifkorrekturansicht
\korrekturansichtfalse

\input{../tex-inputs/latex-vorspann}


\section[Arthur Schnitzler an Ida Dehmel, 25. 2. 1920]{L02337 Arthur Schnitzler an Ida Dehmel, 25. 2. 1920}
\nopagebreak\mylabel{L02337v}
\rehead{ }\normalsize\beginnumbering\briefempfaengerindex{Dehmel, Ida@\textsc{Dehmel, Ida}!zzzSchnitzler, Arthur@\emph{von Arthur Schnitzler}!1920-02-251@{25. 2. 1920}|(be}
\toendnotes[C]{\smallbreak\pagebreak[2]}
\correspDesc{Versand  durch Arthur Schnitzler am 25. 2. 1920 in Wien
\newline{}Erhalt  durch Ida Dehmel im Zeitraum [26. 2. 1920
                  – 1. 3. 1920?] in Blankenese}\toendnotes[C]{\smallbreak}
\Standort{Hamburg, Staats- und Universitätsbibliothek, DA:Br:S:620.}
\physDesc{Brief, 1 Blatt, 2 Seiten, 984 Zeichen
\newline{}Handschrift: schwarze Tinte, lateinische Kurrent}\toendnotes[C]{\smallbreak}
\pstart
           \raggedleft{}{\pb}Wien\oindex{Wien@\textbf{Wien}, \emph{Verwaltungsgebiet}|pw}, 25. Feber 1920\pend
           \vspace{0.5em}
\pstart
           Verehrte Frau, erst heute komm ich Ihnen sagen, wie tief der Tod
               Ihres Gatten\pwindex{Dehmel, Richard 18.\,11.\,1863 Hermsdorf – 8.\,2.\,1920 Blankenese@\textsc{Dehmel, Richard} (18.\,11.\,1863 Hermsdorf – 8.\,2.\,1920 Blankenese), \emph{Schriftsteller, Schriftsteller, Krimiautor}|pw}, dieses großen Dichters, dieses
               hohen Menschen mich erschüttert hat. Als die traurige Nachricht kam, war mir, als
               hätt ich erst vor kurzem persönlich von ihm Abschied genommen, nach einem tagelangen
               von mancherlei aus lebendigster Unterhaltung erfülltem Zusa{\geminationm}ensein: so nahe war er mir in seinem Kriegs-{\pb}Tagebuch\pwindex{Dehmel, Richard 18.\,11.\,1863 Hermsdorf – 8.\,2.\,1920 Blankenese@\textsc{Dehmel, Richard} (18.\,11.\,1863 Hermsdorf – 8.\,2.\,1920 Blankenese), \emph{Schriftsteller, Schriftsteller, Krimiautor}!Zwischen Volk und Menschheit@\strich\emph{Zwischen Volk und Menschheit}|pwv}
               gewesen – ich hatte seine Stimme gehört, wie es mir so oft auch mit seinen Gedichten
               erging, – seinen Blick auf mir gefühlt; – denn in jedem Wort das er schrieb, in jedem
               das er sprach war seine ganze, seine wahrhaftige, seine große Seele. Und wie viele
               Jahre sind es nun schon her, daß ich ihn zum letzten Male gesehn!\pend
           
\pstart
           Meine Frau\pwindex{Schnitzler, Olga 17.\,1.\,1882 Wien – 13.\,1.\,1970 Lugano@\textsc{Schnitzler, Olga} (17.\,1.\,1882 Wien – 13.\,1.\,1970 Lugano), \emph{Schauspielerin, Sängerin}|pwv}, die ihn verehrt
               hat, gleich mir, schließt sich dem Ausdruck meiner innigsten Theilnahme aus vollem
               Herzen an. Wir denken Ihrer in schmerzlich-trostreicher Erinnerung schönerer Zeiten
               und mit den alten freundschaftlichen Gefühlen.\pend
           \pstart Ihr\spacefill\mbox{Arthur Schnitzler}\pend{}\selectlanguage{ngerman}\endnumbering\briefempfaengerindex{Dehmel, Ida@\textsc{Dehmel, Ida}!zzzSchnitzler, Arthur@\emph{von Arthur Schnitzler}!1920-02-251@{25. 2. 1920}|)be}\mylabel{L02337h}  \newcommand{\dateiname}{L02337}\newcommand{\titel}{Arthur Schnitzler an Ida Dehmel, 25. 2. 1920}\newcommand{\editorInnen}{Martin Anton Müller und Gerd-Hermann Susen}%% latex-leseansicht-abspann.tex
%% Abspann für die Leseansicht.
%% Der Schalter \ifkorrekturansicht ist bereits durch den Vorspann gesetzt.

%% latex-abspann.tex
%% Gemeinsamer Abspann für Korrekturansicht und Leseansicht.
%% Setzt den Schalter \ifkorrekturansicht voraus (gesetzt in den
%% einbindenden Dateien latex-korrekturansicht-abspann.tex bzw.
%% latex-leseansicht-abspann.tex).
%% ---------------------------------------------------------------

\normalsize

% Das esempio-Environment wird nur in der Leseansicht benötigt
\ifkorrekturansicht\else
\newenvironment{esempio}[3]%
{
    \vspace{1.5ex}
    \rlap{\underline{#1}}
    \par
    \setlength{\parindent}{0cm}
    \nopagebreak
    \leftskip=#2cm
    \rightskip=#3cm
}
{
    \par
}
\fi

\doendnotes{C}
\bigskip
\vfill

\clearpage

\footnotesize

\ifkorrekturansicht
  \lohead{\textsc{register}}
\fi

% theindex-Environment neu definieren ohne reledmac
\makeatletter
\renewenvironment{theindex}{%
  \ifkorrekturansicht
    \section*{\indexname}%
  \else
    \subsubsection*{Index der erwähnten Entitäten}%
  \fi
  \setlength{\parindent}{0pt}%
  \setlength{\parskip}{0pt plus 0.3pt}%
  \let\item\@idxitem
}{%
  \ifkorrekturansicht\clearpage\fi
}
\makeatother

\IfFileExists{\jobname-pw.ind}{\input{\jobname-pw.ind}}{}

% Quellenangabe nur in der Leseansicht
\ifkorrekturansicht\else
% Fallback-Definitionen, falls die .tex-Datei \titel etc. nicht gesetzt hat
\providecommand{\titel}{}
\providecommand{\editorInnen}{}
\providecommand{\dateiname}{\jobname}

\vspace{3cm}

\vfill

\footnotesize
\textsc{Quelle}: \titel. Herausgegeben von {\editorInnen}. In: \emph{Arthur Schnitzler: Briefwechsel mit Autorinnen und Autoren}.
 Digitale Edition, https://schnitzler-briefe.acdh.oeaw.ac.at/{\dateiname}.html (Stand \today)
\fi

\end{document}


