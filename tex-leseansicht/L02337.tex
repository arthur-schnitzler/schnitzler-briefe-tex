%% latex-korrekturansicht-vorspann.tex
%% Vorspann für die Korrekturansicht.
%% Lädt die gemeinsame Datei latex-vorspann.tex mit gesetztem Schalter.

\newif\ifkorrekturansicht
\korrekturansichttrue

\input{../tex-inputs/latex-vorspann}


\section[Arthur Schnitzler an Ida Dehmel, 25. 2. 1920]{L02337 Arthur Schnitzler an Ida Dehmel, 25. 2. 1920}
\nopagebreak\mylabel{L02337v}
\rehead{ }\normalsize\beginnumbering\briefempfaengerindex{Dehmel, Ida@\textsc{Dehmel, Ida}!zzzSchnitzler, Arthur@\emph{von Arthur Schnitzler}!1920-02-251@{25. 2. 1920}|(be}
\toendnotes[C]{\smallbreak\pagebreak[2]}\Standort{Hamburg, Staats- und Universitätsbibliothek, DA:Br:S:620.}
\physDesc{Brief, 1 Blatt, 2 Seiten, 984 Zeichen
\newline{}Handschrift: schwarze Tinte, lateinische Kurrent}\toendnotes[C]{\smallbreak}
\pstart
           \raggedleft{}{\pb}Wien\oindex{Wien@\textbf{Wien}, \emph{A.ADM2}|pw}, 25. Feber 1920\pend
           \vspace{0.5em}
\pstart
           Verehrte Frau, erst heute komm ich Ihnen sagen, wie tief der Tod
               Ihres Gatten\pwindex{Dehmel, Richard 18.11.1863 – 08.02.1920@\textsc{Dehmel, Richard} (18.11.1863 – 08.02.1920), \emph{Schriftsteller/Schriftstellerin, Schriftsteller/Schriftstellerin, Krimiautor/Krimiautorin}|pw}, dieses großen Dichters, dieses
               hohen Menschen mich erschüttert hat. Als die traurige Nachricht kam, war mir, als
               hätt ich erst vor kurzem persönlich von ihm Abschied genommen, nach einem tagelangen
               von mancherlei aus lebendigster Unterhaltung erfülltem Zusa{\geminationm}ensein: so nahe war er mir in seinem Kriegs-{\pb}Tagebuch\pwindex{Zwischen Volk und Menschheit@\emph{Zwischen Volk und Menschheit}|pwv}
               gewesen – ich hatte seine Stimme gehört, wie es mir so oft auch mit seinen Gedichten
               erging, – seinen Blick auf mir gefühlt; – denn in jedem Wort das er schrieb, in jedem
               das er sprach war seine ganze, seine wahrhaftige, seine große Seele. Und wie viele
               Jahre sind es nun schon her, daß ich ihn zum letzten Male gesehn!\pend
           
\pstart
           Meine Frau\pwindex{Schnitzler, Olga 17.01.1882 – 13.01.1970@\textsc{Schnitzler, Olga} (17.01.1882 – 13.01.1970), \emph{Schauspieler/Schauspielerin, Sänger/Sängerin}|pwv}, die ihn verehrt
               hat, gleich mir, schließt sich dem Ausdruck meiner innigsten Theilnahme aus vollem
               Herzen an. Wir denken Ihrer in schmerzlich-trostreicher Erinnerung schönerer Zeiten
               und mit den alten freundschaftlichen Gefühlen.\pend
           \pstart Ihr\spacefill\mbox{Arthur Schnitzler}\pend{}\selectlanguage{ngerman}\endnumbering\briefempfaengerindex{Dehmel, Ida@\textsc{Dehmel, Ida}!zzzSchnitzler, Arthur@\emph{von Arthur Schnitzler}!1920-02-251@{25. 2. 1920}|)be}\mylabel{L02337h}  \normalsize

\doendnotes{C}
\bigskip
\vfill

\clearpage

\footnotesize

\lohead{\textsc{register}}

% Definiere theindex-Environment komplett neu ohne reledmac
\makeatletter
\renewenvironment{theindex}{%
  \section*{\indexname}%
  \setlength{\parindent}{0pt}%
  \setlength{\parskip}{0pt plus 0.3pt}%
  \let\item\@idxitem
}{%
  \clearpage
}
\makeatother

\IfFileExists{\jobname-pw.ind}{\input{\jobname-pw.ind}}{}

\end{document}

      