%% latex-leseansicht-vorspann.tex
%% Vorspann für die Leseansicht.
%% Lädt die gemeinsame Datei latex-vorspann.tex mit nicht gesetztem Schalter.

\newif\ifkorrekturansicht
\korrekturansichtfalse

\input{../tex-inputs/latex-vorspann}


         
         \renewcommand{\erwaehntePersonen}{Personen: Ida Dehmel, Richard Dehmel, Olga Schnitzler}
         \renewcommand{\erwaehnteOrte}{Orte: Blankenese, Wien}
         \renewcommand{\erwaehnteWerke}{Werke: Zwischen Volk und Menschheit}
               \section[Arthur Schnitzler an Ida Dehmel, 25. 2. 1920]{ Arthur Schnitzler an Ida Dehmel, 25. 2. 1920}\nopagebreak\mylabel{v}\rehead{ }\begin{ledgroupsized}[t]{13cm}\normalsize\beginnumbering \toendnotes[C]{\smallbreak\pagebreak[2]} \Standort{Hamburg, Staats- und Universitätsbibliothek, DA:Br:S:620.}
\physDesc{Brief, 1 Blatt, 2 Seiten
\newline{}Handschrift: schwarze Tinte, lateinische Kurrent}\toendnotes[C]{\smallbreak}\pstart
           \raggedleft{}{\pb}Wien\oindex{Wien@\textbf{Wien}|pw}, 25. Feber 1920\pend
           \pstart
           Verehrte Frau, erst heute komm ich Ihnen sagen, wie tief der
                    Tod Ihres Gatten\pwindex{Dehmel, Richard 18.11.1863 – 08.02.1920@\textsc{Dehmel, Richard} (18.11.1863 – 08.02.1920), \emph{Schriftsteller}|pw}, dieses großen Dichters,
                    dieses hohen Menschen mich erschüttert hat. Als die traurige Nachricht kam, war
                    mir, als hätt ich erst vor kurzem persönlich von ihm Abschied genommen, nach
                    einem tagelangen von mancherlei aus lebendigster Unterhaltung erfülltem Zusa{\geminationm}ensein: so nahe war er mir in seinem Kriegs-{\pb}Tagebuch\pwindex{Dehmel, Richard 18.11.1863 – 08.02.1920@\textsc{Dehmel, Richard} (18.11.1863 – 08.02.1920), \emph{Schriftsteller}!Zwischen Volk und Menschheit1919@\strich\emph{Zwischen Volk und Menschheit} {[}1919{]}|pwv} gewesen – ich hatte seine Stimme gehört, wie es mir so oft
                    auch mit seinen Gedichten erging, – seinen Blick auf mir gefühlt; – denn in
                    jedem Wort das er schrieb, in jedem das er sprach war seine ganze, seine
                    wahrhaftige, seine große Seele. Und wie viele Jahre sind es nun schon her, daß
                    ich ihn zum letzten Male gesehn!\pend
           \pstart
           Meine Frau\pwindex{Schnitzler, Olga 17.01.1882 – 13.01.1970@\textsc{Schnitzler, Olga} (17.01.1882 – 13.01.1970), \emph{Schauspielerin, Sängerin}|pwv}, die ihn verehrt
                    hat, gleich mir, schließt sich dem Ausdruck meiner innigsten Theilnahme aus
                    vollem Herzen an. Wir denken Ihrer in schmerzlich-trostreicher Erinnerung
                    schönerer Zeiten und mit den alten freundschaftlichen Gefühlen.\pend
           \pstart Ihr\spacefill\mbox{Arthur Schnitzler}\pend{}
         
         \endnumbering\mylabel{h}\end{ledgroupsized}  \newcommand{\dateiname}{L02337}\newcommand{\titel}{Arthur Schnitzler an Ida Dehmel, 25. 2. 1920}\newcommand{\editorInnen}{ Martin Anton Müller und Gerd-Hermann Susen}%% latex-leseansicht-abspann.tex
%% Abspann für die Leseansicht.
%% Der Schalter \ifkorrekturansicht ist bereits durch den Vorspann gesetzt.

%% latex-abspann.tex
%% Gemeinsamer Abspann für Korrekturansicht und Leseansicht.
%% Setzt den Schalter \ifkorrekturansicht voraus (gesetzt in den
%% einbindenden Dateien latex-korrekturansicht-abspann.tex bzw.
%% latex-leseansicht-abspann.tex).
%% ---------------------------------------------------------------

\normalsize

% Das esempio-Environment wird nur in der Leseansicht benötigt
\ifkorrekturansicht\else
\newenvironment{esempio}[3]%
{
    \vspace{1.5ex}
    \rlap{\underline{#1}}
    \par
    \setlength{\parindent}{0cm}
    \nopagebreak
    \leftskip=#2cm
    \rightskip=#3cm
}
{
    \par
}
\fi

\doendnotes{C}
\bigskip
\vfill

\clearpage

\footnotesize

\ifkorrekturansicht
  \lohead{\textsc{register}}
\fi

% theindex-Environment neu definieren ohne reledmac
\makeatletter
\renewenvironment{theindex}{%
  \ifkorrekturansicht
    \section*{\indexname}%
  \else
    \subsubsection*{Index der erwähnten Entitäten}%
  \fi
  \setlength{\parindent}{0pt}%
  \setlength{\parskip}{0pt plus 0.3pt}%
  \let\item\@idxitem
}{%
  \ifkorrekturansicht\clearpage\fi
}
\makeatother

\IfFileExists{\jobname-pw.ind}{\input{\jobname-pw.ind}}{}

% Quellenangabe nur in der Leseansicht
\ifkorrekturansicht\else
% Fallback-Definitionen, falls die .tex-Datei \titel etc. nicht gesetzt hat
\providecommand{\titel}{}
\providecommand{\editorInnen}{}
\providecommand{\dateiname}{\jobname}

\vspace{3cm}

\vfill

\footnotesize
\textsc{Quelle}: \titel. Herausgegeben von {\editorInnen}. In: \emph{Arthur Schnitzler: Briefwechsel mit Autorinnen und Autoren}.
 Digitale Edition, https://schnitzler-briefe.acdh.oeaw.ac.at/{\dateiname}.html (Stand \today)
\fi

\end{document}


      