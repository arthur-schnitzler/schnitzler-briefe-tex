%% latex-korrekturansicht-vorspann.tex
%% Vorspann für die Korrekturansicht.
%% Lädt die gemeinsame Datei latex-vorspann.tex mit gesetztem Schalter.

\newif\ifkorrekturansicht
\korrekturansichttrue

\input{../tex-inputs/latex-vorspann}


\section[Therese Rie-Andro an Arthur Schnitzler, 30. 9. 1923]{L02576 Therese Rie-Andro an Arthur Schnitzler, 30. 9. 1923}
\nopagebreak\mylabel{L02576v}
\rehead{ }\normalsize\beginnumbering\briefempfaengerindex{Schnitzler, Arthur@\textsc{Schnitzler, Arthur}!zzzRie, Therese@\emph{von Therese Rie}!1923-09-301@{30. 9. 1923}|(be}
\toendnotes[C]{\smallbreak\pagebreak[2]}\Standort{CUL, Schnitzler, B 658.}
\physDesc{Brief, 1 Blatt, 1 Seite, 1291 Zeichen
\newline{}Schreibmaschine
\newline{}Handschrift: blaue Tinte, lateinische Kurrent (\noindent{}eine Unterstreichung, Grußformel und Unterschrift)
\newline{}Schnitzler: 1) mit Bleistift zwischen erstem und zweitem Absatz: »\textsc{Fliederbusch\pwindex{Fink und Fliederbusch. Komoedie in drei Akten@\emph{Fink und Fliederbusch. Komödie in drei Akten}|pw}}«  2) mit rotem Buntstift beschriftet: »\textsc{Rie-Andro (Fliederbusch\pwindex{Fink und Fliederbusch. Komoedie in drei Akten@\emph{Fink und Fliederbusch. Komödie in drei Akten}|pw})}« und fünf Unterstreichungen}\toendnotes[C]{\smallbreak}
\pstart
           \raggedleft{}{\pb}Wien\oindex{Wien@\textbf{Wien}, \emph{A.ADM2}|pw}, 30. 9. 23\pend
           
\pstart
           \raggedleft{}IV. Schönburgstr. 48\oindex{Schoenburgstrasse@\textbf{Schönburgstraße}, \emph{Straße (K.STR)}|pw}\pend
           
\pstart{}Verehrter Herr Doktor,\pend\vspace{0.5em}
\pstart
           Haben Sie sehr sehr herzlichen Dank! Ich habe mich einen ganzen Nachmittag meiner
               Lieblingsbeschäftigung hingeben können: zu lachen. Wenn Sie freilich auch das »ernste
               Lachen« mit dazu rechnen wollen, das einen überkommt, wenn man das Allzumenschliche
               blosgelegt sieht. Es ist ein sehr \uline{weises}{ }Stück\pwindex{Fink und Fliederbusch. Komoedie in drei Akten@\emph{Fink und Fliederbusch. Komödie in drei Akten}|pwv} und ich weiss jetzt
               genau, warum ich es damals so besonders liebte!\pend
           
\pstart
           Ich schicke Ihnen zugleich den versprochenen Rolland\pwindex{Rolland, Romain 29.01.1866 – 30.12.1944@\textsc{Rolland, Romain} (29.01.1866 – 30.12.1944), \emph{Schriftsteller/Schriftstellerin}|pw}\pwindex{Musikalische Reise ins Land der Vergangenheit@\emph{Musikalische Reise ins Land der Vergangenheit}|pwv}; Sie hätten ihn längst bekommen, aber ich wusste, dass Sie verreist waren.
               Hoffentlich interessiert er Sie – umsomehr, als Sie, wie Stefan Zweig\pwindex{Zweig, Stefan 28.11.1881 – 23.02.1942@\textsc{Zweig, Stefan} (28.11.1881 – 23.02.1942), \emph{Schriftsteller/Schriftstellerin}|pw} mir in Salzburg\oindex{Salzburg@\textbf{Salzburg}, \emph{A.ADM2}|pw} erzählte, mit Rolland\pwindex{Rolland, Romain 29.01.1866 – 30.12.1944@\textsc{Rolland, Romain} (29.01.1866 – 30.12.1944), \emph{Schriftsteller/Schriftstellerin}|pw} dort
                  \label{K_L02576-1v}\edtext{zusammen waren}{\lemma{\textnormal{\emph{zusammen waren}}}\Cendnote{\textnormal{Siehe A. S.: \emph{Tagebuch}, 3. 8. 1923.
               }}}\label{K_L02576-1}. Ein paar Aufsätze finde ich ja ein bischen langweilig, aber der Händel\pwindex{Haendel, Georg Friedrich 23.02.1685 – 14.04.1759@\textsc{Händel, Georg Friedrich} (23.02.1685 – 14.04.1759), \emph{Komponist/Komponistin}|pw}\pwindex{Musikalische Reise ins Land der Vergangenheit@\emph{Musikalische Reise ins Land der Vergangenheit}|pwv} ist ergreifend schön für meinen Geschmack. Auch Metastasio\pwindex{Metastasio, Pietro 1698-01-03 – 1782-04-12@\textsc{Metastasio, Pietro} (1698-01-03 – 1782-04-12), \emph{Schriftsteller/Schriftstellerin, Komponist/Komponistin, Librettist/Librettistin}|pw}\pwindex{Musikalische Reise ins Land der Vergangenheit@\emph{Musikalische Reise ins Land der Vergangenheit}|pwv} mit seinem ganz modernen Musikdramatiker-Empfinden hat mir sehr gefallen und
               der musikwütige Engländer\oindex{England@\textbf{England}, \emph{A.ADM1}|pw}, der Musik so sehr
               vergöttert und so ungern bezahlt, ist auch nicht schlecht.\pend
           
\pstart
           – – Sonderbar ist mirs immer, dass Rolland\pwindex{Rolland, Romain 29.01.1866 – 30.12.1944@\textsc{Rolland, Romain} (29.01.1866 – 30.12.1944), \emph{Schriftsteller/Schriftstellerin}|pw}
               sich um J. S. Bach\pwindex{Bach, Johann Sebastian 21.03.1685 – 28.07.1750@\textsc{Bach, Johann Sebastian} (21.03.1685 – 28.07.1750), \emph{Komponist/Komponistin}|pw} jedes Mal mit ein paar
               bewundernden Worten herumdrückt; aber ihm nie recht in die Nähe will. Vielleicht
               gibts da trotz allem doch nationale Schranken – oder er hat die Hmoll-Messe\pwindex{h-Moll-Messe@\emph{h-Moll-Messe}|pw} nie ordentlich gehö\textcolor{gray}{rt}\pend
           
\pstart
           Nochmals herzlichsten Dark und viele Grüsse1\pend
           
\pstart
           {[}hs.:{]} \textcolor{gray}{Ihre}{\\[\baselineskip]}\spacefill\mbox{Therese Rie.}\pend
           \leftskip=0em{}\selectlanguage{ngerman}\endnumbering\briefempfaengerindex{Schnitzler, Arthur@\textsc{Schnitzler, Arthur}!zzzRie, Therese@\emph{von Therese Rie}!1923-09-301@{30. 9. 1923}|)be}\mylabel{L02576h}  \normalsize

\doendnotes{C}
\bigskip
\vfill

\clearpage

\footnotesize

\lohead{\textsc{register}}

% Definiere theindex-Environment komplett neu ohne reledmac
\makeatletter
\renewenvironment{theindex}{%
  \section*{\indexname}%
  \setlength{\parindent}{0pt}%
  \setlength{\parskip}{0pt plus 0.3pt}%
  \let\item\@idxitem
}{%
  \clearpage
}
\makeatother

\IfFileExists{\jobname-pw.ind}{\input{\jobname-pw.ind}}{}

\end{document}

      