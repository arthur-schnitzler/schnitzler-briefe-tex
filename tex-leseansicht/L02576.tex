%% latex-leseansicht-vorspann.tex
%% Vorspann für die Leseansicht.
%% Lädt die gemeinsame Datei latex-vorspann.tex mit nicht gesetztem Schalter.

\newif\ifkorrekturansicht
\korrekturansichtfalse

\input{../tex-inputs/latex-vorspann}

\begin{center}
            \textcolor{red}{ENTWURF. ENTZIFFERUNG NOCH NICHT KORREKTURGELESEN}
                      \end{center}
            
               \section[Therese Rie-Andro an Arthur Schnitzler, 30. 9. 1923]{ Therese Rie-Andro an Arthur Schnitzler, 30. 9. 1923}\nopagebreak\mylabel{v}\rehead{ }\begin{ledgroupsized}[t]{13cm}\normalsize\beginnumbering\briefempfaengerindex{Schnitzler, Arthur@\textsc{Schnitzler, Arthur}!zzzRie, Therese@\emph{von Therese Rie}!1923-09-301@{30. 9. 1923}|(be} \toendnotes[C]{\smallbreak\pagebreak[2]} \Standort{CUL, Schnitzler, B 658.}
\physDesc{Brief, 1 Blatt, 1 Seite
\newline{}Schreibmaschine
\newline{}Handschrift: blaue Tinte, lateinische Kurrent (\noindent{}eine Unterstreichung, Grußformel und Unterschrift)
\newline{}Schnitzler: 1) mit Bleistift zwischen erstem
                                 und zweitem Absatz: »\textsc{Fliederbusch\pwindex{Schnitzler, Arthur 15.05.1862 – 21.10.1931@\textsc{Schnitzler, Arthur} (15.05.1862 – 21.10.1931), \emph{Schriftsteller, Mediziner}!Fink und Fliederbusch. Komoedie in drei Akten1917@\strich\emph{Fink und Fliederbusch. Komödie in drei Akten} {[}1917{]}|pw}}« 2) mit rotem Buntstift beschriftet: »\textsc{Rie-Andro (Fliederbusch\pwindex{Schnitzler, Arthur 15.05.1862 – 21.10.1931@\textsc{Schnitzler, Arthur} (15.05.1862 – 21.10.1931), \emph{Schriftsteller, Mediziner}!Fink und Fliederbusch. Komoedie in drei Akten1917@\strich\emph{Fink und Fliederbusch. Komödie in drei Akten} {[}1917{]}|pw})}« und fünf Unterstreichungen}\toendnotes[C]{\smallbreak}\pstart
           \raggedleft{}{\pb}Wien\oindex{Wien@\textbf{Wien}|pw}, 30. 9. 23\pend
           \pstart
           \raggedleft{}IV. Schönburgstr. 48\oindex{Schoenburgstrasse@\textbf{Schönburgstraße}|pw}\pend
           \pstart{}Verehrter Herr Doktor,\pend\pstart
           Haben Sie sehr sehr herzlichen Dank! Ich habe mich einen ganzen Nachmittag meiner
               Lieblingsbeschäftigung hingeben können: zu lachen. Wenn Sie freilich auch das »ernste
               Lachen« mit dazu rechnen wollen, das einen überkommt, wenn man das Allzumenschliche
               blosgelegt sieht. Es ist ein sehr \uline{weises}{ }Stück\pwindex{Schnitzler, Arthur 15.05.1862 – 21.10.1931@\textsc{Schnitzler, Arthur} (15.05.1862 – 21.10.1931), \emph{Schriftsteller, Mediziner}!Fink und Fliederbusch. Komoedie in drei Akten1917@\strich\emph{Fink und Fliederbusch. Komödie in drei Akten} {[}1917{]}|pwv} und ich weiss jetzt genau, warum ich es damals so besonders
               liebte!\pend
           \pstart
           Ich schicke Ihnen zugleich den versprochenen Rolland\pwindex{Rolland, Romain 29.01.1866 – 30.12.1944@\textsc{Rolland, Romain} (29.01.1866 – 30.12.1944), \emph{Schriftsteller}|pw}\pwindex{Rolland, Romain 29.01.1866 – 30.12.1944@\textsc{Rolland, Romain} (29.01.1866 – 30.12.1944), \emph{Schriftsteller}!Musikalische Reise ins Land der Vergangenheit1919@\strich\emph{Musikalische Reise ins Land der Vergangenheit} {[}1919{]}|pwv}; Sie hätten ihn längst bekommen, aber ich wusste, dass Sie verreist waren.
               Hoffentlich interessiert er Sie – umsomehr, als Sie, wie Stefan Zweig\pwindex{Zweig, Stefan 28.11.1881 – 23.02.1942@\textsc{Zweig, Stefan} (28.11.1881 – 23.02.1942), \emph{Schriftsteller}|pw} mir in Salzburg\oindex{Salzburg@\textbf{Salzburg}|pw}
               erzählte, mit Rolland\pwindex{Rolland, Romain 29.01.1866 – 30.12.1944@\textsc{Rolland, Romain} (29.01.1866 – 30.12.1944), \emph{Schriftsteller}|pw} dort \label{K_L02576-1v}\edtext{zusammen waren}{\lemma{\textnormal{\emph{zusammen waren}}}\Cendnote{\textnormal{siehe A. S.: \emph{Tagebuch}, 3. 8. 1923}}}\label{K_L02576-1h}. Ein paar Aufsätze finde ich ja ein bischen langweilig, aber der Händel\pwindex{Haendel, Georg Friedrich 23.02.1685 – 14.04.1759@\textsc{Händel, Georg Friedrich} (23.02.1685 – 14.04.1759), \emph{Komponist}|pw}\pwindex{Rolland, Romain 29.01.1866 – 30.12.1944@\textsc{Rolland, Romain} (29.01.1866 – 30.12.1944), \emph{Schriftsteller}!Musikalische Reise ins Land der Vergangenheit1919@\strich\emph{Musikalische Reise ins Land der Vergangenheit} {[}1919{]}|pwv} ist ergreifend schön für meinen Geschmack. Auch Metastasio\pwindex{Metastasio, Pietro 1698-01-03 – 1782-04-12@\textsc{Metastasio, Pietro} (1698-01-03 – 1782-04-12), \emph{Schriftsteller, Komponist, Librettist}|pw}\pwindex{Rolland, Romain 29.01.1866 – 30.12.1944@\textsc{Rolland, Romain} (29.01.1866 – 30.12.1944), \emph{Schriftsteller}!Musikalische Reise ins Land der Vergangenheit1919@\strich\emph{Musikalische Reise ins Land der Vergangenheit} {[}1919{]}|pwv} mit seinem ganz modernen Musikdramatiker-Empfinden hat mir sehr gefallen und
               der musikwütige Engländer\oindex{England@\textbf{England}|pw}, der Musik so sehr
               vergöttert und so ungern bezahlt, ist auch nicht schlecht.\pend
           \pstart
           – – Sonderbar ist mirs immer, dass Rolland\pwindex{Rolland, Romain 29.01.1866 – 30.12.1944@\textsc{Rolland, Romain} (29.01.1866 – 30.12.1944), \emph{Schriftsteller}|pw} sich
               um J. S. Bach\pwindex{Bach, Johann Sebastian 21.03.1685 – 28.07.1750@\textsc{Bach, Johann Sebastian} (21.03.1685 – 28.07.1750), \emph{Komponist}|pw} jedes Mal mit ein paar bewundernden
               Worten herumdrückt; aber ihm nie recht in die Nähe will. Vielleicht gibts da trotz
               allem doch nationale Schranken – oder er hat die Hmoll-Messe\pwindex{Bach, Johann Sebastian 21.03.1685 – 28.07.1750@\textsc{Bach, Johann Sebastian} (21.03.1685 – 28.07.1750), \emph{Komponist}!h-Moll-Messe1733@\strich\emph{h-Moll-Messe} {[}1733{]}|pw} nie ordentlich gehö\textcolor{gray}{rt}\pend
           \pstart
           Nochmals herzlichsten Dark und viele Grüsse1\pend
           \pstart
           {[}hs.:{]} \textcolor{gray}{Ihre}{\\[\baselineskip]}\spacefill\mbox{Therese Rie.}\pend
           \leftskip=0em{}\endnumbering\briefempfaengerindex{Schnitzler, Arthur@\textsc{Schnitzler, Arthur}!zzzRie, Therese@\emph{von Therese Rie}!1923-09-301@{30. 9. 1923}|)be}\mylabel{h}\end{ledgroupsized}  \newcommand{\dateiname}{L02576}\newcommand{\titel}{Therese Rie-Andro an Arthur Schnitzler, 30. 9. 1923}\newcommand{\editorInnen}{Martin Anton Müller und Gerd-Hermann Susen}%% latex-leseansicht-abspann.tex
%% Abspann für die Leseansicht.
%% Der Schalter \ifkorrekturansicht ist bereits durch den Vorspann gesetzt.

%% latex-abspann.tex
%% Gemeinsamer Abspann für Korrekturansicht und Leseansicht.
%% Setzt den Schalter \ifkorrekturansicht voraus (gesetzt in den
%% einbindenden Dateien latex-korrekturansicht-abspann.tex bzw.
%% latex-leseansicht-abspann.tex).
%% ---------------------------------------------------------------

\normalsize

% Das esempio-Environment wird nur in der Leseansicht benötigt
\ifkorrekturansicht\else
\newenvironment{esempio}[3]%
{
    \vspace{1.5ex}
    \rlap{\underline{#1}}
    \par
    \setlength{\parindent}{0cm}
    \nopagebreak
    \leftskip=#2cm
    \rightskip=#3cm
}
{
    \par
}
\fi

\doendnotes{C}
\bigskip
\vfill

\clearpage

\footnotesize

\ifkorrekturansicht
  \lohead{\textsc{register}}
\fi

% theindex-Environment neu definieren ohne reledmac
\makeatletter
\renewenvironment{theindex}{%
  \ifkorrekturansicht
    \section*{\indexname}%
  \else
    \subsubsection*{Index der erwähnten Entitäten}%
  \fi
  \setlength{\parindent}{0pt}%
  \setlength{\parskip}{0pt plus 0.3pt}%
  \let\item\@idxitem
}{%
  \ifkorrekturansicht\clearpage\fi
}
\makeatother

\IfFileExists{\jobname-pw.ind}{\input{\jobname-pw.ind}}{}

% Quellenangabe nur in der Leseansicht
\ifkorrekturansicht\else
% Fallback-Definitionen, falls die .tex-Datei \titel etc. nicht gesetzt hat
\providecommand{\titel}{}
\providecommand{\editorInnen}{}
\providecommand{\dateiname}{\jobname}

\vspace{3cm}

\vfill

\footnotesize
\textsc{Quelle}: \titel. Herausgegeben von {\editorInnen}. In: \emph{Arthur Schnitzler: Briefwechsel mit Autorinnen und Autoren}.
 Digitale Edition, https://schnitzler-briefe.acdh.oeaw.ac.at/{\dateiname}.html (Stand \today)
\fi

\end{document}


      