%% latex-leseansicht-vorspann.tex
%% Vorspann für die Leseansicht.
%% Lädt die gemeinsame Datei latex-vorspann.tex mit nicht gesetztem Schalter.

\newif\ifkorrekturansicht
\korrekturansichtfalse

\input{../tex-inputs/latex-vorspann}


               \section[Arthur Schnitzler an Hugo von Hofmannsthal, 11. 8. 1893]{ Arthur Schnitzler an Hugo von Hofmannsthal, 11. 8. 1893}\nopagebreak\mylabel{v}\rehead{ }\begin{ledgroupsized}[t]{13cm}\normalsize\beginnumbering\briefempfaengerindex{Hofmannsthal, Hugo von@\textsc{Hofmannsthal, Hugo von}!zzzSchnitzler, Arthur@\emph{von Arthur Schnitzler}!1893-08-112@{11. 8. 1893}|(be} \toendnotes[C]{\smallbreak\pagebreak[2]} \Standort{FDH, Hs-30885,38.}
\physDesc{Brief, 2 Blätter (Briefpapier mit Trauerrand), 5 Seiten
\newline{}Handschrift: schwarze Tinte, deutsche Kurrent\newline{}Ordnung: von Schnitzler mutmaßlich bei der Durchsicht der Korrespondenz
                                    1929 mit Bleistift datiert: »11. 8. 93« }\buchAbdrucke{\weitereDrucke{Hugo von Hofmannsthal, Arthur Schnitzler: \emph{Briefwechsel}. Hg. Therese Nickl und Heinrich Schnitzler. Frankfurt am Main: \emph{S. Fischer} 1964, S. 43–44.} }\toendnotes[C]{\smallbreak}\pstart{}{\pb}Lieber Hugo,\pend\pstart
           Ihr Feu{[}i{]}lleton\pwindex{Gabriele DAnnunzio09. 08. 1893@\emph{Gabriele d’Annunzio} {[}09. 08. 1893{]}|pwv} über \textsc{Annunzio}\pwindex{DAnnunzio, Gabriele 12.03.1863 – 01.03.1938@\textsc{D’Annunzio, Gabriele} (12.03.1863 – 01.03.1938), \emph{Schriftsteller}|pw} hab ich mit großer Freude geleſen; es iſt eins Ihrer ſchönſten, mit weiten
               Ausblicken. – Iſt von dem Mann was ins Deutſche überſetzt? –\pend
           \pstart
           – Denken Sie, mir iſt man endlich draufgeko{\geminationm}en, daſs ich
                  \strikeout{i}auf die ſexuellen Inſtincte der Menge ſpeculire und
                  {\pb}meine »cyniſchen\pwindex{Literatur. »Bunte Reihe.« Ein Geschichtenbuch von Moritz Goldschmidt. »Anatol« von Arthur Schnitzler3.8.1893 – 3.8.1893@\emph{Literatur. »Bunte Reihe.« Ein Geschichtenbuch von Moritz Goldschmidt. »Anatol« von Arthur Schnitzler} {[}3.8.1893 – 3.8.1893{]}|pwv}«, »plumpen\pwindex{Literatur. »Bunte Reihe.« Ein Geschichtenbuch von Moritz Goldschmidt. »Anatol« von Arthur Schnitzler3.8.1893 – 3.8.1893@\emph{Literatur. »Bunte Reihe.« Ein Geschichtenbuch von Moritz Goldschmidt. »Anatol« von Arthur Schnitzler} {[}3.8.1893 – 3.8.1893{]}|pwv}«
               Sachen mit verletzender Abſichtlichkeit ſchreibe – (offenbar um mittelſt meiner
               Trivialität viel Geld zu machen.) – Der Ruhm dieſer Entdeckung gebührt der Wiener Abendpoſt\orgindex{Wiener Abendpost@Wiener Abendpost|pw}, welche im übrigen zugleich
               Geſchmack genug hat, die Leichtbeſchwingtheit Ihrer Verſe\pwindex{Einleitung1892@\emph{Einleitung} {[}1892{]}|pwv} zu loben. (Referent Bruno Walden\pwindex{Galliny, Florentine 24.06.1845 – 19.07.1913@\textsc{Galliny, Florentine} (24.06.1845 – 19.07.1913), \emph{Schriftstellerin, Journalistin}|pw}.) –\pend
           \pstart
           Meine Abſicht geht vorläufig dahin {\pb}Ende nächſter Woche
               ins Puſterthal\oindex{Pustertal@\textbf{Pustertal}|pw} zu reiſen, und vielleicht von dort
               per \textsc{Bic.} nach Wien\oindex{Wien@\textbf{Wien}|pw} zurück.
                  (\textsc{Salten}\pwindex{Salten, Felix 06.09.1869 – 08.10.1945@\textsc{Salten, Felix} (06.09.1869 – 08.10.1945), \emph{Schriftsteller, Journalist}|pw} iſt bereits unten.) – \textsc{Paul Goldma{\geminationn}}\pwindex{Goldmann, Paul 31.01.1865 – 25.09.1935@\textsc{Goldmann, Paul} (31.01.1865 – 25.09.1935), \emph{Schriftsteller, Journalist}|pw} will im September nach Salzburg\oindex{Salzburg@\textbf{Salzburg}|pw}
                  ko{\geminationm}en; vielleicht läßt ſich eine Zuſa{\geminationm}enkunft Ende Auguſt arrangiren?\pend
           \pstart
           Wie ſind Ihre Pläne? Schreiben Sie doch was darüber. Arbeiten Sie was? Meine kleine
                  Novelle\pwindex{Schnitzler, Arthur 15.05.1862 – 21.10.1931@\textsc{Schnitzler, Arthur} (15.05.1862 – 21.10.1931), \emph{Schriftsteller, Mediziner}!kleine Komoedie01.08.1895 – 01.08.1895@\strich\emph{Die kleine Komödie} {[}01.08.1895 – 01.08.1895{]}|pw} iſt bis auf wenige Zeilen fertig. {\pb}Das hab ich Ihnen ſchon geſchrieben. – Jetzt ſchreib ich
               ab und zu ein paar Verſe an dem »allegoriſchen« Gedicht\pwindex{Schnitzler, Arthur 15.05.1862 – 21.10.1931@\textsc{Schnitzler, Arthur} (15.05.1862 – 21.10.1931), \emph{Schriftsteller, Mediziner}!Artifex1893@\strich\emph{Artifex} {[}1893{]}|pw}; bedauere aber ſehr, nicht die ausreichende Befähigung dazu zu
               haben. –\pend
           \pstart
           Den Mut zu was größerem, das wird Sie nach alledem nicht wundern, hab ich noch nicht
               erlangt. – Unter vier Augen: das Volkstheater\oindex{Volkstheater@\textbf{Volkstheater}|pw} beginnt
               mit mir \introOben{}(wegen »Märchen\pwindex{Schnitzler, Arthur 15.05.1862 – 21.10.1931@\textsc{Schnitzler, Arthur} (15.05.1862 – 21.10.1931), \emph{Schriftsteller, Mediziner}!Maerchen. Schauspiel in drei Aufzuegen1891 – 1891@\strich\emph{Das Märchen. Schauspiel in drei Aufzügen} {[}1891 – 1891{]}|pw}«)\introOben{}
               zu unterhandeln; ich ſage Ihnen – Zuſtände!! – Weiteres darüber mündlich.\pend
           \pstart
           {\pb}– Wie gehts dem aegyptiſchen\oindex{Aegypten@\textbf{Ägypten}|pw} unanſtändigen Stück\pwindex{Hofmannsthal, Hugo von 01.02.1874 – 15.07.1929@\textsc{Hofmannsthal, Hugo von} (01.02.1874 – 15.07.1929), \emph{Schriftsteller}!Alexanderzug1936@\strich\emph{Alexanderzug} {[}1936{]}|pwv}? – Wenn es \uline{nur}{ }aegyptiſch\oindex{Aegypten@\textbf{Ägypten}|pw} wäre, läge es der Allgemeinheit zu fern!
               – Der Tod \textsc{Kafka\pwindex{Kafka, Eduard Michael 11.03.1869 – 06.08.1893@\textsc{Kafka, Eduard Michael} (11.03.1869 – 06.08.1893), \emph{Redakteur}|pw}}’s iſt Ihnen wohl bekannt worden? –\pend
           \pstart
           – Hören Sie was von \textsc{Fels\pwindex{Fels, Friedrich Michael *~1864@\textsc{Fels, Friedrich Michael} (*~1864), \emph{Journalist}|pw}}? – Schreibt Ihnen Richard\pwindex{Beer-Hofmann, Richard 11.07.1866 – 26.09.1945@\textsc{Beer-Hofmann, Richard} (11.07.1866 – 26.09.1945), \emph{Schriftsteller}|pw}? –\pend
           \pstart
           Sind Sie vergnügt? –\pend
           \pstart
           Herzlich der Ihre{\\[\baselineskip]}\spacefill\mbox{Arthur}\pend
           \leftskip=0em{}\pstart
           Wien\oindex{Wien@\textbf{Wien}|pw}, 11. 8. 93\pend
           \pstart
           \label{T_L00251_1v}\edtext{\uline{Sie müssen \textsc{Bicycle} fahren
                     lernen!}}{\lemma{\textnormal{\emph{Sie … lernen!}}}\Cendnote{\textnormal{quer am linken Rand}}}\label{T_L00251_1h}\pend
           \endnumbering\briefempfaengerindex{Hofmannsthal, Hugo von@\textsc{Hofmannsthal, Hugo von}!zzzSchnitzler, Arthur@\emph{von Arthur Schnitzler}!1893-08-112@{11. 8. 1893}|)be}\mylabel{h}\end{ledgroupsized}  \newcommand{\dateiname}{L00251}\newcommand{\titel}{Arthur Schnitzler an Hugo von Hofmannsthal, 11. 8. 1893}\newcommand{\editorInnen}{Martin Anton Müller und Gerd-Hermann Susen}
            \footnotesize
\begin{ledgroupsized}[t]{11.5cm}
\doendnotes{C}
\end{ledgroupsized}
         %% latex-leseansicht-abspann.tex
%% Abspann für die Leseansicht.
%% Der Schalter \ifkorrekturansicht ist bereits durch den Vorspann gesetzt.

%% latex-abspann.tex
%% Gemeinsamer Abspann für Korrekturansicht und Leseansicht.
%% Setzt den Schalter \ifkorrekturansicht voraus (gesetzt in den
%% einbindenden Dateien latex-korrekturansicht-abspann.tex bzw.
%% latex-leseansicht-abspann.tex).
%% ---------------------------------------------------------------

\normalsize

% Das esempio-Environment wird nur in der Leseansicht benötigt
\ifkorrekturansicht\else
\newenvironment{esempio}[3]%
{
    \vspace{1.5ex}
    \rlap{\underline{#1}}
    \par
    \setlength{\parindent}{0cm}
    \nopagebreak
    \leftskip=#2cm
    \rightskip=#3cm
}
{
    \par
}
\fi

\doendnotes{C}
\bigskip
\vfill

\clearpage

\footnotesize

\ifkorrekturansicht
  \lohead{\textsc{register}}
\fi

% theindex-Environment neu definieren ohne reledmac
\makeatletter
\renewenvironment{theindex}{%
  \ifkorrekturansicht
    \section*{\indexname}%
  \else
    \subsubsection*{Index der erwähnten Entitäten}%
  \fi
  \setlength{\parindent}{0pt}%
  \setlength{\parskip}{0pt plus 0.3pt}%
  \let\item\@idxitem
}{%
  \ifkorrekturansicht\clearpage\fi
}
\makeatother

\IfFileExists{\jobname-pw.ind}{\input{\jobname-pw.ind}}{}

% Quellenangabe nur in der Leseansicht
\ifkorrekturansicht\else
% Fallback-Definitionen, falls die .tex-Datei \titel etc. nicht gesetzt hat
\providecommand{\titel}{}
\providecommand{\editorInnen}{}
\providecommand{\dateiname}{\jobname}

\vspace{3cm}

\vfill

\footnotesize
\textsc{Quelle}: \titel. Herausgegeben von {\editorInnen}. In: \emph{Arthur Schnitzler: Briefwechsel mit Autorinnen und Autoren}.
 Digitale Edition, https://schnitzler-briefe.acdh.oeaw.ac.at/{\dateiname}.html (Stand \today)
\fi

\end{document}


      