%% latex-leseansicht-vorspann.tex
%% Vorspann für die Leseansicht.
%% Lädt die gemeinsame Datei latex-vorspann.tex mit nicht gesetztem Schalter.

\newif\ifkorrekturansicht
\korrekturansichtfalse

\input{../tex-inputs/latex-vorspann}


\section[Arthur Schnitzler an Hugo von Hofmannsthal, 11. 8. 1893]{L00251 Arthur Schnitzler an Hugo von Hofmannsthal, 11. 8. 1893}
\nopagebreak\mylabel{L00251v}
\rehead{ }\normalsize\beginnumbering\briefempfaengerindex{Hofmannsthal, Hugo von@\textsc{Hofmannsthal, Hugo von}!zzzSchnitzler, Arthur@\emph{von Arthur Schnitzler}!1893-08-112@{11. 8. 1893}|(be}
\toendnotes[C]{\smallbreak\pagebreak[2]}
\correspDesc{Versand  durch Arthur Schnitzler am 11. 8. 1893 in Wien
\newline{}Erhalt  durch Hugo von Hofmannsthal im Zeitraum [11. 8. 1893
                  – 15. 8. 1893?] in Wien}\toendnotes[C]{\smallbreak}
\Standort{FDH, Hs-30885,38.}
\physDesc{Brief, 2 Blätter, 5 Seiten, 1698 Zeichen (Briefpapier mit Trauerrand)
\newline{}Handschrift: schwarze Tinte, deutsche Kurrent
\newline{}Ordnung: mit Bleistift von Schnitzler mutmaßlich bei der Durchsicht der Korrespondenz
                                    1929 datiert: »11. 8. 93« }
\buchAbdrucke{\weitereDrucke{Hugo von Hofmannsthal, Arthur Schnitzler: \emph{Briefwechsel}. Herausgegeben von Therese Nickl und Heinrich Schnitzler. Frankfurt am Main: \emph{S. Fischer} 1964, S. 43–44.} }\toendnotes[C]{\smallbreak}
\pstart{}{\pb}Lieber Hugo,\pend\vspace{0.5em}
\pstart
           Ihr Feu{[}i{]}lleton\pwindex{Hofmannsthal, Hugo von 1.\,2.\,1874 Wien – 15.\,7.\,1929 Rodaun@\textsc{Hofmannsthal, Hugo von} (1.\,2.\,1874 Wien – 15.\,7.\,1929 Rodaun), \emph{Schriftsteller}!Gabriele d’Annunzio@\strich\emph{Gabriele d’Annunzio}|pwv} über \textsc{Annunzio}\pwindex{D’Annunzio, Gabriele 12.\,3.\,1863 Pescara – 1.\,3.\,1938 Cargnacco@\textsc{D’Annunzio, Gabriele} (12.\,3.\,1863 Pescara – 1.\,3.\,1938 Cargnacco), \emph{Schriftsteller}|pw} hab ich mit großer Freude geleſen; es iſt eins Ihrer{ }ſchönſten, mit weiten
               Ausblicken. – Iſt von dem Mann was ins Deutſche überſetzt? –\pend
           
\pstart
           – Denken Sie, mir iſt man endlich draufgeko{\geminationm}en, daſs ich
                  \strikeout{i}auf die{ }ſexuellen Inſtincte der Menge{ }ſpeculire und
                  {\pb}meine »cyniſchen\pwindex{Galliny, Florentine 24.\,6.\,1845 Wien – 19.\,7.\,1913 ebd.@\textsc{Galliny, Florentine} (24.\,6.\,1845 Wien – 19.\,7.\,1913 ebd.), \emph{Schriftstellerin, Journalistin}!Literatur. »Bunte Reihe.« Ein Geschichtenbuch von Moritz Goldschmidt. »Anatol« von Arthur Schnitzler@\strich\emph{Literatur. »Bunte Reihe.« Ein Geschichtenbuch von Moritz Goldschmidt. »Anatol« von Arthur Schnitzler}|pwv}«, »plumpen\pwindex{Galliny, Florentine 24.\,6.\,1845 Wien – 19.\,7.\,1913 ebd.@\textsc{Galliny, Florentine} (24.\,6.\,1845 Wien – 19.\,7.\,1913 ebd.), \emph{Schriftstellerin, Journalistin}!Literatur. »Bunte Reihe.« Ein Geschichtenbuch von Moritz Goldschmidt. »Anatol« von Arthur Schnitzler@\strich\emph{Literatur. »Bunte Reihe.« Ein Geschichtenbuch von Moritz Goldschmidt. »Anatol« von Arthur Schnitzler}|pwv}« Sachen mit verletzender Abſichtlichkeit{ }ſchreibe –
               (offenbar um mittelſt meiner Trivialität viel Geld zu machen.) – Der Ruhm dieſer
               Entdeckung gebührt der Wiener Abendpoſt\orgindex{Wiener Abendpost@Wiener Abendpost|pw}, welche
               im übrigen zugleich Geſchmack genug hat, die Leichtbeſchwingtheit Ihrer Verſe\pwindex{Hofmannsthal, Hugo von 1.\,2.\,1874 Wien – 15.\,7.\,1929 Rodaun@\textsc{Hofmannsthal, Hugo von} (1.\,2.\,1874 Wien – 15.\,7.\,1929 Rodaun), \emph{Schriftsteller}!Prolog [zum Anatol]@\strich\emph{Prolog [zum Anatol]}|pwv} zu loben. (Referent Bruno Walden\pwindex{Galliny, Florentine 24.\,6.\,1845 Wien – 19.\,7.\,1913 ebd.@\textsc{Galliny, Florentine} (24.\,6.\,1845 Wien – 19.\,7.\,1913 ebd.), \emph{Schriftstellerin, Journalistin}|pw}.) –\pend
           
\pstart
           Meine Abſicht geht vorläufig dahin {\pb}Ende nächſter Woche
               ins Puſterthal\oindex{Pustertal@\textbf{Pustertal}, \emph{Tal}|pw} zu reiſen, und vielleicht von dort
               per \textsc{Bic.} nach Wien\oindex{Wien@\textbf{Wien}, \emph{Verwaltungsgebiet}|pw}
               zurück. (\textsc{Salten}\pwindex{Salten, Felix 6.\,9.\,1869 Budapest – 8.\,10.\,1945 Zürich@\textsc{Salten, Felix} (6.\,9.\,1869 Budapest – 8.\,10.\,1945 Zürich), \emph{Schriftsteller, Journalist, Chefredakteur}|pw} iſt bereits unten.) – \textsc{Paul Goldma{\geminationn}}\pwindex{Goldmann, Paul 31.\,1.\,1865 Breslau – 25.\,9.\,1935 Wien@\textsc{Goldmann, Paul} (31.\,1.\,1865 Breslau – 25.\,9.\,1935 Wien), \emph{Schriftsteller, Journalist}|pw} will im September nach Salzburg\oindex{Salzburg@\textbf{Salzburg}, \emph{Verwaltungsgebiet}|pw}
                  ko{\geminationm}en; vielleicht läßt{ }ſich eine Zuſa{\geminationm}enkunft Ende Auguſt arrangiren?\pend
           
\pstart
           Wie{ }ſind Ihre Pläne? Schreiben Sie doch was darüber. Arbeiten Sie was? Meine kleine
                  Novelle\pwindex{Schnitzler, Arthur 15.\,5.\,1862 Wien – 21.\,10.\,1931 ebd.@\textsc{Schnitzler, Arthur} (15.\,5.\,1862 Wien – 21.\,10.\,1931 ebd.), \emph{Schriftsteller, Mediziner}!kleine Komödie@\strich\emph{Die kleine Komödie}|pw} iſt bis auf wenige Zeilen fertig. {\pb}Das hab ich Ihnen{ }ſchon geſchrieben. – Jetzt{ }ſchreib ich
               ab und zu ein paar Verſe an dem »allegoriſchen« Gedicht\pwindex{Schnitzler, Arthur 15.\,5.\,1862 Wien – 21.\,10.\,1931 ebd.@\textsc{Schnitzler, Arthur} (15.\,5.\,1862 Wien – 21.\,10.\,1931 ebd.), \emph{Schriftsteller, Mediziner}!Artifex@\strich\emph{Artifex}|pw}; bedauere aber{ }ſehr, nicht die ausreichende Befähigung dazu zu
               haben. –\pend
           
\pstart
           Den Mut zu was größerem, das wird Sie nach alledem nicht wundern, hab ich noch nicht
               erlangt. – Unter vier Augen: das Volkstheater\oindex{Wien@\textbf{Wien}!VII., Neubau@\textbf{VII., Neubau}!Volkstheater@\textbf{Volkstheater}, \emph{Theater}|pw}
               beginnt mit mir \introOben{}(wegen »Märchen\pwindex{Schnitzler, Arthur 15.\,5.\,1862 Wien – 21.\,10.\,1931 ebd.@\textsc{Schnitzler, Arthur} (15.\,5.\,1862 Wien – 21.\,10.\,1931 ebd.), \emph{Schriftsteller, Mediziner}!Märchen. Schauspiel in drei Aufzügen@\strich\emph{Das Märchen. Schauspiel in drei Aufzügen}|pw}«)\introOben{} zu unterhandeln; ich{ }ſage Ihnen – Zuſtände!! – Weiteres
               darüber mündlich.\pend
           
\pstart
           {\pb}– Wie gehts dem aegyptiſchen\oindex{Ägypten@\textbf{Ägypten}|pw} unanſtändigen Stück\pwindex{Hofmannsthal, Hugo von 1.\,2.\,1874 Wien – 15.\,7.\,1929 Rodaun@\textsc{Hofmannsthal, Hugo von} (1.\,2.\,1874 Wien – 15.\,7.\,1929 Rodaun), \emph{Schriftsteller}!Alexanderzug@\strich\emph{Alexanderzug}|pwv}? – Wenn es \uline{nur}{ }aegyptiſch\oindex{Ägypten@\textbf{Ägypten}|pw} wäre, läge es der Allgemeinheit zu
               fern! – Der Tod \textsc{Kafka\pwindex{Kafka, Eduard Michael 11.\,3.\,1869 Wien – 6.\,8.\,1893 Brünn@\textsc{Kafka, Eduard Michael} (11.\,3.\,1869 Wien – 6.\,8.\,1893 Brünn), \emph{Redakteur}|pw}}’s iſt Ihnen wohl bekannt worden? –\pend
           
\pstart
           – Hören Sie was von \textsc{Fels\pwindex{Fels, Friedrich Michael *~1864 Bad Dürkheim@\textsc{Fels, Friedrich Michael} (*~1864 Bad Dürkheim), \emph{Journalist}|pw}}? – Schreibt Ihnen Richard\pwindex{Beer-Hofmann, Richard 11.\,7.\,1866 Wien – 26.\,9.\,1945 New York City@\textsc{Beer-Hofmann, Richard} (11.\,7.\,1866 Wien – 26.\,9.\,1945 New York City), \emph{Schriftsteller}|pw}? –\pend
           
\pstart
           Sind Sie vergnügt? –\pend
           
\pstart
           Herzlich der Ihre{\\[\baselineskip]}\spacefill\mbox{Arthur}\pend
           \leftskip=0em{}
\pstart
           Wien\oindex{Wien@\textbf{Wien}, \emph{Verwaltungsgebiet}|pw}, 11. 8. 93\pend
           
\pstart
           \label{T_L00251-1v}\edtext{\uline{Sie müssen \textsc{Bicycle} fahren
                     lernen!}}{\lemma{\textnormal{\emph{Sie … lernen!}}}\Cendnote{\textnormal{quer am linken Rand}}}\label{T_L00251-1}\pend
           \selectlanguage{ngerman}\endnumbering\briefempfaengerindex{Hofmannsthal, Hugo von@\textsc{Hofmannsthal, Hugo von}!zzzSchnitzler, Arthur@\emph{von Arthur Schnitzler}!1893-08-112@{11. 8. 1893}|)be}\mylabel{L00251h}  \newcommand{\dateiname}{L00251}\newcommand{\titel}{Arthur Schnitzler an Hugo von Hofmannsthal, 11. 8. 1893}\newcommand{\editorInnen}{Martin Anton Müller und Gerd-Hermann Susen}%% latex-leseansicht-abspann.tex
%% Abspann für die Leseansicht.
%% Der Schalter \ifkorrekturansicht ist bereits durch den Vorspann gesetzt.

%% latex-abspann.tex
%% Gemeinsamer Abspann für Korrekturansicht und Leseansicht.
%% Setzt den Schalter \ifkorrekturansicht voraus (gesetzt in den
%% einbindenden Dateien latex-korrekturansicht-abspann.tex bzw.
%% latex-leseansicht-abspann.tex).
%% ---------------------------------------------------------------

\normalsize

% Das esempio-Environment wird nur in der Leseansicht benötigt
\ifkorrekturansicht\else
\newenvironment{esempio}[3]%
{
    \vspace{1.5ex}
    \rlap{\underline{#1}}
    \par
    \setlength{\parindent}{0cm}
    \nopagebreak
    \leftskip=#2cm
    \rightskip=#3cm
}
{
    \par
}
\fi

\doendnotes{C}
\bigskip
\vfill

\clearpage

\footnotesize

\ifkorrekturansicht
  \lohead{\textsc{register}}
\fi

% theindex-Environment neu definieren ohne reledmac
\makeatletter
\renewenvironment{theindex}{%
  \ifkorrekturansicht
    \section*{\indexname}%
  \else
    \subsubsection*{Index der erwähnten Entitäten}%
  \fi
  \setlength{\parindent}{0pt}%
  \setlength{\parskip}{0pt plus 0.3pt}%
  \let\item\@idxitem
}{%
  \ifkorrekturansicht\clearpage\fi
}
\makeatother

\IfFileExists{\jobname-pw.ind}{\input{\jobname-pw.ind}}{}

% Quellenangabe nur in der Leseansicht
\ifkorrekturansicht\else
% Fallback-Definitionen, falls die .tex-Datei \titel etc. nicht gesetzt hat
\providecommand{\titel}{}
\providecommand{\editorInnen}{}
\providecommand{\dateiname}{\jobname}

\vspace{3cm}

\vfill

\footnotesize
\textsc{Quelle}: \titel. Herausgegeben von {\editorInnen}. In: \emph{Arthur Schnitzler: Briefwechsel mit Autorinnen und Autoren}.
 Digitale Edition, https://schnitzler-briefe.acdh.oeaw.ac.at/{\dateiname}.html (Stand \today)
\fi

\end{document}


