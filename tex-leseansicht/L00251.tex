%% latex-korrekturansicht-vorspann.tex
%% Vorspann für die Korrekturansicht.
%% Lädt die gemeinsame Datei latex-vorspann.tex mit gesetztem Schalter.

\newif\ifkorrekturansicht
\korrekturansichttrue

\input{../tex-inputs/latex-vorspann}


\section[Arthur Schnitzler an Hugo von Hofmannsthal, 11. 8. 1893]{L00251 Arthur Schnitzler an Hugo von Hofmannsthal, 11. 8. 1893}
\nopagebreak\mylabel{L00251v}
\rehead{ }\normalsize\beginnumbering\briefempfaengerindex{Hofmannsthal, Hugo von@\textsc{Hofmannsthal, Hugo von}!zzzSchnitzler, Arthur@\emph{von Arthur Schnitzler}!1893-08-112@{11. 8. 1893}|(be}
\toendnotes[C]{\smallbreak\pagebreak[2]}\Standort{FDH, Hs-30885,38.}
\physDesc{Brief, 2 Blätter, 5 Seiten, 1698 Zeichen (Briefpapier mit Trauerrand)
\newline{}Handschrift: schwarze Tinte, deutsche Kurrent
\newline{}Ordnung: mit Bleistift von Schnitzler mutmaßlich bei der Durchsicht der Korrespondenz
                                    1929 datiert: »11. 8. 93« }
\buchAbdrucke{\weitereDrucke{Hugo von Hofmannsthal, Arthur Schnitzler: \emph{Briefwechsel}. Frankfurt am Main: \emph{S. Fischer} 1964, S. 43–44.} }\toendnotes[C]{\smallbreak}
\pstart{}{\pb}Lieber Hugo,\pend\vspace{0.5em}
\pstart
           Ihr Feu{[}i{]}lleton\pwindex{Gabriele DAnnunzio@\emph{Gabriele d’Annunzio}|pwv} über \textsc{Annunzio}\pwindex{DAnnunzio, Gabriele 12.03.1863 – 01.03.1938@\textsc{D’Annunzio, Gabriele} (12.03.1863 – 01.03.1938), \emph{Schriftsteller/Schriftstellerin}|pw} hab ich mit großer Freude geleſen; es iſt eins Ihrer ſchönſten, mit weiten
               Ausblicken. – Iſt von dem Mann was ins Deutſche überſetzt? –\pend
           
\pstart
           – Denken Sie, mir iſt man endlich draufgeko{\geminationm}en, daſs ich
                  \strikeout{i}auf die ſexuellen Inſtincte der Menge ſpeculire und
                  {\pb}meine »cyniſchen\pwindex{Literatur. »Bunte Reihe.« Ein Geschichtenbuch von Moritz Goldschmidt. »Anatol« von Arthur Schnitzler@\emph{Literatur. »Bunte Reihe.« Ein Geschichtenbuch von Moritz Goldschmidt. »Anatol« von Arthur Schnitzler}|pwv}«, »plumpen\pwindex{Literatur. »Bunte Reihe.« Ein Geschichtenbuch von Moritz Goldschmidt. »Anatol« von Arthur Schnitzler@\emph{Literatur. »Bunte Reihe.« Ein Geschichtenbuch von Moritz Goldschmidt. »Anatol« von Arthur Schnitzler}|pwv}« Sachen mit verletzender Abſichtlichkeit ſchreibe –
               (offenbar um mittelſt meiner Trivialität viel Geld zu machen.) – Der Ruhm dieſer
               Entdeckung gebührt der Wiener Abendpoſt\orgindex{Wiener Abendpost@Wiener Abendpost|pw}, welche
               im übrigen zugleich Geſchmack genug hat, die Leichtbeſchwingtheit Ihrer Verſe\pwindex{Einleitung@\emph{Einleitung}|pwv} zu loben. (Referent Bruno Walden\pwindex{Galliny, Florentine 24.06.1845 – 19.07.1913@\textsc{Galliny, Florentine} (24.06.1845 – 19.07.1913), \emph{Schriftsteller/Schriftstellerin, Journalist/Journalistin}|pw}.) –\pend
           
\pstart
           Meine Abſicht geht vorläufig dahin {\pb}Ende nächſter Woche
               ins Puſterthal\oindex{Pustertal@\textbf{Pustertal}, \emph{T.VAL}|pw} zu reiſen, und vielleicht von dort
               per \textsc{Bic.} nach Wien\oindex{Wien@\textbf{Wien}, \emph{A.ADM2}|pw}
               zurück. (\textsc{Salten}\pwindex{Salten, Felix 06.09.1869 – 08.10.1945@\textsc{Salten, Felix} (06.09.1869 – 08.10.1945), \emph{Schriftsteller/Schriftstellerin, Journalist/Journalistin, Chefredakteur/Chefredakteurin}|pw} iſt bereits unten.) – \textsc{Paul Goldma{\geminationn}}\pwindex{Goldmann, Paul 31.01.1865 – 25.09.1935@\textsc{Goldmann, Paul} (31.01.1865 – 25.09.1935), \emph{Schriftsteller/Schriftstellerin, Journalist/Journalistin}|pw} will im September nach Salzburg\oindex{Salzburg@\textbf{Salzburg}, \emph{A.ADM2}|pw}
                  ko{\geminationm}en; vielleicht läßt ſich eine Zuſa{\geminationm}enkunft Ende Auguſt arrangiren?\pend
           
\pstart
           Wie ſind Ihre Pläne? Schreiben Sie doch was darüber. Arbeiten Sie was? Meine kleine
                  Novelle\pwindex{kleine Komoedie@\emph{Die kleine Komödie}|pw} iſt bis auf wenige Zeilen fertig. {\pb}Das hab ich Ihnen ſchon geſchrieben. – Jetzt ſchreib ich
               ab und zu ein paar Verſe an dem »allegoriſchen« Gedicht\pwindex{Artifex@\emph{Artifex}|pw}; bedauere aber ſehr, nicht die ausreichende Befähigung dazu zu
               haben. –\pend
           
\pstart
           Den Mut zu was größerem, das wird Sie nach alledem nicht wundern, hab ich noch nicht
               erlangt. – Unter vier Augen: das Volkstheater\oindex{Volkstheater@\textbf{Volkstheater}, \emph{Theater (K.THE)}|pw}
               beginnt mit mir \introOben{}(wegen »Märchen\pwindex{Maerchen. Schauspiel in drei Aufzuegen@\emph{Das Märchen. Schauspiel in drei Aufzügen}|pw}«)\introOben{} zu unterhandeln; ich ſage Ihnen – Zuſtände!! – Weiteres
               darüber mündlich.\pend
           
\pstart
           {\pb}– Wie gehts dem aegyptiſchen\oindex{Aegypten@\textbf{Ägypten}, \emph{A.PCLI}|pw} unanſtändigen Stück\pwindex{Alexanderzug@\emph{Alexanderzug}|pwv}? – Wenn es \uline{nur}{ }aegyptiſch\oindex{Aegypten@\textbf{Ägypten}, \emph{A.PCLI}|pw} wäre, läge es der Allgemeinheit zu
               fern! – Der Tod \textsc{Kafka\pwindex{Kafka, Eduard Michael 11.03.1869 – 06.08.1893@\textsc{Kafka, Eduard Michael} (11.03.1869 – 06.08.1893), \emph{Redakteur/Redakteurin}|pw}}’s iſt Ihnen wohl bekannt worden? –\pend
           
\pstart
           – Hören Sie was von \textsc{Fels\pwindex{Fels, Friedrich Michael *~1864@\textsc{Fels, Friedrich Michael} (*~1864), \emph{Journalist/Journalistin}|pw}}? – Schreibt Ihnen Richard\pwindex{Beer-Hofmann, Richard 1866-07-11 – 1945-09-26@\textsc{Beer-Hofmann, Richard} (1866-07-11 – 1945-09-26), \emph{Schriftsteller/Schriftstellerin}|pw}? –\pend
           
\pstart
           Sind Sie vergnügt? –\pend
           
\pstart
           Herzlich der Ihre{\\[\baselineskip]}\spacefill\mbox{Arthur}\pend
           \leftskip=0em{}
\pstart
           Wien\oindex{Wien@\textbf{Wien}, \emph{A.ADM2}|pw}, 11. 8. 93\pend
           
\pstart
           \label{T_L00251-1v}\edtext{\uline{Sie müssen \textsc{Bicycle} fahren
                     lernen!}}{\lemma{\textnormal{\emph{Sie … lernen!}}}\Cendnote{\textnormal{quer am linken Rand}}}\label{T_L00251-1}\pend
           \selectlanguage{ngerman}\endnumbering\briefempfaengerindex{Hofmannsthal, Hugo von@\textsc{Hofmannsthal, Hugo von}!zzzSchnitzler, Arthur@\emph{von Arthur Schnitzler}!1893-08-112@{11. 8. 1893}|)be}\mylabel{L00251h}  \normalsize

\doendnotes{C}
\bigskip
\vfill

\clearpage

\footnotesize

\lohead{\textsc{register}}

% Definiere theindex-Environment komplett neu ohne reledmac
\makeatletter
\renewenvironment{theindex}{%
  \section*{\indexname}%
  \setlength{\parindent}{0pt}%
  \setlength{\parskip}{0pt plus 0.3pt}%
  \let\item\@idxitem
}{%
  \clearpage
}
\makeatother

\IfFileExists{\jobname-pw.ind}{\input{\jobname-pw.ind}}{}

\end{document}

      