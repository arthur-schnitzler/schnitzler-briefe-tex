%% latex-korrekturansicht-vorspann.tex
%% Vorspann für die Korrekturansicht.
%% Lädt die gemeinsame Datei latex-vorspann.tex mit gesetztem Schalter.

\newif\ifkorrekturansicht
\korrekturansichttrue

\input{../tex-inputs/latex-vorspann}


\section[ Paul Goldmann an Arthur Schnitzler, 17. 7. {[}1896{]}]{L02782 Paul Goldmann an Arthur Schnitzler, 17. 7. {[}1896{]}}
\nopagebreak\mylabel{L02782v}
\rehead{ }\normalsize\beginnumbering\briefempfaengerindex{Schnitzler, Arthur@\textsc{Schnitzler, Arthur}!zzzGoldmann, Paul@\emph{von Paul Goldmann}!1896-07-171@{17. 7. {[}1896{]}}|(be}
\toendnotes[C]{\smallbreak\pagebreak[2]}\Standort{DLA, A:Schnitzler, HS.NZ85.1.3166.}
\physDesc{Brief, 1 Blatt, 1 Seite, 348 Zeichen
\newline{}Handschrift: blaue Tinte, deutsche Kurrent
\newline{}Schnitzler: mit Bleistift das Jahr »96« vermerkt }\toendnotes[C]{\smallbreak}
\pstart
           {\pb}\textcolor{gray}{\textbf{\textbf{Frankfurter Zeitung\orgindex{Frankfurter Zeitung@Frankfurter Zeitung|pw}}}}\pend
           
\pstart
           \textcolor{gray}{\textbf{(\begin{otherlanguage}{french}Gazette de Francfort\end{otherlanguage}\orgindex{Frankfurter Zeitung@Frankfurter Zeitung|pw}).}}\pend
           
\pstart
           \textcolor{gray}{\textbf{\textbf{\begin{otherlanguage}{french}Fondateur M.\end{otherlanguage}{ }L. Sonnemann\pwindex{Sonnemann, Leopold 1831-10-29 – 1909-10-30@\textsc{Sonnemann, Leopold} (1831-10-29 – 1909-10-30), \emph{Journalist/Journalistin, Herausgeber/Herausgeberin}|pw}.}}}\pend
           
\pstart
           \begin{otherlanguage}{french}\textcolor{gray}{\textbf{Journal\pwindex{Frankfurter Zeitung@\emph{Frankfurter Zeitung}|pwv} politique,
                        financier,}}\end{otherlanguage}\pend
           
\pstart
           \begin{otherlanguage}{french}\textcolor{gray}{\textbf{commercial et littéraire.}}\end{otherlanguage}\pend
           
\pstart
           \begin{otherlanguage}{french}\textcolor{gray}{\textbf{\textbf{Paraissant trois fois par jour.}}}\end{otherlanguage}\hfill \textsc{Paris\oindex{Paris@\textbf{Paris}, \emph{P.PPLC}|pw}}, 17. Juli.\pend
           
\pstart
           \begin{otherlanguage}{french}\textcolor{gray}{\textbf{\textbf{Bureau à Paris\oindex{Paris@\textbf{Paris}, \emph{P.PPLC}|pw}}}}\end{otherlanguage}\pend
           
\pstart
           \begin{otherlanguage}{french}\textcolor{gray}{\textbf{\textbf{24. Rue Feydeau\oindex{rue Feydeau@\textbf{rue Feydeau}, \emph{Straße (K.STR)}|pw}.}}}\end{otherlanguage}\pend
           
\pstart\center{}Mein lieber Freund,\pend\vspace{0.5em}
\pstart
           Einen Brief von mir findeſt Du in Chriſtiana\oindex{Oslo@\textbf{Oslo}, \emph{P.PPLC}|pwv}. Nun werde ich aber vielleicht ſchon am 25. oder 26. Juli abreiſen müſſen, aus
               unvorhergeſehenen Gründen.\pend
           
\pstart
           Bitte, ſchreibe \strikeout{oder} mir ſofort, womöglich
               telegraphire mir: \strikeout{wann}{ }\label{K_L02782-1v}\edtext{bis wann}{\lemma{\textnormal{\emph{bis wann}}}\Cendnote{\textnormal{Schnitzler hielt sich am 2. 8. 1896 und 3. 8. 1896 in Kopenhagen\oindex{Kopenhagen@\textbf{Kopenhagen}, \emph{P.PPLC}|pwk} auf, wo er im Hotel Kongen af Danmark (Hotel König von
                     Dänemark)\oindex{Hotel Kongen af Danmark@\textbf{Hotel Kongen af Danmark}, \emph{Hotel (K.HTL)}|pwk} nächtigte. Am 3. 8. 1896 fuhr er nachmittags
                  weiter nach Skodsborg\oindex{Skodsborg@\textbf{Skodsborg}, \emph{P.PPL}|pwk}. Dort beherbergte ihn
                  das Badehotel\oindex{Badehotellet@\textbf{Badehotellet}, \emph{Hotel (K.HTL)}|pwk}. Schnitzler erreichte dieser Brief vermutlich am 23. 7. 1896 in Trondheim\oindex{Trondheim@\textbf{Trondheim}, \emph{P.PPLA2}|pwk}, wohin er sich seine Post schicken
                  ließ (vgl. Arthur Schnitzler an Richard Beer-Hofmann, 27. 6. 1896 und A. S.: \emph{Tagebuch}, 23. 7. 1896).}}}\label{K_L02782-1} biſt Du in \textsc{Kopenhagen\oindex{Kopenhagen@\textbf{Kopenhagen}, \emph{P.PPLC}|pw}}? In welchem Hotel? Wann und wo in \textsc{Scodsborg\oindex{Skodsborg@\textbf{Skodsborg}, \emph{P.PPL}|pw}}?\pend
           
\pstart
           Viele treue Grüße!{\\[\baselineskip]}Dein{\\[\baselineskip]}\spacefill\mbox{P. G.}\pend
           \leftskip=0em{}
\pstart
           \noindent{}In Eile\pend
           \selectlanguage{ngerman}\endnumbering\briefempfaengerindex{Schnitzler, Arthur@\textsc{Schnitzler, Arthur}!zzzGoldmann, Paul@\emph{von Paul Goldmann}!1896-07-171@{17. 7. {[}1896{]}}|)be}\mylabel{L02782h}  \normalsize

\doendnotes{C}
\bigskip
\vfill

\clearpage

\footnotesize

\lohead{\textsc{register}}

% Definiere theindex-Environment komplett neu ohne reledmac
\makeatletter
\renewenvironment{theindex}{%
  \section*{\indexname}%
  \setlength{\parindent}{0pt}%
  \setlength{\parskip}{0pt plus 0.3pt}%
  \let\item\@idxitem
}{%
  \clearpage
}
\makeatother

\IfFileExists{\jobname-pw.ind}{\input{\jobname-pw.ind}}{}

\end{document}

      