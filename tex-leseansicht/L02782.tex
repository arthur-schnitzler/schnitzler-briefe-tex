%% latex-leseansicht-vorspann.tex
%% Vorspann für die Leseansicht.
%% Lädt die gemeinsame Datei latex-vorspann.tex mit nicht gesetztem Schalter.

\newif\ifkorrekturansicht
\korrekturansichtfalse

\input{../tex-inputs/latex-vorspann}


         
         \renewcommand{\erwaehntePersonen}{Personen: Leopold Sonnemann}
         \renewcommand{\erwaehnteInstitutionen}{Institutionen: Frankfurter Zeitung}
         \renewcommand{\erwaehnteOrte}{Orte: Badehotel, Hotel König von Dänemark, Kopenhagen, Oslo, Paris, Skodsborg, Trondheim, rue Feydeau}
         \renewcommand{\erwaehnteWerke}{Werke: Frankfurter Zeitung}
               \section[ Paul Goldmann an Arthur Schnitzler, 17. 7. {[}1896{]}]{ Paul Goldmann an Arthur Schnitzler, 17. 7. {[}1896{]}}\nopagebreak\mylabel{v}\rehead{ }\begin{ledgroupsized}[t]{13cm}\normalsize\beginnumbering \toendnotes[C]{\smallbreak\pagebreak[2]} \Standort{DLA, A:Schnitzler, HS.NZ85.1.3166.}
\physDesc{Brief, 1 Blatt, 1 Seite, 348 Zeichen
\newline{}Handschrift: blaue Tinte, deutsche Kurrent
\newline{}Schnitzler: mit Bleistift das Jahr »96« vermerkt }\toendnotes[C]{\smallbreak}\pstart
           \noindent{}{\pb}\textcolor{gray}{\textbf{\textbf{Frankfurter Zeitung\orgindex{Frankfurter Zeitung@Frankfurter Zeitung|pw}}}}\pend
           \pstart
           \textcolor{gray}{\textbf{(\begin{otherlanguage}{french}Gazette de Francfort\end{otherlanguage}\orgindex{Frankfurter Zeitung@Frankfurter Zeitung|pw}).}}\pend
           \pstart
           \textcolor{gray}{\textbf{\textbf{\begin{otherlanguage}{french}Fondateur M.\end{otherlanguage}{ }L. Sonnemann\pwindex{Sonnemann, Leopold 1831-10-29 – 1909-10-30@\textsc{Sonnemann, Leopold} (1831-10-29 – 1909-10-30), \emph{Journalist, Herausgeber}|pw}.}}}\pend
           \pstart
           \begin{otherlanguage}{french}\textcolor{gray}{\textbf{Journal\pwindex{?? Werk@Nicht ermittelte Verfasserinnen und Verfasser!Frankfurter Zeitung1856 – 1943@\emph{Frankfurter Zeitung} {[}1856 – 1943{]}|pwv} politique,
                        financier,}}\end{otherlanguage}\pend
           \pstart
           \begin{otherlanguage}{french}\textcolor{gray}{\textbf{commercial et littéraire.}}\end{otherlanguage}\pend
           \pstart
           \begin{otherlanguage}{french}\textcolor{gray}{\textbf{\textbf{Paraissant trois fois par jour.}}}\end{otherlanguage}\hfill \textsc{Paris\oindex{Paris@\textbf{Paris}|pw}}, 17. Juli.\pend
           \pstart
           \begin{otherlanguage}{french}\textcolor{gray}{\textbf{\textbf{Bureau à Paris\oindex{Paris@\textbf{Paris}|pw}}}}\end{otherlanguage}\pend
           \pstart
           \begin{otherlanguage}{french}\textcolor{gray}{\textbf{\textbf{24. Rue Feydeau\oindex{rue Feydeau@\textbf{rue Feydeau}|pw}.}}}\end{otherlanguage}\pend
           \pstart\center{}Mein lieber Freund,\pend\pstart
           Einen Brief von mir findeſt Du in Chriſtiana\oindex{Oslo@\textbf{Oslo}|pwv}. Nun werde ich aber vielleicht ſchon am 25. oder 26. Juli abreiſen müſſen, aus
               unvorhergeſehenen Gründen.\pend
           \pstart
           Bitte, ſchreibe \strikeout{oder} mir ſofort, womöglich
               telegraphire mir: \strikeout{wann}{ }\label{K_L02782-1v}\edtext{bis wann}{\lemma{\textnormal{\emph{bis wann}}}\Cendnote{\textnormal{Schnitzler\pwindex{Schnitzler, Arthur 15.05.1862 – 21.10.1931@\textsc{Schnitzler, Arthur} (15.05.1862 – 21.10.1931), \emph{Schriftsteller, Mediziner}|pwk} hielt sich am 2. 8. 1896 und 3. 8. 1896 in Kopenhagen\oindex{Kopenhagen@\textbf{Kopenhagen}|pwk} auf, wo er im Hotel Kongen af Danmark (Hotel König von
                     Dänemark)\oindex{Hotel Koenig von Daenemark@\textbf{Hotel König von Dänemark}|pwk} nächtigte. Am 3. 8. 1896 fuhr er nachmittags
                  weiter nach Skodsborg\oindex{Skodsborg@\textbf{Skodsborg}|pwk}. Dort beherbergte ihn
                  das Badehotel\oindex{Badehotel@\textbf{Badehotel}|pwk}. Schnitzler\pwindex{Schnitzler, Arthur 15.05.1862 – 21.10.1931@\textsc{Schnitzler, Arthur} (15.05.1862 – 21.10.1931), \emph{Schriftsteller, Mediziner}|pwk} erreichte dieser Brief vermutlich am 23. 7. 1896 in Trondheim\oindex{Trondheim@\textbf{Trondheim}|pwk}, wohin er sich seine Post schicken
                  ließ (vgl. Arthur Schnitzler an Richard Beer-Hofmann, 27. 6. 1896 und A. S.: \emph{Tagebuch}, 23. 7. 1896).}}}\label{K_L02782-1h} biſt Du in \textsc{Kopenhagen\oindex{Kopenhagen@\textbf{Kopenhagen}|pw}}? In welchem Hotel? Wann und wo in \textsc{Scodsborg\oindex{Skodsborg@\textbf{Skodsborg}|pw}}?\pend
           \pstart
           Viele treue Grüße!{\\[\baselineskip]}Dein{\\[\baselineskip]}\spacefill\mbox{P. G.}\pend
           \leftskip=0em{}\pstart
           \noindent{}In Eile\pend
           
         
         \endnumbering\mylabel{h}\end{ledgroupsized}  \newcommand{\dateiname}{L02782}\newcommand{\titel}{Paul Goldmann an Arthur Schnitzler, 17. 7. [1896]}\newcommand{\editorInnen}{Martin Anton Müller und Laura Untner}%% latex-leseansicht-abspann.tex
%% Abspann für die Leseansicht.
%% Der Schalter \ifkorrekturansicht ist bereits durch den Vorspann gesetzt.

%% latex-abspann.tex
%% Gemeinsamer Abspann für Korrekturansicht und Leseansicht.
%% Setzt den Schalter \ifkorrekturansicht voraus (gesetzt in den
%% einbindenden Dateien latex-korrekturansicht-abspann.tex bzw.
%% latex-leseansicht-abspann.tex).
%% ---------------------------------------------------------------

\normalsize

% Das esempio-Environment wird nur in der Leseansicht benötigt
\ifkorrekturansicht\else
\newenvironment{esempio}[3]%
{
    \vspace{1.5ex}
    \rlap{\underline{#1}}
    \par
    \setlength{\parindent}{0cm}
    \nopagebreak
    \leftskip=#2cm
    \rightskip=#3cm
}
{
    \par
}
\fi

\doendnotes{C}
\bigskip
\vfill

\clearpage

\footnotesize

\ifkorrekturansicht
  \lohead{\textsc{register}}
\fi

% theindex-Environment neu definieren ohne reledmac
\makeatletter
\renewenvironment{theindex}{%
  \ifkorrekturansicht
    \section*{\indexname}%
  \else
    \subsubsection*{Index der erwähnten Entitäten}%
  \fi
  \setlength{\parindent}{0pt}%
  \setlength{\parskip}{0pt plus 0.3pt}%
  \let\item\@idxitem
}{%
  \ifkorrekturansicht\clearpage\fi
}
\makeatother

\IfFileExists{\jobname-pw.ind}{\input{\jobname-pw.ind}}{}

% Quellenangabe nur in der Leseansicht
\ifkorrekturansicht\else
% Fallback-Definitionen, falls die .tex-Datei \titel etc. nicht gesetzt hat
\providecommand{\titel}{}
\providecommand{\editorInnen}{}
\providecommand{\dateiname}{\jobname}

\vspace{3cm}

\vfill

\footnotesize
\textsc{Quelle}: \titel. Herausgegeben von {\editorInnen}. In: \emph{Arthur Schnitzler: Briefwechsel mit Autorinnen und Autoren}.
 Digitale Edition, https://schnitzler-briefe.acdh.oeaw.ac.at/{\dateiname}.html (Stand \today)
\fi

\end{document}


      