%% latex-leseansicht-vorspann.tex
%% Vorspann für die Leseansicht.
%% Lädt die gemeinsame Datei latex-vorspann.tex mit nicht gesetztem Schalter.

\newif\ifkorrekturansicht
\korrekturansichtfalse

\input{../tex-inputs/latex-vorspann}


         
         \renewcommand{\erwaehntePersonen}{Personen: Max Eugen Burckhard, Vincenz Chiavacci, Carl Karlweis}
         \renewcommand{\erwaehnteOrte}{Orte: Berlin, Burgtheater, Volkstheater, Wien}
         \renewcommand{\erwaehnteWerke}{Werke: Der Apostel. Schauspiel in drei Aufzügen, Wienerinnen. Lustspiel in drei Akten}
               \section[Hermann Bahr an Arthur Schnitzler, 10. 7. {[}1902{]}]{ Hermann Bahr an Arthur Schnitzler, 10. 7. {[}1902{]}}\nopagebreak\mylabel{v}\rehead{ }\begin{ledgroupsized}[t]{13cm}\normalsize\beginnumbering \toendnotes[C]{\smallbreak\pagebreak[2]} \Standort{CUL, Schnitzler, B 5b.}
\physDesc{Brief, 1 Blatt (Briefpapier mit Trauerrand), 4 Seiten
\newline{}Handschrift: schwarze Tinte, deutsche Kurrent\newline{}Ordnung: mit Bleistift von unbekannter Hand nummeriert: »90« }\buchAbdrucke{\weitereDrucke{Hermann Bahr, Arthur Schnitzler: \emph{Briefwechsel, Aufzeichnungen, Dokumente (1891–1931)}. Hg. Kurt Ifkovits und Martin Anton Müller. Göttingen: \emph{Wallstein} 2018, S. 241.} }\pstart
           \raggedleft{}{\pb}10. Juli\pend
           \pstart\center{}Lieber Arthur!\pend\pstart
           Denſelben Wiſch hat \textsc{Burckhard}\pwindex{Burckhard, Max Eugen 14.07.1854 – 16.03.1912@\textsc{Burckhard, Max Eugen} (14.07.1854 – 16.03.1912), \emph{Schriftsteller, Rechtswissenschaftler, Theaterleiter}|pw} bekommen, voriges Jahr \textsc{Karlweis}\pwindex{Karlweis, Carl 23.11.1850 – 27.10.1901@\textsc{Karlweis, Carl} (23.11.1850 – 27.10.1901), \emph{Schriftsteller}|pw} und \textsc{Chiavacci}\pwindex{Chiavacci, Vincenz 15.06.1847 – 02.02.1916@\textsc{Chiavacci, Vincenz} (15.06.1847 – 02.02.1916), \emph{Schriftsteller, Journalist, Beamter}|pw}, und mit derſelben Wirkung: einer
               Anfrage bei mir. Gesetzlich biſt Du verpflichtet, eine Antwort zu geben. \uline{Ich} werde aber, wenn ich jemals befragt werde,
               antworten, daß ich das Einkommen {\pb}auch meiner
               nächſten Freunde weder kenne noch mir darüber Gedanken mache, weil es mich gar nicht
               intereſſiert.\pend
           \pstart
           Übrigens theile ich Dir der Wahrheit gemäß mit: 1) Daß in der Zeit vom 1. Januar bis
               zum 31. December 1901 überhaupt kein Stück von mir in Berlin\oindex{Berlin@\textbf{Berlin}|pw} aufgeführt wurde; {\pb}2) Daß in Wien\oindex{Wien@\textbf{Wien}|pw} am Deutſchen
                  Volkstheater\oindex{Volkstheater@\textbf{Volkstheater}|pw} noch »Wienerinnen\pwindex{Bahr, Hermann 19.07.1863 – 15.01.1934@\textsc{Bahr, Hermann} (19.07.1863 – 15.01.1934), \emph{Schriftsteller, Kritiker}!Wienerinnen. Lustspiel in drei Akten1900@\strich\emph{Wienerinnen. Lustspiel in drei Akten} {[}1900{]}|pw}« weiter
               gegeben wurde, daß aber der eigentliche Zug dieſes im Oktober 1900 zum erſten Mal
               aufgeführten Stückes im Jänner 1901 bereits vorüber war.\pend
           \pstart
           3) Daß in Wien\oindex{Wien@\textbf{Wien}|pw} am Burgtheater\oindex{Burgtheater@\textbf{Burgtheater}|pw} der »Apoſtel\pwindex{Bahr, Hermann 19.07.1863 – 15.01.1934@\textsc{Bahr, Hermann} (19.07.1863 – 15.01.1934), \emph{Schriftsteller, Kritiker}!Apostel. Schauspiel in drei Aufzuegen1901@\strich\emph{Der Apostel. Schauspiel in drei Aufzügen} {[}1901{]}|pw}« im November
               und December 1901 zehn Mal gegeben, die {\pb}Tantièmen
               hiefür erſt am 4. Januar verrechnet, erſt im Februar von mir behoben wurden und alſo
               nicht \textsc{pro} 1901 fatiert werden konnten. Und nun rechne Dir
               meine Reichthümer aus! Roman oder Novelle habe ich 1901 keine geſchrieben.\pend
           \pstart
           Herzlichſt{\\[\baselineskip]}Dein alter{\\[\baselineskip]}\spacefill\mbox{Hermann}\pend
           \leftskip=0em{}
         
         \endnumbering\mylabel{h}\end{ledgroupsized}  \newcommand{\dateiname}{L01230}\newcommand{\titel}{Hermann Bahr an Arthur Schnitzler, 10. 7. [1902]}\newcommand{\editorInnen}{ Kurt Ifkovits,  Martin Anton Müller}%% latex-leseansicht-abspann.tex
%% Abspann für die Leseansicht.
%% Der Schalter \ifkorrekturansicht ist bereits durch den Vorspann gesetzt.

%% latex-abspann.tex
%% Gemeinsamer Abspann für Korrekturansicht und Leseansicht.
%% Setzt den Schalter \ifkorrekturansicht voraus (gesetzt in den
%% einbindenden Dateien latex-korrekturansicht-abspann.tex bzw.
%% latex-leseansicht-abspann.tex).
%% ---------------------------------------------------------------

\normalsize

% Das esempio-Environment wird nur in der Leseansicht benötigt
\ifkorrekturansicht\else
\newenvironment{esempio}[3]%
{
    \vspace{1.5ex}
    \rlap{\underline{#1}}
    \par
    \setlength{\parindent}{0cm}
    \nopagebreak
    \leftskip=#2cm
    \rightskip=#3cm
}
{
    \par
}
\fi

\doendnotes{C}
\bigskip
\vfill

\clearpage

\footnotesize

\ifkorrekturansicht
  \lohead{\textsc{register}}
\fi

% theindex-Environment neu definieren ohne reledmac
\makeatletter
\renewenvironment{theindex}{%
  \ifkorrekturansicht
    \section*{\indexname}%
  \else
    \subsubsection*{Index der erwähnten Entitäten}%
  \fi
  \setlength{\parindent}{0pt}%
  \setlength{\parskip}{0pt plus 0.3pt}%
  \let\item\@idxitem
}{%
  \ifkorrekturansicht\clearpage\fi
}
\makeatother

\IfFileExists{\jobname-pw.ind}{\input{\jobname-pw.ind}}{}

% Quellenangabe nur in der Leseansicht
\ifkorrekturansicht\else
% Fallback-Definitionen, falls die .tex-Datei \titel etc. nicht gesetzt hat
\providecommand{\titel}{}
\providecommand{\editorInnen}{}
\providecommand{\dateiname}{\jobname}

\vspace{3cm}

\vfill

\footnotesize
\textsc{Quelle}: \titel. Herausgegeben von {\editorInnen}. In: \emph{Arthur Schnitzler: Briefwechsel mit Autorinnen und Autoren}.
 Digitale Edition, https://schnitzler-briefe.acdh.oeaw.ac.at/{\dateiname}.html (Stand \today)
\fi

\end{document}


      