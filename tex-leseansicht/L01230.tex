%% latex-korrekturansicht-vorspann.tex
%% Vorspann für die Korrekturansicht.
%% Lädt die gemeinsame Datei latex-vorspann.tex mit gesetztem Schalter.

\newif\ifkorrekturansicht
\korrekturansichttrue

\input{../tex-inputs/latex-vorspann}


\section[Hermann Bahr an Arthur Schnitzler, 10. 7. {[}1902{]}]{L01230 Hermann Bahr an Arthur Schnitzler, 10. 7. {[}1902{]}}
\nopagebreak\mylabel{L01230v}
\rehead{ }\normalsize\beginnumbering\briefempfaengerindex{Schnitzler, Arthur@\textsc{Schnitzler, Arthur}!zzzBahr, Hermann@\emph{von Hermann Bahr}!1902-07-101@{10. 7. 1902}|(be}
\toendnotes[C]{\smallbreak\pagebreak[2]}\Standort{CUL, Schnitzler, B 5b.}
\physDesc{Brief, 1 Blatt, 4 Seiten, 1086 Zeichen (Briefpapier mit Trauerrand)
\newline{}Handschrift: schwarze Tinte, deutsche Kurrent
\newline{}Ordnung: mit Bleistift von unbekannter Hand nummeriert:
                                    »90« }
\buchAbdrucke{\weitereDrucke{Hermann Bahr, Arthur Schnitzler: \emph{Briefwechsel, Aufzeichnungen, Dokumente (1891–1931)}. Göttingen: \emph{Wallstein} 2018, S. 241.} }\toendnotes[C]{\smallbreak}
\pstart
           \raggedleft{}{\pb}10. Juli\pend
           
\pstart\center{}Lieber Arthur!\pend\vspace{0.5em}
\pstart
           Denſelben \label{K_L01230-1v}\edtext{Wiſch}{\lemma{\textnormal{\emph{Wiſch}}}\Cendnote{\textnormal{Hermann Bahr, Arthur Schnitzler: \emph{Briefwechsel, Aufzeichnungen, Dokumente (1891–1931)}, Leopold Hipp an Arthur Schnitzler, 28. 6. 1902.
               }}}\label{K_L01230-1} hat \textsc{Burckhard}\pwindex{Burckhard, Max Eugen 14.07.1854 – 16.03.1912@\textsc{Burckhard, Max Eugen} (14.07.1854 – 16.03.1912), \emph{Schriftsteller/Schriftstellerin, Rechtswissenschaftler/Rechtswissenschaftlerin, Theaterleiter/Theaterleiterin}|pw} bekommen, voriges Jahr \textsc{Karlweis}\pwindex{Karlweis, Carl 23.11.1850 – 27.10.1901@\textsc{Karlweis, Carl} (23.11.1850 – 27.10.1901), \emph{Schriftsteller/Schriftstellerin}|pw} und \textsc{Chiavacci}\pwindex{Chiavacci, Vincenz 15.06.1847 – 02.02.1916@\textsc{Chiavacci, Vincenz} (15.06.1847 – 02.02.1916), \emph{Schriftsteller/Schriftstellerin, Journalist/Journalistin, Beamter/Beamte}|pw}, und mit derſelben Wirkung: einer Anfrage bei mir. Gesetzlich biſt Du
               verpflichtet, eine Antwort zu geben. \uline{Ich} werde aber,
               wenn ich jemals befragt werde, antworten, daß ich das Einkommen {\pb}auch meiner nächſten Freunde weder kenne noch mir
               darüber Gedanken mache, weil es mich gar nicht intereſſiert.\pend
           
\pstart
           Übrigens theile ich Dir der Wahrheit gemäß mit: 1) Daß in der Zeit vom 1. Januar bis
               zum 31. December 1901 überhaupt kein Stück von mir in Berlin\oindex{Berlin@\textbf{Berlin}, \emph{P.PPLC}|pw} aufgeführt wurde; {\pb}2) Daß in Wien\oindex{Wien@\textbf{Wien}, \emph{A.ADM2}|pw} am Deutſchen
                  Volkstheater\orgindex{Volkstheater@Volkstheater|pw} noch »Wienerinnen\pwindex{Wienerinnen. Lustspiel in drei Akten@\emph{Wienerinnen. Lustspiel in drei Akten}|pw}« weiter
               gegeben wurde, daß aber der eigentliche Zug dieſes im Oktober 1900 zum erſten Mal
               aufgeführten Stückes im Jänner 1901 bereits vorüber war.\pend
           
\pstart
           3) Daß in Wien\oindex{Wien@\textbf{Wien}, \emph{A.ADM2}|pw} am Burgtheater\orgindex{Burgtheater@Burgtheater|pw} der »Apoſtel\pwindex{Apostel. Schauspiel in drei Aufzuegen@\emph{Der Apostel. Schauspiel in drei Aufzügen}|pw}« im November
               und December 1901 zehn Mal gegeben, die {\pb}Tantièmen
               hiefür erſt am 4. Januar verrechnet, erſt im Februar von mir behoben wurden und alſo
               nicht \textsc{pro} 1901 fatiert werden konnten. Und nun rechne Dir
               meine Reichthümer aus! Roman oder Novelle habe ich 1901 keine geſchrieben.\pend
           
\pstart
           Herzlichſt{\\[\baselineskip]}Dein alter{\\[\baselineskip]}\spacefill\mbox{Hermann}\pend
           \leftskip=0em{}\selectlanguage{ngerman}\endnumbering\briefempfaengerindex{Schnitzler, Arthur@\textsc{Schnitzler, Arthur}!zzzBahr, Hermann@\emph{von Hermann Bahr}!1902-07-101@{10. 7. 1902}|)be}\mylabel{L01230h}  \normalsize

\doendnotes{C}
\bigskip
\vfill

\clearpage

\footnotesize

\lohead{\textsc{register}}

% Definiere theindex-Environment komplett neu ohne reledmac
\makeatletter
\renewenvironment{theindex}{%
  \section*{\indexname}%
  \setlength{\parindent}{0pt}%
  \setlength{\parskip}{0pt plus 0.3pt}%
  \let\item\@idxitem
}{%
  \clearpage
}
\makeatother

\IfFileExists{\jobname-pw.ind}{\input{\jobname-pw.ind}}{}

\end{document}

      