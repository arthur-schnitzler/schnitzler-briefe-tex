%% latex-leseansicht-vorspann.tex
%% Vorspann für die Leseansicht.
%% Lädt die gemeinsame Datei latex-vorspann.tex mit nicht gesetztem Schalter.

\newif\ifkorrekturansicht
\korrekturansichtfalse

\input{../tex-inputs/latex-vorspann}


         
         \renewcommand{\erwaehntePersonen}{Personen: Paul Goldmann, Clementine Goldmann}
         \renewcommand{\erwaehnteOrte}{Orte: Berlin, Bethmannstraße, Central-Hotel, Deutsches Theater Berlin, Frankfurt am Main}
         \renewcommand{\erwaehnteWerke}{Werke: Lebendige Stunden. Vier Einakter}
               \section[ Paul Goldmann an Arthur Schnitzler, 29. 12. {[}1901{]}]{ Paul Goldmann an Arthur Schnitzler, 29. 12. {[}1901{]}}\nopagebreak\mylabel{v}\rehead{ }\begin{ledgroupsized}[t]{13cm}\normalsize\beginnumbering\briefempfaengerindex{Schnitzler, Arthur@\textsc{Schnitzler, Arthur}!zzzGoldmann, Paul@\emph{von Paul Goldmann}!1901-12-291@{29. 12. {[}1901{]}}|(be} \toendnotes[C]{\smallbreak\pagebreak[2]} \Standort{DLA, A:Schnitzler, HS.NZ85.1.3171.}
\physDesc{Brief, 1 Blatt, 1 Seite, 530 Zeichen
\newline{}Handschrift: blaue Tinte, deutsche Kurrent
\newline{}Schnitzler: mit Bleistift das Jahr »901.« vermerkt }\toendnotes[C]{\smallbreak}\pstart
           {\pb}Frankfurt\oindex{Frankfurt am Main@\textbf{Frankfurt am Main}|pw}, 29. Dezember.\pend
           \pstart{}Mein lieber Freund,\pend\pstart
           Zu Deinem \label{K_L03098-1v}\edtext{Eintreffen in Berlin\oindex{Berlin@\textbf{Berlin}|pw}}{\lemma{\textnormal{\emph{Eintreffen in Berlin}}}\Cendnote{\textnormal{Schnitzler\pwindex{Schnitzler, Arthur 15.05.1862 – 21.10.1931@\textsc{Schnitzler, Arthur} (15.05.1862 – 21.10.1931), \emph{Schriftsteller, Mediziner}|pwk} war seit dem Vortag, dem 28. 12. 1901, in Berlin\oindex{Berlin@\textbf{Berlin}|pwk} und blieb bis zum 6. 1. 1902.}}}\label{K_L03098-1h}
               wünſche ich Dir alles gute Glück.\pend
           \pstart
           Bitte, ſchreib’ mir gleich (Adreſſe: \textsc{Hotel Central\oindex{Central-Hotel@\textbf{Central-Hotel}|pw}}, \textsc{Bethmannstraße\oindex{Bethmannstrasse@\textbf{Bethmannstraße}|pw}}), wie es auf den \label{K_L03098-2v}\edtext{Proben\pwindex{Schnitzler, Arthur 15.05.1862 – 21.10.1931@\textsc{Schnitzler, Arthur} (15.05.1862 – 21.10.1931), \emph{Schriftsteller, Mediziner}!Lebendige Stunden. Vier Einakter1901-12-23@\strich\emph{Lebendige Stunden. Vier Einakter} {[}1901-12-23{]}|pw}}{\lemma{\textnormal{\emph{Proben}}}\Cendnote{\textnormal{siehe A. S.: \emph{Tagebuch}, 28. 12. 1901, 3. 1. 1902 und Paul Goldmann an Arthur Schnitzler, 31. 12. [1901]}}}\label{K_L03098-2h} geht.\pend
           \pstart
           Ich werde \label{K_L03098-3v}\edtext{Samſtag{ }früh von hier wegfahren, um zu Deiner \textsc{Première\pwindex{Schnitzler, Arthur 15.05.1862 – 21.10.1931@\textsc{Schnitzler, Arthur} (15.05.1862 – 21.10.1931), \emph{Schriftsteller, Mediziner}!Lebendige Stunden. Vier Einakter1901-12-23@\strich\emph{Lebendige Stunden. Vier Einakter} {[}1901-12-23{]}|pw}} in \textsc{Berlin\oindex{Berlin@\textbf{Berlin}|pw}}}{\lemma{\textnormal{\emph{Samſtag … Berlin}}}\Cendnote{\textnormal{Am Samstag, dem 4. 1. 1902, fand am
                     Deutschen Theater Berlin\oindex{Deutsches Theater Berlin@\textbf{Deutsches Theater Berlin}|pwk} die Uraufführung der
                  vier Einakter \emph{Lebendige Stunden}\pwindex{Schnitzler, Arthur 15.05.1862 – 21.10.1931@\textsc{Schnitzler, Arthur} (15.05.1862 – 21.10.1931), \emph{Schriftsteller, Mediziner}!Lebendige Stunden. Vier Einakter1901-12-23@\strich\emph{Lebendige Stunden. Vier Einakter} {[}1901-12-23{]}|pwk} statt.}}}\label{K_L03098-3h}
               zu ſein.\pend
           \pstart
           Bitte, ſorge dafür, daß ich in meiner Wohnung ein Billet vorfinde.\pend
           \pstart
           Meine Mutter\pwindex{Goldmann, Clementine 1842-05-15 – 1924-02-24@\textsc{Goldmann, Clementine} (1842-05-15 – 1924-02-24)|pwv} (die Dich
               grüßen läßt) iſt auch in Frankfurt\oindex{Frankfurt am Main@\textbf{Frankfurt am Main}|pw}.\pend
           \pstart
           Es thut mir unendlich leid, daß Deine \strikeout{An} Anweſenheit
               in Berlin\oindex{Berlin@\textbf{Berlin}|pw} gerade in die Zeit meiner Abweſenheit
               fällt.\pend
           \pstart
           Viele treue Grüße! Dein {\\[\baselineskip]}\spacefill\mbox{Paul Goldmn}\pend
           \leftskip=0em{}
         
         \endnumbering\mylabel{h}\end{ledgroupsized}  \newcommand{\dateiname}{L03098}\newcommand{\titel}{Paul Goldmann an Arthur Schnitzler, 29. 12. [1901]}\newcommand{\editorInnen}{Martin Anton Müller und Laura Untner}%% latex-leseansicht-abspann.tex
%% Abspann für die Leseansicht.
%% Der Schalter \ifkorrekturansicht ist bereits durch den Vorspann gesetzt.

%% latex-abspann.tex
%% Gemeinsamer Abspann für Korrekturansicht und Leseansicht.
%% Setzt den Schalter \ifkorrekturansicht voraus (gesetzt in den
%% einbindenden Dateien latex-korrekturansicht-abspann.tex bzw.
%% latex-leseansicht-abspann.tex).
%% ---------------------------------------------------------------

\normalsize

% Das esempio-Environment wird nur in der Leseansicht benötigt
\ifkorrekturansicht\else
\newenvironment{esempio}[3]%
{
    \vspace{1.5ex}
    \rlap{\underline{#1}}
    \par
    \setlength{\parindent}{0cm}
    \nopagebreak
    \leftskip=#2cm
    \rightskip=#3cm
}
{
    \par
}
\fi

\doendnotes{C}
\bigskip
\vfill

\clearpage

\footnotesize

\ifkorrekturansicht
  \lohead{\textsc{register}}
\fi

% theindex-Environment neu definieren ohne reledmac
\makeatletter
\renewenvironment{theindex}{%
  \ifkorrekturansicht
    \section*{\indexname}%
  \else
    \subsubsection*{Index der erwähnten Entitäten}%
  \fi
  \setlength{\parindent}{0pt}%
  \setlength{\parskip}{0pt plus 0.3pt}%
  \let\item\@idxitem
}{%
  \ifkorrekturansicht\clearpage\fi
}
\makeatother

\IfFileExists{\jobname-pw.ind}{\input{\jobname-pw.ind}}{}

% Quellenangabe nur in der Leseansicht
\ifkorrekturansicht\else
% Fallback-Definitionen, falls die .tex-Datei \titel etc. nicht gesetzt hat
\providecommand{\titel}{}
\providecommand{\editorInnen}{}
\providecommand{\dateiname}{\jobname}

\vspace{3cm}

\vfill

\footnotesize
\textsc{Quelle}: \titel. Herausgegeben von {\editorInnen}. In: \emph{Arthur Schnitzler: Briefwechsel mit Autorinnen und Autoren}.
 Digitale Edition, https://schnitzler-briefe.acdh.oeaw.ac.at/{\dateiname}.html (Stand \today)
\fi

\end{document}


      