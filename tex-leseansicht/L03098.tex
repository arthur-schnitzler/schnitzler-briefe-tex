%% latex-korrekturansicht-vorspann.tex
%% Vorspann für die Korrekturansicht.
%% Lädt die gemeinsame Datei latex-vorspann.tex mit gesetztem Schalter.

\newif\ifkorrekturansicht
\korrekturansichttrue

\input{../tex-inputs/latex-vorspann}


\section[ Paul Goldmann an Arthur Schnitzler, 29. 12. {[}1901{]}]{L03098 Paul Goldmann an Arthur Schnitzler, 29. 12. {[}1901{]}}
\nopagebreak\mylabel{L03098v}
\rehead{ }\normalsize\beginnumbering\briefempfaengerindex{Schnitzler, Arthur@\textsc{Schnitzler, Arthur}!zzzGoldmann, Paul@\emph{von Paul Goldmann}!1901-12-291@{29. 12. {[}1901{]}}|(be}
\toendnotes[C]{\smallbreak\pagebreak[2]}\Standort{DLA, A:Schnitzler, HS.NZ85.1.3171.}
\physDesc{Brief, 1 Blatt, 1 Seite, 530 Zeichen
\newline{}Handschrift: blaue Tinte, deutsche Kurrent
\newline{}Schnitzler: mit Bleistift das Jahr »901.« vermerkt }\toendnotes[C]{\smallbreak}
\pstart
           {\pb}Frankfurt\oindex{Frankfurt am Main@\textbf{Frankfurt am Main}, \emph{P.PPLA3}|pw}, 29. Dezember.\pend
           
\pstart{}Mein lieber Freund,\pend\vspace{0.5em}
\pstart
           Zu Deinem \label{K_L03098-1v}\edtext{Eintreffen in Berlin\oindex{Berlin@\textbf{Berlin}, \emph{P.PPLC}|pw}}{\lemma{\textnormal{\emph{Eintreffen in Berlin}}}\Cendnote{\textnormal{Schnitzler war seit dem Vortag, dem 28. 12. 1901, in Berlin\oindex{Berlin@\textbf{Berlin}, \emph{P.PPLC}|pwk} und blieb bis zum 6. 1. 1902.}}}\label{K_L03098-1}
               wünſche ich Dir alles gute Glück.\pend
           
\pstart
           Bitte, ſchreib’ mir gleich (Adreſſe: \textsc{Hotel Central\oindex{Central-Hotel@\textbf{Central-Hotel}, \emph{Hotel (K.HTL)}|pw}}, \textsc{Bethmannstraße\oindex{Bethmannstrasse@\textbf{Bethmannstraße}, \emph{Straße (K.STR)}|pw}}), wie es auf den \label{K_L03098-2v}\edtext{Proben\pwindex{Lebendige Stunden. Vier Einakter@\emph{Lebendige Stunden. Vier Einakter}|pw}}{\lemma{\textnormal{\emph{Proben}}}\Cendnote{\textnormal{Siehe A. S.: \emph{Tagebuch}, 28. 12. 1901, 3. 1. 1902 und Paul Goldmann an Arthur Schnitzler, 31. 12. [1901].
               }}}\label{K_L03098-2} geht.\pend
           
\pstart
           Ich werde \label{K_L03098-3v}\edtext{Samſtag{ }früh von hier wegfahren, um zu Deiner \textsc{Première\pwindex{Lebendige Stunden. Vier Einakter@\emph{Lebendige Stunden. Vier Einakter}|pw}} in \textsc{Berlin\oindex{Berlin@\textbf{Berlin}, \emph{P.PPLC}|pw}}}{\lemma{\textnormal{\emph{Samſtag … Berlin}}}\Cendnote{\textnormal{Am Samstag, dem 4. 1. 1902 fand am
                     Deutschen Theater Berlin\oindex{Deutsches Theater Berlin@\textbf{Deutsches Theater Berlin}, \emph{Theater (K.THE)}|pwk} die Uraufführung der
                  vier Einakter \emph{Lebendige Stunden}\pwindex{Lebendige Stunden. Vier Einakter@\emph{Lebendige Stunden. Vier Einakter}|pwk} statt.}}}\label{K_L03098-3}
               zu ſein.\pend
           
\pstart
           Bitte, ſorge dafür, daß ich in meiner Wohnung ein Billet vorfinde.\pend
           
\pstart
           Meine Mutter\pwindex{Goldmann, Clementine 1842-05-15 – 1924-02-24@\textsc{Goldmann, Clementine} (1842-05-15 – 1924-02-24)|pwv} (die Dich
               grüßen läßt) iſt auch in Frankfurt\oindex{Frankfurt am Main@\textbf{Frankfurt am Main}, \emph{P.PPLA3}|pw}.\pend
           
\pstart
           Es thut mir unendlich leid, daß Deine \strikeout{An} Anweſenheit
               in Berlin\oindex{Berlin@\textbf{Berlin}, \emph{P.PPLC}|pw} gerade in die Zeit meiner Abweſenheit
               fällt.\pend
           
\pstart
           Viele treue Grüße! Dein {\\[\baselineskip]}\spacefill\mbox{Paul Goldmn}\pend
           \leftskip=0em{}\selectlanguage{ngerman}\endnumbering\briefempfaengerindex{Schnitzler, Arthur@\textsc{Schnitzler, Arthur}!zzzGoldmann, Paul@\emph{von Paul Goldmann}!1901-12-291@{29. 12. {[}1901{]}}|)be}\mylabel{L03098h}  \normalsize

\doendnotes{C}
\bigskip
\vfill

\clearpage

\footnotesize

\lohead{\textsc{register}}

% Definiere theindex-Environment komplett neu ohne reledmac
\makeatletter
\renewenvironment{theindex}{%
  \section*{\indexname}%
  \setlength{\parindent}{0pt}%
  \setlength{\parskip}{0pt plus 0.3pt}%
  \let\item\@idxitem
}{%
  \clearpage
}
\makeatother

\IfFileExists{\jobname-pw.ind}{\input{\jobname-pw.ind}}{}

\end{document}

      