%% latex-korrekturansicht-vorspann.tex
%% Vorspann für die Korrekturansicht.
%% Lädt die gemeinsame Datei latex-vorspann.tex mit gesetztem Schalter.

\newif\ifkorrekturansicht
\korrekturansichttrue

\input{../tex-inputs/latex-vorspann}


\section[Arthur Schnitzler an Hugo von Hofmannsthal, {[}9. 3. 1894{]}]{L00305 Arthur Schnitzler an Hugo von Hofmannsthal, {[}9. 3. 1894{]}}
\nopagebreak\mylabel{L00305v}
\rehead{ }\normalsize\beginnumbering\briefempfaengerindex{Hofmannsthal, Hugo von@\textsc{Hofmannsthal, Hugo von}!zzzSchnitzler, Arthur@\emph{von Arthur Schnitzler}!1894-03-092@{{[}9. 3. 1894{]}}|(be}
\toendnotes[C]{\smallbreak\pagebreak[2]}\Standort{FDH, Hs-30885,42.}
\physDesc{Brief, 1 Blatt, 4 Seiten, 830 Zeichen (Briefpapier mit Trauerrand)
\newline{}Handschrift: Bleistift, deutsche Kurrent
\newline{}Ordnung: mit Bleistift von Schnitzler mutmaßlich bei der Durchsicht der Briefe
                                    1929 datiert: »93« }
\buchAbdrucke{\weitereDrucke{Hugo von Hofmannsthal, Arthur Schnitzler: \emph{Briefwechsel}. Frankfurt am Main: \emph{S. Fischer} 1964, S. 51.} }\toendnotes[C]{\smallbreak}
\pstart
           \raggedleft{}{\pb}\uline{Freitag.}\pend
           \vspace{0.5em}
\pstart
           Liebſter Hugo,{ }So{\geminationn}tag iſt
               nichts bei mir. Vielleicht ko{\geminationm}’ ich um 8,
                  ½ 9 zu \textsc{Karlweis\pwindex{Karlweis, Carl 23.11.1850 – 27.10.1901@\textsc{Karlweis, Carl} (23.11.1850 – 27.10.1901), \emph{Schriftsteller/Schriftstellerin}|pw}}; Sie auch? –\pend
           
\pstart
           Bitte ſehr \label{K_L00305-1v}\edtext{ſchicken Sie doch an Goldmann\pwindex{Goldmann, Paul 31.01.1865 – 25.09.1935@\textsc{Goldmann, Paul} (31.01.1865 – 25.09.1935), \emph{Schriftsteller/Schriftstellerin, Journalist/Journalistin}|pw}}{\lemma{\textnormal{\emph{ſchicken … Goldmann}}}\Cendnote{\textnormal{Siehe Paul Goldmann an Arthur Schnitzler, 28. 2. [1894], der diesen Brief
                  motiviert haben dürfte; vgl. A. S.: \emph{Tagebuch}, 5. 3. 1894.
               }}}\label{K_L00305-1}{ }\textsc{75 rue Richelieu\oindex{rue Richelieu@\textbf{rue Richelieu}, \emph{Straße (K.STR)}|pw}} Ihre Sachen. Er ſchreibt mir ſo oft drum. »Tizian\pwindex{Tod des Tizian. Ein Bruchstueck@\emph{Der Tod des Tizian. Ein Bruchstück}|pw}« und »Thor u Tod\pwindex{Thor und der Tod@\emph{Der Thor und der Tod}|pw}«
               wenigſtens.\pend
           
\pstart
           {\pb}Von \textsc{Albert\pwindex{Albert, Henri 1869-11-16 – 1921-08-03@\textsc{Albert, Henri} (1869-11-16 – 1921-08-03), \emph{Journalist/Journalistin, Kritiker/Kritikerin, Übersetzer/Übersetzerin}|pw}} iſt in der \textsc{Nou\textcolor{gray}{v} Revue}\orgindex{Nouvelle Revue@Nouvelle Revue|pw} eine \label{K_L00305-2v}\edtext{Beſprechg\pwindex{Le nouvel almanach de M. Bierbaum@\emph{Le nouvel almanach de M. Bierbaum}|pwv}}{\lemma{\textnormal{\emph{Beſprechg}}}\Cendnote{\textnormal{Die Besprechung \emph{Le nouvel almanach de M. Bierbaum}\pwindex{Le nouvel almanach de M. Bierbaum@\emph{Le nouvel almanach de M. Bierbaum}|pwk} erschien am
                     1. 3. 1894 im \emph{Mercure de
                        France}\pwindex{Mercure de France@\emph{Mercure de France}|pwk} (S. 243–246).}}}\label{K_L00305-2} des \textsc{Musenalmanachs\pwindex{Moderner Musen-Almanach auf das Jahr 1894. Ein Jahrbuch deutscher Kunst@\emph{Moderner Musen-Almanach auf das Jahr 1894. Ein Jahrbuch deutscher Kunst}|pw}}, in dem Sie u ich mit
               ſehr viel Liebe behandelt ſind. (\label{K_L00305-3v}\edtext{\textsc{Le génial Loris etc.}\pwindex{Le nouvel almanach de M. Bierbaum@\emph{Le nouvel almanach de M. Bierbaum}|pwv}}{\lemma{\textnormal{\emph{Le génial Loris etc.}}}\Cendnote{\textnormal{Die betreffende Stelle findet sich auf S. 245.
               }}}\label{K_L00305-3}). Vielleicht ſchreiben Sie dem Mann auch 2 Zeilen (\textsc{Henri Albert\pwindex{Albert, Henri 1869-11-16 – 1921-08-03@\textsc{Albert, Henri} (1869-11-16 – 1921-08-03), \emph{Journalist/Journalistin, Kritiker/Kritikerin, Übersetzer/Übersetzerin}|pw}, 25 rue Jacob\oindex{rue Jacob@\textbf{rue Jacob}, \emph{Straße (K.STR)}|pw}.})\pend
           
\pstart
           {\pb}– Bei dieſer Gelegenheit eri{\geminationn}er’ ich Sie an Ihre Verſprechung mir Ihre Gedichte zu überſenden.\pend
           
\pstart
           – Haben Sie Nachricht von Richard\pwindex{Beer-Hofmann, Richard 1866-07-11 – 1945-09-26@\textsc{Beer-Hofmann, Richard} (1866-07-11 – 1945-09-26), \emph{Schriftsteller/Schriftstellerin}|pw}? Ich nur
               eine Correſp-Karte mit Adreſſe. –\pend
           
\pstart
           Sind Sie vielleicht Samſtag{ }Abend im {\pb}\textsc{Central\oindex{Cafe Central@\textbf{Café Central}, \emph{Kaffeehaus (K.KAF)}|pw}}, ich meine, nach zehn? –\pend
           
\pstart
           Wann gehn wir ins Arſenal\oindex{Arsenal@\textbf{Arsenal}, \emph{Gebäude (K.GBD)}|pw}? –\pend
           
\pstart
           Und, überhaupt, wann ſehn wir uns wieder? Daſs uns nur \textsc{Trio}’s zuſa{\geminationm}enführen, iſt eigentlich komiſch.\pend
           
\pstart
           Herzlich der Ihre{\\[\baselineskip]}\spacefill\mbox{Arthur.}\pend
           \leftskip=0em{}\selectlanguage{ngerman}\endnumbering\briefempfaengerindex{Hofmannsthal, Hugo von@\textsc{Hofmannsthal, Hugo von}!zzzSchnitzler, Arthur@\emph{von Arthur Schnitzler}!1894-03-092@{{[}9. 3. 1894{]}}|)be}\mylabel{L00305h}  \normalsize

\doendnotes{C}
\bigskip
\vfill

\clearpage

\footnotesize

\lohead{\textsc{register}}

% Definiere theindex-Environment komplett neu ohne reledmac
\makeatletter
\renewenvironment{theindex}{%
  \section*{\indexname}%
  \setlength{\parindent}{0pt}%
  \setlength{\parskip}{0pt plus 0.3pt}%
  \let\item\@idxitem
}{%
  \clearpage
}
\makeatother

\IfFileExists{\jobname-pw.ind}{\input{\jobname-pw.ind}}{}

\end{document}

      