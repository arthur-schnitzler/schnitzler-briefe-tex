%% latex-leseansicht-vorspann.tex
%% Vorspann für die Leseansicht.
%% Lädt die gemeinsame Datei latex-vorspann.tex mit nicht gesetztem Schalter.

\newif\ifkorrekturansicht
\korrekturansichtfalse

\input{../tex-inputs/latex-vorspann}


\section[Arthur Schnitzler an Hugo von Hofmannsthal, {{[}}9. 3. 1894{{]}}]{L00305 Arthur Schnitzler an Hugo von Hofmannsthal, {[}9. 3. 1894{]}}
\nopagebreak\mylabel{L00305v}
\rehead{ }\normalsize\beginnumbering\briefempfaengerindex{Hofmannsthal, Hugo von@\textsc{Hofmannsthal, Hugo von}!zzzSchnitzler, Arthur@\emph{von Arthur Schnitzler}!1894-03-092@{{[}9. 3. 1894{]}}|(be}
\toendnotes[C]{\smallbreak\pagebreak[2]}
\correspDesc{Versand  durch Arthur Schnitzler am [9. 3. 1894] in Wien
\newline{}Erhalt  durch Hugo von Hofmannsthal im Zeitraum [9. 3. 1894
                  – 13. 3. 1894?] in Wien}\toendnotes[C]{\smallbreak}
\Standort{FDH, Hs-30885,42.}
\physDesc{Brief, 1 Blatt, 4 Seiten, 830 Zeichen (Briefpapier mit Trauerrand)
\newline{}Handschrift: Bleistift, deutsche Kurrent
\newline{}Ordnung: mit Bleistift von Schnitzler mutmaßlich bei der Durchsicht der Briefe
                                    1929 datiert: »93« }
\buchAbdrucke{\weitereDrucke{Hugo von Hofmannsthal, Arthur Schnitzler: \emph{Briefwechsel}. Herausgegeben von Therese Nickl und Heinrich Schnitzler. Frankfurt am Main: \emph{S. Fischer} 1964, S. 51.} }\toendnotes[C]{\smallbreak}
\pstart
           \raggedleft{}{\pb}\uline{Freitag.}\pend
           \vspace{0.5em}
\pstart
           Liebſter Hugo,{ }So{\geminationn}tag iſt
               nichts bei mir. Vielleicht ko{\geminationm}’ ich um 8,
                  ½ 9 zu \textsc{Karlweis\pwindex{Karlweis, Carl 23.\,11.\,1850 Wien – 27.\,10.\,1901 ebd.@\textsc{Karlweis, Carl} (23.\,11.\,1850 Wien – 27.\,10.\,1901 ebd.), \emph{Schriftsteller}|pw}}; Sie auch? –\pend
           
\pstart
           Bitte{ }ſehr \label{K_L00305-1v}\edtext{ſchicken Sie doch an Goldmann\pwindex{Goldmann, Paul 31.\,1.\,1865 Breslau – 25.\,9.\,1935 Wien@\textsc{Goldmann, Paul} (31.\,1.\,1865 Breslau – 25.\,9.\,1935 Wien), \emph{Schriftsteller, Journalist}|pw}}{\lemma{\textnormal{\emph{schicken … Goldmann}}}\Cendnote{\textnormal{Siehe XXXX Auszeichnungsfehler: Dokument L02611 nicht gefunden, der diesen Brief
                  motiviert haben dürfte; vgl. A. S.: \emph{Tagebuch}, 5. 3. 1894.
               }}}\label{K_L00305-1}{ }\textsc{75 rue Richelieu\oindex{rue Richelieu@\textbf{rue Richelieu}, \emph{Straße}|pw}} Ihre Sachen. Er{ }ſchreibt mir{ }ſo oft drum. »Tizian\pwindex{Hofmannsthal, Hugo von 1.\,2.\,1874 Wien – 15.\,7.\,1929 Rodaun@\textsc{Hofmannsthal, Hugo von} (1.\,2.\,1874 Wien – 15.\,7.\,1929 Rodaun), \emph{Schriftsteller}!Tod des Tizian. Ein Bruchstück@\strich\emph{Der Tod des Tizian. Ein Bruchstück}|pw}« und »Thor u Tod\pwindex{Hofmannsthal, Hugo von 1.\,2.\,1874 Wien – 15.\,7.\,1929 Rodaun@\textsc{Hofmannsthal, Hugo von} (1.\,2.\,1874 Wien – 15.\,7.\,1929 Rodaun), \emph{Schriftsteller}!Thor und der Tod@\strich\emph{Der Thor und der Tod}|pw}«
               wenigſtens.\pend
           
\pstart
           {\pb}Von \textsc{Albert\pwindex{Albert, Henri 16.\,11.\,1869 Niederbronn-les-Bains – 3.\,8.\,1921 Straßburg@\textsc{Albert, Henri} (16.\,11.\,1869 Niederbronn-les-Bains – 3.\,8.\,1921 Straßburg), \emph{Journalist, Kritiker, Übersetzer}|pw}} iſt in der \textsc{Nou\textcolor{gray}{v} Revue}\orgindex{Nouvelle Revue@Nouvelle Revue|pw} eine \label{K_L00305-2v}\edtext{Beſprechg\pwindex{Albert, Henri 16.\,11.\,1869 Niederbronn-les-Bains – 3.\,8.\,1921 Straßburg@\textsc{Albert, Henri} (16.\,11.\,1869 Niederbronn-les-Bains – 3.\,8.\,1921 Straßburg), \emph{Journalist, Kritiker, Übersetzer}!Le nouvel almanach de M. Bierbaum@\strich\emph{Le nouvel almanach de M. Bierbaum}|pwv}}{\lemma{\textnormal{\emph{Besprechg}}}\Cendnote{\textnormal{Die Besprechung \emph{Le nouvel almanach de M. Bierbaum}\pwindex{Albert, Henri 16.\,11.\,1869 Niederbronn-les-Bains – 3.\,8.\,1921 Straßburg@\textsc{Albert, Henri} (16.\,11.\,1869 Niederbronn-les-Bains – 3.\,8.\,1921 Straßburg), \emph{Journalist, Kritiker, Übersetzer}!Le nouvel almanach de M. Bierbaum@\strich\emph{Le nouvel almanach de M. Bierbaum}|pwk} erschien am
                     1. 3. 1894 im \emph{Mercure de
                        France}\pwindex{Mercure de France@\emph{Mercure de France}|pwk} (S. 243–246).}}}\label{K_L00305-2} des \textsc{Musenalmanachs\pwindex{Moderner Musen-Almanach auf das Jahr 1894. Ein Jahrbuch deutscher Kunst@\emph{Moderner Musen-Almanach auf das Jahr 1894. Ein Jahrbuch deutscher Kunst}|pw}}, in dem Sie u ich mit{ }ſehr viel Liebe behandelt{ }ſind. (\label{K_L00305-3v}\edtext{\textsc{Le génial Loris etc.}\pwindex{Albert, Henri 16.\,11.\,1869 Niederbronn-les-Bains – 3.\,8.\,1921 Straßburg@\textsc{Albert, Henri} (16.\,11.\,1869 Niederbronn-les-Bains – 3.\,8.\,1921 Straßburg), \emph{Journalist, Kritiker, Übersetzer}!Le nouvel almanach de M. Bierbaum@\strich\emph{Le nouvel almanach de M. Bierbaum}|pwv}}{\lemma{\textnormal{\emph{Le génial Loris etc.}}}\Cendnote{\textnormal{Die betreffende Stelle findet sich auf S. 245.
               }}}\label{K_L00305-3}). Vielleicht{ }ſchreiben Sie dem Mann auch 2 Zeilen (\textsc{Henri Albert\pwindex{Albert, Henri 16.\,11.\,1869 Niederbronn-les-Bains – 3.\,8.\,1921 Straßburg@\textsc{Albert, Henri} (16.\,11.\,1869 Niederbronn-les-Bains – 3.\,8.\,1921 Straßburg), \emph{Journalist, Kritiker, Übersetzer}|pw}, 25 rue Jacob\oindex{rue Jacob@\textbf{rue Jacob}, \emph{Straße}|pw}.})\pend
           
\pstart
           {\pb}– Bei dieſer Gelegenheit eri{\geminationn}er’ ich Sie an Ihre Verſprechung mir Ihre Gedichte zu überſenden.\pend
           
\pstart
           – Haben Sie Nachricht von Richard\pwindex{Beer-Hofmann, Richard 11.\,7.\,1866 Wien – 26.\,9.\,1945 New York City@\textsc{Beer-Hofmann, Richard} (11.\,7.\,1866 Wien – 26.\,9.\,1945 New York City), \emph{Schriftsteller}|pw}? Ich nur
               eine Correſp-Karte mit Adreſſe. –\pend
           
\pstart
           Sind Sie vielleicht Samſtag{ }Abend im {\pb}\textsc{Central\oindex{Wien@\textbf{Wien}!I., Innere Stadt@\textbf{I., Innere Stadt}!Café Central@\textbf{Café Central}, \emph{Kaffeehaus}|pw}}, ich meine, nach zehn? –\pend
           
\pstart
           Wann gehn wir ins Arſenal\oindex{Wien@\textbf{Wien}!III., Landstraße@\textbf{III., Landstraße}!Arsenal@\textbf{Arsenal}, \emph{Gebäude}|pw}? –\pend
           
\pstart
           Und, überhaupt, wann{ }ſehn wir uns wieder? Daſs uns nur \textsc{Trio}’s zuſa{\geminationm}enführen, iſt eigentlich komiſch.\pend
           
\pstart
           Herzlich der Ihre{\\[\baselineskip]}\spacefill\mbox{Arthur.}\pend
           \leftskip=0em{}\selectlanguage{ngerman}\endnumbering\briefempfaengerindex{Hofmannsthal, Hugo von@\textsc{Hofmannsthal, Hugo von}!zzzSchnitzler, Arthur@\emph{von Arthur Schnitzler}!1894-03-092@{{[}9. 3. 1894{]}}|)be}\mylabel{L00305h}  \newcommand{\dateiname}{L00305}\newcommand{\titel}{Arthur Schnitzler an Hugo von Hofmannsthal, [9. 3. 1894]}\newcommand{\editorInnen}{Martin Anton Müller und Gerd-Hermann Susen}%% latex-leseansicht-abspann.tex
%% Abspann für die Leseansicht.
%% Der Schalter \ifkorrekturansicht ist bereits durch den Vorspann gesetzt.

%% latex-abspann.tex
%% Gemeinsamer Abspann für Korrekturansicht und Leseansicht.
%% Setzt den Schalter \ifkorrekturansicht voraus (gesetzt in den
%% einbindenden Dateien latex-korrekturansicht-abspann.tex bzw.
%% latex-leseansicht-abspann.tex).
%% ---------------------------------------------------------------

\normalsize

% Das esempio-Environment wird nur in der Leseansicht benötigt
\ifkorrekturansicht\else
\newenvironment{esempio}[3]%
{
    \vspace{1.5ex}
    \rlap{\underline{#1}}
    \par
    \setlength{\parindent}{0cm}
    \nopagebreak
    \leftskip=#2cm
    \rightskip=#3cm
}
{
    \par
}
\fi

\doendnotes{C}
\bigskip
\vfill

\clearpage

\footnotesize

\ifkorrekturansicht
  \lohead{\textsc{register}}
\fi

% theindex-Environment neu definieren ohne reledmac
\makeatletter
\renewenvironment{theindex}{%
  \ifkorrekturansicht
    \section*{\indexname}%
  \else
    \subsubsection*{Index der erwähnten Entitäten}%
  \fi
  \setlength{\parindent}{0pt}%
  \setlength{\parskip}{0pt plus 0.3pt}%
  \let\item\@idxitem
}{%
  \ifkorrekturansicht\clearpage\fi
}
\makeatother

\IfFileExists{\jobname-pw.ind}{\input{\jobname-pw.ind}}{}

% Quellenangabe nur in der Leseansicht
\ifkorrekturansicht\else
% Fallback-Definitionen, falls die .tex-Datei \titel etc. nicht gesetzt hat
\providecommand{\titel}{}
\providecommand{\editorInnen}{}
\providecommand{\dateiname}{\jobname}

\vspace{3cm}

\vfill

\footnotesize
\textsc{Quelle}: \titel. Herausgegeben von {\editorInnen}. In: \emph{Arthur Schnitzler: Briefwechsel mit Autorinnen und Autoren}.
 Digitale Edition, https://schnitzler-briefe.acdh.oeaw.ac.at/{\dateiname}.html (Stand \today)
\fi

\end{document}


