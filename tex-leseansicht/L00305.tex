%% latex-leseansicht-vorspann.tex
%% Vorspann für die Leseansicht.
%% Lädt die gemeinsame Datei latex-vorspann.tex mit nicht gesetztem Schalter.

\newif\ifkorrekturansicht
\korrekturansichtfalse

\input{../tex-inputs/latex-vorspann}


         
         \renewcommand{\erwaehntePersonen}{Personen: Henri Albert, Richard Beer-Hofmann, Paul Goldmann, Hugo von Hofmannsthal, Carl Karlweis}
         \renewcommand{\erwaehnteInstitutionen}{Institutionen: Nouvelle Revue}
         \renewcommand{\erwaehnteOrte}{Orte: Arsenal, Café Central, Wien, rue Jacob, rue Richelieu}
         \renewcommand{\erwaehnteWerke}{Werke: Der Thor und der Tod, Der Tod des Tizian, Le nouvel almanach de M. Bierbaum, Mercure de France, Moderner Musen-Almanach auf das Jahr 1894. Ein Jahrbuch deutscher Kunst}
               \section[Arthur Schnitzler an Hugo von Hofmannsthal, {[}9. 3. 1894{]}]{ Arthur Schnitzler an Hugo von Hofmannsthal, {[}9. 3. 1894{]}}\nopagebreak\mylabel{v}\rehead{ }\begin{ledgroupsized}[t]{13cm}\normalsize\beginnumbering \toendnotes[C]{\smallbreak\pagebreak[2]} \Standort{FDH, Hs-30885,42.}
\physDesc{Brief, 1 Blatt, 4 Seiten (Briefpapier mit Trauerrand)
\newline{}Handschrift: Bleistift, deutsche Kurrent\newline{}Ordnung: von Schnitzler mutmaßlich bei der Durchsicht der Briefe
                                    1929 mit Bleistift datiert: »93« }\buchAbdrucke{\weitereDrucke{Hugo von Hofmannsthal, Arthur Schnitzler: \emph{Briefwechsel}. Hg. Therese Nickl und Heinrich Schnitzler. Frankfurt am Main: \emph{S. Fischer} 1964, S. 51.} }\toendnotes[C]{\smallbreak}\pstart
           \raggedleft{}{\pb}\uline{Freitag.}\pend
           \pstart
           Liebſter Hugo, So{\geminationn}tag iſt
               nichts bei mir. Vielleicht ko{\geminationm}’ ich um 8,
                  ½ 9 zu \textsc{Karlweis\pwindex{Karlweis, Carl 23.11.1850 – 27.10.1901@\textsc{Karlweis, Carl} (23.11.1850 – 27.10.1901), \emph{Schriftsteller}|pw}}; Sie auch? –\pend
           \pstart
           Bitte ſehr \label{K_L00305-2v}\edtext{ſchicken Sie doch an Goldmann\pwindex{Goldmann, Paul 31.01.1865 – 25.09.1935@\textsc{Goldmann, Paul} (31.01.1865 – 25.09.1935), \emph{Schriftsteller, Journalist}|pw}}{\lemma{\textnormal{\emph{ſchicken … Goldmann}}}\Cendnote{\textnormal{siehe Paul Goldmann an Arthur Schnitzler, 28. 2. [1894], der diesen
                     Brief motiviert haben dürfte; Vgl. A. S.: \emph{Tagebuch}, 5. 3. 1894}}}\label{K_L00305-2h}{ }\textsc{75 rue Richelieu\oindex{rue Richelieu@\textbf{rue Richelieu}|pw}} Ihre Sachen. Er ſchreibt mir ſo oft drum. »Tizian\pwindex{Hofmannsthal, Hugo von 1874-02-01 – 1929-07-15@\textsc{Hofmannsthal, Hugo von} (1874-02-01 – 1929-07-15), \emph{Schriftsteller}!Tod des TizianOktober 1892@\strich\emph{Der Tod des Tizian} {[}Oktober 1892{]}|pw}« und »Thor u Tod\pwindex{Hofmannsthal, Hugo von 1874-02-01 – 1929-07-15@\textsc{Hofmannsthal, Hugo von} (1874-02-01 – 1929-07-15), \emph{Schriftsteller}!Thor und der Tod1893@\strich\emph{Der Thor und der Tod} {[}1893{]}|pw}«
               wenigſtens.\pend
           \pstart
           {\pb}Von \textsc{Albert\pwindex{Albert, Henri 1869-11-16 – 1921-08-03@\textsc{Albert, Henri} (1869-11-16 – 1921-08-03), \emph{Journalist, Kritiker, Übersetzer}|pw}} iſt in der \textsc{Nou\textcolor{gray}{v} Revue}\orgindex{Nouvelle Revue@Nouvelle Revue|pw} eine \label{K_L00305_1v}\edtext{Beſprechg\pwindex{Albert, Henri 1869-11-16 – 1921-08-03@\textsc{Albert, Henri} (1869-11-16 – 1921-08-03), \emph{Journalist, Kritiker, Übersetzer}!Le nouvel almanach de M. Bierbaum1. 3. 1894@\strich\emph{Le nouvel almanach de M. Bierbaum} {[}1. 3. 1894{]}|pwv}}{\lemma{\textnormal{\emph{Beſprechg}}}\Cendnote{\textnormal{Die Besprechung \emph{Le nouvel almanach de M. Bierbaum}\pwindex{Albert, Henri 1869-11-16 – 1921-08-03@\textsc{Albert, Henri} (1869-11-16 – 1921-08-03), \emph{Journalist, Kritiker, Übersetzer}!Le nouvel almanach de M. Bierbaum1. 3. 1894@\strich\emph{Le nouvel almanach de M. Bierbaum} {[}1. 3. 1894{]}|pwk} erschien am
                     1. 3. 1894 im \emph{Mercure de
                     France}\pwindex{?? Werk@Nicht ermittelte Verfasserinnen und Verfasser!Mercure de France1890 – 1965@\emph{Mercure de France} {[}1890 – 1965{]}|pwk} (S. 243–246).}}}\label{K_L00305_1h} des \textsc{Musenalmanach\pwindex{Moderner Musen-Almanach auf das Jahr 1894. Ein Jahrbuch deutscher Kunst1893@\emph{Moderner Musen-Almanach auf das Jahr 1894. Ein Jahrbuch deutscher Kunst} {[}1893{]}|pw}s}, in dem Sie u ich mit
               ſehr viel Liebe behandelt ſind. (\label{K_L00305_2v}\edtext{\textsc{Le génial Loris etc.}\pwindex{Albert, Henri 1869-11-16 – 1921-08-03@\textsc{Albert, Henri} (1869-11-16 – 1921-08-03), \emph{Journalist, Kritiker, Übersetzer}!Le nouvel almanach de M. Bierbaum1. 3. 1894@\strich\emph{Le nouvel almanach de M. Bierbaum} {[}1. 3. 1894{]}|pwv}}{\lemma{\textnormal{\emph{Le génial Loris etc.}}}\Cendnote{\textnormal{auf S. 245}}}\label{K_L00305_2h}). Vielleicht ſchreiben Sie dem Mann auch 2 Zeilen (\textsc{Henri Albert\pwindex{Albert, Henri 1869-11-16 – 1921-08-03@\textsc{Albert, Henri} (1869-11-16 – 1921-08-03), \emph{Journalist, Kritiker, Übersetzer}|pw}, 25 rue Jacob\oindex{rue Jacob@\textbf{rue Jacob}|pw}.})\pend
           \pstart
           {\pb}– Bei dieſer Gelegenheit eri{\geminationn}er’ ich Sie an Ihre Verſprechung mir Ihre Gedichte zu überſenden.\pend
           \pstart
           – Haben Sie Nachricht von Richard\pwindex{Beer-Hofmann, Richard 1866-07-11 – 1945-09-26@\textsc{Beer-Hofmann, Richard} (1866-07-11 – 1945-09-26), \emph{Schriftsteller}|pw}? Ich nur
               eine Correſp-Karte mit Adreſſe. –\pend
           \pstart
           Sind Sie vielleicht Samſtag{ }Abend im {\pb}\textsc{Central\oindex{Cafe Central@\textbf{Café Central}|pw}}, ich meine, nach zehn? –\pend
           \pstart
           Wann gehn wir ins Arſenal\oindex{Arsenal@\textbf{Arsenal}|pw}? –\pend
           \pstart
           Und, überhaupt, wann ſehn wir uns wieder? Daſs uns nur \textsc{Trio}’s zuſa{\geminationm}enführen, iſt eigentlich komiſch.\pend
           \pstart
           Herzlich der Ihre{\\[\baselineskip]}\spacefill\mbox{Arthur.}\pend
           \leftskip=0em{}
         
         \endnumbering\mylabel{h}\end{ledgroupsized}  \newcommand{\dateiname}{L00305}\newcommand{\titel}{Arthur Schnitzler an Hugo von Hofmannsthal, [9. 3. 1894]}\newcommand{\editorInnen}{Martin Anton Müller und Gerd-Hermann Susen}%% latex-leseansicht-abspann.tex
%% Abspann für die Leseansicht.
%% Der Schalter \ifkorrekturansicht ist bereits durch den Vorspann gesetzt.

%% latex-abspann.tex
%% Gemeinsamer Abspann für Korrekturansicht und Leseansicht.
%% Setzt den Schalter \ifkorrekturansicht voraus (gesetzt in den
%% einbindenden Dateien latex-korrekturansicht-abspann.tex bzw.
%% latex-leseansicht-abspann.tex).
%% ---------------------------------------------------------------

\normalsize

% Das esempio-Environment wird nur in der Leseansicht benötigt
\ifkorrekturansicht\else
\newenvironment{esempio}[3]%
{
    \vspace{1.5ex}
    \rlap{\underline{#1}}
    \par
    \setlength{\parindent}{0cm}
    \nopagebreak
    \leftskip=#2cm
    \rightskip=#3cm
}
{
    \par
}
\fi

\doendnotes{C}
\bigskip
\vfill

\clearpage

\footnotesize

\ifkorrekturansicht
  \lohead{\textsc{register}}
\fi

% theindex-Environment neu definieren ohne reledmac
\makeatletter
\renewenvironment{theindex}{%
  \ifkorrekturansicht
    \section*{\indexname}%
  \else
    \subsubsection*{Index der erwähnten Entitäten}%
  \fi
  \setlength{\parindent}{0pt}%
  \setlength{\parskip}{0pt plus 0.3pt}%
  \let\item\@idxitem
}{%
  \ifkorrekturansicht\clearpage\fi
}
\makeatother

\IfFileExists{\jobname-pw.ind}{\input{\jobname-pw.ind}}{}

% Quellenangabe nur in der Leseansicht
\ifkorrekturansicht\else
% Fallback-Definitionen, falls die .tex-Datei \titel etc. nicht gesetzt hat
\providecommand{\titel}{}
\providecommand{\editorInnen}{}
\providecommand{\dateiname}{\jobname}

\vspace{3cm}

\vfill

\footnotesize
\textsc{Quelle}: \titel. Herausgegeben von {\editorInnen}. In: \emph{Arthur Schnitzler: Briefwechsel mit Autorinnen und Autoren}.
 Digitale Edition, https://schnitzler-briefe.acdh.oeaw.ac.at/{\dateiname}.html (Stand \today)
\fi

\end{document}


      