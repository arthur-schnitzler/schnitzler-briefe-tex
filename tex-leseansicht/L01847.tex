%% latex-korrekturansicht-vorspann.tex
%% Vorspann für die Korrekturansicht.
%% Lädt die gemeinsame Datei latex-vorspann.tex mit gesetztem Schalter.

\newif\ifkorrekturansicht
\korrekturansichttrue

\input{../tex-inputs/latex-vorspann}


\section[Max Burckhard: Widmungsexemplar Gottfried Wunderlich für Arthur Schnitzler, 20. 6. 1909]{L01847 Max Burckhard: Widmungsexemplar Gottfried Wunderlich für Arthur
               Schnitzler, 20. 6. 1909}
\nopagebreak\mylabel{L01847v}
\rehead{ }\normalsize\beginnumbering\briefempfaengerindex{Schnitzler, Arthur@\textsc{Schnitzler, Arthur}!zzzBurckhard, Max Eugen@\emph{von Max Eugen Burckhard}!1909-06-201@{20. 6. 1909}|(be}
\toendnotes[C]{\smallbreak\pagebreak[2]}\Standort{DLA, G:Schnitzler, Arthur (Sammlung Heinrich Schnitzler).}
\physDesc{Widmung am Vorsatzblatt, 84 Zeichen
\newline{}Handschrift: schwarze Tinte, deutsche Kurrent
\newline{}Ordnung: bei der Enteignung des Exemplars 1938 von
                                 unbekannter Hand mit Bleistift ergänzte Information:
                                    »Widm.« und zwei Stempel: \noindent{}\textcolor{gray}{\textbf{\textit{NATIONAL-BIBLIOTHEK\orgindex{Oesterreichische Nationalbibliothek@Österreichische Nationalbibliothek|pw}{ }WIEN\oindex{Wien@\textbf{Wien}, \emph{A.ADM2}|pw}}}}{ / }\textcolor{gray}{\textbf{\textit{682782-B}}}« }
\pstart
           \noindent{}{\pb}Arthur Schnitzler{\\}überreicht in neuem
               Gewande –\pend
           
\pstart
           in herzlicher Verehrung{\\[\baselineskip]}\spacefill\mbox{D\textsuperscript{r} B.}\pend
           \leftskip=0em{}
\pstart
           20. 6. 09\pend
           \selectlanguage{ngerman}\vspace{1em}{\vspace{1\baselineskip}}
\pstart
           \centering{}{\pb}\textcolor{gray}{\textbf{Gottfried Wunderlich\pwindex{Gottfried Wunderlich. Roman@\emph{Gottfried Wunderlich. Roman}|pw}}}\pend
           
\pstart
           \centering{}\textcolor{gray}{\textbf{Roman}}{\\}\textcolor{gray}{\textbf{von}}{\\}\textcolor{gray}{\textbf{Max Burckhard}}\pend
           
\pstart
           \centering{}\textcolor{gray}{\textbf{Dritte Auflage}}\pend
           {\vspace{1\baselineskip}}
\pstart
           \centering{}\textcolor{gray}{\textbf{S. Fiſcher, Verlag\orgindex{S. Fischer Verlag@S. Fischer Verlag|pw}, Berlin\oindex{Berlin@\textbf{Berlin}, \emph{P.PPLC}|pw}}}\pend
           
\pstart
           \centering{}\textcolor{gray}{\textbf{1909}}\pend
           \selectlanguage{ngerman}\endnumbering\briefempfaengerindex{Schnitzler, Arthur@\textsc{Schnitzler, Arthur}!zzzBurckhard, Max Eugen@\emph{von Max Eugen Burckhard}!1909-06-201@{20. 6. 1909}|)be}\mylabel{L01847h}  \normalsize

\doendnotes{C}
\bigskip
\vfill

\clearpage

\footnotesize

\lohead{\textsc{register}}

% Definiere theindex-Environment komplett neu ohne reledmac
\makeatletter
\renewenvironment{theindex}{%
  \section*{\indexname}%
  \setlength{\parindent}{0pt}%
  \setlength{\parskip}{0pt plus 0.3pt}%
  \let\item\@idxitem
}{%
  \clearpage
}
\makeatother

\IfFileExists{\jobname-pw.ind}{\input{\jobname-pw.ind}}{}

\end{document}

      