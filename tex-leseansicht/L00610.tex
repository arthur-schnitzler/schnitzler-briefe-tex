%% latex-leseansicht-vorspann.tex
%% Vorspann für die Leseansicht.
%% Lädt die gemeinsame Datei latex-vorspann.tex mit nicht gesetztem Schalter.

\newif\ifkorrekturansicht
\korrekturansichtfalse

\input{../tex-inputs/latex-vorspann}


         
         \renewcommand{\erwaehntePersonen}{Personen: Max Eugen Burckhard,  Louis Philippe Robert d’Orléans, duc d’Orléans,  Maria Dorothea von Österreich}
         \renewcommand{\erwaehnteOrte}{Orte: Berlin, Deutsches Theater Berlin, Schloss Schönbrunn, Wien}
         \renewcommand{\erwaehnteWerke}{Werke: Freiwild. Schauspiel in 3 Akten}
               \section[Max Burckhard an Arthur Schnitzler, {[}zwischen 27. 10. und 1. 11. 1896{]}]{ Max Burckhard an Arthur Schnitzler, {[}zwischen 27. 10. und
               1. 11. 1896{]}}\nopagebreak\mylabel{v}\rehead{ }\begin{ledgroupsized}[t]{13cm}\normalsize\beginnumbering \toendnotes[C]{\smallbreak\pagebreak[2]} \Standort{CUL, Schnitzler, B 20.}
\physDesc{Telegramm, 266 Zeichen
\newline{}maschinell
\newline{}Schnitzler: mit Bleistift datiert: »96?« 
\newline{}Ordnung: beschnitten }\toendnotes[C]{\smallbreak}\pstart
           \noindent{}{\pb}herzlichsten dank fuer mittheilung.
               leider habe ich dienstag{ }vormittag generalprobe in schoenbrunn\oindex{Schloss Schoenbrunn@\textbf{Schloss Schönbrunn}|pw} fuer die \label{K_L00610-1v}\edtext{festvorstellung}{\lemma{\textnormal{\emph{festvorstellung}}}\Cendnote{\textnormal{Diese fand am
                     4. 11. 1896, dem Vorabend der Hochzeit von Erzherzogin Marie Dorothea\pwindex{Maria Dorothea von Oesterreich 14.06.1867 – 06.04.1932@\textsc{Maria Dorothea von Österreich} (14.06.1867 – 06.04.1932), \emph{Erzherzogin}|pwk} mit Louis Philippe d’Orleans\pwindex{Louis Philippe Robert DOrleans, duc DOrleans 1869-02-06 – 1926-03-28@\textsc{Louis Philippe Robert d’Orléans, duc d’Orléans} (1869-02-06 – 1926-03-28), \emph{Thronprätendent}|pwk}, statt.}}}\label{K_L00610-1h} die ich unmoeglich
               stuerzen kann. – ich hatte mich \label{K_L00610-2v}\edtext{so
                  gefreut}{\lemma{\textnormal{\emph{so
                  gefreut}}}\Cendnote{\textnormal{Am 3. 11. 1896
                  fand in Berlin\oindex{Berlin@\textbf{Berlin}|pwk} am Deutschen Theater\oindex{Deutsches Theater Berlin@\textbf{Deutsches Theater Berlin}|pwk} die Uraufführung von \emph{Freiwild}\pwindex{Schnitzler, Arthur 15.05.1862 – 21.10.1931@\textsc{Schnitzler, Arthur} (15.05.1862 – 21.10.1931), \emph{Schriftsteller, Mediziner}!Freiwild. Schauspiel in 3 Akten1896@\strich\emph{Freiwild. Schauspiel in 3 Akten} {[}1896{]}|pwk} statt.}}}\label{K_L00610-2h}. – so geht es einem. – herzlichste gruesse und die
               besten wuensche. \spacefill\mbox{= doctor burckhard +}\pend
           
         
         \endnumbering\mylabel{h}\end{ledgroupsized}  \newcommand{\dateiname}{L00610}\newcommand{\titel}{Max Burckhard an Arthur Schnitzler, [zwischen 27. 10. und 1. 11. 1896]}\newcommand{\editorInnen}{Martin Anton Müller und Gerd-Hermann Susen}%% latex-leseansicht-abspann.tex
%% Abspann für die Leseansicht.
%% Der Schalter \ifkorrekturansicht ist bereits durch den Vorspann gesetzt.

%% latex-abspann.tex
%% Gemeinsamer Abspann für Korrekturansicht und Leseansicht.
%% Setzt den Schalter \ifkorrekturansicht voraus (gesetzt in den
%% einbindenden Dateien latex-korrekturansicht-abspann.tex bzw.
%% latex-leseansicht-abspann.tex).
%% ---------------------------------------------------------------

\normalsize

% Das esempio-Environment wird nur in der Leseansicht benötigt
\ifkorrekturansicht\else
\newenvironment{esempio}[3]%
{
    \vspace{1.5ex}
    \rlap{\underline{#1}}
    \par
    \setlength{\parindent}{0cm}
    \nopagebreak
    \leftskip=#2cm
    \rightskip=#3cm
}
{
    \par
}
\fi

\doendnotes{C}
\bigskip
\vfill

\clearpage

\footnotesize

\ifkorrekturansicht
  \lohead{\textsc{register}}
\fi

% theindex-Environment neu definieren ohne reledmac
\makeatletter
\renewenvironment{theindex}{%
  \ifkorrekturansicht
    \section*{\indexname}%
  \else
    \subsubsection*{Index der erwähnten Entitäten}%
  \fi
  \setlength{\parindent}{0pt}%
  \setlength{\parskip}{0pt plus 0.3pt}%
  \let\item\@idxitem
}{%
  \ifkorrekturansicht\clearpage\fi
}
\makeatother

\IfFileExists{\jobname-pw.ind}{\input{\jobname-pw.ind}}{}

% Quellenangabe nur in der Leseansicht
\ifkorrekturansicht\else
% Fallback-Definitionen, falls die .tex-Datei \titel etc. nicht gesetzt hat
\providecommand{\titel}{}
\providecommand{\editorInnen}{}
\providecommand{\dateiname}{\jobname}

\vspace{3cm}

\vfill

\footnotesize
\textsc{Quelle}: \titel. Herausgegeben von {\editorInnen}. In: \emph{Arthur Schnitzler: Briefwechsel mit Autorinnen und Autoren}.
 Digitale Edition, https://schnitzler-briefe.acdh.oeaw.ac.at/{\dateiname}.html (Stand \today)
\fi

\end{document}


      