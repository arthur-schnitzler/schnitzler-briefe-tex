%% latex-korrekturansicht-vorspann.tex
%% Vorspann für die Korrekturansicht.
%% Lädt die gemeinsame Datei latex-vorspann.tex mit gesetztem Schalter.

\newif\ifkorrekturansicht
\korrekturansichttrue

\input{../tex-inputs/latex-vorspann}


\section[Arthur Schnitzler an Joseph Victor Widmann, 8. 2. 1902]{L01201 Arthur Schnitzler an Joseph Victor Widmann, 8. 2. 1902}
\nopagebreak\mylabel{L01201v}
\rehead{ }\normalsize\beginnumbering\briefempfaengerindex{Widmann, Joseph Victor@\textsc{Widmann, Joseph Victor}!zzzSchnitzler, Arthur@\emph{von Arthur Schnitzler}!1902-02-081@{8. 2. 1902}|(be}
\toendnotes[C]{\smallbreak\pagebreak[2]}\Standort{Bern, Burgerbibliothek, N Joseph Viktor Widmann 29 (7).}
\physDesc{Briefkarte, 458 Zeichen
\newline{}Handschrift: schwarze Tinte, deutsche Kurrent
\newline{}Ordnung: 1) Lochung  2) von unbekannter Hand nummeriert: »2.« und
                                 beschriftet: »\textsc{A. Schnitzler} 1/2«}
\pstart{}{\pb}Verehrter Herr Widmann,\pend\vspace{0.5em}
\pstart
            es iſt ſehr liebenswürdig von Ihnen, mir die Nummer des B. B.\pwindex{Bund@\emph{Der Bund}|pw} mit der freundlichen Ankündg des \textsc{Paracelsus}\pwindex{Paracelsus. Versspiel in einem Akt@\emph{Paracelsus. Versspiel in einem Akt}|pw} zu ſenden. Wie gern möcht ich Ihnen einmal perſönlich Dank ſagen für die {\pb}wohlthuende Sympathie, mit d\damage{er} Sie alle meine mehr oder weniger gelungenen Verſuche von Anatol\pwindex{Anatol@\emph{Anatol}|pw} an begleiten; Ihnen auch ſagen, wie ſehr
               mich die Anerkennung von Seiten eines Dichters freut, den ich ſo hoch verehre.\pend
           
\pstart
           Ihr wahrlichſt{\\[\baselineskip]}ergebener \spacefill\mbox{Arthur Schnitzler}\pend
           \leftskip=0em{}
\pstart
           8. 2. 902.\pend
           \selectlanguage{ngerman}\endnumbering\briefempfaengerindex{Widmann, Joseph Victor@\textsc{Widmann, Joseph Victor}!zzzSchnitzler, Arthur@\emph{von Arthur Schnitzler}!1902-02-081@{8. 2. 1902}|)be}\mylabel{L01201h}  \normalsize

\doendnotes{C}
\bigskip
\vfill

\clearpage

\footnotesize

\lohead{\textsc{register}}

% Definiere theindex-Environment komplett neu ohne reledmac
\makeatletter
\renewenvironment{theindex}{%
  \section*{\indexname}%
  \setlength{\parindent}{0pt}%
  \setlength{\parskip}{0pt plus 0.3pt}%
  \let\item\@idxitem
}{%
  \clearpage
}
\makeatother

\IfFileExists{\jobname-pw.ind}{\input{\jobname-pw.ind}}{}

\end{document}

      