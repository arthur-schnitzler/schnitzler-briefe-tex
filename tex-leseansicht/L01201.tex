%% latex-leseansicht-vorspann.tex
%% Vorspann für die Leseansicht.
%% Lädt die gemeinsame Datei latex-vorspann.tex mit nicht gesetztem Schalter.

\newif\ifkorrekturansicht
\korrekturansichtfalse

\input{../tex-inputs/latex-vorspann}


\section[Arthur Schnitzler an Joseph Victor Widmann, 8. 2. 1902]{L01201 Arthur Schnitzler an Joseph Victor Widmann, 8. 2. 1902}
\nopagebreak\mylabel{L01201v}
\rehead{ }\normalsize\beginnumbering\briefempfaengerindex{Widmann, Joseph Victor@\textsc{Widmann, Joseph Victor}!zzzSchnitzler, Arthur@\emph{von Arthur Schnitzler}!1902-02-081@{8. 2. 1902}|(be}
\toendnotes[C]{\smallbreak\pagebreak[2]}
\correspDesc{Versand  durch Arthur Schnitzler am 8. 2. 1902 in Wien
\newline{}Erhalt  durch Joseph Victor Widmann im Zeitraum [9. 2. 1902
                  – 13. 2. 1902?] in Bern}\toendnotes[C]{\smallbreak}
\Standort{Bern, Burgerbibliothek, N Joseph Viktor Widmann 29 (7).}
\physDesc{Briefkarte, 458 Zeichen
\newline{}Handschrift: schwarze Tinte, deutsche Kurrent
\newline{}Ordnung: 1) Lochung  2) von unbekannter Hand nummeriert: »2.« und
                                 beschriftet: »\textsc{A. Schnitzler} 1/2«}
\pstart{}{\pb}Verehrter Herr Widmann,\pend\vspace{0.5em}
\pstart
           es iſt{ }ſehr liebenswürdig von Ihnen, mir die Nummer des B. B.\pwindex{Bund@\emph{Der Bund}|pw} mit der freundlichen Ankündg des \textsc{Paracelsus}\pwindex{Schnitzler, Arthur 15.\,5.\,1862 Wien – 21.\,10.\,1931 ebd.@\textsc{Schnitzler, Arthur} (15.\,5.\,1862 Wien – 21.\,10.\,1931 ebd.), \emph{Schriftsteller, Mediziner}!Paracelsus. Versspiel in einem Akt@\strich\emph{Paracelsus. Versspiel in einem Akt}|pw} zu{ }ſenden. Wie gern möcht ich Ihnen einmal perſönlich Dank{ }ſagen für die {\pb}wohlthuende Sympathie, mit d\damage{er} Sie alle meine mehr oder weniger gelungenen Verſuche von Anatol\pwindex{Schnitzler, Arthur 15.\,5.\,1862 Wien – 21.\,10.\,1931 ebd.@\textsc{Schnitzler, Arthur} (15.\,5.\,1862 Wien – 21.\,10.\,1931 ebd.), \emph{Schriftsteller, Mediziner}!Anatol@\strich\emph{Anatol}|pw} an begleiten; Ihnen auch{ }ſagen, wie{ }ſehr
               mich die Anerkennung von Seiten eines Dichters freut, den ich{ }ſo hoch verehre.\pend
           
\pstart
           Ihr wahrlichſt{\\[\baselineskip]}ergebener \spacefill\mbox{Arthur Schnitzler}\pend
           \leftskip=0em{}
\pstart
           8. 2. 902.\pend
           \selectlanguage{ngerman}\endnumbering\briefempfaengerindex{Widmann, Joseph Victor@\textsc{Widmann, Joseph Victor}!zzzSchnitzler, Arthur@\emph{von Arthur Schnitzler}!1902-02-081@{8. 2. 1902}|)be}\mylabel{L01201h}  \newcommand{\dateiname}{L01201}\newcommand{\titel}{Arthur Schnitzler an Joseph Victor Widmann, 8. 2. 1902}\newcommand{\editorInnen}{Martin Anton Müller und Gerd-Hermann Susen}%% latex-leseansicht-abspann.tex
%% Abspann für die Leseansicht.
%% Der Schalter \ifkorrekturansicht ist bereits durch den Vorspann gesetzt.

%% latex-abspann.tex
%% Gemeinsamer Abspann für Korrekturansicht und Leseansicht.
%% Setzt den Schalter \ifkorrekturansicht voraus (gesetzt in den
%% einbindenden Dateien latex-korrekturansicht-abspann.tex bzw.
%% latex-leseansicht-abspann.tex).
%% ---------------------------------------------------------------

\normalsize

% Das esempio-Environment wird nur in der Leseansicht benötigt
\ifkorrekturansicht\else
\newenvironment{esempio}[3]%
{
    \vspace{1.5ex}
    \rlap{\underline{#1}}
    \par
    \setlength{\parindent}{0cm}
    \nopagebreak
    \leftskip=#2cm
    \rightskip=#3cm
}
{
    \par
}
\fi

\doendnotes{C}
\bigskip
\vfill

\clearpage

\footnotesize

\ifkorrekturansicht
  \lohead{\textsc{register}}
\fi

% theindex-Environment neu definieren ohne reledmac
\makeatletter
\renewenvironment{theindex}{%
  \ifkorrekturansicht
    \section*{\indexname}%
  \else
    \subsubsection*{Index der erwähnten Entitäten}%
  \fi
  \setlength{\parindent}{0pt}%
  \setlength{\parskip}{0pt plus 0.3pt}%
  \let\item\@idxitem
}{%
  \ifkorrekturansicht\clearpage\fi
}
\makeatother

\IfFileExists{\jobname-pw.ind}{\input{\jobname-pw.ind}}{}

% Quellenangabe nur in der Leseansicht
\ifkorrekturansicht\else
% Fallback-Definitionen, falls die .tex-Datei \titel etc. nicht gesetzt hat
\providecommand{\titel}{}
\providecommand{\editorInnen}{}
\providecommand{\dateiname}{\jobname}

\vspace{3cm}

\vfill

\footnotesize
\textsc{Quelle}: \titel. Herausgegeben von {\editorInnen}. In: \emph{Arthur Schnitzler: Briefwechsel mit Autorinnen und Autoren}.
 Digitale Edition, https://schnitzler-briefe.acdh.oeaw.ac.at/{\dateiname}.html (Stand \today)
\fi

\end{document}


