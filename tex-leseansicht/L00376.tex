%% latex-leseansicht-vorspann.tex
%% Vorspann für die Leseansicht.
%% Lädt die gemeinsame Datei latex-vorspann.tex mit nicht gesetztem Schalter.

\newif\ifkorrekturansicht
\korrekturansichtfalse

\input{../tex-inputs/latex-vorspann}


               \section[Arthur Schnitzler an Richard Beer-Hofmann, 5. 10. 1894]{ Arthur Schnitzler an Richard Beer-Hofmann, 5. 10. 1894}\nopagebreak\mylabel{v}\rehead{ }\begin{ledgroupsized}[t]{13cm}\normalsize\beginnumbering\briefempfaengerindex{Beer-Hofmann, Richard@\textsc{Beer-Hofmann, Richard}!zzzSchnitzler, Arthur@\emph{von Arthur Schnitzler}!1894-10-051@{5. 10. 1894}|(be} \toendnotes[C]{\smallbreak\pagebreak[2]} \Standort{YCGL, MSS 31.}
\physDesc{Brief, 2 Blätter, 5 Seiten, Umschlag
\newline{}Handschrift: Bleistift, deutsche Kurrent\newline{}Versand: 1) Stempel: »\nobreak{}\oindex{I., Innere Stadt@\textbf{I., Innere Stadt}|pwk}Wien 1/1, 5. 10. 94, 8–9 V\nobreak{}«.  2) Stempel: »\nobreak{}\oindex{Rom@\textbf{Rom}|pwk}Rom, 7 10-94, 2 S\nobreak{}«. 3) nachgesandt nach »\textsc{Hôtel Quirinal}\oindex{Hotel Quirinale@\textbf{Hotel Quirinale}|pw}«}\buchAbdrucke{\weitereDrucke{1) Arthur Schnitzler: \emph{Briefe 1875–1912}. Hg. Therese Nickl und Heinrich Schnitzler. Frankfurt am Main: \emph{S. Fischer} 1981, S. 229–230.} \weitereDrucke{2) Arthur Schnitzler, Richard Beer-Hofmann: \emph{Briefwechsel 1891–1931}. Hg. Konstanze Fliedl. Wien, Zürich: \emph{Europaverlag} 1992, S. 62–63.} }\toendnotes[C]{\smallbreak}\pstart{}{\pb}\textsc{Dr. Arthur Schnitzler}, Wien,
                     IX. Frankg. 1.\oindex{Frankgasse@\textbf{Frankgasse}|pw}\pend{}{\bigskip}\pstart{}{\pb}Herrn \textsc{Dr. Richard
                     Beer-Hofmann}\pend{}\pstart{}\textsc{Rom\oindex{Rom@\textbf{Rom}|pw}}\pend{}\pstart{}\textsc{a posta ferma}\pend{}\pstart{}\textsc{Italien\oindex{Italien@\textbf{Italien}|pw}}\pend{}{\bigskip}\pstart
           \raggedleft{}{\pb}Wien\oindex{Wien@\textbf{Wien}|pw}, 5. Oct
                  94.\pend
           \pstart{}Lieber Bekannter!\pend\pstart
           Das einzige, was Sie mir von Ihrer italien\oindex{Italien@\textbf{Italien}|pw}. Reiſe
               mittheilen, iſt daſs mein \textsc{Guercino}\pwindex{Guercino 8.2.1591 – 22.12.1666@\textsc{Guercino} (8.2.1591 – 22.12.1666), \emph{Maler}|pw}\pwindex{Guercino 8.2.1591 – 22.12.1666@\textsc{Guercino} (8.2.1591 – 22.12.1666), \emph{Maler}!Verstossung der Hagar1658@\strich\emph{Die Verstoßung der Hagar} {[}1658{]}|pwv} in Mailand\oindex{Mailand@\textbf{Mailand}|pw} hängt. Das ſteht aber ſchon im »\textsc{Lübke}\pwindex{Luebke, Wilhelm 17.01.1826 – 05.04.1893@\textsc{Lübke, Wilhelm} (17.01.1826 – 05.04.1893)!Grundriss der Kunstgeschichte1860 – 1860@\strich\emph{Grundriß der Kunstgeschichte} {[}1860 – 1860{]}|pwv}\pwindex{Luebke, Wilhelm 17.01.1826 – 05.04.1893@\textsc{Lübke, Wilhelm} (17.01.1826 – 05.04.1893)|pw}« – ich muſs Sie alſo, we{\geminationn} Sie überhaupt die
               Abſicht haben, Neuigkeiten aus Italien\oindex{Italien@\textbf{Italien}|pw} an mich zu
               ſchreiben, um sorgfältigere Auswahl bitten. Laſſen Sie ſich nicht etwa einfallen, mir
               aus Rom\oindex{Rom@\textbf{Rom}|pw} zu ſchreiben, daſs dort \textsc{Julius Caesar}\pwindex{Caesar, Gaius Iulius 13.7.100? v. Chr. – 15.3.44 v. Chr.@\textsc{Caesar, Gaius Iulius} (13.7.100? v. Chr. – 15.3.44 v. Chr.), \emph{Politiker, Kaiser, Heerführer}|pw} ermordet wurde – es ſteht im Ploetz\pwindex{Ploetz, Karl 08.07.1819 – 06.02.1881@\textsc{Ploetz, Karl} (08.07.1819 – 06.02.1881), \emph{Gymnasiallehrer}!Auszug aus der alten, mittleren und neueren Geschichte1863 – 1863@\strich\emph{Auszug aus der alten, mittleren und neueren Geschichte} {[}1863 – 1863{]}|pwv}\pwindex{Ploetz, Karl 08.07.1819 – 06.02.1881@\textsc{Ploetz, Karl} (08.07.1819 – 06.02.1881), \emph{Gymnasiallehrer}|pw}! – Dagegen bin ich gern {\pb}bereit, perſönlicheres
               von Ihnen zu erfahren – haben Sie keine von den \label{K_L00376_1v}\edtext{Schweſtern Rondoli\pwindex{Maupassant, Guy de 05.08.1850 – 07.07.1893@\textsc{Maupassant, Guy de} (05.08.1850 – 07.07.1893), \emph{Schriftsteller}!Schwestern Rondoli1884 – 1884@\strich\emph{Die Schwestern Rondoli} {[}1884 – 1884{]}|pwv}}{\lemma{\textnormal{\emph{Schweſtern Rondoli}}}\Cendnote{\textnormal{In der Novelle von Maupassant\pwindex{Maupassant, Guy de 05.08.1850 – 07.07.1893@\textsc{Maupassant, Guy de} (05.08.1850 – 07.07.1893), \emph{Schriftsteller}|pwk} hat die männliche Hauptfigur auf einer Reise eine
                  Liebschaft mit einer Frau, im Folgejahr mit ihrer Schwester.}}}\label{K_L00376_1h} getroffen? –
               Beantworten Sie mir auch gütigſt einige Fragen. 1.) Wa{\geminationn}
                  ko{\geminationm}en Sie zurück? 2.) Wie weit werden Sie Ihre Reiſe
               ausdehnen. 3) Haben Sie was geſchrieben?\pend
           \pstart
           Einige Thatſachen: \uline{Ludaßy}\pwindex{Gans-Ludassy, Julius von 13.04.1858 – 30.09.1922@\textsc{Gans-Ludassy, Julius von} (13.04.1858 – 30.09.1922), \emph{Schriftsteller, Journalist}|pw} iſt Chefred. der Wr. Allg. Ztg.\orgindex{Wiener Allgemeine Zeitung@Wiener Allgemeine Zeitung|pw} (mit einem
               nicht übeln Gehalt) worden. Er rechnet auf das ganze junge Wien\oindex{Wien@\textbf{Wien}|pw}; »alſo« auch auf Sie. (Die Gänſefüße ſind 17gradig.) –\pend
           \pstart
           Morgen iſt die »Schmetterlingsſchlacht\pwindex{\textcolor{red}{\textsuperscript{XXXX1 indx}}!Schmetterlingsschlacht1894@\strich\emph{Die Schmetterlingsschlacht} {[}1894{]}|pw}« – ich hab
                  {\pb}noch keinen Sitz, was mich geradezu aufregt. –\pend
           \pstart
           »Man sagt« iſt durchgefallen. –\pend
           \pstart
           Mein Stück\pwindex{Schnitzler, Arthur 15.05.1862 – 21.10.1931@\textsc{Schnitzler, Arthur} (15.05.1862 – 21.10.1931), \emph{Schriftsteller, Mediziner}!Liebelei. Schauspiel in drei Akten9. 10. 1895@\strich\emph{Liebelei. Schauspiel in drei Akten} {[}9. 10. 1895{]}|pwv} (gefährliche
               Nachbarſchaft der Thatſachen – Sie ſehen, ich bin nicht abergläubiſch, oder erſt
               recht, oder erſt recht gar nicht, oder gar nicht erſt recht gar nicht – ) ist {\dots} hier stock’ ich ſchon — vollendet? {\dotstwo} Nein. Beendet? Nein. Fertig? – Nein. – Ich habe »nur mehr« dran zu feilen.
               Hab ich Ihnen den Titel ſchon geſchrieben?{\dotstwo} »Liebelei\pwindex{Schnitzler, Arthur 15.05.1862 – 21.10.1931@\textsc{Schnitzler, Arthur} (15.05.1862 – 21.10.1931), \emph{Schriftsteller, Mediziner}!Liebelei. Schauspiel in drei Akten9. 10. 1895@\strich\emph{Liebelei. Schauspiel in drei Akten} {[}9. 10. 1895{]}|pw}«. – Anfangs wird er ihnen wahrſcheinlich
               nicht {\pb}gefallen; aber er iſt gut, – auch praktiſch
                  geno{\geminationm}en. –\pend
           \pstart
           Ich lese: \textsc{Rosenkranz\pwindex{Rosenkranz, Karl 23.04.1805 – 14.06.1879@\textsc{Rosenkranz, Karl} (23.04.1805 – 14.06.1879), \emph{Philosoph}|pw}, Diderot\pwindex{Diderot, Denis 05.10.1713 – 31.07.1784@\textsc{Diderot, Denis} (05.10.1713 – 31.07.1784), \emph{Schriftsteller}|pw}\pwindex{?? Werk@Nicht ermittelte Verfasserinnen und Verfasser!Diderots Leben und Werke1866 – 1866@\emph{Diderots Leben und Werke} {[}1866 – 1866{]}|pw}; – Keller\pwindex{Keller, Otto 05.06.1861 – 25.10.1928@\textsc{Keller, Otto} (05.06.1861 – 25.10.1928)|pw}}, Musikgeschichte\pwindex{Keller, Otto 05.06.1861 – 25.10.1928@\textsc{Keller, Otto} (05.06.1861 – 25.10.1928)!Geschichte der Musik1893 – 1893@\strich\emph{Geschichte der Musik} {[}1893 – 1893{]}|pw} u. a. –\pend
           \pstart
           Vorgeleſen wurde mir – ein fünfaktiges Drama in Verſen, in dem aber gewiſs Talent
               ſteckt; \textsc{Phryne}\pwindex{Ebermann, Leo 16.07.1863 – 09.10.1914@\textsc{Ebermann, Leo} (16.07.1863 – 09.10.1914), \emph{Schriftsteller, Journalist, Rechtswissenschaftler}!Athenerin19.9.1896 – 19.9.1896@\strich\emph{Die Athenerin} {[}19.9.1896 – 19.9.1896{]}|pw} von \textsc{Leo Ebermann}\pwindex{Ebermann, Leo 16.07.1863 – 09.10.1914@\textsc{Ebermann, Leo} (16.07.1863 – 09.10.1914), \emph{Schriftsteller, Journalist, Rechtswissenschaftler}|pw}, der mich aber als Menſch und beſonders als Vorleſer ſehr nervös macht: er
               poſirt auf guten Sprecher{\dots}\pend
           \pstart
           Phrrryne{\dotstwo}\pend
           \pstart
           Gawiſs {\dotstwo} du darrrfſt nicht länger lebohn{\dots}\pend
           \pstart
           Meine Gerechtigkeit hat Orgien {\pb}gefeiert; eigentlich
               wollte ich ihm ununterbrochen Ihre Büſte »in’ \strikeout{den} Kop\substVorne{}\textsuperscript{f}\substDazwischen{}p\substHinten{} hereinhaun«. – (Lachen Sie nicht; der Kellner beobachtet Sie. –)\pend
           \pstart
           Leben Sie wohl, ſchreiben Sie mir, und ſeien Sie herzlichſt gegrüßt.\pend
           \pstart Ihr \spacefill\mbox{Arthur}\pend{}\endnumbering\briefempfaengerindex{Beer-Hofmann, Richard@\textsc{Beer-Hofmann, Richard}!zzzSchnitzler, Arthur@\emph{von Arthur Schnitzler}!1894-10-051@{5. 10. 1894}|)be}\mylabel{h}\end{ledgroupsized}  \newcommand{\dateiname}{L00376}\newcommand{\titel}{Arthur Schnitzler an Richard Beer-Hofmann, 5. 10. 1894}\newcommand{\editorInnen}{Martin Anton Müller und Gerd-Hermann Susen}%% latex-leseansicht-abspann.tex
%% Abspann für die Leseansicht.
%% Der Schalter \ifkorrekturansicht ist bereits durch den Vorspann gesetzt.

%% latex-abspann.tex
%% Gemeinsamer Abspann für Korrekturansicht und Leseansicht.
%% Setzt den Schalter \ifkorrekturansicht voraus (gesetzt in den
%% einbindenden Dateien latex-korrekturansicht-abspann.tex bzw.
%% latex-leseansicht-abspann.tex).
%% ---------------------------------------------------------------

\normalsize

% Das esempio-Environment wird nur in der Leseansicht benötigt
\ifkorrekturansicht\else
\newenvironment{esempio}[3]%
{
    \vspace{1.5ex}
    \rlap{\underline{#1}}
    \par
    \setlength{\parindent}{0cm}
    \nopagebreak
    \leftskip=#2cm
    \rightskip=#3cm
}
{
    \par
}
\fi

\doendnotes{C}
\bigskip
\vfill

\clearpage

\footnotesize

\ifkorrekturansicht
  \lohead{\textsc{register}}
\fi

% theindex-Environment neu definieren ohne reledmac
\makeatletter
\renewenvironment{theindex}{%
  \ifkorrekturansicht
    \section*{\indexname}%
  \else
    \subsubsection*{Index der erwähnten Entitäten}%
  \fi
  \setlength{\parindent}{0pt}%
  \setlength{\parskip}{0pt plus 0.3pt}%
  \let\item\@idxitem
}{%
  \ifkorrekturansicht\clearpage\fi
}
\makeatother

\IfFileExists{\jobname-pw.ind}{\input{\jobname-pw.ind}}{}

% Quellenangabe nur in der Leseansicht
\ifkorrekturansicht\else
% Fallback-Definitionen, falls die .tex-Datei \titel etc. nicht gesetzt hat
\providecommand{\titel}{}
\providecommand{\editorInnen}{}
\providecommand{\dateiname}{\jobname}

\vspace{3cm}

\vfill

\footnotesize
\textsc{Quelle}: \titel. Herausgegeben von {\editorInnen}. In: \emph{Arthur Schnitzler: Briefwechsel mit Autorinnen und Autoren}.
 Digitale Edition, https://schnitzler-briefe.acdh.oeaw.ac.at/{\dateiname}.html (Stand \today)
\fi

\end{document}


      