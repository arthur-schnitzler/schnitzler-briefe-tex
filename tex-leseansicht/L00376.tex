%% latex-leseansicht-vorspann.tex
%% Vorspann für die Leseansicht.
%% Lädt die gemeinsame Datei latex-vorspann.tex mit nicht gesetztem Schalter.

\newif\ifkorrekturansicht
\korrekturansichtfalse

\input{../tex-inputs/latex-vorspann}


\section[Arthur Schnitzler an Richard Beer-Hofmann, 5. 10. 1894]{L00376 Arthur Schnitzler an Richard Beer-Hofmann, 5. 10. 1894}
\nopagebreak\mylabel{L00376v}
\rehead{ }\normalsize\beginnumbering\briefempfaengerindex{Beer-Hofmann, Richard@\textsc{Beer-Hofmann, Richard}!zzzSchnitzler, Arthur@\emph{von Arthur Schnitzler}!1894-10-051@{5. 10. 1894}|(be}
\toendnotes[C]{\smallbreak\pagebreak[2]}
\correspDesc{Versand  durch Arthur Schnitzler am 5. 10. 1894 in Wien
\newline{}Erhalt  durch Richard Beer-Hofmann im Zeitraum [5. 10. 1894
                  – 9. 10. 1894?] \textbf{Ort fehlend} }\toendnotes[C]{\smallbreak}
\Standort{YCGL, MSS 31.}
\physDesc{Brief, 2 Blätter, 5 Seiten, Kuvert, 2107 Zeichen
\newline{}Handschrift: Bleistift, deutsche Kurrent
\newline{}Versand: 1) Stempel: »\nobreak{}\oindex{I., Innere Stadt@\textbf{I., Innere Stadt}, \emph{Verwaltungsgebiet}|pwk}Wien 1/1, 5. 10. 94, 8–9 V\nobreak{}«.   2) Stempel: »\nobreak{}\oindex{Rom@\textbf{Rom}, \emph{Hauptstadt}|pwk}Rom, 7 10-94, 2 S\nobreak{}«.  3) nachgesandt nach »\textsc{Hôtel Quirinal}\oindex{Hotel Quirinale@\textbf{Hotel Quirinale}, \emph{Hotel}|pw}«}
\buchAbdrucke{\weitereDrucke{1) Arthur Schnitzler: \emph{Briefe 1875–1912}. Herausgegeben von Therese Nickl und Heinrich Schnitzler. Frankfurt am Main: \emph{S. Fischer} 1981, S. 229–230.} \weitereDrucke{2) Arthur Schnitzler, Richard Beer-Hofmann: \emph{Briefwechsel 1891–1931}. Herausgegeben von Konstanze Fliedl. Wien, Zürich: \emph{Europaverlag} 1992, S. 62–63.} }\toendnotes[C]{\smallbreak}\pstart{}{\pb}\textsc{Dr. Arthur Schnitzler}, Wien, IX. Frankg. 1.\oindex{Wien@\textbf{Wien}!IX., Alsergrund@\textbf{IX., Alsergrund}!Frankgasse 1@\textbf{Frankgasse 1}, \emph{Wohngebäude}|pw}\pend{}{\bigskip}\pstart{}{\pb}Herrn \textsc{Dr. Richard
                     Beer-Hofmann}\pend{}\pstart{}\textsc{Rom\oindex{Rom@\textbf{Rom}, \emph{Hauptstadt}|pw}}\pend{}\pstart{}\textsc{a posta ferma}\pend{}\pstart{}\textsc{Italien\oindex{Italien@\textbf{Italien}|pw}}\pend{}{\bigskip}\vspace{1em}
\pstart
           \raggedleft{}{\pb}Wien\oindex{Wien@\textbf{Wien}, \emph{Verwaltungsgebiet}|pw}, 5. Oct 94.\pend
           
\pstart{}Lieber Bekannter!\pend\vspace{0.5em}
\pstart
           Das einzige, was Sie mir von Ihrer italien\oindex{Italien@\textbf{Italien}|pw}.
               Reiſe mittheilen, iſt daſs mein \textsc{Guercino}\pwindex{Guercino 8.\,2.\,1591 Cento – 22.\,12.\,1666 Bologna@\textsc{Guercino} (8.\,2.\,1591 Cento – 22.\,12.\,1666 Bologna), \emph{Maler}|pw}\pwindex{Guercino 8.\,2.\,1591 Cento – 22.\,12.\,1666 Bologna@\textsc{Guercino} (8.\,2.\,1591 Cento – 22.\,12.\,1666 Bologna), \emph{Maler}!Verstoßung der Hagar@\strich\emph{Die Verstoßung der Hagar}|pwv} in Mailand\oindex{Mailand@\textbf{Mailand}|pw} hängt. Das{ }ſteht aber{ }ſchon im
                  »\textsc{Lübke}\pwindex{Lübke, Wilhelm 17.\,1.\,1826 Dortmund – 5.\,4.\,1893 Karlsruhe@\textsc{Lübke, Wilhelm} (17.\,1.\,1826 Dortmund – 5.\,4.\,1893 Karlsruhe), \emph{Kunsthistoriker}!Grundriß der Kunstgeschichte@\strich\emph{Grundriß der Kunstgeschichte}|pwv}\pwindex{Lübke, Wilhelm 17.\,1.\,1826 Dortmund – 5.\,4.\,1893 Karlsruhe@\textsc{Lübke, Wilhelm} (17.\,1.\,1826 Dortmund – 5.\,4.\,1893 Karlsruhe), \emph{Kunsthistoriker}|pw}« – ich muſs Sie alſo, we{\geminationn} Sie überhaupt die
               Abſicht haben, Neuigkeiten aus Italien\oindex{Italien@\textbf{Italien}|pw} an mich
               zu{ }ſchreiben, um sorgfältigere Auswahl bitten. Laſſen Sie{ }ſich nicht etwa einfallen,
               mir aus Rom\oindex{Rom@\textbf{Rom}, \emph{Hauptstadt}|pw} zu{ }ſchreiben, daſs dort \textsc{Julius Caesar}\pwindex{Caesar, Gaius Iulius 13.7.100? v. Chr. Rom – 15.3.44 v. Chr. ebd.@\textsc{Caesar, Gaius Iulius} (13.7.100? v. Chr. Rom – 15.3.44 v. Chr. ebd.), \emph{Politiker, Kaiser, Heerführer}|pw} ermordet wurde – es{ }ſteht im Ploetz\pwindex{Ploetz, Karl 8.\,7.\,1819 Berlin – 6.\,2.\,1881 Görlitz@\textsc{Ploetz, Karl} (8.\,7.\,1819 Berlin – 6.\,2.\,1881 Görlitz), \emph{Gymnasiallehrer}!Auszug aus der alten, mittleren und neueren Geschichte@\strich\emph{Auszug aus der alten, mittleren und neueren Geschichte}|pwv}\pwindex{Ploetz, Karl 8.\,7.\,1819 Berlin – 6.\,2.\,1881 Görlitz@\textsc{Ploetz, Karl} (8.\,7.\,1819 Berlin – 6.\,2.\,1881 Görlitz), \emph{Gymnasiallehrer}|pw}! – Dagegen bin ich gern {\pb}bereit, perſönlicheres
               von Ihnen zu erfahren – haben Sie keine von den \label{K_L00376-1v}\edtext{Schweſtern Rondoli\pwindex{Maupassant, Guy de 5.\,8.\,1850 Tourville-sur-Arques – 7.\,7.\,1893 Paris@\textsc{Maupassant, Guy de} (5.\,8.\,1850 Tourville-sur-Arques – 7.\,7.\,1893 Paris), \emph{Schriftsteller}!Schwestern Rondoli@\strich\emph{Die Schwestern Rondoli}|pwv}}{\lemma{\textnormal{\emph{Schwestern Rondoli}}}\Cendnote{\textnormal{In der Novelle von Maupassant\pwindex{Maupassant, Guy de 5.\,8.\,1850 Tourville-sur-Arques – 7.\,7.\,1893 Paris@\textsc{Maupassant, Guy de} (5.\,8.\,1850 Tourville-sur-Arques – 7.\,7.\,1893 Paris), \emph{Schriftsteller}|pwk} hat die männliche Hauptfigur auf einer Reise eine
                  Liebschaft mit einer Frau, im Folgejahr mit ihrer Schwester.}}}\label{K_L00376-1} getroffen? –
               Beantworten Sie mir auch gütigſt einige Fragen. 1.) Wa{\geminationn}
                  ko{\geminationm}en Sie zurück? 2.) Wie weit werden Sie Ihre Reiſe
               ausdehnen. 3) Haben Sie was geſchrieben?\pend
           
\pstart
           Einige Thatſachen: \uline{Ludaßy}\pwindex{Gans-Ludassy, Julius von 13.\,4.\,1858 Wien – 30.\,9.\,1922 ebd.@\textsc{Gans-Ludassy, Julius von} (13.\,4.\,1858 Wien – 30.\,9.\,1922 ebd.), \emph{Schriftsteller, Journalist, Herausgeber}|pw} iſt Chefred. der Wr. Allg. Ztg.\orgindex{Wiener Allgemeine Zeitung@Wiener Allgemeine Zeitung|pw} (mit einem
               nicht übeln Gehalt) worden. Er rechnet auf das ganze junge Wien\oindex{Wien@\textbf{Wien}, \emph{Verwaltungsgebiet}|pw}; »alſo« auch auf Sie. (Die Gänſefüße{ }ſind 17gradig.) –\pend
           
\pstart
           Morgen iſt die »Schmetterlingsſchlacht\pwindex{\textcolor{red}{\textsuperscript{XXXX indx1}}!Schmetterlingsschlacht. Komödie in 4 Akten@\strich\emph{Die Schmetterlingsschlacht. Komödie in 4 Akten}|pw}« – ich
               hab {\pb}noch keinen Sitz, was mich geradezu
               aufregt. –\pend
           
\pstart
           »Man sagt« iſt durchgefallen. –\pend
           
\pstart
           Mein Stück\pwindex{Schnitzler, Arthur 15.\,5.\,1862 Wien – 21.\,10.\,1931 ebd.@\textsc{Schnitzler, Arthur} (15.\,5.\,1862 Wien – 21.\,10.\,1931 ebd.), \emph{Schriftsteller, Mediziner}!Liebelei. Schauspiel in drei Akten@\strich\emph{Liebelei. Schauspiel in drei Akten}|pwv} (gefährliche
               Nachbarſchaft der Thatſachen – Sie{ }ſehen, ich bin nicht abergläubiſch, oder erſt
               recht, oder erſt recht gar nicht, oder gar nicht erſt recht gar nicht – ) ist {\dots} hier stock’ ich{ }ſchon – vollendet? {\dotstwo} Nein. Beendet? Nein. Fertig? – Nein. – Ich habe »nur mehr« dran zu feilen.
               Hab ich Ihnen den Titel{ }ſchon geſchrieben?{\dotstwo} »Liebelei\pwindex{Schnitzler, Arthur 15.\,5.\,1862 Wien – 21.\,10.\,1931 ebd.@\textsc{Schnitzler, Arthur} (15.\,5.\,1862 Wien – 21.\,10.\,1931 ebd.), \emph{Schriftsteller, Mediziner}!Liebelei. Schauspiel in drei Akten@\strich\emph{Liebelei. Schauspiel in drei Akten}|pw}«. – Anfangs wird er ihnen wahrſcheinlich
               nicht {\pb}gefallen; aber er iſt gut, – auch praktiſch
                  geno{\geminationm}en. –\pend
           
\pstart
           Ich lese: \textsc{Rosenkranz\pwindex{Rosenkranz, Karl 23.\,4.\,1805 Magdeburg – 14.\,6.\,1879 Kaliningrad@\textsc{Rosenkranz, Karl} (23.\,4.\,1805 Magdeburg – 14.\,6.\,1879 Kaliningrad), \emph{Philosoph}|pw}, Diderot\pwindex{Diderot, Denis 5.\,10.\,1713 Langres – 31.\,7.\,1784 Paris@\textsc{Diderot, Denis} (5.\,10.\,1713 Langres – 31.\,7.\,1784 Paris), \emph{Schriftsteller}|pw}\pwindex{Diderots Leben und Werke@\emph{Diderots Leben und Werke}|pw}; – Keller\pwindex{Keller, Otto 5.\,6.\,1861 Wien – 25.\,10.\,1928 Salzburg@\textsc{Keller, Otto} (5.\,6.\,1861 Wien – 25.\,10.\,1928 Salzburg)|pw}}, Musikgeschichte\pwindex{Keller, Otto 5.\,6.\,1861 Wien – 25.\,10.\,1928 Salzburg@\textsc{Keller, Otto} (5.\,6.\,1861 Wien – 25.\,10.\,1928 Salzburg)!Geschichte der Musik@\strich\emph{Geschichte der Musik}|pw} u. a. –\pend
           
\pstart
           Vorgeleſen wurde mir – ein fünfaktiges Drama in Verſen, in dem aber gewiſs Talent{ }ſteckt; \textsc{Phryne}\pwindex{Ebermann, Leo 16.\,7.\,1863 Draganovka – 9.\,10.\,1914 Wien@\textsc{Ebermann, Leo} (16.\,7.\,1863 Draganovka – 9.\,10.\,1914 Wien), \emph{Schriftsteller, Journalist, Rechtswissenschaftler}!Athenerin. Drama in drei Aufzügen@\strich\emph{Die Athenerin. Drama in drei Aufzügen}|pw} von \textsc{Leo Ebermann}\pwindex{Ebermann, Leo 16.\,7.\,1863 Draganovka – 9.\,10.\,1914 Wien@\textsc{Ebermann, Leo} (16.\,7.\,1863 Draganovka – 9.\,10.\,1914 Wien), \emph{Schriftsteller, Journalist, Rechtswissenschaftler}|pw}, der mich aber als Menſch und beſonders als Vorleſer{ }ſehr nervös macht: er
               poſirt auf guten Sprecher{\dots}\pend
           
\pstart
           Phrrryne{\dotstwo}\pend
           
\pstart
           Gawiſs {\dotstwo} du darrrfſt nicht länger lebohn{\dots}\pend
           
\pstart
           Meine Gerechtigkeit hat Orgien {\pb}gefeiert; eigentlich
               wollte ich ihm ununterbrochen Ihre Büſte »in’ \strikeout{den} Kop\substVorne{}\textsuperscript{f}\substDazwischen{}p\substHinten{} hereinhaun«. – (Lachen Sie nicht; der Kellner beobachtet Sie. –)\pend
           
\pstart
           Leben Sie wohl,{ }ſchreiben Sie mir, und{ }ſeien Sie herzlichſt gegrüßt.\pend
           \pstart Ihr \spacefill\mbox{Arthur}\pend{}\selectlanguage{ngerman}\endnumbering\briefempfaengerindex{Beer-Hofmann, Richard@\textsc{Beer-Hofmann, Richard}!zzzSchnitzler, Arthur@\emph{von Arthur Schnitzler}!1894-10-051@{5. 10. 1894}|)be}\mylabel{L00376h}  \newcommand{\dateiname}{L00376}\newcommand{\titel}{Arthur Schnitzler an Richard Beer-Hofmann, 5. 10. 1894}\newcommand{\editorInnen}{Martin Anton Müller und Gerd-Hermann Susen}%% latex-leseansicht-abspann.tex
%% Abspann für die Leseansicht.
%% Der Schalter \ifkorrekturansicht ist bereits durch den Vorspann gesetzt.

%% latex-abspann.tex
%% Gemeinsamer Abspann für Korrekturansicht und Leseansicht.
%% Setzt den Schalter \ifkorrekturansicht voraus (gesetzt in den
%% einbindenden Dateien latex-korrekturansicht-abspann.tex bzw.
%% latex-leseansicht-abspann.tex).
%% ---------------------------------------------------------------

\normalsize

% Das esempio-Environment wird nur in der Leseansicht benötigt
\ifkorrekturansicht\else
\newenvironment{esempio}[3]%
{
    \vspace{1.5ex}
    \rlap{\underline{#1}}
    \par
    \setlength{\parindent}{0cm}
    \nopagebreak
    \leftskip=#2cm
    \rightskip=#3cm
}
{
    \par
}
\fi

\doendnotes{C}
\bigskip
\vfill

\clearpage

\footnotesize

\ifkorrekturansicht
  \lohead{\textsc{register}}
\fi

% theindex-Environment neu definieren ohne reledmac
\makeatletter
\renewenvironment{theindex}{%
  \ifkorrekturansicht
    \section*{\indexname}%
  \else
    \subsubsection*{Index der erwähnten Entitäten}%
  \fi
  \setlength{\parindent}{0pt}%
  \setlength{\parskip}{0pt plus 0.3pt}%
  \let\item\@idxitem
}{%
  \ifkorrekturansicht\clearpage\fi
}
\makeatother

\IfFileExists{\jobname-pw.ind}{\input{\jobname-pw.ind}}{}

% Quellenangabe nur in der Leseansicht
\ifkorrekturansicht\else
% Fallback-Definitionen, falls die .tex-Datei \titel etc. nicht gesetzt hat
\providecommand{\titel}{}
\providecommand{\editorInnen}{}
\providecommand{\dateiname}{\jobname}

\vspace{3cm}

\vfill

\footnotesize
\textsc{Quelle}: \titel. Herausgegeben von {\editorInnen}. In: \emph{Arthur Schnitzler: Briefwechsel mit Autorinnen und Autoren}.
 Digitale Edition, https://schnitzler-briefe.acdh.oeaw.ac.at/{\dateiname}.html (Stand \today)
\fi

\end{document}


