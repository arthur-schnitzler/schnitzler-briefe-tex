%% latex-korrekturansicht-vorspann.tex
%% Vorspann für die Korrekturansicht.
%% Lädt die gemeinsame Datei latex-vorspann.tex mit gesetztem Schalter.

\newif\ifkorrekturansicht
\korrekturansichttrue

\input{../tex-inputs/latex-vorspann}


\section[Arthur Schnitzler an Richard Beer-Hofmann, 5. 10. 1894]{L00376 Arthur Schnitzler an Richard Beer-Hofmann, 5. 10. 1894}
\nopagebreak\mylabel{L00376v}
\rehead{ }\normalsize\beginnumbering\briefempfaengerindex{Beer-Hofmann, Richard@\textsc{Beer-Hofmann, Richard}!zzzSchnitzler, Arthur@\emph{von Arthur Schnitzler}!1894-10-051@{5. 10. 1894}|(be}
\toendnotes[C]{\smallbreak\pagebreak[2]}\Standort{YCGL, MSS 31.}
\physDesc{Brief, 2 Blätter, 5 Seiten, Umschlag, 2107 Zeichen
\newline{}Handschrift: Bleistift, deutsche Kurrent
\newline{}Versand: 1) Stempel: »\nobreak{}\oindex{I., Innere Stadt@\textbf{I., Innere Stadt}, \emph{A.ADM3}|pwk}Wien 1/1, 5. 10. 94, 8–9 V\nobreak{}«.   2) Stempel: »\nobreak{}\oindex{Rom@\textbf{Rom}, \emph{P.PPLC}|pwk}Rom, 7 10-94, 2 S\nobreak{}«.  3) nachgesandt nach »\textsc{Hôtel Quirinal}\oindex{Hotel Quirinale@\textbf{Hotel Quirinale}, \emph{Hotel (K.HTL)}|pw}«}
\buchAbdrucke{\weitereDrucke{1) Arthur Schnitzler: \emph{Briefe 1875–1912}. Frankfurt am Main: \emph{S. Fischer} 1981, S. 229–230.} \weitereDrucke{2) Arthur Schnitzler, Richard Beer-Hofmann: \emph{Briefwechsel 1891–1931}. Wien, Zürich: \emph{Europaverlag} 1992, S. 62–63.} }\toendnotes[C]{\smallbreak}\pstart{}{\pb}\textsc{Dr. Arthur Schnitzler}, Wien, IX. Frankg. 1.\oindex{Frankgasse 1@\textbf{Frankgasse 1}, \emph{Wohngebäude (K.WHS)}|pw}\pend{}{\bigskip}\pstart{}{\pb}Herrn \textsc{Dr. Richard
                     Beer-Hofmann}\pend{}\pstart{}\textsc{Rom\oindex{Rom@\textbf{Rom}, \emph{P.PPLC}|pw}}\pend{}\pstart{}\textsc{a posta ferma}\pend{}\pstart{}\textsc{Italien\oindex{Italien@\textbf{Italien}, \emph{A.PCLI}|pw}}\pend{}{\bigskip}\vspace{1em}
\pstart
           \raggedleft{}{\pb}Wien\oindex{Wien@\textbf{Wien}, \emph{A.ADM2}|pw}, 5. Oct
                  94.\pend
           
\pstart{}Lieber Bekannter!\pend\vspace{0.5em}
\pstart
           Das einzige, was Sie mir von Ihrer italien\oindex{Italien@\textbf{Italien}, \emph{A.PCLI}|pw}.
               Reiſe mittheilen, iſt daſs mein \textsc{Guercino}\pwindex{Guercino 8.2.1591 – 22.12.1666@\textsc{Guercino} (8.2.1591 – 22.12.1666), \emph{Maler/Malerin}|pw}\pwindex{Verstossung der Hagar@\emph{Die Verstoßung der Hagar}|pwv} in Mailand\oindex{Mailand@\textbf{Mailand}, \emph{P.PPLA}|pw} hängt. Das ſteht aber ſchon im
                  »\textsc{Lübke}\pwindex{Grundriss der Kunstgeschichte@\emph{Grundriß der Kunstgeschichte}|pwv}\pwindex{Luebke, Wilhelm 17.01.1826 – 05.04.1893@\textsc{Lübke, Wilhelm} (17.01.1826 – 05.04.1893), \emph{Kunsthistoriker/Kunsthistorikerin}|pw}« – ich muſs Sie alſo, we{\geminationn} Sie überhaupt die
               Abſicht haben, Neuigkeiten aus Italien\oindex{Italien@\textbf{Italien}, \emph{A.PCLI}|pw} an mich
               zu ſchreiben, um sorgfältigere Auswahl bitten. Laſſen Sie ſich nicht etwa einfallen,
               mir aus Rom\oindex{Rom@\textbf{Rom}, \emph{P.PPLC}|pw} zu ſchreiben, daſs dort \textsc{Julius Caesar}\pwindex{Caesar, Gaius Iulius 13.7.100? v. Chr. – 15.3.44 v. Chr.@\textsc{Caesar, Gaius Iulius} (13.7.100? v. Chr. – 15.3.44 v. Chr.), \emph{Politiker/Politikerin, Kaiser/Kaiserin, Heerführer/Heerführerin}|pw} ermordet wurde – es ſteht im Ploetz\pwindex{Auszug aus der alten, mittleren und neueren Geschichte@\emph{Auszug aus der alten, mittleren und neueren Geschichte}|pwv}\pwindex{Ploetz, Karl 08.07.1819 – 06.02.1881@\textsc{Ploetz, Karl} (08.07.1819 – 06.02.1881), \emph{Gymnasiallehrer/Gymnasiallehrerin}|pw}! – Dagegen bin ich gern {\pb}bereit, perſönlicheres
               von Ihnen zu erfahren – haben Sie keine von den \label{K_L00376-1v}\edtext{Schweſtern Rondoli\pwindex{Schwestern Rondoli@\emph{Die Schwestern Rondoli}|pwv}}{\lemma{\textnormal{\emph{Schweſtern Rondoli}}}\Cendnote{\textnormal{In der Novelle von Maupassant\pwindex{Maupassant, Guy de 05.08.1850 – 07.07.1893@\textsc{Maupassant, Guy de} (05.08.1850 – 07.07.1893), \emph{Schriftsteller/Schriftstellerin}|pwk} hat die männliche Hauptfigur auf einer Reise eine
                  Liebschaft mit einer Frau, im Folgejahr mit ihrer Schwester.}}}\label{K_L00376-1} getroffen? –
               Beantworten Sie mir auch gütigſt einige Fragen. 1.) Wa{\geminationn}
                  ko{\geminationm}en Sie zurück? 2.) Wie weit werden Sie Ihre Reiſe
               ausdehnen. 3) Haben Sie was geſchrieben?\pend
           
\pstart
           Einige Thatſachen: \uline{Ludaßy}\pwindex{Gans-Ludassy, Julius von 13.04.1858 – 30.09.1922@\textsc{Gans-Ludassy, Julius von} (13.04.1858 – 30.09.1922), \emph{Schriftsteller/Schriftstellerin, Journalist/Journalistin, Herausgeber/Herausgeberin}|pw} iſt Chefred. der Wr. Allg. Ztg.\orgindex{Wiener Allgemeine Zeitung@Wiener Allgemeine Zeitung|pw} (mit einem
               nicht übeln Gehalt) worden. Er rechnet auf das ganze junge Wien\oindex{Wien@\textbf{Wien}, \emph{A.ADM2}|pw}; »alſo« auch auf Sie. (Die Gänſefüße ſind 17gradig.) –\pend
           
\pstart
           Morgen iſt die »Schmetterlingsſchlacht\pwindex{Schmetterlingsschlacht. Komoedie in 4 Akten@\emph{Die Schmetterlingsschlacht. Komödie in 4 Akten}|pw}« – ich
               hab {\pb}noch keinen Sitz, was mich geradezu
               aufregt. –\pend
           
\pstart
           »Man sagt« iſt durchgefallen. –\pend
           
\pstart
           Mein Stück\pwindex{Liebelei. Schauspiel in drei Akten@\emph{Liebelei. Schauspiel in drei Akten}|pwv} (gefährliche
               Nachbarſchaft der Thatſachen – Sie ſehen, ich bin nicht abergläubiſch, oder erſt
               recht, oder erſt recht gar nicht, oder gar nicht erſt recht gar nicht – ) ist {\dots} hier stock’ ich ſchon – vollendet? {\dotstwo} Nein. Beendet? Nein. Fertig? – Nein. – Ich habe »nur mehr« dran zu feilen.
               Hab ich Ihnen den Titel ſchon geſchrieben?{\dotstwo} »Liebelei\pwindex{Liebelei. Schauspiel in drei Akten@\emph{Liebelei. Schauspiel in drei Akten}|pw}«. – Anfangs wird er ihnen wahrſcheinlich
               nicht {\pb}gefallen; aber er iſt gut, – auch praktiſch
                  geno{\geminationm}en. –\pend
           
\pstart
           Ich lese: \textsc{Rosenkranz\pwindex{Rosenkranz, Karl 23.04.1805 – 14.06.1879@\textsc{Rosenkranz, Karl} (23.04.1805 – 14.06.1879), \emph{Philosoph/Philosophin}|pw}, Diderot\pwindex{Diderot, Denis 05.10.1713 – 31.07.1784@\textsc{Diderot, Denis} (05.10.1713 – 31.07.1784), \emph{Schriftsteller/Schriftstellerin}|pw}\pwindex{Diderots Leben und Werke@\emph{Diderots Leben und Werke}|pw}; – Keller\pwindex{Keller, Otto 05.06.1861 – 25.10.1928@\textsc{Keller, Otto} (05.06.1861 – 25.10.1928)|pw}}, Musikgeschichte\pwindex{Geschichte der Musik@\emph{Geschichte der Musik}|pw} u. a. –\pend
           
\pstart
           Vorgeleſen wurde mir – ein fünfaktiges Drama in Verſen, in dem aber gewiſs Talent
               ſteckt; \textsc{Phryne}\pwindex{Athenerin. Drama in drei Aufzuegen@\emph{Die Athenerin. Drama in drei Aufzügen}|pw} von \textsc{Leo Ebermann}\pwindex{Ebermann, Leo 16.07.1863 – 09.10.1914@\textsc{Ebermann, Leo} (16.07.1863 – 09.10.1914), \emph{Schriftsteller/Schriftstellerin, Journalist/Journalistin, Rechtswissenschaftler/Rechtswissenschaftlerin}|pw}, der mich aber als Menſch und beſonders als Vorleſer ſehr nervös macht: er
               poſirt auf guten Sprecher{\dots}\pend
           
\pstart
           Phrrryne{\dotstwo}\pend
           
\pstart
           Gawiſs {\dotstwo} du darrrfſt nicht länger lebohn{\dots}\pend
           
\pstart
           Meine Gerechtigkeit hat Orgien {\pb}gefeiert; eigentlich
               wollte ich ihm ununterbrochen Ihre Büſte »in’ \strikeout{den} Kop\substVorne{}\textsuperscript{f}\substDazwischen{}p\substHinten{} hereinhaun«. – (Lachen Sie nicht; der Kellner beobachtet Sie. –)\pend
           
\pstart
           Leben Sie wohl, ſchreiben Sie mir, und ſeien Sie herzlichſt gegrüßt.\pend
           \pstart Ihr \spacefill\mbox{Arthur}\pend{}\selectlanguage{ngerman}\endnumbering\briefempfaengerindex{Beer-Hofmann, Richard@\textsc{Beer-Hofmann, Richard}!zzzSchnitzler, Arthur@\emph{von Arthur Schnitzler}!1894-10-051@{5. 10. 1894}|)be}\mylabel{L00376h}  \normalsize

\doendnotes{C}
\bigskip
\vfill

\clearpage

\footnotesize

\lohead{\textsc{register}}

% Definiere theindex-Environment komplett neu ohne reledmac
\makeatletter
\renewenvironment{theindex}{%
  \section*{\indexname}%
  \setlength{\parindent}{0pt}%
  \setlength{\parskip}{0pt plus 0.3pt}%
  \let\item\@idxitem
}{%
  \clearpage
}
\makeatother

\IfFileExists{\jobname-pw.ind}{\input{\jobname-pw.ind}}{}

\end{document}

      