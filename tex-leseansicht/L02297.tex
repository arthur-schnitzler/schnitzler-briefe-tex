%% latex-korrekturansicht-vorspann.tex
%% Vorspann für die Korrekturansicht.
%% Lädt die gemeinsame Datei latex-vorspann.tex mit gesetztem Schalter.

\newif\ifkorrekturansicht
\korrekturansichttrue

\input{../tex-inputs/latex-vorspann}


\section[Hugo von Hofmannsthal an Arthur Schnitzler, 17. 8. {[}1918{]}]{L02297 Hugo von Hofmannsthal an Arthur Schnitzler, 17. 8. {[}1918{]}}
\nopagebreak\mylabel{L02297v}
\rehead{ }\normalsize\beginnumbering\briefempfaengerindex{Schnitzler, Arthur@\textsc{Schnitzler, Arthur}!zzzHofmannsthal, Hugo von@\emph{von Hugo von Hofmannsthal}!1918-08-171@{17. 8. {[}1918{]}}|(be}
\toendnotes[C]{\smallbreak\pagebreak[2]}\Standort{CUL, Schnitzler, B 43.}
\physDesc{Briefkarte, 960 Zeichen
\newline{}Handschrift: schwarze Tinte, deutsche Kurrent
\newline{}Schnitzler: mit Bleistift die Jahreszahl ergänzt: »18« und beschriftet: »Hugo« 
\newline{}Ordnung: 1) mit Bleistift von Frieda
                                    Pollak\pwindex{Pollak, Frieda 08.12.1881 – 13.07.1937@\textsc{Pollak, Frieda} (08.12.1881 – 13.07.1937), \emph{Sekretär/Sekretärin}|pw} (?) mit dem Buchstaben »A«
                                 (Abgeschrieben/Abschrift) gekennzeichnet  2) mit Bleistift von unbekannter Hand nummeriert: »\strikeout{348}« 3) mit Bleistift von unbekannter Hand nummeriert:
                                    »359«}
\buchAbdrucke{\weitereDrucke{Hugo von Hofmannsthal, Arthur Schnitzler: \emph{Briefwechsel}. Frankfurt am Main: \emph{S. Fischer} 1964, S. 282.} }\toendnotes[C]{\smallbreak}
\pstart
           \raggedleft{}{\pb}Auſſee\oindex{Bad Aussee@\textbf{Bad Aussee}, \emph{P.PPLA3}|pw}{ }17 VIII. \pend
           
\pstart{}mein lieber Arthur\pend\vspace{0.5em}
\pstart
           Ihr \label{K_L02297-1v}\edtext{Buch\pwindex{Casanovas Heimfahrt@\emph{Casanovas Heimfahrt}|pwv}}{\lemma{\textnormal{\emph{Buch}}}\Cendnote{\textnormal{\emph{Casanovas Heimfahrt}\pwindex{Casanovas Heimfahrt@\emph{Casanovas Heimfahrt}|pwk} ist nicht unter den
                  Büchern Hofmannsthals\pwindex{Hofmannsthal, Hugo von 1874-02-01 – 1929-07-15@\textsc{Hofmannsthal, Hugo von} (1874-02-01 – 1929-07-15), \emph{Schriftsteller/Schriftstellerin}|pwk} überliefert.}}}\label{K_L02297-1}
               kam an, u. wenn auch nicht durch Sie ſondern durch Fiſcher\orgindex{S. Fischer Verlag@S. Fischer Verlag|pw}, ſo iſt es ja doch ein Gruß von Ihnen. Ich las es in einem Zug
               durch, es iſt ja die Hand eines Meiſters, die einen raſch u. leicht vorwärts führt,
               alles iſt von einer ſicheren Kunſt, was da ſteht und was nicht da ſteht, die
               Verknüpfungen, die Antitheſen u. der Ausgang. Wie man bei einem Freunde über das
               Künstleriſche hinaus noch nach {\pb}einem Mehr ſucht, ſo war mir hier ſeltſam ein alter Zug wie aus einem Jugendporträt
               von Ihnen, nun aufs neue bewuſstlos ſich accentuierend: die Spielernatur des
               Menſchen, den Sie darſtellen. Er ſpielt eine Partie mit dem Schickſal, haſardiert
               frech, und verliert.\hspace*{1.5em}– Ich wuſste von Ihnen halbwegs
               in dieſen Monaten; durch die Erſchwerung der Verbindungen iſt man ja mehr
               auseinandergehalten, als lebte man in verſchiedenen Städten. Gegenſeitige Achtung u.
               Zuneigung, und viele viele Erinnerungen halten uns aber zuſa{\geminationm}en.\pend
           \pstart Ihr \spacefill\mbox{Hugo.}\pend{}\selectlanguage{ngerman}\endnumbering\briefempfaengerindex{Schnitzler, Arthur@\textsc{Schnitzler, Arthur}!zzzHofmannsthal, Hugo von@\emph{von Hugo von Hofmannsthal}!1918-08-171@{17. 8. {[}1918{]}}|)be}\mylabel{L02297h}  \normalsize

\doendnotes{C}
\bigskip
\vfill

\clearpage

\footnotesize

\lohead{\textsc{register}}

% Definiere theindex-Environment komplett neu ohne reledmac
\makeatletter
\renewenvironment{theindex}{%
  \section*{\indexname}%
  \setlength{\parindent}{0pt}%
  \setlength{\parskip}{0pt plus 0.3pt}%
  \let\item\@idxitem
}{%
  \clearpage
}
\makeatother

\IfFileExists{\jobname-pw.ind}{\input{\jobname-pw.ind}}{}

\end{document}

      