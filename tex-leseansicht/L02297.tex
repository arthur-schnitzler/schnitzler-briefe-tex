\input{../tex-inputs/latex-pdf-vorspann}
\begin{center}
            \textcolor{red}{ENTWURF. ENTZIFFERUNG NOCH NICHT KORREKTURGELESEN}
                      \end{center}
            
               \section[Hugo von Hofmannsthal an Arthur Schnitzler, 17. 8. {[}1918{]}]{ Hugo von Hofmannsthal an Arthur Schnitzler, 17. 8. {[}1918{]}}\nopagebreak\mylabel{v}\rehead{ }\begin{ledgroupsized}[t]{13cm}\normalsize\beginnumbering\briefempfaengerindex{Schnitzler, Arthur@\textsc{Schnitzler, Arthur}!zzzHofmannsthal, Hugo von@\emph{von Hugo von Hofmannsthal}!1918-08-171@{17. 8. {[}1918{]}}|(be} \toendnotes[C]{\smallbreak\pagebreak[2]} \Standort{CUL, Schnitzler, B 43.}
\physDesc{Briefkarte
\newline{}Handschrift: schwarze Tinte, deutsche Kurrent
\newline{}Schnitzler: mit Bleistift die Jahreszahl ergänzt: »18« und beschriftet: »Hugo« \newline{}Ordnung: 1) mit Bleistift von Frieda Pollak\pwindex{Pollak, Frieda 08.12.1881 – 13.07.1937@\textsc{Pollak, Frieda} (08.12.1881 – 13.07.1937), \emph{Sekretärin}|pw} (?) mit dem Buchstaben »A« (Abgeschrieben/Abschrift) gekennzeichnet 2) mit Bleistift von unbekannter Hand nummeriert: »\strikeout{348}«3) mit Bleistift von unbekannter Hand nummeriert: »359«}\buchAbdrucke{\weitereDrucke{Hugo von Hofmannsthal, Arthur Schnitzler: \emph{Briefwechsel}. Hg. Therese Nickl und Heinrich Schnitzler. Frankfurt am Main: \emph{S. Fischer} 1964, S. 282.} }\toendnotes[C]{\smallbreak}\pstart
           \raggedleft{}{\pb}Auſſee\oindex{Bad Aussee@\textbf{Bad Aussee}|pw}{ }17 VIII. \pend
           \pstart{}mein lieber Arthur\pend\pstart
           Ihr \label{K_L02297_1v}\edtext{Buch\pwindex{Schnitzler, Arthur 15.05.1862 – 21.10.1931@\textsc{Schnitzler, Arthur} (15.05.1862 – 21.10.1931), \emph{Schriftsteller, Mediziner}!Casanovas Heimfahrt1.7.1918 – 1.9.1918@\strich\emph{Casanovas Heimfahrt} {[}1.7.1918 – 1.9.1918{]}|pwv}}{\lemma{\textnormal{\emph{Buch}}}\Cendnote{\textnormal{\emph{Casanovas
                     Heimfahrt}\pwindex{Schnitzler, Arthur 15.05.1862 – 21.10.1931@\textsc{Schnitzler, Arthur} (15.05.1862 – 21.10.1931), \emph{Schriftsteller, Mediziner}!Casanovas Heimfahrt1.7.1918 – 1.9.1918@\strich\emph{Casanovas Heimfahrt} {[}1.7.1918 – 1.9.1918{]}|pwk} ist nicht unter den Büchern Hofmannsthals\pwindex{Hofmannsthal, Hugo von 01.02.1874 – 15.07.1929@\textsc{Hofmannsthal, Hugo von} (01.02.1874 – 15.07.1929), \emph{Schriftsteller}|pwk} überliefert.}}}\label{K_L02297_1h} kam an, u. wenn auch nicht durch Sie
               ſondern durch Fiſcher\orgindex{S. Fischer Verlag@S. Fischer Verlag|pw}, ſo iſt es ja doch ein Gruß
               von Ihnen. Ich las es in einem Zug durch, es iſt ja die Hand eines Meiſters, die
               einen raſch u. leicht vorwärts führt, alles iſt von einer ſicheren Kunſt, was da
               ſteht und was nicht da ſteht, die Verknüpfungen, die Antitheſen u. der Ausgang. Wie
               man bei einem Freunde über das Künstleriſche hinaus noch nach {\pb}einem Mehr ſucht, ſo war mir hier
               ſeltſam ein alter Zug wie aus einem Jugendporträt von Ihnen, nun aufs neue bewuſstlos
               ſich accentuierend: die Spielernatur des Menſchen, den Sie darſtellen. Er ſpielt eine
               Partie mit dem Schickſal, haſardiert frech, und verliert.\hspace*{1.5em}– Ich wuſste von Ihnen halbwegs in dieſen Monaten; durch die
               Erſchwerung der Verbindungen iſt man ja mehr auseinandergehalten, als lebte man in
               verſchiedenen Städten. Gegenſeitige Achtung u. Zuneigung, und viele viele
               Erinnerungen halten uns aber zuſa{\geminationm}en.\pend
           \pstart Ihr \spacefill\mbox{Hugo.}\pend{}\endnumbering\briefempfaengerindex{Schnitzler, Arthur@\textsc{Schnitzler, Arthur}!zzzHofmannsthal, Hugo von@\emph{von Hugo von Hofmannsthal}!1918-08-171@{17. 8. {[}1918{]}}|)be}\mylabel{h}\end{ledgroupsized}  \newcommand{\dateiname}{L02297}\newcommand{\titel}{Hugo von Hofmannsthal an Arthur Schnitzler, 17. 8. [1918]}\newcommand{\editorInnen}{Martin Anton Müller und Gerd-Hermann Susen}\input{../tex-inputs/latex-pdf-abspann}
      