%% latex-leseansicht-vorspann.tex
%% Vorspann für die Leseansicht.
%% Lädt die gemeinsame Datei latex-vorspann.tex mit nicht gesetztem Schalter.

\newif\ifkorrekturansicht
\korrekturansichtfalse

\input{../tex-inputs/latex-vorspann}


         
         \renewcommand{\erwaehntePersonen}{Personen: Frieda Pollak}
         \renewcommand{\erwaehnteInstitutionen}{Institutionen: S. Fischer Verlag}
         \renewcommand{\erwaehnteOrte}{Orte: Bad Aussee, Wien}
         \renewcommand{\erwaehnteWerke}{Werke: Casanovas Heimfahrt}
               \section[Hugo von Hofmannsthal an Arthur Schnitzler, 17. 8. {[}1918{]}]{ Hugo von Hofmannsthal an Arthur Schnitzler, 17. 8. {[}1918{]}}\nopagebreak\mylabel{v}\rehead{ }\begin{ledgroupsized}[t]{13cm}\normalsize\beginnumbering \toendnotes[C]{\smallbreak\pagebreak[2]} \Standort{CUL, Schnitzler, B 43.}
\physDesc{Briefkarte, 960 Zeichen
\newline{}Handschrift: schwarze Tinte, deutsche Kurrent
\newline{}Schnitzler: mit Bleistift die Jahreszahl ergänzt: »18« und beschriftet: »Hugo« 
\newline{}Ordnung: 1) mit Bleistift von Frieda
                                    Pollak\pwindex{Pollak, Frieda 08.12.1881 – 13.07.1937@\textsc{Pollak, Frieda} (08.12.1881 – 13.07.1937), \emph{Sekretärin}|pw} (?) mit dem Buchstaben »A«
                                 (Abgeschrieben/Abschrift) gekennzeichnet  2) mit Bleistift von unbekannter Hand nummeriert: »\strikeout{348}« 3) mit Bleistift von unbekannter Hand nummeriert:
                                    »359«}\buchAbdrucke{\weitereDrucke{Hugo von Hofmannsthal, Arthur Schnitzler: \emph{Briefwechsel}. Hg. Therese Nickl und Heinrich Schnitzler. Frankfurt am Main: \emph{S. Fischer} 1964, S. 282.} }\toendnotes[C]{\smallbreak}\pstart
           \raggedleft{}{\pb}Auſſee\oindex{Bad Aussee@\textbf{Bad Aussee}|pw}{ }17 VIII. \pend
           \pstart{}mein lieber Arthur\pend\pstart
           Ihr \label{K_L02297_1v}\edtext{Buch\pwindex{Schnitzler, Arthur 15.05.1862 – 21.10.1931@\textsc{Schnitzler, Arthur} (15.05.1862 – 21.10.1931), \emph{Schriftsteller, Mediziner}!Casanovas Heimfahrt1.7.1918 – 1.9.1918@\strich\emph{Casanovas Heimfahrt} {[}1.7.1918 – 1.9.1918{]}|pwv}}{\lemma{\textnormal{\emph{Buch}}}\Cendnote{\textnormal{\emph{Casanovas Heimfahrt}\pwindex{Schnitzler, Arthur 15.05.1862 – 21.10.1931@\textsc{Schnitzler, Arthur} (15.05.1862 – 21.10.1931), \emph{Schriftsteller, Mediziner}!Casanovas Heimfahrt1.7.1918 – 1.9.1918@\strich\emph{Casanovas Heimfahrt} {[}1.7.1918 – 1.9.1918{]}|pwk} ist nicht unter den
                  Büchern Hofmannsthals\pwindex{Hofmannsthal, Hugo von 1874-02-01 – 1929-07-15@\textsc{Hofmannsthal, Hugo von} (1874-02-01 – 1929-07-15), \emph{Schriftsteller}|pwk} überliefert.}}}\label{K_L02297_1h}
               kam an, u. wenn auch nicht durch Sie ſondern durch Fiſcher\orgindex{S. Fischer Verlag@S. Fischer Verlag|pw}, ſo iſt es ja doch ein Gruß von Ihnen. Ich las es in einem Zug
               durch, es iſt ja die Hand eines Meiſters, die einen raſch u. leicht vorwärts führt,
               alles iſt von einer ſicheren Kunſt, was da ſteht und was nicht da ſteht, die
               Verknüpfungen, die Antitheſen u. der Ausgang. Wie man bei einem Freunde über das
               Künstleriſche hinaus noch nach {\pb}einem Mehr ſucht, ſo war mir hier ſeltſam ein alter Zug wie aus einem Jugendporträt
               von Ihnen, nun aufs neue bewuſstlos ſich accentuierend: die Spielernatur des
               Menſchen, den Sie darſtellen. Er ſpielt eine Partie mit dem Schickſal, haſardiert
               frech, und verliert.\hspace*{1.5em}– Ich wuſste von Ihnen halbwegs
               in dieſen Monaten; durch die Erſchwerung der Verbindungen iſt man ja mehr
               auseinandergehalten, als lebte man in verſchiedenen Städten. Gegenſeitige Achtung u.
               Zuneigung, und viele viele Erinnerungen halten uns aber zuſa{\geminationm}en.\pend
           \pstart Ihr \spacefill\mbox{Hugo.}\pend{}
         
         \endnumbering\mylabel{h}\end{ledgroupsized}  \newcommand{\dateiname}{L02297}\newcommand{\titel}{Hugo von Hofmannsthal an Arthur Schnitzler, 17. 8. [1918]}\newcommand{\editorInnen}{Martin Anton Müller und Gerd-Hermann Susen}%% latex-leseansicht-abspann.tex
%% Abspann für die Leseansicht.
%% Der Schalter \ifkorrekturansicht ist bereits durch den Vorspann gesetzt.

%% latex-abspann.tex
%% Gemeinsamer Abspann für Korrekturansicht und Leseansicht.
%% Setzt den Schalter \ifkorrekturansicht voraus (gesetzt in den
%% einbindenden Dateien latex-korrekturansicht-abspann.tex bzw.
%% latex-leseansicht-abspann.tex).
%% ---------------------------------------------------------------

\normalsize

% Das esempio-Environment wird nur in der Leseansicht benötigt
\ifkorrekturansicht\else
\newenvironment{esempio}[3]%
{
    \vspace{1.5ex}
    \rlap{\underline{#1}}
    \par
    \setlength{\parindent}{0cm}
    \nopagebreak
    \leftskip=#2cm
    \rightskip=#3cm
}
{
    \par
}
\fi

\doendnotes{C}
\bigskip
\vfill

\clearpage

\footnotesize

\ifkorrekturansicht
  \lohead{\textsc{register}}
\fi

% theindex-Environment neu definieren ohne reledmac
\makeatletter
\renewenvironment{theindex}{%
  \ifkorrekturansicht
    \section*{\indexname}%
  \else
    \subsubsection*{Index der erwähnten Entitäten}%
  \fi
  \setlength{\parindent}{0pt}%
  \setlength{\parskip}{0pt plus 0.3pt}%
  \let\item\@idxitem
}{%
  \ifkorrekturansicht\clearpage\fi
}
\makeatother

\IfFileExists{\jobname-pw.ind}{\input{\jobname-pw.ind}}{}

% Quellenangabe nur in der Leseansicht
\ifkorrekturansicht\else
% Fallback-Definitionen, falls die .tex-Datei \titel etc. nicht gesetzt hat
\providecommand{\titel}{}
\providecommand{\editorInnen}{}
\providecommand{\dateiname}{\jobname}

\vspace{3cm}

\vfill

\footnotesize
\textsc{Quelle}: \titel. Herausgegeben von {\editorInnen}. In: \emph{Arthur Schnitzler: Briefwechsel mit Autorinnen und Autoren}.
 Digitale Edition, https://schnitzler-briefe.acdh.oeaw.ac.at/{\dateiname}.html (Stand \today)
\fi

\end{document}


      