%% latex-leseansicht-vorspann.tex
%% Vorspann für die Leseansicht.
%% Lädt die gemeinsame Datei latex-vorspann.tex mit nicht gesetztem Schalter.

\newif\ifkorrekturansicht
\korrekturansichtfalse

\input{../tex-inputs/latex-vorspann}


\section[ Felix Salten an Arthur Schnitzler, 23. 8. 1906]{L03433 Felix Salten an Arthur Schnitzler,  23. 8. 1906}
\nopagebreak\mylabel{L03433v}
\rehead{ }\normalsize\beginnumbering\briefempfaengerindex{Schnitzler, Arthur@\textsc{Schnitzler, Arthur}!zzzSalten, Felix@\emph{von Felix Salten}!1906-08-231@{23. 8. 1906}|(be}
\toendnotes[C]{\smallbreak\pagebreak[2]}
\correspDesc{Versand  durch Felix Salten am 23. 8. 1906 in Bansin
\newline{}Erhalt  durch Arthur Schnitzler am 25. 8. 1906 in Wien}\toendnotes[C]{\smallbreak}
\Standort{CUL, Schnitzler, B 89, B 1.}
\physDesc{Postkarte, 967 Zeichen
\newline{}Handschrift: schwarze Tinte, lateinische Kurrent
\newline{}Versand: Stempel: »\nobreak{}\oindex{Bansin@\textbf{Bansin}|pwk}Seebad Ba\textcolor{gray}{nsin}, 23. 8. 06\nobreak{}«. Stempel: »\nobreak{}\oindex{XVIII., Währing@\textbf{XVIII., Währing}, \emph{Verwaltungsgebiet}|pwk}18/\textsubscript{1} Wien 110, 25. VIII. 06, X, Bestellt\nobreak{}«.  
\newline{}Ordnung: mit Bleistift von unbekannter Hand nummeriert: »224« }\toendnotes[C]{\smallbreak}\pstart{}{\pb}Herrn D\textsuperscript{r} Arthur Schnitzler\pend{}\pstart{}Wien\oindex{Wien@\textbf{Wien}, \emph{Verwaltungsgebiet}|pw}\pend{}\pstart{}XVIII. Spoettelgasse 7\oindex{Wien@\textbf{Wien}!XVIII., Währing@\textbf{XVIII., Währing}!Edmund-Weiß-Gasse 7@\textbf{Edmund-Weiß-Gasse 7}, \emph{Wohngebäude}|pw}\pend{}{\bigskip}\vspace{1em}
\pstart
           \raggedleft{}{\pb}Bansin\oindex{Bansin@\textbf{Bansin}|pw}, 23. VIII. 06.\pend
           \vspace{0.5em}
\pstart
           Lieber, schönen Dank für Ihre Karten aus \label{K_L03433-1v}\edtext{Weimar\oindex{Weimar@\textbf{Weimar}, \emph{Verwaltungsgebiet}|pw}}{\lemma{\textnormal{\emph{Weimar}}}\Cendnote{\textnormal{Schnitzlers Aufenthalt in Weimar\oindex{Weimar@\textbf{Weimar}, \emph{Verwaltungsgebiet}|pwk} fand zwischen 12. 8. 1906 und 16. 8. 1906 statt.}}}\label{K_L03433-1}. Wir bleiben noch
               ca 10–12 Tage hier\oindex{Bansin@\textbf{Bansin}|pwv}, gehen
               dann nach Lübeck\oindex{Lübeck@\textbf{Lübeck}, \emph{Hauptstadt}|pw} u. Hamburg\oindex{Hamburg@\textbf{Hamburg}|pw}, dann nach Weimar\oindex{Weimar@\textbf{Weimar}, \emph{Verwaltungsgebiet}|pw}
               und Eisenach\oindex{Eisenach@\textbf{Eisenach}|pw}. Zuletzt begleitet mich Otti\pwindex{Salten, Ottilie 7.\,3.\,1868 Prag – 22.\,6.\,1942 Zürich@\textsc{Salten, Ottilie} (7.\,3.\,1868 Prag – 22.\,6.\,1942 Zürich), \emph{Schauspielerin}|pw} nach Dresden\oindex{Dresden@\textbf{Dresden}|pw}. Ich bin gegen den 10. Septb. in Wien\oindex{Wien@\textbf{Wien}, \emph{Verwaltungsgebiet}|pw}, und fahre – wahrscheinlich – zu den
               Flottenmanövern in der Adria\oindex{Adriatisches Meer@\textbf{Adriatisches Meer}|pw}. Von da noch ein
               paar Tage Venedig\oindex{Venedig@\textbf{Venedig}|pw}, dann \label{K_L03433-2v}\edtext{definitiv Wien\oindex{Wien@\textbf{Wien}, \emph{Verwaltungsgebiet}|pw}}{\lemma{\textnormal{\emph{definitiv Wien}}}\Cendnote{\textnormal{Salten\pwindex{Salten, Felix 6.\,9.\,1869 Budapest – 8.\,10.\,1945 Zürich@\textsc{Salten, Felix} (6.\,9.\,1869 Budapest – 8.\,10.\,1945 Zürich), \emph{Schriftsteller, Journalist, Chefredakteur}|pwk} hatte bereits vor dem Sommer eine Vertragsauflösung mit \emph{Ullstein}\orgindex{Ullstein Verlag@Ullstein Verlag|pwk} bewirkt, vgl. XXXX Auszeichnungsfehler: Dokument L03430 nicht gefunden. Der seit Jahresbeginn dauernde Aufenthalt in Berlin\oindex{Berlin@\textbf{Berlin}, \emph{Hauptstadt}|pwk} wurde Anfang September beendet und der Wohnsitz wieder nach Wien\oindex{Wien@\textbf{Wien}, \emph{Verwaltungsgebiet}|pwk} verlegt, vgl. A. S.: \emph{Tagebuch}, 2. 8. 1906.}}}\label{K_L03433-2}. Wenn das Wetter schön bleibt,
               könnten Sie wegen eines Tennisplatzes (Vormittag) etwas veranlaßen. Mein Schwager Richard\pwindex{Metzl, Richard 20.\,4.\,1870 Prag – 31.\,10.\,1941 Paris@\textsc{Metzl, Richard} (20.\,4.\,1870 Prag – 31.\,10.\,1941 Paris), \emph{Regisseur, Schauspieler, Theatersekretär}|pw}, der in Reichenau\oindex{Reichenau an der Rax@\textbf{Reichenau an der Rax}, \emph{Verwaltungsgebiet}|pw} mit uns spielte, spielt jetzt noch schärfer und wird ein guter
               Partner sein. Otti\pwindex{Salten, Ottilie 7.\,3.\,1868 Prag – 22.\,6.\,1942 Zürich@\textsc{Salten, Ottilie} (7.\,3.\,1868 Prag – 22.\,6.\,1942 Zürich), \emph{Schauspielerin}|pw} übersiedelt, Sack und Pack,
               am 14. September. Wir sind unsere Wohnung in der Kantstraße\oindex{Kantstraße@\textbf{Kantstraße}, \emph{Straße}|pw} los; müßen sie am 14. schon räumen. Eine Chance! Denn ich hätte sonst die
               ganze Miete für die restliche Vertragszeit, also 5000.– M. vor meiner Abreise
               deponiren müßen, u. hätte dann wer weiß wie viel verloren. Auf bald. Herzliche Grüße
               von uns zu Ihnen. {\\}Ihr {\\}\spacefill\mbox{Salten}\pend
           \selectlanguage{ngerman}\endnumbering\briefempfaengerindex{Schnitzler, Arthur@\textsc{Schnitzler, Arthur}!zzzSalten, Felix@\emph{von Felix Salten}!1906-08-231@{23. 8. 1906}|)be}\mylabel{L03433h}  \newcommand{\dateiname}{L03433}\newcommand{\titel}{Felix Salten an Arthur Schnitzler, 23. 8. 1906}\newcommand{\editorInnen}{Martin Anton Müller und Laura Untner}%% latex-leseansicht-abspann.tex
%% Abspann für die Leseansicht.
%% Der Schalter \ifkorrekturansicht ist bereits durch den Vorspann gesetzt.

%% latex-abspann.tex
%% Gemeinsamer Abspann für Korrekturansicht und Leseansicht.
%% Setzt den Schalter \ifkorrekturansicht voraus (gesetzt in den
%% einbindenden Dateien latex-korrekturansicht-abspann.tex bzw.
%% latex-leseansicht-abspann.tex).
%% ---------------------------------------------------------------

\normalsize

% Das esempio-Environment wird nur in der Leseansicht benötigt
\ifkorrekturansicht\else
\newenvironment{esempio}[3]%
{
    \vspace{1.5ex}
    \rlap{\underline{#1}}
    \par
    \setlength{\parindent}{0cm}
    \nopagebreak
    \leftskip=#2cm
    \rightskip=#3cm
}
{
    \par
}
\fi

\doendnotes{C}
\bigskip
\vfill

\clearpage

\footnotesize

\ifkorrekturansicht
  \lohead{\textsc{register}}
\fi

% theindex-Environment neu definieren ohne reledmac
\makeatletter
\renewenvironment{theindex}{%
  \ifkorrekturansicht
    \section*{\indexname}%
  \else
    \subsubsection*{Index der erwähnten Entitäten}%
  \fi
  \setlength{\parindent}{0pt}%
  \setlength{\parskip}{0pt plus 0.3pt}%
  \let\item\@idxitem
}{%
  \ifkorrekturansicht\clearpage\fi
}
\makeatother

\IfFileExists{\jobname-pw.ind}{\input{\jobname-pw.ind}}{}

% Quellenangabe nur in der Leseansicht
\ifkorrekturansicht\else
% Fallback-Definitionen, falls die .tex-Datei \titel etc. nicht gesetzt hat
\providecommand{\titel}{}
\providecommand{\editorInnen}{}
\providecommand{\dateiname}{\jobname}

\vspace{3cm}

\vfill

\footnotesize
\textsc{Quelle}: \titel. Herausgegeben von {\editorInnen}. In: \emph{Arthur Schnitzler: Briefwechsel mit Autorinnen und Autoren}.
 Digitale Edition, https://schnitzler-briefe.acdh.oeaw.ac.at/{\dateiname}.html (Stand \today)
\fi

\end{document}


