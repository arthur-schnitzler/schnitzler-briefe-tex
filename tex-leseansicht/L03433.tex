%% latex-leseansicht-vorspann.tex
%% Vorspann für die Leseansicht.
%% Lädt die gemeinsame Datei latex-vorspann.tex mit nicht gesetztem Schalter.

\newif\ifkorrekturansicht
\korrekturansichtfalse

\input{../tex-inputs/latex-vorspann}

\begin{center}
            \textcolor{red}{ENTWURF, NICHT FERTIG KORRIGIERT}
                      \end{center}
            
         
         \renewcommand{\erwaehntePersonen}{Personen: Richard Metzl, Ottilie Salten}
         \renewcommand{\erwaehnteOrte}{Orte: Adriatisches Meer, Bansin, Dresden, Edmund-Weiß-Gasse, Eisenach, Hamburg, Kantstraße, Lübeck, Reichenau an der Rax, Venedig, Weimar, Wien}
         \renewcommand{\erwaehnteWerke}{}
               \section[Felix Salten an Arthur Schnitzler, 23. 8. 1906]{ Felix Salten an Arthur Schnitzler, 23. 8. 1906}\nopagebreak\mylabel{v}\rehead{ }\begin{ledgroupsized}[t]{13cm}\normalsize\beginnumbering \toendnotes[C]{\smallbreak\pagebreak[2]} \Standort{CUL, Schnitzler, B 89, B 1.}
\physDesc{Postkarte, 970 Zeichen
\newline{}Handschrift: schwarze Tinte, lateinische Kurrent
\newline{}Versand: Stempel: »\nobreak{}\oindex{Bansin@\textbf{Bansin}|pwk}Seebad Ba\textcolor{gray}{n}{[}sin{]}, 23. 8. 06\nobreak{}«.  
\newline{}Ordnung: mit Bleistift von unbekannter Hand nummeriert:
                                    »224« }\toendnotes[C]{\smallbreak}\pstart{}{\pb}Herrn D\textsuperscript{r} Arthur Schnitzler\pend{}\pstart{}Wien\oindex{Wien@\textbf{Wien}|pw}\pend{}\pstart{}XVIII. Spoettelgasse 7\oindex{Edmund-Weiss-Gasse@\textbf{Edmund-Weiß-Gasse}|pw}\pend{}{\bigskip}\pstart
           \raggedleft{}{\pb}Bansin\oindex{Bansin@\textbf{Bansin}|pw}, 23. VIII. 06.
               \pend
           \pstart
           Lieber, schönen Dank für Ihre Karten aus Weimar\oindex{Weimar@\textbf{Weimar}|pw}. Wir bleiben noch ca 10–12 Tage hier\oindex{Bansin@\textbf{Bansin}|pwv}, gehen dann nach Lübeck\oindex{Luebeck@\textbf{Lübeck}|pw} u. Hamburg\oindex{Hamburg@\textbf{Hamburg}|pw}, dann
               nach Weimar\oindex{Weimar@\textbf{Weimar}|pw} und Eisenach\oindex{Eisenach@\textbf{Eisenach}|pw}. Zuletzt begleitet mich Otti\pwindex{Salten, Ottilie 07.03.1868 – 22.06.1942@\textsc{Salten, Ottilie} (07.03.1868 – 22.06.1942), \emph{Schauspielerin}|pw}
               nach Dresden\oindex{Dresden@\textbf{Dresden}|pw}. Ich bin gegen den 10.
                  Septb. in Wien\oindex{Wien@\textbf{Wien}|pw}, und fahre –
               wahrscheinlich – zu den Flottenmanövern in der Adria\oindex{Adriatisches Meer@\textbf{Adriatisches Meer}|pw}. Von da noch ein paar Tage Venedig\oindex{Venedig@\textbf{Venedig}|pw}, dann definitiv Wien\oindex{Wien@\textbf{Wien}|pw}. Wenn das
               Wetter schön bleibt, könnten Sie wegen eines Tennisplatzes (Vormittag) etwas
               veranlaßen. Mein Schwager Richard\pwindex{Metzl, Richard 20.04.1870 – 31.10.1941@\textsc{Metzl, Richard} (20.04.1870 – 31.10.1941), \emph{Regisseur, Schauspieler, Theatersekretär}|pw}, der in Reichenau\oindex{Reichenau an der Rax@\textbf{Reichenau an der Rax}|pw} mit uns spielte, spielt jetzt noch
               schärfer und wird ein guter Partner sein. Otti\pwindex{Salten, Ottilie 07.03.1868 – 22.06.1942@\textsc{Salten, Ottilie} (07.03.1868 – 22.06.1942), \emph{Schauspielerin}|pw}
               übersiedelt, Sack und Pack, am 14. September. Wir sind unsere Wohnung in
               der Kantstraße\oindex{Kantstrasse@\textbf{Kantstraße}|pw} los; müßen sie am 14. schon
               räumen. Eine Chance! Denn ich hätte sonst die ganze Miete für die restliche
               Vertragszeit, also 5000.– M. vor meiner Abreise deponiren müßen, u. hätte dann wer
               weiß wie viel verloren.\pend
           \pstart
           Auf bald.\pend
           \pstart Herzliche Grüße von uns zu Ihnen. Ihr \spacefill\mbox{Salten}\pend{}
         
         \endnumbering\mylabel{h}\end{ledgroupsized}\begin{anhang}\end{anhang}\newcommand{\dateiname}{L03433}\newcommand{\titel}{Felix Salten an Arthur Schnitzler, 23. 8. 1906}\newcommand{\editorInnen}{Martin Anton Müller und Laura Untner}%% latex-leseansicht-abspann.tex
%% Abspann für die Leseansicht.
%% Der Schalter \ifkorrekturansicht ist bereits durch den Vorspann gesetzt.

%% latex-abspann.tex
%% Gemeinsamer Abspann für Korrekturansicht und Leseansicht.
%% Setzt den Schalter \ifkorrekturansicht voraus (gesetzt in den
%% einbindenden Dateien latex-korrekturansicht-abspann.tex bzw.
%% latex-leseansicht-abspann.tex).
%% ---------------------------------------------------------------

\normalsize

% Das esempio-Environment wird nur in der Leseansicht benötigt
\ifkorrekturansicht\else
\newenvironment{esempio}[3]%
{
    \vspace{1.5ex}
    \rlap{\underline{#1}}
    \par
    \setlength{\parindent}{0cm}
    \nopagebreak
    \leftskip=#2cm
    \rightskip=#3cm
}
{
    \par
}
\fi

\doendnotes{C}
\bigskip
\vfill

\clearpage

\footnotesize

\ifkorrekturansicht
  \lohead{\textsc{register}}
\fi

% theindex-Environment neu definieren ohne reledmac
\makeatletter
\renewenvironment{theindex}{%
  \ifkorrekturansicht
    \section*{\indexname}%
  \else
    \subsubsection*{Index der erwähnten Entitäten}%
  \fi
  \setlength{\parindent}{0pt}%
  \setlength{\parskip}{0pt plus 0.3pt}%
  \let\item\@idxitem
}{%
  \ifkorrekturansicht\clearpage\fi
}
\makeatother

\IfFileExists{\jobname-pw.ind}{\input{\jobname-pw.ind}}{}

% Quellenangabe nur in der Leseansicht
\ifkorrekturansicht\else
% Fallback-Definitionen, falls die .tex-Datei \titel etc. nicht gesetzt hat
\providecommand{\titel}{}
\providecommand{\editorInnen}{}
\providecommand{\dateiname}{\jobname}

\vspace{3cm}

\vfill

\footnotesize
\textsc{Quelle}: \titel. Herausgegeben von {\editorInnen}. In: \emph{Arthur Schnitzler: Briefwechsel mit Autorinnen und Autoren}.
 Digitale Edition, https://schnitzler-briefe.acdh.oeaw.ac.at/{\dateiname}.html (Stand \today)
\fi

\end{document}


      