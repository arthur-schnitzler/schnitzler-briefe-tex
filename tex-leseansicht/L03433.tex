%% latex-korrekturansicht-vorspann.tex
%% Vorspann für die Korrekturansicht.
%% Lädt die gemeinsame Datei latex-vorspann.tex mit gesetztem Schalter.

\newif\ifkorrekturansicht
\korrekturansichttrue

\input{../tex-inputs/latex-vorspann}


\section[ Felix Salten an Arthur Schnitzler, 23. 8. 1906]{L03433 Felix Salten an Arthur Schnitzler, 23. 8. 1906}
\nopagebreak\mylabel{L03433v}
\rehead{ }\normalsize\beginnumbering\briefempfaengerindex{Schnitzler, Arthur@\textsc{Schnitzler, Arthur}!zzzSalten, Felix@\emph{von Felix Salten}!1906-08-231@{23. 8. 1906}|(be}
\toendnotes[C]{\smallbreak\pagebreak[2]}\Standort{CUL, Schnitzler, B 89, B 1.}
\physDesc{Postkarte, 967 Zeichen
\newline{}Handschrift: schwarze Tinte, lateinische Kurrent
\newline{}Versand: Stempel: »\nobreak{}\oindex{Bansin@\textbf{Bansin}, \emph{P.PPL}|pwk}Seebad Ba\textcolor{gray}{nsin}, 23. 8. 06\nobreak{}«. Stempel: »\nobreak{}\oindex{XVIII., Waehring@\textbf{XVIII., Währing}, \emph{A.ADM3}|pwk}18/\textsubscript{1} Wien 110, 25. VIII. 06, X, Bestellt\nobreak{}«.  
\newline{}Ordnung: mit Bleistift von unbekannter Hand nummeriert: »224« }\toendnotes[C]{\smallbreak}\pstart{}{\pb}Herrn D\textsuperscript{r} Arthur Schnitzler\pend{}\pstart{}Wien\oindex{Wien@\textbf{Wien}, \emph{A.ADM2}|pw}\pend{}\pstart{}XVIII. Spoettelgasse 7\oindex{Edmund-Weiss-Gasse 7@\textbf{Edmund-Weiß-Gasse 7}, \emph{Wohngebäude (K.WHS)}|pw}\pend{}{\bigskip}\vspace{1em}
\pstart
           \raggedleft{}{\pb}Bansin\oindex{Bansin@\textbf{Bansin}, \emph{P.PPL}|pw}, 23. VIII. 06. \pend
           \vspace{0.5em}
\pstart
           Lieber, schönen Dank für Ihre Karten aus \label{K_L03433-1v}\edtext{Weimar\oindex{Weimar@\textbf{Weimar}, \emph{A.ADM3}|pw}}{\lemma{\textnormal{\emph{Weimar}}}\Cendnote{\textnormal{Schnitzlers Aufenthalt in Weimar\oindex{Weimar@\textbf{Weimar}, \emph{A.ADM3}|pwk} fand zwischen 12. 8. 1906 und 16. 8. 1906 statt.}}}\label{K_L03433-1}. Wir bleiben noch
               ca 10–12 Tage hier\oindex{Bansin@\textbf{Bansin}, \emph{P.PPL}|pwv}, gehen
               dann nach Lübeck\oindex{Luebeck@\textbf{Lübeck}, \emph{P.PPLA3}|pw} u. Hamburg\oindex{Hamburg@\textbf{Hamburg}, \emph{P.PPLA}|pw}, dann nach Weimar\oindex{Weimar@\textbf{Weimar}, \emph{A.ADM3}|pw}
               und Eisenach\oindex{Eisenach@\textbf{Eisenach}, \emph{P.PPL}|pw}. Zuletzt begleitet mich Otti\pwindex{Salten, Ottilie 07.03.1868 – 22.06.1942@\textsc{Salten, Ottilie} (07.03.1868 – 22.06.1942), \emph{Schauspieler/Schauspielerin}|pw} nach Dresden\oindex{Dresden@\textbf{Dresden}, \emph{P.PPLA}|pw}. Ich bin gegen den 10. Septb. in Wien\oindex{Wien@\textbf{Wien}, \emph{A.ADM2}|pw}, und fahre – wahrscheinlich – zu den
               Flottenmanövern in der Adria\oindex{Adriatisches Meer@\textbf{Adriatisches Meer}, \emph{Meer (N.MER)}|pw}. Von da noch ein
               paar Tage Venedig\oindex{Venedig@\textbf{Venedig}, \emph{P.PPLA}|pw}, dann \label{K_L03433-2v}\edtext{definitiv Wien\oindex{Wien@\textbf{Wien}, \emph{A.ADM2}|pw}}{\lemma{\textnormal{\emph{definitiv Wien}}}\Cendnote{\textnormal{Salten\pwindex{Salten, Felix 06.09.1869 – 08.10.1945@\textsc{Salten, Felix} (06.09.1869 – 08.10.1945), \emph{Schriftsteller/Schriftstellerin, Journalist/Journalistin, Chefredakteur/Chefredakteurin}|pwk} hatte bereits vor dem Sommer eine Vertragsauflösung mit \emph{Ullstein}\orgindex{Ullstein Verlag@Ullstein Verlag|pwk} bewirkt, vgl. Felix Salten an Arthur Schnitzler, 6. 7. 1906. Der seit Jahresbeginn dauernde Aufenthalt in Berlin\oindex{Berlin@\textbf{Berlin}, \emph{P.PPLC}|pwk} wurde Anfang September beendet und der Wohnsitz wieder nach Wien\oindex{Wien@\textbf{Wien}, \emph{A.ADM2}|pwk} verlegt, vgl. A. S.: \emph{Tagebuch}, 2. 8. 1906.}}}\label{K_L03433-2}. Wenn das Wetter schön bleibt,
               könnten Sie wegen eines Tennisplatzes (Vormittag) etwas veranlaßen. Mein Schwager Richard\pwindex{Metzl, Richard 20.04.1870 – 31.10.1941@\textsc{Metzl, Richard} (20.04.1870 – 31.10.1941), \emph{Regisseur/Regisseurin, Schauspieler/Schauspielerin, Theatersekretär/Theatersekretärin}|pw}, der in Reichenau\oindex{Reichenau an der Rax@\textbf{Reichenau an der Rax}, \emph{A.ADM3}|pw} mit uns spielte, spielt jetzt noch schärfer und wird ein guter
               Partner sein. Otti\pwindex{Salten, Ottilie 07.03.1868 – 22.06.1942@\textsc{Salten, Ottilie} (07.03.1868 – 22.06.1942), \emph{Schauspieler/Schauspielerin}|pw} übersiedelt, Sack und Pack,
               am 14. September. Wir sind unsere Wohnung in der Kantstraße\oindex{Kantstrasse@\textbf{Kantstraße}, \emph{Straße (K.STR)}|pw} los; müßen sie am 14. schon räumen. Eine Chance! Denn ich hätte sonst die
               ganze Miete für die restliche Vertragszeit, also 5000.– M. vor meiner Abreise
               deponiren müßen, u. hätte dann wer weiß wie viel verloren. Auf bald. Herzliche Grüße
               von uns zu Ihnen. {\\}Ihr {\\}\spacefill\mbox{Salten}\pend
           \selectlanguage{ngerman}\endnumbering\briefempfaengerindex{Schnitzler, Arthur@\textsc{Schnitzler, Arthur}!zzzSalten, Felix@\emph{von Felix Salten}!1906-08-231@{23. 8. 1906}|)be}\mylabel{L03433h}  \normalsize

\doendnotes{C}
\bigskip
\vfill

\clearpage

\footnotesize

\lohead{\textsc{register}}

% Definiere theindex-Environment komplett neu ohne reledmac
\makeatletter
\renewenvironment{theindex}{%
  \section*{\indexname}%
  \setlength{\parindent}{0pt}%
  \setlength{\parskip}{0pt plus 0.3pt}%
  \let\item\@idxitem
}{%
  \clearpage
}
\makeatother

\IfFileExists{\jobname-pw.ind}{\input{\jobname-pw.ind}}{}

\end{document}

      