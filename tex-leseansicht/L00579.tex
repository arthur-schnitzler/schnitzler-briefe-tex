%% latex-leseansicht-vorspann.tex
%% Vorspann für die Leseansicht.
%% Lädt die gemeinsame Datei latex-vorspann.tex mit nicht gesetztem Schalter.

\newif\ifkorrekturansicht
\korrekturansichtfalse

\input{../tex-inputs/latex-vorspann}


         
         \renewcommand{\erwaehntePersonen}{Personen:  ?? [Frau des Leuchtturmwärters],  ?? [Leuchtturmwärter],  ?? [Teenagertochter eines Leuchtturmwärters], Richard Beer-Hofmann, Paula Beer-Hofmann, Paul Goldmann, Hugo von Hofmannsthal, Henrik Ibsen, Peter Nansen}
         \renewcommand{\erwaehnteOrte}{Orte: Badehotel, Berlin, Göteborg, Kopenhagen, Nordkap, Oslo, Ostsee, Skodsborg, Trondheim, Ven, Wien}
         \renewcommand{\erwaehnteWerke}{Werke: Freiwild. Schauspiel in 3 Akten, Geschichte der beiden Liebespaare}
               \section[Arthur Schnitzler an Hugo von Hofmannsthal, 7. 8. 1896]{ Arthur Schnitzler an Hugo von Hofmannsthal, 7. 8. 1896}\nopagebreak\mylabel{v}\rehead{ }\begin{ledgroupsized}[t]{13cm}\normalsize\beginnumbering \toendnotes[C]{\smallbreak\pagebreak[2]} \Standort{FDH, Hs-30885,51.}
\physDesc{Brief, 2 Blätter, 7 Seiten (Auch das zweite Blatt von
                                Schnitzler datiert)
\newline{}Handschrift: Bleistift, deutsche Kurrent}\buchAbdrucke{\weitereDrucke{1) Hugo von Hofmannsthal, Arthur Schnitzler: \emph{Briefwechsel}. Hg. Therese Nickl und Heinrich Schnitzler. Frankfurt am Main: \emph{S. Fischer} 1964, S. 70–72.} \weitereDrucke{2) Arthur Schnitzler: \emph{Briefe 1875–1912}. Hg. Therese Nickl und Heinrich Schnitzler. Frankfurt am Main: \emph{S. Fischer} 1981, S. 290–292.} }\toendnotes[C]{\smallbreak}\pstart
           \raggedleft{}{\pb}\textsc{Skodsborg}\oindex{Skodsborg@\textbf{Skodsborg}|pw},
                            7. 8. 96\pend
           \pstart
           Lieber Hugo, ſeit So{\geminationn}tag bin ich mit Richard\pwindex{Beer-Hofmann, Richard 1866-07-11 – 1945-09-26@\textsc{Beer-Hofmann, Richard} (1866-07-11 – 1945-09-26), \emph{Schriftsteller}|pw} (und
                        Paula\pwindex{Beer-Hofmann, Paula 25.02.1879 – 30.10.1939@\textsc{Beer-Hofmann, Paula} (25.02.1879 – 30.10.1939)|pw}) zuſa{\geminationm}en; ſeit vorgeſtern iſt auch Paul Goldmann\pwindex{Goldmann, Paul 31.01.1865 – 25.09.1935@\textsc{Goldmann, Paul} (31.01.1865 – 25.09.1935), \emph{Schriftsteller, Journalist}|pw}
                    da, und wir ſind in einem angenehmen Hotel, am Meer, hinter den Häuſern gleich
                    ein wunderſchöner Wald mit Buchen und Tannen, im Wald kleine faſt verſteckte
                    Teiche, und we{\geminationn} man eine halbe Stunde weiter \substVorne{}\textsuperscript{\textcolor{gray}{läuft}}\substDazwischen{}geht\substHinten{}, das freundliche Thal mit lieben kleinen Häuſern und Ort\substVorne{}\textsuperscript{en}\substDazwischen{}ſchaften\substHinten{} (wo wir aber noch nie geweſen ſind). Heute Vormittag ſind wir nach
                    einer kleinen ſchwe{\pb}diſchen Inſel\oindex{Ven@\textbf{Ven}|pwv} hinübergeſegelt, wo nicht
                    viele Menſchen wohnen, ſind in dem netten Haus des Leuchtthurmwächters\pwindex{?? [Leuchtturmwaerter] @\textsc{?? [Leuchtturmwärter]}|pwv} geweſen, und wie wir
                    von dem niedern Thurm herunterſtiegen, fanden wir im Wohnzimmer ein leiſes
                    Harmonium, eine freundliche Hausfrau\pwindex{?? [Frau des Leuchtturmwaerters] @\textsc{?? [Frau des Leuchtturmwärters]}|pwv} und \substVorne{}\textsuperscript{eine}\substDazwischen{}im\substHinten{} Vorzimmer saſs die vierzehnjährige Tochter\pwindex{?? [Teenagertochter eines Leuchtturmwaerters] *~um 1882@\textsc{?? [Teenagertochter eines Leuchtturmwärters]} (*~um 1882)|pwv} des Hauſes, regungslos in einer Ecke des
                    Divans, ſah uns mit prachtvollen braunen Augen an, {\pb}ſtrickte und hatte nur einen Schuh an. Dafür war der andere Strumpf an den
                    Zehen zerriſſen. Das war die junge Dame\pwindex{?? [Teenagertochter eines Leuchtturmwaerters] *~um 1882@\textsc{?? [Teenagertochter eines Leuchtturmwärters]} (*~um 1882)|pwv} von \textsc{Hven}\oindex{Ven@\textbf{Ven}|pw}{\dotstwo}{ }\substVorne{}\textsuperscript{D}\substDazwischen{}I\substHinten{}m Zurückfahren gab es ſo hohe Wellen, daſs man die Oſtſee\oindex{Ostsee@\textbf{Ostsee}|pw} als Meer erkennen durfte; bisher war ſie immer ſo
                    ſtill, daſs man ſich an einem See hätte glauben können. Paula\pwindex{Beer-Hofmann, Paula 25.02.1879 – 30.10.1939@\textsc{Beer-Hofmann, Paula} (25.02.1879 – 30.10.1939)|pw} iſt ſogar ſeekrank geweſen. – Wir werden hier wohl
                    alle bis etwa zum 20. Auguſt bleiben. Nachmittags
                    pflege ich zu arbeiten. Vorher bin ich {\pb}wenig
                        dazugeko{\geminationm}en; nur ein paar Regentage oder
                    -ſtunden auf der Nordcap\oindex{Nordkap@\textbf{Nordkap}|pw}tour bin ich in
                    meiner Kajüte geſeſſen und habe am 2. Akt\pwindex{Schnitzler, Arthur 15.05.1862 – 21.10.1931@\textsc{Schnitzler, Arthur} (15.05.1862 – 21.10.1931), \emph{Schriftsteller, Mediziner}!Freiwild. Schauspiel in 3 Akten1896@\strich\emph{Freiwild. Schauspiel in 3 Akten} {[}1896{]}|pwv} allerlei verſucht. Immerhin ſcheint’s mir, als
                        we{\geminationn} ich theilweiſe in den Intentionen Ihres
                    Briefs, den ich in \textsc{Trondjhem}\oindex{Trondheim@\textbf{Trondheim}|pw} bei meiner Rückkehr gefunden habe, verfahren wäre;
                    denn vor allem hatte ich das Bedürfnis die Scene zwiſchen Ihm
                        un\textcolor{gray}{d} Ihr mit mehr Leben anzufüllen. Ich weiſs noch nicht,
                    ob mir das {\pb}und manches andre, das ich am 2.\pwindex{Schnitzler, Arthur 15.05.1862 – 21.10.1931@\textsc{Schnitzler, Arthur} (15.05.1862 – 21.10.1931), \emph{Schriftsteller, Mediziner}!Freiwild. Schauspiel in 3 Akten1896@\strich\emph{Freiwild. Schauspiel in 3 Akten} {[}1896{]}|pwv} und in den letzten Tagen
                    am 3. Akt\pwindex{Schnitzler, Arthur 15.05.1862 – 21.10.1931@\textsc{Schnitzler, Arthur} (15.05.1862 – 21.10.1931), \emph{Schriftsteller, Mediziner}!Freiwild. Schauspiel in 3 Akten1896@\strich\emph{Freiwild. Schauspiel in 3 Akten} {[}1896{]}|pwv} gearbeitet habe,
                    gelungen iſt; in ein paar Tagen les’ ich die ganze Sache dem Paul\pwindex{Goldmann, Paul 31.01.1865 – 25.09.1935@\textsc{Goldmann, Paul} (31.01.1865 – 25.09.1935), \emph{Schriftsteller, Journalist}|pw} und dem Richard\pwindex{Beer-Hofmann, Richard 1866-07-11 – 1945-09-26@\textsc{Beer-Hofmann, Richard} (1866-07-11 – 1945-09-26), \emph{Schriftsteller}|pw}
                    wieder vor. So wie ichs haben will, bring ichs doch wohl nie zuſa{\geminationm}en. –\pend
           \pstart
           Richard\pwindex{Beer-Hofmann, Richard 1866-07-11 – 1945-09-26@\textsc{Beer-Hofmann, Richard} (1866-07-11 – 1945-09-26), \emph{Schriftsteller}|pw} hat mir von Ihrer Novelle\pwindex{Hofmannsthal, Hugo von 1874-02-01 – 1929-07-15@\textsc{Hofmannsthal, Hugo von} (1874-02-01 – 1929-07-15), \emph{Schriftsteller}!Geschichte der beiden Liebespaare1978@\strich\emph{Geschichte der beiden Liebespaare} {[}1978{]}|pwv} erzählt; auch dſs er Ihnen
                    gerathen, Sie drucken zu laſſen. Solange muſs ich wohl warten bis
                    ich ſie zu leſen bekomme. Wohin werden Sie ſie geben? –\pend
           \pstart
           Meine Reiſe iſt im ganzen ſehr ſchön geweſen; vielleicht iſt die Zeit nur {\pb}etwas zu kurz geweſen, um ſoviel in ſich
                    aufzunehmen.\pend
           \pstart
           Auf der See hab ich merkwürdg viel Kopfſchmerzen gehabt. Von Städten hat mir \textsc{Gothenburg}\oindex{Goeteborg@\textbf{Göteborg}|pw} den
                    ſtärkſten Eindruck gemacht; wahrſcheinlich weil ich dort ganz allein (auch nicht
                    mit zufälligen Bekannten von der Reiſe) herumgegangen bin und am tiefſten
                    geſpürt habe: Wie fremd – wie fern – und dann weil ich nur ein paar Stunden dort
                    geweſen bin und bei jedem Haus, jedem Menſchen {\pb}wußte – dich ſeh ich zum letzten Mal.\pend
           \pstart
           – In \textsc{Christ}.\oindex{Oslo@\textbf{Oslo}|pw} hab ich \textsc{Ibsen}\pwindex{Ibsen, Henrik 20.03.1828 – 23.05.1906@\textsc{Ibsen, Henrik} (20.03.1828 – 23.05.1906), \emph{Schriftsteller}|pw} geſprochen, der mehr zuhörte als redete aber ſehr liebenswürdg war; in \textsc{Kopenhagen}\oindex{Kopenhagen@\textbf{Kopenhagen}|pw}{ }ſind wir (Richard\pwindex{Beer-Hofmann, Richard 1866-07-11 – 1945-09-26@\textsc{Beer-Hofmann, Richard} (1866-07-11 – 1945-09-26), \emph{Schriftsteller}|pw} u ich) mit \textsc{Nansen}\pwindex{Nansen, Peter 20.01.1861 – 31.07.1918@\textsc{Nansen, Peter} (20.01.1861 – 31.07.1918), \emph{Schriftsteller, Journalist, Verleger}|pw} beim Frühſtück geſeſſen, den wir wohl noch ſehen werden. –\pend
           \pstart
           – Bis zum 20. treffen mich Nachrichten hier, Badehotel\oindex{Badehotel@\textbf{Badehotel}|pw}. Es möcht mich freuen, noch zwei Worte von Ihnen
                    zu hören.\pend
           \pstart Leben Sie wohl! Mit vielen herzlichen Grüßen Ihr \spacefill\mbox{ArthSch}\pend{}\pstart
           \textsc{Skodsborg}\oindex{Skodsborg@\textbf{Skodsborg}|pw}{ }7/8 96. \pend
           \pstart
           Nach 20. (–25.) \textsc{Berlin}\oindex{Berlin@\textbf{Berlin}|pw}, aber ſchreiben Sie nach Wien\oindex{Wien@\textbf{Wien}|pw}.\pend
           
         
         \endnumbering\mylabel{h}\end{ledgroupsized}  \newcommand{\dateiname}{L00579}\newcommand{\titel}{Arthur Schnitzler an Hugo von Hofmannsthal, 7. 8. 1896}\newcommand{\editorInnen}{Martin Anton Müller und Gerd-Hermann Susen}%% latex-leseansicht-abspann.tex
%% Abspann für die Leseansicht.
%% Der Schalter \ifkorrekturansicht ist bereits durch den Vorspann gesetzt.

%% latex-abspann.tex
%% Gemeinsamer Abspann für Korrekturansicht und Leseansicht.
%% Setzt den Schalter \ifkorrekturansicht voraus (gesetzt in den
%% einbindenden Dateien latex-korrekturansicht-abspann.tex bzw.
%% latex-leseansicht-abspann.tex).
%% ---------------------------------------------------------------

\normalsize

% Das esempio-Environment wird nur in der Leseansicht benötigt
\ifkorrekturansicht\else
\newenvironment{esempio}[3]%
{
    \vspace{1.5ex}
    \rlap{\underline{#1}}
    \par
    \setlength{\parindent}{0cm}
    \nopagebreak
    \leftskip=#2cm
    \rightskip=#3cm
}
{
    \par
}
\fi

\doendnotes{C}
\bigskip
\vfill

\clearpage

\footnotesize

\ifkorrekturansicht
  \lohead{\textsc{register}}
\fi

% theindex-Environment neu definieren ohne reledmac
\makeatletter
\renewenvironment{theindex}{%
  \ifkorrekturansicht
    \section*{\indexname}%
  \else
    \subsubsection*{Index der erwähnten Entitäten}%
  \fi
  \setlength{\parindent}{0pt}%
  \setlength{\parskip}{0pt plus 0.3pt}%
  \let\item\@idxitem
}{%
  \ifkorrekturansicht\clearpage\fi
}
\makeatother

\IfFileExists{\jobname-pw.ind}{\input{\jobname-pw.ind}}{}

% Quellenangabe nur in der Leseansicht
\ifkorrekturansicht\else
% Fallback-Definitionen, falls die .tex-Datei \titel etc. nicht gesetzt hat
\providecommand{\titel}{}
\providecommand{\editorInnen}{}
\providecommand{\dateiname}{\jobname}

\vspace{3cm}

\vfill

\footnotesize
\textsc{Quelle}: \titel. Herausgegeben von {\editorInnen}. In: \emph{Arthur Schnitzler: Briefwechsel mit Autorinnen und Autoren}.
 Digitale Edition, https://schnitzler-briefe.acdh.oeaw.ac.at/{\dateiname}.html (Stand \today)
\fi

\end{document}


      