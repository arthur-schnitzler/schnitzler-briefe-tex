%% latex-korrekturansicht-vorspann.tex
%% Vorspann für die Korrekturansicht.
%% Lädt die gemeinsame Datei latex-vorspann.tex mit gesetztem Schalter.

\newif\ifkorrekturansicht
\korrekturansichttrue

\input{../tex-inputs/latex-vorspann}


\section[Arthur Schnitzler an Hugo von Hofmannsthal, 7. 8. 1896]{L00579 Arthur Schnitzler an Hugo von Hofmannsthal, 7. 8. 1896}
\nopagebreak\mylabel{L00579v}
\rehead{ }\normalsize\beginnumbering\briefempfaengerindex{Hofmannsthal, Hugo von@\textsc{Hofmannsthal, Hugo von}!zzzSchnitzler, Arthur@\emph{von Arthur Schnitzler}!1896-08-072@{7. 8. 1896}|(be}
\toendnotes[C]{\smallbreak\pagebreak[2]}\Standort{FDH, Hs-30885,51.}
\physDesc{Brief, 2 Blätter, 7 Seiten, 3072 Zeichen (Auch das zweite Blatt von Schnitzler datiert)
\newline{}Handschrift: Bleistift, deutsche Kurrent}
\buchAbdrucke{\weitereDrucke{1) Hugo von Hofmannsthal, Arthur Schnitzler: \emph{Briefwechsel}. Frankfurt am Main: \emph{S. Fischer} 1964, S. 70–72.} \weitereDrucke{2) Arthur Schnitzler: \emph{Briefe 1875–1912}. Frankfurt am Main: \emph{S. Fischer} 1981, S. 290–292.} }\toendnotes[C]{\smallbreak}
\pstart
           \raggedleft{}{\pb}\textsc{Skodsborg}\oindex{Skodsborg@\textbf{Skodsborg}, \emph{P.PPL}|pw}, 7. 8. 96\pend
           \vspace{0.5em}
\pstart
           Lieber Hugo, ſeit So{\geminationn}tag
               bin ich mit Richard\pwindex{Beer-Hofmann, Richard 1866-07-11 – 1945-09-26@\textsc{Beer-Hofmann, Richard} (1866-07-11 – 1945-09-26), \emph{Schriftsteller/Schriftstellerin}|pw} (und Paula\pwindex{Beer-Hofmann, Paula 25.02.1879 – 30.10.1939@\textsc{Beer-Hofmann, Paula} (25.02.1879 – 30.10.1939)|pw}) zuſa{\geminationm}en; ſeit vorgeſtern
               iſt auch Paul Goldmann\pwindex{Goldmann, Paul 31.01.1865 – 25.09.1935@\textsc{Goldmann, Paul} (31.01.1865 – 25.09.1935), \emph{Schriftsteller/Schriftstellerin, Journalist/Journalistin}|pw} da, und wir ſind in
               einem angenehmen Hotel, am Meer, hinter den Häuſern gleich ein wunderſchöner Wald mit
               Buchen und Tannen, im Wald kleine faſt verſteckte Teiche, und we{\geminationn} man eine halbe Stunde weiter \substVorne{}\textsuperscript{\textcolor{gray}{läuft}}\substDazwischen{}geht\substHinten{}, das freundliche Thal mit lieben kleinen Häuſern und Ort\substVorne{}\textsuperscript{en}\substDazwischen{}ſchaften\substHinten{} (wo wir aber noch nie geweſen ſind). Heute Vormittag ſind wir nach einer
               kleinen ſchwe{\pb}diſchen Inſel\oindex{Ven@\textbf{Ven}, \emph{Insel (N.INS)}|pwv} hinübergeſegelt, wo nicht viele
               Menſchen wohnen, ſind in dem netten Haus des Leuchtthurmwächters\pwindex{?? [Leuchtturmwaerter] @\textsc{?? [Leuchtturmwärter]}|pwv} geweſen, und wie wir von dem niedern
               Thurm herunterſtiegen, fanden wir im Wohnzimmer ein leiſes Harmonium, eine
               freundliche Hausfrau\pwindex{?? [Frau des Leuchtturmwaerters] @\textsc{?? [Frau des Leuchtturmwärters]}|pwv} und \substVorne{}\textsuperscript{eine}\substDazwischen{}im\substHinten{} Vorzimmer saſs die vierzehnjährige Tochter\pwindex{?? [Teenagertochter eines Leuchtturmwaerters] *~um 1882@\textsc{?? [Teenagertochter eines Leuchtturmwärters]} (*~um 1882)|pwv} des Hauſes, regungslos in einer Ecke des Divans,
               ſah uns mit prachtvollen braunen Augen an, {\pb}ſtrickte und
               hatte nur einen Schuh an. Dafür war der andere Strumpf an den Zehen zerriſſen. Das
               war die junge Dame\pwindex{?? [Teenagertochter eines Leuchtturmwaerters] *~um 1882@\textsc{?? [Teenagertochter eines Leuchtturmwärters]} (*~um 1882)|pwv} von \textsc{Hven}\oindex{Ven@\textbf{Ven}, \emph{Insel (N.INS)}|pw}{\dotstwo}{ }\substVorne{}\textsuperscript{D}\substDazwischen{}I\substHinten{}m Zurückfahren gab es ſo hohe Wellen, daſs man die Oſtſee\oindex{Ostsee@\textbf{Ostsee}, \emph{Meer (N.MER)}|pw} als Meer erkennen durfte; bisher war ſie immer ſo
               ſtill, daſs man ſich an einem See hätte glauben können. Paula\pwindex{Beer-Hofmann, Paula 25.02.1879 – 30.10.1939@\textsc{Beer-Hofmann, Paula} (25.02.1879 – 30.10.1939)|pw} iſt ſogar ſeekrank geweſen. – Wir werden hier wohl alle
               bis etwa zum 20. Auguſt bleiben. Nachmittags pflege ich zu
               arbeiten. Vorher bin ich {\pb}wenig dazugeko{\geminationm}en; nur ein paar Regentage oder -ſtunden auf der Nordcap\oindex{Nordkap@\textbf{Nordkap}, \emph{Kap (N.KAP)}|pw}tour bin ich in meiner Kajüte geſeſſen
               und habe am 2. Akt\pwindex{Freiwild. Schauspiel in 3 Akten@\emph{Freiwild. Schauspiel in 3 Akten}|pwv} allerlei
               verſucht. Immerhin ſcheint’s mir, als we{\geminationn} ich theilweiſe
               in den Intentionen Ihres Briefs, den ich in \textsc{Trondjhem}\oindex{Trondheim@\textbf{Trondheim}, \emph{P.PPLA2}|pw} bei meiner Rückkehr gefunden habe, verfahren wäre; denn vor allem hatte ich das
               Bedürfnis die Scene zwiſchen Ihm un\textcolor{gray}{d} Ihr mit mehr Leben
               anzufüllen. Ich weiſs noch nicht, ob mir das {\pb}und manches
               andre, das ich am 2.\pwindex{Freiwild. Schauspiel in 3 Akten@\emph{Freiwild. Schauspiel in 3 Akten}|pwv} und in
               den letzten Tagen am 3. Akt\pwindex{Freiwild. Schauspiel in 3 Akten@\emph{Freiwild. Schauspiel in 3 Akten}|pwv}
               gearbeitet habe, gelungen iſt; in ein paar Tagen les’ ich die ganze Sache dem Paul\pwindex{Goldmann, Paul 31.01.1865 – 25.09.1935@\textsc{Goldmann, Paul} (31.01.1865 – 25.09.1935), \emph{Schriftsteller/Schriftstellerin, Journalist/Journalistin}|pw} und dem Richard\pwindex{Beer-Hofmann, Richard 1866-07-11 – 1945-09-26@\textsc{Beer-Hofmann, Richard} (1866-07-11 – 1945-09-26), \emph{Schriftsteller/Schriftstellerin}|pw} wieder vor. So wie ichs haben will, bring ichs doch wohl nie zuſa{\geminationm}en. –\pend
           
\pstart
           Richard\pwindex{Beer-Hofmann, Richard 1866-07-11 – 1945-09-26@\textsc{Beer-Hofmann, Richard} (1866-07-11 – 1945-09-26), \emph{Schriftsteller/Schriftstellerin}|pw} hat mir von Ihrer Novelle\pwindex{Geschichte der beiden Liebespaare@\emph{Geschichte der beiden Liebespaare}|pwv} erzählt; auch dſs er Ihnen gerathen,
               Sie drucken zu laſſen. Solange muſs ich wohl warten bis ich ſie zu leſen
               bekomme. Wohin werden Sie ſie geben? –\pend
           
\pstart
           Meine Reiſe iſt im ganzen ſehr ſchön geweſen; vielleicht iſt die Zeit nur {\pb}etwas zu kurz geweſen, um ſoviel in ſich aufzunehmen.\pend
           
\pstart
           Auf der See hab ich merkwürdg viel Kopfſchmerzen gehabt. Von Städten hat mir \textsc{Gothenburg}\oindex{Goeteborg@\textbf{Göteborg}, \emph{P.PPLA}|pw} den ſtärkſten Eindruck gemacht; wahrſcheinlich weil ich dort ganz allein (auch
               nicht mit zufälligen Bekannten von der Reiſe) herumgegangen bin und am tiefſten
               geſpürt habe: Wie fremd – wie fern – und dann weil ich nur ein paar Stunden dort
               geweſen bin und bei jedem Haus, jedem Menſchen {\pb}wußte –
               dich ſeh ich zum letzten Mal.\pend
           
\pstart
           – In \textsc{Christ}.\oindex{Oslo@\textbf{Oslo}, \emph{P.PPLC}|pw} hab ich \textsc{Ibsen}\pwindex{Ibsen, Henrik 20.03.1828 – 23.05.1906@\textsc{Ibsen, Henrik} (20.03.1828 – 23.05.1906), \emph{Schriftsteller/Schriftstellerin}|pw} geſprochen, der mehr zuhörte als redete aber ſehr liebenswürdg war; in \textsc{Kopenhagen}\oindex{Kopenhagen@\textbf{Kopenhagen}, \emph{P.PPLC}|pw}{ }ſind wir (Richard\pwindex{Beer-Hofmann, Richard 1866-07-11 – 1945-09-26@\textsc{Beer-Hofmann, Richard} (1866-07-11 – 1945-09-26), \emph{Schriftsteller/Schriftstellerin}|pw} u ich) mit \textsc{Nansen}\pwindex{Nansen, Peter 20.01.1861 – 31.07.1918@\textsc{Nansen, Peter} (20.01.1861 – 31.07.1918), \emph{Schriftsteller/Schriftstellerin, Journalist/Journalistin, Verleger/Verlegerin}|pw} beim Frühſtück geſeſſen, den wir wohl noch ſehen werden. –\pend
           
\pstart
           – Bis zum 20. treffen mich Nachrichten hier, Badehotel\oindex{Badehotellet@\textbf{Badehotellet}, \emph{Hotel (K.HTL)}|pw}. Es möcht mich freuen, noch zwei Worte von Ihnen zu
               hören.\pend
           \pstart Leben Sie wohl! Mit vielen herzlichen Grüßen Ihr \spacefill\mbox{ArthSch}\pend{}
\pstart
           \textsc{Skodsborg}\oindex{Skodsborg@\textbf{Skodsborg}, \emph{P.PPL}|pw}{ }7/8 96. \pend
           
\pstart
           Nach 20. (–25.) \textsc{Berlin}\oindex{Berlin@\textbf{Berlin}, \emph{P.PPLC}|pw}, aber ſchreiben Sie nach Wien\oindex{Wien@\textbf{Wien}, \emph{A.ADM2}|pw}.\pend
           \selectlanguage{ngerman}\endnumbering\briefempfaengerindex{Hofmannsthal, Hugo von@\textsc{Hofmannsthal, Hugo von}!zzzSchnitzler, Arthur@\emph{von Arthur Schnitzler}!1896-08-072@{7. 8. 1896}|)be}\mylabel{L00579h}  \normalsize

\doendnotes{C}
\bigskip
\vfill

\clearpage

\footnotesize

\lohead{\textsc{register}}

% Definiere theindex-Environment komplett neu ohne reledmac
\makeatletter
\renewenvironment{theindex}{%
  \section*{\indexname}%
  \setlength{\parindent}{0pt}%
  \setlength{\parskip}{0pt plus 0.3pt}%
  \let\item\@idxitem
}{%
  \clearpage
}
\makeatother

\IfFileExists{\jobname-pw.ind}{\input{\jobname-pw.ind}}{}

\end{document}

      