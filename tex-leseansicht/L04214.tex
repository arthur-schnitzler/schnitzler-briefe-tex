%% latex-leseansicht-vorspann.tex
%% Vorspann für die Leseansicht.
%% Lädt die gemeinsame Datei latex-vorspann.tex mit nicht gesetztem Schalter.

\newif\ifkorrekturansicht
\korrekturansichtfalse

\input{../tex-inputs/latex-vorspann}


\section[Arthur Schnitzler an Romain Rolland, {[}29. 1.?{]} 1916]{L04214 Arthur Schnitzler an Romain Rolland, [29. 1.?] 1916}
\nopagebreak\mylabel{L04214v}
\rehead{ }\normalsize\beginnumbering\briefempfaengerindex{Rolland, Romain@\textsc{Rolland, Romain}!zzzSchnitzler, Arthur@\emph{von Arthur Schnitzler}!1916-01-291@{[29. 1.?] 1916}|(be}
\toendnotes[C]{\smallbreak\pagebreak[2]}
\correspDesc{Versand  durch Arthur Schnitzler am [29. 1.?] 1916 in Wien
\newline{}Erhalt  durch Romain Rolland am 6. 2. 1916 in Genf}\toendnotes[C]{\smallbreak}
\Standort{Paris, Bibliothèque Nationale de France, Fonds Romain Rolland, Cote NAF 28400.}
\physDesc{Telegramm, 255 Zeichen
\newline{}maschinell
\newline{}Versand: 1) »\textcolor{gray}{\textbf{Erhalten von}}{ }K{ }\textcolor{gray}{\textbf{den}}{ }6/2 \textcolor{gray}{\textbf{191}}6{ }\textcolor{gray}{\textbf{um}}{ }3\textcolor{gray}{\textbf{Uhr}}10\textcolor{gray}{\textbf{s}}{ }\textcolor{gray}{Sundy}«   2) Stempel: »\nobreak{}\oindex{Genf@\textbf{Genf}|pwk}Genève Bureau des Télé\textcolor{gray}{grammes}, 6 II 16\nobreak{}«.  3) Stempel: »\nobreak{}Transmissible\nobreak{}«. 
\newline{}Ordnung: mit Bleistift Blatt paginiert: »6« }\toendnotes[C]{\smallbreak}\pstart{}{\pb}romain rolland genf champel\oindex{Champel@\textbf{Champel}|pw}\pend{}\pstart{}hotel beau-sejour\oindex{Hôtel Beau-Séjour@\textbf{Hôtel Beau-Séjour}, \emph{Hotel}|pw}. =\pend{}{\bigskip}\vspace{1em}
\pstart
           \centering{}{\pb}geneve\oindex{Genf@\textbf{Genf}|pw}{ }wien\oindex{Wien@\textbf{Wien}, \emph{Verwaltungsgebiet}|pw} 13.4 5077-25-31/1–8– \label{K_L03884-1v}\edtext{retard guerre}{\lemma{\textnormal{\emph{retard guerre}}}\Cendnote{\textnormal{französisch: Verspätung Krieg}}}\label{K_L03884-1}, ctr =
              \pend
           \vspace{0.5em}
\pstart
           zum \label{K_L03884-2v}\edtext{fuenfzigsten geburtstag}{\lemma{\textnormal{\emph{fuenfzigsten geburtstag}}}\Cendnote{\textnormal{Am
                  29. 1. 1916 war Rolland\pwindex{Rolland, Romain 29.\,1.\,1866 Clamecy – 30.\,12.\,1944 Vézelay@\textsc{Rolland, Romain} (29.\,1.\,1866 Clamecy – 30.\,12.\,1944 Vézelay), \emph{Schriftsteller}|pwk} 50 Jahre alt geworden.}}}\label{K_L03884-2} begluekwuensche ich sie in herzlicher
               verehrung persoenliche begegnung in schoenern zeiten erhoffend\pend
           \pstart = ihr \spacefill\mbox{doktor schnitz\strikeout{e}ler\strikeout{er}+. –.|+}\pend{}\selectlanguage{ngerman}\endnumbering\briefempfaengerindex{Rolland, Romain@\textsc{Rolland, Romain}!zzzSchnitzler, Arthur@\emph{von Arthur Schnitzler}!1916-01-291@{[29. 1.?] 1916}|)be}\mylabel{L04214h}
\begin{anhang}
\end{anhang}\newcommand{\dateiname}{L04214}\newcommand{\titel}{Arthur Schnitzler an Romain Rolland, [29. 1.?] 1916}\newcommand{\editorInnen}{Selma Jahnke und Martin Anton Müller}%% latex-leseansicht-abspann.tex
%% Abspann für die Leseansicht.
%% Der Schalter \ifkorrekturansicht ist bereits durch den Vorspann gesetzt.

%% latex-abspann.tex
%% Gemeinsamer Abspann für Korrekturansicht und Leseansicht.
%% Setzt den Schalter \ifkorrekturansicht voraus (gesetzt in den
%% einbindenden Dateien latex-korrekturansicht-abspann.tex bzw.
%% latex-leseansicht-abspann.tex).
%% ---------------------------------------------------------------

\normalsize

% Das esempio-Environment wird nur in der Leseansicht benötigt
\ifkorrekturansicht\else
\newenvironment{esempio}[3]%
{
    \vspace{1.5ex}
    \rlap{\underline{#1}}
    \par
    \setlength{\parindent}{0cm}
    \nopagebreak
    \leftskip=#2cm
    \rightskip=#3cm
}
{
    \par
}
\fi

\doendnotes{C}
\bigskip
\vfill

\clearpage

\footnotesize

\ifkorrekturansicht
  \lohead{\textsc{register}}
\fi

% theindex-Environment neu definieren ohne reledmac
\makeatletter
\renewenvironment{theindex}{%
  \ifkorrekturansicht
    \section*{\indexname}%
  \else
    \subsubsection*{Index der erwähnten Entitäten}%
  \fi
  \setlength{\parindent}{0pt}%
  \setlength{\parskip}{0pt plus 0.3pt}%
  \let\item\@idxitem
}{%
  \ifkorrekturansicht\clearpage\fi
}
\makeatother

\IfFileExists{\jobname-pw.ind}{\input{\jobname-pw.ind}}{}

% Quellenangabe nur in der Leseansicht
\ifkorrekturansicht\else
% Fallback-Definitionen, falls die .tex-Datei \titel etc. nicht gesetzt hat
\providecommand{\titel}{}
\providecommand{\editorInnen}{}
\providecommand{\dateiname}{\jobname}

\vspace{3cm}

\vfill

\footnotesize
\textsc{Quelle}: \titel. Herausgegeben von {\editorInnen}. In: \emph{Arthur Schnitzler: Briefwechsel mit Autorinnen und Autoren}.
 Digitale Edition, https://schnitzler-briefe.acdh.oeaw.ac.at/{\dateiname}.html (Stand \today)
\fi

\end{document}


