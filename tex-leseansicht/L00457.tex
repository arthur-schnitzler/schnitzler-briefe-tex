%% latex-korrekturansicht-vorspann.tex
%% Vorspann für die Korrekturansicht.
%% Lädt die gemeinsame Datei latex-vorspann.tex mit gesetztem Schalter.

\newif\ifkorrekturansicht
\korrekturansichttrue

\input{../tex-inputs/latex-vorspann}


\section[Richard Beer-Hofmann an Arthur Schnitzler, 23. 6. 1895]{L00457 Richard Beer-Hofmann an Arthur Schnitzler, 23. 6. 1895}
\nopagebreak\mylabel{L00457v}
\rehead{ }\normalsize\beginnumbering\briefempfaengerindex{Schnitzler, Arthur@\textsc{Schnitzler, Arthur}!zzzBeer-Hofmann, Richard@\emph{von Richard Beer-Hofmann}!1895-06-231@{23. 6. 1895}|(be}
\toendnotes[C]{\smallbreak\pagebreak[2]}\Standort{CUL, Schnitzler, B 8.}
\physDesc{Brief, 1 Blatt, 4 Seiten, 954 Zeichen
\newline{}Handschrift: Bleistift, lateinische Kurrent
\newline{}Schnitzler: mit Bleistift nummeriert: »62« }
\buchAbdrucke{\weitereDrucke{Arthur Schnitzler, Richard Beer-Hofmann: \emph{Briefwechsel 1891–1931}. Wien, Zürich: \emph{Europaverlag} 1992, S. 75–76.} }
\pstart
           {\pb}\uline{Zleb}\oindex{Schleb@\textbf{Schleb}, \emph{P.PPL}|pw}{ }23/VI 95\pend
           \vspace{0.5em}
\pstart
           Lieber Arthur!{ }Zleb\oindex{Schleb@\textbf{Schleb}, \emph{P.PPL}|pw} ist mit dem Wagen ¾ Stunden von Caslau\oindex{Cáslav@\textbf{Čáslav}, \emph{P.PPL}|pw} entfernt; ich bin weil man doch am
                  Sonntag nicht in Caslau\oindex{Cáslav@\textbf{Čáslav}, \emph{P.PPL}|pw} bleiben
               kann nach Zleb\oindex{Schleb@\textbf{Schleb}, \emph{P.PPL}|pw} gefahren – Sie begreifen – mit mir
               am Tische zwei unsägliche Cadetten der Reserve, einer aus Neu-Bidschow\oindex{Nový Bydžov@\textbf{Nový Bydžov}, \emph{P.PPL}|pw}, der andere {\pb}aus Benatek\oindex{Benatek@\textbf{Benatek}, \emph{Besiedelter Ort (A.BSO)}|pw}. Jetzt lesen sie Gottseidank böhmische\oindex{Boehmen@\textbf{Böhmen}, \emph{L.RGN}|pw} Zeitungen.\pend
           
\pstart
           Ich bin also voraussichtlich am 29ten, unwahrscheinlicher Weise schon am
                  28ten{ }nachts d. i. 11 Uhr nachts in Wien\oindex{Wien@\textbf{Wien}, \emph{A.ADM2}|pw}, und werde gegen 3. od 4. nach Ischl\oindex{Bad Ischl@\textbf{Bad Ischl}, \emph{P.PPL}|pw} reisen. Ich bin nervös sehr herunter {\pb}so daß ich trotz Müdigkeit nicht
               schlafe. Ich sehne mich nach Ruhe und Arbeiten. –\pend
           
\pstart
           Vielleicht gebe ich mir telegrafisch ein Rendezvous mit Ihnen, wenn ich ankomme. Wann
               sind Sie in Ischl\oindex{Bad Ischl@\textbf{Bad Ischl}, \emph{P.PPL}|pw}? Das können Sie mir zwar sagen,
               schreiben Sie es mir {\pb}aber lieber,
               weil mir jeder Brief woltut.\pend
           
\pstart
           \label{OL515-1v}\label{OL515-1h}Ad Burkhardt\pwindex{Burckhard, Max Eugen 14.07.1854 – 16.03.1912@\textsc{Burckhard, Max Eugen} (14.07.1854 – 16.03.1912), \emph{Schriftsteller/Schriftstellerin, Rechtswissenschaftler/Rechtswissenschaftlerin, Theaterleiter/Theaterleiterin}|pw}: \uline{Bahr\pwindex{Bahr, Hermann 19.07.1863 – 15.01.1934@\textsc{Bahr, Hermann} (19.07.1863 – 15.01.1934), \emph{Schriftsteller/Schriftstellerin, Kritiker/Kritikerin}|pw}, Burkhardt\pwindex{Burckhard, Max Eugen 14.07.1854 – 16.03.1912@\textsc{Burckhard, Max Eugen} (14.07.1854 – 16.03.1912), \emph{Schriftsteller/Schriftstellerin, Rechtswissenschaftler/Rechtswissenschaftlerin, Theaterleiter/Theaterleiterin}|pw}, Lueger\pwindex{Lueger, Karl 24.10.1844 – 10.03.1910@\textsc{Lueger, Karl} (24.10.1844 – 10.03.1910), \emph{Politiker/Politikerin}|pw}}. Aber der Erste ist doch anders. Sie sehen sogar gerecht werde ich hier {\dots}\pend
           
\pstart
           Der »alte Dichter\pwindex{Spaeter Ruhm@\emph{Später Ruhm}|pw}« ist doch schon zusa{\geminationm}engestrichen? \pend
           
\pstart
           Herzlichst Ihr{\\[\baselineskip]}\spacefill\mbox{Richard}\pend
           \leftskip=0em{}\selectlanguage{ngerman}\endnumbering\briefempfaengerindex{Schnitzler, Arthur@\textsc{Schnitzler, Arthur}!zzzBeer-Hofmann, Richard@\emph{von Richard Beer-Hofmann}!1895-06-231@{23. 6. 1895}|)be}\mylabel{L00457h}  \normalsize

\doendnotes{C}
\bigskip
\vfill

\clearpage

\footnotesize

\lohead{\textsc{register}}

% Definiere theindex-Environment komplett neu ohne reledmac
\makeatletter
\renewenvironment{theindex}{%
  \section*{\indexname}%
  \setlength{\parindent}{0pt}%
  \setlength{\parskip}{0pt plus 0.3pt}%
  \let\item\@idxitem
}{%
  \clearpage
}
\makeatother

\IfFileExists{\jobname-pw.ind}{\input{\jobname-pw.ind}}{}

\end{document}

      