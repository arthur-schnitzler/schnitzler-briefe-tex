%% latex-leseansicht-vorspann.tex
%% Vorspann für die Leseansicht.
%% Lädt die gemeinsame Datei latex-vorspann.tex mit nicht gesetztem Schalter.

\newif\ifkorrekturansicht
\korrekturansichtfalse

\input{../tex-inputs/latex-vorspann}


\section[Richard Beer-Hofmann an Arthur Schnitzler, 23. 6. 1895]{L00457 Richard Beer-Hofmann an Arthur Schnitzler, 23. 6. 1895}
\nopagebreak\mylabel{L00457v}
\rehead{ }\normalsize\beginnumbering\briefempfaengerindex{Schnitzler, Arthur@\textsc{Schnitzler, Arthur}!zzzBeer-Hofmann, Richard@\emph{von Richard Beer-Hofmann}!1895-06-231@{23. 6. 1895}|(be}
\toendnotes[C]{\smallbreak\pagebreak[2]}
\correspDesc{Versand  durch Richard Beer-Hofmann am 23. 6. 1895 in Schleb
\newline{}Erhalt  durch Arthur Schnitzler im Zeitraum [24. 6. 1895
                  – 28. 6. 1895?] in Wien}\toendnotes[C]{\smallbreak}
\Standort{CUL, Schnitzler, B 8.}
\physDesc{Brief, 1 Blatt, 4 Seiten, 954 Zeichen
\newline{}Handschrift: Bleistift, lateinische Kurrent
\newline{}Schnitzler: mit Bleistift nummeriert: »62« }
\buchAbdrucke{\weitereDrucke{Arthur Schnitzler, Richard Beer-Hofmann: \emph{Briefwechsel 1891–1931}. Herausgegeben von Konstanze Fliedl. Wien, Zürich: \emph{Europaverlag} 1992, S. 75–76.} }
\pstart
           {\pb}\uline{Zleb}\oindex{Schleb@\textbf{Schleb}|pw}{ }23/VI 95\pend
           \vspace{0.5em}
\pstart
           Lieber Arthur!{ }Zleb\oindex{Schleb@\textbf{Schleb}|pw} ist mit dem Wagen ¾ Stunden von Caslau\oindex{Čáslav@\textbf{Čáslav}|pw} entfernt; ich bin weil man doch am
                  Sonntag nicht in Caslau\oindex{Čáslav@\textbf{Čáslav}|pw} bleiben
               kann nach Zleb\oindex{Schleb@\textbf{Schleb}|pw} gefahren – Sie begreifen – mit mir
               am Tische zwei unsägliche Cadetten der Reserve, einer aus Neu-Bidschow\oindex{Nový Bydžov@\textbf{Nový Bydžov}|pw}, der andere {\pb}aus Benatek\oindex{Benatek@\textbf{Benatek}|pw}. Jetzt lesen sie Gottseidank böhmische\oindex{Böhmen@\textbf{Böhmen}, \emph{Region}|pw} Zeitungen.\pend
           
\pstart
           Ich bin also voraussichtlich am 29ten, unwahrscheinlicher Weise schon am
                  28ten{ }nachts d. i. 11 Uhr nachts in Wien\oindex{Wien@\textbf{Wien}, \emph{Verwaltungsgebiet}|pw}, und werde gegen 3. od 4. nach Ischl\oindex{Bad Ischl@\textbf{Bad Ischl}|pw} reisen. Ich bin nervös sehr herunter {\pb}so daß ich trotz Müdigkeit nicht
               schlafe. Ich sehne mich nach Ruhe und Arbeiten. –\pend
           
\pstart
           Vielleicht gebe ich mir telegrafisch ein Rendezvous mit Ihnen, wenn ich ankomme. Wann
               sind Sie in Ischl\oindex{Bad Ischl@\textbf{Bad Ischl}|pw}? Das können Sie mir zwar sagen,
               schreiben Sie es mir {\pb}aber lieber,
               weil mir jeder Brief woltut.\pend
           
\pstart
           \label{OL515-1v}\label{OL515-1h}Ad Burkhardt\pwindex{Burckhard, Max Eugen 14.\,7.\,1854 Korneuburg – 16.\,3.\,1912 Wien@\textsc{Burckhard, Max Eugen} (14.\,7.\,1854 Korneuburg – 16.\,3.\,1912 Wien), \emph{Schriftsteller, Rechtswissenschaftler, Theaterleiter}|pw}: \uline{Bahr\pwindex{Bahr, Hermann 19.\,7.\,1863 Linz – 15.\,1.\,1934 München@\textsc{Bahr, Hermann} (19.\,7.\,1863 Linz – 15.\,1.\,1934 München), \emph{Schriftsteller, Kritiker}|pw}, Burkhardt\pwindex{Burckhard, Max Eugen 14.\,7.\,1854 Korneuburg – 16.\,3.\,1912 Wien@\textsc{Burckhard, Max Eugen} (14.\,7.\,1854 Korneuburg – 16.\,3.\,1912 Wien), \emph{Schriftsteller, Rechtswissenschaftler, Theaterleiter}|pw}, Lueger\pwindex{Lueger, Karl 24.\,10.\,1844 Wien – 10.\,3.\,1910 ebd.@\textsc{Lueger, Karl} (24.\,10.\,1844 Wien – 10.\,3.\,1910 ebd.), \emph{Politiker}|pw}}. Aber der Erste ist doch anders. Sie sehen sogar gerecht werde ich hier {\dots}\pend
           
\pstart
           Der »alte Dichter\pwindex{Schnitzler, Arthur 15.\,5.\,1862 Wien – 21.\,10.\,1931 ebd.@\textsc{Schnitzler, Arthur} (15.\,5.\,1862 Wien – 21.\,10.\,1931 ebd.), \emph{Schriftsteller, Mediziner}!Später Ruhm@\strich\emph{Später Ruhm}|pw}« ist doch schon zusa{\geminationm}engestrichen?\pend
           
\pstart
           Herzlichst Ihr{\\[\baselineskip]}\spacefill\mbox{Richard}\pend
           \leftskip=0em{}\selectlanguage{ngerman}\endnumbering\briefempfaengerindex{Schnitzler, Arthur@\textsc{Schnitzler, Arthur}!zzzBeer-Hofmann, Richard@\emph{von Richard Beer-Hofmann}!1895-06-231@{23. 6. 1895}|)be}\mylabel{L00457h}  \newcommand{\dateiname}{L00457}\newcommand{\titel}{Richard Beer-Hofmann an Arthur Schnitzler, 23. 6. 1895}\newcommand{\editorInnen}{Herausgegeben von Martin Anton Müller}%% latex-leseansicht-abspann.tex
%% Abspann für die Leseansicht.
%% Der Schalter \ifkorrekturansicht ist bereits durch den Vorspann gesetzt.

%% latex-abspann.tex
%% Gemeinsamer Abspann für Korrekturansicht und Leseansicht.
%% Setzt den Schalter \ifkorrekturansicht voraus (gesetzt in den
%% einbindenden Dateien latex-korrekturansicht-abspann.tex bzw.
%% latex-leseansicht-abspann.tex).
%% ---------------------------------------------------------------

\normalsize

% Das esempio-Environment wird nur in der Leseansicht benötigt
\ifkorrekturansicht\else
\newenvironment{esempio}[3]%
{
    \vspace{1.5ex}
    \rlap{\underline{#1}}
    \par
    \setlength{\parindent}{0cm}
    \nopagebreak
    \leftskip=#2cm
    \rightskip=#3cm
}
{
    \par
}
\fi

\doendnotes{C}
\bigskip
\vfill

\clearpage

\footnotesize

\ifkorrekturansicht
  \lohead{\textsc{register}}
\fi

% theindex-Environment neu definieren ohne reledmac
\makeatletter
\renewenvironment{theindex}{%
  \ifkorrekturansicht
    \section*{\indexname}%
  \else
    \subsubsection*{Index der erwähnten Entitäten}%
  \fi
  \setlength{\parindent}{0pt}%
  \setlength{\parskip}{0pt plus 0.3pt}%
  \let\item\@idxitem
}{%
  \ifkorrekturansicht\clearpage\fi
}
\makeatother

\IfFileExists{\jobname-pw.ind}{\input{\jobname-pw.ind}}{}

% Quellenangabe nur in der Leseansicht
\ifkorrekturansicht\else
% Fallback-Definitionen, falls die .tex-Datei \titel etc. nicht gesetzt hat
\providecommand{\titel}{}
\providecommand{\editorInnen}{}
\providecommand{\dateiname}{\jobname}

\vspace{3cm}

\vfill

\footnotesize
\textsc{Quelle}: \titel. Herausgegeben von {\editorInnen}. In: \emph{Arthur Schnitzler: Briefwechsel mit Autorinnen und Autoren}.
 Digitale Edition, https://schnitzler-briefe.acdh.oeaw.ac.at/{\dateiname}.html (Stand \today)
\fi

\end{document}


