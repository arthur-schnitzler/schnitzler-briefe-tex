%% latex-leseansicht-vorspann.tex
%% Vorspann für die Leseansicht.
%% Lädt die gemeinsame Datei latex-vorspann.tex mit nicht gesetztem Schalter.

\newif\ifkorrekturansicht
\korrekturansichtfalse

\input{../tex-inputs/latex-vorspann}


         
         \renewcommand{\erwaehntePersonen}{Personen: Albert Ehrenstein, Stefan Großmann, Karl Kraus}
         \renewcommand{\erwaehnteInstitutionen}{Institutionen: Die Fackel}
         \renewcommand{\erwaehnteOrte}{Orte: Sternwartestraße 71, Wien}
         \renewcommand{\erwaehnteWerke}{}
               \section[Arthur Schnitzler an Albert Ehrenstein, 9. 2. 1911]{ Arthur Schnitzler an Albert Ehrenstein, 9. 2. 1911}\nopagebreak\mylabel{v}\rehead{ }\begin{ledgroupsized}[t]{13cm}\normalsize\beginnumbering\briefempfaengerindex{Ehrenstein, Albert@\textsc{Ehrenstein, Albert}!zzzSchnitzler, Arthur@\emph{von Arthur Schnitzler}!1911-02-091@{9. 2. 1911}|(be} \toendnotes[C]{\smallbreak\pagebreak[2]} \Standort{Jerusalem, The National Library of Israel, ARC. Ms. Var. 306 1 118.}
\physDesc{Brief, 1 Blatt, 2 Seiten, 1869 Zeichen
\newline{}Schreibmaschine
\newline{}Handschrift: schwarze Tinte, lateinische Kurrent (\noindent{}Korrekturen, Unterschrift)}\buchAbdrucke{\weitereDrucke{Arthur Schnitzler: \emph{Briefe 1875–1912}. Hg. Therese Nickl und Heinrich Schnitzler. Frankfurt am Main: \emph{S. Fischer} 1981, S. 656–657.} }\pstart
           {\pb}\textcolor{gray}{\textbf{Dr. Arthur Schnitzler}}\hfill 9. 2. 1911.\pend
           \pstart
           \textcolor{gray}{\textbf{Wien XVIII. Sternwartestrasse 71\oindex{Sternwartestrasse 71@\textbf{Sternwartestraße 71}|pw}}}\pend
           \pstart{}Sehr geehrter Herr Doktor.\pend\pstart
           Gestern erhielt ich einen Brief von Stefan
                  Grossmann\pwindex{Grossmann, Stefan 19.05.1875 – 03.01.1935@\textsc{Großmann, Stefan} (19.05.1875 – 03.01.1935), \emph{Schriftsteller, Journalist}|pw}, der unter anderem folgende Stelle enthält: »Ein junger Literat
                  \introOben{}(\introOben{}von Talent\introOben{})\introOben{}{ }\introOben{}\textsc{Herr Ehrenstein}\introOben{} erzählt verschiedenen Leuten unter anderm auch dem Fackel\orgindex{Fackel@Die Fackel|pw}-Kraus\pwindex{Kraus, Karl 28.04.1874 – 12.06.1936@\textsc{Kraus, Karl} (28.04.1874 – 12.06.1936), \emph{Schriftsteller, Publizist}|pw}, Sie hätten
               ihm ›bestätigt‹, dass ich meine Macht als Kritiker zu erotischen Erpressungen an
               Schauspielerinnen ausgenützt hätte.« Zugleich bittet er mich um eine Silbe darüber,
               dass ich eine solche Bestätigung nicht gab, \introOben{}»\introOben{}wie ich sie ja
               auch nicht geben konnte.\introOben{}«\introOben{}\pend
           \pstart
           Ich habe Herrn Grossmann\pwindex{Grossmann, Stefan 19.05.1875 – 03.01.1935@\textsc{Großmann, Stefan} (19.05.1875 – 03.01.1935), \emph{Schriftsteller, Journalist}|pw} wie natürlich den
               Tatsachen entsprechend geantwortet, dass ich Ihnen ein solches Gerücht nicht
               bestätigt habe und nicht bestätigen konnte, da ich es von keine Seite, auch von Ihnen
               selbst nicht –, jemals vernommen hatte. Hiemit wäre die Sache nach der einen Seite
               abgetan. Was aber aus der Geschichte leider hervorgeht ist, dass Sie sich befugt
               finden Privatge{\pb}spräche zwischen mir und Ihnen weiter zu tragen –
               in Kreise, die mir äusserlich und innerlich ferne sind und bleiben sollen. Dem
               gegenüber kommt ja meine\introOben{}r\introOben{} Auffassung \introOben{}nach\introOben{} kaum \substVorne{}\textsuperscript{mehr}\substDazwischen{}sonderlich\substHinten{} in Betracht, dass Sie wie dieser Fall beweist, bei solcher Gelegenheit Ihre
               Phantasie in entstellender ja wie es scheint in erfindender Richtung walten lassen.
               Denn wenn ich hier auch die Möglichkeit von Missverständnissen im weitesten Ausmass
               zugestehen wollte, es ist jedenfalls total ausgeschlossen, dass sich Grossmann\pwindex{Grossmann, Stefan 19.05.1875 – 03.01.1935@\textsc{Großmann, Stefan} (19.05.1875 – 03.01.1935), \emph{Schriftsteller, Journalist}|pw} und Kraus\pwindex{Kraus, Karl 28.04.1874 – 12.06.1936@\textsc{Kraus, Karl} (28.04.1874 – 12.06.1936), \emph{Schriftsteller, Publizist}|pw} diese Fabel einfach aus den Fingern gesogen hätten. Dass ich bei
               meinem Ihnen bekannten Ekel vor Literatengezänk – und Geklatsch mich unter diesen
               Umständen genötigt sehe auf die Fortsetzung eines persönlichen Verkehrs mit Ihnen zu
               verzichten, werden Sie ohneweiters einsehen, mit welcher Erklärung die leidige
               Angelegenheit für mich, der ich Wichtigeres zu tun habe, ein für alle Mal erledigt
               ist.\pend
           \pstart
           Hochachtungsvoll{\\[\baselineskip]}\spacefill\mbox{{[}hs.:{]} Dr Arthur Schnitzler}\pend
           \leftskip=0em{}\pstart
           \noindent{}{[}ms.:{]} Herrn Dr. Albert Ehrenstein, Wien\oindex{Wien@\textbf{Wien}|pw}.\pend
           
         
         \endnumbering\mylabel{h}\end{ledgroupsized}  \newcommand{\dateiname}{L02008}\newcommand{\titel}{Arthur Schnitzler an Albert Ehrenstein, 9. 2. 1911}\newcommand{\editorInnen}{Martin Anton Müller und Gerd-Hermann Susen}%% latex-leseansicht-abspann.tex
%% Abspann für die Leseansicht.
%% Der Schalter \ifkorrekturansicht ist bereits durch den Vorspann gesetzt.

%% latex-abspann.tex
%% Gemeinsamer Abspann für Korrekturansicht und Leseansicht.
%% Setzt den Schalter \ifkorrekturansicht voraus (gesetzt in den
%% einbindenden Dateien latex-korrekturansicht-abspann.tex bzw.
%% latex-leseansicht-abspann.tex).
%% ---------------------------------------------------------------

\normalsize

% Das esempio-Environment wird nur in der Leseansicht benötigt
\ifkorrekturansicht\else
\newenvironment{esempio}[3]%
{
    \vspace{1.5ex}
    \rlap{\underline{#1}}
    \par
    \setlength{\parindent}{0cm}
    \nopagebreak
    \leftskip=#2cm
    \rightskip=#3cm
}
{
    \par
}
\fi

\doendnotes{C}
\bigskip
\vfill

\clearpage

\footnotesize

\ifkorrekturansicht
  \lohead{\textsc{register}}
\fi

% theindex-Environment neu definieren ohne reledmac
\makeatletter
\renewenvironment{theindex}{%
  \ifkorrekturansicht
    \section*{\indexname}%
  \else
    \subsubsection*{Index der erwähnten Entitäten}%
  \fi
  \setlength{\parindent}{0pt}%
  \setlength{\parskip}{0pt plus 0.3pt}%
  \let\item\@idxitem
}{%
  \ifkorrekturansicht\clearpage\fi
}
\makeatother

\IfFileExists{\jobname-pw.ind}{\input{\jobname-pw.ind}}{}

% Quellenangabe nur in der Leseansicht
\ifkorrekturansicht\else
% Fallback-Definitionen, falls die .tex-Datei \titel etc. nicht gesetzt hat
\providecommand{\titel}{}
\providecommand{\editorInnen}{}
\providecommand{\dateiname}{\jobname}

\vspace{3cm}

\vfill

\footnotesize
\textsc{Quelle}: \titel. Herausgegeben von {\editorInnen}. In: \emph{Arthur Schnitzler: Briefwechsel mit Autorinnen und Autoren}.
 Digitale Edition, https://schnitzler-briefe.acdh.oeaw.ac.at/{\dateiname}.html (Stand \today)
\fi

\end{document}


      