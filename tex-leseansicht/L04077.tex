%% latex-leseansicht-vorspann.tex
%% Vorspann für die Leseansicht.
%% Lädt die gemeinsame Datei latex-vorspann.tex mit nicht gesetztem Schalter.

\newif\ifkorrekturansicht
\korrekturansichtfalse

\input{../tex-inputs/latex-vorspann}


\section[Arthur Schnitzler an Gustav Schwarzkopf, 2. 8. 1901]{L04077 Arthur Schnitzler an Gustav Schwarzkopf, 2. 8. 1901}
\nopagebreak\mylabel{L04077v}
\rehead{ }\normalsize\beginnumbering\briefempfaengerindex{Schwarzkopf, Gustav@\textsc{Schwarzkopf, Gustav}!zzzSchnitzler, Arthur@\emph{von Arthur Schnitzler}!1901-08-021@{2. 8. 1901}|(be}
\toendnotes[C]{\smallbreak\pagebreak[2]}
\correspDesc{Versand  durch Arthur Schnitzler am 2. 8. 1901 in Vahrn
\newline{}Erhalt  durch Gustav Schwarzkopf im Zeitraum [3. 8. 1901
                  – 7. 8. 1901?] in Wien}\toendnotes[C]{\smallbreak}
\Standort{CUL, Schnitzler, B 96.}
\physDesc{Brief, 1 Blatt, 3 Seiten, 914 Zeichen
\newline{}Handschrift: schwarze Tinte, deutsche Kurrent}\toendnotes[C]{\smallbreak}
\pstart
           \noindent{}{\pb}Lieber Guſtav, Ihre freundliche Karte hab ich erhalten. Wir{ }ſind
               noch immer hier und ich ſuche \textcolor{gray}{nur} gelegentlich nach einem höhern
               Ort, weil{ }ſich \label{K_L04077-1v}\edtext{Paul Goldm.\pwindex{Goldmann, Paul 31.\,1.\,1865 Breslau – 25.\,9.\,1935 Wien@\textsc{Goldmann, Paul} (31.\,1.\,1865 Breslau – 25.\,9.\,1935 Wien), \emph{Schriftsteller, Journalist}|pw} nach einem ſolchen{ }ſehnt}{\lemma{\textnormal{\emph{Paul … sehnt}}}\Cendnote{\textnormal{Vgl. XXXX Auszeichnungsfehler: Dokument L03073 nicht gefunden und XXXX Auszeichnungsfehler: Dokument L03075 nicht gefunden.}}}\label{K_L04077-1},
               habe ſchon mehr oder minder{ }ſchöne Ausflüge gemacht u noch nichts gefunden,
               vielleicht auch aus dem Grund, weil ich es hier ausnehmend behaglich finde. Trotzdem
               iſt es wahrſcheinlich, daſs wir bald weiterfahren.\pend
           
\pstart
           {\pb}Was ſagen Sie zu der letzten Feuilletonungeheuerlichkeit\pwindex{Silberer, Geza 1.\,12.\,1876 Vršac – 5.\,4.\,1938 Wien@\textsc{Silberer, Geza} (1.\,12.\,1876 Vršac – 5.\,4.\,1938 Wien), \emph{Schriftsteller, Journalist}!Irene van Bien@\strich\emph{Irene van Bien}|pwv}
               der N. Fr. Pr\pwindex{Neue Freie Presse@\emph{Neue Freie Presse}|pw}; »\textsc{\label{K_L04077-2v}\edtext{Irene van Bien\pwindex{Silberer, Geza 1.\,12.\,1876 Vršac – 5.\,4.\,1938 Wien@\textsc{Silberer, Geza} (1.\,12.\,1876 Vršac – 5.\,4.\,1938 Wien), \emph{Schriftsteller, Journalist}!Irene van Bien@\strich\emph{Irene van Bien}|pw}}{\lemma{\textnormal{\emph{Irene van Bien}}}\Cendnote{\textnormal{Die Erzählung
                        \emph{Irene van Bien}\pwindex{Silberer, Geza 1.\,12.\,1876 Vršac – 5.\,4.\,1938 Wien@\textsc{Silberer, Geza} (1.\,12.\,1876 Vršac – 5.\,4.\,1938 Wien), \emph{Schriftsteller, Journalist}!Irene van Bien@\strich\emph{Irene van Bien}|pwk} erschien in zwei Teilen
                     am 30. 7. 1901 und am 31. 7. 1901
                           (\emph{Neue Freie Presse}\pwindex{Neue Freie Presse@\emph{Neue Freie Presse}|pwk},
                        Nr. 13.264, Morgenblatt, S. 1–4; Nr. 13.265, Morgenblatt,
                        S. 1–3). Als Autor war Sil
                        Vara\pwindex{Silberer, Geza 1.\,12.\,1876 Vršac – 5.\,4.\,1938 Wien@\textsc{Silberer, Geza} (1.\,12.\,1876 Vršac – 5.\,4.\,1938 Wien), \emph{Schriftsteller, Journalist}|pwk} genannt, das Pseudonym von Geza
                        Silberer\pwindex{Silberer, Geza 1.\,12.\,1876 Vršac – 5.\,4.\,1938 Wien@\textsc{Silberer, Geza} (1.\,12.\,1876 Vršac – 5.\,4.\,1938 Wien), \emph{Schriftsteller, Journalist}|pwk}. Schnitzler
                     entschuldigte sich am XXXX Auszeichnungsfehler: Dokument L04078 nicht gefunden für seine Verwechslung der Autorschaft. Er entschuldigte
                     sich aber nicht für seinen Sexismus, indem er eine Frau als Verfasserin einer
                     Geschichte annahm, in der eine Frau nur durch einen gewaltsam erzwungenen Kuss
                     (Euphemismus für eine Vergewaltigung?) zu lieben lernt.}}}\label{K_L04077-2}}« von »\textsc{Sil Vara}\pwindex{Silberer, Geza 1.\,12.\,1876 Vršac – 5.\,4.\,1938 Wien@\textsc{Silberer, Geza} (1.\,12.\,1876 Vršac – 5.\,4.\,1938 Wien), \emph{Schriftsteller, Journalist}|pw}«! Ich ſchwöre auf Adele Schreiber\pwindex{Schreiber, Adele 29.\,4.\,1872 Wien – 20.\,2.\,1957 Herrliberg@\textsc{Schreiber, Adele} (29.\,4.\,1872 Wien – 20.\,2.\,1957 Herrliberg), \emph{Schriftstellerin, Politikerin, Pädagogin}|pw}, diese
               Literaturweiber können sich die Liebe doch nur unter der Form von
               Nothzuchtsattentaten vorſtellen. Ich hoffe, diese Gans wird ihr Leben lang vergeblich
               von diesem Kukuruz träumen. –\pend
           
\pstart
           Die beiden jungen Damen\pwindex{Schnitzler, Olga 17.\,1.\,1882 Wien – 13.\,1.\,1970 Lugano@\textsc{Schnitzler, Olga} (17.\,1.\,1882 Wien – 13.\,1.\,1970 Lugano), \emph{Schauspielerin, Sängerin}|pwv}\pwindex{Steinrück, Elisabeth 19.\,11.\,1885 Wien – 7.\,4.\,1920 Partenkirchen@\textsc{Steinrück, Elisabeth} (19.\,11.\,1885 Wien – 7.\,4.\,1920 Partenkirchen)|pwv} befinden ſich{ }ſehr wohl und erwiedern Ihre lieben Grüße\pend
           
\pstart
           {\pb}herzlich, gleich Ihrem{\\[\baselineskip]}\spacefill\mbox{ArthSch}\pend
           \leftskip=0em{}
\pstart
           \noindent{}Haben Sie nun wirklich die Abſicht, auch dieſen ganzen Sommer in Wien\oindex{Wien@\textbf{Wien}, \emph{Verwaltungsgebiet}|pw} zu verbringen?\pend
           
\pstart
           \textsc{Vahrn\oindex{Vahrn@\textbf{Vahrn}, \emph{Hauptstadt}|pw}}, 2. 8. 901.\pend
           \selectlanguage{ngerman}\endnumbering\briefempfaengerindex{Schwarzkopf, Gustav@\textsc{Schwarzkopf, Gustav}!zzzSchnitzler, Arthur@\emph{von Arthur Schnitzler}!1901-08-021@{2. 8. 1901}|)be}\mylabel{L04077h}
\begin{anhang}
\end{anhang}\newcommand{\dateiname}{L04077}\newcommand{\titel}{Arthur Schnitzler an Gustav Schwarzkopf, 2. 8. 1901}\newcommand{\editorInnen}{Herausgegeben von Jahnke, SelmaMüller, Martin Anton}%% latex-leseansicht-abspann.tex
%% Abspann für die Leseansicht.
%% Der Schalter \ifkorrekturansicht ist bereits durch den Vorspann gesetzt.

%% latex-abspann.tex
%% Gemeinsamer Abspann für Korrekturansicht und Leseansicht.
%% Setzt den Schalter \ifkorrekturansicht voraus (gesetzt in den
%% einbindenden Dateien latex-korrekturansicht-abspann.tex bzw.
%% latex-leseansicht-abspann.tex).
%% ---------------------------------------------------------------

\normalsize

% Das esempio-Environment wird nur in der Leseansicht benötigt
\ifkorrekturansicht\else
\newenvironment{esempio}[3]%
{
    \vspace{1.5ex}
    \rlap{\underline{#1}}
    \par
    \setlength{\parindent}{0cm}
    \nopagebreak
    \leftskip=#2cm
    \rightskip=#3cm
}
{
    \par
}
\fi

\doendnotes{C}
\bigskip
\vfill

\clearpage

\footnotesize

\ifkorrekturansicht
  \lohead{\textsc{register}}
\fi

% theindex-Environment neu definieren ohne reledmac
\makeatletter
\renewenvironment{theindex}{%
  \ifkorrekturansicht
    \section*{\indexname}%
  \else
    \subsubsection*{Index der erwähnten Entitäten}%
  \fi
  \setlength{\parindent}{0pt}%
  \setlength{\parskip}{0pt plus 0.3pt}%
  \let\item\@idxitem
}{%
  \ifkorrekturansicht\clearpage\fi
}
\makeatother

\IfFileExists{\jobname-pw.ind}{\input{\jobname-pw.ind}}{}

% Quellenangabe nur in der Leseansicht
\ifkorrekturansicht\else
% Fallback-Definitionen, falls die .tex-Datei \titel etc. nicht gesetzt hat
\providecommand{\titel}{}
\providecommand{\editorInnen}{}
\providecommand{\dateiname}{\jobname}

\vspace{3cm}

\vfill

\footnotesize
\textsc{Quelle}: \titel. Herausgegeben von {\editorInnen}. In: \emph{Arthur Schnitzler: Briefwechsel mit Autorinnen und Autoren}.
 Digitale Edition, https://schnitzler-briefe.acdh.oeaw.ac.at/{\dateiname}.html (Stand \today)
\fi

\end{document}


