%% latex-korrekturansicht-vorspann.tex
%% Vorspann für die Korrekturansicht.
%% Lädt die gemeinsame Datei latex-vorspann.tex mit gesetztem Schalter.

\newif\ifkorrekturansicht
\korrekturansichttrue

\input{../tex-inputs/latex-vorspann}


\section[Paul Goldmann an Arthur Schnitzler, 9. 7. 1895]{L02740 Paul Goldmann an Arthur Schnitzler, 9. 7. 1895}
\nopagebreak\mylabel{L02740v}
\rehead{ }\normalsize\beginnumbering\briefempfaengerindex{Schnitzler, Arthur@\textsc{Schnitzler, Arthur}!zzzGoldmann, Paul@\emph{von Paul Goldmann}!1895-07-092@{9. 7. 1895}|(be}
\toendnotes[C]{\smallbreak\pagebreak[2]}\Standort{DLA, A:Schnitzler, HS.NZ85.1.3165.}
\physDesc{Brief, 1 Blatt, 1 Seite, 342 Zeichen
\newline{}Handschrift: schwarze Tinte, deutsche Kurrent
\newline{}Beilage: handschriftlicher Brief, 1 Blatt, 1 Seite, schwarze Tinte,
                                 lateinische Kurrent 
\newline{}Schnitzler: 1) mit schwarzer Tinte das Jahr »95« vermerkt  2) mit rotem Buntstift eine Unterstreichung}\toendnotes[C]{\smallbreak}
\pstart
           {\pb}\textcolor{gray}{\textbf{\textbf{Frankfurter Zeitung\orgindex{Frankfurter Zeitung@Frankfurter Zeitung|pw}}}}\pend
           
\pstart
           \textcolor{gray}{\textbf{(\begin{otherlanguage}{french}Gazette de Francfort\end{otherlanguage}\orgindex{Frankfurter Zeitung@Frankfurter Zeitung|pw}). }}\pend
           
\pstart
           \textcolor{gray}{\textbf{\textbf{\begin{otherlanguage}{french}Fondateur M. L.
                              Sonnemann\pwindex{Sonnemann, Leopold 1831-10-29 – 1909-10-30@\textsc{Sonnemann, Leopold} (1831-10-29 – 1909-10-30), \emph{Journalist/Journalistin, Herausgeber/Herausgeberin}|pw}\end{otherlanguage}.}}}\pend
           
\pstart
           \begin{otherlanguage}{french}\textcolor{gray}{\textbf{Journal\pwindex{Frankfurter Zeitung@\emph{Frankfurter Zeitung}|pwv} politique,
                        financier,}}\end{otherlanguage}\pend
           
\pstart
           \begin{otherlanguage}{french}\textcolor{gray}{\textbf{commercial et littéraire.}}\end{otherlanguage}\pend
           
\pstart
           \begin{otherlanguage}{french}\textcolor{gray}{\textbf{\textbf{Paraissant trois fois par jour.}}}\end{otherlanguage}\hfill \textsc{Paris\oindex{Paris@\textbf{Paris}, \emph{P.PPLC}|pw}}, 9. Juli.\pend
           
\pstart
           \begin{otherlanguage}{french}\textcolor{gray}{\textbf{\textbf{Bureau à Paris\oindex{Paris@\textbf{Paris}, \emph{P.PPLC}|pw}}}}\end{otherlanguage}\pend
           
\pstart
           \begin{otherlanguage}{french}\textcolor{gray}{\textbf{\textbf{24. Rue Feydeau\oindex{rue Feydeau@\textbf{rue Feydeau}, \emph{Straße (K.STR)}|pw}.}}}\end{otherlanguage}\pend
           
\pstart\center{}Mein lieber Freund,\pend\vspace{0.5em}
\pstart
           Eben erhalte ich den beifolgenden Brief von \textsc{Henri Becque\pwindex{Becque, Henry 1837-04-09 – 1899-05-12@\textsc{Becque, Henry} (1837-04-09 – 1899-05-12), \emph{Schriftsteller/Schriftstellerin, Dramatiker/Dramatikerin}|pw}} über »Sterben\pwindex{Sterben. Novelle@\emph{Sterben. Novelle}|pw}«. Nun wollen wir weiter
               ſehen.\pend
           
\pstart
           Herzlichſt {\\[\baselineskip]}Dein {\\[\baselineskip]}\spacefill\mbox{Paul Goldmann.}\pend
           \leftskip=0em{}\selectlanguage{ngerman}\vspace{1em}{\vspace{1\baselineskip}}
\pstart{}{\pb}{[}hs. :{]} \label{K_L02740-1v}\edtext{Mon cher Goldmann}{\lemma{\textnormal{\emph{Mon cher Goldmann}}}\Cendnote{\textnormal{französisch: Mein lieber
                        Goldmann}}}\label{K_L02740-1}\pend\vspace{0.5em}
\pstart
           \label{K_L02740-2v}\edtext{\begin{otherlanguage}{french}Je viens de lire le roman\pwindex{Sterben. Novelle@\emph{Sterben. Novelle}|pwv} de votre ami. C’est très douloureux et \label{K_L02740-3v}\edtext{toût à fait remarquable}{\lemma{\textnormal{\emph{toût à fait remarquable}}}\Cendnote{\textnormal{Vgl. A. S.: \emph{Tagebuch}, 15. 7. 1895.
                  }}}\label{K_L02740-3}. Pourquoi m’avez vous demandé d’en prendre connaissance?\end{otherlanguage}}{\lemma{\textnormal{\emph{Je … connaissance?}}}\Cendnote{\textnormal{französisch: Ich habe eben den
                        Roman\pwindex{Sterben. Novelle@\emph{Sterben. Novelle}|pwv} Ihres Freundes
                     gelesen. Es ist sehr schmerzhaft und vollständig bemerkenswert. Warum haben Sie
                     mich um Kenntnisnahme gebeten?}}}\label{K_L02740-2}\pend
           
\pstart
           \label{K_L02740-4v}\edtext{\begin{otherlanguage}{french}Bien à vous\end{otherlanguage}}{\lemma{\textnormal{\emph{Bien à vous}}}\Cendnote{\textnormal{französisch: Der Ihre}}}\label{K_L02740-4}{\\[\baselineskip]}\spacefill\mbox{Henry Becque\pwindex{Becque, Henry 1837-04-09 – 1899-05-12@\textsc{Becque, Henry} (1837-04-09 – 1899-05-12), \emph{Schriftsteller/Schriftstellerin, Dramatiker/Dramatikerin}|pw}}\pend
           \leftskip=0em{}\selectlanguage{ngerman}\endnumbering\briefempfaengerindex{Schnitzler, Arthur@\textsc{Schnitzler, Arthur}!zzzGoldmann, Paul@\emph{von Paul Goldmann}!1895-07-092@{9. 7. 1895}|)be}\mylabel{L02740h}  \normalsize

\doendnotes{C}
\bigskip
\vfill

\clearpage

\footnotesize

\lohead{\textsc{register}}

% Definiere theindex-Environment komplett neu ohne reledmac
\makeatletter
\renewenvironment{theindex}{%
  \section*{\indexname}%
  \setlength{\parindent}{0pt}%
  \setlength{\parskip}{0pt plus 0.3pt}%
  \let\item\@idxitem
}{%
  \clearpage
}
\makeatother

\IfFileExists{\jobname-pw.ind}{\input{\jobname-pw.ind}}{}

\end{document}

      