%% latex-leseansicht-vorspann.tex
%% Vorspann für die Leseansicht.
%% Lädt die gemeinsame Datei latex-vorspann.tex mit nicht gesetztem Schalter.

\newif\ifkorrekturansicht
\korrekturansichtfalse

\input{../tex-inputs/latex-vorspann}


\section[ Felix Salten an Arthur Schnitzler, 28. 7. 1901]{L03316 Felix Salten an Arthur Schnitzler,  28. 7. 1901}
\nopagebreak\mylabel{L03316v}
\rehead{ }\normalsize\beginnumbering\briefempfaengerindex{Schnitzler, Arthur@\textsc{Schnitzler, Arthur}!zzzSalten, Felix@\emph{von Felix Salten}!1901-07-281@{28. 7. 1901}|(be}
\toendnotes[C]{\smallbreak\pagebreak[2]}
\correspDesc{Versand  durch Felix Salten am 28. 7. 1901 in Bad Ischl
\newline{}Erhalt  durch Arthur Schnitzler im Zeitraum [29. 7. 1901
                  – 2. 8. 1901?] in Vahrn}\toendnotes[C]{\smallbreak}
\Standort{CUL, Schnitzler, B 89, A 2.}
\physDesc{Brief, 1 Blatt, 2 Seiten, 935 Zeichen
\newline{}Handschrift: schwarze Tinte, lateinische Kurrent
\newline{}Ordnung: mit Bleistift von unbekannter Hand nummeriert: »140« }\toendnotes[C]{\smallbreak}
\pstart
           \raggedleft{}{\pb}Ischl\oindex{Bad Ischl@\textbf{Bad Ischl}|pw}, 28. Juli 01\pend
           \vspace{0.5em}
\pstart
           Lieber Freund.{ }Dienstag gehe ich nach Wien\oindex{Wien@\textbf{Wien}, \emph{Verwaltungsgebiet}|pw} zurück. Bleibe dort ein paar Wochen, dann muß ich freilich wieder
               hierher. Dann habe ich noch ein paar Fahrten nach München\oindex{München@\textbf{München}|pw}{ }{\kaufmannsund} nach Berlin\oindex{Berlin@\textbf{Berlin}, \emph{Hauptstadt}|pw} zu
               machen, aber erst im September. Vielleicht ist es nöthig,
               dass ich vorher, Ende August, od. Anfangs Septemb. noch mit Felix\pwindex{Felix, Hugo 19.\,11.\,1866 Budapest – 25.\,8.\,1934 Hollywood@\textsc{Felix, Hugo} (19.\,11.\,1866 Budapest – 25.\,8.\,1934 Hollywood), \emph{Komponist, Chemiker}|pw} zusammentreffe. Er schlägt Verona\oindex{Verona@\textbf{Verona}, \emph{Hauptstadt}|pw} vor, ich Venedig\oindex{Venedig@\textbf{Venedig}|pw}.
               Wenn Sie nun diese Zeit am \label{K_L03316-1v}\edtext{Gardasee\oindex{Lago di Garda@\textbf{Lago di Garda}, \emph{See}|pw}}{\lemma{\textnormal{\emph{Gardasee}}}\Cendnote{\textnormal{Das war nicht der Fall.}}}\label{K_L03316-1} sind,
               könnten wir, falls es Ihnen recht ist dorthin, oder doch in die Nähe kommen. Vor
               wenigen Tagen war Bogumil Zepler\pwindex{Zepler, Bogumil 6.\,5.\,1858 Breslau – 17.\,8.\,1918 Karpacz@\textsc{Zepler, Bogumil} (6.\,5.\,1858 Breslau – 17.\,8.\,1918 Karpacz), \emph{Komponist}|pw} da, mit
               hübschen neuen \label{K_L03316-2v}\edtext{Sachen\pwindex{Willomitzer, Josef 17.\,4.\,1849 Benešov – 3.\,10.\,1900 Prag@\textsc{Willomitzer, Josef} (17.\,4.\,1849 Benešov – 3.\,10.\,1900 Prag), \emph{Schriftsteller, Journalist}!Neue Loreley@\strich\emph{Neue Loreley}|pwv}\pwindex{Vazeh, Mirzä Şäfi um 1796 Ganja – 16.\,11.\,1852 Tiflis@\textsc{Vazeh, Mirzä Şäfi} (um 1796 Ganja – 16.\,11.\,1852 Tiflis), \emph{Schriftsteller}!Hafisa@\strich\emph{Hafisa}|pwv}}{\lemma{\textnormal{\emph{Sachen}}}\Cendnote{\textnormal{Zwei Lieder\pwindex{Willomitzer, Josef 17.\,4.\,1849 Benešov – 3.\,10.\,1900 Prag@\textsc{Willomitzer, Josef} (17.\,4.\,1849 Benešov – 3.\,10.\,1900 Prag), \emph{Schriftsteller, Journalist}!Neue Loreley@\strich\emph{Neue Loreley}|pwkv}\pwindex{Vazeh, Mirzä Şäfi um 1796 Ganja – 16.\,11.\,1852 Tiflis@\textsc{Vazeh, Mirzä Şäfi} (um 1796 Ganja – 16.\,11.\,1852 Tiflis), \emph{Schriftsteller}!Hafisa@\strich\emph{Hafisa}|pwkv} lassen sich nachweisen, wobei nur das
                  zweite bei der Premiere am 16. 11. 1901 im Jung-Wiener Theater zum
                     Lieben Augustin\oindex{Wien@\textbf{Wien}!VI., Mariahilf@\textbf{VI., Mariahilf}!Jung-Wiener Theater zum Lieben Augustin@\textbf{Jung-Wiener Theater zum Lieben Augustin}, \emph{Kabarett}|pwk} aufgeführt wurde: \emph{Neue
                     Loreley}\pwindex{Willomitzer, Josef 17.\,4.\,1849 Benešov – 3.\,10.\,1900 Prag@\textsc{Willomitzer, Josef} (17.\,4.\,1849 Benešov – 3.\,10.\,1900 Prag), \emph{Schriftsteller, Journalist}!Neue Loreley@\strich\emph{Neue Loreley}|pwk} (Balladentext von Josef
                     Willomitzer\pwindex{Willomitzer, Josef 17.\,4.\,1849 Benešov – 3.\,10.\,1900 Prag@\textsc{Willomitzer, Josef} (17.\,4.\,1849 Benešov – 3.\,10.\,1900 Prag), \emph{Schriftsteller, Journalist}|pwk}) und \emph{Hafisa}\pwindex{Vazeh, Mirzä Şäfi um 1796 Ganja – 16.\,11.\,1852 Tiflis@\textsc{Vazeh, Mirzä Şäfi} (um 1796 Ganja – 16.\,11.\,1852 Tiflis), \emph{Schriftsteller}!Hafisa@\strich\emph{Hafisa}|pwk} nach einer
                  Vorlage von Mirzä Şäfi Vazeh\pwindex{Vazeh, Mirzä Şäfi um 1796 Ganja – 16.\,11.\,1852 Tiflis@\textsc{Vazeh, Mirzä Şäfi} (um 1796 Ganja – 16.\,11.\,1852 Tiflis), \emph{Schriftsteller}|pwk} in der
                  Übersetzung von Friedrich von
                  Bodenstedt\pwindex{Bodenstedt, Friedrich von 22.\,4.\,1819 Peine – 18.\,4.\,1892 Wiesbaden@\textsc{Bodenstedt, Friedrich von} (22.\,4.\,1819 Peine – 18.\,4.\,1892 Wiesbaden), \emph{Schriftsteller}|pwk}.}}}\label{K_L03316-2}, die ich erworben habe. Von den Wien\oindex{Wien@\textbf{Wien}, \emph{Verwaltungsgebiet}|pw}er Leuten ist nichts, aber auch noch garnichts da, was die
               Sache allerdings nicht erleichtert. Doch war ich darauf {\pb}so ziemlich vorbereitet.\pend
           
\pstart
           Dass wir \label{K_L03316-3v}\edtext{im selben Zug fuhren}{\lemma{\textnormal{\emph{im selben Zug fuhren}}}\Cendnote{\textnormal{Vermutlich im Zug von Feldkirch\oindex{Feldkirch@\textbf{Feldkirch}, \emph{Hauptstadt}|pwk} nach St.
                     Anton\oindex{St. Anton am Arlberg@\textbf{St. Anton am Arlberg}, \emph{Verwaltungsgebiet}|pwk} am 10. 7. 1901, vgl. XXXX Auszeichnungsfehler: Dokument L03315 nicht gefunden. }}}\label{K_L03316-3} und uns nicht sahen? Von wo –? und bis wohin?\pend
           
\pstart
           Gratuliere zum neuen \label{K_L03316-4v}\edtext{Stück\pwindex{Schnitzler, Arthur 15.\,5.\,1862 Wien – 21.\,10.\,1931 ebd.@\textsc{Schnitzler, Arthur} (15.\,5.\,1862 Wien – 21.\,10.\,1931 ebd.), \emph{Schriftsteller, Mediziner}!einsame Weg. Schauspiel in fünf Akten@\strich\emph{Der einsame Weg. Schauspiel in fünf Akten}|pwv}}{\lemma{\textnormal{\emph{Stück}}}\Cendnote{\textnormal{Siehe XXXX Auszeichnungsfehler: Dokument L02969 nicht gefunden.
               }}}\label{K_L03316-4} und bin sehr neugierig. Die \label{K_L03316-5v}\edtext{Prinzessin Anna\pwindex{Salten, Felix 6.\,9.\,1869 Budapest – 8.\,10.\,1945 Zürich@\textsc{Salten, Felix} (6.\,9.\,1869 Budapest – 8.\,10.\,1945 Zürich), \emph{Schriftsteller, Journalist, Chefredakteur}!Gedenktafel der Prinzessin Anna@\strich\emph{Die Gedenktafel der Prinzessin Anna}|pw} ist erschienen. Soll ich Ihnen
               das Heft der »Insel\pwindex{Insel. Monatsschrift mit Buchschmuck und Illustrationen@\emph{Die Insel. Monatsschrift mit Buchschmuck und Illustrationen}|pw}}{\lemma{\textnormal{\emph{Prinzessin … »Insel}}}\Cendnote{\textnormal{Felix Salten\pwindex{Salten, Felix 6.\,9.\,1869 Budapest – 8.\,10.\,1945 Zürich@\textsc{Salten, Felix} (6.\,9.\,1869 Budapest – 8.\,10.\,1945 Zürich), \emph{Schriftsteller, Journalist, Chefredakteur}|pwk}: \emph{Die Gedenktafel der Prinzessin Anna}\pwindex{Salten, Felix 6.\,9.\,1869 Budapest – 8.\,10.\,1945 Zürich@\textsc{Salten, Felix} (6.\,9.\,1869 Budapest – 8.\,10.\,1945 Zürich), \emph{Schriftsteller, Journalist, Chefredakteur}!Gedenktafel der Prinzessin Anna@\strich\emph{Die Gedenktafel der Prinzessin Anna}|pwk}. In: \emph{Die Insel}\pwindex{Insel. Monatsschrift mit Buchschmuck und Illustrationen@\emph{Die Insel. Monatsschrift mit Buchschmuck und Illustrationen}|pwk}, Jg. 2, Quartal 4, Nr. 10, Juli 1901, S. 67–117.}}}\label{K_L03316-5}« schicken?\pend
           
\pstart
           Herzlichst {\\[\baselineskip]}Ihr {\\[\baselineskip]}\spacefill\mbox{Salten}\pend
           \leftskip=0em{}\selectlanguage{ngerman}\endnumbering\briefempfaengerindex{Schnitzler, Arthur@\textsc{Schnitzler, Arthur}!zzzSalten, Felix@\emph{von Felix Salten}!1901-07-281@{28. 7. 1901}|)be}\mylabel{L03316h}  \newcommand{\dateiname}{L03316}\newcommand{\titel}{Felix Salten an Arthur Schnitzler, 28. 7. 1901}\newcommand{\editorInnen}{Martin Anton Müller und Laura Untner}%% latex-leseansicht-abspann.tex
%% Abspann für die Leseansicht.
%% Der Schalter \ifkorrekturansicht ist bereits durch den Vorspann gesetzt.

%% latex-abspann.tex
%% Gemeinsamer Abspann für Korrekturansicht und Leseansicht.
%% Setzt den Schalter \ifkorrekturansicht voraus (gesetzt in den
%% einbindenden Dateien latex-korrekturansicht-abspann.tex bzw.
%% latex-leseansicht-abspann.tex).
%% ---------------------------------------------------------------

\normalsize

% Das esempio-Environment wird nur in der Leseansicht benötigt
\ifkorrekturansicht\else
\newenvironment{esempio}[3]%
{
    \vspace{1.5ex}
    \rlap{\underline{#1}}
    \par
    \setlength{\parindent}{0cm}
    \nopagebreak
    \leftskip=#2cm
    \rightskip=#3cm
}
{
    \par
}
\fi

\doendnotes{C}
\bigskip
\vfill

\clearpage

\footnotesize

\ifkorrekturansicht
  \lohead{\textsc{register}}
\fi

% theindex-Environment neu definieren ohne reledmac
\makeatletter
\renewenvironment{theindex}{%
  \ifkorrekturansicht
    \section*{\indexname}%
  \else
    \subsubsection*{Index der erwähnten Entitäten}%
  \fi
  \setlength{\parindent}{0pt}%
  \setlength{\parskip}{0pt plus 0.3pt}%
  \let\item\@idxitem
}{%
  \ifkorrekturansicht\clearpage\fi
}
\makeatother

\IfFileExists{\jobname-pw.ind}{\input{\jobname-pw.ind}}{}

% Quellenangabe nur in der Leseansicht
\ifkorrekturansicht\else
% Fallback-Definitionen, falls die .tex-Datei \titel etc. nicht gesetzt hat
\providecommand{\titel}{}
\providecommand{\editorInnen}{}
\providecommand{\dateiname}{\jobname}

\vspace{3cm}

\vfill

\footnotesize
\textsc{Quelle}: \titel. Herausgegeben von {\editorInnen}. In: \emph{Arthur Schnitzler: Briefwechsel mit Autorinnen und Autoren}.
 Digitale Edition, https://schnitzler-briefe.acdh.oeaw.ac.at/{\dateiname}.html (Stand \today)
\fi

\end{document}


