%% latex-korrekturansicht-vorspann.tex
%% Vorspann für die Korrekturansicht.
%% Lädt die gemeinsame Datei latex-vorspann.tex mit gesetztem Schalter.

\newif\ifkorrekturansicht
\korrekturansichttrue

\input{../tex-inputs/latex-vorspann}


\section[ Felix Salten an Arthur Schnitzler, 28. 7. 1901]{L03316 Felix Salten an Arthur Schnitzler, 28. 7. 1901}
\nopagebreak\mylabel{L03316v}
\rehead{ }\normalsize\beginnumbering\briefempfaengerindex{Schnitzler, Arthur@\textsc{Schnitzler, Arthur}!zzzSalten, Felix@\emph{von Felix Salten}!1901-07-281@{28. 7. 1901}|(be}
\toendnotes[C]{\smallbreak\pagebreak[2]}\Standort{CUL, Schnitzler, B 89, A 2.}
\physDesc{Brief, 1 Blatt, 2 Seiten, 935 Zeichen
\newline{}Handschrift: schwarze Tinte, lateinische Kurrent
\newline{}Ordnung: mit Bleistift von unbekannter Hand nummeriert: »140« }\toendnotes[C]{\smallbreak}
\pstart
           \raggedleft{}{\pb}Ischl\oindex{Bad Ischl@\textbf{Bad Ischl}, \emph{P.PPL}|pw}, 28. Juli 01\pend
           \vspace{0.5em}
\pstart
           Lieber Freund.{ }Dienstag gehe ich nach Wien\oindex{Wien@\textbf{Wien}, \emph{A.ADM2}|pw} zurück. Bleibe dort ein paar Wochen, dann muß ich freilich wieder
               hierher. Dann habe ich noch ein paar Fahrten nach München\oindex{Muenchen@\textbf{München}, \emph{P.PPLA}|pw}{ }{\kaufmannsund} nach Berlin\oindex{Berlin@\textbf{Berlin}, \emph{P.PPLC}|pw} zu
               machen, aber erst im September. Vielleicht ist es nöthig,
               dass ich vorher, Ende August, od. Anfangs Septemb. noch mit Felix\pwindex{Felix, Hugo 19.11.1866 – 25.08.1934@\textsc{Felix, Hugo} (19.11.1866 – 25.08.1934), \emph{Komponist/Komponistin, Chemiker/Chemikerin}|pw} zusammentreffe. Er schlägt Verona\oindex{Verona@\textbf{Verona}, \emph{P.PPLA2}|pw} vor, ich Venedig\oindex{Venedig@\textbf{Venedig}, \emph{P.PPLA}|pw}.
               Wenn Sie nun diese Zeit am \label{K_L03316-1v}\edtext{Gardasee\oindex{Lago di Garda@\textbf{Lago di Garda}, \emph{See (N.SEE)}|pw}}{\lemma{\textnormal{\emph{Gardasee}}}\Cendnote{\textnormal{Das war nicht der Fall.}}}\label{K_L03316-1} sind,
               könnten wir, falls es Ihnen recht ist dorthin, oder doch in die Nähe kommen. Vor
               wenigen Tagen war Bogumil Zepler\pwindex{Zepler, Bogumil 06.05.1858 – 17.08.1918@\textsc{Zepler, Bogumil} (06.05.1858 – 17.08.1918), \emph{Komponist/Komponistin}|pw} da, mit
               hübschen neuen \label{K_L03316-2v}\edtext{Sachen\pwindex{Neue Loreley@\emph{Neue Loreley}|pwv}\pwindex{Hafisa@\emph{Hafisa}|pwv}}{\lemma{\textnormal{\emph{Sachen}}}\Cendnote{\textnormal{Zwei Lieder\pwindex{Neue Loreley@\emph{Neue Loreley}|pwkv}\pwindex{Hafisa@\emph{Hafisa}|pwkv} lassen sich nachweisen, wobei nur das
                  zweite bei der Premiere am 16. 11. 1901 im Jung-Wiener Theater zum
                     Lieben Augustin\oindex{Jung-Wiener Theater zum Lieben Augustin@\textbf{Jung-Wiener Theater zum Lieben Augustin}, \emph{Kabarett (K.KBR)}|pwk} aufgeführt wurde: \emph{Neue
                     Loreley}\pwindex{Neue Loreley@\emph{Neue Loreley}|pwk} (Balladentext von Josef
                     Willomitzer\pwindex{Willomitzer, Josef 1849-04-17 – 1900-10-03@\textsc{Willomitzer, Josef} (1849-04-17 – 1900-10-03), \emph{Schriftsteller/Schriftstellerin, Journalist/Journalistin}|pwk}) und \emph{Hafisa}\pwindex{Hafisa@\emph{Hafisa}|pwk} nach einer
                  Vorlage von Mirzä Şäfi Vazeh\pwindex{Vazeh, Mirzae Şaefi um 1796 – 1852-11-16@\textsc{Vazeh, Mirzä Şäfi} (um 1796 – 1852-11-16), \emph{Schriftsteller/Schriftstellerin}|pwk} in der
                  Übersetzung von Friedrich von
                  Bodenstedt\pwindex{Bodenstedt, Friedrich von 22.04.1819 – 18.04.1892@\textsc{Bodenstedt, Friedrich von} (22.04.1819 – 18.04.1892), \emph{Schriftsteller/Schriftstellerin}|pwk}.}}}\label{K_L03316-2}, die ich erworben habe. Von den Wien\oindex{Wien@\textbf{Wien}, \emph{A.ADM2}|pw}er Leuten ist nichts, aber auch noch garnichts da, was die
               Sache allerdings nicht erleichtert. Doch war ich darauf {\pb}so ziemlich vorbereitet.\pend
           
\pstart
           Dass wir \label{K_L03316-3v}\edtext{im selben Zug fuhren}{\lemma{\textnormal{\emph{im selben Zug fuhren}}}\Cendnote{\textnormal{Vermutlich im Zug von Feldkirch\oindex{Feldkirch@\textbf{Feldkirch}, \emph{P.PPLA2}|pwk} nach St.
                     Anton\oindex{St. Anton am Arlberg@\textbf{St. Anton am Arlberg}, \emph{A.ADM3}|pwk} am 10. 7. 1901, vgl. Felix Salten an Arthur Schnitzler, 11. 7. 1901. }}}\label{K_L03316-3} und uns nicht sahen? Von wo –? und bis wohin?\pend
           
\pstart
           Gratuliere zum neuen \label{K_L03316-4v}\edtext{Stück\pwindex{einsame Weg. Schauspiel in fuenf Akten@\emph{Der einsame Weg. Schauspiel in fünf Akten}|pwv}}{\lemma{\textnormal{\emph{Stück}}}\Cendnote{\textnormal{Siehe Arthur Schnitzler an Felix Salten, 10. 8. 1901.
               }}}\label{K_L03316-4} und bin sehr neugierig. Die \label{K_L03316-5v}\edtext{Prinzessin Anna\pwindex{Gedenktafel der Prinzessin Anna@\emph{Die Gedenktafel der Prinzessin Anna}|pw} ist erschienen. Soll ich Ihnen
               das Heft der »Insel\pwindex{Insel. Monatsschrift mit Buchschmuck und Illustrationen@\emph{Die Insel. Monatsschrift mit Buchschmuck und Illustrationen}|pw}}{\lemma{\textnormal{\emph{Prinzessin … »Insel}}}\Cendnote{\textnormal{Felix Salten\pwindex{Salten, Felix 06.09.1869 – 08.10.1945@\textsc{Salten, Felix} (06.09.1869 – 08.10.1945), \emph{Schriftsteller/Schriftstellerin, Journalist/Journalistin, Chefredakteur/Chefredakteurin}|pwk}: \emph{Die Gedenktafel der Prinzessin Anna}\pwindex{Gedenktafel der Prinzessin Anna@\emph{Die Gedenktafel der Prinzessin Anna}|pwk}. In: \emph{Die Insel}\pwindex{Insel. Monatsschrift mit Buchschmuck und Illustrationen@\emph{Die Insel. Monatsschrift mit Buchschmuck und Illustrationen}|pwk}, Jg. 2, Quartal 4, Nr. 10, Juli 1901, S. 67–117.}}}\label{K_L03316-5}« schicken?\pend
           
\pstart
           Herzlichst {\\[\baselineskip]}Ihr {\\[\baselineskip]}\spacefill\mbox{Salten}\pend
           \leftskip=0em{}\selectlanguage{ngerman}\endnumbering\briefempfaengerindex{Schnitzler, Arthur@\textsc{Schnitzler, Arthur}!zzzSalten, Felix@\emph{von Felix Salten}!1901-07-281@{28. 7. 1901}|)be}\mylabel{L03316h}  \normalsize

\doendnotes{C}
\bigskip
\vfill

\clearpage

\footnotesize

\lohead{\textsc{register}}

% Definiere theindex-Environment komplett neu ohne reledmac
\makeatletter
\renewenvironment{theindex}{%
  \section*{\indexname}%
  \setlength{\parindent}{0pt}%
  \setlength{\parskip}{0pt plus 0.3pt}%
  \let\item\@idxitem
}{%
  \clearpage
}
\makeatother

\IfFileExists{\jobname-pw.ind}{\input{\jobname-pw.ind}}{}

\end{document}

      