%% latex-leseansicht-vorspann.tex
%% Vorspann für die Leseansicht.
%% Lädt die gemeinsame Datei latex-vorspann.tex mit nicht gesetztem Schalter.

\newif\ifkorrekturansicht
\korrekturansichtfalse

\input{../tex-inputs/latex-vorspann}


         
         \renewcommand{\erwaehntePersonen}{Personen: Felix Salten, Ferdinand Schmutzer}
         \renewcommand{\erwaehnteInstitutionen}{Institutionen: S. Fischer Verlag}
         \renewcommand{\erwaehnteOrte}{Orte: Berlin, Wien}
         \renewcommand{\erwaehnteWerke}{Werke: Der blinde Geronimo und sein Bruder}
               \section[Arthur Schnitzler und Ferdinand Schmutzer: Widmungsexemplar Der blinde Geronimo und sein Bruder für Felix Salten, {[}29. 5. 1915?{]}]{ Arthur Schnitzler und Ferdinand Schmutzer: Widmungsexemplar Der blinde
               Geronimo und sein Bruder für Felix Salten, {[}29. 5. 1915?{]}}\nopagebreak\mylabel{v}\rehead{ }\begin{ledgroupsized}[t]{13cm}\normalsize\beginnumbering\briefempfaengerindex{Salten, Felix@\textsc{Salten, Felix}!zzzSchnitzler, Arthur@\emph{von Arthur Schnitzler}!1915-05-291@{{[}29. 5. 1915?{]}}|(be} \toendnotes[C]{\smallbreak\pagebreak[2]} \Standort{Wienbibliothek im Rathaus, A-72875/2.Ex., DS-2018-9502.}
\physDesc{Widmung am Schmutztitel, 73 Zeichen
\newline{}Handschrift: schwarze Tinte, lateinische Kurrent}\toendnotes[C]{\smallbreak}\pstart
           \noindent{}\centering{}{\pb}für Felix Salten\pend
           \pstart
           \noindent{}\centering{}mit {\\}herzlichen Grüßen.\pend
           {\bigskip}\pstart
           \noindent{}\centering{}\textcolor{gray}{\textbf{\emph{Herausgegeben 1915}}}{\\}\textcolor{gray}{\textbf{\emph{zu Gunsten der im Feld Erblindeten}}}\pend
           {\bigskip}\pstart
           \noindent{}\centering{}{[}hs. Schmutzer:{]} \spacefill\mbox{Ferd. Schmutzer}{\\}{[}hs. Schnitzler:{]} \spacefill\mbox{Arthur Schnitzler}\pend
           {\bigskip}\pstart
           \noindent{}\centering{}{\pb}\textcolor{gray}{\textbf{\emph{DER BLINDE GERONIMO {\\}UND SEIN
                           BRUDER\pwindex{Schnitzler, Arthur 15.05.1862 – 21.10.1931@\textsc{Schnitzler, Arthur} (15.05.1862 – 21.10.1931), \emph{Schriftsteller, Mediziner}!blinde Geronimo und sein Bruder1915-05-29@\strich\emph{Der blinde Geronimo und sein Bruder} {[}1915-05-29{]}|pw}}}}\pend
           \pstart
           \noindent{}\centering{}\textcolor{gray}{\textbf{\emph{Erzählung}}}{\\}\textcolor{gray}{\textbf{\so{von}}}{\\}\textcolor{gray}{\textbf{\emph{ARTHUR SCHNITZLER}}}\pend
           \pstart
           \noindent{}\centering{}\textcolor{gray}{\textbf{\emph{Mit einer Originalradierung}}}\pend
           \pstart
           \noindent{}\centering{}\textcolor{gray}{\textbf{von}}{\\}\textcolor{gray}{\textbf{\emph{FERDINAND SCHMUTZER}}}\pend
           {\bigskip}\pstart
           \noindent{}\centering{}\textcolor{gray}{\textbf{\emph{S. FISCHER ⋅ VERLAG\orgindex{S. Fischer Verlag@S. Fischer Verlag|pw}}}}{\\}\textcolor{gray}{\textbf{\emph{BERLIN\oindex{Berlin@\textbf{Berlin}|pw}{ }\label{K_L03614-1v}\edtext{1915}{\lemma{\textnormal{\emph{1915}}}\Cendnote{\textnormal{Zur Datierung 
                           siehe A. S.: \emph{Tagebuch}, 29. 5. 1915.
                        }}}\label{K_L03614-1h}}}}\pend
           
         
         \endnumbering\mylabel{h}\end{ledgroupsized}  \newcommand{\dateiname}{L03614}\newcommand{\titel}{Arthur Schnitzler und Ferdinand Schmutzer: Widmungsexemplar Der blinde Geronimo und sein Bruder für Felix Salten, [29. 5. 1915?]}\newcommand{\editorInnen}{Martin Anton Müller und Laura Untner}%% latex-leseansicht-abspann.tex
%% Abspann für die Leseansicht.
%% Der Schalter \ifkorrekturansicht ist bereits durch den Vorspann gesetzt.

%% latex-abspann.tex
%% Gemeinsamer Abspann für Korrekturansicht und Leseansicht.
%% Setzt den Schalter \ifkorrekturansicht voraus (gesetzt in den
%% einbindenden Dateien latex-korrekturansicht-abspann.tex bzw.
%% latex-leseansicht-abspann.tex).
%% ---------------------------------------------------------------

\normalsize

% Das esempio-Environment wird nur in der Leseansicht benötigt
\ifkorrekturansicht\else
\newenvironment{esempio}[3]%
{
    \vspace{1.5ex}
    \rlap{\underline{#1}}
    \par
    \setlength{\parindent}{0cm}
    \nopagebreak
    \leftskip=#2cm
    \rightskip=#3cm
}
{
    \par
}
\fi

\doendnotes{C}
\bigskip
\vfill

\clearpage

\footnotesize

\ifkorrekturansicht
  \lohead{\textsc{register}}
\fi

% theindex-Environment neu definieren ohne reledmac
\makeatletter
\renewenvironment{theindex}{%
  \ifkorrekturansicht
    \section*{\indexname}%
  \else
    \subsubsection*{Index der erwähnten Entitäten}%
  \fi
  \setlength{\parindent}{0pt}%
  \setlength{\parskip}{0pt plus 0.3pt}%
  \let\item\@idxitem
}{%
  \ifkorrekturansicht\clearpage\fi
}
\makeatother

\IfFileExists{\jobname-pw.ind}{\input{\jobname-pw.ind}}{}

% Quellenangabe nur in der Leseansicht
\ifkorrekturansicht\else
% Fallback-Definitionen, falls die .tex-Datei \titel etc. nicht gesetzt hat
\providecommand{\titel}{}
\providecommand{\editorInnen}{}
\providecommand{\dateiname}{\jobname}

\vspace{3cm}

\vfill

\footnotesize
\textsc{Quelle}: \titel. Herausgegeben von {\editorInnen}. In: \emph{Arthur Schnitzler: Briefwechsel mit Autorinnen und Autoren}.
 Digitale Edition, https://schnitzler-briefe.acdh.oeaw.ac.at/{\dateiname}.html (Stand \today)
\fi

\end{document}


      