%% latex-korrekturansicht-vorspann.tex
%% Vorspann für die Korrekturansicht.
%% Lädt die gemeinsame Datei latex-vorspann.tex mit gesetztem Schalter.

\newif\ifkorrekturansicht
\korrekturansichttrue

\input{../tex-inputs/latex-vorspann}


\section[Arthur Schnitzler und Ferdinand Schmutzer: Widmungsexemplar Der blinde Geronimo und sein Bruder für Felix Salten, {[}29. 5. 1915?{]}]{L03614 Arthur Schnitzler und Ferdinand Schmutzer: Widmungsexemplar Der blinde
               Geronimo und sein Bruder für Felix Salten, {[}29. 5. 1915?{]}}
\nopagebreak\mylabel{L03614v}
\rehead{ }\normalsize\beginnumbering\briefempfaengerindex{Salten, Felix@\textsc{Salten, Felix}!zzzSchnitzler, Arthur@\emph{von Arthur Schnitzler}!1915-05-291@{{[}29. 5. 1915?{]}}|(be}
\toendnotes[C]{\smallbreak\pagebreak[2]}\Standort{Wienbibliothek im Rathaus, A-72875/2.Ex., DS-2018-9502.}
\physDesc{Widmung am Schmutztitel, 73 Zeichen
\newline{}Handschrift: schwarze Tinte, lateinische Kurrent}\toendnotes[C]{\smallbreak}
\pstart
           \noindent{}\centering{}{\pb}für Felix Salten\pend
           
\pstart
           \centering{}mit {\\}herzlichen Grüßen.\pend
           {\vspace{1\baselineskip}}
\pstart
           \centering{}\textcolor{gray}{\textbf{\emph{Herausgegeben 1915}}}{\\}\textcolor{gray}{\textbf{\emph{zu Gunsten der im Feld Erblindeten}}}\pend
           {\vspace{1\baselineskip}}
\pstart
           \centering{}{[}hs. :{]} \spacefill\mbox{Ferd. Schmutzer}{\\}{[}hs. :{]} \spacefill\mbox{Arthur Schnitzler}\pend
           \selectlanguage{ngerman}\vspace{1em}{\vspace{1\baselineskip}}
\pstart
           \centering{}{\pb}\textcolor{gray}{\textbf{\emph{DER BLINDE GERONIMO {\\}UND SEIN
                           BRUDER\pwindex{blinde Geronimo und sein Bruder@\emph{Der blinde Geronimo und sein Bruder}|pw}}}}\pend
           
\pstart
           \centering{}\textcolor{gray}{\textbf{\emph{Erzählung}}}{\\}\textcolor{gray}{\textbf{\so{von}}}{\\}\textcolor{gray}{\textbf{\emph{ARTHUR SCHNITZLER}}}\pend
           
\pstart
           \centering{}\textcolor{gray}{\textbf{\emph{Mit einer Originalradierung}}}\pend
           
\pstart
           \centering{}\textcolor{gray}{\textbf{von}}{\\}\textcolor{gray}{\textbf{\emph{FERDINAND SCHMUTZER}}}\pend
           {\vspace{1\baselineskip}}
\pstart
           \centering{}\textcolor{gray}{\textbf{\emph{S. FISCHER ⋅ VERLAG\orgindex{S. Fischer Verlag@S. Fischer Verlag|pw}}}}{\\}\textcolor{gray}{\textbf{\emph{BERLIN\oindex{Berlin@\textbf{Berlin}, \emph{P.PPLC}|pw}{ }\label{K_L03614-1v}\edtext{1915}{\lemma{\textnormal{\emph{1915}}}\Cendnote{\textnormal{Zur Datierung 
                           siehe A. S.: \emph{Tagebuch}, 29. 5. 1915.
                        }}}\label{K_L03614-1}}}}\pend
           \selectlanguage{ngerman}\endnumbering\briefempfaengerindex{Salten, Felix@\textsc{Salten, Felix}!zzzSchnitzler, Arthur@\emph{von Arthur Schnitzler}!1915-05-291@{{[}29. 5. 1915?{]}}|)be}\mylabel{L03614h}  \normalsize

\doendnotes{C}
\bigskip
\vfill

\clearpage

\footnotesize

\lohead{\textsc{register}}

% Definiere theindex-Environment komplett neu ohne reledmac
\makeatletter
\renewenvironment{theindex}{%
  \section*{\indexname}%
  \setlength{\parindent}{0pt}%
  \setlength{\parskip}{0pt plus 0.3pt}%
  \let\item\@idxitem
}{%
  \clearpage
}
\makeatother

\IfFileExists{\jobname-pw.ind}{\input{\jobname-pw.ind}}{}

\end{document}

      