%% latex-leseansicht-vorspann.tex
%% Vorspann für die Leseansicht.
%% Lädt die gemeinsame Datei latex-vorspann.tex mit nicht gesetztem Schalter.

\newif\ifkorrekturansicht
\korrekturansichtfalse

\input{../tex-inputs/latex-vorspann}


\section[Arthur Schnitzler an Georg Brandes, 30. 1. 1922]{L02376 Arthur Schnitzler an Georg Brandes, 30. 1. 1922}
\nopagebreak\mylabel{L02376v}
\rehead{ }\normalsize\beginnumbering\briefempfaengerindex{Brandes, Georg@\textsc{Brandes, Georg}!zzzSchnitzler, Arthur@\emph{von Arthur Schnitzler}!1922-01-301@{30. 1. 1922}|(be}
\toendnotes[C]{\smallbreak\pagebreak[2]}
\correspDesc{Versand  durch Arthur Schnitzler am 30. 1. 1922 in Wien
\newline{}Erhalt  durch Georg Brandes im Zeitraum [31. 1. 1922
                  – 4. 2. 1922?] in Kopenhagen}\toendnotes[C]{\smallbreak}
\Standort{Kopenhagen, Det Kongelige Bibliotek, Georg Brandes Arkiv, box 125.}
\physDesc{Brief, 3 Blätter, 6 Seiten, 5418 Zeichen
\newline{}Handschrift: schwarze Tinte, lateinische Kurrent
\newline{}Ordnung: mit Bleistift von unbekannter Hand beschriftet:
                                    »Schnitzler« und nummeriert:
                                 »44.«, die Blätter durchgezählt
                                    »1«–»3«, wobei bei den letzten
                                 beiden Blättern auch zusätzlich das Datum ergänzt ist: »30/1 22« }
\buchAbdrucke{\weitereDrucke{1) Georg Brandes, Arthur Schnitzler: \emph{Ein Briefwechsel}. Herausgegeben von Kurt Bergel. Bern: \emph{Francke} 1956, S. 133–135.} \weitereDrucke{2) Arthur Schnitzler: \emph{Briefe 1913–1931}. Herausgegeben von Peter Michael Braunwarth, Richard Miklin, Susanne Pertlik und Heinrich Schnitzler. Frankfurt am Main: \emph{S. Fischer} 1984, S. 263–266.} }\toendnotes[C]{\smallbreak}
\pstart
           \raggedleft{}{\pb}Wien\oindex{Wien@\textbf{Wien}, \emph{Verwaltungsgebiet}|pw}, 30. 1. 1922\pend
           \vspace{0.5em}
\pstart
           Mein lieber und verehrter Freund, es trifft sich gut, daß ich Ihnen
               auf Ihren letzten Brief noch zu antworten habe, so darf ich, ganz nebenbei und
               gewissermaßen unabsichtlich die Gelegenheit benutzen und Ihnen zu Ihrem
               80. Geburtstag Glück wünschen, von dem Sie natürlich nichts hören wollen. Aber we{\geminationn} solche Daten auch nicht viel Sinn haben, – man darf zu
               einem solchen Tag rückhaltloser \strikeout{derart} allerlei
               aussprechen, was sonst vielleicht pathetisch oder sentimental klänge, und so erlauben
               Sie mir nur ganz einfach hier niederzuschreiben, daß unter den Menschen, die älter
               sind als ich – und denen ich nicht eben durch die engsten verwandtschaftlichen Bande
               verknüpft war, kaum Einer ist, dem ich so von Herzen und von Geiste zugethan war und
               bin als Ihnen, Georg Brandes – und der mir – nicht nur durch seine Werke, sondern
               durch sein Sein, sein Dasein, – \uline{mein} Bewußtsein von
               seiner Gegenwart {\pb}in der Welt so viel gegeben hat
               als Sie! Möchten Sie doch allen die Sie lieben und bewundern, noch lange erhalten
               bleiben, – und möchte es das Schicksal fügen, daß wir einander wieder einmal
               persönlich begegnen.\pend
           
\pstart
           Was in jenem »Interview\pwindex{\textcolor{red}{\textsuperscript{XXXX indx1}}!?? [nicht ermitteltes dänisches Interview]@\strich\emph{?? [nicht ermitteltes dänisches Interview]}|pwv}«
               gestanden, weiß ich natürlich nicht; – mir war es bisher ganz unbekannt, daß mich ein
                  daenischer\oindex{Dänemark@\textbf{Dänemark}|pw}{ }Journalist\pwindex{Bangert, Julius 2.\,11.\,1880 Kopenhagen – 11.\,8.\,1957 Gentofte Kommune@\textsc{Bangert, Julius} (2.\,11.\,1880 Kopenhagen – 11.\,8.\,1957 Gentofte Kommune), \emph{Journalist}|pwv} interviewt hat; –
               es waren 2 oder 3 Herren aus Daenemark\oindex{Dänemark@\textbf{Dänemark}|pw} im Lauf
               der letzten Jahre bei mir, und ich habe \introOben{}mich\introOben{} mit ihnen \introOben{}über allerlei\introOben{} unterhalten, – hoffentlich war das, was diesen
               Besuchern in Erinnerung verblieben, nicht so confus wie das \label{K_L02376-1v}\edtext{Zeug\pwindex{Gollomb, Joseph 15.\,11.\,1881 Sankt Petersburg – 23.\,5.\,1950 New York City@\textsc{Gollomb, Joseph} (15.\,11.\,1881 Sankt Petersburg – 23.\,5.\,1950 New York City), \emph{Schriftsteller, Journalist}!Dr. Arthur Schnitzler on the Vienna of To-day@\strich\emph{Dr. Arthur Schnitzler on the Vienna of To-day}|pwv}}{\lemma{\textnormal{\emph{Zeug}}}\Cendnote{\textnormal{Unklar; womöglich meint er Joseph Gollomb\pwindex{Gollomb, Joseph 15.\,11.\,1881 Sankt Petersburg – 23.\,5.\,1950 New York City@\textsc{Gollomb, Joseph} (15.\,11.\,1881 Sankt Petersburg – 23.\,5.\,1950 New York City), \emph{Schriftsteller, Journalist}|pwk}: \emph{Dr. Arthur Schnitzler on the Vienna of To-day}\pwindex{Gollomb, Joseph 15.\,11.\,1881 Sankt Petersburg – 23.\,5.\,1950 New York City@\textsc{Gollomb, Joseph} (15.\,11.\,1881 Sankt Petersburg – 23.\,5.\,1950 New York City), \emph{Schriftsteller, Journalist}!Dr. Arthur Schnitzler on the Vienna of To-day@\strich\emph{Dr. Arthur Schnitzler on the Vienna of To-day}|pwk}. In: \emph{New York Evening Post}\pwindex{New-York Evening Post@\emph{New-York Evening Post}|pwk},
                  5. 6. 1920, Sec. 3, S. [1] und S. 12. Siehe A. S.: \emph{»Das Zeitlose ist von kürzester Dauer«}, Joseph Gollomb: Dr. Arthur Schnitzler on the Vienna of To-day, 5. 6. 1920; vgl. A. S.: \emph{Tagebuch}, 2. 7. 1902.}}}\label{K_L02376-1}, was ich
               gleichfalls als »Interview« mit mir, vor einem Jahr in einer amerikanischen\oindex{Amerika@\textbf{Amerika}|pw} Zeitung zu lesen bekam – Nun Sie haben wohl
               ähnliche Erfahrungen gemacht. Es freut mich schon aus Ihrem Brief zu entnehmen, daß
               ich immerhin über {\pb}Sie, lieber Freund, nichts
               böses geäußert zu haben scheine.\pend
           
\pstart
           Mit dem »Reigen\pwindex{Schnitzler, Arthur 15.\,5.\,1862 Wien – 21.\,10.\,1931 ebd.@\textsc{Schnitzler, Arthur} (15.\,5.\,1862 Wien – 21.\,10.\,1931 ebd.), \emph{Schriftsteller, Mediziner}!Reigen. Zehn Dialoge@\strich\emph{Reigen. Zehn Dialoge}|pw}« hab ich freilich allerlei
               dummes erlebt; – was mir aber kaum nah gegangen ist. Das schli{\geminationm}ste erfährt man ja immer (auch das wird Ihnen nicht neu
               sein) nicht von den Gegnern, – sondern von den Freunden, – die den bessern Theil der
               Tapferkeit, die Vorsicht wählen. Aber es ist schon wahr, – unter den zahlreichen
               Affairen meines Lebens, ist es wohl diese letzte, in de\substVorne{}\textsuperscript{en}\substDazwischen{}r\substHinten{} Verlogenheit, Unverstand und Feigheit sich selbst übertroffen haben. (Dabei
               gesteh ich ohne weiteres zu, daß gegen die \uline{Aufführung}
               des »Reigen\pwindex{Schnitzler, Arthur 15.\,5.\,1862 Wien – 21.\,10.\,1931 ebd.@\textsc{Schnitzler, Arthur} (15.\,5.\,1862 Wien – 21.\,10.\,1931 ebd.), \emph{Schriftsteller, Mediziner}!Reigen. Zehn Dialoge@\strich\emph{Reigen. Zehn Dialoge}|pw}« immerhin auch ehrliche Einwendungen
               möglich sind – aber \substVorne{}\textsuperscript{diese}\substDazwischen{}solche\substHinten{} ehrlichen und discutabeln Einwendungen sind eben in hundert Fällen, wo sie
               auch und besser am Platze gewesen wären, \uline{nicht}
               erhoben worden.) Ich lege hier übrigens einen \label{K_L02376-2v}\edtext{Artikel\pwindex{Schnitzler, Arthur 15.\,5.\,1862 Wien – 21.\,10.\,1931 ebd.@\textsc{Schnitzler, Arthur} (15.\,5.\,1862 Wien – 21.\,10.\,1931 ebd.), \emph{Schriftsteller, Mediziner}!Berichtigung. Ein paar Worte zum Gutachten Maximilian Hardens über den »Reigen«@\strich\emph{Berichtigung. Ein paar Worte zum Gutachten Maximilian Hardens über den »Reigen«}|pwv}}{\lemma{\textnormal{\emph{Artikel}}}\Cendnote{\textnormal{Siehe A. S.: \emph{»Das Zeitlose ist von kürzester Dauer«}, Berichtigung. Ein paar Worte zum Gutachten Maximilian Hardens über den »Reigen«, 30. 1. 1921.
               }}}\label{K_L02376-2} bei – das einzige Document, in dem ich \strikeout{selbst}
               mich \introOben{}persönlich\introOben{} zu Worte habe kommen lassen; – er erklärt
               sich selbst.\pend
           
\pstart
           {\pb}Meine beiden Casanova\pwindex{Casanova, Giacomo Girolamo 2.\,4.\,1725 Venedig – 4.\,6.\,1798 Duchcov@\textsc{Casanova, Giacomo Girolamo} (2.\,4.\,1725 Venedig – 4.\,6.\,1798 Duchcov), \emph{Schriftsteller, Abenteurer}|pw}-Sachen, das Lustspiel »die
                  Schwestern\pwindex{Schnitzler, Arthur 15.\,5.\,1862 Wien – 21.\,10.\,1931 ebd.@\textsc{Schnitzler, Arthur} (15.\,5.\,1862 Wien – 21.\,10.\,1931 ebd.), \emph{Schriftsteller, Mediziner}!Schwestern oder Casanova in Spa. Lustspiel in Versen@\strich\emph{Die Schwestern oder Casanova in Spa. Lustspiel in Versen}|pw}« und die Novelle »Casan.
                  Heimfahrt\pwindex{Schnitzler, Arthur 15.\,5.\,1862 Wien – 21.\,10.\,1931 ebd.@\textsc{Schnitzler, Arthur} (15.\,5.\,1862 Wien – 21.\,10.\,1931 ebd.), \emph{Schriftsteller, Mediziner}!Casanovas Heimfahrt@\strich\emph{Casanovas Heimfahrt}|pw}« sind so entstanden, daß mir zwei Stoffe, die schon geraume Zeit
               unter meinen Papieren lagen, durch die Lectüre der Casanova Memoiren\pwindex{Casanova, Giacomo Girolamo 2.\,4.\,1725 Venedig – 4.\,6.\,1798 Duchcov@\textsc{Casanova, Giacomo Girolamo} (2.\,4.\,1725 Venedig – 4.\,6.\,1798 Duchcov), \emph{Schriftsteller, Abenteurer}!Aus meinem Leben@\strich\emph{Aus meinem Leben}|pwv} plötzlich lebendig geworden sind. Die
               Beschäftigung damit bedeutete keine bewußte Abkehr von der Zeit. Zu den Ereignissen
               selbst hätt ich natürlich geschwiegen – gelegentlich mußte man sich nur melden, um
               gegen eine Verläumdung oder gar gegen \label{K_L02376-3v}\edtext{Mißbrauch seiner Unterschrift zu protestiren}{\lemma{\textnormal{\emph{Mißbrauch … protestiren}}}\Cendnote{\textnormal{Schnitzler spielt auf seine angebliche
                  Unterschrift auf einer Protestnote\pwindex{Eine Kundgebung für Toller@\emph{Eine Kundgebung für Toller}|pwkv} gegen die Hinrichtung Ernst
                     Tollers\pwindex{Toller, Ernst 1.\,12.\,1893 Szamocin – 22.\,5.\,1939 New York City@\textsc{Toller, Ernst} (1.\,12.\,1893 Szamocin – 22.\,5.\,1939 New York City), \emph{Schriftsteller, Aktivist}|pwk} an, die am 11. 6. 1919 durch die Presse ging. Schnitzler hatte diese nicht unterschrieben
                  und verfasste in der Folge einen Leserbrief, in dem er sich gegen die ungefragte
                  Verwendung seines Namens verwehrte. Siehe A. S.: \emph{»Das Zeitlose ist von kürzester Dauer«}, Der Protest aus Wien gegen Tollers Hinrichtung, 13. 6. 1919.}}}\label{K_L02376-3} – Überraschungen hab ich eigentlich
               nicht erlebt, – die existiren für Unser Einen doch wohl nur in quantitativer
               Hinsicht.\pend
           
\pstart
           Die Zustände in Wien\oindex{Wien@\textbf{Wien}, \emph{Verwaltungsgebiet}|pw} sind übel genug, – die
               Preissteigerungen phantastisch 1000–2000 fach; – dabei ungeheuer viel Luxus; – und
               mehr stilles Elend als sichtbares. Die denen es am schlechtesten geht, halten weder
               Umzüge noch plündern sie. Wie es weiter gehen soll, weiß niemand. Wirkliche {\pb}Hilfe ka{\geminationn} natürlich
               von außen – auch durch die berühmten Credite, nie und ni{\geminationm}er kommen; – es müßten die außerordentlichen inneren \introOben{}national-\introOben{}oekonomischen Möglichkeiten unseres Landes mit Energie und ohne
               jede Rücksicht auf \introOben{}partei\introOben{}politische \strikeout{und} Interessen ausgenutzt werden; – aber vielleicht ist
               es heute schon zu spät dazu. An ein Zugrundegehen von Wien\oindex{Wien@\textbf{Wien}, \emph{Verwaltungsgebiet}|pw} glaub ich nicht (etwa im Sinne von Venedig\oindex{Venedig@\textbf{Venedig}|pw} –), aber als was es sich erheben und wieder emporblühen soll – und
               wann, das vermag ich nicht vorauszusehen. –\pend
           
\pstart
           In meinen äußeren Verhältnissen – \uline{da wo sie schon die
                  innern sind} hat sich manches verändert. Von meiner Frau\pwindex{Schnitzler, Olga 17.\,1.\,1882 Wien – 13.\,1.\,1970 Lugano@\textsc{Schnitzler, Olga} (17.\,1.\,1882 Wien – 13.\,1.\,1970 Lugano), \emph{Schauspielerin, Sängerin}|pwv} bin ich geschieden, – aber wir sind
               gute Freunde geblieben, – ja in der letzten Zeit wieder geworden, könnte man besser
               sagen. Sie lebt vorläufig in Salzburg\oindex{Salzburg@\textbf{Salzburg}, \emph{Verwaltungsgebiet}|pw}, war aber in
               den letzten Tagen in Wien\oindex{Wien@\textbf{Wien}, \emph{Verwaltungsgebiet}|pw}, und Sie können kaum
               glauben, wie viel wir gerade von Ihnen gesprochen haben. Mein Sohn\pwindex{Schnitzler, Heinrich 9.\,8.\,1902 Hinterbrühl – 12.\,7.\,1982 Wien@\textsc{Schnitzler, Heinrich} (9.\,8.\,1902 Hinterbrühl – 12.\,7.\,1982 Wien), \emph{Regisseur, Schauspieler}|pwv}, der {\pb}heuer zwanzig wird, zeigt sich in \uline{theatralibus} theoretisch und praktisch recht begabt, – auch
               musikalisch leistet er etwas. Dabei fehlt aber jede \uline{falsche} Tendenz ins selbstschöpferische, – d. h. er dilettirt weder als
               Dichter noch als Componist. Ich glaube er ist der geborene Regisseur – und andre
               glauben es auch. Seine Hauptbeschäftigung ist jetzt Shakespeare\pwindex{Shakespeare, William 23.\,4.\,1564? Stratford-upon-Avon – 3.\,5.\,1616 ebd.@\textsc{Shakespeare, William} (23.\,4.\,1564? Stratford-upon-Avon – 3.\,5.\,1616 ebd.), \emph{Schauspieler, Dramatiker}|pw}; eben hat er eine Inszenierung von Maß für Maß\pwindex{Shakespeare, William 23.\,4.\,1564? Stratford-upon-Avon – 3.\,5.\,1616 ebd.@\textsc{Shakespeare, William} (23.\,4.\,1564? Stratford-upon-Avon – 3.\,5.\,1616 ebd.), \emph{Schauspieler, Dramatiker}!Maß für Maß. Schauspiel in fünf Akten@\strich\emph{Maß für Maß. Schauspiel in fünf Akten}|pw} gemacht – er arbeitet in der Hofbibliothek\oindex{Wien@\textbf{Wien}!I., Innere Stadt@\textbf{I., Innere Stadt}!Österreichische Nationalbibliothek@\textbf{Österreichische Nationalbibliothek}, \emph{Bibliothek}|pw} – jetzt Nationalbibliothek\oindex{Wien@\textbf{Wien}!I., Innere Stadt@\textbf{I., Innere Stadt}!Österreichische Nationalbibliothek@\textbf{Österreichische Nationalbibliothek}, \emph{Bibliothek}|pw}, – und hat auch an der Wanderbühne\orgindex{Wanderbühne des österreichischen Volksbildungsamtes@Wanderbühne des österreichischen Volksbildungsamtes|pw} schon kleinere Rollen gespielt. – Meine Tochter Lili\pwindex{Cappellini, Lili 13.\,9.\,1909 Wien – 26.\,7.\,1928 Venedig@\textsc{Cappellini, Lili} (13.\,9.\,1909 Wien – 26.\,7.\,1928 Venedig)|pw}, zwölf vorbei geht ins Gymnasium; –
               declamirt die Jungfrau von Orleans\pwindex{\textcolor{red}{\textsuperscript{XXXX indx1}}!Jungfrau von Orleans@\strich\emph{Die Jungfrau von Orleans}|pw}, schreibt
               »Geschichten«, – und verwickelt mich jeden Morgen in die schwierigsten Gespräche über
               Gott und \introOben{}den\introOben{} freien Willen. Aber Landschaft, Schwimmen und
               Milchchocolade ist ihr glücklicherweise doch noch wichtiger.\pend
           
\pstart
           Und von mir selber we{\geminationn} Sie erlauben schreib ich Ihnen
               nächstens. Freundschaftlich treu\pend
           \pstart Der Ihrige wie immer \spacefill\mbox{Arthur Schnitzler}\pend{}\selectlanguage{ngerman}\endnumbering\briefempfaengerindex{Brandes, Georg@\textsc{Brandes, Georg}!zzzSchnitzler, Arthur@\emph{von Arthur Schnitzler}!1922-01-301@{30. 1. 1922}|)be}\mylabel{L02376h}  \newcommand{\dateiname}{L02376}\newcommand{\titel}{Arthur Schnitzler an Georg Brandes, 30. 1. 1922}\newcommand{\editorInnen}{Martin Anton Müller und Gerd-Hermann Susen}%% latex-leseansicht-abspann.tex
%% Abspann für die Leseansicht.
%% Der Schalter \ifkorrekturansicht ist bereits durch den Vorspann gesetzt.

%% latex-abspann.tex
%% Gemeinsamer Abspann für Korrekturansicht und Leseansicht.
%% Setzt den Schalter \ifkorrekturansicht voraus (gesetzt in den
%% einbindenden Dateien latex-korrekturansicht-abspann.tex bzw.
%% latex-leseansicht-abspann.tex).
%% ---------------------------------------------------------------

\normalsize

% Das esempio-Environment wird nur in der Leseansicht benötigt
\ifkorrekturansicht\else
\newenvironment{esempio}[3]%
{
    \vspace{1.5ex}
    \rlap{\underline{#1}}
    \par
    \setlength{\parindent}{0cm}
    \nopagebreak
    \leftskip=#2cm
    \rightskip=#3cm
}
{
    \par
}
\fi

\doendnotes{C}
\bigskip
\vfill

\clearpage

\footnotesize

\ifkorrekturansicht
  \lohead{\textsc{register}}
\fi

% theindex-Environment neu definieren ohne reledmac
\makeatletter
\renewenvironment{theindex}{%
  \ifkorrekturansicht
    \section*{\indexname}%
  \else
    \subsubsection*{Index der erwähnten Entitäten}%
  \fi
  \setlength{\parindent}{0pt}%
  \setlength{\parskip}{0pt plus 0.3pt}%
  \let\item\@idxitem
}{%
  \ifkorrekturansicht\clearpage\fi
}
\makeatother

\IfFileExists{\jobname-pw.ind}{\input{\jobname-pw.ind}}{}

% Quellenangabe nur in der Leseansicht
\ifkorrekturansicht\else
% Fallback-Definitionen, falls die .tex-Datei \titel etc. nicht gesetzt hat
\providecommand{\titel}{}
\providecommand{\editorInnen}{}
\providecommand{\dateiname}{\jobname}

\vspace{3cm}

\vfill

\footnotesize
\textsc{Quelle}: \titel. Herausgegeben von {\editorInnen}. In: \emph{Arthur Schnitzler: Briefwechsel mit Autorinnen und Autoren}.
 Digitale Edition, https://schnitzler-briefe.acdh.oeaw.ac.at/{\dateiname}.html (Stand \today)
\fi

\end{document}


