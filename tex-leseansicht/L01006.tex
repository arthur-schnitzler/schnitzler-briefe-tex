%% latex-leseansicht-vorspann.tex
%% Vorspann für die Leseansicht.
%% Lädt die gemeinsame Datei latex-vorspann.tex mit nicht gesetztem Schalter.

\newif\ifkorrekturansicht
\korrekturansichtfalse

\input{../tex-inputs/latex-vorspann}


         
         \renewcommand{\erwaehntePersonen}{Personen: Richard Beer-Hofmann, Franz Grillparzer, Gisela Hajek, Paul Schlenther}
         \renewcommand{\erwaehnteInstitutionen}{Institutionen: Burgtheater}
         \renewcommand{\erwaehnteOrte}{Orte: Burgtheater, I., Innere Stadt, Wien, Wollzeile}
         \renewcommand{\erwaehnteWerke}{Werke: Die Gefährtin. Schauspiel in einem Akt, Esther, Paracelsus. Versspiel in einem Akt}
               \section[Arthur Schnitzler an Richard Beer-Hofmann, 19. 12. 1899]{ Arthur Schnitzler an Richard Beer-Hofmann, 19. 12. 1899}\nopagebreak\mylabel{v}\rehead{ }\begin{ledgroupsized}[t]{13cm}\normalsize\beginnumbering \toendnotes[C]{\smallbreak\pagebreak[2]} \Standort{YCGL, MSS 31.}
\physDesc{Kartenbrief
\newline{}Handschrift: Bleistift, deutsche Kurrent\newline{}Versand: 1) Rohrpost  2) Stempel: »\nobreak{}\oindex{I., Innere Stadt@\textbf{I., Innere Stadt}|pwk}Wien 1/1, 19 XII 99, 4 10N\nobreak{}«.  3) Stempel: »\nobreak{}\oindex{I., Innere Stadt@\textbf{I., Innere Stadt}|pwk}Wien 1/1, 19 XII 99, 4 10N\nobreak{}«. }\toendnotes[C]{\smallbreak}\pstart{}{\pb}Herrn \textsc{Dr. Rich.
                     Beer-Hofmann}\pend{}\pstart{}Wien\oindex{Wien@\textbf{Wien}|pw}\pend{}\pstart{}\textsc{I. Wollzeile 15\oindex{Wollzeile@\textbf{Wollzeile}|pw}}.\pend{}{\bigskip}\pstart
           \noindent{}{\pb}lieber Richard, bitte kommen Sie in die \label{K_L01006_1v}\edtext{Loge 6}{\lemma{\textnormal{\emph{Loge 6}}}\Cendnote{\textnormal{Das \emph{Burgtheater}\orgindex{Burgtheater@Burgtheater|pwk}
                        veranstaltete eine gemeinsame Aufführung von Schnitzler\pwindex{Schnitzler, Arthur 15.05.1862 – 21.10.1931@\textsc{Schnitzler, Arthur} (15.05.1862 – 21.10.1931), \emph{Schriftsteller, Mediziner}|pwk}s \emph{Paracelus}\pwindex{Schnitzler, Arthur 15.05.1862 – 21.10.1931@\textsc{Schnitzler, Arthur} (15.05.1862 – 21.10.1931), \emph{Schriftsteller, Mediziner}!Paracelsus. Versspiel in einem Akt01. 11. 1898@\strich\emph{Paracelsus. Versspiel in einem Akt} {[}01. 11. 1898{]}|pwk} und \emph{Die Gefährtin}\pwindex{Schnitzler, Arthur 15.05.1862 – 21.10.1931@\textsc{Schnitzler, Arthur} (15.05.1862 – 21.10.1931), \emph{Schriftsteller, Mediziner}!Gefaehrtin. Schauspiel in einem Akt1899-03-01@\strich\emph{Die Gefährtin. Schauspiel in einem Akt} {[}1899-03-01{]}|pwk}
                        mit dem Dramenfragment \emph{Esther}\pwindex{Grillparzer, Franz 15.01.1791 – 21.01.1872@\textsc{Grillparzer, Franz} (15.01.1791 – 21.01.1872), \emph{Schriftsteller, Beamter}!Esther1868@\strich\emph{Esther} {[}1868{]}|pwk} von Franz Grillparzer\pwindex{Grillparzer, Franz 15.01.1791 – 21.01.1872@\textsc{Grillparzer, Franz} (15.01.1791 – 21.01.1872), \emph{Schriftsteller, Beamter}|pwk}.}}}\label{K_L01006_1h}\pwindex{Grillparzer, Franz 15.01.1791 – 21.01.1872@\textsc{Grillparzer, Franz} (15.01.1791 – 21.01.1872), \emph{Schriftsteller, Beamter}!Esther1868@\strich\emph{Esther} {[}1868{]}|pw}\pwindex{Schnitzler, Arthur 15.05.1862 – 21.10.1931@\textsc{Schnitzler, Arthur} (15.05.1862 – 21.10.1931), \emph{Schriftsteller, Mediziner}!Paracelsus. Versspiel in einem Akt01. 11. 1898@\strich\emph{Paracelsus. Versspiel in einem Akt} {[}01. 11. 1898{]}|pw}\pwindex{Schnitzler, Arthur 15.05.1862 – 21.10.1931@\textsc{Schnitzler, Arthur} (15.05.1862 – 21.10.1931), \emph{Schriftsteller, Mediziner}!Gefaehrtin. Schauspiel in einem Akt1899-03-01@\strich\emph{Die Gefährtin. Schauspiel in einem Akt} {[}1899-03-01{]}|pw}\oindex{Burgtheater@\textbf{Burgtheater}|pwv}, rechts, 1. Gallerie!\pend
           \pstart
           Ich ſelbſt bin bei Schlenther\pwindex{Schlenther, Paul 20.08.1854 – 30.04.1916@\textsc{Schlenther, Paul} (20.08.1854 – 30.04.1916), \emph{Schriftsteller, Kritiker, Theaterleiter}|pw}. Nachtmahlen
                  ka{\geminationn} ich nicht mit Ihnen; ſchweſterlicher\pwindex{Hajek, Gisela 20.12.1867 – 03.02.1953@\textsc{Hajek, Gisela} (20.12.1867 – 03.02.1953)|pwv} Geburtstag.\pend
           \pstart
           Herzlichſt Ihr{\\[\baselineskip]}\spacefill\mbox{Arthur}\pend
           \leftskip=0em{}
         
         \endnumbering\mylabel{h}\end{ledgroupsized}  \newcommand{\dateiname}{L01006}\newcommand{\titel}{Arthur Schnitzler an Richard Beer-Hofmann, 19. 12. 1899}\newcommand{\editorInnen}{Martin Anton Müller und Gerd-Hermann Susen}%% latex-leseansicht-abspann.tex
%% Abspann für die Leseansicht.
%% Der Schalter \ifkorrekturansicht ist bereits durch den Vorspann gesetzt.

%% latex-abspann.tex
%% Gemeinsamer Abspann für Korrekturansicht und Leseansicht.
%% Setzt den Schalter \ifkorrekturansicht voraus (gesetzt in den
%% einbindenden Dateien latex-korrekturansicht-abspann.tex bzw.
%% latex-leseansicht-abspann.tex).
%% ---------------------------------------------------------------

\normalsize

% Das esempio-Environment wird nur in der Leseansicht benötigt
\ifkorrekturansicht\else
\newenvironment{esempio}[3]%
{
    \vspace{1.5ex}
    \rlap{\underline{#1}}
    \par
    \setlength{\parindent}{0cm}
    \nopagebreak
    \leftskip=#2cm
    \rightskip=#3cm
}
{
    \par
}
\fi

\doendnotes{C}
\bigskip
\vfill

\clearpage

\footnotesize

\ifkorrekturansicht
  \lohead{\textsc{register}}
\fi

% theindex-Environment neu definieren ohne reledmac
\makeatletter
\renewenvironment{theindex}{%
  \ifkorrekturansicht
    \section*{\indexname}%
  \else
    \subsubsection*{Index der erwähnten Entitäten}%
  \fi
  \setlength{\parindent}{0pt}%
  \setlength{\parskip}{0pt plus 0.3pt}%
  \let\item\@idxitem
}{%
  \ifkorrekturansicht\clearpage\fi
}
\makeatother

\IfFileExists{\jobname-pw.ind}{\input{\jobname-pw.ind}}{}

% Quellenangabe nur in der Leseansicht
\ifkorrekturansicht\else
% Fallback-Definitionen, falls die .tex-Datei \titel etc. nicht gesetzt hat
\providecommand{\titel}{}
\providecommand{\editorInnen}{}
\providecommand{\dateiname}{\jobname}

\vspace{3cm}

\vfill

\footnotesize
\textsc{Quelle}: \titel. Herausgegeben von {\editorInnen}. In: \emph{Arthur Schnitzler: Briefwechsel mit Autorinnen und Autoren}.
 Digitale Edition, https://schnitzler-briefe.acdh.oeaw.ac.at/{\dateiname}.html (Stand \today)
\fi

\end{document}


      