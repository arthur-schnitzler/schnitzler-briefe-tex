%% latex-leseansicht-vorspann.tex
%% Vorspann für die Leseansicht.
%% Lädt die gemeinsame Datei latex-vorspann.tex mit nicht gesetztem Schalter.

\newif\ifkorrekturansicht
\korrekturansichtfalse

\input{../tex-inputs/latex-vorspann}


         
         \newcommand{\erwaehntePersonen}{Personen: Hermann Bahr, Francesco Petrarca}
         \newcommand{\erwaehnteOrte}{Orte: Casa Petrarca, Helsingør, Italien, Marienlyst, Venedig}
         \newcommand{\erwaehnteWerke}{Werke: Hamlet}
               \section[Arthur und Olga Schnitzler an Hermann Bahr, 14. 7. 1906]{ Arthur und Olga Schnitzler an Hermann Bahr, 14. 7. 1906}\nopagebreak\mylabel{v}\rehead{ }\begin{ledgroupsized}[t]{13cm}\normalsize\beginnumbering \toendnotes[C]{\smallbreak\pagebreak[2]} \Standort{TMW, HS AM 60174 Ba.}
\physDesc{Bildpostkarte
\newline{}Handschrift Arthur Schnitzler: 1) Bleistift, deutsche Kurrent\hspace{1em}2) Bleistift, lateinische Kurrent (\noindent{}Adresse)\hspace{1em}\newline{}Handschrift Olga Schnitzler: Bleistift, lateinische Kurrent\newline{}Versand: 1) Stempel: »\nobreak{}1{[}4{]}. 7. 06, 2–5 E\nobreak{}«.   2) Stempel: »\nobreak{}\oindex{Venedig@\textbf{Venedig}|pwk}Venezia, 16. 7{[}. 06{]}\nobreak{}«. \newline{}Ordnung: Lochung }\buchAbdrucke{\weitereDrucke{1) \emph{14. 7. 1906, Abschrift.} In: Arthur Schnitzler: \emph{The Letters of Arthur Schnitzler to Hermann Bahr}. Edited, annotated, and with an introduction, by Donald G.
                        Daviau. Chapel Hill: \emph{The University of North Carolina Press} 1978, S. 95 (University of North Carolina studies in the Germanic languages
                        and literatures, 89).} \weitereDrucke{2) Hermann Bahr, Arthur Schnitzler: \emph{Briefwechsel, Aufzeichnungen, Dokumente (1891–1931)}. Hg. Kurt Ifkovits und Martin Anton Müller. Göttingen: \emph{Wallstein} 2018, S. 380.} }\toendnotes[C]{\smallbreak}\pstart{}{\pb}Herrn\pend{}\pstart{}Hermann Bahr\pend{}\pstart{}Venezia\oindex{Italien@\textbf{Italien}|pw}\pend{}\pstart{}Casa Petrarca\pwindex{Petrarca, Francesco 1304-07-19 – 1374@\textsc{Petrarca, Francesco} (1304-07-19 – 1374), \emph{Schriftsteller}|pwv}\oindex{Casa Petrarca@\textbf{Casa Petrarca}|pw}\pend{}\pstart{}Italie\oindex{Venedig@\textbf{Venedig}|pw}\pend{}{\bigskip}\pstart
           \noindent{}\centering{}\textcolor{gray}{\textbf{{\pb}Hilsen fra Marienlyst Parken\oindex{Marienlyst@\textbf{Marienlyst}|pw}}}\pend
           \pstart
           \textsc{{\pb}Marienlyst\oindex{Marienlyst@\textbf{Marienlyst}|pw}}, 14. 7. 906\pend
           \pstart
           hier, lieber Hermann, wohnen wir ſeit 14 Tagen. Es ist wunderſchön,
               und we{\geminationn} du herkämſt, kö{\geminationn}teſt du ein angenehmes Leben, ohne Strand zwar, aber auch ohne Brandwunden führen.
               Wir bleiben bis auf weiteres.\pend
           \pstart
           Herzlichſt{\\[\baselineskip]}dein \spacefill\mbox{A.}\pend
           \leftskip=0em{}\pstart
           \noindent{}{\pb}{[}hs. Olga Schnitzler:{]} (\label{T_L01612_1v}\edtext{Dies}{\lemma{\textnormal{\emph{Dies}}}\Cendnote{\textnormal{handschriftlicher Pfeil auf eine
                  Statue}}}\label{T_L01612_1h} ſoll \textsc{Hamlet\pwindex{\textcolor{red}{\textsuperscript{XXXX1 indx}}!Hamlet1600@\strich\emph{Hamlet} {[}1600{]}|pwv} sein.})\pend
           \pstart Herzliche Erwiderung Ihrer lieben Grüsse! { }\spacefill\mbox{Olga Schnitzler.}\pend{}
         
         \endnumbering\mylabel{h}\end{ledgroupsized}  \newcommand{\dateiname}{L01612}\newcommand{\titel}{Arthur und Olga Schnitzler an Hermann Bahr, 14. 7. 1906}\newcommand{\editorInnen}{ Kurt Ifkovits,  Martin Anton Müller}%% latex-leseansicht-abspann.tex
%% Abspann für die Leseansicht.
%% Der Schalter \ifkorrekturansicht ist bereits durch den Vorspann gesetzt.

%% latex-abspann.tex
%% Gemeinsamer Abspann für Korrekturansicht und Leseansicht.
%% Setzt den Schalter \ifkorrekturansicht voraus (gesetzt in den
%% einbindenden Dateien latex-korrekturansicht-abspann.tex bzw.
%% latex-leseansicht-abspann.tex).
%% ---------------------------------------------------------------

\normalsize

% Das esempio-Environment wird nur in der Leseansicht benötigt
\ifkorrekturansicht\else
\newenvironment{esempio}[3]%
{
    \vspace{1.5ex}
    \rlap{\underline{#1}}
    \par
    \setlength{\parindent}{0cm}
    \nopagebreak
    \leftskip=#2cm
    \rightskip=#3cm
}
{
    \par
}
\fi

\doendnotes{C}
\bigskip
\vfill

\clearpage

\footnotesize

\ifkorrekturansicht
  \lohead{\textsc{register}}
\fi

% theindex-Environment neu definieren ohne reledmac
\makeatletter
\renewenvironment{theindex}{%
  \ifkorrekturansicht
    \section*{\indexname}%
  \else
    \subsubsection*{Index der erwähnten Entitäten}%
  \fi
  \setlength{\parindent}{0pt}%
  \setlength{\parskip}{0pt plus 0.3pt}%
  \let\item\@idxitem
}{%
  \ifkorrekturansicht\clearpage\fi
}
\makeatother

\IfFileExists{\jobname-pw.ind}{\input{\jobname-pw.ind}}{}

% Quellenangabe nur in der Leseansicht
\ifkorrekturansicht\else
% Fallback-Definitionen, falls die .tex-Datei \titel etc. nicht gesetzt hat
\providecommand{\titel}{}
\providecommand{\editorInnen}{}
\providecommand{\dateiname}{\jobname}

\vspace{3cm}

\vfill

\footnotesize
\textsc{Quelle}: \titel. Herausgegeben von {\editorInnen}. In: \emph{Arthur Schnitzler: Briefwechsel mit Autorinnen und Autoren}.
 Digitale Edition, https://schnitzler-briefe.acdh.oeaw.ac.at/{\dateiname}.html (Stand \today)
\fi

\end{document}


      