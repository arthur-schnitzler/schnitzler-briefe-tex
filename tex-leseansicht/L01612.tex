%% latex-leseansicht-vorspann.tex
%% Vorspann für die Leseansicht.
%% Lädt die gemeinsame Datei latex-vorspann.tex mit nicht gesetztem Schalter.

\newif\ifkorrekturansicht
\korrekturansichtfalse

\input{../tex-inputs/latex-vorspann}


\section[Arthur und Olga Schnitzler an Hermann Bahr, 14. 7. 1906]{L01612 Arthur und Olga Schnitzler an Hermann Bahr, 14. 7. 1906}
\nopagebreak\mylabel{L01612v}
\rehead{ }\normalsize\beginnumbering\briefempfaengerindex{Bahr, Hermann@\textsc{Bahr, Hermann}!zzzSchnitzler, Olga@\emph{von Olga Schnitzler}!1906-07-141@{14. 7. 1906}|(be}\briefempfaengerindex{Bahr, Hermann@\textsc{Bahr, Hermann}!zzzSchnitzler, Arthur@\emph{von Arthur Schnitzler}!1906-07-141@{14. 7. 1906}|(be}
\toendnotes[C]{\smallbreak\pagebreak[2]}
\correspDesc{Versand  durch Arthur Schnitzler, Olga Schnitzler am 14. 7. 1906 in Helsingør
\newline{}Erhalt  durch Hermann Bahr am 16. 7. 1906 in Venedig}\toendnotes[C]{\smallbreak}
\Standort{TMW, HS AM 60174 Ba.}
\physDesc{Bildpostkarte, 359 Zeichen
\newline{}Handschrift Arthur Schnitzler: Bleistift, deutsche Kurrent
\newline{}Handschrift Olga Schnitzler: Bleistift, lateinische Kurrent
\newline{}Versand: 1) Stempel: »\nobreak{}1{[}4{]}. 7. 06, 2–5 E\nobreak{}«.   2) Stempel: »\nobreak{}\oindex{Venedig@\textbf{Venedig}|pwk}Venezia, 16. 7{[}. 06{]}\nobreak{}«. 
\newline{}Ordnung: Lochung }
\buchAbdrucke{\weitereDrucke{1) \emph{14. 7. 1906, Abschrift.} In: Arthur Schnitzler: \emph{The Letters of Arthur Schnitzler to Hermann Bahr}. Edited, annotated, and with an introduction, by Donald G. Daviau. Chapel Hill: \emph{The University of North Carolina Press} 1978, S. 95 (University of North Carolina studies in the Germanic languages
                        and literatures, 89).} \weitereDrucke{2) Hermann Bahr, Arthur Schnitzler: \emph{Briefwechsel, Aufzeichnungen, Dokumente (1891–1931)}. Herausgegeben von Kurt Ifkovits und Martin Anton Müller. Göttingen: \emph{Wallstein} 2018, S. 380.} }\toendnotes[C]{\smallbreak}\pstart{}\textsc{{\pb}Herrn}\pend{}\pstart{}\textsc{Hermann Bahr}\pend{}\pstart{}\textsc{Venezia\oindex{Italien@\textbf{Italien}|pw}}\pend{}\pstart{}\textsc{Casa Petrarca\pwindex{Petrarca, Francesco 19.\,7.\,1304 Arezzo – 1374 Arquà Petrarca@\textsc{Petrarca, Francesco} (19.\,7.\,1304 Arezzo – 1374 Arquà Petrarca), \emph{Schriftsteller}|pwv}\oindex{Casa Petrarca@\textbf{Casa Petrarca}, \emph{Gebäude}|pw}}\pend{}\pstart{}\textsc{Italie\oindex{Venedig@\textbf{Venedig}|pw}}\pend{}{\bigskip}
\pstart
           \noindent{}\centering{}{\pb}\textcolor{gray}{\textbf{\label{K_L01612-1v}\edtext{Hilsen fra}{\lemma{\textnormal{\emph{Hilsen fra}}}\Cendnote{\textnormal{dänisch: Grüße aus}}}\label{K_L01612-1}{ }Marienlyst Parken\oindex{Marienlyst@\textbf{Marienlyst}, \emph{Gut}|pw}}}\pend
           \vspace{1em}
\pstart
           \textsc{{\pb}Marienlyst\oindex{Marienlyst@\textbf{Marienlyst}, \emph{Gut}|pw}}, 14. 7. 906\pend
           \vspace{0.5em}
\pstart
           hier, lieber Hermann, wohnen wir{ }ſeit 14 Tagen. Es ist wunderſchön,
               und we{\geminationn} du herkämſt, kö{\geminationn}teſt du ein angenehmes Leben, ohne Strand zwar, aber auch ohne Brandwunden führen.
               Wir bleiben bis auf weiteres.\pend
           
\pstart
           Herzlichſt{\\[\baselineskip]}dein \spacefill\mbox{A.}\pend
           \leftskip=0em{}\selectlanguage{ngerman}\vspace{1em}
\pstart
           \noindent{}{\pb}{[}hs. Schnitzler:{]} (\label{T_L01612-1v}\edtext{Dies}{\lemma{\textnormal{\emph{Dies}}}\Cendnote{\textnormal{handschriftlicher Pfeil auf eine
                  Statue}}}\label{T_L01612-1}{ }ſoll \textsc{Hamlet\pwindex{\textcolor{red}{\textsuperscript{XXXX indx1}}!Hamlet@\strich\emph{Hamlet}|pwv} sein.})\pend
           \pstart Herzliche Erwiderung Ihrer lieben Grüsse! { }\spacefill\mbox{Olga Schnitzler.}\pend{}\selectlanguage{ngerman}\endnumbering\briefempfaengerindex{Bahr, Hermann@\textsc{Bahr, Hermann}!zzzSchnitzler, Olga@\emph{von Olga Schnitzler}!1906-07-141@{14. 7. 1906}|)be}\briefempfaengerindex{Bahr, Hermann@\textsc{Bahr, Hermann}!zzzSchnitzler, Arthur@\emph{von Arthur Schnitzler}!1906-07-141@{14. 7. 1906}|)be}\mylabel{L01612h}  \newcommand{\dateiname}{L01612}\newcommand{\titel}{Arthur und Olga Schnitzler an Hermann Bahr, 14. 7. 1906}\newcommand{\editorInnen}{Herausgegeben von Martin Anton Müller}%% latex-leseansicht-abspann.tex
%% Abspann für die Leseansicht.
%% Der Schalter \ifkorrekturansicht ist bereits durch den Vorspann gesetzt.

%% latex-abspann.tex
%% Gemeinsamer Abspann für Korrekturansicht und Leseansicht.
%% Setzt den Schalter \ifkorrekturansicht voraus (gesetzt in den
%% einbindenden Dateien latex-korrekturansicht-abspann.tex bzw.
%% latex-leseansicht-abspann.tex).
%% ---------------------------------------------------------------

\normalsize

% Das esempio-Environment wird nur in der Leseansicht benötigt
\ifkorrekturansicht\else
\newenvironment{esempio}[3]%
{
    \vspace{1.5ex}
    \rlap{\underline{#1}}
    \par
    \setlength{\parindent}{0cm}
    \nopagebreak
    \leftskip=#2cm
    \rightskip=#3cm
}
{
    \par
}
\fi

\doendnotes{C}
\bigskip
\vfill

\clearpage

\footnotesize

\ifkorrekturansicht
  \lohead{\textsc{register}}
\fi

% theindex-Environment neu definieren ohne reledmac
\makeatletter
\renewenvironment{theindex}{%
  \ifkorrekturansicht
    \section*{\indexname}%
  \else
    \subsubsection*{Index der erwähnten Entitäten}%
  \fi
  \setlength{\parindent}{0pt}%
  \setlength{\parskip}{0pt plus 0.3pt}%
  \let\item\@idxitem
}{%
  \ifkorrekturansicht\clearpage\fi
}
\makeatother

\IfFileExists{\jobname-pw.ind}{\input{\jobname-pw.ind}}{}

% Quellenangabe nur in der Leseansicht
\ifkorrekturansicht\else
% Fallback-Definitionen, falls die .tex-Datei \titel etc. nicht gesetzt hat
\providecommand{\titel}{}
\providecommand{\editorInnen}{}
\providecommand{\dateiname}{\jobname}

\vspace{3cm}

\vfill

\footnotesize
\textsc{Quelle}: \titel. Herausgegeben von {\editorInnen}. In: \emph{Arthur Schnitzler: Briefwechsel mit Autorinnen und Autoren}.
 Digitale Edition, https://schnitzler-briefe.acdh.oeaw.ac.at/{\dateiname}.html (Stand \today)
\fi

\end{document}


