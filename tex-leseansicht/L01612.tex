%% latex-korrekturansicht-vorspann.tex
%% Vorspann für die Korrekturansicht.
%% Lädt die gemeinsame Datei latex-vorspann.tex mit gesetztem Schalter.

\newif\ifkorrekturansicht
\korrekturansichttrue

\input{../tex-inputs/latex-vorspann}


\section[Arthur und Olga Schnitzler an Hermann Bahr, 14. 7. 1906]{L01612 Arthur und Olga Schnitzler an Hermann Bahr, 14. 7. 1906}
\nopagebreak\mylabel{L01612v}
\rehead{ }\normalsize\beginnumbering\briefempfaengerindex{Bahr, Hermann@\textsc{Bahr, Hermann}!zzzSchnitzler, Olga@\emph{von Olga Schnitzler}!1906-07-141@{14. 7. 1906}|(be}\briefempfaengerindex{Bahr, Hermann@\textsc{Bahr, Hermann}!zzzSchnitzler, Arthur@\emph{von Arthur Schnitzler}!1906-07-141@{14. 7. 1906}|(be}
\toendnotes[C]{\smallbreak\pagebreak[2]}\Standort{TMW, HS AM 60174 Ba.}
\physDesc{Bildpostkarte, 359 Zeichen
\newline{}Handschrift Arthur Schnitzler: 1) Bleistift, deutsche Kurrent\hspace{1em}2) Bleistift, lateinische Kurrent (\noindent{}Adresse)\hspace{1em}
\newline{}Handschrift Olga Schnitzler: Bleistift, lateinische Kurrent
\newline{}Versand: 1) Stempel: »\nobreak{}1{[}4{]}. 7. 06, 2–5 E\nobreak{}«.   2) Stempel: »\nobreak{}\oindex{Venedig@\textbf{Venedig}, \emph{P.PPLA}|pwk}Venezia, 16. 7{[}. 06{]}\nobreak{}«. 
\newline{}Ordnung: Lochung }
\buchAbdrucke{\weitereDrucke{1) Arthur Schnitzler: \emph{The Letters of Arthur Schnitzler to Hermann Bahr}. Chapel Hill: \emph{The University of North Carolina Press} 1978, S. 95.} \weitereDrucke{2) Hermann Bahr, Arthur Schnitzler: \emph{Briefwechsel, Aufzeichnungen, Dokumente (1891–1931)}. Göttingen: \emph{Wallstein} 2018, S. 380.} }\toendnotes[C]{\smallbreak}\pstart{}{\pb}Herrn\pend{}\pstart{}Hermann Bahr\pend{}\pstart{}Venezia\oindex{Italien@\textbf{Italien}, \emph{A.PCLI}|pw}\pend{}\pstart{}Casa Petrarca\pwindex{Petrarca, Francesco 1304-07-19 – 1374@\textsc{Petrarca, Francesco} (1304-07-19 – 1374), \emph{Schriftsteller/Schriftstellerin}|pwv}\oindex{Casa Petrarca@\textbf{Casa Petrarca}, \emph{Gebäude (K.GBD)}|pw}\pend{}\pstart{}Italie\oindex{Venedig@\textbf{Venedig}, \emph{P.PPLA}|pw}\pend{}{\bigskip}
\pstart
           \noindent{}\centering{}{\pb}\textcolor{gray}{\textbf{\label{K_L01612-1v}\edtext{Hilsen fra}{\lemma{\textnormal{\emph{Hilsen fra}}}\Cendnote{\textnormal{dänisch: Grüße aus}}}\label{K_L01612-1}{ }Marienlyst Parken\oindex{Marienlyst@\textbf{Marienlyst}, \emph{S.EST}|pw}}}\pend
           \vspace{1em}
\pstart
           \textsc{{\pb}Marienlyst\oindex{Marienlyst@\textbf{Marienlyst}, \emph{S.EST}|pw}}, 14. 7. 906\pend
           \vspace{0.5em}
\pstart
           hier, lieber Hermann, wohnen wir ſeit 14 Tagen. Es ist wunderſchön,
               und we{\geminationn} du herkämſt, kö{\geminationn}teſt du ein angenehmes Leben, ohne Strand zwar, aber auch ohne Brandwunden führen.
               Wir bleiben bis auf weiteres.\pend
           
\pstart
           Herzlichſt{\\[\baselineskip]}dein \spacefill\mbox{A.}\pend
           \leftskip=0em{}\selectlanguage{ngerman}\vspace{1em}
\pstart
           \noindent{}{\pb}{[}hs. :{]} (\label{T_L01612-1v}\edtext{Dies}{\lemma{\textnormal{\emph{Dies}}}\Cendnote{\textnormal{handschriftlicher Pfeil auf eine
                  Statue}}}\label{T_L01612-1} ſoll \textsc{Hamlet\pwindex{Hamlet@\emph{Hamlet}|pwv} sein.})\pend
           \pstart Herzliche Erwiderung Ihrer lieben Grüsse! { }\spacefill\mbox{Olga Schnitzler.}\pend{}\selectlanguage{ngerman}\endnumbering\briefempfaengerindex{Bahr, Hermann@\textsc{Bahr, Hermann}!zzzSchnitzler, Olga@\emph{von Olga Schnitzler}!1906-07-141@{14. 7. 1906}|)be}\briefempfaengerindex{Bahr, Hermann@\textsc{Bahr, Hermann}!zzzSchnitzler, Arthur@\emph{von Arthur Schnitzler}!1906-07-141@{14. 7. 1906}|)be}\mylabel{L01612h}  \normalsize

\doendnotes{C}
\bigskip
\vfill

\clearpage

\footnotesize

\lohead{\textsc{register}}

% Definiere theindex-Environment komplett neu ohne reledmac
\makeatletter
\renewenvironment{theindex}{%
  \section*{\indexname}%
  \setlength{\parindent}{0pt}%
  \setlength{\parskip}{0pt plus 0.3pt}%
  \let\item\@idxitem
}{%
  \clearpage
}
\makeatother

\IfFileExists{\jobname-pw.ind}{\input{\jobname-pw.ind}}{}

\end{document}

      