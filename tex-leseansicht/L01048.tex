%% latex-korrekturansicht-vorspann.tex
%% Vorspann für die Korrekturansicht.
%% Lädt die gemeinsame Datei latex-vorspann.tex mit gesetztem Schalter.

\newif\ifkorrekturansicht
\korrekturansichttrue

\input{../tex-inputs/latex-vorspann}


\section[Arthur Schnitzler an Julius Rodenberg, 21. 6. 1900]{L01048 Arthur Schnitzler an Julius Rodenberg, 21. 6. 1900}
\nopagebreak\mylabel{L01048v}
\rehead{ }\normalsize\beginnumbering\briefempfaengerindex{Rodenberg, Julius@\textsc{Rodenberg, Julius}!zzzSchnitzler, Arthur@\emph{von Arthur Schnitzler}!1900-06-212@{21. 6. 1900}|(be}
\toendnotes[C]{\smallbreak\pagebreak[2]}\Standort{Weimar, Klassik Stiftung, 81/X,2,10.}
\physDesc{Brief, 1 Blatt, 3 Seiten, 845 Zeichen
\newline{}Handschrift: schwarze Tinte, deutsche Kurrent}\toendnotes[C]{\smallbreak}
\pstart
           \raggedleft{}{\pb}21. 6. 900{\\}Wien IX. Frankgaſſe 1\oindex{Frankgasse 1@\textbf{Frankgasse 1}, \emph{Wohngebäude (K.WHS)}|pw}.\pend
           
\pstart{}Sehr geehrter Herr Doktor,\pend\vspace{0.5em}
\pstart
           Herr \textsc{Pastor}\pwindex{Pastor, Willy 22.09.1867 – 18.04.1933@\textsc{Pastor, Willy} (22.09.1867 – 18.04.1933), \emph{Schriftsteller/Schriftstellerin}|pw} war ſo freundlich mir auf meine erſte Anfrage Mitte Mai v J. zu
               antworten aber deſweiteren bis zu Ihrer Rückkehr zu verſchieben. Ich nehme an, Sie
               ſind wieder in Berlin\oindex{Berlin@\textbf{Berlin}, \emph{P.PPLC}|pw} und erlaube mir folgendes
                  mitzutheilen\textcolor{gray}{:}\pend
           
\pstart
           1) daſs ich Ihnen meine neue Novelle\pwindex{Frau Bertha Garlan. Roman@\emph{Frau Bertha Garlan. Roman}|pwv} (Titel ſteht noch nicht {\pb}feſt),
               welche etwa 3 Fortsetzg der Dtſch Rundſchau\orgindex{Deutsche Rundschau@Deutsche Rundschau|pw} in
               Anſpruch nähme, innerhalb der nächsten 8 Tage einſenden könnte.\pend
           
\pstart
           2) daſs ich aber darum bitten müßte, mir ein Reſultat \uuline{ganz
                  beſtimmt ſpäteſtens} 10 Tage nach dem Einlaufstage bekannt zu geben\pend
           
\pstart
           3.) und mir im Falle der Annahme einen Termin zu {\pb}beſtimmen.\pend
           
\pstart
           Ich wiederhole nochmals, daſs meiner Empfindg nach das \textsc{Sujet} für die Dtſch Rdſch\orgindex{Deutsche Rundschau@Deutsche Rundschau|pw} nicht ganz
               unbedenklich iſt, und daſs ich vor Abſendg des \textsc{Manuscriptes}\pwindex{Frau Bertha Garlan. Roman@\emph{Frau Bertha Garlan. Roman}|pwv} noch ein Wort von Ihnen erwarte.\pend
           
\pstart
           Hochachtgvoll{\\[\baselineskip]} Ihr ergebn\textcolor{gray}{er}{\\[\baselineskip]}\spacefill\mbox{ArthurSchnitzler}\pend
           \leftskip=0em{}\selectlanguage{ngerman}\endnumbering\briefempfaengerindex{Rodenberg, Julius@\textsc{Rodenberg, Julius}!zzzSchnitzler, Arthur@\emph{von Arthur Schnitzler}!1900-06-212@{21. 6. 1900}|)be}\mylabel{L01048h}  \normalsize

\doendnotes{C}
\bigskip
\vfill

\clearpage

\footnotesize

\lohead{\textsc{register}}

% Definiere theindex-Environment komplett neu ohne reledmac
\makeatletter
\renewenvironment{theindex}{%
  \section*{\indexname}%
  \setlength{\parindent}{0pt}%
  \setlength{\parskip}{0pt plus 0.3pt}%
  \let\item\@idxitem
}{%
  \clearpage
}
\makeatother

\IfFileExists{\jobname-pw.ind}{\input{\jobname-pw.ind}}{}

\end{document}

      