%% latex-leseansicht-vorspann.tex
%% Vorspann für die Leseansicht.
%% Lädt die gemeinsame Datei latex-vorspann.tex mit nicht gesetztem Schalter.

\newif\ifkorrekturansicht
\korrekturansichtfalse

\input{../tex-inputs/latex-vorspann}


         
         \newcommand{\erwaehntePersonen}{Personen: Willy Pastor, Julius Rodenberg}
         \newcommand{\erwaehnteInstitutionen}{Institutionen: Deutsche Rundschau}
         \newcommand{\erwaehnteOrte}{Orte: Berlin, Frankgasse, Wien}
         \newcommand{\erwaehnteWerke}{Werke: Frau Bertha Garlan. Roman}
               \section[Arthur Schnitzler an Julius Rodenberg, 21. 6. 1900]{ Arthur Schnitzler an Julius Rodenberg, 21. 6. 1900}\nopagebreak\mylabel{v}\rehead{ }\begin{ledgroupsized}[t]{13cm}\normalsize\beginnumbering \toendnotes[C]{\smallbreak\pagebreak[2]} \Standort{Weimar, Klassik Stiftung, 81/X,2,10.}
\physDesc{Brief, 1 Blatt, 3 Seiten
\newline{}Handschrift: schwarze Tinte, deutsche Kurrent}\toendnotes[C]{\smallbreak}\pstart
           \raggedleft{}{\pb}21. 6. 900{\\}Wien IX. Frankgaſſe 1\oindex{Frankgasse@\textbf{Frankgasse}|pw}.\pend
           \pstart{}Sehr geehrter Herr Doktor,\pend\pstart
           Herr \textsc{Pastor}\pwindex{Pastor, Willy 22.09.1867 – 18.04.1933@\textsc{Pastor, Willy} (22.09.1867 – 18.04.1933), \emph{Schriftsteller}|pw} war ſo freundlich mir auf meine erſte Anfrage Mitte Mai v J.
                    zu antworten aber deſweiteren bis zu Ihrer Rückkehr zu verſchieben. Ich nehme
                    an, Sie ſind wieder in Berlin\oindex{Berlin@\textbf{Berlin}|pw} und erlaube mir
                    folgendes mitzutheilen\textcolor{gray}{:}\pend
           \pstart
           1) daſs ich Ihnen meine neue Novelle\pwindex{Schnitzler, Arthur 15.05.1862 – 21.10.1931@\textsc{Schnitzler, Arthur} (15.05.1862 – 21.10.1931), \emph{Schriftsteller, Mediziner}!Frau Bertha Garlan. Roman15.1.1901 – 15.3.1901@\strich\emph{Frau Bertha Garlan. Roman} {[}15.1.1901 – 15.3.1901{]}|pwv} (Titel ſteht noch nicht {\pb}feſt), welche etwa 3 Fortsetzg der Dtſch
                        Rundſchau\orgindex{Deutsche Rundschau@Deutsche Rundschau|pw} in Anſpruch nähme, innerhalb der nächsten 8 Tage einſenden
                    könnte.\pend
           \pstart
           2) daſs ich aber darum bitten müßte, mir ein Reſultat \uuline{ganz beſtimmt ſpäteſtens} 10 Tage nach dem Einlaufstage bekannt zu
                    geben\pend
           \pstart
           3.) und mir im Falle der Annahme einen Termin zu {\pb}beſtimmen.\pend
           \pstart
           Ich wiederhole nochmals, daſs meiner Empfindg nach das \textsc{Sujet} für die Dtſch Rdſch\orgindex{Deutsche Rundschau@Deutsche Rundschau|pw} nicht ganz
                    unbedenklich iſt, und daſs ich vor Abſendg des \textsc{Manuscriptes}\pwindex{Schnitzler, Arthur 15.05.1862 – 21.10.1931@\textsc{Schnitzler, Arthur} (15.05.1862 – 21.10.1931), \emph{Schriftsteller, Mediziner}!Frau Bertha Garlan. Roman15.1.1901 – 15.3.1901@\strich\emph{Frau Bertha Garlan. Roman} {[}15.1.1901 – 15.3.1901{]}|pwv} noch ein Wort von Ihnen erwarte.\pend
           \pstart
           Hochachtgvoll{\\[\baselineskip]} Ihr ergebn\textcolor{gray}{er}{\\[\baselineskip]}\spacefill\mbox{ArthurSchnitzler}\pend
           \leftskip=0em{}
         
         \endnumbering\mylabel{h}\end{ledgroupsized}  \newcommand{\dateiname}{L01048}\newcommand{\titel}{Arthur Schnitzler an Julius Rodenberg, 21. 6. 1900}\newcommand{\editorInnen}{Martin Anton Müller und Gerd-Hermann Susen}%% latex-leseansicht-abspann.tex
%% Abspann für die Leseansicht.
%% Der Schalter \ifkorrekturansicht ist bereits durch den Vorspann gesetzt.

%% latex-abspann.tex
%% Gemeinsamer Abspann für Korrekturansicht und Leseansicht.
%% Setzt den Schalter \ifkorrekturansicht voraus (gesetzt in den
%% einbindenden Dateien latex-korrekturansicht-abspann.tex bzw.
%% latex-leseansicht-abspann.tex).
%% ---------------------------------------------------------------

\normalsize

% Das esempio-Environment wird nur in der Leseansicht benötigt
\ifkorrekturansicht\else
\newenvironment{esempio}[3]%
{
    \vspace{1.5ex}
    \rlap{\underline{#1}}
    \par
    \setlength{\parindent}{0cm}
    \nopagebreak
    \leftskip=#2cm
    \rightskip=#3cm
}
{
    \par
}
\fi

\doendnotes{C}
\bigskip
\vfill

\clearpage

\footnotesize

\ifkorrekturansicht
  \lohead{\textsc{register}}
\fi

% theindex-Environment neu definieren ohne reledmac
\makeatletter
\renewenvironment{theindex}{%
  \ifkorrekturansicht
    \section*{\indexname}%
  \else
    \subsubsection*{Index der erwähnten Entitäten}%
  \fi
  \setlength{\parindent}{0pt}%
  \setlength{\parskip}{0pt plus 0.3pt}%
  \let\item\@idxitem
}{%
  \ifkorrekturansicht\clearpage\fi
}
\makeatother

\IfFileExists{\jobname-pw.ind}{\input{\jobname-pw.ind}}{}

% Quellenangabe nur in der Leseansicht
\ifkorrekturansicht\else
% Fallback-Definitionen, falls die .tex-Datei \titel etc. nicht gesetzt hat
\providecommand{\titel}{}
\providecommand{\editorInnen}{}
\providecommand{\dateiname}{\jobname}

\vspace{3cm}

\vfill

\footnotesize
\textsc{Quelle}: \titel. Herausgegeben von {\editorInnen}. In: \emph{Arthur Schnitzler: Briefwechsel mit Autorinnen und Autoren}.
 Digitale Edition, https://schnitzler-briefe.acdh.oeaw.ac.at/{\dateiname}.html (Stand \today)
\fi

\end{document}


      