%% latex-leseansicht-vorspann.tex
%% Vorspann für die Leseansicht.
%% Lädt die gemeinsame Datei latex-vorspann.tex mit nicht gesetztem Schalter.

\newif\ifkorrekturansicht
\korrekturansichtfalse

\input{../tex-inputs/latex-vorspann}


\section[Arthur Schnitzler an Gustav Schwarzkopf, 11. 8. 1902]{L04073 Arthur Schnitzler an Gustav Schwarzkopf, 11. 8. 1902}
\nopagebreak\mylabel{L04073v}
\rehead{ }\normalsize\beginnumbering\briefempfaengerindex{Schwarzkopf, Gustav@\textsc{Schwarzkopf, Gustav}!zzzSchnitzler, Arthur@\emph{von Arthur Schnitzler}!1902-08-111@{11. 8. 1902}|(be}
\toendnotes[C]{\smallbreak\pagebreak[2]}
\correspDesc{Versand  durch Arthur Schnitzler am 11. 8. 1902 in Hinterbrühl
\newline{}Erhalt  durch Gustav Schwarzkopf im Zeitraum [12. 8. 1902 – 16. 8. 1902?] in Wien}\toendnotes[C]{\smallbreak}
\Standort{CUL, Schnitzler, B 96.}
\physDesc{Brief, 1 Blatt, 4 Seiten, 991 Zeichen
\newline{}Handschrift: Bleistift, deutsche Kurrent}\toendnotes[C]{\smallbreak}
\pstart
           \raggedleft{}{\pb}11. 8. 902{\\}\textsc{Hntrbrl.\oindex{Hinterbrühl@\textbf{Hinterbrühl}, \emph{Hauptstadt}|pw}}\pend
           \vspace{0.5em}
\pstart
           lieber Guſtav, ich habe Sie \label{K_L04073-1v}\edtext{geſtern Nachmittag aufgeſucht}{\lemma{\textnormal{\emph{gestern … aufgesucht}}}\Cendnote{\textnormal{Nicht im \emph{Tagebuch}\pwindex{Schnitzler, Arthur 15. 5. 1862 Wien – 21. 10. 1931 ebd.@\textsc{Schnitzler, Arthur} (15. 5. 1862 Wien – 21. 10. 1931 ebd.), \emph{Schriftsteller, Mediziner}!Tagebuch@\strich\emph{Tagebuch}|pwk}-Eintrag zum 10. 8. 1902 erwähnt. }}}\label{K_L04073-1}, um Ihnen die Nachricht von der am
                  9., Nachmittg 4, unter normalem Verlauf erfolgten Geburt eines Jünglings\pwindex{Schnitzler, Heinrich 9.\,8.\,1902 Hinterbrühl – 12.\,7.\,1982 Wien@\textsc{Schnitzler, Heinrich} (9.\,8.\,1902 Hinterbrühl – 12.\,7.\,1982 Wien), \emph{Regisseur, Schauspieler}|pwv} zu überbringen, und
               bei dieſer Gegenheit durch Dr Max\pwindex{Schwarzkopf, Max 12.\,6.\,1857 Wien – 14.\,4.\,1928 ebd.@\textsc{Schwarzkopf, Max} (12.\,6.\,1857 Wien – 14.\,4.\,1928 ebd.), \emph{Rechtsanwalt}|pw} erfahren,
               dſs Sie nach Strobl\oindex{Strobl@\textbf{Strobl}, \emph{Verwaltungsgebiet}|pw} abgedampft ſind. So ſehr ich
               das einerſeits bedaure, ſo bin ich {\pb}doch anderſeits vollko{\geminationm}en einverſtanden und zolle der
                  Energie des Herrn Hiller\pwindex{Hiller, Max 17.\,9.\,1856 Osijek – 13.\,1.\,1941 Wien@\textsc{Hiller, Max} (17.\,9.\,1856 Osijek – 13.\,1.\,1941 Wien), \emph{Industrieller}|pw} meine vollſte Anerke{\geminationn}ung (Grüßen Sie ihn
               bitte und ſeine Frau\pwindex{Hiller, Ida 11.\,11.\,1858 Siklós – 3.\,8.\,1922 Wien@\textsc{Hiller, Ida} (11.\,11.\,1858 Siklós – 3.\,8.\,1922 Wien)|pwv} – ich
               habe natürlich vergeſſen zu \label{K_L04073-2v}\edtext{condoliren\pwindex{Hiller, Babette 6.\,9.\,1837 Prag – 30.\,7.\,1902 Pörtschach am Wörthersee@\textsc{Hiller, Babette} (6.\,9.\,1837 Prag – 30.\,7.\,1902 Pörtschach am Wörthersee)|pwv}}{\lemma{\textnormal{\emph{condoliren}}}\Cendnote{\textnormal{Am 30. 7. 1902 war
                     Babette Hiller\pwindex{Hiller, Babette 6.\,9.\,1837 Prag – 30.\,7.\,1902 Pörtschach am Wörthersee@\textsc{Hiller, Babette} (6.\,9.\,1837 Prag – 30.\,7.\,1902 Pörtschach am Wörthersee)|pwk}, die Mutter von Max Hiller\pwindex{Hiller, Max 17.\,9.\,1856 Osijek – 13.\,1.\,1941 Wien@\textsc{Hiller, Max} (17.\,9.\,1856 Osijek – 13.\,1.\,1941 Wien), \emph{Industrieller}|pwk}, verstorben.}}}\label{K_L04073-2}, entſchuldg
               Sie mich vielleicht?) – Hoffentlich behagen Sie ſich dort ſo wohl, daſs Sie ein paar
               Wochen bleiben, was ich, wie Sie wiſſen, für recht {\pb}vortheilhaft fände. Am Ende{ }ſtürzen
               Sie{ }ſogar \textcolor{gray}{an} der Luft etwas zu arbeiten, was ich, wie Sie
               gleichfalls wiſſen, für das allervortheilhafteſte hielte.\pend
           
\pstart
           Hier ist alles wohl, grüßt Sie, ſoweit es zu ſolchen Intelligenzäußerungen fähig iſt,
               vielmals und wünſcht Ihnen einen ſehr {\pb}angenehmen, wetter- u launegeſegneten Aufenthalt.\pend
           
\pstart
           Von Herzen Ihr{\\[\baselineskip]}\spacefill\mbox{Arthur}\pend
           \leftskip=0em{}\selectlanguage{ngerman}\endnumbering\briefempfaengerindex{Schwarzkopf, Gustav@\textsc{Schwarzkopf, Gustav}!zzzSchnitzler, Arthur@\emph{von Arthur Schnitzler}!1902-08-111@{11. 8. 1902}|)be}\mylabel{L04073h}
\begin{anhang}
\end{anhang}\newcommand{\dateiname}{L04073}\newcommand{\titel}{Arthur Schnitzler an Gustav Schwarzkopf, 11. 8. 1902}\newcommand{\editorInnen}{Herausgegeben von Jahnke, SelmaMüller, Martin Anton}%% latex-leseansicht-abspann.tex
%% Abspann für die Leseansicht.
%% Der Schalter \ifkorrekturansicht ist bereits durch den Vorspann gesetzt.

%% latex-abspann.tex
%% Gemeinsamer Abspann für Korrekturansicht und Leseansicht.
%% Setzt den Schalter \ifkorrekturansicht voraus (gesetzt in den
%% einbindenden Dateien latex-korrekturansicht-abspann.tex bzw.
%% latex-leseansicht-abspann.tex).
%% ---------------------------------------------------------------

\normalsize

% Das esempio-Environment wird nur in der Leseansicht benötigt
\ifkorrekturansicht\else
\newenvironment{esempio}[3]%
{
    \vspace{1.5ex}
    \rlap{\underline{#1}}
    \par
    \setlength{\parindent}{0cm}
    \nopagebreak
    \leftskip=#2cm
    \rightskip=#3cm
}
{
    \par
}
\fi

\doendnotes{C}
\bigskip
\vfill

\clearpage

\footnotesize

\ifkorrekturansicht
  \lohead{\textsc{register}}
\fi

% theindex-Environment neu definieren ohne reledmac
\makeatletter
\renewenvironment{theindex}{%
  \ifkorrekturansicht
    \section*{\indexname}%
  \else
    \subsubsection*{Index der erwähnten Entitäten}%
  \fi
  \setlength{\parindent}{0pt}%
  \setlength{\parskip}{0pt plus 0.3pt}%
  \let\item\@idxitem
}{%
  \ifkorrekturansicht\clearpage\fi
}
\makeatother

\IfFileExists{\jobname-pw.ind}{\input{\jobname-pw.ind}}{}

% Quellenangabe nur in der Leseansicht
\ifkorrekturansicht\else
% Fallback-Definitionen, falls die .tex-Datei \titel etc. nicht gesetzt hat
\providecommand{\titel}{}
\providecommand{\editorInnen}{}
\providecommand{\dateiname}{\jobname}

\vspace{3cm}

\vfill

\footnotesize
\textsc{Quelle}: \titel. Herausgegeben von {\editorInnen}. In: \emph{Arthur Schnitzler: Briefwechsel mit Autorinnen und Autoren}.
 Digitale Edition, https://schnitzler-briefe.acdh.oeaw.ac.at/{\dateiname}.html (Stand \today)
\fi

\end{document}


