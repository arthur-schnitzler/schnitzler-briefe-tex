%% latex-korrekturansicht-vorspann.tex
%% Vorspann für die Korrekturansicht.
%% Lädt die gemeinsame Datei latex-vorspann.tex mit gesetztem Schalter.

\newif\ifkorrekturansicht
\korrekturansichttrue

\input{../tex-inputs/latex-vorspann}


\section[Richard Beer-Hofmann an Arthur Schnitzler, 20. 7. 1910]{L01949 Richard Beer-Hofmann an Arthur Schnitzler, 20. 7. 1910}
\nopagebreak\mylabel{L01949v}
\rehead{ }\normalsize\beginnumbering\briefempfaengerindex{Schnitzler, Arthur@\textsc{Schnitzler, Arthur}!zzzBeer-Hofmann, Richard@\emph{von Richard Beer-Hofmann}!1910-07-201@{20. 7. 1910}|(be}
\toendnotes[C]{\smallbreak\pagebreak[2]}\Standort{CUL, Schnitzler, B 8.}
\physDesc{Bildpostkarte, 274 Zeichen
\newline{}Handschrift: Bleistift, lateinische Kurrent
\newline{}Versand: 1) Stempel: »\nobreak{}\oindex{Bad Aussee@\textbf{Bad Aussee}, \emph{P.PPLA3}|pwk}Aussee in Steiermark, 20. VII. 10\nobreak{}«.   2) mutmaßlich vom Briefträger die Straßenangabe in der Adressierung
                                 gestrichen
\newline{}Schnitzler: mit Bleistift beschriftet: »\textsc{BH}« 
\newline{}Ordnung: mit Bleistift von unbekannter Hand nummeriert:
                                    »235« }
\buchAbdrucke{\weitereDrucke{Arthur Schnitzler, Richard Beer-Hofmann: \emph{Briefwechsel 1891–1931}. Wien, Zürich: \emph{Europaverlag} 1992, S. 211.} }\toendnotes[C]{\smallbreak}\pstart{}{\pb}Herrn\pend{}\pstart{}Arthur Schnitzler\pend{}\pstart{}Wien\oindex{Wien@\textbf{Wien}, \emph{A.ADM2}|pw}\pend{}\pstart{}XVIII Spöttelgasse 7\oindex{Edmund-Weiss-Gasse 7@\textbf{Edmund-Weiß-Gasse 7}, \emph{Wohngebäude (K.WHS)}|pw}\pend{}{\bigskip}
\pstart
           \noindent{}\centering{}{\pb}\textcolor{gray}{\textbf{Aussee\oindex{Bad Aussee@\textbf{Bad Aussee}, \emph{P.PPLA3}|pw} von der Tauscherin\oindex{Tauscherin@\textbf{Tauscherin}, \emph{Ausflugsziel}|pw}.}}\pend
           \vspace{1em}
\pstart
           \noindent{}{\pb}Lieber Arthur, wir
               sind auf 2 Stunden – ohne irgendwelche Besuchsabsichten hier. Das Erste was wir
               erblicken: Rothe Zettel »Das Repertoirestück des k.k.
                  Hofburgtheaters\oindex{Burgtheater@\textbf{Burgtheater}, \emph{S.THTR}|pw} ›Liebelei\pwindex{Liebelei. Schauspiel in drei Akten@\emph{Liebelei. Schauspiel in drei Akten}|pw}‹«.\pend
           
\pstart
           \label{T_L01949-1v}\edtext{Herzliche Grüße, Sie
                  Repertoirebeherrscher!}{\lemma{\textnormal{\emph{Herzliche … Repertoirebeherrscher!}}}\Cendnote{\textnormal{von der rechten
                  Seite weg gegen den Uhrzeigersinn an zwei Seiten entlang des Textes geschrieben,
                  die Unterschrift auf der dritten}}}\label{T_L01949-1}\pend
           \pstart \spacefill\mbox{Richard}\pend{}\selectlanguage{ngerman}\endnumbering\briefempfaengerindex{Schnitzler, Arthur@\textsc{Schnitzler, Arthur}!zzzBeer-Hofmann, Richard@\emph{von Richard Beer-Hofmann}!1910-07-201@{20. 7. 1910}|)be}\mylabel{L01949h}  \normalsize

\doendnotes{C}
\bigskip
\vfill

\clearpage

\footnotesize

\lohead{\textsc{register}}

% Definiere theindex-Environment komplett neu ohne reledmac
\makeatletter
\renewenvironment{theindex}{%
  \section*{\indexname}%
  \setlength{\parindent}{0pt}%
  \setlength{\parskip}{0pt plus 0.3pt}%
  \let\item\@idxitem
}{%
  \clearpage
}
\makeatother

\IfFileExists{\jobname-pw.ind}{\input{\jobname-pw.ind}}{}

\end{document}

      