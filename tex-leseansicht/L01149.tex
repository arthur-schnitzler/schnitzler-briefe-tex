%% latex-korrekturansicht-vorspann.tex
%% Vorspann für die Korrekturansicht.
%% Lädt die gemeinsame Datei latex-vorspann.tex mit gesetztem Schalter.

\newif\ifkorrekturansicht
\korrekturansichttrue

\input{../tex-inputs/latex-vorspann}


\section[Hugo von Hofmannsthal an Arthur Schnitzler, 18. 7. {[}1901{]}]{L01149 Hugo von Hofmannsthal an Arthur Schnitzler, 18. 7. {[}1901{]}}
\nopagebreak\mylabel{L01149v}
\rehead{ }\normalsize\beginnumbering\briefempfaengerindex{Schnitzler, Arthur@\textsc{Schnitzler, Arthur}!zzzHofmannsthal, Hugo von@\emph{von Hugo von Hofmannsthal}!1901-07-181@{18. 7. {[}1901{]}}|(be}
\toendnotes[C]{\smallbreak\pagebreak[2]}\Standort{CUL, Schnitzler, B 43b/1.}
\physDesc{Brief, 1 Blatt, 2 Seiten, 1795 Zeichen
\newline{}Handschrift: schwarze Tinte, deutsche Kurrent
\newline{}Schnitzler: mit Bleistift die Jahreszahl »901« ergänzt 
\newline{}Ordnung: mit Bleistift von unbekannter Hand nummeriert:
                                    »177« }
\buchAbdrucke{\weitereDrucke{1) Hugo von Hofmannsthal, Arthur Schnitzler: \emph{Briefwechsel}. Frankfurt am Main: \emph{S. Fischer} 1964, S. 149–150.} \weitereDrucke{2) Hermann Bahr, Arthur Schnitzler: \emph{Briefwechsel, Aufzeichnungen, Dokumente (1891–1931)}. Göttingen: \emph{Wallstein} 2018, S. 213.} }\toendnotes[C]{\smallbreak}
\pstart
           \raggedleft{}{\pb}18. Juli.\hspace*{1.5em}Rodaun\oindex{Rodaun@\textbf{Rodaun}, \emph{A.ADM4}|pw},\pend
           
\pstart{}mein guter lieber Arthur\pend\vspace{0.5em}
\pstart
           ſchon gleich beim Betreten dieses Hauſes am 1\textsuperscript{ten}\label{K_L01149-1v}\edtext{Juni}{\lemma{\textnormal{\emph{Juni}}}\Cendnote{\textnormal{Von Schnitzler mit Bleistift zu »Juli«
                     korrigiert.}}}\label{K_L01149-1} habe ich mit herzlicher Freude Ihren lieben Brief gefunden, und es iſt mir
               faſt unbegreiflich, daſs 17 Tage vergehen konnten, wo ich wirklich jeden Tag daran
               dachte, Ihnen zu ſchreiben, und immer wieder die eine Viertelſtunde ſich wegrückte.
               Allerdings hab ich in dieſen Tagen mit ziemlicher Haſt und ziemlich viel Einfällen
               den letzten Act des Ballets\pwindex{Triumph der Zeit@\emph{Der Triumph der Zeit}|pwv}
               endlich ausgeführt, ſo daſs von nun an dieſes ziemlich umfangreiche Ding, deſſen
               Werth oder Unwerth ich abſolut nicht abſchätzen kann, unter meinen Arbeiten exiſtiren
               wird. Hoffentlich kann ichs Ihnen im Herbſt vorleſen und es miſsfällt Ihnen
               nicht.\pend
           
\pstart
           Dieſes Aneinander-vorüber-ſchweben in Innsbruck\oindex{Innsbruck@\textbf{Innsbruck}, \emph{A.ADM2}|pw}
               hat mir damals recht leid gethan. Hätte man nicht ein paar Stunden zuſammen ſein
               können? ich glaube daſs wäre für alle vier ein freundlicher Eindruck
               geweſen. Auch iſt doch von Gerty\pwindex{Hofmannsthal, Gertrude von 16.03.1880 – 09.11.1959@\textsc{Hofmannsthal, Gertrude von} (16.03.1880 – 09.11.1959)|pw} eine
               Indiscretion eben ſo wenig zu fürchten wie von mir und überdies hätte man ihr den
               Familiennamen der andern\pwindex{Schnitzler, Olga 17.01.1882 – 13.01.1970@\textsc{Schnitzler, Olga} (17.01.1882 – 13.01.1970), \emph{Schauspieler/Schauspielerin, Sänger/Sängerin}|pwv} gar
               nicht zu ſagen gebraucht. Wir ſind an dieſem Abend noch ins Hofgartengaſthaus\oindex{Hofgartengasthaus@\textbf{Hofgartengasthaus}, \emph{Lokal (K.LKL)}|pw} nachtmahlen gegangen, dem einzigen Ort, wo man
               »im Freien nachtmahlt« und ich habe ſehr gehofft, {\pb}daſs wir uns dort begegnen würden,
               es iſt aber leider nicht der Fall geweſen.\hspace*{2.5em}Mit dem
               Haus und dem Leben hier bin ich ſehr zufrieden, ich will aber nicht viel darüber
               ſagen, ſondern freue mich darauf, es Ihnen zu zeigen. Jetzt wüſste ich ſchon gerne,
               wo ich mir vorſtellen ſoll, daſs Sie ſind.\hspace*{2.5em}Ich will
               nun möglichſt bald anfangen, das große figurenreiche und tragiſche Stück\pwindex{Pompilia oder das Leben@\emph{Pompilia oder das Leben}|pwv} zu ſchreiben, deſſen Stoff mir von
                  Browning\pwindex{Browning, Robert 1812-05-07 – 1889-12-12@\textsc{Browning, Robert} (1812-05-07 – 1889-12-12), \emph{Schriftsteller/Schriftstellerin}|pw} überliefert iſt.\pend
           
\pstart
           Von Menſchen ſehe ich Bahr\pwindex{Bahr, Hermann 19.07.1863 – 15.01.1934@\textsc{Bahr, Hermann} (19.07.1863 – 15.01.1934), \emph{Schriftsteller/Schriftstellerin, Kritiker/Kritikerin}|pw}, der öfter \label{K_L01149-2v}\edtext{herüberkommt}{\lemma{\textnormal{\emph{herüberkommt}}}\Cendnote{\textnormal{Das neu bezogene Haus Hofmannsthals\pwindex{Hofmannsthal, Hugo von 1874-02-01 – 1929-07-15@\textsc{Hofmannsthal, Hugo von} (1874-02-01 – 1929-07-15), \emph{Schriftsteller/Schriftstellerin}|pwk} lag etwa acht Kilometer von dem Bahrs\pwindex{Bahr, Hermann 19.07.1863 – 15.01.1934@\textsc{Bahr, Hermann} (19.07.1863 – 15.01.1934), \emph{Schriftsteller/Schriftstellerin, Kritiker/Kritikerin}|pwk} entfernt.}}}\label{K_L01149-2}, und erwarte nächſtens Andrian\pwindex{Andrian-Werburg, Leopold von 09.05.1875 – 19.11.1951@\textsc{Andrian-Werburg, Leopold von} (09.05.1875 – 19.11.1951), \emph{Schriftsteller/Schriftstellerin, Diplomat/Diplomatin}|pw} für einige Tage.\pend
           
\pstart
           Ich freue mich ſehr auf einen Brief von Ihnen.\pend
           
\pstart
           Von Herzen Ihr{\\[\baselineskip]}\spacefill\mbox{Hugo.}\pend
           \leftskip=0em{}\selectlanguage{ngerman}\endnumbering\briefempfaengerindex{Schnitzler, Arthur@\textsc{Schnitzler, Arthur}!zzzHofmannsthal, Hugo von@\emph{von Hugo von Hofmannsthal}!1901-07-181@{18. 7. {[}1901{]}}|)be}\mylabel{L01149h}  \normalsize

\doendnotes{C}
\bigskip
\vfill

\clearpage

\footnotesize

\lohead{\textsc{register}}

% Definiere theindex-Environment komplett neu ohne reledmac
\makeatletter
\renewenvironment{theindex}{%
  \section*{\indexname}%
  \setlength{\parindent}{0pt}%
  \setlength{\parskip}{0pt plus 0.3pt}%
  \let\item\@idxitem
}{%
  \clearpage
}
\makeatother

\IfFileExists{\jobname-pw.ind}{\input{\jobname-pw.ind}}{}

\end{document}

      