%% latex-leseansicht-vorspann.tex
%% Vorspann für die Leseansicht.
%% Lädt die gemeinsame Datei latex-vorspann.tex mit nicht gesetztem Schalter.

\newif\ifkorrekturansicht
\korrekturansichtfalse

\input{../tex-inputs/latex-vorspann}


         
         \renewcommand{\erwaehntePersonen}{Personen: Leopold von Andrian-Werburg, Hermann Bahr, Robert Browning, Gertrude von Hofmannsthal, Olga Schnitzler}
         \renewcommand{\erwaehnteOrte}{Orte: Hofgartengasthaus, Innsbruck, Rodaun, Wien}
         \renewcommand{\erwaehnteWerke}{Werke: Der Triumph der Zeit, Pompilia oder das Leben}
               \section[Hugo von Hofmannsthal an Arthur Schnitzler, 18. 7. {[}1901{]}]{ Hugo von Hofmannsthal an Arthur Schnitzler, 18. 7. {[}1901{]}}\nopagebreak\mylabel{v}\rehead{ }\begin{ledgroupsized}[t]{13cm}\normalsize\beginnumbering \toendnotes[C]{\smallbreak\pagebreak[2]} \Standort{CUL, Schnitzler, B 43b/1.}
\physDesc{Brief, 1 Blatt, 2 Seiten, 1795 Zeichen
\newline{}Handschrift: schwarze Tinte, deutsche Kurrent
\newline{}Schnitzler: mit Bleistift die Jahreszahl »901« ergänzt 
\newline{}Ordnung: mit Bleistift von unbekannter Hand nummeriert:
                                    »177« }\buchAbdrucke{\weitereDrucke{1) Hugo von Hofmannsthal, Arthur Schnitzler: \emph{Briefwechsel}. Hg. Therese Nickl und Heinrich Schnitzler. Frankfurt am Main: \emph{S. Fischer} 1964, S. 149–150.} \weitereDrucke{2) Hermann Bahr, Arthur Schnitzler: \emph{Briefwechsel, Aufzeichnungen, Dokumente (1891–1931)}. Hg. Kurt Ifkovits und Martin Anton Müller. Göttingen: \emph{Wallstein} 2018, S. 213.} }\toendnotes[C]{\smallbreak}\pstart
           \raggedleft{}{\pb}18. Juli.\hspace*{1.5em}Rodaun\oindex{Rodaun@\textbf{Rodaun}|pw},\pend
           \pstart{}mein guter lieber Arthur\pend\pstart
           ſchon gleich beim Betreten dieses Hauſes am 1\textsuperscript{ten}\label{K_L01149-1v}\edtext{Juni}{\lemma{\textnormal{\emph{Juni}}}\Cendnote{\textnormal{Von Schnitzler mit Bleistift zu »Juli«
                     korrigiert.}}}\label{K_L01149-1h} habe ich mit herzlicher Freude Ihren lieben Brief gefunden, und es iſt mir
               faſt unbegreiflich, daſs 17 Tage vergehen konnten, wo ich wirklich jeden Tag daran
               dachte, Ihnen zu ſchreiben, und immer wieder die eine Viertelſtunde ſich wegrückte.
               Allerdings hab ich in dieſen Tagen mit ziemlicher Haſt und ziemlich viel Einfällen
               den letzten Act des Ballets\pwindex{Hofmannsthal, Hugo von 1874-02-01 – 1929-07-15@\textsc{Hofmannsthal, Hugo von} (1874-02-01 – 1929-07-15), \emph{Schriftsteller}!Triumph der Zeit1. 9. 1901@\strich\emph{Der Triumph der Zeit} {[}1. 9. 1901{]}|pwv}
               endlich ausgeführt, ſo daſs von nun an dieſes ziemlich umfangreiche Ding, deſſen
               Werth oder Unwerth ich abſolut nicht abſchätzen kann, unter meinen Arbeiten exiſtiren
               wird. Hoffentlich kann ichs Ihnen im Herbſt vorleſen und es miſsfällt Ihnen
               nicht.\pend
           \pstart
           Dieſes Aneinander-vorüber-ſchweben in Innsbruck\oindex{Innsbruck@\textbf{Innsbruck}|pw}
               hat mir damals recht leid gethan. Hätte man nicht ein paar Stunden zuſammen ſein
               können? ich glaube daſs wäre für alle vier ein freundlicher Eindruck
               geweſen. Auch iſt doch von Gerty\pwindex{Hofmannsthal, Gertrude von 16.03.1880 – 09.11.1959@\textsc{Hofmannsthal, Gertrude von} (16.03.1880 – 09.11.1959)|pw} eine
               Indiscretion eben ſo wenig zu fürchten wie von mir und überdies hätte man ihr den
               Familiennamen der andern\pwindex{Schnitzler, Olga 17.01.1882 – 13.01.1970@\textsc{Schnitzler, Olga} (17.01.1882 – 13.01.1970), \emph{Schauspielerin, Sängerin}|pwv} gar
               nicht zu ſagen gebraucht. Wir ſind an dieſem Abend noch ins Hofgartengaſthaus\oindex{Hofgartengasthaus@\textbf{Hofgartengasthaus}|pw} nachtmahlen gegangen, dem einzigen Ort, wo man
               »im Freien nachtmahlt« und ich habe ſehr gehofft, {\pb}daſs wir uns dort begegnen würden,
               es iſt aber leider nicht der Fall geweſen.\hspace*{2.5em}Mit dem
               Haus und dem Leben hier bin ich ſehr zufrieden, ich will aber nicht viel darüber
               ſagen, ſondern freue mich darauf, es Ihnen zu zeigen. Jetzt wüſste ich ſchon gerne,
               wo ich mir vorſtellen ſoll, daſs Sie ſind.\hspace*{2.5em}Ich will
               nun möglichſt bald anfangen, das große figurenreiche und tragiſche Stück\pwindex{Hofmannsthal, Hugo von 1874-02-01 – 1929-07-15@\textsc{Hofmannsthal, Hugo von} (1874-02-01 – 1929-07-15), \emph{Schriftsteller}!Pompilia oder das Leben1901@\strich\emph{Pompilia oder das Leben} {[}1901{]}|pwv} zu ſchreiben, deſſen Stoff mir von
                  Browning\pwindex{Browning, Robert 1812-05-07 – 1889-12-12@\textsc{Browning, Robert} (1812-05-07 – 1889-12-12), \emph{Schriftsteller}|pw} überliefert iſt.\pend
           \pstart
           Von Menſchen ſehe ich Bahr\pwindex{Bahr, Hermann 19.07.1863 – 15.01.1934@\textsc{Bahr, Hermann} (19.07.1863 – 15.01.1934), \emph{Schriftsteller, Kritiker}|pw}, der öfter \label{K_L01149-2v}\edtext{herüberkommt}{\lemma{\textnormal{\emph{herüberkommt}}}\Cendnote{\textnormal{Das neu bezogene Haus Hofmannsthal\pwindex{Hofmannsthal, Hugo von 1874-02-01 – 1929-07-15@\textsc{Hofmannsthal, Hugo von} (1874-02-01 – 1929-07-15), \emph{Schriftsteller}|pwk}s lag etwa acht Kilometer von dem Bahrs\pwindex{Bahr, Hermann 19.07.1863 – 15.01.1934@\textsc{Bahr, Hermann} (19.07.1863 – 15.01.1934), \emph{Schriftsteller, Kritiker}|pwk} entfernt.}}}\label{K_L01149-2h}, und erwarte nächſtens Andrian\pwindex{Andrian-Werburg, Leopold von 09.05.1875 – 19.11.1951@\textsc{Andrian-Werburg, Leopold von} (09.05.1875 – 19.11.1951), \emph{Schriftsteller, Diplomat}|pw} für einige Tage.\pend
           \pstart
           Ich freue mich ſehr auf einen Brief von Ihnen.\pend
           \pstart
           Von Herzen Ihr{\\[\baselineskip]}\spacefill\mbox{Hugo.}\pend
           \leftskip=0em{}
         
         \endnumbering\mylabel{h}\end{ledgroupsized}  \newcommand{\dateiname}{L01149}\newcommand{\titel}{Hugo von Hofmannsthal an Arthur Schnitzler, 18. 7. [1901]}\newcommand{\editorInnen}{ Martin Anton Müller und Gerd-Hermann Susen}%% latex-leseansicht-abspann.tex
%% Abspann für die Leseansicht.
%% Der Schalter \ifkorrekturansicht ist bereits durch den Vorspann gesetzt.

%% latex-abspann.tex
%% Gemeinsamer Abspann für Korrekturansicht und Leseansicht.
%% Setzt den Schalter \ifkorrekturansicht voraus (gesetzt in den
%% einbindenden Dateien latex-korrekturansicht-abspann.tex bzw.
%% latex-leseansicht-abspann.tex).
%% ---------------------------------------------------------------

\normalsize

% Das esempio-Environment wird nur in der Leseansicht benötigt
\ifkorrekturansicht\else
\newenvironment{esempio}[3]%
{
    \vspace{1.5ex}
    \rlap{\underline{#1}}
    \par
    \setlength{\parindent}{0cm}
    \nopagebreak
    \leftskip=#2cm
    \rightskip=#3cm
}
{
    \par
}
\fi

\doendnotes{C}
\bigskip
\vfill

\clearpage

\footnotesize

\ifkorrekturansicht
  \lohead{\textsc{register}}
\fi

% theindex-Environment neu definieren ohne reledmac
\makeatletter
\renewenvironment{theindex}{%
  \ifkorrekturansicht
    \section*{\indexname}%
  \else
    \subsubsection*{Index der erwähnten Entitäten}%
  \fi
  \setlength{\parindent}{0pt}%
  \setlength{\parskip}{0pt plus 0.3pt}%
  \let\item\@idxitem
}{%
  \ifkorrekturansicht\clearpage\fi
}
\makeatother

\IfFileExists{\jobname-pw.ind}{\input{\jobname-pw.ind}}{}

% Quellenangabe nur in der Leseansicht
\ifkorrekturansicht\else
% Fallback-Definitionen, falls die .tex-Datei \titel etc. nicht gesetzt hat
\providecommand{\titel}{}
\providecommand{\editorInnen}{}
\providecommand{\dateiname}{\jobname}

\vspace{3cm}

\vfill

\footnotesize
\textsc{Quelle}: \titel. Herausgegeben von {\editorInnen}. In: \emph{Arthur Schnitzler: Briefwechsel mit Autorinnen und Autoren}.
 Digitale Edition, https://schnitzler-briefe.acdh.oeaw.ac.at/{\dateiname}.html (Stand \today)
\fi

\end{document}


      