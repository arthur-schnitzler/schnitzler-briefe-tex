%% latex-korrekturansicht-vorspann.tex
%% Vorspann für die Korrekturansicht.
%% Lädt die gemeinsame Datei latex-vorspann.tex mit gesetztem Schalter.

\newif\ifkorrekturansicht
\korrekturansichttrue

\input{../tex-inputs/latex-vorspann}


\section[Hugo von Hofmannsthal an Arthur Schnitzler, 19. 3. 1892]{L00085 Hugo von Hofmannsthal an Arthur Schnitzler, 19. 3. 1892}
\nopagebreak\mylabel{L00085v}
\rehead{ }\normalsize\beginnumbering\briefempfaengerindex{Schnitzler, Arthur@\textsc{Schnitzler, Arthur}!zzzHofmannsthal, Hugo von@\emph{von Hugo von Hofmannsthal}!1892-03-191@{19. 3. 1892}|(be}
\toendnotes[C]{\smallbreak\pagebreak[2]}\Standort{CUL, Schnitzler, B 43.}
\physDesc{Postkarte, 640 Zeichen
\newline{}Handschrift: schwarze Tinte, deutsche Kurrent
\newline{}Versand: 1) Rohrpost  2) Stempel: »\nobreak{}\oindex{III., Landstrasse@\textbf{III., Landstraße}, \emph{A.ADM3}|pwk}Wien 3/1 40, 19. 3. 92, 1–2N\nobreak{}«.  3) Stempel: »\nobreak{}\oindex{Kaerntnerring@\textbf{Kärntnerring}, \emph{Straße (K.STR)}|pwk}Wien Kärntnerring, 19. 3. 92, 1–2N\nobreak{}«. 
\newline{}Schnitzler: mit Bleistift datiert: »19/3 92« und nummeriert: »20« }
\buchAbdrucke{\weitereDrucke{Hugo von Hofmannsthal, Arthur Schnitzler: \emph{Briefwechsel}. Frankfurt am Main: \emph{S. Fischer} 1964, S. 18.} }\toendnotes[C]{\smallbreak}\pstart{}{\pb}Herrn \textsc{D\textsuperscript{r} Arthur Schnitzler}\pend{}\pstart{}\textsc{Wien\oindex{Wien@\textbf{Wien}, \emph{A.ADM2}|pw}}\pend{}\pstart{}\textsc{I. Kärnthnerring 12\oindex{Kaerntnerring 12/Boesendorferstrasse 11@\textbf{Kärntnerring 12/Bösendorferstraße 11}, \emph{Wohngebäude (K.WHS)}|pw}}\pend{}\pstart{}\textsc{II Stiege 3 Stock}\pend{}{\bigskip}\vspace{1em}
\pstart{}{\pb}Lieber Freund.\pend\vspace{0.5em}
\pstart
           Das erſtemal ſchreibe ich einen Brief an Sie ängſtlich. Ich muſs nämlich ſehr unartig
               ſein. Verzeihen Sie, bitte. Kainz\pwindex{Kainz, Josef 02.01.1858 – 20.09.1910@\textsc{Kainz, Josef} (02.01.1858 – 20.09.1910), \emph{Schauspieler/Schauspielerin}|pw}, dem ich
               irgend einen Sonntag nach Purkersdorf\oindex{Purkersdorf@\textbf{Purkersdorf}, \emph{A.ADM3}|pw} zu kommen
               verſprochen hatte, reiſt Montag nach Graz\oindex{Graz@\textbf{Graz}, \emph{A.ADM2}|pw}, Prag\oindex{Prag@\textbf{Prag}, \emph{A.ADM1}|pw}, Moskau\oindex{Moskau@\textbf{Moskau}, \emph{A.ADM1}|pw}{ }\textsc{etc}. und will mich abſolut morgen draußen haben. Bitte
               bedenken Sie alſo, daſs Kainz\pwindex{Kainz, Josef 02.01.1858 – 20.09.1910@\textsc{Kainz, Josef} (02.01.1858 – 20.09.1910), \emph{Schauspieler/Schauspielerin}|pw} für mich
               dasſelbe vorſtellt, wie Reicher\pwindex{Reicher, Emanuel 18.06.1849 – 15.05.1924@\textsc{Reicher, Emanuel} (18.06.1849 – 15.05.1924), \emph{Schauspieler/Schauspielerin}|pw} für Sie und
               entſchuldigen Sie dieſen Eingriff der Außendinge in das Unſere. Ich komme vielleicht
                  \label{K_L00085-1v}\edtext{Montag}{\lemma{\textnormal{\emph{Montag}}}\Cendnote{\textnormal{Tatsächlich kam Hofmannsthal\pwindex{Hofmannsthal, Hugo von 1874-02-01 – 1929-07-15@\textsc{Hofmannsthal, Hugo von} (1874-02-01 – 1929-07-15), \emph{Schriftsteller/Schriftstellerin}|pwk} am Montag, dem 21. 3. 1892 vorbei.}}}\label{K_L00085-1} zu Ihnen und wir verabreden gleich
               irgend eine Stunde.\pend
           
\pstart
           Herzlichſt{\\[\baselineskip]}\spacefill\mbox{Loris.}\pend
           \leftskip=0em{}
\pstart
           \noindent{}Bitte auch Salten\pwindex{Salten, Felix 06.09.1869 – 08.10.1945@\textsc{Salten, Felix} (06.09.1869 – 08.10.1945), \emph{Schriftsteller/Schriftstellerin, Journalist/Journalistin, Chefredakteur/Chefredakteurin}|pw} grüßen und
                  entſchuldigen.\pend
           \selectlanguage{ngerman}\endnumbering\briefempfaengerindex{Schnitzler, Arthur@\textsc{Schnitzler, Arthur}!zzzHofmannsthal, Hugo von@\emph{von Hugo von Hofmannsthal}!1892-03-191@{19. 3. 1892}|)be}\mylabel{L00085h}  \normalsize

\doendnotes{C}
\bigskip
\vfill

\clearpage

\footnotesize

\lohead{\textsc{register}}

% Definiere theindex-Environment komplett neu ohne reledmac
\makeatletter
\renewenvironment{theindex}{%
  \section*{\indexname}%
  \setlength{\parindent}{0pt}%
  \setlength{\parskip}{0pt plus 0.3pt}%
  \let\item\@idxitem
}{%
  \clearpage
}
\makeatother

\IfFileExists{\jobname-pw.ind}{\input{\jobname-pw.ind}}{}

\end{document}

      