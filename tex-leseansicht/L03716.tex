%% latex-leseansicht-vorspann.tex
%% Vorspann für die Leseansicht.
%% Lädt die gemeinsame Datei latex-vorspann.tex mit nicht gesetztem Schalter.

\newif\ifkorrekturansicht
\korrekturansichtfalse

\input{../tex-inputs/latex-vorspann}


\section[Elsa Plessner an Arthur Schnitzler, 23. 10. 1897]{L03716 Elsa Plessner an Arthur Schnitzler, 23. 10. 1897}
\nopagebreak\mylabel{L03716v}
\rehead{ }\normalsize\beginnumbering\briefempfaengerindex{Schnitzler, Arthur@\textsc{Schnitzler, Arthur}!zzzPlessner, Elsa@\emph{von Elsa Plessner}!1897-10-231@{23. 10. 1897}|(be}
\toendnotes[C]{\smallbreak\pagebreak[2]}
\correspDesc{Versand  durch Elsa Plessner am 23. 10. 1897 in Wien
\newline{}Erhalt  durch Arthur Schnitzler im Zeitraum [24. 10. 1897 – 28. 10. 1897?] in Wien}\toendnotes[C]{\smallbreak}
\Standort{DLA, A:Schnitzler, HS.1985.1.419.}
\physDesc{Brief, 1 Blatt, 2 Seiten, 1080 Zeichen
\newline{}Handschrift: schwarze Tinte, lateinische Kurrent}\toendnotes[C]{\smallbreak}
\pstart
           \raggedleft{}{\pb}Wien, I. Spiegelgasse N\textsuperscript{o} 2\oindex{Wien@\textbf{Wien}!I., Innere Stadt@\textbf{I., Innere Stadt}!Spiegelgasse 2@\textbf{Spiegelgasse 2}, \emph{Wohngebäude}|pw}, den 23. October 1897.\pend
           
\pstart
           \raggedleft{}Telef. N\textsuperscript{r} 7819\pend
           
\pstart\center{}Verehrter Herr Doctor!\pend\vspace{0.5em}
\pstart
           Nachdem ich Sie einige Zeit in Ruhe gelassen habe, übermittle ich heute wieder einmal
               eine neue kleine Arbeit\pwindex{Plessner, Elsa 22.\,8.\,1875 Wien – 7.\,5.\,1932 Alicante@\textsc{Plessner, Elsa} (22.\,8.\,1875 Wien – 7.\,5.\,1932 Alicante), \emph{Schriftstellerin}!Irmedals Kummer@\strich\emph{Irmedals Kummer}|pwv} Ihrem
               Urtheil.\pend
           
\pstart
           Wenn sie Ihnen gefiele, würde ich mich sehr, sehr freuen. Ich glaube, relativ
               anständig, das heißt ohne stylistische Schlampereien gearbeitet zu haben. Vielleicht
               ist dieser Märchenstyl ein wenig »cliché« allein ich habe ihn mit Vorbedacht benutzt,
               und zwar gerade die gebräuchlichsten Wendungen, blos \strikeout{von} der besseren, satirischen Wirkung halber.
               Natürlich nehme ich das Ding\pwindex{Plessner, Elsa 22.\,8.\,1875 Wien – 7.\,5.\,1932 Alicante@\textsc{Plessner, Elsa} (22.\,8.\,1875 Wien – 7.\,5.\,1932 Alicante), \emph{Schriftstellerin}!Irmedals Kummer@\strich\emph{Irmedals Kummer}|pwv}
               nicht als »\begin{otherlanguage}{french}\label{K_L03716-1v}\edtext{grande chose}{\lemma{\textnormal{\emph{grande chose}}}\Cendnote{\textnormal{französisch: große Sache}}}\label{K_L03716-1}{ }\end{otherlanguage}«, allein ich habe seit \uuline{¾} Jahren meine
               Feder überhaupt nur zu Briefen spazieren geführt. Darum ist mir »Irmedals Kum{\pb}mer\pwindex{Plessner, Elsa 22.\,8.\,1875 Wien – 7.\,5.\,1932 Alicante@\textsc{Plessner, Elsa} (22.\,8.\,1875 Wien – 7.\,5.\,1932 Alicante), \emph{Schriftstellerin}!Irmedals Kummer@\strich\emph{Irmedals Kummer}|pw}« sehr werth.– – –\pend
           
\pstart
           \label{K_L03716-2v}\edtext{Ein neues Stück\pwindex{Plessner, Elsa 22.\,8.\,1875 Wien – 7.\,5.\,1932 Alicante@\textsc{Plessner, Elsa} (22.\,8.\,1875 Wien – 7.\,5.\,1932 Alicante), \emph{Schriftstellerin}!erste Kapitel. Schauspiel in drei Akten@\strich\emph{Das erste Kapitel. Schauspiel in drei Akten}|pwuv}\pwindex{Plessner, Elsa 22.\,8.\,1875 Wien – 7.\,5.\,1932 Alicante@\textsc{Plessner, Elsa} (22.\,8.\,1875 Wien – 7.\,5.\,1932 Alicante), \emph{Schriftstellerin}!Ehrlosen. Schauspiel in drei Acten@\strich\emph{Die Ehrlosen. Schauspiel in drei Acten}|pwuv}}{\lemma{\textnormal{\emph{Ein neues Stück}}}\Cendnote{\textnormal{Von Plessner\pwindex{Plessner, Elsa 22.\,8.\,1875 Wien – 7.\,5.\,1932 Alicante@\textsc{Plessner, Elsa} (22.\,8.\,1875 Wien – 7.\,5.\,1932 Alicante), \emph{Schriftstellerin}|pwk} sind
                  zwei Stücke\pwindex{Plessner, Elsa 22.\,8.\,1875 Wien – 7.\,5.\,1932 Alicante@\textsc{Plessner, Elsa} (22.\,8.\,1875 Wien – 7.\,5.\,1932 Alicante), \emph{Schriftstellerin}!erste Kapitel. Schauspiel in drei Akten@\strich\emph{Das erste Kapitel. Schauspiel in drei Akten}|pwkv}\pwindex{Plessner, Elsa 22.\,8.\,1875 Wien – 7.\,5.\,1932 Alicante@\textsc{Plessner, Elsa} (22.\,8.\,1875 Wien – 7.\,5.\,1932 Alicante), \emph{Schriftstellerin}!Ehrlosen. Schauspiel in drei Acten@\strich\emph{Die Ehrlosen. Schauspiel in drei Acten}|pwkv} überliefert, wobei sie zuerst \emph{Die Ehrlosen}\pwindex{Plessner, Elsa 22.\,8.\,1875 Wien – 7.\,5.\,1932 Alicante@\textsc{Plessner, Elsa} (22.\,8.\,1875 Wien – 7.\,5.\,1932 Alicante), \emph{Schriftstellerin}!Ehrlosen. Schauspiel in drei Acten@\strich\emph{Die Ehrlosen. Schauspiel in drei Acten}|pwk} ausgearbeitet 
                  hat. Dessen Entstehung datiert sie im Brief vom XXXX Auszeichnungsfehler: Dokument L03720 nicht gefunden allerdings auf den Herbst 1898, so dass
               es sich um einen nicht realisierten Plan handeln dürfte.}}}\label{K_L03716-2} liegt auf der Pfanne. Ende Dezember dürften Sie davon
               ereilt werden, Sie, verehrter Herr, der Sie so liebenswürdig der Puffer meines
               künstlerischen Zuges sind. – – Wenn es mir endlich einmal was werden möchte. Weiß
               wirklich nicht, wie es ausfallen wird.\pend
           
\pstart
           Abwarten! –\pend
           
\pstart
           Viele, viele Grüße in aufrichtiger, waschechter Verehrung{\\[\baselineskip]}\spacefill\mbox{Elsa Plessner.}\pend
           \leftskip=0em{}\selectlanguage{ngerman}\endnumbering\briefempfaengerindex{Schnitzler, Arthur@\textsc{Schnitzler, Arthur}!zzzPlessner, Elsa@\emph{von Elsa Plessner}!1897-10-231@{23. 10. 1897}|)be}\mylabel{L03716h}  \newcommand{\dateiname}{L03716}\newcommand{\titel}{Elsa Plessner an Arthur Schnitzler, 23. 10. 1897}\newcommand{\editorInnen}{Selma Jahnke und Martin Anton Müller}%% latex-leseansicht-abspann.tex
%% Abspann für die Leseansicht.
%% Der Schalter \ifkorrekturansicht ist bereits durch den Vorspann gesetzt.

%% latex-abspann.tex
%% Gemeinsamer Abspann für Korrekturansicht und Leseansicht.
%% Setzt den Schalter \ifkorrekturansicht voraus (gesetzt in den
%% einbindenden Dateien latex-korrekturansicht-abspann.tex bzw.
%% latex-leseansicht-abspann.tex).
%% ---------------------------------------------------------------

\normalsize

% Das esempio-Environment wird nur in der Leseansicht benötigt
\ifkorrekturansicht\else
\newenvironment{esempio}[3]%
{
    \vspace{1.5ex}
    \rlap{\underline{#1}}
    \par
    \setlength{\parindent}{0cm}
    \nopagebreak
    \leftskip=#2cm
    \rightskip=#3cm
}
{
    \par
}
\fi

\doendnotes{C}
\bigskip
\vfill

\clearpage

\footnotesize

\ifkorrekturansicht
  \lohead{\textsc{register}}
\fi

% theindex-Environment neu definieren ohne reledmac
\makeatletter
\renewenvironment{theindex}{%
  \ifkorrekturansicht
    \section*{\indexname}%
  \else
    \subsubsection*{Index der erwähnten Entitäten}%
  \fi
  \setlength{\parindent}{0pt}%
  \setlength{\parskip}{0pt plus 0.3pt}%
  \let\item\@idxitem
}{%
  \ifkorrekturansicht\clearpage\fi
}
\makeatother

\IfFileExists{\jobname-pw.ind}{\input{\jobname-pw.ind}}{}

% Quellenangabe nur in der Leseansicht
\ifkorrekturansicht\else
% Fallback-Definitionen, falls die .tex-Datei \titel etc. nicht gesetzt hat
\providecommand{\titel}{}
\providecommand{\editorInnen}{}
\providecommand{\dateiname}{\jobname}

\vspace{3cm}

\vfill

\footnotesize
\textsc{Quelle}: \titel. Herausgegeben von {\editorInnen}. In: \emph{Arthur Schnitzler: Briefwechsel mit Autorinnen und Autoren}.
 Digitale Edition, https://schnitzler-briefe.acdh.oeaw.ac.at/{\dateiname}.html (Stand \today)
\fi

\end{document}


