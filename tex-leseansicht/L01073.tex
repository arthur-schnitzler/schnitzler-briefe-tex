%% latex-leseansicht-vorspann.tex
%% Vorspann für die Leseansicht.
%% Lädt die gemeinsame Datei latex-vorspann.tex mit nicht gesetztem Schalter.

\newif\ifkorrekturansicht
\korrekturansichtfalse

\input{../tex-inputs/latex-vorspann}


         
         \newcommand{\erwaehntePersonen}{Personen: Hermann Bahr, Julius Bauer, Jakob Julius David, Alfred Dreyfus, Paul Goldmann, Robert Hirschfeld, Auguste Mercier, Felix Salten, Paul Schlenther, Franz Spehar, Ludwig Speidel}
         \newcommand{\erwaehnteOrte}{Orte: Altaussee, Wien}
         \newcommand{\erwaehnteWerke}{Werke: Der Schleier der Beatrice. Schauspiel in fünf Akten, Erklärung [Schleier der Beatrice]}
               \section[Richard Beer-Hofmann an Arthur Schnitzler, 14. 9. 1900]{ Richard Beer-Hofmann an Arthur Schnitzler, 14. 9. 1900}\nopagebreak\mylabel{v}\rehead{ }\begin{ledgroupsized}[t]{13cm}\normalsize\beginnumbering \toendnotes[C]{\smallbreak\pagebreak[2]} \Standort{CUL, Schnitzler, B 8.}
\physDesc{Telegramm
\newline{}maschinell\newline{}Versand: »\textcolor{gray}{\textbf{{[}Aufgenom{]}men durch}}{ }\textcolor{gray}{\textbf{\textit{/9 F. Spehar\pwindex{Spehar, Franz @\textsc{Spehar, Franz}, \emph{Telegrafenbeamter}|pw}}}}« 
\newline{}Schnitzler: mit Bleistift datiert: »14/9 90« \newline{}Ordnung: 1) beschnitten  2) mit Bleistift von unbekannter Hand nummeriert:
                              »159«}\buchAbdrucke{\weitereDrucke{Arthur Schnitzler, Richard Beer-Hofmann: \emph{Briefwechsel 1891–1931}. Hg. Konstanze Fliedl. Wien, Zürich: \emph{Europaverlag} 1992, S. 151.} }\toendnotes[C]{\smallbreak}\pstart
           \noindent{}{\pb}+ fr altauszee\oindex{Altaussee@\textbf{Altaussee}|pw} 478 30 14{ }7 15 m.–\pend
           \pstart
           komme hoffentlich heute vier uhr nachmittag an moechte dasz sye und paul\pwindex{Goldmann, Paul 31.01.1865 – 25.09.1935@\textsc{Goldmann, Paul} (31.01.1865 – 25.09.1935), \emph{Schriftsteller, Journalist}|pw} mich um halb sechs abholen.
               erfahre soeben die \label{K_L01073_1v}\edtext{mercier\pwindex{Mercier, Auguste 8.12.1833 – 3.3.1921@\textsc{Mercier, Auguste} (8.12.1833 – 3.3.1921), \emph{Politiker}|pw}tat des seehundes\pwindex{Schlenther, Paul 20.08.1854 – 30.04.1916@\textsc{Schlenther, Paul} (20.08.1854 – 30.04.1916), \emph{Schriftsteller, Kritiker, Theaterleiter}|pwv}}{\lemma{\textnormal{\emph{merciertat des seehundes}}}\Cendnote{\textnormal{Paul Schlenther\pwindex{Schlenther, Paul 20.08.1854 – 30.04.1916@\textsc{Schlenther, Paul} (20.08.1854 – 30.04.1916), \emph{Schriftsteller, Kritiker, Theaterleiter}|pwk} hatte nach anfänglichen
                  Zusagen die Aufführung von \emph{Der Schleier der
                     Beatrice}\pwindex{Schnitzler, Arthur 15.05.1862 – 21.10.1931@\textsc{Schnitzler, Arthur} (15.05.1862 – 21.10.1931), \emph{Schriftsteller, Mediziner}!Schleier der Beatrice. Schauspiel in fuenf Akten1900-12-01@\strich\emph{Der Schleier der Beatrice. Schauspiel in fünf Akten} {[}1900-12-01{]}|pwk} doch abgelehnt. Am 14. 9. 1900 druckten mehrere
                  Zeitungen eine \emph{Erklärung}\pwindex{Bahr, Hermann 19.07.1863 – 15.01.1934@\textsc{Bahr, Hermann} (19.07.1863 – 15.01.1934), \emph{Schriftsteller, Kritiker}!Erklaerung [Schleier der Beatrice]1900-09-14@\strich\emph{Erklärung [Schleier der Beatrice]} {[}1900-09-14{]}|pwk}\pwindex{Salten, Felix 06.09.1869 – 08.10.1945@\textsc{Salten, Felix} (06.09.1869 – 08.10.1945), \emph{Schriftsteller, Journalist}!Erklaerung [Schleier der Beatrice]1900-09-14@\strich\emph{Erklärung [Schleier der Beatrice]} {[}1900-09-14{]}|pwk}\pwindex{Bauer, Julius 15.10.1853 – 11.06.1941@\textsc{Bauer, Julius} (15.10.1853 – 11.06.1941), \emph{Schriftsteller, Journalist, Kritiker}!Erklaerung [Schleier der Beatrice]1900-09-14@\strich\emph{Erklärung [Schleier der Beatrice]} {[}1900-09-14{]}|pwk}\pwindex{Hirschfeld, Robert 17.09.1857 – 02.04.1914@\textsc{Hirschfeld, Robert} (17.09.1857 – 02.04.1914), \emph{Journalist, Musikkritiker}!Erklaerung [Schleier der Beatrice]1900-09-14@\strich\emph{Erklärung [Schleier der Beatrice]} {[}1900-09-14{]}|pwk}\pwindex{Speidel, Ludwig 1830-04-11 – 1906-02-03@\textsc{Speidel, Ludwig} (1830-04-11 – 1906-02-03), \emph{Journalist, Kritiker}!Erklaerung [Schleier der Beatrice]1900-09-14@\strich\emph{Erklärung [Schleier der Beatrice]} {[}1900-09-14{]}|pwk}\pwindex{David, Jakob Julius 1859-02-06 – 1906-11-20@\textsc{David, Jakob Julius} (1859-02-06 – 1906-11-20), \emph{Schriftsteller, Journalist}!Erklaerung [Schleier der Beatrice]1900-09-14@\strich\emph{Erklärung [Schleier der Beatrice]} {[}1900-09-14{]}|pwk} – ein heftiger Protest
                  von Hermann Bahr\pwindex{Bahr, Hermann 19.07.1863 – 15.01.1934@\textsc{Bahr, Hermann} (19.07.1863 – 15.01.1934), \emph{Schriftsteller, Kritiker}|pwk}, Julius Bauer\pwindex{Bauer, Julius 15.10.1853 – 11.06.1941@\textsc{Bauer, Julius} (15.10.1853 – 11.06.1941), \emph{Schriftsteller, Journalist, Kritiker}|pwk}, Jakob Julius
                     David\pwindex{David, Jakob Julius 1859-02-06 – 1906-11-20@\textsc{David, Jakob Julius} (1859-02-06 – 1906-11-20), \emph{Schriftsteller, Journalist}|pwk}, Robert Hirschfeld\pwindex{Hirschfeld, Robert 17.09.1857 – 02.04.1914@\textsc{Hirschfeld, Robert} (17.09.1857 – 02.04.1914), \emph{Journalist, Musikkritiker}|pwk}, Felix Salten\pwindex{Salten, Felix 06.09.1869 – 08.10.1945@\textsc{Salten, Felix} (06.09.1869 – 08.10.1945), \emph{Schriftsteller, Journalist}|pwk} und Ludwig Speidel\pwindex{Speidel, Ludwig 1830-04-11 – 1906-02-03@\textsc{Speidel, Ludwig} (1830-04-11 – 1906-02-03), \emph{Journalist, Kritiker}|pwk} gegen die Vorgehensweise. Beer-Hofmann\pwindex{Beer-Hofmann, Richard 1866-07-11 – 1945-09-26@\textsc{Beer-Hofmann, Richard} (1866-07-11 – 1945-09-26), \emph{Schriftsteller}|pwk} stellt mit der Bezugnahme auf den Kriegsminister Auguste Mercier\pwindex{Mercier, Auguste 8.12.1833 – 3.3.1921@\textsc{Mercier, Auguste} (8.12.1833 – 3.3.1921), \emph{Politiker}|pwk} eine Verbindung zum
                  antisemitisch motivierten Dreyfus\pwindex{Dreyfus, Alfred 1859-10-09 – 1935-07-12@\textsc{Dreyfus, Alfred} (1859-10-09 – 1935-07-12), \emph{Militär}|pwk}prozess
                  her.}}}\label{K_L01073_1h} herzlychst \spacefill\mbox{= richard .+}\pend
           
         
         \endnumbering\mylabel{h}\end{ledgroupsized}  \newcommand{\dateiname}{L01073}\newcommand{\titel}{Richard Beer-Hofmann an Arthur Schnitzler, 14. 9. 1900}\newcommand{\editorInnen}{Martin Anton Müller und Gerd-Hermann Susen}%% latex-leseansicht-abspann.tex
%% Abspann für die Leseansicht.
%% Der Schalter \ifkorrekturansicht ist bereits durch den Vorspann gesetzt.

%% latex-abspann.tex
%% Gemeinsamer Abspann für Korrekturansicht und Leseansicht.
%% Setzt den Schalter \ifkorrekturansicht voraus (gesetzt in den
%% einbindenden Dateien latex-korrekturansicht-abspann.tex bzw.
%% latex-leseansicht-abspann.tex).
%% ---------------------------------------------------------------

\normalsize

% Das esempio-Environment wird nur in der Leseansicht benötigt
\ifkorrekturansicht\else
\newenvironment{esempio}[3]%
{
    \vspace{1.5ex}
    \rlap{\underline{#1}}
    \par
    \setlength{\parindent}{0cm}
    \nopagebreak
    \leftskip=#2cm
    \rightskip=#3cm
}
{
    \par
}
\fi

\doendnotes{C}
\bigskip
\vfill

\clearpage

\footnotesize

\ifkorrekturansicht
  \lohead{\textsc{register}}
\fi

% theindex-Environment neu definieren ohne reledmac
\makeatletter
\renewenvironment{theindex}{%
  \ifkorrekturansicht
    \section*{\indexname}%
  \else
    \subsubsection*{Index der erwähnten Entitäten}%
  \fi
  \setlength{\parindent}{0pt}%
  \setlength{\parskip}{0pt plus 0.3pt}%
  \let\item\@idxitem
}{%
  \ifkorrekturansicht\clearpage\fi
}
\makeatother

\IfFileExists{\jobname-pw.ind}{\input{\jobname-pw.ind}}{}

% Quellenangabe nur in der Leseansicht
\ifkorrekturansicht\else
% Fallback-Definitionen, falls die .tex-Datei \titel etc. nicht gesetzt hat
\providecommand{\titel}{}
\providecommand{\editorInnen}{}
\providecommand{\dateiname}{\jobname}

\vspace{3cm}

\vfill

\footnotesize
\textsc{Quelle}: \titel. Herausgegeben von {\editorInnen}. In: \emph{Arthur Schnitzler: Briefwechsel mit Autorinnen und Autoren}.
 Digitale Edition, https://schnitzler-briefe.acdh.oeaw.ac.at/{\dateiname}.html (Stand \today)
\fi

\end{document}


      