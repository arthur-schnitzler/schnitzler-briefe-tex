\input{../tex-inputs/latex-pdf-vorspann}
\begin{center}
            \textcolor{red}{ENTWURF. ENTZIFFERUNG NOCH NICHT KORREKTURGELESEN}
                      \end{center}
            
               \section[Richard Beer-Hofmann an Arthur Schnitzler, 14. 9. 1900]{ Richard Beer-Hofmann an Arthur Schnitzler, 14. 9. 1900}\nopagebreak\mylabel{v}\rehead{ }\begin{ledgroupsized}[t]{13cm}\normalsize\beginnumbering\briefempfaengerindex{Schnitzler, Arthur@\textsc{Schnitzler, Arthur}!zzzBeer-Hofmann, Richard@\emph{von Richard Beer-Hofmann}!1900-09-141@{14. 9. 1900}|(be} \toendnotes[C]{\smallbreak\pagebreak[2]} \Standort{CUL, Schnitzler, B 8.}
\physDesc{Telegramm
\newline{}maschinell\newline{}Versand: »\textcolor{gray}{\textbf{{[}Aufgenom{]}men durch}}{ }\textcolor{gray}{\textbf{\textit{/9 F. Spehar\pwindex{Spehar, Franz @\textsc{Spehar, Franz}, \emph{Telegrafenbeamter}|pw}}}}« 
\newline{}Schnitzler: mit Bleistift datiert: »14/9 90« \newline{}Ordnung: 1) beschnitten 2) mit Bleistift von unbekannter Hand nummeriert:
                              »159«}\buchAbdrucke{\weitereDrucke{Arthur Schnitzler, Richard Beer-Hofmann: \emph{Briefwechsel 1891–1931}. Hg. Konstanze Fliedl. Wien, Zürich: \emph{Europaverlag} 1992, S. 151.} }\toendnotes[C]{\smallbreak}\pstart
           \noindent{}{\pb}+ fr altauszee\oindex{Altaussee@\textbf{Altaussee}|pw} 478 30 14{ }7 15 m.–\pend
           \pstart
           komme hoffentlich heute vier uhr nachmittag an moechte dasz sye und paul\pwindex{Goldmann, Paul 31.01.1865 – 25.09.1935@\textsc{Goldmann, Paul} (31.01.1865 – 25.09.1935), \emph{Schriftsteller, Journalist}|pw} mich um halb sechs abholen.
               erfahre soeben die \label{K_L01073_1v}\edtext{mercier\pwindex{Mercier, Auguste 8.12.1833 – 3.3.1921@\textsc{Mercier, Auguste} (8.12.1833 – 3.3.1921), \emph{Politiker}|pw}tat des seehundes\pwindex{Schlenther, Paul 20.08.1854 – 30.04.1916@\textsc{Schlenther, Paul} (20.08.1854 – 30.04.1916), \emph{Schriftsteller, Kritiker, Theaterleiter}|pwv}}{\lemma{\textnormal{\emph{merciertat des seehundes}}}\Cendnote{\textnormal{Paul Schlenther\pwindex{Schlenther, Paul 20.08.1854 – 30.04.1916@\textsc{Schlenther, Paul} (20.08.1854 – 30.04.1916), \emph{Schriftsteller, Kritiker, Theaterleiter}|pwk} hatte nach anfänglichen
                  Zusagen die Aufführung von \emph{Der Schleier der
                     Beatrice}\pwindex{Schnitzler, Arthur 15.05.1862 – 21.10.1931@\textsc{Schnitzler, Arthur} (15.05.1862 – 21.10.1931), \emph{Schriftsteller, Mediziner}!Schleier der Beatrice. Schauspiel in fuenf Akten1900-12-01 – 1900-12-01@\strich\emph{Der Schleier der Beatrice. Schauspiel in fünf Akten} {[}1900-12-01 – 1900-12-01{]}|pwk} doch abgelehnt. Am 14. 9. 1900 druckten mehrere
                  Zeitungen eine \emph{Erklärung}\pwindex{Bahr, Hermann 19.07.1863 – 15.01.1934@\textsc{Bahr, Hermann} (19.07.1863 – 15.01.1934), \emph{Schriftsteller, Kritiker}!Erklaerung14. 09. 1900@\strich\emph{Erklärung} {[}14. 09. 1900{]}|pwk}\pwindex{Salten, Felix 06.09.1869 – 08.10.1945@\textsc{Salten, Felix} (06.09.1869 – 08.10.1945), \emph{Schriftsteller, Journalist}!Erklaerung14. 09. 1900@\strich\emph{Erklärung} {[}14. 09. 1900{]}|pwk}\pwindex{Bauer, Julius 15.10.1853 – 11.06.1941@\textsc{Bauer, Julius} (15.10.1853 – 11.06.1941), \emph{Schriftsteller, Journalist, Kritiker}!Erklaerung14. 09. 1900@\strich\emph{Erklärung} {[}14. 09. 1900{]}|pwk}\pwindex{Hirschfeld, Robert 17.09.1857 – 02.04.1914@\textsc{Hirschfeld, Robert} (17.09.1857 – 02.04.1914), \emph{Journalist, Musikkritiker}!Erklaerung14. 09. 1900@\strich\emph{Erklärung} {[}14. 09. 1900{]}|pwk}\pwindex{Speidel, Ludwig 11.04.1830 – 03.02.1906@\textsc{Speidel, Ludwig} (11.04.1830 – 03.02.1906), \emph{Journalist, Kritiker}!Erklaerung14. 09. 1900@\strich\emph{Erklärung} {[}14. 09. 1900{]}|pwk}\pwindex{David, Jakob Julius 06.02.1859 – 20.11.1906@\textsc{David, Jakob Julius} (06.02.1859 – 20.11.1906), \emph{Schriftsteller, Journalist}!Erklaerung14. 09. 1900@\strich\emph{Erklärung} {[}14. 09. 1900{]}|pwk} – ein heftiger Protest
                  von Hermann Bahr\pwindex{Bahr, Hermann 19.07.1863 – 15.01.1934@\textsc{Bahr, Hermann} (19.07.1863 – 15.01.1934), \emph{Schriftsteller, Kritiker}|pwk}, Julius Bauer\pwindex{Bauer, Julius 15.10.1853 – 11.06.1941@\textsc{Bauer, Julius} (15.10.1853 – 11.06.1941), \emph{Schriftsteller, Journalist, Kritiker}|pwk}, Jakob Julius
                     David\pwindex{David, Jakob Julius 06.02.1859 – 20.11.1906@\textsc{David, Jakob Julius} (06.02.1859 – 20.11.1906), \emph{Schriftsteller, Journalist}|pwk}, Robert Hirschfeld\pwindex{Hirschfeld, Robert 17.09.1857 – 02.04.1914@\textsc{Hirschfeld, Robert} (17.09.1857 – 02.04.1914), \emph{Journalist, Musikkritiker}|pwk}, Felix Salten\pwindex{Salten, Felix 06.09.1869 – 08.10.1945@\textsc{Salten, Felix} (06.09.1869 – 08.10.1945), \emph{Schriftsteller, Journalist}|pwk} und Ludwig Speidel\pwindex{Speidel, Ludwig 11.04.1830 – 03.02.1906@\textsc{Speidel, Ludwig} (11.04.1830 – 03.02.1906), \emph{Journalist, Kritiker}|pwk} gegen die Vorgehensweise. Beer-Hofmann\pwindex{Beer-Hofmann, Richard 11.07.1866 – 26.09.1945@\textsc{Beer-Hofmann, Richard} (11.07.1866 – 26.09.1945), \emph{Schriftsteller}|pwk} stellt mit der Bezugnahme auf den Kriegsminister Auguste Mercier\pwindex{Mercier, Auguste 8.12.1833 – 3.3.1921@\textsc{Mercier, Auguste} (8.12.1833 – 3.3.1921), \emph{Politiker}|pwk} eine Verbindung zum
                  antisemitisch motivierten Dreyfus\pwindex{Dreyfus, Alfred 09.10.1859 – 12.07.1935@\textsc{Dreyfus, Alfred} (09.10.1859 – 12.07.1935), \emph{Militär}|pwk}prozess
                  her.}}}\label{K_L01073_1h} herzlychst \spacefill\mbox{= richard .+}\pend
           \endnumbering\briefempfaengerindex{Schnitzler, Arthur@\textsc{Schnitzler, Arthur}!zzzBeer-Hofmann, Richard@\emph{von Richard Beer-Hofmann}!1900-09-141@{14. 9. 1900}|)be}\mylabel{h}\end{ledgroupsized}  \newcommand{\dateiname}{L01073}\newcommand{\titel}{Richard Beer-Hofmann an Arthur Schnitzler, 14. 9. 1900}\newcommand{\editorInnen}{Martin Anton Müller und Gerd-Hermann Susen}\input{../tex-inputs/latex-pdf-abspann}
      