%% latex-korrekturansicht-vorspann.tex
%% Vorspann für die Korrekturansicht.
%% Lädt die gemeinsame Datei latex-vorspann.tex mit gesetztem Schalter.

\newif\ifkorrekturansicht
\korrekturansichttrue

\input{../tex-inputs/latex-vorspann}


\section[Olga Schnitzler an Richard und Paula Beer-Hofmann, {[}15. 12. 1909?{]}]{L02560 Olga Schnitzler an Richard und Paula Beer-Hofmann,
               {[}15. 12. 1909?{]}}
\nopagebreak\mylabel{L02560v}
\rehead{ }\normalsize\beginnumbering\briefempfaengerindex{Beer-Hofmann, Paula@\textsc{Beer-Hofmann, Paula}!zzzSchnitzler, Olga@\emph{von Olga Schnitzler}!1909-12-151@{{[}15. 12. 1909{]}?}|(be}\briefempfaengerindex{Beer-Hofmann, Richard@\textsc{Beer-Hofmann, Richard}!zzzSchnitzler, Olga@\emph{von Olga Schnitzler}!1909-12-151@{{[}15. 12. 1909{]}?}|(be}
\toendnotes[C]{\smallbreak\pagebreak[2]}\Standort{YCGL, MSS 31.}
\physDesc{Brief, 1 Blatt, 1 Seite, 278 Zeichen
\newline{}Handschrift: schwarze Tinte, lateinische Kurrent}\toendnotes[C]{\smallbreak}
\pstart
           \noindent{}{\pb}Meine Lieben,{ }Arth. lässt Euch
               um einen Gefallen bitten: Andrian\pwindex{Andrian-Werburg, Leopold von 09.05.1875 – 19.11.1951@\textsc{Andrian-Werburg, Leopold von} (09.05.1875 – 19.11.1951), \emph{Schriftsteller/Schriftstellerin, Diplomat/Diplomatin}|pw} kommt
                  \label{K_L02560-1v}\edtext{heut{ }Abend}{\lemma{\textnormal{\emph{heut Abend}}}\Cendnote{\textnormal{Das Korrespondenzstück ist undatiert.
                  Die Datierung folgt der Annahme, dass es sich um das im \emph{Tagebuch}\pwindex{Tagebuch@\emph{Tagebuch}|pwk} vom 15. 12. 1909 erwähnte Treffen handelt. Beer-Hofmann\pwindex{Beer-Hofmann, Richard 1866-07-11 – 1945-09-26@\textsc{Beer-Hofmann, Richard} (1866-07-11 – 1945-09-26), \emph{Schriftsteller/Schriftstellerin}|pwk} wäre demnach nicht gekommen. Da
                  das Korrespondenzstück im Nachlass im Ordner 1909 abgelegt ist, wird
                  es dem Tagebuch folgend datiert. Möglich wäre aber auch der 3. 4. 1910; auch in
                  diesem Fall war Beer-Hofmann\pwindex{Beer-Hofmann, Richard 1866-07-11 – 1945-09-26@\textsc{Beer-Hofmann, Richard} (1866-07-11 – 1945-09-26), \emph{Schriftsteller/Schriftstellerin}|pwk} nicht bei Schnitzler zum Abendessen. Es ist mit den
                  derzeit zu überblickenden Korrespondenzstücken nicht möglich, eine definitive
                  Datierung vorzunehmen.}}}\label{K_L02560-1} zum Nachtmal, wir bitten Euch, auch zu kommen, oder
               falls die Paula\pwindex{Beer-Hofmann, Paula 25.02.1879 – 30.10.1939@\textsc{Beer-Hofmann, Paula} (25.02.1879 – 30.10.1939)|pw} zu müde ist, so bitten wir
               Sie, lieber Herr Doctor, ev. nach dem Nachtmal. Aber am schönsten wär’s, Ihr kämet
               beide. Herzl. Grüsse,\pend
           \pstart \spacefill\mbox{Olga.}\pend{}\selectlanguage{ngerman}\endnumbering\briefempfaengerindex{Beer-Hofmann, Paula@\textsc{Beer-Hofmann, Paula}!zzzSchnitzler, Olga@\emph{von Olga Schnitzler}!1909-12-151@{{[}15. 12. 1909{]}?}|)be}\briefempfaengerindex{Beer-Hofmann, Richard@\textsc{Beer-Hofmann, Richard}!zzzSchnitzler, Olga@\emph{von Olga Schnitzler}!1909-12-151@{{[}15. 12. 1909{]}?}|)be}\mylabel{L02560h}  \normalsize

\doendnotes{C}
\bigskip
\vfill

\clearpage

\footnotesize

\lohead{\textsc{register}}

% Definiere theindex-Environment komplett neu ohne reledmac
\makeatletter
\renewenvironment{theindex}{%
  \section*{\indexname}%
  \setlength{\parindent}{0pt}%
  \setlength{\parskip}{0pt plus 0.3pt}%
  \let\item\@idxitem
}{%
  \clearpage
}
\makeatother

\IfFileExists{\jobname-pw.ind}{\input{\jobname-pw.ind}}{}

\end{document}

      