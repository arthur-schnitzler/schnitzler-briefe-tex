%% latex-korrekturansicht-vorspann.tex
%% Vorspann für die Korrekturansicht.
%% Lädt die gemeinsame Datei latex-vorspann.tex mit gesetztem Schalter.

\newif\ifkorrekturansicht
\korrekturansichttrue

\input{../tex-inputs/latex-vorspann}


\section[Elsa Plessner an Arthur Schnitzler, 18. 3. 1897]{L03732 Elsa Plessner an Arthur Schnitzler, 18. 3. 1897}
\nopagebreak\mylabel{L03732v}
\rehead{ }\normalsize\beginnumbering\briefempfaengerindex{Schnitzler, Arthur@\textsc{Schnitzler, Arthur}!zzzPlessner, Elsa@\emph{von Elsa Plessner}!1897-03-182@{18. 3. 1897}|(be}
\toendnotes[C]{\smallbreak\pagebreak[2]}\Standort{DLA, A:Schnitzler, HS.1985.1.419.}
\physDesc{Brief,  Blätter, 2 Seiten, 1183 Zeichen
\newline{}Handschrift: , lateinische Kurrent}\toendnotes[C]{\smallbreak}
\pstart
           {\pb}\textcolor{gray}{\textbf{HOTEL KAISERHOF\oindex{Hotel Meranerhof@\textbf{Hotel Meranerhof}, \emph{Hotel (K.HTL)}|pw}}}{\\}\textcolor{gray}{\textbf{A. Ellmenreich\pwindex{Ellmenreich, Alexander 1854-03-12 – 1911-04-30@\textsc{Ellmenreich, Alexander} (1854-03-12 – 1911-04-30), \emph{Hotelier/Hotelière, Hotelbesitzer/Hotelbesitzerin}|pw}.}}\hfill \textcolor{gray}{\textbf{Meran\oindex{Meran@\textbf{Meran}, \emph{P.PPLA3}|pw}}}, 18. III. \textcolor{gray}{\textbf{189}}7\pend
           
\pstart{}Hochverehrter Herr Doctor!\pend\vspace{0.5em}
\pstart
           Nach langer Pause erlaube ich mir heute wieder einmal, Ihre liebenswürdige
               Aufmerksamkeit für eine \label{K_L03732-1v}\edtext{kleine Arbeit\pwindex{glaeserne Kaefig. Eine Parabel@\emph{Der gläserne Käfig. Eine Parabel}|pwv}}{\lemma{\textnormal{\emph{kleine Arbeit}}}\Cendnote{\textnormal{Die Beilage, ein Manuskript der Erzählung \emph{Der gläserne Käfig}\pwindex{glaeserne Kaefig. Eine Parabel@\emph{Der gläserne Käfig. Eine Parabel}|pwk}, ist nicht überliefert.}}}\label{K_L03732-1} zu
               erbitten. – Bitte – was halten Sie davon?? – –\pend
           
\pstart
           Zugleich bitte ich Sie, mir meinen letzten \label{K_L03732-2v}\edtext{Brief}{\lemma{\textnormal{\emph{Brief}}}\Cendnote{\textnormal{Elsa Plessner an Arthur Schnitzler, 13. 1. 1897. }}}\label{K_L03732-2} – aus dem
                  Januar - nicht übel auszulegen. Es that mir krapp nach seiner
               Absendung \uline{schrecklich leid}, ihn geschrieben zu haben.
               Was müssen Sie von diesen unverlangten und ziemlich verworrenen Confidenzen gedacht
               haben!!!! Dieser Gedanke hat mir einige fatale Stunden bereitet!! – Nun – ich habe
               die Lehre daraus gezogen, nie wieder so umgehend und – – unüberlegt einen \label{K_L03732-3v}\edtext{Brief}{\lemma{\textnormal{\emph{Brief}}}\Cendnote{\textnormal{Schnitzlers Brief ist nicht
                  überliefert.}}}\label{K_L03732-3} zu beant{\pb}worten, da man sich sonst zu sehr von
               seiner Stimmung hinreisten lässt. – – –\pend
           
\pstart
            Also bitte – – – vergessen Sie das \label{K_L03732-4v}\edtext{Ungethüm}{\lemma{\textnormal{\emph{Ungethüm}}}\Cendnote{\textnormal{Elsa Plessner an Arthur Schnitzler, 13. 1. 1897. }}}\label{K_L03732-4}. – Es scheint Sie
               übrigens stark verstimmt zu haben, da keine Antwort – – Sie hatten damit Recht.
               Pardon! – – – – – – –\pend
           
\pstart
           Ihr geschätzte Kritik des »gläsernen Käfigs\pwindex{glaeserne Kaefig. Eine Parabel@\emph{Der gläserne Käfig. Eine Parabel}|pw}«,
               erbitte an notirte Wiener\oindex{Wien@\textbf{Wien}, \emph{A.ADM2}|pw} Adresse, da
                  Sonntag dort eintreffe. Die Arbeit\pwindex{glaeserne Kaefig. Eine Parabel@\emph{Der gläserne Käfig. Eine Parabel}|pwv} ist aus dem Januar datirt und lag so
               lange im Pult, da ich kein Vertrauen dazu hatte – – – und habe – Ganz ehrlich!! – Na
               – sie werden richten. Tausend Dank im Voraus – – !\pend
           
\pstart
           Mit alter und neuer Verehrung{\\[\baselineskip]}\spacefill\mbox{Elsa Plessner}\pend
           \leftskip=0em{}\selectlanguage{ngerman}\endnumbering\briefempfaengerindex{Schnitzler, Arthur@\textsc{Schnitzler, Arthur}!zzzPlessner, Elsa@\emph{von Elsa Plessner}!1897-03-182@{18. 3. 1897}|)be}\mylabel{L03732h}
\begin{anhang}
\end{anhang}\normalsize

\doendnotes{C}
\bigskip
\vfill

\clearpage

\footnotesize

\lohead{\textsc{register}}

% Definiere theindex-Environment komplett neu ohne reledmac
\makeatletter
\renewenvironment{theindex}{%
  \section*{\indexname}%
  \setlength{\parindent}{0pt}%
  \setlength{\parskip}{0pt plus 0.3pt}%
  \let\item\@idxitem
}{%
  \clearpage
}
\makeatother

\IfFileExists{\jobname-pw.ind}{\input{\jobname-pw.ind}}{}

\end{document}

      