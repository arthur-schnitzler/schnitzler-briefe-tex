%% latex-leseansicht-vorspann.tex
%% Vorspann für die Leseansicht.
%% Lädt die gemeinsame Datei latex-vorspann.tex mit nicht gesetztem Schalter.

\newif\ifkorrekturansicht
\korrekturansichtfalse

\input{../tex-inputs/latex-vorspann}


\section[Elsa Plessner an Arthur Schnitzler, 18. 3. 1897]{L03732 Elsa Plessner an Arthur Schnitzler, 18. 3. 1897}
\nopagebreak\mylabel{L03732v}
\rehead{ }\normalsize\beginnumbering\briefempfaengerindex{Schnitzler, Arthur@\textsc{Schnitzler, Arthur}!zzzPlessner, Elsa@\emph{von Elsa Plessner}!1897-03-182@{18. 3. 1897}|(be}
\toendnotes[C]{\smallbreak\pagebreak[2]}
\correspDesc{Versand  durch Elsa Plessner am 18. 3. 1897 in Meran
\newline{}Erhalt  durch Arthur Schnitzler im Zeitraum [19. 3. 1897
                  – 23. 3. 1897?] in Wien}\toendnotes[C]{\smallbreak}
\Standort{DLA, A:Schnitzler, HS.1985.1.419.}
\physDesc{Brief,  Blätter, 2 Seiten, 1183 Zeichen
\newline{}Handschrift: lila Tinte, lateinische Kurrent}\toendnotes[C]{\smallbreak}
\pstart
           {\pb}\textcolor{gray}{\textbf{HOTEL KAISERHOF\oindex{Hotel Meranerhof@\textbf{Hotel Meranerhof}, \emph{Hotel}|pw}}}{\\}\textcolor{gray}{\textbf{A. Ellmenreich\pwindex{Ellmenreich, Alexander 12.\,3.\,1854 – 30.\,4.\,1911@\textsc{Ellmenreich, Alexander} (12.\,3.\,1854 – 30.\,4.\,1911), \emph{Hotelier, Hotelbesitzer}|pw}.}}\hfill \textcolor{gray}{\textbf{Meran\oindex{Meran@\textbf{Meran}, \emph{Hauptstadt}|pw},}}{ }18. III. \textcolor{gray}{\textbf{189}}7\pend
           
\pstart\center{}Hochverehrter Herr Doctor!\pend\vspace{0.5em}
\pstart
           Nach langer Pause erlaube ich mir heute wieder einmal, Ihre liebenswürdige
               Aufmerksamkeit für eine \label{K_L03732-1v}\edtext{kleine Arbeit\pwindex{Plessner, Elsa 22.\,8.\,1875 Wien – 7.\,5.\,1932 Alicante@\textsc{Plessner, Elsa} (22.\,8.\,1875 Wien – 7.\,5.\,1932 Alicante), \emph{Schriftstellerin}!gläserne Käfig. Eine Parabel@\strich\emph{Der gläserne Käfig. Eine Parabel}|pwv}}{\lemma{\textnormal{\emph{kleine Arbeit}}}\Cendnote{\textnormal{Die Beilage, ein Manuskript der
                  Erzählung \emph{Der gläserne Käfig}\pwindex{Plessner, Elsa 22.\,8.\,1875 Wien – 7.\,5.\,1932 Alicante@\textsc{Plessner, Elsa} (22.\,8.\,1875 Wien – 7.\,5.\,1932 Alicante), \emph{Schriftstellerin}!gläserne Käfig. Eine Parabel@\strich\emph{Der gläserne Käfig. Eine Parabel}|pwk}, ist nicht
                  überliefert.}}}\label{K_L03732-1} zu erbitten. – Bitte – was halten Sie davon?? – –\pend
           
\pstart
           Zugleich bitte ich Sie, mir meinen letzten \label{K_L03732-2v}\edtext{Brief}{\lemma{\textnormal{\emph{Brief}}}\Cendnote{\textnormal{XXXX Auszeichnungsfehler: Dokument L03712 nicht gefunden. }}}\label{K_L03732-2} – aus dem
                  Januar – nicht übel auszulegen. Es that mir knapp nach seiner
               Absendung \uline{schrecklich leid}, ihn geschrieben zu haben.
               Was müssen Sie von diesen unverlangten und ziemlich verworrenen Confidenzen gedacht
               haben!! Dieser Gedanke hat mir einige fatale Stunden bereitet!! – Nun – ich habe die
               Lehre daraus gezogen, nie wieder so umgehend und – – unüberlegt einen \label{K_L03732-3v}\edtext{Brief}{\lemma{\textnormal{\emph{Brief}}}\Cendnote{\textnormal{Schnitzlers Brief ist nicht
                  überliefert.}}}\label{K_L03732-3} zu beant{\pb}worten, da man sich
               sonst zu sehr von seiner Stimmung hinreisten lässt. – – –\pend
           
\pstart
           Also bitte – – – vergessen Sie das Ungethüm. – Es scheint Sie
               übrigens stark verstimmt zu haben, da keine Antwort – – Sie hatten damit Recht. –
               Pardon! – – – – – – –\pend
           
\pstart
           Ihr geschätzte Kritik des »gläsernen Käfigs\pwindex{Plessner, Elsa 22.\,8.\,1875 Wien – 7.\,5.\,1932 Alicante@\textsc{Plessner, Elsa} (22.\,8.\,1875 Wien – 7.\,5.\,1932 Alicante), \emph{Schriftstellerin}!gläserne Käfig. Eine Parabel@\strich\emph{Der gläserne Käfig. Eine Parabel}|pw}«
               erbitte an notirte Wiener Adresse\oindex{XXXX Ortsangabe fehlt|pwu}, da
                  Sonntag dort eintreffe. Die Arbeit\pwindex{Plessner, Elsa 22.\,8.\,1875 Wien – 7.\,5.\,1932 Alicante@\textsc{Plessner, Elsa} (22.\,8.\,1875 Wien – 7.\,5.\,1932 Alicante), \emph{Schriftstellerin}!gläserne Käfig. Eine Parabel@\strich\emph{Der gläserne Käfig. Eine Parabel}|pwv} ist aus dem Januar datirt und lag so
               lange im Pult, da ich kein Vertrauen dazu hatte – – – und habe – Ganz ehrlich!! – Na
               – Sie werden richten. Tausend Dank im Voraus – – !\pend
           
\pstart
           Mit alter und neuer Verehrung{\\[\baselineskip]}\spacefill\mbox{Elsa Plessner.}\pend
           \leftskip=0em{}\selectlanguage{ngerman}\endnumbering\briefempfaengerindex{Schnitzler, Arthur@\textsc{Schnitzler, Arthur}!zzzPlessner, Elsa@\emph{von Elsa Plessner}!1897-03-182@{18. 3. 1897}|)be}\mylabel{L03732h}  \newcommand{\dateiname}{L03732}\newcommand{\titel}{Elsa Plessner an Arthur Schnitzler, 18. 3. 1897}\newcommand{\editorInnen}{Selma Jahnke und Martin Anton Müller}%% latex-leseansicht-abspann.tex
%% Abspann für die Leseansicht.
%% Der Schalter \ifkorrekturansicht ist bereits durch den Vorspann gesetzt.

%% latex-abspann.tex
%% Gemeinsamer Abspann für Korrekturansicht und Leseansicht.
%% Setzt den Schalter \ifkorrekturansicht voraus (gesetzt in den
%% einbindenden Dateien latex-korrekturansicht-abspann.tex bzw.
%% latex-leseansicht-abspann.tex).
%% ---------------------------------------------------------------

\normalsize

% Das esempio-Environment wird nur in der Leseansicht benötigt
\ifkorrekturansicht\else
\newenvironment{esempio}[3]%
{
    \vspace{1.5ex}
    \rlap{\underline{#1}}
    \par
    \setlength{\parindent}{0cm}
    \nopagebreak
    \leftskip=#2cm
    \rightskip=#3cm
}
{
    \par
}
\fi

\doendnotes{C}
\bigskip
\vfill

\clearpage

\footnotesize

\ifkorrekturansicht
  \lohead{\textsc{register}}
\fi

% theindex-Environment neu definieren ohne reledmac
\makeatletter
\renewenvironment{theindex}{%
  \ifkorrekturansicht
    \section*{\indexname}%
  \else
    \subsubsection*{Index der erwähnten Entitäten}%
  \fi
  \setlength{\parindent}{0pt}%
  \setlength{\parskip}{0pt plus 0.3pt}%
  \let\item\@idxitem
}{%
  \ifkorrekturansicht\clearpage\fi
}
\makeatother

\IfFileExists{\jobname-pw.ind}{\input{\jobname-pw.ind}}{}

% Quellenangabe nur in der Leseansicht
\ifkorrekturansicht\else
% Fallback-Definitionen, falls die .tex-Datei \titel etc. nicht gesetzt hat
\providecommand{\titel}{}
\providecommand{\editorInnen}{}
\providecommand{\dateiname}{\jobname}

\vspace{3cm}

\vfill

\footnotesize
\textsc{Quelle}: \titel. Herausgegeben von {\editorInnen}. In: \emph{Arthur Schnitzler: Briefwechsel mit Autorinnen und Autoren}.
 Digitale Edition, https://schnitzler-briefe.acdh.oeaw.ac.at/{\dateiname}.html (Stand \today)
\fi

\end{document}


