%% latex-korrekturansicht-vorspann.tex
%% Vorspann für die Korrekturansicht.
%% Lädt die gemeinsame Datei latex-vorspann.tex mit gesetztem Schalter.

\newif\ifkorrekturansicht
\korrekturansichttrue

\input{../tex-inputs/latex-vorspann}


\section[Richard Beer-Hofmann an Arthur Schnitzler, {[}7. 10. 1905{]}]{L01558 Richard Beer-Hofmann an Arthur Schnitzler, {[}7. 10. 1905{]}}
\nopagebreak\mylabel{L01558v}
\rehead{ }\normalsize\beginnumbering\briefempfaengerindex{Schnitzler, Arthur@\textsc{Schnitzler, Arthur}!zzzBeer-Hofmann, Richard@\emph{von Richard Beer-Hofmann}!1905-10-071@{{[}7. 10. 1905{]}}|(be}
\toendnotes[C]{\smallbreak\pagebreak[2]}\Standort{CUL, Schnitzler, B 8.}
\physDesc{Brief, 1 Blatt, 1 Seite, 361 Zeichen
\newline{}Handschrift: schwarze Tinte, lateinische Kurrent
\newline{}Schnitzler: mit Bleistift datiert: »7/X 905« 
\newline{}Ordnung: mit Bleistift von unbekannter Hand nummeriert:
                                    »205« }
\buchAbdrucke{\weitereDrucke{Arthur Schnitzler, Richard Beer-Hofmann: \emph{Briefwechsel 1891–1931}. Wien, Zürich: \emph{Europaverlag} 1992, S. 175–176.} }\toendnotes[C]{\smallbreak}
\pstart
           \raggedleft{}{\pb}Samstag\pend
           \vspace{0.5em}
\pstart
           Lieber Arthur! Eine ganze Fülle widriger Dinge (Ja{\geminationm}ern mündlich) hat mich abgehalten Sie bis jetzt
               aufzusuchen. So warte ich bis nach der \label{K_L01558-1v}\edtext{Première\pwindex{Zwischenspiel. Komoedie in drei Akten@\emph{Zwischenspiel. Komödie in drei Akten}|pwv}}{\lemma{\textnormal{\emph{Première}}}\Cendnote{\textnormal{von \emph{Zwischenspiel}\pwindex{Zwischenspiel. Komoedie in drei Akten@\emph{Zwischenspiel. Komödie in drei Akten}|pwk} am 12. 10. 1905}}}\label{K_L01558-1}, da Sie wol jetzt in den Proben stecken. Hugo\pwindex{Hofmannsthal, Hugo von 1874-02-01 – 1929-07-15@\textsc{Hofmannsthal, Hugo von} (1874-02-01 – 1929-07-15), \emph{Schriftsteller/Schriftstellerin}|pw} sagt mir wegen der Sitze. Wenn es geht, möchte ich am liebsten erste
               Reihe rechts – (2 Sitze) haben. Wenn nicht – so gute Sitze als Sie beko{\geminationm}en.\pend
           
\pstart
           Von Herzen Ihr{\\[\baselineskip]}\spacefill\mbox{Richard}\pend
           \leftskip=0em{}\selectlanguage{ngerman}\endnumbering\briefempfaengerindex{Schnitzler, Arthur@\textsc{Schnitzler, Arthur}!zzzBeer-Hofmann, Richard@\emph{von Richard Beer-Hofmann}!1905-10-071@{{[}7. 10. 1905{]}}|)be}\mylabel{L01558h}  \normalsize

\doendnotes{C}
\bigskip
\vfill

\clearpage

\footnotesize

\lohead{\textsc{register}}

% Definiere theindex-Environment komplett neu ohne reledmac
\makeatletter
\renewenvironment{theindex}{%
  \section*{\indexname}%
  \setlength{\parindent}{0pt}%
  \setlength{\parskip}{0pt plus 0.3pt}%
  \let\item\@idxitem
}{%
  \clearpage
}
\makeatother

\IfFileExists{\jobname-pw.ind}{\input{\jobname-pw.ind}}{}

\end{document}

      