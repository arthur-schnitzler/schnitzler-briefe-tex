%% latex-korrekturansicht-vorspann.tex
%% Vorspann für die Korrekturansicht.
%% Lädt die gemeinsame Datei latex-vorspann.tex mit gesetztem Schalter.

\newif\ifkorrekturansicht
\korrekturansichttrue

\input{../tex-inputs/latex-vorspann}


\section[Arthur Schnitzler an Richard Beer-Hofmann, 4. 10. 1897]{L00729 Arthur Schnitzler an Richard Beer-Hofmann, 4. 10. 1897}
\nopagebreak\mylabel{L00729v}
\rehead{ }\normalsize\beginnumbering\briefempfaengerindex{Beer-Hofmann, Richard@\textsc{Beer-Hofmann, Richard}!zzzSchnitzler, Arthur@\emph{von Arthur Schnitzler}!1897-10-041@{4. 10. 1897}|(be}
\toendnotes[C]{\smallbreak\pagebreak[2]}\Standort{CUL, Schnitzler, B 8.1, S. 66.}
\physDesc{Brief, maschinenschriftliche Abschrift1 Blatt, 1 Seite, 386 Zeichen
\newline{}Schreibmaschine
\newline{}Ordnung: von unbekannter Hand nummeriert: »105« }
\buchAbdrucke{\weitereDrucke{Arthur Schnitzler, Richard Beer-Hofmann: \emph{Briefwechsel 1891–1931}. Wien, Zürich: \emph{Europaverlag} 1992, S. 113.} }\toendnotes[C]{\smallbreak}
\pstart
           \raggedleft{}{\pb}Wien\oindex{Wien@\textbf{Wien}, \emph{A.ADM2}|pw}, 4. 10. 97.\pend
           \vspace{0.5em}
\pstart
           Lieber Richard, Sie teleph. mich immer an, wenn ich nicht zu Haus
               bin. Vormittag bin ich nämlich auf dem Land. Schaun Sie doch einmal Nachmittag bevor
               Sie nach Heiligst.\oindex{Heiligenstadt@\textbf{Heiligenstadt}, \emph{P.PPL}|pw} fahren, zu mir herauf. Ich
               möchte auch gern einmal mit Ihnen hinaus. Hugo\pwindex{Hofmannsthal, Hugo von 1874-02-01 – 1929-07-15@\textsc{Hofmannsthal, Hugo von} (1874-02-01 – 1929-07-15), \emph{Schriftsteller/Schriftstellerin}|pw}
               schreibt mir, er kommt nächste Woche nach Wien\oindex{Wien@\textbf{Wien}, \emph{A.ADM2}|pw} und
               möchte Ihnen und mir viel vorlesen.\pend
           
\pstart
           Herzlich Ihr \spacefill\mbox{Arthur.}\pend
           
\pstart
           \noindent{}Ich arbeite sozusagen.\pend
           
\pstart
           (\label{T_L00729-1v}\edtext{w. o.}{\lemma{\textnormal{\emph{w. o.}}}\Cendnote{\textnormal{»wie oben«: Verweis auf frühere Stelle der Briefabschrift. Der
                     Brief wurde in die Wollzeile 15\oindex{Wollzeile@\textbf{Wollzeile}, \emph{Straße (K.STR)}|pwk}
                     geschickt.}}}\label{T_L00729-1})\pend
           \selectlanguage{ngerman}\endnumbering\briefempfaengerindex{Beer-Hofmann, Richard@\textsc{Beer-Hofmann, Richard}!zzzSchnitzler, Arthur@\emph{von Arthur Schnitzler}!1897-10-041@{4. 10. 1897}|)be}\mylabel{L00729h}  \normalsize

\doendnotes{C}
\bigskip
\vfill

\clearpage

\footnotesize

\lohead{\textsc{register}}

% Definiere theindex-Environment komplett neu ohne reledmac
\makeatletter
\renewenvironment{theindex}{%
  \section*{\indexname}%
  \setlength{\parindent}{0pt}%
  \setlength{\parskip}{0pt plus 0.3pt}%
  \let\item\@idxitem
}{%
  \clearpage
}
\makeatother

\IfFileExists{\jobname-pw.ind}{\input{\jobname-pw.ind}}{}

\end{document}

      