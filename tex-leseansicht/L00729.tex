%% latex-leseansicht-vorspann.tex
%% Vorspann für die Leseansicht.
%% Lädt die gemeinsame Datei latex-vorspann.tex mit nicht gesetztem Schalter.

\newif\ifkorrekturansicht
\korrekturansichtfalse

\input{../tex-inputs/latex-vorspann}


         
         \renewcommand{\erwaehntePersonen}{Personen: Richard Beer-Hofmann, Hugo von Hofmannsthal}
         \renewcommand{\erwaehnteOrte}{Orte: Heiligenstadt, Wien, Wollzeile}
         \renewcommand{\erwaehnteWerke}{}
               \section[Arthur Schnitzler an Richard Beer-Hofmann, 4. 10. 1897]{ Arthur Schnitzler an Richard Beer-Hofmann, 4. 10. 1897}\nopagebreak\mylabel{v}\rehead{ }\begin{ledgroupsized}[t]{13cm}\normalsize\beginnumbering\briefempfaengerindex{Beer-Hofmann, Richard@\textsc{Beer-Hofmann, Richard}!zzzSchnitzler, Arthur@\emph{von Arthur Schnitzler}!1897-10-041@{4. 10. 1897}|(be} \toendnotes[C]{\smallbreak\pagebreak[2]} \Standort{CUL, Schnitzler, B 8.1, S. 66.}
\physDesc{Brief, maschinenschriftliche Abschrift, 1 Blatt, 1 Seite, 386 Zeichen
\newline{}Schreibmaschine
\newline{}Ordnung: von unbekannter Hand nummeriert: »105« }\buchAbdrucke{\weitereDrucke{Arthur Schnitzler, Richard Beer-Hofmann: \emph{Briefwechsel 1891–1931}. Hg. Konstanze Fliedl. Wien, Zürich: \emph{Europaverlag} 1992, S. 113.} }\toendnotes[C]{\smallbreak}\pstart
           \raggedleft{}{\pb}Wien\oindex{Wien@\textbf{Wien}|pw}, 4. 10. 97.\pend
           \pstart
           Lieber Richard, Sie teleph. mich immer an, wenn ich nicht zu Haus
               bin. Vormittag bin ich nämlich auf dem Land. Schaun Sie doch einmal Nachmittag bevor
               Sie nach Heiligst.\oindex{Heiligenstadt@\textbf{Heiligenstadt}|pw} fahren, zu mir herauf. Ich
               möchte auch gern einmal mit Ihnen hinaus. Hugo\pwindex{Hofmannsthal, Hugo von 1874-02-01 – 1929-07-15@\textsc{Hofmannsthal, Hugo von} (1874-02-01 – 1929-07-15), \emph{Schriftsteller}|pw}
               schreibt mir, er kommt nächste Woche nach Wien\oindex{Wien@\textbf{Wien}|pw} und
               möchte Ihnen und mir viel vorlesen.\pend
           \pstart
           Herzlich Ihr \spacefill\mbox{Arthur.}\pend
           \pstart
           \noindent{}Ich arbeite sozusagen.\pend
           \pstart
           (\label{T_L00729-1v}\edtext{w. o.}{\lemma{\textnormal{\emph{w. o.}}}\Cendnote{\textnormal{»wie oben«: Verweis auf frühere Stelle der Briefabschrift. Der
                     Brief wurde in die Wollzeile 15\oindex{Wollzeile@\textbf{Wollzeile}|pwk}
                     geschickt.}}}\label{T_L00729-1h})\pend
           
         
         \endnumbering\mylabel{h}\end{ledgroupsized}  \newcommand{\dateiname}{L00729}\newcommand{\titel}{Arthur Schnitzler an Richard Beer-Hofmann, 4. 10. 1897}\newcommand{\editorInnen}{Martin Anton Müller und Gerd-Hermann Susen}%% latex-leseansicht-abspann.tex
%% Abspann für die Leseansicht.
%% Der Schalter \ifkorrekturansicht ist bereits durch den Vorspann gesetzt.

%% latex-abspann.tex
%% Gemeinsamer Abspann für Korrekturansicht und Leseansicht.
%% Setzt den Schalter \ifkorrekturansicht voraus (gesetzt in den
%% einbindenden Dateien latex-korrekturansicht-abspann.tex bzw.
%% latex-leseansicht-abspann.tex).
%% ---------------------------------------------------------------

\normalsize

% Das esempio-Environment wird nur in der Leseansicht benötigt
\ifkorrekturansicht\else
\newenvironment{esempio}[3]%
{
    \vspace{1.5ex}
    \rlap{\underline{#1}}
    \par
    \setlength{\parindent}{0cm}
    \nopagebreak
    \leftskip=#2cm
    \rightskip=#3cm
}
{
    \par
}
\fi

\doendnotes{C}
\bigskip
\vfill

\clearpage

\footnotesize

\ifkorrekturansicht
  \lohead{\textsc{register}}
\fi

% theindex-Environment neu definieren ohne reledmac
\makeatletter
\renewenvironment{theindex}{%
  \ifkorrekturansicht
    \section*{\indexname}%
  \else
    \subsubsection*{Index der erwähnten Entitäten}%
  \fi
  \setlength{\parindent}{0pt}%
  \setlength{\parskip}{0pt plus 0.3pt}%
  \let\item\@idxitem
}{%
  \ifkorrekturansicht\clearpage\fi
}
\makeatother

\IfFileExists{\jobname-pw.ind}{\input{\jobname-pw.ind}}{}

% Quellenangabe nur in der Leseansicht
\ifkorrekturansicht\else
% Fallback-Definitionen, falls die .tex-Datei \titel etc. nicht gesetzt hat
\providecommand{\titel}{}
\providecommand{\editorInnen}{}
\providecommand{\dateiname}{\jobname}

\vspace{3cm}

\vfill

\footnotesize
\textsc{Quelle}: \titel. Herausgegeben von {\editorInnen}. In: \emph{Arthur Schnitzler: Briefwechsel mit Autorinnen und Autoren}.
 Digitale Edition, https://schnitzler-briefe.acdh.oeaw.ac.at/{\dateiname}.html (Stand \today)
\fi

\end{document}


      