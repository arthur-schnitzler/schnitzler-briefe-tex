%% latex-leseansicht-vorspann.tex
%% Vorspann für die Leseansicht.
%% Lädt die gemeinsame Datei latex-vorspann.tex mit nicht gesetztem Schalter.

\newif\ifkorrekturansicht
\korrekturansichtfalse

\input{../tex-inputs/latex-vorspann}

\begin{center}
            \textcolor{red}{ENTWURF, NICHT FERTIG KORRIGIERT}
                      \end{center}
            
         
         \renewcommand{\erwaehntePersonen}{Personen: Richard Beer-Hofmann, Moriz Benedikt, Otto Brahm, Heinrich Conrad, Paul Goldmann, Hugo von Hofmannsthal, Christiane von Hofmannsthal, Josef Kainz, Heinrich Kanner, Henri de Rochefort, Felix Salten, Paul Schlenther, Olga Schnitzler, Gustav Schwarzkopf}
         \renewcommand{\erwaehnteInstitutionen}{Institutionen: Deutsches Theater Berlin, Die Zeit, Die Zeit. Wiener Wochenschrift, Neue Freie Presse, Robert Lutz, Théâtre Antoine, Wiener Allgemeine Zeitung}
         \renewcommand{\erwaehnteOrte}{Orte: Florenz, Medici-Kapelle, San Domenico, Wien}
         \renewcommand{\erwaehnteWerke}{Werke: Abenteuer meines Lebens, Der Schleier der Beatrice. Schauspiel in fünf Akten, Der Weg zum Licht. Ein Salzburger Märchendrama in vier Akten, Der einsame Weg. Schauspiel in fünf Akten, Der grüne Kakadu. Groteske in einem Akt, Die kleine Veronika. Novelle, Dämmerseele, Les Aventures de ma vie, Neue Freie Presse}
               \section[Arthur Schnitzler an Felix Salten, 27. 5. 1902]{ Arthur Schnitzler an Felix Salten, 27. 5. 1902}\nopagebreak\mylabel{v}\rehead{ }\begin{ledgroupsized}[t]{13cm}\normalsize\beginnumbering \toendnotes[C]{\smallbreak\pagebreak[2]} \Standort{Wienbibliothek im Rathaus, ZPH 1681, 2.1.516.}
\physDesc{Brief, 2 Blätter, 8 Seiten
\newline{}Handschrift: Bleistift, deutsche Kurrent\newline{}Ordnung: mit Bleistift von unbekannter Hand Nummerierung der ungeraden Seiten:
                                 »62«–»65« }\toendnotes[C]{\smallbreak}\pstart
           \raggedleft{}{\pb}27. 5. 902\pend
           \pstart
           lieber, ich freu mich ſehr über den guten Eindruck, den Sie von der
                  \label{K_L02974-88v}\edtext{Novellette\pwindex{Schnitzler, Arthur 15.05.1862 – 21.10.1931@\textsc{Schnitzler, Arthur} (15.05.1862 – 21.10.1931), \emph{Schriftsteller, Mediziner}!Daemmerseele18. 5. 1902@\strich\emph{Dämmerseele} {[}18. 5. 1902{]}|pwv}}{\lemma{\textnormal{\emph{Novellette}}}\Cendnote{\textnormal{Arthur Schnitzler\pwindex{Schnitzler, Arthur 15.05.1862 – 21.10.1931@\textsc{Schnitzler, Arthur} (15.05.1862 – 21.10.1931), \emph{Schriftsteller, Mediziner}|pwk}: \emph{Dämmerseele}\pwindex{Schnitzler, Arthur 15.05.1862 – 21.10.1931@\textsc{Schnitzler, Arthur} (15.05.1862 – 21.10.1931), \emph{Schriftsteller, Mediziner}!Daemmerseele18. 5. 1902@\strich\emph{Dämmerseele} {[}18. 5. 1902{]}|pwk}. In: \emph{Neue Freie Presse}\pwindex{Neue Freie Presse1864 – 1939@\emph{Neue Freie Presse} {[}1864 – 1939{]}|pwk}, Nr. 13.553,
                        18. 5. 1902, Morgenblatt, Pfingstbeilage,
                     S. 31–33.}}}\label{K_L02974-88h} in d. N. Fr. Pr.\pwindex{Neue Freie Presse1864 – 1939@\emph{Neue Freie Presse} {[}1864 – 1939{]}|pw} haben; was mir eigentlich ſelten geſchie\textcolor{gray}{h}t, –
               ich war ſelbſt ein bischen unſicher im Urtheil daſs ſie Schwarzk.\pwindex{Schwarzkopf, Gustav 07.11.1853 – 13.11.1939@\textsc{Schwarzkopf, Gustav} (07.11.1853 – 13.11.1939), \emph{Schriftsteller}|pw} nicht mag, iſt ziemlich verſtändlich; – der Einwurf
                  Goldm.\pwindex{Goldmann, Paul 31.01.1865 – 25.09.1935@\textsc{Goldmann, Paul} (31.01.1865 – 25.09.1935), \emph{Schriftsteller, Journalist}|pw}: es handle ſich um Liebe, kaum
               discutirbar; Richard\pwindex{Beer-Hofmann, Richard 1866-07-11 – 1945-09-26@\textsc{Beer-Hofmann, Richard} (1866-07-11 – 1945-09-26), \emph{Schriftsteller}|pw} u Hugo\pwindex{Hofmannsthal, Hugo von 1874-02-01 – 1929-07-15@\textsc{Hofmannsthal, Hugo von} (1874-02-01 – 1929-07-15), \emph{Schriftsteller}|pw} ſcheinen ſie im ganzen gut zu finden, aber {\pb}wie mir ſchien, mit einigem innern
               Widerſtand. Olga\pwindex{Schnitzler, Olga 17.01.1882 – 13.01.1970@\textsc{Schnitzler, Olga} (17.01.1882 – 13.01.1970), \emph{Schauspielerin, Sängerin}|pw} gefiel ſie, als ich ſie ihr
               vorlas, beſonders gut; – die gedruckte hat ſie aber enttäuscht. Meine Bedenken gehen
               nach der Seite des mä{\geminationn}lichen {\dotstwo} ich finde eben kein andres Wort – Helden{\dotstwo}, wo mir was
               zu fehlen ſcheint. Der Titel ko{\geminationm}t mir ſelbſt nach jedem
               Überdenken Ihrer Einwände, nicht un {\pb}glücklich vor. Daſs Sie als der erſte den Schluſs nicht als Pointe empfinden,
               ſondern wohl im Gegentheil gerade als den Ausklang ins ungewiſſe, ferne, mit
               Notwendgkeit weiterflutend, be\textcolor{gray}{rü}hrt mich beſonders angenehm.– \pend
           \pstart
           Paul G.\pwindex{Goldmann, Paul 31.01.1865 – 25.09.1935@\textsc{Goldmann, Paul} (31.01.1865 – 25.09.1935), \emph{Schriftsteller, Journalist}|pw} ist wieder fort; die \label{K_L02974-43v}\edtext{Martin Finder Sachen}{\lemma{\textnormal{\emph{Martin Finder Sachen}}}\Cendnote{\textnormal{Da Salten\pwindex{Salten, Felix 06.09.1869 – 08.10.1945@\textsc{Salten, Felix} (06.09.1869 – 08.10.1945), \emph{Schriftsteller, Journalist}|pwk} bis zum 30. 6. 1902 bei der \emph{Wiener Allgemeinen Zeitung}\orgindex{Wiener Allgemeine Zeitung@Wiener Allgemeine Zeitung|pwk} unter Vertrag stand,
                  veröffentlichte er seine Beiträge für die Wochenschrift \emph{Zeit}\orgindex{Zeit. Wiener Wochenschrift@Die Zeit. Wiener Wochenschrift|pwk} bis dahin unter diesem Pseudonym, in das nur wenige
                  Personen eingeweiht waren.}}}\label{K_L02974-43h} ſind ihm höchlich aufgefallen;– er hat ſich
               gefragt: Was ko{\geminationm}t da für ein {\pb}{[}»{]}Nachwuchs« – er iſt es,
               der in d. N. Fr. Pr.\orgindex{Neue Freie Presse@Neue Freie Presse|pw} mit lebhafteſter Betonung
               von Ihnen ſprach, worauf Bened.\pwindex{Benedikt, Moriz 27.05.1849 – 18.03.1920@\textsc{Benedikt, Moriz} (27.05.1849 – 18.03.1920), \emph{Journalist, Herausgeber}|pw} meinte, er
               dächte ſchon lange Zeit an Sie{\dots} Das will natürlich nicht
               viel heißen; aber ich glaube, we{\geminationn} Sie zu irgend welchen
               Schritten ſich entſchlöſſen (über die natürlich noch geſprochen werden muſs), ſo
               wären hier die Chancen, mindeſtens materiell günſtiger als bei der Zeit\orgindex{Zeit@Die Zeit|pw}. Obwohl {\pb}Kanner\pwindex{Kanner, Heinrich 09.11.1864 – 15.02.1930@\textsc{Kanner, Heinrich} (09.11.1864 – 15.02.1930), \emph{Herausgeber, Publizist}|pw} zu P. G.\pwindex{Goldmann, Paul 31.01.1865 – 25.09.1935@\textsc{Goldmann, Paul} (31.01.1865 – 25.09.1935), \emph{Schriftsteller, Journalist}|pw}, der auch dort von Ihnen redete, geäußert hat: »\label{K_L02974-124v}\edtext{Er wird ja für uns ſchreiben.}{\lemma{\textnormal{\emph{Er … ſchreiben.}}}\Cendnote{\textnormal{Kanner\pwindex{Kanner, Heinrich 09.11.1864 – 15.02.1930@\textsc{Kanner, Heinrich} (09.11.1864 – 15.02.1930), \emph{Herausgeber, Publizist}|pwk} wahrte also Salten\pwindex{Salten, Felix 06.09.1869 – 08.10.1945@\textsc{Salten, Felix} (06.09.1869 – 08.10.1945), \emph{Schriftsteller, Journalist}|pwk}s
                  Pseudonym und verriet nicht, dass dieser schon für die Wochenschrift \emph{Die Zeit}\orgindex{Zeit. Wiener Wochenschrift@Die Zeit. Wiener Wochenschrift|pwk} schrieb, und Bezog sich nur auf die anlaufende
                  Gründung der neuen Tageszeitung, die ab 27. 9. 1902 erschien.}}}\label{K_L02974-124h}«– \pend
           \pstart
           Kainz\pwindex{Kainz, Josef 02.01.1858 – 20.09.1910@\textsc{Kainz, Josef} (02.01.1858 – 20.09.1910), \emph{Schauspieler}|pw} will durchaus im »Weg zum Licht\pwindex{\textcolor{red}{\textsuperscript{XXXX1 indx}}!Weg zum Licht. Ein Salzburger Maerchendrama in vier Akten1902-04-05@\strich\emph{Der Weg zum Licht. Ein Salzburger Märchendrama in vier Akten} {[}1902-04-05{]}|pw}« ſpielen; u Schlenther\pwindex{Schlenther, Paul 20.08.1854 – 30.04.1916@\textsc{Schlenther, Paul} (20.08.1854 – 30.04.1916), \emph{Schriftsteller, Kritiker, Theaterleiter}|pw} dürfte es daher aufführen (So Brahm\pwindex{Brahm, Otto 05.02.1856 – 28.11.1912@\textsc{Brahm, Otto} (05.02.1856 – 28.11.1912), \emph{Theaterleiter, Regisseur}|pw}.) Es iſt recht lächerlich, daſs ein ſolcher Künſtler den \label{K_L02974-1v}\edtext{Hahngikl\pwindex{\textcolor{red}{\textsuperscript{XXXX1 indx}}!Weg zum Licht. Ein Salzburger Maerchendrama in vier Akten1902-04-05@\strich\emph{Der Weg zum Licht. Ein Salzburger Märchendrama in vier Akten} {[}1902-04-05{]}|pwv}}{\lemma{\textnormal{\emph{Hahngikl}}}\Cendnote{\textnormal{laut Figurenliste »ein Dunkelelb
                     vom Untersberg«}}}\label{K_L02974-1h} dem \textsc{Bentivoglio\pwindex{Schnitzler, Arthur 15.05.1862 – 21.10.1931@\textsc{Schnitzler, Arthur} (15.05.1862 – 21.10.1931), \emph{Schriftsteller, Mediziner}!Schleier der Beatrice. Schauspiel in fuenf Akten1900-12-01@\strich\emph{Der Schleier der Beatrice. Schauspiel in fünf Akten} {[}1900-12-01{]}|pwv}} vorzieht; aber es liegt wohl recht tief.– Dem Deutſchen Theater\orgindex{Deutsches Theater Berlin@Deutsches Theater Berlin|pw} geht es hier ausgezeichnet. – Der Kakadu\pwindex{Schnitzler, Arthur 15.05.1862 – 21.10.1931@\textsc{Schnitzler, Arthur} (15.05.1862 – 21.10.1931), \emph{Schriftsteller, Mediziner}!gruene Kakadu. Groteske in einem Akt1. 3. 1899@\strich\emph{Der grüne Kakadu. Groteske in einem Akt} {[}1. 3. 1899{]}|pw} iſt {\pb}bei Antoine\orgindex{Theâtre Antoine@Théâtre Antoine|pw} acceptirt. Über die \textsc{Bea.\pwindex{Schnitzler, Arthur 15.05.1862 – 21.10.1931@\textsc{Schnitzler, Arthur} (15.05.1862 – 21.10.1931), \emph{Schriftsteller, Mediziner}!Schleier der Beatrice. Schauspiel in fuenf Akten1900-12-01@\strich\emph{Der Schleier der Beatrice. Schauspiel in fünf Akten} {[}1900-12-01{]}|pw}} ſpricht Brahm\pwindex{Brahm, Otto 05.02.1856 – 28.11.1912@\textsc{Brahm, Otto} (05.02.1856 – 28.11.1912), \emph{Theaterleiter, Regisseur}|pw} kein Wort.– Ich überdenke
               und scenire mein Stück\pwindex{Schnitzler, Arthur 15.05.1862 – 21.10.1931@\textsc{Schnitzler, Arthur} (15.05.1862 – 21.10.1931), \emph{Schriftsteller, Mediziner}!einsame Weg. Schauspiel in fuenf Akten1904@\strich\emph{Der einsame Weg. Schauspiel in fünf Akten} {[}1904{]}|pwv} u übe
               mich indeſs weiter im Erzählen. \pend
           \pstart
           – Sagen Sie mir doch etwas über Ihre Reiſe, Ihre Arbeiten, Ihre Laune. Daſs Hugo\pwindex{Hofmannsthal, Hugo von 1874-02-01 – 1929-07-15@\textsc{Hofmannsthal, Hugo von} (1874-02-01 – 1929-07-15), \emph{Schriftsteller}|pw} ein ganz kleines Kind beko{\geminationm}en hat, Chriſtiane\pwindex{Hofmannsthal, Christiane von 14.05.1902 – 05.01.1987@\textsc{Hofmannsthal, Christiane von} (14.05.1902 – 05.01.1987)|pw} genannt, wiſſen Sie wohl ſchon.– Heute {\pb}hatten wir beinah einen »Frühlingsabend« –
               lau, ohne Wind und Regen, man faſſt es kaum. – \textsc{\label{K_L02974-11v}\edtext{Rochefort\pwindex{Rochefort, Henri de 1830-01-31 – 1913-06-30@\textsc{Rochefort, Henri de} (1830-01-31 – 1913-06-30), \emph{Schriftsteller, Politiker, Politiker}!Abenteuer meines Lebens1900-11-02@\strich\emph{Abenteuer meines Lebens} {[}1900-11-02{]}|pwuv}\pwindex{Rochefort, Henri de 1830-01-31 – 1913-06-30@\textsc{Rochefort, Henri de} (1830-01-31 – 1913-06-30), \emph{Schriftsteller, Politiker, Politiker}|pw}}{\lemma{\textnormal{\emph{Rochefort}}}\Cendnote{\textnormal{Es dürfte sich um die (gekürzte)
                     deutschsprachige Ausgabe der Autobiografie von Henri Rochefort\pwindex{Rochefort, Henri de 1830-01-31 – 1913-06-30@\textsc{Rochefort, Henri de} (1830-01-31 – 1913-06-30), \emph{Schriftsteller, Politiker, Politiker}|pwk} handeln: \emph{Abenteuer meines Lebens}\pwindex{Rochefort, Henri de 1830-01-31 – 1913-06-30@\textsc{Rochefort, Henri de} (1830-01-31 – 1913-06-30), \emph{Schriftsteller, Politiker, Politiker}!Abenteuer meines Lebens1900-11-02@\strich\emph{Abenteuer meines Lebens} {[}1900-11-02{]}|pwk}. Autorisierte
                        deutsche Bearbeitung von Heinrich
                           Conrad\pwindex{Conrad, Heinrich 19.10.1866 – 1918-12-20@\textsc{Conrad, Heinrich} (19.10.1866 – 1918-12-20), \emph{Übersetzer, Romanist}|pwk}. Stuttgart: \emph{Robert Lutz}\orgindex{Robert Lutz@Robert Lutz|pwk}{ }1900. (Original: \emph{Les Aventures de ma
                        vie}\pwindex{Rochefort, Henri de 1830-01-31 – 1913-06-30@\textsc{Rochefort, Henri de} (1830-01-31 – 1913-06-30), \emph{Schriftsteller, Politiker, Politiker}!Aventures de ma vie1896@\strich\emph{Les Aventures de ma vie} {[}1896{]}|pwk}{ }1896).}}}\label{K_L02974-11h}} wird gegen Schluſs matter; ich beſchäftige mich ein weniges mit Botanik und
               denke wieder manchmal mit Wehmut, wie faul ich mein Leben lang war, und auf wie viel
               beſſerm Grund ich {\pb}ſtehen könnte, we{\geminationn} ich nicht gar ſo ſpät auf mich aufmerkſam geworden
               wäre. \pend
           \pstart
           Leben Sie wohl. Grüßen Sie Florenz\oindex{Florenz@\textbf{Florenz}|pw}, die \textsc{Mediceer} Gräber\oindex{Medici-Kapelle@\textbf{Medici-Kapelle}|pw}, den Garten hinter dem Kloſter zu \textsc{Fiesole}\oindex{San Domenico@\textbf{San Domenico}|pw} und Veronika\pwindex{Salten, Felix 06.09.1869 – 08.10.1945@\textsc{Salten, Felix} (06.09.1869 – 08.10.1945), \emph{Schriftsteller, Journalist}!kleine Veronika. Novelle1902-12-01@\strich\emph{Die kleine Veronika. Novelle} {[}1902-12-01{]}|pw}; – und Bern grüßt den andern
               Hund. \pend
           \pstart
           Herzlichst Ihr {\\[\baselineskip]}\spacefill\mbox{A.}\pend
           \leftskip=0em{}
         
         \endnumbering\mylabel{h}\end{ledgroupsized}\begin{anhang}\end{anhang}\newcommand{\dateiname}{L02974}\newcommand{\titel}{Arthur Schnitzler an Felix Salten, 27. 5. 1902}\newcommand{\editorInnen}{Martin Anton Müller und Laura Untner}%% latex-leseansicht-abspann.tex
%% Abspann für die Leseansicht.
%% Der Schalter \ifkorrekturansicht ist bereits durch den Vorspann gesetzt.

%% latex-abspann.tex
%% Gemeinsamer Abspann für Korrekturansicht und Leseansicht.
%% Setzt den Schalter \ifkorrekturansicht voraus (gesetzt in den
%% einbindenden Dateien latex-korrekturansicht-abspann.tex bzw.
%% latex-leseansicht-abspann.tex).
%% ---------------------------------------------------------------

\normalsize

% Das esempio-Environment wird nur in der Leseansicht benötigt
\ifkorrekturansicht\else
\newenvironment{esempio}[3]%
{
    \vspace{1.5ex}
    \rlap{\underline{#1}}
    \par
    \setlength{\parindent}{0cm}
    \nopagebreak
    \leftskip=#2cm
    \rightskip=#3cm
}
{
    \par
}
\fi

\doendnotes{C}
\bigskip
\vfill

\clearpage

\footnotesize

\ifkorrekturansicht
  \lohead{\textsc{register}}
\fi

% theindex-Environment neu definieren ohne reledmac
\makeatletter
\renewenvironment{theindex}{%
  \ifkorrekturansicht
    \section*{\indexname}%
  \else
    \subsubsection*{Index der erwähnten Entitäten}%
  \fi
  \setlength{\parindent}{0pt}%
  \setlength{\parskip}{0pt plus 0.3pt}%
  \let\item\@idxitem
}{%
  \ifkorrekturansicht\clearpage\fi
}
\makeatother

\IfFileExists{\jobname-pw.ind}{\input{\jobname-pw.ind}}{}

% Quellenangabe nur in der Leseansicht
\ifkorrekturansicht\else
% Fallback-Definitionen, falls die .tex-Datei \titel etc. nicht gesetzt hat
\providecommand{\titel}{}
\providecommand{\editorInnen}{}
\providecommand{\dateiname}{\jobname}

\vspace{3cm}

\vfill

\footnotesize
\textsc{Quelle}: \titel. Herausgegeben von {\editorInnen}. In: \emph{Arthur Schnitzler: Briefwechsel mit Autorinnen und Autoren}.
 Digitale Edition, https://schnitzler-briefe.acdh.oeaw.ac.at/{\dateiname}.html (Stand \today)
\fi

\end{document}


      