%% latex-korrekturansicht-vorspann.tex
%% Vorspann für die Korrekturansicht.
%% Lädt die gemeinsame Datei latex-vorspann.tex mit gesetztem Schalter.

\newif\ifkorrekturansicht
\korrekturansichttrue

\input{../tex-inputs/latex-vorspann}


\section[ Arthur Schnitzler an Felix Salten, 27. 5. 1902]{L02974 Arthur Schnitzler an Felix Salten, 27. 5. 1902}
\nopagebreak\mylabel{L02974v}
\rehead{ }\normalsize\beginnumbering\briefempfaengerindex{Salten, Felix@\textsc{Salten, Felix}!zzzSchnitzler, Arthur@\emph{von Arthur Schnitzler}!1902-05-271@{27. 5. 1902}|(be}
\toendnotes[C]{\smallbreak\pagebreak[2]}\Standort{Wienbibliothek im Rathaus, ZPH 1681, 2.1.516.}
\physDesc{Brief, 2 Blätter, 8 Seiten, 2579 Zeichen
\newline{}Handschrift: Bleistift, deutsche Kurrent
\newline{}Ordnung: mit Bleistift von unbekannter Hand Nummerierung der Doppelseiten des
                                 Konvoluts: »62«–»65« }\toendnotes[C]{\smallbreak}
\pstart
           \raggedleft{}{\pb}27. 5. 902\pend
           \vspace{0.5em}
\pstart
           lieber, ich freue mich ſehr über den guten Eindruck, den Sie von der
                  \label{K_L02974-1v}\edtext{Novellette\pwindex{Daemmerseele@\emph{Dämmerseele}|pwv}}{\lemma{\textnormal{\emph{Novellette}}}\Cendnote{\textnormal{Siehe Felix Salten an Arthur Schnitzler, 22. 5. 1902.
               }}}\label{K_L02974-1} in d. N. Fr. Pr.\pwindex{Neue Freie Presse@\emph{Neue Freie Presse}|pw} haben; was mir
               eigentlich ſelten paſſiert, – ich war ſelbſt ein bischen unſicher im Urtheil. Daſs
               ſie Schwarzk.\pwindex{Schwarzkopf, Gustav 07.11.1853 – 13.11.1939@\textsc{Schwarzkopf, Gustav} (07.11.1853 – 13.11.1939), \emph{Schriftsteller/Schriftstellerin}|pw} nicht mag, iſt ziemlich
               verſtändlich; – der \label{K_L02974-2v}\edtext{Einwurf Goldm.\pwindex{Goldmann, Paul 31.01.1865 – 25.09.1935@\textsc{Goldmann, Paul} (31.01.1865 – 25.09.1935), \emph{Schriftsteller/Schriftstellerin, Journalist/Journalistin}|pw}: es handle ſich um Liebe}{\lemma{\textnormal{\emph{Einwurf … Liebe}}}\Cendnote{\textnormal{Siehe A. S.: \emph{Tagebuch}, 21. 5. 1902.
               }}}\label{K_L02974-2}, kaum discutirbar; Richard\pwindex{Beer-Hofmann, Richard 1866-07-11 – 1945-09-26@\textsc{Beer-Hofmann, Richard} (1866-07-11 – 1945-09-26), \emph{Schriftsteller/Schriftstellerin}|pw} u Hugo\pwindex{Hofmannsthal, Hugo von 1874-02-01 – 1929-07-15@\textsc{Hofmannsthal, Hugo von} (1874-02-01 – 1929-07-15), \emph{Schriftsteller/Schriftstellerin}|pw} ſcheinen ſie im ganzen gut zu finden, aber
                  {\pb}wie mir ſchien, mit einigem innern
               Widerſtand. Olga\pwindex{Schnitzler, Olga 17.01.1882 – 13.01.1970@\textsc{Schnitzler, Olga} (17.01.1882 – 13.01.1970), \emph{Schauspieler/Schauspielerin, Sänger/Sängerin}|pw} gefiel ſie, als ich ſie ihr
               vorlas, beſonders gut; – die gedruckte hat ſie aber enttäuscht. Meine Bedenken gehen
               nach der Seite des mä\textcolor{gray}{{\geminationn}}lichen {\dotstwo} ich f\textcolor{gray}{i}nde eben kein
               andres Wort – Helden{\dots}, wo mir was zu fehlen ſcheint. Der
               Titel ko{\geminationm}t mir, ſelbſt nach jedem Überdenken Ihrer
               Einwände, nicht un{\pb}glücklich vor. Daſs Sie
               als der erſte den Schluſs nicht als Pointe empfinden, ſondern wohl im Gegentheil
               gerade als den Ausklang ins ungewiſſe, ferne, mit Notwendgkeit weiterflutend,
                  be\textcolor{gray}{rü}hrt mich beſonders angenehm. –\pend
           
\pstart
           Paul G.\pwindex{Goldmann, Paul 31.01.1865 – 25.09.1935@\textsc{Goldmann, Paul} (31.01.1865 – 25.09.1935), \emph{Schriftsteller/Schriftstellerin, Journalist/Journalistin}|pw} ist wieder \label{K_L02974-3v}\edtext{fort}{\lemma{\textnormal{\emph{fort}}}\Cendnote{\textnormal{Paul Goldmann\pwindex{Goldmann, Paul 31.01.1865 – 25.09.1935@\textsc{Goldmann, Paul} (31.01.1865 – 25.09.1935), \emph{Schriftsteller/Schriftstellerin, Journalist/Journalistin}|pwk} war über Pfingsten in Wien\oindex{Wien@\textbf{Wien}, \emph{A.ADM2}|pwk} gewesen.}}}\label{K_L02974-3}; die \label{K_L02974-4v}\edtext{Martin Finder Sachen}{\lemma{\textnormal{\emph{Martin Finder Sachen}}}\Cendnote{\textnormal{Da Salten\pwindex{Salten, Felix 06.09.1869 – 08.10.1945@\textsc{Salten, Felix} (06.09.1869 – 08.10.1945), \emph{Schriftsteller/Schriftstellerin, Journalist/Journalistin, Chefredakteur/Chefredakteurin}|pwk} bis zum 30. 6. 1902 bei der \emph{Wiener Allgemeinen Zeitung}\orgindex{Wiener Allgemeine Zeitung@Wiener Allgemeine Zeitung|pwk} unter Vertrag stand, veröffentlichte er seine
                  Beiträge für die Wochenschrift \emph{Die Zeit}\pwindex{Zeit. Wiener Wochenschrift@\emph{Die Zeit. Wiener Wochenschrift}|pwk} bis
                  dahin unter dem Pseudonym »Martin Finder«, in das nur wenige Personen eingeweiht
                  waren.}}}\label{K_L02974-4} ſind ihm höchlich aufgefallen; – er hat ſich gefragt: Was ko{\geminationm}t da für ein {\pb}{[}»{]}Nachwuchs« – er iſt es, der in d \label{K_L02974-5v}\edtext{N. Fr. Pr.\orgindex{Neue Freie Presse@Neue Freie Presse|pw} mit lebhafteſter Betonung von Ihnen
               ſprach, worauf \textsc{Bened.}\pwindex{Benedikt, Moriz 27.05.1849 – 18.03.1920@\textsc{Benedikt, Moriz} (27.05.1849 – 18.03.1920), \emph{Journalist/Journalistin, Herausgeber/Herausgeberin}|pw} meinte, er dächte ſchon lange Zeit an Sie {\dots} Das will
               natürlich nicht viel heißen; aber ich glaube, we{\geminationn} Sie zu
               irgendwelchen Schritten}{\lemma{\textnormal{\emph{N. Fr. Pr. … Schritten}}}\Cendnote{\textnormal{Vgl. Felix Salten an Arthur Schnitzler, 2[3]. 5. 1902.
               }}}\label{K_L02974-5} ſich entſchlöſſen (über die natürlich noch geſprochen werden muſs), ſo wären
               hier die Chancen, mindeſtens materiell günſtiger als bei der Zeit\orgindex{Zeit@Die Zeit|pw}. Obwohl {\pb}Kanner\pwindex{Kanner, Heinrich 09.11.1864 – 15.02.1930@\textsc{Kanner, Heinrich} (09.11.1864 – 15.02.1930), \emph{Herausgeber/Herausgeberin, Publizist/Publizistin}|pw} zu P. G.\pwindex{Goldmann, Paul 31.01.1865 – 25.09.1935@\textsc{Goldmann, Paul} (31.01.1865 – 25.09.1935), \emph{Schriftsteller/Schriftstellerin, Journalist/Journalistin}|pw}, der auch dort von Ihnen redete, geäußert hat: »\label{K_L02974-6v}\edtext{Er wird ja für uns ſchreiben.}{\lemma{\textnormal{\emph{Er … ſchreiben.}}}\Cendnote{\textnormal{Kanner\pwindex{Kanner, Heinrich 09.11.1864 – 15.02.1930@\textsc{Kanner, Heinrich} (09.11.1864 – 15.02.1930), \emph{Herausgeber/Herausgeberin, Publizist/Publizistin}|pwk} wahrte Saltens\pwindex{Salten, Felix 06.09.1869 – 08.10.1945@\textsc{Salten, Felix} (06.09.1869 – 08.10.1945), \emph{Schriftsteller/Schriftstellerin, Journalist/Journalistin, Chefredakteur/Chefredakteurin}|pwk} Pseudonym und erzählte nicht, dass dieser schon
                  begonnen hatte, für die Wochenschrift \emph{Die Zeit}\orgindex{Zeit. Wiener Wochenschrift@Die Zeit. Wiener Wochenschrift|pwk}
                  zu schreiben. Die Auskunft bezog sich nur auf die anlaufende Gründung der neuen
                     Tageszeitung\pwindex{Zeit@\emph{Die Zeit}|pwkv}, die ab dem
                     27. 9. 1902 erschien.}}}\label{K_L02974-6}« –\pend
           
\pstart
           \textsc{Kainz}\pwindex{Kainz, Josef 02.01.1858 – 20.09.1910@\textsc{Kainz, Josef} (02.01.1858 – 20.09.1910), \emph{Schauspieler/Schauspielerin}|pw} will durchaus im »Weg zum Licht\pwindex{Weg zum Licht. Ein Salzburger Maerchendrama in vier Akten@\emph{Der Weg zum Licht. Ein Salzburger Märchendrama in vier Akten}|pw}« ſpielen;
               u \label{K_L02974-7v}\edtext{Schlenther\pwindex{Schlenther, Paul 20.08.1854 – 30.04.1916@\textsc{Schlenther, Paul} (20.08.1854 – 30.04.1916), \emph{Schriftsteller/Schriftstellerin, Kritiker/Kritikerin, Theaterleiter/Theaterleiterin}|pw}\orgindex{Burgtheater@Burgtheater|pwv} dürfte es daher aufführen}{\lemma{\textnormal{\emph{Schlenther … aufführen}}}\Cendnote{\textnormal{Dazu kam es nicht.}}}\label{K_L02974-7} (So Brahm\pwindex{Brahm, Otto 05.02.1856 – 28.11.1912@\textsc{Brahm, Otto} (05.02.1856 – 28.11.1912), \emph{Theaterleiter/Theaterleiterin, Regisseur/Regisseurin}|pw}.) Es iſt recht
               lächerlich, daſs ein ſolcher Künſtler den \label{K_L02974-8v}\edtext{Hahngikl\pwindex{Weg zum Licht. Ein Salzburger Maerchendrama in vier Akten@\emph{Der Weg zum Licht. Ein Salzburger Märchendrama in vier Akten}|pwv}}{\lemma{\textnormal{\emph{Hahngikl}}}\Cendnote{\textnormal{laut Figurenliste »ein Dunkelelb
                     vom Untersberg«}}}\label{K_L02974-8} dem \label{K_L02974-9v}\edtext{\textsc{Bentivoglio\pwindex{Schleier der Beatrice. Schauspiel in fuenf Akten@\emph{Der Schleier der Beatrice. Schauspiel in fünf Akten}|pwv}}}{\lemma{\textnormal{\emph{Bentivoglio}}}\Cendnote{\textnormal{Hauptfigur von \emph{Der Schleier der Beatrice}\pwindex{Schleier der Beatrice. Schauspiel in fuenf Akten@\emph{Der Schleier der Beatrice. Schauspiel in fünf Akten}|pwk}. Zur Ablehnung des Stücks\pwindex{Schleier der Beatrice. Schauspiel in fuenf Akten@\emph{Der Schleier der Beatrice. Schauspiel in fünf Akten}|pwkv} durch das \emph{Burgtheater}\orgindex{Burgtheater@Burgtheater|pwk}{ }siehe Richard Beer-Hofmann an Arthur Schnitzler, 14. 9. 1900.}}}\label{K_L02974-9} vorzieht;
               aber es liegt wohl recht tief. – Dem \label{K_L02974-10v}\edtext{Deutſch Theater\orgindex{Deutsches Theater Berlin@Deutsches Theater Berlin|pw} geht es hier}{\lemma{\textnormal{\emph{Deutſch … hier}}}\Cendnote{\textnormal{Das \emph{Deutsche Theater Berlin}\orgindex{Deutsches Theater Berlin@Deutsches Theater Berlin|pwk} spielte vom 6. 5. 1902 bis zum
                     zum 5. 6. 1902 im Carl-Theater\oindex{Carl-Theater@\textbf{Carl-Theater}, \emph{Theater (K.THE)}|pwk} in
                     Wien\oindex{Wien@\textbf{Wien}, \emph{A.ADM2}|pwk}
                     ein »Gesammt-Gastpiel«.}}}\label{K_L02974-10} ausgezeichnet. – Der \label{K_L02974-11v}\edtext{Kakadu\pwindex{gruene Kakadu. Groteske in einem Akt@\emph{Der grüne Kakadu. Groteske in einem Akt}|pw}\pwindex{Au Perroquet Vert@\emph{Au Perroquet Vert}|pwv} iſt {\pb}bei Antoine\orgindex{Theâtre Antoine@Théâtre Antoine|pw}}{\lemma{\textnormal{\emph{Kakadu iſt bei Antoine}}}\Cendnote{\textnormal{\emph{Au
                     Perroquet Vert}\pwindex{Au Perroquet Vert@\emph{Au Perroquet Vert}|pwk}, die Übersetzung von \emph{Der
                     grüne Kakadu}\pwindex{gruene Kakadu. Groteske in einem Akt@\emph{Der grüne Kakadu. Groteske in einem Akt}|pwk}, hatte am 7. 11. 1903 am \emph{Théâtre Antoine}\orgindex{Theâtre Antoine@Théâtre Antoine|pwk} Premiere.}}}\label{K_L02974-11} acceptirt. –
                  \label{K_L02974-12v}\edtext{Über die \textsc{Bea.\pwindex{Schleier der Beatrice. Schauspiel in fuenf Akten@\emph{Der Schleier der Beatrice. Schauspiel in fünf Akten}|pw}} ſpricht Brahm\pwindex{Brahm, Otto 05.02.1856 – 28.11.1912@\textsc{Brahm, Otto} (05.02.1856 – 28.11.1912), \emph{Theaterleiter/Theaterleiterin, Regisseur/Regisseurin}|pw} kein Wort}{\lemma{\textnormal{\emph{Über … Wort}}}\Cendnote{\textnormal{Nach der Enttäuschung der Uraufführung von
                     \emph{Der Schleier der Beatrice}\pwindex{Schleier der Beatrice. Schauspiel in fuenf Akten@\emph{Der Schleier der Beatrice. Schauspiel in fünf Akten}|pwk} setzte Schnitzler seine Hoffnungen auf eine
                  Inszenierung am \emph{Deutschen Theater Berlin}\orgindex{Deutsches Theater Berlin@Deutsches Theater Berlin|pwk}.
                  Diese fand am 7. 3. 1903 statt.}}}\label{K_L02974-12}. – Ich überdenke und scenire mein Stück\pwindex{einsame Weg. Schauspiel in fuenf Akten@\emph{Der einsame Weg. Schauspiel in fünf Akten}|pwv} u übe mich indeſs weiter
               im Erzählen!\pend
           
\pstart
           – Sagen Sie mir doch etwas über Ihre Reiſe, Ihre Arbeiten, Ihre Laune. Daſs Hugo\pwindex{Hofmannsthal, Hugo von 1874-02-01 – 1929-07-15@\textsc{Hofmannsthal, Hugo von} (1874-02-01 – 1929-07-15), \emph{Schriftsteller/Schriftstellerin}|pw} ein ganz kleines \label{K_L02974-13v}\edtext{Kind\pwindex{Zimmer, Christiane 14.05.1902 – 05.01.1987@\textsc{Zimmer, Christiane} (14.05.1902 – 05.01.1987)|pwv} beko{\geminationm}en hat, Chriſtiane\pwindex{Zimmer, Christiane 14.05.1902 – 05.01.1987@\textsc{Zimmer, Christiane} (14.05.1902 – 05.01.1987)|pw}}{\lemma{\textnormal{\emph{Kind … Chriſtiane}}}\Cendnote{\textnormal{Christiane von Hofmannsthal\pwindex{Zimmer, Christiane 14.05.1902 – 05.01.1987@\textsc{Zimmer, Christiane} (14.05.1902 – 05.01.1987)|pwk} kam am 14. 5. 1902 auf die Welt.}}}\label{K_L02974-13} genannt, wiſſen Sie
               wohl ſchon. – Heute{ }{\pb}hatten wir beinah einen »Frühlingsabend« –
               lau, ohne Wind und Regen, man faſſt es kaum. – \textsc{\label{K_L02974-14v}\edtext{Rochefort\pwindex{Abenteuer meines Lebens@\emph{Abenteuer meines Lebens}|pwuv}\pwindex{Rochefort, Henri de 1830-01-31 – 1913-06-30@\textsc{Rochefort, Henri de} (1830-01-31 – 1913-06-30), \emph{Schriftsteller/Schriftstellerin, Politiker/Politikerin, Journalist/Journalistin}|pw}}{\lemma{\textnormal{\emph{Rochefort}}}\Cendnote{\textnormal{Es dürfte sich um die (gekürzte)
                     deutschsprachige Ausgabe der Autobiografie von Henri Rochefort\pwindex{Rochefort, Henri de 1830-01-31 – 1913-06-30@\textsc{Rochefort, Henri de} (1830-01-31 – 1913-06-30), \emph{Schriftsteller/Schriftstellerin, Politiker/Politikerin, Journalist/Journalistin}|pwk}: \emph{Les
                        Aventures de ma vie}\pwindex{Aventures de ma vie@\emph{Les Aventures de ma vie}|pwk} (1896) handeln: \emph{Abenteuer meines Lebens}\pwindex{Abenteuer meines Lebens@\emph{Abenteuer meines Lebens}|pwk}. Autorisierte
                        deutsche Bearbeitung von Heinrich
                           Conrad\pwindex{Conrad, Heinrich 19.10.1866 – 1918-12-20@\textsc{Conrad, Heinrich} (19.10.1866 – 1918-12-20), \emph{Übersetzer/Übersetzerin, Romanist/Romanistin}|pwk}. Stuttgart\oindex{Stuttgart@\textbf{Stuttgart}, \emph{P.PPLA}|pwk}: \emph{Robert Lutz}\orgindex{Robert Lutz@Robert Lutz|pwk}{ }1900.}}}\label{K_L02974-14}} wird gegen Schluſs matter; ich beſchäftige mich ein weniges mit \label{K_L02974-15v}\edtext{Botanik}{\lemma{\textnormal{\emph{Botanik}}}\Cendnote{\textnormal{Am 23. 5. 1902 besuchte Schnitzler
                  den Botanischen Garten\oindex{Botanischer Garten der Universitaet@\textbf{Botanischer Garten der Universität}, \emph{Park (K.PRK)}|pwk}.}}}\label{K_L02974-15} und denke
               wieder manchmal mit Wehmut, wie faul ich mein Leben lang war, und auf wie viel
                  beſſer\textcolor{gray}{m} Grund ich {\pb}ſtehen könnte, we{\geminationn} ich nicht gar ſo ſpät auf mich
               aufmerkſam geworden wäre.\pend
           
\pstart
           Leben Sie wohl. Grüßen Sie Florenz\oindex{Florenz@\textbf{Florenz}, \emph{P.PPLA}|pw}, die \textsc{Mediceer} Gräber\oindex{Medici-Kapelle@\textbf{Medici-Kapelle}, \emph{Kirche (K.KRC)}|pw}, den Garten hinter dem Kloſter zu \textsc{Fiesole}\oindex{San Domenico@\textbf{San Domenico}, \emph{Kloster (K.KLS)}|pw} und \textsc{Veronika\pwindex{kleine Veronika@\emph{Die kleine Veronika}|pw}}; – und \label{K_L02974-16v}\edtext{Bern}{\lemma{\textnormal{\emph{Bern}}}\Cendnote{\textnormal{Vgl. Paul Goldmann an Arthur Schnitzler, 17. 4. [1902].
               }}}\label{K_L02974-16} grüßt den andern \label{K_L02974-17v}\edtext{Hund}{\lemma{\textnormal{\emph{Hund}}}\Cendnote{\textnormal{Vgl. Felix Salten an Arthur Schnitzler, 20. 5. 1902.
               }}}\label{K_L02974-17}.\pend
           
\pstart
           Herzlichst Ihr {\\[\baselineskip]}\spacefill\mbox{A.}\pend
           \leftskip=0em{}\selectlanguage{ngerman}\endnumbering\briefempfaengerindex{Salten, Felix@\textsc{Salten, Felix}!zzzSchnitzler, Arthur@\emph{von Arthur Schnitzler}!1902-05-271@{27. 5. 1902}|)be}\mylabel{L02974h}  \normalsize

\doendnotes{C}
\bigskip
\vfill

\clearpage

\footnotesize

\lohead{\textsc{register}}

% Definiere theindex-Environment komplett neu ohne reledmac
\makeatletter
\renewenvironment{theindex}{%
  \section*{\indexname}%
  \setlength{\parindent}{0pt}%
  \setlength{\parskip}{0pt plus 0.3pt}%
  \let\item\@idxitem
}{%
  \clearpage
}
\makeatother

\IfFileExists{\jobname-pw.ind}{\input{\jobname-pw.ind}}{}

\end{document}

      