%% latex-leseansicht-vorspann.tex
%% Vorspann für die Leseansicht.
%% Lädt die gemeinsame Datei latex-vorspann.tex mit nicht gesetztem Schalter.

\newif\ifkorrekturansicht
\korrekturansichtfalse

\input{../tex-inputs/latex-vorspann}


         
         \renewcommand{\erwaehntePersonen}{Personen: Richard Beer-Hofmann, Moriz Benedikt, Otto Brahm, Heinrich Conrad, Paul Goldmann, Hugo von Hofmannsthal, Josef Kainz, Heinrich Kanner, Henri de Rochefort, Felix Salten, Paul Schlenther, Olga Schnitzler, Gustav Schwarzkopf, Christiane Zimmer}
         \renewcommand{\erwaehnteInstitutionen}{Institutionen: Burgtheater, Deutsches Theater Berlin, Die Zeit, Die Zeit. Wiener Wochenschrift, Neue Freie Presse, Robert Lutz, Théâtre Antoine, Wiener Allgemeine Zeitung}
         \renewcommand{\erwaehnteOrte}{Orte: Botan. Garten, Carl-Theater, Florenz, Medici-Kapelle, San Domenico, Stuttgart, Wien}
         \renewcommand{\erwaehnteWerke}{Werke: Abenteuer meines Lebens, Au Perroquet Vert, Der Hund von Florenz, Der Schleier der Beatrice. Schauspiel in fünf Akten, Der Weg zum Licht. Ein Salzburger Märchendrama in vier Akten, Der einsame Weg. Schauspiel in fünf Akten, Der grüne Kakadu. Groteske in einem Akt, Die Zeit, Die Zeit. Wiener Wochenschrift, Die kleine Veronika, Dämmerseele, Les Aventures de ma vie, Neue Freie Presse}
               \section[ Arthur Schnitzler an Felix Salten, 27. 5. 1902]{ Arthur Schnitzler an Felix Salten, 27. 5. 1902}\nopagebreak\mylabel{v}\rehead{ }\begin{ledgroupsized}[t]{13cm}\normalsize\beginnumbering\briefempfaengerindex{Salten, Felix@\textsc{Salten, Felix}!zzzSchnitzler, Arthur@\emph{von Arthur Schnitzler}!1902-05-271@{27. 5. 1902}|(be} \toendnotes[C]{\smallbreak\pagebreak[2]} \Standort{Wienbibliothek im Rathaus, ZPH 1681, 2.1.516.}
\physDesc{Brief, 2 Blätter, 8 Seiten, 2579 Zeichen
\newline{}Handschrift: Bleistift, deutsche Kurrent
\newline{}Ordnung: mit Bleistift von unbekannter Hand Nummerierung der Doppelseiten des
                                 Konvoluts: »62«–»65« }\toendnotes[C]{\smallbreak}\pstart
           \raggedleft{}{\pb}27. 5. 902\pend
           \pstart
           lieber, ich freue mich ſehr über den guten Eindruck, den Sie von der
                  \label{K_L02974-1v}\edtext{Novellette\pwindex{Schnitzler, Arthur 15.05.1862 – 21.10.1931@\textsc{Schnitzler, Arthur} (15.05.1862 – 21.10.1931), \emph{Schriftsteller, Mediziner}!Daemmerseele18. 5. 1902@\strich\emph{Dämmerseele} {[}18. 5. 1902{]}|pwv}}{\lemma{\textnormal{\emph{Novellette}}}\Cendnote{\textnormal{Siehe Felix Salten an Arthur Schnitzler, 22. 5. 1902.
               }}}\label{K_L02974-1h} in d. N. Fr. Pr.\pwindex{Neue Freie Presse1864 – 1939@\emph{Neue Freie Presse} {[}1864 – 1939{]}|pw} haben; was mir
               eigentlich ſelten paſſiert, – ich war ſelbſt ein bischen unſicher im Urtheil. Daſs
               ſie Schwarzk.\pwindex{Schwarzkopf, Gustav 07.11.1853 – 13.11.1939@\textsc{Schwarzkopf, Gustav} (07.11.1853 – 13.11.1939), \emph{Schriftsteller}|pw} nicht mag, iſt ziemlich
               verſtändlich; – der \label{K_L02974-2v}\edtext{Einwurf Goldm.\pwindex{Goldmann, Paul 31.01.1865 – 25.09.1935@\textsc{Goldmann, Paul} (31.01.1865 – 25.09.1935), \emph{Schriftsteller, Journalist}|pw}: es handle ſich um Liebe}{\lemma{\textnormal{\emph{Einwurf … Liebe}}}\Cendnote{\textnormal{Siehe A. S.: \emph{Tagebuch}, 21. 5. 1902.
               }}}\label{K_L02974-2h}, kaum discutirbar; Richard\pwindex{Beer-Hofmann, Richard 1866-07-11 – 1945-09-26@\textsc{Beer-Hofmann, Richard} (1866-07-11 – 1945-09-26), \emph{Schriftsteller}|pw} u Hugo\pwindex{Hofmannsthal, Hugo von 1874-02-01 – 1929-07-15@\textsc{Hofmannsthal, Hugo von} (1874-02-01 – 1929-07-15), \emph{Schriftsteller}|pw} ſcheinen ſie im ganzen gut zu finden, aber
                  {\pb}wie mir ſchien, mit einigem innern
               Widerſtand. Olga\pwindex{Schnitzler, Olga 17.01.1882 – 13.01.1970@\textsc{Schnitzler, Olga} (17.01.1882 – 13.01.1970), \emph{Schauspielerin, Sängerin}|pw} gefiel ſie, als ich ſie ihr
               vorlas, beſonders gut; – die gedruckte hat ſie aber enttäuscht. Meine Bedenken gehen
               nach der Seite des mä\textcolor{gray}{{\geminationn}}lichen {\dotstwo} ich f\textcolor{gray}{i}nde eben kein
               andres Wort – Helden{\dots}, wo mir was zu fehlen ſcheint. Der
               Titel ko{\geminationm}t mir, ſelbſt nach jedem Überdenken Ihrer
               Einwände, nicht un{\pb}glücklich vor. Daſs Sie
               als der erſte den Schluſs nicht als Pointe empfinden, ſondern wohl im Gegentheil
               gerade als den Ausklang ins ungewiſſe, ferne, mit Notwendgkeit weiterflutend,
                  be\textcolor{gray}{rü}hrt mich beſonders angenehm. –\pend
           \pstart
           Paul G.\pwindex{Goldmann, Paul 31.01.1865 – 25.09.1935@\textsc{Goldmann, Paul} (31.01.1865 – 25.09.1935), \emph{Schriftsteller, Journalist}|pw} ist wieder \label{K_L02974-3v}\edtext{fort}{\lemma{\textnormal{\emph{fort}}}\Cendnote{\textnormal{Paul Goldmann\pwindex{Goldmann, Paul 31.01.1865 – 25.09.1935@\textsc{Goldmann, Paul} (31.01.1865 – 25.09.1935), \emph{Schriftsteller, Journalist}|pwk} war über Pfingsten in Wien\oindex{Wien@\textbf{Wien}|pwk} gewesen.}}}\label{K_L02974-3h}; die \label{K_L02974-4v}\edtext{Martin Finder Sachen}{\lemma{\textnormal{\emph{Martin Finder Sachen}}}\Cendnote{\textnormal{Da Salten\pwindex{Salten, Felix 06.09.1869 – 08.10.1945@\textsc{Salten, Felix} (06.09.1869 – 08.10.1945), \emph{Schriftsteller, Journalist}|pwk} bis zum 30. 6. 1902 bei der \emph{Wiener Allgemeinen Zeitung}\orgindex{Wiener Allgemeine Zeitung@Wiener Allgemeine Zeitung|pwk} unter Vertrag stand, veröffentlichte er seine
                  Beiträge für die Wochenschrift \emph{Die Zeit}\pwindex{Zeit. Wiener Wochenschrift1894 – 1904@\emph{Die Zeit. Wiener Wochenschrift} {[}1894 – 1904{]}|pwk} bis
                  dahin unter dem Pseudonym »Martin Finder«, in das nur wenige Personen eingeweiht
                  waren.}}}\label{K_L02974-4h} ſind ihm höchlich aufgefallen; – er hat ſich gefragt: Was ko{\geminationm}t da für ein {\pb}{[}»{]}Nachwuchs« – er iſt es, der in d \label{K_L02974-5v}\edtext{N. Fr. Pr.\orgindex{Neue Freie Presse@Neue Freie Presse|pw} mit lebhafteſter Betonung von Ihnen
               ſprach, worauf \textsc{Bened.}\pwindex{Benedikt, Moriz 27.05.1849 – 18.03.1920@\textsc{Benedikt, Moriz} (27.05.1849 – 18.03.1920), \emph{Journalist, Herausgeber}|pw} meinte, er dächte ſchon lange Zeit an Sie {\dots} Das will
               natürlich nicht viel heißen; aber ich glaube, we{\geminationn} Sie zu
               irgendwelchen Schritten}{\lemma{\textnormal{\emph{N. Fr. Pr. … Schritten}}}\Cendnote{\textnormal{Vgl. Felix Salten an Arthur Schnitzler, 2[3]. 5. 1902.
               }}}\label{K_L02974-5h} ſich entſchlöſſen (über die natürlich noch geſprochen werden muſs), ſo wären
               hier die Chancen, mindeſtens materiell günſtiger als bei der Zeit\orgindex{Zeit@Die Zeit|pw}. Obwohl {\pb}Kanner\pwindex{Kanner, Heinrich 09.11.1864 – 15.02.1930@\textsc{Kanner, Heinrich} (09.11.1864 – 15.02.1930), \emph{Herausgeber, Publizist}|pw} zu P. G.\pwindex{Goldmann, Paul 31.01.1865 – 25.09.1935@\textsc{Goldmann, Paul} (31.01.1865 – 25.09.1935), \emph{Schriftsteller, Journalist}|pw}, der auch dort von Ihnen redete, geäußert hat: »\label{K_L02974-6v}\edtext{Er wird ja für uns ſchreiben.}{\lemma{\textnormal{\emph{Er … ſchreiben.}}}\Cendnote{\textnormal{Kanner\pwindex{Kanner, Heinrich 09.11.1864 – 15.02.1930@\textsc{Kanner, Heinrich} (09.11.1864 – 15.02.1930), \emph{Herausgeber, Publizist}|pwk} wahrte Saltens\pwindex{Salten, Felix 06.09.1869 – 08.10.1945@\textsc{Salten, Felix} (06.09.1869 – 08.10.1945), \emph{Schriftsteller, Journalist}|pwk} Pseudonym und erzählte nicht, dass dieser schon
                  begonnen hatte, für die Wochenschrift \emph{Die Zeit}\orgindex{Zeit. Wiener Wochenschrift@Die Zeit. Wiener Wochenschrift|pwk}
                  zu schreiben. Die Auskunft bezog sich nur auf die anlaufende Gründung der neuen
                     Tageszeitung\pwindex{Zeit1902-09-27 – 1919@\emph{Die Zeit} {[}1902-09-27 – 1919{]}|pwkv}, die ab dem
                     27. 9. 1902 erschien.}}}\label{K_L02974-6h}« –\pend
           \pstart
           \textsc{Kainz}\pwindex{Kainz, Josef 02.01.1858 – 20.09.1910@\textsc{Kainz, Josef} (02.01.1858 – 20.09.1910), \emph{Schauspieler}|pw} will durchaus im »Weg zum Licht\pwindex{\textcolor{red}{\textsuperscript{XXXX1 indx}}!Weg zum Licht. Ein Salzburger Maerchendrama in vier Akten1902-04-05@\strich\emph{Der Weg zum Licht. Ein Salzburger Märchendrama in vier Akten} {[}1902-04-05{]}|pw}« ſpielen;
               u \label{K_L02974-7v}\edtext{Schlenther\pwindex{Schlenther, Paul 20.08.1854 – 30.04.1916@\textsc{Schlenther, Paul} (20.08.1854 – 30.04.1916), \emph{Schriftsteller, Kritiker, Theaterleiter}|pw}\orgindex{Burgtheater@Burgtheater|pwv} dürfte es daher aufführen}{\lemma{\textnormal{\emph{Schlenther … aufführen}}}\Cendnote{\textnormal{Dazu kam es nicht.}}}\label{K_L02974-7h} (So Brahm\pwindex{Brahm, Otto 05.02.1856 – 28.11.1912@\textsc{Brahm, Otto} (05.02.1856 – 28.11.1912), \emph{Theaterleiter, Regisseur}|pw}.) Es iſt recht
               lächerlich, daſs ein ſolcher Künſtler den \label{K_L02974-8v}\edtext{Hahngikl\pwindex{\textcolor{red}{\textsuperscript{XXXX1 indx}}!Weg zum Licht. Ein Salzburger Maerchendrama in vier Akten1902-04-05@\strich\emph{Der Weg zum Licht. Ein Salzburger Märchendrama in vier Akten} {[}1902-04-05{]}|pwv}}{\lemma{\textnormal{\emph{Hahngikl}}}\Cendnote{\textnormal{laut Figurenliste »ein Dunkelelb
                     vom Untersberg«}}}\label{K_L02974-8h} dem \label{K_L02974-9v}\edtext{\textsc{Bentivoglio\pwindex{Schnitzler, Arthur 15.05.1862 – 21.10.1931@\textsc{Schnitzler, Arthur} (15.05.1862 – 21.10.1931), \emph{Schriftsteller, Mediziner}!Schleier der Beatrice. Schauspiel in fuenf Akten1900-12-01@\strich\emph{Der Schleier der Beatrice. Schauspiel in fünf Akten} {[}1900-12-01{]}|pwv}}}{\lemma{\textnormal{\emph{Bentivoglio}}}\Cendnote{\textnormal{Hauptfigur von \emph{Der Schleier der Beatrice}\pwindex{Schnitzler, Arthur 15.05.1862 – 21.10.1931@\textsc{Schnitzler, Arthur} (15.05.1862 – 21.10.1931), \emph{Schriftsteller, Mediziner}!Schleier der Beatrice. Schauspiel in fuenf Akten1900-12-01@\strich\emph{Der Schleier der Beatrice. Schauspiel in fünf Akten} {[}1900-12-01{]}|pwk}. Zur Ablehnung des Stücks\pwindex{Schnitzler, Arthur 15.05.1862 – 21.10.1931@\textsc{Schnitzler, Arthur} (15.05.1862 – 21.10.1931), \emph{Schriftsteller, Mediziner}!Schleier der Beatrice. Schauspiel in fuenf Akten1900-12-01@\strich\emph{Der Schleier der Beatrice. Schauspiel in fünf Akten} {[}1900-12-01{]}|pwkv} durch das \emph{Burgtheater}\orgindex{Burgtheater@Burgtheater|pwk}{ }siehe Richard Beer-Hofmann an Arthur Schnitzler, 14. 9. 1900.}}}\label{K_L02974-9h} vorzieht;
               aber es liegt wohl recht tief. – Dem \label{K_L02974-10v}\edtext{Deutſch Theater\orgindex{Deutsches Theater Berlin@Deutsches Theater Berlin|pw} geht es hier}{\lemma{\textnormal{\emph{Deutſch … hier}}}\Cendnote{\textnormal{Das \emph{Deutsche Theater Berlin}\orgindex{Deutsches Theater Berlin@Deutsches Theater Berlin|pwk} spielte vom 6. 5. 1902 bis zum
                     zum 5. 6. 1902 im Carl-Theater\oindex{Carl-Theater@\textbf{Carl-Theater}|pwk} in
                     Wien\oindex{Wien@\textbf{Wien}|pwk}
                     ein »Gesammt-Gastpiel«.}}}\label{K_L02974-10h} ausgezeichnet. – Der \label{K_L02974-11v}\edtext{Kakadu\pwindex{Schnitzler, Arthur 15.05.1862 – 21.10.1931@\textsc{Schnitzler, Arthur} (15.05.1862 – 21.10.1931), \emph{Schriftsteller, Mediziner}!gruene Kakadu. Groteske in einem Akt1. 3. 1899@\strich\emph{Der grüne Kakadu. Groteske in einem Akt} {[}1. 3. 1899{]}|pw}\pwindex{Schnitzler, Arthur 15.05.1862 – 21.10.1931@\textsc{Schnitzler, Arthur} (15.05.1862 – 21.10.1931), \emph{Schriftsteller, Mediziner}!gruene Kakadu. Groteske in einem Akt1. 3. 1899@\strich\emph{Der grüne Kakadu. Groteske in einem Akt} {[}1. 3. 1899{]}|pw}\pwindex{Schnitzler, Arthur 15.05.1862 – 21.10.1931@\textsc{Schnitzler, Arthur} (15.05.1862 – 21.10.1931), \emph{Schriftsteller, Mediziner}!Au Perroquet Vert1903-11-07@\strich\emph{Au Perroquet Vert} {[}1903-11-07{]}|pwv} iſt {\pb}bei Antoine\orgindex{Theâtre Antoine@Théâtre Antoine|pw}}{\lemma{\textnormal{\emph{Kakadu iſt bei Antoine}}}\Cendnote{\textnormal{\emph{Au
                     Perroquet Vert}\pwindex{Schnitzler, Arthur 15.05.1862 – 21.10.1931@\textsc{Schnitzler, Arthur} (15.05.1862 – 21.10.1931), \emph{Schriftsteller, Mediziner}!Au Perroquet Vert1903-11-07@\strich\emph{Au Perroquet Vert} {[}1903-11-07{]}|pwk}, die Übersetzung von \emph{Der
                     grüne Kakadu}\pwindex{Schnitzler, Arthur 15.05.1862 – 21.10.1931@\textsc{Schnitzler, Arthur} (15.05.1862 – 21.10.1931), \emph{Schriftsteller, Mediziner}!gruene Kakadu. Groteske in einem Akt1. 3. 1899@\strich\emph{Der grüne Kakadu. Groteske in einem Akt} {[}1. 3. 1899{]}|pwk}\pwindex{Schnitzler, Arthur 15.05.1862 – 21.10.1931@\textsc{Schnitzler, Arthur} (15.05.1862 – 21.10.1931), \emph{Schriftsteller, Mediziner}!gruene Kakadu. Groteske in einem Akt1. 3. 1899@\strich\emph{Der grüne Kakadu. Groteske in einem Akt} {[}1. 3. 1899{]}|pwk}, hatte am 7. 11. 1903 am \emph{Théâtre Antoine}\orgindex{Theâtre Antoine@Théâtre Antoine|pwk} Premiere.}}}\label{K_L02974-11h} acceptirt. –
                  \label{K_L02974-12v}\edtext{Über die \textsc{Bea.\pwindex{Schnitzler, Arthur 15.05.1862 – 21.10.1931@\textsc{Schnitzler, Arthur} (15.05.1862 – 21.10.1931), \emph{Schriftsteller, Mediziner}!Schleier der Beatrice. Schauspiel in fuenf Akten1900-12-01@\strich\emph{Der Schleier der Beatrice. Schauspiel in fünf Akten} {[}1900-12-01{]}|pw}} ſpricht Brahm\pwindex{Brahm, Otto 05.02.1856 – 28.11.1912@\textsc{Brahm, Otto} (05.02.1856 – 28.11.1912), \emph{Theaterleiter, Regisseur}|pw} kein Wort}{\lemma{\textnormal{\emph{Über … Wort}}}\Cendnote{\textnormal{Nach der Enttäuschung der Uraufführung von
                     \emph{Der Schleier der Beatrice}\pwindex{Schnitzler, Arthur 15.05.1862 – 21.10.1931@\textsc{Schnitzler, Arthur} (15.05.1862 – 21.10.1931), \emph{Schriftsteller, Mediziner}!Schleier der Beatrice. Schauspiel in fuenf Akten1900-12-01@\strich\emph{Der Schleier der Beatrice. Schauspiel in fünf Akten} {[}1900-12-01{]}|pwk} setzte Schnitzler\pwindex{Schnitzler, Arthur 15.05.1862 – 21.10.1931@\textsc{Schnitzler, Arthur} (15.05.1862 – 21.10.1931), \emph{Schriftsteller, Mediziner}|pwk} seine Hoffnungen auf eine
                  Inszenierung am \emph{Deutschen Theater Berlin}\orgindex{Deutsches Theater Berlin@Deutsches Theater Berlin|pwk}.
                  Diese fand am 7. 3. 1903 statt.}}}\label{K_L02974-12h}. – Ich überdenke und scenire mein Stück\pwindex{Schnitzler, Arthur 15.05.1862 – 21.10.1931@\textsc{Schnitzler, Arthur} (15.05.1862 – 21.10.1931), \emph{Schriftsteller, Mediziner}!einsame Weg. Schauspiel in fuenf Akten1904@\strich\emph{Der einsame Weg. Schauspiel in fünf Akten} {[}1904{]}|pwv} u übe mich indeſs weiter
               im Erzählen!\pend
           \pstart
           – Sagen Sie mir doch etwas über Ihre Reiſe, Ihre Arbeiten, Ihre Laune. Daſs Hugo\pwindex{Hofmannsthal, Hugo von 1874-02-01 – 1929-07-15@\textsc{Hofmannsthal, Hugo von} (1874-02-01 – 1929-07-15), \emph{Schriftsteller}|pw} ein ganz kleines \label{K_L02974-13v}\edtext{Kind\pwindex{Zimmer, Christiane 14.05.1902 – 05.01.1987@\textsc{Zimmer, Christiane} (14.05.1902 – 05.01.1987)|pwv} beko{\geminationm}en hat, Chriſtiane\pwindex{Zimmer, Christiane 14.05.1902 – 05.01.1987@\textsc{Zimmer, Christiane} (14.05.1902 – 05.01.1987)|pw}}{\lemma{\textnormal{\emph{Kind … Chriſtiane}}}\Cendnote{\textnormal{Christiane von Hofmannsthal\pwindex{Zimmer, Christiane 14.05.1902 – 05.01.1987@\textsc{Zimmer, Christiane} (14.05.1902 – 05.01.1987)|pwk} kam am 14. 5. 1902 auf die Welt.}}}\label{K_L02974-13h} genannt, wiſſen Sie
               wohl ſchon. – Heute{ }{\pb}hatten wir beinah einen »Frühlingsabend« –
               lau, ohne Wind und Regen, man faſſt es kaum. – \textsc{\label{K_L02974-14v}\edtext{Rochefort\pwindex{Rochefort, Henri de 1830-01-31 – 1913-06-30@\textsc{Rochefort, Henri de} (1830-01-31 – 1913-06-30), \emph{Schriftsteller, Politiker, Journalist}!Abenteuer meines Lebens1900-11-02@\strich\emph{Abenteuer meines Lebens} {[}1900-11-02{]}|pwuv}\pwindex{Rochefort, Henri de 1830-01-31 – 1913-06-30@\textsc{Rochefort, Henri de} (1830-01-31 – 1913-06-30), \emph{Schriftsteller, Politiker, Journalist}|pw}}{\lemma{\textnormal{\emph{Rochefort}}}\Cendnote{\textnormal{Es dürfte sich um die (gekürzte)
                     deutschsprachige Ausgabe der Autobiografie von Henri Rochefort\pwindex{Rochefort, Henri de 1830-01-31 – 1913-06-30@\textsc{Rochefort, Henri de} (1830-01-31 – 1913-06-30), \emph{Schriftsteller, Politiker, Journalist}|pwk}: \emph{Les
                        Aventures de ma vie}\pwindex{Rochefort, Henri de 1830-01-31 – 1913-06-30@\textsc{Rochefort, Henri de} (1830-01-31 – 1913-06-30), \emph{Schriftsteller, Politiker, Journalist}!Aventures de ma vie1896@\strich\emph{Les Aventures de ma vie} {[}1896{]}|pwk} (1896) handeln: \emph{Abenteuer meines Lebens}\pwindex{Rochefort, Henri de 1830-01-31 – 1913-06-30@\textsc{Rochefort, Henri de} (1830-01-31 – 1913-06-30), \emph{Schriftsteller, Politiker, Journalist}!Abenteuer meines Lebens1900-11-02@\strich\emph{Abenteuer meines Lebens} {[}1900-11-02{]}|pwk}. Autorisierte
                        deutsche Bearbeitung von Heinrich
                           Conrad\pwindex{Conrad, Heinrich 19.10.1866 – 1918-12-20@\textsc{Conrad, Heinrich} (19.10.1866 – 1918-12-20), \emph{Übersetzer, Romanist}|pwk}. Stuttgart\oindex{Stuttgart@\textbf{Stuttgart}|pwk}: \emph{Robert Lutz}\orgindex{Robert Lutz@Robert Lutz|pwk}{ }1900.}}}\label{K_L02974-14h}} wird gegen Schluſs matter; ich beſchäftige mich ein weniges mit \label{K_L02974-15v}\edtext{Botanik}{\lemma{\textnormal{\emph{Botanik}}}\Cendnote{\textnormal{Am 23. 5. 1902 besuchte Schnitzler\pwindex{Schnitzler, Arthur 15.05.1862 – 21.10.1931@\textsc{Schnitzler, Arthur} (15.05.1862 – 21.10.1931), \emph{Schriftsteller, Mediziner}|pwk}
                  den Botanischen Garten\oindex{Botan. Garten@\textbf{Botan. Garten}|pwk}.}}}\label{K_L02974-15h} und denke
               wieder manchmal mit Wehmut, wie faul ich mein Leben lang war, und auf wie viel
                  beſſer\textcolor{gray}{m} Grund ich {\pb}ſtehen könnte, we{\geminationn} ich nicht gar ſo ſpät auf mich
               aufmerkſam geworden wäre.\pend
           \pstart
           Leben Sie wohl. Grüßen Sie Florenz\oindex{Florenz@\textbf{Florenz}|pw}, die \textsc{Mediceer} Gräber\oindex{Medici-Kapelle@\textbf{Medici-Kapelle}|pw}, den Garten hinter dem Kloſter zu \textsc{Fiesole}\oindex{San Domenico@\textbf{San Domenico}|pw} und \textsc{Veronika\pwindex{Salten, Felix 06.09.1869 – 08.10.1945@\textsc{Salten, Felix} (06.09.1869 – 08.10.1945), \emph{Schriftsteller, Journalist}!kleine Veronika1902-12-01@\strich\emph{Die kleine Veronika} {[}1902-12-01{]}|pw}}; – und \label{K_L02974-16v}\edtext{Bern}{\lemma{\textnormal{\emph{Bern}}}\Cendnote{\textnormal{Vgl. Paul Goldmann an Arthur Schnitzler, 17. 4. [1902].
               }}}\label{K_L02974-16h} grüßt den andern \label{K_L02974-17v}\edtext{Hund}{\lemma{\textnormal{\emph{Hund}}}\Cendnote{\textnormal{Vgl. Felix Salten an Arthur Schnitzler, 20. 5. 1902.
               }}}\label{K_L02974-17h}.\pend
           \pstart
           Herzlichst Ihr {\\[\baselineskip]}\spacefill\mbox{A.}\pend
           \leftskip=0em{}
         
         \endnumbering\mylabel{h}\end{ledgroupsized}  \newcommand{\dateiname}{L02974}\newcommand{\titel}{Arthur Schnitzler an Felix Salten, 27. 5. 1902}\newcommand{\editorInnen}{Martin Anton Müller und Laura Untner}%% latex-leseansicht-abspann.tex
%% Abspann für die Leseansicht.
%% Der Schalter \ifkorrekturansicht ist bereits durch den Vorspann gesetzt.

%% latex-abspann.tex
%% Gemeinsamer Abspann für Korrekturansicht und Leseansicht.
%% Setzt den Schalter \ifkorrekturansicht voraus (gesetzt in den
%% einbindenden Dateien latex-korrekturansicht-abspann.tex bzw.
%% latex-leseansicht-abspann.tex).
%% ---------------------------------------------------------------

\normalsize

% Das esempio-Environment wird nur in der Leseansicht benötigt
\ifkorrekturansicht\else
\newenvironment{esempio}[3]%
{
    \vspace{1.5ex}
    \rlap{\underline{#1}}
    \par
    \setlength{\parindent}{0cm}
    \nopagebreak
    \leftskip=#2cm
    \rightskip=#3cm
}
{
    \par
}
\fi

\doendnotes{C}
\bigskip
\vfill

\clearpage

\footnotesize

\ifkorrekturansicht
  \lohead{\textsc{register}}
\fi

% theindex-Environment neu definieren ohne reledmac
\makeatletter
\renewenvironment{theindex}{%
  \ifkorrekturansicht
    \section*{\indexname}%
  \else
    \subsubsection*{Index der erwähnten Entitäten}%
  \fi
  \setlength{\parindent}{0pt}%
  \setlength{\parskip}{0pt plus 0.3pt}%
  \let\item\@idxitem
}{%
  \ifkorrekturansicht\clearpage\fi
}
\makeatother

\IfFileExists{\jobname-pw.ind}{\input{\jobname-pw.ind}}{}

% Quellenangabe nur in der Leseansicht
\ifkorrekturansicht\else
% Fallback-Definitionen, falls die .tex-Datei \titel etc. nicht gesetzt hat
\providecommand{\titel}{}
\providecommand{\editorInnen}{}
\providecommand{\dateiname}{\jobname}

\vspace{3cm}

\vfill

\footnotesize
\textsc{Quelle}: \titel. Herausgegeben von {\editorInnen}. In: \emph{Arthur Schnitzler: Briefwechsel mit Autorinnen und Autoren}.
 Digitale Edition, https://schnitzler-briefe.acdh.oeaw.ac.at/{\dateiname}.html (Stand \today)
\fi

\end{document}


      