%% latex-leseansicht-vorspann.tex
%% Vorspann für die Leseansicht.
%% Lädt die gemeinsame Datei latex-vorspann.tex mit nicht gesetztem Schalter.

\newif\ifkorrekturansicht
\korrekturansichtfalse

\input{../tex-inputs/latex-vorspann}

\begin{center}
            \textcolor{red}{ENTWURF, NICHT FERTIG KORRIGIERT}
                      \end{center}
            
         
         \renewcommand{\erwaehntePersonen}{Personen: Richard Beer-Hofmann, Moriz Benedikt, Otto Brahm, Paul Goldmann, Hugo von Hofmannsthal, Christiane von Hofmannsthal, Josef Kainz, Heinrich Kanner, Felix Salten, Olga Schnitzler}
         \renewcommand{\erwaehnteInstitutionen}{Institutionen: Deutsches Theater Berlin, Die Zeit, Neue Freie Presse, Théâtre Antoine}
         \renewcommand{\erwaehnteOrte}{Orte: Florenz, Medici-Kapelle, San Domenico, Wien}
         \renewcommand{\erwaehnteWerke}{Werke: Der Schleier der Beatrice. Schauspiel in fünf Akten, Der grüne Kakadu. Groteske in einem Akt}
               \section[Arthur Schnitzler an Felix Salten, 27. 5. 1902]{ Arthur Schnitzler an Felix Salten, 27. 5. 1902}\nopagebreak\mylabel{v}\rehead{ }\begin{ledgroupsized}[t]{13cm}\normalsize\beginnumbering \toendnotes[C]{\smallbreak\pagebreak[2]} \Standort{Wienbibliothek im Rathaus, ZPH 1681, 2.1.516.}
\physDesc{
\newline{}Handschrift: , deutsche Kurrent}\toendnotes[C]{\smallbreak}\pstart
           \raggedleft{}{\pb}27. 5. 902\pend
           \pstart
           lieber, ich freu mich ſehr über den guten Eindruck, den Sie von der
                  Novellette\textcolor{red}{\textsuperscript{\textbf{KEY}}} in d. N. Fr. Pr.\textcolor{red}{\textsuperscript{\textbf{KEY}}}
               haben; was mir eigentlich ſelten geſchie\textcolor{gray}{h}t, – ich war ſelbſt ein
               bischen unſicher im Urtheil daſs ſie Schwarzk.\pwindex{\textcolor{red}{\textsuperscript{XXXX1 indx}}|pw}
               nicht mag, iſt ziemlich verſtändlich; – der Einwurf Goldm.\pwindex{Goldmann, Paul 31.01.1865 – 25.09.1935@\textsc{Goldmann, Paul} (31.01.1865 – 25.09.1935), \emph{Schriftsteller, Journalist}|pw}: es handle ſich um Liebe, kaum discutirbar; Richard\pwindex{Beer-Hofmann, Richard 1866-07-11 – 1945-09-26@\textsc{Beer-Hofmann, Richard} (1866-07-11 – 1945-09-26), \emph{Schriftsteller}|pw} u Hugo\pwindex{Hofmannsthal, Hugo von 1874-02-01 – 1929-07-15@\textsc{Hofmannsthal, Hugo von} (1874-02-01 – 1929-07-15), \emph{Schriftsteller}|pw}
               ſcheinen ſie im ganzen gut zu finden, aber {\pb}wie mir ſchien, mit einigem innern
               Widerſtand. Olga\pwindex{Schnitzler, Olga 17.01.1882 – 13.01.1970@\textsc{Schnitzler, Olga} (17.01.1882 – 13.01.1970), \emph{Schauspielerin, Sängerin}|pw} gefiel ſie, als ich ſie ihr vorlas,
               beſonders gut; – die gedruckte hat ſie aber enttäuscht. Meine Bedenken gehen nahc der
               Seite des mä{\geminationn}lichen {\dotstwo} ich finde eben kein andres Wort –
               Helden{\dotstwo}, wo mir was zu fehlen ſcheint. Der Titel ko{\geminationm}t
               mir ſelbſt nach jedem Überdenken Ihrer Einwände, nicht un {\pb}glücklich vor. Daſs Sie als der
               erſte den Schluſs nicht als Pointe empfiden, ſondern wohl im Gegentheil gerade als
               den Ausklang ins ungewiſſe, ferne, mit Notwendgkeit weiterflutend,
                  be\textcolor{gray}{rü}hrt mich beſonders angenehm.– \pend
           \pstart
           Paul G.\pwindex{Goldmann, Paul 31.01.1865 – 25.09.1935@\textsc{Goldmann, Paul} (31.01.1865 – 25.09.1935), \emph{Schriftsteller, Journalist}|pw} ist wieder fort; die Martin Finder
               Sachen ſind ihm höchlich aufgefallen;– er hat ſich gefragt: Was ko{\geminationm}t da für ein {\pb}Nachwuchs« – er iſt es, der in
               d. N. Fr. Pr.\orgindex{Neue Freie Presse@Neue Freie Presse|pw} mit lebhafteſter Betonung von Ihnen ſprach,
               worauf Bened.\pwindex{Benedikt, Moriz 27.05.1849 – 18.03.1920@\textsc{Benedikt, Moriz} (27.05.1849 – 18.03.1920), \emph{Journalist, Herausgeber}|pw} meinte, er dächte ſchon lange Zeit an Sie{\dots} Das will natürlich nicht
               viel heißen; aber ich glaube, we{\geminationn} Sie zu irgend welchen
               Schritten ſich entſchlöſſen (über die natürlich noch geſprochen werden muſs), ſo
               wären hier die Chancen, mindeſtens materiell günſtiger als bei der Zeit\orgindex{Zeit@Die Zeit|pw}. Obwohl {\pb}Kanner\pwindex{Kanner, Heinrich 09.11.1864 – 15.02.1930@\textsc{Kanner, Heinrich} (09.11.1864 – 15.02.1930), \emph{Herausgeber, Publizist}|pw} zu P. G.\pwindex{Goldmann, Paul 31.01.1865 – 25.09.1935@\textsc{Goldmann, Paul} (31.01.1865 – 25.09.1935), \emph{Schriftsteller, Journalist}|pw}g, der auch
               dort von Ihnen redete, geäußert hat: »Er wird ja für uns ſchreiben.«– \pend
           \pstart
           Kainz\pwindex{Kainz, Josef 02.01.1858 – 20.09.1910@\textsc{Kainz, Josef} (02.01.1858 – 20.09.1910), \emph{Schauspieler}|pw} will durchaus Im »Weg
                  zum Licht\textcolor{red}{\textsuperscript{\textbf{KEY}}}« ſpielen; u Schlenther\textcolor{red}{\textsuperscript{\textbf{KEY}}} dürfte es daher aufführen (So Brahm\pwindex{Brahm, Otto 05.02.1856 – 28.11.1912@\textsc{Brahm, Otto} (05.02.1856 – 28.11.1912), \emph{Theaterleiter, Regisseur}|pw}.) Es iſt recht lächerlich, daſs ein ſolcher Künſtler den Hahngikl dem
                  \textsc{Bentivoglio\pwindex{Schnitzler, Arthur 15.05.1862 – 21.10.1931@\textsc{Schnitzler, Arthur} (15.05.1862 – 21.10.1931), \emph{Schriftsteller, Mediziner}!Schleier der Beatrice. Schauspiel in fuenf Akten1900-12-01@\strich\emph{Der Schleier der Beatrice. Schauspiel in fünf Akten} {[}1900-12-01{]}|pwv}} vorzieht; aber es liegt wohl recht tief.– Dem Deutſchen
                     Theater\orgindex{Deutsches Theater Berlin@Deutsches Theater Berlin|pw} geht es hier ausgezeichnet. – Der Kakadu\pwindex{Schnitzler, Arthur 15.05.1862 – 21.10.1931@\textsc{Schnitzler, Arthur} (15.05.1862 – 21.10.1931), \emph{Schriftsteller, Mediziner}!gruene Kakadu. Groteske in einem Akt1. 3. 1899@\strich\emph{Der grüne Kakadu. Groteske in einem Akt} {[}1. 3. 1899{]}|pw}
               iſt {\pb}bei Antoine\orgindex{Theâtre Antoine@Théâtre Antoine|pw} acceptirt. Über die \textsc{Bea.\pwindex{Schnitzler, Arthur 15.05.1862 – 21.10.1931@\textsc{Schnitzler, Arthur} (15.05.1862 – 21.10.1931), \emph{Schriftsteller, Mediziner}!Schleier der Beatrice. Schauspiel in fuenf Akten1900-12-01@\strich\emph{Der Schleier der Beatrice. Schauspiel in fünf Akten} {[}1900-12-01{]}|pw}} ſpricht Brahm\pwindex{Brahm, Otto 05.02.1856 – 28.11.1912@\textsc{Brahm, Otto} (05.02.1856 – 28.11.1912), \emph{Theaterleiter, Regisseur}|pw} kein Wort.– Ich überdenke
               und scenire mein Stück\textcolor{red}{\textsuperscript{\textbf{KEY}}} u übe mich
               indeſs weiter im Erzählen. \pend
           \pstart
           – Sagen Sie mir doch etwas über Ihre Reiſe, Ihre Arbeiten, Ihre Laune. Daſs Hugo\pwindex{Hofmannsthal, Hugo von 1874-02-01 – 1929-07-15@\textsc{Hofmannsthal, Hugo von} (1874-02-01 – 1929-07-15), \emph{Schriftsteller}|pw} ein ganz kleines Kind beko{\geminationm}en hat, Chriſtiane\pwindex{Hofmannsthal, Christiane von 14.05.1902 – 05.01.1987@\textsc{Hofmannsthal, Christiane von} (14.05.1902 – 05.01.1987)|pw} genannt, wiſſen Sie wohl ſchon.– Heute {\pb}hatten wir beinah einen
               »Frühlingsabend« – lau, ohne Wind und Regen, man faſſt es kaum. – \textsc{Rochefort\textcolor{red}{\textsuperscript{\textbf{KEY}}}} wird gegen Schluſs matter; ich beſchäftige mich ein weniges mit Botanik und
               denke wieder manchmal mit Wehmut, wie faul ich mein Leben lang war, und auf wie viel
               beſſerm Grund ich {\pb}ſtehen könnte,
                  we{\geminationn} ich nicht gar ſo ſpät auf mich aufmerkſam
               geworden wäre. \pend
           \pstart
           Leben Sie wohl. Grüßen Sie Florenz\oindex{Florenz@\textbf{Florenz}|pw}, die \textsc{Mediceer} Gräber\oindex{Medici-Kapelle@\textbf{Medici-Kapelle}|pw}, den Garten
               hinter dem Kloſter zu \textsc{Fiesole}\oindex{San Domenico@\textbf{San Domenico}|pw} und Veronika\textcolor{red}{\textsuperscript{\textbf{KEY}}}; – und Bern grüßt den andern Hund. \pend
           \pstart
           Herzlichst Ihr {\\[\baselineskip]}\spacefill\mbox{A.}\pend
           \leftskip=0em{}
         
         \endnumbering\mylabel{h}\end{ledgroupsized}\begin{anhang}\end{anhang}\newcommand{\dateiname}{L02974}\newcommand{\titel}{Arthur Schnitzler an Felix Salten, 27. 5. 1902}\newcommand{\editorInnen}{Martin Anton Müller und Laura Untner}%% latex-leseansicht-abspann.tex
%% Abspann für die Leseansicht.
%% Der Schalter \ifkorrekturansicht ist bereits durch den Vorspann gesetzt.

%% latex-abspann.tex
%% Gemeinsamer Abspann für Korrekturansicht und Leseansicht.
%% Setzt den Schalter \ifkorrekturansicht voraus (gesetzt in den
%% einbindenden Dateien latex-korrekturansicht-abspann.tex bzw.
%% latex-leseansicht-abspann.tex).
%% ---------------------------------------------------------------

\normalsize

% Das esempio-Environment wird nur in der Leseansicht benötigt
\ifkorrekturansicht\else
\newenvironment{esempio}[3]%
{
    \vspace{1.5ex}
    \rlap{\underline{#1}}
    \par
    \setlength{\parindent}{0cm}
    \nopagebreak
    \leftskip=#2cm
    \rightskip=#3cm
}
{
    \par
}
\fi

\doendnotes{C}
\bigskip
\vfill

\clearpage

\footnotesize

\ifkorrekturansicht
  \lohead{\textsc{register}}
\fi

% theindex-Environment neu definieren ohne reledmac
\makeatletter
\renewenvironment{theindex}{%
  \ifkorrekturansicht
    \section*{\indexname}%
  \else
    \subsubsection*{Index der erwähnten Entitäten}%
  \fi
  \setlength{\parindent}{0pt}%
  \setlength{\parskip}{0pt plus 0.3pt}%
  \let\item\@idxitem
}{%
  \ifkorrekturansicht\clearpage\fi
}
\makeatother

\IfFileExists{\jobname-pw.ind}{\input{\jobname-pw.ind}}{}

% Quellenangabe nur in der Leseansicht
\ifkorrekturansicht\else
% Fallback-Definitionen, falls die .tex-Datei \titel etc. nicht gesetzt hat
\providecommand{\titel}{}
\providecommand{\editorInnen}{}
\providecommand{\dateiname}{\jobname}

\vspace{3cm}

\vfill

\footnotesize
\textsc{Quelle}: \titel. Herausgegeben von {\editorInnen}. In: \emph{Arthur Schnitzler: Briefwechsel mit Autorinnen und Autoren}.
 Digitale Edition, https://schnitzler-briefe.acdh.oeaw.ac.at/{\dateiname}.html (Stand \today)
\fi

\end{document}


      