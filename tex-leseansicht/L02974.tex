%% latex-leseansicht-vorspann.tex
%% Vorspann für die Leseansicht.
%% Lädt die gemeinsame Datei latex-vorspann.tex mit nicht gesetztem Schalter.

\newif\ifkorrekturansicht
\korrekturansichtfalse

\input{../tex-inputs/latex-vorspann}


\section[ Arthur Schnitzler an Felix Salten, 27. 5. 1902]{L02974 Arthur Schnitzler an Felix Salten,  27. 5. 1902}
\nopagebreak\mylabel{L02974v}
\rehead{ }\normalsize\beginnumbering\briefempfaengerindex{Salten, Felix@\textsc{Salten, Felix}!zzzSchnitzler, Arthur@\emph{von Arthur Schnitzler}!1902-05-271@{27. 5. 1902}|(be}
\toendnotes[C]{\smallbreak\pagebreak[2]}
\correspDesc{Versand  durch Arthur Schnitzler am 27. 5. 1902 in Wien
\newline{}Erhalt  durch Felix Salten im Zeitraum [28. 5. 1902
                  – 1. 6. 1902?] in Florenz}\toendnotes[C]{\smallbreak}
\Standort{Wienbibliothek im Rathaus, ZPH 1681, 2.1.516.}
\physDesc{Brief, 2 Blätter, 8 Seiten, 2579 Zeichen
\newline{}Handschrift: Bleistift, deutsche Kurrent
\newline{}Ordnung: mit Bleistift von unbekannter Hand Nummerierung der Doppelseiten des
                                 Konvoluts: »62«–»65« }\toendnotes[C]{\smallbreak}
\pstart
           \raggedleft{}{\pb}27. 5. 902\pend
           \vspace{0.5em}
\pstart
           lieber, ich freue mich{ }ſehr über den guten Eindruck, den Sie von der
                  \label{K_L02974-1v}\edtext{Novellette\pwindex{Schnitzler, Arthur 15.\,5.\,1862 Wien – 21.\,10.\,1931 ebd.@\textsc{Schnitzler, Arthur} (15.\,5.\,1862 Wien – 21.\,10.\,1931 ebd.), \emph{Schriftsteller, Mediziner}!Dämmerseele@\strich\emph{Dämmerseele}|pwv}}{\lemma{\textnormal{\emph{Novellette}}}\Cendnote{\textnormal{Siehe XXXX Auszeichnungsfehler: Dokument L03330 nicht gefunden.
               }}}\label{K_L02974-1} in d. N. Fr. Pr.\pwindex{Neue Freie Presse@\emph{Neue Freie Presse}|pw} haben; was mir
               eigentlich{ }ſelten paſſiert, – ich war{ }ſelbſt ein bischen unſicher im Urtheil. Daſs{ }ſie Schwarzk.\pwindex{Schwarzkopf, Gustav 7.\,11.\,1853 Wien – 13.\,11.\,1939 ebd.@\textsc{Schwarzkopf, Gustav} (7.\,11.\,1853 Wien – 13.\,11.\,1939 ebd.), \emph{Schriftsteller}|pw} nicht mag, iſt ziemlich
               verſtändlich; – der \label{K_L02974-2v}\edtext{Einwurf Goldm.\pwindex{Goldmann, Paul 31.\,1.\,1865 Breslau – 25.\,9.\,1935 Wien@\textsc{Goldmann, Paul} (31.\,1.\,1865 Breslau – 25.\,9.\,1935 Wien), \emph{Schriftsteller, Journalist}|pw}: es handle{ }ſich um Liebe}{\lemma{\textnormal{\emph{Einwurf … Liebe}}}\Cendnote{\textnormal{Siehe A. S.: \emph{Tagebuch}, 21. 5. 1902.
               }}}\label{K_L02974-2}, kaum discutirbar; Richard\pwindex{Beer-Hofmann, Richard 11.\,7.\,1866 Wien – 26.\,9.\,1945 New York City@\textsc{Beer-Hofmann, Richard} (11.\,7.\,1866 Wien – 26.\,9.\,1945 New York City), \emph{Schriftsteller}|pw} u Hugo\pwindex{Hofmannsthal, Hugo von 1.\,2.\,1874 Wien – 15.\,7.\,1929 Rodaun@\textsc{Hofmannsthal, Hugo von} (1.\,2.\,1874 Wien – 15.\,7.\,1929 Rodaun), \emph{Schriftsteller}|pw}{ }ſcheinen{ }ſie im ganzen gut zu finden, aber
                  {\pb}wie mir{ }ſchien, mit einigem innern
               Widerſtand. Olga\pwindex{Schnitzler, Olga 17.\,1.\,1882 Wien – 13.\,1.\,1970 Lugano@\textsc{Schnitzler, Olga} (17.\,1.\,1882 Wien – 13.\,1.\,1970 Lugano), \emph{Schauspielerin, Sängerin}|pw} gefiel{ }ſie, als ich{ }ſie ihr
               vorlas, beſonders gut; – die gedruckte hat{ }ſie aber enttäuscht. Meine Bedenken gehen
               nach der Seite des mä\textcolor{gray}{{\geminationn}}lichen {\dotstwo} ich f\textcolor{gray}{i}nde eben kein
               andres Wort – Helden{\dots}, wo mir was zu fehlen{ }ſcheint. Der
               Titel ko{\geminationm}t mir,{ }ſelbſt nach jedem Überdenken Ihrer
               Einwände, nicht un{\pb}glücklich vor. Daſs Sie
               als der erſte den Schluſs nicht als Pointe empfinden,{ }ſondern wohl im Gegentheil
               gerade als den Ausklang ins ungewiſſe, ferne, mit Notwendgkeit weiterflutend,
                  be\textcolor{gray}{rü}hrt mich beſonders angenehm. –\pend
           
\pstart
           Paul G.\pwindex{Goldmann, Paul 31.\,1.\,1865 Breslau – 25.\,9.\,1935 Wien@\textsc{Goldmann, Paul} (31.\,1.\,1865 Breslau – 25.\,9.\,1935 Wien), \emph{Schriftsteller, Journalist}|pw} ist wieder \label{K_L02974-3v}\edtext{fort}{\lemma{\textnormal{\emph{fort}}}\Cendnote{\textnormal{Paul Goldmann\pwindex{Goldmann, Paul 31.\,1.\,1865 Breslau – 25.\,9.\,1935 Wien@\textsc{Goldmann, Paul} (31.\,1.\,1865 Breslau – 25.\,9.\,1935 Wien), \emph{Schriftsteller, Journalist}|pwk} war über Pfingsten in Wien\oindex{Wien@\textbf{Wien}, \emph{Verwaltungsgebiet}|pwk} gewesen.}}}\label{K_L02974-3}; die \label{K_L02974-4v}\edtext{Martin Finder Sachen}{\lemma{\textnormal{\emph{Martin Finder Sachen}}}\Cendnote{\textnormal{Da Salten\pwindex{Salten, Felix 6.\,9.\,1869 Budapest – 8.\,10.\,1945 Zürich@\textsc{Salten, Felix} (6.\,9.\,1869 Budapest – 8.\,10.\,1945 Zürich), \emph{Schriftsteller, Journalist, Chefredakteur}|pwk} bis zum 30. 6. 1902 bei der \emph{Wiener Allgemeinen Zeitung}\orgindex{Wiener Allgemeine Zeitung@Wiener Allgemeine Zeitung|pwk} unter Vertrag stand, veröffentlichte er seine
                  Beiträge für die Wochenschrift \emph{Die Zeit}\pwindex{Zeit. Wiener Wochenschrift@\emph{Die Zeit. Wiener Wochenschrift}|pwk} bis
                  dahin unter dem Pseudonym »Martin Finder«, in das nur wenige Personen eingeweiht
                  waren.}}}\label{K_L02974-4}{ }ſind ihm höchlich aufgefallen; – er hat{ }ſich gefragt: Was ko{\geminationm}t da für ein {\pb}{[}»{]}Nachwuchs« – er iſt es, der in d \label{K_L02974-5v}\edtext{N. Fr. Pr.\orgindex{Neue Freie Presse@Neue Freie Presse|pw} mit lebhafteſter Betonung von Ihnen{ }ſprach, worauf \textsc{Bened.}\pwindex{Benedikt, Moriz 27.\,5.\,1849 Kvačice – 18.\,3.\,1920 Wien@\textsc{Benedikt, Moriz} (27.\,5.\,1849 Kvačice – 18.\,3.\,1920 Wien), \emph{Journalist, Herausgeber}|pw} meinte, er dächte{ }ſchon lange Zeit an Sie {\dots} Das will
               natürlich nicht viel heißen; aber ich glaube, we{\geminationn} Sie zu
               irgendwelchen Schritten}{\lemma{\textnormal{\emph{N. Fr. Pr. … Schritten}}}\Cendnote{\textnormal{Vgl. XXXX Auszeichnungsfehler: Dokument L03331 nicht gefunden.
               }}}\label{K_L02974-5}{ }ſich entſchlöſſen (über die natürlich noch geſprochen werden muſs),{ }ſo wären
               hier die Chancen, mindeſtens materiell günſtiger als bei der Zeit\orgindex{Zeit@Die Zeit|pw}. Obwohl {\pb}Kanner\pwindex{Kanner, Heinrich 9.\,11.\,1864 Galați – 15.\,2.\,1930 Wien@\textsc{Kanner, Heinrich} (9.\,11.\,1864 Galați – 15.\,2.\,1930 Wien), \emph{Herausgeber, Publizist}|pw} zu P. G.\pwindex{Goldmann, Paul 31.\,1.\,1865 Breslau – 25.\,9.\,1935 Wien@\textsc{Goldmann, Paul} (31.\,1.\,1865 Breslau – 25.\,9.\,1935 Wien), \emph{Schriftsteller, Journalist}|pw}, der auch dort von Ihnen redete, geäußert hat: »\label{K_L02974-6v}\edtext{Er wird ja für uns{ }ſchreiben.}{\lemma{\textnormal{\emph{Er … schreiben.}}}\Cendnote{\textnormal{Kanner\pwindex{Kanner, Heinrich 9.\,11.\,1864 Galați – 15.\,2.\,1930 Wien@\textsc{Kanner, Heinrich} (9.\,11.\,1864 Galați – 15.\,2.\,1930 Wien), \emph{Herausgeber, Publizist}|pwk} wahrte Saltens\pwindex{Salten, Felix 6.\,9.\,1869 Budapest – 8.\,10.\,1945 Zürich@\textsc{Salten, Felix} (6.\,9.\,1869 Budapest – 8.\,10.\,1945 Zürich), \emph{Schriftsteller, Journalist, Chefredakteur}|pwk} Pseudonym und erzählte nicht, dass dieser schon
                  begonnen hatte, für die Wochenschrift \emph{Die Zeit}\orgindex{Zeit. Wiener Wochenschrift@Die Zeit. Wiener Wochenschrift|pwk}
                  zu schreiben. Die Auskunft bezog sich nur auf die anlaufende Gründung der neuen
                     Tageszeitung\pwindex{Zeit@\emph{Die Zeit}|pwkv}, die ab dem
                     27. 9. 1902 erschien.}}}\label{K_L02974-6}« –\pend
           
\pstart
           \textsc{Kainz}\pwindex{Kainz, Josef 2.\,1.\,1858 Mosonmagyaróvár – 20.\,9.\,1910 Wien@\textsc{Kainz, Josef} (2.\,1.\,1858 Mosonmagyaróvár – 20.\,9.\,1910 Wien), \emph{Schauspieler}|pw} will durchaus im »Weg zum Licht\pwindex{\textcolor{red}{\textsuperscript{XXXX indx1}}!Weg zum Licht. Ein Salzburger Märchendrama in vier Akten@\strich\emph{Der Weg zum Licht. Ein Salzburger Märchendrama in vier Akten}|pw}«{ }ſpielen;
               u \label{K_L02974-7v}\edtext{Schlenther\pwindex{Schlenther, Paul 20.\,8.\,1854 Chernyakhovsk – 30.\,4.\,1916 Berlin@\textsc{Schlenther, Paul} (20.\,8.\,1854 Chernyakhovsk – 30.\,4.\,1916 Berlin), \emph{Schriftsteller, Kritiker, Theaterleiter}|pw}\orgindex{Burgtheater@Burgtheater|pwv} dürfte es daher aufführen}{\lemma{\textnormal{\emph{Schlenther … aufführen}}}\Cendnote{\textnormal{Dazu kam es nicht.}}}\label{K_L02974-7} (So Brahm\pwindex{Brahm, Otto 5.\,2.\,1856 Hamburg – 28.\,11.\,1912 Berlin@\textsc{Brahm, Otto} (5.\,2.\,1856 Hamburg – 28.\,11.\,1912 Berlin), \emph{Theaterleiter, Regisseur}|pw}.) Es iſt recht
               lächerlich, daſs ein{ }ſolcher Künſtler den \label{K_L02974-8v}\edtext{Hahngikl\pwindex{\textcolor{red}{\textsuperscript{XXXX indx1}}!Weg zum Licht. Ein Salzburger Märchendrama in vier Akten@\strich\emph{Der Weg zum Licht. Ein Salzburger Märchendrama in vier Akten}|pwv}}{\lemma{\textnormal{\emph{Hahngikl}}}\Cendnote{\textnormal{laut Figurenliste »ein Dunkelelb
                     vom Untersberg«}}}\label{K_L02974-8} dem \label{K_L02974-9v}\edtext{\textsc{Bentivoglio\pwindex{Schnitzler, Arthur 15.\,5.\,1862 Wien – 21.\,10.\,1931 ebd.@\textsc{Schnitzler, Arthur} (15.\,5.\,1862 Wien – 21.\,10.\,1931 ebd.), \emph{Schriftsteller, Mediziner}!Schleier der Beatrice. Schauspiel in fünf Akten@\strich\emph{Der Schleier der Beatrice. Schauspiel in fünf Akten}|pwv}}}{\lemma{\textnormal{\emph{Bentivoglio}}}\Cendnote{\textnormal{Hauptfigur von \emph{Der Schleier der Beatrice}\pwindex{Schnitzler, Arthur 15.\,5.\,1862 Wien – 21.\,10.\,1931 ebd.@\textsc{Schnitzler, Arthur} (15.\,5.\,1862 Wien – 21.\,10.\,1931 ebd.), \emph{Schriftsteller, Mediziner}!Schleier der Beatrice. Schauspiel in fünf Akten@\strich\emph{Der Schleier der Beatrice. Schauspiel in fünf Akten}|pwk}. Zur Ablehnung des Stücks\pwindex{Schnitzler, Arthur 15.\,5.\,1862 Wien – 21.\,10.\,1931 ebd.@\textsc{Schnitzler, Arthur} (15.\,5.\,1862 Wien – 21.\,10.\,1931 ebd.), \emph{Schriftsteller, Mediziner}!Schleier der Beatrice. Schauspiel in fünf Akten@\strich\emph{Der Schleier der Beatrice. Schauspiel in fünf Akten}|pwkv} durch das \emph{Burgtheater}\orgindex{Burgtheater@Burgtheater|pwk}{ }siehe XXXX Auszeichnungsfehler: Dokument L01073 nicht gefunden.}}}\label{K_L02974-9} vorzieht;
               aber es liegt wohl recht tief. – Dem \label{K_L02974-10v}\edtext{Deutſch Theater\orgindex{Deutsches Theater Berlin@Deutsches Theater Berlin|pw} geht es hier}{\lemma{\textnormal{\emph{Deutsch … hier}}}\Cendnote{\textnormal{Das \emph{Deutsche Theater Berlin}\orgindex{Deutsches Theater Berlin@Deutsches Theater Berlin|pwk} spielte vom 6. 5. 1902 bis zum
                     zum 5. 6. 1902 im Carl-Theater\oindex{Wien@\textbf{Wien}!II., Leopoldstadt@\textbf{II., Leopoldstadt}!Carl-Theater@\textbf{Carl-Theater}, \emph{Theater}|pwk} in
                     Wien\oindex{Wien@\textbf{Wien}, \emph{Verwaltungsgebiet}|pwk}
                     ein »Gesammt-Gastpiel«.}}}\label{K_L02974-10} ausgezeichnet. – Der \label{K_L02974-11v}\edtext{Kakadu\pwindex{Schnitzler, Arthur 15.\,5.\,1862 Wien – 21.\,10.\,1931 ebd.@\textsc{Schnitzler, Arthur} (15.\,5.\,1862 Wien – 21.\,10.\,1931 ebd.), \emph{Schriftsteller, Mediziner}!grüne Kakadu. Groteske in einem Akt@\strich\emph{Der grüne Kakadu. Groteske in einem Akt}|pw}\pwindex{Schnitzler, Arthur 15.\,5.\,1862 Wien – 21.\,10.\,1931 ebd.@\textsc{Schnitzler, Arthur} (15.\,5.\,1862 Wien – 21.\,10.\,1931 ebd.), \emph{Schriftsteller, Mediziner}!Au Perroquet Vert@\strich\emph{Au Perroquet Vert}|pwv} iſt {\pb}bei Antoine\orgindex{Théâtre Antoine@Théâtre Antoine|pw}}{\lemma{\textnormal{\emph{Kakadu ist bei Antoine}}}\Cendnote{\textnormal{\emph{Au
                     Perroquet Vert}\pwindex{Schnitzler, Arthur 15.\,5.\,1862 Wien – 21.\,10.\,1931 ebd.@\textsc{Schnitzler, Arthur} (15.\,5.\,1862 Wien – 21.\,10.\,1931 ebd.), \emph{Schriftsteller, Mediziner}!Au Perroquet Vert@\strich\emph{Au Perroquet Vert}|pwk}, die Übersetzung von \emph{Der
                     grüne Kakadu}\pwindex{Schnitzler, Arthur 15.\,5.\,1862 Wien – 21.\,10.\,1931 ebd.@\textsc{Schnitzler, Arthur} (15.\,5.\,1862 Wien – 21.\,10.\,1931 ebd.), \emph{Schriftsteller, Mediziner}!grüne Kakadu. Groteske in einem Akt@\strich\emph{Der grüne Kakadu. Groteske in einem Akt}|pwk}, hatte am 7. 11. 1903 am \emph{Théâtre Antoine}\orgindex{Théâtre Antoine@Théâtre Antoine|pwk} Premiere.}}}\label{K_L02974-11} acceptirt. –
                  \label{K_L02974-12v}\edtext{Über die \textsc{Bea.\pwindex{Schnitzler, Arthur 15.\,5.\,1862 Wien – 21.\,10.\,1931 ebd.@\textsc{Schnitzler, Arthur} (15.\,5.\,1862 Wien – 21.\,10.\,1931 ebd.), \emph{Schriftsteller, Mediziner}!Schleier der Beatrice. Schauspiel in fünf Akten@\strich\emph{Der Schleier der Beatrice. Schauspiel in fünf Akten}|pw}}{ }ſpricht Brahm\pwindex{Brahm, Otto 5.\,2.\,1856 Hamburg – 28.\,11.\,1912 Berlin@\textsc{Brahm, Otto} (5.\,2.\,1856 Hamburg – 28.\,11.\,1912 Berlin), \emph{Theaterleiter, Regisseur}|pw} kein Wort}{\lemma{\textnormal{\emph{Über … Wort}}}\Cendnote{\textnormal{Nach der Enttäuschung der Uraufführung von
                     \emph{Der Schleier der Beatrice}\pwindex{Schnitzler, Arthur 15.\,5.\,1862 Wien – 21.\,10.\,1931 ebd.@\textsc{Schnitzler, Arthur} (15.\,5.\,1862 Wien – 21.\,10.\,1931 ebd.), \emph{Schriftsteller, Mediziner}!Schleier der Beatrice. Schauspiel in fünf Akten@\strich\emph{Der Schleier der Beatrice. Schauspiel in fünf Akten}|pwk}\eventindex{Lobe-Theater@\textbf{Lobe-Theater}!Uraufführung von Der Schleier der Beatrice, 1.12.1900@Uraufführung von Der Schleier der Beatrice, 1.12.1900|pwk} setzte Schnitzler seine Hoffnungen auf eine
                  Inszenierung am \emph{Deutschen Theater Berlin}\orgindex{Deutsches Theater Berlin@Deutsches Theater Berlin|pwk}.
                  Diese fand am 7. 3. 1903 statt.}}}\label{K_L02974-12}. – Ich überdenke und scenire mein Stück\pwindex{Schnitzler, Arthur 15.\,5.\,1862 Wien – 21.\,10.\,1931 ebd.@\textsc{Schnitzler, Arthur} (15.\,5.\,1862 Wien – 21.\,10.\,1931 ebd.), \emph{Schriftsteller, Mediziner}!einsame Weg. Schauspiel in fünf Akten@\strich\emph{Der einsame Weg. Schauspiel in fünf Akten}|pwv} u übe mich indeſs weiter
               im Erzählen!\pend
           
\pstart
           – Sagen Sie mir doch etwas über Ihre Reiſe, Ihre Arbeiten, Ihre Laune. Daſs Hugo\pwindex{Hofmannsthal, Hugo von 1.\,2.\,1874 Wien – 15.\,7.\,1929 Rodaun@\textsc{Hofmannsthal, Hugo von} (1.\,2.\,1874 Wien – 15.\,7.\,1929 Rodaun), \emph{Schriftsteller}|pw} ein ganz kleines \label{K_L02974-13v}\edtext{Kind\pwindex{Zimmer, Christiane 14.\,5.\,1902 Rodaun – 5.\,1.\,1987 New York City@\textsc{Zimmer, Christiane} (14.\,5.\,1902 Rodaun – 5.\,1.\,1987 New York City)|pwv} beko{\geminationm}en hat, Chriſtiane\pwindex{Zimmer, Christiane 14.\,5.\,1902 Rodaun – 5.\,1.\,1987 New York City@\textsc{Zimmer, Christiane} (14.\,5.\,1902 Rodaun – 5.\,1.\,1987 New York City)|pw}}{\lemma{\textnormal{\emph{Kind … Christiane}}}\Cendnote{\textnormal{Christiane von Hofmannsthal\pwindex{Zimmer, Christiane 14.\,5.\,1902 Rodaun – 5.\,1.\,1987 New York City@\textsc{Zimmer, Christiane} (14.\,5.\,1902 Rodaun – 5.\,1.\,1987 New York City)|pwk} kam am 14. 5. 1902 auf die Welt.}}}\label{K_L02974-13} genannt, wiſſen Sie
               wohl{ }ſchon. – Heute{ }{\pb}hatten wir beinah einen »Frühlingsabend« –
               lau, ohne Wind und Regen, man faſſt es kaum. – \textsc{\label{K_L02974-14v}\edtext{Rochefort\pwindex{Rochefort, Henri de 31.\,1.\,1830 Paris – 30.\,6.\,1913 Aix-les-Bains@\textsc{Rochefort, Henri de} (31.\,1.\,1830 Paris – 30.\,6.\,1913 Aix-les-Bains), \emph{Schriftsteller, Politiker, Journalist}!Abenteuer meines Lebens@\strich\emph{Abenteuer meines Lebens}|pwuv}\pwindex{Rochefort, Henri de 31.\,1.\,1830 Paris – 30.\,6.\,1913 Aix-les-Bains@\textsc{Rochefort, Henri de} (31.\,1.\,1830 Paris – 30.\,6.\,1913 Aix-les-Bains), \emph{Schriftsteller, Politiker, Journalist}|pw}}{\lemma{\textnormal{\emph{Rochefort}}}\Cendnote{\textnormal{Es dürfte sich um die (gekürzte)
                     deutschsprachige Ausgabe der Autobiografie von Henri Rochefort\pwindex{Rochefort, Henri de 31.\,1.\,1830 Paris – 30.\,6.\,1913 Aix-les-Bains@\textsc{Rochefort, Henri de} (31.\,1.\,1830 Paris – 30.\,6.\,1913 Aix-les-Bains), \emph{Schriftsteller, Politiker, Journalist}|pwk}: \emph{Les
                        Aventures de ma vie}\pwindex{Rochefort, Henri de 31.\,1.\,1830 Paris – 30.\,6.\,1913 Aix-les-Bains@\textsc{Rochefort, Henri de} (31.\,1.\,1830 Paris – 30.\,6.\,1913 Aix-les-Bains), \emph{Schriftsteller, Politiker, Journalist}!Aventures de ma vie@\strich\emph{Les Aventures de ma vie}|pwk} (1896) handeln: \emph{Abenteuer meines Lebens}\pwindex{Rochefort, Henri de 31.\,1.\,1830 Paris – 30.\,6.\,1913 Aix-les-Bains@\textsc{Rochefort, Henri de} (31.\,1.\,1830 Paris – 30.\,6.\,1913 Aix-les-Bains), \emph{Schriftsteller, Politiker, Journalist}!Abenteuer meines Lebens@\strich\emph{Abenteuer meines Lebens}|pwk}. Autorisierte
                        deutsche Bearbeitung von Heinrich
                           Conrad\pwindex{Conrad, Heinrich 19.\,10.\,1866 Hamburg – 20.\,12.\,1918 München@\textsc{Conrad, Heinrich} (19.\,10.\,1866 Hamburg – 20.\,12.\,1918 München), \emph{Übersetzer, Romanist}|pwk}. Stuttgart\oindex{Stuttgart@\textbf{Stuttgart}|pwk}: \emph{Robert Lutz}\orgindex{Robert Lutz@Robert Lutz|pwk}{ }1900.}}}\label{K_L02974-14}} wird gegen Schluſs matter; ich beſchäftige mich ein weniges mit \label{K_L02974-15v}\edtext{Botanik}{\lemma{\textnormal{\emph{Botanik}}}\Cendnote{\textnormal{Am 23. 5. 1902 besuchte Schnitzler
                  den Botanischen Garten\oindex{Wien@\textbf{Wien}!III., Landstraße@\textbf{III., Landstraße}!Botanischer Garten der Universität@\textbf{Botanischer Garten der Universität}, \emph{Park}|pwk}.}}}\label{K_L02974-15} und denke
               wieder manchmal mit Wehmut, wie faul ich mein Leben lang war, und auf wie viel
                  beſſer\textcolor{gray}{m} Grund ich {\pb}ſtehen könnte, we{\geminationn} ich nicht gar{ }ſo{ }ſpät auf mich
               aufmerkſam geworden wäre.\pend
           
\pstart
           Leben Sie wohl. Grüßen Sie Florenz\oindex{Florenz@\textbf{Florenz}|pw}, die \textsc{Mediceer} Gräber\oindex{Sagrestia Nuova@\textbf{Sagrestia Nuova}, \emph{Kirche}|pw}, den Garten hinter dem Kloſter zu \textsc{Fiesole}\oindex{San Domenico@\textbf{San Domenico}, \emph{Kloster}|pw} und \textsc{Veronika\pwindex{Salten, Felix 6.\,9.\,1869 Budapest – 8.\,10.\,1945 Zürich@\textsc{Salten, Felix} (6.\,9.\,1869 Budapest – 8.\,10.\,1945 Zürich), \emph{Schriftsteller, Journalist, Chefredakteur}!kleine Veronika@\strich\emph{Die kleine Veronika}|pw}}; – und \label{K_L02974-16v}\edtext{Bern}{\lemma{\textnormal{\emph{Bern}}}\Cendnote{\textnormal{Vgl. XXXX Auszeichnungsfehler: Dokument L03204 nicht gefunden.
               }}}\label{K_L02974-16} grüßt den andern \label{K_L02974-17v}\edtext{Hund}{\lemma{\textnormal{\emph{Hund}}}\Cendnote{\textnormal{Vgl. XXXX Auszeichnungsfehler: Dokument L03357 nicht gefunden.
               }}}\label{K_L02974-17}.\pend
           
\pstart
           Herzlichst Ihr {\\[\baselineskip]}\spacefill\mbox{A.}\pend
           \leftskip=0em{}\selectlanguage{ngerman}\endnumbering\briefempfaengerindex{Salten, Felix@\textsc{Salten, Felix}!zzzSchnitzler, Arthur@\emph{von Arthur Schnitzler}!1902-05-271@{27. 5. 1902}|)be}\mylabel{L02974h}  \newcommand{\dateiname}{L02974}\newcommand{\titel}{Arthur Schnitzler an Felix Salten, 27. 5. 1902}\newcommand{\editorInnen}{Martin Anton Müller und Laura Untner}%% latex-leseansicht-abspann.tex
%% Abspann für die Leseansicht.
%% Der Schalter \ifkorrekturansicht ist bereits durch den Vorspann gesetzt.

%% latex-abspann.tex
%% Gemeinsamer Abspann für Korrekturansicht und Leseansicht.
%% Setzt den Schalter \ifkorrekturansicht voraus (gesetzt in den
%% einbindenden Dateien latex-korrekturansicht-abspann.tex bzw.
%% latex-leseansicht-abspann.tex).
%% ---------------------------------------------------------------

\normalsize

% Das esempio-Environment wird nur in der Leseansicht benötigt
\ifkorrekturansicht\else
\newenvironment{esempio}[3]%
{
    \vspace{1.5ex}
    \rlap{\underline{#1}}
    \par
    \setlength{\parindent}{0cm}
    \nopagebreak
    \leftskip=#2cm
    \rightskip=#3cm
}
{
    \par
}
\fi

\doendnotes{C}
\bigskip
\vfill

\clearpage

\footnotesize

\ifkorrekturansicht
  \lohead{\textsc{register}}
\fi

% theindex-Environment neu definieren ohne reledmac
\makeatletter
\renewenvironment{theindex}{%
  \ifkorrekturansicht
    \section*{\indexname}%
  \else
    \subsubsection*{Index der erwähnten Entitäten}%
  \fi
  \setlength{\parindent}{0pt}%
  \setlength{\parskip}{0pt plus 0.3pt}%
  \let\item\@idxitem
}{%
  \ifkorrekturansicht\clearpage\fi
}
\makeatother

\IfFileExists{\jobname-pw.ind}{\input{\jobname-pw.ind}}{}

% Quellenangabe nur in der Leseansicht
\ifkorrekturansicht\else
% Fallback-Definitionen, falls die .tex-Datei \titel etc. nicht gesetzt hat
\providecommand{\titel}{}
\providecommand{\editorInnen}{}
\providecommand{\dateiname}{\jobname}

\vspace{3cm}

\vfill

\footnotesize
\textsc{Quelle}: \titel. Herausgegeben von {\editorInnen}. In: \emph{Arthur Schnitzler: Briefwechsel mit Autorinnen und Autoren}.
 Digitale Edition, https://schnitzler-briefe.acdh.oeaw.ac.at/{\dateiname}.html (Stand \today)
\fi

\end{document}


