%% latex-leseansicht-vorspann.tex
%% Vorspann für die Leseansicht.
%% Lädt die gemeinsame Datei latex-vorspann.tex mit nicht gesetztem Schalter.

\newif\ifkorrekturansicht
\korrekturansichtfalse

\input{../tex-inputs/latex-vorspann}


\section[Arthur Schnitzler an Stefan Großmann, 24. 12. 1925]{L02459 Arthur Schnitzler an Stefan Großmann, 24. 12. 1925}
\nopagebreak\mylabel{L02459v}
\rehead{ }\normalsize\beginnumbering\briefempfaengerindex{Großmann, Stefan@\textsc{Großmann, Stefan}!zzzSchnitzler, Arthur@\emph{von Arthur Schnitzler}!1925-12-241@{24. 12. 1925}|(be}
\toendnotes[C]{\smallbreak\pagebreak[2]}
\correspDesc{Versand  durch Arthur Schnitzler am 24. 12. 1925 in Wien
\newline{}Erhalt  durch Stefan Großmann im Zeitraum [25. 12. 1925 – 29. 12. 1925?] in Berlin}\toendnotes[C]{\smallbreak}
\Standort{DLA, A:Schnitzler, HS.NZ85.1.896.}
\physDesc{Brief, Durchschlag, 1 Blatt, 1 Seite, 739 Zeichen
\newline{}Schreibmaschine
\newline{}Handschrift: roter Buntstift, lateinische Kurrent (\noindent{}Beschriftung: »Großmann« und vier
                                 Unterstreichungen)}\toendnotes[C]{\smallbreak}
\pstart
           \raggedleft{}{\pb}24. 12. 1925.\pend
           
\pstart{}Verehrter Herr Grossmann.\pend\vspace{0.5em}
\pstart
           Besten Dank für die freundliche Uebersendung der \label{K_L02459-1v}\edtext{Nummer\pwindex{Tage-Buch@\emph{Das Tage-Buch}|pwv} 51\pwindex{Schnitzlers Unterleibsbeschwerden@\emph{Schnitzlers Unterleibsbeschwerden}|pwv}}{\lemma{\textnormal{\emph{Nummer 51}}}\Cendnote{\textnormal{Darin ist folgende Notiz enthalten: »SCHNITZLERS UNTERLEIBSBESCHWERDEN{ / }Adolf Hitler\pwindex{Hitler, Adolf 20.\,4.\,1889 Braunau am Inn – 30.\,4.\,1945 Berlin@\textsc{Hitler, Adolf} (20.\,4.\,1889 Braunau am Inn – 30.\,4.\,1945 Berlin), \emph{Politiker, Diktator, Massenmörder}|pw} hat noch immer in München\oindex{München@\textbf{München}|pw}, das doch längst nicht mehr die
                        dümmste Stadt der Welt sein will, seine täglich erscheinende Zeitung\orgindex{Völkischer Beobachter@Völkischer Beobachter|pwv}. Dort schreibt ein
                        treudeutscher Mann\pwindex{Stolzing-Cerny, Josef 12.\,2.\,1869 Wien – 23.\,7.\,1942 München@\textsc{Stolzing-Cerny, Josef} (12.\,2.\,1869 Wien – 23.\,7.\,1942 München), \emph{Journalist}|pwv}
                        über Arthur Schnitzlers
                        Dichtungen:{ / }›Man könnte sich mit solchen Stücken gewiß abfinden, wenn daraus das Ethos
                        eines Dichters spräche, der, indem er uns die Kehrseite solcher Liebeleien
                        wie in Geschlechtskrankheiten, Unterleibsleiden, nervösen Zerrüttungen und
                        der Degeneration der Masse der großstädtischen Bevölkerung zeigte, warnend
                        und abschreckend wirkte.‹{ / }Kein Zweifel, in Schnitzlers Dichtungen fehlen die Unterleibsbeschwerden, sowohl des Darmes als der
                        anderen Organe. Eine kleine Verstopfung und Schnitzler wäre auch bei Hitler\pwindex{Hitler, Adolf 20.\,4.\,1889 Braunau am Inn – 30.\,4.\,1945 Berlin@\textsc{Hitler, Adolf} (20.\,4.\,1889 Braunau am Inn – 30.\,4.\,1945 Berlin), \emph{Politiker, Diktator, Massenmörder}|pw} ein gemachter Mann.« Jg. 6, H. 51, 19. 12. 1925, S. 1911.}}}\label{K_L02459-1}
               vom 19. Dezember. Aber warum gleich in zehn Exemplaren? Zarte Mahnung,
               weil ich meine terminlose Zusage bisher leider noch kein einziges Mal zu erfüllen
               imstande war? Ich hätte diesmal so viele Zusagen ähnlicher Art zu erfüllen gehabt,
               dass ich mich entschlossen habe keine zu erfüllen und somit will ich auch meinem
               alten Prinzip getreu nichts wirklich versprechen als was ich auch schon im selben
               Augenblick zu halten vermöchte.\pend
           
\pstart
           Der \label{K_L02459-2v}\edtext{Notiz\pwindex{Stolzing-Cerny, Josef 12.\,2.\,1869 Wien – 23.\,7.\,1942 München@\textsc{Stolzing-Cerny, Josef} (12.\,2.\,1869 Wien – 23.\,7.\,1942 München), \emph{Journalist}!Residenztheater. Erstaufführung: Anatol@\strich\emph{Residenztheater. Erstaufführung: Anatol}|pwv}}{\lemma{\textnormal{\emph{Notiz}}}\Cendnote{\textnormal{J. St–g. [ = Josef Stolzing-Cerny]\pwindex{Stolzing-Cerny, Josef 12.\,2.\,1869 Wien – 23.\,7.\,1942 München@\textsc{Stolzing-Cerny, Josef} (12.\,2.\,1869 Wien – 23.\,7.\,1942 München), \emph{Journalist}|pwk}: \emph{Residenztheater. Erstaufführung: Anatol}\pwindex{Stolzing-Cerny, Josef 12.\,2.\,1869 Wien – 23.\,7.\,1942 München@\textsc{Stolzing-Cerny, Josef} (12.\,2.\,1869 Wien – 23.\,7.\,1942 München), \emph{Journalist}!Residenztheater. Erstaufführung: Anatol@\strich\emph{Residenztheater. Erstaufführung: Anatol}|pwk}.
                     In: \emph{Völkischer Beobachter}\pwindex{Völkischer Beobachter@\emph{Völkischer Beobachter}|pwk}, Jg. 38,
                     Nr. 220, 15. 12. 1925, S. 2.}}}\label{K_L02459-2} aus dem Hitler\pwindex{Hitler, Adolf 20.\,4.\,1889 Braunau am Inn – 30.\,4.\,1945 Berlin@\textsc{Hitler, Adolf} (20.\,4.\,1889 Braunau am Inn – 30.\,4.\,1945 Berlin), \emph{Politiker, Diktator, Massenmörder}|pw}-Blatt\orgindex{Völkischer Beobachter@Völkischer Beobachter|pwv} bleibt ein bescheidenes, aber ehrenvolles Plätzchen
               in meiner Sammlung gewahrt.\pend
           
\pstart
           Mit verbindlichen Neujahrsgrüssen{\\[\baselineskip]}Ihr sehr ergebener\pend
           \leftskip=0em{}{\vspace{1\baselineskip}}
\pstart
           Herrn Stefan Grossmann,{\\}Herausgeber des »Tagebuch\orgindex{Tage-Buch@Das Tage-Buch|pw}\label{T_L02459-1v}\edtext{«}{\lemma{\textnormal{\emph{«}}}\Cendnote{\textnormal{an Stelle des Ausführungszeichens steht »§«}}}\label{T_L02459-1}, Berlin\oindex{Berlin@\textbf{Berlin}, \emph{Hauptstadt}|pw}.\pend
           \selectlanguage{ngerman}\endnumbering\briefempfaengerindex{Großmann, Stefan@\textsc{Großmann, Stefan}!zzzSchnitzler, Arthur@\emph{von Arthur Schnitzler}!1925-12-241@{24. 12. 1925}|)be}\mylabel{L02459h}  \newcommand{\dateiname}{L02459}\newcommand{\titel}{Arthur Schnitzler an Stefan Großmann, 24. 12. 1925}\newcommand{\editorInnen}{Martin Anton Müller und Gerd-Hermann Susen}%% latex-leseansicht-abspann.tex
%% Abspann für die Leseansicht.
%% Der Schalter \ifkorrekturansicht ist bereits durch den Vorspann gesetzt.

%% latex-abspann.tex
%% Gemeinsamer Abspann für Korrekturansicht und Leseansicht.
%% Setzt den Schalter \ifkorrekturansicht voraus (gesetzt in den
%% einbindenden Dateien latex-korrekturansicht-abspann.tex bzw.
%% latex-leseansicht-abspann.tex).
%% ---------------------------------------------------------------

\normalsize

% Das esempio-Environment wird nur in der Leseansicht benötigt
\ifkorrekturansicht\else
\newenvironment{esempio}[3]%
{
    \vspace{1.5ex}
    \rlap{\underline{#1}}
    \par
    \setlength{\parindent}{0cm}
    \nopagebreak
    \leftskip=#2cm
    \rightskip=#3cm
}
{
    \par
}
\fi

\doendnotes{C}
\bigskip
\vfill

\clearpage

\footnotesize

\ifkorrekturansicht
  \lohead{\textsc{register}}
\fi

% theindex-Environment neu definieren ohne reledmac
\makeatletter
\renewenvironment{theindex}{%
  \ifkorrekturansicht
    \section*{\indexname}%
  \else
    \subsubsection*{Index der erwähnten Entitäten}%
  \fi
  \setlength{\parindent}{0pt}%
  \setlength{\parskip}{0pt plus 0.3pt}%
  \let\item\@idxitem
}{%
  \ifkorrekturansicht\clearpage\fi
}
\makeatother

\IfFileExists{\jobname-pw.ind}{\input{\jobname-pw.ind}}{}

% Quellenangabe nur in der Leseansicht
\ifkorrekturansicht\else
% Fallback-Definitionen, falls die .tex-Datei \titel etc. nicht gesetzt hat
\providecommand{\titel}{}
\providecommand{\editorInnen}{}
\providecommand{\dateiname}{\jobname}

\vspace{3cm}

\vfill

\footnotesize
\textsc{Quelle}: \titel. Herausgegeben von {\editorInnen}. In: \emph{Arthur Schnitzler: Briefwechsel mit Autorinnen und Autoren}.
 Digitale Edition, https://schnitzler-briefe.acdh.oeaw.ac.at/{\dateiname}.html (Stand \today)
\fi

\end{document}


