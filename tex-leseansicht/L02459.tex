%% latex-leseansicht-vorspann.tex
%% Vorspann für die Leseansicht.
%% Lädt die gemeinsame Datei latex-vorspann.tex mit nicht gesetztem Schalter.

\newif\ifkorrekturansicht
\korrekturansichtfalse

\input{../tex-inputs/latex-vorspann}


         
         \renewcommand{\erwaehntePersonen}{Personen: Stefan Großmann, Adolf Hitler, Josef Stolzing-Cerny}
         \renewcommand{\erwaehnteInstitutionen}{Institutionen: Das Tage-Buch, Völkischer Beobachter}
         \renewcommand{\erwaehnteOrte}{Orte: Berlin, München, Wien}
         \renewcommand{\erwaehnteWerke}{Werke: Das Tage-Buch, Residenztheater. Erstaufführung: Anatol, Schnitzlers Unterleibsbeschwerden, Völkischer Beobachter}
               \section[Arthur Schnitzler an Stefan Großmann, 24. 12. 1925]{ Arthur Schnitzler an Stefan Großmann, 24. 12. 1925}\nopagebreak\mylabel{v}\rehead{ }\begin{ledgroupsized}[t]{13cm}\normalsize\beginnumbering\briefempfaengerindex{Grossmann, Stefan@\textsc{Großmann, Stefan}!zzzSchnitzler, Arthur@\emph{von Arthur Schnitzler}!1925-12-241@{24. 12. 1925}|(be} \toendnotes[C]{\smallbreak\pagebreak[2]} \Standort{DLA, A:Schnitzler, HS.NZ85.1.896.}
\physDesc{Brief, Durchschlag, 1 Blatt, 1 Seite, 739 Zeichen
\newline{}Schreibmaschine
\newline{}Handschrift: roter Buntstift, lateinische Kurrent (\noindent{}Beschriftung: »Großmann« und vier
                                 Unterstreichungen)}\toendnotes[C]{\smallbreak}\pstart
           \raggedleft{}{\pb}24. 12. 1925. \pend
           \pstart{}Verehrter Herr Grossmann.\pend\pstart
           Besten Dank für die freundliche Uebersendung der \label{K_L02459-1v}\edtext{Nummer\pwindex{Tage-Buch1920-01-01 – 1933-01-01@\emph{Das Tage-Buch} {[}1920-01-01 – 1933-01-01{]}|pwv} 51\pwindex{Schnitzlers Unterleibsbeschwerden1925-12-19@\emph{Schnitzlers Unterleibsbeschwerden} {[}1925-12-19{]}|pwv}}{\lemma{\textnormal{\emph{Nummer 51}}}\Cendnote{\textnormal{Darin ist folgende Notiz enthalten: »SCHNITZLERS UNTERLEIBSBESCHWERDEN{ / }Adolf Hitler\pwindex{Hitler, Adolf 20.04.1889 – 30.04.1945@\textsc{Hitler, Adolf} (20.04.1889 – 30.04.1945), \emph{Politiker, Diktator, Massenmörder}|pw} hat noch immer in München\oindex{Muenchen@\textbf{München}|pw}, das doch längst nicht mehr die
                        dümmste Stadt der Welt sein will, seine täglich erscheinende Zeitung\orgindex{Voelkischer Beobachter@Völkischer Beobachter|pwv}. Dort schreibt ein
                        treudeutscher Mann\pwindex{Stolzing-Cerny, Josef 1869-02-12 – 1942-07-23@\textsc{Stolzing-Cerny, Josef} (1869-02-12 – 1942-07-23), \emph{Journalist}|pwv}
                        über Arthur Schnitzlers\pwindex{Schnitzler, Arthur 15.05.1862 – 21.10.1931@\textsc{Schnitzler, Arthur} (15.05.1862 – 21.10.1931), \emph{Schriftsteller, Mediziner}|pw}
                        Dichtungen:{ / }›Man könnte sich mit solchen Stücken gewiß abfinden, wenn daraus das Ethos
                        eines Dichters spräche, der, indem er uns die Kehrseite solcher Liebeleien
                        wie in Geschlechtskrankheiten, Unterleibsleiden, nervösen Zerrüttungen und
                        der Degeneration der Masse der großstädtischen Bevölkerung zeigte, warnend
                        und abschreckend wirkte.‹{ / }Kein Zweifel, in Schnitzlers\pwindex{Schnitzler, Arthur 15.05.1862 – 21.10.1931@\textsc{Schnitzler, Arthur} (15.05.1862 – 21.10.1931), \emph{Schriftsteller, Mediziner}|pw} Dichtungen fehlen die Unterleibsbeschwerden, sowohl des Darmes als der
                        anderen Organe. Eine kleine Verstopfung und Schnitzler\pwindex{Schnitzler, Arthur 15.05.1862 – 21.10.1931@\textsc{Schnitzler, Arthur} (15.05.1862 – 21.10.1931), \emph{Schriftsteller, Mediziner}|pw} wäre auch bei Hitler\pwindex{Hitler, Adolf 20.04.1889 – 30.04.1945@\textsc{Hitler, Adolf} (20.04.1889 – 30.04.1945), \emph{Politiker, Diktator, Massenmörder}|pw} ein gemachter Mann.« Jg. 6, H. 51, 19. 12. 1925, S. 1911.}}}\label{K_L02459-1h}
               vom 19. Dezember. Aber warum gleich in zehn Exemplaren? Zarte Mahnung,
               weil ich meine terminlose Zusage bisher leider noch kein einziges Mal zu erfüllen
               imstande war? Ich hätte diesmal so viele Zusagen ähnlicher Art zu erfüllen gehabt,
               dass ich mich entschlossen habe keine zu erfüllen und somit will ich auch meinem
               alten Prinzip getreu nichts wirklich versprechen als was ich auch schon im selben
               Augenblick zu halten vermöchte.\pend
           \pstart
           Der \label{K_L02459-2v}\edtext{Notiz\pwindex{Residenztheater. Erstauffuehrung: Anatol1925-12-15@\emph{Residenztheater. Erstaufführung: Anatol} {[}1925-12-15{]}|pwv}}{\lemma{\textnormal{\emph{Notiz}}}\Cendnote{\textnormal{J. St–g. [ = Josef Stolzing-Cerny]\pwindex{Stolzing-Cerny, Josef 1869-02-12 – 1942-07-23@\textsc{Stolzing-Cerny, Josef} (1869-02-12 – 1942-07-23), \emph{Journalist}|pwk}: \emph{Residenztheater. Erstaufführung: Anatol}\pwindex{Residenztheater. Erstauffuehrung: Anatol1925-12-15@\emph{Residenztheater. Erstaufführung: Anatol} {[}1925-12-15{]}|pwk}.
                     In: \emph{Völkischer Beobachter}\pwindex{Voelkischer Beobachter1920-12-17 – 1945-04-29@\emph{Völkischer Beobachter} {[}1920-12-17 – 1945-04-29{]}|pwk}, Jg. 38,
                     Nr. 220, 15. 12. 1925, S. 2.}}}\label{K_L02459-2h} aus dem Hitler\pwindex{Hitler, Adolf 20.04.1889 – 30.04.1945@\textsc{Hitler, Adolf} (20.04.1889 – 30.04.1945), \emph{Politiker, Diktator, Massenmörder}|pw}-Blatt\orgindex{Voelkischer Beobachter@Völkischer Beobachter|pwv} bleibt ein bescheidenes, aber ehrenvolles Plätzchen
               in meiner Sammlung gewahrt.\pend
           \pstart
           Mit verbindlichen Neujahrsgrüssen{\\[\baselineskip]}Ihr sehr ergebener\pend
           \leftskip=0em{}{\bigskip}\pstart
           \noindent{}Herrn Stefan Grossmann,{\\}Herausgeber des »Tagebuch\orgindex{Tage-Buch@Das Tage-Buch|pw}\label{T_L02459-1v}\edtext{«}{\lemma{\textnormal{\emph{«}}}\Cendnote{\textnormal{an Stelle des Ausführungszeichens steht »§«}}}\label{T_L02459-1h}, Berlin\oindex{Berlin@\textbf{Berlin}|pw}.\pend
           
         
         \endnumbering\mylabel{h}\end{ledgroupsized}  \newcommand{\dateiname}{L02459}\newcommand{\titel}{Arthur Schnitzler an Stefan Großmann, 24. 12. 1925}\newcommand{\editorInnen}{Martin Anton Müller und Gerd-Hermann Susen}%% latex-leseansicht-abspann.tex
%% Abspann für die Leseansicht.
%% Der Schalter \ifkorrekturansicht ist bereits durch den Vorspann gesetzt.

%% latex-abspann.tex
%% Gemeinsamer Abspann für Korrekturansicht und Leseansicht.
%% Setzt den Schalter \ifkorrekturansicht voraus (gesetzt in den
%% einbindenden Dateien latex-korrekturansicht-abspann.tex bzw.
%% latex-leseansicht-abspann.tex).
%% ---------------------------------------------------------------

\normalsize

% Das esempio-Environment wird nur in der Leseansicht benötigt
\ifkorrekturansicht\else
\newenvironment{esempio}[3]%
{
    \vspace{1.5ex}
    \rlap{\underline{#1}}
    \par
    \setlength{\parindent}{0cm}
    \nopagebreak
    \leftskip=#2cm
    \rightskip=#3cm
}
{
    \par
}
\fi

\doendnotes{C}
\bigskip
\vfill

\clearpage

\footnotesize

\ifkorrekturansicht
  \lohead{\textsc{register}}
\fi

% theindex-Environment neu definieren ohne reledmac
\makeatletter
\renewenvironment{theindex}{%
  \ifkorrekturansicht
    \section*{\indexname}%
  \else
    \subsubsection*{Index der erwähnten Entitäten}%
  \fi
  \setlength{\parindent}{0pt}%
  \setlength{\parskip}{0pt plus 0.3pt}%
  \let\item\@idxitem
}{%
  \ifkorrekturansicht\clearpage\fi
}
\makeatother

\IfFileExists{\jobname-pw.ind}{\input{\jobname-pw.ind}}{}

% Quellenangabe nur in der Leseansicht
\ifkorrekturansicht\else
% Fallback-Definitionen, falls die .tex-Datei \titel etc. nicht gesetzt hat
\providecommand{\titel}{}
\providecommand{\editorInnen}{}
\providecommand{\dateiname}{\jobname}

\vspace{3cm}

\vfill

\footnotesize
\textsc{Quelle}: \titel. Herausgegeben von {\editorInnen}. In: \emph{Arthur Schnitzler: Briefwechsel mit Autorinnen und Autoren}.
 Digitale Edition, https://schnitzler-briefe.acdh.oeaw.ac.at/{\dateiname}.html (Stand \today)
\fi

\end{document}


      