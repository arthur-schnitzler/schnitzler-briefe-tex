%% latex-korrekturansicht-vorspann.tex
%% Vorspann für die Korrekturansicht.
%% Lädt die gemeinsame Datei latex-vorspann.tex mit gesetztem Schalter.

\newif\ifkorrekturansicht
\korrekturansichttrue

\input{../tex-inputs/latex-vorspann}


\section[Richard Beer-Hofmann an Arthur Schnitzler, 16. 9. 1894]{L00370 Richard Beer-Hofmann an Arthur Schnitzler, 16. 9. 1894}
\nopagebreak\mylabel{L00370v}
\rehead{ }\normalsize\beginnumbering\briefempfaengerindex{Schnitzler, Arthur@\textsc{Schnitzler, Arthur}!zzzBeer-Hofmann, Richard@\emph{von Richard Beer-Hofmann}!1894-09-161@{16. 9. 1894}|(be}
\toendnotes[C]{\smallbreak\pagebreak[2]}\Standort{CUL, Schnitzler, B 8.}
\physDesc{Postkarte, 387 Zeichen
\newline{}Handschrift: blaue Tinte, lateinische Kurrent
\newline{}Versand: 1) Stempel: »\nobreak{}\oindex{Bellagio@\textbf{Bellagio}, \emph{A.ADM3}|pwk}Bellagio (Como), 16 9 94\nobreak{}«.   2) Stempel: »\nobreak{}\oindex{IX., Alsergrund@\textbf{IX., Alsergrund}, \emph{A.ADM3}|pwk}Wien 9/3, 18. 9. 94, 8.V, Bestellt\nobreak{}«. 
\newline{}Schnitzler: mit Bleistift nummeriert: »41« }
\buchAbdrucke{\weitereDrucke{Arthur Schnitzler, Richard Beer-Hofmann: \emph{Briefwechsel 1891–1931}. Wien, Zürich: \emph{Europaverlag} 1992, S. 59–60.} }\toendnotes[C]{\smallbreak}\pstart{}{\pb}\textcolor{gray}{\textbf{A}}n Herrn\pend{}\pstart{}D\textsuperscript{r} Arthur Schnitzler\pend{}\pstart{}Wien\oindex{Wien@\textbf{Wien}, \emph{A.ADM2}|pw}\pend{}\pstart{}Austria\oindex{Oesterreich@\textbf{Österreich}, \emph{A.PCLI}|pw}\pend{}\pstart{}IX Frankgasse 1\oindex{Frankgasse 1@\textbf{Frankgasse 1}, \emph{Wohngebäude (K.WHS)}|pw}\pend{}{\bigskip}\vspace{1em}
\pstart
           \noindent{}{\pb}Lieber Arthur! Bin in Bellagio\oindex{Bellagio@\textbf{Bellagio}, \emph{A.ADM3}|pw}, – bis Ende der \label{K_L00370-1v}\edtext{Wochen}{\lemma{\textnormal{\emph{Wochen}}}\Cendnote{\textnormal{österreichisch dialektal für
                  »Woche«}}}\label{K_L00370-1} an den Seen. Wenn Sie was zum Schreiben haben und lieb sind,
               schreiben Sie mir bis Ende der Woche. »Pallanza\oindex{Pallanza@\textbf{Pallanza}, \emph{P.PPL}|pw}.« »Posta fermata«. Adresse mit lat. Lettern. Anfangsbuchstaben \uline{B} unterstrichen.\pend
           \pstart Herzlichst Ihr\spacefill\mbox{Richard}\pend{}
\pstart
           16/IX. 94. \pend
           
\pstart
           Grüße an Hugo\pwindex{Hofmannsthal, Hugo von 1874-02-01 – 1929-07-15@\textsc{Hofmannsthal, Hugo von} (1874-02-01 – 1929-07-15), \emph{Schriftsteller/Schriftstellerin}|pw}{ }Bahr\pwindex{Bahr, Hermann 19.07.1863 – 15.01.1934@\textsc{Bahr, Hermann} (19.07.1863 – 15.01.1934), \emph{Schriftsteller/Schriftstellerin, Kritiker/Kritikerin}|pw} etc.\pend
           
\pstart
           Entschuldigung an A. S.\pwindex{Sandrock, Adele 1863-08-19 – 1937-08-30@\textsc{Sandrock, Adele} (1863-08-19 – 1937-08-30), \emph{Schauspieler/Schauspielerin}|pw} wegen verspäteten
                  Dankes.\pend
           \selectlanguage{ngerman}\endnumbering\briefempfaengerindex{Schnitzler, Arthur@\textsc{Schnitzler, Arthur}!zzzBeer-Hofmann, Richard@\emph{von Richard Beer-Hofmann}!1894-09-161@{16. 9. 1894}|)be}\mylabel{L00370h}  \normalsize

\doendnotes{C}
\bigskip
\vfill

\clearpage

\footnotesize

\lohead{\textsc{register}}

% Definiere theindex-Environment komplett neu ohne reledmac
\makeatletter
\renewenvironment{theindex}{%
  \section*{\indexname}%
  \setlength{\parindent}{0pt}%
  \setlength{\parskip}{0pt plus 0.3pt}%
  \let\item\@idxitem
}{%
  \clearpage
}
\makeatother

\IfFileExists{\jobname-pw.ind}{\input{\jobname-pw.ind}}{}

\end{document}

      