%% latex-korrekturansicht-vorspann.tex
%% Vorspann für die Korrekturansicht.
%% Lädt die gemeinsame Datei latex-vorspann.tex mit gesetztem Schalter.

\newif\ifkorrekturansicht
\korrekturansichttrue

\input{../tex-inputs/latex-vorspann}


\section[Arthur Schnitzler an Richard Beer-Hofmann, {[}14. 11.? 1893{]}]{L00268 Arthur Schnitzler an Richard Beer-Hofmann, {[}14. 11.? 1893{]}}
\nopagebreak\mylabel{L00268v}
\rehead{ }\normalsize\beginnumbering\briefempfaengerindex{Beer-Hofmann, Richard@\textsc{Beer-Hofmann, Richard}!zzzSchnitzler, Arthur@\emph{von Arthur Schnitzler}!1893-11-141@{{[}14. 11.? 1893{]}}|(be}
\toendnotes[C]{\smallbreak\pagebreak[2]}\Standort{YCGL, MSS 31.}
\physDesc{Brief, 1 Blatt, 2 Seiten, 194 Zeichen (Briefpapier mit Trauerrand)
\newline{}Handschrift: Bleistift, deutsche Kurrent}\toendnotes[C]{\smallbreak}
\pstart{}{\pb}Lieber Richard,\pend\vspace{0.5em}
\pstart
           bitte ſehr, ſenden Sie durch Ueberbringer dieſes den \label{K_L00268-1v}\edtext{\textsc{Rosé}\pwindex{Rose, Arnold 24.10.1863 – 25.08.1946@\textsc{Rosé, Arnold} (24.10.1863 – 25.08.1946), \emph{Violinist/Violinistin}|pw}ſitz}{\lemma{\textnormal{\emph{Roséſitz}}}\Cendnote{\textnormal{Das undatierte Korrespondenzstück ist
                  mit Trauerrand versehen und damit nach dem Tod des Vaters\pwindex{Schnitzler, Johann 10.04.1835 – 02.05.1893@\textsc{Schnitzler, Johann} (10.04.1835 – 02.05.1893), \emph{Laryngologe/Laryngologin}|pwkv} anzusetzen, dessen Ordination\oindex{Wohnung und Ordination Johann Schnitzler Burgring 1@\textbf{Wohnung und Ordination Johann Schnitzler Burgring 1}, \emph{Ordination}|pwkv}{ }Schnitzler
                  weiter betreut haben dürfte. Da Schnitzler nach dem Sterbefall für fünf Monate nicht ins Theater ging
                  und am 15. 11. 1893 in eine neue Wohnung übersiedelte, lässt sich ein mögliches
                  Zeitfenster eingrenzen. Arnold Rosé\pwindex{Rose, Arnold 24.10.1863 – 25.08.1946@\textsc{Rosé, Arnold} (24.10.1863 – 25.08.1946), \emph{Violinist/Violinistin}|pwk} war ein beliebter Violinist,
                  dessen Aufführungen Schnitzler gerne
                  besuchte. Im \emph{Tagebuch}\pwindex{Tagebuch@\emph{Tagebuch}|pwk} erwähnt Schnitzler keinen Konzertbesuch, doch
                  im Verzeichnis seiner Theaterbesuche (\emph{CUL}, A 179) notiert
                  er für den 14. 11. 1893 den Besuch eines philharmonischen Konzerts. Ein
                  solches kann für diesen Tag nicht nachgewiesen werden, sehr wohl aber den ersten von 
                  sechs Abenden des \emph{Quartett Rosé}\orgindex{Rose-Quartett@Rosé-Quartett|pwk}, den dieses gemeinsam mit 
                  Anton Rückauf\pwindex{Rueckauf, Anton 13.03.1855 – 19.09.1903@\textsc{Rückauf, Anton} (13.03.1855 – 19.09.1903), \emph{Komponist/Komponistin, Pianist/Pianistin}|pwk} bestritt. Dadurch wird eine unsichere 
               Datierung des Korrespondenzstücks möglich.}}}\label{K_L00268-1}, den Sie wohl noch bei ſich haben, {\pb}\textsc{Burgring 1\oindex{Wohnung und Ordination Johann Schnitzler Burgring 1@\textbf{Wohnung und Ordination Johann Schnitzler Burgring 1}, \emph{Ordination}|pw}}. – (an meinen Namen)\pend
           
\pstart
           Herzlich{\\[\baselineskip]}Ihr \spacefill\mbox{Arthur.}\pend
           \leftskip=0em{}
\pstart
           \noindent{}Seh ich Sie heut Abend? hoffentlich\pend
           \selectlanguage{ngerman}\endnumbering\briefempfaengerindex{Beer-Hofmann, Richard@\textsc{Beer-Hofmann, Richard}!zzzSchnitzler, Arthur@\emph{von Arthur Schnitzler}!1893-11-141@{{[}14. 11.? 1893{]}}|)be}\mylabel{L00268h}  \normalsize

\doendnotes{C}
\bigskip
\vfill

\clearpage

\footnotesize

\lohead{\textsc{register}}

% Definiere theindex-Environment komplett neu ohne reledmac
\makeatletter
\renewenvironment{theindex}{%
  \section*{\indexname}%
  \setlength{\parindent}{0pt}%
  \setlength{\parskip}{0pt plus 0.3pt}%
  \let\item\@idxitem
}{%
  \clearpage
}
\makeatother

\IfFileExists{\jobname-pw.ind}{\input{\jobname-pw.ind}}{}

\end{document}

      