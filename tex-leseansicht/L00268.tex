%% latex-leseansicht-vorspann.tex
%% Vorspann für die Leseansicht.
%% Lädt die gemeinsame Datei latex-vorspann.tex mit nicht gesetztem Schalter.

\newif\ifkorrekturansicht
\korrekturansichtfalse

\input{../tex-inputs/latex-vorspann}

\begin{center}
            \textcolor{red}{ENTWURF. ENTZIFFERUNG NOCH NICHT KORREKTURGELESEN}
                      \end{center}
            
               \section[Arthur Schnitzler an Richard Beer-Hofmann, {[}zwischen 5. 10. und 14. 11. 1893{]}]{ Arthur Schnitzler an Richard Beer-Hofmann, {[}zwischen 5. 10. und
               14. 11. 1893{]}}\nopagebreak\mylabel{v}\rehead{ }\begin{ledgroupsized}[t]{13cm}\normalsize\beginnumbering\briefempfaengerindex{Beer-Hofmann, Richard@\textsc{Beer-Hofmann, Richard}!zzzSchnitzler, Arthur@\emph{von Arthur Schnitzler}!1893-10-051@{{[}zwischen 5. 10. und
                  14. 11. 1893{]}}|(be} \toendnotes[C]{\smallbreak\pagebreak[2]} \Standort{YCGL, MSS 31.}
\physDesc{Brief, 1 Blatt (Briefpapier mit Trauerrand), 2 Seiten
\newline{}Handschrift: Bleistift, deutsche Kurrent}\toendnotes[C]{\smallbreak}\pstart{}{\pb}Lieber Richard,\pend\pstart
           bitte ſehr, ſenden Sie durch Ueberbringer dieſes den \label{K_L00268_1v}\edtext{\textsc{Rosé}\pwindex{Rose, Arnold 24.10.1863 – 25.08.1946@\textsc{Rosé, Arnold} (24.10.1863 – 25.08.1946), \emph{Violinist}|pw}ſitz}{\lemma{\textnormal{\emph{Roséſitz}}}\Cendnote{\textnormal{Arnold Rosé\pwindex{Rose, Arnold 24.10.1863 – 25.08.1946@\textsc{Rosé, Arnold} (24.10.1863 – 25.08.1946), \emph{Violinist}|pwk} 
               war ein beliebter Violinist, dessen Aufführungen Schnitzler\pwindex{Schnitzler, Arthur 15.05.1862 – 21.10.1931@\textsc{Schnitzler, Arthur} (15.05.1862 – 21.10.1931), \emph{Schriftsteller, Mediziner}|pwk} gerne besuchte. Ein offensichtliches
                  Konzert bietet sich in dem Zeitraum aber nicht an, doch trat Rosé\pwindex{Rose, Arnold 24.10.1863 – 25.08.1946@\textsc{Rosé, Arnold} (24.10.1863 – 25.08.1946), \emph{Violinist}|pwk} mehrmals als Begleitmusiker
                  auf. Möglicherweise handelt es sich aber auch um bei \emph{Alexander Rosé Concertbureau}\orgindex{Alexander Rose Concertbuero@Alexander Rosé Concertbüro|pwk} besorgte
               Karten für eine musikalische Vorführung?}}}\label{K_L00268_1h}, den Sie wohl noch bei ſich haben,
                  {\pb}\textsc{\label{K_L00268_2v}\edtext{Burgring 1\oindex{Burgring@\textbf{Burgring}|pw}}{\lemma{\textnormal{\emph{Burgring 1}}}\Cendnote{\textnormal{Das
                     undatierte Korrespondenzstück ist mit Trauerrand versehen und damit nach dem Tod des 
                        Vaters\pwindex{Schnitzler, Johann 10.04.1835 – 02.05.1893@\textsc{Schnitzler, Johann} (10.04.1835 – 02.05.1893), \emph{Laryngologe}|pwkv} anzusetzen. Da Schnitzler\pwindex{Schnitzler, Arthur 15.05.1862 – 21.10.1931@\textsc{Schnitzler, Arthur} (15.05.1862 – 21.10.1931), \emph{Schriftsteller, Mediziner}|pwk} für
                        fünf Monate nicht ins Theater ging und am 15. 11. 1893 übersiedelte,
                        lässt sich das mögliche Zeitfenster weiter eingrenzen.}}}\label{K_L00268_2h}}. – (an meinen
               Namen)\pend
           \pstart
           Herzlich{\\[\baselineskip]}Ihr \spacefill\mbox{Arthur.}\pend
           \leftskip=0em{}\pstart
           \noindent{}Seh ich Sie heut Abend? hoffentlich\pend
           \endnumbering\briefempfaengerindex{Beer-Hofmann, Richard@\textsc{Beer-Hofmann, Richard}!zzzSchnitzler, Arthur@\emph{von Arthur Schnitzler}!1893-10-051@{{[}zwischen 5. 10. und
                  14. 11. 1893{]}}|)be}\mylabel{h}\end{ledgroupsized}  \newcommand{\dateiname}{L00268}\newcommand{\titel}{Arthur Schnitzler an Richard Beer-Hofmann, [zwischen 5. 10. und 14. 11. 1893]}\newcommand{\editorInnen}{Martin Anton Müller und Gerd-Hermann Susen}%% latex-leseansicht-abspann.tex
%% Abspann für die Leseansicht.
%% Der Schalter \ifkorrekturansicht ist bereits durch den Vorspann gesetzt.

%% latex-abspann.tex
%% Gemeinsamer Abspann für Korrekturansicht und Leseansicht.
%% Setzt den Schalter \ifkorrekturansicht voraus (gesetzt in den
%% einbindenden Dateien latex-korrekturansicht-abspann.tex bzw.
%% latex-leseansicht-abspann.tex).
%% ---------------------------------------------------------------

\normalsize

% Das esempio-Environment wird nur in der Leseansicht benötigt
\ifkorrekturansicht\else
\newenvironment{esempio}[3]%
{
    \vspace{1.5ex}
    \rlap{\underline{#1}}
    \par
    \setlength{\parindent}{0cm}
    \nopagebreak
    \leftskip=#2cm
    \rightskip=#3cm
}
{
    \par
}
\fi

\doendnotes{C}
\bigskip
\vfill

\clearpage

\footnotesize

\ifkorrekturansicht
  \lohead{\textsc{register}}
\fi

% theindex-Environment neu definieren ohne reledmac
\makeatletter
\renewenvironment{theindex}{%
  \ifkorrekturansicht
    \section*{\indexname}%
  \else
    \subsubsection*{Index der erwähnten Entitäten}%
  \fi
  \setlength{\parindent}{0pt}%
  \setlength{\parskip}{0pt plus 0.3pt}%
  \let\item\@idxitem
}{%
  \ifkorrekturansicht\clearpage\fi
}
\makeatother

\IfFileExists{\jobname-pw.ind}{\input{\jobname-pw.ind}}{}

% Quellenangabe nur in der Leseansicht
\ifkorrekturansicht\else
% Fallback-Definitionen, falls die .tex-Datei \titel etc. nicht gesetzt hat
\providecommand{\titel}{}
\providecommand{\editorInnen}{}
\providecommand{\dateiname}{\jobname}

\vspace{3cm}

\vfill

\footnotesize
\textsc{Quelle}: \titel. Herausgegeben von {\editorInnen}. In: \emph{Arthur Schnitzler: Briefwechsel mit Autorinnen und Autoren}.
 Digitale Edition, https://schnitzler-briefe.acdh.oeaw.ac.at/{\dateiname}.html (Stand \today)
\fi

\end{document}


      