%% latex-leseansicht-vorspann.tex
%% Vorspann für die Leseansicht.
%% Lädt die gemeinsame Datei latex-vorspann.tex mit nicht gesetztem Schalter.

\newif\ifkorrekturansicht
\korrekturansichtfalse

\input{../tex-inputs/latex-vorspann}


\section[Arthur Schnitzler an Richard Beer-Hofmann, {{[}}14. 11.? 1893{{]}}]{L00268 Arthur Schnitzler an Richard Beer-Hofmann, {[}14. 11.? 1893{]}}
\nopagebreak\mylabel{L00268v}
\rehead{ }\normalsize\beginnumbering\briefempfaengerindex{Beer-Hofmann, Richard@\textsc{Beer-Hofmann, Richard}!zzzSchnitzler, Arthur@\emph{von Arthur Schnitzler}!1893-11-141@{{[}14. 11.? 1893{]}}|(be}
\toendnotes[C]{\smallbreak\pagebreak[2]}
\correspDesc{Versand  durch Arthur Schnitzler am [14. 11.? 1893] in Wien
\newline{}Erhalt  durch Richard Beer-Hofmann am [14. 11.? 1893] in Wien}\toendnotes[C]{\smallbreak}
\Standort{YCGL, MSS 31.}
\physDesc{Brief, 1 Blatt, 2 Seiten, 194 Zeichen (Briefpapier mit Trauerrand)
\newline{}Handschrift: Bleistift, deutsche Kurrent}\toendnotes[C]{\smallbreak}
\pstart{}{\pb}Lieber Richard,\pend\vspace{0.5em}
\pstart
           bitte{ }ſehr,{ }ſenden Sie durch Ueberbringer dieſes den \label{K_L00268-1v}\edtext{\textsc{Rosé}\pwindex{Rosé, Arnold 24.\,10.\,1863 Iași – 25.\,8.\,1946 London@\textsc{Rosé, Arnold} (24.\,10.\,1863 Iași – 25.\,8.\,1946 London), \emph{Violinist}|pw}ſitz}{\lemma{\textnormal{\emph{Rosésitz}}}\Cendnote{\textnormal{Das undatierte Korrespondenzstück ist
                  mit Trauerrand versehen und damit nach dem Tod des Vaters\pwindex{Schnitzler, Johann 10.\,4.\,1835 Nagykanizsa – 2.\,5.\,1893 Wien@\textsc{Schnitzler, Johann} (10.\,4.\,1835 Nagykanizsa – 2.\,5.\,1893 Wien), \emph{Laryngologe}|pwkv} anzusetzen, dessen Ordination\oindex{Wien@\textbf{Wien}!I., Innere Stadt@\textbf{I., Innere Stadt}!Wohnung und Ordination Johann Schnitzler Burgring 1@\textbf{Wohnung und Ordination Johann Schnitzler Burgring 1}, \emph{Ordination}|pwkv}{ }Schnitzler
                  weiter betreut haben dürfte. Da Schnitzler nach dem Sterbefall für fünf Monate nicht ins Theater ging
                  und am 15. 11. 1893 in eine neue Wohnung übersiedelte, lässt sich ein mögliches
                  Zeitfenster eingrenzen. Arnold Rosé\pwindex{Rosé, Arnold 24.\,10.\,1863 Iași – 25.\,8.\,1946 London@\textsc{Rosé, Arnold} (24.\,10.\,1863 Iași – 25.\,8.\,1946 London), \emph{Violinist}|pwk} war ein beliebter Violinist,
                  dessen Aufführungen Schnitzler gerne
                  besuchte. Im \emph{Tagebuch}\pwindex{Schnitzler, Arthur 15.\,5.\,1862 Wien – 21.\,10.\,1931 ebd.@\textsc{Schnitzler, Arthur} (15.\,5.\,1862 Wien – 21.\,10.\,1931 ebd.), \emph{Schriftsteller, Mediziner}!Tagebuch@\strich\emph{Tagebuch}|pwk} erwähnt Schnitzler keinen Konzertbesuch, doch
                  im Verzeichnis seiner Theaterbesuche (\emph{CUL}, A 179) notiert
                  er für den 14. 11. 1893 den Besuch eines philharmonischen Konzerts. Ein
                  solches kann für diesen Tag nicht nachgewiesen werden, sehr wohl aber den ersten von 
                  sechs Abenden des \emph{Quartett Rosé}\orgindex{Rosé-Quartett@Rosé-Quartett|pwk}, den dieses gemeinsam mit 
                  Anton Rückauf\pwindex{Rückauf, Anton 13.\,3.\,1855 Prag – 19.\,9.\,1903 Schloss Alterlaa@\textsc{Rückauf, Anton} (13.\,3.\,1855 Prag – 19.\,9.\,1903 Schloss Alterlaa), \emph{Komponist, Pianist}|pwk} bestritt. Dadurch wird eine unsichere 
               Datierung des Korrespondenzstücks möglich.}}}\label{K_L00268-1}, den Sie wohl noch bei{ }ſich haben, {\pb}\textsc{Burgring 1\oindex{Wien@\textbf{Wien}!I., Innere Stadt@\textbf{I., Innere Stadt}!Wohnung und Ordination Johann Schnitzler Burgring 1@\textbf{Wohnung und Ordination Johann Schnitzler Burgring 1}, \emph{Ordination}|pw}}. – (an meinen Namen)\pend
           
\pstart
           Herzlich{\\[\baselineskip]}Ihr \spacefill\mbox{Arthur.}\pend
           \leftskip=0em{}
\pstart
           \noindent{}Seh ich Sie heut Abend? hoffentlich\pend
           \selectlanguage{ngerman}\endnumbering\briefempfaengerindex{Beer-Hofmann, Richard@\textsc{Beer-Hofmann, Richard}!zzzSchnitzler, Arthur@\emph{von Arthur Schnitzler}!1893-11-141@{{[}14. 11.? 1893{]}}|)be}\mylabel{L00268h}  \newcommand{\dateiname}{L00268}\newcommand{\titel}{Arthur Schnitzler an Richard Beer-Hofmann, [14. 11.? 1893]}\newcommand{\editorInnen}{Martin Anton Müller und Gerd-Hermann Susen}%% latex-leseansicht-abspann.tex
%% Abspann für die Leseansicht.
%% Der Schalter \ifkorrekturansicht ist bereits durch den Vorspann gesetzt.

%% latex-abspann.tex
%% Gemeinsamer Abspann für Korrekturansicht und Leseansicht.
%% Setzt den Schalter \ifkorrekturansicht voraus (gesetzt in den
%% einbindenden Dateien latex-korrekturansicht-abspann.tex bzw.
%% latex-leseansicht-abspann.tex).
%% ---------------------------------------------------------------

\normalsize

% Das esempio-Environment wird nur in der Leseansicht benötigt
\ifkorrekturansicht\else
\newenvironment{esempio}[3]%
{
    \vspace{1.5ex}
    \rlap{\underline{#1}}
    \par
    \setlength{\parindent}{0cm}
    \nopagebreak
    \leftskip=#2cm
    \rightskip=#3cm
}
{
    \par
}
\fi

\doendnotes{C}
\bigskip
\vfill

\clearpage

\footnotesize

\ifkorrekturansicht
  \lohead{\textsc{register}}
\fi

% theindex-Environment neu definieren ohne reledmac
\makeatletter
\renewenvironment{theindex}{%
  \ifkorrekturansicht
    \section*{\indexname}%
  \else
    \subsubsection*{Index der erwähnten Entitäten}%
  \fi
  \setlength{\parindent}{0pt}%
  \setlength{\parskip}{0pt plus 0.3pt}%
  \let\item\@idxitem
}{%
  \ifkorrekturansicht\clearpage\fi
}
\makeatother

\IfFileExists{\jobname-pw.ind}{\input{\jobname-pw.ind}}{}

% Quellenangabe nur in der Leseansicht
\ifkorrekturansicht\else
% Fallback-Definitionen, falls die .tex-Datei \titel etc. nicht gesetzt hat
\providecommand{\titel}{}
\providecommand{\editorInnen}{}
\providecommand{\dateiname}{\jobname}

\vspace{3cm}

\vfill

\footnotesize
\textsc{Quelle}: \titel. Herausgegeben von {\editorInnen}. In: \emph{Arthur Schnitzler: Briefwechsel mit Autorinnen und Autoren}.
 Digitale Edition, https://schnitzler-briefe.acdh.oeaw.ac.at/{\dateiname}.html (Stand \today)
\fi

\end{document}


