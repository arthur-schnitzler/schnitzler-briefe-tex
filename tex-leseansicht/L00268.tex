%% latex-leseansicht-vorspann.tex
%% Vorspann für die Leseansicht.
%% Lädt die gemeinsame Datei latex-vorspann.tex mit nicht gesetztem Schalter.

\newif\ifkorrekturansicht
\korrekturansichtfalse

\input{../tex-inputs/latex-vorspann}


         
         \renewcommand{\erwaehntePersonen}{Personen: Richard Beer-Hofmann, Arnold Rosé, Anton Rückauf, Johann Schnitzler}
         \renewcommand{\erwaehnteInstitutionen}{Institutionen: Rosé-Quartett}
         \renewcommand{\erwaehnteOrte}{Orte: Wien, Wohnung und Ordination Johann Schnitzler Burgring 1}
         \renewcommand{\erwaehnteWerke}{Werke: Tagebuch}
               \section[Arthur Schnitzler an Richard Beer-Hofmann, {[}14. 11.? 1893{]}]{ Arthur Schnitzler an Richard Beer-Hofmann, {[}14. 11.? 1893{]}}\nopagebreak\mylabel{v}\rehead{ }\begin{ledgroupsized}[t]{13cm}\normalsize\beginnumbering\briefempfaengerindex{Beer-Hofmann, Richard@\textsc{Beer-Hofmann, Richard}!zzzSchnitzler, Arthur@\emph{von Arthur Schnitzler}!1893-11-141@{{[}14. 11.? 1893{]}}|(be} \toendnotes[C]{\smallbreak\pagebreak[2]} \Standort{YCGL, MSS 31.}
\physDesc{Brief, 1 Blatt, 2 Seiten, 194 Zeichen (Briefpapier mit Trauerrand)
\newline{}Handschrift: Bleistift, deutsche Kurrent}\toendnotes[C]{\smallbreak}\pstart{}{\pb}Lieber Richard,\pend\pstart
           bitte ſehr, ſenden Sie durch Ueberbringer dieſes den \label{K_L00268-1v}\edtext{\textsc{Rosé}\pwindex{Rose, Arnold 24.10.1863 – 25.08.1946@\textsc{Rosé, Arnold} (24.10.1863 – 25.08.1946), \emph{Violinist}|pw}ſitz}{\lemma{\textnormal{\emph{Roséſitz}}}\Cendnote{\textnormal{Das undatierte Korrespondenzstück ist
                  mit Trauerrand versehen und damit nach dem Tod des Vaters\pwindex{Schnitzler, Johann 10.04.1835 – 02.05.1893@\textsc{Schnitzler, Johann} (10.04.1835 – 02.05.1893), \emph{Laryngologe}|pwkv} anzusetzen, dessen Ordination\oindex{Wohnung und Ordination Johann Schnitzler Burgring 1@\textbf{Wohnung und Ordination Johann Schnitzler Burgring 1}|pwkv}{ }Schnitzler\pwindex{Schnitzler, Arthur 15.05.1862 – 21.10.1931@\textsc{Schnitzler, Arthur} (15.05.1862 – 21.10.1931), \emph{Schriftsteller, Mediziner}|pwk}
                  weiter betreut haben dürfte. Da Schnitzler\pwindex{Schnitzler, Arthur 15.05.1862 – 21.10.1931@\textsc{Schnitzler, Arthur} (15.05.1862 – 21.10.1931), \emph{Schriftsteller, Mediziner}|pwk} nach dem Sterbefall für fünf Monate nicht ins Theater ging
                  und am 15. 11. 1893 in eine neue Wohnung übersiedelte, lässt sich ein mögliches
                  Zeitfenster eingrenzen. Arnold Rosé\pwindex{Rose, Arnold 24.10.1863 – 25.08.1946@\textsc{Rosé, Arnold} (24.10.1863 – 25.08.1946), \emph{Violinist}|pwk} war ein beliebter Violinist,
                  dessen Aufführungen Schnitzler\pwindex{Schnitzler, Arthur 15.05.1862 – 21.10.1931@\textsc{Schnitzler, Arthur} (15.05.1862 – 21.10.1931), \emph{Schriftsteller, Mediziner}|pwk} gerne
                  besuchte. Im \emph{Tagebuch}\pwindex{\textcolor{red}{\textsuperscript{XXXX1 indx}}!Tagebuch1981 – 2000@\strich\emph{Tagebuch} {[}Hrsg., 1981 – 2000{]}|pwk} erwähnt Schnitzler\pwindex{Schnitzler, Arthur 15.05.1862 – 21.10.1931@\textsc{Schnitzler, Arthur} (15.05.1862 – 21.10.1931), \emph{Schriftsteller, Mediziner}|pwk} keinen Konzertbesuch, doch
                  im Verzeichnis seiner Theaterbesuche (\emph{CUL}, A 179) notiert
                  er für den 14. 11. 1893 den Besuch eines philharmonischen Konzerts. Ein
                  solches kann für diesen Tag nicht nachgewiesen werden, sehr wohl aber den ersten von 
                  sechs Abenden des \emph{Quartett Rosé}\orgindex{Rose-Quartett@Rosé-Quartett|pwk}, den dieses gemeinsam mit 
                  Anton Rückauf\pwindex{Rueckauf, Anton 13.03.1855 – 19.09.1903@\textsc{Rückauf, Anton} (13.03.1855 – 19.09.1903), \emph{Komponist, Pianist}|pwk} bestritt. Dadurch wird eine unsichere 
               Datierung des Korrespondenzstücks möglich.}}}\label{K_L00268-1h}, den Sie wohl noch bei ſich haben, {\pb}\textsc{Burgring 1\oindex{Wohnung und Ordination Johann Schnitzler Burgring 1@\textbf{Wohnung und Ordination Johann Schnitzler Burgring 1}|pw}}. – (an meinen Namen)\pend
           \pstart
           Herzlich{\\[\baselineskip]}Ihr \spacefill\mbox{Arthur.}\pend
           \leftskip=0em{}\pstart
           \noindent{}Seh ich Sie heut Abend? hoffentlich\pend
           
         
         \endnumbering\mylabel{h}\end{ledgroupsized}  \newcommand{\dateiname}{L00268}\newcommand{\titel}{Arthur Schnitzler an Richard Beer-Hofmann, [14. 11.? 1893]}\newcommand{\editorInnen}{Martin Anton Müller und Gerd-Hermann Susen}%% latex-leseansicht-abspann.tex
%% Abspann für die Leseansicht.
%% Der Schalter \ifkorrekturansicht ist bereits durch den Vorspann gesetzt.

%% latex-abspann.tex
%% Gemeinsamer Abspann für Korrekturansicht und Leseansicht.
%% Setzt den Schalter \ifkorrekturansicht voraus (gesetzt in den
%% einbindenden Dateien latex-korrekturansicht-abspann.tex bzw.
%% latex-leseansicht-abspann.tex).
%% ---------------------------------------------------------------

\normalsize

% Das esempio-Environment wird nur in der Leseansicht benötigt
\ifkorrekturansicht\else
\newenvironment{esempio}[3]%
{
    \vspace{1.5ex}
    \rlap{\underline{#1}}
    \par
    \setlength{\parindent}{0cm}
    \nopagebreak
    \leftskip=#2cm
    \rightskip=#3cm
}
{
    \par
}
\fi

\doendnotes{C}
\bigskip
\vfill

\clearpage

\footnotesize

\ifkorrekturansicht
  \lohead{\textsc{register}}
\fi

% theindex-Environment neu definieren ohne reledmac
\makeatletter
\renewenvironment{theindex}{%
  \ifkorrekturansicht
    \section*{\indexname}%
  \else
    \subsubsection*{Index der erwähnten Entitäten}%
  \fi
  \setlength{\parindent}{0pt}%
  \setlength{\parskip}{0pt plus 0.3pt}%
  \let\item\@idxitem
}{%
  \ifkorrekturansicht\clearpage\fi
}
\makeatother

\IfFileExists{\jobname-pw.ind}{\input{\jobname-pw.ind}}{}

% Quellenangabe nur in der Leseansicht
\ifkorrekturansicht\else
% Fallback-Definitionen, falls die .tex-Datei \titel etc. nicht gesetzt hat
\providecommand{\titel}{}
\providecommand{\editorInnen}{}
\providecommand{\dateiname}{\jobname}

\vspace{3cm}

\vfill

\footnotesize
\textsc{Quelle}: \titel. Herausgegeben von {\editorInnen}. In: \emph{Arthur Schnitzler: Briefwechsel mit Autorinnen und Autoren}.
 Digitale Edition, https://schnitzler-briefe.acdh.oeaw.ac.at/{\dateiname}.html (Stand \today)
\fi

\end{document}


      