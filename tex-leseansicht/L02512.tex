%% latex-korrekturansicht-vorspann.tex
%% Vorspann für die Korrekturansicht.
%% Lädt die gemeinsame Datei latex-vorspann.tex mit gesetztem Schalter.

\newif\ifkorrekturansicht
\korrekturansichttrue

\input{../tex-inputs/latex-vorspann}


\section[Arthur Schnitzler an Robert Adam, 14. 6. 1929]{L02512 Arthur Schnitzler an Robert Adam, 14. 6. 1929}
\nopagebreak\mylabel{L02512v}
\rehead{ }\normalsize\beginnumbering\briefempfaengerindex{Adam, Robert@\textsc{Adam, Robert}!zzzSchnitzler, Arthur@\emph{von Arthur Schnitzler}!1929-06-141@{14. 6. 1929}|(be}
\toendnotes[C]{\smallbreak\pagebreak[2]}\Standort{DLA, 96.34.2/34.}
\physDesc{Brief, 1 Blatt, 1 Seite, Umschlag, 731 Zeichen (Briefpaper und Umschlag mit Trauerrand)
\newline{}Handschrift: schwarze Tinte, lateinische Kurrent
\newline{}Versand: Stempel: »\nobreak{}\oindex{XVIII., Waehring@\textbf{XVIII., Währing}, \emph{A.ADM3}|pwk}18/\textsubscript{1}Wien
                                       110, 15. \textcolor{gray}{XI}. 29, 7\nobreak{}«.  }\toendnotes[C]{\smallbreak}\pstart{}{\pb}\label{T_L02512-1v}\edtext{\textcolor{gray}{\textbf{A. S.}}}{\lemma{\textnormal{\emph{A. S.}}}\Cendnote{\textnormal{ovaler Absenderkleber}}}\label{T_L02512-1}\pend{}\pstart{}\textcolor{gray}{\textbf{WIEN, XVIII.}}\oindex{XVIII., Waehring@\textbf{XVIII., Währing}, \emph{A.ADM3}|pw}\pend{}\pstart{}\textcolor{gray}{\textbf{STERNWARTESTR. 71}}\oindex{Sternwartestrasse 71@\textbf{Sternwartestraße 71}, \emph{Wohngebäude (K.WHS)}|pw}\pend{}{\bigskip}\pstart{}{\pb}Herrn Ob.Landesger-Rath\pend{}\pstart{}Dr. Rob. Adam Pollak\pend{}\pstart{}Wien XII\oindex{XII., Meidling@\textbf{XII., Meidling}, \emph{A.ADM3}|pw}\pend{}\pstart{}Meidlinger Hauptstr 58\oindex{Meidlinger Hauptstrasse@\textbf{Meidlinger Hauptstraße}, \emph{Straße (K.STR)}|pw}.\pend{}{\bigskip}\vspace{1em}
\pstart
           \raggedleft{}{\pb}Wien\oindex{Wien@\textbf{Wien}, \emph{A.ADM2}|pw}, 14/6 929\pend
           
\pstart{}Verehrter Herr Oberlandesgerichtsrath,\pend\vspace{0.5em}
\pstart
           ich fahre dieser Tage auf den Semmering\oindex{Semmering@\textbf{Semmering}, \emph{A.ADM3}|pw}; nach
               meiner Rückkehr Anfang Juli wird es mir ein besondres Vergnügen sein,
               Sie nach so langer Zeit wieder einmal bei mir zu sehen. Ob eine Bühne sich
               entschließen wird, Ihre Margot\pwindex{Margot und das Jugendgericht@\emph{Margot und das Jugendgericht}|pw} zur Aufführung zu
               bringen, läßt sich schwer voraussagen; die \textcolor{gray}{Galerie}, so lustig sie
               ist – und selbst angeno{\geminationm}en, es stecke mehr bittre
               Wahrheit drin als heitre Erfindung, scheint mir stellenweise in künstlerischem Sinne
               so grob, als daſs ein Theaterpublikum die rechte Freude daran haben sollte.\pend
           \pstart Aber unfehlbar bin ich nicht. Also auf bald, und herzliche Grüße\hspace*{1.5em}Ihr sehr ergebner\spacefill\mbox{ArthSchnitzler}\pend{}\selectlanguage{ngerman}\endnumbering\briefempfaengerindex{Adam, Robert@\textsc{Adam, Robert}!zzzSchnitzler, Arthur@\emph{von Arthur Schnitzler}!1929-06-141@{14. 6. 1929}|)be}\mylabel{L02512h}  \normalsize

\doendnotes{C}
\bigskip
\vfill

\clearpage

\footnotesize

\lohead{\textsc{register}}

% Definiere theindex-Environment komplett neu ohne reledmac
\makeatletter
\renewenvironment{theindex}{%
  \section*{\indexname}%
  \setlength{\parindent}{0pt}%
  \setlength{\parskip}{0pt plus 0.3pt}%
  \let\item\@idxitem
}{%
  \clearpage
}
\makeatother

\IfFileExists{\jobname-pw.ind}{\input{\jobname-pw.ind}}{}

\end{document}

      