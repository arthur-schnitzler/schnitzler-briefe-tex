%% latex-leseansicht-vorspann.tex
%% Vorspann für die Leseansicht.
%% Lädt die gemeinsame Datei latex-vorspann.tex mit nicht gesetztem Schalter.

\newif\ifkorrekturansicht
\korrekturansichtfalse

\input{../tex-inputs/latex-vorspann}


\section[Olga Schnitzler an Richard und Paula Beer-Hofmann, 11. 9. 1910]{L02557 Olga Schnitzler an Richard und Paula Beer-Hofmann, 11. 9. 1910}
\nopagebreak\mylabel{L02557v}
\rehead{ }\normalsize\beginnumbering\briefempfaengerindex{Beer-Hofmann, Paula@\textsc{Beer-Hofmann, Paula}!zzzSchnitzler, Olga@\emph{von Olga Schnitzler}!1910-09-111@{11. 9. 1910}|(be}\briefempfaengerindex{Beer-Hofmann, Richard@\textsc{Beer-Hofmann, Richard}!zzzSchnitzler, Olga@\emph{von Olga Schnitzler}!1910-09-111@{11. 9. 1910}|(be}
\toendnotes[C]{\smallbreak\pagebreak[2]}
\correspDesc{Versand  durch Olga Schnitzler am 11. 9. 1910 in Wien
\newline{}Übermittlung  am 12. 9. 1910 in Wien
\newline{}Erhalt  durch Richard Beer-Hofmann, Paula Beer-Hofmann im Zeitraum [12. 9. 1910
                  – 15. 9. 1910?] in Wien}\toendnotes[C]{\smallbreak}
\Standort{YCGL, MSS 31.}
\physDesc{Postkarte, 446 Zeichen
\newline{}Handschrift: Bleistift, lateinische Kurrent
\newline{}Versand: Stempel: »\nobreak{}\oindex{IX., Alsergrund@\textbf{IX., Alsergrund}, \emph{Verwaltungsgebiet}|pwk}9/4 Wien, 1\textcolor{gray}{2. IX}. 10\nobreak{}«.  }\toendnotes[C]{\smallbreak}\pstart{}{\pb}Wien XVIII\oindex{XVIII., Währing@\textbf{XVIII., Währing}, \emph{Verwaltungsgebiet}|pw}\pend{}\pstart{}Sternwartestrasse 71\oindex{Wien@\textbf{Wien}!XVIII., Währing@\textbf{XVIII., Währing}!Sternwartestraße 71@\textbf{Sternwartestraße 71}, \emph{Wohngebäude}|pw}.\pend{}{\bigskip}\pstart{}Herrn\pend{}\pstart{}Dr. Rich. Beer-Hofma{\geminationn}\pend{}\pstart{}Ischl\oindex{Bad Ischl@\textbf{Bad Ischl}|pw}\pend{}\pstart{}Steinfeld 6\oindex{Steinfeld@\textbf{Steinfeld}|pw}\pend{}{\bigskip}\vspace{1em}
\pstart
           \noindent{}{\pb}Aber eigentlich ist die Paula gemeint, – ich habe
               eine Bitte an sie. Also, liebe Paula, wenn’s Ihnen keine Mühe macht, so lassen Sie
               mir von dort\oindex{Bad Ischl@\textbf{Bad Ischl}|pwv}{ }\uline{per Nachnahme} 3 Schachteln Mondsee\oindex{Mondsee@\textbf{Mondsee}, \emph{Hauptstadt}|pw}r Käse schicken, der Wild\orgindex{Gebrüder Wild – Schmalz- und Käsehandlung@Gebrüder Wild – Schmalz- und Käsehandlung|pw} hat ihn nicht.\pend
           
\pstart
           Danke, Küss die Hand zu wünschen!\pend
           
\pstart
           Viele Grüsse an \substVorne{}\textsuperscript{a}\substDazwischen{}A\substHinten{}lle! Hier ist es sehr schön, es regnet auch zuweilen, kommt bald, – alles
               verziehen!\pend
           \pstart Ihre \spacefill\mbox{Olga.}\pend{}
\pstart
           11. Sept. 1910.\pend
           \selectlanguage{ngerman}\endnumbering\briefempfaengerindex{Beer-Hofmann, Paula@\textsc{Beer-Hofmann, Paula}!zzzSchnitzler, Olga@\emph{von Olga Schnitzler}!1910-09-111@{11. 9. 1910}|)be}\briefempfaengerindex{Beer-Hofmann, Richard@\textsc{Beer-Hofmann, Richard}!zzzSchnitzler, Olga@\emph{von Olga Schnitzler}!1910-09-111@{11. 9. 1910}|)be}\mylabel{L02557h}  \newcommand{\dateiname}{L02557}\newcommand{\titel}{Olga Schnitzler an Richard und Paula Beer-Hofmann, 11. 9. 1910}\newcommand{\editorInnen}{Martin Anton Müller und Gerd-Hermann Susen}%% latex-leseansicht-abspann.tex
%% Abspann für die Leseansicht.
%% Der Schalter \ifkorrekturansicht ist bereits durch den Vorspann gesetzt.

%% latex-abspann.tex
%% Gemeinsamer Abspann für Korrekturansicht und Leseansicht.
%% Setzt den Schalter \ifkorrekturansicht voraus (gesetzt in den
%% einbindenden Dateien latex-korrekturansicht-abspann.tex bzw.
%% latex-leseansicht-abspann.tex).
%% ---------------------------------------------------------------

\normalsize

% Das esempio-Environment wird nur in der Leseansicht benötigt
\ifkorrekturansicht\else
\newenvironment{esempio}[3]%
{
    \vspace{1.5ex}
    \rlap{\underline{#1}}
    \par
    \setlength{\parindent}{0cm}
    \nopagebreak
    \leftskip=#2cm
    \rightskip=#3cm
}
{
    \par
}
\fi

\doendnotes{C}
\bigskip
\vfill

\clearpage

\footnotesize

\ifkorrekturansicht
  \lohead{\textsc{register}}
\fi

% theindex-Environment neu definieren ohne reledmac
\makeatletter
\renewenvironment{theindex}{%
  \ifkorrekturansicht
    \section*{\indexname}%
  \else
    \subsubsection*{Index der erwähnten Entitäten}%
  \fi
  \setlength{\parindent}{0pt}%
  \setlength{\parskip}{0pt plus 0.3pt}%
  \let\item\@idxitem
}{%
  \ifkorrekturansicht\clearpage\fi
}
\makeatother

\IfFileExists{\jobname-pw.ind}{\input{\jobname-pw.ind}}{}

% Quellenangabe nur in der Leseansicht
\ifkorrekturansicht\else
% Fallback-Definitionen, falls die .tex-Datei \titel etc. nicht gesetzt hat
\providecommand{\titel}{}
\providecommand{\editorInnen}{}
\providecommand{\dateiname}{\jobname}

\vspace{3cm}

\vfill

\footnotesize
\textsc{Quelle}: \titel. Herausgegeben von {\editorInnen}. In: \emph{Arthur Schnitzler: Briefwechsel mit Autorinnen und Autoren}.
 Digitale Edition, https://schnitzler-briefe.acdh.oeaw.ac.at/{\dateiname}.html (Stand \today)
\fi

\end{document}


