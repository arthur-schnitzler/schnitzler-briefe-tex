%% latex-leseansicht-vorspann.tex
%% Vorspann für die Leseansicht.
%% Lädt die gemeinsame Datei latex-vorspann.tex mit nicht gesetztem Schalter.

\newif\ifkorrekturansicht
\korrekturansichtfalse

\input{../tex-inputs/latex-vorspann}


\section[Arthur Schnitzler an Georg Engländer, 10. 1. 1919]{L02318 Arthur Schnitzler an Georg Engländer, 10. 1. 1919}
\nopagebreak\mylabel{L02318v}
\rehead{ }\normalsize\beginnumbering\briefempfaengerindex{Engländer, Georg@\textsc{Engländer, Georg}!zzzSchnitzler, Arthur@\emph{von Arthur Schnitzler}!1919-01-101@{10. 1. 1919}|(be}
\toendnotes[C]{\smallbreak\pagebreak[2]}
\correspDesc{Versand  durch Arthur Schnitzler am 10. 1. 1919 in Wien
\newline{}Erhalt  durch Georg Engländer am 26. 2. 1919 in Wien}\toendnotes[C]{\smallbreak}
\Standort{Wien, Österreichische Nationalbibliothek, 228/B8/1-3 LIT MAG.}
\physDesc{Brief, 1 Blatt, 1 Seite, 1092 Zeichen
\newline{}Handschrift: schwarze Tinte, lateinische Kurrent}\toendnotes[C]{\smallbreak}
\pstart
           \raggedleft{}{\pb}Wien\oindex{Wien@\textbf{Wien}, \emph{Verwaltungsgebiet}|pw}, 10. 1. 1919\pend
           
\pstart{}verehrter Herr Engländer,\pend\vspace{0.5em}
\pstart
           zu dem schweren Verlust, den die Welt durch das Hinscheiden Peter Altenbergs\pwindex{Altenberg, Peter 9.\,3.\,1859 Wien – 8.\,1.\,1919 ebd.@\textsc{Altenberg, Peter} (9.\,3.\,1859 Wien – 8.\,1.\,1919 ebd.), \emph{Schriftsteller}|pw} erlitten, bitte ich vor allem Sie als Bruder
               den Ausdruck meines innigsten Beileids entgegenzunehmen. Es hat sich, besonders in
               den spätern Jahren, freilich recht selten gefügt, daß ich ihn gesehen oder gesprochen
               hatte; – was sein kostbares, wundervolles Werk mir – vom ersten Buch\pwindex{Altenberg, Peter 9.\,3.\,1859 Wien – 8.\,1.\,1919 ebd.@\textsc{Altenberg, Peter} (9.\,3.\,1859 Wien – 8.\,1.\,1919 ebd.), \emph{Schriftsteller}!Wie ich es sehe@\strich\emph{Wie ich es sehe}|pwv} an bis zum letzten\pwindex{Altenberg, Peter 9.\,3.\,1859 Wien – 8.\,1.\,1919 ebd.@\textsc{Altenberg, Peter} (9.\,3.\,1859 Wien – 8.\,1.\,1919 ebd.), \emph{Schriftsteller}!Vita ipsa@\strich\emph{Vita ipsa}|pwv}, und in immer steigendem Maße bedeutet hat – und
               immer bedeuten wird, das – ich weiß es – hat er immer gefühlt. Jedem dieser Bücher
               hab ich mich entgegengefreut, jedes hat mich – über alles aesthetische Gefallen
               hinaus, manchmal ganz unabhängig davon, – im Innersten beglückt. Sein Leben ist dahin
               – die »Märchen seines Lebens\pwindex{Altenberg, Peter 9.\,3.\,1859 Wien – 8.\,1.\,1919 ebd.@\textsc{Altenberg, Peter} (9.\,3.\,1859 Wien – 8.\,1.\,1919 ebd.), \emph{Schriftsteller}!Märchen des Lebens@\strich\emph{Märchen des Lebens}|pwv}«
               (er hätte ja jedes Buch so nennen dürfen) werden uns weiter begleiten, – und unsere
               Söhne und Enkel und Urenkel wie uns – unvergänglich wie es eben die Märchen eines
               solchen Dichterlebens sind – wahrer als deren Wahrheiten und Legenden! –\pend
           
\pstart
           In herzlichster Antheilnahme drücke ich Ihnen, verehrter Herr, die Hand als Ihr
               sehr ergebener{\\[\baselineskip]}\spacefill\mbox{Arthur Schnitzler}\pend
           \leftskip=0em{}\selectlanguage{ngerman}\endnumbering\briefempfaengerindex{Engländer, Georg@\textsc{Engländer, Georg}!zzzSchnitzler, Arthur@\emph{von Arthur Schnitzler}!1919-01-101@{10. 1. 1919}|)be}\mylabel{L02318h}  \newcommand{\dateiname}{L02318}\newcommand{\titel}{Arthur Schnitzler an Georg Engländer, 10. 1. 1919}\newcommand{\editorInnen}{Martin Anton Müller und Gerd-Hermann Susen}%% latex-leseansicht-abspann.tex
%% Abspann für die Leseansicht.
%% Der Schalter \ifkorrekturansicht ist bereits durch den Vorspann gesetzt.

%% latex-abspann.tex
%% Gemeinsamer Abspann für Korrekturansicht und Leseansicht.
%% Setzt den Schalter \ifkorrekturansicht voraus (gesetzt in den
%% einbindenden Dateien latex-korrekturansicht-abspann.tex bzw.
%% latex-leseansicht-abspann.tex).
%% ---------------------------------------------------------------

\normalsize

% Das esempio-Environment wird nur in der Leseansicht benötigt
\ifkorrekturansicht\else
\newenvironment{esempio}[3]%
{
    \vspace{1.5ex}
    \rlap{\underline{#1}}
    \par
    \setlength{\parindent}{0cm}
    \nopagebreak
    \leftskip=#2cm
    \rightskip=#3cm
}
{
    \par
}
\fi

\doendnotes{C}
\bigskip
\vfill

\clearpage

\footnotesize

\ifkorrekturansicht
  \lohead{\textsc{register}}
\fi

% theindex-Environment neu definieren ohne reledmac
\makeatletter
\renewenvironment{theindex}{%
  \ifkorrekturansicht
    \section*{\indexname}%
  \else
    \subsubsection*{Index der erwähnten Entitäten}%
  \fi
  \setlength{\parindent}{0pt}%
  \setlength{\parskip}{0pt plus 0.3pt}%
  \let\item\@idxitem
}{%
  \ifkorrekturansicht\clearpage\fi
}
\makeatother

\IfFileExists{\jobname-pw.ind}{\input{\jobname-pw.ind}}{}

% Quellenangabe nur in der Leseansicht
\ifkorrekturansicht\else
% Fallback-Definitionen, falls die .tex-Datei \titel etc. nicht gesetzt hat
\providecommand{\titel}{}
\providecommand{\editorInnen}{}
\providecommand{\dateiname}{\jobname}

\vspace{3cm}

\vfill

\footnotesize
\textsc{Quelle}: \titel. Herausgegeben von {\editorInnen}. In: \emph{Arthur Schnitzler: Briefwechsel mit Autorinnen und Autoren}.
 Digitale Edition, https://schnitzler-briefe.acdh.oeaw.ac.at/{\dateiname}.html (Stand \today)
\fi

\end{document}


