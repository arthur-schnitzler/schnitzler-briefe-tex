%% latex-korrekturansicht-vorspann.tex
%% Vorspann für die Korrekturansicht.
%% Lädt die gemeinsame Datei latex-vorspann.tex mit gesetztem Schalter.

\newif\ifkorrekturansicht
\korrekturansichttrue

\input{../tex-inputs/latex-vorspann}


\section[Arthur Schnitzler an Georg Engländer, 10. 1. 1919]{L02318 Arthur Schnitzler an Georg Engländer, 10. 1. 1919}
\nopagebreak\mylabel{L02318v}
\rehead{ }\normalsize\beginnumbering\briefempfaengerindex{Englaender, Georg@\textsc{Engländer, Georg}!zzzSchnitzler, Arthur@\emph{von Arthur Schnitzler}!1919-01-101@{10. 1. 1919}|(be}
\toendnotes[C]{\smallbreak\pagebreak[2]}\Standort{Wien, Österreichische Nationalbibliothek, 228/B8/1-3 LIT MAG.}
\physDesc{Brief, 1 Blatt, 1 Seite, 1092 Zeichen
\newline{}Handschrift: schwarze Tinte, lateinische Kurrent}\toendnotes[C]{\smallbreak}
\pstart
           \raggedleft{}{\pb}Wien\oindex{Wien@\textbf{Wien}, \emph{A.ADM2}|pw}, 10. 1. 1919\pend
           
\pstart{}verehrter Herr Engländer,\pend\vspace{0.5em}
\pstart
           zu dem schweren Verlust, den die Welt durch das Hinscheiden Peter Altenbergs\pwindex{Altenberg, Peter 09.03.1859 – 08.01.1919@\textsc{Altenberg, Peter} (09.03.1859 – 08.01.1919), \emph{Schriftsteller/Schriftstellerin}|pw} erlitten, bitte ich vor allem Sie als Bruder
               den Ausdruck meines innigsten Beileids entgegenzunehmen. Es hat sich, besonders in
               den spätern Jahren, freilich recht selten gefügt, daß ich ihn gesehen oder gesprochen
               hatte; – was sein kostbares, wundervolles Werk mir – vom ersten Buch\pwindex{Wie ich es sehe@\emph{Wie ich es sehe}|pwv} an bis zum letzten\pwindex{Vita ipsa@\emph{Vita ipsa}|pwv}, und in immer steigendem Maße bedeutet hat – und
               immer bedeuten wird, das – ich weiß es – hat er immer gefühlt. Jedem dieser Bücher
               hab ich mich entgegengefreut, jedes hat mich – über alles aesthetische Gefallen
               hinaus, manchmal ganz unabhängig davon, – im Innersten beglückt. Sein Leben ist dahin
               – die »Märchen seines Lebens\pwindex{Maerchen des Lebens@\emph{Märchen des Lebens}|pwv}«
               (er hätte ja jedes Buch so nennen dürfen) werden uns weiter begleiten, – und unsere
               Söhne und Enkel und Urenkel wie uns – unvergänglich wie es eben die Märchen eines
               solchen Dichterlebens sind – wahrer als deren Wahrheiten und Legenden! –\pend
           
\pstart
           In herzlichster Antheilnahme drücke ich Ihnen, verehrter Herr, die Hand als Ihr
               sehr ergebener{\\[\baselineskip]}\spacefill\mbox{Arthur Schnitzler}\pend
           \leftskip=0em{}\selectlanguage{ngerman}\endnumbering\briefempfaengerindex{Englaender, Georg@\textsc{Engländer, Georg}!zzzSchnitzler, Arthur@\emph{von Arthur Schnitzler}!1919-01-101@{10. 1. 1919}|)be}\mylabel{L02318h}  \normalsize

\doendnotes{C}
\bigskip
\vfill

\clearpage

\footnotesize

\lohead{\textsc{register}}

% Definiere theindex-Environment komplett neu ohne reledmac
\makeatletter
\renewenvironment{theindex}{%
  \section*{\indexname}%
  \setlength{\parindent}{0pt}%
  \setlength{\parskip}{0pt plus 0.3pt}%
  \let\item\@idxitem
}{%
  \clearpage
}
\makeatother

\IfFileExists{\jobname-pw.ind}{\input{\jobname-pw.ind}}{}

\end{document}

      