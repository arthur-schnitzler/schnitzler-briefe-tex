%% latex-leseansicht-vorspann.tex
%% Vorspann für die Leseansicht.
%% Lädt die gemeinsame Datei latex-vorspann.tex mit nicht gesetztem Schalter.

\newif\ifkorrekturansicht
\korrekturansichtfalse

\input{../tex-inputs/latex-vorspann}


\section[Arthur Schnitzler an Stefan Zweig, {{[}}zwischen 14. und 16. 12. 1909?{{]}}]{L03807 Arthur Schnitzler an Stefan Zweig, {[}zwischen 14. und 16. 12. 1909?{]}}
\nopagebreak\mylabel{L03807v}
\rehead{ }\normalsize\beginnumbering\briefempfaengerindex{Zweig, Stefan@\textsc{Zweig, Stefan}!zzzSchnitzler, Arthur@\emph{von Arthur Schnitzler}!1909-12-163@{{[}zwischen 14. und 16. 12. 1909?{]}}|(be}
\toendnotes[C]{\smallbreak\pagebreak[2]}
\correspDesc{Versand  durch Arthur Schnitzler im Zeitraum [zwischen 14. und 16. 12. 1909?] in Wien
\newline{}Erhalt  durch Stefan Zweig im Zeitraum [zwischen 14. und 16. 12. 1909?] in Wien}\toendnotes[C]{\smallbreak}
\Standort{Jerusalem, National Library of Israel, ARC. Ms. Var. 305 1 58 Stefan Zweig Collection.}
\physDesc{Visitenkarte, 92 Zeichen
\newline{}Handschrift: Bleistift, deutsche Kurrent
\newline{}Ordnung: mit Bleistift von unbekannter Hand datiert: »1908(?)« }\toendnotes[C]{\smallbreak}
\pstart
           \noindent{}\centering{}{\pb}\textcolor{gray}{\textbf{D\textsuperscript{}\textsuperscript{r} Arthur Schnitzler}}\pend
           
\pstart
           {\pb}hab mich{ }ſehr über Ihren \label{K_L03807-1v}\edtext{ſchönen Brief}{\lemma{\textnormal{\emph{schönen Brief}}}\Cendnote{\textnormal{Die
                  Visitenkarte ist mit Bleistift geschrieben, was bedeuten dürfte, sie wurde abgegeben, als Schnitzler bei
                  einem Besuch bei Zweig\pwindex{Zweig, Stefan 28.\,11.\,1881 Wien – 23.\,2.\,1942 Petrópolis@\textsc{Zweig, Stefan} (28.\,11.\,1881 Wien – 23.\,2.\,1942 Petrópolis), \emph{Schriftsteller}|pwk} diesen nicht antraf. Wertet man unter dieser Prämisse die
                  frühen Briefe Zweigs\pwindex{Zweig, Stefan 28.\,11.\,1881 Wien – 23.\,2.\,1942 Petrópolis@\textsc{Zweig, Stefan} (28.\,11.\,1881 Wien – 23.\,2.\,1942 Petrópolis), \emph{Schriftsteller}|pwk} nach dem »ſchönen Brief« aus und schließt jene
                  aus, für die eine Antwort Schnitzlers überliefert ist, bleibt der Brief Zweigs\pwindex{Zweig, Stefan 28.\,11.\,1881 Wien – 23.\,2.\,1942 Petrópolis@\textsc{Zweig, Stefan} (28.\,11.\,1881 Wien – 23.\,2.\,1942 Petrópolis), \emph{Schriftsteller}|pwk}
                  vom XXXX Auszeichnungsfehler: Dokument L03624 nicht gefunden als wahrscheinlicher Referenzpunkt. Schnitzler
                  war zwischen 14. 12. 1909 und 17. 12. 1909
                  mehrfach in den inneren Bezirken Wiens\oindex{Wien@\textbf{Wien}, \emph{Verwaltungsgebiet}|pwk} unterwegs und machte Besuche. Am 17. 12. 1909
                  kam es zum Besuch bei Zweig\pwindex{Zweig, Stefan 28.\,11.\,1881 Wien – 23.\,2.\,1942 Petrópolis@\textsc{Zweig, Stefan} (28.\,11.\,1881 Wien – 23.\,2.\,1942 Petrópolis), \emph{Schriftsteller}|pwk}, so dass die vorliegende Visitenkarte in den drei vorangehenden 
                  Tagen abgegeben worden sein dürfte.}}}\label{K_L03807-1} gefreut wollte Ihnen d\textcolor{gray}{ie} Hand
               drücken.\pend
           
\pstart
           Herzlichſt{\\[\baselineskip]}Ihr \spacefill\mbox{A. S.}\pend
           \leftskip=0em{}\selectlanguage{ngerman}\endnumbering\briefempfaengerindex{Zweig, Stefan@\textsc{Zweig, Stefan}!zzzSchnitzler, Arthur@\emph{von Arthur Schnitzler}!1909-12-143@{{[}zwischen 14. und 16. 12. 1909?{]}}|)be}\mylabel{L03807h}  \newcommand{\dateiname}{L03807}\newcommand{\titel}{Arthur Schnitzler an Stefan Zweig, [zwischen 14. und 16. 12. 1909?]}\newcommand{\editorInnen}{Selma Jahnke und Martin Anton Müller}%% latex-leseansicht-abspann.tex
%% Abspann für die Leseansicht.
%% Der Schalter \ifkorrekturansicht ist bereits durch den Vorspann gesetzt.

%% latex-abspann.tex
%% Gemeinsamer Abspann für Korrekturansicht und Leseansicht.
%% Setzt den Schalter \ifkorrekturansicht voraus (gesetzt in den
%% einbindenden Dateien latex-korrekturansicht-abspann.tex bzw.
%% latex-leseansicht-abspann.tex).
%% ---------------------------------------------------------------

\normalsize

% Das esempio-Environment wird nur in der Leseansicht benötigt
\ifkorrekturansicht\else
\newenvironment{esempio}[3]%
{
    \vspace{1.5ex}
    \rlap{\underline{#1}}
    \par
    \setlength{\parindent}{0cm}
    \nopagebreak
    \leftskip=#2cm
    \rightskip=#3cm
}
{
    \par
}
\fi

\doendnotes{C}
\bigskip
\vfill

\clearpage

\footnotesize

\ifkorrekturansicht
  \lohead{\textsc{register}}
\fi

% theindex-Environment neu definieren ohne reledmac
\makeatletter
\renewenvironment{theindex}{%
  \ifkorrekturansicht
    \section*{\indexname}%
  \else
    \subsubsection*{Index der erwähnten Entitäten}%
  \fi
  \setlength{\parindent}{0pt}%
  \setlength{\parskip}{0pt plus 0.3pt}%
  \let\item\@idxitem
}{%
  \ifkorrekturansicht\clearpage\fi
}
\makeatother

\IfFileExists{\jobname-pw.ind}{\input{\jobname-pw.ind}}{}

% Quellenangabe nur in der Leseansicht
\ifkorrekturansicht\else
% Fallback-Definitionen, falls die .tex-Datei \titel etc. nicht gesetzt hat
\providecommand{\titel}{}
\providecommand{\editorInnen}{}
\providecommand{\dateiname}{\jobname}

\vspace{3cm}

\vfill

\footnotesize
\textsc{Quelle}: \titel. Herausgegeben von {\editorInnen}. In: \emph{Arthur Schnitzler: Briefwechsel mit Autorinnen und Autoren}.
 Digitale Edition, https://schnitzler-briefe.acdh.oeaw.ac.at/{\dateiname}.html (Stand \today)
\fi

\end{document}


