%% latex-korrekturansicht-vorspann.tex
%% Vorspann für die Korrekturansicht.
%% Lädt die gemeinsame Datei latex-vorspann.tex mit gesetztem Schalter.

\newif\ifkorrekturansicht
\korrekturansichttrue

\input{../tex-inputs/latex-vorspann}


\section[Arthur Schnitzler an Stefan Zweig, {[}zwischen 14. und 16. 12. 1909?{]}]{L03807 Arthur Schnitzler an Stefan Zweig, {[}zwischen 14. und 16. 12. 1909?{]}}
\nopagebreak\mylabel{L03807v}
\rehead{ }\normalsize\beginnumbering\briefempfaengerindex{Zweig, Stefan@\textsc{Zweig, Stefan}!zzzSchnitzler, Arthur@\emph{von Arthur Schnitzler}!1909-12-163@{{[}zwischen 14. und 16. 12. 1909?{]}}|(be}
\toendnotes[C]{\smallbreak\pagebreak[2]}\Standort{Jerusalem, National Library of Israel, ARC. Ms. Var. 305 1 58 Stefan Zweig Collection.}
\physDesc{Visitenkarte, 1 Blatt, 2 Seiten, 92 Zeichen
\newline{}Handschrift: Bleistift, deutsche Kurrent
\newline{}Ordnung: mit Bleistift von unbekannter Hand datiert: »1908(?)« }\toendnotes[C]{\smallbreak}
\pstart
           \noindent{}\centering{}{\pb}\textcolor{gray}{\textbf{D\textsuperscript{}\textsuperscript{r} Arthur Schnitzler}}\pend
           
\pstart
           {\pb}hab mich ſehr über Ihren \label{K_L03807-1v}\edtext{ſchönen Brief}{\lemma{\textnormal{\emph{ſchönen Brief}}}\Cendnote{\textnormal{Die
                  Visitenkarte ist mit Bleistift geschrieben, was bedeuten dürfte, sie wurde abgegeben, als Schnitzler bei
                  einem Besuch bei Zweig\pwindex{Zweig, Stefan 28.11.1881 – 23.02.1942@\textsc{Zweig, Stefan} (28.11.1881 – 23.02.1942), \emph{Schriftsteller/Schriftstellerin}|pwk} diesen nicht antraf. Wertet man unter dieser Prämisse die
                  frühen Briefe Zweigs\pwindex{Zweig, Stefan 28.11.1881 – 23.02.1942@\textsc{Zweig, Stefan} (28.11.1881 – 23.02.1942), \emph{Schriftsteller/Schriftstellerin}|pwk} nach dem »ſchönen Brief« aus und schließt jene
                  aus, für die eine Antwort Schnitzlers überliefert ist, bleibt der Brief Zweigs\pwindex{Zweig, Stefan 28.11.1881 – 23.02.1942@\textsc{Zweig, Stefan} (28.11.1881 – 23.02.1942), \emph{Schriftsteller/Schriftstellerin}|pwk}
                  vom 13. 12. 1909 als wahrscheinlicher Referenzpunkt. Schnitzler
                  war zwischen 14. 12. 1909 und 17. 12. 1909
                  mehrfach in den inneren Bezirken Wiens\oindex{Wien@\textbf{Wien}, \emph{A.ADM2}|pwk} unterwegs und machte Besuche. Am 17. 12. 1909
                  kam es zum Besuch bei Zweig\pwindex{Zweig, Stefan 28.11.1881 – 23.02.1942@\textsc{Zweig, Stefan} (28.11.1881 – 23.02.1942), \emph{Schriftsteller/Schriftstellerin}|pwk}, so dass die vorliegende Visitenkarte in den drei vorangehenden 
                  Tagen abgegeben worden sein dürfte.}}}\label{K_L03807-1} gefreut wollte Ihnen d\textcolor{gray}{ie} Hand
               drücken.\pend
           
\pstart
           Herzlichſt{\\[\baselineskip]}Ihr \spacefill\mbox{A. S.}\pend
           \leftskip=0em{}\selectlanguage{ngerman}\endnumbering\briefempfaengerindex{Zweig, Stefan@\textsc{Zweig, Stefan}!zzzSchnitzler, Arthur@\emph{von Arthur Schnitzler}!1909-12-143@{{[}zwischen 14. und 16. 12. 1909?{]}}|)be}\mylabel{L03807h}  \normalsize

\doendnotes{C}
\bigskip
\vfill

\clearpage

\footnotesize

\lohead{\textsc{register}}

% Definiere theindex-Environment komplett neu ohne reledmac
\makeatletter
\renewenvironment{theindex}{%
  \section*{\indexname}%
  \setlength{\parindent}{0pt}%
  \setlength{\parskip}{0pt plus 0.3pt}%
  \let\item\@idxitem
}{%
  \clearpage
}
\makeatother

\IfFileExists{\jobname-pw.ind}{\input{\jobname-pw.ind}}{}

\end{document}

      