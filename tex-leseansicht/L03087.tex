%% latex-korrekturansicht-vorspann.tex
%% Vorspann für die Korrekturansicht.
%% Lädt die gemeinsame Datei latex-vorspann.tex mit gesetztem Schalter.

\newif\ifkorrekturansicht
\korrekturansichttrue

\input{../tex-inputs/latex-vorspann}


\section[ Paul Goldmann an Arthur Schnitzler, 28. 9. {[}1901{]}]{L03087 Paul Goldmann an Arthur Schnitzler, 28. 9. {[}1901{]}}
\nopagebreak\mylabel{L03087v}
\rehead{ }\normalsize\beginnumbering\briefempfaengerindex{Schnitzler, Arthur@\textsc{Schnitzler, Arthur}!zzzGoldmann, Paul@\emph{von Paul Goldmann}!1901-09-281@{28. 9. {[}1901{]}}|(be}
\toendnotes[C]{\smallbreak\pagebreak[2]}\Standort{DLA, A:Schnitzler, HS.NZ85.1.3171.}
\physDesc{Brief, 1 Blatt, 2 Seiten, 841 Zeichen
\newline{}Handschrift: blaue Tinte, deutsche Kurrent
\newline{}Schnitzler: 1) mit Bleistift das Jahr »901« vermerkt  2) mit rotem Buntstift drei Unterstreichungen}\toendnotes[C]{\smallbreak}
\pstart
           \raggedleft{}{\pb}\textcolor{gray}{\textbf{DESSAUERSTRASSE 19}}\oindex{Dessauer Strasse@\textbf{Dessauer Straße}, \emph{Straße (K.STR)}|pw}\pend
           
\pstart
           Berlin\oindex{Berlin@\textbf{Berlin}, \emph{P.PPLC}|pw}, 28. September.\pend
           
\pstart\center{}Mein lieber Freund,\pend\vspace{0.5em}
\pstart
           Dank für Deinen lieben Brief!\pend
           
\pstart
           Zu \label{K_L03087-1v}\edtext{\textsc{Glümers\pwindex{Gluemer, Marie 03.07.1867 – 16.11.1925@\textsc{Glümer, Marie} (03.07.1867 – 16.11.1925), \emph{Schauspieler/Schauspielerin}|pw}\pwindex{Gluemer, Auguste 1862-03-16 – 1956@\textsc{Glümer, Auguste} (1862-03-16 – 1956), \emph{Lehrer/Lehrerin}|pw}}}{\lemma{\textnormal{\emph{Glümers}}}\Cendnote{\textnormal{Bezug auf die Verlobung Marie Glümers\pwindex{Gluemer, Marie 03.07.1867 – 16.11.1925@\textsc{Glümer, Marie} (03.07.1867 – 16.11.1925), \emph{Schauspieler/Schauspielerin}|pwk} mit Paul Martin Blaustein\pwindex{Marton, Paul Martin @\textsc{Marton, Paul Martin}, \emph{Schriftsteller/Schriftstellerin, Theaterleiter/Theaterleiterin}|pwk}, siehe A. S.: \emph{Tagebuch}, 27. 9. 1901.}}}\label{K_L03087-1} werde ich nicht hingehen. Ich betrachte überhaupt meine Beziehungen mit ihnen
               als abgeſchloſſen. Es iſt empörend, daß ich, nachdem wir zwei Jahre aufs
               Freundſchaftlichſte verkehrt haben, die Verlobungs-Nachricht\pwindex{neueste Verlobung in Berliner Theaterkreisen@\emph{Die neueste Verlobung in Berliner Theaterkreisen}|pwv} aus dem \label{K_L03087-2v}\edtext{»Lokalanzeiger\pwindex{Berliner Lokal-Anzeiger@\emph{Berliner Lokal-Anzeiger}|pw}«}{\lemma{\textnormal{\emph{»Lokalanzeiger«}}}\Cendnote{\textnormal{[O. V.]: \emph{[Die neueste Verlobung in Berliner
                        Theaterkreisen]}\pwindex{neueste Verlobung in Berliner Theaterkreisen@\emph{Die neueste Verlobung in Berliner Theaterkreisen}|pwk}. In: \emph{Berliner
                        Lokal-Anzeiger}\pwindex{Berliner Lokal-Anzeiger@\emph{Berliner Lokal-Anzeiger}|pwk}, Jg. 19, Nr. 453, 27. 9. 1901, Morgenblatt, 1. Ausgabe, S. 2.}}}\label{K_L03087-2} erfahren
               muß!\pend
           
\pstart
           Im Übrigen iſt es wirklich das Beſte. Der Herr Direktor\pwindex{Marton, Paul Martin @\textsc{Marton, Paul Martin}, \emph{Schriftsteller/Schriftstellerin, Theaterleiter/Theaterleiterin}|pwv} mag ein {\pb}Schwindler ſein, – für ſeine Frau\pwindex{Gluemer, Marie 03.07.1867 – 16.11.1925@\textsc{Glümer, Marie} (03.07.1867 – 16.11.1925), \emph{Schauspieler/Schauspielerin}|pwv} wird er ſchon ſorgen. Vielleicht ſchwindelt er ſich auch hinauf.
               Jedenfalls kommt das arme Mädel\pwindex{Gluemer, Marie 03.07.1867 – 16.11.1925@\textsc{Glümer, Marie} (03.07.1867 – 16.11.1925), \emph{Schauspieler/Schauspielerin}|pwv} aus den ſchlimmſten Exiſtenzſorgen heraus.\pend
           
\pstart
           Ich ſehe ſie noch in Salzburg\oindex{Salzburg@\textbf{Salzburg}, \emph{A.ADM2}|pw}, wie ich ſie mit Dir
               zuſammen \label{K_L03087-3v}\edtext{beſuchte}{\lemma{\textnormal{\emph{beſuchte}}}\Cendnote{\textnormal{Das bezieht sich auf die Zeit, als Schnitzler mit Marie Glümer\pwindex{Gluemer, Marie 03.07.1867 – 16.11.1925@\textsc{Glümer, Marie} (03.07.1867 – 16.11.1925), \emph{Schauspieler/Schauspielerin}|pwk} liiert war, siehe A. S.: \emph{Tagebuch}, 29. 9. 1890.}}}\label{K_L03087-3}. Wer hätte damals das Alles
               geahnt?\pend
           
\pstart
           Ich ſende Dir heut einen \label{K_L03087-4v}\edtext{Artikel\pwindex{?? [Artikel ueber eine russische Judenverfolgung]@\emph{?? [Artikel über eine russische Judenverfolgung]}|pwv}}{\lemma{\textnormal{\emph{Artikel}}}\Cendnote{\textnormal{Beilage nicht erhalten}}}\label{K_L03087-4} von \textsc{Gorki\pwindex{Gorkij, Maxim 1868-03-28 – 1936-06-18@\textsc{Gorkij, Maxim} (1868-03-28 – 1936-06-18), \emph{Schriftsteller/Schriftstellerin}|pw}}, der mich tief ergriffen hat, – die Schilderung einer ruſſ\oindex{Russland@\textbf{Russland}, \emph{A.PCLI}|pwv}iſchen Judenverfolgung.\pend
           
\pstart
           Viele treue Grüße! {\\[\baselineskip]}Dein \spacefill\mbox{Paul Goldmann}\pend
           \leftskip=0em{}
\pstart
           \noindent{}Alles Liebe den beiden Schweſtern\pwindex{Schnitzler, Olga 17.01.1882 – 13.01.1970@\textsc{Schnitzler, Olga} (17.01.1882 – 13.01.1970), \emph{Schauspieler/Schauspielerin, Sänger/Sängerin}|pwv}\pwindex{Steinrueck, Elisabeth 19.11.1885 – 07.04.1920@\textsc{Steinrück, Elisabeth} (19.11.1885 – 07.04.1920)|pwv}!\pend
           \selectlanguage{ngerman}\endnumbering\briefempfaengerindex{Schnitzler, Arthur@\textsc{Schnitzler, Arthur}!zzzGoldmann, Paul@\emph{von Paul Goldmann}!1901-09-281@{28. 9. {[}1901{]}}|)be}\mylabel{L03087h}  \normalsize

\doendnotes{C}
\bigskip
\vfill

\clearpage

\footnotesize

\lohead{\textsc{register}}

% Definiere theindex-Environment komplett neu ohne reledmac
\makeatletter
\renewenvironment{theindex}{%
  \section*{\indexname}%
  \setlength{\parindent}{0pt}%
  \setlength{\parskip}{0pt plus 0.3pt}%
  \let\item\@idxitem
}{%
  \clearpage
}
\makeatother

\IfFileExists{\jobname-pw.ind}{\input{\jobname-pw.ind}}{}

\end{document}

      