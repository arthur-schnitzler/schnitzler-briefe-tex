%% latex-leseansicht-vorspann.tex
%% Vorspann für die Leseansicht.
%% Lädt die gemeinsame Datei latex-vorspann.tex mit nicht gesetztem Schalter.

\newif\ifkorrekturansicht
\korrekturansichtfalse

\input{../tex-inputs/latex-vorspann}


         
         \renewcommand{\erwaehntePersonen}{Personen: Marie Glümer, Auguste Glümer, Paul Goldmann, Maxim Gorkij, Paul Martin Marton, Olga Schnitzler, Elisabeth Steinrück}
         \renewcommand{\erwaehnteOrte}{Orte: Berlin, Dessauer Straße, Russland, Salzburg, Wien}
         \renewcommand{\erwaehnteWerke}{Werke: ?? [Artikel über eine russische Judenverfolgung], Berliner Lokal-Anzeiger, Die neueste Verlobung in Berliner Theaterkreisen}
               \section[ Paul Goldmann an Arthur Schnitzler, 28. 9. {[}1901{]}]{ Paul Goldmann an Arthur Schnitzler, 28. 9. {[}1901{]}}\nopagebreak\mylabel{v}\rehead{ }\begin{ledgroupsized}[t]{13cm}\normalsize\beginnumbering\briefempfaengerindex{Schnitzler, Arthur@\textsc{Schnitzler, Arthur}!zzzGoldmann, Paul@\emph{von Paul Goldmann}!1901-09-281@{28. 9. {[}1901{]}}|(be} \toendnotes[C]{\smallbreak\pagebreak[2]} \Standort{DLA, A:Schnitzler, HS.NZ85.1.3171.}
\physDesc{Brief, 1 Blatt, 2 Seiten, 841 Zeichen
\newline{}Handschrift: blaue Tinte, deutsche Kurrent
\newline{}Schnitzler: 1) mit Bleistift das Jahr »901« vermerkt  2) mit rotem Buntstift drei Unterstreichungen}\toendnotes[C]{\smallbreak}\pstart
           \noindent{}\raggedleft{}{\pb}\textcolor{gray}{\textbf{DESSAUERSTRASSE 19}}\oindex{Dessauer Strasse@\textbf{Dessauer Straße}|pw}\pend
           \pstart
           Berlin\oindex{Berlin@\textbf{Berlin}|pw}, 28. September.\pend
           \pstart\center{}Mein lieber Freund,\pend\pstart
           Dank für Deinen lieben Brief!\pend
           \pstart
           Zu \label{K_L03087-1v}\edtext{\textsc{Glümers\pwindex{Gluemer, Marie 03.07.1867 – 16.11.1925@\textsc{Glümer, Marie} (03.07.1867 – 16.11.1925), \emph{Schauspielerin}|pw}\pwindex{Gluemer, Auguste 16.03.1862 – 1956@\textsc{Glümer, Auguste} (16.03.1862 – 1956)|pw}}}{\lemma{\textnormal{\emph{Glümers}}}\Cendnote{\textnormal{Bezug auf die Verlobung Marie Glümer\pwindex{Gluemer, Marie 03.07.1867 – 16.11.1925@\textsc{Glümer, Marie} (03.07.1867 – 16.11.1925), \emph{Schauspielerin}|pwk}s mit Paul Martin Blaustein\pwindex{Marton, Paul Martin @\textsc{Marton, Paul Martin}, \emph{Schriftsteller, Theaterleiter}|pwk}, siehe A. S.: \emph{Tagebuch}, 27. 9. 1901}}}\label{K_L03087-1h} werde ich nicht hingehen. Ich betrachte überhaupt meine Beziehungen mit ihnen
               als abgeſchloſſen. Es iſt empörend, daß ich, nachdem wir zwei Jahre aufs
               Freundſchaftlichſte verkehrt haben, die Verlobungs-Nachricht\pwindex{?? Werk@Nicht ermittelte Verfasserinnen und Verfasser!neueste Verlobung in Berliner Theaterkreisen1901-09-27@\emph{Die neueste Verlobung in Berliner Theaterkreisen} {[}1901-09-27{]}|pwv} aus dem \label{K_L03087-2v}\edtext{»Lokalanzeiger\pwindex{?? Werk@Nicht ermittelte Verfasserinnen und Verfasser!Berliner Lokal-Anzeiger1883@\emph{Berliner Lokal-Anzeiger} {[}1883{]}|pw}«}{\lemma{\textnormal{\emph{»Lokalanzeiger«}}}\Cendnote{\textnormal{[O. V.]: \emph{[Die neueste Verlobung in Berliner
                        Theaterkreisen]}\pwindex{?? Werk@Nicht ermittelte Verfasserinnen und Verfasser!neueste Verlobung in Berliner Theaterkreisen1901-09-27@\emph{Die neueste Verlobung in Berliner Theaterkreisen} {[}1901-09-27{]}|pwk}. In: \emph{Berliner
                        Lokal-Anzeiger}\pwindex{?? Werk@Nicht ermittelte Verfasserinnen und Verfasser!Berliner Lokal-Anzeiger1883@\emph{Berliner Lokal-Anzeiger} {[}1883{]}|pwk}, Jg. 19, Nr. 453, 27. 9. 1901, Morgenblatt, 1. Ausgabe, S. 2.}}}\label{K_L03087-2h} erfahren
               muß!\pend
           \pstart
           Im Übrigen iſt es wirklich das Beſte. Der Herr Direktor\pwindex{Marton, Paul Martin @\textsc{Marton, Paul Martin}, \emph{Schriftsteller, Theaterleiter}|pwv} mag ein {\pb}Schwindler ſein, – für ſeine Frau\pwindex{Gluemer, Marie 03.07.1867 – 16.11.1925@\textsc{Glümer, Marie} (03.07.1867 – 16.11.1925), \emph{Schauspielerin}|pwv} wird er ſchon ſorgen. Vielleicht ſchwindelt er ſich auch hinauf.
               Jedenfalls kommt das arme Mädel\pwindex{Gluemer, Marie 03.07.1867 – 16.11.1925@\textsc{Glümer, Marie} (03.07.1867 – 16.11.1925), \emph{Schauspielerin}|pwv} aus den ſchlimmſten Exiſtenzſorgen heraus.\pend
           \pstart
           Ich ſehe ſie noch in Salzburg\oindex{Salzburg@\textbf{Salzburg}|pw}, wie ich ſie mit Dir
               zuſammen \label{K_L03087-3v}\edtext{beſuchte}{\lemma{\textnormal{\emph{beſuchte}}}\Cendnote{\textnormal{Das bezieht sich auf die Zeit, als Schnitzler\pwindex{Schnitzler, Arthur 15.05.1862 – 21.10.1931@\textsc{Schnitzler, Arthur} (15.05.1862 – 21.10.1931), \emph{Schriftsteller, Mediziner}|pwk} mit Marie Glümer\pwindex{Gluemer, Marie 03.07.1867 – 16.11.1925@\textsc{Glümer, Marie} (03.07.1867 – 16.11.1925), \emph{Schauspielerin}|pwk} liiert war, siehe A. S.: \emph{Tagebuch}, 29. 9. 1890.}}}\label{K_L03087-3h}. Wer hätte damals das Alles
               geahnt?\pend
           \pstart
           Ich ſende Dir heut einen \label{K_L03087-4v}\edtext{Artikel\pwindex{Gorkij, Maxim 1868-03-28 – 1936-06-18@\textsc{Gorkij, Maxim} (1868-03-28 – 1936-06-18), \emph{Schriftsteller}!?? [Artikel ueber eine russische Judenverfolgung]1901-09@\strich\emph{?? [Artikel über eine russische Judenverfolgung]} {[}1901-09{]}|pwv}}{\lemma{\textnormal{\emph{Artikel}}}\Cendnote{\textnormal{Beilage nicht erhalten}}}\label{K_L03087-4h} von \textsc{Gorki\pwindex{Gorkij, Maxim 1868-03-28 – 1936-06-18@\textsc{Gorkij, Maxim} (1868-03-28 – 1936-06-18), \emph{Schriftsteller}|pw}}, der mich tief ergriffen hat, – die Schilderung einer ruſſ\oindex{Russland@\textbf{Russland}|pwv}iſchen Judenverfolgung.\pend
           \pstart
           Viele treue Grüße! {\\[\baselineskip]}Dein \spacefill\mbox{Paul Goldmann}\pend
           \leftskip=0em{}\pstart
           \noindent{}Alles Liebe den beiden Schweſtern\pwindex{Schnitzler, Olga 17.01.1882 – 13.01.1970@\textsc{Schnitzler, Olga} (17.01.1882 – 13.01.1970), \emph{Schauspielerin, Sängerin}|pwv}\pwindex{Steinrueck, Elisabeth 19.11.1885 – 07.04.1920@\textsc{Steinrück, Elisabeth} (19.11.1885 – 07.04.1920)|pwv}!\pend
           
         
         \endnumbering\mylabel{h}\end{ledgroupsized}  \newcommand{\dateiname}{L03087}\newcommand{\titel}{Paul Goldmann an Arthur Schnitzler, 28. 9. [1901]}\newcommand{\editorInnen}{Martin Anton Müller und Laura Untner}%% latex-leseansicht-abspann.tex
%% Abspann für die Leseansicht.
%% Der Schalter \ifkorrekturansicht ist bereits durch den Vorspann gesetzt.

%% latex-abspann.tex
%% Gemeinsamer Abspann für Korrekturansicht und Leseansicht.
%% Setzt den Schalter \ifkorrekturansicht voraus (gesetzt in den
%% einbindenden Dateien latex-korrekturansicht-abspann.tex bzw.
%% latex-leseansicht-abspann.tex).
%% ---------------------------------------------------------------

\normalsize

% Das esempio-Environment wird nur in der Leseansicht benötigt
\ifkorrekturansicht\else
\newenvironment{esempio}[3]%
{
    \vspace{1.5ex}
    \rlap{\underline{#1}}
    \par
    \setlength{\parindent}{0cm}
    \nopagebreak
    \leftskip=#2cm
    \rightskip=#3cm
}
{
    \par
}
\fi

\doendnotes{C}
\bigskip
\vfill

\clearpage

\footnotesize

\ifkorrekturansicht
  \lohead{\textsc{register}}
\fi

% theindex-Environment neu definieren ohne reledmac
\makeatletter
\renewenvironment{theindex}{%
  \ifkorrekturansicht
    \section*{\indexname}%
  \else
    \subsubsection*{Index der erwähnten Entitäten}%
  \fi
  \setlength{\parindent}{0pt}%
  \setlength{\parskip}{0pt plus 0.3pt}%
  \let\item\@idxitem
}{%
  \ifkorrekturansicht\clearpage\fi
}
\makeatother

\IfFileExists{\jobname-pw.ind}{\input{\jobname-pw.ind}}{}

% Quellenangabe nur in der Leseansicht
\ifkorrekturansicht\else
% Fallback-Definitionen, falls die .tex-Datei \titel etc. nicht gesetzt hat
\providecommand{\titel}{}
\providecommand{\editorInnen}{}
\providecommand{\dateiname}{\jobname}

\vspace{3cm}

\vfill

\footnotesize
\textsc{Quelle}: \titel. Herausgegeben von {\editorInnen}. In: \emph{Arthur Schnitzler: Briefwechsel mit Autorinnen und Autoren}.
 Digitale Edition, https://schnitzler-briefe.acdh.oeaw.ac.at/{\dateiname}.html (Stand \today)
\fi

\end{document}


      