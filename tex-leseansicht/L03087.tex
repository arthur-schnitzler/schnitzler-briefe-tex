%% latex-leseansicht-vorspann.tex
%% Vorspann für die Leseansicht.
%% Lädt die gemeinsame Datei latex-vorspann.tex mit nicht gesetztem Schalter.

\newif\ifkorrekturansicht
\korrekturansichtfalse

\input{../tex-inputs/latex-vorspann}


\section[ Paul Goldmann an Arthur Schnitzler, 28. 9. {[}1901{]}]{L03087 Paul Goldmann an Arthur Schnitzler,  28. 9. [1901]}
\nopagebreak\mylabel{L03087v}
\rehead{ }\normalsize\beginnumbering\briefempfaengerindex{Schnitzler, Arthur@\textsc{Schnitzler, Arthur}!zzzGoldmann, Paul@\emph{von Paul Goldmann}!1901-09-281@{28. 9. [1901]}|(be}
\toendnotes[C]{\smallbreak\pagebreak[2]}
\correspDesc{Versand  durch Paul Goldmann am 28. 9. [1901] in Berlin
\newline{}Erhalt  durch Arthur Schnitzler im Zeitraum [29. 9. 1901
                  – 3. 10. 1901?] in Wien}\toendnotes[C]{\smallbreak}
\Standort{DLA, A:Schnitzler, HS.NZ85.1.3171.}
\physDesc{Brief, 1 Blatt, 2 Seiten, 841 Zeichen
\newline{}Handschrift: blaue Tinte, deutsche Kurrent
\newline{}Schnitzler: 1) mit Bleistift das Jahr »901« vermerkt  2) mit rotem Buntstift drei Unterstreichungen}\toendnotes[C]{\smallbreak}
\pstart
           \raggedleft{}{\pb}\textcolor{gray}{\textbf{DESSAUERSTRASSE 19}}\oindex{Dessauer Straße@\textbf{Dessauer Straße}, \emph{Straße}|pw}\pend
           
\pstart
           Berlin\oindex{Berlin@\textbf{Berlin}, \emph{Hauptstadt}|pw}, 28. September.\pend
           
\pstart\center{}Mein lieber Freund,\pend\vspace{0.5em}
\pstart
           Dank für Deinen lieben Brief!\pend
           
\pstart
           Zu \label{K_L03087-1v}\edtext{\textsc{Glümers\pwindex{Glümer, Marie 3.\,7.\,1867 Wien – 16.\,11.\,1925 München@\textsc{Glümer, Marie} (3.\,7.\,1867 Wien – 16.\,11.\,1925 München), \emph{Schauspielerin}|pw}\pwindex{Glümer, Auguste 16.\,3.\,1862 Wien – 1956@\textsc{Glümer, Auguste} (16.\,3.\,1862 Wien – 1956), \emph{Lehrerin}|pw}}}{\lemma{\textnormal{\emph{Glümers}}}\Cendnote{\textnormal{Bezug auf die Verlobung Marie Glümers\pwindex{Glümer, Marie 3.\,7.\,1867 Wien – 16.\,11.\,1925 München@\textsc{Glümer, Marie} (3.\,7.\,1867 Wien – 16.\,11.\,1925 München), \emph{Schauspielerin}|pwk} mit Paul Martin Blaustein\pwindex{Marton, Paul Martin @\textsc{Marton, Paul Martin}, \emph{Schriftsteller, Theaterleiter}|pwk}, siehe A. S.: \emph{Tagebuch}, 27. 9. 1901.}}}\label{K_L03087-1} werde ich nicht hingehen. Ich betrachte überhaupt meine Beziehungen mit ihnen
               als abgeſchloſſen. Es iſt empörend, daß ich, nachdem wir zwei Jahre aufs
               Freundſchaftlichſte verkehrt haben, die Verlobungs-Nachricht\pwindex{neueste Verlobung in Berliner Theaterkreisen@\emph{Die neueste Verlobung in Berliner Theaterkreisen}|pwv} aus dem \label{K_L03087-2v}\edtext{»Lokalanzeiger\pwindex{Berliner Lokal-Anzeiger@\emph{Berliner Lokal-Anzeiger}|pw}«}{\lemma{\textnormal{\emph{»Lokalanzeiger«}}}\Cendnote{\textnormal{[O. V.]: \emph{[Die neueste Verlobung in Berliner
                        Theaterkreisen]}\pwindex{neueste Verlobung in Berliner Theaterkreisen@\emph{Die neueste Verlobung in Berliner Theaterkreisen}|pwk}. In: \emph{Berliner
                        Lokal-Anzeiger}\pwindex{Berliner Lokal-Anzeiger@\emph{Berliner Lokal-Anzeiger}|pwk}, Jg. 19, Nr. 453, 27. 9. 1901, Morgenblatt, 1. Ausgabe, S. 2.}}}\label{K_L03087-2} erfahren
               muß!\pend
           
\pstart
           Im Übrigen iſt es wirklich das Beſte. Der Herr Direktor\pwindex{Marton, Paul Martin @\textsc{Marton, Paul Martin}, \emph{Schriftsteller, Theaterleiter}|pwv} mag ein {\pb}Schwindler{ }ſein, – für{ }ſeine Frau\pwindex{Glümer, Marie 3.\,7.\,1867 Wien – 16.\,11.\,1925 München@\textsc{Glümer, Marie} (3.\,7.\,1867 Wien – 16.\,11.\,1925 München), \emph{Schauspielerin}|pwv} wird er{ }ſchon{ }ſorgen. Vielleicht{ }ſchwindelt er{ }ſich auch hinauf.
               Jedenfalls kommt das arme Mädel\pwindex{Glümer, Marie 3.\,7.\,1867 Wien – 16.\,11.\,1925 München@\textsc{Glümer, Marie} (3.\,7.\,1867 Wien – 16.\,11.\,1925 München), \emph{Schauspielerin}|pwv} aus den{ }ſchlimmſten Exiſtenzſorgen heraus.\pend
           
\pstart
           Ich{ }ſehe{ }ſie noch in Salzburg\oindex{Salzburg@\textbf{Salzburg}, \emph{Verwaltungsgebiet}|pw}, wie ich{ }ſie mit Dir
               zuſammen \label{K_L03087-3v}\edtext{beſuchte}{\lemma{\textnormal{\emph{besuchte}}}\Cendnote{\textnormal{Das bezieht sich auf die Zeit, als Schnitzler mit Marie Glümer\pwindex{Glümer, Marie 3.\,7.\,1867 Wien – 16.\,11.\,1925 München@\textsc{Glümer, Marie} (3.\,7.\,1867 Wien – 16.\,11.\,1925 München), \emph{Schauspielerin}|pwk} liiert war, siehe A. S.: \emph{Tagebuch}, 29. 9. 1890.}}}\label{K_L03087-3}. Wer hätte damals das Alles
               geahnt?\pend
           
\pstart
           Ich{ }ſende Dir heut einen \label{K_L03087-4v}\edtext{Artikel\pwindex{Gorkij, Maxim 28.\,3.\,1868 Nischni Nowgorod – 18.\,6.\,1936 Moskau@\textsc{Gorkij, Maxim} (28.\,3.\,1868 Nischni Nowgorod – 18.\,6.\,1936 Moskau), \emph{Schriftsteller}!?? [Artikel über eine russische Judenverfolgung]@\strich\emph{?? [Artikel über eine russische Judenverfolgung]}|pwv}}{\lemma{\textnormal{\emph{Artikel}}}\Cendnote{\textnormal{Beilage nicht erhalten}}}\label{K_L03087-4} von \textsc{Gorki\pwindex{Gorkij, Maxim 28.\,3.\,1868 Nischni Nowgorod – 18.\,6.\,1936 Moskau@\textsc{Gorkij, Maxim} (28.\,3.\,1868 Nischni Nowgorod – 18.\,6.\,1936 Moskau), \emph{Schriftsteller}|pw}}, der mich tief ergriffen hat, – die Schilderung einer ruſſ\oindex{Russland@\textbf{Russland}|pwv}iſchen Judenverfolgung.\pend
           
\pstart
           Viele treue Grüße! {\\[\baselineskip]}Dein \spacefill\mbox{Paul Goldmann}\pend
           \leftskip=0em{}
\pstart
           \noindent{}Alles Liebe den beiden Schweſtern\pwindex{Schnitzler, Olga 17.\,1.\,1882 Wien – 13.\,1.\,1970 Lugano@\textsc{Schnitzler, Olga} (17.\,1.\,1882 Wien – 13.\,1.\,1970 Lugano), \emph{Schauspielerin, Sängerin}|pwv}\pwindex{Steinrück, Elisabeth 19.\,11.\,1885 – 7.\,4.\,1920 Partenkirchen@\textsc{Steinrück, Elisabeth} (19.\,11.\,1885 – 7.\,4.\,1920 Partenkirchen)|pwv}!\pend
           \selectlanguage{ngerman}\endnumbering\briefempfaengerindex{Schnitzler, Arthur@\textsc{Schnitzler, Arthur}!zzzGoldmann, Paul@\emph{von Paul Goldmann}!1901-09-281@{28. 9. [1901]}|)be}\mylabel{L03087h}  \newcommand{\dateiname}{L03087}\newcommand{\titel}{Paul Goldmann an Arthur Schnitzler, 28. 9. [1901]}\newcommand{\editorInnen}{Martin Anton Müller und Laura Untner}%% latex-leseansicht-abspann.tex
%% Abspann für die Leseansicht.
%% Der Schalter \ifkorrekturansicht ist bereits durch den Vorspann gesetzt.

%% latex-abspann.tex
%% Gemeinsamer Abspann für Korrekturansicht und Leseansicht.
%% Setzt den Schalter \ifkorrekturansicht voraus (gesetzt in den
%% einbindenden Dateien latex-korrekturansicht-abspann.tex bzw.
%% latex-leseansicht-abspann.tex).
%% ---------------------------------------------------------------

\normalsize

% Das esempio-Environment wird nur in der Leseansicht benötigt
\ifkorrekturansicht\else
\newenvironment{esempio}[3]%
{
    \vspace{1.5ex}
    \rlap{\underline{#1}}
    \par
    \setlength{\parindent}{0cm}
    \nopagebreak
    \leftskip=#2cm
    \rightskip=#3cm
}
{
    \par
}
\fi

\doendnotes{C}
\bigskip
\vfill

\clearpage

\footnotesize

\ifkorrekturansicht
  \lohead{\textsc{register}}
\fi

% theindex-Environment neu definieren ohne reledmac
\makeatletter
\renewenvironment{theindex}{%
  \ifkorrekturansicht
    \section*{\indexname}%
  \else
    \subsubsection*{Index der erwähnten Entitäten}%
  \fi
  \setlength{\parindent}{0pt}%
  \setlength{\parskip}{0pt plus 0.3pt}%
  \let\item\@idxitem
}{%
  \ifkorrekturansicht\clearpage\fi
}
\makeatother

\IfFileExists{\jobname-pw.ind}{\input{\jobname-pw.ind}}{}

% Quellenangabe nur in der Leseansicht
\ifkorrekturansicht\else
% Fallback-Definitionen, falls die .tex-Datei \titel etc. nicht gesetzt hat
\providecommand{\titel}{}
\providecommand{\editorInnen}{}
\providecommand{\dateiname}{\jobname}

\vspace{3cm}

\vfill

\footnotesize
\textsc{Quelle}: \titel. Herausgegeben von {\editorInnen}. In: \emph{Arthur Schnitzler: Briefwechsel mit Autorinnen und Autoren}.
 Digitale Edition, https://schnitzler-briefe.acdh.oeaw.ac.at/{\dateiname}.html (Stand \today)
\fi

\end{document}


