%% latex-korrekturansicht-vorspann.tex
%% Vorspann für die Korrekturansicht.
%% Lädt die gemeinsame Datei latex-vorspann.tex mit gesetztem Schalter.

\newif\ifkorrekturansicht
\korrekturansichttrue

\input{../tex-inputs/latex-vorspann}


\section[Arthur und Olga Schnitzler an Richard Beer-Hofmann, 30. 7. 1912]{L02080 Arthur und Olga Schnitzler an Richard Beer-Hofmann, 30. 7. 1912}
\nopagebreak\mylabel{L02080v}
\rehead{ }\normalsize\beginnumbering\briefempfaengerindex{Beer-Hofmann, Richard@\textsc{Beer-Hofmann, Richard}!zzzSchnitzler, Olga@\emph{von Olga Schnitzler}!1912-07-301@{30. 7. 1912}|(be}\briefempfaengerindex{Beer-Hofmann, Richard@\textsc{Beer-Hofmann, Richard}!zzzSchnitzler, Arthur@\emph{von Arthur Schnitzler}!1912-07-301@{30. 7. 1912}|(be}
\toendnotes[C]{\smallbreak\pagebreak[2]}\Standort{YCGL, MSS 31.}
\physDesc{Bildpostkarte, 252 Zeichen
\newline{}Handschrift Arthur Schnitzler: Bleistift, deutsche Kurrent
\newline{}Handschrift Olga Schnitzler: Bleistift, lateinische Kurrent
\newline{}Versand: 1) Stempel: »\nobreak{}\oindex{Brijuni@\textbf{Brijuni}, \emph{P.PPL}|pwk}Brioni, 30. 7. \textcolor{gray}{12}\nobreak{}«.   2) Stempel: »\nobreak{}\oindex{St. Moritz@\textbf{St. Moritz}, \emph{P.PPL}|pwk}St. Moritz-\textcolor{gray}{Dorf}, 1. \textcolor{gray}{VIII. 12}, XII\nobreak{}«. 
\newline{}Beer-Hofmann: mit Bleistift das Datum der Beantwortung vermerkt: »B
                                       2/VIII 12« }\toendnotes[C]{\smallbreak}\pstart{}{\pb}Herrn \textsc{Dr. Richard
                     Beer-Hofmann –}\pend{}\pstart{}aus Wien\oindex{Wien@\textbf{Wien}, \emph{A.ADM2}|pw}\pend{}\pstart{}\textsc{St. Morit\textcolor{gray}{z}\oindex{St. Moritz@\textbf{St. Moritz}, \emph{P.PPL}|pw}}\pend{}\pstart{}\textsc{im Engadin\oindex{Engadin@\textbf{Engadin}, \emph{T.VAL}|pw}}\pend{}\pstart{}\textsc{Waldhaus\oindex{Hotel Waldhaus am See@\textbf{Hotel Waldhaus am See}, \emph{Hotel (K.HTL)}|pw}}\pend{}{\bigskip}
\pstart
           \noindent{}\centering{}{\pb}\textcolor{gray}{\textbf{Insel Brioni i. d. Adria\oindex{Brijuni@\textbf{Brijuni}, \emph{P.PPL}|pw}.}}\pend
           
\pstart
           \centering{}\textcolor{gray}{\textbf{Straße in Val Madonna\oindex{Val Madonna@\textbf{Val Madonna}, \emph{Tal (N.TAL)}|pw}}}\pend
           \vspace{1em}
\pstart
           \noindent{}{\pb}Herzliche Grüße Ihnen Allen\pwindex{Beer-Hofmann, Naemah 20.12.1898 – 10.11.1971@\textsc{Beer-Hofmann, Naëmah} (20.12.1898 – 10.11.1971)|pwv}\pwindex{Beer-Hofmann, Paula 25.02.1879 – 30.10.1939@\textsc{Beer-Hofmann, Paula} (25.02.1879 – 30.10.1939)|pwv}\pwindex{Beer-Hofmann, Gabriel 09.01.1901 – 24.03.1971@\textsc{Beer-Hofmann, Gabriel} (09.01.1901 – 24.03.1971), \emph{Schriftsteller/Schriftstellerin, Filmagent/Filmagentin}|pwv}\pwindex{Beer-Hofmann, Mirjam 04.09.1897 – 24.12.1984@\textsc{Beer-Hofmann, Mirjam} (04.09.1897 – 24.12.1984)|pwv}. – Es iſt
               wunderſchön hier, wir bleiben noch bis nach 20. 8, fahren da{\geminationn} direct \textsc{Sils Maria}\oindex{Sils im Engadin/Segl@\textbf{Sils im Engadin/Segl}, \emph{A.ADM3}|pw}. Schreiben Sie ein Wort, wies Ihnen allen\pwindex{Beer-Hofmann, Naemah 20.12.1898 – 10.11.1971@\textsc{Beer-Hofmann, Naëmah} (20.12.1898 – 10.11.1971)|pwv}\pwindex{Beer-Hofmann, Paula 25.02.1879 – 30.10.1939@\textsc{Beer-Hofmann, Paula} (25.02.1879 – 30.10.1939)|pwv}\pwindex{Beer-Hofmann, Gabriel 09.01.1901 – 24.03.1971@\textsc{Beer-Hofmann, Gabriel} (09.01.1901 – 24.03.1971), \emph{Schriftsteller/Schriftstellerin, Filmagent/Filmagentin}|pwv}\pwindex{Beer-Hofmann, Mirjam 04.09.1897 – 24.12.1984@\textsc{Beer-Hofmann, Mirjam} (04.09.1897 – 24.12.1984)|pwv} geht.\pend
           
\pstart
           Ihr{\\[\baselineskip]}\spacefill\mbox{Arthur}\pend
           \leftskip=0em{}\selectlanguage{ngerman}\vspace{1em}
\pstart
           \noindent{}{[}hs. :{]} Herzl. Grüsse!\pend
           \pstart \spacefill\mbox{Olga.}\pend{}\selectlanguage{ngerman}\endnumbering\briefempfaengerindex{Beer-Hofmann, Richard@\textsc{Beer-Hofmann, Richard}!zzzSchnitzler, Olga@\emph{von Olga Schnitzler}!1912-07-301@{30. 7. 1912}|)be}\briefempfaengerindex{Beer-Hofmann, Richard@\textsc{Beer-Hofmann, Richard}!zzzSchnitzler, Arthur@\emph{von Arthur Schnitzler}!1912-07-301@{30. 7. 1912}|)be}\mylabel{L02080h}  \normalsize

\doendnotes{C}
\bigskip
\vfill

\clearpage

\footnotesize

\lohead{\textsc{register}}

% Definiere theindex-Environment komplett neu ohne reledmac
\makeatletter
\renewenvironment{theindex}{%
  \section*{\indexname}%
  \setlength{\parindent}{0pt}%
  \setlength{\parskip}{0pt plus 0.3pt}%
  \let\item\@idxitem
}{%
  \clearpage
}
\makeatother

\IfFileExists{\jobname-pw.ind}{\input{\jobname-pw.ind}}{}

\end{document}

      