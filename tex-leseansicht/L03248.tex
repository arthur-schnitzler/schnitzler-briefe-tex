%% latex-korrekturansicht-vorspann.tex
%% Vorspann für die Korrekturansicht.
%% Lädt die gemeinsame Datei latex-vorspann.tex mit gesetztem Schalter.

\newif\ifkorrekturansicht
\korrekturansichttrue

\input{../tex-inputs/latex-vorspann}


\section[ Paul Goldmann an Arthur Schnitzler, 13. 7. 1906]{L03248 Paul Goldmann an Arthur Schnitzler, 13. 7. 1906}
\nopagebreak\mylabel{L03248v}
\rehead{ }\normalsize\beginnumbering\briefempfaengerindex{Schnitzler, Arthur@\textsc{Schnitzler, Arthur}!zzzGoldmann, Paul@\emph{von Paul Goldmann}!1906-07-133@{13. 7. 1906}|(be}
\toendnotes[C]{\smallbreak\pagebreak[2]}\Standort{DLA, A:Schnitzler, HS.NZ85.1.3175.}
\physDesc{Postkarte, 421 Zeichen
\newline{}Handschrift: 1) blaue Tinte, deutsche Kurrent\hspace{1em}2) blaue Tinte, lateinische Kurrent (\noindent{}Adresse)\hspace{1em}
\newline{}Versand: 1) Stempel: »\nobreak{}\oindex{Berlin@\textbf{Berlin}, \emph{P.PPLC}|pwk}Berlin, S.W. 11, 13. 7. 06, 2–3N.\nobreak{}«.   2) Stempel: »\nobreak{}\oindex{Helsingør@\textbf{Helsingør}, \emph{P.PPLA2}|pwk}Helsingør, 14. 7. 06, 11–12F\nobreak{}«. 
\newline{}Schnitzler: mit Bleistift das Jahr »906« vermerkt }\toendnotes[C]{\smallbreak}\pstart{}{\pb}Welt-\textcolor{gray}{\textbf{Poſtkarte}}\pend{}\pstart{}Herrn\pend{}\pstart{}Dr. Arthur Schnitzler (aus Wien\oindex{Wien@\textbf{Wien}, \emph{A.ADM2}|pw})\pend{}\pstart{}Marienlyst\oindex{Marienlyst@\textbf{Marienlyst}, \emph{S.EST}|pw}\pend{}\pstart{}Dänemark\oindex{Daenemark@\textbf{Dänemark}, \emph{A.PCLI}|pw}. \pend{}{\bigskip}\vspace{1em}
\pstart
           \noindent{}{\pb}Berlin\oindex{Berlin@\textbf{Berlin}, \emph{P.PPLC}|pw}, 13. Juli.
                  Lieber Freund, Nach Dänemark\oindex{Daenemark@\textbf{Dänemark}, \emph{A.PCLI}|pw} komme ich nicht – ich ſuche wieder ſtarke
               Gebirgsluft auf u. ſchwanke noch zwiſchen Schweiz\oindex{Schweiz@\textbf{Schweiz}, \emph{A.PCLI}|pw} u. Tirol\oindex{Tirol@\textbf{Tirol}, \emph{A.ADM1}|pw}.
               K\textcolor{gray}{ä}meſt Du nicht vielleicht nach Dänemark\oindex{Daenemark@\textbf{Dänemark}, \emph{A.PCLI}|pw} noch ins \label{K_L03248-1v}\edtext{Gebirge}{\lemma{\textnormal{\emph{Gebirge}}}\Cendnote{\textnormal{Dazu kam es nicht.}}}\label{K_L03248-1}? Ich
               würde mich ſehr freuen, \strikeout{Dich}{ }\strikeout{\textcolor{gray}{da}} irgendwo mit Dir \label{K_L03248-2v}\edtext{zuſammenzutreffen}{\lemma{\textnormal{\emph{zuſammenzutreffen}}}\Cendnote{\textnormal{Schnitzler und Goldmann\pwindex{Goldmann, Paul 31.01.1865 – 25.09.1935@\textsc{Goldmann, Paul} (31.01.1865 – 25.09.1935), \emph{Schriftsteller/Schriftstellerin, Journalist/Journalistin}|pwk} trafen sich erst am 24. 5. 1907 in Wien\oindex{Wien@\textbf{Wien}, \emph{A.ADM2}|pwk} wieder.}}}\label{K_L03248-2}. Herzliche Grüße Dir u.
               Deiner Frau\pwindex{Schnitzler, Olga 17.01.1882 – 13.01.1970@\textsc{Schnitzler, Olga} (17.01.1882 – 13.01.1970), \emph{Schauspieler/Schauspielerin, Sänger/Sängerin}|pwv} von Deinem
               getreuen \spacefill\mbox{Paul Goldmann.}\pend
           
\pstart
           \noindent{}Was macht \textsc{Heinrich Schnitzler\pwindex{Schnitzler, Heinrich 09.08.1902 – 12.07.1982@\textsc{Schnitzler, Heinrich} (09.08.1902 – 12.07.1982), \emph{Regisseur/Regisseurin, Schauspieler/Schauspielerin}|pw}}?\pend
           \selectlanguage{ngerman}\endnumbering\briefempfaengerindex{Schnitzler, Arthur@\textsc{Schnitzler, Arthur}!zzzGoldmann, Paul@\emph{von Paul Goldmann}!1906-07-133@{13. 7. 1906}|)be}\mylabel{L03248h}  \normalsize

\doendnotes{C}
\bigskip
\vfill

\clearpage

\footnotesize

\lohead{\textsc{register}}

% Definiere theindex-Environment komplett neu ohne reledmac
\makeatletter
\renewenvironment{theindex}{%
  \section*{\indexname}%
  \setlength{\parindent}{0pt}%
  \setlength{\parskip}{0pt plus 0.3pt}%
  \let\item\@idxitem
}{%
  \clearpage
}
\makeatother

\IfFileExists{\jobname-pw.ind}{\input{\jobname-pw.ind}}{}

\end{document}

      