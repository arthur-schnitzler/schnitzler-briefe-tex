%% latex-leseansicht-vorspann.tex
%% Vorspann für die Leseansicht.
%% Lädt die gemeinsame Datei latex-vorspann.tex mit nicht gesetztem Schalter.

\newif\ifkorrekturansicht
\korrekturansichtfalse

\input{../tex-inputs/latex-vorspann}


\section[ Paul Goldmann an Arthur Schnitzler, 13. 7. 1906]{L03248 Paul Goldmann an Arthur Schnitzler,  13. 7. 1906}
\nopagebreak\mylabel{L03248v}
\rehead{ }\normalsize\beginnumbering\briefempfaengerindex{Schnitzler, Arthur@\textsc{Schnitzler, Arthur}!zzzGoldmann, Paul@\emph{von Paul Goldmann}!1906-07-133@{13. 7. 1906}|(be}
\toendnotes[C]{\smallbreak\pagebreak[2]}
\correspDesc{Versand  durch Paul Goldmann am 13. 7. 1906 in Berlin
\newline{}Zustellung  am 14. 7. 1906 in Helsingør
\newline{}Erhalt  durch Arthur Schnitzler im Zeitraum [14. 7. 1906
                  – 15. 7. 1906?] in Marienlyst}\toendnotes[C]{\smallbreak}
\Standort{DLA, A:Schnitzler, HS.NZ85.1.3175.}
\physDesc{Postkarte, 421 Zeichen
\newline{}Handschrift: blaue Tinte, deutsche Kurrent
\newline{}Versand: 1) Stempel: »\nobreak{}\oindex{Berlin@\textbf{Berlin}, \emph{Hauptstadt}|pwk}Berlin, S.W. 11, 13. 7. 06, 2–3N.\nobreak{}«.   2) Stempel: »\nobreak{}\oindex{Helsingør@\textbf{Helsingør}, \emph{Hauptstadt}|pwk}Helsingør, 14. 7. 06, 11–12F\nobreak{}«. 
\newline{}Schnitzler: mit Bleistift das Jahr »906« vermerkt }\toendnotes[C]{\smallbreak}\pstart{}{\pb}\textsc{Welt-}\textcolor{gray}{\textbf{Poſtkarte}}\pend{}\pstart{}\textsc{Herrn}\pend{}\pstart{}\textsc{Dr. Arthur Schnitzler (aus Wien\oindex{Wien@\textbf{Wien}, \emph{Verwaltungsgebiet}|pw})}\pend{}\pstart{}\textsc{Marienlyst\oindex{Marienlyst@\textbf{Marienlyst}, \emph{Gut}|pw}}\pend{}\pstart{}\textsc{Dänemark\oindex{Dänemark@\textbf{Dänemark}|pw}.}\pend{}{\bigskip}\vspace{1em}
\pstart
           \noindent{}{\pb}Berlin\oindex{Berlin@\textbf{Berlin}, \emph{Hauptstadt}|pw}, 13. Juli.
                  Lieber Freund, Nach Dänemark\oindex{Dänemark@\textbf{Dänemark}|pw} komme ich nicht – ich{ }ſuche wieder{ }ſtarke
               Gebirgsluft auf u.{ }ſchwanke noch zwiſchen Schweiz\oindex{Schweiz@\textbf{Schweiz}|pw} u. Tirol\oindex{Tirol@\textbf{Tirol}, \emph{Land}|pw}.
               K\textcolor{gray}{ä}meſt Du nicht vielleicht nach Dänemark\oindex{Dänemark@\textbf{Dänemark}|pw} noch ins \label{K_L03248-1v}\edtext{Gebirge}{\lemma{\textnormal{\emph{Gebirge}}}\Cendnote{\textnormal{Dazu kam es nicht.}}}\label{K_L03248-1}? Ich
               würde mich{ }ſehr freuen, \strikeout{Dich}{ }\strikeout{\textcolor{gray}{da}} irgendwo mit Dir \label{K_L03248-2v}\edtext{zuſammenzutreffen}{\lemma{\textnormal{\emph{zusammenzutreffen}}}\Cendnote{\textnormal{Schnitzler und Goldmann\pwindex{Goldmann, Paul 31.\,1.\,1865 Breslau – 25.\,9.\,1935 Wien@\textsc{Goldmann, Paul} (31.\,1.\,1865 Breslau – 25.\,9.\,1935 Wien), \emph{Schriftsteller, Journalist}|pwk} trafen sich erst am 24. 5. 1907 in Wien\oindex{Wien@\textbf{Wien}, \emph{Verwaltungsgebiet}|pwk} wieder.}}}\label{K_L03248-2}. Herzliche Grüße Dir u.
               Deiner Frau\pwindex{Schnitzler, Olga 17.\,1.\,1882 Wien – 13.\,1.\,1970 Lugano@\textsc{Schnitzler, Olga} (17.\,1.\,1882 Wien – 13.\,1.\,1970 Lugano), \emph{Schauspielerin, Sängerin}|pwv} von Deinem
               getreuen \spacefill\mbox{Paul Goldmann.}\pend
           
\pstart
           \noindent{}Was macht \textsc{Heinrich Schnitzler\pwindex{Schnitzler, Heinrich 9.\,8.\,1902 Hinterbrühl – 12.\,7.\,1982 Wien@\textsc{Schnitzler, Heinrich} (9.\,8.\,1902 Hinterbrühl – 12.\,7.\,1982 Wien), \emph{Regisseur, Schauspieler}|pw}}?\pend
           \selectlanguage{ngerman}\endnumbering\briefempfaengerindex{Schnitzler, Arthur@\textsc{Schnitzler, Arthur}!zzzGoldmann, Paul@\emph{von Paul Goldmann}!1906-07-133@{13. 7. 1906}|)be}\mylabel{L03248h}  \newcommand{\dateiname}{L03248}\newcommand{\titel}{Paul Goldmann an Arthur Schnitzler, 13. 7. 1906}\newcommand{\editorInnen}{Martin Anton Müller und Laura Untner}%% latex-leseansicht-abspann.tex
%% Abspann für die Leseansicht.
%% Der Schalter \ifkorrekturansicht ist bereits durch den Vorspann gesetzt.

%% latex-abspann.tex
%% Gemeinsamer Abspann für Korrekturansicht und Leseansicht.
%% Setzt den Schalter \ifkorrekturansicht voraus (gesetzt in den
%% einbindenden Dateien latex-korrekturansicht-abspann.tex bzw.
%% latex-leseansicht-abspann.tex).
%% ---------------------------------------------------------------

\normalsize

% Das esempio-Environment wird nur in der Leseansicht benötigt
\ifkorrekturansicht\else
\newenvironment{esempio}[3]%
{
    \vspace{1.5ex}
    \rlap{\underline{#1}}
    \par
    \setlength{\parindent}{0cm}
    \nopagebreak
    \leftskip=#2cm
    \rightskip=#3cm
}
{
    \par
}
\fi

\doendnotes{C}
\bigskip
\vfill

\clearpage

\footnotesize

\ifkorrekturansicht
  \lohead{\textsc{register}}
\fi

% theindex-Environment neu definieren ohne reledmac
\makeatletter
\renewenvironment{theindex}{%
  \ifkorrekturansicht
    \section*{\indexname}%
  \else
    \subsubsection*{Index der erwähnten Entitäten}%
  \fi
  \setlength{\parindent}{0pt}%
  \setlength{\parskip}{0pt plus 0.3pt}%
  \let\item\@idxitem
}{%
  \ifkorrekturansicht\clearpage\fi
}
\makeatother

\IfFileExists{\jobname-pw.ind}{\input{\jobname-pw.ind}}{}

% Quellenangabe nur in der Leseansicht
\ifkorrekturansicht\else
% Fallback-Definitionen, falls die .tex-Datei \titel etc. nicht gesetzt hat
\providecommand{\titel}{}
\providecommand{\editorInnen}{}
\providecommand{\dateiname}{\jobname}

\vspace{3cm}

\vfill

\footnotesize
\textsc{Quelle}: \titel. Herausgegeben von {\editorInnen}. In: \emph{Arthur Schnitzler: Briefwechsel mit Autorinnen und Autoren}.
 Digitale Edition, https://schnitzler-briefe.acdh.oeaw.ac.at/{\dateiname}.html (Stand \today)
\fi

\end{document}


