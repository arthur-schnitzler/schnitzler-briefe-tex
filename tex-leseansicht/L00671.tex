%% latex-korrekturansicht-vorspann.tex
%% Vorspann für die Korrekturansicht.
%% Lädt die gemeinsame Datei latex-vorspann.tex mit gesetztem Schalter.

\newif\ifkorrekturansicht
\korrekturansichttrue

\input{../tex-inputs/latex-vorspann}


\section[Arthur Schnitzler an Hugo von Hofmannsthal, 26. 4. 1897]{L00671 Arthur Schnitzler an Hugo von Hofmannsthal, 26. 4. 1897}
\nopagebreak\mylabel{L00671v}
\rehead{ }\normalsize\beginnumbering\briefempfaengerindex{Hofmannsthal, Hugo von@\textsc{Hofmannsthal, Hugo von}!zzzSchnitzler, Arthur@\emph{von Arthur Schnitzler}!1897-04-262@{26. 4. 1897}|(be}
\toendnotes[C]{\smallbreak\pagebreak[2]}\Standort{FDH, Hs-30885,56.}
\physDesc{Brief, 1 Blatt, 4 Seiten, 2525 Zeichen
\newline{}Handschrift: schwarze Tinte, deutsche Kurrent}
\buchAbdrucke{\weitereDrucke{1) Hugo von Hofmannsthal, Arthur Schnitzler: \emph{Briefwechsel}. Frankfurt am Main: \emph{S. Fischer} 1964, S. 81–82.} \weitereDrucke{2) Arthur Schnitzler: \emph{Briefe 1875–1912}. Frankfurt am Main: \emph{S. Fischer} 1981, S. 319–320.} }\toendnotes[C]{\smallbreak}
\pstart
           \raggedleft{}{\pb}5 \textsc{rue \introOben{}de\introOben{} Maubeuge}\oindex{rue de Maubeuge@\textbf{rue de Maubeuge}, \emph{Straße (K.STR)}|pw}{\\}\textsc{Paris}\oindex{Paris@\textbf{Paris}, \emph{P.PPLC}|pw}. 2\substVorne{}\textsuperscript{7}\substDazwischen{}6\substHinten{}. 4. 97.\pend
           \vspace{0.5em}
\pstart
           Mein lieber Hugo. Seien Sie mir herzlich gegrüßt. Ich lebe im I{\geminationn}erſten der Stadt, wie ich in Wien\oindex{Wien@\textbf{Wien}, \emph{A.ADM2}|pw} um keinen Preis leben möchte; an der Kreuzung vieler
               Straßen, mitten im Lärm der Geſchäfte u des Verkehrs. Der Zufall hat es gefügt, daſs
               ich gerade hier die Wohnung gefunden habe, wie ich ſie brauche, und günſtige
               Verbindungen von Goldmann\pwindex{Goldmann, Paul 31.01.1865 – 25.09.1935@\textsc{Goldmann, Paul} (31.01.1865 – 25.09.1935), \emph{Schriftsteller/Schriftstellerin, Journalist/Journalistin}|pw} haben ſie mir
               verſchafft. Ich ſage \uline{mir}, obwohl das nicht ganz richtig\pwindex{Reinhard, Marie 1871-03-13 – 1899-03-18@\textsc{Reinhard, Marie} (1871-03-13 – 1899-03-18), \emph{Gesangspädagoge/Gesangspädagogin}|pwv} iſt. Aber ich habe
               mein Zi{\geminationm}er allein u ſo viel Freiheit, als unter den
               bekannten Umſtänden möglich iſt. Manchmal möcht ich wohl lieber ganz allein ſein;
               aber vielleicht iſt {\pb}es nur die Sehnſucht nach der ich
               mich ſehne. Ich bin nemlich bisher wirklich noch nie von Wien\oindex{Wien@\textbf{Wien}, \emph{A.ADM2}|pw} fortgeweſen, ohne dort irgendwen zurück zu laſſen, um den
               ich mehr oder weniger »zittern« mußte; das geht mir vielleicht ab. Im ganzen aber
               fühl ich mich, wie Sie ſagen würden »eher« wohl; insbeſondere tritt das ſonderbare
               ein, was ſich i{\geminationm}er beinah einſtellt, we{\geminationn} ich auf Reiſen, beſſer: we{\geminationn} ich nicht daheim bin; ich bin beinah gänzlich erlöſt von den Bangigkeiten und
               Hypochondrien, die mir das Leben zu Hauſe oft ſo heftig ſtören. Aber \introOben{}auch\introOben{} daſs ich gerade \uline{hier} bin,
               freut mich. Es iſt mir oft, als we{\geminationn} ich hier lieber
               leben möchte als in Wien\oindex{Wien@\textbf{Wien}, \emph{A.ADM2}|pw}; aber das iſt
                  wahrſchein{\pb}lich ein Irrtum. Von allem, was ich hier
               ſchon geſehn, möchte ich Ihnen lieber erſt in Wien\oindex{Wien@\textbf{Wien}, \emph{A.ADM2}|pw}
               erzählen; denn ich frage mich vergeblich, was ich herausſuchen ſollte. Das ſchönſte
               hat mir bisher die Schauſpielerei geboten; es iſt einfach was andres als die
               Deutſchen haben; nicht immer was beſſres vielleicht – aber dem Weſen der Stücke, die
               ſie ſpielen, wunderbar verwandt, was ja ſchließlich doch das wichtigſte iſt. Dramen
               ſcheinen sie ja hier (wo denn???) auch nicht mehr zu ſchreiben; ich habe \textsc{loi de l’homme\pwindex{Maennerrecht@\emph{Männerrecht}|pw}, (Hervieu\pwindex{Hervieu, Paul Ernest 2.9.1857 – 25.10.1915@\textsc{Hervieu, Paul Ernest} (2.9.1857 – 25.10.1915), \emph{Schriftsteller/Schriftstellerin}|pw}); Douloureuse\pwindex{Douloureuse@\emph{La Douloureuse}|pw}
                     (Donnay\pwindex{Donnay, Maurice 12.10.1859 – 31.03.1945@\textsc{Donnay, Maurice} (12.10.1859 – 31.03.1945), \emph{Schriftsteller/Schriftstellerin}|pw}), – Carrière\pwindex{Carriere@\emph{La Carrière}|pw} (Hermant\pwindex{Hermant, Abel 03.02.1862 – 28.09.1950@\textsc{Hermant, Abel} (03.02.1862 – 28.09.1950), \emph{Schriftsteller/Schriftstellerin}|pw}); –
                     Snob\pwindex{Snob@\emph{Snob}|pw} (Guiche\pwindex{Guiches, Gustave 18.6.1860 – 3.8.1935@\textsc{Guiches, Gustave} (18.6.1860 – 3.8.1935), \emph{Schriftsteller/Schriftstellerin}|pw})} – geſehen – es iſt ein vollko{\geminationm}ener Sieg des Feuilletons auf dem Theater. Ich habe {\pb}wohl auch ein bischen das Gefühl des »Menſchenfreunds\pwindex{Alpenkoenig und der Menschenfeind@\emph{Der Alpenkönig und der Menschenfeind}|pwv}« aus dem Raimund\pwindex{Raimund, Ferdinand 01.06.1790 – 05.09.1836@\textsc{Raimund, Ferdinand} (01.06.1790 – 05.09.1836), \emph{Schauspieler/Schauspielerin, Dramatiker/Dramatikerin}|pw}’ſchen Märchen gehabt, – aber können wir wirklichen Menſchen uns auch
               »beſſern«? Mit Bewußtſein entwickeln – das müßte wohl möglich ſein! –\pend
           
\pstart
           – Sagen Sie mir ein Wort, wie es Ihnen und andren Leuten, von denen Sie gerade
               erzählen wollen (was mir jedenfalls erwünſcht wäre) geht. – Ich werde Ende
                  Mai, ſpäteſtens Anfang Juni wieder in Wien\oindex{Wien@\textbf{Wien}, \emph{A.ADM2}|pw}{ }ſein. Das Wetter iſt nicht ſchön; noch ke{\geminationn} ich eigentlich den Pariſ\oindex{Paris@\textbf{Paris}, \emph{P.PPLC}|pw}er Frühling nicht.\pend
           
\pstart
           Grüßen Sie alle, die wir beide gern haben.\pend
           \pstart Herzlich grüßt Sie Ihr \spacefill\mbox{Arthur.}\pend{}
\pstart
           \noindent{}Auch Ihren Eltern\pwindex{Hofmannsthal, Hugo August von 21.12.1841 – 08.12.1915@\textsc{Hofmannsthal, Hugo August von} (21.12.1841 – 08.12.1915), \emph{Bankdirektor/Bankdirektorin}|pwv}\pwindex{Hofmannsthal, Anna von 27.01.1849 – 22.03.1904@\textsc{Hofmannsthal, Anna von} (27.01.1849 – 22.03.1904)|pwv}, bitte, empfehlen Sie mich freundlich.\pend
           \selectlanguage{ngerman}\endnumbering\briefempfaengerindex{Hofmannsthal, Hugo von@\textsc{Hofmannsthal, Hugo von}!zzzSchnitzler, Arthur@\emph{von Arthur Schnitzler}!1897-04-262@{26. 4. 1897}|)be}\mylabel{L00671h}  \normalsize

\doendnotes{C}
\bigskip
\vfill

\clearpage

\footnotesize

\lohead{\textsc{register}}

% Definiere theindex-Environment komplett neu ohne reledmac
\makeatletter
\renewenvironment{theindex}{%
  \section*{\indexname}%
  \setlength{\parindent}{0pt}%
  \setlength{\parskip}{0pt plus 0.3pt}%
  \let\item\@idxitem
}{%
  \clearpage
}
\makeatother

\IfFileExists{\jobname-pw.ind}{\input{\jobname-pw.ind}}{}

\end{document}

      