%% latex-leseansicht-vorspann.tex
%% Vorspann für die Leseansicht.
%% Lädt die gemeinsame Datei latex-vorspann.tex mit nicht gesetztem Schalter.

\newif\ifkorrekturansicht
\korrekturansichtfalse

\input{../tex-inputs/latex-vorspann}


         
         \renewcommand{\erwaehntePersonen}{Personen: Maurice Donnay, Paul Goldmann, Gustave Guiches, Abel Hermant, Paul Ernest Hervieu, Hugo von Hofmannsthal, Hugo August von Hofmannsthal, Anna von Hofmannsthal, Ferdinand Raimund, Marie Reinhard}
         \renewcommand{\erwaehnteOrte}{Orte: Paris, Wien, rue de Maubeuge}
         \renewcommand{\erwaehnteWerke}{Werke: Der Alpenkönig und der Menschenfeind, La Carrière, La Douloureuse, Männerrecht, Snob}
               \section[Arthur Schnitzler an Hugo von Hofmannsthal, 26. 4. 1897]{ Arthur Schnitzler an Hugo von Hofmannsthal, 26. 4. 1897}\nopagebreak\mylabel{v}\rehead{ }\begin{ledgroupsized}[t]{13cm}\normalsize\beginnumbering \toendnotes[C]{\smallbreak\pagebreak[2]} \Standort{FDH, Hs-30885,56.}
\physDesc{Brief, 1 Blatt, 4 Seiten
\newline{}Handschrift: schwarze Tinte, deutsche Kurrent}\buchAbdrucke{\weitereDrucke{1) Hugo von Hofmannsthal, Arthur Schnitzler: \emph{Briefwechsel}. Hg. Therese Nickl und Heinrich Schnitzler. Frankfurt am Main: \emph{S. Fischer} 1964, S. 81–82.} \weitereDrucke{2) Arthur Schnitzler: \emph{Briefe 1875–1912}. Hg. Therese Nickl und Heinrich Schnitzler. Frankfurt am Main: \emph{S. Fischer} 1981, S. 319–320.} }\toendnotes[C]{\smallbreak}\pstart
           \raggedleft{}{\pb}5 \textsc{rue \introOben{}de\introOben{} Maubeuge}\oindex{rue de Maubeuge@\textbf{rue de Maubeuge}|pw}{\\}\textsc{Paris}\oindex{Paris@\textbf{Paris}|pw}. 2\substVorne{}\textsuperscript{7}\substDazwischen{}6\substHinten{}. 4. 97.\pend
           \pstart
           Mein lieber Hugo. Seien Sie mir herzlich gegrüßt. Ich lebe im
                        I{\geminationn}erſten der Stadt, wie ich in Wien\oindex{Wien@\textbf{Wien}|pw} um keinen Preis leben möchte; an der Kreuzung vieler
                    Straßen, mitten im Lärm der Geſchäfte u des Verkehrs. Der Zufall hat es gefügt,
                    daſs ich gerade hier die Wohnung gefunden habe, wie ich ſie brauche, und
                    günſtige Verbindungen von Goldmann\pwindex{Goldmann, Paul 31.01.1865 – 25.09.1935@\textsc{Goldmann, Paul} (31.01.1865 – 25.09.1935), \emph{Schriftsteller, Journalist}|pw} haben ſie
                    mir verſchafft. Ich ſage \uline{mir}, obwohl das nicht
                    ganz richtig\pwindex{Reinhard, Marie 1871-03-13 – 1899-03-18@\textsc{Reinhard, Marie} (1871-03-13 – 1899-03-18), \emph{Gesangspädagogin}|pwv} iſt. Aber ich habe mein Zi{\geminationm}er allein u ſo viel Freiheit, als unter
                    den bekannten Umſtänden möglich iſt. Manchmal möcht ich wohl lieber ganz allein
                    ſein; aber vielleicht iſt {\pb}es nur die Sehnſucht nach
                    der ich mich ſehne. Ich bin nemlich bisher wirklich noch nie von Wien\oindex{Wien@\textbf{Wien}|pw} fortgeweſen, ohne dort irgendwen zurück zu laſſen, um
                    den ich mehr oder weniger »zittern« mußte; das geht mir vielleicht ab. Im ganzen
                    aber fühl ich mich, wie Sie ſagen würden »eher« wohl; insbeſondere tritt das
                    ſonderbare ein, was ſich i{\geminationm}er beinah einſtellt,
                    we{\geminationn} ich auf Reiſen, beſſer: we{\geminationn} ich nicht daheim
                    bin; ich bin beinah gänzlich erlöſt von den Bangigkeiten und Hypochondrien, die
                    mir das Leben zu Hauſe oft ſo heftig ſtören. Aber \introOben{}auch\introOben{}
                    daſs ich gerade \uline{hier} bin, freut mich. Es iſt mir
                    oft, als we{\geminationn} ich hier lieber leben möchte als in Wien\oindex{Wien@\textbf{Wien}|pw}; aber das iſt wahrſchein{\pb}lich ein
                    Irrtum. Von allem, was ich hier ſchon geſehn, möchte ich Ihnen lieber erſt in
                        Wien\oindex{Wien@\textbf{Wien}|pw} erzählen; denn ich frage mich
                    vergeblich, was ich herausſuchen ſollte. Das ſchönſte hat mir bisher die
                    Schauſpielerei geboten; es iſt einfach was andres als die Deutſchen haben; nicht
                    immer was beſſres vielleicht – aber dem Weſen der Stücke, die ſie ſpielen,
                    wunderbar verwandt, was ja ſchließlich doch das wichtigſte iſt. Dramen ſcheinen
                    sie ja hier (wo denn???) auch nicht mehr zu ſchreiben; ich habe \textsc{loi de l’homme\pwindex{Hervieu, Paul Ernest 2.9.1857 – 25.10.1915@\textsc{Hervieu, Paul Ernest} (2.9.1857 – 25.10.1915), \emph{Schriftsteller}!Maennerrecht1897@\strich\emph{Männerrecht} {[}1897{]}|pw}, (Hervieu\pwindex{Hervieu, Paul Ernest 2.9.1857 – 25.10.1915@\textsc{Hervieu, Paul Ernest} (2.9.1857 – 25.10.1915), \emph{Schriftsteller}|pw}); Douloureuse\pwindex{Donnay, Maurice 12.10.1859 – 31.03.1945@\textsc{Donnay, Maurice} (12.10.1859 – 31.03.1945), \emph{Schriftsteller}!Douloureuse1895@\strich\emph{La Douloureuse} {[}1895{]}|pw} (Donnay\pwindex{Donnay, Maurice 12.10.1859 – 31.03.1945@\textsc{Donnay, Maurice} (12.10.1859 – 31.03.1945), \emph{Schriftsteller}|pw}), – Carrière\pwindex{Hermant, Abel 03.02.1862 – 28.09.1950@\textsc{Hermant, Abel} (03.02.1862 – 28.09.1950), \emph{Schriftsteller}!Carriere1894@\strich\emph{La Carrière} {[}1894{]}|pw} (Hermant\pwindex{Hermant, Abel 03.02.1862 – 28.09.1950@\textsc{Hermant, Abel} (03.02.1862 – 28.09.1950), \emph{Schriftsteller}|pw}); – Snob\pwindex{Guiches, Gustave 18.6.1860 – 3.8.1935@\textsc{Guiches, Gustave} (18.6.1860 – 3.8.1935), \emph{Schriftsteller}!Snob1897@\strich\emph{Snob} {[}1897{]}|pw} (Guiche\pwindex{Guiches, Gustave 18.6.1860 – 3.8.1935@\textsc{Guiches, Gustave} (18.6.1860 – 3.8.1935), \emph{Schriftsteller}|pw})} – geſehen – es iſt ein
                    vollko{\geminationm}ener Sieg des Feuilletons auf dem Theater. Ich habe {\pb}wohl auch ein bischen das Gefühl des »Menſchenfreunds\pwindex{Raimund, Ferdinand 01.06.1790 – 05.09.1836@\textsc{Raimund, Ferdinand} (01.06.1790 – 05.09.1836), \emph{Schauspieler, Dramatiker}!Alpenkoenig und der Menschenfeind1828@\strich\emph{Der Alpenkönig und der Menschenfeind} {[}1828{]}|pwv}« aus dem Raimund\pwindex{Raimund, Ferdinand 01.06.1790 – 05.09.1836@\textsc{Raimund, Ferdinand} (01.06.1790 – 05.09.1836), \emph{Schauspieler, Dramatiker}|pw}’ſchen Märchen gehabt, – aber können
                    wir wirklichen Menſchen uns auch »beſſern«? Mit Bewußtſein entwickeln – das müßte
                    wohl möglich ſein! –\pend
           \pstart
           – Sagen Sie mir ein Wort, wie es Ihnen und andren Leuten, von denen Sie gerade
                    erzählen wollen (was mir jedenfalls erwünſcht wäre) geht. – Ich werde Ende
                        Mai, ſpäteſtens Anfang Juni wieder in Wien\oindex{Wien@\textbf{Wien}|pw}{ }ſein. Das Wetter iſt nicht ſchön; noch ke{\geminationn} ich eigentlich den Pariſ\oindex{Paris@\textbf{Paris}|pw}er Frühling nicht.\pend
           \pstart
           Grüßen Sie alle, die wir beide gern haben.\pend
           \pstart Herzlich grüßt Sie Ihr \spacefill\mbox{Arthur.}\pend{}\pstart
           \noindent{}Auch Ihren Eltern\pwindex{Hofmannsthal, Hugo August von 21.12.1841 – 08.12.1915@\textsc{Hofmannsthal, Hugo August von} (21.12.1841 – 08.12.1915), \emph{Bankdirektor}|pwv}\pwindex{Hofmannsthal, Anna von 27.01.1849 – 22.03.1904@\textsc{Hofmannsthal, Anna von} (27.01.1849 – 22.03.1904)|pwv}, bitte, empfehlen Sie mich freundlich.\pend
           
         
         \endnumbering\mylabel{h}\end{ledgroupsized}  \newcommand{\dateiname}{L00671}\newcommand{\titel}{Arthur Schnitzler an Hugo von Hofmannsthal, 26. 4. 1897}\newcommand{\editorInnen}{Martin Anton Müller und Gerd-Hermann Susen}%% latex-leseansicht-abspann.tex
%% Abspann für die Leseansicht.
%% Der Schalter \ifkorrekturansicht ist bereits durch den Vorspann gesetzt.

%% latex-abspann.tex
%% Gemeinsamer Abspann für Korrekturansicht und Leseansicht.
%% Setzt den Schalter \ifkorrekturansicht voraus (gesetzt in den
%% einbindenden Dateien latex-korrekturansicht-abspann.tex bzw.
%% latex-leseansicht-abspann.tex).
%% ---------------------------------------------------------------

\normalsize

% Das esempio-Environment wird nur in der Leseansicht benötigt
\ifkorrekturansicht\else
\newenvironment{esempio}[3]%
{
    \vspace{1.5ex}
    \rlap{\underline{#1}}
    \par
    \setlength{\parindent}{0cm}
    \nopagebreak
    \leftskip=#2cm
    \rightskip=#3cm
}
{
    \par
}
\fi

\doendnotes{C}
\bigskip
\vfill

\clearpage

\footnotesize

\ifkorrekturansicht
  \lohead{\textsc{register}}
\fi

% theindex-Environment neu definieren ohne reledmac
\makeatletter
\renewenvironment{theindex}{%
  \ifkorrekturansicht
    \section*{\indexname}%
  \else
    \subsubsection*{Index der erwähnten Entitäten}%
  \fi
  \setlength{\parindent}{0pt}%
  \setlength{\parskip}{0pt plus 0.3pt}%
  \let\item\@idxitem
}{%
  \ifkorrekturansicht\clearpage\fi
}
\makeatother

\IfFileExists{\jobname-pw.ind}{\input{\jobname-pw.ind}}{}

% Quellenangabe nur in der Leseansicht
\ifkorrekturansicht\else
% Fallback-Definitionen, falls die .tex-Datei \titel etc. nicht gesetzt hat
\providecommand{\titel}{}
\providecommand{\editorInnen}{}
\providecommand{\dateiname}{\jobname}

\vspace{3cm}

\vfill

\footnotesize
\textsc{Quelle}: \titel. Herausgegeben von {\editorInnen}. In: \emph{Arthur Schnitzler: Briefwechsel mit Autorinnen und Autoren}.
 Digitale Edition, https://schnitzler-briefe.acdh.oeaw.ac.at/{\dateiname}.html (Stand \today)
\fi

\end{document}


      