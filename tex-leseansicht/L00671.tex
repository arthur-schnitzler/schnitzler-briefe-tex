%% latex-leseansicht-vorspann.tex
%% Vorspann für die Leseansicht.
%% Lädt die gemeinsame Datei latex-vorspann.tex mit nicht gesetztem Schalter.

\newif\ifkorrekturansicht
\korrekturansichtfalse

\input{../tex-inputs/latex-vorspann}


\section[Arthur Schnitzler an Hugo von Hofmannsthal, 26. 4. 1897]{L00671 Arthur Schnitzler an Hugo von Hofmannsthal, 26. 4. 1897}
\nopagebreak\mylabel{L00671v}
\rehead{ }\normalsize\beginnumbering\briefempfaengerindex{Hofmannsthal, Hugo von@\textsc{Hofmannsthal, Hugo von}!zzzSchnitzler, Arthur@\emph{von Arthur Schnitzler}!1897-04-262@{26. 4. 1897}|(be}
\toendnotes[C]{\smallbreak\pagebreak[2]}
\correspDesc{Versand  durch Arthur Schnitzler am 26. 4. 1897 in Paris
\newline{}Erhalt  durch Hugo von Hofmannsthal im Zeitraum [27. 4. 1897
                  – 1. 5. 1897?] in Wien}\toendnotes[C]{\smallbreak}
\Standort{FDH, Hs-30885,56.}
\physDesc{Brief, 1 Blatt, 4 Seiten, 2525 Zeichen
\newline{}Handschrift: schwarze Tinte, deutsche Kurrent}
\buchAbdrucke{\weitereDrucke{1) Hugo von Hofmannsthal, Arthur Schnitzler: \emph{Briefwechsel}. Herausgegeben von Therese Nickl und Heinrich Schnitzler. Frankfurt am Main: \emph{S. Fischer} 1964, S. 81–82.} \weitereDrucke{2) Arthur Schnitzler: \emph{Briefe 1875–1912}. Herausgegeben von Therese Nickl und Heinrich Schnitzler. Frankfurt am Main: \emph{S. Fischer} 1981, S. 319–320.} }\toendnotes[C]{\smallbreak}
\pstart
           \raggedleft{}{\pb}5 \textsc{rue \introOben{}de\introOben{} Maubeuge}\oindex{5, rue de Maubeuge@\textbf{5, rue de Maubeuge}, \emph{Wohngebäude}|pw}{\\}\textsc{Paris}\oindex{Paris@\textbf{Paris}, \emph{Hauptstadt}|pw}. 2\substVorne{}\textsuperscript{7}\substDazwischen{}6\substHinten{}. 4. 97.\pend
           \vspace{0.5em}
\pstart
           Mein lieber Hugo. Seien Sie mir herzlich gegrüßt. Ich lebe im I{\geminationn}erſten der Stadt, wie ich in Wien\oindex{Wien@\textbf{Wien}, \emph{Verwaltungsgebiet}|pw} um keinen Preis leben möchte; an der Kreuzung vieler
               Straßen, mitten im Lärm der Geſchäfte u des Verkehrs. Der Zufall hat es gefügt, daſs
               ich gerade hier die Wohnung gefunden habe, wie ich{ }ſie brauche, und günſtige
               Verbindungen von Goldmann\pwindex{Goldmann, Paul 31.\,1.\,1865 Breslau – 25.\,9.\,1935 Wien@\textsc{Goldmann, Paul} (31.\,1.\,1865 Breslau – 25.\,9.\,1935 Wien), \emph{Schriftsteller, Journalist}|pw} haben{ }ſie mir
               verſchafft. Ich{ }ſage \uline{mir}, obwohl das nicht ganz richtig\pwindex{Reinhard, Marie 13.\,3.\,1871 Wien – 18.\,3.\,1899 ebd.@\textsc{Reinhard, Marie} (13.\,3.\,1871 Wien – 18.\,3.\,1899 ebd.), \emph{Gesangspädagogin}|pwv} iſt. Aber ich habe
               mein Zi{\geminationm}er allein u{ }ſo viel Freiheit, als unter den
               bekannten Umſtänden möglich iſt. Manchmal möcht ich wohl lieber ganz allein{ }ſein;
               aber vielleicht iſt {\pb}es nur die Sehnſucht nach der ich
               mich{ }ſehne. Ich bin nemlich bisher wirklich noch nie von Wien\oindex{Wien@\textbf{Wien}, \emph{Verwaltungsgebiet}|pw} fortgeweſen, ohne dort irgendwen zurück zu laſſen, um den
               ich mehr oder weniger »zittern« mußte; das geht mir vielleicht ab. Im ganzen aber
               fühl ich mich, wie Sie{ }ſagen würden »eher« wohl; insbeſondere tritt das{ }ſonderbare
               ein, was{ }ſich i{\geminationm}er beinah einſtellt, we{\geminationn} ich auf Reiſen, beſſer: we{\geminationn} ich nicht daheim bin; ich bin beinah gänzlich erlöſt von den Bangigkeiten und
               Hypochondrien, die mir das Leben zu Hauſe oft{ }ſo heftig{ }ſtören. Aber \introOben{}auch\introOben{} daſs ich gerade \uline{hier} bin,
               freut mich. Es iſt mir oft, als we{\geminationn} ich hier lieber
               leben möchte als in Wien\oindex{Wien@\textbf{Wien}, \emph{Verwaltungsgebiet}|pw}; aber das iſt
                  wahrſchein{\pb}lich ein Irrtum. Von allem, was ich hier{ }ſchon geſehn, möchte ich Ihnen lieber erſt in Wien\oindex{Wien@\textbf{Wien}, \emph{Verwaltungsgebiet}|pw}
               erzählen; denn ich frage mich vergeblich, was ich herausſuchen{ }ſollte. Das{ }ſchönſte
               hat mir bisher die Schauſpielerei geboten; es iſt einfach was andres als die
               Deutſchen haben; nicht immer was beſſres vielleicht – aber dem Weſen der Stücke, die{ }ſie{ }ſpielen, wunderbar verwandt, was ja{ }ſchließlich doch das wichtigſte iſt. Dramen{ }ſcheinen sie ja hier (wo denn???) auch nicht mehr zu{ }ſchreiben; ich habe \textsc{loi de l’homme\pwindex{Hervieu, Paul Ernest 2.\,9.\,1857 Neuilly-sur-Seine – 25.\,10.\,1915 Paris@\textsc{Hervieu, Paul Ernest} (2.\,9.\,1857 Neuilly-sur-Seine – 25.\,10.\,1915 Paris), \emph{Schriftsteller}!Männerrecht@\strich\emph{Männerrecht}|pw}, (Hervieu\pwindex{Hervieu, Paul Ernest 2.\,9.\,1857 Neuilly-sur-Seine – 25.\,10.\,1915 Paris@\textsc{Hervieu, Paul Ernest} (2.\,9.\,1857 Neuilly-sur-Seine – 25.\,10.\,1915 Paris), \emph{Schriftsteller}|pw}); Douloureuse\pwindex{Donnay, Maurice 12.\,10.\,1859 Paris – 31.\,3.\,1945 ebd.@\textsc{Donnay, Maurice} (12.\,10.\,1859 Paris – 31.\,3.\,1945 ebd.), \emph{Schriftsteller}!Douloureuse@\strich\emph{La Douloureuse}|pw}
                     (Donnay\pwindex{Donnay, Maurice 12.\,10.\,1859 Paris – 31.\,3.\,1945 ebd.@\textsc{Donnay, Maurice} (12.\,10.\,1859 Paris – 31.\,3.\,1945 ebd.), \emph{Schriftsteller}|pw}), – Carrière\pwindex{Hermant, Abel 3.\,2.\,1862 Paris – 28.\,9.\,1950@\textsc{Hermant, Abel} (3.\,2.\,1862 Paris – 28.\,9.\,1950), \emph{Schriftsteller}!Carrière@\strich\emph{La Carrière}|pw} (Hermant\pwindex{Hermant, Abel 3.\,2.\,1862 Paris – 28.\,9.\,1950@\textsc{Hermant, Abel} (3.\,2.\,1862 Paris – 28.\,9.\,1950), \emph{Schriftsteller}|pw}); –
                     Snob\pwindex{Guiches, Gustave 18.\,6.\,1860 – 3.\,8.\,1935 Paris@\textsc{Guiches, Gustave} (18.\,6.\,1860 – 3.\,8.\,1935 Paris), \emph{Schriftsteller}!Snob@\strich\emph{Snob}|pw} (Guiche\pwindex{Guiches, Gustave 18.\,6.\,1860 – 3.\,8.\,1935 Paris@\textsc{Guiches, Gustave} (18.\,6.\,1860 – 3.\,8.\,1935 Paris), \emph{Schriftsteller}|pw})} – geſehen – es iſt ein vollko{\geminationm}ener Sieg des Feuilletons auf dem Theater. Ich habe {\pb}wohl auch ein bischen das Gefühl des »Menſchenfreunds\pwindex{Raimund, Ferdinand 1.\,6.\,1790 Wien – 5.\,9.\,1836 Pottenstein@\textsc{Raimund, Ferdinand} (1.\,6.\,1790 Wien – 5.\,9.\,1836 Pottenstein), \emph{Schauspieler, Dramatiker}!Alpenkönig und der Menschenfeind@\strich\emph{Der Alpenkönig und der Menschenfeind}|pwv}« aus dem Raimund\pwindex{Raimund, Ferdinand 1.\,6.\,1790 Wien – 5.\,9.\,1836 Pottenstein@\textsc{Raimund, Ferdinand} (1.\,6.\,1790 Wien – 5.\,9.\,1836 Pottenstein), \emph{Schauspieler, Dramatiker}|pw}’ſchen Märchen gehabt, – aber können wir wirklichen Menſchen uns auch
               »beſſern«? Mit Bewußtſein entwickeln – das müßte wohl möglich{ }ſein! –\pend
           
\pstart
           – Sagen Sie mir ein Wort, wie es Ihnen und andren Leuten, von denen Sie gerade
               erzählen wollen (was mir jedenfalls erwünſcht wäre) geht. – Ich werde Ende Mai,{ }ſpäteſtens Anfang Juni wieder in Wien\oindex{Wien@\textbf{Wien}, \emph{Verwaltungsgebiet}|pw}{ }ſein. Das Wetter iſt nicht{ }ſchön; noch ke{\geminationn} ich eigentlich den Pariſ\oindex{Paris@\textbf{Paris}, \emph{Hauptstadt}|pw}er Frühling nicht.\pend
           
\pstart
           Grüßen Sie alle, die wir beide gern haben.\pend
           \pstart Herzlich grüßt Sie Ihr \spacefill\mbox{Arthur.}\pend{}
\pstart
           \noindent{}Auch Ihren Eltern\pwindex{Hofmannsthal, Hugo August von 21.\,12.\,1841 Wien – 8.\,12.\,1915 ebd.@\textsc{Hofmannsthal, Hugo August von} (21.\,12.\,1841 Wien – 8.\,12.\,1915 ebd.), \emph{Bankdirektor}|pwv}\pwindex{Hofmannsthal, Anna von 27.\,1.\,1849 Wien – 22.\,3.\,1904 Sanatorium Fürth@\textsc{Hofmannsthal, Anna von} (27.\,1.\,1849 Wien – 22.\,3.\,1904 Sanatorium Fürth)|pwv}, bitte, empfehlen Sie mich freundlich.\pend
           \selectlanguage{ngerman}\endnumbering\briefempfaengerindex{Hofmannsthal, Hugo von@\textsc{Hofmannsthal, Hugo von}!zzzSchnitzler, Arthur@\emph{von Arthur Schnitzler}!1897-04-262@{26. 4. 1897}|)be}\mylabel{L00671h}  \newcommand{\dateiname}{L00671}\newcommand{\titel}{Arthur Schnitzler an Hugo von Hofmannsthal, 26. 4. 1897}\newcommand{\editorInnen}{Martin Anton Müller und Gerd-Hermann Susen}%% latex-leseansicht-abspann.tex
%% Abspann für die Leseansicht.
%% Der Schalter \ifkorrekturansicht ist bereits durch den Vorspann gesetzt.

%% latex-abspann.tex
%% Gemeinsamer Abspann für Korrekturansicht und Leseansicht.
%% Setzt den Schalter \ifkorrekturansicht voraus (gesetzt in den
%% einbindenden Dateien latex-korrekturansicht-abspann.tex bzw.
%% latex-leseansicht-abspann.tex).
%% ---------------------------------------------------------------

\normalsize

% Das esempio-Environment wird nur in der Leseansicht benötigt
\ifkorrekturansicht\else
\newenvironment{esempio}[3]%
{
    \vspace{1.5ex}
    \rlap{\underline{#1}}
    \par
    \setlength{\parindent}{0cm}
    \nopagebreak
    \leftskip=#2cm
    \rightskip=#3cm
}
{
    \par
}
\fi

\doendnotes{C}
\bigskip
\vfill

\clearpage

\footnotesize

\ifkorrekturansicht
  \lohead{\textsc{register}}
\fi

% theindex-Environment neu definieren ohne reledmac
\makeatletter
\renewenvironment{theindex}{%
  \ifkorrekturansicht
    \section*{\indexname}%
  \else
    \subsubsection*{Index der erwähnten Entitäten}%
  \fi
  \setlength{\parindent}{0pt}%
  \setlength{\parskip}{0pt plus 0.3pt}%
  \let\item\@idxitem
}{%
  \ifkorrekturansicht\clearpage\fi
}
\makeatother

\IfFileExists{\jobname-pw.ind}{\input{\jobname-pw.ind}}{}

% Quellenangabe nur in der Leseansicht
\ifkorrekturansicht\else
% Fallback-Definitionen, falls die .tex-Datei \titel etc. nicht gesetzt hat
\providecommand{\titel}{}
\providecommand{\editorInnen}{}
\providecommand{\dateiname}{\jobname}

\vspace{3cm}

\vfill

\footnotesize
\textsc{Quelle}: \titel. Herausgegeben von {\editorInnen}. In: \emph{Arthur Schnitzler: Briefwechsel mit Autorinnen und Autoren}.
 Digitale Edition, https://schnitzler-briefe.acdh.oeaw.ac.at/{\dateiname}.html (Stand \today)
\fi

\end{document}


