\input{../tex-inputs/latex-pdf-vorspann}
\begin{center}
            \textcolor{red}{ENTWURF. ENTZIFFERUNG NOCH NICHT KORREKTURGELESEN}
                      \end{center}
            
               \section[Hugo von Hofmannsthal an Arthur Schnitzler, {[}19. 11. 1899?{]}]{ Hugo von Hofmannsthal an Arthur Schnitzler, {[}19. 11. 1899?{]}}\nopagebreak\mylabel{v}\rehead{ }\begin{ledgroupsized}[t]{13cm}\normalsize\beginnumbering\briefempfaengerindex{Schnitzler, Arthur@\textsc{Schnitzler, Arthur}!zzzHofmannsthal, Hugo von@\emph{von Hugo von Hofmannsthal}!1899-11-191@{{[}19. 11. 1899?{]}}|(be} \toendnotes[C]{\smallbreak\pagebreak[2]} \Standort{CUL, Schnitzler, B 43.}
\physDesc{Brief, 1 Blatt, 1 Seite
\newline{}Handschrift: Bleistift, deutsche Kurrent
\newline{}Schnitzler: mit Bleistift datiert: »\substVorne{}\textsuperscript{De}\substDazwischen{}Nov\substHinten{} 99.« \newline{}Ordnung: 1) mit Bleistift von unbekannter Hand nummeriert:
                                    »158« 2) mit Bleistift von unbekannter Hand nummeriert:
                                    »161«}\buchAbdrucke{\weitereDrucke{Hugo von Hofmannsthal, Arthur Schnitzler: \emph{Briefwechsel}. Hg. Therese Nickl und Heinrich Schnitzler. Frankfurt am Main: \emph{S. Fischer} 1964, S. 134.} }\toendnotes[C]{\smallbreak}\pstart{}{\pb}lieber\pend\pstart
           leider \label{K_L00998_1v}\edtext{treffe ich Sie \uline{nie}}{\lemma{\textnormal{\emph{treffe ich Sie nie}}}\Cendnote{\textnormal{Dieser Brief ist nur tentativ zu
                  datieren. Im von Schnitzler\pwindex{Schnitzler, Arthur 15.05.1862 – 21.10.1931@\textsc{Schnitzler, Arthur} (15.05.1862 – 21.10.1931), \emph{Schriftsteller, Mediziner}|pwk} angegebenen Monat
                  findet das erste Treffen zwischen den beiden am 26. 11. 1899 statt. Am 5. 11. 1899 ist Schnitzler\pwindex{Schnitzler, Arthur 15.05.1862 – 21.10.1931@\textsc{Schnitzler, Arthur} (15.05.1862 – 21.10.1931), \emph{Schriftsteller, Mediziner}|pwk} bei Beer-Hofmann\pwindex{Beer-Hofmann, Richard 11.07.1866 – 26.09.1945@\textsc{Beer-Hofmann, Richard} (11.07.1866 – 26.09.1945), \emph{Schriftsteller}|pwk} und
                  ärgert sich über Hofmannsthal\pwindex{Hofmannsthal, Hugo von 01.02.1874 – 15.07.1929@\textsc{Hofmannsthal, Hugo von} (01.02.1874 – 15.07.1929), \emph{Schriftsteller}|pwk}, was
                  möglicherweise auf ein Zusammentreffen verweist. Offenbar war zu dieser Zeit ein
                  regelmäßiges Treffen am Sonntag geplant, das Schnitzler\pwindex{Schnitzler, Arthur 15.05.1862 – 21.10.1931@\textsc{Schnitzler, Arthur} (15.05.1862 – 21.10.1931), \emph{Schriftsteller, Mediziner}|pwk} aber erst am Monatsende einhalten konnte. Verzichtet man
                  darauf, das »nie« als Übertreibung zu betrachten und einen gewissen Abstand
                  zwischen den Treffen anzunehmen, bliebe der Sonntag, 19. 11. 1899 als
                  mögliches Datum.}}}\label{K_L00998_1h}.\pend
           \pstart
           Heute abend ko{\geminationm}e ich \uline{eventuell} zu Richard\pwindex{Beer-Hofmann, Richard 11.07.1866 – 26.09.1945@\textsc{Beer-Hofmann, Richard} (11.07.1866 – 26.09.1945), \emph{Schriftsteller}|pw} doch nicht vor
                  ½ 9.\pend
           \pstart
           Herzlich{\\[\baselineskip]}\spacefill\mbox{Hugo}\pend
           \leftskip=0em{}\endnumbering\briefempfaengerindex{Schnitzler, Arthur@\textsc{Schnitzler, Arthur}!zzzHofmannsthal, Hugo von@\emph{von Hugo von Hofmannsthal}!1899-11-191@{{[}19. 11. 1899?{]}}|)be}\mylabel{h}\end{ledgroupsized}  \newcommand{\dateiname}{L00998}\newcommand{\titel}{Hugo von Hofmannsthal an Arthur Schnitzler, [19. 11. 1899?]}\newcommand{\editorInnen}{Martin Anton Müller und Gerd-Hermann Susen}\input{../tex-inputs/latex-pdf-abspann}
      