%% latex-leseansicht-vorspann.tex
%% Vorspann für die Leseansicht.
%% Lädt die gemeinsame Datei latex-vorspann.tex mit nicht gesetztem Schalter.

\newif\ifkorrekturansicht
\korrekturansichtfalse

\input{../tex-inputs/latex-vorspann}


         
         \renewcommand{\erwaehntePersonen}{Personen: Henri Albert, Leopold von Andrian-Werburg, Hermann Bahr, Richard Beer-Hofmann, Samuel Fischer, Hedwig Fischer, Johann Wolfgang von Goethe, Paul Goldmann, Hugo von Hofmannsthal, Leopold Sonnemann}
         \renewcommand{\erwaehnteInstitutionen}{Institutionen: Frankfurter Zeitung, S. Fischer Verlag}
         \renewcommand{\erwaehnteOrte}{Orte: Europa, Paris, Wien, rue Feydeau, rue Jacob}
         \renewcommand{\erwaehnteWerke}{Werke: Das Kind, Der Garten der Erkenntnis, Der Garten der Erkenntnis, Die Zeit. Wiener Wochenschrift, Le Courrier français, Les Jeunes Viennois, Revue des Revues, Sterben. Novelle}
               \section[Paul Goldmann an Arthur Schnitzler, 28. 3. {[}1895{]}]{ Paul Goldmann an Arthur Schnitzler, 28. 3. {[}1895{]}}\nopagebreak\mylabel{v}\rehead{ }\begin{ledgroupsized}[t]{13cm}\normalsize\beginnumbering\briefempfaengerindex{Schnitzler, Arthur@\textsc{Schnitzler, Arthur}!zzzGoldmann, Paul@\emph{von Paul Goldmann}!1895-03-281@{28. 3. {[}1895{]}}|(be} \toendnotes[C]{\smallbreak\pagebreak[2]} \Standort{DLA, A:Schnitzler, HS.NZ85.1.3165.}
\physDesc{Brief, 1 Blatt, 4 Seiten, 1467 Zeichen
\newline{}Handschrift: schwarze Tinte, deutsche Kurrent
\newline{}Schnitzler: 1) mit schwarzer Tinte das Jahr »95« vermerkt  2) mit rotem Buntstift drei Unterstreichungen}\toendnotes[C]{\smallbreak}\pstart
           \noindent{}{\pb}\textcolor{gray}{\textbf{\textbf{Frankfurter Zeitung\orgindex{Frankfurter Zeitung@Frankfurter Zeitung|pw}}}}\pend
           \pstart
           \textcolor{gray}{\textbf{(\begin{otherlanguage}{french}Gazette de Francfort\end{otherlanguage}\orgindex{Frankfurter Zeitung@Frankfurter Zeitung|pw}). }}\pend
           \pstart
           \textcolor{gray}{\textbf{\textbf{\begin{otherlanguage}{french}Fondateur M. L.
                                 Sonnemann\pwindex{Sonnemann, Leopold 1831-10-29 – 1909-10-30@\textsc{Sonnemann, Leopold} (1831-10-29 – 1909-10-30), \emph{Journalist, Herausgeber}|pw}\end{otherlanguage}.}}}\hfill \textsc{Paris\oindex{Paris@\textbf{Paris}|pw}}, 28. März.\pend
           \pstart
           \begin{otherlanguage}{french}\textcolor{gray}{\textbf{Journal politique, financier,}}\end{otherlanguage}\pend
           \pstart
           \begin{otherlanguage}{french}\textcolor{gray}{\textbf{commercial et littéraire.}}\end{otherlanguage}\pend
           \pstart
           \begin{otherlanguage}{french}\textcolor{gray}{\textbf{\textbf{Paraissant trois fois par jour.}}}\end{otherlanguage}\pend
           \pstart
           \begin{otherlanguage}{french}\textcolor{gray}{\textbf{\textbf{Bureau à Paris\oindex{Paris@\textbf{Paris}|pw}:}}}\end{otherlanguage}\pend
           \pstart
           \begin{otherlanguage}{french}\textcolor{gray}{\textbf{\textbf{24. Rue Feydeau\oindex{rue Feydeau@\textbf{rue Feydeau}|pw}.}}}\end{otherlanguage}\pend
           \pstart\center{}Mein lieber Freund,\pend\pstart
           \textsc{Henri Alberts\pwindex{Albert, Henri 1869-11-16 – 1921-08-03@\textsc{Albert, Henri} (1869-11-16 – 1921-08-03), \emph{Journalist, Kritiker, Übersetzer}|pw}}{ }\label{K_L02732-1v}\edtext{Artikel\pwindex{Albert, Henri 1869-11-16 – 1921-08-03@\textsc{Albert, Henri} (1869-11-16 – 1921-08-03), \emph{Journalist, Kritiker, Übersetzer}!Jeunes Viennois01. 04. 1895@\strich\emph{Les Jeunes Viennois} {[}01. 04. 1895{]}|pwv}}{\lemma{\textnormal{\emph{Artikel}}}\Cendnote{\textnormal{Henri Albert\pwindex{Albert, Henri 1869-11-16 – 1921-08-03@\textsc{Albert, Henri} (1869-11-16 – 1921-08-03), \emph{Journalist, Kritiker, Übersetzer}|pwk}: \emph{Les Jeunes Viennois}\pwindex{Albert, Henri 1869-11-16 – 1921-08-03@\textsc{Albert, Henri} (1869-11-16 – 1921-08-03), \emph{Journalist, Kritiker, Übersetzer}!Jeunes Viennois01. 04. 1895@\strich\emph{Les Jeunes Viennois} {[}01. 04. 1895{]}|pwk}. In: \emph{Revue des Revues}\pwindex{?? Werk@Nicht ermittelte Verfasserinnen und Verfasser!Revue des Revues1890 – 1903@\emph{Revue des Revues} {[}1890 – 1903{]}|pwk}, Bd. 13, 1. 4. 1895, S. 8–13.}}}\label{K_L02732-1h} erscheint morgen oder übermorgen in der »\textsc{Revue des Revues\pwindex{?? Werk@Nicht ermittelte Verfasserinnen und Verfasser!Revue des Revues1890 – 1903@\emph{Revue des Revues} {[}1890 – 1903{]}|pw}}«. Ich ſende Dir zwei \label{K_L02732-2v}\edtext{Bürſtenabzüge\pwindex{?? Werk@Nicht ermittelte Verfasserinnen und Verfasser!Revue des Revues1890 – 1903@\emph{Revue des Revues} {[}1890 – 1903{]}|pwv}}{\lemma{\textnormal{\emph{Bürſtenabzüge}}}\Cendnote{\textnormal{Einer der Bürstenabzüge findet sich in
                     Schnitzlers\pwindex{Schnitzler, Arthur 15.05.1862 – 21.10.1931@\textsc{Schnitzler, Arthur} (15.05.1862 – 21.10.1931), \emph{Schriftsteller, Mediziner}|pwk} Zeitungsausschnittsammlung
                  an der \emph{University of Exeter}, Box 37/1.}}}\label{K_L02732-2h}, einen für Dich, einen für \textsc{Richard\pwindex{Beer-Hofmann, Richard 1866-07-11 – 1945-09-26@\textsc{Beer-Hofmann, Richard} (1866-07-11 – 1945-09-26), \emph{Schriftsteller}|pw}}. Der Artikel\pwindex{Albert, Henri 1869-11-16 – 1921-08-03@\textsc{Albert, Henri} (1869-11-16 – 1921-08-03), \emph{Journalist, Kritiker, Übersetzer}!Jeunes Viennois01. 04. 1895@\strich\emph{Les Jeunes Viennois} {[}01. 04. 1895{]}|pwv} hat manche
               Fehler in Auffaſſung und Ausdruck. \textsc{Bahr\pwindex{Bahr, Hermann 19.07.1863 – 15.01.1934@\textsc{Bahr, Hermann} (19.07.1863 – 15.01.1934), \emph{Schriftsteller, Kritiker}|pw}} iſt zu ſehr herausgeſtrichen, Du zu wen\textcolor{gray}{i}g. Aber im Ganzen
               gefällt mir die kleine Abhandlung\pwindex{Albert, Henri 1869-11-16 – 1921-08-03@\textsc{Albert, Henri} (1869-11-16 – 1921-08-03), \emph{Journalist, Kritiker, Übersetzer}!Jeunes Viennois01. 04. 1895@\strich\emph{Les Jeunes Viennois} {[}01. 04. 1895{]}|pwv} und wird Dir wohl auch gefallen.\pend
           \pstart
           Über Deinen lieben ausführlichen Brief habe ich {\pb}mich ſehr gefreut. Ich danke Dir einſtweilen dafür und ſchreibe Dir nächſtens.\pend
           \pstart
           Schreib’, bitte, an \textsc{Henri Albert\pwindex{Albert, Henri 1869-11-16 – 1921-08-03@\textsc{Albert, Henri} (1869-11-16 – 1921-08-03), \emph{Journalist, Kritiker, Übersetzer}|pw}} (21. \textsc{Rue Jacob}\oindex{rue Jacob@\textbf{rue Jacob}|pw}) ein paar Zeilen des Dankes. Auch \textsc{Richard\pwindex{Beer-Hofmann, Richard 1866-07-11 – 1945-09-26@\textsc{Beer-Hofmann, Richard} (1866-07-11 – 1945-09-26), \emph{Schriftsteller}|pw}} ſoll das thun.\pend
           \pstart
           Schreib’ mir, ob Dir der Artikel\pwindex{Albert, Henri 1869-11-16 – 1921-08-03@\textsc{Albert, Henri} (1869-11-16 – 1921-08-03), \emph{Journalist, Kritiker, Übersetzer}!Jeunes Viennois01. 04. 1895@\strich\emph{Les Jeunes Viennois} {[}01. 04. 1895{]}|pwv} gefallen hat, ob ich Dir weiter Pariſ\oindex{Paris@\textbf{Paris}|pw}er Zeitungsartikel ſchicken ſoll, ob Ihr den \textsc{Courrier Français\pwindex{?? Werk@Nicht ermittelte Verfasserinnen und Verfasser!Le Courrier français1884 – 1909@\emph{Le Courrier français} {[}1884 – 1909{]}|pw}} bekommt? Die letzten beiden Fragen muß ich nun ſchon zum dritten Mal ſtellen.
               Oh! Oh! Oh!\pend
           \pstart
           {\pb}Bitte, bitte komm’ nach \textsc{Paris\oindex{Paris@\textbf{Paris}|pw}}!\pend
           \pstart
           Auch \textsc{Richard\pwindex{Beer-Hofmann, Richard 1866-07-11 – 1945-09-26@\textsc{Beer-Hofmann, Richard} (1866-07-11 – 1945-09-26), \emph{Schriftsteller}|pw}} ſoll kommen: es iſt Frühling hier und große Schönheit.\pend
           \pstart
           Über das Buch\pwindex{Garten der Erkenntnis1895@\emph{Der Garten der Erkenntnis} {[}1895{]}|pwv} von \textsc{Andrian\pwindex{Andrian-Werburg, Leopold von 09.05.1875 – 19.11.1951@\textsc{Andrian-Werburg, Leopold von} (09.05.1875 – 19.11.1951), \emph{Schriftsteller, Diplomat}|pw}} bin ich Zeile für Zeile und Wort für Wort Deiner Anſicht. Eine unreife Dilettant\pwindex{Andrian-Werburg, Leopold von 09.05.1875 – 19.11.1951@\textsc{Andrian-Werburg, Leopold von} (09.05.1875 – 19.11.1951), \emph{Schriftsteller, Diplomat}|pwv}en-Arbeit\pwindex{Garten der Erkenntnis1895@\emph{Der Garten der Erkenntnis} {[}1895{]}|pwv},
               mit viel Selbſtgefälligkeit, viel Unklarheit, viel Anempfindung \label{T_L02732-1v}\edtext{\substVorne{}\textsuperscript{.}\substDazwischen{}und einigen ſchönen Wendungen.\substHinten{}}{\lemma{\textnormal{\emph{und … Wendungen.}}}\Cendnote{\textnormal{Goldmann\pwindex{Goldmann, Paul 31.01.1865 – 25.09.1935@\textsc{Goldmann, Paul} (31.01.1865 – 25.09.1935), \emph{Schriftsteller, Journalist}|pwk} tilgte den Punkt am Ende des
                  Satzes nicht. Die Einfügung suggeriert jedoch den Willen, diesen zu
                  streichen.}}}\label{T_L02732-1h} Solche Sachen\pwindex{Garten der Erkenntnis1895@\emph{Der Garten der Erkenntnis} {[}1895{]}|pwv} läßt man in ſeinem Pult liegen und gibt ſie nicht als Buch heraus. Es
               gehört die ganze Urtheilslosigkeit und {\pb}Gewiſſensloſigkeit eines \textsc{Bahr\pwindex{Bahr, Hermann 19.07.1863 – 15.01.1934@\textsc{Bahr, Hermann} (19.07.1863 – 15.01.1934), \emph{Schriftsteller, Kritiker}|pw}} dazu, um das \label{K_L02732-3v}\edtext{als ein\strikeout{e}{ }Literatur-Ereigniß\pwindex{Garten der Erkenntnis1895@\emph{Der Garten der Erkenntnis} {[}1895{]}|pwv} zu
                  proklamiren}{\lemma{\textnormal{\emph{als … proklamiren}}}\Cendnote{\textnormal{Leopold von Andrian-Werburgs\pwindex{Andrian-Werburg, Leopold von 09.05.1875 – 19.11.1951@\textsc{Andrian-Werburg, Leopold von} (09.05.1875 – 19.11.1951), \emph{Schriftsteller, Diplomat}|pwk} Erzählung \emph{Der Garten der Erkenntnis}\pwindex{Garten der Erkenntnis1895@\emph{Der Garten der Erkenntnis} {[}1895{]}|pwk} hatte durch Bahr\pwindex{Bahr, Hermann 19.07.1863 – 15.01.1934@\textsc{Bahr, Hermann} (19.07.1863 – 15.01.1934), \emph{Schriftsteller, Kritiker}|pwk} einen Verleger\orgindex{S. Fischer Verlag@S. Fischer Verlag|pwkv} gefunden: dieser hatte Samuel Fischer\pwindex{Fischer, Samuel 24.12.1859 – 15.10.1934@\textsc{Fischer, Samuel} (24.12.1859 – 15.10.1934), \emph{Verleger}|pwk} am 25. 1. 1895 in einem Brief geschrieben, Andrians\pwindex{Andrian-Werburg, Leopold von 09.05.1875 – 19.11.1951@\textsc{Andrian-Werburg, Leopold von} (09.05.1875 – 19.11.1951), \emph{Schriftsteller, Diplomat}|pwk}{ }Text\pwindex{Garten der Erkenntnis1895@\emph{Der Garten der Erkenntnis} {[}1895{]}|pwkv}
                  wäre »das beste Werk\pwindex{Garten der Erkenntnis1895@\emph{Der Garten der Erkenntnis} {[}1895{]}|pwv}
                     nach meinem Urteile, was bisher die europ\oindex{Europa@\textbf{Europa}|pwv}äische Moderne hervorgebracht hat« (Samuel Fischer\pwindex{Fischer, Samuel 24.12.1859 – 15.10.1934@\textsc{Fischer, Samuel} (24.12.1859 – 15.10.1934), \emph{Verleger}|pwk}, Hedwig Fischer\pwindex{Fischer, Hedwig 08.09.1871 – 11.04.1952@\textsc{Fischer, Hedwig} (08.09.1871 – 11.04.1952)|pwk}: \emph{Briefwechsel mit
                        Autoren}. Herausgegeben von Dierk Rodewald und Corinna Fiedler. Mit
                     einer Einführung von Bernhard Zeller. Frankfurt am Main:
                        \emph{S. Fischer}{ }1989, S. 171–172). Anlässlich des Erscheinens
                  veröffentlichte Bahr\pwindex{Bahr, Hermann 19.07.1863 – 15.01.1934@\textsc{Bahr, Hermann} (19.07.1863 – 15.01.1934), \emph{Schriftsteller, Kritiker}|pwk} eine überschwängliche
                     Rezension\pwindex{Bahr, Hermann 19.07.1863 – 15.01.1934@\textsc{Bahr, Hermann} (19.07.1863 – 15.01.1934), \emph{Schriftsteller, Kritiker}!Garten der Erkenntnis1895-03-16@\strich\emph{Der Garten der Erkenntnis} {[}1895-03-16{]}|pwkv} in der \emph{Zeit}\pwindex{Zeit. Wiener Wochenschrift1894 – 1904@\emph{Die Zeit. Wiener Wochenschrift} {[}1894 – 1904{]}|pwk}: Hermann Bahr\pwindex{Bahr, Hermann 19.07.1863 – 15.01.1934@\textsc{Bahr, Hermann} (19.07.1863 – 15.01.1934), \emph{Schriftsteller, Kritiker}|pwk}: \emph{Der Garten der Erkenntnis}\pwindex{Bahr, Hermann 19.07.1863 – 15.01.1934@\textsc{Bahr, Hermann} (19.07.1863 – 15.01.1934), \emph{Schriftsteller, Kritiker}!Garten der Erkenntnis1895-03-16@\strich\emph{Der Garten der Erkenntnis} {[}1895-03-16{]}|pwk}. In: \emph{Die Zeit. Wiener Wochenschrift}\pwindex{Zeit. Wiener Wochenschrift1894 – 1904@\emph{Die Zeit. Wiener Wochenschrift} {[}1894 – 1904{]}|pwk}, Bd. 2, H. 24, 16. 3. 1895, S. 171–172. Schnitzler\pwindex{Schnitzler, Arthur 15.05.1862 – 21.10.1931@\textsc{Schnitzler, Arthur} (15.05.1862 – 21.10.1931), \emph{Schriftsteller, Mediziner}|pwk} las das Werk\pwindex{Garten der Erkenntnis1895@\emph{Der Garten der Erkenntnis} {[}1895{]}|pwkv} am Tag nach Erscheinen
                  dieser Rezension\pwindex{Bahr, Hermann 19.07.1863 – 15.01.1934@\textsc{Bahr, Hermann} (19.07.1863 – 15.01.1934), \emph{Schriftsteller, Kritiker}!Garten der Erkenntnis1895-03-16@\strich\emph{Der Garten der Erkenntnis} {[}1895-03-16{]}|pwkv}, am 17. 3. 1895, und
                  notierte: »Spuren eines Künstlers, schöne Vergleiche. – Keine
                     Gestaltung, Affectation, Unklarheiten, – unreifer Loris\pwindex{Hofmannsthal, Hugo von 1874-02-01 – 1929-07-15@\textsc{Hofmannsthal, Hugo von} (1874-02-01 – 1929-07-15), \emph{Schriftsteller}|pw} – nicht reifer Goethe\pwindex{Goethe, Johann Wolfgang von 1749-08-28 – 1832-03-22@\textsc{Goethe, Johann Wolfgang von} (1749-08-28 – 1832-03-22), \emph{Schriftsteller}|pw}, wie Bahr\pwindex{Bahr, Hermann 19.07.1863 – 15.01.1934@\textsc{Bahr, Hermann} (19.07.1863 – 15.01.1934), \emph{Schriftsteller, Kritiker}|pw} sagte. – Es mit
                        ›Kind\pwindex{Beer-Hofmann, Richard 1866-07-11 – 1945-09-26@\textsc{Beer-Hofmann, Richard} (1866-07-11 – 1945-09-26), \emph{Schriftsteller}!Kind1893@\strich\emph{Das Kind} {[}1893{]}|pw}‹ oder ›Sterben\pwindex{Schnitzler, Arthur 15.05.1862 – 21.10.1931@\textsc{Schnitzler, Arthur} (15.05.1862 – 21.10.1931), \emph{Schriftsteller, Mediziner}!Sterben. Novelle1894-10-01 – 1894-12-01@\strich\emph{Sterben. Novelle} {[}1894-10-01 – 1894-12-01{]}|pw}‹ vergleichen ist dumm und
                  frech.«}}}\label{K_L02732-3h}! Welch\substVorne{}\textsuperscript{e}\substDazwischen{}’\substHinten{} ein Verderber\pwindex{Bahr, Hermann 19.07.1863 – 15.01.1934@\textsc{Bahr, Hermann} (19.07.1863 – 15.01.1934), \emph{Schriftsteller, Kritiker}|pwv} von
               Geſchmack und Talent!\pend
           \pstart
           Aber nein, ich habe \strikeout{keine} ja keine Zeit, Dir heut zu ſchreiben.\pend
           \pstart
           Auf nächſtens alſo!\pend
           \pstart
           Grüß’ Dich Gott! {\\[\baselineskip]}Dein treuer {\\[\baselineskip]}\spacefill\mbox{Paul Goldmann.}\pend
           \leftskip=0em{}
         
         \endnumbering\mylabel{h}\end{ledgroupsized}  \newcommand{\dateiname}{L02732}\newcommand{\titel}{Paul Goldmann an Arthur Schnitzler, 28. 3. [1895]}\newcommand{\editorInnen}{Martin Anton Müller und Laura Untner}%% latex-leseansicht-abspann.tex
%% Abspann für die Leseansicht.
%% Der Schalter \ifkorrekturansicht ist bereits durch den Vorspann gesetzt.

%% latex-abspann.tex
%% Gemeinsamer Abspann für Korrekturansicht und Leseansicht.
%% Setzt den Schalter \ifkorrekturansicht voraus (gesetzt in den
%% einbindenden Dateien latex-korrekturansicht-abspann.tex bzw.
%% latex-leseansicht-abspann.tex).
%% ---------------------------------------------------------------

\normalsize

% Das esempio-Environment wird nur in der Leseansicht benötigt
\ifkorrekturansicht\else
\newenvironment{esempio}[3]%
{
    \vspace{1.5ex}
    \rlap{\underline{#1}}
    \par
    \setlength{\parindent}{0cm}
    \nopagebreak
    \leftskip=#2cm
    \rightskip=#3cm
}
{
    \par
}
\fi

\doendnotes{C}
\bigskip
\vfill

\clearpage

\footnotesize

\ifkorrekturansicht
  \lohead{\textsc{register}}
\fi

% theindex-Environment neu definieren ohne reledmac
\makeatletter
\renewenvironment{theindex}{%
  \ifkorrekturansicht
    \section*{\indexname}%
  \else
    \subsubsection*{Index der erwähnten Entitäten}%
  \fi
  \setlength{\parindent}{0pt}%
  \setlength{\parskip}{0pt plus 0.3pt}%
  \let\item\@idxitem
}{%
  \ifkorrekturansicht\clearpage\fi
}
\makeatother

\IfFileExists{\jobname-pw.ind}{\input{\jobname-pw.ind}}{}

% Quellenangabe nur in der Leseansicht
\ifkorrekturansicht\else
% Fallback-Definitionen, falls die .tex-Datei \titel etc. nicht gesetzt hat
\providecommand{\titel}{}
\providecommand{\editorInnen}{}
\providecommand{\dateiname}{\jobname}

\vspace{3cm}

\vfill

\footnotesize
\textsc{Quelle}: \titel. Herausgegeben von {\editorInnen}. In: \emph{Arthur Schnitzler: Briefwechsel mit Autorinnen und Autoren}.
 Digitale Edition, https://schnitzler-briefe.acdh.oeaw.ac.at/{\dateiname}.html (Stand \today)
\fi

\end{document}


      