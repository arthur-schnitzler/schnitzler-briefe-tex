%% latex-korrekturansicht-vorspann.tex
%% Vorspann für die Korrekturansicht.
%% Lädt die gemeinsame Datei latex-vorspann.tex mit gesetztem Schalter.

\newif\ifkorrekturansicht
\korrekturansichttrue

\input{../tex-inputs/latex-vorspann}


\section[Paul Goldmann an Arthur Schnitzler, 28. 3. {[}1895{]}]{L02732 Paul Goldmann an Arthur Schnitzler, 28. 3. {[}1895{]}}
\nopagebreak\mylabel{L02732v}
\rehead{ }\normalsize\beginnumbering\briefempfaengerindex{Schnitzler, Arthur@\textsc{Schnitzler, Arthur}!zzzGoldmann, Paul@\emph{von Paul Goldmann}!1895-03-281@{28. 3. {[}1895{]}}|(be}
\toendnotes[C]{\smallbreak\pagebreak[2]}\Standort{DLA, A:Schnitzler, HS.NZ85.1.3165.}
\physDesc{Brief, 1 Blatt, 4 Seiten, 1467 Zeichen
\newline{}Handschrift: schwarze Tinte, deutsche Kurrent
\newline{}Schnitzler: 1) mit schwarzer Tinte das Jahr »95« vermerkt  2) mit rotem Buntstift drei Unterstreichungen}\toendnotes[C]{\smallbreak}
\pstart
           {\pb}\textcolor{gray}{\textbf{\textbf{Frankfurter Zeitung\orgindex{Frankfurter Zeitung@Frankfurter Zeitung|pw}}}}\pend
           
\pstart
           \textcolor{gray}{\textbf{(\begin{otherlanguage}{french}Gazette de Francfort\end{otherlanguage}\orgindex{Frankfurter Zeitung@Frankfurter Zeitung|pw}). }}\pend
           
\pstart
           \textcolor{gray}{\textbf{\textbf{\begin{otherlanguage}{french}Fondateur M. L.
                                 Sonnemann\pwindex{Sonnemann, Leopold 1831-10-29 – 1909-10-30@\textsc{Sonnemann, Leopold} (1831-10-29 – 1909-10-30), \emph{Journalist/Journalistin, Herausgeber/Herausgeberin}|pw}\end{otherlanguage}.}}}\hfill \textsc{Paris\oindex{Paris@\textbf{Paris}, \emph{P.PPLC}|pw}}, 28. März.\pend
           
\pstart
           \begin{otherlanguage}{french}\textcolor{gray}{\textbf{Journal politique, financier,}}\end{otherlanguage}\pend
           
\pstart
           \begin{otherlanguage}{french}\textcolor{gray}{\textbf{commercial et littéraire.}}\end{otherlanguage}\pend
           
\pstart
           \begin{otherlanguage}{french}\textcolor{gray}{\textbf{\textbf{Paraissant trois fois par jour.}}}\end{otherlanguage}\pend
           
\pstart
           \begin{otherlanguage}{french}\textcolor{gray}{\textbf{\textbf{Bureau à Paris\oindex{Paris@\textbf{Paris}, \emph{P.PPLC}|pw}:}}}\end{otherlanguage}\pend
           
\pstart
           \begin{otherlanguage}{french}\textcolor{gray}{\textbf{\textbf{24. Rue Feydeau\oindex{rue Feydeau@\textbf{rue Feydeau}, \emph{Straße (K.STR)}|pw}.}}}\end{otherlanguage}\pend
           
\pstart\center{}Mein lieber Freund,\pend\vspace{0.5em}
\pstart
           \textsc{Henri Alberts\pwindex{Albert, Henri 1869-11-16 – 1921-08-03@\textsc{Albert, Henri} (1869-11-16 – 1921-08-03), \emph{Journalist/Journalistin, Kritiker/Kritikerin, Übersetzer/Übersetzerin}|pw}}{ }\label{K_L02732-1v}\edtext{Artikel\pwindex{Jeunes Viennois@\emph{Les Jeunes Viennois}|pwv}}{\lemma{\textnormal{\emph{Artikel}}}\Cendnote{\textnormal{Henri Albert\pwindex{Albert, Henri 1869-11-16 – 1921-08-03@\textsc{Albert, Henri} (1869-11-16 – 1921-08-03), \emph{Journalist/Journalistin, Kritiker/Kritikerin, Übersetzer/Übersetzerin}|pwk}: \emph{Les Jeunes Viennois}\pwindex{Jeunes Viennois@\emph{Les Jeunes Viennois}|pwk}. In: \emph{Revue des Revues}\pwindex{Revue des Revues@\emph{Revue des Revues}|pwk}, Bd. 13, 1. 4. 1895, S. 8–13.}}}\label{K_L02732-1} erscheint morgen oder übermorgen in der »\textsc{Revue des Revues\pwindex{Revue des Revues@\emph{Revue des Revues}|pw}}«. Ich ſende Dir zwei \label{K_L02732-2v}\edtext{Bürſtenabzüge\pwindex{Revue des Revues@\emph{Revue des Revues}|pwv}}{\lemma{\textnormal{\emph{Bürſtenabzüge}}}\Cendnote{\textnormal{Einer der Bürstenabzüge findet sich in
                     Schnitzlers Zeitungsausschnittsammlung
                  an der \emph{University of Exeter}, Box 37/1.}}}\label{K_L02732-2}, einen für Dich, einen für \textsc{Richard\pwindex{Beer-Hofmann, Richard 1866-07-11 – 1945-09-26@\textsc{Beer-Hofmann, Richard} (1866-07-11 – 1945-09-26), \emph{Schriftsteller/Schriftstellerin}|pw}}. Der Artikel\pwindex{Jeunes Viennois@\emph{Les Jeunes Viennois}|pwv} hat manche
               Fehler in Auffaſſung und Ausdruck. \textsc{Bahr\pwindex{Bahr, Hermann 19.07.1863 – 15.01.1934@\textsc{Bahr, Hermann} (19.07.1863 – 15.01.1934), \emph{Schriftsteller/Schriftstellerin, Kritiker/Kritikerin}|pw}} iſt zu ſehr herausgeſtrichen, Du zu wen\textcolor{gray}{i}g. Aber im Ganzen
               gefällt mir die kleine Abhandlung\pwindex{Jeunes Viennois@\emph{Les Jeunes Viennois}|pwv} und wird Dir wohl auch gefallen.\pend
           
\pstart
           Über Deinen lieben ausführlichen Brief habe ich {\pb}mich ſehr gefreut. Ich danke Dir einſtweilen dafür und ſchreibe Dir nächſtens.\pend
           
\pstart
           Schreib’, bitte, an \textsc{Henri Albert\pwindex{Albert, Henri 1869-11-16 – 1921-08-03@\textsc{Albert, Henri} (1869-11-16 – 1921-08-03), \emph{Journalist/Journalistin, Kritiker/Kritikerin, Übersetzer/Übersetzerin}|pw}} (21. \textsc{Rue Jacob}\oindex{rue Jacob@\textbf{rue Jacob}, \emph{Straße (K.STR)}|pw}) ein paar Zeilen des Dankes. Auch \textsc{Richard\pwindex{Beer-Hofmann, Richard 1866-07-11 – 1945-09-26@\textsc{Beer-Hofmann, Richard} (1866-07-11 – 1945-09-26), \emph{Schriftsteller/Schriftstellerin}|pw}} ſoll das thun.\pend
           
\pstart
           Schreib’ mir, ob Dir der Artikel\pwindex{Jeunes Viennois@\emph{Les Jeunes Viennois}|pwv} gefallen hat, ob ich Dir weiter Pariſ\oindex{Paris@\textbf{Paris}, \emph{P.PPLC}|pw}er Zeitungsartikel ſchicken ſoll, ob Ihr den \textsc{Courrier Français\pwindex{Le Courrier français@\emph{Le Courrier français}|pw}} bekommt? Die letzten beiden Fragen muß ich nun ſchon zum dritten Mal ſtellen.
               Oh! Oh! Oh!\pend
           
\pstart
           {\pb}Bitte, bitte komm’ nach \textsc{Paris\oindex{Paris@\textbf{Paris}, \emph{P.PPLC}|pw}}!\pend
           
\pstart
           Auch \textsc{Richard\pwindex{Beer-Hofmann, Richard 1866-07-11 – 1945-09-26@\textsc{Beer-Hofmann, Richard} (1866-07-11 – 1945-09-26), \emph{Schriftsteller/Schriftstellerin}|pw}} ſoll kommen: es iſt Frühling hier und große Schönheit.\pend
           
\pstart
           Über das Buch\pwindex{Garten der Erkenntnis@\emph{Der Garten der Erkenntnis}|pwv} von \textsc{Andrian\pwindex{Andrian-Werburg, Leopold von 09.05.1875 – 19.11.1951@\textsc{Andrian-Werburg, Leopold von} (09.05.1875 – 19.11.1951), \emph{Schriftsteller/Schriftstellerin, Diplomat/Diplomatin}|pw}} bin ich Zeile für Zeile und Wort für Wort Deiner Anſicht. Eine unreife Dilettant\pwindex{Andrian-Werburg, Leopold von 09.05.1875 – 19.11.1951@\textsc{Andrian-Werburg, Leopold von} (09.05.1875 – 19.11.1951), \emph{Schriftsteller/Schriftstellerin, Diplomat/Diplomatin}|pwv}en-Arbeit\pwindex{Garten der Erkenntnis@\emph{Der Garten der Erkenntnis}|pwv},
               mit viel Selbſtgefälligkeit, viel Unklarheit, viel Anempfindung \label{T_L02732-1v}\edtext{\substVorne{}\textsuperscript{.}\substDazwischen{}und einigen ſchönen Wendungen.\substHinten{}}{\lemma{\textnormal{\emph{und … Wendungen.}}}\Cendnote{\textnormal{Goldmann\pwindex{Goldmann, Paul 31.01.1865 – 25.09.1935@\textsc{Goldmann, Paul} (31.01.1865 – 25.09.1935), \emph{Schriftsteller/Schriftstellerin, Journalist/Journalistin}|pwk} tilgte den Punkt am Ende des
                  Satzes nicht. Die Einfügung suggeriert jedoch den Willen, diesen zu
                  streichen.}}}\label{T_L02732-1} Solche Sachen\pwindex{Garten der Erkenntnis@\emph{Der Garten der Erkenntnis}|pwv} läßt man in ſeinem Pult liegen und gibt ſie nicht als Buch heraus. Es
               gehört die ganze Urtheilslosigkeit und {\pb}Gewiſſensloſigkeit eines \textsc{Bahr\pwindex{Bahr, Hermann 19.07.1863 – 15.01.1934@\textsc{Bahr, Hermann} (19.07.1863 – 15.01.1934), \emph{Schriftsteller/Schriftstellerin, Kritiker/Kritikerin}|pw}} dazu, um das \label{K_L02732-3v}\edtext{als ein\strikeout{e}{ }Literatur-Ereigniß\pwindex{Garten der Erkenntnis@\emph{Der Garten der Erkenntnis}|pwv} zu
                  proklamiren}{\lemma{\textnormal{\emph{als … proklamiren}}}\Cendnote{\textnormal{Leopold von Andrian-Werburgs\pwindex{Andrian-Werburg, Leopold von 09.05.1875 – 19.11.1951@\textsc{Andrian-Werburg, Leopold von} (09.05.1875 – 19.11.1951), \emph{Schriftsteller/Schriftstellerin, Diplomat/Diplomatin}|pwk} Erzählung \emph{Der Garten der Erkenntnis}\pwindex{Garten der Erkenntnis@\emph{Der Garten der Erkenntnis}|pwk} hatte durch Bahr\pwindex{Bahr, Hermann 19.07.1863 – 15.01.1934@\textsc{Bahr, Hermann} (19.07.1863 – 15.01.1934), \emph{Schriftsteller/Schriftstellerin, Kritiker/Kritikerin}|pwk} einen Verleger\orgindex{S. Fischer Verlag@S. Fischer Verlag|pwkv} gefunden: dieser hatte Samuel Fischer\pwindex{Fischer, Samuel 24.12.1859 – 15.10.1934@\textsc{Fischer, Samuel} (24.12.1859 – 15.10.1934), \emph{Verleger/Verlegerin}|pwk} am 25. 1. 1895 in einem Brief geschrieben, Andrians\pwindex{Andrian-Werburg, Leopold von 09.05.1875 – 19.11.1951@\textsc{Andrian-Werburg, Leopold von} (09.05.1875 – 19.11.1951), \emph{Schriftsteller/Schriftstellerin, Diplomat/Diplomatin}|pwk}{ }Text\pwindex{Garten der Erkenntnis@\emph{Der Garten der Erkenntnis}|pwkv}
                  wäre »das beste Werk\pwindex{Garten der Erkenntnis@\emph{Der Garten der Erkenntnis}|pwv}
                     nach meinem Urteile, was bisher die europ\oindex{Europa@\textbf{Europa}, \emph{Kontinent (A.KNT)}|pwv}äische Moderne hervorgebracht hat« (Samuel Fischer\pwindex{Fischer, Samuel 24.12.1859 – 15.10.1934@\textsc{Fischer, Samuel} (24.12.1859 – 15.10.1934), \emph{Verleger/Verlegerin}|pwk}, Hedwig Fischer\pwindex{Fischer, Hedwig 08.09.1871 – 11.04.1952@\textsc{Fischer, Hedwig} (08.09.1871 – 11.04.1952)|pwk}: \emph{Briefwechsel mit
                        Autoren}. Herausgegeben von Dierk Rodewald und Corinna Fiedler. Mit
                     einer Einführung von Bernhard Zeller. Frankfurt am Main:
                        \emph{S. Fischer}{ }1989, S. 171–172). Anlässlich des Erscheinens
                  veröffentlichte Bahr\pwindex{Bahr, Hermann 19.07.1863 – 15.01.1934@\textsc{Bahr, Hermann} (19.07.1863 – 15.01.1934), \emph{Schriftsteller/Schriftstellerin, Kritiker/Kritikerin}|pwk} eine überschwängliche
                     Rezension\pwindex{Garten der Erkenntnis@\emph{Der Garten der Erkenntnis}|pwkv} in der \emph{Zeit}\pwindex{Zeit. Wiener Wochenschrift@\emph{Die Zeit. Wiener Wochenschrift}|pwk}: Hermann Bahr\pwindex{Bahr, Hermann 19.07.1863 – 15.01.1934@\textsc{Bahr, Hermann} (19.07.1863 – 15.01.1934), \emph{Schriftsteller/Schriftstellerin, Kritiker/Kritikerin}|pwk}: \emph{Der Garten der Erkenntnis}\pwindex{Garten der Erkenntnis@\emph{Der Garten der Erkenntnis}|pwk}. In: \emph{Die Zeit. Wiener Wochenschrift}\pwindex{Zeit. Wiener Wochenschrift@\emph{Die Zeit. Wiener Wochenschrift}|pwk}, Bd. 2, H. 24, 16. 3. 1895, S. 171–172. Schnitzler las das Werk\pwindex{Garten der Erkenntnis@\emph{Der Garten der Erkenntnis}|pwkv} am Tag nach Erscheinen
                  dieser Rezension\pwindex{Garten der Erkenntnis@\emph{Der Garten der Erkenntnis}|pwkv}, am 17. 3. 1895, und
                  notierte: »Spuren eines Künstlers, schöne Vergleiche. – Keine
                     Gestaltung, Affectation, Unklarheiten, – unreifer Loris\pwindex{Hofmannsthal, Hugo von 1874-02-01 – 1929-07-15@\textsc{Hofmannsthal, Hugo von} (1874-02-01 – 1929-07-15), \emph{Schriftsteller/Schriftstellerin}|pw} – nicht reifer Goethe\pwindex{Goethe, Johann Wolfgang von 1749-08-28 – 1832-03-22@\textsc{Goethe, Johann Wolfgang von} (1749-08-28 – 1832-03-22), \emph{Schriftsteller/Schriftstellerin}|pw}, wie Bahr\pwindex{Bahr, Hermann 19.07.1863 – 15.01.1934@\textsc{Bahr, Hermann} (19.07.1863 – 15.01.1934), \emph{Schriftsteller/Schriftstellerin, Kritiker/Kritikerin}|pw} sagte. – Es mit
                        ›Kind\pwindex{Kind@\emph{Das Kind}|pw}‹ oder ›Sterben\pwindex{Sterben. Novelle@\emph{Sterben. Novelle}|pw}‹ vergleichen ist dumm und
                  frech.«}}}\label{K_L02732-3}! Welch\substVorne{}\textsuperscript{e}\substDazwischen{}’\substHinten{} ein Verderber\pwindex{Bahr, Hermann 19.07.1863 – 15.01.1934@\textsc{Bahr, Hermann} (19.07.1863 – 15.01.1934), \emph{Schriftsteller/Schriftstellerin, Kritiker/Kritikerin}|pwv} von
               Geſchmack und Talent!\pend
           
\pstart
           Aber nein, ich habe \strikeout{keine} ja keine Zeit, Dir heut zu ſchreiben.\pend
           
\pstart
           Auf nächſtens alſo!\pend
           
\pstart
           Grüß’ Dich Gott! {\\[\baselineskip]}Dein treuer {\\[\baselineskip]}\spacefill\mbox{Paul Goldmann.}\pend
           \leftskip=0em{}\selectlanguage{ngerman}\endnumbering\briefempfaengerindex{Schnitzler, Arthur@\textsc{Schnitzler, Arthur}!zzzGoldmann, Paul@\emph{von Paul Goldmann}!1895-03-281@{28. 3. {[}1895{]}}|)be}\mylabel{L02732h}  \normalsize

\doendnotes{C}
\bigskip
\vfill

\clearpage

\footnotesize

\lohead{\textsc{register}}

% Definiere theindex-Environment komplett neu ohne reledmac
\makeatletter
\renewenvironment{theindex}{%
  \section*{\indexname}%
  \setlength{\parindent}{0pt}%
  \setlength{\parskip}{0pt plus 0.3pt}%
  \let\item\@idxitem
}{%
  \clearpage
}
\makeatother

\IfFileExists{\jobname-pw.ind}{\input{\jobname-pw.ind}}{}

\end{document}

      