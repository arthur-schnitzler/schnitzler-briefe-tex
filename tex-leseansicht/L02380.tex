%% latex-korrekturansicht-vorspann.tex
%% Vorspann für die Korrekturansicht.
%% Lädt die gemeinsame Datei latex-vorspann.tex mit gesetztem Schalter.

\newif\ifkorrekturansicht
\korrekturansichttrue

\input{../tex-inputs/latex-vorspann}


\section[Robert Adam an Arthur Schnitzler, 13. 5. 1922]{L02380 Robert Adam an Arthur Schnitzler, 13. 5. 1922}
\nopagebreak\mylabel{L02380v}
\rehead{ }\normalsize\beginnumbering\briefempfaengerindex{Schnitzler, Arthur@\textsc{Schnitzler, Arthur}!zzzAdam, Robert@\emph{von Robert Adam}!1922-05-131@{13. 5. 1922}|(be}
\toendnotes[C]{\smallbreak\pagebreak[2]}\Standort{Wien, Österreichische Gesellschaft für Literatur, Kopienarchiv Schnitzler, Adam.}
\physDesc{Brief, Fotokopie1 Blatt, 2 Seiten, 568 Zeichen
\newline{}Handschrift: schwarze Tinte, deutsche Kurrent
\newline{}Schnitzler: mit Bleistift beschriftet: »\textsc{Adam}« }
\pstart
           \raggedleft{}{\pb}\textsc{Millstatt\oindex{Millstatt@\textbf{Millstatt}, \emph{A.ADM3}|pw}}, den 13. \textsc{Mai} 1922\pend
           
\pstart\center{}Hochverehrter Herr Doktor!\pend\vspace{0.5em}
\pstart
           Geſtatten Sie mir, mich mit dieſen Zeilen dem Reigen der Gratulanten anzuſchließen,
               der Ihren kommenden 60. Geburtstag als einen für Öſterreich\oindex{Oesterreich@\textbf{Österreich}, \emph{A.PCLI}|pw} und die deutſche Literatur freudigen Werktag feiert. Wie ſehr ich
               Ihre Arbeiten ſchätze, brauche ich Ihnen bei dieſem Anlaß wohl nicht zu wiederholen.
               Möge es Ihnen vergönnt ſein, {\pb}noch viele Jahre
               hindurch Ihr Weſen in Werken auszuſchöpfen und uns die ſüße Reife Ihrer Kunſt
               genießen zu laſſen.\pend
           
\pstart
           Nehmen Sie meine herzlichſten Grüße und Empfehlungen entgegen!\pend
           
\pstart
           Ihr ergebener{\\[\baselineskip]}\spacefill\mbox{D\textsuperscript{r}RAdam}\pend
           \leftskip=0em{}\selectlanguage{ngerman}\endnumbering\briefempfaengerindex{Schnitzler, Arthur@\textsc{Schnitzler, Arthur}!zzzAdam, Robert@\emph{von Robert Adam}!1922-05-131@{13. 5. 1922}|)be}\mylabel{L02380h}  \normalsize

\doendnotes{C}
\bigskip
\vfill

\clearpage

\footnotesize

\lohead{\textsc{register}}

% Definiere theindex-Environment komplett neu ohne reledmac
\makeatletter
\renewenvironment{theindex}{%
  \section*{\indexname}%
  \setlength{\parindent}{0pt}%
  \setlength{\parskip}{0pt plus 0.3pt}%
  \let\item\@idxitem
}{%
  \clearpage
}
\makeatother

\IfFileExists{\jobname-pw.ind}{\input{\jobname-pw.ind}}{}

\end{document}

      