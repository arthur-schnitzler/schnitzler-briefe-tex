%% latex-leseansicht-vorspann.tex
%% Vorspann für die Leseansicht.
%% Lädt die gemeinsame Datei latex-vorspann.tex mit nicht gesetztem Schalter.

\newif\ifkorrekturansicht
\korrekturansichtfalse

\input{../tex-inputs/latex-vorspann}


\section[ Arthur Schnitzler an Felix Salten, 7. 11. 1907]{L03010 Arthur Schnitzler an Felix Salten,  7. 11. 1907}
\nopagebreak\mylabel{L03010v}
\rehead{ }\normalsize\beginnumbering\briefempfaengerindex{Salten, Felix@\textsc{Salten, Felix}!zzzSchnitzler, Arthur@\emph{von Arthur Schnitzler}!1907-11-071@{7. 11. 1907}|(be}
\toendnotes[C]{\smallbreak\pagebreak[2]}
\correspDesc{Versand  durch Arthur Schnitzler am 7. 11. 1907 in Wien
\newline{}Erhalt  durch Felix Salten im Zeitraum [7. 11. 1907
                  – 10. 11. 1907?] in Wien}\toendnotes[C]{\smallbreak}
\Standort{Wienbibliothek im Rathaus, ZPH 1681, 2.1.516.}
\physDesc{Brief, 1 Blatt, 1 Seite, 390 Zeichen
\newline{}Handschrift: schwarze Tinte, lateinische Kurrent
\newline{}Ordnung: mit Bleistift von unbekannter Hand nummeriert: »7« }\toendnotes[C]{\smallbreak}
\pstart
           \raggedleft{}{\pb}7. 11. 907\pend
           
\pstart{}lieber,\pend\vspace{0.5em}
\pstart
           Kainz\pwindex{Kainz, Josef 2.\,1.\,1858 Mosonmagyaróvár – 20.\,9.\,1910 Wien@\textsc{Kainz, Josef} (2.\,1.\,1858 Mosonmagyaróvár – 20.\,9.\,1910 Wien), \emph{Schauspieler}|pw} spielt am Samstag den Fiesco\pwindex{\textcolor{red}{\textsuperscript{XXXX indx1}}!Verschwörung des Fiesko zu Genua@\strich\emph{Die Verschwörung des Fiesko zu Genua}|pw}, Frau Kainz\pwindex{Kainz, Margarethe 31.\,12.\,1885 Berlin – 12.\,2.\,1950 Wien@\textsc{Kainz, Margarethe} (31.\,12.\,1885 Berlin – 12.\,2.\,1950 Wien), \emph{Schauspielerin}|pw} ist bei Ihrer \label{K_L03010-1v}\edtext{Première\pwindex{Salten, Felix 6.\,9.\,1869 Budapest – 8.\,10.\,1945 Zürich@\textsc{Salten, Felix} (6.\,9.\,1869 Budapest – 8.\,10.\,1945 Zürich), \emph{Schriftsteller, Journalist, Chefredakteur}!Vom andern Ufer. Einakter@\strich\emph{Vom andern Ufer. Einakter}|pwv}}{\lemma{\textnormal{\emph{Première}}}\Cendnote{\textnormal{Die Premiere von \emph{Vom
                  andern Ufer}\pwindex{Salten, Felix 6.\,9.\,1869 Budapest – 8.\,10.\,1945 Zürich@\textsc{Salten, Felix} (6.\,9.\,1869 Budapest – 8.\,10.\,1945 Zürich), \emph{Schriftsteller, Journalist, Chefredakteur}!Vom andern Ufer. Einakter@\strich\emph{Vom andern Ufer. Einakter}|pwk} fand 9. 11. 1907 am \emph{Volkstheater}\orgindex{Volkstheater@Volkstheater|pwk}
                  statt. Schnitzler nahm teil.}}}\label{K_L03010-1}, geht aber dann zu Fiesco\pwindex{\textcolor{red}{\textsuperscript{XXXX indx1}}!Verschwörung des Fiesko zu Genua@\strich\emph{Die Verschwörung des Fiesko zu Genua}|pw} hinüber, so
               daß sie wohl beide \introOben{}nachher\introOben{} nicht mit mir sein werden. Richard\pwindex{Beer-Hofmann, Richard 11.\,7.\,1866 Wien – 26.\,9.\,1945 New York City@\textsc{Beer-Hofmann, Richard} (11.\,7.\,1866 Wien – 26.\,9.\,1945 New York City), \emph{Schriftsteller}|pw} sagte mir gestern, er wollte zur \uline{zweiten} Vorstellung
               gehen. Speidels\pwindex{Speidel, Felix 2.\,7.\,1875 Stuttgart – 3.\,10.\,1952 Unterach am Attersee@\textsc{Speidel, Felix} (2.\,7.\,1875 Stuttgart – 3.\,10.\,1952 Unterach am Attersee), \emph{Schriftsteller, Verleger}|pw}\pwindex{Speidel-Haeberle, Else 11.\,7.\,1877 Stuttgart – 21.\,7.\,1937 Augustenfeld@\textsc{Speidel-Haeberle, Else} (11.\,7.\,1877 Stuttgart – 21.\,7.\,1937 Augustenfeld), \emph{Schauspielerin}|pw}
                  sin\textcolor{gray}{d} wohl im Theater\oindex{Wien@\textbf{Wien}!VII., Neubau@\textbf{VII., Neubau}!Volkstheater@\textbf{Volkstheater}, \emph{Theater}|pwv}. Ich würde vorschlagen: Meissl {\kaufmannsund} Schadn\oindex{Wien@\textbf{Wien}!I., Innere Stadt@\textbf{I., Innere Stadt}!Meissl {\kaufmannsund} Schadn@\textbf{Meissl {\kaufmannsund} Schadn}, \emph{Hotel}|pw} wie \label{K_L03010-2v}\edtext{neulich nach dem Walzertraum\pwindex{\textcolor{red}{\textsuperscript{XXXX indx1}}!Walzertraum. Operette in drei Akten@\strich\emph{Ein Walzertraum. Operette in drei Akten}|pw}\pwindex{\textcolor{red}{\textsuperscript{XXXX indx1}}!Walzertraum. Operette in drei Akten@\strich\emph{Ein Walzertraum. Operette in drei Akten}|pw}}{\lemma{\textnormal{\emph{neulich … Walzertraum}}}\Cendnote{\textnormal{Vgl. A. S.: \emph{Tagebuch}, 23. 10. 1907.
               }}}\label{K_L03010-2}. Sie vergessen nicht mir die Loge zu schicken?\pend
           
\pstart
           herzlichst Ihr {\\[\baselineskip]}\spacefill\mbox{Arthur}\pend
           \leftskip=0em{}\selectlanguage{ngerman}\endnumbering\briefempfaengerindex{Salten, Felix@\textsc{Salten, Felix}!zzzSchnitzler, Arthur@\emph{von Arthur Schnitzler}!1907-11-071@{7. 11. 1907}|)be}\mylabel{L03010h}  \newcommand{\dateiname}{L03010}\newcommand{\titel}{Arthur Schnitzler an Felix Salten, 7. 11. 1907}\newcommand{\editorInnen}{Martin Anton Müller und Laura Untner}%% latex-leseansicht-abspann.tex
%% Abspann für die Leseansicht.
%% Der Schalter \ifkorrekturansicht ist bereits durch den Vorspann gesetzt.

%% latex-abspann.tex
%% Gemeinsamer Abspann für Korrekturansicht und Leseansicht.
%% Setzt den Schalter \ifkorrekturansicht voraus (gesetzt in den
%% einbindenden Dateien latex-korrekturansicht-abspann.tex bzw.
%% latex-leseansicht-abspann.tex).
%% ---------------------------------------------------------------

\normalsize

% Das esempio-Environment wird nur in der Leseansicht benötigt
\ifkorrekturansicht\else
\newenvironment{esempio}[3]%
{
    \vspace{1.5ex}
    \rlap{\underline{#1}}
    \par
    \setlength{\parindent}{0cm}
    \nopagebreak
    \leftskip=#2cm
    \rightskip=#3cm
}
{
    \par
}
\fi

\doendnotes{C}
\bigskip
\vfill

\clearpage

\footnotesize

\ifkorrekturansicht
  \lohead{\textsc{register}}
\fi

% theindex-Environment neu definieren ohne reledmac
\makeatletter
\renewenvironment{theindex}{%
  \ifkorrekturansicht
    \section*{\indexname}%
  \else
    \subsubsection*{Index der erwähnten Entitäten}%
  \fi
  \setlength{\parindent}{0pt}%
  \setlength{\parskip}{0pt plus 0.3pt}%
  \let\item\@idxitem
}{%
  \ifkorrekturansicht\clearpage\fi
}
\makeatother

\IfFileExists{\jobname-pw.ind}{\input{\jobname-pw.ind}}{}

% Quellenangabe nur in der Leseansicht
\ifkorrekturansicht\else
% Fallback-Definitionen, falls die .tex-Datei \titel etc. nicht gesetzt hat
\providecommand{\titel}{}
\providecommand{\editorInnen}{}
\providecommand{\dateiname}{\jobname}

\vspace{3cm}

\vfill

\footnotesize
\textsc{Quelle}: \titel. Herausgegeben von {\editorInnen}. In: \emph{Arthur Schnitzler: Briefwechsel mit Autorinnen und Autoren}.
 Digitale Edition, https://schnitzler-briefe.acdh.oeaw.ac.at/{\dateiname}.html (Stand \today)
\fi

\end{document}


