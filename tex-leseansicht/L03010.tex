%% latex-korrekturansicht-vorspann.tex
%% Vorspann für die Korrekturansicht.
%% Lädt die gemeinsame Datei latex-vorspann.tex mit gesetztem Schalter.

\newif\ifkorrekturansicht
\korrekturansichttrue

\input{../tex-inputs/latex-vorspann}


\section[ Arthur Schnitzler an Felix Salten, 7. 11. 1907]{L03010 Arthur Schnitzler an Felix Salten, 7. 11. 1907}
\nopagebreak\mylabel{L03010v}
\rehead{ }\normalsize\beginnumbering\briefempfaengerindex{Salten, Felix@\textsc{Salten, Felix}!zzzSchnitzler, Arthur@\emph{von Arthur Schnitzler}!1907-11-071@{7. 11. 1907}|(be}
\toendnotes[C]{\smallbreak\pagebreak[2]}\Standort{Wienbibliothek im Rathaus, ZPH 1681, 2.1.516.}
\physDesc{Brief, 1 Blatt, 1 Seite, 390 Zeichen
\newline{}Handschrift: schwarze Tinte, lateinische Kurrent
\newline{}Ordnung: mit Bleistift von unbekannter Hand nummeriert: »7« }\toendnotes[C]{\smallbreak}
\pstart
           \raggedleft{}{\pb}7. 11. 907\pend
           
\pstart{}lieber,\pend\vspace{0.5em}
\pstart
           Kainz\pwindex{Kainz, Josef 02.01.1858 – 20.09.1910@\textsc{Kainz, Josef} (02.01.1858 – 20.09.1910), \emph{Schauspieler/Schauspielerin}|pw} spielt am Samstag den Fiesco\pwindex{Verschwoerung des Fiesko zu Genua@\emph{Die Verschwörung des Fiesko zu Genua}|pw}, Frau Kainz\pwindex{Kainz, Margarethe 13.12.1858 – 12.02.1950@\textsc{Kainz, Margarethe} (13.12.1858 – 12.02.1950), \emph{Schauspieler/Schauspielerin}|pw} ist bei Ihrer \label{K_L03010-1v}\edtext{Première\pwindex{Vom andern Ufer. Einakter@\emph{Vom andern Ufer. Einakter}|pwv}}{\lemma{\textnormal{\emph{Première}}}\Cendnote{\textnormal{Die Premiere von \emph{Vom
                  andern Ufer}\pwindex{Vom andern Ufer. Einakter@\emph{Vom andern Ufer. Einakter}|pwk} fand 9. 11. 1907 am \emph{Volkstheater}\orgindex{Volkstheater@Volkstheater|pwk}
                  statt. Schnitzler nahm teil.}}}\label{K_L03010-1}, geht aber dann zu Fiesco\pwindex{Verschwoerung des Fiesko zu Genua@\emph{Die Verschwörung des Fiesko zu Genua}|pw} hinüber, so
               daß sie wohl beide \introOben{}nachher\introOben{} nicht mit mir sein werden. Richard\pwindex{Beer-Hofmann, Richard 1866-07-11 – 1945-09-26@\textsc{Beer-Hofmann, Richard} (1866-07-11 – 1945-09-26), \emph{Schriftsteller/Schriftstellerin}|pw} sagte mir gestern, er wollte zur \uline{zweiten} Vorstellung
               gehen. Speidels\pwindex{Speidel, Felix 02.07.1875 – 1952-10-03@\textsc{Speidel, Felix} (02.07.1875 – 1952-10-03), \emph{Schriftsteller/Schriftstellerin, Verleger/Verlegerin}|pw}\pwindex{Speidel-Haeberle, Else 11.07.1877 – 21.07.1937@\textsc{Speidel-Haeberle, Else} (11.07.1877 – 21.07.1937), \emph{Schauspieler/Schauspielerin}|pw}
                  sin\textcolor{gray}{d} wohl im Theater\oindex{Volkstheater@\textbf{Volkstheater}, \emph{Theater (K.THE)}|pwv}. Ich würde vorschlagen: Meissl {\kaufmannsund} Schadn\oindex{Meissl {\kaufmannsund} Schadn@\textbf{Meissl {\kaufmannsund} Schadn}, \emph{Hotel (K.HTL)}|pw} wie \label{K_L03010-2v}\edtext{neulich nach dem Walzertraum\pwindex{Walzertraum. Operette in drei Akten@\emph{Ein Walzertraum. Operette in drei Akten}|pw}}{\lemma{\textnormal{\emph{neulich … Walzertraum}}}\Cendnote{\textnormal{Vgl. A. S.: \emph{Tagebuch}, 23. 10. 1907.
               }}}\label{K_L03010-2}. Sie vergessen nicht mir die Loge zu schicken?\pend
           
\pstart
           herzlichst Ihr {\\[\baselineskip]}\spacefill\mbox{Arthur}\pend
           \leftskip=0em{}\selectlanguage{ngerman}\endnumbering\briefempfaengerindex{Salten, Felix@\textsc{Salten, Felix}!zzzSchnitzler, Arthur@\emph{von Arthur Schnitzler}!1907-11-071@{7. 11. 1907}|)be}\mylabel{L03010h}  \normalsize

\doendnotes{C}
\bigskip
\vfill

\clearpage

\footnotesize

\lohead{\textsc{register}}

% Definiere theindex-Environment komplett neu ohne reledmac
\makeatletter
\renewenvironment{theindex}{%
  \section*{\indexname}%
  \setlength{\parindent}{0pt}%
  \setlength{\parskip}{0pt plus 0.3pt}%
  \let\item\@idxitem
}{%
  \clearpage
}
\makeatother

\IfFileExists{\jobname-pw.ind}{\input{\jobname-pw.ind}}{}

\end{document}

      