%% latex-leseansicht-vorspann.tex
%% Vorspann für die Leseansicht.
%% Lädt die gemeinsame Datei latex-vorspann.tex mit nicht gesetztem Schalter.

\newif\ifkorrekturansicht
\korrekturansichtfalse

\input{../tex-inputs/latex-vorspann}


\section[Felix Salten an Arthur Schnitzler, 2. 9. 1891]{L03103 Felix Salten an Arthur Schnitzler, 2. 9. 1891}
\nopagebreak\mylabel{L03103v}
\rehead{ }\normalsize\beginnumbering\briefempfaengerindex{Schnitzler, Arthur@\textsc{Schnitzler, Arthur}!zzzSalten, Felix@\emph{von Felix Salten}!1891-09-021@{2. 9. 1891}|(be}
\toendnotes[C]{\smallbreak\pagebreak[2]}
\correspDesc{Versand  durch Felix Salten am 2. 9. 1891 in Miskolc
\newline{}Erhalt  durch Arthur Schnitzler im Zeitraum [3. 9. 1891
                  – 7. 9. 1891?] in Wien}\toendnotes[C]{\smallbreak}
\Standort{CUL, Schnitzler, B 89, A 1.}
\physDesc{Brief, 1 Blatt, 4 Seiten, 1916 Zeichen
\newline{}Handschrift: schwarze Tinte, lateinische Kurrent
\newline{}Ordnung: mit Bleistift von unbekannter Hand nummeriert: »5« }\toendnotes[C]{\smallbreak}
\pstart
           \raggedleft{}{\pb}Miskolcz\oindex{Miskolc@\textbf{Miskolc}|pw}, 2. September 91\pend
           \vspace{0.5em}
\pstart
           Lieber Freund! Vor allem, wie geht es Ihnen? was
               machen Sie? und was hat sich ereignet? –\pend
           
\pstart
           Ich sitze unter der Hängelampe – gut, habe eine langweilige Fahrt gehabt, Umstände u.
               Nachzahlung wegen des Hundes. Mit einem Kondukteur, der nicht deutsch sprach,
               gestritten, – in irgend einen Csaba oder Becse oder so was eine Ziege für einen
               ungrischen Ochsen angesehen, – Reiseeindrücke – wissen Sie, = Becher!\pend
           
\pstart
           {\pb}Hier\oindex{Miskolc@\textbf{Miskolc}|pwv} lebe ich famos. Heute mit einem neu von Papa\pwindex{Salzmann, Philipp 24.\,12.\,1831 Miskolc – 2.\,4.\,1905 Wien@\textsc{Salzmann, Philipp} (24.\,12.\,1831 Miskolc – 2.\,4.\,1905 Wien), \emph{Bergbauunternehmer, Projektemacher}|pwv} gekauften Wagen u. neuen Pferden in’s
                  Bad Tapolcz\oindex{Miskolctapolca@\textbf{Miskolctapolca}|pw} gefahren. Prachtvolles Land\oindex{Ungarn@\textbf{Ungarn}|pwv} ist das wol hier, aber
               die Menschen sollte man ausrotten.\pend
           
\pstart
           Eine grausige Idee: Mich hat es gequält, dass wir an so vielen Alleen, Heerstraßen u.
               Brücken, \strikeout{vor} die fern am Ho\substVorne{}\textsuperscript{z}\substDazwischen{}r\substHinten{}izont sichtbar waren, vorbeifuhren, u. ich i{\geminationm}er
               denken musste\textcolor{gray}{,} dass ich in meinem Leben nie durch diese Allee od.
               über jene Brücke gehen werde.\pend
           
\pstart
           {\pb}\label{K_L03103-1v}\edtext{Aus 
               Mödling\oindex{Mödling@\textbf{Mödling}, \emph{Hauptstadt}|pw}}{\lemma{\textnormal{\emph{Aus 
               Mödling}}}\Cendnote{\textnormal{vermutlich von Bertha Karlsburg\pwindex{Karlsburg, Bertha @\textsc{Karlsburg, Bertha}, \emph{Schauspielerin}|pwk}}}}\label{K_L03103-1} bekomme ich die frappirendsten Briefe. Ich hätte nicht gedacht, dass sie\pwindex{Karlsburg, Bertha @\textsc{Karlsburg, Bertha}, \emph{Schauspielerin}|pwuv} mich wirklich
               noch so lieb hat. Mir geht es in dieser Hinsicht \uline{sehr
                  gut}.\pend
           
\pstart
           Hier ist eine hübsche Orpheum\orgindex{Danzer’s Orpheum@Danzer’s Orpheum|pw}-Sängerin\pwindex{Makay, Paula @\textsc{Makay, Paula}, \emph{Sängerin}|pwv}. Von dem Weib\pwindex{Makay, Paula @\textsc{Makay, Paula}, \emph{Sängerin}|pwv} habe ich Ihnen viel für
               uns psychologisch \substVorne{}\textsuperscript{i}\substDazwischen{}I\substHinten{}nteressantes zu erzählen.\pend
           
\pstart
           Gestern soupirte ich mit ihr u. meinen Brüdern\pwindex{Salzmann, Michael Emil 19.\,1.\,1858 Szigetvár – 26.\,6.\,1908 Wien@\textsc{Salzmann, Michael Emil} (19.\,1.\,1858 Szigetvár – 26.\,6.\,1908 Wien), \emph{Versicherungsbeamter}|pwuv}\pwindex{Salzmann, Ignaz 30.\,12.\,1858 Budapest – 8.\,8.\,1932 Wien@\textsc{Salzmann, Ignaz} (30.\,12.\,1858 Budapest – 8.\,8.\,1932 Wien), \emph{Kaufmann}|pwuv}\pwindex{Salzmann, Theodor 1867 Budapest – 12.\,12.\,1926 Wien@\textsc{Salzmann, Theodor} (1867 Budapest – 12.\,12.\,1926 Wien)|pwuv}\pwindex{Sós, Geza 31.\,12.\,1870 Budapest – 30.\,1.\,1918 ebd.@\textsc{Sós, Geza} (31.\,12.\,1870 Budapest – 30.\,1.\,1918 ebd.), \emph{Bildhauer}|pwuv}. Sie sehen also, dass die Hängelampe nach 10 Uhr
               verlöscht – natürlich brannte sie \substVorne{}\textsuperscript{\textcolor{gray}{um}}\substDazwischen{}in\substHinten{} aller Ruhe wieder, {\pb}da wusste man das Ereignis
               in ganz Miskolcz\oindex{Miskolc@\textbf{Miskolc}|pw} u. erzählte sich, dass die
               »schöne Makay Paula\pwindex{Makay, Paula @\textsc{Makay, Paula}, \emph{Sängerin}|pw} die Einladung eines
                  Hußaren\textcolor{gray}{-}Rittmeisters ausgeschlagen, u. meine angenommen.«\pend
           
\pstart
           Wäre ich jetzt bei Ihnen, u. könnte die illustrirende Geste dazu
                  machen, würde ich sagen: »Famoses Mädel\pwindex{Makay, Paula @\textsc{Makay, Paula}, \emph{Sängerin}|pwv}, – fliegt \label{K_L03103-2v}\edtext{damisch}{\lemma{\textnormal{\emph{damisch}}}\Cendnote{\textnormal{österreichisch: verrückt (hier positiv konnotiert)}}}\label{K_L03103-2} auf mich!«\pend
           
\pstart
           Übrigens, das ganze Milieu des Orpheums\orgindex{Danzer’s Orpheum@Danzer’s Orpheum|pw} (wir luden
               auch einige ihrer Collegen ein) ist sehr interessant.\pend
           
\pstart
           Wie gerne hätte ich Sie jetzt hier\oindex{Miskolc@\textbf{Miskolc}|pwv}!\pend
           
\pstart
           Bitte, schreiben Sie mir ausführlich, verzeihen Sie die abgehackten Sätze, sie sind
               nicht Manier, sondern letztlich eine Folge der Eile, in der ich schreibe.\pend
           
\pstart
           Bald mehr, schreiben Sie gleich. Ihr aufrichtiger {\\[\baselineskip]}\spacefill\mbox{Salten}\pend
           \leftskip=0em{}\selectlanguage{ngerman}\endnumbering\briefempfaengerindex{Schnitzler, Arthur@\textsc{Schnitzler, Arthur}!zzzSalten, Felix@\emph{von Felix Salten}!1891-09-021@{2. 9. 1891}|)be}\mylabel{L03103h}  \newcommand{\dateiname}{L03103}\newcommand{\titel}{Felix Salten an Arthur Schnitzler, 2. 9. 1891}\newcommand{\editorInnen}{Martin Anton Müller und Laura Untner}%% latex-leseansicht-abspann.tex
%% Abspann für die Leseansicht.
%% Der Schalter \ifkorrekturansicht ist bereits durch den Vorspann gesetzt.

%% latex-abspann.tex
%% Gemeinsamer Abspann für Korrekturansicht und Leseansicht.
%% Setzt den Schalter \ifkorrekturansicht voraus (gesetzt in den
%% einbindenden Dateien latex-korrekturansicht-abspann.tex bzw.
%% latex-leseansicht-abspann.tex).
%% ---------------------------------------------------------------

\normalsize

% Das esempio-Environment wird nur in der Leseansicht benötigt
\ifkorrekturansicht\else
\newenvironment{esempio}[3]%
{
    \vspace{1.5ex}
    \rlap{\underline{#1}}
    \par
    \setlength{\parindent}{0cm}
    \nopagebreak
    \leftskip=#2cm
    \rightskip=#3cm
}
{
    \par
}
\fi

\doendnotes{C}
\bigskip
\vfill

\clearpage

\footnotesize

\ifkorrekturansicht
  \lohead{\textsc{register}}
\fi

% theindex-Environment neu definieren ohne reledmac
\makeatletter
\renewenvironment{theindex}{%
  \ifkorrekturansicht
    \section*{\indexname}%
  \else
    \subsubsection*{Index der erwähnten Entitäten}%
  \fi
  \setlength{\parindent}{0pt}%
  \setlength{\parskip}{0pt plus 0.3pt}%
  \let\item\@idxitem
}{%
  \ifkorrekturansicht\clearpage\fi
}
\makeatother

\IfFileExists{\jobname-pw.ind}{\input{\jobname-pw.ind}}{}

% Quellenangabe nur in der Leseansicht
\ifkorrekturansicht\else
% Fallback-Definitionen, falls die .tex-Datei \titel etc. nicht gesetzt hat
\providecommand{\titel}{}
\providecommand{\editorInnen}{}
\providecommand{\dateiname}{\jobname}

\vspace{3cm}

\vfill

\footnotesize
\textsc{Quelle}: \titel. Herausgegeben von {\editorInnen}. In: \emph{Arthur Schnitzler: Briefwechsel mit Autorinnen und Autoren}.
 Digitale Edition, https://schnitzler-briefe.acdh.oeaw.ac.at/{\dateiname}.html (Stand \today)
\fi

\end{document}


