%% latex-korrekturansicht-vorspann.tex
%% Vorspann für die Korrekturansicht.
%% Lädt die gemeinsame Datei latex-vorspann.tex mit gesetztem Schalter.

\newif\ifkorrekturansicht
\korrekturansichttrue

\input{../tex-inputs/latex-vorspann}


\section[Felix Salten an Arthur Schnitzler, 2. 9. 1891]{L03103 Felix Salten an Arthur Schnitzler, 2. 9. 1891}
\nopagebreak\mylabel{L03103v}
\rehead{ }\normalsize\beginnumbering\briefempfaengerindex{Schnitzler, Arthur@\textsc{Schnitzler, Arthur}!zzzSalten, Felix@\emph{von Felix Salten}!1891-09-021@{2. 9. 1891}|(be}
\toendnotes[C]{\smallbreak\pagebreak[2]}\Standort{CUL, Schnitzler, B 89, A 1.}
\physDesc{Brief, 1 Blatt, 4 Seiten, 1916 Zeichen
\newline{}Handschrift: schwarze Tinte, lateinische Kurrent
\newline{}Ordnung: mit Bleistift von unbekannter Hand nummeriert: »5« }\toendnotes[C]{\smallbreak}
\pstart
           \raggedleft{}{\pb}Miskolcz\oindex{Miskolc@\textbf{Miskolc}, \emph{P.PPLA}|pw}, 2. September 91\pend
           \vspace{0.5em}
\pstart
           Lieber Freund! Vor allem, wie geht es Ihnen? was
               machen Sie? und was hat sich ereignet? –\pend
           
\pstart
           Ich sitze unter der Hängelampe – gut, habe eine langweilige Fahrt gehabt, Umstände u.
               Nachzahlung wegen des Hundes. Mit einem Kondukteur, der nicht deutsch sprach,
               gestritten, – in irgend einen Csaba oder Becse oder so was eine Ziege für einen
               ungrischen Ochsen angesehen, – Reiseeindrücke – wissen Sie, = Becher!\pend
           
\pstart
           {\pb}Hier\oindex{Miskolc@\textbf{Miskolc}, \emph{P.PPLA}|pwv} lebe ich famos. Heute mit einem neu von Papa\pwindex{Salzmann, Philipp 1831-12-24 – 1905-04-02@\textsc{Salzmann, Philipp} (1831-12-24 – 1905-04-02), \emph{Bergbauunternehmer/Bergbauunternehmerin, Projektemacher/Projektemacherin}|pwv} gekauften Wagen u. neuen Pferden in’s
                  Bad Tapolcz\oindex{Miskolctapolca@\textbf{Miskolctapolca}, \emph{P.PPL}|pw} gefahren. Prachtvolles Land\oindex{Ungarn@\textbf{Ungarn}, \emph{A.PCLI}|pwv} ist das wol hier, aber
               die Menschen sollte man ausrotten.\pend
           
\pstart
           Eine grausige Idee: Mich hat es gequält, dass wir an so vielen Alleen, Heerstraßen u.
               Brücken, \strikeout{vor} die fern am Ho\substVorne{}\textsuperscript{z}\substDazwischen{}r\substHinten{}izont sichtbar waren, vorbeifuhren, u. ich i{\geminationm}er
               denken musste\textcolor{gray}{,} dass ich in meinem Leben nie durch diese Allee od.
               über jene Brücke gehen werde.\pend
           
\pstart
           {\pb}\label{K_L03103-1v}\edtext{Aus 
               Mödling\oindex{Moedling@\textbf{Mödling}, \emph{P.PPLA3}|pw}}{\lemma{\textnormal{\emph{Aus 
               Mödling}}}\Cendnote{\textnormal{vermutlich von Bertha Karlsburg\pwindex{Karlsburg, Bertha @\textsc{Karlsburg, Bertha}, \emph{Schauspieler/Schauspielerin}|pwk}}}}\label{K_L03103-1} bekomme ich die frappirendsten Briefe. Ich hätte nicht gedacht, dass sie\pwindex{Karlsburg, Bertha @\textsc{Karlsburg, Bertha}, \emph{Schauspieler/Schauspielerin}|pwuv} mich wirklich
               noch so lieb hat. Mir geht es in dieser Hinsicht \uline{sehr
                  gut}.\pend
           
\pstart
           Hier ist eine hübsche Orpheum\orgindex{Danzer s Orpheum@Danzer’s Orpheum|pw}-Sängerin\pwindex{Makay, Paula @\textsc{Makay, Paula}, \emph{Sänger/Sängerin}|pwv}. Von dem Weib\pwindex{Makay, Paula @\textsc{Makay, Paula}, \emph{Sänger/Sängerin}|pwv} habe ich Ihnen viel für
               uns psychologisch \substVorne{}\textsuperscript{i}\substDazwischen{}I\substHinten{}nteressantes zu erzählen.\pend
           
\pstart
           Gestern soupirte ich mit ihr u. meinen Brüdern\pwindex{Salzmann, Michael Emil 1858-01-19 – 1908-06-26@\textsc{Salzmann, Michael Emil} (1858-01-19 – 1908-06-26), \emph{Versicherungsbeamter/Versicherungsbeamtin}|pwuv}\pwindex{Salzmann, Ignaz 1858-12-30 – 1932-08-08@\textsc{Salzmann, Ignaz} (1858-12-30 – 1932-08-08), \emph{Kaufmann/Kauffrau}|pwuv}\pwindex{Salzmann, Theodor 1867 – 1926-12-12@\textsc{Salzmann, Theodor} (1867 – 1926-12-12)|pwuv}\pwindex{Sós, Geza 1870-12-31 – 1918-01-30@\textsc{Sós, Geza} (1870-12-31 – 1918-01-30), \emph{Bildhauer/Bildhauerin}|pwuv}. Sie sehen also, dass die Hängelampe nach 10 Uhr
               verlöscht – natürlich brannte sie \substVorne{}\textsuperscript{\textcolor{gray}{um}}\substDazwischen{}in\substHinten{} aller Ruhe wieder, {\pb}da wusste man das Ereignis
               in ganz Miskolcz\oindex{Miskolc@\textbf{Miskolc}, \emph{P.PPLA}|pw} u. erzählte sich, dass die
               »schöne Makay Paula\pwindex{Makay, Paula @\textsc{Makay, Paula}, \emph{Sänger/Sängerin}|pw} die Einladung eines
                  Hußaren\textcolor{gray}{-}Rittmeisters ausgeschlagen, u. meine angenommen.«\pend
           
\pstart
           Wäre ich jetzt bei Ihnen, u. könnte die illustrirende Geste dazu
                  machen, würde ich sagen: »Famoses Mädel\pwindex{Makay, Paula @\textsc{Makay, Paula}, \emph{Sänger/Sängerin}|pwv}, – fliegt \label{K_L03103-2v}\edtext{damisch}{\lemma{\textnormal{\emph{damisch}}}\Cendnote{\textnormal{österreichisch: verrückt (hier positiv konnotiert)}}}\label{K_L03103-2} auf mich!«\pend
           
\pstart
           Übrigens, das ganze Milieu des Orpheums\orgindex{Danzer s Orpheum@Danzer’s Orpheum|pw} (wir luden
               auch einige ihrer Collegen ein) ist sehr interessant.\pend
           
\pstart
           Wie gerne hätte ich Sie jetzt hier\oindex{Miskolc@\textbf{Miskolc}, \emph{P.PPLA}|pwv}!\pend
           
\pstart
           Bitte, schreiben Sie mir ausführlich, verzeihen Sie die abgehackten Sätze, sie sind
               nicht Manier, sondern letztlich eine Folge der Eile, in der ich schreibe.\pend
           
\pstart
           Bald mehr, schreiben Sie gleich. Ihr aufrichtiger {\\[\baselineskip]}\spacefill\mbox{Salten}\pend
           \leftskip=0em{}\selectlanguage{ngerman}\endnumbering\briefempfaengerindex{Schnitzler, Arthur@\textsc{Schnitzler, Arthur}!zzzSalten, Felix@\emph{von Felix Salten}!1891-09-021@{2. 9. 1891}|)be}\mylabel{L03103h}  \normalsize

\doendnotes{C}
\bigskip
\vfill

\clearpage

\footnotesize

\lohead{\textsc{register}}

% Definiere theindex-Environment komplett neu ohne reledmac
\makeatletter
\renewenvironment{theindex}{%
  \section*{\indexname}%
  \setlength{\parindent}{0pt}%
  \setlength{\parskip}{0pt plus 0.3pt}%
  \let\item\@idxitem
}{%
  \clearpage
}
\makeatother

\IfFileExists{\jobname-pw.ind}{\input{\jobname-pw.ind}}{}

\end{document}

      