%% latex-leseansicht-vorspann.tex
%% Vorspann für die Leseansicht.
%% Lädt die gemeinsame Datei latex-vorspann.tex mit nicht gesetztem Schalter.

\newif\ifkorrekturansicht
\korrekturansichtfalse

\input{../tex-inputs/latex-vorspann}

\begin{center}
            \textcolor{red}{ENTWURF, NICHT FERTIG KORRIGIERT}
                      \end{center}
            
         
         \renewcommand{\erwaehntePersonen}{Personen:  ?? [Partnerin von Paul Goldmann und später Felix Salten], Paula Makay, Philipp Salzmann, Michael Emil Salzmann, Ignaz Salzmann, Theodor Salzmann, Geza Sós}
         \renewcommand{\erwaehnteInstitutionen}{Institutionen: Danzer’s Orpheum}
         \renewcommand{\erwaehnteOrte}{Orte: Miskolc, Miskolctapolca, Mödling, Wien}
         \renewcommand{\erwaehnteWerke}{}
               \section[Felix Salten an Arthur Schnitzler, 2. 9. 1891]{ Felix Salten an Arthur Schnitzler, 2. 9. 1891}\nopagebreak\mylabel{v}\rehead{ }\begin{ledgroupsized}[t]{13cm}\normalsize\beginnumbering \toendnotes[C]{\smallbreak\pagebreak[2]} \Standort{CUL, Schnitzler, B 89, A 1.}
\physDesc{Brief,  Blätter,  Seiten
\newline{}Handschrift: schwarze Tinte, lateinische Kurrent}\toendnotes[C]{\smallbreak}\pstart
           {\pb}Miskolcz\oindex{Miskolc@\textbf{Miskolc}|pw}, 2. September 91\pend
           \pstart
           Lieber Freund! Vor allem, wie geht es Ihnen? Was machen Sie? Und was
               hat sich ereignet? –\pend
           \pstart
           Ich sitze unter der Hängelampe – gut, habe eine langweilige Fahrt gehabt, Umstände u.
               Nachzahlung wegen des Hundes. Mit einem Kondukteur, der nicht deutsch sprach,
               gestritten, – in irgendeinen Csaba oder Becse oder so was eine Ziege für einen
               ungarischen Ochsen angesehen, – Reiseeindrücke – wissen Sie, = Becher!\pend
           \pstart
           {\pb}Hier lebe ich famos. Heute mit
               einem neu von Papa\pwindex{Salzmann, Philipp 1831-12-24 – 1905-04-02@\textsc{Salzmann, Philipp} (1831-12-24 – 1905-04-02), \emph{Bergbauunternehmer}|pwv}
               gekauften Wagen u. neuen Pferden in’s Bad
                  Tapolcz\oindex{Miskolctapolca@\textbf{Miskolctapolca}|pw} gefahren. Prachtvolles Land ist das wol hier, aber die Menschen
               sollte man ausrotten. \pend
           \pstart
           Eine grausige Idee: Mich hat es gequält, dass wir an so vielen Alleen, Heer straßen
               u. Brücken, \strikeout{vor} die fern am Horizont sichtbar waren,
               vorbeifuhren, u. ich i{\geminationm}er denken musste, dass ich in
               meinem Leben nie durch diese Allee oder über jene Brücke gehen werde. \pend
           \pstart
           {\pb}Aus Mödling\oindex{Moedling@\textbf{Mödling}|pw} bekomme ich die frappirendsten
               Briefe. Ich hötte nicht gedacht, dass sie\pwindex{?? [Partnerin von Paul Goldmann und spaeter Felix Salten] 1890 – 1891@\textsc{?? [Partnerin von Paul Goldmann und später Felix Salten]} (1890 – 1891)|pwv} mich wirklich
               noch so lieb hat. Mir geht es in dieser Hinsicht \uline{sehr
                  gut}.\pend
           \pstart
           Hier ist eine hübsche Orpheum\orgindex{Danzer s Orpheum@Danzer’s Orpheum|pw}-Sängerin\pwindex{Makay, Paula @\textsc{Makay, Paula}, \emph{Sänger/Sängerin}|pwv}. Von dem Weib habe ich
               Ihnen viel für uns psychologisch interessantes zu erzählen.\pend
           \pstart
           Gestern soupirte ich mit ihr u. meinen Brüdern\pwindex{Salzmann, Michael Emil 1858-01-19 – 1908-06-26@\textsc{Salzmann, Michael Emil} (1858-01-19 – 1908-06-26), \emph{Versicherungsbeamter}|pwv}\pwindex{Salzmann, Ignaz 1858-12-30 – 1932-08-08@\textsc{Salzmann, Ignaz} (1858-12-30 – 1932-08-08)|pwv}\pwindex{Salzmann, Theodor 1867 – 1926-12-12@\textsc{Salzmann, Theodor} (1867 – 1926-12-12)|pwv}\pwindex{Sós, Geza 1870-12-31 – 1918-01-30@\textsc{Sós, Geza} (1870-12-31 – 1918-01-30), \emph{Bildhauer}|pwv}. Sie sehen also, dass die Hängelampe nach 10 Uhr verlöscht –
               natürlich brannte sie in aller Ruhe wieder, {\pb}da wusste man das Ereignis
               in ganz Miskolcz\oindex{Miskolc@\textbf{Miskolc}|pw} u erzählte sich, dass die
               »schöne Makay Paula\pwindex{Makay, Paula @\textsc{Makay, Paula}, \emph{Sänger/Sängerin}|pw} die Einladung eines
                  Hussaren\textcolor{gray}{-}Rittmeisters ausgeschlagen, u. meine angenommen«. \pend
           \pstart
           Wäre ich jetzt bei Ihnen, u. könnte die illustrirende Geste dazu machen, würde ich
               sagen: »Famoses Mädel, – fliegt \label{K_L03103-1v}\edtext{damisch}{\lemma{\textnormal{\emph{damisch}}}\Cendnote{\textnormal{österreichisch: verrückt
                  (hier positiv konnotiert)}}}\label{K_L03103-1h} auf mich!«\pend
           \pstart
           Übrigens, das ganze Milieu des Orpheum\orgindex{Danzer s Orpheum@Danzer’s Orpheum|pw} (wir luden
               auch einige ihrer Collegen ein) ist sehr interessant.\pend
           \pstart
           Wie gerne hätte ich Sie jetzt hier!\pend
           \pstart
           Bitte, schreiben Sie mir ausführlich, verzeihen Sie die abgehackten Sätze, sie sind
               nicht Manier, sondern letztlich eine Folge der Eile, in der ich schreibe.\pend
           \pstart
           Bald mehr, schreiben Sie gleich. l \pend
           \pstart
           Ihr aufrichtiger{\\[\baselineskip]}\spacefill\mbox{Salten}\pend
           \leftskip=0em{}
         
         \endnumbering\mylabel{h}\end{ledgroupsized}\begin{anhang}\end{anhang}\newcommand{\dateiname}{L03103}\newcommand{\titel}{Felix Salten an Arthur Schnitzler, 2. 9. 1891}\newcommand{\editorInnen}{Martin Anton Müller und Laura Untner}%% latex-leseansicht-abspann.tex
%% Abspann für die Leseansicht.
%% Der Schalter \ifkorrekturansicht ist bereits durch den Vorspann gesetzt.

%% latex-abspann.tex
%% Gemeinsamer Abspann für Korrekturansicht und Leseansicht.
%% Setzt den Schalter \ifkorrekturansicht voraus (gesetzt in den
%% einbindenden Dateien latex-korrekturansicht-abspann.tex bzw.
%% latex-leseansicht-abspann.tex).
%% ---------------------------------------------------------------

\normalsize

% Das esempio-Environment wird nur in der Leseansicht benötigt
\ifkorrekturansicht\else
\newenvironment{esempio}[3]%
{
    \vspace{1.5ex}
    \rlap{\underline{#1}}
    \par
    \setlength{\parindent}{0cm}
    \nopagebreak
    \leftskip=#2cm
    \rightskip=#3cm
}
{
    \par
}
\fi

\doendnotes{C}
\bigskip
\vfill

\clearpage

\footnotesize

\ifkorrekturansicht
  \lohead{\textsc{register}}
\fi

% theindex-Environment neu definieren ohne reledmac
\makeatletter
\renewenvironment{theindex}{%
  \ifkorrekturansicht
    \section*{\indexname}%
  \else
    \subsubsection*{Index der erwähnten Entitäten}%
  \fi
  \setlength{\parindent}{0pt}%
  \setlength{\parskip}{0pt plus 0.3pt}%
  \let\item\@idxitem
}{%
  \ifkorrekturansicht\clearpage\fi
}
\makeatother

\IfFileExists{\jobname-pw.ind}{\input{\jobname-pw.ind}}{}

% Quellenangabe nur in der Leseansicht
\ifkorrekturansicht\else
% Fallback-Definitionen, falls die .tex-Datei \titel etc. nicht gesetzt hat
\providecommand{\titel}{}
\providecommand{\editorInnen}{}
\providecommand{\dateiname}{\jobname}

\vspace{3cm}

\vfill

\footnotesize
\textsc{Quelle}: \titel. Herausgegeben von {\editorInnen}. In: \emph{Arthur Schnitzler: Briefwechsel mit Autorinnen und Autoren}.
 Digitale Edition, https://schnitzler-briefe.acdh.oeaw.ac.at/{\dateiname}.html (Stand \today)
\fi

\end{document}


      