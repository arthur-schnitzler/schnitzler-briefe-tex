%% latex-leseansicht-vorspann.tex
%% Vorspann für die Leseansicht.
%% Lädt die gemeinsame Datei latex-vorspann.tex mit nicht gesetztem Schalter.

\newif\ifkorrekturansicht
\korrekturansichtfalse

\input{../tex-inputs/latex-vorspann}


\section[Arthur Schnitzler an Gustav Schwarzkopf, 28. 9. 1893]{L04103 Arthur Schnitzler an Gustav Schwarzkopf, 28. 9. 1893}
\nopagebreak\mylabel{L04103v}
\rehead{ }\normalsize\beginnumbering\briefempfaengerindex{Schwarzkopf, Gustav@\textsc{Schwarzkopf, Gustav}!zzzSchnitzler, Arthur@\emph{von Arthur Schnitzler}!1893-09-282@{28. 9. 1893}|(be}
\toendnotes[C]{\smallbreak\pagebreak[2]}
\correspDesc{Versand  durch Arthur Schnitzler am 28. 9. 1893 in Wien
\newline{}Erhalt  durch Gustav Schwarzkopf am [28. 9. 1893] in Wien}\toendnotes[C]{\smallbreak}
\Standort{CUL, Schnitzler, B 96.}
\physDesc{Postkarte, 311 Zeichen
\newline{}Handschrift: schwarze Tinte, deutsche Kurrent
\newline{}Versand: 1) Rohrpost  2) Stempel: »\nobreak{}\oindex{I., Innere Stadt@\textbf{I., Innere Stadt}, \emph{Verwaltungsgebiet}|pwk}Wien 1/1 10, \textcolor{gray}{28} IX \textcolor{gray}{93}, 10\textsuperscript{30}\textcolor{gray}{V}\nobreak{}«.  3) Stempel: »\nobreak{}\oindex{I., Innere Stadt@\textbf{I., Innere Stadt}, \emph{Verwaltungsgebiet}|pwk}Wien 1/1 10, \textcolor{gray}{28} IX 93, 11\textsuperscript{30}V\nobreak{}«.  4) Stempel: »\nobreak{}\oindex{I., Innere Stadt@\textbf{I., Innere Stadt}, \emph{Verwaltungsgebiet}|pwk}Wien 1/1, 28 IX 9\textcolor{gray}{3}, 11\textsuperscript{40}V\nobreak{}«. }\toendnotes[C]{\smallbreak}\pstart{}{\pb}\textsc{Herrn Gustav Schwarzkopf}\pend{}\pstart{}Schriftſteller\pend{}\pstart{}Wien\oindex{Wien@\textbf{Wien}, \emph{Verwaltungsgebiet}|pw}\pend{}\pstart{}\textsc{I. Tiefer Graben 23}\oindex{Wien@\textbf{Wien}!I., Innere Stadt@\textbf{I., Innere Stadt}!Tiefer Graben 23@\textbf{Tiefer Graben 23}, \emph{Wohngebäude}|pw}\pend{}{\bigskip}\vspace{1em}
\pstart
           \noindent{}{\pb}Verehrteſter Freund, we{\geminationn} Sie heute
               Abend nach zehn ins \textsc{Griensteidl\oindex{Wien@\textbf{Wien}!I., Innere Stadt@\textbf{I., Innere Stadt}!Café Griensteidl@\textbf{Café Griensteidl}, \emph{Kaffeehaus}|pw}} kämen, wäre das ſehr liebenswürdig von Ihnen. Es handelt ſich vornehmlich um
               die bereits \label{K_L04103-1v}\edtext{neulich}{\lemma{\textnormal{\emph{neulich}}}\Cendnote{\textnormal{Das letzte im \emph{Tagebuch}\pwindex{Schnitzler, Arthur 15. 5. 1862 Wien – 21. 10. 1931 ebd.@\textsc{Schnitzler, Arthur} (15. 5. 1862 Wien – 21. 10. 1931 ebd.), \emph{Schriftsteller, Mediziner}!Tagebuch@\strich\emph{Tagebuch}|pwk} belegte Treffen fand am 7. 9. 1893 statt. Es
                  dürfte aber danach noch eine Begegnung gegeben haben, da Schnitzler von Bahr\pwindex{Bahr, Hermann 19.\,7.\,1863 Linz – 15.\,1.\,1934 München@\textsc{Bahr, Hermann} (19.\,7.\,1863 Linz – 15.\,1.\,1934 München), \emph{Schriftsteller, Kritiker}|pwk}
                  erst in Folge des Briefes vom XXXX Auszeichnungsfehler: Dokument L00263 nicht gefunden bezüglich der Feuilletonreihe kontaktiert haben
                  dürfte.}}}\label{K_L04103-1} berührte \label{K_L04103-2v}\edtext{Feu{[}i{]}lletonreihe\pwindex{Wiener Spaziergänge@\emph{Wiener Spaziergänge}|pwv}}{\lemma{\textnormal{\emph{Feuilletonreihe}}}\Cendnote{\textnormal{Hermann Bahr\pwindex{Bahr, Hermann 19.\,7.\,1863 Linz – 15.\,1.\,1934 München@\textsc{Bahr, Hermann} (19.\,7.\,1863 Linz – 15.\,1.\,1934 München), \emph{Schriftsteller, Kritiker}|pwk} plante eine größere Feuilletonreihe\pwindex{Wiener Spaziergänge@\emph{Wiener Spaziergänge}|pwkv}, \emph{Wiener Spaziergänge}\pwindex{Wiener Spaziergänge@\emph{Wiener Spaziergänge}|pwk}, siehe Hermann Bahr, Arthur Schnitzler: \emph{Briefwechsel, Aufzeichnungen, Dokumente (1891–1931)}, Hermann Bahr: [Notizen zu den Wiener Spaziergängen?], [Oktober 1893?]. Es erschienen aber bis
                  zum Ausscheiden Bahrs\pwindex{Bahr, Hermann 19.\,7.\,1863 Linz – 15.\,1.\,1934 München@\textsc{Bahr, Hermann} (19.\,7.\,1863 Linz – 15.\,1.\,1934 München), \emph{Schriftsteller, Kritiker}|pwk} nur zwei Teile in der
                     \emph{Deutschen Zeitung}\pwindex{Deutsche Zeitung@\emph{Deutsche Zeitung}|pwk}, der eine\pwindex{Schnitzler, Arthur 15. 5. 1862 Wien – 21. 10. 1931 ebd.@\textsc{Schnitzler, Arthur} (15. 5. 1862 Wien – 21. 10. 1931 ebd.), \emph{Schriftsteller, Mediziner}!Spaziergang@\strich\emph{Spaziergang}|pwkv} von Schnitzler. }}}\label{K_L04103-2}. \textsc{Beer-Hofma{\geminationn}\pwindex{Beer-Hofmann, Richard 11.\,7.\,1866 Wien – 26.\,9.\,1945 New York City@\textsc{Beer-Hofmann, Richard} (11.\,7.\,1866 Wien – 26.\,9.\,1945 New York City), \emph{Schriftsteller}|pw}} iſt bestimmt dort. –\pend
           
\pstart
           Herzlicher Gruſs Ihr ergebner{\\[\baselineskip]}\spacefill\mbox{ArthSchn}\pend
           \leftskip=0em{}\selectlanguage{ngerman}\endnumbering\briefempfaengerindex{Schwarzkopf, Gustav@\textsc{Schwarzkopf, Gustav}!zzzSchnitzler, Arthur@\emph{von Arthur Schnitzler}!1893-09-282@{28. 9. 1893}|)be}\mylabel{L04103h}
\begin{anhang}
\end{anhang}\newcommand{\dateiname}{L04103}\newcommand{\titel}{Arthur Schnitzler an Gustav Schwarzkopf, 28. 9. 1893}\newcommand{\editorInnen}{Herausgegeben von Jahnke, SelmaMüller, Martin Anton}%% latex-leseansicht-abspann.tex
%% Abspann für die Leseansicht.
%% Der Schalter \ifkorrekturansicht ist bereits durch den Vorspann gesetzt.

%% latex-abspann.tex
%% Gemeinsamer Abspann für Korrekturansicht und Leseansicht.
%% Setzt den Schalter \ifkorrekturansicht voraus (gesetzt in den
%% einbindenden Dateien latex-korrekturansicht-abspann.tex bzw.
%% latex-leseansicht-abspann.tex).
%% ---------------------------------------------------------------

\normalsize

% Das esempio-Environment wird nur in der Leseansicht benötigt
\ifkorrekturansicht\else
\newenvironment{esempio}[3]%
{
    \vspace{1.5ex}
    \rlap{\underline{#1}}
    \par
    \setlength{\parindent}{0cm}
    \nopagebreak
    \leftskip=#2cm
    \rightskip=#3cm
}
{
    \par
}
\fi

\doendnotes{C}
\bigskip
\vfill

\clearpage

\footnotesize

\ifkorrekturansicht
  \lohead{\textsc{register}}
\fi

% theindex-Environment neu definieren ohne reledmac
\makeatletter
\renewenvironment{theindex}{%
  \ifkorrekturansicht
    \section*{\indexname}%
  \else
    \subsubsection*{Index der erwähnten Entitäten}%
  \fi
  \setlength{\parindent}{0pt}%
  \setlength{\parskip}{0pt plus 0.3pt}%
  \let\item\@idxitem
}{%
  \ifkorrekturansicht\clearpage\fi
}
\makeatother

\IfFileExists{\jobname-pw.ind}{\input{\jobname-pw.ind}}{}

% Quellenangabe nur in der Leseansicht
\ifkorrekturansicht\else
% Fallback-Definitionen, falls die .tex-Datei \titel etc. nicht gesetzt hat
\providecommand{\titel}{}
\providecommand{\editorInnen}{}
\providecommand{\dateiname}{\jobname}

\vspace{3cm}

\vfill

\footnotesize
\textsc{Quelle}: \titel. Herausgegeben von {\editorInnen}. In: \emph{Arthur Schnitzler: Briefwechsel mit Autorinnen und Autoren}.
 Digitale Edition, https://schnitzler-briefe.acdh.oeaw.ac.at/{\dateiname}.html (Stand \today)
\fi

\end{document}


