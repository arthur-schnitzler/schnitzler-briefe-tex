%% latex-leseansicht-vorspann.tex
%% Vorspann für die Leseansicht.
%% Lädt die gemeinsame Datei latex-vorspann.tex mit nicht gesetztem Schalter.

\newif\ifkorrekturansicht
\korrekturansichtfalse

\input{../tex-inputs/latex-vorspann}


\section[Karl Kraus an Arthur Schnitzler, 27. 7. 1893]{L00244 Karl Kraus an Arthur Schnitzler, 27. 7. 1893}
\nopagebreak\mylabel{L00244v}
\rehead{ }\normalsize\beginnumbering\briefempfaengerindex{Schnitzler, Arthur@\textsc{Schnitzler, Arthur}!zzzKraus, Karl@\emph{von Karl Kraus}!1893-07-271@{27. 7. 1893}|(be}
\toendnotes[C]{\smallbreak\pagebreak[2]}
\correspDesc{Versand  durch Karl Kraus am 27. 7. 1893 in Bad Ischl
\newline{}Erhalt  durch Arthur Schnitzler im Zeitraum [28. 7. 1893
                  – 1. 8. 1893?] in Wien}\toendnotes[C]{\smallbreak}
\Standort{CUL, Schnitzler, B 55.}
\physDesc{Postkarte, 865 Zeichen
\newline{}Handschrift: schwarze Tinte, deutsche Kurrent
\newline{}Versand: Stempel: »\nobreak{}\oindex{Bad Ischl@\textbf{Bad Ischl}|pwk}Ischl, 27/7 93, 1–N\nobreak{}«.  
\newline{}Schnitzler: mit Bleistift seitlich des Textes neben die »Fr. Bühne\orgindex{»Freie Bühne« Verein für moderne Literatur@»Freie Bühne« Verein für moderne Literatur|pw}«: »|| \textsc{Hirschfeld\pwindex{Hirschfeld, Robert 17.\,9.\,1857 Žďár nad Sázavou – 2.\,4.\,1914 Salzburg@\textsc{Hirschfeld, Robert} (17.\,9.\,1857 Žďár nad Sázavou – 2.\,4.\,1914 Salzburg), \emph{Journalist, Musikkritiker}|pw}–Wengraf\pwindex{Wengraf, Edmund 9.\,1.\,1860 Mikulov – 8.\,12.\,1933 Wien@\textsc{Wengraf, Edmund} (9.\,1.\,1860 Mikulov – 8.\,12.\,1933 Wien), \emph{Schriftsteller, Journalist, Kaufmann}|pw} –
                                          \textcolor{gray}{frei}? ||}« }
\buchAbdrucke{\weitereDrucke{\emph{Karl Kraus und Arthur Schnitzler. Eine Dokumentation.}Herausgegeben von Reinhard Urbach In: \emph{Literatur und Kritik}, Bd. 49, Oktober 1970, S. 519–520.} }\toendnotes[C]{\smallbreak}\pstart{}{\pb}Herrn Doktor Arthur
                  Schnitzler,\pend{}\pstart{}Schriftſteller\pend{}\pstart{}I. Grillparzerstr. 7\oindex{Wien@\textbf{Wien}!I., Innere Stadt@\textbf{I., Innere Stadt}!Grillparzerstraße@\textbf{Grillparzerstraße}, \emph{Straße}|pw}\pend{}\pstart{}Wien\oindex{Wien@\textbf{Wien}, \emph{Verwaltungsgebiet}|pw}\pend{}{\bigskip}\vspace{1em}
\pstart
           \noindent{}{\pb}\uline{Innigſten Dank}, liebſter Doktor, für den
               lieben Brief! Beifolgend das letzte \uline{Magazin\pwindex{Magazin für die Literatur des Auslandes@\emph{Magazin für die Literatur des Auslandes}|pw}}, das ich erſt heute bekam; es{ }ſteht eine \label{K_L00244-1v}\edtext{Nachricht\pwindex{Kraus, Karl 28.\,4.\,1874 Jičín – 12.\,6.\,1936 Wien@\textsc{Kraus, Karl} (28.\,4.\,1874 Jičín – 12.\,6.\,1936 Wien), \emph{Schriftsteller, Publizist, Schriftsteller}!Wiener Theater. – Luise Sigert. Auferstanden@\strich\emph{Wiener Theater. – Luise Sigert. Auferstanden{\rufezeichen}}|pwv}}{\lemma{\textnormal{\emph{Nachricht}}}\Cendnote{\textnormal{Diese Karte bezieht sich auf ein
                  Gerichtsverfahren, das am 24. 7. 1893 und am 25. 7. 1893
                  in Wien\oindex{Wien@\textbf{Wien}, \emph{Verwaltungsgebiet}|pwk} wegen sexuell zu expliziter
                  Veröffentlichungen in einer Wochenschrift namens \emph{Gesellschaft}\orgindex{Gesellschaft [Wien]@Die Gesellschaft [Wien]|pwk} verhandelt wurde. Dabei wurden Moriz Ehrenfeld\pwindex{Ehrenfeld, Moriz 29.\,9.\,1869 Wien – 13.\,9.\,1900 Czernowitz@\textsc{Ehrenfeld, Moriz} (29.\,9.\,1869 Wien – 13.\,9.\,1900 Czernowitz), \emph{Schriftsteller}|pwk}, Ferdinand
                     Mautner\pwindex{Gumppenberg, Hanns von 4.\,12.\,1866 Landshut – 29.\,3.\,1928 München@\textsc{Gumppenberg, Hanns von} (4.\,12.\,1866 Landshut – 29.\,3.\,1928 München), \emph{Schriftsteller, Kritiker}|pwk} und Alfred Brehmer\pwindex{Brehmer, Arthur 8.\,2.\,1858 Triest – 1.\,12.\,1923 Eichgraben@\textsc{Brehmer, Arthur} (8.\,2.\,1858 Triest – 1.\,12.\,1923 Eichgraben), \emph{Redakteur}|pwk} zu
                  mehrmonatigen Haftstrafen verurteilt. Verteidigt wurden die letzteren beiden von
                     Friedrich Elbogen\pwindex{Elbogen, Friedrich 20.\,5.\,1854 Prag – 15.\,4.\,1909 Wien@\textsc{Elbogen, Friedrich} (20.\,5.\,1854 Prag – 15.\,4.\,1909 Wien), \emph{Schriftsteller, Kritiker, Rechtsanwalt}|pwk}. Brehmer\pwindex{Brehmer, Arthur 8.\,2.\,1858 Triest – 1.\,12.\,1923 Eichgraben@\textsc{Brehmer, Arthur} (8.\,2.\,1858 Triest – 1.\,12.\,1923 Eichgraben), \emph{Redakteur}|pwk} war zudem an einer weiteren Zeitschrift beteiligt,
                     \emph{Wiener Kunst}\orgindex{Wiener Kunst@Wiener Kunst|pwk}. Beide Zeitschriften sind
                  nicht erhalten. Der Konnex, den Kraus\pwindex{Kraus, Karl 28.\,4.\,1874 Jičín – 12.\,6.\,1936 Wien@\textsc{Kraus, Karl} (28.\,4.\,1874 Jičín – 12.\,6.\,1936 Wien), \emph{Schriftsteller, Publizist, Schriftsteller}|pwk}
                  herstellt, bezieht sich auf den letzten Absatz seines Theaterbriefs, erschienen am
                     22. 7. 1893: \emph{Wiener Theater. – Luise Sigert\pwindex{Sigert, Louise *~25.\,5.\,1871 Wien@\textsc{Sigert, Louise} (*~25.\,5.\,1871 Wien), \emph{Schriftstellerin}|pwk}. \emph{Auferstanden!}\pwindex{Sigert, Louise *~25.\,5.\,1871 Wien@\textsc{Sigert, Louise} (*~25.\,5.\,1871 Wien), \emph{Schriftstellerin}!Auferstanden Drama in einem Vorspiele und drei Acten@\strich\emph{Auferstanden{\rufezeichen} Drama in einem Vorspiele und drei Acten}|pwk}}\pwindex{Kraus, Karl 28.\,4.\,1874 Jičín – 12.\,6.\,1936 Wien@\textsc{Kraus, Karl} (28.\,4.\,1874 Jičín – 12.\,6.\,1936 Wien), \emph{Schriftsteller, Publizist, Schriftsteller}!Wiener Theater. – Luise Sigert. Auferstanden@\strich\emph{Wiener Theater. – Luise Sigert. Auferstanden{\rufezeichen}}|pwk}. In: \emph{Das Magazin für Litteratur}\pwindex{Magazin für die Literatur des Auslandes@\emph{Magazin für die Literatur des Auslandes}|pwk}, Jg. 62,
                     Nr. 29, S. 466–467. Darin endet Kraus\pwindex{Kraus, Karl 28.\,4.\,1874 Jičín – 12.\,6.\,1936 Wien@\textsc{Kraus, Karl} (28.\,4.\,1874 Jičín – 12.\,6.\,1936 Wien), \emph{Schriftsteller, Publizist, Schriftsteller}|pwk} mit einer Kritik an der Zeitschrift \emph{Wiener Kunst}\orgindex{Wiener Kunst@Wiener Kunst|pwk} und erwähnt eine geplante Musteraufführung von \emph{Die Weber}\pwindex{Hauptmann, Gerhart 15.\,11.\,1862 Szczawno-Zdrój – 6.\,6.\,1946 Jagniątków@\textsc{Hauptmann, Gerhart} (15.\,11.\,1862 Szczawno-Zdrój – 6.\,6.\,1946 Jagniątków), \emph{Schriftsteller}!Weber. Schauspiel aus den vierziger Jahren@\strich\emph{Die Weber. Schauspiel aus den vierziger Jahren}|pwk} von Gerhart Hauptmann\pwindex{Hauptmann, Gerhart 15.\,11.\,1862 Szczawno-Zdrój – 6.\,6.\,1946 Jagniątków@\textsc{Hauptmann, Gerhart} (15.\,11.\,1862 Szczawno-Zdrój – 6.\,6.\,1946 Jagniątków), \emph{Schriftsteller}|pwk}. Realisiert werden sollte sie unter der
                  Leitung von Friedrich Elbogen\pwindex{Elbogen, Friedrich 20.\,5.\,1854 Prag – 15.\,4.\,1909 Wien@\textsc{Elbogen, Friedrich} (20.\,5.\,1854 Prag – 15.\,4.\,1909 Wien), \emph{Schriftsteller, Kritiker, Rechtsanwalt}|pwk} von der Wien\oindex{Wien@\textbf{Wien}, \emph{Verwaltungsgebiet}|pwk}er \emph{Freien
                     Bühne}\orgindex{»Freie Bühne« Verein für moderne Literatur@»Freie Bühne« Verein für moderne Literatur|pwk}, bei der Robert Hirschfeld\pwindex{Hirschfeld, Georg 11.\,2.\,1873 Berlin – 17.\,1.\,1942 München@\textsc{Hirschfeld, Georg} (11.\,2.\,1873 Berlin – 17.\,1.\,1942 München), \emph{Schriftsteller}|pwk}
                  und Edmund Wengraf\pwindex{Wengraf, Edmund 9.\,1.\,1860 Mikulov – 8.\,12.\,1933 Wien@\textsc{Wengraf, Edmund} (9.\,1.\,1860 Mikulov – 8.\,12.\,1933 Wien), \emph{Schriftsteller, Journalist, Kaufmann}|pwk} federführend waren. Im
                  nächsten Heft erschien eine ungezeichnete Meldung, die auch von Kraus\pwindex{Kraus, Karl 28.\,4.\,1874 Jičín – 12.\,6.\,1936 Wien@\textsc{Kraus, Karl} (28.\,4.\,1874 Jičín – 12.\,6.\,1936 Wien), \emph{Schriftsteller, Publizist, Schriftsteller}|pwk} stammen dürfte und ausführlicher auf
                  das (nicht verwirklichte) Theatervorhaben eingeht (\emph{[Eine Freie Bühne]}\pwindex{Kraus, Karl 28.\,4.\,1874 Jičín – 12.\,6.\,1936 Wien@\textsc{Kraus, Karl} (28.\,4.\,1874 Jičín – 12.\,6.\,1936 Wien), \emph{Schriftsteller, Publizist, Schriftsteller}!Wiener Freie Bühne]@\strich\emph{[Wiener Freie Bühne]}|pwk}, Nr. 30,
                  S. 484).}}}\label{K_L00244-1}, wie ich eben erſt vor 1 Min. entdeckte, drin, die Sie als
               von einem in dieſen Mittheil.{ }ſehr competenten Blatte{ }\introOben{}aus\introOben{} gewiss \uline{freuen} wird.
               Glückauf! – Hauptmacher der Fr. Bühne\orgindex{»Freie Bühne« Verein für moderne Literatur@»Freie Bühne« Verein für moderne Literatur|pw} iſt ja doch
               die »Wiener Kunst\orgindex{Wiener Kunst@Wiener Kunst|pw}« – Revolverblatt!!!! Redacteur
                  Brehmer\pwindex{Brehmer, Arthur 8.\,2.\,1858 Triest – 1.\,12.\,1923 Eichgraben@\textsc{Brehmer, Arthur} (8.\,2.\,1858 Triest – 1.\,12.\,1923 Eichgraben), \emph{Redakteur}|pw} hat{ }ſich \strikeout{ja} jezt auf \label{K_L00244-2v}\edtext{\uline{4 Monate} zurückgezogen}{\lemma{\textnormal{\emph{4 Monate zurückgezogen}}}\Cendnote{\textnormal{D. h. er wurde zu vier Monaten Arrest verurteilt. [O. V.]:
                        \emph{Vergehen gegen die Sittlichkeit –
                        Schluß}\pwindex{Vergehen gegen die Sittlichkeit (Schluß)@\emph{Vergehen gegen die Sittlichkeit (Schluß)}|pwk}. In: \emph{Neue Freie Presse}\pwindex{Neue Freie Presse@\emph{Neue Freie Presse}|pwk},
                     Nr. 10.388, 25. 7. 1893, S. 6.}}}\label{K_L00244-2}.\pend
           
\pstart
           Was{ }ſagen Sie zu dem Proceſſe, der genialen Rede Elbogens\pwindex{Elbogen, Friedrich 20.\,5.\,1854 Prag – 15.\,4.\,1909 Wien@\textsc{Elbogen, Friedrich} (20.\,5.\,1854 Prag – 15.\,4.\,1909 Wien), \emph{Schriftsteller, Kritiker, Rechtsanwalt}|pw} von der \label{K_L00244-3v}\edtext{Hemmung d.
                  \uline{Naturalismus} (!) i. der Kunſt übhpt.}{\lemma{\textnormal{\emph{Hemmung … übhpt.}}}\Cendnote{\textnormal{In seiner Verteidigung hatte Elbogen\pwindex{Elbogen, Friedrich 20.\,5.\,1854 Prag – 15.\,4.\,1909 Wien@\textsc{Elbogen, Friedrich} (20.\,5.\,1854 Prag – 15.\,4.\,1909 Wien), \emph{Schriftsteller, Kritiker, Rechtsanwalt}|pwk} den größeren Zusammenhang
                  hergestellt: »Es handle sich vielmehr um die Hemmung einer neuen
                     Kunstrichtung, des Naturalismus. \textsc{Principiis obsta}. Wenn
                     Sie diesen Anfängen nicht widerstehen, meine Herren Geschworenen, dann ist es
                     mit aller Kunst und Literatur für alle Zeiten aus und vorbei.«
                     (Vgl. [O. V.]: \emph{Vergehen gegen die
                        Sittlichkeit}\pwindex{Vergehen gegen die Sittlichkeit@\emph{Vergehen gegen die Sittlichkeit}|pwk}. In: \emph{Neue Freie
                        Presse}\pwindex{Neue Freie Presse@\emph{Neue Freie Presse}|pwk}, Nr. 10.387, 24. 7. 1893, S. 3–4, hier
                  S. 4).}}}\label{K_L00244-3}{ }\uline{für alle} Zeiten durch Verbot der »Geſellſchaft\orgindex{Gesellschaft [Wien]@Die Gesellschaft [Wien]|pw}«ſchweinigel.\pend
           
\pstart
           Einakter\pwindex{Schnitzler, Arthur 15.\,5.\,1862 Wien – 21.\,10.\,1931 ebd.@\textsc{Schnitzler, Arthur} (15.\,5.\,1862 Wien – 21.\,10.\,1931 ebd.), \emph{Schriftsteller, Mediziner}!Abschiedssouper@\strich\emph{Abschiedssouper}|pwv} geht flott weiter.
               Heut las ich im B. Börſ.courier\orgindex{Berliner Börsen-Courier@Berliner Börsen-Courier|pw} circa \label{K_L00244-4v}\edtext{40 Zeilen\pwindex{Man schreibt uns aus Ischl]@\emph{[Man schreibt uns aus Ischl]}|pwv}}{\lemma{\textnormal{\emph{40 Zeilen}}}\Cendnote{\textnormal{[O. V.]: \emph{[Man schreibt uns aus Ischl]}\pwindex{Man schreibt uns aus Ischl]@\emph{[Man schreibt uns aus Ischl]}|pwk}.
                     In: \emph{Berliner Börsen-Courier}\pwindex{Berliner Börsen-Courier@\emph{Berliner Börsen-Courier}|pwk}, Nr. 343,
                        25. 7. 1893, Morgen-Ausgabe, S. 4.}}}\label{K_L00244-4} über Abſchiedssouper\pwindex{Schnitzler, Arthur 15.\,5.\,1862 Wien – 21.\,10.\,1931 ebd.@\textsc{Schnitzler, Arthur} (15.\,5.\,1862 Wien – 21.\,10.\,1931 ebd.), \emph{Schriftsteller, Mediziner}!Abschiedssouper@\strich\emph{Abschiedssouper}|pw}{ }\uline{gelesen}? Darf ich, daſs \uline{Abschiedss.\pwindex{Schnitzler, Arthur 15.\,5.\,1862 Wien – 21.\,10.\,1931 ebd.@\textsc{Schnitzler, Arthur} (15.\,5.\,1862 Wien – 21.\,10.\,1931 ebd.), \emph{Schriftsteller, Mediziner}!Abschiedssouper@\strich\emph{Abschiedssouper}|pw}} im Residenz\oindex{Residenztheater München@\textbf{Residenztheater München}, \emph{Theater}|pw} angenommen ist, im Magazin\orgindex{Magazin für die Literatur des Auslandes@Magazin für die Literatur des Auslandes|pw}{ }\label{K_L00244-5v}\edtext{publicieren}{\lemma{\textnormal{\emph{publicieren}}}\Cendnote{\textnormal{nicht erschienen}}}\label{K_L00244-5}? 1000 Grüße Ihr \spacefill\mbox{Kraus}\pend
           
\pstart
           \noindent{}Schicken Sie Ihr Drama\pwindex{Schnitzler, Arthur 15.\,5.\,1862 Wien – 21.\,10.\,1931 ebd.@\textsc{Schnitzler, Arthur} (15.\,5.\,1862 Wien – 21.\,10.\,1931 ebd.), \emph{Schriftsteller, Mediziner}!Märchen. Schauspiel in drei Aufzügen@\strich\emph{Das Märchen. Schauspiel in drei Aufzügen}|pwv}
                  hin!!\pend
           \selectlanguage{ngerman}\endnumbering\briefempfaengerindex{Schnitzler, Arthur@\textsc{Schnitzler, Arthur}!zzzKraus, Karl@\emph{von Karl Kraus}!1893-07-271@{27. 7. 1893}|)be}\mylabel{L00244h}  \newcommand{\dateiname}{L00244}\newcommand{\titel}{Karl Kraus an Arthur Schnitzler, 27. 7. 1893}\newcommand{\editorInnen}{Martin Anton Müller und Gerd-Hermann Susen}%% latex-leseansicht-abspann.tex
%% Abspann für die Leseansicht.
%% Der Schalter \ifkorrekturansicht ist bereits durch den Vorspann gesetzt.

%% latex-abspann.tex
%% Gemeinsamer Abspann für Korrekturansicht und Leseansicht.
%% Setzt den Schalter \ifkorrekturansicht voraus (gesetzt in den
%% einbindenden Dateien latex-korrekturansicht-abspann.tex bzw.
%% latex-leseansicht-abspann.tex).
%% ---------------------------------------------------------------

\normalsize

% Das esempio-Environment wird nur in der Leseansicht benötigt
\ifkorrekturansicht\else
\newenvironment{esempio}[3]%
{
    \vspace{1.5ex}
    \rlap{\underline{#1}}
    \par
    \setlength{\parindent}{0cm}
    \nopagebreak
    \leftskip=#2cm
    \rightskip=#3cm
}
{
    \par
}
\fi

\doendnotes{C}
\bigskip
\vfill

\clearpage

\footnotesize

\ifkorrekturansicht
  \lohead{\textsc{register}}
\fi

% theindex-Environment neu definieren ohne reledmac
\makeatletter
\renewenvironment{theindex}{%
  \ifkorrekturansicht
    \section*{\indexname}%
  \else
    \subsubsection*{Index der erwähnten Entitäten}%
  \fi
  \setlength{\parindent}{0pt}%
  \setlength{\parskip}{0pt plus 0.3pt}%
  \let\item\@idxitem
}{%
  \ifkorrekturansicht\clearpage\fi
}
\makeatother

\IfFileExists{\jobname-pw.ind}{\input{\jobname-pw.ind}}{}

% Quellenangabe nur in der Leseansicht
\ifkorrekturansicht\else
% Fallback-Definitionen, falls die .tex-Datei \titel etc. nicht gesetzt hat
\providecommand{\titel}{}
\providecommand{\editorInnen}{}
\providecommand{\dateiname}{\jobname}

\vspace{3cm}

\vfill

\footnotesize
\textsc{Quelle}: \titel. Herausgegeben von {\editorInnen}. In: \emph{Arthur Schnitzler: Briefwechsel mit Autorinnen und Autoren}.
 Digitale Edition, https://schnitzler-briefe.acdh.oeaw.ac.at/{\dateiname}.html (Stand \today)
\fi

\end{document}


