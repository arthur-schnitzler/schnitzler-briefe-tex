%% latex-leseansicht-vorspann.tex
%% Vorspann für die Leseansicht.
%% Lädt die gemeinsame Datei latex-vorspann.tex mit nicht gesetztem Schalter.

\newif\ifkorrekturansicht
\korrekturansichtfalse

\input{../tex-inputs/latex-vorspann}


         
         \renewcommand{\erwaehntePersonen}{Personen: Arthur Brehmer, Moriz Ehrenfeld, Friedrich Elbogen, Hanns von Gumppenberg, Gerhart Hauptmann, Robert Hirschfeld, Georg Hirschfeld, Louise Sigert, Edmund Wengraf}
         \renewcommand{\erwaehnteInstitutionen}{Institutionen: Berliner Börsen-Courier, Die Gesellschaft [Wien], Magazin für die Literatur des Auslandes, Wiener Kunst, »Freie Bühne« Verein für moderne Literatur}
         \renewcommand{\erwaehnteOrte}{Orte: Bad Ischl, Grillparzerstraße, Residenztheater München, Wien}
         \renewcommand{\erwaehnteWerke}{Werke: Abschiedssouper, Auferstanden{\rufezeichen} Drama in einem Vorspiele und drei Acten, Berliner Börsen-Courier, Das Märchen. Schauspiel in drei Aufzügen, Die Weber. Schauspiel aus den vierziger Jahren, Magazin für die Literatur des Auslandes, Neue Freie Presse, Vergehen gegen die Sittlichkeit, Vergehen gegen die Sittlichkeit (Schluß), Wiener Theater. – Luise Sigert. Auferstanden{\rufezeichen}, [Man schreibt uns aus Ischl], [Wiener Freie Bühne]}
               \section[Karl Kraus an Arthur Schnitzler, 27. 7. 1893]{ Karl Kraus an Arthur Schnitzler, 27. 7. 1893}\nopagebreak\mylabel{v}\rehead{ }\begin{ledgroupsized}[t]{13cm}\normalsize\beginnumbering \toendnotes[C]{\smallbreak\pagebreak[2]} \Standort{CUL, Schnitzler, B 55.}
\physDesc{Postkarte, 865 Zeichen
\newline{}Handschrift: schwarze Tinte, deutsche Kurrent
\newline{}Versand: Stempel: »\nobreak{}\oindex{Bad Ischl@\textbf{Bad Ischl}|pwk}Ischl, 27/7 93, 1–N\nobreak{}«.  
\newline{}Schnitzler: mit Bleistift seitlich des Textes neben die »Fr. Bühne\orgindex{»Freie Buehne« Verein fuer moderne Literatur@»Freie Bühne« Verein für moderne Literatur|pw}«: »|| \textsc{Hirschfeld\pwindex{Hirschfeld, Robert 17.09.1857 – 02.04.1914@\textsc{Hirschfeld, Robert} (17.09.1857 – 02.04.1914), \emph{Journalist, Musikkritiker}|pw}–Wengraf\pwindex{Wengraf, Edmund 09.01.1860 – 08.12.1933@\textsc{Wengraf, Edmund} (09.01.1860 – 08.12.1933), \emph{Journalist}|pw} –
                                          \textcolor{gray}{frei}? ||}« }\buchAbdrucke{\weitereDrucke{\emph{Karl Kraus und Arthur Schnitzler. Eine Dokumentation.} Hg. Reinhard Urbach. In: \emph{Literatur und Kritik}, Bd. 49, Oktober 1970, S. 519–520.} }\toendnotes[C]{\smallbreak}\pstart{}{\pb}Herrn Doktor Arthur
                  Schnitzler,\pend{}\pstart{}Schriftſteller\pend{}\pstart{}I. Grillparzerstr. 7\oindex{Grillparzerstrasse@\textbf{Grillparzerstraße}|pw}\pend{}\pstart{}Wien\oindex{Wien@\textbf{Wien}|pw}\pend{}{\bigskip}\pstart
           \noindent{}{\pb}\uline{Innigſten Dank}, liebſter Doktor,  für den
               lieben Brief! Beifolgend das letzte \uline{Magazin\pwindex{?? Werk@Nicht ermittelte Verfasserinnen und Verfasser!Magazin fuer die Literatur des Auslandes1832 – 1915@\emph{Magazin für die Literatur des Auslandes} {[}1832 – 1915{]}|pw}}, das ich erſt heute bekam; es ſteht eine \label{K_L00244-1v}\edtext{Nachricht\pwindex{Kraus, Karl 28.04.1874 – 12.06.1936@\textsc{Kraus, Karl} (28.04.1874 – 12.06.1936), \emph{Schriftsteller, Publizist}!Wiener Theater. – Luise Sigert. Auferstanden22. 07. 1893@\strich\emph{Wiener Theater. – Luise Sigert. Auferstanden{\rufezeichen}} {[}22. 07. 1893{]}|pwv}}{\lemma{\textnormal{\emph{Nachricht}}}\Cendnote{\textnormal{Diese Karte bezieht sich auf ein
                  Gerichtsverfahren, das am 24. und 25. 7. 1893 in Wien\oindex{Wien@\textbf{Wien}|pwk} wegen sexuell zu expliziter
                  Veröffentlichungen in einer Wochenschrift namens \emph{Gesellschaft}\orgindex{Gesellschaft [Wien]@Die Gesellschaft [Wien]|pwk} verhandelt wurde. Dabei wurden Moriz Ehrenfeld\pwindex{Ehrenfeld, Moriz 29.09.1869 – 13.09.1900@\textsc{Ehrenfeld, Moriz} (29.09.1869 – 13.09.1900), \emph{Schriftsteller}|pwk}, Ferdinand
                     Mautner\pwindex{Gumppenberg, Hanns von 04.12.1866 – 29.03.1928@\textsc{Gumppenberg, Hanns von} (04.12.1866 – 29.03.1928), \emph{Schriftsteller, Kritiker}|pwk} und Alfred Brehmer\pwindex{Brehmer, Arthur 08.02.1858 – 01.12.1923@\textsc{Brehmer, Arthur} (08.02.1858 – 01.12.1923), \emph{Redakteur}|pwk} zu
                  mehrmonatigen Haftstrafen verurteilt. Verteidigt wurden die letzteren beiden von
                     Friedrich Elbogen\pwindex{Elbogen, Friedrich 20.05.1854 – 15.04.1909@\textsc{Elbogen, Friedrich} (20.05.1854 – 15.04.1909), \emph{Schriftsteller, Kritiker, Rechtsanwalt}|pwk}. Mit Brehmer\pwindex{Brehmer, Arthur 08.02.1858 – 01.12.1923@\textsc{Brehmer, Arthur} (08.02.1858 – 01.12.1923), \emph{Redakteur}|pwk} gibt es dabei eine Überlappung zu
                  einer weiteren Zeitschrift, \emph{Wiener Kunst}\orgindex{Wiener Kunst@Wiener Kunst|pwk},
                  wobei beide Zeitschriften nicht erhalten sind. Der Konnex, den Kraus\pwindex{Kraus, Karl 28.04.1874 – 12.06.1936@\textsc{Kraus, Karl} (28.04.1874 – 12.06.1936), \emph{Schriftsteller, Publizist}|pwk} herstellt, bezieht sich auf den letzten Absatz seines
                  Theaterbriefs, erschienen am 22. 7. 1893; in \emph{Wiener Theater. – Luise
                        Sigert\pwindex{Sigert, Louise *~25.05.1871@\textsc{Sigert, Louise} (*~25.05.1871), \emph{Schriftstellerin}|pwk}. \emph{Auferstanden!}\pwindex{Sigert, Louise *~25.05.1871@\textsc{Sigert, Louise} (*~25.05.1871), \emph{Schriftstellerin}!Auferstanden Drama in einem Vorspiele und drei Acten1892@\strich\emph{Auferstanden{\rufezeichen} Drama in einem Vorspiele und drei Acten} {[}1892{]}|pwk}}\pwindex{Kraus, Karl 28.04.1874 – 12.06.1936@\textsc{Kraus, Karl} (28.04.1874 – 12.06.1936), \emph{Schriftsteller, Publizist}!Wiener Theater. – Luise Sigert. Auferstanden22. 07. 1893@\strich\emph{Wiener Theater. – Luise Sigert. Auferstanden{\rufezeichen}} {[}22. 07. 1893{]}|pwk} (\emph{Das Magazin für Litteratur}\pwindex{?? Werk@Nicht ermittelte Verfasserinnen und Verfasser!Magazin fuer die Literatur des Auslandes1832 – 1915@\emph{Magazin für die Literatur des Auslandes} {[}1832 – 1915{]}|pwk}, Jg. 62,
                     Nr. 29, S. 466–467.) endet Kraus\pwindex{Kraus, Karl 28.04.1874 – 12.06.1936@\textsc{Kraus, Karl} (28.04.1874 – 12.06.1936), \emph{Schriftsteller, Publizist}|pwk}
                  mit einer Kritik an der Zeitschrift \emph{Wiener
                     Kunst}\orgindex{Wiener Kunst@Wiener Kunst|pwk} und erwähnt eine geplante Musteraufführung von \emph{Die Weber}\pwindex{Hauptmann, Gerhart 15.11.1862 – 06.06.1946@\textsc{Hauptmann, Gerhart} (15.11.1862 – 06.06.1946), \emph{Schriftsteller}!Weber. Schauspiel aus den vierziger Jahren1892@\strich\emph{Die Weber. Schauspiel aus den vierziger Jahren} {[}1892{]}|pwk} von Gerhart
                     Hauptmann\pwindex{Hauptmann, Gerhart 15.11.1862 – 06.06.1946@\textsc{Hauptmann, Gerhart} (15.11.1862 – 06.06.1946), \emph{Schriftsteller}|pwk}. Die Wien\oindex{Wien@\textbf{Wien}|pwk}er \emph{Freie Bühne}\orgindex{»Freie Buehne« Verein fuer moderne Literatur@»Freie Bühne« Verein für moderne Literatur|pwk}, bei der unter anderem auch Robert Hirschfeld\pwindex{Hirschfeld, Georg 11.02.1873 – 17.01.1942@\textsc{Hirschfeld, Georg} (11.02.1873 – 17.01.1942), \emph{Schriftsteller}|pwk} und Edmund Wengraf\pwindex{Wengraf, Edmund 09.01.1860 – 08.12.1933@\textsc{Wengraf, Edmund} (09.01.1860 – 08.12.1933), \emph{Journalist}|pwk} federführend waren, sollte nunmehr unter der
                  Führung von dem Verteidiger Elbogen\pwindex{Elbogen, Friedrich 20.05.1854 – 15.04.1909@\textsc{Elbogen, Friedrich} (20.05.1854 – 15.04.1909), \emph{Schriftsteller, Kritiker, Rechtsanwalt}|pwk}
                  umgesetzt werden. Im nächsten Heft erschien eine ungezeichnete Meldung, die auch
                  von Kraus\pwindex{Kraus, Karl 28.04.1874 – 12.06.1936@\textsc{Kraus, Karl} (28.04.1874 – 12.06.1936), \emph{Schriftsteller, Publizist}|pwk} stammen dürfte und ausführlicher
                  auf das (nicht verwirklichte) Theatervorhaben eingeht (\emph{[Eine Freie Bühne]}\pwindex{Wiener Freie Buehne]29. 07. 1893@\emph{[Wiener Freie Bühne]} {[}29. 07. 1893{]}|pwk}, Nr. 30,
                  S. 484).}}}\label{K_L00244-1h}, wie ich eben erſt vor 1 Min. entdeckte, drin, die Sie als
               von einem in dieſen Mittheil. ſehr competenten Blatte{ }\introOben{}aus\introOben{} gewiss \uline{freuen} wird.
               Glückauf! – Hauptmacher der Fr. Bühne\orgindex{»Freie Buehne« Verein fuer moderne Literatur@»Freie Bühne« Verein für moderne Literatur|pw} iſt ja doch
               die »Wiener Kunst\orgindex{Wiener Kunst@Wiener Kunst|pw}« – Revolverblatt!!!! Redacteur
                  Brehmer\pwindex{Brehmer, Arthur 08.02.1858 – 01.12.1923@\textsc{Brehmer, Arthur} (08.02.1858 – 01.12.1923), \emph{Redakteur}|pw} hat ſich \strikeout{ja} jezt auf \label{K_L00244-2v}\edtext{\uline{4 Monate} zurückgezogen}{\lemma{\textnormal{\emph{4 Monate zurückgezogen}}}\Cendnote{\textnormal{D. h. er wurde zu vier Monaten Arrest verurteilt ([O. V.]:
                        \emph{Vergehen gegen die Sittlichkeit –
                        Schluß}\pwindex{?? Werk@Nicht ermittelte Verfasserinnen und Verfasser!Vergehen gegen die Sittlichkeit (Schluss)25. 07. 1893@\emph{Vergehen gegen die Sittlichkeit (Schluß)} {[}25. 07. 1893{]}|pwk}. In: \emph{Neue Freie Presse}\pwindex{Neue Freie Presse1864 – 1939@\emph{Neue Freie Presse} {[}1864 – 1939{]}|pwk},
                     Nr. 10.388, 25. 7. 1893, S. 6).}}}\label{K_L00244-2h}.\pend
           \pstart
           Was ſagen Sie zu dem Proceſſe, der genialen Rede Elbogens\pwindex{Elbogen, Friedrich 20.05.1854 – 15.04.1909@\textsc{Elbogen, Friedrich} (20.05.1854 – 15.04.1909), \emph{Schriftsteller, Kritiker, Rechtsanwalt}|pw} von der \label{K_L00244-3v}\edtext{Hemmung d.
                  \uline{Naturalismus} (!) i. der Kunſt übhpt.}{\lemma{\textnormal{\emph{Hemmung … übhpt.}}}\Cendnote{\textnormal{In seiner Verteidigung hatte Elbogen\pwindex{Elbogen, Friedrich 20.05.1854 – 15.04.1909@\textsc{Elbogen, Friedrich} (20.05.1854 – 15.04.1909), \emph{Schriftsteller, Kritiker, Rechtsanwalt}|pwk} den größeren Zusammenhang
                  hergestellt: »Es handle sich vielmehr um die Hemmung einer neuen
                     Kunstrichtung, des Naturalismus. \textsc{Principiis obsta}. Wenn
                     Sie diesen Anfängen nicht widerstehen, meine Herren Geschworenen, dann ist es
                     mit aller Kunst und Literatur für alle Zeiten aus und vorbei.«
                     (Vgl. [O. V.]: \emph{Vergehen gegen die
                        Sittlichkeit}\pwindex{?? Werk@Nicht ermittelte Verfasserinnen und Verfasser!Vergehen gegen die Sittlichkeit24. 07. 1893@\emph{Vergehen gegen die Sittlichkeit} {[}24. 07. 1893{]}|pwk}. In: \emph{Neue Freie
                        Presse}\pwindex{Neue Freie Presse1864 – 1939@\emph{Neue Freie Presse} {[}1864 – 1939{]}|pwk}, Nr. 10.387, 24. 7. 1893, S. 3–4, hier
                  S. 4).}}}\label{K_L00244-3h}{ }\uline{für alle} Zeiten durch Verbot der »Geſellſchaft\orgindex{Gesellschaft [Wien]@Die Gesellschaft [Wien]|pw}«ſchweinigel.\pend
           \pstart
           Einakter\pwindex{Schnitzler, Arthur 15.05.1862 – 21.10.1931@\textsc{Schnitzler, Arthur} (15.05.1862 – 21.10.1931), \emph{Schriftsteller, Mediziner}!Abschiedssouper1892@\strich\emph{Abschiedssouper} {[}1892{]}|pwv} geht flott weiter.
               Heut las ich im B. Börſ.courier\orgindex{Berliner Boersen-Courier@Berliner Börsen-Courier|pw} circa \label{K_L00244-4v}\edtext{40 Zeilen\pwindex{?? Werk@Nicht ermittelte Verfasserinnen und Verfasser!Man schreibt uns aus Ischl]25. 07. 1893@\emph{[Man schreibt uns aus Ischl]} {[}25. 07. 1893{]}|pwv}}{\lemma{\textnormal{\emph{40 Zeilen}}}\Cendnote{\textnormal{[O. V.]: \emph{[Man schreibt uns aus Ischl]}\pwindex{?? Werk@Nicht ermittelte Verfasserinnen und Verfasser!Man schreibt uns aus Ischl]25. 07. 1893@\emph{[Man schreibt uns aus Ischl]} {[}25. 07. 1893{]}|pwk}.
                     In: \emph{Berliner Börsen-Courier}\pwindex{?? Werk@Nicht ermittelte Verfasserinnen und Verfasser!Berliner Boersen-Courier1868 – 1933@\emph{Berliner Börsen-Courier} {[}1868 – 1933{]}|pwk}, Nr. 343,
                        25. 7. 1893, Morgen-Ausgabe, S. 4.}}}\label{K_L00244-4h} über Abſchiedssouper\pwindex{Schnitzler, Arthur 15.05.1862 – 21.10.1931@\textsc{Schnitzler, Arthur} (15.05.1862 – 21.10.1931), \emph{Schriftsteller, Mediziner}!Abschiedssouper1892@\strich\emph{Abschiedssouper} {[}1892{]}|pw}{ }\uline{gelesen}? Darf ich, daſs \uline{Abschiedss.\pwindex{Schnitzler, Arthur 15.05.1862 – 21.10.1931@\textsc{Schnitzler, Arthur} (15.05.1862 – 21.10.1931), \emph{Schriftsteller, Mediziner}!Abschiedssouper1892@\strich\emph{Abschiedssouper} {[}1892{]}|pw}} im Residenz\oindex{Residenztheater Muenchen@\textbf{Residenztheater München}|pw} angenommen ist, im Magazin\orgindex{Magazin fuer die Literatur des Auslandes@Magazin für die Literatur des Auslandes|pw}{ }\label{K_L00244-5v}\edtext{publicieren}{\lemma{\textnormal{\emph{publicieren}}}\Cendnote{\textnormal{nicht erschienen}}}\label{K_L00244-5h}? 1000 Grüße Ihr \spacefill\mbox{Kraus}\pend
           \pstart
           \noindent{}Schicken Sie Ihr Drama\pwindex{Schnitzler, Arthur 15.05.1862 – 21.10.1931@\textsc{Schnitzler, Arthur} (15.05.1862 – 21.10.1931), \emph{Schriftsteller, Mediziner}!Maerchen. Schauspiel in drei Aufzuegen1893-12-01@\strich\emph{Das Märchen. Schauspiel in drei Aufzügen} {[}1893-12-01{]}|pwv}
                  hin!!\pend
           
         
         \endnumbering\mylabel{h}\end{ledgroupsized}  \newcommand{\dateiname}{L00244}\newcommand{\titel}{Karl Kraus an Arthur Schnitzler, 27. 7. 1893}\newcommand{\editorInnen}{Martin Anton Müller und Gerd-Hermann Susen}%% latex-leseansicht-abspann.tex
%% Abspann für die Leseansicht.
%% Der Schalter \ifkorrekturansicht ist bereits durch den Vorspann gesetzt.

%% latex-abspann.tex
%% Gemeinsamer Abspann für Korrekturansicht und Leseansicht.
%% Setzt den Schalter \ifkorrekturansicht voraus (gesetzt in den
%% einbindenden Dateien latex-korrekturansicht-abspann.tex bzw.
%% latex-leseansicht-abspann.tex).
%% ---------------------------------------------------------------

\normalsize

% Das esempio-Environment wird nur in der Leseansicht benötigt
\ifkorrekturansicht\else
\newenvironment{esempio}[3]%
{
    \vspace{1.5ex}
    \rlap{\underline{#1}}
    \par
    \setlength{\parindent}{0cm}
    \nopagebreak
    \leftskip=#2cm
    \rightskip=#3cm
}
{
    \par
}
\fi

\doendnotes{C}
\bigskip
\vfill

\clearpage

\footnotesize

\ifkorrekturansicht
  \lohead{\textsc{register}}
\fi

% theindex-Environment neu definieren ohne reledmac
\makeatletter
\renewenvironment{theindex}{%
  \ifkorrekturansicht
    \section*{\indexname}%
  \else
    \subsubsection*{Index der erwähnten Entitäten}%
  \fi
  \setlength{\parindent}{0pt}%
  \setlength{\parskip}{0pt plus 0.3pt}%
  \let\item\@idxitem
}{%
  \ifkorrekturansicht\clearpage\fi
}
\makeatother

\IfFileExists{\jobname-pw.ind}{\input{\jobname-pw.ind}}{}

% Quellenangabe nur in der Leseansicht
\ifkorrekturansicht\else
% Fallback-Definitionen, falls die .tex-Datei \titel etc. nicht gesetzt hat
\providecommand{\titel}{}
\providecommand{\editorInnen}{}
\providecommand{\dateiname}{\jobname}

\vspace{3cm}

\vfill

\footnotesize
\textsc{Quelle}: \titel. Herausgegeben von {\editorInnen}. In: \emph{Arthur Schnitzler: Briefwechsel mit Autorinnen und Autoren}.
 Digitale Edition, https://schnitzler-briefe.acdh.oeaw.ac.at/{\dateiname}.html (Stand \today)
\fi

\end{document}


      