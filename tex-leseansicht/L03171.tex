%% latex-leseansicht-vorspann.tex
%% Vorspann für die Leseansicht.
%% Lädt die gemeinsame Datei latex-vorspann.tex mit nicht gesetztem Schalter.

\newif\ifkorrekturansicht
\korrekturansichtfalse

\input{../tex-inputs/latex-vorspann}


\section[ Felix Salten an Arthur Schnitzler, [27. 4. 1896]]{L03171 Felix Salten an Arthur Schnitzler,  [27. 4. 1896]}
\nopagebreak\mylabel{L03171v}
\rehead{ }\normalsize\beginnumbering\briefempfaengerindex{Schnitzler, Arthur@\textsc{Schnitzler, Arthur}!zzzSalten, Felix@\emph{von Felix Salten}!1896-04-271@{{[}27. 4. 1896{]}}|(be}
\toendnotes[C]{\smallbreak\pagebreak[2]}
\correspDesc{Versand  durch Felix Salten am [27. 4. 1896] in Wien
\newline{}Erhalt  durch Arthur Schnitzler im Zeitraum [27. 4. 1896
                  – 1. 5. 1896?] in Wien}\toendnotes[C]{\smallbreak}
\Standort{CUL, Schnitzler, B 89, A 1.}
\physDesc{Brief, 1 Blatt, 1 Seite, 149 Zeichen
\newline{}Handschrift: Bleistift, lateinische Kurrent
\newline{}Schnitzler: mit Bleistift auf der Vorlage datiert: »27/4 \textcolor{gray}{\textbf{189}}6« 
\newline{}Ordnung: mit Bleistift von unbekannter Hand nummeriert: »70« }\toendnotes[C]{\smallbreak}
\pstart
           {\pb}\textcolor{gray}{\textbf{\textbf{»Wiener Allgemeine
                        Zeitung\orgindex{Wiener Allgemeine Zeitung@Wiener Allgemeine Zeitung|pw}«}}}\pend
           
\pstart
           \textcolor{gray}{\textbf{Redaction:}}\pend
           
\pstart
           \textcolor{gray}{\textbf{\textbf{IX/3, Univerſitätsſtraße Nr. 6\oindex{Wien@\textbf{Wien}!IX., Alsergrund@\textbf{IX., Alsergrund}!Universitätsstraße@\textbf{Universitätsstraße}, \emph{Straße}|pw}\oindex{Wien@\textbf{Wien}!I., Innere Stadt@\textbf{I., Innere Stadt}!Universitätsstraße@\textbf{Universitätsstraße}, \emph{Straße}|pw}.}}}\pend
           
\pstart
           \textcolor{gray}{\textbf{Adminiſtration:}}\hfill \textcolor{gray}{\textbf{Wien\oindex{Wien@\textbf{Wien}, \emph{Verwaltungsgebiet}|pw}, am ..........{ }189{\dots}}}\pend
           
\pstart
           \textcolor{gray}{\textbf{\textbf{I. Wollzeile Nr. 5\oindex{Wien@\textbf{Wien}!I., Innere Stadt@\textbf{I., Innere Stadt}!Wollzeile@\textbf{Wollzeile}, \emph{Straße}|pw}} (im Durchhauſe).}}\pend
           
\pstart
           \textcolor{gray}{\textbf{Telegramm-Adreſſe: »Allgemeine, Wien\oindex{Wien@\textbf{Wien}, \emph{Verwaltungsgebiet}|pw}«.}}\pend
           
\pstart
           \textcolor{gray}{\textbf{Telephon der Redaction: Nr. 805 u. 2180.}}\pend
           
\pstart
           \textcolor{gray}{\textbf{\hspace*{1.5em}„\hspace*{2.5em}„\hspace*{1.5em} Adminiſtration: Nr. 1024.}}\pend
           \vspace{0.5em}
\pstart
           lieber Arthur.{ }Ludaßy\pwindex{Gans-Ludassy, Julius von 13.\,4.\,1858 Wien – 30.\,9.\,1922 ebd.@\textsc{Gans-Ludassy, Julius von} (13.\,4.\,1858 Wien – 30.\,9.\,1922 ebd.), \emph{Schriftsteller, Journalist, Herausgeber}|pw} hat die \label{K_L03171-1v}\edtext{Loge\oindex{Wien@\textbf{Wien}!I., Innere Stadt@\textbf{I., Innere Stadt}!Burgtheater@\textbf{Burgtheater}, \emph{Theater}|pwv}}{\lemma{\textnormal{\emph{Loge}}}\Cendnote{\textnormal{Für die Aufführung von Victorien Sardous\pwindex{Sardou, Victorien 7.\,9.\,1831 Paris – 8.\,11.\,1908 ebd.@\textsc{Sardou, Victorien} (7.\,9.\,1831 Paris – 8.\,11.\,1908 ebd.), \emph{Schriftsteller}|pwk}{ }\emph{Die alten Junggesellen}\pwindex{Sardou, Victorien 7.\,9.\,1831 Paris – 8.\,11.\,1908 ebd.@\textsc{Sardou, Victorien} (7.\,9.\,1831 Paris – 8.\,11.\,1908 ebd.), \emph{Schriftsteller}!alten Junggesellen. Pariser Sittengemälde in 5 Aufzügen@\strich\emph{Die alten Junggesellen. Pariser Sittengemälde in 5 Aufzügen}|pwk} im Burgtheater\oindex{Wien@\textbf{Wien}!I., Innere Stadt@\textbf{I., Innere Stadt}!Burgtheater@\textbf{Burgtheater}, \emph{Theater}|pwk}. Schnitzler besuchte 
                  die Aufführung trotzdem, vgl. A. S.: \emph{Tagebuch}, 27. 4. 1896. 
               }}}\label{K_L03171-1} im letzten Momente \label{K_L03171-2v}\edtext{mit
               Beschlag gelegt}{\lemma{\textnormal{\emph{mit
               Beschlag gelegt}}}\Cendnote{\textnormal{für sich beansprucht}}}\label{K_L03171-2}.\pend
           
\pstart
           Ich werde heute im Griensteidl\oindex{Wien@\textbf{Wien}!I., Innere Stadt@\textbf{I., Innere Stadt}!Café Griensteidl@\textbf{Café Griensteidl}, \emph{Kaffeehaus}|pw} sein. Gegen die Loge\oindex{Wien@\textbf{Wien}!I., Innere Stadt@\textbf{I., Innere Stadt}!Burgtheater@\textbf{Burgtheater}, \emph{Theater}|pwv} kann ich nichts machen.\pend
           \pstart Ihr \spacefill\mbox{Salten.}\pend{}\selectlanguage{ngerman}\endnumbering\briefempfaengerindex{Schnitzler, Arthur@\textsc{Schnitzler, Arthur}!zzzSalten, Felix@\emph{von Felix Salten}!1896-04-271@{{[}27. 4. 1896{]}}|)be}\mylabel{L03171h}  \newcommand{\dateiname}{L03171}\newcommand{\titel}{Felix Salten an Arthur Schnitzler, [27. 4. 1896]}\newcommand{\editorInnen}{Martin Anton Müller und Laura Untner}%% latex-leseansicht-abspann.tex
%% Abspann für die Leseansicht.
%% Der Schalter \ifkorrekturansicht ist bereits durch den Vorspann gesetzt.

%% latex-abspann.tex
%% Gemeinsamer Abspann für Korrekturansicht und Leseansicht.
%% Setzt den Schalter \ifkorrekturansicht voraus (gesetzt in den
%% einbindenden Dateien latex-korrekturansicht-abspann.tex bzw.
%% latex-leseansicht-abspann.tex).
%% ---------------------------------------------------------------

\normalsize

% Das esempio-Environment wird nur in der Leseansicht benötigt
\ifkorrekturansicht\else
\newenvironment{esempio}[3]%
{
    \vspace{1.5ex}
    \rlap{\underline{#1}}
    \par
    \setlength{\parindent}{0cm}
    \nopagebreak
    \leftskip=#2cm
    \rightskip=#3cm
}
{
    \par
}
\fi

\doendnotes{C}
\bigskip
\vfill

\clearpage

\footnotesize

\ifkorrekturansicht
  \lohead{\textsc{register}}
\fi

% theindex-Environment neu definieren ohne reledmac
\makeatletter
\renewenvironment{theindex}{%
  \ifkorrekturansicht
    \section*{\indexname}%
  \else
    \subsubsection*{Index der erwähnten Entitäten}%
  \fi
  \setlength{\parindent}{0pt}%
  \setlength{\parskip}{0pt plus 0.3pt}%
  \let\item\@idxitem
}{%
  \ifkorrekturansicht\clearpage\fi
}
\makeatother

\IfFileExists{\jobname-pw.ind}{\input{\jobname-pw.ind}}{}

% Quellenangabe nur in der Leseansicht
\ifkorrekturansicht\else
% Fallback-Definitionen, falls die .tex-Datei \titel etc. nicht gesetzt hat
\providecommand{\titel}{}
\providecommand{\editorInnen}{}
\providecommand{\dateiname}{\jobname}

\vspace{3cm}

\vfill

\footnotesize
\textsc{Quelle}: \titel. Herausgegeben von {\editorInnen}. In: \emph{Arthur Schnitzler: Briefwechsel mit Autorinnen und Autoren}.
 Digitale Edition, https://schnitzler-briefe.acdh.oeaw.ac.at/{\dateiname}.html (Stand \today)
\fi

\end{document}


