%% latex-korrekturansicht-vorspann.tex
%% Vorspann für die Korrekturansicht.
%% Lädt die gemeinsame Datei latex-vorspann.tex mit gesetztem Schalter.

\newif\ifkorrekturansicht
\korrekturansichttrue

\input{../tex-inputs/latex-vorspann}


\section[ Felix Salten an Arthur Schnitzler, {[}27. 4. 1896{]}]{L03171 Felix Salten an Arthur Schnitzler, {[}27. 4. 1896{]}}
\nopagebreak\mylabel{L03171v}
\rehead{ }\normalsize\beginnumbering\briefempfaengerindex{Schnitzler, Arthur@\textsc{Schnitzler, Arthur}!zzzSalten, Felix@\emph{von Felix Salten}!1896-04-271@{{[}27. 4. 1896{]}}|(be}
\toendnotes[C]{\smallbreak\pagebreak[2]}\Standort{CUL, Schnitzler, B 89, A 1.}
\physDesc{Brief, 1 Blatt, 1 Seite, 149 Zeichen
\newline{}Handschrift: Bleistift, lateinische Kurrent
\newline{}Schnitzler: mit Bleistift auf der Vorlage datiert: »27/4 \textcolor{gray}{\textbf{189}}6« 
\newline{}Ordnung: mit Bleistift von unbekannter Hand nummeriert: »70« }\toendnotes[C]{\smallbreak}
\pstart
           {\pb}\textcolor{gray}{\textbf{\textbf{»Wiener Allgemeine
                        Zeitung\orgindex{Wiener Allgemeine Zeitung@Wiener Allgemeine Zeitung|pw}«}}}\pend
           
\pstart
           \textcolor{gray}{\textbf{Redaction:}}\pend
           
\pstart
           \textcolor{gray}{\textbf{\textbf{IX/3, Univerſitätsſtraße Nr. 6\oindex{Universitaetsstrasse@\textbf{Universitätsstraße}, \emph{Straße (K.STR)}|pw}.}}}\pend
           
\pstart
           \textcolor{gray}{\textbf{Adminiſtration:}}\hfill \textcolor{gray}{\textbf{Wien\oindex{Wien@\textbf{Wien}, \emph{A.ADM2}|pw}, am ..........{ }189{\dots}}}\pend
           
\pstart
           \textcolor{gray}{\textbf{\textbf{I. Wollzeile Nr. 5\oindex{Wollzeile@\textbf{Wollzeile}, \emph{Straße (K.STR)}|pw}} (im Durchhauſe).}}\pend
           
\pstart
           \textcolor{gray}{\textbf{Telegramm-Adreſſe: »Allgemeine, Wien\oindex{Wien@\textbf{Wien}, \emph{A.ADM2}|pw}«.}}\pend
           
\pstart
           \textcolor{gray}{\textbf{Telephon der Redaction: Nr. 805 u. 2180.}}\pend
           
\pstart
           \textcolor{gray}{\textbf{\hspace*{1.5em}„\hspace*{2.5em}„\hspace*{1.5em} Adminiſtration: Nr. 1024.}}\pend
           \vspace{0.5em}
\pstart
           lieber Arthur.{ }Ludaßy\pwindex{Gans-Ludassy, Julius von 13.04.1858 – 30.09.1922@\textsc{Gans-Ludassy, Julius von} (13.04.1858 – 30.09.1922), \emph{Schriftsteller/Schriftstellerin, Journalist/Journalistin, Herausgeber/Herausgeberin}|pw} hat die \label{K_L03171-1v}\edtext{Loge\oindex{Burgtheater@\textbf{Burgtheater}, \emph{S.THTR}|pwv}}{\lemma{\textnormal{\emph{Loge}}}\Cendnote{\textnormal{Für die Aufführung von Victorien Sardous\pwindex{Sardou, Victorien 07.09.1831 – 08.11.1908@\textsc{Sardou, Victorien} (07.09.1831 – 08.11.1908), \emph{Schriftsteller/Schriftstellerin}|pwk}{ }\emph{Die alten Junggesellen}\pwindex{alten Junggesellen. Pariser Sittengemaelde in 5 Aufzuegen@\emph{Die alten Junggesellen. Pariser Sittengemälde in 5 Aufzügen}|pwk} im Burgtheater\oindex{Burgtheater@\textbf{Burgtheater}, \emph{S.THTR}|pwk}. Schnitzler besuchte 
                  die Aufführung trotzdem, vgl. A. S.: \emph{Tagebuch}, 27. 4. 1896. 
               }}}\label{K_L03171-1} im letzten Momente \label{K_L03171-2v}\edtext{mit
               Beschlag gelegt}{\lemma{\textnormal{\emph{mit
               Beschlag gelegt}}}\Cendnote{\textnormal{für sich beansprucht}}}\label{K_L03171-2}.\pend
           
\pstart
           Ich werde heute im Griensteidl\oindex{Cafe Griensteidl@\textbf{Café Griensteidl}, \emph{Kaffeehaus (K.KAF)}|pw} sein. Gegen die Loge\oindex{Burgtheater@\textbf{Burgtheater}, \emph{S.THTR}|pwv} kann ich nichts machen.\pend
           \pstart Ihr \spacefill\mbox{Salten.}\pend{}\selectlanguage{ngerman}\endnumbering\briefempfaengerindex{Schnitzler, Arthur@\textsc{Schnitzler, Arthur}!zzzSalten, Felix@\emph{von Felix Salten}!1896-04-271@{{[}27. 4. 1896{]}}|)be}\mylabel{L03171h}  \normalsize

\doendnotes{C}
\bigskip
\vfill

\clearpage

\footnotesize

\lohead{\textsc{register}}

% Definiere theindex-Environment komplett neu ohne reledmac
\makeatletter
\renewenvironment{theindex}{%
  \section*{\indexname}%
  \setlength{\parindent}{0pt}%
  \setlength{\parskip}{0pt plus 0.3pt}%
  \let\item\@idxitem
}{%
  \clearpage
}
\makeatother

\IfFileExists{\jobname-pw.ind}{\input{\jobname-pw.ind}}{}

\end{document}

      