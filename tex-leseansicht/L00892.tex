%% latex-korrekturansicht-vorspann.tex
%% Vorspann für die Korrekturansicht.
%% Lädt die gemeinsame Datei latex-vorspann.tex mit gesetztem Schalter.

\newif\ifkorrekturansicht
\korrekturansichttrue

\input{../tex-inputs/latex-vorspann}


\section[Arthur Schnitzler an Georg Brandes, 24. 2. 1899]{L00892 Arthur Schnitzler an Georg Brandes, 24. 2. 1899}
\nopagebreak\mylabel{L00892v}
\rehead{ }\normalsize\beginnumbering\briefempfaengerindex{Brandes, Georg@\textsc{Brandes, Georg}!zzzSchnitzler, Arthur@\emph{von Arthur Schnitzler}!1899-02-241@{24. 2. 1899}|(be}
\toendnotes[C]{\smallbreak\pagebreak[2]}\Standort{Kopenhagen, Det Kongelige Bibliotek, Georg Brandes Arkiv, box 125.}
\physDesc{Brief, 1 Blatt, 2 Seiten, 750 Zeichen
\newline{}Handschrift: schwarze Tinte, deutsche Kurrent
\newline{}Ordnung: mit Bleistift von unbekannter Hand nummeriert »14
                                    Schnitz« und das Datum mit einem Fragezeichen
                                 versehen }
\buchAbdrucke{\weitereDrucke{Georg Brandes, Arthur Schnitzler: \emph{Ein Briefwechsel}. Bern: \emph{Francke} 1956, S. 73.} }\toendnotes[C]{\smallbreak}
\pstart
           \raggedleft{}{\pb}24. 2. 99.\pend
           
\pstart{}Verehrteſter Herr Brandes,\pend\vspace{0.5em}
\pstart
           heute ſende ich Ihnen das \textsc{Manuscript} »Der grüne Kakadu\pwindex{gruene Kakadu. Groteske in einem Akt@\emph{Der grüne Kakadu. Groteske in einem Akt}|pw}«. Es iſt der dritte von 3 Einaktern\pwindex{Gefaehrtin. Schauspiel in einem Akt@\emph{Die Gefährtin. Schauspiel in einem Akt}|pwv}\pwindex{Paracelsus. Versspiel in einem Akt@\emph{Paracelsus. Versspiel in einem Akt}|pwv}\pwindex{gruene Kakadu. Groteske in einem Akt@\emph{Der grüne Kakadu. Groteske in einem Akt}|pwv}, die bald
               auch als \label{K_L00892-1v}\edtext{Buch\pwindex{gruene Kakadu – Paracelsus – Die Gefaehrtin. Drei Einakter@\emph{Der grüne Kakadu – Paracelsus – Die Gefährtin. Drei Einakter}|pwv}}{\lemma{\textnormal{\emph{Buch}}}\Cendnote{\textnormal{Die Auslieferung erfolgte Ende
                     April 1899: \emph{Der grüne Kakadu. Paracelsus – Die
                        Gefährtin}\pwindex{gruene Kakadu – Paracelsus – Die Gefaehrtin. Drei Einakter@\emph{Der grüne Kakadu – Paracelsus – Die Gefährtin. Drei Einakter}|pwk}. Drei Einakter von Arthur Schnitzler. Berlin: \emph{S. Fischer}\orgindex{S. Fischer Verlag@S. Fischer Verlag|pwk}{ }1899.}}}\label{K_L00892-1} erſcheinen werden. Aber dieſe »Groteske« möchte ich gern in Ihren
               Händen wiſſen, bevor ſie aufgeführt wird. Die Hoftheatercenſur hat ſie freigegeben,
               nur wenige Stellen (Sie werden ſich beim {\pb}Durchleſen leicht denken können, welche) ſind geſtrichen. Am erſten
                  März wird der Kakadu\pwindex{gruene Kakadu. Groteske in einem Akt@\emph{Der grüne Kakadu. Groteske in einem Akt}|pw} mit den zwei
               anderen Einaktern\pwindex{Gefaehrtin. Schauspiel in einem Akt@\emph{Die Gefährtin. Schauspiel in einem Akt}|pwv}\pwindex{Paracelsus. Versspiel in einem Akt@\emph{Paracelsus. Versspiel in einem Akt}|pwv}
                  zuſa{\geminationm}en aufgeführt. –\pend
           
\pstart
           Ich hoffe, dieſer Brief trifft Sie ſchon in voller Geſundheit an, Ihre Karte vom
                  22. Januar hat ja bereits einen hoffnungsvolleren Ton. Möge ich und
               wir alle, die Sie lieben, bald das allerbeſte von Ihnen hören!\pend
           \pstart Ich grüße Sie von Herzen als Ihr aufrichtig ergebener \spacefill\mbox{Arthur
                  Schnitzler}\pend{}\selectlanguage{ngerman}\endnumbering\briefempfaengerindex{Brandes, Georg@\textsc{Brandes, Georg}!zzzSchnitzler, Arthur@\emph{von Arthur Schnitzler}!1899-02-241@{24. 2. 1899}|)be}\mylabel{L00892h}  \normalsize

\doendnotes{C}
\bigskip
\vfill

\clearpage

\footnotesize

\lohead{\textsc{register}}

% Definiere theindex-Environment komplett neu ohne reledmac
\makeatletter
\renewenvironment{theindex}{%
  \section*{\indexname}%
  \setlength{\parindent}{0pt}%
  \setlength{\parskip}{0pt plus 0.3pt}%
  \let\item\@idxitem
}{%
  \clearpage
}
\makeatother

\IfFileExists{\jobname-pw.ind}{\input{\jobname-pw.ind}}{}

\end{document}

      