%% latex-leseansicht-vorspann.tex
%% Vorspann für die Leseansicht.
%% Lädt die gemeinsame Datei latex-vorspann.tex mit nicht gesetztem Schalter.

\newif\ifkorrekturansicht
\korrekturansichtfalse

\input{../tex-inputs/latex-vorspann}


               \section[Arthur Schnitzler an Hugo von Hofmannsthal, 16. 9. 1907]{ Arthur Schnitzler an Hugo von Hofmannsthal, 16. 9. 1907}\nopagebreak\mylabel{v}\rehead{ }\begin{ledgroupsized}[t]{13cm}\normalsize\beginnumbering\briefempfaengerindex{Hofmannsthal, Hugo von@\textsc{Hofmannsthal, Hugo von}!zzzSchnitzler, Arthur@\emph{von Arthur Schnitzler}!1907-09-161@{16. 9. 1907}|(be} \toendnotes[C]{\smallbreak\pagebreak[2]} \Standort{FDH, Hs-30885,129.}
\physDesc{Brief, 1 Blatt, 1 Seite, maschineller Durchschlag
\newline{}Schreibmaschine
\newline{}Handschrift: 1) Bleistift, deutsche Kurrent (\noindent{}Beschriftung: »\textsc{Hofmsthal}«)\hspace{1em}2) roter Buntstift, deutsche Kurrent (\noindent{}Unterstreichungen)\hspace{1em}\newline{}Ordnung: 1) Lochung 2) mit Bleistift von unbekannter Hand nummeriert:
                                                »129«\newline{}Zusatz: Zusammen mit der fehlenden Unterschrift scheint es
                                            unwahrscheinlich, dass dies das tatsächlich übermittelte
                                            Korrespondenzstück darstellt, obzwar es im Nachlass
                                            Hofmannsthals aufbewahrt ist. Mit großer
                                            Wahrscheinlichkeit dürfte es bei der Durchsicht der
                                            Briefe nach Hofmannsthals Tod 1929
                                            hinzugefügt worden sein. }\buchAbdrucke{\weitereDrucke{Hugo von Hofmannsthal, Arthur Schnitzler: \emph{Briefwechsel}. Hg. Therese Nickl und Heinrich Schnitzler. Frankfurt am Main: \emph{S. Fischer} 1964, S. 231.} }\toendnotes[C]{\smallbreak}\pstart
           \raggedleft{}{\pb}16. Sept. 07.\pend
           \pstart{}Lieber Hugo,\pend\pstart
           Ich danke Ihnen noch sehr für Ihr Telegramm. Der »Morgen\orgindex{Morgen. Wochenschrift fuer deutsche Kultur@Morgen. Wochenschrift für deutsche Kultur|pw}« scheint über meine Forderung nicht angenehm überrascht gewesen
                    zu sein. Sie bieten die \label{T_L01707_1v}\edtext{Hälfte,}{\lemma{\textnormal{\emph{Hälfte,}}}\Cendnote{\textnormal{Fehler: »Hälfte.,«}}}\label{T_L01707_1h} scheinen aber entschlossen,
                    wenn sie auch das Buch\pwindex{Schnitzler, Arthur 15.05.1862 – 21.10.1931@\textsc{Schnitzler, Arthur} (15.05.1862 – 21.10.1931), \emph{Schriftsteller, Mediziner}!Weg ins Freie. Roman1.1.1908 – 1.6.1908@\strich\emph{Der Weg ins Freie. Roman} {[}1.1.1908 – 1.6.1908{]}|pwv}
                    kriegen, höher gehen zu wollen{\dotstwo} Ich habe eigentlich
                    nicht den Eindruck, dass aus der Sache was werden wird. Dieser Schreibebrief hat
                    übrigens einen besonderen Zweck. Ich muss Sie etwas meinen Roman\pwindex{Schnitzler, Arthur 15.05.1862 – 21.10.1931@\textsc{Schnitzler, Arthur} (15.05.1862 – 21.10.1931), \emph{Schriftsteller, Mediziner}!Weg ins Freie. Roman1.1.1908 – 1.6.1908@\strich\emph{Der Weg ins Freie. Roman} {[}1.1.1908 – 1.6.1908{]}|pwv} betreffend fragen. Ist es nicht
                    höchst unwahrscheinlich, dass ein Mensch erst mit acht–neunundzwanzig Jahren
                    seine Diplomatenprüfung ablegt? Wär es aber nicht möglich, dass ein junger
                    Mensch eine \label{T_L01707_2v}\edtext{Staatskarriere}{\lemma{\textnormal{\emph{Staatskarriere}}}\Cendnote{\textnormal{Fehler:
                        »Staastkarriere«}}}\label{T_L01707_2h} einschlägt, Statthalterei zum
                    Beispiel und dass er dann zur Diplomatie übergeht? Ferner: Muss jemand, der die
                    Diplomatenprüfung macht vorher die orientalische
                        Akademie\orgindex{Orientalische Akademie@Orientalische Akademie|pw} besucht haben, oder genügt die Universität?\pend
           \endnumbering\briefempfaengerindex{Hofmannsthal, Hugo von@\textsc{Hofmannsthal, Hugo von}!zzzSchnitzler, Arthur@\emph{von Arthur Schnitzler}!1907-09-161@{16. 9. 1907}|)be}\mylabel{h}\end{ledgroupsized}  \newcommand{\dateiname}{L01707}\newcommand{\titel}{Arthur Schnitzler an Hugo von Hofmannsthal, 16. 9. 1907}\newcommand{\editorInnen}{Martin Anton Müller und Gerd-Hermann Susen}%% latex-leseansicht-abspann.tex
%% Abspann für die Leseansicht.
%% Der Schalter \ifkorrekturansicht ist bereits durch den Vorspann gesetzt.

%% latex-abspann.tex
%% Gemeinsamer Abspann für Korrekturansicht und Leseansicht.
%% Setzt den Schalter \ifkorrekturansicht voraus (gesetzt in den
%% einbindenden Dateien latex-korrekturansicht-abspann.tex bzw.
%% latex-leseansicht-abspann.tex).
%% ---------------------------------------------------------------

\normalsize

% Das esempio-Environment wird nur in der Leseansicht benötigt
\ifkorrekturansicht\else
\newenvironment{esempio}[3]%
{
    \vspace{1.5ex}
    \rlap{\underline{#1}}
    \par
    \setlength{\parindent}{0cm}
    \nopagebreak
    \leftskip=#2cm
    \rightskip=#3cm
}
{
    \par
}
\fi

\doendnotes{C}
\bigskip
\vfill

\clearpage

\footnotesize

\ifkorrekturansicht
  \lohead{\textsc{register}}
\fi

% theindex-Environment neu definieren ohne reledmac
\makeatletter
\renewenvironment{theindex}{%
  \ifkorrekturansicht
    \section*{\indexname}%
  \else
    \subsubsection*{Index der erwähnten Entitäten}%
  \fi
  \setlength{\parindent}{0pt}%
  \setlength{\parskip}{0pt plus 0.3pt}%
  \let\item\@idxitem
}{%
  \ifkorrekturansicht\clearpage\fi
}
\makeatother

\IfFileExists{\jobname-pw.ind}{\input{\jobname-pw.ind}}{}

% Quellenangabe nur in der Leseansicht
\ifkorrekturansicht\else
% Fallback-Definitionen, falls die .tex-Datei \titel etc. nicht gesetzt hat
\providecommand{\titel}{}
\providecommand{\editorInnen}{}
\providecommand{\dateiname}{\jobname}

\vspace{3cm}

\vfill

\footnotesize
\textsc{Quelle}: \titel. Herausgegeben von {\editorInnen}. In: \emph{Arthur Schnitzler: Briefwechsel mit Autorinnen und Autoren}.
 Digitale Edition, https://schnitzler-briefe.acdh.oeaw.ac.at/{\dateiname}.html (Stand \today)
\fi

\end{document}


      