%% latex-korrekturansicht-vorspann.tex
%% Vorspann für die Korrekturansicht.
%% Lädt die gemeinsame Datei latex-vorspann.tex mit gesetztem Schalter.

\newif\ifkorrekturansicht
\korrekturansichttrue

\input{../tex-inputs/latex-vorspann}


\section[Arthur Schnitzler an Hugo von Hofmannsthal, 16. 9. 1907]{L01707 Arthur Schnitzler an Hugo von Hofmannsthal, 16. 9. 1907}
\nopagebreak\mylabel{L01707v}
\rehead{ }\normalsize\beginnumbering\briefempfaengerindex{Hofmannsthal, Hugo von@\textsc{Hofmannsthal, Hugo von}!zzzSchnitzler, Arthur@\emph{von Arthur Schnitzler}!1907-09-161@{16. 9. 1907}|(be}
\toendnotes[C]{\smallbreak\pagebreak[2]}\Standort{FDH, Hs-30885,129.}
\physDesc{Brief, Durchschlag1 Blatt, 1 Seite, 839 Zeichen
\newline{}Schreibmaschine
\newline{}Handschrift: 1) Bleistift, deutsche Kurrent (\noindent{}Beschriftung: »\textsc{Hofmsthal}«)\hspace{1em}2) roter Buntstift, deutsche Kurrent (\noindent{}Unterstreichungen)\hspace{1em}
\newline{}Ordnung: 1) Lochung  2) mit Bleistift von unbekannter Hand nummeriert:
                                    »129«
\newline{}Zusatz: Zusammen mit der fehlenden Unterschrift scheint es
                                 unwahrscheinlich, dass dies das tatsächlich übermittelte
                                 Korrespondenzstück darstellt, obzwar es im Nachlass Hofmannsthals
                                 aufbewahrt ist. Mit großer Wahrscheinlichkeit dürfte es bei der
                                 Durchsicht der Briefe nach Hofmannsthals Tod 1929
                                 hinzugefügt worden sein. }
\buchAbdrucke{\weitereDrucke{Hugo von Hofmannsthal, Arthur Schnitzler: \emph{Briefwechsel}. Frankfurt am Main: \emph{S. Fischer} 1964, S. 231.} }\toendnotes[C]{\smallbreak}
\pstart
           \raggedleft{}{\pb}16. Sept. 07.\pend
           
\pstart{}Lieber Hugo,\pend\vspace{0.5em}
\pstart
           Ich danke Ihnen noch sehr für Ihr Telegramm. Der »Morgen\orgindex{Morgen. Wochenschrift fuer deutsche Kultur@Morgen. Wochenschrift für deutsche Kultur|pw}« scheint über meine Forderung nicht angenehm überrascht gewesen zu
               sein. Sie bieten die \label{T_L01707-1v}\edtext{Hälfte,}{\lemma{\textnormal{\emph{Hälfte,}}}\Cendnote{\textnormal{Fehler: »Hälfte.,«}}}\label{T_L01707-1}
               scheinen aber entschlossen, wenn sie auch das Buch\pwindex{Weg ins Freie. Roman@\emph{Der Weg ins Freie. Roman}|pwv} kriegen, höher gehen zu wollen{\dotstwo} Ich habe eigentlich nicht den Eindruck, dass aus der Sache was werden
               wird. Dieser Schreibebrief hat übrigens einen besonderen Zweck. Ich muss Sie etwas
               meinen Roman\pwindex{Weg ins Freie. Roman@\emph{Der Weg ins Freie. Roman}|pwv} betreffend
               fragen. Ist es nicht höchst unwahrscheinlich, dass ein Mensch erst mit
               acht–neunundzwanzig Jahren seine Diplomatenprüfung ablegt? Wär es aber nicht möglich,
               dass ein junger Mensch eine \label{T_L01707-2v}\edtext{Staatskarriere}{\lemma{\textnormal{\emph{Staatskarriere}}}\Cendnote{\textnormal{Fehler:
                     »Staastkarriere«}}}\label{T_L01707-2} einschlägt, Statthalterei zum Beispiel
               und dass er dann zur Diplomatie übergeht? Ferner: Muss jemand, der die
               Diplomatenprüfung macht vorher die orientalische
                  Akademie\orgindex{Orientalische Akademie@Orientalische Akademie|pw} besucht haben, oder genügt die Universität?\pend
           \selectlanguage{ngerman}\endnumbering\briefempfaengerindex{Hofmannsthal, Hugo von@\textsc{Hofmannsthal, Hugo von}!zzzSchnitzler, Arthur@\emph{von Arthur Schnitzler}!1907-09-161@{16. 9. 1907}|)be}\mylabel{L01707h}  \normalsize

\doendnotes{C}
\bigskip
\vfill

\clearpage

\footnotesize

\lohead{\textsc{register}}

% Definiere theindex-Environment komplett neu ohne reledmac
\makeatletter
\renewenvironment{theindex}{%
  \section*{\indexname}%
  \setlength{\parindent}{0pt}%
  \setlength{\parskip}{0pt plus 0.3pt}%
  \let\item\@idxitem
}{%
  \clearpage
}
\makeatother

\IfFileExists{\jobname-pw.ind}{\input{\jobname-pw.ind}}{}

\end{document}

      