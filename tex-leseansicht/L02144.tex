\input{../tex-inputs/latex-pdf-vorspann}
\begin{center}
            \textcolor{red}{ENTWURF. ENTZIFFERUNG NOCH NICHT KORREKTURGELESEN}
                      \end{center}
            
               \section[Hermann Bahr an Arthur Schnitzler, 20. 7. 1913]{ Hermann Bahr an Arthur Schnitzler, 20. 7. 1913}\nopagebreak\mylabel{v}\rehead{ }\begin{ledgroupsized}[t]{13cm}\normalsize\beginnumbering\briefempfaengerindex{Schnitzler, Arthur@\textsc{Schnitzler, Arthur}!zzzBahr, Hermann@\emph{von Hermann Bahr}!1913-07-201@{20. 7. 1913}|(be} \toendnotes[C]{\smallbreak\pagebreak[2]} \Standort{CUL, Schnitzler, B 5b.}
\physDesc{Bildpostkarte
\newline{}Handschrift: schwarze Tinte, deutsche Kurrent\newline{}Versand: ohne postalischen Übermittlungsvermerk \newline{}Ordnung: mit Bleistift von unbekannter Hand nummeriert:
                                    »178« }\buchAbdrucke{\weitereDrucke{Hermann Bahr, Arthur Schnitzler: \emph{Briefwechsel, Aufzeichnungen, Dokumente (1891–1931)}. Hg. Kurt Ifkovits und Martin Anton Müller. Göttingen: \emph{Wallstein} 2018, S. 490.} }\toendnotes[C]{\smallbreak}\pstart
           \noindent{}\centering{}\textcolor{gray}{\textbf{{\pb}Salzburg, Am Stein\oindex{Aeusserer Stein@\textbf{Äußerer Stein}|pw}}}\pend
           \pstart
           \raggedleft{}{\pb}20. 7. 13\pend
           \pstart{}Lieber Arthur!\pend\pstart
           Herzlichen Dank für Dein ſo liebes Geſchenk, das meine Wand ſchmückt, und für Euer\pwindex{Schnitzler, Olga 17.01.1882 – 13.01.1970@\textsc{Schnitzler, Olga} (17.01.1882 – 13.01.1970), \emph{Schauspielerin, Sängerin}|pwv} gutes Telegramm, das mein
               Herz erfreut!\pend
           \pstart
           Bleibt mir, was Ihr\pwindex{Schnitzler, Olga 17.01.1882 – 13.01.1970@\textsc{Schnitzler, Olga} (17.01.1882 – 13.01.1970), \emph{Schauspielerin, Sängerin}|pwv} mir ſeit ſo
               vielen Jahren ſeid! Ich will immer der Eure\pwindex{Schnitzler, Olga 17.01.1882 – 13.01.1970@\textsc{Schnitzler, Olga} (17.01.1882 – 13.01.1970), \emph{Schauspielerin, Sängerin}|pwv}{ }ſein!\pend
           \pstart \spacefill\mbox{HermannBahr}\pend{}\endnumbering\briefempfaengerindex{Schnitzler, Arthur@\textsc{Schnitzler, Arthur}!zzzBahr, Hermann@\emph{von Hermann Bahr}!1913-07-201@{20. 7. 1913}|)be}\mylabel{h}\end{ledgroupsized}  \newcommand{\dateiname}{L02144}\newcommand{\titel}{Hermann Bahr an Arthur Schnitzler, 20. 7. 1913}\newcommand{\editorInnen}{ Kurt Ifkovits,  Martin Anton Müller}\input{../tex-inputs/latex-pdf-abspann}
      