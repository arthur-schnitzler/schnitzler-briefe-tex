%% latex-leseansicht-vorspann.tex
%% Vorspann für die Leseansicht.
%% Lädt die gemeinsame Datei latex-vorspann.tex mit nicht gesetztem Schalter.

\newif\ifkorrekturansicht
\korrekturansichtfalse

\input{../tex-inputs/latex-vorspann}


         
         \renewcommand{\erwaehntePersonen}{Personen: Fedor Mamroth, Lucien Millevoye, Leopold Sonnemann, Jean Thorel}
         \renewcommand{\erwaehnteInstitutionen}{Institutionen: Frankfurter Zeitung}
         \renewcommand{\erwaehnteOrte}{Orte: Paris, Wien, rue Feydeau}
         \renewcommand{\erwaehnteWerke}{Werke: Adolphe. Anecdote trouvée dans les papiers d’un inconnu, Amourette. Pièce en trois actes. Adaptée de Arthur Schnitzler, Die Affaire Goldmann–Millevoye, Frankfurter Zeitung, Mercure de France}
               \section[ Paul Goldmann an Arthur Schnitzler, 23. 11. {[}1896{]}]{ Paul Goldmann an Arthur Schnitzler, 23. 11. {[}1896{]}}\nopagebreak\mylabel{v}\rehead{ }\begin{ledgroupsized}[t]{13cm}\normalsize\beginnumbering \toendnotes[C]{\smallbreak\pagebreak[2]} \Standort{DLA, A:Schnitzler, HS.NZ85.1.3166.}
\physDesc{Brief, 1 Blatt, 4 Seiten, 1924 Zeichen
\newline{}Handschrift: blaue Tinte, deutsche Kurrent
\newline{}Schnitzler: mit Bleistift das Jahr »96« vermerkt }\toendnotes[C]{\smallbreak}\pstart
           \noindent{}{\pb}\textcolor{gray}{\textbf{\textbf{Frankfurter Zeitung\orgindex{Frankfurter Zeitung@Frankfurter Zeitung|pw}}}}\pend
           \pstart
           \textcolor{gray}{\textbf{(\begin{otherlanguage}{french}Gazette de Francfort\end{otherlanguage}\orgindex{Frankfurter Zeitung@Frankfurter Zeitung|pw}).}}\pend
           \pstart
           \textcolor{gray}{\textbf{\textbf{\begin{otherlanguage}{french}Fondateur M.\end{otherlanguage}{ }L. Sonnemann\pwindex{Sonnemann, Leopold 1831-10-29 – 1909-10-30@\textsc{Sonnemann, Leopold} (1831-10-29 – 1909-10-30), \emph{Journalist, Herausgeber}|pw}.}}}\pend
           \pstart
           \begin{otherlanguage}{french}\textcolor{gray}{\textbf{Journal\pwindex{?? Werk@Nicht ermittelte Verfasserinnen und Verfasser!Frankfurter Zeitung1856 – 1943@\emph{Frankfurter Zeitung} {[}1856 – 1943{]}|pwv} politique,
                        financier,}}\end{otherlanguage}\pend
           \pstart
           \begin{otherlanguage}{french}\textcolor{gray}{\textbf{commercial et littéraire.}}\end{otherlanguage}\pend
           \pstart
           \begin{otherlanguage}{french}\textcolor{gray}{\textbf{\textbf{Paraissant trois fois par jour.}}}\end{otherlanguage}\hfill \textsc{Paris\oindex{Paris@\textbf{Paris}|pw}}, 23. November.\pend
           \pstart
           \begin{otherlanguage}{french}\textcolor{gray}{\textbf{\textbf{Bureau à Paris\oindex{Paris@\textbf{Paris}|pw}}}}\end{otherlanguage}\pend
           \pstart
           \begin{otherlanguage}{french}\textcolor{gray}{\textbf{\textbf{24. Rue Feydeau\oindex{rue Feydeau@\textbf{rue Feydeau}|pw}.}}}\end{otherlanguage}\pend
           \pstart\center{}Mein lieber Freund,\pend\pstart
           Zugleich mit der \label{K_L02791-88v}\edtext{Depeſche\pwindex{Affaire Goldmann–Millevoye1896-11-22@\emph{Die Affaire Goldmann–Millevoye} {[}1896-11-22{]}|pwv}}{\lemma{\textnormal{\emph{Depeſche}}}\Cendnote{\textnormal{F\pwindex{Mamroth, Fedor 21.02.1851 – 25.06.1907@\textsc{Mamroth, Fedor} (21.02.1851 – 25.06.1907), \emph{Journalist, Kritiker}|pwk} [ = Fedor Mamroth\pwindex{Mamroth, Fedor 21.02.1851 – 25.06.1907@\textsc{Mamroth, Fedor} (21.02.1851 – 25.06.1907), \emph{Journalist, Kritiker}|pwk}]: \emph{Die Affaire
                        Goldmann–Millevoye}\pwindex{Affaire Goldmann–Millevoye1896-11-22@\emph{Die Affaire Goldmann–Millevoye} {[}1896-11-22{]}|pwk}. In: \emph{Frankfurter
                        Zeitung}\pwindex{?? Werk@Nicht ermittelte Verfasserinnen und Verfasser!Frankfurter Zeitung1856 – 1943@\emph{Frankfurter Zeitung} {[}1856 – 1943{]}|pwk}, Jg. 41, Nr. 325, 22. 11. 1896, Erstes
                     Morgenblatt, S. 3.}}}\label{K_L02791-88h} an meinen Onkel\pwindex{Mamroth, Fedor 21.02.1851 – 25.06.1907@\textsc{Mamroth, Fedor} (21.02.1851 – 25.06.1907), \emph{Journalist, Kritiker}|pwv} ſandte ich am Samſtag
               eine an Dich ab. \strikeout{D} Dein \label{K_L02791-1v}\edtext{Telegramm}{\lemma{\textnormal{\emph{Telegramm}}}\Cendnote{\textnormal{siehe Arthur Schnitzler an Paul Goldmann, 21. 11. 1896}}}\label{K_L02791-1h}, das \strikeout{N\textcolor{gray}{×}} Nachricht verlangte, hat ſich mit dem \label{K_L02791-2v}\edtext{meinen}{\lemma{\textnormal{\emph{meinen}}}\Cendnote{\textnormal{siehe Paul Goldmann an Arthur Schnitzler, 21. 11. 1896}}}\label{K_L02791-2h} gekreuzt. Dies zur Steuer der hiſtoriſchen Wahrheit.\pend
           \pstart
           Und nun \strikeout{t\textcolor{gray}{a}} tauſend Dank für Deine freundſchaftliche Theilnahme und Deine lieben
               Glückwünſche. Aber glaube nur \strikeout{ja} ja nicht, daß ich
               ein \strikeout{Hed} Held geworden bin. Die Sache iſt eigentlich
               eine große Comödie, mit ſehr wenig Gefahr. Und willſt Du {\pb}wiſſen, was Muth iſt? Muth iſt: wenn man \label{K_L02791-3v}\edtext{vorher}{\lemma{\textnormal{\emph{vorher}}}\Cendnote{\textnormal{vor einem Pistolenduell}}}\label{K_L02791-3h} eine halbe Flaſche Rothwein
               getrunken hat. Muth iſt: wenn Leute da ſind und zuſchauen. Muth iſt: wenn man unter
               gar keinen Umſtänden weglaufen darf. Muth iſt: wenn man nicht an die Gefahr denkt.
               Und Muth iſt, vor Allem, wie bekannt: wenn man überzeugt iſt, es wird Einem doch
               nichts paſſiren.\pend
           \pstart
           Ein Gefühl, das »Muth« heißt, gibt es ſicher nicht. Es gibt nur \uline{ein} Gefühl: die Furcht; und der Muth iſt die Negirung dieſes {\pb}Gefühls, oder, um mich franzöſiſch zu citiren:
                  \label{K_L02791-55v}\edtext{\begin{otherlanguage}{french}\textsc{le courage, c’est l’effort qu’on fait contre la peur}\end{otherlanguage}}{\lemma{\textnormal{\emph{le … peur}}}\Cendnote{\textnormal{französisch: Mut ist Aufwand, den man
                  gegen die Angst aufbringt.}}}\label{K_L02791-55h}.\pend
           \pstart
           Das ſind ſo die \substVorne{}\textsuperscript{w\textcolor{gray}{a}hren}{\allowbreak}\substDazwischen{}wahren\substHinten{} inneren Vorgänge geweſen. Alles Äußerliche war Schauſpiel und Schwindel. Ich
               habe nicht auf den Mann\pwindex{Millevoye, Lucien 1850-08-01 – 1918-03-25@\textsc{Millevoye, Lucien} (1850-08-01 – 1918-03-25), \emph{Politiker, Journalist}|pwv}
               gezielt, er aber hat auf mich gezielt, was aber nichts macht, da \strikeout{i\textcolor{gray}{c}h}{ }\strikeout{e\textcolor{gray}{r}} er ein ſchlechter Schütze iſt. Für meine Poſition hier iſt die Sache gut
               geweſen, bei meinem Blatte\orgindex{Frankfurter Zeitung@Frankfurter Zeitung|pwv}
               hätte ſie mich beinahe meine Stellung gekoſtet (die großen Demokraten ſind gegen das
               Duell). Schlagen mußte ich mich, um nicht als {\pb}Feigling zu erſcheinen. Aber ich hab’ es ungern gethan. Es iſt eigentlich eine
               Kinderei, und hinterher ſchämt man ſich ſehr darüber, daß man nicht verwundet iſt.
               Die Nacht vorher aber hat man Angſt.\pend
           \pstart
           Hoffentlich kann ich Dir eines Tages mit würdigeren Thaten aufwarten.\pend
           \pstart
           Grüß’ Dich Gott, mein lieber Freund. Schreib’ mir bald!\pend
           \pstart
           Dein treuer {\\[\baselineskip]}\spacefill\mbox{Paul Goldmnn}\pend
           \leftskip=0em{}\pstart
           \noindent{}Morgen ſende ich ab\substVorne{}\textsuperscript{:}\substDazwischen{}.\substHinten{}{ }\substVorne{}\textsuperscript{.}\substDazwischen{}1.)\substHinten{} Das Manuſkript der Überſetzung\pwindex{Thorel, Jean 1859-09-11 – 1916-08-20@\textsc{Thorel, Jean} (1859-09-11 – 1916-08-20), \emph{Übersetzer, Schriftsteller}!Amourette. Piece en trois actes. Adaptee de Arthur Schnitzler1897@\strich\emph{Amourette. Pièce en trois actes. Adaptée de Arthur Schnitzler} {[}Übersetzung, 1897{]}|pwv} von \textsc{Thorel\pwindex{Thorel, Jean 1859-09-11 – 1916-08-20@\textsc{Thorel, Jean} (1859-09-11 – 1916-08-20), \emph{Übersetzer, Schriftsteller}|pw}} 2.) den \label{K_L02791-41v}\edtext{»\textsc{Mercure\pwindex{?? Werk@Nicht ermittelte Verfasserinnen und Verfasser!Mercure de France1890 – 1965@\emph{Mercure de France} {[}1890 – 1965{]}|pw}}«}{\lemma{\textnormal{\emph{»Mercure«}}}\Cendnote{\textnormal{Kein zeitnah erschienener
                     Artikel im \emph{Mercure}\pwindex{?? Werk@Nicht ermittelte Verfasserinnen und Verfasser!Mercure de France1890 – 1965@\emph{Mercure de France} {[}1890 – 1965{]}|pwk} bietet sich an,
                     weswegen Goldmann\pwindex{Goldmann, Paul 31.01.1865 – 25.09.1935@\textsc{Goldmann, Paul} (31.01.1865 – 25.09.1935), \emph{Schriftsteller, Journalist}|pwk} das Heft\pwindex{?? Werk@Nicht ermittelte Verfasserinnen und Verfasser!Mercure de France1890 – 1965@\emph{Mercure de France} {[}1890 – 1965{]}|pwkv} geschickt haben könnte, also
                     dürfte es sich um eine allgemeine Beilage gehandelt haben.}}}\label{K_L02791-41h} 3.) »\textsc{Adolphe\pwindex{\textcolor{red}{\textsuperscript{XXXX1 indx}}!Adolphe. Anecdote trouvee dans les papiers Dun inconnu1816@\strich\emph{Adolphe. Anecdote trouvée dans les papiers d’un inconnu} {[}1816{]}|pw}}«. Bitte das Manuſkript\pwindex{Thorel, Jean 1859-09-11 – 1916-08-20@\textsc{Thorel, Jean} (1859-09-11 – 1916-08-20), \emph{Übersetzer, Schriftsteller}!Amourette. Piece en trois actes. Adaptee de Arthur Schnitzler1897@\strich\emph{Amourette. Pièce en trois actes. Adaptée de Arthur Schnitzler} {[}Übersetzung, 1897{]}|pwv}{ }\uline{bald} zurückzuſenden.\pend
           
         
         \endnumbering\mylabel{h}\end{ledgroupsized}  \newcommand{\dateiname}{L02791}\newcommand{\titel}{Paul Goldmann an Arthur Schnitzler, 23. 11. [1896]}\newcommand{\editorInnen}{Martin Anton Müller und Laura Untner}%% latex-leseansicht-abspann.tex
%% Abspann für die Leseansicht.
%% Der Schalter \ifkorrekturansicht ist bereits durch den Vorspann gesetzt.

%% latex-abspann.tex
%% Gemeinsamer Abspann für Korrekturansicht und Leseansicht.
%% Setzt den Schalter \ifkorrekturansicht voraus (gesetzt in den
%% einbindenden Dateien latex-korrekturansicht-abspann.tex bzw.
%% latex-leseansicht-abspann.tex).
%% ---------------------------------------------------------------

\normalsize

% Das esempio-Environment wird nur in der Leseansicht benötigt
\ifkorrekturansicht\else
\newenvironment{esempio}[3]%
{
    \vspace{1.5ex}
    \rlap{\underline{#1}}
    \par
    \setlength{\parindent}{0cm}
    \nopagebreak
    \leftskip=#2cm
    \rightskip=#3cm
}
{
    \par
}
\fi

\doendnotes{C}
\bigskip
\vfill

\clearpage

\footnotesize

\ifkorrekturansicht
  \lohead{\textsc{register}}
\fi

% theindex-Environment neu definieren ohne reledmac
\makeatletter
\renewenvironment{theindex}{%
  \ifkorrekturansicht
    \section*{\indexname}%
  \else
    \subsubsection*{Index der erwähnten Entitäten}%
  \fi
  \setlength{\parindent}{0pt}%
  \setlength{\parskip}{0pt plus 0.3pt}%
  \let\item\@idxitem
}{%
  \ifkorrekturansicht\clearpage\fi
}
\makeatother

\IfFileExists{\jobname-pw.ind}{\input{\jobname-pw.ind}}{}

% Quellenangabe nur in der Leseansicht
\ifkorrekturansicht\else
% Fallback-Definitionen, falls die .tex-Datei \titel etc. nicht gesetzt hat
\providecommand{\titel}{}
\providecommand{\editorInnen}{}
\providecommand{\dateiname}{\jobname}

\vspace{3cm}

\vfill

\footnotesize
\textsc{Quelle}: \titel. Herausgegeben von {\editorInnen}. In: \emph{Arthur Schnitzler: Briefwechsel mit Autorinnen und Autoren}.
 Digitale Edition, https://schnitzler-briefe.acdh.oeaw.ac.at/{\dateiname}.html (Stand \today)
\fi

\end{document}


      