%% latex-leseansicht-vorspann.tex
%% Vorspann für die Leseansicht.
%% Lädt die gemeinsame Datei latex-vorspann.tex mit nicht gesetztem Schalter.

\newif\ifkorrekturansicht
\korrekturansichtfalse

\input{../tex-inputs/latex-vorspann}


\section[ Paul Goldmann an Arthur Schnitzler, 23. 11. {[}1896{]}]{L02791 Paul Goldmann an Arthur Schnitzler,  23. 11. [1896]}
\nopagebreak\mylabel{L02791v}
\rehead{ }\normalsize\beginnumbering\briefempfaengerindex{Schnitzler, Arthur@\textsc{Schnitzler, Arthur}!zzzGoldmann, Paul@\emph{von Paul Goldmann}!1896-11-233@{23. 11. [1896]}|(be}
\toendnotes[C]{\smallbreak\pagebreak[2]}
\correspDesc{Versand  durch Paul Goldmann am 23. 11. [1896] in Paris
\newline{}Erhalt  durch Arthur Schnitzler im Zeitraum [24. 11. 1896 – 28. 11. 1896?] in Wien}\toendnotes[C]{\smallbreak}
\Standort{DLA, A:Schnitzler, HS.NZ85.1.3166.}
\physDesc{Brief, 1 Blatt, 4 Seiten, 1924 Zeichen
\newline{}Handschrift: blaue Tinte, deutsche Kurrent
\newline{}Schnitzler: mit Bleistift das Jahr »96« vermerkt }\toendnotes[C]{\smallbreak}
\pstart
           {\pb}\textcolor{gray}{\textbf{\textbf{Frankfurter Zeitung\orgindex{Frankfurter Zeitung@Frankfurter Zeitung|pw}}}}\pend
           
\pstart
           \textcolor{gray}{\textbf{(\begin{otherlanguage}{french}Gazette de Francfort\end{otherlanguage}\orgindex{Frankfurter Zeitung@Frankfurter Zeitung|pw}).}}\pend
           
\pstart
           \textcolor{gray}{\textbf{\textbf{\begin{otherlanguage}{french}Fondateur M.\end{otherlanguage}{ }L. Sonnemann\pwindex{Sonnemann, Leopold 29.\,10.\,1831 Höchberg – 30.\,10.\,1909 Frankfurt am Main@\textsc{Sonnemann, Leopold} (29.\,10.\,1831 Höchberg – 30.\,10.\,1909 Frankfurt am Main), \emph{Journalist, Herausgeber}|pw}.}}}\pend
           
\pstart
           \begin{otherlanguage}{french}\textcolor{gray}{\textbf{Journal\pwindex{Frankfurter Zeitung@\emph{Frankfurter Zeitung}|pwv} politique,
                        financier,}}\end{otherlanguage}\pend
           
\pstart
           \begin{otherlanguage}{french}\textcolor{gray}{\textbf{commercial et littéraire.}}\end{otherlanguage}\pend
           
\pstart
           \begin{otherlanguage}{french}\textcolor{gray}{\textbf{\textbf{Paraissant trois fois par jour.}}}\end{otherlanguage}\hfill \textsc{Paris\oindex{Paris@\textbf{Paris}, \emph{Hauptstadt}|pw}}, 23. November.\pend
           
\pstart
           \begin{otherlanguage}{french}\textcolor{gray}{\textbf{\textbf{Bureau à Paris\oindex{Paris@\textbf{Paris}, \emph{Hauptstadt}|pw}}}}\end{otherlanguage}\pend
           
\pstart
           \begin{otherlanguage}{french}\textcolor{gray}{\textbf{\textbf{24. Rue Feydeau\oindex{rue Feydeau@\textbf{rue Feydeau}, \emph{Straße}|pw}.}}}\end{otherlanguage}\pend
           
\pstart\center{}Mein lieber Freund,\pend\vspace{0.5em}
\pstart
           Zugleich mit der \label{K_L02791-1v}\edtext{Depeſche\pwindex{Mamroth, Fedor 21.\,2.\,1851 Breslau – 25.\,6.\,1907 Frankfurt am Main@\textsc{Mamroth, Fedor} (21.\,2.\,1851 Breslau – 25.\,6.\,1907 Frankfurt am Main), \emph{Journalist, Kritiker}!Affaire Goldmann–Millevoye@\strich\emph{Die Affaire Goldmann–Millevoye}|pwv}}{\lemma{\textnormal{\emph{Depesche}}}\Cendnote{\textnormal{F\pwindex{Mamroth, Fedor 21.\,2.\,1851 Breslau – 25.\,6.\,1907 Frankfurt am Main@\textsc{Mamroth, Fedor} (21.\,2.\,1851 Breslau – 25.\,6.\,1907 Frankfurt am Main), \emph{Journalist, Kritiker}|pwk} [ = Fedor Mamroth\pwindex{Mamroth, Fedor 21.\,2.\,1851 Breslau – 25.\,6.\,1907 Frankfurt am Main@\textsc{Mamroth, Fedor} (21.\,2.\,1851 Breslau – 25.\,6.\,1907 Frankfurt am Main), \emph{Journalist, Kritiker}|pwk}]: \emph{Die Affaire
                        Goldmann–Millevoye}\pwindex{Mamroth, Fedor 21.\,2.\,1851 Breslau – 25.\,6.\,1907 Frankfurt am Main@\textsc{Mamroth, Fedor} (21.\,2.\,1851 Breslau – 25.\,6.\,1907 Frankfurt am Main), \emph{Journalist, Kritiker}!Affaire Goldmann–Millevoye@\strich\emph{Die Affaire Goldmann–Millevoye}|pwk}. In: \emph{Frankfurter
                        Zeitung}\pwindex{Frankfurter Zeitung@\emph{Frankfurter Zeitung}|pwk}, Jg. 41, Nr. 325, 22. 11. 1896, Erstes
                     Morgenblatt, S. 3.}}}\label{K_L02791-1} an meinen Onkel\pwindex{Mamroth, Fedor 21.\,2.\,1851 Breslau – 25.\,6.\,1907 Frankfurt am Main@\textsc{Mamroth, Fedor} (21.\,2.\,1851 Breslau – 25.\,6.\,1907 Frankfurt am Main), \emph{Journalist, Kritiker}|pwv}{ }ſandte ich am Samſtag
               eine an Dich ab. \strikeout{D} Dein \label{K_L02791-2v}\edtext{Telegramm}{\lemma{\textnormal{\emph{Telegramm}}}\Cendnote{\textnormal{Siehe XXXX Auszeichnungsfehler: Dokument L02684 nicht gefunden.
               }}}\label{K_L02791-2}, das \strikeout{N\textcolor{gray}{×}} Nachricht verlangte, hat{ }ſich mit dem \label{K_L02791-3v}\edtext{meinen}{\lemma{\textnormal{\emph{meinen}}}\Cendnote{\textnormal{Siehe XXXX Auszeichnungsfehler: Dokument L02689 nicht gefunden.
               }}}\label{K_L02791-3} gekreuzt. Dies zur Steuer der hiſtoriſchen Wahrheit.\pend
           
\pstart
           Und nun \strikeout{t\textcolor{gray}{a}} tauſend Dank für Deine freundſchaftliche Theilnahme und Deine lieben
               Glückwünſche. Aber glaube nur \strikeout{ja} ja nicht, daß ich
               ein \strikeout{Hed} Held geworden bin. Die Sache iſt eigentlich
               eine große Comödie, mit{ }ſehr wenig Gefahr. Und willſt Du {\pb}wiſſen, was Muth iſt? Muth iſt: wenn man \label{K_L02791-4v}\edtext{vorher}{\lemma{\textnormal{\emph{vorher}}}\Cendnote{\textnormal{vor einem Pistolenduell}}}\label{K_L02791-4} eine halbe Flaſche Rothwein
               getrunken hat. Muth iſt: wenn Leute da{ }ſind und zuſchauen. Muth iſt: wenn man unter
               gar keinen Umſtänden weglaufen darf. Muth iſt: wenn man nicht an die Gefahr denkt.
               Und Muth iſt, vor Allem, wie bekannt: wenn man überzeugt iſt, es wird Einem doch
               nichts paſſiren.\pend
           
\pstart
           Ein Gefühl, das »Muth« heißt, gibt es{ }ſicher nicht. Es gibt nur \uline{ein} Gefühl: die Furcht; und der Muth iſt die Negirung dieſes {\pb}Gefühls, oder, um mich franzöſiſch zu citiren:
                  \label{K_L02791-5v}\edtext{\begin{otherlanguage}{french}\textsc{le courage, c’est l’effort qu’on fait contre la peur}\end{otherlanguage}}{\lemma{\textnormal{\emph{le … peur}}}\Cendnote{\textnormal{französisch: Mut ist Aufwand, den man
                  gegen die Angst aufbringt.}}}\label{K_L02791-5}.\pend
           
\pstart
           Das{ }ſind{ }ſo die \substVorne{}\textsuperscript{w\textcolor{gray}{a}hren}\substDazwischen{}wahren\substHinten{} inneren Vorgänge geweſen. Alles Äußerliche war Schauſpiel und Schwindel. Ich
               habe nicht auf den Mann\pwindex{Millevoye, Lucien 1.\,8.\,1850 Grenoble – 25.\,3.\,1918 Paris@\textsc{Millevoye, Lucien} (1.\,8.\,1850 Grenoble – 25.\,3.\,1918 Paris), \emph{Politiker, Journalist}|pwv}
               gezielt, er aber hat auf mich gezielt, was aber nichts macht, da \strikeout{i\textcolor{gray}{c}h}{ }\strikeout{e\textcolor{gray}{r}} er ein{ }ſchlechter Schütze iſt. Für meine Poſition hier iſt die Sache gut
               geweſen, bei meinem Blatte\orgindex{Frankfurter Zeitung@Frankfurter Zeitung|pwv}
               hätte{ }ſie mich beinahe meine Stellung gekoſtet (die großen Demokraten{ }ſind gegen das
               Duell). Schlagen mußte ich mich, um nicht als {\pb}Feigling zu erſcheinen. Aber ich hab’ es ungern gethan. Es iſt eigentlich eine
               Kinderei, und hinterher{ }ſchämt man{ }ſich{ }ſehr darüber, daß man nicht verwundet iſt.
               Die Nacht vorher aber hat man Angſt.\pend
           
\pstart
           Hoffentlich kann ich Dir eines Tages mit würdigeren Thaten aufwarten.\pend
           
\pstart
           Grüß’ Dich Gott, mein lieber Freund. Schreib’ mir bald!\pend
           
\pstart
           Dein treuer {\\[\baselineskip]}\spacefill\mbox{Paul Goldmnn}\pend
           \leftskip=0em{}
\pstart
           \noindent{}Morgen{ }ſende ich ab\substVorne{}\textsuperscript{:}\substDazwischen{}.\substHinten{}{ }\substVorne{}\textsuperscript{.}\substDazwischen{}1.)\substHinten{} Das Manuſkript der Überſetzung\pwindex{Schnitzler, Arthur 15.\,5.\,1862 Wien – 21.\,10.\,1931 ebd.@\textsc{Schnitzler, Arthur} (15.\,5.\,1862 Wien – 21.\,10.\,1931 ebd.), \emph{Schriftsteller, Mediziner}!Amourette. Pièce en trois actes. Adaptée de Arthur Schnitzler@\strich\emph{Amourette. Pièce en trois actes. Adaptée de Arthur Schnitzler}|pwv} von \textsc{Thorel\pwindex{Thorel, Jean 11.\,9.\,1859 Éragny – 20.\,8.\,1916 Enghien-les-Bains@\textsc{Thorel, Jean} (11.\,9.\,1859 Éragny – 20.\,8.\,1916 Enghien-les-Bains), \emph{Übersetzer, Dramatiker}|pw}} 2.) den \label{K_L02791-6v}\edtext{»\textsc{Mercure\pwindex{Mercure de France@\emph{Mercure de France}|pw}}«}{\lemma{\textnormal{\emph{»Mercure«}}}\Cendnote{\textnormal{Kein zeitnah erschienener
                     Artikel im \emph{Mercure}\pwindex{Mercure de France@\emph{Mercure de France}|pwk} bietet sich an,
                     weswegen Goldmann\pwindex{Goldmann, Paul 31.\,1.\,1865 Breslau – 25.\,9.\,1935 Wien@\textsc{Goldmann, Paul} (31.\,1.\,1865 Breslau – 25.\,9.\,1935 Wien), \emph{Schriftsteller, Journalist}|pwk} das Heft\pwindex{Mercure de France@\emph{Mercure de France}|pwkv} geschickt haben könnte, also
                     dürfte es sich um eine allgemeine Beilage gehandelt haben.}}}\label{K_L02791-6} 3.) »\textsc{Adolphe\pwindex{\textcolor{red}{\textsuperscript{XXXX indx1}}!Adolphe. Anecdote trouvée dans les papiers d’un inconnu@\strich\emph{Adolphe. Anecdote trouvée dans les papiers d’un inconnu}|pw}}«. Bitte das Manuſkript\pwindex{Schnitzler, Arthur 15.\,5.\,1862 Wien – 21.\,10.\,1931 ebd.@\textsc{Schnitzler, Arthur} (15.\,5.\,1862 Wien – 21.\,10.\,1931 ebd.), \emph{Schriftsteller, Mediziner}!Amourette. Pièce en trois actes. Adaptée de Arthur Schnitzler@\strich\emph{Amourette. Pièce en trois actes. Adaptée de Arthur Schnitzler}|pwv}{ }\uline{bald} zurückzuſenden.\pend
           \selectlanguage{ngerman}\endnumbering\briefempfaengerindex{Schnitzler, Arthur@\textsc{Schnitzler, Arthur}!zzzGoldmann, Paul@\emph{von Paul Goldmann}!1896-11-233@{23. 11. [1896]}|)be}\mylabel{L02791h}  \newcommand{\dateiname}{L02791}\newcommand{\titel}{Paul Goldmann an Arthur Schnitzler, 23. 11. [1896]}\newcommand{\editorInnen}{Martin Anton Müller und Laura Untner}%% latex-leseansicht-abspann.tex
%% Abspann für die Leseansicht.
%% Der Schalter \ifkorrekturansicht ist bereits durch den Vorspann gesetzt.

%% latex-abspann.tex
%% Gemeinsamer Abspann für Korrekturansicht und Leseansicht.
%% Setzt den Schalter \ifkorrekturansicht voraus (gesetzt in den
%% einbindenden Dateien latex-korrekturansicht-abspann.tex bzw.
%% latex-leseansicht-abspann.tex).
%% ---------------------------------------------------------------

\normalsize

% Das esempio-Environment wird nur in der Leseansicht benötigt
\ifkorrekturansicht\else
\newenvironment{esempio}[3]%
{
    \vspace{1.5ex}
    \rlap{\underline{#1}}
    \par
    \setlength{\parindent}{0cm}
    \nopagebreak
    \leftskip=#2cm
    \rightskip=#3cm
}
{
    \par
}
\fi

\doendnotes{C}
\bigskip
\vfill

\clearpage

\footnotesize

\ifkorrekturansicht
  \lohead{\textsc{register}}
\fi

% theindex-Environment neu definieren ohne reledmac
\makeatletter
\renewenvironment{theindex}{%
  \ifkorrekturansicht
    \section*{\indexname}%
  \else
    \subsubsection*{Index der erwähnten Entitäten}%
  \fi
  \setlength{\parindent}{0pt}%
  \setlength{\parskip}{0pt plus 0.3pt}%
  \let\item\@idxitem
}{%
  \ifkorrekturansicht\clearpage\fi
}
\makeatother

\IfFileExists{\jobname-pw.ind}{\input{\jobname-pw.ind}}{}

% Quellenangabe nur in der Leseansicht
\ifkorrekturansicht\else
% Fallback-Definitionen, falls die .tex-Datei \titel etc. nicht gesetzt hat
\providecommand{\titel}{}
\providecommand{\editorInnen}{}
\providecommand{\dateiname}{\jobname}

\vspace{3cm}

\vfill

\footnotesize
\textsc{Quelle}: \titel. Herausgegeben von {\editorInnen}. In: \emph{Arthur Schnitzler: Briefwechsel mit Autorinnen und Autoren}.
 Digitale Edition, https://schnitzler-briefe.acdh.oeaw.ac.at/{\dateiname}.html (Stand \today)
\fi

\end{document}


