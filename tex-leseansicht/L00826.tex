%% latex-leseansicht-vorspann.tex
%% Vorspann für die Leseansicht.
%% Lädt die gemeinsame Datei latex-vorspann.tex mit nicht gesetztem Schalter.

\newif\ifkorrekturansicht
\korrekturansichtfalse

\input{../tex-inputs/latex-vorspann}


\section[Arthur Schnitzler an Hugo von Hofmannsthal, 23. 7. 1898]{L00826 Arthur Schnitzler an Hugo von Hofmannsthal, 23. 7. 1898}
\nopagebreak\mylabel{L00826v}
\rehead{ }\normalsize\beginnumbering\briefempfaengerindex{Hofmannsthal, Hugo von@\textsc{Hofmannsthal, Hugo von}!zzzSchnitzler, Arthur@\emph{von Arthur Schnitzler}!1898-07-231@{23. 7. 1898}|(be}
\toendnotes[C]{\smallbreak\pagebreak[2]}
\correspDesc{Versand  durch Arthur Schnitzler am 23. 7. 1898 in Bad Gastein
\newline{}Erhalt  durch Hugo von Hofmannsthal im Zeitraum [24. 7. 1898
                  – 28. 7. 1898?] in Mödling}\toendnotes[C]{\smallbreak}
\Standort{FDH, Hs-30885,72.}
\physDesc{Brief, 1 Blatt, 4 Seiten, 1284 Zeichen
\newline{}Handschrift: Bleistift, deutsche Kurrent}
\buchAbdrucke{\weitereDrucke{Hugo von Hofmannsthal, Arthur Schnitzler: \emph{Briefwechsel}. Herausgegeben von Therese Nickl und Heinrich Schnitzler. Frankfurt am Main: \emph{S. Fischer} 1964, S. 107.} }\toendnotes[C]{\smallbreak}
\pstart
           \raggedleft{}{\pb}\textsc{Bad Gastein}\oindex{Bad Gastein@\textbf{Bad Gastein}, \emph{Hauptstadt}|pw}{ }23. 7. 98\pend
           \vspace{0.5em}
\pstart
           Mein lieber Hugo, ich riskir noch ein paar Zeilen nach \textsc{Czortków}\oindex{Tschortkiw@\textbf{Tschortkiw}, \emph{Hauptstadt}|pw} – Sie wiſſen{ }ſchon, dſs ich bei Ihren Eltern\pwindex{Hofmannsthal, Hugo August von 21.\,12.\,1841 Wien – 8.\,12.\,1915 ebd.@\textsc{Hofmannsthal, Hugo August von} (21.\,12.\,1841 Wien – 8.\,12.\,1915 ebd.), \emph{Bankdirektor}|pwv}\pwindex{Hofmannsthal, Anna von 27.\,1.\,1849 Wien – 22.\,3.\,1904 Sanatorium Fürth@\textsc{Hofmannsthal, Anna von} (27.\,1.\,1849 Wien – 22.\,3.\,1904 Sanatorium Fürth)|pwv} war, die von viel
               Herzlichkeit gegen mich waren. Ich hab mich{ }ſehr gefreut. Die Sp.
                  Mädeln\pwindex{Schmidl, Paula 13.\,10.\,1874 Wien – 24.\,9.\,1966 Jerusalem@\textsc{Schmidl, Paula} (13.\,10.\,1874 Wien – 24.\,9.\,1966 Jerusalem)|pw}\pwindex{Wassermann, Julie 5.\,12.\,1876 Wien – April 1963 Zürich@\textsc{Wassermann, Julie} (5.\,12.\,1876 Wien – April 1963 Zürich), \emph{Schriftstellerin}|pw}\pwindex{Ulmann, Agnes 23.\,12.\,1875 Wien – 1.\,4.\,1942 New York City@\textsc{Ulmann, Agnes} (23.\,12.\,1875 Wien – 1.\,4.\,1942 New York City), \emph{Malerin, Bildhauerin}|pw}\pwindex{Sgal, Emilie 7.\,5.\,1871 Wien – 3.\,12.\,1938 Den Haag@\textsc{Sgal, Emilie} (7.\,5.\,1871 Wien – 3.\,12.\,1938 Den Haag)|pw}\pwindex{Michaelis, Dora 23.\,5.\,1881 Wien – 22.\,1.\,1946 New York City@\textsc{Michaelis, Dora} (23.\,5.\,1881 Wien – 22.\,1.\,1946 New York City)|pw}\pwindex{Knepler, Sophie 13.\,5.\,1872 – 30.\,10.\,1908@\textsc{Knepler, Sophie} (13.\,5.\,1872 – 30.\,10.\,1908)|pw} haben mich herumgeführt und \introOben{}mir\introOben{} die Stätten
               gezeigt, wo Sie gedichtet haben – es war nur wenig Zeit, die \textsc{Weil{\pb}guni}\oindex{Hotel Weilguni@\textbf{Hotel Weilguni}, \emph{Hotel}|pw}ſche \textsc{table d’hôte} drohte – und{ }ſo kam eine rührende
               Haſt über die Geſchöpfe. Es iſt was hübſches um dieſe kleinen Unſterblichkeiten –
               über die großen werden wir nicht{ }ſo gemütlich plaudern können; fürcht ich; es wird zu{ }ſpät{ }ſein. –\pend
           
\pstart
           Herrliches Wetter hab ich überall; hier ganz beſonders. Montag fahr ich
               nach Salzburg\oindex{Salzburg@\textbf{Salzburg}, \emph{Verwaltungsgebiet}|pw}. Warten Sie {\pb}jedenfalls eine neue Nachricht ab, bevor Sie mir{ }ſchreiben. Auf Richard\pwindex{Beer-Hofmann, Richard 11.\,7.\,1866 Wien – 26.\,9.\,1945 New York City@\textsc{Beer-Hofmann, Richard} (11.\,7.\,1866 Wien – 26.\,9.\,1945 New York City), \emph{Schriftsteller}|pw}{ }ſcheints werden wir verzichten müſſen – doch \uline{Sie}{ }\introOben{}allein\introOben{} werden ihn{ }ſpäter haben, geht aus einem eiligen Brief
               von ihm hervor. –\pend
           
\pstart
           Gearbeitet hab ich nichts; doch iſt trotz allem, was bedrückt, eine gewiſſe Fülle in
               mir, ja{ }ſogar die Neigung dieſer Fülle,{ }ſich zu {\pb}ordnen.\pend
           
\pstart
           Ich hoffe Sie kö{\geminationn}en mir bald{ }ſagen, wie es Ihnen \introOben{}oder vielmehr\introOben{} daſs es Ihnen beſſer geht. Was werden Sie{ }ſchreiben. In mir iſt der Streit zwiſchen dem Stück\pwindex{Schnitzler, Arthur 15.\,5.\,1862 Wien – 21.\,10.\,1931 ebd.@\textsc{Schnitzler, Arthur} (15.\,5.\,1862 Wien – 21.\,10.\,1931 ebd.), \emph{Schriftsteller, Mediziner}!Schleier der Beatrice. Schauspiel in fünf Akten@\strich\emph{Der Schleier der Beatrice. Schauspiel in fünf Akten}|pwv} und dem Roman\pwindex{Schnitzler, Arthur 15.\,5.\,1862 Wien – 21.\,10.\,1931 ebd.@\textsc{Schnitzler, Arthur} (15.\,5.\,1862 Wien – 21.\,10.\,1931 ebd.), \emph{Schriftsteller, Mediziner}!Weg ins Freie. Roman@\strich\emph{Der Weg ins Freie. Roman}|pwv} noch nicht entſchieden.\pend
           
\pstart
           Leben Sie wohl – ich{ }ſende den Brief doch lieber nach Mödling\oindex{Mödling@\textbf{Mödling}, \emph{Hauptstadt}|pw}; möge er Sie heiter u. herzlich begrüßen.\pend
           \pstart Ihr \spacefill\mbox{Arthur.}\pend{}\selectlanguage{ngerman}\endnumbering\briefempfaengerindex{Hofmannsthal, Hugo von@\textsc{Hofmannsthal, Hugo von}!zzzSchnitzler, Arthur@\emph{von Arthur Schnitzler}!1898-07-231@{23. 7. 1898}|)be}\mylabel{L00826h}  \newcommand{\dateiname}{L00826}\newcommand{\titel}{Arthur Schnitzler an Hugo von Hofmannsthal, 23. 7. 1898}\newcommand{\editorInnen}{Martin Anton Müller und Gerd-Hermann Susen}%% latex-leseansicht-abspann.tex
%% Abspann für die Leseansicht.
%% Der Schalter \ifkorrekturansicht ist bereits durch den Vorspann gesetzt.

%% latex-abspann.tex
%% Gemeinsamer Abspann für Korrekturansicht und Leseansicht.
%% Setzt den Schalter \ifkorrekturansicht voraus (gesetzt in den
%% einbindenden Dateien latex-korrekturansicht-abspann.tex bzw.
%% latex-leseansicht-abspann.tex).
%% ---------------------------------------------------------------

\normalsize

% Das esempio-Environment wird nur in der Leseansicht benötigt
\ifkorrekturansicht\else
\newenvironment{esempio}[3]%
{
    \vspace{1.5ex}
    \rlap{\underline{#1}}
    \par
    \setlength{\parindent}{0cm}
    \nopagebreak
    \leftskip=#2cm
    \rightskip=#3cm
}
{
    \par
}
\fi

\doendnotes{C}
\bigskip
\vfill

\clearpage

\footnotesize

\ifkorrekturansicht
  \lohead{\textsc{register}}
\fi

% theindex-Environment neu definieren ohne reledmac
\makeatletter
\renewenvironment{theindex}{%
  \ifkorrekturansicht
    \section*{\indexname}%
  \else
    \subsubsection*{Index der erwähnten Entitäten}%
  \fi
  \setlength{\parindent}{0pt}%
  \setlength{\parskip}{0pt plus 0.3pt}%
  \let\item\@idxitem
}{%
  \ifkorrekturansicht\clearpage\fi
}
\makeatother

\IfFileExists{\jobname-pw.ind}{\input{\jobname-pw.ind}}{}

% Quellenangabe nur in der Leseansicht
\ifkorrekturansicht\else
% Fallback-Definitionen, falls die .tex-Datei \titel etc. nicht gesetzt hat
\providecommand{\titel}{}
\providecommand{\editorInnen}{}
\providecommand{\dateiname}{\jobname}

\vspace{3cm}

\vfill

\footnotesize
\textsc{Quelle}: \titel. Herausgegeben von {\editorInnen}. In: \emph{Arthur Schnitzler: Briefwechsel mit Autorinnen und Autoren}.
 Digitale Edition, https://schnitzler-briefe.acdh.oeaw.ac.at/{\dateiname}.html (Stand \today)
\fi

\end{document}


