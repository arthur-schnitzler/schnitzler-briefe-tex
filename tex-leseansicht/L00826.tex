%% latex-leseansicht-vorspann.tex
%% Vorspann für die Leseansicht.
%% Lädt die gemeinsame Datei latex-vorspann.tex mit nicht gesetztem Schalter.

\newif\ifkorrekturansicht
\korrekturansichtfalse

\input{../tex-inputs/latex-vorspann}


         
         \newcommand{\erwaehntePersonen}{Personen: }
         \newcommand{\erwaehnteInstitutionen}{}
         \newcommand{\erwaehnteOrte}{}
         \newcommand{\erwaehnteWerke}{
               \section[Arthur Schnitzler an Hugo von Hofmannsthal, 23. 7. 1898]{ Arthur Schnitzler an Hugo von Hofmannsthal, 23. 7. 1898}\nopagebreak\mylabel{v}\rehead{ }\begin{ledgroupsized}[t]{13cm}\normalsize\beginnumbering \toendnotes[C]{\smallbreak\pagebreak[2]} \Standort{FDH, Hs-30885,72.}
\physDesc{Brief, 1 Blatt, 4 Seiten
\newline{}Handschrift: Bleistift, deutsche Kurrent}\buchAbdrucke{\weitereDrucke{Hugo von Hofmannsthal, Arthur Schnitzler: \emph{Briefwechsel}. Hg. Therese Nickl und Heinrich Schnitzler. Frankfurt am Main: \emph{S. Fischer} 1964, S. 107.} }\toendnotes[C]{\smallbreak}\pstart
           \raggedleft{}{\pb}\textsc{Bad Gastein}\oindex{XXXX Ortsangabe fehlt|pw}{ }23. 7. 98\pend
           \pstart
           Mein lieber Hugo, ich riskir noch ein paar Zeilen nach \textsc{Czortków}\oindex{XXXX Ortsangabe fehlt|pw} – Sie wiſſen ſchon, dſs ich bei Ihren Eltern\pwindex{\textcolor{red}{\textsuperscript{XXXX1 indx}}|pwv}\pwindex{\textcolor{red}{\textsuperscript{XXXX1 indx}}|pwv} war, die von viel
                    Herzlichkeit gegen mich waren. Ich hab mich ſehr gefreut. Die Sp. Mädeln\pwindex{\textcolor{red}{\textsuperscript{XXXX1 indx}}|pw}\pwindex{\textcolor{red}{\textsuperscript{XXXX1 indx}}|pw}\pwindex{\textcolor{red}{\textsuperscript{XXXX1 indx}}|pw}\pwindex{\textcolor{red}{\textsuperscript{XXXX1 indx}}|pw}\pwindex{\textcolor{red}{\textsuperscript{XXXX1 indx}}|pw}\pwindex{\textcolor{red}{\textsuperscript{XXXX1 indx}}|pw}
                    haben mich herumgeführt und \introOben{}mir\introOben{} die Stätten gezeigt, wo
                    Sie gedichtet haben – es war nur wenig Zeit, die \textsc{Weil{\pb}guni}\oindex{XXXX Ortsangabe fehlt|pw}ſche \textsc{table d’hôte} drohte – und ſo kam eine
                    rührende Haſt über die Geſchöpfe. Es iſt was hübſches um dieſe kleinen
                    Unſterblichkeiten – über die großen werden wir nicht ſo gemütlich plaudern
                    können; fürcht ich; es wird zu ſpät ſein. –\pend
           \pstart
           Herrliches Wetter hab ich überall; hier ganz beſonders. Montag fahr
                    ich nach Salzburg\oindex{XXXX Ortsangabe fehlt|pw}. Warten Sie {\pb}jedenfalls eine neue Nachricht ab, bevor Sie mir
                    ſchreiben. Auf Richard\pwindex{\textcolor{red}{\textsuperscript{XXXX1 indx}}|pw}{ }ſcheints werden wir verzichten müſſen – doch
                        \uline{Sie}{ }\introOben{}allein\introOben{} werden ihn ſpäter haben, geht aus einem eiligen
                    Brief von ihm hervor. –\pend
           \pstart
           Gearbeitet hab ich nichts; doch iſt trotz allem, was bedrückt, eine gewiſſe Fülle
                    in mir, ja ſogar die Neigung dieſer Fülle, ſich zu {\pb}ordnen.\pend
           \pstart
           Ich hoffe Sie kö{\geminationn}en mir bald ſagen, wie es Ihnen
                        \introOben{}oder vielmehr\introOben{} daſs es Ihnen beſſer geht. Was werden
                    Sie ſchreiben. In mir iſt der Streit zwiſchen dem Stück\textcolor{red}{\textsuperscript{XXXX indx}} und dem Roman\textcolor{red}{\textsuperscript{XXXX indx}} noch nicht entſchieden.\pend
           \pstart
           Leben Sie wohl – ich ſende den Brief doch lieber nach Mödling\oindex{XXXX Ortsangabe fehlt|pw}; möge er Sie heiter u. herzlich begrüßen.\pend
           \pstart Ihr \spacefill\mbox{Arthur.}\pend{}
         
         \endnumbering\mylabel{h}\end{ledgroupsized}  \newcommand{\dateiname}{L00826}\newcommand{\titel}{Arthur Schnitzler an Hugo von Hofmannsthal, 23. 7. 1898}\newcommand{\editorInnen}{Martin Anton Müller und Gerd-Hermann Susen}%% latex-leseansicht-abspann.tex
%% Abspann für die Leseansicht.
%% Der Schalter \ifkorrekturansicht ist bereits durch den Vorspann gesetzt.

%% latex-abspann.tex
%% Gemeinsamer Abspann für Korrekturansicht und Leseansicht.
%% Setzt den Schalter \ifkorrekturansicht voraus (gesetzt in den
%% einbindenden Dateien latex-korrekturansicht-abspann.tex bzw.
%% latex-leseansicht-abspann.tex).
%% ---------------------------------------------------------------

\normalsize

% Das esempio-Environment wird nur in der Leseansicht benötigt
\ifkorrekturansicht\else
\newenvironment{esempio}[3]%
{
    \vspace{1.5ex}
    \rlap{\underline{#1}}
    \par
    \setlength{\parindent}{0cm}
    \nopagebreak
    \leftskip=#2cm
    \rightskip=#3cm
}
{
    \par
}
\fi

\doendnotes{C}
\bigskip
\vfill

\clearpage

\footnotesize

\ifkorrekturansicht
  \lohead{\textsc{register}}
\fi

% theindex-Environment neu definieren ohne reledmac
\makeatletter
\renewenvironment{theindex}{%
  \ifkorrekturansicht
    \section*{\indexname}%
  \else
    \subsubsection*{Index der erwähnten Entitäten}%
  \fi
  \setlength{\parindent}{0pt}%
  \setlength{\parskip}{0pt plus 0.3pt}%
  \let\item\@idxitem
}{%
  \ifkorrekturansicht\clearpage\fi
}
\makeatother

\IfFileExists{\jobname-pw.ind}{\input{\jobname-pw.ind}}{}

% Quellenangabe nur in der Leseansicht
\ifkorrekturansicht\else
% Fallback-Definitionen, falls die .tex-Datei \titel etc. nicht gesetzt hat
\providecommand{\titel}{}
\providecommand{\editorInnen}{}
\providecommand{\dateiname}{\jobname}

\vspace{3cm}

\vfill

\footnotesize
\textsc{Quelle}: \titel. Herausgegeben von {\editorInnen}. In: \emph{Arthur Schnitzler: Briefwechsel mit Autorinnen und Autoren}.
 Digitale Edition, https://schnitzler-briefe.acdh.oeaw.ac.at/{\dateiname}.html (Stand \today)
\fi

\end{document}


      