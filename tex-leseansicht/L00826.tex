%% latex-korrekturansicht-vorspann.tex
%% Vorspann für die Korrekturansicht.
%% Lädt die gemeinsame Datei latex-vorspann.tex mit gesetztem Schalter.

\newif\ifkorrekturansicht
\korrekturansichttrue

\input{../tex-inputs/latex-vorspann}


\section[Arthur Schnitzler an Hugo von Hofmannsthal, 23. 7. 1898]{L00826 Arthur Schnitzler an Hugo von Hofmannsthal, 23. 7. 1898}
\nopagebreak\mylabel{L00826v}
\rehead{ }\normalsize\beginnumbering\briefempfaengerindex{Hofmannsthal, Hugo von@\textsc{Hofmannsthal, Hugo von}!zzzSchnitzler, Arthur@\emph{von Arthur Schnitzler}!1898-07-231@{23. 7. 1898}|(be}
\toendnotes[C]{\smallbreak\pagebreak[2]}\Standort{FDH, Hs-30885,72.}
\physDesc{Brief, 1 Blatt, 4 Seiten, 1284 Zeichen
\newline{}Handschrift: Bleistift, deutsche Kurrent}
\buchAbdrucke{\weitereDrucke{Hugo von Hofmannsthal, Arthur Schnitzler: \emph{Briefwechsel}. Frankfurt am Main: \emph{S. Fischer} 1964, S. 107.} }\toendnotes[C]{\smallbreak}
\pstart
           \raggedleft{}{\pb}\textsc{Bad Gastein}\oindex{Bad Gastein@\textbf{Bad Gastein}, \emph{P.PPLA3}|pw}{ }23. 7. 98\pend
           \vspace{0.5em}
\pstart
           Mein lieber Hugo, ich riskir noch ein paar Zeilen nach \textsc{Czortków}\oindex{Tschortkiw@\textbf{Tschortkiw}, \emph{P.PPLA2}|pw} – Sie wiſſen ſchon, dſs ich bei Ihren Eltern\pwindex{Hofmannsthal, Hugo August von 21.12.1841 – 08.12.1915@\textsc{Hofmannsthal, Hugo August von} (21.12.1841 – 08.12.1915), \emph{Bankdirektor/Bankdirektorin}|pwv}\pwindex{Hofmannsthal, Anna von 27.01.1849 – 22.03.1904@\textsc{Hofmannsthal, Anna von} (27.01.1849 – 22.03.1904)|pwv} war, die von viel
               Herzlichkeit gegen mich waren. Ich hab mich ſehr gefreut. Die Sp.
                  Mädeln\pwindex{Schmidl, Paula 13.10.1874 – 24.09.1966@\textsc{Schmidl, Paula} (13.10.1874 – 24.09.1966)|pw}\pwindex{Wassermann, Julie 05.12.1876 – April 1963@\textsc{Wassermann, Julie} (05.12.1876 – April 1963), \emph{Schriftsteller/Schriftstellerin}|pw}\pwindex{Ulmann, Agnes 23. 12. 1875 – 1. 4. 1942@\textsc{Ulmann, Agnes} (23. 12. 1875 – 1. 4. 1942), \emph{Maler/Malerin, Bildhauer/Bildhauerin}|pw}\pwindex{Sgal, Emilie 07.05.1871 – 3.12.1938@\textsc{Sgal, Emilie} (07.05.1871 – 3.12.1938)|pw}\pwindex{Michaelis, Dora 23.05.1881 – 22.01.1946@\textsc{Michaelis, Dora} (23.05.1881 – 22.01.1946)|pw}\pwindex{Knepler, Sophie 13.5.1872 – 30.10.1908@\textsc{Knepler, Sophie} (13.5.1872 – 30.10.1908)|pw} haben mich herumgeführt und \introOben{}mir\introOben{} die Stätten
               gezeigt, wo Sie gedichtet haben – es war nur wenig Zeit, die \textsc{Weil{\pb}guni}\oindex{Hotel Weilguni@\textbf{Hotel Weilguni}, \emph{Hotel (K.HTL)}|pw}ſche \textsc{table d’hôte} drohte – und ſo kam eine rührende
               Haſt über die Geſchöpfe. Es iſt was hübſches um dieſe kleinen Unſterblichkeiten –
               über die großen werden wir nicht ſo gemütlich plaudern können; fürcht ich; es wird zu
               ſpät ſein. –\pend
           
\pstart
           Herrliches Wetter hab ich überall; hier ganz beſonders. Montag fahr ich
               nach Salzburg\oindex{Salzburg@\textbf{Salzburg}, \emph{A.ADM2}|pw}. Warten Sie {\pb}jedenfalls eine neue Nachricht ab, bevor Sie mir
               ſchreiben. Auf Richard\pwindex{Beer-Hofmann, Richard 1866-07-11 – 1945-09-26@\textsc{Beer-Hofmann, Richard} (1866-07-11 – 1945-09-26), \emph{Schriftsteller/Schriftstellerin}|pw}{ }ſcheints werden wir verzichten müſſen – doch \uline{Sie}{ }\introOben{}allein\introOben{} werden ihn ſpäter haben, geht aus einem eiligen Brief
               von ihm hervor. –\pend
           
\pstart
           Gearbeitet hab ich nichts; doch iſt trotz allem, was bedrückt, eine gewiſſe Fülle in
               mir, ja ſogar die Neigung dieſer Fülle, ſich zu {\pb}ordnen.\pend
           
\pstart
           Ich hoffe Sie kö{\geminationn}en mir bald ſagen, wie es Ihnen \introOben{}oder vielmehr\introOben{} daſs es Ihnen beſſer geht. Was werden Sie
               ſchreiben. In mir iſt der Streit zwiſchen dem Stück\pwindex{Schleier der Beatrice. Schauspiel in fuenf Akten@\emph{Der Schleier der Beatrice. Schauspiel in fünf Akten}|pwv} und dem Roman\pwindex{Weg ins Freie. Roman@\emph{Der Weg ins Freie. Roman}|pwv} noch nicht entſchieden.\pend
           
\pstart
           Leben Sie wohl – ich ſende den Brief doch lieber nach Mödling\oindex{Moedling@\textbf{Mödling}, \emph{P.PPLA3}|pw}; möge er Sie heiter u. herzlich begrüßen.\pend
           \pstart Ihr \spacefill\mbox{Arthur.}\pend{}\selectlanguage{ngerman}\endnumbering\briefempfaengerindex{Hofmannsthal, Hugo von@\textsc{Hofmannsthal, Hugo von}!zzzSchnitzler, Arthur@\emph{von Arthur Schnitzler}!1898-07-231@{23. 7. 1898}|)be}\mylabel{L00826h}  \normalsize

\doendnotes{C}
\bigskip
\vfill

\clearpage

\footnotesize

\lohead{\textsc{register}}

% Definiere theindex-Environment komplett neu ohne reledmac
\makeatletter
\renewenvironment{theindex}{%
  \section*{\indexname}%
  \setlength{\parindent}{0pt}%
  \setlength{\parskip}{0pt plus 0.3pt}%
  \let\item\@idxitem
}{%
  \clearpage
}
\makeatother

\IfFileExists{\jobname-pw.ind}{\input{\jobname-pw.ind}}{}

\end{document}

      