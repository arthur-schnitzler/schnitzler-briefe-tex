%% latex-korrekturansicht-vorspann.tex
%% Vorspann für die Korrekturansicht.
%% Lädt die gemeinsame Datei latex-vorspann.tex mit gesetztem Schalter.

\newif\ifkorrekturansicht
\korrekturansichttrue

\input{../tex-inputs/latex-vorspann}


\section[Arthur Schnitzler an Hugo von Hofmannsthal, 8. 7. 1897]{L00694 Arthur Schnitzler an Hugo von Hofmannsthal, 8. 7. 1897}
\nopagebreak\mylabel{L00694v}
\rehead{ }\normalsize\beginnumbering\briefempfaengerindex{Hofmannsthal, Hugo von@\textsc{Hofmannsthal, Hugo von}!zzzSchnitzler, Arthur@\emph{von Arthur Schnitzler}!1897-07-081@{8. 7. 1897}|(be}
\toendnotes[C]{\smallbreak\pagebreak[2]}\Standort{FDH, Hs-30885,59.}
\physDesc{Brief, 1 Blatt, 4 Seiten, 1952 Zeichen
\newline{}Handschrift: schwarze Tinte, deutsche Kurrent}
\buchAbdrucke{\weitereDrucke{1) Hugo von Hofmannsthal, Arthur Schnitzler: \emph{Briefwechsel}. Frankfurt am Main: \emph{S. Fischer} 1964, S. 88–89.} \weitereDrucke{2) Arthur Schnitzler: \emph{Briefe 1875–1912}. Frankfurt am Main: \emph{S. Fischer} 1981, S. 334–335.} }\toendnotes[C]{\smallbreak}
\pstart
           {\pb}\textsc{Ischl}\oindex{Bad Ischl@\textbf{Bad Ischl}, \emph{P.PPL}|pw}{ }8. 7. 97\pend
           \vspace{0.5em}
\pstart
           Mein lieber Hugo, geſtern iſt Ihr Brief aus der Fuſch\oindex{Bad Fusch@\textbf{Bad Fusch}, \emph{A.ADM3}|pw} geko{\geminationm}en. Ich freue mich ſehr,
               dſs es Ihnen gut geht und weiſs dſs manche von den Verſen die Sie »verſuchen«, Ihnen
               gelingen werden. Glauben Sie das nicht ſelbſt? Ich ſelbſt ſchreibe an einem Stück\pwindex{Vermaechtnis. Schauspiel in drei Akten@\emph{Das Vermächtnis. Schauspiel in drei Akten}|pwv}, deſſen zweiten Akt ich
               heute bego{\geminationn}en habe. Es iſt nicht das, was ich mir
                  vorgeno{\geminationm}en habe, ſondern ein andres, das mir als
               Einfall bereits vor ein paar Monaten in Wien\oindex{Wien@\textbf{Wien}, \emph{A.ADM2}|pw}
                  geko{\geminationm}en und mir plötzlich, in den zwei erſten Tagen
               meines Iſchl\oindex{Bad Ischl@\textbf{Bad Ischl}, \emph{P.PPL}|pw}er {\pb}Aufenthalts mit großer Lebendigkeit, Scene für Scene klar geworden iſt. Ich habe
               den erſten Akt\pwindex{Vermaechtnis. Schauspiel in drei Akten@\emph{Das Vermächtnis. Schauspiel in drei Akten}|pwv} mit viel Liebe
               geſchrieben, bin gegen den Schluſs mistrauiſch geworden und fand ihn beim Durchleſen
               vorgeſtern blaſs. Aus verschiedenen Gründen iſt die ganze Sti{\geminationm}ung wieder ins dunklere hineingerathen, aber die
               Hoffnung, dſs es wieder beſſer wird, darf beſtehn. Ich werde weiter arbeiten, wie man
               unter drohenden Wolken weiterfährt; (was doch eigentlich ein recht ſtupider Vergleich
               iſt.) ((Ich hätt ihn doch ausſtreichen können, ganz einfach?)) \pend
           
\pstart
           {\pb}Ich muſs vielleicht bald nach Wien\oindex{Wien@\textbf{Wien}, \emph{A.ADM2}|pw}, da ich in der Wohnungsfrage in der beka{\geminationn}ten, noch mancherlei oder vielmehr alles zu ordnen
               habe. Das urſprünglich geplante Häuschen im Gebirg ist mir weggeſchnappt worden. Es
               iſt ſehr ärgerlich. Natürlich bleibt es trotzdem bei unſerm Salzburg\oindex{Salzburg@\textbf{Salzburg}, \emph{A.ADM2}|pw}, und ich freu mich ſehr darauf. Sagen Sie mir nur
               gleich das genaue Datum, da ich mit den Tagen haushalten muſs.\pend
           
\pstart
           Morgen ſchicke ich Ihnen den 2. Band Mozart\pwindex{Mozart, Wolfgang Amadeus 27.01.1756 – 05.12.1791@\textsc{Mozart, Wolfgang Amadeus} (27.01.1756 – 05.12.1791), \emph{Komponist/Komponistin}|pw}\pwindex{W. A. Mozart@\emph{W. A. Mozart}|pwv}. – Richard\pwindex{Beer-Hofmann, Richard 1866-07-11 – 1945-09-26@\textsc{Beer-Hofmann, Richard} (1866-07-11 – 1945-09-26), \emph{Schriftsteller/Schriftstellerin}|pw} arbeitet wirklich; er ſcheint
               im dritten Capitel\pwindex{Tod Georgs@\emph{Der Tod Georgs}|pwv} zu ſein.
                  {\pb}Wenigſtens hat er kaum zu was anderm Zeit und ist eine
               Radelraunzen wie ein kleines Kind.\pend
           
\pstart
           Neulich bin ich nach Unterach\oindex{Unterach am Attersee@\textbf{Unterach am Attersee}, \emph{P.PPL}|pw} zu Stri\pwindex{Strisower, Bernhard 31.10.1847 – 21.12.1900@\textsc{Strisower, Bernhard} (31.10.1847 – 21.12.1900), \emph{Bankier/Bankierin}|pw}\pwindex{Strisower, Friederike 31.05.1854 – 13.12.1922@\textsc{Strisower, Friederike} (31.05.1854 – 13.12.1922)|pw}’s geradelt; ſonſt mach ich nur
               ganz kleine Spazierfahrten, und plaudre mit einer merkwürdig geſcheiten Frau\pwindex{Freudenthal, Rosa 1862 – 18.06.1905@\textsc{Freudenthal, Rosa} (1862 – 18.06.1905)|pwv}{ }ſehr viel, die Humor hat, und ich verſuche mich zu
               erinnern, ob ich ſchon je eine Frau mit Humor gekannt habe. –\pend
           
\pstart
           Schreiben Sie mir bald.\pend
           
\pstart
           Ich leſe noch immer \textsc{Tolstoi}\pwindex{Tolstoi, Leo N. von 09.09.1828 – 20.11.1910@\textsc{Tolstoi, Leo N. von} (09.09.1828 – 20.11.1910), \emph{Schriftsteller/Schriftstellerin, Schriftsteller/Schriftstellerin, Krimiautor/Krimiautorin}|pw}\pwindex{Krieg und Frieden@\emph{Krieg und Frieden}|pwv} u \textsc{Brandes}\pwindex{Brandes, Georg 04.02.1842 – 19.02.1927@\textsc{Brandes, Georg} (04.02.1842 – 19.02.1927)|pw}\pwindex{William Shakespeare@\emph{William Shakespeare}|pwv}.\pend
           
\pstart
           Herzlich der Ihre{\\[\baselineskip]}\spacefill\mbox{Arthur.}\pend
           \leftskip=0em{}\selectlanguage{ngerman}\endnumbering\briefempfaengerindex{Hofmannsthal, Hugo von@\textsc{Hofmannsthal, Hugo von}!zzzSchnitzler, Arthur@\emph{von Arthur Schnitzler}!1897-07-081@{8. 7. 1897}|)be}\mylabel{L00694h}  \normalsize

\doendnotes{C}
\bigskip
\vfill

\clearpage

\footnotesize

\lohead{\textsc{register}}

% Definiere theindex-Environment komplett neu ohne reledmac
\makeatletter
\renewenvironment{theindex}{%
  \section*{\indexname}%
  \setlength{\parindent}{0pt}%
  \setlength{\parskip}{0pt plus 0.3pt}%
  \let\item\@idxitem
}{%
  \clearpage
}
\makeatother

\IfFileExists{\jobname-pw.ind}{\input{\jobname-pw.ind}}{}

\end{document}

      