%% latex-leseansicht-vorspann.tex
%% Vorspann für die Leseansicht.
%% Lädt die gemeinsame Datei latex-vorspann.tex mit nicht gesetztem Schalter.

\newif\ifkorrekturansicht
\korrekturansichtfalse

\input{../tex-inputs/latex-vorspann}


\section[Arthur Schnitzler an Hugo von Hofmannsthal, 8. 7. 1897]{L00694 Arthur Schnitzler an Hugo von Hofmannsthal, 8. 7. 1897}
\nopagebreak\mylabel{L00694v}
\rehead{ }\normalsize\beginnumbering\briefempfaengerindex{Hofmannsthal, Hugo von@\textsc{Hofmannsthal, Hugo von}!zzzSchnitzler, Arthur@\emph{von Arthur Schnitzler}!1897-07-081@{8. 7. 1897}|(be}
\toendnotes[C]{\smallbreak\pagebreak[2]}
\correspDesc{Versand  durch Arthur Schnitzler am 8. 7. 1897 in Bad Ischl
\newline{}Erhalt  durch Hugo von Hofmannsthal im Zeitraum [9. 7. 1897
                  – 13. 7. 1897?] in Bad Fusch}\toendnotes[C]{\smallbreak}
\Standort{FDH, Hs-30885,59.}
\physDesc{Brief, 1 Blatt, 4 Seiten, 1952 Zeichen
\newline{}Handschrift: schwarze Tinte, deutsche Kurrent}
\buchAbdrucke{\weitereDrucke{1) Hugo von Hofmannsthal, Arthur Schnitzler: \emph{Briefwechsel}. Herausgegeben von Therese Nickl und Heinrich Schnitzler. Frankfurt am Main: \emph{S. Fischer} 1964, S. 88–89.} \weitereDrucke{2) Arthur Schnitzler: \emph{Briefe 1875–1912}. Herausgegeben von Therese Nickl und Heinrich Schnitzler. Frankfurt am Main: \emph{S. Fischer} 1981, S. 334–335.} }\toendnotes[C]{\smallbreak}
\pstart
           {\pb}\textsc{Ischl}\oindex{Bad Ischl@\textbf{Bad Ischl}|pw}{ }8. 7. 97\pend
           \vspace{0.5em}
\pstart
           Mein lieber Hugo, geſtern iſt Ihr Brief aus der Fuſch\oindex{Bad Fusch@\textbf{Bad Fusch}|pw} geko{\geminationm}en. Ich freue mich{ }ſehr,
               dſs es Ihnen gut geht und weiſs dſs manche von den Verſen die Sie »verſuchen«, Ihnen
               gelingen werden. Glauben Sie das nicht{ }ſelbſt? Ich{ }ſelbſt{ }ſchreibe an einem Stück\pwindex{Schnitzler, Arthur 15.\,5.\,1862 Wien – 21.\,10.\,1931 ebd.@\textsc{Schnitzler, Arthur} (15.\,5.\,1862 Wien – 21.\,10.\,1931 ebd.), \emph{Schriftsteller, Mediziner}!Vermächtnis. Schauspiel in drei Akten@\strich\emph{Das Vermächtnis. Schauspiel in drei Akten}|pwv}, deſſen zweiten Akt ich
               heute bego{\geminationn}en habe. Es iſt nicht das, was ich mir
                  vorgeno{\geminationm}en habe,{ }ſondern ein andres, das mir als
               Einfall bereits vor ein paar Monaten in Wien\oindex{Wien@\textbf{Wien}, \emph{Verwaltungsgebiet}|pw}
                  geko{\geminationm}en und mir plötzlich, in den zwei erſten Tagen
               meines Iſchl\oindex{Bad Ischl@\textbf{Bad Ischl}|pw}er {\pb}Aufenthalts mit großer Lebendigkeit, Scene für Scene klar geworden iſt. Ich habe
               den erſten Akt\pwindex{Schnitzler, Arthur 15.\,5.\,1862 Wien – 21.\,10.\,1931 ebd.@\textsc{Schnitzler, Arthur} (15.\,5.\,1862 Wien – 21.\,10.\,1931 ebd.), \emph{Schriftsteller, Mediziner}!Vermächtnis. Schauspiel in drei Akten@\strich\emph{Das Vermächtnis. Schauspiel in drei Akten}|pwv} mit viel Liebe
               geſchrieben, bin gegen den Schluſs mistrauiſch geworden und fand ihn beim Durchleſen
               vorgeſtern blaſs. Aus verschiedenen Gründen iſt die ganze Sti{\geminationm}ung wieder ins dunklere hineingerathen, aber die
               Hoffnung, dſs es wieder beſſer wird, darf beſtehn. Ich werde weiter arbeiten, wie man
               unter drohenden Wolken weiterfährt; (was doch eigentlich ein recht{ }ſtupider Vergleich
               iſt.) ((Ich hätt ihn doch ausſtreichen können, ganz einfach?))\pend
           
\pstart
           {\pb}Ich muſs vielleicht bald nach Wien\oindex{Wien@\textbf{Wien}, \emph{Verwaltungsgebiet}|pw}, da ich in der Wohnungsfrage in der beka{\geminationn}ten, noch mancherlei oder vielmehr alles zu ordnen
               habe. Das urſprünglich geplante Häuschen im Gebirg ist mir weggeſchnappt worden. Es
               iſt{ }ſehr ärgerlich. Natürlich bleibt es trotzdem bei unſerm Salzburg\oindex{Salzburg@\textbf{Salzburg}, \emph{Verwaltungsgebiet}|pw}, und ich freu mich{ }ſehr darauf. Sagen Sie mir nur
               gleich das genaue Datum, da ich mit den Tagen haushalten muſs.\pend
           
\pstart
           Morgen{ }ſchicke ich Ihnen den 2. Band Mozart\pwindex{Mozart, Wolfgang Amadeus 27.\,1.\,1756 Salzburg – 5.\,12.\,1791 Wien@\textsc{Mozart, Wolfgang Amadeus} (27.\,1.\,1756 Salzburg – 5.\,12.\,1791 Wien), \emph{Komponist}|pw}\pwindex{\textcolor{red}{\textsuperscript{XXXX indx1}}!W. A. Mozart@\strich\emph{W. A. Mozart}|pwv}. – Richard\pwindex{Beer-Hofmann, Richard 11.\,7.\,1866 Wien – 26.\,9.\,1945 New York City@\textsc{Beer-Hofmann, Richard} (11.\,7.\,1866 Wien – 26.\,9.\,1945 New York City), \emph{Schriftsteller}|pw} arbeitet wirklich; er{ }ſcheint
               im dritten Capitel\pwindex{Beer-Hofmann, Richard 11.\,7.\,1866 Wien – 26.\,9.\,1945 New York City@\textsc{Beer-Hofmann, Richard} (11.\,7.\,1866 Wien – 26.\,9.\,1945 New York City), \emph{Schriftsteller}!Tod Georgs@\strich\emph{Der Tod Georgs}|pwv} zu{ }ſein.
                  {\pb}Wenigſtens hat er kaum zu was anderm Zeit und ist eine
               Radelraunzen wie ein kleines Kind.\pend
           
\pstart
           Neulich bin ich nach Unterach\oindex{Unterach am Attersee@\textbf{Unterach am Attersee}|pw} zu Stri\pwindex{Strisower, Bernhard 31.\,10.\,1847 Brody [Ukraine] – 21.\,12.\,1900 Wien@\textsc{Strisower, Bernhard} (31.\,10.\,1847 Brody [Ukraine] – 21.\,12.\,1900 Wien), \emph{Bankier}|pw}\pwindex{Strisower, Friederike 31.\,5.\,1854 Wien – 13.\,12.\,1922@\textsc{Strisower, Friederike} (31.\,5.\,1854 Wien – 13.\,12.\,1922)|pw}’s geradelt;{ }ſonſt mach ich nur
               ganz kleine Spazierfahrten, und plaudre mit einer merkwürdig geſcheiten Frau\pwindex{Freudenthal, Rosa 1862 – 18.\,6.\,1905 Berlin@\textsc{Freudenthal, Rosa} (1862 – 18.\,6.\,1905 Berlin)|pwv}{ }ſehr viel, die Humor hat, und ich verſuche mich zu
               erinnern, ob ich{ }ſchon je eine Frau mit Humor gekannt habe. –\pend
           
\pstart
           Schreiben Sie mir bald.\pend
           
\pstart
           Ich leſe noch immer \textsc{Tolstoi}\pwindex{Tolstoi, Lew Nikolajewitsch 9.\,9.\,1828 Yasnaya Polyana – 20.\,11.\,1910 Lev Tolstoy@\textsc{Tolstoi, Lew Nikolajewitsch} (9.\,9.\,1828 Yasnaya Polyana – 20.\,11.\,1910 Lev Tolstoy), \emph{Schriftsteller}|pw}\pwindex{Tolstoi, Lew Nikolajewitsch 9.\,9.\,1828 Yasnaya Polyana – 20.\,11.\,1910 Lev Tolstoy@\textsc{Tolstoi, Lew Nikolajewitsch} (9.\,9.\,1828 Yasnaya Polyana – 20.\,11.\,1910 Lev Tolstoy), \emph{Schriftsteller}!Krieg und Frieden@\strich\emph{Krieg und Frieden}|pwv} u \textsc{Brandes}\pwindex{Brandes, Georg 4.\,2.\,1842 Kopenhagen – 19.\,2.\,1927 ebd.@\textsc{Brandes, Georg} (4.\,2.\,1842 Kopenhagen – 19.\,2.\,1927 ebd.)|pw}\pwindex{Brandes, Georg 4.\,2.\,1842 Kopenhagen – 19.\,2.\,1927 ebd.@\textsc{Brandes, Georg} (4.\,2.\,1842 Kopenhagen – 19.\,2.\,1927 ebd.)!William Shakespeare@\strich\emph{William Shakespeare}|pwv}.\pend
           
\pstart
           Herzlich der Ihre{\\[\baselineskip]}\spacefill\mbox{Arthur.}\pend
           \leftskip=0em{}\selectlanguage{ngerman}\endnumbering\briefempfaengerindex{Hofmannsthal, Hugo von@\textsc{Hofmannsthal, Hugo von}!zzzSchnitzler, Arthur@\emph{von Arthur Schnitzler}!1897-07-081@{8. 7. 1897}|)be}\mylabel{L00694h}  \newcommand{\dateiname}{L00694}\newcommand{\titel}{Arthur Schnitzler an Hugo von Hofmannsthal, 8. 7. 1897}\newcommand{\editorInnen}{Martin Anton Müller und Gerd-Hermann Susen}%% latex-leseansicht-abspann.tex
%% Abspann für die Leseansicht.
%% Der Schalter \ifkorrekturansicht ist bereits durch den Vorspann gesetzt.

%% latex-abspann.tex
%% Gemeinsamer Abspann für Korrekturansicht und Leseansicht.
%% Setzt den Schalter \ifkorrekturansicht voraus (gesetzt in den
%% einbindenden Dateien latex-korrekturansicht-abspann.tex bzw.
%% latex-leseansicht-abspann.tex).
%% ---------------------------------------------------------------

\normalsize

% Das esempio-Environment wird nur in der Leseansicht benötigt
\ifkorrekturansicht\else
\newenvironment{esempio}[3]%
{
    \vspace{1.5ex}
    \rlap{\underline{#1}}
    \par
    \setlength{\parindent}{0cm}
    \nopagebreak
    \leftskip=#2cm
    \rightskip=#3cm
}
{
    \par
}
\fi

\doendnotes{C}
\bigskip
\vfill

\clearpage

\footnotesize

\ifkorrekturansicht
  \lohead{\textsc{register}}
\fi

% theindex-Environment neu definieren ohne reledmac
\makeatletter
\renewenvironment{theindex}{%
  \ifkorrekturansicht
    \section*{\indexname}%
  \else
    \subsubsection*{Index der erwähnten Entitäten}%
  \fi
  \setlength{\parindent}{0pt}%
  \setlength{\parskip}{0pt plus 0.3pt}%
  \let\item\@idxitem
}{%
  \ifkorrekturansicht\clearpage\fi
}
\makeatother

\IfFileExists{\jobname-pw.ind}{\input{\jobname-pw.ind}}{}

% Quellenangabe nur in der Leseansicht
\ifkorrekturansicht\else
% Fallback-Definitionen, falls die .tex-Datei \titel etc. nicht gesetzt hat
\providecommand{\titel}{}
\providecommand{\editorInnen}{}
\providecommand{\dateiname}{\jobname}

\vspace{3cm}

\vfill

\footnotesize
\textsc{Quelle}: \titel. Herausgegeben von {\editorInnen}. In: \emph{Arthur Schnitzler: Briefwechsel mit Autorinnen und Autoren}.
 Digitale Edition, https://schnitzler-briefe.acdh.oeaw.ac.at/{\dateiname}.html (Stand \today)
\fi

\end{document}


