%% latex-leseansicht-vorspann.tex
%% Vorspann für die Leseansicht.
%% Lädt die gemeinsame Datei latex-vorspann.tex mit nicht gesetztem Schalter.

\newif\ifkorrekturansicht
\korrekturansichtfalse

\input{../tex-inputs/latex-vorspann}


         
         \renewcommand{\erwaehntePersonen}{Personen: Felix Salten}
         \renewcommand{\erwaehnteInstitutionen}{Institutionen: Stadttheater Teplitz}
         \renewcommand{\erwaehnteOrte}{Orte: Dresden, Teplice, Wien}
         \renewcommand{\erwaehnteWerke}{}
               \section[ Felix Salten an Arthur Schnitzler, 6. 5. 1899]{ Felix Salten an Arthur Schnitzler, 6. 5. 1899}\nopagebreak\mylabel{v}\rehead{ }\begin{ledgroupsized}[t]{13cm}\normalsize\beginnumbering\briefempfaengerindex{Schnitzler, Arthur@\textsc{Schnitzler, Arthur}!zzzSalten, Felix@\emph{von Felix Salten}!1899-05-061@{6. 5. 1899}|(be} \toendnotes[C]{\smallbreak\pagebreak[2]} \Standort{CUL, Schnitzler, B 89, A 2.}
\physDesc{Brief, 1 Blatt, 3 Seiten, 949 Zeichen
\newline{}Handschrift: schwarze Tinte, lateinische Kurrent
\newline{}Ordnung: mit Bleistift von unbekannter Hand nummeriert: »115« }\toendnotes[C]{\smallbreak}\pstart
           \raggedleft{}{\pb}Teplitz\oindex{Teplice@\textbf{Teplice}|pw}, 6. Mai 99.\pend
           \pstart
           Lieber Freund, ich nehme an, dass die Telegramme von heute, wie die \label{K_L03291-1v}\edtext{Sendung an den Magistrat}{\lemma{\textnormal{\emph{Sendung an den Magistrat}}}\Cendnote{\textnormal{Am 11. 4. 1899 verlautbarte
                  die Stadtregierung von Teplitz\oindex{Teplice@\textbf{Teplice}|pwk}, dass das \emph{Stadttheater}\orgindex{Stadttheater Teplitz@Stadttheater Teplitz|pwk} ab dem 1. 10. 1899 auf vier Jahre zur Pacht frei würde. Salten\pwindex{Salten, Felix 06.09.1869 – 08.10.1945@\textsc{Salten, Felix} (06.09.1869 – 08.10.1945), \emph{Schriftsteller, Journalist, Chefredakteur}|pwk} hatte sich bereits zwei Jahre zuvor darum bemüht
                     (vgl. Felix Salten an Arthur Schnitzler, [10. 1. 1897]) und erneuerte
                  sein Interesse. Die für die Bewerbung benötigten 1000 Gulden könnte er von Schnitzler\pwindex{Schnitzler, Arthur 15.05.1862 – 21.10.1931@\textsc{Schnitzler, Arthur} (15.05.1862 – 21.10.1931), \emph{Schriftsteller, Mediziner}|pwk} ausgeliehen oder vermittelt
                  bekommen haben, doch belegbar ist das nicht. Woran Saltens\pwindex{Salten, Felix 06.09.1869 – 08.10.1945@\textsc{Salten, Felix} (06.09.1869 – 08.10.1945), \emph{Schriftsteller, Journalist, Chefredakteur}|pwk} Bewerbung
                  scheiterte, ist nicht bekannt. Seine fehlende Erfahrung als Theaterleiter dürfte
                  jedenfalls nicht geholfen haben.}}}\label{K_L03291-1h} von Ihnen herrühren, und danke Ihnen sehr
               herzlich dafür. Ich wußte wirklich nicht, dass der Termin so kurz gestellt ist, sonst
               hätte ich mir die Sache vorher geordnet. Überhaupt habe ich mich erst vor ein paar
               Tagen zu Teplitz\oindex{Teplice@\textbf{Teplice}|pw} entschloßen, und schrieb Ihnen
               deshalb vor meiner Abreise kurz »Dresden\oindex{Dresden@\textbf{Dresden}|pw}«, wie
               ich es allen gesagt hatte. Ich hatte {\pb}weder Zeit noch Ruhe, Ihnen
               diese neue Teplitz\oindex{Teplice@\textbf{Teplice}|pw}er Affaire brieflich zu
               erklären. Entschuldigen Sie, bitte, dass ich Sie so plötzlich und so dringend in
               Anspruch nahm. Ich brauche Ihnen wol nicht erst zu sagen, dass die Tausend Gulden
               ganz sicher sind, und dass Sie sie in der kürzesten Zeit (1 Monat längstens) wieder
               erhalten.\pend
           \pstart
           Dienstag früh bin ich wieder in Wien\oindex{Wien@\textbf{Wien}|pw}. Wenn ich zu Hause eine Zeile von Ihnen fände, wo ich Sie
                  Abends{ }{\pb}treffen kann, wär es mir sehr
               lieb.\pend
           \pstart
           Nochmals wärmsten Dank. {\\[\baselineskip]}Herzlichst Ihr {\\[\baselineskip]}\spacefill\mbox{Salten}\pend
           \leftskip=0em{}
         
         \endnumbering\mylabel{h}\end{ledgroupsized}  \newcommand{\dateiname}{L03291}\newcommand{\titel}{Felix Salten an Arthur Schnitzler, 6. 5. 1899}\newcommand{\editorInnen}{Martin Anton Müller und Laura Untner}%% latex-leseansicht-abspann.tex
%% Abspann für die Leseansicht.
%% Der Schalter \ifkorrekturansicht ist bereits durch den Vorspann gesetzt.

%% latex-abspann.tex
%% Gemeinsamer Abspann für Korrekturansicht und Leseansicht.
%% Setzt den Schalter \ifkorrekturansicht voraus (gesetzt in den
%% einbindenden Dateien latex-korrekturansicht-abspann.tex bzw.
%% latex-leseansicht-abspann.tex).
%% ---------------------------------------------------------------

\normalsize

% Das esempio-Environment wird nur in der Leseansicht benötigt
\ifkorrekturansicht\else
\newenvironment{esempio}[3]%
{
    \vspace{1.5ex}
    \rlap{\underline{#1}}
    \par
    \setlength{\parindent}{0cm}
    \nopagebreak
    \leftskip=#2cm
    \rightskip=#3cm
}
{
    \par
}
\fi

\doendnotes{C}
\bigskip
\vfill

\clearpage

\footnotesize

\ifkorrekturansicht
  \lohead{\textsc{register}}
\fi

% theindex-Environment neu definieren ohne reledmac
\makeatletter
\renewenvironment{theindex}{%
  \ifkorrekturansicht
    \section*{\indexname}%
  \else
    \subsubsection*{Index der erwähnten Entitäten}%
  \fi
  \setlength{\parindent}{0pt}%
  \setlength{\parskip}{0pt plus 0.3pt}%
  \let\item\@idxitem
}{%
  \ifkorrekturansicht\clearpage\fi
}
\makeatother

\IfFileExists{\jobname-pw.ind}{\input{\jobname-pw.ind}}{}

% Quellenangabe nur in der Leseansicht
\ifkorrekturansicht\else
% Fallback-Definitionen, falls die .tex-Datei \titel etc. nicht gesetzt hat
\providecommand{\titel}{}
\providecommand{\editorInnen}{}
\providecommand{\dateiname}{\jobname}

\vspace{3cm}

\vfill

\footnotesize
\textsc{Quelle}: \titel. Herausgegeben von {\editorInnen}. In: \emph{Arthur Schnitzler: Briefwechsel mit Autorinnen und Autoren}.
 Digitale Edition, https://schnitzler-briefe.acdh.oeaw.ac.at/{\dateiname}.html (Stand \today)
\fi

\end{document}


      