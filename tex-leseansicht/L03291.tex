%% latex-leseansicht-vorspann.tex
%% Vorspann für die Leseansicht.
%% Lädt die gemeinsame Datei latex-vorspann.tex mit nicht gesetztem Schalter.

\newif\ifkorrekturansicht
\korrekturansichtfalse

\input{../tex-inputs/latex-vorspann}

\begin{center}
            \textcolor{red}{ENTWURF, NICHT FERTIG KORRIGIERT}
                      \end{center}
            
         \renewcommand{\erwaehnteInstitutionen}{Institutionen: Stadttheater (Teplitz)}
         \renewcommand{\erwaehnteOrte}{Orte: Dresden, Teplice, Wien}
         \renewcommand{\erwaehnteWerke}{}
               \section[Felix Salten an Arthur Schnitzler, 6. 9. 1899]{ Felix Salten an Arthur Schnitzler, 6. 9. 1899}\nopagebreak\mylabel{v}\rehead{ }\begin{ledgroupsized}[t]{13cm}\normalsize\beginnumbering \toendnotes[C]{\smallbreak\pagebreak[2]} \Standort{CUL, Schnitzler, B 89, A 2.}
\physDesc{Brief, 1 Blatt, 3 Seiten
\newline{}Handschrift: schwarze Tinte, lateinische Kurrent\newline{}Ordnung: mit Bleistift von unbekannter Hand nummeriert:
                                    »115« }\toendnotes[C]{\smallbreak}\pstart
           \raggedleft{}{\pb}Teplitz\oindex{Teplice@\textbf{Teplice}|pw}, 6. Mai 99. \pend
           \pstart
           Lieber Freund, ich nehme an, dass die Telegramme von heute, wie die
               \label{K_L03291-11v}\edtext{Sendung an den Magistrat}{\lemma{\textnormal{\emph{Sendung an den Magistrat}}}\Cendnote{\textnormal{unklar}}}\label{K_L03291-11h} von Ihnen
               herrühren und danke Ihnen sehr herzlich dafür. Ich wußte wirklich nicht, dass der
               Termin so kurz gestellt ist, sonst hätte ich mir die Sachen vorher geordnet.
               Überhaupt habe ich mich erst vor ein paar Tagen zu Teplitz\oindex{Teplice@\textbf{Teplice}|pw} entschloßen, und schrieb Ihnen deshalb vor meiner Abreise kurz »Dresden\oindex{Dresden@\textbf{Dresden}|pw}«, wie ich es allen gesagt hatte. Ich hatte
                  {\pb}weder Zeit noch Ruhe, Ihnen
               diese \label{K_L03291-1v}\edtext{neue Teplitz\oindex{Teplice@\textbf{Teplice}|pw}er Affaire}{\lemma{\textnormal{\emph{neue Teplitzer Affaire}}}\Cendnote{\textnormal{Die frühere Affäre war der erste Versuch, das \emph{Stadttheater}\orgindex{Stadttheater (Teplitz)@Stadttheater (Teplitz)|pwk} zu übernehmen, vgl. Felix Salten an Arthur Schnitzler, 16. 1. 1897. Am 11. 4. 1899
                     hatte die Stadtregierung von Teplitz\oindex{Teplice@\textbf{Teplice}|pwk} verlautbart, dass das \emph{Stadttheater}\orgindex{Stadttheater (Teplitz)@Stadttheater (Teplitz)|pwk} ab 1. 10. 1899
                     auf vier Jahre zur Pacht vergeben würde. Woran Salten\pwindex{Salten, Felix 06.09.1869 – 08.10.1945@\textsc{Salten, Felix} (06.09.1869 – 08.10.1945), \emph{Schriftsteller, Journalist}|pwk}s 
                     Bewerbung scheiterte, ist nicht bekannt. Möglicherweise gelang es ihm nicht, die
                  notwendigen Summen aufzustellen. Seine fehlende Erfahrung als Theaterleiter dürfte ebenfalls
                  nicht geholfen haben.}}}\label{K_L03291-1h} brieflich zu erklären. Entschuldigen Sie,
               bitte, dass ich Sie so plötzlich und so dringend in Anspruch nahm. Ich brauche Ihnen
               wol nicht erst zu sagen, dass die Tausend Gulden ganz sicher sind, und dass Sie sie
               in der kürzesten Zeit (1 Monat längstens) wieder erhalten. \pend
           \pstart
           Dienstag früh bin ich wieder in Wien\oindex{Wien@\textbf{Wien}|pw}. Wenn ich zu Hause eine Zeile von Ihnen fände, wo ich Sie Abends {\pb}treffen kann, wär es mir sehr lieb.\pend
           \pstart
           Nochmals wärmsten Dank.{\\[\baselineskip]}Herzlichst Ihr{\\[\baselineskip]}\spacefill\mbox{Salten}\pend
           \leftskip=0em{}
         
         \endnumbering\mylabel{h}\end{ledgroupsized}\begin{anhang}\end{anhang}\newcommand{\dateiname}{L03291}\newcommand{\titel}{Felix Salten an Arthur Schnitzler, 6. 9. 1899}\newcommand{\editorInnen}{Martin Anton Müller und Laura Untner}%% latex-leseansicht-abspann.tex
%% Abspann für die Leseansicht.
%% Der Schalter \ifkorrekturansicht ist bereits durch den Vorspann gesetzt.

%% latex-abspann.tex
%% Gemeinsamer Abspann für Korrekturansicht und Leseansicht.
%% Setzt den Schalter \ifkorrekturansicht voraus (gesetzt in den
%% einbindenden Dateien latex-korrekturansicht-abspann.tex bzw.
%% latex-leseansicht-abspann.tex).
%% ---------------------------------------------------------------

\normalsize

% Das esempio-Environment wird nur in der Leseansicht benötigt
\ifkorrekturansicht\else
\newenvironment{esempio}[3]%
{
    \vspace{1.5ex}
    \rlap{\underline{#1}}
    \par
    \setlength{\parindent}{0cm}
    \nopagebreak
    \leftskip=#2cm
    \rightskip=#3cm
}
{
    \par
}
\fi

\doendnotes{C}
\bigskip
\vfill

\clearpage

\footnotesize

\ifkorrekturansicht
  \lohead{\textsc{register}}
\fi

% theindex-Environment neu definieren ohne reledmac
\makeatletter
\renewenvironment{theindex}{%
  \ifkorrekturansicht
    \section*{\indexname}%
  \else
    \subsubsection*{Index der erwähnten Entitäten}%
  \fi
  \setlength{\parindent}{0pt}%
  \setlength{\parskip}{0pt plus 0.3pt}%
  \let\item\@idxitem
}{%
  \ifkorrekturansicht\clearpage\fi
}
\makeatother

\IfFileExists{\jobname-pw.ind}{\input{\jobname-pw.ind}}{}

% Quellenangabe nur in der Leseansicht
\ifkorrekturansicht\else
% Fallback-Definitionen, falls die .tex-Datei \titel etc. nicht gesetzt hat
\providecommand{\titel}{}
\providecommand{\editorInnen}{}
\providecommand{\dateiname}{\jobname}

\vspace{3cm}

\vfill

\footnotesize
\textsc{Quelle}: \titel. Herausgegeben von {\editorInnen}. In: \emph{Arthur Schnitzler: Briefwechsel mit Autorinnen und Autoren}.
 Digitale Edition, https://schnitzler-briefe.acdh.oeaw.ac.at/{\dateiname}.html (Stand \today)
\fi

\end{document}


      