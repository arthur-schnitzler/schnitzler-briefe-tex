%% latex-leseansicht-vorspann.tex
%% Vorspann für die Leseansicht.
%% Lädt die gemeinsame Datei latex-vorspann.tex mit nicht gesetztem Schalter.

\newif\ifkorrekturansicht
\korrekturansichtfalse

\input{../tex-inputs/latex-vorspann}


\section[Arthur Schnitzler an Richard Beer-Hofmann, 20. 7. 1899]{L00948 Arthur Schnitzler an Richard Beer-Hofmann, 20. 7. 1899}
\nopagebreak\mylabel{L00948v}
\rehead{ }\normalsize\beginnumbering\briefempfaengerindex{Beer-Hofmann, Richard@\textsc{Beer-Hofmann, Richard}!zzzSchnitzler, Arthur@\emph{von Arthur Schnitzler}!1899-07-201@{20. 7. 1899}|(be}
\toendnotes[C]{\smallbreak\pagebreak[2]}
\correspDesc{Versand  durch Arthur Schnitzler am 20. 7. 1899 in Velden am Wörthersee
\newline{}Erhalt  durch Richard Beer-Hofmann am 21. 7. 1899 in Seeboden}\toendnotes[C]{\smallbreak}
\Standort{YCGL, MSS 31.}
\physDesc{Briefkarte, , Kuvert, 720 Zeichen
\newline{}Handschrift: Bleistift, deutsche Kurrent
\newline{}Versand: 1) Stempel: »\nobreak{}\oindex{Velden am Wörthersee@\textbf{Velden am Wörthersee}|pwk}Velden \textcolor{gray}{am}
                                       Wörthersee, 20 {[}7.{]}\textcolor{gray}{9}9, 9N\nobreak{}«.   2) Stempel: »\nobreak{}\oindex{Seeboden am Millstättersee@\textbf{Seeboden am Millstättersee}|pwk}{\pb}Seeboden, 21. 7. {[}189{]}9\nobreak{}«. 
\newline{}Beer-Hofmann: eventuell vom Empfänger mit Bleistift am Umschlag datiert: »20. 7.« }
\buchAbdrucke{\weitereDrucke{Arthur Schnitzler, Richard Beer-Hofmann: \emph{Briefwechsel 1891–1931}. Herausgegeben von Konstanze Fliedl. Wien, Zürich: \emph{Europaverlag} 1992, S. 133.} }\toendnotes[C]{\smallbreak}\pstart{}{\pb}\textsc{Dr. Rich. Beer-Hofmann}\pend{}\pstart{}\textsc{Seeboden\oindex{Seeboden am Millstättersee@\textbf{Seeboden am Millstättersee}|pw}}\pend{}\pstart{}\textsc{Villa Platzer\oindex{Villa Platzer@\textbf{Villa Platzer}, \emph{Gebäude}|pw}}\pend{}\pstart{} am Millstätterſee\oindex{Millstätter See@\textbf{Millstätter See}, \emph{See}|pw}\pend{}{\bigskip}\vspace{1em}
\pstart
           \raggedleft{}{\pb}20. 7. 99\pend
           \vspace{0.5em}
\pstart
           lieber Richard, telegr. Sie mir jedenfalls einen Tag früher, bevor
               Sie ko{\geminationm}en. Bleiben Sie da{\geminationn}
               über Nacht hier? – Event. aviſiren Sie auch Robert
                  Hirſchfeld\pwindex{Hirschfeld, Robert 17.\,9.\,1857 Žďár nad Sázavou – 2.\,4.\,1914 Salzburg@\textsc{Hirschfeld, Robert} (17.\,9.\,1857 Žďár nad Sázavou – 2.\,4.\,1914 Salzburg), \emph{Journalist, Musikkritiker}|pw} (\textsc{Krumpendorf}\oindex{Krumpendorf am Wörthersee@\textbf{Krumpendorf am Wörthersee}, \emph{Verwaltungsgebiet}|pw}) wann Sie hier{ }ſind? – An die Tauern\oindex{Hohe Tauern@\textbf{Hohe Tauern}, \emph{Gebirge}|pw}
               glaub ich nicht,{ }ſind mir auch nicht{ }ſehr{ }ſympathiſch. Meinen Sie den Übergang vom
                  Millſtätterſee\oindex{Millstätter See@\textbf{Millstätter See}, \emph{See}|pw}{ }\textsc{resp.}{ }Spital\oindex{Spittal an der Drau@\textbf{Spittal an der Drau}, \emph{Hauptstadt}|pw}
                aus? – Ich habe andre Vorſchläge zu unterbreiten. We{\geminationn} ich nur ahnte, {\pb}ob wir
               1 oder 2 oder 14 Tage zuſa{\geminationm}en bleiben? –\pend
           
\pstart
           Waſſerm.\pwindex{Wassermann, Jakob 10.\,3.\,1873 Fürth – 1.\,1.\,1934 Altaussee@\textsc{Wassermann, Jakob} (10.\,3.\,1873 Fürth – 1.\,1.\,1934 Altaussee), \emph{Schriftsteller}|pw} ko{\geminationm}t
               erſt heut Abend an. –\pend
           
\pstart
           – Geſtern hab ich eine Radtour gemacht, Faakerſee\oindex{Faakersee@\textbf{Faakersee}, \emph{See}|pw}, mit Ihrer\pwindex{Reinhard, Marie 13.\,3.\,1871 Wien – 18.\,3.\,1899 ebd.@\textsc{Reinhard, Marie} (13.\,3.\,1871 Wien – 18.\,3.\,1899 ebd.), \emph{Gesangspädagogin}|pwv}{ }Schweſter\pwindex{Burger, Caroline 11.\,7.\,1869 Wien – 15.\,3.\,1959 ebd.@\textsc{Burger, Caroline} (11.\,7.\,1869 Wien – 15.\,3.\,1959 ebd.)|pwv} und Ihrem Schwager\pwindex{Burger, Rudolf *~6.\,12.\,1866 Wien@\textsc{Burger, Rudolf} (*~6.\,12.\,1866 Wien), \emph{Versicherungsdirektor}|pwv} – es war beinah ganz
               wie im \label{K_L00948-1v}\edtext{vorigen Jahre}{\lemma{\textnormal{\emph{vorigen Jahre}}}\Cendnote{\textnormal{Im Jahr zuvor war er mit Marie Reinhard\pwindex{Reinhard, Marie 13.\,3.\,1871 Wien – 18.\,3.\,1899 ebd.@\textsc{Reinhard, Marie} (13.\,3.\,1871 Wien – 18.\,3.\,1899 ebd.), \emph{Gesangspädagogin}|pwk} und ihrer Schwester Lola Burger\pwindex{Burger, Caroline 11.\,7.\,1869 Wien – 15.\,3.\,1959 ebd.@\textsc{Burger, Caroline} (11.\,7.\,1869 Wien – 15.\,3.\,1959 ebd.)|pwk} im Sommerurlaub. Siehe A. S.: \emph{Tagebuch}, 29. 7. 1898.}}}\label{K_L00948-1} – und –\pend
           
\pstart
           – Es iſt vergeblich ein \label{K_L00948-2v}\edtext{Wort zu{ }ſuchen}{\lemma{\textnormal{\emph{Wort zu suchen}}}\Cendnote{\textnormal{Er trauerte um Marie Reinhard\pwindex{Reinhard, Marie 13.\,3.\,1871 Wien – 18.\,3.\,1899 ebd.@\textsc{Reinhard, Marie} (13.\,3.\,1871 Wien – 18.\,3.\,1899 ebd.), \emph{Gesangspädagogin}|pwk}, die am
                     18. 3. 1899 verstorben war.}}}\label{K_L00948-2}.\pend
           
\pstart
           Leben Sie wohl.{\\[\baselineskip]}Ihr \spacefill\mbox{Arthur.}\pend
           \leftskip=0em{}\selectlanguage{ngerman}\endnumbering\briefempfaengerindex{Beer-Hofmann, Richard@\textsc{Beer-Hofmann, Richard}!zzzSchnitzler, Arthur@\emph{von Arthur Schnitzler}!1899-07-201@{20. 7. 1899}|)be}\mylabel{L00948h}  \newcommand{\dateiname}{L00948}\newcommand{\titel}{Arthur Schnitzler an Richard Beer-Hofmann, 20. 7. 1899}\newcommand{\editorInnen}{Martin Anton Müller und Gerd-Hermann Susen}%% latex-leseansicht-abspann.tex
%% Abspann für die Leseansicht.
%% Der Schalter \ifkorrekturansicht ist bereits durch den Vorspann gesetzt.

%% latex-abspann.tex
%% Gemeinsamer Abspann für Korrekturansicht und Leseansicht.
%% Setzt den Schalter \ifkorrekturansicht voraus (gesetzt in den
%% einbindenden Dateien latex-korrekturansicht-abspann.tex bzw.
%% latex-leseansicht-abspann.tex).
%% ---------------------------------------------------------------

\normalsize

% Das esempio-Environment wird nur in der Leseansicht benötigt
\ifkorrekturansicht\else
\newenvironment{esempio}[3]%
{
    \vspace{1.5ex}
    \rlap{\underline{#1}}
    \par
    \setlength{\parindent}{0cm}
    \nopagebreak
    \leftskip=#2cm
    \rightskip=#3cm
}
{
    \par
}
\fi

\doendnotes{C}
\bigskip
\vfill

\clearpage

\footnotesize

\ifkorrekturansicht
  \lohead{\textsc{register}}
\fi

% theindex-Environment neu definieren ohne reledmac
\makeatletter
\renewenvironment{theindex}{%
  \ifkorrekturansicht
    \section*{\indexname}%
  \else
    \subsubsection*{Index der erwähnten Entitäten}%
  \fi
  \setlength{\parindent}{0pt}%
  \setlength{\parskip}{0pt plus 0.3pt}%
  \let\item\@idxitem
}{%
  \ifkorrekturansicht\clearpage\fi
}
\makeatother

\IfFileExists{\jobname-pw.ind}{\input{\jobname-pw.ind}}{}

% Quellenangabe nur in der Leseansicht
\ifkorrekturansicht\else
% Fallback-Definitionen, falls die .tex-Datei \titel etc. nicht gesetzt hat
\providecommand{\titel}{}
\providecommand{\editorInnen}{}
\providecommand{\dateiname}{\jobname}

\vspace{3cm}

\vfill

\footnotesize
\textsc{Quelle}: \titel. Herausgegeben von {\editorInnen}. In: \emph{Arthur Schnitzler: Briefwechsel mit Autorinnen und Autoren}.
 Digitale Edition, https://schnitzler-briefe.acdh.oeaw.ac.at/{\dateiname}.html (Stand \today)
\fi

\end{document}


