%% latex-korrekturansicht-vorspann.tex
%% Vorspann für die Korrekturansicht.
%% Lädt die gemeinsame Datei latex-vorspann.tex mit gesetztem Schalter.

\newif\ifkorrekturansicht
\korrekturansichttrue

\input{../tex-inputs/latex-vorspann}


\section[Arthur Schnitzler an Richard Beer-Hofmann, 20. 7. 1899]{L00948 Arthur Schnitzler an Richard Beer-Hofmann, 20. 7. 1899}
\nopagebreak\mylabel{L00948v}
\rehead{ }\normalsize\beginnumbering\briefempfaengerindex{Beer-Hofmann, Richard@\textsc{Beer-Hofmann, Richard}!zzzSchnitzler, Arthur@\emph{von Arthur Schnitzler}!1899-07-201@{20. 7. 1899}|(be}
\toendnotes[C]{\smallbreak\pagebreak[2]}\Standort{YCGL, MSS 31.}
\physDesc{Briefkarte, , Umschlag, 720 Zeichen
\newline{}Handschrift: Bleistift, deutsche Kurrent
\newline{}Versand: 1) Stempel: »\nobreak{}\oindex{Velden am Woerthersee@\textbf{Velden am Wörthersee}, \emph{P.PPL}|pwk}Velden \textcolor{gray}{am}
                                       Wörthersee, 20 {[}7.{]}\textcolor{gray}{9}9 , 9N\nobreak{}«.   2) Stempel: »\nobreak{}\oindex{Seeboden@\textbf{Seeboden}, \emph{A.ADM3}|pwk}{\pb}Seeboden, 21. 7. {[}189{]}9\nobreak{}«. 
\newline{}Beer-Hofmann: eventuell vom Empfänger mit Bleistift am Umschlag datiert: »20. 7.« }
\buchAbdrucke{\weitereDrucke{Arthur Schnitzler, Richard Beer-Hofmann: \emph{Briefwechsel 1891–1931}. Wien, Zürich: \emph{Europaverlag} 1992, S. 133.} }\toendnotes[C]{\smallbreak}\pstart{}{\pb}\textsc{Dr. Rich. Beer-Hofmann}\pend{}\pstart{}\textsc{Seeboden\oindex{Seeboden@\textbf{Seeboden}, \emph{A.ADM3}|pw}}\pend{}\pstart{}\textsc{Villa Platzer\oindex{Villa Platzer@\textbf{Villa Platzer}, \emph{Gebäude (K.GBD)}|pw}}\pend{}\pstart{} am Millstätterſee\oindex{Millstaetter See@\textbf{Millstätter See}, \emph{See (N.SEE)}|pw}\pend{}{\bigskip}\vspace{1em}
\pstart
           \raggedleft{}{\pb}20. 7. 99\pend
           \vspace{0.5em}
\pstart
           lieber Richard, telegr. Sie mir jedenfalls einen Tag früher, bevor
               Sie ko{\geminationm}en. Bleiben Sie da{\geminationn}
               über Nacht hier? – Event. aviſiren Sie auch Robert
                  Hirſchfeld\pwindex{Hirschfeld, Robert 17.09.1857 – 02.04.1914@\textsc{Hirschfeld, Robert} (17.09.1857 – 02.04.1914), \emph{Journalist/Journalistin, Musikkritiker/Musikkritikerin}|pw} (\textsc{Krumpendorf}\oindex{Krumpendorf am Woerthersee@\textbf{Krumpendorf am Wörthersee}, \emph{A.ADM3}|pw}) wann Sie hier ſind? – An die Tauern\oindex{Hohe Tauern@\textbf{Hohe Tauern}, \emph{Gebirge (N.GBR)}|pw}
               glaub ich nicht, ſind mir auch nicht ſehr ſympathiſch. Meinen Sie den Übergang vom
                  Millſtätterſee\oindex{Millstaetter See@\textbf{Millstätter See}, \emph{See (N.SEE)}|pw}{ }\textsc{resp.}{ }Spital\oindex{Spittal an der Drau@\textbf{Spittal an der Drau}, \emph{P.PPLA3}|pw}
                aus? – Ich habe andre Vorſchläge zu unterbreiten. We{\geminationn} ich nur ahnte, {\pb}ob wir
               1 oder 2 oder 14 Tage zuſa{\geminationm}en bleiben? – \pend
           
\pstart
           Waſſerm.\pwindex{Wassermann, Jakob 10.03.1873 – 01.01.1934@\textsc{Wassermann, Jakob} (10.03.1873 – 01.01.1934), \emph{Schriftsteller/Schriftstellerin}|pw} ko{\geminationm}t
               erſt heut Abend an. –\pend
           
\pstart
           – Geſtern hab ich eine Radtour gemacht, Faakerſee\oindex{Faakersee@\textbf{Faakersee}, \emph{H.LK}|pw}, mit Ihrer\pwindex{Reinhard, Marie 1871-03-13 – 1899-03-18@\textsc{Reinhard, Marie} (1871-03-13 – 1899-03-18), \emph{Gesangspädagoge/Gesangspädagogin}|pwv}{ }Schweſter\pwindex{Burger, Caroline 11.07.1869 – 15.03.1959@\textsc{Burger, Caroline} (11.07.1869 – 15.03.1959)|pwv} und Ihrem Schwager\pwindex{Burger, Rudolf *~06.12.1866@\textsc{Burger, Rudolf} (*~06.12.1866), \emph{Versicherungsdirektor/Versicherungsdirektorin}|pwv} – es war beinah ganz
               wie im \label{K_L00948-1v}\edtext{vorigen Jahre}{\lemma{\textnormal{\emph{vorigen Jahre}}}\Cendnote{\textnormal{Im Jahr zuvor war er mit Marie Reinhard\pwindex{Reinhard, Marie 1871-03-13 – 1899-03-18@\textsc{Reinhard, Marie} (1871-03-13 – 1899-03-18), \emph{Gesangspädagoge/Gesangspädagogin}|pwk} und ihrer Schwester Lola Burger\pwindex{Burger, Caroline 11.07.1869 – 15.03.1959@\textsc{Burger, Caroline} (11.07.1869 – 15.03.1959)|pwk} im Sommerurlaub. Siehe A. S.: \emph{Tagebuch}, 29. 7. 1898.}}}\label{K_L00948-1} – und –\pend
           
\pstart
           – Es iſt vergeblich ein \label{K_L00948-2v}\edtext{Wort zu
                  ſuchen}{\lemma{\textnormal{\emph{Wort zu
                  ſuchen}}}\Cendnote{\textnormal{Er trauerte um Marie Reinhard\pwindex{Reinhard, Marie 1871-03-13 – 1899-03-18@\textsc{Reinhard, Marie} (1871-03-13 – 1899-03-18), \emph{Gesangspädagoge/Gesangspädagogin}|pwk}, die am
                     18. 3. 1899 verstorben war.}}}\label{K_L00948-2}.\pend
           
\pstart
           Leben Sie wohl.{\\[\baselineskip]}Ihr \spacefill\mbox{Arthur.}\pend
           \leftskip=0em{}\selectlanguage{ngerman}\endnumbering\briefempfaengerindex{Beer-Hofmann, Richard@\textsc{Beer-Hofmann, Richard}!zzzSchnitzler, Arthur@\emph{von Arthur Schnitzler}!1899-07-201@{20. 7. 1899}|)be}\mylabel{L00948h}  \normalsize

\doendnotes{C}
\bigskip
\vfill

\clearpage

\footnotesize

\lohead{\textsc{register}}

% Definiere theindex-Environment komplett neu ohne reledmac
\makeatletter
\renewenvironment{theindex}{%
  \section*{\indexname}%
  \setlength{\parindent}{0pt}%
  \setlength{\parskip}{0pt plus 0.3pt}%
  \let\item\@idxitem
}{%
  \clearpage
}
\makeatother

\IfFileExists{\jobname-pw.ind}{\input{\jobname-pw.ind}}{}

\end{document}

      