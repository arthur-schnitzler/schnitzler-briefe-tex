%% latex-leseansicht-vorspann.tex
%% Vorspann für die Leseansicht.
%% Lädt die gemeinsame Datei latex-vorspann.tex mit nicht gesetztem Schalter.

\newif\ifkorrekturansicht
\korrekturansichtfalse

\input{../tex-inputs/latex-vorspann}


         
         \renewcommand{\erwaehntePersonen}{Personen: Robert Adam, Flavius Belisar, Giacomo Girolamo Casanova}
         \renewcommand{\erwaehnteOrte}{Orte: Spa, Wien, Österreich}
         \renewcommand{\erwaehnteWerke}{Werke: Casanovas Heimfahrt, Der Fremde, Die Schwestern oder Casanova in Spa. Lustspiel in Versen, Die neue Rundschau, Yppl. Idylle in fünf Akten}
               \section[Robert Adam an Arthur Schnitzler, 8. 12. 1918]{ Robert Adam an Arthur Schnitzler, 8. 12. 1918}\nopagebreak\mylabel{v}\rehead{ }\begin{ledgroupsized}[t]{13cm}\normalsize\beginnumbering\briefempfaengerindex{Schnitzler, Arthur@\textsc{Schnitzler, Arthur}!zzzAdam, Robert@\emph{von Robert Adam}!1918-12-081@{8. 12. 1918}|(be} \toendnotes[C]{\smallbreak\pagebreak[2]} \Standort{CUL, Schnitzler, B 1.}
\physDesc{Brief, 1 Blatt, 4 Seiten, 3381 Zeichen
\newline{}Handschrift: schwarze Tinte, deutsche Kurrent
\newline{}Schnitzler: 1) mit Bleistift beschriftet: »\textsc{Adam}«  2) mit rotem Buntstift zwei Unterstreichungen
\newline{}Ordnung: von unbekannter Hand nummeriert: »11« }\Standort{Wien, Österreichische Nationalbibliothek, Cod.ser. 52.263, 227.}
\physDesc{Brief, maschinenschriftliche Abschrift, 1 Blatt, 1 Seite, 3381 Zeichen
\newline{}Schreibmaschine}\toendnotes[C]{\smallbreak}\pstart
           \raggedleft{}{\pb}Wien\oindex{Wien@\textbf{Wien}|pw}, 8. Dezember 1918\pend
           \pstart\center{}Hochverehrter Herr Doktor!\pend\pstart
           Sie haben mir durch die Zuſendung von »\textsc{Casanovas Heimfahrt}\pwindex{Schnitzler, Arthur 15.05.1862 – 21.10.1931@\textsc{Schnitzler, Arthur} (15.05.1862 – 21.10.1931), \emph{Schriftsteller, Mediziner}!Casanovas Heimfahrt1.7.1918 – 1.9.1918@\strich\emph{Casanovas Heimfahrt} {[}1.7.1918 – 1.9.1918{]}|pw}« eine große Freude bereitet, und ich ſage Ihnen herzlichen Dank. Wie ſehr ich
               dieſe Novelle\pwindex{Schnitzler, Arthur 15.05.1862 – 21.10.1931@\textsc{Schnitzler, Arthur} (15.05.1862 – 21.10.1931), \emph{Schriftsteller, Mediziner}!Casanovas Heimfahrt1.7.1918 – 1.9.1918@\strich\emph{Casanovas Heimfahrt} {[}1.7.1918 – 1.9.1918{]}|pwv}, die ich zum
               erſtenmal während des Erſcheinens in der Neuen
                  Rundschau\pwindex{?? Werk@Nicht ermittelte Verfasserinnen und Verfasser!neue Rundschau1904@\emph{Die neue Rundschau} {[}1904{]}|pw} las, als die wundervoll-weiſe und ſüße Frucht einer
               Erzählermeiſterſchaft ſchätze, habe ich Ihnen bereits geſagt. Wenn ich mich geneigt
               fühle, ſie allen Ihren früheren epiſchen Arbeiten voranzuſtellen, mag mich vielleicht
               meine Vorliebe für den Helden, mit deſſen Memoiren ich mich längere Zeit beſchäftigt
               habe, beeinfluſſen; aber daß hier alle Geſtalten, nicht nur der Held, ein eigenes
               Leben lebten, ſodaß es iſt, als ſchüfe der Dichter nicht, wie eine \textsc{laterna magica}, ſondern als beleuchtete er bloß, wie ein
               ſcharfer Scheinwerfer ſchon Exiſtierendes; daß jede Geberde der handelnden Perſonen,
               alles {\pb}Lebende und Lebloſe, das ſie
               umgibt, mit gewaltiger Plaſtik, die doch nie aufhört, das einfachſte und
               ſelbſtverſtändlichſte Ding der Welt zu ſcheinen, hingeſtellt und umriſſen iſt; daß
               auf allen der 181 Seiten des Buchs\pwindex{Schnitzler, Arthur 15.05.1862 – 21.10.1931@\textsc{Schnitzler, Arthur} (15.05.1862 – 21.10.1931), \emph{Schriftsteller, Mediziner}!Casanovas Heimfahrt1.7.1918 – 1.9.1918@\strich\emph{Casanovas Heimfahrt} {[}1.7.1918 – 1.9.1918{]}|pwv} kein Wort zuviel und daher unnütz zu ſein ſcheint, was mir als
               Merkzeichen einer klaſſiſchen Arbeit gilt – das muß und wird jeder Kunſtverſtändige,
               wenn er auch meine Spezialliebe zum Helden nicht teilt, aus vollem Herzen bezeugen.
               Ich bin ſchon außerordentlich auf Ihren jungen \textsc{Casanova}\pwindex{Casanova, Giacomo Girolamo 02.04.1725 – 04.06.1798@\textsc{Casanova, Giacomo Girolamo} (02.04.1725 – 04.06.1798), \emph{Schriftsteller, Abenteurer}|pw} in Spaa\oindex{Spa@\textbf{Spa}|pw}\pwindex{Schnitzler, Arthur 15.05.1862 – 21.10.1931@\textsc{Schnitzler, Arthur} (15.05.1862 – 21.10.1931), \emph{Schriftsteller, Mediziner}!Schwestern oder Casanova in Spa. Lustspiel in Versen01. 10. 1919@\strich\emph{Die Schwestern oder Casanova in Spa. Lustspiel in Versen} {[}01. 10. 1919{]}|pwv} begierig, den wir wohl ſchon längſt kennen gelernt hätten, wenn die politiſche
               Umwälzung nicht gekommen wäre. Bis er erſcheint, will ich mir noch einmal, und nun
               mit Muße und unabhängig von Fortſetzungen, den gealterten Sünder vornehmen und an
               Ihrem Werke lernen, wie man klar und farbig und ſpannend und einfach und doch
               geiſtreich erzählen kann: daß ich dies nicht kann und niemals können werde, iſt
               etwas, was mich manchmal niedergeſchlagen, immer aber vor dem, der es kann,
               ehrfürchtig und beſcheiden {\pb}macht. –\pend
           \pstart
           Die Bitte, die ich in meinem letzten Briefe an Sie ſtellte – Sie möchten ſich über
               das Geſchick meiner zwei Stücke\pwindex{Adam, Robert 20.04.1877 – 16.10.1961@\textsc{Adam, Robert} (20.04.1877 – 16.10.1961), \emph{Schriftsteller, Richter}!Yppl. Idylle in fuenf Akten@\strich\emph{Yppl. Idylle in fünf Akten}|pwv}\pwindex{Adam, Robert 20.04.1877 – 16.10.1961@\textsc{Adam, Robert} (20.04.1877 – 16.10.1961), \emph{Schriftsteller, Richter}!Fremde@\strich\emph{Der Fremde}|pwv} gelegentlich erkundigen – iſt durch die traurigen Ereigniſſe der
               letzten Woche gegenſtandslos geworden; Sie werden einſehen, daß mich wirklich das
               Pech verfolgt – ich glaube ſogar, daß das Theater, das wirklich einmal eines meiner
               Stücke zur Aufführung bringen wollte, zumindeſt am Tage der Erſtaufführung in Flammen
               aufgehen oder Konkurs anſagen würde. Wenn ich alſo Trübſal blaſe – das einzige
               Inſtrument, für das meine muſikaliſche Anlage zureicht –, ſo iſt dieſe Beſchäftigung
               nicht ſo ganz unberechtigt, zumal es, trotz mancher hübſchen neuen Geſetze, nicht
               viel Erquickliches ringsum gibt, das aufheitern oder tröſten könnte – die
               Verhältniſſe haben es mit ſich gebracht, daß ich, der noch vor kurzem aus dem
               Staatsdienſt mich wegſehnte, um die mir noch etwa verbliebene Kraft frei verwerten zu
               können, nunmehr, beim Anblick ſo vieler \label{K_L02315-1v}\edtext{Beliſare\pwindex{Belisar, Flavius 505 – 0565@\textsc{Belisar, Flavius} (505 – 0565), \emph{Militär}|pwv}}{\lemma{\textnormal{\emph{Beliſare}}}\Cendnote{\textnormal{Hier wohl im Sinne der apokryphen
                  Überlieferung, Belisar\pwindex{Belisar, Flavius 505 – 0565@\textsc{Belisar, Flavius} (505 – 0565), \emph{Militär}|pwk} hätte, nach seiner
                  Zeit als Feldherr, die Augen ausgestochen bekommen und als Bettler auf der Straße
                  gelebt.}}}\label{K_L02315-1h}, froh ſein muß, ein feſtes Amt zu bekleiden, und nicht, wie ſo
               mancher meines Alters, auf Stel{\pb}lungsſuche
               gehen zu müſſen. Daß ich aber in der hungernden und frierenden Republik\oindex{Oesterreich@\textbf{Österreich}|pwv} gerade ſo wie im Kaiſerſtaat Tag
               für Tag über Preistreibereien zu Gericht ſitze, als wäre gar nichts geſchehen, als
               beſtünde noch der außerordentliche Kriegszuſtand, das kommt mir manchmal ſo
               grauenhaft vor wie das Weiterwachſen der Haare einer Leiche, die verfault und
               zerfällt. –\pend
           \pstart
           Nochmals beſten Dank! Und die herzlichsten Grüße von Ihrem{\\[\baselineskip]}ergebenen{\\[\baselineskip]}\spacefill\mbox{D\textsuperscript{r}RAdam}\pend
           \leftskip=0em{}
         
         \endnumbering\mylabel{h}\end{ledgroupsized}  \newcommand{\dateiname}{L02315}\newcommand{\titel}{Robert Adam an Arthur Schnitzler, 8. 12. 1918}\newcommand{\editorInnen}{Martin Anton Müller und Gerd-Hermann Susen}%% latex-leseansicht-abspann.tex
%% Abspann für die Leseansicht.
%% Der Schalter \ifkorrekturansicht ist bereits durch den Vorspann gesetzt.

%% latex-abspann.tex
%% Gemeinsamer Abspann für Korrekturansicht und Leseansicht.
%% Setzt den Schalter \ifkorrekturansicht voraus (gesetzt in den
%% einbindenden Dateien latex-korrekturansicht-abspann.tex bzw.
%% latex-leseansicht-abspann.tex).
%% ---------------------------------------------------------------

\normalsize

% Das esempio-Environment wird nur in der Leseansicht benötigt
\ifkorrekturansicht\else
\newenvironment{esempio}[3]%
{
    \vspace{1.5ex}
    \rlap{\underline{#1}}
    \par
    \setlength{\parindent}{0cm}
    \nopagebreak
    \leftskip=#2cm
    \rightskip=#3cm
}
{
    \par
}
\fi

\doendnotes{C}
\bigskip
\vfill

\clearpage

\footnotesize

\ifkorrekturansicht
  \lohead{\textsc{register}}
\fi

% theindex-Environment neu definieren ohne reledmac
\makeatletter
\renewenvironment{theindex}{%
  \ifkorrekturansicht
    \section*{\indexname}%
  \else
    \subsubsection*{Index der erwähnten Entitäten}%
  \fi
  \setlength{\parindent}{0pt}%
  \setlength{\parskip}{0pt plus 0.3pt}%
  \let\item\@idxitem
}{%
  \ifkorrekturansicht\clearpage\fi
}
\makeatother

\IfFileExists{\jobname-pw.ind}{\input{\jobname-pw.ind}}{}

% Quellenangabe nur in der Leseansicht
\ifkorrekturansicht\else
% Fallback-Definitionen, falls die .tex-Datei \titel etc. nicht gesetzt hat
\providecommand{\titel}{}
\providecommand{\editorInnen}{}
\providecommand{\dateiname}{\jobname}

\vspace{3cm}

\vfill

\footnotesize
\textsc{Quelle}: \titel. Herausgegeben von {\editorInnen}. In: \emph{Arthur Schnitzler: Briefwechsel mit Autorinnen und Autoren}.
 Digitale Edition, https://schnitzler-briefe.acdh.oeaw.ac.at/{\dateiname}.html (Stand \today)
\fi

\end{document}


      