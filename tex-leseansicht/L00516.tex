%% latex-leseansicht-vorspann.tex
%% Vorspann für die Leseansicht.
%% Lädt die gemeinsame Datei latex-vorspann.tex mit nicht gesetztem Schalter.

\newif\ifkorrekturansicht
\korrekturansichtfalse

\input{../tex-inputs/latex-vorspann}


\section[Lou Andreas-Salomé an Arthur Schnitzler, {[}25. 11. 1895{]}]{L00516 Lou Andreas-Salomé an Arthur Schnitzler, {[}25. 11. 1895{]}}
\nopagebreak\mylabel{L00516v}
\rehead{ }\normalsize\beginnumbering\briefempfaengerindex{Schnitzler, Arthur@\textsc{Schnitzler, Arthur}!zzzAndreas-Salomé, Lou@\emph{von Lou Andreas-Salomé}!1895-11-251@{{[}25. 11. 1895{]}}|(be}
\toendnotes[C]{\smallbreak\pagebreak[2]}
\correspDesc{Versand  durch Lou Andreas-Salomé am [25. 11. 1895] in Wien
\newline{}Erhalt  durch Arthur Schnitzler im Zeitraum [25. 11. 1895 – 29. 11. 1895?] in Wien}\toendnotes[C]{\smallbreak}
\Standort{CUL, Schnitzler, B 3.}
\physDesc{Brief, 1 Blatt, 2 Seiten, 1088 Zeichen
\newline{}Handschrift: schwarze Tinte, deutsche Kurrent
\newline{}Schnitzler: 1) mit Bleistift datiert »25/11 95«  2) mit rotem Buntstift eine Unterstreichung
\newline{}Ordnung: mit rotem Buntstift von unbekannter Hand nummeriert
                                    »10« }\toendnotes[C]{\smallbreak}
\pstart
           {\pb}Montag Abend.\pend
           
\pstart{}Lieber Herr \textsc{D\textsuperscript{r}},\pend\vspace{0.5em}
\pstart
           danke für die »Liebelei\pwindex{Schnitzler, Arthur 15.\,5.\,1862 Wien – 21.\,10.\,1931 ebd.@\textsc{Schnitzler, Arthur} (15.\,5.\,1862 Wien – 21.\,10.\,1931 ebd.), \emph{Schriftsteller, Mediziner}!Liebelei. Schauspiel in drei Akten@\strich\emph{Liebelei. Schauspiel in drei Akten}|pw}«, die ich heute
               Nachmittag erhalten und{ }ſeitdem geleſen und wieder geleſen habe. Hätte ich{ }ſie{ }ſchon
               vorher gekannt, – den erſten Eindruck von Ihnen{ }ſelbſt anſtatt von den Burgſchauſpielern\oindex{Wien@\textbf{Wien}!I., Innere Stadt@\textbf{I., Innere Stadt}!Burgtheater@\textbf{Burgtheater}, \emph{Theater}|pw} empfangen,{ }ſo würde die (an{ }ſich
               vielleicht nicht{ }ſo großen) Schwächen des Spiels, beſonders des Spiels der Chriſtine\pwindex{Schnitzler, Arthur 15.\,5.\,1862 Wien – 21.\,10.\,1931 ebd.@\textsc{Schnitzler, Arthur} (15.\,5.\,1862 Wien – 21.\,10.\,1931 ebd.), \emph{Schriftsteller, Mediziner}!Liebelei. Schauspiel in drei Akten@\strich\emph{Liebelei. Schauspiel in drei Akten}|pwv}, mir nicht{ }ſo viel
               vom Beſten verwiſcht haben. Ich kam ganz gedrückt aus dem Theater, ich konnte unter
               dem Spiel Ihre Eigenart nicht überall herauserkennen. Es geht ja mit dem »\textsc{Hannele}\pwindex{\textcolor{red}{\textsuperscript{XXXX indx1}}!Hanneles Himmelfahrt. Traumdichtung in zwei Teilen@\strich\emph{Hanneles Himmelfahrt. Traumdichtung in zwei Teilen}|pw}« {\pb}auch{ }ſo: erſt dadurch, daß man das
               Werk{ }ſelbſt kennt, ergänzt und unterſtützt man den Theatereindruck, der{ }ſonſt
               unzulänglich bleibt, und wahrſcheinlich wird es allen intimen und lebensfeinen, \uline{lebenseinfachen} Kunſtwerken{ }ſo ergehen, auch bei guter
               Darſtellung. Das Theater iſt eben nothwendig ein grobes Ding, was ein Dichter aber
               mit{ }ſeiner groben Hülfe in uns hervorrufen will, iſt etwas{ }ſo zartes.\pend
           
\pstart
           Die »Liebelei\pwindex{Schnitzler, Arthur 15.\,5.\,1862 Wien – 21.\,10.\,1931 ebd.@\textsc{Schnitzler, Arthur} (15.\,5.\,1862 Wien – 21.\,10.\,1931 ebd.), \emph{Schriftsteller, Mediziner}!Liebelei. Schauspiel in drei Akten@\strich\emph{Liebelei. Schauspiel in drei Akten}|pw}« iſt wunderſchön. Von Ihnen Dreien\pwindex{Beer-Hofmann, Richard 11.\,7.\,1866 Wien – 26.\,9.\,1945 New York City@\textsc{Beer-Hofmann, Richard} (11.\,7.\,1866 Wien – 26.\,9.\,1945 New York City), \emph{Schriftsteller}|pwv}\pwindex{Hofmannsthal, Hugo von 1.\,2.\,1874 Wien – 15.\,7.\,1929 Rodaun@\textsc{Hofmannsthal, Hugo von} (1.\,2.\,1874 Wien – 15.\,7.\,1929 Rodaun), \emph{Schriftsteller}|pwv}, – von Ihnen
               drei glücklichen Freunden\pwindex{Beer-Hofmann, Richard 11.\,7.\,1866 Wien – 26.\,9.\,1945 New York City@\textsc{Beer-Hofmann, Richard} (11.\,7.\,1866 Wien – 26.\,9.\,1945 New York City), \emph{Schriftsteller}|pwv}\pwindex{Hofmannsthal, Hugo von 1.\,2.\,1874 Wien – 15.\,7.\,1929 Rodaun@\textsc{Hofmannsthal, Hugo von} (1.\,2.\,1874 Wien – 15.\,7.\,1929 Rodaun), \emph{Schriftsteller}|pwv}, –{ }ſind doch Sie der \label{K_L00516-1v}\edtext{Glücklichſte}{\lemma{\textnormal{\emph{Glücklichste}}}\Cendnote{\textnormal{Vgl. A. S.: \emph{Tagebuch}, 19. 5. 1895.
               }}}\label{K_L00516-1}.\pend
           
\pstart
           Mit herzlichem Gruß Ihre{\\[\baselineskip]}\spacefill\mbox{LouAS.}\pend
           \leftskip=0em{}\selectlanguage{ngerman}\endnumbering\briefempfaengerindex{Schnitzler, Arthur@\textsc{Schnitzler, Arthur}!zzzAndreas-Salomé, Lou@\emph{von Lou Andreas-Salomé}!1895-11-251@{{[}25. 11. 1895{]}}|)be}\mylabel{L00516h}  \newcommand{\dateiname}{L00516}\newcommand{\titel}{Lou Andreas-Salomé an Arthur Schnitzler, [25. 11. 1895]}\newcommand{\editorInnen}{Martin Anton Müller und Gerd-Hermann Susen}%% latex-leseansicht-abspann.tex
%% Abspann für die Leseansicht.
%% Der Schalter \ifkorrekturansicht ist bereits durch den Vorspann gesetzt.

%% latex-abspann.tex
%% Gemeinsamer Abspann für Korrekturansicht und Leseansicht.
%% Setzt den Schalter \ifkorrekturansicht voraus (gesetzt in den
%% einbindenden Dateien latex-korrekturansicht-abspann.tex bzw.
%% latex-leseansicht-abspann.tex).
%% ---------------------------------------------------------------

\normalsize

% Das esempio-Environment wird nur in der Leseansicht benötigt
\ifkorrekturansicht\else
\newenvironment{esempio}[3]%
{
    \vspace{1.5ex}
    \rlap{\underline{#1}}
    \par
    \setlength{\parindent}{0cm}
    \nopagebreak
    \leftskip=#2cm
    \rightskip=#3cm
}
{
    \par
}
\fi

\doendnotes{C}
\bigskip
\vfill

\clearpage

\footnotesize

\ifkorrekturansicht
  \lohead{\textsc{register}}
\fi

% theindex-Environment neu definieren ohne reledmac
\makeatletter
\renewenvironment{theindex}{%
  \ifkorrekturansicht
    \section*{\indexname}%
  \else
    \subsubsection*{Index der erwähnten Entitäten}%
  \fi
  \setlength{\parindent}{0pt}%
  \setlength{\parskip}{0pt plus 0.3pt}%
  \let\item\@idxitem
}{%
  \ifkorrekturansicht\clearpage\fi
}
\makeatother

\IfFileExists{\jobname-pw.ind}{\input{\jobname-pw.ind}}{}

% Quellenangabe nur in der Leseansicht
\ifkorrekturansicht\else
% Fallback-Definitionen, falls die .tex-Datei \titel etc. nicht gesetzt hat
\providecommand{\titel}{}
\providecommand{\editorInnen}{}
\providecommand{\dateiname}{\jobname}

\vspace{3cm}

\vfill

\footnotesize
\textsc{Quelle}: \titel. Herausgegeben von {\editorInnen}. In: \emph{Arthur Schnitzler: Briefwechsel mit Autorinnen und Autoren}.
 Digitale Edition, https://schnitzler-briefe.acdh.oeaw.ac.at/{\dateiname}.html (Stand \today)
\fi

\end{document}


