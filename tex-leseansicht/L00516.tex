%% latex-korrekturansicht-vorspann.tex
%% Vorspann für die Korrekturansicht.
%% Lädt die gemeinsame Datei latex-vorspann.tex mit gesetztem Schalter.

\newif\ifkorrekturansicht
\korrekturansichttrue

\input{../tex-inputs/latex-vorspann}


\section[Lou Andreas-Salomé an Arthur Schnitzler, {[}25. 11. 1895{]}]{L00516 Lou Andreas-Salomé an Arthur Schnitzler, {[}25. 11. 1895{]}}
\nopagebreak\mylabel{L00516v}
\rehead{ }\normalsize\beginnumbering\briefempfaengerindex{Schnitzler, Arthur@\textsc{Schnitzler, Arthur}!zzzAndreas-Salome, Lou@\emph{von Lou Andreas-Salomé}!1895-11-251@{{[}25. 11. 1895{]}}|(be}
\toendnotes[C]{\smallbreak\pagebreak[2]}\Standort{CUL, Schnitzler, B 3.}
\physDesc{Brief, 1 Blatt, 2 Seiten, 1088 Zeichen
\newline{}Handschrift: schwarze Tinte, deutsche Kurrent
\newline{}Schnitzler: 1) mit Bleistift datiert »25/11 95«  2) mit rotem Buntstift eine Unterstreichung
\newline{}Ordnung: mit rotem Buntstift von unbekannter Hand nummeriert
                                    »10« }\toendnotes[C]{\smallbreak}
\pstart
           {\pb}Montag Abend.\pend
           
\pstart{}Lieber Herr \textsc{D\textsuperscript{r}},\pend\vspace{0.5em}
\pstart
           danke für die »Liebelei\pwindex{Liebelei. Schauspiel in drei Akten@\emph{Liebelei. Schauspiel in drei Akten}|pw}«, die ich heute
               Nachmittag erhalten und ſeitdem geleſen und wieder geleſen habe. Hätte ich ſie ſchon
               vorher gekannt, – den erſten Eindruck von Ihnen ſelbſt anſtatt von den Burgſchauſpielern\oindex{Burgtheater@\textbf{Burgtheater}, \emph{S.THTR}|pw} empfangen, ſo würde die (an ſich
               vielleicht nicht ſo großen) Schwächen des Spiels, beſonders des Spiels der Chriſtine\pwindex{Liebelei. Schauspiel in drei Akten@\emph{Liebelei. Schauspiel in drei Akten}|pwv}, mir nicht ſo viel
               vom Beſten verwiſcht haben. Ich kam ganz gedrückt aus dem Theater, ich konnte unter
               dem Spiel Ihre Eigenart nicht überall herauserkennen. Es geht ja mit dem »\textsc{Hannele}\pwindex{Hanneles Himmelfahrt. Traumdichtung in zwei Teilen@\emph{Hanneles Himmelfahrt. Traumdichtung in zwei Teilen}|pw}« {\pb}auch ſo: erſt dadurch, daß man das
               Werk ſelbſt kennt, ergänzt und unterſtützt man den Theatereindruck, der ſonſt
               unzulänglich bleibt, und wahrſcheinlich wird es allen intimen und lebensfeinen, \uline{lebenseinfachen} Kunſtwerken ſo ergehen, auch bei guter
               Darſtellung. Das Theater iſt eben nothwendig ein grobes Ding, was ein Dichter aber
               mit ſeiner groben Hülfe in uns hervorrufen will, iſt etwas ſo zartes.\pend
           
\pstart
           Die »Liebelei\pwindex{Liebelei. Schauspiel in drei Akten@\emph{Liebelei. Schauspiel in drei Akten}|pw}« iſt wunderſchön. Von Ihnen Dreien\pwindex{Beer-Hofmann, Richard 1866-07-11 – 1945-09-26@\textsc{Beer-Hofmann, Richard} (1866-07-11 – 1945-09-26), \emph{Schriftsteller/Schriftstellerin}|pwv}\pwindex{Hofmannsthal, Hugo von 1874-02-01 – 1929-07-15@\textsc{Hofmannsthal, Hugo von} (1874-02-01 – 1929-07-15), \emph{Schriftsteller/Schriftstellerin}|pwv}, – von Ihnen
               drei glücklichen Freunden\pwindex{Beer-Hofmann, Richard 1866-07-11 – 1945-09-26@\textsc{Beer-Hofmann, Richard} (1866-07-11 – 1945-09-26), \emph{Schriftsteller/Schriftstellerin}|pwv}\pwindex{Hofmannsthal, Hugo von 1874-02-01 – 1929-07-15@\textsc{Hofmannsthal, Hugo von} (1874-02-01 – 1929-07-15), \emph{Schriftsteller/Schriftstellerin}|pwv}, – ſind doch Sie der \label{K_L00516-1v}\edtext{Glücklichſte}{\lemma{\textnormal{\emph{Glücklichſte}}}\Cendnote{\textnormal{Vgl. A. S.: \emph{Tagebuch}, 19. 5. 1895.
               }}}\label{K_L00516-1}.\pend
           
\pstart
           Mit herzlichem Gruß Ihre{\\[\baselineskip]}\spacefill\mbox{LouAS.}\pend
           \leftskip=0em{}\selectlanguage{ngerman}\endnumbering\briefempfaengerindex{Schnitzler, Arthur@\textsc{Schnitzler, Arthur}!zzzAndreas-Salome, Lou@\emph{von Lou Andreas-Salomé}!1895-11-251@{{[}25. 11. 1895{]}}|)be}\mylabel{L00516h}  \normalsize

\doendnotes{C}
\bigskip
\vfill

\clearpage

\footnotesize

\lohead{\textsc{register}}

% Definiere theindex-Environment komplett neu ohne reledmac
\makeatletter
\renewenvironment{theindex}{%
  \section*{\indexname}%
  \setlength{\parindent}{0pt}%
  \setlength{\parskip}{0pt plus 0.3pt}%
  \let\item\@idxitem
}{%
  \clearpage
}
\makeatother

\IfFileExists{\jobname-pw.ind}{\input{\jobname-pw.ind}}{}

\end{document}

      