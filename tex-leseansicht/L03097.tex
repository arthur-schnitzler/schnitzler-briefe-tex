%% latex-korrekturansicht-vorspann.tex
%% Vorspann für die Korrekturansicht.
%% Lädt die gemeinsame Datei latex-vorspann.tex mit gesetztem Schalter.

\newif\ifkorrekturansicht
\korrekturansichttrue

\input{../tex-inputs/latex-vorspann}


\section[ Paul Goldmann an Arthur Schnitzler und Olga Gussmann, 23. 12. {[}1901{]}]{L03097 Paul Goldmann an Arthur Schnitzler und Olga
               Gussmann, 23. 12. {[}1901{]}}
\nopagebreak\mylabel{L03097v}
\rehead{ }\normalsize\beginnumbering\briefempfaengerindex{Schnitzler, Olga@\textsc{Schnitzler, Olga}!zzzGoldmann, Paul@\emph{von Paul Goldmann}!1901-12-231@{23. 12. {[}1901{]}}|(be}\briefempfaengerindex{Schnitzler, Arthur@\textsc{Schnitzler, Arthur}!zzzGoldmann, Paul@\emph{von Paul Goldmann}!1901-12-231@{23. 12. {[}1901{]}}|(be}
\toendnotes[C]{\smallbreak\pagebreak[2]}\Standort{DLA, A:Schnitzler, HS.NZ85.1.3171.}
\physDesc{Brief, 1 Blatt, 4 Seiten, 2205 Zeichen
\newline{}Handschrift: blaue Tinte, deutsche Kurrent
\newline{}Schnitzler: mit rotem Buntstift eine Unterstreichung }\toendnotes[C]{\smallbreak}
\pstart
           \raggedleft{}{\pb}\textcolor{gray}{\textbf{DESSAUERSTRASSE 19}}\oindex{Dessauer Strasse@\textbf{Dessauer Straße}, \emph{Straße (K.STR)}|pw}\pend
           
\pstart
           Berlin\oindex{Berlin@\textbf{Berlin}, \emph{P.PPLC}|pw}, 23. Dezember.\pend
           
\pstart\center{}Mein lieber Freund,\pend\vspace{0.5em}
\pstart
           Ich fahre heut{ }Mittag nach Frankfurt\oindex{Frankfurt am Main@\textbf{Frankfurt am Main}, \emph{P.PPLA3}|pw}. \label{K_L03097-1v}\edtext{Wenn Du gekommen wäreſt}{\lemma{\textnormal{\emph{Wenn Du gekommen wäreſt}}}\Cendnote{\textnormal{Siehe Paul Goldmann an Arthur Schnitzler, 19. 12. [1901].
               }}}\label{K_L03097-1}, ſo wäre ich erſt morgen gefahren. Ich bedaure
               unendlich, daß ich Dich jetzt nicht ſehen kann.\pend
           
\pstart
           Was Du mir über \label{K_L03097-2v}\edtext{\textsc{Olga}}{\lemma{\textnormal{\emph{Olga}}}\Cendnote{\textnormal{Olga Gussmann\pwindex{Schnitzler, Olga 17.01.1882 – 13.01.1970@\textsc{Schnitzler, Olga} (17.01.1882 – 13.01.1970), \emph{Schauspieler/Schauspielerin, Sänger/Sängerin}|pwk} war erneut schwanger. Am 9. 8. 1902 brachte
                  sie den gemeinsamen Sohn Heinrich\pwindex{Schnitzler, Heinrich 09.08.1902 – 12.07.1982@\textsc{Schnitzler, Heinrich} (09.08.1902 – 12.07.1982), \emph{Regisseur/Regisseurin, Schauspieler/Schauspielerin}|pwk} auf die
                  Welt.}}}\label{K_L03097-2} ſchreibſt, iſt ſehr erfreulich auch für mich, weil es ja, wie ich
               weiß, Euren Wünſchen entſpricht. Ich wünſche von Herzen, daß die kritiſche Zeit
               vorübergehen möge, ohne \strikeout{daß} allzuviel \label{K_L03097-3v}\edtext{Leiden}{\lemma{\textnormal{\emph{Leiden}}}\Cendnote{\textnormal{Siehe A. S.: \emph{Tagebuch}, 23. 12. 1901.
               }}}\label{K_L03097-3} und Aufregung. Ich \substVorne{}\textsuperscript{h\textcolor{gray}{offe}}\substDazwischen{}denke\substHinten{}, daß ſich in Euer Beider Leben Manches freundlicher {\pb}und ruhiger geſtalten wird, wenn dieſe Hoffnung ſich
               erfüllt haben wird. Gern würde ich \textsc{Olga} noch ein paar
               Zeilen ſchreiben. Aber ich habe keine Minute und kann gerade noch raſch dieſen Brief
               fertigſtellen, den \textsc{Olga} auch als einen an ſie gerichteten
               betrachten ſoll. Liebes Fräulein \textsc{Olga}, Ich wünſche Ihnen
               von ganzem Herzen Glück. Und es wird Alles ſchon gut werden.\pend
           
\pstart
           Wenn ich von \label{K_L03097-4v}\edtext{Urtheilsloſigkeit der Wien\oindex{Wien@\textbf{Wien}, \emph{A.ADM2}|pw}er Freunde}{\lemma{\textnormal{\emph{Urtheilsloſigkeit … Freunde}}}\Cendnote{\textnormal{Siehe Paul Goldmann an Arthur Schnitzler, 19. 12. [1901].
               }}}\label{K_L03097-4} geſprochen habe, ſo iſt wieder einmal mein Temperament mit mir durchgegangen.
               Entſchuldige den ſchroffen {\pb}Ausdruck! Daß \label{K_L03097-5v}\edtext{\uline{Du} von »Lebendigen
                  Stunden\pwindex{Lebendige Stunden@\emph{Lebendige Stunden}|pw}« mehr hältſt}{\lemma{\textnormal{\emph{Du … hältſt}}}\Cendnote{\textnormal{Siehe Arthur Schnitzler an Hermann Bahr, 28. 10. 1901.
               }}}\label{K_L03097-5}, als von der »Frau mit dem Dolch\pwindex{Frau mit dem Dolche@\emph{Die Frau mit dem Dolche}|pw}«, kann
               ich begreifen, da das erſte Stück\pwindex{Lebendige Stunden@\emph{Lebendige Stunden}|pwv} Deinem Herzen eben näher ſteht. Ich kann aber nicht verſtehen, wie ein
               objektiv denkender \strikeout{Dritter} Anderer ſich über die
               vorausſichtliche Bühnenwirkung der beiden Stücke\pwindex{Frau mit dem Dolche@\emph{Die Frau mit dem Dolche}|pw}\pwindex{Lebendige Stunden@\emph{Lebendige Stunden}|pw} täuſchen kann. Es iſt klar, daß die »Frau mit dem Dolch\pwindex{Frau mit dem Dolche@\emph{Die Frau mit dem Dolche}|pw}« der Erfolg des Abends ſein wird und daß die »Lebendigen Stunden\pwindex{Lebendige Stunden@\emph{Lebendige Stunden}|pw}\label{T_L03097-1v}\edtext{«,}{\lemma{\textnormal{\emph{«,}}}\Cendnote{\textnormal{korrigiert
                  aus »,««}}}\label{T_L03097-1} wenn nicht die Darſtellung ein Wunder thut, faſt
               wirkungslos bleiben werden. Die »Letzten Masken\pwindex{letzten Masken@\emph{Die letzten Masken}|pw}«
               habe ich auch geleſen – Ich {\pb}konnte es nicht
               ſertigbringen, das Buch\pwindex{Lebendige Stunden. Vier Einakter@\emph{Lebendige Stunden. Vier Einakter}|pwv} auf
               dem Tiſch liegen zu laſſen und bis zur \label{K_L03097-6v}\edtext{\textsc{Première\pwindex{Lebendige Stunden. Vier Einakter@\emph{Lebendige Stunden. Vier Einakter}|pwv}}}{\lemma{\textnormal{\emph{Première}}}\Cendnote{\textnormal{am 4. 1. 1902 am Deutschen Theater Berlin\oindex{Deutsches Theater Berlin@\textbf{Deutsches Theater Berlin}, \emph{Theater (K.THE)}|pwk}}}}\label{K_L03097-6} zu warten. Ich fand darin Geiſtreiches und Feines, hatte aber nicht den
               ſtarken Eindruck, den ich erwartet hatte. Das eigentliche Drama wäre meiner Anſicht
               nach doch geweſen, wenn der Journaliſt\pwindex{letzten Masken@\emph{Die letzten Masken}|pwv} dem Schriftſteller\pwindex{letzten Masken@\emph{Die letzten Masken}|pwv} geſagt hätte, was er ihm zu ſagen hatte. Dann wäre es
               natürlich ein anderes Stück geworden; aber ich weiß nicht, ob \strikeout{\textcolor{gray}{es} nicht d\textcolor{gray}{ram}} nicht ein \uline{Dramatiker} gerade dieſes \strikeout{Stück hätte} andere Stück hätte ſchreiben müſſen. Im
               Übrigen, die Aufführung\pwindex{Lebendige Stunden. Vier Einakter@\emph{Lebendige Stunden. Vier Einakter}|pwv} wird
                  lehren{\dotsfour}\pend
           
\pstart
           Tauſend Grüße, mein lieber Freund! Und frohe Feiertage! {\\[\baselineskip]}Dein \spacefill\mbox{Paul
                  Goldmann}\pend
           \leftskip=0em{}
\pstart
           \noindent{}{\pb}\label{T_L03097-2v}\edtext{Bitte, ſchreib’ mir nach Frankfurt\oindex{Frankfurt am Main@\textbf{Frankfurt am Main}, \emph{P.PPLA3}|pw}: \textsc{Reuterweg} 59\oindex{Reuterweg@\textbf{Reuterweg}, \emph{Straße (K.STR)}|pw}, bei \textsc{Dr. Rosengart\pwindex{Rosengart, Josef 1860-02-08 – 1927-08-04@\textsc{Rosengart, Josef} (1860-02-08 – 1927-08-04), \emph{Arzt/Ärztin}|pw}}.}{\lemma{\textnormal{\emph{Bitte, … Rosengart.}}}\Cendnote{\textnormal{kopfüber am oberen Rand der
                     ersten Seite}}}\label{T_L03097-2}\pend
           \selectlanguage{ngerman}\endnumbering\briefempfaengerindex{Schnitzler, Olga@\textsc{Schnitzler, Olga}!zzzGoldmann, Paul@\emph{von Paul Goldmann}!1901-12-231@{23. 12. {[}1901{]}}|)be}\briefempfaengerindex{Schnitzler, Arthur@\textsc{Schnitzler, Arthur}!zzzGoldmann, Paul@\emph{von Paul Goldmann}!1901-12-231@{23. 12. {[}1901{]}}|)be}\mylabel{L03097h}  \normalsize

\doendnotes{C}
\bigskip
\vfill

\clearpage

\footnotesize

\lohead{\textsc{register}}

% Definiere theindex-Environment komplett neu ohne reledmac
\makeatletter
\renewenvironment{theindex}{%
  \section*{\indexname}%
  \setlength{\parindent}{0pt}%
  \setlength{\parskip}{0pt plus 0.3pt}%
  \let\item\@idxitem
}{%
  \clearpage
}
\makeatother

\IfFileExists{\jobname-pw.ind}{\input{\jobname-pw.ind}}{}

\end{document}

      