%% latex-leseansicht-vorspann.tex
%% Vorspann für die Leseansicht.
%% Lädt die gemeinsame Datei latex-vorspann.tex mit nicht gesetztem Schalter.

\newif\ifkorrekturansicht
\korrekturansichtfalse

\input{../tex-inputs/latex-vorspann}


         
         \renewcommand{\erwaehntePersonen}{Personen: Josef Rosengart, Olga Schnitzler, Heinrich Schnitzler}
         \renewcommand{\erwaehnteOrte}{Orte: Berlin, Dessauer Straße, Deutsches Theater Berlin, Frankfurt am Main, Reuterweg, Wien}
         \renewcommand{\erwaehnteWerke}{Werke: Die Frau mit dem Dolche, Die letzten Masken, Lebendige Stunden, Lebendige Stunden. Vier Einakter}
               \section[ Paul Goldmann an Arthur Schnitzler und Olga Gussmann, 23. 12. {[}1901{]}]{ Paul Goldmann an Arthur Schnitzler und Olga
               Gussmann, 23. 12. {[}1901{]}}\nopagebreak\mylabel{v}\rehead{ }\begin{ledgroupsized}[t]{13cm}\normalsize\beginnumbering \toendnotes[C]{\smallbreak\pagebreak[2]} \Standort{DLA, A:Schnitzler, HS.NZ85.1.3171.}
\physDesc{Brief, 1 Blatt, 4 Seiten, 2205 Zeichen
\newline{}Handschrift: blaue Tinte, deutsche Kurrent
\newline{}Schnitzler: mit rotem Buntstift eine Unterstreichung }\toendnotes[C]{\smallbreak}\pstart
           \noindent{}\raggedleft{}{\pb}\textcolor{gray}{\textbf{DESSAUERSTRASSE 19}}\oindex{Dessauer Strasse@\textbf{Dessauer Straße}|pw}\pend
           \pstart
           Berlin\oindex{Berlin@\textbf{Berlin}|pw}, 23. Dezember.\pend
           \pstart\center{}Mein lieber Freund,\pend\pstart
           Ich fahre heut{ }Mittag nach Frankfurt\oindex{Frankfurt am Main@\textbf{Frankfurt am Main}|pw}. \label{K_L03097-1v}\edtext{Wenn Du gekommen wäreſt}{\lemma{\textnormal{\emph{Wenn Du gekommen wäreſt}}}\Cendnote{\textnormal{siehe Paul Goldmann an Arthur Schnitzler, 19. 12. [1901]}}}\label{K_L03097-1h}, ſo wäre ich erſt morgen gefahren. Ich bedaure
               unendlich, daß ich Dich jetzt nicht ſehen kann.\pend
           \pstart
           Was Du mir über \label{K_L03097-2v}\edtext{\textsc{Olga}}{\lemma{\textnormal{\emph{Olga}}}\Cendnote{\textnormal{Olga Gussmann\pwindex{Schnitzler, Olga 17.01.1882 – 13.01.1970@\textsc{Schnitzler, Olga} (17.01.1882 – 13.01.1970), \emph{Schauspielerin, Sängerin}|pwk} war erneut schwanger. Am 9. 8. 1902 brachte
                  sie den gemeinsamen Sohn Heinrich\pwindex{Schnitzler, Heinrich 09.08.1902 – 12.07.1982@\textsc{Schnitzler, Heinrich} (09.08.1902 – 12.07.1982), \emph{Regisseur, Schauspieler}|pwk} auf die
                  Welt.}}}\label{K_L03097-2h} ſchreibſt, iſt ſehr erfreulich auch für mich, weil es ja, wie ich
               weiß, Euren Wünſchen entſpricht. Ich wünſche von Herzen, daß die kritiſche Zeit
               vorübergehen möge, ohne \strikeout{daß} allzuviel \label{K_L03097-3v}\edtext{Leiden}{\lemma{\textnormal{\emph{Leiden}}}\Cendnote{\textnormal{siehe A. S.: \emph{Tagebuch}, 23. 12. 1901}}}\label{K_L03097-3h} und Aufregung. Ich \substVorne{}\textsuperscript{h\textcolor{gray}{offe}}\substDazwischen{}denke\substHinten{}, daß ſich in Euer Beider Leben Manches freundlicher {\pb}und ruhiger geſtalten wird, wenn dieſe Hoffnung ſich
               erfüllt haben wird. Gern würde ich \textsc{Olga} noch ein paar
               Zeilen ſchreiben. Aber ich habe keine Minute und kann gerade noch raſch dieſen Brief
               fertigſtellen, den \textsc{Olga} auch als einen an ſie gerichteten
               betrachten ſoll. Liebes Fräulein \textsc{Olga}, Ich wünſche Ihnen
               von ganzem Herzen Glück. Und es wird Alles ſchon gut werden.\pend
           \pstart
           Wenn ich von \label{K_L03097-4v}\edtext{Urtheilsloſigkeit der Wien\oindex{Wien@\textbf{Wien}|pw}er Freunde}{\lemma{\textnormal{\emph{Urtheilsloſigkeit … Freunde}}}\Cendnote{\textnormal{siehe Paul Goldmann an Arthur Schnitzler, 19. 12. [1901]}}}\label{K_L03097-4h} geſprochen habe, ſo iſt wieder einmal mein Temperament mit mir durchgegangen.
               Entſchuldige den ſchroffen {\pb}Ausdruck! Daß \label{K_L03097-5v}\edtext{\uline{Du} von »Lebendigen
                  Stunden\pwindex{Schnitzler, Arthur 15.05.1862 – 21.10.1931@\textsc{Schnitzler, Arthur} (15.05.1862 – 21.10.1931), \emph{Schriftsteller, Mediziner}!Lebendige Stunden01. 12. 1901@\strich\emph{Lebendige Stunden} {[}01. 12. 1901{]}|pw}« mehr hältſt}{\lemma{\textnormal{\emph{Du … hältſt}}}\Cendnote{\textnormal{siehe Arthur Schnitzler an Hermann Bahr, 28. 10. 1901}}}\label{K_L03097-5h}, als von der »Frau mit dem Dolch\pwindex{Schnitzler, Arthur 15.05.1862 – 21.10.1931@\textsc{Schnitzler, Arthur} (15.05.1862 – 21.10.1931), \emph{Schriftsteller, Mediziner}!Frau mit dem Dolche1901@\strich\emph{Die Frau mit dem Dolche} {[}1901{]}|pw}«, kann
               ich begreifen, da das erſte Stück\pwindex{Schnitzler, Arthur 15.05.1862 – 21.10.1931@\textsc{Schnitzler, Arthur} (15.05.1862 – 21.10.1931), \emph{Schriftsteller, Mediziner}!Lebendige Stunden01. 12. 1901@\strich\emph{Lebendige Stunden} {[}01. 12. 1901{]}|pwv} Deinem Herzen eben näher ſteht. Ich kann aber nicht verſtehen, wie ein
               objektiv denkender \strikeout{Dritter} Anderer ſich über die
               vorausſichtliche Bühnenwirkung der beiden Stücke\pwindex{Schnitzler, Arthur 15.05.1862 – 21.10.1931@\textsc{Schnitzler, Arthur} (15.05.1862 – 21.10.1931), \emph{Schriftsteller, Mediziner}!Frau mit dem Dolche1901@\strich\emph{Die Frau mit dem Dolche} {[}1901{]}|pw}\pwindex{Schnitzler, Arthur 15.05.1862 – 21.10.1931@\textsc{Schnitzler, Arthur} (15.05.1862 – 21.10.1931), \emph{Schriftsteller, Mediziner}!Lebendige Stunden01. 12. 1901@\strich\emph{Lebendige Stunden} {[}01. 12. 1901{]}|pw} täuſchen kann. Es iſt klar, daß die »Frau mit dem Dolch\pwindex{Schnitzler, Arthur 15.05.1862 – 21.10.1931@\textsc{Schnitzler, Arthur} (15.05.1862 – 21.10.1931), \emph{Schriftsteller, Mediziner}!Frau mit dem Dolche1901@\strich\emph{Die Frau mit dem Dolche} {[}1901{]}|pw}« der Erfolg des Abends ſein wird und daß die »Lebendigen Stunden\pwindex{Schnitzler, Arthur 15.05.1862 – 21.10.1931@\textsc{Schnitzler, Arthur} (15.05.1862 – 21.10.1931), \emph{Schriftsteller, Mediziner}!Lebendige Stunden01. 12. 1901@\strich\emph{Lebendige Stunden} {[}01. 12. 1901{]}|pw}\label{T_L03097-1v}\edtext{«,}{\lemma{\textnormal{\emph{«,}}}\Cendnote{\textnormal{korrigiert
                  aus »,««}}}\label{T_L03097-1h} wenn nicht die Darſtellung ein Wunder thut, faſt
               wirkungslos bleiben werden. Die »Letzten Masken\pwindex{Schnitzler, Arthur 15.05.1862 – 21.10.1931@\textsc{Schnitzler, Arthur} (15.05.1862 – 21.10.1931), \emph{Schriftsteller, Mediziner}!letzten Masken1901@\strich\emph{Die letzten Masken} {[}1901{]}|pw}«
               habe ich auch geleſen – Ich {\pb}konnte es nicht
               ſertigbringen, das Buch\pwindex{Schnitzler, Arthur 15.05.1862 – 21.10.1931@\textsc{Schnitzler, Arthur} (15.05.1862 – 21.10.1931), \emph{Schriftsteller, Mediziner}!Lebendige Stunden. Vier Einakter1901-12-23@\strich\emph{Lebendige Stunden. Vier Einakter} {[}1901-12-23{]}|pwv} auf
               dem Tiſch liegen zu laſſen und bis zur \label{K_L03097-6v}\edtext{\textsc{Première\pwindex{Schnitzler, Arthur 15.05.1862 – 21.10.1931@\textsc{Schnitzler, Arthur} (15.05.1862 – 21.10.1931), \emph{Schriftsteller, Mediziner}!Lebendige Stunden. Vier Einakter1901-12-23@\strich\emph{Lebendige Stunden. Vier Einakter} {[}1901-12-23{]}|pwv}}}{\lemma{\textnormal{\emph{Première}}}\Cendnote{\textnormal{am 4. 1. 1902 am Deutschen Theater Berlin\oindex{Deutsches Theater Berlin@\textbf{Deutsches Theater Berlin}|pwk}}}}\label{K_L03097-6h} zu warten. Ich fand darin Geiſtreiches und Feines, hatte aber nicht den
               ſtarken Eindruck, den ich erwartet hatte. Das eigentliche Drama wäre meiner Anſicht
               nach doch geweſen, wenn der Journaliſt\pwindex{Schnitzler, Arthur 15.05.1862 – 21.10.1931@\textsc{Schnitzler, Arthur} (15.05.1862 – 21.10.1931), \emph{Schriftsteller, Mediziner}!letzten Masken1901@\strich\emph{Die letzten Masken} {[}1901{]}|pwv} dem Schriftſteller\pwindex{Schnitzler, Arthur 15.05.1862 – 21.10.1931@\textsc{Schnitzler, Arthur} (15.05.1862 – 21.10.1931), \emph{Schriftsteller, Mediziner}!letzten Masken1901@\strich\emph{Die letzten Masken} {[}1901{]}|pwv} geſagt hätte, was er ihm zu ſagen hatte. Dann wäre es
               natürlich ein anderes Stück geworden; aber ich weiß nicht, ob \strikeout{\textcolor{gray}{es} nicht d\textcolor{gray}{ram}} nicht ein \uline{Dramatiker} gerade dieſes \strikeout{Stück hätte} andere Stück hätte ſchreiben müſſen. Im
               Übrigen, die Aufführung\pwindex{Schnitzler, Arthur 15.05.1862 – 21.10.1931@\textsc{Schnitzler, Arthur} (15.05.1862 – 21.10.1931), \emph{Schriftsteller, Mediziner}!Lebendige Stunden. Vier Einakter1901-12-23@\strich\emph{Lebendige Stunden. Vier Einakter} {[}1901-12-23{]}|pwv} wird
                  lehren{\dotsfour}\pend
           \pstart
           Tauſend Grüße, mein lieber Freund! Und frohe Feiertage! {\\[\baselineskip]}Dein \spacefill\mbox{Paul
                  Goldmann}\pend
           \leftskip=0em{}\pstart
           \noindent{}{\pb}\label{T_L03097-2v}\edtext{Bitte, ſchreib’ mir nach Frankfurt\oindex{Frankfurt am Main@\textbf{Frankfurt am Main}|pw}: \textsc{Reuterweg} 59\oindex{Reuterweg@\textbf{Reuterweg}|pw}, bei \textsc{Dr. Rosengart\pwindex{Rosengart, Josef 1860-02-08 – 1927-08-04@\textsc{Rosengart, Josef} (1860-02-08 – 1927-08-04), \emph{Arzt}|pw}}.}{\lemma{\textnormal{\emph{Bitte, … Rosengart.}}}\Cendnote{\textnormal{kopfüber am oberen Rand der
                     ersten Seite}}}\label{T_L03097-2h}\pend
           
         
         \endnumbering\mylabel{h}\end{ledgroupsized}  \newcommand{\dateiname}{L03097}\newcommand{\titel}{Paul Goldmann an Arthur Schnitzler und Olga Gussmann, 23. 12. [1901]}\newcommand{\editorInnen}{Martin Anton Müller und Laura Untner}%% latex-leseansicht-abspann.tex
%% Abspann für die Leseansicht.
%% Der Schalter \ifkorrekturansicht ist bereits durch den Vorspann gesetzt.

%% latex-abspann.tex
%% Gemeinsamer Abspann für Korrekturansicht und Leseansicht.
%% Setzt den Schalter \ifkorrekturansicht voraus (gesetzt in den
%% einbindenden Dateien latex-korrekturansicht-abspann.tex bzw.
%% latex-leseansicht-abspann.tex).
%% ---------------------------------------------------------------

\normalsize

% Das esempio-Environment wird nur in der Leseansicht benötigt
\ifkorrekturansicht\else
\newenvironment{esempio}[3]%
{
    \vspace{1.5ex}
    \rlap{\underline{#1}}
    \par
    \setlength{\parindent}{0cm}
    \nopagebreak
    \leftskip=#2cm
    \rightskip=#3cm
}
{
    \par
}
\fi

\doendnotes{C}
\bigskip
\vfill

\clearpage

\footnotesize

\ifkorrekturansicht
  \lohead{\textsc{register}}
\fi

% theindex-Environment neu definieren ohne reledmac
\makeatletter
\renewenvironment{theindex}{%
  \ifkorrekturansicht
    \section*{\indexname}%
  \else
    \subsubsection*{Index der erwähnten Entitäten}%
  \fi
  \setlength{\parindent}{0pt}%
  \setlength{\parskip}{0pt plus 0.3pt}%
  \let\item\@idxitem
}{%
  \ifkorrekturansicht\clearpage\fi
}
\makeatother

\IfFileExists{\jobname-pw.ind}{\input{\jobname-pw.ind}}{}

% Quellenangabe nur in der Leseansicht
\ifkorrekturansicht\else
% Fallback-Definitionen, falls die .tex-Datei \titel etc. nicht gesetzt hat
\providecommand{\titel}{}
\providecommand{\editorInnen}{}
\providecommand{\dateiname}{\jobname}

\vspace{3cm}

\vfill

\footnotesize
\textsc{Quelle}: \titel. Herausgegeben von {\editorInnen}. In: \emph{Arthur Schnitzler: Briefwechsel mit Autorinnen und Autoren}.
 Digitale Edition, https://schnitzler-briefe.acdh.oeaw.ac.at/{\dateiname}.html (Stand \today)
\fi

\end{document}


      