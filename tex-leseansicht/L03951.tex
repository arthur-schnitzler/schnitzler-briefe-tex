%% latex-leseansicht-vorspann.tex
%% Vorspann für die Leseansicht.
%% Lädt die gemeinsame Datei latex-vorspann.tex mit nicht gesetztem Schalter.

\newif\ifkorrekturansicht
\korrekturansichtfalse

\input{../tex-inputs/latex-vorspann}


\section[Arthur Schnitzler an Berta Zuckerkandl, 9. 12. 1911]{L03951 Arthur Schnitzler an Berta Zuckerkandl, 9. 12. 1911}
\nopagebreak\mylabel{L03951v}
\rehead{ }\normalsize\beginnumbering\briefempfaengerindex{Zuckerkandl, Berta@\textsc{Zuckerkandl, Berta}!zzzSchnitzler, Arthur@\emph{von Arthur Schnitzler}!1911-12-091@{9. 12. 1911}|(be}
\toendnotes[C]{\smallbreak\pagebreak[2]}
\correspDesc{Versand  durch Arthur Schnitzler am 9. 12. 1911 in Wien
\newline{}Erhalt  durch Berta Zuckerkandl im Zeitraum [9. 12. 1911
                  – 12. 12. 1911?] in Wien}\toendnotes[C]{\smallbreak}
\Standort{DLA, HS.1985.1.2282.}
\physDesc{Brief, Durchschlag, 1 Blatt, 2 Seiten, 1074 Zeichen
\newline{}Schreibmaschine
\newline{}Handschrift: roter Buntstift, lateinische Kurrent (\noindent{}beschriftet: »\uline{Zuckerkandl}« und »Frk«, vier Unterstreichungen)}\toendnotes[C]{\smallbreak}
\pstart
           \raggedleft{}{\pb}9. 12. 1911.\pend
           
\pstart\center{}Verehrte gnädige Frau.\pend\vspace{0.5em}
\pstart
           Herzlichen Dank für die freundliche Uebersendung des \label{K_L03951-1v}\edtext{Briefes von Madame Lefèvre\pwindex{Lefèvre, A. @\textsc{Lefèvre, A.}, \emph{Übersetzerin}|pw}}{\lemma{\textnormal{\emph{Briefes … Lefèvre}}}\Cendnote{\textnormal{nicht
                  überliefert}}}\label{K_L03951-1}.
               Ich habe also gleich \label{K_L03951-2v}\edtext{an sie
                  geschrieben}{\lemma{\textnormal{\emph{an sie
                  geschrieben}}}\Cendnote{\textnormal{Es sind keine Briefe Schnitzlers an Levèvre\pwindex{Lefèvre, A. @\textsc{Lefèvre, A.}, \emph{Übersetzerin}|pwk} überliefert.}}}\label{K_L03951-2} und sie vor allem um
               Beantwortung der Frage ersucht, ob \strikeout{Ghui}{ }Guitry\pwindex{Guitry, Lucien 13.\,12.\,1860 Paris – 1.\,6.\,1925 ebd.@\textsc{Guitry, Lucien} (13.\,12.\,1860 Paris – 1.\,6.\,1925 ebd.), \emph{Schriftsteller, Schauspieler}|pw} im nächsten Jahre bei Antoine\pwindex{Antoine, André 31.\,1.\,1858 Limoges – 23.\,10.\,1943 Le Pouliguen@\textsc{Antoine, André} (31.\,1.\,1858 Limoges – 23.\,10.\,1943 Le Pouliguen), \emph{Theaterleiter, Schauspieler}|pw} sein wird, ferner ob sie überzeugt ist, dass Antoine\pwindex{Antoine, André 31.\,1.\,1858 Limoges – 23.\,10.\,1943 Le Pouliguen@\textsc{Antoine, André} (31.\,1.\,1858 Limoges – 23.\,10.\,1943 Le Pouliguen), \emph{Theaterleiter, Schauspieler}|pw} ihr eine Uebersetzung besonders gerne
               anvertrauen würde. Denn schliesslich Herr Remon\pwindex{Rémon, Maurice 27.\,11.\,1861 Paris – 20.\,6.\,1945 Mérignac@\textsc{Rémon, Maurice} (27.\,11.\,1861 Paris – 20.\,6.\,1945 Mérignac), \emph{Übersetzer}|pw} hat mir über seine Verbindung mit Antoine\pwindex{Antoine, André 31.\,1.\,1858 Limoges – 23.\,10.\,1943 Le Pouliguen@\textsc{Antoine, André} (31.\,1.\,1858 Limoges – 23.\,10.\,1943 Le Pouliguen), \emph{Theaterleiter, Schauspieler}|pw} ungefähr \label{K_L03951-3v}\edtext{dasselbe
                  geschrieben}{\lemma{\textnormal{\emph{dasselbe
                  geschrieben}}}\Cendnote{\textnormal{Maurice Rémon\pwindex{Rémon, Maurice 27.\,11.\,1861 Paris – 20.\,6.\,1945 Mérignac@\textsc{Rémon, Maurice} (27.\,11.\,1861 Paris – 20.\,6.\,1945 Mérignac), \emph{Übersetzer}|pwk} an Arthur Schnitzler, 1. 11. 1911, vgl.
                     Karl Zieger: \emph{Arthur Schnitzler et la France 1894–1938.
                        Enquête sur une réception}, Villeneuve d’Ascq:
                        \emph{Presses Universitaires du
                        Septentrion} 2012, S. 190.
                  }}}\label{K_L03951-3} und behauptet noch überdies,
               dass seine Uebersetzung\pwindex{Schnitzler, Arthur 15. 5. 1862 Wien – 21. 10. 1931 ebd.@\textsc{Schnitzler, Arthur} (15. 5. 1862 Wien – 21. 10. 1931 ebd.), \emph{Schriftsteller, Mediziner}!femme au poignard@\strich\emph{La femme au poignard}|pwv} der
                  »Frau mit dem Dolch\pwindex{Schnitzler, Arthur 15. 5. 1862 Wien – 21. 10. 1931 ebd.@\textsc{Schnitzler, Arthur} (15. 5. 1862 Wien – 21. 10. 1931 ebd.), \emph{Schriftsteller, Mediziner}!Frau mit dem Dolche@\strich\emph{Die Frau mit dem Dolche}|pw}« heuer bei Antoine\pwindex{Antoine, André 31.\,1.\,1858 Limoges – 23.\,10.\,1943 Le Pouliguen@\textsc{Antoine, André} (31.\,1.\,1858 Limoges – 23.\,10.\,1943 Le Pouliguen), \emph{Theaterleiter, Schauspieler}|pw} zur Aufführung kommen soll. Nebstbei
               scheint er doch in direkter Verbindung mit Guitry\pwindex{Guitry, Lucien 13.\,12.\,1860 Paris – 1.\,6.\,1925 ebd.@\textsc{Guitry, Lucien} (13.\,12.\,1860 Paris – 1.\,6.\,1925 ebd.), \emph{Schriftsteller, Schauspieler}|pw} zu stehen und es kommt in der Angelegenheit des »Weiten Landes\pwindex{Schnitzler, Arthur 15. 5. 1862 Wien – 21. 10. 1931 ebd.@\textsc{Schnitzler, Arthur} (15. 5. 1862 Wien – 21. 10. 1931 ebd.), \emph{Schriftsteller, Mediziner}!weite Land. Tragikomödie in fünf Akten@\strich\emph{Das weite Land. Tragikomödie in fünf Akten}|pw}« doch mehr auf Guitry\pwindex{Guitry, Lucien 13.\,12.\,1860 Paris – 1.\,6.\,1925 ebd.@\textsc{Guitry, Lucien} (13.\,12.\,1860 Paris – 1.\,6.\,1925 ebd.), \emph{Schriftsteller, Schauspieler}|pw} an als auf Antoine\pwindex{Antoine, André 31.\,1.\,1858 Limoges – 23.\,10.\,1943 Le Pouliguen@\textsc{Antoine, André} (31.\,1.\,1858 Limoges – 23.\,10.\,1943 Le Pouliguen), \emph{Theaterleiter, Schauspieler}|pw}, umso mehr als Antoine\pwindex{Antoine, André 31.\,1.\,1858 Limoges – 23.\,10.\,1943 Le Pouliguen@\textsc{Antoine, André} (31.\,1.\,1858 Limoges – 23.\,10.\,1943 Le Pouliguen), \emph{Theaterleiter, Schauspieler}|pw} es
               nicht einmal der Mühe wert hält mir auf meinen direkten und klare Fragen
               formulierenden \label{K_L03951-4v}\edtext{Brief}{\lemma{\textnormal{\emph{Brief}}}\Cendnote{\textnormal{Arthur Schnitzler an André Antoine\pwindex{Antoine, André 31.\,1.\,1858 Limoges – 23.\,10.\,1943 Le Pouliguen@\textsc{Antoine, André} (31.\,1.\,1858 Limoges – 23.\,10.\,1943 Le Pouliguen), \emph{Theaterleiter, Schauspieler}|pwk}, 20. 11. 1911, \emph{Deutsches Literaturarchiv Marbach},
                  HS.1985.1.257.}}}\label{K_L03951-4} zu antworten. {\pb}Wir wollen
               nun die Antwort der Frau Lefèvre\pwindex{Lefèvre, A. @\textsc{Lefèvre, A.}, \emph{Übersetzerin}|pw} abwarten und
               dann weiter sehen.\pend
           
\pstart
           Mit wiederholtem Dank für Ihre Bemühungen und Ihr liebenswürdiges Interesse mit
               vielen{\\[\baselineskip]} Grüssen{\\[\baselineskip]} Ihr aufrichtig ergebener\pend
           \leftskip=0em{}{\vspace{1\baselineskip}}
\pstart
           \noindent{}Frau Berta Zuckerkandl, Wien\oindex{Wien@\textbf{Wien}, \emph{Verwaltungsgebiet}|pw}.\pend
           \selectlanguage{ngerman}\endnumbering\briefempfaengerindex{Zuckerkandl, Berta@\textsc{Zuckerkandl, Berta}!zzzSchnitzler, Arthur@\emph{von Arthur Schnitzler}!1911-12-091@{9. 12. 1911}|)be}\mylabel{L03951h}
\begin{anhang}
\end{anhang}\newcommand{\dateiname}{L03951}\newcommand{\titel}{Arthur Schnitzler an Berta Zuckerkandl, 9. 12. 1911}\newcommand{\editorInnen}{Herausgegeben von Jahnke, SelmaMüller, Martin Anton}%% latex-leseansicht-abspann.tex
%% Abspann für die Leseansicht.
%% Der Schalter \ifkorrekturansicht ist bereits durch den Vorspann gesetzt.

%% latex-abspann.tex
%% Gemeinsamer Abspann für Korrekturansicht und Leseansicht.
%% Setzt den Schalter \ifkorrekturansicht voraus (gesetzt in den
%% einbindenden Dateien latex-korrekturansicht-abspann.tex bzw.
%% latex-leseansicht-abspann.tex).
%% ---------------------------------------------------------------

\normalsize

% Das esempio-Environment wird nur in der Leseansicht benötigt
\ifkorrekturansicht\else
\newenvironment{esempio}[3]%
{
    \vspace{1.5ex}
    \rlap{\underline{#1}}
    \par
    \setlength{\parindent}{0cm}
    \nopagebreak
    \leftskip=#2cm
    \rightskip=#3cm
}
{
    \par
}
\fi

\doendnotes{C}
\bigskip
\vfill

\clearpage

\footnotesize

\ifkorrekturansicht
  \lohead{\textsc{register}}
\fi

% theindex-Environment neu definieren ohne reledmac
\makeatletter
\renewenvironment{theindex}{%
  \ifkorrekturansicht
    \section*{\indexname}%
  \else
    \subsubsection*{Index der erwähnten Entitäten}%
  \fi
  \setlength{\parindent}{0pt}%
  \setlength{\parskip}{0pt plus 0.3pt}%
  \let\item\@idxitem
}{%
  \ifkorrekturansicht\clearpage\fi
}
\makeatother

\IfFileExists{\jobname-pw.ind}{\input{\jobname-pw.ind}}{}

% Quellenangabe nur in der Leseansicht
\ifkorrekturansicht\else
% Fallback-Definitionen, falls die .tex-Datei \titel etc. nicht gesetzt hat
\providecommand{\titel}{}
\providecommand{\editorInnen}{}
\providecommand{\dateiname}{\jobname}

\vspace{3cm}

\vfill

\footnotesize
\textsc{Quelle}: \titel. Herausgegeben von {\editorInnen}. In: \emph{Arthur Schnitzler: Briefwechsel mit Autorinnen und Autoren}.
 Digitale Edition, https://schnitzler-briefe.acdh.oeaw.ac.at/{\dateiname}.html (Stand \today)
\fi

\end{document}


