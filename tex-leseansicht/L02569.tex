%% latex-korrekturansicht-vorspann.tex
%% Vorspann für die Korrekturansicht.
%% Lädt die gemeinsame Datei latex-vorspann.tex mit gesetztem Schalter.

\newif\ifkorrekturansicht
\korrekturansichttrue

\input{../tex-inputs/latex-vorspann}


\section[Therese Rie-Andro an Arthur Schnitzler, 6. 2. 1912]{L02569 Therese Rie-Andro an Arthur Schnitzler, 6. 2. 1912}
\nopagebreak\mylabel{L02569v}
\rehead{ }\normalsize\beginnumbering\briefempfaengerindex{Schnitzler, Arthur@\textsc{Schnitzler, Arthur}!zzzRie, Therese@\emph{von Therese Rie}!1912-02-061@{6. 2. 1912}|(be}
\toendnotes[C]{\smallbreak\pagebreak[2]}\Standort{DLA, A:Schnitzler, 85.1.4310.}
\physDesc{Brief, 1 Blatt, 3 Seiten, 2051 Zeichen
\newline{}Handschrift: blaue Tinte, lateinische Kurrent
\newline{}Schnitzler: 1) mit Bleistift beschriftet: »\textsc{Andro}«  2) mit rotem Buntstift eine Unterstreichung}\toendnotes[C]{\smallbreak}
\pstart
           \raggedleft{}{\pb}Wien\oindex{Wien@\textbf{Wien}, \emph{A.ADM2}|pw}, d. 6. Februar
                  1912.\pend
           
\pstart
           \raggedleft{}IV, Schönburgstr. 48\oindex{Schoenburgstrasse@\textbf{Schönburgstraße}, \emph{Straße (K.STR)}|pw}.\pend
           
\pstart{}Sehr geehrter Herr,\pend\vspace{0.5em}
\pstart
           Hans Pfitzner\pwindex{Pfitzner, Hans 05.05.1869 – 22.05.1949@\textsc{Pfitzner, Hans} (05.05.1869 – 22.05.1949), \emph{Komponist/Komponistin}|pw} sendet Ihnen durch mich die
               Dichtung zu seine\substVorne{}\textsuperscript{r}\substDazwischen{}m\substHinten{} neuesten \substVorne{}\textsuperscript{Arbeit}\substDazwischen{}Musikdrama\substHinten{} »Palestrina\pwindex{Palestrina. Musikalische Legende in drei Akten@\emph{Palestrina. Musikalische Legende in drei Akten}|pw}«, zugleich seinen ersten
                  \uline{dichterischen}{ }Versuch\pwindex{Palestrina. Musikalische Legende in drei Akten@\emph{Palestrina. Musikalische Legende in drei Akten}|pwv}, und bittet Sie, als
               einen der ganz Wenigen, an dessen Urteil ihm gelegen ist, sie zu lesen.\pend
           
\pstart
           Wenn er selbst sich nicht direkt an Sie wendet, liegt es zum Teil an seiner
               Ueberbürdung mit Arbeit (er iſt, wie Sie vielleicht wissen, Direktor der Oper\orgindex{Oper Strassburg@Oper Straßburg|pw} und des Konservatoriums\orgindex{Staedtisches Konservatorium@Städtisches Konservatorium|pw} in Straßburg\oindex{Strassburg@\textbf{Straßburg}, \emph{P.PPLA}|pw} und Leiter
               der Orchesterkonzerte\orgindex{Strassburger Philharmoniker@Straßburger Philharmoniker|pw}), zum Teil an einer
               gewissen Scheu dem Briefschreiben gegenüber, die er mit {\pb}manchen seiner großen Kollegen gemeinsam hat, und \substVorne{}\textsuperscript{\textcolor{gray}{lieber}}\substDazwischen{}wobei\substHinten{} er lieber seine »Jünger« ins Treffen schickt.\pend
           
\pstart
           Pfitzner\pwindex{Pfitzner, Hans 05.05.1869 – 22.05.1949@\textsc{Pfitzner, Hans} (05.05.1869 – 22.05.1949), \emph{Komponist/Komponistin}|pw} weiß, daß Sie seinen Schöpfungen Ihr
               Interesse nicht entsagt haben, wenn sie – leider viel zu wenig! – in Wien\oindex{Wien@\textbf{Wien}, \emph{A.ADM2}|pw} zu hören waren. Vielleicht aber wissen Sie, sehr geehrter
               Herr Doctor, nicht, daß er zu Ihren wärmsten Bewunderern zählt; er hat sich unter
               anderm jahrelang mit Ihrem »Parazelsus\pwindex{Paracelsus. Versspiel in einem Akt@\emph{Paracelsus. Versspiel in einem Akt}|pw}«
               beschäftigt und ich kann es nicht genug beklagen, daß seine Liebe für dieses eminent
               »musikalische« Werk sich nicht zu Musik verdichtet hat. Ich denke i{\geminationm}er, einmal wird das noch werden.\pend
           
\pstart
           Pfitzner\pwindex{Pfitzner, Hans 05.05.1869 – 22.05.1949@\textsc{Pfitzner, Hans} (05.05.1869 – 22.05.1949), \emph{Komponist/Komponistin}|pw} hat seine Dichtung\pwindex{Palestrina. Musikalische Legende in drei Akten@\emph{Palestrina. Musikalische Legende in drei Akten}|pwv} – die Partitur ist erst in den
               allerersten Anfängen vorhanden – in ganz wenigen Exemplaren für Freunde drucken
               lassen. Er hat mich ermächtigt, Ihnen das meine zu senden und ich bitte Sie, es ruhig
               so lange zu behalten, als es Ihnen lieb iſt. Doch bittet mich Pfitzner\pwindex{Pfitzner, Hans 05.05.1869 – 22.05.1949@\textsc{Pfitzner, Hans} (05.05.1869 – 22.05.1949), \emph{Komponist/Komponistin}|pw} sehr, \substVorne{}\textsuperscript{\textcolor{gray}{seine}}\substDazwischen{}die Ueber\substHinten{}sendung seiner Dichtung\pwindex{Palestrina. Musikalische Legende in drei Akten@\emph{Palestrina. Musikalische Legende in drei Akten}|pwv} als einen Akt des innigsten persönlichen Vertrauens aufzufassen und
               auch zu Freunden nicht drüber zu sprechen, ehe nicht auch der musikalische Teil der
               Arbeit vollendet ist.\pend
           
\pstart
           Verzeihen Sie, sehr geehrter Herr, wenn ich Ihnen diese ein wenig drakonischen
                  Besti{\geminationm}ungen des Meisters\pwindex{Pfitzner, Hans 05.05.1869 – 22.05.1949@\textsc{Pfitzner, Hans} (05.05.1869 – 22.05.1949), \emph{Komponist/Komponistin}|pwv} völlig ungeschminkt übermittle; allein ich bin es
               gewöhnt, mich seinen künstlerischen Wünschen unbedingt unterzuordnen und überzeugt,
               daß diese auch bei Ihnen das {\pb}\substVorne{}\textsuperscript{\textcolor{gray}{äußerste}}\substDazwischen{}absoluteste\substHinten{} Verständnis finden werden.\pend
           
\pstart
           Ich begrüße Sie in herzlicher Bewunderung.{\\[\baselineskip]}\spacefill\mbox{L. Andro. (\label{K_L02569-1v}\edtext{R.}{\lemma{\textnormal{\emph{R.}}}\Cendnote{\textnormal{für »Risa«}}}\label{K_L02569-1} Rie.)}\pend
           \leftskip=0em{}\selectlanguage{ngerman}\endnumbering\briefempfaengerindex{Schnitzler, Arthur@\textsc{Schnitzler, Arthur}!zzzRie, Therese@\emph{von Therese Rie}!1912-02-061@{6. 2. 1912}|)be}\mylabel{L02569h}  \normalsize

\doendnotes{C}
\bigskip
\vfill

\clearpage

\footnotesize

\lohead{\textsc{register}}

% Definiere theindex-Environment komplett neu ohne reledmac
\makeatletter
\renewenvironment{theindex}{%
  \section*{\indexname}%
  \setlength{\parindent}{0pt}%
  \setlength{\parskip}{0pt plus 0.3pt}%
  \let\item\@idxitem
}{%
  \clearpage
}
\makeatother

\IfFileExists{\jobname-pw.ind}{\input{\jobname-pw.ind}}{}

\end{document}

      