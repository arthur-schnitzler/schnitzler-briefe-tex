%% latex-leseansicht-vorspann.tex
%% Vorspann für die Leseansicht.
%% Lädt die gemeinsame Datei latex-vorspann.tex mit nicht gesetztem Schalter.

\newif\ifkorrekturansicht
\korrekturansichtfalse

\input{../tex-inputs/latex-vorspann}


\section[Therese Rie-Andro an Arthur Schnitzler, 6. 2. 1912]{L02569 Therese Rie-Andro an Arthur Schnitzler, 6. 2. 1912}
\nopagebreak\mylabel{L02569v}
\rehead{ }\normalsize\beginnumbering\briefempfaengerindex{Schnitzler, Arthur@\textsc{Schnitzler, Arthur}!zzzRie, Therese@\emph{von Therese Rie}!1912-02-061@{6. 2. 1912}|(be}
\toendnotes[C]{\smallbreak\pagebreak[2]}
\correspDesc{Versand  durch Therese Rie am 6. 2. 1912 in Wien
\newline{}Erhalt  durch Arthur Schnitzler im Zeitraum [6. 2. 1912
                  – 10. 2. 1912?] in Wien}\toendnotes[C]{\smallbreak}
\Standort{DLA, A:Schnitzler, 85.1.4310.}
\physDesc{Brief, 1 Blatt, 3 Seiten, 2051 Zeichen
\newline{}Handschrift: blaue Tinte, lateinische Kurrent
\newline{}Schnitzler: 1) mit Bleistift beschriftet: »\textsc{Andro}«  2) mit rotem Buntstift eine Unterstreichung}\toendnotes[C]{\smallbreak}
\pstart
           \raggedleft{}{\pb}Wien\oindex{Wien@\textbf{Wien}, \emph{Verwaltungsgebiet}|pw}, d. 6. Februar 1912.\pend
           
\pstart
           \raggedleft{}IV, Schönburgstr. 48\oindex{Wien@\textbf{Wien}!IV., Wieden@\textbf{IV., Wieden}!Schönburgstraße 48@\textbf{Schönburgstraße 48}, \emph{Wohngebäude}|pw}.\pend
           
\pstart{}Sehr geehrter Herr,\pend\vspace{0.5em}
\pstart
           Hans Pfitzner\pwindex{Pfitzner, Hans 5.\,5.\,1869 Moskau – 22.\,5.\,1949 Salzburg@\textsc{Pfitzner, Hans} (5.\,5.\,1869 Moskau – 22.\,5.\,1949 Salzburg), \emph{Komponist}|pw} sendet Ihnen durch mich die
               Dichtung zu seine\substVorne{}\textsuperscript{r}\substDazwischen{}m\substHinten{} neuesten \substVorne{}\textsuperscript{Arbeit}\substDazwischen{}Musikdrama\substHinten{} »Palestrina\pwindex{Pfitzner, Hans 5.\,5.\,1869 Moskau – 22.\,5.\,1949 Salzburg@\textsc{Pfitzner, Hans} (5.\,5.\,1869 Moskau – 22.\,5.\,1949 Salzburg), \emph{Komponist}!Palestrina. Musikalische Legende in drei Akten@\strich\emph{Palestrina. Musikalische Legende in drei Akten}|pw}«, zugleich seinen ersten
                  \uline{dichterischen}{ }Versuch\pwindex{Pfitzner, Hans 5.\,5.\,1869 Moskau – 22.\,5.\,1949 Salzburg@\textsc{Pfitzner, Hans} (5.\,5.\,1869 Moskau – 22.\,5.\,1949 Salzburg), \emph{Komponist}!Palestrina. Musikalische Legende in drei Akten@\strich\emph{Palestrina. Musikalische Legende in drei Akten}|pwv}, und bittet Sie, als
               einen der ganz Wenigen, an dessen Urteil ihm gelegen ist, sie zu lesen.\pend
           
\pstart
           Wenn er selbst sich nicht direkt an Sie wendet, liegt es zum Teil an seiner
               Ueberbürdung mit Arbeit (er iſt, wie Sie vielleicht wissen, Direktor der Oper\orgindex{Oper Straßburg@Oper Straßburg|pw} und des Konservatoriums\orgindex{Städtisches Konservatorium@Städtisches Konservatorium|pw} in Straßburg\oindex{Straßburg@\textbf{Straßburg}|pw} und Leiter
               der Orchesterkonzerte\orgindex{Straßburger Philharmoniker@Straßburger Philharmoniker|pw}), zum Teil an einer
               gewissen Scheu dem Briefschreiben gegenüber, die er mit {\pb}manchen seiner großen Kollegen gemeinsam hat, und \substVorne{}\textsuperscript{\textcolor{gray}{lieber}}\substDazwischen{}wobei\substHinten{} er lieber seine »Jünger« ins Treffen schickt.\pend
           
\pstart
           Pfitzner\pwindex{Pfitzner, Hans 5.\,5.\,1869 Moskau – 22.\,5.\,1949 Salzburg@\textsc{Pfitzner, Hans} (5.\,5.\,1869 Moskau – 22.\,5.\,1949 Salzburg), \emph{Komponist}|pw} weiß, daß Sie seinen Schöpfungen Ihr
               Interesse nicht entsagt haben, wenn sie – leider viel zu wenig! – in Wien\oindex{Wien@\textbf{Wien}, \emph{Verwaltungsgebiet}|pw} zu hören waren. Vielleicht aber wissen Sie, sehr geehrter
               Herr Doctor, nicht, daß er zu Ihren wärmsten Bewunderern zählt; er hat sich unter
               anderm jahrelang mit Ihrem »Parazelsus\pwindex{Schnitzler, Arthur 15.\,5.\,1862 Wien – 21.\,10.\,1931 ebd.@\textsc{Schnitzler, Arthur} (15.\,5.\,1862 Wien – 21.\,10.\,1931 ebd.), \emph{Schriftsteller, Mediziner}!Paracelsus. Versspiel in einem Akt@\strich\emph{Paracelsus. Versspiel in einem Akt}|pw}«
               beschäftigt und ich kann es nicht genug beklagen, daß seine Liebe für dieses eminent
               »musikalische« Werk sich nicht zu Musik verdichtet hat. Ich denke i{\geminationm}er, einmal wird das noch werden.\pend
           
\pstart
           Pfitzner\pwindex{Pfitzner, Hans 5.\,5.\,1869 Moskau – 22.\,5.\,1949 Salzburg@\textsc{Pfitzner, Hans} (5.\,5.\,1869 Moskau – 22.\,5.\,1949 Salzburg), \emph{Komponist}|pw} hat seine Dichtung\pwindex{Pfitzner, Hans 5.\,5.\,1869 Moskau – 22.\,5.\,1949 Salzburg@\textsc{Pfitzner, Hans} (5.\,5.\,1869 Moskau – 22.\,5.\,1949 Salzburg), \emph{Komponist}!Palestrina. Musikalische Legende in drei Akten@\strich\emph{Palestrina. Musikalische Legende in drei Akten}|pwv} – die Partitur ist erst in den
               allerersten Anfängen vorhanden – in ganz wenigen Exemplaren für Freunde drucken
               lassen. Er hat mich ermächtigt, Ihnen das meine zu senden und ich bitte Sie, es ruhig
               so lange zu behalten, als es Ihnen lieb iſt. Doch bittet mich Pfitzner\pwindex{Pfitzner, Hans 5.\,5.\,1869 Moskau – 22.\,5.\,1949 Salzburg@\textsc{Pfitzner, Hans} (5.\,5.\,1869 Moskau – 22.\,5.\,1949 Salzburg), \emph{Komponist}|pw} sehr, \substVorne{}\textsuperscript{\textcolor{gray}{seine}}\substDazwischen{}die Ueber\substHinten{}sendung seiner Dichtung\pwindex{Pfitzner, Hans 5.\,5.\,1869 Moskau – 22.\,5.\,1949 Salzburg@\textsc{Pfitzner, Hans} (5.\,5.\,1869 Moskau – 22.\,5.\,1949 Salzburg), \emph{Komponist}!Palestrina. Musikalische Legende in drei Akten@\strich\emph{Palestrina. Musikalische Legende in drei Akten}|pwv} als einen Akt des innigsten persönlichen Vertrauens aufzufassen und
               auch zu Freunden nicht drüber zu sprechen, ehe nicht auch der musikalische Teil der
               Arbeit vollendet ist.\pend
           
\pstart
           Verzeihen Sie, sehr geehrter Herr, wenn ich Ihnen diese ein wenig drakonischen
                  Besti{\geminationm}ungen des Meisters\pwindex{Pfitzner, Hans 5.\,5.\,1869 Moskau – 22.\,5.\,1949 Salzburg@\textsc{Pfitzner, Hans} (5.\,5.\,1869 Moskau – 22.\,5.\,1949 Salzburg), \emph{Komponist}|pwv} völlig ungeschminkt übermittle; allein ich bin es
               gewöhnt, mich seinen künstlerischen Wünschen unbedingt unterzuordnen und überzeugt,
               daß diese auch bei Ihnen das {\pb}\substVorne{}\textsuperscript{\textcolor{gray}{äußerste}}\substDazwischen{}absoluteste\substHinten{} Verständnis finden werden.\pend
           
\pstart
           Ich begrüße Sie in herzlicher Bewunderung.{\\[\baselineskip]}\spacefill\mbox{L. Andro. (\label{K_L02569-1v}\edtext{R.}{\lemma{\textnormal{\emph{R.}}}\Cendnote{\textnormal{für »Risa«}}}\label{K_L02569-1} Rie.)}\pend
           \leftskip=0em{}\selectlanguage{ngerman}\endnumbering\briefempfaengerindex{Schnitzler, Arthur@\textsc{Schnitzler, Arthur}!zzzRie, Therese@\emph{von Therese Rie}!1912-02-061@{6. 2. 1912}|)be}\mylabel{L02569h}  \newcommand{\dateiname}{L02569}\newcommand{\titel}{Therese Rie-Andro an Arthur Schnitzler, 6. 2. 1912}\newcommand{\editorInnen}{Martin Anton Müller und Gerd-Hermann Susen}%% latex-leseansicht-abspann.tex
%% Abspann für die Leseansicht.
%% Der Schalter \ifkorrekturansicht ist bereits durch den Vorspann gesetzt.

%% latex-abspann.tex
%% Gemeinsamer Abspann für Korrekturansicht und Leseansicht.
%% Setzt den Schalter \ifkorrekturansicht voraus (gesetzt in den
%% einbindenden Dateien latex-korrekturansicht-abspann.tex bzw.
%% latex-leseansicht-abspann.tex).
%% ---------------------------------------------------------------

\normalsize

% Das esempio-Environment wird nur in der Leseansicht benötigt
\ifkorrekturansicht\else
\newenvironment{esempio}[3]%
{
    \vspace{1.5ex}
    \rlap{\underline{#1}}
    \par
    \setlength{\parindent}{0cm}
    \nopagebreak
    \leftskip=#2cm
    \rightskip=#3cm
}
{
    \par
}
\fi

\doendnotes{C}
\bigskip
\vfill

\clearpage

\footnotesize

\ifkorrekturansicht
  \lohead{\textsc{register}}
\fi

% theindex-Environment neu definieren ohne reledmac
\makeatletter
\renewenvironment{theindex}{%
  \ifkorrekturansicht
    \section*{\indexname}%
  \else
    \subsubsection*{Index der erwähnten Entitäten}%
  \fi
  \setlength{\parindent}{0pt}%
  \setlength{\parskip}{0pt plus 0.3pt}%
  \let\item\@idxitem
}{%
  \ifkorrekturansicht\clearpage\fi
}
\makeatother

\IfFileExists{\jobname-pw.ind}{\input{\jobname-pw.ind}}{}

% Quellenangabe nur in der Leseansicht
\ifkorrekturansicht\else
% Fallback-Definitionen, falls die .tex-Datei \titel etc. nicht gesetzt hat
\providecommand{\titel}{}
\providecommand{\editorInnen}{}
\providecommand{\dateiname}{\jobname}

\vspace{3cm}

\vfill

\footnotesize
\textsc{Quelle}: \titel. Herausgegeben von {\editorInnen}. In: \emph{Arthur Schnitzler: Briefwechsel mit Autorinnen und Autoren}.
 Digitale Edition, https://schnitzler-briefe.acdh.oeaw.ac.at/{\dateiname}.html (Stand \today)
\fi

\end{document}


