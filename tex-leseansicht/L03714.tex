%% latex-korrekturansicht-vorspann.tex
%% Vorspann für die Korrekturansicht.
%% Lädt die gemeinsame Datei latex-vorspann.tex mit gesetztem Schalter.

\newif\ifkorrekturansicht
\korrekturansichttrue

\input{../tex-inputs/latex-vorspann}


\section[Elsa Plessner an Arthur Schnitzler, 7. 8. 1897]{L03714 Elsa Plessner an Arthur Schnitzler, 7. 8. 1897}
\nopagebreak\mylabel{L03714v}
\rehead{ }\normalsize\beginnumbering\briefempfaengerindex{Schnitzler, Arthur@\textsc{Schnitzler, Arthur}!zzzPlessner, Elsa@\emph{von Elsa Plessner}!1897-08-073@{7. 8. 1897}|(be}
\toendnotes[C]{\smallbreak\pagebreak[2]}\Standort{DLA, A:Schnitzler, HS.1985.1.419.}
\physDesc{Brief,  Blätter, 2 Seiten, 482 Zeichen
\newline{}Handschrift: , deutsche Kurrent}\toendnotes[C]{\smallbreak}
\pstart
           {\pb}Wien – Sievering. Fröschlgasse 6\oindex{Froeschelgasse 6@\textbf{Fröschelgasse 6}, \emph{Wohngebäude (K.WHS)}|pw}\pend
           
\pstart
           \raggedleft{}den 7. VIII.\pend
           
\pstart{}Verehrter Herr Doctor!\pend\vspace{0.5em}
\pstart
           Ich thue es doch nicht – d. h. die \label{K_L03714-1v}\edtext{höchstpersönliche Correctur}{\lemma{\textnormal{\emph{höchstpersönliche Correctur}}}\Cendnote{\textnormal{Plessners\pwindex{Plessner, Elsa 22.08.1875 – 01.05.1932@\textsc{Plessner, Elsa} (22.08.1875 – 01.05.1932), \emph{Schriftsteller/Schriftstellerin}|pwk} Text \emph{Der gläserne Käfig}\pwindex{glaeserne Kaefig. Eine Parabel@\emph{Der gläserne Käfig. Eine Parabel}|pwk} erschien im Erstdruck (\emph{Die Zeit}\orgindex{Zeit. Wiener Wochenschrift@Die Zeit. Wiener Wochenschrift|pwk}, Bd. 12, Nr. 149, 7. 8. 1897, S. 95–96) mit unautorisierten Änderungen, u. a. mit der
                  nicht von der Autorin vorgesehenen Gattungsbezeichnung »eine Parabel«. Als erste
                  wutentbrannte Reaktion darauf hatte Plessner\pwindex{Plessner, Elsa 22.08.1875 – 01.05.1932@\textsc{Plessner, Elsa} (22.08.1875 – 01.05.1932), \emph{Schriftsteller/Schriftstellerin}|pwk}
                  angekündigt, die ausliegenden Zeitungsexemplare in allen wichtigen Kaffeehäusern
                  per Hand zu korrigieren, vgl. Elsa Plessner an Arthur Schnitzler, 7. 8. [1897].}}}\label{K_L03714-1}. – Noch geschmackloser – ja lächerlich! Nicht? Muss
               verzichten! Aber sie werden doch meine \label{K_L03714-2v}\edtext{große Bitte}{\lemma{\textnormal{\emph{große Bitte}}}\Cendnote{\textnormal{Schnitzler sollte die »literarischen Kreise«
                  wissen lassen, dass Plessners\pwindex{Plessner, Elsa 22.08.1875 – 01.05.1932@\textsc{Plessner, Elsa} (22.08.1875 – 01.05.1932), \emph{Schriftsteller/Schriftstellerin}|pwk}{ }Text\pwindex{glaeserne Kaefig. Eine Parabel@\emph{Der gläserne Käfig. Eine Parabel}|pwkv} gegen ihren Willen
                  verändert abgedruckt worden war, vgl. Elsa Plessner an Arthur Schnitzler, 7. 8. [1897].}}}\label{K_L03714-2} erfüllen? – Die bleibt aufrecht!! {\pb}War
               furchtbar – namenlos wüthend – erster, begreiflicher Rachegedanke! – Zu dumm! – Aber
               ich bin nicht immer so! Bessere Einsicht kommt meistens nach! – Meistens! Nicht
               immer! –\pend
           
\pstart
           Womit ich zeichne{\\[\baselineskip]} Ihre dankbare{\\[\baselineskip]}\spacefill\mbox{Elsa Plessner}\pend
           \leftskip=0em{}\selectlanguage{ngerman}\endnumbering\briefempfaengerindex{Schnitzler, Arthur@\textsc{Schnitzler, Arthur}!zzzPlessner, Elsa@\emph{von Elsa Plessner}!1897-08-073@{7. 8. 1897}|)be}\mylabel{L03714h}
\begin{anhang}
\end{anhang}\normalsize

\doendnotes{C}
\bigskip
\vfill

\clearpage

\footnotesize

\lohead{\textsc{register}}

% Definiere theindex-Environment komplett neu ohne reledmac
\makeatletter
\renewenvironment{theindex}{%
  \section*{\indexname}%
  \setlength{\parindent}{0pt}%
  \setlength{\parskip}{0pt plus 0.3pt}%
  \let\item\@idxitem
}{%
  \clearpage
}
\makeatother

\IfFileExists{\jobname-pw.ind}{\input{\jobname-pw.ind}}{}

\end{document}

      