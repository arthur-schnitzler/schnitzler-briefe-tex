%% latex-leseansicht-vorspann.tex
%% Vorspann für die Leseansicht.
%% Lädt die gemeinsame Datei latex-vorspann.tex mit nicht gesetztem Schalter.

\newif\ifkorrekturansicht
\korrekturansichtfalse

\input{../tex-inputs/latex-vorspann}


\section[ Paul Goldmann an Arthur Schnitzler, 6. 12. {[}1899{]}]{L02897 Paul Goldmann an Arthur Schnitzler,  6. 12. [1899]}
\nopagebreak\mylabel{L02897v}
\rehead{ }\normalsize\beginnumbering\briefempfaengerindex{Schnitzler, Arthur@\textsc{Schnitzler, Arthur}!zzzGoldmann, Paul@\emph{von Paul Goldmann}!1899-12-061@{6. 12. [1899]}|(be}
\toendnotes[C]{\smallbreak\pagebreak[2]}
\correspDesc{Versand  durch Paul Goldmann am 6. 12. [1899] in Frankfurt am Main
\newline{}Erhalt  durch Arthur Schnitzler im Zeitraum [7. 12. 1899
                  – 11. 12. 1899?] in Wien}\toendnotes[C]{\smallbreak}
\Standort{DLA, A:Schnitzler, HS.NZ85.1.3169.}
\physDesc{Brief, 1 Blatt, 4 Seiten, 1206 Zeichen
\newline{}Handschrift: blaue Tinte, deutsche Kurrent
\newline{}Schnitzler: 1) mit Bleistift das Jahr »99« vermerkt  2) mit rotem Buntstift fünf Unterstreichungen}\toendnotes[C]{\smallbreak}
\pstart
           \centering{}{\pb}Frankfurt\oindex{Frankfurt am Main@\textbf{Frankfurt am Main}, \emph{Hauptstadt}|pw}, 6. Dezember.\pend
           
\pstart\center{}Mein lieber Freund,\pend\vspace{0.5em}
\pstart
           Eine kleine Anfrage, die ich Dich aber bitten muß,{ }ſtreng vertraulich zu behandeln.
               Die »Frankfurter Zeitung\orgindex{Frankfurter Zeitung@Frankfurter Zeitung|pw}«{ }ſucht einen zweiten
               Feuilleton-Redakteur, eine Hilfskraft für \textsc{Dr. Mamroth\pwindex{Mamroth, Fedor 21.\,2.\,1851 Breslau – 25.\,6.\,1907 Frankfurt am Main@\textsc{Mamroth, Fedor} (21.\,2.\,1851 Breslau – 25.\,6.\,1907 Frankfurt am Main), \emph{Journalist, Kritiker}|pw}}; eventuell könnte der Betreffende zugleich das Muſik-Referat übernehmen. Weißt
               Du Jemanden, einen jüngeren oder älteren Mann, der geeignet wäre? Was iſt
               beiſpielsweiſe mit {\pb}\label{K_L02897-1v}\edtext{\textsc{Alfred Gold\pwindex{Gold, Alfred 28.\,6.\,1874 Wien – 24.\,10.\,1958 New York City@\textsc{Gold, Alfred} (28.\,6.\,1874 Wien – 24.\,10.\,1958 New York City), \emph{Schriftsteller, Journalist, Kunsthändler}|pw}}}{\lemma{\textnormal{\emph{Alfred Gold}}}\Cendnote{\textnormal{Alfred Gold\pwindex{Gold, Alfred 28.\,6.\,1874 Wien – 24.\,10.\,1958 New York City@\textsc{Gold, Alfred} (28.\,6.\,1874 Wien – 24.\,10.\,1958 New York City), \emph{Schriftsteller, Journalist, Kunsthändler}|pwk} wurde 1901{ }Berlin\oindex{Berlin@\textbf{Berlin}, \emph{Hauptstadt}|pwk}er Feuilletonkorrespondent der \emph{Frankfurter Zeitung}\orgindex{Frankfurter Zeitung@Frankfurter Zeitung|pwk}. Siehe auch XXXX Auszeichnungsfehler: Dokument L03536 nicht gefunden.}}}\label{K_L02897-1}?\pend
           
\pstart
           Weiter, gleichfalls vertraulich: \textsc{Wassermann\pwindex{Wassermann, Jakob 10.\,3.\,1873 Fürth – 1.\,1.\,1934 Altaussee@\textsc{Wassermann, Jakob} (10.\,3.\,1873 Fürth – 1.\,1.\,1934 Altaussee), \emph{Schriftsteller}|pw}} iſt nicht mehr zu halten. Er hat die Berichterſtattung\orgindex{Frankfurter Zeitung@Frankfurter Zeitung|pwv} gar zu gewiſſenlos geführt. Man wird ihm am
                  1. Januar kündigen. Ich habe bereits Alles gethan,
               um \textsc{Schwarzkopf\pwindex{Schwarzkopf, Gustav 7.\,11.\,1853 Wien – 13.\,11.\,1939 ebd.@\textsc{Schwarzkopf, Gustav} (7.\,11.\,1853 Wien – 13.\,11.\,1939 ebd.), \emph{Schriftsteller}|pw}} die Stelle zu verſchaffen. Mein Onkel\pwindex{Mamroth, Fedor 21.\,2.\,1851 Breslau – 25.\,6.\,1907 Frankfurt am Main@\textsc{Mamroth, Fedor} (21.\,2.\,1851 Breslau – 25.\,6.\,1907 Frankfurt am Main), \emph{Journalist, Kritiker}|pwv} iſt einverſtanden, und wenn mir die \label{K_L02897-2v}\edtext{\begin{otherlanguage}{french}Canaille\end{otherlanguage}}{\lemma{\textnormal{\emph{Canaille}}}\Cendnote{\textnormal{französisch: Schurkin; siehe auch XXXX Auszeichnungsfehler: Dokument L02792 nicht gefunden und XXXX Auszeichnungsfehler: Dokument L02893 nicht gefunden.
               }}}\label{K_L02897-2},{ }ſeine Frau\pwindex{Mamroth, Johanna 19.\,5.\,1872 Frankfurt am Main – 12.\,9.\,1910@\textsc{Mamroth, Johanna} (19.\,5.\,1872 Frankfurt am Main – 12.\,9.\,1910)|pwv}, nicht
               dazwiſchen hetzt, wird es wohl werden. \strikeout{M\textcolor{gray}{irt}} Mir hätte, offen geſtanden, \label{K_L02897-111v}\edtext{\textsc{Hirschfeld\pwindex{Hirschfeld, Robert 17.\,9.\,1857 Žďár nad Sázavou – 2.\,4.\,1914 Salzburg@\textsc{Hirschfeld, Robert} (17.\,9.\,1857 Žďár nad Sázavou – 2.\,4.\,1914 Salzburg), \emph{Journalist, Musikkritiker}|pw}}{ }{\pb}näher gelegen}{\lemma{\textnormal{\emph{Hirschfeld näher gelegen}}}\Cendnote{\textnormal{Während hier Schnitzler{ }Schwarzkopf\pwindex{Schwarzkopf, Gustav 7.\,11.\,1853 Wien – 13.\,11.\,1939 ebd.@\textsc{Schwarzkopf, Gustav} (7.\,11.\,1853 Wien – 13.\,11.\,1939 ebd.), \emph{Schriftsteller}|pwk} zu präferieren scheint, erinnerte sich Salten\pwindex{Salten, Felix 6.\,9.\,1869 Budapest – 8.\,10.\,1945 Zürich@\textsc{Salten, Felix} (6.\,9.\,1869 Budapest – 8.\,10.\,1945 Zürich), \emph{Schriftsteller, Journalist, Chefredakteur}|pwk} anders
                  an diese Vorgänge: »Er hatte ja
                     in einigen kritischen Momenten, als mein Freund versagt. 
                     Das erste Mal, als er in voller Kenntnis meiner prekären materiellen Lage von der Frankfurter Zeitung\orgindex{Frankfurter Zeitung@Frankfurter Zeitung|pw} nach einem Wien\oindex{Wien@\textbf{Wien}, \emph{Verwaltungsgebiet}|pw}er
                     Theaterreferenten gefragt, den überreich mit Stellungen und Korrespondenzen versehenen Dr. Robert Hirschfeld\pwindex{Hirschfeld, Robert 17.\,9.\,1857 Žďár nad Sázavou – 2.\,4.\,1914 Salzburg@\textsc{Hirschfeld, Robert} (17.\,9.\,1857 Žďár nad Sázavou – 2.\,4.\,1914 Salzburg), \emph{Journalist, Musikkritiker}|pw} namhaft machte
                     und als ich ihn fragte, ob denn sein Urteil ich sei der beste,
                     der einzige Wien\oindex{Wien@\textbf{Wien}, \emph{Verwaltungsgebiet}|pw}er Kritiker eine blosse Höflichkeit wäre, sich
                     an die Stirne schlug und ausrief: ›An Sie habe ich vergessen!‹« (\emph{Wienbibliothek im Rathaus}, Nachlass Salten, ZPH 1681/1
                        1.1.1.9.1, [S. 6], vgl. [S. 52], ähnlich [S. 6].) }}}\label{K_L02897-111}. Aber Dir zuliebe{ }ſoll es \textsc{Schwarzkopf\pwindex{Schwarzkopf, Gustav 7.\,11.\,1853 Wien – 13.\,11.\,1939 ebd.@\textsc{Schwarzkopf, Gustav} (7.\,11.\,1853 Wien – 13.\,11.\,1939 ebd.), \emph{Schriftsteller}|pw}}{ }ſein – wenn eben nichts Unvorhergeſehenes dazwiſchen kommt.\pend
           
\pstart
           Viele treue Grüße! {\\[\baselineskip]}Dein {\\[\baselineskip]}\spacefill\mbox{Paul Goldmann}\pend
           \leftskip=0em{}
\pstart
           \noindent{}Was macht \label{K_L02897-3v}\edtext{\textsc{Richards\pwindex{Beer-Hofmann, Richard 11.\,7.\,1866 Wien – 26.\,9.\,1945 New York City@\textsc{Beer-Hofmann, Richard} (11.\,7.\,1866 Wien – 26.\,9.\,1945 New York City), \emph{Schriftsteller}|pw}}{ }Drama\pwindex{Beer-Hofmann, Richard 11.\,7.\,1866 Wien – 26.\,9.\,1945 New York City@\textsc{Beer-Hofmann, Richard} (11.\,7.\,1866 Wien – 26.\,9.\,1945 New York City), \emph{Schriftsteller}!Graf von Charolais. Ein Trauerspiel@\strich\emph{Der Graf von Charolais. Ein Trauerspiel}|pwv}}{\lemma{\textnormal{\emph{Richards Drama}}}\Cendnote{\textnormal{Seit dem Sommer arbeitete Beer-Hofmann\pwindex{Beer-Hofmann, Richard 11.\,7.\,1866 Wien – 26.\,9.\,1945 New York City@\textsc{Beer-Hofmann, Richard} (11.\,7.\,1866 Wien – 26.\,9.\,1945 New York City), \emph{Schriftsteller}|pwk} an dem Trauerspiel \emph{Der Graf von Charolais}\pwindex{Beer-Hofmann, Richard 11.\,7.\,1866 Wien – 26.\,9.\,1945 New York City@\textsc{Beer-Hofmann, Richard} (11.\,7.\,1866 Wien – 26.\,9.\,1945 New York City), \emph{Schriftsteller}!Graf von Charolais. Ein Trauerspiel@\strich\emph{Der Graf von Charolais. Ein Trauerspiel}|pwk} (vgl. XXXX Auszeichnungsfehler: Dokument L00965 nicht gefunden).}}}\label{K_L02897-3}? Und was
                     \label{K_L02897-4v}\edtext{dasjenige\pwindex{Hofmannsthal, Hugo von 1.\,2.\,1874 Wien – 15.\,7.\,1929 Rodaun@\textsc{Hofmannsthal, Hugo von} (1.\,2.\,1874 Wien – 15.\,7.\,1929 Rodaun), \emph{Schriftsteller}!Bergwerk zu Falun@\strich\emph{Das Bergwerk zu Falun}|pwuv} von
                     \textsc{Hoffmannsthal\pwindex{Hofmannsthal, Hugo von 1.\,2.\,1874 Wien – 15.\,7.\,1929 Rodaun@\textsc{Hofmannsthal, Hugo von} (1.\,2.\,1874 Wien – 15.\,7.\,1929 Rodaun), \emph{Schriftsteller}|pw}}}{\lemma{\textnormal{\emph{dasjenige von Hoffmannsthal}}}\Cendnote{\textnormal{Es dürfte das eine Bezugnahme auf \emph{Das Bergwerk zu Falun}\pwindex{Hofmannsthal, Hugo von 1.\,2.\,1874 Wien – 15.\,7.\,1929 Rodaun@\textsc{Hofmannsthal, Hugo von} (1.\,2.\,1874 Wien – 15.\,7.\,1929 Rodaun), \emph{Schriftsteller}!Bergwerk zu Falun@\strich\emph{Das Bergwerk zu Falun}|pwk} darstellen, das Hugo von Hofmannsthal\pwindex{Hofmannsthal, Hugo von 1.\,2.\,1874 Wien – 15.\,7.\,1929 Rodaun@\textsc{Hofmannsthal, Hugo von} (1.\,2.\,1874 Wien – 15.\,7.\,1929 Rodaun), \emph{Schriftsteller}|pwk} am 29. 10. 1899 bei Beer-Hofmann\pwindex{Beer-Hofmann, Richard 11.\,7.\,1866 Wien – 26.\,9.\,1945 New York City@\textsc{Beer-Hofmann, Richard} (11.\,7.\,1866 Wien – 26.\,9.\,1945 New York City), \emph{Schriftsteller}|pwk} in Anwesenheit Schnitzlers vorlas.}}}\label{K_L02897-4}? Letzterer hat
                  mir vor einigen Wochen {\pb}ſein \label{K_L02897-5v}\edtext{Buch\pwindex{Hofmannsthal, Hugo von 1.\,2.\,1874 Wien – 15.\,7.\,1929 Rodaun@\textsc{Hofmannsthal, Hugo von} (1.\,2.\,1874 Wien – 15.\,7.\,1929 Rodaun), \emph{Schriftsteller}!Frau im Fenster. Die Hochzeit der Sobeide. Der Abenteurer und die Sängerin. Theater in Versen@\strich\emph{Die Frau im Fenster. Die Hochzeit der Sobeide. Der Abenteurer und die Sängerin. Theater in Versen}|pwv}}{\lemma{\textnormal{\emph{Buch}}}\Cendnote{\textnormal{\emph{Die Frau im Fenster. Die Hochzeit der Sobeide.
                        Der Abenteurer und die Sängerin. Theater in Versen}\pwindex{Hofmannsthal, Hugo von 1.\,2.\,1874 Wien – 15.\,7.\,1929 Rodaun@\textsc{Hofmannsthal, Hugo von} (1.\,2.\,1874 Wien – 15.\,7.\,1929 Rodaun), \emph{Schriftsteller}!Frau im Fenster. Die Hochzeit der Sobeide. Der Abenteurer und die Sängerin. Theater in Versen@\strich\emph{Die Frau im Fenster. Die Hochzeit der Sobeide. Der Abenteurer und die Sängerin. Theater in Versen}|pwk} war
                     bereits im April 1899 erschienen, aber teilweise versandte es Hofmannsthal\pwindex{Hofmannsthal, Hugo von 1.\,2.\,1874 Wien – 15.\,7.\,1929 Rodaun@\textsc{Hofmannsthal, Hugo von} (1.\,2.\,1874 Wien – 15.\,7.\,1929 Rodaun), \emph{Schriftsteller}|pwk} erst gegen
                     Jahresende.}}}\label{K_L02897-5} geſchickt mit einer Widmung: »in herzlicher Sympathie«. Ich
                  hatte Luſt, ihm meines\pwindex{Goldmann, Paul 31.\,1.\,1865 Breslau – 25.\,9.\,1935 Wien@\textsc{Goldmann, Paul} (31.\,1.\,1865 Breslau – 25.\,9.\,1935 Wien), \emph{Schriftsteller, Journalist}!Sommer in China. Reisebilder@\strich\emph{Ein Sommer in China. Reisebilder}|pwv}
                  zurückzuſchicken mit der Widmung: »in{ }ſympathiſcher Herzlichkeit« – habe es aber
                  unterlaſſen.\pend
           \selectlanguage{ngerman}\endnumbering\briefempfaengerindex{Schnitzler, Arthur@\textsc{Schnitzler, Arthur}!zzzGoldmann, Paul@\emph{von Paul Goldmann}!1899-12-061@{6. 12. [1899]}|)be}\mylabel{L02897h}  \newcommand{\dateiname}{L02897}\newcommand{\titel}{Paul Goldmann an Arthur Schnitzler, 6. 12. [1899]}\newcommand{\editorInnen}{Martin Anton Müller und Laura Untner}%% latex-leseansicht-abspann.tex
%% Abspann für die Leseansicht.
%% Der Schalter \ifkorrekturansicht ist bereits durch den Vorspann gesetzt.

%% latex-abspann.tex
%% Gemeinsamer Abspann für Korrekturansicht und Leseansicht.
%% Setzt den Schalter \ifkorrekturansicht voraus (gesetzt in den
%% einbindenden Dateien latex-korrekturansicht-abspann.tex bzw.
%% latex-leseansicht-abspann.tex).
%% ---------------------------------------------------------------

\normalsize

% Das esempio-Environment wird nur in der Leseansicht benötigt
\ifkorrekturansicht\else
\newenvironment{esempio}[3]%
{
    \vspace{1.5ex}
    \rlap{\underline{#1}}
    \par
    \setlength{\parindent}{0cm}
    \nopagebreak
    \leftskip=#2cm
    \rightskip=#3cm
}
{
    \par
}
\fi

\doendnotes{C}
\bigskip
\vfill

\clearpage

\footnotesize

\ifkorrekturansicht
  \lohead{\textsc{register}}
\fi

% theindex-Environment neu definieren ohne reledmac
\makeatletter
\renewenvironment{theindex}{%
  \ifkorrekturansicht
    \section*{\indexname}%
  \else
    \subsubsection*{Index der erwähnten Entitäten}%
  \fi
  \setlength{\parindent}{0pt}%
  \setlength{\parskip}{0pt plus 0.3pt}%
  \let\item\@idxitem
}{%
  \ifkorrekturansicht\clearpage\fi
}
\makeatother

\IfFileExists{\jobname-pw.ind}{\input{\jobname-pw.ind}}{}

% Quellenangabe nur in der Leseansicht
\ifkorrekturansicht\else
% Fallback-Definitionen, falls die .tex-Datei \titel etc. nicht gesetzt hat
\providecommand{\titel}{}
\providecommand{\editorInnen}{}
\providecommand{\dateiname}{\jobname}

\vspace{3cm}

\vfill

\footnotesize
\textsc{Quelle}: \titel. Herausgegeben von {\editorInnen}. In: \emph{Arthur Schnitzler: Briefwechsel mit Autorinnen und Autoren}.
 Digitale Edition, https://schnitzler-briefe.acdh.oeaw.ac.at/{\dateiname}.html (Stand \today)
\fi

\end{document}


