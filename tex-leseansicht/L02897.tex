%% latex-leseansicht-vorspann.tex
%% Vorspann für die Leseansicht.
%% Lädt die gemeinsame Datei latex-vorspann.tex mit nicht gesetztem Schalter.

\newif\ifkorrekturansicht
\korrekturansichtfalse

\input{../tex-inputs/latex-vorspann}


         
         \renewcommand{\erwaehntePersonen}{Personen: Richard Beer-Hofmann, Alfred Gold, Robert Hirschfeld, Hugo von Hofmannsthal, Fedor Mamroth, Johanna Mamroth, Gustav Schwarzkopf, Jakob Wassermann}
         \renewcommand{\erwaehnteInstitutionen}{Institutionen: Frankfurter Zeitung}
         \renewcommand{\erwaehnteOrte}{Orte: Berlin, Frankfurt am Main, Wien}
         \renewcommand{\erwaehnteWerke}{Werke: Das Bergwerk zu Falun, Der Graf von Charolais. Ein Trauerspiel, Die Frau im Fenster. Die Hochzeit der Sobeide. Der Abenteurer und die Sängerin. Theater in Versen, Ein Sommer in China. Reisebilder}
               \section[ Paul Goldmann an Arthur Schnitzler, 6. 12. {[}1899{]}]{ Paul Goldmann an Arthur Schnitzler, 6. 12. {[}1899{]}}\nopagebreak\mylabel{v}\rehead{ }\begin{ledgroupsized}[t]{13cm}\normalsize\beginnumbering \toendnotes[C]{\smallbreak\pagebreak[2]} \Standort{DLA, A:Schnitzler, HS.NZ85.1.3169.}
\physDesc{Brief, 1 Blatt, 4 Seiten, 1206 Zeichen
\newline{}Handschrift: blaue Tinte, deutsche Kurrent
\newline{}Schnitzler: 1) mit Bleistift das Jahr »99« vermerkt  2) mit rotem Buntstift fünf Unterstreichungen}\toendnotes[C]{\smallbreak}\pstart
           \centering{}{\pb}Frankfurt\oindex{Frankfurt am Main@\textbf{Frankfurt am Main}|pw}, 6. Dezember.\pend
           \pstart\center{}Mein lieber Freund,\pend\pstart
           Eine kleine Anfrage, die ich Dich aber bitten muß, ſtreng vertraulich zu behandeln.
               Die »Frankfurter Zeitung\orgindex{Frankfurter Zeitung@Frankfurter Zeitung|pw}« ſucht einen zweiten
               Feuilleton-Redakteur, eine Hilfskraft für \textsc{Dr. Mamroth\pwindex{Mamroth, Fedor 21.02.1851 – 25.06.1907@\textsc{Mamroth, Fedor} (21.02.1851 – 25.06.1907), \emph{Journalist, Kritiker}|pw}}; eventuell könnte der Betreffende zugleich das Muſik-Referat übernehmen. Weißt
               Du Jemanden, einen jüngeren oder älteren Mann, der geeignet wäre? Was iſt
               beiſpielsweiſe mit {\pb}\label{K_L02897-1v}\edtext{\textsc{Alfred Gold\pwindex{Gold, Alfred 28.06.1874 – 24.10.1958@\textsc{Gold, Alfred} (28.06.1874 – 24.10.1958), \emph{Schriftsteller, Journalist, Händler}|pw}}}{\lemma{\textnormal{\emph{Alfred Gold}}}\Cendnote{\textnormal{Alfred Gold\pwindex{Gold, Alfred 28.06.1874 – 24.10.1958@\textsc{Gold, Alfred} (28.06.1874 – 24.10.1958), \emph{Schriftsteller, Journalist, Händler}|pwk} wurde 1901{ }Berlin\oindex{Berlin@\textbf{Berlin}|pwk}er Feuilletonkorrespondent der \emph{Frankfurter Zeitung}\orgindex{Frankfurter Zeitung@Frankfurter Zeitung|pwk}. Siehe auch Paul Goldmann an Olga und Elisabeth Gussmann, 10. 12. [1901].}}}\label{K_L02897-1h}?\pend
           \pstart
           Weiter, gleichfalls vertraulich: \textsc{Wassermann\pwindex{Wassermann, Jakob 10.03.1873 – 01.01.1934@\textsc{Wassermann, Jakob} (10.03.1873 – 01.01.1934), \emph{Schriftsteller}|pw}} iſt nicht mehr zu halten. Er hat die Berichterſtattung\orgindex{Frankfurter Zeitung@Frankfurter Zeitung|pwv} gar zu gewiſſenlos geführt. Man wird ihm am
                  1. Januar kündigen. Ich habe bereits Alles gethan,
               um \textsc{Schwarzkopf\pwindex{Schwarzkopf, Gustav 07.11.1853 – 13.11.1939@\textsc{Schwarzkopf, Gustav} (07.11.1853 – 13.11.1939), \emph{Schriftsteller}|pw}} die Stelle zu verſchaffen. Mein Onkel\pwindex{Mamroth, Fedor 21.02.1851 – 25.06.1907@\textsc{Mamroth, Fedor} (21.02.1851 – 25.06.1907), \emph{Journalist, Kritiker}|pwv} iſt einverſtanden, und wenn mir die \label{K_L02897-2v}\edtext{\begin{otherlanguage}{french}Canaille\end{otherlanguage}}{\lemma{\textnormal{\emph{Canaille}}}\Cendnote{\textnormal{französisch: Schurkin; siehe auch Paul Goldmann an Arthur Schnitzler, 2. [1.? 1897] und Paul Goldmann an Arthur Schnitzler, 12. 11. [1899]}}}\label{K_L02897-2h}, ſeine Frau\pwindex{Mamroth, Johanna 1872-05-19 – 1910-09-12@\textsc{Mamroth, Johanna} (1872-05-19 – 1910-09-12)|pwv}, nicht
               dazwiſchen hetzt, wird es wohl werden. \strikeout{M\textcolor{gray}{irt}} Mir hätte, offen geſtanden, \label{K_L02897-111v}\edtext{\textsc{Hirschfeld\pwindex{Hirschfeld, Robert 17.09.1857 – 02.04.1914@\textsc{Hirschfeld, Robert} (17.09.1857 – 02.04.1914), \emph{Journalist, Kritiker}|pw}}{ }{\pb}näher gelegen}{\lemma{\textnormal{\emph{Hirschfeld näher gelegen}}}\Cendnote{\textnormal{Während hier Schnitzler\pwindex{Schnitzler, Arthur 15.05.1862 – 21.10.1931@\textsc{Schnitzler, Arthur} (15.05.1862 – 21.10.1931), \emph{Schriftsteller, Mediziner}|pwk}Schwarzkopf\pwindex{Schwarzkopf, Gustav 07.11.1853 – 13.11.1939@\textsc{Schwarzkopf, Gustav} (07.11.1853 – 13.11.1939), \emph{Schriftsteller}|pwk} zu präferieren scheint, erinnerte sich Salten\pwindex{\textcolor{red}{\textsuperscript{XXXX1 indx}}|pwk} anders
                  an diese Vorgänge: »Er hatte ja
                     in einigen kritischen Momenten, als mein Freund versagt. 
                     Das erste Mal, als er in voller Kenntnis meiner prekäten materiellen Lage von der Frankfurter Zeitung\textcolor{red}{\textsuperscript{\textbf{KEY}}} nach einem Wien\oindex{Wien@\textbf{Wien}|pw}er
                     Theaterreferenten gefragt, den überreich mit Stellungen und Korrespondenzen versehenen Dr. Robert Hirschfeld\pwindex{Hirschfeld, Robert 17.09.1857 – 02.04.1914@\textsc{Hirschfeld, Robert} (17.09.1857 – 02.04.1914), \emph{Journalist, Kritiker}|pw} namhaft machte
                     und als ich ihn fragte, ob denn sein Urteil ich sei der beste,
                     der einzige Wien\oindex{Wien@\textbf{Wien}|pw}er Kritiker eine blosse Höflichkeit wäre, sich
                     an die Stirne schlug und ausrief: ›An Sie habe ich vergessen!‹« (\emph{Wienbibliothek im Rathaus}, Nachlass Salten, ZPH 1681/1
                        1.1.1.9.1, [S. 6], vgl. [S. 52], ähnlich [S. 6].) }}}\label{K_L02897-111h}. Aber Dir zuliebe ſoll es \textsc{Schwarzkopf\pwindex{Schwarzkopf, Gustav 07.11.1853 – 13.11.1939@\textsc{Schwarzkopf, Gustav} (07.11.1853 – 13.11.1939), \emph{Schriftsteller}|pw}} ſein – wenn eben nichts Unvorhergeſehenes dazwiſchen kommt.\pend
           \pstart
           Viele treue Grüße! {\\[\baselineskip]}Dein {\\[\baselineskip]}\spacefill\mbox{Paul Goldmann}\pend
           \leftskip=0em{}\pstart
           \noindent{}Was macht \label{K_L02897-3v}\edtext{\textsc{Richard\pwindex{Beer-Hofmann, Richard 1866-07-11 – 1945-09-26@\textsc{Beer-Hofmann, Richard} (1866-07-11 – 1945-09-26), \emph{Schriftsteller}|pw}s}{ }Drama\pwindex{Beer-Hofmann, Richard 1866-07-11 – 1945-09-26@\textsc{Beer-Hofmann, Richard} (1866-07-11 – 1945-09-26), \emph{Schriftsteller}!Graf von Charolais. Ein Trauerspiel1904-12-23@\strich\emph{Der Graf von Charolais. Ein Trauerspiel} {[}1904-12-23{]}|pwv}}{\lemma{\textnormal{\emph{Richards Drama}}}\Cendnote{\textnormal{Seit dem Sommer arbeitete Beer-Hofmann\pwindex{Beer-Hofmann, Richard 1866-07-11 – 1945-09-26@\textsc{Beer-Hofmann, Richard} (1866-07-11 – 1945-09-26), \emph{Schriftsteller}|pwk} an dem Trauerspiel \emph{Der Graf von Charolais}\pwindex{Beer-Hofmann, Richard 1866-07-11 – 1945-09-26@\textsc{Beer-Hofmann, Richard} (1866-07-11 – 1945-09-26), \emph{Schriftsteller}!Graf von Charolais. Ein Trauerspiel1904-12-23@\strich\emph{Der Graf von Charolais. Ein Trauerspiel} {[}1904-12-23{]}|pwk} (vgl. Richard Beer-Hofmann an Arthur Schnitzler, 28. 8. 1899).}}}\label{K_L02897-3h}? Und was
                     \label{K_L02897-4v}\edtext{dasjenige\pwindex{Hofmannsthal, Hugo von 1874-02-01 – 1929-07-15@\textsc{Hofmannsthal, Hugo von} (1874-02-01 – 1929-07-15), \emph{Schriftsteller}!Bergwerk zu Falun1900 – 1933@\strich\emph{Das Bergwerk zu Falun} {[}1900 – 1933{]}|pwuv} von
                     \textsc{Hoffmannsthal\pwindex{Hofmannsthal, Hugo von 1874-02-01 – 1929-07-15@\textsc{Hofmannsthal, Hugo von} (1874-02-01 – 1929-07-15), \emph{Schriftsteller}|pw}}}{\lemma{\textnormal{\emph{dasjenige von Hoffmannsthal}}}\Cendnote{\textnormal{Bezug auf \emph{Das Bergwerk zu Falun}\pwindex{Hofmannsthal, Hugo von 1874-02-01 – 1929-07-15@\textsc{Hofmannsthal, Hugo von} (1874-02-01 – 1929-07-15), \emph{Schriftsteller}!Bergwerk zu Falun1900 – 1933@\strich\emph{Das Bergwerk zu Falun} {[}1900 – 1933{]}|pwk}, das Hugo von Hofmannsthal\pwindex{Hofmannsthal, Hugo von 1874-02-01 – 1929-07-15@\textsc{Hofmannsthal, Hugo von} (1874-02-01 – 1929-07-15), \emph{Schriftsteller}|pwk} am 29. 10. 1899 bei Beer-Hofmann\pwindex{Beer-Hofmann, Richard 1866-07-11 – 1945-09-26@\textsc{Beer-Hofmann, Richard} (1866-07-11 – 1945-09-26), \emph{Schriftsteller}|pwk} in Anwesenheit Schnitzler\pwindex{Schnitzler, Arthur 15.05.1862 – 21.10.1931@\textsc{Schnitzler, Arthur} (15.05.1862 – 21.10.1931), \emph{Schriftsteller, Mediziner}|pwk}s vorlas?}}}\label{K_L02897-4h}? Letzterer hat
                  mir vor einigen Wochen {\pb}ſein \label{K_L02897-5v}\edtext{Buch\pwindex{Hofmannsthal, Hugo von 1874-02-01 – 1929-07-15@\textsc{Hofmannsthal, Hugo von} (1874-02-01 – 1929-07-15), \emph{Schriftsteller}!Frau im Fenster. Die Hochzeit der Sobeide. Der Abenteurer und die
                  Saengerin. Theater in VersenApril 1899@\strich\emph{Die Frau im Fenster. Die Hochzeit der Sobeide. Der Abenteurer und die Sängerin. Theater in Versen} {[}April 1899{]}|pwv}}{\lemma{\textnormal{\emph{Buch}}}\Cendnote{\textnormal{\emph{Die Frau im Fenster. Die Hochzeit der Sobeide.
                        Der Abenteurer und die Sängerin. Theater in Versen}\pwindex{Hofmannsthal, Hugo von 1874-02-01 – 1929-07-15@\textsc{Hofmannsthal, Hugo von} (1874-02-01 – 1929-07-15), \emph{Schriftsteller}!Frau im Fenster. Die Hochzeit der Sobeide. Der Abenteurer und die
                  Saengerin. Theater in VersenApril 1899@\strich\emph{Die Frau im Fenster. Die Hochzeit der Sobeide. Der Abenteurer und die Sängerin. Theater in Versen} {[}April 1899{]}|pwk} erschien zwar
                     bereits im April 1899, aber teilweise versandte es Hofmannsthal\pwindex{Hofmannsthal, Hugo von 1874-02-01 – 1929-07-15@\textsc{Hofmannsthal, Hugo von} (1874-02-01 – 1929-07-15), \emph{Schriftsteller}|pwk} erst gegen
                     Jahresende.}}}\label{K_L02897-5h} geſchickt mit einer Widmung: »in herzlicher Sympathie«. Ich
                  hatte Luſt, ihm meines\pwindex{Goldmann, Paul 31.01.1865 – 25.09.1935@\textsc{Goldmann, Paul} (31.01.1865 – 25.09.1935), \emph{Schriftsteller, Journalist}!Sommer in China. Reisebilder1899-05-02@\strich\emph{Ein Sommer in China. Reisebilder} {[}1899-05-02{]}|pwv}
                  zurückzuſchicken mit der Widmung: »in ſympathiſcher Herzlichkeit« – habe es aber
                  unterlaſſen.\pend
           
         
         \endnumbering\mylabel{h}\end{ledgroupsized}  \newcommand{\dateiname}{L02897}\newcommand{\titel}{Paul Goldmann an Arthur Schnitzler, 6. 12. [1899]}\newcommand{\editorInnen}{Martin Anton Müller und Laura Untner}%% latex-leseansicht-abspann.tex
%% Abspann für die Leseansicht.
%% Der Schalter \ifkorrekturansicht ist bereits durch den Vorspann gesetzt.

%% latex-abspann.tex
%% Gemeinsamer Abspann für Korrekturansicht und Leseansicht.
%% Setzt den Schalter \ifkorrekturansicht voraus (gesetzt in den
%% einbindenden Dateien latex-korrekturansicht-abspann.tex bzw.
%% latex-leseansicht-abspann.tex).
%% ---------------------------------------------------------------

\normalsize

% Das esempio-Environment wird nur in der Leseansicht benötigt
\ifkorrekturansicht\else
\newenvironment{esempio}[3]%
{
    \vspace{1.5ex}
    \rlap{\underline{#1}}
    \par
    \setlength{\parindent}{0cm}
    \nopagebreak
    \leftskip=#2cm
    \rightskip=#3cm
}
{
    \par
}
\fi

\doendnotes{C}
\bigskip
\vfill

\clearpage

\footnotesize

\ifkorrekturansicht
  \lohead{\textsc{register}}
\fi

% theindex-Environment neu definieren ohne reledmac
\makeatletter
\renewenvironment{theindex}{%
  \ifkorrekturansicht
    \section*{\indexname}%
  \else
    \subsubsection*{Index der erwähnten Entitäten}%
  \fi
  \setlength{\parindent}{0pt}%
  \setlength{\parskip}{0pt plus 0.3pt}%
  \let\item\@idxitem
}{%
  \ifkorrekturansicht\clearpage\fi
}
\makeatother

\IfFileExists{\jobname-pw.ind}{\input{\jobname-pw.ind}}{}

% Quellenangabe nur in der Leseansicht
\ifkorrekturansicht\else
% Fallback-Definitionen, falls die .tex-Datei \titel etc. nicht gesetzt hat
\providecommand{\titel}{}
\providecommand{\editorInnen}{}
\providecommand{\dateiname}{\jobname}

\vspace{3cm}

\vfill

\footnotesize
\textsc{Quelle}: \titel. Herausgegeben von {\editorInnen}. In: \emph{Arthur Schnitzler: Briefwechsel mit Autorinnen und Autoren}.
 Digitale Edition, https://schnitzler-briefe.acdh.oeaw.ac.at/{\dateiname}.html (Stand \today)
\fi

\end{document}


      