%% latex-korrekturansicht-vorspann.tex
%% Vorspann für die Korrekturansicht.
%% Lädt die gemeinsame Datei latex-vorspann.tex mit gesetztem Schalter.

\newif\ifkorrekturansicht
\korrekturansichttrue

\input{../tex-inputs/latex-vorspann}


\section[ Paul Goldmann an Arthur Schnitzler, 6. 12. {[}1899{]}]{L02897 Paul Goldmann an Arthur Schnitzler, 6. 12. {[}1899{]}}
\nopagebreak\mylabel{L02897v}
\rehead{ }\normalsize\beginnumbering\briefempfaengerindex{Schnitzler, Arthur@\textsc{Schnitzler, Arthur}!zzzGoldmann, Paul@\emph{von Paul Goldmann}!1899-12-061@{6. 12. {[}1899{]}}|(be}
\toendnotes[C]{\smallbreak\pagebreak[2]}\Standort{DLA, A:Schnitzler, HS.NZ85.1.3169.}
\physDesc{Brief, 1 Blatt, 4 Seiten, 1206 Zeichen
\newline{}Handschrift: blaue Tinte, deutsche Kurrent
\newline{}Schnitzler: 1) mit Bleistift das Jahr »99« vermerkt  2) mit rotem Buntstift fünf Unterstreichungen}\toendnotes[C]{\smallbreak}
\pstart
           \centering{}{\pb}Frankfurt\oindex{Frankfurt am Main@\textbf{Frankfurt am Main}, \emph{P.PPLA3}|pw}, 6. Dezember.\pend
           
\pstart\center{}Mein lieber Freund,\pend\vspace{0.5em}
\pstart
           Eine kleine Anfrage, die ich Dich aber bitten muß, ſtreng vertraulich zu behandeln.
               Die »Frankfurter Zeitung\orgindex{Frankfurter Zeitung@Frankfurter Zeitung|pw}« ſucht einen zweiten
               Feuilleton-Redakteur, eine Hilfskraft für \textsc{Dr. Mamroth\pwindex{Mamroth, Fedor 21.02.1851 – 25.06.1907@\textsc{Mamroth, Fedor} (21.02.1851 – 25.06.1907), \emph{Journalist/Journalistin, Kritiker/Kritikerin}|pw}}; eventuell könnte der Betreffende zugleich das Muſik-Referat übernehmen. Weißt
               Du Jemanden, einen jüngeren oder älteren Mann, der geeignet wäre? Was iſt
               beiſpielsweiſe mit {\pb}\label{K_L02897-1v}\edtext{\textsc{Alfred Gold\pwindex{Gold, Alfred 28.06.1874 – 24.10.1958@\textsc{Gold, Alfred} (28.06.1874 – 24.10.1958), \emph{Schriftsteller/Schriftstellerin, Journalist/Journalistin, Kunsthändler/Kunsthändlerin}|pw}}}{\lemma{\textnormal{\emph{Alfred Gold}}}\Cendnote{\textnormal{Alfred Gold\pwindex{Gold, Alfred 28.06.1874 – 24.10.1958@\textsc{Gold, Alfred} (28.06.1874 – 24.10.1958), \emph{Schriftsteller/Schriftstellerin, Journalist/Journalistin, Kunsthändler/Kunsthändlerin}|pwk} wurde 1901{ }Berlin\oindex{Berlin@\textbf{Berlin}, \emph{P.PPLC}|pwk}er Feuilletonkorrespondent der \emph{Frankfurter Zeitung}\orgindex{Frankfurter Zeitung@Frankfurter Zeitung|pwk}. Siehe auch Paul Goldmann an Olga und Elisabeth Gussmann, 10. 12. [1901].}}}\label{K_L02897-1}?\pend
           
\pstart
           Weiter, gleichfalls vertraulich: \textsc{Wassermann\pwindex{Wassermann, Jakob 10.03.1873 – 01.01.1934@\textsc{Wassermann, Jakob} (10.03.1873 – 01.01.1934), \emph{Schriftsteller/Schriftstellerin}|pw}} iſt nicht mehr zu halten. Er hat die Berichterſtattung\orgindex{Frankfurter Zeitung@Frankfurter Zeitung|pwv} gar zu gewiſſenlos geführt. Man wird ihm am
                  1. Januar kündigen. Ich habe bereits Alles gethan,
               um \textsc{Schwarzkopf\pwindex{Schwarzkopf, Gustav 07.11.1853 – 13.11.1939@\textsc{Schwarzkopf, Gustav} (07.11.1853 – 13.11.1939), \emph{Schriftsteller/Schriftstellerin}|pw}} die Stelle zu verſchaffen. Mein Onkel\pwindex{Mamroth, Fedor 21.02.1851 – 25.06.1907@\textsc{Mamroth, Fedor} (21.02.1851 – 25.06.1907), \emph{Journalist/Journalistin, Kritiker/Kritikerin}|pwv} iſt einverſtanden, und wenn mir die \label{K_L02897-2v}\edtext{\begin{otherlanguage}{french}Canaille\end{otherlanguage}}{\lemma{\textnormal{\emph{Canaille}}}\Cendnote{\textnormal{französisch: Schurkin; siehe auch Paul Goldmann an Arthur Schnitzler, 2. [1.? 1897] und Paul Goldmann an Arthur Schnitzler, 12. 11. [1899].
               }}}\label{K_L02897-2}, ſeine Frau\pwindex{Mamroth, Johanna 1872-05-19 – 1910-09-12@\textsc{Mamroth, Johanna} (1872-05-19 – 1910-09-12)|pwv}, nicht
               dazwiſchen hetzt, wird es wohl werden. \strikeout{M\textcolor{gray}{irt}} Mir hätte, offen geſtanden, \label{K_L02897-111v}\edtext{\textsc{Hirschfeld\pwindex{Hirschfeld, Robert 17.09.1857 – 02.04.1914@\textsc{Hirschfeld, Robert} (17.09.1857 – 02.04.1914), \emph{Journalist/Journalistin, Musikkritiker/Musikkritikerin}|pw}}{ }{\pb}näher gelegen}{\lemma{\textnormal{\emph{Hirschfeld näher gelegen}}}\Cendnote{\textnormal{Während hier Schnitzler{ }Schwarzkopf\pwindex{Schwarzkopf, Gustav 07.11.1853 – 13.11.1939@\textsc{Schwarzkopf, Gustav} (07.11.1853 – 13.11.1939), \emph{Schriftsteller/Schriftstellerin}|pwk} zu präferieren scheint, erinnerte sich Salten\pwindex{Salten, Felix 06.09.1869 – 08.10.1945@\textsc{Salten, Felix} (06.09.1869 – 08.10.1945), \emph{Schriftsteller/Schriftstellerin, Journalist/Journalistin, Chefredakteur/Chefredakteurin}|pwk} anders
                  an diese Vorgänge: »Er hatte ja
                     in einigen kritischen Momenten, als mein Freund versagt. 
                     Das erste Mal, als er in voller Kenntnis meiner prekären materiellen Lage von der Frankfurter Zeitung\orgindex{Frankfurter Zeitung@Frankfurter Zeitung|pw} nach einem Wien\oindex{Wien@\textbf{Wien}, \emph{A.ADM2}|pw}er
                     Theaterreferenten gefragt, den überreich mit Stellungen und Korrespondenzen versehenen Dr. Robert Hirschfeld\pwindex{Hirschfeld, Robert 17.09.1857 – 02.04.1914@\textsc{Hirschfeld, Robert} (17.09.1857 – 02.04.1914), \emph{Journalist/Journalistin, Musikkritiker/Musikkritikerin}|pw} namhaft machte
                     und als ich ihn fragte, ob denn sein Urteil ich sei der beste,
                     der einzige Wien\oindex{Wien@\textbf{Wien}, \emph{A.ADM2}|pw}er Kritiker eine blosse Höflichkeit wäre, sich
                     an die Stirne schlug und ausrief: ›An Sie habe ich vergessen!‹« (\emph{Wienbibliothek im Rathaus}, Nachlass Salten, ZPH 1681/1
                        1.1.1.9.1, [S. 6], vgl. [S. 52], ähnlich [S. 6].) }}}\label{K_L02897-111}. Aber Dir zuliebe ſoll es \textsc{Schwarzkopf\pwindex{Schwarzkopf, Gustav 07.11.1853 – 13.11.1939@\textsc{Schwarzkopf, Gustav} (07.11.1853 – 13.11.1939), \emph{Schriftsteller/Schriftstellerin}|pw}} ſein – wenn eben nichts Unvorhergeſehenes dazwiſchen kommt.\pend
           
\pstart
           Viele treue Grüße! {\\[\baselineskip]}Dein {\\[\baselineskip]}\spacefill\mbox{Paul Goldmann}\pend
           \leftskip=0em{}
\pstart
           \noindent{}Was macht \label{K_L02897-3v}\edtext{\textsc{Richards\pwindex{Beer-Hofmann, Richard 1866-07-11 – 1945-09-26@\textsc{Beer-Hofmann, Richard} (1866-07-11 – 1945-09-26), \emph{Schriftsteller/Schriftstellerin}|pw}}{ }Drama\pwindex{Graf von Charolais. Ein Trauerspiel@\emph{Der Graf von Charolais. Ein Trauerspiel}|pwv}}{\lemma{\textnormal{\emph{Richards Drama}}}\Cendnote{\textnormal{Seit dem Sommer arbeitete Beer-Hofmann\pwindex{Beer-Hofmann, Richard 1866-07-11 – 1945-09-26@\textsc{Beer-Hofmann, Richard} (1866-07-11 – 1945-09-26), \emph{Schriftsteller/Schriftstellerin}|pwk} an dem Trauerspiel \emph{Der Graf von Charolais}\pwindex{Graf von Charolais. Ein Trauerspiel@\emph{Der Graf von Charolais. Ein Trauerspiel}|pwk} (vgl. Richard Beer-Hofmann an Arthur Schnitzler, 28. 8. 1899).}}}\label{K_L02897-3}? Und was
                     \label{K_L02897-4v}\edtext{dasjenige\pwindex{Bergwerk zu Falun@\emph{Das Bergwerk zu Falun}|pwuv} von
                     \textsc{Hoffmannsthal\pwindex{Hofmannsthal, Hugo von 1874-02-01 – 1929-07-15@\textsc{Hofmannsthal, Hugo von} (1874-02-01 – 1929-07-15), \emph{Schriftsteller/Schriftstellerin}|pw}}}{\lemma{\textnormal{\emph{dasjenige von Hoffmannsthal}}}\Cendnote{\textnormal{Es dürfte das eine Bezugnahme auf \emph{Das Bergwerk zu Falun}\pwindex{Bergwerk zu Falun@\emph{Das Bergwerk zu Falun}|pwk} darstellen, das Hugo von Hofmannsthal\pwindex{Hofmannsthal, Hugo von 1874-02-01 – 1929-07-15@\textsc{Hofmannsthal, Hugo von} (1874-02-01 – 1929-07-15), \emph{Schriftsteller/Schriftstellerin}|pwk} am 29. 10. 1899 bei Beer-Hofmann\pwindex{Beer-Hofmann, Richard 1866-07-11 – 1945-09-26@\textsc{Beer-Hofmann, Richard} (1866-07-11 – 1945-09-26), \emph{Schriftsteller/Schriftstellerin}|pwk} in Anwesenheit Schnitzlers vorlas.}}}\label{K_L02897-4}? Letzterer hat
                  mir vor einigen Wochen {\pb}ſein \label{K_L02897-5v}\edtext{Buch\pwindex{Frau im Fenster. Die Hochzeit der Sobeide. Der Abenteurer und die Saengerin. Theater in Versen@\emph{Die Frau im Fenster. Die Hochzeit der Sobeide. Der Abenteurer und die Sängerin. Theater in Versen}|pwv}}{\lemma{\textnormal{\emph{Buch}}}\Cendnote{\textnormal{\emph{Die Frau im Fenster. Die Hochzeit der Sobeide.
                        Der Abenteurer und die Sängerin. Theater in Versen}\pwindex{Frau im Fenster. Die Hochzeit der Sobeide. Der Abenteurer und die Saengerin. Theater in Versen@\emph{Die Frau im Fenster. Die Hochzeit der Sobeide. Der Abenteurer und die Sängerin. Theater in Versen}|pwk} war
                     bereits im April 1899 erschienen, aber teilweise versandte es Hofmannsthal\pwindex{Hofmannsthal, Hugo von 1874-02-01 – 1929-07-15@\textsc{Hofmannsthal, Hugo von} (1874-02-01 – 1929-07-15), \emph{Schriftsteller/Schriftstellerin}|pwk} erst gegen
                     Jahresende.}}}\label{K_L02897-5} geſchickt mit einer Widmung: »in herzlicher Sympathie«. Ich
                  hatte Luſt, ihm meines\pwindex{Sommer in China. Reisebilder@\emph{Ein Sommer in China. Reisebilder}|pwv}
                  zurückzuſchicken mit der Widmung: »in ſympathiſcher Herzlichkeit« – habe es aber
                  unterlaſſen.\pend
           \selectlanguage{ngerman}\endnumbering\briefempfaengerindex{Schnitzler, Arthur@\textsc{Schnitzler, Arthur}!zzzGoldmann, Paul@\emph{von Paul Goldmann}!1899-12-061@{6. 12. {[}1899{]}}|)be}\mylabel{L02897h}  \normalsize

\doendnotes{C}
\bigskip
\vfill

\clearpage

\footnotesize

\lohead{\textsc{register}}

% Definiere theindex-Environment komplett neu ohne reledmac
\makeatletter
\renewenvironment{theindex}{%
  \section*{\indexname}%
  \setlength{\parindent}{0pt}%
  \setlength{\parskip}{0pt plus 0.3pt}%
  \let\item\@idxitem
}{%
  \clearpage
}
\makeatother

\IfFileExists{\jobname-pw.ind}{\input{\jobname-pw.ind}}{}

\end{document}

      