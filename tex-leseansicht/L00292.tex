%% latex-korrekturansicht-vorspann.tex
%% Vorspann für die Korrekturansicht.
%% Lädt die gemeinsame Datei latex-vorspann.tex mit gesetztem Schalter.

\newif\ifkorrekturansicht
\korrekturansichttrue

\input{../tex-inputs/latex-vorspann}


\section[Oscar Blumenthal an Arthur Schnitzler, 16. 1. 1894]{L00292 Oscar Blumenthal an Arthur Schnitzler, 16. 1. 1894}
\nopagebreak\mylabel{L00292v}
\rehead{ }\normalsize\beginnumbering\briefempfaengerindex{Schnitzler, Arthur@\textsc{Schnitzler, Arthur}!zzzBlumenthal, Oskar@\emph{von Oskar Blumenthal}!1894-01-161@{16. 1. 1894}|(be}
\toendnotes[C]{\smallbreak\pagebreak[2]}\Standort{CUL, Schnitzler, B 15.}
\physDesc{Brief, 1 Blatt, 1 Seite, 509 Zeichen
\newline{}Handschrift Schreibkraft: schwarze Tinte, deutsche Kurrent
\newline{}Handschrift Oskar Blumenthal: schwarze Tinte, deutsche Kurrent (\noindent{}Unterschrift)
\newline{}Schnitzler: 1) mit Bleistift auf der Rückseite beschriftet: »\textsc{Blumenthal}«  2) mit rotem Buntstift eine Unterstreichung und nummeriert:
                                    »5«}\toendnotes[C]{\smallbreak}
\pstart
           \centering{}{\pb}\textcolor{gray}{\textbf{\textsc{Lessing-Theater\orgindex{Lessing-Theater@Lessing-Theater|pw}}}}\pend
           
\pstart
           \centering{}\textcolor{gray}{\textbf{\textsc{Director}:}}{\\}\textcolor{gray}{\textbf{DR. OSCAR BLUMENTHAL.}}\pend
           
\pstart
           \raggedleft{}\textcolor{gray}{\textbf{Berlin N.W.\oindex{Berlin@\textbf{Berlin}, \emph{P.PPLC}|pw}, den}}{ }16. Januar \textcolor{gray}{\textbf{189}}4.{\\}\textcolor{gray}{\textbf{Friedrich-Carl-Ufer\oindex{Kapelle-Ufer@\textbf{Kapelle-Ufer}, \emph{Straße (K.STR)}|pw}}}.\pend
           \vspace{0.5em}
\pstart
           \centering{}Werther Herr Doktor!\pend
           
\pstart
           Nach dem wenig ermuthigenden Ausgang der Wien\oindex{Wien@\textbf{Wien}, \emph{A.ADM2}|pw}er
              \label{K_L00292-1v}\edtext{Probeaufführung}{\lemma{\textnormal{\emph{Probeaufführung}}}\Cendnote{\textnormal{Die Uraufführung\eventindex{Volkstheater@\textbf{Volkstheater}!Urauffuehrung von Das Maerchen, 1.12.1893@Uraufführung von Das Märchen, 1.12.1893|pwkv} hatte am 1. 12. 1893 am \emph{Deutschen Volkstheater}\orgindex{Volkstheater@Volkstheater|pwk} stattgefunden. Bereits
                  nach zwei Vorstellungen wurde das Stück abgesetzt.}}}\label{K_L00292-1} des »Märchens\pwindex{Maerchen. Schauspiel in drei Aufzuegen@\emph{Das Märchen. Schauspiel in drei Aufzügen}|pw}« glaube ich, daß wir gut thun werden, vorläufig in Berlin\oindex{Berlin@\textbf{Berlin}, \emph{P.PPLC}|pw} von dem Stücke\pwindex{Maerchen. Schauspiel in drei Aufzuegen@\emph{Das Märchen. Schauspiel in drei Aufzügen}|pwv} abzuſehen. Sehr gerne werde ich gelegentlich einen
               Ihrer Einakter bringen; aber da es ſich hier immer darum handelt, ein begleitendes
               Werk zu finden, das für ſich allein den Abend nicht ausfüllen würde, ſo läßt ſich
               hier beim beſten Willen ein Darſtellungstermin nicht feſtſetzen.\pend
           
\pstart
           Mit beſten Grüßen Ihr ergebener{\\[\baselineskip]}\spacefill\mbox{{[}hs. :{]} Dr. Osc. Blumenthal}\pend
           \leftskip=0em{}\selectlanguage{ngerman}\endnumbering\briefempfaengerindex{Schnitzler, Arthur@\textsc{Schnitzler, Arthur}!zzzBlumenthal, Oskar@\emph{von Oskar Blumenthal}!1894-01-161@{16. 1. 1894}|)be}\mylabel{L00292h}  \normalsize

\doendnotes{C}
\bigskip
\vfill

\clearpage

\footnotesize

\lohead{\textsc{register}}

% Definiere theindex-Environment komplett neu ohne reledmac
\makeatletter
\renewenvironment{theindex}{%
  \section*{\indexname}%
  \setlength{\parindent}{0pt}%
  \setlength{\parskip}{0pt plus 0.3pt}%
  \let\item\@idxitem
}{%
  \clearpage
}
\makeatother

\IfFileExists{\jobname-pw.ind}{\input{\jobname-pw.ind}}{}

\end{document}

      