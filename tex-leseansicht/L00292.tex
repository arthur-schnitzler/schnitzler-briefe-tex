%% latex-leseansicht-vorspann.tex
%% Vorspann für die Leseansicht.
%% Lädt die gemeinsame Datei latex-vorspann.tex mit nicht gesetztem Schalter.

\newif\ifkorrekturansicht
\korrekturansichtfalse

\input{../tex-inputs/latex-vorspann}


         
         \renewcommand{\erwaehntePersonen}{Personen: Oskar Blumenthal}
         \renewcommand{\erwaehnteInstitutionen}{Institutionen: Lessing-Theater}
         \renewcommand{\erwaehnteOrte}{Orte: Berlin, Kapelle-Ufer, Volkstheater, Wien}
         \renewcommand{\erwaehnteWerke}{Werke: Das Märchen. Schauspiel in drei Aufzügen}
               \section[Oscar Blumenthal an Arthur Schnitzler, 16. 1. 1894]{ Oscar Blumenthal an Arthur Schnitzler, 16. 1. 1894}\nopagebreak\mylabel{v}\rehead{ }\begin{ledgroupsized}[t]{13cm}\normalsize\beginnumbering\briefempfaengerindex{Schnitzler, Arthur@\textsc{Schnitzler, Arthur}!zzzBlumenthal, Oskar@\emph{von Oskar Blumenthal}!1894-01-161@{16. 1. 1894}|(be} \toendnotes[C]{\smallbreak\pagebreak[2]} \Standort{CUL, Schnitzler, B 15.}
\physDesc{Brief, 1 Blatt, 1 Seite, 509 Zeichen
\newline{}Handschrift Schreibkraft: schwarze Tinte, deutsche Kurrent\newline{}Handschrift Oskar Blumenthal: schwarze Tinte, deutsche Kurrent (\noindent{}Unterschrift)
\newline{}Schnitzler: 1) mit Bleistift auf der Rückseite beschriftet: »\textsc{Blumenthal}«  2) mit rotem Buntstift eine Unterstreichung und nummeriert:
                                    »5«}\toendnotes[C]{\smallbreak}\pstart
           \noindent{}\centering{}{\pb}\textcolor{gray}{\textbf{\textsc{Lessing-Theater\orgindex{Lessing-Theater@Lessing-Theater|pw}}}}\pend
           \pstart
           \noindent{}\centering{}\textcolor{gray}{\textbf{\textsc{Director}:}}{\\}\textcolor{gray}{\textbf{DR. OSCAR BLUMENTHAL.}}\pend
           \pstart
           \noindent{}\raggedleft{}\textcolor{gray}{\textbf{Berlin N.W.\oindex{Berlin@\textbf{Berlin}|pw}, den}}{ }16. Januar \textcolor{gray}{\textbf{189}}4.{\\}\textcolor{gray}{\textbf{Friedrich-Carl-Ufer\oindex{Kapelle-Ufer@\textbf{Kapelle-Ufer}|pw}}}.\pend
           \pstart
           \centering{}Werther Herr Doktor!\pend
           \pstart
           \noindent{}Nach dem wenig ermuthigenden Ausgang der Wien\oindex{Wien@\textbf{Wien}|pw}er
               \label{K_L00292-1v}\edtext{Probeaufführung}{\lemma{\textnormal{\emph{Probeaufführung}}}\Cendnote{\textnormal{Die Uraufführung hatte am 1. 12. 1893 am \emph{Deutschen Volkstheater}XXXX ORGangabe fehlt stattgefunden. Bereits
                  nach zwei Vorstellungen wurde das Stück abgesetzt.}}}\label{K_L00292-1h} des »Märchens\pwindex{Schnitzler, Arthur 15.05.1862 – 21.10.1931@\textsc{Schnitzler, Arthur} (15.05.1862 – 21.10.1931), \emph{Schriftsteller, Mediziner}!Maerchen. Schauspiel in drei Aufzuegen1893-12-01@\strich\emph{Das Märchen. Schauspiel in drei Aufzügen} {[}1893-12-01{]}|pw}« glaube ich, daß wir gut thun werden, vorläufig in Berlin\oindex{Berlin@\textbf{Berlin}|pw} von dem Stücke\pwindex{Schnitzler, Arthur 15.05.1862 – 21.10.1931@\textsc{Schnitzler, Arthur} (15.05.1862 – 21.10.1931), \emph{Schriftsteller, Mediziner}!Maerchen. Schauspiel in drei Aufzuegen1893-12-01@\strich\emph{Das Märchen. Schauspiel in drei Aufzügen} {[}1893-12-01{]}|pwv} abzuſehen. Sehr gerne werde ich gelegentlich einen
               Ihrer Einakter bringen; aber da es ſich hier immer darum handelt, ein begleitendes
               Werk zu finden, das für ſich allein den Abend nicht ausfüllen würde, ſo läßt ſich
               hier beim beſten Willen ein Darſtellungstermin nicht feſtſetzen.\pend
           \pstart
           Mit beſten Grüßen Ihr ergebener{\\[\baselineskip]}\spacefill\mbox{{[}hs. Blumenthal:{]} Dr. Osc. Blumenthal}\pend
           \leftskip=0em{}
         
         \endnumbering\mylabel{h}\end{ledgroupsized}  \newcommand{\dateiname}{L00292}\newcommand{\titel}{Oscar Blumenthal an Arthur Schnitzler, 16. 1. 1894}\newcommand{\editorInnen}{Martin Anton Müller und Gerd-Hermann Susen}%% latex-leseansicht-abspann.tex
%% Abspann für die Leseansicht.
%% Der Schalter \ifkorrekturansicht ist bereits durch den Vorspann gesetzt.

%% latex-abspann.tex
%% Gemeinsamer Abspann für Korrekturansicht und Leseansicht.
%% Setzt den Schalter \ifkorrekturansicht voraus (gesetzt in den
%% einbindenden Dateien latex-korrekturansicht-abspann.tex bzw.
%% latex-leseansicht-abspann.tex).
%% ---------------------------------------------------------------

\normalsize

% Das esempio-Environment wird nur in der Leseansicht benötigt
\ifkorrekturansicht\else
\newenvironment{esempio}[3]%
{
    \vspace{1.5ex}
    \rlap{\underline{#1}}
    \par
    \setlength{\parindent}{0cm}
    \nopagebreak
    \leftskip=#2cm
    \rightskip=#3cm
}
{
    \par
}
\fi

\doendnotes{C}
\bigskip
\vfill

\clearpage

\footnotesize

\ifkorrekturansicht
  \lohead{\textsc{register}}
\fi

% theindex-Environment neu definieren ohne reledmac
\makeatletter
\renewenvironment{theindex}{%
  \ifkorrekturansicht
    \section*{\indexname}%
  \else
    \subsubsection*{Index der erwähnten Entitäten}%
  \fi
  \setlength{\parindent}{0pt}%
  \setlength{\parskip}{0pt plus 0.3pt}%
  \let\item\@idxitem
}{%
  \ifkorrekturansicht\clearpage\fi
}
\makeatother

\IfFileExists{\jobname-pw.ind}{\input{\jobname-pw.ind}}{}

% Quellenangabe nur in der Leseansicht
\ifkorrekturansicht\else
% Fallback-Definitionen, falls die .tex-Datei \titel etc. nicht gesetzt hat
\providecommand{\titel}{}
\providecommand{\editorInnen}{}
\providecommand{\dateiname}{\jobname}

\vspace{3cm}

\vfill

\footnotesize
\textsc{Quelle}: \titel. Herausgegeben von {\editorInnen}. In: \emph{Arthur Schnitzler: Briefwechsel mit Autorinnen und Autoren}.
 Digitale Edition, https://schnitzler-briefe.acdh.oeaw.ac.at/{\dateiname}.html (Stand \today)
\fi

\end{document}


      