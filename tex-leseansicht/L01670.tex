%% latex-korrekturansicht-vorspann.tex
%% Vorspann für die Korrekturansicht.
%% Lädt die gemeinsame Datei latex-vorspann.tex mit gesetztem Schalter.

\newif\ifkorrekturansicht
\korrekturansichttrue

\input{../tex-inputs/latex-vorspann}


\section[Hermann Bahr an Arthur Schnitzler, 26. 4. 1907]{L01670 Hermann Bahr an Arthur Schnitzler, 26. 4. 1907}
\nopagebreak\mylabel{L01670v}
\rehead{ }\normalsize\beginnumbering\briefempfaengerindex{Schnitzler, Arthur@\textsc{Schnitzler, Arthur}!zzzBahr, Hermann@\emph{von Hermann Bahr}!1907-04-261@{26. 4. 1907}|(be}
\toendnotes[C]{\smallbreak\pagebreak[2]}\Standort{CUL, Schnitzler, B 5b.}
\physDesc{Brief, 1 Blatt, 1 Seite, 347 Zeichen
\newline{}Handschrift Lisa Clarus: blaue Tinte, lateinische Kurrent
\newline{}Handschrift Hermann Bahr: schwarze Tinte (\noindent{}Unterschrift)
\newline{}Ordnung: mit Bleistift von unbekannter Hand nummeriert:
                                    »147« }
\buchAbdrucke{\weitereDrucke{Hermann Bahr, Arthur Schnitzler: \emph{Briefwechsel, Aufzeichnungen, Dokumente (1891–1931)}. Göttingen: \emph{Wallstein} 2018, S. 392.} }\toendnotes[C]{\smallbreak}
\pstart
           \raggedleft{}{\pb}26. 4. 07\pend
           
\pstart{}Lieber Arthur!\pend\vspace{0.5em}
\pstart
           Möchtest Du so lieb sein, mir auch noch den zweiten
                  Band Brehm\pwindex{Brehms Tierleben@\emph{Brehms Tierleben}|pw} zu schicken? Du kriegst dann beide zusammen in ein paar Tagen
               zurück. Ich hoffe nun in der nächsten Woche, wahrscheinlich Dienstag oder Mittwoch,
               meine \label{K_L01670-1v}\edtext{Forschungsreise}{\lemma{\textnormal{\emph{Forschungsreise}}}\Cendnote{\textnormal{Möglicherweise eine doppelte Anspielung:
                  einerseits auf Bahrs\pwindex{Bahr, Hermann 19.07.1863 – 15.01.1934@\textsc{Bahr, Hermann} (19.07.1863 – 15.01.1934), \emph{Schriftsteller/Schriftstellerin, Kritiker/Kritikerin}|pwk} Interesse für die
                  südlichen österreichischen Provinzen, andererseits auf die Suche nach einer
                  Sommervilla. Bahr\pwindex{Bahr, Hermann 19.07.1863 – 15.01.1934@\textsc{Bahr, Hermann} (19.07.1863 – 15.01.1934), \emph{Schriftsteller/Schriftstellerin, Kritiker/Kritikerin}|pwk} traf am
                     3. 5. 1907, also später als angekündigt, in Triest\oindex{Triest@\textbf{Triest}, \emph{A.ADM3}|pwk} ein und kündigte am 8. 5. 1907 die
                  Weiterfahrt nach Sistiana\oindex{Sistiana@\textbf{Sistiana}, \emph{P.PPL}|pwk} an.}}}\label{K_L01670-1} nach Fiume\oindex{Rijeka@\textbf{Rijeka}, \emph{P.PPLA}|pw} und Triest\oindex{Triest@\textbf{Triest}, \emph{A.ADM3}|pw} zu machen. Kommst Du mit?\pend
           
\pstart
           Mit den besten Grüssen an Deine Frau\pwindex{Schnitzler, Olga 17.01.1882 – 13.01.1970@\textsc{Schnitzler, Olga} (17.01.1882 – 13.01.1970), \emph{Schauspieler/Schauspielerin, Sänger/Sängerin}|pwv},{\\[\baselineskip]}herzlichst{\\[\baselineskip]}\spacefill\mbox{{[}hs. :{]} HermannB}\pend
           \leftskip=0em{}\selectlanguage{ngerman}\endnumbering\briefempfaengerindex{Schnitzler, Arthur@\textsc{Schnitzler, Arthur}!zzzBahr, Hermann@\emph{von Hermann Bahr}!1907-04-261@{26. 4. 1907}|)be}\mylabel{L01670h}  \normalsize

\doendnotes{C}
\bigskip
\vfill

\clearpage

\footnotesize

\lohead{\textsc{register}}

% Definiere theindex-Environment komplett neu ohne reledmac
\makeatletter
\renewenvironment{theindex}{%
  \section*{\indexname}%
  \setlength{\parindent}{0pt}%
  \setlength{\parskip}{0pt plus 0.3pt}%
  \let\item\@idxitem
}{%
  \clearpage
}
\makeatother

\IfFileExists{\jobname-pw.ind}{\input{\jobname-pw.ind}}{}

\end{document}

      