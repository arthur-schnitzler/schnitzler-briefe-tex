%% latex-leseansicht-vorspann.tex
%% Vorspann für die Leseansicht.
%% Lädt die gemeinsame Datei latex-vorspann.tex mit nicht gesetztem Schalter.

\newif\ifkorrekturansicht
\korrekturansichtfalse

\input{../tex-inputs/latex-vorspann}


\section[Hermann Bahr an Arthur Schnitzler, 26. 4. 1907]{L01670 Hermann Bahr an Arthur Schnitzler, 26. 4. 1907}
\nopagebreak\mylabel{L01670v}
\rehead{ }\normalsize\beginnumbering\briefempfaengerindex{Schnitzler, Arthur@\textsc{Schnitzler, Arthur}!zzzBahr, Hermann@\emph{von Hermann Bahr}!1907-04-261@{26. 4. 1907}|(be}
\toendnotes[C]{\smallbreak\pagebreak[2]}
\correspDesc{Versand  durch Hermann Bahr am 26. 4. 1907 in Wien
\newline{}Erhalt  durch Arthur Schnitzler im Zeitraum [26. 4. 1907
                  – 30. 4. 1907?] in Wien}\toendnotes[C]{\smallbreak}
\Standort{CUL, Schnitzler, B 5b.}
\physDesc{Brief, 1 Blatt, 1 Seite, 347 Zeichen
\newline{}Handschrift Lisa Clarus: blaue Tinte, lateinische Kurrent
\newline{}Handschrift Hermann Bahr: schwarze Tinte (\noindent{}Unterschrift)
\newline{}Ordnung: mit Bleistift von unbekannter Hand nummeriert:
                                    »147« }
\buchAbdrucke{\weitereDrucke{Hermann Bahr, Arthur Schnitzler: \emph{Briefwechsel, Aufzeichnungen, Dokumente (1891–1931)}. Herausgegeben von Kurt Ifkovits und Martin Anton Müller. Göttingen: \emph{Wallstein} 2018, S. 392.} }\toendnotes[C]{\smallbreak}
\pstart
           \raggedleft{}{\pb}26. 4. 07\pend
           
\pstart{}Lieber Arthur!\pend\vspace{0.5em}
\pstart
           Möchtest Du so lieb sein, mir auch noch den zweiten
                  Band Brehm\pwindex{\textcolor{red}{\textsuperscript{XXXX indx1}}!Brehms Tierleben@\strich\emph{Brehms Tierleben}|pw} zu schicken? Du kriegst dann beide zusammen in ein paar Tagen
               zurück. Ich hoffe nun in der nächsten Woche, wahrscheinlich Dienstag oder Mittwoch,
               meine \label{K_L01670-1v}\edtext{Forschungsreise}{\lemma{\textnormal{\emph{Forschungsreise}}}\Cendnote{\textnormal{Möglicherweise eine doppelte Anspielung:
                  einerseits auf Bahrs\pwindex{Bahr, Hermann 19.\,7.\,1863 Linz – 15.\,1.\,1934 München@\textsc{Bahr, Hermann} (19.\,7.\,1863 Linz – 15.\,1.\,1934 München), \emph{Schriftsteller, Kritiker}|pwk} Interesse für die
                  südlichen österreichischen Provinzen, andererseits auf die Suche nach einer
                  Sommervilla. Bahr\pwindex{Bahr, Hermann 19.\,7.\,1863 Linz – 15.\,1.\,1934 München@\textsc{Bahr, Hermann} (19.\,7.\,1863 Linz – 15.\,1.\,1934 München), \emph{Schriftsteller, Kritiker}|pwk} traf am
                     3. 5. 1907, also später als angekündigt, in Triest\oindex{Triest@\textbf{Triest}, \emph{Verwaltungsgebiet}|pwk} ein und kündigte am 8. 5. 1907 die
                  Weiterfahrt nach Sistiana\oindex{Sistiana@\textbf{Sistiana}|pwk} an.}}}\label{K_L01670-1} nach Fiume\oindex{Rijeka@\textbf{Rijeka}|pw} und Triest\oindex{Triest@\textbf{Triest}, \emph{Verwaltungsgebiet}|pw} zu machen. Kommst Du mit?\pend
           
\pstart
           Mit den besten Grüssen an Deine Frau\pwindex{Schnitzler, Olga 17.\,1.\,1882 Wien – 13.\,1.\,1970 Lugano@\textsc{Schnitzler, Olga} (17.\,1.\,1882 Wien – 13.\,1.\,1970 Lugano), \emph{Schauspielerin, Sängerin}|pwv},{\\[\baselineskip]}herzlichst{\\[\baselineskip]}\spacefill\mbox{{[}hs. Bahr:{]} HermannB}\pend
           \leftskip=0em{}\selectlanguage{ngerman}\endnumbering\briefempfaengerindex{Schnitzler, Arthur@\textsc{Schnitzler, Arthur}!zzzBahr, Hermann@\emph{von Hermann Bahr}!1907-04-261@{26. 4. 1907}|)be}\mylabel{L01670h}  \newcommand{\dateiname}{L01670}\newcommand{\titel}{Hermann Bahr an Arthur Schnitzler, 26. 4. 1907}\newcommand{\editorInnen}{Herausgegeben von Martin Anton Müller}%% latex-leseansicht-abspann.tex
%% Abspann für die Leseansicht.
%% Der Schalter \ifkorrekturansicht ist bereits durch den Vorspann gesetzt.

%% latex-abspann.tex
%% Gemeinsamer Abspann für Korrekturansicht und Leseansicht.
%% Setzt den Schalter \ifkorrekturansicht voraus (gesetzt in den
%% einbindenden Dateien latex-korrekturansicht-abspann.tex bzw.
%% latex-leseansicht-abspann.tex).
%% ---------------------------------------------------------------

\normalsize

% Das esempio-Environment wird nur in der Leseansicht benötigt
\ifkorrekturansicht\else
\newenvironment{esempio}[3]%
{
    \vspace{1.5ex}
    \rlap{\underline{#1}}
    \par
    \setlength{\parindent}{0cm}
    \nopagebreak
    \leftskip=#2cm
    \rightskip=#3cm
}
{
    \par
}
\fi

\doendnotes{C}
\bigskip
\vfill

\clearpage

\footnotesize

\ifkorrekturansicht
  \lohead{\textsc{register}}
\fi

% theindex-Environment neu definieren ohne reledmac
\makeatletter
\renewenvironment{theindex}{%
  \ifkorrekturansicht
    \section*{\indexname}%
  \else
    \subsubsection*{Index der erwähnten Entitäten}%
  \fi
  \setlength{\parindent}{0pt}%
  \setlength{\parskip}{0pt plus 0.3pt}%
  \let\item\@idxitem
}{%
  \ifkorrekturansicht\clearpage\fi
}
\makeatother

\IfFileExists{\jobname-pw.ind}{\input{\jobname-pw.ind}}{}

% Quellenangabe nur in der Leseansicht
\ifkorrekturansicht\else
% Fallback-Definitionen, falls die .tex-Datei \titel etc. nicht gesetzt hat
\providecommand{\titel}{}
\providecommand{\editorInnen}{}
\providecommand{\dateiname}{\jobname}

\vspace{3cm}

\vfill

\footnotesize
\textsc{Quelle}: \titel. Herausgegeben von {\editorInnen}. In: \emph{Arthur Schnitzler: Briefwechsel mit Autorinnen und Autoren}.
 Digitale Edition, https://schnitzler-briefe.acdh.oeaw.ac.at/{\dateiname}.html (Stand \today)
\fi

\end{document}


