%% latex-leseansicht-vorspann.tex
%% Vorspann für die Leseansicht.
%% Lädt die gemeinsame Datei latex-vorspann.tex mit nicht gesetztem Schalter.

\newif\ifkorrekturansicht
\korrekturansichtfalse

\input{../tex-inputs/latex-vorspann}


               \section[Hugo von Hofmannsthal an Arthur Schnitzler, 10. 9. 1905]{ Hugo von Hofmannsthal an Arthur Schnitzler, 10. 9. 1905}\nopagebreak\mylabel{v}\rehead{ }\begin{ledgroupsized}[t]{13cm}\normalsize\beginnumbering\briefempfaengerindex{Schnitzler, Arthur@\textsc{Schnitzler, Arthur}!zzzHofmannsthal, Hugo von@\emph{von Hugo von Hofmannsthal}!1905-09-101@{10. 9. 1905}|(be} \toendnotes[C]{\smallbreak\pagebreak[2]} \buchAlsQuelle{Hugo von Hofmannsthal, Arthur Schnitzler: \emph{Briefwechsel}. Hg. Therese Nickl und Heinrich Schnitzler. Frankfurt am Main: \emph{S. Fischer} 1964, S. 214.}\toendnotes[C]{\smallbreak}\pstart
           \noindent{}{\pb}{[}Telegramm{]}\hfill \label{K_L01544_1v}\edtext{{[}Misurina\oindex{Misurina@\textbf{Misurina}|pw}, 10. September 1905{]}}{\lemma{\textnormal{\emph{Misurina, … 1905}}}\Cendnote{\textnormal{Diese Angabe dürfte falsch sein,
                        da es eine andere Datierung erforderlich machen würde; anzunehmen ist Lueg\oindex{Lueg am Wolfgangsee@\textbf{Lueg am Wolfgangsee}|pwk}.}}}\label{K_L01544_1h}\pend
           \pstart
           \label{K_L01544_2v}\edtext{Große Freude}{\lemma{\textnormal{\emph{Große Freude}}}\Cendnote{\textnormal{Am 9. 9. 1905 meldeten die Zeitungen, dass mit der Annahme von \emph{Zwischenspiel}\pwindex{Schnitzler, Arthur 15.05.1862 – 21.10.1931@\textsc{Schnitzler, Arthur} (15.05.1862 – 21.10.1931), \emph{Schriftsteller, Mediziner}!Zwischenspiel. Komoedie in drei Akten1905-10-12 – 1905-10-12@\strich\emph{Zwischenspiel. Komödie in drei Akten} {[}1905-10-12 – 1905-10-12{]}|pwk} am \emph{Burgtheater}\orgindex{Burgtheater@Burgtheater|pwk} erstmals seit einigen Jahren wieder ein Stück von Jung-Wiener
                  Autoren an einer Wien\oindex{Wien@\textbf{Wien}|pwk}er Bühne aufgeführt werden
                  würde.}}}\label{K_L01544_2h} über Burgtheater\orgindex{Burgtheater@Burgtheater|pw} erbitte paar
               Zeilen näheres Ich arbeite sehr Kommt Ihr nicht doch noch her\hspace*{1.5em}Herrliches Wetter gutes Essen \pend
           \pstart \spacefill\mbox{Hugo}\pend{}\endnumbering\briefempfaengerindex{Schnitzler, Arthur@\textsc{Schnitzler, Arthur}!zzzHofmannsthal, Hugo von@\emph{von Hugo von Hofmannsthal}!1905-09-101@{10. 9. 1905}|)be}\mylabel{h}\end{ledgroupsized}  \newcommand{\dateiname}{L01544}\newcommand{\titel}{Hugo von Hofmannsthal an Arthur Schnitzler, 10. 9. 1905}\newcommand{\editorInnen}{Martin Anton Müller und Gerd-Hermann Susen}%% latex-leseansicht-abspann.tex
%% Abspann für die Leseansicht.
%% Der Schalter \ifkorrekturansicht ist bereits durch den Vorspann gesetzt.

%% latex-abspann.tex
%% Gemeinsamer Abspann für Korrekturansicht und Leseansicht.
%% Setzt den Schalter \ifkorrekturansicht voraus (gesetzt in den
%% einbindenden Dateien latex-korrekturansicht-abspann.tex bzw.
%% latex-leseansicht-abspann.tex).
%% ---------------------------------------------------------------

\normalsize

% Das esempio-Environment wird nur in der Leseansicht benötigt
\ifkorrekturansicht\else
\newenvironment{esempio}[3]%
{
    \vspace{1.5ex}
    \rlap{\underline{#1}}
    \par
    \setlength{\parindent}{0cm}
    \nopagebreak
    \leftskip=#2cm
    \rightskip=#3cm
}
{
    \par
}
\fi

\doendnotes{C}
\bigskip
\vfill

\clearpage

\footnotesize

\ifkorrekturansicht
  \lohead{\textsc{register}}
\fi

% theindex-Environment neu definieren ohne reledmac
\makeatletter
\renewenvironment{theindex}{%
  \ifkorrekturansicht
    \section*{\indexname}%
  \else
    \subsubsection*{Index der erwähnten Entitäten}%
  \fi
  \setlength{\parindent}{0pt}%
  \setlength{\parskip}{0pt plus 0.3pt}%
  \let\item\@idxitem
}{%
  \ifkorrekturansicht\clearpage\fi
}
\makeatother

\IfFileExists{\jobname-pw.ind}{\input{\jobname-pw.ind}}{}

% Quellenangabe nur in der Leseansicht
\ifkorrekturansicht\else
% Fallback-Definitionen, falls die .tex-Datei \titel etc. nicht gesetzt hat
\providecommand{\titel}{}
\providecommand{\editorInnen}{}
\providecommand{\dateiname}{\jobname}

\vspace{3cm}

\vfill

\footnotesize
\textsc{Quelle}: \titel. Herausgegeben von {\editorInnen}. In: \emph{Arthur Schnitzler: Briefwechsel mit Autorinnen und Autoren}.
 Digitale Edition, https://schnitzler-briefe.acdh.oeaw.ac.at/{\dateiname}.html (Stand \today)
\fi

\end{document}


      