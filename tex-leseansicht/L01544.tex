%% latex-korrekturansicht-vorspann.tex
%% Vorspann für die Korrekturansicht.
%% Lädt die gemeinsame Datei latex-vorspann.tex mit gesetztem Schalter.

\newif\ifkorrekturansicht
\korrekturansichttrue

\input{../tex-inputs/latex-vorspann}


\section[Hugo von Hofmannsthal an Arthur Schnitzler, 10. 9. 1905]{L01544 Hugo von Hofmannsthal an Arthur Schnitzler, 10. 9. 1905}
\nopagebreak\mylabel{L01544v}
\rehead{ }\normalsize\beginnumbering\briefempfaengerindex{Schnitzler, Arthur@\textsc{Schnitzler, Arthur}!zzzHofmannsthal, Hugo von@\emph{von Hugo von Hofmannsthal}!1905-09-101@{10. 9. 1905}|(be}
\toendnotes[C]{\smallbreak\pagebreak[2]}\buchAlsQuelle{Hugo von Hofmannsthal, Arthur Schnitzler: \emph{Briefwechsel}. Frankfurt am Main: \emph{S. Fischer} 1964, S. 214.}\toendnotes[C]{\smallbreak}
\pstart
           {\pb}{[}Telegramm{]}\hfill \label{K_L01544-1v}\edtext{{[}Misurina\oindex{Misurina@\textbf{Misurina}, \emph{P.PPL}|pw}, 10. September
                        1905{]}}{\lemma{\textnormal{\emph{Misurina, … 1905}}}\Cendnote{\textnormal{Diese Angabe dürfte falsch sein,
                        da es eine andere Datierung erforderlich machen würde; anzunehmen ist Lueg\oindex{Lueg@\textbf{Lueg}, \emph{Teil eines besiedelten Ortes (A.BSOX)}|pwk}.}}}\label{K_L01544-1}\pend
           \vspace{0.5em}
\pstart
           \label{K_L01544-2v}\edtext{Große Freude}{\lemma{\textnormal{\emph{Große Freude}}}\Cendnote{\textnormal{Am 9. 9. 1905 meldeten die Zeitungen, dass mit der Annahme von
                     \emph{Zwischenspiel}\pwindex{Zwischenspiel. Komoedie in drei Akten@\emph{Zwischenspiel. Komödie in drei Akten}|pwk} am \emph{Burgtheater}\orgindex{Burgtheater@Burgtheater|pwk} erstmals seit einigen Jahren wieder ein Stück
                  von Jung-Wiener Autoren an einer Wien\oindex{Wien@\textbf{Wien}, \emph{A.ADM2}|pwk}er Bühne
                  aufgeführt werden würde. Im Spezifischen bedeutete das, dass die seit der Zurückweisung von \emph{Der Schleier der Beatrice}\pwindex{Schleier der Beatrice. Schauspiel in fuenf Akten@\emph{Der Schleier der Beatrice. Schauspiel in fünf Akten}|pwk}
                  bestehende Eiszeit zwischen Schnitzler und Direktor Paul Schlenther\pwindex{Schlenther, Paul 20.08.1854 – 30.04.1916@\textsc{Schlenther, Paul} (20.08.1854 – 30.04.1916), \emph{Schriftsteller/Schriftstellerin, Kritiker/Kritikerin, Theaterleiter/Theaterleiterin}|pwk} 
                   beendet war. Vgl. Richard Beer-Hofmann an Arthur Schnitzler, 14. 9. 1900.}}}\label{K_L01544-2} über Burgtheater\orgindex{Burgtheater@Burgtheater|pw} erbitte paar Zeilen näheres Ich arbeite sehr Kommt Ihr nicht
               doch noch her\hspace*{1.5em}Herrliches Wetter gutes Essen \pend
           \pstart \spacefill\mbox{Hugo}\pend{}\selectlanguage{ngerman}\endnumbering\briefempfaengerindex{Schnitzler, Arthur@\textsc{Schnitzler, Arthur}!zzzHofmannsthal, Hugo von@\emph{von Hugo von Hofmannsthal}!1905-09-101@{10. 9. 1905}|)be}\mylabel{L01544h}  \normalsize

\doendnotes{C}
\bigskip
\vfill

\clearpage

\footnotesize

\lohead{\textsc{register}}

% Definiere theindex-Environment komplett neu ohne reledmac
\makeatletter
\renewenvironment{theindex}{%
  \section*{\indexname}%
  \setlength{\parindent}{0pt}%
  \setlength{\parskip}{0pt plus 0.3pt}%
  \let\item\@idxitem
}{%
  \clearpage
}
\makeatother

\IfFileExists{\jobname-pw.ind}{\input{\jobname-pw.ind}}{}

\end{document}

      