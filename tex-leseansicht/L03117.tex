%% latex-leseansicht-vorspann.tex
%% Vorspann für die Leseansicht.
%% Lädt die gemeinsame Datei latex-vorspann.tex mit nicht gesetztem Schalter.

\newif\ifkorrekturansicht
\korrekturansichtfalse

\input{../tex-inputs/latex-vorspann}

\begin{center}
            \textcolor{red}{ENTWURF, NICHT FERTIG KORRIGIERT}
                      \end{center}
            
         
         \renewcommand{\erwaehntePersonen}{Personen: Julius Bauer, Paul Horn}
         \renewcommand{\erwaehnteOrte}{Orte: Café Pfob, Café Union, Riedhof, Volkstheater, Wien}
         \renewcommand{\erwaehnteWerke}{Werke: ?? [Feuilleton], Musotte, Tagebuch}
               \section[Felix Salten an Arthur Schnitzler, {[}12. 11. 1892{]}]{ Felix Salten an Arthur Schnitzler, {[}12. 11. 1892{]}}\nopagebreak\mylabel{v}\rehead{ }\begin{ledgroupsized}[t]{13cm}\normalsize\beginnumbering \toendnotes[C]{\smallbreak\pagebreak[2]} \Standort{CUL, Schnitzler, B 89, A 1.}
\physDesc{Brief, 1 Blatt, 2 Seiten
\newline{}Handschrift: Bleistift, lateinische Kurrent\newline{}Ordnung: mit Bleistift von unbekannter Hand nummeriert: »20« }\toendnotes[C]{\smallbreak}\pstart
           \noindent{}{\pb}Verehrtester Freund! Dass es mir sehr, sehr unangenehm ist, mich an
               Sie zu wenden, nach allem, was Sie bereits für mich gethan, können Sie sich denken,
               doppelt, da ich weiss, dass Sie ja selbst nicht viel übrig haben. Allein Sie können
               sich auch hoffentlich denken, wie elend es mir geht, wenn ich es trotz allem thun
               muss, muss, weil ich mir keinen anderen Ausweg weiss. \strikeout{Da}Wenn es halbwegs in ihrer Macht steht so bitte ich Sie sehr, mir
               freundlichst 5 f zu leihen, welche ich {\pb}Ihnen, – da Bauer\pwindex{Bauer, Julius 15.10.1853 – 11.06.1941@\textsc{Bauer, Julius} (15.10.1853 – 11.06.1941), \emph{Schriftsteller, Journalist, Kritiker}|pw} mein Feuilleton\pwindex{Salten, Felix 06.09.1869 – 08.10.1945@\textsc{Salten, Felix} (06.09.1869 – 08.10.1945), \emph{Schriftsteller, Journalist}!?? [Feuilleton]nach Mitte November 1892@\strich\emph{?? [Feuilleton]} {[}nach Mitte November 1892{]}|pwv} dieser Tage zu
               bringen versprach – wohl Ende der nächsten Woche \introOben{}gewiss\introOben{} retour geben kann. \pend
           \pstart
           Kommen Sie heute abend – wenn auch spät – zu Pfob\oindex{Cafe Pfob@\textbf{Café Pfob}|pw}? Ich
               gehe nicht zu \label{K_L03117-2v}\edtext{Musotte\pwindex{\textcolor{red}{\textsuperscript{XXXX1 indx}}!Musotte1891@\strich\emph{Musotte} {[}1891{]}|pw}\pwindex{\textcolor{red}{\textsuperscript{XXXX1 indx}}!Musotte1891@\strich\emph{Musotte} {[}1891{]}|pw}}{\lemma{\textnormal{\emph{Musotte}}}\Cendnote{\textnormal{Schnitzler\pwindex{Schnitzler, Arthur 15.05.1862 – 21.10.1931@\textsc{Schnitzler, Arthur} (15.05.1862 – 21.10.1931), \emph{Schriftsteller, Mediziner}|pwk}s Besuch bei 
                  der Aufführung von \emph{Musotte}\pwindex{\textcolor{red}{\textsuperscript{XXXX1 indx}}!Musotte1891@\strich\emph{Musotte} {[}1891{]}|pwk}\pwindex{\textcolor{red}{\textsuperscript{XXXX1 indx}}!Musotte1891@\strich\emph{Musotte} {[}1891{]}|pwk} lässt sich nur indirekt, durch die Erwähnung 
                  des Volkstheater\oindex{Volkstheater@\textbf{Volkstheater}|pwk}s, aus dem \emph{Tagebuch}\pwindex{Schnitzler, Arthur 15.05.1862 – 21.10.1931@\textsc{Schnitzler, Arthur} (15.05.1862 – 21.10.1931), \emph{Schriftsteller, Mediziner}!Tagebuch1981 – 2000@\strich\emph{Tagebuch} {[}1981 – 2000{]}|pwk}-Eintrag 
                  zum 12. 11. 1892 ableiten.}}}\label{K_L03117-2h}! Oder, da Sie mit Paul\pwindex{Horn, Paul 13.02.1867 – 18.01.1936@\textsc{Horn, Paul} (13.02.1867 – 18.01.1936), \emph{Fabrikant}|pw} soupiren u. wie ich höre Riedhof\oindex{Riedhof@\textbf{Riedhof}|pw},
               Union\oindex{Cafe Union@\textbf{Café Union}|pw}? Besser wäre Pfob\oindex{Cafe Pfob@\textbf{Café Pfob}|pw}
               weil alles heute da sein wird. \pend
           \pstart
           Ihr 
               {\\[\baselineskip]}\spacefill\mbox{Salten}\pend
           \leftskip=0em{}
         
         \endnumbering\mylabel{h}\end{ledgroupsized}\begin{anhang}\end{anhang}\newcommand{\dateiname}{L03117}\newcommand{\titel}{Felix Salten an Arthur Schnitzler, [12. 11. 1892]}\newcommand{\editorInnen}{Martin Anton Müller und Laura Untner}%% latex-leseansicht-abspann.tex
%% Abspann für die Leseansicht.
%% Der Schalter \ifkorrekturansicht ist bereits durch den Vorspann gesetzt.

%% latex-abspann.tex
%% Gemeinsamer Abspann für Korrekturansicht und Leseansicht.
%% Setzt den Schalter \ifkorrekturansicht voraus (gesetzt in den
%% einbindenden Dateien latex-korrekturansicht-abspann.tex bzw.
%% latex-leseansicht-abspann.tex).
%% ---------------------------------------------------------------

\normalsize

% Das esempio-Environment wird nur in der Leseansicht benötigt
\ifkorrekturansicht\else
\newenvironment{esempio}[3]%
{
    \vspace{1.5ex}
    \rlap{\underline{#1}}
    \par
    \setlength{\parindent}{0cm}
    \nopagebreak
    \leftskip=#2cm
    \rightskip=#3cm
}
{
    \par
}
\fi

\doendnotes{C}
\bigskip
\vfill

\clearpage

\footnotesize

\ifkorrekturansicht
  \lohead{\textsc{register}}
\fi

% theindex-Environment neu definieren ohne reledmac
\makeatletter
\renewenvironment{theindex}{%
  \ifkorrekturansicht
    \section*{\indexname}%
  \else
    \subsubsection*{Index der erwähnten Entitäten}%
  \fi
  \setlength{\parindent}{0pt}%
  \setlength{\parskip}{0pt plus 0.3pt}%
  \let\item\@idxitem
}{%
  \ifkorrekturansicht\clearpage\fi
}
\makeatother

\IfFileExists{\jobname-pw.ind}{\input{\jobname-pw.ind}}{}

% Quellenangabe nur in der Leseansicht
\ifkorrekturansicht\else
% Fallback-Definitionen, falls die .tex-Datei \titel etc. nicht gesetzt hat
\providecommand{\titel}{}
\providecommand{\editorInnen}{}
\providecommand{\dateiname}{\jobname}

\vspace{3cm}

\vfill

\footnotesize
\textsc{Quelle}: \titel. Herausgegeben von {\editorInnen}. In: \emph{Arthur Schnitzler: Briefwechsel mit Autorinnen und Autoren}.
 Digitale Edition, https://schnitzler-briefe.acdh.oeaw.ac.at/{\dateiname}.html (Stand \today)
\fi

\end{document}


      