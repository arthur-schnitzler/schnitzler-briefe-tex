%% latex-korrekturansicht-vorspann.tex
%% Vorspann für die Korrekturansicht.
%% Lädt die gemeinsame Datei latex-vorspann.tex mit gesetztem Schalter.

\newif\ifkorrekturansicht
\korrekturansichttrue

\input{../tex-inputs/latex-vorspann}


\section[Felix Salten an Arthur Schnitzler, {[}12. 11. 1892{]}]{L03117 Felix Salten an Arthur Schnitzler, {[}12. 11. 1892{]}}
\nopagebreak\mylabel{L03117v}
\rehead{ }\normalsize\beginnumbering\briefempfaengerindex{Schnitzler, Arthur@\textsc{Schnitzler, Arthur}!zzzSalten, Felix@\emph{von Felix Salten}!1892-11-121@{{[}12. 11. 1892{]}}|(be}
\toendnotes[C]{\smallbreak\pagebreak[2]}\Standort{CUL, Schnitzler, B 89, A 1.}
\physDesc{Brief, 1 Blatt, 2 Seiten, 777 Zeichen
\newline{}Handschrift: Bleistift, lateinische Kurrent
\newline{}Schnitzler: mit Bleistift datiert: »12/XI 92« 
\newline{}Ordnung: mit Bleistift von unbekannter Hand nummeriert: »20« }\toendnotes[C]{\smallbreak}
\pstart
           \noindent{}{\pb}Verehrtester Freund! Dass es mir sehr, sehr unangenehm
               ist, mich an Sie zu wenden, nach allem, was Sie \label{K_L03117-1v}\edtext{bereits für mich gethan}{\lemma{\textnormal{\emph{bereits für mich gethan}}}\Cendnote{\textnormal{Siehe Felix Salten an Arthur Schnitzler, 10. 8. 1892.
               }}}\label{K_L03117-1}, können Sie sich denken, doppelt, da ich weiss, dass Sie ja selbst nicht viel
               übrig haben. Allein Sie können sich auch hoffentlich denken, wie elend es mir geht, \substVorne{}\textsuperscript{\textcolor{gray}{d}}\substDazwischen{}w\substHinten{}enn ich es trotz alledem thun muss, muss, weil ich mir keinen anderen Ausweg weiss\substVorne{}\textsuperscript{, wenn}\substDazwischen{}. Wenn\substHinten{} es halbwegs in Ihrer Macht steht so bitte ich Sie \uline{sehr} mir freundlichst 5 f zu leihen, welche ich {\pb}Ihnen, – da \label{K_L03117-2v}\edtext{Bauer\pwindex{Bauer, Julius 15.10.1853 – 11.06.1941@\textsc{Bauer, Julius} (15.10.1853 – 11.06.1941), \emph{Schriftsteller/Schriftstellerin, Journalist/Journalistin, Kritiker/Kritikerin}|pw} mein Feuilleton\pwindex{?? [Feuilleton]@\emph{?? [Feuilleton]}|pwv}}{\lemma{\textnormal{\emph{Bauer mein Feuilleton}}}\Cendnote{\textnormal{Nicht nachgewiesen. Es ist unklar, ob
                     Saltens\pwindex{Salten, Felix 06.09.1869 – 08.10.1945@\textsc{Salten, Felix} (06.09.1869 – 08.10.1945), \emph{Schriftsteller/Schriftstellerin, Journalist/Journalistin, Chefredakteur/Chefredakteurin}|pwk}{ }Text\pwindex{?? [Feuilleton]@\emph{?? [Feuilleton]}|pwkv} ohne Namensnennung, überhaupt nicht oder zu einem
                  viel späteren Zeitpunkt erschien.}}}\label{K_L03117-2} diese\textcolor{gray}{r} Tage zu bringen
               versprach – wol Ende der nächsten Woche \introOben{}gewiss\introOben{} retour geben
               kann.\pend
           
\pstart
           Kommen Sie heute{ }Abend – wenn auch spät – zu Pfob\oindex{Cafe Pfob@\textbf{Café Pfob}, \emph{Kaffeehaus (K.KAF)}|pw}? Ich
               gehe nicht zu \label{K_L03117-3v}\edtext{Musotte\pwindex{Musotte@\emph{Musotte}|pw}}{\lemma{\textnormal{\emph{Musotte}}}\Cendnote{\textnormal{Schnitzler hatte
                  am 9. 11. 1892 angekündigt, »beinahe ſicher«
                  in die Aufführung von \emph{Musotte}\pwindex{Musotte@\emph{Musotte}|pwk} zu gehen. Ein tatsächlicher Besuch lässt sich nur indirekt, durch die
                  Erwähnung des Volkstheaters\oindex{Volkstheater@\textbf{Volkstheater}, \emph{Theater (K.THE)}|pwk}, im \emph{Tagebuch}\pwindex{Tagebuch@\emph{Tagebuch}|pwk}-Eintrag zum 12. 11. 1892 ableiten. Ein Besuch in einem der
                  genannten Lokale ist für den Tag nicht im \emph{Tagebuch}\pwindex{Tagebuch@\emph{Tagebuch}|pwk} erwähnt.}}}\label{K_L03117-3}! Oder, da Sie mit Paul\pwindex{Horn, Paul 13.02.1867 – 18.01.1936@\textsc{Horn, Paul} (13.02.1867 – 18.01.1936), \emph{Fabrikant/Fabrikantin}|pw} soupiren u. wie ich höre Riedhof\oindex{Riedhof@\textbf{Riedhof}, \emph{Lokal (K.LKL)}|pw}, Union\oindex{Cafe Union@\textbf{Café Union}, \emph{Kaffeehaus (K.KAF)}|pw}? Besser wäre Pfob\oindex{Cafe Pfob@\textbf{Café Pfob}, \emph{Kaffeehaus (K.KAF)}|pw} weil alles heute da sein wird.\pend
           
\pstart
           Ihr {\\[\baselineskip]}\spacefill\mbox{Salten}\pend
           \leftskip=0em{}\selectlanguage{ngerman}\endnumbering\briefempfaengerindex{Schnitzler, Arthur@\textsc{Schnitzler, Arthur}!zzzSalten, Felix@\emph{von Felix Salten}!1892-11-121@{{[}12. 11. 1892{]}}|)be}\mylabel{L03117h}  \normalsize

\doendnotes{C}
\bigskip
\vfill

\clearpage

\footnotesize

\lohead{\textsc{register}}

% Definiere theindex-Environment komplett neu ohne reledmac
\makeatletter
\renewenvironment{theindex}{%
  \section*{\indexname}%
  \setlength{\parindent}{0pt}%
  \setlength{\parskip}{0pt plus 0.3pt}%
  \let\item\@idxitem
}{%
  \clearpage
}
\makeatother

\IfFileExists{\jobname-pw.ind}{\input{\jobname-pw.ind}}{}

\end{document}

      