%% latex-leseansicht-vorspann.tex
%% Vorspann für die Leseansicht.
%% Lädt die gemeinsame Datei latex-vorspann.tex mit nicht gesetztem Schalter.

\newif\ifkorrekturansicht
\korrekturansichtfalse

\input{../tex-inputs/latex-vorspann}


\section[Felix Salten an Arthur Schnitzler, {[}12. 11. 1892{]}]{L03117 Felix Salten an Arthur Schnitzler, {[}12. 11. 1892{]}}
\nopagebreak\mylabel{L03117v}
\rehead{ }\normalsize\beginnumbering\briefempfaengerindex{Schnitzler, Arthur@\textsc{Schnitzler, Arthur}!zzzSalten, Felix@\emph{von Felix Salten}!1892-11-121@{{[}12. 11. 1892{]}}|(be}
\toendnotes[C]{\smallbreak\pagebreak[2]}
\correspDesc{Versand  durch Felix Salten am [12. 11. 1892] in Wien
\newline{}Erhalt  durch Arthur Schnitzler am [12. 11. 1892?] in Wien}\toendnotes[C]{\smallbreak}
\Standort{CUL, Schnitzler, B 89, A 1.}
\physDesc{Brief, 1 Blatt, 2 Seiten, 777 Zeichen
\newline{}Handschrift: Bleistift, lateinische Kurrent
\newline{}Schnitzler: mit Bleistift datiert: »12/XI 92« 
\newline{}Ordnung: mit Bleistift von unbekannter Hand nummeriert: »20« }\toendnotes[C]{\smallbreak}
\pstart
           \noindent{}{\pb}Verehrtester Freund! Dass es mir sehr, sehr unangenehm
               ist, mich an Sie zu wenden, nach allem, was Sie \label{K_L03117-1v}\edtext{bereits für mich gethan}{\lemma{\textnormal{\emph{bereits für mich gethan}}}\Cendnote{\textnormal{Siehe XXXX Auszeichnungsfehler: Dokument L03186 nicht gefunden.
               }}}\label{K_L03117-1}, können Sie sich denken, doppelt, da ich weiss, dass Sie ja selbst nicht viel
               übrig haben. Allein Sie können sich auch hoffentlich denken, wie elend es mir geht, \substVorne{}\textsuperscript{\textcolor{gray}{d}}\substDazwischen{}w\substHinten{}enn ich es trotz alledem thun muss, muss, weil ich mir keinen anderen Ausweg weiss\substVorne{}\textsuperscript{, wenn}\substDazwischen{}. Wenn\substHinten{} es halbwegs in Ihrer Macht steht so bitte ich Sie \uline{sehr} mir freundlichst 5 f zu leihen, welche ich {\pb}Ihnen, – da \label{K_L03117-2v}\edtext{Bauer\pwindex{Bauer, Julius 15.\,10.\,1853 Szigetvár – 11.\,6.\,1941 Wien@\textsc{Bauer, Julius} (15.\,10.\,1853 Szigetvár – 11.\,6.\,1941 Wien), \emph{Schriftsteller, Journalist, Kritiker}|pw} mein Feuilleton\pwindex{Salten, Felix 6.\,9.\,1869 Budapest – 8.\,10.\,1945 Zürich@\textsc{Salten, Felix} (6.\,9.\,1869 Budapest – 8.\,10.\,1945 Zürich), \emph{Schriftsteller, Journalist, Chefredakteur}!?? [Feuilleton]@\strich\emph{?? [Feuilleton]}|pwv}}{\lemma{\textnormal{\emph{Bauer mein Feuilleton}}}\Cendnote{\textnormal{Nicht nachgewiesen. Es ist unklar, ob
                     Saltens\pwindex{Salten, Felix 6.\,9.\,1869 Budapest – 8.\,10.\,1945 Zürich@\textsc{Salten, Felix} (6.\,9.\,1869 Budapest – 8.\,10.\,1945 Zürich), \emph{Schriftsteller, Journalist, Chefredakteur}|pwk}{ }Text\pwindex{Salten, Felix 6.\,9.\,1869 Budapest – 8.\,10.\,1945 Zürich@\textsc{Salten, Felix} (6.\,9.\,1869 Budapest – 8.\,10.\,1945 Zürich), \emph{Schriftsteller, Journalist, Chefredakteur}!?? [Feuilleton]@\strich\emph{?? [Feuilleton]}|pwkv} ohne Namensnennung, überhaupt nicht oder zu einem
                  viel späteren Zeitpunkt erschien.}}}\label{K_L03117-2} diese\textcolor{gray}{r} Tage zu bringen
               versprach – wol Ende der nächsten Woche \introOben{}gewiss\introOben{} retour geben
               kann.\pend
           
\pstart
           Kommen Sie heute{ }Abend – wenn auch spät – zu Pfob\oindex{Wien@\textbf{Wien}!I., Innere Stadt@\textbf{I., Innere Stadt}!Café Pfob@\textbf{Café Pfob}, \emph{Kaffeehaus}|pw}? Ich
               gehe nicht zu \label{K_L03117-3v}\edtext{Musotte\pwindex{\textcolor{red}{\textsuperscript{XXXX indx1}}!Musotte. Schauspiel in drei Akten@\strich\emph{Musotte. Schauspiel in drei Akten}|pw}\pwindex{\textcolor{red}{\textsuperscript{XXXX indx1}}!Musotte. Schauspiel in drei Akten@\strich\emph{Musotte. Schauspiel in drei Akten}|pw}}{\lemma{\textnormal{\emph{Musotte}}}\Cendnote{\textnormal{Schnitzler hatte
                  am XXXX Auszeichnungsfehler: Dokument L00133 nicht gefunden angekündigt, »beinahe{ }ſicher«
                  in die Premiere von \emph{Musotte}\pwindex{\textcolor{red}{\textsuperscript{XXXX indx1}}!Musotte. Schauspiel in drei Akten@\strich\emph{Musotte. Schauspiel in drei Akten}|pwk}\pwindex{\textcolor{red}{\textsuperscript{XXXX indx1}}!Musotte. Schauspiel in drei Akten@\strich\emph{Musotte. Schauspiel in drei Akten}|pwk}\eventindex{Volkstheater@\textbf{Volkstheater}!Premiere von Musotte, 12.11.1892@Premiere von Musotte, 12.11.1892|pwk} zu gehen. Ein tatsächlicher Besuch lässt sich nur indirekt, durch die
                  Erwähnung des Volkstheaters\oindex{Wien@\textbf{Wien}!VII., Neubau@\textbf{VII., Neubau}!Volkstheater@\textbf{Volkstheater}, \emph{Theater}|pwk}, im \emph{Tagebuch}\pwindex{Schnitzler, Arthur 15.\,5.\,1862 Wien – 21.\,10.\,1931 ebd.@\textsc{Schnitzler, Arthur} (15.\,5.\,1862 Wien – 21.\,10.\,1931 ebd.), \emph{Schriftsteller, Mediziner}!Tagebuch@\strich\emph{Tagebuch}|pwk}-Eintrag zum 12. 11. 1892 ableiten. Ein Besuch in einem der
                  genannten Lokale ist für den Tag nicht im \emph{Tagebuch}\pwindex{Schnitzler, Arthur 15.\,5.\,1862 Wien – 21.\,10.\,1931 ebd.@\textsc{Schnitzler, Arthur} (15.\,5.\,1862 Wien – 21.\,10.\,1931 ebd.), \emph{Schriftsteller, Mediziner}!Tagebuch@\strich\emph{Tagebuch}|pwk} erwähnt.}}}\label{K_L03117-3}! Oder, da Sie mit Paul\pwindex{Horn, Paul 13.\,2.\,1867 Wien – 18.\,1.\,1936 Menton@\textsc{Horn, Paul} (13.\,2.\,1867 Wien – 18.\,1.\,1936 Menton), \emph{Fabrikant}|pw} soupiren u. wie ich höre Riedhof\oindex{Wien@\textbf{Wien}!VIII., Josefstadt@\textbf{VIII., Josefstadt}!Riedhof@\textbf{Riedhof}, \emph{Lokal}|pw}, Union\oindex{Wien@\textbf{Wien}!I., Innere Stadt@\textbf{I., Innere Stadt}!Café Union@\textbf{Café Union}, \emph{Kaffeehaus}|pw}? Besser wäre Pfob\oindex{Wien@\textbf{Wien}!I., Innere Stadt@\textbf{I., Innere Stadt}!Café Pfob@\textbf{Café Pfob}, \emph{Kaffeehaus}|pw} weil alles heute da sein wird.\pend
           
\pstart
           Ihr {\\[\baselineskip]}\spacefill\mbox{Salten}\pend
           \leftskip=0em{}\selectlanguage{ngerman}\endnumbering\briefempfaengerindex{Schnitzler, Arthur@\textsc{Schnitzler, Arthur}!zzzSalten, Felix@\emph{von Felix Salten}!1892-11-121@{{[}12. 11. 1892{]}}|)be}\mylabel{L03117h}  \newcommand{\dateiname}{L03117}\newcommand{\titel}{Felix Salten an Arthur Schnitzler, [12. 11. 1892]}\newcommand{\editorInnen}{Martin Anton Müller und Laura Untner}%% latex-leseansicht-abspann.tex
%% Abspann für die Leseansicht.
%% Der Schalter \ifkorrekturansicht ist bereits durch den Vorspann gesetzt.

%% latex-abspann.tex
%% Gemeinsamer Abspann für Korrekturansicht und Leseansicht.
%% Setzt den Schalter \ifkorrekturansicht voraus (gesetzt in den
%% einbindenden Dateien latex-korrekturansicht-abspann.tex bzw.
%% latex-leseansicht-abspann.tex).
%% ---------------------------------------------------------------

\normalsize

% Das esempio-Environment wird nur in der Leseansicht benötigt
\ifkorrekturansicht\else
\newenvironment{esempio}[3]%
{
    \vspace{1.5ex}
    \rlap{\underline{#1}}
    \par
    \setlength{\parindent}{0cm}
    \nopagebreak
    \leftskip=#2cm
    \rightskip=#3cm
}
{
    \par
}
\fi

\doendnotes{C}
\bigskip
\vfill

\clearpage

\footnotesize

\ifkorrekturansicht
  \lohead{\textsc{register}}
\fi

% theindex-Environment neu definieren ohne reledmac
\makeatletter
\renewenvironment{theindex}{%
  \ifkorrekturansicht
    \section*{\indexname}%
  \else
    \subsubsection*{Index der erwähnten Entitäten}%
  \fi
  \setlength{\parindent}{0pt}%
  \setlength{\parskip}{0pt plus 0.3pt}%
  \let\item\@idxitem
}{%
  \ifkorrekturansicht\clearpage\fi
}
\makeatother

\IfFileExists{\jobname-pw.ind}{\input{\jobname-pw.ind}}{}

% Quellenangabe nur in der Leseansicht
\ifkorrekturansicht\else
% Fallback-Definitionen, falls die .tex-Datei \titel etc. nicht gesetzt hat
\providecommand{\titel}{}
\providecommand{\editorInnen}{}
\providecommand{\dateiname}{\jobname}

\vspace{3cm}

\vfill

\footnotesize
\textsc{Quelle}: \titel. Herausgegeben von {\editorInnen}. In: \emph{Arthur Schnitzler: Briefwechsel mit Autorinnen und Autoren}.
 Digitale Edition, https://schnitzler-briefe.acdh.oeaw.ac.at/{\dateiname}.html (Stand \today)
\fi

\end{document}


