%% latex-leseansicht-vorspann.tex
%% Vorspann für die Leseansicht.
%% Lädt die gemeinsame Datei latex-vorspann.tex mit nicht gesetztem Schalter.

\newif\ifkorrekturansicht
\korrekturansichtfalse

\input{../tex-inputs/latex-vorspann}

\begin{center}
            \textcolor{red}{ENTWURF, NICHT FERTIG KORRIGIERT}
                      \end{center}
            
         
         \renewcommand{\erwaehntePersonen}{Personen: Lili Cappellini, Anna Katharina Rehmann, Felix Salten, Ottilie Salten, Paul Salten, Heinrich Schnitzler, Olga Schnitzler}
         \renewcommand{\erwaehnteOrte}{Orte: Bregenz, Gütsch, Hotel de l’Europe, Innsbruck, Kaltenleutgeben, Salzburg, Wien}
         \renewcommand{\erwaehnteWerke}{}
               \section[ Felix Salten an Arthur Schnitzler, 23. 8. 1909]{ Felix Salten an Arthur Schnitzler, 23. 8. 1909}\nopagebreak\mylabel{v}\rehead{ }\begin{ledgroupsized}[t]{13cm}\normalsize\beginnumbering \toendnotes[C]{\smallbreak\pagebreak[2]} \Standort{CUL, Schnitzler, B 89, B 1.}
\physDesc{Briefkarte, 1295 Zeichen
\newline{}Handschrift: schwarze Tinte, lateinische Kurrent
\newline{}Schnitzler: mit Bleistift Vermerk: »\textsc{Salten}« 
\newline{}Ordnung: mit Bleistift von unbekannter Hand nummeriert: »256« }\toendnotes[C]{\smallbreak}\pstart
           \raggedleft{}{\pb}Kaltenleutgeben\oindex{Kaltenleutgeben@\textbf{Kaltenleutgeben}|pw}, 23. VIII. 09\pend
           \pstart
           Lieber,{ }morgen gehe ich nun nach Wien\oindex{Wien@\textbf{Wien}|pw} und Mittwoch{ }Abend nach Innsbruck\oindex{Innsbruck@\textbf{Innsbruck}|pw}. Am 30. u. 31. werde ich in
                  Bregenz\oindex{Bregenz@\textbf{Bregenz}|pw} sein. Ich weiß nicht mehr, wer mir
               gesagt hat, Sie hätten die Absicht, \label{K_L03506-1v}\edtext{nach Bregenz\oindex{Bregenz@\textbf{Bregenz}|pw} zu kommen}{\lemma{\textnormal{\emph{nach Bregenz zu kommen}}}\Cendnote{\textnormal{nicht geschehen}}}\label{K_L03506-1h}. Ist das richtig? Ich wohne Hotel Europe\oindex{Hotel de l Europe@\textbf{Hotel de l’Europe}|pw}. Am 1. Sept. will ich für \textcolor{gray}{2} Tage nach Luzern\oindex{Guetsch@\textbf{Gütsch}|pw}. Träfe ich Sie \label{K_L03506-2v}\edtext{am 4. od. 5. in Salzburg\oindex{Salzburg@\textbf{Salzburg}|pw}}{\lemma{\textnormal{\emph{am 4. od. 5. in Salzburg}}}\Cendnote{\textnormal{nicht geschehen}}}\label{K_L03506-2h}? Geben Sie mir
               vielleicht nach Bregenz\oindex{Bregenz@\textbf{Bregenz}|pw} Nachricht, falls Sie
               nicht selbst hinkommen, was mich natürlich sehr freuen würde. Von dieser Reise gehe
               ich nicht mehr hierher zurück. Otti\pwindex{Salten, Ottilie 07.03.1868 – 22.06.1942@\textsc{Salten, Ottilie} (07.03.1868 – 22.06.1942), \emph{Schauspielerin}|pw}
               übersiedelt heute in acht Tagen mit den Kinder\pwindex{Salten, Paul 11.08.1903 – 08.05.1937@\textsc{Salten, Paul} (11.08.1903 – 08.05.1937), \emph{Filmcutter}|pwv}\pwindex{Rehmann, Anna Katharina 18.08.1904 – 27.03.1977@\textsc{Rehmann, Anna Katharina} (18.08.1904 – 27.03.1977), \emph{Schauspielerin, Übersetzerin}|pwv}n nach Wien\oindex{Wien@\textbf{Wien}|pw}. Ab
                  6. bin ich da, und freue mich aufs \label{K_L03506-3v}\edtext{Tennis, das wir dann gleich wieder
                  aufnehmen}{\lemma{\textnormal{\emph{Tennis, … aufnehmen}}}\Cendnote{\textnormal{siehe A. S.: \emph{Tagebuch}, 6. 9. 1909}}}\label{K_L03506-3h} wollen. Dass Heini\pwindex{Schnitzler, Heinrich 09.08.1902 – 12.07.1982@\textsc{Schnitzler, Heinrich} (09.08.1902 – 12.07.1982), \emph{Regisseur, Schauspieler}|pw}’s \label{K_L03506-4v}\edtext{Schwesterl\pwindex{Cappellini, Lili 13.09.1909 – 26.07.1928@\textsc{Cappellini, Lili} (13.09.1909 – 26.07.1928)|pwv} so bald bevor
                  steht}{\lemma{\textnormal{\emph{Schwesterl … steht}}}\Cendnote{\textnormal{Lili Schnitzler\pwindex{Cappellini, Lili 13.09.1909 – 26.07.1928@\textsc{Cappellini, Lili} (13.09.1909 – 26.07.1928)|pwk} wurde am 13. 9. 1909
                  geboren.}}}\label{K_L03506-4h}, wußte ich nicht. Aber – je eher, je besser! (Vorausgesetzt,
               u. s. w.) Wir senden Ihrer Frau\pwindex{Schnitzler, Olga 17.01.1882 – 13.01.1970@\textsc{Schnitzler, Olga} (17.01.1882 – 13.01.1970), \emph{Schauspielerin, Sängerin}|pwv} viele herzliche Grüße und wünschen ihr von Herzen, dass alles \uline{sehr} gut und sehr leicht sein möge! Grüßen Sie auch
               den lieben Heini\pwindex{Schnitzler, Heinrich 09.08.1902 – 12.07.1982@\textsc{Schnitzler, Heinrich} (09.08.1902 – 12.07.1982), \emph{Regisseur, Schauspieler}|pw} von uns allen. Bald wird man
               Ihnen auch schreiben müßen: »Grüßen Sie Ihre Kinder!« Eigentlich kann mans ja schon
               heute. Also: Grüßen Sie Ihre Kinder\pwindex{Schnitzler, Heinrich 09.08.1902 – 12.07.1982@\textsc{Schnitzler, Heinrich} (09.08.1902 – 12.07.1982), \emph{Regisseur, Schauspieler}|pwv}\pwindex{Cappellini, Lili 13.09.1909 – 26.07.1928@\textsc{Cappellini, Lili} (13.09.1909 – 26.07.1928)|pwv}. – Frau Olga\pwindex{Schnitzler, Olga 17.01.1882 – 13.01.1970@\textsc{Schnitzler, Olga} (17.01.1882 – 13.01.1970), \emph{Schauspielerin, Sängerin}|pw}
               hat Annerl\pwindex{Rehmann, Anna Katharina 18.08.1904 – 27.03.1977@\textsc{Rehmann, Anna Katharina} (18.08.1904 – 27.03.1977), \emph{Schauspielerin, Übersetzerin}|pw} einen entzückenden Brief
               geschrieben, der ihr großen Eindruck macht. Sie will sich selbst {\pb}bedanken, und wird nächstens
               einen Brief diktiren.\pend
           \pstart
           Auf Wiedersehen in Salzburg\oindex{Salzburg@\textbf{Salzburg}|pw} – Bregenz\oindex{Bregenz@\textbf{Bregenz}|pw} oder Wien\oindex{Wien@\textbf{Wien}|pw}. Jedenfalls bald. {\\[\baselineskip]}herzlichst {\\[\baselineskip]}I\textcolor{gray}{h}r {\\[\baselineskip]}\spacefill\mbox{Salten}\pend
           \leftskip=0em{}
         
         \endnumbering\mylabel{h}\end{ledgroupsized}  \newcommand{\dateiname}{L03506}\newcommand{\titel}{Felix Salten an Arthur Schnitzler, 23. 8. 1909}\newcommand{\editorInnen}{Martin Anton Müller und Laura Untner}%% latex-leseansicht-abspann.tex
%% Abspann für die Leseansicht.
%% Der Schalter \ifkorrekturansicht ist bereits durch den Vorspann gesetzt.

%% latex-abspann.tex
%% Gemeinsamer Abspann für Korrekturansicht und Leseansicht.
%% Setzt den Schalter \ifkorrekturansicht voraus (gesetzt in den
%% einbindenden Dateien latex-korrekturansicht-abspann.tex bzw.
%% latex-leseansicht-abspann.tex).
%% ---------------------------------------------------------------

\normalsize

% Das esempio-Environment wird nur in der Leseansicht benötigt
\ifkorrekturansicht\else
\newenvironment{esempio}[3]%
{
    \vspace{1.5ex}
    \rlap{\underline{#1}}
    \par
    \setlength{\parindent}{0cm}
    \nopagebreak
    \leftskip=#2cm
    \rightskip=#3cm
}
{
    \par
}
\fi

\doendnotes{C}
\bigskip
\vfill

\clearpage

\footnotesize

\ifkorrekturansicht
  \lohead{\textsc{register}}
\fi

% theindex-Environment neu definieren ohne reledmac
\makeatletter
\renewenvironment{theindex}{%
  \ifkorrekturansicht
    \section*{\indexname}%
  \else
    \subsubsection*{Index der erwähnten Entitäten}%
  \fi
  \setlength{\parindent}{0pt}%
  \setlength{\parskip}{0pt plus 0.3pt}%
  \let\item\@idxitem
}{%
  \ifkorrekturansicht\clearpage\fi
}
\makeatother

\IfFileExists{\jobname-pw.ind}{\input{\jobname-pw.ind}}{}

% Quellenangabe nur in der Leseansicht
\ifkorrekturansicht\else
% Fallback-Definitionen, falls die .tex-Datei \titel etc. nicht gesetzt hat
\providecommand{\titel}{}
\providecommand{\editorInnen}{}
\providecommand{\dateiname}{\jobname}

\vspace{3cm}

\vfill

\footnotesize
\textsc{Quelle}: \titel. Herausgegeben von {\editorInnen}. In: \emph{Arthur Schnitzler: Briefwechsel mit Autorinnen und Autoren}.
 Digitale Edition, https://schnitzler-briefe.acdh.oeaw.ac.at/{\dateiname}.html (Stand \today)
\fi

\end{document}


      