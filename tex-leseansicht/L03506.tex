%% latex-korrekturansicht-vorspann.tex
%% Vorspann für die Korrekturansicht.
%% Lädt die gemeinsame Datei latex-vorspann.tex mit gesetztem Schalter.

\newif\ifkorrekturansicht
\korrekturansichttrue

\input{../tex-inputs/latex-vorspann}


\section[ Felix Salten an Arthur Schnitzler, 23. 8. 1909]{L03506 Felix Salten an Arthur Schnitzler, 23. 8. 1909}
\nopagebreak\mylabel{L03506v}
\rehead{ }\normalsize\beginnumbering\briefempfaengerindex{Schnitzler, Arthur@\textsc{Schnitzler, Arthur}!zzzSalten, Felix@\emph{von Felix Salten}!1909-08-231@{23. 8. 1909}|(be}
\toendnotes[C]{\smallbreak\pagebreak[2]}\Standort{CUL, Schnitzler, B 89, B 1.}
\physDesc{Briefkarte, 1294 Zeichen
\newline{}Handschrift: schwarze Tinte, lateinische Kurrent
\newline{}Schnitzler: mit Bleistift Vermerk: »\textsc{Salten}« 
\newline{}Ordnung: mit Bleistift von unbekannter Hand nummeriert: »256« }\toendnotes[C]{\smallbreak}
\pstart
           \raggedleft{}{\pb}Kaltenleutgeben\oindex{Kaltenleutgeben@\textbf{Kaltenleutgeben}, \emph{P.PPLA3}|pw}, 23. VIII. 09\pend
           \vspace{0.5em}
\pstart
           Lieber,{ }morgen gehe ich nun nach Wien\oindex{Wien@\textbf{Wien}, \emph{A.ADM2}|pw} und Mittwoch{ }Abend nach Innsbruck\oindex{Innsbruck@\textbf{Innsbruck}, \emph{A.ADM2}|pw}. Am 30. u. 31. werde ich in
                  Bregenz\oindex{Bregenz@\textbf{Bregenz}, \emph{P.PPLA}|pw} sein. Ich weiß nicht mehr, wer mir
               gesagt hat, Sie hätten die Absicht, \label{K_L03506-1v}\edtext{nach Bregenz\oindex{Bregenz@\textbf{Bregenz}, \emph{P.PPLA}|pw} zu kommen}{\lemma{\textnormal{\emph{nach Bregenz zu kommen}}}\Cendnote{\textnormal{Die vorliegende Karte dürfte nach Wien\oindex{Wien@\textbf{Wien}, \emph{A.ADM2}|pwk} adressiert
                  gewesen sein und Schnitzler verpasst haben. Er reiste
                  am Abend des 23. 8. 1909 nach München\oindex{Muenchen@\textbf{München}, \emph{P.PPLA}|pwk}
                  und blieb (mit einer kurzen Unterbrechung) bis zum Abend des 1. 9. 1909.}}}\label{K_L03506-1}. Ist das richtig? Ich wohne Hotel Europe\oindex{Hotel de l Europe [Bregenz]@\textbf{Hotel de l’Europe [Bregenz]}, \emph{Hotel (K.HTL)}|pw}. Am 1. Sept. will ich für \textcolor{gray}{2} Tage nach Luzern\oindex{Guetsch@\textbf{Gütsch}, \emph{T.HLL}|pw}. Träfe ich Sie \label{K_L03506-2v}\edtext{am 4. od. 5. in Salzburg\oindex{Salzburg@\textbf{Salzburg}, \emph{A.ADM2}|pw}}{\lemma{\textnormal{\emph{am 4. od. 5. in Salzburg}}}\Cendnote{\textnormal{Dazu kam es nicht.}}}\label{K_L03506-2}? Geben Sie mir
               vielleicht nach Bregenz\oindex{Bregenz@\textbf{Bregenz}, \emph{P.PPLA}|pw} Nachricht, falls Sie
               nicht selbst hinkommen, was mich natürlich sehr freuen würde. Von dieser Reise gehe
               ich nicht mehr hierher zurück. Otti\pwindex{Salten, Ottilie 07.03.1868 – 22.06.1942@\textsc{Salten, Ottilie} (07.03.1868 – 22.06.1942), \emph{Schauspieler/Schauspielerin}|pw}
               übersiedelt heute in acht Tagen mit den Kinder\pwindex{Salten, Paul 11.08.1903 – 08.05.1937@\textsc{Salten, Paul} (11.08.1903 – 08.05.1937), \emph{Filmcutter/Filmcutterin}|pwv}\pwindex{Rehmann, Anna Katharina 18.08.1904 – 27.03.1977@\textsc{Rehmann, Anna Katharina} (18.08.1904 – 27.03.1977), \emph{Schauspieler/Schauspielerin, Übersetzer/Übersetzerin}|pwv}n nach Wien\oindex{Wien@\textbf{Wien}, \emph{A.ADM2}|pw}. Ab
                  6. bin ich da, und freue mich aufs \label{K_L03506-3v}\edtext{Tennis, das wir dann gleich wieder
                  aufnehmen}{\lemma{\textnormal{\emph{Tennis, … aufnehmen}}}\Cendnote{\textnormal{Siehe A. S.: \emph{Tagebuch}, 6. 9. 1909.
               }}}\label{K_L03506-3} wollen. Dass Heini\pwindex{Schnitzler, Heinrich 09.08.1902 – 12.07.1982@\textsc{Schnitzler, Heinrich} (09.08.1902 – 12.07.1982), \emph{Regisseur/Regisseurin, Schauspieler/Schauspielerin}|pw}’s \label{K_L03506-4v}\edtext{Schwesterl\pwindex{Cappellini, Lili 13.09.1909 – 26.07.1928@\textsc{Cappellini, Lili} (13.09.1909 – 26.07.1928)|pwv} so bald bevor
                  steht}{\lemma{\textnormal{\emph{Schwesterl … steht}}}\Cendnote{\textnormal{Lili Schnitzler\pwindex{Cappellini, Lili 13.09.1909 – 26.07.1928@\textsc{Cappellini, Lili} (13.09.1909 – 26.07.1928)|pwk} wurde am 13. 9. 1909
                  geboren. Warum Salten\pwindex{Salten, Felix 06.09.1869 – 08.10.1945@\textsc{Salten, Felix} (06.09.1869 – 08.10.1945), \emph{Schriftsteller/Schriftstellerin, Journalist/Journalistin, Chefredakteur/Chefredakteurin}|pwk} sicher scheint, dass es ein Mädchen werden sollte, ist unklar.}}}\label{K_L03506-4}, wußte ich nicht. Aber – je eher, je besser! (Vorausgesetzt,
               u. s. w.) Wir senden Ihrer Frau\pwindex{Schnitzler, Olga 17.01.1882 – 13.01.1970@\textsc{Schnitzler, Olga} (17.01.1882 – 13.01.1970), \emph{Schauspieler/Schauspielerin, Sänger/Sängerin}|pwv} viele herzliche Grüße und wünschen ihr von Herzen, dass alles \uline{sehr} gut und sehr leicht sein möge! Grüßen Sie auch
               den lieben Heini\pwindex{Schnitzler, Heinrich 09.08.1902 – 12.07.1982@\textsc{Schnitzler, Heinrich} (09.08.1902 – 12.07.1982), \emph{Regisseur/Regisseurin, Schauspieler/Schauspielerin}|pw} von uns allen. Bald wird man
               Ihnen auch schreiben müßen: »Grüßen Sie Ihre Kinder!« Eigentlich kann mans ja schon
               heute. Also: Grüßen Sie Ihre Kinder\pwindex{Schnitzler, Heinrich 09.08.1902 – 12.07.1982@\textsc{Schnitzler, Heinrich} (09.08.1902 – 12.07.1982), \emph{Regisseur/Regisseurin, Schauspieler/Schauspielerin}|pwv}\pwindex{Cappellini, Lili 13.09.1909 – 26.07.1928@\textsc{Cappellini, Lili} (13.09.1909 – 26.07.1928)|pwv}. – Frau Olga\pwindex{Schnitzler, Olga 17.01.1882 – 13.01.1970@\textsc{Schnitzler, Olga} (17.01.1882 – 13.01.1970), \emph{Schauspieler/Schauspielerin, Sänger/Sängerin}|pw}
               hat Annerl\pwindex{Rehmann, Anna Katharina 18.08.1904 – 27.03.1977@\textsc{Rehmann, Anna Katharina} (18.08.1904 – 27.03.1977), \emph{Schauspieler/Schauspielerin, Übersetzer/Übersetzerin}|pw} einen entzückenden Brief
               geschrieben, der ihr großen Eindruck macht. Sie will sich selbst {\pb}bedanken, und wird nächstens
               einen Brief diktiren.\pend
           
\pstart
           Auf Wiedersehen in Salzburg\oindex{Salzburg@\textbf{Salzburg}, \emph{A.ADM2}|pw} – Bregenz\oindex{Bregenz@\textbf{Bregenz}, \emph{P.PPLA}|pw} oder Wien\oindex{Wien@\textbf{Wien}, \emph{A.ADM2}|pw}. Jedenfalls bald. {\\[\baselineskip]}herzlichst {\\[\baselineskip]}I\textcolor{gray}{h}r {\\[\baselineskip]}\spacefill\mbox{Salten}\pend
           \leftskip=0em{}\selectlanguage{ngerman}\endnumbering\briefempfaengerindex{Schnitzler, Arthur@\textsc{Schnitzler, Arthur}!zzzSalten, Felix@\emph{von Felix Salten}!1909-08-231@{23. 8. 1909}|)be}\mylabel{L03506h}  \normalsize

\doendnotes{C}
\bigskip
\vfill

\clearpage

\footnotesize

\lohead{\textsc{register}}

% Definiere theindex-Environment komplett neu ohne reledmac
\makeatletter
\renewenvironment{theindex}{%
  \section*{\indexname}%
  \setlength{\parindent}{0pt}%
  \setlength{\parskip}{0pt plus 0.3pt}%
  \let\item\@idxitem
}{%
  \clearpage
}
\makeatother

\IfFileExists{\jobname-pw.ind}{\input{\jobname-pw.ind}}{}

\end{document}

      