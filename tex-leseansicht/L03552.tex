%% latex-leseansicht-vorspann.tex
%% Vorspann für die Leseansicht.
%% Lädt die gemeinsame Datei latex-vorspann.tex mit nicht gesetztem Schalter.

\newif\ifkorrekturansicht
\korrekturansichtfalse

\input{../tex-inputs/latex-vorspann}


\section[ Felix Salten an Arthur Schnitzler, 14. 10. 1910]{L03552 Felix Salten an Arthur Schnitzler,  14. 10. 1910}
\nopagebreak\mylabel{L03552v}
\rehead{ }\normalsize\beginnumbering\briefempfaengerindex{Schnitzler, Arthur@\textsc{Schnitzler, Arthur}!zzzSalten, Felix@\emph{von Felix Salten}!1910-10-141@{14. 10. 1910}|(be}
\toendnotes[C]{\smallbreak\pagebreak[2]}
\correspDesc{Versand  durch Felix Salten am 14. 10. 1910 in Wien
\newline{}Erhalt  durch Arthur Schnitzler im Zeitraum [14. 10. 1910 – 17. 10. 1910?] in Wien}\toendnotes[C]{\smallbreak}
\Standort{CUL, Schnitzler, B 89, B 2.}
\physDesc{Brief, 1 Blatt, 1 Seite, 1151 Zeichen
\newline{}Handschrift: schwarze Tinte, lateinische Kurrent
\newline{}Ordnung: mit Bleistift von unbekannter Hand nummeriert: »267« }
\buchAbdrucke{\weitereDrucke{Adele Sandrock, Arthur Schnitzler: \emph{Dilly. Geschichte einer Liebe in Briefen, Bildern und
                        Dokumenten}. Zusammengestellt von Renate Wagner. Wien, München: \emph{Amalthea} 1975, S. 309.} }\toendnotes[C]{\smallbreak}
\pstart
           {\pb}\textcolor{gray}{\textbf{\textsc{Felix Salten}}}\pend
           
\pstart
           \raggedleft{}14. X. 10\pend
           
\pstart{}Lieber,\pend\vspace{0.5em}
\pstart
           ich möchte Ihnen, eh’ Sie \label{K_L03552-1v}\edtext{auf den Semmering\oindex{Semmering@\textbf{Semmering}, \emph{Verwaltungsgebiet}|pw} fahren}{\lemma{\textnormal{\emph{auf den Semmering fahren}}}\Cendnote{\textnormal{Schnitzler fuhr am 16. 10. 1910 auf den
                     Semmering\oindex{Semmering@\textbf{Semmering}, \emph{Verwaltungsgebiet}|pwk} und blieb dort bis zum 19. 10. 1910.}}}\label{K_L03552-1},
               rasch noch mitteilen, dass ich gestern{ }Abends mit Berger\pwindex{Berger, Alfred von 30.\,4.\,1853 Wien – 24.\,8.\,1912 ebd.@\textsc{Berger, Alfred von} (30.\,4.\,1853 Wien – 24.\,8.\,1912 ebd.), \emph{Schriftsteller, Journalist, Theaterleiter}|pw} sprach, und die
               Gelegenheit wahrnahm, ein Wort für die Sandrock\pwindex{Sandrock, Adele 19.\,8.\,1863 Rotterdam – 30.\,8.\,1937 Berlin@\textsc{Sandrock, Adele} (19.\,8.\,1863 Rotterdam – 30.\,8.\,1937 Berlin), \emph{Schauspielerin}|pw} zu sagen. Berger\pwindex{Berger, Alfred von 30.\,4.\,1853 Wien – 24.\,8.\,1912 ebd.@\textsc{Berger, Alfred von} (30.\,4.\,1853 Wien – 24.\,8.\,1912 ebd.), \emph{Schriftsteller, Journalist, Theaterleiter}|pw} ist bereit,
                  \label{K_L03552-2v}\edtext{sie zu engagiren\orgindex{Burgtheater@Burgtheater|pwv}}{\lemma{\textnormal{\emph{sie zu engagiren}}}\Cendnote{\textnormal{Vgl. Hermann Bahr, Arthur Schnitzler: \emph{Briefwechsel, Aufzeichnungen, Dokumente (1891–1931)}, Aufzeichnung von Hugo Thimig, 25. 10. 1910 und Adele Sandrock\pwindex{Sandrock, Adele 19.\,8.\,1863 Rotterdam – 30.\,8.\,1937 Berlin@\textsc{Sandrock, Adele} (19.\,8.\,1863 Rotterdam – 30.\,8.\,1937 Berlin), \emph{Schauspielerin}|pwk}, Arthur Schnitzler: \emph{Dilly. Geschichte
                        einer Liebe in Briefen, Bildern und Dokumenten}. Zusammengestellt von
                     Renate Wagner. Wien, München:
                        \emph{Amalthea}{ }1975, S. 306–315. Zu einem neuerlichen Engagement
                  von Sandrock\pwindex{Sandrock, Adele 19.\,8.\,1863 Rotterdam – 30.\,8.\,1937 Berlin@\textsc{Sandrock, Adele} (19.\,8.\,1863 Rotterdam – 30.\,8.\,1937 Berlin), \emph{Schauspielerin}|pwk} am \emph{Burgtheater}\orgindex{Burgtheater@Burgtheater|pwk} kam es nicht. }}}\label{K_L03552-2}. Bedingungen: sie darf
               nicht gleich, darf überhaupt in diesem ersten Jahr keinen Vorschuß verlangen, weil
               dafür kein Geld zu haben ist und sie dem Direktor\pwindex{Berger, Alfred von 30.\,4.\,1853 Wien – 24.\,8.\,1912 ebd.@\textsc{Berger, Alfred von} (30.\,4.\,1853 Wien – 24.\,8.\,1912 ebd.), \emph{Schriftsteller, Journalist, Theaterleiter}|pwv} mit solchen Affairen Verlegenheiten bereiten würde.
               Dann: sie muß sich für den Anfang mit 8 bis 10.000 Kronen Gage begnügen; muß auch
               wegen Rollen Geduld haben und darf dabei sicher sein, dass sie würdige Aufgaben
               erhält. Berger\pwindex{Berger, Alfred von 30.\,4.\,1853 Wien – 24.\,8.\,1912 ebd.@\textsc{Berger, Alfred von} (30.\,4.\,1853 Wien – 24.\,8.\,1912 ebd.), \emph{Schriftsteller, Journalist, Theaterleiter}|pw}’s Worte: »Ich werde die Sandrock\pwindex{Sandrock, Adele 19.\,8.\,1863 Rotterdam – 30.\,8.\,1937 Berlin@\textsc{Sandrock, Adele} (19.\,8.\,1863 Rotterdam – 30.\,8.\,1937 Berlin), \emph{Schauspielerin}|pw} nicht untergehen laßen!« Dass sie
               neben der Bleibtreu\pwindex{Bleibtreu, Hedwig 23.\,12.\,1868 Linz – 24.\,1.\,1958 Wien@\textsc{Bleibtreu, Hedwig} (23.\,12.\,1868 Linz – 24.\,1.\,1958 Wien), \emph{Schauspielerin}|pw} Platz haben wird, hält er
               für sicher. Vielleicht teilen Sie ihr das mit. Ich glaube, es wird ihr lieber sein
               als ein Varieté-Stück. Sie kann sich, wenn sie die Sache auf dieser Basis betreiben
               will, mit mir in Verbindung setzen. Berger\pwindex{Berger, Alfred von 30.\,4.\,1853 Wien – 24.\,8.\,1912 ebd.@\textsc{Berger, Alfred von} (30.\,4.\,1853 Wien – 24.\,8.\,1912 ebd.), \emph{Schriftsteller, Journalist, Theaterleiter}|pw} ist
               am Sonntag zu Mittag bei mir. Es wäre
               gut, wenn ich bis dahin eine Zeile von der Sandrock\pwindex{Sandrock, Adele 19.\,8.\,1863 Rotterdam – 30.\,8.\,1937 Berlin@\textsc{Sandrock, Adele} (19.\,8.\,1863 Rotterdam – 30.\,8.\,1937 Berlin), \emph{Schauspielerin}|pw} hätte. Auf den Semmering\oindex{Semmering@\textbf{Semmering}, \emph{Verwaltungsgebiet}|pw} kann
               ich leider nicht. Wir wünschen Frau \label{K_L03552-3v}\edtext{Olga\pwindex{Schnitzler, Olga 17.\,1.\,1882 Wien – 13.\,1.\,1970 Lugano@\textsc{Schnitzler, Olga} (17.\,1.\,1882 Wien – 13.\,1.\,1970 Lugano), \emph{Schauspielerin, Sängerin}|pw} schöne Erholung}{\lemma{\textnormal{\emph{Olga schöne Erholung}}}\Cendnote{\textnormal{Siehe A. S.: \emph{Tagebuch}, 7. 10. 1910. }}}\label{K_L03552-3} und
               Ihnen Beiden\pwindex{Schnitzler, Olga 17.\,1.\,1882 Wien – 13.\,1.\,1970 Lugano@\textsc{Schnitzler, Olga} (17.\,1.\,1882 Wien – 13.\,1.\,1970 Lugano), \emph{Schauspielerin, Sängerin}|pwv} gutes
               Wetter!\pend
           
\pstart
           Herzlich von uns\pwindex{Salten, Ottilie 7.\,3.\,1868 Prag – 22.\,6.\,1942 Zürich@\textsc{Salten, Ottilie} (7.\,3.\,1868 Prag – 22.\,6.\,1942 Zürich), \emph{Schauspielerin}|pwv} zu
               Ihnen {\\[\baselineskip]}Ihr {\\[\baselineskip]}\spacefill\mbox{Salten}\pend
           \leftskip=0em{}\selectlanguage{ngerman}\endnumbering\briefempfaengerindex{Schnitzler, Arthur@\textsc{Schnitzler, Arthur}!zzzSalten, Felix@\emph{von Felix Salten}!1910-10-141@{14. 10. 1910}|)be}\mylabel{L03552h}  \newcommand{\dateiname}{L03552}\newcommand{\titel}{Felix Salten an Arthur Schnitzler, 14. 10. 1910}\newcommand{\editorInnen}{Martin Anton Müller und Laura Untner}%% latex-leseansicht-abspann.tex
%% Abspann für die Leseansicht.
%% Der Schalter \ifkorrekturansicht ist bereits durch den Vorspann gesetzt.

%% latex-abspann.tex
%% Gemeinsamer Abspann für Korrekturansicht und Leseansicht.
%% Setzt den Schalter \ifkorrekturansicht voraus (gesetzt in den
%% einbindenden Dateien latex-korrekturansicht-abspann.tex bzw.
%% latex-leseansicht-abspann.tex).
%% ---------------------------------------------------------------

\normalsize

% Das esempio-Environment wird nur in der Leseansicht benötigt
\ifkorrekturansicht\else
\newenvironment{esempio}[3]%
{
    \vspace{1.5ex}
    \rlap{\underline{#1}}
    \par
    \setlength{\parindent}{0cm}
    \nopagebreak
    \leftskip=#2cm
    \rightskip=#3cm
}
{
    \par
}
\fi

\doendnotes{C}
\bigskip
\vfill

\clearpage

\footnotesize

\ifkorrekturansicht
  \lohead{\textsc{register}}
\fi

% theindex-Environment neu definieren ohne reledmac
\makeatletter
\renewenvironment{theindex}{%
  \ifkorrekturansicht
    \section*{\indexname}%
  \else
    \subsubsection*{Index der erwähnten Entitäten}%
  \fi
  \setlength{\parindent}{0pt}%
  \setlength{\parskip}{0pt plus 0.3pt}%
  \let\item\@idxitem
}{%
  \ifkorrekturansicht\clearpage\fi
}
\makeatother

\IfFileExists{\jobname-pw.ind}{\input{\jobname-pw.ind}}{}

% Quellenangabe nur in der Leseansicht
\ifkorrekturansicht\else
% Fallback-Definitionen, falls die .tex-Datei \titel etc. nicht gesetzt hat
\providecommand{\titel}{}
\providecommand{\editorInnen}{}
\providecommand{\dateiname}{\jobname}

\vspace{3cm}

\vfill

\footnotesize
\textsc{Quelle}: \titel. Herausgegeben von {\editorInnen}. In: \emph{Arthur Schnitzler: Briefwechsel mit Autorinnen und Autoren}.
 Digitale Edition, https://schnitzler-briefe.acdh.oeaw.ac.at/{\dateiname}.html (Stand \today)
\fi

\end{document}


