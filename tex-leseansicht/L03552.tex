%% latex-leseansicht-vorspann.tex
%% Vorspann für die Leseansicht.
%% Lädt die gemeinsame Datei latex-vorspann.tex mit nicht gesetztem Schalter.

\newif\ifkorrekturansicht
\korrekturansichtfalse

\input{../tex-inputs/latex-vorspann}

\begin{center}
            \textcolor{red}{ENTWURF, NICHT FERTIG KORRIGIERT}
                      \end{center}
            
         
         \renewcommand{\erwaehntePersonen}{Personen: Alfred von Berger, Hedwig Bleibtreu, Adele Sandrock, Olga Schnitzler}
         \renewcommand{\erwaehnteOrte}{Orte: Semmering, Wien}
         \renewcommand{\erwaehnteWerke}{}
               \section[Felix Salten an Arthur Schnitzler, 14. 10. 1910]{ Felix Salten an Arthur Schnitzler, 14. 10. 1910}\nopagebreak\mylabel{v}\rehead{ }\begin{ledgroupsized}[t]{13cm}\normalsize\beginnumbering \toendnotes[C]{\smallbreak\pagebreak[2]} \Standort{CUL, Schnitzler, B 89, B 2.}
\physDesc{Brief, 1 Blatt, 1 Seite, 1166 Zeichen
\newline{}Handschrift: schwarze Tinte, lateinische Kurrent
\newline{}Ordnung: mit Bleistift von unbekannter Hand nummeriert:
                                    »267« }\pstart
           \noindent{}{\pb}\textcolor{gray}{\textbf{\textsc{Felix Salten}}}\pend
           \pstart
           \raggedleft{}14. X. 10\pend
           \pstart{}Lieber,\pend\pstart
           ich möchte Ihnen, eh’ Sie auf den Semmering\oindex{Semmering@\textbf{Semmering}|pw} fahren,
               rasch noch mitteilen, dass ich gestern Abends mit Berger\pwindex{Berger, Alfred von 30.04.1853 – 24.08.1912@\textsc{Berger, Alfred von} (30.04.1853 – 24.08.1912), \emph{Schriftsteller, Journalist, Theaterleiter}|pw} sprach, und die Gelegenheit wahrnahm, ein Wort für die Sandrock\pwindex{Sandrock, Adele 1863-08-19 – 1937-08-30@\textsc{Sandrock, Adele} (1863-08-19 – 1937-08-30), \emph{Schauspielerin}|pw} zu sagen. Berger\pwindex{Berger, Alfred von 30.04.1853 – 24.08.1912@\textsc{Berger, Alfred von} (30.04.1853 – 24.08.1912), \emph{Schriftsteller, Journalist, Theaterleiter}|pw}
               ist bereit, sie zu engagiren. Bedingungen: sie darf nicht gleich, darf überhaupt in
               diesem ersten Jahr keinen Vorschuß verlangen, weil dafür kein Geld zu haben ist und
               sie dem Direktor mit solchen Affairen Verlegenheiten bereiten würde. Dann: sie muß
               sich für den Anfang mit 8 bis 10.000 Kronen Gage begnügen; muß auch wegen Rollen
               Geduld haben und darf dabei sicher sein, dass sie würdige Aufgaben erhält. Berger\pwindex{Berger, Alfred von 30.04.1853 – 24.08.1912@\textsc{Berger, Alfred von} (30.04.1853 – 24.08.1912), \emph{Schriftsteller, Journalist, Theaterleiter}|pw}’s Worte: »Ich werde die Sandrock\pwindex{Sandrock, Adele 1863-08-19 – 1937-08-30@\textsc{Sandrock, Adele} (1863-08-19 – 1937-08-30), \emph{Schauspielerin}|pw} nicht untergehen laßen!« Dass sie neben der Bleibtreu\pwindex{Bleibtreu, Hedwig 23.12.1868 – 24.01.1958@\textsc{Bleibtreu, Hedwig} (23.12.1868 – 24.01.1958), \emph{Schauspielerin}|pw} Platz haben wird, hält er für sicher. Vielleicht teilen Sie
               ihr das mit. Ich glaube, es und Ihr lieber sein als ein Varieté-Stück. Sie kann sich,
               wenn sie die Sache auf dieser Basis betreiben will, mit mir in Verbindung setzen. Berger\pwindex{Berger, Alfred von 30.04.1853 – 24.08.1912@\textsc{Berger, Alfred von} (30.04.1853 – 24.08.1912), \emph{Schriftsteller, Journalist, Theaterleiter}|pw} ist am Sonntag zu Mittag bei mir. Es wäre gut,
               wenn ist bis dahin eine Zeile von der Sandrock\pwindex{Sandrock, Adele 1863-08-19 – 1937-08-30@\textsc{Sandrock, Adele} (1863-08-19 – 1937-08-30), \emph{Schauspielerin}|pw} hätte.
               Auf den Semmering\oindex{Semmering@\textbf{Semmering}|pw} kann ich leider nicht. Wir wünschen
               Frau Olga\pwindex{Schnitzler, Olga 17.01.1882 – 13.01.1970@\textsc{Schnitzler, Olga} (17.01.1882 – 13.01.1970), \emph{Schauspielerin, Sängerin}|pw} schöne Erholung und Ihnen Beiden gutes
               Wetter! {\\}Herzlich von uns zu Ihnen\pend
           \pstart
           Ihr
               {\\[\baselineskip]}Salten\pend
           \leftskip=0em{}
         
         \endnumbering\mylabel{h}\end{ledgroupsized}\begin{anhang}\end{anhang}\newcommand{\dateiname}{L03552}\newcommand{\titel}{Felix Salten an Arthur Schnitzler, 14. 10. 1910}\newcommand{\editorInnen}{Martin Anton Müller und Laura Untner}%% latex-leseansicht-abspann.tex
%% Abspann für die Leseansicht.
%% Der Schalter \ifkorrekturansicht ist bereits durch den Vorspann gesetzt.

%% latex-abspann.tex
%% Gemeinsamer Abspann für Korrekturansicht und Leseansicht.
%% Setzt den Schalter \ifkorrekturansicht voraus (gesetzt in den
%% einbindenden Dateien latex-korrekturansicht-abspann.tex bzw.
%% latex-leseansicht-abspann.tex).
%% ---------------------------------------------------------------

\normalsize

% Das esempio-Environment wird nur in der Leseansicht benötigt
\ifkorrekturansicht\else
\newenvironment{esempio}[3]%
{
    \vspace{1.5ex}
    \rlap{\underline{#1}}
    \par
    \setlength{\parindent}{0cm}
    \nopagebreak
    \leftskip=#2cm
    \rightskip=#3cm
}
{
    \par
}
\fi

\doendnotes{C}
\bigskip
\vfill

\clearpage

\footnotesize

\ifkorrekturansicht
  \lohead{\textsc{register}}
\fi

% theindex-Environment neu definieren ohne reledmac
\makeatletter
\renewenvironment{theindex}{%
  \ifkorrekturansicht
    \section*{\indexname}%
  \else
    \subsubsection*{Index der erwähnten Entitäten}%
  \fi
  \setlength{\parindent}{0pt}%
  \setlength{\parskip}{0pt plus 0.3pt}%
  \let\item\@idxitem
}{%
  \ifkorrekturansicht\clearpage\fi
}
\makeatother

\IfFileExists{\jobname-pw.ind}{\input{\jobname-pw.ind}}{}

% Quellenangabe nur in der Leseansicht
\ifkorrekturansicht\else
% Fallback-Definitionen, falls die .tex-Datei \titel etc. nicht gesetzt hat
\providecommand{\titel}{}
\providecommand{\editorInnen}{}
\providecommand{\dateiname}{\jobname}

\vspace{3cm}

\vfill

\footnotesize
\textsc{Quelle}: \titel. Herausgegeben von {\editorInnen}. In: \emph{Arthur Schnitzler: Briefwechsel mit Autorinnen und Autoren}.
 Digitale Edition, https://schnitzler-briefe.acdh.oeaw.ac.at/{\dateiname}.html (Stand \today)
\fi

\end{document}


      