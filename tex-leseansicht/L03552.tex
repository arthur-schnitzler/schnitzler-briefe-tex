%% latex-korrekturansicht-vorspann.tex
%% Vorspann für die Korrekturansicht.
%% Lädt die gemeinsame Datei latex-vorspann.tex mit gesetztem Schalter.

\newif\ifkorrekturansicht
\korrekturansichttrue

\input{../tex-inputs/latex-vorspann}


\section[ Felix Salten an Arthur Schnitzler, 14. 10. 1910]{L03552 Felix Salten an Arthur Schnitzler, 14. 10. 1910}
\nopagebreak\mylabel{L03552v}
\rehead{ }\normalsize\beginnumbering\briefempfaengerindex{Schnitzler, Arthur@\textsc{Schnitzler, Arthur}!zzzSalten, Felix@\emph{von Felix Salten}!1910-10-141@{14. 10. 1910}|(be}
\toendnotes[C]{\smallbreak\pagebreak[2]}\Standort{CUL, Schnitzler, B 89, B 2.}
\physDesc{Brief, 1 Blatt, 1 Seite, 1151 Zeichen
\newline{}Handschrift: schwarze Tinte, lateinische Kurrent
\newline{}Ordnung: mit Bleistift von unbekannter Hand nummeriert: »267« }
\buchAbdrucke{\weitereDrucke{Adele Sandrock, Arthur Schnitzler: \emph{Dilly. Geschichte einer Liebe in Briefen, Bildern und
                        Dokumenten}. Wien, München: \emph{Amalthea} 1975, S. 309.} }\toendnotes[C]{\smallbreak}
\pstart
           {\pb}\textcolor{gray}{\textbf{\textsc{Felix Salten}}}\pend
           
\pstart
           \raggedleft{}14. X. 10\pend
           
\pstart{}Lieber,\pend\vspace{0.5em}
\pstart
           ich möchte Ihnen, eh’ Sie \label{K_L03552-1v}\edtext{auf den Semmering\oindex{Semmering@\textbf{Semmering}, \emph{A.ADM3}|pw} fahren}{\lemma{\textnormal{\emph{auf den Semmering fahren}}}\Cendnote{\textnormal{Schnitzler fuhr am 16. 10. 1910 auf den
                     Semmering\oindex{Semmering@\textbf{Semmering}, \emph{A.ADM3}|pwk} und blieb dort bis zum 19. 10. 1910.}}}\label{K_L03552-1},
               rasch noch mitteilen, dass ich gestern{ }Abends mit Berger\pwindex{Berger, Alfred von 30.04.1853 – 24.08.1912@\textsc{Berger, Alfred von} (30.04.1853 – 24.08.1912), \emph{Schriftsteller/Schriftstellerin, Journalist/Journalistin, Theaterleiter/Theaterleiterin}|pw} sprach, und die
               Gelegenheit wahrnahm, ein Wort für die Sandrock\pwindex{Sandrock, Adele 1863-08-19 – 1937-08-30@\textsc{Sandrock, Adele} (1863-08-19 – 1937-08-30), \emph{Schauspieler/Schauspielerin}|pw} zu sagen. Berger\pwindex{Berger, Alfred von 30.04.1853 – 24.08.1912@\textsc{Berger, Alfred von} (30.04.1853 – 24.08.1912), \emph{Schriftsteller/Schriftstellerin, Journalist/Journalistin, Theaterleiter/Theaterleiterin}|pw} ist bereit,
                  \label{K_L03552-2v}\edtext{sie zu engagiren\orgindex{Burgtheater@Burgtheater|pwv}}{\lemma{\textnormal{\emph{sie zu engagiren}}}\Cendnote{\textnormal{Vgl. Hermann Bahr, Arthur Schnitzler: \emph{Briefwechsel, Aufzeichnungen, Dokumente (1891–1931)}, Aufzeichnung von Hugo Thimig, 25. 10. 1910 und Adele Sandrock\pwindex{Sandrock, Adele 1863-08-19 – 1937-08-30@\textsc{Sandrock, Adele} (1863-08-19 – 1937-08-30), \emph{Schauspieler/Schauspielerin}|pwk}, Arthur Schnitzler: \emph{Dilly. Geschichte
                        einer Liebe in Briefen, Bildern und Dokumenten}. Zusammengestellt von
                     Renate Wagner. Wien, München:
                        \emph{Amalthea}{ }1975, S. 306–315. Zu einem neuerlichen Engagement
                  von Sandrock\pwindex{Sandrock, Adele 1863-08-19 – 1937-08-30@\textsc{Sandrock, Adele} (1863-08-19 – 1937-08-30), \emph{Schauspieler/Schauspielerin}|pwk} am \emph{Burgtheater}\orgindex{Burgtheater@Burgtheater|pwk} kam es nicht. }}}\label{K_L03552-2}. Bedingungen: sie darf
               nicht gleich, darf überhaupt in diesem ersten Jahr keinen Vorschuß verlangen, weil
               dafür kein Geld zu haben ist und sie dem Direktor\pwindex{Berger, Alfred von 30.04.1853 – 24.08.1912@\textsc{Berger, Alfred von} (30.04.1853 – 24.08.1912), \emph{Schriftsteller/Schriftstellerin, Journalist/Journalistin, Theaterleiter/Theaterleiterin}|pwv} mit solchen Affairen Verlegenheiten bereiten würde.
               Dann: sie muß sich für den Anfang mit 8 bis 10.000 Kronen Gage begnügen; muß auch
               wegen Rollen Geduld haben und darf dabei sicher sein, dass sie würdige Aufgaben
               erhält. Berger\pwindex{Berger, Alfred von 30.04.1853 – 24.08.1912@\textsc{Berger, Alfred von} (30.04.1853 – 24.08.1912), \emph{Schriftsteller/Schriftstellerin, Journalist/Journalistin, Theaterleiter/Theaterleiterin}|pw}’s Worte: »Ich werde die Sandrock\pwindex{Sandrock, Adele 1863-08-19 – 1937-08-30@\textsc{Sandrock, Adele} (1863-08-19 – 1937-08-30), \emph{Schauspieler/Schauspielerin}|pw} nicht untergehen laßen!« Dass sie
               neben der Bleibtreu\pwindex{Bleibtreu, Hedwig 23.12.1868 – 24.01.1958@\textsc{Bleibtreu, Hedwig} (23.12.1868 – 24.01.1958), \emph{Schauspieler/Schauspielerin}|pw} Platz haben wird, hält er
               für sicher. Vielleicht teilen Sie ihr das mit. Ich glaube, es wird ihr lieber sein
               als ein Varieté-Stück. Sie kann sich, wenn sie die Sache auf dieser Basis betreiben
               will, mit mir in Verbindung setzen. Berger\pwindex{Berger, Alfred von 30.04.1853 – 24.08.1912@\textsc{Berger, Alfred von} (30.04.1853 – 24.08.1912), \emph{Schriftsteller/Schriftstellerin, Journalist/Journalistin, Theaterleiter/Theaterleiterin}|pw} ist
               am Sonntag zu Mittag bei mir. Es wäre
               gut, wenn ich bis dahin eine Zeile von der Sandrock\pwindex{Sandrock, Adele 1863-08-19 – 1937-08-30@\textsc{Sandrock, Adele} (1863-08-19 – 1937-08-30), \emph{Schauspieler/Schauspielerin}|pw} hätte. Auf den Semmering\oindex{Semmering@\textbf{Semmering}, \emph{A.ADM3}|pw} kann
               ich leider nicht. Wir wünschen Frau \label{K_L03552-3v}\edtext{Olga\pwindex{Schnitzler, Olga 17.01.1882 – 13.01.1970@\textsc{Schnitzler, Olga} (17.01.1882 – 13.01.1970), \emph{Schauspieler/Schauspielerin, Sänger/Sängerin}|pw} schöne Erholung}{\lemma{\textnormal{\emph{Olga schöne Erholung}}}\Cendnote{\textnormal{Siehe A. S.: \emph{Tagebuch}, 7. 10. 1910. }}}\label{K_L03552-3} und
               Ihnen Beiden\pwindex{Schnitzler, Olga 17.01.1882 – 13.01.1970@\textsc{Schnitzler, Olga} (17.01.1882 – 13.01.1970), \emph{Schauspieler/Schauspielerin, Sänger/Sängerin}|pwv} gutes
               Wetter!\pend
           
\pstart
           Herzlich von uns\pwindex{Salten, Ottilie 07.03.1868 – 22.06.1942@\textsc{Salten, Ottilie} (07.03.1868 – 22.06.1942), \emph{Schauspieler/Schauspielerin}|pwv} zu
               Ihnen {\\[\baselineskip]}Ihr {\\[\baselineskip]}\spacefill\mbox{Salten}\pend
           \leftskip=0em{}\selectlanguage{ngerman}\endnumbering\briefempfaengerindex{Schnitzler, Arthur@\textsc{Schnitzler, Arthur}!zzzSalten, Felix@\emph{von Felix Salten}!1910-10-141@{14. 10. 1910}|)be}\mylabel{L03552h}  \normalsize

\doendnotes{C}
\bigskip
\vfill

\clearpage

\footnotesize

\lohead{\textsc{register}}

% Definiere theindex-Environment komplett neu ohne reledmac
\makeatletter
\renewenvironment{theindex}{%
  \section*{\indexname}%
  \setlength{\parindent}{0pt}%
  \setlength{\parskip}{0pt plus 0.3pt}%
  \let\item\@idxitem
}{%
  \clearpage
}
\makeatother

\IfFileExists{\jobname-pw.ind}{\input{\jobname-pw.ind}}{}

\end{document}

      