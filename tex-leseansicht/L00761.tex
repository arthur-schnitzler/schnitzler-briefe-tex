%% latex-leseansicht-vorspann.tex
%% Vorspann für die Leseansicht.
%% Lädt die gemeinsame Datei latex-vorspann.tex mit nicht gesetztem Schalter.

\newif\ifkorrekturansicht
\korrekturansichtfalse

\input{../tex-inputs/latex-vorspann}


\section[Hugo von Hofmannsthal an Arthur Schnitzler, {{[}}10.? 1. 1898{{]}}]{L00761 Hugo von Hofmannsthal an Arthur Schnitzler, {[}10.? 1. 1898{]}}
\nopagebreak\mylabel{L00761v}
\rehead{ }\normalsize\beginnumbering\briefempfaengerindex{Schnitzler, Arthur@\textsc{Schnitzler, Arthur}!zzzHofmannsthal, Hugo von@\emph{von Hugo von Hofmannsthal}!1898-01-101@{{[}10.? 1. 1898{]}}|(be}
\toendnotes[C]{\smallbreak\pagebreak[2]}
\correspDesc{Versand  durch Hugo von Hofmannsthal am [10.? 1. 1898] in Wien
\newline{}Erhalt  durch Arthur Schnitzler im Zeitraum [10. 1. 1898
                  – 14. 1. 1898?] in Wien}\toendnotes[C]{\smallbreak}
\Standort{CUL, Schnitzler, B 43.}
\physDesc{Briefkarte, 419 Zeichen
\newline{}Handschrift: schwarze Tinte, deutsche Kurrent
\newline{}Schnitzler: mit Bleistift datiert: »? Jann 98« 
\newline{}Ordnung: mit Bleistift von unbekannter Hand nummeriert: »104« }
\buchAbdrucke{\weitereDrucke{Hugo von Hofmannsthal, Arthur Schnitzler: \emph{Briefwechsel}. Herausgegeben von Therese Nickl und Heinrich Schnitzler. Frankfurt am Main: \emph{S. Fischer} 1964, S. 98.} }\toendnotes[C]{\smallbreak}
\pstart
           \raggedleft{}{\pb}\label{K_L00761-1v}\edtext{Montag}{\lemma{\textnormal{\emph{Montag}}}\Cendnote{\textnormal{Am 5. 1. 1898 wiederholte Brahm\pwindex{Brahm, Otto 5.\,2.\,1856 Hamburg – 28.\,11.\,1912 Berlin@\textsc{Brahm, Otto} (5.\,2.\,1856 Hamburg – 28.\,11.\,1912 Berlin), \emph{Theaterleiter, Regisseur}|pwk} in einem Brief an Schnitzler, dass er \emph{Der Kaiser und Hexe}\pwindex{Hofmannsthal, Hugo von 1.\,2.\,1874 Wien – 15.\,7.\,1929 Rodaun@\textsc{Hofmannsthal, Hugo von} (1.\,2.\,1874 Wien – 15.\,7.\,1929 Rodaun), \emph{Schriftsteller}!Kaiser und die Hexe@\strich\emph{Der Kaiser und die Hexe}|pwk} für misslungen halte. Er hatte sich also seine
                        Meinung gebildet, wenngleich sich das so lesen lässt, dass diese noch nicht
                        kommuniziert worden war. Dieser Brief könnte somit am darauffolgenden Montag
                        geschrieben worden sein. Ein Brief Brahms\pwindex{Brahm, Otto 5.\,2.\,1856 Hamburg – 28.\,11.\,1912 Berlin@\textsc{Brahm, Otto} (5.\,2.\,1856 Hamburg – 28.\,11.\,1912 Berlin), \emph{Theaterleiter, Regisseur}|pwk} an
                           Hofmannsthal\pwindex{Hofmannsthal, Hugo von 1.\,2.\,1874 Wien – 15.\,7.\,1929 Rodaun@\textsc{Hofmannsthal, Hugo von} (1.\,2.\,1874 Wien – 15.\,7.\,1929 Rodaun), \emph{Schriftsteller}|pwk}, in dem er seine
                        Absage mitteilt, ist nicht bekannt. }}}\label{K_L00761-1}\pend
           
\pstart{}mein lieber Arthur,\pend\vspace{0.5em}
\pstart
           »Kaiſer und Hexe\pwindex{Hofmannsthal, Hugo von 1.\,2.\,1874 Wien – 15.\,7.\,1929 Rodaun@\textsc{Hofmannsthal, Hugo von} (1.\,2.\,1874 Wien – 15.\,7.\,1929 Rodaun), \emph{Schriftsteller}!Kaiser und die Hexe@\strich\emph{Der Kaiser und die Hexe}|pw}« gefällt Brahm\pwindex{Brahm, Otto 5.\,2.\,1856 Hamburg – 28.\,11.\,1912 Berlin@\textsc{Brahm, Otto} (5.\,2.\,1856 Hamburg – 28.\,11.\,1912 Berlin), \emph{Theaterleiter, Regisseur}|pw} nicht{ }ſehr (offenbar) und er wird es \uline{nicht}{ }ſpielen.\pend
           
\pstart
           Die künftigen Beziehungen der \textsc{Sorma}\pwindex{Sorma, Agnes 17.\,5.\,1862 Breslau – 10.\,2.\,1927 Crown King@\textsc{Sorma, Agnes} (17.\,5.\,1862 Breslau – 10.\,2.\,1927 Crown King), \emph{Schauspielerin}|pw} zum »Deutſchen Theater\oindex{Deutsches Theater Berlin@\textbf{Deutsches Theater Berlin}, \emph{Theater}|pw}«{ }ſind{ }ſehr unſicher;
               er denkt {\pb}alſo daran, die beiden
               anderen Stücke\pwindex{Hofmannsthal, Hugo von 1.\,2.\,1874 Wien – 15.\,7.\,1929 Rodaun@\textsc{Hofmannsthal, Hugo von} (1.\,2.\,1874 Wien – 15.\,7.\,1929 Rodaun), \emph{Schriftsteller}!Frau im Fenster@\strich\emph{Die Frau im Fenster}|pwv}\pwindex{Hofmannsthal, Hugo von 1.\,2.\,1874 Wien – 15.\,7.\,1929 Rodaun@\textsc{Hofmannsthal, Hugo von} (1.\,2.\,1874 Wien – 15.\,7.\,1929 Rodaun), \emph{Schriftsteller}!Hochzeit der Sobeide@\strich\emph{Die Hochzeit der Sobeide}|pwv} oder
               nur die »junge Frau\pwindex{Hofmannsthal, Hugo von 1.\,2.\,1874 Wien – 15.\,7.\,1929 Rodaun@\textsc{Hofmannsthal, Hugo von} (1.\,2.\,1874 Wien – 15.\,7.\,1929 Rodaun), \emph{Schriftsteller}!Frau im Fenster@\strich\emph{Die Frau im Fenster}|pw}« mit einem (fremden)
               Einacter heuer, ohne die \textsc{Sorma}\pwindex{Sorma, Agnes 17.\,5.\,1862 Breslau – 10.\,2.\,1927 Crown King@\textsc{Sorma, Agnes} (17.\,5.\,1862 Breslau – 10.\,2.\,1927 Crown King), \emph{Schauspielerin}|pw}, zu{ }ſpielen etc{\dots} lauter unangenehme Sachen, worüber
               weiter nichts zu reden. Morgen{ }abend bin \uline{leider} nicht frei.\pend
           \pstart Ihr\spacefill\mbox{Hugo.}\pend{}\selectlanguage{ngerman}\endnumbering\briefempfaengerindex{Schnitzler, Arthur@\textsc{Schnitzler, Arthur}!zzzHofmannsthal, Hugo von@\emph{von Hugo von Hofmannsthal}!1898-01-101@{{[}10.? 1. 1898{]}}|)be}\mylabel{L00761h}  \newcommand{\dateiname}{L00761}\newcommand{\titel}{Hugo von Hofmannsthal an Arthur Schnitzler, [10.? 1. 1898]}\newcommand{\editorInnen}{Martin Anton Müller und Gerd-Hermann Susen}%% latex-leseansicht-abspann.tex
%% Abspann für die Leseansicht.
%% Der Schalter \ifkorrekturansicht ist bereits durch den Vorspann gesetzt.

%% latex-abspann.tex
%% Gemeinsamer Abspann für Korrekturansicht und Leseansicht.
%% Setzt den Schalter \ifkorrekturansicht voraus (gesetzt in den
%% einbindenden Dateien latex-korrekturansicht-abspann.tex bzw.
%% latex-leseansicht-abspann.tex).
%% ---------------------------------------------------------------

\normalsize

% Das esempio-Environment wird nur in der Leseansicht benötigt
\ifkorrekturansicht\else
\newenvironment{esempio}[3]%
{
    \vspace{1.5ex}
    \rlap{\underline{#1}}
    \par
    \setlength{\parindent}{0cm}
    \nopagebreak
    \leftskip=#2cm
    \rightskip=#3cm
}
{
    \par
}
\fi

\doendnotes{C}
\bigskip
\vfill

\clearpage

\footnotesize

\ifkorrekturansicht
  \lohead{\textsc{register}}
\fi

% theindex-Environment neu definieren ohne reledmac
\makeatletter
\renewenvironment{theindex}{%
  \ifkorrekturansicht
    \section*{\indexname}%
  \else
    \subsubsection*{Index der erwähnten Entitäten}%
  \fi
  \setlength{\parindent}{0pt}%
  \setlength{\parskip}{0pt plus 0.3pt}%
  \let\item\@idxitem
}{%
  \ifkorrekturansicht\clearpage\fi
}
\makeatother

\IfFileExists{\jobname-pw.ind}{\input{\jobname-pw.ind}}{}

% Quellenangabe nur in der Leseansicht
\ifkorrekturansicht\else
% Fallback-Definitionen, falls die .tex-Datei \titel etc. nicht gesetzt hat
\providecommand{\titel}{}
\providecommand{\editorInnen}{}
\providecommand{\dateiname}{\jobname}

\vspace{3cm}

\vfill

\footnotesize
\textsc{Quelle}: \titel. Herausgegeben von {\editorInnen}. In: \emph{Arthur Schnitzler: Briefwechsel mit Autorinnen und Autoren}.
 Digitale Edition, https://schnitzler-briefe.acdh.oeaw.ac.at/{\dateiname}.html (Stand \today)
\fi

\end{document}


