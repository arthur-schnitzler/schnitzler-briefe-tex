%% latex-korrekturansicht-vorspann.tex
%% Vorspann für die Korrekturansicht.
%% Lädt die gemeinsame Datei latex-vorspann.tex mit gesetztem Schalter.

\newif\ifkorrekturansicht
\korrekturansichttrue

\input{../tex-inputs/latex-vorspann}


\section[Hugo von Hofmannsthal an Arthur Schnitzler, {[}10.? 1. 1898{]}]{L00761 Hugo von Hofmannsthal an Arthur Schnitzler, {[}10.? 1. 1898{]}}
\nopagebreak\mylabel{L00761v}
\rehead{ }\normalsize\beginnumbering\briefempfaengerindex{Schnitzler, Arthur@\textsc{Schnitzler, Arthur}!zzzHofmannsthal, Hugo von@\emph{von Hugo von Hofmannsthal}!1898-01-101@{{[}10.? 1. 1898{]}}|(be}
\toendnotes[C]{\smallbreak\pagebreak[2]}\Standort{CUL, Schnitzler, B 43.}
\physDesc{Briefkarte, 419 Zeichen
\newline{}Handschrift: schwarze Tinte, deutsche Kurrent
\newline{}Schnitzler: mit Bleistift datiert: »? Jann 98« 
\newline{}Ordnung: mit Bleistift von unbekannter Hand nummeriert: »104« }
\buchAbdrucke{\weitereDrucke{Hugo von Hofmannsthal, Arthur Schnitzler: \emph{Briefwechsel}. Frankfurt am Main: \emph{S. Fischer} 1964, S. 98.} }\toendnotes[C]{\smallbreak}
\pstart
           \raggedleft{}{\pb}\label{K_L00761-1v}\edtext{Montag}{\lemma{\textnormal{\emph{Montag}}}\Cendnote{\textnormal{Am 5. 1. 1898 wiederholte Brahm\pwindex{Brahm, Otto 05.02.1856 – 28.11.1912@\textsc{Brahm, Otto} (05.02.1856 – 28.11.1912), \emph{Theaterleiter/Theaterleiterin, Regisseur/Regisseurin}|pwk} in einem Brief an Schnitzler, dass er \emph{Der Kaiser und Hexe}\pwindex{Kaiser und die Hexe@\emph{Der Kaiser und die Hexe}|pwk} für misslungen halte. Er hatte sich also seine
                        Meinung gebildet, wenngleich sich das so lesen lässt, dass diese noch nicht
                        kommuniziert worden war. Dieser Brief könnte somit am darauffolgenden Montag
                        geschrieben worden sein. Ein Brief Brahms\pwindex{Brahm, Otto 05.02.1856 – 28.11.1912@\textsc{Brahm, Otto} (05.02.1856 – 28.11.1912), \emph{Theaterleiter/Theaterleiterin, Regisseur/Regisseurin}|pwk} an
                           Hofmannsthal\pwindex{Hofmannsthal, Hugo von 1874-02-01 – 1929-07-15@\textsc{Hofmannsthal, Hugo von} (1874-02-01 – 1929-07-15), \emph{Schriftsteller/Schriftstellerin}|pwk}, in dem er seine
                        Absage mitteilt, ist nicht bekannt. }}}\label{K_L00761-1}\pend
           
\pstart{}mein lieber Arthur,\pend\vspace{0.5em}
\pstart
           »Kaiſer und Hexe\pwindex{Kaiser und die Hexe@\emph{Der Kaiser und die Hexe}|pw}« gefällt Brahm\pwindex{Brahm, Otto 05.02.1856 – 28.11.1912@\textsc{Brahm, Otto} (05.02.1856 – 28.11.1912), \emph{Theaterleiter/Theaterleiterin, Regisseur/Regisseurin}|pw} nicht ſehr (offenbar) und er wird es \uline{nicht}{ }ſpielen.\pend
           
\pstart
           Die künftigen Beziehungen der \textsc{Sorma}\pwindex{Sorma, Agnes 17.05.1862 – 10.02.1927@\textsc{Sorma, Agnes} (17.05.1862 – 10.02.1927), \emph{Schauspieler/Schauspielerin}|pw} zum »Deutſchen Theater\oindex{Deutsches Theater Berlin@\textbf{Deutsches Theater Berlin}, \emph{Theater (K.THE)}|pw}« ſind ſehr unſicher;
               er denkt {\pb}alſo daran, die beiden
               anderen Stücke\pwindex{Frau im Fenster@\emph{Die Frau im Fenster}|pwv}\pwindex{Hochzeit der Sobeide@\emph{Die Hochzeit der Sobeide}|pwv} oder
               nur die »junge Frau\pwindex{Frau im Fenster@\emph{Die Frau im Fenster}|pw}« mit einem (fremden)
               Einacter heuer, ohne die \textsc{Sorma}\pwindex{Sorma, Agnes 17.05.1862 – 10.02.1927@\textsc{Sorma, Agnes} (17.05.1862 – 10.02.1927), \emph{Schauspieler/Schauspielerin}|pw}, zu ſpielen etc{\dots} lauter unangenehme Sachen, worüber
               weiter nichts zu reden. Morgen{ }abend bin \uline{leider} nicht frei.\pend
           \pstart Ihr\spacefill\mbox{Hugo.}\pend{}\selectlanguage{ngerman}\endnumbering\briefempfaengerindex{Schnitzler, Arthur@\textsc{Schnitzler, Arthur}!zzzHofmannsthal, Hugo von@\emph{von Hugo von Hofmannsthal}!1898-01-101@{{[}10.? 1. 1898{]}}|)be}\mylabel{L00761h}  \normalsize

\doendnotes{C}
\bigskip
\vfill

\clearpage

\footnotesize

\lohead{\textsc{register}}

% Definiere theindex-Environment komplett neu ohne reledmac
\makeatletter
\renewenvironment{theindex}{%
  \section*{\indexname}%
  \setlength{\parindent}{0pt}%
  \setlength{\parskip}{0pt plus 0.3pt}%
  \let\item\@idxitem
}{%
  \clearpage
}
\makeatother

\IfFileExists{\jobname-pw.ind}{\input{\jobname-pw.ind}}{}

\end{document}

      