\input{../tex-inputs/latex-pdf-vorspann}
\begin{center}
            \textcolor{red}{ENTWURF. ENTZIFFERUNG NOCH NICHT KORREKTURGELESEN}
                      \end{center}
            
               \section[Hugo von Hofmannsthal an Arthur Schnitzler, {[}10.? 1. 1898{]}]{ Hugo von Hofmannsthal an Arthur Schnitzler, {[}10.? 1. 1898{]}}\nopagebreak\mylabel{v}\rehead{ }\begin{ledgroupsized}[t]{13cm}\normalsize\beginnumbering\briefempfaengerindex{Schnitzler, Arthur@\textsc{Schnitzler, Arthur}!zzzHofmannsthal, Hugo von@\emph{von Hugo von Hofmannsthal}!1898-01-101@{{[}10.? 1. 1898{]}}|(be} \toendnotes[C]{\smallbreak\pagebreak[2]} \Standort{CUL, Schnitzler, B 43.}
\physDesc{Briefkarte
\newline{}Handschrift: schwarze Tinte, deutsche Kurrent
\newline{}Schnitzler: mit Bleistift datiert: »? Jann 98« \newline{}Ordnung: mit Bleistift von unbekannter Hand nummeriert:
                                    »104« }\buchAbdrucke{\weitereDrucke{Hugo von Hofmannsthal, Arthur Schnitzler: \emph{Briefwechsel}. Hg. Therese Nickl und Heinrich Schnitzler. Frankfurt am Main: \emph{S. Fischer} 1964, S. 98.} }\toendnotes[C]{\smallbreak}\pstart
           \raggedleft{}{\pb}\label{K_L00761_1v}\edtext{Montag}{\lemma{\textnormal{\emph{Montag}}}\Cendnote{\textnormal{Am 5. 1. 1898 wiederholt Brahm\pwindex{Brahm, Otto 05.02.1856 – 28.11.1912@\textsc{Brahm, Otto} (05.02.1856 – 28.11.1912), \emph{Theaterleiter, Regisseur}|pwk} in einem Brief an Schnitzler\pwindex{Schnitzler, Arthur 15.05.1862 – 21.10.1931@\textsc{Schnitzler, Arthur} (15.05.1862 – 21.10.1931), \emph{Schriftsteller, Mediziner}|pwk}, dass er \emph{Der Kaiser und
                           Hexe}\pwindex{Hofmannsthal, Hugo von 01.02.1874 – 15.07.1929@\textsc{Hofmannsthal, Hugo von} (01.02.1874 – 15.07.1929), \emph{Schriftsteller}!Kaiser und die Hexe1900@\strich\emph{Der Kaiser und die Hexe} {[}1900{]}|pwk} für misslungen halte. Er hatte sich also seine Meinung
                        gebildet, wenngleich sich das so lesen lässt, dass diese noch nicht
                        kommuniziert war. Entsprechend könnte der Brief am darauffolgenden Montag
                        geschrieben sein.}}}\label{K_L00761_1h}\pend
           \pstart{}mein lieber Arthur,\pend\pstart
           »Kaiſer und Hexe\pwindex{Hofmannsthal, Hugo von 01.02.1874 – 15.07.1929@\textsc{Hofmannsthal, Hugo von} (01.02.1874 – 15.07.1929), \emph{Schriftsteller}!Kaiser und die Hexe1900@\strich\emph{Der Kaiser und die Hexe} {[}1900{]}|pw}« gefällt Brahm\pwindex{Brahm, Otto 05.02.1856 – 28.11.1912@\textsc{Brahm, Otto} (05.02.1856 – 28.11.1912), \emph{Theaterleiter, Regisseur}|pw} nicht ſehr (offenbar) und er wird es \uline{nicht}{ }ſpielen.\pend
           \pstart
           Die künftigen Beziehungen der \textsc{Sorma}\pwindex{Sorma, Agnes 17.05.1862 – 10.02.1927@\textsc{Sorma, Agnes} (17.05.1862 – 10.02.1927), \emph{Schauspielerin}|pw} zum »Deutſchen Theater\oindex{Deutsches Theater Berlin@\textbf{Deutsches Theater Berlin}|pw}« ſind ſehr unſicher; er
               denkt {\pb}alſo daran, die beiden
               anderen Stücke\pwindex{Hofmannsthal, Hugo von 01.02.1874 – 15.07.1929@\textsc{Hofmannsthal, Hugo von} (01.02.1874 – 15.07.1929), \emph{Schriftsteller}!Frau im Fenster15. 5. 1898@\strich\emph{Die Frau im Fenster} {[}15. 5. 1898{]}|pwv}\pwindex{Hofmannsthal, Hugo von 01.02.1874 – 15.07.1929@\textsc{Hofmannsthal, Hugo von} (01.02.1874 – 15.07.1929), \emph{Schriftsteller}!Hochzeit der Sobeide1899@\strich\emph{Die Hochzeit der Sobeide} {[}1899{]}|pwv} oder nur
               die »junge Frau\pwindex{Hofmannsthal, Hugo von 01.02.1874 – 15.07.1929@\textsc{Hofmannsthal, Hugo von} (01.02.1874 – 15.07.1929), \emph{Schriftsteller}!Frau im Fenster15. 5. 1898@\strich\emph{Die Frau im Fenster} {[}15. 5. 1898{]}|pw}« mit einem (fremden) Einacter
               heuer, ohne die \textsc{Sorma}\pwindex{Sorma, Agnes 17.05.1862 – 10.02.1927@\textsc{Sorma, Agnes} (17.05.1862 – 10.02.1927), \emph{Schauspielerin}|pw}, zu ſpielen etc{\dots} lauter unangenehme Sachen, worüber
               weiter nichts zu reden. Morgen{ }abend bin \uline{leider} nicht frei.\pend
           \pstart Ihr\spacefill\mbox{Hugo.}\pend{}\endnumbering\briefempfaengerindex{Schnitzler, Arthur@\textsc{Schnitzler, Arthur}!zzzHofmannsthal, Hugo von@\emph{von Hugo von Hofmannsthal}!1898-01-101@{{[}10.? 1. 1898{]}}|)be}\mylabel{h}\end{ledgroupsized}  \newcommand{\dateiname}{L00761}\newcommand{\titel}{Hugo von Hofmannsthal an Arthur Schnitzler, [10.? 1. 1898]}\newcommand{\editorInnen}{Martin Anton Müller und Gerd-Hermann Susen}\input{../tex-inputs/latex-pdf-abspann}
      