%% latex-leseansicht-vorspann.tex
%% Vorspann für die Leseansicht.
%% Lädt die gemeinsame Datei latex-vorspann.tex mit nicht gesetztem Schalter.

\newif\ifkorrekturansicht
\korrekturansichtfalse

\input{../tex-inputs/latex-vorspann}


         
         \newcommand{\erwaehntePersonen}{Personen: }
         \newcommand{\erwaehnteInstitutionen}{}
         \newcommand{\erwaehnteOrte}{}
         \newcommand{\erwaehnteWerke}{
               \section[Hugo von Hofmannsthal an Arthur Schnitzler, {[}10.? 1. 1898{]}]{ Hugo von Hofmannsthal an Arthur Schnitzler, {[}10.? 1. 1898{]}}\nopagebreak\mylabel{v}\rehead{ }\begin{ledgroupsized}[t]{13cm}\normalsize\beginnumbering \toendnotes[C]{\smallbreak\pagebreak[2]} \Standort{CUL, Schnitzler, B 43.}
\physDesc{Briefkarte
\newline{}Handschrift: schwarze Tinte, deutsche Kurrent
\newline{}Schnitzler: mit Bleistift datiert: »? Jann 98« \newline{}Ordnung: mit Bleistift von unbekannter Hand nummeriert:
                                    »104« }\buchAbdrucke{\weitereDrucke{Hugo von Hofmannsthal, Arthur Schnitzler: \emph{Briefwechsel}. Hg. Therese Nickl und Heinrich Schnitzler. Frankfurt am Main: \emph{S. Fischer} 1964, S. 98.} }\toendnotes[C]{\smallbreak}\pstart
           \raggedleft{}{\pb}\label{K_L00761_1v}\edtext{Montag}{\lemma{\textnormal{\emph{Montag}}}\Cendnote{\textnormal{Am 5. 1. 1898 wiederholt Brahm\pwindex{\textcolor{red}{\textsuperscript{XXXX1 indx}}|pwk} in einem Brief an Schnitzler\pwindex{\textcolor{red}{\textsuperscript{XXXX1 indx}}|pwk}, dass er \emph{Der Kaiser und
                           Hexe}\textcolor{red}{\textsuperscript{XXXX indx}} für misslungen halte. Er hatte sich also seine Meinung
                        gebildet, wenngleich sich das so lesen lässt, dass diese noch nicht
                        kommuniziert war. Dieser Brief könnte somit am darauffolgenden Montag
                        geschrieben sein. Ein Brief Brahm\pwindex{\textcolor{red}{\textsuperscript{XXXX1 indx}}|pwk}s an Hofmannsthal\pwindex{\textcolor{red}{\textsuperscript{XXXX1 indx}}|pwk}, in dem
                        er seine Absage mitteilt, ist 
                         nicht bekannt. }}}\label{K_L00761_1h}\pend
           \pstart{}mein lieber Arthur,\pend\pstart
           »Kaiſer und Hexe\textcolor{red}{\textsuperscript{XXXX indx}}« gefällt Brahm\pwindex{\textcolor{red}{\textsuperscript{XXXX1 indx}}|pw} nicht ſehr (offenbar) und er wird es \uline{nicht}{ }ſpielen.\pend
           \pstart
           Die künftigen Beziehungen der \textsc{Sorma}\pwindex{\textcolor{red}{\textsuperscript{XXXX1 indx}}|pw} zum »Deutſchen Theater\oindex{XXXX Ortsangabe fehlt|pw}« ſind ſehr unſicher; er
               denkt {\pb}alſo daran, die beiden
               anderen Stücke\textcolor{red}{\textsuperscript{XXXX indx}}\textcolor{red}{\textsuperscript{XXXX indx}} oder nur
               die »junge Frau\textcolor{red}{\textsuperscript{XXXX indx}}« mit einem (fremden) Einacter
               heuer, ohne die \textsc{Sorma}\pwindex{\textcolor{red}{\textsuperscript{XXXX1 indx}}|pw}, zu ſpielen etc{\dots} lauter unangenehme Sachen, worüber
               weiter nichts zu reden. Morgen{ }abend bin \uline{leider} nicht frei.\pend
           \pstart Ihr\spacefill\mbox{Hugo.}\pend{}
         
         \endnumbering\mylabel{h}\end{ledgroupsized}  \newcommand{\dateiname}{L00761}\newcommand{\titel}{Hugo von Hofmannsthal an Arthur Schnitzler, [10.? 1. 1898]}\newcommand{\editorInnen}{Martin Anton Müller und Gerd-Hermann Susen}%% latex-leseansicht-abspann.tex
%% Abspann für die Leseansicht.
%% Der Schalter \ifkorrekturansicht ist bereits durch den Vorspann gesetzt.

%% latex-abspann.tex
%% Gemeinsamer Abspann für Korrekturansicht und Leseansicht.
%% Setzt den Schalter \ifkorrekturansicht voraus (gesetzt in den
%% einbindenden Dateien latex-korrekturansicht-abspann.tex bzw.
%% latex-leseansicht-abspann.tex).
%% ---------------------------------------------------------------

\normalsize

% Das esempio-Environment wird nur in der Leseansicht benötigt
\ifkorrekturansicht\else
\newenvironment{esempio}[3]%
{
    \vspace{1.5ex}
    \rlap{\underline{#1}}
    \par
    \setlength{\parindent}{0cm}
    \nopagebreak
    \leftskip=#2cm
    \rightskip=#3cm
}
{
    \par
}
\fi

\doendnotes{C}
\bigskip
\vfill

\clearpage

\footnotesize

\ifkorrekturansicht
  \lohead{\textsc{register}}
\fi

% theindex-Environment neu definieren ohne reledmac
\makeatletter
\renewenvironment{theindex}{%
  \ifkorrekturansicht
    \section*{\indexname}%
  \else
    \subsubsection*{Index der erwähnten Entitäten}%
  \fi
  \setlength{\parindent}{0pt}%
  \setlength{\parskip}{0pt plus 0.3pt}%
  \let\item\@idxitem
}{%
  \ifkorrekturansicht\clearpage\fi
}
\makeatother

\IfFileExists{\jobname-pw.ind}{\input{\jobname-pw.ind}}{}

% Quellenangabe nur in der Leseansicht
\ifkorrekturansicht\else
% Fallback-Definitionen, falls die .tex-Datei \titel etc. nicht gesetzt hat
\providecommand{\titel}{}
\providecommand{\editorInnen}{}
\providecommand{\dateiname}{\jobname}

\vspace{3cm}

\vfill

\footnotesize
\textsc{Quelle}: \titel. Herausgegeben von {\editorInnen}. In: \emph{Arthur Schnitzler: Briefwechsel mit Autorinnen und Autoren}.
 Digitale Edition, https://schnitzler-briefe.acdh.oeaw.ac.at/{\dateiname}.html (Stand \today)
\fi

\end{document}


      