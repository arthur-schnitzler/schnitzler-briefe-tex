%% latex-leseansicht-vorspann.tex
%% Vorspann für die Leseansicht.
%% Lädt die gemeinsame Datei latex-vorspann.tex mit nicht gesetztem Schalter.

\newif\ifkorrekturansicht
\korrekturansichtfalse

\input{../tex-inputs/latex-vorspann}


\section[Arthur Schnitzler und Paul Goldmann an Richard Beer-Hofmann, 18. 9. 1893]{L00264 Arthur Schnitzler und Paul Goldmann an Richard Beer-Hofmann, 18. 9. 1893}
\nopagebreak\mylabel{L00264v}
\rehead{ }\normalsize\beginnumbering\briefempfaengerindex{Beer-Hofmann, Richard@\textsc{Beer-Hofmann, Richard}!zzzGoldmann, Paul@\emph{von Paul Goldmann}!1893-09-181@{18. 9. 1893}|(be}\briefempfaengerindex{Beer-Hofmann, Richard@\textsc{Beer-Hofmann, Richard}!zzzSchnitzler, Arthur@\emph{von Arthur Schnitzler}!1893-09-181@{18. 9. 1893}|(be}
\toendnotes[C]{\smallbreak\pagebreak[2]}
\correspDesc{Versand  durch Arthur Schnitzler, Paul Goldmann am 18. 9. 1893 in Salzburg
\newline{}Umleitung  am 25. 9. 1893 in Znaim
\newline{}Erhalt  durch Richard Beer-Hofmann am 25. 9. 1893 in Wien}\toendnotes[C]{\smallbreak}
\Standort{YCGL, MSS 31.}
\physDesc{Brief, 1 Blatt, 2 Seiten, Kuvert, 577 Zeichen (Briefpapier und Umschlag mit Trauerrand )
\newline{}Handschrift Arthur Schnitzler: Bleistift, deutsche Kurrent
\newline{}Handschrift Paul Goldmann: Bleistift, deutsche Kurrent
\newline{}Versand: 1) Stempel: »\nobreak{}\oindex{Salzburg@\textbf{Salzburg}, \emph{Verwaltungsgebiet}|pwk}{\pb}Salzburg Stadt, 18/9 93, 2N\nobreak{}«.   2) Stempel: »\nobreak{}\oindex{Znaim@\textbf{Znaim}, \emph{Hauptstadt}|pwk}Znaim, 25/\textcolor{gray}{9} 93, 8–10V\nobreak{}«.  3) Stempel: »\nobreak{}\oindex{I., Innere Stadt@\textbf{I., Innere Stadt}, \emph{Verwaltungsgebiet}|pwk}Wien 1/1, 25 9. 93, 5–6½ N, Bestellt\nobreak{}«.  4) mit schwarzer Tinte von unbekannter Hand Empfängeradresse geändert zu: »\textsc{Wollzeile Nro. 15\oindex{Wien@\textbf{Wien}!I., Innere Stadt@\textbf{I., Innere Stadt}!Wollzeile 15 (»Berthahof«)@\textbf{Wollzeile 15 (»Berthahof«)}, \emph{Wohngebäude}|pw}{ }Wien\oindex{Wien@\textbf{Wien}, \emph{Verwaltungsgebiet}|pw}}«}\pstart{}{\pb}\textsc{Herrn Dr. Richard Beer-Hofmann}\pend{}\pstart{}kk. Lieutenant a d Reſ. des Kuk Infanterie-Regim. \textsc{Nr. 99}\pend{}\pstart{}\textsc{Znaim\oindex{Znaim@\textbf{Znaim}, \emph{Hauptstadt}|pw}}\pend{}{\bigskip}\vspace{1em}
\pstart
           \raggedleft{}{\pb}\textsc{Salzburg}\oindex{Salzburg@\textbf{Salzburg}, \emph{Verwaltungsgebiet}|pw} 18. 9. 93\pend
           
\pstart{}Lieber Richard,\pend\vspace{0.5em}
\pstart
           wir{ }ſitzen im \textsc{Café Tomaselli}\oindex{Café Tomaselli@\textbf{Café Tomaselli}, \emph{Kaffeehaus}|pw} und grüßen Sie herzlich.\pend
           \pstart \spacefill\mbox{Arthur}\pend{}\selectlanguage{ngerman}\vspace{1em}
\pstart
           \noindent{}{[}hs. Goldmann:{]} Liebſter Freund!\pend
           
\pstart
           Wir feiern{ }ſeit geſtern das große Erinnerungsfeſt. Ich weiß nun alles – bis auf
               Deinen Hund und Deine Cravatten. Es iſt{ }ſo{ }ſchön, \strikeout{bei}{ }{\pb}beiſammen zu{ }ſein!\pend
           
\pstart
           Ich kann leider nicht nach Wien\oindex{Wien@\textbf{Wien}, \emph{Verwaltungsgebiet}|pw}, aber Du mußt nach
                  \textsc{Paris}\oindex{Paris@\textbf{Paris}, \emph{Hauptstadt}|pw}. Du wirſt mir darauf, wie gewöhnlich, nicht antworten. Das macht nichts. Aber
               ich \strikeout{er} erwarte Dich in \textsc{Paris}\oindex{Paris@\textbf{Paris}, \emph{Hauptstadt}|pw}, nächſtens,{ }ſo nächſtens als möglich. Ja? Treuen Gruß!\pend
           
\pstart
           Dein{\\[\baselineskip]}\spacefill\mbox{Paul Goldmann.}\pend
           \leftskip=0em{}\selectlanguage{ngerman}\endnumbering\briefempfaengerindex{Beer-Hofmann, Richard@\textsc{Beer-Hofmann, Richard}!zzzGoldmann, Paul@\emph{von Paul Goldmann}!1893-09-181@{18. 9. 1893}|)be}\briefempfaengerindex{Beer-Hofmann, Richard@\textsc{Beer-Hofmann, Richard}!zzzSchnitzler, Arthur@\emph{von Arthur Schnitzler}!1893-09-181@{18. 9. 1893}|)be}\mylabel{L00264h}  \newcommand{\dateiname}{L00264}\newcommand{\titel}{Arthur Schnitzler und Paul Goldmann an Richard Beer-Hofmann, 18. 9. 1893}\newcommand{\editorInnen}{Martin Anton Müller und Gerd-Hermann Susen}%% latex-leseansicht-abspann.tex
%% Abspann für die Leseansicht.
%% Der Schalter \ifkorrekturansicht ist bereits durch den Vorspann gesetzt.

%% latex-abspann.tex
%% Gemeinsamer Abspann für Korrekturansicht und Leseansicht.
%% Setzt den Schalter \ifkorrekturansicht voraus (gesetzt in den
%% einbindenden Dateien latex-korrekturansicht-abspann.tex bzw.
%% latex-leseansicht-abspann.tex).
%% ---------------------------------------------------------------

\normalsize

% Das esempio-Environment wird nur in der Leseansicht benötigt
\ifkorrekturansicht\else
\newenvironment{esempio}[3]%
{
    \vspace{1.5ex}
    \rlap{\underline{#1}}
    \par
    \setlength{\parindent}{0cm}
    \nopagebreak
    \leftskip=#2cm
    \rightskip=#3cm
}
{
    \par
}
\fi

\doendnotes{C}
\bigskip
\vfill

\clearpage

\footnotesize

\ifkorrekturansicht
  \lohead{\textsc{register}}
\fi

% theindex-Environment neu definieren ohne reledmac
\makeatletter
\renewenvironment{theindex}{%
  \ifkorrekturansicht
    \section*{\indexname}%
  \else
    \subsubsection*{Index der erwähnten Entitäten}%
  \fi
  \setlength{\parindent}{0pt}%
  \setlength{\parskip}{0pt plus 0.3pt}%
  \let\item\@idxitem
}{%
  \ifkorrekturansicht\clearpage\fi
}
\makeatother

\IfFileExists{\jobname-pw.ind}{\input{\jobname-pw.ind}}{}

% Quellenangabe nur in der Leseansicht
\ifkorrekturansicht\else
% Fallback-Definitionen, falls die .tex-Datei \titel etc. nicht gesetzt hat
\providecommand{\titel}{}
\providecommand{\editorInnen}{}
\providecommand{\dateiname}{\jobname}

\vspace{3cm}

\vfill

\footnotesize
\textsc{Quelle}: \titel. Herausgegeben von {\editorInnen}. In: \emph{Arthur Schnitzler: Briefwechsel mit Autorinnen und Autoren}.
 Digitale Edition, https://schnitzler-briefe.acdh.oeaw.ac.at/{\dateiname}.html (Stand \today)
\fi

\end{document}


