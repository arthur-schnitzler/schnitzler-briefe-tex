%% latex-korrekturansicht-vorspann.tex
%% Vorspann für die Korrekturansicht.
%% Lädt die gemeinsame Datei latex-vorspann.tex mit gesetztem Schalter.

\newif\ifkorrekturansicht
\korrekturansichttrue

\input{../tex-inputs/latex-vorspann}


\section[Arthur Schnitzler und Paul Goldmann an Richard Beer-Hofmann, 18. 9. 1893]{L00264 Arthur Schnitzler und Paul Goldmann an Richard Beer-Hofmann,
               18. 9. 1893}
\nopagebreak\mylabel{L00264v}
\rehead{ }\normalsize\beginnumbering\briefempfaengerindex{Beer-Hofmann, Richard@\textsc{Beer-Hofmann, Richard}!zzzGoldmann, Paul@\emph{von Paul Goldmann}!1893-09-181@{18. 9. 1893}|(be}\briefempfaengerindex{Beer-Hofmann, Richard@\textsc{Beer-Hofmann, Richard}!zzzSchnitzler, Arthur@\emph{von Arthur Schnitzler}!1893-09-181@{18. 9. 1893}|(be}
\toendnotes[C]{\smallbreak\pagebreak[2]}\Standort{YCGL, MSS 31.}
\physDesc{Brief, 1 Blatt, 2 Seiten, Umschlag, 577 Zeichen (Briefpapier und Umschlag mit Trauerrand )
\newline{}Handschrift Arthur Schnitzler: Bleistift, deutsche Kurrent
\newline{}Handschrift Paul Goldmann: Bleistift, deutsche Kurrent
\newline{}Versand: 1) Stempel: »\nobreak{}\oindex{Salzburg@\textbf{Salzburg}, \emph{A.ADM2}|pwk}{\pb}Salzburg Stadt, 18/9 93, 2N\nobreak{}«.   2) Stempel: »\nobreak{}\oindex{Znaim@\textbf{Znaim}, \emph{P.PPLA2}|pwk}Znaim, 25/\textcolor{gray}{9} 93, 8–10V\nobreak{}«.  3) Stempel: »\nobreak{}\oindex{I., Innere Stadt@\textbf{I., Innere Stadt}, \emph{A.ADM3}|pwk}Wien 1/1, 25 9. 93, 5–6½ N, Bestellt\nobreak{}«.  4) mit schwarzer Tinte von unbekannter Hand Empfängeradresse geändert zu: »\textsc{Wollzeile Nro. 15\oindex{Wollzeile@\textbf{Wollzeile}, \emph{Straße (K.STR)}|pw}{ }Wien\oindex{Wien@\textbf{Wien}, \emph{A.ADM2}|pw}}«}\pstart{}{\pb}\textsc{Herrn Dr. Richard Beer-Hofmann}\pend{}\pstart{}kk. Lieutenant a d Reſ. des Kuk Infanterie-Regim. \textsc{Nr. 99}\pend{}\pstart{}\textsc{Znaim\oindex{Znaim@\textbf{Znaim}, \emph{P.PPLA2}|pw}}\pend{}{\bigskip}\vspace{1em}
\pstart
           \raggedleft{}{\pb}\textsc{Salzburg}\oindex{Salzburg@\textbf{Salzburg}, \emph{A.ADM2}|pw} 18. 9. 93\pend
           
\pstart{}Lieber Richard,\pend\vspace{0.5em}
\pstart
           wir ſitzen im \textsc{Café Tomaselli}\oindex{Cafe Tomaselli@\textbf{Café Tomaselli}, \emph{Kaffeehaus (K.KAF)}|pw} und grüßen Sie herzlich.\pend
           \pstart \spacefill\mbox{Arthur}\pend{}\selectlanguage{ngerman}\vspace{1em}
\pstart
           \noindent{}{[}hs. :{]} Liebſter Freund!\pend
           
\pstart
           Wir feiern ſeit geſtern das große Erinnerungsfeſt. Ich weiß nun alles – bis auf
               Deinen Hund und Deine Cravatten. Es iſt ſo ſchön, \strikeout{bei}{ }{\pb}beiſammen zu ſein!\pend
           
\pstart
           Ich kann leider nicht nach Wien\oindex{Wien@\textbf{Wien}, \emph{A.ADM2}|pw}, aber Du mußt nach
                  \textsc{Paris}\oindex{Paris@\textbf{Paris}, \emph{P.PPLC}|pw}. Du wirſt mir darauf, wie gewöhnlich, nicht antworten. Das macht nichts. Aber
               ich \strikeout{er} erwarte Dich in \textsc{Paris}\oindex{Paris@\textbf{Paris}, \emph{P.PPLC}|pw}, nächſtens, ſo nächſtens als möglich. Ja? Treuen Gruß!\pend
           
\pstart
           Dein{\\[\baselineskip]}\spacefill\mbox{Paul Goldmann.}\pend
           \leftskip=0em{}\selectlanguage{ngerman}\endnumbering\briefempfaengerindex{Beer-Hofmann, Richard@\textsc{Beer-Hofmann, Richard}!zzzGoldmann, Paul@\emph{von Paul Goldmann}!1893-09-181@{18. 9. 1893}|)be}\briefempfaengerindex{Beer-Hofmann, Richard@\textsc{Beer-Hofmann, Richard}!zzzSchnitzler, Arthur@\emph{von Arthur Schnitzler}!1893-09-181@{18. 9. 1893}|)be}\mylabel{L00264h}  \normalsize

\doendnotes{C}
\bigskip
\vfill

\clearpage

\footnotesize

\lohead{\textsc{register}}

% Definiere theindex-Environment komplett neu ohne reledmac
\makeatletter
\renewenvironment{theindex}{%
  \section*{\indexname}%
  \setlength{\parindent}{0pt}%
  \setlength{\parskip}{0pt plus 0.3pt}%
  \let\item\@idxitem
}{%
  \clearpage
}
\makeatother

\IfFileExists{\jobname-pw.ind}{\input{\jobname-pw.ind}}{}

\end{document}

      