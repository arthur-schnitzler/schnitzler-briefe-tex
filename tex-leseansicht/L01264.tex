%% latex-korrekturansicht-vorspann.tex
%% Vorspann für die Korrekturansicht.
%% Lädt die gemeinsame Datei latex-vorspann.tex mit gesetztem Schalter.

\newif\ifkorrekturansicht
\korrekturansichttrue

\input{../tex-inputs/latex-vorspann}


\section[Hugo von Hofmannsthal an Arthur Schnitzler, 7. 1. {[}1903{]}]{L01264 Hugo von Hofmannsthal an Arthur Schnitzler, 7. 1. {[}1903{]}}
\nopagebreak\mylabel{L01264v}
\rehead{ }\normalsize\beginnumbering\briefempfaengerindex{Schnitzler, Arthur@\textsc{Schnitzler, Arthur}!zzzHofmannsthal, Hugo von@\emph{von Hugo von Hofmannsthal}!1903-01-072@{7. 1. {[}1903{]}}|(be}
\toendnotes[C]{\smallbreak\pagebreak[2]}\Standort{CUL, Schnitzler, B 43.}
\physDesc{Brief, 1 Blatt, 2 Seiten, 416 Zeichen
\newline{}Handschrift: schwarze Tinte, deutsche Kurrent
\newline{}Ordnung: 1) mit Bleistift von unbekannter Hand nummeriert: »\strikeout{210}«  2) mit Bleistift von unbekannter Hand nummeriert:
                                    »192«}
\buchAbdrucke{\weitereDrucke{Hugo von Hofmannsthal, Arthur Schnitzler: \emph{Briefwechsel}. Frankfurt am Main: \emph{S. Fischer} 1964, S. 166.} }\toendnotes[C]{\smallbreak}
\pstart
           \raggedleft{}{\pb}7 I.\pend
           \vspace{0.5em}
\pstart
           lieber, die geſtrige Sache hat für mich ſehr unter der gedrängten
               Zeit gelitten.\hspace*{2.5em}Hier einige Bitten.\pend
           
\pstart
           1.) wie würden Sie mir rathen das Verhältnis zu dem Otway\pwindex{Otway, Thomas 03.03.1652 – 16.04.1685@\textsc{Otway, Thomas} (03.03.1652 – 16.04.1685), \emph{Schriftsteller/Schriftstellerin}|pw}’sſchen{ }Stück\pwindex{Venice Preserv'd@\emph{Venice Preserv'd}|pwv} auf dem Zettel zu
               bezeichnen? \pend
           
\pstart
           2.) welche Beſetzung der 6 großen Rollen (2 Frauen 2 Männer {\pb}2 Senatoren) ſchlagen Sie (mit dem
               vorhandenen Material) für das Burgtheater\orgindex{Burgtheater@Burgtheater|pw}
               vor?\pend
           
\pstart
           Wir wünſchen Ihnen und Olga\pwindex{Schnitzler, Olga 17.01.1882 – 13.01.1970@\textsc{Schnitzler, Olga} (17.01.1882 – 13.01.1970), \emph{Schauspieler/Schauspielerin, Sänger/Sängerin}|pw} viel Freude für den
               kleinen Ausflug.\pend
           
\pstart
           Von Herzen{\\[\baselineskip]}\spacefill\mbox{Hugo.}\pend
           \leftskip=0em{}\selectlanguage{ngerman}\endnumbering\briefempfaengerindex{Schnitzler, Arthur@\textsc{Schnitzler, Arthur}!zzzHofmannsthal, Hugo von@\emph{von Hugo von Hofmannsthal}!1903-01-072@{7. 1. {[}1903{]}}|)be}\mylabel{L01264h}  \normalsize

\doendnotes{C}
\bigskip
\vfill

\clearpage

\footnotesize

\lohead{\textsc{register}}

% Definiere theindex-Environment komplett neu ohne reledmac
\makeatletter
\renewenvironment{theindex}{%
  \section*{\indexname}%
  \setlength{\parindent}{0pt}%
  \setlength{\parskip}{0pt plus 0.3pt}%
  \let\item\@idxitem
}{%
  \clearpage
}
\makeatother

\IfFileExists{\jobname-pw.ind}{\input{\jobname-pw.ind}}{}

\end{document}

      