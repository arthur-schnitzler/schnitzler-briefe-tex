%% latex-korrekturansicht-vorspann.tex
%% Vorspann für die Korrekturansicht.
%% Lädt die gemeinsame Datei latex-vorspann.tex mit gesetztem Schalter.

\newif\ifkorrekturansicht
\korrekturansichttrue

\input{../tex-inputs/latex-vorspann}


\section[Arthur Schnitzler an Bertha von Suttner, 10. 12. 1913]{L02162 Arthur Schnitzler an Bertha von Suttner, 10. 12. 1913}
\nopagebreak\mylabel{L02162v}
\rehead{ }\normalsize\beginnumbering\briefempfaengerindex{Suttner, Bertha von@\textsc{Suttner, Bertha von}!zzzSchnitzler, Arthur@\emph{von Arthur Schnitzler}!1913-12-101@{10. 12. 1913}|(be}
\toendnotes[C]{\smallbreak\pagebreak[2]}\Standort{Genf, United Nations Archives, BvS/27/352-1.}
\physDesc{Briefkarte, 480 Zeichen
\newline{}Handschrift: schwarze Tinte, lateinische Kurrent
\newline{}Ordnung: mit Bleistift von unbekannter Hand beschriftet:
                                    »X14« }\toendnotes[C]{\smallbreak}
\pstart
           {\pb}\textcolor{gray}{\textbf{Dr. Arthur Schnitzler}}\hfill 10. 12. 913\pend
           
\pstart
           \textcolor{gray}{\textbf{Wien XVIII. Sternwartestrasse 71\oindex{Sternwartestrasse 71@\textbf{Sternwartestraße 71}, \emph{Wohngebäude (K.WHS)}|pw}}}\pend
           \vspace{0.5em}
\pstart
           Hochverehrte Frau Baronin,{ }Freitag u. Samstag{ }\damage{s}ind wir leider verhindert. Da nun über\damage{di}es meine Frau\pwindex{Schnitzler, Olga 17.01.1882 – 13.01.1970@\textsc{Schnitzler, Olga} (17.01.1882 – 13.01.1970), \emph{Schauspieler/Schauspielerin, Sänger/Sängerin}|pwv} in der
               nächsten Woche \label{K_L02162-1v}\edtext{oeffent\damage{lic}h zu singen}{\lemma{\textnormal{\emph{oeffentlich zu singen}}}\Cendnote{\textnormal{Siehe A. S.: \emph{Tagebuch}, 16. 12. 1913.
               }}}\label{K_L02162-1} hat, möchten wir sehr bitten, unseren \damage{Be}such in die übernächste Woche verlegen zu dür\damage{fe}n, und werden im Lauf der nächsten anfragen, \damage{w}elcher Tag Ihnen, sehr verehrte Frau Baronin {\pb}genehm
               wäre.\pend
           
\pstart
           Bis dahin empfehlen wir uns Beide\pwindex{Schnitzler, Olga 17.01.1882 – 13.01.1970@\textsc{Schnitzler, Olga} (17.01.1882 – 13.01.1970), \emph{Schauspieler/Schauspielerin, Sänger/Sängerin}|pwv} aufs allerbeste und bitten unserer herzlichsten Verehrung versichert
               zu sein.\pend
           
\pstart
           Ihr sehr ergebner{\\[\baselineskip]}\spacefill\mbox{Arthur Schnitzler}\pend
           \leftskip=0em{}\selectlanguage{ngerman}\endnumbering\briefempfaengerindex{Suttner, Bertha von@\textsc{Suttner, Bertha von}!zzzSchnitzler, Arthur@\emph{von Arthur Schnitzler}!1913-12-101@{10. 12. 1913}|)be}\mylabel{L02162h}  \normalsize

\doendnotes{C}
\bigskip
\vfill

\clearpage

\footnotesize

\lohead{\textsc{register}}

% Definiere theindex-Environment komplett neu ohne reledmac
\makeatletter
\renewenvironment{theindex}{%
  \section*{\indexname}%
  \setlength{\parindent}{0pt}%
  \setlength{\parskip}{0pt plus 0.3pt}%
  \let\item\@idxitem
}{%
  \clearpage
}
\makeatother

\IfFileExists{\jobname-pw.ind}{\input{\jobname-pw.ind}}{}

\end{document}

      