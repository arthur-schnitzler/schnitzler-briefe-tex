%% latex-leseansicht-vorspann.tex
%% Vorspann für die Leseansicht.
%% Lädt die gemeinsame Datei latex-vorspann.tex mit nicht gesetztem Schalter.

\newif\ifkorrekturansicht
\korrekturansichtfalse

\input{../tex-inputs/latex-vorspann}


         
         \renewcommand{\erwaehntePersonen}{Personen: Julius von Gans-Ludassy, Hugo Ganz, Theodor Herzl, Heinrich Kanner, Felix Salten}
         \renewcommand{\erwaehnteInstitutionen}{Institutionen: Bezirksgericht Wien Josefstadt, Concordia, Die Zeit}
         \renewcommand{\erwaehnteOrte}{Orte: Alser Straße, Pötzleinsdorf, Starkfriedgassse, Wien}
         \renewcommand{\erwaehnteWerke}{}
               \section[ Felix Salten an Arthur Schnitzler, {[}18.? 10. 1906{]}]{ Felix Salten an Arthur Schnitzler, {[}18.? 10. 1906{]}}\nopagebreak\mylabel{v}\rehead{ }\begin{ledgroupsized}[t]{13cm}\normalsize\beginnumbering\briefempfaengerindex{Schnitzler, Arthur@\textsc{Schnitzler, Arthur}!zzzSalten, Felix@\emph{von Felix Salten}!1906-10-182@{{[}18.? 10. 1906{]}}|(be} \toendnotes[C]{\smallbreak\pagebreak[2]} \Standort{CUL, Schnitzler, B 89, B 1.}
\physDesc{Brief, 1 Blatt, 3 Seiten, 3912 Zeichen
\newline{}Handschrift: schwarze Tinte, lateinische Kurrent
\newline{}Schnitzler: mit Bleistift datiert: »October 906« 
\newline{}Ordnung: mit Bleistift von unbekannter Hand nummeriert: »225« }\toendnotes[C]{\smallbreak}\pstart
           \raggedleft{}{\pb}\label{K_L03434-1v}\edtext{Donnerstag}{\lemma{\textnormal{\emph{Donnerstag}}}\Cendnote{\textnormal{Schnitzler\pwindex{Schnitzler, Arthur 15.05.1862 – 21.10.1931@\textsc{Schnitzler, Arthur} (15.05.1862 – 21.10.1931), \emph{Schriftsteller, Mediziner}|pwk} datierte das ansonsten nur
                        durch Salten\pwindex{Salten, Felix 06.09.1869 – 08.10.1945@\textsc{Salten, Felix} (06.09.1869 – 08.10.1945), \emph{Schriftsteller, Journalist}|pwk}s Angabe des Wochentags
                        zeitlich näher bestimmte Korrespondenzstück auf »October 906«. Eine weitere Einschränkung ist dadurch möglich, dass am Dienstag, dem 23. 10. 1906, mehrere Zeitungen
                        den Prozessbeginn für den 24. 11. 1906
                        verkündeten. Das deutet auf eine gerichtliche Festsetzung dieses Termins am
                           22. 10. 1906 hin – eben jenem Montag, von
                        dem in diesem Korrespondenzstück die Rede ist. Dass das folgende Schreiben
                           (Felix Salten an Arthur Schnitzler, [20.? 10. 1906]) von einer Verschiebung
                        des Prozessbeginns zeugt, fügt sich problemlos in diesen zeitlichen Ablauf
                        ein.}}}\label{K_L03434-1h}.\pend
           \pstart{}Lieber,\pend\pstart
           Ein Kapitel Ludassy\pwindex{Gans-Ludassy, Julius von 13.04.1858 – 30.09.1922@\textsc{Gans-Ludassy, Julius von} (13.04.1858 – 30.09.1922), \emph{Schriftsteller, Journalist, Herausgeber}|pw}. Es ist langweilig und
               lästig, aber ich muß ein Stückchen Vorgeschichte erwähnen. Wie ich \label{K_L03434-2v}\edtext{mit ihm
               auseinanderkam}{\lemma{\textnormal{\emph{mit ihm
               auseinanderkam}}}\Cendnote{\textnormal{vgl. Felix Salten an Arthur Schnitzler, 9. 3. 1906}}}\label{K_L03434-2h}, wissen Sie ja. Es war der \label{K_L03434-3v}\edtext{Hugo Ganz\pwindex{Ganz, Hugo 24.04.1862 – 02.01.1922@\textsc{Ganz, Hugo} (24.04.1862 – 02.01.1922), \emph{Schriftsteller, Journalist}|pw}-Prozess}{\lemma{\textnormal{\emph{Hugo Ganz-Prozess}}}\Cendnote{\textnormal{Hugo Ganz\pwindex{Ganz, Hugo 24.04.1862 – 02.01.1922@\textsc{Ganz, Hugo} (24.04.1862 – 02.01.1922), \emph{Schriftsteller, Journalist}|pwk} hatte 1903{ }\emph{Die Zeit}\orgindex{Zeit@Die Zeit|pwk} wegen schlechter Behandlung auf
                  Abfertigung geklagt. Die Verhandlungen fanden im Januar 1904 statt (vgl. A. S.: \emph{Tagebuch}, 19. 1. 1904). Die zweite Instanz bestätigte Ende März 1904 das Urteil, demnach Ganz\pwindex{Ganz, Hugo 24.04.1862 – 02.01.1922@\textsc{Ganz, Hugo} (24.04.1862 – 02.01.1922), \emph{Schriftsteller, Journalist}|pwk} 18.000 Kronen zustanden. Der in Folge
                  genannte Heinrich Kanner\pwindex{Kanner, Heinrich 09.11.1864 – 15.02.1930@\textsc{Kanner, Heinrich} (09.11.1864 – 15.02.1930), \emph{Herausgeber, Publizist}|pwk} war der
                  Herausgeber der \emph{Zeit}\orgindex{Zeit@Die Zeit|pwk}, der durch seine
                  schlechten Umgangsformen Ganz\pwindex{Ganz, Hugo 24.04.1862 – 02.01.1922@\textsc{Ganz, Hugo} (24.04.1862 – 02.01.1922), \emph{Schriftsteller, Journalist}|pwk}’ Kündigung
                  bewirkt haben soll.}}}\label{K_L03434-3h} gewesen. Die »Concordia\orgindex{Concordia@Concordia|pw}« ereiferte sich gegen Kanner\pwindex{Kanner, Heinrich 09.11.1864 – 15.02.1930@\textsc{Kanner, Heinrich} (09.11.1864 – 15.02.1930), \emph{Herausgeber, Publizist}|pw}, den ich verteidigte. In der Versammlung saß Ludaßy\pwindex{Gans-Ludassy, Julius von 13.04.1858 – 30.09.1922@\textsc{Gans-Ludassy, Julius von} (13.04.1858 – 30.09.1922), \emph{Schriftsteller, Journalist, Herausgeber}|pw} mit mir an einem Tisch. Ich sagte in meiner Rede, Ludaßy\pwindex{Gans-Ludassy, Julius von 13.04.1858 – 30.09.1922@\textsc{Gans-Ludassy, Julius von} (13.04.1858 – 30.09.1922), \emph{Schriftsteller, Journalist, Herausgeber}|pw} sei als Chef auch heftig gewesen, ohne
               dass die Concordia\orgindex{Concordia@Concordia|pw} u. s. w. Als ich geendigt
               hatte, zischelte mir Ludaßy\pwindex{Gans-Ludassy, Julius von 13.04.1858 – 30.09.1922@\textsc{Gans-Ludassy, Julius von} (13.04.1858 – 30.09.1922), \emph{Schriftsteller, Journalist, Herausgeber}|pw}, der ganz blaß
               war, zu: »Das war geschmacklos und undankbar{\dotstwo}« Ich: »Wofür
               bin ich Ihnen denn Dank schuldig?« Er: »Ich weiß auch Sachen von Ihnen{\dots}« Worauf ich, der ich einerseits fand, es sei vielleicht
               zu viel von mir gewesen, wenn ich bei Gelegenheit Kanner\pwindex{Kanner, Heinrich 09.11.1864 – 15.02.1930@\textsc{Kanner, Heinrich} (09.11.1864 – 15.02.1930), \emph{Herausgeber, Publizist}|pw}s auf Ludaßy\pwindex{Gans-Ludassy, Julius von 13.04.1858 – 30.09.1922@\textsc{Gans-Ludassy, Julius von} (13.04.1858 – 30.09.1922), \emph{Schriftsteller, Journalist, Herausgeber}|pw}’s verjährte
               Brotherren-Grobheit anspielte, andrerseits über die »Sachen«, die er wißen wollte,
               aufgebracht war, ihm sagte: (auch aus versammlungstechnischen Gründen): »Ich werde
               jetzt aussprechen, dass diese Reminiszenz keine Spitze gegen Sie enthielt, und dann
               werden Sie sofort erklären, was Sie von mir wißen.« Er antwortete: »Abgemacht.« Ich
               tat nun meinerseits, wie versprochen. Wie ich ihn aber aufforderte, ja bevor ich ihn
               noch auffordern konnte, nunmehr sein Wort einzulösen, reichte er mir die Hand, mit
               den Worten: »Sei’n wir wieder gut{\dotstwo}« Ich schlug seine Hand
               aus, und begehrte, die »Sachen« zu wißen. Er blieb dabei: »Laßen wir’s gut sein.« Da
               sagte ich ihm, in Erinnerung an manche ähnliche Büberei: »Das ist echt Ihre Art. Wenn
               Sie jetzt nicht sofort mit {\pb}der
               Sprache herausrücken, sind Sie ein feiger Lump{\dots}« oder
                  Kerl {\dotstwo} oder Schuft, oder so was ähnliches. Ludaßy\pwindex{Gans-Ludassy, Julius von 13.04.1858 – 30.09.1922@\textsc{Gans-Ludassy, Julius von} (13.04.1858 – 30.09.1922), \emph{Schriftsteller, Journalist, Herausgeber}|pw} stand vom Tisch auf und seither grüßen
               wir uns nicht mehr.\pend
           \pstart
           Sie erinnern sich dieser abscheulichen Geschichte gewiß; erinnern sich ihrer um so
               eher, als ich sie gleich damals, und hernach noch oft bei Ihnen zum Besten gab, wenn
               wir über Freund Ludaßy\pwindex{Gans-Ludassy, Julius von 13.04.1858 – 30.09.1922@\textsc{Gans-Ludassy, Julius von} (13.04.1858 – 30.09.1922), \emph{Schriftsteller, Journalist, Herausgeber}|pw} und sein Verhältnis zu
               mir, zu Ihnen und zu uns Allen sprachen.\pend
           \pstart
           Diese Geschichte, als die Entstehungsursache seiner Feindschaft gegen mich, habe ich
               vor dem Ehrenrat zu Protokoll gegeben. Herr Ludaßy\pwindex{Gans-Ludassy, Julius von 13.04.1858 – 30.09.1922@\textsc{Gans-Ludassy, Julius von} (13.04.1858 – 30.09.1922), \emph{Schriftsteller, Journalist, Herausgeber}|pw}{ }\uline{leugnet} diesen Vorfall, bezichtigt mich der
               Unwahrheit, und erhebt Ehrenbeleidigungsklage gegen mich, weil ich ihn durch
               Erzählung dieser von mir erlogenen Episode vor dem Ehrenrat dem Gespött preisgegeben
               habe. Die Verhandlung findet Montag, Bezirksgericht Josefstadt\orgindex{Bezirksgericht Wien Josefstadt@Bezirksgericht Wien Josefstadt|pw}, Alserstraße\oindex{Alser Strasse@\textbf{Alser Straße}|pw} statt. Herr Ludaßy\pwindex{Gans-Ludassy, Julius von 13.04.1858 – 30.09.1922@\textsc{Gans-Ludassy, Julius von} (13.04.1858 – 30.09.1922), \emph{Schriftsteller, Journalist, Herausgeber}|pw} will
               damit der Schwurgerichtsverhandlung gegen sich in listiger Weise präludiren.\pend
           \pstart
           Es kommt nun für mich darauf an, zu beweisen, dass ich diesen Vorfall gleich damals,
               nach der Kanner\pwindex{Kanner, Heinrich 09.11.1864 – 15.02.1930@\textsc{Kanner, Heinrich} (09.11.1864 – 15.02.1930), \emph{Herausgeber, Publizist}|pw}-Versammlung, dritten Personen
               erzählt habe. Ich weiß nun, dass ich Ihnen gleich damals ausführlich davon Mitteilung
               machte, um Sie in Kenntnis zu setzen, dass ich mit Ludaßy\pwindex{Gans-Ludassy, Julius von 13.04.1858 – 30.09.1922@\textsc{Gans-Ludassy, Julius von} (13.04.1858 – 30.09.1922), \emph{Schriftsteller, Journalist, Herausgeber}|pw} verfeindet sei. Weiß, dass ich Ihnen im \label{K_L03434-4v}\edtext{Sommer 190\substVorne{}\textsuperscript{5}\substDazwischen{}4\substHinten{}}{\lemma{\textnormal{\emph{Sommer 1904}}}\Cendnote{\textnormal{siehe A. S.: \emph{Tagebuch}, 6. 7. 1904}}}\label{K_L03434-4h} in Pötzleinsdorf\oindex{Poetzleinsdorf@\textbf{Pötzleinsdorf}|pw}, in der Starkfriedgaße\oindex{Starkfriedgassse@\textbf{Starkfriedgassse}|pw}, wo ich damals wohnte, die Sache \uline{wieder} erzählte, worauf Sie mir Ludaßy\pwindex{Gans-Ludassy, Julius von 13.04.1858 – 30.09.1922@\textsc{Gans-Ludassy, Julius von} (13.04.1858 – 30.09.1922), \emph{Schriftsteller, Journalist, Herausgeber}|pw}’s Schmutzwort über Herzl\pwindex{Herzl, Theodor 1860-05-02 – 1904-07-03@\textsc{Herzl, Theodor} (1860-05-02 – 1904-07-03), \emph{Schriftsteller, Journalist}|pw}, das er kurz nach Herzl\pwindex{Herzl, Theodor 1860-05-02 – 1904-07-03@\textsc{Herzl, Theodor} (1860-05-02 – 1904-07-03), \emph{Schriftsteller, Journalist}|pw}’s Tode
               geäußert hatte, gleichsam zur Illustrirung mitteilten.\pend
           \pstart
           Nun bitte ich Sie, mir das zu bezeugen. Sie sind der Einzige, dem ich so oft von der
               Sache sprach. Es ist \uline{wichtig}, dass mir der Wahrheit
               gemäß bezeugt wird, ich habe diesen Vorfall \uline{lange}{ }\uuline{vor} dem Ehrenratsverfahren, \uline{oftmals} und \uline{immer} in \uline{der{\pb}selben Form} erzählt, und immer als die letzte Ursache der Entzweiung
               bezeichnet.\pend
           \pstart
           Die Äußerung über Herzl\pwindex{Herzl, Theodor 1860-05-02 – 1904-07-03@\textsc{Herzl, Theodor} (1860-05-02 – 1904-07-03), \emph{Schriftsteller, Journalist}|pw} wird in der Montag-Verhandlung nicht zur Sprache kommmen. Ich hoffe,
               Sie zögern nicht, mir durch die einfache Constatirung dieser Tatsache in meinem \introOben{}mir\introOben{} aufgedrungenen Abwehrkampf gegen eine der bissigsten
               Canaillen, die es gibt, beizustehen; in einem Kampf, in dem ich ohnehin zu sehr
               allein stehe. Bitte geben Sie mir pneumatisch Nachricht, ob Sie sich dieser Dinge,
               namentlich des Sommers 1904, ec. erinnern, und ob ich Sie
               als Zeugen nennen darf. Das Wesentliche ist, ob Sie – wie ich annehme – Sich
               besinnen, diese Geschichte lange \uline{vor} dem Dezember vo\textcolor{gray}{r.} Jahres und oft vorher von
               mir gehört zu haben.\pend
           \pstart
           herzlichst Ihr {\\[\baselineskip]}\spacefill\mbox{Salten}\pend
           \leftskip=0em{}
         
         \endnumbering\mylabel{h}\end{ledgroupsized}  \newcommand{\dateiname}{L03434}\newcommand{\titel}{Felix Salten an Arthur Schnitzler, [18.? 10. 1906]}\newcommand{\editorInnen}{Martin Anton Müller und Laura Untner}%% latex-leseansicht-abspann.tex
%% Abspann für die Leseansicht.
%% Der Schalter \ifkorrekturansicht ist bereits durch den Vorspann gesetzt.

%% latex-abspann.tex
%% Gemeinsamer Abspann für Korrekturansicht und Leseansicht.
%% Setzt den Schalter \ifkorrekturansicht voraus (gesetzt in den
%% einbindenden Dateien latex-korrekturansicht-abspann.tex bzw.
%% latex-leseansicht-abspann.tex).
%% ---------------------------------------------------------------

\normalsize

% Das esempio-Environment wird nur in der Leseansicht benötigt
\ifkorrekturansicht\else
\newenvironment{esempio}[3]%
{
    \vspace{1.5ex}
    \rlap{\underline{#1}}
    \par
    \setlength{\parindent}{0cm}
    \nopagebreak
    \leftskip=#2cm
    \rightskip=#3cm
}
{
    \par
}
\fi

\doendnotes{C}
\bigskip
\vfill

\clearpage

\footnotesize

\ifkorrekturansicht
  \lohead{\textsc{register}}
\fi

% theindex-Environment neu definieren ohne reledmac
\makeatletter
\renewenvironment{theindex}{%
  \ifkorrekturansicht
    \section*{\indexname}%
  \else
    \subsubsection*{Index der erwähnten Entitäten}%
  \fi
  \setlength{\parindent}{0pt}%
  \setlength{\parskip}{0pt plus 0.3pt}%
  \let\item\@idxitem
}{%
  \ifkorrekturansicht\clearpage\fi
}
\makeatother

\IfFileExists{\jobname-pw.ind}{\input{\jobname-pw.ind}}{}

% Quellenangabe nur in der Leseansicht
\ifkorrekturansicht\else
% Fallback-Definitionen, falls die .tex-Datei \titel etc. nicht gesetzt hat
\providecommand{\titel}{}
\providecommand{\editorInnen}{}
\providecommand{\dateiname}{\jobname}

\vspace{3cm}

\vfill

\footnotesize
\textsc{Quelle}: \titel. Herausgegeben von {\editorInnen}. In: \emph{Arthur Schnitzler: Briefwechsel mit Autorinnen und Autoren}.
 Digitale Edition, https://schnitzler-briefe.acdh.oeaw.ac.at/{\dateiname}.html (Stand \today)
\fi

\end{document}


      