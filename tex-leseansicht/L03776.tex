%% latex-leseansicht-vorspann.tex
%% Vorspann für die Leseansicht.
%% Lädt die gemeinsame Datei latex-vorspann.tex mit nicht gesetztem Schalter.

\newif\ifkorrekturansicht
\korrekturansichtfalse

\input{../tex-inputs/latex-vorspann}


\section[Arthur Schnitzler an Stefan Zweig, 9. 11. 1914]{L03776 Arthur Schnitzler an Stefan Zweig, 9. 11. 1914}
\nopagebreak\mylabel{L03776v}
\rehead{ }\normalsize\beginnumbering\briefempfaengerindex{Zweig, Stefan@\textsc{Zweig, Stefan}!zzzSchnitzler, Arthur@\emph{von Arthur Schnitzler}!1914-11-091@{9. 11. 1914}|(be}
\toendnotes[C]{\smallbreak\pagebreak[2]}
\correspDesc{Versand  durch Arthur Schnitzler am 9. 11. 1914 in Wien
\newline{}Erhalt  durch Stefan Zweig im Zeitraum [9. 11. 1914 – 14. 11. 1914?] in Wien}\toendnotes[C]{\smallbreak}
\Standort{Jerusalem, National Library of Israel, ARC. Ms. Var. 305 1 58 Stefan Zweig Collection.}
\physDesc{Brief, 1 Blatt, 1 Seite, 791 Zeichen
\newline{}Schreibmaschine
\newline{}Handschrift: schwarze Tinte (\noindent{}Unterschrift)}\toendnotes[C]{\smallbreak}
\pstart
           {\pb}\textcolor{gray}{\textbf{Dr. Arthur Schnitzler}}\hfill 9. 11. 1914.\pend
           
\pstart
           \textcolor{gray}{\textbf{Wien XVIII. Sternwartestrasse 71\oindex{Wien@\textbf{Wien}!XVIII., Währing@\textbf{XVIII., Währing}!Sternwartestraße 71@\textbf{Sternwartestraße 71}, \emph{Wohngebäude}|pw}}}\pend
           
\pstart\center{}Lieber Herr Doktor Zweig.\pend\vspace{0.5em}
\pstart
           Wie Sie vielleicht schon erfahren haben, soll eine Internationale Revue gegründet
               werden, für deren Zustandekommen sich hier besonders Dr. Ludo Hartmann\pwindex{Hartmann, Ludo Moritz 2.\,3.\,1865 Stuttgart – 14.\,11.\,1924 Wien@\textsc{Hartmann, Ludo Moritz} (2.\,3.\,1865 Stuttgart – 14.\,11.\,1924 Wien), \emph{Politiker, Historiker, Volksbildner}|pw} einsetzt. Er \label{K_L03776-1v}\edtext{war bei mir}{\lemma{\textnormal{\emph{war bei mir}}}\Cendnote{\textnormal{Vgl. A. S.: \emph{Tagebuch}, 7. 11. 1914.}}}\label{K_L03776-1} unter anderm um mich zu fragen, ob ich
               eine Verbindung zwischen ihm und Romain
                  Rolland\pwindex{Rolland, Romain 29.\,1.\,1866 Clamecy – 30.\,12.\,1944 Vézelay@\textsc{Rolland, Romain} (29.\,1.\,1866 Clamecy – 30.\,12.\,1944 Vézelay), \emph{Schriftsteller}|pw} anbahnen könne. Ich habe mir erlaubt ihn mit dieser Absicht an Sie,
               lieber Herr Doktor, zu weisen und er möchte Sie bitten in obengedachtem Sinn, wenn es
               irgend angeht, \label{K_L03776-2v}\edtext{an Rolland\pwindex{Rolland, Romain 29.\,1.\,1866 Clamecy – 30.\,12.\,1944 Vézelay@\textsc{Rolland, Romain} (29.\,1.\,1866 Clamecy – 30.\,12.\,1944 Vézelay), \emph{Schriftsteller}|pw} zu schreiben}{\lemma{\textnormal{\emph{an Rolland zu schreiben}}}\Cendnote{\textnormal{Am 11. 11. 1914 (Poststempel) schrieb Zweig\pwindex{Zweig, Stefan 28.\,11.\,1881 Wien – 23.\,2.\,1942 Petrópolis@\textsc{Zweig, Stefan} (28.\,11.\,1881 Wien – 23.\,2.\,1942 Petrópolis), \emph{Schriftsteller}|pwk} an Rolland\pwindex{Rolland, Romain 29.\,1.\,1866 Clamecy – 30.\,12.\,1944 Vézelay@\textsc{Rolland, Romain} (29.\,1.\,1866 Clamecy – 30.\,12.\,1944 Vézelay), \emph{Schriftsteller}|pwk}: »Ich verständige Sie gleichzeitig, dass ein Versuch einer
                     neutralen Zeitschrift in der Schweiz\oindex{Schweiz@\textbf{Schweiz}|pw}
                     doppelsprachig unternommen werden soll. Professor Brockhausen\pwindex{Brockhausen, Carl 9.\,5.\,1859 Emmerich am Rhein – 16.\,9.\,1951 Wien@\textsc{Brockhausen, Carl} (9.\,5.\,1859 Emmerich am Rhein – 16.\,9.\,1951 Wien), \emph{Ministerialbeamter, Jurist, Hochschullehrer}|pw}, ein bekannter Nationalöconom und
                     Friedensfreund, wird in dieser Sache von Wien\oindex{Wien@\textbf{Wien}, \emph{Verwaltungsgebiet}|pw} aus delegiert, er wird sicherlich in der Schweiz\oindex{Schweiz@\textbf{Schweiz}|pw} auch Ihre Mitarbeit zu werben suchen, und ich
                     kann Ihnen nur sagen, dass er als rechtlich und tüchtig gilt, seine
                     vortreffliche Absicht nicht zu bezweifeln ist. Die Organisation kann ich nicht
                     beurteilen – hoffentlich setzt er sie Ihnen auseinander.« (Romain Rolland\pwindex{Rolland, Romain 29.\,1.\,1866 Clamecy – 30.\,12.\,1944 Vézelay@\textsc{Rolland, Romain} (29.\,1.\,1866 Clamecy – 30.\,12.\,1944 Vézelay), \emph{Schriftsteller}|pwk}, Stefan Zweig\pwindex{Zweig, Stefan 28.\,11.\,1881 Wien – 23.\,2.\,1942 Petrópolis@\textsc{Zweig, Stefan} (28.\,11.\,1881 Wien – 23.\,2.\,1942 Petrópolis), \emph{Schriftsteller}|pwk}: \emph{Von Welt zu Welt. Briefe
                        einer Freundschaft 1914–1918}. Mit einem Begleitwort von Peter
                     Handke. Aus dem Französischen von Eva und Gerhard Schwewe (Briefe Rollands) und
                     Christel Gersch (Briefe Zweigs). Berlin: \emph{Aufbau
                        Verlag}{ }2014.) Die
                  Zeitschrift wurde nicht realisiert.}}}\label{K_L03776-2}. Interessieren Sie sich für die ganze Angelegenheit, mit der es
               schon in allernächster Zeit Ernst werden soll, so setzen Sie sich mit Ludo Hartmann\pwindex{Hartmann, Ludo Moritz 2.\,3.\,1865 Stuttgart – 14.\,11.\,1924 Wien@\textsc{Hartmann, Ludo Moritz} (2.\,3.\,1865 Stuttgart – 14.\,11.\,1924 Wien), \emph{Politiker, Historiker, Volksbildner}|pw} vielleicht telefonisch in
               Verbindung, nicht wahr?\pend
           
\pstart
           Entschuldigen Sie die Bemühung, seien Sie herzlichst gegrüsst und auf baldiges
               Wiedersehen{\\[\baselineskip]}Ihr \spacefill\mbox{{[}hs.:{]} Arthur Schnitzler}\pend
           \leftskip=0em{}\selectlanguage{ngerman}\endnumbering\briefempfaengerindex{Zweig, Stefan@\textsc{Zweig, Stefan}!zzzSchnitzler, Arthur@\emph{von Arthur Schnitzler}!1914-11-091@{9. 11. 1914}|)be}\mylabel{L03776h}  \newcommand{\dateiname}{L03776}\newcommand{\titel}{Arthur Schnitzler an Stefan Zweig, 9. 11. 1914}\newcommand{\editorInnen}{Selma Jahnke und Martin Anton Müller}%% latex-leseansicht-abspann.tex
%% Abspann für die Leseansicht.
%% Der Schalter \ifkorrekturansicht ist bereits durch den Vorspann gesetzt.

%% latex-abspann.tex
%% Gemeinsamer Abspann für Korrekturansicht und Leseansicht.
%% Setzt den Schalter \ifkorrekturansicht voraus (gesetzt in den
%% einbindenden Dateien latex-korrekturansicht-abspann.tex bzw.
%% latex-leseansicht-abspann.tex).
%% ---------------------------------------------------------------

\normalsize

% Das esempio-Environment wird nur in der Leseansicht benötigt
\ifkorrekturansicht\else
\newenvironment{esempio}[3]%
{
    \vspace{1.5ex}
    \rlap{\underline{#1}}
    \par
    \setlength{\parindent}{0cm}
    \nopagebreak
    \leftskip=#2cm
    \rightskip=#3cm
}
{
    \par
}
\fi

\doendnotes{C}
\bigskip
\vfill

\clearpage

\footnotesize

\ifkorrekturansicht
  \lohead{\textsc{register}}
\fi

% theindex-Environment neu definieren ohne reledmac
\makeatletter
\renewenvironment{theindex}{%
  \ifkorrekturansicht
    \section*{\indexname}%
  \else
    \subsubsection*{Index der erwähnten Entitäten}%
  \fi
  \setlength{\parindent}{0pt}%
  \setlength{\parskip}{0pt plus 0.3pt}%
  \let\item\@idxitem
}{%
  \ifkorrekturansicht\clearpage\fi
}
\makeatother

\IfFileExists{\jobname-pw.ind}{\input{\jobname-pw.ind}}{}

% Quellenangabe nur in der Leseansicht
\ifkorrekturansicht\else
% Fallback-Definitionen, falls die .tex-Datei \titel etc. nicht gesetzt hat
\providecommand{\titel}{}
\providecommand{\editorInnen}{}
\providecommand{\dateiname}{\jobname}

\vspace{3cm}

\vfill

\footnotesize
\textsc{Quelle}: \titel. Herausgegeben von {\editorInnen}. In: \emph{Arthur Schnitzler: Briefwechsel mit Autorinnen und Autoren}.
 Digitale Edition, https://schnitzler-briefe.acdh.oeaw.ac.at/{\dateiname}.html (Stand \today)
\fi

\end{document}


