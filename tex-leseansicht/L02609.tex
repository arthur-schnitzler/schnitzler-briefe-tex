%% latex-korrekturansicht-vorspann.tex
%% Vorspann für die Korrekturansicht.
%% Lädt die gemeinsame Datei latex-vorspann.tex mit gesetztem Schalter.

\newif\ifkorrekturansicht
\korrekturansichttrue

\input{../tex-inputs/latex-vorspann}


\section[Paul Goldmann an Arthur Schnitzler, 17. 2. {[}1894{]}]{L02609 Paul Goldmann an Arthur Schnitzler, 17. 2. {[}1894{]}}
\nopagebreak\mylabel{L02609v}
\rehead{ }\normalsize\beginnumbering\briefempfaengerindex{Schnitzler, Arthur@\textsc{Schnitzler, Arthur}!zzzGoldmann, Paul@\emph{von Paul Goldmann}!1894-02-172@{17. 2. {[}1894{]}}|(be}
\toendnotes[C]{\smallbreak\pagebreak[2]}\Standort{DLA, A:Schnitzler, HS.NZ85.1.3164.}
\physDesc{Brief, 1 Blatt, 4 Seiten, 2377 Zeichen
\newline{}Handschrift: schwarze Tinte, deutsche Kurrent
\newline{}Schnitzler: 1) mit Bleistift auf dem ersten Blatt die Jahreszahl »94« vermerkt  2) mit rotem Buntstift zwei Unterstreichungen}\toendnotes[C]{\smallbreak}
\pstart
           \raggedleft{}{\pb}\textsc{Paris\oindex{Paris@\textbf{Paris}, \emph{P.PPLC}|pw}}, 17. Februar.\pend
           
\pstart\center{}Mein lieber Freund,\pend\vspace{0.5em}
\pstart
           Es iſt nur der Zeitmangel. Ich denke oft an Dich. Stelle Dir ſehr oft vor und es iſt
               doch noch mehr. Spreche auch viel von Dir. Aber ſchreiben? Unmöglich. Und was auch?
               Was ich thue, ſiehst Du aus der Zeitung\pwindex{Frankfurter Zeitung@\emph{Frankfurter Zeitung}|pwv}, wo Du meine Arbeiten mit einer Treue verfolgſt, die mich rührt.
               Nebenher keinen Strich. \textsc{\label{K_L02609-1v}\edtext{Improductivitas absoluta}{\lemma{\textnormal{\emph{Improductivitas absoluta}}}\Cendnote{\textnormal{lateinisch: völlige
                     Unproduktivität}}}\label{K_L02609-1}}. Schädel leer, Herz leer. Verkommene Exiſtenz. Scheußlicher bürgerlicher
               Zuſtand, ſeeliſcher desgleichen. {\pb}Das iſt immer
               dieſelbe Geſchichte. Was willſt Du alſo von mir hören? Mir iſt lieber, ich höre von
               Dir. Das iſt doch wenigſtens eine Freude.\pend
           
\pstart
           Und doch ein kleiner Lichtblick. Einen Menſchen\pwindex{Albert, Henri 1869-11-16 – 1921-08-03@\textsc{Albert, Henri} (1869-11-16 – 1921-08-03), \emph{Journalist/Journalistin, Kritiker/Kritikerin, Übersetzer/Übersetzerin}|pwv} gefunden, den Erſten ſeit Wien\oindex{Wien@\textbf{Wien}, \emph{A.ADM2}|pw}. Heißt \textsc{Henri Albert\pwindex{Albert, Henri 1869-11-16 – 1921-08-03@\textsc{Albert, Henri} (1869-11-16 – 1921-08-03), \emph{Journalist/Journalistin, Kritiker/Kritikerin, Übersetzer/Übersetzerin}|pw}}, Mitte zwanzig. Dasjenige, was wir ſeinerzeit impertinent genug waren, eine
               Wir-Natur zu nennen. Noch mehr: ich glaube beinahe, daß er ein viertes Exemplar iſt
               von der \textsc{Species}{ }\textsc{Arthur} – \textsc{Richard\pwindex{Beer-Hofmann, Richard 1866-07-11 – 1945-09-26@\textsc{Beer-Hofmann, Richard} (1866-07-11 – 1945-09-26), \emph{Schriftsteller/Schriftstellerin}|pw}} – \textsc{Loris\pwindex{Hofmannsthal, Hugo von 1874-02-01 – 1929-07-15@\textsc{Hofmannsthal, Hugo von} (1874-02-01 – 1929-07-15), \emph{Schriftsteller/Schriftstellerin}|pw}}. Noch weiß ichs nicht genau; denn ich habe die Aufrichtigkeit-Diagnoſe noch
               nicht ſtellen können. Alles {\pb}Übrige ſcheint zu
               ſtimmen. Und, oh Wunder, er kennt Euch\pwindex{Beer-Hofmann, Richard 1866-07-11 – 1945-09-26@\textsc{Beer-Hofmann, Richard} (1866-07-11 – 1945-09-26), \emph{Schriftsteller/Schriftstellerin}|pwv}\pwindex{Hofmannsthal, Hugo von 1874-02-01 – 1929-07-15@\textsc{Hofmannsthal, Hugo von} (1874-02-01 – 1929-07-15), \emph{Schriftsteller/Schriftstellerin}|pwv} Alle, hat von Allen\pwindex{Beer-Hofmann, Richard 1866-07-11 – 1945-09-26@\textsc{Beer-Hofmann, Richard} (1866-07-11 – 1945-09-26), \emph{Schriftsteller/Schriftstellerin}|pwv}\pwindex{Hofmannsthal, Hugo von 1874-02-01 – 1929-07-15@\textsc{Hofmannsthal, Hugo von} (1874-02-01 – 1929-07-15), \emph{Schriftsteller/Schriftstellerin}|pwv} geleſen. Nun kennt er Euch\pwindex{Beer-Hofmann, Richard 1866-07-11 – 1945-09-26@\textsc{Beer-Hofmann, Richard} (1866-07-11 – 1945-09-26), \emph{Schriftsteller/Schriftstellerin}|pwv}\pwindex{Hofmannsthal, Hugo von 1874-02-01 – 1929-07-15@\textsc{Hofmannsthal, Hugo von} (1874-02-01 – 1929-07-15), \emph{Schriftsteller/Schriftstellerin}|pwv} natürlich erſt
               recht. Ich habe ihn – auf Widerruf – zum auswärtigen Mitglied unſeres Kreiſes
               ernannt, weil ich ihn lieb gewonnen und dies \strikeout{das} der
               höchſte Orden iſt, das Goldene Vließ, das ich zu vergeben habe. Wenn das keine
               Enttäuſchung iſt – in \textsc{Paris\oindex{Paris@\textbf{Paris}, \emph{P.PPLC}|pw}} haben die Naturen ſolche Untiefen! – ſo iſts ein wahrer Fund geweſen. Er
               correſpondirt von hier\oindex{Paris@\textbf{Paris}, \emph{P.PPLC}|pwv} für die
                  »Freie Bühne\orgindex{Freie Buehne@Freie Bühne|pw}«, ſchreibt außerdem viel in den
               jungen franzöſiſchen Revüen. Als Elſäſſer\pwindex{Albert, Henri 1869-11-16 – 1921-08-03@\textsc{Albert, Henri} (1869-11-16 – 1921-08-03), \emph{Journalist/Journalistin, Kritiker/Kritikerin, Übersetzer/Übersetzerin}|pwv} ſpricht und ſchreibt er deutſch wie franzöſiſch. {\pb}Ich bin hinter ihm her, daß er mir \label{K_L02609-2v}\edtext{über Euch\pwindex{Beer-Hofmann, Richard 1866-07-11 – 1945-09-26@\textsc{Beer-Hofmann, Richard} (1866-07-11 – 1945-09-26), \emph{Schriftsteller/Schriftstellerin}|pwv}\pwindex{Hofmannsthal, Hugo von 1874-02-01 – 1929-07-15@\textsc{Hofmannsthal, Hugo von} (1874-02-01 – 1929-07-15), \emph{Schriftsteller/Schriftstellerin}|pwv} einen Artikel in den »\textsc{Mercure de France\pwindex{Mercure de France@\emph{Mercure de France}|pw}}« oder die »\textsc{Société Nouvelle\pwindex{Societe Nouvelle. Revue internationale. Sociologie, Arts, Sciences, Lettres@\emph{La Société Nouvelle. Revue internationale. Sociologie, Arts, Sciences, Lettres}|pw}}«}{\lemma{\textnormal{\emph{über … Nouvelle«}}}\Cendnote{\textnormal{Bereits wenig später erschien die
                     Rezension\pwindex{Le nouvel almanach de M. Bierbaum@\emph{Le nouvel almanach de M. Bierbaum}|pwkv} des \emph{Modernen Musenalmanach auf das Jahr 1894}\pwindex{Moderner Musen-Almanach auf das Jahr 1894. Ein Jahrbuch deutscher Kunst@\emph{Moderner Musen-Almanach auf das Jahr 1894. Ein Jahrbuch deutscher Kunst}|pwk} im
                     \emph{Mercure de France}\pwindex{Mercure de France@\emph{Mercure de France}|pwk}, in der die Beiträge von Schnitzler und Hofmannsthal\pwindex{Hofmannsthal, Hugo von 1874-02-01 – 1929-07-15@\textsc{Hofmannsthal, Hugo von} (1874-02-01 – 1929-07-15), \emph{Schriftsteller/Schriftstellerin}|pwk} hervorgehoben wurden: Henri Albert\pwindex{Albert, Henri 1869-11-16 – 1921-08-03@\textsc{Albert, Henri} (1869-11-16 – 1921-08-03), \emph{Journalist/Journalistin, Kritiker/Kritikerin, Übersetzer/Übersetzerin}|pwk}: \emph{Le nouvel almanach de M. Bierbaum}\pwindex{Le nouvel almanach de M. Bierbaum@\emph{Le nouvel almanach de M. Bierbaum}|pwk}. In: \emph{Mercure de France}\pwindex{Mercure de France@\emph{Mercure de France}|pwk}, Jg. 10, Nr. 51,
                        März 1894, S. 233–246, hier: S. 244–245.}}}\label{K_L02609-2} macht,
               daß er etwas\pwindex{Weihnachts-Einkaeufe@\emph{Weihnachts-Einkäufe}|pwv} von Dir \label{K_L02609-3v}\edtext{überſetzt}{\lemma{\textnormal{\emph{überſetzt}}}\Cendnote{\textnormal{Arthur Schnitzler: \emph{Les Emplettes de Noël}\pwindex{Weihnachts-Einkaeufe@\emph{Weihnachts-Einkäufe}|pwk}. Übersetzung Henri Albert\pwindex{Albert, Henri 1869-11-16 – 1921-08-03@\textsc{Albert, Henri} (1869-11-16 – 1921-08-03), \emph{Journalist/Journalistin, Kritiker/Kritikerin, Übersetzer/Übersetzerin}|pwk}. In: \emph{L’Idée
                        libre. Revue mensuelle de Littérature et d'Art}\pwindex{L'Idee libre. Revue mensuelle de Litterature et d'Art@\emph{L'Idée libre. Revue mensuelle de Littérature et d'Art}|pwk}, Jg. 3, Nr. 5–6,
                        Mai–Juni 1894, S. 215–225.}}}\label{K_L02609-3}{ }\textsc{etc.} Hoffen wir!\pend
           
\pstart
           Wann kommt endlich Einer von Euch\pwindex{Beer-Hofmann, Richard 1866-07-11 – 1945-09-26@\textsc{Beer-Hofmann, Richard} (1866-07-11 – 1945-09-26), \emph{Schriftsteller/Schriftstellerin}|pwv}\pwindex{Hofmannsthal, Hugo von 1874-02-01 – 1929-07-15@\textsc{Hofmannsthal, Hugo von} (1874-02-01 – 1929-07-15), \emph{Schriftsteller/Schriftstellerin}|pwv} her?\pend
           
\pstart
           Deine Zukunfts-Zuverſicht betreffend Deine Production für dieſes
                  Jahr hat mich unendlich erfreut. Aber was? Und wie gehts Dir ſonſt?
               Perſönliches, perſönliches, mein theurer Freund!\pend
           
\pstart
           Über \textsc{Niemann\pwindex{Niemann, August 1839-06-27 – 1919-09-17@\textsc{Niemann, August} (1839-06-27 – 1919-09-17), \emph{Schriftsteller/Schriftstellerin, Schauspieler/Schauspielerin, Redakteur/Redakteurin}|pw}} bin ich ganz anderer Anſicht. Mich hat das Ding\pwindex{Junggesell. Humoreske@\emph{Der Junggesell. Humoreske}|pwuv} hoch entzückt gerade wegen ſeiner
               Abſichtsloſigkeit, gerade, weil ich in ihm ein einfaches, humorvolles, \strikeout{,}{ }zierliches Kunſtwerk gefunden, von der Höhe des
               intellectuellen Standpunktes abgeſehen. Wer von uns hat da Recht? Und \textsc{Duerer\pwindex{Duerer, Emil @\textsc{Dürer, Emil}|pw}}? Schreib’ mir über \textsc{Duerer\pwindex{Duerer, Emil @\textsc{Dürer, Emil}|pw}}! Herzlichſt und in Treue Dein \spacefill\mbox{Paul Goldmann}\pend
           
\pstart
           \noindent{}{\pb}\label{T_L02609-1v}\edtext{viele herzliche Grüße an die Freunde\pwindex{Beer-Hofmann, Richard 1866-07-11 – 1945-09-26@\textsc{Beer-Hofmann, Richard} (1866-07-11 – 1945-09-26), \emph{Schriftsteller/Schriftstellerin}|pwv}\pwindex{Hofmannsthal, Hugo von 1874-02-01 – 1929-07-15@\textsc{Hofmannsthal, Hugo von} (1874-02-01 – 1929-07-15), \emph{Schriftsteller/Schriftstellerin}|pwv}. Schreib
                  mir bald einen \uline{langen} Brief}{\lemma{\textnormal{\emph{viele … Brief}}}\Cendnote{\textnormal{am oberen Rand auf der ersten Seite}}}\label{T_L02609-1}\pend
           \selectlanguage{ngerman}\endnumbering\briefempfaengerindex{Schnitzler, Arthur@\textsc{Schnitzler, Arthur}!zzzGoldmann, Paul@\emph{von Paul Goldmann}!1894-02-172@{17. 2. {[}1894{]}}|)be}\mylabel{L02609h}  \normalsize

\doendnotes{C}
\bigskip
\vfill

\clearpage

\footnotesize

\lohead{\textsc{register}}

% Definiere theindex-Environment komplett neu ohne reledmac
\makeatletter
\renewenvironment{theindex}{%
  \section*{\indexname}%
  \setlength{\parindent}{0pt}%
  \setlength{\parskip}{0pt plus 0.3pt}%
  \let\item\@idxitem
}{%
  \clearpage
}
\makeatother

\IfFileExists{\jobname-pw.ind}{\input{\jobname-pw.ind}}{}

\end{document}

      