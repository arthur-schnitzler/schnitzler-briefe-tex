%% latex-leseansicht-vorspann.tex
%% Vorspann für die Leseansicht.
%% Lädt die gemeinsame Datei latex-vorspann.tex mit nicht gesetztem Schalter.

\newif\ifkorrekturansicht
\korrekturansichtfalse

\input{../tex-inputs/latex-vorspann}


\section[Paul Goldmann an Arthur Schnitzler, 17. 2. [1894]]{L02609 Paul Goldmann an Arthur Schnitzler, 17. 2. [1894]}
\nopagebreak\mylabel{L02609v}
\rehead{ }\normalsize\beginnumbering\briefempfaengerindex{Schnitzler, Arthur@\textsc{Schnitzler, Arthur}!zzzGoldmann, Paul@\emph{von Paul Goldmann}!1894-02-172@{17. 2. [1894]}|(be}
\toendnotes[C]{\smallbreak\pagebreak[2]}
\correspDesc{Versand  durch Paul Goldmann am 17. 2. [1894] in Paris
\newline{}Erhalt  durch Arthur Schnitzler im Zeitraum [18. 2. 1894
                  – 22. 2. 1894?] in Wien}\toendnotes[C]{\smallbreak}
\Standort{DLA, A:Schnitzler, HS.NZ85.1.3164.}
\physDesc{Brief, 1 Blatt, 4 Seiten, 2377 Zeichen
\newline{}Handschrift: schwarze Tinte, deutsche Kurrent
\newline{}Schnitzler: 1) mit Bleistift auf dem ersten Blatt die Jahreszahl »94« vermerkt  2) mit rotem Buntstift zwei Unterstreichungen}\toendnotes[C]{\smallbreak}
\pstart
           \raggedleft{}{\pb}\textsc{Paris\oindex{Paris@\textbf{Paris}, \emph{Hauptstadt}|pw}}, 17. Februar.\pend
           
\pstart\center{}Mein lieber Freund,\pend\vspace{0.5em}
\pstart
           Es iſt nur der Zeitmangel. Ich denke oft an Dich. Stelle Dir{ }ſehr oft vor und es iſt
               doch noch mehr. Spreche auch viel von Dir. Aber{ }ſchreiben? Unmöglich. Und was auch?
               Was ich thue,{ }ſiehst Du aus der Zeitung\pwindex{Frankfurter Zeitung@\emph{Frankfurter Zeitung}|pwv}, wo Du meine Arbeiten mit einer Treue verfolgſt, die mich rührt.
               Nebenher keinen Strich. \textsc{\label{K_L02609-1v}\edtext{Improductivitas absoluta}{\lemma{\textnormal{\emph{Improductivitas absoluta}}}\Cendnote{\textnormal{lateinisch: völlige
                     Unproduktivität}}}\label{K_L02609-1}}. Schädel leer, Herz leer. Verkommene Exiſtenz. Scheußlicher bürgerlicher
               Zuſtand,{ }ſeeliſcher desgleichen. {\pb}Das iſt immer
               dieſelbe Geſchichte. Was willſt Du alſo von mir hören? Mir iſt lieber, ich höre von
               Dir. Das iſt doch wenigſtens eine Freude.\pend
           
\pstart
           Und doch ein kleiner Lichtblick. Einen Menſchen\pwindex{Albert, Henri 16.\,11.\,1869 Niederbronn-les-Bains – 3.\,8.\,1921 Straßburg@\textsc{Albert, Henri} (16.\,11.\,1869 Niederbronn-les-Bains – 3.\,8.\,1921 Straßburg), \emph{Journalist, Kritiker, Übersetzer}|pwv} gefunden, den Erſten{ }ſeit Wien\oindex{Wien@\textbf{Wien}, \emph{Verwaltungsgebiet}|pw}. Heißt \textsc{Henri Albert\pwindex{Albert, Henri 16.\,11.\,1869 Niederbronn-les-Bains – 3.\,8.\,1921 Straßburg@\textsc{Albert, Henri} (16.\,11.\,1869 Niederbronn-les-Bains – 3.\,8.\,1921 Straßburg), \emph{Journalist, Kritiker, Übersetzer}|pw}}, Mitte zwanzig. Dasjenige, was wir{ }ſeinerzeit impertinent genug waren, eine
               Wir-Natur zu nennen. Noch mehr: ich glaube beinahe, daß er ein viertes Exemplar iſt
               von der \textsc{Species}{ }\textsc{Arthur} – \textsc{Richard\pwindex{Beer-Hofmann, Richard 11.\,7.\,1866 Wien – 26.\,9.\,1945 New York City@\textsc{Beer-Hofmann, Richard} (11.\,7.\,1866 Wien – 26.\,9.\,1945 New York City), \emph{Schriftsteller}|pw}} – \textsc{Loris\pwindex{Hofmannsthal, Hugo von 1.\,2.\,1874 Wien – 15.\,7.\,1929 Rodaun@\textsc{Hofmannsthal, Hugo von} (1.\,2.\,1874 Wien – 15.\,7.\,1929 Rodaun), \emph{Schriftsteller}|pw}}. Noch weiß ichs nicht genau; denn ich habe die Aufrichtigkeit-Diagnoſe noch
               nicht{ }ſtellen können. Alles {\pb}Übrige{ }ſcheint zu{ }ſtimmen. Und, oh Wunder, er kennt Euch\pwindex{Beer-Hofmann, Richard 11.\,7.\,1866 Wien – 26.\,9.\,1945 New York City@\textsc{Beer-Hofmann, Richard} (11.\,7.\,1866 Wien – 26.\,9.\,1945 New York City), \emph{Schriftsteller}|pwv}\pwindex{Hofmannsthal, Hugo von 1.\,2.\,1874 Wien – 15.\,7.\,1929 Rodaun@\textsc{Hofmannsthal, Hugo von} (1.\,2.\,1874 Wien – 15.\,7.\,1929 Rodaun), \emph{Schriftsteller}|pwv} Alle, hat von Allen\pwindex{Beer-Hofmann, Richard 11.\,7.\,1866 Wien – 26.\,9.\,1945 New York City@\textsc{Beer-Hofmann, Richard} (11.\,7.\,1866 Wien – 26.\,9.\,1945 New York City), \emph{Schriftsteller}|pwv}\pwindex{Hofmannsthal, Hugo von 1.\,2.\,1874 Wien – 15.\,7.\,1929 Rodaun@\textsc{Hofmannsthal, Hugo von} (1.\,2.\,1874 Wien – 15.\,7.\,1929 Rodaun), \emph{Schriftsteller}|pwv} geleſen. Nun kennt er Euch\pwindex{Beer-Hofmann, Richard 11.\,7.\,1866 Wien – 26.\,9.\,1945 New York City@\textsc{Beer-Hofmann, Richard} (11.\,7.\,1866 Wien – 26.\,9.\,1945 New York City), \emph{Schriftsteller}|pwv}\pwindex{Hofmannsthal, Hugo von 1.\,2.\,1874 Wien – 15.\,7.\,1929 Rodaun@\textsc{Hofmannsthal, Hugo von} (1.\,2.\,1874 Wien – 15.\,7.\,1929 Rodaun), \emph{Schriftsteller}|pwv} natürlich erſt
               recht. Ich habe ihn – auf Widerruf – zum auswärtigen Mitglied unſeres Kreiſes
               ernannt, weil ich ihn lieb gewonnen und dies \strikeout{das} der
               höchſte Orden iſt, das Goldene Vließ, das ich zu vergeben habe. Wenn das keine
               Enttäuſchung iſt – in \textsc{Paris\oindex{Paris@\textbf{Paris}, \emph{Hauptstadt}|pw}} haben die Naturen{ }ſolche Untiefen! –{ }ſo iſts ein wahrer Fund geweſen. Er
               correſpondirt von hier\oindex{Paris@\textbf{Paris}, \emph{Hauptstadt}|pwv} für die
                  »Freie Bühne\orgindex{Freie Bühne@Freie Bühne|pw}«,{ }ſchreibt außerdem viel in den
               jungen franzöſiſchen Revüen. Als Elſäſſer\pwindex{Albert, Henri 16.\,11.\,1869 Niederbronn-les-Bains – 3.\,8.\,1921 Straßburg@\textsc{Albert, Henri} (16.\,11.\,1869 Niederbronn-les-Bains – 3.\,8.\,1921 Straßburg), \emph{Journalist, Kritiker, Übersetzer}|pwv}{ }ſpricht und{ }ſchreibt er deutſch wie franzöſiſch. {\pb}Ich bin hinter ihm her, daß er mir \label{K_L02609-2v}\edtext{über Euch\pwindex{Beer-Hofmann, Richard 11.\,7.\,1866 Wien – 26.\,9.\,1945 New York City@\textsc{Beer-Hofmann, Richard} (11.\,7.\,1866 Wien – 26.\,9.\,1945 New York City), \emph{Schriftsteller}|pwv}\pwindex{Hofmannsthal, Hugo von 1.\,2.\,1874 Wien – 15.\,7.\,1929 Rodaun@\textsc{Hofmannsthal, Hugo von} (1.\,2.\,1874 Wien – 15.\,7.\,1929 Rodaun), \emph{Schriftsteller}|pwv} einen Artikel in den »\textsc{Mercure de France\pwindex{Mercure de France@\emph{Mercure de France}|pw}}« oder die »\textsc{Société Nouvelle\pwindex{Société Nouvelle. Revue internationale. Sociologie, Arts, Sciences, Lettres@\emph{La Société Nouvelle. Revue internationale. Sociologie, Arts, Sciences, Lettres}|pw}}«}{\lemma{\textnormal{\emph{über … Nouvelle«}}}\Cendnote{\textnormal{Bereits wenig später erschien die
                     Rezension\pwindex{Albert, Henri 16.\,11.\,1869 Niederbronn-les-Bains – 3.\,8.\,1921 Straßburg@\textsc{Albert, Henri} (16.\,11.\,1869 Niederbronn-les-Bains – 3.\,8.\,1921 Straßburg), \emph{Journalist, Kritiker, Übersetzer}!Le nouvel almanach de M. Bierbaum@\strich\emph{Le nouvel almanach de M. Bierbaum}|pwkv} des \emph{Modernen Musenalmanach auf das Jahr 1894}\pwindex{Moderner Musen-Almanach auf das Jahr 1894. Ein Jahrbuch deutscher Kunst@\emph{Moderner Musen-Almanach auf das Jahr 1894. Ein Jahrbuch deutscher Kunst}|pwk} im
                     \emph{Mercure de France}\pwindex{Mercure de France@\emph{Mercure de France}|pwk}, in der die Beiträge von Schnitzler und Hofmannsthal\pwindex{Hofmannsthal, Hugo von 1.\,2.\,1874 Wien – 15.\,7.\,1929 Rodaun@\textsc{Hofmannsthal, Hugo von} (1.\,2.\,1874 Wien – 15.\,7.\,1929 Rodaun), \emph{Schriftsteller}|pwk} hervorgehoben wurden: Henri Albert\pwindex{Albert, Henri 16.\,11.\,1869 Niederbronn-les-Bains – 3.\,8.\,1921 Straßburg@\textsc{Albert, Henri} (16.\,11.\,1869 Niederbronn-les-Bains – 3.\,8.\,1921 Straßburg), \emph{Journalist, Kritiker, Übersetzer}|pwk}: \emph{Le nouvel almanach de M. Bierbaum}\pwindex{Albert, Henri 16.\,11.\,1869 Niederbronn-les-Bains – 3.\,8.\,1921 Straßburg@\textsc{Albert, Henri} (16.\,11.\,1869 Niederbronn-les-Bains – 3.\,8.\,1921 Straßburg), \emph{Journalist, Kritiker, Übersetzer}!Le nouvel almanach de M. Bierbaum@\strich\emph{Le nouvel almanach de M. Bierbaum}|pwk}. In: \emph{Mercure de France}\pwindex{Mercure de France@\emph{Mercure de France}|pwk}, Jg. 10, Nr. 51,
                        März 1894, S. 233–246, hier: S. 244–245.}}}\label{K_L02609-2} macht,
               daß er etwas\pwindex{Schnitzler, Arthur 15.\,5.\,1862 Wien – 21.\,10.\,1931 ebd.@\textsc{Schnitzler, Arthur} (15.\,5.\,1862 Wien – 21.\,10.\,1931 ebd.), \emph{Schriftsteller, Mediziner}!Weihnachts-Einkäufe@\strich\emph{Weihnachts-Einkäufe}|pwv} von Dir \label{K_L02609-3v}\edtext{überſetzt}{\lemma{\textnormal{\emph{übersetzt}}}\Cendnote{\textnormal{Arthur Schnitzler: \emph{Les Emplettes de Noël}\pwindex{Schnitzler, Arthur 15.\,5.\,1862 Wien – 21.\,10.\,1931 ebd.@\textsc{Schnitzler, Arthur} (15.\,5.\,1862 Wien – 21.\,10.\,1931 ebd.), \emph{Schriftsteller, Mediziner}!Weihnachts-Einkäufe@\strich\emph{Weihnachts-Einkäufe}|pwk}. Übersetzung Henri Albert\pwindex{Albert, Henri 16.\,11.\,1869 Niederbronn-les-Bains – 3.\,8.\,1921 Straßburg@\textsc{Albert, Henri} (16.\,11.\,1869 Niederbronn-les-Bains – 3.\,8.\,1921 Straßburg), \emph{Journalist, Kritiker, Übersetzer}|pwk}. In: \emph{L’Idée
                        libre. Revue mensuelle de Littérature et d’Art}\pwindex{Idée libre. Revue mensuelle de Littérature et d'Art@\emph{L’Idée libre. Revue mensuelle de Littérature et d'Art}|pwk}, Jg. 3, Nr. 5–6,
                        Mai–Juni 1894, S. 215–225.}}}\label{K_L02609-3}{ }\textsc{etc.} Hoffen wir!\pend
           
\pstart
           Wann kommt endlich Einer von Euch\pwindex{Beer-Hofmann, Richard 11.\,7.\,1866 Wien – 26.\,9.\,1945 New York City@\textsc{Beer-Hofmann, Richard} (11.\,7.\,1866 Wien – 26.\,9.\,1945 New York City), \emph{Schriftsteller}|pwv}\pwindex{Hofmannsthal, Hugo von 1.\,2.\,1874 Wien – 15.\,7.\,1929 Rodaun@\textsc{Hofmannsthal, Hugo von} (1.\,2.\,1874 Wien – 15.\,7.\,1929 Rodaun), \emph{Schriftsteller}|pwv} her?\pend
           
\pstart
           Deine Zukunfts-Zuverſicht betreffend Deine Production für dieſes
                  Jahr hat mich unendlich erfreut. Aber was? Und wie gehts Dir{ }ſonſt?
               Perſönliches, perſönliches, mein theurer Freund!\pend
           
\pstart
           Über \textsc{Niemann\pwindex{Niemann, August 27.\,6.\,1839 – 17.\,9.\,1919@\textsc{Niemann, August} (27.\,6.\,1839 – 17.\,9.\,1919), \emph{Schriftsteller, Schauspieler, Redakteur}|pw}} bin ich ganz anderer Anſicht. Mich hat das Ding\pwindex{Niemann, August 27.\,6.\,1839 – 17.\,9.\,1919@\textsc{Niemann, August} (27.\,6.\,1839 – 17.\,9.\,1919), \emph{Schriftsteller, Schauspieler, Redakteur}!Junggesell. Humoreske@\strich\emph{Der Junggesell. Humoreske}|pwuv} hoch entzückt gerade wegen{ }ſeiner
               Abſichtsloſigkeit, gerade, weil ich in ihm ein einfaches, humorvolles, \strikeout{,}{ }zierliches Kunſtwerk gefunden, von der Höhe des
               intellectuellen Standpunktes abgeſehen. Wer von uns hat da Recht? Und \textsc{Duerer\pwindex{Dürer, Emil @\textsc{Dürer, Emil}|pw}}? Schreib’ mir über \textsc{Duerer\pwindex{Dürer, Emil @\textsc{Dürer, Emil}|pw}}! Herzlichſt und in Treue Dein \spacefill\mbox{Paul Goldmann}\pend
           
\pstart
           \noindent{}{\pb}\label{T_L02609-1v}\edtext{viele herzliche Grüße an die Freunde\pwindex{Beer-Hofmann, Richard 11.\,7.\,1866 Wien – 26.\,9.\,1945 New York City@\textsc{Beer-Hofmann, Richard} (11.\,7.\,1866 Wien – 26.\,9.\,1945 New York City), \emph{Schriftsteller}|pwv}\pwindex{Hofmannsthal, Hugo von 1.\,2.\,1874 Wien – 15.\,7.\,1929 Rodaun@\textsc{Hofmannsthal, Hugo von} (1.\,2.\,1874 Wien – 15.\,7.\,1929 Rodaun), \emph{Schriftsteller}|pwv}. Schreib
                  mir bald einen \uline{langen} Brief}{\lemma{\textnormal{\emph{viele … Brief}}}\Cendnote{\textnormal{am oberen Rand auf der ersten Seite}}}\label{T_L02609-1}\pend
           \selectlanguage{ngerman}\endnumbering\briefempfaengerindex{Schnitzler, Arthur@\textsc{Schnitzler, Arthur}!zzzGoldmann, Paul@\emph{von Paul Goldmann}!1894-02-172@{17. 2. [1894]}|)be}\mylabel{L02609h}  \newcommand{\dateiname}{L02609}\newcommand{\titel}{Paul Goldmann an Arthur Schnitzler, 17. 2. [1894]}\newcommand{\editorInnen}{Martin Anton Müller und Laura Untner}%% latex-leseansicht-abspann.tex
%% Abspann für die Leseansicht.
%% Der Schalter \ifkorrekturansicht ist bereits durch den Vorspann gesetzt.

%% latex-abspann.tex
%% Gemeinsamer Abspann für Korrekturansicht und Leseansicht.
%% Setzt den Schalter \ifkorrekturansicht voraus (gesetzt in den
%% einbindenden Dateien latex-korrekturansicht-abspann.tex bzw.
%% latex-leseansicht-abspann.tex).
%% ---------------------------------------------------------------

\normalsize

% Das esempio-Environment wird nur in der Leseansicht benötigt
\ifkorrekturansicht\else
\newenvironment{esempio}[3]%
{
    \vspace{1.5ex}
    \rlap{\underline{#1}}
    \par
    \setlength{\parindent}{0cm}
    \nopagebreak
    \leftskip=#2cm
    \rightskip=#3cm
}
{
    \par
}
\fi

\doendnotes{C}
\bigskip
\vfill

\clearpage

\footnotesize

\ifkorrekturansicht
  \lohead{\textsc{register}}
\fi

% theindex-Environment neu definieren ohne reledmac
\makeatletter
\renewenvironment{theindex}{%
  \ifkorrekturansicht
    \section*{\indexname}%
  \else
    \subsubsection*{Index der erwähnten Entitäten}%
  \fi
  \setlength{\parindent}{0pt}%
  \setlength{\parskip}{0pt plus 0.3pt}%
  \let\item\@idxitem
}{%
  \ifkorrekturansicht\clearpage\fi
}
\makeatother

\IfFileExists{\jobname-pw.ind}{\input{\jobname-pw.ind}}{}

% Quellenangabe nur in der Leseansicht
\ifkorrekturansicht\else
% Fallback-Definitionen, falls die .tex-Datei \titel etc. nicht gesetzt hat
\providecommand{\titel}{}
\providecommand{\editorInnen}{}
\providecommand{\dateiname}{\jobname}

\vspace{3cm}

\vfill

\footnotesize
\textsc{Quelle}: \titel. Herausgegeben von {\editorInnen}. In: \emph{Arthur Schnitzler: Briefwechsel mit Autorinnen und Autoren}.
 Digitale Edition, https://schnitzler-briefe.acdh.oeaw.ac.at/{\dateiname}.html (Stand \today)
\fi

\end{document}


