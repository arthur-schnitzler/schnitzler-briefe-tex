%% latex-leseansicht-vorspann.tex
%% Vorspann für die Leseansicht.
%% Lädt die gemeinsame Datei latex-vorspann.tex mit nicht gesetztem Schalter.

\newif\ifkorrekturansicht
\korrekturansichtfalse

\input{../tex-inputs/latex-vorspann}


         \renewcommand{\erwaehnteOrte}{Orte: Kaltenleutgeben, Kaltwasserheilanstalt Winternitz, Wien}
         \renewcommand{\erwaehnteWerke}{}
               \section[ Felix Salten an Arthur Schnitzler, {[}9. 6. 1902{]}]{ Felix Salten an Arthur Schnitzler, {[}9. 6. 1902{]}}\nopagebreak\mylabel{v}\rehead{ }\begin{ledgroupsized}[t]{13cm}\normalsize\beginnumbering \toendnotes[C]{\smallbreak\pagebreak[2]} \Standort{CUL, Schnitzler, B 89, A 2.}
\physDesc{Brief, 1 Blatt, 2 Seiten, 335 Zeichen
\newline{}Handschrift: Bleistift, lateinische Kurrent
\newline{}Schnitzler: mit Bleistift datiert: »\substVorne{}\textsuperscript{15}\substDazwischen{}9\substHinten{}/6 902« 
\newline{}Ordnung: mit Bleistift von unbekannter Hand nummeriert: »157« }\toendnotes[C]{\smallbreak}\pstart
           \noindent{}{\pb}Lieber, für heute{ }Abend bin ich leider schon verabredet. Vielleicht \label{K_L03332-1v}\edtext{Freitag}{\lemma{\textnormal{\emph{Freitag}}}\Cendnote{\textnormal{siehe A. S.: \emph{Tagebuch}, 13. 6. 1902}}}\label{K_L03332-1h}? Das \label{K_L03332-2v}\edtext{Kritik Buch}{\lemma{\textnormal{\emph{Kritik Buch}}}\Cendnote{\textnormal{siehe Arthur Schnitzler an Felix Salten, 25. 3. [1902]}}}\label{K_L03332-2h} will ich nächste Woche auf dem Land fertig machen. Wir ziehen voraussichtlich
                  Montag{ }früh nach Kaltenleutgeben\oindex{Kaltenleutgeben@\textbf{Kaltenleutgeben}|pw} in die
                  Anstalt\oindex{Kaltwasserheilanstalt Winternitz@\textbf{Kaltwasserheilanstalt Winternitz}|pwv}, (Kur wegen
               Schnupfen, Nerven ec.{[}){]} damit ich für den Winter ganz {\pb}beisammen bin,\pend
           \pstart
           herzlichst  Ihr {\\[\baselineskip]}\spacefill\mbox{F. S.}\pend
           \leftskip=0em{}\pstart
           \noindent{}Wenn \uline{Frtg}{ }\uuline{nicht}, bitte eine Zeile.\pend
           
         
         \endnumbering\mylabel{h}\end{ledgroupsized}  \newcommand{\dateiname}{L03332}\newcommand{\titel}{Felix Salten an Arthur Schnitzler, [9. 6. 1902]}\newcommand{\editorInnen}{Martin Anton Müller und Laura Untner}%% latex-leseansicht-abspann.tex
%% Abspann für die Leseansicht.
%% Der Schalter \ifkorrekturansicht ist bereits durch den Vorspann gesetzt.

%% latex-abspann.tex
%% Gemeinsamer Abspann für Korrekturansicht und Leseansicht.
%% Setzt den Schalter \ifkorrekturansicht voraus (gesetzt in den
%% einbindenden Dateien latex-korrekturansicht-abspann.tex bzw.
%% latex-leseansicht-abspann.tex).
%% ---------------------------------------------------------------

\normalsize

% Das esempio-Environment wird nur in der Leseansicht benötigt
\ifkorrekturansicht\else
\newenvironment{esempio}[3]%
{
    \vspace{1.5ex}
    \rlap{\underline{#1}}
    \par
    \setlength{\parindent}{0cm}
    \nopagebreak
    \leftskip=#2cm
    \rightskip=#3cm
}
{
    \par
}
\fi

\doendnotes{C}
\bigskip
\vfill

\clearpage

\footnotesize

\ifkorrekturansicht
  \lohead{\textsc{register}}
\fi

% theindex-Environment neu definieren ohne reledmac
\makeatletter
\renewenvironment{theindex}{%
  \ifkorrekturansicht
    \section*{\indexname}%
  \else
    \subsubsection*{Index der erwähnten Entitäten}%
  \fi
  \setlength{\parindent}{0pt}%
  \setlength{\parskip}{0pt plus 0.3pt}%
  \let\item\@idxitem
}{%
  \ifkorrekturansicht\clearpage\fi
}
\makeatother

\IfFileExists{\jobname-pw.ind}{\input{\jobname-pw.ind}}{}

% Quellenangabe nur in der Leseansicht
\ifkorrekturansicht\else
% Fallback-Definitionen, falls die .tex-Datei \titel etc. nicht gesetzt hat
\providecommand{\titel}{}
\providecommand{\editorInnen}{}
\providecommand{\dateiname}{\jobname}

\vspace{3cm}

\vfill

\footnotesize
\textsc{Quelle}: \titel. Herausgegeben von {\editorInnen}. In: \emph{Arthur Schnitzler: Briefwechsel mit Autorinnen und Autoren}.
 Digitale Edition, https://schnitzler-briefe.acdh.oeaw.ac.at/{\dateiname}.html (Stand \today)
\fi

\end{document}


      