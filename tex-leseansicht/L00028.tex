%% latex-leseansicht-vorspann.tex
%% Vorspann für die Leseansicht.
%% Lädt die gemeinsame Datei latex-vorspann.tex mit nicht gesetztem Schalter.

\newif\ifkorrekturansicht
\korrekturansichtfalse

\input{../tex-inputs/latex-vorspann}


         
         \renewcommand{\erwaehntePersonen}{Personen: Hermann Bahr, Richard Beer-Hofmann,  Franz Joseph I. von Österreich-Ungarn, Hugo von Hofmannsthal, Felix Salten}
         \renewcommand{\erwaehnteOrte}{Orte: Bad Aussee, Bad Ischl, Salzburg, Strobl, Wien}
         \renewcommand{\erwaehnteWerke}{Werke: Allgemeine Kunst-Chronik, Die Mozart-Centenarfeier in Salzburg, Die Überwindung des Naturalismus}
               \section[Arthur Schnitzler an Hugo von Hofmannsthal, 11. 8. 1891]{ Arthur Schnitzler an Hugo von Hofmannsthal, 11. 8. 1891}\nopagebreak\mylabel{v}\rehead{ }\begin{ledgroupsized}[t]{13cm}\normalsize\beginnumbering \toendnotes[C]{\smallbreak\pagebreak[2]} \Standort{FDH, Hs-30885,10.}
\physDesc{Brief, 1 Blatt, 4 Seiten, 912 Zeichen
\newline{}Handschrift: schwarze Tinte, deutsche Kurrent}\buchAbdrucke{\weitereDrucke{1) Hugo von Hofmannsthal, Arthur Schnitzler: \emph{Briefwechsel}. Hg. Therese Nickl und Heinrich Schnitzler. Frankfurt am Main: \emph{S. Fischer} 1964, S. 11–12.} \weitereDrucke{2) Hermann Bahr, Arthur Schnitzler: \emph{Briefwechsel, Aufzeichnungen, Dokumente (1891–1931)}. Hg. Kurt Ifkovits und Martin Anton Müller. Göttingen: \emph{Wallstein} 2018, S. 7.} }\toendnotes[C]{\smallbreak}\pstart
           \raggedleft{}{\pb}Wien\oindex{Wien@\textbf{Wien}|pw}, 11. Aug. 91\pend
           \pstart
           Lieber Freund,\hspace*{2em}es iſt ſehr wahrſcheinlich, daß ich die \label{K_L00028-1v}\edtext{beiden Feiertage}{\lemma{\textnormal{\emph{beiden Feiertage}}}\Cendnote{\textnormal{Der 15. 8. 1891 – Mariä Himmelfahrt –, war ein
                  Samstag. Dienstag, der 18. 8. war Geburtstag des Kaisers Franz Joseph\pwindex{Franz Joseph I. von Oesterreich-Ungarn 18.08.1830 – 21.11.1916@\textsc{Franz Joseph I. von Österreich-Ungarn} (18.08.1830 – 21.11.1916), \emph{Kaiser}|pwk}.}}}\label{K_L00028-1h} in Iſchl\oindex{Bad Ischl@\textbf{Bad Ischl}|pw} bei meinen Leuten verbringe. Bei dieſer Gelegenheit möcht
               ich ſehr gerne mit Ihnen zuſa{\geminationm}en sein. Nicht wahr, Sie
               theilen mir gleich in 2 Zeilen mit, ob Sie am 15. u.
                  16. Auguſt in \textsc{Strobl}\oindex{Strobl@\textbf{Strobl}|pw}{ }ſind, ob Sie eventuell {\pb}nach Iſchl\oindex{Bad Ischl@\textbf{Bad Ischl}|pw} herüber kommen wollen \textsc{etc}. Von meiner Ankunft verſtändige ich Sie jedenfalls. Ich
               will auch dem \textsc{Beer Hofmann}\pwindex{Beer-Hofmann, Richard 1866-07-11 – 1945-09-26@\textsc{Beer-Hofmann, Richard} (1866-07-11 – 1945-09-26), \emph{Schriftsteller}|pw} nach \textsc{Aussee}\oindex{Bad Aussee@\textbf{Bad Aussee}|pw}{ }ſchreiben (im übrigen hab \label{T_L00028-1v}\edtext{auch \uline{ich}}{\lemma{\textnormal{\emph{auch ich}}}\Cendnote{\textnormal{durch Austauschzeichen die
                  Wortreihenfolge von »\uline{ich} auch« geändert.}}}\label{T_L00028-1h} noch keine
               Zeile von ihm erhalten) – vielleicht ſind wir alle drei zusa{\geminationm}en, {\pb}ſpielen Feiertagspöbel,
               und fühlen uns wohl. –\pend
           \pstart
           Ihr \label{K_L00028-2v}\edtext{Salzburger\oindex{Salzburg@\textbf{Salzburg}|pw} Artikel\pwindex{Mozart-Centenarfeier in Salzburg01. 08. 1891@\emph{Die Mozart-Centenarfeier in Salzburg} {[}01. 08. 1891{]}|pwv}}{\lemma{\textnormal{\emph{Salzburger Artikel}}}\Cendnote{\textnormal{Loris\pwindex{Hofmannsthal, Hugo von 1874-02-01 – 1929-07-15@\textsc{Hofmannsthal, Hugo von} (1874-02-01 – 1929-07-15), \emph{Schriftsteller}|pwk}: \emph{Die Mozart-Centenarfeier in Salzburg}\pwindex{Mozart-Centenarfeier in Salzburg01. 08. 1891@\emph{Die Mozart-Centenarfeier in Salzburg} {[}01. 08. 1891{]}|pwk}. In: \emph{Allgemeine Kunst-Chronik}\pwindex{?? Werk@Nicht ermittelte Verfasserinnen und Verfasser!Allgemeine Kunst-Chronik1881 – 1896@\emph{Allgemeine Kunst-Chronik} {[}1881 – 1896{]}|pwk}, Bd. 15, Nr. 16, 1. August-Heft,
                        1. 8. 1891, S. 423–433.}}}\label{K_L00028-2h} war wunderſchön; wohl
               Ihnen, der ſo was im »Halbſchlaf« aufs Papier träumen kann. Ich bin wach, vielleicht
               ſogar überwach; aber es iſt ein verlogener Herbſtmorgen mit einer Barbierbeckensonne!
               – Haben Sie \textsc{Salten}\pwindex{Salten, Felix 06.09.1869 – 08.10.1945@\textsc{Salten, Felix} (06.09.1869 – 08.10.1945), \emph{Schriftsteller, Journalist}|pw}{ }{\pb}\label{K_L00028-3v}\edtext{über \textsc{Bahr}\pwindex{Bahr, Hermann 19.07.1863 – 15.01.1934@\textsc{Bahr, Hermann} (19.07.1863 – 15.01.1934), \emph{Schriftsteller, Kritiker}|pw}\pwindex{Salten, Felix 06.09.1869 – 08.10.1945@\textsc{Salten, Felix} (06.09.1869 – 08.10.1945), \emph{Schriftsteller, Journalist}!Ueberwindung des Naturalismus01. 08. 1891@\strich\emph{Die Überwindung des Naturalismus} {[}01. 08. 1891{]}|pwv}}{\lemma{\textnormal{\emph{über Bahr}}}\Cendnote{\textnormal{Salten\pwindex{Salten, Felix 06.09.1869 – 08.10.1945@\textsc{Salten, Felix} (06.09.1869 – 08.10.1945), \emph{Schriftsteller, Journalist}|pwk}: \emph{Die Überwindung des Naturalismus}\pwindex{Salten, Felix 06.09.1869 – 08.10.1945@\textsc{Salten, Felix} (06.09.1869 – 08.10.1945), \emph{Schriftsteller, Journalist}!Ueberwindung des Naturalismus01. 08. 1891@\strich\emph{Die Überwindung des Naturalismus} {[}01. 08. 1891{]}|pwk}. In: \emph{Allgemeine Kunst-Chronik}\pwindex{?? Werk@Nicht ermittelte Verfasserinnen und Verfasser!Allgemeine Kunst-Chronik1881 – 1896@\emph{Allgemeine Kunst-Chronik} {[}1881 – 1896{]}|pwk}, Bd. 15, Nr. 16, 1. August-Heft,
                        1. 8. 1891, S. 446–447.}}}\label{K_L00028-3h} geleſen? Ich finde –
               vortrefflich! –\pend
           \pstart
           Leben Sie wohl, hoffentlich plaudern wir bald.{\\[\baselineskip]}Ihr\spacefill\mbox{Arth
                  Schnitz}\pend
           \leftskip=0em{}
         
         \endnumbering\mylabel{h}\end{ledgroupsized}  \newcommand{\dateiname}{L00028}\newcommand{\titel}{Arthur Schnitzler an Hugo von Hofmannsthal, 11. 8. 1891}\newcommand{\editorInnen}{ Martin Anton Müller und Gerd-Hermann Susen}%% latex-leseansicht-abspann.tex
%% Abspann für die Leseansicht.
%% Der Schalter \ifkorrekturansicht ist bereits durch den Vorspann gesetzt.

%% latex-abspann.tex
%% Gemeinsamer Abspann für Korrekturansicht und Leseansicht.
%% Setzt den Schalter \ifkorrekturansicht voraus (gesetzt in den
%% einbindenden Dateien latex-korrekturansicht-abspann.tex bzw.
%% latex-leseansicht-abspann.tex).
%% ---------------------------------------------------------------

\normalsize

% Das esempio-Environment wird nur in der Leseansicht benötigt
\ifkorrekturansicht\else
\newenvironment{esempio}[3]%
{
    \vspace{1.5ex}
    \rlap{\underline{#1}}
    \par
    \setlength{\parindent}{0cm}
    \nopagebreak
    \leftskip=#2cm
    \rightskip=#3cm
}
{
    \par
}
\fi

\doendnotes{C}
\bigskip
\vfill

\clearpage

\footnotesize

\ifkorrekturansicht
  \lohead{\textsc{register}}
\fi

% theindex-Environment neu definieren ohne reledmac
\makeatletter
\renewenvironment{theindex}{%
  \ifkorrekturansicht
    \section*{\indexname}%
  \else
    \subsubsection*{Index der erwähnten Entitäten}%
  \fi
  \setlength{\parindent}{0pt}%
  \setlength{\parskip}{0pt plus 0.3pt}%
  \let\item\@idxitem
}{%
  \ifkorrekturansicht\clearpage\fi
}
\makeatother

\IfFileExists{\jobname-pw.ind}{\input{\jobname-pw.ind}}{}

% Quellenangabe nur in der Leseansicht
\ifkorrekturansicht\else
% Fallback-Definitionen, falls die .tex-Datei \titel etc. nicht gesetzt hat
\providecommand{\titel}{}
\providecommand{\editorInnen}{}
\providecommand{\dateiname}{\jobname}

\vspace{3cm}

\vfill

\footnotesize
\textsc{Quelle}: \titel. Herausgegeben von {\editorInnen}. In: \emph{Arthur Schnitzler: Briefwechsel mit Autorinnen und Autoren}.
 Digitale Edition, https://schnitzler-briefe.acdh.oeaw.ac.at/{\dateiname}.html (Stand \today)
\fi

\end{document}


      