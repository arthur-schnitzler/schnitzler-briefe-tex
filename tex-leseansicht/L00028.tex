%% latex-korrekturansicht-vorspann.tex
%% Vorspann für die Korrekturansicht.
%% Lädt die gemeinsame Datei latex-vorspann.tex mit gesetztem Schalter.

\newif\ifkorrekturansicht
\korrekturansichttrue

\input{../tex-inputs/latex-vorspann}


\section[Arthur Schnitzler an Hugo von Hofmannsthal, 11. 8. 1891]{L00028 Arthur Schnitzler an Hugo von Hofmannsthal,11. 8. 1891}
\nopagebreak\mylabel{L00028v}
\rehead{ }\normalsize\beginnumbering\briefempfaengerindex{Hofmannsthal, Hugo von@\textsc{Hofmannsthal, Hugo von}!zzzSchnitzler, Arthur@\emph{von Arthur Schnitzler}!1891-08-111@{11. 8. 1891}|(be}
\toendnotes[C]{\smallbreak\pagebreak[2]}\Standort{FDH, Hs-30885,10.}
\physDesc{Brief, 1 Blatt, 4 Seiten, 912 Zeichen
\newline{}Handschrift: schwarze Tinte, deutsche Kurrent}
\buchAbdrucke{\weitereDrucke{1) Hugo von Hofmannsthal, Arthur Schnitzler: \emph{Briefwechsel}. Frankfurt am Main: \emph{S. Fischer} 1964, S. 11–12.} \weitereDrucke{2) Hermann Bahr, Arthur Schnitzler: \emph{Briefwechsel, Aufzeichnungen, Dokumente (1891–1931)}. Göttingen: \emph{Wallstein} 2018, S. 7.} }\toendnotes[C]{\smallbreak}
\pstart
           \raggedleft{}{\pb}Wien\oindex{Wien@\textbf{Wien}|pw}, 11. Aug. 91\pend
           \vspace{0.5em}
\pstart
           Lieber Freund,\hspace*{2em}es iſt{ }ſehr wahrſcheinlich, daß ich die \label{K_L00028-1v}\edtext{beiden Feiertage}{\lemma{\textnormal{\emph{beiden Feiertage}}}\Cendnote{\textnormal{Der 15. 8. 1891 – Mariä Himmelfahrt –, war ein
                  Samstag. Dienstag, der 18. 8. war Geburtstag des Kaisers Franz Joseph\pwindex{Franz Joseph I. von Oesterreich-Ungarn 18.\,8.\,1830 Wien – 21.\,11.\,1916 ebd.@\textsc{Franz Joseph I. von Österreich-Ungarn} (18.\,8.\,1830 Wien – 21.\,11.\,1916 ebd.), \emph{Kaiser}|pwk}.}}}\label{K_L00028-1} in Iſchl\oindex{Bad Ischl@\textbf{Bad Ischl}|pw} bei meinen Leuten verbringe. Bei dieſer Gelegenheit möcht
               ich{ }ſehr gerne mit Ihnen zuſa{\geminationm}en sein. Nicht wahr, Sie
               theilen mir gleich in 2 Zeilen mit, ob Sie am 15. u.
                  16. Auguſt in \textsc{Strobl}\oindex{Strobl@\textbf{Strobl}|pw}{ }ſind, ob Sie eventuell {\pb}nach Iſchl\oindex{Bad Ischl@\textbf{Bad Ischl}|pw} herüber kommen wollen \textsc{etc}. Von meiner Ankunft verſtändige ich Sie jedenfalls. Ich
               will auch dem \textsc{Beer Hofmann}\pwindex{Beer-Hofmann, Richard 11.\,7.\,1866 Wien – 26.\,9.\,1945 New York City@\textsc{Beer-Hofmann, Richard} (11.\,7.\,1866 Wien – 26.\,9.\,1945 New York City), \emph{Schriftsteller}|pw} nach \textsc{Aussee}\oindex{Bad Aussee@\textbf{Bad Aussee}|pw}{ }ſchreiben (im übrigen hab \label{T_L00028-1v}\edtext{auch \uline{ich}}{\lemma{\textnormal{\emph{auch ich}}}\Cendnote{\textnormal{durch Austauschzeichen die
                  Wortreihenfolge von »\uline{ich} auch« geändert}}}\label{T_L00028-1} noch keine
               Zeile von ihm erhalten) – vielleicht{ }ſind wir alle drei zusa{\geminationm}en, {\pb}ſpielen Feiertagspöbel,
               und fühlen uns wohl. –\pend
           
\pstart
           Ihr \label{K_L00028-2v}\edtext{Salzburger\oindex{Salzburg@\textbf{Salzburg}|pw} ArtikelSEXref\pwindex{Hofmannsthal, Hugo von 1.\,2.\,1874 Wien – 15.\,7.\,1929 Rodaun@\textsc{Hofmannsthal, Hugo von} (1.\,2.\,1874 Wien – 15.\,7.\,1929 Rodaun), \emph{Schriftsteller}!Mozart-Centenarfeier in Salzburg@\strich\emph{Die Mozart-Centenarfeier in Salzburg}|pwv}}{\lemma{\textnormal{\emph{Salzburger Artikel}}}\Cendnote{\textnormal{Loris\pwindex{Hofmannsthal, Hugo von 1.\,2.\,1874 Wien – 15.\,7.\,1929 Rodaun@\textsc{Hofmannsthal, Hugo von} (1.\,2.\,1874 Wien – 15.\,7.\,1929 Rodaun), \emph{Schriftsteller}|pwk}: \emph{Die Mozart-Centenarfeier in Salzburg}SEXref\pwindex{Hofmannsthal, Hugo von 1.\,2.\,1874 Wien – 15.\,7.\,1929 Rodaun@\textsc{Hofmannsthal, Hugo von} (1.\,2.\,1874 Wien – 15.\,7.\,1929 Rodaun), \emph{Schriftsteller}!Mozart-Centenarfeier in Salzburg@\strich\emph{Die Mozart-Centenarfeier in Salzburg}|pwk}. In: \emph{Allgemeine Kunst-Chronik}\pwindex{Allgemeine Kunst-Chronik@\emph{Allgemeine Kunst-Chronik}|pwk}, Bd. 15, Nr. 16, 1. August-Heft,
                        1. 8. 1891, S. 423–433.}}}\label{K_L00028-2} war wunderſchön; wohl
               Ihnen, der{ }ſo was im »Halbſchlaf« aufs Papier träumen kann. Ich bin wach, vielleicht{ }ſogar überwach; aber es iſt ein verlogener Herbſtmorgen mit einer Barbierbeckensonne!
               – Haben Sie \textsc{Salten}\pwindex{Salten, Felix 6.\,9.\,1869 Budapest – 8.\,10.\,1945 Zuerich@\textsc{Salten, Felix} (6.\,9.\,1869 Budapest – 8.\,10.\,1945 Zürich), \emph{Schriftsteller, Journalist, Chefredakteur}|pw}{ }{\pb}\label{K_L00028-3v}\edtext{über \textsc{Bahr}\pwindex{Bahr, Hermann 19.\,7.\,1863 Linz – 15.\,1.\,1934 Muenchen@\textsc{Bahr, Hermann} (19.\,7.\,1863 Linz – 15.\,1.\,1934 München), \emph{Schriftsteller, Kritiker}|pw}SEXref\pwindex{Salten, Felix 6.\,9.\,1869 Budapest – 8.\,10.\,1945 Zuerich@\textsc{Salten, Felix} (6.\,9.\,1869 Budapest – 8.\,10.\,1945 Zürich), \emph{Schriftsteller, Journalist, Chefredakteur}!Ueberwindung des Naturalismus@\strich\emph{Die Überwindung des Naturalismus}|pwv}}{\lemma{\textnormal{\emph{über Bahr}}}\Cendnote{\textnormal{Salten\pwindex{Salten, Felix 6.\,9.\,1869 Budapest – 8.\,10.\,1945 Zuerich@\textsc{Salten, Felix} (6.\,9.\,1869 Budapest – 8.\,10.\,1945 Zürich), \emph{Schriftsteller, Journalist, Chefredakteur}|pwk}: \emph{Die Überwindung des Naturalismus}SEXref\pwindex{Salten, Felix 6.\,9.\,1869 Budapest – 8.\,10.\,1945 Zuerich@\textsc{Salten, Felix} (6.\,9.\,1869 Budapest – 8.\,10.\,1945 Zürich), \emph{Schriftsteller, Journalist, Chefredakteur}!Ueberwindung des Naturalismus@\strich\emph{Die Überwindung des Naturalismus}|pwk}. In: \emph{Allgemeine Kunst-Chronik}\pwindex{Allgemeine Kunst-Chronik@\emph{Allgemeine Kunst-Chronik}|pwk}, Bd. 15, Nr. 16, 1. August-Heft,
                        1. 8. 1891, S. 446–447.}}}\label{K_L00028-3} geleſen? Ich finde –
               vortrefflich! –\pend
           
\pstart
           Leben Sie wohl, hoffentlich plaudern wir bald.{\\[\baselineskip]}Ihr\spacefill\mbox{Arth
                  Schnitz}\pend
           \leftskip=0em{}\selectlanguage{ngerman}\endnumbering\briefempfaengerindex{Hofmannsthal, Hugo von@\textsc{Hofmannsthal, Hugo von}!zzzSchnitzler, Arthur@\emph{von Arthur Schnitzler}!1891-08-111@{11. 8. 1891}|)be}\mylabel{L00028h}  \normalsize

\doendnotes{C}
\bigskip
\vfill

\clearpage

\footnotesize

\lohead{\textsc{register}}

% Definiere theindex-Environment komplett neu ohne reledmac
\makeatletter
\renewenvironment{theindex}{%
  \section*{\indexname}%
  \setlength{\parindent}{0pt}%
  \setlength{\parskip}{0pt plus 0.3pt}%
  \let\item\@idxitem
}{%
  \clearpage
}
\makeatother

\IfFileExists{\jobname-pw.ind}{\input{\jobname-pw.ind}}{}

\end{document}

      