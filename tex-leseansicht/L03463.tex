%% latex-leseansicht-vorspann.tex
%% Vorspann für die Leseansicht.
%% Lädt die gemeinsame Datei latex-vorspann.tex mit nicht gesetztem Schalter.

\newif\ifkorrekturansicht
\korrekturansichtfalse

\input{../tex-inputs/latex-vorspann}


\section[ Paul Goldmann an Arthur Schnitzler, 29. 7. 1908]{L03463 Paul Goldmann an Arthur Schnitzler,  29. 7. 1908}
\nopagebreak\mylabel{L03463v}
\rehead{ }\normalsize\beginnumbering\briefempfaengerindex{Schnitzler, Arthur@\textsc{Schnitzler, Arthur}!zzzGoldmann, Paul@\emph{von Paul Goldmann}!1908-07-291@{29. 7. 1908}|(be}
\toendnotes[C]{\smallbreak\pagebreak[2]}
\correspDesc{Versand  durch Paul Goldmann am 29. 7. 1908 in Berlin
\newline{}Erhalt  durch Arthur Schnitzler im Zeitraum [30. 7. 1908
                  – 3. 8. 1908?] in Seis am Schlern}\toendnotes[C]{\smallbreak}
\Standort{DLA, A:Schnitzler, HS.NZ85.1.3175.}
\physDesc{Bildpostkarte, 426 Zeichen
\newline{}Handschrift: blaue Tinte, deutsche Kurrent
\newline{}Versand: Stempel: »\nobreak{}\oindex{Berlin@\textbf{Berlin}, \emph{Hauptstadt}|pwk}Berlin, S.
                                       W\textcolor{gray}{.} 11, 29. 7. 08, 11–12V.\nobreak{}«.  }\toendnotes[C]{\smallbreak}\pstart{}\textsc{{\pb}Herrn}\pend{}\pstart{}\textsc{Dr. Arthur Schnitzler}\pend{}\pstart{}\textsc{Seis am Schlern\oindex{Seis am Schlern@\textbf{Seis am Schlern}|pw}}\pend{}\pstart{}\textsc{Villa Heufler\oindex{Villa Heufler@\textbf{Villa Heufler}, \emph{Beherbergungsgebäude}|pw}}\pend{}\pstart{}\textsc{Tirol\oindex{Seis am Schlern@\textbf{Seis am Schlern}|pw}.}\pend{}{\bigskip}
\pstart
           \noindent{}{\pb}\textcolor{gray}{\textbf{\textbf{Alt-Berlin\oindex{Berlin@\textbf{Berlin}, \emph{Hauptstadt}|pw}}}}\hspace*{2.5em}\textcolor{gray}{\textbf{Loge Royal-York\oindex{Villa Kamecke@\textbf{Villa Kamecke}, \emph{Gebäude}|pw} in der Dorotheenſtraße\oindex{Dorotheenstraße@\textbf{Dorotheenstraße}, \emph{Straße}|pw} im Jahre 1833.}}\pend
           \vspace{1em}
\pstart
           {\pb}29. 7. 08.\pend
           \vspace{0.5em}
\pstart
           Lieber Freund, Ich danke Dir für Deine Karte u. habe mich{ }ſehr
               gefreut, daß Du wieder einmal meiner gedacht haſt. Ich bin noch in Berlin\oindex{Berlin@\textbf{Berlin}, \emph{Hauptstadt}|pw}\strikeout{,} u. \label{K_L03463-1v}\edtext{verheirate}{\lemma{\textnormal{\emph{verheirate}}}\Cendnote{\textnormal{Goldmann\pwindex{Goldmann, Paul 31.\,1.\,1865 Breslau – 25.\,9.\,1935 Wien@\textsc{Goldmann, Paul} (31.\,1.\,1865 Breslau – 25.\,9.\,1935 Wien), \emph{Schriftsteller, Journalist}|pwk} und Eva Marie Kobler, geb. Fränkel\pwindex{Goldmann, Eva Marie 27.\,10.\,1877 Wien – 2.\,11.\,1937 ebd.@\textsc{Goldmann, Eva Marie} (27.\,10.\,1877 Wien – 2.\,11.\,1937 ebd.)|pwk}, heirateten am 4. 8. 1908. Schnitzler war die aus Wien\oindex{Wien@\textbf{Wien}, \emph{Verwaltungsgebiet}|pwk} stammende
                  Braut spätestens seit 9. 8. 1900 bekannt. Das Paar hatte eine gemeinsame Tochter, die am
                     29. 5. 1911 zur Welt kam: Franziska Goldmann\pwindex{Goldmann, Franziska 29.\,5.\,1911 Berlin – 19.\,8.\,1963 Rio de Janeiro@\textsc{Goldmann, Franziska} (29.\,5.\,1911 Berlin – 19.\,8.\,1963 Rio de Janeiro), \emph{Schauspielerin}|pwk}.}}}\label{K_L03463-1} mich hier nächſte Woche mit
               Frau \textsc{Kobler\pwindex{Goldmann, Eva Marie 27.\,10.\,1877 Wien – 2.\,11.\,1937 ebd.@\textsc{Goldmann, Eva Marie} (27.\,10.\,1877 Wien – 2.\,11.\,1937 ebd.)|pw}}. Wir gehen dann zunächſt nach Marienbad\oindex{Marienbad@\textbf{Marienbad}|pw},
               vielleicht{ }ſpäter noch an die See. Meine zukünftige Frau\pwindex{Goldmann, Eva Marie 27.\,10.\,1877 Wien – 2.\,11.\,1937 ebd.@\textsc{Goldmann, Eva Marie} (27.\,10.\,1877 Wien – 2.\,11.\,1937 ebd.)|pwv} u. ich{ }ſenden Dir u. Deiner Frau\pwindex{Schnitzler, Olga 17.\,1.\,1882 Wien – 13.\,1.\,1970 Lugano@\textsc{Schnitzler, Olga} (17.\,1.\,1882 Wien – 13.\,1.\,1970 Lugano), \emph{Schauspielerin, Sängerin}|pwv} herzliche Grüße.\pend
           
\pstart
           Dein getreuer {\\[\baselineskip]}\spacefill\mbox{Paul Goldmann.}\pend
           \leftskip=0em{}\selectlanguage{ngerman}\endnumbering\briefempfaengerindex{Schnitzler, Arthur@\textsc{Schnitzler, Arthur}!zzzGoldmann, Paul@\emph{von Paul Goldmann}!1908-07-291@{29. 7. 1908}|)be}\mylabel{L03463h}  \newcommand{\dateiname}{L03463}\newcommand{\titel}{Paul Goldmann an Arthur Schnitzler, 29. 7. 1908}\newcommand{\editorInnen}{Martin Anton Müller und Laura Untner}%% latex-leseansicht-abspann.tex
%% Abspann für die Leseansicht.
%% Der Schalter \ifkorrekturansicht ist bereits durch den Vorspann gesetzt.

%% latex-abspann.tex
%% Gemeinsamer Abspann für Korrekturansicht und Leseansicht.
%% Setzt den Schalter \ifkorrekturansicht voraus (gesetzt in den
%% einbindenden Dateien latex-korrekturansicht-abspann.tex bzw.
%% latex-leseansicht-abspann.tex).
%% ---------------------------------------------------------------

\normalsize

% Das esempio-Environment wird nur in der Leseansicht benötigt
\ifkorrekturansicht\else
\newenvironment{esempio}[3]%
{
    \vspace{1.5ex}
    \rlap{\underline{#1}}
    \par
    \setlength{\parindent}{0cm}
    \nopagebreak
    \leftskip=#2cm
    \rightskip=#3cm
}
{
    \par
}
\fi

\doendnotes{C}
\bigskip
\vfill

\clearpage

\footnotesize

\ifkorrekturansicht
  \lohead{\textsc{register}}
\fi

% theindex-Environment neu definieren ohne reledmac
\makeatletter
\renewenvironment{theindex}{%
  \ifkorrekturansicht
    \section*{\indexname}%
  \else
    \subsubsection*{Index der erwähnten Entitäten}%
  \fi
  \setlength{\parindent}{0pt}%
  \setlength{\parskip}{0pt plus 0.3pt}%
  \let\item\@idxitem
}{%
  \ifkorrekturansicht\clearpage\fi
}
\makeatother

\IfFileExists{\jobname-pw.ind}{\input{\jobname-pw.ind}}{}

% Quellenangabe nur in der Leseansicht
\ifkorrekturansicht\else
% Fallback-Definitionen, falls die .tex-Datei \titel etc. nicht gesetzt hat
\providecommand{\titel}{}
\providecommand{\editorInnen}{}
\providecommand{\dateiname}{\jobname}

\vspace{3cm}

\vfill

\footnotesize
\textsc{Quelle}: \titel. Herausgegeben von {\editorInnen}. In: \emph{Arthur Schnitzler: Briefwechsel mit Autorinnen und Autoren}.
 Digitale Edition, https://schnitzler-briefe.acdh.oeaw.ac.at/{\dateiname}.html (Stand \today)
\fi

\end{document}


