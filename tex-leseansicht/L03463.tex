%% latex-korrekturansicht-vorspann.tex
%% Vorspann für die Korrekturansicht.
%% Lädt die gemeinsame Datei latex-vorspann.tex mit gesetztem Schalter.

\newif\ifkorrekturansicht
\korrekturansichttrue

\input{../tex-inputs/latex-vorspann}


\section[ Paul Goldmann an Arthur Schnitzler, 29. 7. 1908]{L03463 Paul Goldmann an Arthur Schnitzler, 29. 7. 1908}
\nopagebreak\mylabel{L03463v}
\rehead{ }\normalsize\beginnumbering\briefempfaengerindex{Schnitzler, Arthur@\textsc{Schnitzler, Arthur}!zzzGoldmann, Paul@\emph{von Paul Goldmann}!1908-07-291@{29. 7. 1908}|(be}
\toendnotes[C]{\smallbreak\pagebreak[2]}\Standort{DLA, A:Schnitzler, HS.NZ85.1.3175.}
\physDesc{Bildpostkarte, 426 Zeichen
\newline{}Handschrift: 1) blaue Tinte, deutsche Kurrent\hspace{1em}2) blaue Tinte, lateinische Kurrent (\noindent{}Adresse)\hspace{1em}
\newline{}Versand: Stempel: »\nobreak{}\oindex{Berlin@\textbf{Berlin}, \emph{P.PPLC}|pwk}Berlin, S.
                                       W\textcolor{gray}{.} 11, 29. 7. 08, 11–12V.\nobreak{}«.  }\toendnotes[C]{\smallbreak}\pstart{}{\pb}Herrn\pend{}\pstart{}Dr. Arthur Schnitzler\pend{}\pstart{}Seis am Schlern\oindex{Seis am Schlern@\textbf{Seis am Schlern}, \emph{P.PPL}|pw}\pend{}\pstart{}Villa Heufler\oindex{Villa Heufler@\textbf{Villa Heufler}, \emph{Beherbergungsgebäude (K.BHB)}|pw}\pend{}\pstart{}Tirol\oindex{Seis am Schlern@\textbf{Seis am Schlern}, \emph{P.PPL}|pw}.\pend{}{\bigskip}
\pstart
           \noindent{}{\pb}\textcolor{gray}{\textbf{\textbf{Alt-Berlin\oindex{Berlin@\textbf{Berlin}, \emph{P.PPLC}|pw}}}}\hspace*{2.5em}\textcolor{gray}{\textbf{Loge Royal-York\oindex{Villa Kamecke@\textbf{Villa Kamecke}, \emph{Gebäude (K.GBD)}|pw} in der Dorotheenſtraße\oindex{Dorotheenstrasse@\textbf{Dorotheenstraße}, \emph{Straße (K.STR)}|pw} im Jahre 1833.}}\pend
           \vspace{1em}
\pstart
           {\pb}29. 7. 08.\pend
           \vspace{0.5em}
\pstart
           Lieber Freund, Ich danke Dir für Deine Karte u. habe mich ſehr
               gefreut, daß Du wieder einmal meiner gedacht haſt. Ich bin noch in Berlin\oindex{Berlin@\textbf{Berlin}, \emph{P.PPLC}|pw}\strikeout{,} u. \label{K_L03463-1v}\edtext{verheirate}{\lemma{\textnormal{\emph{verheirate}}}\Cendnote{\textnormal{Goldmann\pwindex{Goldmann, Paul 31.01.1865 – 25.09.1935@\textsc{Goldmann, Paul} (31.01.1865 – 25.09.1935), \emph{Schriftsteller/Schriftstellerin, Journalist/Journalistin}|pwk} und Eva Marie Kobler, geb. Fränkel\pwindex{Goldmann, Eva Marie 27.10.1877 – 02.11.1937@\textsc{Goldmann, Eva Marie} (27.10.1877 – 02.11.1937)|pwk}, heirateten am 4. 8. 1908. Schnitzler war die aus Wien\oindex{Wien@\textbf{Wien}, \emph{A.ADM2}|pwk} stammende
                  Braut spätestens seit 9. 8. 1900 bekannt. Das Paar hatte eine gemeinsame Tochter, die am
                     29. 5. 1911 zur Welt kam: Franziska Goldmann\pwindex{Goldmann, Franziska 1911-05-29 – 1963-08-19@\textsc{Goldmann, Franziska} (1911-05-29 – 1963-08-19), \emph{Schauspieler/Schauspielerin}|pwk}.}}}\label{K_L03463-1} mich hier nächſte Woche mit
               Frau \textsc{Kobler\pwindex{Goldmann, Eva Marie 27.10.1877 – 02.11.1937@\textsc{Goldmann, Eva Marie} (27.10.1877 – 02.11.1937)|pw}}. Wir gehen dann zunächſt nach Marienbad\oindex{Marienbad@\textbf{Marienbad}, \emph{P.PPL}|pw},
               vielleicht ſpäter noch an die See. Meine zukünftige Frau\pwindex{Goldmann, Eva Marie 27.10.1877 – 02.11.1937@\textsc{Goldmann, Eva Marie} (27.10.1877 – 02.11.1937)|pwv} u. ich ſenden Dir u. Deiner Frau\pwindex{Schnitzler, Olga 17.01.1882 – 13.01.1970@\textsc{Schnitzler, Olga} (17.01.1882 – 13.01.1970), \emph{Schauspieler/Schauspielerin, Sänger/Sängerin}|pwv} herzliche Grüße.\pend
           
\pstart
           Dein getreuer {\\[\baselineskip]}\spacefill\mbox{Paul Goldmann.}\pend
           \leftskip=0em{}\selectlanguage{ngerman}\endnumbering\briefempfaengerindex{Schnitzler, Arthur@\textsc{Schnitzler, Arthur}!zzzGoldmann, Paul@\emph{von Paul Goldmann}!1908-07-291@{29. 7. 1908}|)be}\mylabel{L03463h}  \normalsize

\doendnotes{C}
\bigskip
\vfill

\clearpage

\footnotesize

\lohead{\textsc{register}}

% Definiere theindex-Environment komplett neu ohne reledmac
\makeatletter
\renewenvironment{theindex}{%
  \section*{\indexname}%
  \setlength{\parindent}{0pt}%
  \setlength{\parskip}{0pt plus 0.3pt}%
  \let\item\@idxitem
}{%
  \clearpage
}
\makeatother

\IfFileExists{\jobname-pw.ind}{\input{\jobname-pw.ind}}{}

\end{document}

      