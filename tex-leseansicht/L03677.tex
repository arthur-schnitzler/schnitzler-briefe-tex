%% latex-korrekturansicht-vorspann.tex
%% Vorspann für die Korrekturansicht.
%% Lädt die gemeinsame Datei latex-vorspann.tex mit gesetztem Schalter.

\newif\ifkorrekturansicht
\korrekturansichttrue

\input{../tex-inputs/latex-vorspann}


\section[Stefan und Friderike Zweig an Arthur Schnitzler, {[}18. 2. 1931?{]}]{L03677 Stefan und Friderike Zweig an Arthur Schnitzler, {[}18. 2. 1931?{]}}
\nopagebreak\mylabel{L03677v}
\rehead{ }\normalsize\beginnumbering\briefempfaengerindex{Schnitzler, Arthur@\textsc{Schnitzler, Arthur}!zzzZweig, Friderike Maria@\emph{von Friderike Maria Zweig}!1931-02-181@{{[}18. 2. 1931?{]}}|(be}\briefempfaengerindex{Schnitzler, Arthur@\textsc{Schnitzler, Arthur}!zzzZweig, Stefan@\emph{von Stefan Zweig}!1931-02-181@{{[}18. 2. 1931?{]}}|(be}
\toendnotes[C]{\smallbreak\pagebreak[2]}\Standort{CUL, Schnitzler, B 118.}
\physDesc{Bildpostkarte, 293 Zeichen
\newline{}Handschrift Stefan Zweig: lila Tinte, lateinische Kurrent
\newline{}Handschrift Friderike Maria Zweig: schwarze Tinte, lateinische Kurrent
\newline{}Versand: 1) Stempel: »\nobreak{}\oindex{Antibes@\textbf{Antibes}, \emph{P.PPL}|pwk}Antibes, Son cap, sa plage de Juan de Pins été hiver\nobreak{}«.   2) Stempel: »\nobreak{}\oindex{Antibes@\textbf{Antibes}, \emph{P.PPL}|pwk}Antibes Alpes Maritimes, 18{[}. 2. 1931{]}, 2\nobreak{}«. }
\buchAbdrucke{\weitereDrucke{Stefan Zweig: \emph{Briefwechsel mit Hermann Bahr, Sigmund Freud, Rainer Maria
                        Rilke und Arthur Schnitzler}. Frankfurt am Main: \emph{S. Fischer} 1987, S. 450–451.} }\toendnotes[C]{\smallbreak}\pstart{}{\pb}D\textsuperscript{r} Arthur
                  Schnitzler\pend{}\pstart{}Wien (Autriche)\oindex{Wien@\textbf{Wien}, \emph{A.ADM2}|pw}\pend{}\pstart{}Sternwartestrasse 71\oindex{Sternwartestrasse 71@\textbf{Sternwartestraße 71}, \emph{Wohngebäude (K.WHS)}|pw}\pend{}{\bigskip}
\pstart
           \noindent{}\centering{}{\pb}\textcolor{gray}{\textbf{LA DOUCE FRANCE\oindex{Frankreich@\textbf{Frankreich}, \emph{A.PCLI}|pw} – CÔTE D’AZUR\oindex{Côte DAzur@\textbf{Côte d’Azur}, \emph{L.CST}|pw}}}\pend
           
\pstart
           \centering{}\textcolor{gray}{\textbf{ANTIBES\oindex{Antibes@\textbf{Antibes}, \emph{P.PPL}|pw}}}\pend
           
\pstart
           \centering{}\textcolor{gray}{\textbf{Maisons contruites dans le Roc}}\pend
           \vspace{1em}
\pstart
           {\pb}Antibes, Hotel du Cap\oindex{Hotel du Cap@\textbf{Hotel du Cap}, \emph{Hotel (K.HTL)}|pw}\pend
           
\pstart{}Lieber verehrter Herr Doktor,\pend\vspace{0.5em}
\pstart
           wenn auch räumlich fern, war ich doch mit ganzem Herzen bei Ihrem grossen \label{K_L03677-1v}\edtext{Erfolge}{\lemma{\textnormal{\emph{Erfolge}}}\Cendnote{\textnormal{Zweig\pwindex{Zweig, Stefan 28.11.1881 – 23.02.1942@\textsc{Zweig, Stefan} (28.11.1881 – 23.02.1942), \emph{Schriftsteller/Schriftstellerin}|pwk} war von Februar bis
                     Mitte März 1931 in Antibes\oindex{Antibes@\textbf{Antibes}, \emph{P.PPL}|pwk}.
                  In diese Zeit fällt in Schnitzlers
                  öffentliches Leben die Uraufführung von \emph{Der Gang
                     zum Weiher}\pwindex{Gang zum Weiher. Dramatische Dichtung@\emph{Der Gang zum Weiher. Dramatische Dichtung}|pwk} am 14. 2. 1931 am \emph{Burgtheater}\orgindex{Burgtheater@Burgtheater|pwk}.}}}\label{K_L03677-1} und beglückwünsche Sie auf das herzlichste.\pend
           
\pstart
           Ihr aufrichtiger{\\[\baselineskip]}\spacefill\mbox{Stefan Zweig}\pend
           \leftskip=0em{}\selectlanguage{ngerman}\vspace{1em}
\pstart
           \noindent{}{[}hs. :{]} Viele ergebene Grüße v.\pend
           \pstart \spacefill\mbox{Friderike Zweig}\pend{}\selectlanguage{ngerman}\endnumbering\briefempfaengerindex{Schnitzler, Arthur@\textsc{Schnitzler, Arthur}!zzzZweig, Friderike Maria@\emph{von Friderike Maria Zweig}!1931-02-181@{{[}18. 2. 1931?{]}}|)be}\briefempfaengerindex{Schnitzler, Arthur@\textsc{Schnitzler, Arthur}!zzzZweig, Stefan@\emph{von Stefan Zweig}!1931-02-181@{{[}18. 2. 1931?{]}}|)be}\mylabel{L03677h}
\begin{anhang}
\end{anhang}\normalsize

\doendnotes{C}
\bigskip
\vfill

\clearpage

\footnotesize

\lohead{\textsc{register}}

% Definiere theindex-Environment komplett neu ohne reledmac
\makeatletter
\renewenvironment{theindex}{%
  \section*{\indexname}%
  \setlength{\parindent}{0pt}%
  \setlength{\parskip}{0pt plus 0.3pt}%
  \let\item\@idxitem
}{%
  \clearpage
}
\makeatother

\IfFileExists{\jobname-pw.ind}{\input{\jobname-pw.ind}}{}

\end{document}

      