%% latex-leseansicht-vorspann.tex
%% Vorspann für die Leseansicht.
%% Lädt die gemeinsame Datei latex-vorspann.tex mit nicht gesetztem Schalter.

\newif\ifkorrekturansicht
\korrekturansichtfalse

\input{../tex-inputs/latex-vorspann}


\section[Arthur Schnitzler an Theodor Herzl, 24. 12. 1900]{L03939 Arthur Schnitzler an Theodor Herzl, 24. 12. 1900}
\nopagebreak\mylabel{L03939v}
\rehead{ }\normalsize\beginnumbering\briefempfaengerindex{Herzl, Theodor@\textsc{Herzl, Theodor}!zzzSchnitzler, Arthur@\emph{von Arthur Schnitzler}!1900-12-241@{24. 12. 1900}|(be}
\toendnotes[C]{\smallbreak\pagebreak[2]}
\correspDesc{Versand  durch Arthur Schnitzler am 24. 12. 1900 in Wien
\newline{}Erhalt  durch Theodor Herzl in Wien}\toendnotes[C]{\smallbreak}
\Standort{Jerusalem, Central Zionist Archives, H1:1926-5.}
\physDesc{,  Blätter,  Seiten
\newline{}Handschrift: , deutsche Kurrent}
\buchAbdrucke{\weitereDrucke{Arthur Schnitzler: \emph{Briefe 1875–1912}. Herausgegeben von Therese Nickl und Heinrich Schnitzler. Frankfurt am Main: \emph{S. Fischer} 1981, S. 399–400.} }\toendnotes[C]{\smallbreak}
\pstart
           \noindent{}{\pb}lieber Doctor Herzl, misverſtehen wir
               einander doch nicht. Von einer unfreundlichen Abſicht hab ich kein Wort geſprochen,
               noch hab ich eine im entfernteſten vermuthet; ich sprach nur von einem \uline{Mangel an Rückſicht}, und Sie werden mir bei näherer
               Überlegung zugeſtehen müſſen, daſs ich dazu alle Berechtigung hatte. Oder wie möchten
               Sie es bezeichnen, we{\geminationn} man einen Autor 6 oder 7 Wochen {\pb}vor
                  Weihnachten um einen Beitrag für die Weihnachtsnu{\geminationm}er
               erſucht, der Autor ihn einſendet, auf die Länge aufmerkſam macht, ſich zu Kürzungen
               bereit erklärt, die Antwort erhält, der Antrag ſei angeno{\geminationm}en, die
               Raumſchwierigkeiten zu beheben ſei Sache des Blattes; we{\geminationn} der Autor endlich die
               Correctur erhält, bei Rückſendung der Correctur neuerdings unaufgefordert ſeine
               Bereitſchaft zu Kürzungen erklärt und endlich ein paar Tage vor Weihnachten die
               Mittheilung {\pb}erhält – der Beitrag könne in der Nummer für
               die er beſti{\geminationm}t war – wegen Raumſchwierigkeiten nicht erſcheinen!– Bedenken Sie noch
               weiters, daſs der Autor einer andern Zeitung dieſe Novelle\pwindex{Schnitzler, Arthur 15.\,5.\,1862 Wien – 21.\,10.\,1931 ebd.@\textsc{Schnitzler, Arthur} (15.\,5.\,1862 Wien – 21.\,10.\,1931 ebd.), \emph{Schriftsteller, Mediziner}!Lieutenant Gustl. Novelle@\strich\emph{Lieutenant Gustl. Novelle}|pwv} für den Fall daſs die N. Fr. Pr.\orgindex{Neue Freie Presse@Neue Freie Presse|pw} ſie wegen ihrer Länge nicht in der
               Weihnachtsnu{\geminationm}er bringen könnte, zugeſagt hätte? – Ich glaube wahrhaftig Sie
               haben keine Urſache ſich zu wundern, daſs ich Ihre Mittheilung mit einigem Ummuth
               aufgeno{\geminationm}en habe. Daſs es {\pb}ausſchließlich Erwägungen
               künſtleriſcher Natur ſind, die mir eine \uline{Theilung} der
                  Novelle\pwindex{Schnitzler, Arthur 15.\,5.\,1862 Wien – 21.\,10.\,1931 ebd.@\textsc{Schnitzler, Arthur} (15.\,5.\,1862 Wien – 21.\,10.\,1931 ebd.), \emph{Schriftsteller, Mediziner}!Lieutenant Gustl. Novelle@\strich\emph{Lieutenant Gustl. Novelle}|pwv} unthunlich
               erſcheinen laſſen, brauche ich Ihnen, der ſie kennt, nicht wiederholt zu verſichern.
               Daſs die Neue Freie Preſſe\orgindex{Neue Freie Presse@Neue Freie Presse|pw} nun bei einer anderen
               Gelegenheit der Novelle\pwindex{Schnitzler, Arthur 15.\,5.\,1862 Wien – 21.\,10.\,1931 ebd.@\textsc{Schnitzler, Arthur} (15.\,5.\,1862 Wien – 21.\,10.\,1931 ebd.), \emph{Schriftsteller, Mediziner}!Lieutenant Gustl. Novelle@\strich\emph{Lieutenant Gustl. Novelle}|pwv} zu
               Lieb eine Beilage erſcheinen läßt, verlange ich nicht und wünſche ich nicht und nehme
               ich auf keinem Falle an. Bemühen Sie ſich nicht weiter in meiner Angelegenheit und
               laſſen Sie mich {\pb}Ihnen nochmals ſagen, daſs ich von Ihrer
               guten Geſi{\geminationn}ung mit Vergnügen überzeugt bin. Die Novelle\pwindex{Schnitzler, Arthur 15.\,5.\,1862 Wien – 21.\,10.\,1931 ebd.@\textsc{Schnitzler, Arthur} (15.\,5.\,1862 Wien – 21.\,10.\,1931 ebd.), \emph{Schriftsteller, Mediziner}!Lieutenant Gustl. Novelle@\strich\emph{Lieutenant Gustl. Novelle}|pwv} war ausſchließlich für die Weihnachtsnu{\geminationm}er
               beſti{\geminationm}t u ich brauche ſie nicht erſt formell zurückzuziehen, da ſie in jener Nu{\geminationm}er
               nicht zum Abdruck kommt. Die Sache iſt erledigt und um jedes weitre
               Misverſtändis unmöglich zu machen, erkläre ich hiermit, daſs ich eine Novelle,
               die ſich zum Erſcheinen in Fortſetzungen eignet, der N. Fr. Pr.\orgindex{Neue Freie Presse@Neue Freie Presse|pw}{\pb}für ihr Romanfeu{[}i{]}lleton mit beſonderm
               Vergnügen überreicht hätte. –\pend
           
\pstart
           Getunlichſt grüßend{\\[\baselineskip]}Ihr{\\[\baselineskip]}\spacefill\mbox{Arthur Schnitzler.}\pend
           \leftskip=0em{}
\pstart
           Wien\oindex{Wien@\textbf{Wien}, \emph{Verwaltungsgebiet}|pw}{ }24. 12. 900.\pend
           \selectlanguage{ngerman}\endnumbering\briefempfaengerindex{Herzl, Theodor@\textsc{Herzl, Theodor}!zzzSchnitzler, Arthur@\emph{von Arthur Schnitzler}!1900-12-241@{24. 12. 1900}|)be}\mylabel{L03939h}
\begin{anhang}
\end{anhang}\newcommand{\dateiname}{L03939}\newcommand{\titel}{Arthur Schnitzler an Theodor Herzl, 24. 12. 1900}\newcommand{\editorInnen}{Herausgegeben von Jahnke, SelmaMüller, Martin Anton}%% latex-leseansicht-abspann.tex
%% Abspann für die Leseansicht.
%% Der Schalter \ifkorrekturansicht ist bereits durch den Vorspann gesetzt.

%% latex-abspann.tex
%% Gemeinsamer Abspann für Korrekturansicht und Leseansicht.
%% Setzt den Schalter \ifkorrekturansicht voraus (gesetzt in den
%% einbindenden Dateien latex-korrekturansicht-abspann.tex bzw.
%% latex-leseansicht-abspann.tex).
%% ---------------------------------------------------------------

\normalsize

% Das esempio-Environment wird nur in der Leseansicht benötigt
\ifkorrekturansicht\else
\newenvironment{esempio}[3]%
{
    \vspace{1.5ex}
    \rlap{\underline{#1}}
    \par
    \setlength{\parindent}{0cm}
    \nopagebreak
    \leftskip=#2cm
    \rightskip=#3cm
}
{
    \par
}
\fi

\doendnotes{C}
\bigskip
\vfill

\clearpage

\footnotesize

\ifkorrekturansicht
  \lohead{\textsc{register}}
\fi

% theindex-Environment neu definieren ohne reledmac
\makeatletter
\renewenvironment{theindex}{%
  \ifkorrekturansicht
    \section*{\indexname}%
  \else
    \subsubsection*{Index der erwähnten Entitäten}%
  \fi
  \setlength{\parindent}{0pt}%
  \setlength{\parskip}{0pt plus 0.3pt}%
  \let\item\@idxitem
}{%
  \ifkorrekturansicht\clearpage\fi
}
\makeatother

\IfFileExists{\jobname-pw.ind}{\input{\jobname-pw.ind}}{}

% Quellenangabe nur in der Leseansicht
\ifkorrekturansicht\else
% Fallback-Definitionen, falls die .tex-Datei \titel etc. nicht gesetzt hat
\providecommand{\titel}{}
\providecommand{\editorInnen}{}
\providecommand{\dateiname}{\jobname}

\vspace{3cm}

\vfill

\footnotesize
\textsc{Quelle}: \titel. Herausgegeben von {\editorInnen}. In: \emph{Arthur Schnitzler: Briefwechsel mit Autorinnen und Autoren}.
 Digitale Edition, https://schnitzler-briefe.acdh.oeaw.ac.at/{\dateiname}.html (Stand \today)
\fi

\end{document}


