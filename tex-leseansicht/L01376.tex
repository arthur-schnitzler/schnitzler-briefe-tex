%% latex-korrekturansicht-vorspann.tex
%% Vorspann für die Korrekturansicht.
%% Lädt die gemeinsame Datei latex-vorspann.tex mit gesetztem Schalter.

\newif\ifkorrekturansicht
\korrekturansichttrue

\input{../tex-inputs/latex-vorspann}


\section[Arthur Schnitzler an Hermann Bahr, 22. 2. 1904]{L01376 Arthur Schnitzler an Hermann Bahr, 22. 2. 1904}
\nopagebreak\mylabel{L01376v}
\rehead{ }\normalsize\beginnumbering\briefempfaengerindex{Bahr, Hermann@\textsc{Bahr, Hermann}!zzzSchnitzler, Arthur@\emph{von Arthur Schnitzler}!1904-02-221@{22. 2. 1904}|(be}
\toendnotes[C]{\smallbreak\pagebreak[2]}\Standort{TMW, HS AM 23367 Ba.}
\physDesc{Brief, 2 Blätter, 7 Seiten, 2851 Zeichen
\newline{}Handschrift: schwarze Tinte, deutsche Kurrent}
\buchAbdrucke{\weitereDrucke{1) Arthur Schnitzler: \emph{The Letters of Arthur Schnitzler to Hermann Bahr}. Chapel Hill: \emph{The University of North Carolina Press} 1978, S. 84–85.} \weitereDrucke{2) Hermann Bahr, Arthur Schnitzler: \emph{Briefwechsel, Aufzeichnungen, Dokumente (1891–1931)}. Göttingen: \emph{Wallstein} 2018, S. 303–304.} }\toendnotes[C]{\smallbreak}
\pstart
           \raggedleft{}{\pb}Wien\oindex{Wien@\textbf{Wien}, \emph{A.ADM2}|pw}, 22. 2. 904\pend
           \vspace{0.5em}
\pstart
           mein lieber Hermann, wir waren eben in Hietzing\oindex{XIII., Hietzing@\textbf{XIII., Hietzing}, \emph{A.ADM3}|pw}, mit Hugo’s\pwindex{Hofmannsthal, Gertrude von 16.03.1880 – 09.11.1959@\textsc{Hofmannsthal, Gertrude von} (16.03.1880 – 09.11.1959)|pw}\pwindex{Hofmannsthal, Hugo von 1874-02-01 – 1929-07-15@\textsc{Hofmannsthal, Hugo von} (1874-02-01 – 1929-07-15), \emph{Schriftsteller/Schriftstellerin}|pw} u Richards\pwindex{Beer-Hofmann, Paula 25.02.1879 – 30.10.1939@\textsc{Beer-Hofmann, Paula} (25.02.1879 – 30.10.1939)|pw}\pwindex{Beer-Hofmann, Richard 1866-07-11 – 1945-09-26@\textsc{Beer-Hofmann, Richard} (1866-07-11 – 1945-09-26), \emph{Schriftsteller/Schriftstellerin}|pw} u Karg\pwindex{Karg-Bebenburg, Edgar von 22.12.1872 – 23.06.1905@\textsc{Karg-Bebenburg, Edgar von} (22.12.1872 – 23.06.1905), \emph{Militär/Militärin}|pw} zuſammen, u da hab ich mit großer Freude gehört, daſs du
               dich viel wohler befindeſt. Nun möchte ich aber gern recht bald ein Wort von dir
               ſelbſt vernehmen, und wiſſen, wie es mit deinen Plänen für die nächſte Zeit ſteht.
               Ich bin seit \label{K_L01376-1v}\edtext{Freitag Abend}{\lemma{\textnormal{\emph{Freitag Abend}}}\Cendnote{\textnormal{eigentlich schon seit dem 19. 2. 1904 abends
                  (einem Donnerstag)}}}\label{K_L01376-1} wieder in Wien\oindex{Wien@\textbf{Wien}, \emph{A.ADM2}|pw}; wir
                  (Olga\pwindex{Schnitzler, Olga 17.01.1882 – 13.01.1970@\textsc{Schnitzler, Olga} (17.01.1882 – 13.01.1970), \emph{Schauspieler/Schauspielerin, Sänger/Sängerin}|pw} u ich) waren {\pb}auf der Rückreise einen
               Tag in Dresden\oindex{Dresden@\textbf{Dresden}, \emph{P.PPLA}|pw} und haben allzukurze Stunden in
               der Galerie\oindex{Gemaeldegalerie Alte Meister@\textbf{Gemäldegalerie Alte Meister}, \emph{Galerie (K.GLR)}|pw} verbracht.\pend
           
\pstart
           Über den Einſamen Weg\pwindex{einsame Weg. Schauspiel in fuenf Akten@\emph{Der einsame Weg. Schauspiel in fünf Akten}|pw} haſt du wohl, ſoweit es
               ſich um den äußerlichen Verlauf des erſten Abends handelt, das weſentliche geleſen.
               Es war ein leidlicher Abfall, Huſten und Unruhe von Anbeginn, matter Beifall nach 2.
               u 3. Akt mit Widerſpruch; Gelächter und ſtarker Beifall nach dem 4. Akt, viel Applaus
               und viel Ziſchen am {\pb}Schluſs. Der 2. Abend, ausverkauft, ging beträchtlich beſſer – und nun ſcheint
               ſich, wie ich aus Berlin\oindex{Berlin@\textbf{Berlin}, \emph{P.PPLC}|pw} höre, das Stück, das bei
               einem Theil der Kritik ſehr lebhafte Anerkennung fand, doch einige Zeit halten zu
               wollen. In Wien\oindex{Wien@\textbf{Wien}, \emph{A.ADM2}|pw} war eigentlich nur das Goldmann\pwindex{Goldmann, Paul 31.01.1865 – 25.09.1935@\textsc{Goldmann, Paul} (31.01.1865 – 25.09.1935), \emph{Schriftsteller/Schriftstellerin, Journalist/Journalistin}|pw}’ſche \label{K_L01376-2v}\edtext{Telegra{\geminationm}\pwindex{Schnitzlers »Einsamer Weg« (Telegramm der »Neuen Freien Presse«)@\emph{Schnitzlers »Einsamer Weg« (Telegramm der »Neuen Freien Presse«)}|pwv}}{\lemma{\textnormal{\emph{Telegramm}}}\Cendnote{\textnormal{[O. V.]: \emph{Schnitzlers »Einsamer Weg«
                        (Telegramm der »Neuen Freien Presse«)}\pwindex{Schnitzlers »Einsamer Weg« (Telegramm der »Neuen Freien Presse«)@\emph{Schnitzlers »Einsamer Weg« (Telegramm der »Neuen Freien Presse«)}|pwk}. In: \emph{Neue Freie Presse}\pwindex{Neue Freie Presse@\emph{Neue Freie Presse}|pwk}, Nr. 14.178, 14. 2. 1904,
                     S. 12.}}}\label{K_L01376-2} wirklich ſchlecht – was er mir perſönlich über das Stück\pwindex{einsame Weg. Schauspiel in fuenf Akten@\emph{Der einsame Weg. Schauspiel in fünf Akten}|pwv} zu ſagen wußte, waren nur
               die folgenden Worte, als ich ihn ein paar Tage nach der Première zum Abſchied {\pb}beſuchte\substVorne{}\textsuperscript{,}\substDazwischen{}:\substHinten{} »Ich ſchreibe eben das \label{K_L01376-3v}\edtext{Feuillet\pwindex{Berliner Theater. »Der einsame Weg«. Von Arthur Schnitzler@\emph{Berliner Theater. »Der einsame Weg«. Von Arthur Schnitzler}|pwv}}{\lemma{\textnormal{\emph{Feuillet}}}\Cendnote{\textnormal{Paul Goldmann\pwindex{Goldmann, Paul 31.01.1865 – 25.09.1935@\textsc{Goldmann, Paul} (31.01.1865 – 25.09.1935), \emph{Schriftsteller/Schriftstellerin, Journalist/Journalistin}|pwk}: \emph{Berliner Theater. »Der einsame Weg«. Von Arthur Schnitzler}\pwindex{Berliner Theater. »Der einsame Weg«. Von Arthur Schnitzler@\emph{Berliner Theater. »Der einsame Weg«. Von Arthur Schnitzler}|pwk}.
                     In: \emph{Neue Freie Presse}\pwindex{Neue Freie Presse@\emph{Neue Freie Presse}|pwk}, Nr. 14.187,
                        23. 2. 1904, S. 1–3.}}}\label{K_L01376-3} über den E. W.\pwindex{einsame Weg. Schauspiel in fuenf Akten@\emph{Der einsame Weg. Schauspiel in fünf Akten}|pw} – Du wirſt keine Freude daran haben.« – Die Fehler des Stücks\pwindex{einsame Weg. Schauspiel in fuenf Akten@\emph{Der einsame Weg. Schauspiel in fünf Akten}|pwv}{ }ſpür ich jetzt wie mir vorko{\geminationm}t ſehr genau: Das Verhältnis zwiſchen Sala\pwindex{einsame Weg. Schauspiel in fuenf Akten@\emph{Der einsame Weg. Schauspiel in fünf Akten}|pwv} u \textsc{Johann\textcolor{gray}{a}} müßte ſchon zu Beginn völlig declarirt sein – das iſt ein techniſcher Fehler, de\substVorne{}\textsuperscript{r}\substDazwischen{}n\substHinten{} gutzumachen in meinen Kräften ſtände. Andres aber dürfte in den Mängeln
               meiner Begabung begründet ſein – ſo insbeſondre eine gewiſſe Steifigkeit im Weſen {\pb}Julians. Immerhin
               bleibt es eine ſchwierige Sache von einer Perſon die Meinung verbreiten zu wollen –
               ſie ſei einmal ein Genie geweſen. Ja we{\geminationn} man das Bild
               ins Foyer hängen könnte, das Julian\pwindex{einsame Weg. Schauspiel in fuenf Akten@\emph{Der einsame Weg. Schauspiel in fünf Akten}|pwv} vor 25 Jahren gemalt und das ihn berühmt gemacht hat! Übrigens –
               vielleicht wäre es auch im Augenblick vergeſſen, da man ſich wieder ins Parket
               begibt.\pend
           
\pstart
           Was ich ſelbſt an dem Stück\pwindex{einsame Weg. Schauspiel in fuenf Akten@\emph{Der einsame Weg. Schauspiel in fünf Akten}|pwv}
               wirklich liebe, iſt der fünfte Akt und die {\pb}Geſtalt des Sala\pwindex{einsame Weg. Schauspiel in fuenf Akten@\emph{Der einsame Weg. Schauspiel in fünf Akten}|pwv}, der gegenüber ich mich,
               eigentlich das erſte Mal in meinem Leben, als eine Art von Schöpfer fühle. Und der
               fünfte Akt bedeutet mir zuweilen etwas mehr als der Abschluſs eines Dramas – ja nicht
               viel weniger als der Abschluſs von 42 ſelbſt gelebten Jahren. \introOben{}–\introOben{} Nun ſeh ich mancherlei vor mir, was mir, wenn ich etwas weniger faul,
               etwas weniger zerſtreut, und mit \strikeout{\textcolor{gray}{×}\-\textcolor{gray}{×}\-\textcolor{gray}{×}\-\textcolor{gray}{×}-} wahrer Intenſität
               begabt wäre, nach dem ſonſtigen Stande meines Innern, eigentlich gelingen
               müßte. –\pend
           
\pstart
           {\pb}– Wir haben in Berlin\oindex{Berlin@\textbf{Berlin}, \emph{P.PPLC}|pw} oft von dir geſprochen und alle Leute die
               du kennſt laſſen dich grüßen. Meine sicilianiſchen\oindex{Sizilien@\textbf{Sizilien}, \emph{A.ADM1}|pw} und korfioliſchen\oindex{Korfu@\textbf{Korfu}, \emph{Insel (N.INS)}|pw} Pläne
               weben weiter – wirſt du auch  ſüdlicher wandern und werden wir uns ſehen? Meine Frau\pwindex{Schnitzler, Olga 17.01.1882 – 13.01.1970@\textsc{Schnitzler, Olga} (17.01.1882 – 13.01.1970), \emph{Schauspieler/Schauspielerin, Sänger/Sängerin}|pwv} grüßt dich herzlich, ich
               desgleichen und wir wären ſehr froh, wenn wir bald noch beſſeres, ganz gutes von dir
               hörten.\pend
           
\pstart
           Dein{\\[\baselineskip]}\spacefill\mbox{Arthur}\pend
           \leftskip=0em{}\selectlanguage{ngerman}\endnumbering\briefempfaengerindex{Bahr, Hermann@\textsc{Bahr, Hermann}!zzzSchnitzler, Arthur@\emph{von Arthur Schnitzler}!1904-02-221@{22. 2. 1904}|)be}\mylabel{L01376h}  \normalsize

\doendnotes{C}
\bigskip
\vfill

\clearpage

\footnotesize

\lohead{\textsc{register}}

% Definiere theindex-Environment komplett neu ohne reledmac
\makeatletter
\renewenvironment{theindex}{%
  \section*{\indexname}%
  \setlength{\parindent}{0pt}%
  \setlength{\parskip}{0pt plus 0.3pt}%
  \let\item\@idxitem
}{%
  \clearpage
}
\makeatother

\IfFileExists{\jobname-pw.ind}{\input{\jobname-pw.ind}}{}

\end{document}

      