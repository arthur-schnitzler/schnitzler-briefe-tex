%% latex-leseansicht-vorspann.tex
%% Vorspann für die Leseansicht.
%% Lädt die gemeinsame Datei latex-vorspann.tex mit nicht gesetztem Schalter.

\newif\ifkorrekturansicht
\korrekturansichtfalse

\input{../tex-inputs/latex-vorspann}


\section[Arthur Schnitzler an Hermann Bahr, 22. 2. 1904]{L01376 Arthur Schnitzler an Hermann Bahr, 22. 2. 1904}
\nopagebreak\mylabel{L01376v}
\rehead{ }\normalsize\beginnumbering\briefempfaengerindex{Bahr, Hermann@\textsc{Bahr, Hermann}!zzzSchnitzler, Arthur@\emph{von Arthur Schnitzler}!1904-02-221@{22. 2. 1904}|(be}
\toendnotes[C]{\smallbreak\pagebreak[2]}
\correspDesc{Versand  durch Arthur Schnitzler am 22. 2. 1904 in Wien
\newline{}Erhalt  durch Hermann Bahr im Zeitraum [22. 2. 1904
                  – 26. 2. 1904?] \textbf{Ort fehlend} }\toendnotes[C]{\smallbreak}
\Standort{TMW, HS AM 23367 Ba.}
\physDesc{Brief, 2 Blätter, 7 Seiten, 2851 Zeichen
\newline{}Handschrift: schwarze Tinte, deutsche Kurrent}
\buchAbdrucke{\weitereDrucke{1) \emph{22. 2. 1904.} In: Arthur Schnitzler: \emph{The Letters of Arthur Schnitzler to Hermann Bahr}. Edited, annotated, and with an introduction, by Donald G. Daviau. Chapel Hill: \emph{The University of North Carolina Press} 1978, S. 84–85 (University of North Carolina studies in the Germanic languages
                        and literatures, 89).} \weitereDrucke{2) Hermann Bahr, Arthur Schnitzler: \emph{Briefwechsel, Aufzeichnungen, Dokumente (1891–1931)}. Herausgegeben von Kurt Ifkovits und Martin Anton Müller. Göttingen: \emph{Wallstein} 2018, S. 303–304.} }\toendnotes[C]{\smallbreak}
\pstart
           \raggedleft{}{\pb}Wien\oindex{Wien@\textbf{Wien}, \emph{Verwaltungsgebiet}|pw}, 22. 2. 904\pend
           \vspace{0.5em}
\pstart
           mein lieber Hermann, wir waren eben in Hietzing\oindex{XIII., Hietzing@\textbf{XIII., Hietzing}, \emph{Verwaltungsgebiet}|pw}, mit Hugo’s\pwindex{Hofmannsthal, Gertrude von 16.\,3.\,1880 Wien – 9.\,11.\,1959 Paddington@\textsc{Hofmannsthal, Gertrude von} (16.\,3.\,1880 Wien – 9.\,11.\,1959 Paddington)|pw}\pwindex{Hofmannsthal, Hugo von 1.\,2.\,1874 Wien – 15.\,7.\,1929 Rodaun@\textsc{Hofmannsthal, Hugo von} (1.\,2.\,1874 Wien – 15.\,7.\,1929 Rodaun), \emph{Schriftsteller}|pw} u Richards\pwindex{Beer-Hofmann, Paula 25.\,2.\,1879 Wien – 30.\,10.\,1939 Zürich@\textsc{Beer-Hofmann, Paula} (25.\,2.\,1879 Wien – 30.\,10.\,1939 Zürich)|pw}\pwindex{Beer-Hofmann, Richard 11.\,7.\,1866 Wien – 26.\,9.\,1945 New York City@\textsc{Beer-Hofmann, Richard} (11.\,7.\,1866 Wien – 26.\,9.\,1945 New York City), \emph{Schriftsteller}|pw} u Karg\pwindex{Karg-Bebenburg, Edgar von 22.\,12.\,1872 – 23.\,6.\,1905 Salzburg@\textsc{Karg-Bebenburg, Edgar von} (22.\,12.\,1872 – 23.\,6.\,1905 Salzburg), \emph{Militär}|pw} zuſammen, u da hab ich mit großer Freude gehört, daſs du
               dich viel wohler befindeſt. Nun möchte ich aber gern recht bald ein Wort von dir{ }ſelbſt vernehmen, und wiſſen, wie es mit deinen Plänen für die nächſte Zeit{ }ſteht.
               Ich bin seit \label{K_L01376-1v}\edtext{Freitag Abend}{\lemma{\textnormal{\emph{Freitag Abend}}}\Cendnote{\textnormal{eigentlich schon seit dem 19. 2. 1904 abends
                  (einem Donnerstag)}}}\label{K_L01376-1} wieder in Wien\oindex{Wien@\textbf{Wien}, \emph{Verwaltungsgebiet}|pw}; wir
               (Olga\pwindex{Schnitzler, Olga 17.\,1.\,1882 Wien – 13.\,1.\,1970 Lugano@\textsc{Schnitzler, Olga} (17.\,1.\,1882 Wien – 13.\,1.\,1970 Lugano), \emph{Schauspielerin, Sängerin}|pw} u ich) waren {\pb}auf der Rückreise einen
               Tag in Dresden\oindex{Dresden@\textbf{Dresden}|pw} und haben allzukurze Stunden in
               der Galerie\oindex{Gemäldegalerie Alte Meister@\textbf{Gemäldegalerie Alte Meister}, \emph{Galerie}|pw} verbracht.\pend
           
\pstart
           Über den Einſamen Weg\pwindex{Schnitzler, Arthur 15.\,5.\,1862 Wien – 21.\,10.\,1931 ebd.@\textsc{Schnitzler, Arthur} (15.\,5.\,1862 Wien – 21.\,10.\,1931 ebd.), \emph{Schriftsteller, Mediziner}!einsame Weg. Schauspiel in fünf Akten@\strich\emph{Der einsame Weg. Schauspiel in fünf Akten}|pw} haſt du wohl,{ }ſoweit es{ }ſich um den äußerlichen Verlauf des erſten Abends handelt, das weſentliche geleſen.
               Es war ein leidlicher Abfall, Huſten und Unruhe von Anbeginn, matter Beifall nach 2.
               u 3. Akt mit Widerſpruch; Gelächter und{ }ſtarker Beifall nach dem 4. Akt, viel Applaus
               und viel Ziſchen am {\pb}Schluſs. Der 2. Abend, ausverkauft, ging beträchtlich beſſer – und nun{ }ſcheint{ }ſich, wie ich aus Berlin\oindex{Berlin@\textbf{Berlin}, \emph{Hauptstadt}|pw} höre, das Stück, das bei
               einem Theil der Kritik{ }ſehr lebhafte Anerkennung fand, doch einige Zeit halten zu
               wollen. In Wien\oindex{Wien@\textbf{Wien}, \emph{Verwaltungsgebiet}|pw} war eigentlich nur das Goldmann\pwindex{Goldmann, Paul 31.\,1.\,1865 Breslau – 25.\,9.\,1935 Wien@\textsc{Goldmann, Paul} (31.\,1.\,1865 Breslau – 25.\,9.\,1935 Wien), \emph{Schriftsteller, Journalist}|pw}’ſche \label{K_L01376-2v}\edtext{Telegra{\geminationm}\pwindex{Schnitzlers »Einsamer Weg« (Telegramm der »Neuen Freien Presse«)@\emph{Schnitzlers »Einsamer Weg« (Telegramm der »Neuen Freien Presse«)}|pwv}}{\lemma{\textnormal{\emph{Telegramm}}}\Cendnote{\textnormal{[O. V.]: \emph{Schnitzlers »Einsamer Weg«
                        (Telegramm der »Neuen Freien Presse«)}\pwindex{Schnitzlers »Einsamer Weg« (Telegramm der »Neuen Freien Presse«)@\emph{Schnitzlers »Einsamer Weg« (Telegramm der »Neuen Freien Presse«)}|pwk}. In: \emph{Neue Freie Presse}\pwindex{Neue Freie Presse@\emph{Neue Freie Presse}|pwk}, Nr. 14.178, 14. 2. 1904,
                     S. 12.}}}\label{K_L01376-2} wirklich{ }ſchlecht – was er mir perſönlich über das Stück\pwindex{Schnitzler, Arthur 15.\,5.\,1862 Wien – 21.\,10.\,1931 ebd.@\textsc{Schnitzler, Arthur} (15.\,5.\,1862 Wien – 21.\,10.\,1931 ebd.), \emph{Schriftsteller, Mediziner}!einsame Weg. Schauspiel in fünf Akten@\strich\emph{Der einsame Weg. Schauspiel in fünf Akten}|pwv} zu{ }ſagen wußte, waren nur
               die folgenden Worte, als ich ihn ein paar Tage nach der Première zum Abſchied {\pb}beſuchte\substVorne{}\textsuperscript{,}\substDazwischen{}:\substHinten{} »Ich{ }ſchreibe eben das \label{K_L01376-3v}\edtext{Feuillet\pwindex{Goldmann, Paul 31.\,1.\,1865 Breslau – 25.\,9.\,1935 Wien@\textsc{Goldmann, Paul} (31.\,1.\,1865 Breslau – 25.\,9.\,1935 Wien), \emph{Schriftsteller, Journalist}!Berliner Theater. »Der einsame Weg«. Von Arthur Schnitzler@\strich\emph{Berliner Theater. »Der einsame Weg«. Von Arthur Schnitzler}|pwv}}{\lemma{\textnormal{\emph{Feuillet}}}\Cendnote{\textnormal{Paul Goldmann\pwindex{Goldmann, Paul 31.\,1.\,1865 Breslau – 25.\,9.\,1935 Wien@\textsc{Goldmann, Paul} (31.\,1.\,1865 Breslau – 25.\,9.\,1935 Wien), \emph{Schriftsteller, Journalist}|pwk}: \emph{Berliner Theater. »Der einsame Weg«. Von Arthur Schnitzler}\pwindex{Goldmann, Paul 31.\,1.\,1865 Breslau – 25.\,9.\,1935 Wien@\textsc{Goldmann, Paul} (31.\,1.\,1865 Breslau – 25.\,9.\,1935 Wien), \emph{Schriftsteller, Journalist}!Berliner Theater. »Der einsame Weg«. Von Arthur Schnitzler@\strich\emph{Berliner Theater. »Der einsame Weg«. Von Arthur Schnitzler}|pwk}.
                     In: \emph{Neue Freie Presse}\pwindex{Neue Freie Presse@\emph{Neue Freie Presse}|pwk}, Nr. 14.187,
                        23. 2. 1904, S. 1–3.}}}\label{K_L01376-3} über den E. W.\pwindex{Schnitzler, Arthur 15.\,5.\,1862 Wien – 21.\,10.\,1931 ebd.@\textsc{Schnitzler, Arthur} (15.\,5.\,1862 Wien – 21.\,10.\,1931 ebd.), \emph{Schriftsteller, Mediziner}!einsame Weg. Schauspiel in fünf Akten@\strich\emph{Der einsame Weg. Schauspiel in fünf Akten}|pw} – Du wirſt keine Freude daran haben.« – Die Fehler des Stücks\pwindex{Schnitzler, Arthur 15.\,5.\,1862 Wien – 21.\,10.\,1931 ebd.@\textsc{Schnitzler, Arthur} (15.\,5.\,1862 Wien – 21.\,10.\,1931 ebd.), \emph{Schriftsteller, Mediziner}!einsame Weg. Schauspiel in fünf Akten@\strich\emph{Der einsame Weg. Schauspiel in fünf Akten}|pwv}{ }ſpür ich jetzt wie mir vorko{\geminationm}t{ }ſehr genau: Das Verhältnis zwiſchen Sala\pwindex{Schnitzler, Arthur 15.\,5.\,1862 Wien – 21.\,10.\,1931 ebd.@\textsc{Schnitzler, Arthur} (15.\,5.\,1862 Wien – 21.\,10.\,1931 ebd.), \emph{Schriftsteller, Mediziner}!einsame Weg. Schauspiel in fünf Akten@\strich\emph{Der einsame Weg. Schauspiel in fünf Akten}|pwv} u \textsc{Johann\textcolor{gray}{a}} müßte{ }ſchon zu Beginn völlig declarirt sein – das iſt ein techniſcher Fehler, de\substVorne{}\textsuperscript{r}\substDazwischen{}n\substHinten{} gutzumachen in meinen Kräften{ }ſtände. Andres aber dürfte in den Mängeln
               meiner Begabung begründet{ }ſein –{ }ſo insbeſondre eine gewiſſe Steifigkeit im Weſen {\pb}Julians. Immerhin
               bleibt es eine{ }ſchwierige Sache von einer Perſon die Meinung verbreiten zu wollen –{ }ſie{ }ſei einmal ein Genie geweſen. Ja we{\geminationn} man das Bild
               ins Foyer hängen könnte, das Julian\pwindex{Schnitzler, Arthur 15.\,5.\,1862 Wien – 21.\,10.\,1931 ebd.@\textsc{Schnitzler, Arthur} (15.\,5.\,1862 Wien – 21.\,10.\,1931 ebd.), \emph{Schriftsteller, Mediziner}!einsame Weg. Schauspiel in fünf Akten@\strich\emph{Der einsame Weg. Schauspiel in fünf Akten}|pwv} vor 25 Jahren gemalt und das ihn berühmt gemacht hat! Übrigens –
               vielleicht wäre es auch im Augenblick vergeſſen, da man{ }ſich wieder ins Parket
               begibt.\pend
           
\pstart
           Was ich{ }ſelbſt an dem Stück\pwindex{Schnitzler, Arthur 15.\,5.\,1862 Wien – 21.\,10.\,1931 ebd.@\textsc{Schnitzler, Arthur} (15.\,5.\,1862 Wien – 21.\,10.\,1931 ebd.), \emph{Schriftsteller, Mediziner}!einsame Weg. Schauspiel in fünf Akten@\strich\emph{Der einsame Weg. Schauspiel in fünf Akten}|pwv}
               wirklich liebe, iſt der fünfte Akt und die {\pb}Geſtalt des Sala\pwindex{Schnitzler, Arthur 15.\,5.\,1862 Wien – 21.\,10.\,1931 ebd.@\textsc{Schnitzler, Arthur} (15.\,5.\,1862 Wien – 21.\,10.\,1931 ebd.), \emph{Schriftsteller, Mediziner}!einsame Weg. Schauspiel in fünf Akten@\strich\emph{Der einsame Weg. Schauspiel in fünf Akten}|pwv}, der gegenüber ich mich,
               eigentlich das erſte Mal in meinem Leben, als eine Art von Schöpfer fühle. Und der
               fünfte Akt bedeutet mir zuweilen etwas mehr als der Abschluſs eines Dramas – ja nicht
               viel weniger als der Abschluſs von 42{ }ſelbſt gelebten Jahren. \introOben{}–\introOben{} Nun{ }ſeh ich mancherlei vor mir, was mir, wenn ich etwas weniger faul,
               etwas weniger zerſtreut, und mit \strikeout{\textcolor{gray}{×}\-\textcolor{gray}{×}\-\textcolor{gray}{×}\-\textcolor{gray}{×}-} wahrer Intenſität
               begabt wäre, nach dem{ }ſonſtigen Stande meines Innern, eigentlich gelingen
               müßte. –\pend
           
\pstart
           {\pb}– Wir haben in Berlin\oindex{Berlin@\textbf{Berlin}, \emph{Hauptstadt}|pw} oft von dir geſprochen und alle Leute die
               du kennſt laſſen dich grüßen. Meine sicilianiſchen\oindex{Sizilien@\textbf{Sizilien}, \emph{Land}|pw} und korfioliſchen\oindex{Korfu@\textbf{Korfu}, \emph{Insel}|pw} Pläne
               weben weiter – wirſt du auch  ſüdlicher wandern und werden wir uns{ }ſehen? Meine Frau\pwindex{Schnitzler, Olga 17.\,1.\,1882 Wien – 13.\,1.\,1970 Lugano@\textsc{Schnitzler, Olga} (17.\,1.\,1882 Wien – 13.\,1.\,1970 Lugano), \emph{Schauspielerin, Sängerin}|pwv} grüßt dich herzlich, ich
               desgleichen und wir wären{ }ſehr froh, wenn wir bald noch beſſeres, ganz gutes von dir
               hörten.\pend
           
\pstart
           Dein{\\[\baselineskip]}\spacefill\mbox{Arthur}\pend
           \leftskip=0em{}\selectlanguage{ngerman}\endnumbering\briefempfaengerindex{Bahr, Hermann@\textsc{Bahr, Hermann}!zzzSchnitzler, Arthur@\emph{von Arthur Schnitzler}!1904-02-221@{22. 2. 1904}|)be}\mylabel{L01376h}  \newcommand{\dateiname}{L01376}\newcommand{\titel}{Arthur Schnitzler an Hermann Bahr, 22. 2. 1904}\newcommand{\editorInnen}{Herausgegeben von Martin Anton Müller}%% latex-leseansicht-abspann.tex
%% Abspann für die Leseansicht.
%% Der Schalter \ifkorrekturansicht ist bereits durch den Vorspann gesetzt.

%% latex-abspann.tex
%% Gemeinsamer Abspann für Korrekturansicht und Leseansicht.
%% Setzt den Schalter \ifkorrekturansicht voraus (gesetzt in den
%% einbindenden Dateien latex-korrekturansicht-abspann.tex bzw.
%% latex-leseansicht-abspann.tex).
%% ---------------------------------------------------------------

\normalsize

% Das esempio-Environment wird nur in der Leseansicht benötigt
\ifkorrekturansicht\else
\newenvironment{esempio}[3]%
{
    \vspace{1.5ex}
    \rlap{\underline{#1}}
    \par
    \setlength{\parindent}{0cm}
    \nopagebreak
    \leftskip=#2cm
    \rightskip=#3cm
}
{
    \par
}
\fi

\doendnotes{C}
\bigskip
\vfill

\clearpage

\footnotesize

\ifkorrekturansicht
  \lohead{\textsc{register}}
\fi

% theindex-Environment neu definieren ohne reledmac
\makeatletter
\renewenvironment{theindex}{%
  \ifkorrekturansicht
    \section*{\indexname}%
  \else
    \subsubsection*{Index der erwähnten Entitäten}%
  \fi
  \setlength{\parindent}{0pt}%
  \setlength{\parskip}{0pt plus 0.3pt}%
  \let\item\@idxitem
}{%
  \ifkorrekturansicht\clearpage\fi
}
\makeatother

\IfFileExists{\jobname-pw.ind}{\input{\jobname-pw.ind}}{}

% Quellenangabe nur in der Leseansicht
\ifkorrekturansicht\else
% Fallback-Definitionen, falls die .tex-Datei \titel etc. nicht gesetzt hat
\providecommand{\titel}{}
\providecommand{\editorInnen}{}
\providecommand{\dateiname}{\jobname}

\vspace{3cm}

\vfill

\footnotesize
\textsc{Quelle}: \titel. Herausgegeben von {\editorInnen}. In: \emph{Arthur Schnitzler: Briefwechsel mit Autorinnen und Autoren}.
 Digitale Edition, https://schnitzler-briefe.acdh.oeaw.ac.at/{\dateiname}.html (Stand \today)
\fi

\end{document}


