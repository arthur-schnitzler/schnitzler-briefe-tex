%% latex-leseansicht-vorspann.tex
%% Vorspann für die Leseansicht.
%% Lädt die gemeinsame Datei latex-vorspann.tex mit nicht gesetztem Schalter.

\newif\ifkorrekturansicht
\korrekturansichtfalse

\input{../tex-inputs/latex-vorspann}


         \renewcommand{\erwaehnteWerke}{}
               \section[Arthur Schnitzler an Hermann Bahr, 22. 2. 1904]{ Arthur Schnitzler an Hermann Bahr, 22. 2. 1904}\nopagebreak\mylabel{v}\rehead{ }\begin{ledgroupsized}[t]{13cm}\normalsize\beginnumbering \toendnotes[C]{\smallbreak\pagebreak[2]} \Standort{TMW, HS AM 23367 Ba.}
\physDesc{Brief, 2 Blätter, 7 Seiten
\newline{}Handschrift: schwarze Tinte, deutsche Kurrent}\buchAbdrucke{\weitereDrucke{1) \emph{22. 2. 1904.} In: Arthur Schnitzler: \emph{The Letters of Arthur Schnitzler to Hermann Bahr}. Edited, annotated, and with an introduction, by Donald G.
                        Daviau. Chapel Hill: \emph{The University of North Carolina Press} 1978, S. 84–85 (University of North Carolina studies in the Germanic languages
                        and literatures, 89).} \weitereDrucke{2) Hermann Bahr, Arthur Schnitzler: \emph{Briefwechsel, Aufzeichnungen, Dokumente (1891–1931)}. Hg. Kurt Ifkovits und Martin Anton Müller. Göttingen: \emph{Wallstein} 2018, S. 303–304.} }\toendnotes[C]{\smallbreak}\pstart
           \raggedleft{}{\pb}Wien\oindex{XXXX Ortsangabe fehlt|pw}, 22. 2. 904\pend
           \pstart
           mein lieber Hermann, wir waren eben in Hietzing\oindex{XXXX Ortsangabe fehlt|pw}, mit Hugo’s\pwindex{\textcolor{red}{\textsuperscript{XXXX1 indx}}|pw}\pwindex{\textcolor{red}{\textsuperscript{XXXX1 indx}}|pw} u Richards\pwindex{\textcolor{red}{\textsuperscript{XXXX1 indx}}|pw}\pwindex{\textcolor{red}{\textsuperscript{XXXX1 indx}}|pw} u Karg\pwindex{\textcolor{red}{\textsuperscript{XXXX1 indx}}|pw} zuſammen, u da hab ich mit großer Freude gehört, daſs du dich viel
               wohler befindeſt. Nun möchte ich aber gern recht bald ein Wort von dir ſelbſt
               vernehmen, und wiſſen, wie es mit deinen Plänen für die nächſte Zeit ſteht. Ich bin
               seit \label{K_L01376_1v}\edtext{Freitag Abend}{\lemma{\textnormal{\emph{Freitag Abend}}}\Cendnote{\textnormal{eigentlich schon seit dem
                     19. 2. 1904 abends (einem Donnerstag)}}}\label{K_L01376_1h} wieder in Wien\oindex{XXXX Ortsangabe fehlt|pw}; wir (Olga\pwindex{\textcolor{red}{\textsuperscript{XXXX1 indx}}|pw} u ich) waren
                  {\pb}auf der Rückreise
               einen Tag in Dresden\oindex{XXXX Ortsangabe fehlt|pw} und haben allzukurze Stunden in
               der Galerie\oindex{XXXX Ortsangabe fehlt|pw} verbracht.\pend
           \pstart
           Über den Einſamen Weg\textcolor{red}{\textsuperscript{XXXX indx}} haſt du wohl, ſoweit es ſich
               um den äußerlichen Verlauf des erſten Abends handelt, das weſentliche geleſen. Es war
               ein leidlicher Abfall, Huſten und Unruhe von Anbeginn, matter Beifall nach 2. u 3.
               Akt mit Widerſpruch; Gelächter und ſtarker Beifall nach dem 4. Akt, viel Applaus und
               viel Ziſchen am {\pb}Schluſs. Der 2. Abend, ausverkauft, ging beträchtlich beſſer – und nun ſcheint
               ſich, wie ich aus Berlin\oindex{XXXX Ortsangabe fehlt|pw} höre, das Stück, das bei
               einem Theil der Kritik ſehr lebhafte Anerkennung fand, doch einige Zeit halten zu
               wollen. In Wien\oindex{XXXX Ortsangabe fehlt|pw} war eigentlich nur das Goldmann\pwindex{\textcolor{red}{\textsuperscript{XXXX1 indx}}|pw}’ſche \label{K_L01376_2v}\edtext{Telegra{\geminationm}\textcolor{red}{\textsuperscript{XXXX indx}}}{\lemma{\textnormal{\emph{Telegramm}}}\Cendnote{\textnormal{[O. V.:] \emph{Schnitzlers »Einsamer Weg« (Telegramm
                        der »Neuen Freien Presse«)}\textcolor{red}{\textsuperscript{XXXX indx}}. In: \emph{Neue Freie
                        Presse}\textcolor{red}{\textsuperscript{XXXX indx}}, Nr. 14178, 14. 2. 1904, S. 12.}}}\label{K_L01376_2h}
               wirklich ſchlecht – was er mir perſönlich über das Stück\textcolor{red}{\textsuperscript{XXXX indx}} zu ſagen wußte, waren nur die folgenden Worte, als
               ich ihn ein paar Tage nach der Première zum Abſchied {\pb}beſuchte\substVorne{}\textsuperscript{,}\substDazwischen{}:\substHinten{} »Ich ſchreibe eben das \label{K_L01376_3v}\edtext{Feuillet\textcolor{red}{\textsuperscript{XXXX indx}}}{\lemma{\textnormal{\emph{Feuillet}}}\Cendnote{\textnormal{Paul Goldmann\pwindex{\textcolor{red}{\textsuperscript{XXXX1 indx}}|pwk}: \emph{Berliner Theater. »Der einsame Weg«. Von Arthur Schnitzler}\textcolor{red}{\textsuperscript{XXXX indx}}. In: \emph{Neue Freie Presse}\textcolor{red}{\textsuperscript{XXXX indx}}, Nr. 14187,
                        23. 2. 1904, S. 1–3.}}}\label{K_L01376_3h} über den E. W.\textcolor{red}{\textsuperscript{XXXX indx}} – Du wirſt keine Freude daran haben.« – Die Fehler des Stücks\textcolor{red}{\textsuperscript{XXXX indx}}{ }ſpür ich jetzt wie mir vorko{\geminationm}t ſehr genau: Das Verhältnis zwiſchen Sala\textcolor{red}{\textsuperscript{XXXX indx}} u \textsc{Johann\textcolor{gray}{a}} müßte ſchon zu Beginn völlig declarirt sein – das iſt ein techniſcher Fehler, de\substVorne{}\textsuperscript{r}\substDazwischen{}n\substHinten{} gutzumachen in meinen Kräften ſtände. Andres aber dürfte in den Mängeln
               meiner Begabung begründet ſein – ſo insbeſondre eine gewiſſe Steifigkeit im Weſen {\pb}Julians. Immerhin
               bleibt es eine ſchwierige Sache von einer Perſon die Meinung verbreiten zu wollen –
               ſie ſei einmal ein Genie geweſen. Ja we{\geminationn} man das Bild
               ins Foyer hängen könnte, das Julian\textcolor{red}{\textsuperscript{XXXX indx}} vor 25 Jahren gemalt und das ihn berühmt gemacht hat! Übrigens –
               vielleicht wäre es auch im Augenblick vergeſſen, da man ſich wieder ins Parket
               begibt.\pend
           \pstart
           Was ich ſelbſt an dem Stück\textcolor{red}{\textsuperscript{XXXX indx}}
               wirklich liebe, iſt der fünfte Akt und die {\pb}Geſtalt des Sala\textcolor{red}{\textsuperscript{XXXX indx}}, der gegenüber ich mich,
               eigentlich das erſte Mal in meinem Leben, als eine Art von Schöpfer fühle. Und der
               fünfte Akt bedeutet mir zuweilen etwas mehr als der Abschluſs eines Dramas – ja nicht
               viel weniger als der Abschluſs von 42 ſelbſt gelebten Jahren. \introOben{}–\introOben{} Nun ſeh ich mancherlei vor mir, was mir, wenn ich etwas weniger faul,
               etwas weniger zerſtreut, und mit \strikeout{\textcolor{gray}{×}\-\textcolor{gray}{×}\-\textcolor{gray}{×}\-\textcolor{gray}{×}-} wahrer Intenſität
               begabt wäre, nach dem ſonſtigen Stande meines Innern, eigentlich gelingen
               müßte. –\pend
           \pstart
           {\pb}– Wir haben in Berlin\oindex{XXXX Ortsangabe fehlt|pw} oft von dir geſprochen und alle Leute die du
               kennſt laſſen dich grüßen. Meine sicilianiſchen\oindex{XXXX Ortsangabe fehlt|pw} und
                  korfioliſchen\oindex{XXXX Ortsangabe fehlt|pw} Pläne weben weiter – wirſt du
               auch  ſüdlicher wandern und werden wir uns ſehen? Meine Frau\pwindex{\textcolor{red}{\textsuperscript{XXXX1 indx}}|pwv} grüßt dich herzlich, ich desgleichen und wir wären
               ſehr froh, wenn wir bald noch beſſeres, ganz gutes von dir hörten.\pend
           \pstart
           Dein{\\[\baselineskip]}\spacefill\mbox{Arthur}\pend
           \leftskip=0em{}
         
         \endnumbering\mylabel{h}\end{ledgroupsized}  \newcommand{\dateiname}{L01376}\newcommand{\titel}{Arthur Schnitzler an Hermann Bahr, 22. 2. 1904}\newcommand{\editorInnen}{ Kurt Ifkovits,  Martin Anton Müller}%% latex-leseansicht-abspann.tex
%% Abspann für die Leseansicht.
%% Der Schalter \ifkorrekturansicht ist bereits durch den Vorspann gesetzt.

%% latex-abspann.tex
%% Gemeinsamer Abspann für Korrekturansicht und Leseansicht.
%% Setzt den Schalter \ifkorrekturansicht voraus (gesetzt in den
%% einbindenden Dateien latex-korrekturansicht-abspann.tex bzw.
%% latex-leseansicht-abspann.tex).
%% ---------------------------------------------------------------

\normalsize

% Das esempio-Environment wird nur in der Leseansicht benötigt
\ifkorrekturansicht\else
\newenvironment{esempio}[3]%
{
    \vspace{1.5ex}
    \rlap{\underline{#1}}
    \par
    \setlength{\parindent}{0cm}
    \nopagebreak
    \leftskip=#2cm
    \rightskip=#3cm
}
{
    \par
}
\fi

\doendnotes{C}
\bigskip
\vfill

\clearpage

\footnotesize

\ifkorrekturansicht
  \lohead{\textsc{register}}
\fi

% theindex-Environment neu definieren ohne reledmac
\makeatletter
\renewenvironment{theindex}{%
  \ifkorrekturansicht
    \section*{\indexname}%
  \else
    \subsubsection*{Index der erwähnten Entitäten}%
  \fi
  \setlength{\parindent}{0pt}%
  \setlength{\parskip}{0pt plus 0.3pt}%
  \let\item\@idxitem
}{%
  \ifkorrekturansicht\clearpage\fi
}
\makeatother

\IfFileExists{\jobname-pw.ind}{\input{\jobname-pw.ind}}{}

% Quellenangabe nur in der Leseansicht
\ifkorrekturansicht\else
% Fallback-Definitionen, falls die .tex-Datei \titel etc. nicht gesetzt hat
\providecommand{\titel}{}
\providecommand{\editorInnen}{}
\providecommand{\dateiname}{\jobname}

\vspace{3cm}

\vfill

\footnotesize
\textsc{Quelle}: \titel. Herausgegeben von {\editorInnen}. In: \emph{Arthur Schnitzler: Briefwechsel mit Autorinnen und Autoren}.
 Digitale Edition, https://schnitzler-briefe.acdh.oeaw.ac.at/{\dateiname}.html (Stand \today)
\fi

\end{document}


      