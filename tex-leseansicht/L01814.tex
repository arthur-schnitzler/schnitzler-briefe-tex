%% latex-korrekturansicht-vorspann.tex
%% Vorspann für die Korrekturansicht.
%% Lädt die gemeinsame Datei latex-vorspann.tex mit gesetztem Schalter.

\newif\ifkorrekturansicht
\korrekturansichttrue

\input{../tex-inputs/latex-vorspann}


\section[Arthur Schnitzler an Richard Beer-Hofmann, 2{[}8{]}. 11. 1908]{L01814 Arthur Schnitzler an Richard Beer-Hofmann, 2{[}8{]}. 11. 1908}
\nopagebreak\mylabel{L01814v}
\rehead{ }\normalsize\beginnumbering\briefempfaengerindex{Beer-Hofmann, Richard@\textsc{Beer-Hofmann, Richard}!zzzSchnitzler, Arthur@\emph{von Arthur Schnitzler}!1908-11-284@{2{[}8{]}. 11. 1908}|(be}
\toendnotes[C]{\smallbreak\pagebreak[2]}\Standort{YCGL, MSS 31.}
\physDesc{Brief, 1 Blatt, 3 Seiten, Umschlag, 343 Zeichen
\newline{}Handschrift: Bleistift, deutsche Kurrent}
\buchAbdrucke{\weitereDrucke{Arthur Schnitzler, Richard Beer-Hofmann: \emph{Briefwechsel 1891–1931}. Wien, Zürich: \emph{Europaverlag} 1992, S. 192.} }\toendnotes[C]{\smallbreak}\pstart{}{\pb}\textcolor{gray}{\textbf{Dr. Arthur Schnitzler}}\pend{}\pstart{}\textcolor{gray}{\textbf{Wien XVIII. Spoettelgasse 7\oindex{Edmund-Weiss-Gasse 7@\textbf{Edmund-Weiß-Gasse 7}, \emph{Wohngebäude (K.WHS)}|pw}.}}\pend{}{\bigskip}\pstart{}{\pb}\textsc{Dr. Richard Beer-Hofma{\geminationn}}\pend{}\pstart{}Wien\oindex{XVIII., Waehring@\textbf{XVIII., Währing}, \emph{A.ADM3}|pw}. \pend{}{\bigskip}\vspace{1em}
\pstart
           {\pb}II.\pend
           
\pstart
           \textcolor{gray}{\textbf{Dr. Arthur Schnitzler}}\hfill \label{K_L01814-1v}\edtext{29. 11.}{\lemma{\textnormal{\emph{29. 11.}}}\Cendnote{\textnormal{Bei der Datierung ist Schnitzler ein Fehler
                           unterlaufen.}}}\label{K_L01814-1}\pend
           
\pstart
           \textcolor{gray}{\textbf{Wien XVIII. Spoettelgasse 7\oindex{Edmund-Weiss-Gasse 7@\textbf{Edmund-Weiß-Gasse 7}, \emph{Wohngebäude (K.WHS)}|pw}.}}\pend
           \vspace{0.5em}
\pstart
           Eben ſchrieb ich Ihnen \label{K_L01814-2v}\edtext{den beiliegd
                  Brief}{\lemma{\textnormal{\emph{den beiliegd
                  Brief}}}\Cendnote{\textnormal{Es dürfte sich um den zweiten
                  Brief vom [28. 11. 1908?]
                  handeln. Da der Briefumschlag ohne Briefmarke geblieben ist, dürfte er in den
                  anderen eingelegt gewesen sein.}}}\label{K_L01814-2}. Bleibt alſo nichts andres übrig als den
               morgigen Abend abzuwarten.\pend
           
\pstart
           {\pb}Falls \textsc{Kerr}\pwindex{Kerr, Alfred 25.12.1867 – 12.10.1948@\textsc{Kerr, Alfred} (25.12.1867 – 12.10.1948), \emph{Schriftsteller/Schriftstellerin, Kritiker/Kritikerin}|pw} bei Ihnen ſchriftlich anfrägt, ſo ſchlagen Sie vielleicht auch für morgen Abend
                  \textsc{Meissl}\oindex{Meissl {\kaufmannsund} Schadn@\textbf{Meissl {\kaufmannsund} Schadn}, \emph{Hotel (K.HTL)}|pw} vor. Den ganzen Tag über hab ich \label{K_L01814-3v}\edtext{morgen »geſchäftliche« Beſprechungen}{\lemma{\textnormal{\emph{morgen … Beſprechungen}}}\Cendnote{\textnormal{Das erlaubt die sichere Datierung dieses Korrespondenzstücks, vgl. A. S.: \emph{Tagebuch}, 29. 11. 1908.}}}\label{K_L01814-3}{ }{\pb}(\textsc{Dohnanyi\pwindex{Dohnányi, Ernst von 27.07.1877 – 09.02.1960@\textsc{Dohnányi, Ernst von} (27.07.1877 – 09.02.1960), \emph{Komponist/Komponistin, Pianist/Pianistin}|pw}, Straus\pwindex{Straus, Oscar 06.03.1870 – 11.01.1954@\textsc{Straus, Oscar} (06.03.1870 – 11.01.1954), \emph{Komponist/Komponistin}|pw}, Herzmansky\pwindex{Herzmansky, Bernhard 06.12.1852 – 18.05.1921@\textsc{Herzmansky, Bernhard} (06.12.1852 – 18.05.1921), \emph{Musikverleger/Musikverlegerin}|pw}}.)\pend
           
\pstart
           Ihr{\\[\baselineskip]}\spacefill\mbox{A.}\pend
           \leftskip=0em{}\selectlanguage{ngerman}\endnumbering\briefempfaengerindex{Beer-Hofmann, Richard@\textsc{Beer-Hofmann, Richard}!zzzSchnitzler, Arthur@\emph{von Arthur Schnitzler}!1908-11-284@{2{[}8{]}. 11. 1908}|)be}\mylabel{L01814h}  \normalsize

\doendnotes{C}
\bigskip
\vfill

\clearpage

\footnotesize

\lohead{\textsc{register}}

% Definiere theindex-Environment komplett neu ohne reledmac
\makeatletter
\renewenvironment{theindex}{%
  \section*{\indexname}%
  \setlength{\parindent}{0pt}%
  \setlength{\parskip}{0pt plus 0.3pt}%
  \let\item\@idxitem
}{%
  \clearpage
}
\makeatother

\IfFileExists{\jobname-pw.ind}{\input{\jobname-pw.ind}}{}

\end{document}

      