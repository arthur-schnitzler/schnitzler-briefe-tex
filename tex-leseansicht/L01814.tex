%% latex-leseansicht-vorspann.tex
%% Vorspann für die Leseansicht.
%% Lädt die gemeinsame Datei latex-vorspann.tex mit nicht gesetztem Schalter.

\newif\ifkorrekturansicht
\korrekturansichtfalse

\input{../tex-inputs/latex-vorspann}


\section[Arthur Schnitzler an Richard Beer-Hofmann, 2{[}8{]}. 11. 1908]{L01814 Arthur Schnitzler an Richard Beer-Hofmann, 2[8]. 11. 1908}
\nopagebreak\mylabel{L01814v}
\rehead{ }\normalsize\beginnumbering\briefempfaengerindex{Beer-Hofmann, Richard@\textsc{Beer-Hofmann, Richard}!zzzSchnitzler, Arthur@\emph{von Arthur Schnitzler}!1908-11-284@{2[8]. 11. 1908}|(be}
\toendnotes[C]{\smallbreak\pagebreak[2]}
\correspDesc{Versand  durch Arthur Schnitzler am 2[8]. 11. 1908 in Wien
\newline{}Erhalt  durch Richard Beer-Hofmann im Zeitraum [28. 11. 1908 – 2. 12. 1908?] in Wien}\toendnotes[C]{\smallbreak}
\Standort{YCGL, MSS 31.}
\physDesc{Brief, 1 Blatt, 3 Seiten, Kuvert, 343 Zeichen
\newline{}Handschrift: Bleistift, deutsche Kurrent}
\buchAbdrucke{\weitereDrucke{Arthur Schnitzler, Richard Beer-Hofmann: \emph{Briefwechsel 1891–1931}. Herausgegeben von Konstanze Fliedl. Wien, Zürich: \emph{Europaverlag} 1992, S. 192.} }\toendnotes[C]{\smallbreak}\pstart{}{\pb}\textcolor{gray}{\textbf{Dr. Arthur Schnitzler}}\pend{}\pstart{}\textcolor{gray}{\textbf{Wien XVIII. Spoettelgasse 7\oindex{Wien@\textbf{Wien}!XVIII., Währing@\textbf{XVIII., Währing}!Edmund-Weiß-Gasse 7@\textbf{Edmund-Weiß-Gasse 7}, \emph{Wohngebäude}|pw}.}}\pend{}{\bigskip}\pstart{}{\pb}\textsc{Dr. Richard Beer-Hofma{\geminationn}}\pend{}\pstart{}Wien\oindex{XVIII., Währing@\textbf{XVIII., Währing}, \emph{Verwaltungsgebiet}|pw}. \pend{}{\bigskip}\vspace{1em}
\pstart
           {\pb}II.\pend
           
\pstart
           \textcolor{gray}{\textbf{Dr. Arthur Schnitzler}}\hfill \label{K_L01814-1v}\edtext{29. 11.}{\lemma{\textnormal{\emph{29. 11.}}}\Cendnote{\textnormal{Bei der Datierung ist Schnitzler ein Fehler
                           unterlaufen.}}}\label{K_L01814-1}\pend
           
\pstart
           \textcolor{gray}{\textbf{Wien XVIII. Spoettelgasse 7\oindex{Wien@\textbf{Wien}!XVIII., Währing@\textbf{XVIII., Währing}!Edmund-Weiß-Gasse 7@\textbf{Edmund-Weiß-Gasse 7}, \emph{Wohngebäude}|pw}.}}\pend
           \vspace{0.5em}
\pstart
           Eben{ }ſchrieb ich Ihnen \label{K_L01814-2v}\edtext{den beiliegd
                  Brief}{\lemma{\textnormal{\emph{den beiliegd
                  Brief}}}\Cendnote{\textnormal{Es dürfte sich um den zweiten
                  Brief vom XXXX Auszeichnungsfehler: Dokument L01812 nicht gefunden
                  handeln. Da der Briefumschlag ohne Briefmarke geblieben ist, dürfte er in den
                  anderen eingelegt gewesen sein.}}}\label{K_L01814-2}. Bleibt alſo nichts andres übrig als den
               morgigen Abend abzuwarten.\pend
           
\pstart
           {\pb}Falls \textsc{Kerr}\pwindex{Kerr, Alfred 25.\,12.\,1867 Breslau – 12.\,10.\,1948 Hamburg@\textsc{Kerr, Alfred} (25.\,12.\,1867 Breslau – 12.\,10.\,1948 Hamburg), \emph{Schriftsteller, Kritiker}|pw} bei Ihnen{ }ſchriftlich anfrägt,{ }ſo{ }ſchlagen Sie vielleicht auch für morgen Abend
                  \textsc{Meissl}\oindex{Wien@\textbf{Wien}!I., Innere Stadt@\textbf{I., Innere Stadt}!Meissl {\kaufmannsund} Schadn@\textbf{Meissl {\kaufmannsund} Schadn}, \emph{Hotel}|pw} vor. Den ganzen Tag über hab ich \label{K_L01814-3v}\edtext{morgen »geſchäftliche« Beſprechungen}{\lemma{\textnormal{\emph{morgen … Besprechungen}}}\Cendnote{\textnormal{Das erlaubt die sichere Datierung dieses Korrespondenzstücks, vgl. A. S.: \emph{Tagebuch}, 29. 11. 1908.}}}\label{K_L01814-3}{ }{\pb}(\textsc{Dohnanyi\pwindex{Dohnányi, Ernst von 27.\,7.\,1877 Bratislava – 9.\,2.\,1960 New York City@\textsc{Dohnányi, Ernst von} (27.\,7.\,1877 Bratislava – 9.\,2.\,1960 New York City), \emph{Komponist, Pianist}|pw}, Straus\pwindex{Straus, Oscar 6.\,3.\,1870 Wien – 11.\,1.\,1954 Bad Ischl@\textsc{Straus, Oscar} (6.\,3.\,1870 Wien – 11.\,1.\,1954 Bad Ischl), \emph{Komponist}|pw}, Herzmansky\pwindex{Herzmansky, Bernhard 6.\,12.\,1852 Šternberk – 18.\,5.\,1921 Bad Goisern@\textsc{Herzmansky, Bernhard} (6.\,12.\,1852 Šternberk – 18.\,5.\,1921 Bad Goisern), \emph{Musikverleger}|pw}}.)\pend
           
\pstart
           Ihr{\\[\baselineskip]}\spacefill\mbox{A.}\pend
           \leftskip=0em{}\selectlanguage{ngerman}\endnumbering\briefempfaengerindex{Beer-Hofmann, Richard@\textsc{Beer-Hofmann, Richard}!zzzSchnitzler, Arthur@\emph{von Arthur Schnitzler}!1908-11-284@{2[8]. 11. 1908}|)be}\mylabel{L01814h}  \newcommand{\dateiname}{L01814}\newcommand{\titel}{Arthur Schnitzler an Richard Beer-Hofmann, 2[8]. 11. 1908}\newcommand{\editorInnen}{Martin Anton Müller und Gerd-Hermann Susen}%% latex-leseansicht-abspann.tex
%% Abspann für die Leseansicht.
%% Der Schalter \ifkorrekturansicht ist bereits durch den Vorspann gesetzt.

%% latex-abspann.tex
%% Gemeinsamer Abspann für Korrekturansicht und Leseansicht.
%% Setzt den Schalter \ifkorrekturansicht voraus (gesetzt in den
%% einbindenden Dateien latex-korrekturansicht-abspann.tex bzw.
%% latex-leseansicht-abspann.tex).
%% ---------------------------------------------------------------

\normalsize

% Das esempio-Environment wird nur in der Leseansicht benötigt
\ifkorrekturansicht\else
\newenvironment{esempio}[3]%
{
    \vspace{1.5ex}
    \rlap{\underline{#1}}
    \par
    \setlength{\parindent}{0cm}
    \nopagebreak
    \leftskip=#2cm
    \rightskip=#3cm
}
{
    \par
}
\fi

\doendnotes{C}
\bigskip
\vfill

\clearpage

\footnotesize

\ifkorrekturansicht
  \lohead{\textsc{register}}
\fi

% theindex-Environment neu definieren ohne reledmac
\makeatletter
\renewenvironment{theindex}{%
  \ifkorrekturansicht
    \section*{\indexname}%
  \else
    \subsubsection*{Index der erwähnten Entitäten}%
  \fi
  \setlength{\parindent}{0pt}%
  \setlength{\parskip}{0pt plus 0.3pt}%
  \let\item\@idxitem
}{%
  \ifkorrekturansicht\clearpage\fi
}
\makeatother

\IfFileExists{\jobname-pw.ind}{\input{\jobname-pw.ind}}{}

% Quellenangabe nur in der Leseansicht
\ifkorrekturansicht\else
% Fallback-Definitionen, falls die .tex-Datei \titel etc. nicht gesetzt hat
\providecommand{\titel}{}
\providecommand{\editorInnen}{}
\providecommand{\dateiname}{\jobname}

\vspace{3cm}

\vfill

\footnotesize
\textsc{Quelle}: \titel. Herausgegeben von {\editorInnen}. In: \emph{Arthur Schnitzler: Briefwechsel mit Autorinnen und Autoren}.
 Digitale Edition, https://schnitzler-briefe.acdh.oeaw.ac.at/{\dateiname}.html (Stand \today)
\fi

\end{document}


