\input{../tex-inputs/latex-pdf-vorspann}
\begin{center}
            \textcolor{red}{ENTWURF. ENTZIFFERUNG NOCH NICHT KORREKTURGELESEN}
                      \end{center}
            
               \section[Arthur Schnitzler an Richard Beer-Hofmann, 2{[}8{]}. 11. 1908]{ Arthur Schnitzler an Richard Beer-Hofmann, 2{[}8{]}. 11. 1908}\nopagebreak\mylabel{v}\rehead{ }\begin{ledgroupsized}[t]{13cm}\normalsize\beginnumbering\briefempfaengerindex{Beer-Hofmann, Richard@\textsc{Beer-Hofmann, Richard}!zzzSchnitzler, Arthur@\emph{von Arthur Schnitzler}!1908-11-284@{2{[}8{]}. 11. 1908}|(be} \toendnotes[C]{\smallbreak\pagebreak[2]} \Standort{YCGL, MSS 31.}
\physDesc{Brief, 1 Blatt, 3 Seiten, Umschlag
\newline{}Handschrift: Bleistift, deutsche Kurrent}\buchAbdrucke{\weitereDrucke{Arthur Schnitzler, Richard Beer-Hofmann: \emph{Briefwechsel 1891–1931}. Hg. Konstanze Fliedl. Wien, Zürich: \emph{Europaverlag} 1992, S. 192.} }\toendnotes[C]{\smallbreak}\pstart{}{\pb}\textcolor{gray}{\textbf{Dr. Arthur Schnitzler}}\pend{}\pstart{}\textcolor{gray}{\textbf{Wien XVIII. Spoettelgasse 7\oindex{Edmund-Weiss-Gasse@\textbf{Edmund-Weiß-Gasse}|pw}.}}\pend{}{\bigskip}\pstart{}{\pb}\textsc{Dr. Richard Beer-Hofma{\geminationn}}\pend{}\pstart{}Wien\oindex{XVIII., Waehring@\textbf{XVIII., Währing}|pw}. \pend{}{\bigskip}\pstart
           \noindent{}{\pb}II.\pend
           \pstart
           \textcolor{gray}{\textbf{Dr. Arthur Schnitzler}}\hfill \label{K_L01814_1v}\edtext{29. 11.}{\lemma{\textnormal{\emph{29. 11.}}}\Cendnote{\textnormal{Bei der Datierung ist Schnitzler\pwindex{Schnitzler, Arthur 15.05.1862 – 21.10.1931@\textsc{Schnitzler, Arthur} (15.05.1862 – 21.10.1931), \emph{Schriftsteller, Mediziner}|pwk} ein Fehler
                           unterlaufen.}}}\label{K_L01814_1h}\pend
           \pstart
           \textcolor{gray}{\textbf{Wien XVIII. Spoettelgasse 7\oindex{Edmund-Weiss-Gasse@\textbf{Edmund-Weiß-Gasse}|pw}.}}\pend
           \pstart
           Eben ſchrieb ich Ihnen \label{K_L01814_2v}\edtext{den beiliegd Brief}{\lemma{\textnormal{\emph{den beiliegd Brief}}}\Cendnote{\textnormal{Es dürfte sich um
                  den zweiten Brief vom [28. 11. 1908?] handeln. Da der Briefumschlag ohne
                  Briefmarke geblieben ist, dürfte er in den anderen eingelegt gewesen sein.}}}\label{K_L01814_2h}.
               Bleibt alſo nichts andres übrig als den morgigen Abend abzuwarten.\pend
           \pstart
           {\pb}Falls \textsc{Kerr}\pwindex{Kerr, Alfred 25.12.1867 – 12.10.1948@\textsc{Kerr, Alfred} (25.12.1867 – 12.10.1948), \emph{Schriftsteller, Kritiker}|pw} bei Ihnen ſchriftlich anfrägt, ſo ſchlagen Sie vielleicht auch für morgen Abend
                  \textsc{Meissl}\oindex{Meissl {\kaufmannsund} Schadn@\textbf{Meissl {\kaufmannsund} Schadn}|pw} vor. Den ganzen Tag über hab ich \label{K_L01814_3v}\edtext{morgen »geſchäftliche«
                  Beſprechungen}{\lemma{\textnormal{\emph{morgen … Beſprechungen}}}\Cendnote{\textnormal{Das erlaubt die sichere Datierung dieses Korrespondenzstücks. Vgl. A. S.: \emph{Tagebuch}, 29. 11. 1908}}}\label{K_L01814_3h}{ }{\pb}(\textsc{Dohnanyi\pwindex{Dohnányi, Ernst von 27.07.1877 – 09.02.1960@\textsc{Dohnányi, Ernst von} (27.07.1877 – 09.02.1960), \emph{Komponist, Pianist}|pw}, Straus\pwindex{Straus, Oscar 06.03.1870 – 11.01.1954@\textsc{Straus, Oscar} (06.03.1870 – 11.01.1954), \emph{Komponist}|pw}, Herzmansky\pwindex{Herzmansky, Bernhard 06.12.1852 – 18.05.1921@\textsc{Herzmansky, Bernhard} (06.12.1852 – 18.05.1921), \emph{Musikverleger}|pw}}.)\pend
           \pstart
           Ihr{\\[\baselineskip]}\spacefill\mbox{A.}\pend
           \leftskip=0em{}\endnumbering\briefempfaengerindex{Beer-Hofmann, Richard@\textsc{Beer-Hofmann, Richard}!zzzSchnitzler, Arthur@\emph{von Arthur Schnitzler}!1908-11-284@{2{[}8{]}. 11. 1908}|)be}\mylabel{h}\end{ledgroupsized}  \newcommand{\dateiname}{L01814}\newcommand{\titel}{Arthur Schnitzler an Richard Beer-Hofmann, 2[8]. 11. 1908}\newcommand{\editorInnen}{Martin Anton Müller und Gerd-Hermann Susen}\input{../tex-inputs/latex-pdf-abspann}
      