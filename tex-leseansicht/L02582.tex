%% latex-korrekturansicht-vorspann.tex
%% Vorspann für die Korrekturansicht.
%% Lädt die gemeinsame Datei latex-vorspann.tex mit gesetztem Schalter.

\newif\ifkorrekturansicht
\korrekturansichttrue

\input{../tex-inputs/latex-vorspann}


\section[Karin Michaëlis an Arthur Schnitzler, 21. 10. 1912]{L02582 Karin Michaëlis an Arthur Schnitzler, 21. 10. 1912}
\nopagebreak\mylabel{L02582v}
\rehead{ }\normalsize\beginnumbering\briefempfaengerindex{Schnitzler, Arthur@\textsc{Schnitzler, Arthur}!zzzMichaelis, Karin@\emph{von Karin Michaëlis}!1912-10-211@{21. 10. 1912}|(be}
\toendnotes[C]{\smallbreak\pagebreak[2]}\Standort{DLA, A:Schnitzler, HS.1985.1.4092.}
\physDesc{Brief, 1 Blatt, 2 Seiten, 529 Zeichen
\newline{}Handschrift: schwarze Tinte, lateinische Kurrent
\newline{}Schnitzler: mit Bleistift »\textsc{Michaeli\textcolor{gray}{s}}« }\toendnotes[C]{\smallbreak}
\pstart
           \centering{}{\pb}21 Oct. 1912.\pend
           
\pstart\center{}Sehr geehrter Hrr Dr.!\pend\vspace{0.5em}
\pstart
           Mein Freund Peter Nansen\pwindex{Nansen, Peter 20.01.1861 – 31.07.1918@\textsc{Nansen, Peter} (20.01.1861 – 31.07.1918), \emph{Schriftsteller/Schriftstellerin, Journalist/Journalistin, Verleger/Verlegerin}|pw} aus Kopenhagen\oindex{Kopenhagen@\textbf{Kopenhagen}, \emph{P.PPLC}|pw} ist hier\oindex{Wien@\textbf{Wien}, \emph{A.ADM2}|pwv} und hat den Wunsch (den ich also auch habe) Sie bald zu
               sehen.\pend
           
\pstart
           Wollen Sie mir die Freude machen, \label{K_L02582-1v}\edtext{Morgen}{\lemma{\textnormal{\emph{Morgen}}}\Cendnote{\textnormal{Ein Treffen mit Karin
                        Michaëlis\pwindex{Michaelis, Karin 20.03.1872 – 11.01.1950@\textsc{Michaëlis, Karin} (20.03.1872 – 11.01.1950), \emph{Schriftsteller/Schriftstellerin}|pwk}, Peter Nansen\pwindex{Nansen, Peter 20.01.1861 – 31.07.1918@\textsc{Nansen, Peter} (20.01.1861 – 31.07.1918), \emph{Schriftsteller/Schriftstellerin, Journalist/Journalistin, Verleger/Verlegerin}|pwk} und
                     anderen fand jedenfalls am 25. 10. 1912{ }abends statt.}}}\label{K_L02582-1}, \uline{Dienstag} um \uline{2 Uhr} mit uns im Hause der Freundin\pwindex{Schwarzwald, Eugenie 04.07.1872 – 07.08.1940@\textsc{Schwarzwald, Eugenie} (04.07.1872 – 07.08.1940), \emph{Schriftsteller/Schriftstellerin, Publizist/Publizistin, Lehrer/Lehrerin}|pwv}, bei der ich wohne (Frau Dr. Schwarzwald\pwindex{Schwarzwald, Eugenie 04.07.1872 – 07.08.1940@\textsc{Schwarzwald, Eugenie} (04.07.1872 – 07.08.1940), \emph{Schriftsteller/Schriftstellerin, Publizist/Publizistin, Lehrer/Lehrerin}|pw}{ }VIII Josefstädterstrasse 68\oindex{Josefstaedter Strasse@\textbf{Josefstädter Straße}, \emph{Straße (K.STR)}|pw}) zu
               frühstücken?\pend
           
\pstart
           {\pb}Ich hoffe von Herzen, dass Sie noch nicht vergeben sind
               und bitte um freundliche telefonische (N. 21237) \strikeout{be\textcolor{gray}{n}} Nachricht, ob wir die Freude haben werden, Sie zu begrüssen.\pend
           
\pstart
           Ihre verehrungsvoll ergebene{\\[\baselineskip]}\spacefill\mbox{Karin Michaëlis Stangeland}\pend
           \leftskip=0em{}\selectlanguage{ngerman}\endnumbering\briefempfaengerindex{Schnitzler, Arthur@\textsc{Schnitzler, Arthur}!zzzMichaelis, Karin@\emph{von Karin Michaëlis}!1912-10-211@{21. 10. 1912}|)be}\mylabel{L02582h}  \normalsize

\doendnotes{C}
\bigskip
\vfill

\clearpage

\footnotesize

\lohead{\textsc{register}}

% Definiere theindex-Environment komplett neu ohne reledmac
\makeatletter
\renewenvironment{theindex}{%
  \section*{\indexname}%
  \setlength{\parindent}{0pt}%
  \setlength{\parskip}{0pt plus 0.3pt}%
  \let\item\@idxitem
}{%
  \clearpage
}
\makeatother

\IfFileExists{\jobname-pw.ind}{\input{\jobname-pw.ind}}{}

\end{document}

      