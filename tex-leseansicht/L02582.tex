%% latex-leseansicht-vorspann.tex
%% Vorspann für die Leseansicht.
%% Lädt die gemeinsame Datei latex-vorspann.tex mit nicht gesetztem Schalter.

\newif\ifkorrekturansicht
\korrekturansichtfalse

\input{../tex-inputs/latex-vorspann}


               \section[Karin Michaëlis an Arthur Schnitzler, 21. 10. 1912]{ Karin Michaëlis an Arthur Schnitzler, 21. 10. 1912}\nopagebreak\mylabel{v}\rehead{ }\begin{ledgroupsized}[t]{13cm}\normalsize\beginnumbering\briefempfaengerindex{Schnitzler, Arthur@\textsc{Schnitzler, Arthur}!zzzMichaelis, Karin@\emph{von Karin Michaëlis}!1912-10-211@{21. 10. 1912}|(be} \toendnotes[C]{\smallbreak\pagebreak[2]} \Standort{DLA, A:Schnitzler, HS.1985.1.4092.}
\physDesc{Brief, 1 Blatt, 2 Seiten
\newline{}Handschrift: schwarze Tinte, lateinische Kurrent
\newline{}Schnitzler: mit Bleistift »\textsc{Michaeli\textcolor{gray}{s}}« }\toendnotes[C]{\smallbreak}\pstart
           \centering{}{\pb}21 Oct. 1912.\pend
           \pstart\center{}Sehr geehrter Hrr Dr.!\pend\pstart
           Mein Freund Peter Nansen\pwindex{Nansen, Peter 20.01.1861 – 31.07.1918@\textsc{Nansen, Peter} (20.01.1861 – 31.07.1918), \emph{Schriftsteller, Journalist, Verleger}|pw} aus Kopenhagen\oindex{Kopenhagen@\textbf{Kopenhagen}|pw} ist hier\oindex{Wien@\textbf{Wien}|pwv} und hat den Wunsch (den ich also auch habe) Sie bald zu sehen.\pend
           \pstart
           Wollen Sie mir die Freude machen, \label{K_L02582-1v}\edtext{Morgen}{\lemma{\textnormal{\emph{Morgen}}}\Cendnote{\textnormal{Ein
                     Treffen mit Karin Michaëlis\pwindex{Michaelis, Karin 20.03.1872 – 11.01.1950@\textsc{Michaëlis, Karin} (20.03.1872 – 11.01.1950), \emph{Schriftstellerin}|pwk}, Peter Nansen\pwindex{Nansen, Peter 20.01.1861 – 31.07.1918@\textsc{Nansen, Peter} (20.01.1861 – 31.07.1918), \emph{Schriftsteller, Journalist, Verleger}|pwk} und anderen fand jedenfalls am
                        25. 10. 1912{ }abends statt.}}}\label{K_L02582-1h}, \uline{Dienstag} um \uline{2 Uhr} mit uns im Hause der Freundin\pwindex{Schwarzwald, Eugenie 04.07.1872 – 07.08.1940@\textsc{Schwarzwald, Eugenie} (04.07.1872 – 07.08.1940), \emph{Pädagogin}|pwv}, bei der ich wohne (Frau Dr. Schwarzwald\pwindex{Schwarzwald, Eugenie 04.07.1872 – 07.08.1940@\textsc{Schwarzwald, Eugenie} (04.07.1872 – 07.08.1940), \emph{Pädagogin}|pw}{ }VIII Josefstädterstrasse 68\oindex{Josefstaedter Strasse@\textbf{Josefstädter Straße}|pw}) zu
               frühstücken?\pend
           \pstart
           {\pb}Ich hoffe von Herzen, dass Sie noch nicht vergeben sind
               und bitte um freundliche telefonische (N. 21237) \strikeout{be\textcolor{gray}{n}} Nachricht, ob wir die Freude haben werden, Sie zu
               begrüssen.\pend
           \pstart
           Ihre verehrungsvoll ergebene{\\[\baselineskip]}\spacefill\mbox{Karin Michaëlis
               Stangeland}\pend
           \leftskip=0em{}\endnumbering\briefempfaengerindex{Schnitzler, Arthur@\textsc{Schnitzler, Arthur}!zzzMichaelis, Karin@\emph{von Karin Michaëlis}!1912-10-211@{21. 10. 1912}|)be}\mylabel{h}\end{ledgroupsized}  \newcommand{\dateiname}{L02582}\newcommand{\titel}{Karin Michaëlis an Arthur Schnitzler, 21. 10. 1912}\newcommand{\editorInnen}{Martin Anton Müller und Laura Untner}%% latex-leseansicht-abspann.tex
%% Abspann für die Leseansicht.
%% Der Schalter \ifkorrekturansicht ist bereits durch den Vorspann gesetzt.

%% latex-abspann.tex
%% Gemeinsamer Abspann für Korrekturansicht und Leseansicht.
%% Setzt den Schalter \ifkorrekturansicht voraus (gesetzt in den
%% einbindenden Dateien latex-korrekturansicht-abspann.tex bzw.
%% latex-leseansicht-abspann.tex).
%% ---------------------------------------------------------------

\normalsize

% Das esempio-Environment wird nur in der Leseansicht benötigt
\ifkorrekturansicht\else
\newenvironment{esempio}[3]%
{
    \vspace{1.5ex}
    \rlap{\underline{#1}}
    \par
    \setlength{\parindent}{0cm}
    \nopagebreak
    \leftskip=#2cm
    \rightskip=#3cm
}
{
    \par
}
\fi

\doendnotes{C}
\bigskip
\vfill

\clearpage

\footnotesize

\ifkorrekturansicht
  \lohead{\textsc{register}}
\fi

% theindex-Environment neu definieren ohne reledmac
\makeatletter
\renewenvironment{theindex}{%
  \ifkorrekturansicht
    \section*{\indexname}%
  \else
    \subsubsection*{Index der erwähnten Entitäten}%
  \fi
  \setlength{\parindent}{0pt}%
  \setlength{\parskip}{0pt plus 0.3pt}%
  \let\item\@idxitem
}{%
  \ifkorrekturansicht\clearpage\fi
}
\makeatother

\IfFileExists{\jobname-pw.ind}{\input{\jobname-pw.ind}}{}

% Quellenangabe nur in der Leseansicht
\ifkorrekturansicht\else
% Fallback-Definitionen, falls die .tex-Datei \titel etc. nicht gesetzt hat
\providecommand{\titel}{}
\providecommand{\editorInnen}{}
\providecommand{\dateiname}{\jobname}

\vspace{3cm}

\vfill

\footnotesize
\textsc{Quelle}: \titel. Herausgegeben von {\editorInnen}. In: \emph{Arthur Schnitzler: Briefwechsel mit Autorinnen und Autoren}.
 Digitale Edition, https://schnitzler-briefe.acdh.oeaw.ac.at/{\dateiname}.html (Stand \today)
\fi

\end{document}


      