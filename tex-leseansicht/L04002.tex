%% latex-leseansicht-vorspann.tex
%% Vorspann für die Leseansicht.
%% Lädt die gemeinsame Datei latex-vorspann.tex mit nicht gesetztem Schalter.

\newif\ifkorrekturansicht
\korrekturansichtfalse

\input{../tex-inputs/latex-vorspann}


\section[Berta Zuckerkandl an Arthur Schnitzler, {{[}}zwischen 20. und 30. 5. 1913?{{]}}]{L04002 Berta Zuckerkandl an Arthur Schnitzler, {[}zwischen 20. und 30. 5. 1913?{]}}
\nopagebreak\mylabel{L04002v}
\rehead{ }\normalsize\beginnumbering\briefempfaengerindex{Schnitzler, Arthur@\textsc{Schnitzler, Arthur}!zzzZuckerkandl, Berta@\emph{von Berta Zuckerkandl}!1913-05-301@{{[}zwischen 20. und 30. 5. 1913?{]}}|(be}
\toendnotes[C]{\smallbreak\pagebreak[2]}
\correspDesc{Versand  durch Berta Zuckerkandl im Zeitraum [zwischen 20. und
                  30. 5. 1913?] in Wien
\newline{}Erhalt  durch Arthur Schnitzler in Wien}\toendnotes[C]{\smallbreak}
\Standort{CUL, Schnitzler, B 200.}
\physDesc{Brief, 1 Blatt, 3 Seiten, 372 Zeichen
\newline{}Handschrift: schwarze Tinte, lateinische Kurrent
\newline{}Schnitzler: 1) mit Bleistift beschriftet: »Zuckerkandl«  2) mit rotem Buntstift eine Unterstreichung}\toendnotes[C]{\smallbreak}
\pstart
           \raggedleft{}{\pb}\textcolor{gray}{\textbf{XIX. NUSSWALDGASSE
                  22}}\oindex{Wien@\textbf{Wien}!XIX., Döbling@\textbf{XIX., Döbling}!Nusswaldgasse 22@\textbf{Nusswaldgasse 22}, \emph{Wohngebäude}|pw}\pend
           
\pstart{}Lieber verehrter Herr Doktor!\pend\vspace{0.5em}
\pstart
           Heute habe ich die \label{K_L04002-1v}\edtext{Beate\pwindex{Schnitzler, Arthur 15.\,5.\,1862 Wien – 21.\,10.\,1931 ebd.@\textsc{Schnitzler, Arthur} (15.\,5.\,1862 Wien – 21.\,10.\,1931 ebd.), \emph{Schriftsteller, Mediziner}!Frau Beate und ihr Sohn. Novelle@\strich\emph{Frau Beate und ihr Sohn. Novelle}|pw} gelesen}{\lemma{\textnormal{\emph{Beate gelesen}}}\Cendnote{\textnormal{Am 20. 5. 1913 vermerkt Schnitzler
                  im \emph{Tagebuch}\pwindex{Schnitzler, Arthur 15.\,5.\,1862 Wien – 21.\,10.\,1931 ebd.@\textsc{Schnitzler, Arthur} (15.\,5.\,1862 Wien – 21.\,10.\,1931 ebd.), \emph{Schriftsteller, Mediziner}!Tagebuch@\strich\emph{Tagebuch}|pwk}, dass die Buchausgabe seiner
                  Novelle \emph{Frau Beate und ihr Sohn}\pwindex{Schnitzler, Arthur 15.\,5.\,1862 Wien – 21.\,10.\,1931 ebd.@\textsc{Schnitzler, Arthur} (15.\,5.\,1862 Wien – 21.\,10.\,1931 ebd.), \emph{Schriftsteller, Mediziner}!Frau Beate und ihr Sohn. Novelle@\strich\emph{Frau Beate und ihr Sohn. Novelle}|pwk} vor einem Tag
                  erschienen sei und er damit beginnt, das Buch Freunden zu schenken. Zuckerkandl\pwindex{Zuckerkandl, Berta 13.\,4.\,1864 Wien – 16.\,10.\,1945 Paris@\textsc{Zuckerkandl, Berta} (13.\,4.\,1864 Wien – 16.\,10.\,1945 Paris), \emph{Schriftstellerin, Journalistin, Übersetzerin}|pwk}, die im vorliegenden undatierten
                  Brief angibt, das Buch rasch beendet zu haben, möchte ihren Leseeindruck gleich
                  weitergeben. Ab dem 30. 5. 1913 dokumentiert das \emph{Tagebuch}\pwindex{Schnitzler, Arthur 15.\,5.\,1862 Wien – 21.\,10.\,1931 ebd.@\textsc{Schnitzler, Arthur} (15.\,5.\,1862 Wien – 21.\,10.\,1931 ebd.), \emph{Schriftsteller, Mediziner}!Tagebuch@\strich\emph{Tagebuch}|pwk} eine Woche mit Begegnungen an jedem zweiten Tag, bei denen Zuckerkandl\pwindex{Zuckerkandl, Berta 13.\,4.\,1864 Wien – 16.\,10.\,1945 Paris@\textsc{Zuckerkandl, Berta} (13.\,4.\,1864 Wien – 16.\,10.\,1945 Paris), \emph{Schriftstellerin, Journalistin, Übersetzerin}|pwk} ihre Rückmeldung mündlich hätte
                  erstatten können. Der Brief, der die Freude über die Neuheit zum Ausdruck bringt,
                  ist wahrscheinlich noch im letzten Drittel des Monats Mai 1913
                  entstanden.}}}\label{K_L04002-1}. In einem Zug. – Wie ist es nur möglich so tief in die Menschen
               hieinzu blicken. Wie können {\pb}Sie erfühlen
               – was eine Mutter selbst, in ihrer Liebe zum Sohn, nur ahnt – nicht weiss. Und wie
               rein wird in Ihrer Hand – Alles Unreine. Wie adeln Sie – selbst das {\pb}Gemeine.\pend
           
\pstart
           Ich bin wirklich tief erschüttert worden.\pend
           
\pstart
           Dank. {\\[\baselineskip]}\spacefill\mbox{B. Z.}\pend
           \leftskip=0em{}\selectlanguage{ngerman}\endnumbering\briefempfaengerindex{Schnitzler, Arthur@\textsc{Schnitzler, Arthur}!zzzZuckerkandl, Berta@\emph{von Berta Zuckerkandl}!1913-05-201@{{[}zwischen 20. und 30. 5. 1913?{]}}|)be}\mylabel{L04002h}
\begin{anhang}
\end{anhang}\newcommand{\dateiname}{L04002}\newcommand{\titel}{Berta Zuckerkandl an Arthur Schnitzler, [zwischen 20. und 30. 5. 1913?]}\newcommand{\editorInnen}{Herausgegeben von Jahnke, SelmaMüller, Martin Anton}%% latex-leseansicht-abspann.tex
%% Abspann für die Leseansicht.
%% Der Schalter \ifkorrekturansicht ist bereits durch den Vorspann gesetzt.

%% latex-abspann.tex
%% Gemeinsamer Abspann für Korrekturansicht und Leseansicht.
%% Setzt den Schalter \ifkorrekturansicht voraus (gesetzt in den
%% einbindenden Dateien latex-korrekturansicht-abspann.tex bzw.
%% latex-leseansicht-abspann.tex).
%% ---------------------------------------------------------------

\normalsize

% Das esempio-Environment wird nur in der Leseansicht benötigt
\ifkorrekturansicht\else
\newenvironment{esempio}[3]%
{
    \vspace{1.5ex}
    \rlap{\underline{#1}}
    \par
    \setlength{\parindent}{0cm}
    \nopagebreak
    \leftskip=#2cm
    \rightskip=#3cm
}
{
    \par
}
\fi

\doendnotes{C}
\bigskip
\vfill

\clearpage

\footnotesize

\ifkorrekturansicht
  \lohead{\textsc{register}}
\fi

% theindex-Environment neu definieren ohne reledmac
\makeatletter
\renewenvironment{theindex}{%
  \ifkorrekturansicht
    \section*{\indexname}%
  \else
    \subsubsection*{Index der erwähnten Entitäten}%
  \fi
  \setlength{\parindent}{0pt}%
  \setlength{\parskip}{0pt plus 0.3pt}%
  \let\item\@idxitem
}{%
  \ifkorrekturansicht\clearpage\fi
}
\makeatother

\IfFileExists{\jobname-pw.ind}{\input{\jobname-pw.ind}}{}

% Quellenangabe nur in der Leseansicht
\ifkorrekturansicht\else
% Fallback-Definitionen, falls die .tex-Datei \titel etc. nicht gesetzt hat
\providecommand{\titel}{}
\providecommand{\editorInnen}{}
\providecommand{\dateiname}{\jobname}

\vspace{3cm}

\vfill

\footnotesize
\textsc{Quelle}: \titel. Herausgegeben von {\editorInnen}. In: \emph{Arthur Schnitzler: Briefwechsel mit Autorinnen und Autoren}.
 Digitale Edition, https://schnitzler-briefe.acdh.oeaw.ac.at/{\dateiname}.html (Stand \today)
\fi

\end{document}


