%% latex-korrekturansicht-vorspann.tex
%% Vorspann für die Korrekturansicht.
%% Lädt die gemeinsame Datei latex-vorspann.tex mit gesetztem Schalter.

\newif\ifkorrekturansicht
\korrekturansichttrue

\input{../tex-inputs/latex-vorspann}


\section[Richard Beer-Hofmann an Arthur Schnitzler, 14. 7. 1899]{L00942 Richard Beer-Hofmann an Arthur Schnitzler, 14. 7. 1899}
\nopagebreak\mylabel{L00942v}
\rehead{ }\normalsize\beginnumbering\briefempfaengerindex{Schnitzler, Arthur@\textsc{Schnitzler, Arthur}!zzzBeer-Hofmann, Richard@\emph{von Richard Beer-Hofmann}!1899-07-142@{14. 7. 1899}|(be}
\toendnotes[C]{\smallbreak\pagebreak[2]}\Standort{CUL, Schnitzler, B 8.}
\physDesc{Brief, 1 Blatt, 4 Seiten, 1173 Zeichen
\newline{}Handschrift: Bleistift, lateinische Kurrent
\newline{}Ordnung: mit Bleistift von unbekannter Hand nummeriert:
                                    »132« }
\buchAbdrucke{\weitereDrucke{Arthur Schnitzler, Richard Beer-Hofmann: \emph{Briefwechsel 1891–1931}. Wien, Zürich: \emph{Europaverlag} 1992, S. 132.} }\toendnotes[C]{\smallbreak}
\pstart
           \centering{}{\pb}Seeboden\oindex{Seeboden@\textbf{Seeboden}, \emph{A.ADM3}|pw}{ }14/VII 99\pend
           \vspace{0.5em}
\pstart
           Lieber Arthur! Das »Vielleicht« konnte sich doch selbstverständlich
               nur auf die gemeinschaftliche Tour beziehen. Ich wünsche – aber das ist ja
               selbstverständlich, – ich hoffe mit einer Wahrscheinlichkeit von 75{\%} daß wir in den letzten Julitagen eine gemeinschaftliche
               Tour machen können. Vielleicht daß wir von hier aus {\pb}am 25 od.
                  26 über die Tauern\oindex{Hohe Tauern@\textbf{Hohe Tauern}, \emph{Gebirge (N.GBR)}|pw} nach Salzburg\oindex{Salzburg@\textbf{Salzburg}, \emph{A.ADM2}|pw}{ }\strikeout{m} gehen – dort 2 Tage bleiben (1 Tag davon muß ich
               nach Ischl\oindex{Bad Ischl@\textbf{Bad Ischl}, \emph{P.PPL}|pw}{ }\introOben{}od.\introOben{}{ }Aussee\oindex{Bad Aussee@\textbf{Bad Aussee}, \emph{P.PPLA3}|pw}) dann nach Bayreuth\oindex{Bayreuth@\textbf{Bayreuth}, \emph{P.PPLA2}|pw} am 31 – und von dort München\oindex{Muenchen@\textbf{München}, \emph{P.PPLA}|pw}{ }Innsbruck\oindex{Innsbruck@\textbf{Innsbruck}, \emph{A.ADM2}|pw}{ }Franzensfeste\oindex{Franzensfeste@\textbf{Franzensfeste}, \emph{A.ADM3}|pw}\substVorne{}\textsuperscript{–}\substDazwischen{}(\substHinten{}eventuell begleite ich Sie nach Bozen\oindex{Bozen@\textbf{Bozen}, \emph{P.PPLA2}|pw}\introOben{})\introOben{} zurück. Vorher möchte ich Sie gewiß gerne hier oder in
                  Millstatt\oindex{Millstatt@\textbf{Millstatt}, \emph{A.ADM3}|pw} haben.\pend
           
\pstart
           Meine ganze Reserve im Ausdruck datirt nur aus der Nervosi{\pb}tät Pläne zu machen, und aus der
                  zweiten\strikeout{,} Nervosität ob ich bis zu Ihrer Ankunft
                  fertig\pwindex{Tod Georgs@\emph{Der Tod Georgs}|pwv} sein werde. Ihre
               Adresse in Velden\oindex{Velden am Woerthersee@\textbf{Velden am Wörthersee}, \emph{P.PPL}|pw} haben Sie mir noch nicht
               angegeben. Von Herzen\pend
           
\pstart
           Ihr{\\[\baselineskip]}\spacefill\mbox{Richard}\pend
           \leftskip=0em{}
\pstart
           \noindent{}Bitte sagen Sie Schwarzkopf\pwindex{Schwarzkopf, Gustav 07.11.1853 – 13.11.1939@\textsc{Schwarzkopf, Gustav} (07.11.1853 – 13.11.1939), \emph{Schriftsteller/Schriftstellerin}|pw} daß ich zu
                     versti{\geminationm}t war um ihm zu schreiben – ich weiß schon,
                  er wird sagen: »u wenn er nicht {\pb}versti{\geminationm}t ist schreibt er?« Aber ich lasse \substVorne{}\textsuperscript{I}\substDazwischen{}i\substHinten{}hn herzlich grüßen und ich würde mich mehr – als er glaubt – freuen wenn
                  er hieher käme.\pend
           
\pstart
           – Ich \uline{habe} geschrieben »versti{\geminationm}t war«. Diese Vergangenheit ist unberechtigt.\pend
           \selectlanguage{ngerman}\endnumbering\briefempfaengerindex{Schnitzler, Arthur@\textsc{Schnitzler, Arthur}!zzzBeer-Hofmann, Richard@\emph{von Richard Beer-Hofmann}!1899-07-142@{14. 7. 1899}|)be}\mylabel{L00942h}  \normalsize

\doendnotes{C}
\bigskip
\vfill

\clearpage

\footnotesize

\lohead{\textsc{register}}

% Definiere theindex-Environment komplett neu ohne reledmac
\makeatletter
\renewenvironment{theindex}{%
  \section*{\indexname}%
  \setlength{\parindent}{0pt}%
  \setlength{\parskip}{0pt plus 0.3pt}%
  \let\item\@idxitem
}{%
  \clearpage
}
\makeatother

\IfFileExists{\jobname-pw.ind}{\input{\jobname-pw.ind}}{}

\end{document}

      