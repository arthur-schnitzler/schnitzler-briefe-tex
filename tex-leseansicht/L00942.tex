%% latex-leseansicht-vorspann.tex
%% Vorspann für die Leseansicht.
%% Lädt die gemeinsame Datei latex-vorspann.tex mit nicht gesetztem Schalter.

\newif\ifkorrekturansicht
\korrekturansichtfalse

\input{../tex-inputs/latex-vorspann}

\begin{center}
            \textcolor{red}{ENTWURF. ENTZIFFERUNG NOCH NICHT KORREKTURGELESEN}
                      \end{center}
            
               \section[Richard Beer-Hofmann an Arthur Schnitzler, 14. 7. 1899]{ Richard Beer-Hofmann an Arthur Schnitzler, 14. 7. 1899}\nopagebreak\mylabel{v}\rehead{ }\begin{ledgroupsized}[t]{13cm}\normalsize\beginnumbering\briefempfaengerindex{Schnitzler, Arthur@\textsc{Schnitzler, Arthur}!zzzBeer-Hofmann, Richard@\emph{von Richard Beer-Hofmann}!1899-07-142@{14. 7. 1899}|(be} \toendnotes[C]{\smallbreak\pagebreak[2]} \Standort{CUL, Schnitzler, B 8.}
\physDesc{Brief, 1 Blatt, 4 Seiten
\newline{}Handschrift: Bleistift, lateinische Kurrent\newline{}Ordnung: mit Bleistift von unbekannter Hand nummeriert: »132« }\buchAbdrucke{\weitereDrucke{Arthur Schnitzler, Richard Beer-Hofmann: \emph{Briefwechsel 1891–1931}. Hg. Konstanze Fliedl. Wien, Zürich: \emph{Europaverlag} 1992, S. 132.} }\toendnotes[C]{\smallbreak}\pstart
           \centering{}{\pb}Seeboden\oindex{Seeboden@\textbf{Seeboden}|pw}{ }14/VII 99\pend
           \pstart
           Lieber Arthur! Das »Vielleicht« konnte sich doch selbstverständlich
               nur auf die gemeinschaftliche Tour beziehen. Ich wünsche – aber das ist ja
               selbstverständlich, – ich hoffe mit einer Wahrscheinlichkeit von 75{\%} daß wir in den letzten Julitagen eine gemeinschaftliche
               Tour machen können. Vielleicht daß wir von hier aus {\pb}am 25 od.
                  26 über die Tauern\oindex{Hohe Tauern@\textbf{Hohe Tauern}|pw} nach Salzburg\oindex{Salzburg@\textbf{Salzburg}|pw}{ }\strikeout{m} gehen – dort 2 Tage bleiben (1 Tag davon muß ich nach
                  Ischl\oindex{Bad Ischl@\textbf{Bad Ischl}|pw}{ }\introOben{}od.\introOben{}{ }Aussee\oindex{Bad Aussee@\textbf{Bad Aussee}|pw}) dann nach Bayreuth\oindex{Bayreuth@\textbf{Bayreuth}|pw} am 31 – und von dort München\oindex{Muenchen@\textbf{München}|pw}{ }Innsbruck\oindex{Innsbruck@\textbf{Innsbruck}|pw}{ }Franzensfeste\oindex{Franzensfeste@\textbf{Franzensfeste}|pw}\substVorne{}\textsuperscript{–}\substDazwischen{}(\substHinten{}eventuell begleite ich Sie nach Bozen\oindex{Bozen@\textbf{Bozen}|pw}\introOben{})\introOben{} zurück. Vorher möchte ich Sie gewiß gerne hier oder in Millstatt\oindex{Millstatt@\textbf{Millstatt}|pw} haben.\pend
           \pstart
           Meine ganze Reserve im Ausdruck datirt nur aus der Nervosi{\pb}tät Pläne zu machen, und aus der
                  zweiten\strikeout{,} Nervosität ob ich bis zu Ihrer Ankunft
                  fertig\pwindex{Beer-Hofmann, Richard 11.07.1866 – 26.09.1945@\textsc{Beer-Hofmann, Richard} (11.07.1866 – 26.09.1945), \emph{Schriftsteller}!Tod Georgs1900@\strich\emph{Der Tod Georgs} {[}1900{]}|pwv} sein werde. Ihre
               Adresse in Velden\oindex{Velden@\textbf{Velden}|pw} haben Sie mir noch nicht
               angegeben. Von Herzen\pend
           \pstart
           Ihr{\\[\baselineskip]}\spacefill\mbox{Richard}\pend
           \leftskip=0em{}\pstart
           \noindent{}Bitte sagen Sie Schwarzkopf\pwindex{Schwarzkopf, Gustav 07.11.1853 – 13.11.1939@\textsc{Schwarzkopf, Gustav} (07.11.1853 – 13.11.1939), \emph{Schriftsteller}|pw} daß ich zu
                     versti{\geminationm}t war um ihm zu schreiben – ich weiß schon,
                  er wird sagen: »u wenn er nicht {\pb}versti{\geminationm}t ist schreibt er?« Aber ich lasse \substVorne{}\textsuperscript{I}\substDazwischen{}i\substHinten{}hn herzlich grüßen und ich würde mich mehr – als er glaubt – freuen wenn
                  er hieher käme.\pend
           \pstart
           – Ich \uline{habe} geschrieben »versti{\geminationm}t war«. Diese Vergangenheit ist unberechtigt.\pend
           \endnumbering\briefempfaengerindex{Schnitzler, Arthur@\textsc{Schnitzler, Arthur}!zzzBeer-Hofmann, Richard@\emph{von Richard Beer-Hofmann}!1899-07-142@{14. 7. 1899}|)be}\mylabel{h}\end{ledgroupsized}  \newcommand{\dateiname}{L00942}\newcommand{\titel}{Richard Beer-Hofmann an Arthur Schnitzler, 14. 7. 1899}\newcommand{\editorInnen}{Martin Anton Müller und Gerd-Hermann Susen}%% latex-leseansicht-abspann.tex
%% Abspann für die Leseansicht.
%% Der Schalter \ifkorrekturansicht ist bereits durch den Vorspann gesetzt.

%% latex-abspann.tex
%% Gemeinsamer Abspann für Korrekturansicht und Leseansicht.
%% Setzt den Schalter \ifkorrekturansicht voraus (gesetzt in den
%% einbindenden Dateien latex-korrekturansicht-abspann.tex bzw.
%% latex-leseansicht-abspann.tex).
%% ---------------------------------------------------------------

\normalsize

% Das esempio-Environment wird nur in der Leseansicht benötigt
\ifkorrekturansicht\else
\newenvironment{esempio}[3]%
{
    \vspace{1.5ex}
    \rlap{\underline{#1}}
    \par
    \setlength{\parindent}{0cm}
    \nopagebreak
    \leftskip=#2cm
    \rightskip=#3cm
}
{
    \par
}
\fi

\doendnotes{C}
\bigskip
\vfill

\clearpage

\footnotesize

\ifkorrekturansicht
  \lohead{\textsc{register}}
\fi

% theindex-Environment neu definieren ohne reledmac
\makeatletter
\renewenvironment{theindex}{%
  \ifkorrekturansicht
    \section*{\indexname}%
  \else
    \subsubsection*{Index der erwähnten Entitäten}%
  \fi
  \setlength{\parindent}{0pt}%
  \setlength{\parskip}{0pt plus 0.3pt}%
  \let\item\@idxitem
}{%
  \ifkorrekturansicht\clearpage\fi
}
\makeatother

\IfFileExists{\jobname-pw.ind}{\input{\jobname-pw.ind}}{}

% Quellenangabe nur in der Leseansicht
\ifkorrekturansicht\else
% Fallback-Definitionen, falls die .tex-Datei \titel etc. nicht gesetzt hat
\providecommand{\titel}{}
\providecommand{\editorInnen}{}
\providecommand{\dateiname}{\jobname}

\vspace{3cm}

\vfill

\footnotesize
\textsc{Quelle}: \titel. Herausgegeben von {\editorInnen}. In: \emph{Arthur Schnitzler: Briefwechsel mit Autorinnen und Autoren}.
 Digitale Edition, https://schnitzler-briefe.acdh.oeaw.ac.at/{\dateiname}.html (Stand \today)
\fi

\end{document}


      