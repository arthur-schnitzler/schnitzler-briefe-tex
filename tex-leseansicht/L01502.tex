%% latex-leseansicht-vorspann.tex
%% Vorspann für die Leseansicht.
%% Lädt die gemeinsame Datei latex-vorspann.tex mit nicht gesetztem Schalter.

\newif\ifkorrekturansicht
\korrekturansichtfalse

\input{../tex-inputs/latex-vorspann}


         
         \renewcommand{\erwaehntePersonen}{Personen: Richard Beer-Hofmann, Paula Beer-Hofmann, Gertrude von Hofmannsthal, Hugo von Hofmannsthal}
         \renewcommand{\erwaehnteOrte}{Orte: Genua, Liesingerstraße, Mittelmeer, Ottakringer Bräu, Rodaun, Wien}
         \renewcommand{\erwaehnteWerke}{}
               \section[Arthur Schnitzler an Richard Beer-Hofmann, 26. 2. 1905]{ Arthur Schnitzler an Richard Beer-Hofmann, 26. 2. 1905}\nopagebreak\mylabel{v}\rehead{ }\begin{ledgroupsized}[t]{13cm}\normalsize\beginnumbering \toendnotes[C]{\smallbreak\pagebreak[2]} \Standort{YCGL, MSS 31.}
\physDesc{Kartenbrief, 444 Zeichen
\newline{}Handschrift: schwarze Tinte, deutsche Kurrent
\newline{}Versand: 1) Stempel: »\nobreak{}Wien 68, 26. 2. 05, 5–6N\nobreak{}«.   2) Stempel: »\nobreak{}\oindex{Rodaun@\textbf{Rodaun}|pwk}Rodaun, 27. 2. \textcolor{gray}{05}, 7–9V\nobreak{}«. }\toendnotes[C]{\smallbreak}\pstart{}{\pb}Herrn \textsc{Dr. Richard
                     Beer-Hofmann}\pend{}\pstart{}\textsc{Rodaun\oindex{Rodaun@\textbf{Rodaun}|pw}}\pend{}\pstart{}\textsc{Liesinger Straße 2}\oindex{Liesingerstrasse@\textbf{Liesingerstraße}|pw}. \pend{}{\bigskip}\pstart
           \raggedleft{}{\pb}So{\geminationn}tag 26. 2. 905.\pend
           \pstart
           lieber Richard, ich reiſe am \label{K_L01502-1v}\edtext{Freitag 3.{ }}{\lemma{\textnormal{\emph{Freitag 3. }}}\Cendnote{\textnormal{siehe A. S.: \emph{Tagebuch}, 3. 3. 1905}}}\label{K_L01502-1h} Genua\oindex{Genua@\textbf{Genua}|pw} zu Mittelmeer\oindex{Mittelmeer@\textbf{Mittelmeer}|pw}zwecken; und, unter günſtigen Umſtänden bin ich erſt \label{K_L01502-2v}\edtext{gegen den 20.}{\lemma{\textnormal{\emph{gegen den 20.}}}\Cendnote{\textnormal{vgl. A. S.: \emph{Tagebuch}, 18. 3. 1905}}}\label{K_L01502-2h} wieder hier\substVorne{}\textsuperscript{?}\substDazwischen{}.\substHinten{}\pend
           \pstart
           Könnte man ſich nicht vorher doch einmal ſehen? Den Hugo’s\pwindex{Hofmannsthal, Gertrude von 16.03.1880 – 09.11.1959@\textsc{Hofmannsthal, Gertrude von} (16.03.1880 – 09.11.1959)|pw}\pwindex{Hofmannsthal, Hugo von 1874-02-01 – 1929-07-15@\textsc{Hofmannsthal, Hugo von} (1874-02-01 – 1929-07-15), \emph{Schriftsteller}|pw} hab ich für \label{K_L01502-3v}\edtext{Mittwoch}{\lemma{\textnormal{\emph{Mittwoch}}}\Cendnote{\textnormal{Das Treffen fand, ohne das Ehepaar Hofmannsthal\pwindex{Hofmannsthal, Gertrude von 16.03.1880 – 09.11.1959@\textsc{Hofmannsthal, Gertrude von} (16.03.1880 – 09.11.1959)|pwk}\pwindex{Hofmannsthal, Hugo von 1874-02-01 – 1929-07-15@\textsc{Hofmannsthal, Hugo von} (1874-02-01 – 1929-07-15), \emph{Schriftsteller}|pwk}, am Donnerstag statt;
                     siehe A. S.: \emph{Tagebuch}, 2. 3. 1905}}}\label{K_L01502-3h}{ }Abend, Hietzing\oindex{Ottakringer Braeu@\textbf{Ottakringer Bräu}|pw} geſchrieben; kommen
               Sie etwa auch mit Paula\pwindex{Beer-Hofmann, Paula 25.02.1879 – 30.10.1939@\textsc{Beer-Hofmann, Paula} (25.02.1879 – 30.10.1939)|pw}? Oder wollen Sie nicht
               endlich einmal bei uns eſſen?\pend
           \pstart
           Laſſen Sie jedenfalls ein Wort hören.\pend
           \pstart
           Herzlichſt Ihr{\\[\baselineskip]}\spacefill\mbox{A.}\pend
           \leftskip=0em{}
         
         \endnumbering\mylabel{h}\end{ledgroupsized}  \newcommand{\dateiname}{L01502}\newcommand{\titel}{Arthur Schnitzler an Richard Beer-Hofmann, 26. 2. 1905}\newcommand{\editorInnen}{Martin Anton Müller und Gerd-Hermann Susen}%% latex-leseansicht-abspann.tex
%% Abspann für die Leseansicht.
%% Der Schalter \ifkorrekturansicht ist bereits durch den Vorspann gesetzt.

%% latex-abspann.tex
%% Gemeinsamer Abspann für Korrekturansicht und Leseansicht.
%% Setzt den Schalter \ifkorrekturansicht voraus (gesetzt in den
%% einbindenden Dateien latex-korrekturansicht-abspann.tex bzw.
%% latex-leseansicht-abspann.tex).
%% ---------------------------------------------------------------

\normalsize

% Das esempio-Environment wird nur in der Leseansicht benötigt
\ifkorrekturansicht\else
\newenvironment{esempio}[3]%
{
    \vspace{1.5ex}
    \rlap{\underline{#1}}
    \par
    \setlength{\parindent}{0cm}
    \nopagebreak
    \leftskip=#2cm
    \rightskip=#3cm
}
{
    \par
}
\fi

\doendnotes{C}
\bigskip
\vfill

\clearpage

\footnotesize

\ifkorrekturansicht
  \lohead{\textsc{register}}
\fi

% theindex-Environment neu definieren ohne reledmac
\makeatletter
\renewenvironment{theindex}{%
  \ifkorrekturansicht
    \section*{\indexname}%
  \else
    \subsubsection*{Index der erwähnten Entitäten}%
  \fi
  \setlength{\parindent}{0pt}%
  \setlength{\parskip}{0pt plus 0.3pt}%
  \let\item\@idxitem
}{%
  \ifkorrekturansicht\clearpage\fi
}
\makeatother

\IfFileExists{\jobname-pw.ind}{\input{\jobname-pw.ind}}{}

% Quellenangabe nur in der Leseansicht
\ifkorrekturansicht\else
% Fallback-Definitionen, falls die .tex-Datei \titel etc. nicht gesetzt hat
\providecommand{\titel}{}
\providecommand{\editorInnen}{}
\providecommand{\dateiname}{\jobname}

\vspace{3cm}

\vfill

\footnotesize
\textsc{Quelle}: \titel. Herausgegeben von {\editorInnen}. In: \emph{Arthur Schnitzler: Briefwechsel mit Autorinnen und Autoren}.
 Digitale Edition, https://schnitzler-briefe.acdh.oeaw.ac.at/{\dateiname}.html (Stand \today)
\fi

\end{document}


      