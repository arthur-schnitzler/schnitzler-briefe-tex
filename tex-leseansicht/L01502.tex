\input{../tex-inputs/latex-pdf-vorspann}
\begin{center}
            \textcolor{red}{ENTWURF. ENTZIFFERUNG NOCH NICHT KORREKTURGELESEN}
                      \end{center}
            
               \section[Arthur Schnitzler an Richard Beer-Hofmann, 26. 2. 1905]{ Arthur Schnitzler an Richard Beer-Hofmann, 26. 2. 1905}\nopagebreak\mylabel{v}\rehead{ }\begin{ledgroupsized}[t]{13cm}\normalsize\beginnumbering\briefempfaengerindex{Beer-Hofmann, Richard@\textsc{Beer-Hofmann, Richard}!zzzSchnitzler, Arthur@\emph{von Arthur Schnitzler}!1905-02-261@{26. 2. 1905}|(be} \toendnotes[C]{\smallbreak\pagebreak[2]} \Standort{YCGL, MSS 31.}
\physDesc{Kartenbrief
\newline{}Handschrift: schwarze Tinte, deutsche Kurrent\newline{}Versand: 1) Stempel: »\nobreak{}Wien 68, 26. 2. 05, 5–6N\nobreak{}«.  2) Stempel: »\nobreak{}\oindex{Rodaun@\textbf{Rodaun}|pwk}Rodaun, 27. 2. \textcolor{gray}{05}, 7–9V\nobreak{}«. }\toendnotes[C]{\smallbreak}\pstart{}{\pb}Herrn \textsc{Dr. Richard
                     Beer-Hofmann}\pend{}\pstart{}\textsc{Rodaun\oindex{Rodaun@\textbf{Rodaun}|pw}}\pend{}\pstart{}\textsc{Liesinger Straße 2}\oindex{Liesingerstrasse@\textbf{Liesingerstraße}|pw}. \pend{}{\bigskip}\pstart
           \raggedleft{}{\pb}So{\geminationn}tag 26. 2. 905.\pend
           \pstart
           lieber Richard, ich reiſe am \label{K_L01502_1v}\edtext{Freitag 3.{ }}{\lemma{\textnormal{\emph{Freitag 3. }}}\Cendnote{\textnormal{siehe A. S.: \emph{Tagebuch}, 3. 3. 1905}}}\label{K_L01502_1h} Genua\oindex{Genua@\textbf{Genua}|pw} zu Mittelmeer\oindex{Mittelmeer@\textbf{Mittelmeer}|pw}zwecken; und, unter günſtigen Umſtänden bin ich erſt \label{K_L01502_2v}\edtext{gegen den 20.}{\lemma{\textnormal{\emph{gegen den 20.}}}\Cendnote{\textnormal{vgl. A. S.: \emph{Tagebuch}, 18. 3. 1905}}}\label{K_L01502_2h} wieder hier\substVorne{}\textsuperscript{?}\substDazwischen{}.\substHinten{}\pend
           \pstart
           Könnte man ſich nicht vorher doch einmal ſehen? Den Hugo’s\pwindex{Hofmannsthal, Gertrude von 16.03.1880 – 09.11.1959@\textsc{Hofmannsthal, Gertrude von} (16.03.1880 – 09.11.1959)|pw}\pwindex{Hofmannsthal, Hugo von 01.02.1874 – 15.07.1929@\textsc{Hofmannsthal, Hugo von} (01.02.1874 – 15.07.1929), \emph{Schriftsteller}|pw} hab ich für \label{K_L01502_3v}\edtext{Mittwoch}{\lemma{\textnormal{\emph{Mittwoch}}}\Cendnote{\textnormal{Das Treffen fand, ohne das Ehepaar Hofmannsthal\pwindex{Hofmannsthal, Gertrude von 16.03.1880 – 09.11.1959@\textsc{Hofmannsthal, Gertrude von} (16.03.1880 – 09.11.1959)|pwk}\pwindex{Hofmannsthal, Hugo von 01.02.1874 – 15.07.1929@\textsc{Hofmannsthal, Hugo von} (01.02.1874 – 15.07.1929), \emph{Schriftsteller}|pwk}, am Donnerstag statt; siehe A. S.: \emph{Tagebuch}, 2. 3. 1905}}}\label{K_L01502_3h}{ }Abend, Hietzing\oindex{Ottakringer Braeu@\textbf{Ottakringer Bräu}|pw} geſchrieben; kommen Sie
               etwa auch mit Paula\pwindex{Beer-Hofmann, Paula 25.02.1879 – 30.10.1939@\textsc{Beer-Hofmann, Paula} (25.02.1879 – 30.10.1939)|pw}? Oder wollen Sie nicht
               endlich einmal bei uns eſſen?\pend
           \pstart
           Laſſen Sie jedenfalls ein Wort hören.\pend
           \pstart
           Herzlichſt Ihr{\\[\baselineskip]}\spacefill\mbox{A.}\pend
           \leftskip=0em{}\endnumbering\briefempfaengerindex{Beer-Hofmann, Richard@\textsc{Beer-Hofmann, Richard}!zzzSchnitzler, Arthur@\emph{von Arthur Schnitzler}!1905-02-261@{26. 2. 1905}|)be}\mylabel{h}\end{ledgroupsized}  \newcommand{\dateiname}{L01502}\newcommand{\titel}{Arthur Schnitzler an Richard Beer-Hofmann, 26. 2. 1905}\newcommand{\editorInnen}{Martin Anton Müller und Gerd-Hermann Susen}\input{../tex-inputs/latex-pdf-abspann}
      