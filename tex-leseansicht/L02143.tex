%% latex-leseansicht-vorspann.tex
%% Vorspann für die Leseansicht.
%% Lädt die gemeinsame Datei latex-vorspann.tex mit nicht gesetztem Schalter.

\newif\ifkorrekturansicht
\korrekturansichtfalse

\input{../tex-inputs/latex-vorspann}


\section[Arthur Schnitzler an Frank Wedekind, 19. 7. 1913]{L02143 Arthur Schnitzler an Frank Wedekind, 19. 7. 1913}
\nopagebreak\mylabel{L02143v}
\rehead{ }\normalsize\beginnumbering\briefempfaengerindex{Wedekind, Frank@\textsc{Wedekind, Frank}!zzzSchnitzler, Arthur@\emph{von Arthur Schnitzler}!1913-07-193@{19. 7. 1913}|(be}
\toendnotes[C]{\smallbreak\pagebreak[2]}
\correspDesc{Versand  durch Arthur Schnitzler am 19. 7. 1913 in Wien
\newline{}Erhalt  durch Frank Wedekind im Zeitraum [19. 7. 1913
                  – 23. 7. 1913?] \textbf{Ort fehlend} }\toendnotes[C]{\smallbreak}
\Standort{München, Monacensia, FW B 159.}
\physDesc{Briefkarte, 602 Zeichen
\newline{}Handschrift: schwarze Tinte, deutsche Kurrent
\newline{}Ordnung: 1) mit blauem Buntstift von unbekannter Hand datiert: »Aug. 13«  2) Lochung}
\buchAbdrucke{\weitereDrucke{1) Peter Michael Braunwarth: \emph{In Reife und Überreife.} In: \emph{Die Presse}, 24. 9. 2004, Sec. Spectrum, S. IV.} \weitereDrucke{2) \emph{Frank Wedekinds Korrespondenz digital}. (7. 10. 2024) \url{https://briefedition.wedekind.h-da.de/view/document/single.xhtml?contentType=1&documentId=1554}.} }\toendnotes[C]{\smallbreak}
\pstart
           \raggedleft{}19/7 91\textcolor{gray}{3}\pend
           
\pstart
           {\pb}\textcolor{gray}{\textbf{Dr. Arthur Schnitzler}}{\\}\textcolor{gray}{\textbf{Wien XVIII. Sternwartestrasse 71\oindex{Wien@\textbf{Wien}!XVIII., Währing@\textbf{XVIII., Währing}!Sternwartestraße 71@\textbf{Sternwartestraße 71}, \emph{Wohngebäude}|pw}}}\pend
           
\pstart{}verehrter Herr Wedekind,\pend\vspace{0.5em}
\pstart
           erſt heute, da bei uns alles wieder in Ordnung iſt und wir uns zur Abreiſe rüſten,
               dank ich Ihnen für Ihre lieben theilnahmsvollen Zeilen, die Sie anläßlich der
               Erkrankung unſeres Sohnes\pwindex{Schnitzler, Heinrich 9.\,8.\,1902 Hinterbrühl – 12.\,7.\,1982 Wien@\textsc{Schnitzler, Heinrich} (9.\,8.\,1902 Hinterbrühl – 12.\,7.\,1982 Wien), \emph{Regisseur, Schauspieler}|pwv} an
               uns gerich{\pb}tet haben. Glücklicherweiſe iſt
               die Sache von Anfang an leicht verlaufen, und wir hatten mehr Unannehmlichkeiten als
               Sorgen.\pend
           
\pstart
           Sie, mein{ }ſehr verehrter lieber Herr Wedekind u Ihre \substVorne{}\textsuperscript{li}\substDazwischen{}verehrte\substHinten{}{ }Gattin\pwindex{Wedekind, Tilly 11.\,4.\,1886 Graz – 20.\,4.\,1970 München@\textsc{Wedekind, Tilly} (11.\,4.\,1886 Graz – 20.\,4.\,1970 München), \emph{Schauspielerin}|pwv} bei guter Gelegenheit
               wiederzuſehen hoffen meine Frau u ich von Herzen. Wie{ }ſchade daſs wir diesmal Sie
               beide und »Franziska\pwindex{Wedekind, Frank 24.\,7.\,1864 Hannover – 9.\,3.\,1918 München@\textsc{Wedekind, Frank} (24.\,7.\,1864 Hannover – 9.\,3.\,1918 München), \emph{Schriftsteller, Schauspieler, Schriftsteller}!Franziska. Ein Modernes Mysterium in fünf Akten@\strich\emph{Franziska. Ein Modernes Mysterium in fünf Akten}|pw}« verſäumt haben!\pend
           \pstart Viele Grüße von Ihrem \spacefill\mbox{Arthur Schnitzler}\pend{}\selectlanguage{ngerman}\endnumbering\briefempfaengerindex{Wedekind, Frank@\textsc{Wedekind, Frank}!zzzSchnitzler, Arthur@\emph{von Arthur Schnitzler}!1913-07-193@{19. 7. 1913}|)be}\mylabel{L02143h}  \newcommand{\dateiname}{L02143}\newcommand{\titel}{Arthur Schnitzler an Frank Wedekind, 19. 7. 1913}\newcommand{\editorInnen}{Martin Anton Müller und Gerd-Hermann Susen}%% latex-leseansicht-abspann.tex
%% Abspann für die Leseansicht.
%% Der Schalter \ifkorrekturansicht ist bereits durch den Vorspann gesetzt.

%% latex-abspann.tex
%% Gemeinsamer Abspann für Korrekturansicht und Leseansicht.
%% Setzt den Schalter \ifkorrekturansicht voraus (gesetzt in den
%% einbindenden Dateien latex-korrekturansicht-abspann.tex bzw.
%% latex-leseansicht-abspann.tex).
%% ---------------------------------------------------------------

\normalsize

% Das esempio-Environment wird nur in der Leseansicht benötigt
\ifkorrekturansicht\else
\newenvironment{esempio}[3]%
{
    \vspace{1.5ex}
    \rlap{\underline{#1}}
    \par
    \setlength{\parindent}{0cm}
    \nopagebreak
    \leftskip=#2cm
    \rightskip=#3cm
}
{
    \par
}
\fi

\doendnotes{C}
\bigskip
\vfill

\clearpage

\footnotesize

\ifkorrekturansicht
  \lohead{\textsc{register}}
\fi

% theindex-Environment neu definieren ohne reledmac
\makeatletter
\renewenvironment{theindex}{%
  \ifkorrekturansicht
    \section*{\indexname}%
  \else
    \subsubsection*{Index der erwähnten Entitäten}%
  \fi
  \setlength{\parindent}{0pt}%
  \setlength{\parskip}{0pt plus 0.3pt}%
  \let\item\@idxitem
}{%
  \ifkorrekturansicht\clearpage\fi
}
\makeatother

\IfFileExists{\jobname-pw.ind}{\input{\jobname-pw.ind}}{}

% Quellenangabe nur in der Leseansicht
\ifkorrekturansicht\else
% Fallback-Definitionen, falls die .tex-Datei \titel etc. nicht gesetzt hat
\providecommand{\titel}{}
\providecommand{\editorInnen}{}
\providecommand{\dateiname}{\jobname}

\vspace{3cm}

\vfill

\footnotesize
\textsc{Quelle}: \titel. Herausgegeben von {\editorInnen}. In: \emph{Arthur Schnitzler: Briefwechsel mit Autorinnen und Autoren}.
 Digitale Edition, https://schnitzler-briefe.acdh.oeaw.ac.at/{\dateiname}.html (Stand \today)
\fi

\end{document}


