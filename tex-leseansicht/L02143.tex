%% latex-korrekturansicht-vorspann.tex
%% Vorspann für die Korrekturansicht.
%% Lädt die gemeinsame Datei latex-vorspann.tex mit gesetztem Schalter.

\newif\ifkorrekturansicht
\korrekturansichttrue

\input{../tex-inputs/latex-vorspann}


\section[Arthur Schnitzler an Frank Wedekind, 19. 7. 1913]{L02143 Arthur Schnitzler an Frank Wedekind, 19. 7. 1913}
\nopagebreak\mylabel{L02143v}
\rehead{ }\normalsize\beginnumbering\briefempfaengerindex{Wedekind, Frank@\textsc{Wedekind, Frank}!zzzSchnitzler, Arthur@\emph{von Arthur Schnitzler}!1913-07-193@{19. 7. 1913}|(be}
\toendnotes[C]{\smallbreak\pagebreak[2]}\Standort{München, Monacensia, FW B 159.}
\physDesc{Briefkarte, 602 Zeichen
\newline{}Handschrift: schwarze Tinte, deutsche Kurrent
\newline{}Ordnung: 1) mit blauem Buntstift von unbekannter Hand datiert: »Aug. 13«  2) Lochung}
\buchAbdrucke{\weitereDrucke{\emph{Die Presse}, 24. 9. 2004, Sec. Spectrum, S. IV.} }\toendnotes[C]{\smallbreak}
\pstart
           \raggedleft{}19/7 91\textcolor{gray}{3}\pend
           
\pstart
           {\pb}\textcolor{gray}{\textbf{Dr. Arthur Schnitzler}}{\\}\textcolor{gray}{\textbf{Wien XVIII. Sternwartestrasse 71\oindex{Sternwartestrasse 71@\textbf{Sternwartestraße 71}, \emph{Wohngebäude (K.WHS)}|pw}}}\pend
           
\pstart{}verehrter Herr Wedekind,\pend\vspace{0.5em}
\pstart
           erſt heute, da bei uns alles wieder in Ordnung iſt und wir uns zur Abreiſe rüſten,
               dank ich Ihnen für Ihre lieben theilnahmsvollen Zeilen, die Sie anläßlich der
               Erkrankung unſeres Sohnes\pwindex{Schnitzler, Heinrich 09.08.1902 – 12.07.1982@\textsc{Schnitzler, Heinrich} (09.08.1902 – 12.07.1982), \emph{Regisseur/Regisseurin, Schauspieler/Schauspielerin}|pwv} an
               uns gerich{\pb}tet haben. Glücklicherweiſe iſt
               die Sache von Anfang an leicht verlaufen, und wir hatten mehr Unannehmlichkeiten als
               Sorgen.\pend
           
\pstart
           Sie, mein ſehr verehrter lieber Herr Wedekind u Ihre \substVorne{}\textsuperscript{li}\substDazwischen{}verehrte\substHinten{}{ }Gattin\pwindex{Wedekind, Tilly 11.04.1886 – 20.04.1970@\textsc{Wedekind, Tilly} (11.04.1886 – 20.04.1970), \emph{Schauspieler/Schauspielerin}|pwv} bei guter Gelegenheit
               wiederzuſehen hoffen meine Frau u ich von Herzen. Wie ſchade daſs wir diesmal Sie
               beide und »Franziska\pwindex{Franziska@\emph{Franziska}|pw}« verſäumt haben!\pend
           \pstart Viele Grüße von Ihrem \spacefill\mbox{Arthur Schnitzler}\pend{}\selectlanguage{ngerman}\endnumbering\briefempfaengerindex{Wedekind, Frank@\textsc{Wedekind, Frank}!zzzSchnitzler, Arthur@\emph{von Arthur Schnitzler}!1913-07-193@{19. 7. 1913}|)be}\mylabel{L02143h}  \normalsize

\doendnotes{C}
\bigskip
\vfill

\clearpage

\footnotesize

\lohead{\textsc{register}}

% Definiere theindex-Environment komplett neu ohne reledmac
\makeatletter
\renewenvironment{theindex}{%
  \section*{\indexname}%
  \setlength{\parindent}{0pt}%
  \setlength{\parskip}{0pt plus 0.3pt}%
  \let\item\@idxitem
}{%
  \clearpage
}
\makeatother

\IfFileExists{\jobname-pw.ind}{\input{\jobname-pw.ind}}{}

\end{document}

      