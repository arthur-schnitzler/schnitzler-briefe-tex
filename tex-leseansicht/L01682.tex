\input{../tex-inputs/latex-pdf-vorspann}
\begin{center}
            \textcolor{red}{ENTWURF. ENTZIFFERUNG NOCH NICHT KORREKTURGELESEN}
                      \end{center}
            
               \section[Hugo von Hofmannsthal an Arthur Schnitzler, 8. 6. 1907]{ Hugo von Hofmannsthal an Arthur Schnitzler, 8. 6. 1907}\nopagebreak\mylabel{v}\rehead{ }\begin{ledgroupsized}[t]{13cm}\normalsize\beginnumbering\briefempfaengerindex{Schnitzler, Arthur@\textsc{Schnitzler, Arthur}!zzzHofmannsthal, Hugo von@\emph{von Hugo von Hofmannsthal}!1907-06-081@{8. 6. 1907}|(be} \toendnotes[C]{\smallbreak\pagebreak[2]} \Standort{CUL, Schnitzler, B 43.}
\physDesc{Bildpostkarte
\newline{}Handschrift: schwarze Tinte, deutsche Kurrent\newline{}Versand: 1) Stempel: »\nobreak{}\oindex{Santa Maria Novella@\textbf{Santa Maria Novella}|pwk}Firenze Ferrovia, 8 6 07, \textcolor{gray}{4}S\nobreak{}«.  2) Stempel: »\nobreak{}\oindex{XVIII., Waehring@\textbf{XVIII., Währing}|pwk}18/1 Wien 110, 10 VII 07, XII\nobreak{}«. \newline{}Ordnung: 1) mit Bleistift von unbekannter Hand nummeriert: »\strikeout{277}« 2) mit Bleistift von unbekannter Hand nummeriert: »274«}\buchAbdrucke{\weitereDrucke{Hugo von Hofmannsthal, Arthur Schnitzler: \emph{Briefwechsel}. Hg. Therese Nickl und Heinrich Schnitzler. Frankfurt am Main: \emph{S. Fischer} 1964, S. 228.} }\pstart{}{\pb}\textsc{Herrn D\textsuperscript{r} Arthur
                  Schnitzler}\pend{}\pstart{}\textsc{Wien}\oindex{Wien@\textbf{Wien}|pw}\pend{}\pstart{}\textsc{XVIII Spöttelgasse 7}\oindex{Edmund-Weiss-Gasse@\textbf{Edmund-Weiß-Gasse}|pw}\pend{}\pstart{}\textsc{Austria\oindex{Oesterreich@\textbf{Österreich}|pw}}\pend{}{\bigskip}\pstart
           \noindent{}\centering{}\textcolor{gray}{\textbf{{\pb}Desposizione di Croce\pwindex{Fra Bartolommeo 1472-03-28 – 1517-10-31@\textsc{Fra Bartolommeo} (1472-03-28 – 1517-10-31), \emph{Maler}!Trauer um den toten Christus1511/1512?@\strich\emph{Trauer um den toten Christus} {[}1511/1512?{]}|pw} (Fra Bartolomeo\pwindex{Fra Bartolommeo 1472-03-28 – 1517-10-31@\textsc{Fra Bartolommeo} (1472-03-28 – 1517-10-31), \emph{Maler}|pw})}}\pend
           \pstart
           \raggedleft{}8 VI.\pend
           \pstart
           Viele Grüße – ich mache viele ſchöne Landpartien.\pend
           \pstart \spacefill\mbox{Hugo.}\pend{}\endnumbering\briefempfaengerindex{Schnitzler, Arthur@\textsc{Schnitzler, Arthur}!zzzHofmannsthal, Hugo von@\emph{von Hugo von Hofmannsthal}!1907-06-081@{8. 6. 1907}|)be}\mylabel{h}\end{ledgroupsized}  \newcommand{\dateiname}{L01682}\newcommand{\titel}{Hugo von Hofmannsthal an Arthur Schnitzler, 8. 6. 1907}\newcommand{\editorInnen}{Martin Anton Müller und Gerd-Hermann Susen}\input{../tex-inputs/latex-pdf-abspann}
      