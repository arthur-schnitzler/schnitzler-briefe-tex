%% latex-korrekturansicht-vorspann.tex
%% Vorspann für die Korrekturansicht.
%% Lädt die gemeinsame Datei latex-vorspann.tex mit gesetztem Schalter.

\newif\ifkorrekturansicht
\korrekturansichttrue

\input{../tex-inputs/latex-vorspann}


\section[Hugo von Hofmannsthal an Arthur Schnitzler, 3. 11. {[}1903{]}]{L01334 Hugo von Hofmannsthal an Arthur Schnitzler, 3. 11. {[}1903{]}}
\nopagebreak\mylabel{L01334v}
\rehead{ }\normalsize\beginnumbering\briefempfaengerindex{Schnitzler, Arthur@\textsc{Schnitzler, Arthur}!zzzHofmannsthal, Hugo von@\emph{von Hugo von Hofmannsthal}!1903-11-031@{3. 11. {[}1903{]}}|(be}
\toendnotes[C]{\smallbreak\pagebreak[2]}\Standort{CUL, Schnitzler, B 43.}
\physDesc{Brief, 1 Blatt, 3 Seiten, 592 Zeichen
\newline{}Handschrift: schwarze Tinte, deutsche Kurrent
\newline{}Schnitzler: mit Bleistift die Jahreszahl ergänzt: »903« 
\newline{}Ordnung: 1) mit Bleistift von unbekannter Hand nummeriert: »\strikeout{211}«  2) mit Bleistift von unbekannter Hand nummeriert: »204«}
\buchAbdrucke{\weitereDrucke{Hugo von Hofmannsthal, Arthur Schnitzler: \emph{Briefwechsel}. Frankfurt am Main: \emph{S. Fischer} 1964, S. 175–176.} }\toendnotes[C]{\smallbreak}
\pstart
           \raggedleft{}{\pb}3 XI.\pend
           
\pstart{}lieber, \pend\vspace{0.5em}
\pstart
           Hauptmann\pwindex{Hauptmann, Gerhart 15.11.1862 – 06.06.1946@\textsc{Hauptmann, Gerhart} (15.11.1862 – 06.06.1946), \emph{Schriftsteller/Schriftstellerin}|pw}, Brahm\pwindex{Brahm, Otto 05.02.1856 – 28.11.1912@\textsc{Brahm, Otto} (05.02.1856 – 28.11.1912), \emph{Theaterleiter/Theaterleiterin, Regisseur/Regisseurin}|pw}, Harden\pwindex{Harden, Maximilian 20.10.1861 – 30.10.1927@\textsc{Harden, Maximilian} (20.10.1861 – 30.10.1927), \emph{Schriftsteller/Schriftstellerin, Publizist/Publizistin}|pw} laſſen Sie herzlich
               grüßen. Mittlerer bittet dringend, ihn \uuline{unverweilt} zu
               verſtändigen, wie bald er Ihr Stück\pwindex{einsame Weg. Schauspiel in fuenf Akten@\emph{Der einsame Weg. Schauspiel in fünf Akten}|pwv} erwarten darf. Er hat große \textsc{chancen}, es \uline{baldigſt} zu ſpielen.\pend
           
\pstart
           Aber Vorleſen! Bitten leſen Sie es\pwindex{einsame Weg. Schauspiel in fuenf Akten@\emph{Der einsame Weg. Schauspiel in fünf Akten}|pwv} vor. Das ſind ſo gemüthliche Abende. Bei {\pb}Ihnen, bei Richard\pwindex{Beer-Hofmann, Richard 1866-07-11 – 1945-09-26@\textsc{Beer-Hofmann, Richard} (1866-07-11 – 1945-09-26), \emph{Schriftsteller/Schriftstellerin}|pw}, wo immer. Hoffentlich bald.\pend
           
\pstart
           Von Herzen{\\[\baselineskip]}\spacefill\mbox{Hugo}\pend
           \leftskip=0em{}
\pstart
           \noindent{}P. S. Gerty\pwindex{Hofmannsthal, Gertrude von 16.03.1880 – 09.11.1959@\textsc{Hofmannsthal, Gertrude von} (16.03.1880 – 09.11.1959)|pw} und das neue baby\pwindex{Hofmannsthal, Franz von 20.10.1903 – 13.07.1929@\textsc{Hofmannsthal, Franz von} (20.10.1903 – 13.07.1929)|pwv} ſind wohl, Elektra\pwindex{Elektra. Tragoedie in einem Aufzug@\emph{Elektra. Tragödie in einem Aufzug}|pw} in Berlin\oindex{Berlin@\textbf{Berlin}, \emph{P.PPLC}|pw}
                  desgleichen. Die Bekannten des Bearbeiters haben dort vorläufig für 7 oder 8
                  Vorſtellungen alle Plätze vorgemerkt. Es iſt doch ein Glück, \substVorne{}\textsuperscript{wenn}\substDazwischen{}daſs\substHinten{} man ſo viele {\pb}Bekannte
                  hat und daſs Dr. \label{K_L01334-1v}\edtext{Goldmann\pwindex{Goldmann, Paul 31.01.1865 – 25.09.1935@\textsc{Goldmann, Paul} (31.01.1865 – 25.09.1935), \emph{Schriftsteller/Schriftstellerin, Journalist/Journalistin}|pw} nicht zu ihnen gehört}{\lemma{\textnormal{\emph{Goldmann … gehört}}}\Cendnote{\textnormal{Anspielung auf dessen Depesche\pwindex{Aus Berlin [Elektra-Premiere]@\emph{Aus Berlin [Elektra-Premiere]}|pwkv}:
                        »Aus \so{Berlin}\oindex{Berlin@\textbf{Berlin}, \emph{P.PPLC}|pw}
                        telegraphiert unſer Korreſpondent\pwindex{Goldmann, Paul 31.01.1865 – 25.09.1935@\textsc{Goldmann, Paul} (31.01.1865 – 25.09.1935), \emph{Schriftsteller/Schriftstellerin, Journalist/Journalistin}|pwv}: Im Kleinen Theater\orgindex{Kleines Theater@Kleines Theater|pw} wurde
                        heute die Tragödie ›\so{Elektra}\pwindex{Elektra. Tragoedie in einem Aufzug@\emph{Elektra. Tragödie in einem Aufzug}|pw}‹
                           aufgeführt. Der Theaterzettel kündigte ein Trauerſpiel von Hugo v.
                              \so{Hofmannsthal}\pwindex{Hofmannsthal, Hugo von 1874-02-01 – 1929-07-15@\textsc{Hofmannsthal, Hugo von} (1874-02-01 – 1929-07-15), \emph{Schriftsteller/Schriftstellerin}|pw} nach \so{Sophokles}\pwindex{Sophokles 497/496? v. u. Z. – 406/405 v. u. Z.@\textsc{Sophokles} (497/496? v. u. Z. – 406/405 v. u. Z.), \emph{Schriftsteller/Schriftstellerin}|pw}\pwindex{Elektra. Tragoedie@\emph{Elektra. Tragödie}|pw} an, und
                           der Theaterzettel hatte recht. Hofmannsthal\pwindex{Hofmannsthal, Hugo von 1874-02-01 – 1929-07-15@\textsc{Hofmannsthal, Hugo von} (1874-02-01 – 1929-07-15), \emph{Schriftsteller/Schriftstellerin}|pw} hat aus der alten
                           Tragödie ein modernes Schauerdrama mit Maeterlinck\pwindex{Maeterlinck, Maurice 29.08.1862 – 06.05.1949@\textsc{Maeterlinck, Maurice} (29.08.1862 – 06.05.1949), \emph{Schriftsteller/Schriftstellerin}|pw}-Anklängen und
                        aus der Elektra eine perverſe, in blutigen Halluzinationen ſchwelgende
                        Megäre gemacht. Von der Hoheit der Geſtalten der alten Tragödie iſt nichts
                        übrig geblieben. In dieſer modernen Faſſung ergreift das Drama nicht mehr,
                        und man kann nur mit Staunen all den ſeltſamen Bildern und Gleichniſſen
                        folgen, mit denen Hofmannsthal\pwindex{Hofmannsthal, Hugo von 1874-02-01 – 1929-07-15@\textsc{Hofmannsthal, Hugo von} (1874-02-01 – 1929-07-15), \emph{Schriftsteller/Schriftstellerin}|pw} den Dialog, den er gänzlich neu
                        geſchrieben hat, überfüllt hat und die er mit nervöſer Haſt hintereinander
                        herjagt. Als der Vorhang fiel, herrſchte zunächſt ein minutenlanges
                        Schweigen der Verblüffung. Dann übernahmen die Freunde des Bearbeiters, die
                        in großer Zahl anweſend waren, die Führung und zeigten dem ſchwankenden
                        Publikum den Weg. Ihr Beiſall übertönte die Oppoſition, und
                           \so{Hofmannsthal}\pwindex{Hofmannsthal, Hugo von 1874-02-01 – 1929-07-15@\textsc{Hofmannsthal, Hugo von} (1874-02-01 – 1929-07-15), \emph{Schriftsteller/Schriftstellerin}|pw} konnte vier- oder fünfmal vor dem Vorhang
                           erſcheinen. Frau \so{Eyſoldt}\pwindex{Eysoldt, Gertrud 30.11.1870 – 05.01.1955@\textsc{Eysoldt, Gertrud} (30.11.1870 – 05.01.1955), \emph{Theaterleiter/Theaterleiterin, Schauspieler/Schauspielerin}|pw} ſpielte die Elektra genau ſo
                        abſonderlich und pervers, wie der Bearbeiter die Figur geſtaltet hatte.
                        Einen großen Stil hatte allein die Darſtellung der Klytämneſtra durch Frau
                           \so{Bertens}\pwindex{Bertens, Rosa 1860 – 1934-10-05@\textsc{Bertens, Rosa} (1860 – 1934-10-05), \emph{Schauspieler/Schauspielerin, Filmschauspieler/Filmschauspielerin}|pw}.« \emph{Neue
                           Freie Presse,}\pwindex{Neue Freie Presse@\emph{Neue Freie Presse}|pwk} Nr. 14.073, 31. 10. 1903,
                        Morgenblatt, S. 11.}}}\label{K_L01334-1}.\pend
           \selectlanguage{ngerman}\endnumbering\briefempfaengerindex{Schnitzler, Arthur@\textsc{Schnitzler, Arthur}!zzzHofmannsthal, Hugo von@\emph{von Hugo von Hofmannsthal}!1903-11-031@{3. 11. {[}1903{]}}|)be}\mylabel{L01334h}  \normalsize

\doendnotes{C}
\bigskip
\vfill

\clearpage

\footnotesize

\lohead{\textsc{register}}

% Definiere theindex-Environment komplett neu ohne reledmac
\makeatletter
\renewenvironment{theindex}{%
  \section*{\indexname}%
  \setlength{\parindent}{0pt}%
  \setlength{\parskip}{0pt plus 0.3pt}%
  \let\item\@idxitem
}{%
  \clearpage
}
\makeatother

\IfFileExists{\jobname-pw.ind}{\input{\jobname-pw.ind}}{}

\end{document}

      