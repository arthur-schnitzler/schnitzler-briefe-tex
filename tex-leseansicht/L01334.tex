%% latex-leseansicht-vorspann.tex
%% Vorspann für die Leseansicht.
%% Lädt die gemeinsame Datei latex-vorspann.tex mit nicht gesetztem Schalter.

\newif\ifkorrekturansicht
\korrekturansichtfalse

\input{../tex-inputs/latex-vorspann}


\section[Hugo von Hofmannsthal an Arthur Schnitzler, 3. 11. [1903]]{L01334 Hugo von Hofmannsthal an Arthur Schnitzler, 3. 11. [1903]}
\nopagebreak\mylabel{L01334v}
\rehead{ }\normalsize\beginnumbering\briefempfaengerindex{Schnitzler, Arthur@\textsc{Schnitzler, Arthur}!zzzHofmannsthal, Hugo von@\emph{von Hugo von Hofmannsthal}!1903-11-031@{3. 11. [1903]}|(be}
\toendnotes[C]{\smallbreak\pagebreak[2]}
\correspDesc{Versand  durch Hugo von Hofmannsthal am 3. 11. [1903] in Berlin
\newline{}Erhalt  durch Arthur Schnitzler im Zeitraum [4. 11. 1903
                  – 8. 11. 1903?] in Wien}\toendnotes[C]{\smallbreak}
\Standort{CUL, Schnitzler, B 43.}
\physDesc{Brief, 1 Blatt, 3 Seiten, 592 Zeichen
\newline{}Handschrift: schwarze Tinte, deutsche Kurrent
\newline{}Schnitzler: mit Bleistift die Jahreszahl ergänzt: »903« 
\newline{}Ordnung: 1) mit Bleistift von unbekannter Hand nummeriert: »\strikeout{211}«  2) mit Bleistift von unbekannter Hand nummeriert: »204«}
\buchAbdrucke{\weitereDrucke{Hugo von Hofmannsthal, Arthur Schnitzler: \emph{Briefwechsel}. Herausgegeben von Therese Nickl und Heinrich Schnitzler. Frankfurt am Main: \emph{S. Fischer} 1964, S. 175–176.} }\toendnotes[C]{\smallbreak}
\pstart
           \raggedleft{}{\pb}3 XI.\pend
           
\pstart{}lieber,\pend\vspace{0.5em}
\pstart
           Hauptmann\pwindex{Hauptmann, Gerhart 15.\,11.\,1862 Szczawno-Zdrój – 6.\,6.\,1946 Jagniątków@\textsc{Hauptmann, Gerhart} (15.\,11.\,1862 Szczawno-Zdrój – 6.\,6.\,1946 Jagniątków), \emph{Schriftsteller}|pw}, Brahm\pwindex{Brahm, Otto 5.\,2.\,1856 Hamburg – 28.\,11.\,1912 Berlin@\textsc{Brahm, Otto} (5.\,2.\,1856 Hamburg – 28.\,11.\,1912 Berlin), \emph{Theaterleiter, Regisseur}|pw}, Harden\pwindex{Harden, Maximilian 20.\,10.\,1861 Berlin – 30.\,10.\,1927 Montana@\textsc{Harden, Maximilian} (20.\,10.\,1861 Berlin – 30.\,10.\,1927 Montana), \emph{Schriftsteller, Publizist}|pw} laſſen Sie herzlich
               grüßen. Mittlerer bittet dringend, ihn \uuline{unverweilt} zu
               verſtändigen, wie bald er Ihr Stück\pwindex{Schnitzler, Arthur 15.\,5.\,1862 Wien – 21.\,10.\,1931 ebd.@\textsc{Schnitzler, Arthur} (15.\,5.\,1862 Wien – 21.\,10.\,1931 ebd.), \emph{Schriftsteller, Mediziner}!einsame Weg. Schauspiel in fünf Akten@\strich\emph{Der einsame Weg. Schauspiel in fünf Akten}|pwv} erwarten darf. Er hat große \textsc{chancen}, es \uline{baldigſt} zu{ }ſpielen.\pend
           
\pstart
           Aber Vorleſen! Bitten leſen Sie es\pwindex{Schnitzler, Arthur 15.\,5.\,1862 Wien – 21.\,10.\,1931 ebd.@\textsc{Schnitzler, Arthur} (15.\,5.\,1862 Wien – 21.\,10.\,1931 ebd.), \emph{Schriftsteller, Mediziner}!einsame Weg. Schauspiel in fünf Akten@\strich\emph{Der einsame Weg. Schauspiel in fünf Akten}|pwv} vor. Das{ }ſind{ }ſo gemüthliche Abende. Bei {\pb}Ihnen, bei Richard\pwindex{Beer-Hofmann, Richard 11.\,7.\,1866 Wien – 26.\,9.\,1945 New York City@\textsc{Beer-Hofmann, Richard} (11.\,7.\,1866 Wien – 26.\,9.\,1945 New York City), \emph{Schriftsteller}|pw}, wo immer. Hoffentlich bald.\pend
           
\pstart
           Von Herzen{\\[\baselineskip]}\spacefill\mbox{Hugo}\pend
           \leftskip=0em{}
\pstart
           \noindent{}P. S. Gerty\pwindex{Hofmannsthal, Gertrude von 16.\,3.\,1880 Wien – 9.\,11.\,1959 Paddington@\textsc{Hofmannsthal, Gertrude von} (16.\,3.\,1880 Wien – 9.\,11.\,1959 Paddington)|pw} und das neue baby\pwindex{Hofmannsthal, Franz von 20.\,10.\,1903 Wien – 13.\,7.\,1929 ebd.@\textsc{Hofmannsthal, Franz von} (20.\,10.\,1903 Wien – 13.\,7.\,1929 ebd.)|pwv}{ }ſind wohl, Elektra\pwindex{Hofmannsthal, Hugo von 1.\,2.\,1874 Wien – 15.\,7.\,1929 Rodaun@\textsc{Hofmannsthal, Hugo von} (1.\,2.\,1874 Wien – 15.\,7.\,1929 Rodaun), \emph{Schriftsteller}!Elektra. Tragödie in einem Aufzug@\strich\emph{Elektra. Tragödie in einem Aufzug}|pw} in Berlin\oindex{Berlin@\textbf{Berlin}, \emph{Hauptstadt}|pw}
                  desgleichen. Die Bekannten des Bearbeiters haben dort vorläufig für 7 oder 8
                  Vorſtellungen alle Plätze vorgemerkt. Es iſt doch ein Glück, \substVorne{}\textsuperscript{wenn}\substDazwischen{}daſs\substHinten{} man{ }ſo viele {\pb}Bekannte
                  hat und daſs Dr. \label{K_L01334-1v}\edtext{Goldmann\pwindex{Goldmann, Paul 31.\,1.\,1865 Breslau – 25.\,9.\,1935 Wien@\textsc{Goldmann, Paul} (31.\,1.\,1865 Breslau – 25.\,9.\,1935 Wien), \emph{Schriftsteller, Journalist}|pw} nicht zu ihnen gehört}{\lemma{\textnormal{\emph{Goldmann … gehört}}}\Cendnote{\textnormal{Anspielung auf dessen Depesche\pwindex{Aus Berlin [Elektra-Premiere]@\emph{Aus Berlin [Elektra-Premiere]}|pwkv}:
                        »Aus \so{Berlin}\oindex{Berlin@\textbf{Berlin}, \emph{Hauptstadt}|pw}
                        telegraphiert unſer Korreſpondent\pwindex{Goldmann, Paul 31.\,1.\,1865 Breslau – 25.\,9.\,1935 Wien@\textsc{Goldmann, Paul} (31.\,1.\,1865 Breslau – 25.\,9.\,1935 Wien), \emph{Schriftsteller, Journalist}|pwv}: Im Kleinen Theater\orgindex{Kleines Theater@Kleines Theater|pw} wurde
                        heute die Tragödie ›\so{Elektra}\pwindex{Hofmannsthal, Hugo von 1.\,2.\,1874 Wien – 15.\,7.\,1929 Rodaun@\textsc{Hofmannsthal, Hugo von} (1.\,2.\,1874 Wien – 15.\,7.\,1929 Rodaun), \emph{Schriftsteller}!Elektra. Tragödie in einem Aufzug@\strich\emph{Elektra. Tragödie in einem Aufzug}|pw}‹
                           aufgeführt. Der Theaterzettel kündigte ein Trauerſpiel von Hugo v.
                              \so{Hofmannsthal}\pwindex{Hofmannsthal, Hugo von 1.\,2.\,1874 Wien – 15.\,7.\,1929 Rodaun@\textsc{Hofmannsthal, Hugo von} (1.\,2.\,1874 Wien – 15.\,7.\,1929 Rodaun), \emph{Schriftsteller}|pw} nach \so{Sophokles}\pwindex{Sophokles 497/496? v.\,u.\,Z. Kolonos – 406/405 v.\,u.\,Z. Athen@\textsc{Sophokles} (497/496? v.\,u.\,Z. Kolonos – 406/405 v.\,u.\,Z. Athen), \emph{Schriftsteller}|pw}\pwindex{Sophokles 497/496? v.\,u.\,Z. Kolonos – 406/405 v.\,u.\,Z. Athen@\textsc{Sophokles} (497/496? v.\,u.\,Z. Kolonos – 406/405 v.\,u.\,Z. Athen), \emph{Schriftsteller}!Elektra. Tragödie@\strich\emph{Elektra. Tragödie}|pw} an, und
                           der Theaterzettel hatte recht. Hofmannsthal\pwindex{Hofmannsthal, Hugo von 1.\,2.\,1874 Wien – 15.\,7.\,1929 Rodaun@\textsc{Hofmannsthal, Hugo von} (1.\,2.\,1874 Wien – 15.\,7.\,1929 Rodaun), \emph{Schriftsteller}|pw} hat aus der alten
                           Tragödie ein modernes Schauerdrama mit Maeterlinck\pwindex{Maeterlinck, Maurice 29.\,8.\,1862 Gent – 6.\,5.\,1949 Nizza@\textsc{Maeterlinck, Maurice} (29.\,8.\,1862 Gent – 6.\,5.\,1949 Nizza), \emph{Schriftsteller}|pw}-Anklängen und
                        aus der Elektra eine perverſe, in blutigen Halluzinationen{ }ſchwelgende
                        Megäre gemacht. Von der Hoheit der Geſtalten der alten Tragödie iſt nichts
                        übrig geblieben. In dieſer modernen Faſſung ergreift das Drama nicht mehr,
                        und man kann nur mit Staunen all den{ }ſeltſamen Bildern und Gleichniſſen
                        folgen, mit denen Hofmannsthal\pwindex{Hofmannsthal, Hugo von 1.\,2.\,1874 Wien – 15.\,7.\,1929 Rodaun@\textsc{Hofmannsthal, Hugo von} (1.\,2.\,1874 Wien – 15.\,7.\,1929 Rodaun), \emph{Schriftsteller}|pw} den Dialog, den er gänzlich neu
                        geſchrieben hat, überfüllt hat und die er mit nervöſer Haſt hintereinander
                        herjagt. Als der Vorhang fiel, herrſchte zunächſt ein minutenlanges
                        Schweigen der Verblüffung. Dann übernahmen die Freunde des Bearbeiters, die
                        in großer Zahl anweſend waren, die Führung und zeigten dem{ }ſchwankenden
                        Publikum den Weg. Ihr Beiſall übertönte die Oppoſition, und
                           \so{Hofmannsthal}\pwindex{Hofmannsthal, Hugo von 1.\,2.\,1874 Wien – 15.\,7.\,1929 Rodaun@\textsc{Hofmannsthal, Hugo von} (1.\,2.\,1874 Wien – 15.\,7.\,1929 Rodaun), \emph{Schriftsteller}|pw} konnte vier- oder fünfmal vor dem Vorhang
                           erſcheinen. Frau \so{Eyſoldt}\pwindex{Eysoldt, Gertrud 30.\,11.\,1870 Pirna – 5.\,1.\,1955 Ohlstadt@\textsc{Eysoldt, Gertrud} (30.\,11.\,1870 Pirna – 5.\,1.\,1955 Ohlstadt), \emph{Theaterleiterin, Schauspielerin}|pw}{ }ſpielte die Elektra genau{ }ſo
                        abſonderlich und pervers, wie der Bearbeiter die Figur geſtaltet hatte.
                        Einen großen Stil hatte allein die Darſtellung der Klytämneſtra durch Frau
                           \so{Bertens}\pwindex{Bertens, Rosa 1860 Istanbul – 5.\,10.\,1934@\textsc{Bertens, Rosa} (1860 Istanbul – 5.\,10.\,1934), \emph{Schauspielerin, Filmschauspielerin}|pw}.« \emph{Neue
                           Freie Presse,}\pwindex{Neue Freie Presse@\emph{Neue Freie Presse}|pwk} Nr. 14.073, 31. 10. 1903,
                        Morgenblatt, S. 11.}}}\label{K_L01334-1}.\pend
           \selectlanguage{ngerman}\endnumbering\briefempfaengerindex{Schnitzler, Arthur@\textsc{Schnitzler, Arthur}!zzzHofmannsthal, Hugo von@\emph{von Hugo von Hofmannsthal}!1903-11-031@{3. 11. [1903]}|)be}\mylabel{L01334h}  \newcommand{\dateiname}{L01334}\newcommand{\titel}{Hugo von Hofmannsthal an Arthur Schnitzler, 3. 11. [1903]}\newcommand{\editorInnen}{Martin Anton Müller und Gerd-Hermann Susen}%% latex-leseansicht-abspann.tex
%% Abspann für die Leseansicht.
%% Der Schalter \ifkorrekturansicht ist bereits durch den Vorspann gesetzt.

%% latex-abspann.tex
%% Gemeinsamer Abspann für Korrekturansicht und Leseansicht.
%% Setzt den Schalter \ifkorrekturansicht voraus (gesetzt in den
%% einbindenden Dateien latex-korrekturansicht-abspann.tex bzw.
%% latex-leseansicht-abspann.tex).
%% ---------------------------------------------------------------

\normalsize

% Das esempio-Environment wird nur in der Leseansicht benötigt
\ifkorrekturansicht\else
\newenvironment{esempio}[3]%
{
    \vspace{1.5ex}
    \rlap{\underline{#1}}
    \par
    \setlength{\parindent}{0cm}
    \nopagebreak
    \leftskip=#2cm
    \rightskip=#3cm
}
{
    \par
}
\fi

\doendnotes{C}
\bigskip
\vfill

\clearpage

\footnotesize

\ifkorrekturansicht
  \lohead{\textsc{register}}
\fi

% theindex-Environment neu definieren ohne reledmac
\makeatletter
\renewenvironment{theindex}{%
  \ifkorrekturansicht
    \section*{\indexname}%
  \else
    \subsubsection*{Index der erwähnten Entitäten}%
  \fi
  \setlength{\parindent}{0pt}%
  \setlength{\parskip}{0pt plus 0.3pt}%
  \let\item\@idxitem
}{%
  \ifkorrekturansicht\clearpage\fi
}
\makeatother

\IfFileExists{\jobname-pw.ind}{\input{\jobname-pw.ind}}{}

% Quellenangabe nur in der Leseansicht
\ifkorrekturansicht\else
% Fallback-Definitionen, falls die .tex-Datei \titel etc. nicht gesetzt hat
\providecommand{\titel}{}
\providecommand{\editorInnen}{}
\providecommand{\dateiname}{\jobname}

\vspace{3cm}

\vfill

\footnotesize
\textsc{Quelle}: \titel. Herausgegeben von {\editorInnen}. In: \emph{Arthur Schnitzler: Briefwechsel mit Autorinnen und Autoren}.
 Digitale Edition, https://schnitzler-briefe.acdh.oeaw.ac.at/{\dateiname}.html (Stand \today)
\fi

\end{document}


