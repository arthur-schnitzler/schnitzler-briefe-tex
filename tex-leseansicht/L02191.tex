%% latex-leseansicht-vorspann.tex
%% Vorspann für die Leseansicht.
%% Lädt die gemeinsame Datei latex-vorspann.tex mit nicht gesetztem Schalter.

\newif\ifkorrekturansicht
\korrekturansichtfalse

\input{../tex-inputs/latex-vorspann}


\section[Richard Beer-Hofmann an Arthur Schnitzler, 10. 8. 1914]{L02191 Richard Beer-Hofmann an Arthur Schnitzler, 10. 8. 1914}
\nopagebreak\mylabel{L02191v}
\rehead{ }\normalsize\beginnumbering\briefempfaengerindex{Schnitzler, Arthur@\textsc{Schnitzler, Arthur}!zzzBeer-Hofmann, Richard@\emph{von Richard Beer-Hofmann}!1914-08-101@{10. 8. 1914}|(be}
\toendnotes[C]{\smallbreak\pagebreak[2]}
\correspDesc{Versand  durch Richard Beer-Hofmann am 10. 8. 1914 in Weißenbach am Attersee
\newline{}Übermittlung  am 11. 8. 1914 in Weißenbach am Attersee
\newline{}Weiterleitung  am 16. 8. 1914 in Celerina
\newline{}Weiterleitung  am 17. 8. 1914 in Bad Ragaz
\newline{}Erhalt  durch Arthur Schnitzler am [2. 9. 1914?] \textbf{Ort fehlend} }\toendnotes[C]{\smallbreak}
\Standort{CUL, Schnitzler, B 8.}
\physDesc{Bildpostkarte, 349 Zeichen
\newline{}Handschrift: Bleistift, lateinische Kurrent
\newline{}Versand: 1) Stempel: »\nobreak{}\oindex{Weißenbach am Attersee@\textbf{Weißenbach am Attersee}, \emph{Verwaltungsgebiet}|pwk}Weissenbach am Attersee, 1\textcolor{gray}{1}. VIII. 14\nobreak{}«.   2) Stempel: »\nobreak{}\oindex{Celerina@\textbf{Celerina}|pwk}Celerina (Graubünden), 16. VIII. 14, 1\nobreak{}«.  3) postalischer Nachsendevermerk: »Hotel Lattmann\oindex{Hotel Lattmann@\textbf{Hotel Lattmann}, \emph{Hotel}|pw}, Ragaz\oindex{Bad Ragaz@\textbf{Bad Ragaz}|pw}«  4) Stempel: »\nobreak{}\oindex{Bad Ragaz@\textbf{Bad Ragaz}|pwk}Ragaz, 17. VIII. 14, 3\nobreak{}«.  5) postalischer Nachsendevermerk: »Wien XVIII\oindex{XVIII., Währing@\textbf{XVIII., Währing}, \emph{Verwaltungsgebiet}|pw}, Sternwartestr. 71\oindex{Wien@\textbf{Wien}!XVIII., Währing@\textbf{XVIII., Währing}!Sternwartestraße 71@\textbf{Sternwartestraße 71}, \emph{Wohngebäude}|pw}« 
\newline{}Ordnung: mit Bleistift von unbekannter Hand nummeriert:
                                    »259« }
\buchAbdrucke{\weitereDrucke{Arthur Schnitzler, Richard Beer-Hofmann: \emph{Briefwechsel 1891–1931}. Herausgegeben von Konstanze Fliedl. Wien, Zürich: \emph{Europaverlag} 1992, S. 220.} }\toendnotes[C]{\smallbreak}\pstart{}{\pb}Herrn D\textsuperscript{r} Arthur Schnitzler\pend{}\pstart{}Schweiz\oindex{Schweiz@\textbf{Schweiz}|pw}\pend{}\pstart{}Celerina\oindex{Celerina@\textbf{Celerina}|pw}\pend{}\pstart{}Cresta Palace\oindex{Cresta Palace@\textbf{Cresta Palace}, \emph{Hotel}|pw}\pend{}{\bigskip}
\pstart
           \noindent{}\centering{}{\pb}\textcolor{gray}{\textbf{Salzkammergut\oindex{Salzkammergut@\textbf{Salzkammergut}, \emph{Region}|pw}. Weissenbach am Attersee\oindex{Weißenbach am Attersee@\textbf{Weißenbach am Attersee}, \emph{Verwaltungsgebiet}|pw}.}}\pend
           \vspace{1em}
\pstart
           \raggedleft{}{\pb}10/VIII. 14\pend
           \vspace{0.5em}
\pstart
           Lieber Arthur! Ich war für zwei Tage – getrieben von Unruhe – in Wien\oindex{Wien@\textbf{Wien}, \emph{Verwaltungsgebiet}|pw} und sah dass es zwecklos wäre \uline{jetzt} dorthin mit den Kindern\pwindex{Beer-Hofmann, Naëmah 20.\,12.\,1898 Wien – 10.\,11.\,1971 New York City@\textsc{Beer-Hofmann, Naëmah} (20.\,12.\,1898 Wien – 10.\,11.\,1971 New York City)|pwv}\pwindex{Beer-Hofmann, Mirjam 4.\,9.\,1897 Wien – 24.\,12.\,1984 New York City@\textsc{Beer-Hofmann, Mirjam} (4.\,9.\,1897 Wien – 24.\,12.\,1984 New York City)|pwv}\pwindex{Beer-Hofmann, Gabriel 9.\,1.\,1901 Wien – 24.\,3.\,1971 St Albans@\textsc{Beer-Hofmann, Gabriel} (9.\,1.\,1901 Wien – 24.\,3.\,1971 St Albans), \emph{Schriftsteller, Filmagent}|pwv} zurückzugehen. So
               bleibe ich noch – wie lange? – hier. \uline{Zu} weit vom
               Schuss sein ist auch unerträglich. Was ists mit Kaufmann\pwindex{Kaufmann, Arthur 4.\,4.\,1872 Iași – 25.\,7.\,1938 Wien@\textsc{Kaufmann, Arthur} (4.\,4.\,1872 Iași – 25.\,7.\,1938 Wien), \emph{Rechtswissenschaftler, Privatgelehrte, Privatier}|pw}, Leo\pwindex{Van-Jung, Leo 15.\,10.\,1866 Odessa – 2.\,7.\,1939 Riga@\textsc{Van-Jung, Leo} (15.\,10.\,1866 Odessa – 2.\,7.\,1939 Riga), \emph{Gesangspädagoge, Mathematiker}|pw}, Bella\pwindex{Vengerova, Isabella 1.\,3.\,1877 Minsk – 7.\,2.\,1956 New York City@\textsc{Vengerova, Isabella} (1.\,3.\,1877 Minsk – 7.\,2.\,1956 New York City), \emph{Musikpädagogin, Pianistin}|pw}?\pend
           
\pstart
           Alles Herzliche von uns!{\\[\baselineskip]}\spacefill\mbox{Richard}\pend
           \leftskip=0em{}\selectlanguage{ngerman}\endnumbering\briefempfaengerindex{Schnitzler, Arthur@\textsc{Schnitzler, Arthur}!zzzBeer-Hofmann, Richard@\emph{von Richard Beer-Hofmann}!1914-08-101@{10. 8. 1914}|)be}\mylabel{L02191h}  \newcommand{\dateiname}{L02191}\newcommand{\titel}{Richard Beer-Hofmann an Arthur Schnitzler, 10. 8. 1914}\newcommand{\editorInnen}{Martin Anton Müller und Gerd-Hermann Susen}%% latex-leseansicht-abspann.tex
%% Abspann für die Leseansicht.
%% Der Schalter \ifkorrekturansicht ist bereits durch den Vorspann gesetzt.

%% latex-abspann.tex
%% Gemeinsamer Abspann für Korrekturansicht und Leseansicht.
%% Setzt den Schalter \ifkorrekturansicht voraus (gesetzt in den
%% einbindenden Dateien latex-korrekturansicht-abspann.tex bzw.
%% latex-leseansicht-abspann.tex).
%% ---------------------------------------------------------------

\normalsize

% Das esempio-Environment wird nur in der Leseansicht benötigt
\ifkorrekturansicht\else
\newenvironment{esempio}[3]%
{
    \vspace{1.5ex}
    \rlap{\underline{#1}}
    \par
    \setlength{\parindent}{0cm}
    \nopagebreak
    \leftskip=#2cm
    \rightskip=#3cm
}
{
    \par
}
\fi

\doendnotes{C}
\bigskip
\vfill

\clearpage

\footnotesize

\ifkorrekturansicht
  \lohead{\textsc{register}}
\fi

% theindex-Environment neu definieren ohne reledmac
\makeatletter
\renewenvironment{theindex}{%
  \ifkorrekturansicht
    \section*{\indexname}%
  \else
    \subsubsection*{Index der erwähnten Entitäten}%
  \fi
  \setlength{\parindent}{0pt}%
  \setlength{\parskip}{0pt plus 0.3pt}%
  \let\item\@idxitem
}{%
  \ifkorrekturansicht\clearpage\fi
}
\makeatother

\IfFileExists{\jobname-pw.ind}{\input{\jobname-pw.ind}}{}

% Quellenangabe nur in der Leseansicht
\ifkorrekturansicht\else
% Fallback-Definitionen, falls die .tex-Datei \titel etc. nicht gesetzt hat
\providecommand{\titel}{}
\providecommand{\editorInnen}{}
\providecommand{\dateiname}{\jobname}

\vspace{3cm}

\vfill

\footnotesize
\textsc{Quelle}: \titel. Herausgegeben von {\editorInnen}. In: \emph{Arthur Schnitzler: Briefwechsel mit Autorinnen und Autoren}.
 Digitale Edition, https://schnitzler-briefe.acdh.oeaw.ac.at/{\dateiname}.html (Stand \today)
\fi

\end{document}


