%% latex-leseansicht-vorspann.tex
%% Vorspann für die Leseansicht.
%% Lädt die gemeinsame Datei latex-vorspann.tex mit nicht gesetztem Schalter.

\newif\ifkorrekturansicht
\korrekturansichtfalse

\input{../tex-inputs/latex-vorspann}


               \section[Richard Beer-Hofmann an Arthur Schnitzler, 10. 8. 1914]{ Richard Beer-Hofmann an Arthur Schnitzler, 10. 8. 1914}\nopagebreak\mylabel{v}\rehead{ }\begin{ledgroupsized}[t]{13cm}\normalsize\beginnumbering\briefempfaengerindex{Schnitzler, Arthur@\textsc{Schnitzler, Arthur}!zzzBeer-Hofmann, Richard@\emph{von Richard Beer-Hofmann}!1914-08-101@{10. 8. 1914}|(be} \toendnotes[C]{\smallbreak\pagebreak[2]} \Standort{CUL, Schnitzler, B 8.}
\physDesc{Bildpostkarte
\newline{}Handschrift: Bleistift, lateinische Kurrent\newline{}Versand: 1) Stempel: »\nobreak{}\oindex{Weissenbach am Attersee@\textbf{Weißenbach am Attersee}|pwk}Weissenbach am Attersee, 1\textcolor{gray}{1}. VII. 14\nobreak{}«.  2) Stempel: »\nobreak{}\oindex{Celerina@\textbf{Celerina}|pwk}Celerina (Graubünden), 16. VII. 14, 1\nobreak{}«. 3) postalischer Nachsendevermerk: »Hotel Lattmann\oindex{Hotel Lattmann@\textbf{Hotel Lattmann}|pw}, Ragaz\oindex{Bad Ragaz@\textbf{Bad Ragaz}|pw}« 4) Stempel: »\nobreak{}\oindex{Bad Ragaz@\textbf{Bad Ragaz}|pwk}Ragaz, 17. VII. 14, 3\nobreak{}«. 5) postalischer Nachsendevermerk: »Wien XVIII\oindex{XVIII., Waehring@\textbf{XVIII., Währing}|pw}, Sternwartestr. 71\oindex{Sternwartestrasse@\textbf{Sternwartestraße}|pw}« \newline{}Ordnung: mit Bleistift von unbekannter Hand nummeriert:
                                    »259« }\buchAbdrucke{\weitereDrucke{Arthur Schnitzler, Richard Beer-Hofmann: \emph{Briefwechsel 1891–1931}. Hg. Konstanze Fliedl. Wien, Zürich: \emph{Europaverlag} 1992, S. 220.} }\toendnotes[C]{\smallbreak}\pstart{}{\pb}Herrn D\textsuperscript{r} Arthur Schnitzler\pend{}\pstart{}Schweiz\oindex{Schweiz@\textbf{Schweiz}|pw}\pend{}\pstart{}Celerina\oindex{Celerina@\textbf{Celerina}|pw}\pend{}\pstart{}Cresta Palace\oindex{Cresta Palace@\textbf{Cresta Palace}|pw}\pend{}{\bigskip}\pstart
           \noindent{}\centering{}{\pb}\textcolor{gray}{\textbf{Salzkammergut\oindex{Salzkammergut@\textbf{Salzkammergut}|pw}. Weissenbach am Attersee\oindex{Weissenbach am Attersee@\textbf{Weißenbach am Attersee}|pw}.}}\pend
           \pstart
           \raggedleft{}10/VIII. 14\pend
           \pstart
           Lieber Arthur! Ich war für zwei Tage – getrieben von Unruhe – in Wien\oindex{Wien@\textbf{Wien}|pw} und sah dass es zwecklos wäre \uline{jetzt} dorthin mit den Kindern\pwindex{Beer-Hofmann, Naemah 20.12.1898 – 10.11.1971@\textsc{Beer-Hofmann, Naëmah} (20.12.1898 – 10.11.1971)|pwv}\pwindex{Beer-Hofmann, Mirjam 04.09.1897 – 24.12.1984@\textsc{Beer-Hofmann, Mirjam} (04.09.1897 – 24.12.1984)|pwv}\pwindex{Beer-Hofmann, Gabriel 09.01.1901 – 24.03.1971@\textsc{Beer-Hofmann, Gabriel} (09.01.1901 – 24.03.1971), \emph{Schriftsteller, Filmagent}|pwv} zurückzugehen. So bleibe
               ich noch – wie lange? – hier. \uline{Zu} weit vom Schuss sein
               ist auch unerträglich. Was ists mit Kaufmann\pwindex{Kaufmann, Arthur 04.04.1872 – 25.07.1938@\textsc{Kaufmann, Arthur} (04.04.1872 – 25.07.1938), \emph{Rechtswissenschaftler, Privatgelehrte, Privatier}|pw}, Leo\pwindex{Van-Jung, Leo 15.10.1866 – 02.07.1939@\textsc{Van-Jung, Leo} (15.10.1866 – 02.07.1939), \emph{Gesangspädagoge, Mathematiker}|pw}, Bella\pwindex{Vengerova, Isabella 01.03.1877 – 07.02.1956@\textsc{Vengerova, Isabella} (01.03.1877 – 07.02.1956), \emph{Musikpädagogin, Pianistin}|pw}?\pend
           \pstart
           Alles Herzliche von uns!{\\[\baselineskip]}\spacefill\mbox{Richard}\pend
           \leftskip=0em{}\endnumbering\briefempfaengerindex{Schnitzler, Arthur@\textsc{Schnitzler, Arthur}!zzzBeer-Hofmann, Richard@\emph{von Richard Beer-Hofmann}!1914-08-101@{10. 8. 1914}|)be}\mylabel{h}\end{ledgroupsized}  \newcommand{\dateiname}{L02191}\newcommand{\titel}{Richard Beer-Hofmann an Arthur Schnitzler, 10. 8. 1914}\newcommand{\editorInnen}{Martin Anton Müller und Gerd-Hermann Susen}%% latex-leseansicht-abspann.tex
%% Abspann für die Leseansicht.
%% Der Schalter \ifkorrekturansicht ist bereits durch den Vorspann gesetzt.

%% latex-abspann.tex
%% Gemeinsamer Abspann für Korrekturansicht und Leseansicht.
%% Setzt den Schalter \ifkorrekturansicht voraus (gesetzt in den
%% einbindenden Dateien latex-korrekturansicht-abspann.tex bzw.
%% latex-leseansicht-abspann.tex).
%% ---------------------------------------------------------------

\normalsize

% Das esempio-Environment wird nur in der Leseansicht benötigt
\ifkorrekturansicht\else
\newenvironment{esempio}[3]%
{
    \vspace{1.5ex}
    \rlap{\underline{#1}}
    \par
    \setlength{\parindent}{0cm}
    \nopagebreak
    \leftskip=#2cm
    \rightskip=#3cm
}
{
    \par
}
\fi

\doendnotes{C}
\bigskip
\vfill

\clearpage

\footnotesize

\ifkorrekturansicht
  \lohead{\textsc{register}}
\fi

% theindex-Environment neu definieren ohne reledmac
\makeatletter
\renewenvironment{theindex}{%
  \ifkorrekturansicht
    \section*{\indexname}%
  \else
    \subsubsection*{Index der erwähnten Entitäten}%
  \fi
  \setlength{\parindent}{0pt}%
  \setlength{\parskip}{0pt plus 0.3pt}%
  \let\item\@idxitem
}{%
  \ifkorrekturansicht\clearpage\fi
}
\makeatother

\IfFileExists{\jobname-pw.ind}{\input{\jobname-pw.ind}}{}

% Quellenangabe nur in der Leseansicht
\ifkorrekturansicht\else
% Fallback-Definitionen, falls die .tex-Datei \titel etc. nicht gesetzt hat
\providecommand{\titel}{}
\providecommand{\editorInnen}{}
\providecommand{\dateiname}{\jobname}

\vspace{3cm}

\vfill

\footnotesize
\textsc{Quelle}: \titel. Herausgegeben von {\editorInnen}. In: \emph{Arthur Schnitzler: Briefwechsel mit Autorinnen und Autoren}.
 Digitale Edition, https://schnitzler-briefe.acdh.oeaw.ac.at/{\dateiname}.html (Stand \today)
\fi

\end{document}


      