%% latex-korrekturansicht-vorspann.tex
%% Vorspann für die Korrekturansicht.
%% Lädt die gemeinsame Datei latex-vorspann.tex mit gesetztem Schalter.

\newif\ifkorrekturansicht
\korrekturansichttrue

\input{../tex-inputs/latex-vorspann}


\section[Richard Beer-Hofmann an Arthur Schnitzler, 10. 8. 1914]{L02191 Richard Beer-Hofmann an Arthur Schnitzler, 10. 8. 1914}
\nopagebreak\mylabel{L02191v}
\rehead{ }\normalsize\beginnumbering\briefempfaengerindex{Schnitzler, Arthur@\textsc{Schnitzler, Arthur}!zzzBeer-Hofmann, Richard@\emph{von Richard Beer-Hofmann}!1914-08-101@{10. 8. 1914}|(be}
\toendnotes[C]{\smallbreak\pagebreak[2]}\Standort{CUL, Schnitzler, B 8.}
\physDesc{Bildpostkarte, 349 Zeichen
\newline{}Handschrift: Bleistift, lateinische Kurrent
\newline{}Versand: 1) Stempel: »\nobreak{}\oindex{Weissenbach am Attersee@\textbf{Weißenbach am Attersee}, \emph{A.ADM3}|pwk}Weissenbach am Attersee, 1\textcolor{gray}{1}. VIII. 14\nobreak{}«.   2) Stempel: »\nobreak{}\oindex{Celerina@\textbf{Celerina}, \emph{P.PPL}|pwk}Celerina (Graubünden), 16. VIII. 14, 1\nobreak{}«.  3) postalischer Nachsendevermerk: »Hotel Lattmann\oindex{Hotel Lattmann@\textbf{Hotel Lattmann}, \emph{Hotel (K.HTL)}|pw}, Ragaz\oindex{Bad Ragaz@\textbf{Bad Ragaz}, \emph{P.PPL}|pw}«  4) Stempel: »\nobreak{}\oindex{Bad Ragaz@\textbf{Bad Ragaz}, \emph{P.PPL}|pwk}Ragaz, 17. VIII. 14, 3\nobreak{}«.  5) postalischer Nachsendevermerk: »Wien XVIII\oindex{XVIII., Waehring@\textbf{XVIII., Währing}, \emph{A.ADM3}|pw}, Sternwartestr. 71\oindex{Sternwartestrasse 71@\textbf{Sternwartestraße 71}, \emph{Wohngebäude (K.WHS)}|pw}« 
\newline{}Ordnung: mit Bleistift von unbekannter Hand nummeriert:
                                    »259« }
\buchAbdrucke{\weitereDrucke{Arthur Schnitzler, Richard Beer-Hofmann: \emph{Briefwechsel 1891–1931}. Wien, Zürich: \emph{Europaverlag} 1992, S. 220.} }\toendnotes[C]{\smallbreak}\pstart{}{\pb}Herrn D\textsuperscript{r} Arthur Schnitzler\pend{}\pstart{}Schweiz\oindex{Schweiz@\textbf{Schweiz}, \emph{A.PCLI}|pw}\pend{}\pstart{}Celerina\oindex{Celerina@\textbf{Celerina}, \emph{P.PPL}|pw}\pend{}\pstart{}Cresta Palace\oindex{Cresta Palace@\textbf{Cresta Palace}, \emph{Hotel (K.HTL)}|pw}\pend{}{\bigskip}
\pstart
           \noindent{}\centering{}{\pb}\textcolor{gray}{\textbf{Salzkammergut\oindex{Salzkammergut@\textbf{Salzkammergut}, \emph{L.RGN}|pw}. Weissenbach am Attersee\oindex{Weissenbach am Attersee@\textbf{Weißenbach am Attersee}, \emph{A.ADM3}|pw}.}}\pend
           \vspace{1em}
\pstart
           \raggedleft{}{\pb}10/VIII. 14\pend
           \vspace{0.5em}
\pstart
           Lieber Arthur! Ich war für zwei Tage – getrieben von Unruhe – in Wien\oindex{Wien@\textbf{Wien}, \emph{A.ADM2}|pw} und sah dass es zwecklos wäre \uline{jetzt} dorthin mit den Kindern\pwindex{Beer-Hofmann, Naemah 20.12.1898 – 10.11.1971@\textsc{Beer-Hofmann, Naëmah} (20.12.1898 – 10.11.1971)|pwv}\pwindex{Beer-Hofmann, Mirjam 04.09.1897 – 24.12.1984@\textsc{Beer-Hofmann, Mirjam} (04.09.1897 – 24.12.1984)|pwv}\pwindex{Beer-Hofmann, Gabriel 09.01.1901 – 24.03.1971@\textsc{Beer-Hofmann, Gabriel} (09.01.1901 – 24.03.1971), \emph{Schriftsteller/Schriftstellerin, Filmagent/Filmagentin}|pwv} zurückzugehen. So
               bleibe ich noch – wie lange? – hier. \uline{Zu} weit vom
               Schuss sein ist auch unerträglich. Was ists mit Kaufmann\pwindex{Kaufmann, Arthur 04.04.1872 – 25.07.1938@\textsc{Kaufmann, Arthur} (04.04.1872 – 25.07.1938), \emph{Rechtswissenschaftler/Rechtswissenschaftlerin, Privatgelehrte/Privatgelehrte, Privatier/Privatière}|pw}, Leo\pwindex{Van-Jung, Leo 15.10.1866 – 02.07.1939@\textsc{Van-Jung, Leo} (15.10.1866 – 02.07.1939), \emph{Gesangspädagoge/Gesangspädagogin, Mathematiker/Mathematikerin}|pw}, Bella\pwindex{Vengerova, Isabella 01.03.1877 – 07.02.1956@\textsc{Vengerova, Isabella} (01.03.1877 – 07.02.1956), \emph{Musikpädagoge/Musikpädagogin, Pianist/Pianistin}|pw}?\pend
           
\pstart
           Alles Herzliche von uns!{\\[\baselineskip]}\spacefill\mbox{Richard}\pend
           \leftskip=0em{}\selectlanguage{ngerman}\endnumbering\briefempfaengerindex{Schnitzler, Arthur@\textsc{Schnitzler, Arthur}!zzzBeer-Hofmann, Richard@\emph{von Richard Beer-Hofmann}!1914-08-101@{10. 8. 1914}|)be}\mylabel{L02191h}  \normalsize

\doendnotes{C}
\bigskip
\vfill

\clearpage

\footnotesize

\lohead{\textsc{register}}

% Definiere theindex-Environment komplett neu ohne reledmac
\makeatletter
\renewenvironment{theindex}{%
  \section*{\indexname}%
  \setlength{\parindent}{0pt}%
  \setlength{\parskip}{0pt plus 0.3pt}%
  \let\item\@idxitem
}{%
  \clearpage
}
\makeatother

\IfFileExists{\jobname-pw.ind}{\input{\jobname-pw.ind}}{}

\end{document}

      