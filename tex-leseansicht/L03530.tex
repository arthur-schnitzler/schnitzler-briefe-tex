%% latex-leseansicht-vorspann.tex
%% Vorspann für die Leseansicht.
%% Lädt die gemeinsame Datei latex-vorspann.tex mit nicht gesetztem Schalter.

\newif\ifkorrekturansicht
\korrekturansichtfalse

\input{../tex-inputs/latex-vorspann}


\section[ Paul Goldmann an Olga Gussmann, 29. 4. [1901]]{L03530 Paul Goldmann an Olga Gussmann,  29. 4. [1901]}
\nopagebreak\mylabel{L03530v}
\rehead{ }\normalsize\beginnumbering\briefempfaengerindex{Schnitzler, Olga@\textsc{Schnitzler, Olga}!zzzGoldmann, Paul@\emph{von Paul Goldmann}!1901-04-291@{29. 4. [1901]}|(be}
\toendnotes[C]{\smallbreak\pagebreak[2]}
\correspDesc{Versand  durch Paul Goldmann am 29. 4. [1901] in Berlin
\newline{}Erhalt  durch Olga Gussmann im Zeitraum [30. 4. 1901
                  – 4. 5. 1901?] in Wien}\toendnotes[C]{\smallbreak}
\Standort{DLA, A:Schnitzler, HS.NZ85.1.5247.}
\physDesc{Brief, 1 Blatt, 3 Seiten, 971 Zeichen
\newline{}Handschrift: blaue Tinte, deutsche Kurrent}\toendnotes[C]{\smallbreak}
\pstart
           \raggedleft{}{\pb}\textcolor{gray}{\textbf{DESSAUERSTRASSE 19\oindex{Dessauer Straße@\textbf{Dessauer Straße}, \emph{Straße}|pw}}}\pend
           
\pstart
           Berlin\oindex{Berlin@\textbf{Berlin}, \emph{Hauptstadt}|pw}, 29. April.\pend
           
\pstart\center{}Liebes Fräulein \textsc{Olga},\pend\vspace{0.5em}
\pstart
           Ich habe heut{ }ſehr wenig Zeit und kann Ihnen nur in
               Eile für Ihren Brief danken und Ihnen die Hand drücken. Sicherlich haben Sie einen
               großen \label{K_L03530-1v}\edtext{Erfolg}{\lemma{\textnormal{\emph{Erfolg}}}\Cendnote{\textnormal{Am 28. 4. 1901 trat Olga Gussmann\pwindex{Schnitzler, Olga 17.\,1.\,1882 Wien – 13.\,1.\,1970 Lugano@\textsc{Schnitzler, Olga} (17.\,1.\,1882 Wien – 13.\,1.\,1970 Lugano), \emph{Schauspielerin, Sängerin}|pwk}
                  in einer Schulvorstellung des Konservatoriums\oindex{Wien@\textbf{Wien}!I., Innere Stadt@\textbf{I., Innere Stadt}!Konservatorium der Gesellschaft der Musikfreunde@\textbf{Konservatorium der Gesellschaft der Musikfreunde}, \emph{Konservatorium}|pwk} in
                     Friedrich Hebbels\pwindex{Hebbel, Friedrich 18.\,3.\,1813 Wesselburen – 13.\,12.\,1863 Wien@\textsc{Hebbel, Friedrich} (18.\,3.\,1813 Wesselburen – 13.\,12.\,1863 Wien), \emph{Schriftsteller}|pwk}{ }\emph{Maria Magdalena}\pwindex{Hebbel, Friedrich 18.\,3.\,1813 Wesselburen – 13.\,12.\,1863 Wien@\textsc{Hebbel, Friedrich} (18.\,3.\,1813 Wesselburen – 13.\,12.\,1863 Wien), \emph{Schriftsteller}!Maria Magdalena. Ein bürgerliches Trauerspiel in drei Akten@\strich\emph{Maria Magdalena. Ein bürgerliches Trauerspiel in drei Akten}|pwk} auf. Siehe XXXX Auszeichnungsfehler: Dokument L01110 nicht gefunden.}}}\label{K_L03530-1} gehabt. Ich erwarte bald Bericht.
               Schicken Sie mir, bitte, auch einige \label{K_L03530-2v}\edtext{Zeitungsausſchnitte}{\lemma{\textnormal{\emph{Zeitungsausschnitte}}}\Cendnote{\textnormal{Vgl. XXXX Auszeichnungsfehler: Dokument L03527 nicht gefunden.
               }}}\label{K_L03530-2}. Hätte man nicht ein Referat in der N. Fr.
                  Pr.\pwindex{Neue Freie Presse@\emph{Neue Freie Presse}|pw} veranlaſſen können? Warum haben Sie mir nicht \substVorne{}\textsuperscript{vorher}\substDazwischen{}vorher\substHinten{} geſchrieben?\pend
           
\pstart
           {\pb}Über \label{K_L03530-3v}\edtext{\textsc{Salten\pwindex{Salten, Felix 6.\,9.\,1869 Budapest – 8.\,10.\,1945 Zürich@\textsc{Salten, Felix} (6.\,9.\,1869 Budapest – 8.\,10.\,1945 Zürich), \emph{Schriftsteller, Journalist, Chefredakteur}|pw}}}{\lemma{\textnormal{\emph{Salten}}}\Cendnote{\textnormal{Hatte dieser eine Besprechung der
                  Aufführung abgelehnt? Überraschend, aber möglich, wäre ein Bezug auf das im
                  Entstehen begriffene \emph{Jung-Wiener Theater zum
                     lieben Augustin}\orgindex{Jung-Wiener Theater zum Lieben Augustin@Jung-Wiener Theater zum Lieben Augustin|pwk}, vgl. XXXX Auszeichnungsfehler: Dokument L03067 nicht gefunden. }}}\label{K_L03530-3} bin ich ganz Ihrer Anſicht.\pend
           
\pstart
           Ob ich einen \label{K_L03530-4v}\edtext{Theil des Sommers mit
               Ihnen verbringen}{\lemma{\textnormal{\emph{Theil … verbringen}}}\Cendnote{\textnormal{Siehe XXXX Auszeichnungsfehler: Dokument L03064 nicht gefunden.
               }}}\label{K_L03530-4} werde, weiß ich noch nicht. Ich hätte Luſt, mich in ein{ }ſehr wildes Land{ }ſchicken zu laſſen, weit, weit weg.\pend
           
\pstart
           Daß ihre Schweſter \textsc{Liesl\pwindex{Steinrück, Elisabeth 19.\,11.\,1885 – 7.\,4.\,1920 Partenkirchen@\textsc{Steinrück, Elisabeth} (19.\,11.\,1885 – 7.\,4.\,1920 Partenkirchen)|pw}} meinen Brief noch immer nicht beantwortet hat, iſt ganz einfach empörend. Sagen
               Sie, bitte, dieſem jungen Geſchöpf\pwindex{Steinrück, Elisabeth 19.\,11.\,1885 – 7.\,4.\,1920 Partenkirchen@\textsc{Steinrück, Elisabeth} (19.\,11.\,1885 – 7.\,4.\,1920 Partenkirchen)|pwv}, daß ich{ }ſie zur Erbin meines ungeheuren Vermögens eingeſetzt ha\substVorne{}\textsuperscript{\textcolor{gray}{be}}\substDazwischen{}tte\substHinten{}, daß ich{ }ſie aber infolge ihres pietätloſen Verhaltens wieder {\pb}aus meinem Teſtament geſtrichen habe.\pend
           
\pstart
           Herzliche Grüße an Sie \strikeout{Beide}\pwindex{Steinrück, Elisabeth 19.\,11.\,1885 – 7.\,4.\,1920 Partenkirchen@\textsc{Steinrück, Elisabeth} (19.\,11.\,1885 – 7.\,4.\,1920 Partenkirchen)|pwv}{ }Beide\pwindex{Steinrück, Elisabeth 19.\,11.\,1885 – 7.\,4.\,1920 Partenkirchen@\textsc{Steinrück, Elisabeth} (19.\,11.\,1885 – 7.\,4.\,1920 Partenkirchen)|pwv} und an Herrn \textsc{Paul\pwindex{Marx, Paul 4.\,6.\,1861 Breslau – 27.\,11.\,1919 Berlin@\textsc{Marx, Paul} (4.\,6.\,1861 Breslau – 27.\,11.\,1919 Berlin), \emph{Journalist, Kritiker}|pw}} von {\\[\baselineskip]}Ihrem ergebenen {\\[\baselineskip]}\spacefill\mbox{Dr. Paul Goldmann.}\pend
           \leftskip=0em{}\selectlanguage{ngerman}\endnumbering\briefempfaengerindex{Schnitzler, Olga@\textsc{Schnitzler, Olga}!zzzGoldmann, Paul@\emph{von Paul Goldmann}!1901-04-291@{29. 4. [1901]}|)be}\mylabel{L03530h}  \newcommand{\dateiname}{L03530}\newcommand{\titel}{Paul Goldmann an Olga Gussmann, 29. 4. [1901]}\newcommand{\editorInnen}{Martin Anton Müller und Laura Untner}%% latex-leseansicht-abspann.tex
%% Abspann für die Leseansicht.
%% Der Schalter \ifkorrekturansicht ist bereits durch den Vorspann gesetzt.

%% latex-abspann.tex
%% Gemeinsamer Abspann für Korrekturansicht und Leseansicht.
%% Setzt den Schalter \ifkorrekturansicht voraus (gesetzt in den
%% einbindenden Dateien latex-korrekturansicht-abspann.tex bzw.
%% latex-leseansicht-abspann.tex).
%% ---------------------------------------------------------------

\normalsize

% Das esempio-Environment wird nur in der Leseansicht benötigt
\ifkorrekturansicht\else
\newenvironment{esempio}[3]%
{
    \vspace{1.5ex}
    \rlap{\underline{#1}}
    \par
    \setlength{\parindent}{0cm}
    \nopagebreak
    \leftskip=#2cm
    \rightskip=#3cm
}
{
    \par
}
\fi

\doendnotes{C}
\bigskip
\vfill

\clearpage

\footnotesize

\ifkorrekturansicht
  \lohead{\textsc{register}}
\fi

% theindex-Environment neu definieren ohne reledmac
\makeatletter
\renewenvironment{theindex}{%
  \ifkorrekturansicht
    \section*{\indexname}%
  \else
    \subsubsection*{Index der erwähnten Entitäten}%
  \fi
  \setlength{\parindent}{0pt}%
  \setlength{\parskip}{0pt plus 0.3pt}%
  \let\item\@idxitem
}{%
  \ifkorrekturansicht\clearpage\fi
}
\makeatother

\IfFileExists{\jobname-pw.ind}{\input{\jobname-pw.ind}}{}

% Quellenangabe nur in der Leseansicht
\ifkorrekturansicht\else
% Fallback-Definitionen, falls die .tex-Datei \titel etc. nicht gesetzt hat
\providecommand{\titel}{}
\providecommand{\editorInnen}{}
\providecommand{\dateiname}{\jobname}

\vspace{3cm}

\vfill

\footnotesize
\textsc{Quelle}: \titel. Herausgegeben von {\editorInnen}. In: \emph{Arthur Schnitzler: Briefwechsel mit Autorinnen und Autoren}.
 Digitale Edition, https://schnitzler-briefe.acdh.oeaw.ac.at/{\dateiname}.html (Stand \today)
\fi

\end{document}


