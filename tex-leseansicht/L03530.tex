%% latex-korrekturansicht-vorspann.tex
%% Vorspann für die Korrekturansicht.
%% Lädt die gemeinsame Datei latex-vorspann.tex mit gesetztem Schalter.

\newif\ifkorrekturansicht
\korrekturansichttrue

\input{../tex-inputs/latex-vorspann}


\section[ Paul Goldmann an Olga Gussmann, 29. 4. {[}1901{]}]{L03530 Paul Goldmann an Olga Gussmann, 29. 4. {[}1901{]}}
\nopagebreak\mylabel{L03530v}
\rehead{ }\normalsize\beginnumbering\briefempfaengerindex{Schnitzler, Olga@\textsc{Schnitzler, Olga}!zzzGoldmann, Paul@\emph{von Paul Goldmann}!1901-04-291@{29. 4. {[}1901{]}}|(be}
\toendnotes[C]{\smallbreak\pagebreak[2]}\Standort{DLA, A:Schnitzler, HS.NZ85.1.5247.}
\physDesc{Brief, 1 Blatt, 3 Seiten, 971 Zeichen
\newline{}Handschrift: blaue Tinte, deutsche Kurrent}\toendnotes[C]{\smallbreak}
\pstart
           \raggedleft{}{\pb}\textcolor{gray}{\textbf{DESSAUERSTRASSE 19\oindex{Dessauer Strasse@\textbf{Dessauer Straße}, \emph{Straße (K.STR)}|pw}}}\pend
           
\pstart
           Berlin\oindex{Berlin@\textbf{Berlin}, \emph{P.PPLC}|pw}, 29. April.\pend
           
\pstart\center{}Liebes Fräulein \textsc{Olga},\pend\vspace{0.5em}
\pstart
           Ich habe heut ſehr wenig Zeit und kann Ihnen nur in
               Eile für Ihren Brief danken und Ihnen die Hand drücken. Sicherlich haben Sie einen
               großen \label{K_L03530-1v}\edtext{Erfolg}{\lemma{\textnormal{\emph{Erfolg}}}\Cendnote{\textnormal{Am 28. 4. 1901 trat Olga Gussmann\pwindex{Schnitzler, Olga 17.01.1882 – 13.01.1970@\textsc{Schnitzler, Olga} (17.01.1882 – 13.01.1970), \emph{Schauspieler/Schauspielerin, Sänger/Sängerin}|pwk}
                  in einer Schulvorstellung des Konservatoriums\oindex{Konservatorium der Gesellschaft der Musikfreunde@\textbf{Konservatorium der Gesellschaft der Musikfreunde}, \emph{Konservatorium (K.KON)}|pwk} in
                     Friedrich Hebbels\pwindex{Hebbel, Friedrich 18.03.1813 – 13.12.1863@\textsc{Hebbel, Friedrich} (18.03.1813 – 13.12.1863), \emph{Schriftsteller/Schriftstellerin}|pwk}{ }\emph{Maria Magdalena}\pwindex{Maria Magdalena. Ein buergerliches Trauerspiel in drei Akten@\emph{Maria Magdalena. Ein bürgerliches Trauerspiel in drei Akten}|pwk} auf. Siehe Arthur Schnitzler an Hermann Bahr, 19. 4. 1901.}}}\label{K_L03530-1} gehabt. Ich erwarte bald Bericht.
               Schicken Sie mir, bitte, auch einige \label{K_L03530-2v}\edtext{Zeitungsausſchnitte}{\lemma{\textnormal{\emph{Zeitungsausſchnitte}}}\Cendnote{\textnormal{Vgl. Paul Goldmann an Olga Gussmann, 10. 5. [1901].
               }}}\label{K_L03530-2}. Hätte man nicht ein Referat in der N. Fr.
                  Pr.\pwindex{Neue Freie Presse@\emph{Neue Freie Presse}|pw} veranlaſſen können? Warum haben Sie mir nicht \substVorne{}\textsuperscript{vorher}\substDazwischen{}vorher\substHinten{} geſchrieben?\pend
           
\pstart
           {\pb}Über \label{K_L03530-3v}\edtext{\textsc{Salten\pwindex{Salten, Felix 06.09.1869 – 08.10.1945@\textsc{Salten, Felix} (06.09.1869 – 08.10.1945), \emph{Schriftsteller/Schriftstellerin, Journalist/Journalistin, Chefredakteur/Chefredakteurin}|pw}}}{\lemma{\textnormal{\emph{Salten}}}\Cendnote{\textnormal{Hatte dieser eine Besprechung der
                  Aufführung abgelehnt? Überraschend, aber möglich, wäre ein Bezug auf das im
                  Entstehen begriffene \emph{Jung-Wiener Theater zum
                     lieben Augustin}\orgindex{Jung-Wiener Theater zum Lieben Augustin@Jung-Wiener Theater zum Lieben Augustin|pwk}, vgl. Paul Goldmann an Arthur Schnitzler, 16. 5. [1901]. }}}\label{K_L03530-3} bin ich ganz Ihrer Anſicht.\pend
           
\pstart
           Ob ich einen \label{K_L03530-4v}\edtext{Theil des Sommers mit
               Ihnen verbringen}{\lemma{\textnormal{\emph{Theil … verbringen}}}\Cendnote{\textnormal{Siehe Paul Goldmann an Arthur Schnitzler, 26. 4. [1901].
               }}}\label{K_L03530-4} werde, weiß ich noch nicht. Ich hätte Luſt, mich in ein ſehr wildes Land
               ſchicken zu laſſen, weit, weit weg.\pend
           
\pstart
           Daß ihre Schweſter \textsc{Liesl\pwindex{Steinrueck, Elisabeth 19.11.1885 – 07.04.1920@\textsc{Steinrück, Elisabeth} (19.11.1885 – 07.04.1920)|pw}} meinen Brief noch immer nicht beantwortet hat, iſt ganz einfach empörend. Sagen
               Sie, bitte, dieſem jungen Geſchöpf\pwindex{Steinrueck, Elisabeth 19.11.1885 – 07.04.1920@\textsc{Steinrück, Elisabeth} (19.11.1885 – 07.04.1920)|pwv}, daß ich ſie zur Erbin meines ungeheuren Vermögens eingeſetzt ha\substVorne{}\textsuperscript{\textcolor{gray}{be}}\substDazwischen{}tte\substHinten{}, daß ich ſie aber infolge ihres pietätloſen Verhaltens wieder {\pb}aus meinem Teſtament geſtrichen habe.\pend
           
\pstart
           Herzliche Grüße an Sie \strikeout{Beide}\pwindex{Steinrueck, Elisabeth 19.11.1885 – 07.04.1920@\textsc{Steinrück, Elisabeth} (19.11.1885 – 07.04.1920)|pwv}{ }Beide\pwindex{Steinrueck, Elisabeth 19.11.1885 – 07.04.1920@\textsc{Steinrück, Elisabeth} (19.11.1885 – 07.04.1920)|pwv} und an Herrn \textsc{Paul\pwindex{Marx, Paul 04.06.1861 – 27.11.1919@\textsc{Marx, Paul} (04.06.1861 – 27.11.1919), \emph{Journalist/Journalistin, Kritiker/Kritikerin}|pw}} von {\\[\baselineskip]}Ihrem ergebenen {\\[\baselineskip]}\spacefill\mbox{Dr. Paul Goldmann.}\pend
           \leftskip=0em{}\selectlanguage{ngerman}\endnumbering\briefempfaengerindex{Schnitzler, Olga@\textsc{Schnitzler, Olga}!zzzGoldmann, Paul@\emph{von Paul Goldmann}!1901-04-291@{29. 4. {[}1901{]}}|)be}\mylabel{L03530h}  \normalsize

\doendnotes{C}
\bigskip
\vfill

\clearpage

\footnotesize

\lohead{\textsc{register}}

% Definiere theindex-Environment komplett neu ohne reledmac
\makeatletter
\renewenvironment{theindex}{%
  \section*{\indexname}%
  \setlength{\parindent}{0pt}%
  \setlength{\parskip}{0pt plus 0.3pt}%
  \let\item\@idxitem
}{%
  \clearpage
}
\makeatother

\IfFileExists{\jobname-pw.ind}{\input{\jobname-pw.ind}}{}

\end{document}

      