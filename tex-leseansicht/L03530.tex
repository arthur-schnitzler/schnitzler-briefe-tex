%% latex-leseansicht-vorspann.tex
%% Vorspann für die Leseansicht.
%% Lädt die gemeinsame Datei latex-vorspann.tex mit nicht gesetztem Schalter.

\newif\ifkorrekturansicht
\korrekturansichtfalse

\input{../tex-inputs/latex-vorspann}

\begin{center}
            \textcolor{red}{ENTWURF, NICHT FERTIG KORRIGIERT}
                      \end{center}
            
         
         \renewcommand{\erwaehntePersonen}{Personen: Olga Schnitzler}
         \renewcommand{\erwaehnteOrte}{Orte: Berlin, Wien}
         \renewcommand{\erwaehnteWerke}{}
               \section[ Paul Goldmann an Olga XXXX Gussmann/Schnitzler, 29. 4. {[}XXXX{]}]{ Paul Goldmann an Olga XXXX Gussmann/Schnitzler, 29. 4. {[}XXXX{]}}\nopagebreak\mylabel{v}\rehead{ }\begin{ledgroupsized}[t]{13cm}\normalsize\beginnumbering \toendnotes[C]{\smallbreak\pagebreak[2]} \Standort{DLA, A:Schnitzler, HS.1985.1.5247.}
\physDesc{,  Blätter,  Seiten
\newline{}Handschrift: , deutsche Kurrent}\toendnotes[C]{\smallbreak}{\pb}\textcolor{gray}{\textbf{DESSAUERSTRASSE 19\oindex{XXXX Ortsangabe fehlt|pw}}}\textcolor{red}{\textsuperscript{\textbf{KEY}}}\pstart
           Berlin\oindex{Berlin@\textbf{Berlin}|pw}, 29. April. 29.
                     April.\pend
           \pstart{}Liebes Fräulein \textsc{Olga},\pend\pstart
           \pend
           \pstart
           Ich habe heut ſehr wenig Zeit und kann Ihnen nur in Eile für Ihren Brief
               danken und Ihnen die Hand drücken. Sicherlich haben Sie einen großen Erfolg gehabt.
               Ich erwarte bald Bericht. Schicken Sie mir, bitte, auch einige Zeitungsausſchnitte.
               Hätte man nicht ein Referat in der N. Fr. Pr.\textcolor{red}{\textsuperscript{\textbf{KEY}}} veranlaſſen
               können? Waum haben Sie vorher mir nicht \strikeout{vorher} geſchrieben? {\pb}\pend
           \pstart
           Über \textsc{Salten\textcolor{red}{\textsuperscript{\textbf{KEY}}}} bin ich ganz Ihrer Anſicht.\pend
           \pstart
           Ob ich einen Theil des Sommers mit Ihnen verbringen werde, weiß ich noch nicht. Ich
               hätte Luſt, mich in ein ſehr wildes Land ſchicken zu laſſen, weit, weit weg.\pend
           \pstart
           Daß ihre Schweſter \textsc{Liesl\pwindex{\textcolor{red}{\textsuperscript{XXXX1 indx}}|pw}} meinen Brief noch immer nicht beantwortet hat, iſt ganz einfach empörend. Sagen
               Sie, bitte, dieſem jungen Geſchöpf\textcolor{red}{\textsuperscript{\textbf{KEY}}}, daß ich ſie zur
               Erbin meines ungeheuren Vermögens eingeſetzt hatte, daß ich ſie aber infolge ihres
               pietätloſen Verhaltens wieder {\pb}
               aus meinem Teſtament geſtrichen habe. {\\[\baselineskip]}Herzliche Grüße an Sie \strikeout{Beide}\textcolor{red}{\textsuperscript{\textbf{KEY}}}\pend
           \leftskip=0em{}\pstart
           {\\[\baselineskip]}Beide\textcolor{red}{\textsuperscript{\textbf{KEY}}} und an Herrn \textsc{Paul\textcolor{red}{\textsuperscript{\textbf{KEY}}}} von\pend
           \leftskip=0em{}\pstart
           {\\[\baselineskip]}Ihrem ergebenen\pend
           \leftskip=0em{}\pstart
           {\\[\baselineskip]}\spacefill\mbox{Dr. Paul Goldmann.}\pend
           \leftskip=0em{}
         
         \endnumbering\mylabel{h}\end{ledgroupsized}\begin{anhang}\end{anhang}\newcommand{\dateiname}{L03530}\newcommand{\titel}{Paul Goldmann an Olga XXXX Gussmann/Schnitzler, 29. 4. [XXXX]}\newcommand{\editorInnen}{Martin Anton Müller und Laura Untner}%% latex-leseansicht-abspann.tex
%% Abspann für die Leseansicht.
%% Der Schalter \ifkorrekturansicht ist bereits durch den Vorspann gesetzt.

%% latex-abspann.tex
%% Gemeinsamer Abspann für Korrekturansicht und Leseansicht.
%% Setzt den Schalter \ifkorrekturansicht voraus (gesetzt in den
%% einbindenden Dateien latex-korrekturansicht-abspann.tex bzw.
%% latex-leseansicht-abspann.tex).
%% ---------------------------------------------------------------

\normalsize

% Das esempio-Environment wird nur in der Leseansicht benötigt
\ifkorrekturansicht\else
\newenvironment{esempio}[3]%
{
    \vspace{1.5ex}
    \rlap{\underline{#1}}
    \par
    \setlength{\parindent}{0cm}
    \nopagebreak
    \leftskip=#2cm
    \rightskip=#3cm
}
{
    \par
}
\fi

\doendnotes{C}
\bigskip
\vfill

\clearpage

\footnotesize

\ifkorrekturansicht
  \lohead{\textsc{register}}
\fi

% theindex-Environment neu definieren ohne reledmac
\makeatletter
\renewenvironment{theindex}{%
  \ifkorrekturansicht
    \section*{\indexname}%
  \else
    \subsubsection*{Index der erwähnten Entitäten}%
  \fi
  \setlength{\parindent}{0pt}%
  \setlength{\parskip}{0pt plus 0.3pt}%
  \let\item\@idxitem
}{%
  \ifkorrekturansicht\clearpage\fi
}
\makeatother

\IfFileExists{\jobname-pw.ind}{\input{\jobname-pw.ind}}{}

% Quellenangabe nur in der Leseansicht
\ifkorrekturansicht\else
% Fallback-Definitionen, falls die .tex-Datei \titel etc. nicht gesetzt hat
\providecommand{\titel}{}
\providecommand{\editorInnen}{}
\providecommand{\dateiname}{\jobname}

\vspace{3cm}

\vfill

\footnotesize
\textsc{Quelle}: \titel. Herausgegeben von {\editorInnen}. In: \emph{Arthur Schnitzler: Briefwechsel mit Autorinnen und Autoren}.
 Digitale Edition, https://schnitzler-briefe.acdh.oeaw.ac.at/{\dateiname}.html (Stand \today)
\fi

\end{document}


      