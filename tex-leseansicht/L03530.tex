%% latex-leseansicht-vorspann.tex
%% Vorspann für die Leseansicht.
%% Lädt die gemeinsame Datei latex-vorspann.tex mit nicht gesetztem Schalter.

\newif\ifkorrekturansicht
\korrekturansichtfalse

\input{../tex-inputs/latex-vorspann}

\begin{center}
            \textcolor{red}{ENTWURF, NICHT FERTIG KORRIGIERT}
                      \end{center}
            
         
         \renewcommand{\erwaehntePersonen}{Personen: Paul Goldmann, Friedrich Hebbel, Paul Marx, Felix Salten, Olga Schnitzler, Elisabeth Steinrück}
         \renewcommand{\erwaehnteInstitutionen}{Institutionen: Jung-Wiener Theater zum Lieben Augustin}
         \renewcommand{\erwaehnteOrte}{Orte: Berlin, Dessauer Straße, Konservatorium der Gesellschaft der Musikfreunde, Wien}
         \renewcommand{\erwaehnteWerke}{Werke: Maria Magdalena. Ein bürgerliches Trauerspiel in drei Akten, Neue Freie Presse}
               \section[ Paul Goldmann an Olga Gussmann, 29. 4. {[}1901{]}]{ Paul Goldmann an Olga Gussmann, 29. 4. {[}1901{]}}\nopagebreak\mylabel{v}\rehead{ }\begin{ledgroupsized}[t]{13cm}\normalsize\beginnumbering\briefempfaengerindex{Schnitzler, Olga@\textsc{Schnitzler, Olga}!zzzGoldmann, Paul@\emph{von Paul Goldmann}!1901-04-291@{29. 4. {[}1901{]}}|(be} \toendnotes[C]{\smallbreak\pagebreak[2]} \Standort{DLA, A:Schnitzler, HS.NZ85.1.5247.}
\physDesc{Brief, 1 Blatt, 3 Seiten, 969 Zeichen
\newline{}Handschrift: blaue Tinte, deutsche Kurrent}\toendnotes[C]{\smallbreak}\pstart
           \noindent{}\raggedleft{}{\pb}\textcolor{gray}{\textbf{DESSAUERSTRASSE 19\oindex{Dessauer Strasse@\textbf{Dessauer Straße}|pw}}}\pend
           \pstart
           Berlin\oindex{Berlin@\textbf{Berlin}|pw}, 29. April.\pend
           \pstart\center{}Liebes Fräulein \textsc{Olga},\pend\pstart
           Ich habe heut ſehr wenig Zeit und kann Ihnen nur in
               Eile für Ihren Brief danken und Ihnen die Hand drücken. Sicherlich haben Sie einen
               großen \label{K_L03530-1v}\edtext{Erfolg}{\lemma{\textnormal{\emph{Erfolg}}}\Cendnote{\textnormal{Am 28. 4. 1901 hatte Olga Gussmann\pwindex{Schnitzler, Olga 17.01.1882 – 13.01.1970@\textsc{Schnitzler, Olga} (17.01.1882 – 13.01.1970), \emph{Schauspielerin, Sängerin}|pwk}
                  einen Auftritt bei einer Vorstellung des Konservatorium\oindex{Konservatorium der Gesellschaft der Musikfreunde@\textbf{Konservatorium der Gesellschaft der Musikfreunde}|pwk}s von Friedrich
                     Hebbel\pwindex{Hebbel, Friedrich 18.03.1813 – 13.12.1863@\textsc{Hebbel, Friedrich} (18.03.1813 – 13.12.1863), \emph{Schriftsteller}|pwk}s \emph{Maria Magdalena}\pwindex{Hebbel, Friedrich 18.03.1813 – 13.12.1863@\textsc{Hebbel, Friedrich} (18.03.1813 – 13.12.1863), \emph{Schriftsteller}!Maria Magdalena. Ein buergerliches Trauerspiel in drei Akten1844@\strich\emph{Maria Magdalena. Ein bürgerliches Trauerspiel in drei Akten} {[}1844{]}|pwk} gehabt. Siehe Arthur Schnitzler an Hermann Bahr, 19. 4. 1901.}}}\label{K_L03530-1h} gehabt. Ich
               erwarte bald Bericht. Schicken Sie mir, bitte, auch einige \label{K_L03530-2v}\edtext{Zeitungsausſchnitte}{\lemma{\textnormal{\emph{Zeitungsausſchnitte}}}\Cendnote{\textnormal{siehe Paul Goldmann an Olga Gussmann, 10. 5. [1901]}}}\label{K_L03530-2h}. Hätte man nicht ein Referat in der N. Fr.
                  Pr.\pwindex{Neue Freie Presse1864 – 1939@\emph{Neue Freie Presse} {[}1864 – 1939{]}|pw} veranlaſſen können? Warum haben Sie mir nicht \substVorne{}\textsuperscript{vorher}{\allowbreak}\substDazwischen{}vorher\substHinten{} geſchrieben?\pend
           \pstart
           {\pb}Über \label{K_L03530-3v}\edtext{\textsc{Salten\pwindex{Salten, Felix 06.09.1869 – 08.10.1945@\textsc{Salten, Felix} (06.09.1869 – 08.10.1945), \emph{Schriftsteller, Journalist}|pw}}}{\lemma{\textnormal{\emph{Salten}}}\Cendnote{\textnormal{womöglich Bezug auf das \emph{Jung-Wiener Theater zum lieben Augustin}\orgindex{Jung-Wiener Theater zum Lieben Augustin@Jung-Wiener Theater zum Lieben Augustin|pwk}, siehe Paul Goldmann an Arthur Schnitzler, 16. 5. [1901]}}}\label{K_L03530-3h} bin ich ganz Ihrer Anſicht.\pend
           \pstart
           Ob ich einen \label{K_L03530-4v}\edtext{Theil des Sommers mit
               Ihnen verbringen}{\lemma{\textnormal{\emph{Theil … verbringen}}}\Cendnote{\textnormal{siehe Paul Goldmann an Olga Gussmann, 10. 5. [1901]}}}\label{K_L03530-4h} werde, weiß ich noch nicht. Ich hätte Luſt, mich in ein ſehr wildes Land
               ſchicken zu laſſen, weit, weit weg.\pend
           \pstart
           Daß ihre Schweſter \textsc{Liesl\pwindex{Steinrueck, Elisabeth 19.11.1885 – 07.04.1920@\textsc{Steinrück, Elisabeth} (19.11.1885 – 07.04.1920)|pw}} meinen Brief noch immer nicht beantwortet hat, iſt ganz einfach empörend. Sagen
               Sie, bitte, dieſem jungen \label{K_L03530-5v}\edtext{Geſchöpf\pwindex{Steinrueck, Elisabeth 19.11.1885 – 07.04.1920@\textsc{Steinrück, Elisabeth} (19.11.1885 – 07.04.1920)|pwv}}{\lemma{\textnormal{\emph{Geſchöpf}}}\Cendnote{\textnormal{Das war humorvoll gemeint.}}}\label{K_L03530-5h}, daß
               ich ſie zur Erbin meines ungeheuren Vermögens eingeſetzt hatte, daß ich ſie aber
               infolge ihres pietätloſen Verhaltens wieder {\pb}aus
               meinem Teſtament geſtrichen habe.\pend
           \pstart
           Herzliche Grüße an Sie \strikeout{Beide}\pwindex{Steinrueck, Elisabeth 19.11.1885 – 07.04.1920@\textsc{Steinrück, Elisabeth} (19.11.1885 – 07.04.1920)|pwv}{ }Beide\pwindex{Steinrueck, Elisabeth 19.11.1885 – 07.04.1920@\textsc{Steinrück, Elisabeth} (19.11.1885 – 07.04.1920)|pwv} und an Herrn \textsc{Paul\pwindex{Marx, Paul 04.06.1861 – 27.11.1919@\textsc{Marx, Paul} (04.06.1861 – 27.11.1919), \emph{Journalist, Kritiker}|pw}} von {\\[\baselineskip]}Ihrem ergebenen {\\[\baselineskip]}\spacefill\mbox{Dr. Paul Goldmann.}\pend
           \leftskip=0em{}
         
         \endnumbering\mylabel{h}\end{ledgroupsized}\begin{anhang}\end{anhang}\newcommand{\dateiname}{L03530}\newcommand{\titel}{Paul Goldmann an Olga Gussmann, 29. 4. [1901]}\newcommand{\editorInnen}{Martin Anton Müller und Laura Untner}%% latex-leseansicht-abspann.tex
%% Abspann für die Leseansicht.
%% Der Schalter \ifkorrekturansicht ist bereits durch den Vorspann gesetzt.

%% latex-abspann.tex
%% Gemeinsamer Abspann für Korrekturansicht und Leseansicht.
%% Setzt den Schalter \ifkorrekturansicht voraus (gesetzt in den
%% einbindenden Dateien latex-korrekturansicht-abspann.tex bzw.
%% latex-leseansicht-abspann.tex).
%% ---------------------------------------------------------------

\normalsize

% Das esempio-Environment wird nur in der Leseansicht benötigt
\ifkorrekturansicht\else
\newenvironment{esempio}[3]%
{
    \vspace{1.5ex}
    \rlap{\underline{#1}}
    \par
    \setlength{\parindent}{0cm}
    \nopagebreak
    \leftskip=#2cm
    \rightskip=#3cm
}
{
    \par
}
\fi

\doendnotes{C}
\bigskip
\vfill

\clearpage

\footnotesize

\ifkorrekturansicht
  \lohead{\textsc{register}}
\fi

% theindex-Environment neu definieren ohne reledmac
\makeatletter
\renewenvironment{theindex}{%
  \ifkorrekturansicht
    \section*{\indexname}%
  \else
    \subsubsection*{Index der erwähnten Entitäten}%
  \fi
  \setlength{\parindent}{0pt}%
  \setlength{\parskip}{0pt plus 0.3pt}%
  \let\item\@idxitem
}{%
  \ifkorrekturansicht\clearpage\fi
}
\makeatother

\IfFileExists{\jobname-pw.ind}{\input{\jobname-pw.ind}}{}

% Quellenangabe nur in der Leseansicht
\ifkorrekturansicht\else
% Fallback-Definitionen, falls die .tex-Datei \titel etc. nicht gesetzt hat
\providecommand{\titel}{}
\providecommand{\editorInnen}{}
\providecommand{\dateiname}{\jobname}

\vspace{3cm}

\vfill

\footnotesize
\textsc{Quelle}: \titel. Herausgegeben von {\editorInnen}. In: \emph{Arthur Schnitzler: Briefwechsel mit Autorinnen und Autoren}.
 Digitale Edition, https://schnitzler-briefe.acdh.oeaw.ac.at/{\dateiname}.html (Stand \today)
\fi

\end{document}


      