%% latex-leseansicht-vorspann.tex
%% Vorspann für die Leseansicht.
%% Lädt die gemeinsame Datei latex-vorspann.tex mit nicht gesetztem Schalter.

\newif\ifkorrekturansicht
\korrekturansichtfalse

\input{../tex-inputs/latex-vorspann}


\section[Arthur Schnitzler an Gustav Schwarzkopf, {[}25. 10. 1903?{]}]{L04201 Arthur Schnitzler an Gustav Schwarzkopf, {[}25. 10. 1903?{]}}
\nopagebreak\mylabel{L04201v}
\rehead{ }\normalsize\beginnumbering\briefempfaengerindex{Schwarzkopf, Gustav@\textsc{Schwarzkopf, Gustav}!zzzSchnitzler, Arthur@\emph{von Arthur Schnitzler}!1903-10-251@{{[}25. 10. 1903?{]}}|(be}
\toendnotes[C]{\smallbreak\pagebreak[2]}
\correspDesc{Versand  durch Arthur Schnitzler am [25. 10. 1903?] in Wien
\newline{}Erhalt  durch Gustav Schwarzkopf im Zeitraum [25. 10. 1903 – 28. 10. 1903?] in Wien}\toendnotes[C]{\smallbreak}
\Standort{CUL, Schnitzler, B 96.}
\physDesc{Brief, 1 Blatt, 2 Seiten, 375 Zeichen
\newline{}Handschrift: Bleistift, deutsche Kurrent}\toendnotes[C]{\smallbreak}
\pstart
           \raggedleft{}{\pb}\textsc{Spöttelgasse 7\oindex{Wien@\textbf{Wien}!XVIII., Währing@\textbf{XVIII., Währing}!Edmund-Weiß-Gasse 7@\textbf{Edmund-Weiß-Gasse 7}, \emph{Wohngebäude}|pw}}\pend
           
\pstart
           \raggedleft{}Wien\oindex{Wien@\textbf{Wien}, \emph{Verwaltungsgebiet}|pw}, So{\geminationn}tag\pend
           \vspace{0.5em}
\pstart
           lieber Guſtav, möchten Sie \label{K_L04201-1v}\edtext{heute So{\geminationn}tag bei uns ein Nachtmahl einnehmen,
               verſchärft durch Vorleſen\eventindex{Edmund-Weiß-Gasse 7@\textbf{Edmund-Weiß-Gasse 7}!Private Lesung von Zwischenspiel, 18.6.1905@Private Lesung von Zwischenspiel, 18.6.1905|pwv} von fünf unſänglich kauzigen Acten\pwindex{Schnitzler, Arthur 15. 5. 1862 Wien – 21. 10. 1931 ebd.@\textsc{Schnitzler, Arthur} (15. 5. 1862 Wien – 21. 10. 1931 ebd.), \emph{Schriftsteller, Mediziner}!Zwischenspiel. Komödie in drei Akten@\strich\emph{Zwischenspiel. Komödie in drei Akten}|pwv}}{\lemma{\textnormal{\emph{heute … Acten}}}\Cendnote{\textnormal{Die Datierung gelingt
               durch das Zusammenlesen mehrerer Details: eine private Lesung eines fünfaktigen Werkes an einem Sonntag oder Montag, nur für Schwarzkopf\pwindex{Schwarzkopf, Gustav 7.\,11.\,1853 Wien – 13.\,11.\,1939 ebd.@\textsc{Schwarzkopf, Gustav} (7.\,11.\,1853 Wien – 13.\,11.\,1939 ebd.), \emph{Schriftsteller}|pwk} und
                  eventuell Olga\pwindex{Schnitzler, Olga 17.\,1.\,1882 Wien – 13.\,1.\,1970 Lugano@\textsc{Schnitzler, Olga} (17.\,1.\,1882 Wien – 13.\,1.\,1970 Lugano), \emph{Schauspielerin, Sängerin}|pwk}. Das trifft nur auf die Lesung\eventindex{Edmund-Weiß-Gasse 7@\textbf{Edmund-Weiß-Gasse 7}!Private Lesung von Zwischenspiel, 18.6.1905@Private Lesung von Zwischenspiel, 18.6.1905|pwkv} von
                  \emph{Der einsame Weg}\pwindex{Schnitzler, Arthur 15. 5. 1862 Wien – 21. 10. 1931 ebd.@\textsc{Schnitzler, Arthur} (15. 5. 1862 Wien – 21. 10. 1931 ebd.), \emph{Schriftsteller, Mediziner}!Zwischenspiel. Komödie in drei Akten@\strich\emph{Zwischenspiel. Komödie in drei Akten}|pwk} zu, die am Sonntag, dem 25. 10. 1903 
                  stattfand.}}}\label{K_L04201-1}? We{\geminationn}\uline{ja}, ſo bitte ſehr, ſeien Sie um 6, ½ 7 bei
               uns.\pend
           
\pstart
           Geht’s nicht, ſo gilt {\pb}die gleiche Einladung für Morgen Montag.\pend
           
\pstart
           Antwort keineswegs nöthig. Sind Sie um ½ 7{ }ſpäteſtens nicht da –{ }ſo nehme ich an
               Sie kommen morgen\pend
           
\pstart
           Herzlichſt Ihr{\\[\baselineskip]}\spacefill\mbox{A.}\pend
           \leftskip=0em{}\selectlanguage{ngerman}\endnumbering\briefempfaengerindex{Schwarzkopf, Gustav@\textsc{Schwarzkopf, Gustav}!zzzSchnitzler, Arthur@\emph{von Arthur Schnitzler}!1903-10-251@{{[}25. 10. 1903?{]}}|)be}\mylabel{L04201h}
\begin{anhang}
\end{anhang}\newcommand{\dateiname}{L04201}\newcommand{\titel}{Arthur Schnitzler an Gustav Schwarzkopf, [25. 10. 1903?]}\newcommand{\editorInnen}{Herausgegeben von Jahnke, SelmaMüller, Martin Anton}%% latex-leseansicht-abspann.tex
%% Abspann für die Leseansicht.
%% Der Schalter \ifkorrekturansicht ist bereits durch den Vorspann gesetzt.

%% latex-abspann.tex
%% Gemeinsamer Abspann für Korrekturansicht und Leseansicht.
%% Setzt den Schalter \ifkorrekturansicht voraus (gesetzt in den
%% einbindenden Dateien latex-korrekturansicht-abspann.tex bzw.
%% latex-leseansicht-abspann.tex).
%% ---------------------------------------------------------------

\normalsize

% Das esempio-Environment wird nur in der Leseansicht benötigt
\ifkorrekturansicht\else
\newenvironment{esempio}[3]%
{
    \vspace{1.5ex}
    \rlap{\underline{#1}}
    \par
    \setlength{\parindent}{0cm}
    \nopagebreak
    \leftskip=#2cm
    \rightskip=#3cm
}
{
    \par
}
\fi

\doendnotes{C}
\bigskip
\vfill

\clearpage

\footnotesize

\ifkorrekturansicht
  \lohead{\textsc{register}}
\fi

% theindex-Environment neu definieren ohne reledmac
\makeatletter
\renewenvironment{theindex}{%
  \ifkorrekturansicht
    \section*{\indexname}%
  \else
    \subsubsection*{Index der erwähnten Entitäten}%
  \fi
  \setlength{\parindent}{0pt}%
  \setlength{\parskip}{0pt plus 0.3pt}%
  \let\item\@idxitem
}{%
  \ifkorrekturansicht\clearpage\fi
}
\makeatother

\IfFileExists{\jobname-pw.ind}{\input{\jobname-pw.ind}}{}

% Quellenangabe nur in der Leseansicht
\ifkorrekturansicht\else
% Fallback-Definitionen, falls die .tex-Datei \titel etc. nicht gesetzt hat
\providecommand{\titel}{}
\providecommand{\editorInnen}{}
\providecommand{\dateiname}{\jobname}

\vspace{3cm}

\vfill

\footnotesize
\textsc{Quelle}: \titel. Herausgegeben von {\editorInnen}. In: \emph{Arthur Schnitzler: Briefwechsel mit Autorinnen und Autoren}.
 Digitale Edition, https://schnitzler-briefe.acdh.oeaw.ac.at/{\dateiname}.html (Stand \today)
\fi

\end{document}


