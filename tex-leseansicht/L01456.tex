\input{../tex-inputs/latex-pdf-vorspann}
\begin{center}
            \textcolor{red}{ENTWURF. ENTZIFFERUNG NOCH NICHT KORREKTURGELESEN}
                      \end{center}
            
               \section[Hugo von Hofmannsthal an Arthur Schnitzler, {[}16?. 10. 1904{]}]{ Hugo von Hofmannsthal an Arthur Schnitzler, {[}16?. 10. 1904{]}}\nopagebreak\mylabel{v}\rehead{ }\begin{ledgroupsized}[t]{13cm}\normalsize\beginnumbering\briefempfaengerindex{Schnitzler, Arthur@\textsc{Schnitzler, Arthur}!zzzHofmannsthal, Hugo von@\emph{von Hugo von Hofmannsthal}!1904-10-161@{{[}16?. 10. 1904{]}}|(be} \toendnotes[C]{\smallbreak\pagebreak[2]} \Standort{CUL, Schnitzler, B 43.}
\physDesc{Brief, 1 Blatt, 1 Seite
\newline{}Handschrift: schwarze Tinte, deutsche Kurrent
\newline{}Schnitzler: mit Bleistift datiert: »1\substVorne{}\textsuperscript{3}\substDazwischen{}6\substHinten{}. 10. 904« \newline{}Ordnung: 1) mit Bleistift von unbekannter Hand nummeriert: »200« 2) mit Bleistift von unbekannter Hand  nummeriert: »239«}\buchAbdrucke{\weitereDrucke{Hugo von Hofmannsthal, Arthur Schnitzler: \emph{Briefwechsel}. Hg. Therese Nickl und Heinrich Schnitzler. Frankfurt am Main: \emph{S. Fischer} 1964, S. 207.} }\toendnotes[C]{\smallbreak}\pstart{}{\pb}lieber \pend\pstart
           danke ſchön. Kein \textsc{rendez-vous} dieſe Woche?\hspace*{1.5em}Den ganzen \label{K_L01456_1v}\edtext{November bin ich dann
                  fort}{\lemma{\textnormal{\emph{November … fort}}}\Cendnote{\textnormal{Von
                     Anfang November bis zum 25. 11. 1904 war er
                  auf einer Waffenübung in Olmütz\oindex{Olomouc@\textbf{Olomouc}|pwk}.}}}\label{K_L01456_1h}. Man
               wird ſterben und einander viel zu wenig genoſſen haben!\hspace*{1.5em}Nur Do{\geminationn}erstag{ }ſind wir \uline{nicht}
               frei.\pend
           \pstart Ihr \spacefill\mbox{Hugo.}\pend{}\endnumbering\briefempfaengerindex{Schnitzler, Arthur@\textsc{Schnitzler, Arthur}!zzzHofmannsthal, Hugo von@\emph{von Hugo von Hofmannsthal}!1904-10-161@{{[}16?. 10. 1904{]}}|)be}\mylabel{h}\end{ledgroupsized}  \newcommand{\dateiname}{L01456}\newcommand{\titel}{Hugo von Hofmannsthal an Arthur Schnitzler, [16?. 10. 1904]}\newcommand{\editorInnen}{Martin Anton Müller und Gerd-Hermann Susen}\input{../tex-inputs/latex-pdf-abspann}
      