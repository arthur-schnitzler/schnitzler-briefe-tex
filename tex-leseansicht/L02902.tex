%% latex-leseansicht-vorspann.tex
%% Vorspann für die Leseansicht.
%% Lädt die gemeinsame Datei latex-vorspann.tex mit nicht gesetztem Schalter.

\newif\ifkorrekturansicht
\korrekturansichtfalse

\input{../tex-inputs/latex-vorspann}


\section[Paul Goldmann, Marie Glümer, Auguste Chlum und Moritz Coschell an Arthur Schnitzler, 11. 1. 1900]{L02902 Paul Goldmann, Marie Glümer, Auguste Chlum und Moritz Coschell an
               Arthur Schnitzler,  11. 1. 1900}
\nopagebreak\mylabel{L02902v}
\rehead{ }\normalsize\beginnumbering\briefempfaengerindex{Schnitzler, Arthur@\textsc{Schnitzler, Arthur}!zzzCoschell, Moritz@\emph{von Moritz Coschell}!1900-01-111@{11. 1. 1900}|(be}\briefempfaengerindex{Schnitzler, Arthur@\textsc{Schnitzler, Arthur}!zzzGlümer, Auguste@\emph{von Auguste Glümer}!1900-01-111@{11. 1. 1900}|(be}\briefempfaengerindex{Schnitzler, Arthur@\textsc{Schnitzler, Arthur}!zzzGlümer, Marie@\emph{von Marie Glümer}!1900-01-111@{11. 1. 1900}|(be}\briefempfaengerindex{Schnitzler, Arthur@\textsc{Schnitzler, Arthur}!zzzGoldmann, Paul@\emph{von Paul Goldmann}!1900-01-111@{11. 1. 1900}|(be}
\toendnotes[C]{\smallbreak\pagebreak[2]}
\correspDesc{Versand  durch Paul Goldmann, Marie Glümer, Auguste Chlum, Moritz Coschell am 11. 1. 1900 in Berlin
\newline{}Erhalt  durch Arthur Schnitzler im Zeitraum [12. 1. 1900
                  – 16. 1. 1900?] in Wien}\toendnotes[C]{\smallbreak}
\Standort{DLA, A:Schnitzler, HS.NZ85.1.3170.}
\physDesc{Postkarte, 611 Zeichen
\newline{}Handschrift Paul Goldmann: schwarze Tinte, deutsche Kurrent (\noindent{}auch die Unterstreichung von »Volkstheater\orgindex{Volkstheater@Volkstheater|pw}« und das Fußnotenzeichen stammen von Goldmann\pwindex{Goldmann, Paul 31.\,1.\,1865 Breslau – 25.\,9.\,1935 Wien@\textsc{Goldmann, Paul} (31.\,1.\,1865 Breslau – 25.\,9.\,1935 Wien), \emph{Schriftsteller, Journalist}|pw})
\newline{}Handschrift Marie Glümer: schwarze Tinte, deutsche Kurrent
\newline{}Handschrift Auguste Glümer: schwarze Tinte, lateinische Kurrent
\newline{}Handschrift Moritz Coschell: schwarze Tinte, lateinische Kurrent
\newline{}Versand: Stempel: »\nobreak{}\oindex{Berlin@\textbf{Berlin}, \emph{Hauptstadt}|pwk}Berlin, W., 13. 1. 00, 12–1 V.\nobreak{}«. Stempel: »\nobreak{}\oindex{IX., Alsergrund@\textbf{IX., Alsergrund}, \emph{Verwaltungsgebiet}|pwk}Wien 9/3 72, 14. 1{[}.{]} 00, 9{[}. V{]}, {[}Bestellt{]}\nobreak{}«.  }\toendnotes[C]{\smallbreak}\pstart{}\textsc{{\pb}Herrn}\pend{}\pstart{}\textsc{Dr. Arthur Schnitzler}\pend{}\pstart{}\textsc{Wien\oindex{Wien@\textbf{Wien}, \emph{Verwaltungsgebiet}|pw}}\pend{}\pstart{}\textsc{IX. Frankgaſse 1\oindex{Wien@\textbf{Wien}!IX., Alsergrund@\textbf{IX., Alsergrund}!Frankgasse 1@\textbf{Frankgasse 1}, \emph{Wohngebäude}|pw}.}\pend{}{\bigskip}\vspace{1em}
\pstart
           {\pb}Berlin\oindex{Berlin@\textbf{Berlin}, \emph{Hauptstadt}|pw} (leider)\pend
           
\pstart
           \raggedleft{}Den 11. 1. 1900.\pend
           \vspace{0.5em}
\pstart
           Lieber Freund, Ich{ }ſende Dir einen herzlichen Gruß aus
               der \label{K_L02902-1v}\edtext{Paſſauerſtraße 37\oindex{Passauerstraße@\textbf{Passauerstraße}, \emph{Straße}|pw}}{\lemma{\textnormal{\emph{Passauerstraße 37}}}\Cendnote{\textnormal{Wohnort von Auguste Chlum\pwindex{Glümer, Auguste 16.\,3.\,1862 Wien – 1956@\textsc{Glümer, Auguste} (16.\,3.\,1862 Wien – 1956), \emph{Lehrerin}|pwk} und Marie
                     Glümer\pwindex{Glümer, Marie 3.\,7.\,1867 Wien – 16.\,11.\,1925 München@\textsc{Glümer, Marie} (3.\,7.\,1867 Wien – 16.\,11.\,1925 München), \emph{Schauspielerin}|pwk}}}}\label{K_L02902-1}. Wir haben von Dir geſprochen – und auch ein wenig von \textsc{Hoffmannsthal\pwindex{Hofmannsthal, Hugo von 1.\,2.\,1874 Wien – 15.\,7.\,1929 Rodaun@\textsc{Hofmannsthal, Hugo von} (1.\,2.\,1874 Wien – 15.\,7.\,1929 Rodaun), \emph{Schriftsteller}|pw}}. \strikeout{D\textcolor{gray}{ein} tr} Dein treuer
                  \spacefill\mbox{Paul Goldmann\textcolor{gray}{.}}\pend
           
\pstart
           {[}hs. Glümer:{]} Ja, vom lieben Hofmannsthal\pwindex{Hofmannsthal, Hugo von 1.\,2.\,1874 Wien – 15.\,7.\,1929 Rodaun@\textsc{Hofmannsthal, Hugo von} (1.\,2.\,1874 Wien – 15.\,7.\,1929 Rodaun), \emph{Schriftsteller}|pw} haben wir geſprochen. – \label{K_L02902-2v}\edtext{G.}{\lemma{\textnormal{\emph{G.}}}\Cendnote{\textnormal{Goldmann\pwindex{Goldmann, Paul 31.\,1.\,1865 Breslau – 25.\,9.\,1935 Wien@\textsc{Goldmann, Paul} (31.\,1.\,1865 Breslau – 25.\,9.\,1935 Wien), \emph{Schriftsteller, Journalist}|pwk}}}}\label{K_L02902-2} iſt{ }ſehr lieb, wir{ }ſind wieder einmal ganz glücklich – Sonſt geht es elend. –
               Was iſt das für eine \label{K_L02902-3v}\edtext{Gouvernante\pwindex{Schnitzler, Arthur 15.\,5.\,1862 Wien – 21.\,10.\,1931 ebd.@\textsc{Schnitzler, Arthur} (15.\,5.\,1862 Wien – 21.\,10.\,1931 ebd.), \emph{Schriftsteller, Mediziner}!Gouvernante@\strich\emph{Die Gouvernante}|pw}}{\lemma{\textnormal{\emph{Gouvernante}}}\Cendnote{\textnormal{Bezug auf eine Zeitungsmeldung, dass Schnitzler einen Vierakter mit dem Titel \emph{Die Gouvernante}\pwindex{Schnitzler, Arthur 15.\,5.\,1862 Wien – 21.\,10.\,1931 ebd.@\textsc{Schnitzler, Arthur} (15.\,5.\,1862 Wien – 21.\,10.\,1931 ebd.), \emph{Schriftsteller, Mediziner}!Gouvernante@\strich\emph{Die Gouvernante}|pwk} abgeschlossen hätte und dieser
                  im \emph{Volkstheater}\orgindex{Volkstheater@Volkstheater|pwk} aufgeführt werden sollte.
                     Vgl. [O. V.]: \emph{Theater, Kunst und
                        Literatur}\pwindex{Theater, Kunst und Literatur [Die Gouvernante fertiggestellt]@\emph{Theater, Kunst und Literatur [Die Gouvernante fertiggestellt]}|pwk}. In: \emph{Wiener Allgemeine
                        Zeitung}\pwindex{Wiener Allgemeine Zeitung@\emph{Wiener Allgemeine Zeitung}|pwk}, Nr. 6554, 12. 1. 1900,
                     6 Uhr-Blatt, S. 3.}}}\label{K_L02902-3} im \uline{Volkstheater\orgindex{Volkstheater@Volkstheater|pw}}\label{T_L02902-1v}\edtext{\footnote{\noindent{}{[}hs. Goldmann:{]} Das unterſtrichene Wort{ }ſoll »Volkstheater\orgindex{Volkstheater@Volkstheater|pw}« heißen und nicht »Kohlrabi«.}}{\lemma{\textnormal{\emph{Das … »Kohlrabi«.}}}\Cendnote{\textnormal{kopfüber am oberen Rand}}}\label{T_L02902-1}? G.
               weiß auch nichts. \uline{\textsc{Telefonirt}}?\pend
           
\pstart
           {[}hs. Glümer:{]} Die \label{K_L02902-4v}\edtext{Herzogin}{\lemma{\textnormal{\emph{Herzogin}}}\Cendnote{\textnormal{unklare Anspielung}}}\label{K_L02902-4}
               küſst Sie auf die \textsc{Dichterstirne}!\pend
           
\pstart
           {[}hs. Coschell:{]} Heute beim Fischer\pwindex{Fischer, Samuel 24.\,12.\,1859 Liptovský Mikuláš – 15.\,10.\,1934 Berlin@\textsc{Fischer, Samuel} (24.\,12.\,1859 Liptovský Mikuláš – 15.\,10.\,1934 Berlin), \emph{Verleger}|pw} gewesen und über Anatol\pwindex{Schnitzler, Arthur 15.\,5.\,1862 Wien – 21.\,10.\,1931 ebd.@\textsc{Schnitzler, Arthur} (15.\,5.\,1862 Wien – 21.\,10.\,1931 ebd.), \emph{Schriftsteller, Mediziner}!Anatol@\strich\emph{Anatol}|pw}
               conferiert. Mache noch einige Ze\textcolor{gray}{i}chnungen. Brief folgt.\pend
           
\pstart
           1000 Grüße{\\[\baselineskip]}\spacefill\mbox{Coschell}\pend
           \leftskip=0em{}\selectlanguage{ngerman}\endnumbering\briefempfaengerindex{Schnitzler, Arthur@\textsc{Schnitzler, Arthur}!zzzCoschell, Moritz@\emph{von Moritz Coschell}!1900-01-111@{11. 1. 1900}|)be}\briefempfaengerindex{Schnitzler, Arthur@\textsc{Schnitzler, Arthur}!zzzGlümer, Auguste@\emph{von Auguste Glümer}!1900-01-111@{11. 1. 1900}|)be}\briefempfaengerindex{Schnitzler, Arthur@\textsc{Schnitzler, Arthur}!zzzGlümer, Marie@\emph{von Marie Glümer}!1900-01-111@{11. 1. 1900}|)be}\briefempfaengerindex{Schnitzler, Arthur@\textsc{Schnitzler, Arthur}!zzzGoldmann, Paul@\emph{von Paul Goldmann}!1900-01-111@{11. 1. 1900}|)be}\mylabel{L02902h}  \newcommand{\dateiname}{L02902}\newcommand{\titel}{Paul Goldmann, Marie Glümer, Auguste Chlum und Moritz Coschell an Arthur Schnitzler, 11. 1. 1900}\newcommand{\editorInnen}{Martin Anton Müller und Laura Untner}%% latex-leseansicht-abspann.tex
%% Abspann für die Leseansicht.
%% Der Schalter \ifkorrekturansicht ist bereits durch den Vorspann gesetzt.

%% latex-abspann.tex
%% Gemeinsamer Abspann für Korrekturansicht und Leseansicht.
%% Setzt den Schalter \ifkorrekturansicht voraus (gesetzt in den
%% einbindenden Dateien latex-korrekturansicht-abspann.tex bzw.
%% latex-leseansicht-abspann.tex).
%% ---------------------------------------------------------------

\normalsize

% Das esempio-Environment wird nur in der Leseansicht benötigt
\ifkorrekturansicht\else
\newenvironment{esempio}[3]%
{
    \vspace{1.5ex}
    \rlap{\underline{#1}}
    \par
    \setlength{\parindent}{0cm}
    \nopagebreak
    \leftskip=#2cm
    \rightskip=#3cm
}
{
    \par
}
\fi

\doendnotes{C}
\bigskip
\vfill

\clearpage

\footnotesize

\ifkorrekturansicht
  \lohead{\textsc{register}}
\fi

% theindex-Environment neu definieren ohne reledmac
\makeatletter
\renewenvironment{theindex}{%
  \ifkorrekturansicht
    \section*{\indexname}%
  \else
    \subsubsection*{Index der erwähnten Entitäten}%
  \fi
  \setlength{\parindent}{0pt}%
  \setlength{\parskip}{0pt plus 0.3pt}%
  \let\item\@idxitem
}{%
  \ifkorrekturansicht\clearpage\fi
}
\makeatother

\IfFileExists{\jobname-pw.ind}{\input{\jobname-pw.ind}}{}

% Quellenangabe nur in der Leseansicht
\ifkorrekturansicht\else
% Fallback-Definitionen, falls die .tex-Datei \titel etc. nicht gesetzt hat
\providecommand{\titel}{}
\providecommand{\editorInnen}{}
\providecommand{\dateiname}{\jobname}

\vspace{3cm}

\vfill

\footnotesize
\textsc{Quelle}: \titel. Herausgegeben von {\editorInnen}. In: \emph{Arthur Schnitzler: Briefwechsel mit Autorinnen und Autoren}.
 Digitale Edition, https://schnitzler-briefe.acdh.oeaw.ac.at/{\dateiname}.html (Stand \today)
\fi

\end{document}


