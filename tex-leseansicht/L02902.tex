%% latex-korrekturansicht-vorspann.tex
%% Vorspann für die Korrekturansicht.
%% Lädt die gemeinsame Datei latex-vorspann.tex mit gesetztem Schalter.

\newif\ifkorrekturansicht
\korrekturansichttrue

\input{../tex-inputs/latex-vorspann}


\section[Paul Goldmann, Marie Glümer, Auguste Chlum und Moritz Coschell an Arthur Schnitzler, 11. 1. 1900]{L02902 Paul Goldmann, Marie Glümer, Auguste Chlum und Moritz Coschell an
               Arthur Schnitzler, 11. 1. 1900}
\nopagebreak\mylabel{L02902v}
\rehead{ }\normalsize\beginnumbering\briefempfaengerindex{Schnitzler, Arthur@\textsc{Schnitzler, Arthur}!zzzCoschell, Moritz@\emph{von Moritz Coschell}!1900-01-111@{11. 1. 1900}|(be}\briefempfaengerindex{Schnitzler, Arthur@\textsc{Schnitzler, Arthur}!zzzGluemer, Auguste@\emph{von Auguste Glümer}!1900-01-111@{11. 1. 1900}|(be}\briefempfaengerindex{Schnitzler, Arthur@\textsc{Schnitzler, Arthur}!zzzGluemer, Marie@\emph{von Marie Glümer}!1900-01-111@{11. 1. 1900}|(be}\briefempfaengerindex{Schnitzler, Arthur@\textsc{Schnitzler, Arthur}!zzzGoldmann, Paul@\emph{von Paul Goldmann}!1900-01-111@{11. 1. 1900}|(be}
\toendnotes[C]{\smallbreak\pagebreak[2]}\Standort{DLA, A:Schnitzler, HS.NZ85.1.3170.}
\physDesc{Postkarte, 611 Zeichen
\newline{}Handschrift Paul Goldmann: 1) schwarze Tinte, deutsche Kurrent (\noindent{}auch die Unterstreichung von »Volkstheater\orgindex{Volkstheater@Volkstheater|pw}« und das Fußnotenzeichen stammen von Goldmann\pwindex{Goldmann, Paul 31.01.1865 – 25.09.1935@\textsc{Goldmann, Paul} (31.01.1865 – 25.09.1935), \emph{Schriftsteller/Schriftstellerin, Journalist/Journalistin}|pw})\hspace{1em}2) schwarze Tinte, lateinische Kurrent (\noindent{}Adresse)\hspace{1em}
\newline{}Handschrift Marie Glümer: schwarze Tinte, deutsche Kurrent
\newline{}Handschrift Auguste Glümer: schwarze Tinte, lateinische Kurrent
\newline{}Handschrift Moritz Coschell: schwarze Tinte, lateinische Kurrent
\newline{}Versand: Stempel: »\nobreak{}\oindex{Berlin@\textbf{Berlin}, \emph{P.PPLC}|pwk}Berlin, W., 13. 1. 00, 12–1 V.\nobreak{}«. Stempel: »\nobreak{}\oindex{IX., Alsergrund@\textbf{IX., Alsergrund}, \emph{A.ADM3}|pwk}Wien 9/3 72, 14. 1{[}.{]} 00, 9{[}. V{]}, {[}Bestellt{]}\nobreak{}«.  }\toendnotes[C]{\smallbreak}\pstart{}{\pb}Herrn\pend{}\pstart{}Dr. Arthur Schnitzler\pend{}\pstart{}Wien\oindex{Wien@\textbf{Wien}, \emph{A.ADM2}|pw}\pend{}\pstart{}IX. Frankgaſse 1\oindex{Frankgasse 1@\textbf{Frankgasse 1}, \emph{Wohngebäude (K.WHS)}|pw}.\pend{}{\bigskip}\vspace{1em}
\pstart
           {\pb}Berlin\oindex{Berlin@\textbf{Berlin}, \emph{P.PPLC}|pw} (leider)\pend
           
\pstart
           \raggedleft{}Den 11. 1. 1900.\pend
           \vspace{0.5em}
\pstart
           Lieber Freund, Ich ſende Dir einen herzlichen Gruß aus
               der \label{K_L02902-1v}\edtext{Paſſauerſtraße 37\oindex{Passauerstrasse@\textbf{Passauerstraße}, \emph{Straße (K.STR)}|pw}}{\lemma{\textnormal{\emph{Paſſauerſtraße 37}}}\Cendnote{\textnormal{Wohnort von Auguste Chlum\pwindex{Gluemer, Auguste 1862-03-16 – 1956@\textsc{Glümer, Auguste} (1862-03-16 – 1956), \emph{Lehrer/Lehrerin}|pwk} und Marie
                     Glümer\pwindex{Gluemer, Marie 03.07.1867 – 16.11.1925@\textsc{Glümer, Marie} (03.07.1867 – 16.11.1925), \emph{Schauspieler/Schauspielerin}|pwk}}}}\label{K_L02902-1}. Wir haben von Dir geſprochen – und auch ein wenig von \textsc{Hoffmannsthal\pwindex{Hofmannsthal, Hugo von 1874-02-01 – 1929-07-15@\textsc{Hofmannsthal, Hugo von} (1874-02-01 – 1929-07-15), \emph{Schriftsteller/Schriftstellerin}|pw}}. \strikeout{D\textcolor{gray}{ein} tr} Dein treuer
                  \spacefill\mbox{Paul Goldmann\textcolor{gray}{.}}\pend
           
\pstart
           {[}hs. :{]} Ja, vom lieben Hofmannsthal\pwindex{Hofmannsthal, Hugo von 1874-02-01 – 1929-07-15@\textsc{Hofmannsthal, Hugo von} (1874-02-01 – 1929-07-15), \emph{Schriftsteller/Schriftstellerin}|pw} haben wir geſprochen. – \label{K_L02902-2v}\edtext{G.}{\lemma{\textnormal{\emph{G.}}}\Cendnote{\textnormal{Goldmann\pwindex{Goldmann, Paul 31.01.1865 – 25.09.1935@\textsc{Goldmann, Paul} (31.01.1865 – 25.09.1935), \emph{Schriftsteller/Schriftstellerin, Journalist/Journalistin}|pwk}}}}\label{K_L02902-2} iſt ſehr lieb, wir ſind wieder einmal ganz glücklich – Sonſt geht es elend. –
               Was iſt das für eine \label{K_L02902-3v}\edtext{Gouvernante\pwindex{Gouvernante@\emph{Die Gouvernante}|pw}}{\lemma{\textnormal{\emph{Gouvernante}}}\Cendnote{\textnormal{Bezug auf eine Zeitungsmeldung, dass Schnitzler einen Vierakter mit dem Titel \emph{Die Gouvernante}\pwindex{Gouvernante@\emph{Die Gouvernante}|pwk} abgeschlossen hätte und dieser
                  im \emph{Volkstheater}\orgindex{Volkstheater@Volkstheater|pwk} aufgeführt werden sollte.
                     Vgl. [O. V.]: \emph{Theater, Kunst und
                        Literatur}\pwindex{Theater, Kunst und Literatur [Die Gouvernante fertiggestellt]@\emph{Theater, Kunst und Literatur [Die Gouvernante fertiggestellt]}|pwk}. In: \emph{Wiener Allgemeine
                        Zeitung}\pwindex{Wiener Allgemeine Zeitung@\emph{Wiener Allgemeine Zeitung}|pwk}, Nr. 6554, 12. 1. 1900,
                     6 Uhr-Blatt, S. 3.}}}\label{K_L02902-3} im \uline{Volkstheater\orgindex{Volkstheater@Volkstheater|pw}}\label{T_L02902-1v}\edtext{\noindent{}{[}hs. :{]} Das unterſtrichene Wort ſoll »Volkstheater\orgindex{Volkstheater@Volkstheater|pw}« heißen und nicht »Kohlrabi«.}{\lemma{\textnormal{\emph{Das … »Kohlrabi«.}}}\Cendnote{\textnormal{kopfüber am oberen Rand}}}\label{T_L02902-1}? G.
               weiß auch nichts. \uline{\textsc{Telefonirt}}?\pend
           
\pstart
           {[}hs. :{]} Die \label{K_L02902-4v}\edtext{Herzogin}{\lemma{\textnormal{\emph{Herzogin}}}\Cendnote{\textnormal{unklare Anspielung}}}\label{K_L02902-4}
               küſst Sie auf die \textsc{Dichterstirne}!\pend
           
\pstart
           {[}hs. :{]} Heute beim Fischer\pwindex{Fischer, Samuel 24.12.1859 – 15.10.1934@\textsc{Fischer, Samuel} (24.12.1859 – 15.10.1934), \emph{Verleger/Verlegerin}|pw} gewesen und über Anatol\pwindex{Anatol@\emph{Anatol}|pw}
               conferiert. Mache noch einige Ze\textcolor{gray}{i}chnungen. Brief folgt.\pend
           
\pstart
           1000 Grüße{\\[\baselineskip]}\spacefill\mbox{Coschell}\pend
           \leftskip=0em{}\selectlanguage{ngerman}\endnumbering\briefempfaengerindex{Schnitzler, Arthur@\textsc{Schnitzler, Arthur}!zzzCoschell, Moritz@\emph{von Moritz Coschell}!1900-01-111@{11. 1. 1900}|)be}\briefempfaengerindex{Schnitzler, Arthur@\textsc{Schnitzler, Arthur}!zzzGluemer, Auguste@\emph{von Auguste Glümer}!1900-01-111@{11. 1. 1900}|)be}\briefempfaengerindex{Schnitzler, Arthur@\textsc{Schnitzler, Arthur}!zzzGluemer, Marie@\emph{von Marie Glümer}!1900-01-111@{11. 1. 1900}|)be}\briefempfaengerindex{Schnitzler, Arthur@\textsc{Schnitzler, Arthur}!zzzGoldmann, Paul@\emph{von Paul Goldmann}!1900-01-111@{11. 1. 1900}|)be}\mylabel{L02902h}  \normalsize

\doendnotes{C}
\bigskip
\vfill

\clearpage

\footnotesize

\lohead{\textsc{register}}

% Definiere theindex-Environment komplett neu ohne reledmac
\makeatletter
\renewenvironment{theindex}{%
  \section*{\indexname}%
  \setlength{\parindent}{0pt}%
  \setlength{\parskip}{0pt plus 0.3pt}%
  \let\item\@idxitem
}{%
  \clearpage
}
\makeatother

\IfFileExists{\jobname-pw.ind}{\input{\jobname-pw.ind}}{}

\end{document}

      