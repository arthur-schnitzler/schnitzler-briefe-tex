%% latex-leseansicht-vorspann.tex
%% Vorspann für die Leseansicht.
%% Lädt die gemeinsame Datei latex-vorspann.tex mit nicht gesetztem Schalter.

\newif\ifkorrekturansicht
\korrekturansichtfalse

\input{../tex-inputs/latex-vorspann}


         
         \renewcommand{\erwaehntePersonen}{Personen: Samuel Fischer, Hugo von Hofmannsthal}
         \renewcommand{\erwaehnteInstitutionen}{Institutionen: Volkstheater}
         \renewcommand{\erwaehnteOrte}{Orte: Berlin, Frankgasse, IX., Alsergrund, Passauerstraße, Wien}
         \renewcommand{\erwaehnteWerke}{Werke: Anatol, Die Gouvernante, Theater, Kunst und Literatur [Gouvernante], Wiener Allgemeine Zeitung}
               \section[Paul Goldmann, Marie Glümer, Auguste Chlum und Moritz Coschell an Arthur Schnitzler, 11. 1. 1900]{ Paul Goldmann, Marie Glümer, Auguste Chlum und Moritz Coschell an
               Arthur Schnitzler, 11. 1. 1900}\nopagebreak\mylabel{v}\rehead{ }\begin{ledgroupsized}[t]{13cm}\normalsize\beginnumbering \toendnotes[C]{\smallbreak\pagebreak[2]} \Standort{DLA, A:Schnitzler, HS.NZ85.1.3170.}
\physDesc{Postkarte
\newline{}Handschrift Paul Goldmann: 1) schwarze Tinte, deutsche Kurrent (\noindent{}auch die Unterstreichung von »Volkstheater\orgindex{Volkstheater@Volkstheater|pw}« und das Fußnotenzeichen stammen von Goldmann\pwindex{Goldmann, Paul 31.01.1865 – 25.09.1935@\textsc{Goldmann, Paul} (31.01.1865 – 25.09.1935), \emph{Schriftsteller, Journalist}|pw})\hspace{1em}2) schwarze Tinte, lateinische Kurrent (\noindent{}Adresse)\hspace{1em}\newline{}Handschrift Marie Glümer: schwarze Tinte, deutsche Kurrent\newline{}Handschrift Auguste Glümer: schwarze Tinte, lateinische Kurrent\newline{}Handschrift Moritz Coschell: schwarze Tinte, lateinische Kurrent\newline{}Versand: Stempel: »\nobreak{}\oindex{Berlin@\textbf{Berlin}|pwk}Berlin, W., 13. 1. 00, 12–1 V.\nobreak{}«. Stempel: »\nobreak{}\oindex{IX., Alsergrund@\textbf{IX., Alsergrund}|pwk}Wien 9/3 72, 14. 1{[}.{]} 00, 9{[}. V{]}, {[}Bestellt{]}\nobreak{}«.  }\toendnotes[C]{\smallbreak}\pstart{}{\pb}Herrn\pend{}\pstart{}Dr. Arthur Schnitzler\pend{}\pstart{}Wien\oindex{Wien@\textbf{Wien}|pw}\pend{}\pstart{}IX. Frankgaſse 1\oindex{Frankgasse@\textbf{Frankgasse}|pw}.\pend{}{\bigskip}\pstart
           {\pb}Berlin\oindex{Berlin@\textbf{Berlin}|pw} (leider)\pend
           \pstart
           \raggedleft{}Den 11. 1. 1900.\pend
           \pstart
           Lieber Freund, Ich ſende Dir einen herzlichen Gruß aus
               der \label{K_L02902-1v}\edtext{Paſſauerſtraße 37\oindex{Passauerstrasse@\textbf{Passauerstraße}|pw}}{\lemma{\textnormal{\emph{Paſſauerſtraße 37}}}\Cendnote{\textnormal{Wohnort von Auguste Chlum\pwindex{Gluemer, Auguste 16.03.1862 – 1956@\textsc{Glümer, Auguste} (16.03.1862 – 1956)|pwk} und Marie
                     Glümer\pwindex{Gluemer, Marie 03.07.1867 – 16.11.1925@\textsc{Glümer, Marie} (03.07.1867 – 16.11.1925), \emph{Schauspielerin}|pwk}}}}\label{K_L02902-1h}. Wir haben von Dir geſprochen – und auch ein wenig von \textsc{Hoffmannsthal\pwindex{Hofmannsthal, Hugo von 1874-02-01 – 1929-07-15@\textsc{Hofmannsthal, Hugo von} (1874-02-01 – 1929-07-15), \emph{Schriftsteller}|pw}}. \strikeout{D\textcolor{gray}{ein} tr} Dein treuer
                  \spacefill\mbox{Paul Goldmann\textcolor{gray}{.}}\pend
           \pstart
           {[}hs. Marie Glümer:{]} Ja, vom lieben Hofmannsthal\pwindex{Hofmannsthal, Hugo von 1874-02-01 – 1929-07-15@\textsc{Hofmannsthal, Hugo von} (1874-02-01 – 1929-07-15), \emph{Schriftsteller}|pw} haben wir geſprochen. – \label{K_L02902-66v}\edtext{G.}{\lemma{\textnormal{\emph{G.}}}\Cendnote{\textnormal{Goldmann\pwindex{Goldmann, Paul 31.01.1865 – 25.09.1935@\textsc{Goldmann, Paul} (31.01.1865 – 25.09.1935), \emph{Schriftsteller, Journalist}|pwk}}}}\label{K_L02902-66h} iſt ſehr lieb, wir ſind wieder einmal ganz glücklich – Sonſt geht es elend. –
               Was iſt das für eine \label{K_L02902-4v}\edtext{Gouvernante\pwindex{Schnitzler, Arthur 15.05.1862 – 21.10.1931@\textsc{Schnitzler, Arthur} (15.05.1862 – 21.10.1931), \emph{Schriftsteller, Mediziner}!Gouvernante2. 12. 1903@\strich\emph{Die Gouvernante} {[}2. 12. 1903{]}|pw}}{\lemma{\textnormal{\emph{Gouvernante}}}\Cendnote{\textnormal{Bezug auf eine Zeitungsmeldung, dass Schnitzler\pwindex{Schnitzler, Arthur 15.05.1862 – 21.10.1931@\textsc{Schnitzler, Arthur} (15.05.1862 – 21.10.1931), \emph{Schriftsteller, Mediziner}|pwk} einen Vierakter mit dem Titel \emph{Die Gouvernante}\pwindex{Schnitzler, Arthur 15.05.1862 – 21.10.1931@\textsc{Schnitzler, Arthur} (15.05.1862 – 21.10.1931), \emph{Schriftsteller, Mediziner}!Gouvernante2. 12. 1903@\strich\emph{Die Gouvernante} {[}2. 12. 1903{]}|pwk} abgeschlossen hätte und dieser im \emph{Volkstheater}\orgindex{Volkstheater@Volkstheater|pwk} aufgeführt werden sollte.
                     Vgl. [O. V.]: \emph{Theater, Kunst und
                        Literatur}\pwindex{?? Werk@Nicht ermittelte Verfasserinnen und Verfasser!Theater, Kunst und Literatur [Gouvernante]1900-01-12@\emph{Theater, Kunst und Literatur [Gouvernante]} {[}1900-01-12{]}|pwk}. In: \emph{Wiener Allgemeine
                        Zeitung}\pwindex{?? Werk@Nicht ermittelte Verfasserinnen und Verfasser!Wiener Allgemeine Zeitung1.3.1880 – 11.2.1934@\emph{Wiener Allgemeine Zeitung} {[}1.3.1880 – 11.2.1934{]}|pwk}, Nr. 6554, 12. 1. 1900,
                     6 Uhr-Blatt, S. 3.}}}\label{K_L02902-4h} im \uline{Volkstheater\orgindex{Volkstheater@Volkstheater|pw}}\footnote{\noindent{}{[}hs. Goldmann:{]} \label{T_L02902-1v}\toendnotes[C]{\begin{minipage}[t]{4em}{\makebox[3.6em][r]{\tiny{Fußnote}}}\end{minipage}\begin{minipage}[t]{\dimexpr\linewidth-4em}\textit{Das unterſtrichene Wort ſoll »Volkstheater« heißen und nicht
                        »Kohlrabi«.}\,{]} kopfüber am oberen
                        Rand\end{minipage}\par}Das unterſtrichene Wort ſoll »Volkstheater\orgindex{Volkstheater@Volkstheater|pw}« heißen und nicht
                        »Kohlrabi«.\label{T_L02902-1h}}? G. weiß auch nichts. \uline{\textsc{Telefonirt}}?\pend
           \pstart
           {[}hs. Auguste Glümer:{]} Die \label{K_L02902-2v}\edtext{Herzogin}{\lemma{\textnormal{\emph{Herzogin}}}\Cendnote{\textnormal{unklare Anspielung}}}\label{K_L02902-2h}
               küſst Sie auf die \textsc{Dichterstirne}!\pend
           \pstart
           {[}hs. Coschell:{]} Heute beim Fischer\pwindex{Fischer, Samuel 24.12.1859 – 15.10.1934@\textsc{Fischer, Samuel} (24.12.1859 – 15.10.1934), \emph{Verleger}|pw} gewesen und über Anatol\pwindex{Schnitzler, Arthur 15.05.1862 – 21.10.1931@\textsc{Schnitzler, Arthur} (15.05.1862 – 21.10.1931), \emph{Schriftsteller, Mediziner}!Anatol1892-10-29@\strich\emph{Anatol} {[}1892-10-29{]}|pw}
               conferiert. Mache noch einige Ze\textcolor{gray}{i}chnungen. Brief folgt.\pend
           \pstart
           1000 Grüße{\\[\baselineskip]}\spacefill\mbox{Coschell}\pend
           \leftskip=0em{}
         
         \endnumbering\mylabel{h}\end{ledgroupsized}  \newcommand{\dateiname}{L02902}\newcommand{\titel}{Paul Goldmann, Marie Glümer, Auguste Chlum und Moritz Coschell an Arthur Schnitzler, 11. 1. 1900}\newcommand{\editorInnen}{Martin Anton Müller und Laura Untner}%% latex-leseansicht-abspann.tex
%% Abspann für die Leseansicht.
%% Der Schalter \ifkorrekturansicht ist bereits durch den Vorspann gesetzt.

%% latex-abspann.tex
%% Gemeinsamer Abspann für Korrekturansicht und Leseansicht.
%% Setzt den Schalter \ifkorrekturansicht voraus (gesetzt in den
%% einbindenden Dateien latex-korrekturansicht-abspann.tex bzw.
%% latex-leseansicht-abspann.tex).
%% ---------------------------------------------------------------

\normalsize

% Das esempio-Environment wird nur in der Leseansicht benötigt
\ifkorrekturansicht\else
\newenvironment{esempio}[3]%
{
    \vspace{1.5ex}
    \rlap{\underline{#1}}
    \par
    \setlength{\parindent}{0cm}
    \nopagebreak
    \leftskip=#2cm
    \rightskip=#3cm
}
{
    \par
}
\fi

\doendnotes{C}
\bigskip
\vfill

\clearpage

\footnotesize

\ifkorrekturansicht
  \lohead{\textsc{register}}
\fi

% theindex-Environment neu definieren ohne reledmac
\makeatletter
\renewenvironment{theindex}{%
  \ifkorrekturansicht
    \section*{\indexname}%
  \else
    \subsubsection*{Index der erwähnten Entitäten}%
  \fi
  \setlength{\parindent}{0pt}%
  \setlength{\parskip}{0pt plus 0.3pt}%
  \let\item\@idxitem
}{%
  \ifkorrekturansicht\clearpage\fi
}
\makeatother

\IfFileExists{\jobname-pw.ind}{\input{\jobname-pw.ind}}{}

% Quellenangabe nur in der Leseansicht
\ifkorrekturansicht\else
% Fallback-Definitionen, falls die .tex-Datei \titel etc. nicht gesetzt hat
\providecommand{\titel}{}
\providecommand{\editorInnen}{}
\providecommand{\dateiname}{\jobname}

\vspace{3cm}

\vfill

\footnotesize
\textsc{Quelle}: \titel. Herausgegeben von {\editorInnen}. In: \emph{Arthur Schnitzler: Briefwechsel mit Autorinnen und Autoren}.
 Digitale Edition, https://schnitzler-briefe.acdh.oeaw.ac.at/{\dateiname}.html (Stand \today)
\fi

\end{document}


      