%% latex-korrekturansicht-vorspann.tex
%% Vorspann für die Korrekturansicht.
%% Lädt die gemeinsame Datei latex-vorspann.tex mit gesetztem Schalter.

\newif\ifkorrekturansicht
\korrekturansichttrue

\input{../tex-inputs/latex-vorspann}


\section[Hugo von Hofmannsthal an Arthur Schnitzler, 16. 12. 1904]{L01480 Hugo von Hofmannsthal an Arthur Schnitzler, 16. 12. 1904}
\nopagebreak\mylabel{L01480v}
\rehead{ }\normalsize\beginnumbering\briefempfaengerindex{Schnitzler, Arthur@\textsc{Schnitzler, Arthur}!zzzHofmannsthal, Hugo von@\emph{von Hugo von Hofmannsthal}!1904-12-161@{16. 12. 1904}|(be}
\toendnotes[C]{\smallbreak\pagebreak[2]}\Standort{CUL, Schnitzler, B 43.}
\physDesc{Postkarte, 246 Zeichen
\newline{}Handschrift: schwarze Tinte, lateinische Kurrent
\newline{}Versand: 1) Stempel: »\nobreak{}\oindex{Rodaun@\textbf{Rodaun}, \emph{A.ADM4}|pwk}Rodaun, 16 12 04, \textcolor{gray}{6}N\nobreak{}«.   2) Stempel: »\nobreak{}\oindex{XVIII., Waehring@\textbf{XVIII., Währing}, \emph{A.ADM3}|pwk}18/2 Wien 113, 17. 12. \textcolor{gray}{0}4, Bestellt\nobreak{}«.  3) mit schwarzer Tinte von unbekannter Hand die Bezirksnummer um den
                                 Postrayon erweitert: »/1«, was im Zusammenhang mit
                                 dem Empfangsstempel vom Postrayon 18/2 stehen dürfte
\newline{}Schnitzler: mit Bleistift datiert: »17/12 904« 
\newline{}Ordnung: 1) mit Bleistift von unbekannter Hand nummeriert:
                                    »219«  2) mit Bleistift von unbekannter Hand nummeriert:
                                    »244«}
\buchAbdrucke{\weitereDrucke{Hugo von Hofmannsthal, Arthur Schnitzler: \emph{Briefwechsel}. Frankfurt am Main: \emph{S. Fischer} 1964, S. 208.} }\toendnotes[C]{\smallbreak}\pstart{}{\pb}Herrn D\textsuperscript{r} Arthur Schnitzler\pend{}\pstart{}Wien\oindex{Wien@\textbf{Wien}, \emph{A.ADM2}|pw}\pend{}\pstart{}XVIII Spöttelgasse 7\oindex{Edmund-Weiss-Gasse 7@\textbf{Edmund-Weiß-Gasse 7}, \emph{Wohngebäude (K.WHS)}|pw}\pend{}{\bigskip}\vspace{1em}
\pstart
           {\pb}Freitag.\pend
           \vspace{0.5em}
\pstart
           Freuen uns auf \label{K_L01480-1v}\edtext{Mittwoch}{\lemma{\textnormal{\emph{Mittwoch}}}\Cendnote{\textnormal{Vgl. A. S.: \emph{Tagebuch}, 21. 12. 1890.
               }}}\label{K_L01480-1}.\pend
           
\pstart
           Wir beide möchten schon gegen ½ 7 ko{\geminationm}en,
                  Papa\pwindex{Hofmannsthal, Hugo August von 21.12.1841 – 08.12.1915@\textsc{Hofmannsthal, Hugo August von} (21.12.1841 – 08.12.1915), \emph{Bankdirektor/Bankdirektorin}|pwv} etwas später.\pend
           
\pstart
           Herzlich{\\[\baselineskip]}\spacefill\mbox{Hugo}\pend
           \leftskip=0em{}
\pstart
           \noindent{}Richard\pwindex{Beer-Hofmann, Richard 1866-07-11 – 1945-09-26@\textsc{Beer-Hofmann, Richard} (1866-07-11 – 1945-09-26), \emph{Schriftsteller/Schriftstellerin}|pw} ist dort\oindex{Berlin@\textbf{Berlin}, \emph{P.PPLC}|pw}. Herzzerreißende Première\pwindex{Graf von Charolais. Ein Trauerspiel@\emph{Der Graf von Charolais. Ein Trauerspiel}|pwv} soll 23\textsuperscript{ten} sein. Höflich\pwindex{Hoeflich, Lucie 20.02.1883 – 08.10.1956@\textsc{Höflich, Lucie} (20.02.1883 – 08.10.1956), \emph{Schauspieler/Schauspielerin}|pw} und Sorma\pwindex{Sorma, Agnes 17.05.1862 – 10.02.1927@\textsc{Sorma, Agnes} (17.05.1862 – 10.02.1927), \emph{Schauspieler/Schauspielerin}|pw} hat er schon nahezu umgebracht.\pend
           \selectlanguage{ngerman}\endnumbering\briefempfaengerindex{Schnitzler, Arthur@\textsc{Schnitzler, Arthur}!zzzHofmannsthal, Hugo von@\emph{von Hugo von Hofmannsthal}!1904-12-161@{16. 12. 1904}|)be}\mylabel{L01480h}  \normalsize

\doendnotes{C}
\bigskip
\vfill

\clearpage

\footnotesize

\lohead{\textsc{register}}

% Definiere theindex-Environment komplett neu ohne reledmac
\makeatletter
\renewenvironment{theindex}{%
  \section*{\indexname}%
  \setlength{\parindent}{0pt}%
  \setlength{\parskip}{0pt plus 0.3pt}%
  \let\item\@idxitem
}{%
  \clearpage
}
\makeatother

\IfFileExists{\jobname-pw.ind}{\input{\jobname-pw.ind}}{}

\end{document}

      