%% latex-leseansicht-vorspann.tex
%% Vorspann für die Leseansicht.
%% Lädt die gemeinsame Datei latex-vorspann.tex mit nicht gesetztem Schalter.

\newif\ifkorrekturansicht
\korrekturansichtfalse

\input{../tex-inputs/latex-vorspann}


\section[Arthur Schnitzler an Hermann Bahr, 16. 12. 1907]{L01741 Arthur Schnitzler an Hermann Bahr, 16. 12. 1907}
\nopagebreak\mylabel{L01741v}
\rehead{ }\normalsize\beginnumbering\briefempfaengerindex{Bahr, Hermann@\textsc{Bahr, Hermann}!zzzSchnitzler, Arthur@\emph{von Arthur Schnitzler}!1907-12-161@{16. 12. 1907}|(be}
\toendnotes[C]{\smallbreak\pagebreak[2]}
\correspDesc{Versand  durch Arthur Schnitzler am 16. 12. 1907 in Wien
\newline{}Erhalt  durch Hermann Bahr im Zeitraum [16. 12. 1907 – 20. 12. 1907?] in Wien}\toendnotes[C]{\smallbreak}
\Standort{TMW, HS AM 23388 Ba.}
\physDesc{Brief, 1 Blatt, 3 Seiten, 841 Zeichen
\newline{}Handschrift: schwarze Tinte, lateinische Kurrent
\newline{}Ordnung: Lochung }
\buchAbdrucke{\weitereDrucke{1) \emph{16. 12. 1907.} In: Arthur Schnitzler: \emph{The Letters of Arthur Schnitzler to Hermann Bahr}. Edited, annotated, and with an introduction, by Donald G. Daviau. Chapel Hill: \emph{The University of North Carolina Press} 1978, S. 100 (University of North Carolina studies in the Germanic languages
                        and literatures, 89).} \weitereDrucke{2) Hermann Bahr, Arthur Schnitzler: \emph{Briefwechsel, Aufzeichnungen, Dokumente (1891–1931)}. Herausgegeben von Kurt Ifkovits und Martin Anton Müller. Göttingen: \emph{Wallstein} 2018, S. 398.} }\toendnotes[C]{\smallbreak}
\pstart
           \raggedleft{}{\pb}\uuline{Vertraulich}\pend
           
\pstart
           \textcolor{gray}{\textbf{Dr. Arthur Schnitzler}}\hfill 16/12 907\pend
           
\pstart
           \textcolor{gray}{\textbf{Wien XVIII. Spoettelgasse 7\oindex{Wien@\textbf{Wien}!XVIII., Währing@\textbf{XVIII., Währing}!Edmund-Weiß-Gasse 7@\textbf{Edmund-Weiß-Gasse 7}, \emph{Wohngebäude}|pw}.}}\pend
           
\pstart{}lieber Hermann,\pend\vspace{0.5em}
\pstart
           ich weiss nicht, ob du noch \label{K_L01741-1v}\edtext{in Wien\oindex{Wien@\textbf{Wien}, \emph{Verwaltungsgebiet}|pw}}{\lemma{\textnormal{\emph{in Wien}}}\Cendnote{\textnormal{Bahr\pwindex{Bahr, Hermann 19.\,7.\,1863 Linz – 15.\,1.\,1934 München@\textsc{Bahr, Hermann} (19.\,7.\,1863 Linz – 15.\,1.\,1934 München), \emph{Schriftsteller, Kritiker}|pwk} war nicht mehr in Berlin\oindex{Berlin@\textbf{Berlin}, \emph{Hauptstadt}|pwk}, doch möglicherweise auf dem Semmering\oindex{Semmering@\textbf{Semmering}, \emph{Verwaltungsgebiet}|pwk}.}}}\label{K_L01741-1} bist – schreibe dir jedenfalls an deine Wr\oindex{Wien@\textbf{Wien}, \emph{Verwaltungsgebiet}|pw} Adresse, aufsuchen kö{\geminationn}t ich dich keineswegs, weil meine Frau\pwindex{Schnitzler, Olga 17.\,1.\,1882 Wien – 13.\,1.\,1970 Lugano@\textsc{Schnitzler, Olga} (17.\,1.\,1882 Wien – 13.\,1.\,1970 Lugano), \emph{Schauspielerin, Sängerin}|pwv} sich eben in Reconvalescenz von einem
               Scharlach befindet – (doch schon gekräftigt genug, um dich herzlich zu grüßen und dir
               mit mir zu dem \label{K_L01741-2v}\edtext{nachtigalligen Erfolg\pwindex{Bahr, Hermann 19.\,7.\,1863 Linz – 15.\,1.\,1934 München@\textsc{Bahr, Hermann} (19.\,7.\,1863 Linz – 15.\,1.\,1934 München), \emph{Schriftsteller, Kritiker}!gelbe Nachtigall@\strich\emph{Die gelbe Nachtigall}|pwv}}{\lemma{\textnormal{\emph{nachtigalligen Erfolg}}}\Cendnote{\textnormal{Die Uraufführung\eventindex{Lessing-Theater@\textbf{Lessing-Theater}!Uraufführung von Die gelbe Nachtigall, 10.12.1907@Uraufführung von Die gelbe Nachtigall, 10.12.1907|pwkv} von \emph{Die gelbe Nachtigall}\pwindex{Bahr, Hermann 19.\,7.\,1863 Linz – 15.\,1.\,1934 München@\textsc{Bahr, Hermann} (19.\,7.\,1863 Linz – 15.\,1.\,1934 München), \emph{Schriftsteller, Kritiker}!gelbe Nachtigall@\strich\emph{Die gelbe Nachtigall}|pwk} hatte am 10. 12. 1907 am \emph{Lessing-Theater}\orgindex{Lessing-Theater@Lessing-Theater|pwk} in Berlin\oindex{Berlin@\textbf{Berlin}, \emph{Hauptstadt}|pwk} stattgefunden. }}}\label{K_L01741-2} schönstens zu gratuliren) –
               Also \uuline{unter uns}{ }{\pb}formeller Antrag des
                  Hebbeltheater\oindex{Hebbel-Theater@\textbf{Hebbel-Theater}, \emph{Theater}|pw} liegt mir vor: Beatrice\pwindex{Schnitzler, Arthur 15.\,5.\,1862 Wien – 21.\,10.\,1931 ebd.@\textsc{Schnitzler, Arthur} (15.\,5.\,1862 Wien – 21.\,10.\,1931 ebd.), \emph{Schriftsteller, Mediziner}!Schleier der Beatrice. Schauspiel in fünf Akten@\strich\emph{Der Schleier der Beatrice. Schauspiel in fünf Akten}|pw} nächste Saison, Ritscher\pwindex{Ritscher, Helene 2.\,6.\,1888 Zalaegerszeg – 27.\,11.\,1964 Hamburg@\textsc{Ritscher, Helene} (2.\,6.\,1888 Zalaegerszeg – 27.\,11.\,1964 Hamburg), \emph{Schauspielerin}|pw} als Beatrice\pwindex{Schnitzler, Arthur 15.\,5.\,1862 Wien – 21.\,10.\,1931 ebd.@\textsc{Schnitzler, Arthur} (15.\,5.\,1862 Wien – 21.\,10.\,1931 ebd.), \emph{Schriftsteller, Mediziner}!Schleier der Beatrice. Schauspiel in fünf Akten@\strich\emph{Der Schleier der Beatrice. Schauspiel in fünf Akten}|pwv}. Meine Frage an dich: hältst dus 1) für wahrscheinlich, dass Reinhardt\pwindex{Reinhardt, Max 9.\,9.\,1873 Baden bei Wien – 30.\,10.\,1943 New York City@\textsc{Reinhardt, Max} (9.\,9.\,1873 Baden bei Wien – 30.\,10.\,1943 New York City), \emph{Theaterleiter, Regisseur, Schauspieler}|pw} auf die Beatrice\pwindex{Schnitzler, Arthur 15.\,5.\,1862 Wien – 21.\,10.\,1931 ebd.@\textsc{Schnitzler, Arthur} (15.\,5.\,1862 Wien – 21.\,10.\,1931 ebd.), \emph{Schriftsteller, Mediziner}!Schleier der Beatrice. Schauspiel in fünf Akten@\strich\emph{Der Schleier der Beatrice. Schauspiel in fünf Akten}|pw} reflectirt\strikeout{e}? 2) hältst
               du, im Jafalle Deutsches Theater\oindex{Deutsches Theater Berlin@\textbf{Deutsches Theater Berlin}, \emph{Theater}|pw} für praktischer
               als \strikeout{für}{ }Hebbeltheater\oindex{Hebbel-Theater@\textbf{Hebbel-Theater}, \emph{Theater}|pw}? 3) Zu welcher Zeit wäre Reinhardt\pwindex{Reinhardt, Max 9.\,9.\,1873 Baden bei Wien – 30.\,10.\,1943 New York City@\textsc{Reinhardt, Max} (9.\,9.\,1873 Baden bei Wien – 30.\,10.\,1943 New York City), \emph{Theaterleiter, Regisseur, Schauspieler}|pw} zu einer fixen Entscheidg zu
                  veranlassen?\strikeout{)}\pend
           
\pstart
           {\pb}– Du bist nicht böse,
               wenn ich dich nochmals um vollko{\geminationm}en \uline{vertrauliche}
               Behandlg der Angelegenheit ersuche.\pend
           
\pstart
           herzlichst der Deine,{\\[\baselineskip]}\spacefill\mbox{Arthur}\pend
           \leftskip=0em{}\selectlanguage{ngerman}\endnumbering\briefempfaengerindex{Bahr, Hermann@\textsc{Bahr, Hermann}!zzzSchnitzler, Arthur@\emph{von Arthur Schnitzler}!1907-12-161@{16. 12. 1907}|)be}\mylabel{L01741h}  \newcommand{\dateiname}{L01741}\newcommand{\titel}{Arthur Schnitzler an Hermann Bahr, 16. 12. 1907}\newcommand{\editorInnen}{Herausgegeben von Martin Anton Müller}%% latex-leseansicht-abspann.tex
%% Abspann für die Leseansicht.
%% Der Schalter \ifkorrekturansicht ist bereits durch den Vorspann gesetzt.

%% latex-abspann.tex
%% Gemeinsamer Abspann für Korrekturansicht und Leseansicht.
%% Setzt den Schalter \ifkorrekturansicht voraus (gesetzt in den
%% einbindenden Dateien latex-korrekturansicht-abspann.tex bzw.
%% latex-leseansicht-abspann.tex).
%% ---------------------------------------------------------------

\normalsize

% Das esempio-Environment wird nur in der Leseansicht benötigt
\ifkorrekturansicht\else
\newenvironment{esempio}[3]%
{
    \vspace{1.5ex}
    \rlap{\underline{#1}}
    \par
    \setlength{\parindent}{0cm}
    \nopagebreak
    \leftskip=#2cm
    \rightskip=#3cm
}
{
    \par
}
\fi

\doendnotes{C}
\bigskip
\vfill

\clearpage

\footnotesize

\ifkorrekturansicht
  \lohead{\textsc{register}}
\fi

% theindex-Environment neu definieren ohne reledmac
\makeatletter
\renewenvironment{theindex}{%
  \ifkorrekturansicht
    \section*{\indexname}%
  \else
    \subsubsection*{Index der erwähnten Entitäten}%
  \fi
  \setlength{\parindent}{0pt}%
  \setlength{\parskip}{0pt plus 0.3pt}%
  \let\item\@idxitem
}{%
  \ifkorrekturansicht\clearpage\fi
}
\makeatother

\IfFileExists{\jobname-pw.ind}{\input{\jobname-pw.ind}}{}

% Quellenangabe nur in der Leseansicht
\ifkorrekturansicht\else
% Fallback-Definitionen, falls die .tex-Datei \titel etc. nicht gesetzt hat
\providecommand{\titel}{}
\providecommand{\editorInnen}{}
\providecommand{\dateiname}{\jobname}

\vspace{3cm}

\vfill

\footnotesize
\textsc{Quelle}: \titel. Herausgegeben von {\editorInnen}. In: \emph{Arthur Schnitzler: Briefwechsel mit Autorinnen und Autoren}.
 Digitale Edition, https://schnitzler-briefe.acdh.oeaw.ac.at/{\dateiname}.html (Stand \today)
\fi

\end{document}


