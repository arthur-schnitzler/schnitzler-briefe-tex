%% latex-leseansicht-vorspann.tex
%% Vorspann für die Leseansicht.
%% Lädt die gemeinsame Datei latex-vorspann.tex mit nicht gesetztem Schalter.

\newif\ifkorrekturansicht
\korrekturansichtfalse

\input{../tex-inputs/latex-vorspann}


\section[Paul Goldmann an Arthur Schnitzler, 21. 12. {[}1895{]}]{L02760 Paul Goldmann an Arthur Schnitzler, 21. 12. [1895]}
\nopagebreak\mylabel{L02760v}
\rehead{ }\normalsize\beginnumbering\briefempfaengerindex{Schnitzler, Arthur@\textsc{Schnitzler, Arthur}!zzzGoldmann, Paul@\emph{von Paul Goldmann}!1895-12-212@{21. 12. [1895]}|(be}
\toendnotes[C]{\smallbreak\pagebreak[2]}
\correspDesc{Versand  durch Paul Goldmann am 21. 12. [1895] in Paris
\newline{}Erhalt  durch Arthur Schnitzler im Zeitraum [22. 12. 1895 – 26. 12. 1895?] in Wien}\toendnotes[C]{\smallbreak}
\Standort{DLA, A:Schnitzler, HS.NZ85.1.3165.}
\physDesc{Brief, 1 Blatt, 4 Seiten, 1332 Zeichen
\newline{}Handschrift: blaue Tinte, deutsche Kurrent
\newline{}Schnitzler: mit Bleistift das Jahr »95« vermerkt }
\pstart
           {\pb}\textcolor{gray}{\textbf{\textbf{Frankfurter Zeitung\orgindex{Frankfurter Zeitung@Frankfurter Zeitung|pw}}}}\pend
           
\pstart
           \textcolor{gray}{\textbf{(\begin{otherlanguage}{french}Gazette de Francfort\end{otherlanguage}\orgindex{Frankfurter Zeitung@Frankfurter Zeitung|pw}).}}\pend
           
\pstart
           \textcolor{gray}{\textbf{\textbf{\begin{otherlanguage}{french}Fondateur M. L.
                              Sonnemann\pwindex{Sonnemann, Leopold 29.\,10.\,1831 Höchberg – 30.\,10.\,1909 Frankfurt am Main@\textsc{Sonnemann, Leopold} (29.\,10.\,1831 Höchberg – 30.\,10.\,1909 Frankfurt am Main), \emph{Journalist, Herausgeber}|pw}\end{otherlanguage}.}}}\pend
           
\pstart
           \begin{otherlanguage}{french}\textcolor{gray}{\textbf{Journal politique, financier,}}\end{otherlanguage}\pend
           
\pstart
           \begin{otherlanguage}{french}\textcolor{gray}{\textbf{commercial et littéraire.}}\end{otherlanguage}\pend
           
\pstart
           \begin{otherlanguage}{french}\textcolor{gray}{\textbf{\textbf{Paraissant trois fois par jour.}}}\end{otherlanguage}\hfill \textsc{Paris\oindex{Paris@\textbf{Paris}, \emph{Hauptstadt}|pw}}, 21. December.\pend
           
\pstart
           \begin{otherlanguage}{french}\textcolor{gray}{\textbf{\textbf{Bureau à Paris\oindex{Paris@\textbf{Paris}, \emph{Hauptstadt}|pw}:}}}\end{otherlanguage}\pend
           
\pstart
           \begin{otherlanguage}{french}\textcolor{gray}{\textbf{\textbf{24. Rue Feydeau\oindex{rue Feydeau@\textbf{rue Feydeau}, \emph{Straße}|pw}.}}}\end{otherlanguage}\pend
           \vspace{0.5em}{\vspace{1\baselineskip}}
\pstart
           Schöne Geſchichte, mein lieber Freund! Ich bekomme
               eben Deinen Brief, die Viſitkarte iſt darin, das Geld iſt herausgenommen. Auf dem
               Umſchlag iſt ein Vermerk der franzöſiſchen Poſt\orgindex{Französische Post@Französische Post|pw}
               zu leſen, daß der Brief mit einer Öffnung von 2 Centimeter angekommen iſt, welche
               Öffnung die Poſt\orgindex{Französische Post@Französische Post|pw} gewiſſenhaft \strikeout{verklebt} verklebt hat – nachdem das Geld herausgenommen
               worden. Zu machen iſt da kaum etwas. Ich richte{ }ſofort eine Reclamation an die franzöſiſche Poſt\orgindex{Französische Post@Französische Post|pw}, wozu ich das Couvert brauche
               (ſonſt hätte ich dirs geſchickt). {\pb}Du{ }ſelbſt haſt
               hoffentlich{ }ſchon auf Grund meiner Depeſche reclamirt. Nützen wird es nichts; Gott
               weiß, wo in Europa\oindex{Europa@\textbf{Europa}|pw} das Geld{ }ſich jetzt
               herumtreibt. Die Poſt\orgindex{Französische Post@Französische Post|pw} iſt nicht haftbar; denn das
               Geld war nicht declarirt, und der Brief, wofür{ }ſie einzig haftet, iſt angekommen.
               Frage immerhin einen Advokaten, ob man nicht auf Grund der von der Poſt\orgindex{Französische Post@Französische Post|pw}{ }ſelbſt conſtatirten \uline{Beſchädigung} des Briefes einen Schadens-Anſpruch erheben kann. {\pb}Aber, Kind, welche Unvorſichtigkeit! 3 Goldſtücke im
               einfachen Couvert! Das \uline{muß} man ja{ }ſtehlen. Ich{ }ſelbſt
               würde es{ }ſtehlen, wenn ich Poſtbeamter wäre. Warum haſt Du mir keine Poſtanweiſung
               geſchickt? Das wäre{ }ſogar noch billiger geweſen.\pend
           
\pstart
           Ich ärgere mich furchtbar\substVorne{}\textsuperscript{.}\substDazwischen{},\substHinten{} und ich denke nach, ob ich nicht irgendwie daran{ }ſchuld bin\substVorne{}\textsuperscript{.}\substDazwischen{},\substHinten{} – aber nein, ich glaube nicht.\pend
           
\pstart
           {\pb}Was nun?\pend
           
\pstart
           Viele treue Grüße! {\\[\baselineskip]}Dein{\\[\baselineskip]}\spacefill\mbox{Paul Goldmnn.}\pend
           \leftskip=0em{}\selectlanguage{ngerman}\endnumbering\briefempfaengerindex{Schnitzler, Arthur@\textsc{Schnitzler, Arthur}!zzzGoldmann, Paul@\emph{von Paul Goldmann}!1895-12-212@{21. 12. [1895]}|)be}\mylabel{L02760h}  \newcommand{\dateiname}{L02760}\newcommand{\titel}{Paul Goldmann an Arthur Schnitzler, 21. 12. [1895]}\newcommand{\editorInnen}{Martin Anton Müller und Laura Untner}%% latex-leseansicht-abspann.tex
%% Abspann für die Leseansicht.
%% Der Schalter \ifkorrekturansicht ist bereits durch den Vorspann gesetzt.

%% latex-abspann.tex
%% Gemeinsamer Abspann für Korrekturansicht und Leseansicht.
%% Setzt den Schalter \ifkorrekturansicht voraus (gesetzt in den
%% einbindenden Dateien latex-korrekturansicht-abspann.tex bzw.
%% latex-leseansicht-abspann.tex).
%% ---------------------------------------------------------------

\normalsize

% Das esempio-Environment wird nur in der Leseansicht benötigt
\ifkorrekturansicht\else
\newenvironment{esempio}[3]%
{
    \vspace{1.5ex}
    \rlap{\underline{#1}}
    \par
    \setlength{\parindent}{0cm}
    \nopagebreak
    \leftskip=#2cm
    \rightskip=#3cm
}
{
    \par
}
\fi

\doendnotes{C}
\bigskip
\vfill

\clearpage

\footnotesize

\ifkorrekturansicht
  \lohead{\textsc{register}}
\fi

% theindex-Environment neu definieren ohne reledmac
\makeatletter
\renewenvironment{theindex}{%
  \ifkorrekturansicht
    \section*{\indexname}%
  \else
    \subsubsection*{Index der erwähnten Entitäten}%
  \fi
  \setlength{\parindent}{0pt}%
  \setlength{\parskip}{0pt plus 0.3pt}%
  \let\item\@idxitem
}{%
  \ifkorrekturansicht\clearpage\fi
}
\makeatother

\IfFileExists{\jobname-pw.ind}{\input{\jobname-pw.ind}}{}

% Quellenangabe nur in der Leseansicht
\ifkorrekturansicht\else
% Fallback-Definitionen, falls die .tex-Datei \titel etc. nicht gesetzt hat
\providecommand{\titel}{}
\providecommand{\editorInnen}{}
\providecommand{\dateiname}{\jobname}

\vspace{3cm}

\vfill

\footnotesize
\textsc{Quelle}: \titel. Herausgegeben von {\editorInnen}. In: \emph{Arthur Schnitzler: Briefwechsel mit Autorinnen und Autoren}.
 Digitale Edition, https://schnitzler-briefe.acdh.oeaw.ac.at/{\dateiname}.html (Stand \today)
\fi

\end{document}


