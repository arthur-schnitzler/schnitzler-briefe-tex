%% latex-leseansicht-vorspann.tex
%% Vorspann für die Leseansicht.
%% Lädt die gemeinsame Datei latex-vorspann.tex mit nicht gesetztem Schalter.

\newif\ifkorrekturansicht
\korrekturansichtfalse

\input{../tex-inputs/latex-vorspann}

\begin{center}
            \textcolor{red}{ENTWURF, NICHT FERTIG KORRIGIERT}
                      \end{center}
            
         
         \renewcommand{\erwaehntePersonen}{Personen: Paul Goldmann, Eva Marie Goldmann}
         \renewcommand{\erwaehnteInstitutionen}{Institutionen: Neue Freie Presse}
         \renewcommand{\erwaehnteOrte}{Orte: Bendlerstraße, Berlin, Velden am Wörthersee, Wien}
         \renewcommand{\erwaehnteWerke}{Werke: ?? [Roman mit erotischen Schilderungen]}
               \section[ Paul Goldmann an Arthur Schnitzler, 2. 8. 1931]{ Paul Goldmann an Arthur Schnitzler, 2. 8. 1931}\nopagebreak\mylabel{v}\rehead{ }\begin{ledgroupsized}[t]{13cm}\normalsize\beginnumbering\briefempfaengerindex{Schnitzler, Arthur@\textsc{Schnitzler, Arthur}!zzzGoldmann, Paul@\emph{von Paul Goldmann}!1931-08-021@{2. 8. 1931}|(be} \toendnotes[C]{\smallbreak\pagebreak[2]} \Standort{DLA, A:Schnitzler, HS.NZ85.1.3176.}
\physDesc{Brief, 1 Blatt, 1 Seite, 543 Zeichen
\newline{}Schreibmaschine
\newline{}Handschrift: lila Tinte, lateinische Kurrent (\noindent{}zwei Korrekturen, Schlussformel und Unterschrift)}\toendnotes[C]{\smallbreak}\pstart
           \noindent{}{\pb}\textcolor{gray}{\textbf{Dr. Paul Goldmann}}\hfill \textcolor{gray}{\textbf{Berlin W. 10\oindex{Berlin@\textbf{Berlin}|pw}}}\pend
           \pstart
           \textcolor{gray}{\textbf{Vertreter der »Neuen Freien
                           Presse\orgindex{Neue Freie Presse@Neue Freie Presse|pw}«}}\hfill \textcolor{gray}{\textbf{Bendlerſtraße 36\oindex{Bendlerstrasse@\textbf{Bendlerstraße}|pw}.}}\pend
           \pstart
           \raggedleft{}\textcolor{gray}{\textbf{Tel. B 2, Lützow 9142}}\pend
           \pstart
           \raggedleft{}2. 8. 31.\pend
           \pstart\center{}Lieber Freund,\pend\pstart
           Als rekommandierte Drucksache übersende ich Dir das \label{K_L03518-1v}\edtext{Buch\pwindex{?? Werk@Nicht ermittelte Verfasserinnen und Verfasser!?? [Roman mit erotischen Schilderungen]@\emph{?? [Roman mit erotischen Schilderungen]}|pwv}}{\lemma{\textnormal{\emph{Buch}}}\Cendnote{\textnormal{nicht ermittelt, siehe Paul Goldmann an Arthur Schnitzler, 19. 5. 1931}}}\label{K_L03518-1h}, das Du so freundlich warst, mir zu borgen, und ich danke Dir herzlich\introOben{}st\introOben{} dafür. Es hat mich sehr interessiert, und ich finde, dass
               es, auch abgesehen von seinen erotischen Schilderungen, verdient, gelesen zu
               werden.\pend
           \pstart
           Ich hoffe, dass Dich meine Sendungen nicht in Wien\oindex{Wien@\textbf{Wien}|pw}
               erreichen und dass Du an irgendeinem schönen Erholungsort Deinen Sommer verbringst.
               Ich gehe nächster Tage nach Velden am
               Wörthersee\oindex{Velden am Woerthersee@\textbf{Velden am Wörthersee}|pw}.\pend
           \pstart
           Mit vielen herzlichen Grüssen, auch von meiner Frau\pwindex{Goldmann, Eva Marie 27.10.1877 – 02.11.1937@\textsc{Goldmann, Eva Marie} (27.10.1877 – 02.11.1937)|pwv}, {\\[\baselineskip]}{[}hs.:{]} Dein {\\[\baselineskip]}\spacefill\mbox{Paul Goldmann\textcolor{gray}{.}}\pend
           \leftskip=0em{}
         
         \endnumbering\mylabel{h}\end{ledgroupsized}  \newcommand{\dateiname}{L03518}\newcommand{\titel}{Paul Goldmann an Arthur Schnitzler, 2. 8. 1931}\newcommand{\editorInnen}{Martin Anton Müller und Laura Untner}%% latex-leseansicht-abspann.tex
%% Abspann für die Leseansicht.
%% Der Schalter \ifkorrekturansicht ist bereits durch den Vorspann gesetzt.

%% latex-abspann.tex
%% Gemeinsamer Abspann für Korrekturansicht und Leseansicht.
%% Setzt den Schalter \ifkorrekturansicht voraus (gesetzt in den
%% einbindenden Dateien latex-korrekturansicht-abspann.tex bzw.
%% latex-leseansicht-abspann.tex).
%% ---------------------------------------------------------------

\normalsize

% Das esempio-Environment wird nur in der Leseansicht benötigt
\ifkorrekturansicht\else
\newenvironment{esempio}[3]%
{
    \vspace{1.5ex}
    \rlap{\underline{#1}}
    \par
    \setlength{\parindent}{0cm}
    \nopagebreak
    \leftskip=#2cm
    \rightskip=#3cm
}
{
    \par
}
\fi

\doendnotes{C}
\bigskip
\vfill

\clearpage

\footnotesize

\ifkorrekturansicht
  \lohead{\textsc{register}}
\fi

% theindex-Environment neu definieren ohne reledmac
\makeatletter
\renewenvironment{theindex}{%
  \ifkorrekturansicht
    \section*{\indexname}%
  \else
    \subsubsection*{Index der erwähnten Entitäten}%
  \fi
  \setlength{\parindent}{0pt}%
  \setlength{\parskip}{0pt plus 0.3pt}%
  \let\item\@idxitem
}{%
  \ifkorrekturansicht\clearpage\fi
}
\makeatother

\IfFileExists{\jobname-pw.ind}{\input{\jobname-pw.ind}}{}

% Quellenangabe nur in der Leseansicht
\ifkorrekturansicht\else
% Fallback-Definitionen, falls die .tex-Datei \titel etc. nicht gesetzt hat
\providecommand{\titel}{}
\providecommand{\editorInnen}{}
\providecommand{\dateiname}{\jobname}

\vspace{3cm}

\vfill

\footnotesize
\textsc{Quelle}: \titel. Herausgegeben von {\editorInnen}. In: \emph{Arthur Schnitzler: Briefwechsel mit Autorinnen und Autoren}.
 Digitale Edition, https://schnitzler-briefe.acdh.oeaw.ac.at/{\dateiname}.html (Stand \today)
\fi

\end{document}


      