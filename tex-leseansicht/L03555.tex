%% latex-leseansicht-vorspann.tex
%% Vorspann für die Leseansicht.
%% Lädt die gemeinsame Datei latex-vorspann.tex mit nicht gesetztem Schalter.

\newif\ifkorrekturansicht
\korrekturansichtfalse

\input{../tex-inputs/latex-vorspann}


\section[ Felix Salten an Arthur Schnitzler, [26. 1. 1912]]{L03555 Felix Salten an Arthur Schnitzler,  [26. 1. 1912]}
\nopagebreak\mylabel{L03555v}
\rehead{ }\normalsize\beginnumbering\briefempfaengerindex{Schnitzler, Arthur@\textsc{Schnitzler, Arthur}!zzzSalten, Felix@\emph{von Felix Salten}!1912-01-261@{{[}26. 1. 1912{]}}|(be}
\toendnotes[C]{\smallbreak\pagebreak[2]}
\correspDesc{Versand  durch Felix Salten am [26. 1. 1912] in Wien
\newline{}Erhalt  durch Arthur Schnitzler im Zeitraum [26. 1. 1912
                  – 29. 1. 1912?] in Wien}\toendnotes[C]{\smallbreak}
\Standort{CUL, Schnitzler, B 89, B 2.}
\physDesc{Briefkarte, 574 Zeichen
\newline{}Handschrift: Bleistift, lateinische Kurrent
\newline{}Schnitzler: mit Bleistift datiert: »26/1 912« 
\newline{}Ordnung: mit Bleistift von unbekannter Hand nummeriert: »270« }\toendnotes[C]{\smallbreak}
\pstart
           \centering{}{\pb}\textcolor{gray}{\textbf{\textsc{Felix Salten}}}\pend
           
\pstart
           \raggedleft{}Freitag.\pend
           
\pstart{}Lieber,\pend\vspace{0.5em}
\pstart
           \label{K_L03555-1v}\edtext{Bauer\pwindex{Bauer, Julius 15.\,10.\,1853 Szigetvár – 11.\,6.\,1941 Wien@\textsc{Bauer, Julius} (15.\,10.\,1853 Szigetvár – 11.\,6.\,1941 Wien), \emph{Schriftsteller, Journalist, Kritiker}|pw}}{\lemma{\textnormal{\emph{Bauer}}}\Cendnote{\textnormal{Julius Bauer\pwindex{Bauer, Julius 15.\,10.\,1853 Szigetvár – 11.\,6.\,1941 Wien@\textsc{Bauer, Julius} (15.\,10.\,1853 Szigetvár – 11.\,6.\,1941 Wien), \emph{Schriftsteller, Journalist, Kritiker}|pwk} bereitete die »Damenspende« des \emph{Concordia}\orgindex{Concordia. Journalisten- und Schriftstellerverein@Concordia. Journalisten- und Schriftstellerverein|pwk}balls am
                  12. 2. 1912 vor, die in diesem Jahr als \emph{Lessing-Almanach}\pwindex{Lessing Almanach@\emph{Lessing Almanach}|pwk} einen Beitrag
                  zur Gründung eines Lessing\pwindex{Lessing, Gotthold Ephraim 22.\,1.\,1729 Kamenz – 15.\,2.\,1781 Braunschweig@\textsc{Lessing, Gotthold Ephraim} (22.\,1.\,1729 Kamenz – 15.\,2.\,1781 Braunschweig), \emph{Schriftsteller, Kritiker, Philosoph}|pwk}-Museums in Wien\oindex{Wien@\textbf{Wien}, \emph{Verwaltungsgebiet}|pwk}
                  liefern sollte. Schnitzler steuerte einen Aphorismus\pwindex{Schnitzler, Arthur 15.\,5.\,1862 Wien – 21.\,10.\,1931 ebd.@\textsc{Schnitzler, Arthur} (15.\,5.\,1862 Wien – 21.\,10.\,1931 ebd.), \emph{Schriftsteller, Mediziner}!Um einer Partei anzugehören]@\strich\emph{[Um einer Partei anzugehören]}|pwkv} bei und folgte also der Bitte um Abänderung nicht, die im vorliegenden Schreiben
                  geäußert wird.}}}\label{K_L03555-1} wendet sich wieder einmal an mich. (weil
               Sie kein Telefon haben) Er bittet mich, Sie aufmerksam zu machen, dass Ihr Beitrag\pwindex{Schnitzler, Arthur 15.\,5.\,1862 Wien – 21.\,10.\,1931 ebd.@\textsc{Schnitzler, Arthur} (15.\,5.\,1862 Wien – 21.\,10.\,1931 ebd.), \emph{Schriftsteller, Mediziner}!Um einer Partei anzugehören]@\strich\emph{[Um einer Partei anzugehören]}|pwv} (für den er Ihnen
               bestens dankt) \introOben{}als\introOben{} der einzige, nicht auf Lessing\pwindex{Lessing, Gotthold Ephraim 22.\,1.\,1729 Kamenz – 15.\,2.\,1781 Braunschweig@\textsc{Lessing, Gotthold Ephraim} (22.\,1.\,1729 Kamenz – 15.\,2.\,1781 Braunschweig), \emph{Schriftsteller, Kritiker, Philosoph}|pw} zu beziehende dastehen würde in jener fabelhaften Ballspende\pwindex{Lessing Almanach@\emph{Lessing Almanach}|pwv}, welche durchaus Lessing\pwindex{Lessing, Gotthold Ephraim 22.\,1.\,1729 Kamenz – 15.\,2.\,1781 Braunschweig@\textsc{Lessing, Gotthold Ephraim} (22.\,1.\,1729 Kamenz – 15.\,2.\,1781 Braunschweig), \emph{Schriftsteller, Kritiker, Philosoph}|pw} gewidmet ist. Er läßt Sie bitten, ihm
                  heute oder morgen –
               weil es schon sehr eilt – irgend etwas Lessing\pwindex{Lessing, Gotthold Ephraim 22.\,1.\,1729 Kamenz – 15.\,2.\,1781 Braunschweig@\textsc{Lessing, Gotthold Ephraim} (22.\,1.\,1729 Kamenz – 15.\,2.\,1781 Braunschweig), \emph{Schriftsteller, Kritiker, Philosoph}|pw}-sagendes zu spenden. Und er wird dann, um Ihre Antwort zu hören, bei
               mir anrufen. (Weil Sie kein Telefon u. s. w.)\pend
           
\pstart
           Auf baldiges Wiedersehen u. herzlichste Grüße von Haus zu Haus {\\[\baselineskip]}Ihr {\\[\baselineskip]}\spacefill\mbox{Salten}\pend
           \leftskip=0em{}\selectlanguage{ngerman}\endnumbering\briefempfaengerindex{Schnitzler, Arthur@\textsc{Schnitzler, Arthur}!zzzSalten, Felix@\emph{von Felix Salten}!1912-01-261@{{[}26. 1. 1912{]}}|)be}\mylabel{L03555h}  \newcommand{\dateiname}{L03555}\newcommand{\titel}{Felix Salten an Arthur Schnitzler, [26. 1. 1912]}\newcommand{\editorInnen}{Martin Anton Müller und Laura Untner}%% latex-leseansicht-abspann.tex
%% Abspann für die Leseansicht.
%% Der Schalter \ifkorrekturansicht ist bereits durch den Vorspann gesetzt.

%% latex-abspann.tex
%% Gemeinsamer Abspann für Korrekturansicht und Leseansicht.
%% Setzt den Schalter \ifkorrekturansicht voraus (gesetzt in den
%% einbindenden Dateien latex-korrekturansicht-abspann.tex bzw.
%% latex-leseansicht-abspann.tex).
%% ---------------------------------------------------------------

\normalsize

% Das esempio-Environment wird nur in der Leseansicht benötigt
\ifkorrekturansicht\else
\newenvironment{esempio}[3]%
{
    \vspace{1.5ex}
    \rlap{\underline{#1}}
    \par
    \setlength{\parindent}{0cm}
    \nopagebreak
    \leftskip=#2cm
    \rightskip=#3cm
}
{
    \par
}
\fi

\doendnotes{C}
\bigskip
\vfill

\clearpage

\footnotesize

\ifkorrekturansicht
  \lohead{\textsc{register}}
\fi

% theindex-Environment neu definieren ohne reledmac
\makeatletter
\renewenvironment{theindex}{%
  \ifkorrekturansicht
    \section*{\indexname}%
  \else
    \subsubsection*{Index der erwähnten Entitäten}%
  \fi
  \setlength{\parindent}{0pt}%
  \setlength{\parskip}{0pt plus 0.3pt}%
  \let\item\@idxitem
}{%
  \ifkorrekturansicht\clearpage\fi
}
\makeatother

\IfFileExists{\jobname-pw.ind}{\input{\jobname-pw.ind}}{}

% Quellenangabe nur in der Leseansicht
\ifkorrekturansicht\else
% Fallback-Definitionen, falls die .tex-Datei \titel etc. nicht gesetzt hat
\providecommand{\titel}{}
\providecommand{\editorInnen}{}
\providecommand{\dateiname}{\jobname}

\vspace{3cm}

\vfill

\footnotesize
\textsc{Quelle}: \titel. Herausgegeben von {\editorInnen}. In: \emph{Arthur Schnitzler: Briefwechsel mit Autorinnen und Autoren}.
 Digitale Edition, https://schnitzler-briefe.acdh.oeaw.ac.at/{\dateiname}.html (Stand \today)
\fi

\end{document}


