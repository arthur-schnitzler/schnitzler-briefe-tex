%% latex-korrekturansicht-vorspann.tex
%% Vorspann für die Korrekturansicht.
%% Lädt die gemeinsame Datei latex-vorspann.tex mit gesetztem Schalter.

\newif\ifkorrekturansicht
\korrekturansichttrue

\input{../tex-inputs/latex-vorspann}


\section[ Felix Salten an Arthur Schnitzler, {[}26. 1. 1912{]}]{L03555 Felix Salten an Arthur Schnitzler, {[}26. 1. 1912{]}}
\nopagebreak\mylabel{L03555v}
\rehead{ }\normalsize\beginnumbering\briefempfaengerindex{Schnitzler, Arthur@\textsc{Schnitzler, Arthur}!zzzSalten, Felix@\emph{von Felix Salten}!1912-01-261@{{[}26. 1. 1912{]}}|(be}
\toendnotes[C]{\smallbreak\pagebreak[2]}\Standort{CUL, Schnitzler, B 89, B 2.}
\physDesc{Briefkarte, 574 Zeichen
\newline{}Handschrift: Bleistift, lateinische Kurrent
\newline{}Schnitzler: mit Bleistift datiert: »26/1 912« 
\newline{}Ordnung: mit Bleistift von unbekannter Hand nummeriert: »270« }\toendnotes[C]{\smallbreak}
\pstart
           \centering{}{\pb}\textcolor{gray}{\textbf{\textsc{Felix Salten}}}\pend
           
\pstart
           \raggedleft{}Freitag.\pend
           
\pstart{}Lieber,\pend\vspace{0.5em}
\pstart
           \label{K_L03555-1v}\edtext{Bauer\pwindex{Bauer, Julius 15.10.1853 – 11.06.1941@\textsc{Bauer, Julius} (15.10.1853 – 11.06.1941), \emph{Schriftsteller/Schriftstellerin, Journalist/Journalistin, Kritiker/Kritikerin}|pw}}{\lemma{\textnormal{\emph{Bauer}}}\Cendnote{\textnormal{Julius Bauer\pwindex{Bauer, Julius 15.10.1853 – 11.06.1941@\textsc{Bauer, Julius} (15.10.1853 – 11.06.1941), \emph{Schriftsteller/Schriftstellerin, Journalist/Journalistin, Kritiker/Kritikerin}|pwk} bereitete die »Damenspende« des \emph{Concordia}\orgindex{Concordia. Journalisten- und Schriftstellerverein@Concordia. Journalisten- und Schriftstellerverein|pwk}balls am
                  12. 2. 1912 vor, die in diesem Jahr als \emph{Lessing-Almanach}\pwindex{Lessing Almanach@\emph{Lessing Almanach}|pwk} einen Beitrag
                  zur Gründung eines Lessing\pwindex{Lessing, Gotthold Ephraim 22.01.1729 – 15.02.1781@\textsc{Lessing, Gotthold Ephraim} (22.01.1729 – 15.02.1781), \emph{Schriftsteller/Schriftstellerin, Kritiker/Kritikerin, Philosoph/Philosophin}|pwk}-Museums in Wien\oindex{Wien@\textbf{Wien}, \emph{A.ADM2}|pwk}
                  liefern sollte. Schnitzler steuerte einen Aphorismus\pwindex{Um einer Partei anzugehoeren]@\emph{[Um einer Partei anzugehören]}|pwkv} bei und folgte also der Bitte um Abänderung nicht, die im vorliegenden Schreiben
                  geäußert wird.}}}\label{K_L03555-1} wendet sich wieder einmal an mich. (weil
               Sie kein Telefon haben) Er bittet mich, Sie aufmerksam zu machen, dass Ihr Beitrag\pwindex{Um einer Partei anzugehoeren]@\emph{[Um einer Partei anzugehören]}|pwv} (für den er Ihnen
               bestens dankt) \introOben{}als\introOben{} der einzige, nicht auf Lessing\pwindex{Lessing, Gotthold Ephraim 22.01.1729 – 15.02.1781@\textsc{Lessing, Gotthold Ephraim} (22.01.1729 – 15.02.1781), \emph{Schriftsteller/Schriftstellerin, Kritiker/Kritikerin, Philosoph/Philosophin}|pw} zu beziehende dastehen würde in jener fabelhaften Ballspende\pwindex{Lessing Almanach@\emph{Lessing Almanach}|pwv}, welche durchaus Lessing\pwindex{Lessing, Gotthold Ephraim 22.01.1729 – 15.02.1781@\textsc{Lessing, Gotthold Ephraim} (22.01.1729 – 15.02.1781), \emph{Schriftsteller/Schriftstellerin, Kritiker/Kritikerin, Philosoph/Philosophin}|pw} gewidmet ist. Er läßt Sie bitten, ihm
                  heute oder morgen –
               weil es schon sehr eilt – irgend etwas Lessing\pwindex{Lessing, Gotthold Ephraim 22.01.1729 – 15.02.1781@\textsc{Lessing, Gotthold Ephraim} (22.01.1729 – 15.02.1781), \emph{Schriftsteller/Schriftstellerin, Kritiker/Kritikerin, Philosoph/Philosophin}|pw}-sagendes zu spenden. Und er wird dann, um Ihre Antwort zu hören, bei
               mir anrufen. (Weil Sie kein Telefon u. s. w.)\pend
           
\pstart
           Auf baldiges Wiedersehen u. herzlichste Grüße von Haus zu Haus {\\[\baselineskip]}Ihr {\\[\baselineskip]}\spacefill\mbox{Salten}\pend
           \leftskip=0em{}\selectlanguage{ngerman}\endnumbering\briefempfaengerindex{Schnitzler, Arthur@\textsc{Schnitzler, Arthur}!zzzSalten, Felix@\emph{von Felix Salten}!1912-01-261@{{[}26. 1. 1912{]}}|)be}\mylabel{L03555h}  \normalsize

\doendnotes{C}
\bigskip
\vfill

\clearpage

\footnotesize

\lohead{\textsc{register}}

% Definiere theindex-Environment komplett neu ohne reledmac
\makeatletter
\renewenvironment{theindex}{%
  \section*{\indexname}%
  \setlength{\parindent}{0pt}%
  \setlength{\parskip}{0pt plus 0.3pt}%
  \let\item\@idxitem
}{%
  \clearpage
}
\makeatother

\IfFileExists{\jobname-pw.ind}{\input{\jobname-pw.ind}}{}

\end{document}

      