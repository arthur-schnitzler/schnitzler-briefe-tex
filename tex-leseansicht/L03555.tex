%% latex-leseansicht-vorspann.tex
%% Vorspann für die Leseansicht.
%% Lädt die gemeinsame Datei latex-vorspann.tex mit nicht gesetztem Schalter.

\newif\ifkorrekturansicht
\korrekturansichtfalse

\input{../tex-inputs/latex-vorspann}

\begin{center}
            \textcolor{red}{ENTWURF, NICHT FERTIG KORRIGIERT}
                      \end{center}
            
         
         \renewcommand{\erwaehntePersonen}{Personen: Julius Bauer, Gotthold Ephraim Lessing}
         \renewcommand{\erwaehnteOrte}{Orte: Wien}
         \renewcommand{\erwaehnteWerke}{Werke: Lessing Almanach, [Um einer Partei anzugehören]}
               \section[Felix Salten an Arthur Schnitzler, {[}26. 1. 1912{]}]{ Felix Salten an Arthur Schnitzler, {[}26. 1. 1912{]}}\nopagebreak\mylabel{v}\rehead{ }\begin{ledgroupsized}[t]{13cm}\normalsize\beginnumbering \toendnotes[C]{\smallbreak\pagebreak[2]} \Standort{CUL, Schnitzler, B 89, B 2.}
\physDesc{Briefkarte, 583 Zeichen
\newline{}Handschrift: Bleistift, lateinische Kurrent
\newline{}Schnitzler: mit Bleistift datiert:
                                          »26/1 {[}1{]}912« 
\newline{}Ordnung: mit Bleistift von unbekannter Hand nummeriert:
                                    »270« }\toendnotes[C]{\smallbreak}\pstart
           \noindent{}\centering{}{\pb}\textcolor{gray}{\textbf{\textsc{Felix Salten}}}\pend
           \pstart
           \raggedleft{}Freitag. \pend
           \pstart{}Lieber,\pend\pstart
           Bauer\pwindex{Bauer, Julius 15.10.1853 – 11.06.1941@\textsc{Bauer, Julius} (15.10.1853 – 11.06.1941), \emph{Schriftsteller, Journalist, Kritiker}|pw} wendet sich wieder einmal an mich, (weil
               Sie kein Telefon haben.) Er bittet mich, Sie aufmerksam zu machen, dass Ihr Beitrag\pwindex{Schnitzler, Arthur 15.05.1862 – 21.10.1931@\textsc{Schnitzler, Arthur} (15.05.1862 – 21.10.1931), \emph{Schriftsteller, Mediziner}!Um einer Partei anzugehoeren]1912@\strich\emph{[Um einer Partei anzugehören]} {[}1912{]}|pwv} (für den er Ihnen
               bestes dankt) \introOben{}als\introOben{} der einzige, nicht auf Lessing\pwindex{Lessing, Gotthold Ephraim 22.01.1729 – 15.02.1781@\textsc{Lessing, Gotthold Ephraim} (22.01.1729 – 15.02.1781), \emph{Schriftsteller, Bibliothekar}|pw} zu beziehende darstellen würde in jener fabelhaften
                  Ballspende\pwindex{Lessing Almanach1912@\emph{Lessing Almanach} {[}1912{]}|pwv}, welche durchaus
                  Lessing\pwindex{Lessing, Gotthold Ephraim 22.01.1729 – 15.02.1781@\textsc{Lessing, Gotthold Ephraim} (22.01.1729 – 15.02.1781), \emph{Schriftsteller, Bibliothekar}|pw} gewidmet ist. Er läßt Sie bitten,
               ihm heute oder morgen – weil es schon sehr eilt – irgend etwas Lessing\pwindex{Lessing, Gotthold Ephraim 22.01.1729 – 15.02.1781@\textsc{Lessing, Gotthold Ephraim} (22.01.1729 – 15.02.1781), \emph{Schriftsteller, Bibliothekar}|pw}-sagendes zu spenden. Und er er wird dann, um Ihre
               Antwort zu hören, bei mir anrufen. (Weil Sie kein Telefon u. s. w.)\pend
           \pstart
           Auf baldiges Wiedersehen u. herzlichste Grüße von Haus zu Haus {\\[\baselineskip]}Ihr {\\[\baselineskip]}\spacefill\mbox{Salten}\pend
           \leftskip=0em{}
         
         \endnumbering\mylabel{h}\end{ledgroupsized}\begin{anhang}\end{anhang}\newcommand{\dateiname}{L03555}\newcommand{\titel}{Felix Salten an Arthur Schnitzler, [26. 1. 1912]}\newcommand{\editorInnen}{Martin Anton Müller und Laura Untner}%% latex-leseansicht-abspann.tex
%% Abspann für die Leseansicht.
%% Der Schalter \ifkorrekturansicht ist bereits durch den Vorspann gesetzt.

%% latex-abspann.tex
%% Gemeinsamer Abspann für Korrekturansicht und Leseansicht.
%% Setzt den Schalter \ifkorrekturansicht voraus (gesetzt in den
%% einbindenden Dateien latex-korrekturansicht-abspann.tex bzw.
%% latex-leseansicht-abspann.tex).
%% ---------------------------------------------------------------

\normalsize

% Das esempio-Environment wird nur in der Leseansicht benötigt
\ifkorrekturansicht\else
\newenvironment{esempio}[3]%
{
    \vspace{1.5ex}
    \rlap{\underline{#1}}
    \par
    \setlength{\parindent}{0cm}
    \nopagebreak
    \leftskip=#2cm
    \rightskip=#3cm
}
{
    \par
}
\fi

\doendnotes{C}
\bigskip
\vfill

\clearpage

\footnotesize

\ifkorrekturansicht
  \lohead{\textsc{register}}
\fi

% theindex-Environment neu definieren ohne reledmac
\makeatletter
\renewenvironment{theindex}{%
  \ifkorrekturansicht
    \section*{\indexname}%
  \else
    \subsubsection*{Index der erwähnten Entitäten}%
  \fi
  \setlength{\parindent}{0pt}%
  \setlength{\parskip}{0pt plus 0.3pt}%
  \let\item\@idxitem
}{%
  \ifkorrekturansicht\clearpage\fi
}
\makeatother

\IfFileExists{\jobname-pw.ind}{\input{\jobname-pw.ind}}{}

% Quellenangabe nur in der Leseansicht
\ifkorrekturansicht\else
% Fallback-Definitionen, falls die .tex-Datei \titel etc. nicht gesetzt hat
\providecommand{\titel}{}
\providecommand{\editorInnen}{}
\providecommand{\dateiname}{\jobname}

\vspace{3cm}

\vfill

\footnotesize
\textsc{Quelle}: \titel. Herausgegeben von {\editorInnen}. In: \emph{Arthur Schnitzler: Briefwechsel mit Autorinnen und Autoren}.
 Digitale Edition, https://schnitzler-briefe.acdh.oeaw.ac.at/{\dateiname}.html (Stand \today)
\fi

\end{document}


      