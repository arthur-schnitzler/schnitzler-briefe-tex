%% latex-leseansicht-vorspann.tex
%% Vorspann für die Leseansicht.
%% Lädt die gemeinsame Datei latex-vorspann.tex mit nicht gesetztem Schalter.

\newif\ifkorrekturansicht
\korrekturansichtfalse

\input{../tex-inputs/latex-vorspann}


         
         \renewcommand{\erwaehntePersonen}{Personen: Paul Goldmann, Theodore Rottenberg}
         \renewcommand{\erwaehnteOrte}{Orte: Berlin, Frankfurt am Main, Prag, Wien}
         \renewcommand{\erwaehnteWerke}{}
               \section[ Paul Goldmann und Theodore Rottenberg an Arthur Schnitzler, {[}1{]}9. 3. {[}1902{]}]{ Paul Goldmann und Theodore Rottenberg an Arthur
               Schnitzler, {[}1{]}9. 3. {[}1902{]}}\nopagebreak\mylabel{v}\rehead{ }\begin{ledgroupsized}[t]{13cm}\normalsize\beginnumbering\briefempfaengerindex{Schnitzler, Arthur@\textsc{Schnitzler, Arthur}!zzzRottenberg, Theodore@\emph{von Theodore Rottenberg}!1902-03-191@{{[}1{]}9. 3. {[}1902{]}}|(be}\briefempfaengerindex{Schnitzler, Arthur@\textsc{Schnitzler, Arthur}!zzzGoldmann, Paul@\emph{von Paul Goldmann}!1902-03-191@{{[}1{]}9. 3. {[}1902{]}}|(be} \toendnotes[C]{\smallbreak\pagebreak[2]} \Standort{DLA, A:Schnitzler, HS.NZ85.1.3172.}
\physDesc{Brief, 1 Blatt, 1 Seite, 147 Zeichen
\newline{}Handschrift Paul Goldmann: schwarze Tinte, deutsche Kurrent\newline{}Handschrift Theodore Rottenberg: schwarze Tinte, deutsche Kurrent
\newline{}Schnitzler: mit Bleistift das Jahr »90\textcolor{gray}{2}« vermerkt }\toendnotes[C]{\smallbreak}\pstart
           \raggedleft{}{\pb}Berlin\oindex{Berlin@\textbf{Berlin}|pw}, \label{K_L03199-1v}\edtext{\textcolor{gray}{1}9. März}{\lemma{\textnormal{\emph{19. März}}}\Cendnote{\textnormal{Die
                        erste Ziffer der Datumsangabe ist nicht mit Sicherheit zu lesen. Da Goldmann\pwindex{Goldmann, Paul 31.01.1865 – 25.09.1935@\textsc{Goldmann, Paul} (31.01.1865 – 25.09.1935), \emph{Schriftsteller, Journalist}|pwk} am 20. 3. [1902] von der
                        Anwesenheit Rottenbergs\pwindex{Rottenberg, Theodore 1875-09-07 – 1945-04-05@\textsc{Rottenberg, Theodore} (1875-09-07 – 1945-04-05)|pwk} in Berlin\oindex{Berlin@\textbf{Berlin}|pwk} schrieb und sich am 29. 3. 1902 in Prag\oindex{Prag@\textbf{Prag}|pwk} aufhielt, scheint die Datierung auf
                        den 19. 3. 1902 aber verlässlich.}}}\label{K_L03199-1h}.\pend
           \pstart
           Es grüßen Dich, mein lieber Freund, Zwei, die ſich lieb haben, nämlich\pend
           \pstart
           1.) \textsc{Paul Goldmann}\pend
           \pstart
           2.) {[}hs. Rottenberg:{]} Frau {\dotsfour} aus Frankfurt\oindex{Frankfurt am Main@\textbf{Frankfurt am Main}|pw}\textcolor{gray}{,} (Sie wissen
               ſchon!)\pend
           
         
         \endnumbering\mylabel{h}\end{ledgroupsized}  \newcommand{\dateiname}{L03199}\newcommand{\titel}{Paul Goldmann und Theodore Rottenberg an Arthur Schnitzler, [1]9. 3. [1902]}\newcommand{\editorInnen}{Martin Anton Müller und Laura Untner}%% latex-leseansicht-abspann.tex
%% Abspann für die Leseansicht.
%% Der Schalter \ifkorrekturansicht ist bereits durch den Vorspann gesetzt.

%% latex-abspann.tex
%% Gemeinsamer Abspann für Korrekturansicht und Leseansicht.
%% Setzt den Schalter \ifkorrekturansicht voraus (gesetzt in den
%% einbindenden Dateien latex-korrekturansicht-abspann.tex bzw.
%% latex-leseansicht-abspann.tex).
%% ---------------------------------------------------------------

\normalsize

% Das esempio-Environment wird nur in der Leseansicht benötigt
\ifkorrekturansicht\else
\newenvironment{esempio}[3]%
{
    \vspace{1.5ex}
    \rlap{\underline{#1}}
    \par
    \setlength{\parindent}{0cm}
    \nopagebreak
    \leftskip=#2cm
    \rightskip=#3cm
}
{
    \par
}
\fi

\doendnotes{C}
\bigskip
\vfill

\clearpage

\footnotesize

\ifkorrekturansicht
  \lohead{\textsc{register}}
\fi

% theindex-Environment neu definieren ohne reledmac
\makeatletter
\renewenvironment{theindex}{%
  \ifkorrekturansicht
    \section*{\indexname}%
  \else
    \subsubsection*{Index der erwähnten Entitäten}%
  \fi
  \setlength{\parindent}{0pt}%
  \setlength{\parskip}{0pt plus 0.3pt}%
  \let\item\@idxitem
}{%
  \ifkorrekturansicht\clearpage\fi
}
\makeatother

\IfFileExists{\jobname-pw.ind}{\input{\jobname-pw.ind}}{}

% Quellenangabe nur in der Leseansicht
\ifkorrekturansicht\else
% Fallback-Definitionen, falls die .tex-Datei \titel etc. nicht gesetzt hat
\providecommand{\titel}{}
\providecommand{\editorInnen}{}
\providecommand{\dateiname}{\jobname}

\vspace{3cm}

\vfill

\footnotesize
\textsc{Quelle}: \titel. Herausgegeben von {\editorInnen}. In: \emph{Arthur Schnitzler: Briefwechsel mit Autorinnen und Autoren}.
 Digitale Edition, https://schnitzler-briefe.acdh.oeaw.ac.at/{\dateiname}.html (Stand \today)
\fi

\end{document}


      