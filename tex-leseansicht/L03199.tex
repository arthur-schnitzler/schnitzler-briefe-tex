%% latex-korrekturansicht-vorspann.tex
%% Vorspann für die Korrekturansicht.
%% Lädt die gemeinsame Datei latex-vorspann.tex mit gesetztem Schalter.

\newif\ifkorrekturansicht
\korrekturansichttrue

\input{../tex-inputs/latex-vorspann}


\section[ Paul Goldmann und Theodore Rottenberg an Arthur Schnitzler, {[}1{]}9. 3. {[}1902{]}]{L03199 Paul Goldmann und Theodore Rottenberg an Arthur
               Schnitzler, {[}1{]}9. 3. {[}1902{]}}
\nopagebreak\mylabel{L03199v}
\rehead{ }\normalsize\beginnumbering\briefempfaengerindex{Schnitzler, Arthur@\textsc{Schnitzler, Arthur}!zzzRottenberg, Theodore@\emph{von Theodore Rottenberg}!1902-03-191@{{[}1{]}9. 3. {[}1902{]}}|(be}\briefempfaengerindex{Schnitzler, Arthur@\textsc{Schnitzler, Arthur}!zzzGoldmann, Paul@\emph{von Paul Goldmann}!1902-03-191@{{[}1{]}9. 3. {[}1902{]}}|(be}
\toendnotes[C]{\smallbreak\pagebreak[2]}\Standort{DLA, A:Schnitzler, HS.NZ85.1.3172.}
\physDesc{Brief, 1 Blatt, 1 Seite, 147 Zeichen
\newline{}Handschrift Paul Goldmann: schwarze Tinte, deutsche Kurrent
\newline{}Handschrift Theodore Rottenberg: schwarze Tinte, deutsche Kurrent
\newline{}Schnitzler: mit Bleistift das Jahr »90\textcolor{gray}{2}« vermerkt }\toendnotes[C]{\smallbreak}
\pstart
           \raggedleft{}{\pb}Berlin\oindex{Berlin@\textbf{Berlin}, \emph{P.PPLC}|pw}, \label{K_L03199-1v}\edtext{\textcolor{gray}{1}9. März}{\lemma{\textnormal{\emph{19. März}}}\Cendnote{\textnormal{Die
                        erste Ziffer der Datumsangabe ist nicht mit Sicherheit zu lesen. Da Goldmann\pwindex{Goldmann, Paul 31.01.1865 – 25.09.1935@\textsc{Goldmann, Paul} (31.01.1865 – 25.09.1935), \emph{Schriftsteller/Schriftstellerin, Journalist/Journalistin}|pwk} am 20. 3. [1902] von der
                        Anwesenheit Rottenbergs\pwindex{Rottenberg, Theodore 1875-09-07 – 1945-04-05@\textsc{Rottenberg, Theodore} (1875-09-07 – 1945-04-05)|pwk} in Berlin\oindex{Berlin@\textbf{Berlin}, \emph{P.PPLC}|pwk} schrieb und sich am 29. 3. 1902 in Prag\oindex{Prag@\textbf{Prag}, \emph{A.ADM1}|pwk} aufhielt, scheint die Datierung auf
                        den 19. 3. 1902 aber verlässlich.}}}\label{K_L03199-1}.\pend
           \vspace{0.5em}
\pstart
           Es grüßen Dich, mein lieber Freund, Zwei, die ſich lieb haben, nämlich\pend
           
\pstart
           1.) \textsc{Paul Goldmann}\pend
           
\pstart
           2.) {[}hs. :{]} Frau {\dotsfour} aus Frankfurt\oindex{Frankfurt am Main@\textbf{Frankfurt am Main}, \emph{P.PPLA3}|pw}\textcolor{gray}{,} (Sie wissen
               ſchon!)\pend
           \selectlanguage{ngerman}\endnumbering\briefempfaengerindex{Schnitzler, Arthur@\textsc{Schnitzler, Arthur}!zzzRottenberg, Theodore@\emph{von Theodore Rottenberg}!1902-03-191@{{[}1{]}9. 3. {[}1902{]}}|)be}\briefempfaengerindex{Schnitzler, Arthur@\textsc{Schnitzler, Arthur}!zzzGoldmann, Paul@\emph{von Paul Goldmann}!1902-03-191@{{[}1{]}9. 3. {[}1902{]}}|)be}\mylabel{L03199h}  \normalsize

\doendnotes{C}
\bigskip
\vfill

\clearpage

\footnotesize

\lohead{\textsc{register}}

% Definiere theindex-Environment komplett neu ohne reledmac
\makeatletter
\renewenvironment{theindex}{%
  \section*{\indexname}%
  \setlength{\parindent}{0pt}%
  \setlength{\parskip}{0pt plus 0.3pt}%
  \let\item\@idxitem
}{%
  \clearpage
}
\makeatother

\IfFileExists{\jobname-pw.ind}{\input{\jobname-pw.ind}}{}

\end{document}

      