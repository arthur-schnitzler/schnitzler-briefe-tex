%% latex-korrekturansicht-vorspann.tex
%% Vorspann für die Korrekturansicht.
%% Lädt die gemeinsame Datei latex-vorspann.tex mit gesetztem Schalter.

\newif\ifkorrekturansicht
\korrekturansichttrue

\input{../tex-inputs/latex-vorspann}


\section[Arthur Schnitzler an Richard Beer-Hofmann, 28. 11. 1908]{L01811 Arthur Schnitzler an Richard Beer-Hofmann, 28. 11. 1908}
\nopagebreak\mylabel{L01811v}
\rehead{ }\normalsize\beginnumbering\briefempfaengerindex{Beer-Hofmann, Richard@\textsc{Beer-Hofmann, Richard}!zzzSchnitzler, Arthur@\emph{von Arthur Schnitzler}!1908-11-281@{28. 11. 1908}|(be}
\toendnotes[C]{\smallbreak\pagebreak[2]}\Standort{YCGL, MSS 31.}
\physDesc{Brief, 1 Blatt, 4 Seiten, Umschlag, 616 Zeichen
\newline{}Handschrift: Bleistift, deutsche Kurrent
\newline{}Versand: ohne postalischen Übermittlungsvermerk }
\buchAbdrucke{\weitereDrucke{Arthur Schnitzler, Richard Beer-Hofmann: \emph{Briefwechsel 1891–1931}. Wien, Zürich: \emph{Europaverlag} 1992, S. 191–192.} }\toendnotes[C]{\smallbreak}\pstart{}{\pb}\textcolor{gray}{\textbf{Dr. Arthur Schnitzler}}\pend{}\pstart{}\textcolor{gray}{\textbf{Wien XVIII. Spoettelgasse 7\oindex{Edmund-Weiss-Gasse 7@\textbf{Edmund-Weiß-Gasse 7}, \emph{Wohngebäude (K.WHS)}|pw}.}}\pend{}{\bigskip}\pstart{}{\pb}\textsc{Dr. Richard Beer Hofma{\geminationn}}\pend{}\pstart{}Wien\oindex{Wien@\textbf{Wien}, \emph{A.ADM2}|pw}\pend{}{\bigskip}\vspace{1em}
\pstart
           {\pb}\textcolor{gray}{\textbf{Dr. Arthur Schnitzler}}\hfill 28/11 08\pend
           
\pstart
           \textcolor{gray}{\textbf{Wien XVIII. Spoettelgasse 7\oindex{Edmund-Weiss-Gasse 7@\textbf{Edmund-Weiß-Gasse 7}, \emph{Wohngebäude (K.WHS)}|pw}.}}\pend
           
\pstart{}lieber Richard,\pend\vspace{0.5em}
\pstart
           we{\geminationn}{ }\textsc{Kerr}\pwindex{Kerr, Alfred 25.12.1867 – 12.10.1948@\textsc{Kerr, Alfred} (25.12.1867 – 12.10.1948), \emph{Schriftsteller/Schriftstellerin, Kritiker/Kritikerin}|pw} jetzt bei Ihnen iſt (er war gegen 1 bei mir ohne mich zu treffen)
               ſo fragen Sie ihn bitte, wie lang er hier bleibt und arrangiren Sie es {\pb}womöglich daſs wir morgen nach der \label{K_L01811-1v}\edtext{Heine\pwindex{Heine, Heinrich 13.12.1797 – 17.02.1856@\textsc{Heine, Heinrich} (13.12.1797 – 17.02.1856), \emph{Schriftsteller/Schriftstellerin}|pw}{ }Sache}{\lemma{\textnormal{\emph{Heine Sache}}}\Cendnote{\textnormal{Am 29. 11. 1908 fand im Bösendorfer-Saal\oindex{Boesendorfer-Saal@\textbf{Bösendorfer-Saal}, \emph{Veranstaltungsgebäude (K.VSB)}|pwk} die Heine\pwindex{Heine, Heinrich 13.12.1797 – 17.02.1856@\textsc{Heine, Heinrich} (13.12.1797 – 17.02.1856), \emph{Schriftsteller/Schriftstellerin}|pwk}-Feier
                  des \emph{Vereins für Kunst und Kultur}\orgindex{Verein fuer Kunst und Kultur@Verein für Kunst und Kultur|pwk} statt. Alfred Kerr\pwindex{Kerr, Alfred 25.12.1867 – 12.10.1948@\textsc{Kerr, Alfred} (25.12.1867 – 12.10.1948), \emph{Schriftsteller/Schriftstellerin, Kritiker/Kritikerin}|pwk} hielt zu Beginn der
                  Veranstaltung einen Vortrag über Heinrich Heine\pwindex{Heine, Heinrich 13.12.1797 – 17.02.1856@\textsc{Heine, Heinrich} (13.12.1797 – 17.02.1856), \emph{Schriftsteller/Schriftstellerin}|pwk}. Schnitzler war anwesend, anschließend
                  speisten sie im Meissl {\kaufmannsund} Schadn\oindex{Meissl {\kaufmannsund} Schadn@\textbf{Meissl {\kaufmannsund} Schadn}, \emph{Hotel (K.HTL)}|pwk} (vgl. A. S.: \emph{Tagebuch}, 29. 11. 1908).}}}\label{K_L01811-1} mit ihm allein (bei \textsc{Meissl}\oindex{Meissl {\kaufmannsund} Schadn@\textbf{Meissl {\kaufmannsund} Schadn}, \emph{Hotel (K.HTL)}|pw}) nachtmahlen. Und we{\geminationn} Sie ev. heute Abends mit ihm
               ſind, ſchreiben {\pb}Sie mir ein unverbindl Wort (wir ſind
               im Concert{ }\textsc{Dohnanyi}\pwindex{Dohnányi, Ernst von 27.07.1877 – 09.02.1960@\textsc{Dohnányi, Ernst von} (27.07.1877 – 09.02.1960), \emph{Komponist/Komponistin, Pianist/Pianistin}|pw})\pend
           
\pstart
           Montag fahren wir aller Wahrſcheinlichke\textcolor{gray}{it} nach \textsc{Semmering}\oindex{Semmering@\textbf{Semmering}, \emph{A.ADM3}|pw} – auf 2–3 Tage, vielleicht {\pb}ko{\geminationm}t \textsc{Kerr}\pwindex{Kerr, Alfred 25.12.1867 – 12.10.1948@\textsc{Kerr, Alfred} (25.12.1867 – 12.10.1948), \emph{Schriftsteller/Schriftstellerin, Kritiker/Kritikerin}|pw} hinauf\textcolor{gray}{?}\pend
           
\pstart
           – All dies an Sie, verzeihen Sie, weil \textsc{Kerr}\pwindex{Kerr, Alfred 25.12.1867 – 12.10.1948@\textsc{Kerr, Alfred} (25.12.1867 – 12.10.1948), \emph{Schriftsteller/Schriftstellerin, Kritiker/Kritikerin}|pw} behauptet hat, noch keine Adreſſe zu haben.\pend
           
\pstart
           Herzlichſt Ihr{\\[\baselineskip]}\spacefill\mbox{A.}\pend
           \leftskip=0em{}
\pstart
           \noindent{}Auch heute nach 5 bin ich zu Hauſe.\pend
           \selectlanguage{ngerman}\endnumbering\briefempfaengerindex{Beer-Hofmann, Richard@\textsc{Beer-Hofmann, Richard}!zzzSchnitzler, Arthur@\emph{von Arthur Schnitzler}!1908-11-281@{28. 11. 1908}|)be}\mylabel{L01811h}  \normalsize

\doendnotes{C}
\bigskip
\vfill

\clearpage

\footnotesize

\lohead{\textsc{register}}

% Definiere theindex-Environment komplett neu ohne reledmac
\makeatletter
\renewenvironment{theindex}{%
  \section*{\indexname}%
  \setlength{\parindent}{0pt}%
  \setlength{\parskip}{0pt plus 0.3pt}%
  \let\item\@idxitem
}{%
  \clearpage
}
\makeatother

\IfFileExists{\jobname-pw.ind}{\input{\jobname-pw.ind}}{}

\end{document}

      