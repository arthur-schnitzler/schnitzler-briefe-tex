%% latex-leseansicht-vorspann.tex
%% Vorspann für die Leseansicht.
%% Lädt die gemeinsame Datei latex-vorspann.tex mit nicht gesetztem Schalter.

\newif\ifkorrekturansicht
\korrekturansichtfalse

\input{../tex-inputs/latex-vorspann}

\begin{center}
            \textcolor{red}{ENTWURF, NICHT FERTIG KORRIGIERT}
                      \end{center}
            
         
         \renewcommand{\erwaehntePersonen}{Personen: Hansi Niese, Anna Katharina Rehmann, Ottilie Salten, Alfred von Schik-Markenau, Leo Stein}
         \renewcommand{\erwaehnteInstitutionen}{Institutionen: Raimund-Theater}
         \renewcommand{\erwaehnteOrte}{Orte: Wien}
         \renewcommand{\erwaehnteWerke}{Werke: Eduard, der Herzensdieb. Posse mit Gesang in fünf Bildern}
               \section[Felix Salten an Arthur Schnitzler, {[}15. 12. 1904{]}]{ Felix Salten an Arthur Schnitzler, {[}15. 12. 1904{]}}\nopagebreak\mylabel{v}\rehead{ }\begin{ledgroupsized}[t]{13cm}\normalsize\beginnumbering \toendnotes[C]{\smallbreak\pagebreak[2]} \Standort{CUL, Schnitzler, B 89, B 1.}
\physDesc{Brief, 1 Blatt, 1 Seite, 517 Zeichen
\newline{}Handschrift: schwarze Tinte, lateinische Kurrent
\newline{}Schnitzler: mit Bleistift datiert: »15/12 {[}1{]}904« 
\newline{}Ordnung: mit Bleistift von unbekannter Hand nummeriert:
                                    »193« }\toendnotes[C]{\smallbreak}\pstart
           \raggedleft{}{\pb}Donnerstag\pend
           \pstart
           Lieber, ich hab’ es der Niese\pwindex{Niese, Hansi 10.11.1875 – 04.04.1934@\textsc{Niese, Hansi} (10.11.1875 – 04.04.1934), \emph{Schauspielerin}|pw}
               leider schon versprechen müssen, dass ich Samstag zu der \label{K_L03400-1v}\edtext{Première\pwindex{Stein, Leo 25.03.1861 – 28.07.1921@\textsc{Stein, Leo} (25.03.1861 – 28.07.1921), \emph{Schriftsteller, Schriftsteller, Librettist}!Eduard, der Herzensdieb. Posse mit Gesang in fuenf Bildern1904-12-18@\strich\emph{Eduard, der Herzensdieb. Posse mit Gesang in fünf Bildern} {[}1904-12-18{]}|pwv}\pwindex{Schik-Markenau, Alfred von 04.02.1868 – 03.06.1929@\textsc{Schik-Markenau, Alfred von} (04.02.1868 – 03.06.1929), \emph{Journalist/Journalistin}!Eduard, der Herzensdieb. Posse mit Gesang in fuenf Bildern1904-12-18@\strich\emph{Eduard, der Herzensdieb. Posse mit Gesang in fünf Bildern} {[}1904-12-18{]}|pwv}}{\lemma{\textnormal{\emph{Première}}}\Cendnote{\textnormal{Am 18. 12. 1904 fand die
                  Uraufführung von \emph{Eduard, der Herzensdieb. Posse
                     mit Gesang in fünf Bildern}\pwindex{Stein, Leo 25.03.1861 – 28.07.1921@\textsc{Stein, Leo} (25.03.1861 – 28.07.1921), \emph{Schriftsteller, Schriftsteller, Librettist}!Eduard, der Herzensdieb. Posse mit Gesang in fuenf Bildern1904-12-18@\strich\emph{Eduard, der Herzensdieb. Posse mit Gesang in fünf Bildern} {[}1904-12-18{]}|pwk}\pwindex{Schik-Markenau, Alfred von 04.02.1868 – 03.06.1929@\textsc{Schik-Markenau, Alfred von} (04.02.1868 – 03.06.1929), \emph{Journalist/Journalistin}!Eduard, der Herzensdieb. Posse mit Gesang in fuenf Bildern1904-12-18@\strich\emph{Eduard, der Herzensdieb. Posse mit Gesang in fünf Bildern} {[}1904-12-18{]}|pwk} von Leo
                     Stein\pwindex{Stein, Leo 25.03.1861 – 28.07.1921@\textsc{Stein, Leo} (25.03.1861 – 28.07.1921), \emph{Schriftsteller, Schriftsteller, Librettist}|pwk} und Alfred von Schik-Markenau\pwindex{Schik-Markenau, Alfred von 04.02.1868 – 03.06.1929@\textsc{Schik-Markenau, Alfred von} (04.02.1868 – 03.06.1929), \emph{Journalist/Journalistin}|pwk}
                  am \emph{Raimund-Theater}\orgindex{Raimund-Theater@Raimund-Theater|pwk} statt. Hansi Niese\pwindex{Niese, Hansi 10.11.1875 – 04.04.1934@\textsc{Niese, Hansi} (10.11.1875 – 04.04.1934), \emph{Schauspielerin}|pwk} gab die weibliche Hauptrolle.}}}\label{K_L03400-1h} gehe.
               Vielleicht sehen wir uns also an einem anderen Abend, Montag oder Dienstag, was ich
               Ihnen aber erst Samstag, wenn das Repertoire da
                  ist{[},{]} sagen kann. Otti\pwindex{Salten, Ottilie 07.03.1868 – 22.06.1942@\textsc{Salten, Ottilie} (07.03.1868 – 22.06.1942), \emph{Schauspielerin}|pw}
               ist schon zurück, wird aber die nächsten Wochen nicht für länger vom Haus fortkönnen,
               weil das Mäderl\pwindex{Rehmann, Anna Katharina 18.08.1904 – 27.03.1977@\textsc{Rehmann, Anna Katharina} (18.08.1904 – 27.03.1977), \emph{Schauspielerin}|pwv} geimpft
               wurde, und sie braucht.\pend
           \pstart
           Was Sie mit dem »\label{K_L03400-2v}\edtext{sich in Schulden
               gestürzt haben}{\lemma{\textnormal{\emph{sich … haben}}}\Cendnote{\textnormal{siehe Arthur Schnitzler an Felix Salten, 13. 12. 1904}}}\label{K_L03400-2h}« meinen, verstehe ich nicht. In Wien\oindex{Wien@\textbf{Wien}|pw} sind
               Sie doch eher Gläubiger. \pend
           \pstart Herzlich Ihr \spacefill\mbox{Salten}\pend{}
         
         \endnumbering\mylabel{h}\end{ledgroupsized}\begin{anhang}\end{anhang}\newcommand{\dateiname}{L03400}\newcommand{\titel}{Felix Salten an Arthur Schnitzler, [15. 12. 1904]}\newcommand{\editorInnen}{Martin Anton Müller und Laura Untner}%% latex-leseansicht-abspann.tex
%% Abspann für die Leseansicht.
%% Der Schalter \ifkorrekturansicht ist bereits durch den Vorspann gesetzt.

%% latex-abspann.tex
%% Gemeinsamer Abspann für Korrekturansicht und Leseansicht.
%% Setzt den Schalter \ifkorrekturansicht voraus (gesetzt in den
%% einbindenden Dateien latex-korrekturansicht-abspann.tex bzw.
%% latex-leseansicht-abspann.tex).
%% ---------------------------------------------------------------

\normalsize

% Das esempio-Environment wird nur in der Leseansicht benötigt
\ifkorrekturansicht\else
\newenvironment{esempio}[3]%
{
    \vspace{1.5ex}
    \rlap{\underline{#1}}
    \par
    \setlength{\parindent}{0cm}
    \nopagebreak
    \leftskip=#2cm
    \rightskip=#3cm
}
{
    \par
}
\fi

\doendnotes{C}
\bigskip
\vfill

\clearpage

\footnotesize

\ifkorrekturansicht
  \lohead{\textsc{register}}
\fi

% theindex-Environment neu definieren ohne reledmac
\makeatletter
\renewenvironment{theindex}{%
  \ifkorrekturansicht
    \section*{\indexname}%
  \else
    \subsubsection*{Index der erwähnten Entitäten}%
  \fi
  \setlength{\parindent}{0pt}%
  \setlength{\parskip}{0pt plus 0.3pt}%
  \let\item\@idxitem
}{%
  \ifkorrekturansicht\clearpage\fi
}
\makeatother

\IfFileExists{\jobname-pw.ind}{\input{\jobname-pw.ind}}{}

% Quellenangabe nur in der Leseansicht
\ifkorrekturansicht\else
% Fallback-Definitionen, falls die .tex-Datei \titel etc. nicht gesetzt hat
\providecommand{\titel}{}
\providecommand{\editorInnen}{}
\providecommand{\dateiname}{\jobname}

\vspace{3cm}

\vfill

\footnotesize
\textsc{Quelle}: \titel. Herausgegeben von {\editorInnen}. In: \emph{Arthur Schnitzler: Briefwechsel mit Autorinnen und Autoren}.
 Digitale Edition, https://schnitzler-briefe.acdh.oeaw.ac.at/{\dateiname}.html (Stand \today)
\fi

\end{document}


      