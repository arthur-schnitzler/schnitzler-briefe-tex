%% latex-korrekturansicht-vorspann.tex
%% Vorspann für die Korrekturansicht.
%% Lädt die gemeinsame Datei latex-vorspann.tex mit gesetztem Schalter.

\newif\ifkorrekturansicht
\korrekturansichttrue

\input{../tex-inputs/latex-vorspann}


\section[ Felix Salten an Arthur Schnitzler, {[}15. 12. 1904{]}]{L03400 Felix Salten an Arthur Schnitzler, {[}15. 12. 1904{]}}
\nopagebreak\mylabel{L03400v}
\rehead{ }\normalsize\beginnumbering\briefempfaengerindex{Schnitzler, Arthur@\textsc{Schnitzler, Arthur}!zzzSalten, Felix@\emph{von Felix Salten}!1904-12-151@{{[}15. 12. 1904{]}}|(be}
\toendnotes[C]{\smallbreak\pagebreak[2]}\Standort{CUL, Schnitzler, B 89, B 1.}
\physDesc{Brief, 1 Blatt, 1 Seite, 510 Zeichen
\newline{}Handschrift: schwarze Tinte, lateinische Kurrent
\newline{}Schnitzler: mit Bleistift datiert: »15/12 904« 
\newline{}Ordnung: mit Bleistift von unbekannter Hand nummeriert: »193« }\toendnotes[C]{\smallbreak}
\pstart
           \raggedleft{}{\pb}Donnerstag\pend
           \vspace{0.5em}
\pstart
           Lieber, ich hab’ es der Niese\pwindex{Niese, Hansi 10.11.1875 – 04.04.1934@\textsc{Niese, Hansi} (10.11.1875 – 04.04.1934), \emph{Schauspieler/Schauspielerin}|pw}
               leider schon versprechen müssen, dass ich Samstag zu
               der \label{K_L03400-1v}\edtext{Première\pwindex{Eduard, der Herzensdieb. Posse mit Gesang in fuenf Bildern@\emph{Eduard, der Herzensdieb. Posse mit Gesang in fünf Bildern}|pwv}}{\lemma{\textnormal{\emph{Première}}}\Cendnote{\textnormal{Am 17. 12. 1904 fand die Uraufführung von \emph{Eduard, der Herzensdieb. Posse mit Gesang in fünf Bildern}\pwindex{Eduard, der Herzensdieb. Posse mit Gesang in fuenf Bildern@\emph{Eduard, der Herzensdieb. Posse mit Gesang in fünf Bildern}|pwk} von Leo Stein\pwindex{Stein, Leo 25.03.1861 – 28.07.1921@\textsc{Stein, Leo} (25.03.1861 – 28.07.1921), \emph{Schriftsteller/Schriftstellerin, Dramatiker/Dramatikerin, Librettist/Librettistin}|pwk} und Alfred von Schik-Markenau\pwindex{Schik-Markenau, Alfred von 04.02.1868 – 03.06.1929@\textsc{Schik-Markenau, Alfred von} (04.02.1868 – 03.06.1929), \emph{Journalist/Journalistin, Redakteur/Redakteurin, Autor/Autorin}|pwk} im Raimund-Theater\oindex{Raimund-Theater@\textbf{Raimund-Theater}, \emph{Theater (K.THE)}|pwk} statt. Hansi Niese\pwindex{Niese, Hansi 10.11.1875 – 04.04.1934@\textsc{Niese, Hansi} (10.11.1875 – 04.04.1934), \emph{Schauspieler/Schauspielerin}|pwk}
                  gab die weibliche Hauptrolle.}}}\label{K_L03400-1} gehe. Vielleicht \label{K_L03400-2v}\edtext{sehen wir uns}{\lemma{\textnormal{\emph{sehen wir uns}}}\Cendnote{\textnormal{Nachweislich sahen sich Salten\pwindex{Salten, Felix 06.09.1869 – 08.10.1945@\textsc{Salten, Felix} (06.09.1869 – 08.10.1945), \emph{Schriftsteller/Schriftstellerin, Journalist/Journalistin, Chefredakteur/Chefredakteurin}|pwk} und Schnitzler erst am 23. 12. 1904
                  wieder.}}}\label{K_L03400-2} also an einem anderen Abend, Montag oder Dienstag, was ich Ihnen aber
               erst Samstag, wenn das Repertoire da
                  ist{[},{]} sagen kann. Otti\pwindex{Salten, Ottilie 07.03.1868 – 22.06.1942@\textsc{Salten, Ottilie} (07.03.1868 – 22.06.1942), \emph{Schauspieler/Schauspielerin}|pw}
               ist schon zurück, wird aber die nächsten Wochen nicht für länger vom Haus fortkönnen,
               weil das Mäderl\pwindex{Rehmann, Anna Katharina 18.08.1904 – 27.03.1977@\textsc{Rehmann, Anna Katharina} (18.08.1904 – 27.03.1977), \emph{Schauspieler/Schauspielerin, Übersetzer/Übersetzerin}|pwv} geimpft
               wurde, und sie braucht.\pend
           
\pstart
           Was Sie mit dem »\label{K_L03400-3v}\edtext{sich in Schulden
               gestürzt haben}{\lemma{\textnormal{\emph{sich … haben}}}\Cendnote{\textnormal{Siehe Arthur Schnitzler an Felix Salten, 13. 12. 1904.
               }}}\label{K_L03400-3}« meinen, verstehe ich nicht. In Wien\oindex{Wien@\textbf{Wien}, \emph{A.ADM2}|pw} sind
               Sie doch eher Gläubiger.\pend
           
\pstart
           herzlich {\\[\baselineskip]}Ihr \spacefill\mbox{Salten}\pend
           \leftskip=0em{}\selectlanguage{ngerman}\endnumbering\briefempfaengerindex{Schnitzler, Arthur@\textsc{Schnitzler, Arthur}!zzzSalten, Felix@\emph{von Felix Salten}!1904-12-151@{{[}15. 12. 1904{]}}|)be}\mylabel{L03400h}  \normalsize

\doendnotes{C}
\bigskip
\vfill

\clearpage

\footnotesize

\lohead{\textsc{register}}

% Definiere theindex-Environment komplett neu ohne reledmac
\makeatletter
\renewenvironment{theindex}{%
  \section*{\indexname}%
  \setlength{\parindent}{0pt}%
  \setlength{\parskip}{0pt plus 0.3pt}%
  \let\item\@idxitem
}{%
  \clearpage
}
\makeatother

\IfFileExists{\jobname-pw.ind}{\input{\jobname-pw.ind}}{}

\end{document}

      