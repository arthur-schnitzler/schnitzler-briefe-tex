%% latex-korrekturansicht-vorspann.tex
%% Vorspann für die Korrekturansicht.
%% Lädt die gemeinsame Datei latex-vorspann.tex mit gesetztem Schalter.

\newif\ifkorrekturansicht
\korrekturansichttrue

\input{../tex-inputs/latex-vorspann}


\section[ Felix Salten an Arthur Schnitzler, 29. 1. 1906]{L03413 Felix Salten an Arthur Schnitzler, 29. 1. 1906}
\nopagebreak\mylabel{L03413v}
\rehead{ }\normalsize\beginnumbering\briefempfaengerindex{Schnitzler, Arthur@\textsc{Schnitzler, Arthur}!zzzSalten, Felix@\emph{von Felix Salten}!1906-01-294@{29. 1. 1906}|(be}
\toendnotes[C]{\smallbreak\pagebreak[2]}\Standort{CUL, Schnitzler, B 89, B 1.}
\physDesc{Postkarte, 777 Zeichen
\newline{}Handschrift: schwarze Tinte, lateinische Kurrent
\newline{}Versand: Stempel: »\nobreak{}\oindex{Berlin@\textbf{Berlin}, \emph{P.PPLC}|pwk}Berlin, S. W. 68, 29. 1. 06, 2–3 N\nobreak{}«. Stempel: »\nobreak{}\oindex{XVIII., Waehring@\textbf{XVIII., Währing}, \emph{A.ADM3}|pwk}18/1 Wien 110, 30 I 06, X, Bestellt\nobreak{}«.  
\newline{}Ordnung: mit Bleistift von unbekannter Hand nummeriert: »204a« }\toendnotes[C]{\smallbreak}\pstart{}{\pb}Herrn D\textsuperscript{r} Arthur Schnitzler\pend{}\pstart{}Wien XVIII\oindex{XVIII., Waehring@\textbf{XVIII., Währing}, \emph{A.ADM3}|pw}\pend{}\pstart{}Spöttelgasse 7\oindex{Edmund-Weiss-Gasse 7@\textbf{Edmund-Weiß-Gasse 7}, \emph{Wohngebäude (K.WHS)}|pw}\pend{}{\bigskip}\vspace{1em}
\pstart
           \raggedleft{}{\pb}Berlin\oindex{Berlin@\textbf{Berlin}, \emph{P.PPLC}|pw}, 29. I. 06\pend
           \vspace{0.5em}
\pstart
           Lieber, wir sind also \label{K_L03413-1v}\edtext{vorigen Dienstag}{\lemma{\textnormal{\emph{vorigen Dienstag}}}\Cendnote{\textnormal{Salten\pwindex{Salten, Felix 06.09.1869 – 08.10.1945@\textsc{Salten, Felix} (06.09.1869 – 08.10.1945), \emph{Schriftsteller/Schriftstellerin, Journalist/Journalistin, Chefredakteur/Chefredakteurin}|pwk} dürfte sich auf den dem letzten
                  Dienstag vorangehenden Dienstag, den 16. 1. 1906
                  beziehen. Für Samstag, den 14. 1. 1906 hatte Schnitzler
                  den Abschied in Wien\oindex{Wien@\textbf{Wien}, \emph{A.ADM2}|pwk} festgehalten. Die
                  Formulierung ist jedoch soweit offen, dass er auch nach dem Abschied noch eine
                  Woche in Wien\oindex{Wien@\textbf{Wien}, \emph{A.ADM2}|pwk} geblieben sein könnte und hier
                  also vom 23. 1. 1906 die Rede ist.}}}\label{K_L03413-1} hier
               angekommen, und schon am Donnerst\textcolor{gray}{a}g
               habe ich die \label{K_L03413-2v}\edtext{Geschäfte
                  übernommen}{\lemma{\textnormal{\emph{Geschäfte
                  übernommen}}}\Cendnote{\textnormal{Salten\pwindex{Salten, Felix 06.09.1869 – 08.10.1945@\textsc{Salten, Felix} (06.09.1869 – 08.10.1945), \emph{Schriftsteller/Schriftstellerin, Journalist/Journalistin, Chefredakteur/Chefredakteurin}|pwk} übernahm die Chefredaktion der \emph{B. Z. am Mittag}\orgindex{B.Z. am Mittag@B.Z. am Mittag|pwk} und der \emph{Berliner Morgenpost}\orgindex{Berliner Morgenpost@Berliner Morgenpost|pwk}, die beide zum \emph{Ullstein-Konzern}\orgindex{Ullstein Verlag@Ullstein Verlag|pwk} gehörten. Das Engagement dauerte nur ein
                  halbes Jahr. Im September 1906 kehrte er mit seiner Familie nach Wien\oindex{Wien@\textbf{Wien}, \emph{A.ADM2}|pwk} zurück.}}}\label{K_L03413-2}. Da bin ich denn gleich so
               tief in Arbeit gerathen, dass ich weiter nichts von Berlin\oindex{Berlin@\textbf{Berlin}, \emph{P.PPLC}|pw} bemerke. Wir wohnen im »Saxonia\oindex{Hotel Saxonia@\textbf{Hotel Saxonia}, \emph{Hotel (K.HTL)}|pw}«, nahe am Potsdamer Platz\oindex{Potsdamer Platz@\textbf{Potsdamer Platz}, \emph{Platz (K.PLT)}|pw},
               schöne Zimmer aber elende Bedienung. Heute haben wir
               eine Wohnung gemiethet: Charlottenburg\oindex{Charlottenburg@\textbf{Charlottenburg}, \emph{P.PPLX}|pw}, Kantstraße 34\oindex{Kantstrasse@\textbf{Kantstraße}, \emph{Straße (K.STR)}|pw}, dieselbe Straße, in der das Theater d. Westens\orgindex{Theater des Westens@Theater des Westens|pw} ist. Morgen sind wir schon drin. Die Freiwohnung, die mir angeboten war, wollte
               ich nicht beziehen, weil mir vor dem zweimaligen Übersiedeln graut. Otti\pwindex{Salten, Ottilie 07.03.1868 – 22.06.1942@\textsc{Salten, Ottilie} (07.03.1868 – 22.06.1942), \emph{Schauspieler/Schauspielerin}|pw} u. den Kindern\pwindex{Rehmann, Anna Katharina 18.08.1904 – 27.03.1977@\textsc{Rehmann, Anna Katharina} (18.08.1904 – 27.03.1977), \emph{Schauspieler/Schauspielerin, Übersetzer/Übersetzerin}|pwv}\pwindex{Salten, Paul 11.08.1903 – 08.05.1937@\textsc{Salten, Paul} (11.08.1903 – 08.05.1937), \emph{Filmcutter/Filmcutterin}|pwv} geht es gut. \label{K_L03413-3v}\edtext{Wann kommen Sie}{\lemma{\textnormal{\emph{Wann kommen Sie}}}\Cendnote{\textnormal{Schnitzler hielt sich die nächsten Male zwischen 4. 2. 1906 und 7. 2. 1906 sowie
                  zwischen 18. 2. 1906
                  und 27. 2. 1906 in
                     Berlin\oindex{Berlin@\textbf{Berlin}, \emph{P.PPLC}|pwk} auf.}}}\label{K_L03413-3}? Wir freuen uns schon
               darauf! Wissen Sie, dass \label{K_L03413-4v}\edtext{Brahm\pwindex{Brahm, Otto 05.02.1856 – 28.11.1912@\textsc{Brahm, Otto} (05.02.1856 – 28.11.1912), \emph{Theaterleiter/Theaterleiterin, Regisseur/Regisseurin}|pw} am 5. Feber 50 J. alt}{\lemma{\textnormal{\emph{Brahm … alt}}}\Cendnote{\textnormal{Schnitzler war von Brahm\pwindex{Brahm, Otto 05.02.1856 – 28.11.1912@\textsc{Brahm, Otto} (05.02.1856 – 28.11.1912), \emph{Theaterleiter/Theaterleiterin, Regisseur/Regisseurin}|pwk} persönlich informiert
                  (siehe Otto Brahm an Arthur Schnitzler, 27. 1. 1906. In: 
                     \emph{Der Briefwechsel Arthur Schnitzler – Otto Brahm}.
                     Vollständige Ausgabe. Herausgegeben, eingeleitet und erläutert von Oskar
                     Seidlin. Tübingen: \emph{Niemeyer}{ }1975, S. 218)
                     und nahm an der Geburtstagsfeier
                  teil, vgl. A. S.: \emph{Tagebuch}, 5. 2. 1906 und Arthur Schnitzler an Richard Beer-Hofmann, 30. 1. 1906.}}}\label{K_L03413-4} wird?\pend
           
\pstart
           Viele herzlichste Grüße von uns an Sie Drei\pwindex{Schnitzler, Olga 17.01.1882 – 13.01.1970@\textsc{Schnitzler, Olga} (17.01.1882 – 13.01.1970), \emph{Schauspieler/Schauspielerin, Sänger/Sängerin}|pw}\pwindex{Schnitzler, Heinrich 09.08.1902 – 12.07.1982@\textsc{Schnitzler, Heinrich} (09.08.1902 – 12.07.1982), \emph{Regisseur/Regisseurin, Schauspieler/Schauspielerin}|pw}{\\[\baselineskip]}Ihr \spacefill\mbox{S.}\pend
           \leftskip=0em{}\selectlanguage{ngerman}\endnumbering\briefempfaengerindex{Schnitzler, Arthur@\textsc{Schnitzler, Arthur}!zzzSalten, Felix@\emph{von Felix Salten}!1906-01-294@{29. 1. 1906}|)be}\mylabel{L03413h}  \normalsize

\doendnotes{C}
\bigskip
\vfill

\clearpage

\footnotesize

\lohead{\textsc{register}}

% Definiere theindex-Environment komplett neu ohne reledmac
\makeatletter
\renewenvironment{theindex}{%
  \section*{\indexname}%
  \setlength{\parindent}{0pt}%
  \setlength{\parskip}{0pt plus 0.3pt}%
  \let\item\@idxitem
}{%
  \clearpage
}
\makeatother

\IfFileExists{\jobname-pw.ind}{\input{\jobname-pw.ind}}{}

\end{document}

      