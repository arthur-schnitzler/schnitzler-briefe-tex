%% latex-leseansicht-vorspann.tex
%% Vorspann für die Leseansicht.
%% Lädt die gemeinsame Datei latex-vorspann.tex mit nicht gesetztem Schalter.

\newif\ifkorrekturansicht
\korrekturansichtfalse

\input{../tex-inputs/latex-vorspann}


\section[ Felix Salten an Arthur Schnitzler, 29. 1. 1906]{L03413 Felix Salten an Arthur Schnitzler,  29. 1. 1906}
\nopagebreak\mylabel{L03413v}
\rehead{ }\normalsize\beginnumbering\briefempfaengerindex{Schnitzler, Arthur@\textsc{Schnitzler, Arthur}!zzzSalten, Felix@\emph{von Felix Salten}!1906-01-294@{29. 1. 1906}|(be}
\toendnotes[C]{\smallbreak\pagebreak[2]}
\correspDesc{Versand  durch Felix Salten am 29. 1. 1906 in Berlin
\newline{}Erhalt  durch Arthur Schnitzler am 30. 1. 1906 in Wien}\toendnotes[C]{\smallbreak}
\Standort{CUL, Schnitzler, B 89, B 1.}
\physDesc{Postkarte, 777 Zeichen
\newline{}Handschrift: schwarze Tinte, lateinische Kurrent
\newline{}Versand: Stempel: »\nobreak{}\oindex{Berlin@\textbf{Berlin}, \emph{Hauptstadt}|pwk}Berlin, S. W. 68, 29. 1. 06, 2–3 N\nobreak{}«. Stempel: »\nobreak{}\oindex{XVIII., Währing@\textbf{XVIII., Währing}, \emph{Verwaltungsgebiet}|pwk}18/1 Wien 110, 30 I 06, X, Bestellt\nobreak{}«.  
\newline{}Ordnung: mit Bleistift von unbekannter Hand nummeriert: »204a« }\toendnotes[C]{\smallbreak}\pstart{}{\pb}Herrn D\textsuperscript{r} Arthur Schnitzler\pend{}\pstart{}Wien XVIII\oindex{XVIII., Währing@\textbf{XVIII., Währing}, \emph{Verwaltungsgebiet}|pw}\pend{}\pstart{}Spöttelgasse 7\oindex{Wien@\textbf{Wien}!XVIII., Währing@\textbf{XVIII., Währing}!Edmund-Weiß-Gasse 7@\textbf{Edmund-Weiß-Gasse 7}, \emph{Wohngebäude}|pw}\pend{}{\bigskip}\vspace{1em}
\pstart
           \raggedleft{}{\pb}Berlin\oindex{Berlin@\textbf{Berlin}, \emph{Hauptstadt}|pw}, 29. I. 06\pend
           \vspace{0.5em}
\pstart
           Lieber, wir sind also \label{K_L03413-1v}\edtext{vorigen Dienstag}{\lemma{\textnormal{\emph{vorigen Dienstag}}}\Cendnote{\textnormal{Salten\pwindex{Salten, Felix 6.\,9.\,1869 Budapest – 8.\,10.\,1945 Zürich@\textsc{Salten, Felix} (6.\,9.\,1869 Budapest – 8.\,10.\,1945 Zürich), \emph{Schriftsteller, Journalist, Chefredakteur}|pwk} dürfte sich auf den dem letzten
                  Dienstag vorangehenden Dienstag, den 16. 1. 1906
                  beziehen. Für Samstag, den 14. 1. 1906 hatte Schnitzler
                  den Abschied in Wien\oindex{Wien@\textbf{Wien}, \emph{Verwaltungsgebiet}|pwk} festgehalten. Die
                  Formulierung ist jedoch soweit offen, dass er auch nach dem Abschied noch eine
                  Woche in Wien\oindex{Wien@\textbf{Wien}, \emph{Verwaltungsgebiet}|pwk} geblieben sein könnte und hier
                  also vom 23. 1. 1906 die Rede ist.}}}\label{K_L03413-1} hier
               angekommen, und schon am Donnerst\textcolor{gray}{a}g
               habe ich die \label{K_L03413-2v}\edtext{Geschäfte
                  übernommen}{\lemma{\textnormal{\emph{Geschäfte
                  übernommen}}}\Cendnote{\textnormal{Salten\pwindex{Salten, Felix 6.\,9.\,1869 Budapest – 8.\,10.\,1945 Zürich@\textsc{Salten, Felix} (6.\,9.\,1869 Budapest – 8.\,10.\,1945 Zürich), \emph{Schriftsteller, Journalist, Chefredakteur}|pwk} übernahm die Chefredaktion der \emph{B. Z. am Mittag}\orgindex{B.Z. am Mittag@B.Z. am Mittag|pwk} und der \emph{Berliner Morgenpost}\orgindex{Berliner Morgenpost@Berliner Morgenpost|pwk}, die beide zum \emph{Ullstein-Konzern}\orgindex{Ullstein Verlag@Ullstein Verlag|pwk} gehörten. Das Engagement dauerte nur ein
                  halbes Jahr. Im September 1906 kehrte er mit seiner Familie nach Wien\oindex{Wien@\textbf{Wien}, \emph{Verwaltungsgebiet}|pwk} zurück.}}}\label{K_L03413-2}. Da bin ich denn gleich so
               tief in Arbeit gerathen, dass ich weiter nichts von Berlin\oindex{Berlin@\textbf{Berlin}, \emph{Hauptstadt}|pw} bemerke. Wir wohnen im »Saxonia\oindex{Hotel Saxonia@\textbf{Hotel Saxonia}, \emph{Hotel}|pw}«, nahe am Potsdamer Platz\oindex{Potsdamer Platz@\textbf{Potsdamer Platz}, \emph{Platz}|pw},
               schöne Zimmer aber elende Bedienung. Heute haben wir
               eine Wohnung gemiethet: Charlottenburg\oindex{Charlottenburg@\textbf{Charlottenburg}, \emph{Ehemaliger Ort}|pw}, Kantstraße 34\oindex{Kantstraße@\textbf{Kantstraße}, \emph{Straße}|pw}, dieselbe Straße, in der das Theater d. Westens\orgindex{Theater des Westens@Theater des Westens|pw} ist. Morgen sind wir schon drin. Die Freiwohnung, die mir angeboten war, wollte
               ich nicht beziehen, weil mir vor dem zweimaligen Übersiedeln graut. Otti\pwindex{Salten, Ottilie 7.\,3.\,1868 Prag – 22.\,6.\,1942 Zürich@\textsc{Salten, Ottilie} (7.\,3.\,1868 Prag – 22.\,6.\,1942 Zürich), \emph{Schauspielerin}|pw} u. den Kindern\pwindex{Rehmann, Anna Katharina 18.\,8.\,1904 Wien – 27.\,3.\,1977 Zürich@\textsc{Rehmann, Anna Katharina} (18.\,8.\,1904 Wien – 27.\,3.\,1977 Zürich), \emph{Schauspielerin, Übersetzerin}|pwv}\pwindex{Salten, Paul 11.\,8.\,1903 Wien – 8.\,5.\,1937 ebd.@\textsc{Salten, Paul} (11.\,8.\,1903 Wien – 8.\,5.\,1937 ebd.), \emph{Filmcutter}|pwv} geht es gut. \label{K_L03413-3v}\edtext{Wann kommen Sie}{\lemma{\textnormal{\emph{Wann kommen Sie}}}\Cendnote{\textnormal{Schnitzler hielt sich die nächsten Male zwischen 4. 2. 1906 und 7. 2. 1906 sowie
                  zwischen 18. 2. 1906
                  und 27. 2. 1906 in
                     Berlin\oindex{Berlin@\textbf{Berlin}, \emph{Hauptstadt}|pwk} auf.}}}\label{K_L03413-3}? Wir freuen uns schon
               darauf! Wissen Sie, dass \label{K_L03413-4v}\edtext{Brahm\pwindex{Brahm, Otto 5.\,2.\,1856 Hamburg – 28.\,11.\,1912 Berlin@\textsc{Brahm, Otto} (5.\,2.\,1856 Hamburg – 28.\,11.\,1912 Berlin), \emph{Theaterleiter, Regisseur}|pw} am 5. Feber 50 J. alt}{\lemma{\textnormal{\emph{Brahm … alt}}}\Cendnote{\textnormal{Schnitzler war von Brahm\pwindex{Brahm, Otto 5.\,2.\,1856 Hamburg – 28.\,11.\,1912 Berlin@\textsc{Brahm, Otto} (5.\,2.\,1856 Hamburg – 28.\,11.\,1912 Berlin), \emph{Theaterleiter, Regisseur}|pwk} persönlich informiert
                  (siehe Otto Brahm an Arthur Schnitzler, 27. 1. 1906. In: 
                     \emph{Der Briefwechsel Arthur Schnitzler – Otto Brahm}.
                     Vollständige Ausgabe. Herausgegeben, eingeleitet und erläutert von Oskar
                     Seidlin. Tübingen: \emph{Niemeyer}{ }1975, S. 218)
                     und nahm an der Geburtstagsfeier
                  teil, vgl. A. S.: \emph{Tagebuch}, 5. 2. 1906 und XXXX Auszeichnungsfehler: Dokument L01580 nicht gefunden.}}}\label{K_L03413-4} wird?\pend
           
\pstart
           Viele herzlichste Grüße von uns an Sie Drei\pwindex{Schnitzler, Olga 17.\,1.\,1882 Wien – 13.\,1.\,1970 Lugano@\textsc{Schnitzler, Olga} (17.\,1.\,1882 Wien – 13.\,1.\,1970 Lugano), \emph{Schauspielerin, Sängerin}|pw}\pwindex{Schnitzler, Heinrich 9.\,8.\,1902 Hinterbrühl – 12.\,7.\,1982 Wien@\textsc{Schnitzler, Heinrich} (9.\,8.\,1902 Hinterbrühl – 12.\,7.\,1982 Wien), \emph{Regisseur, Schauspieler}|pw}{\\[\baselineskip]}Ihr \spacefill\mbox{S.}\pend
           \leftskip=0em{}\selectlanguage{ngerman}\endnumbering\briefempfaengerindex{Schnitzler, Arthur@\textsc{Schnitzler, Arthur}!zzzSalten, Felix@\emph{von Felix Salten}!1906-01-294@{29. 1. 1906}|)be}\mylabel{L03413h}  \newcommand{\dateiname}{L03413}\newcommand{\titel}{Felix Salten an Arthur Schnitzler, 29. 1. 1906}\newcommand{\editorInnen}{Martin Anton Müller und Laura Untner}%% latex-leseansicht-abspann.tex
%% Abspann für die Leseansicht.
%% Der Schalter \ifkorrekturansicht ist bereits durch den Vorspann gesetzt.

%% latex-abspann.tex
%% Gemeinsamer Abspann für Korrekturansicht und Leseansicht.
%% Setzt den Schalter \ifkorrekturansicht voraus (gesetzt in den
%% einbindenden Dateien latex-korrekturansicht-abspann.tex bzw.
%% latex-leseansicht-abspann.tex).
%% ---------------------------------------------------------------

\normalsize

% Das esempio-Environment wird nur in der Leseansicht benötigt
\ifkorrekturansicht\else
\newenvironment{esempio}[3]%
{
    \vspace{1.5ex}
    \rlap{\underline{#1}}
    \par
    \setlength{\parindent}{0cm}
    \nopagebreak
    \leftskip=#2cm
    \rightskip=#3cm
}
{
    \par
}
\fi

\doendnotes{C}
\bigskip
\vfill

\clearpage

\footnotesize

\ifkorrekturansicht
  \lohead{\textsc{register}}
\fi

% theindex-Environment neu definieren ohne reledmac
\makeatletter
\renewenvironment{theindex}{%
  \ifkorrekturansicht
    \section*{\indexname}%
  \else
    \subsubsection*{Index der erwähnten Entitäten}%
  \fi
  \setlength{\parindent}{0pt}%
  \setlength{\parskip}{0pt plus 0.3pt}%
  \let\item\@idxitem
}{%
  \ifkorrekturansicht\clearpage\fi
}
\makeatother

\IfFileExists{\jobname-pw.ind}{\input{\jobname-pw.ind}}{}

% Quellenangabe nur in der Leseansicht
\ifkorrekturansicht\else
% Fallback-Definitionen, falls die .tex-Datei \titel etc. nicht gesetzt hat
\providecommand{\titel}{}
\providecommand{\editorInnen}{}
\providecommand{\dateiname}{\jobname}

\vspace{3cm}

\vfill

\footnotesize
\textsc{Quelle}: \titel. Herausgegeben von {\editorInnen}. In: \emph{Arthur Schnitzler: Briefwechsel mit Autorinnen und Autoren}.
 Digitale Edition, https://schnitzler-briefe.acdh.oeaw.ac.at/{\dateiname}.html (Stand \today)
\fi

\end{document}


