%% latex-leseansicht-vorspann.tex
%% Vorspann für die Leseansicht.
%% Lädt die gemeinsame Datei latex-vorspann.tex mit nicht gesetztem Schalter.

\newif\ifkorrekturansicht
\korrekturansichtfalse

\input{../tex-inputs/latex-vorspann}


         
         \renewcommand{\erwaehntePersonen}{Personen: Otto Brahm, Anna Katharina Rehmann, Felix Salten, Ottilie Salten, Paul Salten, Olga Schnitzler, Heinrich Schnitzler}
         \renewcommand{\erwaehnteInstitutionen}{Institutionen: Theater des Westens}
         \renewcommand{\erwaehnteOrte}{Orte: Berlin, Charlottenburg, Edmund-Weiß-Gasse 7, Hotel Saxonia, Kantstraße, Potsdamer Platz, Wien, XVIII., Währing}
         \renewcommand{\erwaehnteWerke}{}
               \section[ Felix Salten an Arthur Schnitzler, 29. 1. 1906]{ Felix Salten an Arthur Schnitzler, 29. 1. 1906}\nopagebreak\mylabel{v}\rehead{ }\begin{ledgroupsized}[t]{13cm}\normalsize\beginnumbering\briefempfaengerindex{Schnitzler, Arthur@\textsc{Schnitzler, Arthur}!zzzSalten, Felix@\emph{von Felix Salten}!1906-01-294@{29. 1. 1906}|(be} \toendnotes[C]{\smallbreak\pagebreak[2]} \Standort{CUL, Schnitzler, B 89, B 1.}
\physDesc{Postkarte, 778 Zeichen
\newline{}Handschrift: schwarze Tinte, lateinische Kurrent
\newline{}Versand: Stempel: »\nobreak{}\oindex{Berlin@\textbf{Berlin}|pwk}Berlin, S. W. 68, 29. 1. 06, 2–3 N\nobreak{}«. Stempel: »\nobreak{}\oindex{XVIII., Waehring@\textbf{XVIII., Währing}|pwk}18/1 Wien 110, 30 I 06, X, Bestellt\nobreak{}«.  
\newline{}Ordnung: mit Bleistift von unbekannter Hand nummeriert: »204a« }\toendnotes[C]{\smallbreak}\pstart{}{\pb}Herrn D\textsuperscript{r} Arthur Schnitzler\pend{}\pstart{}Wien XVIII\oindex{XVIII., Waehring@\textbf{XVIII., Währing}|pw}\pend{}\pstart{}Spöttelgasse 7\oindex{Edmund-Weiss-Gasse 7@\textbf{Edmund-Weiß-Gasse 7}|pw}\pend{}{\bigskip}\pstart
           \raggedleft{}{\pb}Berlin\oindex{Berlin@\textbf{Berlin}|pw}, 29. I. 06\pend
           \pstart
           Lieber, wir sind also \label{K_L03413-1v}\edtext{vorigen Dienstag}{\lemma{\textnormal{\emph{vorigen Dienstag}}}\Cendnote{\textnormal{Salten\pwindex{Salten, Felix 06.09.1869 – 08.10.1945@\textsc{Salten, Felix} (06.09.1869 – 08.10.1945), \emph{Schriftsteller, Journalist}|pwk} dürfte sich auf den dem letzten
                  Dienstag vorangehenden Dienstag, den 16. 1. 1906
                  beziehen. Für den Samstag, den 14. 1. 1906 hatte Schnitzler\pwindex{Schnitzler, Arthur 15.05.1862 – 21.10.1931@\textsc{Schnitzler, Arthur} (15.05.1862 – 21.10.1931), \emph{Schriftsteller, Mediziner}|pwk}
                  den Abschied in Wien\oindex{Wien@\textbf{Wien}|pwk} festgehalten. Die
                  Formulierung ist jedoch soweit offen, dass er auch nach dem Abschied noch eine
                  Woche in Wien\oindex{Wien@\textbf{Wien}|pwk} geblieben sein könnte und hier
                  also vom 23. 1. 1906 die Rede ist.}}}\label{K_L03413-1h} hier
               angekommen, und schon am Donnerst\textcolor{gray}{a}g
               habe ich die \label{K_L03413-33v}\edtext{Geschäfte
                  übernommen}{\lemma{\textnormal{\emph{Geschäfte
                  übernommen}}}\Cendnote{\textnormal{Salten\pwindex{Salten, Felix 06.09.1869 – 08.10.1945@\textsc{Salten, Felix} (06.09.1869 – 08.10.1945), \emph{Schriftsteller, Journalist}|pwk} übernahm die Chefredaktion der \emph{B. Z. am Mittag}XXXX ORGangabe fehlt und der \emph{Berliner Morgenpost}XXXX ORGangabe fehlt, die beide zum \emph{Ullstein-Konzern}XXXX ORGangabe fehlt gehörten. Das Engagement dauerte nur ein
                  halbes Jahr. Im September 1906 kehrte er mit seiner Familie nach Wien\oindex{Wien@\textbf{Wien}|pwk} zurück.}}}\label{K_L03413-33h}. Da bin ich denn gleich so
               tief in Arbeit gerathen, dass ich weiter nichts von Berlin\oindex{Berlin@\textbf{Berlin}|pw} bemerke. Wir wohnen im »Saxonia\oindex{Hotel Saxonia@\textbf{Hotel Saxonia}|pw}«, nahe am Potsdamer Platz\oindex{Potsdamer Platz@\textbf{Potsdamer Platz}|pw},
               schöne Zimmer aber elende Bedienung. Heute haben wir
               eine Wohnung gemiethet: Charlottenburg\oindex{Charlottenburg@\textbf{Charlottenburg}|pw}, Kantstraße 34\oindex{Kantstrasse@\textbf{Kantstraße}|pw}, dieselbe Straße, in der das Theater d. Westens\orgindex{Theater des Westens@Theater des Westens|pw} ist. Morgen sind wir schon drin. Die Freiwohnung, die mir angeboten war, wollte
               ich nicht beziehen, weil mir vor dem zweimaligen Übersiedeln graut. Otti\pwindex{Salten, Ottilie 07.03.1868 – 22.06.1942@\textsc{Salten, Ottilie} (07.03.1868 – 22.06.1942), \emph{Schauspielerin}|pw} u. den Kindern\pwindex{Rehmann, Anna Katharina 18.08.1904 – 27.03.1977@\textsc{Rehmann, Anna Katharina} (18.08.1904 – 27.03.1977), \emph{Schauspielerin, Übersetzerin}|pwv}\pwindex{Salten, Paul 11.08.1903 – 08.05.1937@\textsc{Salten, Paul} (11.08.1903 – 08.05.1937), \emph{Filmcutter}|pwv} geht es gut. \label{K_L03413-2v}\edtext{Wann kommen Sie}{\lemma{\textnormal{\emph{Wann kommen Sie}}}\Cendnote{\textnormal{Schnitzler\pwindex{Schnitzler, Arthur 15.05.1862 – 21.10.1931@\textsc{Schnitzler, Arthur} (15.05.1862 – 21.10.1931), \emph{Schriftsteller, Mediziner}|pwk} hielt sich die nächsten Male zwischen 4. 2. 1906 und 7. 2. 1906 sowie
                  zwischen 18. 2. 1906
                  und 27. 2. 1906 in
                     Berlin\oindex{Berlin@\textbf{Berlin}|pwk} auf.}}}\label{K_L03413-2h}? Wir freuen uns schon
               darauf! Wissen Sie, dass \label{K_L03413-3v}\edtext{Brahm\pwindex{Brahm, Otto 05.02.1856 – 28.11.1912@\textsc{Brahm, Otto} (05.02.1856 – 28.11.1912), \emph{Theaterleiter, Regisseur}|pw} am 5. Feber 50 J. alt}{\lemma{\textnormal{\emph{Brahm … alt}}}\Cendnote{\textnormal{Schnitzler\pwindex{Schnitzler, Arthur 15.05.1862 – 21.10.1931@\textsc{Schnitzler, Arthur} (15.05.1862 – 21.10.1931), \emph{Schriftsteller, Mediziner}|pwk} war von Brahm\pwindex{Brahm, Otto 05.02.1856 – 28.11.1912@\textsc{Brahm, Otto} (05.02.1856 – 28.11.1912), \emph{Theaterleiter, Regisseur}|pwk} persönlich informiert
                     (siehe Otto Brahm an Arthur Schnitzler, 27. 1. 1906, \emph{Brahm/Schnitzler} 218)
                     und nahm an der Geburtstagsfeier
                  teil, vgl. A. S.: \emph{Tagebuch}, 5. 2. 1906 und Arthur Schnitzler an Richard Beer-Hofmann, 30. 1. 1906.}}}\label{K_L03413-3h} wird?\pend
           \pstart
           Viele herzlichste Grüße von uns an Sie Drei\pwindex{Schnitzler, Olga 17.01.1882 – 13.01.1970@\textsc{Schnitzler, Olga} (17.01.1882 – 13.01.1970), \emph{Schauspielerin, Sängerin}|pw}\pwindex{Schnitzler, Heinrich 09.08.1902 – 12.07.1982@\textsc{Schnitzler, Heinrich} (09.08.1902 – 12.07.1982), \emph{Regisseur, Schauspieler}|pw}{\\[\baselineskip]}Ihr \spacefill\mbox{S.}\pend
           \leftskip=0em{}
         
         \endnumbering\mylabel{h}\end{ledgroupsized}  \newcommand{\dateiname}{L03413}\newcommand{\titel}{Felix Salten an Arthur Schnitzler, 29. 1. 1906}\newcommand{\editorInnen}{Martin Anton Müller und Laura Untner}%% latex-leseansicht-abspann.tex
%% Abspann für die Leseansicht.
%% Der Schalter \ifkorrekturansicht ist bereits durch den Vorspann gesetzt.

%% latex-abspann.tex
%% Gemeinsamer Abspann für Korrekturansicht und Leseansicht.
%% Setzt den Schalter \ifkorrekturansicht voraus (gesetzt in den
%% einbindenden Dateien latex-korrekturansicht-abspann.tex bzw.
%% latex-leseansicht-abspann.tex).
%% ---------------------------------------------------------------

\normalsize

% Das esempio-Environment wird nur in der Leseansicht benötigt
\ifkorrekturansicht\else
\newenvironment{esempio}[3]%
{
    \vspace{1.5ex}
    \rlap{\underline{#1}}
    \par
    \setlength{\parindent}{0cm}
    \nopagebreak
    \leftskip=#2cm
    \rightskip=#3cm
}
{
    \par
}
\fi

\doendnotes{C}
\bigskip
\vfill

\clearpage

\footnotesize

\ifkorrekturansicht
  \lohead{\textsc{register}}
\fi

% theindex-Environment neu definieren ohne reledmac
\makeatletter
\renewenvironment{theindex}{%
  \ifkorrekturansicht
    \section*{\indexname}%
  \else
    \subsubsection*{Index der erwähnten Entitäten}%
  \fi
  \setlength{\parindent}{0pt}%
  \setlength{\parskip}{0pt plus 0.3pt}%
  \let\item\@idxitem
}{%
  \ifkorrekturansicht\clearpage\fi
}
\makeatother

\IfFileExists{\jobname-pw.ind}{\input{\jobname-pw.ind}}{}

% Quellenangabe nur in der Leseansicht
\ifkorrekturansicht\else
% Fallback-Definitionen, falls die .tex-Datei \titel etc. nicht gesetzt hat
\providecommand{\titel}{}
\providecommand{\editorInnen}{}
\providecommand{\dateiname}{\jobname}

\vspace{3cm}

\vfill

\footnotesize
\textsc{Quelle}: \titel. Herausgegeben von {\editorInnen}. In: \emph{Arthur Schnitzler: Briefwechsel mit Autorinnen und Autoren}.
 Digitale Edition, https://schnitzler-briefe.acdh.oeaw.ac.at/{\dateiname}.html (Stand \today)
\fi

\end{document}


      