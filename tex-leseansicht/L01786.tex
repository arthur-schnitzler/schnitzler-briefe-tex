%% latex-leseansicht-vorspann.tex
%% Vorspann für die Leseansicht.
%% Lädt die gemeinsame Datei latex-vorspann.tex mit nicht gesetztem Schalter.

\newif\ifkorrekturansicht
\korrekturansichtfalse

\input{../tex-inputs/latex-vorspann}


               \section[Arthur Schnitzler an Hugo von Hofmannsthal, 6. 8. 1908]{ Arthur Schnitzler an Hugo von Hofmannsthal, 6. 8. 1908}\nopagebreak\mylabel{v}\rehead{ }\begin{ledgroupsized}[t]{13cm}\normalsize\beginnumbering\briefempfaengerindex{Hofmannsthal, Hugo von@\textsc{Hofmannsthal, Hugo von}!zzzSchnitzler, Arthur@\emph{von Arthur Schnitzler}!1908-08-061@{6. 8. 1908}|(be} \toendnotes[C]{\smallbreak\pagebreak[2]} \Standort{FDH, Hs-30885,132.}
\physDesc{Brief, 1 Blatt, 4 Seiten
\newline{}Handschrift: schwarze Tinte, lateinische Kurrent}\buchAbdrucke{\weitereDrucke{Hugo von Hofmannsthal, Arthur Schnitzler: \emph{Briefwechsel}. Hg. Therese Nickl und Heinrich Schnitzler. Frankfurt am Main: \emph{S. Fischer} 1964, S. 239.} }\toendnotes[C]{\smallbreak}\pstart
           {\pb}\textcolor{gray}{\textbf{Dr. Arthur Schnitzler}}\hfill Seis am Schlern\oindex{Seis am Schlern@\textbf{Seis am Schlern}|pw},\pend
           \pstart
           \textcolor{gray}{\textbf{Wien XVIII. Spoettelgasse 7\oindex{Edmund-Weiss-Gasse@\textbf{Edmund-Weiß-Gasse}|pw}.}}\hfill 6. 8. 08\pend
           \pstart{}lieber Hugo, \pend\pstart
           Sie sehen, wir sind noch i{\geminationm}er da, und wahrscheinlich
               bleiben wir bis ungefähr 20. we{\geminationn} nicht
               länger. Seit 10 Tagen ist Wasserma{\geminationn}\pwindex{Wassermann, Jakob 10.03.1873 – 01.01.1934@\textsc{Wassermann, Jakob} (10.03.1873 – 01.01.1934), \emph{Schriftsteller}|pw} hier, Agnes Speyer\pwindex{Ulmann, Agnes 23. 12. 1875 – 1. 4. 1942@\textsc{Ulmann, Agnes} (23. 12. 1875 – 1. 4. 1942), \emph{Malerin, Bildhauerin}|pw}, Doctor Kaufmann\pwindex{Kaufmann, Arthur 04.04.1872 – 25.07.1938@\textsc{Kaufmann, Arthur} (04.04.1872 – 25.07.1938), \emph{Rechtswissenschaftler, Privatgelehrte, Privatier}|pw}, und gestern sind wir von einer sehr schönen Partie
                  zurückgeko{\geminationm}en: – Seis\oindex{Seis am Schlern@\textbf{Seis am Schlern}|pw} – Weisslahnbad\oindex{Weisslahnbad@\textbf{Weisslahnbad}|pw} – Jungbru{\geminationn}thal\oindex{Jungbrunntal@\textbf{Jungbrunntal}|pw} – Schlern\oindex{Schlern@\textbf{Schlern}|pw} – Seis\oindex{Seis am Schlern@\textbf{Seis am Schlern}|pw}; – besonders der
               (etwa 5stündg Spaziergang von hier nach Weisslahnbad\oindex{Weisslahnbad@\textbf{Weisslahnbad}|pw} gehört zu {\pb}den schönsten, die man
               träumen ka{\geminationn}, und ist, wie die ganze Gegend, nicht
               berühmt genug. Vor 8 Tagen ist Brahm\pwindex{Brahm, Otto 05.02.1856 – 28.11.1912@\textsc{Brahm, Otto} (05.02.1856 – 28.11.1912), \emph{Theaterleiter, Regisseur}|pw} abgereist,
               der sich nicht weniger als drei Wochen lang hier aufgehalten hat, und, trotz allerlei
               mehr oder weniger fundirten Hypochondrien, in guter Laune und ebensolchem
               Wohlbefinden.\pend
           \pstart
           Von hier aus mach ich mit Olga\pwindex{Schnitzler, Olga 17.01.1882 – 13.01.1970@\textsc{Schnitzler, Olga} (17.01.1882 – 13.01.1970), \emph{Schauspielerin, Sängerin}|pw} eine kleine Reise;
               wohin steht noch nicht fest – Martino\oindex{San Martino di Castrozza@\textbf{San Martino di Castrozza}|pw} oder Campiglio\oindex{Madonna di Campiglio@\textbf{Madonna di Campiglio}|pw}, event. München\oindex{Muenchen@\textbf{München}|pw} zum Schluss. – Dass Sie zu {\pb}meinem Roman\pwindex{Schnitzler, Arthur 15.05.1862 – 21.10.1931@\textsc{Schnitzler, Arthur} (15.05.1862 – 21.10.1931), \emph{Schriftsteller, Mediziner}!Weg ins Freie. Roman1.1.1908 – 1.6.1908@\strich\emph{Der Weg ins Freie. Roman} {[}1.1.1908 – 1.6.1908{]}|pwv} kein glückliches Verhältnis
               gefunden haben, war in der That nicht schwer zu merken. Und so sehr ich Ihrem
               Ausspruch beisti{\geminationm}e, dass Sie zwischen mir und meinen
               Arbeiten keine Grenze ziehen können, ich empfinde ihn als doppelt u. zehnfach wahr
               gegenüber einem Werk, das mich in Gedanke u Ausführung durch manches reife und \introOben{}höchst\introOben{} bewußte Jahr meines Lebens vornehmlich beschäftigt hat.
               Als so wahr erweist es sich, was Sie selbst zu spüren scheinen, wie es kaum denkbar
               ist, zum Dichter eines Werks, das für eine {\pb}ganze
               Entwicklungsperiode \substVorne{}\textsuperscript{eines}\substDazwischen{}\label{T_L01786_1v}\edtext{dieses}{\lemma{\textnormal{\emph{dieses}}}\Cendnote{\textnormal{In der ersten Schicht schrieb er »dieses«,
                        ersetzte es dann durch »eines«, um dann wieder zu
                           »dieses« zurückzukehren.}}}\label{T_L01786_1h}\substHinten{} Dichters bedeutend ist, in einem glücklichern Verhältnis zu stehen als zu
               der Dichtung selbst und dass ich Ihnen für den Takt dankbar bin, der es Sie als
               richtig erkennen liess, jedes weitre Wort über ein Werk\pwindex{Schnitzler, Arthur 15.05.1862 – 21.10.1931@\textsc{Schnitzler, Arthur} (15.05.1862 – 21.10.1931), \emph{Schriftsteller, Mediziner}!Weg ins Freie. Roman1.1.1908 – 1.6.1908@\strich\emph{Der Weg ins Freie. Roman} {[}1.1.1908 – 1.6.1908{]}|pw} zu unterlassen, das nichts vermocht hat als Sie zu verstören und von
               dem mir ein unverlierbar und untrüglich starkes \introOben{}\strikeout{\textcolor{gray}{×}\-\textcolor{gray}{×}\-\textcolor{gray}{×}\-\textcolor{gray}{×}\-\textcolor{gray}{×}\-\textcolor{gray}{×}\-\textcolor{gray}{×}\-\textcolor{gray}{×}\-\textcolor{gray}{×}\-\textcolor{gray}{×}}\introOben{} Nachgefühl in der Seele geblieben ist. –\pend
           \pstart
           Auf Wiedersehen im Herbst; Sie bringen hoffentlich viel schönes zum
               vorlesen mit, – lassen Sie sichs beide in Sils\oindex{Sils im Engadin@\textbf{Sils im Engadin}|pw}
               wohlergehen.\pend
           \pstart Wir grüßen herzlichst.\spacefill\mbox{Arthur.}\pend{}\endnumbering\briefempfaengerindex{Hofmannsthal, Hugo von@\textsc{Hofmannsthal, Hugo von}!zzzSchnitzler, Arthur@\emph{von Arthur Schnitzler}!1908-08-061@{6. 8. 1908}|)be}\mylabel{h}\end{ledgroupsized}  \newcommand{\dateiname}{L01786}\newcommand{\titel}{Arthur Schnitzler an Hugo von Hofmannsthal, 6. 8. 1908}\newcommand{\editorInnen}{Martin Anton Müller und Gerd-Hermann Susen}
            \footnotesize
\begin{ledgroupsized}[t]{11.5cm}
\doendnotes{C}
\end{ledgroupsized}
         %% latex-leseansicht-abspann.tex
%% Abspann für die Leseansicht.
%% Der Schalter \ifkorrekturansicht ist bereits durch den Vorspann gesetzt.

%% latex-abspann.tex
%% Gemeinsamer Abspann für Korrekturansicht und Leseansicht.
%% Setzt den Schalter \ifkorrekturansicht voraus (gesetzt in den
%% einbindenden Dateien latex-korrekturansicht-abspann.tex bzw.
%% latex-leseansicht-abspann.tex).
%% ---------------------------------------------------------------

\normalsize

% Das esempio-Environment wird nur in der Leseansicht benötigt
\ifkorrekturansicht\else
\newenvironment{esempio}[3]%
{
    \vspace{1.5ex}
    \rlap{\underline{#1}}
    \par
    \setlength{\parindent}{0cm}
    \nopagebreak
    \leftskip=#2cm
    \rightskip=#3cm
}
{
    \par
}
\fi

\doendnotes{C}
\bigskip
\vfill

\clearpage

\footnotesize

\ifkorrekturansicht
  \lohead{\textsc{register}}
\fi

% theindex-Environment neu definieren ohne reledmac
\makeatletter
\renewenvironment{theindex}{%
  \ifkorrekturansicht
    \section*{\indexname}%
  \else
    \subsubsection*{Index der erwähnten Entitäten}%
  \fi
  \setlength{\parindent}{0pt}%
  \setlength{\parskip}{0pt plus 0.3pt}%
  \let\item\@idxitem
}{%
  \ifkorrekturansicht\clearpage\fi
}
\makeatother

\IfFileExists{\jobname-pw.ind}{\input{\jobname-pw.ind}}{}

% Quellenangabe nur in der Leseansicht
\ifkorrekturansicht\else
% Fallback-Definitionen, falls die .tex-Datei \titel etc. nicht gesetzt hat
\providecommand{\titel}{}
\providecommand{\editorInnen}{}
\providecommand{\dateiname}{\jobname}

\vspace{3cm}

\vfill

\footnotesize
\textsc{Quelle}: \titel. Herausgegeben von {\editorInnen}. In: \emph{Arthur Schnitzler: Briefwechsel mit Autorinnen und Autoren}.
 Digitale Edition, https://schnitzler-briefe.acdh.oeaw.ac.at/{\dateiname}.html (Stand \today)
\fi

\end{document}


      