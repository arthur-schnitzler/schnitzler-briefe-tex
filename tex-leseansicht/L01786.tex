%% latex-korrekturansicht-vorspann.tex
%% Vorspann für die Korrekturansicht.
%% Lädt die gemeinsame Datei latex-vorspann.tex mit gesetztem Schalter.

\newif\ifkorrekturansicht
\korrekturansichttrue

\input{../tex-inputs/latex-vorspann}


\section[Arthur Schnitzler an Hugo von Hofmannsthal, 6. 8. 1908]{L01786 Arthur Schnitzler an Hugo von Hofmannsthal, 6. 8. 1908}
\nopagebreak\mylabel{L01786v}
\rehead{ }\normalsize\beginnumbering\briefempfaengerindex{Hofmannsthal, Hugo von@\textsc{Hofmannsthal, Hugo von}!zzzSchnitzler, Arthur@\emph{von Arthur Schnitzler}!1908-08-061@{6. 8. 1908}|(be}
\toendnotes[C]{\smallbreak\pagebreak[2]}\Standort{FDH, Hs-30885,132.}
\physDesc{Brief, 1 Blatt, 4 Seiten, 1884 Zeichen
\newline{}Handschrift: schwarze Tinte, lateinische Kurrent}
\buchAbdrucke{\weitereDrucke{Hugo von Hofmannsthal, Arthur Schnitzler: \emph{Briefwechsel}. Frankfurt am Main: \emph{S. Fischer} 1964, S. 239.} }\toendnotes[C]{\smallbreak}
\pstart
           
\pstart
           {\pb}\textcolor{gray}{\textbf{Dr. Arthur Schnitzler}}\pend
           
\pstart
           \raggedleft{}Seis am Schlern\oindex{Seis am Schlern@\textbf{Seis am Schlern}, \emph{P.PPL}|pw},\pend
           \pend
           
\pstart
           \textcolor{gray}{\textbf{Wien XVIII. Spoettelgasse 7\oindex{Edmund-Weiss-Gasse 7@\textbf{Edmund-Weiß-Gasse 7}, \emph{Wohngebäude (K.WHS)}|pw}.}}\hfill 6. 8. 08\pend
           
\pstart{}lieber Hugo, \pend\vspace{0.5em}
\pstart
           Sie sehen, wir sind noch i{\geminationm}er da, und wahrscheinlich
               bleiben wir bis ungefähr 20. we{\geminationn} nicht
               länger. Seit 10 Tagen ist Wasserma{\geminationn}\pwindex{Wassermann, Jakob 10.03.1873 – 01.01.1934@\textsc{Wassermann, Jakob} (10.03.1873 – 01.01.1934), \emph{Schriftsteller/Schriftstellerin}|pw} hier, Agnes Speyer\pwindex{Ulmann, Agnes 23. 12. 1875 – 1. 4. 1942@\textsc{Ulmann, Agnes} (23. 12. 1875 – 1. 4. 1942), \emph{Maler/Malerin, Bildhauer/Bildhauerin}|pw}, Doctor Kaufmann\pwindex{Kaufmann, Arthur 04.04.1872 – 25.07.1938@\textsc{Kaufmann, Arthur} (04.04.1872 – 25.07.1938), \emph{Rechtswissenschaftler/Rechtswissenschaftlerin, Privatgelehrte/Privatgelehrte, Privatier/Privatière}|pw}, und gestern sind wir von einer sehr
               schönen Partie zurückgeko{\geminationm}en: – Seis\oindex{Seis am Schlern@\textbf{Seis am Schlern}, \emph{P.PPL}|pw} – Weisslahnbad\oindex{Weisslahnbad@\textbf{Weisslahnbad}, \emph{Teil eines besiedelten Ortes (A.BSOX)}|pw} – Jungbru{\geminationn}thal\oindex{Jungbrunntal@\textbf{Jungbrunntal}, \emph{Tal (N.TAL)}|pw} – Schlern\oindex{Schlern@\textbf{Schlern}, \emph{Berg (N.BRG)}|pw} – Seis\oindex{Seis am Schlern@\textbf{Seis am Schlern}, \emph{P.PPL}|pw}; – besonders der (etwa 5stündg Spaziergang von hier nach Weisslahnbad\oindex{Weisslahnbad@\textbf{Weisslahnbad}, \emph{Teil eines besiedelten Ortes (A.BSOX)}|pw} gehört zu {\pb}den schönsten, die man träumen ka{\geminationn}, und ist, wie die ganze Gegend, nicht berühmt genug.
               Vor 8 Tagen ist Brahm\pwindex{Brahm, Otto 05.02.1856 – 28.11.1912@\textsc{Brahm, Otto} (05.02.1856 – 28.11.1912), \emph{Theaterleiter/Theaterleiterin, Regisseur/Regisseurin}|pw} abgereist, der sich
               nicht weniger als drei Wochen lang hier aufgehalten hat, und, trotz allerlei mehr
               oder weniger fundirten Hypochondrien, in guter Laune und ebensolchem
               Wohlbefinden.\pend
           
\pstart
           Von hier aus mach ich mit Olga\pwindex{Schnitzler, Olga 17.01.1882 – 13.01.1970@\textsc{Schnitzler, Olga} (17.01.1882 – 13.01.1970), \emph{Schauspieler/Schauspielerin, Sänger/Sängerin}|pw} eine kleine
               Reise; wohin steht noch nicht fest – Martino\oindex{San Martino di Castrozza@\textbf{San Martino di Castrozza}, \emph{P.PPL}|pw}
               oder Campiglio\oindex{Madonna di Campiglio@\textbf{Madonna di Campiglio}, \emph{P.PPL}|pw}, event. München\oindex{Muenchen@\textbf{München}, \emph{P.PPLA}|pw} zum Schluss. – Dass Sie zu {\pb}meinem Roman\pwindex{Weg ins Freie. Roman@\emph{Der Weg ins Freie. Roman}|pwv} kein
               glückliches Verhältnis gefunden haben, war in der That nicht schwer zu merken. Und so
               sehr ich Ihrem Ausspruch beisti{\geminationm}e, dass Sie zwischen mir
               und meinen Arbeiten keine Grenze ziehen können, ich empfinde ihn als doppelt u.
               zehnfach wahr gegenüber einem Werk, das mich in Gedanke u Ausführung durch manches
               reife und \introOben{}höchst\introOben{} bewußte Jahr meines Lebens vornehmlich
               beschäftigt hat. Als so wahr erweist es sich, was Sie selbst zu spüren scheinen, wie
               es kaum denkbar ist, zum Dichter eines Werks, das für eine {\pb}ganze Entwicklungsperiode \substVorne{}\textsuperscript{eines}\substDazwischen{}\label{T_L01786-1v}\edtext{dieses}{\lemma{\textnormal{\emph{dieses}}}\Cendnote{\textnormal{In der ersten Schicht schrieb er »dieses«,
                        ersetzte es dann durch »eines«, um dann wieder zu
                           »dieses« zurückzukehren.}}}\label{T_L01786-1}\substHinten{} Dichters bedeutend ist, in einem glücklichern Verhältnis zu stehen als zu
               der Dichtung selbst und dass ich Ihnen für den Takt dankbar bin, der es Sie als
               richtig erkennen liess, jedes weitre Wort über ein Werk\pwindex{Weg ins Freie. Roman@\emph{Der Weg ins Freie. Roman}|pw} zu unterlassen, das nichts vermocht hat als Sie zu verstören und von
               dem mir ein unverlierbar und untrüglich starkes \introOben{}\strikeout{\textcolor{gray}{×}\-\textcolor{gray}{×}\-\textcolor{gray}{×}\-\textcolor{gray}{×}\-\textcolor{gray}{×}\-\textcolor{gray}{×}\-\textcolor{gray}{×}\-\textcolor{gray}{×}\-\textcolor{gray}{×}\-\textcolor{gray}{×}}\introOben{} Nachgefühl in der Seele geblieben ist. –\pend
           
\pstart
           Auf Wiedersehen im Herbst; Sie bringen hoffentlich viel schönes zum
               vorlesen mit, – lassen Sie sichs beide in Sils\oindex{Sils im Engadin/Segl@\textbf{Sils im Engadin/Segl}, \emph{A.ADM3}|pw}
               wohlergehen.\pend
           \pstart Wir grüßen herzlichst.\spacefill\mbox{Arthur.}\pend{}\selectlanguage{ngerman}\endnumbering\briefempfaengerindex{Hofmannsthal, Hugo von@\textsc{Hofmannsthal, Hugo von}!zzzSchnitzler, Arthur@\emph{von Arthur Schnitzler}!1908-08-061@{6. 8. 1908}|)be}\mylabel{L01786h}  \normalsize

\doendnotes{C}
\bigskip
\vfill

\clearpage

\footnotesize

\lohead{\textsc{register}}

% Definiere theindex-Environment komplett neu ohne reledmac
\makeatletter
\renewenvironment{theindex}{%
  \section*{\indexname}%
  \setlength{\parindent}{0pt}%
  \setlength{\parskip}{0pt plus 0.3pt}%
  \let\item\@idxitem
}{%
  \clearpage
}
\makeatother

\IfFileExists{\jobname-pw.ind}{\input{\jobname-pw.ind}}{}

\end{document}

      