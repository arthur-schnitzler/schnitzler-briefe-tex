%% latex-korrekturansicht-vorspann.tex
%% Vorspann für die Korrekturansicht.
%% Lädt die gemeinsame Datei latex-vorspann.tex mit gesetztem Schalter.

\newif\ifkorrekturansicht
\korrekturansichttrue

\input{../tex-inputs/latex-vorspann}


\section[Arthur Schnitzler an Hugo von Hofmannsthal, {[}25.? 7. 1892{]}]{L00107 Arthur Schnitzler an Hugo von Hofmannsthal, {[}25.? 7. 1892{]}}
\nopagebreak\mylabel{L00107v}
\rehead{ }\normalsize\beginnumbering\briefempfaengerindex{Hofmannsthal, Hugo von@\textsc{Hofmannsthal, Hugo von}!zzzSchnitzler, Arthur@\emph{von Arthur Schnitzler}!1892-07-251@{{[}25.? 7. 1892{]}}|(be}
\toendnotes[C]{\smallbreak\pagebreak[2]}\Standort{FDH, Hs-30885,23.}
\physDesc{Brief, 1 Blatt, 3 Seiten, 653 Zeichen
\newline{}Handschrift: Bleistift, deutsche Kurrent
\newline{}Ordnung: 1) mit Bleistift von Schnitzler mutmaßlich bei der Durchsicht der Korrespondenz
                                 1929 datiert: »92«  2) mit Bleistift von unbekannter Hand datiert: »So{\geminationm}er«}
\buchAbdrucke{\weitereDrucke{Hugo von Hofmannsthal, Arthur Schnitzler: \emph{Briefwechsel}. Frankfurt am Main: \emph{S. Fischer} 1964, S. 24.} }\toendnotes[C]{\smallbreak}
\pstart
           \noindent{}{\pb}Lieber Loris! Nächſtens mehr! Heute nur eine
               Frage. – Mein Anatol Cyclus\pwindex{Anatol@\emph{Anatol}|pw} erſcheint im
                  October im \textsc{Bibl. Bureau}\orgindex{Bibliographisches Bureau@Bibliographisches Bureau|pw} (nächſtens näheres). – Ihr Gedicht\pwindex{Einleitung@\emph{Einleitung}|pwv} leitet die Sa{\geminationm}lung ein; wollen Sie ihm
               irgend einen Namen geben; haben Sie ſonſt irgendwelche Wünſche? Möchten Sie im {\pb}Inhalt verzeichnet ſein? –\pend
           
\pstart
           – In ein paar Tagen beginnt die Drucklegung.\pend
           
\pstart
           Auf Ihren erfreulichen Brief muſs ich Ihnen noch antworten. – Bitte baldige
               Auskunft. – Haben Sie ſchon bemerkt, wie miſerabel die »Agonie\pwindex{Agonie@\emph{Agonie}|pw}« iſt? – Gut iſt nur {\pb}Frage an das Schickſal\pwindex{Frage an das Schicksal@\emph{Die Frage an das Schicksal}|pw} wie Epiſode\pwindex{Episode@\emph{Episode}|pw}.\pend
           
\pstart
           Wie gehts Ihrem Stück? –\pend
           
\pstart
           Meine Novelle\pwindex{Sterben. Novelle@\emph{Sterben. Novelle}|pwv} iſt in 2, 3
               Tagen beendet – ich habe nemlich Zeit, während der Ordinationsſtunde zu
               ſchreiben!\pend
           \pstart Ihr \spacefill\mbox{Arthur}\pend{}\selectlanguage{ngerman}\endnumbering\briefempfaengerindex{Hofmannsthal, Hugo von@\textsc{Hofmannsthal, Hugo von}!zzzSchnitzler, Arthur@\emph{von Arthur Schnitzler}!1892-07-251@{{[}25.? 7. 1892{]}}|)be}\mylabel{L00107h}  \normalsize

\doendnotes{C}
\bigskip
\vfill

\clearpage

\footnotesize

\lohead{\textsc{register}}

% Definiere theindex-Environment komplett neu ohne reledmac
\makeatletter
\renewenvironment{theindex}{%
  \section*{\indexname}%
  \setlength{\parindent}{0pt}%
  \setlength{\parskip}{0pt plus 0.3pt}%
  \let\item\@idxitem
}{%
  \clearpage
}
\makeatother

\IfFileExists{\jobname-pw.ind}{\input{\jobname-pw.ind}}{}

\end{document}

      