\input{../tex-inputs/latex-pdf-vorspann}
\begin{center}
            \textcolor{red}{ENTWURF. ENTZIFFERUNG NOCH NICHT KORREKTURGELESEN}
                      \end{center}
            
               \section[Arthur Schnitzler an Hugo von Hofmannsthal, {[}25.? 7. 1892{]}]{ Arthur Schnitzler an Hugo von Hofmannsthal, {[}25.? 7. 1892{]}}\nopagebreak\mylabel{v}\rehead{ }\begin{ledgroupsized}[t]{13cm}\normalsize\beginnumbering\briefempfaengerindex{Hofmannsthal, Hugo von@\textsc{Hofmannsthal, Hugo von}!zzzSchnitzler, Arthur@\emph{von Arthur Schnitzler}!1892-07-251@{{[}25.? 7. 1892{]}}|(be} \toendnotes[C]{\smallbreak\pagebreak[2]} \Standort{FDH, Hs-30885,23.}
\physDesc{Brief, 1 Blatt, 3 Seiten
\newline{}Handschrift: Bleistift, deutsche Kurrent\newline{}Ordnung: von unbekannter Hand datiert: »So{\geminationm}er 92« }\buchAbdrucke{\weitereDrucke{Hugo von Hofmannsthal, Arthur Schnitzler: \emph{Briefwechsel}. Hg. Therese Nickl und Heinrich Schnitzler. Frankfurt am Main: \emph{S. Fischer} 1964, S. 24.} }\toendnotes[C]{\smallbreak}\pstart
           \noindent{}{\pb}Lieber Loris! Nächſtens mehr! Heute nur
                    eine Frage. – Mein Anatol Cyclus\pwindex{Schnitzler, Arthur 15.05.1862 – 21.10.1931@\textsc{Schnitzler, Arthur} (15.05.1862 – 21.10.1931), \emph{Schriftsteller, Mediziner}!Anatol1892-10-29 – 1892-10-29@\strich\emph{Anatol} {[}1892-10-29 – 1892-10-29{]}|pw} erſcheint im
                        October im \textsc{Bibl. Bureau}\orgindex{Bibliographisches Bureau@Bibliographisches Bureau|pw} (nächſtens näheres). – Ihr Gedicht\pwindex{Einleitung1892@\emph{Einleitung} {[}1892{]}|pwv} leitet die Sa{\geminationm}lung ein; wollen
                    Sie ihm irgend einen Namen geben; haben Sie ſonſt irgendwelche Wünſche? Möchten
                    Sie im {\pb}Inhalt verzeichnet ſein? –\pend
           \pstart
           – In ein paar Tagen beginnt die Drucklegung.\pend
           \pstart
           Auf Ihren erfreulichen Brief muſs ich Ihnen noch antworten. – Bitte baldige
                    Auskunft. – Haben Sie ſchon bemerkt, wie miſerabel die »Agonie\pwindex{Schnitzler, Arthur 15.05.1862 – 21.10.1931@\textsc{Schnitzler, Arthur} (15.05.1862 – 21.10.1931), \emph{Schriftsteller, Mediziner}!Agonie1892@\strich\emph{Agonie} {[}1892{]}|pw}« iſt? – Gut iſt nur {\pb}Frage an das Schickſal\pwindex{Schnitzler, Arthur 15.05.1862 – 21.10.1931@\textsc{Schnitzler, Arthur} (15.05.1862 – 21.10.1931), \emph{Schriftsteller, Mediziner}!Frage an das Schicksal01. 05. 1890@\strich\emph{Die Frage an das Schicksal} {[}01. 05. 1890{]}|pw} wie Epiſode\pwindex{Schnitzler, Arthur 15.05.1862 – 21.10.1931@\textsc{Schnitzler, Arthur} (15.05.1862 – 21.10.1931), \emph{Schriftsteller, Mediziner}!Episode8. 09. 1889@\strich\emph{Episode} {[}8. 09. 1889{]}|pw}.\pend
           \pstart
           Wie gehts Ihrem Stück? –\pend
           \pstart
           Meine Novelle\pwindex{Schnitzler, Arthur 15.05.1862 – 21.10.1931@\textsc{Schnitzler, Arthur} (15.05.1862 – 21.10.1931), \emph{Schriftsteller, Mediziner}!Sterben. Novelle1.10.1894 – 1.12.1894@\strich\emph{Sterben. Novelle} {[}1.10.1894 – 1.12.1894{]}|pwv} iſt in 2, 3
                    Tagen beendet – ich habe nemlich Zeit, während der Ordinationsſtunde zu
                    ſchreiben!\pend
           \pstart Ihr \spacefill\mbox{Arthur}\pend{}\endnumbering\briefempfaengerindex{Hofmannsthal, Hugo von@\textsc{Hofmannsthal, Hugo von}!zzzSchnitzler, Arthur@\emph{von Arthur Schnitzler}!1892-07-251@{{[}25.? 7. 1892{]}}|)be}\mylabel{h}\end{ledgroupsized}  \newcommand{\dateiname}{L00107}\newcommand{\titel}{Arthur Schnitzler an Hugo von Hofmannsthal, [25.? 7. 1892]}\newcommand{\editorInnen}{Martin Anton Müller und Gerd-Hermann Susen}\input{../tex-inputs/latex-pdf-abspann}
      