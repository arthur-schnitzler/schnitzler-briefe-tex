%% latex-leseansicht-vorspann.tex
%% Vorspann für die Leseansicht.
%% Lädt die gemeinsame Datei latex-vorspann.tex mit nicht gesetztem Schalter.

\newif\ifkorrekturansicht
\korrekturansichtfalse

\input{../tex-inputs/latex-vorspann}


         
         \renewcommand{\erwaehntePersonen}{Personen: Georg Brandes, Gertrud Rung}
         \renewcommand{\erwaehnteOrte}{Orte: Kopenhagen, Wien}
         \renewcommand{\erwaehnteWerke}{Werke: Die Frau des Richters. Novelle}
               \section[Georg Brandes an Arthur Schnitzler, 30. 12. 1925]{ Georg Brandes an Arthur Schnitzler, 30. 12. 1925}\nopagebreak\mylabel{v}\rehead{ }\begin{ledgroupsized}[t]{13cm}\normalsize\beginnumbering\briefempfaengerindex{Schnitzler, Arthur@\textsc{Schnitzler, Arthur}!zzzBrandes, Georg@\emph{von Georg Brandes}!1925-12-301@{30. 12. 1925}|(be} \toendnotes[C]{\smallbreak\pagebreak[2]} \Standort{CUL, Schnitzler, B 17.}
\physDesc{Briefkarte, 1064 Zeichen
\newline{}Handschrift: schwarze Tinte, lateinische Kurrent
\newline{}Schnitzler: mit rotem Buntstift zwei Unterstreichungen 
\newline{}Ordnung: mit Bleistift von unbekannter Hand nummeriert:
                                    »61« }\buchAbdrucke{\weitereDrucke{Georg Brandes, Arthur Schnitzler: \emph{Ein Briefwechsel}. Hg. Kurt Bergel. Bern: \emph{Francke} 1956, S. 150–151.} }\pstart
           \raggedleft{}{\pb}Kopenhagen\oindex{Kopenhagen@\textbf{Kopenhagen}|pw}{ }30 Dec 25\pend
           \pstart{}Mein liebster Freund\pend\pstart
           Das Jahr ist zu Ende, und ich habe Ihnen unendliches zu danken, dass es in Wien\oindex{Wien@\textbf{Wien}|pw} für mich einigermassen gut ablief. Sie als
               Artzt wissen, dass uralte Menschen meistens beschwerlich sind. Sie haben es mich
               nicht fühlen lassen, aber Ihr Haus in Wien\oindex{Wien@\textbf{Wien}|pw} ist mir
               ein Heim gewesen. Sie haben wol in 35 Jahren unsere Freundschaft ununterbrochen
               bewahrt, \uline{obschon Sie immer mehr leisteten, als ich im
                  Stande war}. Ihre Gastfreundschaft Frau Rung\pwindex{Rung, Gertrud 26.03.1882 – 25.04.1959@\textsc{Rung, Gertrud} (26.03.1882 – 25.04.1959), \emph{Übersetzerin, Sekretärin}|pw} und mir {\pb}gegenüber
               wird mir unvergesslich sein, was freilich ein bischen lächerlich klingt, da die 84
               jährigen sich gewöhnlich nicht lange einer Erinnerung erfreuen können.\pend
           \pstart
           N’importe! So lange wir das Tageslicht sehen, tut es nicht viel, ob wir uns schneller
               oder langsamer bewegen. – Ich habe Ihnen noch nicht für die feine Erzählung \uline{Die Frau des Richters}\pwindex{Schnitzler, Arthur 15.05.1862 – 21.10.1931@\textsc{Schnitzler, Arthur} (15.05.1862 – 21.10.1931), \emph{Schriftsteller, Mediziner}!Frau des Richters. Novelle7.8.1925 – 15.8.1925@\strich\emph{Die Frau des Richters. Novelle} {[}7.8.1925 – 15.8.1925{]}|pw} gedankt, nicht, dass ich Sie weniger schätze, aber ich hatte sie schon irgendwo
               gelesen, bevor sie in Buchform erschien. Mit Freude las ich, dass Sie \uline{Teatererfolge} haben. Arm, wie wir alle sind, ist das
               von Nutzen. Aus vollem Herzen\pend
           \pstart Ihr \spacefill\mbox{Georg Brandes}\pend{}
         
         \endnumbering\mylabel{h}\end{ledgroupsized}  \newcommand{\dateiname}{L02462}\newcommand{\titel}{Georg Brandes an Arthur Schnitzler, 30. 12. 1925}\newcommand{\editorInnen}{Martin Anton Müller und Gerd-Hermann Susen}%% latex-leseansicht-abspann.tex
%% Abspann für die Leseansicht.
%% Der Schalter \ifkorrekturansicht ist bereits durch den Vorspann gesetzt.

%% latex-abspann.tex
%% Gemeinsamer Abspann für Korrekturansicht und Leseansicht.
%% Setzt den Schalter \ifkorrekturansicht voraus (gesetzt in den
%% einbindenden Dateien latex-korrekturansicht-abspann.tex bzw.
%% latex-leseansicht-abspann.tex).
%% ---------------------------------------------------------------

\normalsize

% Das esempio-Environment wird nur in der Leseansicht benötigt
\ifkorrekturansicht\else
\newenvironment{esempio}[3]%
{
    \vspace{1.5ex}
    \rlap{\underline{#1}}
    \par
    \setlength{\parindent}{0cm}
    \nopagebreak
    \leftskip=#2cm
    \rightskip=#3cm
}
{
    \par
}
\fi

\doendnotes{C}
\bigskip
\vfill

\clearpage

\footnotesize

\ifkorrekturansicht
  \lohead{\textsc{register}}
\fi

% theindex-Environment neu definieren ohne reledmac
\makeatletter
\renewenvironment{theindex}{%
  \ifkorrekturansicht
    \section*{\indexname}%
  \else
    \subsubsection*{Index der erwähnten Entitäten}%
  \fi
  \setlength{\parindent}{0pt}%
  \setlength{\parskip}{0pt plus 0.3pt}%
  \let\item\@idxitem
}{%
  \ifkorrekturansicht\clearpage\fi
}
\makeatother

\IfFileExists{\jobname-pw.ind}{\input{\jobname-pw.ind}}{}

% Quellenangabe nur in der Leseansicht
\ifkorrekturansicht\else
% Fallback-Definitionen, falls die .tex-Datei \titel etc. nicht gesetzt hat
\providecommand{\titel}{}
\providecommand{\editorInnen}{}
\providecommand{\dateiname}{\jobname}

\vspace{3cm}

\vfill

\footnotesize
\textsc{Quelle}: \titel. Herausgegeben von {\editorInnen}. In: \emph{Arthur Schnitzler: Briefwechsel mit Autorinnen und Autoren}.
 Digitale Edition, https://schnitzler-briefe.acdh.oeaw.ac.at/{\dateiname}.html (Stand \today)
\fi

\end{document}


      