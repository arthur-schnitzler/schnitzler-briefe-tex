%% latex-leseansicht-vorspann.tex
%% Vorspann für die Leseansicht.
%% Lädt die gemeinsame Datei latex-vorspann.tex mit nicht gesetztem Schalter.

\newif\ifkorrekturansicht
\korrekturansichtfalse

\input{../tex-inputs/latex-vorspann}


               \section[Hermann Bahr an Arthur Schnitzler, 18. 1. 1909]{ Hermann Bahr an Arthur Schnitzler, 18. 1. 1909}\nopagebreak\mylabel{v}\rehead{ }\begin{ledgroupsized}[t]{13cm}\normalsize\beginnumbering\briefempfaengerindex{Schnitzler, Arthur@\textsc{Schnitzler, Arthur}!zzzBahr, Hermann@\emph{von Hermann Bahr}!1909-01-181@{18. 1. 1909}|(be} \toendnotes[C]{\smallbreak\pagebreak[2]} \Standort{CUL, Schnitzler, B 5b.}
\physDesc{Brief, 1 Blatt, 2 Seiten
\newline{}Handschrift: blaue Tinte, deutsche Kurrent
\newline{}Schnitzler: mit Bleistift beschriftet: »Bahr« \newline{}Ordnung: mit Bleistift von unbekannter Hand
                           nummeriert: »155« }\buchAbdrucke{\weitereDrucke{Hermann Bahr, Arthur Schnitzler: \emph{Briefwechsel, Aufzeichnungen, Dokumente (1891–1931)}. Hg. Kurt Ifkovits und Martin Anton Müller. Göttingen: \emph{Wallstein} 2018, S. 414.} }\toendnotes[C]{\smallbreak}\pstart
           \raggedleft{}{\pb}Wien XIII/\textsubscript{7}\oindex{Ober Sankt Veit@\textbf{Ober Sankt Veit}|pw}{\\}18. 1. 09\pend
           \pstart\center{}Lieber Arthur!\pend\pstart
           Danke ſchön für Deine ſo liebe Karte. Ich komme eben vom Semmering\oindex{Semmering@\textbf{Semmering}|pw} (wo ich übrigens Deinen Bruder Julius\pwindex{Schnitzler, Julius 13.07.1865 – 29.06.1939@\textsc{Schnitzler, Julius} (13.07.1865 – 29.06.1939), \emph{Chirurg}|pw}{ }ſtolz im \textsc{Nizza\oindex{Nizza@\textbf{Nizza}|pw}{ }Express} vorüber ſauſen ſah), hab einen
               ſcheußlichen Hexenſchuß, ſitz in einem durch Überſchwemmung aus einem geplatzten
               Waſſerrohr faſt demolierten Haus und ſoll in zwei Tagen nach Dresden\oindex{Dresden@\textbf{Dresden}|pw} zur \label{K_L01824_1v}\edtext{Strauß\pwindex{Strauss, Richard 11.06.1864 – 08.09.1949@\textsc{Strauss, Richard} (11.06.1864 – 08.09.1949), \emph{Theaterleiter, Komponist, Dirigent}|pw}-Elektra\pwindex{Strauss, Richard 11.06.1864 – 08.09.1949@\textsc{Strauss, Richard} (11.06.1864 – 08.09.1949), \emph{Theaterleiter, Komponist, Dirigent}!Elektra (op. 58)25.1.1909 – 25.1.1909@\strich\emph{Elektra (op. 58)} {[}Vertonung, 25.1.1909 – 25.1.1909{]}|pw}\pwindex{\textcolor{red}{\textsuperscript{XXXX1 indx}}!Elektra. Tragoedie in einem Aufzug1903@\strich\emph{Elektra. Tragödie in einem Aufzug} {[}1903{]}|pw}-Première}{\lemma{\textnormal{\emph{Strauß-Elektra-Première}}}\Cendnote{\textnormal{Am
                     25. 1. 1909, Bahr\pwindex{Bahr, Hermann 19.07.1863 – 15.01.1934@\textsc{Bahr, Hermann} (19.07.1863 – 15.01.1934), \emph{Schriftsteller, Kritiker}|pwk} war vom
                     23. bis zum 26. in Dresden\oindex{Dresden@\textbf{Dresden}|pwk}.}}}\label{K_L01824_1h}, weshalb ich, Dir herzlichſt für Deinen guten Willen
               dankend, Dich bitten muß, Deine ſo liebe Abſicht erſt auszuführen, bis ich nächſte
               Woche von Dresden\oindex{Dresden@\textbf{Dresden}|pw} zurück, halbwegs in Ordnung und
               auch mit den drei letzten Kapiteln meines neuen \label{K_L01824_2v}\edtext{Romans\pwindex{Bahr, Hermann 19.07.1863 – 15.01.1934@\textsc{Bahr, Hermann} (19.07.1863 – 15.01.1934), \emph{Schriftsteller, Kritiker}!Drut1909@\strich\emph{Drut} {[}1909{]}|pwv}}{\lemma{\textnormal{\emph{Romans}}}\Cendnote{\textnormal{Hermann Bahr\pwindex{Bahr, Hermann 19.07.1863 – 15.01.1934@\textsc{Bahr, Hermann} (19.07.1863 – 15.01.1934), \emph{Schriftsteller, Kritiker}|pwk}: \emph{Drut. Roman}\pwindex{Bahr, Hermann 19.07.1863 – 15.01.1934@\textsc{Bahr, Hermann} (19.07.1863 – 15.01.1934), \emph{Schriftsteller, Kritiker}!Drut1909@\strich\emph{Drut} {[}1909{]}|pwk}. Berlin: \emph{S. Fischer}\orgindex{S. Fischer Verlag@S. Fischer Verlag|pwk}{ }1909.}}}\label{K_L01824_2h} aus dem Roheſten bin, worauf ich anzufangen hoffe, wieder einem
               Menſchen zu gleichen.\pend
           \pstart
           {\pb}Ich freue mich unendlich \substVorne{}\textsuperscript{D}\substDazwischen{}a\substHinten{}uf Dich, ich hab Dir ja ſo viel, ſo viel zu ſagen und manchmal ist mir ſchon
               ordentlich bang nach Dir. Nur hat ſich mein Leben allmälig ſo merkwürdig geſtellt,
               daß ich mir ſchon wirklich \strikeout{nicht} manchmal vorkomme,
               nicht mehr auf der Erde zu ſein, ſondern nur noch ein hinten her, neben bei irgendwo
               mitſauſendes, nachwirbelndes \label{K_L01824_3v}\edtext{Gehängſel}{\lemma{\textnormal{\emph{Gehängſel}}}\Cendnote{\textnormal{Anhängsel}}}\label{K_L01824_3h}!\pend
           \pstart
           Grüß Deine liebe Frau\pwindex{Schnitzler, Olga 17.01.1882 – 13.01.1970@\textsc{Schnitzler, Olga} (17.01.1882 – 13.01.1970), \emph{Schauspielerin, Sängerin}|pwv} herzlichſt
               von mir, auch den Sohn\pwindex{Schnitzler, Heinrich 09.08.1902 – 12.07.1982@\textsc{Schnitzler, Heinrich} (09.08.1902 – 12.07.1982), \emph{Regisseur, Schauspieler}|pwv}, Herrn
               Sohn muß man jetzt wol bald ſchon ſagen.\pend
           \pstart
           Herzlichſt{\\[\baselineskip]}immer Dein{\\[\baselineskip]}\spacefill\mbox{Hermann}\pend
           \leftskip=0em{}\endnumbering\briefempfaengerindex{Schnitzler, Arthur@\textsc{Schnitzler, Arthur}!zzzBahr, Hermann@\emph{von Hermann Bahr}!1909-01-181@{18. 1. 1909}|)be}\mylabel{h}\end{ledgroupsized}  \newcommand{\dateiname}{L01824}\newcommand{\titel}{Hermann Bahr an Arthur Schnitzler, 18. 1. 1909}\newcommand{\editorInnen}{ Kurt Ifkovits,  Martin Anton Müller}
            \footnotesize
\begin{ledgroupsized}[t]{11.5cm}
\doendnotes{C}
\end{ledgroupsized}
         %% latex-leseansicht-abspann.tex
%% Abspann für die Leseansicht.
%% Der Schalter \ifkorrekturansicht ist bereits durch den Vorspann gesetzt.

%% latex-abspann.tex
%% Gemeinsamer Abspann für Korrekturansicht und Leseansicht.
%% Setzt den Schalter \ifkorrekturansicht voraus (gesetzt in den
%% einbindenden Dateien latex-korrekturansicht-abspann.tex bzw.
%% latex-leseansicht-abspann.tex).
%% ---------------------------------------------------------------

\normalsize

% Das esempio-Environment wird nur in der Leseansicht benötigt
\ifkorrekturansicht\else
\newenvironment{esempio}[3]%
{
    \vspace{1.5ex}
    \rlap{\underline{#1}}
    \par
    \setlength{\parindent}{0cm}
    \nopagebreak
    \leftskip=#2cm
    \rightskip=#3cm
}
{
    \par
}
\fi

\doendnotes{C}
\bigskip
\vfill

\clearpage

\footnotesize

\ifkorrekturansicht
  \lohead{\textsc{register}}
\fi

% theindex-Environment neu definieren ohne reledmac
\makeatletter
\renewenvironment{theindex}{%
  \ifkorrekturansicht
    \section*{\indexname}%
  \else
    \subsubsection*{Index der erwähnten Entitäten}%
  \fi
  \setlength{\parindent}{0pt}%
  \setlength{\parskip}{0pt plus 0.3pt}%
  \let\item\@idxitem
}{%
  \ifkorrekturansicht\clearpage\fi
}
\makeatother

\IfFileExists{\jobname-pw.ind}{\input{\jobname-pw.ind}}{}

% Quellenangabe nur in der Leseansicht
\ifkorrekturansicht\else
% Fallback-Definitionen, falls die .tex-Datei \titel etc. nicht gesetzt hat
\providecommand{\titel}{}
\providecommand{\editorInnen}{}
\providecommand{\dateiname}{\jobname}

\vspace{3cm}

\vfill

\footnotesize
\textsc{Quelle}: \titel. Herausgegeben von {\editorInnen}. In: \emph{Arthur Schnitzler: Briefwechsel mit Autorinnen und Autoren}.
 Digitale Edition, https://schnitzler-briefe.acdh.oeaw.ac.at/{\dateiname}.html (Stand \today)
\fi

\end{document}


      