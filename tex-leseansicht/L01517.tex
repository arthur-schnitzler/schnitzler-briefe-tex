%% latex-korrekturansicht-vorspann.tex
%% Vorspann für die Korrekturansicht.
%% Lädt die gemeinsame Datei latex-vorspann.tex mit gesetztem Schalter.

\newif\ifkorrekturansicht
\korrekturansichttrue

\input{../tex-inputs/latex-vorspann}


\section[Hugo und Gerty von Hofmannsthal an Arthur Schnitzler, 11. 5. 1905]{L01517 Hugo und Gerty von Hofmannsthal an Arthur Schnitzler,
               11. 5. 1905}
\nopagebreak\mylabel{L01517v}
\rehead{ }\normalsize\beginnumbering\briefempfaengerindex{Schnitzler, Arthur@\textsc{Schnitzler, Arthur}!zzzHofmannsthal, Gertrude von@\emph{von Gertrude von Hofmannsthal}!1905-05-112@{11. 5. 1905}|(be}\briefempfaengerindex{Schnitzler, Arthur@\textsc{Schnitzler, Arthur}!zzzHofmannsthal, Hugo von@\emph{von Hugo von Hofmannsthal}!1905-05-112@{11. 5. 1905}|(be}
\toendnotes[C]{\smallbreak\pagebreak[2]}\Standort{CUL, Schnitzler, B 43.}
\physDesc{Bildpostkarte, 55 Zeichen
\newline{}Handschrift Hugo von Hofmannsthal: Bleistift, lateinische Kurrent
\newline{}Handschrift Gertrude von Hofmannsthal: Bleistift
\newline{}Versand: 1) Stempel: »\nobreak{}\oindex{Versailles@\textbf{Versailles}, \emph{P.PPLA2}|pwk}Versailles à Paris, 12 Mai 05\nobreak{}«.   2) mit Bleistift von unbekannter Hand die Bezirksangabe ergänzt:
                                    »XVIII/1«
\newline{}Ordnung: 1) mit Bleistift von unbekannter Hand nummeriert: »\strikeout{242}«  2) mit Bleistift von unbekannter Hand nummeriert:
                                    »253«}
\buchAbdrucke{\weitereDrucke{Hugo von Hofmannsthal, Arthur Schnitzler: \emph{Briefwechsel}. Frankfurt am Main: \emph{S. Fischer} 1964, S. 211.} }\toendnotes[C]{\smallbreak}\pstart{}{\pb}D\textsuperscript{r}
                  Arthur Schnitzler\pend{}\pstart{}Wien\oindex{Wien@\textbf{Wien}, \emph{A.ADM2}|pw}\pend{}\pstart{}Autriche\oindex{Oesterreich@\textbf{Österreich}, \emph{A.PCLI}|pw}\pend{}{\bigskip}
\pstart
           \noindent{}\centering{}{\pb}\textcolor{gray}{\textbf{124. \emph{Trianon artistique}. – Le Moulin\oindex{Versailles@\textbf{Versailles}, \emph{P.PPLA2}|pwv}.}}\pend
           \vspace{1em}
\pstart
           {\pb}Viele Grüße \spacefill\mbox{Hugo}{\\[\baselineskip]}\spacefill\mbox{{[}hs. :{]} Gerty}\pend
           \leftskip=0em{}
\pstart
           11 V\pend
           \selectlanguage{ngerman}\endnumbering\briefempfaengerindex{Schnitzler, Arthur@\textsc{Schnitzler, Arthur}!zzzHofmannsthal, Gertrude von@\emph{von Gertrude von Hofmannsthal}!1905-05-112@{11. 5. 1905}|)be}\briefempfaengerindex{Schnitzler, Arthur@\textsc{Schnitzler, Arthur}!zzzHofmannsthal, Hugo von@\emph{von Hugo von Hofmannsthal}!1905-05-112@{11. 5. 1905}|)be}\mylabel{L01517h}  \normalsize

\doendnotes{C}
\bigskip
\vfill

\clearpage

\footnotesize

\lohead{\textsc{register}}

% Definiere theindex-Environment komplett neu ohne reledmac
\makeatletter
\renewenvironment{theindex}{%
  \section*{\indexname}%
  \setlength{\parindent}{0pt}%
  \setlength{\parskip}{0pt plus 0.3pt}%
  \let\item\@idxitem
}{%
  \clearpage
}
\makeatother

\IfFileExists{\jobname-pw.ind}{\input{\jobname-pw.ind}}{}

\end{document}

      