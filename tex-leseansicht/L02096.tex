%% latex-korrekturansicht-vorspann.tex
%% Vorspann für die Korrekturansicht.
%% Lädt die gemeinsame Datei latex-vorspann.tex mit gesetztem Schalter.

\newif\ifkorrekturansicht
\korrekturansichttrue

\input{../tex-inputs/latex-vorspann}


\section[Hugo von Hofmannsthal an Arthur Schnitzler, 13. 11. 1912]{L02096 Hugo von Hofmannsthal an Arthur Schnitzler, 13. 11. 1912}
\nopagebreak\mylabel{L02096v}
\rehead{ }\normalsize\beginnumbering\briefempfaengerindex{Schnitzler, Arthur@\textsc{Schnitzler, Arthur}!zzzHofmannsthal, Hugo von@\emph{von Hugo von Hofmannsthal}!1912-11-131@{13. 11. 1912}|(be}
\toendnotes[C]{\smallbreak\pagebreak[2]}\Standort{CUL, Schnitzler, B 43.}
\physDesc{Postkarte, 572 Zeichen
\newline{}Handschrift: 1) schwarze Tinte, deutsche Kurrent\hspace{1em}2) schwarze Tinte, lateinische Kurrent (\noindent{}Adresse)\hspace{1em}
\newline{}Versand: Stempel: »\nobreak{}\oindex{Rodaun@\textbf{Rodaun}, \emph{A.ADM4}|pwk}Rodaun, 14 11 12, 3N\nobreak{}«.  
\newline{}Ordnung: 1) mit Bleistift von unbekannter Hand nummeriert:
                                    »381«  2) mit Bleistift von unbekannter Hand nummeriert:
                                    »342«}
\buchAbdrucke{\weitereDrucke{Hugo von Hofmannsthal, Arthur Schnitzler: \emph{Briefwechsel}. Frankfurt am Main: \emph{S. Fischer} 1964, S. 269.} }\toendnotes[C]{\smallbreak}\pstart{}{\pb}Herrn D\textsuperscript{r} Arthur Schnitzler\pend{}\pstart{}Wien\oindex{Wien@\textbf{Wien}, \emph{A.ADM2}|pw}\pend{}\pstart{}XVIII. Sternwartestrasse 71\oindex{Sternwartestrasse 71@\textbf{Sternwartestraße 71}, \emph{Wohngebäude (K.WHS)}|pw}.\pend{}{\bigskip}\vspace{1em}
\pstart
           \centering{}{\pb}13 XI.\pend
           \vspace{0.5em}
\pstart
           Retourniere gleicher Poſt im So{\geminationm}er entliehene Bücher.
                  Varnhagen\pwindex{Varnhagen-Ense, Karl August von 21.02.1785 – 10.10.1858@\textsc{Varnhagen-Ense, Karl August von} (21.02.1785 – 10.10.1858), \emph{Schriftsteller/Schriftstellerin, Diplomat/Diplomatin}|pw} Band III.\pwindex{Tagebuecher@\emph{Tagebücher}|pwv} hat Waſſermann\pwindex{Wassermann, Jakob 10.03.1873 – 01.01.1934@\textsc{Wassermann, Jakob} (10.03.1873 – 01.01.1934), \emph{Schriftsteller/Schriftstellerin}|pw} trotz meines Widerſtrebens an ſich
                  geno{\geminationm}en, auf \uline{eigene
                  Verantwortung}, {\pb}und Ihnen
               in Wien\oindex{Wien@\textbf{Wien}, \emph{A.ADM2}|pw}{ }ſofort zurückzuſtellen geſchworen.\pend
           
\pstart
           Ich gehe, nach Überlegung, Sonntag{ }abends zu dem Hauptmann\pwindex{Hauptmann, Gerhart 15.11.1862 – 06.06.1946@\textsc{Hauptmann, Gerhart} (15.11.1862 – 06.06.1946), \emph{Schriftsteller/Schriftstellerin}|pw}-banquett
               der \textsc{Concordia}\orgindex{Concordia. Journalisten- und Schriftstellerverein@Concordia. Journalisten- und Schriftstellerverein|pw} weil ich es abſurd finde, daſs einem Menſchen wie H.\pwindex{Hauptmann, Gerhart 15.11.1862 – 06.06.1946@\textsc{Hauptmann, Gerhart} (15.11.1862 – 06.06.1946), \emph{Schriftsteller/Schriftstellerin}|pw} gegenüber, nicht ein anſtändiger Menſch an dem ganzen Tisch
               ſitzt.\pend
           
\pstart
           Wäre ſehr froh, wenn Sie allenfalls ſchon zurück wären und ſich gleichfalls \label{K_L02096-1v}\edtext{hinzugehen entſchlöſſen}{\lemma{\textnormal{\emph{hinzugehen entſchlöſſen}}}\Cendnote{\textnormal{Schnitzler ging am 17. 11. 1912 zum Hauptmann\pwindex{Hauptmann, Gerhart 15.11.1862 – 06.06.1946@\textsc{Hauptmann, Gerhart} (15.11.1862 – 06.06.1946), \emph{Schriftsteller/Schriftstellerin}|pwk}bankett, Hofmannsthal\pwindex{Hofmannsthal, Hugo von 1874-02-01 – 1929-07-15@\textsc{Hofmannsthal, Hugo von} (1874-02-01 – 1929-07-15), \emph{Schriftsteller/Schriftstellerin}|pwk} wegen eines Streits mit Salten\pwindex{Salten, Felix 06.09.1869 – 08.10.1945@\textsc{Salten, Felix} (06.09.1869 – 08.10.1945), \emph{Schriftsteller/Schriftstellerin, Journalist/Journalistin, Chefredakteur/Chefredakteurin}|pwk} nicht (vgl. A. S.: \emph{Tagebuch}, 15. 11. 1912).}}}\label{K_L02096-1}.\pend
           
\pstart
           Herzlich{\\[\baselineskip]}\spacefill\mbox{Hugo.}\pend
           \leftskip=0em{}\selectlanguage{ngerman}\endnumbering\briefempfaengerindex{Schnitzler, Arthur@\textsc{Schnitzler, Arthur}!zzzHofmannsthal, Hugo von@\emph{von Hugo von Hofmannsthal}!1912-11-131@{13. 11. 1912}|)be}\mylabel{L02096h}  \normalsize

\doendnotes{C}
\bigskip
\vfill

\clearpage

\footnotesize

\lohead{\textsc{register}}

% Definiere theindex-Environment komplett neu ohne reledmac
\makeatletter
\renewenvironment{theindex}{%
  \section*{\indexname}%
  \setlength{\parindent}{0pt}%
  \setlength{\parskip}{0pt plus 0.3pt}%
  \let\item\@idxitem
}{%
  \clearpage
}
\makeatother

\IfFileExists{\jobname-pw.ind}{\input{\jobname-pw.ind}}{}

\end{document}

      