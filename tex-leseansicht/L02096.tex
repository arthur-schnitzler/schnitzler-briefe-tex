%% latex-leseansicht-vorspann.tex
%% Vorspann für die Leseansicht.
%% Lädt die gemeinsame Datei latex-vorspann.tex mit nicht gesetztem Schalter.

\newif\ifkorrekturansicht
\korrekturansichtfalse

\input{../tex-inputs/latex-vorspann}


               \section[Hugo von Hofmannsthal an Arthur Schnitzler, 13. 11. 1912]{ Hugo von Hofmannsthal an Arthur Schnitzler, 13. 11. 1912}\nopagebreak\mylabel{v}\rehead{ }\begin{ledgroupsized}[t]{13cm}\normalsize\beginnumbering\briefempfaengerindex{Schnitzler, Arthur@\textsc{Schnitzler, Arthur}!zzzHofmannsthal, Hugo von@\emph{von Hugo von Hofmannsthal}!1912-11-131@{13. 11. 1912}|(be} \toendnotes[C]{\smallbreak\pagebreak[2]} \Standort{CUL, Schnitzler, B 43.}
\physDesc{Postkarte
\newline{}Handschrift: schwarze Tinte, deutsche Kurrent\newline{}Versand: Stempel: »\nobreak{}\oindex{Rodaun@\textbf{Rodaun}|pwk}Rodaun, 14 11 12, 3N\nobreak{}«.  \newline{}Ordnung: 1) mit Bleistift von unbekannter Hand nummeriert:
                                    »381« 2) mit Bleistift von unbekannter Hand nummeriert:
                                    »342«}\buchAbdrucke{\weitereDrucke{Hugo von Hofmannsthal, Arthur Schnitzler: \emph{Briefwechsel}. Hg. Therese Nickl und Heinrich Schnitzler. Frankfurt am Main: \emph{S. Fischer} 1964, S. 269.} }\toendnotes[C]{\smallbreak}\pstart{}{\pb}\textsc{Herrn D\textsuperscript{r} Arthur Schnitzler}\pend{}\pstart{}\textsc{Wien}\oindex{Wien@\textbf{Wien}|pw}\pend{}\pstart{}\textsc{XVIII. Sternwartestrasse 71\oindex{Sternwartestrasse@\textbf{Sternwartestraße}|pw}.}\pend{}{\bigskip}\pstart
           \centering{}{\pb}13 XI.\pend
           \pstart
           Retourniere gleicher Poſt im So{\geminationm}er entliehene Bücher.
                  Varnhagen\pwindex{Varnhagen-Ense, Karl August von 21.02.1785 – 10.10.1858@\textsc{Varnhagen-Ense, Karl August von} (21.02.1785 – 10.10.1858), \emph{Schriftsteller, Diplomat}|pw} Band III.\pwindex{Varnhagen-Ense, Karl August von 21.02.1785 – 10.10.1858@\textsc{Varnhagen-Ense, Karl August von} (21.02.1785 – 10.10.1858), \emph{Schriftsteller, Diplomat}!Tagebuecher1861 – 1870@\strich\emph{Tagebücher} {[}1861 – 1870{]}|pwv} hat Waſſermann\pwindex{Wassermann, Jakob 10.03.1873 – 01.01.1934@\textsc{Wassermann, Jakob} (10.03.1873 – 01.01.1934), \emph{Schriftsteller}|pw} trotz meines Widerſtrebens an ſich geno{\geminationm}en, auf \uline{eigene
                  Verantwortung}, {\pb}und Ihnen
               in Wien\oindex{Wien@\textbf{Wien}|pw}{ }ſofort zurückzuſtellen geſchworen.\pend
           \pstart
           Ich gehe, nach Überlegung, Sonntag{ }abends zu dem Hauptmann\pwindex{Hauptmann, Gerhart 15.11.1862 – 06.06.1946@\textsc{Hauptmann, Gerhart} (15.11.1862 – 06.06.1946), \emph{Schriftsteller}|pw}-banquett der
                  \textsc{Concordia}\orgindex{Concordia@Concordia|pw} weil ich es abſurd finde, daſs einem Menſchen wie H.\pwindex{Hauptmann, Gerhart 15.11.1862 – 06.06.1946@\textsc{Hauptmann, Gerhart} (15.11.1862 – 06.06.1946), \emph{Schriftsteller}|pw} gegenüber, nicht ein anſtändiger Menſch an dem ganzen Tisch
               ſitzt.\pend
           \pstart
           Wäre ſehr froh, wenn Sie allenfalls ſchon zurück wären und ſich gleichfalls \label{K_L02096_1v}\edtext{hinzugehen entſchlöſſen}{\lemma{\textnormal{\emph{hinzugehen entſchlöſſen}}}\Cendnote{\textnormal{Schnitzler\pwindex{Schnitzler, Arthur 15.05.1862 – 21.10.1931@\textsc{Schnitzler, Arthur} (15.05.1862 – 21.10.1931), \emph{Schriftsteller, Mediziner}|pwk} ging hin (Vgl. A. S.: \emph{Tagebuch}, 17. 11. 1912), Hofmannsthal\pwindex{Hofmannsthal, Hugo von 01.02.1874 – 15.07.1929@\textsc{Hofmannsthal, Hugo von} (01.02.1874 – 15.07.1929), \emph{Schriftsteller}|pwk} wegen eines Streits mit Salten\pwindex{Salten, Felix 06.09.1869 – 08.10.1945@\textsc{Salten, Felix} (06.09.1869 – 08.10.1945), \emph{Schriftsteller, Journalist}|pwk} nicht (Vgl. A. S.: \emph{Tagebuch}, 15. 11. 1912).}}}\label{K_L02096_1h}.\pend
           \pstart
           Herzlich{\\[\baselineskip]}\spacefill\mbox{Hugo.}\pend
           \leftskip=0em{}\endnumbering\briefempfaengerindex{Schnitzler, Arthur@\textsc{Schnitzler, Arthur}!zzzHofmannsthal, Hugo von@\emph{von Hugo von Hofmannsthal}!1912-11-131@{13. 11. 1912}|)be}\mylabel{h}\end{ledgroupsized}  \newcommand{\dateiname}{L02096}\newcommand{\titel}{Hugo von Hofmannsthal an Arthur Schnitzler, 13. 11. 1912}\newcommand{\editorInnen}{Martin Anton Müller und Gerd-Hermann Susen}
            \footnotesize
\begin{ledgroupsized}[t]{11.5cm}
\doendnotes{C}
\end{ledgroupsized}
         %% latex-leseansicht-abspann.tex
%% Abspann für die Leseansicht.
%% Der Schalter \ifkorrekturansicht ist bereits durch den Vorspann gesetzt.

%% latex-abspann.tex
%% Gemeinsamer Abspann für Korrekturansicht und Leseansicht.
%% Setzt den Schalter \ifkorrekturansicht voraus (gesetzt in den
%% einbindenden Dateien latex-korrekturansicht-abspann.tex bzw.
%% latex-leseansicht-abspann.tex).
%% ---------------------------------------------------------------

\normalsize

% Das esempio-Environment wird nur in der Leseansicht benötigt
\ifkorrekturansicht\else
\newenvironment{esempio}[3]%
{
    \vspace{1.5ex}
    \rlap{\underline{#1}}
    \par
    \setlength{\parindent}{0cm}
    \nopagebreak
    \leftskip=#2cm
    \rightskip=#3cm
}
{
    \par
}
\fi

\doendnotes{C}
\bigskip
\vfill

\clearpage

\footnotesize

\ifkorrekturansicht
  \lohead{\textsc{register}}
\fi

% theindex-Environment neu definieren ohne reledmac
\makeatletter
\renewenvironment{theindex}{%
  \ifkorrekturansicht
    \section*{\indexname}%
  \else
    \subsubsection*{Index der erwähnten Entitäten}%
  \fi
  \setlength{\parindent}{0pt}%
  \setlength{\parskip}{0pt plus 0.3pt}%
  \let\item\@idxitem
}{%
  \ifkorrekturansicht\clearpage\fi
}
\makeatother

\IfFileExists{\jobname-pw.ind}{\input{\jobname-pw.ind}}{}

% Quellenangabe nur in der Leseansicht
\ifkorrekturansicht\else
% Fallback-Definitionen, falls die .tex-Datei \titel etc. nicht gesetzt hat
\providecommand{\titel}{}
\providecommand{\editorInnen}{}
\providecommand{\dateiname}{\jobname}

\vspace{3cm}

\vfill

\footnotesize
\textsc{Quelle}: \titel. Herausgegeben von {\editorInnen}. In: \emph{Arthur Schnitzler: Briefwechsel mit Autorinnen und Autoren}.
 Digitale Edition, https://schnitzler-briefe.acdh.oeaw.ac.at/{\dateiname}.html (Stand \today)
\fi

\end{document}


      