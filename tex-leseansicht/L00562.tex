%% latex-korrekturansicht-vorspann.tex
%% Vorspann für die Korrekturansicht.
%% Lädt die gemeinsame Datei latex-vorspann.tex mit gesetztem Schalter.

\newif\ifkorrekturansicht
\korrekturansichttrue

\input{../tex-inputs/latex-vorspann}


\section[Richard Beer-Hofmann an Arthur Schnitzler, 9. 7. 1896]{L00562 Richard Beer-Hofmann an Arthur Schnitzler, 9. 7. 1896}
\nopagebreak\mylabel{L00562v}
\rehead{ }\normalsize\beginnumbering\briefempfaengerindex{Schnitzler, Arthur@\textsc{Schnitzler, Arthur}!zzzBeer-Hofmann, Richard@\emph{von Richard Beer-Hofmann}!1896-07-092@{9. 7. 1896}|(be}
\toendnotes[C]{\smallbreak\pagebreak[2]}\Standort{CUL, Schnitzler, B 8.}
\physDesc{Brief, 1 Blatt, 3 Seiten, 665 Zeichen
\newline{}Handschrift: Bleistift, lateinische Kurrent
\newline{}Ordnung: mit Bleistift von unbekannter Hand nummeriert:
                                    »90« }
\buchAbdrucke{\weitereDrucke{Arthur Schnitzler, Richard Beer-Hofmann: \emph{Briefwechsel 1891–1931}. Wien, Zürich: \emph{Europaverlag} 1992, S. 92.} }
\pstart
           \raggedleft{}{\pb}Fürberg\oindex{Fuerberg@\textbf{Fürberg}, \emph{P.PPL}|pw}{ }9/VII 96\pend
           \vspace{0.5em}
\pstart
           Lieber Arthur! Ich reise heute von Fürberg\oindex{Fuerberg@\textbf{Fürberg}, \emph{P.PPL}|pw}; die Leute die unsere Wohnung für den So{\geminationm}er gemiethet haben ko{\geminationm}en morgen, und in den Dachzi{\geminationm}ern, die dumpf und unruhig sind halt ichs nicht aus.
               Ich gehe also auf einige Tage nach Salzburg\oindex{Salzburg@\textbf{Salzburg}, \emph{A.ADM2}|pw}.\pend
           
\pstart
           Gegen \uline{20} dürfte ich in Kopenhagen\oindex{Kopenhagen@\textbf{Kopenhagen}, \emph{P.PPLC}|pw} sein.
               Schreiben Sie poste restante hin. Nicht {\pb}nur nicht für eine Karte auch nicht
               für einen Brief eignet sich mein versti{\geminationm}ter Zustand.
                  Versti{\geminationm}t ist so richtig. Es klingt alles falsch und
               hässlich. Also, ich will ja nichts mit dem Brief als daß Sie in Trondjhem\oindex{Trondheim@\textbf{Trondheim}, \emph{P.PPLA2}|pw} einen Gruß von mir vorfinden. Ich grüße Sie, und
               wünsche Ihnen heitere sonnige {\pb}Fahrt. Und wir sehen uns ja bald?\pend
           
\pstart
           Herzlichst Ihr{\\[\baselineskip]}\spacefill\mbox{Richard}\pend
           \leftskip=0em{}\selectlanguage{ngerman}\endnumbering\briefempfaengerindex{Schnitzler, Arthur@\textsc{Schnitzler, Arthur}!zzzBeer-Hofmann, Richard@\emph{von Richard Beer-Hofmann}!1896-07-092@{9. 7. 1896}|)be}\mylabel{L00562h}  \normalsize

\doendnotes{C}
\bigskip
\vfill

\clearpage

\footnotesize

\lohead{\textsc{register}}

% Definiere theindex-Environment komplett neu ohne reledmac
\makeatletter
\renewenvironment{theindex}{%
  \section*{\indexname}%
  \setlength{\parindent}{0pt}%
  \setlength{\parskip}{0pt plus 0.3pt}%
  \let\item\@idxitem
}{%
  \clearpage
}
\makeatother

\IfFileExists{\jobname-pw.ind}{\input{\jobname-pw.ind}}{}

\end{document}

      