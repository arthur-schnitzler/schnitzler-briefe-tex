%% latex-leseansicht-vorspann.tex
%% Vorspann für die Leseansicht.
%% Lädt die gemeinsame Datei latex-vorspann.tex mit nicht gesetztem Schalter.

\newif\ifkorrekturansicht
\korrekturansichtfalse

\input{../tex-inputs/latex-vorspann}


\section[Arthur Schnitzler an Hugo von Hofmannsthal, 22. 10. 1911]{L02041 Arthur Schnitzler an Hugo von Hofmannsthal, 22. 10. 1911}
\nopagebreak\mylabel{L02041v}
\rehead{ }\normalsize\beginnumbering\briefempfaengerindex{Hofmannsthal, Hugo von@\textsc{Hofmannsthal, Hugo von}!zzzSchnitzler, Arthur@\emph{von Arthur Schnitzler}!1911-10-221@{22. 10. 1911}|(be}
\toendnotes[C]{\smallbreak\pagebreak[2]}
\correspDesc{Versand  durch Arthur Schnitzler am 22. 10. 1911 in Wien
\newline{}Erhalt  durch Hugo von Hofmannsthal im Zeitraum [22. 10. 1911 – 26. 10. 1911?] in Wien}\toendnotes[C]{\smallbreak}
\Standort{FDH, Hs-30885,144.}
\physDesc{Briefkarte, 483 Zeichen
\newline{}Handschrift: schwarze Tinte, deutsche Kurrent}
\buchAbdrucke{\weitereDrucke{Hugo von Hofmannsthal, Arthur Schnitzler: \emph{Briefwechsel}. Herausgegeben von Therese Nickl und Heinrich Schnitzler. Frankfurt am Main: \emph{S. Fischer} 1964, S. 264.} }\toendnotes[C]{\smallbreak}
\pstart
           \raggedleft{}{\pb}22/X 911\pend
           
\pstart
           \textcolor{gray}{\textbf{A. S.}}\pend
           \vspace{0.5em}
\pstart
           mein lieber Hugo, ich danke für Ihr liebes Telegra{\geminationm} aus Neubeuern\oindex{Neubeuern@\textbf{Neubeuern}|pw}, das ich für alle Fälle{ }ſchon nach Rodaun\oindex{Wien@\textbf{Wien}!XXIII., Liesing@\textbf{XXIII., Liesing}!Rodaun@\textbf{Rodaun}, \emph{Region}|pw} beantworte. Ich \label{K_L02041-1v}\edtext{reiſe}{\lemma{\textnormal{\emph{reise}}}\Cendnote{\textnormal{Schnitzler reiste am 29. 10. 1911 über
                            Prag\oindex{Prag@\textbf{Prag}, \emph{Land}|pwk} nach Berlin\oindex{Berlin@\textbf{Berlin}, \emph{Hauptstadt}|pwk}, Hamburg\oindex{Hamburg@\textbf{Hamburg}|pwk},
                            München\oindex{München@\textbf{München}|pwk} und Garmisch-Partenkirchen\oindex{Garmisch-Partenkirchen@\textbf{Garmisch-Partenkirchen}, \emph{Hauptstadt}|pwk}. Am 17. 11. 1911 war er
                        wieder in Wien\oindex{Wien@\textbf{Wien}, \emph{Verwaltungsgebiet}|pwk}.}}}\label{K_L02041-1} Ende der Woche ab,
                        Prag\oindex{Prag@\textbf{Prag}, \emph{Land}|pw}, Dresden\oindex{Dresden@\textbf{Dresden}|pw} (Vorleſungen), – da{\geminationn}{ }Berlin\oindex{Berlin@\textbf{Berlin}, \emph{Hauptstadt}|pw} – Hamburg\oindex{Hamburg@\textbf{Hamburg}|pw} (Beatrice\pwindex{Schnitzler, Arthur 15.\,5.\,1862 Wien – 21.\,10.\,1931 ebd.@\textsc{Schnitzler, Arthur} (15.\,5.\,1862 Wien – 21.\,10.\,1931 ebd.), \emph{Schriftsteller, Mediziner}!Schleier der Beatrice. Schauspiel in fünf Akten@\strich\emph{Der Schleier der Beatrice. Schauspiel in fünf Akten}|pw}, Weites Land\pwindex{Schnitzler, Arthur 15.\,5.\,1862 Wien – 21.\,10.\,1931 ebd.@\textsc{Schnitzler, Arthur} (15.\,5.\,1862 Wien – 21.\,10.\,1931 ebd.), \emph{Schriftsteller, Mediziner}!weite Land. Tragikomödie in fünf Akten@\strich\emph{Das weite Land. Tragikomödie in fünf Akten}|pw}, Anatol\pwindex{Schnitzler, Arthur 15.\,5.\,1862 Wien – 21.\,10.\,1931 ebd.@\textsc{Schnitzler, Arthur} (15.\,5.\,1862 Wien – 21.\,10.\,1931 ebd.), \emph{Schriftsteller, Mediziner}!Anatol@\strich\emph{Anatol}|pw}) bin gegen {\pb}Mitte November zurück. Vorher werden wir einander wohl kaum{ }ſehen.
                    Für Herbſt und Winter aber hoff ich ein häufigeres
                        Zuſa{\geminationm}enſein als es mir die letzten Jahre
                    beſchieden war. Was iſt’s mit »Jedermann\pwindex{Hofmannsthal, Hugo von 1.\,2.\,1874 Wien – 15.\,7.\,1929 Rodaun@\textsc{Hofmannsthal, Hugo von} (1.\,2.\,1874 Wien – 15.\,7.\,1929 Rodaun), \emph{Schriftsteller}!Jedermann. Das Spiel vom Sterben des reichen Mannes@\strich\emph{Jedermann. Das Spiel vom Sterben des reichen Mannes}|pw}«
                    und Allerlei?\pend
           
\pstart
           Wir grüßen Euch herzlichſt!{\\[\baselineskip]}Ihr{\\[\baselineskip]}\spacefill\mbox{Arthur.}\pend
           \leftskip=0em{}\selectlanguage{ngerman}\endnumbering\briefempfaengerindex{Hofmannsthal, Hugo von@\textsc{Hofmannsthal, Hugo von}!zzzSchnitzler, Arthur@\emph{von Arthur Schnitzler}!1911-10-221@{22. 10. 1911}|)be}\mylabel{L02041h}  \newcommand{\dateiname}{L02041}\newcommand{\titel}{Arthur Schnitzler an Hugo von Hofmannsthal, 22. 10. 1911}\newcommand{\editorInnen}{Martin Anton Müller und Gerd-Hermann Susen}%% latex-leseansicht-abspann.tex
%% Abspann für die Leseansicht.
%% Der Schalter \ifkorrekturansicht ist bereits durch den Vorspann gesetzt.

%% latex-abspann.tex
%% Gemeinsamer Abspann für Korrekturansicht und Leseansicht.
%% Setzt den Schalter \ifkorrekturansicht voraus (gesetzt in den
%% einbindenden Dateien latex-korrekturansicht-abspann.tex bzw.
%% latex-leseansicht-abspann.tex).
%% ---------------------------------------------------------------

\normalsize

% Das esempio-Environment wird nur in der Leseansicht benötigt
\ifkorrekturansicht\else
\newenvironment{esempio}[3]%
{
    \vspace{1.5ex}
    \rlap{\underline{#1}}
    \par
    \setlength{\parindent}{0cm}
    \nopagebreak
    \leftskip=#2cm
    \rightskip=#3cm
}
{
    \par
}
\fi

\doendnotes{C}
\bigskip
\vfill

\clearpage

\footnotesize

\ifkorrekturansicht
  \lohead{\textsc{register}}
\fi

% theindex-Environment neu definieren ohne reledmac
\makeatletter
\renewenvironment{theindex}{%
  \ifkorrekturansicht
    \section*{\indexname}%
  \else
    \subsubsection*{Index der erwähnten Entitäten}%
  \fi
  \setlength{\parindent}{0pt}%
  \setlength{\parskip}{0pt plus 0.3pt}%
  \let\item\@idxitem
}{%
  \ifkorrekturansicht\clearpage\fi
}
\makeatother

\IfFileExists{\jobname-pw.ind}{\input{\jobname-pw.ind}}{}

% Quellenangabe nur in der Leseansicht
\ifkorrekturansicht\else
% Fallback-Definitionen, falls die .tex-Datei \titel etc. nicht gesetzt hat
\providecommand{\titel}{}
\providecommand{\editorInnen}{}
\providecommand{\dateiname}{\jobname}

\vspace{3cm}

\vfill

\footnotesize
\textsc{Quelle}: \titel. Herausgegeben von {\editorInnen}. In: \emph{Arthur Schnitzler: Briefwechsel mit Autorinnen und Autoren}.
 Digitale Edition, https://schnitzler-briefe.acdh.oeaw.ac.at/{\dateiname}.html (Stand \today)
\fi

\end{document}


