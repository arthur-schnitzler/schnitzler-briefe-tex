\input{../tex-inputs/latex-pdf-vorspann}
\begin{center}
            \textcolor{red}{ENTWURF. ENTZIFFERUNG NOCH NICHT KORREKTURGELESEN}
                      \end{center}
            
               \section[Arthur Schnitzler an Hugo von Hofmannsthal, 22. 10. 1911]{ Arthur Schnitzler an Hugo von Hofmannsthal, 22. 10. 1911}\nopagebreak\mylabel{v}\rehead{ }\begin{ledgroupsized}[t]{13cm}\normalsize\beginnumbering\briefempfaengerindex{Hofmannsthal, Hugo von@\textsc{Hofmannsthal, Hugo von}!zzzSchnitzler, Arthur@\emph{von Arthur Schnitzler}!1911-10-221@{22. 10. 1911}|(be} \toendnotes[C]{\smallbreak\pagebreak[2]} \Standort{FDH, Hs-30885,144.}
\physDesc{Briefkarte
\newline{}Handschrift: schwarze Tinte, deutsche Kurrent}\buchAbdrucke{\weitereDrucke{Hugo von Hofmannsthal, Arthur Schnitzler: \emph{Briefwechsel}. Hg. Therese Nickl und Heinrich Schnitzler. Frankfurt am Main: \emph{S. Fischer} 1964, S. 264.} }\pstart
           \raggedleft{}{\pb}22/X 911\pend
           \pstart
           \textcolor{gray}{\textbf{A. S.}}\pend
           \pstart
           mein lieber Hugo, ich danke für Ihr liebes Telegra{\geminationm} aus Neubeuern\oindex{Neubeuern@\textbf{Neubeuern}|pw}, das
               ich für alle Fälle ſchon nach Rodaun\oindex{Rodaun@\textbf{Rodaun}|pw} beantworte. Ich
               reiſe Ende der Woche ab, Prag\oindex{Prag@\textbf{Prag}|pw}, Dresden\oindex{Dresden@\textbf{Dresden}|pw} (Vorleſungen), – da{\geminationn}{ }Berlin\oindex{Berlin@\textbf{Berlin}|pw} – Hamburg\oindex{Hamburg@\textbf{Hamburg}|pw}
                  (Beatrice\pwindex{Schnitzler, Arthur 15.05.1862 – 21.10.1931@\textsc{Schnitzler, Arthur} (15.05.1862 – 21.10.1931), \emph{Schriftsteller, Mediziner}!Schleier der Beatrice. Schauspiel in fuenf Akten1900-12-01 – 1900-12-01@\strich\emph{Der Schleier der Beatrice. Schauspiel in fünf Akten} {[}1900-12-01 – 1900-12-01{]}|pw}, Weites
                  Land\pwindex{Schnitzler, Arthur 15.05.1862 – 21.10.1931@\textsc{Schnitzler, Arthur} (15.05.1862 – 21.10.1931), \emph{Schriftsteller, Mediziner}!weite Land. Tragikomoedie in fuenf Akten1910-10-20@\strich\emph{Das weite Land. Tragikomödie in fünf Akten} {[}1910-10-20{]}|pw}, Anatol\pwindex{Schnitzler, Arthur 15.05.1862 – 21.10.1931@\textsc{Schnitzler, Arthur} (15.05.1862 – 21.10.1931), \emph{Schriftsteller, Mediziner}!Anatol1892-10-29 – 1892-10-29@\strich\emph{Anatol} {[}1892-10-29 – 1892-10-29{]}|pw}) bin gegen {\pb}Mitte November zurück. Vorher werden wir einander wohl kaum ſehen. Für
                  Herbſt und Winter aber hoff ich ein häufigeres Zuſa{\geminationm}enſein als es mir die letzten Jahre beſchieden war. Was
               iſt’s mit »Jedermann\pwindex{Hofmannsthal, Hugo von 01.02.1874 – 15.07.1929@\textsc{Hofmannsthal, Hugo von} (01.02.1874 – 15.07.1929), \emph{Schriftsteller}!Jedermann. Das Spiel vom Sterben des reichen Mannes1911@\strich\emph{Jedermann. Das Spiel vom Sterben des reichen Mannes} {[}1911{]}|pw}« und Allerlei? \pend
           \pstart
           Wir grüßen Euch herzlichſt!{\\[\baselineskip]}Ihr{\\[\baselineskip]}\spacefill\mbox{Arthur.}\pend
           \leftskip=0em{}\endnumbering\briefempfaengerindex{Hofmannsthal, Hugo von@\textsc{Hofmannsthal, Hugo von}!zzzSchnitzler, Arthur@\emph{von Arthur Schnitzler}!1911-10-221@{22. 10. 1911}|)be}\mylabel{h}\end{ledgroupsized}  \newcommand{\dateiname}{L02041}\newcommand{\titel}{Arthur Schnitzler an Hugo von Hofmannsthal, 22. 10. 1911}\newcommand{\editorInnen}{Martin Anton Müller und Gerd-Hermann Susen}\input{../tex-inputs/latex-pdf-abspann}
      