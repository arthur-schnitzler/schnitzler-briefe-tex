%% latex-korrekturansicht-vorspann.tex
%% Vorspann für die Korrekturansicht.
%% Lädt die gemeinsame Datei latex-vorspann.tex mit gesetztem Schalter.

\newif\ifkorrekturansicht
\korrekturansichttrue

\input{../tex-inputs/latex-vorspann}


\section[Arthur Schnitzler an Hugo von Hofmannsthal, 22. 10. 1911]{L02041 Arthur Schnitzler an Hugo von Hofmannsthal, 22. 10. 1911}
\nopagebreak\mylabel{L02041v}
\rehead{ }\normalsize\beginnumbering\briefempfaengerindex{Hofmannsthal, Hugo von@\textsc{Hofmannsthal, Hugo von}!zzzSchnitzler, Arthur@\emph{von Arthur Schnitzler}!1911-10-221@{22. 10. 1911}|(be}
\toendnotes[C]{\smallbreak\pagebreak[2]}\Standort{FDH, Hs-30885,144.}
\physDesc{Briefkarte, 483 Zeichen
\newline{}Handschrift: schwarze Tinte, deutsche Kurrent}
\buchAbdrucke{\weitereDrucke{Hugo von Hofmannsthal, Arthur Schnitzler: \emph{Briefwechsel}. Frankfurt am Main: \emph{S. Fischer} 1964, S. 264.} }\toendnotes[C]{\smallbreak}
\pstart
           \raggedleft{}{\pb}22/X 911\pend
           
\pstart
           \textcolor{gray}{\textbf{A. S.}}\pend
           \vspace{0.5em}
\pstart
           mein lieber Hugo, ich danke für Ihr liebes Telegra{\geminationm} aus Neubeuern\oindex{Neubeuern@\textbf{Neubeuern}, \emph{P.PPL}|pw}, das ich für alle Fälle ſchon nach Rodaun\oindex{Rodaun@\textbf{Rodaun}, \emph{A.ADM4}|pw} beantworte. Ich \label{K_L02041-1v}\edtext{reiſe}{\lemma{\textnormal{\emph{reiſe}}}\Cendnote{\textnormal{Schnitzler reiste am 29. 10. 1911 über
                            Prag\oindex{Prag@\textbf{Prag}, \emph{A.ADM1}|pwk} nach Berlin\oindex{Berlin@\textbf{Berlin}, \emph{P.PPLC}|pwk}, Hamburg\oindex{Hamburg@\textbf{Hamburg}, \emph{P.PPLA}|pwk},
                            München\oindex{XXXX Ortsangabe fehlt|pwk} und Garmisch-Partenkirchen\oindex{XXXX Ortsangabe fehlt|pwk}. Am 17. 11. 1911 war er
                        wieder in Wien\oindex{Wien@\textbf{Wien}, \emph{A.ADM2}|pwk}.}}}\label{K_L02041-1} Ende der Woche ab,
                        Prag\oindex{Prag@\textbf{Prag}, \emph{A.ADM1}|pw}, Dresden\oindex{Dresden@\textbf{Dresden}, \emph{P.PPLA}|pw} (Vorleſungen), – da{\geminationn}{ }Berlin\oindex{Berlin@\textbf{Berlin}, \emph{P.PPLC}|pw} – Hamburg\oindex{Hamburg@\textbf{Hamburg}, \emph{P.PPLA}|pw} (Beatrice\pwindex{Schleier der Beatrice. Schauspiel in fuenf Akten@\emph{Der Schleier der Beatrice. Schauspiel in fünf Akten}|pw}, Weites Land\pwindex{weite Land. Tragikomoedie in fuenf Akten@\emph{Das weite Land. Tragikomödie in fünf Akten}|pw}, Anatol\pwindex{Anatol@\emph{Anatol}|pw}) bin gegen {\pb}Mitte November zurück. Vorher werden wir einander wohl kaum ſehen.
                    Für Herbſt und Winter aber hoff ich ein häufigeres
                        Zuſa{\geminationm}enſein als es mir die letzten Jahre
                    beſchieden war. Was iſt’s mit »Jedermann\pwindex{Jedermann. Das Spiel vom Sterben des reichen Mannes@\emph{Jedermann. Das Spiel vom Sterben des reichen Mannes}|pw}«
                    und Allerlei? \pend
           
\pstart
           Wir grüßen Euch herzlichſt!{\\[\baselineskip]}Ihr{\\[\baselineskip]}\spacefill\mbox{Arthur.}\pend
           \leftskip=0em{}\selectlanguage{ngerman}\endnumbering\briefempfaengerindex{Hofmannsthal, Hugo von@\textsc{Hofmannsthal, Hugo von}!zzzSchnitzler, Arthur@\emph{von Arthur Schnitzler}!1911-10-221@{22. 10. 1911}|)be}\mylabel{L02041h}  \normalsize

\doendnotes{C}
\bigskip
\vfill

\clearpage

\footnotesize

\lohead{\textsc{register}}

% Definiere theindex-Environment komplett neu ohne reledmac
\makeatletter
\renewenvironment{theindex}{%
  \section*{\indexname}%
  \setlength{\parindent}{0pt}%
  \setlength{\parskip}{0pt plus 0.3pt}%
  \let\item\@idxitem
}{%
  \clearpage
}
\makeatother

\IfFileExists{\jobname-pw.ind}{\input{\jobname-pw.ind}}{}

\end{document}

      