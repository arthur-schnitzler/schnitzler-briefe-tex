%% latex-leseansicht-vorspann.tex
%% Vorspann für die Leseansicht.
%% Lädt die gemeinsame Datei latex-vorspann.tex mit nicht gesetztem Schalter.

\newif\ifkorrekturansicht
\korrekturansichtfalse

\input{../tex-inputs/latex-vorspann}


\section[Stefan Zweig an Arthur Schnitzler, 25. 8. 1917]{L03662 Stefan Zweig an Arthur Schnitzler, 25. 8. 1917}
\nopagebreak\mylabel{L03662v}
\rehead{ }\normalsize\beginnumbering\briefempfaengerindex{Schnitzler, Arthur@\textsc{Schnitzler, Arthur}!zzzZweig, Stefan@\emph{von Stefan Zweig}!1917-08-251@{25. 8. 1917}|(be}
\toendnotes[C]{\smallbreak\pagebreak[2]}
\correspDesc{Versand  durch Stefan Zweig am 25. 8. 1917 in Wien
\newline{}Erhalt  durch Arthur Schnitzler im Zeitraum [26. 8. 1917 – 30. 8. 1917?] in Wien}\toendnotes[C]{\smallbreak}
\Standort{CUL, Schnitzler, B 118.}
\physDesc{Brief, 1 Blatt, 4 Seiten, 2610 Zeichen
\newline{}Handschrift: blaue Tinte, lateinische Kurrent
\newline{}Schnitzler: 1) mit Bleistift »\textsc{Zweig}«  2) mit rotem Buntstift zwei Unterstreichungen}
\buchAbdrucke{\weitereDrucke{Stefan Zweig: \emph{Briefwechsel mit Hermann Bahr, Sigmund Freud, Rainer Maria
                        Rilke und Arthur Schnitzler}. Herausgegeben von Jeffrey B. Berlin, Hans-Ulrich Lindken und Donald A. Prater. Frankfurt am Main: \emph{S. Fischer} 1987, S. 406–408.} }\toendnotes[C]{\smallbreak}
\pstart
           {\pb}Kalksburg\oindex{Wien@\textbf{Wien}!XXIII., Liesing@\textbf{XXIII., Liesing}!Haselbrunnerstraße 12@\textbf{Haselbrunnerstraße 12}, \emph{Wohngebäude}|pw}, 25. August 1917\pend
           \vspace{0.5em}
\pstart
           Verehrter lieber Herr Doktor, vielen Dank für Ihren guten \label{K_L03662-1v}\edtext{Brief}{\lemma{\textnormal{\emph{Brief}}}\Cendnote{\textnormal{XXXX Auszeichnungsfehler: Dokument L03786 nicht gefunden.}}}\label{K_L03662-1}. Es liess mir doch keine Ruhe, ehe ich nicht Ihren Doktor Gräsler\pwindex{Schnitzler, Arthur 15. 5. 1862 Wien – 21. 10. 1931 ebd.@\textsc{Schnitzler, Arthur} (15. 5. 1862 Wien – 21. 10. 1931 ebd.), \emph{Schriftsteller, Mediziner}!Doktor Gräsler, Badearzt@\strich\emph{Doktor Gräsler, Badearzt}|pw} noch einmal gelesen hatte und ich zögere
               nicht, zu bekennen, dass d\substVorne{}\textsuperscript{ie}\substDazwischen{}as\substHinten{} Unverst\substVorne{}\textsuperscript{ä}\substDazwischen{}ehen\substHinten{} des Schlusses mein Fehler war. Ich fühle \strikeout{es}
               jetzt besser: dass diese scheinbare Episodenreihe (als die ich das Buch\pwindex{Schnitzler, Arthur 15. 5. 1862 Wien – 21. 10. 1931 ebd.@\textsc{Schnitzler, Arthur} (15. 5. 1862 Wien – 21. 10. 1931 ebd.), \emph{Schriftsteller, Mediziner}!Doktor Gräsler, Badearzt@\strich\emph{Doktor Gräsler, Badearzt}|pwv} zuerst
               empfand) der concentrierte Lebenszustand dieses Menschen ist, das einzige Dasein im
               höheren Sinn der Spannung zwischen einer gleichgiltigen Vergangenheit und Zukunft.
               Und ich empfinde jetzt erst, nachdem ich nicht mehr so hitzig zu Ende las, die ganze
                  {\pb}künstlerische Plastik seiner
               Pychologie. Freilich, wie ist sie verborgen, wie wenige werden spüren, dass hier der
               Anker bis in die Untiefe, ins letzte Geheimnis der Lebensangst gelegt ist und werden
               meinen – wie ja ich selbst zuerst – das Schiff steure ein wenig willkürlich auf
               seiner Strömung! Es ist mit solcher Noblesse das Ungesagte gesagt und das Gesagte
               wieder um seine Härte gebracht, dass ich mich im Sinne der höhern Gerechtigkeit (von
               der wir doch einzig den fruchtbaren Dank haben) unendlich freue, noch einmal dies Buch\pwindex{Schnitzler, Arthur 15. 5. 1862 Wien – 21. 10. 1931 ebd.@\textsc{Schnitzler, Arthur} (15. 5. 1862 Wien – 21. 10. 1931 ebd.), \emph{Schriftsteller, Mediziner}!Doktor Gräsler, Badearzt@\strich\emph{Doktor Gräsler, Badearzt}|pwv} begonnen und beendet zu
               haben. Dass Sie, wie schon in der letzten Novelle\pwindex{Schnitzler, Arthur 15. 5. 1862 Wien – 21. 10. 1931 ebd.@\textsc{Schnitzler, Arthur} (15. 5. 1862 Wien – 21. 10. 1931 ebd.), \emph{Schriftsteller, Mediziner}!Frau Beate und ihr Sohn. Novelle@\strich\emph{Frau Beate und ihr Sohn. Novelle}|pwv}, dem einsti{\pb}gen allzu wienerischen\oindex{Wien@\textbf{Wien}, \emph{Verwaltungsgebiet}|pw} Milieu ausgewichen sind, das gesellschaftliche Problem als
               geringes gegenüber dem blutmässigen, tiefinnersten empfinden, spüre ich mit geradezu
                  \uline{persönlicher} Dankbarkeit. Ich weiss nicht, ob ich
               es trotz meiner grossen und nie gebeugten menschlichen Verehrung vermocht hätte,
               Ihrem Werke auf die Dauer treu zu bleiben, wenn es im
               bourg{[}e{]}oisen, oft typisch wienerischen\oindex{Wien@\textbf{Wien}, \emph{Verwaltungsgebiet}|pw} Problem sich begrenzt hätte. Ich habe an Ihrem Werk immer die
               Bücher am meisten geliebt, wo der Inhalt ein allmenschlicher, allgiltiger war (Frau Berta Garlan\pwindex{Schnitzler, Arthur 15. 5. 1862 Wien – 21. 10. 1931 ebd.@\textsc{Schnitzler, Arthur} (15. 5. 1862 Wien – 21. 10. 1931 ebd.), \emph{Schriftsteller, Mediziner}!Frau Bertha Garlan. Roman@\strich\emph{Frau Bertha Garlan. Roman}|pw}, Reigen\pwindex{Schnitzler, Arthur 15. 5. 1862 Wien – 21. 10. 1931 ebd.@\textsc{Schnitzler, Arthur} (15. 5. 1862 Wien – 21. 10. 1931 ebd.), \emph{Schriftsteller, Mediziner}!Reigen. Zehn Dialoge@\strich\emph{Reigen. Zehn Dialoge}|pw}, Ruf des Lebens\pwindex{Schnitzler, Arthur 15. 5. 1862 Wien – 21. 10. 1931 ebd.@\textsc{Schnitzler, Arthur} (15. 5. 1862 Wien – 21. 10. 1931 ebd.), \emph{Schriftsteller, Mediziner}!Ruf des Lebens. Schauspiel in drei Akten@\strich\emph{Der Ruf des Lebens. Schauspiel in drei Akten}|pw},
               die letzten Novellen\pwindex{Schnitzler, Arthur 15. 5. 1862 Wien – 21. 10. 1931 ebd.@\textsc{Schnitzler, Arthur} (15. 5. 1862 Wien – 21. 10. 1931 ebd.), \emph{Schriftsteller, Mediziner}!Frau Beate und ihr Sohn. Novelle@\strich\emph{Frau Beate und ihr Sohn. Novelle}|pwu}\pwindex{Schnitzler, Arthur 15. 5. 1862 Wien – 21. 10. 1931 ebd.@\textsc{Schnitzler, Arthur} (15. 5. 1862 Wien – 21. 10. 1931 ebd.), \emph{Schriftsteller, Mediziner}!Masken und Wunder. Novellen@\strich\emph{Masken und Wunder. Novellen}|pwu}) und
               bin so glücklich, dass diese Verinnerlichung {\pb}in den letzten Jahren so fortschreitet,
               in denselben Jahren, in denen sie bei den meisten abzunehmen pflegt. Dass Ihnen heute
               künstlerisch nur mehr wichtig ist, was \strikeout{\textcolor{gray}{all}} menschlich wichtig ist: das Blutproblem, die
               Gefühle des na{[}c{]}kten, nicht bloss des socialen Menschen \introOben{}scheint mir Verheissung und Erfüllung\introOben{}. Die
                  Bourg{[}e{]}oisie ist mit allen ihren Problemen \uline{im letzten}, glaube ich, unfruchtbar, es sei denn, dass
               man sie mit dem Hass und der Verachtung anfasst, die sie verdient. Darüber bin ich
               persönlich in diesem Kriege ganz klar geworden und gerade an den Besten wie Werfel\pwindex{Werfel, Franz 10.\,9.\,1890 Prag – 26.\,8.\,1945 Beverly Hills@\textsc{Werfel, Franz} (10.\,9.\,1890 Prag – 26.\,8.\,1945 Beverly Hills), \emph{Schriftsteller}|pw}, spüre ich diese Wahrheit bestätigt.\pend
           
\pstart
           Mein Buch\pwindex{Zweig, Stefan 28.\,11.\,1881 Wien – 23.\,2.\,1942 Petrópolis@\textsc{Zweig, Stefan} (28.\,11.\,1881 Wien – 23.\,2.\,1942 Petrópolis), \emph{Schriftsteller}!Jeremias. Eine dramatische Dichtung in neun Bildern@\strich\emph{Jeremias. Eine dramatische Dichtung in neun Bildern}|pwv} zögert und zögert noch: ich
               habe nur Bühnenexemplare. Aber \label{K_L03662-2v}\edtext{in paar Tagen}{\lemma{\textnormal{\emph{in paar Tagen}}}\Cendnote{\textnormal{Zum
               3. 10. 1917 vermerkte Schnitzler den
                  Abschluss der Lektüre von \emph{Jeremias}\pwindex{Zweig, Stefan 28.\,11.\,1881 Wien – 23.\,2.\,1942 Petrópolis@\textsc{Zweig, Stefan} (28.\,11.\,1881 Wien – 23.\,2.\,1942 Petrópolis), \emph{Schriftsteller}!Jeremias. Eine dramatische Dichtung in neun Bildern@\strich\emph{Jeremias. Eine dramatische Dichtung in neun Bildern}|pwk}.}}}\label{K_L03662-2} ist es doch in Ihren Händen!\pend
           
\pstart
           Mit vielen Grüssen Ihr getreuer{\\[\baselineskip]}\spacefill\mbox{Stefan Zweig}\pend
           \leftskip=0em{}\selectlanguage{ngerman}\endnumbering\briefempfaengerindex{Schnitzler, Arthur@\textsc{Schnitzler, Arthur}!zzzZweig, Stefan@\emph{von Stefan Zweig}!1917-08-251@{25. 8. 1917}|)be}\mylabel{L03662h}  \newcommand{\dateiname}{L03662}\newcommand{\titel}{Stefan Zweig an Arthur Schnitzler, 25. 8. 1917}\newcommand{\editorInnen}{Selma Jahnke und Martin Anton Müller}%% latex-leseansicht-abspann.tex
%% Abspann für die Leseansicht.
%% Der Schalter \ifkorrekturansicht ist bereits durch den Vorspann gesetzt.

%% latex-abspann.tex
%% Gemeinsamer Abspann für Korrekturansicht und Leseansicht.
%% Setzt den Schalter \ifkorrekturansicht voraus (gesetzt in den
%% einbindenden Dateien latex-korrekturansicht-abspann.tex bzw.
%% latex-leseansicht-abspann.tex).
%% ---------------------------------------------------------------

\normalsize

% Das esempio-Environment wird nur in der Leseansicht benötigt
\ifkorrekturansicht\else
\newenvironment{esempio}[3]%
{
    \vspace{1.5ex}
    \rlap{\underline{#1}}
    \par
    \setlength{\parindent}{0cm}
    \nopagebreak
    \leftskip=#2cm
    \rightskip=#3cm
}
{
    \par
}
\fi

\doendnotes{C}
\bigskip
\vfill

\clearpage

\footnotesize

\ifkorrekturansicht
  \lohead{\textsc{register}}
\fi

% theindex-Environment neu definieren ohne reledmac
\makeatletter
\renewenvironment{theindex}{%
  \ifkorrekturansicht
    \section*{\indexname}%
  \else
    \subsubsection*{Index der erwähnten Entitäten}%
  \fi
  \setlength{\parindent}{0pt}%
  \setlength{\parskip}{0pt plus 0.3pt}%
  \let\item\@idxitem
}{%
  \ifkorrekturansicht\clearpage\fi
}
\makeatother

\IfFileExists{\jobname-pw.ind}{\input{\jobname-pw.ind}}{}

% Quellenangabe nur in der Leseansicht
\ifkorrekturansicht\else
% Fallback-Definitionen, falls die .tex-Datei \titel etc. nicht gesetzt hat
\providecommand{\titel}{}
\providecommand{\editorInnen}{}
\providecommand{\dateiname}{\jobname}

\vspace{3cm}

\vfill

\footnotesize
\textsc{Quelle}: \titel. Herausgegeben von {\editorInnen}. In: \emph{Arthur Schnitzler: Briefwechsel mit Autorinnen und Autoren}.
 Digitale Edition, https://schnitzler-briefe.acdh.oeaw.ac.at/{\dateiname}.html (Stand \today)
\fi

\end{document}


