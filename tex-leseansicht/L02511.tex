\input{../tex-inputs/latex-pdf-vorspann}
\begin{center}
            \textcolor{red}{ENTWURF. ENTZIFFERUNG NOCH NICHT KORREKTURGELESEN}
                      \end{center}
            
               \section[Arthur Schnitzler: Widmungsexemplar Spiel im Morgengrauen für Hugo von Hofmannsthal, 11. 6. 1929]{ Arthur Schnitzler: Widmungsexemplar Spiel im Morgengrauen für Hugo von
               Hofmannsthal, 11. 6. 1929}\nopagebreak\mylabel{v}\rehead{ }\begin{ledgroupsized}[t]{13cm}\normalsize\beginnumbering\briefempfaengerindex{Hofmannsthal, Hugo von@\textsc{Hofmannsthal, Hugo von}!zzzSchnitzler, Arthur@\emph{von Arthur Schnitzler}!1929-06-111@{11. 6. 1929}|(be} \toendnotes[C]{\smallbreak\pagebreak[2]} \Standort{FDH, FDH 1938.}
\physDesc{Widmung am Vorsatzblatt
\newline{}Handschrift: schwarze Tinte, lateinische Kurrent}\buchAbdrucke{\weitereDrucke{Hugo von Hofmannsthal: \emph{Bibliothek}. Hg. Ellen Ritter † in Zusammenarbeit mit Dalia Bukauskaité und
                        Konrad Heumann. Frankfurt am Main: \emph{S. Fischer} 2011, S. 606 (Sämtliche Werke. Kritische Ausgabe, XL).} }\toendnotes[C]{\smallbreak}\pstart
           \noindent{}{\pb}Meinem lieben Hugo\pend
           \pstart
           herzlich{\\[\baselineskip]}\introOben{}wie immer\introOben{}{\\[\baselineskip]}\spacefill\mbox{Arthur}\pend
           \leftskip=0em{}\pstart
           Wien\oindex{Wien@\textbf{Wien}|pw}{ }\label{K_L02511_1v}\edtext{11. 6. 929}{\lemma{\textnormal{\emph{11. 6. 929}}}\Cendnote{\textnormal{am 9. 3. 1927 vom \emph{Börsenblatt für den deutschen Buchhandel}\pwindex{Boersenblatt fuer den deutschen Buchhandel1843-01-03@\emph{Börsenblatt für den deutschen Buchhandel}|pwk}
                        als Neuerscheinung gemeldet}}}\label{K_L02511_1h}.\pend
           {\bigskip}\pstart
           \noindent{}\centering{}{\pb}\textcolor{gray}{\textbf{SPIEL IM MORGENGRAUEN\pwindex{Schnitzler, Arthur 15.05.1862 – 21.10.1931@\textsc{Schnitzler, Arthur} (15.05.1862 – 21.10.1931), \emph{Schriftsteller, Mediziner}!Spiel im Morgengrauen. Novelle5.12.1926 – 9.1.1927@\strich\emph{Spiel im Morgengrauen. Novelle} {[}5.12.1926 – 9.1.1927{]}|pw}}}\pend
           \pstart
           \noindent{}\centering{}\textcolor{gray}{\textbf{NOVELLE}}{\\}\textcolor{gray}{\textbf{VON}}{\\}\textcolor{gray}{\textbf{ARTHUR SCHNITZLER}}\pend
           {\bigskip}\pstart
           \noindent{}\centering{}\textcolor{gray}{\textbf{S. FISCHER / VERLAG\orgindex{S. Fischer Verlag@S. Fischer Verlag|pw} / BERLIN\oindex{Berlin@\textbf{Berlin}|pw}}}\pend
           \endnumbering\briefempfaengerindex{Hofmannsthal, Hugo von@\textsc{Hofmannsthal, Hugo von}!zzzSchnitzler, Arthur@\emph{von Arthur Schnitzler}!1929-06-111@{11. 6. 1929}|)be}\mylabel{h}\end{ledgroupsized}  \newcommand{\dateiname}{L02511}\newcommand{\titel}{Arthur Schnitzler: Widmungsexemplar Spiel im Morgengrauen für Hugo von Hofmannsthal, 11. 6. 1929}\newcommand{\editorInnen}{Martin Anton Müller und Gerd-Hermann Susen}\input{../tex-inputs/latex-pdf-abspann}
      