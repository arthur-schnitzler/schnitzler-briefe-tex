%% latex-leseansicht-vorspann.tex
%% Vorspann für die Leseansicht.
%% Lädt die gemeinsame Datei latex-vorspann.tex mit nicht gesetztem Schalter.

\newif\ifkorrekturansicht
\korrekturansichtfalse

\input{../tex-inputs/latex-vorspann}


         
         \renewcommand{\erwaehntePersonen}{Personen: Hugo von Hofmannsthal}
         \renewcommand{\erwaehnteInstitutionen}{Institutionen: S. Fischer Verlag}
         \renewcommand{\erwaehnteOrte}{Orte: Berlin, Wien}
         \renewcommand{\erwaehnteWerke}{Werke: Börsenblatt für den Deutschen Buchhandel, Spiel im Morgengrauen. Novelle}
               \section[Arthur Schnitzler: Widmungsexemplar Spiel im Morgengrauen für Hugo Hofmannsthal, 11. 6. 1929]{ Arthur Schnitzler: Widmungsexemplar Spiel im Morgengrauen für Hugo 
               Hofmannsthal, 11. 6. 1929}\nopagebreak\mylabel{v}\rehead{ }\begin{ledgroupsized}[t]{13cm}\normalsize\beginnumbering\briefempfaengerindex{Hofmannsthal, Hugo von@\textsc{Hofmannsthal, Hugo von}!zzzSchnitzler, Arthur@\emph{von Arthur Schnitzler}!1929-06-111@{11. 6. 1929}|(be} \toendnotes[C]{\smallbreak\pagebreak[2]} \Standort{FDH, FDH 1938.}
\physDesc{Widmung am Vorsatzblatt, 59 Zeichen
\newline{}Handschrift: schwarze Tinte, lateinische Kurrent}\buchAbdrucke{\weitereDrucke{Hugo von Hofmannsthal: \emph{Bibliothek}. Hg. Ellen Ritter † in Zusammenarbeit mit Dalia Bukauskaité und Konrad
                        Heumann. Frankfurt am Main: \emph{S. Fischer} 2011, S. 606 (Sämtliche Werke. Kritische Ausgabe, XL).} }\toendnotes[C]{\smallbreak}\pstart
           \noindent{}{\pb}Meinem lieben Hugo\pend
           \pstart
           herzlich{\\[\baselineskip]}\introOben{}wie immer\introOben{}{\\[\baselineskip]}\spacefill\mbox{Arthur}\pend
           \leftskip=0em{}\pstart
           Wien\oindex{Wien@\textbf{Wien}|pw}{ }\label{K_L02511-1v}\edtext{11. 6. 929}{\lemma{\textnormal{\emph{11. 6. 929}}}\Cendnote{\textnormal{am 9. 3. 1927 vom \emph{Börsenblatt für den deutschen Buchhandel}\pwindex{?? Werk@Nicht ermittelte Verfasserinnen und Verfasser!Boersenblatt fuer den Deutschen Buchhandel1843-01-03@\emph{Börsenblatt für den Deutschen Buchhandel} {[}1843-01-03{]}|pwk}
                        als Neuerscheinung gemeldet}}}\label{K_L02511-1h}.\pend
           {\bigskip}\pstart
           \noindent{}\centering{}{\pb}\textcolor{gray}{\textbf{SPIEL IM MORGENGRAUEN\pwindex{Schnitzler, Arthur 15.05.1862 – 21.10.1931@\textsc{Schnitzler, Arthur} (15.05.1862 – 21.10.1931), \emph{Schriftsteller, Mediziner}!Spiel im Morgengrauen. Novelle5.12.1926 – 9.1.1927@\strich\emph{Spiel im Morgengrauen. Novelle} {[}5.12.1926 – 9.1.1927{]}|pw}}}\pend
           \pstart
           \noindent{}\centering{}\textcolor{gray}{\textbf{NOVELLE}}{\\}\textcolor{gray}{\textbf{VON}}{\\}\textcolor{gray}{\textbf{ARTHUR SCHNITZLER}}\pend
           {\bigskip}\pstart
           \noindent{}\centering{}\textcolor{gray}{\textbf{S. FISCHER / VERLAG\orgindex{S. Fischer Verlag@S. Fischer Verlag|pw} / BERLIN\oindex{Berlin@\textbf{Berlin}|pw}}}\pend
           
         
         \endnumbering\mylabel{h}\end{ledgroupsized}  \newcommand{\dateiname}{L02511}\newcommand{\titel}{Arthur Schnitzler: Widmungsexemplar Spiel im Morgengrauen für Hugo Hofmannsthal, 11. 6. 1929}\newcommand{\editorInnen}{Martin Anton Müller und Gerd-Hermann Susen}%% latex-leseansicht-abspann.tex
%% Abspann für die Leseansicht.
%% Der Schalter \ifkorrekturansicht ist bereits durch den Vorspann gesetzt.

%% latex-abspann.tex
%% Gemeinsamer Abspann für Korrekturansicht und Leseansicht.
%% Setzt den Schalter \ifkorrekturansicht voraus (gesetzt in den
%% einbindenden Dateien latex-korrekturansicht-abspann.tex bzw.
%% latex-leseansicht-abspann.tex).
%% ---------------------------------------------------------------

\normalsize

% Das esempio-Environment wird nur in der Leseansicht benötigt
\ifkorrekturansicht\else
\newenvironment{esempio}[3]%
{
    \vspace{1.5ex}
    \rlap{\underline{#1}}
    \par
    \setlength{\parindent}{0cm}
    \nopagebreak
    \leftskip=#2cm
    \rightskip=#3cm
}
{
    \par
}
\fi

\doendnotes{C}
\bigskip
\vfill

\clearpage

\footnotesize

\ifkorrekturansicht
  \lohead{\textsc{register}}
\fi

% theindex-Environment neu definieren ohne reledmac
\makeatletter
\renewenvironment{theindex}{%
  \ifkorrekturansicht
    \section*{\indexname}%
  \else
    \subsubsection*{Index der erwähnten Entitäten}%
  \fi
  \setlength{\parindent}{0pt}%
  \setlength{\parskip}{0pt plus 0.3pt}%
  \let\item\@idxitem
}{%
  \ifkorrekturansicht\clearpage\fi
}
\makeatother

\IfFileExists{\jobname-pw.ind}{\input{\jobname-pw.ind}}{}

% Quellenangabe nur in der Leseansicht
\ifkorrekturansicht\else
% Fallback-Definitionen, falls die .tex-Datei \titel etc. nicht gesetzt hat
\providecommand{\titel}{}
\providecommand{\editorInnen}{}
\providecommand{\dateiname}{\jobname}

\vspace{3cm}

\vfill

\footnotesize
\textsc{Quelle}: \titel. Herausgegeben von {\editorInnen}. In: \emph{Arthur Schnitzler: Briefwechsel mit Autorinnen und Autoren}.
 Digitale Edition, https://schnitzler-briefe.acdh.oeaw.ac.at/{\dateiname}.html (Stand \today)
\fi

\end{document}


      