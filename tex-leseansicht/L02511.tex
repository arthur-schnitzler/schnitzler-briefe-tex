%% latex-korrekturansicht-vorspann.tex
%% Vorspann für die Korrekturansicht.
%% Lädt die gemeinsame Datei latex-vorspann.tex mit gesetztem Schalter.

\newif\ifkorrekturansicht
\korrekturansichttrue

\input{../tex-inputs/latex-vorspann}


\section[Arthur Schnitzler: Widmungsexemplar Spiel im Morgengrauen für Hugo Hofmannsthal, 11. 6. 1929]{L02511 Arthur Schnitzler: Widmungsexemplar Spiel im Morgengrauen für Hugo 
               Hofmannsthal, 11. 6. 1929}
\nopagebreak\mylabel{L02511v}
\rehead{ }\normalsize\beginnumbering\briefempfaengerindex{Hofmannsthal, Hugo von@\textsc{Hofmannsthal, Hugo von}!zzzSchnitzler, Arthur@\emph{von Arthur Schnitzler}!1929-06-111@{11. 6. 1929}|(be}
\toendnotes[C]{\smallbreak\pagebreak[2]}\Standort{FDH, FDH 1938.}
\physDesc{Widmung am Vorsatzblatt, 59 Zeichen
\newline{}Handschrift: schwarze Tinte, lateinische Kurrent}
\buchAbdrucke{\weitereDrucke{Hugo von Hofmannsthal: \emph{Bibliothek}. Frankfurt am Main: \emph{S. Fischer} 2011, S. 606.} }\toendnotes[C]{\smallbreak}
\pstart
           \noindent{}{\pb}Meinem lieben Hugo\pend
           
\pstart
           herzlich{\\[\baselineskip]}\introOben{}wie immer\introOben{}{\\[\baselineskip]}\spacefill\mbox{Arthur}\pend
           \leftskip=0em{}
\pstart
           Wien\oindex{Wien@\textbf{Wien}, \emph{A.ADM2}|pw}{ }\label{K_L02511-1v}\edtext{11. 6. 929}{\lemma{\textnormal{\emph{11. 6. 929}}}\Cendnote{\textnormal{am 9. 3. 1927 vom \emph{Börsenblatt für den deutschen Buchhandel}\pwindex{Boersenblatt fuer den Deutschen Buchhandel@\emph{Börsenblatt für den Deutschen Buchhandel}|pwk}
                        als Neuerscheinung gemeldet}}}\label{K_L02511-1}.\pend
           \selectlanguage{ngerman}\vspace{1em}{\vspace{1\baselineskip}}
\pstart
           \centering{}{\pb}\textcolor{gray}{\textbf{SPIEL IM MORGENGRAUEN\pwindex{Spiel im Morgengrauen. Novelle@\emph{Spiel im Morgengrauen. Novelle}|pw}}}\pend
           
\pstart
           \centering{}\textcolor{gray}{\textbf{NOVELLE}}{\\}\textcolor{gray}{\textbf{VON}}{\\}\textcolor{gray}{\textbf{ARTHUR SCHNITZLER}}\pend
           {\vspace{1\baselineskip}}
\pstart
           \centering{}\textcolor{gray}{\textbf{S. FISCHER / VERLAG\orgindex{S. Fischer Verlag@S. Fischer Verlag|pw} / BERLIN\oindex{Berlin@\textbf{Berlin}, \emph{P.PPLC}|pw}}}\pend
           \selectlanguage{ngerman}\endnumbering\briefempfaengerindex{Hofmannsthal, Hugo von@\textsc{Hofmannsthal, Hugo von}!zzzSchnitzler, Arthur@\emph{von Arthur Schnitzler}!1929-06-111@{11. 6. 1929}|)be}\mylabel{L02511h}  \normalsize

\doendnotes{C}
\bigskip
\vfill

\clearpage

\footnotesize

\lohead{\textsc{register}}

% Definiere theindex-Environment komplett neu ohne reledmac
\makeatletter
\renewenvironment{theindex}{%
  \section*{\indexname}%
  \setlength{\parindent}{0pt}%
  \setlength{\parskip}{0pt plus 0.3pt}%
  \let\item\@idxitem
}{%
  \clearpage
}
\makeatother

\IfFileExists{\jobname-pw.ind}{\input{\jobname-pw.ind}}{}

\end{document}

      