%% latex-leseansicht-vorspann.tex
%% Vorspann für die Leseansicht.
%% Lädt die gemeinsame Datei latex-vorspann.tex mit nicht gesetztem Schalter.

\newif\ifkorrekturansicht
\korrekturansichtfalse

\input{../tex-inputs/latex-vorspann}


\section[Arthur Schnitzler an Gustav Schwarzkopf, 20. 4. 1899]{L04133 Arthur Schnitzler an Gustav Schwarzkopf, 20. 4. 1899}
\nopagebreak\mylabel{L04133v}
\rehead{ }\normalsize\beginnumbering\briefempfaengerindex{Schwarzkopf, Gustav@\textsc{Schwarzkopf, Gustav}!zzzSchnitzler, Arthur@\emph{von Arthur Schnitzler}!1899-04-201@{20. 4. 1899}|(be}
\toendnotes[C]{\smallbreak\pagebreak[2]}
\correspDesc{Versand  durch Arthur Schnitzler am 20. 4. 1899 in Wien
\newline{}Erhalt  durch Gustav Schwarzkopf im Zeitraum [20. 4. 1899
                  – 23. 4. 1899?] in Wien}\toendnotes[C]{\smallbreak}
\Standort{CUL, Schnitzler, B 96.}
\physDesc{Postkarte, 217 Zeichen
\newline{}Handschrift: schwarze Tinte, deutsche Kurrent
\newline{}Versand: Stempel: »\nobreak{}\oindex{I., Innere Stadt@\textbf{I., Innere Stadt}, \emph{Verwaltungsgebiet}|pwk}Wien 1/1, 20 IV 99, 2 30V\nobreak{}«.  }\toendnotes[C]{\smallbreak}\pstart{}{\pb}Herrn \textsc{Gustav
                     Schwarzkopf}\pend{}\pstart{}Wien\oindex{Wien@\textbf{Wien}, \emph{Verwaltungsgebiet}|pw}\pend{}\pstart{}\textsc{I. Tiefer Graben 23}\oindex{Wien@\textbf{Wien}!I., Innere Stadt@\textbf{I., Innere Stadt}!Tiefer Graben 23@\textbf{Tiefer Graben 23}, \emph{Wohngebäude}|pw}.\pend{}{\bigskip}\vspace{1em}
\pstart
           \noindent{}{\pb}Lieber Guſtav, erwarten Sie mich heute Nachmittg nicht. Dagegen
               hoffe ich Sie im \textsc{Kaiserhof\oindex{Wien@\textbf{Wien}!I., Innere Stadt@\textbf{I., Innere Stadt}!Café Kaiserhof (Inh. Johann Wortner) [Wien]@\textbf{Café Kaiserhof (Inh. Johann Wortner) [Wien]}, \emph{Kaffeehaus}|pw}} (\label{K_L04133-1v}\edtext{nach \textsc{Cyrano\pwindex{Rostand, Edmond 1.\,4.\,1868 Marseille – 2.\,12.\,1918 Paris@\textsc{Rostand, Edmond} (1.\,4.\,1868 Marseille – 2.\,12.\,1918 Paris), \emph{Schriftsteller}!Cyrano von Bergerac. Romantische Komödie in fünf Aufzügen@\strich\emph{Cyrano von Bergerac. Romantische Komödie in fünf Aufzügen}|pw}}\eventindex{Burgtheater@\textbf{Burgtheater}!Aufführung von Cyrano von Bergerac, 20.4.1899@Aufführung von Cyrano von Bergerac, 20.4.1899|pwv}}{\lemma{\textnormal{\emph{nach Cyrano}}}\Cendnote{\textnormal{Schnitzler besuchte die die Aufführung
                     von \emph{Cyrano von Bergerac}\pwindex{Rostand, Edmond 1.\,4.\,1868 Marseille – 2.\,12.\,1918 Paris@\textsc{Rostand, Edmond} (1.\,4.\,1868 Marseille – 2.\,12.\,1918 Paris), \emph{Schriftsteller}!Cyrano von Bergerac. Romantische Komödie in fünf Aufzügen@\strich\emph{Cyrano von Bergerac. Romantische Komödie in fünf Aufzügen}|pwk} am 20. 4. 1899\eventindex{Burgtheater@\textbf{Burgtheater}!Aufführung von Cyrano von Bergerac, 20.4.1899@Aufführung von Cyrano von Bergerac, 20.4.1899|pwk} im Burgtheater\oindex{Wien@\textbf{Wien}!I., Innere Stadt@\textbf{I., Innere Stadt}!Burgtheater@\textbf{Burgtheater}, \emph{Theater}|pwk} nicht, er scheint aber zu wissen, dass es Schwarzkopf\pwindex{Schwarzkopf, Gustav 7.\,11.\,1853 Wien – 13.\,11.\,1939 ebd.@\textsc{Schwarzkopf, Gustav} (7.\,11.\,1853 Wien – 13.\,11.\,1939 ebd.), \emph{Schriftsteller}|pwk} tat.}}}\label{K_L04133-1}) zu ſehn; wenn nicht, ſo hören Sie
               morgen von mir.\pend
           \pstart Herzlichſt Ihr \spacefill\mbox{Arthur}\pend{}\selectlanguage{ngerman}\endnumbering\briefempfaengerindex{Schwarzkopf, Gustav@\textsc{Schwarzkopf, Gustav}!zzzSchnitzler, Arthur@\emph{von Arthur Schnitzler}!1899-04-201@{20. 4. 1899}|)be}\mylabel{L04133h}
\begin{anhang}
\end{anhang}\newcommand{\dateiname}{L04133}\newcommand{\titel}{Arthur Schnitzler an Gustav Schwarzkopf, 20. 4. 1899}\newcommand{\editorInnen}{Herausgegeben von Jahnke, SelmaMüller, Martin Anton}%% latex-leseansicht-abspann.tex
%% Abspann für die Leseansicht.
%% Der Schalter \ifkorrekturansicht ist bereits durch den Vorspann gesetzt.

%% latex-abspann.tex
%% Gemeinsamer Abspann für Korrekturansicht und Leseansicht.
%% Setzt den Schalter \ifkorrekturansicht voraus (gesetzt in den
%% einbindenden Dateien latex-korrekturansicht-abspann.tex bzw.
%% latex-leseansicht-abspann.tex).
%% ---------------------------------------------------------------

\normalsize

% Das esempio-Environment wird nur in der Leseansicht benötigt
\ifkorrekturansicht\else
\newenvironment{esempio}[3]%
{
    \vspace{1.5ex}
    \rlap{\underline{#1}}
    \par
    \setlength{\parindent}{0cm}
    \nopagebreak
    \leftskip=#2cm
    \rightskip=#3cm
}
{
    \par
}
\fi

\doendnotes{C}
\bigskip
\vfill

\clearpage

\footnotesize

\ifkorrekturansicht
  \lohead{\textsc{register}}
\fi

% theindex-Environment neu definieren ohne reledmac
\makeatletter
\renewenvironment{theindex}{%
  \ifkorrekturansicht
    \section*{\indexname}%
  \else
    \subsubsection*{Index der erwähnten Entitäten}%
  \fi
  \setlength{\parindent}{0pt}%
  \setlength{\parskip}{0pt plus 0.3pt}%
  \let\item\@idxitem
}{%
  \ifkorrekturansicht\clearpage\fi
}
\makeatother

\IfFileExists{\jobname-pw.ind}{\input{\jobname-pw.ind}}{}

% Quellenangabe nur in der Leseansicht
\ifkorrekturansicht\else
% Fallback-Definitionen, falls die .tex-Datei \titel etc. nicht gesetzt hat
\providecommand{\titel}{}
\providecommand{\editorInnen}{}
\providecommand{\dateiname}{\jobname}

\vspace{3cm}

\vfill

\footnotesize
\textsc{Quelle}: \titel. Herausgegeben von {\editorInnen}. In: \emph{Arthur Schnitzler: Briefwechsel mit Autorinnen und Autoren}.
 Digitale Edition, https://schnitzler-briefe.acdh.oeaw.ac.at/{\dateiname}.html (Stand \today)
\fi

\end{document}


