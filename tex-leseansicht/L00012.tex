%% latex-korrekturansicht-vorspann.tex
%% Vorspann für die Korrekturansicht.
%% Lädt die gemeinsame Datei latex-vorspann.tex mit gesetztem Schalter.

\newif\ifkorrekturansicht
\korrekturansichttrue

\input{../tex-inputs/latex-vorspann}


\section[Hugo von Hofmannsthal an Arthur Schnitzler, {[}6. 5. 1891?{]}]{L00012 Hugo von Hofmannsthal an Arthur Schnitzler, {[}6. 5. 1891?{]}}
\nopagebreak\mylabel{L00012v}
\rehead{ }\normalsize\beginnumbering\briefempfaengerindex{Schnitzler, Arthur@\textsc{Schnitzler, Arthur}!zzzHofmannsthal, Hugo von@\emph{von Hugo von Hofmannsthal}!1891-05-061@{{[}6. 5. 1891?{]}}|(be}
\toendnotes[C]{\smallbreak\pagebreak[2]}\Standort{CUL, Schnitzler, B 43.}
\physDesc{Visitenkarte, 272 Zeichen
\newline{}Handschrift: Bleistift, deutsche Kurrent
\newline{}Ordnung: 1) mit Bleistift von Frieda
                                    Pollak\pwindex{Pollak, Frieda 08.12.1881 – 13.07.1937@\textsc{Pollak, Frieda} (08.12.1881 – 13.07.1937), \emph{Sekretär/Sekretärin}|pw} (?) mit dem Buchstaben »A«
                                 (Abgeschrieben/Abschrift) gekennzeichnet  2) mit Bleistift von unbekannter Hand datiert: »9\strikeout{1.}0«
\newline{}Zusatz: Visitenkarte des Vaters Hugo August von Hofmannsthal }
\buchAbdrucke{\weitereDrucke{Hugo von Hofmannsthal, Arthur Schnitzler: \emph{Briefwechsel}. Frankfurt am Main: \emph{S. Fischer} 1964, S. 7.} }\toendnotes[C]{\smallbreak}
\pstart
           \noindent{}\centering{}{\pb}\textcolor{gray}{\textbf{Hugo von Hofmannsthal\pwindex{Hofmannsthal, Hugo August von 21.12.1841 – 08.12.1915@\textsc{Hofmannsthal, Hugo August von} (21.12.1841 – 08.12.1915), \emph{Bankdirektor/Bankdirektorin}|pw}}}\pend
           
\pstart
           dankt beſchämt und warm für Alkandis Lied\pwindex{Alkandi s Lied@\emph{Alkandi’s Lied}|pw}, die
               5 Worte auf dem Titelblatt {\pb}und
               den hübſchen Gedanken, aus einer Höflichkeit der Form eine Höflichkeit des Herzens zu
               machen. Sehen wir uns, falls ich heute den \label{K_L00012-1v}\edtext{Naturaliſtennaturausflug}{\lemma{\textnormal{\emph{Naturaliſtennaturausflug}}}\Cendnote{\textnormal{Hierbei handelt es sich womöglich um die »Landpartie der
                  Naturalisten«, die Schnitzler am 6. 5. 1891 in einer Stoffnotiz
                  erwähnt, vgl. Hermann Bahr, Arthur Schnitzler: \emph{Briefwechsel, Aufzeichnungen, Dokumente (1891–1931)}, Arthur Schnitzler: Die Landpartie der Naturalisten, Stoffnotiz, 6. 5. 1891.}}}\label{K_L00012-1} mitmache? Müßige Frage, gleichviel\hspace*{1.5em}\textsc{à bientôt}\pend
           \selectlanguage{ngerman}\endnumbering\briefempfaengerindex{Schnitzler, Arthur@\textsc{Schnitzler, Arthur}!zzzHofmannsthal, Hugo von@\emph{von Hugo von Hofmannsthal}!1891-05-061@{{[}6. 5. 1891?{]}}|)be}\mylabel{L00012h}  \normalsize

\doendnotes{C}
\bigskip
\vfill

\clearpage

\footnotesize

\lohead{\textsc{register}}

% Definiere theindex-Environment komplett neu ohne reledmac
\makeatletter
\renewenvironment{theindex}{%
  \section*{\indexname}%
  \setlength{\parindent}{0pt}%
  \setlength{\parskip}{0pt plus 0.3pt}%
  \let\item\@idxitem
}{%
  \clearpage
}
\makeatother

\IfFileExists{\jobname-pw.ind}{\input{\jobname-pw.ind}}{}

\end{document}

      