%% latex-leseansicht-vorspann.tex
%% Vorspann für die Leseansicht.
%% Lädt die gemeinsame Datei latex-vorspann.tex mit nicht gesetztem Schalter.

\newif\ifkorrekturansicht
\korrekturansichtfalse

\input{../tex-inputs/latex-vorspann}


         
         \renewcommand{\erwaehntePersonen}{Personen:  ?? [Bräutigam von Hildegard Mitis], Hilda von Mitis, Maximilian von Mitis}
         \renewcommand{\erwaehnteOrte}{Orte: Bratislava, Italien, Paris, Wien}
         \renewcommand{\erwaehnteWerke}{}
               \section[Paul Goldmann an Arthur Schnitzler, 18. 12. {[}1894{]}]{ Paul Goldmann an Arthur Schnitzler, 18. 12. {[}1894{]}}\nopagebreak\mylabel{v}\rehead{ }\begin{ledgroupsized}[t]{13cm}\normalsize\beginnumbering \toendnotes[C]{\smallbreak\pagebreak[2]} \Standort{DLA, A:Schnitzler, HS.NZ85.1.3164.}
\physDesc{Brief, 2 Blätter, 8 Seiten
\newline{}Handschrift: schwarze Tinte, deutsche Kurrent
\newline{}Schnitzler: mit Bleistift die Jahreszahl »94« vermerkt }\toendnotes[C]{\smallbreak}\pstart
           \raggedleft{}{\pb}18. December.\pend
           \pstart\center{}Mein lieber Freund,\pend\pstart
           Ich glaube, ich empfinde mehr Reue als Schmerz. Das iſt ein furchtbares Gefühl. Das
                  \strikeout{ar\textcolor{gray}{n}}{ }\label{K_mets_Goldmann_94-partII-4v}\edtext{arme Mädel\pwindex{Mitis, Hilda von 1876-08-30 – 1894-12-14@\textsc{Mitis, Hilda von} (1876-08-30 – 1894-12-14), \emph{Schriftstellerin, Telefonistin}|pwv}}{\lemma{\textnormal{\emph{arme Mädel}}}\Cendnote{\textnormal{Hilda von Mitis\pwindex{Mitis, Hilda von 1876-08-30 – 1894-12-14@\textsc{Mitis, Hilda von} (1876-08-30 – 1894-12-14), \emph{Schriftstellerin, Telefonistin}|pwk}, die sich am
                  14. 12. 1894 in Bratislava\oindex{Bratislava@\textbf{Bratislava}|pwk} im Wald erschossen hatte.}}}\label{K_mets_Goldmann_94-partII-4h} iſt ſymboliſch für meine verſäumte
               Jugend. Ein Anderer hätte im ſtolzen Kraftbewußtſein ſich mit dieſer ſchönen Blume\pwindex{Mitis, Hilda von 1876-08-30 – 1894-12-14@\textsc{Mitis, Hilda von} (1876-08-30 – 1894-12-14), \emph{Schriftstellerin, Telefonistin}|pwv} geſchmückt und ihren
               Duft genoſſen. Ich habe ſchwächlich genörgelt und gezweifelt. Liebt ſie mich? Lügt
               ſie nicht? Das war nicht das Grübeln der Denker-Natur, ſondern, wie geſagt, Schwäche,
               mangelnde Beſitzergreifungs-Kraft. Es war in ihr zu Anfang gewiß eine kleine Flamme.
               Aber ſie iſt {\pb}raſch verlöſcht, weil ich mich in
               meine Schale zurückzog und nicht glauben wollte. Es hätten herrliche Tage werden
               können und Sonnenſchein für ein ganzes Leben. Statt deſſen wurde es nur, wie Alles in
               meinem Leben, ein verſäumtes Glück, ein nicht zu Ende gelebtes Erlebniß. Seit Jahren
               plagt mich die Reue darüber. Und es iſt ſo eigenthümlich für meinen jetzigen
               Seelenzuſtand, daß mich auf einmal die Angſt befällt, wo ich in die Dreißig komme,
               die Angſt, daß ich \strikeout{\textcolor{gray}{d}} meine Jugend nicht genoſſen, daß ich herrliche Gelegenheiten verſäumt habe.
               Ich will alſo raſch nachholen. So denke ich ſeit vorigem Sommer daran, mich in den
               Ferien {\pb}mit dem Mädel\pwindex{Mitis, Hilda von 1876-08-30 – 1894-12-14@\textsc{Mitis, Hilda von} (1876-08-30 – 1894-12-14), \emph{Schriftstellerin, Telefonistin}|pwv} zu treffen oder gar ſie nach \textsc{Paris\oindex{Paris@\textbf{Paris}|pw}} kommen zu laſſen, wo ihr Platz wäre. Ich will ihr ſchreiben und verſäume es
               natürlich, wie ich Alles verſäume. Nun kommt an einem grauen Morgen dieſe Nachricht.
               Das heißt für mich viel mehr, als Du ahnen kannſt. Nicht blos ein armes liebes Ding\pwindex{Mitis, Hilda von 1876-08-30 – 1894-12-14@\textsc{Mitis, Hilda von} (1876-08-30 – 1894-12-14), \emph{Schriftstellerin, Telefonistin}|pwv} iſt todt, das mir Gutes
               gethan – ſondern: »Die Jugend iſt vorbei, unwiderruflich vorbei. Man lebt nicht
               wieder, was man einmal zu leben unterlaſſen.«\pend
           \pstart
           Ich habe merkwürdig oft an ſie gedacht. Nicht etwa dieſe dumme romantiſche Geſchichte
               von der hinterdrein kommenden Liebe. Aber \strikeout{n} es war
               die Überzeugung, daß ſie\pwindex{Mitis, Hilda von 1876-08-30 – 1894-12-14@\textsc{Mitis, Hilda von} (1876-08-30 – 1894-12-14), \emph{Schriftstellerin, Telefonistin}|pwv} ein
               ſelten köſtliches Menſchenkind\pwindex{Mitis, Hilda von 1876-08-30 – 1894-12-14@\textsc{Mitis, Hilda von} (1876-08-30 – 1894-12-14), \emph{Schriftstellerin, Telefonistin}|pwv} geweſen {\pb}und daß ich ſie hätte
                  heut noch wenn auch vielleicht nicht lieben, ſo
               doch genießen können. Das iſt übrigens bei mir das ſelbe. Ich kann nicht lieben, nur
               genießen. Ich bin ſeitdem ſtärker geworden; ich war für ſie gereift; nun hätte ich
               ſie mir holen mögen. Einer meiner Lieblings-Träume war: »Reich, und eine Reiſe nach
                  Italien\oindex{Italien@\textbf{Italien}|pw} mit ihr.«\pend
           \pstart
           Ich habe ihre Briefe wieder geleſen und gierig nach Spuren von Falſchheit, Poſe,
               Hyſterie geſucht. Das wäre Balſam geweſen für meine Reue. Ich glaube auch, daß ſie
               mich nicht geliebt hat. Aber ich glaube auch, daß das meine Schuld war. Und neben den
                  {\pb}ſchlimmen Spuren habe ich doch viel einſache
               Güte, Herzigkeit und Poeſie gefunden. Ich glaube beinahe: ſie iſt die einzige Frau
               geweſen, die mich \strikeout{\textcolor{gray}{ver}} verſtanden hat. Das nagt, das nagt. Oh ich blöder Thor!\pend
           \pstart
           Ich glaube auch, ſie hat ſich an mich anlehnen wollen, um das Künſtleriſche in ihr
               zur Entwickelung zu bringen. Ich habe ſie weggeſtoßen. Nicht einmal geſchrieben habe
               ich ihr. Und das Nicht-Schreiben war eine Heuchelei. Denn, wie geſagt, ich dachte
               viel an ſie. Vielleicht, wenn ſie mich um ſich gewußt hätte, wäre ſie nicht in den
               Wald {\pb}gegangen, ſich erſchießen. Ich hätte, ihr laut
               zurufen müſſen, was ich all’ die Jahre dachte: »Kommen Sie nach \textsc{Paris\oindex{Paris@\textbf{Paris}|pw}}!« Ich glaube beinahe, ich habe eine Verantwortung daran, daß dieſe köſtliche
                  Menſchenblume\pwindex{Mitis, Hilda von 1876-08-30 – 1894-12-14@\textsc{Mitis, Hilda von} (1876-08-30 – 1894-12-14), \emph{Schriftstellerin, Telefonistin}|pwv} verkümmert
               iſt. Meine einzige Genugthuung wäre, wenn ich wüßte, daß ſie mich vergeſſen hat. Aber
               wie das erfahren?\pend
           \pstart
           Denk’ nur, dieſer Tod\textcolor{gray}{.} Wie ſtolz, wie heldenmüthig! Er ſagt: »Sie
               war eine edle Frau. Du haſt es nicht verſtanden. Zu ſpät.«\pend
           \pstart
           Ich ſehe mich mit ihr bei Dir, in Deinem lieben {\pb}Zimmer. Es iſt unfaßbar, daß das Alles verloren iſt. Schatten und Reue. Das »Zu
               ſpät« brennt wie Feuer auf dem Herzen.\pend
           \pstart
           Könnteſt Du nicht noch etwas über ihr\pwindex{Mitis, Hilda von 1876-08-30 – 1894-12-14@\textsc{Mitis, Hilda von} (1876-08-30 – 1894-12-14), \emph{Schriftstellerin, Telefonistin}|pwv} Leben erfahren? Ich möchte hören, daß ſie liederlich geweſen iſt, daß
               ſie banal geworden iſt. Auch möchte ich wiſſen, \strikeout{\textcolor{gray}{×}} warum ſie geſtorben iſt. Liebe zum Vater\pwindex{Mitis, Maximilian von 1840-07-12 – 1894-12-10@\textsc{Mitis, Maximilian von} (1840-07-12 – 1894-12-10), \emph{Sekretär}|pwv}? Ich glaube nicht. Sie hat einen kleinen dummen Lieutenaut\pwindex{?? [Braeutigam von Hildegard Mitis] @\textsc{?? [Bräutigam von Hildegard Mitis]}|pwv} zum Bräutigam\pwindex{?? [Braeutigam von Hildegard Mitis] @\textsc{?? [Bräutigam von Hildegard Mitis]}|pwv} gehabt und ihn ſehr
               geliebt. Der mag ihr auf ihre »Unmoral« gekommen ſein und ſie weggeſtoßen {\pb}haben. Dann ſtarb der \label{K_L02628-2v}\edtext{Vater\pwindex{Mitis, Maximilian von 1840-07-12 – 1894-12-10@\textsc{Mitis, Maximilian von} (1840-07-12 – 1894-12-10), \emph{Sekretär}|pwv}}{\lemma{\textnormal{\emph{Vater}}}\Cendnote{\textnormal{Maximilian von Mitis\pwindex{Mitis, Maximilian von 1840-07-12 – 1894-12-10@\textsc{Mitis, Maximilian von} (1840-07-12 – 1894-12-10), \emph{Sekretär}|pwk} starb vier Tage vor
                  seiner Tochter\pwindex{Mitis, Hilda von 1876-08-30 – 1894-12-14@\textsc{Mitis, Hilda von} (1876-08-30 – 1894-12-14), \emph{Schriftstellerin, Telefonistin}|pwkv}.}}}\label{K_L02628-2h}.
               Nun kam die unendliche Vereinſamung über ſie, vielleicht auch die Noth. Darum hat
               ſies gethan.\pend
           \pstart
           Wenn es einen gnädigen Gott gäbe, hätte ich an jenem Tage im Preßburg\oindex{Bratislava@\textbf{Bratislava}|pw}er Walde ſein müſſen. Wie ich ſie ins Leben
               zurückgetragen hätte auf meinen Armen!\pend
           \pstart
           Nun kommen mir die Thränen.\pend
           \pstart
           Siehſt Du nun, wie verfehlt mein Leben iſt?\pend
           \pstart
           Grüß’ Dich Gott, theurer Freund!{\\[\baselineskip]}Dein{\\[\baselineskip]}\spacefill\mbox{Paul Goldmann}\pend
           \leftskip=0em{}
         
         \endnumbering\mylabel{h}\end{ledgroupsized}  \newcommand{\dateiname}{L02628}\newcommand{\titel}{Paul Goldmann an Arthur Schnitzler, 18. 12. [1894]}\newcommand{\editorInnen}{Martin Anton Müller und Laura Untner}%% latex-leseansicht-abspann.tex
%% Abspann für die Leseansicht.
%% Der Schalter \ifkorrekturansicht ist bereits durch den Vorspann gesetzt.

%% latex-abspann.tex
%% Gemeinsamer Abspann für Korrekturansicht und Leseansicht.
%% Setzt den Schalter \ifkorrekturansicht voraus (gesetzt in den
%% einbindenden Dateien latex-korrekturansicht-abspann.tex bzw.
%% latex-leseansicht-abspann.tex).
%% ---------------------------------------------------------------

\normalsize

% Das esempio-Environment wird nur in der Leseansicht benötigt
\ifkorrekturansicht\else
\newenvironment{esempio}[3]%
{
    \vspace{1.5ex}
    \rlap{\underline{#1}}
    \par
    \setlength{\parindent}{0cm}
    \nopagebreak
    \leftskip=#2cm
    \rightskip=#3cm
}
{
    \par
}
\fi

\doendnotes{C}
\bigskip
\vfill

\clearpage

\footnotesize

\ifkorrekturansicht
  \lohead{\textsc{register}}
\fi

% theindex-Environment neu definieren ohne reledmac
\makeatletter
\renewenvironment{theindex}{%
  \ifkorrekturansicht
    \section*{\indexname}%
  \else
    \subsubsection*{Index der erwähnten Entitäten}%
  \fi
  \setlength{\parindent}{0pt}%
  \setlength{\parskip}{0pt plus 0.3pt}%
  \let\item\@idxitem
}{%
  \ifkorrekturansicht\clearpage\fi
}
\makeatother

\IfFileExists{\jobname-pw.ind}{\input{\jobname-pw.ind}}{}

% Quellenangabe nur in der Leseansicht
\ifkorrekturansicht\else
% Fallback-Definitionen, falls die .tex-Datei \titel etc. nicht gesetzt hat
\providecommand{\titel}{}
\providecommand{\editorInnen}{}
\providecommand{\dateiname}{\jobname}

\vspace{3cm}

\vfill

\footnotesize
\textsc{Quelle}: \titel. Herausgegeben von {\editorInnen}. In: \emph{Arthur Schnitzler: Briefwechsel mit Autorinnen und Autoren}.
 Digitale Edition, https://schnitzler-briefe.acdh.oeaw.ac.at/{\dateiname}.html (Stand \today)
\fi

\end{document}


      