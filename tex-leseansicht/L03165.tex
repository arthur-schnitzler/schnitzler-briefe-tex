%% latex-leseansicht-vorspann.tex
%% Vorspann für die Leseansicht.
%% Lädt die gemeinsame Datei latex-vorspann.tex mit nicht gesetztem Schalter.

\newif\ifkorrekturansicht
\korrekturansichtfalse

\input{../tex-inputs/latex-vorspann}


         
         \renewcommand{\erwaehntePersonen}{Personen: Felix Salten}
         \renewcommand{\erwaehnteInstitutionen}{Institutionen: Burgtheater}
         \renewcommand{\erwaehnteOrte}{Orte: Wien}
         \renewcommand{\erwaehnteWerke}{Werke: Liebelei. Schauspiel in drei Akten}
               \section[ Felix Salten an Arthur Schnitzler, {[}7. 9. 1895{]}]{ Felix Salten an Arthur Schnitzler, {[}7. 9. 1895{]}}\nopagebreak\mylabel{v}\rehead{ }\begin{ledgroupsized}[t]{13cm}\normalsize\beginnumbering \toendnotes[C]{\smallbreak\pagebreak[2]} \Standort{CUL, Schnitzler, B 89, A 1.}
\physDesc{Brief, 1 Blatt, 1 Seite, 89 Zeichen (als Briefpapier wurde das Fragment eines Probenplans des Burgtheaters\orgindex{Burgtheater@Burgtheater|pw}
                                 verwendet, der vorgedruckte Text mit Bleistift durch mehrfache Übermalung markiert)
\newline{}Handschrift: Bleistift, lateinische Kurrent
\newline{}Schnitzler: mit Bleistift datiert: »7/9 95« 
\newline{}Ordnung: mit Bleistift von unbekannter Hand nummeriert: »65« }\toendnotes[C]{\smallbreak}\pstart
           \noindent{}\centering{}{\pb}\textcolor{gray}{\textbf{Reſerven: Alle unlängſt oder oft gegebenen, daher als
                        feſtſtehend zu erachtenden Vorſtellungen.}}\pend
           \settowidth{\longeste}{Neu:}\settowidth{\longestz}{Neu einſtudirt und in Scene geſetzt:}\settowidth{\longestd}{Repriſen:}\settowidth{\longestv}{}\settowidth{\longestf}{}\addtolength\longeste{1em}
        \addtolength\longestz{1em}
        \addtolength\longestd{1em}
      \pstart\noindent\makebox[\the\longeste][l]{\textcolor{gray}{\textbf{Neu:}}}\makebox[\the\longestz][l]{\textcolor{gray}{\textbf{Neu einſtudirt und in Scene geſetzt:}}}
                  \makebox[\the\longestd][l]{\textcolor{gray}{\textbf{Repriſen:}}}\pend\pstart\noindent\makebox[\the\longeste][l]{\textcolor{gray}{\textbf{\label{K_L03165-1v}\edtext{Liebelei}{\lemma{\textnormal{\emph{Liebelei}}}\Cendnote{\textnormal{mit einem Pfeil
                              markiert}}}\label{K_L03165-1h}\pwindex{Schnitzler, Arthur 15.05.1862 – 21.10.1931@\textsc{Schnitzler, Arthur} (15.05.1862 – 21.10.1931), \emph{Schriftsteller, Mediziner}!Liebelei. Schauspiel in drei Akten1895-10-09@\strich\emph{Liebelei. Schauspiel in drei Akten} {[}1895-10-09{]}|pw}}}}\makebox[\the\longestz][l]{}
                  \makebox[\the\longestd][l]{}\pend\pstart
           lieber Arthur! Wenn Sie schon \label{K_L03165-2v}\edtext{hier\oindex{Wien@\textbf{Wien}|pwv}}{\lemma{\textnormal{\emph{hier}}}\Cendnote{\textnormal{Schnitzler\pwindex{Schnitzler, Arthur 15.05.1862 – 21.10.1931@\textsc{Schnitzler, Arthur} (15.05.1862 – 21.10.1931), \emph{Schriftsteller, Mediziner}|pwk} kehrte an diesem Tag nach Wien\oindex{Wien@\textbf{Wien}|pwk} zurück. Nachweislich sah er Salten\pwindex{Salten, Felix 06.09.1869 – 08.10.1945@\textsc{Salten, Felix} (06.09.1869 – 08.10.1945), \emph{Schriftsteller, Journalist}|pwk} erst am 12. 9. 1895
                  wieder.}}}\label{K_L03165-2h} sind, laßen Sie michs für Nachmittg wissen\pend
           \pstart
           herzl. Ihr {\\[\baselineskip]}\spacefill\mbox{Salten.}\pend
           \leftskip=0em{}
         
         \endnumbering\mylabel{h}\end{ledgroupsized}  \newcommand{\dateiname}{L03165}\newcommand{\titel}{Felix Salten an Arthur Schnitzler, [7. 9. 1895]}\newcommand{\editorInnen}{Martin Anton Müller und Laura Untner}%% latex-leseansicht-abspann.tex
%% Abspann für die Leseansicht.
%% Der Schalter \ifkorrekturansicht ist bereits durch den Vorspann gesetzt.

%% latex-abspann.tex
%% Gemeinsamer Abspann für Korrekturansicht und Leseansicht.
%% Setzt den Schalter \ifkorrekturansicht voraus (gesetzt in den
%% einbindenden Dateien latex-korrekturansicht-abspann.tex bzw.
%% latex-leseansicht-abspann.tex).
%% ---------------------------------------------------------------

\normalsize

% Das esempio-Environment wird nur in der Leseansicht benötigt
\ifkorrekturansicht\else
\newenvironment{esempio}[3]%
{
    \vspace{1.5ex}
    \rlap{\underline{#1}}
    \par
    \setlength{\parindent}{0cm}
    \nopagebreak
    \leftskip=#2cm
    \rightskip=#3cm
}
{
    \par
}
\fi

\doendnotes{C}
\bigskip
\vfill

\clearpage

\footnotesize

\ifkorrekturansicht
  \lohead{\textsc{register}}
\fi

% theindex-Environment neu definieren ohne reledmac
\makeatletter
\renewenvironment{theindex}{%
  \ifkorrekturansicht
    \section*{\indexname}%
  \else
    \subsubsection*{Index der erwähnten Entitäten}%
  \fi
  \setlength{\parindent}{0pt}%
  \setlength{\parskip}{0pt plus 0.3pt}%
  \let\item\@idxitem
}{%
  \ifkorrekturansicht\clearpage\fi
}
\makeatother

\IfFileExists{\jobname-pw.ind}{\input{\jobname-pw.ind}}{}

% Quellenangabe nur in der Leseansicht
\ifkorrekturansicht\else
% Fallback-Definitionen, falls die .tex-Datei \titel etc. nicht gesetzt hat
\providecommand{\titel}{}
\providecommand{\editorInnen}{}
\providecommand{\dateiname}{\jobname}

\vspace{3cm}

\vfill

\footnotesize
\textsc{Quelle}: \titel. Herausgegeben von {\editorInnen}. In: \emph{Arthur Schnitzler: Briefwechsel mit Autorinnen und Autoren}.
 Digitale Edition, https://schnitzler-briefe.acdh.oeaw.ac.at/{\dateiname}.html (Stand \today)
\fi

\end{document}


      