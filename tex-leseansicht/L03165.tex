%% latex-leseansicht-vorspann.tex
%% Vorspann für die Leseansicht.
%% Lädt die gemeinsame Datei latex-vorspann.tex mit nicht gesetztem Schalter.

\newif\ifkorrekturansicht
\korrekturansichtfalse

\input{../tex-inputs/latex-vorspann}


\section[ Felix Salten an Arthur Schnitzler, {[}7. 9. 1895{]}]{L03165 Felix Salten an Arthur Schnitzler,  [7. 9. 1895]}
\nopagebreak\mylabel{L03165v}
\rehead{ }\normalsize\beginnumbering\briefempfaengerindex{Schnitzler, Arthur@\textsc{Schnitzler, Arthur}!zzzSalten, Felix@\emph{von Felix Salten}!1895-09-071@{{[}7. 9. 1895{]}}|(be}
\toendnotes[C]{\smallbreak\pagebreak[2]}
\correspDesc{Versand  durch Felix Salten am [7. 9. 1895] in Wien
\newline{}Erhalt  durch Arthur Schnitzler am [7. 9. 1895?] in Wien}\toendnotes[C]{\smallbreak}
\Standort{CUL, Schnitzler, B 89, A 1.}
\physDesc{Brief, 1 Blatt, 1 Seite, 89 Zeichen (als Briefpapier wurde das Fragment eines Probenplans des Burgtheaters\orgindex{Burgtheater@Burgtheater|pw}
                                 verwendet, der vorgedruckte Text mit Bleistift durch mehrfache Übermalung markiert)
\newline{}Handschrift: Bleistift, lateinische Kurrent
\newline{}Schnitzler: mit Bleistift datiert: »7/9 95« 
\newline{}Ordnung: mit Bleistift von unbekannter Hand nummeriert: »65« }\toendnotes[C]{\smallbreak}
\pstart
           \centering{}{\pb}\textcolor{gray}{\textbf{Reſerven: Alle unlängſt oder oft gegebenen, daher als
                        feſtſtehend zu erachtenden Vorſtellungen.}}\pend
           \vspace{0.5em}\settowidth{\longeste}{Liebelei}\settowidth{\longestz}{Neu einstudirt und in Scene gesetzt:mm}\settowidth{\longestd}{Reprisen:m}\settowidth{\longestv}{}\settowidth{\longestf}{}\addtolength\longeste{1em}
        \addtolength\longestz{1em}
        \addtolength\longestd{1em}
      \pstart\noindent\makebox[\the\longeste][l]{\textcolor{gray}{\textbf{Neu:}}}\makebox[\the\longestz][l]{\textcolor{gray}{\textbf{Neu einſtudirt und in Scene geſetzt:}}}
                  \makebox[\the\longestd][l]{\textcolor{gray}{\textbf{Repriſen:}}}\pend\pstart\noindent\makebox[\the\longeste][l]{\textcolor{gray}{\textbf{\label{K_L03165-1v}\edtext{Liebelei}{\lemma{\textnormal{\emph{Liebelei}}}\Cendnote{\textnormal{mit einem Pfeil
                              markiert}}}\label{K_L03165-1}\pwindex{Schnitzler, Arthur 15.\,5.\,1862 Wien – 21.\,10.\,1931 ebd.@\textsc{Schnitzler, Arthur} (15.\,5.\,1862 Wien – 21.\,10.\,1931 ebd.), \emph{Schriftsteller, Mediziner}!Liebelei. Schauspiel in drei Akten@\strich\emph{Liebelei. Schauspiel in drei Akten}|pw}}}}\makebox[\the\longestz][l]{}
                  \makebox[\the\longestd][l]{}\pend
\pstart
           lieber Arthur! Wenn Sie schon \label{K_L03165-2v}\edtext{hier\oindex{Wien@\textbf{Wien}, \emph{Verwaltungsgebiet}|pwv}}{\lemma{\textnormal{\emph{hier}}}\Cendnote{\textnormal{Schnitzler kehrte an diesem Tag nach Wien\oindex{Wien@\textbf{Wien}, \emph{Verwaltungsgebiet}|pwk} zurück. Nachweislich sah er Salten\pwindex{Salten, Felix 6.\,9.\,1869 Budapest – 8.\,10.\,1945 Zürich@\textsc{Salten, Felix} (6.\,9.\,1869 Budapest – 8.\,10.\,1945 Zürich), \emph{Schriftsteller, Journalist, Chefredakteur}|pwk} erst am 12. 9. 1895
                  wieder.}}}\label{K_L03165-2} sind, laßen Sie michs für Nachmittg wissen\pend
           
\pstart
           herzl. Ihr {\\[\baselineskip]}\spacefill\mbox{Salten.}\pend
           \leftskip=0em{}\selectlanguage{ngerman}\endnumbering\briefempfaengerindex{Schnitzler, Arthur@\textsc{Schnitzler, Arthur}!zzzSalten, Felix@\emph{von Felix Salten}!1895-09-071@{{[}7. 9. 1895{]}}|)be}\mylabel{L03165h}  \newcommand{\dateiname}{L03165}\newcommand{\titel}{Felix Salten an Arthur Schnitzler, [7. 9. 1895]}\newcommand{\editorInnen}{Martin Anton Müller und Laura Untner}%% latex-leseansicht-abspann.tex
%% Abspann für die Leseansicht.
%% Der Schalter \ifkorrekturansicht ist bereits durch den Vorspann gesetzt.

%% latex-abspann.tex
%% Gemeinsamer Abspann für Korrekturansicht und Leseansicht.
%% Setzt den Schalter \ifkorrekturansicht voraus (gesetzt in den
%% einbindenden Dateien latex-korrekturansicht-abspann.tex bzw.
%% latex-leseansicht-abspann.tex).
%% ---------------------------------------------------------------

\normalsize

% Das esempio-Environment wird nur in der Leseansicht benötigt
\ifkorrekturansicht\else
\newenvironment{esempio}[3]%
{
    \vspace{1.5ex}
    \rlap{\underline{#1}}
    \par
    \setlength{\parindent}{0cm}
    \nopagebreak
    \leftskip=#2cm
    \rightskip=#3cm
}
{
    \par
}
\fi

\doendnotes{C}
\bigskip
\vfill

\clearpage

\footnotesize

\ifkorrekturansicht
  \lohead{\textsc{register}}
\fi

% theindex-Environment neu definieren ohne reledmac
\makeatletter
\renewenvironment{theindex}{%
  \ifkorrekturansicht
    \section*{\indexname}%
  \else
    \subsubsection*{Index der erwähnten Entitäten}%
  \fi
  \setlength{\parindent}{0pt}%
  \setlength{\parskip}{0pt plus 0.3pt}%
  \let\item\@idxitem
}{%
  \ifkorrekturansicht\clearpage\fi
}
\makeatother

\IfFileExists{\jobname-pw.ind}{\input{\jobname-pw.ind}}{}

% Quellenangabe nur in der Leseansicht
\ifkorrekturansicht\else
% Fallback-Definitionen, falls die .tex-Datei \titel etc. nicht gesetzt hat
\providecommand{\titel}{}
\providecommand{\editorInnen}{}
\providecommand{\dateiname}{\jobname}

\vspace{3cm}

\vfill

\footnotesize
\textsc{Quelle}: \titel. Herausgegeben von {\editorInnen}. In: \emph{Arthur Schnitzler: Briefwechsel mit Autorinnen und Autoren}.
 Digitale Edition, https://schnitzler-briefe.acdh.oeaw.ac.at/{\dateiname}.html (Stand \today)
\fi

\end{document}


