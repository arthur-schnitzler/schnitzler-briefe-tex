%% latex-korrekturansicht-vorspann.tex
%% Vorspann für die Korrekturansicht.
%% Lädt die gemeinsame Datei latex-vorspann.tex mit gesetztem Schalter.

\newif\ifkorrekturansicht
\korrekturansichttrue

\input{../tex-inputs/latex-vorspann}


\section[Hugo von Hofmannsthal an Arthur Schnitzler, {[}23. 6. 1904{]}]{L01409 Hugo von Hofmannsthal an Arthur Schnitzler, {[}23. 6. 1904{]}}
\nopagebreak\mylabel{L01409v}
\rehead{ }\normalsize\beginnumbering\briefempfaengerindex{Schnitzler, Arthur@\textsc{Schnitzler, Arthur}!zzzHofmannsthal, Hugo von@\emph{von Hugo von Hofmannsthal}!1904-06-231@{{[}23. 6. 1904{]}}|(be}
\toendnotes[C]{\smallbreak\pagebreak[2]}\Standort{CUL, Schnitzler, B 43.}
\physDesc{Brief, 1 Blatt, 2 Seiten, 390 Zeichen
\newline{}Handschrift: schwarze Tinte, deutsche Kurrent
\newline{}Schnitzler: mit Bleistift datiert: »23/6 904.« 
\newline{}Ordnung: 1) mit Bleistift von unbekannter Hand nummeriert: »\strikeout{239}«  2) mit Bleistift von unbekannter Hand nummeriert:
                                    »225«}
\buchAbdrucke{\weitereDrucke{Hugo von Hofmannsthal, Arthur Schnitzler: \emph{Briefwechsel}. Frankfurt am Main: \emph{S. Fischer} 1964, S. 189.} }\toendnotes[C]{\smallbreak}
\pstart{}{\pb}lieber\pend\vspace{0.5em}
\pstart
           1.) wie gehts Ihnen\pend
           
\pstart
           2.) bitte ko{\geminationm}en Sie nächſten Do{\geminationn}erstag, weil Mittwoch das Kinderfräulein\pwindex{?? [Kinderfrau bei Hofmannsthal] 23.6.1904 – 23.6.1904@\textsc{?? [Kinderfrau bei Hofmannsthal]} (23.6.1904 – 23.6.1904)|pwv} Ausgang
               hat\pend
           
\pstart
           3.) wir nehmen als ſelbſtverſtändlich an, daſs Ihr Liſl\pwindex{Steinrueck, Elisabeth 19.11.1885 – 07.04.1920@\textsc{Steinrück, Elisabeth} (19.11.1885 – 07.04.1920)|pw} mitbringt\pend
           
\pstart
           4.) Olga\pwindex{Schnitzler, Olga 17.01.1882 – 13.01.1970@\textsc{Schnitzler, Olga} (17.01.1882 – 13.01.1970), \emph{Schauspieler/Schauspielerin, Sänger/Sängerin}|pw}{ }ſoll nur ja nicht etwa in der Abſicht, damit {\pb}einen guten Zweck zu erreichen,
               irgendwie unſere Geſpräche über die Bären\pwindex{Beer-Hofmann, Paula 25.02.1879 – 30.10.1939@\textsc{Beer-Hofmann, Paula} (25.02.1879 – 30.10.1939)|pw}\pwindex{Beer-Hofmann, Richard 1866-07-11 – 1945-09-26@\textsc{Beer-Hofmann, Richard} (1866-07-11 – 1945-09-26), \emph{Schriftsteller/Schriftstellerin}|pw} gegen Frl. Mütter\pwindex{Muetter, Franziska 25.05.1858 – 23.02.1919@\textsc{Mütter, Franziska} (25.05.1858 – 23.02.1919), \emph{Sänger/Sängerin, Gesangspädagoge/Gesangspädagogin}|pw} erwähnen. Es
               würde daraus ganz ſicher etwas unangenehmes entſtehen.\pend
           
\pstart
           Von Herzen{\\[\baselineskip]}\spacefill\mbox{Hugo.}\pend
           \leftskip=0em{}\selectlanguage{ngerman}\endnumbering\briefempfaengerindex{Schnitzler, Arthur@\textsc{Schnitzler, Arthur}!zzzHofmannsthal, Hugo von@\emph{von Hugo von Hofmannsthal}!1904-06-231@{{[}23. 6. 1904{]}}|)be}\mylabel{L01409h}  \normalsize

\doendnotes{C}
\bigskip
\vfill

\clearpage

\footnotesize

\lohead{\textsc{register}}

% Definiere theindex-Environment komplett neu ohne reledmac
\makeatletter
\renewenvironment{theindex}{%
  \section*{\indexname}%
  \setlength{\parindent}{0pt}%
  \setlength{\parskip}{0pt plus 0.3pt}%
  \let\item\@idxitem
}{%
  \clearpage
}
\makeatother

\IfFileExists{\jobname-pw.ind}{\input{\jobname-pw.ind}}{}

\end{document}

      