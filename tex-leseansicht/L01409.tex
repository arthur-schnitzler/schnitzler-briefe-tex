%% latex-leseansicht-vorspann.tex
%% Vorspann für die Leseansicht.
%% Lädt die gemeinsame Datei latex-vorspann.tex mit nicht gesetztem Schalter.

\newif\ifkorrekturansicht
\korrekturansichtfalse

\input{../tex-inputs/latex-vorspann}


         
         \renewcommand{\erwaehntePersonen}{Personen:  ?? [Kinderfrau bei Hofmannsthal], Paula Beer-Hofmann, Richard Beer-Hofmann, Franziska Mütter, Olga Schnitzler, Elisabeth Steinrück}
         \renewcommand{\erwaehnteOrte}{Orte: Wien}
         \renewcommand{\erwaehnteWerke}{}
               \section[Hugo von Hofmannsthal an Arthur Schnitzler, {[}23. 6. 1904{]}]{ Hugo von Hofmannsthal an Arthur Schnitzler, {[}23. 6. 1904{]}}\nopagebreak\mylabel{v}\rehead{ }\begin{ledgroupsized}[t]{13cm}\normalsize\beginnumbering \toendnotes[C]{\smallbreak\pagebreak[2]} \Standort{CUL, Schnitzler, B 43.}
\physDesc{Brief, 1 Blatt, 2 Seiten
\newline{}Handschrift: schwarze Tinte, deutsche Kurrent
\newline{}Schnitzler: mit Bleistift datiert: »23/6 904.« \newline{}Ordnung: 1) mit Bleistift von unbekannter Hand nummeriert: »\strikeout{239}«  2) mit Bleistift von unbekannter Hand nummeriert: »225«}\buchAbdrucke{\weitereDrucke{Hugo von Hofmannsthal, Arthur Schnitzler: \emph{Briefwechsel}. Hg. Therese Nickl und Heinrich Schnitzler. Frankfurt am Main: \emph{S. Fischer} 1964, S. 189.} }\toendnotes[C]{\smallbreak}\pstart{}{\pb}lieber\pend\pstart
           1.) wie gehts Ihnen\pend
           \pstart
           2.) bitte ko{\geminationm}en Sie nächſten Do{\geminationn}erstag, weil Mittwoch das Kinderfräulein\pwindex{?? [Kinderfrau bei Hofmannsthal] 23.6.1904 – 23.6.1904@\textsc{?? [Kinderfrau bei Hofmannsthal]} (23.6.1904 – 23.6.1904)|pwv} Ausgang hat\pend
           \pstart
           3.) wir nehmen als ſelbſtverſtändlich an, daſs Ihr Liſl\pwindex{Steinrueck, Elisabeth 19.11.1885 – 07.04.1920@\textsc{Steinrück, Elisabeth} (19.11.1885 – 07.04.1920)|pw} mitbringt\pend
           \pstart
           4.) Olga\pwindex{Schnitzler, Olga 17.01.1882 – 13.01.1970@\textsc{Schnitzler, Olga} (17.01.1882 – 13.01.1970), \emph{Schauspielerin, Sängerin}|pw}{ }ſoll nur ja nicht etwa in der Abſicht, damit {\pb}einen guten Zweck zu erreichen,
               irgendwie unſere Geſpräche über die Bären\pwindex{Beer-Hofmann, Paula 25.02.1879 – 30.10.1939@\textsc{Beer-Hofmann, Paula} (25.02.1879 – 30.10.1939)|pw}\pwindex{Beer-Hofmann, Richard 1866-07-11 – 1945-09-26@\textsc{Beer-Hofmann, Richard} (1866-07-11 – 1945-09-26), \emph{Schriftsteller}|pw}
               gegen Frl. Mütter\pwindex{Muetter, Franziska 25.05.1858 – 23.02.1919@\textsc{Mütter, Franziska} (25.05.1858 – 23.02.1919), \emph{Sängerin, Gesangspädagogin}|pw} erwähnen. Es würde daraus ganz
               ſicher etwas unangenehmes entſtehen.\pend
           \pstart
           Von Herzen{\\[\baselineskip]}\spacefill\mbox{Hugo.}\pend
           \leftskip=0em{}
         
         \endnumbering\mylabel{h}\end{ledgroupsized}  \newcommand{\dateiname}{L01409}\newcommand{\titel}{Hugo von Hofmannsthal an Arthur Schnitzler, [23. 6. 1904]}\newcommand{\editorInnen}{Martin Anton Müller und Gerd-Hermann Susen}%% latex-leseansicht-abspann.tex
%% Abspann für die Leseansicht.
%% Der Schalter \ifkorrekturansicht ist bereits durch den Vorspann gesetzt.

%% latex-abspann.tex
%% Gemeinsamer Abspann für Korrekturansicht und Leseansicht.
%% Setzt den Schalter \ifkorrekturansicht voraus (gesetzt in den
%% einbindenden Dateien latex-korrekturansicht-abspann.tex bzw.
%% latex-leseansicht-abspann.tex).
%% ---------------------------------------------------------------

\normalsize

% Das esempio-Environment wird nur in der Leseansicht benötigt
\ifkorrekturansicht\else
\newenvironment{esempio}[3]%
{
    \vspace{1.5ex}
    \rlap{\underline{#1}}
    \par
    \setlength{\parindent}{0cm}
    \nopagebreak
    \leftskip=#2cm
    \rightskip=#3cm
}
{
    \par
}
\fi

\doendnotes{C}
\bigskip
\vfill

\clearpage

\footnotesize

\ifkorrekturansicht
  \lohead{\textsc{register}}
\fi

% theindex-Environment neu definieren ohne reledmac
\makeatletter
\renewenvironment{theindex}{%
  \ifkorrekturansicht
    \section*{\indexname}%
  \else
    \subsubsection*{Index der erwähnten Entitäten}%
  \fi
  \setlength{\parindent}{0pt}%
  \setlength{\parskip}{0pt plus 0.3pt}%
  \let\item\@idxitem
}{%
  \ifkorrekturansicht\clearpage\fi
}
\makeatother

\IfFileExists{\jobname-pw.ind}{\input{\jobname-pw.ind}}{}

% Quellenangabe nur in der Leseansicht
\ifkorrekturansicht\else
% Fallback-Definitionen, falls die .tex-Datei \titel etc. nicht gesetzt hat
\providecommand{\titel}{}
\providecommand{\editorInnen}{}
\providecommand{\dateiname}{\jobname}

\vspace{3cm}

\vfill

\footnotesize
\textsc{Quelle}: \titel. Herausgegeben von {\editorInnen}. In: \emph{Arthur Schnitzler: Briefwechsel mit Autorinnen und Autoren}.
 Digitale Edition, https://schnitzler-briefe.acdh.oeaw.ac.at/{\dateiname}.html (Stand \today)
\fi

\end{document}


      