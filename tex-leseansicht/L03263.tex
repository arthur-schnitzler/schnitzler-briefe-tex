%% latex-leseansicht-vorspann.tex
%% Vorspann für die Leseansicht.
%% Lädt die gemeinsame Datei latex-vorspann.tex mit nicht gesetztem Schalter.

\newif\ifkorrekturansicht
\korrekturansichtfalse

\input{../tex-inputs/latex-vorspann}

\begin{center}
            \textcolor{red}{ENTWURF, NICHT FERTIG KORRIGIERT}
                      \end{center}
            
         
         \renewcommand{\erwaehntePersonen}{Personen: Paul Blasel}
         \renewcommand{\erwaehnteInstitutionen}{Institutionen: Stadttheater (Teplitz)}
         \renewcommand{\erwaehnteOrte}{Orte: Teplice, Wien}
         \renewcommand{\erwaehnteWerke}{}
               \section[Felix Salten an Arthur Schnitzler, 16. 1. 1897]{ Felix Salten an Arthur Schnitzler, 16. 1. 1897}\nopagebreak\mylabel{v}\rehead{ }\begin{ledgroupsized}[t]{13cm}\normalsize\beginnumbering \toendnotes[C]{\smallbreak\pagebreak[2]} \Standort{CUL, Schnitzler, B 89, A 2.}
\physDesc{Brief, 1 Blatt, 2 Seiten, 336 Zeichen
\newline{}Handschrift: Bleistift, lateinische Kurrent
\newline{}Ordnung: mit Bleistift von unbekannter Hand nummeriert:
                                    »85« }\toendnotes[C]{\smallbreak}\pstart
           {\pb}Teplitz\oindex{Teplice@\textbf{Teplice}|pw}, 16/I. 97\pend
           \pstart
           Lieber Freund! Heute habe ich alles eingeleitet. Die Chancen sind
               meiner Ansicht nach nur gering, obwol man mir das Gegentheil zu sagen versucht.
               Schade, dass Sie sich nicht entschließen können. \uline{Das}
               wäre die absolute Sicherheit. Die {\pb}Stadt\oindex{Teplice@\textbf{Teplice}|pwv} ist reizend und billig.
               Das \label{K_L03263-1v}\edtext{Theater\orgindex{Stadttheater (Teplitz)@Stadttheater (Teplitz)|pwv}}{\lemma{\textnormal{\emph{Theater}}}\Cendnote{\textnormal{Paul Blasel\pwindex{Blasel, Paul 1855-06-29 – 1940-06-21@\textsc{Blasel, Paul} (1855-06-29 – 1940-06-21), \emph{Opernsänger, Schauspieler, Theaterdirektor}|pwk} hatte zum Jahreswechsel
                  bekanntgegeben, dass er nach zwei Saisonen die Leitung des \emph{Stadttheaters}\orgindex{Stadttheater (Teplitz)@Stadttheater (Teplitz)|pwk} mit Ablauf der Saison zurückgeben werde. Ob
                     Salten\pwindex{Salten, Felix 06.09.1869 – 08.10.1945@\textsc{Salten, Felix} (06.09.1869 – 08.10.1945), \emph{Schriftsteller, Journalist}|pwk} sich tatsächlich um die Nachfolge
                  bewarb, ist offen.}}}\label{K_L03263-1h} prachtvoll.\pend
           \pstart
           Auf Wiedersehen Dienstag. {\\[\baselineskip]}Herzlich {\\[\baselineskip]}Ihr {\\[\baselineskip]}\spacefill\mbox{Salten}\pend
           \leftskip=0em{}
         
         \endnumbering\mylabel{h}\end{ledgroupsized}\begin{anhang}\end{anhang}\newcommand{\dateiname}{L03263}\newcommand{\titel}{Felix Salten an Arthur Schnitzler, 16. 1. 1897}\newcommand{\editorInnen}{Martin Anton Müller und Laura Untner}%% latex-leseansicht-abspann.tex
%% Abspann für die Leseansicht.
%% Der Schalter \ifkorrekturansicht ist bereits durch den Vorspann gesetzt.

%% latex-abspann.tex
%% Gemeinsamer Abspann für Korrekturansicht und Leseansicht.
%% Setzt den Schalter \ifkorrekturansicht voraus (gesetzt in den
%% einbindenden Dateien latex-korrekturansicht-abspann.tex bzw.
%% latex-leseansicht-abspann.tex).
%% ---------------------------------------------------------------

\normalsize

% Das esempio-Environment wird nur in der Leseansicht benötigt
\ifkorrekturansicht\else
\newenvironment{esempio}[3]%
{
    \vspace{1.5ex}
    \rlap{\underline{#1}}
    \par
    \setlength{\parindent}{0cm}
    \nopagebreak
    \leftskip=#2cm
    \rightskip=#3cm
}
{
    \par
}
\fi

\doendnotes{C}
\bigskip
\vfill

\clearpage

\footnotesize

\ifkorrekturansicht
  \lohead{\textsc{register}}
\fi

% theindex-Environment neu definieren ohne reledmac
\makeatletter
\renewenvironment{theindex}{%
  \ifkorrekturansicht
    \section*{\indexname}%
  \else
    \subsubsection*{Index der erwähnten Entitäten}%
  \fi
  \setlength{\parindent}{0pt}%
  \setlength{\parskip}{0pt plus 0.3pt}%
  \let\item\@idxitem
}{%
  \ifkorrekturansicht\clearpage\fi
}
\makeatother

\IfFileExists{\jobname-pw.ind}{\input{\jobname-pw.ind}}{}

% Quellenangabe nur in der Leseansicht
\ifkorrekturansicht\else
% Fallback-Definitionen, falls die .tex-Datei \titel etc. nicht gesetzt hat
\providecommand{\titel}{}
\providecommand{\editorInnen}{}
\providecommand{\dateiname}{\jobname}

\vspace{3cm}

\vfill

\footnotesize
\textsc{Quelle}: \titel. Herausgegeben von {\editorInnen}. In: \emph{Arthur Schnitzler: Briefwechsel mit Autorinnen und Autoren}.
 Digitale Edition, https://schnitzler-briefe.acdh.oeaw.ac.at/{\dateiname}.html (Stand \today)
\fi

\end{document}


      