%% latex-leseansicht-vorspann.tex
%% Vorspann für die Leseansicht.
%% Lädt die gemeinsame Datei latex-vorspann.tex mit nicht gesetztem Schalter.

\newif\ifkorrekturansicht
\korrekturansichtfalse

\input{../tex-inputs/latex-vorspann}


\section[ Felix Salten an Arthur Schnitzler, 16. 1. 1897]{L03263 Felix Salten an Arthur Schnitzler,  16. 1. 1897}
\nopagebreak\mylabel{L03263v}
\rehead{ }\normalsize\beginnumbering\briefempfaengerindex{Schnitzler, Arthur@\textsc{Schnitzler, Arthur}!zzzSalten, Felix@\emph{von Felix Salten}!1897-01-164@{16. 1. 1897}|(be}
\toendnotes[C]{\smallbreak\pagebreak[2]}
\correspDesc{Versand  durch Felix Salten am 16. 1. 1897 in Teplice
\newline{}Erhalt  durch Arthur Schnitzler im Zeitraum [17. 1. 1897
                  – 21. 1. 1897?] in Wien}\toendnotes[C]{\smallbreak}
\Standort{CUL, Schnitzler, B 89, A 2.}
\physDesc{Brief, 1 Blatt, 2 Seiten, 333 Zeichen
\newline{}Handschrift: Bleistift, lateinische Kurrent
\newline{}Ordnung: mit Bleistift von unbekannter Hand nummeriert: »85« }\toendnotes[C]{\smallbreak}
\pstart
           \raggedleft{}{\pb}Teplitz\oindex{Teplice@\textbf{Teplice}|pw}, 16/I. 97\pend
           \vspace{0.5em}
\pstart
           Lieber Freund!{ }Heute habe ich alles \label{K_L03263-1v}\edtext{eingeleitet}{\lemma{\textnormal{\emph{eingeleitet}}}\Cendnote{\textnormal{Vgl. XXXX Auszeichnungsfehler: Dokument L03262 nicht gefunden.
               }}}\label{K_L03263-1}. Die Chancen sind meiner Ansicht nach nur gering, obwol man mir das
               Gegentheil zu sagen versucht. Schade, dass Sie sich \label{K_L03263-2v}\edtext{nicht entschließen}{\lemma{\textnormal{\emph{nicht entschließen}}}\Cendnote{\textnormal{Es gibt keine Hinweise, dass Schnitzler ernsthaft überlegte, mit Salten\pwindex{Salten, Felix 6.\,9.\,1869 Budapest – 8.\,10.\,1945 Zürich@\textsc{Salten, Felix} (6.\,9.\,1869 Budapest – 8.\,10.\,1945 Zürich), \emph{Schriftsteller, Journalist, Chefredakteur}|pwk} gemeinsam ein Theater zu führen. Überhaupt dürfte
                   Schnitzler nie wirklich erwogen haben,
                  ein Theater zu leiten.}}}\label{K_L03263-2} können. \uline{Das} wäre die
               absolute Sicherheit. Die {\pb}Stadt\oindex{Teplice@\textbf{Teplice}|pwv} ist reizend und billig.
               Das Theater\orgindex{Stadttheater Teplitz@Stadttheater Teplitz|pwv} prachtvoll.\pend
           
\pstart
           Auf Wiedersehen \label{K_L03263-3v}\edtext{Dienstag}{\lemma{\textnormal{\emph{Dienstag}}}\Cendnote{\textnormal{Vermutlich wollten beide zur Lesung von Max Burckhard\pwindex{Burckhard, Max Eugen 14.\,7.\,1854 Korneuburg – 16.\,3.\,1912 Wien@\textsc{Burckhard, Max Eugen} (14.\,7.\,1854 Korneuburg – 16.\,3.\,1912 Wien), \emph{Schriftsteller, Rechtswissenschaftler, Theaterleiter}|pwk}\eventindex{Österreichischer Ingenieur- und Architektenverein@\textbf{Österreichischer Ingenieur- und Architektenverein}!Lesung von Max Burckhard, 19.1.1897@Lesung von Max Burckhard, 19.1.1897|pwk} im Österreichischen Ingenieur- und Architektenverein\oindex{Wien@\textbf{Wien}!I., Innere Stadt@\textbf{I., Innere Stadt}!Österreichischer Ingenieur- und Architektenverein@\textbf{Österreichischer Ingenieur- und Architektenverein}|pwk}. Burckhard\pwindex{Burckhard, Max Eugen 14.\,7.\,1854 Korneuburg – 16.\,3.\,1912 Wien@\textsc{Burckhard, Max Eugen} (14.\,7.\,1854 Korneuburg – 16.\,3.\,1912 Wien), \emph{Schriftsteller, Rechtswissenschaftler, Theaterleiter}|pwk} las für Mitglieder der \emph{Grillparzer-Gesellschaft}\orgindex{Grillparzer-Gesellschaft@Grillparzer-Gesellschaft|pwk} zwei eigene
                  Erzählungen, \emph{In der Schule des Lebens}\pwindex{Burckhard, Max Eugen 14.\,7.\,1854 Korneuburg – 16.\,3.\,1912 Wien@\textsc{Burckhard, Max Eugen} (14.\,7.\,1854 Korneuburg – 16.\,3.\,1912 Wien), \emph{Schriftsteller, Rechtswissenschaftler, Theaterleiter}!In der Schule des Lebens@\strich\emph{In der Schule des Lebens}|pwk} und
                     \emph{Dulfein}\pwindex{Burckhard, Max Eugen 14.\,7.\,1854 Korneuburg – 16.\,3.\,1912 Wien@\textsc{Burckhard, Max Eugen} (14.\,7.\,1854 Korneuburg – 16.\,3.\,1912 Wien), \emph{Schriftsteller, Rechtswissenschaftler, Theaterleiter}!Dulfein. Ein Liebesmärchen@\strich\emph{Dulfein. Ein Liebesmärchen}|pwk}. Vgl. A. S.: \emph{Tagebuch}, 19. 1. 1897.}}}\label{K_L03263-3}. {\\[\baselineskip]}Herzlich {\\[\baselineskip]}Ihr {\\[\baselineskip]}\spacefill\mbox{Salten}\pend
           \leftskip=0em{}\selectlanguage{ngerman}\endnumbering\briefempfaengerindex{Schnitzler, Arthur@\textsc{Schnitzler, Arthur}!zzzSalten, Felix@\emph{von Felix Salten}!1897-01-164@{16. 1. 1897}|)be}\mylabel{L03263h}  \newcommand{\dateiname}{L03263}\newcommand{\titel}{Felix Salten an Arthur Schnitzler, 16. 1. 1897}\newcommand{\editorInnen}{Martin Anton Müller und Laura Untner}%% latex-leseansicht-abspann.tex
%% Abspann für die Leseansicht.
%% Der Schalter \ifkorrekturansicht ist bereits durch den Vorspann gesetzt.

%% latex-abspann.tex
%% Gemeinsamer Abspann für Korrekturansicht und Leseansicht.
%% Setzt den Schalter \ifkorrekturansicht voraus (gesetzt in den
%% einbindenden Dateien latex-korrekturansicht-abspann.tex bzw.
%% latex-leseansicht-abspann.tex).
%% ---------------------------------------------------------------

\normalsize

% Das esempio-Environment wird nur in der Leseansicht benötigt
\ifkorrekturansicht\else
\newenvironment{esempio}[3]%
{
    \vspace{1.5ex}
    \rlap{\underline{#1}}
    \par
    \setlength{\parindent}{0cm}
    \nopagebreak
    \leftskip=#2cm
    \rightskip=#3cm
}
{
    \par
}
\fi

\doendnotes{C}
\bigskip
\vfill

\clearpage

\footnotesize

\ifkorrekturansicht
  \lohead{\textsc{register}}
\fi

% theindex-Environment neu definieren ohne reledmac
\makeatletter
\renewenvironment{theindex}{%
  \ifkorrekturansicht
    \section*{\indexname}%
  \else
    \subsubsection*{Index der erwähnten Entitäten}%
  \fi
  \setlength{\parindent}{0pt}%
  \setlength{\parskip}{0pt plus 0.3pt}%
  \let\item\@idxitem
}{%
  \ifkorrekturansicht\clearpage\fi
}
\makeatother

\IfFileExists{\jobname-pw.ind}{\input{\jobname-pw.ind}}{}

% Quellenangabe nur in der Leseansicht
\ifkorrekturansicht\else
% Fallback-Definitionen, falls die .tex-Datei \titel etc. nicht gesetzt hat
\providecommand{\titel}{}
\providecommand{\editorInnen}{}
\providecommand{\dateiname}{\jobname}

\vspace{3cm}

\vfill

\footnotesize
\textsc{Quelle}: \titel. Herausgegeben von {\editorInnen}. In: \emph{Arthur Schnitzler: Briefwechsel mit Autorinnen und Autoren}.
 Digitale Edition, https://schnitzler-briefe.acdh.oeaw.ac.at/{\dateiname}.html (Stand \today)
\fi

\end{document}


