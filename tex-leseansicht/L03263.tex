%% latex-korrekturansicht-vorspann.tex
%% Vorspann für die Korrekturansicht.
%% Lädt die gemeinsame Datei latex-vorspann.tex mit gesetztem Schalter.

\newif\ifkorrekturansicht
\korrekturansichttrue

\input{../tex-inputs/latex-vorspann}


\section[ Felix Salten an Arthur Schnitzler, 16. 1. 1897]{L03263 Felix Salten an Arthur Schnitzler, 16. 1. 1897}
\nopagebreak\mylabel{L03263v}
\rehead{ }\normalsize\beginnumbering\briefempfaengerindex{Schnitzler, Arthur@\textsc{Schnitzler, Arthur}!zzzSalten, Felix@\emph{von Felix Salten}!1897-01-164@{16. 1. 1897}|(be}
\toendnotes[C]{\smallbreak\pagebreak[2]}\Standort{CUL, Schnitzler, B 89, A 2.}
\physDesc{Brief, 1 Blatt, 2 Seiten, 333 Zeichen
\newline{}Handschrift: Bleistift, lateinische Kurrent
\newline{}Ordnung: mit Bleistift von unbekannter Hand nummeriert: »85« }\toendnotes[C]{\smallbreak}
\pstart
           \raggedleft{}{\pb}Teplitz\oindex{Teplice@\textbf{Teplice}, \emph{P.PPL}|pw}, 16/I. 97\pend
           \vspace{0.5em}
\pstart
           Lieber Freund!{ }Heute habe ich alles \label{K_L03263-1v}\edtext{eingeleitet}{\lemma{\textnormal{\emph{eingeleitet}}}\Cendnote{\textnormal{Vgl. Felix Salten an Arthur Schnitzler, [10. 1. 1897].
               }}}\label{K_L03263-1}. Die Chancen sind meiner Ansicht nach nur gering, obwol man mir das
               Gegentheil zu sagen versucht. Schade, dass Sie sich \label{K_L03263-2v}\edtext{nicht entschließen}{\lemma{\textnormal{\emph{nicht entschließen}}}\Cendnote{\textnormal{Es gibt keine Hinweise, dass Schnitzler ernsthaft überlegte, mit Salten\pwindex{Salten, Felix 06.09.1869 – 08.10.1945@\textsc{Salten, Felix} (06.09.1869 – 08.10.1945), \emph{Schriftsteller/Schriftstellerin, Journalist/Journalistin, Chefredakteur/Chefredakteurin}|pwk} gemeinsam ein Theater zu führen. Überhaupt dürfte
                   Schnitzler nie wirklich erwogen haben,
                  ein Theater zu leiten.}}}\label{K_L03263-2} können. \uline{Das} wäre die
               absolute Sicherheit. Die {\pb}Stadt\oindex{Teplice@\textbf{Teplice}, \emph{P.PPL}|pwv} ist reizend und billig.
               Das Theater\orgindex{Stadttheater Teplitz@Stadttheater Teplitz|pwv} prachtvoll.\pend
           
\pstart
           Auf Wiedersehen \label{K_L03263-3v}\edtext{Dienstag}{\lemma{\textnormal{\emph{Dienstag}}}\Cendnote{\textnormal{Vermutlich wollten beide zur Lesung von Max Burckhard\pwindex{Burckhard, Max Eugen 14.07.1854 – 16.03.1912@\textsc{Burckhard, Max Eugen} (14.07.1854 – 16.03.1912), \emph{Schriftsteller/Schriftstellerin, Rechtswissenschaftler/Rechtswissenschaftlerin, Theaterleiter/Theaterleiterin}|pwk} im Österreichischen Ingenieur- und Architektenverein\oindex{Oesterreichischer Ingenieur- und Architektenverein@\textbf{Österreichischer Ingenieur- und Architektenverein}, \emph{Vereinslokal (K.VRN)}|pwk}. Burckhard\pwindex{Burckhard, Max Eugen 14.07.1854 – 16.03.1912@\textsc{Burckhard, Max Eugen} (14.07.1854 – 16.03.1912), \emph{Schriftsteller/Schriftstellerin, Rechtswissenschaftler/Rechtswissenschaftlerin, Theaterleiter/Theaterleiterin}|pwk} las für Mitglieder der \emph{Grillparzer-Gesellschaft}\orgindex{Grillparzer-Gesellschaft@Grillparzer-Gesellschaft|pwk} zwei eigene
                  Erzählungen, \emph{In der Schule des Lebens}\pwindex{In der Schule des Lebens@\emph{In der Schule des Lebens}|pwk} und
                     \emph{Dulfein}\pwindex{Dulfein. Ein Liebesmaerchen@\emph{Dulfein. Ein Liebesmärchen}|pwk}. Vgl. A. S.: \emph{Tagebuch}, 19. 1. 1897.}}}\label{K_L03263-3}. {\\[\baselineskip]}Herzlich {\\[\baselineskip]}Ihr {\\[\baselineskip]}\spacefill\mbox{Salten}\pend
           \leftskip=0em{}\selectlanguage{ngerman}\endnumbering\briefempfaengerindex{Schnitzler, Arthur@\textsc{Schnitzler, Arthur}!zzzSalten, Felix@\emph{von Felix Salten}!1897-01-164@{16. 1. 1897}|)be}\mylabel{L03263h}  \normalsize

\doendnotes{C}
\bigskip
\vfill

\clearpage

\footnotesize

\lohead{\textsc{register}}

% Definiere theindex-Environment komplett neu ohne reledmac
\makeatletter
\renewenvironment{theindex}{%
  \section*{\indexname}%
  \setlength{\parindent}{0pt}%
  \setlength{\parskip}{0pt plus 0.3pt}%
  \let\item\@idxitem
}{%
  \clearpage
}
\makeatother

\IfFileExists{\jobname-pw.ind}{\input{\jobname-pw.ind}}{}

\end{document}

      