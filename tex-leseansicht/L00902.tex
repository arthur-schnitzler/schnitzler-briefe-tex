%% latex-leseansicht-vorspann.tex
%% Vorspann für die Leseansicht.
%% Lädt die gemeinsame Datei latex-vorspann.tex mit nicht gesetztem Schalter.

\newif\ifkorrekturansicht
\korrekturansichtfalse

\input{../tex-inputs/latex-vorspann}


\section[Jakob Julius David an Arthur Schnitzler, {{[}}8. 3. 1899?{{]}}]{L00902 Jakob Julius David an Arthur Schnitzler, {[}8. 3. 1899?{]}}
\nopagebreak\mylabel{L00902v}
\rehead{ }\normalsize\beginnumbering\briefempfaengerindex{Schnitzler, Arthur@\textsc{Schnitzler, Arthur}!zzzDavid, Jakob Julius@\emph{von Jakob Julius David}!1899-03-082@{{[}8. 3. 1899?{]}}|(be}
\toendnotes[C]{\smallbreak\pagebreak[2]}
\correspDesc{Versand  durch Jakob Julius David am [8. 3. 1899?] in Wien
\newline{}Erhalt  durch Arthur Schnitzler im Zeitraum [8. 3. 1899
                  – 12. 3. 1899?] in Wien}\toendnotes[C]{\smallbreak}
\Standort{TMW, HS Schn 1/93/1.}
\physDesc{Visitenkarte, 188 Zeichen
\newline{}Handschrift: schwarze Tinte, lateinische Kurrent
\newline{}Ordnung: mit Bleistift von unbekannter Hand nummeriert:
                                    »6a« }\toendnotes[C]{\smallbreak}
\pstart
           \centering{}{\pb}\textcolor{gray}{\textbf{D\textsuperscript{r} J. J. David}}\pend
           
\pstart\center{}{\pb}Verehrter Herr!\pend\vspace{0.5em}
\pstart
           schön Dank. Die Logen \label{K_L00902-1v}\edtext{opponirten auch
                  gestern}{\lemma{\textnormal{\emph{opponirten auch
                  gestern}}}\Cendnote{\textnormal{Die Karte ist undatiert.
                  Sofern sie einen Anschluss an eine erhaltene Kommunikation darstellt, bietet sich
                  der 8. 3. 1899 an. Am Vortag dürfte David\pwindex{David, Jakob Julius 6.\,2.\,1859 Hranice – 20.\,11.\,1906 Wien@\textsc{David, Jakob Julius} (6.\,2.\,1859 Hranice – 20.\,11.\,1906 Wien), \emph{Schriftsteller, Journalist}|pwk} die ihm von Schnitzler
                  verschafften Freikarten für einen neuerlichen Besuch der drei Einakter \emph{Der grüne Kakadu – Paracelsus – Die Gefährtin}\pwindex{Schnitzler, Arthur 15.\,5.\,1862 Wien – 21.\,10.\,1931 ebd.@\textsc{Schnitzler, Arthur} (15.\,5.\,1862 Wien – 21.\,10.\,1931 ebd.), \emph{Schriftsteller, Mediziner}!grüne Kakadu – Paracelsus – Die Gefährtin. Drei Einakter@\strich\emph{Der grüne Kakadu – Paracelsus – Die Gefährtin. Drei Einakter}|pwk}
                     benutzt haben. Bereits in seiner Rezension der Uraufführung\eventindex{Burgtheater@\textbf{Burgtheater}!Uraufführung von Der grüne Kakadu – Paracelsus – Die Gefährtin. Drei Einakter, 1.3.1899@Uraufführung von Der grüne Kakadu – Paracelsus – Die Gefährtin. Drei Einakter, 1.3.1899|pwkv} – \emph{Aus ungleichen Tagen}\pwindex{David, Jakob Julius 6.\,2.\,1859 Hranice – 20.\,11.\,1906 Wien@\textsc{David, Jakob Julius} (6.\,2.\,1859 Hranice – 20.\,11.\,1906 Wien), \emph{Schriftsteller, Journalist}!Aus ungleichen Tagen@\strich\emph{Aus ungleichen Tagen}|pwk} – hatte er von der geteilten Aufnahme
                  durch das Publikum berichtet.}}}\label{K_L00902-1}. So beßer, wenn sie sich daran ärgern.\pend
           
\pstart
           Es wird mich immer freuen, wenn sich Gelegenheit zu einer Aussprache gäbe.\pend
           
\pstart
           Bestens Ihr{\\[\baselineskip]}\spacefill\mbox{David}\pend
           \leftskip=0em{}\selectlanguage{ngerman}\endnumbering\briefempfaengerindex{Schnitzler, Arthur@\textsc{Schnitzler, Arthur}!zzzDavid, Jakob Julius@\emph{von Jakob Julius David}!1899-03-082@{{[}8. 3. 1899?{]}}|)be}\mylabel{L00902h}  \newcommand{\dateiname}{L00902}\newcommand{\titel}{Jakob Julius David an Arthur Schnitzler, [8. 3. 1899?]}\newcommand{\editorInnen}{Martin Anton Müller und Gerd-Hermann Susen}%% latex-leseansicht-abspann.tex
%% Abspann für die Leseansicht.
%% Der Schalter \ifkorrekturansicht ist bereits durch den Vorspann gesetzt.

%% latex-abspann.tex
%% Gemeinsamer Abspann für Korrekturansicht und Leseansicht.
%% Setzt den Schalter \ifkorrekturansicht voraus (gesetzt in den
%% einbindenden Dateien latex-korrekturansicht-abspann.tex bzw.
%% latex-leseansicht-abspann.tex).
%% ---------------------------------------------------------------

\normalsize

% Das esempio-Environment wird nur in der Leseansicht benötigt
\ifkorrekturansicht\else
\newenvironment{esempio}[3]%
{
    \vspace{1.5ex}
    \rlap{\underline{#1}}
    \par
    \setlength{\parindent}{0cm}
    \nopagebreak
    \leftskip=#2cm
    \rightskip=#3cm
}
{
    \par
}
\fi

\doendnotes{C}
\bigskip
\vfill

\clearpage

\footnotesize

\ifkorrekturansicht
  \lohead{\textsc{register}}
\fi

% theindex-Environment neu definieren ohne reledmac
\makeatletter
\renewenvironment{theindex}{%
  \ifkorrekturansicht
    \section*{\indexname}%
  \else
    \subsubsection*{Index der erwähnten Entitäten}%
  \fi
  \setlength{\parindent}{0pt}%
  \setlength{\parskip}{0pt plus 0.3pt}%
  \let\item\@idxitem
}{%
  \ifkorrekturansicht\clearpage\fi
}
\makeatother

\IfFileExists{\jobname-pw.ind}{\input{\jobname-pw.ind}}{}

% Quellenangabe nur in der Leseansicht
\ifkorrekturansicht\else
% Fallback-Definitionen, falls die .tex-Datei \titel etc. nicht gesetzt hat
\providecommand{\titel}{}
\providecommand{\editorInnen}{}
\providecommand{\dateiname}{\jobname}

\vspace{3cm}

\vfill

\footnotesize
\textsc{Quelle}: \titel. Herausgegeben von {\editorInnen}. In: \emph{Arthur Schnitzler: Briefwechsel mit Autorinnen und Autoren}.
 Digitale Edition, https://schnitzler-briefe.acdh.oeaw.ac.at/{\dateiname}.html (Stand \today)
\fi

\end{document}


