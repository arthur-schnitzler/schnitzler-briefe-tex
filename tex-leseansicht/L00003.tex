%% latex-leseansicht-vorspann.tex
%% Vorspann für die Leseansicht.
%% Lädt die gemeinsame Datei latex-vorspann.tex mit nicht gesetztem Schalter.

\newif\ifkorrekturansicht
\korrekturansichtfalse

\input{../tex-inputs/latex-vorspann}


         
         \renewcommand{\erwaehntePersonen}{Personen: Gerhart Hauptmann, Michael Konstantin}
         \renewcommand{\erwaehnteInstitutionen}{Institutionen: Budapester Tagblatt, Moderne Dichtung/Moderne Rundschau}
         \renewcommand{\erwaehnteOrte}{Orte: Brünn, Kärntnerring 12/Bösendorferstraße 11, Ordination Dr. Arthur Schnitzler Giselastraße 11, Výstaviště, Wien, Österreich}
         \renewcommand{\erwaehnteWerke}{Werke: Anatols Hochzeitsmorgen, Budapester Tagblatt, Der Apostel, Die Frage an das Schicksal, Moderne Dichtung. Monatsschrift für Literatur und Kritik}
               \section[Michael Konstantin an Arthur Schnitzler, 22. 5. 1890]{ Michael Konstantin an Arthur Schnitzler, 22. 5. 1890}\nopagebreak\mylabel{v}\rehead{ }\begin{ledgroupsized}[t]{13cm}\normalsize\beginnumbering\briefempfaengerindex{Schnitzler, Arthur@\textsc{Schnitzler, Arthur}!zzzKonstantin, Michael@\emph{von Michael Konstantin}!1890-05-221@{22. 5. 1890}|(be} \toendnotes[C]{\smallbreak\pagebreak[2]} \Standort{DLA, A:Schnitzler, HS.NZ85.1.3750.}
\physDesc{Postkarte, 557 Zeichen
\newline{}Handschrift: schwarze Tinte, deutsche Kurrent
\newline{}Versand: 1) Stempel: »\nobreak{}\oindex{Bruenn@\textbf{Brünn}|pwk}Brünn Bahnhof Brno nádraží, 22 5 90\nobreak{}«.   2) Stempel: »\nobreak{}{[}Wi{]}en, 23 5 90, 8.F\nobreak{}«. 
\newline{}Schnitzler: mit rotem Buntstift zwei Unterstreichungen }\toendnotes[C]{\smallbreak}\pstart{}{\pb}Herrn \textsc{Arthur Schnitzler}\pend{}\pstart{}\textsc{Wien\oindex{Wien@\textbf{Wien}|pw}}\pend{}\pstart{}I Giselastraße 11\oindex{Ordination Dr. Arthur Schnitzler Giselastrasse 11@\textbf{Ordination Dr. Arthur Schnitzler Giselastraße 11}|pw}\pend{}{\bigskip}\pstart
           \raggedleft{}{\pb}Brünn\oindex{Bruenn@\textbf{Brünn}|pw}{ }22/5 1890\pend
           \pstart
           \textcolor{gray}{\textbf{Moderne Dichtung\pwindex{Moderne Dichtung. Monatsschrift fuer Literatur und Kritik1890-01-01 – 1890-12-31@\emph{Moderne Dichtung. Monatsschrift für Literatur und Kritik} {[}1890-01-01 – 1890-12-31{]}|pw}.}}\hfill Herrn \textsc{Arthur Schnitzler}\pend
           \pstart
           \textcolor{gray}{\textbf{Monatsſchrift für Literatur und Kritik.}}\hfill \textsc{\uline{Wien\oindex{Wien@\textbf{Wien}|pw}}}\pend
           \pstart
           \textcolor{gray}{\textbf{Redaction\orgindex{Moderne Dichtung/Moderne Rundschau@Moderne Dichtung/Moderne Rundschau|pwv}.}}\hfill I Giſelaſtraße 11\oindex{Kaerntnerring 12/Boesendorferstrasse 11@\textbf{Kärntnerring 12/Bösendorferstraße 11}|pw}\pend
           \pstart
           \textcolor{gray}{\textbf{Brünn, Schreibwaldſtraße 35\oindex{Výstaviště@\textbf{Výstaviště}|pw}.}}\pend
           \pstart{}Geehrter Herr!\pend\pstart
           Die Handlungsweiſe des \textsc{B. Tgbtt.}\orgindex{Budapester Tagblatt@Budapester Tagblatt|pw} iſt einfach eine \label{K_L00003-1v}\edtext{Gemeinheit}{\lemma{\textnormal{\emph{Gemeinheit}}}\Cendnote{\textnormal{Es dürfte sich um den unerlaubten und
                  korrumpierten Nachdruck von \emph{Die Frage an das
                     Schicksal}\pwindex{Schnitzler, Arthur 15.05.1862 – 21.10.1931@\textsc{Schnitzler, Arthur} (15.05.1862 – 21.10.1931), \emph{Schriftsteller, Mediziner}!Frage an das Schicksal01. 05. 1890@\strich\emph{Die Frage an das Schicksal} {[}01. 05. 1890{]}|pwk} im \emph{Budapester Tageblatt}\pwindex{?? Werk@Nicht ermittelte Verfasserinnen und Verfasser!Budapester Tagblatt1.5.1884 – 2.4.1915@\emph{Budapester Tagblatt} {[}1.5.1884 – 2.4.1915{]}|pwk} vom
                     13. 5. 1890 handeln. Er basiert auf dem Erstdruck in der \emph{Modernen Dichtung}\pwindex{Moderne Dichtung. Monatsschrift fuer Literatur und Kritik1890-01-01 – 1890-12-31@\emph{Moderne Dichtung. Monatsschrift für Literatur und Kritik} {[}1890-01-01 – 1890-12-31{]}|pwk} vom
                  1. 5. 1890.}}}\label{K_L00003-1h}. Ich werde Gelegenheit nehmen der Redaction
               derſelben meine Meinung zu ſagen.\pend
           \pstart
           Die Plauderei »\textsc{Anatols Hochzeitsmorgen\pwindex{Schnitzler, Arthur 15.05.1862 – 21.10.1931@\textsc{Schnitzler, Arthur} (15.05.1862 – 21.10.1931), \emph{Schriftsteller, Mediziner}!Anatols Hochzeitsmorgen01. 07. 1890@\strich\emph{Anatols Hochzeitsmorgen} {[}01. 07. 1890{]}|pw}}« ſenden Sie gefl. baldigſt ein; wenn verwendbar, würde ich dieſelbe gerne im
                  \label{K_L00003-2v}\edtext{Juliheft\pwindex{Moderne Dichtung. Monatsschrift fuer Literatur und Kritik1890-01-01 – 1890-12-31@\emph{Moderne Dichtung. Monatsschrift für Literatur und Kritik} {[}1890-01-01 – 1890-12-31{]}|pwv}}{\lemma{\textnormal{\emph{Juliheft}}}\Cendnote{\textnormal{Am 7. 4. 1890 hatte Michael Konstantin\pwindex{Konstantin, Michael 1855-04-26 – 1911-05-18@\textsc{Konstantin, Michael} (1855-04-26 – 1911-05-18), \emph{Politiker, Administrator}|pwk} an Gerhart Hauptmann\pwindex{Hauptmann, Gerhart 15.11.1862 – 06.06.1946@\textsc{Hauptmann, Gerhart} (15.11.1862 – 06.06.1946), \emph{Schriftsteller}|pwk} geschrieben, »daß wir es uns zur
                     Ehre rechnen würden, Ihnen unser Heft 7 widmen zu dürfen.« Konstantin
                  bat um die Einsendung eines Fotos und einer Novelle; Hauptmann\pwindex{Hauptmann, Gerhart 15.11.1862 – 06.06.1946@\textsc{Hauptmann, Gerhart} (15.11.1862 – 06.06.1946), \emph{Schriftsteller}|pwk}{ }schickte beides, und mit \emph{Der Apostel}\pwindex{Hauptmann, Gerhart 15.11.1862 – 06.06.1946@\textsc{Hauptmann, Gerhart} (15.11.1862 – 06.06.1946), \emph{Schriftsteller}!Apostel01. 07. 1890@\strich\emph{Der Apostel} {[}01. 07. 1890{]}|pwk} begann dann auch das Heft (Gerhart Hauptmann\pwindex{Hauptmann, Gerhart 15.11.1862 – 06.06.1946@\textsc{Hauptmann, Gerhart} (15.11.1862 – 06.06.1946), \emph{Schriftsteller}|pwk}: \emph{Notiz-Kalender. 1889–1891.} Hg. von Martin Machatzke. Frankfurt am
                     Main 1982, S. 237). Auf den Seiten 431–442 findet sich
                  Schnitzlers \emph{Anatols Hochzeitsmorgen}\pwindex{Schnitzler, Arthur 15.05.1862 – 21.10.1931@\textsc{Schnitzler, Arthur} (15.05.1862 – 21.10.1931), \emph{Schriftsteller, Mediziner}!Anatols Hochzeitsmorgen01. 07. 1890@\strich\emph{Anatols Hochzeitsmorgen} {[}01. 07. 1890{]}|pwk}.}}}\label{K_L00003-2h}
               bringen, in welchem vornehmlich Oesterreich\oindex{Oesterreich@\textbf{Österreich}|pw}er
               das Wort führen werden. Ich ſende vom Maiheft\pwindex{Moderne Dichtung. Monatsschrift fuer Literatur und Kritik1890-01-01 – 1890-12-31@\emph{Moderne Dichtung. Monatsschrift für Literatur und Kritik} {[}1890-01-01 – 1890-12-31{]}|pwv} 5 Exempl. als Belegnu{\geminationm}ern an Ihre Adreſſe.\pend
           \pstart
           Hochachtungsvoll{\\[\baselineskip]}\textcolor{gray}{\textbf{\textit{»Moderne Dichtung«}}}\spacefill\mbox{Michael Konstantin.}\pend
           \leftskip=0em{}
         
         \endnumbering\mylabel{h}\end{ledgroupsized}  \newcommand{\dateiname}{L00003}\newcommand{\titel}{Michael Konstantin an Arthur Schnitzler, 22. 5. 1890}\newcommand{\editorInnen}{Martin Anton Müller und Gerd-Hermann Susen}%% latex-leseansicht-abspann.tex
%% Abspann für die Leseansicht.
%% Der Schalter \ifkorrekturansicht ist bereits durch den Vorspann gesetzt.

%% latex-abspann.tex
%% Gemeinsamer Abspann für Korrekturansicht und Leseansicht.
%% Setzt den Schalter \ifkorrekturansicht voraus (gesetzt in den
%% einbindenden Dateien latex-korrekturansicht-abspann.tex bzw.
%% latex-leseansicht-abspann.tex).
%% ---------------------------------------------------------------

\normalsize

% Das esempio-Environment wird nur in der Leseansicht benötigt
\ifkorrekturansicht\else
\newenvironment{esempio}[3]%
{
    \vspace{1.5ex}
    \rlap{\underline{#1}}
    \par
    \setlength{\parindent}{0cm}
    \nopagebreak
    \leftskip=#2cm
    \rightskip=#3cm
}
{
    \par
}
\fi

\doendnotes{C}
\bigskip
\vfill

\clearpage

\footnotesize

\ifkorrekturansicht
  \lohead{\textsc{register}}
\fi

% theindex-Environment neu definieren ohne reledmac
\makeatletter
\renewenvironment{theindex}{%
  \ifkorrekturansicht
    \section*{\indexname}%
  \else
    \subsubsection*{Index der erwähnten Entitäten}%
  \fi
  \setlength{\parindent}{0pt}%
  \setlength{\parskip}{0pt plus 0.3pt}%
  \let\item\@idxitem
}{%
  \ifkorrekturansicht\clearpage\fi
}
\makeatother

\IfFileExists{\jobname-pw.ind}{\input{\jobname-pw.ind}}{}

% Quellenangabe nur in der Leseansicht
\ifkorrekturansicht\else
% Fallback-Definitionen, falls die .tex-Datei \titel etc. nicht gesetzt hat
\providecommand{\titel}{}
\providecommand{\editorInnen}{}
\providecommand{\dateiname}{\jobname}

\vspace{3cm}

\vfill

\footnotesize
\textsc{Quelle}: \titel. Herausgegeben von {\editorInnen}. In: \emph{Arthur Schnitzler: Briefwechsel mit Autorinnen und Autoren}.
 Digitale Edition, https://schnitzler-briefe.acdh.oeaw.ac.at/{\dateiname}.html (Stand \today)
\fi

\end{document}


      