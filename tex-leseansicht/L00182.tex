%% latex-leseansicht-vorspann.tex
%% Vorspann für die Leseansicht.
%% Lädt die gemeinsame Datei latex-vorspann.tex mit nicht gesetztem Schalter.

\newif\ifkorrekturansicht
\korrekturansichtfalse

\input{../tex-inputs/latex-vorspann}


\section[Eduard Michael Kafka an Arthur Schnitzler, 24. 2. 1893]{L00182 Eduard Michael Kafka an Arthur Schnitzler, 24. 2. 1893}
\nopagebreak\mylabel{L00182v}
\rehead{ }\normalsize\beginnumbering\briefempfaengerindex{Schnitzler, Arthur@\textsc{Schnitzler, Arthur}!zzzKafka, Eduard Michael@\emph{von Eduard Michael Kafka}!1893-02-241@{24. 2. 1893}|(be}
\toendnotes[C]{\smallbreak\pagebreak[2]}
\correspDesc{Versand  durch Eduard Michael Kafka am 24. 2. 1893 in Breslau
\newline{}Erhalt  durch Arthur Schnitzler im Zeitraum [25. 2. 1893
                  – 1. 3. 1893?] in Wien}\toendnotes[C]{\smallbreak}
\Standort{DLA, A:Schnitzler, HS.NZ85.1.3604.}
\physDesc{Brief, 2 Blätter, 6 Seiten, 2729 Zeichen
\newline{}Handschrift: schwarze Tinte, deutsche Kurrent
\newline{}Schnitzler: mit rotem Buntstift mehrere Unterstreichungen }\toendnotes[C]{\smallbreak}
\pstart
           \raggedleft{}{\pb}24/II 93.{\\}\textsc{Breslau}\oindex{Breslau@\textbf{Breslau}|pw},{\\}\textsc{Hotel Galisch}\oindex{Hotel Galisch@\textbf{Hotel Galisch}, \emph{Hotel}|pw}.\pend
           
\pstart{}Lieber Schnitzler,\pend\vspace{0.5em}
\pstart
           bitte,{ }ſchreiben Sie mir freundlichſt, was Fels\pwindex{Fels, Friedrich Michael *~1864 Bad Dürkheim@\textsc{Fels, Friedrich Michael} (*~1864 Bad Dürkheim), \emph{Journalist}|pw}
               macht. Iſt er wirklich in Meran\oindex{Meran@\textbf{Meran}, \emph{Hauptstadt}|pw}, wie \textsc{Bahr}\pwindex{Bahr, Hermann 19.\,7.\,1863 Linz – 15.\,1.\,1934 München@\textsc{Bahr, Hermann} (19.\,7.\,1863 Linz – 15.\,1.\,1934 München), \emph{Schriftsteller, Kritiker}|pw} mir erzählte. Ich möchte \substVorne{}\textsuperscript{I}\substDazwischen{}i\substHinten{}hn gerne, wenn’s geht, in den nächſten Tagen beſuchen.\pend
           
\pstart
           Ich traf \textsc{Bahr}\pwindex{Bahr, Hermann 19.\,7.\,1863 Linz – 15.\,1.\,1934 München@\textsc{Bahr, Hermann} (19.\,7.\,1863 Linz – 15.\,1.\,1934 München), \emph{Schriftsteller, Kritiker}|pw} in \textsc{Berlin}\oindex{Berlin@\textbf{Berlin}, \emph{Hauptstadt}|pw}, vor einigen Tagen bei der \textsc{»Gaea\pwindex{Goldschmidt, Adalbert von 5.\,5.\,1848 Wien – 21.\,12.\,1906 ebd.@\textsc{Goldschmidt, Adalbert von} (5.\,5.\,1848 Wien – 21.\,12.\,1906 ebd.), \emph{Schriftsteller, Komponist}!Gaea. Musikdrama@\strich\emph{Gaea. Musikdrama}|pw}«vorlesung}. \textsc{Berti Goldschmidt\pwindex{Goldschmidt, Adalbert von 5.\,5.\,1848 Wien – 21.\,12.\,1906 ebd.@\textsc{Goldschmidt, Adalbert von} (5.\,5.\,1848 Wien – 21.\,12.\,1906 ebd.), \emph{Schriftsteller, Komponist}|pw}} hat dort einen ganz koloſſalen Erfolg damit gehabt. \textsc{Reicher}\pwindex{Reicher, Emanuel 18.\,6.\,1849 Bochnia – 15.\,5.\,1924 Berlin@\textsc{Reicher, Emanuel} (18.\,6.\,1849 Bochnia – 15.\,5.\,1924 Berlin), \emph{Schauspieler}|pw} las aber auch mit einer Meiſterſchaft, die sich in Worten nicht aus{\pb}drücken läßt: er bot eine unglaubliche,
               unübertreffliche Leiſtung, die ihm auf der ganzen Welt keiner nachmachen kann.\pend
           
\pstart
           Ich sprach in \textsc{Berlin}\oindex{Berlin@\textbf{Berlin}, \emph{Hauptstadt}|pw} mit \textsc{Rittner}\pwindex{Rittner, Rudolf 30.\,6.\,1869 Bílý Potok – 4.\,2.\,1943 ebd.@\textsc{Rittner, Rudolf} (30.\,6.\,1869 Bílý Potok – 4.\,2.\,1943 ebd.), \emph{Theaterleiter, Schauspieler}|pw} über die \textsc{Anatol\pwindex{Schnitzler, Arthur 15.\,5.\,1862 Wien – 21.\,10.\,1931 ebd.@\textsc{Schnitzler, Arthur} (15.\,5.\,1862 Wien – 21.\,10.\,1931 ebd.), \emph{Schriftsteller, Mediziner}!Anatol@\strich\emph{Anatol}|pw}}ſachen. Bitte,{ }ſenden Sie ein Ex. an ihn, O.
                     Schillingſtr. 14\textsubscript{II.}\oindex{Schillingstraße@\textbf{Schillingstraße}, \emph{Straße}|pw}, – er wird \label{T_L00182-1v}\edtext{ſich{ }ſicher}{\lemma{\textnormal{\emph{sich sicher}}}\Cendnote{\textnormal{durch Linien umgestellt von »ſicher{ }ſich«}}}\label{T_L00182-1} für die Sachen einſetzen, wenn Sie ihn in einem lieben
               Brief überdies noch recht{ }ſchön darum bitten.\pend
           
\pstart
           Auch an \textsc{Jarno}\pwindex{Jarno, Josef 24.\,8.\,1865 Budapest – 11.\,1.\,1932 Wien@\textsc{Jarno, Josef} (24.\,8.\,1865 Budapest – 11.\,1.\,1932 Wien), \emph{Theaterleiter, Schauspieler}|pw}, bitte,{ }ſchreiben Sie; die beiden jungen Leute können Ihnen {\pb}ganz außerordentlich viel nutzen.\pend
           
\pstart
           Ich bin jetzt mit \textsc{Reicher}\pwindex{Reicher, Emanuel 18.\,6.\,1849 Bochnia – 15.\,5.\,1924 Berlin@\textsc{Reicher, Emanuel} (18.\,6.\,1849 Bochnia – 15.\,5.\,1924 Berlin), \emph{Schauspieler}|pw} für ein paar Tage nach \textsc{Breslau}\oindex{Breslau@\textbf{Breslau}|pw} gefahren: er{ }ſpielt morgen hier den \introOben{}König im\introOben{}{ }\textsc{Talisman}\pwindex{\textcolor{red}{\textsuperscript{XXXX indx1}}!Talisman. Dramatisches Märchen@\strich\emph{Der Talisman. Dramatisches Märchen}|pw} zum erſtenmale: ich bin{ }ſehr geſpannt, was er damit machen wird.\pend
           
\pstart
           An’s Magazin\orgindex{Magazin für die Literatur des Auslandes@Magazin für die Literatur des Auslandes|pw} würde ich Ihnen raten, doch einmal
               ein \textsc{Manuscript} zu{ }ſenden: ich höre doch von verſchiedenen
               Seiten, Sie hätten eine{ }ſo hübſche Novelle\pwindex{Schnitzler, Arthur 15.\,5.\,1862 Wien – 21.\,10.\,1931 ebd.@\textsc{Schnitzler, Arthur} (15.\,5.\,1862 Wien – 21.\,10.\,1931 ebd.), \emph{Schriftsteller, Mediziner}!Sterben. Novelle@\strich\emph{Sterben. Novelle}|pwv} geſchrieben. Auch dem {\pb}\textsc{Berliner Tagblatt}\orgindex{Berliner Tageblatt@Berliner Tageblatt|pw}, wo Sie viele Freunde haben, in erſter Linie \textsc{D\textsuperscript{r}{ }Levysohn\pwindex{Levysohn, Arthur 23.\,3.\,1841 Zielona Góra – 11.\,4.\,1908 Meran@\textsc{Levysohn, Arthur} (23.\,3.\,1841 Zielona Góra – 11.\,4.\,1908 Meran), \emph{Chefredakteur}|pw}}{ }ſelbſt, u \textsc{Neumann Hofer\pwindex{Neumann-Hofer, Gilbert Otto 4.\,2.\,1857 Bol’shiye Berezhki – 14.\,4.\,1941 Detmold@\textsc{Neumann-Hofer, Gilbert Otto} (4.\,2.\,1857 Bol’shiye Berezhki – 14.\,4.\,1941 Detmold), \emph{Kritiker, Theaterleiter}|pw}}, der Sie{ }ſehr{ }ſchätzt, möchte ich doch an Ihrer Stelle einmal eine kleine
               Skizze{ }ſenden.\pend
           
\pstart
           Was iſt denn mit Ihrem neuen Stück\pwindex{Schnitzler, Arthur 15.\,5.\,1862 Wien – 21.\,10.\,1931 ebd.@\textsc{Schnitzler, Arthur} (15.\,5.\,1862 Wien – 21.\,10.\,1931 ebd.), \emph{Schriftsteller, Mediziner}!Familie@\strich\emph{Familie}|pwuv}? Bitte,{ }ſchreiben Sie mir ausführlich über
               dasſelbe. – Sie wiſſen, Sie haben einen aufrichtigen, guten Freund in mir: vielleicht
               kann ich Ihnen irgendwie behilflich{ }ſein: ich bin ja jetzt \textsc{Weltvagabund} im großen Stil, heut da, morgen dort, u. überall doch nur
               gerade in \uline{den} Kreiſen, die Sie brauchen. Alſo!\pend
           
\pstart
           Herzlichſt Ihr{\\[\baselineskip]}\spacefill\mbox{Kafka}\pend
           \leftskip=0em{}
\pstart
           \noindent{}{\pb}\textsc{P.S.}\pend
           
\pstart
           Jetzt habe ich richtig gerade an das vergeſſen, \substVorne{}\textsuperscript{warum}\substDazwischen{}deſſentwegen\substHinten{} ich Ihnen eigentlich{ }ſchreiben wollte.\pend
           
\pstart
           \textsc{Reicher}\pwindex{Reicher, Emanuel 18.\,6.\,1849 Bochnia – 15.\,5.\,1924 Berlin@\textsc{Reicher, Emanuel} (18.\,6.\,1849 Bochnia – 15.\,5.\,1924 Berlin), \emph{Schauspieler}|pw} las geſtern bei einer \textsc{Soiree} hier, welcher ich
                  gleichfalls beiwohnte, Ihre Frage an das
                     Schickſal\pwindex{Schnitzler, Arthur 15.\,5.\,1862 Wien – 21.\,10.\,1931 ebd.@\textsc{Schnitzler, Arthur} (15.\,5.\,1862 Wien – 21.\,10.\,1931 ebd.), \emph{Schriftsteller, Mediziner}!Frage an das Schicksal@\strich\emph{Die Frage an das Schicksal}|pw}. Mit richtigem Beifall. Und natürlich in brillanter Weiſe. \textsc{Reicher}\pwindex{Reicher, Emanuel 18.\,6.\,1849 Bochnia – 15.\,5.\,1924 Berlin@\textsc{Reicher, Emanuel} (18.\,6.\,1849 Bochnia – 15.\,5.\,1924 Berlin), \emph{Schauspieler}|pw} iſt unermüdlich für Ihren Ruhm thätig. Sie{ }ſollten ihm doch wieder mal{ }ſchreiben. {\pb}Daſs er Ihnen nicht i{\geminationm}er antwortet, daraus dürfen Sie sich nichts machen:
                  er hat ja wirklich{ }ſo haarſträubend viel zu thun.\pend
           
\pstart
           Grüßen Sie mir doch freundlichſt unſren lieben \textsc{Loris\pwindex{Hofmannsthal, Hugo von 1.\,2.\,1874 Wien – 15.\,7.\,1929 Rodaun@\textsc{Hofmannsthal, Hugo von} (1.\,2.\,1874 Wien – 15.\,7.\,1929 Rodaun), \emph{Schriftsteller}|pw}} u. die »anderen«. Hat noch i{\geminationm}er keiner Luſt,{ }ſein Bündel zu{ }ſchnüren u. nach Berlin\oindex{Berlin@\textbf{Berlin}, \emph{Hauptstadt}|pw} zu
                  wandern?\pend
           
\pstart
           Wenn ich nur{ }ſchon wüßte, wohin ich von hier hinreiſen{ }ſoll! Nach Hamburg\oindex{Hamburg@\textbf{Hamburg}|pw} oder nach München\oindex{München@\textbf{München}|pw}? Oder{ }ſoll ich zu Holländer\pwindex{Hollaender, Felix 1.\,11.\,1867 Głubczyce – 29.\,5.\,1931 Berlin@\textsc{Hollaender, Felix} (1.\,11.\,1867 Głubczyce – 29.\,5.\,1931 Berlin), \emph{Schriftsteller, Theaterleiter, Regisseur}|pw}, der Sie beſtens \label{T_L00182-2v}\edtext{grüßen läßt}{\lemma{\textnormal{\emph{grüßen läßt}}}\Cendnote{\textnormal{weiter am linken
                     Rand}}}\label{T_L00182-2}, nach Schreiberhau\oindex{Szklarska Poręba@\textbf{Szklarska Poręba}|pw}? Bis zum
                     15. März darf ich mich goldener Freiheit freuen!\pend
           
\pstart
           \spacefill\mbox{EMKafka.}\pend
           
\pstart
           \label{T_L00182-3v}\edtext{Briefe treffen mich am beſten
                  jeweilig durch das \textsc{literarische} Auskunftsbureau \textsc{Clemens Freyer\orgindex{Literarisches Bureau Clemens Freyer@Literarisches Bureau Clemens Freyer|pw}, Berlin, Wilhelmſtr 94/96\oindex{Wilhelmstraße@\textbf{Wilhelmstraße}, \emph{Straße}|pw}}, das mir alles nachſendet.}{\lemma{\textnormal{\emph{Briefe … nachsendet.}}}\Cendnote{\textnormal{auf
                     dem ersten Blatt über Anrede und Datum eingefügt}}}\label{T_L00182-3}\pend
           \selectlanguage{ngerman}\endnumbering\briefempfaengerindex{Schnitzler, Arthur@\textsc{Schnitzler, Arthur}!zzzKafka, Eduard Michael@\emph{von Eduard Michael Kafka}!1893-02-241@{24. 2. 1893}|)be}\mylabel{L00182h}  \newcommand{\dateiname}{L00182}\newcommand{\titel}{Eduard Michael Kafka an Arthur Schnitzler, 24. 2. 1893}\newcommand{\editorInnen}{Martin Anton Müller und Gerd-Hermann Susen}%% latex-leseansicht-abspann.tex
%% Abspann für die Leseansicht.
%% Der Schalter \ifkorrekturansicht ist bereits durch den Vorspann gesetzt.

%% latex-abspann.tex
%% Gemeinsamer Abspann für Korrekturansicht und Leseansicht.
%% Setzt den Schalter \ifkorrekturansicht voraus (gesetzt in den
%% einbindenden Dateien latex-korrekturansicht-abspann.tex bzw.
%% latex-leseansicht-abspann.tex).
%% ---------------------------------------------------------------

\normalsize

% Das esempio-Environment wird nur in der Leseansicht benötigt
\ifkorrekturansicht\else
\newenvironment{esempio}[3]%
{
    \vspace{1.5ex}
    \rlap{\underline{#1}}
    \par
    \setlength{\parindent}{0cm}
    \nopagebreak
    \leftskip=#2cm
    \rightskip=#3cm
}
{
    \par
}
\fi

\doendnotes{C}
\bigskip
\vfill

\clearpage

\footnotesize

\ifkorrekturansicht
  \lohead{\textsc{register}}
\fi

% theindex-Environment neu definieren ohne reledmac
\makeatletter
\renewenvironment{theindex}{%
  \ifkorrekturansicht
    \section*{\indexname}%
  \else
    \subsubsection*{Index der erwähnten Entitäten}%
  \fi
  \setlength{\parindent}{0pt}%
  \setlength{\parskip}{0pt plus 0.3pt}%
  \let\item\@idxitem
}{%
  \ifkorrekturansicht\clearpage\fi
}
\makeatother

\IfFileExists{\jobname-pw.ind}{\input{\jobname-pw.ind}}{}

% Quellenangabe nur in der Leseansicht
\ifkorrekturansicht\else
% Fallback-Definitionen, falls die .tex-Datei \titel etc. nicht gesetzt hat
\providecommand{\titel}{}
\providecommand{\editorInnen}{}
\providecommand{\dateiname}{\jobname}

\vspace{3cm}

\vfill

\footnotesize
\textsc{Quelle}: \titel. Herausgegeben von {\editorInnen}. In: \emph{Arthur Schnitzler: Briefwechsel mit Autorinnen und Autoren}.
 Digitale Edition, https://schnitzler-briefe.acdh.oeaw.ac.at/{\dateiname}.html (Stand \today)
\fi

\end{document}


