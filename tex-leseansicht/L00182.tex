%% latex-leseansicht-vorspann.tex
%% Vorspann für die Leseansicht.
%% Lädt die gemeinsame Datei latex-vorspann.tex mit nicht gesetztem Schalter.

\newif\ifkorrekturansicht
\korrekturansichtfalse

\input{../tex-inputs/latex-vorspann}


               \section[Eduard Michael Kafka an Arthur Schnitzler, 24. 2. 1893]{ Eduard Michael Kafka an Arthur Schnitzler, 24. 2. 1893}\nopagebreak\mylabel{v}\rehead{ }\begin{ledgroupsized}[t]{13cm}\normalsize\beginnumbering\briefempfaengerindex{Schnitzler, Arthur@\textsc{Schnitzler, Arthur}!zzzKafka, Eduard Michael@\emph{von Eduard Michael Kafka}!1893-02-241@{24. 2. 1893}|(be} \toendnotes[C]{\smallbreak\pagebreak[2]} \Standort{DLA, A:Schnitzler, HS.NZ85.1.3604.}
\physDesc{Brief, 2 Blätter, 6 Seiten
\newline{}Handschrift: schwarze Tinte, deutsche Kurrent
\newline{}Schnitzler: mit rotem Buntstift mehrere Unterstreichungen }\toendnotes[C]{\smallbreak}\pstart
           \raggedleft{}{\pb}24/II 93.{\\}\textsc{Breslau}\oindex{Breslau@\textbf{Breslau}|pw},{\\}\textsc{Hotel Galisch}\oindex{Hotel Galisch@\textbf{Hotel Galisch}|pw}.\pend
           \pstart{}Lieber Schnitzler,\pend\pstart
           bitte, ſchreiben Sie mir freundlichſt, was Fels\pwindex{Fels, Friedrich Michael *~1864@\textsc{Fels, Friedrich Michael} (*~1864), \emph{Journalist}|pw}
               macht. Iſt er wirklich in Meran\oindex{Meran@\textbf{Meran}|pw}, wie \textsc{Bahr}\pwindex{Bahr, Hermann 19.07.1863 – 15.01.1934@\textsc{Bahr, Hermann} (19.07.1863 – 15.01.1934), \emph{Schriftsteller, Kritiker}|pw} mir erzählte. Ich möchte \substVorne{}\textsuperscript{I}\substDazwischen{}i\substHinten{}hn gerne, wenn’s geht, in den nächſten Tagen beſuchen.\pend
           \pstart
           Ich traf \textsc{Bahr}\pwindex{Bahr, Hermann 19.07.1863 – 15.01.1934@\textsc{Bahr, Hermann} (19.07.1863 – 15.01.1934), \emph{Schriftsteller, Kritiker}|pw} in \textsc{Berlin}\oindex{Berlin@\textbf{Berlin}|pw}, vor einigen Tagen bei der \textsc{»Gaea\pwindex{Goldschmidt, Adalbert von 05.05.1848 – 21.12.1906@\textsc{Goldschmidt, Adalbert von} (05.05.1848 – 21.12.1906), \emph{Schriftsteller, Komponist}!Gaea1893@\strich\emph{Gaea} {[}1893{]}|pw}«vorlesung}. \textsc{Berti Goldschmidt\pwindex{Goldschmidt, Adalbert von 05.05.1848 – 21.12.1906@\textsc{Goldschmidt, Adalbert von} (05.05.1848 – 21.12.1906), \emph{Schriftsteller, Komponist}|pw}} hat dort einen ganz koloſſalen Erfolg damit gehabt. \textsc{Reicher}\pwindex{Reicher, Emanuel 18.06.1849 – 15.05.1924@\textsc{Reicher, Emanuel} (18.06.1849 – 15.05.1924), \emph{Schauspieler}|pw} las aber auch mit einer Meiſterſchaft, die sich in Worten nicht aus{\pb}drücken läßt: er bot eine unglaubliche,
               unübertreffliche Leiſtung, die ihm auf der ganzen Welt keiner nachmachen kann.\pend
           \pstart
           Ich sprach in \textsc{Berlin}\oindex{Berlin@\textbf{Berlin}|pw} mit \textsc{Rittner}\pwindex{Rittner, Rudolf 30.06.1869 – 04.02.1943@\textsc{Rittner, Rudolf} (30.06.1869 – 04.02.1943), \emph{Theaterleiter, Schauspieler}|pw} über die \textsc{Anatol\pwindex{Schnitzler, Arthur 15.05.1862 – 21.10.1931@\textsc{Schnitzler, Arthur} (15.05.1862 – 21.10.1931), \emph{Schriftsteller, Mediziner}!Anatol1892-10-29 – 1892-10-29@\strich\emph{Anatol} {[}1892-10-29 – 1892-10-29{]}|pw}}ſachen. Bitte, ſenden Sie ein Ex. an ihn, O.
                     Schillingſtr. 14\textsubscript{II.}\oindex{Schillingstrasse@\textbf{Schillingstraße}|pw}, – er wird \label{T_L00182_1v}\edtext{ſich ſicher}{\lemma{\textnormal{\emph{ſich ſicher}}}\Cendnote{\textnormal{durch Linien umgestellt von »ſicher
                     ſich«}}}\label{T_L00182_1h} für die Sachen einſetzen, wenn Sie ihn in einem lieben
               Brief überdies noch recht ſchön darum bitten.\pend
           \pstart
           Auch an \textsc{Jarno}\pwindex{Jarno, Josef 24.08.1865 – 11.01.1932@\textsc{Jarno, Josef} (24.08.1865 – 11.01.1932), \emph{Theaterleiter, Schauspieler}|pw}, bitte, ſchreiben Sie; die beiden jungen Leute können Ihnen {\pb}ganz außerordentlich viel nutzen.\pend
           \pstart
           Ich bin jetzt mit \textsc{Reicher}\pwindex{Reicher, Emanuel 18.06.1849 – 15.05.1924@\textsc{Reicher, Emanuel} (18.06.1849 – 15.05.1924), \emph{Schauspieler}|pw} für ein paar Tage nach \textsc{Breslau}\oindex{Breslau@\textbf{Breslau}|pw} gefahren: er ſpielt morgen hier den \introOben{}König im\introOben{}{ }\textsc{Talisman}\pwindex{\textcolor{red}{\textsuperscript{XXXX1 indx}}!Talisman. Dramatisches Maerchen1892@\strich\emph{Der Talisman. Dramatisches Märchen} {[}1892{]}|pw} zum erſtenmale: ich bin ſehr geſpannt, was er damit machen wird.\pend
           \pstart
           An’s Magazin\orgindex{Magazin fuer die Literatur des Auslandes@Magazin für die Literatur des Auslandes|pw} würde ich Ihnen raten, doch einmal ein
                  \textsc{Manuscript} zu ſenden: ich höre doch von verſchiedenen
               Seiten, Sie hätten eine ſo hübſche Novelle\pwindex{Schnitzler, Arthur 15.05.1862 – 21.10.1931@\textsc{Schnitzler, Arthur} (15.05.1862 – 21.10.1931), \emph{Schriftsteller, Mediziner}!Sterben. Novelle1.10.1894 – 1.12.1894@\strich\emph{Sterben. Novelle} {[}1.10.1894 – 1.12.1894{]}|pwv} geſchrieben. Auch dem {\pb}\textsc{Berliner Tagblatt}\orgindex{Berliner Tageblatt@Berliner Tageblatt|pw}, wo Sie viele Freunde haben, in erſter Linie \textsc{D\textsuperscript{r}{ }Levysohn\pwindex{Levysohn, Arthur 23.3.1841 – 11.4.1908@\textsc{Levysohn, Arthur} (23.3.1841 – 11.4.1908), \emph{Chefredakteur}|pw}}{ }ſelbſt, u \textsc{Neumann Hofer\pwindex{Neumann-Hofer, Gilbert Otto 04.02.1857 – 14.04.1941@\textsc{Neumann-Hofer, Gilbert Otto} (04.02.1857 – 14.04.1941), \emph{Kritiker, Theaterleiter}|pw}}, der Sie ſehr ſchätzt, möchte ich doch an Ihrer Stelle einmal eine kleine
               Skizze ſenden.\pend
           \pstart
           Was iſt denn mit Ihrem neuen Stück\pwindex{Schnitzler, Arthur 15.05.1862 – 21.10.1931@\textsc{Schnitzler, Arthur} (15.05.1862 – 21.10.1931), \emph{Schriftsteller, Mediziner}!Familie1977@\strich\emph{Familie} {[}1977{]}|pwuv}? Bitte, ſchreiben Sie mir ausführlich über
               dasſelbe. – Sie wiſſen, Sie haben einen aufrichtigen, guten Freund in mir: vielleicht
               kann ich Ihnen irgendwie behilflich ſein: ich bin ja jetzt \textsc{Weltvagabund} im großen Stil, heut da, morgen dort, u. überall doch nur
               gerade in \uline{den} Kreiſen, die Sie brauchen. Alſo!\pend
           \pstart
           Herzlichſt Ihr{\\[\baselineskip]}\spacefill\mbox{Kafka}\pend
           \leftskip=0em{}\pstart
           \noindent{}{\pb}\textsc{P.S.}\pend
           \pstart
           Jetzt habe ich richtig gerade an das vergeſſen, \substVorne{}\textsuperscript{warum}\substDazwischen{}deſſentwegen\substHinten{} ich Ihnen eigentlich ſchreiben wollte.\pend
           \pstart
           \textsc{Reicher}\pwindex{Reicher, Emanuel 18.06.1849 – 15.05.1924@\textsc{Reicher, Emanuel} (18.06.1849 – 15.05.1924), \emph{Schauspieler}|pw} las geſtern bei einer \textsc{Soiree} hier, welcher ich
                  gleichfalls beiwohnte, Ihre Frage an das
                     Schickſal\pwindex{Schnitzler, Arthur 15.05.1862 – 21.10.1931@\textsc{Schnitzler, Arthur} (15.05.1862 – 21.10.1931), \emph{Schriftsteller, Mediziner}!Frage an das Schicksal01. 05. 1890@\strich\emph{Die Frage an das Schicksal} {[}01. 05. 1890{]}|pw}. Mit richtigem Beifall. Und natürlich in brillanter Weiſe. \textsc{Reicher}\pwindex{Reicher, Emanuel 18.06.1849 – 15.05.1924@\textsc{Reicher, Emanuel} (18.06.1849 – 15.05.1924), \emph{Schauspieler}|pw} iſt unermüdlich für Ihren Ruhm thätig. Sie ſollten ihm doch wieder mal
                  ſchreiben. {\pb}Daſs er Ihnen nicht i{\geminationm}er antwortet, daraus dürfen Sie sich nichts machen:
                  er hat ja wirklich ſo haarſträubend viel zu thun.\pend
           \pstart
           Grüßen Sie mir doch freundlichſt unſren lieben \textsc{Loris\pwindex{Hofmannsthal, Hugo von 01.02.1874 – 15.07.1929@\textsc{Hofmannsthal, Hugo von} (01.02.1874 – 15.07.1929), \emph{Schriftsteller}|pw}} u. die »anderen«. Hat noch i{\geminationm}er keiner Luſt,
                  ſein Bündel zu ſchnüren u. nach Berlin\oindex{Berlin@\textbf{Berlin}|pw} zu
                  wandern?\pend
           \pstart
           Wenn ich nur ſchon wüßte, wohin ich von hier hinreiſen ſoll! Nach Hamburg\oindex{Hamburg@\textbf{Hamburg}|pw} oder nach München\oindex{Muenchen@\textbf{München}|pw}? Oder ſoll ich zu Holländer\pwindex{Hollaender, Felix 01.11.1867 – 29.05.1931@\textsc{Hollaender, Felix} (01.11.1867 – 29.05.1931), \emph{Schriftsteller, Theaterleiter, Regisseur}|pw},
                  der Sie beſtens \label{T_L00182_2v}\edtext{grüßen läßt}{\lemma{\textnormal{\emph{grüßen läßt}}}\Cendnote{\textnormal{weiter am linken Rand}}}\label{T_L00182_2h}, nach Schreiberhau\oindex{Szklarska Poręba@\textbf{Szklarska Poręba}|pw}? Bis zum 15. März darf
                  ich mich goldener Freiheit freuen!\pend
           \pstart
           \spacefill\mbox{EMKafka.}\pend
           \pstart
           \label{T_L00182_3v}\edtext{Briefe treffen mich am beſten
                  jeweilig durch das \textsc{literarische} Auskunftsbureau \textsc{Clemens Freyer\orgindex{Literarisches Bureau Clemens Freyer@Literarisches Bureau Clemens Freyer|pw}, Berlin, Wilhelmſtr 94/96\oindex{Wilhelmstrasse@\textbf{Wilhelmstraße}|pw}}, das mir alles nachſendet.}{\lemma{\textnormal{\emph{Briefe … nachſendet.}}}\Cendnote{\textnormal{auf
                     dem ersten Blatt über Anrede und Datum eingefügt}}}\label{T_L00182_3h}\pend
                     \endnumbering\briefempfaengerindex{Schnitzler, Arthur@\textsc{Schnitzler, Arthur}!zzzKafka, Eduard Michael@\emph{von Eduard Michael Kafka}!1893-02-241@{24. 2. 1893}|)be}\mylabel{h}\end{ledgroupsized}  \newcommand{\dateiname}{L00182}\newcommand{\titel}{Eduard Michael Kafka an Arthur Schnitzler, 24. 2. 1893}\newcommand{\editorInnen}{Martin Anton Müller und Gerd-Hermann Susen}
            \footnotesize
\begin{ledgroupsized}[t]{11.5cm}
\doendnotes{C}
\end{ledgroupsized}
         %% latex-leseansicht-abspann.tex
%% Abspann für die Leseansicht.
%% Der Schalter \ifkorrekturansicht ist bereits durch den Vorspann gesetzt.

%% latex-abspann.tex
%% Gemeinsamer Abspann für Korrekturansicht und Leseansicht.
%% Setzt den Schalter \ifkorrekturansicht voraus (gesetzt in den
%% einbindenden Dateien latex-korrekturansicht-abspann.tex bzw.
%% latex-leseansicht-abspann.tex).
%% ---------------------------------------------------------------

\normalsize

% Das esempio-Environment wird nur in der Leseansicht benötigt
\ifkorrekturansicht\else
\newenvironment{esempio}[3]%
{
    \vspace{1.5ex}
    \rlap{\underline{#1}}
    \par
    \setlength{\parindent}{0cm}
    \nopagebreak
    \leftskip=#2cm
    \rightskip=#3cm
}
{
    \par
}
\fi

\doendnotes{C}
\bigskip
\vfill

\clearpage

\footnotesize

\ifkorrekturansicht
  \lohead{\textsc{register}}
\fi

% theindex-Environment neu definieren ohne reledmac
\makeatletter
\renewenvironment{theindex}{%
  \ifkorrekturansicht
    \section*{\indexname}%
  \else
    \subsubsection*{Index der erwähnten Entitäten}%
  \fi
  \setlength{\parindent}{0pt}%
  \setlength{\parskip}{0pt plus 0.3pt}%
  \let\item\@idxitem
}{%
  \ifkorrekturansicht\clearpage\fi
}
\makeatother

\IfFileExists{\jobname-pw.ind}{\input{\jobname-pw.ind}}{}

% Quellenangabe nur in der Leseansicht
\ifkorrekturansicht\else
% Fallback-Definitionen, falls die .tex-Datei \titel etc. nicht gesetzt hat
\providecommand{\titel}{}
\providecommand{\editorInnen}{}
\providecommand{\dateiname}{\jobname}

\vspace{3cm}

\vfill

\footnotesize
\textsc{Quelle}: \titel. Herausgegeben von {\editorInnen}. In: \emph{Arthur Schnitzler: Briefwechsel mit Autorinnen und Autoren}.
 Digitale Edition, https://schnitzler-briefe.acdh.oeaw.ac.at/{\dateiname}.html (Stand \today)
\fi

\end{document}


      