%% latex-korrekturansicht-vorspann.tex
%% Vorspann für die Korrekturansicht.
%% Lädt die gemeinsame Datei latex-vorspann.tex mit gesetztem Schalter.

\newif\ifkorrekturansicht
\korrekturansichttrue

\input{../tex-inputs/latex-vorspann}


\section[Eduard Michael Kafka an Arthur Schnitzler, 24. 2. 1893]{L00182 Eduard Michael Kafka an Arthur Schnitzler, 24. 2. 1893}
\nopagebreak\mylabel{L00182v}
\rehead{ }\normalsize\beginnumbering\briefempfaengerindex{Schnitzler, Arthur@\textsc{Schnitzler, Arthur}!zzzKafka, Eduard Michael@\emph{von Eduard Michael Kafka}!1893-02-241@{24. 2. 1893}|(be}
\toendnotes[C]{\smallbreak\pagebreak[2]}\Standort{DLA, A:Schnitzler, HS.NZ85.1.3604.}
\physDesc{Brief, 2 Blätter, 6 Seiten, 2729 Zeichen
\newline{}Handschrift: schwarze Tinte, deutsche Kurrent
\newline{}Schnitzler: mit rotem Buntstift mehrere Unterstreichungen }\toendnotes[C]{\smallbreak}
\pstart
           \raggedleft{}{\pb}24/II 93.{\\}\textsc{Breslau}\oindex{Breslau@\textbf{Breslau}, \emph{P.PPLA}|pw},{\\}\textsc{Hotel Galisch}\oindex{Hotel Galisch@\textbf{Hotel Galisch}, \emph{Hotel (K.HTL)}|pw}.\pend
           
\pstart{}Lieber Schnitzler,\pend\vspace{0.5em}
\pstart
           bitte, ſchreiben Sie mir freundlichſt, was Fels\pwindex{Fels, Friedrich Michael *~1864@\textsc{Fels, Friedrich Michael} (*~1864), \emph{Journalist/Journalistin}|pw}
               macht. Iſt er wirklich in Meran\oindex{Meran@\textbf{Meran}, \emph{P.PPLA3}|pw}, wie \textsc{Bahr}\pwindex{Bahr, Hermann 19.07.1863 – 15.01.1934@\textsc{Bahr, Hermann} (19.07.1863 – 15.01.1934), \emph{Schriftsteller/Schriftstellerin, Kritiker/Kritikerin}|pw} mir erzählte. Ich möchte \substVorne{}\textsuperscript{I}\substDazwischen{}i\substHinten{}hn gerne, wenn’s geht, in den nächſten Tagen beſuchen.\pend
           
\pstart
           Ich traf \textsc{Bahr}\pwindex{Bahr, Hermann 19.07.1863 – 15.01.1934@\textsc{Bahr, Hermann} (19.07.1863 – 15.01.1934), \emph{Schriftsteller/Schriftstellerin, Kritiker/Kritikerin}|pw} in \textsc{Berlin}\oindex{Berlin@\textbf{Berlin}, \emph{P.PPLC}|pw}, vor einigen Tagen bei der \textsc{»Gaea\pwindex{Gaea. Musikdrama@\emph{Gaea. Musikdrama}|pw}«vorlesung}. \textsc{Berti Goldschmidt\pwindex{Goldschmidt, Adalbert von 1848-05-05 – 1906-12-21@\textsc{Goldschmidt, Adalbert von} (1848-05-05 – 1906-12-21), \emph{Schriftsteller/Schriftstellerin, Komponist/Komponistin}|pw}} hat dort einen ganz koloſſalen Erfolg damit gehabt. \textsc{Reicher}\pwindex{Reicher, Emanuel 18.06.1849 – 15.05.1924@\textsc{Reicher, Emanuel} (18.06.1849 – 15.05.1924), \emph{Schauspieler/Schauspielerin}|pw} las aber auch mit einer Meiſterſchaft, die sich in Worten nicht aus{\pb}drücken läßt: er bot eine unglaubliche,
               unübertreffliche Leiſtung, die ihm auf der ganzen Welt keiner nachmachen kann.\pend
           
\pstart
           Ich sprach in \textsc{Berlin}\oindex{Berlin@\textbf{Berlin}, \emph{P.PPLC}|pw} mit \textsc{Rittner}\pwindex{Rittner, Rudolf 30.06.1869 – 04.02.1943@\textsc{Rittner, Rudolf} (30.06.1869 – 04.02.1943), \emph{Theaterleiter/Theaterleiterin, Schauspieler/Schauspielerin}|pw} über die \textsc{Anatol\pwindex{Anatol@\emph{Anatol}|pw}}ſachen. Bitte, ſenden Sie ein Ex. an ihn, O.
                     Schillingſtr. 14\textsubscript{II.}\oindex{Schillingstrasse@\textbf{Schillingstraße}, \emph{Straße (K.STR)}|pw}, – er wird \label{T_L00182-1v}\edtext{ſich ſicher}{\lemma{\textnormal{\emph{ſich ſicher}}}\Cendnote{\textnormal{durch Linien umgestellt von »ſicher
                     ſich«}}}\label{T_L00182-1} für die Sachen einſetzen, wenn Sie ihn in einem lieben
               Brief überdies noch recht ſchön darum bitten.\pend
           
\pstart
           Auch an \textsc{Jarno}\pwindex{Jarno, Josef 24.08.1865 – 11.01.1932@\textsc{Jarno, Josef} (24.08.1865 – 11.01.1932), \emph{Theaterleiter/Theaterleiterin, Schauspieler/Schauspielerin}|pw}, bitte, ſchreiben Sie; die beiden jungen Leute können Ihnen {\pb}ganz außerordentlich viel nutzen.\pend
           
\pstart
           Ich bin jetzt mit \textsc{Reicher}\pwindex{Reicher, Emanuel 18.06.1849 – 15.05.1924@\textsc{Reicher, Emanuel} (18.06.1849 – 15.05.1924), \emph{Schauspieler/Schauspielerin}|pw} für ein paar Tage nach \textsc{Breslau}\oindex{Breslau@\textbf{Breslau}, \emph{P.PPLA}|pw} gefahren: er ſpielt morgen hier den \introOben{}König im\introOben{}{ }\textsc{Talisman}\pwindex{Talisman. Dramatisches Maerchen@\emph{Der Talisman. Dramatisches Märchen}|pw} zum erſtenmale: ich bin ſehr geſpannt, was er damit machen wird.\pend
           
\pstart
           An’s Magazin\orgindex{Magazin fuer die Literatur des Auslandes@Magazin für die Literatur des Auslandes|pw} würde ich Ihnen raten, doch einmal
               ein \textsc{Manuscript} zu ſenden: ich höre doch von verſchiedenen
               Seiten, Sie hätten eine ſo hübſche Novelle\pwindex{Sterben. Novelle@\emph{Sterben. Novelle}|pwv} geſchrieben. Auch dem {\pb}\textsc{Berliner Tagblatt}\orgindex{Berliner Tageblatt@Berliner Tageblatt|pw}, wo Sie viele Freunde haben, in erſter Linie \textsc{D\textsuperscript{r}{ }Levysohn\pwindex{Levysohn, Arthur 23.3.1841 – 11.4.1908@\textsc{Levysohn, Arthur} (23.3.1841 – 11.4.1908), \emph{Chefredakteur/Chefredakteurin}|pw}}{ }ſelbſt, u \textsc{Neumann Hofer\pwindex{Neumann-Hofer, Gilbert Otto 04.02.1857 – 14.04.1941@\textsc{Neumann-Hofer, Gilbert Otto} (04.02.1857 – 14.04.1941), \emph{Kritiker/Kritikerin, Theaterleiter/Theaterleiterin}|pw}}, der Sie ſehr ſchätzt, möchte ich doch an Ihrer Stelle einmal eine kleine
               Skizze ſenden.\pend
           
\pstart
           Was iſt denn mit Ihrem neuen Stück\pwindex{Familie@\emph{Familie}|pwuv}? Bitte, ſchreiben Sie mir ausführlich über
               dasſelbe. – Sie wiſſen, Sie haben einen aufrichtigen, guten Freund in mir: vielleicht
               kann ich Ihnen irgendwie behilflich ſein: ich bin ja jetzt \textsc{Weltvagabund} im großen Stil, heut da, morgen dort, u. überall doch nur
               gerade in \uline{den} Kreiſen, die Sie brauchen. Alſo!\pend
           
\pstart
           Herzlichſt Ihr{\\[\baselineskip]}\spacefill\mbox{Kafka}\pend
           \leftskip=0em{}
\pstart
           \noindent{}{\pb}\textsc{P.S.}\pend
           
\pstart
           Jetzt habe ich richtig gerade an das vergeſſen, \substVorne{}\textsuperscript{warum}\substDazwischen{}deſſentwegen\substHinten{} ich Ihnen eigentlich ſchreiben wollte.\pend
           
\pstart
           \textsc{Reicher}\pwindex{Reicher, Emanuel 18.06.1849 – 15.05.1924@\textsc{Reicher, Emanuel} (18.06.1849 – 15.05.1924), \emph{Schauspieler/Schauspielerin}|pw} las geſtern bei einer \textsc{Soiree} hier, welcher ich
                  gleichfalls beiwohnte, Ihre Frage an das
                     Schickſal\pwindex{Frage an das Schicksal@\emph{Die Frage an das Schicksal}|pw}. Mit richtigem Beifall. Und natürlich in brillanter Weiſe. \textsc{Reicher}\pwindex{Reicher, Emanuel 18.06.1849 – 15.05.1924@\textsc{Reicher, Emanuel} (18.06.1849 – 15.05.1924), \emph{Schauspieler/Schauspielerin}|pw} iſt unermüdlich für Ihren Ruhm thätig. Sie ſollten ihm doch wieder mal
                  ſchreiben. {\pb}Daſs er Ihnen nicht i{\geminationm}er antwortet, daraus dürfen Sie sich nichts machen:
                  er hat ja wirklich ſo haarſträubend viel zu thun.\pend
           
\pstart
           Grüßen Sie mir doch freundlichſt unſren lieben \textsc{Loris\pwindex{Hofmannsthal, Hugo von 1874-02-01 – 1929-07-15@\textsc{Hofmannsthal, Hugo von} (1874-02-01 – 1929-07-15), \emph{Schriftsteller/Schriftstellerin}|pw}} u. die »anderen«. Hat noch i{\geminationm}er keiner Luſt,
                  ſein Bündel zu ſchnüren u. nach Berlin\oindex{Berlin@\textbf{Berlin}, \emph{P.PPLC}|pw} zu
                  wandern?\pend
           
\pstart
           Wenn ich nur ſchon wüßte, wohin ich von hier hinreiſen ſoll! Nach Hamburg\oindex{Hamburg@\textbf{Hamburg}, \emph{P.PPLA}|pw} oder nach München\oindex{Muenchen@\textbf{München}, \emph{P.PPLA}|pw}? Oder ſoll ich zu Holländer\pwindex{Hollaender, Felix 01.11.1867 – 29.05.1931@\textsc{Hollaender, Felix} (01.11.1867 – 29.05.1931), \emph{Schriftsteller/Schriftstellerin, Theaterleiter/Theaterleiterin, Regisseur/Regisseurin}|pw}, der Sie beſtens \label{T_L00182-2v}\edtext{grüßen läßt}{\lemma{\textnormal{\emph{grüßen läßt}}}\Cendnote{\textnormal{weiter am linken
                     Rand}}}\label{T_L00182-2}, nach Schreiberhau\oindex{Szklarska Poręba@\textbf{Szklarska Poręba}, \emph{P.PPL}|pw}? Bis zum
                     15. März darf ich mich goldener Freiheit freuen!\pend
           
\pstart
           \spacefill\mbox{EMKafka.}\pend
           
\pstart
           \label{T_L00182-3v}\edtext{Briefe treffen mich am beſten
                  jeweilig durch das \textsc{literarische} Auskunftsbureau \textsc{Clemens Freyer\orgindex{Literarisches Bureau Clemens Freyer@Literarisches Bureau Clemens Freyer|pw}, Berlin, Wilhelmſtr 94/96\oindex{Wilhelmstrasse@\textbf{Wilhelmstraße}, \emph{Straße (K.STR)}|pw}}, das mir alles nachſendet.}{\lemma{\textnormal{\emph{Briefe … nachſendet.}}}\Cendnote{\textnormal{auf
                     dem ersten Blatt über Anrede und Datum eingefügt}}}\label{T_L00182-3}\pend
           \selectlanguage{ngerman}\endnumbering\briefempfaengerindex{Schnitzler, Arthur@\textsc{Schnitzler, Arthur}!zzzKafka, Eduard Michael@\emph{von Eduard Michael Kafka}!1893-02-241@{24. 2. 1893}|)be}\mylabel{L00182h}  \normalsize

\doendnotes{C}
\bigskip
\vfill

\clearpage

\footnotesize

\lohead{\textsc{register}}

% Definiere theindex-Environment komplett neu ohne reledmac
\makeatletter
\renewenvironment{theindex}{%
  \section*{\indexname}%
  \setlength{\parindent}{0pt}%
  \setlength{\parskip}{0pt plus 0.3pt}%
  \let\item\@idxitem
}{%
  \clearpage
}
\makeatother

\IfFileExists{\jobname-pw.ind}{\input{\jobname-pw.ind}}{}

\end{document}

      