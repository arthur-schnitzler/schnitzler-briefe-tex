%% latex-korrekturansicht-vorspann.tex
%% Vorspann für die Korrekturansicht.
%% Lädt die gemeinsame Datei latex-vorspann.tex mit gesetztem Schalter.

\newif\ifkorrekturansicht
\korrekturansichttrue

\input{../tex-inputs/latex-vorspann}


\section[Arthur Schnitzler an Gerty Hofmannsthal, 17. 2. 1931]{L02543 Arthur Schnitzler an Gerty Hofmannsthal, 17. 2. 1931}
\nopagebreak\mylabel{L02543v}
\rehead{ }\normalsize\beginnumbering\briefempfaengerindex{Hofmannsthal, Gertrude von@\textsc{Hofmannsthal, Gertrude von}!zzzSchnitzler, Arthur@\emph{von Arthur Schnitzler}!1931-02-171@{17. 2. 1931}|(be}
\toendnotes[C]{\smallbreak\pagebreak[2]}\Standort{FDH, Hs-31346,4.}
\physDesc{Postkarte, 291 Zeichen
\newline{}Handschrift: schwarze Tinte, lateinische Kurrent
\newline{}Versand: Stempel: »\nobreak{}Wien 68, \textcolor{gray}{1}7. II. {[}1931{]}\nobreak{}«.  }\toendnotes[C]{\smallbreak}\pstart{}{\pb}\label{T_L02543-1v}\edtext{\textcolor{gray}{\textbf{A. S.}}}{\lemma{\textnormal{\emph{A. S.}}}\Cendnote{\textnormal{ovaler Absenderkleber}}}\label{T_L02543-1}\pend{}\pstart{}\textcolor{gray}{\textbf{WIEN, XVIII.}}\oindex{XVIII., Waehring@\textbf{XVIII., Währing}, \emph{A.ADM3}|pw}\pend{}\pstart{}\textcolor{gray}{\textbf{STERNWARTESTR. 71}}\oindex{Sternwartestrasse 71@\textbf{Sternwartestraße 71}, \emph{Wohngebäude (K.WHS)}|pw}\pend{}{\bigskip}\pstart{}Frau Gerty von Hofmannsthal\pend{}\pstart{}Wien IV\oindex{IV., Wieden@\textbf{IV., Wieden}, \emph{A.ADM3}|pw}\pend{}\pstart{}Mozartgasse 4\oindex{Mozartgasse@\textbf{Mozartgasse}, \emph{Straße (K.STR)}|pw}\pend{}{\bigskip}\vspace{1em}
\pstart
           \raggedleft{}{\pb}Wien\oindex{Wien@\textbf{Wien}, \emph{A.ADM2}|pw}{ }\label{K_L02543-1v}\edtext{18/2 931}{\lemma{\textnormal{\emph{18/2 931}}}\Cendnote{\textnormal{Beide Rollstempel weisen
                        unzweifelhaft eine »7« aus, sodass Schnitzler falsch datiert haben dürfte.}}}\label{K_L02543-1}\pend
           \vspace{0.5em}
\pstart
           liebe Gerty, ich danke Ihnen sehr un\textcolor{gray}{d} hoffe Sie
               baldigst zu sehen. Sie haben mir Ihre Tel. Nummer nicht gesagt, hier, zur Revanche
               die meine: \textsc{\label{K_L02543-2v}\edtext{A 10.0.81}{\lemma{\textnormal{\emph{A 10.0.81}}}\Cendnote{\textnormal{Es handelt sich um eine Geheimnummer. Im offiziellen
                     Adressbuch steht Schnitzler bis zum Tod
                     mit der Nummer »A-14.432«.}}}\label{K_L02543-2}}. Bitte rufen Sie mich an, damit wir was ausmachen können.\pend
           
\pstart
           Alles herzliche{\\[\baselineskip]}Ihr \spacefill\mbox{Arthur.}\pend
           \leftskip=0em{}\selectlanguage{ngerman}\endnumbering\briefempfaengerindex{Hofmannsthal, Gertrude von@\textsc{Hofmannsthal, Gertrude von}!zzzSchnitzler, Arthur@\emph{von Arthur Schnitzler}!1931-02-171@{17. 2. 1931}|)be}\mylabel{L02543h}  \normalsize

\doendnotes{C}
\bigskip
\vfill

\clearpage

\footnotesize

\lohead{\textsc{register}}

% Definiere theindex-Environment komplett neu ohne reledmac
\makeatletter
\renewenvironment{theindex}{%
  \section*{\indexname}%
  \setlength{\parindent}{0pt}%
  \setlength{\parskip}{0pt plus 0.3pt}%
  \let\item\@idxitem
}{%
  \clearpage
}
\makeatother

\IfFileExists{\jobname-pw.ind}{\input{\jobname-pw.ind}}{}

\end{document}

      