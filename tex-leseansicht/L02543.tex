%% latex-leseansicht-vorspann.tex
%% Vorspann für die Leseansicht.
%% Lädt die gemeinsame Datei latex-vorspann.tex mit nicht gesetztem Schalter.

\newif\ifkorrekturansicht
\korrekturansichtfalse

\input{../tex-inputs/latex-vorspann}


\section[Arthur Schnitzler an Gerty Hofmannsthal, 17. 2. 1931]{L02543 Arthur Schnitzler an Gerty Hofmannsthal, 17. 2. 1931}
\nopagebreak\mylabel{L02543v}
\rehead{ }\normalsize\beginnumbering\briefempfaengerindex{Hofmannsthal, Gertrude von@\textsc{Hofmannsthal, Gertrude von}!zzzSchnitzler, Arthur@\emph{von Arthur Schnitzler}!1931-02-171@{17. 2. 1931}|(be}
\toendnotes[C]{\smallbreak\pagebreak[2]}
\correspDesc{Versand  durch Arthur Schnitzler am 17. 2. 1931 in Wien
\newline{}Erhalt  durch Gerty Hofmannsthal im Zeitraum [17. 2. 1931
                  – 21. 2. 1931?] in Wien}\toendnotes[C]{\smallbreak}
\Standort{FDH, Hs-31346,4.}
\physDesc{Postkarte, 291 Zeichen
\newline{}Handschrift: schwarze Tinte, lateinische Kurrent
\newline{}Versand: Stempel: »\nobreak{}\oindex{Wien@\textbf{Wien}, \emph{Verwaltungsgebiet}|pwk}Wien 68, \textcolor{gray}{1}7. II. {[}1931{]}\nobreak{}«.  }\toendnotes[C]{\smallbreak}\pstart{}{\pb}\label{T_L02543-1v}\edtext{\textcolor{gray}{\textbf{A. S.}}}{\lemma{\textnormal{\emph{A. S.}}}\Cendnote{\textnormal{ovaler Absenderkleber}}}\label{T_L02543-1}\pend{}\pstart{}\textcolor{gray}{\textbf{WIEN, XVIII.}}\oindex{XVIII., Währing@\textbf{XVIII., Währing}, \emph{Verwaltungsgebiet}|pw}\pend{}\pstart{}\textcolor{gray}{\textbf{STERNWARTESTR. 71}}\oindex{Wien@\textbf{Wien}!XVIII., Währing@\textbf{XVIII., Währing}!Sternwartestraße 71@\textbf{Sternwartestraße 71}, \emph{Wohngebäude}|pw}\pend{}{\bigskip}\pstart{}Frau Gerty von Hofmannsthal\pend{}\pstart{}Wien IV\oindex{IV., Wieden@\textbf{IV., Wieden}, \emph{Verwaltungsgebiet}|pw}\pend{}\pstart{}Mozartgasse 4\oindex{Wien@\textbf{Wien}!IV., Wieden@\textbf{IV., Wieden}!Mozartgasse@\textbf{Mozartgasse}, \emph{Straße}|pw}\pend{}{\bigskip}\vspace{1em}
\pstart
           \raggedleft{}{\pb}Wien\oindex{Wien@\textbf{Wien}, \emph{Verwaltungsgebiet}|pw}{ }\label{K_L02543-1v}\edtext{18/2 931}{\lemma{\textnormal{\emph{18/2 931}}}\Cendnote{\textnormal{Beide Rollstempel weisen
                        unzweifelhaft eine »7« aus, sodass Schnitzler falsch datiert haben dürfte.}}}\label{K_L02543-1}\pend
           \vspace{0.5em}
\pstart
           liebe Gerty, ich danke Ihnen sehr un\textcolor{gray}{d} hoffe Sie
               baldigst zu sehen. Sie haben mir Ihre Tel. Nummer nicht gesagt, hier, zur Revanche
               die meine: \textsc{\label{K_L02543-2v}\edtext{A 10.0.81}{\lemma{\textnormal{\emph{A 10.0.81}}}\Cendnote{\textnormal{Es handelt sich um eine Geheimnummer. Im offiziellen
                     Adressbuch steht Schnitzler bis zum Tod
                     mit der Nummer »A-14.432«.}}}\label{K_L02543-2}}. Bitte rufen Sie mich an, damit wir was ausmachen können.\pend
           
\pstart
           Alles herzliche{\\[\baselineskip]}Ihr \spacefill\mbox{Arthur.}\pend
           \leftskip=0em{}\selectlanguage{ngerman}\endnumbering\briefempfaengerindex{Hofmannsthal, Gertrude von@\textsc{Hofmannsthal, Gertrude von}!zzzSchnitzler, Arthur@\emph{von Arthur Schnitzler}!1931-02-171@{17. 2. 1931}|)be}\mylabel{L02543h}  \newcommand{\dateiname}{L02543}\newcommand{\titel}{Arthur Schnitzler an Gerty Hofmannsthal, 17. 2. 1931}\newcommand{\editorInnen}{Martin Anton Müller und Gerd-Hermann Susen}%% latex-leseansicht-abspann.tex
%% Abspann für die Leseansicht.
%% Der Schalter \ifkorrekturansicht ist bereits durch den Vorspann gesetzt.

%% latex-abspann.tex
%% Gemeinsamer Abspann für Korrekturansicht und Leseansicht.
%% Setzt den Schalter \ifkorrekturansicht voraus (gesetzt in den
%% einbindenden Dateien latex-korrekturansicht-abspann.tex bzw.
%% latex-leseansicht-abspann.tex).
%% ---------------------------------------------------------------

\normalsize

% Das esempio-Environment wird nur in der Leseansicht benötigt
\ifkorrekturansicht\else
\newenvironment{esempio}[3]%
{
    \vspace{1.5ex}
    \rlap{\underline{#1}}
    \par
    \setlength{\parindent}{0cm}
    \nopagebreak
    \leftskip=#2cm
    \rightskip=#3cm
}
{
    \par
}
\fi

\doendnotes{C}
\bigskip
\vfill

\clearpage

\footnotesize

\ifkorrekturansicht
  \lohead{\textsc{register}}
\fi

% theindex-Environment neu definieren ohne reledmac
\makeatletter
\renewenvironment{theindex}{%
  \ifkorrekturansicht
    \section*{\indexname}%
  \else
    \subsubsection*{Index der erwähnten Entitäten}%
  \fi
  \setlength{\parindent}{0pt}%
  \setlength{\parskip}{0pt plus 0.3pt}%
  \let\item\@idxitem
}{%
  \ifkorrekturansicht\clearpage\fi
}
\makeatother

\IfFileExists{\jobname-pw.ind}{\input{\jobname-pw.ind}}{}

% Quellenangabe nur in der Leseansicht
\ifkorrekturansicht\else
% Fallback-Definitionen, falls die .tex-Datei \titel etc. nicht gesetzt hat
\providecommand{\titel}{}
\providecommand{\editorInnen}{}
\providecommand{\dateiname}{\jobname}

\vspace{3cm}

\vfill

\footnotesize
\textsc{Quelle}: \titel. Herausgegeben von {\editorInnen}. In: \emph{Arthur Schnitzler: Briefwechsel mit Autorinnen und Autoren}.
 Digitale Edition, https://schnitzler-briefe.acdh.oeaw.ac.at/{\dateiname}.html (Stand \today)
\fi

\end{document}


