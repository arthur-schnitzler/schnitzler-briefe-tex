%% latex-leseansicht-vorspann.tex
%% Vorspann für die Leseansicht.
%% Lädt die gemeinsame Datei latex-vorspann.tex mit nicht gesetztem Schalter.

\newif\ifkorrekturansicht
\korrekturansichtfalse

\input{../tex-inputs/latex-vorspann}

\begin{center}
            \textcolor{red}{ENTWURF, NICHT FERTIG KORRIGIERT}
                      \end{center}
            
         
         \renewcommand{\erwaehntePersonen}{Personen:  ?? [Eigentümer der Cottagegasse 37],  ?? [Haushaltshilfe der Familie Salten 1912], Samuel Fischer, Hedwig Fischer, Richard Kralik, Ottilie Salten, Olga Schnitzler, Elisabeth Steinrück, Julius Ferdinand Wollf, Johanna Sophie Wollf}
         \renewcommand{\erwaehnteInstitutionen}{Institutionen: Burgtheater}
         \renewcommand{\erwaehnteOrte}{Orte: Berghof, Dresden, München, Sanatorium Hera, Unterach am Attersee, Wien}
         \renewcommand{\erwaehnteWerke}{}
               \section[Felix Salten an Olga Schnitzler, 2. 9. 1912]{ Felix Salten an Olga Schnitzler, 2. 9. 1912}\nopagebreak\mylabel{v}\rehead{ }\begin{ledgroupsized}[t]{13cm}\normalsize\beginnumbering \toendnotes[C]{\smallbreak\pagebreak[2]} \Standort{CUL, Schnitzler, B 89, B 2.}
\physDesc{Briefkarte, 1873 Zeichen
\newline{}Handschrift: schwarze Tinte, lateinische Kurrent
\newline{}Ordnung: mit Bleistift von unbekannter Hand nummeriert: »274« }\toendnotes[C]{\smallbreak}\pstart
           \raggedleft{}{\pb}Berghof\oindex{Berghof@\textbf{Berghof}|pw}, 2. IX. 12. \pend
           \pstart{}Verehrte, liebe Frau Olga,\pend\pstart
           vielen Dank für den lieben Brief und für Arthur\pwindex{Schnitzler, Arthur 15.05.1862 – 21.10.1931@\textsc{Schnitzler, Arthur} (15.05.1862 – 21.10.1931), \emph{Schriftsteller, Mediziner}|pw}s Karten. Wir haben eine ziemlich unruhige Zeit noch nicht ganz hinter
               uns. Wollfs\pwindex{Wollf, Julius Ferdinand 22.05.1871 – 01.03.1942@\textsc{Wollf, Julius Ferdinand} (22.05.1871 – 01.03.1942), \emph{Journalist, Herausgeber, Verleger}|pw}\pwindex{Wollf, Johanna Sophie 1877-10-18 – 1942-02-27@\textsc{Wollf, Johanna Sophie} (1877-10-18 – 1942-02-27)|pw} aus Dresden\oindex{Dresden@\textbf{Dresden}|pw} sind drei Wochen lang bei uns gewesen und wir haben uns
               sehr mit Ihnen gefreut. Wir konnten uns deshalb zu keinem ganzen Behagen kommen, weil
               es fast unaufhörlich geregnet hat, und weil Otti\pwindex{Salten, Ottilie 07.03.1868 – 22.06.1942@\textsc{Salten, Ottilie} (07.03.1868 – 22.06.1942), \emph{Schauspielerin}|pw} mit ihrer Gesundheit nicht ganz in Ordnung war. Nun ist sie seit
                  Mittwoch in Wien\oindex{Wien@\textbf{Wien}|pw}, im Sanatorium
                  »Hera\oindex{Sanatorium Hera@\textbf{Sanatorium Hera}|pw}« und hat am Donnerstag eine kleine
               Operation überstanden. Es ist alles sehr gut gegangen: sie befindet sich schon viel
               besser und es ist möglich, dass Sie übermorgen oder Donnerstag schon wieder hier sein
               wird. Bei alledem – angenehm ist sowas ja nie, weder für Otti\pwindex{Salten, Ottilie 07.03.1868 – 22.06.1942@\textsc{Salten, Ottilie} (07.03.1868 – 22.06.1942), \emph{Schauspielerin}|pw}, die allein, nur vom Stubenmädchen\pwindex{?? [Haushaltshilfe der Familie Salten 1912] @\textsc{?? [Haushaltshilfe der Familie Salten 1912]}|pwv} begleitet, in Wien\oindex{Wien@\textbf{Wien}|pw}
               sein muß, noch für mich, der hier nur warten und sonst nichts nützliches für sie tun
               kann. Vielleicht haben wir hier noch ein paar Wochen Zeit, dass Otti\pwindex{Salten, Ottilie 07.03.1868 – 22.06.1942@\textsc{Salten, Ottilie} (07.03.1868 – 22.06.1942), \emph{Schauspielerin}|pw}{ }{\pb}sich erholen kann. Ohnehin
               graut uns ein bischen vor dem \label{K_L03560-1v}\edtext{Umzug}{\lemma{\textnormal{\emph{Umzug}}}\Cendnote{\textnormal{nicht eruiert. Eventuell handelte es
                  sich um eine größere Wohnung im selben Haus?}}}\label{K_L03560-1h} in Wien\oindex{Wien@\textbf{Wien}|pw}, vor allen Geschichten, die wir mit dem Haus,
               den Möbeln, den Handwerkern und zunächst mit dem Hausherrn\pwindex{?? [Eigentuemer der Cottagegasse 37] @\textsc{?? [Eigentümer der Cottagegasse 37]}|pwv} haben werden, der mich wieder und immer wieder zu schröpfen
               sucht. \pend
           \pstart
           Ich freu mich \uline{sehr}, dass es Ihrer Schwester\pwindex{Steinrueck, Elisabeth 19.11.1885 – 07.04.1920@\textsc{Steinrück, Elisabeth} (19.11.1885 – 07.04.1920)|pwv} gut geht. Bitte, grüßen Sie sie
               vielmals von uns! Haben Sie nun in München\oindex{Muenchen@\textbf{München}|pw} Ihre
               Konzertreise zusammengestellt? Ich bin sehr neugierig darauf, und wüßte gern, wann
               und wohin Sie gehen. Jedenfalls werde ich Sie aber doch gewiss vorher noch singen
               hören, was ich mir lebhaft wünsche, und möchte, wenn Sie’s gestatten, auch Ihr
               Progamm als Privatkonzert zu hören bekommen. Ich bin jetzt so ziemlich sicher, dass
               Sie an Ihrer Wirkung Freude haben werden, wenn Sie wieder öffentlich singen. \pend
           \pstart
           Was haben Sie dazu gesagt, dass Herrv. Kralik\pwindex{Kralik, Richard 1852-10-01 – 1934-02-04@\textsc{Kralik, Richard} (1852-10-01 – 1934-02-04), \emph{Schriftsteller}|pw}
               für das Burgtheater\orgindex{Burgtheater@Burgtheater|pw} kandidirt wird?
               Symptomatisch! \pend
           \pstart
           Noch herzliche Grüße von uns allen, ebenso von Fischers\pwindex{Fischer, Samuel 24.12.1859 – 15.10.1934@\textsc{Fischer, Samuel} (24.12.1859 – 15.10.1934), \emph{Verleger}|pw}\pwindex{Fischer, Hedwig 08.09.1871 – 11.04.1952@\textsc{Fischer, Hedwig} (08.09.1871 – 11.04.1952)|pw}.\pend
           \pstart Aufrichtig Ihr \spacefill\mbox{Felix Salten}\pend{}
         
         \endnumbering\mylabel{h}\end{ledgroupsized}\begin{anhang}\end{anhang}\newcommand{\dateiname}{L03560}\newcommand{\titel}{Felix Salten an Olga Schnitzler, 2. 9. 1912}\newcommand{\editorInnen}{Martin Anton Müller und Laura Untner}%% latex-leseansicht-abspann.tex
%% Abspann für die Leseansicht.
%% Der Schalter \ifkorrekturansicht ist bereits durch den Vorspann gesetzt.

%% latex-abspann.tex
%% Gemeinsamer Abspann für Korrekturansicht und Leseansicht.
%% Setzt den Schalter \ifkorrekturansicht voraus (gesetzt in den
%% einbindenden Dateien latex-korrekturansicht-abspann.tex bzw.
%% latex-leseansicht-abspann.tex).
%% ---------------------------------------------------------------

\normalsize

% Das esempio-Environment wird nur in der Leseansicht benötigt
\ifkorrekturansicht\else
\newenvironment{esempio}[3]%
{
    \vspace{1.5ex}
    \rlap{\underline{#1}}
    \par
    \setlength{\parindent}{0cm}
    \nopagebreak
    \leftskip=#2cm
    \rightskip=#3cm
}
{
    \par
}
\fi

\doendnotes{C}
\bigskip
\vfill

\clearpage

\footnotesize

\ifkorrekturansicht
  \lohead{\textsc{register}}
\fi

% theindex-Environment neu definieren ohne reledmac
\makeatletter
\renewenvironment{theindex}{%
  \ifkorrekturansicht
    \section*{\indexname}%
  \else
    \subsubsection*{Index der erwähnten Entitäten}%
  \fi
  \setlength{\parindent}{0pt}%
  \setlength{\parskip}{0pt plus 0.3pt}%
  \let\item\@idxitem
}{%
  \ifkorrekturansicht\clearpage\fi
}
\makeatother

\IfFileExists{\jobname-pw.ind}{\input{\jobname-pw.ind}}{}

% Quellenangabe nur in der Leseansicht
\ifkorrekturansicht\else
% Fallback-Definitionen, falls die .tex-Datei \titel etc. nicht gesetzt hat
\providecommand{\titel}{}
\providecommand{\editorInnen}{}
\providecommand{\dateiname}{\jobname}

\vspace{3cm}

\vfill

\footnotesize
\textsc{Quelle}: \titel. Herausgegeben von {\editorInnen}. In: \emph{Arthur Schnitzler: Briefwechsel mit Autorinnen und Autoren}.
 Digitale Edition, https://schnitzler-briefe.acdh.oeaw.ac.at/{\dateiname}.html (Stand \today)
\fi

\end{document}


      