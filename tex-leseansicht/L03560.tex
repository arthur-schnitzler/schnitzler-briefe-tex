%% latex-leseansicht-vorspann.tex
%% Vorspann für die Leseansicht.
%% Lädt die gemeinsame Datei latex-vorspann.tex mit nicht gesetztem Schalter.

\newif\ifkorrekturansicht
\korrekturansichtfalse

\input{../tex-inputs/latex-vorspann}


\section[ Felix Salten an Olga Schnitzler, 2. 9. 1912]{L03560 Felix Salten an Olga Schnitzler,  2. 9. 1912}
\nopagebreak\mylabel{L03560v}
\rehead{ }\normalsize\beginnumbering\briefempfaengerindex{Schnitzler, Olga@\textsc{Schnitzler, Olga}!zzzSalten, Felix@\emph{von Felix Salten}!1912-09-022@{2. 9. 1912}|(be}
\toendnotes[C]{\smallbreak\pagebreak[2]}
\correspDesc{Versand  durch Felix Salten am 2. 9. 1912 in Unterach am Attersee
\newline{}Erhalt  durch Olga Schnitzler im Zeitraum [3. 9. 1912
                  – 7. 9. 1912?] in Wien}\toendnotes[C]{\smallbreak}
\Standort{CUL, Schnitzler, B 89, B 2.}
\physDesc{Briefkarte, 1863 Zeichen
\newline{}Handschrift: schwarze Tinte, lateinische Kurrent
\newline{}Ordnung: mit Bleistift von unbekannter Hand nummeriert: »274« }\toendnotes[C]{\smallbreak}
\pstart
           \raggedleft{}{\pb}Berghof\oindex{Berghof@\textbf{Berghof}, \emph{Wohngebäude}|pw}, 2. IX. 12.\pend
           
\pstart{}Verehrte, liebe Frau Olga,\pend\vspace{0.5em}
\pstart
           vielen Dank für den lieben Brief und für \label{K_L03560-1v}\edtext{Arthurs Karten}{\lemma{\textnormal{\emph{Arthurs Karten}}}\Cendnote{\textnormal{nicht erhalten}}}\label{K_L03560-1}. Wir haben eine ziemlich unruhige Zeit
               noch nicht ganz hinter uns. Wollfs\pwindex{Wollf, Julius Ferdinand 22.\,5.\,1871 Koblenz – 1.\,3.\,1942 Dresden@\textsc{Wollf, Julius Ferdinand} (22.\,5.\,1871 Koblenz – 1.\,3.\,1942 Dresden), \emph{Journalist, Herausgeber, Verleger}|pw}\pwindex{Wollf, Johanna Sophie 18.\,10.\,1877 Mannheim – 27.\,2.\,1942 Dresden@\textsc{Wollf, Johanna Sophie} (18.\,10.\,1877 Mannheim – 27.\,2.\,1942 Dresden)|pw}
               aus Dresden\oindex{Dresden@\textbf{Dresden}|pw} sind drei Wochen lang bei uns gewesen
               und wir haben uns sehr mit Ihnen gefreut. Wir konnten nur deshalb zu keinem ganzen
               Behagen kommen, weil es fast unaufhörlich geregnet hat, und weil Otti\pwindex{Salten, Ottilie 7.\,3.\,1868 Prag – 22.\,6.\,1942 Zürich@\textsc{Salten, Ottilie} (7.\,3.\,1868 Prag – 22.\,6.\,1942 Zürich), \emph{Schauspielerin}|pw} mit ihrer Gesundheit nicht ganz in Ordnung war. Nun ist
               sie seit Mittwoch in Wien\oindex{Wien@\textbf{Wien}, \emph{Verwaltungsgebiet}|pw}, im Sanatorium »Hera\oindex{Wien@\textbf{Wien}!IX., Alsergrund@\textbf{IX., Alsergrund}!Sanatorium Hera@\textbf{Sanatorium Hera}, \emph{Sanatorium}|pw}«, und hat am
                  Donnerstag eine kleine Operation überstanden. Es
               ist alles sehr gut gegangen: sie befindet sich schon viel besser und es ist möglich,
               dass Sie übermorgen oder Donnerstag schon wieder hier\oindex{Unterach am Attersee@\textbf{Unterach am Attersee}|pwv} sein wird. Bei alledem – angenehm ist sowas ja nie, weder für Otti\pwindex{Salten, Ottilie 7.\,3.\,1868 Prag – 22.\,6.\,1942 Zürich@\textsc{Salten, Ottilie} (7.\,3.\,1868 Prag – 22.\,6.\,1942 Zürich), \emph{Schauspielerin}|pw}, die allein, nur vom Stubenmädchen\pwindex{?? [Haushaltshilfe der Familie Salten 1912] @\textsc{?? [Haushaltshilfe der Familie Salten 1912]}|pwv} begleitet, in Wien\oindex{Wien@\textbf{Wien}, \emph{Verwaltungsgebiet}|pw} sein muß, noch für mich, der hier nur warten
               und sonst nichts nützliches für sie tun kann. Vielleicht haben wir hier noch ein paar
               Wochen Zeit, dass Otti\pwindex{Salten, Ottilie 7.\,3.\,1868 Prag – 22.\,6.\,1942 Zürich@\textsc{Salten, Ottilie} (7.\,3.\,1868 Prag – 22.\,6.\,1942 Zürich), \emph{Schauspielerin}|pw}{ }{\pb}sich erholen kann. Ohnehin
               graut uns ein bischen vor dem \label{K_L03560-2v}\edtext{Umzug}{\lemma{\textnormal{\emph{Umzug}}}\Cendnote{\textnormal{Eventuell
                  handelte es sich um eine größere Wohnung im selben Haus\oindex{Wien@\textbf{Wien}!XVIII., Währing@\textbf{XVIII., Währing}!Cottagegasse@\textbf{Cottagegasse}, \emph{Straße}|pwkv}\oindex{Wien@\textbf{Wien}!XIX., Döbling@\textbf{XIX., Döbling}!Cottagegasse@\textbf{Cottagegasse}, \emph{Straße}|pwkv}. Jedenfalls wohnte Salten\pwindex{Salten, Felix 6.\,9.\,1869 Budapest – 8.\,10.\,1945 Zürich@\textsc{Salten, Felix} (6.\,9.\,1869 Budapest – 8.\,10.\,1945 Zürich), \emph{Schriftsteller, Journalist, Chefredakteur}|pwk} seit
                     25. 1. 1910 und bis zu seiner Emigration
                     1938 in der Cottagegasse 37\oindex{Wien@\textbf{Wien}!XVIII., Währing@\textbf{XVIII., Währing}!Cottagegasse@\textbf{Cottagegasse}, \emph{Straße}|pwk}\oindex{Wien@\textbf{Wien}!XIX., Döbling@\textbf{XIX., Döbling}!Cottagegasse@\textbf{Cottagegasse}, \emph{Straße}|pwk}. }}}\label{K_L03560-2} in Wien\oindex{Wien@\textbf{Wien}, \emph{Verwaltungsgebiet}|pw},
               vor allen Geschichten, die wir mit dem Haus\oindex{Wien@\textbf{Wien}!XVIII., Währing@\textbf{XVIII., Währing}!Cottagegasse@\textbf{Cottagegasse}, \emph{Straße}|pwv}\oindex{Wien@\textbf{Wien}!XIX., Döbling@\textbf{XIX., Döbling}!Cottagegasse@\textbf{Cottagegasse}, \emph{Straße}|pwv}, den Möbeln, den Handwerkern und zunächst mit dem
                  \label{K_L03560-3v}\edtext{Hausherr\pwindex{Schwarz, Emil @\textsc{Schwarz, Emil}, \emph{Vermieter, Beamter}|pwv}n}{\lemma{\textnormal{\emph{Hausherrn}}}\Cendnote{\textnormal{Emil Schwarz\pwindex{Schwarz, Emil @\textsc{Schwarz, Emil}, \emph{Vermieter, Beamter}|pwk}}}}\label{K_L03560-3} haben werden, der mich wieder und immer wieder zu schröpfen sucht.\pend
           
\pstart
           Ich freu mich \uline{sehr}, dass es Ihrer Schwester\pwindex{Steinrück, Elisabeth 19.\,11.\,1885 – 7.\,4.\,1920 Partenkirchen@\textsc{Steinrück, Elisabeth} (19.\,11.\,1885 – 7.\,4.\,1920 Partenkirchen)|pwv} gut geht. Bitte, grüßen Sie sie
               vielmals von uns\pwindex{Salten, Ottilie 7.\,3.\,1868 Prag – 22.\,6.\,1942 Zürich@\textsc{Salten, Ottilie} (7.\,3.\,1868 Prag – 22.\,6.\,1942 Zürich), \emph{Schauspielerin}|pwv}! Haben Sie
               nun \label{K_L03560-4v}\edtext{in München\oindex{München@\textbf{München}|pw} Ihre Konzertreise zusa{\geminationm}engestellt}{\lemma{\textnormal{\emph{in … zusammengestellt}}}\Cendnote{\textnormal{Gemeint war wohl der
                  kurze Zwischenstopp in München\oindex{München@\textbf{München}|pwk} am 29. 8. 1912. Zu einer
                  Konzertreise kam es nicht.}}}\label{K_L03560-4}? Ich bin sehr neugierig darauf, und wüßte gern,
               wann und wohin Sie gehen. Jedenfalls werde ich Sie aber doch gewiss vorher noch
               singen hören, was ich mir lebhaft wünsche, und möchte, wenn Sie’s gestatten, auch Ihr
               Progamm als Privatkonzert zu hören bekommen. Ich bin jetzt so ziemlich sicher, dass
               Sie an Ihrer Wirkung Freude haben werden, wenn Sie wieder öffentlich singen.\pend
           
\pstart
           Was haben Sie dazu gesagt, dass Herr v. Kralik\pwindex{Kralik, Richard 1.\,10.\,1852 Lenora – 4.\,2.\,1934 Wien@\textsc{Kralik, Richard} (1.\,10.\,1852 Lenora – 4.\,2.\,1934 Wien), \emph{Schriftsteller}|pw}
               für das Burgtheater\orgindex{Burgtheater@Burgtheater|pw} kandidirt wird?
               Symptomatisch!\pend
           
\pstart
           V\textcolor{gray}{ie}le herzliche Grüße von uns allen, ebenso von Fischers\pwindex{Fischer, Samuel 24.\,12.\,1859 Liptovský Mikuláš – 15.\,10.\,1934 Berlin@\textsc{Fischer, Samuel} (24.\,12.\,1859 Liptovský Mikuláš – 15.\,10.\,1934 Berlin), \emph{Verleger}|pw}\pwindex{Fischer, Hedwig 8.\,9.\,1871 Szczecin – 11.\,4.\,1952 Königstein im Taunus@\textsc{Fischer, Hedwig} (8.\,9.\,1871 Szczecin – 11.\,4.\,1952 Königstein im Taunus)|pw}.\pend
           \pstart Aufrichtig Ihr \spacefill\mbox{Felix Salten}\pend{}\selectlanguage{ngerman}\endnumbering\briefempfaengerindex{Schnitzler, Olga@\textsc{Schnitzler, Olga}!zzzSalten, Felix@\emph{von Felix Salten}!1912-09-022@{2. 9. 1912}|)be}\mylabel{L03560h}  \newcommand{\dateiname}{L03560}\newcommand{\titel}{Felix Salten an Olga Schnitzler, 2. 9. 1912}\newcommand{\editorInnen}{Martin Anton Müller und Laura Untner}%% latex-leseansicht-abspann.tex
%% Abspann für die Leseansicht.
%% Der Schalter \ifkorrekturansicht ist bereits durch den Vorspann gesetzt.

%% latex-abspann.tex
%% Gemeinsamer Abspann für Korrekturansicht und Leseansicht.
%% Setzt den Schalter \ifkorrekturansicht voraus (gesetzt in den
%% einbindenden Dateien latex-korrekturansicht-abspann.tex bzw.
%% latex-leseansicht-abspann.tex).
%% ---------------------------------------------------------------

\normalsize

% Das esempio-Environment wird nur in der Leseansicht benötigt
\ifkorrekturansicht\else
\newenvironment{esempio}[3]%
{
    \vspace{1.5ex}
    \rlap{\underline{#1}}
    \par
    \setlength{\parindent}{0cm}
    \nopagebreak
    \leftskip=#2cm
    \rightskip=#3cm
}
{
    \par
}
\fi

\doendnotes{C}
\bigskip
\vfill

\clearpage

\footnotesize

\ifkorrekturansicht
  \lohead{\textsc{register}}
\fi

% theindex-Environment neu definieren ohne reledmac
\makeatletter
\renewenvironment{theindex}{%
  \ifkorrekturansicht
    \section*{\indexname}%
  \else
    \subsubsection*{Index der erwähnten Entitäten}%
  \fi
  \setlength{\parindent}{0pt}%
  \setlength{\parskip}{0pt plus 0.3pt}%
  \let\item\@idxitem
}{%
  \ifkorrekturansicht\clearpage\fi
}
\makeatother

\IfFileExists{\jobname-pw.ind}{\input{\jobname-pw.ind}}{}

% Quellenangabe nur in der Leseansicht
\ifkorrekturansicht\else
% Fallback-Definitionen, falls die .tex-Datei \titel etc. nicht gesetzt hat
\providecommand{\titel}{}
\providecommand{\editorInnen}{}
\providecommand{\dateiname}{\jobname}

\vspace{3cm}

\vfill

\footnotesize
\textsc{Quelle}: \titel. Herausgegeben von {\editorInnen}. In: \emph{Arthur Schnitzler: Briefwechsel mit Autorinnen und Autoren}.
 Digitale Edition, https://schnitzler-briefe.acdh.oeaw.ac.at/{\dateiname}.html (Stand \today)
\fi

\end{document}


