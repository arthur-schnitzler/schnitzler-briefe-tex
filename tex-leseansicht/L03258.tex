%% latex-korrekturansicht-vorspann.tex
%% Vorspann für die Korrekturansicht.
%% Lädt die gemeinsame Datei latex-vorspann.tex mit gesetztem Schalter.

\newif\ifkorrekturansicht
\korrekturansichttrue

\input{../tex-inputs/latex-vorspann}


\section[ Paul Goldmann an Arthur Schnitzler, 6. 9. 1907]{L03258 Paul Goldmann an Arthur Schnitzler, 6. 9. 1907}
\nopagebreak\mylabel{L03258v}
\rehead{ }\normalsize\beginnumbering\briefempfaengerindex{Schnitzler, Arthur@\textsc{Schnitzler, Arthur}!zzzGoldmann, Paul@\emph{von Paul Goldmann}!1907-09-061@{6. 9. 1907}|(be}
\toendnotes[C]{\smallbreak\pagebreak[2]}\Standort{DLA, A:Schnitzler, HS.NZ85.1.3175.}
\physDesc{Bildpostkarte, 252 Zeichen
\newline{}Handschrift: 1) schwarze Tinte, deutsche Kurrent\hspace{1em}2) schwarze Tinte, lateinische Kurrent (\noindent{}Adresse)\hspace{1em}
\newline{}Versand: Stempel: »\nobreak{}\oindex{Karersee@\textbf{Karersee}, \emph{See (N.SEE)}|pwk}Karersee, 6. IX. 07\nobreak{}«.  }\toendnotes[C]{\smallbreak}\pstart{}{\pb}Herrn\pend{}\pstart{}Dr. Arthur Schnitzler\pend{}\pstart{}Wien\oindex{Wien@\textbf{Wien}, \emph{A.ADM2}|pw}\pend{}\pstart{}XVIII. Spöttelgaſse 7\oindex{Edmund-Weiss-Gasse 7@\textbf{Edmund-Weiß-Gasse 7}, \emph{Wohngebäude (K.WHS)}|pw}.\pend{}{\bigskip}
\pstart
           \noindent{}\centering{}{\pb}\textcolor{gray}{\textbf{Karersee-Hotel\oindex{Grand Hotel Carezza@\textbf{Grand Hotel Carezza}, \emph{Hotel (K.HTL)}|pw} (1650 m) in Tirol\oindex{Tirol@\textbf{Tirol}, \emph{A.ADM1}|pw}. }}\pend
           \vspace{1em}
\pstart
           {\pb}6. 9. 07.\pend
           \vspace{0.5em}
\pstart
           Es hat mir unendlich leid getan, lieber Freund, Dich
               in \label{K_L03258-1v}\edtext{Karerſee\oindex{Karersee@\textbf{Karersee}, \emph{See (N.SEE)}|pw} nicht mehr angetroffen}{\lemma{\textnormal{\emph{Karerſee … angetroffen}}}\Cendnote{\textnormal{Schnitzler hielt sich zwischen 29. 8. 1907 und 31. 8. 1907 in Karersee\oindex{Karersee@\textbf{Karersee}, \emph{See (N.SEE)}|pwk} auf.}}}\label{K_L03258-1} zu haben. Hoffentlich
                  \label{K_L03258-2v}\edtext{ſehen wir uns}{\lemma{\textnormal{\emph{ſehen wir uns}}}\Cendnote{\textnormal{Schnitzler und Goldmann\pwindex{Goldmann, Paul 31.01.1865 – 25.09.1935@\textsc{Goldmann, Paul} (31.01.1865 – 25.09.1935), \emph{Schriftsteller/Schriftstellerin, Journalist/Journalistin}|pwk} sahen sich das nächste Mal am 8. 10. 1907 in Wien\oindex{Wien@\textbf{Wien}, \emph{A.ADM2}|pwk}.}}}\label{K_L03258-2} Ende Sept. in Wien\oindex{Wien@\textbf{Wien}, \emph{A.ADM2}|pw}.\pend
           
\pstart
           Herzliche Grüße Dir u. Deiner Frau\pwindex{Schnitzler, Olga 17.01.1882 – 13.01.1970@\textsc{Schnitzler, Olga} (17.01.1882 – 13.01.1970), \emph{Schauspieler/Schauspielerin, Sänger/Sängerin}|pwv} von Deinem {\\[\baselineskip]}\spacefill\mbox{Paul Goldmann.}\pend
           \leftskip=0em{}\selectlanguage{ngerman}\endnumbering\briefempfaengerindex{Schnitzler, Arthur@\textsc{Schnitzler, Arthur}!zzzGoldmann, Paul@\emph{von Paul Goldmann}!1907-09-061@{6. 9. 1907}|)be}\mylabel{L03258h}  \normalsize

\doendnotes{C}
\bigskip
\vfill

\clearpage

\footnotesize

\lohead{\textsc{register}}

% Definiere theindex-Environment komplett neu ohne reledmac
\makeatletter
\renewenvironment{theindex}{%
  \section*{\indexname}%
  \setlength{\parindent}{0pt}%
  \setlength{\parskip}{0pt plus 0.3pt}%
  \let\item\@idxitem
}{%
  \clearpage
}
\makeatother

\IfFileExists{\jobname-pw.ind}{\input{\jobname-pw.ind}}{}

\end{document}

      