%% latex-leseansicht-vorspann.tex
%% Vorspann für die Leseansicht.
%% Lädt die gemeinsame Datei latex-vorspann.tex mit nicht gesetztem Schalter.

\newif\ifkorrekturansicht
\korrekturansichtfalse

\input{../tex-inputs/latex-vorspann}

\begin{center}
            \textcolor{red}{ENTWURF, NICHT FERTIG KORRIGIERT}
                      \end{center}
            
         
         \renewcommand{\erwaehntePersonen}{Personen: Heinrich Kanner, Isidor Singer}
         \renewcommand{\erwaehnteInstitutionen}{Institutionen: Die Zeit}
         \renewcommand{\erwaehnteOrte}{Orte: Niederösterreich, Steiermark, Wien, Wipplingerstraße}
         \renewcommand{\erwaehnteWerke}{Werke: Die Zeit, Die griechische Tänzerin. Novellette}
               \section[Felix Salten an Arthur Schnitzler, 2. 9. 1902]{ Felix Salten an Arthur Schnitzler, 2. 9. 1902}\nopagebreak\mylabel{v}\rehead{ }\begin{ledgroupsized}[t]{13cm}\normalsize\beginnumbering \toendnotes[C]{\smallbreak\pagebreak[2]} \Standort{CUL, Schnitzler, B 89, A 2.}
\physDesc{Brief, 1 Blatt, 1 Seite
\newline{}Handschrift: schwarze Tinte, lateinische Kurrent\newline{}Ordnung: mit Bleistift von unbekannter Hand nummeriert:
                                    »158« }\toendnotes[C]{\smallbreak}\pstart
           \noindent{}{\pb}\textcolor{gray}{\textbf{Die}}\hfill \textcolor{gray}{\textbf{WIEN\oindex{Wien@\textbf{Wien}|pw},}}{ }2. Septemb. \textcolor{gray}{\textbf{190}}2.\pend
           \pstart
           \textcolor{gray}{\textbf{ZEIT\orgindex{Zeit@Die Zeit|pw}}}\pend
           \pstart
           \textcolor{gray}{\textbf{HERAUSGEBER:}}\pend
           \pstart
           \textcolor{gray}{\textbf{PROF. DR. I. SINGER\pwindex{Singer, Isidor 16.01.1857 – 08.12.1927@\textsc{Singer, Isidor} (16.01.1857 – 08.12.1927), \emph{Journalist, Herausgeber, Soziologe}|pw}}}\pend
           \pstart
           \textcolor{gray}{\textbf{DR. HEINRICH KANNER\pwindex{Kanner, Heinrich 09.11.1864 – 15.02.1930@\textsc{Kanner, Heinrich} (09.11.1864 – 15.02.1930), \emph{Herausgeber, Publizist}|pw}}}\pend
           \pstart
           \textcolor{gray}{\textbf{REDACTION:}}\pend
           \pstart
           \textcolor{gray}{\textbf{I/\textsubscript{21}
                           WIPPLINGERSTRASSE 38\oindex{Wipplingerstrasse@\textbf{Wipplingerstraße}|pw}}}\pend
           \pstart
           Lieber –telefonisch konnte ich Sie nicht mehr erreichen, als heute
               Mittag Ihr Brief kam. Das Ganze ist selbstverständlich ein Irrthum. D\textsuperscript{r} Kanner\pwindex{Kanner, Heinrich 09.11.1864 – 15.02.1930@\textsc{Kanner, Heinrich} (09.11.1864 – 15.02.1930), \emph{Herausgeber, Publizist}|pw} acceptirte
               s. Z. Ihre Honorarforderung sofort u. willig und hat nur vergessen die Su{\geminationm}e dem Prof. Singer\pwindex{Singer, Isidor 16.01.1857 – 08.12.1927@\textsc{Singer, Isidor} (16.01.1857 – 08.12.1927), \emph{Journalist, Herausgeber, Soziologe}|pw}, der die Caße führt, bekannt zu geben. Dieser wieder dachte bei
               Absendung des Honorares nicht an ein besonderes Übereinkommen und hat auch nicht
               danach gefragt. In dem jetzt herrschenden Arbeits-Trubel hat ein derartiger Irrthum
               wol nichts Verletzendes an sich und darf wol als entschuldbar gelten. Die fehlenden
               120 Kronen gehen natürlich gleich an Sie ab. Ich hoffe, Sie nehmen diesen
               Zwischenfall nicht zum Anlaß, mich mit der \label{K_L03333-1v}\edtext{Novelle\pwindex{Schnitzler, Arthur 15.05.1862 – 21.10.1931@\textsc{Schnitzler, Arthur} (15.05.1862 – 21.10.1931), \emph{Schriftsteller, Mediziner}!griechische Taenzerin. Novellette28. 09. 1902@\strich\emph{Die griechische Tänzerin. Novellette} {[}28. 09. 1902{]}|pwv}}{\lemma{\textnormal{\emph{Novelle}}}\Cendnote{\textnormal{Arthur Schnitzler\pwindex{Schnitzler, Arthur 15.05.1862 – 21.10.1931@\textsc{Schnitzler, Arthur} (15.05.1862 – 21.10.1931), \emph{Schriftsteller, Mediziner}|pwk}: \emph{Die griechische Tänzerin}\pwindex{Schnitzler, Arthur 15.05.1862 – 21.10.1931@\textsc{Schnitzler, Arthur} (15.05.1862 – 21.10.1931), \emph{Schriftsteller, Mediziner}!griechische Taenzerin. Novellette28. 09. 1902@\strich\emph{Die griechische Tänzerin. Novellette} {[}28. 09. 1902{]}|pwk}. In: \emph{Die Zeit}\pwindex{Zeit1902-09-27 – 1919@\emph{Die Zeit} {[}1902-09-27 – 1919{]}|pwk}, Jg. 1, Nr. 2, 28. 9. 1902,
                     Morgenblatt, Beilage: Sonntags-Zeit, S. 4–7.}}}\label{K_L03333-1h} sitzen zu laßen, und
               hoffe weiter, Sie haben das Mscpt, wie besprochen, auf Ihre \label{K_L03333-23v}\edtext{Reise}{\lemma{\textnormal{\emph{Reise}}}\Cendnote{\textnormal{Schnitzler\pwindex{Schnitzler, Arthur 15.05.1862 – 21.10.1931@\textsc{Schnitzler, Arthur} (15.05.1862 – 21.10.1931), \emph{Schriftsteller, Mediziner}|pwk} war von 2. 9. 1902 bis zum
                     7. 9. 1902 mit
                  dem Fahrrad in der Niederösterreich\oindex{Niederoesterreich@\textbf{Niederösterreich}|pwk} und der
                     Steiermark\oindex{Steiermark@\textbf{Steiermark}|pwk} unterwegs.}}}\label{K_L03333-23h} mitgenommen,
               denn es wäre mir doch äußerst unangenehm, wenn Sie, ohne weitere Aufklärung
               abzuwarten (die ja auch durch telef. Anruf sofort zu erhalten war) die Sache beiseite
               gelegt hätten. Mir ist der Vorfall doppelt unangenehm, weil er mit einem anderen fast
               auf die Stunde zusammentrifft, und ich jetzt mit dem von mir ausgeworbenen
               Mitarbeitern ziemlich beschämt dastehe. \pend
           \pstart
           Herzlichst Ihr {\\[\baselineskip]}\spacefill\mbox{Salten.}\pend
           \leftskip=0em{}
         
         \endnumbering\mylabel{h}\end{ledgroupsized}\begin{anhang}\end{anhang}\newcommand{\dateiname}{L03333}\newcommand{\titel}{Felix Salten an Arthur Schnitzler, 2. 9. 1902}\newcommand{\editorInnen}{Martin Anton Müller und Laura Untner}%% latex-leseansicht-abspann.tex
%% Abspann für die Leseansicht.
%% Der Schalter \ifkorrekturansicht ist bereits durch den Vorspann gesetzt.

%% latex-abspann.tex
%% Gemeinsamer Abspann für Korrekturansicht und Leseansicht.
%% Setzt den Schalter \ifkorrekturansicht voraus (gesetzt in den
%% einbindenden Dateien latex-korrekturansicht-abspann.tex bzw.
%% latex-leseansicht-abspann.tex).
%% ---------------------------------------------------------------

\normalsize

% Das esempio-Environment wird nur in der Leseansicht benötigt
\ifkorrekturansicht\else
\newenvironment{esempio}[3]%
{
    \vspace{1.5ex}
    \rlap{\underline{#1}}
    \par
    \setlength{\parindent}{0cm}
    \nopagebreak
    \leftskip=#2cm
    \rightskip=#3cm
}
{
    \par
}
\fi

\doendnotes{C}
\bigskip
\vfill

\clearpage

\footnotesize

\ifkorrekturansicht
  \lohead{\textsc{register}}
\fi

% theindex-Environment neu definieren ohne reledmac
\makeatletter
\renewenvironment{theindex}{%
  \ifkorrekturansicht
    \section*{\indexname}%
  \else
    \subsubsection*{Index der erwähnten Entitäten}%
  \fi
  \setlength{\parindent}{0pt}%
  \setlength{\parskip}{0pt plus 0.3pt}%
  \let\item\@idxitem
}{%
  \ifkorrekturansicht\clearpage\fi
}
\makeatother

\IfFileExists{\jobname-pw.ind}{\input{\jobname-pw.ind}}{}

% Quellenangabe nur in der Leseansicht
\ifkorrekturansicht\else
% Fallback-Definitionen, falls die .tex-Datei \titel etc. nicht gesetzt hat
\providecommand{\titel}{}
\providecommand{\editorInnen}{}
\providecommand{\dateiname}{\jobname}

\vspace{3cm}

\vfill

\footnotesize
\textsc{Quelle}: \titel. Herausgegeben von {\editorInnen}. In: \emph{Arthur Schnitzler: Briefwechsel mit Autorinnen und Autoren}.
 Digitale Edition, https://schnitzler-briefe.acdh.oeaw.ac.at/{\dateiname}.html (Stand \today)
\fi

\end{document}


      