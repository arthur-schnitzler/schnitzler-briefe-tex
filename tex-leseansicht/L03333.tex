%% latex-leseansicht-vorspann.tex
%% Vorspann für die Leseansicht.
%% Lädt die gemeinsame Datei latex-vorspann.tex mit nicht gesetztem Schalter.

\newif\ifkorrekturansicht
\korrekturansichtfalse

\input{../tex-inputs/latex-vorspann}


         
         \renewcommand{\erwaehntePersonen}{Personen: Heinrich Kanner, Felix Salten, Isidor Singer}
         \renewcommand{\erwaehnteInstitutionen}{Institutionen: Die Zeit}
         \renewcommand{\erwaehnteOrte}{Orte: Leoben, Niederösterreich, Steiermark, Wien, Wipplingerstraße}
         \renewcommand{\erwaehnteWerke}{Werke: Die Zeit, Die griechische Tänzerin. Novellette}
               \section[ Felix Salten an Arthur Schnitzler, 2. 9. 1902]{ Felix Salten an Arthur Schnitzler, 2. 9. 1902}\nopagebreak\mylabel{v}\rehead{ }\begin{ledgroupsized}[t]{13cm}\normalsize\beginnumbering\briefempfaengerindex{Schnitzler, Arthur@\textsc{Schnitzler, Arthur}!zzzSalten, Felix@\emph{von Felix Salten}!1902-09-022@{2. 9. 1902}|(be} \toendnotes[C]{\smallbreak\pagebreak[2]} \Standort{CUL, Schnitzler, B 89, A 2.}
\physDesc{Brief, 1 Blatt, 1 Seite, 1175 Zeichen
\newline{}Handschrift: schwarze Tinte, lateinische Kurrent
\newline{}Ordnung: mit Bleistift von unbekannter Hand nummeriert: »158« }\toendnotes[C]{\smallbreak}\pstart
           \noindent{}{\pb}\textcolor{gray}{\textbf{DIE}}\pend
           \pstart
           \textcolor{gray}{\textbf{ZEIT\orgindex{Zeit@Die Zeit|pw}}}\hfill \textcolor{gray}{\textbf{\emph{WIEN\oindex{Wien@\textbf{Wien}|pw}},}}{ }2. Septemb. \textcolor{gray}{\textbf{\emph{190}}}2.\pend
           \pstart
           \textcolor{gray}{\textbf{\textsc{\textbf{\so{Wiener Tagblatt}}}}}\pend
           \pstart
           \textcolor{gray}{\textbf{HERAUSGEBER:}}\pend
           \pstart
           \textcolor{gray}{\textbf{\textbf{PROF. DR. I. SINGER\pwindex{Singer, Isidor 16.01.1857 – 08.12.1927@\textsc{Singer, Isidor} (16.01.1857 – 08.12.1927), \emph{Journalist, Herausgeber, Soziologe}|pw}}}}\pend
           \pstart
           \textcolor{gray}{\textbf{\textbf{DR. HEINRICH KANNER\pwindex{Kanner, Heinrich 09.11.1864 – 15.02.1930@\textsc{Kanner, Heinrich} (09.11.1864 – 15.02.1930), \emph{Herausgeber, Publizist}|pw}}}}\pend
           \pstart
           \textcolor{gray}{\textbf{\textbf{\so{REDACTION:}}}}\pend
           \pstart
           \textcolor{gray}{\textbf{I/\textsubscript{21},
                           WIPPLINGERSTRASSE 38\oindex{Wipplingerstrasse@\textbf{Wipplingerstraße}|pw}}}\pend
           \pstart
           Lieber – telefonisch konnte ich Sie nicht mehr erreichen, als heute{ }Mittag Ihr Brief kam. Das Ganze ist selbstverständlich ein Irrthum. D\textsuperscript{r}{ }Kanner\pwindex{Kanner, Heinrich 09.11.1864 – 15.02.1930@\textsc{Kanner, Heinrich} (09.11.1864 – 15.02.1930), \emph{Herausgeber, Publizist}|pw} acceptirte \label{K_L03333-1v}\edtext{s. Z.}{\lemma{\textnormal{\emph{s. Z.}}}\Cendnote{\textnormal{seiner
                  Zeit}}}\label{K_L03333-1h} Ihre Honorarforderung sofort u. willig und hat nur vergessen die Su{\geminationm}e dem Prof. Singer\pwindex{Singer, Isidor 16.01.1857 – 08.12.1927@\textsc{Singer, Isidor} (16.01.1857 – 08.12.1927), \emph{Journalist, Herausgeber, Soziologe}|pw}, der die Caße führt, bekannt zu geben. Dieser wieder dachte bei
               Absendung des Honorares nicht an ein besonderes Übereinkommen und hat auch nicht
               danach gefragt. In dem jetzt herrschenden Arbeits-Trubel hat ein derartiger Irrthum
               wol nichts \substVorne{}\textsuperscript{v}\substDazwischen{}V\substHinten{}erletzendes an sich und darf wol als entschuldbar gelten. Die
               fehlenden \label{K_L03333-11v}\edtext{120 Kronen}{\lemma{\textnormal{\emph{120 Kronen}}}\Cendnote{\textnormal{Der Betrag entspricht etwa 1.000€ im Jahr 2024.}}}\label{K_L03333-11h} gehen natürlich gleich an Sie ab. Ich hoffe, Sie nehmen diesen
               Zwischenfall nicht zum Anlaß, mich mit der \label{K_L03333-2v}\edtext{Novelle\pwindex{Schnitzler, Arthur 15.05.1862 – 21.10.1931@\textsc{Schnitzler, Arthur} (15.05.1862 – 21.10.1931), \emph{Schriftsteller, Mediziner}!griechische Taenzerin. Novellette28. 09. 1902@\strich\emph{Die griechische Tänzerin. Novellette} {[}28. 09. 1902{]}|pwv}}{\lemma{\textnormal{\emph{Novelle}}}\Cendnote{\textnormal{Arthur Schnitzler\pwindex{Schnitzler, Arthur 15.05.1862 – 21.10.1931@\textsc{Schnitzler, Arthur} (15.05.1862 – 21.10.1931), \emph{Schriftsteller, Mediziner}|pwk}: \emph{Die griechische Tänzerin}\pwindex{Schnitzler, Arthur 15.05.1862 – 21.10.1931@\textsc{Schnitzler, Arthur} (15.05.1862 – 21.10.1931), \emph{Schriftsteller, Mediziner}!griechische Taenzerin. Novellette28. 09. 1902@\strich\emph{Die griechische Tänzerin. Novellette} {[}28. 09. 1902{]}|pwk}. In: \emph{Die Zeit}\pwindex{Zeit1902-09-27 – 1919@\emph{Die Zeit} {[}1902-09-27 – 1919{]}|pwk}, Jg. 1, Nr. 2, 28. 9. 1902, Morgenblatt, Beilage: Sonntags-Zeit,
                  S. 4–7.}}}\label{K_L03333-2h} sitzen zu laßen, und hoffe weiter, Sie haben das Mscpt\pwindex{Schnitzler, Arthur 15.05.1862 – 21.10.1931@\textsc{Schnitzler, Arthur} (15.05.1862 – 21.10.1931), \emph{Schriftsteller, Mediziner}!griechische Taenzerin. Novellette28. 09. 1902@\strich\emph{Die griechische Tänzerin. Novellette} {[}28. 09. 1902{]}|pwv}, wie besprochen, auf Ihre
                  \label{K_L03333-3v}\edtext{Reise}{\lemma{\textnormal{\emph{Reise}}}\Cendnote{\textnormal{Schnitzler\pwindex{Schnitzler, Arthur 15.05.1862 – 21.10.1931@\textsc{Schnitzler, Arthur} (15.05.1862 – 21.10.1931), \emph{Schriftsteller, Mediziner}|pwk} machte von 2. 9. 1902 bis zum 7. 9. 1902 eine
                     Radtour in der Steiermark\oindex{Steiermark@\textbf{Steiermark}|pwk} und in Niederösterreich\oindex{Niederoesterreich@\textbf{Niederösterreich}|pwk}.}}}\label{K_L03333-3h} mitgenommen, denn
               es wäre mir doch äußerst unangenehm, wenn Sie, ohne weitere Aufklärung abzuwarten
               (die ja auch durch telef. Anrufen sofort zu erhalten war) die Sache beiseite gelegt
               hätten. Mir ist der Vorfall doppelt unangenehm, weil er mit einem anderen fast auf
               die Stunde zusammentrifft, und ich jetzt mit dem von mir angeworbenen Mitarbeitern
               ziemlich beschämt dastehe.\pend
           \pstart
           herzlichst Ihr {\\[\baselineskip]}\spacefill\mbox{Salten.}\pend
           \leftskip=0em{}
         
         \endnumbering\mylabel{h}\end{ledgroupsized}  \newcommand{\dateiname}{L03333}\newcommand{\titel}{Felix Salten an Arthur Schnitzler, 2. 9. 1902}\newcommand{\editorInnen}{Martin Anton Müller und Laura Untner}%% latex-leseansicht-abspann.tex
%% Abspann für die Leseansicht.
%% Der Schalter \ifkorrekturansicht ist bereits durch den Vorspann gesetzt.

%% latex-abspann.tex
%% Gemeinsamer Abspann für Korrekturansicht und Leseansicht.
%% Setzt den Schalter \ifkorrekturansicht voraus (gesetzt in den
%% einbindenden Dateien latex-korrekturansicht-abspann.tex bzw.
%% latex-leseansicht-abspann.tex).
%% ---------------------------------------------------------------

\normalsize

% Das esempio-Environment wird nur in der Leseansicht benötigt
\ifkorrekturansicht\else
\newenvironment{esempio}[3]%
{
    \vspace{1.5ex}
    \rlap{\underline{#1}}
    \par
    \setlength{\parindent}{0cm}
    \nopagebreak
    \leftskip=#2cm
    \rightskip=#3cm
}
{
    \par
}
\fi

\doendnotes{C}
\bigskip
\vfill

\clearpage

\footnotesize

\ifkorrekturansicht
  \lohead{\textsc{register}}
\fi

% theindex-Environment neu definieren ohne reledmac
\makeatletter
\renewenvironment{theindex}{%
  \ifkorrekturansicht
    \section*{\indexname}%
  \else
    \subsubsection*{Index der erwähnten Entitäten}%
  \fi
  \setlength{\parindent}{0pt}%
  \setlength{\parskip}{0pt plus 0.3pt}%
  \let\item\@idxitem
}{%
  \ifkorrekturansicht\clearpage\fi
}
\makeatother

\IfFileExists{\jobname-pw.ind}{\input{\jobname-pw.ind}}{}

% Quellenangabe nur in der Leseansicht
\ifkorrekturansicht\else
% Fallback-Definitionen, falls die .tex-Datei \titel etc. nicht gesetzt hat
\providecommand{\titel}{}
\providecommand{\editorInnen}{}
\providecommand{\dateiname}{\jobname}

\vspace{3cm}

\vfill

\footnotesize
\textsc{Quelle}: \titel. Herausgegeben von {\editorInnen}. In: \emph{Arthur Schnitzler: Briefwechsel mit Autorinnen und Autoren}.
 Digitale Edition, https://schnitzler-briefe.acdh.oeaw.ac.at/{\dateiname}.html (Stand \today)
\fi

\end{document}


      