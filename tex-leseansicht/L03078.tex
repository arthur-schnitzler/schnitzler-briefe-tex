%% latex-korrekturansicht-vorspann.tex
%% Vorspann für die Korrekturansicht.
%% Lädt die gemeinsame Datei latex-vorspann.tex mit gesetztem Schalter.

\newif\ifkorrekturansicht
\korrekturansichttrue

\input{../tex-inputs/latex-vorspann}


\section[ Paul Goldmann an Arthur Schnitzler, 7. 8. 1901]{L03078 Paul Goldmann an Arthur Schnitzler, 7. 8. 1901}
\nopagebreak\mylabel{L03078v}
\rehead{ }\normalsize\beginnumbering\briefempfaengerindex{Schnitzler, Arthur@\textsc{Schnitzler, Arthur}!zzzGoldmann, Paul@\emph{von Paul Goldmann}!1901-08-071@{7. 8. 1901}|(be}
\toendnotes[C]{\smallbreak\pagebreak[2]}\Standort{DLA, A:Schnitzler, HS.NZ85.1.3171.}
\physDesc{Postkarte, 113 Zeichen
\newline{}Handschrift: 1) schwarze Tinte, deutsche Kurrent\hspace{1em}2) schwarze Tinte, lateinische Kurrent (\noindent{}Adresse)\hspace{1em}
\newline{}Versand: Stempel: »\nobreak{}\oindex{Hoehlenstein@\textbf{Höhlenstein}, \emph{P.PPLQ}|pwk}Landr{[}o{]}{ }\textcolor{gray}{Höhlenstein}, {[}7. 8.{]}\textcolor{gray}{01}\nobreak{}«.  }\toendnotes[C]{\smallbreak}\pstart{}{\pb}Herrn\pend{}\pstart{}Dr. Arthur Schnitzler\pend{}\pstart{}\strikeout{V}{ }Vahrn\oindex{Vahrn@\textbf{Vahrn}, \emph{P.PPLA3}|pw}\pend{}\pstart{}bei Brixen\oindex{Brixen@\textbf{Brixen}, \emph{P.PPLA3}|pw}.\pend{}{\bigskip}\vspace{1em}
\pstart
           \centering{}{\pb}\textsc{Landro\oindex{Hoehlenstein@\textbf{Höhlenstein}, \emph{P.PPLQ}|pw}}, 7. Auguſt.\pend
           \vspace{0.5em}
\pstart
           \label{K_L03078-1v}\edtext{\textsc{Kerr\pwindex{Kerr, Alfred 25.12.1867 – 12.10.1948@\textsc{Kerr, Alfred} (25.12.1867 – 12.10.1948), \emph{Schriftsteller/Schriftstellerin, Kritiker/Kritikerin}|pw}} wohnt}{\lemma{\textnormal{\emph{Kerr wohnt}}}\Cendnote{\textnormal{Schnitzler wollte Alfred Kerr\pwindex{Kerr, Alfred 25.12.1867 – 12.10.1948@\textsc{Kerr, Alfred} (25.12.1867 – 12.10.1948), \emph{Schriftsteller/Schriftstellerin, Kritiker/Kritikerin}|pwk} wegen eines möglichen Treffens im Sommer 1901 kontaktieren (vgl. Arthur Schnitzler an Richard Beer-Hofmann, 17. 8. 1901). Am {[}20. 8. 1901{]} schrieb Kerr\pwindex{Kerr, Alfred 25.12.1867 – 12.10.1948@\textsc{Kerr, Alfred} (25.12.1867 – 12.10.1948), \emph{Schriftsteller/Schriftstellerin, Kritiker/Kritikerin}|pwk} an
                     Schnitzler, dass sie sich erst im September in Wien\oindex{Wien@\textbf{Wien}, \emph{A.ADM2}|pwk}
                  sehen könnten. Auf den Brief schrieb Schnitzler mit schwarzer Tinte: »\textsc{Luitpol\textcolor{gray}{ds}} 43\oindex{Luitpoldstrasse@\textbf{Luitpoldstraße}, \emph{Straße (K.STR)}|pw}« (\emph{CUL}, B 50).}}}\label{K_L03078-1}: \textsc{Luitpoldstraſse} 43\oindex{Luitpoldstrasse@\textbf{Luitpoldstraße}, \emph{Straße (K.STR)}|pw}.\pend
           
\pstart
           Viele Grüße an Alle!\pend
           \pstart \spacefill\mbox{P. G.}\pend{}\selectlanguage{ngerman}\endnumbering\briefempfaengerindex{Schnitzler, Arthur@\textsc{Schnitzler, Arthur}!zzzGoldmann, Paul@\emph{von Paul Goldmann}!1901-08-071@{7. 8. 1901}|)be}\mylabel{L03078h}  \normalsize

\doendnotes{C}
\bigskip
\vfill

\clearpage

\footnotesize

\lohead{\textsc{register}}

% Definiere theindex-Environment komplett neu ohne reledmac
\makeatletter
\renewenvironment{theindex}{%
  \section*{\indexname}%
  \setlength{\parindent}{0pt}%
  \setlength{\parskip}{0pt plus 0.3pt}%
  \let\item\@idxitem
}{%
  \clearpage
}
\makeatother

\IfFileExists{\jobname-pw.ind}{\input{\jobname-pw.ind}}{}

\end{document}

      