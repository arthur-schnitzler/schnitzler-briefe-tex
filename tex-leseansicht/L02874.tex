%% latex-korrekturansicht-vorspann.tex
%% Vorspann für die Korrekturansicht.
%% Lädt die gemeinsame Datei latex-vorspann.tex mit gesetztem Schalter.

\newif\ifkorrekturansicht
\korrekturansichttrue

\input{../tex-inputs/latex-vorspann}


\section[ Paul Goldmann an Arthur Schnitzler, 1. 5. 1899]{L02874 Paul Goldmann an Arthur Schnitzler, 1. 5. 1899}
\nopagebreak\mylabel{L02874v}
\rehead{ }\normalsize\beginnumbering\briefempfaengerindex{Schnitzler, Arthur@\textsc{Schnitzler, Arthur}!zzzGoldmann, Paul@\emph{von Paul Goldmann}!1899-05-012@{1. 5. 1899}|(be}
\toendnotes[C]{\smallbreak\pagebreak[2]}\Standort{DLA, A:Schnitzler, HS.NZ85.1.3169.}
\physDesc{Brief, 1 Blatt, 4 Seiten, 2209 Zeichen
\newline{}Handschrift: schwarze Tinte, deutsche Kurrent}\toendnotes[C]{\smallbreak}
\pstart
           {\pb}\textcolor{gray}{\textbf{\textbf{Frankfurter Zeitung}}}\orgindex{Frankfurter Zeitung@Frankfurter Zeitung|pw}\hfill \textcolor{gray}{\textbf{\textbf{Frankfurt a. M.\oindex{Frankfurt am Main@\textbf{Frankfurt am Main}, \emph{P.PPLA3}|pw},}}}{ }1. Mai \textcolor{gray}{\textbf{189}}9.\pend
           
\pstart
           \textcolor{gray}{\textbf{und}}\pend
           
\pstart
           \textcolor{gray}{\textbf{Handelsblatt.}}\pend
           
\pstart
           \textcolor{gray}{\textbf{\textbf{Redaktion\orgindex{Frankfurter Zeitung@Frankfurter Zeitung|pwv}.}\noindent{}\textcolor{gray}{\textbf{Für die Redaktion\orgindex{Frankfurter Zeitung@Frankfurter Zeitung|pwv} beſtimmte Briefe und Sendungen wolle man
                                 \so{nicht} an die Perſon eines Redakteurs,
                              ſondern ſtets \textbf{an die Redaktion der Frankfurter Zeitung\orgindex{Frankfurter Zeitung@Frankfurter Zeitung|pw}} adreſſiren.}}}}\pend
           
\pstart
           \textcolor{gray}{\textbf{Telegramm-Adreſſe:}}\pend
           
\pstart
           \textcolor{gray}{\textbf{\textbf{Zeitung\orgindex{Frankfurter Zeitung@Frankfurter Zeitung|pwv}{ }Frankfurt Main\oindex{Frankfurt am Main@\textbf{Frankfurt am Main}, \emph{P.PPLA3}|pw}.}}}\pend
           
\pstart{}Mein lieber Freund,\pend\vspace{0.5em}
\pstart
           Ich ſehe aus den hier eingetroffenen Berlin\oindex{Berlin@\textbf{Berlin}, \emph{P.PPLC}|pw}er
               Blättern, wie groß Dein \label{K_L02874-1v}\edtext{Erfolg\pwindex{gruene Kakadu – Paracelsus – Die Gefaehrtin. Drei Einakter@\emph{Der grüne Kakadu – Paracelsus – Die Gefährtin. Drei Einakter}|pwv}}{\lemma{\textnormal{\emph{Erfolg}}}\Cendnote{\textnormal{Vor allem \emph{Der grüne Kakadu}\pwindex{gruene Kakadu. Groteske in einem Akt@\emph{Der grüne Kakadu. Groteske in einem Akt}|pwk} wurde bei der Premiere von \emph{Der grüne Kakadu – Paracelsus – Die Gefährtin. Drei
                     Einakter}\pwindex{gruene Kakadu – Paracelsus – Die Gefaehrtin. Drei Einakter@\emph{Der grüne Kakadu – Paracelsus – Die Gefährtin. Drei Einakter}|pwk} am Deutschen Theater\oindex{Deutsches Theater Berlin@\textbf{Deutsches Theater Berlin}, \emph{Theater (K.THE)}|pwk} besonders
                  gut aufgenommen, vgl. A. S.: \emph{Tagebuch}, 29. 4. 1899.}}}\label{K_L02874-1} geweſen iſt, und beglückwünſche Dich nochmals von ganzem Herzen. Ich
               erwarte mir davon gute Wirkungen auf Deine Gemüthsverfaſſung, wenigſtens einen neuen
               Anſporn zur Arbeit. Daß Du alle die Dir geſpendeten Ehren \strikeout{als} im gegenwärtigen Moment als nutzlos empfindeſt, kann ich begreifen.
               Aber ich bin froh, daß Du in dieſen Tagen wenigſtens äußerlich mit etwas Anderem \strikeout{b\textcolor{gray}{e}} beſchäftigt geweſen biſt, als mit Deinem \label{K_L02874-2v}\edtext{Schmerz}{\lemma{\textnormal{\emph{Schmerz}}}\Cendnote{\textnormal{wegen Marie Reinhards\pwindex{Reinhard, Marie 1871-03-13 – 1899-03-18@\textsc{Reinhard, Marie} (1871-03-13 – 1899-03-18), \emph{Gesangspädagoge/Gesangspädagogin}|pwk} Tod am 18. 3. 1899}}}\label{K_L02874-2}; und auch dieſer wird und muß milder, weniger {\pb}blutig werden. Aber ſonſt, wie geſagt, iſt mir Deine
               Stimmung ſo \strikeout{\textcolor{gray}{×}} verſtändlich! Was Du \strikeout{\textcolor{gray}{h}\textcolor{gray}{×}\-\textcolor{gray}{×}\-\textcolor{gray}{×}} in dieſem Augenblick \strikeout{\textcolor{gray}{em}} empfindeſt, habe ich mein ganzes Leben lang gefühlt. Immer dieſe furchtbare
               Leere. Ich habe nie mit Jemandem theilen können, Dir aber war dieſes hohe Glück
               wenigſtens einige Jahre lang gegeben, und es wird Dir \strikeout{\textcolor{gray}{wie}} wieder beſchieden ſein. Ich habe zur Ausfüllung meiner Exiſtenz, zur
               Befriedigung all’ meiner Sehnſucht nie etwas gehabt, als meine Arbeit, – und welche
               Arbeit! Die Arbeit, an die ich früher geglaubt, mißachte ich jetzt, als etwas
               Gekünſteltes und Weſenloſes. Nur das Menſchliche hat Werth, – nur das, was wir
               leben.\pend
           
\pstart
           Ich hab’ mich ſelten ſo in Dein \strikeout{\textcolor{gray}{×}}{ }{\pb}Empfinden hineinverſetzen können, wie \strikeout{\textcolor{gray}{gege}} in dieſem Falle, und ich meine, wenn ich bei Dir wäre, könnte ich Dir \strikeout{\textcolor{gray}{m}} Manches Tröſtliche ſagen. Daß Du nicht nach Frankfurt\oindex{Frankfurt am Main@\textbf{Frankfurt am Main}, \emph{P.PPLA3}|pw} kommen magſt, bringt mir eine \substVorne{}\textsuperscript{\textcolor{gray}{×}\-\textcolor{gray}{×}\-\textcolor{gray}{×}\-\textcolor{gray}{×}}\substDazwischen{}ſchmerzliche\substHinten{} Enttäuſchung. Ich erfuhr heut{ }Morgen, daß ich Ende dieſer Woche nach Berlin\oindex{Berlin@\textbf{Berlin}, \emph{P.PPLC}|pw} gehen ſoll, und dachte einen Augenblick daran, Dirs zu telegraphiren
               und \introOben{}Dich\introOben{} zu bitten, \strikeout{daß} daß
               Du mich dort erwarteſt. Aber \strikeout{D\textcolor{gray}{ein}} als ich Deinen Brief bekam, entſchloß ich mich, lieber nicht zu telegraphiren;
               es wäre ja auch ohnedies nutzlos geweſen.\pend
           
\pstart
           Wenn Du jetzt \label{K_L02874-3v}\edtext{wieder in Wien\oindex{Wien@\textbf{Wien}, \emph{A.ADM2}|pw}}{\lemma{\textnormal{\emph{wieder in Wien}}}\Cendnote{\textnormal{Schnitzler kehrte am 2. 5. 1899 nach Wien\oindex{Wien@\textbf{Wien}, \emph{A.ADM2}|pwk} zurück.}}}\label{K_L02874-3}{ }\strikeout{b\textcolor{gray}{i}ſt} biſt, ſo quäle Dich
               wenigſtens nicht ſelbſt, wie Du es bisher gethan haſt. Beſonders {\pb}dieſe \label{K_L02874-4v}\edtext{Reiſe nach Graz\oindex{Graz@\textbf{Graz}, \emph{A.ADM2}|pw}}{\lemma{\textnormal{\emph{Reiſe nach Graz}}}\Cendnote{\textnormal{Siehe A. S.: \emph{Tagebuch}, 1. 4. 1899.
               }}}\label{K_L02874-4} war eine fürchterliche Geſchichte. Laß’ den Schmerz ſeinen natürlichen Lauf
               nehmen, wie Du als Arzt mit den Krankheiten thuſt, und behandle ihn nicht mit
               Gewaltkuren!\pend
           
\pstart
           \begin{otherlanguage}{french}\textsc{Adieu}\end{otherlanguage}, mein lieber Freund! {\\[\baselineskip]}Dein {\\[\baselineskip]}\spacefill\mbox{Paul Goldmann}\pend
           \leftskip=0em{}
\pstart
           \noindent{}Ich gehe nach Berlin\oindex{Berlin@\textbf{Berlin}, \emph{P.PPLC}|pw}, dann wahrſcheinlich nach
                  dem Haag\oindex{Den Haag@\textbf{Den Haag}, \emph{P.PPLG}|pw} zur \label{K_L02874-5v}\edtext{Friedens-Conferenz}{\lemma{\textnormal{\emph{Friedens-Conferenz}}}\Cendnote{\textnormal{Die Haag\oindex{Den Haag@\textbf{Den Haag}, \emph{P.PPLG}|pwk}er
                     Friedenskonferenz fand von 18. 5. 1899 bis 29. 7. 1899 statt.}}}\label{K_L02874-5}. \strikeout{B} Briefe erreichen mich ſtets über Frankfurt\oindex{Frankfurt am Main@\textbf{Frankfurt am Main}, \emph{P.PPLA3}|pw}.\pend
           \selectlanguage{ngerman}\endnumbering\briefempfaengerindex{Schnitzler, Arthur@\textsc{Schnitzler, Arthur}!zzzGoldmann, Paul@\emph{von Paul Goldmann}!1899-05-012@{1. 5. 1899}|)be}\mylabel{L02874h}  \normalsize

\doendnotes{C}
\bigskip
\vfill

\clearpage

\footnotesize

\lohead{\textsc{register}}

% Definiere theindex-Environment komplett neu ohne reledmac
\makeatletter
\renewenvironment{theindex}{%
  \section*{\indexname}%
  \setlength{\parindent}{0pt}%
  \setlength{\parskip}{0pt plus 0.3pt}%
  \let\item\@idxitem
}{%
  \clearpage
}
\makeatother

\IfFileExists{\jobname-pw.ind}{\input{\jobname-pw.ind}}{}

\end{document}

      