\input{../tex-inputs/latex-pdf-vorspann}
\begin{center}
            \textcolor{red}{ENTWURF. ENTZIFFERUNG NOCH NICHT KORREKTURGELESEN}
                      \end{center}
            
               \section[Adalbert Seligmann an Arthur Schnitzler, 25. 5. 1900]{ Adalbert Seligmann an Arthur Schnitzler, 25. 5. 1900}\nopagebreak\mylabel{v}\rehead{ }\begin{ledgroupsized}[t]{13cm}\normalsize\beginnumbering\briefempfaengerindex{Schnitzler, Arthur@\textsc{Schnitzler, Arthur}!zzzSeligmann, Adalbert Franz@\emph{von Adalbert Franz Seligmann}!1900-05-251@{25. 5. 1900}|(be} \toendnotes[C]{\smallbreak\pagebreak[2]} \Standort{TMW, HS Schn 4/61/1.}
\physDesc{Brief, 1 Blatt, 1 Seite
\newline{}Handschrift: schwarze Tinte, deutsche Kurrent
\newline{}Schnitzler: mit Bleistift beschriftet: »\textsc{Seligma{\geminationn}}« und nummeriert: »2« }\toendnotes[C]{\smallbreak}\pstart
           \raggedleft{}{\pb}25/5 1900\pend
           \pstart
           Verehrter Freund! Meinen beſten Dank für die Sendung\pwindex{Schnitzler, Arthur 15.05.1862 – 21.10.1931@\textsc{Schnitzler, Arthur} (15.05.1862 – 21.10.1931), \emph{Schriftsteller, Mediziner}!Reigen. Zehn Dialoge1900@\strich\emph{Reigen. Zehn Dialoge} {[}1900{]}|pwv}! Die Sachen ſind wirklich
                    reizend, und eine Fülle der verſchiedenſten Beobachtungen iſt darin, die aus
                    Mittheilungen nicht geſchöpft ſein können. Das muß man ſelber mitgemacht haben –
                    ich gratulire Ihnen noch nachträglich dazu!\pend
           \pstart
           Nochmals herzlich dankend{\\[\baselineskip]}Ihr{\\[\baselineskip]}\spacefill\mbox{Seligmann}\pend
           \leftskip=0em{}\endnumbering\briefempfaengerindex{Schnitzler, Arthur@\textsc{Schnitzler, Arthur}!zzzSeligmann, Adalbert Franz@\emph{von Adalbert Franz Seligmann}!1900-05-251@{25. 5. 1900}|)be}\mylabel{h}\end{ledgroupsized}  \newcommand{\dateiname}{L01040}\newcommand{\titel}{Adalbert Seligmann an Arthur Schnitzler, 25. 5. 1900}\newcommand{\editorInnen}{Martin Anton Müller und Gerd-Hermann Susen}\input{../tex-inputs/latex-pdf-abspann}
      