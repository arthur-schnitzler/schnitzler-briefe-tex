%% latex-leseansicht-vorspann.tex
%% Vorspann für die Leseansicht.
%% Lädt die gemeinsame Datei latex-vorspann.tex mit nicht gesetztem Schalter.

\newif\ifkorrekturansicht
\korrekturansichtfalse

\input{../tex-inputs/latex-vorspann}


         
         \newcommand{\erwaehntePersonen}{Personen: Richard Beer-Hofmann, Christine Schönberger}
         \newcommand{\erwaehnteOrte}{Orte: Brühl, Schloß Schönbrunn, Wien, XIX., Döbling}
         \newcommand{\erwaehnteWerke}{Werke: Das Märchen der 672. Nacht, Der neue Actäon}
               \section[Hugo von Hofmannsthal an Arthur Schnitzler, {[}28. 4. 1895{]}]{ Hugo von Hofmannsthal an Arthur Schnitzler, {[}28. 4. 1895{]}}\nopagebreak\mylabel{v}\rehead{ }\begin{ledgroupsized}[t]{13cm}\normalsize\beginnumbering \toendnotes[C]{\smallbreak\pagebreak[2]} \Standort{CUL, Schnitzler, B 43.}
\physDesc{Brief, 1 Blatt (Briefpapier mit aufgeprägtem Wappen), 3 Seiten
\newline{}Handschrift: schwarze Tinte, deutsche Kurrent
\newline{}Schnitzler: mit Bleistift datiert: »28/4 95« und nummeriert: »70« }\buchAbdrucke{\weitereDrucke{Hugo von Hofmannsthal, Arthur Schnitzler: \emph{Briefwechsel}. Hg. Therese Nickl und Heinrich Schnitzler. Frankfurt am Main: \emph{S. Fischer} 1964, S. 53.} }\pstart{}{\pb}mein lieber
                        Arthur,\pend\pstart
           ich mache die beſten Fortſchritte, fahre jeden Tag nach Schönbrunn\oindex{Schloss Schoenbrunn@\textbf{Schloß Schönbrunn}|pw} oder Döbling\oindex{XIX., Doebling@\textbf{XIX., Döbling}|pw}
                    und kann ſchon 1 ½ Stunden ohne Ermüdung gehen. Morgen bin ich durch Familie
                    occupiert. Übermorgen will ich ſchon in der Früh zur Tini\pwindex{Schoenberger, Christine 1875-11-17 – 1971-02-03@\textsc{Schönberger, Christine} (1875-11-17 – 1971-02-03), \emph{Gastwirtin}|pw} fahren, vielleicht {\pb}dort das Märchen\pwindex{Hofmannsthal, Hugo von 1874-02-01 – 1929-07-15@\textsc{Hofmannsthal, Hugo von} (1874-02-01 – 1929-07-15), \emph{Schriftsteller}!Maerchen der 672. Nacht2.11.1895 – 16.11.1895@\strich\emph{Das Märchen der 672. Nacht} {[}2.11.1895 – 16.11.1895{]}|pw} fertigſchreiben oder wenn das ſchon fertig wäre,
                    eine Geſchichte des Actäon\pwindex{Hofmannsthal, Hugo von 1874-02-01 – 1929-07-15@\textsc{Hofmannsthal, Hugo von} (1874-02-01 – 1929-07-15), \emph{Schriftsteller}!neue Actaeon1978@\strich\emph{Der neue Actäon} {[}1978{]}|pw} anfangen. Ich hab
                    dem Richard\pwindex{Beer-Hofmann, Richard 1866-07-11 – 1945-09-26@\textsc{Beer-Hofmann, Richard} (1866-07-11 – 1945-09-26), \emph{Schriftsteller}|pw} geſchrieben, ob er mir nicht
                    nachfahren will. Es wär ſehr ſchön, wenn Sie mit ihm ſich über ſo etwas einigen
                    würden. Den Nachmittag könnten wir dann wo anders hin, von der Brühl\oindex{Bruehl@\textbf{Brühl}|pw} aus.\pend
           \pstart
           {\pb}Jedenfalls rechne ich
                    darauf, mit Ihnen in der nächſten Woche mindeſtens einen Abend zu
                    verbringen.\pend
           \pstart
           Herzlich{\\[\baselineskip]} Ihr{\\[\baselineskip]}\spacefill\mbox{Hugo.}\pend
           \leftskip=0em{}
         
         \endnumbering\mylabel{h}\end{ledgroupsized}  \newcommand{\dateiname}{L00434}\newcommand{\titel}{Hugo von Hofmannsthal an Arthur Schnitzler, [28. 4. 1895]}\newcommand{\editorInnen}{Martin Anton Müller und Gerd-Hermann Susen}%% latex-leseansicht-abspann.tex
%% Abspann für die Leseansicht.
%% Der Schalter \ifkorrekturansicht ist bereits durch den Vorspann gesetzt.

%% latex-abspann.tex
%% Gemeinsamer Abspann für Korrekturansicht und Leseansicht.
%% Setzt den Schalter \ifkorrekturansicht voraus (gesetzt in den
%% einbindenden Dateien latex-korrekturansicht-abspann.tex bzw.
%% latex-leseansicht-abspann.tex).
%% ---------------------------------------------------------------

\normalsize

% Das esempio-Environment wird nur in der Leseansicht benötigt
\ifkorrekturansicht\else
\newenvironment{esempio}[3]%
{
    \vspace{1.5ex}
    \rlap{\underline{#1}}
    \par
    \setlength{\parindent}{0cm}
    \nopagebreak
    \leftskip=#2cm
    \rightskip=#3cm
}
{
    \par
}
\fi

\doendnotes{C}
\bigskip
\vfill

\clearpage

\footnotesize

\ifkorrekturansicht
  \lohead{\textsc{register}}
\fi

% theindex-Environment neu definieren ohne reledmac
\makeatletter
\renewenvironment{theindex}{%
  \ifkorrekturansicht
    \section*{\indexname}%
  \else
    \subsubsection*{Index der erwähnten Entitäten}%
  \fi
  \setlength{\parindent}{0pt}%
  \setlength{\parskip}{0pt plus 0.3pt}%
  \let\item\@idxitem
}{%
  \ifkorrekturansicht\clearpage\fi
}
\makeatother

\IfFileExists{\jobname-pw.ind}{\input{\jobname-pw.ind}}{}

% Quellenangabe nur in der Leseansicht
\ifkorrekturansicht\else
% Fallback-Definitionen, falls die .tex-Datei \titel etc. nicht gesetzt hat
\providecommand{\titel}{}
\providecommand{\editorInnen}{}
\providecommand{\dateiname}{\jobname}

\vspace{3cm}

\vfill

\footnotesize
\textsc{Quelle}: \titel. Herausgegeben von {\editorInnen}. In: \emph{Arthur Schnitzler: Briefwechsel mit Autorinnen und Autoren}.
 Digitale Edition, https://schnitzler-briefe.acdh.oeaw.ac.at/{\dateiname}.html (Stand \today)
\fi

\end{document}


      