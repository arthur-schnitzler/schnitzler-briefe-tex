%% latex-leseansicht-vorspann.tex
%% Vorspann für die Leseansicht.
%% Lädt die gemeinsame Datei latex-vorspann.tex mit nicht gesetztem Schalter.

\newif\ifkorrekturansicht
\korrekturansichtfalse

\input{../tex-inputs/latex-vorspann}


\section[Hugo von Hofmannsthal an Arthur Schnitzler, {[}28. 4. 1895{]}]{L00434 Hugo von Hofmannsthal an Arthur Schnitzler, {[}28. 4. 1895{]}}
\nopagebreak\mylabel{L00434v}
\rehead{ }\normalsize\beginnumbering\briefempfaengerindex{Schnitzler, Arthur@\textsc{Schnitzler, Arthur}!zzzHofmannsthal, Hugo von@\emph{von Hugo von Hofmannsthal}!1895-04-282@{{[}28. 4. 1895{]}}|(be}
\toendnotes[C]{\smallbreak\pagebreak[2]}
\correspDesc{Versand  durch Hugo von Hofmannsthal am [28. 4. 1895] in Wien
\newline{}Erhalt  durch Arthur Schnitzler im Zeitraum [28. 4. 1895
                  – 2. 5. 1895?] in Wien}\toendnotes[C]{\smallbreak}
\Standort{CUL, Schnitzler, B 43.}
\physDesc{Brief, 1 Blatt, 3 Seiten, 658 Zeichen (aufgeprägtes Wappen)
\newline{}Handschrift: schwarze Tinte, deutsche Kurrent
\newline{}Schnitzler: mit Bleistift datiert: »28/4 95« und nummeriert: »70« }
\buchAbdrucke{\weitereDrucke{Hugo von Hofmannsthal, Arthur Schnitzler: \emph{Briefwechsel}. Herausgegeben von Therese Nickl und Heinrich Schnitzler. Frankfurt am Main: \emph{S. Fischer} 1964, S. 53.} }
\pstart{}{\pb}mein lieber Arthur,\pend\vspace{0.5em}
\pstart
           ich mache die beſten Fortſchritte, fahre jeden Tag nach Schönbrunn\oindex{Wien@\textbf{Wien}!XIII., Hietzing@\textbf{XIII., Hietzing}!Schloss Schönbrunn@\textbf{Schloss Schönbrunn}, \emph{Schloss}|pw} oder Döbling\oindex{XIX., Döbling@\textbf{XIX., Döbling}, \emph{Verwaltungsgebiet}|pw} und
               kann{ }ſchon 1 ½ Stunden ohne Ermüdung gehen. Morgen bin ich durch Familie occupiert.
               Übermorgen will ich{ }ſchon in der Früh zur Tini\pwindex{Kepert, Christine 17.\,11.\,1875 – 3.\,2.\,1971 Wien@\textsc{Kepert, Christine} (17.\,11.\,1875 – 3.\,2.\,1971 Wien), \emph{Gastwirtin}|pw}
               fahren, vielleicht {\pb}dort das Märchen\pwindex{Hofmannsthal, Hugo von 1.\,2.\,1874 Wien – 15.\,7.\,1929 Rodaun@\textsc{Hofmannsthal, Hugo von} (1.\,2.\,1874 Wien – 15.\,7.\,1929 Rodaun), \emph{Schriftsteller}!Märchen der 672. Nacht@\strich\emph{Das Märchen der 672. Nacht}|pw} fertigſchreiben oder wenn das{ }ſchon
               fertig wäre, eine Geſchichte des Actäon\pwindex{Hofmannsthal, Hugo von 1.\,2.\,1874 Wien – 15.\,7.\,1929 Rodaun@\textsc{Hofmannsthal, Hugo von} (1.\,2.\,1874 Wien – 15.\,7.\,1929 Rodaun), \emph{Schriftsteller}!neue Actäon@\strich\emph{Der neue Actäon}|pw}
               anfangen. Ich hab dem Richard\pwindex{Beer-Hofmann, Richard 11.\,7.\,1866 Wien – 26.\,9.\,1945 New York City@\textsc{Beer-Hofmann, Richard} (11.\,7.\,1866 Wien – 26.\,9.\,1945 New York City), \emph{Schriftsteller}|pw} geſchrieben, ob
               er mir nicht nachfahren will. Es wär{ }ſehr{ }ſchön, wenn Sie mit ihm{ }ſich über{ }ſo etwas
               einigen würden. Den Nachmittag könnten wir dann wo anders hin, von der Brühl\oindex{Brühl@\textbf{Brühl}, \emph{Tal}|pw} aus.\pend
           
\pstart
           {\pb}Jedenfalls rechne ich darauf, mit
               Ihnen in der nächſten Woche mindeſtens einen Abend zu verbringen.\pend
           
\pstart
           Herzlich{\\[\baselineskip]} Ihr{\\[\baselineskip]}\spacefill\mbox{Hugo.}\pend
           \leftskip=0em{}\selectlanguage{ngerman}\endnumbering\briefempfaengerindex{Schnitzler, Arthur@\textsc{Schnitzler, Arthur}!zzzHofmannsthal, Hugo von@\emph{von Hugo von Hofmannsthal}!1895-04-282@{{[}28. 4. 1895{]}}|)be}\mylabel{L00434h}  \newcommand{\dateiname}{L00434}\newcommand{\titel}{Hugo von Hofmannsthal an Arthur Schnitzler, [28. 4. 1895]}\newcommand{\editorInnen}{Martin Anton Müller und Gerd-Hermann Susen}%% latex-leseansicht-abspann.tex
%% Abspann für die Leseansicht.
%% Der Schalter \ifkorrekturansicht ist bereits durch den Vorspann gesetzt.

%% latex-abspann.tex
%% Gemeinsamer Abspann für Korrekturansicht und Leseansicht.
%% Setzt den Schalter \ifkorrekturansicht voraus (gesetzt in den
%% einbindenden Dateien latex-korrekturansicht-abspann.tex bzw.
%% latex-leseansicht-abspann.tex).
%% ---------------------------------------------------------------

\normalsize

% Das esempio-Environment wird nur in der Leseansicht benötigt
\ifkorrekturansicht\else
\newenvironment{esempio}[3]%
{
    \vspace{1.5ex}
    \rlap{\underline{#1}}
    \par
    \setlength{\parindent}{0cm}
    \nopagebreak
    \leftskip=#2cm
    \rightskip=#3cm
}
{
    \par
}
\fi

\doendnotes{C}
\bigskip
\vfill

\clearpage

\footnotesize

\ifkorrekturansicht
  \lohead{\textsc{register}}
\fi

% theindex-Environment neu definieren ohne reledmac
\makeatletter
\renewenvironment{theindex}{%
  \ifkorrekturansicht
    \section*{\indexname}%
  \else
    \subsubsection*{Index der erwähnten Entitäten}%
  \fi
  \setlength{\parindent}{0pt}%
  \setlength{\parskip}{0pt plus 0.3pt}%
  \let\item\@idxitem
}{%
  \ifkorrekturansicht\clearpage\fi
}
\makeatother

\IfFileExists{\jobname-pw.ind}{\input{\jobname-pw.ind}}{}

% Quellenangabe nur in der Leseansicht
\ifkorrekturansicht\else
% Fallback-Definitionen, falls die .tex-Datei \titel etc. nicht gesetzt hat
\providecommand{\titel}{}
\providecommand{\editorInnen}{}
\providecommand{\dateiname}{\jobname}

\vspace{3cm}

\vfill

\footnotesize
\textsc{Quelle}: \titel. Herausgegeben von {\editorInnen}. In: \emph{Arthur Schnitzler: Briefwechsel mit Autorinnen und Autoren}.
 Digitale Edition, https://schnitzler-briefe.acdh.oeaw.ac.at/{\dateiname}.html (Stand \today)
\fi

\end{document}


