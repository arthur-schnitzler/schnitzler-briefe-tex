%% latex-korrekturansicht-vorspann.tex
%% Vorspann für die Korrekturansicht.
%% Lädt die gemeinsame Datei latex-vorspann.tex mit gesetztem Schalter.

\newif\ifkorrekturansicht
\korrekturansichttrue

\input{../tex-inputs/latex-vorspann}


\section[Hugo von Hofmannsthal an Arthur Schnitzler, {[}28. 4. 1895{]}]{L00434 Hugo von Hofmannsthal an Arthur Schnitzler, {[}28. 4. 1895{]}}
\nopagebreak\mylabel{L00434v}
\rehead{ }\normalsize\beginnumbering\briefempfaengerindex{Schnitzler, Arthur@\textsc{Schnitzler, Arthur}!zzzHofmannsthal, Hugo von@\emph{von Hugo von Hofmannsthal}!1895-04-282@{{[}28. 4. 1895{]}}|(be}
\toendnotes[C]{\smallbreak\pagebreak[2]}\Standort{CUL, Schnitzler, B 43.}
\physDesc{Brief, 1 Blatt, 3 Seiten, 658 Zeichen (aufgeprägtes Wappen)
\newline{}Handschrift: schwarze Tinte, deutsche Kurrent
\newline{}Schnitzler: mit Bleistift datiert: »28/4 95« und nummeriert: »70« }
\buchAbdrucke{\weitereDrucke{Hugo von Hofmannsthal, Arthur Schnitzler: \emph{Briefwechsel}. Frankfurt am Main: \emph{S. Fischer} 1964, S. 53.} }
\pstart{}{\pb}mein lieber Arthur,\pend\vspace{0.5em}
\pstart
           ich mache die beſten Fortſchritte, fahre jeden Tag nach Schönbrunn\oindex{Schloss Schoenbrunn@\textbf{Schloss Schönbrunn}, \emph{Schloss (K.SLS)}|pw} oder Döbling\oindex{XIX., Doebling@\textbf{XIX., Döbling}, \emph{A.ADM3}|pw} und
               kann ſchon 1 ½ Stunden ohne Ermüdung gehen. Morgen bin ich durch Familie occupiert.
               Übermorgen will ich ſchon in der Früh zur Tini\pwindex{Schoenberger, Christine 1875-11-17 – 1971-02-03@\textsc{Schönberger, Christine} (1875-11-17 – 1971-02-03), \emph{Gastwirt/Gastwirtin}|pw}
               fahren, vielleicht {\pb}dort das Märchen\pwindex{Maerchen der 672. Nacht@\emph{Das Märchen der 672. Nacht}|pw} fertigſchreiben oder wenn das ſchon
               fertig wäre, eine Geſchichte des Actäon\pwindex{neue Actaeon@\emph{Der neue Actäon}|pw}
               anfangen. Ich hab dem Richard\pwindex{Beer-Hofmann, Richard 1866-07-11 – 1945-09-26@\textsc{Beer-Hofmann, Richard} (1866-07-11 – 1945-09-26), \emph{Schriftsteller/Schriftstellerin}|pw} geſchrieben, ob
               er mir nicht nachfahren will. Es wär ſehr ſchön, wenn Sie mit ihm ſich über ſo etwas
               einigen würden. Den Nachmittag könnten wir dann wo anders hin, von der Brühl\oindex{Bruehl@\textbf{Brühl}, \emph{Tal (N.TAL)}|pw} aus.\pend
           
\pstart
           {\pb}Jedenfalls rechne ich darauf, mit
               Ihnen in der nächſten Woche mindeſtens einen Abend zu verbringen.\pend
           
\pstart
           Herzlich{\\[\baselineskip]} Ihr{\\[\baselineskip]}\spacefill\mbox{Hugo.}\pend
           \leftskip=0em{}\selectlanguage{ngerman}\endnumbering\briefempfaengerindex{Schnitzler, Arthur@\textsc{Schnitzler, Arthur}!zzzHofmannsthal, Hugo von@\emph{von Hugo von Hofmannsthal}!1895-04-282@{{[}28. 4. 1895{]}}|)be}\mylabel{L00434h}  \normalsize

\doendnotes{C}
\bigskip
\vfill

\clearpage

\footnotesize

\lohead{\textsc{register}}

% Definiere theindex-Environment komplett neu ohne reledmac
\makeatletter
\renewenvironment{theindex}{%
  \section*{\indexname}%
  \setlength{\parindent}{0pt}%
  \setlength{\parskip}{0pt plus 0.3pt}%
  \let\item\@idxitem
}{%
  \clearpage
}
\makeatother

\IfFileExists{\jobname-pw.ind}{\input{\jobname-pw.ind}}{}

\end{document}

      