%% latex-leseansicht-vorspann.tex
%% Vorspann für die Leseansicht.
%% Lädt die gemeinsame Datei latex-vorspann.tex mit nicht gesetztem Schalter.

\newif\ifkorrekturansicht
\korrekturansichtfalse

\input{../tex-inputs/latex-vorspann}


\section[Hugo Hofmannsthal an Arthur Schnitzler, 3. 6. 1929]{L02509 Hugo Hofmannsthal an Arthur Schnitzler, 3. 6. 1929}
\nopagebreak\mylabel{L02509v}
\rehead{ }\normalsize\beginnumbering\briefempfaengerindex{Schnitzler, Arthur@\textsc{Schnitzler, Arthur}!zzzHofmannsthal, Hugo von@\emph{von Hugo von Hofmannsthal}!1929-06-031@{3. 6. 1929}|(be}
\toendnotes[C]{\smallbreak\pagebreak[2]}
\correspDesc{Versand  durch Hugo von Hofmannsthal am 3. 6. 1929 in Rodaun
\newline{}Erhalt  durch Arthur Schnitzler im Zeitraum [4. 6. 1929
                  – 8. 6. 1929?] in Wien}\toendnotes[C]{\smallbreak}
\Standort{CUL, Schnitzler, B 43.}
\physDesc{Brief, 1 Blatt, 2 Seiten, 2084 Zeichen
\newline{}Handschrift: schwarze Tinte, lateinische Kurrent
\newline{}Schnitzler: mit rotem Buntstift mehrere Unterstreichungen 
\newline{}Ordnung: 1) mit Bleistift von unbekannter Hand nummeriert: »\strikeout{372}«  2) mit Bleistift von unbekannter Hand nummeriert:
                                    »381«}
\buchAbdrucke{\weitereDrucke{Hugo von Hofmannsthal, Arthur Schnitzler: \emph{Briefwechsel}. Herausgegeben von Therese Nickl und Heinrich Schnitzler. Frankfurt am Main: \emph{S. Fischer} 1964, S. 312.} }\toendnotes[C]{\smallbreak}
\pstart
           {\pb}Rodaun\oindex{Wien@\textbf{Wien}!XXIII., Liesing@\textbf{XXIII., Liesing}!Rodaun@\textbf{Rodaun}, \emph{Region}|pw}{ }3. Juni 29\pend
           
\pstart{}mein lieber Arthur,\pend\vspace{0.5em}
\pstart
           so waren Sie also in der Zwischenzeit nicht verreist. Sie haben den Besuch Ihres Schwiegersohnes\pwindex{Cappellini, Arnoldo 10.\,8.\,1889 Venedig – 8.\,5.\,1954 Rom@\textsc{Cappellini, Arnoldo} (10.\,8.\,1889 Venedig – 8.\,5.\,1954 Rom)|pwv} hier
               empfangen, statt mit ihm zu reisen, \strikeout{und} Sie waren
               eine Woche lang recht unwohl, sind aber gottlob wieder völlig davon hergestellt –
               dies alles, wenn ich den Bericht der guten Freundin B. Z.\pwindex{Zuckerkandl, Berta 13.\,4.\,1864 Wien – 16.\,10.\,1945 Paris@\textsc{Zuckerkandl, Berta} (13.\,4.\,1864 Wien – 16.\,10.\,1945 Paris), \emph{Schriftstellerin, Journalistin, Übersetzerin}|pw} recht verstehe.\pend
           
\pstart
           Ich war 14 Tage, genau 13 Tage, in Italien\oindex{Italien@\textbf{Italien}|pw}, bis
               gegen Rom\oindex{Rom@\textbf{Rom}, \emph{Hauptstadt}|pw} hin, ohne das eigentlich röm\oindex{Rom@\textbf{Rom}, \emph{Hauptstadt}|pw}ische Gebiet zu berühren. Es waren sehr schöne
               Tage.\pend
           
\pstart
           Vor dem Wegfahren las ich sehr viel in Ihren Sachen, erzählendes u. dramatisches \substVorne{}\textsuperscript{D}\substDazwischen{}d\substHinten{}urcheinander, alles mit dem größten Vergnügen. Ja, so gut Leutnant Gustl\pwindex{Schnitzler, Arthur 15.\,5.\,1862 Wien – 21.\,10.\,1931 ebd.@\textsc{Schnitzler, Arthur} (15.\,5.\,1862 Wien – 21.\,10.\,1931 ebd.), \emph{Schriftsteller, Mediziner}!Lieutenant Gustl. Novelle@\strich\emph{Lieutenant Gustl. Novelle}|pw} erzählt ist, »Fräulein Else\pwindex{Schnitzler, Arthur 15.\,5.\,1862 Wien – 21.\,10.\,1931 ebd.@\textsc{Schnitzler, Arthur} (15.\,5.\,1862 Wien – 21.\,10.\,1931 ebd.), \emph{Schriftsteller, Mediziner}!Fräulein Else@\strich\emph{Fräulein Else}|pw}« schlägt ihn freilich noch; das ist i{\geminationn}erhalb der deutschen Literatur wirklich ein genre für
               sich, das Sie geschaffen haben. Sehr großen Eindruck machte mir auch der »Einsame Weg\pwindex{Schnitzler, Arthur 15.\,5.\,1862 Wien – 21.\,10.\,1931 ebd.@\textsc{Schnitzler, Arthur} (15.\,5.\,1862 Wien – 21.\,10.\,1931 ebd.), \emph{Schriftsteller, Mediziner}!einsame Weg. Schauspiel in fünf Akten@\strich\emph{Der einsame Weg. Schauspiel in fünf Akten}|pw}«; so wenige Figuren eigentlich, und
               ein so großer Reichtum erreicht. Den Roman\pwindex{Schnitzler, Arthur 15.\,5.\,1862 Wien – 21.\,10.\,1931 ebd.@\textsc{Schnitzler, Arthur} (15.\,5.\,1862 Wien – 21.\,10.\,1931 ebd.), \emph{Schriftsteller, Mediziner}!Weg ins Freie. Roman@\strich\emph{Der Weg ins Freie. Roman}|pwv} habe ich {\pb}auch wieder gelesen, so wie Sie es
               vorschlugen, von Capitel V bis zum Ende. Aber ich habe diese Arbeit nun einmal
               weniger gern, und ich kö{\geminationn}te es auch begründen. Die
               Einwände begi{\geminationn}en bei der Hauptfigur, die mir nicht ganz
               consistent erscheint (ihr Äußeres und Inneres nicht ganz übereinsti{\geminationm}end) – aber der Haupteinwand geht tiefer. Aber darüber
               müsste man sich, wenn überhaupt, mündlich \substVorne{}\textsuperscript{sprechen}\substDazwischen{}unterhalten\substHinten{}. – Vor ein paar Tagen, gegen Abend, kam ich zurück, wollte \introOben{}mir\introOben{} irgend ein Buch suchen, und griff wieder nach einem von
               Ihnen: nach den Dämmerseelen\pwindex{Schnitzler, Arthur 15.\,5.\,1862 Wien – 21.\,10.\,1931 ebd.@\textsc{Schnitzler, Arthur} (15.\,5.\,1862 Wien – 21.\,10.\,1931 ebd.), \emph{Schriftsteller, Mediziner}!Dämmerseelen. Novellen@\strich\emph{Dämmerseelen. Novellen}|pw}, und las dann alle
               5 oder 6 Geschichten mit der größten Bewunderung. Dieser schwebende Ton und diese
               bezaubernde Leichtigkeit (\damage{\textcolor{gray}{nicht}} ohne Unheimlichkeit dabei) gehört wirklich nur Ihnen. Vielleicht ist dies,
               alles in allem, Ihr meisterhaftestes Buch; aber man soll keine Censuren austeilen. –
               Ich möchte Sie so gerne bald \label{K_L02509-1v}\edtext{wiedersehen}{\lemma{\textnormal{\emph{wiedersehen}}}\Cendnote{\textnormal{In Folge fand das letzte
                  Treffen der beiden statt, vgl. A. S.: \emph{Tagebuch}, 11. 6. 1929.
               }}}\label{K_L02509-1}. B. Z.\pwindex{Zuckerkandl, Berta 13.\,4.\,1864 Wien – 16.\,10.\,1945 Paris@\textsc{Zuckerkandl, Berta} (13.\,4.\,1864 Wien – 16.\,10.\,1945 Paris), \emph{Schriftstellerin, Journalistin, Übersetzerin}|pw}{ }sagt mir, Sie fahren gerne Auto. Kann ich Sie nicht
               abholen, für einen halben Tag, – vor- oder nachmittag oder wie es Ihnen passt? Ich
               brauche nicht zu sagen, dass es mir die größte Freude machen würde. Rufen Sie
               vielleicht einmal zwischen 9\textsuperscript{h} und 10\textsuperscript{h}{ }Rodaun\oindex{Wien@\textbf{Wien}!XXIII., Liesing@\textbf{XXIII., Liesing}!Rodaun@\textbf{Rodaun}, \emph{Region}|pw} N. 3 an?\pend
           \pstart Von Herzen Ihr\spacefill\mbox{Hugo.}\pend{}\selectlanguage{ngerman}\endnumbering\briefempfaengerindex{Schnitzler, Arthur@\textsc{Schnitzler, Arthur}!zzzHofmannsthal, Hugo von@\emph{von Hugo von Hofmannsthal}!1929-06-031@{3. 6. 1929}|)be}\mylabel{L02509h}  \newcommand{\dateiname}{L02509}\newcommand{\titel}{Hugo Hofmannsthal an Arthur Schnitzler, 3. 6. 1929}\newcommand{\editorInnen}{Martin Anton Müller und Gerd-Hermann Susen}%% latex-leseansicht-abspann.tex
%% Abspann für die Leseansicht.
%% Der Schalter \ifkorrekturansicht ist bereits durch den Vorspann gesetzt.

%% latex-abspann.tex
%% Gemeinsamer Abspann für Korrekturansicht und Leseansicht.
%% Setzt den Schalter \ifkorrekturansicht voraus (gesetzt in den
%% einbindenden Dateien latex-korrekturansicht-abspann.tex bzw.
%% latex-leseansicht-abspann.tex).
%% ---------------------------------------------------------------

\normalsize

% Das esempio-Environment wird nur in der Leseansicht benötigt
\ifkorrekturansicht\else
\newenvironment{esempio}[3]%
{
    \vspace{1.5ex}
    \rlap{\underline{#1}}
    \par
    \setlength{\parindent}{0cm}
    \nopagebreak
    \leftskip=#2cm
    \rightskip=#3cm
}
{
    \par
}
\fi

\doendnotes{C}
\bigskip
\vfill

\clearpage

\footnotesize

\ifkorrekturansicht
  \lohead{\textsc{register}}
\fi

% theindex-Environment neu definieren ohne reledmac
\makeatletter
\renewenvironment{theindex}{%
  \ifkorrekturansicht
    \section*{\indexname}%
  \else
    \subsubsection*{Index der erwähnten Entitäten}%
  \fi
  \setlength{\parindent}{0pt}%
  \setlength{\parskip}{0pt plus 0.3pt}%
  \let\item\@idxitem
}{%
  \ifkorrekturansicht\clearpage\fi
}
\makeatother

\IfFileExists{\jobname-pw.ind}{\input{\jobname-pw.ind}}{}

% Quellenangabe nur in der Leseansicht
\ifkorrekturansicht\else
% Fallback-Definitionen, falls die .tex-Datei \titel etc. nicht gesetzt hat
\providecommand{\titel}{}
\providecommand{\editorInnen}{}
\providecommand{\dateiname}{\jobname}

\vspace{3cm}

\vfill

\footnotesize
\textsc{Quelle}: \titel. Herausgegeben von {\editorInnen}. In: \emph{Arthur Schnitzler: Briefwechsel mit Autorinnen und Autoren}.
 Digitale Edition, https://schnitzler-briefe.acdh.oeaw.ac.at/{\dateiname}.html (Stand \today)
\fi

\end{document}


