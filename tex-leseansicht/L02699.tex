%% latex-leseansicht-vorspann.tex
%% Vorspann für die Leseansicht.
%% Lädt die gemeinsame Datei latex-vorspann.tex mit nicht gesetztem Schalter.

\newif\ifkorrekturansicht
\korrekturansichtfalse

\input{../tex-inputs/latex-vorspann}


         
         \renewcommand{\erwaehntePersonen}{Personen:  ?? [Dänischer Maler in Paris, 1892], Marie Glümer, Paul Goldmann, Arthur Klein, Leopold Sonnemann}
         \renewcommand{\erwaehnteInstitutionen}{Institutionen: Frankfurter Zeitung, Neues Deutsches Theater}
         \renewcommand{\erwaehnteOrte}{Orte: Berlin, Deutschland, Dänemark, Frankreich, Paris, Prag, Wien, rue Richelieu}
         \renewcommand{\erwaehnteWerke}{Werke: Das Märchen. Schauspiel in drei Aufzügen, Tagebuch}
               \section[Paul Goldmann an Arthur Schnitzler, 27. 6. {[}1892{]}]{ Paul Goldmann an Arthur Schnitzler, 27. 6. {[}1892{]}}\nopagebreak\mylabel{v}\rehead{ }\begin{ledgroupsized}[t]{13cm}\normalsize\beginnumbering\briefempfaengerindex{Schnitzler, Arthur@\textsc{Schnitzler, Arthur}!zzzGoldmann, Paul@\emph{von Paul Goldmann}!1892-06-271@{27. 6. {[}1892{]}}|(be} \toendnotes[C]{\smallbreak\pagebreak[2]} \Standort{DLA, A:Schnitzler, HS.NZ85.1.3163.}
\physDesc{Brief, 2 Blätter, 6 Seiten, 2691 Zeichen
\newline{}Handschrift: schwarze Tinte, deutsche Kurrent
\newline{}Schnitzler: mit Bleistift das Jahr »92« vermerkt }\toendnotes[C]{\smallbreak}\pstart
           \noindent{}{\pb}\textcolor{gray}{\textbf{Frankfurter Zeitung\orgindex{Frankfurter Zeitung@Frankfurter Zeitung|pw}.}}\pend
           \pstart
           \textcolor{gray}{\textbf{(Gazette de
                     Francfort\orgindex{Frankfurter Zeitung@Frankfurter Zeitung|pw}.)}}\pend
           \pstart
           \textcolor{gray}{\textbf{\begin{otherlanguage}{french}Directeur\end{otherlanguage}: \textbf{M. L. Sonnemann\pwindex{Sonnemann, Leopold 1831-10-29 – 1909-10-30@\textsc{Sonnemann, Leopold} (1831-10-29 – 1909-10-30), \emph{Journalist, Herausgeber}|pw}}.}}\hfill \textsc{Paris\oindex{Paris@\textbf{Paris}|pw}}, 27. Juni.\pend
           \pstart
           \textcolor{gray}{\textbf{\begin{otherlanguage}{french}Journal politique, financier,\end{otherlanguage}}}\pend
           \pstart
           \textcolor{gray}{\textbf{\begin{otherlanguage}{french}commercial et litteraire.\end{otherlanguage}}}\pend
           \pstart
           \textcolor{gray}{\textbf{\begin{otherlanguage}{french}\textbf{Paraissant trois fois par jour}\end{otherlanguage}}}\pend
           \pstart
           \textcolor{gray}{\textbf{\begin{otherlanguage}{french}\textbf{Bureaux à Paris\oindex{Paris@\textbf{Paris}|pw}:}\end{otherlanguage}}}\pend
           \pstart
           \textcolor{gray}{\textbf{\begin{otherlanguage}{french}\textbf{rue Richelieu 75\oindex{rue Richelieu@\textbf{rue Richelieu}|pw}.}\end{otherlanguage}}}\pend
           \pstart\center{}Mein lieber Arthur!\pend\pstart
           Mir ſcheint, wir haben uns im ſelben Moment hingeſetzt, um aneinander zu ſchreiben.
               Auch das ſoll als ein liebes Zeichen genommen werden. Wie unendlich, aus tiefſtem
               Herzen froh Du mich mit Deinem Brief gemacht haſt, kann ich Dir nicht ſagen. Ich bin
               ſo ſtolz, ſo ſtolz auf dieſe treue Freundſchaft, die Du mir entgegenbringſt. Und das
               iſt das einzige wirkliche Gut, das mir das Leben bisher geboten. Ich habe heut wieder einmal {\pb}nach
               langer Zeit ein warmes Aufwallen von Glück im Herzen gehabt und danke das Dir. Oh{\dots} doch laſſen wir die Gefühle. Mein Privatleben verlange
               nicht zu wiſſen. Ich wüßte auch nicht, wie ich es Dir ſchildern ſollte in ſeiner Öde
               und Verlaſſenheit. Ich bin ein armer einſamer Narr, und betrinke mich an Arbeit, um
               das auf Stunden zu vergeſſen – mein bewährtes Recept. Verkehr außer \textsc{Arthur Klein\pwindex{Klein, Arthur 27.11.1868 – 28.07.1943@\textsc{Klein, Arthur} (27.11.1868 – 28.07.1943)|pw}} nur ein ſeltſamer \label{K_L02699-1v}\edtext{Burſch\pwindex{?? [Daenischer Maler in Paris, 1892] @\textsc{?? [Dänischer Maler in Paris, 1892]}|pwv}}{\lemma{\textnormal{\emph{Burſch}}}\Cendnote{\textnormal{nicht identifiziert}}}\label{K_L02699-1h} von einem dän\oindex{Daenemark@\textbf{Dänemark}|pwv}iſchen Maler\pwindex{?? [Daenischer Maler in Paris, 1892] @\textsc{?? [Dänischer Maler in Paris, 1892]}|pwv}, viel mehr Millionärsſohn\pwindex{?? [Daenischer Maler in Paris, 1892] @\textsc{?? [Dänischer Maler in Paris, 1892]}|pwv}, der gern großer Künſtler
               werden möchte und an ſeinem Dilettantismus {\pb}und an
               unglücklicher Liebe zugrunde geht. Seltſamer, ſehr lieber Menſch\pwindex{?? [Daenischer Maler in Paris, 1892] @\textsc{?? [Dänischer Maler in Paris, 1892]}|pwv}, der ſich zweifellos in den nächſten
               Jahren erſchießen wird. Um ihn herum ein oder zwei \label{K_L02699-2v}\edtext{Freunde}{\lemma{\textnormal{\emph{Freunde}}}\Cendnote{\textnormal{nicht
                  identifiziert}}}\label{K_L02699-2h}, auch deutſch\oindex{Deutschland@\textbf{Deutschland}|pwv}e Millionärsſöhne, gutmüthig, mit künſtleriſchen Inſpirationen,
               inoffenſiv. \textsc{Arthur Schnitzler} iſt in dieſem Kreiſe ein
               bekannter Begriff; ich leſe Dich vor, ich ſchildere Dich \textsc{etc. etc.} In franz\oindex{Frankreich@\textbf{Frankreich}|pwv}öſiſche Kreiſe {[}ist{]} nicht hineinzukommen. Der \label{K_L02699-3v}\edtext{\textsc{\begin{otherlanguage}{french}sale Prussien\end{otherlanguage}}}{\lemma{\textnormal{\emph{sale Prussien}}}\Cendnote{\textnormal{französisch: schmutziger Preuße}}}\label{K_L02699-3h}{ }\strikeout{iſt wie} klebt Einem wie ein Peſthauch an, vor dem
               ſich alle Thüren {\pb}verſperren{\dotsfour}\pend
           \pstart
           Thu’ mir den einzigen Gefallen, laß’ Dich nicht \label{K_L02699-4v}\edtext{in \textsc{Prag\oindex{Prag@\textbf{Prag}|pw}}}{\lemma{\textnormal{\emph{in Prag}}}\Cendnote{\textnormal{Über das ganze Jahr 1892 gab es Bemühungen, \emph{Das
                     Märchen}\pwindex{Schnitzler, Arthur 15.05.1862 – 21.10.1931@\textsc{Schnitzler, Arthur} (15.05.1862 – 21.10.1931), \emph{Schriftsteller, Mediziner}!Maerchen. Schauspiel in drei Aufzuegen1893-12-01@\strich\emph{Das Märchen. Schauspiel in drei Aufzügen} {[}1893-12-01{]}|pwk} am \emph{Neuen Deutschen Theater}\orgindex{Neues Deutsches Theater@Neues Deutsches Theater|pwk} in
                     Prag\oindex{Prag@\textbf{Prag}|pwk} aufzuführen. Am 4. 1. 1892 notierte
                     Schnitzler\pwindex{Schnitzler, Arthur 15.05.1862 – 21.10.1931@\textsc{Schnitzler, Arthur} (15.05.1862 – 21.10.1931), \emph{Schriftsteller, Mediziner}|pwk} im \emph{Tagebuch}\pwindex{Schnitzler, Arthur 15.05.1862 – 21.10.1931@\textsc{Schnitzler, Arthur} (15.05.1862 – 21.10.1931), \emph{Schriftsteller, Mediziner}!Tagebuch1981 – 2000@\strich\emph{Tagebuch} {[}1981 – 2000{]}|pwk} die Zusage. Das Schauspiel\pwindex{Schnitzler, Arthur 15.05.1862 – 21.10.1931@\textsc{Schnitzler, Arthur} (15.05.1862 – 21.10.1931), \emph{Schriftsteller, Mediziner}!Maerchen. Schauspiel in drei Aufzuegen1893-12-01@\strich\emph{Das Märchen. Schauspiel in drei Aufzügen} {[}1893-12-01{]}|pwkv} sollte im Oktober des Jahres aufgeführt werden
                     (vgl. A. S.: \emph{Tagebuch}, 6. 1. 1892, 6. 8. 1892).
                  Letztendlich wurde die Aufführung jedoch untersagt (vgl. A. S.: \emph{Tagebuch}, 9. 1. 1893, 12. 1. 1893). }}}\label{K_L02699-4h} aufführen! In \textsc{Prag\oindex{Prag@\textbf{Prag}|pw}} kann man Dich erſtens nicht verſtehen und zweitens nicht ſpielen. Die Sache\pwindex{Schnitzler, Arthur 15.05.1862 – 21.10.1931@\textsc{Schnitzler, Arthur} (15.05.1862 – 21.10.1931), \emph{Schriftsteller, Mediziner}!Maerchen. Schauspiel in drei Aufzuegen1893-12-01@\strich\emph{Das Märchen. Schauspiel in drei Aufzügen} {[}1893-12-01{]}|pwv} muß Mißerfolg haben, und
               damit verdirbſt Du Dir dann Deine Berlin\oindex{Berlin@\textbf{Berlin}|pw}er
               Aufführung. Warte ruhig ab! Glaube mir, Deine Zeit \uline{muß} kommen. Aber über \textsc{Prag\oindex{Prag@\textbf{Prag}|pw}} geht man nicht zur Höhe der Künſtlerſchaft{\dotsfour}\pend
           \pstart
           Es freut mich unſäglich zu hören, daß Du an der Arbeit biſt. Schaffe, liebſter
               Freund, und werde nicht {\pb}müde! Du biſt der Einzige von
               uns, der eine Zukunft hat!\pend
           \pstart
           Und \label{K_L02699-5v}\edtext{\uline{das\pwindex{Gluemer, Marie 03.07.1867 – 16.11.1925@\textsc{Glümer, Marie} (03.07.1867 – 16.11.1925), \emph{Schauspielerin}|pwv}}}{\lemma{\textnormal{\emph{das}}}\Cendnote{\textnormal{Bezug auf die seit 1889
                  andauernde Beziehung zwischen Schnitzler\pwindex{Schnitzler, Arthur 15.05.1862 – 21.10.1931@\textsc{Schnitzler, Arthur} (15.05.1862 – 21.10.1931), \emph{Schriftsteller, Mediziner}|pwk} und
                     Marie Glümer\pwindex{Gluemer, Marie 03.07.1867 – 16.11.1925@\textsc{Glümer, Marie} (03.07.1867 – 16.11.1925), \emph{Schauspielerin}|pwk}.}}}\label{K_L02699-5h} dauert auch noch
               fort? Ich kenne mich nicht mehr aus: iſt es gut? iſt es ſchlimm? Da gibt es nur
               Eines: die Dinge zu Ende leben; und \strikeout{iſt} kommt kein
               Ende, ſo iſt es deshalb, weil es vielleicht keines gibt. Obwohl ich glaube, daß, wenn
               Du Dich einmal losriſſeſt und in die Welt hinausgingſt, die herrliche, große, Dir die
               zwei weißen Arme\pwindex{Gluemer, Marie 03.07.1867 – 16.11.1925@\textsc{Glümer, Marie} (03.07.1867 – 16.11.1925), \emph{Schauspielerin}|pwv} doch zu eng
               erſcheinen würden, die jetzt Deinen {\pb}Lebenskreis
               begrenzen. Verſuche es! Einen Monat! Komm hierher, oder irgendwohin! Sieh’ Dir die
                  Sache\pwindex{Gluemer, Marie 03.07.1867 – 16.11.1925@\textsc{Glümer, Marie} (03.07.1867 – 16.11.1925), \emph{Schauspielerin}|pwv} von außen an! Ich
               meine, Du biſt die Probe Dir ſchuldig und denen, die an Dich glauben. Geht’s
                  nicht\strikeout{,} ohne das verteufelte Glück\pwindex{Gluemer, Marie 03.07.1867 – 16.11.1925@\textsc{Glümer, Marie} (03.07.1867 – 16.11.1925), \emph{Schauspielerin}|pwv}, ſo kannſt Du ja immer noch
               heimkehren.\pend
           \pstart
           Sei innigſt umarmt! Tauſend Dank! {\\[\baselineskip]}Dein{\\[\baselineskip]}treuer{\\[\baselineskip]}\spacefill\mbox{Paul Goldmann.}\pend
           \leftskip=0em{}
         
         \endnumbering\mylabel{h}\end{ledgroupsized}  \newcommand{\dateiname}{L02699}\newcommand{\titel}{Paul Goldmann an Arthur Schnitzler, 27. 6. [1892]}\newcommand{\editorInnen}{Martin Anton Müller und Laura Untner}%% latex-leseansicht-abspann.tex
%% Abspann für die Leseansicht.
%% Der Schalter \ifkorrekturansicht ist bereits durch den Vorspann gesetzt.

%% latex-abspann.tex
%% Gemeinsamer Abspann für Korrekturansicht und Leseansicht.
%% Setzt den Schalter \ifkorrekturansicht voraus (gesetzt in den
%% einbindenden Dateien latex-korrekturansicht-abspann.tex bzw.
%% latex-leseansicht-abspann.tex).
%% ---------------------------------------------------------------

\normalsize

% Das esempio-Environment wird nur in der Leseansicht benötigt
\ifkorrekturansicht\else
\newenvironment{esempio}[3]%
{
    \vspace{1.5ex}
    \rlap{\underline{#1}}
    \par
    \setlength{\parindent}{0cm}
    \nopagebreak
    \leftskip=#2cm
    \rightskip=#3cm
}
{
    \par
}
\fi

\doendnotes{C}
\bigskip
\vfill

\clearpage

\footnotesize

\ifkorrekturansicht
  \lohead{\textsc{register}}
\fi

% theindex-Environment neu definieren ohne reledmac
\makeatletter
\renewenvironment{theindex}{%
  \ifkorrekturansicht
    \section*{\indexname}%
  \else
    \subsubsection*{Index der erwähnten Entitäten}%
  \fi
  \setlength{\parindent}{0pt}%
  \setlength{\parskip}{0pt plus 0.3pt}%
  \let\item\@idxitem
}{%
  \ifkorrekturansicht\clearpage\fi
}
\makeatother

\IfFileExists{\jobname-pw.ind}{\input{\jobname-pw.ind}}{}

% Quellenangabe nur in der Leseansicht
\ifkorrekturansicht\else
% Fallback-Definitionen, falls die .tex-Datei \titel etc. nicht gesetzt hat
\providecommand{\titel}{}
\providecommand{\editorInnen}{}
\providecommand{\dateiname}{\jobname}

\vspace{3cm}

\vfill

\footnotesize
\textsc{Quelle}: \titel. Herausgegeben von {\editorInnen}. In: \emph{Arthur Schnitzler: Briefwechsel mit Autorinnen und Autoren}.
 Digitale Edition, https://schnitzler-briefe.acdh.oeaw.ac.at/{\dateiname}.html (Stand \today)
\fi

\end{document}


      