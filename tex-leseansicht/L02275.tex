%% latex-leseansicht-vorspann.tex
%% Vorspann für die Leseansicht.
%% Lädt die gemeinsame Datei latex-vorspann.tex mit nicht gesetztem Schalter.

\newif\ifkorrekturansicht
\korrekturansichtfalse

\input{../tex-inputs/latex-vorspann}


\section[Robert Adam an Arthur Schnitzler, 12. 10. 1917]{L02275 Robert Adam an Arthur Schnitzler, 12. 10. 1917}
\nopagebreak\mylabel{L02275v}
\rehead{ }\normalsize\beginnumbering\briefempfaengerindex{Schnitzler, Arthur@\textsc{Schnitzler, Arthur}!zzzAdam, Robert@\emph{von Robert Adam}!1917-10-121@{12. 10. 1917}|(be}
\toendnotes[C]{\smallbreak\pagebreak[2]}
\correspDesc{Versand  durch Robert Adam am 12. 10. 1917 in Wien
\newline{}Erhalt  durch Arthur Schnitzler im Zeitraum [12. 10. 1917 – 16. 10. 1917?] in Wien}\toendnotes[C]{\smallbreak}
\Standort{DLA, A:Schnitzler, HS.NZ85.1.4230,21.}
\physDesc{Brief, 1 Blatt, 3 Seiten, 1822 Zeichen
\newline{}Handschrift: schwarze Tinte, deutsche Kurrent
\newline{}Schnitzler: 1) mit Bleistift beschriftet: »\textsc{Adam}«  2) mit rotem Buntstift eine Unterstreichung}\Standort{Wien, Österreichische Nationalbibliothek, Cod.ser. 52.263, 202.}
\physDesc{Brief, maschinenschriftliche Abschrift, 1 Blatt, 1 Seite, 1822 Zeichen
\newline{}Schreibmaschine}
\pstart
           \raggedleft{}{\pb}Wien\oindex{Wien@\textbf{Wien}, \emph{Verwaltungsgebiet}|pw}, am 12. Oktober 1917.\pend
           
\pstart{}Hochverehrter Herr Doktor!\pend\vspace{0.5em}
\pstart
           Ich überſende Ihnen (da ich glaube, daß Sie es mir geſtatten) meine jüngſte
               Tragikomödie, »Juda\pwindex{Adam, Robert 20.\,4.\,1877 Wien – 16.\,10.\,1961 Baden bei Wien@\textsc{Adam, Robert} (20.\,4.\,1877 Wien – 16.\,10.\,1961 Baden bei Wien), \emph{Schriftsteller, Richter}!Ende des Judas@\strich\emph{Das Ende des Judas}|pw}«, die{ }ſoeben fertiggewordene
               Arbeit des letzten Halbjahres, mit der Bitte,{ }ſie zu leſen, und mit der Bitte um Rat,
               was ich damit anfangen{ }ſoll. Ich habe das Gefühl, daß es das erſte Theaterſtück iſt,
               das ich geſchrieben habe; ob es, mit meinen anderen Arbeiten verglichen, einen
               Fortſchritt bedeutet oder aber einen Rückſchritt, das kann ich{ }ſelbſt, und gar jetzt{ }ſchon, nicht beurteilen. Bühnenwirkſam dürfte es{ }ſein, wenigſtens in{ }ſeiner zweiten
               Hälfte; aber ob nicht mein Stoff {\pb}knabenhaft-töricht
               iſt, fragen immer wieder nicht zu widerlegende Skrupel (denen allerdings eine dem
               Milieu des Stückes gemäße Gegenfrage zu antworten weiß: welcher Theaterſtoff iſt
               nicht kindiſch?) Mit einem Worte: ich{ }ſtehe meiner Arbeit nun, da{ }ſie vollendet iſt,
               mit{ }ſehr{ }ſchwankenden Gefühlen und urteilslos gegenüber.\pend
           
\pstart
           So bin ich auf den erſten Eindruck, den{ }ſie auf Sie, hochverehrter Herr Doktor,
               machen wird,{ }ſehr geſpannt und{ }ſehe Ihrem Urteil, das Sie mir ja wohl nicht weigern
               werden, mit Angſt und Beben entgegen. Iſt das Ganze als Ganzes etwas wert oder nicht?
               Daß mir gewiſſe Einzelheiten nicht mißlungen{ }ſind, glaube ich allerdings. –\pend
           
\pstart
           Und wenn das Stück etwas wert {\pb}ſein{ }ſollte:{ }ſoll ich’s
               dem Burgtheater\orgindex{Burgtheater@Burgtheater|pw} und dem Münchner Hoftheater\orgindex{Nationaltheater München@Nationaltheater München|pw} einreichen? oder{ }ſoll ich mein Heil bei
               akatholiſchen Theatern{ }ſuchen?\pend
           
\pstart
           Wenn ich wenigſtens zur »jungen Generation« gehörte! Aber ach! ich darf mich nicht
               mehr zu ihr zählen (und Gott möge mich vor{ }ſolchem bewahren!) und zur »alten
               Generation« habe ich auch nicht mehr gehört. Wo{ }ſoll ich ein Plätzlein an der Sonne{ }ſuchen? –\pend
           
\pstart
           Indem ich Sie bitte, mir die 180 Seiten lange Einſendung nicht zu verübeln,
               verbleibe ich mit den ergeben ten Grüßen Ihr{\\[\baselineskip]}\spacefill\mbox{Robert Adam}\pend
           \leftskip=0em{}\selectlanguage{ngerman}\endnumbering\briefempfaengerindex{Schnitzler, Arthur@\textsc{Schnitzler, Arthur}!zzzAdam, Robert@\emph{von Robert Adam}!1917-10-121@{12. 10. 1917}|)be}\mylabel{L02275h}  \newcommand{\dateiname}{L02275}\newcommand{\titel}{Robert Adam an Arthur Schnitzler, 12. 10. 1917}\newcommand{\editorInnen}{Martin Anton Müller und Gerd-Hermann Susen}%% latex-leseansicht-abspann.tex
%% Abspann für die Leseansicht.
%% Der Schalter \ifkorrekturansicht ist bereits durch den Vorspann gesetzt.

%% latex-abspann.tex
%% Gemeinsamer Abspann für Korrekturansicht und Leseansicht.
%% Setzt den Schalter \ifkorrekturansicht voraus (gesetzt in den
%% einbindenden Dateien latex-korrekturansicht-abspann.tex bzw.
%% latex-leseansicht-abspann.tex).
%% ---------------------------------------------------------------

\normalsize

% Das esempio-Environment wird nur in der Leseansicht benötigt
\ifkorrekturansicht\else
\newenvironment{esempio}[3]%
{
    \vspace{1.5ex}
    \rlap{\underline{#1}}
    \par
    \setlength{\parindent}{0cm}
    \nopagebreak
    \leftskip=#2cm
    \rightskip=#3cm
}
{
    \par
}
\fi

\doendnotes{C}
\bigskip
\vfill

\clearpage

\footnotesize

\ifkorrekturansicht
  \lohead{\textsc{register}}
\fi

% theindex-Environment neu definieren ohne reledmac
\makeatletter
\renewenvironment{theindex}{%
  \ifkorrekturansicht
    \section*{\indexname}%
  \else
    \subsubsection*{Index der erwähnten Entitäten}%
  \fi
  \setlength{\parindent}{0pt}%
  \setlength{\parskip}{0pt plus 0.3pt}%
  \let\item\@idxitem
}{%
  \ifkorrekturansicht\clearpage\fi
}
\makeatother

\IfFileExists{\jobname-pw.ind}{\input{\jobname-pw.ind}}{}

% Quellenangabe nur in der Leseansicht
\ifkorrekturansicht\else
% Fallback-Definitionen, falls die .tex-Datei \titel etc. nicht gesetzt hat
\providecommand{\titel}{}
\providecommand{\editorInnen}{}
\providecommand{\dateiname}{\jobname}

\vspace{3cm}

\vfill

\footnotesize
\textsc{Quelle}: \titel. Herausgegeben von {\editorInnen}. In: \emph{Arthur Schnitzler: Briefwechsel mit Autorinnen und Autoren}.
 Digitale Edition, https://schnitzler-briefe.acdh.oeaw.ac.at/{\dateiname}.html (Stand \today)
\fi

\end{document}


