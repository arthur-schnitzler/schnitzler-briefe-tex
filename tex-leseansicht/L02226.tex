%% latex-leseansicht-vorspann.tex
%% Vorspann für die Leseansicht.
%% Lädt die gemeinsame Datei latex-vorspann.tex mit nicht gesetztem Schalter.

\newif\ifkorrekturansicht
\korrekturansichtfalse

\input{../tex-inputs/latex-vorspann}


         
         \renewcommand{\erwaehntePersonen}{Personen: Robert Adam}
         \renewcommand{\erwaehnteOrte}{Orte: Meidlinger Hauptstraße, Wien}
         \renewcommand{\erwaehnteWerke}{}
               \section[Robert Adam an Arthur Schnitzler, 10. 5. 1916]{ Robert Adam an Arthur Schnitzler, 10. 5. 1916}\nopagebreak\mylabel{v}\rehead{ }\begin{ledgroupsized}[t]{13cm}\normalsize\beginnumbering\briefempfaengerindex{Schnitzler, Arthur@\textsc{Schnitzler, Arthur}!zzzAdam, Robert@\emph{von Robert Adam}!1916-05-101@{10. 5. 1916}|(be} \toendnotes[C]{\smallbreak\pagebreak[2]} \Standort{DLA, A:Schnitzler, HS.NZ85.1.4230,13.}
\physDesc{Brief, 1 Blatt, 2 Seiten, 558 Zeichen
\newline{}Handschrift: schwarze Tinte, deutsche Kurrent
\newline{}Schnitzler: 1) mit Bleistift beschriftet: »\textsc{Adam}« und: »\textsc{XII. Meidl Hpts 58}\oindex{Meidlinger Hauptstrasse@\textbf{Meidlinger Hauptstraße}|pw}«  2) mit rotem Buntstift eine Unterstreichung}\pstart
           \raggedleft{}{\pb}Wien\oindex{Meidlinger Hauptstrasse@\textbf{Meidlinger Hauptstraße}|pw}, am 10. Mai 1916\pend
           \pstart{}Hochgeehrter Herr Doktor!\pend\pstart
           Ich möchte gerne alles vermeiden, was Ihnen als Aufdringlichkeit erſcheinen könnte,
               und doch drängt es mich, bei Ihnen wieder einmal vorzuſprechen, um Ihnen mein Herz
               auszuſchütten und etwas Ermutigung zu haben. Darf ich, da ich bei meinen letzten
               Beſuchen nicht das Glück hatte, Sie anzutreffen, mir die Anfrage erlauben, ob und
               wann ich bei Ihnen vor{\pb}ſprechen könnte, ohne Sie zu
               ſtören?\pend
           \pstart
           Ich bitte Sie, hochverehrter Herr Doktor, mir dieſe Behelligung nicht übel zu
               nehmen.\pend
           \pstart
           Mit den ergebenſten Grüßen\pend
           \pstart
           Ihr{\\[\baselineskip]}\spacefill\mbox{Robert Adam}\pend
           \leftskip=0em{}
         
         \endnumbering\mylabel{h}\end{ledgroupsized}  \newcommand{\dateiname}{L02226}\newcommand{\titel}{Robert Adam an Arthur Schnitzler, 10. 5. 1916}\newcommand{\editorInnen}{Martin Anton Müller und Gerd-Hermann Susen}%% latex-leseansicht-abspann.tex
%% Abspann für die Leseansicht.
%% Der Schalter \ifkorrekturansicht ist bereits durch den Vorspann gesetzt.

%% latex-abspann.tex
%% Gemeinsamer Abspann für Korrekturansicht und Leseansicht.
%% Setzt den Schalter \ifkorrekturansicht voraus (gesetzt in den
%% einbindenden Dateien latex-korrekturansicht-abspann.tex bzw.
%% latex-leseansicht-abspann.tex).
%% ---------------------------------------------------------------

\normalsize

% Das esempio-Environment wird nur in der Leseansicht benötigt
\ifkorrekturansicht\else
\newenvironment{esempio}[3]%
{
    \vspace{1.5ex}
    \rlap{\underline{#1}}
    \par
    \setlength{\parindent}{0cm}
    \nopagebreak
    \leftskip=#2cm
    \rightskip=#3cm
}
{
    \par
}
\fi

\doendnotes{C}
\bigskip
\vfill

\clearpage

\footnotesize

\ifkorrekturansicht
  \lohead{\textsc{register}}
\fi

% theindex-Environment neu definieren ohne reledmac
\makeatletter
\renewenvironment{theindex}{%
  \ifkorrekturansicht
    \section*{\indexname}%
  \else
    \subsubsection*{Index der erwähnten Entitäten}%
  \fi
  \setlength{\parindent}{0pt}%
  \setlength{\parskip}{0pt plus 0.3pt}%
  \let\item\@idxitem
}{%
  \ifkorrekturansicht\clearpage\fi
}
\makeatother

\IfFileExists{\jobname-pw.ind}{\input{\jobname-pw.ind}}{}

% Quellenangabe nur in der Leseansicht
\ifkorrekturansicht\else
% Fallback-Definitionen, falls die .tex-Datei \titel etc. nicht gesetzt hat
\providecommand{\titel}{}
\providecommand{\editorInnen}{}
\providecommand{\dateiname}{\jobname}

\vspace{3cm}

\vfill

\footnotesize
\textsc{Quelle}: \titel. Herausgegeben von {\editorInnen}. In: \emph{Arthur Schnitzler: Briefwechsel mit Autorinnen und Autoren}.
 Digitale Edition, https://schnitzler-briefe.acdh.oeaw.ac.at/{\dateiname}.html (Stand \today)
\fi

\end{document}


      