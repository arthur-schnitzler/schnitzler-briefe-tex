%% latex-korrekturansicht-vorspann.tex
%% Vorspann für die Korrekturansicht.
%% Lädt die gemeinsame Datei latex-vorspann.tex mit gesetztem Schalter.

\newif\ifkorrekturansicht
\korrekturansichttrue

\input{../tex-inputs/latex-vorspann}


\section[Robert Adam an Arthur Schnitzler, 10. 5. 1916]{L02226 Robert Adam an Arthur Schnitzler, 10. 5. 1916}
\nopagebreak\mylabel{L02226v}
\rehead{ }\normalsize\beginnumbering\briefempfaengerindex{Schnitzler, Arthur@\textsc{Schnitzler, Arthur}!zzzAdam, Robert@\emph{von Robert Adam}!1916-05-101@{10. 5. 1916}|(be}
\toendnotes[C]{\smallbreak\pagebreak[2]}\Standort{DLA, A:Schnitzler, HS.NZ85.1.4230,13.}
\physDesc{Brief, 1 Blatt, 2 Seiten, 558 Zeichen
\newline{}Handschrift: schwarze Tinte, deutsche Kurrent
\newline{}Schnitzler: 1) mit Bleistift beschriftet: »\textsc{Adam}« und: »\textsc{XII. Meidl Hpts 58}\oindex{Meidlinger Hauptstrasse@\textbf{Meidlinger Hauptstraße}, \emph{Straße (K.STR)}|pw}«  2) mit rotem Buntstift eine Unterstreichung}
\pstart
           \raggedleft{}{\pb}Wien\oindex{Meidlinger Hauptstrasse@\textbf{Meidlinger Hauptstraße}, \emph{Straße (K.STR)}|pw}, am 10. Mai 1916\pend
           
\pstart{}Hochgeehrter Herr Doktor!\pend\vspace{0.5em}
\pstart
           Ich möchte gerne alles vermeiden, was Ihnen als Aufdringlichkeit erſcheinen könnte,
               und doch drängt es mich, bei Ihnen wieder einmal vorzuſprechen, um Ihnen mein Herz
               auszuſchütten und etwas Ermutigung zu haben. Darf ich, da ich bei meinen letzten
               Beſuchen nicht das Glück hatte, Sie anzutreffen, mir die Anfrage erlauben, ob und
               wann ich bei Ihnen vor{\pb}ſprechen könnte, ohne Sie zu
               ſtören?\pend
           
\pstart
           Ich bitte Sie, hochverehrter Herr Doktor, mir dieſe Behelligung nicht übel zu
               nehmen.\pend
           
\pstart
           Mit den ergebenſten Grüßen\pend
           
\pstart
           Ihr{\\[\baselineskip]}\spacefill\mbox{Robert Adam}\pend
           \leftskip=0em{}\selectlanguage{ngerman}\endnumbering\briefempfaengerindex{Schnitzler, Arthur@\textsc{Schnitzler, Arthur}!zzzAdam, Robert@\emph{von Robert Adam}!1916-05-101@{10. 5. 1916}|)be}\mylabel{L02226h}  \normalsize

\doendnotes{C}
\bigskip
\vfill

\clearpage

\footnotesize

\lohead{\textsc{register}}

% Definiere theindex-Environment komplett neu ohne reledmac
\makeatletter
\renewenvironment{theindex}{%
  \section*{\indexname}%
  \setlength{\parindent}{0pt}%
  \setlength{\parskip}{0pt plus 0.3pt}%
  \let\item\@idxitem
}{%
  \clearpage
}
\makeatother

\IfFileExists{\jobname-pw.ind}{\input{\jobname-pw.ind}}{}

\end{document}

      