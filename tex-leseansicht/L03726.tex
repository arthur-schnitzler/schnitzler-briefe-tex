%% latex-leseansicht-vorspann.tex
%% Vorspann für die Leseansicht.
%% Lädt die gemeinsame Datei latex-vorspann.tex mit nicht gesetztem Schalter.

\newif\ifkorrekturansicht
\korrekturansichtfalse

\input{../tex-inputs/latex-vorspann}


\section[Elsa Plessner an Arthur Schnitzler, 24. 6. 1900]{L03726 Elsa Plessner an Arthur Schnitzler, 24. 6. 1900}
\nopagebreak\mylabel{L03726v}
\rehead{ }\normalsize\beginnumbering\briefempfaengerindex{Schnitzler, Arthur@\textsc{Schnitzler, Arthur}!zzzPlessner, Elsa@\emph{von Elsa Plessner}!1900-06-241@{24. 6. 1900}|(be}
\toendnotes[C]{\smallbreak\pagebreak[2]}
\correspDesc{Versand  durch Elsa Plessner am 24. 6. 1900 in Wien
\newline{}Erhalt  durch Arthur Schnitzler im Zeitraum [24. 6. 1900 – 27. 6. 1900?] in Wien}\toendnotes[C]{\smallbreak}
\Standort{DLA, A:Schnitzler, HS.1985.1.419.}
\physDesc{Brief, 1 Blatt, 2 Seiten, 979 Zeichen (Briefpapier mit Muster aus Hufeisen und vierblättrigem Kleeblatt)
\newline{}Handschrift: schwarze Tinte, lateinische Kurrent
\newline{}Schnitzler: mit rotem Buntstift eine Unterstreichung }\toendnotes[C]{\smallbreak}
\pstart
           \raggedleft{}{\pb}Wien I. Kärnthnerstraße N\textsuperscript{o} 10\oindex{Wien@\textbf{Wien}!I., Innere Stadt@\textbf{I., Innere Stadt}!Kärntner Straße 10@\textbf{Kärntner Straße 10}, \emph{Wohngebäude}|pw}\pend
           
\pstart
           \raggedleft{}den 24. 6. 1900\pend
           
\pstart{}Verehrter Herr Doctor!\pend\vspace{0.5em}
\pstart
           Der »\label{K_L03726-1v}\edtext{neuen d. Rundschau\pwindex{Neue Deutsche Rundschau@\emph{Neue Deutsche Rundschau}|pw}« entnehme ich, dass ein neues Buch aus Ihrer Feder, »Reigen\pwindex{Schnitzler, Arthur 15. 5. 1862 Wien – 21. 10. 1931 ebd.@\textsc{Schnitzler, Arthur} (15. 5. 1862 Wien – 21. 10. 1931 ebd.), \emph{Schriftsteller, Mediziner}!Reigen. Zehn Dialoge@\strich\emph{Reigen. Zehn Dialoge}|pw}}{\lemma{\textnormal{\emph{neuen … »Reigen}}}\Cendnote{\textnormal{Alfred Kerr\pwindex{Kerr, Alfred 25.\,12.\,1867 Breslau – 12.\,10.\,1948 Hamburg@\textsc{Kerr, Alfred} (25.\,12.\,1867 Breslau – 12.\,10.\,1948 Hamburg), \emph{Schriftsteller, Kritiker}|pwk}: \emph{Aus der Wiener Mappe}\pwindex{Kerr, Alfred 25.\,12.\,1867 Breslau – 12.\,10.\,1948 Hamburg@\textsc{Kerr, Alfred} (25.\,12.\,1867 Breslau – 12.\,10.\,1948 Hamburg), \emph{Schriftsteller, Kritiker}!Aus der Wiener Mappe@\strich\emph{Aus der Wiener Mappe}|pwk}. In: \emph{Neue deutsche Rundschau}\pwindex{Neue Deutsche Rundschau@\emph{Neue Deutsche Rundschau}|pwk}, Jg. 11, Nr. 6,
                        Juni 1900, S. 660–666. Darin
                     (S. 666) lobte Kerr\pwindex{Kerr, Alfred 25.\,12.\,1867 Breslau – 12.\,10.\,1948 Hamburg@\textsc{Kerr, Alfred} (25.\,12.\,1867 Breslau – 12.\,10.\,1948 Hamburg), \emph{Schriftsteller, Kritiker}|pwk} den \emph{Reigen}\pwindex{Schnitzler, Arthur 15. 5. 1862 Wien – 21. 10. 1931 ebd.@\textsc{Schnitzler, Arthur} (15. 5. 1862 Wien – 21. 10. 1931 ebd.), \emph{Schriftsteller, Mediziner}!Reigen. Zehn Dialoge@\strich\emph{Reigen. Zehn Dialoge}|pwk},
                  erwähnte aber, dass er nicht käuflich zu erwerben sei, denn »Unsre Besten
                     haben kein Vertrauen zu dieser Gegenwart«. Schnitzler verschenkte
                  das Buch zu dieser Zeit als Privatdruck an Freunde und Kollegen, wobei er jedes Exemplar mit einer
               persönlichen Widmung versah.}}}\label{K_L03726-1}« das Licht der Welt
               erblickt hat. Gleichzeitig kommt aber die betrübsame Kunde, dass »Reigen\pwindex{Schnitzler, Arthur 15. 5. 1862 Wien – 21. 10. 1931 ebd.@\textsc{Schnitzler, Arthur} (15. 5. 1862 Wien – 21. 10. 1931 ebd.), \emph{Schriftsteller, Mediziner}!Reigen. Zehn Dialoge@\strich\emph{Reigen. Zehn Dialoge}|pw}« für profane Menschenkinder nicht zugänglich ist. – Nun
               erlaube ich mir, Sie zu fragen und um Nachricht zu bitten, wie, wann, wo und wieso ich
               doch vielleicht das Buch\pwindex{Schnitzler, Arthur 15. 5. 1862 Wien – 21. 10. 1931 ebd.@\textsc{Schnitzler, Arthur} (15. 5. 1862 Wien – 21. 10. 1931 ebd.), \emph{Schriftsteller, Mediziner}!Reigen. Zehn Dialoge@\strich\emph{Reigen. Zehn Dialoge}|pwv} in
               die Hand bekommen könnte. Sie können sich wohl vorstellen, dass mich {\pb}\substVorne{}\textsuperscript{J}\substDazwischen{}j\substHinten{}ede Ihrer Arbeiten ungemein interessiert. Nicht wahr? Ich hoffe also, dass Sie nicht böse sind, wenn ich Sie direct interpelliere und dabei
               auf meine Eigenschaft als »Literaturbeflissene« Bezug nehme\substVorne{}\textsuperscript{,}\substDazwischen{}.\substHinten{}\pend
           
\pstart
           Sollten Sie aber triftige \label{K_L03726-2v}\edtext{Gründe}{\lemma{\textnormal{\emph{Gründe}}}\Cendnote{\textnormal{Es ist kein 
                  Dankschreiben von Plessner\pwindex{Plessner, Elsa 22.\,8.\,1875 Wien – 7.\,5.\,1932 Alicante@\textsc{Plessner, Elsa} (22.\,8.\,1875 Wien – 7.\,5.\,1932 Alicante), \emph{Schriftstellerin}|pwk} überliefert und auch überhaupt ist bislang kein Exemplar des Erstdrucks von
                  \emph{Reigen}\pwindex{Schnitzler, Arthur 15. 5. 1862 Wien – 21. 10. 1931 ebd.@\textsc{Schnitzler, Arthur} (15. 5. 1862 Wien – 21. 10. 1931 ebd.), \emph{Schriftsteller, Mediziner}!Reigen. Zehn Dialoge@\strich\emph{Reigen. Zehn Dialoge}|pwk} nachgewiesen, das Schnitzler einer Frau gewidmet hätte. Folglich
                  dürfte er es ihr vermutlich nicht geschickt haben.
            }}}\label{K_L03726-2} haben, mich trotzdem unter die profanen
               Menschenkinder einzureihen, so werde ich mich Ihrer Einsicht fügen und
               selbstverständlich keinen weiteren Versuch machen, mich in den Besitz des Buches\pwindex{Schnitzler, Arthur 15. 5. 1862 Wien – 21. 10. 1931 ebd.@\textsc{Schnitzler, Arthur} (15. 5. 1862 Wien – 21. 10. 1931 ebd.), \emph{Schriftsteller, Mediziner}!Reigen. Zehn Dialoge@\strich\emph{Reigen. Zehn Dialoge}|pwv} zu setzen.\pend
           
\pstart
           Mit vorzüglicher Hochachtung grüßt in alter Verehrung{\\[\baselineskip]}\spacefill\mbox{Elsa Plessner.}\pend
           \leftskip=0em{}\selectlanguage{ngerman}\endnumbering\briefempfaengerindex{Schnitzler, Arthur@\textsc{Schnitzler, Arthur}!zzzPlessner, Elsa@\emph{von Elsa Plessner}!1900-06-241@{24. 6. 1900}|)be}\mylabel{L03726h}  \newcommand{\dateiname}{L03726}\newcommand{\titel}{Elsa Plessner an Arthur Schnitzler, 24. 6. 1900}\newcommand{\editorInnen}{Selma Jahnke und Martin Anton Müller}%% latex-leseansicht-abspann.tex
%% Abspann für die Leseansicht.
%% Der Schalter \ifkorrekturansicht ist bereits durch den Vorspann gesetzt.

%% latex-abspann.tex
%% Gemeinsamer Abspann für Korrekturansicht und Leseansicht.
%% Setzt den Schalter \ifkorrekturansicht voraus (gesetzt in den
%% einbindenden Dateien latex-korrekturansicht-abspann.tex bzw.
%% latex-leseansicht-abspann.tex).
%% ---------------------------------------------------------------

\normalsize

% Das esempio-Environment wird nur in der Leseansicht benötigt
\ifkorrekturansicht\else
\newenvironment{esempio}[3]%
{
    \vspace{1.5ex}
    \rlap{\underline{#1}}
    \par
    \setlength{\parindent}{0cm}
    \nopagebreak
    \leftskip=#2cm
    \rightskip=#3cm
}
{
    \par
}
\fi

\doendnotes{C}
\bigskip
\vfill

\clearpage

\footnotesize

\ifkorrekturansicht
  \lohead{\textsc{register}}
\fi

% theindex-Environment neu definieren ohne reledmac
\makeatletter
\renewenvironment{theindex}{%
  \ifkorrekturansicht
    \section*{\indexname}%
  \else
    \subsubsection*{Index der erwähnten Entitäten}%
  \fi
  \setlength{\parindent}{0pt}%
  \setlength{\parskip}{0pt plus 0.3pt}%
  \let\item\@idxitem
}{%
  \ifkorrekturansicht\clearpage\fi
}
\makeatother

\IfFileExists{\jobname-pw.ind}{\input{\jobname-pw.ind}}{}

% Quellenangabe nur in der Leseansicht
\ifkorrekturansicht\else
% Fallback-Definitionen, falls die .tex-Datei \titel etc. nicht gesetzt hat
\providecommand{\titel}{}
\providecommand{\editorInnen}{}
\providecommand{\dateiname}{\jobname}

\vspace{3cm}

\vfill

\footnotesize
\textsc{Quelle}: \titel. Herausgegeben von {\editorInnen}. In: \emph{Arthur Schnitzler: Briefwechsel mit Autorinnen und Autoren}.
 Digitale Edition, https://schnitzler-briefe.acdh.oeaw.ac.at/{\dateiname}.html (Stand \today)
\fi

\end{document}


