%% latex-korrekturansicht-vorspann.tex
%% Vorspann für die Korrekturansicht.
%% Lädt die gemeinsame Datei latex-vorspann.tex mit gesetztem Schalter.

\newif\ifkorrekturansicht
\korrekturansichttrue

\input{../tex-inputs/latex-vorspann}


\section[Elsa Plessner an Arthur Schnitzler, 24. 6. 1900]{L03726 Elsa Plessner an Arthur Schnitzler, 24. 6. 1900}
\nopagebreak\mylabel{L03726v}
\rehead{ }\normalsize\beginnumbering\briefempfaengerindex{Schnitzler, Arthur@\textsc{Schnitzler, Arthur}!zzzPlessner, Elsa@\emph{von Elsa Plessner}!1900-06-241@{24. 6. 1900}|(be}
\toendnotes[C]{\smallbreak\pagebreak[2]}\Standort{DLA, A:Schnitzler, HS.1985.1.419.}
\physDesc{Brief,  Blätter, 2 Seiten
\newline{}Handschrift: , lateinische Kurrent
\newline{}Schnitzler: eine Unterstreichung }\toendnotes[C]{\smallbreak}
\pstart
           {\pb}Wien I. Kärnthnerstraße N\textsuperscript{o} 10\oindex{Kaerntner Strasse 10@\textbf{Kärntner Straße 10}, \emph{Wohngebäude (K.WHS)}|pw}\textsuperscript{o} 10\pend
           
\pstart
           \raggedleft{}den 24. 6. 1900\pend
           
\pstart{}Verehrter Herr Doctor!\pend\vspace{0.5em}
\pstart
           Der »\label{K_L03726-87v}\edtext{neuen d. Rundschau\pwindex{Neue Deutsche Rundschau@\emph{Neue Deutsche Rundschau}|pw}« entnehme ich, dass ein neues Buch aus Ihrer Feder »Reigen\pwindex{Reigen. Zehn Dialoge@\emph{Reigen. Zehn Dialoge}|pw}}{\lemma{\textnormal{\emph{neuen … »Reigen}}}\Cendnote{\textnormal{Alfred Kerr\pwindex{Kerr, Alfred 25.12.1867 – 12.10.1948@\textsc{Kerr, Alfred} (25.12.1867 – 12.10.1948), \emph{Schriftsteller/Schriftstellerin, Kritiker/Kritikerin}|pwk}: \emph{Aus der Wiener Mappe}\pwindex{Aus der Wiener Mappe@\emph{Aus der Wiener Mappe}|pwk}. In: \emph{Neue deutsche Rundschau}\pwindex{Neue Deutsche Rundschau@\emph{Neue Deutsche Rundschau}|pwk}, Jg. 11, Nr. 6,
                        Juni 1900, S. 660–666. Darin
                     (S. 666) lobte Kerr\pwindex{Kerr, Alfred 25.12.1867 – 12.10.1948@\textsc{Kerr, Alfred} (25.12.1867 – 12.10.1948), \emph{Schriftsteller/Schriftstellerin, Kritiker/Kritikerin}|pwk} den \emph{Reigen}\pwindex{Reigen. Zehn Dialoge@\emph{Reigen. Zehn Dialoge}|pwk},
                  erwähnte aber, dass er nicht käuflich zu erwerben sei, denn »Unsre Besten
                     haben kein Vertrauen zu dieser Gegenwart«. Schnitzler verschenkte
                  das Buch zu dieser Zeit als Privatdruck an Freunde.}}}\label{K_L03726-87}« das Licht der Welt
               erblickt hat. Gleichzeitig kommt aber die betrübsame Kunde, dass »Reigen\pwindex{Reigen. Zehn Dialoge@\emph{Reigen. Zehn Dialoge}|pw}« für profane Menschenkinder nicht zugänglich ist. – Nun
               erlaube ich mir, Sie zu fragen und um Nachricht zu bitten wie, wann, wo und wieso ich
               doch vielleicht das Buch\pwindex{Reigen. Zehn Dialoge@\emph{Reigen. Zehn Dialoge}|pwv} in
               die Hand bekommen könnte. Sie können sich wohl vorstellen, dass mich {\pb}\substVorne{}\textsuperscript{J}\substDazwischen{}j\substHinten{}ede Ihrer Arbeiten ungemein interessiert. Nicht wahr?\pend
           
\pstart
           Ich hoffe also, dass Sie nicht böse sind, wenn ich Sie direct interpelliere und dabei
               auf meine Eigenschaft als »Literaturbeflissene« Bezug nehme.\strikeout{,}\pend
           
\pstart
           Sollten Sie aber triftige Gründe haben, mich trotzdem unter die profanen
               Menschenkinder einzureihen, so werde ich mich Ihrer Einsicht fügen und
               selbstverständlich keinen weiteren Versuch machen, mich in den Besitz des Buches\pwindex{Reigen. Zehn Dialoge@\emph{Reigen. Zehn Dialoge}|pwv} zu setzen.\pend
           
\pstart
           Mit vorzüglicher Hochachtung grüßt in alter Verehrung{\\[\baselineskip]}\spacefill\mbox{Elsa Plessner.}\pend
           \leftskip=0em{}\selectlanguage{ngerman}\endnumbering\briefempfaengerindex{Schnitzler, Arthur@\textsc{Schnitzler, Arthur}!zzzPlessner, Elsa@\emph{von Elsa Plessner}!1900-06-241@{24. 6. 1900}|)be}\mylabel{L03726h}
\begin{anhang}
\end{anhang}\normalsize

\doendnotes{C}
\bigskip
\vfill

\clearpage

\footnotesize

\lohead{\textsc{register}}

% Definiere theindex-Environment komplett neu ohne reledmac
\makeatletter
\renewenvironment{theindex}{%
  \section*{\indexname}%
  \setlength{\parindent}{0pt}%
  \setlength{\parskip}{0pt plus 0.3pt}%
  \let\item\@idxitem
}{%
  \clearpage
}
\makeatother

\IfFileExists{\jobname-pw.ind}{\input{\jobname-pw.ind}}{}

\end{document}

      