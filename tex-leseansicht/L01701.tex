%% latex-leseansicht-vorspann.tex
%% Vorspann für die Leseansicht.
%% Lädt die gemeinsame Datei latex-vorspann.tex mit nicht gesetztem Schalter.

\newif\ifkorrekturansicht
\korrekturansichtfalse

\input{../tex-inputs/latex-vorspann}


\section[Richard Beer-Hofmann an Arthur Schnitzler, 23. 8. 1907]{L01701 Richard Beer-Hofmann an Arthur Schnitzler, 23. 8. 1907}
\nopagebreak\mylabel{L01701v}
\rehead{ }\normalsize\beginnumbering\briefempfaengerindex{Schnitzler, Arthur@\textsc{Schnitzler, Arthur}!zzzBeer-Hofmann, Richard@\emph{von Richard Beer-Hofmann}!1907-08-231@{23. 8. 1907}|(be}
\toendnotes[C]{\smallbreak\pagebreak[2]}
\correspDesc{Versand  durch Richard Beer-Hofmann am 23. 8. 1907 in Velden am Wörthersee
\newline{}Erhalt  durch Arthur Schnitzler im Zeitraum [24. 8. 1907
                  – 28. 8. 1907?] in Welsberg-Taisten}\toendnotes[C]{\smallbreak}
\Standort{CUL, Schnitzler, B 8.}
\physDesc{Brief, 2 Blätter, 4 Seiten, 1581 Zeichen (Briefpapier mit Trauerrand)
\newline{}Handschrift: Bleistift, lateinische Kurrent
\newline{}Ordnung: mit Bleistift von unbekannter Hand nummeriert:
                                    »211« }
\buchAbdrucke{\weitereDrucke{Arthur Schnitzler, Richard Beer-Hofmann: \emph{Briefwechsel 1891–1931}. Herausgegeben von Konstanze Fliedl. Wien, Zürich: \emph{Europaverlag} 1992, S. 183.} }\toendnotes[C]{\smallbreak}
\pstart
           \raggedleft{}{\pb}Velden\oindex{Velden am Wörthersee@\textbf{Velden am Wörthersee}|pw}{ }23/VIII 07\pend
           \vspace{0.5em}
\pstart
           Lieber Arthur! Ihre Karte vom 19. erhalte ich heute
               nachgeschickt – nach Villach\oindex{Villach@\textbf{Villach}, \emph{Verwaltungsgebiet}|pw}, wo ich bis heute
               Früh war. Wir sind am 20. von Wien\oindex{Wien@\textbf{Wien}, \emph{Verwaltungsgebiet}|pw}
               weg, haben Veld\uline{es}\oindex{Bled@\textbf{Bled}|pw} angesehen, dann doch aber hier – bei Wahliss\oindex{Etablissement Ernst Wahliss@\textbf{Etablissement Ernst Wahliss}, \emph{Hotel}|pw} – Wohnung geno{\geminationm}en. Wollen hier acht
               Tage ungefähr bleiben, wenn uns kühles Wetter nicht vorher südlich treibt. Dann – bei
               schönem Wetter über ein Stück der Dolomitenstrasse\oindex{Große Dolomitenstraße@\textbf{Große Dolomitenstraße}, \emph{Straße}|pw} nach {\pb}Bozen\oindex{Bozen@\textbf{Bozen}, \emph{Hauptstadt}|pw} – schliesslich Lido\oindex{Lido@\textbf{Lido}|pw}, bei kühlem Wetter direkt an den Lido\oindex{Lido@\textbf{Lido}|pw}! I{\geminationm}erhin ist ziemliche
               Wahrscheinlichkeit vorhanden dass wir zwischen 2 – und
                  5 September in Bozen\oindex{Bozen@\textbf{Bozen}, \emph{Hauptstadt}|pw} oder Bozens\oindex{Bozen@\textbf{Bozen}, \emph{Hauptstadt}|pw} Nähe sind.\pend
           
\pstart
           {\pb}Wenn Sie getreulich Ihren
               Aufenthalt mir melden, eventuell auch mir sagen wohin ich restante Briefe oder
                  Telegra{\geminationm}e richten soll\strikeout{en}, können wir uns vielleicht doch treffen – was sehr schön wäre.\pend
           
\pstart
           {\pb}Wenn Sie Goldmann\pwindex{Goldmann, Paul 31.\,1.\,1865 Breslau – 25.\,9.\,1935 Wien@\textsc{Goldmann, Paul} (31.\,1.\,1865 Breslau – 25.\,9.\,1935 Wien), \emph{Schriftsteller, Journalist}|pw} sehen, sagen Sie ihm, bitte, dass ich ihm sehr für
               seine lieben Zeilen danke, dass ich ihm \label{K_L01701-1v}\edtext{als Mamroth\pwindex{Mamroth, Fedor 21.\,2.\,1851 Breslau – 25.\,6.\,1907 Frankfurt am Main@\textsc{Mamroth, Fedor} (21.\,2.\,1851 Breslau – 25.\,6.\,1907 Frankfurt am Main), \emph{Journalist, Kritiker}|pw} starb}{\lemma{\textnormal{\emph{als Mamroth starb}}}\Cendnote{\textnormal{am 25. 6. 1907}}}\label{K_L01701-1}, nicht schrieb, weil ich um diese Zeit den Kopf mit der beabsichtigten
               Operation an Papa\pwindex{Beer, Hermann 10.\,8.\,1835 Radiměř – 3.\,10.\,1902 Wien@\textsc{Beer, Hermann} (10.\,8.\,1835 Radiměř – 3.\,10.\,1902 Wien), \emph{Rechtsanwalt}|pwv} – die dann
               unterblieb – voll hatte, und ruhigere Tage abwarten wollte um ihm zu schreiben. Ich
               will mich aber nie mehr selbst auf »ruhige« oder »ruhigere« Tage vertrösten, ich {\pb}entdecke – vielleicht ein bischen
               zu spät – daß es keine giebt, – nie giebt, für Leute wie ich bin, zumindest
               nicht.\pend
           
\pstart
           Geht Goldmann\pwindex{Goldmann, Paul 31.\,1.\,1865 Breslau – 25.\,9.\,1935 Wien@\textsc{Goldmann, Paul} (31.\,1.\,1865 Breslau – 25.\,9.\,1935 Wien), \emph{Schriftsteller, Journalist}|pw} mit Ihnen dann sagen Sie mir wie
               lange er \substVorne{}\textsuperscript{bei}\substDazwischen{}mit\substHinten{} Ihnen bleibt, vielleicht kann ich es (wenn {\pb}es sich nur um 1–2 Tage handelt) so
               einrichten, daß ich ihn noch treffe. Geht er aber Wien\oindex{Wien@\textbf{Wien}, \emph{Verwaltungsgebiet}|pw}wärts, so liegen wir an seiner Route und erwarten ein Telegra{\geminationm} »\uline{Wahliss\oindex{Etablissement Ernst Wahliss@\textbf{Etablissement Ernst Wahliss}, \emph{Hotel}|pw} – Velden\oindex{Velden am Wörthersee@\textbf{Velden am Wörthersee}|pw}}« wann er hieher ko{\geminationm}mt. Alles Herzliche Ihnen und
               Frau Olga\pwindex{Schnitzler, Olga 17.\,1.\,1882 Wien – 13.\,1.\,1970 Lugano@\textsc{Schnitzler, Olga} (17.\,1.\,1882 Wien – 13.\,1.\,1970 Lugano), \emph{Schauspielerin, Sängerin}|pw}.\pend
           \pstart Ihr \spacefill\mbox{Richard}\pend{}\selectlanguage{ngerman}\endnumbering\briefempfaengerindex{Schnitzler, Arthur@\textsc{Schnitzler, Arthur}!zzzBeer-Hofmann, Richard@\emph{von Richard Beer-Hofmann}!1907-08-231@{23. 8. 1907}|)be}\mylabel{L01701h}  \newcommand{\dateiname}{L01701}\newcommand{\titel}{Richard Beer-Hofmann an Arthur Schnitzler, 23. 8. 1907}\newcommand{\editorInnen}{Martin Anton Müller und Gerd-Hermann Susen}%% latex-leseansicht-abspann.tex
%% Abspann für die Leseansicht.
%% Der Schalter \ifkorrekturansicht ist bereits durch den Vorspann gesetzt.

%% latex-abspann.tex
%% Gemeinsamer Abspann für Korrekturansicht und Leseansicht.
%% Setzt den Schalter \ifkorrekturansicht voraus (gesetzt in den
%% einbindenden Dateien latex-korrekturansicht-abspann.tex bzw.
%% latex-leseansicht-abspann.tex).
%% ---------------------------------------------------------------

\normalsize

% Das esempio-Environment wird nur in der Leseansicht benötigt
\ifkorrekturansicht\else
\newenvironment{esempio}[3]%
{
    \vspace{1.5ex}
    \rlap{\underline{#1}}
    \par
    \setlength{\parindent}{0cm}
    \nopagebreak
    \leftskip=#2cm
    \rightskip=#3cm
}
{
    \par
}
\fi

\doendnotes{C}
\bigskip
\vfill

\clearpage

\footnotesize

\ifkorrekturansicht
  \lohead{\textsc{register}}
\fi

% theindex-Environment neu definieren ohne reledmac
\makeatletter
\renewenvironment{theindex}{%
  \ifkorrekturansicht
    \section*{\indexname}%
  \else
    \subsubsection*{Index der erwähnten Entitäten}%
  \fi
  \setlength{\parindent}{0pt}%
  \setlength{\parskip}{0pt plus 0.3pt}%
  \let\item\@idxitem
}{%
  \ifkorrekturansicht\clearpage\fi
}
\makeatother

\IfFileExists{\jobname-pw.ind}{\input{\jobname-pw.ind}}{}

% Quellenangabe nur in der Leseansicht
\ifkorrekturansicht\else
% Fallback-Definitionen, falls die .tex-Datei \titel etc. nicht gesetzt hat
\providecommand{\titel}{}
\providecommand{\editorInnen}{}
\providecommand{\dateiname}{\jobname}

\vspace{3cm}

\vfill

\footnotesize
\textsc{Quelle}: \titel. Herausgegeben von {\editorInnen}. In: \emph{Arthur Schnitzler: Briefwechsel mit Autorinnen und Autoren}.
 Digitale Edition, https://schnitzler-briefe.acdh.oeaw.ac.at/{\dateiname}.html (Stand \today)
\fi

\end{document}


