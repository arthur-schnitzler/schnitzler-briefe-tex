%% latex-leseansicht-vorspann.tex
%% Vorspann für die Leseansicht.
%% Lädt die gemeinsame Datei latex-vorspann.tex mit nicht gesetztem Schalter.

\newif\ifkorrekturansicht
\korrekturansichtfalse

\input{../tex-inputs/latex-vorspann}


\section[ Paul Goldmann an Arthur Schnitzler, 22. 3. {[}1897{]}]{L02806 Paul Goldmann an Arthur Schnitzler,  22. 3. [1897]}
\nopagebreak\mylabel{L02806v}
\rehead{ }\normalsize\beginnumbering\briefempfaengerindex{Schnitzler, Arthur@\textsc{Schnitzler, Arthur}!zzzGoldmann, Paul@\emph{von Paul Goldmann}!1897-03-223@{22. 3. [1897]}|(be}
\toendnotes[C]{\smallbreak\pagebreak[2]}
\correspDesc{Versand  durch Paul Goldmann am 22. 3. [1897] in Paris
\newline{}Erhalt  durch Arthur Schnitzler im Zeitraum [23. 3. 1897
                  – 27. 3. 1897?] in Wien}\toendnotes[C]{\smallbreak}
\Standort{DLA, A:Schnitzler, HS.NZ85.1.3167.}
\physDesc{Brief, 2 Blätter, 7 Seiten, 3084 Zeichen
\newline{}Handschrift: blaue Tinte, deutsche Kurrent
\newline{}Schnitzler: mit Bleistift das Jahr »97« vermerkt }\toendnotes[C]{\smallbreak}
\pstart
           {\pb}\textcolor{gray}{\textbf{\textbf{Frankfurter Zeitung\orgindex{Frankfurter Zeitung@Frankfurter Zeitung|pw}}}}\pend
           
\pstart
           \textcolor{gray}{\textbf{(\begin{otherlanguage}{french}Gazette de Francfort\end{otherlanguage}\orgindex{Frankfurter Zeitung@Frankfurter Zeitung|pw}).}}\pend
           
\pstart
           \textcolor{gray}{\textbf{\textbf{\begin{otherlanguage}{french}Fondateur M.\end{otherlanguage}{ }L. Sonnemann\pwindex{Sonnemann, Leopold 29.\,10.\,1831 Höchberg – 30.\,10.\,1909 Frankfurt am Main@\textsc{Sonnemann, Leopold} (29.\,10.\,1831 Höchberg – 30.\,10.\,1909 Frankfurt am Main), \emph{Journalist, Herausgeber}|pw}.}}}\pend
           
\pstart
           \begin{otherlanguage}{french}\textcolor{gray}{\textbf{Journal politique, financier,}}\end{otherlanguage}\pend
           
\pstart
           \begin{otherlanguage}{french}\textcolor{gray}{\textbf{commercial et littéraire.}}\end{otherlanguage}\pend
           
\pstart
           \begin{otherlanguage}{french}\textcolor{gray}{\textbf{\textbf{Paraissant trois fois par jour.}}}\end{otherlanguage}\hfill \textsc{Paris\oindex{Paris@\textbf{Paris}, \emph{Hauptstadt}|pw}}, 22. März.\pend
           
\pstart
           \begin{otherlanguage}{french}\textcolor{gray}{\textbf{\textbf{Bureau à Paris\oindex{Paris@\textbf{Paris}, \emph{Hauptstadt}|pw}}}}\end{otherlanguage}\pend
           
\pstart
           \begin{otherlanguage}{french}\textcolor{gray}{\textbf{\textbf{24. Rue Feydeau\oindex{rue Feydeau@\textbf{rue Feydeau}, \emph{Straße}|pw}.}}}\end{otherlanguage}\pend
           
\pstart\center{}Mein lieber Freund,\pend\vspace{0.5em}
\pstart
           Haſt Du{ }ſchon \textsc{Nansens\pwindex{Nansen, Peter 20.\,1.\,1861 Kopenhagen – 31.\,7.\,1918 Mariager@\textsc{Nansen, Peter} (20.\,1.\,1861 Kopenhagen – 31.\,7.\,1918 Mariager), \emph{Schriftsteller, Journalist, Verleger}|pw}}{ }\strikeout{\textsc{A}}{ }Artikel\pwindex{Nansen, Peter 20.\,1.\,1861 Kopenhagen – 31.\,7.\,1918 Mariager@\textsc{Nansen, Peter} (20.\,1.\,1861 Kopenhagen – 31.\,7.\,1918 Mariager), \emph{Schriftsteller, Journalist, Verleger}!Arthur Schnitzler. »Elskovsleg«s Forfatter@\strich\emph{Arthur Schnitzler. »Elskovsleg«s Forfatter}|pwv} Dir \label{K_L02806-1v}\edtext{überſetzen laſſen}{\lemma{\textnormal{\emph{übersetzen lassen}}}\Cendnote{\textnormal{Das hatte Schnitzler
                  jedenfalls vor, vgl. 
                     \emph{Peter Nansen – Arthur Schnitzler. Ein Briefwechsel zweier
                        Geistesverwandter}. Herausgegeben, kommentiert und mit einem Nachwort
                     versehen von Karin Bang. Roskilde: \emph{Zentrum für
                        österreichisch-nordische Kulturstudien}{ }2003, S. 7 (Småskrifter fra CØNK / Kleine Schriften
                     von ZÖNK 9).
               }}}\label{K_L02806-1}? Er iſt ungemein lieb und herzlich geſchrieben und{ }ſehr ehrenvoll für uns
               Alle, insbeſondere natürlich für Dich.\pend
           
\pstart
           Je näher die Zeit heranrückt, wo ich Dich hier wiederſehen werde, mit umſo größerer
               Freude denke ich daran. Hab’ nur keine Furcht, daß ich mich werde von Arbeit
               Deinetwegen abhalten laſſen. Die Arbeit läßt mich hier einfach nicht los, wenn{ }ſie
               einmal da iſt. Ich denke, wir werden namentlich am \strikeout{Tage}{ }Abend beiſammen{ }ſein können, und oft auch am Tage. {\pb}\strikeout{Das} Die Hotel-Zimmer werde ich miethen,{ }ſobald Du mir
               Deine Ankunft anzeigſt. Nur möchte ich auch eine kleine Idee von dem Preiſe haben,
               den Du zu zahlen gedenkſt. \strikeout{\textcolor{gray}{Nenne}} Nenne mir ein \textsc{Maximum}: etwa 8 bis 10 \textsc{Francs}{ }\textsc{pro} Tag und \textsc{pro} Zimmer,
                  alſo 16 bis 20 \textsc{Francs}{ }\textsc{pro} Tag? Ich hoffe, ich bekomme es billiger, aber ich will
               doch wiſſen, wie weit ich im Nothfall gehen darf?\pend
           
\pstart
           Welche Unannehmlichkeiten es im Gefolge haben{ }ſollte, wenn Ihr unter Eurem wahren
               Namen Euch im \textsc{Hotel} einſchreibt, iſt mir dunkel. Ich kenne
               nur Fälle, wo es für Leute {\pb}\strikeout{U\textcolor{gray}{nn}} Unannehmlichkeiten im Gefolge gehabt hat, weil{ }ſie unter{ }ſalſchen Namen
               abgeſtiegen{ }ſind. Die Polizei hat auch in \textsc{Paris\oindex{Paris@\textbf{Paris}, \emph{Hauptstadt}|pw}} nichts dagegen, daß ein Menſch{ }ſeinen wahren Namen führt.\pend
           
\pstart
           Auch bei der Idee, mir \textsc{Virginia}-Cigarren zuzuſenden,
               erkenne ich Dich wieder. Vielleicht gar in einem \label{K_L02806-2v}\edtext{recommandirten Briefe}{\lemma{\textnormal{\emph{recommandirten Briefe}}}\Cendnote{\textnormal{Siehe XXXX Auszeichnungsfehler: Dokument L02760 nicht gefunden.
               }}}\label{K_L02806-2}? Wiſſe denn, oh Freund, daß in Frankreich\oindex{Frankreich@\textbf{Frankreich}|pw}
               das Tabaks-Monopol beſteht. Jede Einfuhr \strikeout{fremd}
               ausländiſcher Cigarren iſt verboten. Privatleute müſſen, um Cigarren-Sendungen \strikeout{empf} aus dem {\pb}Auslande
               empfangen zu dürfen, eine beſondere Import-Erlaubniß vom Finanz-Miniſterium\orgindex{Französisches Finanzministerium@Französisches Finanzministerium|pw} haben. Du kannſt \textsc{Virginia}-Cigarren nur{ }ſo nach Frankreich\oindex{Frankreich@\textbf{Frankreich}|pw}
               bringen, daß Du{ }ſie{ }ſelbſt mit Dir nimmſt. An der Grenze{ }ſagſt Du dann, daß Du Dich
               zwei Monate in Frankreich\oindex{Frankreich@\textbf{Frankreich}|pw} aufhalten willſt und
               für dieſe Zeit Dich mit Cigarren verſehen willſt. Dieſe Cigarren verzollſt Du dann
               (was eine Unſumme \strikeout{Gel} Geldes koſten wird). Oder aber,
               wenn Du Courage haſt, (die haſt Du aber wahrſcheinlich nicht), {\pb}ſo{ }ſagſt Du gar nichts und verſuchſt die Cigarren
               einfach durchzuſchmuggeln.\pend
           
\pstart
           Dein \textsc{Bicycle}{ }ſollſt Du gewiß mitnehmen. Die Umgebung von
                  \textsc{Paris\oindex{Paris@\textbf{Paris}, \emph{Hauptstadt}|pw}} iſt eigens für \textsc{Bicycle}-Touren geſchaffen. Du wirſt
               hier zahlloſe und herrliche Ausflüge mit Deiner Maſchine machen können{\dotsfive}\pend
           
\pstart
           Traurig iſt es, daß Du Dir Dein junges und{ }ſchönes Leben \strikeout{mit} durch ein Bischen \label{K_L02806-3v}\edtext{Ohrenklingen}{\lemma{\textnormal{\emph{Ohrenklingen}}}\Cendnote{\textnormal{Das Ohrenklingen aufgrund ders Otosklerose war gerade wieder akut, vgl. A. S.: \emph{Tagebuch}, 12. 3. 1897.
               }}}\label{K_L02806-3} verbittern läßt. Für mich iſt das gerade ein Beweis Deiner {\pb}\strikeout{G\textcolor{gray}{e}ſ\textcolor{gray}{×}} Geſundheit. Denn wenn Du irgend ein ernſtes Leiden hätteſt,{ }ſo könnteſt Du
               nicht auf das Ohrenklingen achten. So concentrirt{ }ſich darauf all’ \strikeout{d\textcolor{gray}{e}} Deine hypochondriſche Grübelei, die{ }ſonſt, Gott{ }ſei gelobt, kein \textsc{Sujet} in Deinem Organismus findet. Laß’ es doch klingen, zum
               Teufel, und denke nicht daran! Wenn Du nicht Medicin{ }ſtudirt hätteſt, würdeſt Du gar
               nicht darauf achten!\pend
           
\pstart
           \strikeout{N\textcolor{gray}{un}} Nun erfahre ich wohl bald den genauen Tag Deiner Ankunft. {\pb}Das wird{ }ſchön werden! \strikeout{\textcolor{gray}{×}\-\textcolor{gray}{×}\-\textcolor{gray}{×}}\pend
           
\pstart
           Traurig iſt nur, daß ich zu Oſtern auf 10 bis 14 Tage
               nach Frankfurt\oindex{Frankfurt am Main@\textbf{Frankfurt am Main}, \emph{Hauptstadt}|pw} muß. Nach \textsc{Nizza\oindex{Nizza@\textbf{Nizza}, \emph{Hauptstadt}|pw}} gehe ich nicht mehr.\pend
           
\pstart
           Wie hat »Liebelei\pwindex{Schnitzler, Arthur 15.\,5.\,1862 Wien – 21.\,10.\,1931 ebd.@\textsc{Schnitzler, Arthur} (15.\,5.\,1862 Wien – 21.\,10.\,1931 ebd.), \emph{Schriftsteller, Mediziner}!Liebelei. Schauspiel in drei Akten@\strich\emph{Liebelei. Schauspiel in drei Akten}|pw}« eigentlich in \textsc{Kopenhagen\oindex{Kopenhagen@\textbf{Kopenhagen}, \emph{Hauptstadt}|pw}}{ }\label{K_L02806-4v}\edtext{gefallen}{\lemma{\textnormal{\emph{gefallen}}}\Cendnote{\textnormal{\emph{Liebelei}\pwindex{Schnitzler, Arthur 15.\,5.\,1862 Wien – 21.\,10.\,1931 ebd.@\textsc{Schnitzler, Arthur} (15.\,5.\,1862 Wien – 21.\,10.\,1931 ebd.), \emph{Schriftsteller, Mediziner}!Liebelei. Schauspiel in drei Akten@\strich\emph{Liebelei. Schauspiel in drei Akten}|pwk} wurde als \emph{Elskovsleg. Skuespil i 3 akter}\pwindex{Schnitzler, Arthur 15.\,5.\,1862 Wien – 21.\,10.\,1931 ebd.@\textsc{Schnitzler, Arthur} (15.\,5.\,1862 Wien – 21.\,10.\,1931 ebd.), \emph{Schriftsteller, Mediziner}!Elskovsleg. Skuespil i 3 akter@\strich\emph{Elskovsleg. Skuespil i 3 akter}|pwk} am 9. 3. 1897
                  am \emph{Folketeatret}\orgindex{Folketeatret@Folketeatret|pwk} uraufgeführt. Obgleich das
                  Stück von der Presse gelobt wurde, war es laut Nansen\pwindex{Nansen, Peter 20.\,1.\,1861 Kopenhagen – 31.\,7.\,1918 Mariager@\textsc{Nansen, Peter} (20.\,1.\,1861 Kopenhagen – 31.\,7.\,1918 Mariager), \emph{Schriftsteller, Journalist, Verleger}|pwk} aufgrund der schauspielerischen Leistungen kein wirklicher Erfolg.
                     Vgl. \emph{Peter Nansen – Arthur Schnitzler. Ein Briefwechsel
                        zweier Geistesverwandter}. Herausgegeben, kommentiert und mit einem
                     Nachwort versehen von Karin Bang. Roskilde:
                        \emph{Zentrum für österreichisch-nordische Kulturstudien}{ }2003, S. 8–9. (Småskrifter fra CØNK / Kleine Schriften
                     von ZÖNK 9)}}}\label{K_L02806-4}?\pend
           
\pstart
           Sei von Herzen gegrüßt und{ }ſchreibe bald\textcolor{gray}{!}\pend
           
\pstart
           Dein treuer {\\[\baselineskip]}\spacefill\mbox{Paul Goldm}\pend
           \leftskip=0em{}\selectlanguage{ngerman}\endnumbering\briefempfaengerindex{Schnitzler, Arthur@\textsc{Schnitzler, Arthur}!zzzGoldmann, Paul@\emph{von Paul Goldmann}!1897-03-223@{22. 3. [1897]}|)be}\mylabel{L02806h}  \newcommand{\dateiname}{L02806}\newcommand{\titel}{Paul Goldmann an Arthur Schnitzler, 22. 3. [1897]}\newcommand{\editorInnen}{Martin Anton Müller und Laura Untner}%% latex-leseansicht-abspann.tex
%% Abspann für die Leseansicht.
%% Der Schalter \ifkorrekturansicht ist bereits durch den Vorspann gesetzt.

%% latex-abspann.tex
%% Gemeinsamer Abspann für Korrekturansicht und Leseansicht.
%% Setzt den Schalter \ifkorrekturansicht voraus (gesetzt in den
%% einbindenden Dateien latex-korrekturansicht-abspann.tex bzw.
%% latex-leseansicht-abspann.tex).
%% ---------------------------------------------------------------

\normalsize

% Das esempio-Environment wird nur in der Leseansicht benötigt
\ifkorrekturansicht\else
\newenvironment{esempio}[3]%
{
    \vspace{1.5ex}
    \rlap{\underline{#1}}
    \par
    \setlength{\parindent}{0cm}
    \nopagebreak
    \leftskip=#2cm
    \rightskip=#3cm
}
{
    \par
}
\fi

\doendnotes{C}
\bigskip
\vfill

\clearpage

\footnotesize

\ifkorrekturansicht
  \lohead{\textsc{register}}
\fi

% theindex-Environment neu definieren ohne reledmac
\makeatletter
\renewenvironment{theindex}{%
  \ifkorrekturansicht
    \section*{\indexname}%
  \else
    \subsubsection*{Index der erwähnten Entitäten}%
  \fi
  \setlength{\parindent}{0pt}%
  \setlength{\parskip}{0pt plus 0.3pt}%
  \let\item\@idxitem
}{%
  \ifkorrekturansicht\clearpage\fi
}
\makeatother

\IfFileExists{\jobname-pw.ind}{\input{\jobname-pw.ind}}{}

% Quellenangabe nur in der Leseansicht
\ifkorrekturansicht\else
% Fallback-Definitionen, falls die .tex-Datei \titel etc. nicht gesetzt hat
\providecommand{\titel}{}
\providecommand{\editorInnen}{}
\providecommand{\dateiname}{\jobname}

\vspace{3cm}

\vfill

\footnotesize
\textsc{Quelle}: \titel. Herausgegeben von {\editorInnen}. In: \emph{Arthur Schnitzler: Briefwechsel mit Autorinnen und Autoren}.
 Digitale Edition, https://schnitzler-briefe.acdh.oeaw.ac.at/{\dateiname}.html (Stand \today)
\fi

\end{document}


