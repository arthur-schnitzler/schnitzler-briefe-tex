%% latex-korrekturansicht-vorspann.tex
%% Vorspann für die Korrekturansicht.
%% Lädt die gemeinsame Datei latex-vorspann.tex mit gesetztem Schalter.

\newif\ifkorrekturansicht
\korrekturansichttrue

\input{../tex-inputs/latex-vorspann}


\section[ Paul Goldmann an Arthur Schnitzler, 22. 3. {[}1897{]}]{L02806 Paul Goldmann an Arthur Schnitzler, 22. 3. {[}1897{]}}
\nopagebreak\mylabel{L02806v}
\rehead{ }\normalsize\beginnumbering\briefempfaengerindex{Schnitzler, Arthur@\textsc{Schnitzler, Arthur}!zzzGoldmann, Paul@\emph{von Paul Goldmann}!1897-03-223@{22. 3. {[}1897{]}}|(be}
\toendnotes[C]{\smallbreak\pagebreak[2]}\Standort{DLA, A:Schnitzler, HS.NZ85.1.3167.}
\physDesc{Brief, 2 Blätter, 7 Seiten, 3084 Zeichen
\newline{}Handschrift: blaue Tinte, deutsche Kurrent
\newline{}Schnitzler: mit Bleistift das Jahr »97« vermerkt }\toendnotes[C]{\smallbreak}
\pstart
           {\pb}\textcolor{gray}{\textbf{\textbf{Frankfurter Zeitung\orgindex{Frankfurter Zeitung@Frankfurter Zeitung|pw}}}}\pend
           
\pstart
           \textcolor{gray}{\textbf{(\begin{otherlanguage}{french}Gazette de Francfort\end{otherlanguage}\orgindex{Frankfurter Zeitung@Frankfurter Zeitung|pw}).}}\pend
           
\pstart
           \textcolor{gray}{\textbf{\textbf{\begin{otherlanguage}{french}Fondateur M.\end{otherlanguage}{ }L. Sonnemann\pwindex{Sonnemann, Leopold 1831-10-29 – 1909-10-30@\textsc{Sonnemann, Leopold} (1831-10-29 – 1909-10-30), \emph{Journalist/Journalistin, Herausgeber/Herausgeberin}|pw}.}}}\pend
           
\pstart
           \begin{otherlanguage}{french}\textcolor{gray}{\textbf{Journal politique, financier,}}\end{otherlanguage}\pend
           
\pstart
           \begin{otherlanguage}{french}\textcolor{gray}{\textbf{commercial et littéraire.}}\end{otherlanguage}\pend
           
\pstart
           \begin{otherlanguage}{french}\textcolor{gray}{\textbf{\textbf{Paraissant trois fois par jour.}}}\end{otherlanguage}\hfill \textsc{Paris\oindex{Paris@\textbf{Paris}, \emph{P.PPLC}|pw}}, 22. März.\pend
           
\pstart
           \begin{otherlanguage}{french}\textcolor{gray}{\textbf{\textbf{Bureau à Paris\oindex{Paris@\textbf{Paris}, \emph{P.PPLC}|pw}}}}\end{otherlanguage}\pend
           
\pstart
           \begin{otherlanguage}{french}\textcolor{gray}{\textbf{\textbf{24. Rue Feydeau\oindex{rue Feydeau@\textbf{rue Feydeau}, \emph{Straße (K.STR)}|pw}.}}}\end{otherlanguage}\pend
           
\pstart\center{}Mein lieber Freund,\pend\vspace{0.5em}
\pstart
           Haſt Du ſchon \textsc{Nansens\pwindex{Nansen, Peter 20.01.1861 – 31.07.1918@\textsc{Nansen, Peter} (20.01.1861 – 31.07.1918), \emph{Schriftsteller/Schriftstellerin, Journalist/Journalistin, Verleger/Verlegerin}|pw}}{ }\strikeout{\textsc{A}}{ }Artikel\pwindex{Arthur Schnitzler. »Elskovsleg«s Forfatter@\emph{Arthur Schnitzler. »Elskovsleg«s Forfatter}|pwv} Dir \label{K_L02806-1v}\edtext{überſetzen laſſen}{\lemma{\textnormal{\emph{überſetzen laſſen}}}\Cendnote{\textnormal{Das hatte Schnitzler
                  jedenfalls vor, vgl. 
                     \emph{Peter Nansen – Arthur Schnitzler. Ein Briefwechsel zweier
                        Geistesverwandter}. Herausgegeben, kommentiert und mit einem Nachwort
                     versehen von Karin Bang. Roskilde: \emph{Zentrum für
                        österreichisch-nordische Kulturstudien}{ }2003, S. 7 (Småskrifter fra CØNK / Kleine Schriften
                     von ZÖNK 9).
               }}}\label{K_L02806-1}? Er iſt ungemein lieb und herzlich geſchrieben und ſehr ehrenvoll für uns
               Alle, insbeſondere natürlich für Dich.\pend
           
\pstart
           Je näher die Zeit heranrückt, wo ich Dich hier wiederſehen werde, mit umſo größerer
               Freude denke ich daran. Hab’ nur keine Furcht, daß ich mich werde von Arbeit
               Deinetwegen abhalten laſſen. Die Arbeit läßt mich hier einfach nicht los, wenn ſie
               einmal da iſt. Ich denke, wir werden namentlich am \strikeout{Tage}{ }Abend beiſammen ſein können, und oft auch am Tage. {\pb}\strikeout{Das} Die Hotel-Zimmer werde ich miethen, ſobald Du mir
               Deine Ankunft anzeigſt. Nur möchte ich auch eine kleine Idee von dem Preiſe haben,
               den Du zu zahlen gedenkſt. \strikeout{\textcolor{gray}{Nenne}} Nenne mir ein \textsc{Maximum}: etwa 8 bis 10 \textsc{Francs}{ }\textsc{pro} Tag und \textsc{pro} Zimmer,
                  alſo 16 bis 20 \textsc{Francs}{ }\textsc{pro} Tag? Ich hoffe, ich bekomme es billiger, aber ich will
               doch wiſſen, wie weit ich im Nothfall gehen darf?\pend
           
\pstart
           Welche Unannehmlichkeiten es im Gefolge haben ſollte, wenn Ihr unter Eurem wahren
               Namen Euch im \textsc{Hotel} einſchreibt, iſt mir dunkel. Ich kenne
               nur Fälle, wo es für Leute {\pb}\strikeout{U\textcolor{gray}{nn}} Unannehmlichkeiten im Gefolge gehabt hat, weil ſie unter ſalſchen Namen
               abgeſtiegen ſind. Die Polizei hat auch in \textsc{Paris\oindex{Paris@\textbf{Paris}, \emph{P.PPLC}|pw}} nichts dagegen, daß ein Menſch ſeinen wahren Namen führt.\pend
           
\pstart
           Auch bei der Idee, mir \textsc{Virginia}-Cigarren zuzuſenden,
               erkenne ich Dich wieder. Vielleicht gar in einem \label{K_L02806-2v}\edtext{recommandirten Briefe}{\lemma{\textnormal{\emph{recommandirten Briefe}}}\Cendnote{\textnormal{Siehe Paul Goldmann an Arthur Schnitzler, 21. 12. [1895].
               }}}\label{K_L02806-2}? Wiſſe denn, oh Freund, daß in Frankreich\oindex{Frankreich@\textbf{Frankreich}, \emph{A.PCLI}|pw}
               das Tabaks-Monopol beſteht. Jede Einfuhr \strikeout{fremd}
               ausländiſcher Cigarren iſt verboten. Privatleute müſſen, um Cigarren-Sendungen \strikeout{empf} aus dem {\pb}Auslande
               empfangen zu dürfen, eine beſondere Import-Erlaubniß vom Finanz-Miniſterium\orgindex{Franzoesisches Finanzministerium@Französisches Finanzministerium|pw} haben. Du kannſt \textsc{Virginia}-Cigarren nur ſo nach Frankreich\oindex{Frankreich@\textbf{Frankreich}, \emph{A.PCLI}|pw}
               bringen, daß Du ſie ſelbſt mit Dir nimmſt. An der Grenze ſagſt Du dann, daß Du Dich
               zwei Monate in Frankreich\oindex{Frankreich@\textbf{Frankreich}, \emph{A.PCLI}|pw} aufhalten willſt und
               für dieſe Zeit Dich mit Cigarren verſehen willſt. Dieſe Cigarren verzollſt Du dann
               (was eine Unſumme \strikeout{Gel} Geldes koſten wird). Oder aber,
               wenn Du Courage haſt, (die haſt Du aber wahrſcheinlich nicht), {\pb}ſo ſagſt Du gar nichts und verſuchſt die Cigarren
               einfach durchzuſchmuggeln.\pend
           
\pstart
           Dein \textsc{Bicycle} ſollſt Du gewiß mitnehmen. Die Umgebung von
                  \textsc{Paris\oindex{Paris@\textbf{Paris}, \emph{P.PPLC}|pw}} iſt eigens für \textsc{Bicycle}-Touren geſchaffen. Du wirſt
               hier zahlloſe und herrliche Ausflüge mit Deiner Maſchine machen können{\dotsfive}\pend
           
\pstart
           Traurig iſt es, daß Du Dir Dein junges und ſchönes Leben \strikeout{mit} durch ein Bischen \label{K_L02806-3v}\edtext{Ohrenklingen}{\lemma{\textnormal{\emph{Ohrenklingen}}}\Cendnote{\textnormal{Das Ohrenklingen aufgrund ders Otosklerose war gerade wieder akut, vgl. A. S.: \emph{Tagebuch}, 12. 3. 1897.
               }}}\label{K_L02806-3} verbittern läßt. Für mich iſt das gerade ein Beweis Deiner {\pb}\strikeout{G\textcolor{gray}{e}ſ\textcolor{gray}{×}} Geſundheit. Denn wenn Du irgend ein ernſtes Leiden hätteſt, ſo könnteſt Du
               nicht auf das Ohrenklingen achten. So concentrirt ſich darauf all’ \strikeout{d\textcolor{gray}{e}} Deine hypochondriſche Grübelei, die ſonſt, Gott ſei gelobt, kein \textsc{Sujet} in Deinem Organismus findet. Laß’ es doch klingen, zum
               Teufel, und denke nicht daran! Wenn Du nicht Medicin ſtudirt hätteſt, würdeſt Du gar
               nicht darauf achten!\pend
           
\pstart
           \strikeout{N\textcolor{gray}{un}} Nun erfahre ich wohl bald den genauen Tag Deiner Ankunft. {\pb}Das wird ſchön werden! \strikeout{\textcolor{gray}{×}\-\textcolor{gray}{×}\-\textcolor{gray}{×}}\pend
           
\pstart
           Traurig iſt nur, daß ich zu Oſtern auf 10 bis 14 Tage
               nach Frankfurt\oindex{Frankfurt am Main@\textbf{Frankfurt am Main}, \emph{P.PPLA3}|pw} muß. Nach \textsc{Nizza\oindex{Nizza@\textbf{Nizza}, \emph{P.PPLA2}|pw}} gehe ich nicht mehr.\pend
           
\pstart
           Wie hat »Liebelei\pwindex{Liebelei. Schauspiel in drei Akten@\emph{Liebelei. Schauspiel in drei Akten}|pw}« eigentlich in \textsc{Kopenhagen\oindex{Kopenhagen@\textbf{Kopenhagen}, \emph{P.PPLC}|pw}}{ }\label{K_L02806-4v}\edtext{gefallen}{\lemma{\textnormal{\emph{gefallen}}}\Cendnote{\textnormal{\emph{Liebelei}\pwindex{Liebelei. Schauspiel in drei Akten@\emph{Liebelei. Schauspiel in drei Akten}|pwk} wurde als \emph{Elskovsleg. Skuespil i 3 akter}\pwindex{Elskovsleg. Skuespil i 3 akter@\emph{Elskovsleg. Skuespil i 3 akter}|pwk} am 9. 3. 1897
                  am \emph{Folketeatret}\orgindex{Folketeatret@Folketeatret|pwk} uraufgeführt. Obgleich das
                  Stück von der Presse gelobt wurde, war es laut Nansen\pwindex{Nansen, Peter 20.01.1861 – 31.07.1918@\textsc{Nansen, Peter} (20.01.1861 – 31.07.1918), \emph{Schriftsteller/Schriftstellerin, Journalist/Journalistin, Verleger/Verlegerin}|pwk} aufgrund der schauspielerischen Leistungen kein wirklicher Erfolg.
                     Vgl. \emph{Peter Nansen – Arthur Schnitzler. Ein Briefwechsel
                        zweier Geistesverwandter}. Herausgegeben, kommentiert und mit einem
                     Nachwort versehen von Karin Bang. Roskilde:
                        \emph{Zentrum für österreichisch-nordische Kulturstudien}{ }2003, S. 8–9. (Småskrifter fra CØNK / Kleine Schriften
                     von ZÖNK 9)}}}\label{K_L02806-4}?\pend
           
\pstart
           Sei von Herzen gegrüßt und ſchreibe bald\textcolor{gray}{!}\pend
           
\pstart
           Dein treuer {\\[\baselineskip]}\spacefill\mbox{Paul Goldm}\pend
           \leftskip=0em{}\selectlanguage{ngerman}\endnumbering\briefempfaengerindex{Schnitzler, Arthur@\textsc{Schnitzler, Arthur}!zzzGoldmann, Paul@\emph{von Paul Goldmann}!1897-03-223@{22. 3. {[}1897{]}}|)be}\mylabel{L02806h}  \normalsize

\doendnotes{C}
\bigskip
\vfill

\clearpage

\footnotesize

\lohead{\textsc{register}}

% Definiere theindex-Environment komplett neu ohne reledmac
\makeatletter
\renewenvironment{theindex}{%
  \section*{\indexname}%
  \setlength{\parindent}{0pt}%
  \setlength{\parskip}{0pt plus 0.3pt}%
  \let\item\@idxitem
}{%
  \clearpage
}
\makeatother

\IfFileExists{\jobname-pw.ind}{\input{\jobname-pw.ind}}{}

\end{document}

      