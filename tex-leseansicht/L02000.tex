%% latex-leseansicht-vorspann.tex
%% Vorspann für die Leseansicht.
%% Lädt die gemeinsame Datei latex-vorspann.tex mit nicht gesetztem Schalter.

\newif\ifkorrekturansicht
\korrekturansichtfalse

\input{../tex-inputs/latex-vorspann}


\section[Arthur Schnitzler an Georg Brandes, 19. 1. 1911]{L02000 Arthur Schnitzler an Georg Brandes, 19. 1. 1911}
\nopagebreak\mylabel{L02000v}
\rehead{ }\normalsize\beginnumbering\briefempfaengerindex{Brandes, Georg@\textsc{Brandes, Georg}!zzzSchnitzler, Arthur@\emph{von Arthur Schnitzler}!1911-01-191@{19. 1. 1911}|(be}
\toendnotes[C]{\smallbreak\pagebreak[2]}
\correspDesc{Versand  durch Arthur Schnitzler am 19. 1. 1911 in Wien
\newline{}Erhalt  durch Georg Brandes im Zeitraum [19. 1. 1911
                  – 23. 1. 1911?] \textbf{Ort fehlend} }\toendnotes[C]{\smallbreak}
\Standort{Kopenhagen, Det Kongelige Bibliotek, Georg Brandes Arkiv, box 125.}
\physDesc{Brief, 3 Blätter, 5 Seiten, 3541 Zeichen (Zählung der Blätter 2 und 3)
\newline{}Handschrift: schwarze Tinte, lateinische Kurrent
\newline{}Ordnung: 1) auf der letzten Seite von unbekannter Hand mit Bleistift
                                 geschrieben: »\textsc{Schnitzler}«  2) mit Bleistift von unbekannter Hand nummeriert
                                    »30.«, Blatt 2 und 3 mit Datum »19/1 11« versehen und zwei Unterstreichungen}
\buchAbdrucke{\weitereDrucke{1) Georg Brandes, Arthur Schnitzler: \emph{Ein Briefwechsel}. Herausgegeben von Kurt Bergel. Bern: \emph{Francke} 1956, S. 99–100.} \weitereDrucke{2) Arthur Schnitzler: \emph{Briefe 1875–1912}. Herausgegeben von Therese Nickl und Heinrich Schnitzler. Frankfurt am Main: \emph{S. Fischer} 1981, S. 649–651.} }\toendnotes[C]{\smallbreak}
\pstart
           \raggedleft{}{\pb}XVIII. Sternwartestr 71\oindex{Wien@\textbf{Wien}!XVIII., Währing@\textbf{XVIII., Währing}!Sternwartestraße 71@\textbf{Sternwartestraße 71}, \emph{Wohngebäude}|pw}{\\}Wien\oindex{Wien@\textbf{Wien}, \emph{Verwaltungsgebiet}|pw}, 19. I. 911.\pend
           
\pstart{}Verehrter Herr Brandes,\pend\vspace{0.5em}
\pstart
           mit Ihrem Brief über den Medardus\pwindex{Schnitzler, Arthur 15.\,5.\,1862 Wien – 21.\,10.\,1931 ebd.@\textsc{Schnitzler, Arthur} (15.\,5.\,1862 Wien – 21.\,10.\,1931 ebd.), \emph{Schriftsteller, Mediziner}!junge Medardus. Dramatische Historie in einem Vorspiel und fünf Aufzügen@\strich\emph{Der junge Medardus. Dramatische Historie in einem Vorspiel und fünf Aufzügen}|pw} hab ich mich
               sehr gefreut. Der Erfolg hier dauert an; das Burgtheater\oindex{Wien@\textbf{Wien}!I., Innere Stadt@\textbf{I., Innere Stadt}!Burgtheater@\textbf{Burgtheater}, \emph{Theater}|pw} hatte seit Jahren nicht eine solche Reihe von ausverkauften
               Häusern; übrigens ist es eine vortreffliche Aufführung, und es wäre mir eine
               wirkliche Genugthuung we{\geminationn} Sie sie einmal sehen könnten.
               Natürlich ist unendlich viel gestrichen; darunter Scenen von bedeutender Wichtigkeit
               – und ich selbst war der Streicher; von der alten Theatererfahrung ausgehend, daß das
               Publikum gegen Längen empfindlicher ist als gegen Lücken. Ich hatte das Stück
               geschrieben, ohne die Eventualität einer Aufführung überhaupt in Betracht zu ziehen,
               ließ meine Phantasie und meine Feder laufen, wie es ihnen beliebte, {\pb}hatte aber natürlich immer die lebendigen
               Bühnenbilder vor mir, ohne recht zu glauben, daß es mir vergö{\geminationn}t sein würde, sie je in Wirklichkeit zu erblicken.
               Schon Schlenther\pwindex{Schlenther, Paul 20.\,8.\,1854 Chernyakhovsk – 30.\,4.\,1916 Berlin@\textsc{Schlenther, Paul} (20.\,8.\,1854 Chernyakhovsk – 30.\,4.\,1916 Berlin), \emph{Schriftsteller, Kritiker, Theaterleiter}|pw} nahm das Stück\pwindex{Schnitzler, Arthur 15.\,5.\,1862 Wien – 21.\,10.\,1931 ebd.@\textsc{Schnitzler, Arthur} (15.\,5.\,1862 Wien – 21.\,10.\,1931 ebd.), \emph{Schriftsteller, Mediziner}!junge Medardus. Dramatische Historie in einem Vorspiel und fünf Aufzügen@\strich\emph{Der junge Medardus. Dramatische Historie in einem Vorspiel und fünf Aufzügen}|pwv} an, konnte sich aber, in bekannter
               Weise nicht entschliessen, seine Absicht zur That zu machen; erst dem Baron Berger\pwindex{Berger, Alfred von 30.\,4.\,1853 Wien – 24.\,8.\,1912 ebd.@\textsc{Berger, Alfred von} (30.\,4.\,1853 Wien – 24.\,8.\,1912 ebd.), \emph{Schriftsteller, Journalist, Theaterleiter}|pw} verdankt \strikeout{ich} das Stück\pwindex{Schnitzler, Arthur 15.\,5.\,1862 Wien – 21.\,10.\,1931 ebd.@\textsc{Schnitzler, Arthur} (15.\,5.\,1862 Wien – 21.\,10.\,1931 ebd.), \emph{Schriftsteller, Mediziner}!junge Medardus. Dramatische Historie in einem Vorspiel und fünf Aufzügen@\strich\emph{Der junge Medardus. Dramatische Historie in einem Vorspiel und fünf Aufzügen}|pwv} sein
               Erwachen zum Bühnenleben. Seither ist schon manches andre fertig geworden und Sie,
               verehrter Freund, der allen meinen Arbeiten mit so wohlthuendem Interesse
               entgegenkommt, werden natürlich auch in den neuen und neuesten Fällen die
               Consequenzen zu tragen haben. –\pend
           
\pstart
           Denken Sie nicht dran, nach langer Zeit endlich wieder nach Wien\oindex{Wien@\textbf{Wien}, \emph{Verwaltungsgebiet}|pw} zu kommen? Wie gern möchte ich mit Ihnen reden, Sie in
               meinem Hause begrüßen – »Mein Haus\oindex{Wien@\textbf{Wien}!XVIII., Währing@\textbf{XVIII., Währing}!Sternwartestraße 71@\textbf{Sternwartestraße 71}, \emph{Wohngebäude}|pwv}« sag ich, denn im vergangenen {\pb}Sommer hab ich von Frau Bleibtreu\pwindex{Bleibtreu, Hedwig 23.\,12.\,1868 Linz – 24.\,1.\,1958 Wien@\textsc{Bleibtreu, Hedwig} (23.\,12.\,1868 Linz – 24.\,1.\,1958 Wien), \emph{Schauspielerin}|pw}, der Wittwe des Schauspielers Römpler\pwindex{Römpler, Alexander 12.\,3.\,1860 Berlin – 18.\,12.\,1909 Wien@\textsc{Römpler, Alexander} (12.\,3.\,1860 Berlin – 18.\,12.\,1909 Wien), \emph{Schauspieler, Schauspiellehrer}|pw} – (sie spielt die Frau Klaehr\pwindex{Schnitzler, Arthur 15.\,5.\,1862 Wien – 21.\,10.\,1931 ebd.@\textsc{Schnitzler, Arthur} (15.\,5.\,1862 Wien – 21.\,10.\,1931 ebd.), \emph{Schriftsteller, Mediziner}!junge Medardus. Dramatische Historie in einem Vorspiel und fünf Aufzügen@\strich\emph{Der junge Medardus. Dramatische Historie in einem Vorspiel und fünf Aufzügen}|pwv} im Medardus\pwindex{Schnitzler, Arthur 15.\,5.\,1862 Wien – 21.\,10.\,1931 ebd.@\textsc{Schnitzler, Arthur} (15.\,5.\,1862 Wien – 21.\,10.\,1931 ebd.), \emph{Schriftsteller, Mediziner}!junge Medardus. Dramatische Historie in einem Vorspiel und fünf Aufzügen@\strich\emph{Der junge Medardus. Dramatische Historie in einem Vorspiel und fünf Aufzügen}|pw}),
               eine kleine Villa\oindex{Wien@\textbf{Wien}!XVIII., Währing@\textbf{XVIII., Währing}!Sternwartestraße 71@\textbf{Sternwartestraße 71}, \emph{Wohngebäude}|pwv} im Cottage\oindex{Wien@\textbf{Wien}!XVIII., Währing@\textbf{XVIII., Währing}!Währinger Cottage@\textbf{Währinger Cottage}, \emph{Teil eines besiedelten Ortes}|pw} gekauft die ich mit
                  Frau\pwindex{Schnitzler, Olga 17.\,1.\,1882 Wien – 13.\,1.\,1970 Lugano@\textsc{Schnitzler, Olga} (17.\,1.\,1882 Wien – 13.\,1.\,1970 Lugano), \emph{Schauspielerin, Sängerin}|pwv} und Kindern\pwindex{Schnitzler, Heinrich 9.\,8.\,1902 Hinterbrühl – 12.\,7.\,1982 Wien@\textsc{Schnitzler, Heinrich} (9.\,8.\,1902 Hinterbrühl – 12.\,7.\,1982 Wien), \emph{Regisseur, Schauspieler}|pwv}\pwindex{Cappellini, Lili 13.\,9.\,1909 Wien – 26.\,7.\,1928 Venedig@\textsc{Cappellini, Lili} (13.\,9.\,1909 Wien – 26.\,7.\,1928 Venedig)|pwv} – (den Buben\pwindex{Schnitzler, Heinrich 9.\,8.\,1902 Hinterbrühl – 12.\,7.\,1982 Wien@\textsc{Schnitzler, Heinrich} (9.\,8.\,1902 Hinterbrühl – 12.\,7.\,1982 Wien), \emph{Regisseur, Schauspieler}|pwv}, der jetzt 8 Jahre ist, kennen Sie
               von Marienlyst\oindex{Marienlyst@\textbf{Marienlyst}, \emph{Gut}|pw} her, das Mädchen\pwindex{Cappellini, Lili 13.\,9.\,1909 Wien – 26.\,7.\,1928 Venedig@\textsc{Cappellini, Lili} (13.\,9.\,1909 Wien – 26.\,7.\,1928 Venedig)|pwv} ist kaum anderthalb Jahre alt)
               bewohne.\pend
           
\pstart
           So darf ich \introOben{}mich\introOben{} mancher inneren wie äußeren Erfolge
               erfreuen, und empfinde das viele Gute, das mir vom Schicksal beschieden, zuweilen so
               stark, daß ich jenes stetig fortschreitende Ohrenleiden, von dem ich seit 15 Jahren
               geplagt bin, gern als einen Polykratesring ansehen möchte – we{\geminationn}{ }\strikeout{\textcolor{gray}{ich}} auch als einen allzu werthvollen – und jedenfalls als einen, den kein Fischer
               der Welt mir jemals zurückbringen wird. –\pend
           
\pstart
           {\pb}Beer Hofmann\pwindex{Beer-Hofmann, Richard 11.\,7.\,1866 Wien – 26.\,9.\,1945 New York City@\textsc{Beer-Hofmann, Richard} (11.\,7.\,1866 Wien – 26.\,9.\,1945 New York City), \emph{Schriftsteller}|pw} mit seiner Frau\pwindex{Beer-Hofmann, Paula 25.\,2.\,1879 Wien – 30.\,10.\,1939 Zürich@\textsc{Beer-Hofmann, Paula} (25.\,2.\,1879 Wien – 30.\,10.\,1939 Zürich)|pwv} und seinen drei Kindern\pwindex{Beer-Hofmann, Mirjam 4.\,9.\,1897 Wien – 24.\,12.\,1984 New York City@\textsc{Beer-Hofmann, Mirjam} (4.\,9.\,1897 Wien – 24.\,12.\,1984 New York City)|pwv}\pwindex{Beer-Hofmann, Gabriel 9.\,1.\,1901 Wien – 24.\,3.\,1971 St Albans@\textsc{Beer-Hofmann, Gabriel} (9.\,1.\,1901 Wien – 24.\,3.\,1971 St Albans), \emph{Schriftsteller, Filmagent}|pwv}\pwindex{Beer-Hofmann, Naëmah 20.\,12.\,1898 Wien – 10.\,11.\,1971 New York City@\textsc{Beer-Hofmann, Naëmah} (20.\,12.\,1898 Wien – 10.\,11.\,1971 New York City)|pwv} wohnt ganz nahe
					von mir, in einem sehr schönen Haus\oindex{Wien@\textbf{Wien}!XVIII., Währing@\textbf{XVIII., Währing}!Hasenauerstraße 59@\textbf{Hasenauerstraße 59}, \emph{Wohngebäude}|pwv}, das ihm der Architekt Josef
                  Hoffmann\pwindex{Hoffmann, Josef 15.\,12.\,1870 Brtnice – 7.\,5.\,1956 Wien@\textsc{Hoffmann, Josef} (15.\,12.\,1870 Brtnice – 7.\,5.\,1956 Wien), \emph{Architekt, Kunstgewerbler}|pw} gebaut hat, und arbeitet nicht so viel, als er seinem Talent nach
               verpflichtet oder verurtheilt wäre. Sie sollten wieder einmal herkommen, – womöglich
               im Mai – man könnte einander so vieles erzählen; – in einer Stunde etwa zehn Mal so
               viel, als in zwei Briefen steht; das beste, was man von Menschen hat, die einem werth
               sind, bleiben doch die zwanglosen Unterhaltungen, die von der ganzen Atmosphäre der
               Persönlichkeit umgeben sind – was ist dagegen die gewollte Condensation und
               Praecision eines noch so herzlich intendirten Schreibens? {\pb}In Briefen will man was besti{\geminationm}tes sagen; – man dankt, man berichtet – man bezweckt; –
               in Gesprächen läßt man sich und den andern viel reiner leben, – man mag mit hundert
               Geheimnissen voneinander scheiden; – die Stimme, der Tonfall, die Geste geben selbst
               Befangenheiten, ja Unaufrichtigkeiten (die zwischen uns nicht zu befürchten sind)
               jene beste und einzige Wahrheit, an der wir uns erl\substVorne{}\textsuperscript{\textcolor{gray}{e}}\substDazwischen{}a\substHinten{}ben dürfen: Gegenwart.\pend
           
\pstart
           Dies soll Sie natürlich nur bestimmen (o welche Kraft traue ich schiefen Aphorismen
               zu!) nach Wien\oindex{Wien@\textbf{Wien}, \emph{Verwaltungsgebiet}|pw} zu reisen – aber Sie ja nicht
               abhalten, mich bald wieder durch ein paar geschriebene Worte zu erfreuen. In
               herzlicher Verehrung\pend
           
\pstart
           Ihr{\\[\baselineskip]}\spacefill\mbox{Arthur Schnitzler}\pend
           \leftskip=0em{}\selectlanguage{ngerman}\endnumbering\briefempfaengerindex{Brandes, Georg@\textsc{Brandes, Georg}!zzzSchnitzler, Arthur@\emph{von Arthur Schnitzler}!1911-01-191@{19. 1. 1911}|)be}\mylabel{L02000h}  \newcommand{\dateiname}{L02000}\newcommand{\titel}{Arthur Schnitzler an Georg Brandes, 19. 1. 1911}\newcommand{\editorInnen}{Martin Anton Müller und Gerd-Hermann Susen}%% latex-leseansicht-abspann.tex
%% Abspann für die Leseansicht.
%% Der Schalter \ifkorrekturansicht ist bereits durch den Vorspann gesetzt.

%% latex-abspann.tex
%% Gemeinsamer Abspann für Korrekturansicht und Leseansicht.
%% Setzt den Schalter \ifkorrekturansicht voraus (gesetzt in den
%% einbindenden Dateien latex-korrekturansicht-abspann.tex bzw.
%% latex-leseansicht-abspann.tex).
%% ---------------------------------------------------------------

\normalsize

% Das esempio-Environment wird nur in der Leseansicht benötigt
\ifkorrekturansicht\else
\newenvironment{esempio}[3]%
{
    \vspace{1.5ex}
    \rlap{\underline{#1}}
    \par
    \setlength{\parindent}{0cm}
    \nopagebreak
    \leftskip=#2cm
    \rightskip=#3cm
}
{
    \par
}
\fi

\doendnotes{C}
\bigskip
\vfill

\clearpage

\footnotesize

\ifkorrekturansicht
  \lohead{\textsc{register}}
\fi

% theindex-Environment neu definieren ohne reledmac
\makeatletter
\renewenvironment{theindex}{%
  \ifkorrekturansicht
    \section*{\indexname}%
  \else
    \subsubsection*{Index der erwähnten Entitäten}%
  \fi
  \setlength{\parindent}{0pt}%
  \setlength{\parskip}{0pt plus 0.3pt}%
  \let\item\@idxitem
}{%
  \ifkorrekturansicht\clearpage\fi
}
\makeatother

\IfFileExists{\jobname-pw.ind}{\input{\jobname-pw.ind}}{}

% Quellenangabe nur in der Leseansicht
\ifkorrekturansicht\else
% Fallback-Definitionen, falls die .tex-Datei \titel etc. nicht gesetzt hat
\providecommand{\titel}{}
\providecommand{\editorInnen}{}
\providecommand{\dateiname}{\jobname}

\vspace{3cm}

\vfill

\footnotesize
\textsc{Quelle}: \titel. Herausgegeben von {\editorInnen}. In: \emph{Arthur Schnitzler: Briefwechsel mit Autorinnen und Autoren}.
 Digitale Edition, https://schnitzler-briefe.acdh.oeaw.ac.at/{\dateiname}.html (Stand \today)
\fi

\end{document}


