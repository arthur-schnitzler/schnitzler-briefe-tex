%% latex-korrekturansicht-vorspann.tex
%% Vorspann für die Korrekturansicht.
%% Lädt die gemeinsame Datei latex-vorspann.tex mit gesetztem Schalter.

\newif\ifkorrekturansicht
\korrekturansichttrue

\input{../tex-inputs/latex-vorspann}


\section[Stefan Großmann an Arthur Schnitzler, 2. 6. 1926]{L02477 Stefan Großmann an Arthur Schnitzler, 2. 6. 1926}
\nopagebreak\mylabel{L02477v}
\rehead{ }\normalsize\beginnumbering\briefempfaengerindex{Schnitzler, Arthur@\textsc{Schnitzler, Arthur}!zzzGrossmann, Stefan@\emph{von Stefan Großmann}!1926-06-021@{2. 6. 1926}|(be}
\toendnotes[C]{\smallbreak\pagebreak[2]}\Standort{DLA, A:Schnitzler, HS.NZ85.1.3232.}
\physDesc{Brief, 1 Blatt, 1 Seite, 586 Zeichen
\newline{}Schreibmaschine
\newline{}Handschrift: schwarze Tinte, deutsche Kurrent (\noindent{}Unterschrift)
\newline{}Schnitzler: mit rotem Buntstift zwei Unterstreichungen }\toendnotes[C]{\smallbreak}
\pstart
           \centering{}{\pb}\textcolor{gray}{\textbf{Das Tage-Buch\orgindex{Tage-Buch@Das Tage-Buch|pw}}}\pend
           
\pstart
           \centering{}\textcolor{gray}{\textbf{\emph{Herausgeber: Stefan Großmann und Leopold Schwarzschild\pwindex{Schwarzschild, Leopold 1891-12-08 – 1950-10-02@\textsc{Schwarzschild, Leopold} (1891-12-08 – 1950-10-02), \emph{Publizist/Publizistin}|pw}}}}\pend
           
\pstart
           \centering{}\textcolor{gray}{\textbf{Tagebuchverlag m. b. H., Berlin SW 19\oindex{Berlin@\textbf{Berlin}, \emph{P.PPLC}|pw}}}\pend
           
\pstart
           \centering{}\textcolor{gray}{\textbf{BEUTHSTRASSE 19\oindex{Beuthstrasse@\textbf{Beuthstrasse}, \emph{Straße (K.STR)}|pw}}}\pend
           
\pstart
           \centering{}\textcolor{gray}{\textbf{\emph{Telegramm-Adresse: Tagebuch Berlin\oindex{Berlin@\textbf{Berlin}, \emph{P.PPLC}|pw} ⋅ Fernsprecher: Merkur 8790–8792}}}\pend
           
\pstart
           \centering{}\textcolor{gray}{\textbf{\emph{\so{Sprechstunde der Redaktion: 12–1 Uhr}}}}\pend
           
\pstart
           \centering{}\textcolor{gray}{\textbf{*}}\pend
           
\pstart
           Tgb./Gr./Schl.\hfill Berlin\oindex{Berlin@\textbf{Berlin}, \emph{P.PPLC}|pw}, den 2. Juni
                     1926.\pend
           {\vspace{1\baselineskip}}
\pstart
           \raggedleft{}Herrn\pend
           
\pstart
           \raggedleft{}Dr. Arthur \so{Schnitzler}\pend
           
\pstart
           \raggedleft{}\so{Wien} XVIII\oindex{XVIII., Waehring@\textbf{XVIII., Währing}, \emph{A.ADM3}|pw}\pend
           
\pstart
           \raggedleft{}Sternwartestr. 71.\oindex{Sternwartestrasse 71@\textbf{Sternwartestraße 71}, \emph{Wohngebäude (K.WHS)}|pw}\pend
           
\pstart{}Verehrter Herr Doktor Schnitzler!\pend\vspace{0.5em}
\pstart
           Ehe noch Ihr Brief kam, hatte ich das Versehen bemerkt und unseren Redakteur\pwindex{Ossietzky, Carl 1889-10-03 – 1938-05-04@\textsc{Ossietzky, Carl} (1889-10-03 – 1938-05-04), \emph{Schriftsteller/Schriftstellerin, Herausgeber/Herausgeberin, Publizist/Publizistin}|pwv} zur Rede gestellt.
               Selbstverständlich erscheint im nächsten Heft eine Richtigstellung\pwindex{Bemerkungen [Korrektur]@\emph{Bemerkungen [Korrektur]}|pwv}.\pend
           
\pstart
           Ich danke Ihnen sehr für die Liebenswürdigkeit Ihres Briefes und empfinde es nur als
               etwas bitter, dass Sie auf meine eigentliche Anfrage, ob Sie aus dem
               unveröffentlichten Buch\pwindex{Buch der Sprueche und Bedenken@\emph{Buch der Sprüche und Bedenken}|pwv} nicht
               mehreres fürs TAGE-BUCH\orgindex{Tage-Buch@Das Tage-Buch|pw} uns geben könnten, nicht geantwortet haben.\pend
           
\pstart
           Mit dankbaren Grüssen bin ich{\\[\baselineskip]}Ihr sehr ergebener{\\[\baselineskip]}\spacefill\mbox{{[}hs.:{]} Großmann}\pend
           \leftskip=0em{}\selectlanguage{ngerman}\endnumbering\briefempfaengerindex{Schnitzler, Arthur@\textsc{Schnitzler, Arthur}!zzzGrossmann, Stefan@\emph{von Stefan Großmann}!1926-06-021@{2. 6. 1926}|)be}\mylabel{L02477h}  \normalsize

\doendnotes{C}
\bigskip
\vfill

\clearpage

\footnotesize

\lohead{\textsc{register}}

% Definiere theindex-Environment komplett neu ohne reledmac
\makeatletter
\renewenvironment{theindex}{%
  \section*{\indexname}%
  \setlength{\parindent}{0pt}%
  \setlength{\parskip}{0pt plus 0.3pt}%
  \let\item\@idxitem
}{%
  \clearpage
}
\makeatother

\IfFileExists{\jobname-pw.ind}{\input{\jobname-pw.ind}}{}

\end{document}

      