%% latex-leseansicht-vorspann.tex
%% Vorspann für die Leseansicht.
%% Lädt die gemeinsame Datei latex-vorspann.tex mit nicht gesetztem Schalter.

\newif\ifkorrekturansicht
\korrekturansichtfalse

\input{../tex-inputs/latex-vorspann}

\begin{center}
            \textcolor{red}{ENTWURF, NICHT FERTIG KORRIGIERT}
                      \end{center}
            
         
         \renewcommand{\erwaehntePersonen}{Personen: Hermann Bahr, Felix Salten}
         \renewcommand{\erwaehnteOrte}{Orte: Gießhübl, Mödling, Rodaun, Wien}
         \renewcommand{\erwaehnteWerke}{}
               \section[Arthur Schnitzler an Felix Salten, {[}2. 5. 1895?{]}]{ Arthur Schnitzler an Felix Salten, {[}2. 5. 1895?{]}}\nopagebreak\mylabel{v}\rehead{ }\begin{ledgroupsized}[t]{13cm}\normalsize\beginnumbering \toendnotes[C]{\smallbreak\pagebreak[2]} \Standort{Wienbibliothek im Rathaus, ZPH 1681, 2.1.516.}
\physDesc{Brief, 1 Blatt, 2 Seiten, 266 Zeichen (Briefpapier mit Trauerrand)
\newline{}Handschrift: Bleistift, deutsche Kurrent
\newline{}Ordnung: mit Bleistift von unbekannter Hand Nummerierung der Blätter des
                                 Konvoluts: »31« }\toendnotes[C]{\smallbreak}\pstart{}{\pb}Lieber Salten, \pend\pstart
           \textsc{Bahr\pwindex{Bahr, Hermann 19.07.1863 – 15.01.1934@\textsc{Bahr, Hermann} (19.07.1863 – 15.01.1934), \emph{Schriftsteller, Kritiker}|pw}} hat mir \label{K_L03040-1v}\edtext{abgeſchrieben}{\lemma{\textnormal{\emph{abgeſchrieben}}}\Cendnote{\textnormal{Das dürfte sich auf Hermann Bahr an Arthur Schnitzler, 2. 5. 1894 beziehen, was die Datierung dieses
                  Korrespondenzstücks ermöglicht. Am 3. 5. 1895 machten Salten\pwindex{Salten, Felix 06.09.1869 – 08.10.1945@\textsc{Salten, Felix} (06.09.1869 – 08.10.1945), \emph{Schriftsteller, Journalist}|pwk} und Schnitzler\pwindex{Schnitzler, Arthur 15.05.1862 – 21.10.1931@\textsc{Schnitzler, Arthur} (15.05.1862 – 21.10.1931), \emph{Schriftsteller, Mediziner}|pwk}
                  einen gemeinsamen Ausflug nach Mödling\oindex{Moedling@\textbf{Mödling}|pwk}, Gießhübl\oindex{Giesshuebl@\textbf{Gießhübl}|pwk} und Rodaun\oindex{Rodaun@\textbf{Rodaun}|pwk}.}}}\label{K_L03040-1h}, alſo ſind wahrſcheinlich wir zwei allein. Bitte holen Sie
               mich alſo entweder um \label{K_L03040-12v}\edtext{¾ 9}{\lemma{\textnormal{\emph{¾ 9}}}\Cendnote{\textnormal{8 Uhr 45}}}\label{K_L03040-12h}{ }\introOben{}früh\introOben{} von Hauſe ab – oder ſorgen {\pb}Sie dafür, daſs eine Abſage bereits um
                  ½ 8 Morgens bei mir iſt, was ich übrigens nicht hoffe. \pend
           \pstart
           Herzliche Grüße {\\[\baselineskip]}\spacefill\mbox{Arthur.}\pend
           \leftskip=0em{}
         
         \endnumbering\mylabel{h}\end{ledgroupsized}\begin{anhang}\end{anhang}\newcommand{\dateiname}{L03040}\newcommand{\titel}{Arthur Schnitzler an Felix Salten, [2. 5. 1895?]}\newcommand{\editorInnen}{Martin Anton Müller und Laura Untner}%% latex-leseansicht-abspann.tex
%% Abspann für die Leseansicht.
%% Der Schalter \ifkorrekturansicht ist bereits durch den Vorspann gesetzt.

%% latex-abspann.tex
%% Gemeinsamer Abspann für Korrekturansicht und Leseansicht.
%% Setzt den Schalter \ifkorrekturansicht voraus (gesetzt in den
%% einbindenden Dateien latex-korrekturansicht-abspann.tex bzw.
%% latex-leseansicht-abspann.tex).
%% ---------------------------------------------------------------

\normalsize

% Das esempio-Environment wird nur in der Leseansicht benötigt
\ifkorrekturansicht\else
\newenvironment{esempio}[3]%
{
    \vspace{1.5ex}
    \rlap{\underline{#1}}
    \par
    \setlength{\parindent}{0cm}
    \nopagebreak
    \leftskip=#2cm
    \rightskip=#3cm
}
{
    \par
}
\fi

\doendnotes{C}
\bigskip
\vfill

\clearpage

\footnotesize

\ifkorrekturansicht
  \lohead{\textsc{register}}
\fi

% theindex-Environment neu definieren ohne reledmac
\makeatletter
\renewenvironment{theindex}{%
  \ifkorrekturansicht
    \section*{\indexname}%
  \else
    \subsubsection*{Index der erwähnten Entitäten}%
  \fi
  \setlength{\parindent}{0pt}%
  \setlength{\parskip}{0pt plus 0.3pt}%
  \let\item\@idxitem
}{%
  \ifkorrekturansicht\clearpage\fi
}
\makeatother

\IfFileExists{\jobname-pw.ind}{\input{\jobname-pw.ind}}{}

% Quellenangabe nur in der Leseansicht
\ifkorrekturansicht\else
% Fallback-Definitionen, falls die .tex-Datei \titel etc. nicht gesetzt hat
\providecommand{\titel}{}
\providecommand{\editorInnen}{}
\providecommand{\dateiname}{\jobname}

\vspace{3cm}

\vfill

\footnotesize
\textsc{Quelle}: \titel. Herausgegeben von {\editorInnen}. In: \emph{Arthur Schnitzler: Briefwechsel mit Autorinnen und Autoren}.
 Digitale Edition, https://schnitzler-briefe.acdh.oeaw.ac.at/{\dateiname}.html (Stand \today)
\fi

\end{document}


      