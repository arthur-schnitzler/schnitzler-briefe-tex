%% latex-leseansicht-vorspann.tex
%% Vorspann für die Leseansicht.
%% Lädt die gemeinsame Datei latex-vorspann.tex mit nicht gesetztem Schalter.

\newif\ifkorrekturansicht
\korrekturansichtfalse

\input{../tex-inputs/latex-vorspann}


\section[ Arthur Schnitzler an Felix Salten, [2. 5. 1894?]]{L03040 Arthur Schnitzler an Felix Salten,  [2. 5. 1894?]}
\nopagebreak\mylabel{L03040v}
\rehead{ }\normalsize\beginnumbering\briefempfaengerindex{Salten, Felix@\textsc{Salten, Felix}!zzzSchnitzler, Arthur@\emph{von Arthur Schnitzler}!1894-05-022@{{[}2. 5. 1894?{]}}|(be}
\toendnotes[C]{\smallbreak\pagebreak[2]}
\correspDesc{Versand  durch Arthur Schnitzler am [2. 5. 1894?] in Wien
\newline{}Erhalt  durch Felix Salten am [2. 5. 1894?] in Wien}\toendnotes[C]{\smallbreak}
\Standort{Wienbibliothek im Rathaus, ZPH 1681, 2.1.516.}
\physDesc{Brief, 1 Blatt, 2 Seiten, 263 Zeichen (Briefpapier mit Trauerrand)
\newline{}Handschrift: Bleistift, deutsche Kurrent
\newline{}Ordnung: mit Bleistift von unbekannter Hand nummeriert: »31« }\toendnotes[C]{\smallbreak}
\pstart{}{\pb}Lieber Salten,\pend\vspace{0.5em}
\pstart
           \textsc{Bahr\pwindex{Bahr, Hermann 19.\,7.\,1863 Linz – 15.\,1.\,1934 München@\textsc{Bahr, Hermann} (19.\,7.\,1863 Linz – 15.\,1.\,1934 München), \emph{Schriftsteller, Kritiker}|pw}} hat uns \label{K_L03040-1v}\edtext{abgeſchrieben}{\lemma{\textnormal{\emph{abgeschrieben}}}\Cendnote{\textnormal{Schnitzler dürfte sich auf dieses
                  Korrespondenzstück beziehen: XXXX Auszeichnungsfehler: Dokument L00318 nicht gefunden.
                  Dadurch wird die Datierung des vorliegenden Korrespondenzstücks möglich. Am 3. 5. 1894 machten
                     Salten\pwindex{Salten, Felix 6.\,9.\,1869 Budapest – 8.\,10.\,1945 Zürich@\textsc{Salten, Felix} (6.\,9.\,1869 Budapest – 8.\,10.\,1945 Zürich), \emph{Schriftsteller, Journalist, Chefredakteur}|pwk} und Schnitzler einen gemeinsamen Ausflug nach Mödling\oindex{Mödling@\textbf{Mödling}, \emph{Hauptstadt}|pwk}, Gießhübl\oindex{Gießhübl@\textbf{Gießhübl}, \emph{Hauptstadt}|pwk} und
                     Rodaun\oindex{Wien@\textbf{Wien}!XXIII., Liesing@\textbf{XXIII., Liesing}!Rodaun@\textbf{Rodaun}, \emph{Region}|pwk}.}}}\label{K_L03040-1}, alſo{ }ſind wahrſcheinlich
               wir zwei allein. Bitte holen Sie mich alſo entweder \introOben{}früh\introOben{} um
                  \label{K_L03040-2v}\edtext{¾ 9}{\lemma{\textnormal{\emph{¾ 9}}}\Cendnote{\textnormal{8 Uhr 45}}}\label{K_L03040-2} von Hauſe ab – oder{ }ſorgen {\pb}Sie dafür, daſs eine Abſage bereits
               um ½ 8 Morgens bei mir iſt, was ich übrigens nicht hoffe.\pend
           
\pstart
           Herzliche Grüße {\\[\baselineskip]}\spacefill\mbox{Arthur.}\pend
           \leftskip=0em{}\selectlanguage{ngerman}\endnumbering\briefempfaengerindex{Salten, Felix@\textsc{Salten, Felix}!zzzSchnitzler, Arthur@\emph{von Arthur Schnitzler}!1894-05-022@{{[}2. 5. 1894?{]}}|)be}\mylabel{L03040h}  \newcommand{\dateiname}{L03040}\newcommand{\titel}{Arthur Schnitzler an Felix Salten, [2. 5. 1894?]}\newcommand{\editorInnen}{Martin Anton Müller und Laura Untner}%% latex-leseansicht-abspann.tex
%% Abspann für die Leseansicht.
%% Der Schalter \ifkorrekturansicht ist bereits durch den Vorspann gesetzt.

%% latex-abspann.tex
%% Gemeinsamer Abspann für Korrekturansicht und Leseansicht.
%% Setzt den Schalter \ifkorrekturansicht voraus (gesetzt in den
%% einbindenden Dateien latex-korrekturansicht-abspann.tex bzw.
%% latex-leseansicht-abspann.tex).
%% ---------------------------------------------------------------

\normalsize

% Das esempio-Environment wird nur in der Leseansicht benötigt
\ifkorrekturansicht\else
\newenvironment{esempio}[3]%
{
    \vspace{1.5ex}
    \rlap{\underline{#1}}
    \par
    \setlength{\parindent}{0cm}
    \nopagebreak
    \leftskip=#2cm
    \rightskip=#3cm
}
{
    \par
}
\fi

\doendnotes{C}
\bigskip
\vfill

\clearpage

\footnotesize

\ifkorrekturansicht
  \lohead{\textsc{register}}
\fi

% theindex-Environment neu definieren ohne reledmac
\makeatletter
\renewenvironment{theindex}{%
  \ifkorrekturansicht
    \section*{\indexname}%
  \else
    \subsubsection*{Index der erwähnten Entitäten}%
  \fi
  \setlength{\parindent}{0pt}%
  \setlength{\parskip}{0pt plus 0.3pt}%
  \let\item\@idxitem
}{%
  \ifkorrekturansicht\clearpage\fi
}
\makeatother

\IfFileExists{\jobname-pw.ind}{\input{\jobname-pw.ind}}{}

% Quellenangabe nur in der Leseansicht
\ifkorrekturansicht\else
% Fallback-Definitionen, falls die .tex-Datei \titel etc. nicht gesetzt hat
\providecommand{\titel}{}
\providecommand{\editorInnen}{}
\providecommand{\dateiname}{\jobname}

\vspace{3cm}

\vfill

\footnotesize
\textsc{Quelle}: \titel. Herausgegeben von {\editorInnen}. In: \emph{Arthur Schnitzler: Briefwechsel mit Autorinnen und Autoren}.
 Digitale Edition, https://schnitzler-briefe.acdh.oeaw.ac.at/{\dateiname}.html (Stand \today)
\fi

\end{document}


