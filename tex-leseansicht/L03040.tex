%% latex-korrekturansicht-vorspann.tex
%% Vorspann für die Korrekturansicht.
%% Lädt die gemeinsame Datei latex-vorspann.tex mit gesetztem Schalter.

\newif\ifkorrekturansicht
\korrekturansichttrue

\input{../tex-inputs/latex-vorspann}


\section[ Arthur Schnitzler an Felix Salten, {[}2. 5. 1894?{]}]{L03040 Arthur Schnitzler an Felix Salten, {[}2. 5. 1894?{]}}
\nopagebreak\mylabel{L03040v}
\rehead{ }\normalsize\beginnumbering\briefempfaengerindex{Salten, Felix@\textsc{Salten, Felix}!zzzSchnitzler, Arthur@\emph{von Arthur Schnitzler}!1894-05-022@{{[}2. 5. 1894?{]}}|(be}
\toendnotes[C]{\smallbreak\pagebreak[2]}\Standort{Wienbibliothek im Rathaus, ZPH 1681, 2.1.516.}
\physDesc{Brief, 1 Blatt, 2 Seiten, 263 Zeichen (Briefpapier mit Trauerrand)
\newline{}Handschrift: Bleistift, deutsche Kurrent
\newline{}Ordnung: mit Bleistift von unbekannter Hand nummeriert: »31« }\toendnotes[C]{\smallbreak}
\pstart{}{\pb}Lieber Salten,\pend\vspace{0.5em}
\pstart
           \textsc{Bahr\pwindex{Bahr, Hermann 19.07.1863 – 15.01.1934@\textsc{Bahr, Hermann} (19.07.1863 – 15.01.1934), \emph{Schriftsteller/Schriftstellerin, Kritiker/Kritikerin}|pw}} hat uns \label{K_L03040-1v}\edtext{abgeſchrieben}{\lemma{\textnormal{\emph{abgeſchrieben}}}\Cendnote{\textnormal{Schnitzler dürfte sich auf dieses
                  Korrespondenzstück beziehen: Hermann Bahr an Arthur Schnitzler, 2. 5. 1894.
                  Dadurch wird die Datierung des vorliegenden Korrespondenzstücks möglich. Am 3. 5. 1894 machten
                     Salten\pwindex{Salten, Felix 06.09.1869 – 08.10.1945@\textsc{Salten, Felix} (06.09.1869 – 08.10.1945), \emph{Schriftsteller/Schriftstellerin, Journalist/Journalistin, Chefredakteur/Chefredakteurin}|pwk} und Schnitzler einen gemeinsamen Ausflug nach Mödling\oindex{Moedling@\textbf{Mödling}, \emph{P.PPLA3}|pwk}, Gießhübl\oindex{Giesshuebl@\textbf{Gießhübl}, \emph{P.PPLA3}|pwk} und
                     Rodaun\oindex{Rodaun@\textbf{Rodaun}, \emph{A.ADM4}|pwk}.}}}\label{K_L03040-1}, alſo ſind wahrſcheinlich
               wir zwei allein. Bitte holen Sie mich alſo entweder \introOben{}früh\introOben{} um
                  \label{K_L03040-2v}\edtext{¾ 9}{\lemma{\textnormal{\emph{¾ 9}}}\Cendnote{\textnormal{8 Uhr 45}}}\label{K_L03040-2} von Hauſe ab – oder
               ſorgen {\pb}Sie dafür, daſs eine Abſage bereits
               um ½ 8 Morgens bei mir iſt, was ich übrigens nicht hoffe.\pend
           
\pstart
           Herzliche Grüße {\\[\baselineskip]}\spacefill\mbox{Arthur.}\pend
           \leftskip=0em{}\selectlanguage{ngerman}\endnumbering\briefempfaengerindex{Salten, Felix@\textsc{Salten, Felix}!zzzSchnitzler, Arthur@\emph{von Arthur Schnitzler}!1894-05-022@{{[}2. 5. 1894?{]}}|)be}\mylabel{L03040h}  \normalsize

\doendnotes{C}
\bigskip
\vfill

\clearpage

\footnotesize

\lohead{\textsc{register}}

% Definiere theindex-Environment komplett neu ohne reledmac
\makeatletter
\renewenvironment{theindex}{%
  \section*{\indexname}%
  \setlength{\parindent}{0pt}%
  \setlength{\parskip}{0pt plus 0.3pt}%
  \let\item\@idxitem
}{%
  \clearpage
}
\makeatother

\IfFileExists{\jobname-pw.ind}{\input{\jobname-pw.ind}}{}

\end{document}

      