%% latex-leseansicht-vorspann.tex
%% Vorspann für die Leseansicht.
%% Lädt die gemeinsame Datei latex-vorspann.tex mit nicht gesetztem Schalter.

\newif\ifkorrekturansicht
\korrekturansichtfalse

\input{../tex-inputs/latex-vorspann}


         
         \renewcommand{\erwaehntePersonen}{Personen: Max Eugen Burckhard, Heinrich Schnitzler}
         \renewcommand{\erwaehnteOrte}{Orte: Edlach, Franzosenschanze, Niederösterreich, Reichenau an der Rax, St. Gilgen}
         \renewcommand{\erwaehnteWerke}{}
               \section[Max Burckhard an Arthur Schnitzler, 29. 7. 1909]{ Max Burckhard an Arthur Schnitzler, 29. 7. 1909}\nopagebreak\mylabel{v}\rehead{ }\begin{ledgroupsized}[t]{13cm}\normalsize\beginnumbering \toendnotes[C]{\smallbreak\pagebreak[2]} \Standort{CUL, Schnitzler, B 20.}
\physDesc{Bildpostkarte, 214 Zeichen
\newline{}Handschrift: schwarze Tinte, deutsche Kurrent
\newline{}Versand: Stempel: »\nobreak{}\oindex{St. Gilgen@\textbf{St. Gilgen}|pwk}St. Gilgen, 29. VII. 09, 3\nobreak{}«.  
\newline{}Schnitzler: mit Bleistift beschriftet »\textsc{Burckhard}« 
\newline{}Ordnung: mit Bleistift von unbekannter Hand
                                    nummeriert: »25« }\toendnotes[C]{\smallbreak}\pstart{}{\pb}H. Dr Artur Schnitzler\pend{}\pstart{}\textsc{Edlach}\oindex{Edlach@\textbf{Edlach}|pw}\pend{}\pstart{}bei Reichenau\oindex{Reichenau an der Rax@\textbf{Reichenau an der Rax}|pw}\pend{}\pstart{}NÖ\oindex{Niederoesterreich@\textbf{Niederösterreich}|pw}\pend{}{\bigskip}\pstart
           \noindent{}\centering{}{\pb}{[}Burckhards Haus auf der Franzosenschanze\oindex{Franzosenschanze@\textbf{Franzosenschanze}|pw} in St. Gilgen\oindex{St. Gilgen@\textbf{St. Gilgen}|pw}{]}\pend
           \pstart
           {\pb}Herzlichſten Glückwunſch zur raſchen
               Geneſung des Kleinen\pwindex{Schnitzler, Heinrich 09.08.1902 – 12.07.1982@\textsc{Schnitzler, Heinrich} (09.08.1902 – 12.07.1982), \emph{Regisseur, Schauspieler}|pwv} und
               alles Schöne und Gute für lieben verehrten Eltern.\pend
           \pstart \spacefill\mbox{DrBurc\textcolor{gray}{kh}ard}\pend{}
         
         \endnumbering\mylabel{h}\end{ledgroupsized}  \newcommand{\dateiname}{L01861}\newcommand{\titel}{Max Burckhard an Arthur Schnitzler, 29. 7. 1909}\newcommand{\editorInnen}{Martin Anton Müller und Gerd-Hermann Susen}%% latex-leseansicht-abspann.tex
%% Abspann für die Leseansicht.
%% Der Schalter \ifkorrekturansicht ist bereits durch den Vorspann gesetzt.

%% latex-abspann.tex
%% Gemeinsamer Abspann für Korrekturansicht und Leseansicht.
%% Setzt den Schalter \ifkorrekturansicht voraus (gesetzt in den
%% einbindenden Dateien latex-korrekturansicht-abspann.tex bzw.
%% latex-leseansicht-abspann.tex).
%% ---------------------------------------------------------------

\normalsize

% Das esempio-Environment wird nur in der Leseansicht benötigt
\ifkorrekturansicht\else
\newenvironment{esempio}[3]%
{
    \vspace{1.5ex}
    \rlap{\underline{#1}}
    \par
    \setlength{\parindent}{0cm}
    \nopagebreak
    \leftskip=#2cm
    \rightskip=#3cm
}
{
    \par
}
\fi

\doendnotes{C}
\bigskip
\vfill

\clearpage

\footnotesize

\ifkorrekturansicht
  \lohead{\textsc{register}}
\fi

% theindex-Environment neu definieren ohne reledmac
\makeatletter
\renewenvironment{theindex}{%
  \ifkorrekturansicht
    \section*{\indexname}%
  \else
    \subsubsection*{Index der erwähnten Entitäten}%
  \fi
  \setlength{\parindent}{0pt}%
  \setlength{\parskip}{0pt plus 0.3pt}%
  \let\item\@idxitem
}{%
  \ifkorrekturansicht\clearpage\fi
}
\makeatother

\IfFileExists{\jobname-pw.ind}{\input{\jobname-pw.ind}}{}

% Quellenangabe nur in der Leseansicht
\ifkorrekturansicht\else
% Fallback-Definitionen, falls die .tex-Datei \titel etc. nicht gesetzt hat
\providecommand{\titel}{}
\providecommand{\editorInnen}{}
\providecommand{\dateiname}{\jobname}

\vspace{3cm}

\vfill

\footnotesize
\textsc{Quelle}: \titel. Herausgegeben von {\editorInnen}. In: \emph{Arthur Schnitzler: Briefwechsel mit Autorinnen und Autoren}.
 Digitale Edition, https://schnitzler-briefe.acdh.oeaw.ac.at/{\dateiname}.html (Stand \today)
\fi

\end{document}


      