%% latex-leseansicht-vorspann.tex
%% Vorspann für die Leseansicht.
%% Lädt die gemeinsame Datei latex-vorspann.tex mit nicht gesetztem Schalter.

\newif\ifkorrekturansicht
\korrekturansichtfalse

\input{../tex-inputs/latex-vorspann}


\section[ Felix Salten an Arthur Schnitzler, 21. 7. 1896]{L03175 Felix Salten an Arthur Schnitzler,  21. 7. 1896}
\nopagebreak\mylabel{L03175v}
\rehead{ }\normalsize\beginnumbering\briefempfaengerindex{Schnitzler, Arthur@\textsc{Schnitzler, Arthur}!zzzSalten, Felix@\emph{von Felix Salten}!1896-07-211@{21. 7. 1896}|(be}
\toendnotes[C]{\smallbreak\pagebreak[2]}
\correspDesc{Versand  durch Felix Salten am 21. 7. 1896 in Wien
\newline{}Erhalt  durch Arthur Schnitzler am [23. 7. 1896?] in Trondheim}\toendnotes[C]{\smallbreak}
\Standort{CUL, Schnitzler, B 89, A 1.}
\physDesc{Brief, 1 Blatt, 4 Seiten, 2208 Zeichen
\newline{}Handschrift: Bleistift, lateinische Kurrent
\newline{}Schnitzler: mit Bleistift die Jahreszahl »96« ergänzt 
\newline{}Ordnung: mit Bleistift von unbekannter Hand nummeriert: »74« }\toendnotes[C]{\smallbreak}
\pstart
           \raggedleft{}{\pb}Wien\oindex{Wien@\textbf{Wien}, \emph{Verwaltungsgebiet}|pw}, den 21. Juli\pend
           \vspace{0.5em}
\pstart
           Lieber Arthur, in dieser Welt geht garnichts vor, und es ist ganz
               gleichgiltig, ob man jetzt in Iglau\oindex{Jihlava@\textbf{Jihlava}|pw} lebt oder
               auf dem \label{K_L03175-1v}\edtext{Nordcap\oindex{Nordkap@\textbf{Nordkap}, \emph{Kap}|pw}}{\lemma{\textnormal{\emph{Nordcap}}}\Cendnote{\textnormal{Womöglich reagiert Salten\pwindex{Salten, Felix 6.\,9.\,1869 Budapest – 8.\,10.\,1945 Zürich@\textsc{Salten, Felix} (6.\,9.\,1869 Budapest – 8.\,10.\,1945 Zürich), \emph{Schriftsteller, Journalist, Chefredakteur}|pwk} schon
                  auf eine Karte vom Nordkap\oindex{Nordkap@\textbf{Nordkap}, \emph{Kap}|pwk}, das 
                  Schnitzler am 19. 7. 1896 besucht hatte.}}}\label{K_L03175-1} ist. Auf dem Nordcap\oindex{Nordkap@\textbf{Nordkap}, \emph{Kap}|pw} ist’s besser, das ist das Ganze. Von
               grossen Ereignissen hab ich Ihnen nur zu melden, dass \label{K_L03175-2v}\edtext{Frau Seiler-Willborn\pwindex{Seiler-Willborn, Ilma 1850 Ungarn – 16.\,7.\,1896 Wien@\textsc{Seiler-Willborn, Ilma} (1850 Ungarn – 16.\,7.\,1896 Wien), \emph{Schauspielerin}|pw}
               plötzlich gestorben}{\lemma{\textnormal{\emph{Frau … gestorben}}}\Cendnote{\textnormal{Die Schauspielerin
                     Ilma Seiler-Willborn\pwindex{Seiler-Willborn, Ilma 1850 Ungarn – 16.\,7.\,1896 Wien@\textsc{Seiler-Willborn, Ilma} (1850 Ungarn – 16.\,7.\,1896 Wien), \emph{Schauspielerin}|pwk} war am 16. 7. 1896 in Wien\oindex{Wien@\textbf{Wien}, \emph{Verwaltungsgebiet}|pwk}
                  verstorben.}}}\label{K_L03175-2} ist, ferner dass man in Ischl\oindex{Bad Ischl@\textbf{Bad Ischl}|pw} nächstens Ihre \label{K_L03175-3v}\edtext{»Liebelei\pwindex{Schnitzler, Arthur 15.\,5.\,1862 Wien – 21.\,10.\,1931 ebd.@\textsc{Schnitzler, Arthur} (15.\,5.\,1862 Wien – 21.\,10.\,1931 ebd.), \emph{Schriftsteller, Mediziner}!Liebelei. Schauspiel in drei Akten@\strich\emph{Liebelei. Schauspiel in drei Akten}|pw}« aufführen}{\lemma{\textnormal{\emph{»Liebelei« aufführen}}}\Cendnote{\textnormal{durch das \emph{Saisontheather
                     Ischl}\orgindex{Saisontheater Ischl@Saisontheater Ischl|pwk}}}}\label{K_L03175-3} wird. Doch dürfte Sie weder der eine noch der andere Unglücksfall zu sehr
               erschüttern. Diesen Sonntag bin ich in Ischl\oindex{Bad Ischl@\textbf{Bad Ischl}|pw} gewesen, vielmehr in Aussee\oindex{Bad Aussee@\textbf{Bad Aussee}, \emph{Hauptstadt}|pw}, denn ich fuhr gleich in der Früh mit Frl. M.\pwindex{Salten, Ottilie 7.\,3.\,1868 Prag – 22.\,6.\,1942 Zürich@\textsc{Salten, Ottilie} (7.\,3.\,1868 Prag – 22.\,6.\,1942 Zürich), \emph{Schauspielerin}|pw}{ }{\pb}dahin. Es schüttete in
               Strömen und wir blieben den ganzen Tag bei Frau Mitterwurzer\pwindex{Mitterwurzer, Wilhelmine 27.\,3.\,1848 Freiburg im Breisgau – 3.\,8.\,1909 Wien@\textsc{Mitterwurzer, Wilhelmine} (27.\,3.\,1848 Freiburg im Breisgau – 3.\,8.\,1909 Wien), \emph{Schauspielerin}|pw}. Ich gehe nun doch nicht ins Ampezzothal\oindex{Valle d’Ampezzo@\textbf{Valle d’Ampezzo}, \emph{Tal}|pw}. Meine Adresse vom 1–7. Aug. ist jetzt Ischl\oindex{Bad Ischl@\textbf{Bad Ischl}|pw}. Von da an München\oindex{München@\textbf{München}|pw} bis zum 12. und von da ab Salzburg\oindex{Salzburg@\textbf{Salzburg}, \emph{Verwaltungsgebiet}|pw} bis zum 20. August. Wir fahren wie
               Sie daraus sehen von Salzburg\oindex{Salzburg@\textbf{Salzburg}, \emph{Verwaltungsgebiet}|pw} per Rad nach München\oindex{München@\textbf{München}|pw}, von da über Schliersee\oindex{Schliersee@\textbf{Schliersee}|pw}, Tegernsee\oindex{Tegernsee@\textbf{Tegernsee}|pw}
               nach Innsbruck\oindex{Innsbruck@\textbf{Innsbruck}, \emph{Verwaltungsgebiet}|pw} und von dort nach Salzburg\oindex{Salzburg@\textbf{Salzburg}, \emph{Verwaltungsgebiet}|pw}. Das ist Alles. Indessen bin ich ununterbrochen zu
               Hause, lese und arbeite. Zeitlin\pwindex{Zeitlin, Alexander 15.\,7.\,1872 Tiflis – 4.\,3.\,1946 New York City@\textsc{Zeitlin, Alexander} (15.\,7.\,1872 Tiflis – 4.\,3.\,1946 New York City), \emph{Bildhauer}|pw} hat keinen
                  \label{K_L03175-4v}\edtext{Preis}{\lemma{\textnormal{\emph{Preis}}}\Cendnote{\textnormal{der \emph{Akademie der bildenden
                     Künste}\orgindex{Akademie der Bildenden Künste Wien@Akademie der Bildenden Künste Wien|pwk} in Wien\oindex{Wien@\textbf{Wien}, \emph{Verwaltungsgebiet}|pwk}}}}\label{K_L03175-4} beko{\geminationm}en, Popper\pwindex{Pongrácz, Szigfrid 14.\,6.\,1872 Brünn – 6.\,2.\,1929 Budapest@\textsc{Pongrácz, Szigfrid} (14.\,6.\,1872 Brünn – 6.\,2.\,1929 Budapest), \emph{Bildhauer}|pw}, der mit einer geradezu herrlichen Gruppe »Adam und Eva\pwindex{Pongrácz, Szigfrid 14.\,6.\,1872 Brünn – 6.\,2.\,1929 Budapest@\textsc{Pongrácz, Szigfrid} (14.\,6.\,1872 Brünn – 6.\,2.\,1929 Budapest), \emph{Bildhauer}!Adam und Eva@\strich\emph{Adam und Eva}|pw}« um den {\pb}Rompreis concurrirte{[},{]}
               wurde mit dem Specialschulpreis abgefunden. Ich schrieb einen \label{K_L03175-5v}\edtext{Leitartikel\pwindex{Salten, Felix 6.\,9.\,1869 Budapest – 8.\,10.\,1945 Zürich@\textsc{Salten, Felix} (6.\,9.\,1869 Budapest – 8.\,10.\,1945 Zürich), \emph{Schriftsteller, Journalist, Chefredakteur}!Schülerausstellung der Akademie@\strich\emph{Die Schülerausstellung der Akademie}|pwv}}{\lemma{\textnormal{\emph{Leitartikel}}}\Cendnote{\textnormal{f. s.\pwindex{Salten, Felix 6.\,9.\,1869 Budapest – 8.\,10.\,1945 Zürich@\textsc{Salten, Felix} (6.\,9.\,1869 Budapest – 8.\,10.\,1945 Zürich), \emph{Schriftsteller, Journalist, Chefredakteur}|pwk} [ = Felix Salten\pwindex{Salten, Felix 6.\,9.\,1869 Budapest – 8.\,10.\,1945 Zürich@\textsc{Salten, Felix} (6.\,9.\,1869 Budapest – 8.\,10.\,1945 Zürich), \emph{Schriftsteller, Journalist, Chefredakteur}|pwk}]: \emph{Die
                        Schülerausstellung der Akademie}\pwindex{Salten, Felix 6.\,9.\,1869 Budapest – 8.\,10.\,1945 Zürich@\textsc{Salten, Felix} (6.\,9.\,1869 Budapest – 8.\,10.\,1945 Zürich), \emph{Schriftsteller, Journalist, Chefredakteur}!Schülerausstellung der Akademie@\strich\emph{Die Schülerausstellung der Akademie}|pwk}. In: \emph{Wiener Allgemeine Zeitung}\pwindex{Wiener Allgemeine Zeitung@\emph{Wiener Allgemeine Zeitung}|pwk}, Nr. 5517, 21. 7. 1896, S. 4.}}}\label{K_L03175-5} über die Zustände an der Akademie\orgindex{Akademie der Bildenden Künste Wien@Akademie der Bildenden Künste Wien|pw}, musste aber zahm sein, da man in kein
               Wespennest stechen will. Doch denke ich mich in der \label{K_L03175-6v}\edtext{Frankft. Ztg\pwindex{Frankfurter Zeitung@\emph{Frankfurter Zeitung}|pw} weitläufiger über die Sache
                  auszulaßen}{\lemma{\textnormal{\emph{Frankft. … auszulaßen}}}\Cendnote{\textnormal{nicht 
                  nachgewiesen}}}\label{K_L03175-6}. Dass \label{K_L03175-7v}\edtext{Edmond de Goncourt\pwindex{Goncourt, Edmond Huot de 26.\,5.\,1822 Nancy – 16.\,7.\,1896 Draveil@\textsc{Goncourt, Edmond Huot de} (26.\,5.\,1822 Nancy – 16.\,7.\,1896 Draveil), \emph{Schriftsteller}|pw} tot}{\lemma{\textnormal{\emph{Edmond de Goncourt tot}}}\Cendnote{\textnormal{Edmond
                     de Goncourt\pwindex{Goncourt, Edmond Huot de 26.\,5.\,1822 Nancy – 16.\,7.\,1896 Draveil@\textsc{Goncourt, Edmond Huot de} (26.\,5.\,1822 Nancy – 16.\,7.\,1896 Draveil), \emph{Schriftsteller}|pwk} starb am 16. 7. 1896 in Draveil\oindex{Draveil@\textbf{Draveil}|pwk}.}}}\label{K_L03175-7} ist, werden Sie
               vielleicht schon erfahren haben. Er starb in dem Schloße\oindex{Haus von Alphonse Daudet@\textbf{Haus von Alphonse Daudet}, \emph{Wohngebäude}|pwv} von Daudet\pwindex{Daudet, Alphonse 13.\,5.\,1840 Nîmes – 16.\,11.\,1897 Paris@\textsc{Daudet, Alphonse} (13.\,5.\,1840 Nîmes – 16.\,11.\,1897 Paris), \emph{Schriftsteller}|pw}.
               Die  Wien\oindex{Wien@\textbf{Wien}, \emph{Verwaltungsgebiet}|pw}er Schornalisten, welche die letzte
                  \label{K_L03175-8v}\edtext{Flegelei\pwindex{† Edmond de Goncourt@\emph{† Edmond de Goncourt}|pwv}}{\lemma{\textnormal{\emph{Flegelei}}}\Cendnote{\textnormal{[Max Nordau\pwindex{Nordau, Max 29.\,7.\,1849 Budapest – 22.\,1.\,1923 Paris@\textsc{Nordau, Max} (29.\,7.\,1849 Budapest – 22.\,1.\,1923 Paris), \emph{Schriftsteller, Mediziner}|pwk}]: \emph{† Edmond de Goncourt}\pwindex{† Edmond de Goncourt@\emph{† Edmond de Goncourt}|pwk}. In: \emph{Neue Freie Presse}\pwindex{Neue Freie Presse@\emph{Neue Freie Presse}|pwk}, Nr. 11.457, 17. 7. 1896, Morgenblatt, S. 5.}}}\label{K_L03175-8}{ }Nordau\pwindex{Nordau, Max 29.\,7.\,1849 Budapest – 22.\,1.\,1923 Paris@\textsc{Nordau, Max} (29.\,7.\,1849 Budapest – 22.\,1.\,1923 Paris), \emph{Schriftsteller, Mediziner}|pw}’s als Quelle über Goncourt\pwindex{Goncourt, Edmond Huot de 26.\,5.\,1822 Nancy – 16.\,7.\,1896 Draveil@\textsc{Goncourt, Edmond Huot de} (26.\,5.\,1822 Nancy – 16.\,7.\,1896 Draveil), \emph{Schriftsteller}|pw} benützten, schrieben in guten \label{K_L03175-9v}\edtext{Notizelach}{\lemma{\textnormal{\emph{Notizelach}}}\Cendnote{\textnormal{Durch Anhang einer jiddischen Endsilbe spielt Salten\pwindex{Salten, Felix 6.\,9.\,1869 Budapest – 8.\,10.\,1945 Zürich@\textsc{Salten, Felix} (6.\,9.\,1869 Budapest – 8.\,10.\,1945 Zürich), \emph{Schriftsteller, Journalist, Chefredakteur}|pwk} darauf an, dass Nordau\pwindex{Nordau, Max 29.\,7.\,1849 Budapest – 22.\,1.\,1923 Paris@\textsc{Nordau, Max} (29.\,7.\,1849 Budapest – 22.\,1.\,1923 Paris), \emph{Schriftsteller, Mediziner}|pwk} Jude war und überhaupt die Wien\oindex{Wien@\textbf{Wien}, \emph{Verwaltungsgebiet}|pwk}er Presselandschaft in Verruf stand, hauptsächlich aus Juden
                  zusammengesetzt zu sein. Da Salten\pwindex{Salten, Felix 6.\,9.\,1869 Budapest – 8.\,10.\,1945 Zürich@\textsc{Salten, Felix} (6.\,9.\,1869 Budapest – 8.\,10.\,1945 Zürich), \emph{Schriftsteller, Journalist, Chefredakteur}|pwk} selbst
                  jüdischer Abstammung war, dürfte er damit weniger einen unmittelbaren antisemitischen Reflex ausdrücken,
                  als ein negatives Bild bestimmter journalistischer Praktiken
                  zeichnen.}}}\label{K_L03175-9}, er sei der populärste und \uline{platteste}{ }{\pb}Schriftsteller Frankreichs\oindex{Frankreich@\textbf{Frankreich}|pw} gewesen. Herr Ohnet\pwindex{Ohnet, Georges 3.\,4.\,1848 Paris – 5.\,5.\,1918 ebd.@\textsc{Ohnet, Georges} (3.\,4.\,1848 Paris – 5.\,5.\,1918 ebd.), \emph{Schriftsteller}|pw} würde sich freuen. Nach seinem Testament wird eine »freie Akademie\orgindex{Académie Goncourt@Académie Goncourt|pwv}« gegründet, deren Präsident
                  Daudet\pwindex{Daudet, Alphonse 13.\,5.\,1840 Nîmes – 16.\,11.\,1897 Paris@\textsc{Daudet, Alphonse} (13.\,5.\,1840 Nîmes – 16.\,11.\,1897 Paris), \emph{Schriftsteller}|pw} ist, und deren einzelne Mitglieder
               eine Rente von 6000 Frcs aus dem Vermögen Goncourts\pwindex{Goncourt, Jules Huot de 17.\,12.\,1830 Paris – 20.\,6.\,1870 ebd.@\textsc{Goncourt, Jules Huot de} (17.\,12.\,1830 Paris – 20.\,6.\,1870 ebd.), \emph{Schriftsteller}|pw}\pwindex{Goncourt, Edmond Huot de 26.\,5.\,1822 Nancy – 16.\,7.\,1896 Draveil@\textsc{Goncourt, Edmond Huot de} (26.\,5.\,1822 Nancy – 16.\,7.\,1896 Draveil), \emph{Schriftsteller}|pw} erhalten. Diese Lust der Fran\oindex{Frankreich@\textbf{Frankreich}|pwv}zosen nach Vereinigungen und ihr Verlangen, dass die
               Berühmtheit durch Zeremonien bestätigt werde, hat etwas, wenn auch nicht viel von
               unserem »hohen Orden«, der freilich schöner ist. Schon deshalb weil er nicht
               exisitirt. Schreiben Sie bald und grüßen Richard\pwindex{Beer-Hofmann, Richard 11.\,7.\,1866 Wien – 26.\,9.\,1945 New York City@\textsc{Beer-Hofmann, Richard} (11.\,7.\,1866 Wien – 26.\,9.\,1945 New York City), \emph{Schriftsteller}|pw}. Die Zeitungen schicke ich Ihnen nun schon nach \label{K_L03175-10v}\edtext{Kopenhagen\oindex{Kopenhagen@\textbf{Kopenhagen}, \emph{Hauptstadt}|pw}}{\lemma{\textnormal{\emph{Kopenhagen}}}\Cendnote{\textnormal{Schnitzler hielt sich vom 2. 8. 1896 bis zum 3. 8. 1896 sowie am
                     22. 8. 1896 in
                     Kopenhagen\oindex{Kopenhagen@\textbf{Kopenhagen}, \emph{Hauptstadt}|pwk} auf. Dazwischen war er in Skodsborg\oindex{Skodsborg@\textbf{Skodsborg}|pwk}.}}}\label{K_L03175-10}.\pend
           
\pstart
           Herzlichst Ihr {\\[\baselineskip]}\spacefill\mbox{Salten}\pend
           \leftskip=0em{}\selectlanguage{ngerman}\endnumbering\briefempfaengerindex{Schnitzler, Arthur@\textsc{Schnitzler, Arthur}!zzzSalten, Felix@\emph{von Felix Salten}!1896-07-211@{21. 7. 1896}|)be}\mylabel{L03175h}  \newcommand{\dateiname}{L03175}\newcommand{\titel}{Felix Salten an Arthur Schnitzler, 21. 7. 1896}\newcommand{\editorInnen}{Martin Anton Müller und Laura Untner}%% latex-leseansicht-abspann.tex
%% Abspann für die Leseansicht.
%% Der Schalter \ifkorrekturansicht ist bereits durch den Vorspann gesetzt.

%% latex-abspann.tex
%% Gemeinsamer Abspann für Korrekturansicht und Leseansicht.
%% Setzt den Schalter \ifkorrekturansicht voraus (gesetzt in den
%% einbindenden Dateien latex-korrekturansicht-abspann.tex bzw.
%% latex-leseansicht-abspann.tex).
%% ---------------------------------------------------------------

\normalsize

% Das esempio-Environment wird nur in der Leseansicht benötigt
\ifkorrekturansicht\else
\newenvironment{esempio}[3]%
{
    \vspace{1.5ex}
    \rlap{\underline{#1}}
    \par
    \setlength{\parindent}{0cm}
    \nopagebreak
    \leftskip=#2cm
    \rightskip=#3cm
}
{
    \par
}
\fi

\doendnotes{C}
\bigskip
\vfill

\clearpage

\footnotesize

\ifkorrekturansicht
  \lohead{\textsc{register}}
\fi

% theindex-Environment neu definieren ohne reledmac
\makeatletter
\renewenvironment{theindex}{%
  \ifkorrekturansicht
    \section*{\indexname}%
  \else
    \subsubsection*{Index der erwähnten Entitäten}%
  \fi
  \setlength{\parindent}{0pt}%
  \setlength{\parskip}{0pt plus 0.3pt}%
  \let\item\@idxitem
}{%
  \ifkorrekturansicht\clearpage\fi
}
\makeatother

\IfFileExists{\jobname-pw.ind}{\input{\jobname-pw.ind}}{}

% Quellenangabe nur in der Leseansicht
\ifkorrekturansicht\else
% Fallback-Definitionen, falls die .tex-Datei \titel etc. nicht gesetzt hat
\providecommand{\titel}{}
\providecommand{\editorInnen}{}
\providecommand{\dateiname}{\jobname}

\vspace{3cm}

\vfill

\footnotesize
\textsc{Quelle}: \titel. Herausgegeben von {\editorInnen}. In: \emph{Arthur Schnitzler: Briefwechsel mit Autorinnen und Autoren}.
 Digitale Edition, https://schnitzler-briefe.acdh.oeaw.ac.at/{\dateiname}.html (Stand \today)
\fi

\end{document}


