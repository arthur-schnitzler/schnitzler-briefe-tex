%% latex-leseansicht-vorspann.tex
%% Vorspann für die Leseansicht.
%% Lädt die gemeinsame Datei latex-vorspann.tex mit nicht gesetztem Schalter.

\newif\ifkorrekturansicht
\korrekturansichtfalse

\input{../tex-inputs/latex-vorspann}

\begin{center}
            \textcolor{red}{ENTWURF, NICHT FERTIG KORRIGIERT}
                      \end{center}
            
         
         \renewcommand{\erwaehntePersonen}{Personen: Richard Beer-Hofmann, Alphonse Daudet, Edmond Huot de Goncourt, Jules Huot de Goncourt, Wilhelmine Mitterwurzer, Max Nordau, Szigfrid Pongrácz, Ottilie Salten, Ilma Seiler-Willborn, Alexander Zeitlin}
         \renewcommand{\erwaehnteInstitutionen}{Institutionen: Académie Goncourt, Akademie der Bildenden Künste Wien}
         \renewcommand{\erwaehnteOrte}{Orte: Bad Aussee, Bad Ischl, Frankreich, Haus von Alphonse Daudet, Innsbruck, Jihlava, Kopenhagen, München, Nordkap, Salzburg, Schliersee, Tegernsee, Valle d’Ampezzo, Wien}
         \renewcommand{\erwaehnteWerke}{Werke: Adam und Eva, Die Schülerausstellung der Akademie, Frankfurter Zeitung, Liebelei. Schauspiel in drei Akten, Neue Freie Presse, Wiener Allgemeine Zeitung, † Edmond de Goncourt}
               \section[Felix Salten an Arthur Schnitzler, 21. 7. 1896]{ Felix Salten an Arthur Schnitzler, 21. 7. 1896}\nopagebreak\mylabel{v}\rehead{ }\begin{ledgroupsized}[t]{13cm}\normalsize\beginnumbering \toendnotes[C]{\smallbreak\pagebreak[2]} \Standort{CUL, Schnitzler, B 89, A 1.}
\physDesc{, 1 Blatt, 4 Seiten
\newline{}Handschrift: Bleistift, lateinische Kurrent
\newline{}Schnitzler: mit Bleistift die Jahreszahl »96« ergänzt }\toendnotes[C]{\smallbreak}\pstart
           \raggedleft{}{\pb}Wien\oindex{Wien@\textbf{Wien}|pw}, den 21. Juli\pend
           \pstart
           Lieber Arthur, in dieser Welt geht garnichts vor, und es ist ganz
               gleichgiltig, ob man jetzt in Iglau\oindex{Jihlava@\textbf{Jihlava}|pw} lebt oder
               auf dem Nordcap\oindex{Nordkap@\textbf{Nordkap}|pw} ist. Auf dem Nordcap\oindex{Nordkap@\textbf{Nordkap}|pw} ist’s besser, da ist das Ganze. Von grossen
               Ereignissen hab ich Ihnen nur zu melden, dass Frau Seiler-Willborn\pwindex{Seiler-Willborn, Ilma 1850 – 1896-07-16@\textsc{Seiler-Willborn, Ilma} (1850 – 1896-07-16), \emph{Schauspielerin}|pw} plötzlich gestorben ist, ferner dass man in Ischl\oindex{Bad Ischl@\textbf{Bad Ischl}|pw} nächstens Ihre »Liebelei\pwindex{Schnitzler, Arthur 15.05.1862 – 21.10.1931@\textsc{Schnitzler, Arthur} (15.05.1862 – 21.10.1931), \emph{Schriftsteller, Mediziner}!Liebelei. Schauspiel in drei Akten1895-10-09@\strich\emph{Liebelei. Schauspiel in drei Akten} {[}1895-10-09{]}|pw}« aufführen wird, doch dürfte sie weder der eine noch der andere
               Unglücksfall zu sehr erschüttern. Diesen Sonntag bin ich in Ischl\oindex{Bad Ischl@\textbf{Bad Ischl}|pw} gewesen, vielmehr in Aussee\oindex{Bad Aussee@\textbf{Bad Aussee}|pw}, denn ich fuhr gleich in der Früh mit Frl. M.\pwindex{Salten, Ottilie 07.03.1868 – 22.06.1942@\textsc{Salten, Ottilie} (07.03.1868 – 22.06.1942), \emph{Schauspielerin}|pw}{ }{\pb}dahin. Es schüttete in
               Strömen und wir blieben den ganzen Tag bei Frau Mitterwurzer\pwindex{Mitterwurzer, Wilhelmine 27.03.1848 – 03.08.1909@\textsc{Mitterwurzer, Wilhelmine} (27.03.1848 – 03.08.1909), \emph{Schauspielerin}|pw}. Ich gehe nun doch nicht ins Ampezzothal\oindex{Valle DAmpezzo@\textbf{Valle d’Ampezzo}|pw}. Meine Adresse vom 1.–7. Aug. ist
               jetzt Ischl\oindex{Bad Ischl@\textbf{Bad Ischl}|pw}, von da an München\oindex{Muenchen@\textbf{München}|pw} bis zum 12. und von da ab Salzburg\oindex{Salzburg@\textbf{Salzburg}|pw} bis zum 20. Aug. Wir
               fahren wie Sie daraus sehen von Salzburg\oindex{Salzburg@\textbf{Salzburg}|pw} per Rad
               nach München\oindex{Muenchen@\textbf{München}|pw}, von da über Schliersee\oindex{Schliersee@\textbf{Schliersee}|pw}, Tegernsee\oindex{Tegernsee@\textbf{Tegernsee}|pw}
               nach Innsbruck\oindex{Innsbruck@\textbf{Innsbruck}|pw} und von dort nach Salzburg\oindex{Salzburg@\textbf{Salzburg}|pw}. Das ist Alles. Indessen bin ich ununterbrochen zu
               Hause, lese und arbeite. Zeitlin\pwindex{Zeitlin, Alexander 15.07.1872 – 04.03.1946@\textsc{Zeitlin, Alexander} (15.07.1872 – 04.03.1946), \emph{Bildender Künstler/Bildende Künstlerin >> Bildhauer/Bildhauerin}|pw} hat keinen
               Preis bekommen, Popper\pwindex{Pongrácz, Szigfrid 1872-06-14 – 1929-02-06@\textsc{Pongrácz, Szigfrid} (1872-06-14 – 1929-02-06), \emph{Bildhauer}|pw}, der mit einer
               geradezu herrlichen Gruppe »Adam und Eva\pwindex{Pongrácz, Szigfrid 1872-06-14 – 1929-02-06@\textsc{Pongrácz, Szigfrid} (1872-06-14 – 1929-02-06), \emph{Bildhauer}!Adam und Eva1896@\strich\emph{Adam und Eva} {[}1896{]}|pw}« um
               den {\pb}Rompreis concurrirte,
               wurde mit dem Specialschulpreis abgefunden. Ich schrieb einen \label{K_L03175-1v}\edtext{Leitartikel\pwindex{Schuelerausstellung der Akademie1896-07-21@\emph{Die Schülerausstellung der Akademie} {[}1896-07-21{]}|pwv}}{\lemma{\textnormal{\emph{Leitartikel}}}\Cendnote{\textnormal{f. s.\pwindex{Salten, Felix 06.09.1869 – 08.10.1945@\textsc{Salten, Felix} (06.09.1869 – 08.10.1945), \emph{Schriftsteller, Journalist}|pwk} [ = Felix Salten\pwindex{Salten, Felix 06.09.1869 – 08.10.1945@\textsc{Salten, Felix} (06.09.1869 – 08.10.1945), \emph{Schriftsteller, Journalist}|pwk}]: \emph{Die Schülerausstellung der Akademie}\pwindex{Schuelerausstellung der Akademie1896-07-21@\emph{Die Schülerausstellung der Akademie} {[}1896-07-21{]}|pwk}. In: \emph{Wiener Allgemeine Zeitung}\pwindex{?? Werk@Nicht ermittelte Verfasserinnen und Verfasser!Wiener Allgemeine Zeitung1.3.1880 – 11.2.1934@\emph{Wiener Allgemeine Zeitung} {[}1.3.1880 – 11.2.1934{]}|pwk}, Nr. 5.517,
                        21. 7. 1896, S. 4.}}}\label{K_L03175-1h} über die
               Zustände an der Akademie\orgindex{Akademie der Bildenden Kuenste Wien@Akademie der Bildenden Künste Wien|pw}, musste aber zahm sein,
               da man in kein Wespennest stechen will. Doch denke ich mich in der Frankft. Ztg.\pwindex{?? Werk@Nicht ermittelte Verfasserinnen und Verfasser!Frankfurter Zeitung1856 – 1943@\emph{Frankfurter Zeitung} {[}1856 – 1943{]}|pw} weitläufiger über die Sache auszulassen\textcolor{red}{\textsuperscript{\textbf{KEY}}}. Dass Edmond de Goncourt\pwindex{Goncourt, Edmond Huot de 26.05.1822 – 16.07.1896@\textsc{Goncourt, Edmond Huot de} (26.05.1822 – 16.07.1896), \emph{Schriftsteller}|pw} tot ist, werden Sie vielleicht schon erfahren haben. Er
               starb in dem Schloße\oindex{Haus von Alphonse Daudet@\textbf{Haus von Alphonse Daudet}|pwv} von Daudet\pwindex{Daudet, Alphonse 13.05.1840 – 16.11.1897@\textsc{Daudet, Alphonse} (13.05.1840 – 16.11.1897), \emph{Schriftsteller}|pw}. Die  Wien\oindex{Wien@\textbf{Wien}|pw}er Schornalisten, welche die letzte \label{K_L03175-221v}\edtext{Flegelei\pwindex{† Edmond de Goncourt1896-07-17@\emph{† Edmond de Goncourt} {[}1896-07-17{]}|pwv}}{\lemma{\textnormal{\emph{Flegelei}}}\Cendnote{\textnormal{[O. V.  = Max Nordau\pwindex{Nordau, Max 29.07.1849 – 22.01.1923@\textsc{Nordau, Max} (29.07.1849 – 22.01.1923), \emph{Schriftsteller, Mediziner}|pwk}]:
                        \emph{† Edmond de Goncourt}\pwindex{† Edmond de Goncourt1896-07-17@\emph{† Edmond de Goncourt} {[}1896-07-17{]}|pwk}. In: \emph{Neue Freie Presse}\pwindex{Neue Freie Presse1864 – 1939@\emph{Neue Freie Presse} {[}1864 – 1939{]}|pwk}, Nr. 11.457,
                        17. 7. 1896, Morgenblatt,
                  S. 5.}}}\label{K_L03175-221h}{ }Nordau\pwindex{Nordau, Max 29.07.1849 – 22.01.1923@\textsc{Nordau, Max} (29.07.1849 – 22.01.1923), \emph{Schriftsteller, Mediziner}|pw}’s als Quelle über Goncourt\pwindex{Goncourt, Edmond Huot de 26.05.1822 – 16.07.1896@\textsc{Goncourt, Edmond Huot de} (26.05.1822 – 16.07.1896), \emph{Schriftsteller}|pw} benützten, schrieben in guten \label{K_L03175-44v}\edtext{Notizelach}{\lemma{\textnormal{\emph{Notizelach}}}\Cendnote{\textnormal{Durch Anhang einer jiddischen Endsilbe alludiert Salten\pwindex{Salten, Felix 06.09.1869 – 08.10.1945@\textsc{Salten, Felix} (06.09.1869 – 08.10.1945), \emph{Schriftsteller, Journalist}|pwk} daran, dass Nordau\pwindex{Nordau, Max 29.07.1849 – 22.01.1923@\textsc{Nordau, Max} (29.07.1849 – 22.01.1923), \emph{Schriftsteller, Mediziner}|pwk} Jude war
                  und überhaupt die Wien\oindex{Wien@\textbf{Wien}|pwk}er Presselandschaft in
                  Verruf stand, nur von Juden bevölkert zu werden. Da Salten\pwindex{Salten, Felix 06.09.1869 – 08.10.1945@\textsc{Salten, Felix} (06.09.1869 – 08.10.1945), \emph{Schriftsteller, Journalist}|pwk} selber jüdischer Abstammung war, dürfte damit
                  weniger ein antisemitischer Reflex gemeint sein, als eine als jüdisch
                  wahrgenommene Berichterstattung das Ziel seiner Kritik darstellen.}}}\label{K_L03175-44h}, er sei
               der populärste und \uline{platteste}{ }{\pb}Schriftsteller Frankreichs\oindex{Frankreich@\textbf{Frankreich}|pw} gewesen. Herr Ohnet\oindex{XXXX Ortsangabe fehlt|pw} würde sich freuen. Nach seinem Testament wird eine »freie Akademie\orgindex{Academie Goncourt@Académie Goncourt|pwv}« gegründet, deren Präsident
                  Daudet\pwindex{Daudet, Alphonse 13.05.1840 – 16.11.1897@\textsc{Daudet, Alphonse} (13.05.1840 – 16.11.1897), \emph{Schriftsteller}|pw} ist, und deren einzelne Mitglieder
               eine Rente von 6000 Frcs aus dem Vermögen Goncourts\pwindex{Goncourt, Jules Huot de 17.12.1830 – 20.06.1870@\textsc{Goncourt, Jules Huot de} (17.12.1830 – 20.06.1870), \emph{Schriftsteller}|pw}\pwindex{Goncourt, Edmond Huot de 26.05.1822 – 16.07.1896@\textsc{Goncourt, Edmond Huot de} (26.05.1822 – 16.07.1896), \emph{Schriftsteller}|pw} erhalten. \pend
           \pstart
           Diese Lust der Franzosen nach Vereinigungen und ihr Verlangen, dass die Berühmtheit
               durch Zeremonien bestätigt werde, hat etwas, wenn auch nicht viel von unseren »hohen
               Orden«, der freilich schöner ist. Schon deshalb weil er nicht exisitiert. Schreiben
               Sie bald und grüßen Richard\pwindex{Beer-Hofmann, Richard 1866-07-11 – 1945-09-26@\textsc{Beer-Hofmann, Richard} (1866-07-11 – 1945-09-26), \emph{Schriftsteller}|pw}. Die Zeitungen
               schicke ich Ihnen nun schon nach Kopenhagen\oindex{Kopenhagen@\textbf{Kopenhagen}|pw}. \pend
           \pstart
           Herzlichst Ihr {\\[\baselineskip]}\spacefill\mbox{Salten}\pend
           \leftskip=0em{}
         
         \endnumbering\mylabel{h}\end{ledgroupsized}\begin{anhang}\end{anhang}\newcommand{\dateiname}{L03175}\newcommand{\titel}{Felix Salten an Arthur Schnitzler, 21. 7. 1896}\newcommand{\editorInnen}{Martin Anton Müller und Laura Untner}%% latex-leseansicht-abspann.tex
%% Abspann für die Leseansicht.
%% Der Schalter \ifkorrekturansicht ist bereits durch den Vorspann gesetzt.

%% latex-abspann.tex
%% Gemeinsamer Abspann für Korrekturansicht und Leseansicht.
%% Setzt den Schalter \ifkorrekturansicht voraus (gesetzt in den
%% einbindenden Dateien latex-korrekturansicht-abspann.tex bzw.
%% latex-leseansicht-abspann.tex).
%% ---------------------------------------------------------------

\normalsize

% Das esempio-Environment wird nur in der Leseansicht benötigt
\ifkorrekturansicht\else
\newenvironment{esempio}[3]%
{
    \vspace{1.5ex}
    \rlap{\underline{#1}}
    \par
    \setlength{\parindent}{0cm}
    \nopagebreak
    \leftskip=#2cm
    \rightskip=#3cm
}
{
    \par
}
\fi

\doendnotes{C}
\bigskip
\vfill

\clearpage

\footnotesize

\ifkorrekturansicht
  \lohead{\textsc{register}}
\fi

% theindex-Environment neu definieren ohne reledmac
\makeatletter
\renewenvironment{theindex}{%
  \ifkorrekturansicht
    \section*{\indexname}%
  \else
    \subsubsection*{Index der erwähnten Entitäten}%
  \fi
  \setlength{\parindent}{0pt}%
  \setlength{\parskip}{0pt plus 0.3pt}%
  \let\item\@idxitem
}{%
  \ifkorrekturansicht\clearpage\fi
}
\makeatother

\IfFileExists{\jobname-pw.ind}{\input{\jobname-pw.ind}}{}

% Quellenangabe nur in der Leseansicht
\ifkorrekturansicht\else
% Fallback-Definitionen, falls die .tex-Datei \titel etc. nicht gesetzt hat
\providecommand{\titel}{}
\providecommand{\editorInnen}{}
\providecommand{\dateiname}{\jobname}

\vspace{3cm}

\vfill

\footnotesize
\textsc{Quelle}: \titel. Herausgegeben von {\editorInnen}. In: \emph{Arthur Schnitzler: Briefwechsel mit Autorinnen und Autoren}.
 Digitale Edition, https://schnitzler-briefe.acdh.oeaw.ac.at/{\dateiname}.html (Stand \today)
\fi

\end{document}


      