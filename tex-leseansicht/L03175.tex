%% latex-korrekturansicht-vorspann.tex
%% Vorspann für die Korrekturansicht.
%% Lädt die gemeinsame Datei latex-vorspann.tex mit gesetztem Schalter.

\newif\ifkorrekturansicht
\korrekturansichttrue

\input{../tex-inputs/latex-vorspann}


\section[ Felix Salten an Arthur Schnitzler, 21. 7. 1896]{L03175 Felix Salten an Arthur Schnitzler, 21. 7. 1896}
\nopagebreak\mylabel{L03175v}
\rehead{ }\normalsize\beginnumbering\briefempfaengerindex{Schnitzler, Arthur@\textsc{Schnitzler, Arthur}!zzzSalten, Felix@\emph{von Felix Salten}!1896-07-211@{21. 7. 1896}|(be}
\toendnotes[C]{\smallbreak\pagebreak[2]}\Standort{CUL, Schnitzler, B 89, A 1.}
\physDesc{Brief, 1 Blatt, 4 Seiten, 2208 Zeichen
\newline{}Handschrift: Bleistift, lateinische Kurrent
\newline{}Schnitzler: mit Bleistift die Jahreszahl »96« ergänzt 
\newline{}Ordnung: mit Bleistift von unbekannter Hand nummeriert: »74« }\toendnotes[C]{\smallbreak}
\pstart
           \raggedleft{}{\pb}Wien\oindex{Wien@\textbf{Wien}, \emph{A.ADM2}|pw}, den 21. Juli\pend
           \vspace{0.5em}
\pstart
           Lieber Arthur, in dieser Welt geht garnichts vor, und es ist ganz
               gleichgiltig, ob man jetzt in Iglau\oindex{Jihlava@\textbf{Jihlava}, \emph{P.PPLA}|pw} lebt oder
               auf dem \label{K_L03175-1v}\edtext{Nordcap\oindex{Nordkap@\textbf{Nordkap}, \emph{Kap (N.KAP)}|pw}}{\lemma{\textnormal{\emph{Nordcap}}}\Cendnote{\textnormal{Womöglich reagiert Salten\pwindex{Salten, Felix 06.09.1869 – 08.10.1945@\textsc{Salten, Felix} (06.09.1869 – 08.10.1945), \emph{Schriftsteller/Schriftstellerin, Journalist/Journalistin, Chefredakteur/Chefredakteurin}|pwk} schon
                  auf eine Karte vom Nordkap\oindex{Nordkap@\textbf{Nordkap}, \emph{Kap (N.KAP)}|pwk}, das 
                  Schnitzler am 19. 7. 1896 besucht hatte.}}}\label{K_L03175-1} ist. Auf dem Nordcap\oindex{Nordkap@\textbf{Nordkap}, \emph{Kap (N.KAP)}|pw} ist’s besser, das ist das Ganze. Von
               grossen Ereignissen hab ich Ihnen nur zu melden, dass \label{K_L03175-2v}\edtext{Frau Seiler-Willborn\pwindex{Seiler-Willborn, Ilma 1850 – 1896-07-16@\textsc{Seiler-Willborn, Ilma} (1850 – 1896-07-16), \emph{Schauspieler/Schauspielerin}|pw}
               plötzlich gestorben}{\lemma{\textnormal{\emph{Frau … gestorben}}}\Cendnote{\textnormal{Die Schauspielerin
                     Ilma Seiler-Willborn\pwindex{Seiler-Willborn, Ilma 1850 – 1896-07-16@\textsc{Seiler-Willborn, Ilma} (1850 – 1896-07-16), \emph{Schauspieler/Schauspielerin}|pwk} war am 16. 7. 1896 in Wien\oindex{Wien@\textbf{Wien}, \emph{A.ADM2}|pwk}
                  verstorben.}}}\label{K_L03175-2} ist, ferner dass man in Ischl\oindex{Bad Ischl@\textbf{Bad Ischl}, \emph{P.PPL}|pw} nächstens Ihre \label{K_L03175-3v}\edtext{»Liebelei\pwindex{Liebelei. Schauspiel in drei Akten@\emph{Liebelei. Schauspiel in drei Akten}|pw}« aufführen}{\lemma{\textnormal{\emph{»Liebelei« aufführen}}}\Cendnote{\textnormal{durch das \emph{Saisontheather
                     Ischl}\orgindex{Saisontheater Ischl@Saisontheater Ischl|pwk}}}}\label{K_L03175-3} wird. Doch dürfte Sie weder der eine noch der andere Unglücksfall zu sehr
               erschüttern. Diesen Sonntag bin ich in Ischl\oindex{Bad Ischl@\textbf{Bad Ischl}, \emph{P.PPL}|pw} gewesen, vielmehr in Aussee\oindex{Bad Aussee@\textbf{Bad Aussee}, \emph{P.PPLA3}|pw}, denn ich fuhr gleich in der Früh mit Frl. M.\pwindex{Salten, Ottilie 07.03.1868 – 22.06.1942@\textsc{Salten, Ottilie} (07.03.1868 – 22.06.1942), \emph{Schauspieler/Schauspielerin}|pw}{ }{\pb}dahin. Es schüttete in
               Strömen und wir blieben den ganzen Tag bei Frau Mitterwurzer\pwindex{Mitterwurzer, Wilhelmine 27.03.1848 – 03.08.1909@\textsc{Mitterwurzer, Wilhelmine} (27.03.1848 – 03.08.1909), \emph{Schauspieler/Schauspielerin}|pw}. Ich gehe nun doch nicht ins Ampezzothal\oindex{Valle DAmpezzo@\textbf{Valle d’Ampezzo}, \emph{T.VAL}|pw}. Meine Adresse vom 1–7. Aug. ist jetzt Ischl\oindex{Bad Ischl@\textbf{Bad Ischl}, \emph{P.PPL}|pw}. Von da an München\oindex{Muenchen@\textbf{München}, \emph{P.PPLA}|pw} bis zum 12. und von da ab Salzburg\oindex{Salzburg@\textbf{Salzburg}, \emph{A.ADM2}|pw} bis zum 20. August. Wir fahren wie
               Sie daraus sehen von Salzburg\oindex{Salzburg@\textbf{Salzburg}, \emph{A.ADM2}|pw} per Rad nach München\oindex{Muenchen@\textbf{München}, \emph{P.PPLA}|pw}, von da über Schliersee\oindex{Schliersee@\textbf{Schliersee}, \emph{P.PPL}|pw}, Tegernsee\oindex{Tegernsee@\textbf{Tegernsee}, \emph{P.PPL}|pw}
               nach Innsbruck\oindex{Innsbruck@\textbf{Innsbruck}, \emph{A.ADM2}|pw} und von dort nach Salzburg\oindex{Salzburg@\textbf{Salzburg}, \emph{A.ADM2}|pw}. Das ist Alles. Indessen bin ich ununterbrochen zu
               Hause, lese und arbeite. Zeitlin\pwindex{Zeitlin, Alexander 15.07.1872 – 04.03.1946@\textsc{Zeitlin, Alexander} (15.07.1872 – 04.03.1946), \emph{Bildhauer/Bildhauerin}|pw} hat keinen
                  \label{K_L03175-4v}\edtext{Preis}{\lemma{\textnormal{\emph{Preis}}}\Cendnote{\textnormal{der \emph{Akademie der bildenden
                     Künste}\orgindex{Akademie der Bildenden Kuenste Wien@Akademie der Bildenden Künste Wien|pwk} in Wien\oindex{Wien@\textbf{Wien}, \emph{A.ADM2}|pwk}}}}\label{K_L03175-4} beko{\geminationm}en, Popper\pwindex{Pongrácz, Szigfrid 1872-06-14 – 1929-02-06@\textsc{Pongrácz, Szigfrid} (1872-06-14 – 1929-02-06), \emph{Bildhauer/Bildhauerin}|pw}, der mit einer geradezu herrlichen Gruppe »Adam und Eva\pwindex{Adam und Eva@\emph{Adam und Eva}|pw}« um den {\pb}Rompreis concurrirte{[},{]}
               wurde mit dem Specialschulpreis abgefunden. Ich schrieb einen \label{K_L03175-5v}\edtext{Leitartikel\pwindex{Schuelerausstellung der Akademie@\emph{Die Schülerausstellung der Akademie}|pwv}}{\lemma{\textnormal{\emph{Leitartikel}}}\Cendnote{\textnormal{f. s.\pwindex{Salten, Felix 06.09.1869 – 08.10.1945@\textsc{Salten, Felix} (06.09.1869 – 08.10.1945), \emph{Schriftsteller/Schriftstellerin, Journalist/Journalistin, Chefredakteur/Chefredakteurin}|pwk} [ = Felix Salten\pwindex{Salten, Felix 06.09.1869 – 08.10.1945@\textsc{Salten, Felix} (06.09.1869 – 08.10.1945), \emph{Schriftsteller/Schriftstellerin, Journalist/Journalistin, Chefredakteur/Chefredakteurin}|pwk}]: \emph{Die
                        Schülerausstellung der Akademie}\pwindex{Schuelerausstellung der Akademie@\emph{Die Schülerausstellung der Akademie}|pwk}. In: \emph{Wiener Allgemeine Zeitung}\pwindex{Wiener Allgemeine Zeitung@\emph{Wiener Allgemeine Zeitung}|pwk}, Nr. 5517, 21. 7. 1896, S. 4.}}}\label{K_L03175-5} über die Zustände an der Akademie\orgindex{Akademie der Bildenden Kuenste Wien@Akademie der Bildenden Künste Wien|pw}, musste aber zahm sein, da man in kein
               Wespennest stechen will. Doch denke ich mich in der \label{K_L03175-6v}\edtext{Frankft. Ztg\pwindex{Frankfurter Zeitung@\emph{Frankfurter Zeitung}|pw} weitläufiger über die Sache
                  auszulaßen}{\lemma{\textnormal{\emph{Frankft. … auszulaßen}}}\Cendnote{\textnormal{nicht 
                  nachgewiesen}}}\label{K_L03175-6}. Dass \label{K_L03175-7v}\edtext{Edmond de Goncourt\pwindex{Goncourt, Edmond Huot de 26.05.1822 – 16.07.1896@\textsc{Goncourt, Edmond Huot de} (26.05.1822 – 16.07.1896), \emph{Schriftsteller/Schriftstellerin}|pw} tot}{\lemma{\textnormal{\emph{Edmond de Goncourt tot}}}\Cendnote{\textnormal{Edmond
                     de Goncourt\pwindex{Goncourt, Edmond Huot de 26.05.1822 – 16.07.1896@\textsc{Goncourt, Edmond Huot de} (26.05.1822 – 16.07.1896), \emph{Schriftsteller/Schriftstellerin}|pwk} starb am 16. 7. 1896 in Draveil\oindex{Draveil@\textbf{Draveil}, \emph{P.PPL}|pwk}.}}}\label{K_L03175-7} ist, werden Sie
               vielleicht schon erfahren haben. Er starb in dem Schloße\oindex{Haus von Alphonse Daudet@\textbf{Haus von Alphonse Daudet}, \emph{Wohngebäude (K.WHS)}|pwv} von Daudet\pwindex{Daudet, Alphonse 13.05.1840 – 16.11.1897@\textsc{Daudet, Alphonse} (13.05.1840 – 16.11.1897), \emph{Schriftsteller/Schriftstellerin}|pw}.
               Die  Wien\oindex{Wien@\textbf{Wien}, \emph{A.ADM2}|pw}er Schornalisten, welche die letzte
                  \label{K_L03175-8v}\edtext{Flegelei\pwindex{† Edmond de Goncourt@\emph{† Edmond de Goncourt}|pwv}}{\lemma{\textnormal{\emph{Flegelei}}}\Cendnote{\textnormal{[Max Nordau\pwindex{Nordau, Max 29.07.1849 – 22.01.1923@\textsc{Nordau, Max} (29.07.1849 – 22.01.1923), \emph{Schriftsteller/Schriftstellerin, Mediziner/Medizinerin}|pwk}]: \emph{† Edmond de Goncourt}\pwindex{† Edmond de Goncourt@\emph{† Edmond de Goncourt}|pwk}. In: \emph{Neue Freie Presse}\pwindex{Neue Freie Presse@\emph{Neue Freie Presse}|pwk}, Nr. 11.457, 17. 7. 1896, Morgenblatt, S. 5.}}}\label{K_L03175-8}{ }Nordau\pwindex{Nordau, Max 29.07.1849 – 22.01.1923@\textsc{Nordau, Max} (29.07.1849 – 22.01.1923), \emph{Schriftsteller/Schriftstellerin, Mediziner/Medizinerin}|pw}’s als Quelle über Goncourt\pwindex{Goncourt, Edmond Huot de 26.05.1822 – 16.07.1896@\textsc{Goncourt, Edmond Huot de} (26.05.1822 – 16.07.1896), \emph{Schriftsteller/Schriftstellerin}|pw} benützten, schrieben in guten \label{K_L03175-9v}\edtext{Notizelach}{\lemma{\textnormal{\emph{Notizelach}}}\Cendnote{\textnormal{Durch Anhang einer jiddischen Endsilbe spielt Salten\pwindex{Salten, Felix 06.09.1869 – 08.10.1945@\textsc{Salten, Felix} (06.09.1869 – 08.10.1945), \emph{Schriftsteller/Schriftstellerin, Journalist/Journalistin, Chefredakteur/Chefredakteurin}|pwk} darauf an, dass Nordau\pwindex{Nordau, Max 29.07.1849 – 22.01.1923@\textsc{Nordau, Max} (29.07.1849 – 22.01.1923), \emph{Schriftsteller/Schriftstellerin, Mediziner/Medizinerin}|pwk} Jude war und überhaupt die Wien\oindex{Wien@\textbf{Wien}, \emph{A.ADM2}|pwk}er Presselandschaft in Verruf stand, hauptsächlich aus Juden
                  zusammengesetzt zu sein. Da Salten\pwindex{Salten, Felix 06.09.1869 – 08.10.1945@\textsc{Salten, Felix} (06.09.1869 – 08.10.1945), \emph{Schriftsteller/Schriftstellerin, Journalist/Journalistin, Chefredakteur/Chefredakteurin}|pwk} selbst
                  jüdischer Abstammung war, dürfte er damit weniger einen unmittelbaren antisemitischen Reflex ausdrücken,
                  als ein negatives Bild bestimmter journalistischer Praktiken
                  zeichnen.}}}\label{K_L03175-9}, er sei der populärste und \uline{platteste}{ }{\pb}Schriftsteller Frankreichs\oindex{Frankreich@\textbf{Frankreich}, \emph{A.PCLI}|pw} gewesen. Herr Ohnet\pwindex{Ohnet, Georges 03.04.1848 – 05.05.1918@\textsc{Ohnet, Georges} (03.04.1848 – 05.05.1918), \emph{Schriftsteller/Schriftstellerin}|pw} würde sich freuen. Nach seinem Testament wird eine »freie Akademie\orgindex{Academie Goncourt@Académie Goncourt|pwv}« gegründet, deren Präsident
                  Daudet\pwindex{Daudet, Alphonse 13.05.1840 – 16.11.1897@\textsc{Daudet, Alphonse} (13.05.1840 – 16.11.1897), \emph{Schriftsteller/Schriftstellerin}|pw} ist, und deren einzelne Mitglieder
               eine Rente von 6000 Frcs aus dem Vermögen Goncourts\pwindex{Goncourt, Jules Huot de 17.12.1830 – 20.06.1870@\textsc{Goncourt, Jules Huot de} (17.12.1830 – 20.06.1870), \emph{Schriftsteller/Schriftstellerin}|pw}\pwindex{Goncourt, Edmond Huot de 26.05.1822 – 16.07.1896@\textsc{Goncourt, Edmond Huot de} (26.05.1822 – 16.07.1896), \emph{Schriftsteller/Schriftstellerin}|pw} erhalten. Diese Lust der Fran\oindex{Frankreich@\textbf{Frankreich}, \emph{A.PCLI}|pwv}zosen nach Vereinigungen und ihr Verlangen, dass die
               Berühmtheit durch Zeremonien bestätigt werde, hat etwas, wenn auch nicht viel von
               unserem »hohen Orden«, der freilich schöner ist. Schon deshalb weil er nicht
               exisitirt. Schreiben Sie bald und grüßen Richard\pwindex{Beer-Hofmann, Richard 1866-07-11 – 1945-09-26@\textsc{Beer-Hofmann, Richard} (1866-07-11 – 1945-09-26), \emph{Schriftsteller/Schriftstellerin}|pw}. Die Zeitungen schicke ich Ihnen nun schon nach \label{K_L03175-10v}\edtext{Kopenhagen\oindex{Kopenhagen@\textbf{Kopenhagen}, \emph{P.PPLC}|pw}}{\lemma{\textnormal{\emph{Kopenhagen}}}\Cendnote{\textnormal{Schnitzler hielt sich vom 2. 8. 1896 bis zum 3. 8. 1896 sowie am
                     22. 8. 1896 in
                     Kopenhagen\oindex{Kopenhagen@\textbf{Kopenhagen}, \emph{P.PPLC}|pwk} auf. Dazwischen war er in Skodsborg\oindex{Skodsborg@\textbf{Skodsborg}, \emph{P.PPL}|pwk}.}}}\label{K_L03175-10}.\pend
           
\pstart
           Herzlichst Ihr {\\[\baselineskip]}\spacefill\mbox{Salten}\pend
           \leftskip=0em{}\selectlanguage{ngerman}\endnumbering\briefempfaengerindex{Schnitzler, Arthur@\textsc{Schnitzler, Arthur}!zzzSalten, Felix@\emph{von Felix Salten}!1896-07-211@{21. 7. 1896}|)be}\mylabel{L03175h}  \normalsize

\doendnotes{C}
\bigskip
\vfill

\clearpage

\footnotesize

\lohead{\textsc{register}}

% Definiere theindex-Environment komplett neu ohne reledmac
\makeatletter
\renewenvironment{theindex}{%
  \section*{\indexname}%
  \setlength{\parindent}{0pt}%
  \setlength{\parskip}{0pt plus 0.3pt}%
  \let\item\@idxitem
}{%
  \clearpage
}
\makeatother

\IfFileExists{\jobname-pw.ind}{\input{\jobname-pw.ind}}{}

\end{document}

      