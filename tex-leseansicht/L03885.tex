%% latex-leseansicht-vorspann.tex
%% Vorspann für die Leseansicht.
%% Lädt die gemeinsame Datei latex-vorspann.tex mit nicht gesetztem Schalter.

\newif\ifkorrekturansicht
\korrekturansichtfalse

\input{../tex-inputs/latex-vorspann}


\section[Arthur Schnitzler an Romain Rolland, 7. 1. 1915]{L03885 Arthur Schnitzler an Romain Rolland, 7. 1. 1915}
\nopagebreak\mylabel{L03885v}
\rehead{ }\normalsize\beginnumbering\briefempfaengerindex{Rolland, Romain@\textsc{Rolland, Romain}!zzzSchnitzler, Arthur@\emph{von Arthur Schnitzler}!1915-01-071@{7. 1. 1915}|(be}
\toendnotes[C]{\smallbreak\pagebreak[2]}
\correspDesc{Versand  durch Arthur Schnitzler am 7. 1. 1915 in Wien
\newline{}Erhalt  durch Romain Rolland im Zeitraum [8. 1. 1915 –
                  11. 1. 1915?] in Genf}\toendnotes[C]{\smallbreak}
\Standort{Paris, Bibliothèque Nationale de France, Fonds Romain Rolland, Cote NAF 28400.}
\physDesc{Brief, 2 Blätter, 2 Seiten, Kuvert, 1312 Zeichen
\newline{}Schreibmaschine
\newline{}Handschrift: schwarze Tinte (\noindent{}Unterschrift, Unterstreichung und Ergänzung zweier Satzzeichen)
\newline{}Versand: 1) Stempel: »\nobreak{}\oindex{I., Innere Stadt@\textbf{I., Innere Stadt}, \emph{Verwaltungsgebiet}|pwk}1/1 Wien, 7. I. 15, 6\nobreak{}«.   2) Stempel: »\nobreak{}\oindex{I., Innere Stadt@\textbf{I., Innere Stadt}, \emph{Verwaltungsgebiet}|pwk}Wien 1, Überprüft\nobreak{}«. 
\newline{}Rolland: mit schwarzer Tinte Datierung: »7/1/1915« und Vermerk: »\uline{ARL}« 
\newline{}Ordnung: 1) mit Bleistift Kuvert nummeriert: »2«  2) mit Bleistift Blätter (einschliesslich des Kuverts) paginiert: »4« – »5«}\Standort{DLA, A:Schnitzler, 85.1.1714.}
\physDesc{BriefDurchschlag, , 1312 Zeichen
\newline{}Schreibmaschine}
\buchAbdrucke{\weitereDrucke{Arthur Schnitzler: \emph{Briefe 1913–1931}. Herausgegeben von Peter Michael Braunwarth, Richard Miklin, Susanne Pertlik und Heinrich Schnitzler. Frankfurt am Main: \emph{S. Fischer} 1984, S. 69–70.} }\toendnotes[C]{\smallbreak}\pstart{}{\pb}\textcolor{gray}{\textbf{Dr. Arthur Schnitzler}}\pend{}\pstart{}\textcolor{gray}{\textbf{Wien XVIII. Sternwartestrasse 71\oindex{Wien@\textbf{Wien}!XVIII., Währing@\textbf{XVIII., Währing}!Sternwartestraße 71@\textbf{Sternwartestraße 71}, \emph{Wohngebäude}|pw}}}\pend{}{\bigskip}\pstart{}{\pb}Herrn Romain Rolland\pend{}\pstart{}Genève\oindex{Genf@\textbf{Genf}|pw}\pend{}\pstart{}Hotel Beau-Séjour\oindex{Hôtel Beau-Séjour@\textbf{Hôtel Beau-Séjour}, \emph{Hotel}|pw}.\pend{}\pstart{}Schweiz\oindex{Schweiz@\textbf{Schweiz}|pw}.\pend{}{\bigskip}\vspace{1em}
\pstart
           {\pb}\textcolor{gray}{\textbf{Dr. Arthur Schnitzler}}\hfill 7. 1. 1915.\pend
           
\pstart
           \textcolor{gray}{\textbf{Wien XVIII. Sternwartestrasse 71\oindex{Wien@\textbf{Wien}!XVIII., Währing@\textbf{XVIII., Währing}!Sternwartestraße 71@\textbf{Sternwartestraße 71}, \emph{Wohngebäude}|pw}}}\pend
           
\pstart\center{}Verehrter Herr Rolland.\pend\vspace{0.5em}
\pstart
           Das Journal de Genêve\pwindex{Schnitzler, Arthur 15. 5. 1862 Wien – 21. 10. 1931 ebd.@\textsc{Schnitzler, Arthur} (15. 5. 1862 Wien – 21. 10. 1931 ebd.), \emph{Schriftsteller, Mediziner}!Une protestation d’Arthur Schnitzler@\strich\emph{Une protestation d’Arthur Schnitzler}|pwv}\pwindex{Journal de Genève@\emph{Journal de Genève}|pw} ist \label{K_L03885-1v}\edtext{nicht an mich gelangt}{\lemma{\textnormal{\emph{nicht an mich gelangt}}}\Cendnote{\textnormal{Er erhielt seinen Protest\pwindex{Schnitzler, Arthur 15. 5. 1862 Wien – 21. 10. 1931 ebd.@\textsc{Schnitzler, Arthur} (15. 5. 1862 Wien – 21. 10. 1931 ebd.), \emph{Schriftsteller, Mediziner}!Une protestation d’Arthur Schnitzler@\strich\emph{Une protestation d’Arthur Schnitzler}|pwkv} erst am 17. 1. 1915.}}}\label{K_L03885-1},
               während die Züricher Zeitung\pwindex{Schnitzler, Arthur 15. 5. 1862 Wien – 21. 10. 1931 ebd.@\textsc{Schnitzler, Arthur} (15. 5. 1862 Wien – 21. 10. 1931 ebd.), \emph{Schriftsteller, Mediziner}!Brief Artur Schnitzlers@\strich\emph{Ein Brief Artur Schnitzlers}|pwv}\pwindex{Neue Zürcher Zeitung@\emph{Neue Zürcher Zeitung}|pw} gestern von der Redaktion\orgindex{Neue Zürcher Zeitung@Neue Zürcher Zeitung|pwv}
               aus mit erheblicher Verspätung bei mir angekommen ist. Die Zensur entschliesst sich
               wahrscheinlich besonders schwer Zeitungen in französischer Sprache durchzulassen\introOben{},\introOben{} und so werde ich vorläufig darauf verzichten müssen, Ihre
                  Uebersetzung\pwindex{Schnitzler, Arthur 15. 5. 1862 Wien – 21. 10. 1931 ebd.@\textsc{Schnitzler, Arthur} (15. 5. 1862 Wien – 21. 10. 1931 ebd.), \emph{Schriftsteller, Mediziner}!Une protestation d’Arthur Schnitzler@\strich\emph{Une protestation d’Arthur Schnitzler}|pwv} meiner Erklärung\pwindex{Schnitzler, Arthur 15. 5. 1862 Wien – 21. 10. 1931 ebd.@\textsc{Schnitzler, Arthur} (15. 5. 1862 Wien – 21. 10. 1931 ebd.), \emph{Schriftsteller, Mediziner}!Brief Artur Schnitzlers@\strich\emph{Ein Brief Artur Schnitzlers}|pw} zu lesen, wenn Sie vielleicht nicht
               doch noch einen Versuch machen wollen, mindestens den betreffenden \uline{Ausschnitt} unter Couvert mir zuzuschicken. Die Zensur
               wird es hoffentlich als politisch gefahrlos erkennen, mir einen von mir selbst
               verfassten und von Romain Rolland übersetzten Protest\pwindex{Schnitzler, Arthur 15. 5. 1862 Wien – 21. 10. 1931 ebd.@\textsc{Schnitzler, Arthur} (15. 5. 1862 Wien – 21. 10. 1931 ebd.), \emph{Schriftsteller, Mediziner}!Une protestation d’Arthur Schnitzler@\strich\emph{Une protestation d’Arthur Schnitzler}|pwv} zur Lektüre frei zu geben.\pend
           
\pstart
           Lassen Sie mich Ihnen heute nochmals für Ihre freundliche Bemühung, sowie für Ihren
               {\pb}letzten, so liebenswürdigen \label{K_L03885-2v}\edtext{Brief}{\lemma{\textnormal{\emph{Brief}}}\Cendnote{\textnormal{XXXX Auszeichnungsfehler: Dokument L03882 nicht gefunden?}}}\label{K_L03885-2} herzlich danken. Immer
               wieder lesen wir in der letzten Zeit in Feldpostbriefen, dass die feindlichen
               Soldaten, die einander in den Schützengräben gegenüberliegen, in den Kampfpausen
               einander Höflichkeiten, Rücksichten, Gefälligkeiten, ja
               achtungsvoll-freundschaftliche Gesinnung erweisen\introOben{};\introOben{} wie
               denken Sie, mein verehrter Herr Rolland, über die Einführung von Schützengräben für
               Journalisten und Diplomaten?\pend
           
\pstart
           Seien Sie herzlichst gegrüsst{\\[\baselineskip]}Ihr sehr
                  ergebener{\\[\baselineskip]}\spacefill\mbox{{[}hs.:{]} Arthur Schnitzler}\pend
           \leftskip=0em{}\selectlanguage{ngerman}\endnumbering\briefempfaengerindex{Rolland, Romain@\textsc{Rolland, Romain}!zzzSchnitzler, Arthur@\emph{von Arthur Schnitzler}!1915-01-071@{7. 1. 1915}|)be}\mylabel{L03885h}
\begin{anhang}
\end{anhang}\newcommand{\dateiname}{L03885}\newcommand{\titel}{Arthur Schnitzler an Romain Rolland, 7. 1. 1915}\newcommand{\editorInnen}{Selma Jahnke und Martin Anton Müller}%% latex-leseansicht-abspann.tex
%% Abspann für die Leseansicht.
%% Der Schalter \ifkorrekturansicht ist bereits durch den Vorspann gesetzt.

%% latex-abspann.tex
%% Gemeinsamer Abspann für Korrekturansicht und Leseansicht.
%% Setzt den Schalter \ifkorrekturansicht voraus (gesetzt in den
%% einbindenden Dateien latex-korrekturansicht-abspann.tex bzw.
%% latex-leseansicht-abspann.tex).
%% ---------------------------------------------------------------

\normalsize

% Das esempio-Environment wird nur in der Leseansicht benötigt
\ifkorrekturansicht\else
\newenvironment{esempio}[3]%
{
    \vspace{1.5ex}
    \rlap{\underline{#1}}
    \par
    \setlength{\parindent}{0cm}
    \nopagebreak
    \leftskip=#2cm
    \rightskip=#3cm
}
{
    \par
}
\fi

\doendnotes{C}
\bigskip
\vfill

\clearpage

\footnotesize

\ifkorrekturansicht
  \lohead{\textsc{register}}
\fi

% theindex-Environment neu definieren ohne reledmac
\makeatletter
\renewenvironment{theindex}{%
  \ifkorrekturansicht
    \section*{\indexname}%
  \else
    \subsubsection*{Index der erwähnten Entitäten}%
  \fi
  \setlength{\parindent}{0pt}%
  \setlength{\parskip}{0pt plus 0.3pt}%
  \let\item\@idxitem
}{%
  \ifkorrekturansicht\clearpage\fi
}
\makeatother

\IfFileExists{\jobname-pw.ind}{\input{\jobname-pw.ind}}{}

% Quellenangabe nur in der Leseansicht
\ifkorrekturansicht\else
% Fallback-Definitionen, falls die .tex-Datei \titel etc. nicht gesetzt hat
\providecommand{\titel}{}
\providecommand{\editorInnen}{}
\providecommand{\dateiname}{\jobname}

\vspace{3cm}

\vfill

\footnotesize
\textsc{Quelle}: \titel. Herausgegeben von {\editorInnen}. In: \emph{Arthur Schnitzler: Briefwechsel mit Autorinnen und Autoren}.
 Digitale Edition, https://schnitzler-briefe.acdh.oeaw.ac.at/{\dateiname}.html (Stand \today)
\fi

\end{document}


