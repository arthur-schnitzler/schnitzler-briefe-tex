%% latex-korrekturansicht-vorspann.tex
%% Vorspann für die Korrekturansicht.
%% Lädt die gemeinsame Datei latex-vorspann.tex mit gesetztem Schalter.

\newif\ifkorrekturansicht
\korrekturansichttrue

\input{../tex-inputs/latex-vorspann}


\section[Arthur Schnitzler an Richard Beer-Hofmann, 22. 9. 1891]{L00041 Arthur Schnitzler an Richard Beer-Hofmann, 22. 9. 1891}
\nopagebreak\mylabel{L00041v}
\rehead{ }\normalsize\beginnumbering\briefempfaengerindex{Beer-Hofmann, Richard@\textsc{Beer-Hofmann, Richard}!zzzSchnitzler, Arthur@\emph{von Arthur Schnitzler}!1891-09-221@{22. 9. 1891}|(be}
\toendnotes[C]{\smallbreak\pagebreak[2]}\Standort{YCGL, MSS 31.}
\physDesc{Kartenbrief, 363 Zeichen
\newline{}Handschrift: 1) Bleistift, deutsche Kurrent\hspace{1em}2) Bleistift, lateinische Kurrent (\noindent{}Adresse)\hspace{1em}
\newline{}Versand: 1) Stempel: »\nobreak{}\oindex{Halle (Saale)@\textbf{Halle (Saale)}, \emph{P.PPL}|pwk}Halle Saale 2, 22. 9. 91, 9–10 N\nobreak{}«.   2) Stempel: »\nobreak{}\oindex{III., Landstrasse@\textbf{III., Landstraße}, \emph{A.ADM3}|pwk}Wien 3/2, 24 9 91, 8 10. V, Bestellt\nobreak{}«. }
\buchAbdrucke{\weitereDrucke{1) Arthur Schnitzler: \emph{Briefe 1875–1912}. Frankfurt am Main: \emph{S. Fischer} 1981, S. 121.} \weitereDrucke{2) Arthur Schnitzler, Richard Beer-Hofmann: \emph{Briefwechsel 1891–1931}. Wien, Zürich: \emph{Europaverlag} 1992, S. 32.} }\toendnotes[C]{\smallbreak}\pstart{}{\pb}Herrn Dr. Rich. Beer-Hofmann\pend{}\pstart{}Wien\oindex{Wien@\textbf{Wien}, \emph{A.ADM2}|pw}\pend{}\pstart{}III Seidlgasse 30\oindex{Seidlgasse@\textbf{Seidlgasse}, \emph{Straße (K.STR)}|pw}\pend{}{\bigskip}\vspace{1em}
\pstart
           \noindent{}{\pb}Lieber Richard, das muſs man erleben, dieſes Halle\oindex{Halle (Saale)@\textbf{Halle (Saale)}, \emph{P.PPL}|pw}! Tramways, die an die Ehrlichkeit der Menſchen glauben –
               im Waggon ſind Käſtchen, wo {\pb}man ſein Fahrgeld
               hineinwirft. – Und dieſe Menſchen ſelbſt – I{\geminationm}erfort ſ\substVorne{}\textsuperscript{\textcolor{gray}{in}}\substDazwischen{}chr\substHinten{}eien ſie und ſind ſtolz auf \label{K_L00041-1v}\edtext{das geeinte deutſche Reich\oindex{Deutschland@\textbf{Deutschland}, \emph{A.PCLI}|pw}}{\lemma{\textnormal{\emph{das … Reich}}}\Cendnote{\textnormal{Am 2. 9. 1891 hatte sich
                  zum 20. Mal der Tag von Sedan\oindex{Sedan@\textbf{Sedan}, \emph{P.PPLA3}|pwk} (Ende des Deutsch\oindex{Deutschland@\textbf{Deutschland}, \emph{A.PCLI}|pwk}-Französischen\oindex{Frankreich@\textbf{Frankreich}, \emph{A.PCLI}|pwk} Krieges von 1870/1871) gejährt,
                  der im Deutschen Reich\oindex{Deutschland@\textbf{Deutschland}, \emph{A.PCLI}|pwk} als Tag der Einheit
                  galt. Vgl. Arthur Schnitzler an Hugo von Hofmannsthal, 1. 9. 1895.
               }}}\label{K_L00041-1}. Lauter Nationalparvenus. – Ich ko{\geminationm}e bald.
                  Ihr\spacefill\mbox{Arthur}\pend
           \selectlanguage{ngerman}\endnumbering\briefempfaengerindex{Beer-Hofmann, Richard@\textsc{Beer-Hofmann, Richard}!zzzSchnitzler, Arthur@\emph{von Arthur Schnitzler}!1891-09-221@{22. 9. 1891}|)be}\mylabel{L00041h}  \normalsize

\doendnotes{C}
\bigskip
\vfill

\clearpage

\footnotesize

\lohead{\textsc{register}}

% Definiere theindex-Environment komplett neu ohne reledmac
\makeatletter
\renewenvironment{theindex}{%
  \section*{\indexname}%
  \setlength{\parindent}{0pt}%
  \setlength{\parskip}{0pt plus 0.3pt}%
  \let\item\@idxitem
}{%
  \clearpage
}
\makeatother

\IfFileExists{\jobname-pw.ind}{\input{\jobname-pw.ind}}{}

\end{document}

      