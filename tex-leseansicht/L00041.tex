%% latex-leseansicht-vorspann.tex
%% Vorspann für die Leseansicht.
%% Lädt die gemeinsame Datei latex-vorspann.tex mit nicht gesetztem Schalter.

\newif\ifkorrekturansicht
\korrekturansichtfalse

\input{../tex-inputs/latex-vorspann}


         
         \renewcommand{\erwaehntePersonen}{Personen: Richard Beer-Hofmann}
         \renewcommand{\erwaehnteOrte}{Orte: Deutschland, Frankreich, Halle (Saale), III., Landstraße, Sedan, Seidlgasse, Wien}
         \renewcommand{\erwaehnteWerke}{}
               \section[Arthur Schnitzler an Richard Beer-Hofmann, 22. 9. 1891]{ Arthur Schnitzler an Richard Beer-Hofmann, 22. 9. 1891}\nopagebreak\mylabel{v}\rehead{ }\begin{ledgroupsized}[t]{13cm}\normalsize\beginnumbering\briefempfaengerindex{Beer-Hofmann, Richard@\textsc{Beer-Hofmann, Richard}!zzzSchnitzler, Arthur@\emph{von Arthur Schnitzler}!1891-09-221@{22. 9. 1891}|(be} \toendnotes[C]{\smallbreak\pagebreak[2]} \Standort{YCGL, MSS 31.}
\physDesc{Kartenbrief, 363 Zeichen
\newline{}Handschrift: 1) Bleistift, deutsche Kurrent\hspace{1em}2) Bleistift, lateinische Kurrent (\noindent{}Adresse)\hspace{1em}
\newline{}Versand: 1) Stempel: »\nobreak{}\oindex{Halle (Saale)@\textbf{Halle (Saale)}|pwk}Halle Saale 2, 22. 9. 91, 9–10 N\nobreak{}«.   2) Stempel: »\nobreak{}\oindex{III., Landstrasse@\textbf{III., Landstraße}|pwk}Wien 3/2, 24 9 91, 8 10. V, Bestellt\nobreak{}«. }\buchAbdrucke{\weitereDrucke{1) Arthur Schnitzler: \emph{Briefe 1875–1912}. Hg. Therese Nickl und Heinrich Schnitzler. Frankfurt am Main: \emph{S. Fischer} 1981, S. 121.} \weitereDrucke{2) Arthur Schnitzler, Richard Beer-Hofmann: \emph{Briefwechsel 1891–1931}. Hg. Konstanze Fliedl. Wien, Zürich: \emph{Europaverlag} 1992, S. 32.} }\toendnotes[C]{\smallbreak}\pstart{}{\pb}Herrn Dr. Rich. Beer-Hofmann\pend{}\pstart{}Wien\oindex{Wien@\textbf{Wien}|pw}\pend{}\pstart{}III Seidlgasse 30\oindex{Seidlgasse@\textbf{Seidlgasse}|pw}\pend{}{\bigskip}\pstart
           \noindent{}{\pb}Lieber Richard, das muſs man erleben, dieſes Halle\oindex{Halle (Saale)@\textbf{Halle (Saale)}|pw}! Tramways, die an die Ehrlichkeit der Menſchen glauben –
               im Waggon ſind Käſtchen, wo {\pb}man ſein Fahrgeld
               hineinwirft. – Und dieſe Menſchen ſelbſt – I{\geminationm}erfort ſ\substVorne{}\textsuperscript{\textcolor{gray}{in}}\substDazwischen{}chr\substHinten{}eien ſie und ſind ſtolz auf \label{K_L00041-1v}\edtext{das geeinte deutſche Reich\oindex{Deutschland@\textbf{Deutschland}|pw}}{\lemma{\textnormal{\emph{das … Reich}}}\Cendnote{\textnormal{Am 2. 9. 1891 hatte sich
                  zum 20. Mal der Tag von Sedan\oindex{Sedan@\textbf{Sedan}|pwk} (Ende des Deutsch\oindex{Deutschland@\textbf{Deutschland}|pwk}-Französischen\oindex{Frankreich@\textbf{Frankreich}|pwk} Krieges von 1870/1871) gejährt,
                  der im Deutschen Reich\oindex{Deutschland@\textbf{Deutschland}|pwk} als Tag der Einheit
                  galt. Vgl. Arthur Schnitzler an Hugo von Hofmannsthal, 1. 9. 1895.
               }}}\label{K_L00041-1h}. Lauter Nationalparvenus. – Ich ko{\geminationm}e bald.
                  Ihr\spacefill\mbox{Arthur}\pend
           
         
         \endnumbering\mylabel{h}\end{ledgroupsized}  \newcommand{\dateiname}{L00041}\newcommand{\titel}{Arthur Schnitzler an Richard Beer-Hofmann, 22. 9. 1891}\newcommand{\editorInnen}{Martin Anton Müller und Gerd-Hermann Susen}%% latex-leseansicht-abspann.tex
%% Abspann für die Leseansicht.
%% Der Schalter \ifkorrekturansicht ist bereits durch den Vorspann gesetzt.

%% latex-abspann.tex
%% Gemeinsamer Abspann für Korrekturansicht und Leseansicht.
%% Setzt den Schalter \ifkorrekturansicht voraus (gesetzt in den
%% einbindenden Dateien latex-korrekturansicht-abspann.tex bzw.
%% latex-leseansicht-abspann.tex).
%% ---------------------------------------------------------------

\normalsize

% Das esempio-Environment wird nur in der Leseansicht benötigt
\ifkorrekturansicht\else
\newenvironment{esempio}[3]%
{
    \vspace{1.5ex}
    \rlap{\underline{#1}}
    \par
    \setlength{\parindent}{0cm}
    \nopagebreak
    \leftskip=#2cm
    \rightskip=#3cm
}
{
    \par
}
\fi

\doendnotes{C}
\bigskip
\vfill

\clearpage

\footnotesize

\ifkorrekturansicht
  \lohead{\textsc{register}}
\fi

% theindex-Environment neu definieren ohne reledmac
\makeatletter
\renewenvironment{theindex}{%
  \ifkorrekturansicht
    \section*{\indexname}%
  \else
    \subsubsection*{Index der erwähnten Entitäten}%
  \fi
  \setlength{\parindent}{0pt}%
  \setlength{\parskip}{0pt plus 0.3pt}%
  \let\item\@idxitem
}{%
  \ifkorrekturansicht\clearpage\fi
}
\makeatother

\IfFileExists{\jobname-pw.ind}{\input{\jobname-pw.ind}}{}

% Quellenangabe nur in der Leseansicht
\ifkorrekturansicht\else
% Fallback-Definitionen, falls die .tex-Datei \titel etc. nicht gesetzt hat
\providecommand{\titel}{}
\providecommand{\editorInnen}{}
\providecommand{\dateiname}{\jobname}

\vspace{3cm}

\vfill

\footnotesize
\textsc{Quelle}: \titel. Herausgegeben von {\editorInnen}. In: \emph{Arthur Schnitzler: Briefwechsel mit Autorinnen und Autoren}.
 Digitale Edition, https://schnitzler-briefe.acdh.oeaw.ac.at/{\dateiname}.html (Stand \today)
\fi

\end{document}


      