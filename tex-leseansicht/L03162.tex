%% latex-korrekturansicht-vorspann.tex
%% Vorspann für die Korrekturansicht.
%% Lädt die gemeinsame Datei latex-vorspann.tex mit gesetztem Schalter.

\newif\ifkorrekturansicht
\korrekturansichttrue

\input{../tex-inputs/latex-vorspann}


\section[ Felix Salten an Arthur Schnitzler, {[}10.? 8. 1895{]}]{L03162 Felix Salten an Arthur Schnitzler, {[}10.? 8. 1895{]}}
\nopagebreak\mylabel{L03162v}
\rehead{ }\normalsize\beginnumbering\briefempfaengerindex{Schnitzler, Arthur@\textsc{Schnitzler, Arthur}!zzzSalten, Felix@\emph{von Felix Salten}!1895-08-101@{{[}10.? 8. 1895{]}}|(be}
\toendnotes[C]{\smallbreak\pagebreak[2]}\Standort{CUL, Schnitzler, B 89, A 1.}
\physDesc{Brief, 1 Blatt, 1 Seite, 236 Zeichen
\newline{}Handschrift: Bleistift, lateinische Kurrent
\newline{}Schnitzler: mit Bleistift datiert: »10? 8/95« 
\newline{}Ordnung: mit Bleistift von unbekannter Hand nummeriert: »62« }\toendnotes[C]{\smallbreak}
\pstart
           {\pb}\textcolor{gray}{\textbf{\textbf{»Wiener Allgemeine
                        Zeitung\orgindex{Wiener Allgemeine Zeitung@Wiener Allgemeine Zeitung|pw}«}}}\pend
           
\pstart
           \textcolor{gray}{\textbf{Redaction:}}\pend
           
\pstart
           \textcolor{gray}{\textbf{\textbf{IX/3, Univerſitätsſtraße Nr. 6\oindex{Universitaetsstrasse@\textbf{Universitätsstraße}, \emph{Straße (K.STR)}|pw}.}}}\pend
           
\pstart
           \textcolor{gray}{\textbf{Adminiſtration:}}\hfill \textcolor{gray}{\textbf{Wien\oindex{Wien@\textbf{Wien}, \emph{A.ADM2}|pw}, am ..........{ }189{\dots}}}\pend
           
\pstart
           \textcolor{gray}{\textbf{\textbf{I. Wollzeile Nr. 5\oindex{Wollzeile@\textbf{Wollzeile}, \emph{Straße (K.STR)}|pw}} (im Durchhauſe).}}\pend
           
\pstart
           \textcolor{gray}{\textbf{Telegramm-Adreſſe: »Allgemeine, Wien\oindex{Wien@\textbf{Wien}, \emph{A.ADM2}|pw}«.}}\pend
           
\pstart
           \textcolor{gray}{\textbf{Telephon der Redaction: Nr. 805 u. 2180.}}\pend
           
\pstart
           \textcolor{gray}{\textbf{\hspace*{1.5em}„\hspace*{1.5em}„\hspace*{1.5em} Adminiſtration: Nr. 1024.}}\pend
           \vspace{0.5em}
\pstart
           Lieber Arthur! Ich denke, es ist nicht nötig \label{K_L03162-1v}\edtext{morgen{ }Nachmittag}{\lemma{\textnormal{\emph{morgen Nachmittag}}}\Cendnote{\textnormal{Die mit Fragezeichen versehene Datierung
                     Schnitzlers stimmt damit überein, dass
                     Schnitzler und Salten\pwindex{Salten, Felix 06.09.1869 – 08.10.1945@\textsc{Salten, Felix} (06.09.1869 – 08.10.1945), \emph{Schriftsteller/Schriftstellerin, Journalist/Journalistin, Chefredakteur/Chefredakteurin}|pwk} sich unmittelbar am Tag nach Schnitzlers Rückkehr aus Ischl\oindex{Bad Ischl@\textbf{Bad Ischl}, \emph{P.PPL}|pwk}, am 11. 8. 1895, trafen und den Abend miteinander
                  verbrachten.}}}\label{K_L03162-1} in das heisse Caféhaus zu gehen. Am besten kommen Sie
               vielleicht \substVorne{}\textsuperscript{zu}\substDazwischen{}gl\substHinten{}eich zu mir. Ich bin den ganzen Nachmittag von 2\textsuperscript{h} an zu Hause\oindex{Hoerlgasse 16@\textbf{Hörlgasse 16}, \emph{Wohngebäude (K.WHS)}|pwv}, bis
                  6 Uhr. Übrigens auch Vor{\pb}mittag.\pend
           
\pstart
           Herzlich {\\[\baselineskip]}Ihr {\\[\baselineskip]}\spacefill\mbox{Salten}\pend
           \leftskip=0em{}\selectlanguage{ngerman}\endnumbering\briefempfaengerindex{Schnitzler, Arthur@\textsc{Schnitzler, Arthur}!zzzSalten, Felix@\emph{von Felix Salten}!1895-08-101@{{[}10.? 8. 1895{]}}|)be}\mylabel{L03162h}  \normalsize

\doendnotes{C}
\bigskip
\vfill

\clearpage

\footnotesize

\lohead{\textsc{register}}

% Definiere theindex-Environment komplett neu ohne reledmac
\makeatletter
\renewenvironment{theindex}{%
  \section*{\indexname}%
  \setlength{\parindent}{0pt}%
  \setlength{\parskip}{0pt plus 0.3pt}%
  \let\item\@idxitem
}{%
  \clearpage
}
\makeatother

\IfFileExists{\jobname-pw.ind}{\input{\jobname-pw.ind}}{}

\end{document}

      