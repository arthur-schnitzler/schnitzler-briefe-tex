%% latex-leseansicht-vorspann.tex
%% Vorspann für die Leseansicht.
%% Lädt die gemeinsame Datei latex-vorspann.tex mit nicht gesetztem Schalter.

\newif\ifkorrekturansicht
\korrekturansichtfalse

\input{../tex-inputs/latex-vorspann}


\section[ Paul Goldmann an Arthur Schnitzler, 1[7?]. 5. [1904]]{L03443 Paul Goldmann an Arthur Schnitzler,  1[7?]. 5. [1904]}
\nopagebreak\mylabel{L03443v}
\rehead{ }\normalsize\beginnumbering\briefempfaengerindex{Schnitzler, Arthur@\textsc{Schnitzler, Arthur}!zzzGoldmann, Paul@\emph{von Paul Goldmann}!1904-05-172@{1[7?]. 5. [1904]}|(be}
\toendnotes[C]{\smallbreak\pagebreak[2]}
\correspDesc{Versand  durch Paul Goldmann am 1[7?]. 5. [1904] in Berlin
\newline{}Erhalt  durch Arthur Schnitzler im Zeitraum [18. 5. 1904
                  – 22. 5. 1904?] in Taormina?}\toendnotes[C]{\smallbreak}
\Standort{DLA, A:Schnitzler, HS.NZ85.1.3174.}
\physDesc{Brief, 1 Blatt, 3 Seiten, 721 Zeichen
\newline{}Handschrift: blaue Tinte, deutsche Kurrent
\newline{}Schnitzler: mit Bleistift das Jahr »904« vermerkt }\toendnotes[C]{\smallbreak}
\pstart
           \raggedleft{}{\pb}\textcolor{gray}{\textbf{DESSAUERSTRASSE 19\oindex{Dessauer Straße@\textbf{Dessauer Straße}, \emph{Straße}|pw}}}\pend
           
\pstart
           Berlin\oindex{Berlin@\textbf{Berlin}, \emph{Hauptstadt}|pw}, 1\textcolor{gray}{7}. Mai.\pend
           
\pstart{}Mein lieber Freund,\pend\vspace{0.5em}
\pstart
           Ich danke Dir und Deiner Frau\pwindex{Schnitzler, Olga 17.\,1.\,1882 Wien – 13.\,1.\,1970 Lugano@\textsc{Schnitzler, Olga} (17.\,1.\,1882 Wien – 13.\,1.\,1970 Lugano), \emph{Schauspielerin, Sängerin}|pwv}
               vielmals für Eure Karten von \label{K_L03443-1v}\edtext{unterwegs}{\lemma{\textnormal{\emph{unterwegs}}}\Cendnote{\textnormal{ Zwischen 1. 5. 1904 und 29. 5. 1904 reisten
                     Arthur und Olga Schnitzler\pwindex{Schnitzler, Olga 17.\,1.\,1882 Wien – 13.\,1.\,1970 Lugano@\textsc{Schnitzler, Olga} (17.\,1.\,1882 Wien – 13.\,1.\,1970 Lugano), \emph{Schauspielerin, Sängerin}|pwk} nach Italien\oindex{Italien@\textbf{Italien}|pwk}. In Rom\oindex{Rom@\textbf{Rom}, \emph{Hauptstadt}|pwk}, wo die von Goldmann\pwindex{Goldmann, Paul 31.\,1.\,1865 Breslau – 25.\,9.\,1935 Wien@\textsc{Goldmann, Paul} (31.\,1.\,1865 Breslau – 25.\,9.\,1935 Wien), \emph{Schriftsteller, Journalist}|pwk} erwähnte Bildpostkarte abgeschickt
                  worden sein dürfte, waren sie vom 3. 5. 1904 bis zum 8. 5. 1904. In Folge
                  reisten sie weiter nach Neapel\oindex{Neapel@\textbf{Neapel}|pwk}, Pompeji\oindex{Pompeji@\textbf{Pompeji}, \emph{Ausgrabung}|pwk}, Palermo\oindex{Palermo@\textbf{Palermo}|pwk} und Taormina\oindex{Taormina@\textbf{Taormina}, \emph{Hauptstadt}|pwk}. }}}\label{K_L03443-1} und
               freue mich{ }ſehr, daß Eure Reiſe zur Ausführung gekommen iſt. Jetzt im Frühling muß es
               herrlich{ }ſein da unten; und der Anblick des Petersdoms\oindex{Petersdom@\textbf{Petersdom}, \emph{Kirche}|pw} auf Deiner Karte, den ich noch nie geſehen habe, hat {\pb}auch in mir \strikeout{g\textcolor{gray}{ro}} eine große Sehnſucht nach Italien\oindex{Italien@\textbf{Italien}|pw}
               wachgerufen. Aber ich kann{ }ſie nicht befriedigen. Denn meinen Urlaub muß ich diesmal
               ernſtlich zur Stärkung meiner Geſundheit verwenden; und darum bin ich entſchloſſen,
               nach Marienbad\oindex{Marienbad@\textbf{Marienbad}|pw} zu gehen.\pend
           
\pstart
           \label{K_L03443-2v}\edtext{Grüßt mir alſo Italien\oindex{Italien@\textbf{Italien}|pw}}{\lemma{\textnormal{\emph{Grüßt mir also Italien}}}\Cendnote{\textnormal{Im Brief vom XXXX Auszeichnungsfehler: Dokument L03444 nicht gefunden schrieb Goldmann\pwindex{Goldmann, Paul 31.\,1.\,1865 Breslau – 25.\,9.\,1935 Wien@\textsc{Goldmann, Paul} (31.\,1.\,1865 Breslau – 25.\,9.\,1935 Wien), \emph{Schriftsteller, Journalist}|pwk}, dass er mangels Adresse seine
                  Briefe nach Wien\oindex{Wien@\textbf{Wien}, \emph{Verwaltungsgebiet}|pwk} richte. Ob Schnitzler diesen Brief nachgesandt bekam oder erst nach
                  seiner Rückkehr vorfand, ist nicht zu bestimmen.}}}\label{K_L03443-2} und genießt die{ }ſchönen
               Tage dieſer Reiſe aus vollem Herzen!\pend
           
\pstart
           Neues weiß ich aus {\pb}Berlin\oindex{Berlin@\textbf{Berlin}, \emph{Hauptstadt}|pw} nicht zu melden.\pend
           
\pstart
           Viele herzliche Grüße Dir und Deiner Frau\pwindex{Schnitzler, Olga 17.\,1.\,1882 Wien – 13.\,1.\,1970 Lugano@\textsc{Schnitzler, Olga} (17.\,1.\,1882 Wien – 13.\,1.\,1970 Lugano), \emph{Schauspielerin, Sängerin}|pwv} von {\\[\baselineskip]}Deinem getreuen {\\[\baselineskip]}\spacefill\mbox{Paul Goldmn}\pend
           \leftskip=0em{}\selectlanguage{ngerman}\endnumbering\briefempfaengerindex{Schnitzler, Arthur@\textsc{Schnitzler, Arthur}!zzzGoldmann, Paul@\emph{von Paul Goldmann}!1904-05-172@{1[7?]. 5. [1904]}|)be}\mylabel{L03443h}  \newcommand{\dateiname}{L03443}\newcommand{\titel}{Paul Goldmann an Arthur Schnitzler, 1[7?]. 5. [1904]}\newcommand{\editorInnen}{Martin Anton Müller und Laura Untner}%% latex-leseansicht-abspann.tex
%% Abspann für die Leseansicht.
%% Der Schalter \ifkorrekturansicht ist bereits durch den Vorspann gesetzt.

%% latex-abspann.tex
%% Gemeinsamer Abspann für Korrekturansicht und Leseansicht.
%% Setzt den Schalter \ifkorrekturansicht voraus (gesetzt in den
%% einbindenden Dateien latex-korrekturansicht-abspann.tex bzw.
%% latex-leseansicht-abspann.tex).
%% ---------------------------------------------------------------

\normalsize

% Das esempio-Environment wird nur in der Leseansicht benötigt
\ifkorrekturansicht\else
\newenvironment{esempio}[3]%
{
    \vspace{1.5ex}
    \rlap{\underline{#1}}
    \par
    \setlength{\parindent}{0cm}
    \nopagebreak
    \leftskip=#2cm
    \rightskip=#3cm
}
{
    \par
}
\fi

\doendnotes{C}
\bigskip
\vfill

\clearpage

\footnotesize

\ifkorrekturansicht
  \lohead{\textsc{register}}
\fi

% theindex-Environment neu definieren ohne reledmac
\makeatletter
\renewenvironment{theindex}{%
  \ifkorrekturansicht
    \section*{\indexname}%
  \else
    \subsubsection*{Index der erwähnten Entitäten}%
  \fi
  \setlength{\parindent}{0pt}%
  \setlength{\parskip}{0pt plus 0.3pt}%
  \let\item\@idxitem
}{%
  \ifkorrekturansicht\clearpage\fi
}
\makeatother

\IfFileExists{\jobname-pw.ind}{\input{\jobname-pw.ind}}{}

% Quellenangabe nur in der Leseansicht
\ifkorrekturansicht\else
% Fallback-Definitionen, falls die .tex-Datei \titel etc. nicht gesetzt hat
\providecommand{\titel}{}
\providecommand{\editorInnen}{}
\providecommand{\dateiname}{\jobname}

\vspace{3cm}

\vfill

\footnotesize
\textsc{Quelle}: \titel. Herausgegeben von {\editorInnen}. In: \emph{Arthur Schnitzler: Briefwechsel mit Autorinnen und Autoren}.
 Digitale Edition, https://schnitzler-briefe.acdh.oeaw.ac.at/{\dateiname}.html (Stand \today)
\fi

\end{document}


