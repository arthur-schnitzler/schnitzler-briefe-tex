%% latex-korrekturansicht-vorspann.tex
%% Vorspann für die Korrekturansicht.
%% Lädt die gemeinsame Datei latex-vorspann.tex mit gesetztem Schalter.

\newif\ifkorrekturansicht
\korrekturansichttrue

\input{../tex-inputs/latex-vorspann}


\section[ Paul Goldmann an Arthur Schnitzler, 1{[}7?{]}. 5. {[}1904{]}]{L03443 Paul Goldmann an Arthur Schnitzler, 1{[}7?{]}. 5. {[}1904{]}}
\nopagebreak\mylabel{L03443v}
\rehead{ }\normalsize\beginnumbering\briefempfaengerindex{Schnitzler, Arthur@\textsc{Schnitzler, Arthur}!zzzGoldmann, Paul@\emph{von Paul Goldmann}!1904-05-172@{1{[}7?{]}. 5. {[}1904{]}}|(be}
\toendnotes[C]{\smallbreak\pagebreak[2]}\Standort{DLA, A:Schnitzler, HS.NZ85.1.3174.}
\physDesc{Brief, 1 Blatt, 3 Seiten, 721 Zeichen
\newline{}Handschrift: blaue Tinte, deutsche Kurrent
\newline{}Schnitzler: mit Bleistift das Jahr »904« vermerkt }\toendnotes[C]{\smallbreak}
\pstart
           \raggedleft{}{\pb}\textcolor{gray}{\textbf{DESSAUERSTRASSE 19\oindex{Dessauer Strasse@\textbf{Dessauer Straße}, \emph{Straße (K.STR)}|pw}}}\pend
           
\pstart
           Berlin\oindex{Berlin@\textbf{Berlin}, \emph{P.PPLC}|pw}, 1\textcolor{gray}{7}. Mai.\pend
           
\pstart{}Mein lieber Freund,\pend\vspace{0.5em}
\pstart
           Ich danke Dir und Deiner Frau\pwindex{Schnitzler, Olga 17.01.1882 – 13.01.1970@\textsc{Schnitzler, Olga} (17.01.1882 – 13.01.1970), \emph{Schauspieler/Schauspielerin, Sänger/Sängerin}|pwv}
               vielmals für Eure Karten von \label{K_L03443-1v}\edtext{unterwegs}{\lemma{\textnormal{\emph{unterwegs}}}\Cendnote{\textnormal{ Zwischen 1. 5. 1904 und 29. 5. 1904 reisten
                     Arthur und Olga Schnitzler\pwindex{Schnitzler, Olga 17.01.1882 – 13.01.1970@\textsc{Schnitzler, Olga} (17.01.1882 – 13.01.1970), \emph{Schauspieler/Schauspielerin, Sänger/Sängerin}|pwk} nach Italien\oindex{Italien@\textbf{Italien}, \emph{A.PCLI}|pwk}. In Rom\oindex{Rom@\textbf{Rom}, \emph{P.PPLC}|pwk}, wo die von Goldmann\pwindex{Goldmann, Paul 31.01.1865 – 25.09.1935@\textsc{Goldmann, Paul} (31.01.1865 – 25.09.1935), \emph{Schriftsteller/Schriftstellerin, Journalist/Journalistin}|pwk} erwähnte Bildpostkarte abgeschickt
                  worden sein dürfte, waren sie vom 3. 5. 1904 bis zum 8. 5. 1904. In Folge
                  reisten sie weiter nach Neapel\oindex{Neapel@\textbf{Neapel}, \emph{P.PPLA}|pwk}, Pompeji\oindex{Pompeji@\textbf{Pompeji}, \emph{S.ANS}|pwk}, Palermo\oindex{Palermo@\textbf{Palermo}, \emph{P.PPLA}|pwk} und Taormina\oindex{Taormina@\textbf{Taormina}, \emph{P.PPLA3}|pwk}. }}}\label{K_L03443-1} und
               freue mich ſehr, daß Eure Reiſe zur Ausführung gekommen iſt. Jetzt im Frühling muß es
               herrlich ſein da unten; und der Anblick des Petersdoms\oindex{Petersdom@\textbf{Petersdom}, \emph{Kirche (K.KRC)}|pw} auf Deiner Karte, den ich noch nie geſehen habe, hat {\pb}auch in mir \strikeout{g\textcolor{gray}{ro}} eine große Sehnſucht nach Italien\oindex{Italien@\textbf{Italien}, \emph{A.PCLI}|pw}
               wachgerufen. Aber ich kann ſie nicht befriedigen. Denn meinen Urlaub muß ich diesmal
               ernſtlich zur Stärkung meiner Geſundheit verwenden; und darum bin ich entſchloſſen,
               nach Marienbad\oindex{Marienbad@\textbf{Marienbad}, \emph{P.PPL}|pw} zu gehen.\pend
           
\pstart
           \label{K_L03443-2v}\edtext{Grüßt mir alſo Italien\oindex{Italien@\textbf{Italien}, \emph{A.PCLI}|pw}}{\lemma{\textnormal{\emph{Grüßt mir alſo Italien}}}\Cendnote{\textnormal{Im Brief vom 26. 5. [1904] schrieb Goldmann\pwindex{Goldmann, Paul 31.01.1865 – 25.09.1935@\textsc{Goldmann, Paul} (31.01.1865 – 25.09.1935), \emph{Schriftsteller/Schriftstellerin, Journalist/Journalistin}|pwk}, dass er mangels Adresse seine
                  Briefe nach Wien\oindex{Wien@\textbf{Wien}, \emph{A.ADM2}|pwk} richte. Ob Schnitzler diesen Brief nachgesandt bekam oder erst nach
                  seiner Rückkehr vorfand, ist nicht zu bestimmen.}}}\label{K_L03443-2} und genießt die ſchönen
               Tage dieſer Reiſe aus vollem Herzen!\pend
           
\pstart
           Neues weiß ich aus {\pb}Berlin\oindex{Berlin@\textbf{Berlin}, \emph{P.PPLC}|pw} nicht zu melden.\pend
           
\pstart
           Viele herzliche Grüße Dir und Deiner Frau\pwindex{Schnitzler, Olga 17.01.1882 – 13.01.1970@\textsc{Schnitzler, Olga} (17.01.1882 – 13.01.1970), \emph{Schauspieler/Schauspielerin, Sänger/Sängerin}|pwv} von {\\[\baselineskip]}Deinem getreuen {\\[\baselineskip]}\spacefill\mbox{Paul Goldmn}\pend
           \leftskip=0em{}\selectlanguage{ngerman}\endnumbering\briefempfaengerindex{Schnitzler, Arthur@\textsc{Schnitzler, Arthur}!zzzGoldmann, Paul@\emph{von Paul Goldmann}!1904-05-172@{1{[}7?{]}. 5. {[}1904{]}}|)be}\mylabel{L03443h}  \normalsize

\doendnotes{C}
\bigskip
\vfill

\clearpage

\footnotesize

\lohead{\textsc{register}}

% Definiere theindex-Environment komplett neu ohne reledmac
\makeatletter
\renewenvironment{theindex}{%
  \section*{\indexname}%
  \setlength{\parindent}{0pt}%
  \setlength{\parskip}{0pt plus 0.3pt}%
  \let\item\@idxitem
}{%
  \clearpage
}
\makeatother

\IfFileExists{\jobname-pw.ind}{\input{\jobname-pw.ind}}{}

\end{document}

      