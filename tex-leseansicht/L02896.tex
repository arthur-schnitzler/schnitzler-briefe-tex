%% latex-leseansicht-vorspann.tex
%% Vorspann für die Leseansicht.
%% Lädt die gemeinsame Datei latex-vorspann.tex mit nicht gesetztem Schalter.

\newif\ifkorrekturansicht
\korrekturansichtfalse

\input{../tex-inputs/latex-vorspann}

\begin{center}
            \textcolor{red}{ENTWURF, NICHT FERTIG KORRIGIERT}
                      \end{center}
            
         
         \renewcommand{\erwaehntePersonen}{Personen: Berthold Frischauer, Theodore Rottenberg}
         \renewcommand{\erwaehnteInstitutionen}{Institutionen: Frankfurter Zeitung, Neue Freie Presse}
         \renewcommand{\erwaehnteOrte}{Orte: Berlin, Frankfurt am Main, Gärtnerweg, Paris, Wien}
         \renewcommand{\erwaehnteWerke}{}
               \section[ Paul Goldmann an Arthur Schnitzler, 4. 12. {[}1899{]}]{ Paul Goldmann an Arthur Schnitzler, 4. 12. {[}1899{]}}\nopagebreak\mylabel{v}\rehead{ }\begin{ledgroupsized}[t]{13cm}\normalsize\beginnumbering \toendnotes[C]{\smallbreak\pagebreak[2]} \Standort{DLA, A:Schnitzler, HS.NZ85.1.3169.}
\physDesc{Brief, 1 Blatt, 2 Seiten
\newline{}Handschrift: schwarze Tinte, deutsche Kurrent
\newline{}Schnitzler: 1) mit Bleistift das Jahr »99« vermerkt  2) mit rotem Buntstift eine Unterstreichung}\toendnotes[C]{\smallbreak}\pstart
           \noindent{}{\pb}\textcolor{gray}{\textbf{\textsc{Frankfurter Zeitung}}}\orgindex{Frankfurter Zeitung@Frankfurter Zeitung|pw}\hfill \textcolor{gray}{\textbf{Frankfurt a. M.\oindex{Frankfurt am Main@\textbf{Frankfurt am Main}|pw},}}{ }4. Dezember.\pend
           \pstart
           \textsc{\textcolor{gray}{\textbf{und}}}\pend
           \pstart
           \textcolor{gray}{\textbf{\textsc{Handelsblatt.}}}\pend
           \pstart
           \textcolor{gray}{\textbf{\textsc{Redaktion\orgindex{Frankfurter Zeitung@Frankfurter Zeitung|pwv}.\footnote{\noindent{}\textcolor{gray}{\textbf{\textsc{Für die Redaktion\orgindex{Frankfurter Zeitung@Frankfurter Zeitung|pwv} beſtimmte Briefe und Sendungen
                                    wolle man \so{nicht} an die Perſon eines
                                    Redakteurs, ſondern ſtets \textbf{an die Redaktion der
                                          Frankfurter Zeitung\orgindex{Frankfurter Zeitung@Frankfurter Zeitung|pw}} adreſſiren.}}}}}}}\pend
           \pstart
           \textcolor{gray}{\textbf{\textsc{Telegramm-Adreſſe:}}}\pend
           \pstart
           \textcolor{gray}{\textbf{\textsc{Zeitung\orgindex{Frankfurter Zeitung@Frankfurter Zeitung|pwv}{ }Frankfurt Main\oindex{Frankfurt am Main@\textbf{Frankfurt am Main}|pw}.}}}\pend
           \pstart{}Mein lieber Freund,\pend\pstart
           Kurzes \textsc{Résumé} der Ereigniſſe in den letzten Tagen:\pend
           \pstart
           Ich erfahre, daß meine Freundin\pwindex{Rottenberg, Theodore 1875-09-07 – 1945-04-05@\textsc{Rottenberg, Theodore} (1875-09-07 – 1945-04-05)|pwv} mir zwei furchtbare \label{K_L02896-1v}\edtext{Tratſchereien}{\lemma{\textnormal{\emph{Tratſchereien}}}\Cendnote{\textnormal{nicht ermittelt}}}\label{K_L02896-1h}
               gemacht hat, und breche mit ihr.\pend
           \pstart
           Ich vernehme, daß \label{K_L02896-2v}\edtext{\textsc{Frischauer\pwindex{Frischauer, Berthold 1851-09-09 – 1924-02-04@\textsc{Frischauer, Berthold} (1851-09-09 – 1924-02-04), \emph{Journalist}|pw}} nach \textsc{Paris\oindex{Paris@\textbf{Paris}|pw}} zurückkehrt}{\lemma{\textnormal{\emph{Frischauer … zurückkehrt}}}\Cendnote{\textnormal{als Korrespondent der
                     \emph{Neuen Freien Presse}\orgindex{Neue Freie Presse@Neue Freie Presse|pwk}}}}\label{K_L02896-2h}, telegraphire an die »Neue Freie Preſſe\orgindex{Neue Freie Presse@Neue Freie Presse|pw}«,
               erhalte telegraphiſche Antwort, ich möchte mich zum ſofortigen Eintritt für den Berlin\oindex{Berlin@\textbf{Berlin}|pw}er Poſten bereit erklären, kündige bei der
                  Frankfurter Zeitung\orgindex{Frankfurter Zeitung@Frankfurter Zeitung|pw} und bin hier ſeit geſtern entlaſſen.\pend
           \pstart
           Von meinen Stimmungen rede ich nicht.\pend
           \pstart
           {\pb}Viele treue Grüße! {\\[\baselineskip]}Dein {\\[\baselineskip]}\spacefill\mbox{Paul Goldmann}\pend
           \leftskip=0em{}\pstart
           \noindent{}\uuline{\textsc{Gärtnerweg 47}\oindex{Gaertnerweg@\textbf{Gärtnerweg}|pw}.}\pend
           
         
         \endnumbering\mylabel{h}\end{ledgroupsized}  \newcommand{\dateiname}{L02896}\newcommand{\titel}{Paul Goldmann an Arthur Schnitzler, 4. 12. [1899]}\newcommand{\editorInnen}{Martin Anton Müller und Laura Untner}%% latex-leseansicht-abspann.tex
%% Abspann für die Leseansicht.
%% Der Schalter \ifkorrekturansicht ist bereits durch den Vorspann gesetzt.

%% latex-abspann.tex
%% Gemeinsamer Abspann für Korrekturansicht und Leseansicht.
%% Setzt den Schalter \ifkorrekturansicht voraus (gesetzt in den
%% einbindenden Dateien latex-korrekturansicht-abspann.tex bzw.
%% latex-leseansicht-abspann.tex).
%% ---------------------------------------------------------------

\normalsize

% Das esempio-Environment wird nur in der Leseansicht benötigt
\ifkorrekturansicht\else
\newenvironment{esempio}[3]%
{
    \vspace{1.5ex}
    \rlap{\underline{#1}}
    \par
    \setlength{\parindent}{0cm}
    \nopagebreak
    \leftskip=#2cm
    \rightskip=#3cm
}
{
    \par
}
\fi

\doendnotes{C}
\bigskip
\vfill

\clearpage

\footnotesize

\ifkorrekturansicht
  \lohead{\textsc{register}}
\fi

% theindex-Environment neu definieren ohne reledmac
\makeatletter
\renewenvironment{theindex}{%
  \ifkorrekturansicht
    \section*{\indexname}%
  \else
    \subsubsection*{Index der erwähnten Entitäten}%
  \fi
  \setlength{\parindent}{0pt}%
  \setlength{\parskip}{0pt plus 0.3pt}%
  \let\item\@idxitem
}{%
  \ifkorrekturansicht\clearpage\fi
}
\makeatother

\IfFileExists{\jobname-pw.ind}{\input{\jobname-pw.ind}}{}

% Quellenangabe nur in der Leseansicht
\ifkorrekturansicht\else
% Fallback-Definitionen, falls die .tex-Datei \titel etc. nicht gesetzt hat
\providecommand{\titel}{}
\providecommand{\editorInnen}{}
\providecommand{\dateiname}{\jobname}

\vspace{3cm}

\vfill

\footnotesize
\textsc{Quelle}: \titel. Herausgegeben von {\editorInnen}. In: \emph{Arthur Schnitzler: Briefwechsel mit Autorinnen und Autoren}.
 Digitale Edition, https://schnitzler-briefe.acdh.oeaw.ac.at/{\dateiname}.html (Stand \today)
\fi

\end{document}


      