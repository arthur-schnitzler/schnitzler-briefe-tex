%% latex-leseansicht-vorspann.tex
%% Vorspann für die Leseansicht.
%% Lädt die gemeinsame Datei latex-vorspann.tex mit nicht gesetztem Schalter.

\newif\ifkorrekturansicht
\korrekturansichtfalse

\input{../tex-inputs/latex-vorspann}


\section[ Paul Goldmann an Arthur Schnitzler, 4. 12. {[}1899{]}]{L02896 Paul Goldmann an Arthur Schnitzler,  4. 12. [1899]}
\nopagebreak\mylabel{L02896v}
\rehead{ }\normalsize\beginnumbering\briefempfaengerindex{Schnitzler, Arthur@\textsc{Schnitzler, Arthur}!zzzGoldmann, Paul@\emph{von Paul Goldmann}!1899-12-042@{4. 12. [1899]}|(be}
\toendnotes[C]{\smallbreak\pagebreak[2]}
\correspDesc{Versand  durch Paul Goldmann am 4. 12. [1899] in Frankfurt am Main
\newline{}Erhalt  durch Arthur Schnitzler im Zeitraum [5. 12. 1899
                  – 9. 12. 1899?] in Wien}\toendnotes[C]{\smallbreak}
\Standort{DLA, A:Schnitzler, HS.NZ85.1.3169.}
\physDesc{Brief, 1 Blatt, 2 Seiten, 530 Zeichen
\newline{}Handschrift: schwarze Tinte, deutsche Kurrent
\newline{}Schnitzler: 1) mit Bleistift das Jahr »99« vermerkt  2) mit rotem Buntstift eine Unterstreichung}\toendnotes[C]{\smallbreak}
\pstart
           {\pb}\textcolor{gray}{\textbf{\textbf{Frankfurter Zeitung}}}\orgindex{Frankfurter Zeitung@Frankfurter Zeitung|pw}\hfill \textcolor{gray}{\textbf{\textbf{Frankfurt a. M.\oindex{Frankfurt am Main@\textbf{Frankfurt am Main}, \emph{Hauptstadt}|pw},}}}{ }4. Dezember.\pend
           
\pstart
           \textcolor{gray}{\textbf{und}}\pend
           
\pstart
           \textcolor{gray}{\textbf{Handelsblatt.}}\pend
           
\pstart
           \textcolor{gray}{\textbf{\textbf{Redaktion\orgindex{Frankfurter Zeitung@Frankfurter Zeitung|pwv}.}\footnote{\noindent{}\textcolor{gray}{\textbf{Für die Redaktion\orgindex{Frankfurter Zeitung@Frankfurter Zeitung|pwv} beſtimmte Briefe und Sendungen wolle man
                                 \so{nicht} an die Perſon eines Redakteurs,{ }ſondern{ }ſtets \textbf{an die Redaktion der Frankfurter Zeitung\orgindex{Frankfurter Zeitung@Frankfurter Zeitung|pw}} adreſſiren.}}}}}\pend
           
\pstart
           \textcolor{gray}{\textbf{Telegramm-Adreſſe:}}\pend
           
\pstart
           \textcolor{gray}{\textbf{\textbf{Zeitung\orgindex{Frankfurter Zeitung@Frankfurter Zeitung|pwv}{ }Frankfurt Main\oindex{Frankfurt am Main@\textbf{Frankfurt am Main}, \emph{Hauptstadt}|pw}}}}\pend
           
\pstart{}Mein lieber Freund,\pend\vspace{0.5em}
\pstart
           Kurzes \textsc{Résumé} der Ereigniſſe in den letzten Tagen:\pend
           
\pstart
           Ich erfahre, daß meine Freundin\pwindex{Rottenberg, Theodore 7.\,9.\,1875 – 5.\,4.\,1945 Limburg an der Lahn@\textsc{Rottenberg, Theodore} (7.\,9.\,1875 – 5.\,4.\,1945 Limburg an der Lahn)|pwv} mir zwei furchtbare \label{K_L02896-1v}\edtext{Tratſchereien}{\lemma{\textnormal{\emph{Tratschereien}}}\Cendnote{\textnormal{nicht ermittelt}}}\label{K_L02896-1}
               gemacht hat, und breche mit ihr.\pend
           
\pstart
           Ich vernehme, daß \label{K_L02896-2v}\edtext{\textsc{Frischauer\pwindex{Frischauer, Berthold 9.\,9.\,1851 Brünn – 4.\,2.\,1924 Wien@\textsc{Frischauer, Berthold} (9.\,9.\,1851 Brünn – 4.\,2.\,1924 Wien), \emph{Journalist}|pw}} nach \textsc{Paris\oindex{Paris@\textbf{Paris}, \emph{Hauptstadt}|pw}} zurückkehrt}{\lemma{\textnormal{\emph{Frischauer … zurückkehrt}}}\Cendnote{\textnormal{als Korrespondent der
                     \emph{Neuen Freien Presse}\orgindex{Neue Freie Presse@Neue Freie Presse|pwk}}}}\label{K_L02896-2}, telegraphire an die »Neue Freie Preſſe\orgindex{Neue Freie Presse@Neue Freie Presse|pw}«,
               erhalte telegraphiſche Antwort, ich möchte mich zum{ }ſofortigen Eintritt für den Berlin\oindex{Berlin@\textbf{Berlin}, \emph{Hauptstadt}|pw}er Poſten bereit erklären, kündige bei der
                  Frankfurter Zeitung\orgindex{Frankfurter Zeitung@Frankfurter Zeitung|pw} und bin hier{ }ſeit geſtern entlaſſen.\pend
           
\pstart
           Von meinen Stimmungen rede ich nicht.\pend
           
\pstart
           {\pb}Viele treue Grüße! {\\[\baselineskip]}Dein {\\[\baselineskip]}\spacefill\mbox{Paul Goldmann}\pend
           \leftskip=0em{}
\pstart
           \noindent{}\uuline{\textsc{Gärtnerweg 47}\oindex{Gärtnerweg@\textbf{Gärtnerweg}, \emph{Straße}|pw}.}\pend
           \selectlanguage{ngerman}\endnumbering\briefempfaengerindex{Schnitzler, Arthur@\textsc{Schnitzler, Arthur}!zzzGoldmann, Paul@\emph{von Paul Goldmann}!1899-12-042@{4. 12. [1899]}|)be}\mylabel{L02896h}  \newcommand{\dateiname}{L02896}\newcommand{\titel}{Paul Goldmann an Arthur Schnitzler, 4. 12. [1899]}\newcommand{\editorInnen}{Martin Anton Müller und Laura Untner}%% latex-leseansicht-abspann.tex
%% Abspann für die Leseansicht.
%% Der Schalter \ifkorrekturansicht ist bereits durch den Vorspann gesetzt.

%% latex-abspann.tex
%% Gemeinsamer Abspann für Korrekturansicht und Leseansicht.
%% Setzt den Schalter \ifkorrekturansicht voraus (gesetzt in den
%% einbindenden Dateien latex-korrekturansicht-abspann.tex bzw.
%% latex-leseansicht-abspann.tex).
%% ---------------------------------------------------------------

\normalsize

% Das esempio-Environment wird nur in der Leseansicht benötigt
\ifkorrekturansicht\else
\newenvironment{esempio}[3]%
{
    \vspace{1.5ex}
    \rlap{\underline{#1}}
    \par
    \setlength{\parindent}{0cm}
    \nopagebreak
    \leftskip=#2cm
    \rightskip=#3cm
}
{
    \par
}
\fi

\doendnotes{C}
\bigskip
\vfill

\clearpage

\footnotesize

\ifkorrekturansicht
  \lohead{\textsc{register}}
\fi

% theindex-Environment neu definieren ohne reledmac
\makeatletter
\renewenvironment{theindex}{%
  \ifkorrekturansicht
    \section*{\indexname}%
  \else
    \subsubsection*{Index der erwähnten Entitäten}%
  \fi
  \setlength{\parindent}{0pt}%
  \setlength{\parskip}{0pt plus 0.3pt}%
  \let\item\@idxitem
}{%
  \ifkorrekturansicht\clearpage\fi
}
\makeatother

\IfFileExists{\jobname-pw.ind}{\input{\jobname-pw.ind}}{}

% Quellenangabe nur in der Leseansicht
\ifkorrekturansicht\else
% Fallback-Definitionen, falls die .tex-Datei \titel etc. nicht gesetzt hat
\providecommand{\titel}{}
\providecommand{\editorInnen}{}
\providecommand{\dateiname}{\jobname}

\vspace{3cm}

\vfill

\footnotesize
\textsc{Quelle}: \titel. Herausgegeben von {\editorInnen}. In: \emph{Arthur Schnitzler: Briefwechsel mit Autorinnen und Autoren}.
 Digitale Edition, https://schnitzler-briefe.acdh.oeaw.ac.at/{\dateiname}.html (Stand \today)
\fi

\end{document}


