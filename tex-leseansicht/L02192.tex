%% latex-leseansicht-vorspann.tex
%% Vorspann für die Leseansicht.
%% Lädt die gemeinsame Datei latex-vorspann.tex mit nicht gesetztem Schalter.

\newif\ifkorrekturansicht
\korrekturansichtfalse

\input{../tex-inputs/latex-vorspann}


\section[Arthur und Olga Schnitzler an Richard Beer-Hofmann, 20. 8. 1914]{L02192 Arthur und Olga Schnitzler an Richard Beer-Hofmann, 20. 8. 1914}
\nopagebreak\mylabel{L02192v}
\rehead{ }\normalsize\beginnumbering\briefempfaengerindex{Beer-Hofmann, Richard@\textsc{Beer-Hofmann, Richard}!zzzSchnitzler, Olga@\emph{von Olga Schnitzler}!1914-08-201@{20. 8. 1914}|(be}\briefempfaengerindex{Beer-Hofmann, Richard@\textsc{Beer-Hofmann, Richard}!zzzSchnitzler, Arthur@\emph{von Arthur Schnitzler}!1914-08-201@{20. 8. 1914}|(be}
\toendnotes[C]{\smallbreak\pagebreak[2]}
\correspDesc{Versand  durch Arthur Schnitzler, Olga Schnitzler am 20. 8. 1914 in Salzburg
\newline{}Erhalt  durch Richard Beer-Hofmann im Zeitraum [21. 8. 1914
                  – 25. 8. 1914?] in Weißenbach am Attersee}\toendnotes[C]{\smallbreak}
\Standort{YCGL, MSS 31.}
\physDesc{Bildpostkarte, 364 Zeichen
\newline{}Handschrift Arthur Schnitzler: Bleistift, deutsche Kurrent
\newline{}Handschrift Olga Schnitzler: schwarze Tinte, lateinische Kurrent
\newline{}Versand: Stempel: »\nobreak{}\oindex{Salzburg@\textbf{Salzburg}, \emph{Verwaltungsgebiet}|pwk}Salzburg 2, 20. VIII. 14, 9\nobreak{}«.  
\newline{}Beer-Hofmann: mit blauem Buntstift den Erhalt und die Beantwortung
                                    markiert: »E. B.« 
\newline{}Zusatz: eine ursprüngliche Adressierung an eine andere Person durch
                                 Übermalung unkenntlich gemacht }\pstart{}{\pb}Hrn\pend{}\pstart{}\textsc{Dr. Richard Beerhofmann}\pend{}\pstart{}\textsc{Weißenbach\oindex{Weißenbach am Attersee@\textbf{Weißenbach am Attersee}, \emph{Verwaltungsgebiet}|pw}.}\pend{}\pstart{}\textsc{am Attersee\oindex{Attersee@\textbf{Attersee}, \emph{See}|pw}}\pend{}\pstart{}\textsc{Ob. Oesterreich}\oindex{Oberösterreich@\textbf{Oberösterreich}, \emph{Land}|pw}\pend{}{\bigskip}
\pstart
           \noindent{}\centering{}{\pb}\textcolor{gray}{\textbf{Salzburg\oindex{Salzburg@\textbf{Salzburg}, \emph{Verwaltungsgebiet}|pw}. Hohensalzburg\oindex{Festung Hohensalzburg@\textbf{Festung Hohensalzburg}, \emph{Gebäude}|pw}.}}\pend
           \vspace{1em}
\pstart
           \noindent{}{\pb}lieber Richard,{ }ſind Sie noch in Weißenbach\oindex{Weißenbach am Attersee@\textbf{Weißenbach am Attersee}, \emph{Verwaltungsgebiet}|pw}? Wir{ }ſind geſtern nach einer endloſen Fahrt in Salzburg\oindex{Salzburg@\textbf{Salzburg}, \emph{Verwaltungsgebiet}|pw} angek\damage{\textcolor{gray}{o{\geminationm}en}} und wollen morgen nach Iſchl\oindex{Bad Ischl@\textbf{Bad Ischl}|pw}, auf ganz
               wenige Tage, zum Zwecke des Ausruhens vor der Heimfahrt. Schreiben Sie mir bitte eine
               Zeile \textsc{\uline{post rest}}. Herzlichſt Ihr\pend
           \pstart \spacefill\mbox{Arthur Schnitzler}\pend{}
\pstart
           \raggedleft{}20/8 914\pend
           \selectlanguage{ngerman}\vspace{1em}
\pstart
           \noindent{}{[}hs. Schnitzler:{]} Herzlichst Ihre\pend
           \pstart \spacefill\mbox{O.}\pend{}\selectlanguage{ngerman}\endnumbering\briefempfaengerindex{Beer-Hofmann, Richard@\textsc{Beer-Hofmann, Richard}!zzzSchnitzler, Olga@\emph{von Olga Schnitzler}!1914-08-201@{20. 8. 1914}|)be}\briefempfaengerindex{Beer-Hofmann, Richard@\textsc{Beer-Hofmann, Richard}!zzzSchnitzler, Arthur@\emph{von Arthur Schnitzler}!1914-08-201@{20. 8. 1914}|)be}\mylabel{L02192h}  \newcommand{\dateiname}{L02192}\newcommand{\titel}{Arthur und Olga Schnitzler an Richard Beer-Hofmann, 20. 8. 1914}\newcommand{\editorInnen}{Martin Anton Müller und Gerd-Hermann Susen}%% latex-leseansicht-abspann.tex
%% Abspann für die Leseansicht.
%% Der Schalter \ifkorrekturansicht ist bereits durch den Vorspann gesetzt.

%% latex-abspann.tex
%% Gemeinsamer Abspann für Korrekturansicht und Leseansicht.
%% Setzt den Schalter \ifkorrekturansicht voraus (gesetzt in den
%% einbindenden Dateien latex-korrekturansicht-abspann.tex bzw.
%% latex-leseansicht-abspann.tex).
%% ---------------------------------------------------------------

\normalsize

% Das esempio-Environment wird nur in der Leseansicht benötigt
\ifkorrekturansicht\else
\newenvironment{esempio}[3]%
{
    \vspace{1.5ex}
    \rlap{\underline{#1}}
    \par
    \setlength{\parindent}{0cm}
    \nopagebreak
    \leftskip=#2cm
    \rightskip=#3cm
}
{
    \par
}
\fi

\doendnotes{C}
\bigskip
\vfill

\clearpage

\footnotesize

\ifkorrekturansicht
  \lohead{\textsc{register}}
\fi

% theindex-Environment neu definieren ohne reledmac
\makeatletter
\renewenvironment{theindex}{%
  \ifkorrekturansicht
    \section*{\indexname}%
  \else
    \subsubsection*{Index der erwähnten Entitäten}%
  \fi
  \setlength{\parindent}{0pt}%
  \setlength{\parskip}{0pt plus 0.3pt}%
  \let\item\@idxitem
}{%
  \ifkorrekturansicht\clearpage\fi
}
\makeatother

\IfFileExists{\jobname-pw.ind}{\input{\jobname-pw.ind}}{}

% Quellenangabe nur in der Leseansicht
\ifkorrekturansicht\else
% Fallback-Definitionen, falls die .tex-Datei \titel etc. nicht gesetzt hat
\providecommand{\titel}{}
\providecommand{\editorInnen}{}
\providecommand{\dateiname}{\jobname}

\vspace{3cm}

\vfill

\footnotesize
\textsc{Quelle}: \titel. Herausgegeben von {\editorInnen}. In: \emph{Arthur Schnitzler: Briefwechsel mit Autorinnen und Autoren}.
 Digitale Edition, https://schnitzler-briefe.acdh.oeaw.ac.at/{\dateiname}.html (Stand \today)
\fi

\end{document}


