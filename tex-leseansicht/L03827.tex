%% latex-korrekturansicht-vorspann.tex
%% Vorspann für die Korrekturansicht.
%% Lädt die gemeinsame Datei latex-vorspann.tex mit gesetztem Schalter.

\newif\ifkorrekturansicht
\korrekturansichttrue

\input{../tex-inputs/latex-vorspann}


\section[Theodor Herzl an Arthur Schnitzler, 4. 5. 1893]{L03827 Theodor Herzl an Arthur Schnitzler, 4. 5. 1893}
\nopagebreak\mylabel{L03827v}
\rehead{ }\normalsize\beginnumbering\briefempfaengerindex{Schnitzler, Arthur@\textsc{Schnitzler, Arthur}!zzzHerzl, Theodor@\emph{von Theodor Herzl}!1893-05-041@{4. 5. 1892}|(be}
\toendnotes[C]{\smallbreak\pagebreak[2]}\Standort{CUL, Schnitzler, B 39.}
\physDesc{Brief, 1 Blatt, 1 Seite, 535 Zeichen
\newline{}Handschrift: schwarze Tinte, lateinische Kurrent
\newline{}Ordnung: mit Bleistift von unbekannter Hand nummeriert: »8« }\toendnotes[C]{\smallbreak}
\pstart
           {\pb}\textcolor{gray}{\textbf{NOUVELLE PRESSE LIBRE }}\orgindex{Neue Freie Presse@Neue Freie Presse|pw}\hfill \textcolor{gray}{\textbf{8, Rue de Monceau }}\oindex{8, Rue de Monceau@\textbf{8, Rue de Monceau}, \emph{Wohngebäude (K.WHS)}|pw}\pend
           
\pstart
           \textcolor{gray}{\textbf{D\textsuperscript{R} TH. HERZL}}\pend
           
\pstart{}Mein lieber Freund!\pend\vspace{0.5em}
\pstart
           Sehr erschüttert lese ich \label{K_L03827-1v}\edtext{in der
                  Zeitung}{\lemma{\textnormal{\emph{in der
                  Zeitung}}}\Cendnote{\textnormal{Johann Schnitzler starb am
                     2. 5. 1893. Am Folgetag wurde eine Traueranzeige der Familie\pwindex{Schnitzler, Julius 13.07.1865 – 29.06.1939@\textsc{Schnitzler, Julius} (13.07.1865 – 29.06.1939), \emph{Chirurg/Chirurgin}|pwkv}\pwindex{Hajek, Gisela 20.12.1867 – 03.02.1953@\textsc{Hajek, Gisela} (20.12.1867 – 03.02.1953)|pwkv}\pwindex{Schnitzler, Louise 1840-07-08 – 1911-09-09@\textsc{Schnitzler, Louise} (1840-07-08 – 1911-09-09)|pwkv}\pwindex{Hajek, Markus 25.11.1861 – 04.04.1941@\textsc{Hajek, Markus} (25.11.1861 – 04.04.1941), \emph{Mediziner/Medizinerin, Laryngologe/Laryngologin}|pwkv}\pwindex{Wilheim, Johanna 1839? – September 1925@\textsc{Wilheim, Johanna} (1839? – September 1925)|pwkv} in der \emph{Presse}\pwindex{Presse@\emph{Die Presse}|pwk}, der \emph{Neuen Freien Presse}\pwindex{Neue Freie Presse@\emph{Neue Freie Presse}|pwk} und der \emph{Wiener Zeitung}\pwindex{Wiener Zeitung@\emph{Wiener Zeitung}|pwk} gedruckt. Zwei Nachrufe gab die \emph{Wiener Allgemeine Zeitung}\pwindex{Wiener Allgemeine Zeitung@\emph{Wiener Allgemeine Zeitung}|pwk} (\emph{Professor Dr. Johann Schnitzler}. In: \emph{Wiener Allgemeine Zeitung}\pwindex{Wiener Allgemeine Zeitung@\emph{Wiener Allgemeine Zeitung}|pwk}, Nr. 4528,
                        3. 5. 1893, S. 6).}}}\label{K_L03827-1}, dass Ihr Vater\pwindex{Schnitzler, Johann 10.04.1835 – 02.05.1893@\textsc{Schnitzler, Johann} (10.04.1835 – 02.05.1893), \emph{Laryngologe/Laryngologin}|pwv} gestorben ist. \pend
           
\pstart
           Wie arm ist unsere Rede, wenn wir einen wirklichen grossen Schmerz vor uns haben. \pend
           
\pstart
           Ein stummer Händedruck sagt die Theilnahme noch am besten – lassen Sie diese Zeilen
               dafür gelten. \pend
           
\pstart
           Sagen Sie auch Ihrer verehrten Frau Mutter\pwindex{Schnitzler, Louise 1840-07-08 – 1911-09-09@\textsc{Schnitzler, Louise} (1840-07-08 – 1911-09-09)|pwv} und Ihren lieben Geschwistern\pwindex{Schnitzler, Julius 13.07.1865 – 29.06.1939@\textsc{Schnitzler, Julius} (13.07.1865 – 29.06.1939), \emph{Chirurg/Chirurgin}|pwv}\pwindex{Hajek, Gisela 20.12.1867 – 03.02.1953@\textsc{Hajek, Gisela} (20.12.1867 – 03.02.1953)|pwv}, dass ich zu denen
               gehöre, die an Ihrem schweren Verlust \introOben{}am\introOben{} Innigsten
               theilnehmen.\pend
           
\pstart
           Leben Sie wohl, mein lieber Schnitzler und glauben Sie an die Freundschaft
               {\\[\baselineskip]}Ihres herzlich ergebenen {\\[\baselineskip]}\spacefill\mbox{Th Herzl}\pend
           \leftskip=0em{}
\pstart
           Paris\oindex{Paris@\textbf{Paris}, \emph{P.PPLC}|pw}{ }4 Mai 93\pend
           \selectlanguage{ngerman}\endnumbering\briefempfaengerindex{Schnitzler, Arthur@\textsc{Schnitzler, Arthur}!zzzHerzl, Theodor@\emph{von Theodor Herzl}!1893-05-041@{4. 5. 1892}|)be}\mylabel{L03827h}
\begin{anhang}
\end{anhang}\normalsize

\doendnotes{C}
\bigskip
\vfill

\clearpage

\footnotesize

\lohead{\textsc{register}}

% Definiere theindex-Environment komplett neu ohne reledmac
\makeatletter
\renewenvironment{theindex}{%
  \section*{\indexname}%
  \setlength{\parindent}{0pt}%
  \setlength{\parskip}{0pt plus 0.3pt}%
  \let\item\@idxitem
}{%
  \clearpage
}
\makeatother

\IfFileExists{\jobname-pw.ind}{\input{\jobname-pw.ind}}{}

\end{document}

      