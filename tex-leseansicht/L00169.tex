%% latex-leseansicht-vorspann.tex
%% Vorspann für die Leseansicht.
%% Lädt die gemeinsame Datei latex-vorspann.tex mit nicht gesetztem Schalter.

\newif\ifkorrekturansicht
\korrekturansichtfalse

\input{../tex-inputs/latex-vorspann}

\begin{center}
            \textcolor{red}{ENTWURF. ENTZIFFERUNG NOCH NICHT KORREKTURGELESEN}
                      \end{center}
            
               \section[Arthur Schnitzler an Richard Beer-Hofmann, {[}1. 2. 1893?{]}]{ Arthur Schnitzler an Richard Beer-Hofmann, {[}1. 2. 1893?{]}}\nopagebreak\mylabel{v}\rehead{ }\begin{ledgroupsized}[t]{13cm}\normalsize\beginnumbering\briefempfaengerindex{Beer-Hofmann, Richard@\textsc{Beer-Hofmann, Richard}!zzzSchnitzler, Arthur@\emph{von Arthur Schnitzler}!1893-02-011@{1. 2. {[}1893?{]}}|(be} \toendnotes[C]{\smallbreak\pagebreak[2]} \Standort{YCGL, MSS 31.}
\physDesc{Kartenbrief
\newline{}Handschrift: Bleistift, deutsche Kurrent\newline{}Versand: 1) Rohrpost 2) Stempel: »\nobreak{}Wien, {[}1{]} \textcolor{gray}{II 93}, 9 15 V\nobreak{}«. 3) Stempel: »\nobreak{}\oindex{I., Innere Stadt@\textbf{I., Innere Stadt}|pwk}Wien 1/1, 1 II 92, 10 10V\nobreak{}«. }\toendnotes[C]{\smallbreak}\pstart{}{\pb}Hrn\pend{}\pstart{}\textsc{Dr Rich Beer Hofmann}\pend{}\pstart{}\textsc{Wien\oindex{Wien@\textbf{Wien}|pw}}\pend{}\pstart{}\textsc{I Wollzeile 15\oindex{Wollzeile@\textbf{Wollzeile}|pw}}\pend{}{\bigskip}\pstart{}{\pb}Mein lieber Richard\pend\pstart
           ich geh auf die \label{K_L00169_1v}\edtext{\textsc{Opernredoute}}{\lemma{\textnormal{\emph{Opernredoute}}}\Cendnote{\textnormal{Der Maskenball in der Oper, die
                  Opernredoute, fand am 1. 2. 1893 statt. Einlass war um
                     22 Uhr. Obzwar der zweite Poststempel eindeutig auf
                     1892 verweist und das 93 des ersten Poststempels
                  nicht mit letzter Sicherheit zu entziffern ist, sprechen mehrere Gründe dafür,
                  einen falsch eingestellten Stempel anzunehmen: 1892 war die
                  Opernredoute am 31. 1., Schnitzler\pwindex{Schnitzler, Arthur 15.05.1862 – 21.10.1931@\textsc{Schnitzler, Arthur} (15.05.1862 – 21.10.1931), \emph{Schriftsteller, Mediziner}|pwk}
                  besuchte sie nicht. Sowohl die Anwesenheit Beer-Hofmann\pwindex{Beer-Hofmann, Richard 11.07.1866 – 26.09.1945@\textsc{Beer-Hofmann, Richard} (11.07.1866 – 26.09.1945), \emph{Schriftsteller}|pwk}s auf dem Ball 1893, als auch ein Besuch Schnitzler\pwindex{Schnitzler, Arthur 15.05.1862 – 21.10.1931@\textsc{Schnitzler, Arthur} (15.05.1862 – 21.10.1931), \emph{Schriftsteller, Mediziner}|pwk}s im Carl-Theater\oindex{Carl-Theater@\textbf{Carl-Theater}|pwk}s lassen sich 1893 nachweisen.}}}\label{K_L00169_1h}. Wollen Sie vorher mit
               mir ſoupiren? Oder ſich im Café mit mir treffen? –\pend
           \pstart Ihr \spacefill\mbox{Arthur}\pend{}\pstart
           \noindent{}\label{T_L00169_1v}\edtext{Oder gehn}{\lemma{\textnormal{\emph{Oder gehn}}}\Cendnote{\textnormal{am rechten Rand}}}\label{T_L00169_1h} Sie auch ins \label{K_L00169_2v}\edtext{Carlth\oindex{Carl-Theater@\textbf{Carl-Theater}|pw}}{\lemma{\textnormal{\emph{Carlth}}}\Cendnote{\textnormal{Schnitzler\pwindex{Schnitzler, Arthur 15.05.1862 – 21.10.1931@\textsc{Schnitzler, Arthur} (15.05.1862 – 21.10.1931), \emph{Schriftsteller, Mediziner}|pwk} besuchte die Aufführung von \emph{Madame Mongodin}\pwindex{\textcolor{red}{\textsuperscript{XXXX1 indx}}!Madame Mongodin1891 – 1891@\strich\emph{Madame Mongodin} {[}1891 – 1891{]}|pwk}\pwindex{\textcolor{red}{\textsuperscript{XXXX1 indx}}!Madame Mongodin1891 – 1891@\strich\emph{Madame Mongodin} {[}1891 – 1891{]}|pwk}, die um 7 Uhr
                     begann.}}}\label{K_L00169_2h}\pend
           \pstart
           {\pb}(\textsc{I was \uline{not} on the \label{K_L00169_3v}\edtext{Weisse
                     Kreuz Ball}{\lemma{\textnormal{\emph{Weisse
                     Kreuz Ball}}}\Cendnote{\textnormal{Dieser hatte am
                        Vorabend, dem 31. 1. 1893 stattgefunden.}}}\label{K_L00169_3h})}\pend
           \endnumbering\briefempfaengerindex{Beer-Hofmann, Richard@\textsc{Beer-Hofmann, Richard}!zzzSchnitzler, Arthur@\emph{von Arthur Schnitzler}!1893-02-011@{1. 2. {[}1893?{]}}|)be}\mylabel{h}\end{ledgroupsized}  \newcommand{\dateiname}{L00169}\newcommand{\titel}{Arthur Schnitzler an Richard Beer-Hofmann, [1. 2. 1893?]}\newcommand{\editorInnen}{Martin Anton Müller und Gerd-Hermann Susen}%% latex-leseansicht-abspann.tex
%% Abspann für die Leseansicht.
%% Der Schalter \ifkorrekturansicht ist bereits durch den Vorspann gesetzt.

%% latex-abspann.tex
%% Gemeinsamer Abspann für Korrekturansicht und Leseansicht.
%% Setzt den Schalter \ifkorrekturansicht voraus (gesetzt in den
%% einbindenden Dateien latex-korrekturansicht-abspann.tex bzw.
%% latex-leseansicht-abspann.tex).
%% ---------------------------------------------------------------

\normalsize

% Das esempio-Environment wird nur in der Leseansicht benötigt
\ifkorrekturansicht\else
\newenvironment{esempio}[3]%
{
    \vspace{1.5ex}
    \rlap{\underline{#1}}
    \par
    \setlength{\parindent}{0cm}
    \nopagebreak
    \leftskip=#2cm
    \rightskip=#3cm
}
{
    \par
}
\fi

\doendnotes{C}
\bigskip
\vfill

\clearpage

\footnotesize

\ifkorrekturansicht
  \lohead{\textsc{register}}
\fi

% theindex-Environment neu definieren ohne reledmac
\makeatletter
\renewenvironment{theindex}{%
  \ifkorrekturansicht
    \section*{\indexname}%
  \else
    \subsubsection*{Index der erwähnten Entitäten}%
  \fi
  \setlength{\parindent}{0pt}%
  \setlength{\parskip}{0pt plus 0.3pt}%
  \let\item\@idxitem
}{%
  \ifkorrekturansicht\clearpage\fi
}
\makeatother

\IfFileExists{\jobname-pw.ind}{\input{\jobname-pw.ind}}{}

% Quellenangabe nur in der Leseansicht
\ifkorrekturansicht\else
% Fallback-Definitionen, falls die .tex-Datei \titel etc. nicht gesetzt hat
\providecommand{\titel}{}
\providecommand{\editorInnen}{}
\providecommand{\dateiname}{\jobname}

\vspace{3cm}

\vfill

\footnotesize
\textsc{Quelle}: \titel. Herausgegeben von {\editorInnen}. In: \emph{Arthur Schnitzler: Briefwechsel mit Autorinnen und Autoren}.
 Digitale Edition, https://schnitzler-briefe.acdh.oeaw.ac.at/{\dateiname}.html (Stand \today)
\fi

\end{document}


      