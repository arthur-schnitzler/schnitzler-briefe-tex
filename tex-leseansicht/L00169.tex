%% latex-korrekturansicht-vorspann.tex
%% Vorspann für die Korrekturansicht.
%% Lädt die gemeinsame Datei latex-vorspann.tex mit gesetztem Schalter.

\newif\ifkorrekturansicht
\korrekturansichttrue

\input{../tex-inputs/latex-vorspann}


\section[Arthur Schnitzler an Richard Beer-Hofmann, {[}1. 2. 1893?{]}]{L00169 Arthur Schnitzler an Richard Beer-Hofmann, {[}1. 2. 1893?{]}}
\nopagebreak\mylabel{L00169v}
\rehead{ }\normalsize\beginnumbering\briefempfaengerindex{Beer-Hofmann, Richard@\textsc{Beer-Hofmann, Richard}!zzzSchnitzler, Arthur@\emph{von Arthur Schnitzler}!1893-02-011@{1. 2. {[}1893?{]}}|(be}
\toendnotes[C]{\smallbreak\pagebreak[2]}\Standort{YCGL, MSS 31.}
\physDesc{Kartenbrief, 230 Zeichen
\newline{}Handschrift: Bleistift, deutsche Kurrent
\newline{}Versand: 1) Rohrpost  2) Stempel: »\nobreak{}Wien, {[}1{]}\textcolor{gray}{II 93}, 9 15 V\nobreak{}«.  3) Stempel: »\nobreak{}\oindex{I., Innere Stadt@\textbf{I., Innere Stadt}, \emph{A.ADM3}|pwk}Wien 1/1, 1 II 92, 10 10V\nobreak{}«. }\toendnotes[C]{\smallbreak}\pstart{}{\pb}Hrn\pend{}\pstart{}\textsc{Dr Rich Beer Hofmann}\pend{}\pstart{}\textsc{Wien\oindex{Wien@\textbf{Wien}, \emph{A.ADM2}|pw}}\pend{}\pstart{}\textsc{I Wollzeile 15\oindex{Wollzeile@\textbf{Wollzeile}, \emph{Straße (K.STR)}|pw}}\pend{}{\bigskip}\vspace{1em}
\pstart{}{\pb}Mein lieber Richard\pend\vspace{0.5em}
\pstart
           ich geh auf die \label{K_L00169-1v}\edtext{\textsc{Opernredoute}}{\lemma{\textnormal{\emph{Opernredoute}}}\Cendnote{\textnormal{Der Maskenball in der Oper, die
                  Opernredoute, fand am 1. 2. 1893 statt. Einlass war um
                     22 Uhr. Obzwar der zweite Poststempel eindeutig auf
                     1892 verweist und die »93« des ersten Poststempels
                  nicht mit letzter Sicherheit zu entziffern ist, sprechen mehrere Gründe dafür,
                  einen falsch eingestellten Stempel anzunehmen: 1892 war die
                  Opernredoute am 31. 1., Schnitzler besuchte sie nicht. Sowohl die Anwesenheit Beer-Hofmanns\pwindex{Beer-Hofmann, Richard 1866-07-11 – 1945-09-26@\textsc{Beer-Hofmann, Richard} (1866-07-11 – 1945-09-26), \emph{Schriftsteller/Schriftstellerin}|pwk} auf dem Ball 1893 als auch ein
                  Besuch Schnitzlers im Carl-Theaters\oindex{Carl-Theater@\textbf{Carl-Theater}, \emph{Theater (K.THE)}|pwk} lassen sich 1893
                  nachweisen.}}}\label{K_L00169-1}. Wollen Sie vorher mit mir ſoupiren? Oder ſich im Café mit mir
               treffen? –\pend
           \pstart Ihr \spacefill\mbox{Arthur}\pend{}
\pstart
           \noindent{}\label{T_L00169-1v}\edtext{Oder gehn}{\lemma{\textnormal{\emph{Oder gehn}}}\Cendnote{\textnormal{am rechten Rand}}}\label{T_L00169-1} Sie auch ins \label{K_L00169-2v}\edtext{Carlth\oindex{Carl-Theater@\textbf{Carl-Theater}, \emph{Theater (K.THE)}|pw}}{\lemma{\textnormal{\emph{Carlth}}}\Cendnote{\textnormal{Schnitzler besuchte die Aufführung von
                        \emph{Madame Mongodin}\pwindex{Madame Mongodin. Schwank in drei Akten@\emph{Madame Mongodin. Schwank in drei Akten}|pwk}, die um
                        7 Uhr begann.}}}\label{K_L00169-2}\pend
           
\pstart
           {\pb}(\textsc{I was \uline{not} on the \label{K_L00169-3v}\edtext{Weisse
                     Kreuz Ball}{\lemma{\textnormal{\emph{Weisse
                     Kreuz Ball}}}\Cendnote{\textnormal{Dieser hatte am
                        Vorabend, dem 31. 1. 1893 stattgefunden.}}}\label{K_L00169-3})}\pend
           \selectlanguage{ngerman}\endnumbering\briefempfaengerindex{Beer-Hofmann, Richard@\textsc{Beer-Hofmann, Richard}!zzzSchnitzler, Arthur@\emph{von Arthur Schnitzler}!1893-02-011@{1. 2. {[}1893?{]}}|)be}\mylabel{L00169h}  \normalsize

\doendnotes{C}
\bigskip
\vfill

\clearpage

\footnotesize

\lohead{\textsc{register}}

% Definiere theindex-Environment komplett neu ohne reledmac
\makeatletter
\renewenvironment{theindex}{%
  \section*{\indexname}%
  \setlength{\parindent}{0pt}%
  \setlength{\parskip}{0pt plus 0.3pt}%
  \let\item\@idxitem
}{%
  \clearpage
}
\makeatother

\IfFileExists{\jobname-pw.ind}{\input{\jobname-pw.ind}}{}

\end{document}

      