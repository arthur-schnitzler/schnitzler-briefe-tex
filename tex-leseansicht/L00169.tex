%% latex-leseansicht-vorspann.tex
%% Vorspann für die Leseansicht.
%% Lädt die gemeinsame Datei latex-vorspann.tex mit nicht gesetztem Schalter.

\newif\ifkorrekturansicht
\korrekturansichtfalse

\input{../tex-inputs/latex-vorspann}


\section[Arthur Schnitzler an Richard Beer-Hofmann, {[}1. 2. 1893?{]}]{L00169 Arthur Schnitzler an Richard Beer-Hofmann, {[}1. 2. 1893?{]}}
\nopagebreak\mylabel{L00169v}
\rehead{ }\normalsize\beginnumbering\briefempfaengerindex{Beer-Hofmann, Richard@\textsc{Beer-Hofmann, Richard}!zzzSchnitzler, Arthur@\emph{von Arthur Schnitzler}!1893-02-011@{1. 2. [1893?]}|(be}
\toendnotes[C]{\smallbreak\pagebreak[2]}
\correspDesc{Versand  durch Arthur Schnitzler am 1. 2. [1893?] in Wien
\newline{}Erhalt  durch Richard Beer-Hofmann am 1. 2. [1893?] in Wien}\toendnotes[C]{\smallbreak}
\Standort{YCGL, MSS 31.}
\physDesc{Kartenbrief, 230 Zeichen
\newline{}Handschrift: Bleistift, deutsche Kurrent
\newline{}Versand: 1) Rohrpost  2) Stempel: »\nobreak{}\oindex{Wien@\textbf{Wien}, \emph{Verwaltungsgebiet}|pwk}Wien, {[}1{]}\textcolor{gray}{II 93}, 9 15 V\nobreak{}«.  3) Stempel: »\nobreak{}\oindex{I., Innere Stadt@\textbf{I., Innere Stadt}, \emph{Verwaltungsgebiet}|pwk}Wien 1/1, 1 II 92, 10 10V\nobreak{}«. }\toendnotes[C]{\smallbreak}\pstart{}{\pb}Hrn\pend{}\pstart{}\textsc{Dr Rich Beer Hofmann}\pend{}\pstart{}\textsc{Wien\oindex{Wien@\textbf{Wien}, \emph{Verwaltungsgebiet}|pw}}\pend{}\pstart{}\textsc{I Wollzeile 15\oindex{Wien@\textbf{Wien}!I., Innere Stadt@\textbf{I., Innere Stadt}!Wollzeile 15 (»Berthahof«)@\textbf{Wollzeile 15 (»Berthahof«)}, \emph{Wohngebäude}|pw}}\pend{}{\bigskip}\vspace{1em}
\pstart{}{\pb}Mein lieber Richard\pend\vspace{0.5em}
\pstart
           ich geh auf die \label{K_L00169-1v}\edtext{\textsc{Opernredoute}}{\lemma{\textnormal{\emph{Opernredoute}}}\Cendnote{\textnormal{Der Maskenball in der Oper, die
                  Opernredoute, fand am 1. 2. 1893 statt. Einlass war um
                     22 Uhr. Obzwar der zweite Poststempel eindeutig auf
                     1892 verweist und die »93« des ersten Poststempels
                  nicht mit letzter Sicherheit zu entziffern ist, sprechen mehrere Gründe dafür,
                  einen falsch eingestellten Stempel anzunehmen: 1892 war die
                  Opernredoute am 31. 1., Schnitzler besuchte sie nicht. Sowohl die Anwesenheit Beer-Hofmanns\pwindex{Beer-Hofmann, Richard 11.\,7.\,1866 Wien – 26.\,9.\,1945 New York City@\textsc{Beer-Hofmann, Richard} (11.\,7.\,1866 Wien – 26.\,9.\,1945 New York City), \emph{Schriftsteller}|pwk} auf dem Ball 1893 als auch ein
                  Besuch Schnitzlers im Carl-Theaters\oindex{Wien@\textbf{Wien}!II., Leopoldstadt@\textbf{II., Leopoldstadt}!Carl-Theater@\textbf{Carl-Theater}, \emph{Theater}|pwk} lassen sich 1893
                  nachweisen.}}}\label{K_L00169-1}. Wollen Sie vorher mit mir{ }ſoupiren? Oder{ }ſich im Café mit mir
               treffen? –\pend
           \pstart Ihr \spacefill\mbox{Arthur}\pend{}
\pstart
           \noindent{}\label{T_L00169-1v}\edtext{Oder gehn}{\lemma{\textnormal{\emph{Oder gehn}}}\Cendnote{\textnormal{am rechten Rand}}}\label{T_L00169-1} Sie auch ins \label{K_L00169-2v}\edtext{Carlth\oindex{Wien@\textbf{Wien}!II., Leopoldstadt@\textbf{II., Leopoldstadt}!Carl-Theater@\textbf{Carl-Theater}, \emph{Theater}|pw}}{\lemma{\textnormal{\emph{Carlth}}}\Cendnote{\textnormal{Schnitzler besuchte die Aufführung von
                        \emph{Madame Mongodin}\pwindex{\textcolor{red}{\textsuperscript{XXXX indx1}}!Madame Mongodin. Schwank in drei Akten@\strich\emph{Madame Mongodin. Schwank in drei Akten}|pwk}\pwindex{\textcolor{red}{\textsuperscript{XXXX indx1}}!Madame Mongodin. Schwank in drei Akten@\strich\emph{Madame Mongodin. Schwank in drei Akten}|pwk}, die um
                        7 Uhr begann.}}}\label{K_L00169-2}\pend
           
\pstart
           {\pb}(\textsc{I was \uline{not} on the \label{K_L00169-3v}\edtext{Weisse
                     Kreuz Ball}{\lemma{\textnormal{\emph{Weisse
                     Kreuz Ball}}}\Cendnote{\textnormal{Dieser hatte am
                        Vorabend, dem 31. 1. 1893 stattgefunden.}}}\label{K_L00169-3})}\pend
           \selectlanguage{ngerman}\endnumbering\briefempfaengerindex{Beer-Hofmann, Richard@\textsc{Beer-Hofmann, Richard}!zzzSchnitzler, Arthur@\emph{von Arthur Schnitzler}!1893-02-011@{1. 2. [1893?]}|)be}\mylabel{L00169h}  \newcommand{\dateiname}{L00169}\newcommand{\titel}{Arthur Schnitzler an Richard Beer-Hofmann, [1. 2. 1893?]}\newcommand{\editorInnen}{Martin Anton Müller und Gerd-Hermann Susen}%% latex-leseansicht-abspann.tex
%% Abspann für die Leseansicht.
%% Der Schalter \ifkorrekturansicht ist bereits durch den Vorspann gesetzt.

%% latex-abspann.tex
%% Gemeinsamer Abspann für Korrekturansicht und Leseansicht.
%% Setzt den Schalter \ifkorrekturansicht voraus (gesetzt in den
%% einbindenden Dateien latex-korrekturansicht-abspann.tex bzw.
%% latex-leseansicht-abspann.tex).
%% ---------------------------------------------------------------

\normalsize

% Das esempio-Environment wird nur in der Leseansicht benötigt
\ifkorrekturansicht\else
\newenvironment{esempio}[3]%
{
    \vspace{1.5ex}
    \rlap{\underline{#1}}
    \par
    \setlength{\parindent}{0cm}
    \nopagebreak
    \leftskip=#2cm
    \rightskip=#3cm
}
{
    \par
}
\fi

\doendnotes{C}
\bigskip
\vfill

\clearpage

\footnotesize

\ifkorrekturansicht
  \lohead{\textsc{register}}
\fi

% theindex-Environment neu definieren ohne reledmac
\makeatletter
\renewenvironment{theindex}{%
  \ifkorrekturansicht
    \section*{\indexname}%
  \else
    \subsubsection*{Index der erwähnten Entitäten}%
  \fi
  \setlength{\parindent}{0pt}%
  \setlength{\parskip}{0pt plus 0.3pt}%
  \let\item\@idxitem
}{%
  \ifkorrekturansicht\clearpage\fi
}
\makeatother

\IfFileExists{\jobname-pw.ind}{\input{\jobname-pw.ind}}{}

% Quellenangabe nur in der Leseansicht
\ifkorrekturansicht\else
% Fallback-Definitionen, falls die .tex-Datei \titel etc. nicht gesetzt hat
\providecommand{\titel}{}
\providecommand{\editorInnen}{}
\providecommand{\dateiname}{\jobname}

\vspace{3cm}

\vfill

\footnotesize
\textsc{Quelle}: \titel. Herausgegeben von {\editorInnen}. In: \emph{Arthur Schnitzler: Briefwechsel mit Autorinnen und Autoren}.
 Digitale Edition, https://schnitzler-briefe.acdh.oeaw.ac.at/{\dateiname}.html (Stand \today)
\fi

\end{document}


