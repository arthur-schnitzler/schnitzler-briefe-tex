%% latex-korrekturansicht-vorspann.tex
%% Vorspann für die Korrekturansicht.
%% Lädt die gemeinsame Datei latex-vorspann.tex mit gesetztem Schalter.

\newif\ifkorrekturansicht
\korrekturansichttrue

\input{../tex-inputs/latex-vorspann}


\section[ Paul Goldmann an Arthur Schnitzler, 31. 12. 1910]{L03473 Paul Goldmann an Arthur Schnitzler, 31. 12. 1910}
\nopagebreak\mylabel{L03473v}
\rehead{ }\normalsize\beginnumbering\briefempfaengerindex{Schnitzler, Arthur@\textsc{Schnitzler, Arthur}!zzzGoldmann, Paul@\emph{von Paul Goldmann}!1910-12-312@{31. 12. 1910}|(be}
\toendnotes[C]{\smallbreak\pagebreak[2]}\Standort{DLA, A:Schnitzler, HS.NZ85.1.3175.}
\physDesc{Brief, 1 Blatt, 1 Seite, 244 Zeichen
\newline{}Handschrift: schwarze Tinte, deutsche Kurrent}\toendnotes[C]{\smallbreak}
\pstart
           \centering{}{\pb}\textcolor{gray}{\textbf{Hotel Sacher\oindex{Hotel Sacher@\textbf{Hotel Sacher}, \emph{Hotel (K.HTL)}|pw}}}\pend
           
\pstart
           \textcolor{gray}{\textbf{Telefon Nr 8008.}}\hfill \textcolor{gray}{\textbf{Wien I.\oindex{Wien@\textbf{Wien}, \emph{A.ADM2}|pw}}}\pend
           
\pstart
           31. 12. 10.\pend
           
\pstart{}Lieber Freund,\pend\vspace{0.5em}
\pstart
           Ich danke Dir herzlichſt für die Überſendung der Exemplars von Deinem \label{K_L03473-1v}\edtext{Drama\pwindex{junge Medardus. Dramatische Historie in einem Vorspiel und fuenf Aufzuegen@\emph{Der junge Medardus. Dramatische Historie in einem Vorspiel und fünf Aufzügen}|pwv}}{\lemma{\textnormal{\emph{Drama}}}\Cendnote{\textnormal{Arthur Schnitzler: 
                     \emph{Der junge Medardus. Dramatische Historie in einem Vorspiel und fünf Aufzügen}\pwindex{junge Medardus. Dramatische Historie in einem Vorspiel und fuenf Aufzuegen@\emph{Der junge Medardus. Dramatische Historie in einem Vorspiel und fünf Aufzügen}|pwk}. 
                     Berlin: \emph{S. Fischer}\orgindex{S. Fischer Verlag@S. Fischer Verlag|pwk}{ }1910. Am
                  26. 10. 1910 wurde das Buch vom \emph{Börsenblatt für den deutschen Buchhandel}\pwindex{Boersenblatt fuer den Deutschen Buchhandel@\emph{Börsenblatt für den Deutschen Buchhandel}|pwk} als
                  Neuerscheinung gemeldet.}}}\label{K_L03473-1}, habe mich aufrichtig darüber gefreut u. wünſche Dir, Deiner Frau\pwindex{Schnitzler, Olga 17.01.1882 – 13.01.1970@\textsc{Schnitzler, Olga} (17.01.1882 – 13.01.1970), \emph{Schauspieler/Schauspielerin, Sänger/Sängerin}|pwv} u. Deinen Kindern\pwindex{Schnitzler, Heinrich 09.08.1902 – 12.07.1982@\textsc{Schnitzler, Heinrich} (09.08.1902 – 12.07.1982), \emph{Regisseur/Regisseurin, Schauspieler/Schauspielerin}|pwv}\pwindex{Cappellini, Lili 13.09.1909 – 26.07.1928@\textsc{Cappellini, Lili} (13.09.1909 – 26.07.1928)|pwv} ein
               glückliches neues Jahr. – Mit vielen Grüßen {\\}Dein {\\}\spacefill\mbox{Paul Goldmann.}\pend
           \selectlanguage{ngerman}\endnumbering\briefempfaengerindex{Schnitzler, Arthur@\textsc{Schnitzler, Arthur}!zzzGoldmann, Paul@\emph{von Paul Goldmann}!1910-12-312@{31. 12. 1910}|)be}\mylabel{L03473h}  \normalsize

\doendnotes{C}
\bigskip
\vfill

\clearpage

\footnotesize

\lohead{\textsc{register}}

% Definiere theindex-Environment komplett neu ohne reledmac
\makeatletter
\renewenvironment{theindex}{%
  \section*{\indexname}%
  \setlength{\parindent}{0pt}%
  \setlength{\parskip}{0pt plus 0.3pt}%
  \let\item\@idxitem
}{%
  \clearpage
}
\makeatother

\IfFileExists{\jobname-pw.ind}{\input{\jobname-pw.ind}}{}

\end{document}

      