%% latex-korrekturansicht-vorspann.tex
%% Vorspann für die Korrekturansicht.
%% Lädt die gemeinsame Datei latex-vorspann.tex mit gesetztem Schalter.

\newif\ifkorrekturansicht
\korrekturansichttrue

\input{../tex-inputs/latex-vorspann}


\section[ Arthur Schnitzler an Felix Salten, 20. 10. 1911]{L02949 Arthur Schnitzler an Felix Salten, 20. 10. 1911}
\nopagebreak\mylabel{L02949v}
\rehead{ }\normalsize\beginnumbering\briefempfaengerindex{Salten, Felix@\textsc{Salten, Felix}!zzzSchnitzler, Arthur@\emph{von Arthur Schnitzler}!1911-10-201@{20. 10. 1911}|(be}
\toendnotes[C]{\smallbreak\pagebreak[2]}\Standort{DLA, A:Schnitzler, HS.NZ85.1.1751.}
\physDesc{Brief, maschinenschriftliche Abschrift1 Blatt, 1 Seite, 1717 Zeichen
\newline{}maschinell
\newline{}Ordnung: mit schwarzer Tinte Vermerk »\textsc{Salten}« }
\buchAbdrucke{\weitereDrucke{Arthur Schnitzler: \emph{Briefe 1875–1912}. Frankfurt am Main: \emph{S. Fischer} 1981, S. 675–676.} }\toendnotes[C]{\smallbreak}
\pstart
           \raggedleft{}{\pb}Wien\oindex{Wien@\textbf{Wien}, \emph{A.ADM2}|pw}, 20. 10. 1911.\pend
           
\pstart{}Lieber,\pend\vspace{0.5em}
\pstart
           Ihre zwei \label{K_L02949-1v}\edtext{Feuilletons\pwindex{Burgtheater (»Das weite Land«, Tragikomoedie in fuenf Akten von Arthur Schnitzler. – Zum erstenmal am 14. Oktober 1911)@\emph{Burgtheater (»Das weite Land«, Tragikomödie in fünf Akten von Arthur Schnitzler. – Zum erstenmal am 14. Oktober 1911)}|pwv}\pwindex{Burgtheater. »Das weite Land.« Tragikomoedie von Arthur Schnitzler@\emph{Burgtheater. »Das weite Land.« Tragikomödie von Arthur Schnitzler}|pwv}}{\lemma{\textnormal{\emph{Feuilletons}}}\Cendnote{\textnormal{Felix Salten\pwindex{Salten, Felix 06.09.1869 – 08.10.1945@\textsc{Salten, Felix} (06.09.1869 – 08.10.1945), \emph{Schriftsteller/Schriftstellerin, Journalist/Journalistin, Chefredakteur/Chefredakteurin}|pwk}: \emph{Burgtheater (»Das weite Land«, Tragikomödie in fünf Akten
                        von Arthur Schnitzler. – Zum erstenmal am 14. Oktober 1911)}\pwindex{Burgtheater (»Das weite Land«, Tragikomoedie in fuenf Akten von Arthur Schnitzler. – Zum erstenmal am 14. Oktober 1911)@\emph{Burgtheater (»Das weite Land«, Tragikomödie in fünf Akten von Arthur Schnitzler. – Zum erstenmal am 14. Oktober 1911)}|pwk}. In: \emph{Die Zeit}\pwindex{Zeit@\emph{Die Zeit}|pwk}, Jg. 10, Nr. 3254, 15. 11. 1911, S. 1–3; Felix Salten\pwindex{Salten, Felix 06.09.1869 – 08.10.1945@\textsc{Salten, Felix} (06.09.1869 – 08.10.1945), \emph{Schriftsteller/Schriftstellerin, Journalist/Journalistin, Chefredakteur/Chefredakteurin}|pwk}: \emph{Burgtheater. »Das weite Land.« Tragikomödie von Arthur Schnitzler}\pwindex{Burgtheater. »Das weite Land.« Tragikomoedie von Arthur Schnitzler@\emph{Burgtheater. »Das weite Land.« Tragikomödie von Arthur Schnitzler}|pwk}. In:
                        \emph{Pester Lloyd}\pwindex{Pester Lloyd@\emph{Pester Lloyd}|pwk}, Jg. 58, Nr. 246, 17. 11. 1911, Morgenblatt, S. 1–2.}}}\label{K_L02949-1}
               sind – muss man es erst sagen – sehr schoen. In Hinsicht auf sehr Wesentliches aber
               bin ich voellig anderer Ansicht, muss es sein, nicht nur weil ich das Stueck\pwindex{weite Land. Tragikomoedie in fuenf Akten@\emph{Das weite Land. Tragikomödie in fünf Akten}|pwv} geschrieben habe, sondern weil ich
               zu der ganzen Frage der ethischen Werturteile, ueber Figuren innerhalb von
               Kunstwerken offenbar anders stehe wie Sie.\pend
           
\pstart
           Darf ich Ihnen ein verwunderliches Missverstaendnis aufklaeren{[},{]}
               das Ihr Feuilleton\pwindex{Burgtheater. »Das weite Land.« Tragikomoedie von Arthur Schnitzler@\emph{Burgtheater. »Das weite Land.« Tragikomödie von Arthur Schnitzler}|pwv} im »Lloyd\pwindex{Pester Lloyd@\emph{Pester Lloyd}|pw}« enthaelt? Hofreiter\pwindex{weite Land. Tragikomoedie in fuenf Akten@\emph{Das weite Land. Tragikomödie in fünf Akten}|pwv} denkt nicht daran am Schluss des
                  Stueck\pwindex{weite Land. Tragikomoedie in fuenf Akten@\emph{Das weite Land. Tragikomödie in fünf Akten}|pwv}es »ein braver Kindesvater\pwindex{Burgtheater. »Das weite Land.« Tragikomoedie von Arthur Schnitzler@\emph{Burgtheater. »Das weite Land.« Tragikomödie von Arthur Schnitzler}|pwv}« zu werden, so
               wenig ich daran gedacht habe, das irgendwen glauben zu machen. Und es liegt nicht der
               leiseste Grund vor{[},{]} mir so etwas, was wirklich eine Banalitaet waere, zuzumuten.
               (Ausser bei Ihnen habe ich diese Zumutung nur unter Dutzenden ein einziges Mal
               gefunden). Erinnern Sie sich nur: Genia\pwindex{weite Land. Tragikomoedie in fuenf Akten@\emph{Das weite Land. Tragikomödie in fünf Akten}|pwv} in ihrem letzten Gespraech mit Hofreiter\pwindex{weite Land. Tragikomoedie in fuenf Akten@\emph{Das weite Land. Tragikomödie in fünf Akten}|pwv} besinnt sich ploetzlich: {[}»{]}Percy kommt\pwindex{weite Land. Tragikomoedie in fuenf Akten@\emph{Das weite Land. Tragikomödie in fünf Akten}|pwv}«. Darauf er: »Den erwart ich noch – denn die
                  Andern – na! (Handbewegung)\pwindex{weite Land. Tragikomoedie in fuenf Akten@\emph{Das weite Land. Tragikomödie in fünf Akten}|pwv}«. Er ist also jedenfalls entschlossen ihn zu
               erwarten; und dass er dann, wenn die Stimme Percys\pwindex{weite Land. Tragikomoedie in fuenf Akten@\emph{Das weite Land. Tragikomödie in fünf Akten}|pwv} im Garten toent, so weit bewegt ist (gerade in der Empfindung: nun
               ist das auch zu Ende), um leise aufzuwimmern, \label{T_L02949-1v}\edtext{das}{\lemma{\textnormal{\emph{das}}}\Cendnote{\textnormal{In der Vorlage steht »dass«.}}}\label{T_L02949-1} ist meines Erachtens kein Anlass
               zu vermuten, dass damit eine Art innerer Umkehr eingeleitet oder angedeutet sein
               sollte. Ich war himmelweit davon entfernt ein solches
                  Missverstae{[}n{]}dnis auch nur fuer moeglich zu halten. (Sonst
               haette ich Hofreiter\pwindex{weite Land. Tragikomoedie in fuenf Akten@\emph{Das weite Land. Tragikomödie in fünf Akten}|pwv} am
               Schlusse ausrufen lassen: »Nun auf nach Amerika\oindex{Amerika@\textbf{Amerika}, \emph{kein passender Code gefunden}|pw}«). \pend
           
\pstart
           Naechsten fahre ich ueber Prag\oindex{Prag@\textbf{Prag}, \emph{A.ADM1}|pw}, Dresden\oindex{Dresden@\textbf{Dresden}, \emph{P.PPLA}|pw}{ }\label{K_L02949-2v}\edtext{nach Berlin\oindex{Berlin@\textbf{Berlin}, \emph{P.PPLC}|pw} und Hamburg\oindex{Hamburg@\textbf{Hamburg}, \emph{P.PPLA}|pw}}{\lemma{\textnormal{\emph{nach Berlin und Hamburg}}}\Cendnote{\textnormal{In Berlin\oindex{Berlin@\textbf{Berlin}, \emph{P.PPLC}|pwk} kam Schnitzler am 2. 11. 1911 an. Am
                     5. 11. 1911
                  reiste er weiter nach Hamburg\oindex{Hamburg@\textbf{Hamburg}, \emph{P.PPLA}|pwk}, wo er bis zum
                     9. 11. 1911
                  blieb.}}}\label{K_L02949-2}, dort »Beatrice\pwindex{Schleier der Beatrice. Schauspiel in fuenf Akten@\emph{Der Schleier der Beatrice. Schauspiel in fünf Akten}|pw}«, »Weites Land\pwindex{weite Land. Tragikomoedie in fuenf Akten@\emph{Das weite Land. Tragikomödie in fünf Akten}|pw}«, »Anatol\pwindex{Anatol@\emph{Anatol}|pw}« zu sehen. Wann ist die \label{K_L02949-3v}\edtext{Dagobert\pwindex{gute Koenig Dagobert. Lustspiel in vier Aufzuegen@\emph{Der gute König Dagobert. Lustspiel in vier Aufzügen}|pw}-Generalprobe}{\lemma{\textnormal{\emph{Dagobert-Generalprobe}}}\Cendnote{\textnormal{Salten\pwindex{Salten, Felix 06.09.1869 – 08.10.1945@\textsc{Salten, Felix} (06.09.1869 – 08.10.1945), \emph{Schriftsteller/Schriftstellerin, Journalist/Journalistin, Chefredakteur/Chefredakteurin}|pwk} hatte das Stück \emph{Le Bon Roi Dagobert}\pwindex{Le Bon Roi Dagobert@\emph{Le Bon Roi Dagobert}|pwk} von André Rivoire\pwindex{Rivoire, Andre 05.05.1872 – 19.08.1930@\textsc{Rivoire, André} (05.05.1872 – 19.08.1930), \emph{Schriftsteller/Schriftstellerin}|pwk} auf Deutsch bearbeitet. Die Uraufführung hatte die Übersetzung\pwindex{gute Koenig Dagobert. Lustspiel in vier Aufzuegen@\emph{Der gute König Dagobert. Lustspiel in vier Aufzügen}|pwkv} am 19. 1. 1910 am \emph{Deutschen Theater}\orgindex{Deutsches Theater Berlin@Deutsches Theater Berlin|pwk} in Berlin\oindex{Berlin@\textbf{Berlin}, \emph{P.PPLC}|pwk} erlebt.
                  In Wien\oindex{Wien@\textbf{Wien}, \emph{A.ADM2}|pwk} fand die Premiere am 18. 11. 1911 am \emph{Deutschen Volkstheater}\orgindex{Volkstheater@Volkstheater|pwk} statt, die Generalprobe am Vortag. Schnitzler besuchte erst
                  die Aufführung am 5. 12. 1911.}}}\label{K_L02949-3}? Darf man ihr beiwohnen?\pend
           
\pstart
           Auf baldiges Wiedersehen. {\\[\baselineskip]}herzlichst Ihr {\\[\baselineskip]}\pend
           \leftskip=0em{}{\vspace{1\baselineskip}}
\pstart
           \noindent{}Felix Salten\pend
           
\pstart
           (Weites Land\pwindex{weite Land. Tragikomoedie in fuenf Akten@\emph{Das weite Land. Tragikomödie in fünf Akten}|pw})\pend
           \selectlanguage{ngerman}\endnumbering\briefempfaengerindex{Salten, Felix@\textsc{Salten, Felix}!zzzSchnitzler, Arthur@\emph{von Arthur Schnitzler}!1911-10-201@{20. 10. 1911}|)be}\mylabel{L02949h}  \normalsize

\doendnotes{C}
\bigskip
\vfill

\clearpage

\footnotesize

\lohead{\textsc{register}}

% Definiere theindex-Environment komplett neu ohne reledmac
\makeatletter
\renewenvironment{theindex}{%
  \section*{\indexname}%
  \setlength{\parindent}{0pt}%
  \setlength{\parskip}{0pt plus 0.3pt}%
  \let\item\@idxitem
}{%
  \clearpage
}
\makeatother

\IfFileExists{\jobname-pw.ind}{\input{\jobname-pw.ind}}{}

\end{document}

      