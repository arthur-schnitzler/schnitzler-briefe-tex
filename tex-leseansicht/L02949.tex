%% latex-leseansicht-vorspann.tex
%% Vorspann für die Leseansicht.
%% Lädt die gemeinsame Datei latex-vorspann.tex mit nicht gesetztem Schalter.

\newif\ifkorrekturansicht
\korrekturansichtfalse

\input{../tex-inputs/latex-vorspann}

\begin{center}
            \textcolor{red}{ENTWURF, NICHT FERTIG KORRIGIERT}
                      \end{center}
            
         
         \renewcommand{\erwaehntePersonen}{Personen: André Rivoire, Felix Salten}
         \renewcommand{\erwaehnteInstitutionen}{Institutionen: Deutsches Theater Berlin}
         \renewcommand{\erwaehnteOrte}{Orte: Amerika, Berlin, Dresden, Hamburg, Prag, Volkstheater, Wien}
         \renewcommand{\erwaehnteWerke}{Werke: Anatol, Burgtheater (»Das weite Land«, Tragikkomödie in fünf Akten von Arthur Schnitzler. – Zum erstenmal am 14. Oktober 1911), Burgtheater. »Das weite Land.« Tragikomödie von Arthur Schnitzler, Das weite Land. Tragikomödie in fünf Akten, Der Schleier der Beatrice. Schauspiel in fünf Akten, Der gute König Dagobert. Lustspiel in vier Aufzügen, Die Zeit, Le Bon Roi Dagobert, Pester Lloyd}
               \section[Arthur Schnitzler an Felix Salten, 20. 10. 1911]{ Arthur Schnitzler an Felix Salten, 20. 10. 1911}\nopagebreak\mylabel{v}\rehead{ }\begin{ledgroupsized}[t]{13cm}\normalsize\beginnumbering \toendnotes[C]{\smallbreak\pagebreak[2]} \Standort{DLA, A:Schnitzler, HS.NZ85.1.1751.}
\physDesc{Brief, Maschinenschriftliche Abschrift, 1 Blatt, 1 Seite, 1713 Zeichen
\newline{}maschinell
\newline{}Ordnung: mit schwarzer Tinte Vermerk »Salten« }\buchAbdrucke{\weitereDrucke{Arthur Schnitzler: \emph{Briefe 1875–1912}. Hg. Therese Nickl und Heinrich Schnitzler. Frankfurt am Main: \emph{S. Fischer} 1981, S. 675–676.} }\toendnotes[C]{\smallbreak}\pstart
           \raggedleft{}{\pb}Wien\oindex{Wien@\textbf{Wien}|pw}, 20. 10. 1911. \pend
           \pstart{}Lieber,\pend\pstart
           Ihre zwei \label{K_L02949-1v}\edtext{Feuilletons\pwindex{Salten, Felix 06.09.1869 – 08.10.1945@\textsc{Salten, Felix} (06.09.1869 – 08.10.1945), \emph{Schriftsteller, Journalist}!Burgtheater (»Das weite Land«, Tragikkomoedie in fuenf Akten von Arthur
                  Schnitzler. – Zum erstenmal am 14. Oktober 1911)15. 10. 1911@\strich\emph{Burgtheater (»Das weite Land«, Tragikkomödie in fünf Akten von Arthur Schnitzler. – Zum erstenmal am 14. Oktober 1911)} {[}15. 10. 1911{]}|pwv}\pwindex{Salten, Felix 06.09.1869 – 08.10.1945@\textsc{Salten, Felix} (06.09.1869 – 08.10.1945), \emph{Schriftsteller, Journalist}!Burgtheater. »Das weite Land.« Tragikomoedie von Arthur Schnitzler1911-11-17@\strich\emph{Burgtheater. »Das weite Land.« Tragikomödie von Arthur Schnitzler} {[}1911-11-17{]}|pwv}}{\lemma{\textnormal{\emph{Feuilletons}}}\Cendnote{\textnormal{Felix Salten\pwindex{Salten, Felix 06.09.1869 – 08.10.1945@\textsc{Salten, Felix} (06.09.1869 – 08.10.1945), \emph{Schriftsteller, Journalist}|pwk}: \emph{Burgtheater (»Das weite Land«, Tragikkomödie in fünf Akten
                        von Arthur Schnitzler. – Zum erstenmal am 14. Oktober 1911)}\pwindex{Salten, Felix 06.09.1869 – 08.10.1945@\textsc{Salten, Felix} (06.09.1869 – 08.10.1945), \emph{Schriftsteller, Journalist}!Burgtheater (»Das weite Land«, Tragikkomoedie in fuenf Akten von Arthur
                  Schnitzler. – Zum erstenmal am 14. Oktober 1911)15. 10. 1911@\strich\emph{Burgtheater (»Das weite Land«, Tragikkomödie in fünf Akten von Arthur Schnitzler. – Zum erstenmal am 14. Oktober 1911)} {[}15. 10. 1911{]}|pwk}. In: \emph{Die Zeit}\pwindex{Zeit1902-09-27 – 1919@\emph{Die Zeit} {[}1902-09-27 – 1919{]}|pwk}, Jg. 10, Nr. 3254,
                        15. 11. 1911, S. 1–3. Felix Salten\pwindex{Salten, Felix 06.09.1869 – 08.10.1945@\textsc{Salten, Felix} (06.09.1869 – 08.10.1945), \emph{Schriftsteller, Journalist}|pwk}:
                     \emph{Burgtheater. »Das weite Land.« Tragikomödie
                        von Arthur Schnitzler}\pwindex{Salten, Felix 06.09.1869 – 08.10.1945@\textsc{Salten, Felix} (06.09.1869 – 08.10.1945), \emph{Schriftsteller, Journalist}!Burgtheater. »Das weite Land.« Tragikomoedie von Arthur Schnitzler1911-11-17@\strich\emph{Burgtheater. »Das weite Land.« Tragikomödie von Arthur Schnitzler} {[}1911-11-17{]}|pwk}. In: \emph{Pester
                        Lloyd}\pwindex{?? Werk@Nicht ermittelte Verfasserinnen und Verfasser!Pester LloydNone@\emph{Pester Lloyd} {[}None{]}|pwk}, Jg. 58, Nr. 246, 17. 11. 1911, Morgenblatt,
                     S. 1–2.}}}\label{K_L02949-1h} sind – muss man es erst sagen – sehr schoen. In Hinsicht
               auf sehr Wesentliches aber bin ich voellig anderer Ansicht, muss es sein, nicht nur
               weil ich das Stueck geschrieben habe, sondern weil ich zu der ganzen Frage der
               ethischen Werturteile, ueber Figuren innerhalb von Kunstwerken offenbar anders stehe
               wie Sie.\pend
           \pstart
           Darf ich Ihnen ein verwunderliches Missverstaendnis aufklaeren das Ihr Feuilleton\pwindex{Salten, Felix 06.09.1869 – 08.10.1945@\textsc{Salten, Felix} (06.09.1869 – 08.10.1945), \emph{Schriftsteller, Journalist}!Burgtheater. »Das weite Land.« Tragikomoedie von Arthur Schnitzler1911-11-17@\strich\emph{Burgtheater. »Das weite Land.« Tragikomödie von Arthur Schnitzler} {[}1911-11-17{]}|pwv} im Lloyd\pwindex{?? Werk@Nicht ermittelte Verfasserinnen und Verfasser!Pester LloydNone@\emph{Pester Lloyd} {[}None{]}|pw} enthielt? Hofreiter\pwindex{Schnitzler, Arthur 15.05.1862 – 21.10.1931@\textsc{Schnitzler, Arthur} (15.05.1862 – 21.10.1931), \emph{Schriftsteller, Mediziner}!weite Land. Tragikomoedie in fuenf Akten1910-10-20@\strich\emph{Das weite Land. Tragikomödie in fünf Akten} {[}1910-10-20{]}|pw} denkt nicht daran am Schluss des Stueckes\pwindex{Schnitzler, Arthur 15.05.1862 – 21.10.1931@\textsc{Schnitzler, Arthur} (15.05.1862 – 21.10.1931), \emph{Schriftsteller, Mediziner}!weite Land. Tragikomoedie in fuenf Akten1910-10-20@\strich\emph{Das weite Land. Tragikomödie in fünf Akten} {[}1910-10-20{]}|pwv} »ein braver Kindesvater« zu werden, so wenig ich
               daran gedacht habe, das irgendwen glauben zu machen. Und es liegt nicht der leiseste
               Grund vor mir so etwas, was wirklich eine Banalitaet waere, zuzumuten. (Ausser bei
               Ihnen habe ich diese Zumutung nur unter Dutzenden ein einziges Mal gefunden).
               Erinnern Sie sich nur: Genia\pwindex{Schnitzler, Arthur 15.05.1862 – 21.10.1931@\textsc{Schnitzler, Arthur} (15.05.1862 – 21.10.1931), \emph{Schriftsteller, Mediziner}!weite Land. Tragikomoedie in fuenf Akten1910-10-20@\strich\emph{Das weite Land. Tragikomödie in fünf Akten} {[}1910-10-20{]}|pwv}
               in ihrem letzten Gespraech mit Hofreiter\pwindex{Schnitzler, Arthur 15.05.1862 – 21.10.1931@\textsc{Schnitzler, Arthur} (15.05.1862 – 21.10.1931), \emph{Schriftsteller, Mediziner}!weite Land. Tragikomoedie in fuenf Akten1910-10-20@\strich\emph{Das weite Land. Tragikomödie in fünf Akten} {[}1910-10-20{]}|pwv} besinnt sich ploetzlich: {[}»{]}Percy kommt\pwindex{Schnitzler, Arthur 15.05.1862 – 21.10.1931@\textsc{Schnitzler, Arthur} (15.05.1862 – 21.10.1931), \emph{Schriftsteller, Mediziner}!weite Land. Tragikomoedie in fuenf Akten1910-10-20@\strich\emph{Das weite Land. Tragikomödie in fünf Akten} {[}1910-10-20{]}|pw}«. Darauf er: »Den erwart ich noch – denn die Andern – na!
                  (Handbewegung)\pwindex{Schnitzler, Arthur 15.05.1862 – 21.10.1931@\textsc{Schnitzler, Arthur} (15.05.1862 – 21.10.1931), \emph{Schriftsteller, Mediziner}!weite Land. Tragikomoedie in fuenf Akten1910-10-20@\strich\emph{Das weite Land. Tragikomödie in fünf Akten} {[}1910-10-20{]}|pwv}«. Er ist also jedenfalls entschlossen ihn zu erwarten; und
               dass er dann, wenn die Stimme Percys\pwindex{Schnitzler, Arthur 15.05.1862 – 21.10.1931@\textsc{Schnitzler, Arthur} (15.05.1862 – 21.10.1931), \emph{Schriftsteller, Mediziner}!weite Land. Tragikomoedie in fuenf Akten1910-10-20@\strich\emph{Das weite Land. Tragikomödie in fünf Akten} {[}1910-10-20{]}|pw} im Garten
               toent, so weit bewegt ist (gerade in der Empfindung: nun ist das auch zu Ende), um
               leise aufzuwimmern, dass ist meines Erachtens kein Anlass zu vermuten, dass damit
               eine Art innerer Umkehr eingeleitet oder angedeutet sein sollte. Ich war himmelweit
               davon entfernt ein solches Missverstae{[}n{]}dnis auch nur fuer
               moeglich zu halten. (Sonst hatte ich Hofreiter\pwindex{Schnitzler, Arthur 15.05.1862 – 21.10.1931@\textsc{Schnitzler, Arthur} (15.05.1862 – 21.10.1931), \emph{Schriftsteller, Mediziner}!weite Land. Tragikomoedie in fuenf Akten1910-10-20@\strich\emph{Das weite Land. Tragikomödie in fünf Akten} {[}1910-10-20{]}|pw}
               am Schlusse ausrufen lassen: »Nun auf nach Amerika\oindex{Amerika@\textbf{Amerika}|pw}«). \pend
           \pstart
           Naechsten fahre ich ueber Prag\oindex{Prag@\textbf{Prag}|pw}, Dresden\oindex{Dresden@\textbf{Dresden}|pw} nach Berlin\oindex{Berlin@\textbf{Berlin}|pw} und Hamburg\oindex{Hamburg@\textbf{Hamburg}|pw}, dort »Beatrice\pwindex{Schnitzler, Arthur 15.05.1862 – 21.10.1931@\textsc{Schnitzler, Arthur} (15.05.1862 – 21.10.1931), \emph{Schriftsteller, Mediziner}!Schleier der Beatrice. Schauspiel in fuenf Akten1900-12-01@\strich\emph{Der Schleier der Beatrice. Schauspiel in fünf Akten} {[}1900-12-01{]}|pw}«, »Weites Land\pwindex{Schnitzler, Arthur 15.05.1862 – 21.10.1931@\textsc{Schnitzler, Arthur} (15.05.1862 – 21.10.1931), \emph{Schriftsteller, Mediziner}!weite Land. Tragikomoedie in fuenf Akten1910-10-20@\strich\emph{Das weite Land. Tragikomödie in fünf Akten} {[}1910-10-20{]}|pw}«, »Anatol\pwindex{Schnitzler, Arthur 15.05.1862 – 21.10.1931@\textsc{Schnitzler, Arthur} (15.05.1862 – 21.10.1931), \emph{Schriftsteller, Mediziner}!Anatol1892-10-29@\strich\emph{Anatol} {[}1892-10-29{]}|pw}« zu sehen. Wann ist die \label{K_L02949-2v}\edtext{Dagobert-Generalprobe\pwindex{Rivoire, Andre 05.05.1872 – 19.08.1930@\textsc{Rivoire, André} (05.05.1872 – 19.08.1930), \emph{Schriftsteller}!gute Koenig Dagobert. Lustspiel in vier Aufzuegen1910@\strich\emph{Der gute König Dagobert. Lustspiel in vier Aufzügen} {[}1910{]}|pw}}{\lemma{\textnormal{\emph{Dagobert-Generalprobe}}}\Cendnote{\textnormal{Salten\pwindex{Salten, Felix 06.09.1869 – 08.10.1945@\textsc{Salten, Felix} (06.09.1869 – 08.10.1945), \emph{Schriftsteller, Journalist}|pwk} hatte das Stück \emph{Le Bon Roi Dagobert}\pwindex{Rivoire, Andre 05.05.1872 – 19.08.1930@\textsc{Rivoire, André} (05.05.1872 – 19.08.1930), \emph{Schriftsteller}!Le Bon Roi Dagobert1908@\strich\emph{Le Bon Roi Dagobert} {[}1908{]}|pwk} von André Rivoire\pwindex{Rivoire, Andre 05.05.1872 – 19.08.1930@\textsc{Rivoire, André} (05.05.1872 – 19.08.1930), \emph{Schriftsteller}|pwk} auf deutsch bearbeitet. Die Uraufführung erlebte die
                  Übersetzung am 19. 1. 1910 am \emph{Deutschen Theater}\orgindex{Deutsches Theater Berlin@Deutsches Theater Berlin|pwk} in Berlin\oindex{Berlin@\textbf{Berlin}|pwk}. In Wien\oindex{Wien@\textbf{Wien}|pwk} fand die Premiere am
                     18. 11. 1911 am Deutschen
                     Volkstheater\oindex{Volkstheater@\textbf{Volkstheater}|pwk} statt, die Generalprobe wohl am Vortag. Schnitzler\pwindex{Schnitzler, Arthur 15.05.1862 – 21.10.1931@\textsc{Schnitzler, Arthur} (15.05.1862 – 21.10.1931), \emph{Schriftsteller, Mediziner}|pwk} besuchte erst die Aufführung am 5. 12. 1911.}}}\label{K_L02949-2h}
               darf man ihr beiwohnen? \pend
           \pstart
           Auf baldiges Wiedersehen.{\\[\baselineskip]} herzlichst Ihr{\\[\baselineskip]}\pend
           \leftskip=0em{}{\bigskip}\pstart
           \noindent{}Felix Salten\pend
           \pstart
           (Weites Land\pwindex{Schnitzler, Arthur 15.05.1862 – 21.10.1931@\textsc{Schnitzler, Arthur} (15.05.1862 – 21.10.1931), \emph{Schriftsteller, Mediziner}!weite Land. Tragikomoedie in fuenf Akten1910-10-20@\strich\emph{Das weite Land. Tragikomödie in fünf Akten} {[}1910-10-20{]}|pw})\pend
           
         
         \endnumbering\mylabel{h}\end{ledgroupsized}\begin{anhang}\end{anhang}\newcommand{\dateiname}{L02949}\newcommand{\titel}{Arthur Schnitzler an Felix Salten, 20. 10. 1911}\newcommand{\editorInnen}{Martin Anton Müller und Laura Untner}%% latex-leseansicht-abspann.tex
%% Abspann für die Leseansicht.
%% Der Schalter \ifkorrekturansicht ist bereits durch den Vorspann gesetzt.

%% latex-abspann.tex
%% Gemeinsamer Abspann für Korrekturansicht und Leseansicht.
%% Setzt den Schalter \ifkorrekturansicht voraus (gesetzt in den
%% einbindenden Dateien latex-korrekturansicht-abspann.tex bzw.
%% latex-leseansicht-abspann.tex).
%% ---------------------------------------------------------------

\normalsize

% Das esempio-Environment wird nur in der Leseansicht benötigt
\ifkorrekturansicht\else
\newenvironment{esempio}[3]%
{
    \vspace{1.5ex}
    \rlap{\underline{#1}}
    \par
    \setlength{\parindent}{0cm}
    \nopagebreak
    \leftskip=#2cm
    \rightskip=#3cm
}
{
    \par
}
\fi

\doendnotes{C}
\bigskip
\vfill

\clearpage

\footnotesize

\ifkorrekturansicht
  \lohead{\textsc{register}}
\fi

% theindex-Environment neu definieren ohne reledmac
\makeatletter
\renewenvironment{theindex}{%
  \ifkorrekturansicht
    \section*{\indexname}%
  \else
    \subsubsection*{Index der erwähnten Entitäten}%
  \fi
  \setlength{\parindent}{0pt}%
  \setlength{\parskip}{0pt plus 0.3pt}%
  \let\item\@idxitem
}{%
  \ifkorrekturansicht\clearpage\fi
}
\makeatother

\IfFileExists{\jobname-pw.ind}{\input{\jobname-pw.ind}}{}

% Quellenangabe nur in der Leseansicht
\ifkorrekturansicht\else
% Fallback-Definitionen, falls die .tex-Datei \titel etc. nicht gesetzt hat
\providecommand{\titel}{}
\providecommand{\editorInnen}{}
\providecommand{\dateiname}{\jobname}

\vspace{3cm}

\vfill

\footnotesize
\textsc{Quelle}: \titel. Herausgegeben von {\editorInnen}. In: \emph{Arthur Schnitzler: Briefwechsel mit Autorinnen und Autoren}.
 Digitale Edition, https://schnitzler-briefe.acdh.oeaw.ac.at/{\dateiname}.html (Stand \today)
\fi

\end{document}


      