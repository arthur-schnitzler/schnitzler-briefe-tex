%% latex-leseansicht-vorspann.tex
%% Vorspann für die Leseansicht.
%% Lädt die gemeinsame Datei latex-vorspann.tex mit nicht gesetztem Schalter.

\newif\ifkorrekturansicht
\korrekturansichtfalse

\input{../tex-inputs/latex-vorspann}


\section[Hugo von Hofmannsthal u. a. an Arthur Schnitzler, 15. {[}7. 1914{]}]{L02187 Hugo von Hofmannsthal u. a. an Arthur Schnitzler, 15. [7. 1914]}
\nopagebreak\mylabel{L02187v}
\rehead{ }\normalsize\beginnumbering\briefempfaengerindex{Schnitzler, Arthur@\textsc{Schnitzler, Arthur}!zzzSchwarzkopf, Gustav@\emph{von Gustav Schwarzkopf}!1914-07-152@{15. [7. 1914]}|(be}\briefempfaengerindex{Schnitzler, Arthur@\textsc{Schnitzler, Arthur}!zzzBeer-Hofmann, Paula@\emph{von Paula Beer-Hofmann}!1914-07-152@{15. [7. 1914]}|(be}\briefempfaengerindex{Schnitzler, Arthur@\textsc{Schnitzler, Arthur}!zzzBeer-Hofmann, Richard@\emph{von Richard Beer-Hofmann}!1914-07-152@{15. [7. 1914]}|(be}\briefempfaengerindex{Schnitzler, Arthur@\textsc{Schnitzler, Arthur}!zzzHofmannsthal, Gertrude von@\emph{von Gertrude von Hofmannsthal}!1914-07-152@{15. [7. 1914]}|(be}\briefempfaengerindex{Schnitzler, Arthur@\textsc{Schnitzler, Arthur}!zzzHofmannsthal, Hugo von@\emph{von Hugo von Hofmannsthal}!1914-07-152@{15. [7. 1914]}|(be}
\toendnotes[C]{\smallbreak\pagebreak[2]}
\correspDesc{Versand  durch Hugo von Hofmannsthal, Gerty von Hofmannsthal, Richard Beer-Hofmann, Paula Beer-Hofmann, Gustav Schwarzkopf am 15. [7. 1914] in Weißenbach am Attersee
\newline{}Umleitung  in Lunz am See
\newline{}Erhalt  durch Arthur Schnitzler im Zeitraum [16. 7. 1914
                  – 20. 7. 1914?] in Wien}\toendnotes[C]{\smallbreak}
\Standort{CUL, Schnitzler, B 43.}
\physDesc{Bildpostkarte, 231 Zeichen
\newline{}Handschrift Hugo von Hofmannsthal: Bleistift, lateinische Kurrent
\newline{}Handschrift Gertrude von Hofmannsthal: Bleistift
\newline{}Handschrift Richard Beer-Hofmann: Bleistift
\newline{}Handschrift Paula Beer-Hofmann: Bleistift
\newline{}Handschrift Gustav Schwarzkopf: Bleistift
\newline{}Versand: 1) Stempel: »\nobreak{}\oindex{Weißenbach am Attersee@\textbf{Weißenbach am Attersee}, \emph{Verwaltungsgebiet}|pwk}{[}Weissenbach{]} Atter\textcolor{gray}{see}\nobreak{}«.   2) mit schwarzer Tinte von unbekannter Hand nachgesandt nach Wien XVIII\oindex{XVIII., Währing@\textbf{XVIII., Währing}, \emph{Verwaltungsgebiet}|pw}, Sternwartestr. 71\oindex{Wien@\textbf{Wien}!XVIII., Währing@\textbf{XVIII., Währing}!Sternwartestraße 71@\textbf{Sternwartestraße 71}, \emph{Wohngebäude}|pw}
\newline{}Schnitzler: mit Bleistift beschriftet: »\textsc{Hofmannsthal}« 
\newline{}Ordnung: 1) mit Bleistift von unbekannter Hand nummeriert: »\strikeout{374}«  2) mit Bleistift von unbekannter Hand nummeriert:
                                    »347a«}
\buchAbdrucke{\weitereDrucke{Hugo von Hofmannsthal, Arthur Schnitzler: \emph{Briefwechsel}. Herausgegeben von Therese Nickl und Heinrich Schnitzler. Frankfurt am Main: \emph{S. Fischer} 1964, S. 276.} }\pstart{}{\pb}Herrn D\textsuperscript{r} Arthur Schnitzler\pend{}\pstart{}p. A. Herrn Hugo Schmiedl\pwindex{Schmidl, Hugo 6.\,11.\,1869 Wien – 22.\,10.\,1923 ebd.@\textsc{Schmidl, Hugo} (6.\,11.\,1869 Wien – 22.\,10.\,1923 ebd.), \emph{Industrieller}|pw}\pend{}\pstart{}Lunz\oindex{Lunz am See@\textbf{Lunz am See}, \emph{Hauptstadt}|pw}\pend{}\pstart{}Niederoest.\oindex{Niederösterreich@\textbf{Niederösterreich}, \emph{Land}|pw}\pend{}{\bigskip}
\pstart
           \noindent{}\centering{}{\pb}\textcolor{gray}{\textbf{Salzkammergut\oindex{Salzkammergut@\textbf{Salzkammergut}, \emph{Region}|pw}. Weissenbach am Attersee\oindex{Weißenbach am Attersee@\textbf{Weißenbach am Attersee}, \emph{Verwaltungsgebiet}|pw}.}}\pend
           \vspace{1em}
\pstart
           \raggedleft{}{\pb}den 15\textsuperscript{ten}\pend
           \vspace{0.5em}
\pstart
           Unter wolkenlosem Hi{\geminationm}el sitzen wir friedlich zusa{\geminationm}en und wünschen Euch das Gleiche.\pend
           
\pstart
           \spacefill\mbox{Hugo}{\\[\baselineskip]}\spacefill\mbox{{[}hs. Hofmannsthal:{]} Gerty}{\\[\baselineskip]}\spacefill\mbox{{[}hs. Schwarzkopf:{]} Gustav.}{\\[\baselineskip]}\spacefill\mbox{{[}hs. Beer-Hofmann:{]} Richard}{\\[\baselineskip]}\spacefill\mbox{{[}hs. Beer-Hofmann:{]} Paula}\pend
           \leftskip=0em{}
\pstart
           \noindent{}{[}hs. Hofmannsthal:{]} Schreibet einmal ein Wort!\pend
           
\pstart
           Aussee, Obertressen 14\oindex{Obertressen@\textbf{Obertressen}|pw}\pend
           \selectlanguage{ngerman}\endnumbering\briefempfaengerindex{Schnitzler, Arthur@\textsc{Schnitzler, Arthur}!zzzSchwarzkopf, Gustav@\emph{von Gustav Schwarzkopf}!1914-07-152@{15. [7. 1914]}|)be}\briefempfaengerindex{Schnitzler, Arthur@\textsc{Schnitzler, Arthur}!zzzBeer-Hofmann, Paula@\emph{von Paula Beer-Hofmann}!1914-07-152@{15. [7. 1914]}|)be}\briefempfaengerindex{Schnitzler, Arthur@\textsc{Schnitzler, Arthur}!zzzBeer-Hofmann, Richard@\emph{von Richard Beer-Hofmann}!1914-07-152@{15. [7. 1914]}|)be}\briefempfaengerindex{Schnitzler, Arthur@\textsc{Schnitzler, Arthur}!zzzHofmannsthal, Gertrude von@\emph{von Gertrude von Hofmannsthal}!1914-07-152@{15. [7. 1914]}|)be}\briefempfaengerindex{Schnitzler, Arthur@\textsc{Schnitzler, Arthur}!zzzHofmannsthal, Hugo von@\emph{von Hugo von Hofmannsthal}!1914-07-152@{15. [7. 1914]}|)be}\mylabel{L02187h}  \newcommand{\dateiname}{L02187}\newcommand{\titel}{Hugo von Hofmannsthal u. a. an Arthur Schnitzler, 15. [7. 1914]}\newcommand{\editorInnen}{Martin Anton Müller und Gerd-Hermann Susen}%% latex-leseansicht-abspann.tex
%% Abspann für die Leseansicht.
%% Der Schalter \ifkorrekturansicht ist bereits durch den Vorspann gesetzt.

%% latex-abspann.tex
%% Gemeinsamer Abspann für Korrekturansicht und Leseansicht.
%% Setzt den Schalter \ifkorrekturansicht voraus (gesetzt in den
%% einbindenden Dateien latex-korrekturansicht-abspann.tex bzw.
%% latex-leseansicht-abspann.tex).
%% ---------------------------------------------------------------

\normalsize

% Das esempio-Environment wird nur in der Leseansicht benötigt
\ifkorrekturansicht\else
\newenvironment{esempio}[3]%
{
    \vspace{1.5ex}
    \rlap{\underline{#1}}
    \par
    \setlength{\parindent}{0cm}
    \nopagebreak
    \leftskip=#2cm
    \rightskip=#3cm
}
{
    \par
}
\fi

\doendnotes{C}
\bigskip
\vfill

\clearpage

\footnotesize

\ifkorrekturansicht
  \lohead{\textsc{register}}
\fi

% theindex-Environment neu definieren ohne reledmac
\makeatletter
\renewenvironment{theindex}{%
  \ifkorrekturansicht
    \section*{\indexname}%
  \else
    \subsubsection*{Index der erwähnten Entitäten}%
  \fi
  \setlength{\parindent}{0pt}%
  \setlength{\parskip}{0pt plus 0.3pt}%
  \let\item\@idxitem
}{%
  \ifkorrekturansicht\clearpage\fi
}
\makeatother

\IfFileExists{\jobname-pw.ind}{\input{\jobname-pw.ind}}{}

% Quellenangabe nur in der Leseansicht
\ifkorrekturansicht\else
% Fallback-Definitionen, falls die .tex-Datei \titel etc. nicht gesetzt hat
\providecommand{\titel}{}
\providecommand{\editorInnen}{}
\providecommand{\dateiname}{\jobname}

\vspace{3cm}

\vfill

\footnotesize
\textsc{Quelle}: \titel. Herausgegeben von {\editorInnen}. In: \emph{Arthur Schnitzler: Briefwechsel mit Autorinnen und Autoren}.
 Digitale Edition, https://schnitzler-briefe.acdh.oeaw.ac.at/{\dateiname}.html (Stand \today)
\fi

\end{document}


