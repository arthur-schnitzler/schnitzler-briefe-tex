%% latex-leseansicht-vorspann.tex
%% Vorspann für die Leseansicht.
%% Lädt die gemeinsame Datei latex-vorspann.tex mit nicht gesetztem Schalter.

\newif\ifkorrekturansicht
\korrekturansichtfalse

\input{../tex-inputs/latex-vorspann}


\section[ Paul Goldmann an Arthur Schnitzler, 6. 10. {[}1902{]}]{L03225 Paul Goldmann an Arthur Schnitzler,  6. 10. [1902]}
\nopagebreak\mylabel{L03225v}
\rehead{ }\normalsize\beginnumbering\briefempfaengerindex{Schnitzler, Arthur@\textsc{Schnitzler, Arthur}!zzzGoldmann, Paul@\emph{von Paul Goldmann}!1902-10-061@{6. 10. [1902]}|(be}
\toendnotes[C]{\smallbreak\pagebreak[2]}
\correspDesc{Versand  durch Paul Goldmann am 6. 10. [1902] in Berlin
\newline{}Erhalt  durch Arthur Schnitzler im Zeitraum [7. 10. 1902
                  – 11. 10. 1902?] in Wien}\toendnotes[C]{\smallbreak}
\Standort{DLA, A:Schnitzler, HS.NZ85.1.3172.}
\physDesc{Brief, 1 Blatt, 2 Seiten, 651 Zeichen
\newline{}Handschrift: blaue Tinte, deutsche Kurrent
\newline{}Schnitzler: 1) mit Bleistift das Jahr »902« vermerkt  2) mit rotem Buntstift vier Unterstreichungen}\toendnotes[C]{\smallbreak}
\pstart
           \raggedleft{}{\pb}\textcolor{gray}{\textbf{DESSAUERSTRASSE 19}}\oindex{Dessauer Straße@\textbf{Dessauer Straße}, \emph{Straße}|pw}\pend
           
\pstart
           Berlin\oindex{Berlin@\textbf{Berlin}, \emph{Hauptstadt}|pw}, 6. Okt.\pend
           
\pstart\center{}Mein lieber Freund,\pend\vspace{0.5em}
\pstart
           Mit \textsc{Lindau\pwindex{Lindau, Paul 3.\,6.\,1839 Magdeburg – 31.\,1.\,1919 Berlin@\textsc{Lindau, Paul} (3.\,6.\,1839 Magdeburg – 31.\,1.\,1919 Berlin), \emph{Schriftsteller, Kritiker, Theaterleiter}|pw}}{ }ſtehe ich gegenwärtig{ }ſehr{ }ſchlecht. Die Gründe erzähle ich Dir mündlich. Ich
               kann ihm alſo das \label{K_L03225-1v}\edtext{Stück\pwindex{Schnitzler, Arthur 15.\,5.\,1862 Wien – 21.\,10.\,1931 ebd.@\textsc{Schnitzler, Arthur} (15.\,5.\,1862 Wien – 21.\,10.\,1931 ebd.), \emph{Schriftsteller, Mediziner}!einsame Weg. Schauspiel in fünf Akten@\strich\emph{Der einsame Weg. Schauspiel in fünf Akten}|pwuv}}{\lemma{\textnormal{\emph{Stück}}}\Cendnote{\textnormal{Von welchem Stück die Rede war, ist
                  ungeklärt. Es dürfte sich nicht um \emph{Der Schleier
                     der Beatrice}\pwindex{Schnitzler, Arthur 15.\,5.\,1862 Wien – 21.\,10.\,1931 ebd.@\textsc{Schnitzler, Arthur} (15.\,5.\,1862 Wien – 21.\,10.\,1931 ebd.), \emph{Schriftsteller, Mediziner}!Schleier der Beatrice. Schauspiel in fünf Akten@\strich\emph{Der Schleier der Beatrice. Schauspiel in fünf Akten}|pwk} gehandelt haben, da Paul
                     Lindau\pwindex{Lindau, Paul 3.\,6.\,1839 Magdeburg – 31.\,1.\,1919 Berlin@\textsc{Lindau, Paul} (3.\,6.\,1839 Magdeburg – 31.\,1.\,1919 Berlin), \emph{Schriftsteller, Kritiker, Theaterleiter}|pwk} bereits in einem Brief an Schnitzler vom 11. 9. 1900 das Stück\pwindex{Schnitzler, Arthur 15.\,5.\,1862 Wien – 21.\,10.\,1931 ebd.@\textsc{Schnitzler, Arthur} (15.\,5.\,1862 Wien – 21.\,10.\,1931 ebd.), \emph{Schriftsteller, Mediziner}!Schleier der Beatrice. Schauspiel in fünf Akten@\strich\emph{Der Schleier der Beatrice. Schauspiel in fünf Akten}|pwkv} für das \emph{Berliner Theater}\orgindex{Berliner Theater@Berliner Theater|pwk} abgelehnt hatte (vgl.
                        \emph{Cambridge University Library}, B 60). Eventuell
                  handelte es sich um das zum Zeitpunkt noch nicht fertiggestellte nächste Stück,
                     \emph{Der einsame Weg}\pwindex{Schnitzler, Arthur 15.\,5.\,1862 Wien – 21.\,10.\,1931 ebd.@\textsc{Schnitzler, Arthur} (15.\,5.\,1862 Wien – 21.\,10.\,1931 ebd.), \emph{Schriftsteller, Mediziner}!einsame Weg. Schauspiel in fünf Akten@\strich\emph{Der einsame Weg. Schauspiel in fünf Akten}|pwk}, an dessen viertem Akt Schnitzler zuletzt gearbeitet hatte.}}}\label{K_L03225-1}
               einſtweilen nicht einreichen. Aber wie \textsc{Lindau\pwindex{Lindau, Paul 3.\,6.\,1839 Magdeburg – 31.\,1.\,1919 Berlin@\textsc{Lindau, Paul} (3.\,6.\,1839 Magdeburg – 31.\,1.\,1919 Berlin), \emph{Schriftsteller, Kritiker, Theaterleiter}|pw}}{ }ſchon iſt, kann{ }ſich die Situation raſch ändern; und dann{ }ſtehe ich{ }ſelbſtverſtändlich zu Deiner Verfügung.\pend
           
\pstart
           \label{K_L03225-2v}\edtext{\textsc{Felix\pwindex{Felix, Hugo 19.\,11.\,1866 Budapest – 25.\,8.\,1934 Hollywood@\textsc{Felix, Hugo} (19.\,11.\,1866 Budapest – 25.\,8.\,1934 Hollywood), \emph{Komponist, Chemiker}|pw}}}{\lemma{\textnormal{\emph{Felix}}}\Cendnote{\textnormal{Siehe XXXX Auszeichnungsfehler: Dokument L03223 nicht gefunden.
               }}}\label{K_L03225-2} habe ich Deine Antwort {\pb}übermittelt; er{ }ſandte
               mir ein ganz beglücktes Telegramm.\pend
           
\pstart
           \label{K_L03225-3v}\edtext{\textsc{Fulda\pwindex{Fulda, Ludwig 15.\,7.\,1862 Frankfurt am Main – 30.\,3.\,1939 Berlin@\textsc{Fulda, Ludwig} (15.\,7.\,1862 Frankfurt am Main – 30.\,3.\,1939 Berlin), \emph{Schriftsteller, Übersetzer}!Kaltwasser. Lustspiel in drei Aufzügen@\strich\emph{Kaltwasser. Lustspiel in drei Aufzügen}|pwv}\pwindex{Fulda, Ludwig 15.\,7.\,1862 Frankfurt am Main – 30.\,3.\,1939 Berlin@\textsc{Fulda, Ludwig} (15.\,7.\,1862 Frankfurt am Main – 30.\,3.\,1939 Berlin), \emph{Schriftsteller, Übersetzer}|pw}}}{\lemma{\textnormal{\emph{Fulda}}}\Cendnote{\textnormal{Ludwig Fuldas\pwindex{Fulda, Ludwig 15.\,7.\,1862 Frankfurt am Main – 30.\,3.\,1939 Berlin@\textsc{Fulda, Ludwig} (15.\,7.\,1862 Frankfurt am Main – 30.\,3.\,1939 Berlin), \emph{Schriftsteller, Übersetzer}|pwk} dreiaktiges Lustspiel \emph{Kaltwasser}\pwindex{Fulda, Ludwig 15.\,7.\,1862 Frankfurt am Main – 30.\,3.\,1939 Berlin@\textsc{Fulda, Ludwig} (15.\,7.\,1862 Frankfurt am Main – 30.\,3.\,1939 Berlin), \emph{Schriftsteller, Übersetzer}!Kaltwasser. Lustspiel in drei Aufzügen@\strich\emph{Kaltwasser. Lustspiel in drei Aufzügen}|pwk} hatte am 5. 10. 1902 die Uraufführung\eventindex{Lessing-Theater@\textbf{Lessing-Theater}!Uraufführung von Kaltwasser, 5.10.1902@Uraufführung von Kaltwasser, 5.10.1902|pwkv} am Berlin\oindex{Berlin@\textbf{Berlin}, \emph{Hauptstadt}|pwk}er \emph{Lessing-Theater}\orgindex{Lessing-Theater@Lessing-Theater|pwk}.}}}\label{K_L03225-3} iſt
               bös durchgefallen.\pend
           
\pstart
           Kann ich die \label{K_L03225-4v}\edtext{\textsc{Musset\pwindex{Musset, Alfred de 11.\,12.\,1810 Paris – 2.\,5.\,1857 ebd.@\textsc{Musset, Alfred de} (11.\,12.\,1810 Paris – 2.\,5.\,1857 ebd.), \emph{Schriftsteller}!ne faut jurer de rien@\strich\emph{Il ne faut jurer de rien}|pwv}\pwindex{Musset, Alfred de 11.\,12.\,1810 Paris – 2.\,5.\,1857 ebd.@\textsc{Musset, Alfred de} (11.\,12.\,1810 Paris – 2.\,5.\,1857 ebd.), \emph{Schriftsteller}|pw}}-Überſetzung\pwindex{Musset, Alfred de 11.\,12.\,1810 Paris – 2.\,5.\,1857 ebd.@\textsc{Musset, Alfred de} (11.\,12.\,1810 Paris – 2.\,5.\,1857 ebd.), \emph{Schriftsteller}!Man soll nichts verschwören. Komödie in 3 Akten@\strich\emph{Man soll nichts verschwören. Komödie in 3 Akten}|pwv}}{\lemma{\textnormal{\emph{Musset-Übersetzung}}}\Cendnote{\textnormal{Alfred de Musset\pwindex{Musset, Alfred de 11.\,12.\,1810 Paris – 2.\,5.\,1857 ebd.@\textsc{Musset, Alfred de} (11.\,12.\,1810 Paris – 2.\,5.\,1857 ebd.), \emph{Schriftsteller}|pwk}: \emph{Man soll nichts verschwören}\pwindex{Musset, Alfred de 11.\,12.\,1810 Paris – 2.\,5.\,1857 ebd.@\textsc{Musset, Alfred de} (11.\,12.\,1810 Paris – 2.\,5.\,1857 ebd.), \emph{Schriftsteller}!Man soll nichts verschwören. Komödie in 3 Akten@\strich\emph{Man soll nichts verschwören. Komödie in 3 Akten}|pwk}. Aus dem Französischen von
                        Paul Goldmann\pwindex{Goldmann, Paul 31.\,1.\,1865 Breslau – 25.\,9.\,1935 Wien@\textsc{Goldmann, Paul} (31.\,1.\,1865 Breslau – 25.\,9.\,1935 Wien), \emph{Schriftsteller, Journalist}|pwk}. Frankfurt am Main\oindex{Frankfurt am Main@\textbf{Frankfurt am Main}, \emph{Hauptstadt}|pwk}: \emph{Rütten
                           {\kaufmannsund} Loening}\orgindex{Rütten und Loening@Rütten {\kaufmannsund}  Loening|pwk}{ }1902. Die Uraufführung\eventindex{Ständetheater@\textbf{Ständetheater}!Premiere von Man soll nichts verschwören, 5.3.1903@Premiere von Man soll nichts verschwören, 5.3.1903|pwkv} des Stücks\pwindex{Musset, Alfred de 11.\,12.\,1810 Paris – 2.\,5.\,1857 ebd.@\textsc{Musset, Alfred de} (11.\,12.\,1810 Paris – 2.\,5.\,1857 ebd.), \emph{Schriftsteller}!ne faut jurer de rien@\strich\emph{Il ne faut jurer de rien}|pwkv} in der Übersetzung\pwindex{Musset, Alfred de 11.\,12.\,1810 Paris – 2.\,5.\,1857 ebd.@\textsc{Musset, Alfred de} (11.\,12.\,1810 Paris – 2.\,5.\,1857 ebd.), \emph{Schriftsteller}!Man soll nichts verschwören. Komödie in 3 Akten@\strich\emph{Man soll nichts verschwören. Komödie in 3 Akten}|pwkv}{ }Goldmanns\pwindex{Goldmann, Paul 31.\,1.\,1865 Breslau – 25.\,9.\,1935 Wien@\textsc{Goldmann, Paul} (31.\,1.\,1865 Breslau – 25.\,9.\,1935 Wien), \emph{Schriftsteller, Journalist}|pwk} fand am 5. 3. 1903 im Deutschen
                     Landestheater\oindex{Ständetheater@\textbf{Ständetheater}, \emph{Theater}|pwk} in Prag\oindex{Prag@\textbf{Prag}, \emph{Land}|pwk} statt. Eine
                  Aufführung am Wien\oindex{Wien@\textbf{Wien}, \emph{Verwaltungsgebiet}|pwk}er Volkstheater\oindex{Wien@\textbf{Wien}!VII., Neubau@\textbf{VII., Neubau}!Volkstheater@\textbf{Volkstheater}, \emph{Theater}|pwk} fand nicht statt.}}}\label{K_L03225-4} dem Volkstheater\orgindex{Volkstheater@Volkstheater|pw} einreichen? Mit \textsc{Schlenther\pwindex{Schlenther, Paul 20.\,8.\,1854 Chernyakhovsk – 30.\,4.\,1916 Berlin@\textsc{Schlenther, Paul} (20.\,8.\,1854 Chernyakhovsk – 30.\,4.\,1916 Berlin), \emph{Schriftsteller, Kritiker, Theaterleiter}|pw}\orgindex{Burgtheater@Burgtheater|pwv}} will ich nichts zu thun haben.\pend
           
\pstart
           Iſt \label{K_L03225-5v}\edtext{\textsc{Olga\pwindex{Schnitzler, Olga 17.\,1.\,1882 Wien – 13.\,1.\,1970 Lugano@\textsc{Schnitzler, Olga} (17.\,1.\,1882 Wien – 13.\,1.\,1970 Lugano), \emph{Schauspielerin, Sängerin}|pw}} wieder ganz geſund}{\lemma{\textnormal{\emph{Olga wieder ganz gesund}}}\Cendnote{\textnormal{Siehe XXXX Auszeichnungsfehler: Dokument L03223 nicht gefunden.
               }}}\label{K_L03225-5}?\pend
           
\pstart
           Ich denke auch, die \label{K_L03225-6v}\edtext{»Zeit\pwindex{Zeit@\emph{Die Zeit}|pw}«}{\lemma{\textnormal{\emph{»Zeit«}}}\Cendnote{\textnormal{Siehe XXXX Auszeichnungsfehler: Dokument L03224 nicht gefunden.
               }}}\label{K_L03225-6} wird{ }ſich noch{ }ſehr gut machen. Die \label{K_L03225-7v}\edtext{N. Fr. Pr.\orgindex{Neue Freie Presse@Neue Freie Presse|pw}}{\lemma{\textnormal{\emph{N. Fr. Pr.}}}\Cendnote{\textnormal{In welcher spezifischen Weise bei der
                     \emph{Neuen Freie Presse}\orgindex{Neue Freie Presse@Neue Freie Presse|pwk} in den ersten zwei
                  Wochen nach dem ersten Erscheinen der ersten Nummer der Tageszeitung \emph{Die Zeit}\pwindex{Zeit@\emph{Die Zeit}|pwk} Entspannung eingetreten ist, ließ sich
                  nicht ermitteln. Siehe auch XXXX Auszeichnungsfehler: Dokument L03072 nicht gefunden.}}}\label{K_L03225-7} frohlockt zu früh. Viele treue Grüße!\pend
           \pstart Dein \spacefill\mbox{Paul Goldm}\pend{}\selectlanguage{ngerman}\endnumbering\briefempfaengerindex{Schnitzler, Arthur@\textsc{Schnitzler, Arthur}!zzzGoldmann, Paul@\emph{von Paul Goldmann}!1902-10-061@{6. 10. [1902]}|)be}\mylabel{L03225h}  \newcommand{\dateiname}{L03225}\newcommand{\titel}{Paul Goldmann an Arthur Schnitzler, 6. 10. [1902]}\newcommand{\editorInnen}{Martin Anton Müller und Laura Untner}%% latex-leseansicht-abspann.tex
%% Abspann für die Leseansicht.
%% Der Schalter \ifkorrekturansicht ist bereits durch den Vorspann gesetzt.

%% latex-abspann.tex
%% Gemeinsamer Abspann für Korrekturansicht und Leseansicht.
%% Setzt den Schalter \ifkorrekturansicht voraus (gesetzt in den
%% einbindenden Dateien latex-korrekturansicht-abspann.tex bzw.
%% latex-leseansicht-abspann.tex).
%% ---------------------------------------------------------------

\normalsize

% Das esempio-Environment wird nur in der Leseansicht benötigt
\ifkorrekturansicht\else
\newenvironment{esempio}[3]%
{
    \vspace{1.5ex}
    \rlap{\underline{#1}}
    \par
    \setlength{\parindent}{0cm}
    \nopagebreak
    \leftskip=#2cm
    \rightskip=#3cm
}
{
    \par
}
\fi

\doendnotes{C}
\bigskip
\vfill

\clearpage

\footnotesize

\ifkorrekturansicht
  \lohead{\textsc{register}}
\fi

% theindex-Environment neu definieren ohne reledmac
\makeatletter
\renewenvironment{theindex}{%
  \ifkorrekturansicht
    \section*{\indexname}%
  \else
    \subsubsection*{Index der erwähnten Entitäten}%
  \fi
  \setlength{\parindent}{0pt}%
  \setlength{\parskip}{0pt plus 0.3pt}%
  \let\item\@idxitem
}{%
  \ifkorrekturansicht\clearpage\fi
}
\makeatother

\IfFileExists{\jobname-pw.ind}{\input{\jobname-pw.ind}}{}

% Quellenangabe nur in der Leseansicht
\ifkorrekturansicht\else
% Fallback-Definitionen, falls die .tex-Datei \titel etc. nicht gesetzt hat
\providecommand{\titel}{}
\providecommand{\editorInnen}{}
\providecommand{\dateiname}{\jobname}

\vspace{3cm}

\vfill

\footnotesize
\textsc{Quelle}: \titel. Herausgegeben von {\editorInnen}. In: \emph{Arthur Schnitzler: Briefwechsel mit Autorinnen und Autoren}.
 Digitale Edition, https://schnitzler-briefe.acdh.oeaw.ac.at/{\dateiname}.html (Stand \today)
\fi

\end{document}


