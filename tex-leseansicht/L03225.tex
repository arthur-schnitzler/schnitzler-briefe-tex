%% latex-leseansicht-vorspann.tex
%% Vorspann für die Leseansicht.
%% Lädt die gemeinsame Datei latex-vorspann.tex mit nicht gesetztem Schalter.

\newif\ifkorrekturansicht
\korrekturansichtfalse

\input{../tex-inputs/latex-vorspann}

\begin{center}
            \textcolor{red}{ENTWURF, NICHT FERTIG KORRIGIERT}
                      \end{center}
            
         
         \renewcommand{\erwaehntePersonen}{Personen: Olga Schnitzler}
         \renewcommand{\erwaehnteOrte}{Orte: Berlin, Dessauer Straße, Wien}
         \renewcommand{\erwaehnteWerke}{}
               \section[ Paul Goldmann an Arthur Schnitzler, 6. 10. {[}1902{]}]{ Paul Goldmann an Arthur Schnitzler, 6. 10. {[}1902{]}}\nopagebreak\mylabel{v}\rehead{ }\begin{ledgroupsized}[t]{13cm}\normalsize\beginnumbering \toendnotes[C]{\smallbreak\pagebreak[2]} \Standort{DLA, A:Schnitzler, HS.NZ85.1.3172.}
\physDesc{Brief, 1 Blatt, 2 Seiten
\newline{}Handschrift: blaue Tinte, deutsche Kurrent
\newline{}Schnitzler: 1) mit Bleistift das Jahr »{[}1{]}902«
                                            vermerkt  2) mit rotem Buntstift vier Unterstreichungen}\pstart
           \noindent{}\raggedleft{}{\pb}\textcolor{gray}{\textbf{DESSAUERSTRASSE 19}}\oindex{Dessauer Strasse@\textbf{Dessauer Straße}|pw}\pend
           \pstart
           Berlin\oindex{Berlin@\textbf{Berlin}|pw}, 6. Okt.\pend
           \pstart\center{}Mein lieber Freund,\pend\pstart
           Mit \textsc{Lindau\textcolor{red}{\textsuperscript{\textbf{KEY}}}} ſtehe ich gegenwärtig ſehr ſchlecht. Die Gründe erzähle ich Dir mündlich.
                    Ich kann ihm alſo das Stück\textcolor{red}{\textsuperscript{\textbf{KEY}}} einſtweilen nicht
                    einreichen. Aber wie \textsc{Lindau\textcolor{red}{\textsuperscript{\textbf{KEY}}}} ſchon iſt, kann ſich die Situation raſch ändern; und dann ſtehe ich
                    ſelbſtverſtändlich zu Deiner Verfügung. \pend
           \pstart
           \textsc{Felix\textcolor{red}{\textsuperscript{\textbf{KEY}}}} habe ich Deine Antwort {\pb} übermittelt; er
                    ſandte mir ein ganz beglücktes Telegramm. \pend
           \pstart
           \textsc{Fulda\textcolor{red}{\textsuperscript{\textbf{KEY}}}\textcolor{red}{\textsuperscript{\textbf{KEY}}}\textcolor{red}{\textsuperscript{\textbf{KEY}}}\textcolor{red}{\textsuperscript{\textbf{KEY}}}\textcolor{red}{\textsuperscript{\textbf{KEY}}}\textcolor{red}{\textsuperscript{\textbf{KEY}}}} iſt bös durchgefallen.\textcolor{red}{\textsuperscript{\textbf{KEY}}} iſt bös durchgefallen. \pend
           \pstart
           Kann ich die \textsc{Musset\textcolor{red}{\textsuperscript{\textbf{KEY}}}}-Überſetzung\textcolor{red}{\textsuperscript{\textbf{KEY}}} dem Volkstheater\textcolor{red}{\textsuperscript{\textbf{KEY}}} einreichen? Mit \textsc{Schlenther\textcolor{red}{\textsuperscript{\textbf{KEY}}}\textcolor{red}{\textsuperscript{\textbf{KEY}}}\textcolor{red}{\textsuperscript{\textbf{KEY}}}\textcolor{red}{\textsuperscript{\textbf{KEY}}}\textcolor{red}{\textsuperscript{\textbf{KEY}}}\textcolor{red}{\textsuperscript{\textbf{KEY}}}}\textcolor{red}{\textsuperscript{\textbf{KEY}}} will ich nichts zu thun haben. \pend
           \pstart
           Iſt \textsc{Olga\pwindex{Schnitzler, Olga 17.01.1882 – 13.01.1970@\textsc{Schnitzler, Olga} (17.01.1882 – 13.01.1970), \emph{Schauspielerin, Sängerin}|pw}} wieder ganz geſund? \pend
           \pstart
           Ich denke auch, die »Zeit\textcolor{red}{\textsuperscript{\textbf{KEY}}}« wird ſich noch ſehr
                    gut machen. Die N. Fr. Pr.\textcolor{red}{\textsuperscript{\textbf{KEY}}} frohlockt zu früh. Viele
                    treue Grüße! {\\[\baselineskip]}Dein \spacefill\mbox{Paul Goldm }\pend
           \leftskip=0em{}
         
         \endnumbering\mylabel{h}\end{ledgroupsized}\begin{anhang}\end{anhang}\newcommand{\dateiname}{L03225}\newcommand{\titel}{Paul Goldmann an Arthur Schnitzler, 6. 10. [1902]}\newcommand{\editorInnen}{Martin Anton Müller und Laura Untner}%% latex-leseansicht-abspann.tex
%% Abspann für die Leseansicht.
%% Der Schalter \ifkorrekturansicht ist bereits durch den Vorspann gesetzt.

%% latex-abspann.tex
%% Gemeinsamer Abspann für Korrekturansicht und Leseansicht.
%% Setzt den Schalter \ifkorrekturansicht voraus (gesetzt in den
%% einbindenden Dateien latex-korrekturansicht-abspann.tex bzw.
%% latex-leseansicht-abspann.tex).
%% ---------------------------------------------------------------

\normalsize

% Das esempio-Environment wird nur in der Leseansicht benötigt
\ifkorrekturansicht\else
\newenvironment{esempio}[3]%
{
    \vspace{1.5ex}
    \rlap{\underline{#1}}
    \par
    \setlength{\parindent}{0cm}
    \nopagebreak
    \leftskip=#2cm
    \rightskip=#3cm
}
{
    \par
}
\fi

\doendnotes{C}
\bigskip
\vfill

\clearpage

\footnotesize

\ifkorrekturansicht
  \lohead{\textsc{register}}
\fi

% theindex-Environment neu definieren ohne reledmac
\makeatletter
\renewenvironment{theindex}{%
  \ifkorrekturansicht
    \section*{\indexname}%
  \else
    \subsubsection*{Index der erwähnten Entitäten}%
  \fi
  \setlength{\parindent}{0pt}%
  \setlength{\parskip}{0pt plus 0.3pt}%
  \let\item\@idxitem
}{%
  \ifkorrekturansicht\clearpage\fi
}
\makeatother

\IfFileExists{\jobname-pw.ind}{\input{\jobname-pw.ind}}{}

% Quellenangabe nur in der Leseansicht
\ifkorrekturansicht\else
% Fallback-Definitionen, falls die .tex-Datei \titel etc. nicht gesetzt hat
\providecommand{\titel}{}
\providecommand{\editorInnen}{}
\providecommand{\dateiname}{\jobname}

\vspace{3cm}

\vfill

\footnotesize
\textsc{Quelle}: \titel. Herausgegeben von {\editorInnen}. In: \emph{Arthur Schnitzler: Briefwechsel mit Autorinnen und Autoren}.
 Digitale Edition, https://schnitzler-briefe.acdh.oeaw.ac.at/{\dateiname}.html (Stand \today)
\fi

\end{document}


      