%% latex-leseansicht-vorspann.tex
%% Vorspann für die Leseansicht.
%% Lädt die gemeinsame Datei latex-vorspann.tex mit nicht gesetztem Schalter.

\newif\ifkorrekturansicht
\korrekturansichtfalse

\input{../tex-inputs/latex-vorspann}


         
         \renewcommand{\erwaehntePersonen}{Personen: Hugo Felix, Ludwig Fulda, Paul Lindau, Alfred de Musset, Paul Schlenther, Olga Schnitzler}
         \renewcommand{\erwaehnteInstitutionen}{Institutionen: Berliner Theater, Burgtheater, Neue Freie Presse, Rütten {\kaufmannsund}  Loening, Volkstheater}
         \renewcommand{\erwaehnteOrte}{Orte: Berlin, Dessauer Straße, Frankfurt am Main, Lessing-Theater, Prag, Ständetheater, Volkstheater, Wien}
         \renewcommand{\erwaehnteWerke}{Werke: Der Schleier der Beatrice. Schauspiel in fünf Akten, Der einsame Weg. Schauspiel in fünf Akten, Die Zeit, Il ne faut jurer de rien, Kaltwasser. Lustspiel in drei Aufzügen, Man soll nichts verschwören. Komödie}
               \section[ Paul Goldmann an Arthur Schnitzler, 6. 10. {[}1902{]}]{ Paul Goldmann an Arthur Schnitzler, 6. 10. {[}1902{]}}\nopagebreak\mylabel{v}\rehead{ }\begin{ledgroupsized}[t]{13cm}\normalsize\beginnumbering \toendnotes[C]{\smallbreak\pagebreak[2]} \Standort{DLA, A:Schnitzler, HS.NZ85.1.3172.}
\physDesc{Brief, 1 Blatt, 2 Seiten, 651 Zeichen
\newline{}Handschrift: blaue Tinte, deutsche Kurrent
\newline{}Schnitzler: 1) mit Bleistift das Jahr »{[}1{]}902« vermerkt  2) mit rotem Buntstift vier Unterstreichungen}\toendnotes[C]{\smallbreak}\pstart
           \noindent{}\raggedleft{}{\pb}\textcolor{gray}{\textbf{DESSAUERSTRASSE 19}}\oindex{Dessauer Strasse@\textbf{Dessauer Straße}|pw}\pend
           \pstart
           Berlin\oindex{Berlin@\textbf{Berlin}|pw}, 6. Okt.\pend
           \pstart\center{}Mein lieber Freund,\pend\pstart
           Mit \textsc{Lindau\pwindex{Lindau, Paul 03.06.1839 – 31.01.1919@\textsc{Lindau, Paul} (03.06.1839 – 31.01.1919), \emph{Schriftsteller, Kritiker, Theaterleiter}|pw}} ſtehe ich gegenwärtig ſehr ſchlecht. Die Gründe erzähle ich Dir mündlich. Ich
               kann ihm alſo das \label{K_L03225-32v}\edtext{Stück\pwindex{Schnitzler, Arthur 15.05.1862 – 21.10.1931@\textsc{Schnitzler, Arthur} (15.05.1862 – 21.10.1931), \emph{Schriftsteller, Mediziner}!einsame Weg. Schauspiel in fuenf Akten1904@\strich\emph{Der einsame Weg. Schauspiel in fünf Akten} {[}1904{]}|pwuv}}{\lemma{\textnormal{\emph{Stück}}}\Cendnote{\textnormal{Von welchem Stück die Rede war, ist
                  ungeklärt. Es dürfte sich nicht um \emph{Der Schleier
                     der Beatrice}\pwindex{Schnitzler, Arthur 15.05.1862 – 21.10.1931@\textsc{Schnitzler, Arthur} (15.05.1862 – 21.10.1931), \emph{Schriftsteller, Mediziner}!Schleier der Beatrice. Schauspiel in fuenf Akten1900-12-01@\strich\emph{Der Schleier der Beatrice. Schauspiel in fünf Akten} {[}1900-12-01{]}|pwk} gehandelt haben, da Paul
                     Lindau\pwindex{Lindau, Paul 03.06.1839 – 31.01.1919@\textsc{Lindau, Paul} (03.06.1839 – 31.01.1919), \emph{Schriftsteller, Kritiker, Theaterleiter}|pwk} bereits in einem Brief an Schnitzler\pwindex{Schnitzler, Arthur 15.05.1862 – 21.10.1931@\textsc{Schnitzler, Arthur} (15.05.1862 – 21.10.1931), \emph{Schriftsteller, Mediziner}|pwk} vom 11. 9. 1900 das Stück\pwindex{Schnitzler, Arthur 15.05.1862 – 21.10.1931@\textsc{Schnitzler, Arthur} (15.05.1862 – 21.10.1931), \emph{Schriftsteller, Mediziner}!Schleier der Beatrice. Schauspiel in fuenf Akten1900-12-01@\strich\emph{Der Schleier der Beatrice. Schauspiel in fünf Akten} {[}1900-12-01{]}|pwkv} für das \emph{Berliner Theater}\orgindex{Berliner Theater@Berliner Theater|pwk} abgelehnt hatte (vgl.
                        \emph{Cambridge University Library}, B 60). Eventuell
                  handelte es sich um das zum Zeitpunkt noch nicht fertiggestellte nächste Stück,
                     \emph{Der einsame Weg}\pwindex{Schnitzler, Arthur 15.05.1862 – 21.10.1931@\textsc{Schnitzler, Arthur} (15.05.1862 – 21.10.1931), \emph{Schriftsteller, Mediziner}!einsame Weg. Schauspiel in fuenf Akten1904@\strich\emph{Der einsame Weg. Schauspiel in fünf Akten} {[}1904{]}|pwk}, an dessen viertem Akt Schnitzler\pwindex{Schnitzler, Arthur 15.05.1862 – 21.10.1931@\textsc{Schnitzler, Arthur} (15.05.1862 – 21.10.1931), \emph{Schriftsteller, Mediziner}|pwk} zuletzt gearbeitet hatte.}}}\label{K_L03225-32h}
               einſtweilen nicht einreichen. Aber wie \textsc{Lindau\pwindex{Lindau, Paul 03.06.1839 – 31.01.1919@\textsc{Lindau, Paul} (03.06.1839 – 31.01.1919), \emph{Schriftsteller, Kritiker, Theaterleiter}|pw}} ſchon iſt, kann ſich die Situation raſch ändern; und dann ſtehe ich
               ſelbſtverſtändlich zu Deiner Verfügung.\pend
           \pstart
           \label{K_L03225-2v}\edtext{\textsc{Felix\pwindex{Felix, Hugo 19.11.1866 – 25.08.1934@\textsc{Felix, Hugo} (19.11.1866 – 25.08.1934), \emph{Komponist, Chemiker}|pw}}}{\lemma{\textnormal{\emph{Felix}}}\Cendnote{\textnormal{siehe Paul Goldmann an Arthur Schnitzler, 2. [10. 1902]}}}\label{K_L03225-2h} habe ich Deine Antwort {\pb}übermittelt; er ſandte
               mir ein ganz beglücktes Telegramm.\pend
           \pstart
           \label{K_L03225-3v}\edtext{\textsc{Fulda\pwindex{Fulda, Ludwig 15.07.1862 – 30.03.1939@\textsc{Fulda, Ludwig} (15.07.1862 – 30.03.1939), \emph{Schriftsteller, Übersetzer}!Kaltwasser. Lustspiel in drei Aufzuegen1902@\strich\emph{Kaltwasser. Lustspiel in drei Aufzügen} {[}1902{]}|pwv}\pwindex{Fulda, Ludwig 15.07.1862 – 30.03.1939@\textsc{Fulda, Ludwig} (15.07.1862 – 30.03.1939), \emph{Schriftsteller, Übersetzer}|pw}}}{\lemma{\textnormal{\emph{Fulda}}}\Cendnote{\textnormal{Ludwig Fulda\pwindex{Fulda, Ludwig 15.07.1862 – 30.03.1939@\textsc{Fulda, Ludwig} (15.07.1862 – 30.03.1939), \emph{Schriftsteller, Übersetzer}|pwk}s dreiaktiges Lustspiel \emph{Kaltwasser}\pwindex{Fulda, Ludwig 15.07.1862 – 30.03.1939@\textsc{Fulda, Ludwig} (15.07.1862 – 30.03.1939), \emph{Schriftsteller, Übersetzer}!Kaltwasser. Lustspiel in drei Aufzuegen1902@\strich\emph{Kaltwasser. Lustspiel in drei Aufzügen} {[}1902{]}|pwk} hatte am 5. 10. 1902 die Uraufführung am Berlin\oindex{Berlin@\textbf{Berlin}|pwk}er Lessing-Theater\oindex{Lessing-Theater@\textbf{Lessing-Theater}|pwk}.}}}\label{K_L03225-3h} iſt
               bös durchgefallen.\pend
           \pstart
           Kann ich die \label{K_L03225-4v}\edtext{\textsc{Musset\pwindex{Musset, Alfred de 11.12.1810 – 02.05.1857@\textsc{Musset, Alfred de} (11.12.1810 – 02.05.1857), \emph{Schriftsteller}!ne faut jurer de rien1836@\strich\emph{Il ne faut jurer de rien} {[}1836{]}|pwv}\pwindex{Musset, Alfred de 11.12.1810 – 02.05.1857@\textsc{Musset, Alfred de} (11.12.1810 – 02.05.1857), \emph{Schriftsteller}|pw}}-Überſetzung\pwindex{Goldmann, Paul 31.01.1865 – 25.09.1935@\textsc{Goldmann, Paul} (31.01.1865 – 25.09.1935), \emph{Schriftsteller, Journalist}!Man soll nichts verschwoeren. Komoedie1902-10-17@\strich\emph{Man soll nichts verschwören. Komödie} {[}Übersetzung, 1902-10-17{]}|pwv}}{\lemma{\textnormal{\emph{Musset-Überſetzung}}}\Cendnote{\textnormal{Alfred de Musset\pwindex{Musset, Alfred de 11.12.1810 – 02.05.1857@\textsc{Musset, Alfred de} (11.12.1810 – 02.05.1857), \emph{Schriftsteller}|pwk}: \emph{Man soll nichts verschwören}\pwindex{Goldmann, Paul 31.01.1865 – 25.09.1935@\textsc{Goldmann, Paul} (31.01.1865 – 25.09.1935), \emph{Schriftsteller, Journalist}!Man soll nichts verschwoeren. Komoedie1902-10-17@\strich\emph{Man soll nichts verschwören. Komödie} {[}Übersetzung, 1902-10-17{]}|pwk}. Aus dem Französischen von
                        Paul Goldmann\pwindex{Goldmann, Paul 31.01.1865 – 25.09.1935@\textsc{Goldmann, Paul} (31.01.1865 – 25.09.1935), \emph{Schriftsteller, Journalist}|pwk}. Frankfurt am Main\oindex{Frankfurt am Main@\textbf{Frankfurt am Main}|pwk}: \emph{Rütten
                           {\kaufmannsund} Loening}\orgindex{Ruetten und Loening@Rütten {\kaufmannsund}  Loening|pwk}{ }1902. Die Uraufführung des Stück\pwindex{Musset, Alfred de 11.12.1810 – 02.05.1857@\textsc{Musset, Alfred de} (11.12.1810 – 02.05.1857), \emph{Schriftsteller}!ne faut jurer de rien1836@\strich\emph{Il ne faut jurer de rien} {[}1836{]}|pwkv}s in der Übersetzung\pwindex{Goldmann, Paul 31.01.1865 – 25.09.1935@\textsc{Goldmann, Paul} (31.01.1865 – 25.09.1935), \emph{Schriftsteller, Journalist}!Man soll nichts verschwoeren. Komoedie1902-10-17@\strich\emph{Man soll nichts verschwören. Komödie} {[}Übersetzung, 1902-10-17{]}|pwkv}{ }Goldmann\pwindex{Goldmann, Paul 31.01.1865 – 25.09.1935@\textsc{Goldmann, Paul} (31.01.1865 – 25.09.1935), \emph{Schriftsteller, Journalist}|pwk}s fand am 5. 3. 1903 im Deutschen
                     Landestheater\oindex{Staendetheater@\textbf{Ständetheater}|pwk} in Prag\oindex{Prag@\textbf{Prag}|pwk} statt. Eine
                  Aufführung am Wien\oindex{Wien@\textbf{Wien}|pwk}er Volkstheater\oindex{Volkstheater@\textbf{Volkstheater}|pwk} fand nicht statt.}}}\label{K_L03225-4h} dem Volkstheater\orgindex{Volkstheater@Volkstheater|pw} einreichen? Mit \textsc{Schlenther\pwindex{Schlenther, Paul 20.08.1854 – 30.04.1916@\textsc{Schlenther, Paul} (20.08.1854 – 30.04.1916), \emph{Schriftsteller, Kritiker, Theaterleiter}|pw}\orgindex{Burgtheater@Burgtheater|pwv}} will ich nichts zu thun haben.\pend
           \pstart
           Iſt \label{K_L03225-5v}\edtext{\textsc{Olga\pwindex{Schnitzler, Olga 17.01.1882 – 13.01.1970@\textsc{Schnitzler, Olga} (17.01.1882 – 13.01.1970), \emph{Schauspielerin, Sängerin}|pw}} wieder ganz geſund}{\lemma{\textnormal{\emph{Olga wieder ganz geſund}}}\Cendnote{\textnormal{siehe Paul Goldmann an Arthur Schnitzler, 2. [10. 1902]}}}\label{K_L03225-5h}?\pend
           \pstart
           Ich denke auch, die \label{K_L03225-6v}\edtext{»Zeit\pwindex{Zeit1902-09-27 – 1919@\emph{Die Zeit} {[}1902-09-27 – 1919{]}|pw}«}{\lemma{\textnormal{\emph{»Zeit«}}}\Cendnote{\textnormal{siehe Paul Goldmann an Arthur Schnitzler, 16. 9. [1902]}}}\label{K_L03225-6h} wird ſich noch ſehr gut machen. Die \label{K_L03225-7v}\edtext{N. Fr. Pr.\orgindex{Neue Freie Presse@Neue Freie Presse|pw}}{\lemma{\textnormal{\emph{N. Fr. Pr.}}}\Cendnote{\textnormal{In welcher spezifischen Weise bei der
                     \emph{Neuen Freie Presse}\orgindex{Neue Freie Presse@Neue Freie Presse|pwk} in den ersten zwei
                  Wochen nach dem ersten Erscheinen der ersten Nummer der Tageszeitung \emph{Die Zeit}\pwindex{Zeit1902-09-27 – 1919@\emph{Die Zeit} {[}1902-09-27 – 1919{]}|pwk} Entspannung eingetrat, ließ sich
                  nicht ermitteln. Siehe auch Paul Goldmann an Arthur Schnitzler und Olga
               Gussmann, 7. 7. [1901].}}}\label{K_L03225-7h} frohlockt zu früh. Viele treue Grüße!\pend
           \pstart Dein \spacefill\mbox{Paul Goldm}\pend{}
         
         \endnumbering\mylabel{h}\end{ledgroupsized}  \newcommand{\dateiname}{L03225}\newcommand{\titel}{Paul Goldmann an Arthur Schnitzler, 6. 10. [1902]}\newcommand{\editorInnen}{Martin Anton Müller und Laura Untner}%% latex-leseansicht-abspann.tex
%% Abspann für die Leseansicht.
%% Der Schalter \ifkorrekturansicht ist bereits durch den Vorspann gesetzt.

%% latex-abspann.tex
%% Gemeinsamer Abspann für Korrekturansicht und Leseansicht.
%% Setzt den Schalter \ifkorrekturansicht voraus (gesetzt in den
%% einbindenden Dateien latex-korrekturansicht-abspann.tex bzw.
%% latex-leseansicht-abspann.tex).
%% ---------------------------------------------------------------

\normalsize

% Das esempio-Environment wird nur in der Leseansicht benötigt
\ifkorrekturansicht\else
\newenvironment{esempio}[3]%
{
    \vspace{1.5ex}
    \rlap{\underline{#1}}
    \par
    \setlength{\parindent}{0cm}
    \nopagebreak
    \leftskip=#2cm
    \rightskip=#3cm
}
{
    \par
}
\fi

\doendnotes{C}
\bigskip
\vfill

\clearpage

\footnotesize

\ifkorrekturansicht
  \lohead{\textsc{register}}
\fi

% theindex-Environment neu definieren ohne reledmac
\makeatletter
\renewenvironment{theindex}{%
  \ifkorrekturansicht
    \section*{\indexname}%
  \else
    \subsubsection*{Index der erwähnten Entitäten}%
  \fi
  \setlength{\parindent}{0pt}%
  \setlength{\parskip}{0pt plus 0.3pt}%
  \let\item\@idxitem
}{%
  \ifkorrekturansicht\clearpage\fi
}
\makeatother

\IfFileExists{\jobname-pw.ind}{\input{\jobname-pw.ind}}{}

% Quellenangabe nur in der Leseansicht
\ifkorrekturansicht\else
% Fallback-Definitionen, falls die .tex-Datei \titel etc. nicht gesetzt hat
\providecommand{\titel}{}
\providecommand{\editorInnen}{}
\providecommand{\dateiname}{\jobname}

\vspace{3cm}

\vfill

\footnotesize
\textsc{Quelle}: \titel. Herausgegeben von {\editorInnen}. In: \emph{Arthur Schnitzler: Briefwechsel mit Autorinnen und Autoren}.
 Digitale Edition, https://schnitzler-briefe.acdh.oeaw.ac.at/{\dateiname}.html (Stand \today)
\fi

\end{document}


      