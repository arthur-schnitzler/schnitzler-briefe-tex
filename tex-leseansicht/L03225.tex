%% latex-korrekturansicht-vorspann.tex
%% Vorspann für die Korrekturansicht.
%% Lädt die gemeinsame Datei latex-vorspann.tex mit gesetztem Schalter.

\newif\ifkorrekturansicht
\korrekturansichttrue

\input{../tex-inputs/latex-vorspann}


\section[ Paul Goldmann an Arthur Schnitzler, 6. 10. {[}1902{]}]{L03225 Paul Goldmann an Arthur Schnitzler, 6. 10. {[}1902{]}}
\nopagebreak\mylabel{L03225v}
\rehead{ }\normalsize\beginnumbering\briefempfaengerindex{Schnitzler, Arthur@\textsc{Schnitzler, Arthur}!zzzGoldmann, Paul@\emph{von Paul Goldmann}!1902-10-061@{6. 10. {[}1902{]}}|(be}
\toendnotes[C]{\smallbreak\pagebreak[2]}\Standort{DLA, A:Schnitzler, HS.NZ85.1.3172.}
\physDesc{Brief, 1 Blatt, 2 Seiten, 651 Zeichen
\newline{}Handschrift: blaue Tinte, deutsche Kurrent
\newline{}Schnitzler: 1) mit Bleistift das Jahr »902« vermerkt  2) mit rotem Buntstift vier Unterstreichungen}\toendnotes[C]{\smallbreak}
\pstart
           \raggedleft{}{\pb}\textcolor{gray}{\textbf{DESSAUERSTRASSE 19}}\oindex{Dessauer Strasse@\textbf{Dessauer Straße}, \emph{Straße (K.STR)}|pw}\pend
           
\pstart
           Berlin\oindex{Berlin@\textbf{Berlin}, \emph{P.PPLC}|pw}, 6. Okt.\pend
           
\pstart\center{}Mein lieber Freund,\pend\vspace{0.5em}
\pstart
           Mit \textsc{Lindau\pwindex{Lindau, Paul 03.06.1839 – 31.01.1919@\textsc{Lindau, Paul} (03.06.1839 – 31.01.1919), \emph{Schriftsteller/Schriftstellerin, Kritiker/Kritikerin, Theaterleiter/Theaterleiterin}|pw}} ſtehe ich gegenwärtig ſehr ſchlecht. Die Gründe erzähle ich Dir mündlich. Ich
               kann ihm alſo das \label{K_L03225-1v}\edtext{Stück\pwindex{einsame Weg. Schauspiel in fuenf Akten@\emph{Der einsame Weg. Schauspiel in fünf Akten}|pwuv}}{\lemma{\textnormal{\emph{Stück}}}\Cendnote{\textnormal{Von welchem Stück die Rede war, ist
                  ungeklärt. Es dürfte sich nicht um \emph{Der Schleier
                     der Beatrice}\pwindex{Schleier der Beatrice. Schauspiel in fuenf Akten@\emph{Der Schleier der Beatrice. Schauspiel in fünf Akten}|pwk} gehandelt haben, da Paul
                     Lindau\pwindex{Lindau, Paul 03.06.1839 – 31.01.1919@\textsc{Lindau, Paul} (03.06.1839 – 31.01.1919), \emph{Schriftsteller/Schriftstellerin, Kritiker/Kritikerin, Theaterleiter/Theaterleiterin}|pwk} bereits in einem Brief an Schnitzler vom 11. 9. 1900 das Stück\pwindex{Schleier der Beatrice. Schauspiel in fuenf Akten@\emph{Der Schleier der Beatrice. Schauspiel in fünf Akten}|pwkv} für das \emph{Berliner Theater}\orgindex{Berliner Theater@Berliner Theater|pwk} abgelehnt hatte (vgl.
                        \emph{Cambridge University Library}, B 60). Eventuell
                  handelte es sich um das zum Zeitpunkt noch nicht fertiggestellte nächste Stück,
                     \emph{Der einsame Weg}\pwindex{einsame Weg. Schauspiel in fuenf Akten@\emph{Der einsame Weg. Schauspiel in fünf Akten}|pwk}, an dessen viertem Akt Schnitzler zuletzt gearbeitet hatte.}}}\label{K_L03225-1}
               einſtweilen nicht einreichen. Aber wie \textsc{Lindau\pwindex{Lindau, Paul 03.06.1839 – 31.01.1919@\textsc{Lindau, Paul} (03.06.1839 – 31.01.1919), \emph{Schriftsteller/Schriftstellerin, Kritiker/Kritikerin, Theaterleiter/Theaterleiterin}|pw}} ſchon iſt, kann ſich die Situation raſch ändern; und dann ſtehe ich
               ſelbſtverſtändlich zu Deiner Verfügung.\pend
           
\pstart
           \label{K_L03225-2v}\edtext{\textsc{Felix\pwindex{Felix, Hugo 19.11.1866 – 25.08.1934@\textsc{Felix, Hugo} (19.11.1866 – 25.08.1934), \emph{Komponist/Komponistin, Chemiker/Chemikerin}|pw}}}{\lemma{\textnormal{\emph{Felix}}}\Cendnote{\textnormal{Siehe Paul Goldmann an Arthur Schnitzler, 2. [10. 1902].
               }}}\label{K_L03225-2} habe ich Deine Antwort {\pb}übermittelt; er ſandte
               mir ein ganz beglücktes Telegramm.\pend
           
\pstart
           \label{K_L03225-3v}\edtext{\textsc{Fulda\pwindex{Kaltwasser. Lustspiel in drei Aufzuegen@\emph{Kaltwasser. Lustspiel in drei Aufzügen}|pwv}\pwindex{Fulda, Ludwig 15.07.1862 – 30.03.1939@\textsc{Fulda, Ludwig} (15.07.1862 – 30.03.1939), \emph{Schriftsteller/Schriftstellerin, Übersetzer/Übersetzerin}|pw}}}{\lemma{\textnormal{\emph{Fulda}}}\Cendnote{\textnormal{Ludwig Fuldas\pwindex{Fulda, Ludwig 15.07.1862 – 30.03.1939@\textsc{Fulda, Ludwig} (15.07.1862 – 30.03.1939), \emph{Schriftsteller/Schriftstellerin, Übersetzer/Übersetzerin}|pwk} dreiaktiges Lustspiel \emph{Kaltwasser}\pwindex{Kaltwasser. Lustspiel in drei Aufzuegen@\emph{Kaltwasser. Lustspiel in drei Aufzügen}|pwk} hatte am 5. 10. 1902 die Uraufführung am Berlin\oindex{Berlin@\textbf{Berlin}, \emph{P.PPLC}|pwk}er \emph{Lessing-Theater}\orgindex{Lessing-Theater@Lessing-Theater|pwk}.}}}\label{K_L03225-3} iſt
               bös durchgefallen.\pend
           
\pstart
           Kann ich die \label{K_L03225-4v}\edtext{\textsc{Musset\pwindex{ne faut jurer de rien@\emph{Il ne faut jurer de rien}|pwv}\pwindex{Musset, Alfred de 11.12.1810 – 02.05.1857@\textsc{Musset, Alfred de} (11.12.1810 – 02.05.1857), \emph{Schriftsteller/Schriftstellerin}|pw}}-Überſetzung\pwindex{Man soll nichts verschwoeren. Komoedie in 3 Akten@\emph{Man soll nichts verschwören. Komödie in 3 Akten}|pwv}}{\lemma{\textnormal{\emph{Musset-Überſetzung}}}\Cendnote{\textnormal{Alfred de Musset\pwindex{Musset, Alfred de 11.12.1810 – 02.05.1857@\textsc{Musset, Alfred de} (11.12.1810 – 02.05.1857), \emph{Schriftsteller/Schriftstellerin}|pwk}: \emph{Man soll nichts verschwören}\pwindex{Man soll nichts verschwoeren. Komoedie in 3 Akten@\emph{Man soll nichts verschwören. Komödie in 3 Akten}|pwk}. Aus dem Französischen von
                        Paul Goldmann\pwindex{Goldmann, Paul 31.01.1865 – 25.09.1935@\textsc{Goldmann, Paul} (31.01.1865 – 25.09.1935), \emph{Schriftsteller/Schriftstellerin, Journalist/Journalistin}|pwk}. Frankfurt am Main\oindex{Frankfurt am Main@\textbf{Frankfurt am Main}, \emph{P.PPLA3}|pwk}: \emph{Rütten
                           {\kaufmannsund} Loening}\orgindex{Ruetten und Loening@Rütten {\kaufmannsund}  Loening|pwk}{ }1902. Die Uraufführung des Stücks\pwindex{ne faut jurer de rien@\emph{Il ne faut jurer de rien}|pwkv} in der Übersetzung\pwindex{Man soll nichts verschwoeren. Komoedie in 3 Akten@\emph{Man soll nichts verschwören. Komödie in 3 Akten}|pwkv}{ }Goldmanns\pwindex{Goldmann, Paul 31.01.1865 – 25.09.1935@\textsc{Goldmann, Paul} (31.01.1865 – 25.09.1935), \emph{Schriftsteller/Schriftstellerin, Journalist/Journalistin}|pwk} fand am 5. 3. 1903 im Deutschen
                     Landestheater\oindex{Staendetheater@\textbf{Ständetheater}, \emph{Theater (K.THE)}|pwk} in Prag\oindex{Prag@\textbf{Prag}, \emph{A.ADM1}|pwk} statt. Eine
                  Aufführung am Wien\oindex{Wien@\textbf{Wien}, \emph{A.ADM2}|pwk}er Volkstheater\oindex{Volkstheater@\textbf{Volkstheater}, \emph{Theater (K.THE)}|pwk} fand nicht statt.}}}\label{K_L03225-4} dem Volkstheater\orgindex{Volkstheater@Volkstheater|pw} einreichen? Mit \textsc{Schlenther\pwindex{Schlenther, Paul 20.08.1854 – 30.04.1916@\textsc{Schlenther, Paul} (20.08.1854 – 30.04.1916), \emph{Schriftsteller/Schriftstellerin, Kritiker/Kritikerin, Theaterleiter/Theaterleiterin}|pw}\orgindex{Burgtheater@Burgtheater|pwv}} will ich nichts zu thun haben.\pend
           
\pstart
           Iſt \label{K_L03225-5v}\edtext{\textsc{Olga\pwindex{Schnitzler, Olga 17.01.1882 – 13.01.1970@\textsc{Schnitzler, Olga} (17.01.1882 – 13.01.1970), \emph{Schauspieler/Schauspielerin, Sänger/Sängerin}|pw}} wieder ganz geſund}{\lemma{\textnormal{\emph{Olga wieder ganz geſund}}}\Cendnote{\textnormal{Siehe Paul Goldmann an Arthur Schnitzler, 2. [10. 1902].
               }}}\label{K_L03225-5}?\pend
           
\pstart
           Ich denke auch, die \label{K_L03225-6v}\edtext{»Zeit\pwindex{Zeit@\emph{Die Zeit}|pw}«}{\lemma{\textnormal{\emph{»Zeit«}}}\Cendnote{\textnormal{Siehe Paul Goldmann an Arthur Schnitzler, 16. 9. [1902].
               }}}\label{K_L03225-6} wird ſich noch ſehr gut machen. Die \label{K_L03225-7v}\edtext{N. Fr. Pr.\orgindex{Neue Freie Presse@Neue Freie Presse|pw}}{\lemma{\textnormal{\emph{N. Fr. Pr.}}}\Cendnote{\textnormal{In welcher spezifischen Weise bei der
                     \emph{Neuen Freie Presse}\orgindex{Neue Freie Presse@Neue Freie Presse|pwk} in den ersten zwei
                  Wochen nach dem ersten Erscheinen der ersten Nummer der Tageszeitung \emph{Die Zeit}\pwindex{Zeit@\emph{Die Zeit}|pwk} Entspannung eingetreten ist, ließ sich
                  nicht ermitteln. Siehe auch Paul Goldmann an Arthur Schnitzler und Olga
               Gussmann, 7. 7. [1901].}}}\label{K_L03225-7} frohlockt zu früh. Viele treue Grüße!\pend
           \pstart Dein \spacefill\mbox{Paul Goldm}\pend{}\selectlanguage{ngerman}\endnumbering\briefempfaengerindex{Schnitzler, Arthur@\textsc{Schnitzler, Arthur}!zzzGoldmann, Paul@\emph{von Paul Goldmann}!1902-10-061@{6. 10. {[}1902{]}}|)be}\mylabel{L03225h}  \normalsize

\doendnotes{C}
\bigskip
\vfill

\clearpage

\footnotesize

\lohead{\textsc{register}}

% Definiere theindex-Environment komplett neu ohne reledmac
\makeatletter
\renewenvironment{theindex}{%
  \section*{\indexname}%
  \setlength{\parindent}{0pt}%
  \setlength{\parskip}{0pt plus 0.3pt}%
  \let\item\@idxitem
}{%
  \clearpage
}
\makeatother

\IfFileExists{\jobname-pw.ind}{\input{\jobname-pw.ind}}{}

\end{document}

      