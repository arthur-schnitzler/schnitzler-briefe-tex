\input{../tex-inputs/latex-pdf-vorspann}
\begin{center}
            \textcolor{red}{ENTWURF. ENTZIFFERUNG NOCH NICHT KORREKTURGELESEN}
                      \end{center}
            
               \section[Arthur Schnitzler an Richard Dehmel, 16. 2. 1902]{ Arthur Schnitzler an Richard Dehmel, 16. 2. 1902}\nopagebreak\mylabel{v}\rehead{ }\begin{ledgroupsized}[t]{13cm}\normalsize\beginnumbering\briefempfaengerindex{Dehmel, Richard@\textsc{Dehmel, Richard}!zzzSchnitzler, Arthur@\emph{von Arthur Schnitzler}!1902-02-161@{16. 2. 1902}|(be} \toendnotes[C]{\smallbreak\pagebreak[2]} \Standort{Hamburg, Staats- und Universitätsbibliothek, DA:Br:S:617.}
\physDesc{Briefkarte
\newline{}Handschrift: schwarze Tinte, deutsche Kurrent}\toendnotes[C]{\smallbreak}\pstart
           \noindent{}{\pb}Verehrteſter Herr Dehmel, die letzten
                    Worte der zwei letzten Zeilen lauten\pend
           \pstart
           – Zeit –\pwindex{Schnitzler, Arthur 15.05.1862 – 21.10.1931@\textsc{Schnitzler, Arthur} (15.05.1862 – 21.10.1931), \emph{Schriftsteller, Mediziner}!Schleier der Beatrice. Schauspiel in fuenf Akten1900-12-01 – 1900-12-01@\strich\emph{Der Schleier der Beatrice. Schauspiel in fünf Akten} {[}1900-12-01 – 1900-12-01{]}|pwv}\pend
           \pstart
           – weit.\pwindex{Schnitzler, Arthur 15.05.1862 – 21.10.1931@\textsc{Schnitzler, Arthur} (15.05.1862 – 21.10.1931), \emph{Schriftsteller, Mediziner}!Schleier der Beatrice. Schauspiel in fuenf Akten1900-12-01 – 1900-12-01@\strich\emph{Der Schleier der Beatrice. Schauspiel in fünf Akten} {[}1900-12-01 – 1900-12-01{]}|pwv}\pend
           \pstart
           Herzlich grüßend Ihr ſehr ergebener{\\[\baselineskip]}\spacefill\mbox{Arthur Schnitzler}\pend
           \leftskip=0em{}\pstart
           16. 2. 902.\pend
           \endnumbering\briefempfaengerindex{Dehmel, Richard@\textsc{Dehmel, Richard}!zzzSchnitzler, Arthur@\emph{von Arthur Schnitzler}!1902-02-161@{16. 2. 1902}|)be}\mylabel{h}\end{ledgroupsized}  \newcommand{\dateiname}{L01203}\newcommand{\titel}{Arthur Schnitzler an Richard Dehmel, 16. 2. 1902}\newcommand{\editorInnen}{ Martin Anton Müller und Gerd-Hermann Susen}\input{../tex-inputs/latex-pdf-abspann}
      