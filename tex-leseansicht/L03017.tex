%% latex-leseansicht-vorspann.tex
%% Vorspann für die Leseansicht.
%% Lädt die gemeinsame Datei latex-vorspann.tex mit nicht gesetztem Schalter.

\newif\ifkorrekturansicht
\korrekturansichtfalse

\input{../tex-inputs/latex-vorspann}


\section[ Arthur Schnitzler an Felix Salten, 30. 8. 1910]{L03017 Arthur Schnitzler an Felix Salten,  30. 8. 1910}
\nopagebreak\mylabel{L03017v}
\rehead{ }\normalsize\beginnumbering\briefempfaengerindex{Salten, Felix@\textsc{Salten, Felix}!zzzSchnitzler, Arthur@\emph{von Arthur Schnitzler}!1910-08-301@{30. 8. 1910}|(be}
\toendnotes[C]{\smallbreak\pagebreak[2]}
\correspDesc{Versand  durch Arthur Schnitzler am 30. 8. 1910 in Bad Ischl
\newline{}Erhalt  durch Felix Salten im Zeitraum [31. 8. 1910
                  – 4. 9. 1910?] in Unterach am Attersee}\toendnotes[C]{\smallbreak}
\Standort{Wienbibliothek im Rathaus, ZPH 1681, 2.1.516.}
\physDesc{Brief, 1 Blatt, 2 Seiten, 512 Zeichen
\newline{}Handschrift: Bleistift, deutsche Kurrent
\newline{}Ordnung: mit Bleistift von unbekannter Hand nummeriert: »9« }\toendnotes[C]{\smallbreak}
\pstart
           {\pb}\textcolor{gray}{\textbf{Dr. Arthur Schnitzler}}\hfill 30/8 1910\pend
           
\pstart
           \textcolor{gray}{\textbf{Wien XVIII. Spoettelgasse 7\oindex{Wien@\textbf{Wien}!XVIII., Währing@\textbf{XVIII., Währing}!Edmund-Weiß-Gasse 7@\textbf{Edmund-Weiß-Gasse 7}, \emph{Wohngebäude}|pw}.}}\hfill \textsc{Ischl\oindex{Bad Ischl@\textbf{Bad Ischl}|pw}, Pens. Petter\oindex{Hotel und Pension Rudolfshöhe (Leopold Petter)@\textbf{Hotel und Pension Rudolfshöhe (Leopold Petter)}, \emph{Hotel}|pw}}\pend
           \vspace{0.5em}
\pstart
           lieber,{ }\label{K_L03017-1v}\edtext{Frankfurt\eventindex{Frankfurt am Main@\textbf{Frankfurt am Main}!Uraufführung von Liebelei. Oper, 18.9.1910@Uraufführung von Liebelei. Oper, 18.9.1910|pwv}\oindex{Frankfurt am Main@\textbf{Frankfurt am Main}, \emph{Hauptstadt}|pw} iſt verſchoben,{ }ſo{ }ſind wir alſo doch von \textsc{Partenkirchen\oindex{Partenkirchen@\textbf{Partenkirchen}, \emph{Teil eines besiedelten Ortes}|pw}} über \textsc{München\oindex{München@\textbf{München}|pw}} – \textsc{Salzburg\oindex{Salzburg@\textbf{Salzburg}, \emph{Verwaltungsgebiet}|pw}}{ }hieher\oindex{Bad Ischl@\textbf{Bad Ischl}|pwv}}{\lemma{\textnormal{\emph{Frankfurt … hieher}}}\Cendnote{\textnormal{Die
                  Uraufführung der \emph{Liebelei-Oper}\pwindex{Schnitzler, Arthur 15.\,5.\,1862 Wien – 21.\,10.\,1931 ebd.@\textsc{Schnitzler, Arthur} (15.\,5.\,1862 Wien – 21.\,10.\,1931 ebd.), \emph{Schriftsteller, Mediziner}!Liebelei. Oper in drei Akten@\strich\emph{Liebelei. Oper in drei Akten}|pwk}\eventindex{Frankfurt am Main@\textbf{Frankfurt am Main}!Uraufführung von Liebelei. Oper, 18.9.1910@Uraufführung von Liebelei. Oper, 18.9.1910|pwk}, vertont durch
                     Franz Neumann\pwindex{Neumann, Franz 16.\,6.\,1874 Přerov – 25.\,2.\,1929 Brünn@\textsc{Neumann, Franz} (16.\,6.\,1874 Přerov – 25.\,2.\,1929 Brünn), \emph{Theaterleiter, Komponist, Dirigent}|pwk}, wurde auf den 18. 9. 1910
                  verschoben. Schnitzler hielt sich dafür
                  zwischen 15. 9. 1910
                  und 19. 9. 1910 in
                     Frankfurt am Main\oindex{Frankfurt am Main@\textbf{Frankfurt am Main}, \emph{Hauptstadt}|pwk} auf. In Bad Ischl\oindex{Bad Ischl@\textbf{Bad Ischl}|pwk} war er zwischen 29. 8. 1910 und 5. 9. 1910.}}}\label{K_L03017-1}, wo wir ein paar Tage (bei
                  Mama\pwindex{Schnitzler, Louise 8.\,7.\,1840 Kőszeg – 9.\,9.\,1911 Wien@\textsc{Schnitzler, Louise} (8.\,7.\,1840 Kőszeg – 9.\,9.\,1911 Wien)|pwv}) bleiben wollen. Zu
               größeren Ausflügen fühlen wir uns nicht friſch genug, nach den mancherlei Erregungen
               der letzten Zeit, und{ }ſchlagen Ihnen vor, {\pb}ob
               Sie nicht Beide\pwindex{Salten, Ottilie 7.\,3.\,1868 Prag – 22.\,6.\,1942 Zürich@\textsc{Salten, Ottilie} (7.\,3.\,1868 Prag – 22.\,6.\,1942 Zürich), \emph{Schauspielerin}|pwv} dieſer Tage,
               etwa \label{K_L03017-2v}\edtext{Donnerſtag oder Freitag
                zu uns herüber ko{\geminationm}en}{\lemma{\textnormal{\emph{Donnerstag … kommen}}}\Cendnote{\textnormal{Siehe A. S.: \emph{Tagebuch}, 1. 9. 1910.
               }}}\label{K_L03017-2} möchten?
               Und ob{ }ſich nicht Fiſchers\pwindex{Fischer, Samuel 24.\,12.\,1859 Liptovský Mikuláš – 15.\,10.\,1934 Berlin@\textsc{Fischer, Samuel} (24.\,12.\,1859 Liptovský Mikuláš – 15.\,10.\,1934 Berlin), \emph{Verleger}|pw}\pwindex{Fischer, Hedwig 8.\,9.\,1871 Szczecin – 11.\,4.\,1952 Königstein im Taunus@\textsc{Fischer, Hedwig} (8.\,9.\,1871 Szczecin – 11.\,4.\,1952 Königstein im Taunus)|pw}
               anſchließen wollten? Wir würden uns{ }ſehr freuen. Laſſen Sie baldigſt ein Wort
               hören.\pend
           
\pstart
           Herzlichſt Ihr {\\[\baselineskip]}\spacefill\mbox{A.}\pend
           \leftskip=0em{}\selectlanguage{ngerman}\endnumbering\briefempfaengerindex{Salten, Felix@\textsc{Salten, Felix}!zzzSchnitzler, Arthur@\emph{von Arthur Schnitzler}!1910-08-301@{30. 8. 1910}|)be}\mylabel{L03017h}  \newcommand{\dateiname}{L03017}\newcommand{\titel}{Arthur Schnitzler an Felix Salten, 30. 8. 1910}\newcommand{\editorInnen}{Martin Anton Müller und Laura Untner}%% latex-leseansicht-abspann.tex
%% Abspann für die Leseansicht.
%% Der Schalter \ifkorrekturansicht ist bereits durch den Vorspann gesetzt.

%% latex-abspann.tex
%% Gemeinsamer Abspann für Korrekturansicht und Leseansicht.
%% Setzt den Schalter \ifkorrekturansicht voraus (gesetzt in den
%% einbindenden Dateien latex-korrekturansicht-abspann.tex bzw.
%% latex-leseansicht-abspann.tex).
%% ---------------------------------------------------------------

\normalsize

% Das esempio-Environment wird nur in der Leseansicht benötigt
\ifkorrekturansicht\else
\newenvironment{esempio}[3]%
{
    \vspace{1.5ex}
    \rlap{\underline{#1}}
    \par
    \setlength{\parindent}{0cm}
    \nopagebreak
    \leftskip=#2cm
    \rightskip=#3cm
}
{
    \par
}
\fi

\doendnotes{C}
\bigskip
\vfill

\clearpage

\footnotesize

\ifkorrekturansicht
  \lohead{\textsc{register}}
\fi

% theindex-Environment neu definieren ohne reledmac
\makeatletter
\renewenvironment{theindex}{%
  \ifkorrekturansicht
    \section*{\indexname}%
  \else
    \subsubsection*{Index der erwähnten Entitäten}%
  \fi
  \setlength{\parindent}{0pt}%
  \setlength{\parskip}{0pt plus 0.3pt}%
  \let\item\@idxitem
}{%
  \ifkorrekturansicht\clearpage\fi
}
\makeatother

\IfFileExists{\jobname-pw.ind}{\input{\jobname-pw.ind}}{}

% Quellenangabe nur in der Leseansicht
\ifkorrekturansicht\else
% Fallback-Definitionen, falls die .tex-Datei \titel etc. nicht gesetzt hat
\providecommand{\titel}{}
\providecommand{\editorInnen}{}
\providecommand{\dateiname}{\jobname}

\vspace{3cm}

\vfill

\footnotesize
\textsc{Quelle}: \titel. Herausgegeben von {\editorInnen}. In: \emph{Arthur Schnitzler: Briefwechsel mit Autorinnen und Autoren}.
 Digitale Edition, https://schnitzler-briefe.acdh.oeaw.ac.at/{\dateiname}.html (Stand \today)
\fi

\end{document}


