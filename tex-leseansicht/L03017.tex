%% latex-korrekturansicht-vorspann.tex
%% Vorspann für die Korrekturansicht.
%% Lädt die gemeinsame Datei latex-vorspann.tex mit gesetztem Schalter.

\newif\ifkorrekturansicht
\korrekturansichttrue

\input{../tex-inputs/latex-vorspann}


\section[ Arthur Schnitzler an Felix Salten, 30. 8. 1910]{L03017 Arthur Schnitzler an Felix Salten, 30. 8. 1910}
\nopagebreak\mylabel{L03017v}
\rehead{ }\normalsize\beginnumbering\briefempfaengerindex{Salten, Felix@\textsc{Salten, Felix}!zzzSchnitzler, Arthur@\emph{von Arthur Schnitzler}!1910-08-301@{30. 8. 1910}|(be}
\toendnotes[C]{\smallbreak\pagebreak[2]}\Standort{Wienbibliothek im Rathaus, ZPH 1681, 2.1.516.}
\physDesc{Brief, 1 Blatt, 2 Seiten, 512 Zeichen
\newline{}Handschrift: Bleistift, deutsche Kurrent
\newline{}Ordnung: mit Bleistift von unbekannter Hand nummeriert: »9« }\toendnotes[C]{\smallbreak}
\pstart
           {\pb}\textcolor{gray}{\textbf{Dr. Arthur Schnitzler}}\hfill 30/8 1910\pend
           
\pstart
           \textcolor{gray}{\textbf{Wien XVIII. Spoettelgasse 7\oindex{Edmund-Weiss-Gasse 7@\textbf{Edmund-Weiß-Gasse 7}, \emph{Wohngebäude (K.WHS)}|pw}.}}\hfill \textsc{Ischl\oindex{Bad Ischl@\textbf{Bad Ischl}, \emph{P.PPL}|pw}, Pens. Petter\oindex{Hotel und Pension Rudolfshoehe (Leopold Petter)@\textbf{Hotel und Pension Rudolfshöhe (Leopold Petter)}, \emph{Hotel (K.HTL)}|pw}}\pend
           \vspace{0.5em}
\pstart
           lieber,{ }\label{K_L03017-1v}\edtext{Frankfurt\oindex{Frankfurt am Main@\textbf{Frankfurt am Main}, \emph{P.PPLA3}|pw} iſt verſchoben, ſo ſind wir alſo doch von \textsc{Partenkirchen\oindex{Partenkirchen@\textbf{Partenkirchen}, \emph{Teil eines besiedelten Ortes (A.BSOX)}|pw}} über \textsc{München\oindex{Muenchen@\textbf{München}, \emph{P.PPLA}|pw}} – \textsc{Salzburg\oindex{Salzburg@\textbf{Salzburg}, \emph{A.ADM2}|pw}}{ }hieher\oindex{Bad Ischl@\textbf{Bad Ischl}, \emph{P.PPL}|pwv}}{\lemma{\textnormal{\emph{Frankfurt … hieher}}}\Cendnote{\textnormal{Die
                  Uraufführung der \emph{Liebelei-Oper}\pwindex{Liebelei. Oper in drei Akten@\emph{Liebelei. Oper in drei Akten}|pwk}, vertont durch
                     Franz Neumann\pwindex{Neumann, Franz 16.06.1874 – 25.02.1929@\textsc{Neumann, Franz} (16.06.1874 – 25.02.1929), \emph{Theaterleiter/Theaterleiterin, Komponist/Komponistin, Dirigent/Dirigentin}|pwk}, wurde auf den 18. 9. 1910
                  verschoben. Schnitzler hielt sich dafür
                  zwischen 15. 9. 1910
                  und 19. 9. 1910 in
                     Frankfurt am Main\oindex{Frankfurt am Main@\textbf{Frankfurt am Main}, \emph{P.PPLA3}|pwk} auf. In Bad Ischl\oindex{Bad Ischl@\textbf{Bad Ischl}, \emph{P.PPL}|pwk} war er zwischen 29. 8. 1910 und 5. 9. 1910.}}}\label{K_L03017-1}, wo wir ein paar Tage (bei
                  Mama\pwindex{Schnitzler, Louise 1840-07-08 – 1911-09-09@\textsc{Schnitzler, Louise} (1840-07-08 – 1911-09-09)|pwv}) bleiben wollen. Zu
               größeren Ausflügen fühlen wir uns nicht friſch genug, nach den mancherlei Erregungen
               der letzten Zeit, und ſchlagen Ihnen vor, {\pb}ob
               Sie nicht Beide\pwindex{Salten, Ottilie 07.03.1868 – 22.06.1942@\textsc{Salten, Ottilie} (07.03.1868 – 22.06.1942), \emph{Schauspieler/Schauspielerin}|pwv} dieſer Tage,
               etwa \label{K_L03017-2v}\edtext{Donnerſtag oder Freitag
                zu uns herüber ko{\geminationm}en}{\lemma{\textnormal{\emph{Donnerſtag … kommen}}}\Cendnote{\textnormal{Siehe A. S.: \emph{Tagebuch}, 1. 9. 1910.
               }}}\label{K_L03017-2} möchten?
               Und ob ſich nicht Fiſchers\pwindex{Fischer, Samuel 24.12.1859 – 15.10.1934@\textsc{Fischer, Samuel} (24.12.1859 – 15.10.1934), \emph{Verleger/Verlegerin}|pw}\pwindex{Fischer, Hedwig 08.09.1871 – 11.04.1952@\textsc{Fischer, Hedwig} (08.09.1871 – 11.04.1952)|pw}
               anſchließen wollten? Wir würden uns ſehr freuen. Laſſen Sie baldigſt ein Wort
               hören.\pend
           
\pstart
           Herzlichſt Ihr {\\[\baselineskip]}\spacefill\mbox{A.}\pend
           \leftskip=0em{}\selectlanguage{ngerman}\endnumbering\briefempfaengerindex{Salten, Felix@\textsc{Salten, Felix}!zzzSchnitzler, Arthur@\emph{von Arthur Schnitzler}!1910-08-301@{30. 8. 1910}|)be}\mylabel{L03017h}  \normalsize

\doendnotes{C}
\bigskip
\vfill

\clearpage

\footnotesize

\lohead{\textsc{register}}

% Definiere theindex-Environment komplett neu ohne reledmac
\makeatletter
\renewenvironment{theindex}{%
  \section*{\indexname}%
  \setlength{\parindent}{0pt}%
  \setlength{\parskip}{0pt plus 0.3pt}%
  \let\item\@idxitem
}{%
  \clearpage
}
\makeatother

\IfFileExists{\jobname-pw.ind}{\input{\jobname-pw.ind}}{}

\end{document}

      