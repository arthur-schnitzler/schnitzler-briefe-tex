%% latex-korrekturansicht-vorspann.tex
%% Vorspann für die Korrekturansicht.
%% Lädt die gemeinsame Datei latex-vorspann.tex mit gesetztem Schalter.

\newif\ifkorrekturansicht
\korrekturansichttrue

\input{../tex-inputs/latex-vorspann}


\section[Theodor Herzl an Arthur Schnitzler, 14. 8. 1885]{L03821 Theodor Herzl an Arthur Schnitzler, 14. 8. 1885}
\nopagebreak\mylabel{L03821v}
\rehead{ }\normalsize\beginnumbering\briefempfaengerindex{Schnitzler, Arthur@\textsc{Schnitzler, Arthur}!zzzHerzl, Theodor@\emph{von Theodor Herzl}!1885-08-141@{14. 8. 1885}|(be}
\toendnotes[C]{\smallbreak\pagebreak[2]}\Standort{CUL, Schnitzler, B 39.}
\physDesc{Brief, 1 Blatt, 1 Seite, 545 Zeichen
\newline{}Handschrift: schwarze Tinte, lateinische Kurrent
\newline{}Ordnung: mit Bleistift von unbekannter Hand nummeriert: »2« }\toendnotes[C]{\smallbreak}
\pstart
           \raggedleft{}{\pb}Hotel Kammer\oindex{Hotel Kammer@\textbf{Hotel Kammer}, \emph{Hotel (K.HTL)}|pw}, 14/8 885\pend
           
\pstart{}Theuerster Doctor!\pend\vspace{0.5em}
\pstart
           Ihre \label{K_L03821-1v}\edtext{Depesche}{\lemma{\textnormal{\emph{Depesche}}}\Cendnote{\textnormal{nicht überliefert}}}\label{K_L03821-1}, Ihren lieben Brief habe ich mit Vergnügen und Bedauern
               erhalten. Letzteres hat seinen Grund darin, dass ich erst morgen mich
               auf die \label{K_L03821-2v}\edtext{holländischen\oindex{Niederlande@\textbf{Niederlande}, \emph{A.PCLI}|pw} Strümpfe}{\lemma{\textnormal{\emph{holländischen Strümpfe}}}\Cendnote{\textnormal{Herzl\pwindex{Herzl, Theodor 1860-05-02 – 1904-07-03@\textsc{Herzl, Theodor} (1860-05-02 – 1904-07-03), \emph{Schriftsteller/Schriftstellerin, Journalist/Journalistin}|pwk} brach am
                     15. 8. 1885 zu einer Studien- und journalistischen Reise
                  nach Belgien\oindex{Belgien@\textbf{Belgien}, \emph{A.PCLI}|pwk} und Holland\oindex{Niederlande@\textbf{Niederlande}, \emph{A.PCLI}|pwk} auf, von der er am
                     11. 9. 1885 zurückkehrte.}}}\label{K_L03821-2} mache. Ich konnte meinen
               guten Alten\pwindex{Herzl, Jeanette 1836-07-28 – 1911-02-20@\textsc{Herzl, Jeanette} (1836-07-28 – 1911-02-20)|pw}\pwindex{Herzl, Jakob 1837-03-14 – 1902-06-09@\textsc{Herzl, Jakob} (1837-03-14 – 1902-06-09), \emph{Bankdirektor/Bankdirektorin, Großkaufmann/Großkauffrau}|pw} nicht diesen \label{K_L03821-3v}\edtext{Tag stehlen}{\lemma{\textnormal{\emph{Tag stehlen}}}\Cendnote{\textnormal{Laut \emph{Tagebuch}\pwindex{Tagebuch@\emph{Tagebuch}|pwk} hatte Schnitzler{ }Herzl\pwindex{Herzl, Theodor 1860-05-02 – 1904-07-03@\textsc{Herzl, Theodor} (1860-05-02 – 1904-07-03), \emph{Schriftsteller/Schriftstellerin, Journalist/Journalistin}|pwk} am 8. 8. 1885 in Kammer\oindex{Kammer@\textbf{Kammer}, \emph{P.PPL}|pwk} getroffen. Möglicherweise war ein Treffen oder ein Stück gemeinsamer Reise am
                     14. 8. 1885 angedacht, dem Tag, an dem Schnitzler von Ischl\oindex{Bad Ischl@\textbf{Bad Ischl}, \emph{P.PPL}|pwk} nach Innsbruck\oindex{Innsbruck@\textbf{Innsbruck}, \emph{A.ADM2}|pwk} reiste. Aber Herzl\pwindex{Herzl, Theodor 1860-05-02 – 1904-07-03@\textsc{Herzl, Theodor} (1860-05-02 – 1904-07-03), \emph{Schriftsteller/Schriftstellerin, Journalist/Journalistin}|pwk}, der in Kammer\oindex{Kammer@\textbf{Kammer}, \emph{P.PPL}|pwk}
                     seine Eltern\pwindex{Herzl, Jeanette 1836-07-28 – 1911-02-20@\textsc{Herzl, Jeanette} (1836-07-28 – 1911-02-20)|pwkv}\pwindex{Herzl, Jakob 1837-03-14 – 1902-06-09@\textsc{Herzl, Jakob} (1837-03-14 – 1902-06-09), \emph{Bankdirektor/Bankdirektorin, Großkaufmann/Großkauffrau}|pwkv} besuchte, trat seine
                     Reise über München\oindex{Muenchen@\textbf{München}, \emph{P.PPLA}|pwk} erst am Folgetag an.}}}\label{K_L03821-3}, obgleich im Tagediebstahl bereits
               Einiges geleistet habe. \pend
           
\pstart
           Ihre Depesche konnte ich nicht erwidern, weil ich Ihre Adresse nicht hatte. –
               Nochmals besten Dank! \pend
           
\pstart
           Ich wünsche Ihnen eine sonnige, vergnügte Fahrt!\pend
           
\pstart
           Glauben Sie an die freundschaftlichen Gefühle Ihres aufrichtig ergebenen{\\[\baselineskip]}\spacefill\mbox{Herzl}\pend
           \leftskip=0em{}\selectlanguage{ngerman}\endnumbering\briefempfaengerindex{Schnitzler, Arthur@\textsc{Schnitzler, Arthur}!zzzHerzl, Theodor@\emph{von Theodor Herzl}!1885-08-141@{14. 8. 1885}|)be}\mylabel{L03821h}
\begin{anhang}
\end{anhang}\normalsize

\doendnotes{C}
\bigskip
\vfill

\clearpage

\footnotesize

\lohead{\textsc{register}}

% Definiere theindex-Environment komplett neu ohne reledmac
\makeatletter
\renewenvironment{theindex}{%
  \section*{\indexname}%
  \setlength{\parindent}{0pt}%
  \setlength{\parskip}{0pt plus 0.3pt}%
  \let\item\@idxitem
}{%
  \clearpage
}
\makeatother

\IfFileExists{\jobname-pw.ind}{\input{\jobname-pw.ind}}{}

\end{document}

      