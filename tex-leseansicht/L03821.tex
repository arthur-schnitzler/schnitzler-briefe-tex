%% latex-leseansicht-vorspann.tex
%% Vorspann für die Leseansicht.
%% Lädt die gemeinsame Datei latex-vorspann.tex mit nicht gesetztem Schalter.

\newif\ifkorrekturansicht
\korrekturansichtfalse

\input{../tex-inputs/latex-vorspann}


\section[Theodor Herzl an Arthur Schnitzler, 14. 8. 1885]{L03821 Theodor Herzl an Arthur Schnitzler, 14. 8. 1885}
\nopagebreak\mylabel{L03821v}
\rehead{ }\normalsize\beginnumbering\briefempfaengerindex{Schnitzler, Arthur@\textsc{Schnitzler, Arthur}!zzzHerzl, Theodor@\emph{von Theodor Herzl}!1885-08-141@{14. 8. 1885}|(be}
\toendnotes[C]{\smallbreak\pagebreak[2]}
\correspDesc{Versand  durch Theodor Herzl am 14. 8. 1885 in Schörfling
\newline{}Erhalt  durch Arthur Schnitzler im Zeitraum [15. 8. 1885
                  – 19. 8. 1885?] in Wien}\toendnotes[C]{\smallbreak}
\Standort{CUL, Schnitzler, B 39.}
\physDesc{Brief, 1 Blatt, 1 Seite, 545 Zeichen
\newline{}Handschrift: schwarze Tinte, lateinische Kurrent
\newline{}Ordnung: mit Bleistift von unbekannter Hand nummeriert: »2« }
\buchAbdrucke{\weitereDrucke{Theodor Herzl: \emph{Briefe und
                        autobiographische Notizen 1866–1895}. Bearbeitet von Johannes Wachten in Zusammenarbeit mit Chaya Harel, Daisy Tycho und Manfred Winkler. Berlin, Frankfurt am Main, Wien: \emph{Propyläen} 1983, S. 199 (Briefe und Tagebücher. Herausgegeben von Alex Bein, Hermann Greive, Moshe Schaerf, Julius H. Schoeps und Johannes Wachten, 1).} }\toendnotes[C]{\smallbreak}
\pstart
           \raggedleft{}{\pb}Hotel Kammer\oindex{Hotel Kammer@\textbf{Hotel Kammer}, \emph{Hotel}|pw}, 14/8 885\pend
           
\pstart{}Theuerster Doctor!\pend\vspace{0.5em}
\pstart
           Ihre \label{K_L03821-1v}\edtext{Depesche}{\lemma{\textnormal{\emph{Depesche}}}\Cendnote{\textnormal{nicht überliefert}}}\label{K_L03821-1}, Ihren lieben Brief habe ich mit Vergnügen und Bedauern
               erhalten. Letzteres hat seinen Grund darin, dass ich erst morgen mich
               auf die \label{K_L03821-2v}\edtext{holländischen\oindex{Niederlande@\textbf{Niederlande}|pw} Strümpfe}{\lemma{\textnormal{\emph{holländischen Strümpfe}}}\Cendnote{\textnormal{Herzl\pwindex{Herzl, Theodor 2.\,5.\,1860 Budapest – 3.\,7.\,1904 Edlach@\textsc{Herzl, Theodor} (2.\,5.\,1860 Budapest – 3.\,7.\,1904 Edlach), \emph{Schriftsteller, Journalist}|pwk} brach am
                     15. 8. 1885 zu einer Studien- und journalistischen Reise
                  nach Belgien\oindex{Belgien@\textbf{Belgien}|pwk} und Holland\oindex{Niederlande@\textbf{Niederlande}|pwk} auf, von der er am
                     11. 9. 1885 zurückkehrte.}}}\label{K_L03821-2} mache. Ich konnte meinen
               guten Alten\pwindex{Herzl, Jeanette 28.\,7.\,1836 Budapest – 20.\,2.\,1911 Wien@\textsc{Herzl, Jeanette} (28.\,7.\,1836 Budapest – 20.\,2.\,1911 Wien)|pw}\pwindex{Herzl, Jakob 14.\,3.\,1837 Zemun – 9.\,6.\,1902 Wien@\textsc{Herzl, Jakob} (14.\,3.\,1837 Zemun – 9.\,6.\,1902 Wien), \emph{Bankdirektor, Großkaufmann}|pw} nicht diesen \label{K_L03821-3v}\edtext{Tag stehlen}{\lemma{\textnormal{\emph{Tag stehlen}}}\Cendnote{\textnormal{Laut \emph{Tagebuch}\pwindex{Schnitzler, Arthur 15.\,5.\,1862 Wien – 21.\,10.\,1931 ebd.@\textsc{Schnitzler, Arthur} (15.\,5.\,1862 Wien – 21.\,10.\,1931 ebd.), \emph{Schriftsteller, Mediziner}!Tagebuch@\strich\emph{Tagebuch}|pwk} hatte Schnitzler{ }Herzl\pwindex{Herzl, Theodor 2.\,5.\,1860 Budapest – 3.\,7.\,1904 Edlach@\textsc{Herzl, Theodor} (2.\,5.\,1860 Budapest – 3.\,7.\,1904 Edlach), \emph{Schriftsteller, Journalist}|pwk} am 8. 8. 1885 in Kammer\oindex{Kammer@\textbf{Kammer}|pwk} getroffen. Möglicherweise war ein Treffen oder ein Stück gemeinsamer Reise am
                     14. 8. 1885 angedacht, dem Tag, an dem Schnitzler von Ischl\oindex{Bad Ischl@\textbf{Bad Ischl}|pwk} nach Innsbruck\oindex{Innsbruck@\textbf{Innsbruck}, \emph{Verwaltungsgebiet}|pwk} reiste. Aber Herzl\pwindex{Herzl, Theodor 2.\,5.\,1860 Budapest – 3.\,7.\,1904 Edlach@\textsc{Herzl, Theodor} (2.\,5.\,1860 Budapest – 3.\,7.\,1904 Edlach), \emph{Schriftsteller, Journalist}|pwk}, der in Kammer\oindex{Kammer@\textbf{Kammer}|pwk}
                     seine Eltern\pwindex{Herzl, Jeanette 28.\,7.\,1836 Budapest – 20.\,2.\,1911 Wien@\textsc{Herzl, Jeanette} (28.\,7.\,1836 Budapest – 20.\,2.\,1911 Wien)|pwkv}\pwindex{Herzl, Jakob 14.\,3.\,1837 Zemun – 9.\,6.\,1902 Wien@\textsc{Herzl, Jakob} (14.\,3.\,1837 Zemun – 9.\,6.\,1902 Wien), \emph{Bankdirektor, Großkaufmann}|pwkv} besuchte, trat seine
                     Reise über München\oindex{München@\textbf{München}|pwk} erst am Folgetag an.}}}\label{K_L03821-3}, obgleich im Tagediebstahl bereits
               Einiges geleistet habe.\pend
           
\pstart
           Ihre Depesche konnte ich nicht erwidern, weil ich Ihre Adresse nicht hatte. –
               Nochmals besten Dank!\pend
           
\pstart
           Ich wünsche Ihnen eine sonnige, vergnügte Fahrt!\pend
           
\pstart
           Glauben Sie an die freundschaftlichen Gefühle Ihres aufrichtig ergebenen{\\[\baselineskip]}\spacefill\mbox{Herzl}\pend
           \leftskip=0em{}\selectlanguage{ngerman}\endnumbering\briefempfaengerindex{Schnitzler, Arthur@\textsc{Schnitzler, Arthur}!zzzHerzl, Theodor@\emph{von Theodor Herzl}!1885-08-141@{14. 8. 1885}|)be}\mylabel{L03821h}
\begin{anhang}
\end{anhang}\newcommand{\dateiname}{L03821}\newcommand{\titel}{Theodor Herzl an Arthur Schnitzler, 14. 8. 1885}\newcommand{\editorInnen}{Selma Jahnke und Martin Anton Müller}%% latex-leseansicht-abspann.tex
%% Abspann für die Leseansicht.
%% Der Schalter \ifkorrekturansicht ist bereits durch den Vorspann gesetzt.

%% latex-abspann.tex
%% Gemeinsamer Abspann für Korrekturansicht und Leseansicht.
%% Setzt den Schalter \ifkorrekturansicht voraus (gesetzt in den
%% einbindenden Dateien latex-korrekturansicht-abspann.tex bzw.
%% latex-leseansicht-abspann.tex).
%% ---------------------------------------------------------------

\normalsize

% Das esempio-Environment wird nur in der Leseansicht benötigt
\ifkorrekturansicht\else
\newenvironment{esempio}[3]%
{
    \vspace{1.5ex}
    \rlap{\underline{#1}}
    \par
    \setlength{\parindent}{0cm}
    \nopagebreak
    \leftskip=#2cm
    \rightskip=#3cm
}
{
    \par
}
\fi

\doendnotes{C}
\bigskip
\vfill

\clearpage

\footnotesize

\ifkorrekturansicht
  \lohead{\textsc{register}}
\fi

% theindex-Environment neu definieren ohne reledmac
\makeatletter
\renewenvironment{theindex}{%
  \ifkorrekturansicht
    \section*{\indexname}%
  \else
    \subsubsection*{Index der erwähnten Entitäten}%
  \fi
  \setlength{\parindent}{0pt}%
  \setlength{\parskip}{0pt plus 0.3pt}%
  \let\item\@idxitem
}{%
  \ifkorrekturansicht\clearpage\fi
}
\makeatother

\IfFileExists{\jobname-pw.ind}{\input{\jobname-pw.ind}}{}

% Quellenangabe nur in der Leseansicht
\ifkorrekturansicht\else
% Fallback-Definitionen, falls die .tex-Datei \titel etc. nicht gesetzt hat
\providecommand{\titel}{}
\providecommand{\editorInnen}{}
\providecommand{\dateiname}{\jobname}

\vspace{3cm}

\vfill

\footnotesize
\textsc{Quelle}: \titel. Herausgegeben von {\editorInnen}. In: \emph{Arthur Schnitzler: Briefwechsel mit Autorinnen und Autoren}.
 Digitale Edition, https://schnitzler-briefe.acdh.oeaw.ac.at/{\dateiname}.html (Stand \today)
\fi

\end{document}


