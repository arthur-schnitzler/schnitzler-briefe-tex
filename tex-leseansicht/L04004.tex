%% latex-leseansicht-vorspann.tex
%% Vorspann für die Leseansicht.
%% Lädt die gemeinsame Datei latex-vorspann.tex mit nicht gesetztem Schalter.

\newif\ifkorrekturansicht
\korrekturansichtfalse

\input{../tex-inputs/latex-vorspann}


\section[Berta Zuckerkandl an Arthur Schnitzler, {{[}}7. oder 14. 11. 1911?{{]}}]{L04004 Berta Zuckerkandl an Arthur Schnitzler, {[}7. oder 14. 11. 1911?{]}}
\nopagebreak\mylabel{L04004v}
\rehead{ }\normalsize\beginnumbering\briefempfaengerindex{Schnitzler, Arthur@\textsc{Schnitzler, Arthur}!zzzZuckerkandl, Berta@\emph{von Berta Zuckerkandl}!1911-11-142@{{[}7. oder 14. 11. 1911?{]}}|(be}
\toendnotes[C]{\smallbreak\pagebreak[2]}
\correspDesc{Versand  durch Berta Zuckerkandl im Zeitraum [7. oder
                  14. 11. 1911?] in Wien
\newline{}Erhalt  durch Arthur Schnitzler in Wien}\toendnotes[C]{\smallbreak}
\Standort{CUL, Schnitzler, B 200.}
\physDesc{Brief, 1 Blatt, 4 Seiten, 2050 Zeichen
\newline{}Handschrift: schwarze Tinte, lateinische Kurrent
\newline{}Beilage: Ausschnitt aus Le Figaro\pwindex{Le Figaro@\emph{Le Figaro}|pw}
                                 mit Zeitungsartikel von Berta
                                    Zuckerkandl\pwindex{Zuckerkandl, Berta 13.\,4.\,1864 Wien – 16.\,10.\,1945 Paris@\textsc{Zuckerkandl, Berta} (13.\,4.\,1864 Wien – 16.\,10.\,1945 Paris), \emph{Schriftstellerin, Journalistin, Übersetzerin}|pw}: \emph{Courrier des Théatres. Au jour
                                       le jour. Shakespeare disait [...]}\pwindex{Zuckerkandl, Berta 13.\,4.\,1864 Wien – 16.\,10.\,1945 Paris@\textsc{Zuckerkandl, Berta} (13.\,4.\,1864 Wien – 16.\,10.\,1945 Paris), \emph{Schriftstellerin, Journalistin, Übersetzerin}!Courrier des Théatres. Au jour le jour. Shakespeare disait [...] [Über die Wiener Uraufführung von Das weite Land]@\strich\emph{Courrier des Théatres. Au jour le jour. Shakespeare disait [...] [Über die Wiener Uraufführung von Das weite Land]}|pwk}. In: \emph{Le Figaro}\pwindex{Le Figaro@\emph{Le Figaro}|pwk}, Jg. 57, Serie 3, Nr. 304,
                                       31. 10. 1911, S. 6. 
\newline{}Schnitzler: beschriftet: »Zucker« }\toendnotes[C]{\smallbreak}
\pstart
           \raggedleft{}{\pb}\label{K_L04004-1v}\edtext{Dienstag}{\lemma{\textnormal{\emph{Dienstag}}}\Cendnote{\textnormal{Die im Brief erwähnten Schreiben mit
                     Anfragen bezüglich der Übersetzung von \emph{Das
                        weite Land}\pwindex{Schnitzler, Arthur 15. 5. 1862 Wien – 21. 10. 1931 ebd.@\textsc{Schnitzler, Arthur} (15. 5. 1862 Wien – 21. 10. 1931 ebd.), \emph{Schriftsteller, Mediziner}!weite Land. Tragikomödie in fünf Akten@\strich\emph{Das weite Land. Tragikomödie in fünf Akten}|pwk} sind die Briefe von Wilhelm Bauer\pwindex{Bauer, Wilhelm 27.\,11.\,1854 Zollingen – 11.\,9.\,1923 Paris@\textsc{Bauer, Wilhelm} (27.\,11.\,1854 Zollingen – 11.\,9.\,1923 Paris)|pwk} (31. 10. 1911) und Maurice Rémon\pwindex{Rémon, Maurice 27.\,11.\,1861 Paris – 20.\,6.\,1945 Mérignac@\textsc{Rémon, Maurice} (27.\,11.\,1861 Paris – 20.\,6.\,1945 Mérignac), \emph{Übersetzer}|pwk} (1. 11. 1911) an Schnitzler. Dieser hatte Wien\oindex{Wien@\textbf{Wien}, \emph{Verwaltungsgebiet}|pwk} am 30. 10. 1911 Richtung Prag\oindex{Prag@\textbf{Prag}, \emph{Land}|pwk} verlassen und kehrte erst am 17. 11. 1911 nach
                     Stationen in Dresden\oindex{Dresden@\textbf{Dresden}|pwk}, Berlin\oindex{Berlin@\textbf{Berlin}, \emph{Hauptstadt}|pwk}, Hamburg\oindex{Hamburg@\textbf{Hamburg}|pwk}, München\oindex{München@\textbf{München}|pwk} und Garmisch-Partenkirchen\oindex{Garmisch-Partenkirchen@\textbf{Garmisch-Partenkirchen}, \emph{Hauptstadt}|pwk} dorthin zurück, wo er am 18. 11. 1911{ }Berta Zuckerkandl\pwindex{Zuckerkandl, Berta 13.\,4.\,1864 Wien – 16.\,10.\,1945 Paris@\textsc{Zuckerkandl, Berta} (13.\,4.\,1864 Wien – 16.\,10.\,1945 Paris), \emph{Schriftstellerin, Journalistin, Übersetzerin}|pwk} traf und die
                     Angelegenheit besprach. In dieser Zeitspanne, in der Schnitzler die Briefe der Bewerber\pwindex{Rémon, Maurice 27.\,11.\,1861 Paris – 20.\,6.\,1945 Mérignac@\textsc{Rémon, Maurice} (27.\,11.\,1861 Paris – 20.\,6.\,1945 Mérignac), \emph{Übersetzer}|pwkv}\pwindex{Bauer, Wilhelm 27.\,11.\,1854 Zollingen – 11.\,9.\,1923 Paris@\textsc{Bauer, Wilhelm} (27.\,11.\,1854 Zollingen – 11.\,9.\,1923 Paris)|pwkv} auf die Reise
                     nachgesandt und dann von ihm an Zuckerkandl\pwindex{Zuckerkandl, Berta 13.\,4.\,1864 Wien – 16.\,10.\,1945 Paris@\textsc{Zuckerkandl, Berta} (13.\,4.\,1864 Wien – 16.\,10.\,1945 Paris), \emph{Schriftstellerin, Journalistin, Übersetzerin}|pwk} weitergeleitet wurden, liegen zwei Dienstage. Der
                     vorliegende Brief wurde demnach am 7. oder am
                        14. 11. 1911 abgefasst.}}}\label{K_L04004-1}.\pend
           
\pstart{}Verehrter Herr Doktor!\pend\vspace{0.5em}
\pstart
           Anbei sende ich Ihnen \label{K_L04004-2v}\edtext{die Notiz\pwindex{Zuckerkandl, Berta 13.\,4.\,1864 Wien – 16.\,10.\,1945 Paris@\textsc{Zuckerkandl, Berta} (13.\,4.\,1864 Wien – 16.\,10.\,1945 Paris), \emph{Schriftstellerin, Journalistin, Übersetzerin}!Courrier des Théatres. Au jour le jour. Shakespeare disait [...] [Über die Wiener Uraufführung von Das weite Land]@\strich\emph{Courrier des Théatres. Au jour le jour. Shakespeare disait [...] [Über die Wiener Uraufführung von Das weite Land]}|pwv}}{\lemma{\textnormal{\emph{die Notiz}}}\Cendnote{\textnormal{\emph{Courrier des Théatres. Au jour le jour.
                        Shakespeare disait [...]}\pwindex{Zuckerkandl, Berta 13.\,4.\,1864 Wien – 16.\,10.\,1945 Paris@\textsc{Zuckerkandl, Berta} (13.\,4.\,1864 Wien – 16.\,10.\,1945 Paris), \emph{Schriftstellerin, Journalistin, Übersetzerin}!Courrier des Théatres. Au jour le jour. Shakespeare disait [...] [Über die Wiener Uraufführung von Das weite Land]@\strich\emph{Courrier des Théatres. Au jour le jour. Shakespeare disait [...] [Über die Wiener Uraufführung von Das weite Land]}|pwk}. In: \emph{Le
                        Figaro}\pwindex{Le Figaro@\emph{Le Figaro}|pwk}, Jg. 57, Serie 3, Nr. 304, 31. 10. 1911,
                     S. 6, siehe unten.}}}\label{K_L04004-2} die ich im »Figaro\pwindex{Le Figaro@\emph{Le Figaro}|pw}« erscheinen liess. Daraufhin haben Sie die mir \label{K_L04004-3v}\edtext{eingesendeten Briefe}{\lemma{\textnormal{\emph{eingesendeten Briefe}}}\Cendnote{\textnormal{Wilhelm Bauer\pwindex{Bauer, Wilhelm 27.\,11.\,1854 Zollingen – 11.\,9.\,1923 Paris@\textsc{Bauer, Wilhelm} (27.\,11.\,1854 Zollingen – 11.\,9.\,1923 Paris)|pwk} an Arthur
                        Schnitzler, 31. 10. 1911, und Maurice Rémon\pwindex{Rémon, Maurice 27.\,11.\,1861 Paris – 20.\,6.\,1945 Mérignac@\textsc{Rémon, Maurice} (27.\,11.\,1861 Paris – 20.\,6.\,1945 Mérignac), \emph{Übersetzer}|pwk} an Arthur Schnitzler, 1. 11. 1911, vgl. Karl Zieger:
                        \emph{Arthur Schnitzler et la France 1894–1938. Enquête sur une
                        réception}, Villeneuve d’Ascq:
                        \emph{Presses Universitaires du
                           Septentrion} 2012, S. 190.}}}\label{K_L04004-3}{\pb}wol erhalten. Besonders Herrn Rémons\pwindex{Rémon, Maurice 27.\,11.\,1861 Paris – 20.\,6.\,1945 Mérignac@\textsc{Rémon, Maurice} (27.\,11.\,1861 Paris – 20.\,6.\,1945 Mérignac), \emph{Übersetzer}|pw}{ }\uline{Hinweis} auf Guitry\pwindex{Guitry, Lucien 13.\,12.\,1860 Paris – 1.\,6.\,1925 ebd.@\textsc{Guitry, Lucien} (13.\,12.\,1860 Paris – 1.\,6.\,1925 ebd.), \emph{Schriftsteller, Schauspieler}|pw} hat diesen Ursprung – denn ich schrieb die Notiz\pwindex{Zuckerkandl, Berta 13.\,4.\,1864 Wien – 16.\,10.\,1945 Paris@\textsc{Zuckerkandl, Berta} (13.\,4.\,1864 Wien – 16.\,10.\,1945 Paris), \emph{Schriftstellerin, Journalistin, Übersetzerin}!Courrier des Théatres. Au jour le jour. Shakespeare disait [...] [Über die Wiener Uraufführung von Das weite Land]@\strich\emph{Courrier des Théatres. Au jour le jour. Shakespeare disait [...] [Über die Wiener Uraufführung von Das weite Land]}|pwv} eben so – dass Guitry\pwindex{Guitry, Lucien 13.\,12.\,1860 Paris – 1.\,6.\,1925 ebd.@\textsc{Guitry, Lucien} (13.\,12.\,1860 Paris – 1.\,6.\,1925 ebd.), \emph{Schriftsteller, Schauspieler}|pw} aufmerksam werden musste.\pend
           
\pstart
           {\pb}Selbstverständlich möchte ich aber gar
               nicht eine bessere Möglichkeit stören. Und wenn Sie glauben dass Herr Rémon\pwindex{Rémon, Maurice 27.\,11.\,1861 Paris – 20.\,6.\,1945 Mérignac@\textsc{Rémon, Maurice} (27.\,11.\,1861 Paris – 20.\,6.\,1945 Mérignac), \emph{Übersetzer}|pw} rascher zum Ziel ko{\geminationm}t – so trete ich gerne zurück. Die Haupt{\pb}sache bleibt die Aufführung\pwindex{Schnitzler, Arthur 15. 5. 1862 Wien – 21. 10. 1931 ebd.@\textsc{Schnitzler, Arthur} (15. 5. 1862 Wien – 21. 10. 1931 ebd.), \emph{Schriftsteller, Mediziner}!weite Land. Tragikomödie in fünf Akten@\strich\emph{Das weite Land. Tragikomödie in fünf Akten}|pwv}. Sollte aber Herr Rémon\pwindex{Rémon, Maurice 27.\,11.\,1861 Paris – 20.\,6.\,1945 Mérignac@\textsc{Rémon, Maurice} (27.\,11.\,1861 Paris – 20.\,6.\,1945 Mérignac), \emph{Übersetzer}|pw} sich mit mir in Verbindung setzen wollen
               – so bin ich auch dazu bereit. – Bitte handeln Sie (ohne jede Behinderung) ganz nach
               bester Einsicht.\pend
           
\pstart
           \label{T_L04004-1v}\edtext{Herzlichst}{\lemma{\textnormal{\emph{Herzlichst}}}\Cendnote{\textnormal{Sie schreibt: »herzlicht«.}}}\label{T_L04004-1} grüssend {\\[\baselineskip]}\spacefill\mbox{Berta Zuckerkandl}\pend
           \leftskip=0em{}\selectlanguage{ngerman}\vspace{1em}
\pstart
           \noindent{}{\pb}\begin{otherlanguage}{french}\label{K_L04004-4v}\edtext{Shakespeare\pwindex{Shakespeare, William 23.\,4.\,1564? Stratford-upon-Avon – 3.\,5.\,1616 ebd.@\textsc{Shakespeare, William} (23.\,4.\,1564? Stratford-upon-Avon – 3.\,5.\,1616 ebd.), \emph{Schauspieler, Dramatiker}|pw} disait:}{\lemma{\textnormal{\emph{Shakespeare disait:}}}\Cendnote{\textnormal{französisch: Shakespeare\pwindex{Shakespeare, William 23.\,4.\,1564? Stratford-upon-Avon – 3.\,5.\,1616 ebd.@\textsc{Shakespeare, William} (23.\,4.\,1564? Stratford-upon-Avon – 3.\,5.\,1616 ebd.), \emph{Schauspieler, Dramatiker}|pwk} sagte: »Du weißt, eine fremde Seele, ist ein dunkler
                        Wald{\dots}« Für Arthur Schnitzler, den Wiener\oindex{Wien@\textbf{Wien}, \emph{Verwaltungsgebiet}|pwk}
                     Dramatiker (an dessen \emph{\emph{Abschiedssouper}\pwindex{Schnitzler, Arthur 15. 5. 1862 Wien – 21. 10. 1931 ebd.@\textsc{Schnitzler, Arthur} (15. 5. 1862 Wien – 21. 10. 1931 ebd.), \emph{Schriftsteller, Mediziner}!Abschiedssouper@\strich\emph{Abschiedssouper}|pwk}} man sich erinnert), ist die Seele der Menschen ein »weites Land«. Unter
                     diesem Titel hat er gerade am \emph{Burgtheater}\orgindex{Burgtheater@Burgtheater|pwk}
                     in Wien\oindex{Wien@\textbf{Wien}, \emph{Verwaltungsgebiet}|pwk} mit großem Erfolg eine Tragikomödie\pwindex{Schnitzler, Arthur 15. 5. 1862 Wien – 21. 10. 1931 ebd.@\textsc{Schnitzler, Arthur} (15. 5. 1862 Wien – 21. 10. 1931 ebd.), \emph{Schriftsteller, Mediziner}!weite Land. Tragikomödie in fünf Akten@\strich\emph{Das weite Land. Tragikomödie in fünf Akten}|pwkv} gegeben, die
                     ganz Europa\oindex{Europa@\textbf{Europa}|pwk} erobern wird. Die Psychologie
                     der Figuren und die feinsinnige und tiefgründige Beobachtung des menschlichen
                     Bewusstseins werden diesem Stück\pwindex{Schnitzler, Arthur 15. 5. 1862 Wien – 21. 10. 1931 ebd.@\textsc{Schnitzler, Arthur} (15. 5. 1862 Wien – 21. 10. 1931 ebd.), \emph{Schriftsteller, Mediziner}!weite Land. Tragikomödie in fünf Akten@\strich\emph{Das weite Land. Tragikomödie in fünf Akten}|pwkv}, einem Meisterwerk, einen besonderen Platz in der Hochachtung
                     aller Kenner sichern. Die Hauptrolle, ein großer Industrieller, war für Joseph Kainz\pwindex{Kainz, Josef 2.\,1.\,1858 Mosonmagyaróvár – 20.\,9.\,1910 Wien@\textsc{Kainz, Josef} (2.\,1.\,1858 Mosonmagyaróvár – 20.\,9.\,1910 Wien), \emph{Schauspieler}|pwk} bestimmt, den großen
                     Schauspieler, um dessen Verlust Österreich\oindex{Österreich-Ungarn@\textbf{Österreich-Ungarn}|pwk}
                     noch immer trauert!}}}\label{K_L04004-4} »\label{K_L04004-5v}\edtext{L’âme d'autrui, vois-tu, c’est une forêt obscure}{\lemma{\textnormal{\emph{L’âme … obscure}}}\Cendnote{\textnormal{Das Zitat stammt nicht aus Shakespeares\pwindex{Shakespeare, William 23.\,4.\,1564? Stratford-upon-Avon – 3.\,5.\,1616 ebd.@\textsc{Shakespeare, William} (23.\,4.\,1564? Stratford-upon-Avon – 3.\,5.\,1616 ebd.), \emph{Schauspieler, Dramatiker}|pwk} Werken sondern aus dem 17. Kapitel von Ivan Sergeevič Turgenevs\pwindex{Turgenev, Ivan Sergeevič 9.\,11.\,1818 Orjol – 3.\,9.\,1883 Bougival@\textsc{Turgenev, Ivan Sergeevič} (9.\,11.\,1818 Orjol – 3.\,9.\,1883 Bougival), \emph{Schriftsteller}|pwk} Roman \emph{Dvorjanskoe gnezdo}\pwindex{Turgenev, Ivan Sergeevič 9.\,11.\,1818 Orjol – 3.\,9.\,1883 Bougival@\textsc{Turgenev, Ivan Sergeevič} (9.\,11.\,1818 Orjol – 3.\,9.\,1883 Bougival), \emph{Schriftsteller}!Dvorjanskoe gnezdo@\strich\emph{Dvorjanskoe gnezdo}|pwk} (deutsch: \emph{Das adelige Nest}\pwindex{Turgenev, Ivan Sergeevič 9.\,11.\,1818 Orjol – 3.\,9.\,1883 Bougival@\textsc{Turgenev, Ivan Sergeevič} (9.\,11.\,1818 Orjol – 3.\,9.\,1883 Bougival), \emph{Schriftsteller}!adelige Nest@\strich\emph{Das adelige Nest}|pwk}).}}}\label{K_L04004-5}{\dots}« Pour M. Arthur
                     Schnitzler, l’auteur dramatique viennois\oindex{Wien@\textbf{Wien}, \emph{Verwaltungsgebiet}|pw} (dont on se rappelle \emph{Souper d'adieu\pwindex{Schnitzler, Arthur 15. 5. 1862 Wien – 21. 10. 1931 ebd.@\textsc{Schnitzler, Arthur} (15. 5. 1862 Wien – 21. 10. 1931 ebd.), \emph{Schriftsteller, Mediziner}!Souper d’Adieu@\strich\emph{Souper d’Adieu}|pw}\pwindex{Schnitzler, Arthur 15. 5. 1862 Wien – 21. 10. 1931 ebd.@\textsc{Schnitzler, Arthur} (15. 5. 1862 Wien – 21. 10. 1931 ebd.), \emph{Schriftsteller, Mediziner}!Abschiedssouper@\strich\emph{Abschiedssouper}|pw}}), l'âme des hommes est un »vaste pays«. Sous ce titre, il vient de donner au
                     théâttre de la Cour\orgindex{Burgtheater@Burgtheater|pw}, à Vienne\oindex{Wien@\textbf{Wien}, \emph{Verwaltungsgebiet}|pw}, avec un succès retentissant, une tragi-comédie\pwindex{Schnitzler, Arthur 15. 5. 1862 Wien – 21. 10. 1931 ebd.@\textsc{Schnitzler, Arthur} (15. 5. 1862 Wien – 21. 10. 1931 ebd.), \emph{Schriftsteller, Mediziner}!weite Land. Tragikomödie in fünf Akten@\strich\emph{Das weite Land. Tragikomödie in fünf Akten}|pwv} qui fera le tour de l’Europe\oindex{Europa@\textbf{Europa}|pw}. La psychologie des caractères,
                  l’observation qui s’y remarque, délicate et profonde, de la conscience humaine,
                  assureront à cette pièce\pwindex{Schnitzler, Arthur 15. 5. 1862 Wien – 21. 10. 1931 ebd.@\textsc{Schnitzler, Arthur} (15. 5. 1862 Wien – 21. 10. 1931 ebd.), \emph{Schriftsteller, Mediziner}!weite Land. Tragikomödie in fünf Akten@\strich\emph{Das weite Land. Tragikomödie in fünf Akten}|pwv},
                  qui est une œuvre, une place à part dans l'estime de tous les connaisseurs. Le
                  rôle principal, un type de grand industriel, était destiné à Joseph \label{T_L04004-2v}\edtext{Kaïnz}{\lemma{\textnormal{\emph{Kaïnz}}}\Cendnote{\textnormal{Im Text steht:
                        »Haïnz«.}}}\label{T_L04004-2}\pwindex{Kainz, Josef 2.\,1.\,1858 Mosonmagyaróvár – 20.\,9.\,1910 Wien@\textsc{Kainz, Josef} (2.\,1.\,1858 Mosonmagyaróvár – 20.\,9.\,1910 Wien), \emph{Schauspieler}|pw}, le grand comédien dont l’Autriche\oindex{Österreich-Ungarn@\textbf{Österreich-Ungarn}|pw}
                  pleure encore la perte!\end{otherlanguage}\pend
           
\pstart
           \begin{otherlanguage}{french}\label{K_L04004-6v}\edtext{– Enfin, voilà donc une pièce\pwindex{Schnitzler, Arthur 15. 5. 1862 Wien – 21. 10. 1931 ebd.@\textsc{Schnitzler, Arthur} (15. 5. 1862 Wien – 21. 10. 1931 ebd.), \emph{Schriftsteller, Mediziner}!weite Land. Tragikomödie in fünf Akten@\strich\emph{Das weite Land. Tragikomödie in fünf Akten}|pwv} ; une création
                  vraiment belle à tenter! s’écria ce dernier sur son lit de mort, quand on lui lut
                     \emph{le Vaste pays\pwindex{Schnitzler, Arthur 15. 5. 1862 Wien – 21. 10. 1931 ebd.@\textsc{Schnitzler, Arthur} (15. 5. 1862 Wien – 21. 10. 1931 ebd.), \emph{Schriftsteller, Mediziner}!weite Land. Tragikomödie in fünf Akten@\strich\emph{Das weite Land. Tragikomödie in fünf Akten}|pw}}. }{\lemma{\textnormal{\emph{– … pays.}}}\Cendnote{\textnormal{französisch: – Endlich, hier
                     ist also ein Stück\pwindex{Schnitzler, Arthur 15. 5. 1862 Wien – 21. 10. 1931 ebd.@\textsc{Schnitzler, Arthur} (15. 5. 1862 Wien – 21. 10. 1931 ebd.), \emph{Schriftsteller, Mediziner}!weite Land. Tragikomödie in fünf Akten@\strich\emph{Das weite Land. Tragikomödie in fünf Akten}|pwkv}, ein
                     wirklich schönes Werk, das einen Versuch wert ist! rief er auf seinem
                     Sterbebett aus, als man ihm \emph{\emph{Das weite Land}\pwindex{Schnitzler, Arthur 15. 5. 1862 Wien – 21. 10. 1931 ebd.@\textsc{Schnitzler, Arthur} (15. 5. 1862 Wien – 21. 10. 1931 ebd.), \emph{Schriftsteller, Mediziner}!weite Land. Tragikomödie in fünf Akten@\strich\emph{Das weite Land. Tragikomödie in fünf Akten}|pwk}} vorlas.}}}\label{K_L04004-6}\end{otherlanguage}\pend
           
\pstart
           \begin{otherlanguage}{french}\label{K_L04004-7v}\edtext{Joseph \label{T_L04004-3v}\edtext{Kaïnz}{\lemma{\textnormal{\emph{Kaïnz}}}\Cendnote{\textnormal{Im
                        Text steht: »Haïnz«.}}}\label{T_L04004-3}\pwindex{Kainz, Josef 2.\,1.\,1858 Mosonmagyaróvár – 20.\,9.\,1910 Wien@\textsc{Kainz, Josef} (2.\,1.\,1858 Mosonmagyaróvár – 20.\,9.\,1910 Wien), \emph{Schauspieler}|pw} eût été superbe dans \emph{le Vaste Pays\pwindex{Schnitzler, Arthur 15. 5. 1862 Wien – 21. 10. 1931 ebd.@\textsc{Schnitzler, Arthur} (15. 5. 1862 Wien – 21. 10. 1931 ebd.), \emph{Schriftsteller, Mediziner}!weite Land. Tragikomödie in fünf Akten@\strich\emph{Das weite Land. Tragikomödie in fünf Akten}|pw}}. Il eût mis dans tout leur relief les côtés hautains, douloureux, amers,
                  passionnés du personnage. Pour jouer en France\oindex{Frankreich@\textbf{Frankreich}|pw}
                  et en Europe\oindex{Europa@\textbf{Europa}|pw} un tel rôle, il faudrait Lucien Guitry\pwindex{Guitry, Lucien 13.\,12.\,1860 Paris – 1.\,6.\,1925 ebd.@\textsc{Guitry, Lucien} (13.\,12.\,1860 Paris – 1.\,6.\,1925 ebd.), \emph{Schriftsteller, Schauspieler}|pw}; lui seul aurait la puissance
                  de talent nécessaire à cette création. Pendant les entr' actes de la pièce\pwindex{Schnitzler, Arthur 15. 5. 1862 Wien – 21. 10. 1931 ebd.@\textsc{Schnitzler, Arthur} (15. 5. 1862 Wien – 21. 10. 1931 ebd.), \emph{Schriftsteller, Mediziner}!weite Land. Tragikomödie in fünf Akten@\strich\emph{Das weite Land. Tragikomödie in fünf Akten}|pw}, le soir de la première\eventindex{Burgtheater@\textbf{Burgtheater}!Premiere von Das weite Land, 14.10.1911 [I.]@Premiere von Das weite Land, 14.10.1911 [I.]|pwv}, il n’ y avait qu’
                  une voix : »Voilà un rôle pour Lucien
                     Guitry\pwindex{Guitry, Lucien 13.\,12.\,1860 Paris – 1.\,6.\,1925 ebd.@\textsc{Guitry, Lucien} (13.\,12.\,1860 Paris – 1.\,6.\,1925 ebd.), \emph{Schriftsteller, Schauspieler}|pw}! Comme Guitry\pwindex{Guitry, Lucien 13.\,12.\,1860 Paris – 1.\,6.\,1925 ebd.@\textsc{Guitry, Lucien} (13.\,12.\,1860 Paris – 1.\,6.\,1925 ebd.), \emph{Schriftsteller, Schauspieler}|pw} serait beau
                  dans ce personnage!...« Hommage précieux, hommage significatif rendu au grand artiste\pwindex{Guitry, Lucien 13.\,12.\,1860 Paris – 1.\,6.\,1925 ebd.@\textsc{Guitry, Lucien} (13.\,12.\,1860 Paris – 1.\,6.\,1925 ebd.), \emph{Schriftsteller, Schauspieler}|pwv} que l' Europe\oindex{Europa@\textbf{Europa}|pw} entière nous envie! }{\lemma{\textnormal{\emph{Joseph … envie!}}}\Cendnote{\textnormal{Joseph Kainz\pwindex{Kainz, Josef 2.\,1.\,1858 Mosonmagyaróvár – 20.\,9.\,1910 Wien@\textsc{Kainz, Josef} (2.\,1.\,1858 Mosonmagyaróvár – 20.\,9.\,1910 Wien), \emph{Schauspieler}|pwk} wäre großartig in \emph{\emph{Das weite Land}\pwindex{Schnitzler, Arthur 15. 5. 1862 Wien – 21. 10. 1931 ebd.@\textsc{Schnitzler, Arthur} (15. 5. 1862 Wien – 21. 10. 1931 ebd.), \emph{Schriftsteller, Mediziner}!weite Land. Tragikomödie in fünf Akten@\strich\emph{Das weite Land. Tragikomödie in fünf Akten}|pwk}} gewesen. Er hätte die hochmütigen, schmerzhaften, die bitteren und
                     leidenschaftlichen Seiten der Figur voll zur Geltung gebracht. Um eine solche
                     Rolle in Frankreich\oindex{Frankreich@\textbf{Frankreich}|pwk} und Europa\oindex{Europa@\textbf{Europa}|pwk} zu spielen, bräuchte man Lucien Guitry\pwindex{Guitry, Lucien 13.\,12.\,1860 Paris – 1.\,6.\,1925 ebd.@\textsc{Guitry, Lucien} (13.\,12.\,1860 Paris – 1.\,6.\,1925 ebd.), \emph{Schriftsteller, Schauspieler}|pwk}; nur er allein hätte das nötige Talent
                     für diese Kunst. Während der Pausen des \emph{Stücks}\pwindex{Schnitzler, Arthur 15. 5. 1862 Wien – 21. 10. 1931 ebd.@\textsc{Schnitzler, Arthur} (15. 5. 1862 Wien – 21. 10. 1931 ebd.), \emph{Schriftsteller, Mediziner}!weite Land. Tragikomödie in fünf Akten@\strich\emph{Das weite Land. Tragikomödie in fünf Akten}|pwk} am Abend der Premiere\eventindex{Burgtheater@\textbf{Burgtheater}!Premiere von Das weite Land, 14.10.1911 [I.]@Premiere von Das weite Land, 14.10.1911 [I.]|pwkv} gab es nur eine Stimme: »Das ist eine Rolle
                     für Lucien Guitry\pwindex{Guitry, Lucien 13.\,12.\,1860 Paris – 1.\,6.\,1925 ebd.@\textsc{Guitry, Lucien} (13.\,12.\,1860 Paris – 1.\,6.\,1925 ebd.), \emph{Schriftsteller, Schauspieler}|pwk}! Wie gut Guitry\pwindex{Guitry, Lucien 13.\,12.\,1860 Paris – 1.\,6.\,1925 ebd.@\textsc{Guitry, Lucien} (13.\,12.\,1860 Paris – 1.\,6.\,1925 ebd.), \emph{Schriftsteller, Schauspieler}|pwk} in dieser Rolle wäre! {\dots}« Eine prachtvolle Hommage, eine aussagekräftige
                     Hommage an den großen Künstler\pwindex{Guitry, Lucien 13.\,12.\,1860 Paris – 1.\,6.\,1925 ebd.@\textsc{Guitry, Lucien} (13.\,12.\,1860 Paris – 1.\,6.\,1925 ebd.), \emph{Schriftsteller, Schauspieler}|pwkv}, um den uns ganz Europa\oindex{Europa@\textbf{Europa}|pwk}
                     beneidet!}}}\label{K_L04004-7}\end{otherlanguage}\pend
           \selectlanguage{ngerman}\endnumbering\briefempfaengerindex{Schnitzler, Arthur@\textsc{Schnitzler, Arthur}!zzzZuckerkandl, Berta@\emph{von Berta Zuckerkandl}!1911-11-072@{{[}7. oder 14. 11. 1911?{]}}|)be}\mylabel{L04004h}
\begin{anhang}
\end{anhang}\newcommand{\dateiname}{L04004}\newcommand{\titel}{Berta Zuckerkandl an Arthur Schnitzler, [7. oder 14. 11. 1911?]}\newcommand{\editorInnen}{Herausgegeben von Jahnke, SelmaMüller, Martin Anton}%% latex-leseansicht-abspann.tex
%% Abspann für die Leseansicht.
%% Der Schalter \ifkorrekturansicht ist bereits durch den Vorspann gesetzt.

%% latex-abspann.tex
%% Gemeinsamer Abspann für Korrekturansicht und Leseansicht.
%% Setzt den Schalter \ifkorrekturansicht voraus (gesetzt in den
%% einbindenden Dateien latex-korrekturansicht-abspann.tex bzw.
%% latex-leseansicht-abspann.tex).
%% ---------------------------------------------------------------

\normalsize

% Das esempio-Environment wird nur in der Leseansicht benötigt
\ifkorrekturansicht\else
\newenvironment{esempio}[3]%
{
    \vspace{1.5ex}
    \rlap{\underline{#1}}
    \par
    \setlength{\parindent}{0cm}
    \nopagebreak
    \leftskip=#2cm
    \rightskip=#3cm
}
{
    \par
}
\fi

\doendnotes{C}
\bigskip
\vfill

\clearpage

\footnotesize

\ifkorrekturansicht
  \lohead{\textsc{register}}
\fi

% theindex-Environment neu definieren ohne reledmac
\makeatletter
\renewenvironment{theindex}{%
  \ifkorrekturansicht
    \section*{\indexname}%
  \else
    \subsubsection*{Index der erwähnten Entitäten}%
  \fi
  \setlength{\parindent}{0pt}%
  \setlength{\parskip}{0pt plus 0.3pt}%
  \let\item\@idxitem
}{%
  \ifkorrekturansicht\clearpage\fi
}
\makeatother

\IfFileExists{\jobname-pw.ind}{\input{\jobname-pw.ind}}{}

% Quellenangabe nur in der Leseansicht
\ifkorrekturansicht\else
% Fallback-Definitionen, falls die .tex-Datei \titel etc. nicht gesetzt hat
\providecommand{\titel}{}
\providecommand{\editorInnen}{}
\providecommand{\dateiname}{\jobname}

\vspace{3cm}

\vfill

\footnotesize
\textsc{Quelle}: \titel. Herausgegeben von {\editorInnen}. In: \emph{Arthur Schnitzler: Briefwechsel mit Autorinnen und Autoren}.
 Digitale Edition, https://schnitzler-briefe.acdh.oeaw.ac.at/{\dateiname}.html (Stand \today)
\fi

\end{document}


