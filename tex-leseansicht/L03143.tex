%% latex-leseansicht-vorspann.tex
%% Vorspann für die Leseansicht.
%% Lädt die gemeinsame Datei latex-vorspann.tex mit nicht gesetztem Schalter.

\newif\ifkorrekturansicht
\korrekturansichtfalse

\input{../tex-inputs/latex-vorspann}


\section[ Felix Salten an Arthur Schnitzler, 30. 8. 1894]{L03143 Felix Salten an Arthur Schnitzler,  30. 8. 1894}
\nopagebreak\mylabel{L03143v}
\rehead{ }\normalsize\beginnumbering\briefempfaengerindex{Schnitzler, Arthur@\textsc{Schnitzler, Arthur}!zzzSalten, Felix@\emph{von Felix Salten}!1894-08-301@{30. 8. 1894}|(be}
\toendnotes[C]{\smallbreak\pagebreak[2]}
\correspDesc{Versand  durch Felix Salten am 30. 8. 1894 in Wien
\newline{}Erhalt  durch Arthur Schnitzler am 31. 8. 1894 in Bad Ischl}\toendnotes[C]{\smallbreak}
\Standort{CUL, Schnitzler, B 89, A 1.}
\physDesc{Postkarte, 364 Zeichen
\newline{}Handschrift: Bleistift, lateinische Kurrent
\newline{}Versand: 1) Stempel: »\nobreak{}\oindex{III., Landstraße@\textbf{III., Landstraße}, \emph{Verwaltungsgebiet}|pwk}Wien 3/1 66, 30. 8. 94, 11–12V\nobreak{}«.   2) Stempel: »\nobreak{}\oindex{Bad Ischl@\textbf{Bad Ischl}|pwk}Ischl, 31/8 94, 7–F\nobreak{}«. 
\newline{}Ordnung: mit Bleistift von unbekannter Hand nummeriert: »44« }\toendnotes[C]{\smallbreak}\pstart{}{\pb}Herrn D\textsuperscript{r} Arthur Schnitzler\pend{}\pstart{}Ischl\oindex{Bad Ischl@\textbf{Bad Ischl}|pw}\pend{}\pstart{}Pension Leopold Petter.\oindex{Hotel und Pension Rudolfshöhe (Leopold Petter)@\textbf{Hotel und Pension Rudolfshöhe (Leopold Petter)}, \emph{Hotel}|pw}\pend{}{\bigskip}\vspace{1em}
\pstart
           \noindent{}{\pb}Lieber Freund, ich höre soeben, dass im letzten Heft
               der »Zukunft\pwindex{Zukunft@\emph{Die Zukunft}|pw}« ein \label{K_L03143-1v}\edtext{Artikel\pwindex{Marholm, Laura 19.\,4.\,1854 Riga – 6.\,10.\,1928 Jūrmala@\textsc{Marholm, Laura} (19.\,4.\,1854 Riga – 6.\,10.\,1928 Jūrmala), \emph{Schriftstellerin}!Märchen@\strich\emph{Ein Märchen}|pwv}}{\lemma{\textnormal{\emph{Artikel}}}\Cendnote{\textnormal{Laura Marholm\pwindex{Marholm, Laura 19.\,4.\,1854 Riga – 6.\,10.\,1928 Jūrmala@\textsc{Marholm, Laura} (19.\,4.\,1854 Riga – 6.\,10.\,1928 Jūrmala), \emph{Schriftstellerin}|pwk}: \emph{Ein Märchen}\pwindex{Marholm, Laura 19.\,4.\,1854 Riga – 6.\,10.\,1928 Jūrmala@\textsc{Marholm, Laura} (19.\,4.\,1854 Riga – 6.\,10.\,1928 Jūrmala), \emph{Schriftstellerin}!Märchen@\strich\emph{Ein Märchen}|pwk}. In: \emph{Die
                        Zukunft}\pwindex{Zukunft@\emph{Die Zukunft}|pwk}, Jg. 8, 25. 8. 1894,
                     S. 368–371.}}}\label{K_L03143-1} der Laura
                  Marholm\pwindex{Marholm, Laura 19.\,4.\,1854 Riga – 6.\,10.\,1928 Jūrmala@\textsc{Marholm, Laura} (19.\,4.\,1854 Riga – 6.\,10.\,1928 Jūrmala), \emph{Schriftstellerin}|pw} über Ihr »Märchen\pwindex{Schnitzler, Arthur 15.\,5.\,1862 Wien – 21.\,10.\,1931 ebd.@\textsc{Schnitzler, Arthur} (15.\,5.\,1862 Wien – 21.\,10.\,1931 ebd.), \emph{Schriftsteller, Mediziner}!Märchen. Schauspiel in drei Aufzügen@\strich\emph{Das Märchen. Schauspiel in drei Aufzügen}|pw}« steht. Falls
               Sie’s noch nicht gehört haben, zeige ich’s Ihnen an. Der Aufsatz\pwindex{Marholm, Laura 19.\,4.\,1854 Riga – 6.\,10.\,1928 Jūrmala@\textsc{Marholm, Laura} (19.\,4.\,1854 Riga – 6.\,10.\,1928 Jūrmala), \emph{Schriftstellerin}!Märchen@\strich\emph{Ein Märchen}|pwv} soll \label{K_L03143-2v}\edtext{sehr schön u. anerkennend}{\lemma{\textnormal{\emph{sehr … anerkennend}}}\Cendnote{\textnormal{Siehe A. S.: \emph{Tagebuch}, 4. 9. 1894.
               }}}\label{K_L03143-2} sein. Ich werde mich jedesfalls drum kümmern.\pend
           
\pstart
           Grüssen sie Herrn Doktor \label{K_L03143-3v}\edtext{Goldmann\pwindex{Goldmann, Paul 31.\,1.\,1865 Breslau – 25.\,9.\,1935 Wien@\textsc{Goldmann, Paul} (31.\,1.\,1865 Breslau – 25.\,9.\,1935 Wien), \emph{Schriftsteller, Journalist}|pw}}{\lemma{\textnormal{\emph{Goldmann}}}\Cendnote{\textnormal{Schnitzler und Goldmann\pwindex{Goldmann, Paul 31.\,1.\,1865 Breslau – 25.\,9.\,1935 Wien@\textsc{Goldmann, Paul} (31.\,1.\,1865 Breslau – 25.\,9.\,1935 Wien), \emph{Schriftsteller, Journalist}|pwk} hielten sich beide in Ischl\oindex{Bad Ischl@\textbf{Bad Ischl}|pwk} auf.}}}\label{K_L03143-3}.\pend
           \pstart Herzlichst \spacefill\mbox{Salten.}\pend{}\selectlanguage{ngerman}\endnumbering\briefempfaengerindex{Schnitzler, Arthur@\textsc{Schnitzler, Arthur}!zzzSalten, Felix@\emph{von Felix Salten}!1894-08-301@{30. 8. 1894}|)be}\mylabel{L03143h}  \newcommand{\dateiname}{L03143}\newcommand{\titel}{Felix Salten an Arthur Schnitzler, 30. 8. 1894}\newcommand{\editorInnen}{Martin Anton Müller und Laura Untner}%% latex-leseansicht-abspann.tex
%% Abspann für die Leseansicht.
%% Der Schalter \ifkorrekturansicht ist bereits durch den Vorspann gesetzt.

%% latex-abspann.tex
%% Gemeinsamer Abspann für Korrekturansicht und Leseansicht.
%% Setzt den Schalter \ifkorrekturansicht voraus (gesetzt in den
%% einbindenden Dateien latex-korrekturansicht-abspann.tex bzw.
%% latex-leseansicht-abspann.tex).
%% ---------------------------------------------------------------

\normalsize

% Das esempio-Environment wird nur in der Leseansicht benötigt
\ifkorrekturansicht\else
\newenvironment{esempio}[3]%
{
    \vspace{1.5ex}
    \rlap{\underline{#1}}
    \par
    \setlength{\parindent}{0cm}
    \nopagebreak
    \leftskip=#2cm
    \rightskip=#3cm
}
{
    \par
}
\fi

\doendnotes{C}
\bigskip
\vfill

\clearpage

\footnotesize

\ifkorrekturansicht
  \lohead{\textsc{register}}
\fi

% theindex-Environment neu definieren ohne reledmac
\makeatletter
\renewenvironment{theindex}{%
  \ifkorrekturansicht
    \section*{\indexname}%
  \else
    \subsubsection*{Index der erwähnten Entitäten}%
  \fi
  \setlength{\parindent}{0pt}%
  \setlength{\parskip}{0pt plus 0.3pt}%
  \let\item\@idxitem
}{%
  \ifkorrekturansicht\clearpage\fi
}
\makeatother

\IfFileExists{\jobname-pw.ind}{\input{\jobname-pw.ind}}{}

% Quellenangabe nur in der Leseansicht
\ifkorrekturansicht\else
% Fallback-Definitionen, falls die .tex-Datei \titel etc. nicht gesetzt hat
\providecommand{\titel}{}
\providecommand{\editorInnen}{}
\providecommand{\dateiname}{\jobname}

\vspace{3cm}

\vfill

\footnotesize
\textsc{Quelle}: \titel. Herausgegeben von {\editorInnen}. In: \emph{Arthur Schnitzler: Briefwechsel mit Autorinnen und Autoren}.
 Digitale Edition, https://schnitzler-briefe.acdh.oeaw.ac.at/{\dateiname}.html (Stand \today)
\fi

\end{document}


