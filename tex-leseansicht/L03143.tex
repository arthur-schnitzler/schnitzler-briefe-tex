%% latex-korrekturansicht-vorspann.tex
%% Vorspann für die Korrekturansicht.
%% Lädt die gemeinsame Datei latex-vorspann.tex mit gesetztem Schalter.

\newif\ifkorrekturansicht
\korrekturansichttrue

\input{../tex-inputs/latex-vorspann}


\section[ Felix Salten an Arthur Schnitzler, 30. 8. 1894]{L03143 Felix Salten an Arthur Schnitzler, 30. 8. 1894}
\nopagebreak\mylabel{L03143v}
\rehead{ }\normalsize\beginnumbering\briefempfaengerindex{Schnitzler, Arthur@\textsc{Schnitzler, Arthur}!zzzSalten, Felix@\emph{von Felix Salten}!1894-08-301@{30. 8. 1894}|(be}
\toendnotes[C]{\smallbreak\pagebreak[2]}\Standort{CUL, Schnitzler, B 89, A 1.}
\physDesc{Postkarte, 364 Zeichen
\newline{}Handschrift: Bleistift, lateinische Kurrent
\newline{}Versand: 1) Stempel: »\nobreak{}\oindex{III., Landstrasse@\textbf{III., Landstraße}, \emph{A.ADM3}|pwk}Wien 3/1 66, 30. 8. 94, 11–12V\nobreak{}«.   2) Stempel: »\nobreak{}\oindex{Bad Ischl@\textbf{Bad Ischl}, \emph{P.PPL}|pwk}Ischl, 31/8 94, 7–F\nobreak{}«. 
\newline{}Ordnung: mit Bleistift von unbekannter Hand nummeriert: »44« }\toendnotes[C]{\smallbreak}\pstart{}{\pb}Herrn D\textsuperscript{r} Arthur Schnitzler\pend{}\pstart{}Ischl\oindex{Bad Ischl@\textbf{Bad Ischl}, \emph{P.PPL}|pw}\pend{}\pstart{}Pension Leopold Petter.\oindex{Hotel und Pension Rudolfshoehe (Leopold Petter)@\textbf{Hotel und Pension Rudolfshöhe (Leopold Petter)}, \emph{Hotel (K.HTL)}|pw}\pend{}{\bigskip}\vspace{1em}
\pstart
           \noindent{}{\pb}Lieber Freund, ich höre soeben, dass im letzten Heft
               der »Zukunft\pwindex{Zukunft@\emph{Die Zukunft}|pw}« ein \label{K_L03143-1v}\edtext{Artikel\pwindex{Maerchen@\emph{Ein Märchen}|pwv}}{\lemma{\textnormal{\emph{Artikel}}}\Cendnote{\textnormal{Laura Marholm\pwindex{Marholm, Laura 19.04.1854 – 06.10.1928@\textsc{Marholm, Laura} (19.04.1854 – 06.10.1928), \emph{Schriftsteller/Schriftstellerin}|pwk}: \emph{Ein Märchen}\pwindex{Maerchen@\emph{Ein Märchen}|pwk}. In: \emph{Die
                        Zukunft}\pwindex{Zukunft@\emph{Die Zukunft}|pwk}, Jg. 8, 25. 8. 1894,
                     S. 368–371.}}}\label{K_L03143-1} der Laura
                  Marholm\pwindex{Marholm, Laura 19.04.1854 – 06.10.1928@\textsc{Marholm, Laura} (19.04.1854 – 06.10.1928), \emph{Schriftsteller/Schriftstellerin}|pw} über Ihr »Märchen\pwindex{Maerchen. Schauspiel in drei Aufzuegen@\emph{Das Märchen. Schauspiel in drei Aufzügen}|pw}« steht. Falls
               Sie’s noch nicht gehört haben, zeige ich’s Ihnen an. Der Aufsatz\pwindex{Maerchen@\emph{Ein Märchen}|pwv} soll \label{K_L03143-2v}\edtext{sehr schön u. anerkennend}{\lemma{\textnormal{\emph{sehr … anerkennend}}}\Cendnote{\textnormal{Siehe A. S.: \emph{Tagebuch}, 4. 9. 1894.
               }}}\label{K_L03143-2} sein. Ich werde mich jedesfalls drum kümmern.\pend
           
\pstart
           Grüssen sie Herrn Doktor \label{K_L03143-3v}\edtext{Goldmann\pwindex{Goldmann, Paul 31.01.1865 – 25.09.1935@\textsc{Goldmann, Paul} (31.01.1865 – 25.09.1935), \emph{Schriftsteller/Schriftstellerin, Journalist/Journalistin}|pw}}{\lemma{\textnormal{\emph{Goldmann}}}\Cendnote{\textnormal{Schnitzler und Goldmann\pwindex{Goldmann, Paul 31.01.1865 – 25.09.1935@\textsc{Goldmann, Paul} (31.01.1865 – 25.09.1935), \emph{Schriftsteller/Schriftstellerin, Journalist/Journalistin}|pwk} hielten sich beide in Ischl\oindex{Bad Ischl@\textbf{Bad Ischl}, \emph{P.PPL}|pwk} auf.}}}\label{K_L03143-3}.\pend
           \pstart Herzlichst \spacefill\mbox{Salten.}\pend{}\selectlanguage{ngerman}\endnumbering\briefempfaengerindex{Schnitzler, Arthur@\textsc{Schnitzler, Arthur}!zzzSalten, Felix@\emph{von Felix Salten}!1894-08-301@{30. 8. 1894}|)be}\mylabel{L03143h}  \normalsize

\doendnotes{C}
\bigskip
\vfill

\clearpage

\footnotesize

\lohead{\textsc{register}}

% Definiere theindex-Environment komplett neu ohne reledmac
\makeatletter
\renewenvironment{theindex}{%
  \section*{\indexname}%
  \setlength{\parindent}{0pt}%
  \setlength{\parskip}{0pt plus 0.3pt}%
  \let\item\@idxitem
}{%
  \clearpage
}
\makeatother

\IfFileExists{\jobname-pw.ind}{\input{\jobname-pw.ind}}{}

\end{document}

      