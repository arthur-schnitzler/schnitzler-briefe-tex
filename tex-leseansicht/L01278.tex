%% latex-leseansicht-vorspann.tex
%% Vorspann für die Leseansicht.
%% Lädt die gemeinsame Datei latex-vorspann.tex mit nicht gesetztem Schalter.

\newif\ifkorrekturansicht
\korrekturansichtfalse

\input{../tex-inputs/latex-vorspann}


         
         \renewcommand{\erwaehntePersonen}{Personen: Richard Dehmel}
         \renewcommand{\erwaehnteOrte}{Orte: Wien}
         \renewcommand{\erwaehnteWerke}{Werke: Neue Deutsche Rundschau, Zwei Menschen. Roman in Romanzen}
               \section[Arthur Schnitzler an Richard Dehmel, 22. 3. 1903]{ Arthur Schnitzler an Richard Dehmel, 22. 3. 1903}\nopagebreak\mylabel{v}\rehead{ }\begin{ledgroupsized}[t]{13cm}\normalsize\beginnumbering \toendnotes[C]{\smallbreak\pagebreak[2]} \Standort{Hamburg, Staats- und Universitätsbibliothek, DA:Br:S:618.}
\physDesc{Brief, 1 Blatt, 2 Seiten, 434 Zeichen
\newline{}Handschrift: schwarze Tinte, deutsche Kurrent}\toendnotes[C]{\smallbreak}\pstart{}{\pb}Verehrteſter Herr Dehmel,\pend\pstart
           für die freundliche Überſendung Ihres neuen Buches\pwindex{Dehmel, Richard 18.11.1863 – 08.02.1920@\textsc{Dehmel, Richard} (18.11.1863 – 08.02.1920), \emph{Schriftsteller}!Zwei Menschen. Roman in Romanzen1903@\strich\emph{Zwei Menschen. Roman in Romanzen} {[}1903{]}|pwv} danke ich Ihnen herzlich. In der N. D. R.\pwindex{Neue Deutsche Rundschau1894-01-01 – 1903-12-31@\emph{Neue Deutsche Rundschau} {[}1894-01-01 – 1903-12-31{]}|pw} war wohl ein \label{K_L01278_1v}\edtext{Theil\pwindex{Dehmel, Richard 18.11.1863 – 08.02.1920@\textsc{Dehmel, Richard} (18.11.1863 – 08.02.1920), \emph{Schriftsteller}!Zwei Menschen. Roman in Romanzen1903@\strich\emph{Zwei Menschen. Roman in Romanzen} {[}1903{]}|pwv}}{\lemma{\textnormal{\emph{Theil}}}\Cendnote{\textnormal{Im Januar-Heft erschienen mehrere
                  Romanzen (Richard Dehmel\pwindex{Dehmel, Richard 18.11.1863 – 08.02.1920@\textsc{Dehmel, Richard} (18.11.1863 – 08.02.1920), \emph{Schriftsteller}|pwk}: \emph{Zwei Menschen. Romanzen}\pwindex{Dehmel, Richard 18.11.1863 – 08.02.1920@\textsc{Dehmel, Richard} (18.11.1863 – 08.02.1920), \emph{Schriftsteller}!Zwei Menschen. Roman in Romanzen1903@\strich\emph{Zwei Menschen. Roman in Romanzen} {[}1903{]}|pwk}. In: \emph{Neue Deutsche Rundschau}\pwindex{Neue Deutsche Rundschau1894-01-01 – 1903-12-31@\emph{Neue Deutsche Rundschau} {[}1894-01-01 – 1903-12-31{]}|pwk}, Jg. 14, H. 1,
                        15. 1. 1903, S. 49–76).}}}\label{K_L01278_1h} davon abgedruckt; was ich
               dort las, hat mich außerordentlich ergriffen und ich hab es dem allerſchönſten
               zugerechnet, was ich von Ihnen {\pb}kenne. Nun freue ich mich
               ſehr, liebgewonnenes bekanntes \substVorne{}\textsuperscript{\textcolor{gray}{neu}}\substDazwischen{}in\substHinten{} ein\substVorne{}\textsuperscript{e}\substDazwischen{}em\substHinten{} herbeigewünſchte\substVorne{}\textsuperscript{s}\substDazwischen{}n\substHinten{} ganze\substVorne{}\textsuperscript{s}\substDazwischen{}n\substHinten{} aufzunehmen.\pend
           \pstart
           Ihr Sie aufrichtig hochſchätzender{\\[\baselineskip]}\spacefill\mbox{Arthur Schnitzler}\pend
           \leftskip=0em{}\pstart
           Wien\oindex{Wien@\textbf{Wien}|pw}{ }22/3 903\pend
           
         
         \endnumbering\mylabel{h}\end{ledgroupsized}  \newcommand{\dateiname}{L01278}\newcommand{\titel}{Arthur Schnitzler an Richard Dehmel, 22. 3. 1903}\newcommand{\editorInnen}{ Martin Anton Müller und Gerd-Hermann Susen}%% latex-leseansicht-abspann.tex
%% Abspann für die Leseansicht.
%% Der Schalter \ifkorrekturansicht ist bereits durch den Vorspann gesetzt.

%% latex-abspann.tex
%% Gemeinsamer Abspann für Korrekturansicht und Leseansicht.
%% Setzt den Schalter \ifkorrekturansicht voraus (gesetzt in den
%% einbindenden Dateien latex-korrekturansicht-abspann.tex bzw.
%% latex-leseansicht-abspann.tex).
%% ---------------------------------------------------------------

\normalsize

% Das esempio-Environment wird nur in der Leseansicht benötigt
\ifkorrekturansicht\else
\newenvironment{esempio}[3]%
{
    \vspace{1.5ex}
    \rlap{\underline{#1}}
    \par
    \setlength{\parindent}{0cm}
    \nopagebreak
    \leftskip=#2cm
    \rightskip=#3cm
}
{
    \par
}
\fi

\doendnotes{C}
\bigskip
\vfill

\clearpage

\footnotesize

\ifkorrekturansicht
  \lohead{\textsc{register}}
\fi

% theindex-Environment neu definieren ohne reledmac
\makeatletter
\renewenvironment{theindex}{%
  \ifkorrekturansicht
    \section*{\indexname}%
  \else
    \subsubsection*{Index der erwähnten Entitäten}%
  \fi
  \setlength{\parindent}{0pt}%
  \setlength{\parskip}{0pt plus 0.3pt}%
  \let\item\@idxitem
}{%
  \ifkorrekturansicht\clearpage\fi
}
\makeatother

\IfFileExists{\jobname-pw.ind}{\input{\jobname-pw.ind}}{}

% Quellenangabe nur in der Leseansicht
\ifkorrekturansicht\else
% Fallback-Definitionen, falls die .tex-Datei \titel etc. nicht gesetzt hat
\providecommand{\titel}{}
\providecommand{\editorInnen}{}
\providecommand{\dateiname}{\jobname}

\vspace{3cm}

\vfill

\footnotesize
\textsc{Quelle}: \titel. Herausgegeben von {\editorInnen}. In: \emph{Arthur Schnitzler: Briefwechsel mit Autorinnen und Autoren}.
 Digitale Edition, https://schnitzler-briefe.acdh.oeaw.ac.at/{\dateiname}.html (Stand \today)
\fi

\end{document}


      