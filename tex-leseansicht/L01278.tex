%% latex-korrekturansicht-vorspann.tex
%% Vorspann für die Korrekturansicht.
%% Lädt die gemeinsame Datei latex-vorspann.tex mit gesetztem Schalter.

\newif\ifkorrekturansicht
\korrekturansichttrue

\input{../tex-inputs/latex-vorspann}


\section[Arthur Schnitzler an Richard Dehmel, 22. 3. 1903]{L01278 Arthur Schnitzler an Richard Dehmel, 22. 3. 1903}
\nopagebreak\mylabel{L01278v}
\rehead{ }\normalsize\beginnumbering\briefempfaengerindex{Dehmel, Richard@\textsc{Dehmel, Richard}!zzzSchnitzler, Arthur@\emph{von Arthur Schnitzler}!1903-03-221@{22. 3. 1903}|(be}
\toendnotes[C]{\smallbreak\pagebreak[2]}\Standort{Hamburg, Staats- und Universitätsbibliothek, DA:Br:S:618.}
\physDesc{Brief, 1 Blatt, 2 Seiten, 434 Zeichen
\newline{}Handschrift: schwarze Tinte, deutsche Kurrent}\toendnotes[C]{\smallbreak}
\pstart{}{\pb}Verehrteſter Herr Dehmel,\pend\vspace{0.5em}
\pstart
           für die freundliche Überſendung Ihres neuen Buches\pwindex{Zwei Menschen. Roman in Romanzen@\emph{Zwei Menschen. Roman in Romanzen}|pwv} danke ich Ihnen herzlich. In der N. D. R.\pwindex{Neue Deutsche Rundschau@\emph{Neue Deutsche Rundschau}|pw} war wohl ein \label{K_L01278-1v}\edtext{Theil\pwindex{Zwei Menschen. Roman in Romanzen@\emph{Zwei Menschen. Roman in Romanzen}|pwv}}{\lemma{\textnormal{\emph{Theil}}}\Cendnote{\textnormal{Im Januar-Heft erschienen mehrere
                  Romanzen. (Richard Dehmel\pwindex{Dehmel, Richard 18.11.1863 – 08.02.1920@\textsc{Dehmel, Richard} (18.11.1863 – 08.02.1920), \emph{Schriftsteller/Schriftstellerin, Schriftsteller/Schriftstellerin, Krimiautor/Krimiautorin}|pwk}: \emph{Zwei Menschen. Romanzen}\pwindex{Zwei Menschen. Roman in Romanzen@\emph{Zwei Menschen. Roman in Romanzen}|pwk}. In: \emph{Neue Deutsche Rundschau}\pwindex{Neue Deutsche Rundschau@\emph{Neue Deutsche Rundschau}|pwk}, Jg. 14, H. 1,
                        15. 1. 1903, S. 49–76.)}}}\label{K_L01278-1} davon abgedruckt; was ich
               dort las, hat mich außerordentlich ergriffen und ich hab es dem allerſchönſten
               zugerechnet, was ich von Ihnen {\pb}kenne. Nun freue ich mich
               ſehr, liebgewonnenes bekanntes \substVorne{}\textsuperscript{\textcolor{gray}{neu}}\substDazwischen{}in\substHinten{} ein\substVorne{}\textsuperscript{e}\substDazwischen{}em\substHinten{} herbeigewünſchte\substVorne{}\textsuperscript{s}\substDazwischen{}n\substHinten{} ganze\substVorne{}\textsuperscript{s}\substDazwischen{}n\substHinten{} aufzunehmen.\pend
           
\pstart
           Ihr Sie aufrichtig hochſchätzender{\\[\baselineskip]}\spacefill\mbox{Arthur Schnitzler}\pend
           \leftskip=0em{}
\pstart
           Wien\oindex{Wien@\textbf{Wien}, \emph{A.ADM2}|pw}{ }22/3 903\pend
           \selectlanguage{ngerman}\endnumbering\briefempfaengerindex{Dehmel, Richard@\textsc{Dehmel, Richard}!zzzSchnitzler, Arthur@\emph{von Arthur Schnitzler}!1903-03-221@{22. 3. 1903}|)be}\mylabel{L01278h}  \normalsize

\doendnotes{C}
\bigskip
\vfill

\clearpage

\footnotesize

\lohead{\textsc{register}}

% Definiere theindex-Environment komplett neu ohne reledmac
\makeatletter
\renewenvironment{theindex}{%
  \section*{\indexname}%
  \setlength{\parindent}{0pt}%
  \setlength{\parskip}{0pt plus 0.3pt}%
  \let\item\@idxitem
}{%
  \clearpage
}
\makeatother

\IfFileExists{\jobname-pw.ind}{\input{\jobname-pw.ind}}{}

\end{document}

      