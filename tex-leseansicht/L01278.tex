%% latex-leseansicht-vorspann.tex
%% Vorspann für die Leseansicht.
%% Lädt die gemeinsame Datei latex-vorspann.tex mit nicht gesetztem Schalter.

\newif\ifkorrekturansicht
\korrekturansichtfalse

\input{../tex-inputs/latex-vorspann}


\section[Arthur Schnitzler an Richard Dehmel, 22. 3. 1903]{L01278 Arthur Schnitzler an Richard Dehmel, 22. 3. 1903}
\nopagebreak\mylabel{L01278v}
\rehead{ }\normalsize\beginnumbering\briefempfaengerindex{Dehmel, Richard@\textsc{Dehmel, Richard}!zzzSchnitzler, Arthur@\emph{von Arthur Schnitzler}!1903-03-221@{22. 3. 1903}|(be}
\toendnotes[C]{\smallbreak\pagebreak[2]}
\correspDesc{Versand  durch Arthur Schnitzler am 22. 3. 1903 in Wien
\newline{}Erhalt  durch Richard Dehmel im Zeitraum [22. 3. 1903
                  – 26. 3. 1903?] \textbf{Ort fehlend} }\toendnotes[C]{\smallbreak}
\Standort{Hamburg, Staats- und Universitätsbibliothek, DA:Br:S:618.}
\physDesc{Brief, 1 Blatt, 2 Seiten, 434 Zeichen
\newline{}Handschrift: schwarze Tinte, deutsche Kurrent}\toendnotes[C]{\smallbreak}
\pstart{}{\pb}Verehrteſter Herr Dehmel,\pend\vspace{0.5em}
\pstart
           für die freundliche Überſendung Ihres neuen \label{K_L01278-1v}\edtext{Buches\pwindex{Dehmel, Richard 18.\,11.\,1863 Hermsdorf – 8.\,2.\,1920 Blankenese@\textsc{Dehmel, Richard} (18.\,11.\,1863 Hermsdorf – 8.\,2.\,1920 Blankenese), \emph{Schriftsteller, Schriftsteller, Krimiautor}!Zwei Menschen. Roman in Romanzen@\strich\emph{Zwei Menschen. Roman in Romanzen}|pwv}}{\lemma{\textnormal{\emph{Buches}}}\Cendnote{\textnormal{Richard Dehmel\pwindex{Dehmel, Richard 18.\,11.\,1863 Hermsdorf – 8.\,2.\,1920 Blankenese@\textsc{Dehmel, Richard} (18.\,11.\,1863 Hermsdorf – 8.\,2.\,1920 Blankenese), \emph{Schriftsteller, Schriftsteller, Krimiautor}|pwk}:
                     \emph{Zwei Menschen. Roman in Romanzen}\pwindex{Dehmel, Richard 18.\,11.\,1863 Hermsdorf – 8.\,2.\,1920 Blankenese@\textsc{Dehmel, Richard} (18.\,11.\,1863 Hermsdorf – 8.\,2.\,1920 Blankenese), \emph{Schriftsteller, Schriftsteller, Krimiautor}!Zwei Menschen. Roman in Romanzen@\strich\emph{Zwei Menschen. Roman in Romanzen}|pwk}. Berlin: \emph{Schuster {\kaufmannsund} Loeffler}\orgindex{Schuster und Loeffler@Schuster {\kaufmannsund}  Loeffler|pwk}{ }1903.}}}\label{K_L01278-1} danke ich Ihnen herzlich. In der N. D. R.\pwindex{Neue Deutsche Rundschau@\emph{Neue Deutsche Rundschau}|pw} war wohl ein \label{K_L01278-2v}\edtext{Theil\pwindex{Dehmel, Richard 18.\,11.\,1863 Hermsdorf – 8.\,2.\,1920 Blankenese@\textsc{Dehmel, Richard} (18.\,11.\,1863 Hermsdorf – 8.\,2.\,1920 Blankenese), \emph{Schriftsteller, Schriftsteller, Krimiautor}!Zwei Menschen. Roman in Romanzen@\strich\emph{Zwei Menschen. Roman in Romanzen}|pwv}}{\lemma{\textnormal{\emph{Theil}}}\Cendnote{\textnormal{Im Januar-Heft erschienen mehrere
                  Romanzen als Vorabdruck: Richard Dehmel\pwindex{Dehmel, Richard 18.\,11.\,1863 Hermsdorf – 8.\,2.\,1920 Blankenese@\textsc{Dehmel, Richard} (18.\,11.\,1863 Hermsdorf – 8.\,2.\,1920 Blankenese), \emph{Schriftsteller, Schriftsteller, Krimiautor}|pwk}: \emph{Zwei Menschen. Romanzen}\pwindex{Dehmel, Richard 18.\,11.\,1863 Hermsdorf – 8.\,2.\,1920 Blankenese@\textsc{Dehmel, Richard} (18.\,11.\,1863 Hermsdorf – 8.\,2.\,1920 Blankenese), \emph{Schriftsteller, Schriftsteller, Krimiautor}!Zwei Menschen. Romanzen@\strich\emph{Zwei Menschen. Romanzen}|pwk}. In: \emph{Neue Deutsche Rundschau}\pwindex{Neue Deutsche Rundschau@\emph{Neue Deutsche Rundschau}|pwk}, Jg. 14, H. 1,
                        15. 1. 1903, S. 54–76.}}}\label{K_L01278-2} davon abgedruckt; was ich
               dort las, hat mich außerordentlich ergriffen und ich hab es dem allerſchönſten
               zugerechnet, was ich von Ihnen {\pb}kenne. Nun freue ich mich{ }ſehr, liebgewonnenes bekanntes \substVorne{}\textsuperscript{\textcolor{gray}{neu}}\substDazwischen{}in\substHinten{} ein\substVorne{}\textsuperscript{e}\substDazwischen{}em\substHinten{} herbeigewünſchte\substVorne{}\textsuperscript{s}\substDazwischen{}n\substHinten{} ganze\substVorne{}\textsuperscript{s}\substDazwischen{}n\substHinten{} aufzunehmen.\pend
           
\pstart
           Ihr Sie aufrichtig hochſchätzender{\\[\baselineskip]}\spacefill\mbox{Arthur Schnitzler}\pend
           \leftskip=0em{}
\pstart
           Wien\oindex{Wien@\textbf{Wien}, \emph{Verwaltungsgebiet}|pw}{ }22/3 903\pend
           \selectlanguage{ngerman}\endnumbering\briefempfaengerindex{Dehmel, Richard@\textsc{Dehmel, Richard}!zzzSchnitzler, Arthur@\emph{von Arthur Schnitzler}!1903-03-221@{22. 3. 1903}|)be}\mylabel{L01278h}  \newcommand{\dateiname}{L01278}\newcommand{\titel}{Arthur Schnitzler an Richard Dehmel, 22. 3. 1903}\newcommand{\editorInnen}{Martin Anton Müller und Gerd-Hermann Susen}%% latex-leseansicht-abspann.tex
%% Abspann für die Leseansicht.
%% Der Schalter \ifkorrekturansicht ist bereits durch den Vorspann gesetzt.

%% latex-abspann.tex
%% Gemeinsamer Abspann für Korrekturansicht und Leseansicht.
%% Setzt den Schalter \ifkorrekturansicht voraus (gesetzt in den
%% einbindenden Dateien latex-korrekturansicht-abspann.tex bzw.
%% latex-leseansicht-abspann.tex).
%% ---------------------------------------------------------------

\normalsize

% Das esempio-Environment wird nur in der Leseansicht benötigt
\ifkorrekturansicht\else
\newenvironment{esempio}[3]%
{
    \vspace{1.5ex}
    \rlap{\underline{#1}}
    \par
    \setlength{\parindent}{0cm}
    \nopagebreak
    \leftskip=#2cm
    \rightskip=#3cm
}
{
    \par
}
\fi

\doendnotes{C}
\bigskip
\vfill

\clearpage

\footnotesize

\ifkorrekturansicht
  \lohead{\textsc{register}}
\fi

% theindex-Environment neu definieren ohne reledmac
\makeatletter
\renewenvironment{theindex}{%
  \ifkorrekturansicht
    \section*{\indexname}%
  \else
    \subsubsection*{Index der erwähnten Entitäten}%
  \fi
  \setlength{\parindent}{0pt}%
  \setlength{\parskip}{0pt plus 0.3pt}%
  \let\item\@idxitem
}{%
  \ifkorrekturansicht\clearpage\fi
}
\makeatother

\IfFileExists{\jobname-pw.ind}{\input{\jobname-pw.ind}}{}

% Quellenangabe nur in der Leseansicht
\ifkorrekturansicht\else
% Fallback-Definitionen, falls die .tex-Datei \titel etc. nicht gesetzt hat
\providecommand{\titel}{}
\providecommand{\editorInnen}{}
\providecommand{\dateiname}{\jobname}

\vspace{3cm}

\vfill

\footnotesize
\textsc{Quelle}: \titel. Herausgegeben von {\editorInnen}. In: \emph{Arthur Schnitzler: Briefwechsel mit Autorinnen und Autoren}.
 Digitale Edition, https://schnitzler-briefe.acdh.oeaw.ac.at/{\dateiname}.html (Stand \today)
\fi

\end{document}


