\input{../tex-inputs/latex-pdf-vorspann}
\begin{center}
            \textcolor{red}{ENTWURF. ENTZIFFERUNG NOCH NICHT KORREKTURGELESEN}
                      \end{center}
            
               \section[Arthur Schnitzler an Richard Beer-Hofmann, {[}17. 5. 1910?{]}]{ Arthur Schnitzler an Richard Beer-Hofmann, {[}17. 5. 1910?{]}}\nopagebreak\mylabel{v}\rehead{ }\begin{ledgroupsized}[t]{13cm}\normalsize\beginnumbering\briefempfaengerindex{Beer-Hofmann, Richard@\textsc{Beer-Hofmann, Richard}!zzzSchnitzler, Arthur@\emph{von Arthur Schnitzler}!1910-05-171@{{[}17. 5. 1910?{]}}|(be} \toendnotes[C]{\smallbreak\pagebreak[2]} \Standort{YCGL, MSS 31.}
\physDesc{Kartenbrief
\newline{}Handschrift: Bleistift, deutsche Kurrent\newline{}Versand: ohne postalischen Übermittlungsvermerk }\toendnotes[C]{\smallbreak}\pstart{}{\pb}\textsc{Herrn Dr. Rich. Beer-Hofmann}\pend{}\pstart{}Wien\oindex{Wien@\textbf{Wien}|pw}.\pend{}\pstart{}\textsc{XVIII. Hasenauerstr 59\oindex{Hasenauerstrasse@\textbf{Hasenauerstraße}|pw}}\pend{}{\bigskip}\pstart{}{\pb}lieber Richard,\pend\pstart
           Herr \label{K_L01931_1v}\edtext{Rh.\pwindex{Reinhardt, Max 09.09.1873 – 30.10.1943@\textsc{Reinhardt, Max} (09.09.1873 – 30.10.1943), \emph{Theaterleiter, Regisseur, Schauspieler}|pw}}{\lemma{\textnormal{\emph{Rh.}}}\Cendnote{\textnormal{Das Korrespondenzstück ist undatiert.
                  Unter Berücksichtigung der nachgewiesenen Aufenthalte Max Reinhardt\pwindex{Reinhardt, Max 09.09.1873 – 30.10.1943@\textsc{Reinhardt, Max} (09.09.1873 – 30.10.1943), \emph{Theaterleiter, Regisseur, Schauspieler}|pwk}s in Wien\oindex{Wien@\textbf{Wien}|pwk}
                  scheint das Schreiben vom [17. 5. 1910?] auf diesen Brief die Antwort zu geben.}}}\label{K_L01931_1h} wohnt, wie er beiläufig im
               Geſpräch mit mir erwähnte, in einem Hotel \substVorne{}\textsuperscript{a}\substDazwischen{}i\substHinten{}n der Nähe der Nordbahn\oindex{Nordbahnhof@\textbf{Nordbahnhof}|pw} – in welchem weiſs
               ich nicht. Er reiſt Samſtg fort\pend
           \pstart
           Herzlichſt\hspace*{1.5em}Ihr {\\[\baselineskip]}\spacefill\mbox{A.}\pend
           \leftskip=0em{}\endnumbering\briefempfaengerindex{Beer-Hofmann, Richard@\textsc{Beer-Hofmann, Richard}!zzzSchnitzler, Arthur@\emph{von Arthur Schnitzler}!1910-05-171@{{[}17. 5. 1910?{]}}|)be}\mylabel{h}\end{ledgroupsized}  \newcommand{\dateiname}{L01931}\newcommand{\titel}{Arthur Schnitzler an Richard Beer-Hofmann, [17. 5. 1910?]}\newcommand{\editorInnen}{Martin Anton Müller und Gerd-Hermann Susen}\input{../tex-inputs/latex-pdf-abspann}
      