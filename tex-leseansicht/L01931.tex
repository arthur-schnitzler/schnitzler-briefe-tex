%% latex-korrekturansicht-vorspann.tex
%% Vorspann für die Korrekturansicht.
%% Lädt die gemeinsame Datei latex-vorspann.tex mit gesetztem Schalter.

\newif\ifkorrekturansicht
\korrekturansichttrue

\input{../tex-inputs/latex-vorspann}


\section[Arthur Schnitzler an Richard Beer-Hofmann, {[}17. 5. 1910?{]}]{L01931 Arthur Schnitzler an Richard Beer-Hofmann, {[}17. 5. 1910?{]}}
\nopagebreak\mylabel{L01931v}
\rehead{ }\normalsize\beginnumbering\briefempfaengerindex{Beer-Hofmann, Richard@\textsc{Beer-Hofmann, Richard}!zzzSchnitzler, Arthur@\emph{von Arthur Schnitzler}!1910-05-171@{{[}17. 5. 1910?{]}}|(be}
\toendnotes[C]{\smallbreak\pagebreak[2]}\Standort{YCGL, MSS 31.}
\physDesc{Kartenbrief, 233 Zeichen
\newline{}Handschrift: Bleistift, deutsche Kurrent
\newline{}Versand: ohne postalischen Übermittlungsvermerk }\toendnotes[C]{\smallbreak}\pstart{}{\pb}\textsc{Herrn Dr. Rich. Beer-Hofmann}\pend{}\pstart{}Wien\oindex{Wien@\textbf{Wien}, \emph{A.ADM2}|pw}.\pend{}\pstart{}\textsc{XVIII. Hasenauerstr 59\oindex{Hasenauerstrasse 59@\textbf{Hasenauerstraße 59}, \emph{Wohngebäude (K.WHS)}|pw}}\pend{}{\bigskip}\vspace{1em}
\pstart{}{\pb}lieber Richard,\pend\vspace{0.5em}
\pstart
           Herr \label{K_L01931-1v}\edtext{Rh.\pwindex{Reinhardt, Max 09.09.1873 – 30.10.1943@\textsc{Reinhardt, Max} (09.09.1873 – 30.10.1943), \emph{Theaterleiter/Theaterleiterin, Regisseur/Regisseurin, Schauspieler/Schauspielerin}|pw}}{\lemma{\textnormal{\emph{Rh.}}}\Cendnote{\textnormal{Das Korrespondenzstück ist undatiert.
                  Unter Berücksichtigung der nachgewiesenen Aufenthalte Max Reinhardts\pwindex{Reinhardt, Max 09.09.1873 – 30.10.1943@\textsc{Reinhardt, Max} (09.09.1873 – 30.10.1943), \emph{Theaterleiter/Theaterleiterin, Regisseur/Regisseurin, Schauspieler/Schauspielerin}|pwk} in Wien\oindex{Wien@\textbf{Wien}, \emph{A.ADM2}|pwk}
                  scheint das Schreiben vom [17. 5. 1910?] auf diesen Brief die Antwort zu geben.}}}\label{K_L01931-1} wohnt, wie er
               beiläufig im Geſpräch mit mir erwähnte, in einem Hotel \substVorne{}\textsuperscript{a}\substDazwischen{}i\substHinten{}n der Nähe der Nordbahn\oindex{Nordbahnhof@\textbf{Nordbahnhof}, \emph{Bahnhofsgebäude (K.BHF)}|pw} – in welchem
               weiſs ich nicht. Er reiſt Samſtg fort\pend
           
\pstart
           Herzlichſt\hspace*{1.5em}Ihr {\\[\baselineskip]}\spacefill\mbox{A.}\pend
           \leftskip=0em{}\selectlanguage{ngerman}\endnumbering\briefempfaengerindex{Beer-Hofmann, Richard@\textsc{Beer-Hofmann, Richard}!zzzSchnitzler, Arthur@\emph{von Arthur Schnitzler}!1910-05-171@{{[}17. 5. 1910?{]}}|)be}\mylabel{L01931h}  \normalsize

\doendnotes{C}
\bigskip
\vfill

\clearpage

\footnotesize

\lohead{\textsc{register}}

% Definiere theindex-Environment komplett neu ohne reledmac
\makeatletter
\renewenvironment{theindex}{%
  \section*{\indexname}%
  \setlength{\parindent}{0pt}%
  \setlength{\parskip}{0pt plus 0.3pt}%
  \let\item\@idxitem
}{%
  \clearpage
}
\makeatother

\IfFileExists{\jobname-pw.ind}{\input{\jobname-pw.ind}}{}

\end{document}

      