%% latex-korrekturansicht-vorspann.tex
%% Vorspann für die Korrekturansicht.
%% Lädt die gemeinsame Datei latex-vorspann.tex mit gesetztem Schalter.

\newif\ifkorrekturansicht
\korrekturansichttrue

\input{../tex-inputs/latex-vorspann}


\section[ Felix Salten an Arthur Schnitzler, 27. 7. 1895]{L03131 Felix Salten an Arthur Schnitzler, 27. 7. 1895}
\nopagebreak\mylabel{L03131v}
\rehead{ }\normalsize\beginnumbering\briefempfaengerindex{Schnitzler, Arthur@\textsc{Schnitzler, Arthur}!zzzSalten, Felix@\emph{von Felix Salten}!1895-07-271@{27. 7. 1895}|(be}
\toendnotes[C]{\smallbreak\pagebreak[2]}\Standort{CUL, Schnitzler, B 89, A 1.}
\physDesc{Postkarte, 380 Zeichen
\newline{}Handschrift: Bleistift, lateinische Kurrent
\newline{}Versand: 1) Stempel: »\nobreak{}\oindex{IX., Alsergrund@\textbf{IX., Alsergrund}, \emph{A.ADM3}|pwk}Wien 9/3 72, 27. 7. 95, 3–4 N\nobreak{}«.   2) Stempel: »\nobreak{}\oindex{Bad Ischl@\textbf{Bad Ischl}, \emph{P.PPL}|pwk}Ischl, 28/7 95, 7F\nobreak{}«. 
\newline{}Ordnung: mit Bleistift von unbekannter Hand nummeriert: »59« }\toendnotes[C]{\smallbreak}\pstart{}{\pb}Herrn D\textsuperscript{r} Arthur Schnitzler\pend{}\pstart{}Ischl\oindex{Bad Ischl@\textbf{Bad Ischl}, \emph{P.PPL}|pw}.\pend{}\pstart{}Pension Leopold\oindex{Hotel und Pension Rudolfshoehe (Leopold Petter)@\textbf{Hotel und Pension Rudolfshöhe (Leopold Petter)}, \emph{Hotel (K.HTL)}|pw}.\pend{}{\bigskip}\vspace{1em}
\pstart
           \noindent{}{\pb}Lieber Arthur, möglicherweise, ja fast bestimmt komme
               ich \label{K_L03131-1v}\edtext{Montag in 8 Tagen auf einen Tag nach Ischl\oindex{Bad Ischl@\textbf{Bad Ischl}, \emph{P.PPL}|pw}}{\lemma{\textnormal{\emph{Montag … Ischl}}}\Cendnote{\textnormal{Siehe A. S.: \emph{Tagebuch}, 5. 8. 1895.
               }}}\label{K_L03131-1} weswegen ich jedoch keineswegs auf \substVorne{}\textsuperscript{i}\substDazwischen{}I\substHinten{}hren Brief verzichte. Dann können wir ja
               alles weitere besprechen. Die \label{K_L03131-2v}\edtext{Feuilletons\pwindex{Muenchener Kunstausstellungen. I. Im koenigl. Glaspalast@\emph{Die Münchener Kunstausstellungen. I. Im königl. Glaspalast}|pwv}\pwindex{Muenchener Kunstausstellungen. II. Im koenigl. Glaspalast@\emph{Die Münchener Kunstausstellungen. II. Im königl. Glaspalast}|pwv}\pwindex{Muenchener Brief. (Orig.-Corr. der »Wiener Allg. Ztg.«)@\emph{Münchener Brief. (Orig.-Corr. der »Wiener Allg. Ztg.«)}|pwv}}{\lemma{\textnormal{\emph{Feuilletons}}}\Cendnote{\textnormal{Siehe Felix Salten an Arthur Schnitzler, 22. 7. 1895.
               }}}\label{K_L03131-2} laße ich heute noch absenden. \label{K_L03131-3v}\edtext{Rich. Engländer\pwindex{Altenberg, Peter 09.03.1859 – 08.01.1919@\textsc{Altenberg, Peter} (09.03.1859 – 08.01.1919), \emph{Schriftsteller/Schriftstellerin}|pw} wohnt in Gmunden\oindex{Gmunden@\textbf{Gmunden}, \emph{P.PPL}|pw}}{\lemma{\textnormal{\emph{Rich. … Gmunden}}}\Cendnote{\textnormal{Siehe dazu auch Peter Altenberg an Arthur Schnitzler, [30. 7. 1895].
               }}}\label{K_L03131-3} beim »Goldenen Brunnen\oindex{Goldener Brunnen@\textbf{Goldener Brunnen}, \emph{S.HTL}|pw}«. – Auf
               Wiedersehen.{\\}Herzlichst Ihr \spacefill\mbox{Salten}\pend
           \selectlanguage{ngerman}\endnumbering\briefempfaengerindex{Schnitzler, Arthur@\textsc{Schnitzler, Arthur}!zzzSalten, Felix@\emph{von Felix Salten}!1895-07-271@{27. 7. 1895}|)be}\mylabel{L03131h}  \normalsize

\doendnotes{C}
\bigskip
\vfill

\clearpage

\footnotesize

\lohead{\textsc{register}}

% Definiere theindex-Environment komplett neu ohne reledmac
\makeatletter
\renewenvironment{theindex}{%
  \section*{\indexname}%
  \setlength{\parindent}{0pt}%
  \setlength{\parskip}{0pt plus 0.3pt}%
  \let\item\@idxitem
}{%
  \clearpage
}
\makeatother

\IfFileExists{\jobname-pw.ind}{\input{\jobname-pw.ind}}{}

\end{document}

      