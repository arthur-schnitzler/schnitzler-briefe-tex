%% latex-leseansicht-vorspann.tex
%% Vorspann für die Leseansicht.
%% Lädt die gemeinsame Datei latex-vorspann.tex mit nicht gesetztem Schalter.

\newif\ifkorrekturansicht
\korrekturansichtfalse

\input{../tex-inputs/latex-vorspann}

\begin{center}
            \textcolor{red}{ENTWURF, NICHT FERTIG KORRIGIERT}
                      \end{center}
            
         
         \renewcommand{\erwaehntePersonen}{Personen: Peter Altenberg}
         \renewcommand{\erwaehnteOrte}{Orte: Bad Ischl, Gmunden, Goldener Brunnen, Hotel und Pension Rudolfshöhe (Leopold Petter), IX., Alsergrund, Wien}
         \renewcommand{\erwaehnteWerke}{Werke: Die Münchener Kunstausstellungen. I. Im königl. Glaspalast, Die Münchener Kunstausstellungen. II. Im königl. Glaspalast, Münchener Brief. (Orig.-Corr. der »Wiener Allg. Ztg.«)}
               \section[Felix Salten an Arthur Schnitzler, 27. 7. 1895]{ Felix Salten an Arthur Schnitzler, 27. 7. 1895}\nopagebreak\mylabel{v}\rehead{ }\begin{ledgroupsized}[t]{13cm}\normalsize\beginnumbering \toendnotes[C]{\smallbreak\pagebreak[2]} \Standort{CUL, Schnitzler, B 89, A 1.}
\physDesc{Postkarte, 383 Zeichen
\newline{}Handschrift: Bleistift, lateinische Kurrent
\newline{}Versand: 1) Stempel: »\nobreak{}\oindex{IX., Alsergrund@\textbf{IX., Alsergrund}|pwk}Wien 9/3 72, 27. 7. 1895, 3–4N\nobreak{}«.   2) Stempel: »\nobreak{}\oindex{Bad Ischl@\textbf{Bad Ischl}|pwk}Ischl, 28/7. 95, 7{[}–{]}\textcolor{gray}{9}\nobreak{}«. 
\newline{}Ordnung: mit Bleistift von unbekannter Hand nummeriert:
                                    »59« }\toendnotes[C]{\smallbreak}\pstart{}{\pb}Herrn D\textsuperscript{r} Arthur Schnitzler\pend{}\pstart{}Ischl\oindex{Bad Ischl@\textbf{Bad Ischl}|pw}. \pend{}\pstart{}Pension Leopold\oindex{Hotel und Pension Rudolfshoehe (Leopold Petter)@\textbf{Hotel und Pension Rudolfshöhe (Leopold Petter)}|pw}.\pend{}{\bigskip}\pstart
           \noindent{}{\pb}Lieber Arthur, möglicherweise, ja fast bestimmt komme ich Montag in
               8 Tagen auf einen Tag nach Ischl\oindex{Bad Ischl@\textbf{Bad Ischl}|pw}, weswegen ich
               jedoch keineswegs auf Ihren Brief verzichte. Dann können wir ja alles weitere
               besprechen. Die Feuilletons\pwindex{Salten, Felix 06.09.1869 – 08.10.1945@\textsc{Salten, Felix} (06.09.1869 – 08.10.1945), \emph{Schriftsteller, Journalist}!Muenchener Kunstausstellungen. I. Im koenigl. Glaspalast1895-07-24@\strich\emph{Die Münchener Kunstausstellungen. I. Im königl. Glaspalast} {[}1895-07-24{]}|pwv}\pwindex{Salten, Felix 06.09.1869 – 08.10.1945@\textsc{Salten, Felix} (06.09.1869 – 08.10.1945), \emph{Schriftsteller, Journalist}!Muenchener Kunstausstellungen. II. Im koenigl. Glaspalast1895-07-25@\strich\emph{Die Münchener Kunstausstellungen. II. Im königl. Glaspalast} {[}1895-07-25{]}|pwv}\pwindex{Muenchener Brief. (Orig.-Corr. der »Wiener Allg. Ztg.«)1895-07-06@\emph{Münchener Brief. (Orig.-Corr. der »Wiener Allg. Ztg.«)} {[}1895-07-06{]}|pwv} laße ich heute noch absenden. Rich. Engländer\pwindex{Altenberg, Peter 09.03.1859 – 08.01.1919@\textsc{Altenberg, Peter} (09.03.1859 – 08.01.1919), \emph{Schriftsteller}|pw} wohnt in Gmunden\oindex{Gmunden@\textbf{Gmunden}|pw} beim »Goldenen Brunnen\oindex{Goldener Brunnen@\textbf{Goldener Brunnen}|pw}«.\pend
           \pstart
           Auf Wiedersehen.{\\[\baselineskip]}Herzlichst Ihr \spacefill\mbox{Salten}\pend
           \leftskip=0em{}
         
         \endnumbering\mylabel{h}\end{ledgroupsized}\begin{anhang}\end{anhang}\newcommand{\dateiname}{L03131}\newcommand{\titel}{Felix Salten an Arthur Schnitzler, 27. 7. 1895}\newcommand{\editorInnen}{Martin Anton Müller und Laura Untner}%% latex-leseansicht-abspann.tex
%% Abspann für die Leseansicht.
%% Der Schalter \ifkorrekturansicht ist bereits durch den Vorspann gesetzt.

%% latex-abspann.tex
%% Gemeinsamer Abspann für Korrekturansicht und Leseansicht.
%% Setzt den Schalter \ifkorrekturansicht voraus (gesetzt in den
%% einbindenden Dateien latex-korrekturansicht-abspann.tex bzw.
%% latex-leseansicht-abspann.tex).
%% ---------------------------------------------------------------

\normalsize

% Das esempio-Environment wird nur in der Leseansicht benötigt
\ifkorrekturansicht\else
\newenvironment{esempio}[3]%
{
    \vspace{1.5ex}
    \rlap{\underline{#1}}
    \par
    \setlength{\parindent}{0cm}
    \nopagebreak
    \leftskip=#2cm
    \rightskip=#3cm
}
{
    \par
}
\fi

\doendnotes{C}
\bigskip
\vfill

\clearpage

\footnotesize

\ifkorrekturansicht
  \lohead{\textsc{register}}
\fi

% theindex-Environment neu definieren ohne reledmac
\makeatletter
\renewenvironment{theindex}{%
  \ifkorrekturansicht
    \section*{\indexname}%
  \else
    \subsubsection*{Index der erwähnten Entitäten}%
  \fi
  \setlength{\parindent}{0pt}%
  \setlength{\parskip}{0pt plus 0.3pt}%
  \let\item\@idxitem
}{%
  \ifkorrekturansicht\clearpage\fi
}
\makeatother

\IfFileExists{\jobname-pw.ind}{\input{\jobname-pw.ind}}{}

% Quellenangabe nur in der Leseansicht
\ifkorrekturansicht\else
% Fallback-Definitionen, falls die .tex-Datei \titel etc. nicht gesetzt hat
\providecommand{\titel}{}
\providecommand{\editorInnen}{}
\providecommand{\dateiname}{\jobname}

\vspace{3cm}

\vfill

\footnotesize
\textsc{Quelle}: \titel. Herausgegeben von {\editorInnen}. In: \emph{Arthur Schnitzler: Briefwechsel mit Autorinnen und Autoren}.
 Digitale Edition, https://schnitzler-briefe.acdh.oeaw.ac.at/{\dateiname}.html (Stand \today)
\fi

\end{document}


      