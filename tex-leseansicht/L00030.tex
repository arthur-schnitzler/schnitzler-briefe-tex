%% latex-leseansicht-vorspann.tex
%% Vorspann für die Leseansicht.
%% Lädt die gemeinsame Datei latex-vorspann.tex mit nicht gesetztem Schalter.

\newif\ifkorrekturansicht
\korrekturansichtfalse

\input{../tex-inputs/latex-vorspann}


\section[Arthur Schnitzler an Hugo von Hofmannsthal, {[}11. 8. 1891{]}]{L00030 Arthur Schnitzler an Hugo von Hofmannsthal, {[}11. 8. 1891{]}}
\nopagebreak\mylabel{L00030v}
\rehead{ }\normalsize\beginnumbering\briefempfaengerindex{Hofmannsthal, Hugo von@\textsc{Hofmannsthal, Hugo von}!zzzSchnitzler, Arthur@\emph{von Arthur Schnitzler}!1891-08-113@{{[}11. 8. 1891{]}}|(be}
\toendnotes[C]{\smallbreak\pagebreak[2]}
\correspDesc{Versand  durch Arthur Schnitzler am [11. 8. 1891] in Wien
\newline{}Erhalt  durch Hugo von Hofmannsthal im Zeitraum [12. 8. 1891
                  – 16. 8. 1891?] in Strobl}\toendnotes[C]{\smallbreak}
\Standort{FDH, Hs-30885,14.}
\physDesc{Briefkarte, 364 Zeichen
\newline{}Handschrift: schwarze Tinte, deutsche Kurrent
\newline{}Ordnung: mit Bleistift von unbekannter Hand datiert: »Aug 91« }
\buchAbdrucke{\weitereDrucke{Hugo von Hofmannsthal, Arthur Schnitzler: \emph{Briefwechsel}. Herausgegeben von Therese Nickl und Heinrich Schnitzler. Frankfurt am Main: \emph{S. Fischer} 1964, S. 12.} }\toendnotes[C]{\smallbreak}
\pstart
           \noindent{}{\pb}Lieber Loris\hspace*{1.5em}\label{K_L00030-1v}\edtext{eben habe ich an Richard \textsc{Beer-Hofmann}\pwindex{Beer-Hofmann, Richard 11.\,7.\,1866 Wien – 26.\,9.\,1945 New York City@\textsc{Beer-Hofmann, Richard} (11.\,7.\,1866 Wien – 26.\,9.\,1945 New York City), \emph{Schriftsteller}|pw} geſchrieben}{\lemma{\textnormal{\emph{eben … geschrieben}}}\Cendnote{\textnormal{Vgl. XXXX Auszeichnungsfehler: Dokument L00029 nicht gefunden.
               }}}\label{K_L00030-1}, er möge womöglich So{\geminationn}tag 16. 8.{ }Vormittag nach Iſchl\oindex{Bad Ischl@\textbf{Bad Ischl}|pw} herüber zu
               kommen. Da ich{ }ſchon am So{\geminationn}tag{ }Abend wieder nach Wien\oindex{Wien@\textbf{Wien}, \emph{Verwaltungsgebiet}|pw} fahre, wäre es
               reizend {\pb}von Ihnen, auch{ }ſchon So{\geminationn}tag{ }Vormittag nach Iſchl\oindex{Bad Ischl@\textbf{Bad Ischl}|pw} zu{ }ſauſen\introOben{}(\introOben{}, wo ich die Adresse \textsc{\uline{Pension Leopold}}\oindex{Hotel und Pension Rudolfshöhe (Leopold Petter)@\textbf{Hotel und Pension Rudolfshöhe (Leopold Petter)}, \emph{Hotel}|pw} habe\introOben{})\introOben{}.\pend
           
\pstart
           Mit herzlichem Gruſs und in der angenehmen Erwartung Sie zu{ }ſehen{\\[\baselineskip]}Ihr
                  \spacefill\mbox{Arthur}\pend
           \leftskip=0em{}\selectlanguage{ngerman}\endnumbering\briefempfaengerindex{Hofmannsthal, Hugo von@\textsc{Hofmannsthal, Hugo von}!zzzSchnitzler, Arthur@\emph{von Arthur Schnitzler}!1891-08-113@{{[}11. 8. 1891{]}}|)be}\mylabel{L00030h}  \newcommand{\dateiname}{L00030}\newcommand{\titel}{Arthur Schnitzler an Hugo von Hofmannsthal, [11. 8. 1891]}\newcommand{\editorInnen}{Martin Anton Müller und Gerd-Hermann Susen}%% latex-leseansicht-abspann.tex
%% Abspann für die Leseansicht.
%% Der Schalter \ifkorrekturansicht ist bereits durch den Vorspann gesetzt.

%% latex-abspann.tex
%% Gemeinsamer Abspann für Korrekturansicht und Leseansicht.
%% Setzt den Schalter \ifkorrekturansicht voraus (gesetzt in den
%% einbindenden Dateien latex-korrekturansicht-abspann.tex bzw.
%% latex-leseansicht-abspann.tex).
%% ---------------------------------------------------------------

\normalsize

% Das esempio-Environment wird nur in der Leseansicht benötigt
\ifkorrekturansicht\else
\newenvironment{esempio}[3]%
{
    \vspace{1.5ex}
    \rlap{\underline{#1}}
    \par
    \setlength{\parindent}{0cm}
    \nopagebreak
    \leftskip=#2cm
    \rightskip=#3cm
}
{
    \par
}
\fi

\doendnotes{C}
\bigskip
\vfill

\clearpage

\footnotesize

\ifkorrekturansicht
  \lohead{\textsc{register}}
\fi

% theindex-Environment neu definieren ohne reledmac
\makeatletter
\renewenvironment{theindex}{%
  \ifkorrekturansicht
    \section*{\indexname}%
  \else
    \subsubsection*{Index der erwähnten Entitäten}%
  \fi
  \setlength{\parindent}{0pt}%
  \setlength{\parskip}{0pt plus 0.3pt}%
  \let\item\@idxitem
}{%
  \ifkorrekturansicht\clearpage\fi
}
\makeatother

\IfFileExists{\jobname-pw.ind}{\input{\jobname-pw.ind}}{}

% Quellenangabe nur in der Leseansicht
\ifkorrekturansicht\else
% Fallback-Definitionen, falls die .tex-Datei \titel etc. nicht gesetzt hat
\providecommand{\titel}{}
\providecommand{\editorInnen}{}
\providecommand{\dateiname}{\jobname}

\vspace{3cm}

\vfill

\footnotesize
\textsc{Quelle}: \titel. Herausgegeben von {\editorInnen}. In: \emph{Arthur Schnitzler: Briefwechsel mit Autorinnen und Autoren}.
 Digitale Edition, https://schnitzler-briefe.acdh.oeaw.ac.at/{\dateiname}.html (Stand \today)
\fi

\end{document}


