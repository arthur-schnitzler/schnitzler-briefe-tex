%% latex-korrekturansicht-vorspann.tex
%% Vorspann für die Korrekturansicht.
%% Lädt die gemeinsame Datei latex-vorspann.tex mit gesetztem Schalter.

\newif\ifkorrekturansicht
\korrekturansichttrue

\input{../tex-inputs/latex-vorspann}


\section[Arthur Schnitzler an Hugo von Hofmannsthal, {[}11. 8. 1891{]}]{L00030 Arthur Schnitzler an Hugo von Hofmannsthal, {[}11. 8. 1891{]}}
\nopagebreak\mylabel{L00030v}
\rehead{ }\normalsize\beginnumbering\briefempfaengerindex{Hofmannsthal, Hugo von@\textsc{Hofmannsthal, Hugo von}!zzzSchnitzler, Arthur@\emph{von Arthur Schnitzler}!1891-08-113@{{[}11. 8. 1891{]}}|(be}
\toendnotes[C]{\smallbreak\pagebreak[2]}\Standort{FDH, Hs-30885,14.}
\physDesc{Briefkarte, 364 Zeichen
\newline{}Handschrift: schwarze Tinte, deutsche Kurrent
\newline{}Ordnung: mit Bleistift von unbekannter Hand datiert: »Aug 91« }
\buchAbdrucke{\weitereDrucke{Hugo von Hofmannsthal, Arthur Schnitzler: \emph{Briefwechsel}. Frankfurt am Main: \emph{S. Fischer} 1964, S. 12.} }\toendnotes[C]{\smallbreak}
\pstart
           \noindent{}{\pb}Lieber Loris\hspace*{1.5em}\label{K_L00030-1v}\edtext{eben habe ich an Richard \textsc{Beer-Hofmann}\pwindex{Beer-Hofmann, Richard 1866-07-11 – 1945-09-26@\textsc{Beer-Hofmann, Richard} (1866-07-11 – 1945-09-26), \emph{Schriftsteller/Schriftstellerin}|pw} geſchrieben}{\lemma{\textnormal{\emph{eben … geſchrieben}}}\Cendnote{\textnormal{Vgl. Arthur Schnitzler an Richard Beer-Hofmann, 11. 8. 1891.
               }}}\label{K_L00030-1}, er möge womöglich So{\geminationn}tag 16. 8.{ }Vormittag nach Iſchl\oindex{Bad Ischl@\textbf{Bad Ischl}, \emph{P.PPL}|pw} herüber zu
               kommen. Da ich ſchon am So{\geminationn}tag{ }Abend wieder nach Wien\oindex{Wien@\textbf{Wien}, \emph{A.ADM2}|pw} fahre, wäre es
               reizend {\pb}von Ihnen, auch ſchon So{\geminationn}tag{ }Vormittag nach Iſchl\oindex{Bad Ischl@\textbf{Bad Ischl}, \emph{P.PPL}|pw} zu ſauſen\introOben{}(\introOben{}, wo ich die Adresse \textsc{\uline{Pension Leopold}}\oindex{Hotel und Pension Rudolfshoehe (Leopold Petter)@\textbf{Hotel und Pension Rudolfshöhe (Leopold Petter)}, \emph{Hotel (K.HTL)}|pw} habe\introOben{})\introOben{}.\pend
           
\pstart
           Mit herzlichem Gruſs und in der angenehmen Erwartung Sie zu ſehen{\\[\baselineskip]}Ihr
                  \spacefill\mbox{Arthur}\pend
           \leftskip=0em{}\selectlanguage{ngerman}\endnumbering\briefempfaengerindex{Hofmannsthal, Hugo von@\textsc{Hofmannsthal, Hugo von}!zzzSchnitzler, Arthur@\emph{von Arthur Schnitzler}!1891-08-113@{{[}11. 8. 1891{]}}|)be}\mylabel{L00030h}  \normalsize

\doendnotes{C}
\bigskip
\vfill

\clearpage

\footnotesize

\lohead{\textsc{register}}

% Definiere theindex-Environment komplett neu ohne reledmac
\makeatletter
\renewenvironment{theindex}{%
  \section*{\indexname}%
  \setlength{\parindent}{0pt}%
  \setlength{\parskip}{0pt plus 0.3pt}%
  \let\item\@idxitem
}{%
  \clearpage
}
\makeatother

\IfFileExists{\jobname-pw.ind}{\input{\jobname-pw.ind}}{}

\end{document}

      