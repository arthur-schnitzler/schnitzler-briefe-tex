%% latex-leseansicht-vorspann.tex
%% Vorspann für die Leseansicht.
%% Lädt die gemeinsame Datei latex-vorspann.tex mit nicht gesetztem Schalter.

\newif\ifkorrekturansicht
\korrekturansichtfalse

\input{../tex-inputs/latex-vorspann}


         \renewcommand{\erwaehnteOrte}{Orte: Altaussee, Bad Ischl, Gasthaus Breitlahner, Gerlos, Gerlospass, Gossensass, Jenbach, Mayrhofen, Mooserboden, Pfitscher Joch, Salzburg, Schwarzensteingrund, Sterzing, Zell am Ziller, Zemmgrund, Zillertal}
         \renewcommand{\erwaehnteWerke}{Werke: Deutsche Alpen}
               \section[Richard Beer-Hofmann an Arthur Schnitzler, 2. 8. 1900]{ Richard Beer-Hofmann an Arthur Schnitzler, 2. 8. 1900}\nopagebreak\mylabel{v}\rehead{ }\begin{ledgroupsized}[t]{13cm}\normalsize\beginnumbering \toendnotes[C]{\smallbreak\pagebreak[2]} \Standort{CUL, Schnitzler, B 8.}
\physDesc{Brief, 1 Blatt, 2 Seiten, 943 Zeichen
\newline{}Handschrift: schwarze Tinte, lateinische Kurrent
\newline{}Ordnung: mit Bleistift von unbekannter Hand nummeriert:
                                    »156« }\buchAbdrucke{\weitereDrucke{Arthur Schnitzler, Richard Beer-Hofmann: \emph{Briefwechsel 1891–1931}. Hg. Konstanze Fliedl. Wien, Zürich: \emph{Europaverlag} 1992, S. 149.} }\toendnotes[C]{\smallbreak}\pstart
           \raggedleft{}{\pb}Alt-Aussee\oindex{Altaussee@\textbf{Altaussee}|pw}{ }2/VIII 1900\pend
           \pstart
           Lieber Arthur! Bei Durchsicht des Reisebuches\pwindex{?? Werk@Nicht ermittelte Verfasserinnen und Verfasser!Deutsche Alpen1877@\emph{Deutsche Alpen} {[}1877{]}|pwv} stoße ich auf folgende Tour die ich Moserboden\oindex{Mooserboden@\textbf{Mooserboden}|pw}, Gerlos\oindex{Gerlos@\textbf{Gerlos}|pw} etc. vorziehen würde: Von Jenbach\oindex{Jenbach@\textbf{Jenbach}|pw} durchs Zillerthal\oindex{Zillertal@\textbf{Zillertal}|pw} nach \uline{Sterzing}\oindex{Sterzing@\textbf{Sterzing}|pw} (11 \introOben{}Bahn-\introOben{}Minuten von Gossensass\oindex{Gossensass@\textbf{Gossensass}|pw})\pend
           \settowidth{\longeste}{Folgende Eintheilung:}\settowidth{\longestz}{7.2Salzburg7.28 Frühmm}\settowidth{\longestd}{}\settowidth{\longestv}{}\settowidth{\longestf}{}\addtolength\longeste{1em}
        \addtolength\longestz{1em}
      \pstart\noindent\makebox[\the\longeste][l]{Folgende Eintheilung:}\makebox[\the\longestz][l]{\strikeout{7.2}{ }Salzburg\oindex{Salzburg@\textbf{Salzburg}|pw}{ }7.28 Früh}
                  \pend\pstart\noindent\makebox[\the\longeste][l]{}\makebox[\the\longestz][l]{Jenbach\oindex{Jenbach@\textbf{Jenbach}|pw}{ }12.28 Mittagessen}
                  \pend\pstart\noindent\makebox[\the\longeste][l]{}\makebox[\the\longestz][l]{Post von Jenbach\oindex{Jenbach@\textbf{Jenbach}|pw} ab um
                        2.30}
                  \pend\pstart\noindent\makebox[\the\longeste][l]{}\makebox[\the\longestz][l]{über Zell im Zillerth.\oindex{Zell am Ziller@\textbf{Zell am Ziller}|pw} nach Mayerhofen\oindex{Mayrhofen@\textbf{Mayrhofen}|pw}{ }7.50}
                  \pend\pstart\noindent\makebox[\the\longeste][l]{}\makebox[\the\longestz][l]{Mit Wagen also vielleicht 3 ½–4 Stunden}
                  \pend\pstart
           In Mayerhofen\oindex{Mayrhofen@\textbf{Mayrhofen}|pw} übernachten – dann über \introOben{}den\introOben{}{ }Ze{\geminationm}grund\oindex{Zemmgrund@\textbf{Zemmgrund}|pw}, Schwarzensteingrund\oindex{Schwarzensteingrund@\textbf{Schwarzensteingrund}|pw} und das Pfitscher Joch\oindex{Pfitscher Joch@\textbf{Pfitscher Joch}|pw} nach Sterzing\oindex{Sterzing@\textbf{Sterzing}|pw} 15–16 Stunden. = 2 Tage. Im Mayer\pwindex{?? Werk@Nicht ermittelte Verfasserinnen und Verfasser!Deutsche Alpen1877@\emph{Deutsche Alpen} {[}1877{]}|pw} II. Band pag. 229. Jedenfalls – nach dem Reisebuch – viel lohnender
               als Gerlos\oindex{Gerlos@\textbf{Gerlos}|pw} etc. Eventuell\pend
           \settowidth{\longeste}{}\settowidth{\longestz}{Salzburgabx3.12 Nachm.}\settowidth{\longestd}{}\settowidth{\longestv}{}\settowidth{\longestf}{}\addtolength\longeste{1em}
        \addtolength\longestz{1em}
      \pstart\noindent\makebox[\the\longeste][l]{}\makebox[\the\longestz][l]{Salzburg\oindex{Salzburg@\textbf{Salzburg}|pw} ab 3.12 Nachm.}
                  \pend\pstart\noindent\makebox[\the\longeste][l]{I Tag.}\makebox[\the\longestz][l]{Jenbach\oindex{Jenbach@\textbf{Jenbach}|pw}{ }8.45 Abends}
                  \pend\pstart\noindent\makebox[\the\longeste][l]{}\makebox[\the\longestz][l]{übernachten}
                  \pend\settowidth{\longeste}{}\settowidth{\longestz}{Jenbachab. Post.x8.00 Früh}\settowidth{\longestd}{}\settowidth{\longestv}{}\settowidth{\longestf}{}\addtolength\longeste{1em}
        \addtolength\longestz{1em}
      \pstart\noindent\makebox[\the\longeste][l]{}\makebox[\the\longestz][l]{Jenbach\oindex{Jenbach@\textbf{Jenbach}|pw} ab. Post. 8.00 Früh}
                  \pend\pstart\noindent\makebox[\the\longeste][l]{II Tag}\makebox[\the\longestz][l]{Mayerhofen\oindex{Mayrhofen@\textbf{Mayrhofen}|pw}. 1.30.
                     Mittagessen}
                  \pend\pstart\noindent\makebox[\the\longeste][l]{}\makebox[\the\longestz][l]{und bequemer Reitweg bis zum Breitlahner\oindex{Gasthaus Breitlahner@\textbf{Gasthaus Breitlahner}|pw} (zu Fuß 5 ½ Stunden)}
                  \pend\settowidth{\longeste}{III Tagx}\settowidth{\longestz}{(nachxSterzing10-11 Stunden)}\settowidth{\longestd}{}\settowidth{\longestv}{}\settowidth{\longestf}{}\addtolength\longeste{1em}
        \addtolength\longestz{1em}
      \pstart\noindent\makebox[\the\longeste][l]{III Tag }\makebox[\the\longestz][l]{(nach Sterzing\oindex{Sterzing@\textbf{Sterzing}|pw} 10-11 Stunden)}
                  \pend\pstart
           {\pb}Überlegen Sie, und schreiben Sie
               mir noch genau \uline{wann} sie in Salzburg\oindex{Salzburg@\textbf{Salzburg}|pw} sind.\pend
           \pstart
           Gerlospass\oindex{Gerlospass@\textbf{Gerlospass}|pw} mißbillige ich. – Moserboden\oindex{Mooserboden@\textbf{Mooserboden}|pw} nicht.\pend
           \pstart
           Herzlichst Ihr{\\[\baselineskip]}\spacefill\mbox{Richard}\pend
           \leftskip=0em{}
         
         \endnumbering\mylabel{h}\end{ledgroupsized}  \newcommand{\dateiname}{L01062}\newcommand{\titel}{Richard Beer-Hofmann an Arthur Schnitzler, 2. 8. 1900}\newcommand{\editorInnen}{Martin Anton Müller und Gerd-Hermann Susen}%% latex-leseansicht-abspann.tex
%% Abspann für die Leseansicht.
%% Der Schalter \ifkorrekturansicht ist bereits durch den Vorspann gesetzt.

%% latex-abspann.tex
%% Gemeinsamer Abspann für Korrekturansicht und Leseansicht.
%% Setzt den Schalter \ifkorrekturansicht voraus (gesetzt in den
%% einbindenden Dateien latex-korrekturansicht-abspann.tex bzw.
%% latex-leseansicht-abspann.tex).
%% ---------------------------------------------------------------

\normalsize

% Das esempio-Environment wird nur in der Leseansicht benötigt
\ifkorrekturansicht\else
\newenvironment{esempio}[3]%
{
    \vspace{1.5ex}
    \rlap{\underline{#1}}
    \par
    \setlength{\parindent}{0cm}
    \nopagebreak
    \leftskip=#2cm
    \rightskip=#3cm
}
{
    \par
}
\fi

\doendnotes{C}
\bigskip
\vfill

\clearpage

\footnotesize

\ifkorrekturansicht
  \lohead{\textsc{register}}
\fi

% theindex-Environment neu definieren ohne reledmac
\makeatletter
\renewenvironment{theindex}{%
  \ifkorrekturansicht
    \section*{\indexname}%
  \else
    \subsubsection*{Index der erwähnten Entitäten}%
  \fi
  \setlength{\parindent}{0pt}%
  \setlength{\parskip}{0pt plus 0.3pt}%
  \let\item\@idxitem
}{%
  \ifkorrekturansicht\clearpage\fi
}
\makeatother

\IfFileExists{\jobname-pw.ind}{\input{\jobname-pw.ind}}{}

% Quellenangabe nur in der Leseansicht
\ifkorrekturansicht\else
% Fallback-Definitionen, falls die .tex-Datei \titel etc. nicht gesetzt hat
\providecommand{\titel}{}
\providecommand{\editorInnen}{}
\providecommand{\dateiname}{\jobname}

\vspace{3cm}

\vfill

\footnotesize
\textsc{Quelle}: \titel. Herausgegeben von {\editorInnen}. In: \emph{Arthur Schnitzler: Briefwechsel mit Autorinnen und Autoren}.
 Digitale Edition, https://schnitzler-briefe.acdh.oeaw.ac.at/{\dateiname}.html (Stand \today)
\fi

\end{document}


      