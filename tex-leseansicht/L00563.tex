\input{../tex-inputs/latex-pdf-vorspann}
\begin{center}
            \textcolor{red}{ENTWURF. ENTZIFFERUNG NOCH NICHT KORREKTURGELESEN}
                      \end{center}
            
               \section[Arthur Schnitzler an Richard Beer-Hofmann, 15. 7. 1896]{ Arthur Schnitzler an Richard Beer-Hofmann, 15. 7. 1896}\nopagebreak\mylabel{v}\rehead{ }\begin{ledgroupsized}[t]{13cm}\normalsize\beginnumbering\briefempfaengerindex{Beer-Hofmann, Richard@\textsc{Beer-Hofmann, Richard}!zzzSchnitzler, Arthur@\emph{von Arthur Schnitzler}!1896-07-151@{15. 7. 1896}|(be} \toendnotes[C]{\smallbreak\pagebreak[2]} \Standort{YCGL, MSS 31.}
\physDesc{Postkarte
\newline{}Handschrift: Bleistift, deutsche Kurrent\newline{}Versand: 1) Stempel: »\nobreak{}\oindex{Trondheim@\textbf{Trondheim}|pwk}Trondhjem, 15. VII. 96\nobreak{}«.  2) Stempel: »\nobreak{}\oindex{Kopenhagen@\textbf{Kopenhagen}|pwk}Kjobenhavn, 17. 7. 96, 50M6\nobreak{}«. }\buchAbdrucke{\weitereDrucke{Arthur Schnitzler, Richard Beer-Hofmann: \emph{Briefwechsel 1891–1931}. Hg. Konstanze Fliedl. Wien, Zürich: \emph{Europaverlag} 1992, S. 93.} }\pstart{}{\pb}Herrn \textsc{Dr. Richard
                     Beer-Hofmann}\pend{}\pstart{}\textsc{Kopenhagen}\oindex{Kopenhagen@\textbf{Kopenhagen}|pw}\pend{}\pstart{}\textsc{post rest.}\pend{}{\bigskip}\pstart
           \raggedleft{}{\pb}Trondhjem\oindex{Trondheim@\textbf{Trondheim}|pw}{ }15/7 96\pend
           \pstart
           Mein lieber Richard, ich freue mich hier Ihren Brief gefunden zu
               haben – ich antwort Ihnen gleich dieſe 2 Zeilen, weil es heut Abend wieder weiter und
                  i{\geminationm}er weiter geht. Glauben Sie mir, Sie haben viel
               verſäumt – nun eben ko{\geminationm} ich aus dem Dom von Drontheim\oindex{Nidarosdom@\textbf{Nidarosdom}|pw} und hab Ihnen was gekauft.\pend
           \pstart
           Sie wiſſen ja, wo mich Ihren Briefe treffen. – Arbeiten werden Sie hoffentlich mehr
               als ich –?\pend
           \pstart
           Herzlich{\\[\baselineskip]}Ihr \spacefill\mbox{Arthur.}\pend
           \leftskip=0em{}\endnumbering\briefempfaengerindex{Beer-Hofmann, Richard@\textsc{Beer-Hofmann, Richard}!zzzSchnitzler, Arthur@\emph{von Arthur Schnitzler}!1896-07-151@{15. 7. 1896}|)be}\mylabel{h}\end{ledgroupsized}  \newcommand{\dateiname}{L00563}\newcommand{\titel}{Arthur Schnitzler an Richard Beer-Hofmann, 15. 7. 1896}\newcommand{\editorInnen}{Martin Anton Müller und Gerd-Hermann Susen}\input{../tex-inputs/latex-pdf-abspann}
      