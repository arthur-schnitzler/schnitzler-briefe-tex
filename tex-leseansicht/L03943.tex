%% latex-leseansicht-vorspann.tex
%% Vorspann für die Leseansicht.
%% Lädt die gemeinsame Datei latex-vorspann.tex mit nicht gesetztem Schalter.

\newif\ifkorrekturansicht
\korrekturansichtfalse

\input{../tex-inputs/latex-vorspann}


\section[Arthur Schnitzler an Theodor Herzl, 11. 5. 1893]{L03943 Arthur Schnitzler an Theodor Herzl, 11. 5. 1893}
\nopagebreak\mylabel{L03943v}
\rehead{ }\normalsize\beginnumbering\briefempfaengerindex{Herzl, Theodor@\textsc{Herzl, Theodor}!zzzSchnitzler, Arthur@\emph{von Arthur Schnitzler}!1893-05-111@{11. 5. 1893}|(be}
\toendnotes[C]{\smallbreak\pagebreak[2]}
\correspDesc{Versand  durch Arthur Schnitzler am 11. 5. 1893 in Wien
\newline{}Erhalt  durch Theodor Herzl im Zeitraum [12. 5. 1893 – 16. 5. 1893?] in Paris}\toendnotes[C]{\smallbreak}
\Standort{Jerusalem, Central Zionist Archives, H1:1924-5.}
\physDesc{Brief, 1 Blatt, 3 Seiten, 597 Zeichen
\newline{}Handschrift: schwarze Tinte, deutsche Kurrent
\newline{}Ordnung: mit Bleistift von unbekannter Hand innerhalb das Konvoluts paginiert:
                                 »19« }\toendnotes[C]{\smallbreak}
\pstart{}{\pb}Verehrteſter Freund,\pend\vspace{0.5em}
\pstart
           innigen Dank für Ihre freundliche Teilnahme\pwindex{Schnitzler, Johann 10.\,4.\,1835 Nagykanizsa – 2.\,5.\,1893 Wien@\textsc{Schnitzler, Johann} (10.\,4.\,1835 Nagykanizsa – 2.\,5.\,1893 Wien), \emph{Laryngologe}|pwv}; \substVorne{}\textsuperscript{d}\substDazwischen{}m\substHinten{}eine Angehörigen danken Ihnen gleichfalls aufs wärmſte! Man muſs
               ja auf alles mit den gleichen Worten danken, aber gerade{ }ſo wie bei den Ausdrücken Ihres
               Mitgefühls manche Saiten in mir {\pb}mitvibrirten, darf ich wohl auch hoffen, daſs Sie aus
               den Worten meines Dankes mehr herausleſen, als das kühle \label{K_L03943-1v}\edtext{\textsc{p. r.}}{\lemma{\textnormal{\emph{p. r.}}}\Cendnote{\textnormal{pro
                  recipiendo (lateinisch): für den Empfang}}}\label{K_L03943-1}, das für alle ist!\pend
           
\pstart
           Haben Sie Ihr
               Verſprechen{ }ſchon ganz vergeſſen? Ihr heutiges \label{K_L03943-2v}\edtext{\textsc{Feu{[}{]}lleton}\pwindex{Herzl, Theodor 2.\,5.\,1860 Budapest – 3.\,7.\,1904 Edlach@\textsc{Herzl, Theodor} (2.\,5.\,1860 Budapest – 3.\,7.\,1904 Edlach), \emph{Schriftsteller, Journalist}!Firniß. Bilder aus dem Pariser Kunstleben. Karl VII. in Chinon@\strich\emph{Firniß. Bilder aus dem Pariser Kunstleben. Karl VII. in Chinon}|pwv}}{\lemma{\textnormal{\emph{Feulleton}}}\Cendnote{\textnormal{Theodor Herzl\pwindex{Herzl, Theodor 2.\,5.\,1860 Budapest – 3.\,7.\,1904 Edlach@\textsc{Herzl, Theodor} (2.\,5.\,1860 Budapest – 3.\,7.\,1904 Edlach), \emph{Schriftsteller, Journalist}|pwk}: \emph{Firniß. Bilder aus dem Pariser Kunstleben. Karl VII. in
                        Chinon}\pwindex{Herzl, Theodor 2.\,5.\,1860 Budapest – 3.\,7.\,1904 Edlach@\textsc{Herzl, Theodor} (2.\,5.\,1860 Budapest – 3.\,7.\,1904 Edlach), \emph{Schriftsteller, Journalist}!Firniß. Bilder aus dem Pariser Kunstleben. Karl VII. in Chinon@\strich\emph{Firniß. Bilder aus dem Pariser Kunstleben. Karl VII. in Chinon}|pwk}. In: \emph{Neue Freie Presse}\pwindex{Neue Freie Presse@\emph{Neue Freie Presse}|pwk},
                     Nr. 10.314, 11. 5. 1893, Morgenblatt,
                     S. 1–4. }}}\label{K_L03943-2} hat mir wieder gezeigt, {\pb}einen wie{ }ſchönen Dienſt
               Sie mir eben jetzt mit den alten Stücken erwieſen!\pend
           
\pstart
           Herzlich der Ihre{\\[\baselineskip]}\spacefill\mbox{ArthSchnitzler}\pend
           \leftskip=0em{}
\pstart
           11. 5. 93{ }Wien\oindex{Wien@\textbf{Wien}, \emph{Verwaltungsgebiet}|pw}.\pend
           \selectlanguage{ngerman}\endnumbering\briefempfaengerindex{Herzl, Theodor@\textsc{Herzl, Theodor}!zzzSchnitzler, Arthur@\emph{von Arthur Schnitzler}!1893-05-111@{11. 5. 1893}|)be}\mylabel{L03943h}
\begin{anhang}
\end{anhang}\newcommand{\dateiname}{L03943}\newcommand{\titel}{Arthur Schnitzler an Theodor Herzl, 11. 5. 1893}\newcommand{\editorInnen}{Herausgegeben von Jahnke, SelmaMüller, Martin Anton}%% latex-leseansicht-abspann.tex
%% Abspann für die Leseansicht.
%% Der Schalter \ifkorrekturansicht ist bereits durch den Vorspann gesetzt.

%% latex-abspann.tex
%% Gemeinsamer Abspann für Korrekturansicht und Leseansicht.
%% Setzt den Schalter \ifkorrekturansicht voraus (gesetzt in den
%% einbindenden Dateien latex-korrekturansicht-abspann.tex bzw.
%% latex-leseansicht-abspann.tex).
%% ---------------------------------------------------------------

\normalsize

% Das esempio-Environment wird nur in der Leseansicht benötigt
\ifkorrekturansicht\else
\newenvironment{esempio}[3]%
{
    \vspace{1.5ex}
    \rlap{\underline{#1}}
    \par
    \setlength{\parindent}{0cm}
    \nopagebreak
    \leftskip=#2cm
    \rightskip=#3cm
}
{
    \par
}
\fi

\doendnotes{C}
\bigskip
\vfill

\clearpage

\footnotesize

\ifkorrekturansicht
  \lohead{\textsc{register}}
\fi

% theindex-Environment neu definieren ohne reledmac
\makeatletter
\renewenvironment{theindex}{%
  \ifkorrekturansicht
    \section*{\indexname}%
  \else
    \subsubsection*{Index der erwähnten Entitäten}%
  \fi
  \setlength{\parindent}{0pt}%
  \setlength{\parskip}{0pt plus 0.3pt}%
  \let\item\@idxitem
}{%
  \ifkorrekturansicht\clearpage\fi
}
\makeatother

\IfFileExists{\jobname-pw.ind}{\input{\jobname-pw.ind}}{}

% Quellenangabe nur in der Leseansicht
\ifkorrekturansicht\else
% Fallback-Definitionen, falls die .tex-Datei \titel etc. nicht gesetzt hat
\providecommand{\titel}{}
\providecommand{\editorInnen}{}
\providecommand{\dateiname}{\jobname}

\vspace{3cm}

\vfill

\footnotesize
\textsc{Quelle}: \titel. Herausgegeben von {\editorInnen}. In: \emph{Arthur Schnitzler: Briefwechsel mit Autorinnen und Autoren}.
 Digitale Edition, https://schnitzler-briefe.acdh.oeaw.ac.at/{\dateiname}.html (Stand \today)
\fi

\end{document}


