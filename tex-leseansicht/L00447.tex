%% latex-leseansicht-vorspann.tex
%% Vorspann für die Leseansicht.
%% Lädt die gemeinsame Datei latex-vorspann.tex mit nicht gesetztem Schalter.

\newif\ifkorrekturansicht
\korrekturansichtfalse

\input{../tex-inputs/latex-vorspann}

\begin{center}
            \textcolor{red}{ENTWURF. ENTZIFFERUNG NOCH NICHT KORREKTURGELESEN}
                      \end{center}
            
               \section[Friedrich M. Fels an Arthur Schnitzler, {[}1. Hälfte Juni 1895{]}]{ Friedrich M. Fels an Arthur Schnitzler, {[}1. Hälfte Juni
                    1895{]}}\nopagebreak\mylabel{v}\rehead{ }\begin{ledgroupsized}[t]{13cm}\normalsize\beginnumbering\briefempfaengerindex{Schnitzler, Arthur@\textsc{Schnitzler, Arthur}!zzzFels, Friedrich Michael@\emph{von Friedrich Michael Fels}!1895-06-011@{{[}1. Hälfte Juni
                        1895{]}}|(be} \toendnotes[C]{\smallbreak\pagebreak[2]} \Standort{DLA, A:Schnitzler, HS.NZ85.1.2956.}
\physDesc{Brief, 1 Blatt, 3 Seiten
\newline{}Handschrift: schwarze Tinte, lateinische Kurrent
\newline{}Schnitzler: mit Bleistift nummeriert: »23« und datiert: »\introOben{}\textsc{Anfg}{ }\textsc{ca Mitte Juni 95}\introOben{}« }\toendnotes[C]{\smallbreak}\pstart
           \raggedleft{}{\pb}Zürich I, Waldma{\geminationn}straſse 10, III. St.\oindex{Waldmannstrasse@\textbf{Waldmannstraße}|pw}\pend
           \pstart{}Lieber Dr. Schnitzler!\pend\pstart
           Verzeihen Sie, daſs ich Sie bis jetzt ohne Nachricht lieſs; aber einmal schrieb
                    mir Magaziner\pwindex{Magaziner, Viktor Adalbert 15.10.1867 – 10.1921@\textsc{Magaziner, Viktor Adalbert} (15.10.1867 – 10.1921), \emph{Buchhalter}|pw}, er habe Sie gesprochen und
                    Ihnen von mir erzählt, und da{\geminationn} wünschten Sie Briefe
                    und \introOben{}ich\introOben{} brachte es bisher nur zu Karten. Endlich aber –
                    das kö{\geminationn}en Sie sich denken – war ich in der ersten
                    Zeit in trostloser Sti{\geminationm}ung, und aus der heraus
                    mochte ich Ihnen nicht schreiben, ich wollte wenigstens vorher erfahren, ob ich
                    überhaupt noch werde leben kö{\geminationn}en; we{\geminationn} auch noch nicht, wie ich werde leben kö{\geminationn}en. Der erste Tag hier brachte mir gleich
                    Enttäuschungen: Spitteler\pwindex{Spitteler, Carl 24.04.1845 – 29.12.1924@\textsc{Spitteler, Carl} (24.04.1845 – 29.12.1924), \emph{Schriftsteller}|pw} ist nicht \substVorne{}\textsuperscript{\textcolor{gray}{der}}\substDazwischen{}mehr\substHinten{} Feuilletonredakteur der Neuen Zürcher
                        Zeitung\orgindex{Neue Zuercher Zeitung@Neue Zürcher Zeitung|pw}, Widman\pwindex{Widmann, Joseph Victor 20.02.1842 – 06.11.1911@\textsc{Widmann, Joseph Victor} (20.02.1842 – 06.11.1911), \emph{Schriftsteller, Journalist}|pw} wohnt z. Z. in
                    Italien, der Beka{\geminationn}te\pwindex{?? [Bekannter von Magaziner in Zuerich] 1.6.1895 – 1.6.1895@\textsc{?? [Bekannter von Magaziner in Zürich]} (1.6.1895 – 1.6.1895)|pwv}, an den mich Magaziner\pwindex{Magaziner, Viktor Adalbert 15.10.1867 – 10.1921@\textsc{Magaziner, Viktor Adalbert} (15.10.1867 – 10.1921), \emph{Buchhalter}|pw} empfahl, ist ein eckelhafter Lump, ein
                    Reporterjüngling miserabelster Sorte. Dazu die Nachricht, daſs ich auch hier
                    wahrscheinlich werde ausgewiesen werden. Nun zeigte es sich auch diesmal, daſs
                    nichts so heiſs gegeſsen, wie gekocht wird. Die N. Z. Z.\pwindex{Neue Zuercher Zeitung12. 1. 1780@\emph{Neue Zürcher Zeitung}|pw} hat bereits ein Feuilleton von mir acceptiert und wird
                    weitere acceptieren, mit Widman\pwindex{Widmann, Joseph Victor 20.02.1842 – 06.11.1911@\textsc{Widmann, Joseph Victor} (20.02.1842 – 06.11.1911), \emph{Schriftsteller, Journalist}|pw} wird bei
                    seiner Rückkehr auch etwas zu machen sein, und was die Hauptsache anlangt, so
                    werde ich wahrscheinlich gegen Erlag einer Kaution von 1,500 frcs in monatlichen
                    Raten à 20 frcs hier bleiben kö{\geminationn}en. Freilich
                    wird{[}s{]} mir in {\pb}der ersten Zeit miserabel gehen; de{\geminationn} das Leben
                    hier ist furchtbar teuer, oder beſser gesagt das Existenzminimum liegt viel
                    höher als in Wien\oindex{Wien@\textbf{Wien}|pw}. Mit 50 fl monatlich ka{\geminationn} man einfach nicht leben. Ich muſs auf alle Weise
                    zu verdienen suchen. Die Preſse\orgindex{Presse@Die Presse|pw} hat seit
                    1 Monat ein Feuilleton\pwindex{Fels, Friedrich Michael *~1864@\textsc{Fels, Friedrich Michael} (*~1864), \emph{Journalist}!Zeichen der Zeit10.7.1895 – 11.7.1895@\strich\emph{Zeichen der Zeit} {[}10.7.1895 – 11.7.1895{]}|pwv} von
                    mir und druckt es nicht; obgleich es angeno{\geminationm}en ist.
                    Sie würden mich sehr verpflichten, we{\geminationn} Sie deshalb
                    mit Hirschfeld\pwindex{Hirschfeld, Robert 17.09.1857 – 02.04.1914@\textsc{Hirschfeld, Robert} (17.09.1857 – 02.04.1914), \emph{Journalist, Musikkritiker}|pw} redeten oder, falls er schon
                    abgereist ist, ihm wenigstens schrieben. Soll ich ihm auch schreiben? und wohin?
                    und was? Auch Wengraf\pwindex{Wengraf, Edmund 09.01.1860 – 08.12.1933@\textsc{Wengraf, Edmund} (09.01.1860 – 08.12.1933), \emph{Journalist}|pw}–Osten\pwindex{Osten, Heinrich 16.08.1855 – 01.08.1931@\textsc{Osten, Heinrich} (16.08.1855 – 01.08.1931), \emph{Schriftsteller, Journalist}|pw} rühren sich nicht; ich habe, seit ich hier bin,
                    kein Belegexemplar\pwindex{Neue Revue. Wiener Literatur-Zeitung1894 – 1898@\emph{Neue Revue. Wiener Literatur-Zeitung}|pwv}\pwindex{Fels, Friedrich Michael *~1864@\textsc{Fels, Friedrich Michael} (*~1864), \emph{Journalist}!Aesthetische Zeitfragen22.5.1895 – 22.5.1895@\strich\emph{Ästhetische Zeitfragen} {[}22.5.1895 – 22.5.1895{]}|pwv} erhalten, obgleich sie meine Adreſse doch wiſsen.\pend
           \pstart
           Vom Zürcher\oindex{Zuerich@\textbf{Zürich}|pw} literarischen Leben ka{\geminationn} ich Ihnen noch nichts sagen; ich ke{\geminationn}e noch niemanden. Henckell\pwindex{Henckell, Karl Friedrich 17.04.1864 – 30.07.1929@\textsc{Henckell, Karl Friedrich} (17.04.1864 – 30.07.1929), \emph{Schriftsteller}|pw} ist verreist, mit M. R. v.
                        Stern\pwindex{Stern, Maurice Reinhold 1860-04-03 – 1938-10-28@\textsc{Stern, Maurice Reinhold} (1860-04-03 – 1938-10-28), \emph{Schriftsteller, Journalist}|pw} verkehrt niemand; wird mir nichts übrig bleiben, als Ilse Frapan\pwindex{Frapan, Ilse 1849-02-03 – 1908-12-02@\textsc{Frapan, Ilse} (1849-02-03 – 1908-12-02), \emph{Schriftstellerin}|pw} aufzusuchen und mir ihre
                    Novellen vorlesen zu laſsen. Bölsche\pwindex{Boelsche, Wilhelm 02.01.1861 – 31.08.1939@\textsc{Bölsche, Wilhelm} (02.01.1861 – 31.08.1939), \emph{Schriftsteller, Publizist}|pw} lebt
                    wieder in Berlin\oindex{Berlin@\textbf{Berlin}|pw}, Halbe\pwindex{Halbe, Max 04.10.1865 – 30.11.1944@\textsc{Halbe, Max} (04.10.1865 – 30.11.1944), \emph{Schriftsteller}|pw} in München\oindex{Muenchen@\textbf{München}|pw}. Windberg\pwindex{Windberg 1895 – 1895@\textsc{Windberg} (1895 – 1895)|pw} hab ich getroffen und treff ich
                    oft; er ist noch mein Trost. Auſserdem ka{\geminationn} ich von
                    anständigen Menschen hier den Schauspieler Néher\pwindex{Neher, Louis 1.6.1895 – 1.6.1895@\textsc{Neher, Louis} (1.6.1895 – 1.6.1895), \emph{Schauspieler}|pw}, früher bei den Meiningern\orgindex{Meininger@Meininger|pw}, und
                    einen ungar\oindex{Ungarn@\textbf{Ungarn}|pw}ischen Studenten\pwindex{?? [Ungarischer Student in Zuerich] 1895 – 1895@\textsc{?? [Ungarischer Student in Zürich]} (1895 – 1895)|pwv}; sonst besteht die
                    Fremdenkolonie gröſstenteils aus Lumpenpack. Übrigens ist die Erfahrung zu
                    machen, daſs die deutschen\oindex{Deutschland@\textbf{Deutschland}|pw} und österreichischen\oindex{Oesterreich@\textbf{Österreich}|pw}{ }{\pb}Deserteure; deren hier eine Unmaſse
                    lebt, viel anständiger sind als die in der Heimat nicht beanständigten, mit den
                        \textcolor{gray}{wun}dervollsten Taſsen versehenen Fremden – wobei ich
                    nicht pro domo rede. Mit den Zürchern\oindex{Zuerich@\textbf{Zürich}|pw} läſst
                    sich schwer was anfangen; man muſs viel überwinden. Übrigens muſs, will und
                    werde ich mich angewöhnen und selbst ein ganzer Zürcher\oindex{Zuerich@\textbf{Zürich}|pw} werden, Familie gründen etc, was dazu gehört. Halten Sie mir
                    den Daumen, daſs mich das Mädel mag. Da{\geminationn} werd ich
                    in zwei Jahren Bürger \introOben{}sein\introOben{} und heiraten.\pend
           \pstart
           Schreiben Sie mir einmal; auſser von Magaziner\pwindex{Magaziner, Viktor Adalbert 15.10.1867 – 10.1921@\textsc{Magaziner, Viktor Adalbert} (15.10.1867 – 10.1921), \emph{Buchhalter}|pw} hab ich von niemandem Nachricht, und Sie wiſsen nicht, wie
                    ich danach lechze.\pend
           \pstart
           Herzlichst{\\[\baselineskip]}Ihr{\\[\baselineskip]}dankbar ergebener{\\[\baselineskip]}\spacefill\mbox{Fels}\pend
           \leftskip=0em{}\pstart
           \noindent{}Bitte, grüſsen Sie Beer-Hofma{\geminationn}\pwindex{Beer-Hofmann, Richard 11.07.1866 – 26.09.1945@\textsc{Beer-Hofmann, Richard} (11.07.1866 – 26.09.1945), \emph{Schriftsteller}|pw}, Hofma{\geminationn}sthal\pwindex{Hofmannsthal, Hugo von 01.02.1874 – 15.07.1929@\textsc{Hofmannsthal, Hugo von} (01.02.1874 – 15.07.1929), \emph{Schriftsteller}|pw}, Salten\pwindex{Salten, Felix 06.09.1869 – 08.10.1945@\textsc{Salten, Felix} (06.09.1869 – 08.10.1945), \emph{Schriftsteller, Journalist}|pw}.\pend
           \endnumbering\briefempfaengerindex{Schnitzler, Arthur@\textsc{Schnitzler, Arthur}!zzzFels, Friedrich Michael@\emph{von Friedrich Michael Fels}!1895-06-011@{{[}1. Hälfte Juni
                        1895{]}}|)be}\mylabel{h}\end{ledgroupsized}  \newcommand{\dateiname}{L00447}\newcommand{\titel}{Friedrich M. Fels an Arthur Schnitzler, [1. Hälfte Juni 1895]}\newcommand{\editorInnen}{Martin Anton Müller und Gerd-Hermann Susen}%% latex-leseansicht-abspann.tex
%% Abspann für die Leseansicht.
%% Der Schalter \ifkorrekturansicht ist bereits durch den Vorspann gesetzt.

%% latex-abspann.tex
%% Gemeinsamer Abspann für Korrekturansicht und Leseansicht.
%% Setzt den Schalter \ifkorrekturansicht voraus (gesetzt in den
%% einbindenden Dateien latex-korrekturansicht-abspann.tex bzw.
%% latex-leseansicht-abspann.tex).
%% ---------------------------------------------------------------

\normalsize

% Das esempio-Environment wird nur in der Leseansicht benötigt
\ifkorrekturansicht\else
\newenvironment{esempio}[3]%
{
    \vspace{1.5ex}
    \rlap{\underline{#1}}
    \par
    \setlength{\parindent}{0cm}
    \nopagebreak
    \leftskip=#2cm
    \rightskip=#3cm
}
{
    \par
}
\fi

\doendnotes{C}
\bigskip
\vfill

\clearpage

\footnotesize

\ifkorrekturansicht
  \lohead{\textsc{register}}
\fi

% theindex-Environment neu definieren ohne reledmac
\makeatletter
\renewenvironment{theindex}{%
  \ifkorrekturansicht
    \section*{\indexname}%
  \else
    \subsubsection*{Index der erwähnten Entitäten}%
  \fi
  \setlength{\parindent}{0pt}%
  \setlength{\parskip}{0pt plus 0.3pt}%
  \let\item\@idxitem
}{%
  \ifkorrekturansicht\clearpage\fi
}
\makeatother

\IfFileExists{\jobname-pw.ind}{\input{\jobname-pw.ind}}{}

% Quellenangabe nur in der Leseansicht
\ifkorrekturansicht\else
% Fallback-Definitionen, falls die .tex-Datei \titel etc. nicht gesetzt hat
\providecommand{\titel}{}
\providecommand{\editorInnen}{}
\providecommand{\dateiname}{\jobname}

\vspace{3cm}

\vfill

\footnotesize
\textsc{Quelle}: \titel. Herausgegeben von {\editorInnen}. In: \emph{Arthur Schnitzler: Briefwechsel mit Autorinnen und Autoren}.
 Digitale Edition, https://schnitzler-briefe.acdh.oeaw.ac.at/{\dateiname}.html (Stand \today)
\fi

\end{document}


      