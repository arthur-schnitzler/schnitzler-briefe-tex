%% latex-leseansicht-vorspann.tex
%% Vorspann für die Leseansicht.
%% Lädt die gemeinsame Datei latex-vorspann.tex mit nicht gesetztem Schalter.

\newif\ifkorrekturansicht
\korrekturansichtfalse

\input{../tex-inputs/latex-vorspann}


\section[ Arthur Schnitzler und Hugo von Hofmannsthal an Felix Salten, [1.] 7. 1902]{L02977 Arthur Schnitzler und Hugo von Hofmannsthal an Felix
               Salten,  [1.] 7. 1902}
\nopagebreak\mylabel{L02977v}
\rehead{ }\normalsize\beginnumbering\briefempfaengerindex{Salten, Felix@\textsc{Salten, Felix}!zzzHofmannsthal, Hugo von@\emph{von Hugo von Hofmannsthal}!1902-07-011@{[1.] 7. 1902}|(be}\briefempfaengerindex{Salten, Felix@\textsc{Salten, Felix}!zzzSchnitzler, Arthur@\emph{von Arthur Schnitzler}!1902-07-011@{[1.] 7. 1902}|(be}
\toendnotes[C]{\smallbreak\pagebreak[2]}
\correspDesc{Versand  durch Arthur Schnitzler, Hugo von Hofmannsthal am [1.] 7. 1902 in Lofer
\newline{}Erhalt  durch Felix Salten am 2. 7. 1902 in Kaltenleutgeben}\toendnotes[C]{\smallbreak}
\Standort{Wienbibliothek im Rathaus, ZPH 1681, 2.1.516.}
\physDesc{Bildpostkarte, 136 Zeichen
\newline{}Handschrift Arthur Schnitzler: Bleistift, deutsche Kurrent
\newline{}Handschrift Hugo von Hofmannsthal: Bleistift, lateinische Kurrent
\newline{}Versand: 1) Stempel: »\nobreak{}\oindex{Lofer@\textbf{Lofer}, \emph{Hauptstadt}|pwk}Lofer, \textcolor{gray}{1.} 7. {[}1902{]}\nobreak{}«.   2) Stempel: »\nobreak{}\oindex{Kaltenleutgeben@\textbf{Kaltenleutgeben}, \emph{Hauptstadt}|pwk}Kaltenleutgeben, 2. 7. 02, 12–1N, Bestellt\nobreak{}«. 
\newline{}Ordnung: mit Bleistift von unbekannter Hand nummeriert: »4« }\toendnotes[C]{\smallbreak}\pstart{}{\pb}\textsc{N.Oe.\oindex{Niederösterreich@\textbf{Niederösterreich}, \emph{Land}|pw}}\pend{}\pstart{}Herrn Felix Salten\pend{}\pstart{}\textsc{Kaltenleutgeben\oindex{Kaltenleutgeben@\textbf{Kaltenleutgeben}, \emph{Hauptstadt}|pw}}\pend{}\pstart{}\textsc{bei}{ }Wien\oindex{Wien@\textbf{Wien}, \emph{Verwaltungsgebiet}|pw}\pend{}\pstart{}\textsc{Anstalt Winternitz\oindex{Kaltwasserheilanstalt Winternitz@\textbf{Kaltwasserheilanstalt Winternitz}, \emph{Sanatorium}|pw}}\pend{}{\bigskip}
\pstart
           \noindent{}\centering{}{\pb}\textcolor{gray}{\textbf{Gasthof zur Post\oindex{Gasthof zur Post@\textbf{Gasthof zur Post}, \emph{Gastgewerbegebäude}|pw}}}\pend
           
\pstart
           \centering{}\textcolor{gray}{\textbf{Lofer\oindex{Lofer@\textbf{Lofer}, \emph{Hauptstadt}|pw}}}\pend
           \vspace{1em}
\pstart
           \noindent{}{\pb}Herzl\textcolor{gray}{.} Grüße von
               demſelben Ort, wo wir vor \label{K_L02977-1v}\edtext{7 Jahren}{\lemma{\textnormal{\emph{7 Jahren}}}\Cendnote{\textnormal{Siehe A. S.: \emph{Tagebuch}, 24. 8. 1895. }}}\label{K_L02977-1}{ }\textsc{etc\textcolor{gray}{.}}\pend
           \pstart Ihr \spacefill\mbox{Arthur}\pend{}\selectlanguage{ngerman}\vspace{1em}
\pstart
           \noindent{}{[}hs. Hofmannsthal:{]} Gruss \spacefill\mbox{Hugo}\pend
           \selectlanguage{ngerman}\endnumbering\briefempfaengerindex{Salten, Felix@\textsc{Salten, Felix}!zzzHofmannsthal, Hugo von@\emph{von Hugo von Hofmannsthal}!1902-07-011@{[1.] 7. 1902}|)be}\briefempfaengerindex{Salten, Felix@\textsc{Salten, Felix}!zzzSchnitzler, Arthur@\emph{von Arthur Schnitzler}!1902-07-011@{[1.] 7. 1902}|)be}\mylabel{L02977h}  \newcommand{\dateiname}{L02977}\newcommand{\titel}{Arthur Schnitzler und Hugo von Hofmannsthal an Felix Salten, [1.] 7. 1902}\newcommand{\editorInnen}{Martin Anton Müller und Laura Untner}%% latex-leseansicht-abspann.tex
%% Abspann für die Leseansicht.
%% Der Schalter \ifkorrekturansicht ist bereits durch den Vorspann gesetzt.

%% latex-abspann.tex
%% Gemeinsamer Abspann für Korrekturansicht und Leseansicht.
%% Setzt den Schalter \ifkorrekturansicht voraus (gesetzt in den
%% einbindenden Dateien latex-korrekturansicht-abspann.tex bzw.
%% latex-leseansicht-abspann.tex).
%% ---------------------------------------------------------------

\normalsize

% Das esempio-Environment wird nur in der Leseansicht benötigt
\ifkorrekturansicht\else
\newenvironment{esempio}[3]%
{
    \vspace{1.5ex}
    \rlap{\underline{#1}}
    \par
    \setlength{\parindent}{0cm}
    \nopagebreak
    \leftskip=#2cm
    \rightskip=#3cm
}
{
    \par
}
\fi

\doendnotes{C}
\bigskip
\vfill

\clearpage

\footnotesize

\ifkorrekturansicht
  \lohead{\textsc{register}}
\fi

% theindex-Environment neu definieren ohne reledmac
\makeatletter
\renewenvironment{theindex}{%
  \ifkorrekturansicht
    \section*{\indexname}%
  \else
    \subsubsection*{Index der erwähnten Entitäten}%
  \fi
  \setlength{\parindent}{0pt}%
  \setlength{\parskip}{0pt plus 0.3pt}%
  \let\item\@idxitem
}{%
  \ifkorrekturansicht\clearpage\fi
}
\makeatother

\IfFileExists{\jobname-pw.ind}{\input{\jobname-pw.ind}}{}

% Quellenangabe nur in der Leseansicht
\ifkorrekturansicht\else
% Fallback-Definitionen, falls die .tex-Datei \titel etc. nicht gesetzt hat
\providecommand{\titel}{}
\providecommand{\editorInnen}{}
\providecommand{\dateiname}{\jobname}

\vspace{3cm}

\vfill

\footnotesize
\textsc{Quelle}: \titel. Herausgegeben von {\editorInnen}. In: \emph{Arthur Schnitzler: Briefwechsel mit Autorinnen und Autoren}.
 Digitale Edition, https://schnitzler-briefe.acdh.oeaw.ac.at/{\dateiname}.html (Stand \today)
\fi

\end{document}


