%% latex-leseansicht-vorspann.tex
%% Vorspann für die Leseansicht.
%% Lädt die gemeinsame Datei latex-vorspann.tex mit nicht gesetztem Schalter.

\newif\ifkorrekturansicht
\korrekturansichtfalse

\input{../tex-inputs/latex-vorspann}

\begin{center}
            \textcolor{red}{ENTWURF. ENTZIFFERUNG NOCH NICHT KORREKTURGELESEN}
                      \end{center}
            
               \section[Arthur Schnitzler an Hugo von Hofmannsthal, {[}15. 1. 1894{]}]{ Arthur Schnitzler an Hugo von Hofmannsthal, {[}15. 1. 1894{]}}\nopagebreak\mylabel{v}\rehead{ }\begin{ledgroupsized}[t]{13cm}\normalsize\beginnumbering\briefempfaengerindex{Hofmannsthal, Hugo von@\textsc{Hofmannsthal, Hugo von}!zzzSchnitzler, Arthur@\emph{von Arthur Schnitzler}!1894-01-151@{{[}15. 1. 1894{]}}|(be} \toendnotes[C]{\smallbreak\pagebreak[2]} \Standort{FDH, Hs-30885,40.}
\physDesc{Brief, 1 Blatt (Briefpapier mit Trauerrand), 3 Seiten
\newline{}Handschrift: schwarze Tinte, deutsche Kurrent\newline{}Ordnung: von unbekannter Hand datiert: »93« }\buchAbdrucke{\weitereDrucke{Hugo von Hofmannsthal, Arthur Schnitzler: \emph{Briefwechsel}. Hg. Therese Nickl und Heinrich Schnitzler. Frankfurt am Main: \emph{S. Fischer} 1964, S. 48–49.} }\toendnotes[C]{\smallbreak}\pstart{}{\pb}Lieber Hugo,\pend\pstart
           \label{K_L00291_1v}\edtext{Sonntag}{\lemma{\textnormal{\emph{Sonntag}}}\Cendnote{\textnormal{Schnitzler\pwindex{Schnitzler, Arthur 15.05.1862 – 21.10.1931@\textsc{Schnitzler, Arthur} (15.05.1862 – 21.10.1931), \emph{Schriftsteller, Mediziner}|pwk} und Hofmannsthal\pwindex{Hofmannsthal, Hugo von 01.02.1874 – 15.07.1929@\textsc{Hofmannsthal, Hugo von} (01.02.1874 – 15.07.1929), \emph{Schriftsteller}|pwk} besuchten die angesprochene Aufführung am
                     21. 1. 1894, die im Zuge eines Gastspiels am Carltheater\oindex{Carl-Theater@\textbf{Carl-Theater}|pwk} stattfand (A. S.: \emph{Tagebuch}, 21. 1. 1893, Hugo von Hofmannsthal\pwindex{Hofmannsthal, Hugo von 01.02.1874 – 15.07.1929@\textsc{Hofmannsthal, Hugo von} (01.02.1874 – 15.07.1929), \emph{Schriftsteller}|pwk}: \emph{Aufzeichnungen}. Hg. Rudolf Hirsch † und Ellen Ritter † in
                     Zusammenarbeit mit Konrad Heumann und Peter Michael Braunwarth. Frankfurt am
                     Main: \emph{S. Fischer}\orgindex{S. Fischer Verlag@S. Fischer Verlag|pwk}{ }2013, S. 265 (\emph{Sämtliche Werke},
                     XXXIX)).}}}\label{K_L00291_1h} gibt \textsc{Mounet-Sully}\pwindex{Mounet-Sully, Jean 27.02.1841 – 01.03.1916@\textsc{Mounet-Sully, Jean} (27.02.1841 – 01.03.1916), \emph{Schauspieler, Rechtsanwalt}|pw} den \textsc{Hamlet}\pwindex{\textcolor{red}{\textsuperscript{XXXX1 indx}}!Hamlet1600@\strich\emph{Hamlet} {[}1600{]}|pw}; da möcht ich gern hineingehn. Sie auch? Soll ich für uns beide Sitze nehmen?
               Was für eine Su{\geminationm}e {\pb}wollen Sie
               eventuell dieſem Zwecke widmen?\pend
           \pstart
           – \label{K_L00291_2v}\edtext{Heut}{\lemma{\textnormal{\emph{Heut}}}\Cendnote{\textnormal{Am 15. 1. 1894 war Schnitzler\pwindex{Schnitzler, Arthur 15.05.1862 – 21.10.1931@\textsc{Schnitzler, Arthur} (15.05.1862 – 21.10.1931), \emph{Schriftsteller, Mediziner}|pwk} in
                  der Premiere von \emph{Der ungläubige Thomas}\pwindex{Jacoby, Wilhelm 08.03.1855 – 20.02.1925@\textsc{Jacoby, Wilhelm} (08.03.1855 – 20.02.1925), \emph{Schriftsteller}!unglaeubige Thomas1893@\strich\emph{Der ungläubige Thomas} {[}1893{]}|pwk}\pwindex{Laufs, Carl 20.12.1858 – 13.08.1900@\textsc{Laufs, Carl} (20.12.1858 – 13.08.1900), \emph{Schriftsteller}!unglaeubige Thomas1893@\strich\emph{Der ungläubige Thomas} {[}1893{]}|pwk} von Karl Laufs\pwindex{Laufs, Carl 20.12.1858 – 13.08.1900@\textsc{Laufs, Carl} (20.12.1858 – 13.08.1900), \emph{Schriftsteller}|pwk} und Wilhelm Jacoby\pwindex{Jacoby, Wilhelm 08.03.1855 – 20.02.1925@\textsc{Jacoby, Wilhelm} (08.03.1855 – 20.02.1925), \emph{Schriftsteller}|pwk} am Raimundtheater\oindex{Raimund-Theater@\textbf{Raimund-Theater}|pwk}. (\emph{Cambridge University Library}, A 179)}}}\label{K_L00291_2h} geh
               ich zum ungläubigen \textsc{Thomas}\pwindex{Jacoby, Wilhelm 08.03.1855 – 20.02.1925@\textsc{Jacoby, Wilhelm} (08.03.1855 – 20.02.1925), \emph{Schriftsteller}!unglaeubige Thomas1893@\strich\emph{Der ungläubige Thomas} {[}1893{]}|pw}\pwindex{Laufs, Carl 20.12.1858 – 13.08.1900@\textsc{Laufs, Carl} (20.12.1858 – 13.08.1900), \emph{Schriftsteller}!unglaeubige Thomas1893@\strich\emph{Der ungläubige Thomas} {[}1893{]}|pw}, \label{K_L00291_3v}\edtext{morgen}{\lemma{\textnormal{\emph{morgen}}}\Cendnote{\textnormal{Victorien Sardou\pwindex{Sardou, Victorien 07.09.1831 – 08.11.1908@\textsc{Sardou, Victorien} (07.09.1831 – 08.11.1908), \emph{Schriftsteller}|pwk}s \emph{Madame Sans-Gêne}\pwindex{\textcolor{red}{\textsuperscript{XXXX1 indx}}!Madame Sans-Gêne1893@\strich\emph{Madame Sans-Gêne} {[}1893{]}|pwk}\pwindex{Sardou, Victorien 07.09.1831 – 08.11.1908@\textsc{Sardou, Victorien} (07.09.1831 – 08.11.1908), \emph{Schriftsteller}!Madame Sans-Gêne1893@\strich\emph{Madame Sans-Gêne} {[}1893{]}|pwk} wurde am 16. 1. 1894 im Deutschen
                     Volkstheater\oindex{Volkstheater@\textbf{Volkstheater}|pwk} gegeben, Schnitzler\pwindex{Schnitzler, Arthur 15.05.1862 – 21.10.1931@\textsc{Schnitzler, Arthur} (15.05.1862 – 21.10.1931), \emph{Schriftsteller, Mediziner}|pwk} war
                  anwesend. (\emph{Cambridge University Library}, A 179)}}}\label{K_L00291_3h} zu
                  \textsc{Madame Sans-gêne}\pwindex{\textcolor{red}{\textsuperscript{XXXX1 indx}}!Madame Sans-Gêne1893@\strich\emph{Madame Sans-Gêne} {[}1893{]}|pw}\pwindex{Sardou, Victorien 07.09.1831 – 08.11.1908@\textsc{Sardou, Victorien} (07.09.1831 – 08.11.1908), \emph{Schriftsteller}!Madame Sans-Gêne1893@\strich\emph{Madame Sans-Gêne} {[}1893{]}|pw}. Bin äußerſt kunſtſinnig. –\pend
           \pstart
           – Beifolgende ergreifende \label{K_L00291_4v}\edtext{Erzählung}{\lemma{\textnormal{\emph{Erzählung}}}\Cendnote{\textnormal{Nicht
                  identifiziert.}}}\label{K_L00291_4h} iſt mit Andacht zu leſen.\pend
           \pstart {\pb}Herzlich Ihr Arthur, der eine baldige Antwort erwartet. – \pend{}\pstart
           \noindent{}\uline{Montag.}\pend
           \endnumbering\briefempfaengerindex{Hofmannsthal, Hugo von@\textsc{Hofmannsthal, Hugo von}!zzzSchnitzler, Arthur@\emph{von Arthur Schnitzler}!1894-01-151@{{[}15. 1. 1894{]}}|)be}\mylabel{h}\end{ledgroupsized}  \newcommand{\dateiname}{L00291}\newcommand{\titel}{Arthur Schnitzler an Hugo von Hofmannsthal, [15. 1. 1894]}\newcommand{\editorInnen}{Martin Anton Müller und Gerd-Hermann Susen}%% latex-leseansicht-abspann.tex
%% Abspann für die Leseansicht.
%% Der Schalter \ifkorrekturansicht ist bereits durch den Vorspann gesetzt.

%% latex-abspann.tex
%% Gemeinsamer Abspann für Korrekturansicht und Leseansicht.
%% Setzt den Schalter \ifkorrekturansicht voraus (gesetzt in den
%% einbindenden Dateien latex-korrekturansicht-abspann.tex bzw.
%% latex-leseansicht-abspann.tex).
%% ---------------------------------------------------------------

\normalsize

% Das esempio-Environment wird nur in der Leseansicht benötigt
\ifkorrekturansicht\else
\newenvironment{esempio}[3]%
{
    \vspace{1.5ex}
    \rlap{\underline{#1}}
    \par
    \setlength{\parindent}{0cm}
    \nopagebreak
    \leftskip=#2cm
    \rightskip=#3cm
}
{
    \par
}
\fi

\doendnotes{C}
\bigskip
\vfill

\clearpage

\footnotesize

\ifkorrekturansicht
  \lohead{\textsc{register}}
\fi

% theindex-Environment neu definieren ohne reledmac
\makeatletter
\renewenvironment{theindex}{%
  \ifkorrekturansicht
    \section*{\indexname}%
  \else
    \subsubsection*{Index der erwähnten Entitäten}%
  \fi
  \setlength{\parindent}{0pt}%
  \setlength{\parskip}{0pt plus 0.3pt}%
  \let\item\@idxitem
}{%
  \ifkorrekturansicht\clearpage\fi
}
\makeatother

\IfFileExists{\jobname-pw.ind}{\input{\jobname-pw.ind}}{}

% Quellenangabe nur in der Leseansicht
\ifkorrekturansicht\else
% Fallback-Definitionen, falls die .tex-Datei \titel etc. nicht gesetzt hat
\providecommand{\titel}{}
\providecommand{\editorInnen}{}
\providecommand{\dateiname}{\jobname}

\vspace{3cm}

\vfill

\footnotesize
\textsc{Quelle}: \titel. Herausgegeben von {\editorInnen}. In: \emph{Arthur Schnitzler: Briefwechsel mit Autorinnen und Autoren}.
 Digitale Edition, https://schnitzler-briefe.acdh.oeaw.ac.at/{\dateiname}.html (Stand \today)
\fi

\end{document}


      