%% latex-korrekturansicht-vorspann.tex
%% Vorspann für die Korrekturansicht.
%% Lädt die gemeinsame Datei latex-vorspann.tex mit gesetztem Schalter.

\newif\ifkorrekturansicht
\korrekturansichttrue

\input{../tex-inputs/latex-vorspann}


\section[Arthur Schnitzler an Hugo von Hofmannsthal, {[}15. 1. 1894{]}]{L00291 Arthur Schnitzler an Hugo von Hofmannsthal, {[}15. 1. 1894{]}}
\nopagebreak\mylabel{L00291v}
\rehead{ }\normalsize\beginnumbering\briefempfaengerindex{Hofmannsthal, Hugo von@\textsc{Hofmannsthal, Hugo von}!zzzSchnitzler, Arthur@\emph{von Arthur Schnitzler}!1894-01-151@{{[}15. 1. 1894{]}}|(be}
\toendnotes[C]{\smallbreak\pagebreak[2]}\Standort{FDH, Hs-30885,40.}
\physDesc{Brief, 1 Blatt, 3 Seiten, 392 Zeichen (Briefpapier mit Trauerrand)
\newline{}Handschrift: schwarze Tinte, deutsche Kurrent
\newline{}Ordnung: von unbekannter Hand datiert: »93« }
\buchAbdrucke{\weitereDrucke{Hugo von Hofmannsthal, Arthur Schnitzler: \emph{Briefwechsel}. Frankfurt am Main: \emph{S. Fischer} 1964, S. 48–49.} }\toendnotes[C]{\smallbreak}
\pstart{}{\pb}Lieber Hugo,\pend\vspace{0.5em}
\pstart
           \label{K_L00291-1v}\edtext{Sonntag}{\lemma{\textnormal{\emph{Sonntag}}}\Cendnote{\textnormal{Schnitzler und Hofmannsthal\pwindex{Hofmannsthal, Hugo von 1874-02-01 – 1929-07-15@\textsc{Hofmannsthal, Hugo von} (1874-02-01 – 1929-07-15), \emph{Schriftsteller/Schriftstellerin}|pwk} besuchten die angesprochene Aufführung am
                     21. 1. 1894, die im Zuge eines Gastspiels am Carl-Theater\oindex{Carl-Theater@\textbf{Carl-Theater}, \emph{Theater (K.THE)}|pwk} stattfand (A. S.: \emph{Tagebuch}, 21. 1. 1893, Hugo von Hofmannsthal\pwindex{Hofmannsthal, Hugo von 1874-02-01 – 1929-07-15@\textsc{Hofmannsthal, Hugo von} (1874-02-01 – 1929-07-15), \emph{Schriftsteller/Schriftstellerin}|pwk}: \emph{Aufzeichnungen}. Herausgegeben von Rudolf Hirsch † und Ellen Ritter † in
                     Zusammenarbeit mit Konrad Heumann und Peter Michael Braunwarth. Frankfurt am
                     Main: \emph{S. Fischer}\orgindex{S. Fischer Verlag@S. Fischer Verlag|pwk}{ }2013, S. 265 (\emph{Sämtliche Werke},
                     XXXIX)).}}}\label{K_L00291-1} gibt \textsc{Mounet-Sully}\pwindex{Mounet-Sully, Jean 27.02.1841 – 01.03.1916@\textsc{Mounet-Sully, Jean} (27.02.1841 – 01.03.1916), \emph{Schauspieler/Schauspielerin, Rechtsanwalt/Rechtsanwältin}|pw} den \textsc{Hamlet}\pwindex{Hamlet@\emph{Hamlet}|pw}; da möcht ich gern hineingehn. Sie auch? Soll ich für uns beide Sitze nehmen?
               Was für eine Su{\geminationm}e {\pb}wollen Sie
               eventuell dieſem Zwecke widmen?\pend
           
\pstart
           – \label{K_L00291-2v}\edtext{Heut}{\lemma{\textnormal{\emph{Heut}}}\Cendnote{\textnormal{Am 15. 1. 1894 war Schnitzler in der Premiere von \emph{Der
                     ungläubige Thomas}\pwindex{unglaeubige Thomas@\emph{Der ungläubige Thomas}|pwk} von Karl Laufs\pwindex{Laufs, Carl 20.12.1858 – 13.08.1900@\textsc{Laufs, Carl} (20.12.1858 – 13.08.1900), \emph{Schriftsteller/Schriftstellerin}|pwk} und
                     Wilhelm Jacoby\pwindex{Jacoby, Wilhelm 08.03.1855 – 20.02.1925@\textsc{Jacoby, Wilhelm} (08.03.1855 – 20.02.1925), \emph{Schriftsteller/Schriftstellerin}|pwk} am Raimundtheater\oindex{Raimund-Theater@\textbf{Raimund-Theater}, \emph{Theater (K.THE)}|pwk}. (\emph{Cambridge University Library}, A 179.)}}}\label{K_L00291-2} geh
               ich zum ungläubigen \textsc{Thomas}\pwindex{unglaeubige Thomas@\emph{Der ungläubige Thomas}|pw}, \label{K_L00291-3v}\edtext{morgen}{\lemma{\textnormal{\emph{morgen}}}\Cendnote{\textnormal{Victorien Sardous\pwindex{Sardou, Victorien 07.09.1831 – 08.11.1908@\textsc{Sardou, Victorien} (07.09.1831 – 08.11.1908), \emph{Schriftsteller/Schriftstellerin}|pwk}{ }\emph{Madame Sans-Gêne}\pwindex{Madame Sans-Gêne. Comedie en 3 actes et 1 prologue@\emph{Madame Sans-Gêne. Comédie en 3 actes et 1 prologue}|pwk} wurde am 16. 1. 1894 im Deutschen Volkstheater\oindex{Volkstheater@\textbf{Volkstheater}, \emph{Theater (K.THE)}|pwk} gegeben, Schnitzler war anwesend. (\emph{Cambridge University Library}, A 179.)}}}\label{K_L00291-3} zu
                  \textsc{Madame Sans-gêne}\pwindex{Madame Sans-Gêne. Comedie en 3 actes et 1 prologue@\emph{Madame Sans-Gêne. Comédie en 3 actes et 1 prologue}|pw}. Bin äußerſt kunſtſinnig. –\pend
           
\pstart
           – Beifolgende ergreifende \label{K_L00291-4v}\edtext{Erzählung}{\lemma{\textnormal{\emph{Erzählung}}}\Cendnote{\textnormal{nicht
                  identifiziert}}}\label{K_L00291-4} iſt mit Andacht zu leſen.\pend
           \pstart {\pb}Herzlich Ihr Arthur, der eine baldige Antwort erwartet. – \pend{}
\pstart
           \noindent{}\uline{Montag.}\pend
           \selectlanguage{ngerman}\endnumbering\briefempfaengerindex{Hofmannsthal, Hugo von@\textsc{Hofmannsthal, Hugo von}!zzzSchnitzler, Arthur@\emph{von Arthur Schnitzler}!1894-01-151@{{[}15. 1. 1894{]}}|)be}\mylabel{L00291h}  \normalsize

\doendnotes{C}
\bigskip
\vfill

\clearpage

\footnotesize

\lohead{\textsc{register}}

% Definiere theindex-Environment komplett neu ohne reledmac
\makeatletter
\renewenvironment{theindex}{%
  \section*{\indexname}%
  \setlength{\parindent}{0pt}%
  \setlength{\parskip}{0pt plus 0.3pt}%
  \let\item\@idxitem
}{%
  \clearpage
}
\makeatother

\IfFileExists{\jobname-pw.ind}{\input{\jobname-pw.ind}}{}

\end{document}

      