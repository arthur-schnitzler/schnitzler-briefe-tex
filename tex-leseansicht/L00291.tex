%% latex-leseansicht-vorspann.tex
%% Vorspann für die Leseansicht.
%% Lädt die gemeinsame Datei latex-vorspann.tex mit nicht gesetztem Schalter.

\newif\ifkorrekturansicht
\korrekturansichtfalse

\input{../tex-inputs/latex-vorspann}


\section[Arthur Schnitzler an Hugo von Hofmannsthal, {{[}}15. 1. 1894{{]}}]{L00291 Arthur Schnitzler an Hugo von Hofmannsthal, {[}15. 1. 1894{]}}
\nopagebreak\mylabel{L00291v}
\rehead{ }\normalsize\beginnumbering\briefempfaengerindex{Hofmannsthal, Hugo von@\textsc{Hofmannsthal, Hugo von}!zzzSchnitzler, Arthur@\emph{von Arthur Schnitzler}!1894-01-151@{{[}15. 1. 1894{]}}|(be}
\toendnotes[C]{\smallbreak\pagebreak[2]}
\correspDesc{Versand  durch Arthur Schnitzler am [15. 1. 1894] in Wien
\newline{}Erhalt  durch Hugo von Hofmannsthal im Zeitraum [15. 1. 1894
                  – 19. 1. 1894?] in Wien}\toendnotes[C]{\smallbreak}
\Standort{FDH, Hs-30885,40.}
\physDesc{Brief, 1 Blatt, 3 Seiten, 392 Zeichen (Briefpapier mit Trauerrand)
\newline{}Handschrift: schwarze Tinte, deutsche Kurrent
\newline{}Ordnung: von unbekannter Hand datiert: »93« }
\buchAbdrucke{\weitereDrucke{Hugo von Hofmannsthal, Arthur Schnitzler: \emph{Briefwechsel}. Herausgegeben von Therese Nickl und Heinrich Schnitzler. Frankfurt am Main: \emph{S. Fischer} 1964, S. 48–49.} }\toendnotes[C]{\smallbreak}
\pstart{}{\pb}Lieber Hugo,\pend\vspace{0.5em}
\pstart
           \label{K_L00291-1v}\edtext{Sonntag}{\lemma{\textnormal{\emph{Sonntag}}}\Cendnote{\textnormal{Schnitzler und Hofmannsthal\pwindex{Hofmannsthal, Hugo von 1.\,2.\,1874 Wien – 15.\,7.\,1929 Rodaun@\textsc{Hofmannsthal, Hugo von} (1.\,2.\,1874 Wien – 15.\,7.\,1929 Rodaun), \emph{Schriftsteller}|pwk} besuchten die angesprochene Aufführung am
                     21. 1. 1894, die im Zuge eines Gastspiels am Carl-Theater\oindex{Wien@\textbf{Wien}!II., Leopoldstadt@\textbf{II., Leopoldstadt}!Carl-Theater@\textbf{Carl-Theater}, \emph{Theater}|pwk} stattfand (A. S.: \emph{Tagebuch}, 21. 1. 1893, Hugo von Hofmannsthal\pwindex{Hofmannsthal, Hugo von 1.\,2.\,1874 Wien – 15.\,7.\,1929 Rodaun@\textsc{Hofmannsthal, Hugo von} (1.\,2.\,1874 Wien – 15.\,7.\,1929 Rodaun), \emph{Schriftsteller}|pwk}: \emph{Aufzeichnungen}. Herausgegeben von Rudolf Hirsch † und Ellen Ritter † in
                     Zusammenarbeit mit Konrad Heumann und Peter Michael Braunwarth. Frankfurt am
                     Main: \emph{S. Fischer}\orgindex{S. Fischer Verlag@S. Fischer Verlag|pwk}{ }2013, S. 265 (\emph{Sämtliche Werke},
                     XXXIX)).}}}\label{K_L00291-1} gibt \textsc{Mounet-Sully}\pwindex{Mounet-Sully, Jean 27.\,2.\,1841 Bergerac – 1.\,3.\,1916 Paris@\textsc{Mounet-Sully, Jean} (27.\,2.\,1841 Bergerac – 1.\,3.\,1916 Paris), \emph{Schauspieler, Rechtsanwalt}|pw} den \textsc{Hamlet}\pwindex{\textcolor{red}{\textsuperscript{XXXX indx1}}!Hamlet@\strich\emph{Hamlet}|pw}; da möcht ich gern hineingehn. Sie auch? Soll ich für uns beide Sitze nehmen?
               Was für eine Su{\geminationm}e {\pb}wollen Sie
               eventuell dieſem Zwecke widmen?\pend
           
\pstart
           – \label{K_L00291-2v}\edtext{Heut}{\lemma{\textnormal{\emph{Heut}}}\Cendnote{\textnormal{Am 15. 1. 1894 war Schnitzler in der Premiere von \emph{Der
                     ungläubige Thomas}\pwindex{Jacoby, Wilhelm 8.\,3.\,1855 Mainz – 20.\,2.\,1925 Wiesbaden@\textsc{Jacoby, Wilhelm} (8.\,3.\,1855 Mainz – 20.\,2.\,1925 Wiesbaden), \emph{Schriftsteller}!ungläubige Thomas@\strich\emph{Der ungläubige Thomas}|pwk}\pwindex{Laufs, Carl 20.\,12.\,1858 Mainz – 13.\,8.\,1900 Kassel@\textsc{Laufs, Carl} (20.\,12.\,1858 Mainz – 13.\,8.\,1900 Kassel), \emph{Schriftsteller}!ungläubige Thomas@\strich\emph{Der ungläubige Thomas}|pwk} von Karl Laufs\pwindex{Laufs, Carl 20.\,12.\,1858 Mainz – 13.\,8.\,1900 Kassel@\textsc{Laufs, Carl} (20.\,12.\,1858 Mainz – 13.\,8.\,1900 Kassel), \emph{Schriftsteller}|pwk} und
                     Wilhelm Jacoby\pwindex{Jacoby, Wilhelm 8.\,3.\,1855 Mainz – 20.\,2.\,1925 Wiesbaden@\textsc{Jacoby, Wilhelm} (8.\,3.\,1855 Mainz – 20.\,2.\,1925 Wiesbaden), \emph{Schriftsteller}|pwk} am Raimundtheater\oindex{Wien@\textbf{Wien}!VI., Mariahilf@\textbf{VI., Mariahilf}!Raimund-Theater@\textbf{Raimund-Theater}, \emph{Theater}|pwk}. (\emph{Cambridge University Library}, A 179.)}}}\label{K_L00291-2} geh
               ich zum ungläubigen \textsc{Thomas}\pwindex{Jacoby, Wilhelm 8.\,3.\,1855 Mainz – 20.\,2.\,1925 Wiesbaden@\textsc{Jacoby, Wilhelm} (8.\,3.\,1855 Mainz – 20.\,2.\,1925 Wiesbaden), \emph{Schriftsteller}!ungläubige Thomas@\strich\emph{Der ungläubige Thomas}|pw}\pwindex{Laufs, Carl 20.\,12.\,1858 Mainz – 13.\,8.\,1900 Kassel@\textsc{Laufs, Carl} (20.\,12.\,1858 Mainz – 13.\,8.\,1900 Kassel), \emph{Schriftsteller}!ungläubige Thomas@\strich\emph{Der ungläubige Thomas}|pw}, \label{K_L00291-3v}\edtext{morgen}{\lemma{\textnormal{\emph{morgen}}}\Cendnote{\textnormal{Victorien Sardous\pwindex{Sardou, Victorien 7.\,9.\,1831 Paris – 8.\,11.\,1908 ebd.@\textsc{Sardou, Victorien} (7.\,9.\,1831 Paris – 8.\,11.\,1908 ebd.), \emph{Schriftsteller}|pwk}{ }\emph{Madame Sans-Gêne}\pwindex{\textcolor{red}{\textsuperscript{XXXX indx1}}!Madame Sans-Gêne. Comédie en 3 actes et 1 prologue@\strich\emph{Madame Sans-Gêne. Comédie en 3 actes et 1 prologue}|pwk}\pwindex{Sardou, Victorien 7.\,9.\,1831 Paris – 8.\,11.\,1908 ebd.@\textsc{Sardou, Victorien} (7.\,9.\,1831 Paris – 8.\,11.\,1908 ebd.), \emph{Schriftsteller}!Madame Sans-Gêne. Comédie en 3 actes et 1 prologue@\strich\emph{Madame Sans-Gêne. Comédie en 3 actes et 1 prologue}|pwk} wurde am 16. 1. 1894 im Deutschen Volkstheater\oindex{Wien@\textbf{Wien}!VII., Neubau@\textbf{VII., Neubau}!Volkstheater@\textbf{Volkstheater}, \emph{Theater}|pwk} gegeben, Schnitzler war anwesend. (\emph{Cambridge University Library}, A 179.)}}}\label{K_L00291-3} zu
                  \textsc{Madame Sans-gêne}\pwindex{\textcolor{red}{\textsuperscript{XXXX indx1}}!Madame Sans-Gêne. Comédie en 3 actes et 1 prologue@\strich\emph{Madame Sans-Gêne. Comédie en 3 actes et 1 prologue}|pw}\pwindex{Sardou, Victorien 7.\,9.\,1831 Paris – 8.\,11.\,1908 ebd.@\textsc{Sardou, Victorien} (7.\,9.\,1831 Paris – 8.\,11.\,1908 ebd.), \emph{Schriftsteller}!Madame Sans-Gêne. Comédie en 3 actes et 1 prologue@\strich\emph{Madame Sans-Gêne. Comédie en 3 actes et 1 prologue}|pw}. Bin äußerſt kunſtſinnig. –\pend
           
\pstart
           – Beifolgende ergreifende \label{K_L00291-4v}\edtext{Erzählung}{\lemma{\textnormal{\emph{Erzählung}}}\Cendnote{\textnormal{nicht
                  identifiziert}}}\label{K_L00291-4} iſt mit Andacht zu leſen.\pend
           \pstart {\pb}Herzlich Ihr Arthur, der eine baldige Antwort erwartet. –\pend{}
\pstart
           \noindent{}\uline{Montag.}\pend
           \selectlanguage{ngerman}\endnumbering\briefempfaengerindex{Hofmannsthal, Hugo von@\textsc{Hofmannsthal, Hugo von}!zzzSchnitzler, Arthur@\emph{von Arthur Schnitzler}!1894-01-151@{{[}15. 1. 1894{]}}|)be}\mylabel{L00291h}  \newcommand{\dateiname}{L00291}\newcommand{\titel}{Arthur Schnitzler an Hugo von Hofmannsthal, [15. 1. 1894]}\newcommand{\editorInnen}{Martin Anton Müller und Gerd-Hermann Susen}%% latex-leseansicht-abspann.tex
%% Abspann für die Leseansicht.
%% Der Schalter \ifkorrekturansicht ist bereits durch den Vorspann gesetzt.

%% latex-abspann.tex
%% Gemeinsamer Abspann für Korrekturansicht und Leseansicht.
%% Setzt den Schalter \ifkorrekturansicht voraus (gesetzt in den
%% einbindenden Dateien latex-korrekturansicht-abspann.tex bzw.
%% latex-leseansicht-abspann.tex).
%% ---------------------------------------------------------------

\normalsize

% Das esempio-Environment wird nur in der Leseansicht benötigt
\ifkorrekturansicht\else
\newenvironment{esempio}[3]%
{
    \vspace{1.5ex}
    \rlap{\underline{#1}}
    \par
    \setlength{\parindent}{0cm}
    \nopagebreak
    \leftskip=#2cm
    \rightskip=#3cm
}
{
    \par
}
\fi

\doendnotes{C}
\bigskip
\vfill

\clearpage

\footnotesize

\ifkorrekturansicht
  \lohead{\textsc{register}}
\fi

% theindex-Environment neu definieren ohne reledmac
\makeatletter
\renewenvironment{theindex}{%
  \ifkorrekturansicht
    \section*{\indexname}%
  \else
    \subsubsection*{Index der erwähnten Entitäten}%
  \fi
  \setlength{\parindent}{0pt}%
  \setlength{\parskip}{0pt plus 0.3pt}%
  \let\item\@idxitem
}{%
  \ifkorrekturansicht\clearpage\fi
}
\makeatother

\IfFileExists{\jobname-pw.ind}{\input{\jobname-pw.ind}}{}

% Quellenangabe nur in der Leseansicht
\ifkorrekturansicht\else
% Fallback-Definitionen, falls die .tex-Datei \titel etc. nicht gesetzt hat
\providecommand{\titel}{}
\providecommand{\editorInnen}{}
\providecommand{\dateiname}{\jobname}

\vspace{3cm}

\vfill

\footnotesize
\textsc{Quelle}: \titel. Herausgegeben von {\editorInnen}. In: \emph{Arthur Schnitzler: Briefwechsel mit Autorinnen und Autoren}.
 Digitale Edition, https://schnitzler-briefe.acdh.oeaw.ac.at/{\dateiname}.html (Stand \today)
\fi

\end{document}


