%% latex-leseansicht-vorspann.tex
%% Vorspann für die Leseansicht.
%% Lädt die gemeinsame Datei latex-vorspann.tex mit nicht gesetztem Schalter.

\newif\ifkorrekturansicht
\korrekturansichtfalse

\input{../tex-inputs/latex-vorspann}


\section[Arthur Schnitzler an Gustav Schwarzkopf, 21. 7. 1899]{L04141 Arthur Schnitzler an Gustav Schwarzkopf, 21. 7. 1899}
\nopagebreak\mylabel{L04141v}
\rehead{ }\normalsize\beginnumbering\briefempfaengerindex{Schwarzkopf, Gustav@\textsc{Schwarzkopf, Gustav}!zzzSchnitzler, Arthur@\emph{von Arthur Schnitzler}!1899-07-211@{21. 7. 1899}|(be}
\toendnotes[C]{\smallbreak\pagebreak[2]}
\correspDesc{Versand  durch Arthur Schnitzler am 21. 7. 1899 in Velden am Wörthersee
\newline{}Erhalt  durch Gustav Schwarzkopf im Zeitraum [22. 7. 1899 – 26. 7. 1899?] in Wien}\toendnotes[C]{\smallbreak}
\Standort{CUL, Schnitzler, B 96.}
\physDesc{Briefkarte, 913 Zeichen
\newline{}Handschrift: Bleistift, deutsche Kurrent}\toendnotes[C]{\smallbreak}
\pstart
           \noindent{}{\pb}lieber Guſtav, meine \label{K_L04141-1v}\edtext{Karte}{\lemma{\textnormal{\emph{Karte}}}\Cendnote{\textnormal{{XXXX ref}. XXXX }}}\label{K_L04141-1} haben Sie wohl erhalten.
               Es iſt hier\oindex{Velden am Wörthersee@\textbf{Velden am Wörthersee}|pwv} ſehr hübſch;
               Gegend, Radwege, beſonders das Bad\oindex{Strandbad Velden@\textbf{Strandbad Velden}, \emph{Schwimmbad}|pwv};
               die Penſion\oindex{Pension Pundschu@\textbf{Pension Pundschu}, \emph{Hotel}|pwv}sgeſellſchaft
               albern, aber das Eſſen vorzüglich. Gearbeitet hab ich noch nichts – und den 1. Akt
               des Shawl\pwindex{Schnitzler, Arthur 15. 5. 1862 Wien – 21. 10. 1931 ebd.@\textsc{Schnitzler, Arthur} (15. 5. 1862 Wien – 21. 10. 1931 ebd.), \emph{Schriftsteller, Mediziner}!Schleier der Beatrice. Schauspiel in fünf Akten@\strich\emph{Der Schleier der Beatrice. Schauspiel in fünf Akten}|pw} durchgeleſen. – \label{K_L04141-2v}\edtext{Vorgeſtern}{\lemma{\textnormal{\emph{Vorgestern}}}\Cendnote{\textnormal{Vgl. A. S.: \emph{Tagebuch}, 19. 7. 1899. }}}\label{K_L04141-2} hab
               ich mit den Burgers\pwindex{Burger, Rudolf *~6.\,12.\,1866 Wien@\textsc{Burger, Rudolf} (*~6.\,12.\,1866 Wien), \emph{Versicherungsdirektor}|pw}\pwindex{Burger, Caroline 11.\,7.\,1869 Wien – 15.\,3.\,1959 ebd.@\textsc{Burger, Caroline} (11.\,7.\,1869 Wien – 15.\,3.\,1959 ebd.)|pw} eine Radpartie
                  Faakerſee\oindex{Faakersee@\textbf{Faakersee}, \emph{See}|pw} gemacht – es waren \label{K_L04141-3v}\edtext{drei ſtatt vier Perſonen}{\lemma{\textnormal{\emph{drei statt vier Personen}}}\Cendnote{\textnormal{Er bezieht sich auf seine am 18. 3. 1899
                  verstorbene Partnerin Marie Reinhard\pwindex{Reinhard, Marie 13.\,3.\,1871 Wien – 18.\,3.\,1899 ebd.@\textsc{Reinhard, Marie} (13.\,3.\,1871 Wien – 18.\,3.\,1899 ebd.), \emph{Gesangspädagogin}|pwk}, die
                  Schwester von Lola Burger\pwindex{Burger, Caroline 11.\,7.\,1869 Wien – 15.\,3.\,1959 ebd.@\textsc{Burger, Caroline} (11.\,7.\,1869 Wien – 15.\,3.\,1959 ebd.)|pwk}.}}}\label{K_L04141-3} – geſtern
               war ich wieder mit ihnen in Pörtſchach\oindex{Pörtschach am Wörthersee@\textbf{Pörtschach am Wörthersee}|pw}
               zuſammen, ſaßen am See\oindex{Wörthersee@\textbf{Wörthersee}, \emph{See}|pwv}. In
               dieſen Sonnentagen, fern von den Wien\oindex{Wien@\textbf{Wien}, \emph{Verwaltungsgebiet}|pw}er Räuſchen,
               geht alles wieder neu auf, \strikeout{u\textcolor{gray}{nd}{ }\textcolor{gray}{×}\-\textcolor{gray}{×}\-\textcolor{gray}{×}\-\textcolor{gray}{×}}{\pb}\strikeout{\textcolor{gray}{[unleserliche Zeile{]} }{ }\textcolor{gray}{×}\-\textcolor{gray}{×}\-\textcolor{gray}{×}\-\textcolor{gray}{×}\-\textcolor{gray}{×}\-\textcolor{gray}{×}\-\textcolor{gray}{×}} Alle Worte ko{\geminationm}en mir ſo klein und dumm vor;
               daher hab ich die vorige Zeile geſtrichen. – In den einſamen Stunden iſt es
               unerträglich, unerträglich. – Geſtern Abd iſt \textsc{Wasserm.\pwindex{Wassermann, Jakob 10.\,3.\,1873 Fürth – 1.\,1.\,1934 Altaussee@\textsc{Wassermann, Jakob} (10.\,3.\,1873 Fürth – 1.\,1.\,1934 Altaussee), \emph{Schriftsteller}|pw}} gekommen, der hier ſeinen Roman\pwindex{Wassermann, Jakob 10.\,3.\,1873 Fürth – 1.\,1.\,1934 Altaussee@\textsc{Wassermann, Jakob} (10.\,3.\,1873 Fürth – 1.\,1.\,1934 Altaussee), \emph{Schriftsteller}!Geschichte der jungen Renate Fuchs@\strich\emph{Die Geschichte der jungen Renate Fuchs}|pwv}
               abſchreiben will. \label{K_L04141-4v}\edtext{Richard\pwindex{Beer-Hofmann, Richard 11.\,7.\,1866 Wien – 26.\,9.\,1945 New York City@\textsc{Beer-Hofmann, Richard} (11.\,7.\,1866 Wien – 26.\,9.\,1945 New York City), \emph{Schriftsteller}|pw} ko{\geminationm}t}{\lemma{\textnormal{\emph{Richard kommt}}}\Cendnote{\textnormal{Er kam am Montag, dem 24. 7. 1899.}}}\label{K_L04141-4}{ }\substVorne{}\textsuperscript{morgen}\substDazwischen{}Montag\substHinten{} oder übermorgen herüber. –\pend
           
\pstart
           – Wir \label{K_L04141-45v}\edtext{bleiben wahrſcheinlich bis Mitte nächſter Woche}{\lemma{\textnormal{\emph{bleiben … Woche}}}\Cendnote{\textnormal{Die Abreise fand am
                  Freitag, dem 28. 7. 1899 statt.}}}\label{K_L04141-45} da. – (Wir: Mama\pwindex{Schnitzler, Louise 8.\,7.\,1840 Kőszeg – 9.\,9.\,1911 Wien@\textsc{Schnitzler, Louise} (8.\,7.\,1840 Kőszeg – 9.\,9.\,1911 Wien)|pwv}, Giſela\pwindex{Hajek, Gisela 20.\,12.\,1867 Wien – 3.\,2.\,1953 Cambridge@\textsc{Hajek, Gisela} (20.\,12.\,1867 Wien – 3.\,2.\,1953 Cambridge)|pw}, – wohl auch \textsc{Wasserm.\pwindex{Wassermann, Jakob 10.\,3.\,1873 Fürth – 1.\,1.\,1934 Altaussee@\textsc{Wassermann, Jakob} (10.\,3.\,1873 Fürth – 1.\,1.\,1934 Altaussee), \emph{Schriftsteller}|pw}})\pend
           \pstart Seien Sie herzlich gegrüßt. Ihr \spacefill\mbox{A. S.}\pend{}
\pstart
           21. 7. \substVorne{}\textsuperscript{1}\substDazwischen{}9\substHinten{}9{\\}Vormittg\pend
           \selectlanguage{ngerman}\endnumbering\briefempfaengerindex{Schwarzkopf, Gustav@\textsc{Schwarzkopf, Gustav}!zzzSchnitzler, Arthur@\emph{von Arthur Schnitzler}!1899-07-211@{21. 7. 1899}|)be}\mylabel{L04141h}
\begin{anhang}
\end{anhang}\newcommand{\dateiname}{L04141}\newcommand{\titel}{Arthur Schnitzler an Gustav Schwarzkopf, 21. 7. 1899}\newcommand{\editorInnen}{Herausgegeben von Jahnke, SelmaMüller, Martin Anton}%% latex-leseansicht-abspann.tex
%% Abspann für die Leseansicht.
%% Der Schalter \ifkorrekturansicht ist bereits durch den Vorspann gesetzt.

%% latex-abspann.tex
%% Gemeinsamer Abspann für Korrekturansicht und Leseansicht.
%% Setzt den Schalter \ifkorrekturansicht voraus (gesetzt in den
%% einbindenden Dateien latex-korrekturansicht-abspann.tex bzw.
%% latex-leseansicht-abspann.tex).
%% ---------------------------------------------------------------

\normalsize

% Das esempio-Environment wird nur in der Leseansicht benötigt
\ifkorrekturansicht\else
\newenvironment{esempio}[3]%
{
    \vspace{1.5ex}
    \rlap{\underline{#1}}
    \par
    \setlength{\parindent}{0cm}
    \nopagebreak
    \leftskip=#2cm
    \rightskip=#3cm
}
{
    \par
}
\fi

\doendnotes{C}
\bigskip
\vfill

\clearpage

\footnotesize

\ifkorrekturansicht
  \lohead{\textsc{register}}
\fi

% theindex-Environment neu definieren ohne reledmac
\makeatletter
\renewenvironment{theindex}{%
  \ifkorrekturansicht
    \section*{\indexname}%
  \else
    \subsubsection*{Index der erwähnten Entitäten}%
  \fi
  \setlength{\parindent}{0pt}%
  \setlength{\parskip}{0pt plus 0.3pt}%
  \let\item\@idxitem
}{%
  \ifkorrekturansicht\clearpage\fi
}
\makeatother

\IfFileExists{\jobname-pw.ind}{\input{\jobname-pw.ind}}{}

% Quellenangabe nur in der Leseansicht
\ifkorrekturansicht\else
% Fallback-Definitionen, falls die .tex-Datei \titel etc. nicht gesetzt hat
\providecommand{\titel}{}
\providecommand{\editorInnen}{}
\providecommand{\dateiname}{\jobname}

\vspace{3cm}

\vfill

\footnotesize
\textsc{Quelle}: \titel. Herausgegeben von {\editorInnen}. In: \emph{Arthur Schnitzler: Briefwechsel mit Autorinnen und Autoren}.
 Digitale Edition, https://schnitzler-briefe.acdh.oeaw.ac.at/{\dateiname}.html (Stand \today)
\fi

\end{document}


