%% latex-korrekturansicht-vorspann.tex
%% Vorspann für die Korrekturansicht.
%% Lädt die gemeinsame Datei latex-vorspann.tex mit gesetztem Schalter.

\newif\ifkorrekturansicht
\korrekturansichttrue

\input{../tex-inputs/latex-vorspann}


\section[Elsa Plessner an Arthur Schnitzler, 9. 1. 1900]{L03723 Elsa Plessner an Arthur Schnitzler, 9. 1. 1900}
\nopagebreak\mylabel{L03723v}
\rehead{ }\normalsize\beginnumbering\briefempfaengerindex{Schnitzler, Arthur@\textsc{Schnitzler, Arthur}!zzzPlessner, Elsa@\emph{von Elsa Plessner}!1900-01-091@{9. 1. 1900}|(be}
\toendnotes[C]{\smallbreak\pagebreak[2]}\Standort{DLA, A:Schnitzler, HS.1985.1.419.}
\physDesc{Brief,  Blätter, 4 Seiten, 1640 Zeichen
\newline{}Handschrift: , lateinische Kurrent}\toendnotes[C]{\smallbreak}
\pstart
           {\pb}Wien I. Kärnthnerstraße N\textsuperscript{o} 10\oindex{Kaerntner Strasse 10@\textbf{Kärntner Straße 10}, \emph{Wohngebäude (K.WHS)}|pw}\pend
           
\pstart
           \raggedleft{} den 9. Januar 1900\pend
           
\pstart{}Verehrter Herr Doctor!\pend\vspace{0.5em}
\pstart
           Nach langer Zeit erlaube ich mir heute wieder einmal Sie an mein
               bescheidenes Vorhandensein zu erinnern und Ihr Urtheil über eine Arbeit\pwindex{erste Kapitel. Schauspiel in drei Akten@\emph{Das erste Kapitel. Schauspiel in drei Akten}|pwv} zu erbitten. Es ist wiederum ein
                  Stück\pwindex{erste Kapitel. Schauspiel in drei Akten@\emph{Das erste Kapitel. Schauspiel in drei Akten}|pwv}, ein dreiactiges
               Schauspiel, das ich Ihrer Kritik unterbreite, indem ich die leise Hoffnung hege, dass
               dieses – mein letztes Opus\pwindex{erste Kapitel. Schauspiel in drei Akten@\emph{Das erste Kapitel. Schauspiel in drei Akten}|pwv} –
               Ihnen nicht allzusehr missfallen dürfte.\pend
           
\pstart
           {\pb}Ich kann es noch immer nicht verschmerzen, dass ich Ihnen – der Sie doch
               seinerzeit einige Hoffnungen auf mein Talent setzten – fortwährend Enttäuschungen
               bereitet habe und wenn ich auch ein Jahr lang von mir nichts habe hören lassen, so
               dürfen Sie nicht glauben, dass mich \label{K_L03723-1v}\edtext{Ihr abfälliges Utheil}{\lemma{\textnormal{\emph{Ihr abfälliges Utheil}}}\Cendnote{\textnormal{Schnitzlers Brief aus dem Januar 1899 ist
                  nicht überliefert, zu Plessners\pwindex{Plessner, Elsa 22.08.1875 – 01.05.1932@\textsc{Plessner, Elsa} (22.08.1875 – 01.05.1932), \emph{Schriftsteller/Schriftstellerin}|pwk} Reaktion auf
                  seine Kritik vgl. Elsa Plessner an Arthur Schnitzler, 26. 1. 1899.}}}\label{K_L03723-1}
               über das letzte Stück\pwindex{Ehrlosen. Schauspiel in drei Acten@\emph{Die Ehrlosen. Schauspiel in drei Acten}|pwv}
               abgeschreckt habe. Welches äußerliche Schicksal dieses Schauspiel\pwindex{erste Kapitel. Schauspiel in drei Akten@\emph{Das erste Kapitel. Schauspiel in drei Akten}|pwv} auch erleben möge – Sie bleiben
               doch die Verkörperung meines beßeren, literarischen Ich – d. h. ich habe so eine {\pb}dunkle Empfindung als ob ich Ihnen Rechenschaft schuldig wäre, über die
               Verwaltung meines Talentes – .\pend
           
\pstart
           Diese meine Ansicht ist ja vielleicht etwas zeitraubend für Sie – aber wenn etwas
               Ersprießliches dabei herauskommen sollte, glaube ich doch, dass es Ihnen ein wenig
               Spaß macht.\pend
           
\pstart
           Beifolgende feingeäderte Seelengeschichte\pwindex{erste Kapitel. Schauspiel in drei Akten@\emph{Das erste Kapitel. Schauspiel in drei Akten}|pwv} zartester Structur dürfte Ihnen beweisen, dass ich mich
               bemühe, nicht zu verflachen. Ob mein Bemühen auch {\pb}Erfolg hatte – das
               bitte ich Sie, mir zu sagen. Besonders der dritte Act liegt mir, am Herzen und ich
               erwarte Ihr Urtheil über »das erste Capitel\pwindex{erste Kapitel. Schauspiel in drei Akten@\emph{Das erste Kapitel. Schauspiel in drei Akten}|pw}«
               mit unbeschreiblicher Spannung.\pend
           
\pstart
           Seien Sie mir nicht böse über diesen neuerlichen Überfall und üben Sie – wie immer –
               strenges Recht.\pend
           
\pstart
           Mit vorzüglicher Hochachtung{\\[\baselineskip]}\spacefill\mbox{Elsa Plessner}\pend
           \leftskip=0em{}\selectlanguage{ngerman}\endnumbering\briefempfaengerindex{Schnitzler, Arthur@\textsc{Schnitzler, Arthur}!zzzPlessner, Elsa@\emph{von Elsa Plessner}!1900-01-091@{9. 1. 1900}|)be}\mylabel{L03723h}
\begin{anhang}
\end{anhang}\normalsize

\doendnotes{C}
\bigskip
\vfill

\clearpage

\footnotesize

\lohead{\textsc{register}}

% Definiere theindex-Environment komplett neu ohne reledmac
\makeatletter
\renewenvironment{theindex}{%
  \section*{\indexname}%
  \setlength{\parindent}{0pt}%
  \setlength{\parskip}{0pt plus 0.3pt}%
  \let\item\@idxitem
}{%
  \clearpage
}
\makeatother

\IfFileExists{\jobname-pw.ind}{\input{\jobname-pw.ind}}{}

\end{document}

      