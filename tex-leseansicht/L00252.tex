%% latex-leseansicht-vorspann.tex
%% Vorspann für die Leseansicht.
%% Lädt die gemeinsame Datei latex-vorspann.tex mit nicht gesetztem Schalter.

\newif\ifkorrekturansicht
\korrekturansichtfalse

\input{../tex-inputs/latex-vorspann}


         
         \newcommand{\erwaehntePersonen}{Personen:  ?? [Regimentsarzt], Alois Bahr}
         \newcommand{\erwaehnteInstitutionen}{}
         \newcommand{\erwaehnteOrte}{Orte: Böhmen, Wien}
         \newcommand{\erwaehnteWerke}{Werke: Amerika, An der schönen blauen Donau}
               \section[Hermann Bahr an Arthur Schnitzler, {[}12. 8. 1893{]}]{ Hermann Bahr an Arthur Schnitzler, {[}12. 8. 1893{]}}\nopagebreak\mylabel{v}\rehead{ }\begin{ledgroupsized}[t]{13cm}\normalsize\beginnumbering \toendnotes[C]{\smallbreak\pagebreak[2]} \Standort{CUL, Schnitzler, B 5b.}
\physDesc{Brief, 1 Blatt, 3 Seiten
\newline{}Handschrift: schwarze Tinte, deutsche Kurrent
\newline{}Schnitzler: mit Bleistift datiert: »Mitte Aug 93« \newline{}Ordnung: 1) mit rotem Buntstift von unbekannter Hand nummeriert:
                                    »12«  2) mit Bleistift von unbekannter Hand nummeriert:
                                    »12«}\buchAbdrucke{\weitereDrucke{Hermann Bahr, Arthur Schnitzler: \emph{Briefwechsel, Aufzeichnungen, Dokumente (1891–1931)}. Hg. Kurt Ifkovits und Martin Anton Müller. Göttingen: \emph{Wallstein} 2018, S. 36.} }\toendnotes[C]{\smallbreak}\pstart\center{}{\pb}Lieber Freund!\pend\pstart
           Ich bin verzweifelt. Ihr Brief trifft mich im Packen – \label{K_L00252_1v}\edtext{ich verreiſe}{\lemma{\textnormal{\emph{ich verreiſe}}}\Cendnote{\textnormal{an
                  seinen Vater\pwindex{Bahr, Alois 11.04.1834 – 05.09.1898@\textsc{Bahr, Alois} (11.04.1834 – 05.09.1898), \emph{Notar, Politiker}|pwkv},
                     12. 8. 1893: »Ich verreise heute Abend auf einige Tage nach
                        Böhmen\oindex{Boehmen@\textbf{Böhmen}|pw} und kann keine Adresse angeben, da
                     ich sie selber noch nicht weiß und mich auch nirgends länger als ein paar
                     Stunden aufhalten werde.« (\emph{Theatermuseum Wien}, AM 50775 Ba)}}}\label{K_L00252_1h} heute auf ein paar Tage.
               Ich fange alſo ſofort zu suchen an – denn irgendwo habe ich ja dieſes verruchte
                  \label{K_L00252_2v}\edtext{Amerika\pwindex{Schnitzler, Arthur 15.05.1862 – 21.10.1931@\textsc{Schnitzler, Arthur} (15.05.1862 – 21.10.1931), \emph{Schriftsteller, Mediziner}!Amerika01. 05. 1889@\strich\emph{Amerika} {[}01. 05. 1889{]}|pw}}{\lemma{\textnormal{\emph{Amerika}}}\Cendnote{\textnormal{Arthur Schnitzler\pwindex{Schnitzler, Arthur 15.05.1862 – 21.10.1931@\textsc{Schnitzler, Arthur} (15.05.1862 – 21.10.1931), \emph{Schriftsteller, Mediziner}|pwk}: \emph{Amerika}\pwindex{Schnitzler, Arthur 15.05.1862 – 21.10.1931@\textsc{Schnitzler, Arthur} (15.05.1862 – 21.10.1931), \emph{Schriftsteller, Mediziner}!Amerika01. 05. 1889@\strich\emph{Amerika} {[}01. 05. 1889{]}|pwk}. In: \emph{An der schönen
                        blauen Donau}\pwindex{?? Werk@Nicht ermittelte Verfasserinnen und Verfasser!der schoenen blauen Donau1886 – 1896@\emph{An der schönen blauen Donau} {[}1886 – 1896{]}|pwk}, Jg. 4, H. 9, {[}1. 5.{]} 1889,
                  S. 197.}}}\label{K_L00252_2h}, aber wo? Ich habe alles von unterſt zu oberſt gekehrt –
               bisher umſonſt. Mittwoch komme ich {\pb}auf ein oder
               zwei Tage zurück u. will dann wie ein Sträfling ſuchen. Sind Sie ſehr böſe, we{\geminationn} ich Sie bis dahin vertröſte?\pend
           \pstart
           Ich muß dann ohnehin zu Ihnen um Ihnen wegen des Regimentsarztes\pwindex{?? [Regimentsarzt] 1893 – 1893@\textsc{?? [Regimentsarzt]} (1893 – 1893)|pwv} zu danken u. Sie zu fragen, in welcher
               Weiſe es für mich angemeſſen ist, mich bei dem Herrn zu \textsc{revan{\pb}chieren}.\pend
           \pstart
           In großer Haſt{\\[\baselineskip]}Ihr treuer{\\[\baselineskip]}\spacefill\mbox{Bahr}\pend
           \leftskip=0em{}\pstart
           \noindent{}Schreiben Sie uns doch einmal ein Feuilleton!\pend
           
         
         \endnumbering\mylabel{h}\end{ledgroupsized}  \newcommand{\dateiname}{L00252}\newcommand{\titel}{Hermann Bahr an Arthur Schnitzler, [12. 8. 1893]}\newcommand{\editorInnen}{ Kurt Ifkovits,  Martin Anton Müller}%% latex-leseansicht-abspann.tex
%% Abspann für die Leseansicht.
%% Der Schalter \ifkorrekturansicht ist bereits durch den Vorspann gesetzt.

%% latex-abspann.tex
%% Gemeinsamer Abspann für Korrekturansicht und Leseansicht.
%% Setzt den Schalter \ifkorrekturansicht voraus (gesetzt in den
%% einbindenden Dateien latex-korrekturansicht-abspann.tex bzw.
%% latex-leseansicht-abspann.tex).
%% ---------------------------------------------------------------

\normalsize

% Das esempio-Environment wird nur in der Leseansicht benötigt
\ifkorrekturansicht\else
\newenvironment{esempio}[3]%
{
    \vspace{1.5ex}
    \rlap{\underline{#1}}
    \par
    \setlength{\parindent}{0cm}
    \nopagebreak
    \leftskip=#2cm
    \rightskip=#3cm
}
{
    \par
}
\fi

\doendnotes{C}
\bigskip
\vfill

\clearpage

\footnotesize

\ifkorrekturansicht
  \lohead{\textsc{register}}
\fi

% theindex-Environment neu definieren ohne reledmac
\makeatletter
\renewenvironment{theindex}{%
  \ifkorrekturansicht
    \section*{\indexname}%
  \else
    \subsubsection*{Index der erwähnten Entitäten}%
  \fi
  \setlength{\parindent}{0pt}%
  \setlength{\parskip}{0pt plus 0.3pt}%
  \let\item\@idxitem
}{%
  \ifkorrekturansicht\clearpage\fi
}
\makeatother

\IfFileExists{\jobname-pw.ind}{\input{\jobname-pw.ind}}{}

% Quellenangabe nur in der Leseansicht
\ifkorrekturansicht\else
% Fallback-Definitionen, falls die .tex-Datei \titel etc. nicht gesetzt hat
\providecommand{\titel}{}
\providecommand{\editorInnen}{}
\providecommand{\dateiname}{\jobname}

\vspace{3cm}

\vfill

\footnotesize
\textsc{Quelle}: \titel. Herausgegeben von {\editorInnen}. In: \emph{Arthur Schnitzler: Briefwechsel mit Autorinnen und Autoren}.
 Digitale Edition, https://schnitzler-briefe.acdh.oeaw.ac.at/{\dateiname}.html (Stand \today)
\fi

\end{document}


      