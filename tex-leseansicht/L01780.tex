%% latex-korrekturansicht-vorspann.tex
%% Vorspann für die Korrekturansicht.
%% Lädt die gemeinsame Datei latex-vorspann.tex mit gesetztem Schalter.

\newif\ifkorrekturansicht
\korrekturansichttrue

\input{../tex-inputs/latex-vorspann}


\section[Olga und Arthur Schnitzler an Hermann Bahr, 6. 7. 1908]{L01780 Olga und Arthur Schnitzler an Hermann Bahr, 6. 7. 1908}
\nopagebreak\mylabel{L01780v}
\rehead{ }\normalsize\beginnumbering\briefempfaengerindex{Bahr, Hermann@\textsc{Bahr, Hermann}!zzzSchnitzler, Arthur@\emph{von Arthur Schnitzler}!1908-07-061@{6. 7. 1908}|(be}\briefempfaengerindex{Bahr, Hermann@\textsc{Bahr, Hermann}!zzzSchnitzler, Olga@\emph{von Olga Schnitzler}!1908-07-061@{6. 7. 1908}|(be}
\toendnotes[C]{\smallbreak\pagebreak[2]}\Standort{TMW, HS AM 60163 Ba.}
\physDesc{Bildpostkarte, 285 Zeichen
\newline{}Handschrift Olga Schnitzler: schwarze Tinte, lateinische Kurrent
\newline{}Handschrift Arthur Schnitzler: schwarze Tinte, deutsche Kurrent
\newline{}Versand: Stempel: »\nobreak{}6. 7. 8\nobreak{}«.  
\newline{}Ordnung: Lochung }
\buchAbdrucke{\weitereDrucke{1) Arthur Schnitzler: \emph{The Letters of Arthur Schnitzler to Hermann Bahr}. Chapel Hill: \emph{The University of North Carolina Press} 1978, S. 102.} \weitereDrucke{2) Hermann Bahr, Arthur Schnitzler: \emph{Briefwechsel, Aufzeichnungen, Dokumente (1891–1931)}. Göttingen: \emph{Wallstein} 2018, S. 405.} }\toendnotes[C]{\smallbreak}\pstart{}{\pb}Herrn\pend{}\pstart{}Hermann Bahr\pend{}\pstart{}Ober St. Veit bei Wien\oindex{Ober Sankt Veit@\textbf{Ober Sankt Veit}, \emph{P.PPLX}|pw}\pend{}\pstart{}Veitlissengasse.\oindex{Veitlissengasse@\textbf{Veitlissengasse}, \emph{Straße (K.STR)}|pw}\pend{}{\bigskip}
\pstart
           \noindent{}\centering{}{\pb}\textcolor{gray}{\textbf{Tirol: \label{T_L01780-1v}\edtext{\uline{Villa Heufler, Seis am Schlern}}{\lemma{\textnormal{\emph{Villa … Schlern}}}\Cendnote{\textnormal{Unterstreichung mit schwarzer
                        Tinte}}}\label{T_L01780-1}\oindex{Villa Heufler@\textbf{Villa Heufler}, \emph{Beherbergungsgebäude (K.BHB)}|pw}, 1000m. Nach dem Aquarell\pwindex{Partie in Seis am Schlern@\emph{Partie in Seis am Schlern}|pwv} von F. A. C. M.
                     Reisch\pwindex{Reisch, Franz August Carl Maria 1862-05-01 – 1942?@\textsc{Reisch, Franz August Carl Maria} (1862-05-01 – 1942?), \emph{Maler/Malerin}|pw}, Meran\oindex{Meran@\textbf{Meran}, \emph{P.PPLA3}|pw}.}}\pend
           \vspace{1em}
\pstart
           \raggedleft{}{\pb}6. Juli{\\}08.\pend
           
\pstart{}Lieber Herr Bahr,\pend\vspace{0.5em}
\pstart
           wir\pwindex{Schnitzler, Olga 17.01.1882 – 13.01.1970@\textsc{Schnitzler, Olga} (17.01.1882 – 13.01.1970), \emph{Schauspieler/Schauspielerin, Sänger/Sängerin}|pwv} haben Ihr wunderschönes
                  \label{K_L01780-1v}\edtext{Feuilleton über Moppchen\pwindex{Hartleben, Katharina Selma 1868 – 1930@\textsc{Hartleben, Katharina Selma} (1868 – 1930), \emph{Schriftsteller/Schriftstellerin}|pw}\pwindex{Moppchen@\emph{Moppchen}|pwv}}{\lemma{\textnormal{\emph{Feuilleton über Moppchen}}}\Cendnote{\textnormal{Hermann Bahr\pwindex{Bahr, Hermann 19.07.1863 – 15.01.1934@\textsc{Bahr, Hermann} (19.07.1863 – 15.01.1934), \emph{Schriftsteller/Schriftstellerin, Kritiker/Kritikerin}|pwk}: \emph{Moppchen}\pwindex{Moppchen@\emph{Moppchen}|pwk}. In: \emph{Neue Freie
                        Presse}\pwindex{Neue Freie Presse@\emph{Neue Freie Presse}|pwk}, Nr. 15.757, 4. 7. 1908. Morgenblatt, S. 1–5
                  (»Moppchen« war der Spitzname von Otto Erich
                     Hartlebens\pwindex{Hartleben, Otto Erich 03.06.1864 – 11.02.1905@\textsc{Hartleben, Otto Erich} (03.06.1864 – 11.02.1905), \emph{Schriftsteller/Schriftstellerin}|pwk} Ehefrau Selma\pwindex{Hartleben, Katharina Selma 1868 – 1930@\textsc{Hartleben, Katharina Selma} (1868 – 1930), \emph{Schriftsteller/Schriftstellerin}|pwk}).}}}\label{K_L01780-1}
               mit Ergriffenheit gelesen, schicken Ihnen die herzlichsten Grüsse und viel gute
               Wünsche für den Sommer.\pend
           \pstart \spacefill\mbox{Olga Schnitzler.}\pend{}\selectlanguage{ngerman}\vspace{1em}
\pstart
           \noindent{}{[}hs. :{]} He\damage{rzl}ichſt dein{\\}\spacefill\mbox{Arthur.}\pend
           \selectlanguage{ngerman}\vspace{1em}
\pstart
           \noindent{}{\pb}{[}hs. :{]} \label{T_L01780-2v}\edtext{Unser Balcon.}{\lemma{\textnormal{\emph{Unser Balcon.}}}\Cendnote{\textnormal{auf dem Motiv mit einem Pfeil markiert}}}\label{T_L01780-2}\pend
           \selectlanguage{ngerman}\endnumbering\briefempfaengerindex{Bahr, Hermann@\textsc{Bahr, Hermann}!zzzSchnitzler, Arthur@\emph{von Arthur Schnitzler}!1908-07-061@{6. 7. 1908}|)be}\briefempfaengerindex{Bahr, Hermann@\textsc{Bahr, Hermann}!zzzSchnitzler, Olga@\emph{von Olga Schnitzler}!1908-07-061@{6. 7. 1908}|)be}\mylabel{L01780h}  \normalsize

\doendnotes{C}
\bigskip
\vfill

\clearpage

\footnotesize

\lohead{\textsc{register}}

% Definiere theindex-Environment komplett neu ohne reledmac
\makeatletter
\renewenvironment{theindex}{%
  \section*{\indexname}%
  \setlength{\parindent}{0pt}%
  \setlength{\parskip}{0pt plus 0.3pt}%
  \let\item\@idxitem
}{%
  \clearpage
}
\makeatother

\IfFileExists{\jobname-pw.ind}{\input{\jobname-pw.ind}}{}

\end{document}

      