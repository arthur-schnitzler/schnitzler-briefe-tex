%% latex-leseansicht-vorspann.tex
%% Vorspann für die Leseansicht.
%% Lädt die gemeinsame Datei latex-vorspann.tex mit nicht gesetztem Schalter.

\newif\ifkorrekturansicht
\korrekturansichtfalse

\input{../tex-inputs/latex-vorspann}


\section[Olga und Arthur Schnitzler an Hermann Bahr, 6. 7. 1908]{L01780 Olga und Arthur Schnitzler an Hermann Bahr, 6. 7. 1908}
\nopagebreak\mylabel{L01780v}
\rehead{ }\normalsize\beginnumbering\briefempfaengerindex{Bahr, Hermann@\textsc{Bahr, Hermann}!zzzSchnitzler, Arthur@\emph{von Arthur Schnitzler}!1908-07-061@{6. 7. 1908}|(be}\briefempfaengerindex{Bahr, Hermann@\textsc{Bahr, Hermann}!zzzSchnitzler, Olga@\emph{von Olga Schnitzler}!1908-07-061@{6. 7. 1908}|(be}
\toendnotes[C]{\smallbreak\pagebreak[2]}
\correspDesc{Versand  durch Olga Schnitzler, Arthur Schnitzler am 6. 7. 1908 in Seis am Schlern
\newline{}Erhalt  durch Hermann Bahr im Zeitraum [7. 7. 1908
                  – 11. 7. 1908?] in Wien}\toendnotes[C]{\smallbreak}
\Standort{TMW, HS AM 60163 Ba.}
\physDesc{Bildpostkarte, 285 Zeichen
\newline{}Handschrift Olga Schnitzler: schwarze Tinte, lateinische Kurrent
\newline{}Handschrift Arthur Schnitzler: schwarze Tinte, deutsche Kurrent
\newline{}Versand: Stempel: »\nobreak{}6. 7. 8\nobreak{}«.  
\newline{}Ordnung: Lochung }
\buchAbdrucke{\weitereDrucke{1) \emph{6. 7. 1908, Abschrift.} In: Arthur Schnitzler: \emph{The Letters of Arthur Schnitzler to Hermann Bahr}. Edited, annotated, and with an introduction, by Donald G. Daviau. Chapel Hill: \emph{The University of North Carolina Press} 1978, S. 102 (University of North Carolina studies in the Germanic languages
                        and literatures, 89).} \weitereDrucke{2) Hermann Bahr, Arthur Schnitzler: \emph{Briefwechsel, Aufzeichnungen, Dokumente (1891–1931)}. Herausgegeben von Kurt Ifkovits und Martin Anton Müller. Göttingen: \emph{Wallstein} 2018, S. 405.} }\toendnotes[C]{\smallbreak}\pstart{}{\pb}Herrn\pend{}\pstart{}Hermann Bahr\pend{}\pstart{}Ober St. Veit bei Wien\oindex{Wien@\textbf{Wien}!XIII., Hietzing@\textbf{XIII., Hietzing}!Ober Sankt Veit@\textbf{Ober Sankt Veit}, \emph{Ehemaliger Ort}|pw}\pend{}\pstart{}Veitlissengasse.\oindex{Wien@\textbf{Wien}!XIII., Hietzing@\textbf{XIII., Hietzing}!Veitlissengasse@\textbf{Veitlissengasse}, \emph{Straße}|pw}\pend{}{\bigskip}
\pstart
           \noindent{}\centering{}{\pb}\textcolor{gray}{\textbf{Tirol: \label{T_L01780-1v}\edtext{\uline{Villa Heufler, Seis am Schlern}}{\lemma{\textnormal{\emph{Villa … Schlern}}}\Cendnote{\textnormal{Unterstreichung mit schwarzer
                        Tinte}}}\label{T_L01780-1}\oindex{Villa Heufler@\textbf{Villa Heufler}, \emph{Beherbergungsgebäude}|pw}, 1000m. Nach dem Aquarell\pwindex{Reisch, Franz August Carl Maria 1.\,5.\,1862 Wien – 1942? Meran@\textsc{Reisch, Franz August Carl Maria} (1.\,5.\,1862 Wien – 1942? Meran), \emph{Maler}!Partie in Seis am Schlern@\strich\emph{Partie in Seis am Schlern}|pwv} von F. A. C. M.
                     Reisch\pwindex{Reisch, Franz August Carl Maria 1.\,5.\,1862 Wien – 1942? Meran@\textsc{Reisch, Franz August Carl Maria} (1.\,5.\,1862 Wien – 1942? Meran), \emph{Maler}|pw}, Meran\oindex{Meran@\textbf{Meran}, \emph{Hauptstadt}|pw}.}}\pend
           \vspace{1em}
\pstart
           \raggedleft{}{\pb}6. Juli{\\}08.\pend
           
\pstart{}Lieber Herr Bahr,\pend\vspace{0.5em}
\pstart
           wir\pwindex{Schnitzler, Olga 17.\,1.\,1882 Wien – 13.\,1.\,1970 Lugano@\textsc{Schnitzler, Olga} (17.\,1.\,1882 Wien – 13.\,1.\,1970 Lugano), \emph{Schauspielerin, Sängerin}|pwv} haben Ihr wunderschönes
                  \label{K_L01780-1v}\edtext{Feuilleton über Moppchen\pwindex{Hartleben, Katharina Selma 1868 Halle (Saale) – 1930 Berlin@\textsc{Hartleben, Katharina Selma} (1868 Halle (Saale) – 1930 Berlin), \emph{Schriftstellerin}|pw}\pwindex{Bahr, Hermann 19.\,7.\,1863 Linz – 15.\,1.\,1934 München@\textsc{Bahr, Hermann} (19.\,7.\,1863 Linz – 15.\,1.\,1934 München), \emph{Schriftsteller, Kritiker}!Moppchen@\strich\emph{Moppchen}|pwv}}{\lemma{\textnormal{\emph{Feuilleton über Moppchen}}}\Cendnote{\textnormal{Hermann Bahr\pwindex{Bahr, Hermann 19.\,7.\,1863 Linz – 15.\,1.\,1934 München@\textsc{Bahr, Hermann} (19.\,7.\,1863 Linz – 15.\,1.\,1934 München), \emph{Schriftsteller, Kritiker}|pwk}: \emph{Moppchen}\pwindex{Bahr, Hermann 19.\,7.\,1863 Linz – 15.\,1.\,1934 München@\textsc{Bahr, Hermann} (19.\,7.\,1863 Linz – 15.\,1.\,1934 München), \emph{Schriftsteller, Kritiker}!Moppchen@\strich\emph{Moppchen}|pwk}. In: \emph{Neue Freie
                        Presse}\pwindex{Neue Freie Presse@\emph{Neue Freie Presse}|pwk}, Nr. 15.757, 4. 7. 1908. Morgenblatt, S. 1–5
                  (»Moppchen« war der Spitzname von Otto Erich
                     Hartlebens\pwindex{Hartleben, Otto Erich 3.\,6.\,1864 Clausthal-Zellerfeld – 11.\,2.\,1905 Salò@\textsc{Hartleben, Otto Erich} (3.\,6.\,1864 Clausthal-Zellerfeld – 11.\,2.\,1905 Salò), \emph{Schriftsteller}|pwk} Ehefrau Selma\pwindex{Hartleben, Katharina Selma 1868 Halle (Saale) – 1930 Berlin@\textsc{Hartleben, Katharina Selma} (1868 Halle (Saale) – 1930 Berlin), \emph{Schriftstellerin}|pwk}).}}}\label{K_L01780-1}
               mit Ergriffenheit gelesen, schicken Ihnen die herzlichsten Grüsse und viel gute
               Wünsche für den Sommer.\pend
           \pstart \spacefill\mbox{Olga Schnitzler.}\pend{}\selectlanguage{ngerman}\vspace{1em}
\pstart
           \noindent{}{[}hs. Schnitzler:{]} He\damage{rzl}ichſt dein{\\}\spacefill\mbox{Arthur.}\pend
           \selectlanguage{ngerman}\vspace{1em}
\pstart
           \noindent{}{\pb}{[}hs. Schnitzler:{]} \label{T_L01780-2v}\edtext{Unser Balcon.}{\lemma{\textnormal{\emph{Unser Balcon.}}}\Cendnote{\textnormal{auf dem Motiv mit einem Pfeil markiert}}}\label{T_L01780-2}\pend
           \selectlanguage{ngerman}\endnumbering\briefempfaengerindex{Bahr, Hermann@\textsc{Bahr, Hermann}!zzzSchnitzler, Arthur@\emph{von Arthur Schnitzler}!1908-07-061@{6. 7. 1908}|)be}\briefempfaengerindex{Bahr, Hermann@\textsc{Bahr, Hermann}!zzzSchnitzler, Olga@\emph{von Olga Schnitzler}!1908-07-061@{6. 7. 1908}|)be}\mylabel{L01780h}  \newcommand{\dateiname}{L01780}\newcommand{\titel}{Olga und Arthur Schnitzler an Hermann Bahr, 6. 7. 1908}\newcommand{\editorInnen}{Herausgegeben von Martin Anton Müller}%% latex-leseansicht-abspann.tex
%% Abspann für die Leseansicht.
%% Der Schalter \ifkorrekturansicht ist bereits durch den Vorspann gesetzt.

%% latex-abspann.tex
%% Gemeinsamer Abspann für Korrekturansicht und Leseansicht.
%% Setzt den Schalter \ifkorrekturansicht voraus (gesetzt in den
%% einbindenden Dateien latex-korrekturansicht-abspann.tex bzw.
%% latex-leseansicht-abspann.tex).
%% ---------------------------------------------------------------

\normalsize

% Das esempio-Environment wird nur in der Leseansicht benötigt
\ifkorrekturansicht\else
\newenvironment{esempio}[3]%
{
    \vspace{1.5ex}
    \rlap{\underline{#1}}
    \par
    \setlength{\parindent}{0cm}
    \nopagebreak
    \leftskip=#2cm
    \rightskip=#3cm
}
{
    \par
}
\fi

\doendnotes{C}
\bigskip
\vfill

\clearpage

\footnotesize

\ifkorrekturansicht
  \lohead{\textsc{register}}
\fi

% theindex-Environment neu definieren ohne reledmac
\makeatletter
\renewenvironment{theindex}{%
  \ifkorrekturansicht
    \section*{\indexname}%
  \else
    \subsubsection*{Index der erwähnten Entitäten}%
  \fi
  \setlength{\parindent}{0pt}%
  \setlength{\parskip}{0pt plus 0.3pt}%
  \let\item\@idxitem
}{%
  \ifkorrekturansicht\clearpage\fi
}
\makeatother

\IfFileExists{\jobname-pw.ind}{\input{\jobname-pw.ind}}{}

% Quellenangabe nur in der Leseansicht
\ifkorrekturansicht\else
% Fallback-Definitionen, falls die .tex-Datei \titel etc. nicht gesetzt hat
\providecommand{\titel}{}
\providecommand{\editorInnen}{}
\providecommand{\dateiname}{\jobname}

\vspace{3cm}

\vfill

\footnotesize
\textsc{Quelle}: \titel. Herausgegeben von {\editorInnen}. In: \emph{Arthur Schnitzler: Briefwechsel mit Autorinnen und Autoren}.
 Digitale Edition, https://schnitzler-briefe.acdh.oeaw.ac.at/{\dateiname}.html (Stand \today)
\fi

\end{document}


