%% latex-leseansicht-vorspann.tex
%% Vorspann für die Leseansicht.
%% Lädt die gemeinsame Datei latex-vorspann.tex mit nicht gesetztem Schalter.

\newif\ifkorrekturansicht
\korrekturansichtfalse

\input{../tex-inputs/latex-vorspann}


         
         \renewcommand{\erwaehntePersonen}{Personen: Paul Goldmann, Olga Schnitzler, Elisabeth Steinrück}
         \renewcommand{\erwaehnteOrte}{Orte: Dolomiten, Lago di Garda, Pörtschach, Vahrn, Valle d’Ampezzo, Wien}
         \renewcommand{\erwaehnteWerke}{}
               \section[ Paul Goldmann an Arthur Schnitzler, 1. 8. {[}1901{]}]{ Paul Goldmann an Arthur Schnitzler, 1. 8. {[}1901{]}}\nopagebreak\mylabel{v}\rehead{ }\begin{ledgroupsized}[t]{13cm}\normalsize\beginnumbering\briefempfaengerindex{Schnitzler, Arthur@\textsc{Schnitzler, Arthur}!zzzGoldmann, Paul@\emph{von Paul Goldmann}!1901-08-012@{1. 8. {[}1901{]}}|(be} \toendnotes[C]{\smallbreak\pagebreak[2]} \Standort{DLA, A:Schnitzler, HS.NZ85.1.3171.}
\physDesc{Brief, 1 Blatt, 2 Seiten, 1199 Zeichen
\newline{}Handschrift: blaue Tinte, deutsche Kurrent
\newline{}Schnitzler: mit Bleistift das Jahr »901« vermerkt }\toendnotes[C]{\smallbreak}\pstart
           \raggedleft{}{\pb}\textsc{Pörtschach\oindex{Poertschach@\textbf{Pörtschach}|pw}}, 1. Auguſt.\pend
           \pstart\center{}Mein lieber Freund,\pend\pstart
           Dank für Deinen lieben Brief.\pend
           \pstart
           Ich muß fort von hier, denn ich kann nicht ſchlafen. Die warme, matte Luft bekommt
               mir ſchlecht. In \label{K_L03076-1v}\edtext{\textsc{Vahrn\oindex{Vahrn@\textbf{Vahrn}|pw}}}{\lemma{\textnormal{\emph{Vahrn}}}\Cendnote{\textnormal{Offenbar hatte Schnitzler\pwindex{Schnitzler, Arthur 15.05.1862 – 21.10.1931@\textsc{Schnitzler, Arthur} (15.05.1862 – 21.10.1931), \emph{Schriftsteller, Mediziner}|pwk} vorgeschlagen, dass Goldmann\pwindex{Goldmann, Paul 31.01.1865 – 25.09.1935@\textsc{Goldmann, Paul} (31.01.1865 – 25.09.1935), \emph{Schriftsteller, Journalist}|pwk} nach Vahrn\oindex{Vahrn@\textbf{Vahrn}|pwk}
                  kommen sollte, wo er sich seit 13. 7. 1901 und noch bis 12. 8. 1901 aufhielt.}}}\label{K_L03076-1h} wäre es dieſelbe
               Geſchichte. Ich muß höher hinauf, in ſtarke und kühle Luft. Euch\pwindex{Schnitzler, Olga 17.01.1882 – 13.01.1970@\textsc{Schnitzler, Olga} (17.01.1882 – 13.01.1970), \emph{Schauspielerin, Sängerin}|pwv} wiederzuſehen wäre ſchön. Aber Wochen
               lang keine Nacht ſchlafen, iſt kein Spaß. Da Du alſo noch nichts Hohes gefunden haſt,
               muß ich ſelbſt ſuchen. Ich gehe von hier in die Dolomiten\oindex{Dolomiten@\textbf{Dolomiten}|pw}. Werde das \textsc{Ampezzo}-Thal\oindex{Valle DAmpezzo@\textbf{Valle d’Ampezzo}|pw} durchprobiren. Wo ich ſchlafen kann,
               bleibe ich ein paar Tage. Es wird ſich alſo leider ſo fügen, {\pb}daß ich erſt den Schluß meines Urlaubs mit Euch\pwindex{Schnitzler, Olga 17.01.1882 – 13.01.1970@\textsc{Schnitzler, Olga} (17.01.1882 – 13.01.1970), \emph{Schauspielerin, Sängerin}|pwv} verbringen kann, wenn Ihr
               in \textsc{Vahrn\oindex{Vahrn@\textbf{Vahrn}|pw}} bleibt. \strikeout{Ende} Ende Auguſt muß ich in Wien\oindex{Wien@\textbf{Wien}|pw} ſein. Samſtag{ }früh fahre ich von hier ab. Da ich nicht weiß, wo ich bleiben werde,
               kann ich Dir noch keine Adreſſe \strikeout{g} geben. Aber das muß
               ſich Sonntag oder Montag entſcheiden. Ich ſchreibe Dir dann ſofort. Laß’ alſo das Suchen
               ſein! Da Du Dich in \textsc{Vahrn\oindex{Vahrn@\textbf{Vahrn}|pw}} wohl fühlſt, bleibe dort. Wenn ich meine Nerven zur Raiſon gebracht haben
               werde, komme ich zu Euch\pwindex{Schnitzler, Olga 17.01.1882 – 13.01.1970@\textsc{Schnitzler, Olga} (17.01.1882 – 13.01.1970), \emph{Schauspielerin, Sängerin}|pwv}, –
                  dorthin\oindex{Vahrn@\textbf{Vahrn}|pwv} oder an den Gardaſee\oindex{Lago di Garda@\textbf{Lago di Garda}|pw}. Einſtweilen geht es mir recht elend.
               Es iſt eine ganz verfluchte Geſchichte, wenn man nicht ſchläft. Viele treue Grüße Dir
               und den lieben Mädchen\pwindex{Schnitzler, Olga 17.01.1882 – 13.01.1970@\textsc{Schnitzler, Olga} (17.01.1882 – 13.01.1970), \emph{Schauspielerin, Sängerin}|pwv}\pwindex{Steinrueck, Elisabeth 19.11.1885 – 07.04.1920@\textsc{Steinrück, Elisabeth} (19.11.1885 – 07.04.1920)|pwv}!\pend
           \pstart Dein \spacefill\mbox{Paul Goldmann}\pend{}
         
         \endnumbering\mylabel{h}\end{ledgroupsized}  \newcommand{\dateiname}{L03076}\newcommand{\titel}{Paul Goldmann an Arthur Schnitzler, 1. 8. [1901]}\newcommand{\editorInnen}{Martin Anton Müller und Laura Untner}%% latex-leseansicht-abspann.tex
%% Abspann für die Leseansicht.
%% Der Schalter \ifkorrekturansicht ist bereits durch den Vorspann gesetzt.

%% latex-abspann.tex
%% Gemeinsamer Abspann für Korrekturansicht und Leseansicht.
%% Setzt den Schalter \ifkorrekturansicht voraus (gesetzt in den
%% einbindenden Dateien latex-korrekturansicht-abspann.tex bzw.
%% latex-leseansicht-abspann.tex).
%% ---------------------------------------------------------------

\normalsize

% Das esempio-Environment wird nur in der Leseansicht benötigt
\ifkorrekturansicht\else
\newenvironment{esempio}[3]%
{
    \vspace{1.5ex}
    \rlap{\underline{#1}}
    \par
    \setlength{\parindent}{0cm}
    \nopagebreak
    \leftskip=#2cm
    \rightskip=#3cm
}
{
    \par
}
\fi

\doendnotes{C}
\bigskip
\vfill

\clearpage

\footnotesize

\ifkorrekturansicht
  \lohead{\textsc{register}}
\fi

% theindex-Environment neu definieren ohne reledmac
\makeatletter
\renewenvironment{theindex}{%
  \ifkorrekturansicht
    \section*{\indexname}%
  \else
    \subsubsection*{Index der erwähnten Entitäten}%
  \fi
  \setlength{\parindent}{0pt}%
  \setlength{\parskip}{0pt plus 0.3pt}%
  \let\item\@idxitem
}{%
  \ifkorrekturansicht\clearpage\fi
}
\makeatother

\IfFileExists{\jobname-pw.ind}{\input{\jobname-pw.ind}}{}

% Quellenangabe nur in der Leseansicht
\ifkorrekturansicht\else
% Fallback-Definitionen, falls die .tex-Datei \titel etc. nicht gesetzt hat
\providecommand{\titel}{}
\providecommand{\editorInnen}{}
\providecommand{\dateiname}{\jobname}

\vspace{3cm}

\vfill

\footnotesize
\textsc{Quelle}: \titel. Herausgegeben von {\editorInnen}. In: \emph{Arthur Schnitzler: Briefwechsel mit Autorinnen und Autoren}.
 Digitale Edition, https://schnitzler-briefe.acdh.oeaw.ac.at/{\dateiname}.html (Stand \today)
\fi

\end{document}


      