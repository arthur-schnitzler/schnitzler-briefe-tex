%% latex-leseansicht-vorspann.tex
%% Vorspann für die Leseansicht.
%% Lädt die gemeinsame Datei latex-vorspann.tex mit nicht gesetztem Schalter.

\newif\ifkorrekturansicht
\korrekturansichtfalse

\input{../tex-inputs/latex-vorspann}


\section[ Paul Goldmann an Arthur Schnitzler, 1. 8. [1901]]{L03076 Paul Goldmann an Arthur Schnitzler,  1. 8. [1901]}
\nopagebreak\mylabel{L03076v}
\rehead{ }\normalsize\beginnumbering\briefempfaengerindex{Schnitzler, Arthur@\textsc{Schnitzler, Arthur}!zzzGoldmann, Paul@\emph{von Paul Goldmann}!1901-08-012@{1. 8. [1901]}|(be}
\toendnotes[C]{\smallbreak\pagebreak[2]}
\correspDesc{Versand  durch Paul Goldmann am 1. 8. [1901] in Pörtschach
\newline{}Erhalt  durch Arthur Schnitzler im Zeitraum [2. 8. 1901
                  – 6. 8. 1901?] in Vahrn}\toendnotes[C]{\smallbreak}
\Standort{DLA, A:Schnitzler, HS.NZ85.1.3171.}
\physDesc{Brief, 1 Blatt, 2 Seiten, 1199 Zeichen
\newline{}Handschrift: blaue Tinte, deutsche Kurrent
\newline{}Schnitzler: mit Bleistift das Jahr »901« vermerkt }\toendnotes[C]{\smallbreak}
\pstart
           \raggedleft{}{\pb}\textsc{Pörtschach\oindex{Pörtschach am Wörthersee@\textbf{Pörtschach am Wörthersee}|pw}}, 1. Auguſt.\pend
           
\pstart\center{}Mein lieber Freund,\pend\vspace{0.5em}
\pstart
           Dank für Deinen lieben Brief.\pend
           
\pstart
           Ich muß fort von hier, denn ich kann nicht{ }ſchlafen. Die warme, matte Luft bekommt
               mir{ }ſchlecht. In \label{K_L03076-1v}\edtext{\textsc{Vahrn\oindex{Vahrn@\textbf{Vahrn}, \emph{Hauptstadt}|pw}}}{\lemma{\textnormal{\emph{Vahrn}}}\Cendnote{\textnormal{Offenbar hatte Schnitzler vorgeschlagen, dass Goldmann\pwindex{Goldmann, Paul 31.\,1.\,1865 Breslau – 25.\,9.\,1935 Wien@\textsc{Goldmann, Paul} (31.\,1.\,1865 Breslau – 25.\,9.\,1935 Wien), \emph{Schriftsteller, Journalist}|pwk} nach Vahrn\oindex{Vahrn@\textbf{Vahrn}, \emph{Hauptstadt}|pwk}
                  kommen sollte, wo er sich seit 13. 7. 1901 und noch bis 12. 8. 1901 aufhielt.}}}\label{K_L03076-1} wäre es dieſelbe
               Geſchichte. Ich muß höher hinauf, in{ }ſtarke und kühle Luft. Euch\pwindex{Schnitzler, Olga 17.\,1.\,1882 Wien – 13.\,1.\,1970 Lugano@\textsc{Schnitzler, Olga} (17.\,1.\,1882 Wien – 13.\,1.\,1970 Lugano), \emph{Schauspielerin, Sängerin}|pwv} wiederzuſehen wäre{ }ſchön. Aber Wochen
               lang keine Nacht{ }ſchlafen, iſt kein Spaß. Da Du alſo noch nichts Hohes gefunden haſt,
               muß ich{ }ſelbſt{ }ſuchen. Ich gehe von hier in die Dolomiten\oindex{Dolomiten@\textbf{Dolomiten}, \emph{Gebirge}|pw}. Werde das \textsc{Ampezzo}-Thal\oindex{Valle d’Ampezzo@\textbf{Valle d’Ampezzo}, \emph{Tal}|pw} durchprobiren. Wo ich{ }ſchlafen kann,
               bleibe ich ein paar Tage. Es wird{ }ſich alſo leider{ }ſo fügen, {\pb}daß ich erſt den Schluß meines Urlaubs mit Euch\pwindex{Schnitzler, Olga 17.\,1.\,1882 Wien – 13.\,1.\,1970 Lugano@\textsc{Schnitzler, Olga} (17.\,1.\,1882 Wien – 13.\,1.\,1970 Lugano), \emph{Schauspielerin, Sängerin}|pwv} verbringen kann, wenn Ihr
               in \textsc{Vahrn\oindex{Vahrn@\textbf{Vahrn}, \emph{Hauptstadt}|pw}} bleibt. \strikeout{Ende} Ende Auguſt muß ich in Wien\oindex{Wien@\textbf{Wien}, \emph{Verwaltungsgebiet}|pw}{ }ſein. Samſtag{ }früh fahre ich von hier ab. Da ich nicht weiß, wo ich bleiben werde,
               kann ich Dir noch keine Adreſſe \strikeout{g} geben. Aber das muß{ }ſich Sonntag oder Montag entſcheiden. Ich{ }ſchreibe Dir dann{ }ſofort. Laß’ alſo das Suchen{ }ſein! Da Du Dich in \textsc{Vahrn\oindex{Vahrn@\textbf{Vahrn}, \emph{Hauptstadt}|pw}} wohl fühlſt, bleibe dort. Wenn ich meine Nerven zur Raiſon gebracht haben
               werde, komme ich zu Euch\pwindex{Schnitzler, Olga 17.\,1.\,1882 Wien – 13.\,1.\,1970 Lugano@\textsc{Schnitzler, Olga} (17.\,1.\,1882 Wien – 13.\,1.\,1970 Lugano), \emph{Schauspielerin, Sängerin}|pwv}, –
                  dorthin\oindex{Vahrn@\textbf{Vahrn}, \emph{Hauptstadt}|pwv} oder an den Gardaſee\oindex{Lago di Garda@\textbf{Lago di Garda}, \emph{See}|pw}. Einſtweilen geht es mir recht elend.
               Es iſt eine ganz verfluchte Geſchichte, wenn man nicht{ }ſchläft. Viele treue Grüße Dir
               und den lieben Mädchen\pwindex{Schnitzler, Olga 17.\,1.\,1882 Wien – 13.\,1.\,1970 Lugano@\textsc{Schnitzler, Olga} (17.\,1.\,1882 Wien – 13.\,1.\,1970 Lugano), \emph{Schauspielerin, Sängerin}|pwv}\pwindex{Steinrück, Elisabeth 19.\,11.\,1885 – 7.\,4.\,1920 Partenkirchen@\textsc{Steinrück, Elisabeth} (19.\,11.\,1885 – 7.\,4.\,1920 Partenkirchen)|pwv}!\pend
           \pstart Dein \spacefill\mbox{Paul Goldmann}\pend{}\selectlanguage{ngerman}\endnumbering\briefempfaengerindex{Schnitzler, Arthur@\textsc{Schnitzler, Arthur}!zzzGoldmann, Paul@\emph{von Paul Goldmann}!1901-08-012@{1. 8. [1901]}|)be}\mylabel{L03076h}  \newcommand{\dateiname}{L03076}\newcommand{\titel}{Paul Goldmann an Arthur Schnitzler, 1. 8. [1901]}\newcommand{\editorInnen}{Martin Anton Müller und Laura Untner}%% latex-leseansicht-abspann.tex
%% Abspann für die Leseansicht.
%% Der Schalter \ifkorrekturansicht ist bereits durch den Vorspann gesetzt.

%% latex-abspann.tex
%% Gemeinsamer Abspann für Korrekturansicht und Leseansicht.
%% Setzt den Schalter \ifkorrekturansicht voraus (gesetzt in den
%% einbindenden Dateien latex-korrekturansicht-abspann.tex bzw.
%% latex-leseansicht-abspann.tex).
%% ---------------------------------------------------------------

\normalsize

% Das esempio-Environment wird nur in der Leseansicht benötigt
\ifkorrekturansicht\else
\newenvironment{esempio}[3]%
{
    \vspace{1.5ex}
    \rlap{\underline{#1}}
    \par
    \setlength{\parindent}{0cm}
    \nopagebreak
    \leftskip=#2cm
    \rightskip=#3cm
}
{
    \par
}
\fi

\doendnotes{C}
\bigskip
\vfill

\clearpage

\footnotesize

\ifkorrekturansicht
  \lohead{\textsc{register}}
\fi

% theindex-Environment neu definieren ohne reledmac
\makeatletter
\renewenvironment{theindex}{%
  \ifkorrekturansicht
    \section*{\indexname}%
  \else
    \subsubsection*{Index der erwähnten Entitäten}%
  \fi
  \setlength{\parindent}{0pt}%
  \setlength{\parskip}{0pt plus 0.3pt}%
  \let\item\@idxitem
}{%
  \ifkorrekturansicht\clearpage\fi
}
\makeatother

\IfFileExists{\jobname-pw.ind}{\input{\jobname-pw.ind}}{}

% Quellenangabe nur in der Leseansicht
\ifkorrekturansicht\else
% Fallback-Definitionen, falls die .tex-Datei \titel etc. nicht gesetzt hat
\providecommand{\titel}{}
\providecommand{\editorInnen}{}
\providecommand{\dateiname}{\jobname}

\vspace{3cm}

\vfill

\footnotesize
\textsc{Quelle}: \titel. Herausgegeben von {\editorInnen}. In: \emph{Arthur Schnitzler: Briefwechsel mit Autorinnen und Autoren}.
 Digitale Edition, https://schnitzler-briefe.acdh.oeaw.ac.at/{\dateiname}.html (Stand \today)
\fi

\end{document}


