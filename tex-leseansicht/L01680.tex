%% latex-korrekturansicht-vorspann.tex
%% Vorspann für die Korrekturansicht.
%% Lädt die gemeinsame Datei latex-vorspann.tex mit gesetztem Schalter.

\newif\ifkorrekturansicht
\korrekturansichttrue

\input{../tex-inputs/latex-vorspann}


\section[Gerty von Hofmannsthal an Arthur Schnitzler, {[}29. 5. 1907{]}]{L01680 Gerty von Hofmannsthal an Arthur Schnitzler, {[}29. 5. 1907{]}}
\nopagebreak\mylabel{L01680v}
\rehead{ }\normalsize\beginnumbering\briefempfaengerindex{Schnitzler, Arthur@\textsc{Schnitzler, Arthur}!zzzHofmannsthal, Gertrude von@\emph{von Gertrude von Hofmannsthal}!1907-05-291@{{[}29. 5. 1907{]}}|(be}
\toendnotes[C]{\smallbreak\pagebreak[2]}\Standort{CUL, Schnitzler, B 43.}
\physDesc{Brief, 1 Blatt, 2 Seiten, 607 Zeichen
\newline{}Handschrift: schwarze Tinte, lateinische Kurrent
\newline{}Schnitzler: mit Bleistift datiert: »29/5 907« 
\newline{}Ordnung: 1) mit Bleistift von unbekannter Hand nummeriert: »\strikeout{276}«  2) mit Bleistift von unbekannter Hand nummeriert:
                                    »278«}
\buchAbdrucke{\weitereDrucke{Hugo von Hofmannsthal, Arthur Schnitzler: \emph{Briefwechsel}. Frankfurt am Main: \emph{S. Fischer} 1964, S. 375–376.} }
\pstart
           \noindent{}{\pb}Lieber Arthur,{ }Hugo\pwindex{Hofmannsthal, Hugo von 1874-02-01 – 1929-07-15@\textsc{Hofmannsthal, Hugo von} (1874-02-01 – 1929-07-15), \emph{Schriftsteller/Schriftstellerin}|pw} schreibt mir eben, dass er bis 3ten
                  Juni in Perugia, Hotel Brufani\oindex{Hotel Brufani@\textbf{Hotel Brufani}, \emph{Hotel (K.HTL)}|pw} ist.
               Gestern war er in Ravenna\oindex{Ravenna@\textbf{Ravenna}, \emph{P.PPLA2}|pw} und ist von dort mit
               der Eisenbahn die Küste entlang bis Rimini\oindex{Rimini@\textbf{Rimini}, \emph{P.PPLA2}|pw}
               gefahren, dann nach Ancona\oindex{Ancona@\textbf{Ancona}, \emph{P.PPLA}|pw}. Heute sind sie nach
                  Gubbio\oindex{Gubbio@\textbf{Gubbio}, \emph{P.PPLA3}|pw} und von dort fahren sie\pwindex{Hofmannsthal, Hugo August von 21.12.1841 – 08.12.1915@\textsc{Hofmannsthal, Hugo August von} (21.12.1841 – 08.12.1915), \emph{Bankdirektor/Bankdirektorin}|pw}{ }\strikeout{wieder} nach Perugia\oindex{Perugia@\textbf{Perugia}, \emph{P.PPLA}|pw}. Ich höre, dass es {\pb}der Gräfin Thun\pwindex{Thun-Hohenstein-Salm-Reifferscheidt, Christiane von 12.06.1859 – 06.08.1935@\textsc{Thun-Hohenstein-Salm-Reifferscheidt, Christiane von} (12.06.1859 – 06.08.1935), \emph{Schriftsteller/Schriftstellerin}|pw} weiter gut geht, und ich hoffe, dass jetzt die grosse
               Gefahr schon vorüber ist glauben Sie nicht?\pend
           
\pstart
           Ich komme natürlich furchtbar gern hinüber, nehme auch auf jeden Fall meine
               Tennissachen mit. Welche Stunden sind Ihnen am liebsten?\pend
           
\pstart
           Auf jeden Fall frage ich mich teleph. an.\pend
           \pstart Herzliche Grüsse Ihnen und Olga\pwindex{Schnitzler, Olga 17.01.1882 – 13.01.1970@\textsc{Schnitzler, Olga} (17.01.1882 – 13.01.1970), \emph{Schauspieler/Schauspielerin, Sänger/Sängerin}|pw}\hspace*{1.5em}Ihre \spacefill\mbox{Gerty}\pend{}\selectlanguage{ngerman}\endnumbering\briefempfaengerindex{Schnitzler, Arthur@\textsc{Schnitzler, Arthur}!zzzHofmannsthal, Gertrude von@\emph{von Gertrude von Hofmannsthal}!1907-05-291@{{[}29. 5. 1907{]}}|)be}\mylabel{L01680h}  \normalsize

\doendnotes{C}
\bigskip
\vfill

\clearpage

\footnotesize

\lohead{\textsc{register}}

% Definiere theindex-Environment komplett neu ohne reledmac
\makeatletter
\renewenvironment{theindex}{%
  \section*{\indexname}%
  \setlength{\parindent}{0pt}%
  \setlength{\parskip}{0pt plus 0.3pt}%
  \let\item\@idxitem
}{%
  \clearpage
}
\makeatother

\IfFileExists{\jobname-pw.ind}{\input{\jobname-pw.ind}}{}

\end{document}

      