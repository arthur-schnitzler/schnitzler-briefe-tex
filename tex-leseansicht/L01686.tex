%% latex-leseansicht-vorspann.tex
%% Vorspann für die Leseansicht.
%% Lädt die gemeinsame Datei latex-vorspann.tex mit nicht gesetztem Schalter.

\newif\ifkorrekturansicht
\korrekturansichtfalse

\input{../tex-inputs/latex-vorspann}


\section[Arthur Schnitzler an Hugo von Hofmannsthal, 25. 6. 1907]{L01686 Arthur Schnitzler an Hugo von Hofmannsthal, 25. 6. 1907}
\nopagebreak\mylabel{L01686v}
\rehead{ }\normalsize\beginnumbering\briefempfaengerindex{Hofmannsthal, Hugo von@\textsc{Hofmannsthal, Hugo von}!zzzSchnitzler, Arthur@\emph{von Arthur Schnitzler}!1907-06-251@{25. 6. 1907}|(be}
\toendnotes[C]{\smallbreak\pagebreak[2]}
\correspDesc{Versand  durch Arthur Schnitzler am 25. 6. 1907 in Wien
\newline{}Erhalt  durch Hugo von Hofmannsthal im Zeitraum [26. 6. 1907
                  – 30. 6. 1907?] in Lido}\toendnotes[C]{\smallbreak}
\Standort{FDH, Hs-30885,128.}
\physDesc{Brief, 1 Blatt, 3 Seiten, 1194 Zeichen
\newline{}Handschrift: schwarze Tinte, deutsche Kurrent}
\buchAbdrucke{\weitereDrucke{Hugo von Hofmannsthal, Arthur Schnitzler: \emph{Briefwechsel}. Herausgegeben von Therese Nickl und Heinrich Schnitzler. Frankfurt am Main: \emph{S. Fischer} 1964, S. 229–230.} }\toendnotes[C]{\smallbreak}
\pstart
           \raggedleft{}{\pb}Wien\oindex{Wien@\textbf{Wien}, \emph{Verwaltungsgebiet}|pw}{ }25. 6. 907\pend
           
\pstart{}Mein lieber Hugo,\pend\vspace{0.5em}
\pstart
           morgen fahren wir nach Villach\oindex{Villach@\textbf{Villach}, \emph{Verwaltungsgebiet}|pw}; – von dort aus
               wollen wir uns umſehen, ob wir irgd was (\textsc{Veldes\oindex{Bled@\textbf{Bled}|pw}? Wochein\oindex{Die Wochein@\textbf{Die Wochein}, \emph{Hotel}|pw}}? oder{ }ſonſt wo) – we{\geminationn}s gut geht, zu längerem
               Aufenthalt finden. Den Buben\pwindex{Schnitzler, Heinrich 9.\,8.\,1902 Hinterbrühl – 12.\,7.\,1982 Wien@\textsc{Schnitzler, Heinrich} (9.\,8.\,1902 Hinterbrühl – 12.\,7.\,1982 Wien), \emph{Regisseur, Schauspieler}|pwv}
               laſſen wir erſt nachko{\geminationm}en we{\geminationn} wir wiſſen, wo unſres Bleibens. Der Roman\pwindex{Schnitzler, Arthur 15.\,5.\,1862 Wien – 21.\,10.\,1931 ebd.@\textsc{Schnitzler, Arthur} (15.\,5.\,1862 Wien – 21.\,10.\,1931 ebd.), \emph{Schriftsteller, Mediziner}!Weg ins Freie. Roman@\strich\emph{Der Weg ins Freie. Roman}|pwv}, den ich nun tüchtig durchfeile, zum großen Theil
               natürlich neu{ }ſchreibe, zieht mit. Das Winterſtück\pwindex{Schnitzler, Arthur 15.\,5.\,1862 Wien – 21.\,10.\,1931 ebd.@\textsc{Schnitzler, Arthur} (15.\,5.\,1862 Wien – 21.\,10.\,1931 ebd.), \emph{Schriftsteller, Mediziner}!Wort. Tragikomödie in fünf Akten@\strich\emph{Das Wort. Tragikomödie in fünf Akten}|pwv}{ }{\pb}hab ich weggeschmiſſen; nicht weggelegt, da ich in ein{ }ſchlechtes Verhältnis dazu gerieth. Irgend ein Wurzelfehler war da,{ }ſo daſs ich durch
               corrigiren nicht weiter kam. Vielleicht muſs der Stoff in andre Erde geſetzt werden,
               doch weiſs ich noch nicht in welche. Vorläufig gehn mir andre theatralische Einfälle
               näher. – Wir haben in der letzten Zeit viele Leute geſehen; es gab manche{ }ſehr gute
               Stunden, mit Richard\pwindex{Beer-Hofmann, Richard 11.\,7.\,1866 Wien – 26.\,9.\,1945 New York City@\textsc{Beer-Hofmann, Richard} (11.\,7.\,1866 Wien – 26.\,9.\,1945 New York City), \emph{Schriftsteller}|pw}, \textsc{Wasserman\textcolor{gray}{n}}\pwindex{Wassermann, Jakob 10.\,3.\,1873 Fürth – 1.\,1.\,1934 Altaussee@\textsc{Wassermann, Jakob} (10.\,3.\,1873 Fürth – 1.\,1.\,1934 Altaussee), \emph{Schriftsteller}|pw}, Kainz\pwindex{Kainz, Josef 2.\,1.\,1858 Mosonmagyaróvár – 20.\,9.\,1910 Wien@\textsc{Kainz, Josef} (2.\,1.\,1858 Mosonmagyaróvár – 20.\,9.\,1910 Wien), \emph{Schauspieler}|pw}, \introOben{}\textsc{Fred}\pwindex{W. Fred 29.\,6.\,1879 Wien – 23.\,10.\,1922 Berlin@\textsc{W. Fred} (29.\,6.\,1879 Wien – 23.\,10.\,1922 Berlin), \emph{Schriftsteller, Journalist}|pw}, und and\textcolor{gray}{re}\introOben{}; auch das \textsc{Tennis} war{ }ſchön – nur lockt es mich {\pb}doch ins einſamere. Der Gräfin Thun\pwindex{Thun-Hohenstein-Salm-Reifferscheidt, Christiane von 12.\,6.\,1859 Doksy – 6.\,8.\,1935 Prag@\textsc{Thun-Hohenstein-Salm-Reifferscheidt, Christiane von} (12.\,6.\,1859 Doksy – 6.\,8.\,1935 Prag), \emph{Schriftstellerin}|pw} hab ich die Dä{\geminationm}erſeelen\pwindex{Schnitzler, Arthur 15.\,5.\,1862 Wien – 21.\,10.\,1931 ebd.@\textsc{Schnitzler, Arthur} (15.\,5.\,1862 Wien – 21.\,10.\,1931 ebd.), \emph{Schriftsteller, Mediziner}!Dämmerseelen. Novellen@\strich\emph{Dämmerseelen. Novellen}|pw} geſchickt;{ }ſie hat in einem{ }ſehr
               liebenswürdg Telegra{\geminationm} gedankt. Wie lange bleiben Sie
               noch am Lido\oindex{Lido@\textbf{Lido}|pw}? Von endgiltigem Zeltaufſchlag
               verſtändige ich Sie gleich. Ich hoffe Sie leſen im September was
               wundervolles vor.\pend
           
\pstart
           Seien Sie, un\textcolor{gray}{d}{ }Gerty\pwindex{Hofmannsthal, Gertrude von 16.\,3.\,1880 Wien – 9.\,11.\,1959 Paddington@\textsc{Hofmannsthal, Gertrude von} (16.\,3.\,1880 Wien – 9.\,11.\,1959 Paddington)|pw} herzlichſt gegrüßt, von \textcolor{gray}{O}lga\pwindex{Schnitzler, Olga 17.\,1.\,1882 Wien – 13.\,1.\,1970 Lugano@\textsc{Schnitzler, Olga} (17.\,1.\,1882 Wien – 13.\,1.\,1970 Lugano), \emph{Schauspielerin, Sängerin}|pw} u mir.\pend
           
\pstart
           Ihr{\\[\baselineskip]}\spacefill\mbox{Arthur}\pend
           \leftskip=0em{}\selectlanguage{ngerman}\endnumbering\briefempfaengerindex{Hofmannsthal, Hugo von@\textsc{Hofmannsthal, Hugo von}!zzzSchnitzler, Arthur@\emph{von Arthur Schnitzler}!1907-06-251@{25. 6. 1907}|)be}\mylabel{L01686h}  \newcommand{\dateiname}{L01686}\newcommand{\titel}{Arthur Schnitzler an Hugo von Hofmannsthal, 25. 6. 1907}\newcommand{\editorInnen}{Martin Anton Müller und Gerd-Hermann Susen}%% latex-leseansicht-abspann.tex
%% Abspann für die Leseansicht.
%% Der Schalter \ifkorrekturansicht ist bereits durch den Vorspann gesetzt.

%% latex-abspann.tex
%% Gemeinsamer Abspann für Korrekturansicht und Leseansicht.
%% Setzt den Schalter \ifkorrekturansicht voraus (gesetzt in den
%% einbindenden Dateien latex-korrekturansicht-abspann.tex bzw.
%% latex-leseansicht-abspann.tex).
%% ---------------------------------------------------------------

\normalsize

% Das esempio-Environment wird nur in der Leseansicht benötigt
\ifkorrekturansicht\else
\newenvironment{esempio}[3]%
{
    \vspace{1.5ex}
    \rlap{\underline{#1}}
    \par
    \setlength{\parindent}{0cm}
    \nopagebreak
    \leftskip=#2cm
    \rightskip=#3cm
}
{
    \par
}
\fi

\doendnotes{C}
\bigskip
\vfill

\clearpage

\footnotesize

\ifkorrekturansicht
  \lohead{\textsc{register}}
\fi

% theindex-Environment neu definieren ohne reledmac
\makeatletter
\renewenvironment{theindex}{%
  \ifkorrekturansicht
    \section*{\indexname}%
  \else
    \subsubsection*{Index der erwähnten Entitäten}%
  \fi
  \setlength{\parindent}{0pt}%
  \setlength{\parskip}{0pt plus 0.3pt}%
  \let\item\@idxitem
}{%
  \ifkorrekturansicht\clearpage\fi
}
\makeatother

\IfFileExists{\jobname-pw.ind}{\input{\jobname-pw.ind}}{}

% Quellenangabe nur in der Leseansicht
\ifkorrekturansicht\else
% Fallback-Definitionen, falls die .tex-Datei \titel etc. nicht gesetzt hat
\providecommand{\titel}{}
\providecommand{\editorInnen}{}
\providecommand{\dateiname}{\jobname}

\vspace{3cm}

\vfill

\footnotesize
\textsc{Quelle}: \titel. Herausgegeben von {\editorInnen}. In: \emph{Arthur Schnitzler: Briefwechsel mit Autorinnen und Autoren}.
 Digitale Edition, https://schnitzler-briefe.acdh.oeaw.ac.at/{\dateiname}.html (Stand \today)
\fi

\end{document}


