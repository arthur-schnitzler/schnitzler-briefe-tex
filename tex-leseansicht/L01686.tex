\input{../tex-inputs/latex-pdf-vorspann}
\begin{center}
            \textcolor{red}{ENTWURF. ENTZIFFERUNG NOCH NICHT KORREKTURGELESEN}
                      \end{center}
            
               \section[Arthur Schnitzler an Hugo von Hofmannsthal, 25. 6. 1907]{ Arthur Schnitzler an Hugo von Hofmannsthal, 25. 6. 1907}\nopagebreak\mylabel{v}\rehead{ }\begin{ledgroupsized}[t]{13cm}\normalsize\beginnumbering\briefempfaengerindex{Hofmannsthal, Hugo von@\textsc{Hofmannsthal, Hugo von}!zzzSchnitzler, Arthur@\emph{von Arthur Schnitzler}!1907-06-251@{25. 6. 1907}|(be} \toendnotes[C]{\smallbreak\pagebreak[2]} \Standort{FDH, Hs-30885,128.}
\physDesc{Brief, 1 Blatt, 3 Seiten
\newline{}Handschrift: schwarze Tinte, deutsche Kurrent}\buchAbdrucke{\weitereDrucke{Hugo von Hofmannsthal, Arthur Schnitzler: \emph{Briefwechsel}. Hg. Therese Nickl und Heinrich Schnitzler. Frankfurt am Main: \emph{S. Fischer} 1964, S. 229–230.} }\toendnotes[C]{\smallbreak}\pstart
           \raggedleft{}{\pb}Wien\oindex{Wien@\textbf{Wien}|pw}{ }25. 6. 907\pend
           \pstart{}Mein lieber Hugo, \pend\pstart
           morgen fahren wir nach Villach\oindex{Villach@\textbf{Villach}|pw}; – von dort aus
               wollen wir uns umſehen, ob wir irgd was (\textsc{Veldes\oindex{Veldes@\textbf{Veldes}|pw}? Wochein\oindex{Die Wochein@\textbf{Die Wochein}|pw}}? oder
               ſonſt wo) – we{\geminationn}s gut geht, zu längerem Aufenthalt finden. Den Buben\pwindex{Schnitzler, Heinrich 09.08.1902 – 12.07.1982@\textsc{Schnitzler, Heinrich} (09.08.1902 – 12.07.1982), \emph{Regisseur, Schauspieler}|pwv} laſſen wir erſt nachko{\geminationm}en we{\geminationn} wir wiſſen, wo
               unſres Bleibens. Der Roman\pwindex{Schnitzler, Arthur 15.05.1862 – 21.10.1931@\textsc{Schnitzler, Arthur} (15.05.1862 – 21.10.1931), \emph{Schriftsteller, Mediziner}!Weg ins Freie. Roman1.1.1908 – 1.6.1908@\strich\emph{Der Weg ins Freie. Roman} {[}1.1.1908 – 1.6.1908{]}|pwv}, den
               ich nun tüchtig durchfeile, zum großen Theil natürlich neu ſchreibe, zieht mit. Das
                  Winterſtück\pwindex{Schnitzler, Arthur 15.05.1862 – 21.10.1931@\textsc{Schnitzler, Arthur} (15.05.1862 – 21.10.1931), \emph{Schriftsteller, Mediziner}!Wort. Tragikomoedie in fuenf Akten1966@\strich\emph{Das Wort. Tragikomödie in fünf Akten} {[}1966{]}|pwv}{ }{\pb}hab ich weggeschmiſſen; nicht
               weggelegt, da ich in ein ſchlechtes Verhältnis dazu gerieth. Irgend ein Wurzelfehler
               war da, ſo daſs ich durch corrigiren nicht weiter kam. Vielleicht muſs der Stoff in
               andre Erde geſetzt werden, doch weiſs ich noch nicht in welche. Vorläufig gehn mir
               andre theatralische Einfälle näher. – Wir haben in der letzten Zeit viele Leute
               geſehen; es gab manche ſehr gute Stunden, mit Richard\pwindex{Beer-Hofmann, Richard 11.07.1866 – 26.09.1945@\textsc{Beer-Hofmann, Richard} (11.07.1866 – 26.09.1945), \emph{Schriftsteller}|pw}, \textsc{Wasserman\textcolor{gray}{n}}\pwindex{Wassermann, Jakob 10.03.1873 – 01.01.1934@\textsc{Wassermann, Jakob} (10.03.1873 – 01.01.1934), \emph{Schriftsteller}|pw}, Kainz\pwindex{Kainz, Josef 02.01.1858 – 20.09.1910@\textsc{Kainz, Josef} (02.01.1858 – 20.09.1910), \emph{Schauspieler}|pw}, \introOben{}\textsc{Fred}\pwindex{W. Fred 29.06.1879 – 23.10.1922@\textsc{W. Fred} (29.06.1879 – 23.10.1922), \emph{Schriftsteller, Journalist}|pw},
                  und and\textcolor{gray}{re}\introOben{}; auch das \textsc{Tennis} war ſchön – nur lockt
               es mich {\pb}doch ins einſamere. Der Gräfin Thun\pwindex{Thun-Hohenstein-Salm-Reifferscheidt, Christiane von 12.06.1859 – 06.08.1935@\textsc{Thun-Hohenstein-Salm-Reifferscheidt, Christiane von} (12.06.1859 – 06.08.1935), \emph{Schriftstellerin}|pw} hab ich die Dä{\geminationm}erſeelen\pwindex{Schnitzler, Arthur 15.05.1862 – 21.10.1931@\textsc{Schnitzler, Arthur} (15.05.1862 – 21.10.1931), \emph{Schriftsteller, Mediziner}!Daemmerseelen. Novellen1907@\strich\emph{Dämmerseelen. Novellen} {[}1907{]}|pw} geſchickt; ſie hat in einem ſehr
               liebenswürdg Telegra{\geminationm} gedankt. Wie lange bleiben Sie
               noch am Lido\oindex{Lido@\textbf{Lido}|pw}? Von endgiltigem Zeltaufſchlag
               verſtändige ich Sie gleich. Ich hoffe Sie leſen im September was
               wundervolles vor.\pend
           \pstart
           Seien Sie, un\textcolor{gray}{d}{ }Gerty\pwindex{Hofmannsthal, Gertrude von 16.03.1880 – 09.11.1959@\textsc{Hofmannsthal, Gertrude von} (16.03.1880 – 09.11.1959)|pw} herzlichſt gegrüßt, von \textcolor{gray}{O}lga\pwindex{Schnitzler, Olga 17.01.1882 – 13.01.1970@\textsc{Schnitzler, Olga} (17.01.1882 – 13.01.1970), \emph{Schauspielerin, Sängerin}|pw} u mir.\pend
           \pstart
           Ihr{\\[\baselineskip]}\spacefill\mbox{Arthur}\pend
           \leftskip=0em{}\endnumbering\briefempfaengerindex{Hofmannsthal, Hugo von@\textsc{Hofmannsthal, Hugo von}!zzzSchnitzler, Arthur@\emph{von Arthur Schnitzler}!1907-06-251@{25. 6. 1907}|)be}\mylabel{h}\end{ledgroupsized}  \newcommand{\dateiname}{L01686}\newcommand{\titel}{Arthur Schnitzler an Hugo von Hofmannsthal, 25. 6. 1907}\newcommand{\editorInnen}{Martin Anton Müller und Gerd-Hermann Susen}\input{../tex-inputs/latex-pdf-abspann}
      