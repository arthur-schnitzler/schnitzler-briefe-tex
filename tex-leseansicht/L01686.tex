%% latex-korrekturansicht-vorspann.tex
%% Vorspann für die Korrekturansicht.
%% Lädt die gemeinsame Datei latex-vorspann.tex mit gesetztem Schalter.

\newif\ifkorrekturansicht
\korrekturansichttrue

\input{../tex-inputs/latex-vorspann}


\section[Arthur Schnitzler an Hugo von Hofmannsthal, 25. 6. 1907]{L01686 Arthur Schnitzler an Hugo von Hofmannsthal, 25. 6. 1907}
\nopagebreak\mylabel{L01686v}
\rehead{ }\normalsize\beginnumbering\briefempfaengerindex{Hofmannsthal, Hugo von@\textsc{Hofmannsthal, Hugo von}!zzzSchnitzler, Arthur@\emph{von Arthur Schnitzler}!1907-06-251@{25. 6. 1907}|(be}
\toendnotes[C]{\smallbreak\pagebreak[2]}\Standort{FDH, Hs-30885,128.}
\physDesc{Brief, 1 Blatt, 3 Seiten, 1194 Zeichen
\newline{}Handschrift: schwarze Tinte, deutsche Kurrent}
\buchAbdrucke{\weitereDrucke{Hugo von Hofmannsthal, Arthur Schnitzler: \emph{Briefwechsel}. Frankfurt am Main: \emph{S. Fischer} 1964, S. 229–230.} }\toendnotes[C]{\smallbreak}
\pstart
           \raggedleft{}{\pb}Wien\oindex{Wien@\textbf{Wien}, \emph{A.ADM2}|pw}{ }25. 6. 907\pend
           
\pstart{}Mein lieber Hugo, \pend\vspace{0.5em}
\pstart
           morgen fahren wir nach Villach\oindex{Villach@\textbf{Villach}, \emph{A.ADM3}|pw}; – von dort aus
               wollen wir uns umſehen, ob wir irgd was (\textsc{Veldes\oindex{Bled@\textbf{Bled}, \emph{P.PPLA}|pw}? Wochein\oindex{Die Wochein@\textbf{Die Wochein}, \emph{Hotel (K.HTL)}|pw}}? oder ſonſt wo) – we{\geminationn}s gut geht, zu längerem
               Aufenthalt finden. Den Buben\pwindex{Schnitzler, Heinrich 09.08.1902 – 12.07.1982@\textsc{Schnitzler, Heinrich} (09.08.1902 – 12.07.1982), \emph{Regisseur/Regisseurin, Schauspieler/Schauspielerin}|pwv}
               laſſen wir erſt nachko{\geminationm}en we{\geminationn} wir wiſſen, wo unſres Bleibens. Der Roman\pwindex{Weg ins Freie. Roman@\emph{Der Weg ins Freie. Roman}|pwv}, den ich nun tüchtig durchfeile, zum großen Theil
               natürlich neu ſchreibe, zieht mit. Das Winterſtück\pwindex{Wort. Tragikomoedie in fuenf Akten@\emph{Das Wort. Tragikomödie in fünf Akten}|pwv}{ }{\pb}hab ich weggeschmiſſen; nicht weggelegt, da ich in ein
               ſchlechtes Verhältnis dazu gerieth. Irgend ein Wurzelfehler war da, ſo daſs ich durch
               corrigiren nicht weiter kam. Vielleicht muſs der Stoff in andre Erde geſetzt werden,
               doch weiſs ich noch nicht in welche. Vorläufig gehn mir andre theatralische Einfälle
               näher. – Wir haben in der letzten Zeit viele Leute geſehen; es gab manche ſehr gute
               Stunden, mit Richard\pwindex{Beer-Hofmann, Richard 1866-07-11 – 1945-09-26@\textsc{Beer-Hofmann, Richard} (1866-07-11 – 1945-09-26), \emph{Schriftsteller/Schriftstellerin}|pw}, \textsc{Wasserman\textcolor{gray}{n}}\pwindex{Wassermann, Jakob 10.03.1873 – 01.01.1934@\textsc{Wassermann, Jakob} (10.03.1873 – 01.01.1934), \emph{Schriftsteller/Schriftstellerin}|pw}, Kainz\pwindex{Kainz, Josef 02.01.1858 – 20.09.1910@\textsc{Kainz, Josef} (02.01.1858 – 20.09.1910), \emph{Schauspieler/Schauspielerin}|pw}, \introOben{}\textsc{Fred}\pwindex{W. Fred 29.06.1879 – 23.10.1922@\textsc{W. Fred} (29.06.1879 – 23.10.1922), \emph{Schriftsteller/Schriftstellerin, Journalist/Journalistin}|pw}, und and\textcolor{gray}{re}\introOben{}; auch das \textsc{Tennis} war ſchön – nur lockt es mich {\pb}doch ins einſamere. Der Gräfin Thun\pwindex{Thun-Hohenstein-Salm-Reifferscheidt, Christiane von 12.06.1859 – 06.08.1935@\textsc{Thun-Hohenstein-Salm-Reifferscheidt, Christiane von} (12.06.1859 – 06.08.1935), \emph{Schriftsteller/Schriftstellerin}|pw} hab ich die Dä{\geminationm}erſeelen\pwindex{Daemmerseelen. Novellen@\emph{Dämmerseelen. Novellen}|pw} geſchickt; ſie hat in einem ſehr
               liebenswürdg Telegra{\geminationm} gedankt. Wie lange bleiben Sie
               noch am Lido\oindex{Lido@\textbf{Lido}, \emph{P.PPL}|pw}? Von endgiltigem Zeltaufſchlag
               verſtändige ich Sie gleich. Ich hoffe Sie leſen im September was
               wundervolles vor.\pend
           
\pstart
           Seien Sie, un\textcolor{gray}{d}{ }Gerty\pwindex{Hofmannsthal, Gertrude von 16.03.1880 – 09.11.1959@\textsc{Hofmannsthal, Gertrude von} (16.03.1880 – 09.11.1959)|pw} herzlichſt gegrüßt, von \textcolor{gray}{O}lga\pwindex{Schnitzler, Olga 17.01.1882 – 13.01.1970@\textsc{Schnitzler, Olga} (17.01.1882 – 13.01.1970), \emph{Schauspieler/Schauspielerin, Sänger/Sängerin}|pw} u mir.\pend
           
\pstart
           Ihr{\\[\baselineskip]}\spacefill\mbox{Arthur}\pend
           \leftskip=0em{}\selectlanguage{ngerman}\endnumbering\briefempfaengerindex{Hofmannsthal, Hugo von@\textsc{Hofmannsthal, Hugo von}!zzzSchnitzler, Arthur@\emph{von Arthur Schnitzler}!1907-06-251@{25. 6. 1907}|)be}\mylabel{L01686h}  \normalsize

\doendnotes{C}
\bigskip
\vfill

\clearpage

\footnotesize

\lohead{\textsc{register}}

% Definiere theindex-Environment komplett neu ohne reledmac
\makeatletter
\renewenvironment{theindex}{%
  \section*{\indexname}%
  \setlength{\parindent}{0pt}%
  \setlength{\parskip}{0pt plus 0.3pt}%
  \let\item\@idxitem
}{%
  \clearpage
}
\makeatother

\IfFileExists{\jobname-pw.ind}{\input{\jobname-pw.ind}}{}

\end{document}

      