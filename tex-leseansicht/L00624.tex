%% latex-leseansicht-vorspann.tex
%% Vorspann für die Leseansicht.
%% Lädt die gemeinsame Datei latex-vorspann.tex mit nicht gesetztem Schalter.

\newif\ifkorrekturansicht
\korrekturansichtfalse

\input{../tex-inputs/latex-vorspann}


\section[Hugo von Hofmannsthal an Arthur Schnitzler, 23. 11. 1896]{L00624 Hugo von Hofmannsthal an Arthur Schnitzler, 23. 11. 1896}
\nopagebreak\mylabel{L00624v}
\rehead{ }\normalsize\beginnumbering\briefempfaengerindex{Schnitzler, Arthur@\textsc{Schnitzler, Arthur}!zzzHofmannsthal, Hugo von@\emph{von Hugo von Hofmannsthal}!1896-11-232@{23. 11. 1896}|(be}
\toendnotes[C]{\smallbreak\pagebreak[2]}
\correspDesc{Versand  durch Hugo von Hofmannsthal am 23. 11. 1896 in Wien
\newline{}Erhalt  durch Arthur Schnitzler am 24. 11. 1896 in Wien}\toendnotes[C]{\smallbreak}
\Standort{CUL, Schnitzler, B 43.}
\physDesc{Postkarte, 293 Zeichen
\newline{}Handschrift: Bleistift, deutsche Kurrent
\newline{}Versand: 1) Stempel: »\nobreak{}\oindex{I., Innere Stadt@\textbf{I., Innere Stadt}, \emph{Verwaltungsgebiet}|pwk}Wien 1/1, 23. 11. {[}96{]}, 11–12.N\nobreak{}«.   2) Stempel: »\nobreak{}\oindex{IX., Alsergrund@\textbf{IX., Alsergrund}, \emph{Verwaltungsgebiet}|pwk}Wien 9/3, 24. 11. 1896, 8.V, Bestellt\nobreak{}«. 
\newline{}Schnitzler: mit Bleistift datiert: »24/11 96« 
\newline{}Ordnung: mit Bleistift von unbekannter Hand nummeriert:
                                    »83« }
\buchAbdrucke{\weitereDrucke{Hugo von Hofmannsthal, Arthur Schnitzler: \emph{Briefwechsel}. Herausgegeben von Therese Nickl und Heinrich Schnitzler. Frankfurt am Main: \emph{S. Fischer} 1964, S. 76–77.} }\pstart{}{\pb}\textsc{Herrn D\textsuperscript{r} Arthur Schnitzler}\pend{}\pstart{}\textsc{IX Franckgasse} 1\oindex{Wien@\textbf{Wien}!IX., Alsergrund@\textbf{IX., Alsergrund}!Frankgasse 1@\textbf{Frankgasse 1}, \emph{Wohngebäude}|pw}\pend{}{\bigskip}\vspace{1em}
\pstart{}{\pb}lieber Arthur\pend\vspace{0.5em}
\pstart
           mir fällt ein daſs glaub ich der \textsc{Poldy}\pwindex{Andrian-Werburg, Leopold von 9.\,5.\,1875 Berlin – 19.\,11.\,1951 Fribourg@\textsc{Andrian-Werburg, Leopold von} (9.\,5.\,1875 Berlin – 19.\,11.\,1951 Fribourg), \emph{Schriftsteller, Diplomat}|pw} morgen abend zu Ihnen kommen will. Falls es also bei Richard\pwindex{Beer-Hofmann, Richard 11.\,7.\,1866 Wien – 26.\,9.\,1945 New York City@\textsc{Beer-Hofmann, Richard} (11.\,7.\,1866 Wien – 26.\,9.\,1945 New York City), \emph{Schriftsteller}|pw} iſt, bitte{ }ſchreiben Sie ihm eine Zeile (\textsc{Habsburgergasse 5}\oindex{Wien@\textbf{Wien}!I., Innere Stadt@\textbf{I., Innere Stadt}!Habsburgergasse@\textbf{Habsburgergasse}, \emph{Straße}|pw}) ich weiß es{ }ſelbſt nicht weiß auch nicht ob ich kommen oder arbeiten
               werde.\pend
           
\pstart
           Ihr{\\[\baselineskip]}\spacefill\mbox{Hugo.}\pend
           \leftskip=0em{}\selectlanguage{ngerman}\endnumbering\briefempfaengerindex{Schnitzler, Arthur@\textsc{Schnitzler, Arthur}!zzzHofmannsthal, Hugo von@\emph{von Hugo von Hofmannsthal}!1896-11-232@{23. 11. 1896}|)be}\mylabel{L00624h}  \newcommand{\dateiname}{L00624}\newcommand{\titel}{Hugo von Hofmannsthal an Arthur Schnitzler, 23. 11. 1896}\newcommand{\editorInnen}{Martin Anton Müller und Gerd-Hermann Susen}%% latex-leseansicht-abspann.tex
%% Abspann für die Leseansicht.
%% Der Schalter \ifkorrekturansicht ist bereits durch den Vorspann gesetzt.

%% latex-abspann.tex
%% Gemeinsamer Abspann für Korrekturansicht und Leseansicht.
%% Setzt den Schalter \ifkorrekturansicht voraus (gesetzt in den
%% einbindenden Dateien latex-korrekturansicht-abspann.tex bzw.
%% latex-leseansicht-abspann.tex).
%% ---------------------------------------------------------------

\normalsize

% Das esempio-Environment wird nur in der Leseansicht benötigt
\ifkorrekturansicht\else
\newenvironment{esempio}[3]%
{
    \vspace{1.5ex}
    \rlap{\underline{#1}}
    \par
    \setlength{\parindent}{0cm}
    \nopagebreak
    \leftskip=#2cm
    \rightskip=#3cm
}
{
    \par
}
\fi

\doendnotes{C}
\bigskip
\vfill

\clearpage

\footnotesize

\ifkorrekturansicht
  \lohead{\textsc{register}}
\fi

% theindex-Environment neu definieren ohne reledmac
\makeatletter
\renewenvironment{theindex}{%
  \ifkorrekturansicht
    \section*{\indexname}%
  \else
    \subsubsection*{Index der erwähnten Entitäten}%
  \fi
  \setlength{\parindent}{0pt}%
  \setlength{\parskip}{0pt plus 0.3pt}%
  \let\item\@idxitem
}{%
  \ifkorrekturansicht\clearpage\fi
}
\makeatother

\IfFileExists{\jobname-pw.ind}{\input{\jobname-pw.ind}}{}

% Quellenangabe nur in der Leseansicht
\ifkorrekturansicht\else
% Fallback-Definitionen, falls die .tex-Datei \titel etc. nicht gesetzt hat
\providecommand{\titel}{}
\providecommand{\editorInnen}{}
\providecommand{\dateiname}{\jobname}

\vspace{3cm}

\vfill

\footnotesize
\textsc{Quelle}: \titel. Herausgegeben von {\editorInnen}. In: \emph{Arthur Schnitzler: Briefwechsel mit Autorinnen und Autoren}.
 Digitale Edition, https://schnitzler-briefe.acdh.oeaw.ac.at/{\dateiname}.html (Stand \today)
\fi

\end{document}


