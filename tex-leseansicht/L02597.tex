%% latex-leseansicht-vorspann.tex
%% Vorspann für die Leseansicht.
%% Lädt die gemeinsame Datei latex-vorspann.tex mit nicht gesetztem Schalter.

\newif\ifkorrekturansicht
\korrekturansichtfalse

\input{../tex-inputs/latex-vorspann}


\section[Arthur Schnitzler an Marie Herzfeld, 20.\,4.\,1909]{L02597 Arthur Schnitzler an Marie Herzfeld, 20.\,4.\,1909}
\nopagebreak\mylabel{L02597v}
\rehead{ }\normalsize\beginnumbering\briefempfaengerindex{Herzfeld, Marie@\textsc{Herzfeld, Marie}!zzzSchnitzler, Arthur@\emph{von Arthur Schnitzler}!1909-04-201@{20.\,4.\,1909}|(be}
\toendnotes[C]{\smallbreak\pagebreak[2]}
\correspDesc{Versand  durch Arthur Schnitzler am 20. 4. 1909 in Wien
\newline{}Erhalt  durch Marie Herzfeld im Zeitraum [20. 4. 1909
                  – 24. 4. 1909?] in Wien}\toendnotes[C]{\smallbreak}
\Standort{DLA, A:Schnitzler, HS.1985.1.993.}
\physDesc{Brief, Durchschlag, 1 Blatt, 1 Seite, 930 Zeichen
\newline{}Schreibmaschine
\newline{}Handschrift: 1) Bleistift, lateinische Kurrent (\noindent{}Vermerk »\textsc{Herzfeld}«)\hspace{1em}2) roter Buntstift (\noindent{}mit rotem Buntstift drei Unterstreichungen)\hspace{1em}}\toendnotes[C]{\smallbreak}
\pstart
           \raggedleft{}{\pb}20. April 09.\pend
           
\pstart{}Verehrtes Fräulein,\pend\vspace{0.5em}
\pstart
           Frau Tesi\pwindex{Rotenstern-Tesi, Anna *~11.\,1.\,1871 Odessa@\textsc{Rotenstern-Tesi, Anna} (*~11.\,1.\,1871 Odessa), \emph{Übersetzerin}|pw} wird von ihrem Gedächtnis getäuscht,
               wenn Sie Ihnen sagte, dass ich ihr von der Revolutionshochzeit\pwindex{Revolutionsbryllup. Skuespil i tre Akter@\emph{Revolutionsbryllup. Skuespil i tre Akter}|pw} gesprochen hätte. Ich habe von dem Stück\pwindex{Revolutionsbryllup. Skuespil i tre Akter@\emph{Revolutionsbryllup. Skuespil i tre Akter}|pwv} schon das beste gehört, habe es aber
               bisher weder gelesen noch gesehen. Dass Frau
                  Tesi\pwindex{Rotenstern-Tesi, Anna *~11.\,1.\,1871 Odessa@\textsc{Rotenstern-Tesi, Anna} (*~11.\,1.\,1871 Odessa), \emph{Übersetzerin}|pw} einiges von mir übersetzt hat stimmt. Meine direkten Verhandlungen
               fandem mit ihrem Gatten Herrn Rottenstern
                  Swestitsch\pwindex{Rotenstern, Peter 10.\,1.\,1868 Odessa – 1944@\textsc{Rotenstern, Peter} (10.\,1.\,1868 Odessa – 1944), \emph{Journalist, Übersetzer}|pw} statt. Beide\pwindex{Rotenstern-Tesi, Anna *~11.\,1.\,1871 Odessa@\textsc{Rotenstern-Tesi, Anna} (*~11.\,1.\,1871 Odessa), \emph{Übersetzerin}|pwv}\pwindex{Rotenstern, Peter 10.\,1.\,1868 Odessa – 1944@\textsc{Rotenstern, Peter} (10.\,1.\,1868 Odessa – 1944), \emph{Journalist, Übersetzer}|pwv} scheinen mir, soweit es die
               Konventionsverhältnisse zwischen Oesterreich\oindex{Österreich@\textbf{Österreich}|pw}
               und Russland\oindex{Russland@\textbf{Russland}|pw} zulassen, verlässliche Menschen.
                  \label{K_L02597-1v}\edtext{Ich habe von ihnen, sowohl für Zwischenspiel\pwindex{Schnitzler, Arthur 15.\,5.\,1862 Wien – 21.\,10.\,1931 ebd.@\textsc{Schnitzler, Arthur} (15.\,5.\,1862 Wien – 21.\,10.\,1931 ebd.), \emph{Schriftsteller, Mediziner}!Zwischenspiel. Komödie in drei Akten@\strich\emph{Zwischenspiel. Komödie in drei Akten}|pw} als für Ruf des Lebens\pwindex{Schnitzler, Arthur 15.\,5.\,1862 Wien – 21.\,10.\,1931 ebd.@\textsc{Schnitzler, Arthur} (15.\,5.\,1862 Wien – 21.\,10.\,1931 ebd.), \emph{Schriftsteller, Mediziner}!Ruf des Lebens. Schauspiel in drei Akten@\strich\emph{Der Ruf des Lebens. Schauspiel in drei Akten}|pw}, wenn ich mich recht erinnere auch für den einsamen Weg\pwindex{Schnitzler, Arthur 15.\,5.\,1862 Wien – 21.\,10.\,1931 ebd.@\textsc{Schnitzler, Arthur} (15.\,5.\,1862 Wien – 21.\,10.\,1931 ebd.), \emph{Schriftsteller, Mediziner}!einsame Weg. Schauspiel in fünf Akten@\strich\emph{Der einsame Weg. Schauspiel in fünf Akten}|pw}}{\lemma{\textnormal{\emph{Ich … Weg}}}\Cendnote{\textnormal{Die \emph{Übersetzung}\pwindex{Schnitzler, Arthur 15.\,5.\,1862 Wien – 21.\,10.\,1931 ebd.@\textsc{Schnitzler, Arthur} (15.\,5.\,1862 Wien – 21.\,10.\,1931 ebd.), \emph{Schriftsteller, Mediziner}!Zwischenspiel, russisch]@\strich\emph{[Zwischenspiel, russisch]}|pwk} des \emph{Zwischenspiels}\pwindex{Schnitzler, Arthur 15.\,5.\,1862 Wien – 21.\,10.\,1931 ebd.@\textsc{Schnitzler, Arthur} (15.\,5.\,1862 Wien – 21.\,10.\,1931 ebd.), \emph{Schriftsteller, Mediziner}!Zwischenspiel. Komödie in drei Akten@\strich\emph{Zwischenspiel. Komödie in drei Akten}|pwk}
                  erschien 1905, jene\pwindex{Schnitzler, Arthur 15.\,5.\,1862 Wien – 21.\,10.\,1931 ebd.@\textsc{Schnitzler, Arthur} (15.\,5.\,1862 Wien – 21.\,10.\,1931 ebd.), \emph{Schriftsteller, Mediziner}!Ruf des Lebens, russisch]@\strich\emph{[Der Ruf des Lebens, russisch]}|pwkv} von \emph{Der Ruf des Lebens}\pwindex{Schnitzler, Arthur 15.\,5.\,1862 Wien – 21.\,10.\,1931 ebd.@\textsc{Schnitzler, Arthur} (15.\,5.\,1862 Wien – 21.\,10.\,1931 ebd.), \emph{Schriftsteller, Mediziner}!Ruf des Lebens. Schauspiel in drei Akten@\strich\emph{Der Ruf des Lebens. Schauspiel in drei Akten}|pwk}{ }1906 und jene\pwindex{Schnitzler, Arthur 15.\,5.\,1862 Wien – 21.\,10.\,1931 ebd.@\textsc{Schnitzler, Arthur} (15.\,5.\,1862 Wien – 21.\,10.\,1931 ebd.), \emph{Schriftsteller, Mediziner}!einsame Weg, russisch I]@\strich\emph{[Der einsame Weg, russisch I]}|pwkv}
                  von \emph{Der einsame Weg}\pwindex{Schnitzler, Arthur 15.\,5.\,1862 Wien – 21.\,10.\,1931 ebd.@\textsc{Schnitzler, Arthur} (15.\,5.\,1862 Wien – 21.\,10.\,1931 ebd.), \emph{Schriftsteller, Mediziner}!einsame Weg. Schauspiel in fünf Akten@\strich\emph{Der einsame Weg. Schauspiel in fünf Akten}|pwk}{ }1904.}}}\label{K_L02597-1} einige recht minimale Summen, / je 300 Kronen/ als
               Tantiemengarantie erhalten. Weitere Gelder flossen mir nie zu., was aber wie gesagt
               an den traurigen Rechtsverhältnissen zwischen Russland\oindex{Russland@\textbf{Russland}|pw} und Oesterreich\oindex{Österreich@\textbf{Österreich}|pw} liegen mag.
               Wie es scheint haben andre österr.\oindex{Österreich@\textbf{Österreich}|pw} und deutsche\oindex{Deutschland@\textbf{Deutschland}|pw} Autoren auch keine bessern Erfahrungen
               gemacht.\pend
           \selectlanguage{ngerman}\endnumbering\briefempfaengerindex{Herzfeld, Marie@\textsc{Herzfeld, Marie}!zzzSchnitzler, Arthur@\emph{von Arthur Schnitzler}!1909-04-201@{20.\,4.\,1909}|)be}\mylabel{L02597h}  \newcommand{\dateiname}{L02597}\newcommand{\titel}{Arthur Schnitzler an Marie Herzfeld, 20. 4. 1909}\newcommand{\editorInnen}{Martin Anton Müller und Laura Untner}%% latex-leseansicht-abspann.tex
%% Abspann für die Leseansicht.
%% Der Schalter \ifkorrekturansicht ist bereits durch den Vorspann gesetzt.

%% latex-abspann.tex
%% Gemeinsamer Abspann für Korrekturansicht und Leseansicht.
%% Setzt den Schalter \ifkorrekturansicht voraus (gesetzt in den
%% einbindenden Dateien latex-korrekturansicht-abspann.tex bzw.
%% latex-leseansicht-abspann.tex).
%% ---------------------------------------------------------------

\normalsize

% Das esempio-Environment wird nur in der Leseansicht benötigt
\ifkorrekturansicht\else
\newenvironment{esempio}[3]%
{
    \vspace{1.5ex}
    \rlap{\underline{#1}}
    \par
    \setlength{\parindent}{0cm}
    \nopagebreak
    \leftskip=#2cm
    \rightskip=#3cm
}
{
    \par
}
\fi

\doendnotes{C}
\bigskip
\vfill

\clearpage

\footnotesize

\ifkorrekturansicht
  \lohead{\textsc{register}}
\fi

% theindex-Environment neu definieren ohne reledmac
\makeatletter
\renewenvironment{theindex}{%
  \ifkorrekturansicht
    \section*{\indexname}%
  \else
    \subsubsection*{Index der erwähnten Entitäten}%
  \fi
  \setlength{\parindent}{0pt}%
  \setlength{\parskip}{0pt plus 0.3pt}%
  \let\item\@idxitem
}{%
  \ifkorrekturansicht\clearpage\fi
}
\makeatother

\IfFileExists{\jobname-pw.ind}{\input{\jobname-pw.ind}}{}

% Quellenangabe nur in der Leseansicht
\ifkorrekturansicht\else
% Fallback-Definitionen, falls die .tex-Datei \titel etc. nicht gesetzt hat
\providecommand{\titel}{}
\providecommand{\editorInnen}{}
\providecommand{\dateiname}{\jobname}

\vspace{3cm}

\vfill

\footnotesize
\textsc{Quelle}: \titel. Herausgegeben von {\editorInnen}. In: \emph{Arthur Schnitzler: Briefwechsel mit Autorinnen und Autoren}.
 Digitale Edition, https://schnitzler-briefe.acdh.oeaw.ac.at/{\dateiname}.html (Stand \today)
\fi

\end{document}


