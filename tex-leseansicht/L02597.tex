%% latex-korrekturansicht-vorspann.tex
%% Vorspann für die Korrekturansicht.
%% Lädt die gemeinsame Datei latex-vorspann.tex mit gesetztem Schalter.

\newif\ifkorrekturansicht
\korrekturansichttrue

\input{../tex-inputs/latex-vorspann}


\section[Arthur Schnitzler an Marie Herzfeld, 20. 4. 1909]{L02597 Arthur Schnitzler an Marie Herzfeld, 20. 4. 1909}
\nopagebreak\mylabel{L02597v}
\rehead{ }\normalsize\beginnumbering\briefempfaengerindex{Herzfeld, Marie@\textsc{Herzfeld, Marie}!zzzSchnitzler, Arthur@\emph{von Arthur Schnitzler}!1909-04-201@{20. 4. 1909}|(be}
\toendnotes[C]{\smallbreak\pagebreak[2]}\Standort{DLA, A:Schnitzler, HS.1985.1.993.}
\physDesc{Brief, Durchschlag1 Blatt, 1 Seite, 930 Zeichen
\newline{}Schreibmaschine
\newline{}Handschrift: 1) Bleistift, lateinische Kurrent (\noindent{}Vermerk »\textsc{Herzfeld}«)\hspace{1em}2) roter Buntstift (\noindent{}mit rotem Buntstift drei Unterstreichungen)\hspace{1em}}\toendnotes[C]{\smallbreak}
\pstart
           \raggedleft{}{\pb}20. April 09.\pend
           
\pstart{}Verehrtes Fräulein,\pend\vspace{0.5em}
\pstart
           Frau Tesi\pwindex{Rotenstern-Tesi, Anna *~1871-01-11@\textsc{Rotenstern-Tesi, Anna} (*~1871-01-11), \emph{Übersetzer/Übersetzerin}|pw} wird von ihrem Gedächtnis getäuscht,
               wenn Sie Ihnen sagte, dass ich ihr von der Revolutionshochzeit\pwindex{Revolutionsbryllup. Skuespil i tre Akter@\emph{Revolutionsbryllup. Skuespil i tre Akter}|pw} gesprochen hätte. Ich habe von dem Stück\pwindex{Revolutionsbryllup. Skuespil i tre Akter@\emph{Revolutionsbryllup. Skuespil i tre Akter}|pwv} schon das beste gehört, habe es aber
               bisher weder gelesen noch gesehen. Dass Frau
                  Tesi\pwindex{Rotenstern-Tesi, Anna *~1871-01-11@\textsc{Rotenstern-Tesi, Anna} (*~1871-01-11), \emph{Übersetzer/Übersetzerin}|pw} einiges von mir übersetzt hat stimmt. Meine direkten Verhandlungen
               fandem mit ihrem Gatten Herrn Rottenstern
                  Swestitsch\pwindex{Rotenstern, Peter 10.01.1868 – 1944@\textsc{Rotenstern, Peter} (10.01.1868 – 1944), \emph{Journalist/Journalistin, Übersetzer/Übersetzerin}|pw} statt. Beide\pwindex{Rotenstern-Tesi, Anna *~1871-01-11@\textsc{Rotenstern-Tesi, Anna} (*~1871-01-11), \emph{Übersetzer/Übersetzerin}|pwv}\pwindex{Rotenstern, Peter 10.01.1868 – 1944@\textsc{Rotenstern, Peter} (10.01.1868 – 1944), \emph{Journalist/Journalistin, Übersetzer/Übersetzerin}|pwv} scheinen mir, soweit es die
               Konventionsverhältnisse zwischen Oesterreich\oindex{Oesterreich@\textbf{Österreich}, \emph{A.PCLI}|pw}
               und Russland\oindex{Russland@\textbf{Russland}, \emph{A.PCLI}|pw} zulassen, verlässliche Menschen.
                  \label{K_L02597-1v}\edtext{Ich habe von ihnen, sowohl für Zwischenspiel\pwindex{Zwischenspiel. Komoedie in drei Akten@\emph{Zwischenspiel. Komödie in drei Akten}|pw} als für Ruf des Lebens\pwindex{Ruf des Lebens. Schauspiel in drei Akten@\emph{Der Ruf des Lebens. Schauspiel in drei Akten}|pw}, wenn ich mich recht erinnere auch für den einsamen Weg\pwindex{einsame Weg. Schauspiel in fuenf Akten@\emph{Der einsame Weg. Schauspiel in fünf Akten}|pw}}{\lemma{\textnormal{\emph{Ich … Weg}}}\Cendnote{\textnormal{Die \emph{Übersetzung}\pwindex{Zwischenspiel, russisch]@\emph{[Zwischenspiel, russisch]}|pwk} des \emph{Zwischenspiels}\pwindex{Zwischenspiel. Komoedie in drei Akten@\emph{Zwischenspiel. Komödie in drei Akten}|pwk}
                  erschien 1905, jene\pwindex{Ruf des Lebens, russisch]@\emph{[Der Ruf des Lebens, russisch]}|pwkv} von \emph{Der Ruf des Lebens}\pwindex{Ruf des Lebens. Schauspiel in drei Akten@\emph{Der Ruf des Lebens. Schauspiel in drei Akten}|pwk}{ }1906 und jene\pwindex{einsame Weg, russisch I]@\emph{[Der einsame Weg, russisch I]}|pwkv}
                  von \emph{Der einsame Weg}\pwindex{einsame Weg. Schauspiel in fuenf Akten@\emph{Der einsame Weg. Schauspiel in fünf Akten}|pwk}{ }1904.}}}\label{K_L02597-1} einige recht minimale Summen, / je 300 Kronen/ als
               Tantiemengarantie erhalten. Weitere Gelder flossen mir nie zu., was aber wie gesagt
               an den traurigen Rechtsverhältnissen zwischen Russland\oindex{Russland@\textbf{Russland}, \emph{A.PCLI}|pw} und Oesterreich\oindex{Oesterreich@\textbf{Österreich}, \emph{A.PCLI}|pw} liegen mag.
               Wie es scheint haben andre österr.\oindex{Oesterreich@\textbf{Österreich}, \emph{A.PCLI}|pw} und deutsche\oindex{Deutschland@\textbf{Deutschland}, \emph{A.PCLI}|pw} Autoren auch keine bessern Erfahrungen
               gemacht.\pend
           \selectlanguage{ngerman}\endnumbering\briefempfaengerindex{Herzfeld, Marie@\textsc{Herzfeld, Marie}!zzzSchnitzler, Arthur@\emph{von Arthur Schnitzler}!1909-04-201@{20. 4. 1909}|)be}\mylabel{L02597h}  \normalsize

\doendnotes{C}
\bigskip
\vfill

\clearpage

\footnotesize

\lohead{\textsc{register}}

% Definiere theindex-Environment komplett neu ohne reledmac
\makeatletter
\renewenvironment{theindex}{%
  \section*{\indexname}%
  \setlength{\parindent}{0pt}%
  \setlength{\parskip}{0pt plus 0.3pt}%
  \let\item\@idxitem
}{%
  \clearpage
}
\makeatother

\IfFileExists{\jobname-pw.ind}{\input{\jobname-pw.ind}}{}

\end{document}

      