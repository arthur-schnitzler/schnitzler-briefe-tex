%% latex-korrekturansicht-vorspann.tex
%% Vorspann für die Korrekturansicht.
%% Lädt die gemeinsame Datei latex-vorspann.tex mit gesetztem Schalter.

\newif\ifkorrekturansicht
\korrekturansichttrue

\input{../tex-inputs/latex-vorspann}


\section[Hermann Bahr und Therese Strisower an Arthur Schnitzler, {[}26.?{]} 8. 1898]{L00839 Hermann Bahr und Therese Strisower an Arthur Schnitzler,
               {[}26.?{]} 8. 1898}
\nopagebreak\mylabel{L00839v}
\rehead{ }\normalsize\beginnumbering\briefempfaengerindex{Schnitzler, Arthur@\textsc{Schnitzler, Arthur}!zzzHorn, Therese@\emph{von Therese Horn}!1898-08-261@{{[}26.?{]} 8. 1898}|(be}\briefempfaengerindex{Schnitzler, Arthur@\textsc{Schnitzler, Arthur}!zzzBahr, Hermann@\emph{von Hermann Bahr}!1898-08-261@{{[}26.?{]} 8. 1898}|(be}
\toendnotes[C]{\smallbreak\pagebreak[2]}\Standort{CUL, Schnitzler, B 5b.}
\physDesc{Bildpostkarte, 189 Zeichen
\newline{}Handschrift Hermann Bahr: Bleistift, deutsche Kurrent
\newline{}Handschrift Therese Horn: Bleistift, lateinische Kurrent
\newline{}Versand: 1) Stempel: »\nobreak{}\oindex{Carbonin@\textbf{Carbonin}, \emph{P.PPL}|pwk}Schluderb{[}ach{]}, 2\textcolor{gray}{×} 8 98\nobreak{}«.   2) Stempel: »\nobreak{}\oindex{IX., Alsergrund@\textbf{IX., Alsergrund}, \emph{A.ADM3}|pwk}Wien 9/3, 28. 8. 98, 9.V, Bestellt\nobreak{}«. 
\newline{}Ordnung: mit Bleistift von unbekannter Hand nummeriert:
                                    »59« }
\buchAbdrucke{\weitereDrucke{Hermann Bahr, Arthur Schnitzler: \emph{Briefwechsel, Aufzeichnungen, Dokumente (1891–1931)}. Göttingen: \emph{Wallstein} 2018, S. 161.} }\toendnotes[C]{\smallbreak}\pstart{}{\pb}Herrn \textsc{D\textsuperscript{r} Arthur Schnitzler}\pend{}\pstart{}\textsc{Wien IX\oindex{Wien@\textbf{Wien}, \emph{A.ADM2}|pw}}\pend{}\pstart{}\textsc{Frankgasse 1\oindex{Frankgasse 1@\textbf{Frankgasse 1}, \emph{Wohngebäude (K.WHS)}|pw}}\pend{}{\bigskip}
\pstart
           \noindent{}\centering{}{\pb}\textcolor{gray}{\textbf{Landro\oindex{Hoehlenstein@\textbf{Höhlenstein}, \emph{P.PPLQ}|pw} mit Monte Cristallo\oindex{Monte Cristallo@\textbf{Monte Cristallo}, \emph{Berg (N.BRG)}|pw}.}}\pend
           \vspace{1em}
\pstart
           \noindent{}{\pb}Warum biſt Du nicht hier? Telegrafiere ſofort\pend
           \pstart Deinem \spacefill\mbox{Hermann}\pend{}
\pstart
           {[}hs. :{]} Warum waren Sie nicht hier? \label{T_L00839-1v}\edtext{Telegrafieren Sie sofort Ihrer Risa,}{\lemma{\textnormal{\emph{Telegrafieren … Risa,}}}\Cendnote{\textnormal{quer am rechten Rand}}}\label{T_L00839-1}{ }\label{T_L00839-2v}\edtext{aber schon nach Unterach\oindex{Unterach am Attersee@\textbf{Unterach am Attersee}, \emph{P.PPL}|pw}.}{\lemma{\textnormal{\emph{aber … Unterach.}}}\Cendnote{\textnormal{am oberen
                  Rand auf dem Kopf}}}\label{T_L00839-2}\pend
           \selectlanguage{ngerman}\endnumbering\briefempfaengerindex{Schnitzler, Arthur@\textsc{Schnitzler, Arthur}!zzzHorn, Therese@\emph{von Therese Horn}!1898-08-261@{{[}26.?{]} 8. 1898}|)be}\briefempfaengerindex{Schnitzler, Arthur@\textsc{Schnitzler, Arthur}!zzzBahr, Hermann@\emph{von Hermann Bahr}!1898-08-261@{{[}26.?{]} 8. 1898}|)be}\mylabel{L00839h}  \normalsize

\doendnotes{C}
\bigskip
\vfill

\clearpage

\footnotesize

\lohead{\textsc{register}}

% Definiere theindex-Environment komplett neu ohne reledmac
\makeatletter
\renewenvironment{theindex}{%
  \section*{\indexname}%
  \setlength{\parindent}{0pt}%
  \setlength{\parskip}{0pt plus 0.3pt}%
  \let\item\@idxitem
}{%
  \clearpage
}
\makeatother

\IfFileExists{\jobname-pw.ind}{\input{\jobname-pw.ind}}{}

\end{document}

      