%% latex-leseansicht-vorspann.tex
%% Vorspann für die Leseansicht.
%% Lädt die gemeinsame Datei latex-vorspann.tex mit nicht gesetztem Schalter.

\newif\ifkorrekturansicht
\korrekturansichtfalse

\input{../tex-inputs/latex-vorspann}


\section[Hermann Bahr und Therese Strisower an Arthur Schnitzler, {[}26.?{]} 8. 1898]{L00839 Hermann Bahr und Therese Strisower an Arthur Schnitzler, [26.?] 8. 1898}
\nopagebreak\mylabel{L00839v}
\rehead{ }\normalsize\beginnumbering\briefempfaengerindex{Schnitzler, Arthur@\textsc{Schnitzler, Arthur}!zzzHorn, Therese@\emph{von Therese Horn}!1898-08-261@{[26.?] 8. 1898}|(be}\briefempfaengerindex{Schnitzler, Arthur@\textsc{Schnitzler, Arthur}!zzzBahr, Hermann@\emph{von Hermann Bahr}!1898-08-261@{[26.?] 8. 1898}|(be}
\toendnotes[C]{\smallbreak\pagebreak[2]}
\correspDesc{Versand  durch Hermann Bahr, Therese Strisower am [26.?] 8. 1898 in Carbonin
\newline{}Erhalt  durch Arthur Schnitzler am 29. 8. 1898 in Wien}\toendnotes[C]{\smallbreak}
\Standort{CUL, Schnitzler, B 5b.}
\physDesc{Bildpostkarte, 189 Zeichen
\newline{}Handschrift Hermann Bahr: Bleistift, deutsche Kurrent
\newline{}Handschrift Therese Horn: Bleistift, lateinische Kurrent
\newline{}Versand: 1) Stempel: »\nobreak{}\oindex{Carbonin@\textbf{Carbonin}|pwk}Schluderb{[}ach{]}, 2\textcolor{gray}{×} 8 98\nobreak{}«.   2) Stempel: »\nobreak{}\oindex{IX., Alsergrund@\textbf{IX., Alsergrund}, \emph{Verwaltungsgebiet}|pwk}Wien 9/3, 28. 8. 98, 9.V, Bestellt\nobreak{}«. 
\newline{}Ordnung: mit Bleistift von unbekannter Hand nummeriert:
                                    »59« }
\buchAbdrucke{\weitereDrucke{Hermann Bahr, Arthur Schnitzler: \emph{Briefwechsel, Aufzeichnungen, Dokumente (1891–1931)}. Herausgegeben von Kurt Ifkovits und Martin Anton Müller. Göttingen: \emph{Wallstein} 2018, S. 161.} }\toendnotes[C]{\smallbreak}\pstart{}{\pb}Herrn \textsc{D\textsuperscript{r} Arthur Schnitzler}\pend{}\pstart{}\textsc{Wien IX\oindex{Wien@\textbf{Wien}, \emph{Verwaltungsgebiet}|pw}}\pend{}\pstart{}\textsc{Frankgasse 1\oindex{Wien@\textbf{Wien}!IX., Alsergrund@\textbf{IX., Alsergrund}!Frankgasse 1@\textbf{Frankgasse 1}, \emph{Wohngebäude}|pw}}\pend{}{\bigskip}
\pstart
           \noindent{}\centering{}{\pb}\textcolor{gray}{\textbf{Landro\oindex{Höhlenstein@\textbf{Höhlenstein}|pw} mit Monte Cristallo\oindex{Monte Cristallo@\textbf{Monte Cristallo}, \emph{Berg}|pw}.}}\pend
           \vspace{1em}
\pstart
           \noindent{}{\pb}Warum biſt Du nicht hier? Telegrafiere{ }ſofort\pend
           \pstart Deinem \spacefill\mbox{Hermann}\pend{}
\pstart
           {[}hs. Horn:{]} Warum waren Sie nicht hier? \label{T_L00839-1v}\edtext{Telegrafieren Sie sofort Ihrer Risa,}{\lemma{\textnormal{\emph{Telegrafieren … Risa,}}}\Cendnote{\textnormal{quer am rechten Rand}}}\label{T_L00839-1}{ }\label{T_L00839-2v}\edtext{aber schon nach Unterach\oindex{Unterach am Attersee@\textbf{Unterach am Attersee}|pw}.}{\lemma{\textnormal{\emph{aber … Unterach.}}}\Cendnote{\textnormal{am oberen
                  Rand auf dem Kopf}}}\label{T_L00839-2}\pend
           \selectlanguage{ngerman}\endnumbering\briefempfaengerindex{Schnitzler, Arthur@\textsc{Schnitzler, Arthur}!zzzHorn, Therese@\emph{von Therese Horn}!1898-08-261@{[26.?] 8. 1898}|)be}\briefempfaengerindex{Schnitzler, Arthur@\textsc{Schnitzler, Arthur}!zzzBahr, Hermann@\emph{von Hermann Bahr}!1898-08-261@{[26.?] 8. 1898}|)be}\mylabel{L00839h}  \newcommand{\dateiname}{L00839}\newcommand{\titel}{Hermann Bahr und Therese Strisower an Arthur Schnitzler, [26.?] 8. 1898}\newcommand{\editorInnen}{Herausgegeben von Martin Anton Müller}%% latex-leseansicht-abspann.tex
%% Abspann für die Leseansicht.
%% Der Schalter \ifkorrekturansicht ist bereits durch den Vorspann gesetzt.

%% latex-abspann.tex
%% Gemeinsamer Abspann für Korrekturansicht und Leseansicht.
%% Setzt den Schalter \ifkorrekturansicht voraus (gesetzt in den
%% einbindenden Dateien latex-korrekturansicht-abspann.tex bzw.
%% latex-leseansicht-abspann.tex).
%% ---------------------------------------------------------------

\normalsize

% Das esempio-Environment wird nur in der Leseansicht benötigt
\ifkorrekturansicht\else
\newenvironment{esempio}[3]%
{
    \vspace{1.5ex}
    \rlap{\underline{#1}}
    \par
    \setlength{\parindent}{0cm}
    \nopagebreak
    \leftskip=#2cm
    \rightskip=#3cm
}
{
    \par
}
\fi

\doendnotes{C}
\bigskip
\vfill

\clearpage

\footnotesize

\ifkorrekturansicht
  \lohead{\textsc{register}}
\fi

% theindex-Environment neu definieren ohne reledmac
\makeatletter
\renewenvironment{theindex}{%
  \ifkorrekturansicht
    \section*{\indexname}%
  \else
    \subsubsection*{Index der erwähnten Entitäten}%
  \fi
  \setlength{\parindent}{0pt}%
  \setlength{\parskip}{0pt plus 0.3pt}%
  \let\item\@idxitem
}{%
  \ifkorrekturansicht\clearpage\fi
}
\makeatother

\IfFileExists{\jobname-pw.ind}{\input{\jobname-pw.ind}}{}

% Quellenangabe nur in der Leseansicht
\ifkorrekturansicht\else
% Fallback-Definitionen, falls die .tex-Datei \titel etc. nicht gesetzt hat
\providecommand{\titel}{}
\providecommand{\editorInnen}{}
\providecommand{\dateiname}{\jobname}

\vspace{3cm}

\vfill

\footnotesize
\textsc{Quelle}: \titel. Herausgegeben von {\editorInnen}. In: \emph{Arthur Schnitzler: Briefwechsel mit Autorinnen und Autoren}.
 Digitale Edition, https://schnitzler-briefe.acdh.oeaw.ac.at/{\dateiname}.html (Stand \today)
\fi

\end{document}


