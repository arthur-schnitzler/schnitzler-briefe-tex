%% latex-leseansicht-vorspann.tex
%% Vorspann für die Leseansicht.
%% Lädt die gemeinsame Datei latex-vorspann.tex mit nicht gesetztem Schalter.

\newif\ifkorrekturansicht
\korrekturansichtfalse

\input{../tex-inputs/latex-vorspann}


\section[Arthur Schnitzler an Berta Zuckerkandl, 14. 5. 1927]{L03970 Arthur Schnitzler an Berta Zuckerkandl, 14. 5. 1927}
\nopagebreak\mylabel{L03970v}
\rehead{ }\normalsize\beginnumbering\briefempfaengerindex{Zuckerkandl, Berta@\textsc{Zuckerkandl, Berta}!zzzSchnitzler, Arthur@\emph{von Arthur Schnitzler}!1927-05-141@{14. 5. 1927}|(be}
\toendnotes[C]{\smallbreak\pagebreak[2]}
\correspDesc{Versand  durch Arthur Schnitzler am 14. 5. 1927 in Wien
\newline{}Erhalt  durch Berta Zuckerkandl im Zeitraum [14. 5. 1927
                  – 17. 5. 1927?] in Wien}\toendnotes[C]{\smallbreak}
\Standort{DLA, HS.1985.1.2282.}
\physDesc{Brief, Durchschlag, 1 Blatt, 1 Seite, 1535 Zeichen
\newline{}Schreibmaschine
\newline{}Handschrift: roter Buntstift, lateinische Kurrent (\noindent{}beschriftet: »\uline{Zuckerkandl}«, neun
                                 Unterstreichungen)}\toendnotes[C]{\smallbreak}
\pstart
           \raggedleft{}{\pb}14. 5. 1927.\pend
           
\pstart{}Liebe und verehrte Frau Hofrätin.\pend\vspace{0.5em}
\pstart
           Mit Beziehung auf unsere Gespräche heute und gestern möchte
               ich, meine Ansichten zusammenfassend, wiederholen, dass ich es durchaus im Interesse
               und zwar nicht nur im materi{[}e{]}llen Interesse von Paul Géraldy\pwindex{Géraldy, Paul 6.\,3.\,1885 Paris – 9.\,3.\,1983 Neuilly-sur-Seine@\textsc{Géraldy, Paul} (6.\,3.\,1885 Paris – 9.\,3.\,1983 Neuilly-sur-Seine), \emph{Schriftsteller}|pw} fände, wenn er den Antrag Zsolnay\pwindex{Zsolnay, Paul 12.\,6.\,1895 Budapest – 13.\,5.\,1961 Wien@\textsc{Zsolnay, Paul} (12.\,6.\,1895 Budapest – 13.\,5.\,1961 Wien), \emph{Verleger}|pw} annähme, selbst für den Fall, dass das
                  \label{K_L03970-1v}\edtext{\begin{otherlanguage}{french}à
                  valoir\end{otherlanguage}}{\lemma{\textnormal{\emph{à
                  valoir}}}\Cendnote{\textnormal{französisch:
                  Vorschuss}}}\label{K_L03970-1} für die Bühnenaufführungen ein geringeres wäre, als es ihm von
               anderer Seite geboten wird. Dieses Minus wird ja selbstverständlich durch einen
               Theatererfolg sehr rasch hereingebracht; andererseits ist auch für die literarische
               Stellung eines ausländischen Autors, mag sein Ruf auch noch so fest schon gegründet
               sein, der Vorteil nicht zu unterschätzen mit einer \label{K_L03970-2v}\edtext{deutschen Gesammtausgabe}{\lemma{\textnormal{\emph{deutschen Gesammtausgabe}}}\Cendnote{\textnormal{Im November 1927 erschien ein Band mit Gedichten
                  von Paul Géraldy\pwindex{Géraldy, Paul 6.\,3.\,1885 Paris – 9.\,3.\,1983 Neuilly-sur-Seine@\textsc{Géraldy, Paul} (6.\,3.\,1885 Paris – 9.\,3.\,1983 Neuilly-sur-Seine), \emph{Schriftsteller}|pwk}, ins Französische\oindex{Frankreich@\textbf{Frankreich}|pwk} übersetzt
                  von Clara Katharina Pollaczek\pwindex{Pollaczek, Clara Katharina 15.\,1.\,1875 Wien – 22.\,7.\,1951 ebd.@\textsc{Pollaczek, Clara Katharina} (15.\,1.\,1875 Wien – 22.\,7.\,1951 ebd.), \emph{Schriftstellerin}|pwk} (Paul Géraldy\pwindex{Géraldy, Paul 6.\,3.\,1885 Paris – 9.\,3.\,1983 Neuilly-sur-Seine@\textsc{Géraldy, Paul} (6.\,3.\,1885 Paris – 9.\,3.\,1983 Neuilly-sur-Seine), \emph{Schriftsteller}|pwk}: \emph{Du und ich. Gedichte}\pwindex{Géraldy, Paul 6.\,3.\,1885 Paris – 9.\,3.\,1983 Neuilly-sur-Seine@\textsc{Géraldy, Paul} (6.\,3.\,1885 Paris – 9.\,3.\,1983 Neuilly-sur-Seine), \emph{Schriftsteller}!Du und ich. Gedichte@\strich\emph{Du und ich. Gedichte}|pwk}. Deutsche Nachdichtung von Clara Katharina Pollaczek\pwindex{Pollaczek, Clara Katharina 15.\,1.\,1875 Wien – 22.\,7.\,1951 ebd.@\textsc{Pollaczek, Clara Katharina} (15.\,1.\,1875 Wien – 22.\,7.\,1951 ebd.), \emph{Schriftstellerin}|pwk},
                     Berlin, Wien,
                     Leipzig: \emph{Zsolnay}\orgindex{Paul Zsolnay Verlag@Paul Zsolnay Verlag|pwk}
                     1927), im Jahr darauf eine deutsche Edition seiner \emph{Dramen}\pwindex{Géraldy, Paul 6.\,3.\,1885 Paris – 9.\,3.\,1983 Neuilly-sur-Seine@\textsc{Géraldy, Paul} (6.\,3.\,1885 Paris – 9.\,3.\,1983 Neuilly-sur-Seine), \emph{Schriftsteller}!Aimer (Aimée)@\strich\emph{Aimer (Aimée)}|pwk}\pwindex{Géraldy, Paul 6.\,3.\,1885 Paris – 9.\,3.\,1983 Neuilly-sur-Seine@\textsc{Géraldy, Paul} (6.\,3.\,1885 Paris – 9.\,3.\,1983 Neuilly-sur-Seine), \emph{Schriftsteller}!Robert und Marianne@\strich\emph{Robert und Marianne}|pwk}\pwindex{Géraldy, Paul 6.\,3.\,1885 Paris – 9.\,3.\,1983 Neuilly-sur-Seine@\textsc{Géraldy, Paul} (6.\,3.\,1885 Paris – 9.\,3.\,1983 Neuilly-sur-Seine), \emph{Schriftsteller}!noces d’argent@\strich\emph{Les noces d’argent}|pwk}\pwindex{Géraldy, Paul 6.\,3.\,1885 Paris – 9.\,3.\,1983 Neuilly-sur-Seine@\textsc{Géraldy, Paul} (6.\,3.\,1885 Paris – 9.\,3.\,1983 Neuilly-sur-Seine), \emph{Schriftsteller}!Grands Garçons, comédie en un acte@\strich\emph{Les Grands Garçons, comédie en un acte}|pwk} in einer
                  Übersetzung von Berta Zuckerkandl\pwindex{Zuckerkandl, Berta 13.\,4.\,1864 Wien – 16.\,10.\,1945 Paris@\textsc{Zuckerkandl, Berta} (13.\,4.\,1864 Wien – 16.\,10.\,1945 Paris), \emph{Schriftstellerin, Journalistin, Übersetzerin}|pwk} (Paul Géraldy\pwindex{Géraldy, Paul 6.\,3.\,1885 Paris – 9.\,3.\,1983 Neuilly-sur-Seine@\textsc{Géraldy, Paul} (6.\,3.\,1885 Paris – 9.\,3.\,1983 Neuilly-sur-Seine), \emph{Schriftsteller}|pwk}: \emph{Theater}\pwindex{Géraldy, Paul 6.\,3.\,1885 Paris – 9.\,3.\,1983 Neuilly-sur-Seine@\textsc{Géraldy, Paul} (6.\,3.\,1885 Paris – 9.\,3.\,1983 Neuilly-sur-Seine), \emph{Schriftsteller}!Theater@\strich\emph{Theater}|pwk}. Berlin,
                     Wien, Leipzig: \emph{Paul Zsolnay Verlag}\orgindex{Paul Zsolnay Verlag@Paul Zsolnay Verlag|pwk}{ }1928).}}}\label{K_L03970-2} herauszukommen, überdies in einem so
               rührigen Verlag, als es der Verlag Zsolnay\orgindex{Paul Zsolnay Verlag@Paul Zsolnay Verlag|pw} ist.
               Und einen weiteren bedeutungsvollen Vorteil sehe ich darin, dass dann eben
               Bühnenvertrieb und Buchvertrieb in derselben Hand vereinigt wäre. Ich denke, dass
               sich Paul Géraldy\pwindex{Géraldy, Paul 6.\,3.\,1885 Paris – 9.\,3.\,1983 Neuilly-sur-Seine@\textsc{Géraldy, Paul} (6.\,3.\,1885 Paris – 9.\,3.\,1983 Neuilly-sur-Seine), \emph{Schriftsteller}|pw} solchen Erwägungen umso
               weniger wird verschliessen können, als ja seine \label{K_L03970-3v}\edtext{Verhandlungen mit Ihnen}{\lemma{\textnormal{\emph{Verhandlungen mit Ihnen}}}\Cendnote{\textnormal{Es ging um die Übersetzungen von Berta Zuckerkandl\pwindex{Zuckerkandl, Berta 13.\,4.\,1864 Wien – 16.\,10.\,1945 Paris@\textsc{Zuckerkandl, Berta} (13.\,4.\,1864 Wien – 16.\,10.\,1945 Paris), \emph{Schriftstellerin, Journalistin, Übersetzerin}|pwk} für eine Edition von Géraldys\pwindex{Géraldy, Paul 6.\,3.\,1885 Paris – 9.\,3.\,1983 Neuilly-sur-Seine@\textsc{Géraldy, Paul} (6.\,3.\,1885 Paris – 9.\,3.\,1983 Neuilly-sur-Seine), \emph{Schriftsteller}|pwk}{ }\emph{Dramen}\pwindex{Géraldy, Paul 6.\,3.\,1885 Paris – 9.\,3.\,1983 Neuilly-sur-Seine@\textsc{Géraldy, Paul} (6.\,3.\,1885 Paris – 9.\,3.\,1983 Neuilly-sur-Seine), \emph{Schriftsteller}!Theater@\strich\emph{Theater}|pwk}.}}}\label{K_L03970-3}, liebe
               Freundin, schon recht weit, ja nahezu zum Abschluss gediehen waren. Es steht Ihnen
               natürlich frei von diesem Schreiben Géraldy\pwindex{Géraldy, Paul 6.\,3.\,1885 Paris – 9.\,3.\,1983 Neuilly-sur-Seine@\textsc{Géraldy, Paul} (6.\,3.\,1885 Paris – 9.\,3.\,1983 Neuilly-sur-Seine), \emph{Schriftsteller}|pw}
               gegenüber, den ich bei dieser Gelegenheit herzlichst zu grüssen bitte, Gebrauch zu
               machen.\pend
           
\pstart
           Hier noch die Adresse von Mlle. Madeleine
                  Lindauer\pwindex{Lindauer, Madeleine 30.\,12.\,1899 Paris – 6.\,7.\,1935 ebd.@\textsc{Lindauer, Madeleine} (30.\,12.\,1899 Paris – 6.\,7.\,1935 ebd.), \emph{Übersetzerin}|pw}, 6, Rue Anatole de la Forge,
                  Paris\oindex{6, Rue Anatole de la Forge@\textbf{6, Rue Anatole de la Forge}, \emph{Wohngebäude}|pw}.\pend
           \pstart Mit den besten Wünschen für Ihre Reise, die hoffentlich nach allen Richtungen
               hin angenehm und erfolgreich für Sie sein wird, bin ich, wie immer, Ihr aufrichtig
               ergebener\pend{}\selectlanguage{ngerman}\endnumbering\briefempfaengerindex{Zuckerkandl, Berta@\textsc{Zuckerkandl, Berta}!zzzSchnitzler, Arthur@\emph{von Arthur Schnitzler}!1927-05-141@{14. 5. 1927}|)be}\mylabel{L03970h}
\begin{anhang}
\end{anhang}\newcommand{\dateiname}{L03970}\newcommand{\titel}{Arthur Schnitzler an Berta Zuckerkandl, 14. 5. 1927}\newcommand{\editorInnen}{Herausgegeben von Jahnke, SelmaMüller, Martin Anton}%% latex-leseansicht-abspann.tex
%% Abspann für die Leseansicht.
%% Der Schalter \ifkorrekturansicht ist bereits durch den Vorspann gesetzt.

%% latex-abspann.tex
%% Gemeinsamer Abspann für Korrekturansicht und Leseansicht.
%% Setzt den Schalter \ifkorrekturansicht voraus (gesetzt in den
%% einbindenden Dateien latex-korrekturansicht-abspann.tex bzw.
%% latex-leseansicht-abspann.tex).
%% ---------------------------------------------------------------

\normalsize

% Das esempio-Environment wird nur in der Leseansicht benötigt
\ifkorrekturansicht\else
\newenvironment{esempio}[3]%
{
    \vspace{1.5ex}
    \rlap{\underline{#1}}
    \par
    \setlength{\parindent}{0cm}
    \nopagebreak
    \leftskip=#2cm
    \rightskip=#3cm
}
{
    \par
}
\fi

\doendnotes{C}
\bigskip
\vfill

\clearpage

\footnotesize

\ifkorrekturansicht
  \lohead{\textsc{register}}
\fi

% theindex-Environment neu definieren ohne reledmac
\makeatletter
\renewenvironment{theindex}{%
  \ifkorrekturansicht
    \section*{\indexname}%
  \else
    \subsubsection*{Index der erwähnten Entitäten}%
  \fi
  \setlength{\parindent}{0pt}%
  \setlength{\parskip}{0pt plus 0.3pt}%
  \let\item\@idxitem
}{%
  \ifkorrekturansicht\clearpage\fi
}
\makeatother

\IfFileExists{\jobname-pw.ind}{\input{\jobname-pw.ind}}{}

% Quellenangabe nur in der Leseansicht
\ifkorrekturansicht\else
% Fallback-Definitionen, falls die .tex-Datei \titel etc. nicht gesetzt hat
\providecommand{\titel}{}
\providecommand{\editorInnen}{}
\providecommand{\dateiname}{\jobname}

\vspace{3cm}

\vfill

\footnotesize
\textsc{Quelle}: \titel. Herausgegeben von {\editorInnen}. In: \emph{Arthur Schnitzler: Briefwechsel mit Autorinnen und Autoren}.
 Digitale Edition, https://schnitzler-briefe.acdh.oeaw.ac.at/{\dateiname}.html (Stand \today)
\fi

\end{document}


