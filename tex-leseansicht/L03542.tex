%% latex-leseansicht-vorspann.tex
%% Vorspann für die Leseansicht.
%% Lädt die gemeinsame Datei latex-vorspann.tex mit nicht gesetztem Schalter.

\newif\ifkorrekturansicht
\korrekturansichtfalse

\input{../tex-inputs/latex-vorspann}


\section[ Eva Marie Goldmann an Arthur Schnitzler, 1. 1. 1927]{L03542 Eva Marie Goldmann an Arthur Schnitzler,  1. 1. 1927}
\nopagebreak\mylabel{L03542v}
\rehead{ }\normalsize\beginnumbering\briefempfaengerindex{Schnitzler, Arthur@\textsc{Schnitzler, Arthur}!zzzGoldmann, Eva Marie@\emph{von Eva Marie Goldmann}!1927-01-011@{1. 1. 1927}|(be}
\toendnotes[C]{\smallbreak\pagebreak[2]}
\correspDesc{Versand  durch Eva Marie Goldmann am 1. 1. 1927 in Berlin
\newline{}Erhalt  durch Arthur Schnitzler im Zeitraum [1. 1. 1927
                  – 5. 1. 1927?] in Berlin}\toendnotes[C]{\smallbreak}
\Standort{DLA, A:Schnitzler, HS.NZ85.1.3160.}
\physDesc{Briefkarte, 434 Zeichen
\newline{}Handschrift: schwarze Tinte, lateinische Kurrent
\newline{}Schnitzler: mit rotem Buntstift Vermerk »Goldma{[}nn{]}\pwindex{Goldmann, Eva Marie 27.\,10.\,1877 Wien – 2.\,11.\,1937 ebd.@\textsc{Goldmann, Eva Marie} (27.\,10.\,1877 Wien – 2.\,11.\,1937 ebd.)|pw}« }\toendnotes[C]{\smallbreak}
\pstart
           \textcolor{gray}{\textbf{E G}}\hfill {\pb}\textcolor{gray}{\textbf{BERLIN, W. 10\oindex{Berlin@\textbf{Berlin}, \emph{Hauptstadt}|pw}}}\pend
           
\pstart
           \raggedleft{}\textcolor{gray}{\textbf{BENDLERSTRASSE 36\oindex{Stauffenbergstraße@\textbf{Stauffenbergstraße}, \emph{Straße}|pw}}}{ }1. \textcolor{gray}{I}. 27.\pend
           
\pstart{}Verehrter Herr Doctor,\pend\vspace{0.5em}
\pstart
           nehmen Sie vielen und herzlichen Dank für die herrlichen Blumen, für Ihr freundliches
               Gedenken und für die guten Wünsche, die wir alle auf das herzlichste erwiedern. {\pb}Mit Ihrem \label{K_L03542-1v}\edtext{Besuch}{\lemma{\textnormal{\emph{Besuch}}}\Cendnote{\textnormal{am 31. 12. 1926}}}\label{K_L03542-1} haben wir uns alle noch besonders gefreut und hoffen, dass Sie ihn bald
                  \label{K_L03542-2v}\edtext{wiederholen}{\lemma{\textnormal{\emph{wiederholen}}}\Cendnote{\textnormal{Schnitzler besuchte die Familie Goldmann\pwindex{Goldmann, Eva Marie 27.\,10.\,1877 Wien – 2.\,11.\,1937 ebd.@\textsc{Goldmann, Eva Marie} (27.\,10.\,1877 Wien – 2.\,11.\,1937 ebd.)|pwkv}\pwindex{Goldmann, Paul 31.\,1.\,1865 Breslau – 25.\,9.\,1935 Wien@\textsc{Goldmann, Paul} (31.\,1.\,1865 Breslau – 25.\,9.\,1935 Wien), \emph{Schriftsteller, Journalist}|pwkv}\pwindex{Goldmann, Franziska 29.\,5.\,1911 Berlin – 19.\,8.\,1963 Rio de Janeiro@\textsc{Goldmann, Franziska} (29.\,5.\,1911 Berlin – 19.\,8.\,1963 Rio de Janeiro), \emph{Schauspielerin}|pwkv} das nächste Mal fast ein Jahr später, am 5. 12. 1927.}}}\label{K_L03542-2}
               werden, denn \label{K_L03542-3v}\edtext{G. s. D.}{\lemma{\textnormal{\emph{G. s. D.}}}\Cendnote{\textnormal{Gott sei Dank}}}\label{K_L03542-3} ist ja Berlin\oindex{Berlin@\textbf{Berlin}, \emph{Hauptstadt}|pw} eine \label{T_L03542-1v}\edtext{Stadt}{\lemma{\textnormal{\emph{Stadt}}}\Cendnote{\textnormal{korrigiert
                  aus »Statt«}}}\label{T_L03542-1}, in der man immer wieder zu tun hat.\pend
           
\pstart
           Mit den schönsten Grüssen und Wünschen {\\[\baselineskip]}Ihre ergebene {\\[\baselineskip]}\spacefill\mbox{EvaGoldmann.}\pend
           \leftskip=0em{}\selectlanguage{ngerman}\endnumbering\briefempfaengerindex{Schnitzler, Arthur@\textsc{Schnitzler, Arthur}!zzzGoldmann, Eva Marie@\emph{von Eva Marie Goldmann}!1927-01-011@{1. 1. 1927}|)be}\mylabel{L03542h}  \newcommand{\dateiname}{L03542}\newcommand{\titel}{Eva Marie Goldmann an Arthur Schnitzler, 1. 1. 1927}\newcommand{\editorInnen}{Martin Anton Müller und Laura Untner}%% latex-leseansicht-abspann.tex
%% Abspann für die Leseansicht.
%% Der Schalter \ifkorrekturansicht ist bereits durch den Vorspann gesetzt.

%% latex-abspann.tex
%% Gemeinsamer Abspann für Korrekturansicht und Leseansicht.
%% Setzt den Schalter \ifkorrekturansicht voraus (gesetzt in den
%% einbindenden Dateien latex-korrekturansicht-abspann.tex bzw.
%% latex-leseansicht-abspann.tex).
%% ---------------------------------------------------------------

\normalsize

% Das esempio-Environment wird nur in der Leseansicht benötigt
\ifkorrekturansicht\else
\newenvironment{esempio}[3]%
{
    \vspace{1.5ex}
    \rlap{\underline{#1}}
    \par
    \setlength{\parindent}{0cm}
    \nopagebreak
    \leftskip=#2cm
    \rightskip=#3cm
}
{
    \par
}
\fi

\doendnotes{C}
\bigskip
\vfill

\clearpage

\footnotesize

\ifkorrekturansicht
  \lohead{\textsc{register}}
\fi

% theindex-Environment neu definieren ohne reledmac
\makeatletter
\renewenvironment{theindex}{%
  \ifkorrekturansicht
    \section*{\indexname}%
  \else
    \subsubsection*{Index der erwähnten Entitäten}%
  \fi
  \setlength{\parindent}{0pt}%
  \setlength{\parskip}{0pt plus 0.3pt}%
  \let\item\@idxitem
}{%
  \ifkorrekturansicht\clearpage\fi
}
\makeatother

\IfFileExists{\jobname-pw.ind}{\input{\jobname-pw.ind}}{}

% Quellenangabe nur in der Leseansicht
\ifkorrekturansicht\else
% Fallback-Definitionen, falls die .tex-Datei \titel etc. nicht gesetzt hat
\providecommand{\titel}{}
\providecommand{\editorInnen}{}
\providecommand{\dateiname}{\jobname}

\vspace{3cm}

\vfill

\footnotesize
\textsc{Quelle}: \titel. Herausgegeben von {\editorInnen}. In: \emph{Arthur Schnitzler: Briefwechsel mit Autorinnen und Autoren}.
 Digitale Edition, https://schnitzler-briefe.acdh.oeaw.ac.at/{\dateiname}.html (Stand \today)
\fi

\end{document}


