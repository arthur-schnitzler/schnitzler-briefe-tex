%% latex-leseansicht-vorspann.tex
%% Vorspann für die Leseansicht.
%% Lädt die gemeinsame Datei latex-vorspann.tex mit nicht gesetztem Schalter.

\newif\ifkorrekturansicht
\korrekturansichtfalse

\input{../tex-inputs/latex-vorspann}

\begin{center}
            \textcolor{red}{ENTWURF, NICHT FERTIG KORRIGIERT}
                      \end{center}
            
         
         \renewcommand{\erwaehntePersonen}{Personen: Eva Marie Goldmann, Paul Goldmann, Franziska Goldmann}
         \renewcommand{\erwaehnteOrte}{Orte: Bendlerstraße, Berlin}
         \renewcommand{\erwaehnteWerke}{}
               \section[ Eva Marie Goldmann an Arthur Schnitzler, 1. 1. 1927]{ Eva Marie Goldmann an Arthur Schnitzler, 1. 1. 1927}\nopagebreak\mylabel{v}\rehead{ }\begin{ledgroupsized}[t]{13cm}\normalsize\beginnumbering \toendnotes[C]{\smallbreak\pagebreak[2]} \Standort{DLA, A:Schnitzler, HS.NZ85.1.3160.}
\physDesc{Briefkarte, 434 Zeichen
\newline{}Handschrift: schwarze Tinte, lateinische Kurrent
\newline{}Schnitzler: mit rotem Buntstift Vermerk »Goldma{[}nn{]}\pwindex{Goldmann, Eva Marie 27.10.1877 – 02.11.1937@\textsc{Goldmann, Eva Marie} (27.10.1877 – 02.11.1937)|pw}« }\toendnotes[C]{\smallbreak}\pstart
           \noindent{}\textcolor{gray}{\textbf{E G}}\hfill {\pb}\textcolor{gray}{\textbf{BERLIN, W. 10\oindex{Berlin@\textbf{Berlin}|pw}}}\pend
           \pstart
           \raggedleft{}\textcolor{gray}{\textbf{BENDLERSTRASSE 36\oindex{Bendlerstrasse@\textbf{Bendlerstraße}|pw}}}{ }1. \textcolor{gray}{I}. 27.
               \pend
           \pstart{}Verehrter Herr Doctor,\pend\pstart
           nehmen Sie vielen und herzlichen Dank für die herrlichen Blumen, für Ihr freundliches
               Gedenken und für die guten Wünsche, die wir alle auf das herzlichste erwiedern. {\pb}Mit Ihrem \label{K_L03542-1v}\edtext{Besuch}{\lemma{\textnormal{\emph{Besuch}}}\Cendnote{\textnormal{am 31. 12. 1926}}}\label{K_L03542-1h} haben wir uns alle noch besonders gefreut und hoffen, dass Sie ihn bald
                  \label{K_L03542-2v}\edtext{wiederholen}{\lemma{\textnormal{\emph{wiederholen}}}\Cendnote{\textnormal{Schnitzler\pwindex{Schnitzler, Arthur 15.05.1862 – 21.10.1931@\textsc{Schnitzler, Arthur} (15.05.1862 – 21.10.1931), \emph{Schriftsteller, Mediziner}|pwk} besuchte die Familie Goldmann\pwindex{Goldmann, Eva Marie 27.10.1877 – 02.11.1937@\textsc{Goldmann, Eva Marie} (27.10.1877 – 02.11.1937)|pwkv}\pwindex{Goldmann, Paul 31.01.1865 – 25.09.1935@\textsc{Goldmann, Paul} (31.01.1865 – 25.09.1935), \emph{Schriftsteller, Journalist}|pwkv}\pwindex{Goldmann, Franziska 1911-05-29 – 1963-08-19@\textsc{Goldmann, Franziska} (1911-05-29 – 1963-08-19), \emph{Schauspielerin}|pwkv} das
                  nächste Mal fast ein Jahr später, am 5. 12. 1927.}}}\label{K_L03542-2h} werden, denn \label{K_L03542-3v}\edtext{G. s. D.}{\lemma{\textnormal{\emph{G. s. D.}}}\Cendnote{\textnormal{Gott sei
                  Dank}}}\label{K_L03542-3h} ist ja Berlin\oindex{Berlin@\textbf{Berlin}|pw} eine \label{T_L03542-1v}\edtext{Stadt}{\lemma{\textnormal{\emph{Stadt}}}\Cendnote{\textnormal{korrigiert aus
               »Statt«}}}\label{T_L03542-1h}, in der man immer wieder zu tun hat.\pend
           \pstart
           Mit den schönsten Grüssen und Wünschen {\\[\baselineskip]}Ihre ergebene {\\[\baselineskip]}\spacefill\mbox{EvaGoldmann.}\pend
           \leftskip=0em{}
         
         \endnumbering\mylabel{h}\end{ledgroupsized}  \newcommand{\dateiname}{L03542}\newcommand{\titel}{Eva Marie Goldmann an Arthur Schnitzler, 1. 1. 1927}\newcommand{\editorInnen}{Martin Anton Müller und Laura Untner}%% latex-leseansicht-abspann.tex
%% Abspann für die Leseansicht.
%% Der Schalter \ifkorrekturansicht ist bereits durch den Vorspann gesetzt.

%% latex-abspann.tex
%% Gemeinsamer Abspann für Korrekturansicht und Leseansicht.
%% Setzt den Schalter \ifkorrekturansicht voraus (gesetzt in den
%% einbindenden Dateien latex-korrekturansicht-abspann.tex bzw.
%% latex-leseansicht-abspann.tex).
%% ---------------------------------------------------------------

\normalsize

% Das esempio-Environment wird nur in der Leseansicht benötigt
\ifkorrekturansicht\else
\newenvironment{esempio}[3]%
{
    \vspace{1.5ex}
    \rlap{\underline{#1}}
    \par
    \setlength{\parindent}{0cm}
    \nopagebreak
    \leftskip=#2cm
    \rightskip=#3cm
}
{
    \par
}
\fi

\doendnotes{C}
\bigskip
\vfill

\clearpage

\footnotesize

\ifkorrekturansicht
  \lohead{\textsc{register}}
\fi

% theindex-Environment neu definieren ohne reledmac
\makeatletter
\renewenvironment{theindex}{%
  \ifkorrekturansicht
    \section*{\indexname}%
  \else
    \subsubsection*{Index der erwähnten Entitäten}%
  \fi
  \setlength{\parindent}{0pt}%
  \setlength{\parskip}{0pt plus 0.3pt}%
  \let\item\@idxitem
}{%
  \ifkorrekturansicht\clearpage\fi
}
\makeatother

\IfFileExists{\jobname-pw.ind}{\input{\jobname-pw.ind}}{}

% Quellenangabe nur in der Leseansicht
\ifkorrekturansicht\else
% Fallback-Definitionen, falls die .tex-Datei \titel etc. nicht gesetzt hat
\providecommand{\titel}{}
\providecommand{\editorInnen}{}
\providecommand{\dateiname}{\jobname}

\vspace{3cm}

\vfill

\footnotesize
\textsc{Quelle}: \titel. Herausgegeben von {\editorInnen}. In: \emph{Arthur Schnitzler: Briefwechsel mit Autorinnen und Autoren}.
 Digitale Edition, https://schnitzler-briefe.acdh.oeaw.ac.at/{\dateiname}.html (Stand \today)
\fi

\end{document}


      