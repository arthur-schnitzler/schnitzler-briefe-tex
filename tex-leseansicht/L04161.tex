%% latex-leseansicht-vorspann.tex
%% Vorspann für die Leseansicht.
%% Lädt die gemeinsame Datei latex-vorspann.tex mit nicht gesetztem Schalter.

\newif\ifkorrekturansicht
\korrekturansichtfalse

\input{../tex-inputs/latex-vorspann}


\section[Arthur Schnitzler an Gustav Schwarzkopf, 1. 6. 1913]{L04161 Arthur Schnitzler an Gustav Schwarzkopf, 1. 6. 1913}
\nopagebreak\mylabel{L04161v}
\rehead{ }\normalsize\beginnumbering\briefempfaengerindex{Schwarzkopf, Gustav@\textsc{Schwarzkopf, Gustav}!zzzSchnitzler, Arthur@\emph{von Arthur Schnitzler}!1913-06-011@{1. 6. 1913}|(be}
\toendnotes[C]{\smallbreak\pagebreak[2]}
\correspDesc{Versand  durch Arthur Schnitzler am 1. 6. 1913 in Wien
\newline{}Erhalt  durch Gustav Schwarzkopf im Zeitraum [2. 6. 1913 – 6. 6. 1913?] in Baden bei Wien}\toendnotes[C]{\smallbreak}
\Standort{CUL, Schnitzler, B 96.}
\physDesc{Kartenbrief, 544 Zeichen
\newline{}Handschrift: Bleistift, deutsche Kurrent
\newline{}Versand: Stempel: »\nobreak{}\oindex{Wien@\textbf{Wien}, \emph{Verwaltungsgebiet}|pwk}Wien, 1. VI. 13, 6\nobreak{}«.  }\toendnotes[C]{\smallbreak}\pstart{}{\pb}XVIII\oindex{XVIII., Währing@\textbf{XVIII., Währing}, \emph{Verwaltungsgebiet}|pw}\pend{}\pstart{}\textsc{Sternwartestr 71}\oindex{Wien@\textbf{Wien}!XVIII., Währing@\textbf{XVIII., Währing}!Sternwartestraße 71@\textbf{Sternwartestraße 71}, \emph{Wohngebäude}|pw}. \pend{}{\bigskip}\pstart{}\textsc{Herrn Gustav Schwarzkopf}\pend{}\pstart{}Baden\oindex{Baden bei Wien@\textbf{Baden bei Wien}, \emph{Hauptstadt}|pw}\pend{}\pstart{}bei Wien\oindex{Wien@\textbf{Wien}, \emph{Verwaltungsgebiet}|pw}\pend{}\pstart{}Pension Villa Quisisana\oindex{Pension Villa Quisisana@\textbf{Pension Villa Quisisana}, \emph{Beherbergungsgebäude}|pw}\pend{}\pstart{}(Helenenthal\oindex{Helenental@\textbf{Helenental}, \emph{Tal}|pw})\pend{}{\bigskip}\vspace{1em}
\pstart
           \raggedleft{}{\pb}1/6 913\pend
           \vspace{0.5em}
\pstart
           lieber Guſtav,{ }\label{K_L04161-1v}\edtext{geſtern iſt Liesl\pwindex{Steinrück, Elisabeth 19.\,11.\,1885 Wien – 7.\,4.\,1920 Partenkirchen@\textsc{Steinrück, Elisabeth} (19.\,11.\,1885 Wien – 7.\,4.\,1920 Partenkirchen)|pw} angeko{\geminationm}en}{\lemma{\textnormal{\emph{gestern … angekommen}}}\Cendnote{\textnormal{Vgl. A. S.: \emph{Tagebuch}, 31. 5. 1913. }}}\label{K_L04161-1} (recht
               gut ausſehend) wohnt nun (Erholungsheim\oindex{Wien@\textbf{Wien}!XVIII., Währing@\textbf{XVIII., Währing}!Erholungsheim Erna Patak@\textbf{Erholungsheim Erna Patak}, \emph{Beherbergungsgebäude}|pw}) nah
               von uns, bleibt bis Ende der Woche; auch wir denken am \label{K_L04161-2v}\edtext{10. zu verreiſen}{\lemma{\textnormal{\emph{10. zu verreisen}}}\Cendnote{\textnormal{ Die Abreise fand bereits am Abend des 9. 6. 1913 statt.
               }}}\label{K_L04161-2}. Ich frage daher an, wie lange Sie in Baden\oindex{Baden bei Wien@\textbf{Baden bei Wien}, \emph{Hauptstadt}|pw} bleiben werden, de{\geminationn}{ }Liesl\pwindex{Steinrück, Elisabeth 19.\,11.\,1885 Wien – 7.\,4.\,1920 Partenkirchen@\textsc{Steinrück, Elisabeth} (19.\,11.\,1885 Wien – 7.\,4.\,1920 Partenkirchen)|pw} möchte
               Sie natürlich ſehr \label{K_L04161-3v}\edtext{gerne ſehen}{\lemma{\textnormal{\emph{gerne sehen}}}\Cendnote{\textnormal{Vgl. A. S.: \emph{Tagebuch}, 8. 6. 1913. }}}\label{K_L04161-3}, u wir
               vor unſrer Abreiſe natürlich auch. Alſo ſchreiben Sie bitte ein Wort, damit wir ein
               Arrangement treffen. Es geht Ihnen hoffentlich gut.\pend
           
\pstart
           Wir grüßen alle herzlichſt{\\[\baselineskip]} Ihr \spacefill\mbox{Arthur}\pend
           \leftskip=0em{}\selectlanguage{ngerman}\endnumbering\briefempfaengerindex{Schwarzkopf, Gustav@\textsc{Schwarzkopf, Gustav}!zzzSchnitzler, Arthur@\emph{von Arthur Schnitzler}!1913-06-011@{1. 6. 1913}|)be}\mylabel{L04161h}
\begin{anhang}
\end{anhang}\newcommand{\dateiname}{L04161}\newcommand{\titel}{Arthur Schnitzler an Gustav Schwarzkopf, 1. 6. 1913}\newcommand{\editorInnen}{Herausgegeben von Jahnke, SelmaMüller, Martin Anton}%% latex-leseansicht-abspann.tex
%% Abspann für die Leseansicht.
%% Der Schalter \ifkorrekturansicht ist bereits durch den Vorspann gesetzt.

%% latex-abspann.tex
%% Gemeinsamer Abspann für Korrekturansicht und Leseansicht.
%% Setzt den Schalter \ifkorrekturansicht voraus (gesetzt in den
%% einbindenden Dateien latex-korrekturansicht-abspann.tex bzw.
%% latex-leseansicht-abspann.tex).
%% ---------------------------------------------------------------

\normalsize

% Das esempio-Environment wird nur in der Leseansicht benötigt
\ifkorrekturansicht\else
\newenvironment{esempio}[3]%
{
    \vspace{1.5ex}
    \rlap{\underline{#1}}
    \par
    \setlength{\parindent}{0cm}
    \nopagebreak
    \leftskip=#2cm
    \rightskip=#3cm
}
{
    \par
}
\fi

\doendnotes{C}
\bigskip
\vfill

\clearpage

\footnotesize

\ifkorrekturansicht
  \lohead{\textsc{register}}
\fi

% theindex-Environment neu definieren ohne reledmac
\makeatletter
\renewenvironment{theindex}{%
  \ifkorrekturansicht
    \section*{\indexname}%
  \else
    \subsubsection*{Index der erwähnten Entitäten}%
  \fi
  \setlength{\parindent}{0pt}%
  \setlength{\parskip}{0pt plus 0.3pt}%
  \let\item\@idxitem
}{%
  \ifkorrekturansicht\clearpage\fi
}
\makeatother

\IfFileExists{\jobname-pw.ind}{\input{\jobname-pw.ind}}{}

% Quellenangabe nur in der Leseansicht
\ifkorrekturansicht\else
% Fallback-Definitionen, falls die .tex-Datei \titel etc. nicht gesetzt hat
\providecommand{\titel}{}
\providecommand{\editorInnen}{}
\providecommand{\dateiname}{\jobname}

\vspace{3cm}

\vfill

\footnotesize
\textsc{Quelle}: \titel. Herausgegeben von {\editorInnen}. In: \emph{Arthur Schnitzler: Briefwechsel mit Autorinnen und Autoren}.
 Digitale Edition, https://schnitzler-briefe.acdh.oeaw.ac.at/{\dateiname}.html (Stand \today)
\fi

\end{document}


