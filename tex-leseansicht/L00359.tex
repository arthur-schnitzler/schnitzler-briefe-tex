%% latex-korrekturansicht-vorspann.tex
%% Vorspann für die Korrekturansicht.
%% Lädt die gemeinsame Datei latex-vorspann.tex mit gesetztem Schalter.

\newif\ifkorrekturansicht
\korrekturansichttrue

\input{../tex-inputs/latex-vorspann}


\section[Arthur Schnitzler an Richard Beer-Hofmann, {[}30. 7. 1894{]}]{L00359 Arthur Schnitzler an Richard Beer-Hofmann, {[}30. 7. 1894{]}}
\nopagebreak\mylabel{L00359v}
\rehead{ }\normalsize\beginnumbering\briefempfaengerindex{Beer-Hofmann, Richard@\textsc{Beer-Hofmann, Richard}!zzzSchnitzler, Arthur@\emph{von Arthur Schnitzler}!1894-07-301@{{[}30. 7. 1894{]}}|(be}
\toendnotes[C]{\smallbreak\pagebreak[2]}\Standort{YCGL, MSS 31.}
\physDesc{Telegramm, 200 Zeichen
\newline{}maschinell
\newline{}Versand: Stempel des Telegrafenbeamten: »{[}30{]}/7 94 W.196 Freinthaller\pwindex{Freinthaller, Hermann @\textsc{Freinthaller, Hermann}, \emph{Postbeamter/Postbeamtin}|pw}« }\pstart{}{\pb}richard beer hofman\pend{}\pstart{}ischl\oindex{Bad Ischl@\textbf{Bad Ischl}, \emph{P.PPL}|pw}{ }egelmoos 22\oindex{Eglmoosgasse@\textbf{Eglmoosgasse}, \emph{Bezirk (A.BZK)}|pw}\pend{}{\bigskip}\vspace{1em}
\pstart
           \noindent{}{\pb}de wien\oindex{Wien@\textbf{Wien}, \emph{A.ADM2}|pw} 72+ 1718
               26 9/50= \pend
           
\pstart
           bitte telegrafiren sie wann sie mit hugo\pwindex{Hofmannsthal, Hugo von 1874-02-01 – 1929-07-15@\textsc{Hofmannsthal, Hugo von} (1874-02-01 – 1929-07-15), \emph{Schriftsteller/Schriftstellerin}|pw}{ }salzburg\oindex{Salzburg@\textbf{Salzburg}, \emph{A.ADM2}|pw} zusammentreffen ich koennte
               hoechstwahrscheinlich schon zwejter august dort sein\pend
           \pstart = herzlich\spacefill\mbox{arthur}\pend{}\selectlanguage{ngerman}\endnumbering\briefempfaengerindex{Beer-Hofmann, Richard@\textsc{Beer-Hofmann, Richard}!zzzSchnitzler, Arthur@\emph{von Arthur Schnitzler}!1894-07-301@{{[}30. 7. 1894{]}}|)be}\mylabel{L00359h}  \normalsize

\doendnotes{C}
\bigskip
\vfill

\clearpage

\footnotesize

\lohead{\textsc{register}}

% Definiere theindex-Environment komplett neu ohne reledmac
\makeatletter
\renewenvironment{theindex}{%
  \section*{\indexname}%
  \setlength{\parindent}{0pt}%
  \setlength{\parskip}{0pt plus 0.3pt}%
  \let\item\@idxitem
}{%
  \clearpage
}
\makeatother

\IfFileExists{\jobname-pw.ind}{\input{\jobname-pw.ind}}{}

\end{document}

      