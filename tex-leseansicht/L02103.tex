%% latex-leseansicht-vorspann.tex
%% Vorspann für die Leseansicht.
%% Lädt die gemeinsame Datei latex-vorspann.tex mit nicht gesetztem Schalter.

\newif\ifkorrekturansicht
\korrekturansichtfalse

\input{../tex-inputs/latex-vorspann}


         
         \renewcommand{\erwaehntePersonen}{Personen: Georg Brandes, Pablo Casals, Hugo von Hofmannsthal, Olga Schnitzler}
         \renewcommand{\erwaehnteOrte}{Orte: Auerbach (Vogtland), Berlin, I., Innere Stadt, Rodaun, Sternwartestraße, Wien, XVIII., Währing}
         \renewcommand{\erwaehnteWerke}{Werke: Professor Bernhardi. Komödie in fünf Akten}
               \section[Hugo von Hofmannsthal an Arthur Schnitzler, 21. 11. 1912]{ Hugo von Hofmannsthal an Arthur Schnitzler, 21. 11. 1912}\nopagebreak\mylabel{v}\rehead{ }\begin{ledgroupsized}[t]{13cm}\normalsize\beginnumbering \toendnotes[C]{\smallbreak\pagebreak[2]} \Standort{CUL, Schnitzler, B 43.}
\physDesc{Postkarte, 445 Zeichen
\newline{}Handschrift: schwarze Tinte, deutsche Kurrent
\newline{}Versand: 1) Rohrpost  2) Stempel: »\nobreak{}\oindex{I., Innere Stadt@\textbf{I., Innere Stadt}|pwk}\textcolor{gray}{1/1} Wien 11, 21 XI 12, XII\nobreak{}«.  3) Stempel: »\nobreak{}\oindex{XVIII., Waehring@\textbf{XVIII., Währing}|pwk}18/1 Wien 111, 21 \textcolor{gray}{X}I 12, XII\textsuperscript{\textcolor{gray}{1}0}\nobreak{}«. 
\newline{}Ordnung: 1) mit Bleistift von unbekannter Hand nummeriert: »\strikeout{382}«  2) mit Bleistift von unbekannter Hand nummeriert:
                                    »343«}\buchAbdrucke{\weitereDrucke{Hugo von Hofmannsthal, Arthur Schnitzler: \emph{Briefwechsel}. Hg. Therese Nickl und Heinrich Schnitzler. Frankfurt am Main: \emph{S. Fischer} 1964, S. 270.} }\toendnotes[C]{\smallbreak}\pstart{}{\pb}Hofmannsthal\pend{}{\bigskip}\pstart{}\textsc{Herrn D\textsuperscript{r} Arthur Schnitzler}\pend{}\pstart{}\textsc{Wien}\oindex{Wien@\textbf{Wien}|pw}\pend{}\pstart{}XVIII Sternwartestrasse 71\oindex{XXXX Ortsangabe fehlt|pw}\pend{}{\bigskip}\pstart
           \noindent{}{\pb}lieber, erwartete i{\geminationm}er ein Wort \strikeout{fin} von Ihnen!\hspace*{1.5em}Nun
                  \label{K_L02103-1v}\edtext{Freitag}{\lemma{\textnormal{\emph{Freitag}}}\Cendnote{\textnormal{Schnitzler\pwindex{Schnitzler, Arthur 15.05.1862 – 21.10.1931@\textsc{Schnitzler, Arthur} (15.05.1862 – 21.10.1931), \emph{Schriftsteller, Mediziner}|pwk} dürfte ihn zu einem Abend
                  anlässlich des Besuchs von Georg Brandes\pwindex{Brandes, Georg 04.02.1842 – 19.02.1927@\textsc{Brandes, Georg} (04.02.1842 – 19.02.1927)|pwk}
                  geladen haben.}}}\label{K_L02103-1h} gerade haben wir Plätze zu \textsc{Casals}\pwindex{Casals, Pablo 29.12.1876 – 22.10.1973@\textsc{Casals, Pablo} (29.12.1876 – 22.10.1973), \emph{Cellist}|pw}. Das iſt eine Muſik die mir ſo viel Freude macht, daſs ich die Plätze wirklich
               nicht aufgeben möchte. Alſo dann auf Wiederſehen nach dem \label{K_L02103-2v}\edtext{12\textsuperscript{ten} December}{\lemma{\textnormal{\emph{12\textsuperscript{ten} December}}}\Cendnote{\textnormal{Schnitzler\pwindex{Schnitzler, Arthur 15.05.1862 – 21.10.1931@\textsc{Schnitzler, Arthur} (15.05.1862 – 21.10.1931), \emph{Schriftsteller, Mediziner}|pwk} war vom 23. 11. 1912 bis zum 2. 12. 1912 in Berlin\oindex{Berlin@\textbf{Berlin}|pwk}, wo die Uraufführung von \emph{Professor Bernhardi}\pwindex{Schnitzler, Arthur 15.05.1862 – 21.10.1931@\textsc{Schnitzler, Arthur} (15.05.1862 – 21.10.1931), \emph{Schriftsteller, Mediziner}!Professor Bernhardi. Komoedie in fuenf Akten1912@\strich\emph{Professor Bernhardi. Komödie in fünf Akten} {[}1912{]}|pwk} stattfand. Hofmannsthal\pwindex{Hofmannsthal, Hugo von 1874-02-01 – 1929-07-15@\textsc{Hofmannsthal, Hugo von} (1874-02-01 – 1929-07-15), \emph{Schriftsteller}|pwk} reiste am 30. 11. 1912 nach Auerbach (Vogtland)\oindex{Auerbach (Vogtland)@\textbf{Auerbach (Vogtland)}|pwk} und in Folge an mehrere
                  deutsche Orte. In Berlin\oindex{Berlin@\textbf{Berlin}|pwk} war er zwischen
                     6. 12. 1912 und 12. 12. 1912. Er kehrte am
                     15. 12. 1912 nach Rodaun\oindex{Rodaun@\textbf{Rodaun}|pwk}
                  zurück.}}}\label{K_L02103-2h}! Es wird wohl die längſte Pauſe in unſerem bisherigen Verkehr
               geweſen sein! Vielleicht bin ich zur Première\pwindex{Schnitzler, Arthur 15.05.1862 – 21.10.1931@\textsc{Schnitzler, Arthur} (15.05.1862 – 21.10.1931), \emph{Schriftsteller, Mediziner}!Professor Bernhardi. Komoedie in fuenf Akten1912@\strich\emph{Professor Bernhardi. Komödie in fünf Akten} {[}1912{]}|pwv} in Berlin\oindex{Berlin@\textbf{Berlin}|pw}! \pend
           \pstart Alles Gute an Olga\pwindex{Schnitzler, Olga 17.01.1882 – 13.01.1970@\textsc{Schnitzler, Olga} (17.01.1882 – 13.01.1970), \emph{Schauspielerin, Sängerin}|pw}.
                  Ihr\spacefill\mbox{Hugo}\pend{}
         
         \endnumbering\mylabel{h}\end{ledgroupsized}  \newcommand{\dateiname}{L02103}\newcommand{\titel}{Hugo von Hofmannsthal an Arthur Schnitzler, 21. 11. 1912}\newcommand{\editorInnen}{Martin Anton Müller und Gerd-Hermann Susen}%% latex-leseansicht-abspann.tex
%% Abspann für die Leseansicht.
%% Der Schalter \ifkorrekturansicht ist bereits durch den Vorspann gesetzt.

%% latex-abspann.tex
%% Gemeinsamer Abspann für Korrekturansicht und Leseansicht.
%% Setzt den Schalter \ifkorrekturansicht voraus (gesetzt in den
%% einbindenden Dateien latex-korrekturansicht-abspann.tex bzw.
%% latex-leseansicht-abspann.tex).
%% ---------------------------------------------------------------

\normalsize

% Das esempio-Environment wird nur in der Leseansicht benötigt
\ifkorrekturansicht\else
\newenvironment{esempio}[3]%
{
    \vspace{1.5ex}
    \rlap{\underline{#1}}
    \par
    \setlength{\parindent}{0cm}
    \nopagebreak
    \leftskip=#2cm
    \rightskip=#3cm
}
{
    \par
}
\fi

\doendnotes{C}
\bigskip
\vfill

\clearpage

\footnotesize

\ifkorrekturansicht
  \lohead{\textsc{register}}
\fi

% theindex-Environment neu definieren ohne reledmac
\makeatletter
\renewenvironment{theindex}{%
  \ifkorrekturansicht
    \section*{\indexname}%
  \else
    \subsubsection*{Index der erwähnten Entitäten}%
  \fi
  \setlength{\parindent}{0pt}%
  \setlength{\parskip}{0pt plus 0.3pt}%
  \let\item\@idxitem
}{%
  \ifkorrekturansicht\clearpage\fi
}
\makeatother

\IfFileExists{\jobname-pw.ind}{\input{\jobname-pw.ind}}{}

% Quellenangabe nur in der Leseansicht
\ifkorrekturansicht\else
% Fallback-Definitionen, falls die .tex-Datei \titel etc. nicht gesetzt hat
\providecommand{\titel}{}
\providecommand{\editorInnen}{}
\providecommand{\dateiname}{\jobname}

\vspace{3cm}

\vfill

\footnotesize
\textsc{Quelle}: \titel. Herausgegeben von {\editorInnen}. In: \emph{Arthur Schnitzler: Briefwechsel mit Autorinnen und Autoren}.
 Digitale Edition, https://schnitzler-briefe.acdh.oeaw.ac.at/{\dateiname}.html (Stand \today)
\fi

\end{document}


      