%% latex-leseansicht-vorspann.tex
%% Vorspann für die Leseansicht.
%% Lädt die gemeinsame Datei latex-vorspann.tex mit nicht gesetztem Schalter.

\newif\ifkorrekturansicht
\korrekturansichtfalse

\input{../tex-inputs/latex-vorspann}


\section[Hugo von Hofmannsthal an Arthur Schnitzler, 21.\,11.\,1912]{L02103 Hugo von Hofmannsthal an Arthur Schnitzler, 21.\,11.\,1912}
\nopagebreak\mylabel{L02103v}
\rehead{ }\normalsize\beginnumbering\briefempfaengerindex{Schnitzler, Arthur@\textsc{Schnitzler, Arthur}!zzzHofmannsthal, Hugo von@\emph{von Hugo von Hofmannsthal}!1912-11-211@{21. 11. 1912}|(be}
\toendnotes[C]{\smallbreak\pagebreak[2]}
\correspDesc{Versand  durch Hugo von Hofmannsthal am 21. 11. 1912 in Wien
\newline{}Erhalt  durch Arthur Schnitzler am 21. 11. 1912 in Wien}\toendnotes[C]{\smallbreak}
\Standort{CUL, Schnitzler, B 43.}
\physDesc{Postkarte, 445 Zeichen
\newline{}Handschrift: schwarze Tinte, deutsche Kurrent
\newline{}Versand: 1) Rohrpost  2) Stempel: »\nobreak{}\oindex{I., Innere Stadt@\textbf{I., Innere Stadt}, \emph{Verwaltungsgebiet}|pwk}\textcolor{gray}{1/1} Wien 11, 21 XI 12, XII\nobreak{}«.  3) Stempel: »\nobreak{}\oindex{XVIII., Währing@\textbf{XVIII., Währing}, \emph{Verwaltungsgebiet}|pwk}18/1 Wien 111, 21 \textcolor{gray}{X}I 12, XII\textsuperscript{\textcolor{gray}{1}0}\nobreak{}«. 
\newline{}Ordnung: 1) mit Bleistift von unbekannter Hand nummeriert: »\strikeout{382}«  2) mit Bleistift von unbekannter Hand nummeriert:
                                    »343«}
\buchAbdrucke{\weitereDrucke{Hugo von Hofmannsthal, Arthur Schnitzler: \emph{Briefwechsel}. Herausgegeben von Therese Nickl und Heinrich Schnitzler. Frankfurt am Main: \emph{S. Fischer} 1964, S. 270.} }\toendnotes[C]{\smallbreak}\pstart{}{\pb}Hofmannsthal\pend{}{\bigskip}\pstart{}\textsc{Herrn D\textsuperscript{r} Arthur Schnitzler}\pend{}\pstart{}\textsc{Wien}\oindex{Wien@\textbf{Wien}, \emph{Verwaltungsgebiet}|pw}\pend{}\pstart{}XVIII Sternwartestrasse 71\oindex{Wien@\textbf{Wien}!XVIII., Währing@\textbf{XVIII., Währing}!Sternwartestraße 71@\textbf{Sternwartestraße 71}, \emph{Wohngebäude}|pw}\pend{}{\bigskip}\vspace{1em}
\pstart
           \noindent{}{\pb}lieber, erwartete i{\geminationm}er ein Wort \strikeout{fin} von Ihnen!\hspace*{1.5em}Nun
                  \label{K_L02103-1v}\edtext{Freitag}{\lemma{\textnormal{\emph{Freitag}}}\Cendnote{\textnormal{Schnitzler dürfte ihn zu einem Abend
                  anlässlich des Besuchs von Georg Brandes\pwindex{Brandes, Georg 4.\,2.\,1842 Kopenhagen – 19.\,2.\,1927 ebd.@\textsc{Brandes, Georg} (4.\,2.\,1842 Kopenhagen – 19.\,2.\,1927 ebd.)|pwk}
                  geladen haben.}}}\label{K_L02103-1} gerade haben wir Plätze zu \textsc{Casals}\pwindex{Casals, Pablo 29.\,12.\,1876 El Vendrell – 22.\,10.\,1973 San Juan@\textsc{Casals, Pablo} (29.\,12.\,1876 El Vendrell – 22.\,10.\,1973 San Juan), \emph{Cellist}|pw}. Das iſt eine Muſik die mir{ }ſo viel Freude macht, daſs ich die Plätze wirklich
               nicht aufgeben möchte. Alſo dann auf Wiederſehen nach dem \label{K_L02103-2v}\edtext{12\textsuperscript{ten} December}{\lemma{\textnormal{\emph{12\textsuperscript{ten} December}}}\Cendnote{\textnormal{Schnitzler war vom 23. 11. 1912 bis zum 2. 12. 1912 in Berlin\oindex{Berlin@\textbf{Berlin}, \emph{Hauptstadt}|pwk}, wo die Uraufführung\eventindex{Kleines Theater@\textbf{Kleines Theater}!Uraufführung von Professor Bernhardi, 28.11.1912@Uraufführung von Professor Bernhardi, 28.11.1912|pwkv} von \emph{Professor Bernhardi}\pwindex{Schnitzler, Arthur 15.\,5.\,1862 Wien – 21.\,10.\,1931 ebd.@\textsc{Schnitzler, Arthur} (15.\,5.\,1862 Wien – 21.\,10.\,1931 ebd.), \emph{Schriftsteller, Mediziner}!Professor Bernhardi. Komödie in fünf Akten@\strich\emph{Professor Bernhardi. Komödie in fünf Akten}|pwk} stattfand. Hofmannsthal\pwindex{Hofmannsthal, Hugo von 1.\,2.\,1874 Wien – 15.\,7.\,1929 Rodaun@\textsc{Hofmannsthal, Hugo von} (1.\,2.\,1874 Wien – 15.\,7.\,1929 Rodaun), \emph{Schriftsteller}|pwk} reiste am 30. 11. 1912 nach Auerbach (Vogtland)\oindex{Auerbach (Vogtland)@\textbf{Auerbach (Vogtland)}|pwk} und in Folge an mehrere
                  deutsche Orte. In Berlin\oindex{Berlin@\textbf{Berlin}, \emph{Hauptstadt}|pwk} war er zwischen
                     6. 12. 1912 und 12. 12. 1912. Er kehrte am
                     15. 12. 1912 nach Rodaun\oindex{Wien@\textbf{Wien}!XXIII., Liesing@\textbf{XXIII., Liesing}!Rodaun@\textbf{Rodaun}, \emph{Region}|pwk}
                  zurück.}}}\label{K_L02103-2}! Es wird wohl die längſte Pauſe in unſerem bisherigen Verkehr
                geweſen sein! Vielleicht bin ich zur Première\eventindex{Kleines Theater@\textbf{Kleines Theater}!Uraufführung von Professor Bernhardi, 28.11.1912@Uraufführung von Professor Bernhardi, 28.11.1912|pwv}\pwindex{Schnitzler, Arthur 15.\,5.\,1862 Wien – 21.\,10.\,1931 ebd.@\textsc{Schnitzler, Arthur} (15.\,5.\,1862 Wien – 21.\,10.\,1931 ebd.), \emph{Schriftsteller, Mediziner}!Professor Bernhardi. Komödie in fünf Akten@\strich\emph{Professor Bernhardi. Komödie in fünf Akten}|pwv} in Berlin\oindex{Berlin@\textbf{Berlin}, \emph{Hauptstadt}|pw}!\pend
           \pstart Alles Gute an Olga\pwindex{Schnitzler, Olga 17.\,1.\,1882 Wien – 13.\,1.\,1970 Lugano@\textsc{Schnitzler, Olga} (17.\,1.\,1882 Wien – 13.\,1.\,1970 Lugano), \emph{Schauspielerin, Sängerin}|pw}.
                  Ihr\spacefill\mbox{Hugo}\pend{}\selectlanguage{ngerman}\endnumbering\briefempfaengerindex{Schnitzler, Arthur@\textsc{Schnitzler, Arthur}!zzzHofmannsthal, Hugo von@\emph{von Hugo von Hofmannsthal}!1912-11-211@{21. 11. 1912}|)be}\mylabel{L02103h}  \newcommand{\dateiname}{L02103}\newcommand{\titel}{Hugo von Hofmannsthal an Arthur Schnitzler, 21. 11. 1912}\newcommand{\editorInnen}{Martin Anton Müller und Gerd-Hermann Susen}%% latex-leseansicht-abspann.tex
%% Abspann für die Leseansicht.
%% Der Schalter \ifkorrekturansicht ist bereits durch den Vorspann gesetzt.

%% latex-abspann.tex
%% Gemeinsamer Abspann für Korrekturansicht und Leseansicht.
%% Setzt den Schalter \ifkorrekturansicht voraus (gesetzt in den
%% einbindenden Dateien latex-korrekturansicht-abspann.tex bzw.
%% latex-leseansicht-abspann.tex).
%% ---------------------------------------------------------------

\normalsize

% Das esempio-Environment wird nur in der Leseansicht benötigt
\ifkorrekturansicht\else
\newenvironment{esempio}[3]%
{
    \vspace{1.5ex}
    \rlap{\underline{#1}}
    \par
    \setlength{\parindent}{0cm}
    \nopagebreak
    \leftskip=#2cm
    \rightskip=#3cm
}
{
    \par
}
\fi

\doendnotes{C}
\bigskip
\vfill

\clearpage

\footnotesize

\ifkorrekturansicht
  \lohead{\textsc{register}}
\fi

% theindex-Environment neu definieren ohne reledmac
\makeatletter
\renewenvironment{theindex}{%
  \ifkorrekturansicht
    \section*{\indexname}%
  \else
    \subsubsection*{Index der erwähnten Entitäten}%
  \fi
  \setlength{\parindent}{0pt}%
  \setlength{\parskip}{0pt plus 0.3pt}%
  \let\item\@idxitem
}{%
  \ifkorrekturansicht\clearpage\fi
}
\makeatother

\IfFileExists{\jobname-pw.ind}{\input{\jobname-pw.ind}}{}

% Quellenangabe nur in der Leseansicht
\ifkorrekturansicht\else
% Fallback-Definitionen, falls die .tex-Datei \titel etc. nicht gesetzt hat
\providecommand{\titel}{}
\providecommand{\editorInnen}{}
\providecommand{\dateiname}{\jobname}

\vspace{3cm}

\vfill

\footnotesize
\textsc{Quelle}: \titel. Herausgegeben von {\editorInnen}. In: \emph{Arthur Schnitzler: Briefwechsel mit Autorinnen und Autoren}.
 Digitale Edition, https://schnitzler-briefe.acdh.oeaw.ac.at/{\dateiname}.html (Stand \today)
\fi

\end{document}


