%% latex-leseansicht-vorspann.tex
%% Vorspann für die Leseansicht.
%% Lädt die gemeinsame Datei latex-vorspann.tex mit nicht gesetztem Schalter.

\newif\ifkorrekturansicht
\korrekturansichtfalse

\input{../tex-inputs/latex-vorspann}


\section[ Felix Salten an Arthur Schnitzler, 6. 6. 1930]{L03591 Felix Salten an Arthur Schnitzler,  6. 6. 1930}
\nopagebreak\mylabel{L03591v}
\rehead{ }\normalsize\beginnumbering\briefempfaengerindex{Schnitzler, Arthur@\textsc{Schnitzler, Arthur}!zzzSalten, Felix@\emph{von Felix Salten}!1930-06-061@{6. 6. 1930}|(be}
\toendnotes[C]{\smallbreak\pagebreak[2]}
\correspDesc{Versand  durch Felix Salten am 6. 6. 1930 in Grand Canyon Village
\newline{}Übermittlung  am 7. 6. 1930 in Grand Canyon
\newline{}Erhalt  durch Arthur Schnitzler im Zeitraum [22. 6. 1930
                  – 5. 7. 1930?] in Wien}\toendnotes[C]{\smallbreak}
\Standort{CUL, Schnitzler, B 89, B 2.}
\physDesc{Bildpostkarte, 135 Zeichen
\newline{}Handschrift: blaue Tinte, lateinische Kurrent
\newline{}Versand: Stempel: »\nobreak{}\oindex{Grand Canyon@\textbf{Grand Canyon}|pwk}Grand Canyon Ariz., Jun 7 1930, 8 PM\nobreak{}«.  
\newline{}Ordnung: mit Bleistift von unbekannter Hand nummeriert: »305« }\toendnotes[C]{\smallbreak}\pstart{}{\pb}Europe\oindex{Europa@\textbf{Europa}|pw}\pend{}\pstart{}Austria\oindex{Österreich@\textbf{Österreich}|pw}\pend{}\pstart{}Herrn D\textsuperscript{r} Arthur Schnitzler\pend{}\pstart{}Wien\oindex{Wien@\textbf{Wien}, \emph{Verwaltungsgebiet}|pw}\pend{}\pstart{}18. Sternwartestrasse 71\oindex{Wien@\textbf{Wien}!XVIII., Währing@\textbf{XVIII., Währing}!Sternwartestraße 71@\textbf{Sternwartestraße 71}, \emph{Wohngebäude}|pw}\pend{}{\bigskip}
\pstart
           \noindent{}\centering{}{\pb}\textcolor{gray}{\textbf{H-3959 \begin{otherlanguage}{english}ON KAIBAB TRAIL\oindex{Old Kaibab Trail@\textbf{Old Kaibab Trail}|pw} TO THE NORTH RIM\oindex{Grand Canyon@\textbf{Grand Canyon}|pwv}\end{otherlanguage},}}\pend
           
\pstart
           \centering{}\textcolor{gray}{\textbf{GRAND CANYON NATIONAL PARK\oindex{Grand Canyon@\textbf{Grand Canyon}|pw}, ARIZONA\oindex{Arizona@\textbf{Arizona}, \emph{Land}|pw}.}}\pend
           
\pstart
           {\pb}\textcolor{gray}{\textbf{\begin{otherlanguage}{english}Kaibab Trail\oindex{Old Kaibab Trail@\textbf{Old Kaibab Trail}|pw} to North Rim\oindex{Grand Canyon@\textbf{Grand Canyon}|pwv} of Grand Canyon\oindex{Grand Canyon@\textbf{Grand Canyon}|pw} for several miles follows
                        along the steep walls bordering Bright
                           Angel Creek\oindex{Bright Angel Creek@\textbf{Bright Angel Creek}, \emph{Fluss}|pw}, a clear water mountain stream\end{otherlanguage}.}}\pend
           \vspace{1em}
\pstart
           \raggedleft{}{\pb}El Tovar\oindex{El Tovar Hotel@\textbf{El Tovar Hotel}, \emph{Hotel}|pw}, 6. 6. 30\pend
           \vspace{0.5em}
\pstart
           Fremde Landschaft – Heimweh!\pend
           
\pstart
           Herzlich Ihr {\\[\baselineskip]}\spacefill\mbox{Felix Salten}\pend
           \leftskip=0em{}\selectlanguage{ngerman}\endnumbering\briefempfaengerindex{Schnitzler, Arthur@\textsc{Schnitzler, Arthur}!zzzSalten, Felix@\emph{von Felix Salten}!1930-06-061@{6. 6. 1930}|)be}\mylabel{L03591h}  \newcommand{\dateiname}{L03591}\newcommand{\titel}{Felix Salten an Arthur Schnitzler, 6. 6. 1930}\newcommand{\editorInnen}{Martin Anton Müller und Laura Untner}%% latex-leseansicht-abspann.tex
%% Abspann für die Leseansicht.
%% Der Schalter \ifkorrekturansicht ist bereits durch den Vorspann gesetzt.

%% latex-abspann.tex
%% Gemeinsamer Abspann für Korrekturansicht und Leseansicht.
%% Setzt den Schalter \ifkorrekturansicht voraus (gesetzt in den
%% einbindenden Dateien latex-korrekturansicht-abspann.tex bzw.
%% latex-leseansicht-abspann.tex).
%% ---------------------------------------------------------------

\normalsize

% Das esempio-Environment wird nur in der Leseansicht benötigt
\ifkorrekturansicht\else
\newenvironment{esempio}[3]%
{
    \vspace{1.5ex}
    \rlap{\underline{#1}}
    \par
    \setlength{\parindent}{0cm}
    \nopagebreak
    \leftskip=#2cm
    \rightskip=#3cm
}
{
    \par
}
\fi

\doendnotes{C}
\bigskip
\vfill

\clearpage

\footnotesize

\ifkorrekturansicht
  \lohead{\textsc{register}}
\fi

% theindex-Environment neu definieren ohne reledmac
\makeatletter
\renewenvironment{theindex}{%
  \ifkorrekturansicht
    \section*{\indexname}%
  \else
    \subsubsection*{Index der erwähnten Entitäten}%
  \fi
  \setlength{\parindent}{0pt}%
  \setlength{\parskip}{0pt plus 0.3pt}%
  \let\item\@idxitem
}{%
  \ifkorrekturansicht\clearpage\fi
}
\makeatother

\IfFileExists{\jobname-pw.ind}{\input{\jobname-pw.ind}}{}

% Quellenangabe nur in der Leseansicht
\ifkorrekturansicht\else
% Fallback-Definitionen, falls die .tex-Datei \titel etc. nicht gesetzt hat
\providecommand{\titel}{}
\providecommand{\editorInnen}{}
\providecommand{\dateiname}{\jobname}

\vspace{3cm}

\vfill

\footnotesize
\textsc{Quelle}: \titel. Herausgegeben von {\editorInnen}. In: \emph{Arthur Schnitzler: Briefwechsel mit Autorinnen und Autoren}.
 Digitale Edition, https://schnitzler-briefe.acdh.oeaw.ac.at/{\dateiname}.html (Stand \today)
\fi

\end{document}


