%% latex-leseansicht-vorspann.tex
%% Vorspann für die Leseansicht.
%% Lädt die gemeinsame Datei latex-vorspann.tex mit nicht gesetztem Schalter.

\newif\ifkorrekturansicht
\korrekturansichtfalse

\input{../tex-inputs/latex-vorspann}


\section[Arthur Schnitzler an Richard Beer-Hofmann, 31. 1. 1893]{L00167 Arthur Schnitzler an Richard Beer-Hofmann, 31. 1. 1893}
\nopagebreak\mylabel{L00167v}
\rehead{ }\normalsize\beginnumbering\briefempfaengerindex{Beer-Hofmann, Richard@\textsc{Beer-Hofmann, Richard}!zzzSchnitzler, Arthur@\emph{von Arthur Schnitzler}!1893-01-312@{31. 1. 1893}|(be}
\toendnotes[C]{\smallbreak\pagebreak[2]}
\correspDesc{Versand  durch Arthur Schnitzler am 31. 1. 1893 in Wien
\newline{}Erhalt  durch Richard Beer-Hofmann im Zeitraum [31. 1. 1893
                  – 4. 2. 1893?] in Wien}\toendnotes[C]{\smallbreak}
\Standort{YCGL, MSS 31.}
\physDesc{Briefkarte, , Kuvert, 372 Zeichen
\newline{}Handschrift: Bleistift, deutsche Kurrent
\newline{}Versand: ohne postalischen Übermittlungsvermerk }
\buchAbdrucke{\weitereDrucke{Arthur Schnitzler, Richard Beer-Hofmann: \emph{Briefwechsel 1891–1931}. Herausgegeben von Konstanze Fliedl. Wien, Zürich: \emph{Europaverlag} 1992, S. 41.} }\toendnotes[C]{\smallbreak}\pstart{}{\pb}\textsc{Hrn Dr. Rich. Beer H}\damage{ofmann}\pend{}\pstart{}\textsc{Wollzeile 15}\oindex{Wien@\textbf{Wien}!I., Innere Stadt@\textbf{I., Innere Stadt}!Wollzeile 15 (»Berthahof«)@\textbf{Wollzeile 15 (»Berthahof«)}, \emph{Wohngebäude}|pw}\pend{}\pstart{}{\pb}Dſtm. bez.\pend{}{\bigskip}\vspace{1em}
\pstart
           {\pb}\strikeout{\textcolor{gray}{\textbf{PROFESSOR{ }SCHNITZLER\pwindex{Schnitzler, Johann 10.\,4.\,1835 Nagykanizsa – 2.\,5.\,1893 Wien@\textsc{Schnitzler, Johann} (10.\,4.\,1835 Nagykanizsa – 2.\,5.\,1893 Wien), \emph{Laryngologe}|pw}}}}\hfill \textsc{\uline{I Grillparzerstraße 7}}\oindex{Wien@\textbf{Wien}!I., Innere Stadt@\textbf{I., Innere Stadt}!Grillparzerstraße@\textbf{Grillparzerstraße}, \emph{Straße}|pw}.\pend
           \vspace{0.5em}
\pstart
           Lieber Richard! Voilà – aber was?! Sie \uuline{vergaßen} mir die Karte zu{ }ſenden!! Bitte entweder um Aufklärg oder um die
               Karte! Ja? {\pb}Dem Löbl\pwindex{Löbl, Emil 5.\,2.\,1863 Wien – 26.\,8.\,1942 ebd.@\textsc{Löbl, Emil} (5.\,2.\,1863 Wien – 26.\,8.\,1942 ebd.), \emph{Schriftsteller, Journalist}|pw} hab ich um eine Redoutekarte geſchrieben. Sollt ich{ }ſie kriegen,{ }ſo geh
               ich! Sie erfahrens rechtzeitig! Vorher \substVorne{}\textsuperscript{\textcolor{gray}{bitt}}\substDazwischen{}geh\substHinten{} ich \substVorne{}\textsuperscript{eine}\substDazwischen{}zu\substHinten{}{ }Mongodin\pwindex{\textcolor{red}{\textsuperscript{XXXX indx1}}!Madame Mongodin. Schwank in drei Akten@\strich\emph{Madame Mongodin. Schwank in drei Akten}|pw}\pwindex{\textcolor{red}{\textsuperscript{XXXX indx1}}!Madame Mongodin. Schwank in drei Akten@\strich\emph{Madame Mongodin. Schwank in drei Akten}|pw}\pend
           
\pstart
           – Alſo bitte die Karte!\pend
           
\pstart
           Herzlich{\\[\baselineskip]}\label{T_L00167-1v}\edtext{Ihr \spacefill\mbox{Arthur}}{\lemma{\textnormal{\emph{Ihr Arthur}}}\Cendnote{\textnormal{am Papier links von
                     »Herzlich«, aber durch den Bleistiftdruck als zwei Schreibakte
                  zu erkennen}}}\label{T_L00167-1}\pend
           \leftskip=0em{}\selectlanguage{ngerman}\endnumbering\briefempfaengerindex{Beer-Hofmann, Richard@\textsc{Beer-Hofmann, Richard}!zzzSchnitzler, Arthur@\emph{von Arthur Schnitzler}!1893-01-312@{31. 1. 1893}|)be}\mylabel{L00167h}  \newcommand{\dateiname}{L00167}\newcommand{\titel}{Arthur Schnitzler an Richard Beer-Hofmann, 31. 1. 1893}\newcommand{\editorInnen}{Martin Anton Müller und Gerd-Hermann Susen}%% latex-leseansicht-abspann.tex
%% Abspann für die Leseansicht.
%% Der Schalter \ifkorrekturansicht ist bereits durch den Vorspann gesetzt.

%% latex-abspann.tex
%% Gemeinsamer Abspann für Korrekturansicht und Leseansicht.
%% Setzt den Schalter \ifkorrekturansicht voraus (gesetzt in den
%% einbindenden Dateien latex-korrekturansicht-abspann.tex bzw.
%% latex-leseansicht-abspann.tex).
%% ---------------------------------------------------------------

\normalsize

% Das esempio-Environment wird nur in der Leseansicht benötigt
\ifkorrekturansicht\else
\newenvironment{esempio}[3]%
{
    \vspace{1.5ex}
    \rlap{\underline{#1}}
    \par
    \setlength{\parindent}{0cm}
    \nopagebreak
    \leftskip=#2cm
    \rightskip=#3cm
}
{
    \par
}
\fi

\doendnotes{C}
\bigskip
\vfill

\clearpage

\footnotesize

\ifkorrekturansicht
  \lohead{\textsc{register}}
\fi

% theindex-Environment neu definieren ohne reledmac
\makeatletter
\renewenvironment{theindex}{%
  \ifkorrekturansicht
    \section*{\indexname}%
  \else
    \subsubsection*{Index der erwähnten Entitäten}%
  \fi
  \setlength{\parindent}{0pt}%
  \setlength{\parskip}{0pt plus 0.3pt}%
  \let\item\@idxitem
}{%
  \ifkorrekturansicht\clearpage\fi
}
\makeatother

\IfFileExists{\jobname-pw.ind}{\input{\jobname-pw.ind}}{}

% Quellenangabe nur in der Leseansicht
\ifkorrekturansicht\else
% Fallback-Definitionen, falls die .tex-Datei \titel etc. nicht gesetzt hat
\providecommand{\titel}{}
\providecommand{\editorInnen}{}
\providecommand{\dateiname}{\jobname}

\vspace{3cm}

\vfill

\footnotesize
\textsc{Quelle}: \titel. Herausgegeben von {\editorInnen}. In: \emph{Arthur Schnitzler: Briefwechsel mit Autorinnen und Autoren}.
 Digitale Edition, https://schnitzler-briefe.acdh.oeaw.ac.at/{\dateiname}.html (Stand \today)
\fi

\end{document}


