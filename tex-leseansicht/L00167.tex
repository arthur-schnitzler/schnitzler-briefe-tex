%% latex-korrekturansicht-vorspann.tex
%% Vorspann für die Korrekturansicht.
%% Lädt die gemeinsame Datei latex-vorspann.tex mit gesetztem Schalter.

\newif\ifkorrekturansicht
\korrekturansichttrue

\input{../tex-inputs/latex-vorspann}


\section[Arthur Schnitzler an Richard Beer-Hofmann, 31. 1. 1893]{L00167 Arthur Schnitzler an Richard Beer-Hofmann, 31. 1. 1893}
\nopagebreak\mylabel{L00167v}
\rehead{ }\normalsize\beginnumbering\briefempfaengerindex{Beer-Hofmann, Richard@\textsc{Beer-Hofmann, Richard}!zzzSchnitzler, Arthur@\emph{von Arthur Schnitzler}!1893-01-312@{31. 1. 1893}|(be}
\toendnotes[C]{\smallbreak\pagebreak[2]}\Standort{YCGL, MSS 31.}
\physDesc{Briefkarte, , Umschlag, 372 Zeichen
\newline{}Handschrift: Bleistift, deutsche Kurrent
\newline{}Versand: ohne postalischen Übermittlungsvermerk }
\buchAbdrucke{\weitereDrucke{Arthur Schnitzler, Richard Beer-Hofmann: \emph{Briefwechsel 1891–1931}. Wien, Zürich: \emph{Europaverlag} 1992, S. 41.} }\toendnotes[C]{\smallbreak}\pstart{}{\pb}\textsc{Hrn Dr. Rich. Beer H}\damage{ofmann}\pend{}\pstart{}\textsc{Wollzeile 15}\oindex{Wollzeile@\textbf{Wollzeile}, \emph{Straße (K.STR)}|pw}\pend{}\pstart{}{\pb}Dſtm. bez.\pend{}{\bigskip}\vspace{1em}
\pstart
           {\pb}\strikeout{\textcolor{gray}{\textbf{PROFESSOR{ }SCHNITZLER\pwindex{Schnitzler, Johann 10.04.1835 – 02.05.1893@\textsc{Schnitzler, Johann} (10.04.1835 – 02.05.1893), \emph{Laryngologe/Laryngologin}|pw}}}}\hfill \textsc{\uline{I Grillparzerstraße 7}}\oindex{Grillparzerstrasse@\textbf{Grillparzerstraße}, \emph{R.ST}|pw}. \pend
           \vspace{0.5em}
\pstart
           Lieber Richard! Voilà – aber was?! Sie \uuline{vergaßen} mir die Karte zu ſenden!! Bitte entweder um Aufklärg oder um die
               Karte! Ja? {\pb}Dem Löbl\pwindex{Loebl, Emil 05.02.1863 – 26.08.1942@\textsc{Löbl, Emil} (05.02.1863 – 26.08.1942), \emph{Schriftsteller/Schriftstellerin, Journalist/Journalistin}|pw} hab ich um eine Redoutekarte geſchrieben. Sollt ich ſie kriegen, ſo geh
               ich! Sie erfahrens rechtzeitig! Vorher \substVorne{}\textsuperscript{\textcolor{gray}{bitt}}\substDazwischen{}geh\substHinten{} ich \substVorne{}\textsuperscript{eine}\substDazwischen{}zu\substHinten{}{ }Mongodin\pwindex{Madame Mongodin. Schwank in drei Akten@\emph{Madame Mongodin. Schwank in drei Akten}|pw}\pend
           
\pstart
           – Alſo bitte die Karte!\pend
           
\pstart
           Herzlich{\\[\baselineskip]}\label{T_L00167-1v}\edtext{Ihr \spacefill\mbox{Arthur}}{\lemma{\textnormal{\emph{Ihr Arthur}}}\Cendnote{\textnormal{am Papier links von
                     »Herzlich«, aber durch den Bleistiftdruck als zwei Schreibakte
                  zu erkennen}}}\label{T_L00167-1}\pend
           \leftskip=0em{}\selectlanguage{ngerman}\endnumbering\briefempfaengerindex{Beer-Hofmann, Richard@\textsc{Beer-Hofmann, Richard}!zzzSchnitzler, Arthur@\emph{von Arthur Schnitzler}!1893-01-312@{31. 1. 1893}|)be}\mylabel{L00167h}  \normalsize

\doendnotes{C}
\bigskip
\vfill

\clearpage

\footnotesize

\lohead{\textsc{register}}

% Definiere theindex-Environment komplett neu ohne reledmac
\makeatletter
\renewenvironment{theindex}{%
  \section*{\indexname}%
  \setlength{\parindent}{0pt}%
  \setlength{\parskip}{0pt plus 0.3pt}%
  \let\item\@idxitem
}{%
  \clearpage
}
\makeatother

\IfFileExists{\jobname-pw.ind}{\input{\jobname-pw.ind}}{}

\end{document}

      