%% latex-leseansicht-vorspann.tex
%% Vorspann für die Leseansicht.
%% Lädt die gemeinsame Datei latex-vorspann.tex mit nicht gesetztem Schalter.

\newif\ifkorrekturansicht
\korrekturansichtfalse

\input{../tex-inputs/latex-vorspann}


\section[ Felix Salten an Arthur Schnitzler, {[}14?. 1. 1895{]}]{L03148 Felix Salten an Arthur Schnitzler,  [14?. 1. 1895]}
\nopagebreak\mylabel{L03148v}
\rehead{ }\normalsize\beginnumbering\briefempfaengerindex{Schnitzler, Arthur@\textsc{Schnitzler, Arthur}!zzzSalten, Felix@\emph{von Felix Salten}!1895-01-142@{{[}14?. 1. 1895{]}}|(be}
\toendnotes[C]{\smallbreak\pagebreak[2]}
\correspDesc{Versand  durch Felix Salten am [14?. 1. 1895] in Wien
\newline{}Erhalt  durch Arthur Schnitzler am [14. 1. 1895] in Wien}\toendnotes[C]{\smallbreak}
\Standort{CUL, Schnitzler, B 89, A 1.}
\physDesc{Brief, 1 Blatt, 2 Seiten, 432 Zeichen
\newline{}Handschrift: Bleistift, lateinische Kurrent
\newline{}Schnitzler: mit Bleistift datiert: »1\substVorne{}\textsuperscript{2}\substDazwischen{}3\substHinten{}/1 95« 
\newline{}Ordnung: mit Bleistift von unbekannter Hand nummeriert: »49« }\toendnotes[C]{\smallbreak}
\pstart
           \noindent{}{\pb}Lieber Freund,{ }\label{K_L03148-1v}\edtext{Lotte\pwindex{Pohl-Glas, Charlotte 1.\,1.\,1873 Wien – 15.\,2.\,1944 Zürich@\textsc{Pohl-Glas, Charlotte} (1.\,1.\,1873 Wien – 15.\,2.\,1944 Zürich), \emph{Schriftstellerin, Politikerin, Sozialistin}|pw} geht morgen in Haft}{\lemma{\textnormal{\emph{Lotte … Haft}}}\Cendnote{\textnormal{Siehe XXXX Auszeichnungsfehler: Dokument L03142 nicht gefunden.
               }}}\label{K_L03148-1} und ich habe heute für sie einiges zu kaufen.
               Sie schreibt mir eben um Geld, und bittet mich, da ihre Leute nichts für sie thun
               wollen. Nun ist erst \label{K_L03148-2v}\edtext{morgen{ }der 15\textsuperscript{te}}{\lemma{\textnormal{\emph{morgen der 15\textsuperscript{te}}}}\Cendnote{\textnormal{Schnitzler datierte den Brief auf den 13. 1. 1895, doch nahm er dabei eine Überschreibung
                  vor und änderte den 12. ab, den er zuerst
                  geschrieben haben dürfte. Diese Unsicherheit und Saltens\pwindex{Salten, Felix 6.\,9.\,1869 Budapest – 8.\,10.\,1945 Zürich@\textsc{Salten, Felix} (6.\,9.\,1869 Budapest – 8.\,10.\,1945 Zürich), \emph{Schriftsteller, Journalist, Chefredakteur}|pwk} Aussage, dass »morgen{ }der 15\textsuperscript{te}« sei, sind Gründe für die Datierung des Briefes auf den 14. In jedem Fall dürfte Schnitzler am 14. 1. 1895 den Brief erhalten haben, da eine
                  Aussage zu diesem Tag im \emph{Tagebuch}\pwindex{Schnitzler, Arthur 15.\,5.\,1862 Wien – 21.\,10.\,1931 ebd.@\textsc{Schnitzler, Arthur} (15.\,5.\,1862 Wien – 21.\,10.\,1931 ebd.), \emph{Schriftsteller, Mediziner}!Tagebuch@\strich\emph{Tagebuch}|pwk} dadurch motiviert scheint: »Saltens\pwindex{Salten, Felix 6.\,9.\,1869 Budapest – 8.\,10.\,1945 Zürich@\textsc{Salten, Felix} (6.\,9.\,1869 Budapest – 8.\,10.\,1945 Zürich), \emph{Schriftsteller, Journalist, Chefredakteur}|pw}{ }Gel.\pwindex{Pohl-Glas, Charlotte 1.\,1.\,1873 Wien – 15.\,2.\,1944 Zürich@\textsc{Pohl-Glas, Charlotte} (1.\,1.\,1873 Wien – 15.\,2.\,1944 Zürich), \emph{Schriftstellerin, Politikerin, Sozialistin}|pwv} wird morgen (wegen social. Geschichten) eingesperrt. Der
                     Glückliche.«}}}\label{K_L03148-2}, und ich bitte Sie deswegen \uline{recht sehr}, mir \uline{bis morgen} mit fl. 10.{\char`~} zu helfen. Ich erhalte \uline{morgen{ }3 Uhr Gage}, und gebe Ihnen {\pb}\uline{mein Wort}, dass ich Ihnen das Geld \uline{morgen{ }Nachmittag} sofort hinüberbringe.\pend
           
\pstart
           Besten Dank im Voraus. Herzlichst Ihr {\\[\baselineskip]}\spacefill\mbox{Salten}\pend
           \leftskip=0em{}\selectlanguage{ngerman}\endnumbering\briefempfaengerindex{Schnitzler, Arthur@\textsc{Schnitzler, Arthur}!zzzSalten, Felix@\emph{von Felix Salten}!1895-01-142@{{[}14?. 1. 1895{]}}|)be}\mylabel{L03148h}  \newcommand{\dateiname}{L03148}\newcommand{\titel}{Felix Salten an Arthur Schnitzler, [14?. 1. 1895]}\newcommand{\editorInnen}{Martin Anton Müller und Laura Untner}%% latex-leseansicht-abspann.tex
%% Abspann für die Leseansicht.
%% Der Schalter \ifkorrekturansicht ist bereits durch den Vorspann gesetzt.

%% latex-abspann.tex
%% Gemeinsamer Abspann für Korrekturansicht und Leseansicht.
%% Setzt den Schalter \ifkorrekturansicht voraus (gesetzt in den
%% einbindenden Dateien latex-korrekturansicht-abspann.tex bzw.
%% latex-leseansicht-abspann.tex).
%% ---------------------------------------------------------------

\normalsize

% Das esempio-Environment wird nur in der Leseansicht benötigt
\ifkorrekturansicht\else
\newenvironment{esempio}[3]%
{
    \vspace{1.5ex}
    \rlap{\underline{#1}}
    \par
    \setlength{\parindent}{0cm}
    \nopagebreak
    \leftskip=#2cm
    \rightskip=#3cm
}
{
    \par
}
\fi

\doendnotes{C}
\bigskip
\vfill

\clearpage

\footnotesize

\ifkorrekturansicht
  \lohead{\textsc{register}}
\fi

% theindex-Environment neu definieren ohne reledmac
\makeatletter
\renewenvironment{theindex}{%
  \ifkorrekturansicht
    \section*{\indexname}%
  \else
    \subsubsection*{Index der erwähnten Entitäten}%
  \fi
  \setlength{\parindent}{0pt}%
  \setlength{\parskip}{0pt plus 0.3pt}%
  \let\item\@idxitem
}{%
  \ifkorrekturansicht\clearpage\fi
}
\makeatother

\IfFileExists{\jobname-pw.ind}{\input{\jobname-pw.ind}}{}

% Quellenangabe nur in der Leseansicht
\ifkorrekturansicht\else
% Fallback-Definitionen, falls die .tex-Datei \titel etc. nicht gesetzt hat
\providecommand{\titel}{}
\providecommand{\editorInnen}{}
\providecommand{\dateiname}{\jobname}

\vspace{3cm}

\vfill

\footnotesize
\textsc{Quelle}: \titel. Herausgegeben von {\editorInnen}. In: \emph{Arthur Schnitzler: Briefwechsel mit Autorinnen und Autoren}.
 Digitale Edition, https://schnitzler-briefe.acdh.oeaw.ac.at/{\dateiname}.html (Stand \today)
\fi

\end{document}


