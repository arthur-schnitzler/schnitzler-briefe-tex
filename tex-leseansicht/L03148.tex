%% latex-korrekturansicht-vorspann.tex
%% Vorspann für die Korrekturansicht.
%% Lädt die gemeinsame Datei latex-vorspann.tex mit gesetztem Schalter.

\newif\ifkorrekturansicht
\korrekturansichttrue

\input{../tex-inputs/latex-vorspann}


\section[ Felix Salten an Arthur Schnitzler, {[}14?. 1. 1895{]}]{L03148 Felix Salten an Arthur Schnitzler, {[}14?. 1. 1895{]}}
\nopagebreak\mylabel{L03148v}
\rehead{ }\normalsize\beginnumbering\briefempfaengerindex{Schnitzler, Arthur@\textsc{Schnitzler, Arthur}!zzzSalten, Felix@\emph{von Felix Salten}!1895-01-141@{{[}14?. 1. 1895{]}}|(be}
\toendnotes[C]{\smallbreak\pagebreak[2]}\Standort{CUL, Schnitzler, B 89, A 1.}
\physDesc{Brief, 1 Blatt, 2 Seiten, 432 Zeichen
\newline{}Handschrift: Bleistift, lateinische Kurrent
\newline{}Schnitzler: mit Bleistift datiert: »1\substVorne{}\textsuperscript{2}\substDazwischen{}3\substHinten{}/1 95« 
\newline{}Ordnung: mit Bleistift von unbekannter Hand nummeriert: »49« }\toendnotes[C]{\smallbreak}
\pstart
           \noindent{}{\pb}Lieber Freund,{ }\label{K_L03148-1v}\edtext{Lotte\pwindex{Pohl-Glas, Charlotte 1873-01-01 – 1944-02-15@\textsc{Pohl-Glas, Charlotte} (1873-01-01 – 1944-02-15), \emph{Schriftsteller/Schriftstellerin, Politiker/Politikerin, Sozialist/Sozialistin}|pw} geht morgen in Haft}{\lemma{\textnormal{\emph{Lotte … Haft}}}\Cendnote{\textnormal{Siehe Felix Salten an Arthur Schnitzler, 7. 8. 1894.
               }}}\label{K_L03148-1} und ich habe heute für sie einiges zu kaufen.
               Sie schreibt mir eben um Geld, und bittet mich, da ihre Leute nichts für sie thun
               wollen. Nun ist erst \label{K_L03148-2v}\edtext{morgen{ }der 15\textsuperscript{te}}{\lemma{\textnormal{\emph{morgen der 15\textsuperscript{te}}}}\Cendnote{\textnormal{Schnitzler datierte den Brief auf den 13. 1. 1895, doch nahm er dabei eine Überschreibung
                  vor und änderte den 12. ab, den er zuerst
                  geschrieben haben dürfte. Diese Unsicherheit und Saltens\pwindex{Salten, Felix 06.09.1869 – 08.10.1945@\textsc{Salten, Felix} (06.09.1869 – 08.10.1945), \emph{Schriftsteller/Schriftstellerin, Journalist/Journalistin, Chefredakteur/Chefredakteurin}|pwk} Aussage, dass »morgen{ }der 15\textsuperscript{te}« sei, sind Gründe für die Datierung des Briefes auf den 14. In jedem Fall dürfte Schnitzler am 14. 1. 1895 den Brief erhalten haben, da eine
                  Aussage zu diesem Tag im \emph{Tagebuch}\pwindex{Tagebuch@\emph{Tagebuch}|pwk} dadurch motiviert scheint: »Saltens\pwindex{Salten, Felix 06.09.1869 – 08.10.1945@\textsc{Salten, Felix} (06.09.1869 – 08.10.1945), \emph{Schriftsteller/Schriftstellerin, Journalist/Journalistin, Chefredakteur/Chefredakteurin}|pw}{ }Gel.\pwindex{Pohl-Glas, Charlotte 1873-01-01 – 1944-02-15@\textsc{Pohl-Glas, Charlotte} (1873-01-01 – 1944-02-15), \emph{Schriftsteller/Schriftstellerin, Politiker/Politikerin, Sozialist/Sozialistin}|pwv} wird morgen (wegen social. Geschichten) eingesperrt. Der
                     Glückliche.«}}}\label{K_L03148-2}, und ich bitte Sie deswegen \uline{recht sehr}, mir \uline{bis morgen} mit fl. 10.{\char`~} zu helfen. Ich erhalte \uline{morgen{ }3 Uhr Gage}, und gebe Ihnen {\pb}\uline{mein Wort}, dass ich Ihnen das Geld \uline{morgen{ }Nachmittag} sofort hinüberbringe.\pend
           
\pstart
           Besten Dank im Voraus. Herzlichst Ihr {\\[\baselineskip]}\spacefill\mbox{Salten}\pend
           \leftskip=0em{}\selectlanguage{ngerman}\endnumbering\briefempfaengerindex{Schnitzler, Arthur@\textsc{Schnitzler, Arthur}!zzzSalten, Felix@\emph{von Felix Salten}!1895-01-141@{{[}14?. 1. 1895{]}}|)be}\mylabel{L03148h}  \normalsize

\doendnotes{C}
\bigskip
\vfill

\clearpage

\footnotesize

\lohead{\textsc{register}}

% Definiere theindex-Environment komplett neu ohne reledmac
\makeatletter
\renewenvironment{theindex}{%
  \section*{\indexname}%
  \setlength{\parindent}{0pt}%
  \setlength{\parskip}{0pt plus 0.3pt}%
  \let\item\@idxitem
}{%
  \clearpage
}
\makeatother

\IfFileExists{\jobname-pw.ind}{\input{\jobname-pw.ind}}{}

\end{document}

      