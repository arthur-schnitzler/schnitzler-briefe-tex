%% latex-leseansicht-vorspann.tex
%% Vorspann für die Leseansicht.
%% Lädt die gemeinsame Datei latex-vorspann.tex mit nicht gesetztem Schalter.

\newif\ifkorrekturansicht
\korrekturansichtfalse

\input{../tex-inputs/latex-vorspann}


         
         \renewcommand{\erwaehntePersonen}{Personen: Otto Brahm, Gabriele D’Annunzio, Ludwig Fulda, Ludwig Ganghofer, Wilhelm von Scholz}
         \renewcommand{\erwaehnteInstitutionen}{Institutionen: Burgtheater, Deutsches Theater Berlin, Münchener Litterarische Gesellschaft, Residenztheater München}
         \renewcommand{\erwaehnteOrte}{Orte: Berlin, München, Wien}
         \renewcommand{\erwaehnteWerke}{Werke: Der grüne Kakadu. Groteske in einem Akt, Mein Fürst, Traum eines Frühlingsmorgens}
               \section[Arthur Schnitzler an Ludwig Ganghofer, 4. 2. 1899]{ Arthur Schnitzler an Ludwig Ganghofer, 4. 2. 1899}\nopagebreak\mylabel{v}\rehead{ }\begin{ledgroupsized}[t]{13cm}\normalsize\beginnumbering \toendnotes[C]{\smallbreak\pagebreak[2]} \Standort{München, Monacensia, Nachl. Ludwig Ganghofer, B 170.}
\physDesc{Brief, 1 Blatt, 3 Seiten, 942 Zeichen
\newline{}Handschrift: schwarze Tinte, deutsche Kurrent}\toendnotes[C]{\smallbreak}\pstart
           \noindent{}{\pb}Sehr geehrter Herr, mein Telegramm hat Ihnen bereits mitgetheilt,
               dſs der »grüne Kakadu\pwindex{Schnitzler, Arthur 15.05.1862 – 21.10.1931@\textsc{Schnitzler, Arthur} (15.05.1862 – 21.10.1931), \emph{Schriftsteller, Mediziner}!gruene Kakadu. Groteske in einem Akt1. 3. 1899@\strich\emph{Der grüne Kakadu. Groteske in einem Akt} {[}1. 3. 1899{]}|pw}\pwindex{Schnitzler, Arthur 15.05.1862 – 21.10.1931@\textsc{Schnitzler, Arthur} (15.05.1862 – 21.10.1931), \emph{Schriftsteller, Mediziner}!gruene Kakadu. Groteske in einem Akt1. 3. 1899@\strich\emph{Der grüne Kakadu. Groteske in einem Akt} {[}1. 3. 1899{]}|pw}« (mit einigen Strichen
               natürlich) am Burgtheater\orgindex{Burgtheater@Burgtheater|pw} zur Aufführg kommt. Das
               ſoll zu \label{K_L00884-1v}\edtext{Anfang März}{\lemma{\textnormal{\emph{Anfang März}}}\Cendnote{\textnormal{Die Uraufführung fand am 1. 3. 1899
                  statt.}}}\label{K_L00884-1h} geſchehen. Nun habe ich auch mit \textsc{Fulda}\pwindex{Fulda, Ludwig 15.07.1862 – 30.03.1939@\textsc{Fulda, Ludwig} (15.07.1862 – 30.03.1939), \emph{Schriftsteller, Übersetzer}|pw}, der eben in Wien\oindex{Wien@\textbf{Wien}|pw} iſt, wegen der Berlin\oindex{Berlin@\textbf{Berlin}|pw}er Prem. früher geſprochen, und die Zuſage
               erhalten, daſs der »Kakadu\pwindex{Schnitzler, Arthur 15.05.1862 – 21.10.1931@\textsc{Schnitzler, Arthur} (15.05.1862 – 21.10.1931), \emph{Schriftsteller, Mediziner}!gruene Kakadu. Groteske in einem Akt1. 3. 1899@\strich\emph{Der grüne Kakadu. Groteske in einem Akt} {[}1. 3. 1899{]}|pw}\pwindex{Schnitzler, Arthur 15.05.1862 – 21.10.1931@\textsc{Schnitzler, Arthur} (15.05.1862 – 21.10.1931), \emph{Schriftsteller, Mediziner}!gruene Kakadu. Groteske in einem Akt1. 3. 1899@\strich\emph{Der grüne Kakadu. Groteske in einem Akt} {[}1. 3. 1899{]}|pw}« {\pb}\label{K_L00884-2v}\edtext{Anfang April}{\lemma{\textnormal{\emph{Anfang April}}}\Cendnote{\textnormal{Die Premiere am \emph{Deutschen Theater}\orgindex{Deutsches Theater Berlin@Deutsches Theater Berlin|pwk} fand am 29. 4. 1899
                  statt.}}}\label{K_L00884-2h}, ſpäteſtens 10. in Berlin\oindex{Berlin@\textbf{Berlin}|pw} geſpielt werden wird. Ich möchte Sie alſo bitten, das Stück nicht
               früher zu geben; mir wäre es am liebſten, we{\geminationn} Sie es
               etwa um den 15. April herum herausbringen könnten, ſo daſs ich von Berlin\oindex{Berlin@\textbf{Berlin}|pw} aus zu Ihren Proben reiſen könnte. Eine
               Aufführg in München\oindex{Muenchen@\textbf{München}|pw} vor Berlin\oindex{Berlin@\textbf{Berlin}|pw} wäre mir in Hinblick auf frühere Verabredungen {\pb}mit Brahm\pwindex{Brahm, Otto 05.02.1856 – 28.11.1912@\textsc{Brahm, Otto} (05.02.1856 – 28.11.1912), \emph{Theaterleiter, Regisseur}|pw} und Fulda\pwindex{Fulda, Ludwig 15.07.1862 – 30.03.1939@\textsc{Fulda, Ludwig} (15.07.1862 – 30.03.1939), \emph{Schriftsteller, Übersetzer}|pw}, \uline{nicht} erwünſcht und ich hoffe, es hat keine Schwierigkeiten für Sie,
               die \label{K_L00884-3v}\edtext{Aufführg bis Mitte
                  April}{\lemma{\textnormal{\emph{Aufführg bis Mitte
                  April}}}\Cendnote{\textnormal{Die Aufführung durch die \emph{Münchener Litterarische Gesellschaft}\orgindex{Muenchener Litterarische Gesellschaft@Münchener Litterarische Gesellschaft|pwk} fand am Tag der Berlin\oindex{Berlin@\textbf{Berlin}|pwk}er Premiere, am 29. 4. 1899,
                  im \emph{Residenztheater}\orgindex{Residenztheater Muenchen@Residenztheater München|pwk} statt.}}}\label{K_L00884-3h}
               hinauszuſchieben.\pend
           \pstart
           Iſt ſchon eine Wahl in Hinſicht auf das \label{K_L00884-4v}\edtext{Stück\pwindex{DAnnunzio, Gabriele 12.03.1863 – 01.03.1938@\textsc{D’Annunzio, Gabriele} (12.03.1863 – 01.03.1938), \emph{Schriftsteller}!Traum eines Fruehlingsmorgens1897-06-15@\strich\emph{Traum eines Frühlingsmorgens} {[}1897-06-15{]}|pw}\pwindex{Scholz, Wilhelm von 15.07.1874 – 29.05.1969@\textsc{Scholz, Wilhelm von} (15.07.1874 – 29.05.1969), \emph{Schriftsteller, Kulturfunktionär}!Mein Fuerst1898@\strich\emph{Mein Fürst} {[}1898{]}|pw}}{\lemma{\textnormal{\emph{Stück}}}\Cendnote{\textnormal{Gegeben wurde es mit \emph{Traum eines Frühlingsmorgens}\pwindex{DAnnunzio, Gabriele 12.03.1863 – 01.03.1938@\textsc{D’Annunzio, Gabriele} (12.03.1863 – 01.03.1938), \emph{Schriftsteller}!Traum eines Fruehlingsmorgens1897-06-15@\strich\emph{Traum eines Frühlingsmorgens} {[}1897-06-15{]}|pwk} von Gabriele D’Annunzio\pwindex{DAnnunzio, Gabriele 12.03.1863 – 01.03.1938@\textsc{D’Annunzio, Gabriele} (12.03.1863 – 01.03.1938), \emph{Schriftsteller}|pwk} und \emph{Mein Fürst}\pwindex{Scholz, Wilhelm von 15.07.1874 – 29.05.1969@\textsc{Scholz, Wilhelm von} (15.07.1874 – 29.05.1969), \emph{Schriftsteller, Kulturfunktionär}!Mein Fuerst1898@\strich\emph{Mein Fürst} {[}1898{]}|pwk} von Wilhelm von
                  Scholz\pwindex{Scholz, Wilhelm von 15.07.1874 – 29.05.1969@\textsc{Scholz, Wilhelm von} (15.07.1874 – 29.05.1969), \emph{Schriftsteller, Kulturfunktionär}|pwk}.}}}\label{K_L00884-4h} getroffen, das zum Kakadu\pwindex{Schnitzler, Arthur 15.05.1862 – 21.10.1931@\textsc{Schnitzler, Arthur} (15.05.1862 – 21.10.1931), \emph{Schriftsteller, Mediziner}!gruene Kakadu. Groteske in einem Akt1. 3. 1899@\strich\emph{Der grüne Kakadu. Groteske in einem Akt} {[}1. 3. 1899{]}|pw}\pwindex{Schnitzler, Arthur 15.05.1862 – 21.10.1931@\textsc{Schnitzler, Arthur} (15.05.1862 – 21.10.1931), \emph{Schriftsteller, Mediziner}!gruene Kakadu. Groteske in einem Akt1. 3. 1899@\strich\emph{Der grüne Kakadu. Groteske in einem Akt} {[}1. 3. 1899{]}|pw}
               gegeben werden ſoll?\pend
           \pstart
           In beſondrer Hochſchätzg ergebenſt{\\[\baselineskip]}\spacefill\mbox{DrArthur Schnitzler}\pend
           \leftskip=0em{}\pstart
           Wien\oindex{Wien@\textbf{Wien}|pw}, 4. Feber 99.\pend
           
         
         \endnumbering\mylabel{h}\end{ledgroupsized}  \newcommand{\dateiname}{L00884}\newcommand{\titel}{Arthur Schnitzler an Ludwig Ganghofer, 4. 2. 1899}\newcommand{\editorInnen}{Martin Anton Müller und Gerd-Hermann Susen}%% latex-leseansicht-abspann.tex
%% Abspann für die Leseansicht.
%% Der Schalter \ifkorrekturansicht ist bereits durch den Vorspann gesetzt.

%% latex-abspann.tex
%% Gemeinsamer Abspann für Korrekturansicht und Leseansicht.
%% Setzt den Schalter \ifkorrekturansicht voraus (gesetzt in den
%% einbindenden Dateien latex-korrekturansicht-abspann.tex bzw.
%% latex-leseansicht-abspann.tex).
%% ---------------------------------------------------------------

\normalsize

% Das esempio-Environment wird nur in der Leseansicht benötigt
\ifkorrekturansicht\else
\newenvironment{esempio}[3]%
{
    \vspace{1.5ex}
    \rlap{\underline{#1}}
    \par
    \setlength{\parindent}{0cm}
    \nopagebreak
    \leftskip=#2cm
    \rightskip=#3cm
}
{
    \par
}
\fi

\doendnotes{C}
\bigskip
\vfill

\clearpage

\footnotesize

\ifkorrekturansicht
  \lohead{\textsc{register}}
\fi

% theindex-Environment neu definieren ohne reledmac
\makeatletter
\renewenvironment{theindex}{%
  \ifkorrekturansicht
    \section*{\indexname}%
  \else
    \subsubsection*{Index der erwähnten Entitäten}%
  \fi
  \setlength{\parindent}{0pt}%
  \setlength{\parskip}{0pt plus 0.3pt}%
  \let\item\@idxitem
}{%
  \ifkorrekturansicht\clearpage\fi
}
\makeatother

\IfFileExists{\jobname-pw.ind}{\input{\jobname-pw.ind}}{}

% Quellenangabe nur in der Leseansicht
\ifkorrekturansicht\else
% Fallback-Definitionen, falls die .tex-Datei \titel etc. nicht gesetzt hat
\providecommand{\titel}{}
\providecommand{\editorInnen}{}
\providecommand{\dateiname}{\jobname}

\vspace{3cm}

\vfill

\footnotesize
\textsc{Quelle}: \titel. Herausgegeben von {\editorInnen}. In: \emph{Arthur Schnitzler: Briefwechsel mit Autorinnen und Autoren}.
 Digitale Edition, https://schnitzler-briefe.acdh.oeaw.ac.at/{\dateiname}.html (Stand \today)
\fi

\end{document}


      