%% latex-leseansicht-vorspann.tex
%% Vorspann für die Leseansicht.
%% Lädt die gemeinsame Datei latex-vorspann.tex mit nicht gesetztem Schalter.

\newif\ifkorrekturansicht
\korrekturansichtfalse

\input{../tex-inputs/latex-vorspann}


\section[Arthur Schnitzler an Ludwig Ganghofer, 4. 2. 1899]{L00884 Arthur Schnitzler an Ludwig Ganghofer, 4. 2. 1899}
\nopagebreak\mylabel{L00884v}
\rehead{ }\normalsize\beginnumbering\briefempfaengerindex{Ganghofer, Ludwig@\textsc{Ganghofer, Ludwig}!zzzSchnitzler, Arthur@\emph{von Arthur Schnitzler}!1899-02-041@{4. 2. 1899}|(be}
\toendnotes[C]{\smallbreak\pagebreak[2]}
\correspDesc{Versand  durch Arthur Schnitzler am 4. 2. 1899 in Wien
\newline{}Erhalt  durch Ludwig Ganghofer im Zeitraum [5. 2. 1899
                  – 9. 2. 1899?] in München}\toendnotes[C]{\smallbreak}
\Standort{München, Monacensia, Nachl. Ludwig Ganghofer, B 170.}
\physDesc{Brief, 1 Blatt, 3 Seiten, 942 Zeichen
\newline{}Handschrift: schwarze Tinte, deutsche Kurrent}\toendnotes[C]{\smallbreak}
\pstart
           \noindent{}{\pb}Sehr geehrter Herr, mein Telegramm hat Ihnen bereits mitgetheilt,
               dſs der »grüne Kakadu\pwindex{Schnitzler, Arthur 15.\,5.\,1862 Wien – 21.\,10.\,1931 ebd.@\textsc{Schnitzler, Arthur} (15.\,5.\,1862 Wien – 21.\,10.\,1931 ebd.), \emph{Schriftsteller, Mediziner}!grüne Kakadu. Groteske in einem Akt@\strich\emph{Der grüne Kakadu. Groteske in einem Akt}|pw}« (mit einigen Strichen
               natürlich) am Burgtheater\orgindex{Burgtheater@Burgtheater|pw} zur Aufführg kommt. Das{ }ſoll zu \label{K_L00884-1v}\edtext{Anfang März}{\lemma{\textnormal{\emph{Anfang März}}}\Cendnote{\textnormal{Die Uraufführung\eventindex{Burgtheater@\textbf{Burgtheater}!Uraufführung von Der grüne Kakadu – Paracelsus – Die Gefährtin. Drei Einakter, 1.3.1899@Uraufführung von Der grüne Kakadu – Paracelsus – Die Gefährtin. Drei Einakter, 1.3.1899|pwkv} fand am 1. 3. 1899
                  statt.}}}\label{K_L00884-1} geſchehen. Nun habe ich auch mit \textsc{Fulda}\pwindex{Fulda, Ludwig 15.\,7.\,1862 Frankfurt am Main – 30.\,3.\,1939 Berlin@\textsc{Fulda, Ludwig} (15.\,7.\,1862 Frankfurt am Main – 30.\,3.\,1939 Berlin), \emph{Schriftsteller, Übersetzer}|pw}, der eben in Wien\oindex{Wien@\textbf{Wien}, \emph{Verwaltungsgebiet}|pw} iſt, wegen der Berlin\oindex{Berlin@\textbf{Berlin}, \emph{Hauptstadt}|pw}er Prem. früher geſprochen, und die Zuſage
               erhalten, daſs der »Kakadu\pwindex{Schnitzler, Arthur 15.\,5.\,1862 Wien – 21.\,10.\,1931 ebd.@\textsc{Schnitzler, Arthur} (15.\,5.\,1862 Wien – 21.\,10.\,1931 ebd.), \emph{Schriftsteller, Mediziner}!grüne Kakadu. Groteske in einem Akt@\strich\emph{Der grüne Kakadu. Groteske in einem Akt}|pw}« {\pb}\label{K_L00884-2v}\edtext{Anfang April}{\lemma{\textnormal{\emph{Anfang April}}}\Cendnote{\textnormal{Die Premiere\eventindex{Deutsches Theater Berlin@\textbf{Deutsches Theater Berlin}!Berliner Premiere von Der grüne Kakadu, 29.4.1899@Berliner Premiere von Der grüne Kakadu, 29.4.1899|pwkv} am \emph{Deutschen Theater}\orgindex{Deutsches Theater Berlin@Deutsches Theater Berlin|pwk} fand am 29. 4. 1899
                  statt.}}}\label{K_L00884-2},{ }ſpäteſtens 10. in Berlin\oindex{Berlin@\textbf{Berlin}, \emph{Hauptstadt}|pw} geſpielt werden wird. Ich möchte Sie alſo bitten, das Stück nicht
               früher zu geben; mir wäre es am liebſten, we{\geminationn} Sie es
               etwa um den 15. April herum herausbringen könnten,{ }ſo daſs ich von Berlin\oindex{Berlin@\textbf{Berlin}, \emph{Hauptstadt}|pw} aus zu Ihren Proben reiſen könnte. Eine
               Aufführg in München\oindex{München@\textbf{München}|pw} vor Berlin\oindex{Berlin@\textbf{Berlin}, \emph{Hauptstadt}|pw} wäre mir in Hinblick auf frühere Verabredungen {\pb}mit Brahm\pwindex{Brahm, Otto 5.\,2.\,1856 Hamburg – 28.\,11.\,1912 Berlin@\textsc{Brahm, Otto} (5.\,2.\,1856 Hamburg – 28.\,11.\,1912 Berlin), \emph{Theaterleiter, Regisseur}|pw} und Fulda\pwindex{Fulda, Ludwig 15.\,7.\,1862 Frankfurt am Main – 30.\,3.\,1939 Berlin@\textsc{Fulda, Ludwig} (15.\,7.\,1862 Frankfurt am Main – 30.\,3.\,1939 Berlin), \emph{Schriftsteller, Übersetzer}|pw}, \uline{nicht} erwünſcht und ich hoffe, es hat keine Schwierigkeiten für Sie,
               die \label{K_L00884-3v}\edtext{Aufführg bis Mitte April}{\lemma{\textnormal{\emph{Aufführg bis Mitte
                  April}}}\Cendnote{\textnormal{Die Aufführung\eventindex{Residenztheater München@\textbf{Residenztheater München}!Premiere von Traum eines Frühlingsmorgens, Mein Fürst, Der grüne Kakadu, 29.4.1899@Premiere von Traum eines Frühlingsmorgens, Mein Fürst, Der grüne Kakadu, 29.4.1899|pwkv} durch die \emph{Münchener Litterarische Gesellschaft}\orgindex{Münchener Litterarische Gesellschaft@Münchener Litterarische Gesellschaft|pwk} fand am Tag der Berlin\oindex{Berlin@\textbf{Berlin}, \emph{Hauptstadt}|pwk}er Premiere\eventindex{Deutsches Theater Berlin@\textbf{Deutsches Theater Berlin}!Berliner Premiere von Der grüne Kakadu, 29.4.1899@Berliner Premiere von Der grüne Kakadu, 29.4.1899|pwkv}, am 29. 4. 1899,
                  im \emph{Residenztheater}\orgindex{Residenztheater München@Residenztheater München|pwk} statt.}}}\label{K_L00884-3}
               hinauszuſchieben.\pend
           
\pstart
           Iſt{ }ſchon eine Wahl in Hinſicht auf das \label{K_L00884-4v}\edtext{Stück\pwindex{D’Annunzio, Gabriele 12.\,3.\,1863 Pescara – 1.\,3.\,1938 Cargnacco@\textsc{D’Annunzio, Gabriele} (12.\,3.\,1863 Pescara – 1.\,3.\,1938 Cargnacco), \emph{Schriftsteller}!Traum eines Frühlingsmorgens@\strich\emph{Traum eines Frühlingsmorgens}|pw}\pwindex{Scholz, Wilhelm von 15.\,7.\,1874 Berlin – 29.\,5.\,1969 Schloss Seeheim@\textsc{Scholz, Wilhelm von} (15.\,7.\,1874 Berlin – 29.\,5.\,1969 Schloss Seeheim), \emph{Schriftsteller, Kulturfunktionär}!Mein Fürst@\strich\emph{Mein Fürst}|pw}}{\lemma{\textnormal{\emph{Stück}}}\Cendnote{\textnormal{Gegeben wurde es mit \emph{Traum eines Frühlingsmorgens}\pwindex{D’Annunzio, Gabriele 12.\,3.\,1863 Pescara – 1.\,3.\,1938 Cargnacco@\textsc{D’Annunzio, Gabriele} (12.\,3.\,1863 Pescara – 1.\,3.\,1938 Cargnacco), \emph{Schriftsteller}!Traum eines Frühlingsmorgens@\strich\emph{Traum eines Frühlingsmorgens}|pwk} von Gabriele D’Annunzio\pwindex{D’Annunzio, Gabriele 12.\,3.\,1863 Pescara – 1.\,3.\,1938 Cargnacco@\textsc{D’Annunzio, Gabriele} (12.\,3.\,1863 Pescara – 1.\,3.\,1938 Cargnacco), \emph{Schriftsteller}|pwk} und \emph{Mein Fürst}\pwindex{Scholz, Wilhelm von 15.\,7.\,1874 Berlin – 29.\,5.\,1969 Schloss Seeheim@\textsc{Scholz, Wilhelm von} (15.\,7.\,1874 Berlin – 29.\,5.\,1969 Schloss Seeheim), \emph{Schriftsteller, Kulturfunktionär}!Mein Fürst@\strich\emph{Mein Fürst}|pwk} von Wilhelm von
                  Scholz\pwindex{Scholz, Wilhelm von 15.\,7.\,1874 Berlin – 29.\,5.\,1969 Schloss Seeheim@\textsc{Scholz, Wilhelm von} (15.\,7.\,1874 Berlin – 29.\,5.\,1969 Schloss Seeheim), \emph{Schriftsteller, Kulturfunktionär}|pwk}.}}}\label{K_L00884-4} getroffen, das zum Kakadu\pwindex{Schnitzler, Arthur 15.\,5.\,1862 Wien – 21.\,10.\,1931 ebd.@\textsc{Schnitzler, Arthur} (15.\,5.\,1862 Wien – 21.\,10.\,1931 ebd.), \emph{Schriftsteller, Mediziner}!grüne Kakadu. Groteske in einem Akt@\strich\emph{Der grüne Kakadu. Groteske in einem Akt}|pw}
               gegeben werden{ }ſoll?\pend
           
\pstart
           In beſondrer Hochſchätzg ergebenſt{\\[\baselineskip]}\spacefill\mbox{DrArthur Schnitzler}\pend
           \leftskip=0em{}
\pstart
           Wien\oindex{Wien@\textbf{Wien}, \emph{Verwaltungsgebiet}|pw}, 4. Feber 99.\pend
           \selectlanguage{ngerman}\endnumbering\briefempfaengerindex{Ganghofer, Ludwig@\textsc{Ganghofer, Ludwig}!zzzSchnitzler, Arthur@\emph{von Arthur Schnitzler}!1899-02-041@{4. 2. 1899}|)be}\mylabel{L00884h}  \newcommand{\dateiname}{L00884}\newcommand{\titel}{Arthur Schnitzler an Ludwig Ganghofer, 4. 2. 1899}\newcommand{\editorInnen}{Martin Anton Müller und Gerd-Hermann Susen}%% latex-leseansicht-abspann.tex
%% Abspann für die Leseansicht.
%% Der Schalter \ifkorrekturansicht ist bereits durch den Vorspann gesetzt.

%% latex-abspann.tex
%% Gemeinsamer Abspann für Korrekturansicht und Leseansicht.
%% Setzt den Schalter \ifkorrekturansicht voraus (gesetzt in den
%% einbindenden Dateien latex-korrekturansicht-abspann.tex bzw.
%% latex-leseansicht-abspann.tex).
%% ---------------------------------------------------------------

\normalsize

% Das esempio-Environment wird nur in der Leseansicht benötigt
\ifkorrekturansicht\else
\newenvironment{esempio}[3]%
{
    \vspace{1.5ex}
    \rlap{\underline{#1}}
    \par
    \setlength{\parindent}{0cm}
    \nopagebreak
    \leftskip=#2cm
    \rightskip=#3cm
}
{
    \par
}
\fi

\doendnotes{C}
\bigskip
\vfill

\clearpage

\footnotesize

\ifkorrekturansicht
  \lohead{\textsc{register}}
\fi

% theindex-Environment neu definieren ohne reledmac
\makeatletter
\renewenvironment{theindex}{%
  \ifkorrekturansicht
    \section*{\indexname}%
  \else
    \subsubsection*{Index der erwähnten Entitäten}%
  \fi
  \setlength{\parindent}{0pt}%
  \setlength{\parskip}{0pt plus 0.3pt}%
  \let\item\@idxitem
}{%
  \ifkorrekturansicht\clearpage\fi
}
\makeatother

\IfFileExists{\jobname-pw.ind}{\input{\jobname-pw.ind}}{}

% Quellenangabe nur in der Leseansicht
\ifkorrekturansicht\else
% Fallback-Definitionen, falls die .tex-Datei \titel etc. nicht gesetzt hat
\providecommand{\titel}{}
\providecommand{\editorInnen}{}
\providecommand{\dateiname}{\jobname}

\vspace{3cm}

\vfill

\footnotesize
\textsc{Quelle}: \titel. Herausgegeben von {\editorInnen}. In: \emph{Arthur Schnitzler: Briefwechsel mit Autorinnen und Autoren}.
 Digitale Edition, https://schnitzler-briefe.acdh.oeaw.ac.at/{\dateiname}.html (Stand \today)
\fi

\end{document}


