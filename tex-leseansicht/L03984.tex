%% latex-leseansicht-vorspann.tex
%% Vorspann für die Leseansicht.
%% Lädt die gemeinsame Datei latex-vorspann.tex mit nicht gesetztem Schalter.

\newif\ifkorrekturansicht
\korrekturansichtfalse

\input{../tex-inputs/latex-vorspann}


\section[Arthur Schnitzler an Berta Zuckerkandl, 10. 8. 1929]{L03984 Arthur Schnitzler an Berta Zuckerkandl, 10. 8. 1929}
\nopagebreak\mylabel{L03984v}
\rehead{ }\normalsize\beginnumbering\briefempfaengerindex{Zuckerkandl, Berta@\textsc{Zuckerkandl, Berta}!zzzSchnitzler, Arthur@\emph{von Arthur Schnitzler}!1929-08-101@{10. 8. 1929}|(be}
\toendnotes[C]{\smallbreak\pagebreak[2]}
\correspDesc{Versand  durch Arthur Schnitzler am 10. 8. 1929 in Wien
\newline{}Erhalt  durch Berta Zuckerkandl im Zeitraum [11. 8. 1929
                  – 15. 8. 1929?] \textbf{Ort fehlend} }\toendnotes[C]{\smallbreak}
\Standort{Wien, Österreichische Nationalbibliothek, 405/B78/5 LIT MAG.}
\physDesc{Brief, 1 Blatt, 2 Seiten, 859 Zeichen (Briefpapier mit Trauerrand)
\newline{}Handschrift: schwarze Tinte, lateinische Kurrent}\toendnotes[C]{\smallbreak}
\pstart
           \raggedleft{}{\pb}Wien\oindex{Wien@\textbf{Wien}, \emph{Verwaltungsgebiet}|pw}{ }10. 8. 929.\pend
           \vspace{0.5em}
\pstart
           Liebe verehrte Freundin, zu der \label{K_L03984-1v}\edtext{Ehrenlegion\orgindex{ordre national de la Légion d’Honneur@L’ordre national de la Légion d’Honneur|pw}}{\lemma{\textnormal{\emph{Ehrenlegion}}}\Cendnote{\textnormal{Berta Zuckerkandl\pwindex{Zuckerkandl, Berta 13.\,4.\,1864 Wien – 16.\,10.\,1945 Paris@\textsc{Zuckerkandl, Berta} (13.\,4.\,1864 Wien – 16.\,10.\,1945 Paris), \emph{Schriftstellerin, Journalistin, Übersetzerin}|pwk} erhielt 1929
                  für ihre Verdienste bei der Vermittlung französischer\oindex{Frankreich@\textbf{Frankreich}|pwk} Literatur den französischen\oindex{Frankreich@\textbf{Frankreich}|pwk} Verdienstorden als Ritter der \emph{Ehrenlegion}\orgindex{ordre national de la Légion d’Honneur@L’ordre national de la Légion d’Honneur|pwk} verliehen.}}}\label{K_L03984-1} gratulire ich Ihnen von
               ganzem Herzen. De{\geminationn} diesmal ist die Auszeichnung nichts
               weniger als eine leere Form, und selten wird sie so nach Verdienst ertheilt. Daſs Sie
               nebstbei, minder sichtbar, \introOben{}schon lange\introOben{} eine Ehrenlegion des
               Herzens tragen, und daſs, unter anderm auch darum, kaum einer der zahlreichen
               Glückwünsche, die Sie heute empfangen werden, nicht ehrlich empfunden sein dürfte,
               dieses Bewußtsein wird Ihnen hoffentlich auch eine gewisse Genugthuung gewähren.
               Bleiben Sie liebe verehrte Freundin was Sie sind, gütig, {\pb}thätig und jung! Wie gern hab ich diese
               Gelegenheit ergriffen Ihnen meine tiefste und wärmste Sympathie auszudrücken.
               Erhalten Sie mir auch die Ihre. Ich rechne sie mit zu meinem werthvollsten seelischen
               Besitz.\pend
           
\pstart
           Tausend Grüße; und alles Freundliche auch den Ihren!{\\[\baselineskip]}Ihr{\\[\baselineskip]}\spacefill\mbox{ArthurSchnitzler}\pend
           \leftskip=0em{}\selectlanguage{ngerman}\endnumbering\briefempfaengerindex{Zuckerkandl, Berta@\textsc{Zuckerkandl, Berta}!zzzSchnitzler, Arthur@\emph{von Arthur Schnitzler}!1929-08-101@{10. 8. 1929}|)be}\mylabel{L03984h}
\begin{anhang}
\end{anhang}\newcommand{\dateiname}{L03984}\newcommand{\titel}{Arthur Schnitzler an Berta Zuckerkandl, 10. 8. 1929}\newcommand{\editorInnen}{Herausgegeben von Jahnke, SelmaMüller, Martin Anton}%% latex-leseansicht-abspann.tex
%% Abspann für die Leseansicht.
%% Der Schalter \ifkorrekturansicht ist bereits durch den Vorspann gesetzt.

%% latex-abspann.tex
%% Gemeinsamer Abspann für Korrekturansicht und Leseansicht.
%% Setzt den Schalter \ifkorrekturansicht voraus (gesetzt in den
%% einbindenden Dateien latex-korrekturansicht-abspann.tex bzw.
%% latex-leseansicht-abspann.tex).
%% ---------------------------------------------------------------

\normalsize

% Das esempio-Environment wird nur in der Leseansicht benötigt
\ifkorrekturansicht\else
\newenvironment{esempio}[3]%
{
    \vspace{1.5ex}
    \rlap{\underline{#1}}
    \par
    \setlength{\parindent}{0cm}
    \nopagebreak
    \leftskip=#2cm
    \rightskip=#3cm
}
{
    \par
}
\fi

\doendnotes{C}
\bigskip
\vfill

\clearpage

\footnotesize

\ifkorrekturansicht
  \lohead{\textsc{register}}
\fi

% theindex-Environment neu definieren ohne reledmac
\makeatletter
\renewenvironment{theindex}{%
  \ifkorrekturansicht
    \section*{\indexname}%
  \else
    \subsubsection*{Index der erwähnten Entitäten}%
  \fi
  \setlength{\parindent}{0pt}%
  \setlength{\parskip}{0pt plus 0.3pt}%
  \let\item\@idxitem
}{%
  \ifkorrekturansicht\clearpage\fi
}
\makeatother

\IfFileExists{\jobname-pw.ind}{\input{\jobname-pw.ind}}{}

% Quellenangabe nur in der Leseansicht
\ifkorrekturansicht\else
% Fallback-Definitionen, falls die .tex-Datei \titel etc. nicht gesetzt hat
\providecommand{\titel}{}
\providecommand{\editorInnen}{}
\providecommand{\dateiname}{\jobname}

\vspace{3cm}

\vfill

\footnotesize
\textsc{Quelle}: \titel. Herausgegeben von {\editorInnen}. In: \emph{Arthur Schnitzler: Briefwechsel mit Autorinnen und Autoren}.
 Digitale Edition, https://schnitzler-briefe.acdh.oeaw.ac.at/{\dateiname}.html (Stand \today)
\fi

\end{document}


