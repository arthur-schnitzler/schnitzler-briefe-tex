%% latex-leseansicht-vorspann.tex
%% Vorspann für die Leseansicht.
%% Lädt die gemeinsame Datei latex-vorspann.tex mit nicht gesetztem Schalter.

\newif\ifkorrekturansicht
\korrekturansichtfalse

\input{../tex-inputs/latex-vorspann}


\section[Arthur Schnitzler an Georg Brandes, 25. 4. 1901]{L01114 Arthur Schnitzler an Georg Brandes, 25. 4. 1901}
\nopagebreak\mylabel{L01114v}
\rehead{ }\normalsize\beginnumbering\briefempfaengerindex{Brandes, Georg@\textsc{Brandes, Georg}!zzzSchnitzler, Arthur@\emph{von Arthur Schnitzler}!1901-04-251@{25. 4. 1901}|(be}
\toendnotes[C]{\smallbreak\pagebreak[2]}
\correspDesc{Versand  durch Arthur Schnitzler am 25. 4. 1901 in Wien
\newline{}Erhalt  durch Georg Brandes im Zeitraum [25. 4. 1901
                  – 29. 4. 1901?] \textbf{Ort fehlend} }\toendnotes[C]{\smallbreak}
\Standort{Kopenhagen, Det Kongelige Bibliotek, Georg Brandes Arkiv, box 125.}
\physDesc{Brief, 2 Blätter, 8 Seiten, 2751 Zeichen
\newline{}Handschrift: schwarze Tinte, deutsche Kurrent
\newline{}Ordnung: mit Bleistift von unbekannter Hand nummeriert: »21. \textsc{Schnitzler}«, die Datierung auf der ersten Seite des zweiten Blattes
                                 mit Bleistift wiederholt }
\buchAbdrucke{\weitereDrucke{Georg Brandes, Arthur Schnitzler: \emph{Ein Briefwechsel}. Herausgegeben von Kurt Bergel. Bern: \emph{Francke} 1956, S. 83–84.} }\toendnotes[C]{\smallbreak}
\pstart
           \raggedleft{}{\pb}Wien\oindex{Wien@\textbf{Wien}, \emph{Verwaltungsgebiet}|pw}, 25. 4. 901.\pend
           
\pstart{}Lieber Herr Brandes,\pend\vspace{0.5em}
\pstart
           \label{K_L01114-1v}\edtext{\textsc{Paul Goldmann}\pwindex{Goldmann, Paul 31.\,1.\,1865 Breslau – 25.\,9.\,1935 Wien@\textsc{Goldmann, Paul} (31.\,1.\,1865 Breslau – 25.\,9.\,1935 Wien), \emph{Schriftsteller, Journalist}|pw}}{\lemma{\textnormal{\emph{Paul Goldmann}}}\Cendnote{\textnormal{XXXX Auszeichnungsfehler: Dokument L03064 nicht gefunden.
               }}}\label{K_L01114-1} hat mir \textsc{Politiken}\pwindex{Politiken@\emph{Politiken}|pw} mit Ihrem \label{K_L01114-2v}\edtext{Artikel\pwindex{Brandes, Georg 4.\,2.\,1842 Kopenhagen – 19.\,2.\,1927 ebd.@\textsc{Brandes, Georg} (4.\,2.\,1842 Kopenhagen – 19.\,2.\,1927 ebd.)!Skikkelser og Tanker. Arthur Schnitzler@\strich\emph{Skikkelser og Tanker. Arthur Schnitzler}|pwv}}{\lemma{\textnormal{\emph{Artikel}}}\Cendnote{\textnormal{Georg Brandes\pwindex{Brandes, Georg 4.\,2.\,1842 Kopenhagen – 19.\,2.\,1927 ebd.@\textsc{Brandes, Georg} (4.\,2.\,1842 Kopenhagen – 19.\,2.\,1927 ebd.)|pwk}: \emph{Skikkelser og Tanker. Arthur Schnitzler}\pwindex{Brandes, Georg 4.\,2.\,1842 Kopenhagen – 19.\,2.\,1927 ebd.@\textsc{Brandes, Georg} (4.\,2.\,1842 Kopenhagen – 19.\,2.\,1927 ebd.)!Skikkelser og Tanker. Arthur Schnitzler@\strich\emph{Skikkelser og Tanker. Arthur Schnitzler}|pwk}. In: \emph{Politiken}\pwindex{Politiken@\emph{Politiken}|pwk}, Nr. 98, 9. 4. 1901,
                     S. 1. Parallel dazu kam es zu einem zweiten Abdruck, der sich in Schnitzlers Zeitungsausschnitten
                     (Exeter, box 37/2) findet und aus \emph{Göteborgs Handels- och Sjöfartstidning}\orgindex{Göteborgs Handels- och Sjöfartstidning@Göteborgs Handels- och Sjöfartstidning|pwk} vom 9. 4. 1901
                  stammt.}}}\label{K_L01114-2} über mich geſandt und ich verſuchte däniſch\oindex{Dänemark@\textbf{Dänemark}|pw} zu verſtehen, was mir nur zum Theil gelang; die Neue Freie Preſſe\pwindex{Neue Freie Presse@\emph{Neue Freie Presse}|pw} kam mir zu \label{K_L01114-3v}\edtext{Hilfe\pwindex{Brandes, Georg 4.\,2.\,1842 Kopenhagen – 19.\,2.\,1927 ebd.@\textsc{Brandes, Georg} (4.\,2.\,1842 Kopenhagen – 19.\,2.\,1927 ebd.)!Arthur Schnitzler@\strich\emph{Arthur Schnitzler}|pwv}}{\lemma{\textnormal{\emph{Hilfe}}}\Cendnote{\textnormal{Georg Brandes\pwindex{Brandes, Georg 4.\,2.\,1842 Kopenhagen – 19.\,2.\,1927 ebd.@\textsc{Brandes, Georg} (4.\,2.\,1842 Kopenhagen – 19.\,2.\,1927 ebd.)|pwk}: \emph{Arthur Schnitzler}\pwindex{Brandes, Georg 4.\,2.\,1842 Kopenhagen – 19.\,2.\,1927 ebd.@\textsc{Brandes, Georg} (4.\,2.\,1842 Kopenhagen – 19.\,2.\,1927 ebd.)!Arthur Schnitzler@\strich\emph{Arthur Schnitzler}|pwk}. In: \emph{Neue Freie Presse}\pwindex{Neue Freie Presse@\emph{Neue Freie Presse}|pwk}, Nr. 13.166, 21. 4. 1901, Morgenblatt,
                     S. 32–33.}}}\label{K_L01114-3} – und Sie können{ }ſich denken, wie{ }ſehr ich mich gefreut
               habe, als ich nun alles, was Sie über mich{ }ſchrieben, we{\geminationn} auch nur in der Überſetzung leſen konnte. Laſſen Sie mich Ihnen die Hand drücken –
               und {\pb}weiter nichts{ }ſagen – wie es Ihnen ja gewiſs
               am liebſten iſt.\pend
           
\pstart
           Sie haben hoffentlich meine Karte aus Rom\oindex{Rom@\textbf{Rom}, \emph{Hauptstadt}|pw}
               bekommen und wiſſen, dſs ich \textsc{Ellen Key}\pwindex{Key, Ellen 11.\,12.\,1849 Sundsholm – 25.\,4.\,1926 Gut ”Strand“ [Vättersee]@\textsc{Key, Ellen} (11.\,12.\,1849 Sundsholm – 25.\,4.\,1926 Gut ”Strand“ [Vättersee]), \emph{Schriftstellerin, Schriftstellerin, Pädagogin}|pw} ke{\geminationn}engelernt habe, die mir zu meiner Freude
               erzählte, dſs Sie den letzten Winter in vollkommener Geſundheit verbracht haben.
               Wenige Tage nachdem ich \textsc{Ellen Key}\pwindex{Key, Ellen 11.\,12.\,1849 Sundsholm – 25.\,4.\,1926 Gut ”Strand“ [Vättersee]@\textsc{Key, Ellen} (11.\,12.\,1849 Sundsholm – 25.\,4.\,1926 Gut ”Strand“ [Vättersee]), \emph{Schriftstellerin, Schriftstellerin, Pädagogin}|pw}, deren Weſen mir wahrhaft wohl that, bei \textsc{Wasserma{\geminationn}s}\pwindex{Wassermann, Jakob 10.\,3.\,1873 Fürth – 1.\,1.\,1934 Altaussee@\textsc{Wassermann, Jakob} (10.\,3.\,1873 Fürth – 1.\,1.\,1934 Altaussee), \emph{Schriftsteller}|pw}\pwindex{Wassermann, Julie 5.\,12.\,1876 Wien – April 1963 Zürich@\textsc{Wassermann, Julie} (5.\,12.\,1876 Wien – April 1963 Zürich), \emph{Schriftstellerin}|pw} kennen gelernt, traf ich{ }ſie ein zweites Mal und {\pb}\textsc{Helge Rhode}\pwindex{Rode, Helge 16.\,10.\,1870 Kopenhagen – 23.\,3.\,1937 Frederiksberg@\textsc{Rode, Helge} (16.\,10.\,1870 Kopenhagen – 23.\,3.\,1937 Frederiksberg), \emph{Schriftsteller}|pw}, den{ }ſie mitbrachte. Ich war kaum zwei Wochen in Rom\oindex{Rom@\textbf{Rom}, \emph{Hauptstadt}|pw}, eben genug, um zu wiſſen, wie man es ein nächſtes Mal anzufangen hat,
               um{ }ſeine Zeit gut auszunützen. Von Rom\oindex{Rom@\textbf{Rom}, \emph{Hauptstadt}|pw} ging ich
               nach Florenz\oindex{Florenz@\textbf{Florenz}|pw}, wo ich mit meiner Mama\pwindex{Schnitzler, Louise 8.\,7.\,1840 Kőszeg – 9.\,9.\,1911 Wien@\textsc{Schnitzler, Louise} (8.\,7.\,1840 Kőszeg – 9.\,9.\,1911 Wien)|pwv} Rendezvous hatte – aber den Frühling
               fand ich nirgends. Man fror beinah immer.\pend
           
\pstart
           Sie waren – oder{ }ſind noch? – in Berlin\oindex{Berlin@\textbf{Berlin}, \emph{Hauptstadt}|pw}, wie mir
                  Georg Hirſchfeld\pwindex{Hirschfeld, Georg 11.\,2.\,1873 Berlin – 17.\,1.\,1942 München@\textsc{Hirschfeld, Georg} (11.\,2.\,1873 Berlin – 17.\,1.\,1942 München), \emph{Schriftsteller}|pw}{ }{\pb}ſchrieb; wann ko{\geminationm}en
               Sie wieder zu uns? Sie würden nicht viel verändert finden – \textsc{Beer Hofmann}\pwindex{Beer-Hofmann, Richard 11.\,7.\,1866 Wien – 26.\,9.\,1945 New York City@\textsc{Beer-Hofmann, Richard} (11.\,7.\,1866 Wien – 26.\,9.\,1945 New York City), \emph{Schriftsteller}|pw} hat nun auch zu{ }ſeinen Töchtern\pwindex{Beer-Hofmann, Mirjam 4.\,9.\,1897 Wien – 24.\,12.\,1984 New York City@\textsc{Beer-Hofmann, Mirjam} (4.\,9.\,1897 Wien – 24.\,12.\,1984 New York City)|pwv}\pwindex{Beer-Hofmann, Naëmah 20.\,12.\,1898 Wien – 10.\,11.\,1971 New York City@\textsc{Beer-Hofmann, Naëmah} (20.\,12.\,1898 Wien – 10.\,11.\,1971 New York City)|pwv} einen Sohn\pwindex{Beer-Hofmann, Gabriel 9.\,1.\,1901 Wien – 24.\,3.\,1971 St Albans@\textsc{Beer-Hofmann, Gabriel} (9.\,1.\,1901 Wien – 24.\,3.\,1971 St Albans), \emph{Schriftsteller, Filmagent}|pwv} beko{\geminationm}en, aber von dem iſt
               begreiflicherweiſe noch nicht viel zu erzählen. Ich werde diesmal wahrſcheinlich{ }ſehr
               bald ins Gebirge reiſen; und nach mancherlei Kleinigkeiten, die ich in der letzten
               Zeit gemacht, mich wohl endlich wieder \introOben{}an\introOben{} was größeres {\pb}wagen. Einen kleinen Roman\pwindex{Schnitzler, Arthur 15.\,5.\,1862 Wien – 21.\,10.\,1931 ebd.@\textsc{Schnitzler, Arthur} (15.\,5.\,1862 Wien – 21.\,10.\,1931 ebd.), \emph{Schriftsteller, Mediziner}!Frau Bertha Garlan. Roman@\strich\emph{Frau Bertha Garlan. Roman}|pwv}, den ich vorigen Winter{ }ſchrieb, haben Sie wohl{ }ſchon erhalten. Die \textsc{Beatrice}\pwindex{Schnitzler, Arthur 15.\,5.\,1862 Wien – 21.\,10.\,1931 ebd.@\textsc{Schnitzler, Arthur} (15.\,5.\,1862 Wien – 21.\,10.\,1931 ebd.), \emph{Schriftsteller, Mediziner}!Schleier der Beatrice. Schauspiel in fünf Akten@\strich\emph{Der Schleier der Beatrice. Schauspiel in fünf Akten}|pw} iſt im Dezember einige Male in Breſlau\oindex{Breslau@\textbf{Breslau}|pw} geſpielt worden, ohne beſonderes Glück. Auch war die Darſtellung
               recht{ }ſchwach. Eine gute Aufführung müßte dem Stück wohl Erfolg bringen. Aber das Burgtheater\oindex{Wien@\textbf{Wien}!I., Innere Stadt@\textbf{I., Innere Stadt}!Burgtheater@\textbf{Burgtheater}, \emph{Theater}|pw} hat wichtigeres zu thun. –\pend
           \pstart Leben Sie wohl und{ }ſeien Sie herzlich gegrüßt von Ihrem treuen
                  \spacefill\mbox{ArthurSchnitzler}\pend{}
\pstart
           \noindent{}{\pb}Dieſer Tage erſcheint eine Novelle von mir,
                  die ich Ihnen natürlich{ }ſchicken werde, Lieutenant
                     Guſtl\pwindex{Schnitzler, Arthur 15.\,5.\,1862 Wien – 21.\,10.\,1931 ebd.@\textsc{Schnitzler, Arthur} (15.\,5.\,1862 Wien – 21.\,10.\,1931 ebd.), \emph{Schriftsteller, Mediziner}!Lieutenant Gustl. Novelle@\strich\emph{Lieutenant Gustl. Novelle}|pw}, – Sie haben{ }ſie vielleicht in der N. Fr. Pr.\orgindex{Neue Freie Presse@Neue Freie Presse|pw} geleſen. Wegen dieſer Novelle{ }ſtehe ich – (da ich noch \textsc{Militärarzt} »in der Evidenz« bin) in »ehrengerichtlicher«
                  Unterſuchung und werde wahrſcheinlich meine \textsc{Charge}
                  verlieren. Wenn Sie die Novelle {\pb}noch nicht
                  kennen und{ }ſie leſen werden – und{ }ſich dieſer Mittheilung erinnern – wird Ihnen
                  wieder manches »oeſterreichiſche\oindex{Österreich@\textbf{Österreich}|pw}« klar
                     werden.\hspace*{1.5em}Die Sache iſt für mich natürlich
                  gleichgiltig – da ich ja mit den Leuten nichts mehr zu thun habe und meine Charge
                  nur im Kriegsfall von Bedeutung wäre – aber{ }ſie iſt charakteriſtiſch für {\pb}die man könnte{ }ſagen naïve Heuchelei in
                  Kreiſen, von denen man in gewiſſem Sinne i{\geminationm}er
                  abhängig iſt; we{\geminationn}{ }ſie auch keine unmittelbare Macht über einen
                  beſitzen.\pend
           
\pstart
           Ihr \spacefill\mbox{A. S.}\pend
           \selectlanguage{ngerman}\endnumbering\briefempfaengerindex{Brandes, Georg@\textsc{Brandes, Georg}!zzzSchnitzler, Arthur@\emph{von Arthur Schnitzler}!1901-04-251@{25. 4. 1901}|)be}\mylabel{L01114h}  \newcommand{\dateiname}{L01114}\newcommand{\titel}{Arthur Schnitzler an Georg Brandes, 25. 4. 1901}\newcommand{\editorInnen}{Martin Anton Müller und Gerd-Hermann Susen}%% latex-leseansicht-abspann.tex
%% Abspann für die Leseansicht.
%% Der Schalter \ifkorrekturansicht ist bereits durch den Vorspann gesetzt.

%% latex-abspann.tex
%% Gemeinsamer Abspann für Korrekturansicht und Leseansicht.
%% Setzt den Schalter \ifkorrekturansicht voraus (gesetzt in den
%% einbindenden Dateien latex-korrekturansicht-abspann.tex bzw.
%% latex-leseansicht-abspann.tex).
%% ---------------------------------------------------------------

\normalsize

% Das esempio-Environment wird nur in der Leseansicht benötigt
\ifkorrekturansicht\else
\newenvironment{esempio}[3]%
{
    \vspace{1.5ex}
    \rlap{\underline{#1}}
    \par
    \setlength{\parindent}{0cm}
    \nopagebreak
    \leftskip=#2cm
    \rightskip=#3cm
}
{
    \par
}
\fi

\doendnotes{C}
\bigskip
\vfill

\clearpage

\footnotesize

\ifkorrekturansicht
  \lohead{\textsc{register}}
\fi

% theindex-Environment neu definieren ohne reledmac
\makeatletter
\renewenvironment{theindex}{%
  \ifkorrekturansicht
    \section*{\indexname}%
  \else
    \subsubsection*{Index der erwähnten Entitäten}%
  \fi
  \setlength{\parindent}{0pt}%
  \setlength{\parskip}{0pt plus 0.3pt}%
  \let\item\@idxitem
}{%
  \ifkorrekturansicht\clearpage\fi
}
\makeatother

\IfFileExists{\jobname-pw.ind}{\input{\jobname-pw.ind}}{}

% Quellenangabe nur in der Leseansicht
\ifkorrekturansicht\else
% Fallback-Definitionen, falls die .tex-Datei \titel etc. nicht gesetzt hat
\providecommand{\titel}{}
\providecommand{\editorInnen}{}
\providecommand{\dateiname}{\jobname}

\vspace{3cm}

\vfill

\footnotesize
\textsc{Quelle}: \titel. Herausgegeben von {\editorInnen}. In: \emph{Arthur Schnitzler: Briefwechsel mit Autorinnen und Autoren}.
 Digitale Edition, https://schnitzler-briefe.acdh.oeaw.ac.at/{\dateiname}.html (Stand \today)
\fi

\end{document}


