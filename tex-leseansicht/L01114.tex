%% latex-korrekturansicht-vorspann.tex
%% Vorspann für die Korrekturansicht.
%% Lädt die gemeinsame Datei latex-vorspann.tex mit gesetztem Schalter.

\newif\ifkorrekturansicht
\korrekturansichttrue

\input{../tex-inputs/latex-vorspann}


\section[Arthur Schnitzler an Georg Brandes, 25. 4. 1901]{L01114 Arthur Schnitzler an Georg Brandes, 25. 4. 1901}
\nopagebreak\mylabel{L01114v}
\rehead{ }\normalsize\beginnumbering\briefempfaengerindex{Brandes, Georg@\textsc{Brandes, Georg}!zzzSchnitzler, Arthur@\emph{von Arthur Schnitzler}!1901-04-251@{25. 4. 1901}|(be}
\toendnotes[C]{\smallbreak\pagebreak[2]}\Standort{Kopenhagen, Det Kongelige Bibliotek, Georg Brandes Arkiv, box 125.}
\physDesc{Brief, 2 Blätter, 8 Seiten, 2751 Zeichen
\newline{}Handschrift: schwarze Tinte, deutsche Kurrent
\newline{}Ordnung: mit Bleistift von unbekannter Hand nummeriert: »21. \textsc{Schnitzler}«, die Datierung auf der ersten Seite des zweiten Blattes
                                 mit Bleistift wiederholt }
\buchAbdrucke{\weitereDrucke{Georg Brandes, Arthur Schnitzler: \emph{Ein Briefwechsel}. Bern: \emph{Francke} 1956, S. 83–84.} }\toendnotes[C]{\smallbreak}
\pstart
           \raggedleft{}{\pb}Wien\oindex{Wien@\textbf{Wien}, \emph{A.ADM2}|pw}, 25. 4. 901.\pend
           
\pstart{}Lieber Herr Brandes,\pend\vspace{0.5em}
\pstart
           \label{K_L01114-1v}\edtext{\textsc{Paul Goldmann}\pwindex{Goldmann, Paul 31.01.1865 – 25.09.1935@\textsc{Goldmann, Paul} (31.01.1865 – 25.09.1935), \emph{Schriftsteller/Schriftstellerin, Journalist/Journalistin}|pw}}{\lemma{\textnormal{\emph{Paul Goldmann}}}\Cendnote{\textnormal{Paul Goldmann an Arthur Schnitzler, 26. 4. [1901].
               }}}\label{K_L01114-1} hat mir \textsc{Politiken}\pwindex{Politiken@\emph{Politiken}|pw} mit Ihrem \label{K_L01114-2v}\edtext{Artikel\pwindex{Skikkelser og Tanker. Arthur Schnitzler@\emph{Skikkelser og Tanker. Arthur Schnitzler}|pwv}}{\lemma{\textnormal{\emph{Artikel}}}\Cendnote{\textnormal{Georg Brandes\pwindex{Brandes, Georg 04.02.1842 – 19.02.1927@\textsc{Brandes, Georg} (04.02.1842 – 19.02.1927)|pwk}: \emph{Skikkelser og Tanker. Arthur Schnitzler}\pwindex{Skikkelser og Tanker. Arthur Schnitzler@\emph{Skikkelser og Tanker. Arthur Schnitzler}|pwk}. In: \emph{Politiken}\pwindex{Politiken@\emph{Politiken}|pwk}, Nr. 98, 9. 4. 1901,
                     S. 1. Parallel dazu kam es zu einem zweiten Abdruck, der sich in Schnitzlers Zeitungsausschnitten
                     (Exeter, box 37/2) findet und aus \emph{Göteborgs Handels- och Sjöfartstidning}\orgindex{Goeteborgs Handels- och Sjoefartstidning@Göteborgs Handels- och Sjöfartstidning|pwk} vom 9. 4. 1901
                  stammt.}}}\label{K_L01114-2} über mich geſandt und ich verſuchte däniſch\oindex{Daenemark@\textbf{Dänemark}, \emph{A.PCLI}|pw} zu verſtehen, was mir nur zum Theil gelang; die Neue Freie Preſſe\pwindex{Neue Freie Presse@\emph{Neue Freie Presse}|pw} kam mir zu \label{K_L01114-3v}\edtext{Hilfe\pwindex{Arthur Schnitzler@\emph{Arthur Schnitzler}|pwv}}{\lemma{\textnormal{\emph{Hilfe}}}\Cendnote{\textnormal{Georg Brandes\pwindex{Brandes, Georg 04.02.1842 – 19.02.1927@\textsc{Brandes, Georg} (04.02.1842 – 19.02.1927)|pwk}: \emph{Arthur Schnitzler}\pwindex{Arthur Schnitzler@\emph{Arthur Schnitzler}|pwk}. In: \emph{Neue Freie Presse}\pwindex{Neue Freie Presse@\emph{Neue Freie Presse}|pwk}, Nr. 13.166, 21. 4. 1901, Morgenblatt,
                     S. 32–33.}}}\label{K_L01114-3} – und Sie können ſich denken, wie ſehr ich mich gefreut
               habe, als ich nun alles, was Sie über mich ſchrieben, we{\geminationn} auch nur in der Überſetzung leſen konnte. Laſſen Sie mich Ihnen die Hand drücken –
               und {\pb}weiter nichts ſagen – wie es Ihnen ja gewiſs
               am liebſten iſt.\pend
           
\pstart
           Sie haben hoffentlich meine Karte aus Rom\oindex{Rom@\textbf{Rom}, \emph{P.PPLC}|pw}
               bekommen und wiſſen, dſs ich \textsc{Ellen Key}\pwindex{Key, Ellen 11.12.1849 – 25.04.1926@\textsc{Key, Ellen} (11.12.1849 – 25.04.1926), \emph{Schriftsteller/Schriftstellerin, Schriftsteller/Schriftstellerin, Pädagoge/Pädagogin}|pw} ke{\geminationn}engelernt habe, die mir zu meiner Freude
               erzählte, dſs Sie den letzten Winter in vollkommener Geſundheit verbracht haben.
               Wenige Tage nachdem ich \textsc{Ellen Key}\pwindex{Key, Ellen 11.12.1849 – 25.04.1926@\textsc{Key, Ellen} (11.12.1849 – 25.04.1926), \emph{Schriftsteller/Schriftstellerin, Schriftsteller/Schriftstellerin, Pädagoge/Pädagogin}|pw}, deren Weſen mir wahrhaft wohl that, bei \textsc{Wasserma{\geminationn}s}\pwindex{Wassermann, Jakob 10.03.1873 – 01.01.1934@\textsc{Wassermann, Jakob} (10.03.1873 – 01.01.1934), \emph{Schriftsteller/Schriftstellerin}|pw}\pwindex{Wassermann, Julie 05.12.1876 – April 1963@\textsc{Wassermann, Julie} (05.12.1876 – April 1963), \emph{Schriftsteller/Schriftstellerin}|pw} kennen gelernt, traf ich ſie ein zweites Mal und {\pb}\textsc{Helge Rhode}\pwindex{Rode, Helge 16.10.1870 – 23.03.1937@\textsc{Rode, Helge} (16.10.1870 – 23.03.1937), \emph{Schriftsteller/Schriftstellerin}|pw}, den ſie mitbrachte. Ich war kaum zwei Wochen in Rom\oindex{Rom@\textbf{Rom}, \emph{P.PPLC}|pw}, eben genug, um zu wiſſen, wie man es ein nächſtes Mal anzufangen hat,
               um ſeine Zeit gut auszunützen. Von Rom\oindex{Rom@\textbf{Rom}, \emph{P.PPLC}|pw} ging ich
               nach Florenz\oindex{Florenz@\textbf{Florenz}, \emph{P.PPLA}|pw}, wo ich mit meiner Mama\pwindex{Schnitzler, Louise 1840-07-08 – 1911-09-09@\textsc{Schnitzler, Louise} (1840-07-08 – 1911-09-09)|pwv} Rendezvous hatte – aber den Frühling
               fand ich nirgends. Man fror beinah immer.\pend
           
\pstart
           Sie waren – oder ſind noch? – in Berlin\oindex{Berlin@\textbf{Berlin}, \emph{P.PPLC}|pw}, wie mir
                  Georg Hirſchfeld\pwindex{Hirschfeld, Georg 11.02.1873 – 17.01.1942@\textsc{Hirschfeld, Georg} (11.02.1873 – 17.01.1942), \emph{Schriftsteller/Schriftstellerin}|pw}{ }{\pb}ſchrieb; wann ko{\geminationm}en
               Sie wieder zu uns? Sie würden nicht viel verändert finden – \textsc{Beer Hofmann}\pwindex{Beer-Hofmann, Richard 1866-07-11 – 1945-09-26@\textsc{Beer-Hofmann, Richard} (1866-07-11 – 1945-09-26), \emph{Schriftsteller/Schriftstellerin}|pw} hat nun auch zu ſeinen Töchtern\pwindex{Beer-Hofmann, Mirjam 04.09.1897 – 24.12.1984@\textsc{Beer-Hofmann, Mirjam} (04.09.1897 – 24.12.1984)|pwv}\pwindex{Beer-Hofmann, Naemah 20.12.1898 – 10.11.1971@\textsc{Beer-Hofmann, Naëmah} (20.12.1898 – 10.11.1971)|pwv} einen Sohn\pwindex{Beer-Hofmann, Gabriel 09.01.1901 – 24.03.1971@\textsc{Beer-Hofmann, Gabriel} (09.01.1901 – 24.03.1971), \emph{Schriftsteller/Schriftstellerin, Filmagent/Filmagentin}|pwv} beko{\geminationm}en, aber von dem iſt
               begreiflicherweiſe noch nicht viel zu erzählen. Ich werde diesmal wahrſcheinlich ſehr
               bald ins Gebirge reiſen; und nach mancherlei Kleinigkeiten, die ich in der letzten
               Zeit gemacht, mich wohl endlich wieder \introOben{}an\introOben{} was größeres {\pb}wagen. Einen kleinen Roman\pwindex{Frau Bertha Garlan. Roman@\emph{Frau Bertha Garlan. Roman}|pwv}, den ich vorigen Winter{ }ſchrieb, haben Sie wohl ſchon erhalten. Die \textsc{Beatrice}\pwindex{Schleier der Beatrice. Schauspiel in fuenf Akten@\emph{Der Schleier der Beatrice. Schauspiel in fünf Akten}|pw} iſt im Dezember einige Male in Breſlau\oindex{Breslau@\textbf{Breslau}, \emph{P.PPLA}|pw} geſpielt worden, ohne beſonderes Glück. Auch war die Darſtellung
               recht ſchwach. Eine gute Aufführung müßte dem Stück wohl Erfolg bringen. Aber das Burgtheater\oindex{Burgtheater@\textbf{Burgtheater}, \emph{S.THTR}|pw} hat wichtigeres zu thun. –\pend
           \pstart Leben Sie wohl und ſeien Sie herzlich gegrüßt von Ihrem treuen
                  \spacefill\mbox{ArthurSchnitzler}\pend{}
\pstart
           \noindent{}{\pb}Dieſer Tage erſcheint eine Novelle von mir,
                  die ich Ihnen natürlich ſchicken werde, Lieutenant
                     Guſtl\pwindex{Lieutenant Gustl. Novelle@\emph{Lieutenant Gustl. Novelle}|pw}, – Sie haben ſie vielleicht in der N. Fr. Pr.\orgindex{Neue Freie Presse@Neue Freie Presse|pw} geleſen. Wegen dieſer Novelle ſtehe ich – (da ich noch \textsc{Militärarzt} »in der Evidenz« bin) in »ehrengerichtlicher«
                  Unterſuchung und werde wahrſcheinlich meine \textsc{Charge}
                  verlieren. Wenn Sie die Novelle {\pb}noch nicht
                  kennen und ſie leſen werden – und ſich dieſer Mittheilung erinnern – wird Ihnen
                  wieder manches »oeſterreichiſche\oindex{Oesterreich@\textbf{Österreich}, \emph{A.PCLI}|pw}« klar
                     werden.\hspace*{1.5em}Die Sache iſt für mich natürlich
                  gleichgiltig – da ich ja mit den Leuten nichts mehr zu thun habe und meine Charge
                  nur im Kriegsfall von Bedeutung wäre – aber ſie iſt charakteriſtiſch für {\pb}die man könnte ſagen naïve Heuchelei in
                  Kreiſen, von denen man in gewiſſem Sinne i{\geminationm}er
                  abhängig iſt; we{\geminationn}{ }ſie auch keine unmittelbare Macht über einen
                  beſitzen.\pend
           
\pstart
           Ihr \spacefill\mbox{A. S.}\pend
           \selectlanguage{ngerman}\endnumbering\briefempfaengerindex{Brandes, Georg@\textsc{Brandes, Georg}!zzzSchnitzler, Arthur@\emph{von Arthur Schnitzler}!1901-04-251@{25. 4. 1901}|)be}\mylabel{L01114h}  \normalsize

\doendnotes{C}
\bigskip
\vfill

\clearpage

\footnotesize

\lohead{\textsc{register}}

% Definiere theindex-Environment komplett neu ohne reledmac
\makeatletter
\renewenvironment{theindex}{%
  \section*{\indexname}%
  \setlength{\parindent}{0pt}%
  \setlength{\parskip}{0pt plus 0.3pt}%
  \let\item\@idxitem
}{%
  \clearpage
}
\makeatother

\IfFileExists{\jobname-pw.ind}{\input{\jobname-pw.ind}}{}

\end{document}

      