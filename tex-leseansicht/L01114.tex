%% latex-leseansicht-vorspann.tex
%% Vorspann für die Leseansicht.
%% Lädt die gemeinsame Datei latex-vorspann.tex mit nicht gesetztem Schalter.

\newif\ifkorrekturansicht
\korrekturansichtfalse

\input{../tex-inputs/latex-vorspann}


               \section[Arthur Schnitzler an Georg Brandes, 25. 4. 1901]{ Arthur Schnitzler an Georg Brandes, 25. 4. 1901}\nopagebreak\mylabel{v}\rehead{ }\begin{ledgroupsized}[t]{13cm}\normalsize\beginnumbering\briefempfaengerindex{Brandes, Georg@\textsc{Brandes, Georg}!zzzSchnitzler, Arthur@\emph{von Arthur Schnitzler}!1901-04-251@{25. 4. 1901}|(be} \toendnotes[C]{\smallbreak\pagebreak[2]} \Standort{Kopenhagen, Det Kongelige Bibliotek, Georg Brandes Arkiv, box 125.}
\physDesc{Brief, 2 Blätter, 8 Seiten
\newline{}Handschrift: schwarze Tinte, deutsche Kurrent\newline{}Ordnung: mit Bleistift von unbekannter Hand nummeriert: »21. \textsc{Schnitzler}«, die Datierung auf der ersten Seite des zweiten
                                    Blattes mit Bleistift wiederholt }\buchAbdrucke{\weitereDrucke{Georg Brandes, Arthur Schnitzler: \emph{Ein Briefwechsel}. Hg. Kurt Bergel. Bern: \emph{Francke} 1956, S. 83–84.} }\toendnotes[C]{\smallbreak}\pstart
           \raggedleft{}{\pb}Wien\oindex{Wien@\textbf{Wien}|pw}, 25. 4. 901.\pend
           \pstart{}Lieber Herr Brandes,\pend\pstart
           \textsc{Paul Goldmann}\pwindex{Goldmann, Paul 31.01.1865 – 25.09.1935@\textsc{Goldmann, Paul} (31.01.1865 – 25.09.1935), \emph{Schriftsteller, Journalist}|pw} hat mir \textsc{Politiken}\pwindex{Politiken1. 1. 1884@\emph{Politiken}|pw} mit Ihrem \label{K_L01114_1v}\edtext{Artikel\pwindex{Brandes, Georg 04.02.1842 – 19.02.1927@\textsc{Brandes, Georg} (04.02.1842 – 19.02.1927)!Arthur Schnitzler [daenisch]09. 04. 1901@\strich\emph{Arthur Schnitzler [dänisch]} {[}09. 04. 1901{]}|pwv}}{\lemma{\textnormal{\emph{Artikel}}}\Cendnote{\textnormal{Es dürfte sich um einen Fehler Schnitzler\pwindex{Schnitzler, Arthur 15.05.1862 – 21.10.1931@\textsc{Schnitzler, Arthur} (15.05.1862 – 21.10.1931), \emph{Schriftsteller, Mediziner}|pwk}s handeln. Zumindest findet
                        sich der Text in seinen Zeitungsausschnitten (Exeter, box 37/2)
                        mit dem Titelzusatz »För Handelstidning« als Ausschnitt aus
                        \emph{Göteborgs Handels- och Sjöfartstidning}\orgindex{Goeteborgs Handels- och Sjoefartstidning@Göteborgs Handels- och Sjöfartstidning|pwk}
                        vom 9. 4. 1901.}}}\label{K_L01114_1h} über mich geſandt und ich verſuchte däniſch\oindex{Daenemark@\textbf{Dänemark}|pw} zu verſtehen, was mir nur zum Theil
                    gelang; die Neue Freie Preſſe\orgindex{Neue Freie Presse@Neue Freie Presse|pw} kam mir zu Hilfe\pwindex{Brandes, Georg 04.02.1842 – 19.02.1927@\textsc{Brandes, Georg} (04.02.1842 – 19.02.1927)!Arthur Schnitzler21. 04. 1901@\strich\emph{Arthur Schnitzler} {[}21. 04. 1901{]}|pwv} – und Sie können ſich
                    denken, wie ſehr ich mich gefreut habe, als ich nun alles, was Sie über mich
                    ſchrieben, we{\geminationn} auch nur in der Überſetzung leſen
                    konnte. Laſſen Sie mich Ihnen die Hand drücken – und {\pb}weiter nichts ſagen – wie es Ihnen ja gewiſs
                    am liebſten iſt.\pend
           \pstart
           Sie haben hoffentlich meine Karte aus Rom\oindex{Rom@\textbf{Rom}|pw}
                    bekommen und wiſſen, dſs ich \textsc{Ellen Key}\pwindex{Key, Ellen 11.12.1849 – 25.04.1926@\textsc{Key, Ellen} (11.12.1849 – 25.04.1926), \emph{Schriftsteller/Schriftstellerin}|pw} ke{\geminationn}engelernt habe, die mir zu meiner Freude
                    erzählte, dſs Sie den letzten Winter in vollkommener Geſundheit verbracht haben.
                    Wenige Tage nachdem ich \textsc{Ellen Key}\pwindex{Key, Ellen 11.12.1849 – 25.04.1926@\textsc{Key, Ellen} (11.12.1849 – 25.04.1926), \emph{Schriftsteller/Schriftstellerin}|pw}, deren Weſen mir wahrhaft wohl that, bei \textsc{Wasserma{\geminationn}s}\pwindex{Wassermann, Jakob 10.03.1873 – 01.01.1934@\textsc{Wassermann, Jakob} (10.03.1873 – 01.01.1934), \emph{Schriftsteller}|pw}\pwindex{Wassermann, Julie 05.12.1876 – April 1963@\textsc{Wassermann, Julie} (05.12.1876 – April 1963), \emph{Schriftstellerin}|pw} kennen gelernt, traf ich ſie ein zweites Mal und {\pb}\textsc{Helge Rhode}\pwindex{Rode, Helge 16.10.1870 – 23.03.1937@\textsc{Rode, Helge} (16.10.1870 – 23.03.1937), \emph{Schriftsteller}|pw}, den ſie mitbrachte. Ich war kaum zwei Wochen in Rom\oindex{Rom@\textbf{Rom}|pw}, eben genug, um zu wiſſen, wie man es ein nächſtes Mal
                    anzufangen hat, um ſeine Zeit gut auszunützen. Von Rom\oindex{Rom@\textbf{Rom}|pw} ging ich nach Florenz\oindex{Florenz@\textbf{Florenz}|pw}, wo ich mit
                    meiner Mama\pwindex{Schnitzler, Louise 08.07.1840 – 09.09.1911@\textsc{Schnitzler, Louise} (08.07.1840 – 09.09.1911)|pwv} Rendezvous
                    hatte – aber den Frühling fand ich nirgends. Man fror beinah immer.\pend
           \pstart
           Sie waren – oder ſind noch? – in Berlin\oindex{Berlin@\textbf{Berlin}|pw}, wie mir
                        Georg Hirſchfeld\pwindex{Hirschfeld, Georg 11.02.1873 – 17.01.1942@\textsc{Hirschfeld, Georg} (11.02.1873 – 17.01.1942), \emph{Schriftsteller}|pw}{ }{\pb}ſchrieb; wann ko{\geminationm}en Sie wieder zu uns? Sie würden nicht viel verändert finden – \textsc{Beer Hofmann}\pwindex{Beer-Hofmann, Richard 11.07.1866 – 26.09.1945@\textsc{Beer-Hofmann, Richard} (11.07.1866 – 26.09.1945), \emph{Schriftsteller}|pw} hat nun auch zu ſeinen Töchtern\pwindex{Beer-Hofmann, Mirjam 04.09.1897 – 24.12.1984@\textsc{Beer-Hofmann, Mirjam} (04.09.1897 – 24.12.1984)|pwv}\pwindex{Beer-Hofmann, Naemah 20.12.1898 – 10.11.1971@\textsc{Beer-Hofmann, Naëmah} (20.12.1898 – 10.11.1971)|pwv} einen Sohn\pwindex{Beer-Hofmann, Gabriel 09.01.1901 – 24.03.1971@\textsc{Beer-Hofmann, Gabriel} (09.01.1901 – 24.03.1971), \emph{Schriftsteller, Filmagent}|pwv} beko{\geminationm}en, aber von
                    dem iſt begreiflicherweiſe noch nicht viel zu erzählen. Ich werde diesmal
                    wahrſcheinlich ſehr bald ins Gebirge reiſen; und nach mancherlei Kleinigkeiten,
                    die ich in der letzten Zeit gemacht, mich wohl endlich wieder \introOben{}an\introOben{} was größeres {\pb}wagen. Einen
                    kleinen Roman\pwindex{Schnitzler, Arthur 15.05.1862 – 21.10.1931@\textsc{Schnitzler, Arthur} (15.05.1862 – 21.10.1931), \emph{Schriftsteller, Mediziner}!Frau Bertha Garlan. Roman15.1.1901 – 15.3.1901@\strich\emph{Frau Bertha Garlan. Roman} {[}15.1.1901 – 15.3.1901{]}|pwv}, den ich
                        vorigen Winter{ }ſchrieb, haben Sie wohl ſchon erhalten. Die
                        \textsc{Beatrice}\pwindex{Schnitzler, Arthur 15.05.1862 – 21.10.1931@\textsc{Schnitzler, Arthur} (15.05.1862 – 21.10.1931), \emph{Schriftsteller, Mediziner}!Schleier der Beatrice. Schauspiel in fuenf Akten1900-12-01 – 1900-12-01@\strich\emph{Der Schleier der Beatrice. Schauspiel in fünf Akten} {[}1900-12-01 – 1900-12-01{]}|pw} iſt im Dezember einige Male in Breſlau\oindex{Breslau@\textbf{Breslau}|pw} geſpielt worden, ohne beſonderes Glück. Auch war die
                    Darſtellung recht ſchwach. Eine gute Aufführung müßte dem Stück wohl Erfolg
                    bringen. Aber das Burgtheater\oindex{Burgtheater@\textbf{Burgtheater}|pw} hat wichtigeres zu
                    thun. –\pend
           \pstart Leben Sie wohl und ſeien Sie herzlich gegrüßt von Ihrem treuen
                        \spacefill\mbox{ArthurSchnitzler}\pend{}\pstart
           \noindent{}{\pb}Dieſer Tage erſcheint eine Novelle von
                        mir, die ich Ihnen natürlich ſchicken werde, Lieutenant Guſtl\pwindex{Schnitzler, Arthur 15.05.1862 – 21.10.1931@\textsc{Schnitzler, Arthur} (15.05.1862 – 21.10.1931), \emph{Schriftsteller, Mediziner}!Lieutenant Gustl. Novelle25. 12. 1900@\strich\emph{Lieutenant Gustl. Novelle} {[}25. 12. 1900{]}|pw}, – Sie haben ſie vielleicht in der N. Fr. Pr.\orgindex{Neue Freie Presse@Neue Freie Presse|pw} geleſen. Wegen dieſer Novelle
                        ſtehe ich – (da ich noch \textsc{Militärarzt} »in der
                        Evidenz« bin) in »ehrengerichtlicher« Unterſuchung und werde wahrſcheinlich
                        meine \textsc{Charge} verlieren. Wenn Sie die Novelle {\pb}noch nicht kennen und ſie leſen werden –
                        und ſich dieſer Mittheilung erinnern – wird Ihnen wieder manches »oeſterreichiſche\oindex{Oesterreich@\textbf{Österreich}|pw}« klar werden.\hspace*{1.5em}Die Sache iſt für mich natürlich gleichgiltig
                        – da ich ja mit den Leuten nichts mehr zu thun habe und meine Charge nur im
                        Kriegsfall von Bedeutung wäre – aber ſie iſt charakteriſtiſch für {\pb}die man könnte ſagen naïve Heuchelei in
                        Kreiſen, von denen man in gewiſſem Sinne i{\geminationm}er
                        abhängig iſt; we{\geminationn}{ }ſie auch keine unmittelbare Macht über
                        einen beſitzen.\pend
           \pstart
           Ihr \spacefill\mbox{A. S.}\pend
                     \endnumbering\briefempfaengerindex{Brandes, Georg@\textsc{Brandes, Georg}!zzzSchnitzler, Arthur@\emph{von Arthur Schnitzler}!1901-04-251@{25. 4. 1901}|)be}\mylabel{h}\end{ledgroupsized}  \newcommand{\dateiname}{L01114}\newcommand{\titel}{Arthur Schnitzler an Georg Brandes, 25. 4. 1901}\newcommand{\editorInnen}{Martin Anton Müller und Gerd-Hermann Susen}
            \footnotesize
\begin{ledgroupsized}[t]{11.5cm}
\doendnotes{C}
\end{ledgroupsized}
         %% latex-leseansicht-abspann.tex
%% Abspann für die Leseansicht.
%% Der Schalter \ifkorrekturansicht ist bereits durch den Vorspann gesetzt.

%% latex-abspann.tex
%% Gemeinsamer Abspann für Korrekturansicht und Leseansicht.
%% Setzt den Schalter \ifkorrekturansicht voraus (gesetzt in den
%% einbindenden Dateien latex-korrekturansicht-abspann.tex bzw.
%% latex-leseansicht-abspann.tex).
%% ---------------------------------------------------------------

\normalsize

% Das esempio-Environment wird nur in der Leseansicht benötigt
\ifkorrekturansicht\else
\newenvironment{esempio}[3]%
{
    \vspace{1.5ex}
    \rlap{\underline{#1}}
    \par
    \setlength{\parindent}{0cm}
    \nopagebreak
    \leftskip=#2cm
    \rightskip=#3cm
}
{
    \par
}
\fi

\doendnotes{C}
\bigskip
\vfill

\clearpage

\footnotesize

\ifkorrekturansicht
  \lohead{\textsc{register}}
\fi

% theindex-Environment neu definieren ohne reledmac
\makeatletter
\renewenvironment{theindex}{%
  \ifkorrekturansicht
    \section*{\indexname}%
  \else
    \subsubsection*{Index der erwähnten Entitäten}%
  \fi
  \setlength{\parindent}{0pt}%
  \setlength{\parskip}{0pt plus 0.3pt}%
  \let\item\@idxitem
}{%
  \ifkorrekturansicht\clearpage\fi
}
\makeatother

\IfFileExists{\jobname-pw.ind}{\input{\jobname-pw.ind}}{}

% Quellenangabe nur in der Leseansicht
\ifkorrekturansicht\else
% Fallback-Definitionen, falls die .tex-Datei \titel etc. nicht gesetzt hat
\providecommand{\titel}{}
\providecommand{\editorInnen}{}
\providecommand{\dateiname}{\jobname}

\vspace{3cm}

\vfill

\footnotesize
\textsc{Quelle}: \titel. Herausgegeben von {\editorInnen}. In: \emph{Arthur Schnitzler: Briefwechsel mit Autorinnen und Autoren}.
 Digitale Edition, https://schnitzler-briefe.acdh.oeaw.ac.at/{\dateiname}.html (Stand \today)
\fi

\end{document}


      