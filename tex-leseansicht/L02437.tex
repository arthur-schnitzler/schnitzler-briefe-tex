%% latex-korrekturansicht-vorspann.tex
%% Vorspann für die Korrekturansicht.
%% Lädt die gemeinsame Datei latex-vorspann.tex mit gesetztem Schalter.

\newif\ifkorrekturansicht
\korrekturansichttrue

\input{../tex-inputs/latex-vorspann}


\section[Gerty Hofmannsthal an Arthur Schnitzler, 7. 3. 1925]{L02437 Gerty Hofmannsthal an Arthur Schnitzler, 7. 3. 1925}
\nopagebreak\mylabel{L02437v}
\rehead{ }\normalsize\beginnumbering\briefempfaengerindex{Schnitzler, Arthur@\textsc{Schnitzler, Arthur}!zzzHofmannsthal, Gertrude von@\emph{von Gertrude von Hofmannsthal}!1925-03-071@{7. 3. 1925}|(be}
\toendnotes[C]{\smallbreak\pagebreak[2]}\Standort{CUL, Schnitzler, B 43.}
\physDesc{Postkarte, 645 Zeichen
\newline{}Handschrift: schwarze Tinte, lateinische Kurrent
\newline{}Versand: Stempel: »\nobreak{}\oindex{I., Innere Stadt@\textbf{I., Innere Stadt}, \emph{A.ADM3}|pwk}1/1 Wien 15, 8. III. 25, VIII\nobreak{}«.  
\newline{}Schnitzler: mit Bleistift beschriftet »\textsc{Gerty Hofmannst}« und die Jahreszahl beim Datum ergänzt: »25« 
\newline{}Ordnung: 1) mit Bleistift von unbekannter Hand nummeriert: »\strikeout{386}«  2) mit Bleistift von unbekannter Hand nummeriert: »\strikeout{389}«}
\buchAbdrucke{\weitereDrucke{Hugo von Hofmannsthal, Arthur Schnitzler: \emph{Briefwechsel}. Frankfurt am Main: \emph{S. Fischer} 1964, S. 394.} }\toendnotes[C]{\smallbreak}\pstart{}{\pb}S. H.\pend{}\pstart{}Herrn Dr. Arthur Schnitzler\pend{}\pstart{}Wien XVIII\oindex{XVIII., Waehring@\textbf{XVIII., Währing}, \emph{A.ADM3}|pw}\pend{}\pstart{}Sternwartestrasse 71\oindex{Sternwartestrasse 71@\textbf{Sternwartestraße 71}, \emph{Wohngebäude (K.WHS)}|pw}\pend{}{\bigskip}\vspace{1em}
\pstart
           \raggedleft{}{\pb}7/III\pend
           \vspace{0.5em}
\pstart
           Lieber Arthur, ich verdanke Ihnen den schönen Abend \label{K_L02437-1v}\edtext{neulich}{\lemma{\textnormal{\emph{neulich}}}\Cendnote{\textnormal{Vgl. A. S.: \emph{Tagebuch}, 1. 3. 1925.
               }}}\label{K_L02437-1} und habe mich wirklich wunderbar unterhalten. Waldau\pwindex{Waldau, Gustav 27.02.1871 – 25.05.1958@\textsc{Waldau, Gustav} (27.02.1871 – 25.05.1958), \emph{Schauspieler/Schauspielerin}|pw} war doch \uline{ganz}
               reizend!\pend
           
\pstart
           Da Sie neulich so rührend waren mir zu helfen so will ich Ihnen noch sagen, dass
               leider meine Depesche Hugo\pwindex{Hofmannsthal, Hugo von 1874-02-01 – 1929-07-15@\textsc{Hofmannsthal, Hugo von} (1874-02-01 – 1929-07-15), \emph{Schriftsteller/Schriftstellerin}|pw} nicht mehr erreicht
               hat. Ich verschiebe jetzt die ganze \label{K_L02437-2v}\edtext{Auseinandersetzung}{\lemma{\textnormal{\emph{Auseinandersetzung}}}\Cendnote{\textnormal{Gerty Hofmannsthal\pwindex{Hofmannsthal, Gertrude von 16.03.1880 – 09.11.1959@\textsc{Hofmannsthal, Gertrude von} (16.03.1880 – 09.11.1959)|pwk} sollte einen
                  Vortrag Hugo Hofmannsthals\pwindex{Hofmannsthal, Hugo von 1874-02-01 – 1929-07-15@\textsc{Hofmannsthal, Hugo von} (1874-02-01 – 1929-07-15), \emph{Schriftsteller/Schriftstellerin}|pwk} verschieben, aber der
                  Veranstalter hatte mit einer Strafzahlung gedroht.}}}\label{K_L02437-2} bis nach Hugos\pwindex{Hofmannsthal, Hugo von 1874-02-01 – 1929-07-15@\textsc{Hofmannsthal, Hugo von} (1874-02-01 – 1929-07-15), \emph{Schriftsteller/Schriftstellerin}|pw}{ }{\pb}Rückkunft. Auch würden weitere Briefe
               von mir (ohne Hilfe) die Sache nur abschwächen. Ein bisschen schien er schon
               »kleiner« in seiner Antwort!\pend
           
\pstart
           Von Hugo\pwindex{Hofmannsthal, Hugo von 1874-02-01 – 1929-07-15@\textsc{Hofmannsthal, Hugo von} (1874-02-01 – 1929-07-15), \emph{Schriftsteller/Schriftstellerin}|pw} das erste Telegr. auf dem Meer dass
               er sehr zufrieden ist.\pend
           \pstart Viele herzliche Grüsse und nochmals Dank Ihre \spacefill\mbox{Gerty}\pend{}\selectlanguage{ngerman}\endnumbering\briefempfaengerindex{Schnitzler, Arthur@\textsc{Schnitzler, Arthur}!zzzHofmannsthal, Gertrude von@\emph{von Gertrude von Hofmannsthal}!1925-03-071@{7. 3. 1925}|)be}\mylabel{L02437h}  \normalsize

\doendnotes{C}
\bigskip
\vfill

\clearpage

\footnotesize

\lohead{\textsc{register}}

% Definiere theindex-Environment komplett neu ohne reledmac
\makeatletter
\renewenvironment{theindex}{%
  \section*{\indexname}%
  \setlength{\parindent}{0pt}%
  \setlength{\parskip}{0pt plus 0.3pt}%
  \let\item\@idxitem
}{%
  \clearpage
}
\makeatother

\IfFileExists{\jobname-pw.ind}{\input{\jobname-pw.ind}}{}

\end{document}

      