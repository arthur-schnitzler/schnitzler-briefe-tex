%% latex-korrekturansicht-vorspann.tex
%% Vorspann für die Korrekturansicht.
%% Lädt die gemeinsame Datei latex-vorspann.tex mit gesetztem Schalter.

\newif\ifkorrekturansicht
\korrekturansichttrue

\input{../tex-inputs/latex-vorspann}


\section[Hugo von Hofmannsthal an Arthur Schnitzler, 24. {[}12?.{]} 1903]{L01331 Hugo von Hofmannsthal an Arthur Schnitzler, 24. {[}12?.{]} 1903}
\nopagebreak\mylabel{L01331v}
\rehead{ }\normalsize\beginnumbering\briefempfaengerindex{Schnitzler, Arthur@\textsc{Schnitzler, Arthur}!zzzHofmannsthal, Hugo von@\emph{von Hugo von Hofmannsthal}!1903-12-241@{24. {[}12?.{]} 1903}|(be}
\toendnotes[C]{\smallbreak\pagebreak[2]}\Standort{CUL, Schnitzler, B 43.}
\physDesc{Postkarte, 155 Zeichen
\newline{}Handschrift: schwarze Tinte, deutsche Kurrent
\newline{}Versand: Stempel: »\nobreak{}\oindex{Rodaun@\textbf{Rodaun}, \emph{A.ADM4}|pwk}Rodaun, 24 {[}12{]} 03 , 7–\textcolor{gray}{9N}\nobreak{}«.  
\newline{}Ordnung: 1) mit Bleistift von unbekannter Hand nummeriert: »\strikeout{218}«  2) mit Bleistift von unbekannter Hand nummeriert:
                                    »210«}
\buchAbdrucke{\weitereDrucke{Hugo von Hofmannsthal, Arthur Schnitzler: \emph{Briefwechsel}. Frankfurt am Main: \emph{S. Fischer} 1964, S. 178.} }\toendnotes[C]{\smallbreak}\pstart{}{\pb}\textsc{Herrn D\textsuperscript{r} Arthur Schnitzler}\pend{}\pstart{}\textsc{Wien}\oindex{Wien@\textbf{Wien}, \emph{A.ADM2}|pw}\pend{}\pstart{}\strikeout{IX}{ }\textsc{XVIII. Spöttelgasse 7}\oindex{Edmund-Weiss-Gasse 7@\textbf{Edmund-Weiß-Gasse 7}, \emph{Wohngebäude (K.WHS)}|pw}. \pend{}{\bigskip}\vspace{1em}
\pstart
           \noindent{}{\pb}Ich habe nun wieder den \label{K_L01331-1v}\edtext{Brief}{\lemma{\textnormal{\emph{Brief}}}\Cendnote{\textnormal{Gemeint ist die 
               nicht erhaltene Antwort auf: Hugo von Hofmannsthal an Arthur Schnitzler, [zwischen 14. und
               23. 12. 1903?]. Da jenes Telegramm auf »Ende 1903« datiert ist,
               muss diese Karte ebenfalls in den letzten Tagen des Jahres 1903 zu verorten sein. Damit lässt sich die nicht zu entziffernde Monatsangabe des Poststempels
               konstruieren.}}}\label{K_L01331-1}
               verloren. Bitte ſchreiben Sie mir wie der Verein\orgindex{Neue akademische Vereinigung@Neue akademische Vereinigung|pwv} in Brünn\oindex{Bruenn@\textbf{Brünn}, \emph{P.PPLA}|pw} heißt.
               Herzlich\pend
           \pstart \spacefill\mbox{Hugo}\pend{}\selectlanguage{ngerman}\endnumbering\briefempfaengerindex{Schnitzler, Arthur@\textsc{Schnitzler, Arthur}!zzzHofmannsthal, Hugo von@\emph{von Hugo von Hofmannsthal}!1903-12-241@{24. {[}12?.{]} 1903}|)be}\mylabel{L01331h}  \normalsize

\doendnotes{C}
\bigskip
\vfill

\clearpage

\footnotesize

\lohead{\textsc{register}}

% Definiere theindex-Environment komplett neu ohne reledmac
\makeatletter
\renewenvironment{theindex}{%
  \section*{\indexname}%
  \setlength{\parindent}{0pt}%
  \setlength{\parskip}{0pt plus 0.3pt}%
  \let\item\@idxitem
}{%
  \clearpage
}
\makeatother

\IfFileExists{\jobname-pw.ind}{\input{\jobname-pw.ind}}{}

\end{document}

      