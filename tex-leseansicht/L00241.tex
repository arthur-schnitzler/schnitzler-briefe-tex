%% latex-korrekturansicht-vorspann.tex
%% Vorspann für die Korrekturansicht.
%% Lädt die gemeinsame Datei latex-vorspann.tex mit gesetztem Schalter.

\newif\ifkorrekturansicht
\korrekturansichttrue

\input{../tex-inputs/latex-vorspann}


\section[Hermann Bahr an Arthur Schnitzler, {[}24. 7. 1893{]}]{L00241 Hermann Bahr an Arthur Schnitzler, {[}24. 7. 1893{]}}
\nopagebreak\mylabel{L00241v}
\rehead{ }\normalsize\beginnumbering\briefempfaengerindex{Schnitzler, Arthur@\textsc{Schnitzler, Arthur}!zzzBahr, Hermann@\emph{von Hermann Bahr}!1893-07-241@{{[}24. 7. 1893{]}}|(be}
\toendnotes[C]{\smallbreak\pagebreak[2]}\Standort{CUL, Schnitzler, B 5b.}
\physDesc{Brief, 1 Blatt, 1 Seite, 250 Zeichen
\newline{}Handschrift: Bleistift, deutsche Kurrent
\newline{}Schnitzler: mit Bleistift datiert: »24. 7. 93« 
\newline{}Ordnung: 1) mit Bleistift von unbekannter Hand nummeriert:
                                    »11«  2) mit rotem Buntstift von unbekannter Hand nummeriert:
                                    »11«}
\buchAbdrucke{\weitereDrucke{Hermann Bahr, Arthur Schnitzler: \emph{Briefwechsel, Aufzeichnungen, Dokumente (1891–1931)}. Göttingen: \emph{Wallstein} 2018, S. 36.} }\toendnotes[C]{\smallbreak}
\pstart
           {\pb}\textcolor{gray}{\textbf{Deutſche Zeitung\orgindex{Deutsche Zeitung@Deutsche Zeitung|pw}}}\pend
           
\pstart
           \textcolor{gray}{\textbf{Wien\oindex{Wien@\textbf{Wien}, \emph{A.ADM2}|pw}}}\pend
           
\pstart
           \textcolor{gray}{\textbf{IX., Pelikangaſſe 4\oindex{Pelikangasse@\textbf{Pelikangasse}, \emph{Straße (K.STR)}|pw}.}}\pend
           
\pstart\center{}Lieber Freund!\pend\vspace{0.5em}
\pstart
           Von Ihrer Anfrage über Loris\pwindex{Hofmannsthal, Hugo von 1874-02-01 – 1929-07-15@\textsc{Hofmannsthal, Hugo von} (1874-02-01 – 1929-07-15), \emph{Schriftsteller/Schriftstellerin}|pw} hat man mir
               nichts mitgeteilt. Ich ko{\geminationm}e morgen entweder zwiſchen
                  3 u. 4{ }\label{K_L00241-1v}\edtext{Burgring\oindex{Wohnung und Ordination Johann Schnitzler Burgring 1@\textbf{Wohnung und Ordination Johann Schnitzler Burgring 1}, \emph{Ordination}|pw}\oindex{Wohnung und Ordination Johann Schnitzler Burgring 1@\textbf{Wohnung und Ordination Johann Schnitzler Burgring 1}, \emph{Ordination}|pwv}}{\lemma{\textnormal{\emph{Burgring}}}\Cendnote{\textnormal{Schnitzler
                  dürfte nach dem Tod seines Vaters\pwindex{Schnitzler, Johann 10.04.1835 – 02.05.1893@\textsc{Schnitzler, Johann} (10.04.1835 – 02.05.1893), \emph{Laryngologe/Laryngologin}|pwkv} dessen Ordination\oindex{Wohnung und Ordination Johann Schnitzler Burgring 1@\textbf{Wohnung und Ordination Johann Schnitzler Burgring 1}, \emph{Ordination}|pwkv} weiter betreut haben.}}}\label{K_L00241-1} oder um ½ 5{ }Grillparzerſtr\oindex{Grillparzerstrasse@\textbf{Grillparzerstraße}, \emph{R.ST}|pw}. Daß Sie \uline{uns} u. nur uns keine Notiz über \label{K_L00241-2v}\edtext{\textsc{Ischler}\oindex{Bad Ischl@\textbf{Bad Ischl}, \emph{P.PPL}|pw} Aufführung}{\lemma{\textnormal{\emph{Ischler Aufführung}}}\Cendnote{\textnormal{Uraufführung\eventindex{Lehártheater@\textbf{Lehártheater}!Urauffuehrung von Abschiedssouper, Auffuehrung von Fraeulein Frau, 14.7.1893@Uraufführung von Abschiedssouper, Aufführung von Fräulein Frau, 14.7.1893|pwkv} von \emph{Abschiedssouper}\pwindex{Abschiedssouper@\emph{Abschiedssouper}|pwk}, 14. 7. 1893.
               }}}\label{K_L00241-2} geſchickt, iſt nicht ſchön.\pend
           
\pstart
           Herzlichſt{\\[\baselineskip]}Ihr{\\[\baselineskip]}\spacefill\mbox{HermannBahr}\pend
           \leftskip=0em{}\selectlanguage{ngerman}\endnumbering\briefempfaengerindex{Schnitzler, Arthur@\textsc{Schnitzler, Arthur}!zzzBahr, Hermann@\emph{von Hermann Bahr}!1893-07-241@{{[}24. 7. 1893{]}}|)be}\mylabel{L00241h}  \normalsize

\doendnotes{C}
\bigskip
\vfill

\clearpage

\footnotesize

\lohead{\textsc{register}}

% Definiere theindex-Environment komplett neu ohne reledmac
\makeatletter
\renewenvironment{theindex}{%
  \section*{\indexname}%
  \setlength{\parindent}{0pt}%
  \setlength{\parskip}{0pt plus 0.3pt}%
  \let\item\@idxitem
}{%
  \clearpage
}
\makeatother

\IfFileExists{\jobname-pw.ind}{\input{\jobname-pw.ind}}{}

\end{document}

      