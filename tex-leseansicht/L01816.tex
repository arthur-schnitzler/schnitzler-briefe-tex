\input{../tex-inputs/latex-pdf-vorspann}
\begin{center}
            \textcolor{red}{ENTWURF. ENTZIFFERUNG NOCH NICHT KORREKTURGELESEN}
                      \end{center}
            
               \section[Richard Beer-Hofmann an Olga Schnitzler, {[}30. 11. 1908?{]}]{ Richard Beer-Hofmann an Olga Schnitzler,
               {[}30. 11. 1908?{]}}\nopagebreak\mylabel{v}\rehead{ }\begin{ledgroupsized}[t]{13cm}\normalsize\beginnumbering\briefempfaengerindex{Schnitzler, Arthur@\textsc{Schnitzler, Arthur}!zzzBeer-Hofmann, Richard@\emph{von Richard Beer-Hofmann}!1908-11-302@{{[}30. 11. 1908?{]}}|(be} \toendnotes[C]{\smallbreak\pagebreak[2]} \Standort{CUL, Schnitzler, B 8.}
\physDesc{Brief, 1 Blatt, 1 Seite
\newline{}Handschrift: Bleistift, lateinische Kurrent
\newline{}Schnitzler: mit Bleistift beschriftet: »\textsc{BH}« \newline{}Ordnung: mit Bleistift von unbekannter Hand nummeriert: »278c« }\buchAbdrucke{\weitereDrucke{Arthur Schnitzler, Richard Beer-Hofmann: \emph{Briefwechsel 1891–1931}. Hg. Konstanze Fliedl. Wien, Zürich: \emph{Europaverlag} 1992, S. 215.} }\toendnotes[C]{\smallbreak}\settowidth{\longeste}{Göttliche Gesellschaft!}\settowidth{\longestz}{kleine Bäckerei}\settowidth{\longestd}{}\settowidth{\longestv}{}\settowidth{\longestf}{}\addtolength\longeste{1em}
        \addtolength\longestz{1em}
      \pstart\noindent\makebox[\the\longeste][l]{{\pb}\label{KLL01816_OS-1v}\edtext{Ein fürstliches}{\lemma{\textnormal{\emph{Ein fürstliches}}}\Cendnote{\textnormal{Die
                           Speisekarte ist undatiert. Sie fügt sich als mögliche Antwort auf die Frage
                           Olga Schnitzler\pwindex{Schnitzler, Olga 17.01.1882 – 13.01.1970@\textsc{Schnitzler, Olga} (17.01.1882 – 13.01.1970), \emph{Schauspielerin, Sängerin}|pwk}s nach dem Speiseplan
                        ein und wird entsprechend an diese Stelle gereiht.}}}\label{KLL01816_OS-1h}:}\makebox[\the\longestz][l]{Gansleberragout}
                  \pend\pstart\noindent\makebox[\the\longeste][l]{Königliche:}\makebox[\the\longestz][l]{Perlhühner}
                  \pend\pstart\noindent\makebox[\the\longeste][l]{Päbstliche:}\makebox[\the\longestz][l]{kleine Bäckerei}
                  \pend\pstart\noindent\makebox[\the\longeste][l]{Göttliche Gesellschaft!}\makebox[\the\longestz][l]{}
                  \pend\pstart \spacefill\mbox{Der Koch – nein – Küchen-Chef}\pend{}\endnumbering\briefempfaengerindex{Schnitzler, Arthur@\textsc{Schnitzler, Arthur}!zzzBeer-Hofmann, Richard@\emph{von Richard Beer-Hofmann}!1908-11-302@{{[}30. 11. 1908?{]}}|)be}\mylabel{h}\end{ledgroupsized}  \newcommand{\dateiname}{L01816}\newcommand{\titel}{Richard Beer-Hofmann an Olga Schnitzler, [30. 11. 1908?]}\newcommand{\editorInnen}{Martin Anton Müller und Gerd-Hermann Susen}\input{../tex-inputs/latex-pdf-abspann}
      