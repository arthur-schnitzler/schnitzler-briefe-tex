%% latex-leseansicht-vorspann.tex
%% Vorspann für die Leseansicht.
%% Lädt die gemeinsame Datei latex-vorspann.tex mit nicht gesetztem Schalter.

\newif\ifkorrekturansicht
\korrekturansichtfalse

\input{../tex-inputs/latex-vorspann}


               \section[Felix Braun an Arthur Schnitzler, 25. 12. 1927]{ Felix Braun an Arthur Schnitzler, 25. 12. 1927}\nopagebreak\mylabel{v}\rehead{ }\begin{ledgroupsized}[t]{13cm}\normalsize\beginnumbering\briefempfaengerindex{Schnitzler, Arthur@\textsc{Schnitzler, Arthur}!zzzBraun, Felix@\emph{von Felix Braun}!1927-12-251@{25. 12. 1927}|(be} \toendnotes[C]{\smallbreak\pagebreak[2]} \Standort{CUL, Schnitzler, B 19.}
\physDesc{Brief, 1 Blatt, 4 Seiten
\newline{}Handschrift: schwarze Tinte, deutsche Kurrent
\newline{}Schnitzler: 1) auf der ersten Seite mit Bleistift beschrieben mit »\textsc{Fel. Braun}{ }Siever. Str. 191\oindex{Sieveringer Strasse@\textbf{Sieveringer Straße}|pw}« und  2) mit rotem Buntstift Vermerk: »Aph{[}orismen{]}« und mehrere
                                 Unterstreichungen}\Standort{DLA, A:Schnitzler, HS.NZ85.1.2604.}
\physDesc{Brief, 1 Blatt, 1 Seite, maschinelle Abschrift
\newline{}Schreibmaschine}\toendnotes[C]{\smallbreak}\pstart
           \centering{}{\pb}Wien\oindex{Wien@\textbf{Wien}|pw}, den 25. XII. 27.\pend
           \pstart{}Verehrter Herr Doktor!\pend\pstart
           Für Ihr neues Werk\pwindex{Schnitzler, Arthur 15.05.1862 – 21.10.1931@\textsc{Schnitzler, Arthur} (15.05.1862 – 21.10.1931), \emph{Schriftsteller, Mediziner}!Buch der Sprueche und Bedenken1927@\strich\emph{Buch der Sprüche und Bedenken} {[}1927{]}|pwv}, die liebe
               Weihnachtsüberraſchung, ſage ich den Dank eines zwiefach Beſchenkten. Ich wollte
               warten, bis ich das ganze Buch geleſen, doch wurde ſein Gewicht immer ſchwerer, und
               obwohl ich nach der Kenntnis von etwa der Hälfte ausſprechen darf, daß ich um ſeinen
               Geist weiß, unterbreche ich die Lektüre, um ein Dankwort an Sie zu richten. – \pend
           \pstart
           Ich hatte gefürchtet, daß mir Ihr Buch nicht genug \introOben{}nahe\introOben{} ſein
               möchte – das Gegenteil erweiſt ſich ſchon jetzt. Was geradezu be\damage{gl}ückend für {\pb}mich war, war das
               Zuſehen der Geburt einer Frömmigkeit aus dem Geiſte des Zweifels. Ich bewundere die
               Ehrlichkeit und die Kraft des Denkers in Ihnen – manches ist ſo philoſophiſch wie nur
               ein Traktat der deutſchen \textsc{Transcendental}-Philoſophie –, und
               ich kann nicht \introOben{}nur\introOben{} von dem älteren, lebenskundigeren, auch
               von dem ſchärfer und ſtrenger blickenden Geiſt, der hier rein männlich und ringend
               waltet, lernen. Manches Ihrer Worte mutet, bis in die Sprache hinein, die vollendet
               iſt, wie aus der \textsc{Antike} an.\pend
           \pstart
           Das iſt ein Buch\pwindex{Schnitzler, Arthur 15.05.1862 – 21.10.1931@\textsc{Schnitzler, Arthur} (15.05.1862 – 21.10.1931), \emph{Schriftsteller, Mediziner}!Buch der Sprueche und Bedenken1927@\strich\emph{Buch der Sprüche und Bedenken} {[}1927{]}|pwv}, das mich lange
               begleiten wird. Sehr, ſehr ſchön iſt es, ſcheinbar ganz Geiſtgeſtalt, doch das
               Erlebte iſt überall ſpürbar. {\pb}Welch ein
               Reichtum an inneren Blicken! Auf S. 111 \label{K_L02494_1v}\edtext{Nr. 48}{\lemma{\textnormal{\emph{Nr. 48}}}\Cendnote{\textnormal{»So mancher glaubt,
                     immer noch einem verlorenen Glücke nachzuweinen und es ist längst nur mehr der
                     abgeschiedene Schmerz darum, dem seine Tränen fließen.«}}}\label{K_L02494_1h} und auf
               S. 121 \label{K_L02494_2v}\edtext{Nr. 80}{\lemma{\textnormal{\emph{Nr. 80}}}\Cendnote{\textnormal{»Ein tragikomisches Schicksal: sein
                     Leben zerstört zu wissen und niemand haben, an dessen Brust man sich darüber
                     ausweinen möchte als allein das Wesen von dem es zerstört
                  wurde.«}}}\label{K_L02494_2h} trafen mich ſelbst.\pend
           \pstart
           Es ist ſehr gut, daß dieſes Buch von Ihnen da iſt, eben aus den Gründen, die Sie in
               der Vorrede anführen. Unter den Sprüchen in Verſen fehlt mir ein Gedicht von Ihnen,
               das ich als Knabe in einer \label{K_L02494_3v}\edtext{Weihnachtsbeilage}{\lemma{\textnormal{\emph{Weihnachtsbeilage}}}\Cendnote{\textnormal{richtig:
                  Pfingstbeilage.}}}\label{K_L02494_3h} las und ſeither in mir trage:\pend
           \stanza{}»Ich hab dir viel gegeben,\pwindex{Schnitzler, Arthur 15.05.1862 – 21.10.1931@\textsc{Schnitzler, Arthur} (15.05.1862 – 21.10.1931), \emph{Schriftsteller, Mediziner}!Zum Abschied21. 05. 1899@\strich\emph{Zum Abschied} {[}21. 05. 1899{]}|pwv}\newverse{}Bewahr’ es gut {\dots}\pwindex{Schnitzler, Arthur 15.05.1862 – 21.10.1931@\textsc{Schnitzler, Arthur} (15.05.1862 – 21.10.1931), \emph{Schriftsteller, Mediziner}!Zum Abschied21. 05. 1899@\strich\emph{Zum Abschied} {[}21. 05. 1899{]}|pwv}«\stanzaend{}\pstart
           das iſt ein wunderbares Gedicht, ein Kryſtall, und ſollte ſichtbar ſein.\pend
           \pstart
           Zum Jahrbeginn wünſche ich Ihnen, verehrter Herr Doktor, viel Liebes und
               Freudiges, und ſo bleibe {\pb}ich, nochmals
               von Herzen für Ihr Geschenk dankend,\hspace*{1.5em}Ihr wahrhaft ergebener{\\[\baselineskip]}\spacefill\mbox{Felix Braun.}\pend
           \leftskip=0em{}\endnumbering\briefempfaengerindex{Schnitzler, Arthur@\textsc{Schnitzler, Arthur}!zzzBraun, Felix@\emph{von Felix Braun}!1927-12-251@{25. 12. 1927}|)be}\mylabel{h}\end{ledgroupsized}  \newcommand{\dateiname}{L02494}\newcommand{\titel}{Felix Braun an Arthur Schnitzler, 25. 12. 1927}\newcommand{\editorInnen}{Martin Anton Müller und Gerd-Hermann Susen}%% latex-leseansicht-abspann.tex
%% Abspann für die Leseansicht.
%% Der Schalter \ifkorrekturansicht ist bereits durch den Vorspann gesetzt.

%% latex-abspann.tex
%% Gemeinsamer Abspann für Korrekturansicht und Leseansicht.
%% Setzt den Schalter \ifkorrekturansicht voraus (gesetzt in den
%% einbindenden Dateien latex-korrekturansicht-abspann.tex bzw.
%% latex-leseansicht-abspann.tex).
%% ---------------------------------------------------------------

\normalsize

% Das esempio-Environment wird nur in der Leseansicht benötigt
\ifkorrekturansicht\else
\newenvironment{esempio}[3]%
{
    \vspace{1.5ex}
    \rlap{\underline{#1}}
    \par
    \setlength{\parindent}{0cm}
    \nopagebreak
    \leftskip=#2cm
    \rightskip=#3cm
}
{
    \par
}
\fi

\doendnotes{C}
\bigskip
\vfill

\clearpage

\footnotesize

\ifkorrekturansicht
  \lohead{\textsc{register}}
\fi

% theindex-Environment neu definieren ohne reledmac
\makeatletter
\renewenvironment{theindex}{%
  \ifkorrekturansicht
    \section*{\indexname}%
  \else
    \subsubsection*{Index der erwähnten Entitäten}%
  \fi
  \setlength{\parindent}{0pt}%
  \setlength{\parskip}{0pt plus 0.3pt}%
  \let\item\@idxitem
}{%
  \ifkorrekturansicht\clearpage\fi
}
\makeatother

\IfFileExists{\jobname-pw.ind}{\input{\jobname-pw.ind}}{}

% Quellenangabe nur in der Leseansicht
\ifkorrekturansicht\else
% Fallback-Definitionen, falls die .tex-Datei \titel etc. nicht gesetzt hat
\providecommand{\titel}{}
\providecommand{\editorInnen}{}
\providecommand{\dateiname}{\jobname}

\vspace{3cm}

\vfill

\footnotesize
\textsc{Quelle}: \titel. Herausgegeben von {\editorInnen}. In: \emph{Arthur Schnitzler: Briefwechsel mit Autorinnen und Autoren}.
 Digitale Edition, https://schnitzler-briefe.acdh.oeaw.ac.at/{\dateiname}.html (Stand \today)
\fi

\end{document}


      