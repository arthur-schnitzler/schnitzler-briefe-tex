%% latex-leseansicht-vorspann.tex
%% Vorspann für die Leseansicht.
%% Lädt die gemeinsame Datei latex-vorspann.tex mit nicht gesetztem Schalter.

\newif\ifkorrekturansicht
\korrekturansichtfalse

\input{../tex-inputs/latex-vorspann}


         
         \newcommand{\erwaehntePersonen}{Personen: }
         \newcommand{\erwaehnteInstitutionen}{}
         \newcommand{\erwaehnteOrte}{}
         \newcommand{\erwaehnteWerke}{
               \section[Max Burckhard an Arthur Schnitzler, 5. 2. 1898]{ Max Burckhard an Arthur Schnitzler, 5. 2. 1898}\nopagebreak\mylabel{v}\rehead{ }\begin{ledgroupsized}[t]{13cm}\normalsize\beginnumbering \toendnotes[C]{\smallbreak\pagebreak[2]} \Standort{CUL, Schnitzler, B 20.}
\physDesc{Brief, 1 Blatt, 1 Seite
\newline{}Handschrift: schwarze Tinte, deutsche Kurrent\newline{}Ordnung: mit Bleistift von unbekannter Hand nummeriert:
                                    »12« }\toendnotes[C]{\smallbreak}\pstart
           \raggedleft{}{\pb}Wien\oindex{XXXX Ortsangabe fehlt|pw}{ }5. 2. 98.\pend
           \pstart{}Sehr verehrter lieber Herr Doctor!\pend\pstart
           Ich gratuliere Ihnen \uline{von Herzen} zu Ihrem geſtrigen
               ſchönen \label{K_L00773_1v}\edtext{Erfolg}{\lemma{\textnormal{\emph{Erfolg}}}\Cendnote{\textnormal{die Wien\oindex{XXXX Ortsangabe fehlt|pwk}er
                  Premiere von \emph{Freiwild}\textcolor{red}{\textsuperscript{XXXX indx}} am Carl-Theater\oindex{XXXX Ortsangabe fehlt|pwk} am 4. 2. 1898.}}}\label{K_L00773_1h}, den mir die Morgenblätter melden. Adieu ſage
               ich \uline{Ihnen} nicht, denn wir bleiben ja doch gute
               Nachbarn und ich darf ja auch ſagen \uline{gute Freunde}.
               Habe ich einmal ein biſſel Luft, ſo bin ich ſo frei zu Ihnen hinabzukommen und Ihnen
               auch noch mündlich zu sagen, wie herzlich mich Ihre Anweſenheit am \label{K_L00773_2v}\edtext{Mittwoch}{\lemma{\textnormal{\emph{Mittwoch}}}\Cendnote{\textnormal{beim Bankett zu Ehren Burckhards\pwindex{\textcolor{red}{\textsuperscript{XXXX1 indx}}|pwk}, das als Reaktion auf dessen Ablösung als Direktor des Burgtheater\oindex{XXXX Ortsangabe fehlt|pwk}s, am 2. 2. 1898 stattfand.}}}\label{K_L00773_2h} gefreut hat. Ihr
               Sie aufrichtig verehrender\pend
           \pstart
           \spacefill\mbox{D\textsuperscript{r}Burckhard}{\\[\baselineskip]}Herzlichſte Grüße!\pend
           \leftskip=0em{}
         
         \endnumbering\mylabel{h}\end{ledgroupsized}  \newcommand{\dateiname}{L00773}\newcommand{\titel}{Max Burckhard an Arthur Schnitzler, 5. 2. 1898}\newcommand{\editorInnen}{Martin Anton Müller und Gerd-Hermann Susen}%% latex-leseansicht-abspann.tex
%% Abspann für die Leseansicht.
%% Der Schalter \ifkorrekturansicht ist bereits durch den Vorspann gesetzt.

%% latex-abspann.tex
%% Gemeinsamer Abspann für Korrekturansicht und Leseansicht.
%% Setzt den Schalter \ifkorrekturansicht voraus (gesetzt in den
%% einbindenden Dateien latex-korrekturansicht-abspann.tex bzw.
%% latex-leseansicht-abspann.tex).
%% ---------------------------------------------------------------

\normalsize

% Das esempio-Environment wird nur in der Leseansicht benötigt
\ifkorrekturansicht\else
\newenvironment{esempio}[3]%
{
    \vspace{1.5ex}
    \rlap{\underline{#1}}
    \par
    \setlength{\parindent}{0cm}
    \nopagebreak
    \leftskip=#2cm
    \rightskip=#3cm
}
{
    \par
}
\fi

\doendnotes{C}
\bigskip
\vfill

\clearpage

\footnotesize

\ifkorrekturansicht
  \lohead{\textsc{register}}
\fi

% theindex-Environment neu definieren ohne reledmac
\makeatletter
\renewenvironment{theindex}{%
  \ifkorrekturansicht
    \section*{\indexname}%
  \else
    \subsubsection*{Index der erwähnten Entitäten}%
  \fi
  \setlength{\parindent}{0pt}%
  \setlength{\parskip}{0pt plus 0.3pt}%
  \let\item\@idxitem
}{%
  \ifkorrekturansicht\clearpage\fi
}
\makeatother

\IfFileExists{\jobname-pw.ind}{\input{\jobname-pw.ind}}{}

% Quellenangabe nur in der Leseansicht
\ifkorrekturansicht\else
% Fallback-Definitionen, falls die .tex-Datei \titel etc. nicht gesetzt hat
\providecommand{\titel}{}
\providecommand{\editorInnen}{}
\providecommand{\dateiname}{\jobname}

\vspace{3cm}

\vfill

\footnotesize
\textsc{Quelle}: \titel. Herausgegeben von {\editorInnen}. In: \emph{Arthur Schnitzler: Briefwechsel mit Autorinnen und Autoren}.
 Digitale Edition, https://schnitzler-briefe.acdh.oeaw.ac.at/{\dateiname}.html (Stand \today)
\fi

\end{document}


      