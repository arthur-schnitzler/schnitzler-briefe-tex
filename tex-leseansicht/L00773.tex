%% latex-leseansicht-vorspann.tex
%% Vorspann für die Leseansicht.
%% Lädt die gemeinsame Datei latex-vorspann.tex mit nicht gesetztem Schalter.

\newif\ifkorrekturansicht
\korrekturansichtfalse

\input{../tex-inputs/latex-vorspann}


\section[Max Burckhard an Arthur Schnitzler, 5. 2. 1898]{L00773 Max Burckhard an Arthur Schnitzler, 5. 2. 1898}
\nopagebreak\mylabel{L00773v}
\rehead{ }\normalsize\beginnumbering\briefempfaengerindex{Schnitzler, Arthur@\textsc{Schnitzler, Arthur}!zzzBurckhard, Max Eugen@\emph{von Max Eugen Burckhard}!1898-02-051@{5. 2. 1898}|(be}
\toendnotes[C]{\smallbreak\pagebreak[2]}
\correspDesc{Versand  durch Max Burckhard am 5. 2. 1898 in Wien
\newline{}Erhalt  durch Arthur Schnitzler im Zeitraum [5. 2. 1898
                  – 9. 2. 1898?] in Wien}\toendnotes[C]{\smallbreak}
\Standort{CUL, Schnitzler, B 20.}
\physDesc{Brief, 1 Blatt, 1 Seite, 482 Zeichen
\newline{}Handschrift: schwarze Tinte, deutsche Kurrent
\newline{}Ordnung: mit Bleistift von unbekannter Hand nummeriert:
                                    »12« }\toendnotes[C]{\smallbreak}
\pstart
           \raggedleft{}{\pb}Wien\oindex{Wien@\textbf{Wien}, \emph{Verwaltungsgebiet}|pw}{ }5. 2. 98.\pend
           
\pstart{}Sehr verehrter lieber Herr Doctor!\pend\vspace{0.5em}
\pstart
           Ich gratuliere Ihnen \uline{von Herzen} zu Ihrem geſtrigen{ }ſchönen \label{K_L00773-1v}\edtext{Erfolg}{\lemma{\textnormal{\emph{Erfolg}}}\Cendnote{\textnormal{Die Wien\oindex{Wien@\textbf{Wien}, \emph{Verwaltungsgebiet}|pwk}er Premiere von \emph{Freiwild}\pwindex{Schnitzler, Arthur 15.\,5.\,1862 Wien – 21.\,10.\,1931 ebd.@\textsc{Schnitzler, Arthur} (15.\,5.\,1862 Wien – 21.\,10.\,1931 ebd.), \emph{Schriftsteller, Mediziner}!Freiwild. Schauspiel in 3 Akten@\strich\emph{Freiwild. Schauspiel in 3 Akten}|pwk} fand am 4. 2. 1898 am \emph{Carl-Theater}\orgindex{Carl-Theater@Carl-Theater|pwk} statt.}}}\label{K_L00773-1}, den mir die
               Morgenblätter melden. Adieu{ }ſage ich \uline{Ihnen} nicht,
               denn wir bleiben ja doch gute Nachbarn und ich darf ja auch{ }ſagen \uline{gute Freunde}. Habe ich einmal ein biſſel Luft,{ }ſo bin
               ich{ }ſo frei zu Ihnen hinabzukommen und Ihnen auch noch mündlich zu sagen, wie
               herzlich mich Ihre Anweſenheit am \label{K_L00773-2v}\edtext{Mittwoch}{\lemma{\textnormal{\emph{Mittwoch}}}\Cendnote{\textnormal{Schnitzler hat am Bankett zu Ehren Burckhards\pwindex{Burckhard, Max Eugen 14.\,7.\,1854 Korneuburg – 16.\,3.\,1912 Wien@\textsc{Burckhard, Max Eugen} (14.\,7.\,1854 Korneuburg – 16.\,3.\,1912 Wien), \emph{Schriftsteller, Rechtswissenschaftler, Theaterleiter}|pwk} teilgenommen, das als Reaktion auf dessen
                  Ablösung als Direktor des Burgtheaters\oindex{Wien@\textbf{Wien}!I., Innere Stadt@\textbf{I., Innere Stadt}!Burgtheater@\textbf{Burgtheater}, \emph{Theater}|pwk} am 2. 2. 1898
                  stattgefunden hat.}}}\label{K_L00773-2} gefreut hat. Ihr Sie aufrichtig verehrender\pend
           
\pstart
           \spacefill\mbox{D\textsuperscript{r}Burckhard}{\\[\baselineskip]}Herzlichſte Grüße!\pend
           \leftskip=0em{}\selectlanguage{ngerman}\endnumbering\briefempfaengerindex{Schnitzler, Arthur@\textsc{Schnitzler, Arthur}!zzzBurckhard, Max Eugen@\emph{von Max Eugen Burckhard}!1898-02-051@{5. 2. 1898}|)be}\mylabel{L00773h}  \newcommand{\dateiname}{L00773}\newcommand{\titel}{Max Burckhard an Arthur Schnitzler, 5. 2. 1898}\newcommand{\editorInnen}{Martin Anton Müller und Gerd-Hermann Susen}%% latex-leseansicht-abspann.tex
%% Abspann für die Leseansicht.
%% Der Schalter \ifkorrekturansicht ist bereits durch den Vorspann gesetzt.

%% latex-abspann.tex
%% Gemeinsamer Abspann für Korrekturansicht und Leseansicht.
%% Setzt den Schalter \ifkorrekturansicht voraus (gesetzt in den
%% einbindenden Dateien latex-korrekturansicht-abspann.tex bzw.
%% latex-leseansicht-abspann.tex).
%% ---------------------------------------------------------------

\normalsize

% Das esempio-Environment wird nur in der Leseansicht benötigt
\ifkorrekturansicht\else
\newenvironment{esempio}[3]%
{
    \vspace{1.5ex}
    \rlap{\underline{#1}}
    \par
    \setlength{\parindent}{0cm}
    \nopagebreak
    \leftskip=#2cm
    \rightskip=#3cm
}
{
    \par
}
\fi

\doendnotes{C}
\bigskip
\vfill

\clearpage

\footnotesize

\ifkorrekturansicht
  \lohead{\textsc{register}}
\fi

% theindex-Environment neu definieren ohne reledmac
\makeatletter
\renewenvironment{theindex}{%
  \ifkorrekturansicht
    \section*{\indexname}%
  \else
    \subsubsection*{Index der erwähnten Entitäten}%
  \fi
  \setlength{\parindent}{0pt}%
  \setlength{\parskip}{0pt plus 0.3pt}%
  \let\item\@idxitem
}{%
  \ifkorrekturansicht\clearpage\fi
}
\makeatother

\IfFileExists{\jobname-pw.ind}{\input{\jobname-pw.ind}}{}

% Quellenangabe nur in der Leseansicht
\ifkorrekturansicht\else
% Fallback-Definitionen, falls die .tex-Datei \titel etc. nicht gesetzt hat
\providecommand{\titel}{}
\providecommand{\editorInnen}{}
\providecommand{\dateiname}{\jobname}

\vspace{3cm}

\vfill

\footnotesize
\textsc{Quelle}: \titel. Herausgegeben von {\editorInnen}. In: \emph{Arthur Schnitzler: Briefwechsel mit Autorinnen und Autoren}.
 Digitale Edition, https://schnitzler-briefe.acdh.oeaw.ac.at/{\dateiname}.html (Stand \today)
\fi

\end{document}


