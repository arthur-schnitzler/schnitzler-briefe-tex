%% latex-korrekturansicht-vorspann.tex
%% Vorspann für die Korrekturansicht.
%% Lädt die gemeinsame Datei latex-vorspann.tex mit gesetztem Schalter.

\newif\ifkorrekturansicht
\korrekturansichttrue

\input{../tex-inputs/latex-vorspann}


\section[Richard Beer-Hofmann an Arthur Schnitzler, 8. 9. 1904]{L01440 Richard Beer-Hofmann an Arthur Schnitzler, 8. 9. 1904}
\nopagebreak\mylabel{L01440v}
\rehead{ }\normalsize\beginnumbering\briefempfaengerindex{Schnitzler, Arthur@\textsc{Schnitzler, Arthur}!zzzBeer-Hofmann, Richard@\emph{von Richard Beer-Hofmann}!1904-09-081@{8. 9. 1904}|(be}
\toendnotes[C]{\smallbreak\pagebreak[2]}\Standort{CUL, Schnitzler, B 8.}
\physDesc{Brief, 1 Blatt, 1 Seite, 426 Zeichen
\newline{}Handschrift: blaue Tinte, lateinische Kurrent
\newline{}Ordnung: mit Bleistift von unbekannter Hand nummeriert:
                                    »188« }\toendnotes[C]{\smallbreak}
\pstart
           \centering{}{\pb}Aussee\oindex{Bad Aussee@\textbf{Bad Aussee}, \emph{P.PPLA3}|pw}{ }8/IX. 04.\pend
           \vspace{0.5em}
\pstart
           Lieber Arthur!{ }Zauner\pwindex{Zauner, Franz @\textsc{Zauner, Franz}, \emph{Stukkateur/Stukkateurin}|pwu} (nicht der Bildhauer\pwindex{Zauner, Franz Anton von 1746-07-05 – 1822-03-22@\textsc{Zauner, Franz Anton von} (1746-07-05 – 1822-03-22), \emph{Bildhauer/Bildhauerin}|pwv} – \label{K_L01440-1v}\edtext{der ist todt}{\lemma{\textnormal{\emph{der ist todt}}}\Cendnote{\textnormal{Unter
                  der Annahme, dass es sich bei Zauner\pwindex{Zauner, Franz @\textsc{Zauner, Franz}, \emph{Stukkateur/Stukkateurin}|pwk}
                  ebenfalls um einen »Franz Zauner« handelt, könnte es sich um einen Stukkateur
                  handeln, der 1904 in Wien\oindex{Wien@\textbf{Wien}, \emph{A.ADM2}|pwk} tätig
                  war.}}}\label{K_L01440-1}) sendet an Sie das Gewünschte.\pend
           
\pstart
           Wenn es so fortregnet bleiben Sie ja wol kaum in Lueg\oindex{Lueg@\textbf{Lueg}, \emph{Teil eines besiedelten Ortes (A.BSOX)}|pw}. Schreiben Sie mir, ob Sie nicht doch lieber ko{\geminationm}en, wo das Wetter durch meine Anwesenheit sich ja
               wesentlich mildert. Wenn Sie nicht ko{\geminationm}en – verständigen
               Sie mich, \strikeout{was} ob und wann Sie nach Salzburg\oindex{Salzburg@\textbf{Salzburg}, \emph{A.ADM2}|pw} gehen. Ich möchte nächste Woche – Beginn – auf 2 Tage
               hin.\pend
           
\pstart
           Herzlichst{\\[\baselineskip]}Ihr{\\[\baselineskip]}\spacefill\mbox{Richard.}\pend
           \leftskip=0em{}\selectlanguage{ngerman}\endnumbering\briefempfaengerindex{Schnitzler, Arthur@\textsc{Schnitzler, Arthur}!zzzBeer-Hofmann, Richard@\emph{von Richard Beer-Hofmann}!1904-09-081@{8. 9. 1904}|)be}\mylabel{L01440h}  \normalsize

\doendnotes{C}
\bigskip
\vfill

\clearpage

\footnotesize

\lohead{\textsc{register}}

% Definiere theindex-Environment komplett neu ohne reledmac
\makeatletter
\renewenvironment{theindex}{%
  \section*{\indexname}%
  \setlength{\parindent}{0pt}%
  \setlength{\parskip}{0pt plus 0.3pt}%
  \let\item\@idxitem
}{%
  \clearpage
}
\makeatother

\IfFileExists{\jobname-pw.ind}{\input{\jobname-pw.ind}}{}

\end{document}

      