%% latex-leseansicht-vorspann.tex
%% Vorspann für die Leseansicht.
%% Lädt die gemeinsame Datei latex-vorspann.tex mit nicht gesetztem Schalter.

\newif\ifkorrekturansicht
\korrekturansichtfalse

\input{../tex-inputs/latex-vorspann}


         
         \renewcommand{\erwaehntePersonen}{Personen: Franz Zauner, Franz Anton von Zauner}
         \renewcommand{\erwaehnteOrte}{Orte: Bad Aussee, Lueg am Wolfgangsee, Salzburg, St. Gilgen, Wien}
         \renewcommand{\erwaehnteWerke}{}
               \section[Richard Beer-Hofmann an Arthur Schnitzler, 8. 9. 1904]{ Richard Beer-Hofmann an Arthur Schnitzler, 8. 9. 1904}\nopagebreak\mylabel{v}\rehead{ }\begin{ledgroupsized}[t]{13cm}\normalsize\beginnumbering \toendnotes[C]{\smallbreak\pagebreak[2]} \Standort{CUL, Schnitzler, B 8.}
\physDesc{Brief, 1 Blatt, 1 Seite, 426 Zeichen
\newline{}Handschrift: blaue Tinte, lateinische Kurrent
\newline{}Ordnung: mit Bleistift von unbekannter Hand nummeriert:
                                    »188« }\toendnotes[C]{\smallbreak}\pstart
           \centering{}{\pb}Aussee\oindex{Bad Aussee@\textbf{Bad Aussee}|pw}{ }8/IX. 04.\pend
           \pstart
           Lieber Arthur!{ }Zauner\pwindex{Zauner, Franz @\textsc{Zauner, Franz}, \emph{Stukkateur}|pwu} (nicht der Bildhauer\pwindex{Zauner, Franz Anton von 1746-07-05 – 1822-03-22@\textsc{Zauner, Franz Anton von} (1746-07-05 – 1822-03-22), \emph{Bildhauer}|pwv} – \label{K_L01440-1v}\edtext{der ist todt}{\lemma{\textnormal{\emph{der ist todt}}}\Cendnote{\textnormal{Unter
                  der Annahme, dass es sich bei Zauner\pwindex{Zauner, Franz @\textsc{Zauner, Franz}, \emph{Stukkateur}|pwk}
                  ebenfalls um einen »Franz Zauner« handelt, könnte es sich um einen Stukkateur
                  handeln, der 1904 in Wien\oindex{Wien@\textbf{Wien}|pwk} tätig
                  war.}}}\label{K_L01440-1h}) sendet an Sie das Gewünschte.\pend
           \pstart
           Wenn es so fortregnet bleiben Sie ja wol kaum in Lueg\oindex{Lueg am Wolfgangsee@\textbf{Lueg am Wolfgangsee}|pw}. Schreiben Sie mir, ob Sie nicht doch lieber ko{\geminationm}en, wo das Wetter durch meine Anwesenheit sich ja
               wesentlich mildert. Wenn Sie nicht ko{\geminationm}en – verständigen
               Sie mich, \strikeout{was} ob und wann Sie nach Salzburg\oindex{Salzburg@\textbf{Salzburg}|pw} gehen. Ich möchte nächste Woche – Beginn – auf 2 Tage
               hin.\pend
           \pstart
           Herzlichst{\\[\baselineskip]}Ihr{\\[\baselineskip]}\spacefill\mbox{Richard.}\pend
           \leftskip=0em{}
         
         \endnumbering\mylabel{h}\end{ledgroupsized}  \newcommand{\dateiname}{L01440}\newcommand{\titel}{Richard Beer-Hofmann an Arthur Schnitzler, 8. 9. 1904}\newcommand{\editorInnen}{Martin Anton Müller und Gerd-Hermann Susen}%% latex-leseansicht-abspann.tex
%% Abspann für die Leseansicht.
%% Der Schalter \ifkorrekturansicht ist bereits durch den Vorspann gesetzt.

%% latex-abspann.tex
%% Gemeinsamer Abspann für Korrekturansicht und Leseansicht.
%% Setzt den Schalter \ifkorrekturansicht voraus (gesetzt in den
%% einbindenden Dateien latex-korrekturansicht-abspann.tex bzw.
%% latex-leseansicht-abspann.tex).
%% ---------------------------------------------------------------

\normalsize

% Das esempio-Environment wird nur in der Leseansicht benötigt
\ifkorrekturansicht\else
\newenvironment{esempio}[3]%
{
    \vspace{1.5ex}
    \rlap{\underline{#1}}
    \par
    \setlength{\parindent}{0cm}
    \nopagebreak
    \leftskip=#2cm
    \rightskip=#3cm
}
{
    \par
}
\fi

\doendnotes{C}
\bigskip
\vfill

\clearpage

\footnotesize

\ifkorrekturansicht
  \lohead{\textsc{register}}
\fi

% theindex-Environment neu definieren ohne reledmac
\makeatletter
\renewenvironment{theindex}{%
  \ifkorrekturansicht
    \section*{\indexname}%
  \else
    \subsubsection*{Index der erwähnten Entitäten}%
  \fi
  \setlength{\parindent}{0pt}%
  \setlength{\parskip}{0pt plus 0.3pt}%
  \let\item\@idxitem
}{%
  \ifkorrekturansicht\clearpage\fi
}
\makeatother

\IfFileExists{\jobname-pw.ind}{\input{\jobname-pw.ind}}{}

% Quellenangabe nur in der Leseansicht
\ifkorrekturansicht\else
% Fallback-Definitionen, falls die .tex-Datei \titel etc. nicht gesetzt hat
\providecommand{\titel}{}
\providecommand{\editorInnen}{}
\providecommand{\dateiname}{\jobname}

\vspace{3cm}

\vfill

\footnotesize
\textsc{Quelle}: \titel. Herausgegeben von {\editorInnen}. In: \emph{Arthur Schnitzler: Briefwechsel mit Autorinnen und Autoren}.
 Digitale Edition, https://schnitzler-briefe.acdh.oeaw.ac.at/{\dateiname}.html (Stand \today)
\fi

\end{document}


      