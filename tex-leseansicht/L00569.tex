%% latex-korrekturansicht-vorspann.tex
%% Vorspann für die Korrekturansicht.
%% Lädt die gemeinsame Datei latex-vorspann.tex mit gesetztem Schalter.

\newif\ifkorrekturansicht
\korrekturansichttrue

\input{../tex-inputs/latex-vorspann}


\section[Arthur Schnitzler an Richard Beer-Hofmann, 2{[}6?{]}. 7. 1896]{L00569 Arthur Schnitzler an Richard Beer-Hofmann, 2{[}6?{]}. 7. 1896}
\nopagebreak\mylabel{L00569v}
\rehead{ }\normalsize\beginnumbering\briefempfaengerindex{Beer-Hofmann, Richard@\textsc{Beer-Hofmann, Richard}!zzzSchnitzler, Arthur@\emph{von Arthur Schnitzler}!1896-07-262@{2{[}6?{]}. 7. 1896}|(be}
\toendnotes[C]{\smallbreak\pagebreak[2]}\Standort{YCGL, MSS 31.}
\physDesc{Postkarte, 182 Zeichen
\newline{}Handschrift: 1) Bleistift, deutsche Kurrent\hspace{1em}2) Bleistift, lateinische Kurrent (\noindent{}Adresse)\hspace{1em}
\newline{}Versand: 1) Stempel: »\nobreak{}\oindex{Oslo@\textbf{Oslo}, \emph{P.PPLC}|pwk}Kristiania, {[}26{]} VII 96\nobreak{}«.   2) Stempel: »\nobreak{}\oindex{Kopenhagen@\textbf{Kopenhagen}, \emph{P.PPLC}|pwk}Kjøbenhavn, 27. 7. 96, 10MB\nobreak{}«. }\pstart{}{\pb}Dr. \strikeout{Paul} Rich.
                  Beer-Hofmann\pend{}\pstart{}Kopenhagen\oindex{Kopenhagen@\textbf{Kopenhagen}, \emph{P.PPLC}|pw}\pend{}\pstart{}post rest.\pend{}{\bigskip}\vspace{1em}
\pstart
           \noindent{}{\pb}Lieber Richard,{ }ſobald Sie Ihre Adreſſe \textsc{Kopenhagen}\oindex{Kopenhagen@\textbf{Kopenhagen}, \emph{P.PPLC}|pw} u näheres \textsc{Skodsborg}\oindex{Skodsborg@\textbf{Skodsborg}, \emph{P.PPL}|pw} wiſſen, telegrafir\textcolor{gray}{en} Sie \textsc{Goldmann\pwindex{Goldmann, Paul 31.01.1865 – 25.09.1935@\textsc{Goldmann, Paul} (31.01.1865 – 25.09.1935), \emph{Schriftsteller/Schriftstellerin, Journalist/Journalistin}|pw}}{ }\textsc{Paris\oindex{Paris@\textbf{Paris}, \emph{P.PPLC}|pw}}{ }\textsc{24 rue Feydeau\oindex{rue Feydeau@\textbf{rue Feydeau}, \emph{Straße (K.STR)}|pw}}\pend
           
\pstart
           Herzlich Ihr{\\[\baselineskip]}\spacefill\mbox{Arthur}\pend
           \leftskip=0em{}\selectlanguage{ngerman}\endnumbering\briefempfaengerindex{Beer-Hofmann, Richard@\textsc{Beer-Hofmann, Richard}!zzzSchnitzler, Arthur@\emph{von Arthur Schnitzler}!1896-07-262@{2{[}6?{]}. 7. 1896}|)be}\mylabel{L00569h}  \normalsize

\doendnotes{C}
\bigskip
\vfill

\clearpage

\footnotesize

\lohead{\textsc{register}}

% Definiere theindex-Environment komplett neu ohne reledmac
\makeatletter
\renewenvironment{theindex}{%
  \section*{\indexname}%
  \setlength{\parindent}{0pt}%
  \setlength{\parskip}{0pt plus 0.3pt}%
  \let\item\@idxitem
}{%
  \clearpage
}
\makeatother

\IfFileExists{\jobname-pw.ind}{\input{\jobname-pw.ind}}{}

\end{document}

      