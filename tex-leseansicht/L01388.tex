\input{../tex-inputs/latex-pdf-vorspann}
\begin{center}
            \textcolor{red}{ENTWURF. ENTZIFFERUNG NOCH NICHT KORREKTURGELESEN}
                      \end{center}
            
               \section[Georg Brandes an Arthur Schnitzler, {[}März 1904{]}]{ Georg Brandes an Arthur Schnitzler, {[}März 1904{]}}\nopagebreak\mylabel{v}\rehead{ }\begin{ledgroupsized}[t]{13cm}\normalsize\beginnumbering\briefempfaengerindex{Schnitzler, Arthur@\textsc{Schnitzler, Arthur}!zzzBrandes, Georg@\emph{von Georg Brandes}!1904-03-011@{{[}März 1904{]}}|(be} \toendnotes[C]{\smallbreak\pagebreak[2]} \Standort{CUL, Schnitzler, B 17.}
\physDesc{Brief
\newline{}Handschrift: blaue Tinte, lateinische Kurrent
\newline{}Schnitzler: mit Bleistift datiert: »März 904« \newline{}Ordnung: mit Bleistift von unbekannter Hand nummeriert:
                                    »29« \newline{}Zusatz: auf dem abgeschnittenen und
                                    teilweise abgerissenen Blatt steht von unbekannter Hand mit
                                    schwarzer Tinte »Julius« }\buchAbdrucke{\weitereDrucke{Georg Brandes, Arthur Schnitzler: \emph{Ein Briefwechsel}. Hg. Kurt Bergel. Bern: \emph{Francke} 1956, S. 91.} }\toendnotes[C]{\smallbreak}\pstart
           \noindent{}\centering{}{\pb}Georg Brandes\pend
           \pstart
           \noindent{}herzlich grüssend und für die letzte schöne Zusendung\pwindex{Schnitzler, Arthur 15.05.1862 – 21.10.1931@\textsc{Schnitzler, Arthur} (15.05.1862 – 21.10.1931), \emph{Schriftsteller, Mediziner}!einsame Weg. Schauspiel in fuenf Akten1904@\strich\emph{Der einsame Weg. Schauspiel in fünf Akten} {[}1904{]}|pwv} dankend\pend
           \endnumbering\briefempfaengerindex{Schnitzler, Arthur@\textsc{Schnitzler, Arthur}!zzzBrandes, Georg@\emph{von Georg Brandes}!1904-03-011@{{[}März 1904{]}}|)be}\mylabel{h}\end{ledgroupsized}  \newcommand{\dateiname}{L01388}\newcommand{\titel}{Georg Brandes an Arthur Schnitzler, [März 1904]}\newcommand{\editorInnen}{Martin Anton Müller und Gerd-Hermann Susen}\input{../tex-inputs/latex-pdf-abspann}
      