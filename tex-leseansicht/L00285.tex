%% latex-leseansicht-vorspann.tex
%% Vorspann für die Leseansicht.
%% Lädt die gemeinsame Datei latex-vorspann.tex mit nicht gesetztem Schalter.

\newif\ifkorrekturansicht
\korrekturansichtfalse

\input{../tex-inputs/latex-vorspann}


\section[Arthur Schnitzler an Hermann Bahr, 2. 12. 1893]{L00285 Arthur Schnitzler an Hermann Bahr, 2. 12. 1893}
\nopagebreak\mylabel{L00285v}
\rehead{ }\textcolor{red}{WERKINDEX FEHLER}\normalsize\beginnumbering\briefempfaengerindex{Bahr, Hermann@\textsc{Bahr, Hermann}!zzzSchnitzler, Arthur@\emph{von Arthur Schnitzler}!1893-12-021@{2. 12. 1893}|(be}
\toendnotes[C]{\smallbreak\pagebreak[2]}
\correspDesc{Versand  durch Arthur Schnitzler am 2. 12. 1893 in Wien
\newline{}Erhalt  durch Hermann Bahr im Zeitraum [2. 12. 1893
                  – 6. 12. 1893?] in Wien}\toendnotes[C]{\smallbreak}
\Standort{TMW, FS PK266826.}
\physDesc{Fotografie, 1 Blatt, 1 Seite, 76 Zeichen (Fotografie von Carl
                                    Pietzner\pwindex{Pietzner, Carl 9.\,4.\,1853 Wriezen – 25.\,11.\,1927 Wien@\textsc{Pietzner, Carl} (9.\,4.\,1853 Wriezen – 25.\,11.\,1927 Wien), \emph{Fotograf}|pw})
\newline{}Handschrift: schwarze Tinte, deutsche Kurrent
\newline{}Zusatz: von unbekannter Hand Plattennummer auf der Rückseite vermerkt:
                                    »13955.« }
\buchAbdrucke{\weitereDrucke{Hermann Bahr, Arthur Schnitzler: \emph{Briefwechsel, Aufzeichnungen, Dokumente (1891–1931)}. Herausgegeben von Kurt Ifkovits und Martin Anton Müller. Göttingen: \emph{Wallstein} 2018, S. 54.} }\begin{figure}[H]\centering\includegraphics[width=10cm]{../tex-inputs/img/Arthur-Schnitzler_an_Hermann-Bahr_00005.jpg}\end{figure}\vspace{1em}
\pstart
           \noindent{}{\pb}Meinem lieben Freund \textsc{Hermann
                  Bahr} in herzlicher Verehrung\pwindex{Pietzner, Carl 9.\,4.\,1853 Wriezen – 25.\,11.\,1927 Wien@\textsc{Pietzner, Carl} (9.\,4.\,1853 Wriezen – 25.\,11.\,1927 Wien), \emph{Fotograf}!Arthur Schnitzler (1893)@\strich\emph{Arthur Schnitzler (1893)}|pw}\pend
           \pstart \spacefill\mbox{ArthSch}\pend{}
\pstart
           \noindent{}\uline{Wien\oindex{Wien@\textbf{Wien}, \emph{Verwaltungsgebiet}|pw}, 2. 12 93.}\pend
           
\pstart
           \centering{}\textcolor{gray}{\textbf{\textsc{K. u. k. Hof-Photograph C. Pietzner}\pwindex{Pietzner, Carl 9.\,4.\,1853 Wriezen – 25.\,11.\,1927 Wien@\textsc{Pietzner, Carl} (9.\,4.\,1853 Wriezen – 25.\,11.\,1927 Wien), \emph{Fotograf}|pw}}}\pend
           \selectlanguage{ngerman}\endnumbering\briefempfaengerindex{Bahr, Hermann@\textsc{Bahr, Hermann}!zzzSchnitzler, Arthur@\emph{von Arthur Schnitzler}!1893-12-021@{2. 12. 1893}|)be}\mylabel{L00285h}  \newcommand{\dateiname}{L00285}\newcommand{\titel}{Arthur Schnitzler an Hermann Bahr, 2. 12. 1893}\newcommand{\editorInnen}{Herausgegeben von Martin Anton Müller}%% latex-leseansicht-abspann.tex
%% Abspann für die Leseansicht.
%% Der Schalter \ifkorrekturansicht ist bereits durch den Vorspann gesetzt.

%% latex-abspann.tex
%% Gemeinsamer Abspann für Korrekturansicht und Leseansicht.
%% Setzt den Schalter \ifkorrekturansicht voraus (gesetzt in den
%% einbindenden Dateien latex-korrekturansicht-abspann.tex bzw.
%% latex-leseansicht-abspann.tex).
%% ---------------------------------------------------------------

\normalsize

% Das esempio-Environment wird nur in der Leseansicht benötigt
\ifkorrekturansicht\else
\newenvironment{esempio}[3]%
{
    \vspace{1.5ex}
    \rlap{\underline{#1}}
    \par
    \setlength{\parindent}{0cm}
    \nopagebreak
    \leftskip=#2cm
    \rightskip=#3cm
}
{
    \par
}
\fi

\doendnotes{C}
\bigskip
\vfill

\clearpage

\footnotesize

\ifkorrekturansicht
  \lohead{\textsc{register}}
\fi

% theindex-Environment neu definieren ohne reledmac
\makeatletter
\renewenvironment{theindex}{%
  \ifkorrekturansicht
    \section*{\indexname}%
  \else
    \subsubsection*{Index der erwähnten Entitäten}%
  \fi
  \setlength{\parindent}{0pt}%
  \setlength{\parskip}{0pt plus 0.3pt}%
  \let\item\@idxitem
}{%
  \ifkorrekturansicht\clearpage\fi
}
\makeatother

\IfFileExists{\jobname-pw.ind}{\input{\jobname-pw.ind}}{}

% Quellenangabe nur in der Leseansicht
\ifkorrekturansicht\else
% Fallback-Definitionen, falls die .tex-Datei \titel etc. nicht gesetzt hat
\providecommand{\titel}{}
\providecommand{\editorInnen}{}
\providecommand{\dateiname}{\jobname}

\vspace{3cm}

\vfill

\footnotesize
\textsc{Quelle}: \titel. Herausgegeben von {\editorInnen}. In: \emph{Arthur Schnitzler: Briefwechsel mit Autorinnen und Autoren}.
 Digitale Edition, https://schnitzler-briefe.acdh.oeaw.ac.at/{\dateiname}.html (Stand \today)
\fi

\end{document}


