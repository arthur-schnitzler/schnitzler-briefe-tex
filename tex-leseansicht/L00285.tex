%% latex-korrekturansicht-vorspann.tex
%% Vorspann für die Korrekturansicht.
%% Lädt die gemeinsame Datei latex-vorspann.tex mit gesetztem Schalter.

\newif\ifkorrekturansicht
\korrekturansichttrue

\input{../tex-inputs/latex-vorspann}


\section[Arthur Schnitzler an Hermann Bahr, 2. 12. 1893]{L00285 Arthur Schnitzler an Hermann Bahr, 2. 12. 1893}
\nopagebreak\mylabel{L00285v}
\rehead{ }\pwindex{XXXX Abgedrucktes Werk, Nummer nicht vorhanden|pwt}\normalsize\beginnumbering\briefempfaengerindex{Bahr, Hermann@\textsc{Bahr, Hermann}!zzzSchnitzler, Arthur@\emph{von Arthur Schnitzler}!1893-12-021@{2. 12. 1893}|(be}
\toendnotes[C]{\smallbreak\pagebreak[2]}\Standort{TMW, FS PK266826.}
\physDesc{Fotografie, 1 Blatt, 1 Seite, 76 Zeichen (Fotografie von Carl
                                    Pietzner\pwindex{Pietzner, Carl 1853-04-09 – 1927-11-25@\textsc{Pietzner, Carl} (1853-04-09 – 1927-11-25), \emph{Fotograf/Fotografin}|pw})
\newline{}Handschrift: schwarze Tinte, deutsche Kurrent
\newline{}Zusatz: von unbekannter Hand Plattennummer auf der Rückseite vermerkt:
                                    »13955.« }
\buchAbdrucke{\weitereDrucke{Hermann Bahr, Arthur Schnitzler: \emph{Briefwechsel, Aufzeichnungen, Dokumente (1891–1931)}. Göttingen: \emph{Wallstein} 2018, S. 54.} }\begin{figure}[H]\centering\includegraphics[width=10cm]{../tex-inputs/img/FS_PK277797alt_V.jpg}\end{figure}\vspace{1em}
\pstart
           \noindent{}{\pb}Meinem lieben Freund \textsc{Hermann
                  Bahr} in herzlicher Verehrung\pwindex{Arthur Schnitzler (1893)@\emph{Arthur Schnitzler (1893)}|pw}\pend
           \pstart \spacefill\mbox{ArthSch}\pend{}
\pstart
           \noindent{}\uline{Wien\oindex{Wien@\textbf{Wien}, \emph{A.ADM2}|pw}, 2. 12
                        93.}\pend
           
\pstart
           \centering{}\textcolor{gray}{\textbf{\textsc{K. u. k. Hof-Photograph C. Pietzner}\pwindex{Pietzner, Carl 1853-04-09 – 1927-11-25@\textsc{Pietzner, Carl} (1853-04-09 – 1927-11-25), \emph{Fotograf/Fotografin}|pw}}}\pend
           \selectlanguage{ngerman}\endnumbering\briefempfaengerindex{Bahr, Hermann@\textsc{Bahr, Hermann}!zzzSchnitzler, Arthur@\emph{von Arthur Schnitzler}!1893-12-021@{2. 12. 1893}|)be}\mylabel{L00285h}  \normalsize

\doendnotes{C}
\bigskip
\vfill

\clearpage

\footnotesize

\lohead{\textsc{register}}

% Definiere theindex-Environment komplett neu ohne reledmac
\makeatletter
\renewenvironment{theindex}{%
  \section*{\indexname}%
  \setlength{\parindent}{0pt}%
  \setlength{\parskip}{0pt plus 0.3pt}%
  \let\item\@idxitem
}{%
  \clearpage
}
\makeatother

\IfFileExists{\jobname-pw.ind}{\input{\jobname-pw.ind}}{}

\end{document}

      