%% latex-leseansicht-vorspann.tex
%% Vorspann für die Leseansicht.
%% Lädt die gemeinsame Datei latex-vorspann.tex mit nicht gesetztem Schalter.

\newif\ifkorrekturansicht
\korrekturansichtfalse

\input{../tex-inputs/latex-vorspann}


         
         \renewcommand{\erwaehntePersonen}{Personen: Hermann Bahr, Richard Beer-Hofmann, Hugo von Hofmannsthal, Hilda von Mitis,  Porges}
         \renewcommand{\erwaehnteOrte}{Orte: Brüssel, Paris, Wien}
         \renewcommand{\erwaehnteWerke}{Werke: Das Märchen. Schauspiel in drei Aufzügen, Die Hochzeit von Valeni, Tagebuch}
               \section[Paul Goldmann an Arthur Schnitzler, 18. 12. 1891]{ Paul Goldmann an Arthur Schnitzler, 18. 12. 1891}\nopagebreak\mylabel{v}\rehead{ }\begin{ledgroupsized}[t]{13cm}\normalsize\beginnumbering \toendnotes[C]{\smallbreak\pagebreak[2]} \Standort{DLA, A:Schnitzler, HS.NZ85.1.3162.}
\physDesc{Brief, 1 Blatt, 4 Seiten, 2569 Zeichen
\newline{}Handschrift: schwarze Tinte, deutsche Kurrent
\newline{}Schnitzler: mit rotem Buntstift auf der dritten und vierten Seite je eine
                                 seitliche Markierung }\toendnotes[C]{\smallbreak}\pstart
           \raggedleft{}{\pb}\textsc{Paris\oindex{Paris@\textbf{Paris}|pw}}, 18. December 1891.\pend
           \pstart\center{}Mein lieber Arthur!\pend\pstart
           Unſer alter Streit! Aber ich fürchte, Deine Kunſt läuft in einen Irrweg hinein, wenn
               Du Dich immer wieder von dieſen Ideen leiten läßt. Darum noch raſch drei Worte. Es
               gibt keine Kunſt, meine ich, die ſo \strikeout{f\textcolor{gray}{a}} den Maſſen angehört, als die dramatiſche. Es iſt ſogar das Weſen dieſer Kunſt
               und ihre eigentliche Aufgabe: Alles in den Maſſen ſichtbaren und fühlbaren
               Proportionen auszudrücken. Der Dramatiker bearbeitet nicht ſeinen Stoff, ſondern das
               Publicum. Das Publicum iſt das Rohmaterial des Bühnendichters. Und die Kunſt, ein
               Stück zu ſchreiben, iſt eigentlich die Kunſt, ſich ein Publicum \textsc{resp.} ſich \uline{das} Publicum {\pb}zu dem ſeinen zu machen. Wer alſo bei ſeinen
               dramatiſchen Arbeiten von der Maſſe abſtrahiren will, gleicht dem Maler, der ſeine
               Bilder in die Luft malt. Es gibt kein Theater für Fünf, es gibt nur \uline{ein} Theater für \uline{Alle}.
               Stücke für fünf Leute ſchreiben iſt keine Kunſt mehr, ſondern ein Sport. Anderſeits
               iſt es weit gefehlt, daß alle Stücke »Hochzeiten von
                     \textsc{Valeni}\pwindex{\textcolor{red}{\textsuperscript{XXXX1 indx}}!Hochzeit von ValeniNone@\strich\emph{Die Hochzeit von Valeni} {[}None{]}|pw}\pwindex{\textcolor{red}{\textsuperscript{XXXX1 indx}}!Hochzeit von ValeniNone@\strich\emph{Die Hochzeit von Valeni} {[}None{]}|pw}« ſein müßten. Man ſoll nicht \uline{theatraliſch} ſein,
               ſondern nur dramatiſch. Intim, fein, ſenſitiv, meinetwegen, aber \uuline{dramatiſch}. Und der letzte Act\pwindex{Schnitzler, Arthur 15.05.1862 – 21.10.1931@\textsc{Schnitzler, Arthur} (15.05.1862 – 21.10.1931), \emph{Schriftsteller, Mediziner}!Maerchen. Schauspiel in drei Aufzuegen1893-12-01@\strich\emph{Das Märchen. Schauspiel in drei Aufzügen} {[}1893-12-01{]}|pwv} des »Märchen\pwindex{Schnitzler, Arthur 15.05.1862 – 21.10.1931@\textsc{Schnitzler, Arthur} (15.05.1862 – 21.10.1931), \emph{Schriftsteller, Mediziner}!Maerchen. Schauspiel in drei Aufzuegen1893-12-01@\strich\emph{Das Märchen. Schauspiel in drei Aufzügen} {[}1893-12-01{]}|pw}s« iſt
               nicht dramatiſch. Daß du aber ein Dramatiker biſt, {\pb}das beweiſt der erſte Act. Alſo keine künſtlichen Syntheſen einer neuen Kunſt,
               bitte! Die Erfindung der neuen Kunſt iſt nur ein Auskunftsmittel, um den
               Schwierigkeiten der alten auszuweichen. Darum ſollſt Du ſchreiben – Du kannſt es, ich
               gebe Dir mein Ehrenwort – aber keine Stücke für Zimmer mit rother Ampel-Beleuchtung
               und heruntergelaſſenen Jalouſien{\dotsfive}\pend
           \pstart
           \textsc{Hermann Bahr\pwindex{Bahr, Hermann 19.07.1863 – 15.01.1934@\textsc{Bahr, Hermann} (19.07.1863 – 15.01.1934), \emph{Schriftsteller, Kritiker}|pw}}? Wieſo \label{K_L02676-5v}\edtext{kommt der zu Euch}{\lemma{\textnormal{\emph{kommt der zu Euch}}}\Cendnote{\textnormal{Bahr\pwindex{Bahr, Hermann 19.07.1863 – 15.01.1934@\textsc{Bahr, Hermann} (19.07.1863 – 15.01.1934), \emph{Schriftsteller, Kritiker}|pwk} lebte seit 28. 11. 1891
                  wieder in Wien\oindex{Wien@\textbf{Wien}|pwk} und frequentierte auch private
                  Treffen mit Schnitzler\pwindex{Schnitzler, Arthur 15.05.1862 – 21.10.1931@\textsc{Schnitzler, Arthur} (15.05.1862 – 21.10.1931), \emph{Schriftsteller, Mediziner}|pwk}, Beer-Hofmann\pwindex{Beer-Hofmann, Richard 1866-07-11 – 1945-09-26@\textsc{Beer-Hofmann, Richard} (1866-07-11 – 1945-09-26), \emph{Schriftsteller}|pwk} und Hofmannsthal\pwindex{Hofmannsthal, Hugo von 1874-02-01 – 1929-07-15@\textsc{Hofmannsthal, Hugo von} (1874-02-01 – 1929-07-15), \emph{Schriftsteller}|pwk}.}}}\label{K_L02676-5h}? {\dots}\pend
           \pstart
           \textsc{Richard\pwindex{Beer-Hofmann, Richard 1866-07-11 – 1945-09-26@\textsc{Beer-Hofmann, Richard} (1866-07-11 – 1945-09-26), \emph{Schriftsteller}|pw}} thut mir ſehr weh, weil er mir nicht ſchreibt{\dotsfive}\pend
           \pstart
           Ich? Verlange nichts zu hören! Troſtlos! Der Käfig, der bisher in Brüſſel\oindex{Bruessel@\textbf{Brüssel}|pw} ſtand, iſt nun nach Paris\oindex{Paris@\textbf{Paris}|pw} übertragen; und die Gefangenſchaft wird nur {\pb}umſo bitterer dadurch, daß Paris\oindex{Paris@\textbf{Paris}|pw} vor den Gitterſtäben zu ſehen iſt. Talentlos, muthlos,
               gewiſſenlos! Langſchläferiſch und zeitvergeuderiſch! Am 1. Januar ſoll ich meinen Dienſt beginnen u. weiß nicht \uline{das} davon! Sechs Monate höchſtens wird’s dauern; dann
               ſchicken ſie mich fort. Faul, faul bin ich. Ich hab’s jetzt heraus: wir nennen uns
               andere, um einen Vorwand zu haben, charakterlos zu ſein{\dotsfour}\pend
           \pstart
           Mit Empfehlungen kannſt Du mir unendllich nützen. Ich bin faſt ganz im Stich gelaſſen
               worden u. brauche Beziehungen wie das Brot. Schaff’ mir, bitte, was Du mir ſchaffen
               kannſt. Auch wenn die andern Freunde mir ein wenig helfen wollten, wäre ich ſehr
               dankbar. Oder gar Dein Herr Porges\pwindex{Porges @\textsc{Porges}|pwu}! Grüße Dich Gott, mein lieber Alter!\pend
           \pstart Dein \spacefill\mbox{Paul Goldmann}\pend{}\pstart
           \noindent{}\label{T_L02676-1v}\edtext{\textsc{Hildegarde\pwindex{Mitis, Hilda von 1876-08-30 – 1894-12-14@\textsc{Mitis, Hilda von} (1876-08-30 – 1894-12-14), \emph{Schriftstellerin, Telefonistin}|pwuv}} haſt Du nie \label{K_L02676-1v}\edtext{geſehen}{\lemma{\textnormal{\emph{geſehen}}}\Cendnote{\textnormal{In Schnitzler\pwindex{Schnitzler, Arthur 15.05.1862 – 21.10.1931@\textsc{Schnitzler, Arthur} (15.05.1862 – 21.10.1931), \emph{Schriftsteller, Mediziner}|pwk}s \emph{Tagebuch}\pwindex{Schnitzler, Arthur 15.05.1862 – 21.10.1931@\textsc{Schnitzler, Arthur} (15.05.1862 – 21.10.1931), \emph{Schriftsteller, Mediziner}!Tagebuch1981 – 2000@\strich\emph{Tagebuch} {[}1981 – 2000{]}|pwk} ist kein
                     Treffen vermerkt.}}}\label{K_L02676-1h}?}{\lemma{\textnormal{\emph{Hildegarde … geſehen?}}}\Cendnote{\textnormal{kopfüber
                     am oberen Rand}}}\label{T_L02676-1h}\pend
           
         
         \endnumbering\mylabel{h}\end{ledgroupsized}  \newcommand{\dateiname}{L02676}\newcommand{\titel}{Paul Goldmann an Arthur Schnitzler, 18. 12. 1891}\newcommand{\editorInnen}{Martin Anton Müller und Laura Untner}%% latex-leseansicht-abspann.tex
%% Abspann für die Leseansicht.
%% Der Schalter \ifkorrekturansicht ist bereits durch den Vorspann gesetzt.

%% latex-abspann.tex
%% Gemeinsamer Abspann für Korrekturansicht und Leseansicht.
%% Setzt den Schalter \ifkorrekturansicht voraus (gesetzt in den
%% einbindenden Dateien latex-korrekturansicht-abspann.tex bzw.
%% latex-leseansicht-abspann.tex).
%% ---------------------------------------------------------------

\normalsize

% Das esempio-Environment wird nur in der Leseansicht benötigt
\ifkorrekturansicht\else
\newenvironment{esempio}[3]%
{
    \vspace{1.5ex}
    \rlap{\underline{#1}}
    \par
    \setlength{\parindent}{0cm}
    \nopagebreak
    \leftskip=#2cm
    \rightskip=#3cm
}
{
    \par
}
\fi

\doendnotes{C}
\bigskip
\vfill

\clearpage

\footnotesize

\ifkorrekturansicht
  \lohead{\textsc{register}}
\fi

% theindex-Environment neu definieren ohne reledmac
\makeatletter
\renewenvironment{theindex}{%
  \ifkorrekturansicht
    \section*{\indexname}%
  \else
    \subsubsection*{Index der erwähnten Entitäten}%
  \fi
  \setlength{\parindent}{0pt}%
  \setlength{\parskip}{0pt plus 0.3pt}%
  \let\item\@idxitem
}{%
  \ifkorrekturansicht\clearpage\fi
}
\makeatother

\IfFileExists{\jobname-pw.ind}{\input{\jobname-pw.ind}}{}

% Quellenangabe nur in der Leseansicht
\ifkorrekturansicht\else
% Fallback-Definitionen, falls die .tex-Datei \titel etc. nicht gesetzt hat
\providecommand{\titel}{}
\providecommand{\editorInnen}{}
\providecommand{\dateiname}{\jobname}

\vspace{3cm}

\vfill

\footnotesize
\textsc{Quelle}: \titel. Herausgegeben von {\editorInnen}. In: \emph{Arthur Schnitzler: Briefwechsel mit Autorinnen und Autoren}.
 Digitale Edition, https://schnitzler-briefe.acdh.oeaw.ac.at/{\dateiname}.html (Stand \today)
\fi

\end{document}


      