%% latex-korrekturansicht-vorspann.tex
%% Vorspann für die Korrekturansicht.
%% Lädt die gemeinsame Datei latex-vorspann.tex mit gesetztem Schalter.

\newif\ifkorrekturansicht
\korrekturansichttrue

\input{../tex-inputs/latex-vorspann}


\section[Paul Goldmann an Arthur Schnitzler, 18. 12. 1891]{L02676 Paul Goldmann an Arthur Schnitzler, 18. 12. 1891}
\nopagebreak\mylabel{L02676v}
\rehead{ }\normalsize\beginnumbering\briefempfaengerindex{Schnitzler, Arthur@\textsc{Schnitzler, Arthur}!zzzGoldmann, Paul@\emph{von Paul Goldmann}!1891-12-182@{18. 12. 1891}|(be}
\toendnotes[C]{\smallbreak\pagebreak[2]}\Standort{DLA, A:Schnitzler, HS.NZ85.1.3162.}
\physDesc{Brief, 1 Blatt, 4 Seiten, 2569 Zeichen
\newline{}Handschrift: schwarze Tinte, deutsche Kurrent
\newline{}Schnitzler: mit rotem Buntstift auf der dritten und vierten Seite je eine
                                 seitliche Markierung }\toendnotes[C]{\smallbreak}
\pstart
           \raggedleft{}{\pb}\textsc{Paris\oindex{Paris@\textbf{Paris}, \emph{P.PPLC}|pw}}, 18. December 1891.\pend
           
\pstart\center{}Mein lieber Arthur!\pend\vspace{0.5em}
\pstart
           Unſer alter Streit! Aber ich fürchte, Deine Kunſt läuft in einen Irrweg hinein, wenn
               Du Dich immer wieder von dieſen Ideen leiten läßt. Darum noch raſch drei Worte. Es
               gibt keine Kunſt, meine ich, die ſo \strikeout{f\textcolor{gray}{a}} den Maſſen angehört, als die dramatiſche. Es iſt ſogar das Weſen dieſer Kunſt
               und ihre eigentliche Aufgabe: Alles in den Maſſen ſichtbaren und fühlbaren
               Proportionen auszudrücken. Der Dramatiker bearbeitet nicht ſeinen Stoff, ſondern das
               Publicum. Das Publicum iſt das Rohmaterial des Bühnendichters. Und die Kunſt, ein
               Stück zu ſchreiben, iſt eigentlich die Kunſt, ſich ein Publicum \textsc{resp.} ſich \uline{das} Publicum {\pb}zu dem ſeinen zu machen. Wer alſo bei ſeinen
               dramatiſchen Arbeiten von der Maſſe abſtrahiren will, gleicht dem Maler, der ſeine
               Bilder in die Luft malt. Es gibt kein Theater für Fünf, es gibt nur \uline{ein} Theater für \uline{Alle}.
               Stücke für fünf Leute ſchreiben iſt keine Kunſt mehr, ſondern ein Sport. Anderſeits
               iſt es weit gefehlt, daß alle Stücke »Hochzeiten von
                     \textsc{Valeni}\pwindex{Hochzeit von Valeni@\emph{Die Hochzeit von Valeni}|pw}« ſein müßten. Man ſoll nicht \uline{theatraliſch} ſein,
               ſondern nur dramatiſch. Intim, fein, ſenſitiv, meinetwegen, aber \uuline{dramatiſch}. Und der letzte Act\pwindex{Maerchen. Schauspiel in drei Aufzuegen@\emph{Das Märchen. Schauspiel in drei Aufzügen}|pwv} des »Märchens\pwindex{Maerchen. Schauspiel in drei Aufzuegen@\emph{Das Märchen. Schauspiel in drei Aufzügen}|pw}« iſt
               nicht dramatiſch. Daß du aber ein Dramatiker biſt, {\pb}das beweiſt der erſte Act. Alſo keine künſtlichen Syntheſen einer neuen Kunſt,
               bitte! Die Erfindung der neuen Kunſt iſt nur ein Auskunftsmittel, um den
               Schwierigkeiten der alten auszuweichen. Darum ſollſt Du ſchreiben – Du kannſt es, ich
               gebe Dir mein Ehrenwort – aber keine Stücke für Zimmer mit rother Ampel-Beleuchtung
               und heruntergelaſſenen Jalouſien{\dotsfive}\pend
           
\pstart
           \textsc{Hermann Bahr\pwindex{Bahr, Hermann 19.07.1863 – 15.01.1934@\textsc{Bahr, Hermann} (19.07.1863 – 15.01.1934), \emph{Schriftsteller/Schriftstellerin, Kritiker/Kritikerin}|pw}}? Wieſo \label{K_L02676-1v}\edtext{kommt der zu Euch}{\lemma{\textnormal{\emph{kommt der zu Euch}}}\Cendnote{\textnormal{Bahr\pwindex{Bahr, Hermann 19.07.1863 – 15.01.1934@\textsc{Bahr, Hermann} (19.07.1863 – 15.01.1934), \emph{Schriftsteller/Schriftstellerin, Kritiker/Kritikerin}|pwk} lebte seit 28. 11. 1891
                  wieder in Wien\oindex{Wien@\textbf{Wien}, \emph{A.ADM2}|pwk} und frequentierte auch private
                  Treffen mit Schnitzler, Beer-Hofmann\pwindex{Beer-Hofmann, Richard 1866-07-11 – 1945-09-26@\textsc{Beer-Hofmann, Richard} (1866-07-11 – 1945-09-26), \emph{Schriftsteller/Schriftstellerin}|pwk} und Hofmannsthal\pwindex{Hofmannsthal, Hugo von 1874-02-01 – 1929-07-15@\textsc{Hofmannsthal, Hugo von} (1874-02-01 – 1929-07-15), \emph{Schriftsteller/Schriftstellerin}|pwk}.}}}\label{K_L02676-1}? {\dots}\pend
           
\pstart
           \textsc{Richard\pwindex{Beer-Hofmann, Richard 1866-07-11 – 1945-09-26@\textsc{Beer-Hofmann, Richard} (1866-07-11 – 1945-09-26), \emph{Schriftsteller/Schriftstellerin}|pw}} thut mir ſehr weh, weil er mir nicht ſchreibt{\dotsfive}\pend
           
\pstart
           Ich? Verlange nichts zu hören! Troſtlos! Der Käfig, der bisher in Brüſſel\oindex{Bruessel@\textbf{Brüssel}, \emph{P.PPLC}|pw} ſtand, iſt nun nach Paris\oindex{Paris@\textbf{Paris}, \emph{P.PPLC}|pw} übertragen; und die Gefangenſchaft wird nur {\pb}umſo bitterer dadurch, daß Paris\oindex{Paris@\textbf{Paris}, \emph{P.PPLC}|pw} vor den Gitterſtäben zu ſehen iſt. Talentlos, muthlos,
               gewiſſenlos! Langſchläferiſch und zeitvergeuderiſch! Am 1. Januar ſoll ich meinen Dienſt beginnen u. weiß nicht \uline{das} davon! Sechs Monate höchſtens wird’s dauern; dann
               ſchicken ſie mich fort. Faul, faul bin ich. Ich hab’s jetzt heraus: wir nennen uns
               andere, um einen Vorwand zu haben, charakterlos zu ſein{\dotsfour}\pend
           
\pstart
           Mit Empfehlungen kannſt Du mir unendllich nützen. Ich bin faſt ganz im Stich gelaſſen
               worden u. brauche Beziehungen wie das Brot. Schaff’ mir, bitte, was Du mir ſchaffen
               kannſt. Auch wenn die andern Freunde mir ein wenig helfen wollten, wäre ich ſehr
               dankbar. Oder gar Dein Herr Porges\pwindex{Porges @\textsc{Porges}|pwu}! Grüße Dich Gott, mein lieber Alter!\pend
           \pstart Dein \spacefill\mbox{Paul Goldmann}\pend{}
\pstart
           \noindent{}\label{T_L02676-1v}\edtext{\textsc{Hildegarde\pwindex{Mitis, Hilda von 1876-08-30 – 1894-12-14@\textsc{Mitis, Hilda von} (1876-08-30 – 1894-12-14), \emph{Schriftsteller/Schriftstellerin, Telefonist/Telefonistin}|pwuv}} haſt Du nie \label{K_L02676-2v}\edtext{geſehen}{\lemma{\textnormal{\emph{geſehen}}}\Cendnote{\textnormal{In Schnitzlers{ }\emph{Tagebuch}\pwindex{Tagebuch@\emph{Tagebuch}|pwk} ist kein
                     Treffen vermerkt.}}}\label{K_L02676-2}?}{\lemma{\textnormal{\emph{Hildegarde … geſehen?}}}\Cendnote{\textnormal{kopfüber
                     am oberen Rand}}}\label{T_L02676-1}\pend
           \selectlanguage{ngerman}\endnumbering\briefempfaengerindex{Schnitzler, Arthur@\textsc{Schnitzler, Arthur}!zzzGoldmann, Paul@\emph{von Paul Goldmann}!1891-12-182@{18. 12. 1891}|)be}\mylabel{L02676h}  \normalsize

\doendnotes{C}
\bigskip
\vfill

\clearpage

\footnotesize

\lohead{\textsc{register}}

% Definiere theindex-Environment komplett neu ohne reledmac
\makeatletter
\renewenvironment{theindex}{%
  \section*{\indexname}%
  \setlength{\parindent}{0pt}%
  \setlength{\parskip}{0pt plus 0.3pt}%
  \let\item\@idxitem
}{%
  \clearpage
}
\makeatother

\IfFileExists{\jobname-pw.ind}{\input{\jobname-pw.ind}}{}

\end{document}

      