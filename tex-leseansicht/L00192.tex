%% latex-korrekturansicht-vorspann.tex
%% Vorspann für die Korrekturansicht.
%% Lädt die gemeinsame Datei latex-vorspann.tex mit gesetztem Schalter.

\newif\ifkorrekturansicht
\korrekturansichttrue

\input{../tex-inputs/latex-vorspann}


\section[Karl Kraus an Arthur Schnitzler, 22. 3. 1893]{L00192 Karl Kraus an Arthur Schnitzler, 22. 3. 1893}
\nopagebreak\mylabel{L00192v}
\rehead{ }\normalsize\beginnumbering\briefempfaengerindex{Schnitzler, Arthur@\textsc{Schnitzler, Arthur}!zzzKraus, Karl@\emph{von Karl Kraus}!1893-03-221@{22. 3. 1893}|(be}
\toendnotes[C]{\smallbreak\pagebreak[2]}\Standort{CUL, Schnitzler, B 55.}
\physDesc{Postkarte, 320 Zeichen
\newline{}Handschrift: Bleistift, deutsche Kurrent
\newline{}Versand: Stempel: »\nobreak{}\oindex{I., Innere Stadt@\textbf{I., Innere Stadt}, \emph{A.ADM3}|pwk}Wien 1/1, 22. 3. 93, 12–1 N\nobreak{}«.  }
\buchAbdrucke{\weitereDrucke{\emph{Literatur und Kritik}, Bd. 49, Oktober 1970, S. 517.} }\pstart{}{\pb}
                  Herrn Schriftſteller
               \pend{}\pstart{}
                  med. D
                  \textsuperscript{r}
                   Arthur Schnitzler
               \pend{}\pstart{}Wien I\oindex{I., Innere Stadt@\textbf{I., Innere Stadt}, \emph{A.ADM3}|pw}\pend{}\pstart{}Grillparzerstraße 7\oindex{Grillparzerstrasse@\textbf{Grillparzerstraße}, \emph{R.ST}|pw}
                  .
               \pend{}{\bigskip}\vspace{1em}
\pstart{}{\pb}
                  Lieber Herr Doctor!
               \pend\vspace{0.5em}
\pstart
           
               Beſten Dank für Ihre freundliche Antwort. Daſs Sie mit ſich ſprechen laſſen werden,
               wuſste ich ja längst. Ich komme heute, 
               Mittwoch
               , nach der Vorſtellung
               der »
               \uline{Kriemhilde\pwindex{Kriemhilde@\emph{Kriemhilde}|pw}}
               « i. e. nach 
               10 Uhr
                ins 
               Grienſteidl\oindex{Cafe Griensteidl@\textbf{Café Griensteidl}, \emph{Kaffeehaus (K.KAF)}|pw}
               .
            \pend
           
\pstart
           
               Ergebenſten Gruß
               {\\[\baselineskip]}
               Ihr
               \hspace*{3.5em}\spacefill\mbox{KarlKraus}\pend
           \leftskip=0em{}\selectlanguage{ngerman}\endnumbering\briefempfaengerindex{Schnitzler, Arthur@\textsc{Schnitzler, Arthur}!zzzKraus, Karl@\emph{von Karl Kraus}!1893-03-221@{22. 3. 1893}|)be}\mylabel{L00192h}  \normalsize

\doendnotes{C}
\bigskip
\vfill

\clearpage

\footnotesize

\lohead{\textsc{register}}

% Definiere theindex-Environment komplett neu ohne reledmac
\makeatletter
\renewenvironment{theindex}{%
  \section*{\indexname}%
  \setlength{\parindent}{0pt}%
  \setlength{\parskip}{0pt plus 0.3pt}%
  \let\item\@idxitem
}{%
  \clearpage
}
\makeatother

\IfFileExists{\jobname-pw.ind}{\input{\jobname-pw.ind}}{}

\end{document}

      