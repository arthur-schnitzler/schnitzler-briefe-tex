%% latex-leseansicht-vorspann.tex
%% Vorspann für die Leseansicht.
%% Lädt die gemeinsame Datei latex-vorspann.tex mit nicht gesetztem Schalter.

\newif\ifkorrekturansicht
\korrekturansichtfalse

\input{../tex-inputs/latex-vorspann}


\section[Hugo von Hofmannsthal an Arthur Schnitzler, {{[}}17. 2. 1892{{]}}]{L00072 Hugo von Hofmannsthal an Arthur Schnitzler, {[}17. 2. 1892{]}}
\nopagebreak\mylabel{L00072v}
\rehead{ }\normalsize\beginnumbering\briefempfaengerindex{Schnitzler, Arthur@\textsc{Schnitzler, Arthur}!zzzHofmannsthal, Hugo von@\emph{von Hugo von Hofmannsthal}!1892-02-171@{{[}17. 2. 1892{]}}|(be}
\toendnotes[C]{\smallbreak\pagebreak[2]}
\correspDesc{Versand  durch Hugo von Hofmannsthal am [17. 2. 1892] in Wien
\newline{}Erhalt  durch Arthur Schnitzler im Zeitraum [17. 2. 1892
                  – 21. 2. 1892?] in Wien}\toendnotes[C]{\smallbreak}
\Standort{CUL, Schnitzler, B 43.}
\physDesc{Brief, 1 Blatt, 1 Seite, 668 Zeichen
\newline{}Handschrift: schwarze Tinte, deutsche Kurrent
\newline{}Schnitzler: mit Bleistift datiert: »17/2 92« 
\newline{}Ordnung: mit Bleistift von unbekannter Hand nummeriert:
                                    »17« }
\buchAbdrucke{\weitereDrucke{1) Hugo von Hofmannsthal, Arthur Schnitzler: \emph{Briefwechsel}. Herausgegeben von Therese Nickl und Heinrich Schnitzler. Frankfurt am Main: \emph{S. Fischer} 1964, S. 16.} \weitereDrucke{2) Hermann Bahr, Arthur Schnitzler: \emph{Briefwechsel, Aufzeichnungen, Dokumente (1891–1931)}. Herausgegeben von Kurt Ifkovits und Martin Anton Müller. Göttingen: \emph{Wallstein} 2018, S. 21.} }\toendnotes[C]{\smallbreak}
\pstart
           \noindent{}\centering{}{\pb}Thatsachen:\pend
           
\pstart
           1.) Bitte adreſſieren Sie den
               beiliegenden Wiſch an Herrn Lothar\pwindex{Lothar, Rudolf 23.\,2.\,1865 Budapest – 2.\,10.\,1943 ebd.@\textsc{Lothar, Rudolf} (23.\,2.\,1865 Budapest – 2.\,10.\,1943 ebd.), \emph{Schriftsteller, Journalist, Theaterdirektor}|pw} und{ }ſchicken
               Sie ihn weg.\pend
           
\pstart
           2.) Maeterlinck\pwindex{Maeterlinck, Maurice 29.\,8.\,1862 Gent – 6.\,5.\,1949 Nizza@\textsc{Maeterlinck, Maurice} (29.\,8.\,1862 Gent – 6.\,5.\,1949 Nizza), \emph{Schriftsteller}|pw} hat mich zur Überſetzung
               freundlichſt autoriſiert.\pend
           
\pstart
           3.) Die Empfehlung an die Palmay\pwindex{Pálmay, Ilka 21.\,9.\,1859 Uzhhorod – 17.\,2.\,1944 Budapest@\textsc{Pálmay, Ilka} (21.\,9.\,1859 Uzhhorod – 17.\,2.\,1944 Budapest), \emph{Schriftstellerin, Schauspielerin, Sängerin}|pw} habe ich
               verlangt und werde{ }ſie Bahr\pwindex{Bahr, Hermann 19.\,7.\,1863 Linz – 15.\,1.\,1934 München@\textsc{Bahr, Hermann} (19.\,7.\,1863 Linz – 15.\,1.\,1934 München), \emph{Schriftsteller, Kritiker}|pw} nächſtens{ }ſchicken.\pend
           
\pstart
           4.) Vielleicht könnte Kafka\pwindex{Kafka, Eduard Michael 11.\,3.\,1869 Wien – 6.\,8.\,1893 Brünn@\textsc{Kafka, Eduard Michael} (11.\,3.\,1869 Wien – 6.\,8.\,1893 Brünn), \emph{Redakteur}|pw} die erſten
               Vierteljahrsbeiträge raſch einkaſſieren und uns gegen Garantie durch perſönliche
               Unterſchrift leihen. Das wären doch vielleicht 200 fl.\pend
           
\pstart
           5.) Suchen Sie Bauer\pwindex{Bauer, Arnold um 1840 – 19.\,7.\,1893 Baden bei Wien@\textsc{Bauer, Arnold} (um 1840 – 19.\,7.\,1893 Baden bei Wien), \emph{Mediziner, Herausgeber, Buchhändler}|pw} gegenüber uns wichtig und
               ernſt zu machen und trachten Sie, \introOben{}daß\introOben{} das erſte Heft\orgindex{Wiener Literatur-Zeitung@Wiener Literatur-Zeitung|pwv} möglichſt bald erſcheint.
               An die Premièren: Fulda\pwindex{Fulda, Ludwig 15.\,7.\,1862 Frankfurt am Main – 30.\,3.\,1939 Berlin@\textsc{Fulda, Ludwig} (15.\,7.\,1862 Frankfurt am Main – 30.\,3.\,1939 Berlin), \emph{Schriftsteller, Übersetzer}|pw} »Sclavin\pwindex{Fulda, Ludwig 15.\,7.\,1862 Frankfurt am Main – 30.\,3.\,1939 Berlin@\textsc{Fulda, Ludwig} (15.\,7.\,1862 Frankfurt am Main – 30.\,3.\,1939 Berlin), \emph{Schriftsteller, Übersetzer}!Sklavin. Schauspiel in vier Aufzügen@\strich\emph{Die Sklavin. Schauspiel in vier Aufzügen}|pw}«, \textsc{Griselidis}\pwindex{\textcolor{red}{\textsuperscript{XXXX indx1}}!Grisélidis. Oper in drei Akten und einem Prolog@\strich\emph{Grisélidis. Oper in drei Akten und einem Prolog}|pw}\pwindex{\textcolor{red}{\textsuperscript{XXXX indx1}}!Grisélidis. Oper in drei Akten und einem Prolog@\strich\emph{Grisélidis. Oper in drei Akten und einem Prolog}|pw} und Schleſinger\pwindex{Schlesinger, Sigmund 15.\,6.\,1832 Nové Mesto nad Váhom – 7.\,3.\,1918 Wien@\textsc{Schlesinger, Sigmund} (15.\,6.\,1832 Nové Mesto nad Váhom – 7.\,3.\,1918 Wien), \emph{Schriftsteller}|pw} »\textsc{Derby}\pwindex{Schlesinger, Sigmund 15.\,6.\,1832 Nové Mesto nad Váhom – 7.\,3.\,1918 Wien@\textsc{Schlesinger, Sigmund} (15.\,6.\,1832 Nové Mesto nad Váhom – 7.\,3.\,1918 Wien), \emph{Schriftsteller}!Derby@\strich\emph{Derby}|pw}« läſst{ }ſich künſtleriſch und{ }ſocial unendlich viel anhängen.\pend
           \pstart \spacefill\mbox{Loris.}\pend{}\selectlanguage{ngerman}\endnumbering\briefempfaengerindex{Schnitzler, Arthur@\textsc{Schnitzler, Arthur}!zzzHofmannsthal, Hugo von@\emph{von Hugo von Hofmannsthal}!1892-02-171@{{[}17. 2. 1892{]}}|)be}\mylabel{L00072h}  \newcommand{\dateiname}{L00072}\newcommand{\titel}{Hugo von Hofmannsthal an Arthur Schnitzler, [17. 2. 1892]}\newcommand{\editorInnen}{Herausgegeben von Martin Anton Müller}%% latex-leseansicht-abspann.tex
%% Abspann für die Leseansicht.
%% Der Schalter \ifkorrekturansicht ist bereits durch den Vorspann gesetzt.

%% latex-abspann.tex
%% Gemeinsamer Abspann für Korrekturansicht und Leseansicht.
%% Setzt den Schalter \ifkorrekturansicht voraus (gesetzt in den
%% einbindenden Dateien latex-korrekturansicht-abspann.tex bzw.
%% latex-leseansicht-abspann.tex).
%% ---------------------------------------------------------------

\normalsize

% Das esempio-Environment wird nur in der Leseansicht benötigt
\ifkorrekturansicht\else
\newenvironment{esempio}[3]%
{
    \vspace{1.5ex}
    \rlap{\underline{#1}}
    \par
    \setlength{\parindent}{0cm}
    \nopagebreak
    \leftskip=#2cm
    \rightskip=#3cm
}
{
    \par
}
\fi

\doendnotes{C}
\bigskip
\vfill

\clearpage

\footnotesize

\ifkorrekturansicht
  \lohead{\textsc{register}}
\fi

% theindex-Environment neu definieren ohne reledmac
\makeatletter
\renewenvironment{theindex}{%
  \ifkorrekturansicht
    \section*{\indexname}%
  \else
    \subsubsection*{Index der erwähnten Entitäten}%
  \fi
  \setlength{\parindent}{0pt}%
  \setlength{\parskip}{0pt plus 0.3pt}%
  \let\item\@idxitem
}{%
  \ifkorrekturansicht\clearpage\fi
}
\makeatother

\IfFileExists{\jobname-pw.ind}{\input{\jobname-pw.ind}}{}

% Quellenangabe nur in der Leseansicht
\ifkorrekturansicht\else
% Fallback-Definitionen, falls die .tex-Datei \titel etc. nicht gesetzt hat
\providecommand{\titel}{}
\providecommand{\editorInnen}{}
\providecommand{\dateiname}{\jobname}

\vspace{3cm}

\vfill

\footnotesize
\textsc{Quelle}: \titel. Herausgegeben von {\editorInnen}. In: \emph{Arthur Schnitzler: Briefwechsel mit Autorinnen und Autoren}.
 Digitale Edition, https://schnitzler-briefe.acdh.oeaw.ac.at/{\dateiname}.html (Stand \today)
\fi

\end{document}


