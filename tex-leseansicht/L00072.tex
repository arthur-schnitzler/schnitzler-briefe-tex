%% latex-leseansicht-vorspann.tex
%% Vorspann für die Leseansicht.
%% Lädt die gemeinsame Datei latex-vorspann.tex mit nicht gesetztem Schalter.

\newif\ifkorrekturansicht
\korrekturansichtfalse

\input{../tex-inputs/latex-vorspann}

\begin{center}
            \textcolor{red}{ENTWURF. ENTZIFFERUNG NOCH NICHT KORREKTURGELESEN}
                      \end{center}
            
               \section[Hugo von Hofmannsthal an Arthur Schnitzler, {[}17. 2. 1892{]}]{ Hugo von Hofmannsthal an Arthur Schnitzler, {[}17. 2. 1892{]}}\nopagebreak\mylabel{v}\rehead{ }\begin{ledgroupsized}[t]{13cm}\normalsize\beginnumbering\briefempfaengerindex{Schnitzler, Arthur@\textsc{Schnitzler, Arthur}!zzzHofmannsthal, Hugo von@\emph{von Hugo von Hofmannsthal}!1892-02-171@{{[}17. 2. 1892{]}}|(be} \toendnotes[C]{\smallbreak\pagebreak[2]} \Standort{CUL, Schnitzler, B 43.}
\physDesc{Brief, 1 Blatt, 1 Seite
\newline{}Handschrift: schwarze Tinte, deutsche Kurrent
\newline{}Schnitzler: mit Bleistift datiert: »17/2 92« \newline{}Ordnung: mit Bleistift von unbekannter Hand nummeriert:
                                        »17« }\buchAbdrucke{\weitereDrucke{1) Hugo von Hofmannsthal, Arthur Schnitzler: \emph{Briefwechsel}. Hg. Therese Nickl und Heinrich Schnitzler. Frankfurt am Main: \emph{S. Fischer} 1964, S. 16.} \weitereDrucke{2) Hermann Bahr, Arthur Schnitzler: \emph{Briefwechsel, Aufzeichnungen, Dokumente
                                (1891–1931)}. Hg. Kurt Ifkovits und Martin Anton Müller. Göttingen: \emph{Wallstein} 2018, S. 21.} }\toendnotes[C]{\smallbreak}\pstart
           \noindent{}\centering{}{\pb}Thatsachen: \pend
           \pstart
           \noindent{}1.) Bitte adreſſieren Sie\pwindex{Schnitzler, Arthur 15.05.1862 – 21.10.1931@\textsc{Schnitzler, Arthur} (15.05.1862 – 21.10.1931), \emph{Schriftsteller, Mediziner}|pwv} den beiliegenden
                    Wiſch an Herrn Lothar\pwindex{Lothar, Rudolf 23.2.1865 – 2.10.1943@\textsc{Lothar, Rudolf} (23.2.1865 – 2.10.1943), \emph{Schriftsteller, Journalist, Theaterdirektor}|pw} und ſchicken Sie ihn
                    weg.\pend
           \pstart
           2.) Maeterlinck\pwindex{Maeterlinck, Maurice 29.08.1862 – 06.05.1949@\textsc{Maeterlinck, Maurice} (29.08.1862 – 06.05.1949), \emph{Schriftsteller}|pw} hat mich zur Überſetzung
                    freundlichſt autoriſiert.\pend
           \pstart
           3.) Die Empfehlung an die Palmay\pwindex{Pálmay, Ilka 1859-09-21 – 1944-02-17@\textsc{Pálmay, Ilka} (1859-09-21 – 1944-02-17), \emph{Schriftstellerin, Schauspielerin, Sängerin}|pw} habe ich
                    verlangt und werde ſie Bahr\pwindex{Bahr, Hermann 19.07.1863 – 15.01.1934@\textsc{Bahr, Hermann} (19.07.1863 – 15.01.1934), \emph{Schriftsteller, Kritiker}|pw} nächſtens
                    ſchicken.\pend
           \pstart
           4.) Vielleicht könnte Kafka\pwindex{Kafka, Eduard Michael 11.03.1869 – 06.08.1893@\textsc{Kafka, Eduard Michael} (11.03.1869 – 06.08.1893), \emph{Redakteur}|pw} die erſten
                    Vierteljahrsbeiträge raſch einkaſſieren und uns gegen Garantie durch perſönliche
                    Unterſchrift leihen. Das wären doch vielleicht 200 fl.\pend
           \pstart
           5.) Suchen Sie Bauer\pwindex{Bauer, Arnold um 1840 – 1893-07-19@\textsc{Bauer, Arnold} (um 1840 – 1893-07-19), \emph{Mediziner, Herausgeber, Buchhändler}|pw} gegenüber uns wichtig
                    und ernſt zu machen und trachten Sie, \introOben{}daß\introOben{} das erſte Heft\orgindex{Wiener Literatur-Zeitung@Wiener Literatur-Zeitung|pwv} möglichſt bald erſcheint. An
                    die Premièren: Fulda\pwindex{Fulda, Ludwig 15.07.1862 – 30.03.1939@\textsc{Fulda, Ludwig} (15.07.1862 – 30.03.1939), \emph{Schriftsteller, Übersetzer}|pw} »Sclavin\pwindex{Fulda, Ludwig 15.07.1862 – 30.03.1939@\textsc{Fulda, Ludwig} (15.07.1862 – 30.03.1939), \emph{Schriftsteller, Übersetzer}!Sklavin. Schauspiel in vier Aufzuegen1891@\strich\emph{Die Sklavin. Schauspiel in vier Aufzügen} {[}1891{]}|pw}«, \textsc{Griselidis}\pwindex{\textcolor{red}{\textsuperscript{XXXX1 indx}}!Griselidis. Oper in drei Akten und einem Prolog1891@\strich\emph{Grisélidis. Oper in drei Akten und einem Prolog} {[}1891{]}|pw}\pwindex{\textcolor{red}{\textsuperscript{XXXX1 indx}}!Griselidis. Oper in drei Akten und einem Prolog1891@\strich\emph{Grisélidis. Oper in drei Akten und einem Prolog} {[}1891{]}|pw} und Schleſinger\pwindex{Schlesinger, Sigmund 15.06.1832 – 07.03.1918@\textsc{Schlesinger, Sigmund} (15.06.1832 – 07.03.1918), \emph{Schriftsteller}|pw} »\textsc{Derby}\pwindex{Schlesinger, Sigmund 15.06.1832 – 07.03.1918@\textsc{Schlesinger, Sigmund} (15.06.1832 – 07.03.1918), \emph{Schriftsteller}!Derby1889@\strich\emph{Derby} {[}1889{]}|pw}« läſst ſich künſtleriſch und ſocial unendlich viel anhängen.\pend
           \pstart \spacefill\mbox{Loris.}\pend{}\endnumbering\briefempfaengerindex{Schnitzler, Arthur@\textsc{Schnitzler, Arthur}!zzzHofmannsthal, Hugo von@\emph{von Hugo von Hofmannsthal}!1892-02-171@{{[}17. 2. 1892{]}}|)be}\mylabel{h}\end{ledgroupsized}  \newcommand{\dateiname}{L00072}\newcommand{\titel}{Hugo von Hofmannsthal an Arthur Schnitzler, [17. 2. 1892]}\newcommand{\editorInnen}{ Martin Anton Müller und Gerd-Hermann Susen}%% latex-leseansicht-abspann.tex
%% Abspann für die Leseansicht.
%% Der Schalter \ifkorrekturansicht ist bereits durch den Vorspann gesetzt.

%% latex-abspann.tex
%% Gemeinsamer Abspann für Korrekturansicht und Leseansicht.
%% Setzt den Schalter \ifkorrekturansicht voraus (gesetzt in den
%% einbindenden Dateien latex-korrekturansicht-abspann.tex bzw.
%% latex-leseansicht-abspann.tex).
%% ---------------------------------------------------------------

\normalsize

% Das esempio-Environment wird nur in der Leseansicht benötigt
\ifkorrekturansicht\else
\newenvironment{esempio}[3]%
{
    \vspace{1.5ex}
    \rlap{\underline{#1}}
    \par
    \setlength{\parindent}{0cm}
    \nopagebreak
    \leftskip=#2cm
    \rightskip=#3cm
}
{
    \par
}
\fi

\doendnotes{C}
\bigskip
\vfill

\clearpage

\footnotesize

\ifkorrekturansicht
  \lohead{\textsc{register}}
\fi

% theindex-Environment neu definieren ohne reledmac
\makeatletter
\renewenvironment{theindex}{%
  \ifkorrekturansicht
    \section*{\indexname}%
  \else
    \subsubsection*{Index der erwähnten Entitäten}%
  \fi
  \setlength{\parindent}{0pt}%
  \setlength{\parskip}{0pt plus 0.3pt}%
  \let\item\@idxitem
}{%
  \ifkorrekturansicht\clearpage\fi
}
\makeatother

\IfFileExists{\jobname-pw.ind}{\input{\jobname-pw.ind}}{}

% Quellenangabe nur in der Leseansicht
\ifkorrekturansicht\else
% Fallback-Definitionen, falls die .tex-Datei \titel etc. nicht gesetzt hat
\providecommand{\titel}{}
\providecommand{\editorInnen}{}
\providecommand{\dateiname}{\jobname}

\vspace{3cm}

\vfill

\footnotesize
\textsc{Quelle}: \titel. Herausgegeben von {\editorInnen}. In: \emph{Arthur Schnitzler: Briefwechsel mit Autorinnen und Autoren}.
 Digitale Edition, https://schnitzler-briefe.acdh.oeaw.ac.at/{\dateiname}.html (Stand \today)
\fi

\end{document}


      