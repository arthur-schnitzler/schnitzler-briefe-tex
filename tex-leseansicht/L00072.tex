%% latex-korrekturansicht-vorspann.tex
%% Vorspann für die Korrekturansicht.
%% Lädt die gemeinsame Datei latex-vorspann.tex mit gesetztem Schalter.

\newif\ifkorrekturansicht
\korrekturansichttrue

\input{../tex-inputs/latex-vorspann}


\section[Hugo von Hofmannsthal an Arthur Schnitzler, {[}17. 2. 1892{]}]{L00072 Hugo von Hofmannsthal an Arthur Schnitzler, {[}17. 2. 1892{]}}
\nopagebreak\mylabel{L00072v}
\rehead{ }\normalsize\beginnumbering\briefempfaengerindex{Schnitzler, Arthur@\textsc{Schnitzler, Arthur}!zzzHofmannsthal, Hugo von@\emph{von Hugo von Hofmannsthal}!1892-02-171@{{[}17. 2. 1892{]}}|(be}
\toendnotes[C]{\smallbreak\pagebreak[2]}\Standort{CUL, Schnitzler, B 43.}
\physDesc{Brief, 1 Blatt, 1 Seite, 668 Zeichen
\newline{}Handschrift: schwarze Tinte, deutsche Kurrent
\newline{}Schnitzler: mit Bleistift datiert: »17/2 92« 
\newline{}Ordnung: mit Bleistift von unbekannter Hand nummeriert:
                                    »17« }
\buchAbdrucke{\weitereDrucke{1) Hugo von Hofmannsthal, Arthur Schnitzler: \emph{Briefwechsel}. Frankfurt am Main: \emph{S. Fischer} 1964, S. 16.} \weitereDrucke{2) Hermann Bahr, Arthur Schnitzler: \emph{Briefwechsel, Aufzeichnungen, Dokumente (1891–1931)}. Göttingen: \emph{Wallstein} 2018, S. 21.} }\toendnotes[C]{\smallbreak}
\pstart
           \noindent{}\centering{}{\pb}Thatsachen: \pend
           
\pstart
           1.) Bitte adreſſieren Sie den
               beiliegenden Wiſch an Herrn Lothar\pwindex{Lothar, Rudolf 23.2.1865 – 2.10.1943@\textsc{Lothar, Rudolf} (23.2.1865 – 2.10.1943), \emph{Schriftsteller/Schriftstellerin, Journalist/Journalistin, Theaterdirektor/Theaterdirektorin}|pw} und ſchicken
               Sie ihn weg.\pend
           
\pstart
           2.) Maeterlinck\pwindex{Maeterlinck, Maurice 29.08.1862 – 06.05.1949@\textsc{Maeterlinck, Maurice} (29.08.1862 – 06.05.1949), \emph{Schriftsteller/Schriftstellerin}|pw} hat mich zur Überſetzung
               freundlichſt autoriſiert.\pend
           
\pstart
           3.) Die Empfehlung an die Palmay\pwindex{Pálmay, Ilka 1859-09-21 – 1944-02-17@\textsc{Pálmay, Ilka} (1859-09-21 – 1944-02-17), \emph{Schriftsteller/Schriftstellerin, Schauspieler/Schauspielerin, Sänger/Sängerin}|pw} habe ich
               verlangt und werde ſie Bahr\pwindex{Bahr, Hermann 19.07.1863 – 15.01.1934@\textsc{Bahr, Hermann} (19.07.1863 – 15.01.1934), \emph{Schriftsteller/Schriftstellerin, Kritiker/Kritikerin}|pw} nächſtens
               ſchicken.\pend
           
\pstart
           4.) Vielleicht könnte Kafka\pwindex{Kafka, Eduard Michael 11.03.1869 – 06.08.1893@\textsc{Kafka, Eduard Michael} (11.03.1869 – 06.08.1893), \emph{Redakteur/Redakteurin}|pw} die erſten
               Vierteljahrsbeiträge raſch einkaſſieren und uns gegen Garantie durch perſönliche
               Unterſchrift leihen. Das wären doch vielleicht 200 fl.\pend
           
\pstart
           5.) Suchen Sie Bauer\pwindex{Bauer, Arnold um 1840 – 1893-07-19@\textsc{Bauer, Arnold} (um 1840 – 1893-07-19), \emph{Mediziner/Medizinerin, Herausgeber/Herausgeberin, Buchhändler/Buchhändlerin}|pw} gegenüber uns wichtig und
               ernſt zu machen und trachten Sie, \introOben{}daß\introOben{} das erſte Heft\orgindex{Wiener Literatur-Zeitung@Wiener Literatur-Zeitung|pwv} möglichſt bald erſcheint.
               An die Premièren: Fulda\pwindex{Fulda, Ludwig 15.07.1862 – 30.03.1939@\textsc{Fulda, Ludwig} (15.07.1862 – 30.03.1939), \emph{Schriftsteller/Schriftstellerin, Übersetzer/Übersetzerin}|pw} »Sclavin\pwindex{Sklavin. Schauspiel in vier Aufzuegen@\emph{Die Sklavin. Schauspiel in vier Aufzügen}|pw}«, \textsc{Griselidis}\pwindex{Griselidis. Oper in drei Akten und einem Prolog@\emph{Grisélidis. Oper in drei Akten und einem Prolog}|pw} und Schleſinger\pwindex{Schlesinger, Sigmund 15.06.1832 – 07.03.1918@\textsc{Schlesinger, Sigmund} (15.06.1832 – 07.03.1918), \emph{Schriftsteller/Schriftstellerin}|pw} »\textsc{Derby}\pwindex{Derby@\emph{Derby}|pw}« läſst ſich künſtleriſch und ſocial unendlich viel anhängen.\pend
           \pstart \spacefill\mbox{Loris.}\pend{}\selectlanguage{ngerman}\endnumbering\briefempfaengerindex{Schnitzler, Arthur@\textsc{Schnitzler, Arthur}!zzzHofmannsthal, Hugo von@\emph{von Hugo von Hofmannsthal}!1892-02-171@{{[}17. 2. 1892{]}}|)be}\mylabel{L00072h}  \normalsize

\doendnotes{C}
\bigskip
\vfill

\clearpage

\footnotesize

\lohead{\textsc{register}}

% Definiere theindex-Environment komplett neu ohne reledmac
\makeatletter
\renewenvironment{theindex}{%
  \section*{\indexname}%
  \setlength{\parindent}{0pt}%
  \setlength{\parskip}{0pt plus 0.3pt}%
  \let\item\@idxitem
}{%
  \clearpage
}
\makeatother

\IfFileExists{\jobname-pw.ind}{\input{\jobname-pw.ind}}{}

\end{document}

      