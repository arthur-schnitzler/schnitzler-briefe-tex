%% latex-korrekturansicht-vorspann.tex
%% Vorspann für die Korrekturansicht.
%% Lädt die gemeinsame Datei latex-vorspann.tex mit gesetztem Schalter.

\newif\ifkorrekturansicht
\korrekturansichttrue

\input{../tex-inputs/latex-vorspann}


\section[Richard Dehmel, Gerhart Hauptmann und Jakob Wassermann an Arthur Schnitzler, 3. 12. 1909]{L01892 Richard Dehmel, Gerhart Hauptmann und Jakob Wassermann an Arthur
               Schnitzler, 3. 12. 1909}
\nopagebreak\mylabel{L01892v}
\rehead{ }\normalsize\beginnumbering\briefempfaengerindex{Schnitzler, Arthur@\textsc{Schnitzler, Arthur}!zzzWassermann, Jakob@\emph{von Jakob Wassermann}!1909-12-031@{3. 12. 1909}|(be}\briefempfaengerindex{Schnitzler, Arthur@\textsc{Schnitzler, Arthur}!zzzHauptmann, Gerhart@\emph{von Gerhart Hauptmann}!1909-12-031@{3. 12. 1909}|(be}\briefempfaengerindex{Schnitzler, Arthur@\textsc{Schnitzler, Arthur}!zzzDehmel, Richard@\emph{von Richard Dehmel}!1909-12-031@{3. 12. 1909}|(be}
\toendnotes[C]{\smallbreak\pagebreak[2]}\Standort{CUL, Schnitzler, B 26.}
\physDesc{Brief, 1 Blatt, 1 Seite, 2882 Zeichen
\newline{}Schreibmaschine
\newline{}Beilage: gedruckte Grußadresse an Fischer\pwindex{Fischer, Samuel 24.12.1859 – 15.10.1934@\textsc{Fischer, Samuel} (24.12.1859 – 15.10.1934), \emph{Verleger/Verlegerin}|pw}, 1 Blatt, 2 Seiten 
\newline{}Schnitzler: mit rotem Buntstift eine Unterstreichung 
\newline{}Ordnung: mit Bleistift von unbekannter Hand in der linken oberen Ecke
                                 Vermerk: »\textsc{D.}« }
\pstart
           \raggedleft{}{\pb}Berlin\oindex{Berlin@\textbf{Berlin}, \emph{P.PPLC}|pw}, 3. Dezember 09\pend
           
\pstart
           Herrn Dr. Arthur Schnitzler\pend
           
\pstart
           \uline{Wien}\oindex{Wien@\textbf{Wien}, \emph{A.ADM2}|pw}\pend
           
\pstart{}Sehr geehrter Herr,\pend\vspace{0.5em}
\pstart
           Am 24. Dezember dieses Jahres begeht unser Verleger, Herr Fischer\pwindex{Fischer, Samuel 24.12.1859 – 15.10.1934@\textsc{Fischer, Samuel} (24.12.1859 – 15.10.1934), \emph{Verleger/Verlegerin}|pw}, seinen 50. Geburtstag. Es wird zu
               diesem Tage eine Adresse an ihn geplant, die zu unterzeichnen wir auch Sie, geehrter
               Herr, bitten. Den Wortlaut der Adresse überreichen wir hiermit; desgleichen in
               hergerichtetem Couvert das Kärtchen, auf das der Name zu schreiben ist und das der
               Adresse beigefügt werden soll. Die Adresse wird von Herrn E. R. Weiss\pwindex{Weiss, Emil Rudolf 12.10.1875 – 07.11.1942@\textsc{Weiß, Emil Rudolf} (12.10.1875 – 07.11.1942), \emph{Künstler/Künstlerin, Typograf/Typografin}|pw} entworfen und unter seiner Leitung ausgeführt
               werden; die Kosten sollen von den Unterzeichner der Adresse aufgebracht werden und
               werden nur wenige Mark für jeden Unterzeichner betragen.\pend
           
\pstart
           Mit der Bitte um möglichst schleunige Uebersendung Ihrer Unterschrift\pend
           
\pstart
           hochachtungsvoll{\\[\baselineskip]}das Comité{\\[\baselineskip]}i. A.{\\[\baselineskip]}\spacefill\mbox{Richard Dehmel\hspace*{1.5em}Gerhart Hauptmann\hspace*{1.5em}Jakob Wassermann}\pend
           \leftskip=0em{}\selectlanguage{ngerman}\vspace{1em}
\pstart{}{\pb}Lieber Herr Fischer\pwindex{Fischer, Samuel 24.12.1859 – 15.10.1934@\textsc{Fischer, Samuel} (24.12.1859 – 15.10.1934), \emph{Verleger/Verlegerin}|pw},\pend\vspace{0.5em}
\pstart
           Ihr fünfzigster Geburtstag scheint uns mehr als ein bloßes privates Fest zu bedeuten;
               und Sie selbst werden beim Rückblick auf Ihr Leben Ihre öffentliche Tätigkeit mit
               besonderer Ergriffenheit betrachten. Sie haben in einer Zeit, wo man in Deutschland\oindex{Deutschland@\textbf{Deutschland}, \emph{A.PCLI}|pw} von mitlebender Literatur wenig wissen
               wollte, vielem Neuen, Interessanten und Bedeutenden, das jetzt gefestigt und bewährt
               ist, anfänglich aber noch in der Gärung lag und zum Streit herausforderte, mutig und
               zuversichtlich, als gerechter Mittler, die Öffentlichkeit erschlossen. Charakter und
               Organisation, nicht der Zufall, haben eine Gemeinde von Gleichstrebenden um Sie
               gebildet. Wir kennen die Schwierigkeiten Ihrer Aufgabe. Denn Ihre Schöpfung, die
               einen ganzen Komplex von Tätigen der verschiedensten Kategorien vereinigt, ist auf
               dem schwierigen Grenzgebiet aufgeführt, wo die Künste und Wissenschaften mit den
               ökonomischen Mächten zusammenstoßen. Sie haben erfahren, daß das Geistige keine
               isolierte Macht ist, kein Schrankenloses und Unbedingtes, sondern daß es auf allen
               Seiten von den wirtschaftlichen Gewalten bedroht, gehemmt und gebunden wird. Es war
               Ihre Aufgabe, Ihre Natur, Ihr Wille, {\pb}diese Gebundenheit in einer edlen Weise wieder zu lösen. In einem so vielfältigen
               Getriebe, in so verantwortungsreichen Beziehungen abhängig von der Mode, von der
               Gunst des Publikums, in der Enge des Wettkampfs, mitemporgehoben von der Energie
               eines allgemeinen nationalen Aufschwungs, der die sittlichen Kräfte nicht selten zu
               lähmen drohte, haben Sie Ihre Sache, welche die Sache der Besten war und ist, auf ein
               nicht mehr unbestrittenes Postament und Ihren Namen in die Reihe der geehrten Namen
               gestellt. Sie sind, in unbefangener Menschlichkeit, immer mehr an Ihrem Werke
               gewachsen; Sie repräsentieren es; wir begrüßen dieses Beispiel der begeistert
               besonnen Hingabe, der Sachlichkeit und des wahrhaften Ernstes und fühlen uns herzlich
               verpflichtet, Ihnen Dank zu sagen und für den weiteren Weg Glück und Vollbringung zu
               wünschen.\pend
           
\pstart
           Im Auftrage:{\\[\baselineskip]}\spacefill\mbox{Richard Dehmel}{\\[\baselineskip]}\spacefill\mbox{Gerhart Hauptmann}{\\[\baselineskip]}\spacefill\mbox{Jakob Wassermann}\pend
           \leftskip=0em{}\selectlanguage{ngerman}\endnumbering\briefempfaengerindex{Schnitzler, Arthur@\textsc{Schnitzler, Arthur}!zzzWassermann, Jakob@\emph{von Jakob Wassermann}!1909-12-031@{3. 12. 1909}|)be}\briefempfaengerindex{Schnitzler, Arthur@\textsc{Schnitzler, Arthur}!zzzHauptmann, Gerhart@\emph{von Gerhart Hauptmann}!1909-12-031@{3. 12. 1909}|)be}\briefempfaengerindex{Schnitzler, Arthur@\textsc{Schnitzler, Arthur}!zzzDehmel, Richard@\emph{von Richard Dehmel}!1909-12-031@{3. 12. 1909}|)be}\mylabel{L01892h}  \normalsize

\doendnotes{C}
\bigskip
\vfill

\clearpage

\footnotesize

\lohead{\textsc{register}}

% Definiere theindex-Environment komplett neu ohne reledmac
\makeatletter
\renewenvironment{theindex}{%
  \section*{\indexname}%
  \setlength{\parindent}{0pt}%
  \setlength{\parskip}{0pt plus 0.3pt}%
  \let\item\@idxitem
}{%
  \clearpage
}
\makeatother

\IfFileExists{\jobname-pw.ind}{\input{\jobname-pw.ind}}{}

\end{document}

      