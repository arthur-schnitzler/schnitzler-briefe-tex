%% latex-leseansicht-vorspann.tex
%% Vorspann für die Leseansicht.
%% Lädt die gemeinsame Datei latex-vorspann.tex mit nicht gesetztem Schalter.

\newif\ifkorrekturansicht
\korrekturansichtfalse

\input{../tex-inputs/latex-vorspann}


\section[Elsa Plessner an Arthur Schnitzler, 4. 1. 1899]{L03719 Elsa Plessner an Arthur Schnitzler, 4. 1. 1899}
\nopagebreak\mylabel{L03719v}
\rehead{ }\normalsize\beginnumbering\briefempfaengerindex{Schnitzler, Arthur@\textsc{Schnitzler, Arthur}!zzzPlessner, Elsa@\emph{von Elsa Plessner}!1899-01-042@{4. 1. 1899}|(be}
\toendnotes[C]{\smallbreak\pagebreak[2]}
\correspDesc{Versand  durch Elsa Plessner am 4. 1. 1899 in Wien
\newline{}Erhalt  durch Arthur Schnitzler im Zeitraum [4. 1. 1899 – 7. 1. 1899?] in Wien}\toendnotes[C]{\smallbreak}
\Standort{DLA, A:Schnitzler, HS.1985.1.419.}
\physDesc{Brief, 1 Blatt, 3 Seiten, 748 Zeichen (Briefpapier mit Blumenmotiv (Schneeglöckchen) auf
                                 S. 1)
\newline{}Handschrift: schwarze Tinte, lateinische Kurrent}\toendnotes[C]{\smallbreak}
\pstart
           \raggedleft{}{\pb}\uline{Wien} I. Spiegelgasse 2\oindex{Wien@\textbf{Wien}!I., Innere Stadt@\textbf{I., Innere Stadt}!Spiegelgasse 2@\textbf{Spiegelgasse 2}, \emph{Wohngebäude}|pw}.\pend
           
\pstart
           \raggedleft{}den 4. I. 9\uuline{\edtext{9}{\Cendnote{sechsfach unterstrichen}}}.\pend
           
\pstart{}Verehrter Herr Doctor!\pend\vspace{0.5em}
\pstart
           Herzlichen Dank für Ihren lie\substVorne{}\textsuperscript{f}\substDazwischen{}b\substHinten{}en \label{K_L03719-1v}\edtext{Brief}{\lemma{\textnormal{\emph{Brief}}}\Cendnote{\textnormal{nicht überliefert}}}\label{K_L03719-1} aus
               dem vorigen Jahr. – D. h. Sie sind noch nicht an die neue 9 gewöhnt! Ihren
               freundlichen Rath werde ich sehr gern befolgen – m. w. – machen wir! Die
               Arbeit\pwindex{Plessner, Elsa 22.\,8.\,1875 Wien – 7.\,5.\,1932 Alicante@\textsc{Plessner, Elsa} (22.\,8.\,1875 Wien – 7.\,5.\,1932 Alicante), \emph{Schriftstellerin}!neue Lehrer. Novelle@\strich\emph{Der neue Lehrer. Novelle}|pwv}, jetzt \label{K_L03719-2v}\edtext{\uline{fast ein Jahr} alt}{\lemma{\textnormal{\emph{fast ein Jahr alt}}}\Cendnote{\textnormal{Im Brief 
                  vom XXXX Auszeichnungsfehler: Dokument L03728 nicht gefunden nennt sie »Juli 98« als Entstehungszeit von \emph{Der neue Lehrer}\pwindex{Plessner, Elsa 22.\,8.\,1875 Wien – 7.\,5.\,1932 Alicante@\textsc{Plessner, Elsa} (22.\,8.\,1875 Wien – 7.\,5.\,1932 Alicante), \emph{Schriftstellerin}!neue Lehrer. Novelle@\strich\emph{Der neue Lehrer. Novelle}|pwk}.}}}\label{K_L03719-2}, ist mir {\pb}doch ein
               bisschen aus Herz gewachsen!! –\pend
           
\pstart
           Momentan nichts anderes vor – ! Bin sehr froh, dass \label{K_L03719-3v}\edtext{noch nicht gedruckt}{\lemma{\textnormal{\emph{noch nicht gedruckt}}}\Cendnote{\textnormal{Plessner\pwindex{Plessner, Elsa 22.\,8.\,1875 Wien – 7.\,5.\,1932 Alicante@\textsc{Plessner, Elsa} (22.\,8.\,1875 Wien – 7.\,5.\,1932 Alicante), \emph{Schriftstellerin}|pwk} hatte \emph{Der neue Lehrer}\pwindex{Plessner, Elsa 22.\,8.\,1875 Wien – 7.\,5.\,1932 Alicante@\textsc{Plessner, Elsa} (22.\,8.\,1875 Wien – 7.\,5.\,1932 Alicante), \emph{Schriftstellerin}!neue Lehrer. Novelle@\strich\emph{Der neue Lehrer. Novelle}|pwk} bei der Zeitschrift \emph{Die
                     Wage}\pwindex{Wage. Eine Wiener Wochenschrift@\emph{Die Wage. Eine Wiener Wochenschrift}|pwk} eingereicht, aber wieder zurückgezogen, vgl. XXXX Auszeichnungsfehler: Dokument L03718 nicht gefunden.}}}\label{K_L03719-3}!\pend
           
\pstart
           Köstlich ist es, wenn Sie als Greis posieren! Die \label{K_L03719-5v}\edtext{zehn oder elf Jahre
               Altersunterschied}{\lemma{\textnormal{\emph{zehn … Altersunterschied}}}\Cendnote{\textnormal{Schnitzler kam 1862 auf die Welt,
                  Plessner\pwindex{Plessner, Elsa 22.\,8.\,1875 Wien – 7.\,5.\,1932 Alicante@\textsc{Plessner, Elsa} (22.\,8.\,1875 Wien – 7.\,5.\,1932 Alicante), \emph{Schriftstellerin}|pwk}{ }1875, es
                  lagen also 13 Jahre Altersunterschied zwischen ihnen.}}}\label{K_L03719-5} haben doch noch kein solches Gewicht!! Oder haben sie noch immer
               Einkehr-Stimmung – {\pb}– immer Sylvester-\begin{otherlanguage}{french}\label{K_L03719-4v}\edtext{lendemain}{\lemma{\textnormal{\emph{lendemain}}}\Cendnote{\textnormal{französisch:
                  Folgetag}}}\label{K_L03719-4}\end{otherlanguage}? – (um nicht zu sagen Kater?). Dann wünsche gute Besserung und
               den pikanten Hering in irgend welcher erfrischender Verkleidung!! –\pend
           
\pstart
           Herzlich grüßt{\\[\baselineskip]}\spacefill\mbox{Elsa Plessner.}\pend
           \leftskip=0em{}\selectlanguage{ngerman}\endnumbering\briefempfaengerindex{Schnitzler, Arthur@\textsc{Schnitzler, Arthur}!zzzPlessner, Elsa@\emph{von Elsa Plessner}!1899-01-042@{4. 1. 1899}|)be}\mylabel{L03719h}  \newcommand{\dateiname}{L03719}\newcommand{\titel}{Elsa Plessner an Arthur Schnitzler, 4. 1. 1899}\newcommand{\editorInnen}{Selma Jahnke und Martin Anton Müller}%% latex-leseansicht-abspann.tex
%% Abspann für die Leseansicht.
%% Der Schalter \ifkorrekturansicht ist bereits durch den Vorspann gesetzt.

%% latex-abspann.tex
%% Gemeinsamer Abspann für Korrekturansicht und Leseansicht.
%% Setzt den Schalter \ifkorrekturansicht voraus (gesetzt in den
%% einbindenden Dateien latex-korrekturansicht-abspann.tex bzw.
%% latex-leseansicht-abspann.tex).
%% ---------------------------------------------------------------

\normalsize

% Das esempio-Environment wird nur in der Leseansicht benötigt
\ifkorrekturansicht\else
\newenvironment{esempio}[3]%
{
    \vspace{1.5ex}
    \rlap{\underline{#1}}
    \par
    \setlength{\parindent}{0cm}
    \nopagebreak
    \leftskip=#2cm
    \rightskip=#3cm
}
{
    \par
}
\fi

\doendnotes{C}
\bigskip
\vfill

\clearpage

\footnotesize

\ifkorrekturansicht
  \lohead{\textsc{register}}
\fi

% theindex-Environment neu definieren ohne reledmac
\makeatletter
\renewenvironment{theindex}{%
  \ifkorrekturansicht
    \section*{\indexname}%
  \else
    \subsubsection*{Index der erwähnten Entitäten}%
  \fi
  \setlength{\parindent}{0pt}%
  \setlength{\parskip}{0pt plus 0.3pt}%
  \let\item\@idxitem
}{%
  \ifkorrekturansicht\clearpage\fi
}
\makeatother

\IfFileExists{\jobname-pw.ind}{\input{\jobname-pw.ind}}{}

% Quellenangabe nur in der Leseansicht
\ifkorrekturansicht\else
% Fallback-Definitionen, falls die .tex-Datei \titel etc. nicht gesetzt hat
\providecommand{\titel}{}
\providecommand{\editorInnen}{}
\providecommand{\dateiname}{\jobname}

\vspace{3cm}

\vfill

\footnotesize
\textsc{Quelle}: \titel. Herausgegeben von {\editorInnen}. In: \emph{Arthur Schnitzler: Briefwechsel mit Autorinnen und Autoren}.
 Digitale Edition, https://schnitzler-briefe.acdh.oeaw.ac.at/{\dateiname}.html (Stand \today)
\fi

\end{document}


