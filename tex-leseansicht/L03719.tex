%% latex-korrekturansicht-vorspann.tex
%% Vorspann für die Korrekturansicht.
%% Lädt die gemeinsame Datei latex-vorspann.tex mit gesetztem Schalter.

\newif\ifkorrekturansicht
\korrekturansichttrue

\input{../tex-inputs/latex-vorspann}


\section[Elsa Plessner an Arthur Schnitzler, 4. 1. 1899]{L03719 Elsa Plessner an Arthur Schnitzler, 4. 1. 1899}
\nopagebreak\mylabel{L03719v}
\rehead{ }\normalsize\beginnumbering\briefempfaengerindex{Schnitzler, Arthur@\textsc{Schnitzler, Arthur}!zzzPlessner, Elsa@\emph{von Elsa Plessner}!1899-01-042@{4. 1. 1899}|(be}
\toendnotes[C]{\smallbreak\pagebreak[2]}\Standort{DLA, A:Schnitzler, HS.1985.1.419.}
\physDesc{Brief, 1 Blatt, 3 Seiten, 744 Zeichen (Briefpapier mit Blumenmotiv (Schneeglöckchen) auf
                                 S. 1)
\newline{}Handschrift: , lateinische Kurrent}\toendnotes[C]{\smallbreak}
\pstart
           \raggedleft{}{\pb}Wien I. Spiegelgasse 2\oindex{Spiegelgasse 2@\textbf{Spiegelgasse 2}, \emph{Wohngebäude (K.WHS)}|pw}.\pend
           
\pstart
           \raggedleft{}den 4. I. 9\uuline{\edtext{9}{\Cendnote{sechsfach unterstrichen}}}.\pend
           
\pstart{}Verehrter Herr Doctor!\pend\vspace{0.5em}
\pstart
           Herzlichen Dank für Ihren lieben \label{K_L03719-1v}\edtext{Brief}{\lemma{\textnormal{\emph{Brief}}}\Cendnote{\textnormal{nicht überliefert}}}\label{K_L03719-1} aus
               dem vorigen Jahr. – D. h. Sie sind noch nicht an die neue 9 gewöhnt! Ihren
               freundlichen Rath werde ich sehr gern befolgen – m. w. – machen wir! Die
                  Arbeit\pwindex{neue Lehrer. Novelle@\emph{Der neue Lehrer. Novelle}|pwv}, jetzt \uline{fast ein Jahr} alt, ist mir {\pb}doch ein
               bisschen aus Herz gewachsen!!\pend
           
\pstart
           Momentan nichts anderes vor – ! Bin sehr froh, dass \label{K_L03719-2v}\edtext{noch nicht gedruckt}{\lemma{\textnormal{\emph{noch nicht gedruckt}}}\Cendnote{\textnormal{Elsa Plessner\pwindex{Plessner, Elsa 22.08.1875 – 01.05.1932@\textsc{Plessner, Elsa} (22.08.1875 – 01.05.1932), \emph{Schriftsteller/Schriftstellerin}|pwk} hatte
                  einen längere Novelle bei der Zeitschrift \emph{Die
                     Wage}\pwindex{Wage. Eine Wiener Wochenschrift@\emph{Die Wage. Eine Wiener Wochenschrift}|pwk} eingereicht, aber wieder zurückgezogen, weil sie den geforderten
                  Eingriffen in den Text nicht zustimmte. Vermutlich handelte es sich um die Novelle
                     \emph{Der neue Lehrer}\pwindex{neue Lehrer. Novelle@\emph{Der neue Lehrer. Novelle}|pwk}.}}}\label{K_L03719-2}!\pend
           
\pstart
           Köstlich ist es, wenn Sie als Greis posieren! Die zehn oder elf Jahre
               Altersunterschied haben doch noch kein solches Gewicht!! Oder haben sie noch immer
               Einkehr-Stimmung – {\pb}immer Sylvester-\begin{otherlanguage}{french}\label{K_L03719-3v}\edtext{lendemain}{\lemma{\textnormal{\emph{lendemain}}}\Cendnote{\textnormal{französisch:
                  Folgetag}}}\label{K_L03719-3}\end{otherlanguage}? – (um nicht zu sagen Kater?). Dann wünsche gute Besserung und
               den pikanten Hering in irgend welcher erfrischender Verkleidung!!\pend
           
\pstart
           Herzlich grüßt{\\[\baselineskip]}\spacefill\mbox{Elsa Plessner}\pend
           \leftskip=0em{}\selectlanguage{ngerman}\endnumbering\briefempfaengerindex{Schnitzler, Arthur@\textsc{Schnitzler, Arthur}!zzzPlessner, Elsa@\emph{von Elsa Plessner}!1899-01-042@{4. 1. 1899}|)be}\mylabel{L03719h}
\begin{anhang}
\end{anhang}\normalsize

\doendnotes{C}
\bigskip
\vfill

\clearpage

\footnotesize

\lohead{\textsc{register}}

% Definiere theindex-Environment komplett neu ohne reledmac
\makeatletter
\renewenvironment{theindex}{%
  \section*{\indexname}%
  \setlength{\parindent}{0pt}%
  \setlength{\parskip}{0pt plus 0.3pt}%
  \let\item\@idxitem
}{%
  \clearpage
}
\makeatother

\IfFileExists{\jobname-pw.ind}{\input{\jobname-pw.ind}}{}

\end{document}

      