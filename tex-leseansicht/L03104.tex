%% latex-leseansicht-vorspann.tex
%% Vorspann für die Leseansicht.
%% Lädt die gemeinsame Datei latex-vorspann.tex mit nicht gesetztem Schalter.

\newif\ifkorrekturansicht
\korrekturansichtfalse

\input{../tex-inputs/latex-vorspann}

\begin{center}
            \textcolor{red}{ENTWURF, NICHT FERTIG KORRIGIERT}
                      \end{center}
            
         \renewcommand{\erwaehnteOrte}{Orte: Hotel Stadt Pest, Miskolc, Wien}
         \renewcommand{\erwaehnteWerke}{}
               \section[Felix Salten an Arthur Schnitzler, {[}10. 9. 1891{]}]{ Felix Salten an Arthur Schnitzler, {[}10. 9. 1891{]}}\nopagebreak\mylabel{v}\rehead{ }\begin{ledgroupsized}[t]{13cm}\normalsize\beginnumbering \toendnotes[C]{\smallbreak\pagebreak[2]} \Standort{CUL, Schnitzler, B 89, A 1.}
\physDesc{Brief, 1 Blatt, 2 Seiten
\newline{}Handschrift: schwarze Tinte, lateinische Kurrent\newline{}Ordnung: mit Bleistift von unbekannter Hand nummeriert: »6« }\pstart
           \noindent{}{\pb}Lieber Freund! Warum habe ich bis heute keinen Brief? Ich bin außer
               mir. \pend
           \pstart
           Ich leide hier entsetzlich unter einem nie geahnten Rückfall, und stehe Qualen aus,
               die nur Sie sich vorstellen können, und nun deute ich mir Ihr Stillschweigen auf die
               gräßlichste Weise. Ich stelle mir vor, wer weiß, was Sie erfahren haben, \strikeout{u.} das Sie mir nicht verschweigen können, dass Sie mich
               aber hier nicht in Aufregung versetzen wollen, so schreiben lieber Sie garnicht. Oder
               ich vermuthe, wer weiss, wie es Ihnen \strikeout{bes} ergeht, und
               bin schrecklich aufgeregt darüber. Schreiben Sie mir gleich, \uline{was i{\geminationm}er auch} geschehen sein mag. \pend
           \pstart
           Es ist nicht freundschaftlich gerade von {\pb}Ihnen, mich in eine derartige
               Situation zu versetzen. Am liebsten wäre mir, sie nähmen sich die Mühe und
               depeschirten mir zwei aufklärende Worte!\pend
           \pstart
           Ich grüße Sie besti{\geminationm}t als{\\[\baselineskip]}Ihr aufrichtiger{\\[\baselineskip]}\spacefill\mbox{Salten}\pend
           \leftskip=0em{}\pstart
           Miskolcz\oindex{Miskolc@\textbf{Miskolc}|pw}, Hotel Stadt Pest\oindex{Hotel Stadt Pest@\textbf{Hotel Stadt Pest}|pw}.\pend
           
         
         \endnumbering\mylabel{h}\end{ledgroupsized}\begin{anhang}\end{anhang}\newcommand{\dateiname}{L03104}\newcommand{\titel}{Felix Salten an Arthur Schnitzler, [10. 9. 1891]}\newcommand{\editorInnen}{Martin Anton Müller und Laura Untner}%% latex-leseansicht-abspann.tex
%% Abspann für die Leseansicht.
%% Der Schalter \ifkorrekturansicht ist bereits durch den Vorspann gesetzt.

%% latex-abspann.tex
%% Gemeinsamer Abspann für Korrekturansicht und Leseansicht.
%% Setzt den Schalter \ifkorrekturansicht voraus (gesetzt in den
%% einbindenden Dateien latex-korrekturansicht-abspann.tex bzw.
%% latex-leseansicht-abspann.tex).
%% ---------------------------------------------------------------

\normalsize

% Das esempio-Environment wird nur in der Leseansicht benötigt
\ifkorrekturansicht\else
\newenvironment{esempio}[3]%
{
    \vspace{1.5ex}
    \rlap{\underline{#1}}
    \par
    \setlength{\parindent}{0cm}
    \nopagebreak
    \leftskip=#2cm
    \rightskip=#3cm
}
{
    \par
}
\fi

\doendnotes{C}
\bigskip
\vfill

\clearpage

\footnotesize

\ifkorrekturansicht
  \lohead{\textsc{register}}
\fi

% theindex-Environment neu definieren ohne reledmac
\makeatletter
\renewenvironment{theindex}{%
  \ifkorrekturansicht
    \section*{\indexname}%
  \else
    \subsubsection*{Index der erwähnten Entitäten}%
  \fi
  \setlength{\parindent}{0pt}%
  \setlength{\parskip}{0pt plus 0.3pt}%
  \let\item\@idxitem
}{%
  \ifkorrekturansicht\clearpage\fi
}
\makeatother

\IfFileExists{\jobname-pw.ind}{\input{\jobname-pw.ind}}{}

% Quellenangabe nur in der Leseansicht
\ifkorrekturansicht\else
% Fallback-Definitionen, falls die .tex-Datei \titel etc. nicht gesetzt hat
\providecommand{\titel}{}
\providecommand{\editorInnen}{}
\providecommand{\dateiname}{\jobname}

\vspace{3cm}

\vfill

\footnotesize
\textsc{Quelle}: \titel. Herausgegeben von {\editorInnen}. In: \emph{Arthur Schnitzler: Briefwechsel mit Autorinnen und Autoren}.
 Digitale Edition, https://schnitzler-briefe.acdh.oeaw.ac.at/{\dateiname}.html (Stand \today)
\fi

\end{document}


      