%% latex-korrekturansicht-vorspann.tex
%% Vorspann für die Korrekturansicht.
%% Lädt die gemeinsame Datei latex-vorspann.tex mit gesetztem Schalter.

\newif\ifkorrekturansicht
\korrekturansichttrue

\input{../tex-inputs/latex-vorspann}


\section[Arthur Schnitzler: Widmungsexemplar Marionetten für Hugo von Hofmannsthal, {[}23.?{]} 3. 1906]{L01593 Arthur Schnitzler: Widmungsexemplar Marionetten für Hugo von
               Hofmannsthal, {[}23.?{]} 3. 1906}
\nopagebreak\mylabel{L01593v}
\rehead{ }\normalsize\beginnumbering\briefempfaengerindex{Hofmannsthal, Hugo von@\textsc{Hofmannsthal, Hugo von}!zzzSchnitzler, Arthur@\emph{von Arthur Schnitzler}!1906-03-232@{{[}23.?{]} 3. 1906}|(be}
\toendnotes[C]{\smallbreak\pagebreak[2]}\Standort{FDH, FDH 1936.}
\physDesc{Widmung am Vorsatzblatt, 40 Zeichen
\newline{}Handschrift: schwarze Tinte, deutsche Kurrent
\newline{}Hofmannsthal: handschriftliche Notiz im Buchinneren: »\noindent{}Und wenn ich Sie vor mir stehen sehe, bereit dem
                                       ehrfurchtgebietenden Willen Ihres Vaters zu trotzen mit wem,
                                       mit wem vergleiche ich Sie treffender als mit jenem Xerxes\pwindex{Xerxes I. um 519 v. u. Z. – 4.8.485 v. u. Z.@\textsc{Xerxes I.} (um 519 v. u. Z. – 4.8.485 v. u. Z.), \emph{König/Königin}|pw} der \introOben{}Éstultissima furia jactantia in der Raserei
                                          der Selbstüberhebung\introOben{} sich anschickte die Wogen des
                                       Hellespont zu peitschen und dem majestätischen Meeresgott
                                       Fesseln anzulegen?{ / }ein weiblicher Bruder jenes Commodus\pwindex{Commodus 161 – 192@\textsc{Commodus} (161 – 192), \emph{Kaiser/Kaiserin}|pw} (beim II\textsuperscript{ten}
                                       Mal){ / }Schluss der II\textsuperscript{ten} Scene Jourdain –
                                          Lucile\pwindex{Dom Juan ou le Festin de pierre@\emph{Dom Juan ou le Festin de pierre}|pwv}{ / }L. Es gibt nichts was Sie erweichen könnte{ / }J Nein{ / }L. Nun denn (lächelt){ / }J. klopft sie auf die Backen.{ / }Menschen meiner Art u mein Ranges« }
\buchAbdrucke{\weitereDrucke{Hugo von Hofmannsthal: \emph{Bibliothek}. Frankfurt am Main: \emph{S. Fischer} 2011, S. 605.} }\toendnotes[C]{\smallbreak}
\pstart
           \noindent{}{\pb}Meinem lieben Hugo\pend
           \pstart \spacefill\mbox{Arthur}\pend{}
\pstart
           \noindent{}Wien\oindex{Wien@\textbf{Wien}, \emph{A.ADM2}|pw}{ }\label{K_L01593-1v}\edtext{März 906}{\lemma{\textnormal{\emph{März 906}}}\Cendnote{\textnormal{Die Datierung folgt der Widmung an
                           Bahr\pwindex{Bahr, Hermann 19.07.1863 – 15.01.1934@\textsc{Bahr, Hermann} (19.07.1863 – 15.01.1934), \emph{Schriftsteller/Schriftstellerin, Kritiker/Kritikerin}|pwk}, 23. 3. 1906.}}}\label{K_L01593-1}.\pend
           \selectlanguage{ngerman}\vspace{1em}{\vspace{1\baselineskip}}
\pstart
           \centering{}{\pb}\textcolor{gray}{\textbf{\so{MARIONETTEN}\pwindex{Marionetten. Drei Einakter@\emph{Marionetten. Drei Einakter}|pw}}}\pend
           
\pstart
           \centering{}\textcolor{gray}{\textbf{Drei Einakter von}}{\\}\textcolor{gray}{\textbf{\so{Arthur Schnitzler}}}\pend
           {\vspace{1\baselineskip}}
\pstart
           \centering{}\textcolor{gray}{\textbf{\so{S. Fischer, Verlag}\orgindex{S. Fischer Verlag@S. Fischer Verlag|pw}\so{,{ }}\so{Berlin}\oindex{Berlin@\textbf{Berlin}, \emph{P.PPLC}|pw}}}\pend
           
\pstart
           \centering{}\textcolor{gray}{\textbf{1906}}\pend
           \selectlanguage{ngerman}\endnumbering\briefempfaengerindex{Hofmannsthal, Hugo von@\textsc{Hofmannsthal, Hugo von}!zzzSchnitzler, Arthur@\emph{von Arthur Schnitzler}!1906-03-232@{{[}23.?{]} 3. 1906}|)be}\mylabel{L01593h}  \normalsize

\doendnotes{C}
\bigskip
\vfill

\clearpage

\footnotesize

\lohead{\textsc{register}}

% Definiere theindex-Environment komplett neu ohne reledmac
\makeatletter
\renewenvironment{theindex}{%
  \section*{\indexname}%
  \setlength{\parindent}{0pt}%
  \setlength{\parskip}{0pt plus 0.3pt}%
  \let\item\@idxitem
}{%
  \clearpage
}
\makeatother

\IfFileExists{\jobname-pw.ind}{\input{\jobname-pw.ind}}{}

\end{document}

      