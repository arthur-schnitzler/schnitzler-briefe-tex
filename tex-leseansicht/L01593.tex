%% latex-leseansicht-vorspann.tex
%% Vorspann für die Leseansicht.
%% Lädt die gemeinsame Datei latex-vorspann.tex mit nicht gesetztem Schalter.

\newif\ifkorrekturansicht
\korrekturansichtfalse

\input{../tex-inputs/latex-vorspann}


\section[Arthur Schnitzler: Widmungsexemplar Marionetten für Hugo von Hofmannsthal, [23.?] 3. 1906]{L01593 Arthur Schnitzler: Widmungsexemplar Marionetten für Hugo von
               Hofmannsthal, [23.?] 3. 1906}
\nopagebreak\mylabel{L01593v}
\rehead{ }\normalsize\beginnumbering\briefempfaengerindex{Hofmannsthal, Hugo von@\textsc{Hofmannsthal, Hugo von}!zzzSchnitzler, Arthur@\emph{von Arthur Schnitzler}!1906-03-232@{[23.?] 3. 1906}|(be}
\toendnotes[C]{\smallbreak\pagebreak[2]}
\correspDesc{Versand  durch Arthur Schnitzler am [23.?] 3. 1906 in Wien
\newline{}Erhalt  durch Hugo von Hofmannsthal im Zeitraum [23. 3. 1906
                  – 27. 3. 1906?] \textbf{Ort fehlend} }\toendnotes[C]{\smallbreak}
\Standort{FDH, FDH 1936.}
\physDesc{Widmung am Vorsatzblatt, 40 Zeichen
\newline{}Handschrift: schwarze Tinte, deutsche Kurrent
\newline{}Hofmannsthal: handschriftliche Notiz im Buchinneren: »\noindent{}Und wenn ich Sie vor mir stehen sehe, bereit dem
                                       ehrfurchtgebietenden Willen Ihres Vaters zu trotzen mit wem,
                                       mit wem vergleiche ich Sie treffender als mit jenem Xerxes\pwindex{Xerxes I. um 519 v. u. Z. Iran – 4.\,8.\,485 v.\,u.\,Z.@\textsc{Xerxes I.} (um 519 v. u. Z. Iran – 4.\,8.\,485 v.\,u.\,Z.), \emph{König}|pw} der \introOben{}Éstultissima furia jactantia in der Raserei
                                          der Selbstüberhebung\introOben{} sich anschickte die Wogen des
                                       Hellespont zu peitschen und dem majestätischen Meeresgott
                                       Fesseln anzulegen?{ / }ein weiblicher Bruder jenes Commodus\pwindex{Commodus 161 – 192@\textsc{Commodus} (161 – 192), \emph{Kaiser}|pw} (beim II\textsuperscript{ten}
                                       Mal){ / }Schluss der II\textsuperscript{ten} Scene Jourdain –
                                          Lucile\pwindex{\textcolor{red}{\textsuperscript{XXXX indx1}}!Dom Juan ou le Festin de pierre@\strich\emph{Dom Juan ou le Festin de pierre}|pwv}{ / }L. Es gibt nichts was Sie erweichen könnte{ / }J Nein{ / }L. Nun denn (lächelt){ / }J. klopft sie auf die Backen.{ / }Menschen meiner Art u mein Ranges« }
\buchAbdrucke{\weitereDrucke{Hugo von Hofmannsthal: \emph{Bibliothek}. Herausgegeben von Ellen Ritter † in Zusammenarbeit mit Dalia Bukauskaité und Konrad Heumann. Frankfurt am Main: \emph{S. Fischer} 2011, S. 605 (Sämtliche Werke. Kritische Ausgabe, XL).} }\toendnotes[C]{\smallbreak}
\pstart
           \noindent{}{\pb}Meinem lieben Hugo\pend
           \pstart \spacefill\mbox{Arthur}\pend{}
\pstart
           \noindent{}Wien\oindex{Wien@\textbf{Wien}, \emph{Verwaltungsgebiet}|pw}{ }\label{K_L01593-1v}\edtext{März 906}{\lemma{\textnormal{\emph{März 906}}}\Cendnote{\textnormal{Die Datierung folgt der Widmung an
                           Bahr\pwindex{Bahr, Hermann 19.\,7.\,1863 Linz – 15.\,1.\,1934 München@\textsc{Bahr, Hermann} (19.\,7.\,1863 Linz – 15.\,1.\,1934 München), \emph{Schriftsteller, Kritiker}|pwk}, XXXX Auszeichnungsfehler: Dokument L01592 nicht gefunden.}}}\label{K_L01593-1}.\pend
           \selectlanguage{ngerman}\vspace{1em}{\vspace{1\baselineskip}}
\pstart
           \centering{}{\pb}\textcolor{gray}{\textbf{\so{MARIONETTEN}\pwindex{Schnitzler, Arthur 15.\,5.\,1862 Wien – 21.\,10.\,1931 ebd.@\textsc{Schnitzler, Arthur} (15.\,5.\,1862 Wien – 21.\,10.\,1931 ebd.), \emph{Schriftsteller, Mediziner}!Marionetten. Drei Einakter@\strich\emph{Marionetten. Drei Einakter}|pw}}}\pend
           
\pstart
           \centering{}\textcolor{gray}{\textbf{Drei Einakter von}}{\\}\textcolor{gray}{\textbf{\so{Arthur Schnitzler}}}\pend
           {\vspace{1\baselineskip}}
\pstart
           \centering{}\textcolor{gray}{\textbf{\so{S. Fischer, Verlag}\orgindex{S. Fischer Verlag@S. Fischer Verlag|pw}\so{,}\hspace*{1em}\so{Berlin}\oindex{Berlin@\textbf{Berlin}, \emph{Hauptstadt}|pw}}}\pend
           
\pstart
           \centering{}\textcolor{gray}{\textbf{1906}}\pend
           \selectlanguage{ngerman}\endnumbering\briefempfaengerindex{Hofmannsthal, Hugo von@\textsc{Hofmannsthal, Hugo von}!zzzSchnitzler, Arthur@\emph{von Arthur Schnitzler}!1906-03-232@{[23.?] 3. 1906}|)be}\mylabel{L01593h}  \newcommand{\dateiname}{L01593}\newcommand{\titel}{Arthur Schnitzler: Widmungsexemplar Marionetten für Hugo von Hofmannsthal, [23.?] 3. 1906}\newcommand{\editorInnen}{Martin Anton Müller und Gerd-Hermann Susen}%% latex-leseansicht-abspann.tex
%% Abspann für die Leseansicht.
%% Der Schalter \ifkorrekturansicht ist bereits durch den Vorspann gesetzt.

%% latex-abspann.tex
%% Gemeinsamer Abspann für Korrekturansicht und Leseansicht.
%% Setzt den Schalter \ifkorrekturansicht voraus (gesetzt in den
%% einbindenden Dateien latex-korrekturansicht-abspann.tex bzw.
%% latex-leseansicht-abspann.tex).
%% ---------------------------------------------------------------

\normalsize

% Das esempio-Environment wird nur in der Leseansicht benötigt
\ifkorrekturansicht\else
\newenvironment{esempio}[3]%
{
    \vspace{1.5ex}
    \rlap{\underline{#1}}
    \par
    \setlength{\parindent}{0cm}
    \nopagebreak
    \leftskip=#2cm
    \rightskip=#3cm
}
{
    \par
}
\fi

\doendnotes{C}
\bigskip
\vfill

\clearpage

\footnotesize

\ifkorrekturansicht
  \lohead{\textsc{register}}
\fi

% theindex-Environment neu definieren ohne reledmac
\makeatletter
\renewenvironment{theindex}{%
  \ifkorrekturansicht
    \section*{\indexname}%
  \else
    \subsubsection*{Index der erwähnten Entitäten}%
  \fi
  \setlength{\parindent}{0pt}%
  \setlength{\parskip}{0pt plus 0.3pt}%
  \let\item\@idxitem
}{%
  \ifkorrekturansicht\clearpage\fi
}
\makeatother

\IfFileExists{\jobname-pw.ind}{\input{\jobname-pw.ind}}{}

% Quellenangabe nur in der Leseansicht
\ifkorrekturansicht\else
% Fallback-Definitionen, falls die .tex-Datei \titel etc. nicht gesetzt hat
\providecommand{\titel}{}
\providecommand{\editorInnen}{}
\providecommand{\dateiname}{\jobname}

\vspace{3cm}

\vfill

\footnotesize
\textsc{Quelle}: \titel. Herausgegeben von {\editorInnen}. In: \emph{Arthur Schnitzler: Briefwechsel mit Autorinnen und Autoren}.
 Digitale Edition, https://schnitzler-briefe.acdh.oeaw.ac.at/{\dateiname}.html (Stand \today)
\fi

\end{document}


