%% latex-leseansicht-vorspann.tex
%% Vorspann für die Leseansicht.
%% Lädt die gemeinsame Datei latex-vorspann.tex mit nicht gesetztem Schalter.

\newif\ifkorrekturansicht
\korrekturansichtfalse

\input{../tex-inputs/latex-vorspann}


\section[Olga Schnitzler an Paula Beer-Hofmann, 23. 1. 1910]{L02553 Olga Schnitzler an Paula Beer-Hofmann, 23. 1. 1910}
\nopagebreak\mylabel{L02553v}
\rehead{ }\normalsize\beginnumbering\briefempfaengerindex{Beer-Hofmann, Paula@\textsc{Beer-Hofmann, Paula}!zzzSchnitzler, Olga@\emph{von Olga Schnitzler}!1910-01-231@{23. 1. 1910}|(be}
\toendnotes[C]{\smallbreak\pagebreak[2]}
\correspDesc{Versand  durch Olga Schnitzler am 23. 1. 1910 in Dresden
\newline{}Erhalt  durch Paula Beer-Hofmann im Zeitraum [24. 1. 1910
                  – 28. 1. 1910?] in Wien}\toendnotes[C]{\smallbreak}
\Standort{YCGL, MSS 31.}
\physDesc{Bildpostkarte, 400 Zeichen
\newline{}Handschrift: schwarze Tinte, lateinische Kurrent
\newline{}Versand: Stempel: »\nobreak{}\oindex{Dresden@\textbf{Dresden}|pwk}Dresden, 23. 1. 10, 5–6N\nobreak{}«.  }\toendnotes[C]{\smallbreak}\pstart{}{\pb}Frau Paula Beer-Hofmann\pend{}\pstart{}Wien XIX\oindex{XIX., Döbling@\textbf{XIX., Döbling}, \emph{Verwaltungsgebiet}|pw}\pend{}\pstart{}Hasenauerstrasse 59\oindex{Wien@\textbf{Wien}!XVIII., Währing@\textbf{XVIII., Währing}!Hasenauerstraße 59@\textbf{Hasenauerstraße 59}, \emph{Wohngebäude}|pw}.\pend{}{\bigskip}
\pstart
           \noindent{}\centering{}{\pb}\textcolor{gray}{\textbf{\textsc{Hotel Bellevue\oindex{Hotel Bellevue [Dresden]@\textbf{Hotel Bellevue [Dresden]}, \emph{Hotel}|pw}, Dresden\oindex{Dresden@\textbf{Dresden}|pw}}}}\pend
           
\pstart
           \centering{}\textcolor{gray}{\textbf{Great sitting-room\hspace*{1.5em}Grosser
                     Salon\hspace*{1.5em}Grand Salon}}\pend
           \vspace{1em}
\pstart
           \noindent{}{\pb}Liebe geliebte Paula, es war so schön hier, ungeheuer erfrischend.
               Ein \label{K_L02553-1v}\edtext{grosser Erfolg\pwindex{\textcolor{red}{\textsuperscript{XXXX indx1}}!Schleier der Pierrette. Pantomime in drei Bildern@\strich\emph{Der Schleier der Pierrette. Pantomime in drei Bildern}|pwv}}{\lemma{\textnormal{\emph{grosser Erfolg}}}\Cendnote{\textnormal{die Uraufführung\eventindex{Semperoper@\textbf{Semperoper}!Uraufführung von Der Schleier der Pierrette, Premiere von Versiegelt, 22.1.1910@Uraufführung von Der Schleier der Pierrette, Premiere von Versiegelt, 22.1.1910|pwkv} von \emph{Der Schleier der Pierrette}\pwindex{\textcolor{red}{\textsuperscript{XXXX indx1}}!Schleier der Pierrette. Pantomime in drei Bildern@\strich\emph{Der Schleier der Pierrette. Pantomime in drei Bildern}|pwk} am 22. 1. 1910}}}\label{K_L02553-1}, Schuch\pwindex{Schuch, Ernst von 30.\,11.\,1846 Graz – 10.\,5.\,1914 Dresden@\textsc{Schuch, Ernst von} (30.\,11.\,1846 Graz – 10.\,5.\,1914 Dresden), \emph{Dirigent}|pw} ist prachtvoll, die Musik
               entzückend. Wie schön wär’s gewesen wenn Ihr auch da gewesen wärt. Von Dohnányi\pwindex{Dohnányi, Ernst von 27.\,7.\,1877 Bratislava – 9.\,2.\,1960 New York City@\textsc{Dohnányi, Ernst von} (27.\,7.\,1877 Bratislava – 9.\,2.\,1960 New York City), \emph{Komponist, Pianist}|pw} sind Freunde\pwindex{Benzon, Otto Carl Waldemar 17.\,1.\,1856 Kopenhagen – 16.\,5.\,1927@\textsc{Benzon, Otto Carl Waldemar} (17.\,1.\,1856 Kopenhagen – 16.\,5.\,1927), \emph{Schriftsteller, Fabrikant}|pwv} sogar aus Dänemark\oindex{Dänemark@\textbf{Dänemark}|pw} hier. Schönes Wetter, freudige Stimmung.\pend
           
\pstart
           Leben Sie wol, auf Wiedersehen! Grüssen Sie den Herrn D\textsuperscript{r}\pwindex{Beer-Hofmann, Richard 11.\,7.\,1866 Wien – 26.\,9.\,1945 New York City@\textsc{Beer-Hofmann, Richard} (11.\,7.\,1866 Wien – 26.\,9.\,1945 New York City), \emph{Schriftsteller}|pwv} u. die Kinder\pwindex{Beer-Hofmann, Gabriel 9.\,1.\,1901 Wien – 24.\,3.\,1971 St Albans@\textsc{Beer-Hofmann, Gabriel} (9.\,1.\,1901 Wien – 24.\,3.\,1971 St Albans), \emph{Schriftsteller, Filmagent}|pwv}\pwindex{Beer-Hofmann, Mirjam 4.\,9.\,1897 Wien – 24.\,12.\,1984 New York City@\textsc{Beer-Hofmann, Mirjam} (4.\,9.\,1897 Wien – 24.\,12.\,1984 New York City)|pwv}\pwindex{Beer-Hofmann, Naëmah 20.\,12.\,1898 Wien – 10.\,11.\,1971 New York City@\textsc{Beer-Hofmann, Naëmah} (20.\,12.\,1898 Wien – 10.\,11.\,1971 New York City)|pwv}.{\\[\baselineskip]}Ihre\spacefill\mbox{Olga.}\pend
           \leftskip=0em{}
\pstart
           23. 1. 10.\pend
           \selectlanguage{ngerman}\endnumbering\briefempfaengerindex{Beer-Hofmann, Paula@\textsc{Beer-Hofmann, Paula}!zzzSchnitzler, Olga@\emph{von Olga Schnitzler}!1910-01-231@{23. 1. 1910}|)be}\mylabel{L02553h}  \newcommand{\dateiname}{L02553}\newcommand{\titel}{Olga Schnitzler an Paula Beer-Hofmann, 23. 1. 1910}\newcommand{\editorInnen}{Martin Anton Müller und Gerd-Hermann Susen}%% latex-leseansicht-abspann.tex
%% Abspann für die Leseansicht.
%% Der Schalter \ifkorrekturansicht ist bereits durch den Vorspann gesetzt.

%% latex-abspann.tex
%% Gemeinsamer Abspann für Korrekturansicht und Leseansicht.
%% Setzt den Schalter \ifkorrekturansicht voraus (gesetzt in den
%% einbindenden Dateien latex-korrekturansicht-abspann.tex bzw.
%% latex-leseansicht-abspann.tex).
%% ---------------------------------------------------------------

\normalsize

% Das esempio-Environment wird nur in der Leseansicht benötigt
\ifkorrekturansicht\else
\newenvironment{esempio}[3]%
{
    \vspace{1.5ex}
    \rlap{\underline{#1}}
    \par
    \setlength{\parindent}{0cm}
    \nopagebreak
    \leftskip=#2cm
    \rightskip=#3cm
}
{
    \par
}
\fi

\doendnotes{C}
\bigskip
\vfill

\clearpage

\footnotesize

\ifkorrekturansicht
  \lohead{\textsc{register}}
\fi

% theindex-Environment neu definieren ohne reledmac
\makeatletter
\renewenvironment{theindex}{%
  \ifkorrekturansicht
    \section*{\indexname}%
  \else
    \subsubsection*{Index der erwähnten Entitäten}%
  \fi
  \setlength{\parindent}{0pt}%
  \setlength{\parskip}{0pt plus 0.3pt}%
  \let\item\@idxitem
}{%
  \ifkorrekturansicht\clearpage\fi
}
\makeatother

\IfFileExists{\jobname-pw.ind}{\input{\jobname-pw.ind}}{}

% Quellenangabe nur in der Leseansicht
\ifkorrekturansicht\else
% Fallback-Definitionen, falls die .tex-Datei \titel etc. nicht gesetzt hat
\providecommand{\titel}{}
\providecommand{\editorInnen}{}
\providecommand{\dateiname}{\jobname}

\vspace{3cm}

\vfill

\footnotesize
\textsc{Quelle}: \titel. Herausgegeben von {\editorInnen}. In: \emph{Arthur Schnitzler: Briefwechsel mit Autorinnen und Autoren}.
 Digitale Edition, https://schnitzler-briefe.acdh.oeaw.ac.at/{\dateiname}.html (Stand \today)
\fi

\end{document}


