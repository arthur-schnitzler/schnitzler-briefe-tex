\input{../tex-inputs/latex-pdf-vorspann}
\begin{center}
            \textcolor{red}{ENTWURF. ENTZIFFERUNG NOCH NICHT KORREKTURGELESEN}
                      \end{center}
            
               \section[Bertha von Suttner an Arthur Schnitzler, 3. 5. 1893]{ Bertha von Suttner an Arthur Schnitzler, 3. 5. 1893}\nopagebreak\mylabel{v}\rehead{ }\begin{ledgroupsized}[t]{13cm}\normalsize\beginnumbering\briefempfaengerindex{Schnitzler, Arthur@\textsc{Schnitzler, Arthur}!zzzSuttner, Bertha von@\emph{von Bertha von Suttner}!1893-05-033@{3. 5. 1893}|(be} \toendnotes[C]{\smallbreak\pagebreak[2]} \Standort{CUL, Schnitzler, B 104.}
\physDesc{Briefkarte
\newline{}Handschrift: schwarze Tinte, lateinische Kurrent}\Standort{DLA, A:Schnitzler, HS.NZ85.1.4773.}
\physDesc{1 Blatt, 1 Seite, maschinelle Abschrift}\toendnotes[C]{\smallbreak}\pstart
           \noindent{}{\pb}\textcolor{gray}{\textbf{\label{T_L00210-1v}\edtext{BS}{\lemma{\textnormal{\emph{BS}}}\Cendnote{\textnormal{Monogramm und
                                    Krone in Golddruck}}}\label{T_L00210-1h}}}\hfill Harmannsdorf\oindex{Harmannsdorf@\textbf{Harmannsdorf}|pw}{ }3/5 93.\pend
           \pstart{}Hochgeehrter Herr College\pend\pstart
           Innigstes Beileid zu dem grossen so unzeitigen und unerwarteten Verlust! Ich
                    hatte den Vorzug, den Betrauerten\pwindex{Schnitzler, Johann 10.04.1835 – 02.05.1893@\textsc{Schnitzler, Johann} (10.04.1835 – 02.05.1893), \emph{Laryngologe}|pwv} persönlich zu kennen und die Nachricht von seinem Tode hat
                    mich schmerzlich bewegt. Mit theilnahms{\pb}vollem Händedruck\pend
           \pstart
           Ihre{\\[\baselineskip]}\spacefill\mbox{Bertha v. Suttner}\pend
           \leftskip=0em{}\pstart
           \noindent{}(Die eine lebhafte Verehrerin Ihres funkelnden Talentes ist)\pend
           \endnumbering\briefempfaengerindex{Schnitzler, Arthur@\textsc{Schnitzler, Arthur}!zzzSuttner, Bertha von@\emph{von Bertha von Suttner}!1893-05-033@{3. 5. 1893}|)be}\mylabel{h}\end{ledgroupsized}  \newcommand{\dateiname}{L00210}\newcommand{\titel}{Bertha von Suttner an Arthur Schnitzler, 3. 5. 1893}\newcommand{\editorInnen}{Martin Anton Müller und Gerd-Hermann Susen}\input{../tex-inputs/latex-pdf-abspann}
      