%% latex-korrekturansicht-vorspann.tex
%% Vorspann für die Korrekturansicht.
%% Lädt die gemeinsame Datei latex-vorspann.tex mit gesetztem Schalter.

\newif\ifkorrekturansicht
\korrekturansichttrue

\input{../tex-inputs/latex-vorspann}


\section[Bertha von Suttner an Arthur Schnitzler, 3. 5. 1893]{L00210 Bertha von Suttner an Arthur Schnitzler, 3. 5. 1893}
\nopagebreak\mylabel{L00210v}
\rehead{ }\normalsize\beginnumbering\briefempfaengerindex{Schnitzler, Arthur@\textsc{Schnitzler, Arthur}!zzzSuttner, Bertha von@\emph{von Bertha von Suttner}!1893-05-033@{3. 5. 1893}|(be}
\toendnotes[C]{\smallbreak\pagebreak[2]}\Standort{CUL, Schnitzler, B 104.}
\physDesc{Briefkarte, 350 Zeichen
\newline{}Handschrift: schwarze Tinte, lateinische Kurrent}\Standort{DLA, A:Schnitzler, HS.NZ85.1.4773.}
\physDesc{maschinenschriftliche Abschrift1 Blatt, 1 Seite, 350 Zeichen
\newline{}Schreibmaschine}\toendnotes[C]{\smallbreak}
\pstart
           {\pb}\textcolor{gray}{\textbf{\label{T_L00210-1v}\edtext{BS}{\lemma{\textnormal{\emph{BS}}}\Cendnote{\textnormal{Monogramm und Krone in Golddruck}}}\label{T_L00210-1}}}\hfill Harmannsdorf\oindex{Harmannsdorf@\textbf{Harmannsdorf}, \emph{P.PPL}|pw}{ }3/5 93.\pend
           
\pstart{}Hochgeehrter Herr College\pend\vspace{0.5em}
\pstart
           Innigstes Beileid zu dem grossen so unzeitigen und unerwarteten Verlust! Ich hatte
               den Vorzug, den Betrauerten\pwindex{Schnitzler, Johann 10.04.1835 – 02.05.1893@\textsc{Schnitzler, Johann} (10.04.1835 – 02.05.1893), \emph{Laryngologe/Laryngologin}|pwv}
               persönlich zu kennen und die Nachricht von seinem Tode hat mich schmerzlich bewegt.
               Mit theilnahms{\pb}vollem Händedruck\pend
           
\pstart
           Ihre{\\[\baselineskip]}\spacefill\mbox{Bertha v. Suttner}\pend
           \leftskip=0em{}
\pstart
           \noindent{}(Die eine lebhafte Verehrerin Ihres funkelnden Talentes ist)\pend
           \selectlanguage{ngerman}\endnumbering\briefempfaengerindex{Schnitzler, Arthur@\textsc{Schnitzler, Arthur}!zzzSuttner, Bertha von@\emph{von Bertha von Suttner}!1893-05-033@{3. 5. 1893}|)be}\mylabel{L00210h}  \normalsize

\doendnotes{C}
\bigskip
\vfill

\clearpage

\footnotesize

\lohead{\textsc{register}}

% Definiere theindex-Environment komplett neu ohne reledmac
\makeatletter
\renewenvironment{theindex}{%
  \section*{\indexname}%
  \setlength{\parindent}{0pt}%
  \setlength{\parskip}{0pt plus 0.3pt}%
  \let\item\@idxitem
}{%
  \clearpage
}
\makeatother

\IfFileExists{\jobname-pw.ind}{\input{\jobname-pw.ind}}{}

\end{document}

      