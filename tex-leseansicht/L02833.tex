%% latex-leseansicht-vorspann.tex
%% Vorspann für die Leseansicht.
%% Lädt die gemeinsame Datei latex-vorspann.tex mit nicht gesetztem Schalter.

\newif\ifkorrekturansicht
\korrekturansichtfalse

\input{../tex-inputs/latex-vorspann}


         
         \renewcommand{\erwaehntePersonen}{Personen: Emma Altmann, Paul Altmann, Carl Altmann, Kasimir Felix Badeni, Hermann Bahr, Max Eugen Burckhard, Alfred Dreyfus, Paul Goldmann, Lucien Millevoye, Marie Reinhard, William Shakespeare, Leopold Sonnemann, Alice Ziegler, Émile Zola}
         \renewcommand{\erwaehnteInstitutionen}{Institutionen: Frankfurter Zeitung}
         \renewcommand{\erwaehnteOrte}{Orte: Deutschland, Paris, Prag, Wien, rue de la Bourse}
         \renewcommand{\erwaehnteWerke}{Werke: Aux Syndiqués de Francfort, Das Vermächtnis. Schauspiel in drei Akten, Die Toten schweigen, Die Zeit. Wiener Wochenschrift, Freiwild. Schauspiel in 3 Akten, La Patrie. Organe de la défense nationale, Le Figaro, Le Syndicat, L’Express du Midi. Organe quotidien de Défense Sociale et Religieuse, Tagebuch, Weihnachts-Einkäufe, À la chambre, ’s Katherl. (Volksstück in fünf Aufzügen von Max Burckhard. Zum ersten Mal aufgeführt im Raimundtheater am 25. November 1897.), ’s Katherl. Volksstück in fünf Aufzügen}
               \section[ Paul Goldmann an Arthur Schnitzler, 10. 12. {[}1897{]}]{ Paul Goldmann an Arthur Schnitzler, 10. 12. {[}1897{]}}\nopagebreak\mylabel{v}\rehead{ }\begin{ledgroupsized}[t]{13cm}\normalsize\beginnumbering \toendnotes[C]{\smallbreak\pagebreak[2]} \Standort{DLA, A:Schnitzler, HS.NZ85.1.3167.}
\physDesc{Brief, 2 Blätter, 7 Seiten, 2961 Zeichen
\newline{}Handschrift: blaue Tinte, deutsche Kurrent
\newline{}Schnitzler: 1) mit Bleistift das Jahr »97« vermerkt  2) mit rotem Buntstift drei Unterstreichungen}\toendnotes[C]{\smallbreak}\pstart
           \noindent{}{\pb}\textcolor{gray}{\textbf{\textbf{Frankfurter Zeitung\orgindex{Frankfurter Zeitung@Frankfurter Zeitung|pw}}}}\pend
           \pstart
           \textcolor{gray}{\textbf{(\begin{otherlanguage}{french}Gazette de Francfort\end{otherlanguage}\orgindex{Frankfurter Zeitung@Frankfurter Zeitung|pw}).}}\pend
           \pstart
           \textcolor{gray}{\textbf{\textbf{\begin{otherlanguage}{french}Fondateur M.\end{otherlanguage}{ }L. Sonnemann\pwindex{Sonnemann, Leopold 1831-10-29 – 1909-10-30@\textsc{Sonnemann, Leopold} (1831-10-29 – 1909-10-30), \emph{Journalist, Herausgeber}|pw}.}}}\pend
           \pstart
           \begin{otherlanguage}{french}\textcolor{gray}{\textbf{Journal politique, financier,}}\end{otherlanguage}\pend
           \pstart
           \begin{otherlanguage}{french}\textcolor{gray}{\textbf{commercial et littéraire.}}\end{otherlanguage}\pend
           \pstart
           \begin{otherlanguage}{french}\textcolor{gray}{\textbf{\textbf{Paraissant trois fois par jour.}}}\end{otherlanguage}\hfill \textsc{Paris\oindex{Paris@\textbf{Paris}|pw}}, 10. December.\pend
           \pstart
           \begin{otherlanguage}{french}\textcolor{gray}{\textbf{\textbf{Bureau à Paris\oindex{Paris@\textbf{Paris}|pw}}}}\end{otherlanguage}\pend
           \pstart
           \begin{otherlanguage}{french}\textcolor{gray}{\textbf{\textbf{10 \so{Rue de la Bourse}\oindex{rue de la Bourse@\textbf{rue de la Bourse}|pw}.}}}\end{otherlanguage}\pend
           \pstart\center{}Mein lieber Freund,\pend\pstart
           Endlich ein freier Augenblick! Ich habe eine Reihe furchtbar aufgeregter Tage hinter
               mir. Die Geſchichte fing an mit einem \label{K_L02833-1v}\edtext{Artikel\pwindex{Millevoye, Lucien 1850-08-01 – 1918-03-25@\textsc{Millevoye, Lucien} (1850-08-01 – 1918-03-25), \emph{Politiker, Journalist}!Aux Syndiques de Francfort1897-12-04@\strich\emph{Aux Syndiqués de Francfort} {[}1897-12-04{]}|pwv} von \textsc{Millevoye\pwindex{Millevoye, Lucien 1850-08-01 – 1918-03-25@\textsc{Millevoye, Lucien} (1850-08-01 – 1918-03-25), \emph{Politiker, Journalist}|pw}}}{\lemma{\textnormal{\emph{Artikel von Millevoye}}}\Cendnote{\textnormal{Ende November und Anfang Dezember 1897
                  erschienen fast täglich Kommentare zur Affäre Dreyfus\pwindex{Dreyfus, Alfred 1859-10-09 – 1935-07-12@\textsc{Dreyfus, Alfred} (1859-10-09 – 1935-07-12), \emph{Militär}|pwk} von Lucien Millevoye\pwindex{Millevoye, Lucien 1850-08-01 – 1918-03-25@\textsc{Millevoye, Lucien} (1850-08-01 – 1918-03-25), \emph{Politiker, Journalist}|pwk} in
                  der von ihm geleiteten Zeitung \emph{La Patrie}\pwindex{Patrie. Organe de la defense nationale1841 – 1937@\emph{La Patrie. Organe de la défense nationale} {[}1841 – 1937{]}|pwk}. Goldmann\pwindex{Goldmann, Paul 31.01.1865 – 25.09.1935@\textsc{Goldmann, Paul} (31.01.1865 – 25.09.1935), \emph{Schriftsteller, Journalist}|pwk} bezog sich auf folgenden Artikel\pwindex{Millevoye, Lucien 1850-08-01 – 1918-03-25@\textsc{Millevoye, Lucien} (1850-08-01 – 1918-03-25), \emph{Politiker, Journalist}!Aux Syndiques de Francfort1897-12-04@\strich\emph{Aux Syndiqués de Francfort} {[}1897-12-04{]}|pwkv}: Lucien Millevoye\pwindex{Millevoye, Lucien 1850-08-01 – 1918-03-25@\textsc{Millevoye, Lucien} (1850-08-01 – 1918-03-25), \emph{Politiker, Journalist}|pwk}: \emph{Aux Syndiqués de Francfort}\pwindex{Millevoye, Lucien 1850-08-01 – 1918-03-25@\textsc{Millevoye, Lucien} (1850-08-01 – 1918-03-25), \emph{Politiker, Journalist}!Aux Syndiques de Francfort1897-12-04@\strich\emph{Aux Syndiqués de Francfort} {[}1897-12-04{]}|pwk}. In: \emph{La Patrie. Organe de la defense nationale}\pwindex{Patrie. Organe de la defense nationale1841 – 1937@\emph{La Patrie. Organe de la défense nationale} {[}1841 – 1937{]}|pwk}, Jg. 57,
                     Nr. 5, 4. 12. 1897, S. 1. Als Beilage ist
                  der Artikel\pwindex{Millevoye, Lucien 1850-08-01 – 1918-03-25@\textsc{Millevoye, Lucien} (1850-08-01 – 1918-03-25), \emph{Politiker, Journalist}!Aux Syndiques de Francfort1897-12-04@\strich\emph{Aux Syndiqués de Francfort} {[}1897-12-04{]}|pwkv} nicht
                  erhalten.}}}\label{K_L02833-1h}, der mich mit Koth bewarf. Ich lege ihn Dir bei, damit Du ſiehſt,
                  \strikeout{in} welchen Ton die Polemik in dieſen heißen Tagen
               angenommen hat und was man ſich Alles ſagen laſſen muß, wenn man ruhig und beſcheiden
               für ſeine Überzeugung eintritt. Sonntag kam der
                  \label{K_L02833-2v}\edtext{Einbruch}{\lemma{\textnormal{\emph{Einbruch}}}\Cendnote{\textnormal{Darüber wurde auch berichtet: [O. V.]: \emph{À la chambre}\pwindex{?? Werk@Nicht ermittelte Verfasserinnen und Verfasser!À la chambre1897-12-07@\emph{À la chambre} {[}1897-12-07{]}|pwk}. In: \emph{L’Express du Midi. Organe quotidien de Défense Sociale et Religieuse}\pwindex{?? Werk@Nicht ermittelte Verfasserinnen und Verfasser!Express du Midi. Organe quotidien de Defense Sociale et Religieuse1891 – 1938@\emph{L’Express du Midi. Organe quotidien de Défense Sociale et Religieuse} {[}1891 – 1938{]}|pwk},
                     Jg. 7, Nr. 2077, 7. 12. 1897,
                  S. [2].}}}\label{K_L02833-2h}, von dem Du wohl in den Blättern geleſen haſt. Man hat mir
               meine Briefe geſtohlen, {\pb}Briefe von meiner Familie
               und von Dir. Wahrſcheinlich war der Einbruch eine verkleidete Hausſuchung. Irgend ein
               officieller Dummkopf hat vielleicht geglaubt, daß \substVorne{}\textsuperscript{\textcolor{gray}{×}}\substDazwischen{}e\substHinten{}r bei mir Documente zum Fall \textsc{Dreyfus\pwindex{Dreyfus, Alfred 1859-10-09 – 1935-07-12@\textsc{Dreyfus, Alfred} (1859-10-09 – 1935-07-12), \emph{Militär}|pw}} finden könnte oder \strikeout{doc} documentariſche Beweiſe
               für die Exiſtenz des famoſen »\label{K_L02833-3v}\edtext{Syndicats}{\lemma{\textnormal{\emph{Syndicats}}}\Cendnote{\textnormal{Bezug auf das
                  vermeintliche »Judensyndikat« hinter der Dreyfus\pwindex{Dreyfus, Alfred 1859-10-09 – 1935-07-12@\textsc{Dreyfus, Alfred} (1859-10-09 – 1935-07-12), \emph{Militär}|pwk}-Affäre (vgl. Emile Zola\pwindex{Zola, Emile 02.04.1840 – 29.09.1902@\textsc{Zola, Émile} (02.04.1840 – 29.09.1902), \emph{Schriftsteller, Journalist}|pwk}: \emph{Le Syndicat}\pwindex{Zola, Emile 02.04.1840 – 29.09.1902@\textsc{Zola, Émile} (02.04.1840 – 29.09.1902), \emph{Schriftsteller, Journalist}!Le Syndicat1897-12-01@\strich\emph{Le Syndicat} {[}1897-12-01{]}|pwk}. In: \emph{Le
                        Figaro}\pwindex{Le Figaro1826-01-15@\emph{Le Figaro} {[}1826-01-15{]}|pwk}, Jg. 43, Nr. 335, 1. 12. 1897,
                     S. 1).}}}\label{K_L02833-3h}« (das nie exiſtirt hat). Tagelang hat ſich hier die Preſſe
               mit mir beſchäftigt, und obwohl kein böſes Wort gegen mich gefallen iſt, ſo iſt es
               doch unheimlich, als Deutſch\oindex{Deutschland@\textbf{Deutschland}|pwv}er
               in ſo leidenſchaftlich bewegter Zeit im Mittelpunkt des Intereſſes zu ſtehen.\pend
           \pstart
           {\pb}Endlich alſo kann ich ein wenig aufathmen, und
               endlich kann ich Dir Deinen ſo lieben und ſchönen Brief beantworten. Ich habe mich
               von Herzen über Deine \label{K_L02833-4v}\edtext{Prag\oindex{Prag@\textbf{Prag}|pw}er Erfolge\pwindex{Schnitzler, Arthur 15.05.1862 – 21.10.1931@\textsc{Schnitzler, Arthur} (15.05.1862 – 21.10.1931), \emph{Schriftsteller, Mediziner}!Freiwild. Schauspiel in 3 Akten1896@\strich\emph{Freiwild. Schauspiel in 3 Akten} {[}1896{]}|pwv}\pwindex{Schnitzler, Arthur 15.05.1862 – 21.10.1931@\textsc{Schnitzler, Arthur} (15.05.1862 – 21.10.1931), \emph{Schriftsteller, Mediziner}!Toten schweigen01. 10. 1897@\strich\emph{Die Toten schweigen} {[}01. 10. 1897{]}|pwv}\pwindex{Schnitzler, Arthur 15.05.1862 – 21.10.1931@\textsc{Schnitzler, Arthur} (15.05.1862 – 21.10.1931), \emph{Schriftsteller, Mediziner}!Weihnachts-Einkaeufe24. 12. 1891@\strich\emph{Weihnachts-Einkäufe} {[}24. 12. 1891{]}|pwv}}{\lemma{\textnormal{\emph{Prager Erfolge}}}\Cendnote{\textnormal{siehe Paul Goldmann an Arthur Schnitzler, 23. 12. [1897]}}}\label{K_L02833-4h} gefreut. Es iſt gut, daß das Alles noch vor die Zeit des \label{K_L02833-5v}\edtext{Aufruhrs}{\lemma{\textnormal{\emph{Aufruhrs}}}\Cendnote{\textnormal{Auslöser waren gewaltvolle Proteste als Reaktion auf die Badeni\pwindex{Badeni, Kasimir Felix 1846-10-14 – 1909-07-09@\textsc{Badeni, Kasimir Felix} (1846-10-14 – 1909-07-09), \emph{Politiker, Jurist}|pwkv}sche
                  Sprachverordnung, die sich von Ende November bis
                  Anfang Dezember 1897 erstreckten. Auch Schnitzler\pwindex{Schnitzler, Arthur 15.05.1862 – 21.10.1931@\textsc{Schnitzler, Arthur} (15.05.1862 – 21.10.1931), \emph{Schriftsteller, Mediziner}|pwk} notierte die »Unruhen,
                     politischer Natur« am 28. 11. 1897 – vor seiner Abreise aus Prag – im
                     \emph{Tagebuch}\pwindex{Schnitzler, Arthur 15.05.1862 – 21.10.1931@\textsc{Schnitzler, Arthur} (15.05.1862 – 21.10.1931), \emph{Schriftsteller, Mediziner}!Tagebuch1981 – 2000@\strich\emph{Tagebuch} {[}1981 – 2000{]}|pwk}.}}}\label{K_L02833-5h} gefallen iſt, ſonſt wäre
               es für Dich auch recht ungemüthlich in \textsc{Prag\oindex{Prag@\textbf{Prag}|pw}} geworden. Mich erſtaunt nur, daß Du Dich ſonſt nicht wohler dort gefühlt haſt.
               Denn es ſoll eine ſehr ſchöne Stadt\oindex{Prag@\textbf{Prag}|pwv} ſein.\pend
           \pstart
           Für Deinen Bericht über das kleine \label{K_L02833-6v}\edtext{Fräulein\pwindex{Ziegler, Alice 1880-01-05 – Dezember 1943@\textsc{Ziegler, Alice} (1880-01-05 – Dezember 1943)|pwv}}{\lemma{\textnormal{\emph{Fräulein}}}\Cendnote{\textnormal{siehe Paul Goldmann an Arthur Schnitzler, 19. 11. [1897]}}}\label{K_L02833-6h} danke ich Dir von ganzem Herzen. Er hat mich ſehr nachdenklich geſtimmt.
               Deine Beobachtungen {\pb}ſind zweifelsohne richtig,
               Deine \strikeout{Schluſ\textcolor{gray}{ſe}} Schlüſſe nicht weniger. Es wäre vielleicht ſehr unklug von mir, wenn ich
               irgend etwas thäte. Ich werde auch wahrſcheinlich nichts thun. Aber anderſeits übt
               gerade dieſe halbe Kindlichkeit auf mich \strikeout{e\textcolor{gray}{×}} einen ungeheuren Reiz aus. Du meinſt, das ſei \textsc{Perversion}. Ich weiß es nicht, aber der Reiz beſteht. Und er wird
               hundertfach verſtärkt durch das Pariſ\oindex{Paris@\textbf{Paris}|pw}er Leben.
               Wenn man ſo Jahre lang mitten unter \label{K_L02833-7v}\edtext{\begin{otherlanguage}{french}Raffinement\end{otherlanguage}}{\lemma{\textnormal{\emph{Raffinement}}}\Cendnote{\textnormal{Fremdwort mit Ursprung im Französischen:
                  Feinheit}}}\label{K_L02833-7h} und Proſtitution gelebt hat (wie es das Loos des Fremden in \textsc{Paris\oindex{Paris@\textbf{Paris}|pw}} iſt), ſo bekommt man eine unendliche Sehnſucht nach {\pb}Einfachheit und Reinheit. Und wenn man außerdem noch
               zum poetiſchen Träumen \strikeout{aufgelegt}{ }\strikeout{iſt\textcolor{gray}{,} ſo} angelegt iſt, ſo liebt man
               die unfertigen Dinge. Die Poeſie beſteht darin, daß man den Dingen etwas hinzufügt.
               Das iſt der Reiz des halben Kindes für den Träumer, und darum bleibt \strikeout{\textcolor{gray}{×}\-\textcolor{gray}{×}} ihm die fertige Frau gleichgiltig. Nebenbei geſagt übrigens: Welche Frau iſt
               überhaupt fertig?\pend
           \pstart
           Bitte, liebſter Freund, ſchreib’ mir bald. In dieſer {\pb}Welt voll Feindſeligkeiten ſehne ich mich ſehr nach einem guten Worte von Dir.\pend
           \pstart
           Fragen, die beſonders zu beantworten wären: Was macht Deine Freundin\pwindex{Reinhard, Marie 1871-03-13 – 1899-03-18@\textsc{Reinhard, Marie} (1871-03-13 – 1899-03-18), \emph{Gesangspädagogin}|pwv}? \strikeout{Was} Wie ſteht es mit Deinem \label{K_L02833-8v}\edtext{neuen Stück\pwindex{Schnitzler, Arthur 15.05.1862 – 21.10.1931@\textsc{Schnitzler, Arthur} (15.05.1862 – 21.10.1931), \emph{Schriftsteller, Mediziner}!Vermaechtnis. Schauspiel in drei Akten1898@\strich\emph{Das Vermächtnis. Schauspiel in drei Akten} {[}1898{]}|pwv}}{\lemma{\textnormal{\emph{neuen Stück}}}\Cendnote{\textnormal{Schnitzler\pwindex{Schnitzler, Arthur 15.05.1862 – 21.10.1931@\textsc{Schnitzler, Arthur} (15.05.1862 – 21.10.1931), \emph{Schriftsteller, Mediziner}|pwk} arbeitete intensiv an dem
                  Schauspiel \emph{Das Vermächtnis}\pwindex{Schnitzler, Arthur 15.05.1862 – 21.10.1931@\textsc{Schnitzler, Arthur} (15.05.1862 – 21.10.1931), \emph{Schriftsteller, Mediziner}!Vermaechtnis. Schauspiel in drei Akten1898@\strich\emph{Das Vermächtnis. Schauspiel in drei Akten} {[}1898{]}|pwk}, hatte dabei
                  jedoch einige Schwierigkeiten, die er immer wieder im \emph{Tagebuch}\pwindex{Schnitzler, Arthur 15.05.1862 – 21.10.1931@\textsc{Schnitzler, Arthur} (15.05.1862 – 21.10.1931), \emph{Schriftsteller, Mediziner}!Tagebuch1981 – 2000@\strich\emph{Tagebuch} {[}1981 – 2000{]}|pwk} festhielt (vgl. z. B. 9. 12. 1897).}}}\label{K_L02833-8h}?
               Und was iſt mit dem Stück\pwindex{Burckhard, Max Eugen 14.07.1854 – 16.03.1912@\textsc{Burckhard, Max Eugen} (14.07.1854 – 16.03.1912), \emph{Schriftsteller, Wissenschaftler, Theaterleiter}! s Katherl. Volksstueck in fuenf Aufzuegen1897-11-25@\strich\emph{’s Katherl. Volksstück in fünf Aufzügen} {[}1897-11-25{]}|pwv} von
                  \textsc{Burckhardt\pwindex{Burckhard, Max Eugen 14.07.1854 – 16.03.1912@\textsc{Burckhard, Max Eugen} (14.07.1854 – 16.03.1912), \emph{Schriftsteller, Wissenschaftler, Theaterleiter}|pw}}, welches der alberne \textsc{Bahr\pwindex{Bahr, Hermann 19.07.1863 – 15.01.1934@\textsc{Bahr, Hermann} (19.07.1863 – 15.01.1934), \emph{Schriftsteller, Kritiker}|pw}} mit \textsc{Shakespeare\pwindex{Shakespeare, William 23.4.1564? – 03.05.1616@\textsc{Shakespeare, William} (23.4.1564? – 03.05.1616), \emph{Schauspieler, Schriftsteller}|pw}}{ }\label{K_L02833-9v}\edtext{vergleicht\pwindex{Bahr, Hermann 19.07.1863 – 15.01.1934@\textsc{Bahr, Hermann} (19.07.1863 – 15.01.1934), \emph{Schriftsteller, Kritiker}! s Katherl. (Volksstueck in fuenf Aufzuegen von Max Burckhard. Zum ersten Mal
                  aufgefuehrt im Raimundtheater am 25. November 1897.)1897-11-27@\strich\emph{’s Katherl. (Volksstück in fünf Aufzügen von Max Burckhard. Zum ersten Mal aufgeführt im Raimundtheater am 25. November 1897.)} {[}1897-11-27{]}|pwv}}{\lemma{\textnormal{\emph{vergleicht}}}\Cendnote{\textnormal{Hermann Bahr\pwindex{Bahr, Hermann 19.07.1863 – 15.01.1934@\textsc{Bahr, Hermann} (19.07.1863 – 15.01.1934), \emph{Schriftsteller, Kritiker}|pwk}: \emph{’s Katherl. (Volksstück in fünf Aufzügen von Max Burckhard.
                        Zum ersten Mal aufgeführt im Raimundtheater am 25. November 1897.)}\pwindex{Bahr, Hermann 19.07.1863 – 15.01.1934@\textsc{Bahr, Hermann} (19.07.1863 – 15.01.1934), \emph{Schriftsteller, Kritiker}! s Katherl. (Volksstueck in fuenf Aufzuegen von Max Burckhard. Zum ersten Mal
                  aufgefuehrt im Raimundtheater am 25. November 1897.)1897-11-27@\strich\emph{’s Katherl. (Volksstück in fünf Aufzügen von Max Burckhard. Zum ersten Mal aufgeführt im Raimundtheater am 25. November 1897.)} {[}1897-11-27{]}|pwk}. In:
                        \emph{Die Zeit}\pwindex{Zeit. Wiener Wochenschrift1894 – 1904@\emph{Die Zeit. Wiener Wochenschrift} {[}1894 – 1904{]}|pwk}, Bd. 13, Nr. 165, 27. 11. 1897, S. 141.}}}\label{K_L02833-9h}?\pend
           \pstart
           Sei von Herzen gegrüßt.\pend
           \pstart
           Dein treuer {\\[\baselineskip]}\spacefill\mbox{Paul Goldmann.}\pend
           \leftskip=0em{}\pstart
           \noindent{}Bitte, grüße doch auch einmal Frau \textsc{Altmann}\pwindex{Altmann, Emma 22.10.1849 – 31.12.1930@\textsc{Altmann, Emma} (22.10.1849 – 31.12.1930)|pwv} und deren {\pb}Söhne\pwindex{Altmann, Paul 25.01.1877 – 17.03.1953@\textsc{Altmann, Paul} (25.01.1877 – 17.03.1953), \emph{Angestellter}|pwv}\pwindex{Altmann, Carl 31.7.1838 – 28.1.1889@\textsc{Altmann, Carl} (31.7.1838 – 28.1.1889), \emph{Kaufmann}|pwv}, wenn Du
                  ſie ſiehſt.\pend
           
         
         \endnumbering\mylabel{h}\end{ledgroupsized}  \newcommand{\dateiname}{L02833}\newcommand{\titel}{Paul Goldmann an Arthur Schnitzler, 10. 12. [1897]}\newcommand{\editorInnen}{Martin Anton Müller und Laura Untner}%% latex-leseansicht-abspann.tex
%% Abspann für die Leseansicht.
%% Der Schalter \ifkorrekturansicht ist bereits durch den Vorspann gesetzt.

%% latex-abspann.tex
%% Gemeinsamer Abspann für Korrekturansicht und Leseansicht.
%% Setzt den Schalter \ifkorrekturansicht voraus (gesetzt in den
%% einbindenden Dateien latex-korrekturansicht-abspann.tex bzw.
%% latex-leseansicht-abspann.tex).
%% ---------------------------------------------------------------

\normalsize

% Das esempio-Environment wird nur in der Leseansicht benötigt
\ifkorrekturansicht\else
\newenvironment{esempio}[3]%
{
    \vspace{1.5ex}
    \rlap{\underline{#1}}
    \par
    \setlength{\parindent}{0cm}
    \nopagebreak
    \leftskip=#2cm
    \rightskip=#3cm
}
{
    \par
}
\fi

\doendnotes{C}
\bigskip
\vfill

\clearpage

\footnotesize

\ifkorrekturansicht
  \lohead{\textsc{register}}
\fi

% theindex-Environment neu definieren ohne reledmac
\makeatletter
\renewenvironment{theindex}{%
  \ifkorrekturansicht
    \section*{\indexname}%
  \else
    \subsubsection*{Index der erwähnten Entitäten}%
  \fi
  \setlength{\parindent}{0pt}%
  \setlength{\parskip}{0pt plus 0.3pt}%
  \let\item\@idxitem
}{%
  \ifkorrekturansicht\clearpage\fi
}
\makeatother

\IfFileExists{\jobname-pw.ind}{\input{\jobname-pw.ind}}{}

% Quellenangabe nur in der Leseansicht
\ifkorrekturansicht\else
% Fallback-Definitionen, falls die .tex-Datei \titel etc. nicht gesetzt hat
\providecommand{\titel}{}
\providecommand{\editorInnen}{}
\providecommand{\dateiname}{\jobname}

\vspace{3cm}

\vfill

\footnotesize
\textsc{Quelle}: \titel. Herausgegeben von {\editorInnen}. In: \emph{Arthur Schnitzler: Briefwechsel mit Autorinnen und Autoren}.
 Digitale Edition, https://schnitzler-briefe.acdh.oeaw.ac.at/{\dateiname}.html (Stand \today)
\fi

\end{document}


      