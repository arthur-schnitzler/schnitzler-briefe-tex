%% latex-leseansicht-vorspann.tex
%% Vorspann für die Leseansicht.
%% Lädt die gemeinsame Datei latex-vorspann.tex mit nicht gesetztem Schalter.

\newif\ifkorrekturansicht
\korrekturansichtfalse

\input{../tex-inputs/latex-vorspann}


\section[ Paul Goldmann an Arthur Schnitzler, 10. 12. {[}1897{]}]{L02833 Paul Goldmann an Arthur Schnitzler,  10. 12. [1897]}
\nopagebreak\mylabel{L02833v}
\rehead{ }\normalsize\beginnumbering\briefempfaengerindex{Schnitzler, Arthur@\textsc{Schnitzler, Arthur}!zzzGoldmann, Paul@\emph{von Paul Goldmann}!1897-12-101@{10. 12. [1897]}|(be}
\toendnotes[C]{\smallbreak\pagebreak[2]}
\correspDesc{Versand  durch Paul Goldmann am 10. 12. [1897] in Paris
\newline{}Erhalt  durch Arthur Schnitzler im Zeitraum [11. 12. 1897 – 15. 12. 1897?] in Wien}\toendnotes[C]{\smallbreak}
\Standort{DLA, A:Schnitzler, HS.NZ85.1.3167.}
\physDesc{Brief, 2 Blätter, 7 Seiten, 2961 Zeichen
\newline{}Handschrift: blaue Tinte, deutsche Kurrent
\newline{}Schnitzler: 1) mit Bleistift das Jahr »97« vermerkt  2) mit rotem Buntstift drei Unterstreichungen}\toendnotes[C]{\smallbreak}
\pstart
           {\pb}\textcolor{gray}{\textbf{\textbf{Frankfurter Zeitung\orgindex{Frankfurter Zeitung@Frankfurter Zeitung|pw}}}}\pend
           
\pstart
           \textcolor{gray}{\textbf{(\begin{otherlanguage}{french}Gazette de Francfort\end{otherlanguage}\orgindex{Frankfurter Zeitung@Frankfurter Zeitung|pw}).}}\pend
           
\pstart
           \textcolor{gray}{\textbf{\textbf{\begin{otherlanguage}{french}Fondateur M.\end{otherlanguage}{ }L. Sonnemann\pwindex{Sonnemann, Leopold 29.\,10.\,1831 Höchberg – 30.\,10.\,1909 Frankfurt am Main@\textsc{Sonnemann, Leopold} (29.\,10.\,1831 Höchberg – 30.\,10.\,1909 Frankfurt am Main), \emph{Journalist, Herausgeber}|pw}.}}}\pend
           
\pstart
           \begin{otherlanguage}{french}\textcolor{gray}{\textbf{Journal politique, financier,}}\end{otherlanguage}\pend
           
\pstart
           \begin{otherlanguage}{french}\textcolor{gray}{\textbf{commercial et littéraire.}}\end{otherlanguage}\pend
           
\pstart
           \begin{otherlanguage}{french}\textcolor{gray}{\textbf{\textbf{Paraissant trois fois par jour.}}}\end{otherlanguage}\hfill \textsc{Paris\oindex{Paris@\textbf{Paris}, \emph{Hauptstadt}|pw}}, 10. December.\pend
           
\pstart
           \begin{otherlanguage}{french}\textcolor{gray}{\textbf{\textbf{Bureau à Paris\oindex{Paris@\textbf{Paris}, \emph{Hauptstadt}|pw}}}}\end{otherlanguage}\pend
           
\pstart
           \begin{otherlanguage}{french}\textcolor{gray}{\textbf{\textbf{10 \so{Rue de la Bourse}\oindex{rue de la Bourse@\textbf{rue de la Bourse}, \emph{Straße}|pw}.}}}\end{otherlanguage}\pend
           
\pstart\center{}Mein lieber Freund,\pend\vspace{0.5em}
\pstart
           Endlich ein freier Augenblick! Ich habe eine Reihe furchtbar aufgeregter Tage hinter
               mir. Die Geſchichte fing an mit einem \label{K_L02833-1v}\edtext{Artikel\pwindex{Millevoye, Lucien 1.\,8.\,1850 Grenoble – 25.\,3.\,1918 Paris@\textsc{Millevoye, Lucien} (1.\,8.\,1850 Grenoble – 25.\,3.\,1918 Paris), \emph{Politiker, Journalist}!Aux Syndiqués de Francfort@\strich\emph{Aux Syndiqués de Francfort}|pwv} von \textsc{Millevoye\pwindex{Millevoye, Lucien 1.\,8.\,1850 Grenoble – 25.\,3.\,1918 Paris@\textsc{Millevoye, Lucien} (1.\,8.\,1850 Grenoble – 25.\,3.\,1918 Paris), \emph{Politiker, Journalist}|pw}}}{\lemma{\textnormal{\emph{Artikel von Millevoye}}}\Cendnote{\textnormal{Ende November und Anfang Dezember 1897
                  erschienen fast täglich Kommentare zur Affäre Dreyfus\pwindex{Dreyfus, Alfred 9.\,10.\,1859 Mulhouse – 12.\,7.\,1935 Paris@\textsc{Dreyfus, Alfred} (9.\,10.\,1859 Mulhouse – 12.\,7.\,1935 Paris), \emph{Militär}|pwk} von Lucien Millevoye\pwindex{Millevoye, Lucien 1.\,8.\,1850 Grenoble – 25.\,3.\,1918 Paris@\textsc{Millevoye, Lucien} (1.\,8.\,1850 Grenoble – 25.\,3.\,1918 Paris), \emph{Politiker, Journalist}|pwk} in
                  der von ihm geleiteten Zeitung \emph{La Patrie}\pwindex{Patrie. Organe de la défense nationale@\emph{La Patrie. Organe de la défense nationale}|pwk}. Goldmann\pwindex{Goldmann, Paul 31.\,1.\,1865 Breslau – 25.\,9.\,1935 Wien@\textsc{Goldmann, Paul} (31.\,1.\,1865 Breslau – 25.\,9.\,1935 Wien), \emph{Schriftsteller, Journalist}|pwk} bezog sich auf folgenden Artikel\pwindex{Millevoye, Lucien 1.\,8.\,1850 Grenoble – 25.\,3.\,1918 Paris@\textsc{Millevoye, Lucien} (1.\,8.\,1850 Grenoble – 25.\,3.\,1918 Paris), \emph{Politiker, Journalist}!Aux Syndiqués de Francfort@\strich\emph{Aux Syndiqués de Francfort}|pwkv}: Lucien Millevoye\pwindex{Millevoye, Lucien 1.\,8.\,1850 Grenoble – 25.\,3.\,1918 Paris@\textsc{Millevoye, Lucien} (1.\,8.\,1850 Grenoble – 25.\,3.\,1918 Paris), \emph{Politiker, Journalist}|pwk}: \emph{Aux Syndiqués de Francfort}\pwindex{Millevoye, Lucien 1.\,8.\,1850 Grenoble – 25.\,3.\,1918 Paris@\textsc{Millevoye, Lucien} (1.\,8.\,1850 Grenoble – 25.\,3.\,1918 Paris), \emph{Politiker, Journalist}!Aux Syndiqués de Francfort@\strich\emph{Aux Syndiqués de Francfort}|pwk}. In: \emph{La Patrie. Organe de la defense nationale}\pwindex{Patrie. Organe de la défense nationale@\emph{La Patrie. Organe de la défense nationale}|pwk}, Jg. 57,
                     Nr. 5, 4. 12. 1897, S. 1. Als Beilage ist
                  der Artikel\pwindex{Millevoye, Lucien 1.\,8.\,1850 Grenoble – 25.\,3.\,1918 Paris@\textsc{Millevoye, Lucien} (1.\,8.\,1850 Grenoble – 25.\,3.\,1918 Paris), \emph{Politiker, Journalist}!Aux Syndiqués de Francfort@\strich\emph{Aux Syndiqués de Francfort}|pwkv} nicht
                  erhalten.}}}\label{K_L02833-1}, der mich mit Koth bewarf. Ich lege ihn Dir bei, damit Du{ }ſiehſt,
                  \strikeout{in} welchen Ton die Polemik in dieſen heißen Tagen
               angenommen hat und was man{ }ſich Alles{ }ſagen laſſen muß, wenn man ruhig und beſcheiden
               für{ }ſeine Überzeugung eintritt. Sonntag kam der
                  \label{K_L02833-2v}\edtext{Einbruch}{\lemma{\textnormal{\emph{Einbruch}}}\Cendnote{\textnormal{Darüber wurde auch berichtet: [O. V.]: \emph{À la chambre}\pwindex{À la chambre@\emph{À la chambre}|pwk}. In: \emph{L’Express du Midi. Organe quotidien de Défense Sociale et Religieuse}\pwindex{Express du Midi. Organe quotidien de Défense Sociale et Religieuse@\emph{L’Express du Midi. Organe quotidien de Défense Sociale et Religieuse}|pwk},
                     Jg. 7, Nr. 2077, 7. 12. 1897,
                  S. [2].}}}\label{K_L02833-2}, von dem Du wohl in den Blättern geleſen haſt. Man hat mir
               meine Briefe geſtohlen, {\pb}Briefe von meiner Familie
               und von Dir. Wahrſcheinlich war der Einbruch eine verkleidete Hausſuchung. Irgend ein
               officieller Dummkopf hat vielleicht geglaubt, daß \substVorne{}\textsuperscript{\textcolor{gray}{×}}\substDazwischen{}e\substHinten{}r bei mir Documente zum Fall \textsc{Dreyfus\pwindex{Dreyfus, Alfred 9.\,10.\,1859 Mulhouse – 12.\,7.\,1935 Paris@\textsc{Dreyfus, Alfred} (9.\,10.\,1859 Mulhouse – 12.\,7.\,1935 Paris), \emph{Militär}|pw}} finden könnte oder \strikeout{doc} documentariſche Beweiſe
               für die Exiſtenz des famoſen »\label{K_L02833-3v}\edtext{Syndicats}{\lemma{\textnormal{\emph{Syndicats}}}\Cendnote{\textnormal{Bezug auf das
                  vermeintliche »Judensyndikat« hinter der Dreyfus\pwindex{Dreyfus, Alfred 9.\,10.\,1859 Mulhouse – 12.\,7.\,1935 Paris@\textsc{Dreyfus, Alfred} (9.\,10.\,1859 Mulhouse – 12.\,7.\,1935 Paris), \emph{Militär}|pwk}-Affäre (vgl. Emile Zola\pwindex{Zola, Émile 2.\,4.\,1840 Paris – 29.\,9.\,1902 ebd.@\textsc{Zola, Émile} (2.\,4.\,1840 Paris – 29.\,9.\,1902 ebd.), \emph{Schriftsteller, Journalist}|pwk}: \emph{Le Syndicat}\pwindex{Zola, Émile 2.\,4.\,1840 Paris – 29.\,9.\,1902 ebd.@\textsc{Zola, Émile} (2.\,4.\,1840 Paris – 29.\,9.\,1902 ebd.), \emph{Schriftsteller, Journalist}!Le Syndicat@\strich\emph{Le Syndicat}|pwk}. In: \emph{Le
                        Figaro}\pwindex{Le Figaro@\emph{Le Figaro}|pwk}, Jg. 43, Nr. 335, 1. 12. 1897,
                     S. 1).}}}\label{K_L02833-3}« (das nie exiſtirt hat). Tagelang hat{ }ſich hier die Preſſe
               mit mir beſchäftigt, und obwohl kein böſes Wort gegen mich gefallen iſt,{ }ſo iſt es
               doch unheimlich, als Deutſch\oindex{Deutschland@\textbf{Deutschland}|pwv}er
               in{ }ſo leidenſchaftlich bewegter Zeit im Mittelpunkt des Intereſſes zu{ }ſtehen.\pend
           
\pstart
           {\pb}Endlich alſo kann ich ein wenig aufathmen, und
               endlich kann ich Dir Deinen{ }ſo lieben und{ }ſchönen Brief beantworten. Ich habe mich
               von Herzen über Deine \label{K_L02833-4v}\edtext{Prag\oindex{Prag@\textbf{Prag}, \emph{Land}|pw}er Erfolge\pwindex{Schnitzler, Arthur 15.\,5.\,1862 Wien – 21.\,10.\,1931 ebd.@\textsc{Schnitzler, Arthur} (15.\,5.\,1862 Wien – 21.\,10.\,1931 ebd.), \emph{Schriftsteller, Mediziner}!Freiwild. Schauspiel in 3 Akten@\strich\emph{Freiwild. Schauspiel in 3 Akten}|pwv}\pwindex{Schnitzler, Arthur 15.\,5.\,1862 Wien – 21.\,10.\,1931 ebd.@\textsc{Schnitzler, Arthur} (15.\,5.\,1862 Wien – 21.\,10.\,1931 ebd.), \emph{Schriftsteller, Mediziner}!Toten schweigen@\strich\emph{Die Toten schweigen}|pwv}\pwindex{Schnitzler, Arthur 15.\,5.\,1862 Wien – 21.\,10.\,1931 ebd.@\textsc{Schnitzler, Arthur} (15.\,5.\,1862 Wien – 21.\,10.\,1931 ebd.), \emph{Schriftsteller, Mediziner}!Weihnachts-Einkäufe@\strich\emph{Weihnachts-Einkäufe}|pwv}}{\lemma{\textnormal{\emph{Prager Erfolge}}}\Cendnote{\textnormal{Siehe XXXX Auszeichnungsfehler: Dokument L02834 nicht gefunden.
               }}}\label{K_L02833-4} gefreut. Es iſt gut, daß das Alles noch vor die Zeit des \label{K_L02833-5v}\edtext{Aufruhrs}{\lemma{\textnormal{\emph{Aufruhrs}}}\Cendnote{\textnormal{Auslöser waren gewaltvolle Proteste als Reaktion auf die Badeni\pwindex{Badeni, Kasimir Felix 14.\,10.\,1846 Galizien – 9.\,7.\,1909 ebd.@\textsc{Badeni, Kasimir Felix} (14.\,10.\,1846 Galizien – 9.\,7.\,1909 ebd.), \emph{Politiker, Jurist}|pwkv}sche
                  Sprachverordnung, die sich von Ende November bis
                  Anfang Dezember 1897 erstreckten. Auch Schnitzler notierte die »Unruhen,
                     politischer Natur« am 28. 11. 1897 – vor seiner Abreise aus Prag – im
                     \emph{Tagebuch}\pwindex{Schnitzler, Arthur 15.\,5.\,1862 Wien – 21.\,10.\,1931 ebd.@\textsc{Schnitzler, Arthur} (15.\,5.\,1862 Wien – 21.\,10.\,1931 ebd.), \emph{Schriftsteller, Mediziner}!Tagebuch@\strich\emph{Tagebuch}|pwk}.}}}\label{K_L02833-5} gefallen iſt,{ }ſonſt wäre
               es für Dich auch recht ungemüthlich in \textsc{Prag\oindex{Prag@\textbf{Prag}, \emph{Land}|pw}} geworden. Mich erſtaunt nur, daß Du Dich{ }ſonſt nicht wohler dort gefühlt haſt.
               Denn es{ }ſoll eine{ }ſehr{ }ſchöne Stadt\oindex{Prag@\textbf{Prag}, \emph{Land}|pwv}{ }ſein.\pend
           
\pstart
           Für Deinen Bericht über das kleine \label{K_L02833-6v}\edtext{Fräulein\pwindex{Ziegler, Alice 5.\,1.\,1880 Prag – Dezember 1943 Konzentrationslager Auschwitz-Birkenau@\textsc{Ziegler, Alice} (5.\,1.\,1880 Prag – Dezember 1943 Konzentrationslager Auschwitz-Birkenau)|pwv}}{\lemma{\textnormal{\emph{Fräulein}}}\Cendnote{\textnormal{Siehe XXXX Auszeichnungsfehler: Dokument L02831 nicht gefunden.
               }}}\label{K_L02833-6} danke ich Dir von ganzem Herzen. Er hat mich{ }ſehr nachdenklich geſtimmt.
               Deine Beobachtungen {\pb}ſind zweifelsohne richtig,
               Deine \strikeout{Schluſ\textcolor{gray}{ſe}} Schlüſſe nicht weniger. Es wäre vielleicht{ }ſehr unklug von mir, wenn ich
               irgend etwas thäte. Ich werde auch wahrſcheinlich nichts thun. Aber anderſeits übt
               gerade dieſe halbe Kindlichkeit auf mich \strikeout{e\textcolor{gray}{×}} einen ungeheuren Reiz aus. Du meinſt, das{ }ſei \textsc{Perversion}. Ich weiß es nicht, aber der Reiz beſteht. Und er wird
               hundertfach verſtärkt durch das Pariſ\oindex{Paris@\textbf{Paris}, \emph{Hauptstadt}|pw}er Leben.
               Wenn man{ }ſo Jahre lang mitten unter \label{K_L02833-7v}\edtext{\begin{otherlanguage}{french}Raffinement\end{otherlanguage}}{\lemma{\textnormal{\emph{Raffinement}}}\Cendnote{\textnormal{Fremdwort mit Ursprung im Französischen:
                  Feinheit}}}\label{K_L02833-7} und Proſtitution gelebt hat (wie es das Loos des Fremden in \textsc{Paris\oindex{Paris@\textbf{Paris}, \emph{Hauptstadt}|pw}} iſt),{ }ſo bekommt man eine unendliche Sehnſucht nach {\pb}Einfachheit und Reinheit. Und wenn man außerdem noch
               zum poetiſchen Träumen \strikeout{aufgelegt}{ }\strikeout{iſt\textcolor{gray}{,}{ }ſo} angelegt iſt,{ }ſo liebt man
               die unfertigen Dinge. Die Poeſie beſteht darin, daß man den Dingen etwas hinzufügt.
               Das iſt der Reiz des halben Kindes für den Träumer, und darum bleibt \strikeout{\textcolor{gray}{×}\-\textcolor{gray}{×}} ihm die fertige Frau gleichgiltig. Nebenbei geſagt übrigens: Welche Frau iſt
               überhaupt fertig?\pend
           
\pstart
           Bitte, liebſter Freund,{ }ſchreib’ mir bald. In dieſer {\pb}Welt voll Feindſeligkeiten{ }ſehne ich mich{ }ſehr nach einem guten Worte von Dir.\pend
           
\pstart
           Fragen, die beſonders zu beantworten wären: Was macht Deine Freundin\pwindex{Reinhard, Marie 13.\,3.\,1871 Wien – 18.\,3.\,1899 ebd.@\textsc{Reinhard, Marie} (13.\,3.\,1871 Wien – 18.\,3.\,1899 ebd.), \emph{Gesangspädagogin}|pwv}? \strikeout{Was} Wie{ }ſteht es mit Deinem \label{K_L02833-8v}\edtext{neuen Stück\pwindex{Schnitzler, Arthur 15.\,5.\,1862 Wien – 21.\,10.\,1931 ebd.@\textsc{Schnitzler, Arthur} (15.\,5.\,1862 Wien – 21.\,10.\,1931 ebd.), \emph{Schriftsteller, Mediziner}!Vermächtnis. Schauspiel in drei Akten@\strich\emph{Das Vermächtnis. Schauspiel in drei Akten}|pwv}}{\lemma{\textnormal{\emph{neuen Stück}}}\Cendnote{\textnormal{Schnitzler arbeitete intensiv an dem
                  Schauspiel \emph{Das Vermächtnis}\pwindex{Schnitzler, Arthur 15.\,5.\,1862 Wien – 21.\,10.\,1931 ebd.@\textsc{Schnitzler, Arthur} (15.\,5.\,1862 Wien – 21.\,10.\,1931 ebd.), \emph{Schriftsteller, Mediziner}!Vermächtnis. Schauspiel in drei Akten@\strich\emph{Das Vermächtnis. Schauspiel in drei Akten}|pwk}, hatte dabei
                  jedoch einige Schwierigkeiten, die er immer wieder im \emph{Tagebuch}\pwindex{Schnitzler, Arthur 15.\,5.\,1862 Wien – 21.\,10.\,1931 ebd.@\textsc{Schnitzler, Arthur} (15.\,5.\,1862 Wien – 21.\,10.\,1931 ebd.), \emph{Schriftsteller, Mediziner}!Tagebuch@\strich\emph{Tagebuch}|pwk} festhielt (vgl. z. B. 9. 12. 1897).}}}\label{K_L02833-8}?
               Und was iſt mit dem Stück\pwindex{Burckhard, Max Eugen 14.\,7.\,1854 Korneuburg – 16.\,3.\,1912 Wien@\textsc{Burckhard, Max Eugen} (14.\,7.\,1854 Korneuburg – 16.\,3.\,1912 Wien), \emph{Schriftsteller, Rechtswissenschaftler, Theaterleiter}!’s Katherl. Volksstück in fünf Aufzügen@\strich\emph{’s Katherl. Volksstück in fünf Aufzügen}|pwv} von
                  \textsc{Burckhardt\pwindex{Burckhard, Max Eugen 14.\,7.\,1854 Korneuburg – 16.\,3.\,1912 Wien@\textsc{Burckhard, Max Eugen} (14.\,7.\,1854 Korneuburg – 16.\,3.\,1912 Wien), \emph{Schriftsteller, Rechtswissenschaftler, Theaterleiter}|pw}}, welches der alberne \textsc{Bahr\pwindex{Bahr, Hermann 19.\,7.\,1863 Linz – 15.\,1.\,1934 München@\textsc{Bahr, Hermann} (19.\,7.\,1863 Linz – 15.\,1.\,1934 München), \emph{Schriftsteller, Kritiker}|pw}} mit \textsc{Shakespeare\pwindex{Shakespeare, William 23.\,4.\,1564? Stratford-upon-Avon – 3.\,5.\,1616 ebd.@\textsc{Shakespeare, William} (23.\,4.\,1564? Stratford-upon-Avon – 3.\,5.\,1616 ebd.), \emph{Schauspieler, Dramatiker}|pw}}{ }\label{K_L02833-9v}\edtext{vergleicht\pwindex{Bahr, Hermann 19.\,7.\,1863 Linz – 15.\,1.\,1934 München@\textsc{Bahr, Hermann} (19.\,7.\,1863 Linz – 15.\,1.\,1934 München), \emph{Schriftsteller, Kritiker}!’s Katherl. (Volksstück in fünf Aufzügen von Max Burckhard. Zum ersten Mal aufgeführt im Raimundtheater am 25. November 1897.)@\strich\emph{’s Katherl. (Volksstück in fünf Aufzügen von Max Burckhard. Zum ersten Mal aufgeführt im Raimundtheater am 25. November 1897.)}|pwv}}{\lemma{\textnormal{\emph{vergleicht}}}\Cendnote{\textnormal{Hermann Bahr\pwindex{Bahr, Hermann 19.\,7.\,1863 Linz – 15.\,1.\,1934 München@\textsc{Bahr, Hermann} (19.\,7.\,1863 Linz – 15.\,1.\,1934 München), \emph{Schriftsteller, Kritiker}|pwk}: \emph{’s Katherl. (Volksstück in fünf Aufzügen von Max Burckhard.
                        Zum ersten Mal aufgeführt im Raimundtheater am 25. November 1897)}\pwindex{Bahr, Hermann 19.\,7.\,1863 Linz – 15.\,1.\,1934 München@\textsc{Bahr, Hermann} (19.\,7.\,1863 Linz – 15.\,1.\,1934 München), \emph{Schriftsteller, Kritiker}!’s Katherl. (Volksstück in fünf Aufzügen von Max Burckhard. Zum ersten Mal aufgeführt im Raimundtheater am 25. November 1897.)@\strich\emph{’s Katherl. (Volksstück in fünf Aufzügen von Max Burckhard. Zum ersten Mal aufgeführt im Raimundtheater am 25. November 1897.)}|pwk}. In:
                        \emph{Die Zeit}\pwindex{Zeit. Wiener Wochenschrift@\emph{Die Zeit. Wiener Wochenschrift}|pwk}, Bd. 13, Nr. 165, 27. 11. 1897, S. 141.}}}\label{K_L02833-9}?\pend
           
\pstart
           Sei von Herzen gegrüßt.\pend
           
\pstart
           Dein treuer {\\[\baselineskip]}\spacefill\mbox{Paul Goldmann.}\pend
           \leftskip=0em{}
\pstart
           \noindent{}Bitte, grüße doch auch einmal Frau \textsc{Altmann}\pwindex{Altmann, Emma 22.\,10.\,1849 Budapest – 31.\,12.\,1930 Wien@\textsc{Altmann, Emma} (22.\,10.\,1849 Budapest – 31.\,12.\,1930 Wien)|pwv} und deren {\pb}Söhne\pwindex{Altmann, Paul 25.\,1.\,1877 Budapest – 17.\,3.\,1953 Wien@\textsc{Altmann, Paul} (25.\,1.\,1877 Budapest – 17.\,3.\,1953 Wien), \emph{Angestellter}|pwv}\pwindex{Altmann, Carl 31.\,7.\,1838 – 28.\,1.\,1889 Wien@\textsc{Altmann, Carl} (31.\,7.\,1838 – 28.\,1.\,1889 Wien), \emph{Kaufmann}|pwv}, wenn Du{ }ſie{ }ſiehſt.\pend
           \selectlanguage{ngerman}\endnumbering\briefempfaengerindex{Schnitzler, Arthur@\textsc{Schnitzler, Arthur}!zzzGoldmann, Paul@\emph{von Paul Goldmann}!1897-12-101@{10. 12. [1897]}|)be}\mylabel{L02833h}  \newcommand{\dateiname}{L02833}\newcommand{\titel}{Paul Goldmann an Arthur Schnitzler, 10. 12. [1897]}\newcommand{\editorInnen}{Martin Anton Müller und Laura Untner}%% latex-leseansicht-abspann.tex
%% Abspann für die Leseansicht.
%% Der Schalter \ifkorrekturansicht ist bereits durch den Vorspann gesetzt.

%% latex-abspann.tex
%% Gemeinsamer Abspann für Korrekturansicht und Leseansicht.
%% Setzt den Schalter \ifkorrekturansicht voraus (gesetzt in den
%% einbindenden Dateien latex-korrekturansicht-abspann.tex bzw.
%% latex-leseansicht-abspann.tex).
%% ---------------------------------------------------------------

\normalsize

% Das esempio-Environment wird nur in der Leseansicht benötigt
\ifkorrekturansicht\else
\newenvironment{esempio}[3]%
{
    \vspace{1.5ex}
    \rlap{\underline{#1}}
    \par
    \setlength{\parindent}{0cm}
    \nopagebreak
    \leftskip=#2cm
    \rightskip=#3cm
}
{
    \par
}
\fi

\doendnotes{C}
\bigskip
\vfill

\clearpage

\footnotesize

\ifkorrekturansicht
  \lohead{\textsc{register}}
\fi

% theindex-Environment neu definieren ohne reledmac
\makeatletter
\renewenvironment{theindex}{%
  \ifkorrekturansicht
    \section*{\indexname}%
  \else
    \subsubsection*{Index der erwähnten Entitäten}%
  \fi
  \setlength{\parindent}{0pt}%
  \setlength{\parskip}{0pt plus 0.3pt}%
  \let\item\@idxitem
}{%
  \ifkorrekturansicht\clearpage\fi
}
\makeatother

\IfFileExists{\jobname-pw.ind}{\input{\jobname-pw.ind}}{}

% Quellenangabe nur in der Leseansicht
\ifkorrekturansicht\else
% Fallback-Definitionen, falls die .tex-Datei \titel etc. nicht gesetzt hat
\providecommand{\titel}{}
\providecommand{\editorInnen}{}
\providecommand{\dateiname}{\jobname}

\vspace{3cm}

\vfill

\footnotesize
\textsc{Quelle}: \titel. Herausgegeben von {\editorInnen}. In: \emph{Arthur Schnitzler: Briefwechsel mit Autorinnen und Autoren}.
 Digitale Edition, https://schnitzler-briefe.acdh.oeaw.ac.at/{\dateiname}.html (Stand \today)
\fi

\end{document}


