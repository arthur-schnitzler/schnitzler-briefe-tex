%% latex-leseansicht-vorspann.tex
%% Vorspann für die Leseansicht.
%% Lädt die gemeinsame Datei latex-vorspann.tex mit nicht gesetztem Schalter.

\newif\ifkorrekturansicht
\korrekturansichtfalse

\input{../tex-inputs/latex-vorspann}


\section[Fedor Mamroth an Arthur Schnitzler, 17. 11. 1892]{L00135 Fedor Mamroth an Arthur Schnitzler, 17. 11. 1892}
\nopagebreak\mylabel{L00135v}
\rehead{ }\normalsize\beginnumbering\briefempfaengerindex{Schnitzler, Arthur@\textsc{Schnitzler, Arthur}!zzzMamroth, Fedor@\emph{von Fedor Mamroth}!1892-11-171@{17. 11. 1892}|(be}
\toendnotes[C]{\smallbreak\pagebreak[2]}
\correspDesc{Versand  durch Fedor Mamroth am 17. 11. 1892 in Frankfurt am Main
\newline{}Erhalt  durch Arthur Schnitzler im Zeitraum [18. 11. 1892 –
            22. 11. 1892?] in Wien}\toendnotes[C]{\smallbreak}
\Standort{CUL, Schnitzler, B 68.}
\physDesc{Brief, 1 Blatt, 2 Seiten, 1378 Zeichen
\newline{}Handschrift: blaue Tinte, deutsche Kurrent
\newline{}Schnitzler: 1) mit Bleistift nummeriert: »3.«  2) mit rotem Buntstift eine Unterstreichung}\toendnotes[C]{\smallbreak}
\pstart
           {\pb}\textcolor{gray}{\textbf{\textsc{Frankfurter Zeitung}}}{\\}\textsc{\textcolor{gray}{\textbf{und}}}{\\}\textcolor{gray}{\textbf{\textsc{Handelsblatt.}}}\orgindex{Frankfurter Zeitung@Frankfurter Zeitung|pw}\pend
           
\pstart
           
\pstart
           \textcolor{gray}{\textbf{\textsc{Redaction.\footnote{\noindent{}\textcolor{gray}{\textbf{\textsc{Für die Redaktion bestimmte Briefe und Sendungen wolle
                          man \so{nicht} an die Person eines Redakteurs,
                          sondern stets \textbf{an die Redaktion der Frankfurter
                            Zeitung} adressiren}}}.}}}}\pend
           
\pstart
           \raggedleft{}\textcolor{gray}{\textbf{\textsc{Frankfurt a. M.\oindex{Frankfurt am Main@\textbf{Frankfurt am Main}, \emph{Hauptstadt}|pw},}}}{ }17. Novbr. \textsc{\textcolor{gray}{\textbf{189}}}2\pend
           \pend
           
\pstart
           \textcolor{gray}{\textbf{\textsc{Telegramm-Adresse:}}}\pend
           
\pstart
           \textcolor{gray}{\textbf{\textsc{Zeitung Frankfurt Main.}}}\pend
           
\pstart{}Sehr verehrter Herr Doctor!\pend\vspace{0.5em}
\pstart
           Wollte ich mein langes u.{ }ſcheinbar{ }ſo unartiges Stillſchweigen zu erklären u. zu
          entſchuldigen{ }ſuchen,{ }ſo würde ich{ }ſoviel Zeit u. Energie dazu brauchen, daß gleich wieder
          die Exiſtenz \uline{dieſes} Briefes bedroht wäre. Begnügen Sie
          Sich deshalb mit der Verſicherung meiner warmen Sympathie u. meiner herzlichen
          Ergebenheit. Es ging nicht anders u. wenn Sie mich umbringen: In Angelegenheit des »Märchen\pwindex{Schnitzler, Arthur 15. 5. 1862 Wien – 21. 10. 1931 ebd.@\textsc{Schnitzler, Arthur} (15. 5. 1862 Wien – 21. 10. 1931 ebd.), \emph{Schriftsteller, Mediziner}!Märchen. Schauspiel in drei Aufzügen@\strich\emph{Das Märchen. Schauspiel in drei Aufzügen}|pw}«{ }ſind mir die Hände gebunden; ich habe (außer{ }ſchlechten) keinerlei Beziehungen zur hieſigen Theaterleitung, und überdies bin ich der
          ungeſchickteſte Menſch, wenn es darauf ankommt, mir und meinen Freunden zu nützen. Dieſes
          Talent muß man mit auf die Welt bringen wie der impertinente Burſche Herr Lothar\pwindex{Lothar, Rudolf 23.\,2.\,1865 Budapest – 2.\,10.\,1943 ebd.@\textsc{Lothar, Rudolf} (23.\,2.\,1865 Budapest – 2.\,10.\,1943 ebd.), \emph{Schriftsteller, Journalist, Theaterdirektor}|pw}, der{ }ſich \label{K_L00135-1v}\edtext{jüngſthin}{\lemma{\textnormal{\emph{jüngsthin}}}\Cendnote{\textnormal{Die Uraufführung von \emph{Cäsar
                Borgia’s Ende}\pwindex{Lothar, Rudolf 23.\,2.\,1865 Budapest – 2.\,10.\,1943 ebd.@\textsc{Lothar, Rudolf} (23.\,2.\,1865 Budapest – 2.\,10.\,1943 ebd.), \emph{Schriftsteller, Journalist, Theaterdirektor}!Cäsar Borgia’s Ende@\strich\emph{Cäsar Borgia’s Ende}|pwk}\eventindex{Frankfurter Stadttheater@\textbf{Frankfurter Stadttheater}!Uraufführung von Cäsar Borgia’s Ende, 12.11.1892@Uraufführung von Cäsar Borgia’s Ende, 12.11.1892|pwk} fand am 12. 11. 1892 im örtlichen \emph{Schauspielhaus}\orgindex{Frankfurter Stadttheater@Frankfurter Stadttheater|pwk} statt.}}}\label{K_L00135-1} von hier aus
          inſcenierte.\pend
           
\pstart
           Die neuen Dialoge\pwindex{Schnitzler, Arthur 15. 5. 1862 Wien – 21. 10. 1931 ebd.@\textsc{Schnitzler, Arthur} (15. 5. 1862 Wien – 21. 10. 1931 ebd.), \emph{Schriftsteller, Mediziner}!Anatol@\strich\emph{Anatol}|pwv}{ }ſandte ich dem Berlin\oindex{Berlin@\textbf{Berlin}, \emph{Hauptstadt}|pw}er Herrn\pwindex{Sack, Eduard 31.\,8.\,1831 Dunajek – 20.\,5.\,1908 Frankfurt am Main@\textsc{Sack, Eduard} (31.\,8.\,1831 Dunajek – 20.\,5.\,1908 Frankfurt am Main), \emph{Journalist, Pädagoge}|pwv}, der in neueſter Zeit bei uns{ }ſchöngeiſtige Literatur beſpricht, mit warmer Empfehlung. Jetzt wollen wir{ }ſehen, was
            \label{K_L00135-2v}\edtext{geſchieht}{\lemma{\textnormal{\emph{geschieht}}}\Cendnote{\textnormal{Eine Rezension von \emph{Anatol}\pwindex{Schnitzler, Arthur 15. 5. 1862 Wien – 21. 10. 1931 ebd.@\textsc{Schnitzler, Arthur} (15. 5. 1862 Wien – 21. 10. 1931 ebd.), \emph{Schriftsteller, Mediziner}!Anatol@\strich\emph{Anatol}|pwk}
            dürfte nicht erschienen sein.}}}\label{K_L00135-2}. Die Novelle\pwindex{Schnitzler, Arthur 15. 5. 1862 Wien – 21. 10. 1931 ebd.@\textsc{Schnitzler, Arthur} (15. 5. 1862 Wien – 21. 10. 1931 ebd.), \emph{Schriftsteller, Mediziner}!Sterben. Novelle@\strich\emph{Sterben. Novelle}|pwv}{ }ſchicken Sie mir gefälligſt, wenn Sie{ }ſich jeder
          Alluſion {\pb}auf das Gerücht, wonach es zweierlei
          Menſchen auf der Welt gebe, enthalten haben. Nein,{ }ſchicken Sie{ }ſie mir in jedem Falle,
          ich bin neugierig darnach u. verſpreche Ihnen, die Arbeit \uline{bald} zu leſen.\pend
           
\pstart
           Leben Sie wohl,{ }ſehr verehrter Herr Doctor,{ }ſeien Sie herzlichſt gegrüßt u. entſchuldigen
          Sie die innere u. äußere Müdigkeit dieſer Zeilen.\pend
           
\pstart
           Ihr{\\[\baselineskip]}ergebener{\\[\baselineskip]}\spacefill\mbox{FMamroth}\pend
           \leftskip=0em{}\selectlanguage{ngerman}\endnumbering\briefempfaengerindex{Schnitzler, Arthur@\textsc{Schnitzler, Arthur}!zzzMamroth, Fedor@\emph{von Fedor Mamroth}!1892-11-171@{17. 11. 1892}|)be}\mylabel{L00135h}  \newcommand{\dateiname}{L00135}\newcommand{\titel}{Fedor Mamroth an Arthur Schnitzler, 17. 11. 1892}\newcommand{\editorInnen}{Martin Anton Müller und Gerd-Hermann Susen}%% latex-leseansicht-abspann.tex
%% Abspann für die Leseansicht.
%% Der Schalter \ifkorrekturansicht ist bereits durch den Vorspann gesetzt.

%% latex-abspann.tex
%% Gemeinsamer Abspann für Korrekturansicht und Leseansicht.
%% Setzt den Schalter \ifkorrekturansicht voraus (gesetzt in den
%% einbindenden Dateien latex-korrekturansicht-abspann.tex bzw.
%% latex-leseansicht-abspann.tex).
%% ---------------------------------------------------------------

\normalsize

% Das esempio-Environment wird nur in der Leseansicht benötigt
\ifkorrekturansicht\else
\newenvironment{esempio}[3]%
{
    \vspace{1.5ex}
    \rlap{\underline{#1}}
    \par
    \setlength{\parindent}{0cm}
    \nopagebreak
    \leftskip=#2cm
    \rightskip=#3cm
}
{
    \par
}
\fi

\doendnotes{C}
\bigskip
\vfill

\clearpage

\footnotesize

\ifkorrekturansicht
  \lohead{\textsc{register}}
\fi

% theindex-Environment neu definieren ohne reledmac
\makeatletter
\renewenvironment{theindex}{%
  \ifkorrekturansicht
    \section*{\indexname}%
  \else
    \subsubsection*{Index der erwähnten Entitäten}%
  \fi
  \setlength{\parindent}{0pt}%
  \setlength{\parskip}{0pt plus 0.3pt}%
  \let\item\@idxitem
}{%
  \ifkorrekturansicht\clearpage\fi
}
\makeatother

\IfFileExists{\jobname-pw.ind}{\input{\jobname-pw.ind}}{}

% Quellenangabe nur in der Leseansicht
\ifkorrekturansicht\else
% Fallback-Definitionen, falls die .tex-Datei \titel etc. nicht gesetzt hat
\providecommand{\titel}{}
\providecommand{\editorInnen}{}
\providecommand{\dateiname}{\jobname}

\vspace{3cm}

\vfill

\footnotesize
\textsc{Quelle}: \titel. Herausgegeben von {\editorInnen}. In: \emph{Arthur Schnitzler: Briefwechsel mit Autorinnen und Autoren}.
 Digitale Edition, https://schnitzler-briefe.acdh.oeaw.ac.at/{\dateiname}.html (Stand \today)
\fi

\end{document}


