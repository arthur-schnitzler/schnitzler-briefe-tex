%% latex-korrekturansicht-vorspann.tex
%% Vorspann für die Korrekturansicht.
%% Lädt die gemeinsame Datei latex-vorspann.tex mit gesetztem Schalter.

\newif\ifkorrekturansicht
\korrekturansichttrue

\input{../tex-inputs/latex-vorspann}


\section[Fedor Mamroth an Arthur Schnitzler, 17. 11. 1892]{L00135 Fedor Mamroth an Arthur Schnitzler, 17. 11. 1892}
\nopagebreak\mylabel{L00135v}
\rehead{ }\normalsize\beginnumbering\briefempfaengerindex{Schnitzler, Arthur@\textsc{Schnitzler, Arthur}!zzzMamroth, Fedor@\emph{von Fedor Mamroth}!1892-11-171@{17. 11. 1892}|(be}
\toendnotes[C]{\smallbreak\pagebreak[2]}\Standort{CUL, Schnitzler, B 68.}
\physDesc{Brief, 1 Blatt, 2 Seiten, 1377 Zeichen
\newline{}Handschrift: blaue Tinte, deutsche Kurrent
\newline{}Schnitzler: 1) mit Bleistift nummeriert: »3.«  2) mit rotem Buntstift eine Unterstreichung}\toendnotes[C]{\smallbreak}
\pstart
           {\pb}\textcolor{gray}{\textbf{\textsc{Frankfurter Zeitung}}}{\\}\textsc{\textcolor{gray}{\textbf{und}}}{\\}\textcolor{gray}{\textbf{\textsc{Handelsblatt.}}}\orgindex{Frankfurter Zeitung@Frankfurter Zeitung|pw}\pend
           
\pstart
           
\pstart
           \textcolor{gray}{\textbf{\textsc{Redaction.\noindent{}\textcolor{gray}{\textbf{\textsc{Für die Redaktion bestimmte Briefe und Sendungen wolle
                          man \so{nicht} an die Person eines Redakteurs,
                          sondern stets \textbf{an die Redaktion der Frankfurter
                            Zeitung} adressiren}}}.}}}\pend
           
\pstart
           \raggedleft{}\textcolor{gray}{\textbf{\textsc{Frankfurt a. M.\oindex{Frankfurt am Main@\textbf{Frankfurt am Main}, \emph{P.PPLA3}|pw},}}}{ }17. Novbr. \textsc{\textcolor{gray}{\textbf{189}}}2\pend
           \pend
           
\pstart
           \textcolor{gray}{\textbf{\textsc{Telegramm-Adresse:}}}\pend
           
\pstart
           \textcolor{gray}{\textbf{\textsc{Zeitung Frankfurt Main.}}}\pend
           
\pstart{}Sehr verehrter Herr Doctor!\pend\vspace{0.5em}
\pstart
           Wollte ich mein langes u. ſcheinbar ſo unartiges Stillſchweigen zu erklären u. zu
          entſchuldigen ſuchen, ſo würde ich ſoviel Zeit u. Energie dazu brauchen, daß gleich wieder
          die Exiſtenz \uline{dieſes} Briefes bedroht wäre. Begnügen Sie
          Sich deshalb mit der Verſicherung meiner warmen Sympathie u. meiner herzlichen
          Ergebenheit. Es ging nicht anders u. wenn Sie mich umbringen: In Angelegenheit des »Märchen\pwindex{Maerchen. Schauspiel in drei Aufzuegen@\emph{Das Märchen. Schauspiel in drei Aufzügen}|pw}« ſind mir die Hände gebunden; ich habe (außer
          ſchlechten) keinerlei Beziehungen zur hieſigen Theaterleitung, und überdies bin ich der
          ungeſchickteſte Menſch, wenn es darauf ankommt, mir und meinen Freunden zu nützen. Dieſes
          Talent muß man mit auf die Welt bringen wie der impertinente Burſche Herr Lothar\pwindex{Lothar, Rudolf 23.2.1865 – 2.10.1943@\textsc{Lothar, Rudolf} (23.2.1865 – 2.10.1943), \emph{Schriftsteller/Schriftstellerin, Journalist/Journalistin, Theaterdirektor/Theaterdirektorin}|pw}, der ſich \label{K_L00135-1v}\edtext{jüngſthin}{\lemma{\textnormal{\emph{jüngſthin}}}\Cendnote{\textnormal{Die \emph{Uraufführung von \emph{Cäsar
                Borgia’s Ende}\pwindex{Caesar Borgia s Ende@\emph{Cäsar Borgia’s Ende}|pwk}}\eventindex{Frankfurter Stadttheater@\textbf{Frankfurter Stadttheater}!Urauffuehrung von Caesar Borgia s Ende, 12.11.1892@Uraufführung von Cäsar Borgia’s Ende, 12.11.1892|pwk} fand am 12. 11. 1892 im örtlichen \emph{Schauspielhaus}XXXX ORGangabe fehlt statt.}}}\label{K_L00135-1} von hier aus
          inſcenierte.\pend
           
\pstart
           Die neuen Dialoge\pwindex{Anatol@\emph{Anatol}|pwv}{ }ſandte ich dem Berlin\oindex{Berlin@\textbf{Berlin}, \emph{P.PPLC}|pw}er Herrn\pwindex{Sack, Eduard 31.8.1831 – 20.5.1908@\textsc{Sack, Eduard} (31.8.1831 – 20.5.1908), \emph{Journalist/Journalistin, Pädagoge/Pädagogin}|pwv}, der in neueſter Zeit bei uns
          ſchöngeiſtige Literatur beſpricht, mit warmer Empfehlung. Jetzt wollen wir ſehen, was
            \label{K_L00135-2v}\edtext{geſchieht}{\lemma{\textnormal{\emph{geſchieht}}}\Cendnote{\textnormal{Eine Rezension von \emph{Anatol}\pwindex{Anatol@\emph{Anatol}|pwk}
            dürfte nicht erschienen sein.}}}\label{K_L00135-2}. Die Novelle\pwindex{Sterben. Novelle@\emph{Sterben. Novelle}|pwv}{ }ſchicken Sie mir gefälligſt, wenn Sie ſich jeder
          Alluſion {\pb}auf das Gerücht, wonach es zweierlei
          Menſchen auf der Welt gebe, enthalten haben. Nein, ſchicken Sie ſie mir in jedem Falle,
          ich bin neugierig darnach u. verſpreche Ihnen, die Arbeit \uline{bald} zu leſen.\pend
           
\pstart
           Leben Sie wohl, ſehr verehrter Herr Doctor, ſeien Sie herzlichſt gegrüßt u. entſchuldigen
          Sie die innere u. äußere Müdigkeit dieſer Zeilen.\pend
           
\pstart
           Ihr{\\[\baselineskip]}ergebener{\\[\baselineskip]}\spacefill\mbox{FMamroth}\pend
           \leftskip=0em{}\selectlanguage{ngerman}\endnumbering\briefempfaengerindex{Schnitzler, Arthur@\textsc{Schnitzler, Arthur}!zzzMamroth, Fedor@\emph{von Fedor Mamroth}!1892-11-171@{17. 11. 1892}|)be}\mylabel{L00135h}  \normalsize

\doendnotes{C}
\bigskip
\vfill

\clearpage

\footnotesize

\lohead{\textsc{register}}

% Definiere theindex-Environment komplett neu ohne reledmac
\makeatletter
\renewenvironment{theindex}{%
  \section*{\indexname}%
  \setlength{\parindent}{0pt}%
  \setlength{\parskip}{0pt plus 0.3pt}%
  \let\item\@idxitem
}{%
  \clearpage
}
\makeatother

\IfFileExists{\jobname-pw.ind}{\input{\jobname-pw.ind}}{}

\end{document}

      