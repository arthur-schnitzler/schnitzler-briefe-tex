%% latex-leseansicht-vorspann.tex
%% Vorspann für die Leseansicht.
%% Lädt die gemeinsame Datei latex-vorspann.tex mit nicht gesetztem Schalter.

\newif\ifkorrekturansicht
\korrekturansichtfalse

\input{../tex-inputs/latex-vorspann}


               \section[Fedor Mamroth an Arthur Schnitzler, 17. 11. 1892]{ Fedor Mamroth an Arthur Schnitzler, 17. 11. 1892}\nopagebreak\mylabel{v}\rehead{ }\begin{ledgroupsized}[t]{13cm}\normalsize\beginnumbering\briefempfaengerindex{Schnitzler, Arthur@\textsc{Schnitzler, Arthur}!zzzMamroth, Fedor@\emph{von Fedor Mamroth}!1892-11-171@{17. 11. 1892}|(be} \toendnotes[C]{\smallbreak\pagebreak[2]} \Standort{CUL, Schnitzler, B 68.}
\physDesc{Brief, 1 Blatt, 2 Seiten
\newline{}Handschrift: blaue Tinte, deutsche Kurrent
\newline{}Schnitzler: 1) mit Bleistift nummeriert: »3.« 2) mit rotem Buntstift eine Unterstreichung}\toendnotes[C]{\smallbreak}\pstart
           \noindent{}{\pb}\textcolor{gray}{\textbf{\textsc{Frankfurter Zeitung}}}{\\}\textsc{\textcolor{gray}{\textbf{und}}}{\\}\textcolor{gray}{\textbf{\textsc{Handelsblatt.}}}\orgindex{Frankfurter Zeitung@Frankfurter Zeitung|pw}\pend
           \pstart
           \textcolor{gray}{\textbf{\textsc{Redaction.\footnote{\noindent{}\textcolor{gray}{\textbf{\textsc{Für die Redaktion bestimmte
                                                  Briefe und Sendungen wolle man \so{nicht} an die Person eines
                                                  Redakteurs, sondern stets \textbf{an die
                                                  Redaktion der Frankfurter Zeitung}
                                                  adressiren}}}.}}}}\hfill \textcolor{gray}{\textbf{\textsc{Frankfurt a. M.\oindex{Frankfurt am Main@\textbf{Frankfurt am Main}|pw},}}}{ }17. Novbr. \textsc{\textcolor{gray}{\textbf{189}}}2\pend
           \pstart
           \textcolor{gray}{\textbf{\textsc{Telegramm-Adresse:}}}\pend
           \pstart
           \textcolor{gray}{\textbf{\textsc{Zeitung Frankfurt Main.}}}\pend
           \pstart{}Sehr verehrter Herr Doctor!\pend\pstart
           Wollte ich mein langes u. ſcheinbar ſo unartiges Stillſchweigen zu erklären u. zu
                    entſchuldigen ſuchen, ſo würde ich ſoviel Zeit u. Energie dazu brauchen, daß
                    gleich wieder die Exiſtenz \uline{dieſes} Briefes
                    bedroht wäre. Begnügen Sie Sich deshalb mit der Verſicherung meiner
                    warmen Sympathie u. meiner herzlichen Ergebenheit. Es ging nicht anders u. wenn
                    Sie mich umbringen: In Angelegenheit des »Märchen\pwindex{Schnitzler, Arthur 15.05.1862 – 21.10.1931@\textsc{Schnitzler, Arthur} (15.05.1862 – 21.10.1931), \emph{Schriftsteller, Mediziner}!Maerchen. Schauspiel in drei Aufzuegen1891 – 1891@\strich\emph{Das Märchen. Schauspiel in drei Aufzügen} {[}1891 – 1891{]}|pw}« ſind mir die Hände gebunden; ich habe (außer ſchlechten)
                    keinerlei Beziehungen zur hieſigen Theaterleitung, und überdies bin ich der
                    ungeſchickteſte Menſch, wenn es darauf ankommt, mir und meinen Freunden zu
                    nützen. Dieſes Talent muß man mit auf die Welt bringen wie der impertinente
                    Burſche Herr Lothar\pwindex{Lothar, Rudolf 23.2.1865 – 2.10.1943@\textsc{Lothar, Rudolf} (23.2.1865 – 2.10.1943), \emph{Schriftsteller, Journalist, Theaterdirektor}|pw}, der ſich \label{K_L00135_1v}\edtext{jüngſthin}{\lemma{\textnormal{\emph{jüngſthin}}}\Cendnote{\textnormal{Die Uraufführung von \emph{Cäsar
                            Borgia’s Ende}\pwindex{Lothar, Rudolf 23.2.1865 – 2.10.1943@\textsc{Lothar, Rudolf} (23.2.1865 – 2.10.1943), \emph{Schriftsteller, Journalist, Theaterdirektor}!Caesar Borgia s Ende1892 – 1892@\strich\emph{Cäsar Borgia’s Ende} {[}1892 – 1892{]}|pwk} fand am 12. 11. 1892 im örtlichen \emph{Schauspielhaus}\orgindex{Frankfurter Staedtisches Schauspielhaus@Frankfurter Städtisches Schauspielhaus|pwk}
                   statt.}}}\label{K_L00135_1h} von hier aus
                    inſcenierte.\pend
           \pstart
           Die neuen Dialoge\pwindex{Schnitzler, Arthur 15.05.1862 – 21.10.1931@\textsc{Schnitzler, Arthur} (15.05.1862 – 21.10.1931), \emph{Schriftsteller, Mediziner}!Anatol1892-10-29 – 1892-10-29@\strich\emph{Anatol} {[}1892-10-29 – 1892-10-29{]}|pwv}{ }ſandte ich
                    dem Berlin\oindex{Berlin@\textbf{Berlin}|pw}er Herrn\pwindex{Sack, Eduard 31.8.1831 – 20.5.1908@\textsc{Sack, Eduard} (31.8.1831 – 20.5.1908), \emph{Journalist, Pädagoge}|pwv}, der in neueſter Zeit
                    bei uns ſchöngeiſtige Literatur beſpricht, mit warmer Empfehlung. Jetzt wollen
                    wir ſehen, was \label{K_L00135_2v}\edtext{geſchieht}{\lemma{\textnormal{\emph{geſchieht}}}\Cendnote{\textnormal{Eine Rezension von \emph{Anatol}\pwindex{Schnitzler, Arthur 15.05.1862 – 21.10.1931@\textsc{Schnitzler, Arthur} (15.05.1862 – 21.10.1931), \emph{Schriftsteller, Mediziner}!Anatol1892-10-29 – 1892-10-29@\strich\emph{Anatol} {[}1892-10-29 – 1892-10-29{]}|pwk} dürfte nicht erschienen sein.}}}\label{K_L00135_2h}. Die Novelle\pwindex{Schnitzler, Arthur 15.05.1862 – 21.10.1931@\textsc{Schnitzler, Arthur} (15.05.1862 – 21.10.1931), \emph{Schriftsteller, Mediziner}!Sterben. Novelle1.10.1894 – 1.12.1894@\strich\emph{Sterben. Novelle} {[}1.10.1894 – 1.12.1894{]}|pwv}{ }ſchicken Sie mir
                    gefälligſt, wenn Sie ſich jeder Alluſion {\pb}auf das Gerücht, wonach es zweierlei
                    Menſchen auf der Welt gebe, enthalten haben. Nein, ſchicken Sie ſie mir in jedem
                    Falle, ich bin neugierig darnach u. verſpreche Ihnen, die Arbeit \uline{bald} zu leſen.\pend
           \pstart
           Leben Sie wohl, ſehr verehrter Herr Doctor, ſeien Sie herzlichſt gegrüßt u.
                    entſchuldigen Sie die innere u. äußere Müdigkeit dieſer Zeilen.\pend
           \pstart
           Ihr{\\[\baselineskip]}ergebener{\\[\baselineskip]}\spacefill\mbox{FMamroth}\pend
           \leftskip=0em{}\endnumbering\briefempfaengerindex{Schnitzler, Arthur@\textsc{Schnitzler, Arthur}!zzzMamroth, Fedor@\emph{von Fedor Mamroth}!1892-11-171@{17. 11. 1892}|)be}\mylabel{h}\end{ledgroupsized}  \newcommand{\dateiname}{L00135}\newcommand{\titel}{Fedor Mamroth an Arthur Schnitzler, 17. 11. 1892}\newcommand{\editorInnen}{Martin Anton Müller und Gerd-Hermann Susen}%% latex-leseansicht-abspann.tex
%% Abspann für die Leseansicht.
%% Der Schalter \ifkorrekturansicht ist bereits durch den Vorspann gesetzt.

%% latex-abspann.tex
%% Gemeinsamer Abspann für Korrekturansicht und Leseansicht.
%% Setzt den Schalter \ifkorrekturansicht voraus (gesetzt in den
%% einbindenden Dateien latex-korrekturansicht-abspann.tex bzw.
%% latex-leseansicht-abspann.tex).
%% ---------------------------------------------------------------

\normalsize

% Das esempio-Environment wird nur in der Leseansicht benötigt
\ifkorrekturansicht\else
\newenvironment{esempio}[3]%
{
    \vspace{1.5ex}
    \rlap{\underline{#1}}
    \par
    \setlength{\parindent}{0cm}
    \nopagebreak
    \leftskip=#2cm
    \rightskip=#3cm
}
{
    \par
}
\fi

\doendnotes{C}
\bigskip
\vfill

\clearpage

\footnotesize

\ifkorrekturansicht
  \lohead{\textsc{register}}
\fi

% theindex-Environment neu definieren ohne reledmac
\makeatletter
\renewenvironment{theindex}{%
  \ifkorrekturansicht
    \section*{\indexname}%
  \else
    \subsubsection*{Index der erwähnten Entitäten}%
  \fi
  \setlength{\parindent}{0pt}%
  \setlength{\parskip}{0pt plus 0.3pt}%
  \let\item\@idxitem
}{%
  \ifkorrekturansicht\clearpage\fi
}
\makeatother

\IfFileExists{\jobname-pw.ind}{\input{\jobname-pw.ind}}{}

% Quellenangabe nur in der Leseansicht
\ifkorrekturansicht\else
% Fallback-Definitionen, falls die .tex-Datei \titel etc. nicht gesetzt hat
\providecommand{\titel}{}
\providecommand{\editorInnen}{}
\providecommand{\dateiname}{\jobname}

\vspace{3cm}

\vfill

\footnotesize
\textsc{Quelle}: \titel. Herausgegeben von {\editorInnen}. In: \emph{Arthur Schnitzler: Briefwechsel mit Autorinnen und Autoren}.
 Digitale Edition, https://schnitzler-briefe.acdh.oeaw.ac.at/{\dateiname}.html (Stand \today)
\fi

\end{document}


      