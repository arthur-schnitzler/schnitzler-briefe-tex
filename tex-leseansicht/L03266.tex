%% latex-leseansicht-vorspann.tex
%% Vorspann für die Leseansicht.
%% Lädt die gemeinsame Datei latex-vorspann.tex mit nicht gesetztem Schalter.

\newif\ifkorrekturansicht
\korrekturansichtfalse

\input{../tex-inputs/latex-vorspann}

\begin{center}
            \textcolor{red}{ENTWURF, NICHT FERTIG KORRIGIERT}
                      \end{center}
            
         
         \renewcommand{\erwaehntePersonen}{Personen: Richard Beer-Hofmann,  Elisabeth von Österreich-Ungarn,  Franz Joseph I. von Österreich-Ungarn, Hugo von Hofmannsthal, Richard Metzl, Louise Metzl, Ottilie Salten, Adele Sandrock}
         \renewcommand{\erwaehnteInstitutionen}{Institutionen: Franz-Joseph-Orden}
         \renewcommand{\erwaehnteOrte}{Orte: Ostsee, Paris, Riga, Russland, Wien}
         \renewcommand{\erwaehnteWerke}{}
               \section[Felix Salten an Arthur Schnitzler, 23. 5. 1897]{ Felix Salten an Arthur Schnitzler, 23. 5. 1897}\nopagebreak\mylabel{v}\rehead{ }\begin{ledgroupsized}[t]{13cm}\normalsize\beginnumbering \toendnotes[C]{\smallbreak\pagebreak[2]} \Standort{CUL, Schnitzler, B 89, A 2.}
\physDesc{Brief, 2 Blätter, 6 Seiten, 5226 Zeichen
\newline{}Handschrift: schwarze Tinte, lateinische Kurrent
\newline{}Ordnung: mit Bleistift von unbekannter Hand nummeriert:
                                    »89« }\toendnotes[C]{\smallbreak}\pstart
           {\pb}Wien\oindex{Wien@\textbf{Wien}|pw}, 23. Mai 97\pend
           \pstart
           Lieber Arthur! Unsere Briefe haben sich gekreuzt. Am Tage, nach
               welchem meiner abgesendet war, empfing ich den Ihrigen. Ich konnte nicht mehr
               erwarten, von Ihnen Nachricht zu erhalten, denn da alle anderen Briefe von Ihnen
               bekamen, musste ich denken, mein Schreiben sei verloren gegangen. \pend
           \pstart
           Die gute Stimmung, in der ich kürzlich an Sie geschrieben, läßt nach. Vierzehn Tage
               Regenwetter liegen dazwischen, und der Frühling hat nun ein Ende. Ich fühle mich nach
               und nach wieder beschwert von allen Gewichten, die sonst immer mein Wesen drücken.
               Alle meine Wünsche concentriren sich nun darauf etwas fertig zu bringen. Wenn das
               geschehen könnte, wäre ich wesentlich gefestigter. Aber Sie können sich nicht denken,
               wie sehr ich an dem Gefühl der Unwichtigkeit leide, sobald ich mir meine {\pb}Arbeiten fertig vorstelle und
               gedruckt und unter allen anderem wirkend, was es in der Kunst gibt. Es ist das
               niederdrückendste Gefühl, und man ist wie gelähmt, wenn diese Empfindung Einem
               vorzurechnen beginnt, in wie vieler Beziehung man mit allem, was man machen möchte
               und könnte, entbehrlich sei. Dass ich damit allein bleibe, ist mir oft schwer genug.
               Und ich bin jetzt ganz allein. Ich möchte das Einmal ganz klar aussprechen. Damit ich
               später nicht mehr in Andeutungen darauf zurückkommen brauche. Ich möchte es umso
               eher, als ich es jetzt ohne die mindeste Bitterkeit thun kann u. meine sonstige
               starke Empfindlichkeit ungerechnet, über Empfindlichkeiten hinweg bin: Ich habe außer
               zu Ihnen, weder zu Hugo\pwindex{Hofmannsthal, Hugo von 1874-02-01 – 1929-07-15@\textsc{Hofmannsthal, Hugo von} (1874-02-01 – 1929-07-15), \emph{Schriftsteller}|pw} und noch viel weniger
               zu Beer-Hofmann\pwindex{Beer-Hofmann, Richard 1866-07-11 – 1945-09-26@\textsc{Beer-Hofmann, Richard} (1866-07-11 – 1945-09-26), \emph{Schriftsteller}|pw} Beziehungen irgend welcher
               Art. Sie suchen mich nicht und nirgends und ich sie nicht. {\pb}Bei der großen Schätzung, auf
               welcher mein Verkehr mit ihnen beruhte, war ich zunächst darauf hingewiesen, die
               Schuld an dieser Wandlung in mir allein zu suchen. Das hat mir manche, sehr verzagte
               Stunde verursacht. Jetzt bin ich mir über die inneren und äußeren Gründe, über die
               Art, in welcher diese Gründe mit dem Leben im Allgemeinen und mit den \introOben{}beteiligten\introOben{} Personen im Besonderen zusammenhängen vollständig
               klar, und deshalb spreche ich es – wie um der Ordnung willen – aus. Ich thue \introOben{}es\introOben{} übrigens auch, weil ich nicht weiß, ob nicht in ähnlicher
               oder anderer Weise diese Angelegenheit mit Ihnen besprochen wurde, und weil ich in
               Ihnen gewiss nicht das Gefühl erhalten wissen möchte, als hätten Sie mir etwas
               schonend zu verschweigen. Schließlich bitte ich Sie, zu glauben, dass ich durchaus
               keine gegentheiligen Versicherungen, auch keine Confidencen provoziren wollte. Nicht
               wahr, das glauben Sie mir? \pend
           \pstart
           {\pb}Mit meinen Arbeiten geht es
               mir merkwürdig. So vielerlei durcheinander, so viel neue Ausblicke. durch alte
               Stoffe, so viele neue Pläne haben mich selten auf einmal beschäftigt. Und wenn das
               Gefühl der großen Unwichtigkeit mich nicht hinderte, käme ich wol rascher vorwärts.
               Schließlich wird ja doch der Todesgedanke, der mich immer mehr und mehr meiner
               bemächtigt, seine Wirkung ausüben, und mich an ein Ziel führen. Ist es nicht
               sonderbar, dass ich an den Tod unabläßiger denn je, aber ohne Qual und ohne Angst, ja
               beinahe mit Neugierde denke? Ich bilde mir aus mehrfachen Gründen ein, dass ich mit
               fünfunddreißig Jahren an einem Märztag weggehen werde, und ich denke daran, wie an
               ein Unternehmen, dessen Zustandekommen zum Theil in meiner Macht liegt. Es ist nicht
               Selbstmord, warum sonst die fünfundreißig? Aber es ist so, als müsste ich in diesen
                  {\pb}acht Jahren, die es bis
               dahin noch sind, alles erledigen, und als wäre ich dann eben à jour. Meine
               Gesundheit, der alte Bronchialkatarrh, der ja doch einmal die Lunge angreifen muss, –
               meine Arbeiten, mein Lieben, alles scheint mir so, als könne es nicht länger
               vorhalten als bis zu jenem Märztag im Jahre 1905. Jedenfalls hängt meine
               Sorglosigkeit und der Glaube an eine Wendung der Dinge, die nun bald eintreten müsse
               mit dieser Vorstellung zusammen, und ich habe wenigstens die Zuversicht davon, nicht
               einen Tag früher zu sterben. \pend
           \pstart
           Gestern erhielt ich von Frl. Sandrock\pwindex{Sandrock, Adele 1863-08-19 – 1937-08-30@\textsc{Sandrock, Adele} (1863-08-19 – 1937-08-30), \emph{Schauspielerin}|pw} einen
               wunderschönen Brief, so echt, wie ich noch nichts von ihr gehört. »Ich liege an der
                  Ostsee\oindex{Ostsee@\textbf{Ostsee}|pw} und blicke mit meinen blauen Augen
               zum blauen Himmel empor.« so beginnt es, und es ist, als ob sie in ihrem gelösten
               Wesen mit diesen ironischen Worten eine direkte Verbindung zwischen sich und der Welt
               gefunden hätte. Dann kommen Sätze: »Gestern trat ich hier auf, und heute liegt Riga\oindex{Riga@\textbf{Riga}|pw}{ }\label{K_L03266-1v}\edtext{in Fraisen}{\lemma{\textnormal{\emph{in Fraisen}}}\Cendnote{\textnormal{veralteter Ausdruck von: in tödlichen Krämpfen, an einem
                  gefährlichen Anfall leidend}}}\label{K_L03266-1h}.« oder: »was geht in Dir vor? gehe ich {\pb}Dir ab? (Ich habe sie zwei
               Monate nicht gesehen und nichts von ihr gehört) Fällst Du nicht tot vom Boden bei dem
               Gedanken mich bald wieder zu sehen?« ec. Schließlich läuft das Ganze auf die Bitte
               hinaus, ihre Triumphe in die Zeitung zu geben. – Der Bruder\pwindex{Metzl, Richard 20.04.1870 – 31.10.1941@\textsc{Metzl, Richard} (20.04.1870 – 31.10.1941), \emph{Regisseur, Schauspieler, Theatersekretär}|pwv} des Fräulein M.\pwindex{Salten, Ottilie 07.03.1868 – 22.06.1942@\textsc{Salten, Ottilie} (07.03.1868 – 22.06.1942), \emph{Schauspielerin}|pw} hat in Rußland\oindex{Russland@\textbf{Russland}|pw}
               anläßlich der Anwesenheit unseres Kaisers\pwindex{Franz Joseph I. von Oesterreich-Ungarn 18.08.1830 – 21.11.1916@\textsc{Franz Joseph I. von Österreich-Ungarn} (18.08.1830 – 21.11.1916), \emph{Kaiser}|pwv} den Franz Josefs-Orden\orgindex{Franz-Joseph-Orden@Franz-Joseph-Orden|pw}
               erhalten. Heute war er hier, im Smoking und schwarzer Binde, mit der Rosette im
               Knopfloch bei seiner Mutter\pwindex{Metzl, Louise 1832-08-06 – 1909-09-10@\textsc{Metzl, Louise} (1832-08-06 – 1909-09-10)|pwv}
               zu Besuch. Gespräch: Thema: unsere Kaiserin\pwindex{Elisabeth von Oesterreich-Ungarn 24.12.1837 – 10.9.1898@\textsc{Elisabeth von Österreich-Ungarn} (24.12.1837 – 10.9.1898), \emph{Königin, Kaiserin}|pwv}. \uline{Ich}: Sie lebt wunderschön, – so
               allein, von nichts bekümmert. \uline{Er}: Sie könnte es noch
               schöner haben. \uline{Ich}: Wieso denn? \uline{Er}: Sie könnte Protektorin aller Wolhtätigkeitsvereine sein!
               (wörtlich.) \pend
           \pstart
           Ich weiß wieder nicht, ob dieser Brief Sie in Paris\oindex{Paris@\textbf{Paris}|pw} noch trifft. Wenn Sie ihn erhalten, dann bitte zeigen Sie’s mir, wenn
               auch nur auf einer Postkarte, an. \pend
           \pstart
            Herzlich Ihr {\\[\baselineskip]}\spacefill\mbox{Salten}\pend
           \leftskip=0em{}
         
         \endnumbering\mylabel{h}\end{ledgroupsized}\begin{anhang}\end{anhang}\newcommand{\dateiname}{L03266}\newcommand{\titel}{Felix Salten an Arthur Schnitzler, 23. 5. 1897}\newcommand{\editorInnen}{Martin Anton Müller und Laura Untner}%% latex-leseansicht-abspann.tex
%% Abspann für die Leseansicht.
%% Der Schalter \ifkorrekturansicht ist bereits durch den Vorspann gesetzt.

%% latex-abspann.tex
%% Gemeinsamer Abspann für Korrekturansicht und Leseansicht.
%% Setzt den Schalter \ifkorrekturansicht voraus (gesetzt in den
%% einbindenden Dateien latex-korrekturansicht-abspann.tex bzw.
%% latex-leseansicht-abspann.tex).
%% ---------------------------------------------------------------

\normalsize

% Das esempio-Environment wird nur in der Leseansicht benötigt
\ifkorrekturansicht\else
\newenvironment{esempio}[3]%
{
    \vspace{1.5ex}
    \rlap{\underline{#1}}
    \par
    \setlength{\parindent}{0cm}
    \nopagebreak
    \leftskip=#2cm
    \rightskip=#3cm
}
{
    \par
}
\fi

\doendnotes{C}
\bigskip
\vfill

\clearpage

\footnotesize

\ifkorrekturansicht
  \lohead{\textsc{register}}
\fi

% theindex-Environment neu definieren ohne reledmac
\makeatletter
\renewenvironment{theindex}{%
  \ifkorrekturansicht
    \section*{\indexname}%
  \else
    \subsubsection*{Index der erwähnten Entitäten}%
  \fi
  \setlength{\parindent}{0pt}%
  \setlength{\parskip}{0pt plus 0.3pt}%
  \let\item\@idxitem
}{%
  \ifkorrekturansicht\clearpage\fi
}
\makeatother

\IfFileExists{\jobname-pw.ind}{\input{\jobname-pw.ind}}{}

% Quellenangabe nur in der Leseansicht
\ifkorrekturansicht\else
% Fallback-Definitionen, falls die .tex-Datei \titel etc. nicht gesetzt hat
\providecommand{\titel}{}
\providecommand{\editorInnen}{}
\providecommand{\dateiname}{\jobname}

\vspace{3cm}

\vfill

\footnotesize
\textsc{Quelle}: \titel. Herausgegeben von {\editorInnen}. In: \emph{Arthur Schnitzler: Briefwechsel mit Autorinnen und Autoren}.
 Digitale Edition, https://schnitzler-briefe.acdh.oeaw.ac.at/{\dateiname}.html (Stand \today)
\fi

\end{document}


      