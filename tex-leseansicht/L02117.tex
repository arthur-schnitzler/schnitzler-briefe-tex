%% latex-leseansicht-vorspann.tex
%% Vorspann für die Leseansicht.
%% Lädt die gemeinsame Datei latex-vorspann.tex mit nicht gesetztem Schalter.

\newif\ifkorrekturansicht
\korrekturansichtfalse

\input{../tex-inputs/latex-vorspann}


\section[Robert Adam an Arthur Schnitzler, 2. 4. 1913]{L02117 Robert Adam an Arthur Schnitzler, 2. 4. 1913}
\nopagebreak\mylabel{L02117v}
\rehead{ }\normalsize\beginnumbering\briefempfaengerindex{Schnitzler, Arthur@\textsc{Schnitzler, Arthur}!zzzAdam, Robert@\emph{von Robert Adam}!1913-04-021@{2. 4. 1913}|(be}
\toendnotes[C]{\smallbreak\pagebreak[2]}
\correspDesc{Versand  durch Robert Adam am 2. 4. 1913 in Zistersdorf
\newline{}Erhalt  durch Arthur Schnitzler im Zeitraum [3. 4. 1913
                  – 7. 4. 1913?] in Wien}\toendnotes[C]{\smallbreak}
\Standort{DLA, A:Schnitzler, HS.NZ85.1.4230,5.}
\physDesc{Brief, 1 Blatt, 2 Seiten, 931 Zeichen
\newline{}Handschrift: schwarze Tinte, deutsche Kurrent
\newline{}Schnitzler: 1) mit Bleistift beschriftet: »\textsc{Adam}«  2) mit rotem Buntstift eine Unterstreichung}\Standort{Wien, Österreichische Nationalbibliothek, Cod.ser. 52.266, 155.}
\physDesc{handschriftliche Abschrift. 1 Blatt, 1 Seite, 931 Zeichen
\newline{}Handschrift: schwarze Tinte, Gabelsberger Kurzschrift}\Standort{Wien, Österreichische Nationalbibliothek, Cod.ser. 52.266, 155.}
\physDesc{maschinenschriftliche Abschrift, 1 Blatt, 1 Seite, 931 Zeichen
\newline{}Schreibmaschine}
\pstart
           \raggedleft{}{\pb}Ziſtersdorf\oindex{Zistersdorf@\textbf{Zistersdorf}, \emph{Verwaltungsgebiet}|pw}, am 2. April 1913.\pend
           
\pstart{}Hochverehrter Herr Doktor!\pend\vspace{0.5em}
\pstart
           Das freundliche Intereſſe, das Sie seinerzeit meiner Komödie Die Geſchichte des Alî ibn Bekkâr mit Schams an-Nahâr\pwindex{Adam, Robert 20.\,4.\,1877 Wien – 16.\,10.\,1961 Baden bei Wien@\textsc{Adam, Robert} (20.\,4.\,1877 Wien – 16.\,10.\,1961 Baden bei Wien), \emph{Schriftsteller, Richter}!Geschichte des Alî ibn Bekkâr mit Schams an-Nahâr@\strich\emph{Die Geschichte des Alî ibn Bekkâr mit Schams an-Nahâr}|pw} und vor
               zwei Jahren dem Manuſkript der Komödie: Neidhard\pwindex{Adam, Robert 20.\,4.\,1877 Wien – 16.\,10.\,1961 Baden bei Wien@\textsc{Adam, Robert} (20.\,4.\,1877 Wien – 16.\,10.\,1961 Baden bei Wien), \emph{Schriftsteller, Richter}!Neidhard@\strich\emph{Neidhard}|pw}
               entgegenbrachten, ermutigt mich, hochverehrter Herr Doktor, neuerlich mit einer Bitte
               an Sie heranzutreten.\pend
           
\pstart
           Ich habe in meiner ländlichen Abgeſchiedenheit kürzlich eine dramatiſche Studie zum
               Abſchluß gebracht, die ich \textsc{Fatme}\pwindex{Adam, Robert 20.\,4.\,1877 Wien – 16.\,10.\,1961 Baden bei Wien@\textsc{Adam, Robert} (20.\,4.\,1877 Wien – 16.\,10.\,1961 Baden bei Wien), \emph{Schriftsteller, Richter}!Fatme@\strich\emph{Fatme}|pw} nennen will. Es{ }ſind vier Proſa-Akte von nicht allzu großem Umfange.\pend
           
\pstart
           {\pb}Darf ich mir erlauben, hochverehrter Herr Doktor,
               Ihnen das Manuſkript,{ }ſobald die Schreibmaſchinenabſchrift fertiggeſtellt iſt,
               einzuſenden?\pend
           
\pstart
           Ich weiß, daß ich Ihre Güte und Zeit in unbilligem Maße in Anſpruch nehme; aber Sie
               waren bisher der Einzige, der{ }ſich meiner annahm, und ich{ }ſetze meine ganze Hoffnung
               in Ihre Güte.\pend
           
\pstart
           Mit den ergebenſten Grüßen\pend
           
\pstart
           Ihr{\\[\baselineskip]}\spacefill\mbox{Robert Adam}\pend
           \leftskip=0em{}
\pstart
           \noindent{}\raggedleft{}(Bezirksrichter Dr Robert Adam{\\}Pollak, Ziſtersdorf{ }\textsc{N. Ö.}\oindex{Zistersdorf@\textbf{Zistersdorf}, \emph{Verwaltungsgebiet}|pw})\pend
           \selectlanguage{ngerman}\endnumbering\briefempfaengerindex{Schnitzler, Arthur@\textsc{Schnitzler, Arthur}!zzzAdam, Robert@\emph{von Robert Adam}!1913-04-021@{2. 4. 1913}|)be}\mylabel{L02117h}  \newcommand{\dateiname}{L02117}\newcommand{\titel}{Robert Adam an Arthur Schnitzler, 2. 4. 1913}\newcommand{\editorInnen}{Martin Anton Müller und Gerd-Hermann Susen}%% latex-leseansicht-abspann.tex
%% Abspann für die Leseansicht.
%% Der Schalter \ifkorrekturansicht ist bereits durch den Vorspann gesetzt.

%% latex-abspann.tex
%% Gemeinsamer Abspann für Korrekturansicht und Leseansicht.
%% Setzt den Schalter \ifkorrekturansicht voraus (gesetzt in den
%% einbindenden Dateien latex-korrekturansicht-abspann.tex bzw.
%% latex-leseansicht-abspann.tex).
%% ---------------------------------------------------------------

\normalsize

% Das esempio-Environment wird nur in der Leseansicht benötigt
\ifkorrekturansicht\else
\newenvironment{esempio}[3]%
{
    \vspace{1.5ex}
    \rlap{\underline{#1}}
    \par
    \setlength{\parindent}{0cm}
    \nopagebreak
    \leftskip=#2cm
    \rightskip=#3cm
}
{
    \par
}
\fi

\doendnotes{C}
\bigskip
\vfill

\clearpage

\footnotesize

\ifkorrekturansicht
  \lohead{\textsc{register}}
\fi

% theindex-Environment neu definieren ohne reledmac
\makeatletter
\renewenvironment{theindex}{%
  \ifkorrekturansicht
    \section*{\indexname}%
  \else
    \subsubsection*{Index der erwähnten Entitäten}%
  \fi
  \setlength{\parindent}{0pt}%
  \setlength{\parskip}{0pt plus 0.3pt}%
  \let\item\@idxitem
}{%
  \ifkorrekturansicht\clearpage\fi
}
\makeatother

\IfFileExists{\jobname-pw.ind}{\input{\jobname-pw.ind}}{}

% Quellenangabe nur in der Leseansicht
\ifkorrekturansicht\else
% Fallback-Definitionen, falls die .tex-Datei \titel etc. nicht gesetzt hat
\providecommand{\titel}{}
\providecommand{\editorInnen}{}
\providecommand{\dateiname}{\jobname}

\vspace{3cm}

\vfill

\footnotesize
\textsc{Quelle}: \titel. Herausgegeben von {\editorInnen}. In: \emph{Arthur Schnitzler: Briefwechsel mit Autorinnen und Autoren}.
 Digitale Edition, https://schnitzler-briefe.acdh.oeaw.ac.at/{\dateiname}.html (Stand \today)
\fi

\end{document}


