%% latex-korrekturansicht-vorspann.tex
%% Vorspann für die Korrekturansicht.
%% Lädt die gemeinsame Datei latex-vorspann.tex mit gesetztem Schalter.

\newif\ifkorrekturansicht
\korrekturansichttrue

\input{../tex-inputs/latex-vorspann}


\section[Robert Adam an Arthur Schnitzler, 2. 4. 1913]{L02117 Robert Adam an Arthur Schnitzler, 2. 4. 1913}
\nopagebreak\mylabel{L02117v}
\rehead{ }\normalsize\beginnumbering\briefempfaengerindex{Schnitzler, Arthur@\textsc{Schnitzler, Arthur}!zzzAdam, Robert@\emph{von Robert Adam}!1913-04-021@{2. 4. 1913}|(be}
\toendnotes[C]{\smallbreak\pagebreak[2]}\Standort{DLA, A:Schnitzler, HS.NZ85.1.4230,5.}
\physDesc{Brief, 1 Blatt, 2 Seiten, 931 Zeichen
\newline{}Handschrift: schwarze Tinte, deutsche Kurrent
\newline{}Schnitzler: 1) mit Bleistift beschriftet: »\textsc{Adam}«  2) mit rotem Buntstift eine Unterstreichung}\Standort{Wien, Österreichische Nationalbibliothek, Cod.ser. 52.266, 155.}
\physDesc{handschriftliche Abschrift1 Blatt, 1 Seite, 931 Zeichen
\newline{}Handschrift: schwarze Tinte, Gabelsberger Kurzschrift}\Standort{Wien, Österreichische Nationalbibliothek, Cod.ser. 52.266, 155.}
\physDesc{maschinenschriftliche Abschrift1 Blatt, 1 Seite, 931 Zeichen
\newline{}Schreibmaschine}
\pstart
           \raggedleft{}{\pb}Ziſtersdorf\oindex{Zistersdorf@\textbf{Zistersdorf}, \emph{A.ADM3}|pw}, am 2. April
                  1913.\pend
           
\pstart{}Hochverehrter Herr Doktor!\pend\vspace{0.5em}
\pstart
           Das freundliche Intereſſe, das Sie seinerzeit meiner Komödie Die Geſchichte des Alî ibn Bekkâr mit Schams an-Nahâr\pwindex{Geschichte des Alî ibn Bekkâr mit Schams an-Nahâr@\emph{Die Geschichte des Alî ibn Bekkâr mit Schams an-Nahâr}|pw} und vor
               zwei Jahren dem Manuſkript der Komödie: Neidhard\pwindex{Neidhard@\emph{Neidhard}|pw}
               entgegenbrachten, ermutigt mich, hochverehrter Herr Doktor, neuerlich mit einer Bitte
               an Sie heranzutreten.\pend
           
\pstart
           Ich habe in meiner ländlichen Abgeſchiedenheit kürzlich eine dramatiſche Studie zum
               Abſchluß gebracht, die ich \textsc{Fatme}\pwindex{Fatme@\emph{Fatme}|pw} nennen will. Es ſind vier Proſa-Akte von nicht allzu großem Umfange.\pend
           
\pstart
           {\pb}Darf ich mir erlauben, hochverehrter Herr Doktor,
               Ihnen das Manuſkript, ſobald die Schreibmaſchinenabſchrift fertiggeſtellt iſt,
               einzuſenden?\pend
           
\pstart
           Ich weiß, daß ich Ihre Güte und Zeit in unbilligem Maße in Anſpruch nehme; aber Sie
               waren bisher der Einzige, der ſich meiner annahm, und ich ſetze meine ganze Hoffnung
               in Ihre Güte.\pend
           
\pstart
           Mit den ergebenſten Grüßen\pend
           
\pstart
           Ihr{\\[\baselineskip]}\spacefill\mbox{Robert Adam}\pend
           \leftskip=0em{}
\pstart
           \noindent{}\raggedleft{}(Bezirksrichter Dr Robert Adam{\\}Pollak, Ziſtersdorf{ }\textsc{N. Ö.}\oindex{Zistersdorf@\textbf{Zistersdorf}, \emph{A.ADM3}|pw})\pend
           \selectlanguage{ngerman}\endnumbering\briefempfaengerindex{Schnitzler, Arthur@\textsc{Schnitzler, Arthur}!zzzAdam, Robert@\emph{von Robert Adam}!1913-04-021@{2. 4. 1913}|)be}\mylabel{L02117h}  \normalsize

\doendnotes{C}
\bigskip
\vfill

\clearpage

\footnotesize

\lohead{\textsc{register}}

% Definiere theindex-Environment komplett neu ohne reledmac
\makeatletter
\renewenvironment{theindex}{%
  \section*{\indexname}%
  \setlength{\parindent}{0pt}%
  \setlength{\parskip}{0pt plus 0.3pt}%
  \let\item\@idxitem
}{%
  \clearpage
}
\makeatother

\IfFileExists{\jobname-pw.ind}{\input{\jobname-pw.ind}}{}

\end{document}

      