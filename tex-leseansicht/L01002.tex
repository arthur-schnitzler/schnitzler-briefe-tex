%% latex-korrekturansicht-vorspann.tex
%% Vorspann für die Korrekturansicht.
%% Lädt die gemeinsame Datei latex-vorspann.tex mit gesetztem Schalter.

\newif\ifkorrekturansicht
\korrekturansichttrue

\input{../tex-inputs/latex-vorspann}


\section[Arthur Schnitzler an Richard Beer-Hofmann, 28. 11. 1899]{L01002 Arthur Schnitzler an Richard Beer-Hofmann, 28. 11. 1899}
\nopagebreak\mylabel{L01002v}
\rehead{ }\normalsize\beginnumbering\briefempfaengerindex{Beer-Hofmann, Richard@\textsc{Beer-Hofmann, Richard}!zzzSchnitzler, Arthur@\emph{von Arthur Schnitzler}!1899-11-281@{28. 11. 1899}|(be}
\toendnotes[C]{\smallbreak\pagebreak[2]}\Standort{YCGL, MSS 31.}
\physDesc{Postkarte, 341 Zeichen
\newline{}Handschrift: Bleistift, deutsche Kurrent
\newline{}Versand: Stempel: »\nobreak{}\oindex{I., Innere Stadt@\textbf{I., Innere Stadt}, \emph{A.ADM3}|pwk}Wien 1/1, 28. 11. 99, 10–11N\nobreak{}«.  
\newline{}Ordnung: mit Bleistift von unbekannter Hand datiert: »28. 11.« }\pstart{}{\pb}Herrn \textsc{Dr. Richard
                     Beer-Hofmann}\pend{}\pstart{}Wien\oindex{Wien@\textbf{Wien}, \emph{A.ADM2}|pw}\pend{}\pstart{}\textsc{I. Wollzeile 15\oindex{Wollzeile@\textbf{Wollzeile}, \emph{Straße (K.STR)}|pw}}.\pend{}{\bigskip}\vspace{1em}
\pstart
           \noindent{}{\pb}lieber, unmöglich für den Homburger
                  Prinzen\pwindex{Prinz Friedrich von Homburg oder die Schlacht bei Fehrbellin@\emph{Prinz Friedrich von Homburg oder die Schlacht bei Fehrbellin}|pw} was zu verſchaffen. Ich mir ſelber durch \textsc{Rosenbaum\pwindex{Rosenbaum, Richard 04.11.1867 – 25.06.1942@\textsc{Rosenbaum, Richard} (04.11.1867 – 25.06.1942), \emph{Dramaturg/Dramaturgin, Verleger/Verlegerin}|pw}}, der aber nur \uline{einen} verſorgen kann.
               Intendanz-Erlaſs, wegen der Angriffe in den Zeitungen, dſs man an der Kaſſe nie was
               kriegt. Alſo durch Dienſtma{\geminationn} an der Kaſſe noch am
               eheſten möglich.\pend
           \pstart Herzlich Ihr \spacefill\mbox{Arthur}\pend{}\selectlanguage{ngerman}\endnumbering\briefempfaengerindex{Beer-Hofmann, Richard@\textsc{Beer-Hofmann, Richard}!zzzSchnitzler, Arthur@\emph{von Arthur Schnitzler}!1899-11-281@{28. 11. 1899}|)be}\mylabel{L01002h}  \normalsize

\doendnotes{C}
\bigskip
\vfill

\clearpage

\footnotesize

\lohead{\textsc{register}}

% Definiere theindex-Environment komplett neu ohne reledmac
\makeatletter
\renewenvironment{theindex}{%
  \section*{\indexname}%
  \setlength{\parindent}{0pt}%
  \setlength{\parskip}{0pt plus 0.3pt}%
  \let\item\@idxitem
}{%
  \clearpage
}
\makeatother

\IfFileExists{\jobname-pw.ind}{\input{\jobname-pw.ind}}{}

\end{document}

      