%% latex-leseansicht-vorspann.tex
%% Vorspann für die Leseansicht.
%% Lädt die gemeinsame Datei latex-vorspann.tex mit nicht gesetztem Schalter.

\newif\ifkorrekturansicht
\korrekturansichtfalse

\input{../tex-inputs/latex-vorspann}


\section[Theodor Herzl an Arthur Schnitzler, 27. 11. 1894]{L03837 Theodor Herzl an Arthur Schnitzler, 27. 11. 1894}
\nopagebreak\mylabel{L03837v}
\rehead{ }\normalsize\beginnumbering\briefempfaengerindex{Schnitzler, Arthur@\textsc{Schnitzler, Arthur}!zzzHerzl, Theodor@\emph{von Theodor Herzl}!1894-11-271@{27. 11. 1894}|(be}
\toendnotes[C]{\smallbreak\pagebreak[2]}
\correspDesc{Versand  durch Theodor Herzl am 27. 11. 1894 in Paris
\newline{}Erhalt  durch Arthur Schnitzler im Zeitraum [28. 11. 1894 – 2. 12. 1894?] in Wien}\toendnotes[C]{\smallbreak}
\Standort{CUL, Schnitzler, B 39.}
\physDesc{Brief, 2 Blätter, 8 Seiten, 4627 Zeichen
\newline{}Handschrift: schwarze Tinte, lateinische Kurrent (\noindent{}Nummerierung des zweiten Bogens: »2 Bl.«)
\newline{}Ordnung: 1) mit Bleistift von unbekannter Hand nummeriert: »16« und
                                 »B39«  2) mit Bleistift mutmaßlich von Leon
                                    Kellner\pwindex{Kellner, Leon 17.\,4.\,1859 Tarnów – 5.\,12.\,1928 Wien@\textsc{Kellner, Leon} (17.\,4.\,1859 Tarnów – 5.\,12.\,1928 Wien), \emph{Zionist, Literaturhistoriker, Anglist}|pw} Markierung interessanter Stellen 3) mit rotem Buntstift eine Anstreichung}
\buchAbdrucke{\weitereDrucke{Theodor Herzl: \emph{Briefe und autobiographische Notizen 1866–1895}. Bearbeitet von Johannes Wachten in Zusammenarbeit mit Chaya Harel, Daisy Tycho und Manfred Winkler. Berlin, Frankfurt am Main, Wien: \emph{Propyläen} 1983, S. 557–559 (Briefe und Tagebücher. Herausgegeben von Alex Bein, Hermann Greive, Moshe Schaerf, Julius H. Schoeps und Johannes Wachten, 1).} }\toendnotes[C]{\smallbreak}
\pstart
           \raggedleft{}{\pb}Palais Bourbon\oindex{Palais Bourbon@\textbf{Palais Bourbon}, \emph{Regierungsgebäude}|pw}\pend
           
\pstart
           \raggedleft{}27. XI. 94\pend
           
\pstart{}Mein lieber Schnitzler!\pend\vspace{0.5em}
\pstart
           Diese Zeilen schreibe ich Ihnen auf der Galerie des Hauses\oindex{Palais Bourbon@\textbf{Palais Bourbon}, \emph{Regierungsgebäude}|pwv} in Augenblicken die ich meiner \label{K_L03837-1v}\edtext{Strafknechtschaft}{\lemma{\textnormal{\emph{Strafknechtschaft}}}\Cendnote{\textnormal{Herzl\pwindex{Herzl, Theodor 2.\,5.\,1860 Budapest – 3.\,7.\,1904 Edlach@\textsc{Herzl, Theodor} (2.\,5.\,1860 Budapest – 3.\,7.\,1904 Edlach), \emph{Schriftsteller, Journalist}|pwk} arbeitete als Korrespondent für die
                     \emph{Neue Freie Presse}\orgindex{Neue Freie Presse@Neue Freie Presse|pwk} in Paris\oindex{Paris@\textbf{Paris}, \emph{Hauptstadt}|pwk}.}}}\label{K_L03837-1} abzwacke. Sie dürfen daher keinen Brief
               erwarten, der so schön wäre wie Ihre Novelle\pwindex{Schnitzler, Arthur 15.\,5.\,1862 Wien – 21.\,10.\,1931 ebd.@\textsc{Schnitzler, Arthur} (15.\,5.\,1862 Wien – 21.\,10.\,1931 ebd.), \emph{Schriftsteller, Mediziner}!Sterben. Novelle@\strich\emph{Sterben. Novelle}|pwv}.\pend
           
\pstart
           Ich beginne mit den Einwendungen. Ich hatte – obwol ich das Buch\pwindex{Schnitzler, Arthur 15.\,5.\,1862 Wien – 21.\,10.\,1931 ebd.@\textsc{Schnitzler, Arthur} (15.\,5.\,1862 Wien – 21.\,10.\,1931 ebd.), \emph{Schriftsteller, Mediziner}!Sterben. Novelle@\strich\emph{Sterben. Novelle}|pwv} mit den Augen einer besonderen Neigung
               und in einem Athem bis zu Ende las – einigemal das Gefühl der Länge. Ziffermässig
               ausgedrückt sind vielleicht 25 Seiten zuviel, wovon sich 10 über das ganze Buch\pwindex{Schnitzler, Arthur 15.\,5.\,1862 Wien – 21.\,10.\,1931 ebd.@\textsc{Schnitzler, Arthur} (15.\,5.\,1862 Wien – 21.\,10.\,1931 ebd.), \emph{Schriftsteller, Mediziner}!Sterben. Novelle@\strich\emph{Sterben. Novelle}|pwv} vertheilen und 15 auf die
               Einleitung. Diese halte ich für verfehlt. Man muss {\pb}Vertrauen zum Verfasser haben, um
               darüber hinauszukommen. Dieses Vertrauen habe ich, haben die, die Sie kennen schon
               heute. Die Vielen werden es erst haben, wenn sich Ihre glänzende Laufbahn in der
               Literatur erfüllt haben wird.\pend
           
\pstart
           Nicht als ob ich gegen die \label{K_L03837-2v}\edtext{grisaille}{\lemma{\textnormal{\emph{grisaille}}}\Cendnote{\textnormal{Malerei in
                  Grautönen}}}\label{K_L03837-2} dieses Anfangs empfindungslos wäre. Das ist fein wie das Ganze,
               aber es gibt Grenzen in der Feinheit. Darüber werden wir später noch reden.\pend
           
\pstart
           Ihre ganze Novelle\pwindex{Schnitzler, Arthur 15.\,5.\,1862 Wien – 21.\,10.\,1931 ebd.@\textsc{Schnitzler, Arthur} (15.\,5.\,1862 Wien – 21.\,10.\,1931 ebd.), \emph{Schriftsteller, Mediziner}!Sterben. Novelle@\strich\emph{Sterben. Novelle}|pwv} ist eine
               feine Arbeit in Grau. Aber eben desshalb hätte ich für den Anfang lebhaftere Farben
               gewünscht. Bedenken Sie, was in Ihre Arbeit\pwindex{Schnitzler, Arthur 15.\,5.\,1862 Wien – 21.\,10.\,1931 ebd.@\textsc{Schnitzler, Arthur} (15.\,5.\,1862 Wien – 21.\,10.\,1931 ebd.), \emph{Schriftsteller, Mediziner}!Sterben. Novelle@\strich\emph{Sterben. Novelle}|pwv} hineingekommen wäre, wenn wir diesen Felix in der
               Gestalt kennen {\pb}gelernt hätten, in der
               er Marie verführte. So ist es nur der letzte Akt einer Tragodie. Ich kann mir wohl
               denken, dass der Arzt in Ihnen gegen \strikeout{den} ein
               brüskeres Auftreten der Krankheit protestirt, der Arzt der sich im Uebrigen – was ich
               zuerst mit Erstaunen dann verstehend bemerkte – so discret zuruckhält.\pend
           
\pstart
           Also das Liebesbacchanal war zuvörderst zu zeigen – so kurz Sie wollen, da es sich
               nicht darum handelt – denn Menschen kann man nur aus ihrer Geschichte verstehen,
               bedauern und bewundern.\pend
           
\pstart
           Die übers übrige Buch\pwindex{Schnitzler, Arthur 15.\,5.\,1862 Wien – 21.\,10.\,1931 ebd.@\textsc{Schnitzler, Arthur} (15.\,5.\,1862 Wien – 21.\,10.\,1931 ebd.), \emph{Schriftsteller, Mediziner}!Sterben. Novelle@\strich\emph{Sterben. Novelle}|pwv}
               vertheilten überflüssigen Seiten sind leere Stellen, Wiederholungen die nicht als
               solche beabsichtigt sind und eine ganz kleine Manier in der {\pb}Naturschilderung, die mir später darum
               unangehehm auffiel, weil sie mich anfangs entzückt hatte.\pend
           
\pstart
           Jetzt bin ich mit dem Tadel fertig. Im Uebrigen ist es ein kleines Meisterwerk mit
               vielen Farben in Grau und der erreichten Vollendung im Verschweigen.\pend
           
\pstart
           Loben muss ich die Sicherheit der Psychologie. Es ist eine \label{K_L03837-3v}\edtext{Marivaudage}{\lemma{\textnormal{\emph{Marivaudage}}}\Cendnote{\textnormal{Tändelei in galantem Stil, am französischen Rokokoschriftsteller Pierre Carlet de Marivaux\pwindex{Marivaux, Pierre Carlet de 4.\,2.\,1688 Paris – 12.\,2.\,1763 ebd.@\textsc{Marivaux, Pierre Carlet de} (4.\,2.\,1688 Paris – 12.\,2.\,1763 ebd.), \emph{Schriftsteller}|pwk} orientierter
                  Begriff der Literaturkritik, der gegen Ende des 19. Jahrhunderts eine
                  postitive Neubewertung erfuhr}}}\label{K_L03837-3} im Traurigen. Viele kleine Züge von grosser
               Wahrheit und die Absicht die immer gegenwärtig, ist nirgends verletzend.\pend
           
\pstart
           Die Sprache ist ausgezeichnet. Sie haben ganz Recht, den »Wiener\oindex{Wien@\textbf{Wien}, \emph{Verwaltungsgebiet}|pw} Stil«, der jetzt gemacht wird, nicht mitzumachen. Ich
               muss nach ersten Mustern greifen, um für Ihren Vortrag Vergleiche zu finden. Die Salzburger\oindex{Salzburg@\textbf{Salzburg}, \emph{Verwaltungsgebiet}|pw} Stellen erinnern {\pb}mich an Gottfried Keller\pwindex{Keller, Gottfried 19.\,7.\,1819 Zürich – 16.\,7.\,1890 ebd.@\textsc{Keller, Gottfried} (19.\,7.\,1819 Zürich – 16.\,7.\,1890 ebd.), \emph{Schriftsteller}|pw}. Allerdings sind sie schwächer, zarter.\pend
           
\pstart
           Ich kann Ihnen sagen, dass ich mancke Stellen zweimal las wegen des Gesanges und
               Duftes der darin ist.\pend
           
\pstart
           Sie sind also von den Anatol Sachen\pwindex{Schnitzler, Arthur 15.\,5.\,1862 Wien – 21.\,10.\,1931 ebd.@\textsc{Schnitzler, Arthur} (15.\,5.\,1862 Wien – 21.\,10.\,1931 ebd.), \emph{Schriftsteller, Mediziner}!Anatol@\strich\emph{Anatol}|pw} weiter
               gekommen u. zw. in der Tiefe.\pend
           
\pstart
           Nachdem ich Ihre Novelle\pwindex{Schnitzler, Arthur 15.\,5.\,1862 Wien – 21.\,10.\,1931 ebd.@\textsc{Schnitzler, Arthur} (15.\,5.\,1862 Wien – 21.\,10.\,1931 ebd.), \emph{Schriftsteller, Mediziner}!Sterben. Novelle@\strich\emph{Sterben. Novelle}|pwv}
               gelesen habe ich mich noch einmal über Ihren Brief über mein Stück\pwindex{Herzl, Theodor 2.\,5.\,1860 Budapest – 3.\,7.\,1904 Edlach@\textsc{Herzl, Theodor} (2.\,5.\,1860 Budapest – 3.\,7.\,1904 Edlach), \emph{Schriftsteller, Journalist}!neue Ghetto. Schauspiel in vier Acten@\strich\emph{Das neue Ghetto. Schauspiel in vier Acten}|pwv} gefreut.\pend
           
\pstart
           Es ist natürlich, dass ich alle Ihre Einwendungen beim Umarbeiten beherzigen werde.
               Es wäre mir noch lieber gewesen, wenn Sie jede einzelne Stelle, die Ihnen Bedenken
               erregte, angemerkt hätten. {\pb}Doch werde
               ich mich wol auch so zurechtfinden. Insbesondere den Fehler der Selbsterläuterung
               werde ich zu vermeiden trachten. Vergessen Sie aber nicht, dass man – banausischer
               Gedanke – auf der Bühne gröber sein muss, weil man keine oder wenig feine Hörer hat.
               Und es muss auf die Bühne, es muss, es muss. Darum hab’ ichs geschrieben. Es muss ins
               Volk!\pend
           
\pstart
           Darum begnügt es sich auch kühn zu sein u. will nicht trotzig sein. Sonst hört man
               mich nicht bis zu Ende an. Ich rede zu einem Volk von Antisemiten!\pend
           
\pstart
           Darin werde ich also nichts {\pb}ändern. Der
               von Ihnen bemängelte Charakter der Frau ist eine Schwierigkeit. Ich kann das
               Schlechte nur auf eine Art »liebenswürdig« machen: indem ich einen Schleier von
               Dummheit darüber breite. Dann wird’s aber lustspielmässig. Nun ist aber diese Frau
               als ein Factor der ihn ins Ghetto zurückdrängt nöthig.\pend
           
\pstart
           Was meinen Sie mit de\substVorne{}\textsuperscript{m}\substDazwischen{}r\substHinten{} schlechten \strikeout{Auft} Einführung Bichlers? Es wäre
               mir angenehm, darauf sofort Ihre Antwort zu erhalten, weil ich in den nächsten Tagen
               wieder hoffen kann für mich zu arbeiten. Dann soll die Abschrift\pwindex{Herzl, Theodor 2.\,5.\,1860 Budapest – 3.\,7.\,1904 Edlach@\textsc{Herzl, Theodor} (2.\,5.\,1860 Budapest – 3.\,7.\,1904 Edlach), \emph{Schriftsteller, Journalist}!neue Ghetto. Schauspiel in vier Acten@\strich\emph{Das neue Ghetto. Schauspiel in vier Acten}|pwv} rasch gemacht werden u. an Sie
               gehen, wie wir übereinkamen. Sie sagten mir noch nicht ob Ihnen mein Begleitbriefplan
               zusagt. Auch muss {\pb}ich Namen u. Adresse
               des Notars oder Advocaten Schick\pwindex{Schik, Friedrich *~6.\,9.\,1857 Wien@\textsc{Schik, Friedrich} (*~6.\,9.\,1857 Wien), \emph{Notar, Journalist, Dramaturg}|pw} wissen um ihn
               im Begleitbrief anzugeben. Sie müssen auch Schick\pwindex{Schik, Friedrich *~6.\,9.\,1857 Wien@\textsc{Schik, Friedrich} (*~6.\,9.\,1857 Wien), \emph{Notar, Journalist, Dramaturg}|pw} fragen, ob er geneigt ist.\pend
           
\pstart
           Ihre Antwort bitte ich als einfachen Brief rue de
                  Monceau\oindex{8, rue de Monceau@\textbf{8, rue de Monceau}, \emph{Wohngebäude}|pw} zu adressiren.\pend
           
\pstart
           Zur Vorsicht sprechen Sie von \strikeout{Ihrem} dem Stück\pwindex{Herzl, Theodor 2.\,5.\,1860 Budapest – 3.\,7.\,1904 Edlach@\textsc{Herzl, Theodor} (2.\,5.\,1860 Budapest – 3.\,7.\,1904 Edlach), \emph{Schriftsteller, Journalist}!neue Ghetto. Schauspiel in vier Acten@\strich\emph{Das neue Ghetto. Schauspiel in vier Acten}|pwv}, als wärs von Ihnen. Ich
               gebe Ihnen übrigens keine Detailcautelen für unsere Correspondenz an. Ich verlasse
               mich ganz auf Sie\pend
           
\pstart
           und bin mit herzlichen Grüssen{\\[\baselineskip]} Ihr Freund{\\[\baselineskip]}\spacefill\mbox{Herzl}\pend
           \leftskip=0em{}\selectlanguage{ngerman}\endnumbering\briefempfaengerindex{Schnitzler, Arthur@\textsc{Schnitzler, Arthur}!zzzHerzl, Theodor@\emph{von Theodor Herzl}!1894-11-271@{27. 11. 1894}|)be}\mylabel{L03837h}
\begin{anhang}
\end{anhang}\newcommand{\dateiname}{L03837}\newcommand{\titel}{Theodor Herzl an Arthur Schnitzler, 27. 11. 1894}\newcommand{\editorInnen}{Selma Jahnke und Martin Anton Müller}%% latex-leseansicht-abspann.tex
%% Abspann für die Leseansicht.
%% Der Schalter \ifkorrekturansicht ist bereits durch den Vorspann gesetzt.

%% latex-abspann.tex
%% Gemeinsamer Abspann für Korrekturansicht und Leseansicht.
%% Setzt den Schalter \ifkorrekturansicht voraus (gesetzt in den
%% einbindenden Dateien latex-korrekturansicht-abspann.tex bzw.
%% latex-leseansicht-abspann.tex).
%% ---------------------------------------------------------------

\normalsize

% Das esempio-Environment wird nur in der Leseansicht benötigt
\ifkorrekturansicht\else
\newenvironment{esempio}[3]%
{
    \vspace{1.5ex}
    \rlap{\underline{#1}}
    \par
    \setlength{\parindent}{0cm}
    \nopagebreak
    \leftskip=#2cm
    \rightskip=#3cm
}
{
    \par
}
\fi

\doendnotes{C}
\bigskip
\vfill

\clearpage

\footnotesize

\ifkorrekturansicht
  \lohead{\textsc{register}}
\fi

% theindex-Environment neu definieren ohne reledmac
\makeatletter
\renewenvironment{theindex}{%
  \ifkorrekturansicht
    \section*{\indexname}%
  \else
    \subsubsection*{Index der erwähnten Entitäten}%
  \fi
  \setlength{\parindent}{0pt}%
  \setlength{\parskip}{0pt plus 0.3pt}%
  \let\item\@idxitem
}{%
  \ifkorrekturansicht\clearpage\fi
}
\makeatother

\IfFileExists{\jobname-pw.ind}{\input{\jobname-pw.ind}}{}

% Quellenangabe nur in der Leseansicht
\ifkorrekturansicht\else
% Fallback-Definitionen, falls die .tex-Datei \titel etc. nicht gesetzt hat
\providecommand{\titel}{}
\providecommand{\editorInnen}{}
\providecommand{\dateiname}{\jobname}

\vspace{3cm}

\vfill

\footnotesize
\textsc{Quelle}: \titel. Herausgegeben von {\editorInnen}. In: \emph{Arthur Schnitzler: Briefwechsel mit Autorinnen und Autoren}.
 Digitale Edition, https://schnitzler-briefe.acdh.oeaw.ac.at/{\dateiname}.html (Stand \today)
\fi

\end{document}


