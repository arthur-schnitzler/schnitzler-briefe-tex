\input{../tex-inputs/latex-pdf-vorspann}
\begin{center}
            \textcolor{red}{ENTWURF. ENTZIFFERUNG NOCH NICHT KORREKTURGELESEN}
                      \end{center}
            
               \section[Hugo von Hofmannsthal an Arthur Schnitzler, 12. 6. 1912]{ Hugo von Hofmannsthal an Arthur Schnitzler, 12. 6. 1912}\nopagebreak\mylabel{v}\rehead{ }\begin{ledgroupsized}[t]{13cm}\normalsize\beginnumbering\briefempfaengerindex{Schnitzler, Arthur@\textsc{Schnitzler, Arthur}!zzzHofmannsthal, Hugo von@\emph{von Hugo von Hofmannsthal}!1912-06-121@{12. 6. 1912}|(be} \toendnotes[C]{\smallbreak\pagebreak[2]} \Standort{DLA, A:Schnitzler/Kopien, HS.NZ85.1.5726,1-2.}
\physDesc{Brief, 3 Blätter, 5 Seiten, Fotokopie
\newline{}Handschrift: schwarze Tinte, deutsche Kurrent
\newline{}Schnitzler: mit rotem Buntstift (?) beschriftet: »\textsc{Hugo}« }\buchAbdrucke{\weitereDrucke{Hugo von Hofmannsthal, Arthur Schnitzler: \emph{Briefwechsel}. Hg. Therese Nickl und Heinrich Schnitzler. Frankfurt am Main: \emph{S. Fischer} 1964, S. 265–267.} }\toendnotes[C]{\smallbreak}\pstart
           \raggedleft{}{\pb}Rodaun\oindex{Rodaun@\textbf{Rodaun}|pw}{ }12. VI 912\pend
           \pstart{}mein lieber Arthur\pend\pstart
           den \label{K_L02074_1v}\edtext{fünfzehnten Mai}{\lemma{\textnormal{\emph{fünfzehnten Mai}}}\Cendnote{\textnormal{Schnitzler\pwindex{Schnitzler, Arthur 15.05.1862 – 21.10.1931@\textsc{Schnitzler, Arthur} (15.05.1862 – 21.10.1931), \emph{Schriftsteller, Mediziner}|pwk}s 50. Geburtstag.}}}\label{K_L02074_1h}, von Perugia\oindex{Perugia@\textbf{Perugia}|pw} nach Rom\oindex{Rom@\textbf{Rom}|pw}
               fahrend, ſtundenlang ſtill neben dem Chauffeur\pwindex{?? [Chauffeur der Adlerwerke] 1912-04-30 – 1912-05-24@\textsc{?? [Chauffeur der Adlerwerke]} (1912-04-30 – 1912-05-24)|pwv}, habe ich mit rechter Herzlichkeit an Sie gedacht und aus den
               vielen Jahren unſerer Freundſchaft iſt unzählbar Vieles an mir vorübergeflogen,
               Augenblicke die Ihnen wohl entſchwunden ſind und in welchen mir Ihr Weſen oder wie
               ſoll ich’s nennen: das Gefühl des Lebens, vermittelt durch das Geſicht eines
               Menſchen, durch einen Blick aus den Augen des andern – ſehr nahe kam und die ich nie
               verlieren werde, ſolange ich lebe. Viele Menſchen ſind mir ſeitdem nahe geko{\geminationm}en, auch jetzt noch bin ich nicht abgeſtumpfter, nicht
               unempfänglicher für die Annäherung eines Menſchen, aber das kann mir wohl nie
               wiederkommen, was damals die Verknüpfung mit Ihnen und Richard\pwindex{Beer-Hofmann, Richard 11.07.1866 – 26.09.1945@\textsc{Beer-Hofmann, Richard} (11.07.1866 – 26.09.1945), \emph{Schriftsteller}|pw} zuerſt mir ſchenkte. Für mich \introOben{}vor allem\introOben{} war
               es ein Augenblick, deſſen {\pb}gleichen nie wiederkommen
               konnte. Frühreif und doch unendlich unerfahren trat ich aus der abſoluten Einſamkeit
               meiner frühen Jugend hervor – da waren Sie für mich nicht nur ein Menſch, ein Freund,
               ſondern eine neue Verknüpfung mit der Welt, Sie waren ſelbſt für mich eine ganze Welt
               – \strikeout{ſo} genug verwandt meiner eigenen, daſs ich alles
               darin leſen konnte wie ein ſchönes anziehendes Buch, genug fremd, daſs mich alles
               daran verwunderte, reizte, durch Geheimnis anzog, durch ſeine Miſchung von Trauer und
               Fröhlichkeit, von großer Schwere und geiſtiger Leichtigkeit bezauberte. Tauſende von
               Begegnungen haben ihr Gewicht in die gleiche Schale getan, Ihre Bücher ſind geko{\geminationm}en eins nach dem Anderen – und alles iſt geblieben wie
               in jenem erſten Jahr. Nie in dieſen zwanzig Jahren war es mir gleichgiltig Ihnen zu
               begegnen, nie habe ich mit Gleichgiltigkeit die Seiten in einem Ihrer Bücher
               umgewandt.\pend
           \pstart
           {\pb}Das große Glück und das unauflösliche Geheimnis, von
               einem Weſen, das zur gleichen Zeit lebt, gleichzeitig die rein geiſtige Einwirkung
               des Dichters und die menſchliche des Menſchen zu erfahren, – hinter jedem geiſtigen
               Product den Menſchen zu fühlen, deſſen Nähe mehr ſagt als die Zeilen enthalten
               können, – andererſeits das Hin- und Wieder des freundſchaftlichen Verkehrs, das dem
               Andern Abgeſchaute und Abgefühlte ſogleich in Kunſtwerken vergeiſtigt und erhöht
               wiederzufinden – dies iſt mir durch Sie widerfahren, und dies verbindet mich mit
               Ihnen in einer Weiſe die mir teuer iſt, ſo teuer daſs ich dies nicht in viele Worte
               auseinanderlegen könnte noch wollte, weder heute noch an einem ſpäteren Tag.\pend
           \pstart
           Meine Gedanken über dieſes Alles waren viel reicher an Umfang und an Tiefe, als ich
               es jetzt hier ausdrücken kann, aber eben darum war es mir ganz {\pb}unmöglich, ja ſelbſt in Gedanken fernliegend, Ihnen in
               eben dieſen Tagen zu ſchreiben. Ihrer Natur liegt alles Demonſtrative ſo fern, daſs
               Sie dies ohne weiteres verſtehen.\pend
           \pstart
           Hier her zurückgekommen, vor 5 Tagen, war das Packet von Fiſcher\pwindex{Fischer, Samuel 24.12.1859 – 15.10.1934@\textsc{Fischer, Samuel} (24.12.1859 – 15.10.1934), \emph{Verleger}|pw} mit Ihren erzählenden
                  Schriften\pwindex{Schnitzler, Arthur 15.05.1862 – 21.10.1931@\textsc{Schnitzler, Arthur} (15.05.1862 – 21.10.1931), \emph{Schriftsteller, Mediziner}!Erzaehlende Schriften1912 – 1912@\strich\emph{Erzählende Schriften} {[}1912 – 1912{]}|pw} das erſte, was mir in die Hand kam. Ich blätterte irgend einen Band
               auf, las da und dort eine halbe Seite, alles iſt mir ja ſo wohlbekannt, daſs ich die
               Erzählungen nach vorne und rückwärts im Flug ergänzte und alles berührte mich mit
               einer Vertrautheit als wäre es Ihr Geſicht das mir entgegenſähe und alles ſchien mir
               auch ſo unabgeschloſſen im ſchönen Sinn, ſo nach vorne und rückwärts deutend, ſo
               fragend und in mich hineinſchauend, wie ein Geſicht. Dann erſt ſchlug ich das
               vorderſte Blatt auf, das nun wirklich Ihr Geſicht\pwindex{\textcolor{red}{\textsuperscript{XXXX1 indx}}!Arthur Schnitzler 1911@\strich\emph{Arthur Schnitzler} {[}1911{]}|pwv} enthält, woran ich Tauſend kleine Züge habe ſich
               bilden, ſich vertiefen ſehen, und das dieſe Züge auf kleinem Raum ſo treu und
               gefühlvoll wiedergibt, und unverſehens ſtürzten mir {\pb}Thränen aus den Augen, ein Weinen ſeltener Art, woran gar nichts ſchmerzliches,
               ſondern nur etwas vielverknüpfendes war.\pend
           \pstart
           Wie leben Sie, mein lieber Arthur, und wo leben Sie? Seid Ihr hier – wie ich es hoffe
               – dann kommt jetzt bald einmal zu uns, laßt dieſes eine Mal im Jahr nicht auch aus
               unſeren Gebräuchen verſchwinden – \pend
           \pstart
           Ich wäre ſehr froh über eine Karte oder einen Anruf. Jeder Tag iſt uns recht.\pend
           \pstart Von Herzen Ihr\spacefill\mbox{Hugo.}\pend{}\endnumbering\briefempfaengerindex{Schnitzler, Arthur@\textsc{Schnitzler, Arthur}!zzzHofmannsthal, Hugo von@\emph{von Hugo von Hofmannsthal}!1912-06-121@{12. 6. 1912}|)be}\mylabel{h}\end{ledgroupsized}  \newcommand{\dateiname}{L02074}\newcommand{\titel}{Hugo von Hofmannsthal an Arthur Schnitzler, 12. 6. 1912}\newcommand{\editorInnen}{Martin Anton Müller und Gerd-Hermann Susen}\input{../tex-inputs/latex-pdf-abspann}
      