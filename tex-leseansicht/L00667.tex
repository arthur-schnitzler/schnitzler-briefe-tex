%% latex-korrekturansicht-vorspann.tex
%% Vorspann für die Korrekturansicht.
%% Lädt die gemeinsame Datei latex-vorspann.tex mit gesetztem Schalter.

\newif\ifkorrekturansicht
\korrekturansichttrue

\input{../tex-inputs/latex-vorspann}


\section[Richard Beer-Hofmann an Arthur Schnitzler, 21. 4. 1897]{L00667 Richard Beer-Hofmann an Arthur Schnitzler, 21. 4. 1897}
\nopagebreak\mylabel{L00667v}
\rehead{ }\normalsize\beginnumbering\briefempfaengerindex{Schnitzler, Arthur@\textsc{Schnitzler, Arthur}!zzzBeer-Hofmann, Richard@\emph{von Richard Beer-Hofmann}!1897-04-211@{21. 4. 1897}|(be}
\toendnotes[C]{\smallbreak\pagebreak[2]}\Standort{CUL, Schnitzler, B 8.}
\physDesc{Brief, 1 Blatt, 2 Seiten, 735 Zeichen
\newline{}Handschrift: Bleistift, lateinische Kurrent
\newline{}Schnitzler: mit Bleistift die Jahreszahl ergänzt: »97« 
\newline{}Ordnung: mit Bleistift von unbekannter Hand nummeriert: »94« }
\buchAbdrucke{\weitereDrucke{Arthur Schnitzler, Richard Beer-Hofmann: \emph{Briefwechsel 1891–1931}. Wien, Zürich: \emph{Europaverlag} 1992, S. 102.} }\toendnotes[C]{\smallbreak}
\pstart
           \raggedleft{}{\pb}Wien\oindex{Wien@\textbf{Wien}, \emph{A.ADM2}|pw}{ }21/IV{\\}½ 12 Nachts{\\}im Caffée.\pend
           
\pstart{}Lieber Arthur!\pend\vspace{0.5em}
\pstart
           Ich hab heute Ihren Brief beko{\geminationm}en. Ich habe noch nie
               einen Menschen gesehen, der sich so sehr schämt sich einzugestehn {\pb}daß er sich wolfühlt. No ja – es
               geht Ihnen eben gut; sagen Sie »Unberufen« und gestehen Sie es sich ein.\pend
           
\pstart
           Hier nichts Neues; nur Zaccone\pwindex{Zacconi, Ermete 14.09.1857 – 14.10.1948@\textsc{Zacconi, Ermete} (14.09.1857 – 14.10.1948), \emph{Regisseur/Regisseurin, Schauspieler/Schauspielerin}|pw} – ein
               Schauspieler den ich von Rom\oindex{Rom@\textbf{Rom}, \emph{P.PPLC}|pw} aus kannte. {\pb}Ein ganz Großer. »Techniker«
               schreien die Leute die nicht einmal Technik haben\pend
           
\pstart
           Ich arbeite. \label{K_L00667-1v}\edtext{Salten\pwindex{Salten, Felix 06.09.1869 – 08.10.1945@\textsc{Salten, Felix} (06.09.1869 – 08.10.1945), \emph{Schriftsteller/Schriftstellerin, Journalist/Journalistin, Chefredakteur/Chefredakteurin}|pw} ist seit Tagen ich weiß nicht wo mit ich weiß nicht
                  wem}{\lemma{\textnormal{\emph{Salten … wem}}}\Cendnote{\textnormal{Vgl. Arthur Schnitzler an Felix Salten, 26. 4. 1897 und Felix Salten an Arthur Schnitzler, 5. 5. 1897.
               }}}\label{K_L00667-1}. Georg
                  Hirschfeld\pwindex{Hirschfeld, Georg 11.02.1873 – 17.01.1942@\textsc{Hirschfeld, Georg} (11.02.1873 – 17.01.1942), \emph{Schriftsteller/Schriftstellerin}|pw} unsichtbar. Schreiben {\pb}Sie bald den verheißenen
               »wirklichen Brief«. Ich grüße von Herzen Paul\pwindex{Goldmann, Paul 31.01.1865 – 25.09.1935@\textsc{Goldmann, Paul} (31.01.1865 – 25.09.1935), \emph{Schriftsteller/Schriftstellerin, Journalist/Journalistin}|pw};
               er soll aus der Tatsache daß ich Ihnen schreibe keine Folgerungen für mein
               schreibfaules Verhältniß zu ihm ableiten. Herzlichst\pend
           \pstart \spacefill\mbox{Richard}\pend{}\selectlanguage{ngerman}\endnumbering\briefempfaengerindex{Schnitzler, Arthur@\textsc{Schnitzler, Arthur}!zzzBeer-Hofmann, Richard@\emph{von Richard Beer-Hofmann}!1897-04-211@{21. 4. 1897}|)be}\mylabel{L00667h}  \normalsize

\doendnotes{C}
\bigskip
\vfill

\clearpage

\footnotesize

\lohead{\textsc{register}}

% Definiere theindex-Environment komplett neu ohne reledmac
\makeatletter
\renewenvironment{theindex}{%
  \section*{\indexname}%
  \setlength{\parindent}{0pt}%
  \setlength{\parskip}{0pt plus 0.3pt}%
  \let\item\@idxitem
}{%
  \clearpage
}
\makeatother

\IfFileExists{\jobname-pw.ind}{\input{\jobname-pw.ind}}{}

\end{document}

      