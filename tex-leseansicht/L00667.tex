%% latex-leseansicht-vorspann.tex
%% Vorspann für die Leseansicht.
%% Lädt die gemeinsame Datei latex-vorspann.tex mit nicht gesetztem Schalter.

\newif\ifkorrekturansicht
\korrekturansichtfalse

\input{../tex-inputs/latex-vorspann}


\section[Richard Beer-Hofmann an Arthur Schnitzler, 21. 4. 1897]{L00667 Richard Beer-Hofmann an Arthur Schnitzler, 21. 4. 1897}
\nopagebreak\mylabel{L00667v}
\rehead{ }\normalsize\beginnumbering\briefempfaengerindex{Schnitzler, Arthur@\textsc{Schnitzler, Arthur}!zzzBeer-Hofmann, Richard@\emph{von Richard Beer-Hofmann}!1897-04-211@{21. 4. 1897}|(be}
\toendnotes[C]{\smallbreak\pagebreak[2]}
\correspDesc{Versand  durch Richard Beer-Hofmann am 21. 4. 1897 in Wien
\newline{}Erhalt  durch Arthur Schnitzler im Zeitraum [22. 4. 1897
                  – 26. 4. 1897?] in Paris}\toendnotes[C]{\smallbreak}
\Standort{CUL, Schnitzler, B 8.}
\physDesc{Brief, 1 Blatt, 2 Seiten, 735 Zeichen
\newline{}Handschrift: Bleistift, lateinische Kurrent
\newline{}Schnitzler: mit Bleistift die Jahreszahl ergänzt: »97« 
\newline{}Ordnung: mit Bleistift von unbekannter Hand nummeriert: »94« }
\buchAbdrucke{\weitereDrucke{Arthur Schnitzler, Richard Beer-Hofmann: \emph{Briefwechsel 1891–1931}. Herausgegeben von Konstanze Fliedl. Wien, Zürich: \emph{Europaverlag} 1992, S. 102.} }\toendnotes[C]{\smallbreak}
\pstart
           \raggedleft{}{\pb}Wien\oindex{Wien@\textbf{Wien}, \emph{Verwaltungsgebiet}|pw}{ }21/IV{\\}½ 12 Nachts{\\}im Caffée.\pend
           
\pstart{}Lieber Arthur!\pend\vspace{0.5em}
\pstart
           Ich hab heute Ihren Brief beko{\geminationm}en. Ich habe noch nie
               einen Menschen gesehen, der sich so sehr schämt sich einzugestehn {\pb}daß er sich wolfühlt. No ja – es
               geht Ihnen eben gut; sagen Sie »Unberufen« und gestehen Sie es sich ein.\pend
           
\pstart
           Hier nichts Neues; nur Zaccone\pwindex{Zacconi, Ermete 14.\,9.\,1857 Montecchio Emilia – 14.\,10.\,1948 Viareggio@\textsc{Zacconi, Ermete} (14.\,9.\,1857 Montecchio Emilia – 14.\,10.\,1948 Viareggio), \emph{Regisseur, Schauspieler}|pw} – ein
               Schauspieler den ich von Rom\oindex{Rom@\textbf{Rom}, \emph{Hauptstadt}|pw} aus kannte. {\pb}Ein ganz Großer. »Techniker«
               schreien die Leute die nicht einmal Technik haben\pend
           
\pstart
           Ich arbeite. \label{K_L00667-1v}\edtext{Salten\pwindex{Salten, Felix 6.\,9.\,1869 Budapest – 8.\,10.\,1945 Zürich@\textsc{Salten, Felix} (6.\,9.\,1869 Budapest – 8.\,10.\,1945 Zürich), \emph{Schriftsteller, Journalist, Chefredakteur}|pw} ist seit Tagen ich weiß nicht wo mit ich weiß nicht
                  wem}{\lemma{\textnormal{\emph{Salten … wem}}}\Cendnote{\textnormal{Vgl. XXXX Auszeichnungsfehler: Dokument L02963 nicht gefunden und XXXX Auszeichnungsfehler: Dokument L03264 nicht gefunden.
               }}}\label{K_L00667-1}. Georg
                  Hirschfeld\pwindex{Hirschfeld, Georg 11.\,2.\,1873 Berlin – 17.\,1.\,1942 München@\textsc{Hirschfeld, Georg} (11.\,2.\,1873 Berlin – 17.\,1.\,1942 München), \emph{Schriftsteller}|pw} unsichtbar. Schreiben {\pb}Sie bald den verheißenen
               »wirklichen Brief«. Ich grüße von Herzen Paul\pwindex{Goldmann, Paul 31.\,1.\,1865 Breslau – 25.\,9.\,1935 Wien@\textsc{Goldmann, Paul} (31.\,1.\,1865 Breslau – 25.\,9.\,1935 Wien), \emph{Schriftsteller, Journalist}|pw};
               er soll aus der Tatsache daß ich Ihnen schreibe keine Folgerungen für mein
               schreibfaules Verhältniß zu ihm ableiten. Herzlichst\pend
           \pstart \spacefill\mbox{Richard}\pend{}\selectlanguage{ngerman}\endnumbering\briefempfaengerindex{Schnitzler, Arthur@\textsc{Schnitzler, Arthur}!zzzBeer-Hofmann, Richard@\emph{von Richard Beer-Hofmann}!1897-04-211@{21. 4. 1897}|)be}\mylabel{L00667h}  \newcommand{\dateiname}{L00667}\newcommand{\titel}{Richard Beer-Hofmann an Arthur Schnitzler, 21. 4. 1897}\newcommand{\editorInnen}{Martin Anton Müller und Gerd-Hermann Susen}%% latex-leseansicht-abspann.tex
%% Abspann für die Leseansicht.
%% Der Schalter \ifkorrekturansicht ist bereits durch den Vorspann gesetzt.

%% latex-abspann.tex
%% Gemeinsamer Abspann für Korrekturansicht und Leseansicht.
%% Setzt den Schalter \ifkorrekturansicht voraus (gesetzt in den
%% einbindenden Dateien latex-korrekturansicht-abspann.tex bzw.
%% latex-leseansicht-abspann.tex).
%% ---------------------------------------------------------------

\normalsize

% Das esempio-Environment wird nur in der Leseansicht benötigt
\ifkorrekturansicht\else
\newenvironment{esempio}[3]%
{
    \vspace{1.5ex}
    \rlap{\underline{#1}}
    \par
    \setlength{\parindent}{0cm}
    \nopagebreak
    \leftskip=#2cm
    \rightskip=#3cm
}
{
    \par
}
\fi

\doendnotes{C}
\bigskip
\vfill

\clearpage

\footnotesize

\ifkorrekturansicht
  \lohead{\textsc{register}}
\fi

% theindex-Environment neu definieren ohne reledmac
\makeatletter
\renewenvironment{theindex}{%
  \ifkorrekturansicht
    \section*{\indexname}%
  \else
    \subsubsection*{Index der erwähnten Entitäten}%
  \fi
  \setlength{\parindent}{0pt}%
  \setlength{\parskip}{0pt plus 0.3pt}%
  \let\item\@idxitem
}{%
  \ifkorrekturansicht\clearpage\fi
}
\makeatother

\IfFileExists{\jobname-pw.ind}{\input{\jobname-pw.ind}}{}

% Quellenangabe nur in der Leseansicht
\ifkorrekturansicht\else
% Fallback-Definitionen, falls die .tex-Datei \titel etc. nicht gesetzt hat
\providecommand{\titel}{}
\providecommand{\editorInnen}{}
\providecommand{\dateiname}{\jobname}

\vspace{3cm}

\vfill

\footnotesize
\textsc{Quelle}: \titel. Herausgegeben von {\editorInnen}. In: \emph{Arthur Schnitzler: Briefwechsel mit Autorinnen und Autoren}.
 Digitale Edition, https://schnitzler-briefe.acdh.oeaw.ac.at/{\dateiname}.html (Stand \today)
\fi

\end{document}


