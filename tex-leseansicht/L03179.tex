%% latex-korrekturansicht-vorspann.tex
%% Vorspann für die Korrekturansicht.
%% Lädt die gemeinsame Datei latex-vorspann.tex mit gesetztem Schalter.

\newif\ifkorrekturansicht
\korrekturansichttrue

\input{../tex-inputs/latex-vorspann}


\section[ Felix Salten an Arthur Schnitzler, {[}27. 8. 1896{]}]{L03179 Felix Salten an Arthur Schnitzler, {[}27. 8. 1896{]}}
\nopagebreak\mylabel{L03179v}
\rehead{ }\normalsize\beginnumbering\briefempfaengerindex{Schnitzler, Arthur@\textsc{Schnitzler, Arthur}!zzzSalten, Felix@\emph{von Felix Salten}!1896-08-271@{{[}27. 8. 1896{]}}|(be}
\toendnotes[C]{\smallbreak\pagebreak[2]}\Standort{CUL, Schnitzler, B 89, A 1.}
\physDesc{Brief, 1 Blatt, 1 Seite, 348 Zeichen
\newline{}Handschrift: schwarze Tinte, lateinische Kurrent
\newline{}Schnitzler: mit Bleistift datiert: »2\substVorne{}\textsuperscript{9}\substDazwischen{}7\substHinten{}/8 96« 
\newline{}Ordnung: mit Bleistift von unbekannter Hand nummeriert: »78« }\toendnotes[C]{\smallbreak}
\pstart
           \raggedleft{}{\pb}Donnerstag.\pend
           \vspace{0.5em}
\pstart
           Lieber Freund, ich bin seit heute{ }hier\oindex{Wien@\textbf{Wien}, \emph{A.ADM2}|pwv}, und freue mich sehr, Sie
               recht \label{K_L03179-1v}\edtext{bald wieder zu sehen}{\lemma{\textnormal{\emph{bald wieder zu sehen}}}\Cendnote{\textnormal{Die beiden sahen sich bereits am Tag von Schnitzlers Rückkehr, am 29. 8. 1896 wieder.}}}\label{K_L03179-1}. Es gibt Vieles zu erzählen. Das \label{K_L03179-2v}\edtext{»Freiwild\pwindex{Freiwild. Schauspiel in 3 Akten@\emph{Freiwild. Schauspiel in 3 Akten}|pw}« bekomme ich
               doch zu hören}{\lemma{\textnormal{\emph{»Freiwild« … hören}}}\Cendnote{\textnormal{Schnitzler hatte Salten\pwindex{Salten, Felix 06.09.1869 – 08.10.1945@\textsc{Salten, Felix} (06.09.1869 – 08.10.1945), \emph{Schriftsteller/Schriftstellerin, Journalist/Journalistin, Chefredakteur/Chefredakteurin}|pwk} bereits am 3. 5. 1896 aus \emph{Freiwild}\pwindex{Freiwild. Schauspiel in 3 Akten@\emph{Freiwild. Schauspiel in 3 Akten}|pwk} vorgelesen.}}}\label{K_L03179-2}, nicht? Ich werde mich dafür
               revanchiren. Nach \label{K_L03179-3v}\edtext{Berlin\oindex{Berlin@\textbf{Berlin}, \emph{P.PPLC}|pw}}{\lemma{\textnormal{\emph{Berlin}}}\Cendnote{\textnormal{Schnitzler war zwischen 22. 8. 1896 und 26. 8. 1896 – auf dem
                  Rückweg von seiner Skandinavien\oindex{Skandinavien@\textbf{Skandinavien}, \emph{Region}|pwk}reise – in Berlin\oindex{Berlin@\textbf{Berlin}, \emph{P.PPLC}|pwk} gewesen.}}}\label{K_L03179-3} konnte ich Ihnen nichts
               mehr schreiben, ich hatte Ihre Karte verlegt\textcolor{gray}{,} und wusste keine
               Adreße.\pend
           
\pstart
           Also auf bald, {\\[\baselineskip]}herzlichst Ihr {\\[\baselineskip]}\spacefill\mbox{Salten}\pend
           \leftskip=0em{}\selectlanguage{ngerman}\endnumbering\briefempfaengerindex{Schnitzler, Arthur@\textsc{Schnitzler, Arthur}!zzzSalten, Felix@\emph{von Felix Salten}!1896-08-271@{{[}27. 8. 1896{]}}|)be}\mylabel{L03179h}  \normalsize

\doendnotes{C}
\bigskip
\vfill

\clearpage

\footnotesize

\lohead{\textsc{register}}

% Definiere theindex-Environment komplett neu ohne reledmac
\makeatletter
\renewenvironment{theindex}{%
  \section*{\indexname}%
  \setlength{\parindent}{0pt}%
  \setlength{\parskip}{0pt plus 0.3pt}%
  \let\item\@idxitem
}{%
  \clearpage
}
\makeatother

\IfFileExists{\jobname-pw.ind}{\input{\jobname-pw.ind}}{}

\end{document}

      