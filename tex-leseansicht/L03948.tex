%% latex-leseansicht-vorspann.tex
%% Vorspann für die Leseansicht.
%% Lädt die gemeinsame Datei latex-vorspann.tex mit nicht gesetztem Schalter.

\newif\ifkorrekturansicht
\korrekturansichtfalse

\input{../tex-inputs/latex-vorspann}


\section[Arthur Schnitzler an Berta Zuckerkandl, 29. 10. 1923]{L03948 Arthur Schnitzler an Berta Zuckerkandl, 29. 10. 1923}
\nopagebreak\mylabel{L03948v}
\rehead{ }\normalsize\beginnumbering\briefempfaengerindex{Zuckerkandl, Berta@\textsc{Zuckerkandl, Berta}!zzzSchnitzler, Arthur@\emph{von Arthur Schnitzler}!1923-10-291@{29. 10. 1923}|(be}
\toendnotes[C]{\smallbreak\pagebreak[2]}
\correspDesc{Versand  durch Arthur Schnitzler am 29. 10. 1923 in Wien
\newline{}Erhalt  durch Berta Zuckerkandl im Zeitraum [30. 10. 1923 – 3. 11. 1923?] in Paris}\toendnotes[C]{\smallbreak}
\Standort{DLA, HS.1985.1.2282.}
\physDesc{Brief, Durchschlag, 1 Blatt, 2 Seiten, 2150 Zeichen
\newline{}Schreibmaschine
\newline{}Handschrift: 1) roter Buntstift, lateinische Kurrent (\noindent{}beschriftet: »\uline{Zuckerkandl}« und »Frankr«, neun Unterstreichungen)\hspace{1em}2) Bleistift, lateinische Kurrent (\noindent{}eine Ergänzung)\hspace{1em}}\toendnotes[C]{\smallbreak}
\pstart
           \raggedleft{}{\pb}29\strikeout{9}. 10. 1923.\pend
           
\pstart{}Liebe und verehrte Frau Hofrätin.\pend\vspace{0.5em}
\pstart
           Meinem \label{K_L03948-1v}\edtext{letzten Schreiben}{\lemma{\textnormal{\emph{letzten Schreiben}}}\Cendnote{\textnormal{Siehe XXXX Auszeichnungsfehler: Dokument L03947 nicht gefunden.}}}\label{K_L03948-1} über die
               Zusammenstellung des französischen\oindex{Frankreich@\textbf{Frankreich}|pw} Einakterbandes
               lasse ich heute meinen Vorschlag bezüglich des \label{K_L03948-2v}\edtext{Novellenbandes\pwindex{Schnitzler, Arthur 15. 5. 1862 Wien – 21. 10. 1931 ebd.@\textsc{Schnitzler, Arthur} (15. 5. 1862 Wien – 21. 10. 1931 ebd.), \emph{Schriftsteller, Mediziner}!pénombre des âmes@\strich\emph{La pénombre des âmes}|pwv}}{\lemma{\textnormal{\emph{Novellenbandes}}}\Cendnote{\textnormal{Erst 1930 erschien die
                  Novellensammlung \emph{La pénombre des âmes}\pwindex{Schnitzler, Arthur 15. 5. 1862 Wien – 21. 10. 1931 ebd.@\textsc{Schnitzler, Arthur} (15. 5. 1862 Wien – 21. 10. 1931 ebd.), \emph{Schriftsteller, Mediziner}!pénombre des âmes@\strich\emph{La pénombre des âmes}|pwk} bei \emph{Éditions Stock}\orgindex{Éditions Stock@Éditions Stock|pwk}, die bis auf \emph{Die Hirtenflöte}\pwindex{Schnitzler, Arthur 15. 5. 1862 Wien – 21. 10. 1931 ebd.@\textsc{Schnitzler, Arthur} (15. 5. 1862 Wien – 21. 10. 1931 ebd.), \emph{Schriftsteller, Mediziner}!Hirtenflöte. Novelle@\strich\emph{Die Hirtenflöte. Novelle}|pwk} alle in diesem Brief genannten Novellen\pwindex{Schnitzler, Arthur 15. 5. 1862 Wien – 21. 10. 1931 ebd.@\textsc{Schnitzler, Arthur} (15. 5. 1862 Wien – 21. 10. 1931 ebd.), \emph{Schriftsteller, Mediziner}!Schicksal des Freiherrn von Leisenbohg. Novellette@\strich\emph{Das Schicksal des Freiherrn von Leisenbohg. Novellette}|pwkv}\pwindex{Schnitzler, Arthur 15. 5. 1862 Wien – 21. 10. 1931 ebd.@\textsc{Schnitzler, Arthur} (15. 5. 1862 Wien – 21. 10. 1931 ebd.), \emph{Schriftsteller, Mediziner}!Mörder. Novelle@\strich\emph{Der Mörder. Novelle}|pwkv}\pwindex{Schnitzler, Arthur 15. 5. 1862 Wien – 21. 10. 1931 ebd.@\textsc{Schnitzler, Arthur} (15. 5. 1862 Wien – 21. 10. 1931 ebd.), \emph{Schriftsteller, Mediziner}!Toten schweigen@\strich\emph{Die Toten schweigen}|pwkv}\pwindex{Schnitzler, Arthur 15. 5. 1862 Wien – 21. 10. 1931 ebd.@\textsc{Schnitzler, Arthur} (15. 5. 1862 Wien – 21. 10. 1931 ebd.), \emph{Schriftsteller, Mediziner}!Ehrentag@\strich\emph{Der Ehrentag}|pwkv}\pwindex{Schnitzler, Arthur 15. 5. 1862 Wien – 21. 10. 1931 ebd.@\textsc{Schnitzler, Arthur} (15. 5. 1862 Wien – 21. 10. 1931 ebd.), \emph{Schriftsteller, Mediziner}!Tod des Junggesellen. Novelle@\strich\emph{Der Tod des Junggesellen. Novelle}|pwkv}\pwindex{Schnitzler, Arthur 15. 5. 1862 Wien – 21. 10. 1931 ebd.@\textsc{Schnitzler, Arthur} (15. 5. 1862 Wien – 21. 10. 1931 ebd.), \emph{Schriftsteller, Mediziner}!blinde Geronimo und sein Bruder@\strich\emph{Der blinde Geronimo und sein Bruder}|pwkv}
                  und vier\pwindex{Schnitzler, Arthur 15. 5. 1862 Wien – 21. 10. 1931 ebd.@\textsc{Schnitzler, Arthur} (15. 5. 1862 Wien – 21. 10. 1931 ebd.), \emph{Schriftsteller, Mediziner}!Blumen@\strich\emph{Blumen}|pwkv}\pwindex{Schnitzler, Arthur 15. 5. 1862 Wien – 21. 10. 1931 ebd.@\textsc{Schnitzler, Arthur} (15. 5. 1862 Wien – 21. 10. 1931 ebd.), \emph{Schriftsteller, Mediziner}!Tagebuch der Redegonda@\strich\emph{Das Tagebuch der Redegonda}|pwkv}\pwindex{Schnitzler, Arthur 15. 5. 1862 Wien – 21. 10. 1931 ebd.@\textsc{Schnitzler, Arthur} (15. 5. 1862 Wien – 21. 10. 1931 ebd.), \emph{Schriftsteller, Mediziner}!tote Gabriel. Novelle@\strich\emph{Der tote Gabriel. Novelle}|pwkv}\pwindex{Schnitzler, Arthur 15. 5. 1862 Wien – 21. 10. 1931 ebd.@\textsc{Schnitzler, Arthur} (15. 5. 1862 Wien – 21. 10. 1931 ebd.), \emph{Schriftsteller, Mediziner}!Frau des Weisen. Erzählung@\strich\emph{Die Frau des Weisen. Erzählung}|pwkv} weitere enthält.}}}\label{K_L03948-2}
               folgen. Mein \label{K_L03948-3v}\edtext{Brief an Geraldy\pwindex{Géraldy, Paul 6.\,3.\,1885 Paris – 9.\,3.\,1983 Neuilly-sur-Seine@\textsc{Géraldy, Paul} (6.\,3.\,1885 Paris – 9.\,3.\,1983 Neuilly-sur-Seine), \emph{Schriftsteller}|pw}}{\lemma{\textnormal{\emph{Brief an Geraldy}}}\Cendnote{\textnormal{Arthug Schnitzler an Paul Géraldy\pwindex{Géraldy, Paul 6.\,3.\,1885 Paris – 9.\,3.\,1983 Neuilly-sur-Seine@\textsc{Géraldy, Paul} (6.\,3.\,1885 Paris – 9.\,3.\,1983 Neuilly-sur-Seine), \emph{Schriftsteller}|pwk}, 18. 4. 1922, \emph{Deutsches Literaturarchiv Marbach},
                     HS.1985.1.811,1.}}}\label{K_L03948-3} vom
                  18. 4. 1922 ist offenbar verloren gegangen. Ich hatte damals mit Ihnen, verehrteste Frau Hofrätin
               als die vor allem in Betracht für einen solchen Band\pwindex{Schnitzler, Arthur 15. 5. 1862 Wien – 21. 10. 1931 ebd.@\textsc{Schnitzler, Arthur} (15. 5. 1862 Wien – 21. 10. 1931 ebd.), \emph{Schriftsteller, Mediziner}!pénombre des âmes@\strich\emph{La pénombre des âmes}|pwv} kommenden Novellen die folgenden bezeichnet: »Die Hir{[}t{]}enflöte\pwindex{Schnitzler, Arthur 15. 5. 1862 Wien – 21. 10. 1931 ebd.@\textsc{Schnitzler, Arthur} (15. 5. 1862 Wien – 21. 10. 1931 ebd.), \emph{Schriftsteller, Mediziner}!Hirtenflöte. Novelle@\strich\emph{Die Hirtenflöte. Novelle}|pw}«, »Das Schicksal des Freiherrn von Leisenbogh\pwindex{Schnitzler, Arthur 15. 5. 1862 Wien – 21. 10. 1931 ebd.@\textsc{Schnitzler, Arthur} (15. 5. 1862 Wien – 21. 10. 1931 ebd.), \emph{Schriftsteller, Mediziner}!Schicksal des Freiherrn von Leisenbohg. Novellette@\strich\emph{Das Schicksal des Freiherrn von Leisenbohg. Novellette}|pw}«, »Der Mörder\pwindex{Schnitzler, Arthur 15. 5. 1862 Wien – 21. 10. 1931 ebd.@\textsc{Schnitzler, Arthur} (15. 5. 1862 Wien – 21. 10. 1931 ebd.), \emph{Schriftsteller, Mediziner}!Mörder. Novelle@\strich\emph{Der Mörder. Novelle}|pw}«, »Die
                  Toten schweigen\pwindex{Schnitzler, Arthur 15. 5. 1862 Wien – 21. 10. 1931 ebd.@\textsc{Schnitzler, Arthur} (15. 5. 1862 Wien – 21. 10. 1931 ebd.), \emph{Schriftsteller, Mediziner}!Toten schweigen@\strich\emph{Die Toten schweigen}|pw}«, »Ehrentag\pwindex{Schnitzler, Arthur 15. 5. 1862 Wien – 21. 10. 1931 ebd.@\textsc{Schnitzler, Arthur} (15. 5. 1862 Wien – 21. 10. 1931 ebd.), \emph{Schriftsteller, Mediziner}!Ehrentag@\strich\emph{Der Ehrentag}|pw}«. \introOben{}Tod des Junggesellen\pwindex{Schnitzler, Arthur 15. 5. 1862 Wien – 21. 10. 1931 ebd.@\textsc{Schnitzler, Arthur} (15. 5. 1862 Wien – 21. 10. 1931 ebd.), \emph{Schriftsteller, Mediziner}!Tod des Junggesellen. Novelle@\strich\emph{Der Tod des Junggesellen. Novelle}|pw}, Geronimo\pwindex{Schnitzler, Arthur 15. 5. 1862 Wien – 21. 10. 1931 ebd.@\textsc{Schnitzler, Arthur} (15. 5. 1862 Wien – 21. 10. 1931 ebd.), \emph{Schriftsteller, Mediziner}!blinde Geronimo und sein Bruder@\strich\emph{Der blinde Geronimo und sein Bruder}|pw}\introOben{}\pend
           
\pstart
           Was frühere Autorisationen anbelangt, wäre zu
               bemerken, dass der »\label{K_L03948-4v}\edtext{Blinde Geronimo\pwindex{Schnitzler, Arthur 15. 5. 1862 Wien – 21. 10. 1931 ebd.@\textsc{Schnitzler, Arthur} (15. 5. 1862 Wien – 21. 10. 1931 ebd.), \emph{Schriftsteller, Mediziner}!aveugle et son frère@\strich\emph{L’aveugle et son frère}|pwv}\pwindex{Schnitzler, Arthur 15. 5. 1862 Wien – 21. 10. 1931 ebd.@\textsc{Schnitzler, Arthur} (15. 5. 1862 Wien – 21. 10. 1931 ebd.), \emph{Schriftsteller, Mediziner}!blinde Geronimo und sein Bruder@\strich\emph{Der blinde Geronimo und sein Bruder}|pw}}{\lemma{\textnormal{\emph{Blinde Geronimo}}}\Cendnote{\textnormal{Arthur Schnitzler: \emph{L’aveugle et son frère}\pwindex{Schnitzler, Arthur 15. 5. 1862 Wien – 21. 10. 1931 ebd.@\textsc{Schnitzler, Arthur} (15. 5. 1862 Wien – 21. 10. 1931 ebd.), \emph{Schriftsteller, Mediziner}!aveugle et son frère@\strich\emph{L’aveugle et son frère}|pwk}. Traduction de Maurice Rémon\pwindex{Rémon, Maurice 27.\,11.\,1861 Paris – 20.\,6.\,1945 Mérignac@\textsc{Rémon, Maurice} (27.\,11.\,1861 Paris – 20.\,6.\,1945 Mérignac), \emph{Übersetzer}|pwk} et Wilhelm Bauer\pwindex{Bauer, Wilhelm 27.\,11.\,1854 Zollingen – 11.\,9.\,1923 Paris@\textsc{Bauer, Wilhelm} (27.\,11.\,1854 Zollingen – 11.\,9.\,1923 Paris)|pwk}. In: \emph{La
                        Grande Revue}\pwindex{Grande Revue@\emph{La Grande Revue}|pwk}, Jg. 16, Nr. 8, 25. 4. 1912,
                     S. 749–773.}}}\label{K_L03948-4}«, übersetzt von Rémon\pwindex{Rémon, Maurice 27.\,11.\,1861 Paris – 20.\,6.\,1945 Mérignac@\textsc{Rémon, Maurice} (27.\,11.\,1861 Paris – 20.\,6.\,1945 Mérignac), \emph{Übersetzer}|pw} und Bauer\pwindex{Bauer, Wilhelm 27.\,11.\,1854 Zollingen – 11.\,9.\,1923 Paris@\textsc{Bauer, Wilhelm} (27.\,11.\,1854 Zollingen – 11.\,9.\,1923 Paris)|pw} am
                  25. 4. 1912 in der Grande
               Revue\pwindex{Grande Revue@\emph{La Grande Revue}|pw}, der »\label{K_L03948-5v}\edtext{Ehrentag}{\lemma{\textnormal{\emph{Ehrentag}}}\Cendnote{\textnormal{Arthur Schnitzler: \emph{Jour de gloire}\pwindex{Schnitzler, Arthur 15. 5. 1862 Wien – 21. 10. 1931 ebd.@\textsc{Schnitzler, Arthur} (15. 5. 1862 Wien – 21. 10. 1931 ebd.), \emph{Schriftsteller, Mediziner}!Jour de gloire@\strich\emph{Jour de gloire}|pwk}. Traduit de l'allemand par N. Valentin\pwindex{Valentin, Noémi @\textsc{Valentin, Noémi}, \emph{Übersetzerin}|pwk} et M. Rémon\pwindex{Rémon, Maurice 27.\,11.\,1861 Paris – 20.\,6.\,1945 Mérignac@\textsc{Rémon, Maurice} (27.\,11.\,1861 Paris – 20.\,6.\,1945 Mérignac), \emph{Übersetzer}|pwk}. In: \emph{La
                              Revue de Paris}\pwindex{Schnitzler, Arthur 15. 5. 1862 Wien – 21. 10. 1931 ebd.@\textsc{Schnitzler, Arthur} (15. 5. 1862 Wien – 21. 10. 1931 ebd.), \emph{Schriftsteller, Mediziner}!Ehrentag@\strich\emph{Der Ehrentag}|pwk}, Jg. 10, Nr. 8, 1. 8. 1903,
                           S. 569–589.}}}\label{K_L03948-5}\pwindex{Schnitzler, Arthur 15. 5. 1862 Wien – 21. 10. 1931 ebd.@\textsc{Schnitzler, Arthur} (15. 5. 1862 Wien – 21. 10. 1931 ebd.), \emph{Schriftsteller, Mediziner}!Ehrentag@\strich\emph{Der Ehrentag}|pw}\pwindex{Schnitzler, Arthur 15. 5. 1862 Wien – 21. 10. 1931 ebd.@\textsc{Schnitzler, Arthur} (15. 5. 1862 Wien – 21. 10. 1931 ebd.), \emph{Schriftsteller, Mediziner}!Jour de gloire@\strich\emph{Jour de gloire}|pwv}«, übersetzt von Vallentin\pwindex{Valentin, Noémi @\textsc{Valentin, Noémi}, \emph{Übersetzerin}|pw} und Rémon\pwindex{Rémon, Maurice 27.\,11.\,1861 Paris – 20.\,6.\,1945 Mérignac@\textsc{Rémon, Maurice} (27.\,11.\,1861 Paris – 20.\,6.\,1945 Mérignac), \emph{Übersetzer}|pw} am 1. 8. 1903 in der Revue de Paris\pwindex{Revue de Paris@\emph{La Revue de Paris}|pw}, »\label{K_L03948-6v}\edtext{Die Toten schweigen\pwindex{Schnitzler, Arthur 15. 5. 1862 Wien – 21. 10. 1931 ebd.@\textsc{Schnitzler, Arthur} (15. 5. 1862 Wien – 21. 10. 1931 ebd.), \emph{Schriftsteller, Mediziner}!Toten schweigen@\strich\emph{Die Toten schweigen}|pw}}{\lemma{\textnormal{\emph{Die Toten schweigen}}}\Cendnote{\textnormal{Arthur Schnitzler: \emph{Les Morts se taisent}\pwindex{Schnitzler, Arthur 15. 5. 1862 Wien – 21. 10. 1931 ebd.@\textsc{Schnitzler, Arthur} (15. 5. 1862 Wien – 21. 10. 1931 ebd.), \emph{Schriftsteller, Mediziner}!Morts se taisent@\strich\emph{Les Morts se taisent}|pwk}. Traduction de N. Valentin\pwindex{Valentin, Noémi @\textsc{Valentin, Noémi}, \emph{Übersetzerin}|pwk} et M. Rémon\pwindex{Rémon, Maurice 27.\,11.\,1861 Paris – 20.\,6.\,1945 Mérignac@\textsc{Rémon, Maurice} (27.\,11.\,1861 Paris – 20.\,6.\,1945 Mérignac), \emph{Übersetzer}|pwk}. In: \emph{La
                           Revue de Paris}\pwindex{Revue de Paris@\emph{La Revue de Paris}|pwk}, Jg. 9, H. 6, 1. 6. 1902,
                        S. 601–619.}}}\label{K_L03948-6}\pwindex{Schnitzler, Arthur 15. 5. 1862 Wien – 21. 10. 1931 ebd.@\textsc{Schnitzler, Arthur} (15. 5. 1862 Wien – 21. 10. 1931 ebd.), \emph{Schriftsteller, Mediziner}!Morts se taisent@\strich\emph{Les Morts se taisent}|pwv}«, übersetzt von Vallentin\pwindex{Valentin, Noémi @\textsc{Valentin, Noémi}, \emph{Übersetzerin}|pw} und Rémon\pwindex{Rémon, Maurice 27.\,11.\,1861 Paris – 20.\,6.\,1945 Mérignac@\textsc{Rémon, Maurice} (27.\,11.\,1861 Paris – 20.\,6.\,1945 Mérignac), \emph{Übersetzer}|pw} im Jahre 1902 in der Nouvelle Revue\pwindex{Revue de Paris@\emph{La Revue de Paris}|pwv}\pwindex{Nouvelle Revue@\emph{La Nouvelle Revue}|pw} erschienen sind. Ich weiss nicht, ob nach dem französischen\oindex{Frankreich@\textbf{Frankreich}|pw} Urheberrecht sich dadurch die Uebersetzer\pwindex{Valentin, Noémi @\textsc{Valentin, Noémi}, \emph{Übersetzerin}|pwv}\pwindex{Rémon, Maurice 27.\,11.\,1861 Paris – 20.\,6.\,1945 Mérignac@\textsc{Rémon, Maurice} (27.\,11.\,1861 Paris – 20.\,6.\,1945 Mérignac), \emph{Übersetzer}|pwv}\pwindex{Bauer, Wilhelm 27.\,11.\,1854 Zollingen – 11.\,9.\,1923 Paris@\textsc{Bauer, Wilhelm} (27.\,11.\,1854 Zollingen – 11.\,9.\,1923 Paris)|pwv} in
               dauerndem Besitz der Autorisation befinden und ob sie es verbieten können, dass,
               insbesondere nach einer Reihe von Jahren, (in unserem Fall 10–20) in ein Sammelbuch
               neue Uebersetzungen der gleichen Novellen\pwindex{Schnitzler, Arthur 15. 5. 1862 Wien – 21. 10. 1931 ebd.@\textsc{Schnitzler, Arthur} (15. 5. 1862 Wien – 21. 10. 1931 ebd.), \emph{Schriftsteller, Mediziner}!Ehrentag@\strich\emph{Der Ehrentag}|pwv}\pwindex{Schnitzler, Arthur 15. 5. 1862 Wien – 21. 10. 1931 ebd.@\textsc{Schnitzler, Arthur} (15. 5. 1862 Wien – 21. 10. 1931 ebd.), \emph{Schriftsteller, Mediziner}!Toten schweigen@\strich\emph{Die Toten schweigen}|pwv}\pwindex{Schnitzler, Arthur 15. 5. 1862 Wien – 21. 10. 1931 ebd.@\textsc{Schnitzler, Arthur} (15. 5. 1862 Wien – 21. 10. 1931 ebd.), \emph{Schriftsteller, Mediziner}!blinde Geronimo und sein Bruder@\strich\emph{Der blinde Geronimo und sein Bruder}|pwv}, Aufnahme finden.\pend
           
\pstart
           In keinem Fall aber kann ich mir vorstellen, dass sich mit den früheren Uebersetzern\pwindex{Valentin, Noémi @\textsc{Valentin, Noémi}, \emph{Übersetzerin}|pwv}\pwindex{Rémon, Maurice 27.\,11.\,1861 Paris – 20.\,6.\,1945 Mérignac@\textsc{Rémon, Maurice} (27.\,11.\,1861 Paris – 20.\,6.\,1945 Mérignac), \emph{Übersetzer}|pwv}\pwindex{Bauer, Wilhelm 27.\,11.\,1854 Zollingen – 11.\,9.\,1923 Paris@\textsc{Bauer, Wilhelm} (27.\,11.\,1854 Zollingen – 11.\,9.\,1923 Paris)|pwv} nicht irgend eine Einigung durch den Verlag Stock\orgindex{Éditions Stock@Éditions Stock|pw} erzielen liesse.\pend
           
\pstart
           Was den »Mörder\pwindex{Schnitzler, Arthur 15. 5. 1862 Wien – 21. 10. 1931 ebd.@\textsc{Schnitzler, Arthur} (15. 5. 1862 Wien – 21. 10. 1931 ebd.), \emph{Schriftsteller, Mediziner}!Mörder. Novelle@\strich\emph{Der Mörder. Novelle}|pw}« anbelangt, so bemerke ich, dass
                  i{[}h{]}n Herrn Marcel
                  Livan\pwindex{Livane, Marcel @\textsc{Livane, Marcel}, \emph{Übersetzer}|pw}, Paris \strikeout{9}, Le Peletier 91\oindex{9, Rue Le Peletier@\textbf{9, Rue Le Peletier}, \emph{Wohngebäude}|pw}, übersetzt hat, dass \label{K_L03948-7v}\edtext{das Erscheinen dieser Novelle\pwindex{Schnitzler, Arthur 15. 5. 1862 Wien – 21. 10. 1931 ebd.@\textsc{Schnitzler, Arthur} (15. 5. 1862 Wien – 21. 10. 1931 ebd.), \emph{Schriftsteller, Mediziner}!Mörder. Novelle@\strich\emph{Der Mörder. Novelle}|pwv} in der »Humanité\pwindex{L'Humanité@\emph{L'Humanité}|pw}« bevorstand}{\lemma{\textnormal{\emph{das … bevorstand}}}\Cendnote{\textnormal{Der Text\pwindex{Schnitzler, Arthur 15. 5. 1862 Wien – 21. 10. 1931 ebd.@\textsc{Schnitzler, Arthur} (15. 5. 1862 Wien – 21. 10. 1931 ebd.), \emph{Schriftsteller, Mediziner}!Mörder. Novelle@\strich\emph{Der Mörder. Novelle}|pwkv} erschien nicht in der
                  Zeitschrift \emph{Humanité}\pwindex{L'Humanité@\emph{L'Humanité}|pwk}.}}}\label{K_L03948-7},
               dass aber {\pb}mein letzter \label{K_L03948-8v}\edtext{Brief an M. Livane\pwindex{Livane, Marcel @\textsc{Livane, Marcel}, \emph{Übersetzer}|pw}}{\lemma{\textnormal{\emph{Brief an M. Livane}}}\Cendnote{\textnormal{Arthur Schnitzler an Marcel Livane\pwindex{Livane, Marcel @\textsc{Livane, Marcel}, \emph{Übersetzer}|pwk}, 1. 10. 1923, Durchschlag im \emph{Deutschen Literaturarchiv Marbach},
                  HS.1985.1.1320. Unter dieser Signatur finden sich noch die Durchschläge
                  dreier weiterer Briefe, in denen Schnitzler
                  nach der ersten Kontaktaufnahme durch Livane\pwindex{Livane, Marcel @\textsc{Livane, Marcel}, \emph{Übersetzer}|pwk} die Bedigungen für eine Übertragung der Übersetzungsrechte der
                  Novelle \emph{Der Mörder}\pwindex{Schnitzler, Arthur 15. 5. 1862 Wien – 21. 10. 1931 ebd.@\textsc{Schnitzler, Arthur} (15. 5. 1862 Wien – 21. 10. 1931 ebd.), \emph{Schriftsteller, Mediziner}!Mörder. Novelle@\strich\emph{Der Mörder. Novelle}|pwk} erläutert
                     (6. 10. 1922), die urspünglich vereinbarte Frist, bis zu der die
                     Novelle\pwindex{Schnitzler, Arthur 15. 5. 1862 Wien – 21. 10. 1931 ebd.@\textsc{Schnitzler, Arthur} (15. 5. 1862 Wien – 21. 10. 1931 ebd.), \emph{Schriftsteller, Mediziner}!Mörder. Novelle@\strich\emph{Der Mörder. Novelle}|pwkv} erschienen sein
                  soll, bis zum 30. 6. 1923 verlängert (6. 11. 1922) und
                     Livane\pwindex{Livane, Marcel @\textsc{Livane, Marcel}, \emph{Übersetzer}|pwk} für die Erwägung zur Publikation
                  eines kleinen Bandes mit etwa drei ins Französische übersetzten Novellen Schnitzlers das Buch \emph{Die griechische Tänzerin. Novellen}\pwindex{Schnitzler, Arthur 15. 5. 1862 Wien – 21. 10. 1931 ebd.@\textsc{Schnitzler, Arthur} (15. 5. 1862 Wien – 21. 10. 1931 ebd.), \emph{Schriftsteller, Mediziner}!griechische Tänzerin. Novellen@\strich\emph{Die griechische Tänzerin. Novellen}|pwk} zukommen lässt
                     (9. 4. 1923).}}}\label{K_L03948-8} als unbestellbar an mich zurückkam.\pend
           
\pstart
           Hat übrigens M. Boutelleau\pwindex{Chardonne, Jacques 2.\,1.\,1884 Barbezieux-Saint-Hilaire – 29.\,5.\,1968 La Frette-sur-Seine@\textsc{Chardonne, Jacques} (2.\,1.\,1884 Barbezieux-Saint-Hilaire – 29.\,5.\,1968 La Frette-sur-Seine), \emph{Schriftsteller, Verleger}|pw} (Verlag Stock\orgindex{Éditions Stock@Éditions Stock|pw}) in seinem Verlag\orgindex{Éditions Stock@Éditions Stock|pwv} nicht einen deutsch verstehenden Vertrauensmann, der
               uns bei der Auswahl der Novellen für den französischen\oindex{Frankreich@\textbf{Frankreich}|pw} Geschmack behilflich sein könnte?\pend
           
\pstart
           Die Bedingungen für den Novellenband\pwindex{Schnitzler, Arthur 15. 5. 1862 Wien – 21. 10. 1931 ebd.@\textsc{Schnitzler, Arthur} (15. 5. 1862 Wien – 21. 10. 1931 ebd.), \emph{Schriftsteller, Mediziner}!pénombre des âmes@\strich\emph{La pénombre des âmes}|pwv} wären natürlich die gleichen, wie für den Einakter-Band{\dotstwo} Wenn die Sache nicht eilt, können wir all das Nähere,
               verehrte Frau Hofrätin, nach Ihrem Eintreffen \label{K_L03948-9v}\edtext{in Wien\oindex{Wien@\textbf{Wien}, \emph{Verwaltungsgebiet}|pw} besprechen}{\lemma{\textnormal{\emph{in Wien besprechen}}}\Cendnote{\textnormal{Vgl. A. S.: \emph{Tagebuch}, 28. 4. 1923.}}}\label{K_L03948-9}. Im
               Prinzip liegen ja die Dinge wohl klar.\pend
           
\pstart
           Für heute nur mehr herzlichste Grüsse und{\\[\baselineskip]} vielen Dank. Auf ein baldiges
               gutes Wiedersehen.{\\[\baselineskip]} Ihr\pend
           \leftskip=0em{}{\vspace{1\baselineskip}}
\pstart
           \noindent{}Frau Hofrätin Berta Zuckerkandl,{\\}Paris\oindex{Paris@\textbf{Paris}, \emph{Hauptstadt}|pw}.\pend
           \selectlanguage{ngerman}\endnumbering\briefempfaengerindex{Zuckerkandl, Berta@\textsc{Zuckerkandl, Berta}!zzzSchnitzler, Arthur@\emph{von Arthur Schnitzler}!1923-10-291@{29. 10. 1923}|)be}\mylabel{L03948h}
\begin{anhang}
\end{anhang}\newcommand{\dateiname}{L03948}\newcommand{\titel}{Arthur Schnitzler an Berta Zuckerkandl, 29. 10. 1923}\newcommand{\editorInnen}{Herausgegeben von Jahnke, SelmaMüller, Martin Anton}%% latex-leseansicht-abspann.tex
%% Abspann für die Leseansicht.
%% Der Schalter \ifkorrekturansicht ist bereits durch den Vorspann gesetzt.

%% latex-abspann.tex
%% Gemeinsamer Abspann für Korrekturansicht und Leseansicht.
%% Setzt den Schalter \ifkorrekturansicht voraus (gesetzt in den
%% einbindenden Dateien latex-korrekturansicht-abspann.tex bzw.
%% latex-leseansicht-abspann.tex).
%% ---------------------------------------------------------------

\normalsize

% Das esempio-Environment wird nur in der Leseansicht benötigt
\ifkorrekturansicht\else
\newenvironment{esempio}[3]%
{
    \vspace{1.5ex}
    \rlap{\underline{#1}}
    \par
    \setlength{\parindent}{0cm}
    \nopagebreak
    \leftskip=#2cm
    \rightskip=#3cm
}
{
    \par
}
\fi

\doendnotes{C}
\bigskip
\vfill

\clearpage

\footnotesize

\ifkorrekturansicht
  \lohead{\textsc{register}}
\fi

% theindex-Environment neu definieren ohne reledmac
\makeatletter
\renewenvironment{theindex}{%
  \ifkorrekturansicht
    \section*{\indexname}%
  \else
    \subsubsection*{Index der erwähnten Entitäten}%
  \fi
  \setlength{\parindent}{0pt}%
  \setlength{\parskip}{0pt plus 0.3pt}%
  \let\item\@idxitem
}{%
  \ifkorrekturansicht\clearpage\fi
}
\makeatother

\IfFileExists{\jobname-pw.ind}{\input{\jobname-pw.ind}}{}

% Quellenangabe nur in der Leseansicht
\ifkorrekturansicht\else
% Fallback-Definitionen, falls die .tex-Datei \titel etc. nicht gesetzt hat
\providecommand{\titel}{}
\providecommand{\editorInnen}{}
\providecommand{\dateiname}{\jobname}

\vspace{3cm}

\vfill

\footnotesize
\textsc{Quelle}: \titel. Herausgegeben von {\editorInnen}. In: \emph{Arthur Schnitzler: Briefwechsel mit Autorinnen und Autoren}.
 Digitale Edition, https://schnitzler-briefe.acdh.oeaw.ac.at/{\dateiname}.html (Stand \today)
\fi

\end{document}


