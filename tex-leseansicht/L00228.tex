%% latex-korrekturansicht-vorspann.tex
%% Vorspann für die Korrekturansicht.
%% Lädt die gemeinsame Datei latex-vorspann.tex mit gesetztem Schalter.

\newif\ifkorrekturansicht
\korrekturansichttrue

\input{../tex-inputs/latex-vorspann}


\section[Arthur Schnitzler an Richard Beer-Hofmann, 30. 6. 1893]{L00228 Arthur Schnitzler an Richard Beer-Hofmann, 30. 6. 1893}
\nopagebreak\mylabel{L00228v}
\rehead{ }\normalsize\beginnumbering\briefempfaengerindex{Beer-Hofmann, Richard@\textsc{Beer-Hofmann, Richard}!zzzSchnitzler, Arthur@\emph{von Arthur Schnitzler}!1893-06-301@{30. 6. 1893}|(be}
\toendnotes[C]{\smallbreak\pagebreak[2]}\Standort{YCGL, MSS 31.}
\physDesc{Brief, 1 Blatt, 2 Seiten, Umschlag, 433 Zeichen (Umschlag und Briefpapier mit Trauerrand )
\newline{}Handschrift: Bleistift, deutsche Kurrent
\newline{}Versand: 1) Stempel: »\nobreak{}\oindex{IX., Alsergrund@\textbf{IX., Alsergrund}, \emph{A.ADM3}|pwk}Wien 9/2, 30. 6. 93, 10–11 V\nobreak{}«.   2) Stempel: »\nobreak{}\oindex{Bad Ischl@\textbf{Bad Ischl}, \emph{P.PPL}|pwk}Ischl, 1 7 93, 7 F\nobreak{}«. }
\buchAbdrucke{\weitereDrucke{Arthur Schnitzler, Richard Beer-Hofmann: \emph{Briefwechsel 1891–1931}. Wien, Zürich: \emph{Europaverlag} 1992, S. 45–46.} }\toendnotes[C]{\smallbreak}\pstart{}{\pb}\textsc{Herrn Dr. Rich. Beer-Hofmann}\pend{}\pstart{}\textsc{Ischl\oindex{Bad Ischl@\textbf{Bad Ischl}, \emph{P.PPL}|pw}}\pend{}\pstart{}\textsc{Schulgasse 8}\oindex{Schulgasse@\textbf{Schulgasse}, \emph{Straße (K.STR)}|pw}.\pend{}{\bigskip}\vspace{1em}
\pstart
           \raggedleft{}{\pb}30/6\pend
           
\pstart{}Lieber Richard,\pend\vspace{0.5em}
\pstart
           aller Wahrſcheinlichkeit nach bin ich So{\geminationn}tag{ }Früh mit meiner Mama\pwindex{Schnitzler, Louise 1840-07-08 – 1911-09-09@\textsc{Schnitzler, Louise} (1840-07-08 – 1911-09-09)|pwv} in Iſchl\oindex{Bad Ischl@\textbf{Bad Ischl}, \emph{P.PPL}|pw}. – Cigaretten u Parfum für
               Sie ſtehen bereit. –\pend
           
\pstart
           \textsc{Jarno}\pwindex{Jarno, Josef 24.08.1865 – 11.01.1932@\textsc{Jarno, Josef} (24.08.1865 – 11.01.1932), \emph{Theaterleiter/Theaterleiterin, Schauspieler/Schauspielerin}|pw} tritt ja ſchon auf. – Schon geſprochen? –\pend
           
\pstart
           Ich werde bei Leopold\oindex{Hotel und Pension Rudolfshoehe (Leopold Petter)@\textbf{Hotel und Pension Rudolfshöhe (Leopold Petter)}, \emph{Hotel (K.HTL)}|pw} wohnen u. täglich
               ſtundenlang Bicycle fahren – was ein {\pb}wirkliches
               Vergnügen iſt. – \textsc{Loris}\pwindex{Hofmannsthal, Hugo von 1874-02-01 – 1929-07-15@\textsc{Hofmannsthal, Hugo von} (1874-02-01 – 1929-07-15), \emph{Schriftsteller/Schriftstellerin}|pw} wirds auch lernen. –\pend
           
\pstart
           Adieu einſtweilen, ich freue mich ſehr, Sie wiederzuſehen.\pend
           \pstart Herzlich der Ihre \spacefill\mbox{Arth}\pend{}\selectlanguage{ngerman}\endnumbering\briefempfaengerindex{Beer-Hofmann, Richard@\textsc{Beer-Hofmann, Richard}!zzzSchnitzler, Arthur@\emph{von Arthur Schnitzler}!1893-06-301@{30. 6. 1893}|)be}\mylabel{L00228h}  \normalsize

\doendnotes{C}
\bigskip
\vfill

\clearpage

\footnotesize

\lohead{\textsc{register}}

% Definiere theindex-Environment komplett neu ohne reledmac
\makeatletter
\renewenvironment{theindex}{%
  \section*{\indexname}%
  \setlength{\parindent}{0pt}%
  \setlength{\parskip}{0pt plus 0.3pt}%
  \let\item\@idxitem
}{%
  \clearpage
}
\makeatother

\IfFileExists{\jobname-pw.ind}{\input{\jobname-pw.ind}}{}

\end{document}

      