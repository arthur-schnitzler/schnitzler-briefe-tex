\input{../tex-inputs/latex-pdf-vorspann}
\begin{center}
            \textcolor{red}{ENTWURF. ENTZIFFERUNG NOCH NICHT KORREKTURGELESEN}
                      \end{center}
            
               \section[Arthur Schnitzler an Richard Beer-Hofmann, 30. 6. 1893]{ Arthur Schnitzler an Richard Beer-Hofmann, 30. 6. 1893}\nopagebreak\mylabel{v}\rehead{ }\begin{ledgroupsized}[t]{13cm}\normalsize\beginnumbering\briefempfaengerindex{Beer-Hofmann, Richard@\textsc{Beer-Hofmann, Richard}!zzzSchnitzler, Arthur@\emph{von Arthur Schnitzler}!1893-06-301@{30. 6. 1893}|(be} \toendnotes[C]{\smallbreak\pagebreak[2]} \Standort{YCGL, MSS 31.}
\physDesc{Brief, 1 Blatt (Briefpapier mit Trauerrand), 2 Seiten, Umschlag mit Trauerrand
\newline{}Handschrift: Bleistift, deutsche Kurrent\newline{}Versand: 1) Stempel: »\nobreak{}Wien 9/2, 30. 6. 93, 10–11 V\nobreak{}«.  2) Stempel: »\nobreak{}\oindex{Bad Ischl@\textbf{Bad Ischl}|pwk}Ischl, 1 7 93, 7 F\nobreak{}«. }\buchAbdrucke{\weitereDrucke{Arthur Schnitzler, Richard Beer-Hofmann: \emph{Briefwechsel 1891–1931}. Hg. Konstanze Fliedl. Wien, Zürich: \emph{Europaverlag} 1992, S. 45–46.} }\toendnotes[C]{\smallbreak}\pstart{}{\pb}\textsc{Herrn Dr. Rich. Beer-Hofmann}\pend{}\pstart{}\textsc{Ischl\oindex{Bad Ischl@\textbf{Bad Ischl}|pw}}\pend{}\pstart{}\textsc{Schulgasse 8}\oindex{Schulgasse@\textbf{Schulgasse}|pw}.\pend{}{\bigskip}\pstart
           \raggedleft{}{\pb}30/6\pend
           \pstart{}Lieber Richard,\pend\pstart
           aller Wahrſcheinlichkeit nach bin ich So{\geminationn}tag{ }Früh mit meiner Mama\pwindex{Schnitzler, Louise 08.07.1840 – 09.09.1911@\textsc{Schnitzler, Louise} (08.07.1840 – 09.09.1911)|pwv} in Iſchl\oindex{Bad Ischl@\textbf{Bad Ischl}|pw}. – Cigaretten u Parfum für
               Sie ſtehen bereit. –\pend
           \pstart
           \textsc{Jarno}\pwindex{Jarno, Josef 24.08.1865 – 11.01.1932@\textsc{Jarno, Josef} (24.08.1865 – 11.01.1932), \emph{Theaterleiter, Schauspieler}|pw} tritt ja ſchon auf. – Schon geſprochen? –\pend
           \pstart
           Ich werde bei Leopold\oindex{Hotel und Pension Rudolfshoehe (Leopold Petter)@\textbf{Hotel und Pension Rudolfshöhe (Leopold Petter)}|pw} wohnen u. täglich ſtundenlang
               Bicycle fahren – was ein {\pb}wirkliches Vergnügen iſt. –
                  \textsc{Loris}\pwindex{Hofmannsthal, Hugo von 01.02.1874 – 15.07.1929@\textsc{Hofmannsthal, Hugo von} (01.02.1874 – 15.07.1929), \emph{Schriftsteller}|pw} wirds auch lernen. –\pend
           \pstart
           Adieu einſtweilen, ich freue mich ſehr, Sie wiederzuſehen.\pend
           \pstart Herzlich der Ihre \spacefill\mbox{Arth}\pend{}\endnumbering\briefempfaengerindex{Beer-Hofmann, Richard@\textsc{Beer-Hofmann, Richard}!zzzSchnitzler, Arthur@\emph{von Arthur Schnitzler}!1893-06-301@{30. 6. 1893}|)be}\mylabel{h}\end{ledgroupsized}  \newcommand{\dateiname}{L00228}\newcommand{\titel}{Arthur Schnitzler an Richard Beer-Hofmann, 30. 6. 1893}\newcommand{\editorInnen}{Martin Anton Müller und Gerd-Hermann Susen}\input{../tex-inputs/latex-pdf-abspann}
      