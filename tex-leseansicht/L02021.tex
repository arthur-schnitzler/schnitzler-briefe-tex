%% latex-korrekturansicht-vorspann.tex
%% Vorspann für die Korrekturansicht.
%% Lädt die gemeinsame Datei latex-vorspann.tex mit gesetztem Schalter.

\newif\ifkorrekturansicht
\korrekturansichttrue

\input{../tex-inputs/latex-vorspann}


\section[Hugo von Hofmannsthal an Arthur Schnitzler, 2. 6. {[}1911{]}]{L02021 Hugo von Hofmannsthal an Arthur Schnitzler, 2. 6. {[}1911{]}}
\nopagebreak\mylabel{L02021v}
\rehead{ }\normalsize\beginnumbering\briefempfaengerindex{Schnitzler, Arthur@\textsc{Schnitzler, Arthur}!zzzHofmannsthal, Hugo von@\emph{von Hugo von Hofmannsthal}!1911-06-021@{2. 6. {[}1911{]}}|(be}
\toendnotes[C]{\smallbreak\pagebreak[2]}\Standort{CUL, Schnitzler, B 43.}
\physDesc{Brief, 1 Blatt, 2 Seiten, 629 Zeichen
\newline{}Handschrift: schwarze Tinte, deutsche Kurrent
\newline{}Schnitzler: mit Bleistift die Jahreszahl ergänzt: »911« und beschriftet: »Hugo« 
\newline{}Ordnung: 1) mit Bleistift von unbekannter Hand nummeriert: »\strikeout{321}«  2) mit Bleistift von unbekannter Hand nummeriert:
                                    »330«}
\buchAbdrucke{\weitereDrucke{Hugo von Hofmannsthal, Arthur Schnitzler: \emph{Briefwechsel}. Frankfurt am Main: \emph{S. Fischer} 1964, S. 261.} }\toendnotes[C]{\smallbreak}
\pstart
           \raggedleft{}{\pb}2. VI.{ }R\oindex{Rodaun@\textbf{Rodaun}, \emph{A.ADM4}|pw}\pend
           
\pstart{}mein lieber Arthur\pend\vspace{0.5em}
\pstart
           ich war minder lang in Paris\oindex{Paris@\textbf{Paris}, \emph{P.PPLC}|pw} als ich zu bleiben
               mir vorgeſetzt hatte – beim \label{K_L02021-1v}\edtext{Zurückkommen}{\lemma{\textnormal{\emph{Zurückkommen}}}\Cendnote{\textnormal{am
                     11. 5. 1911}}}\label{K_L02021-1} war meine Vorfreude groß, Sie nun bald zu ſehen, ausgiebig zu ſehen und mehr
               als einmal, die vielen Fäden fortzuſpinnen, die uns verbinden und von denen ja
               niemals einer abgeriſſen ist, freute mich {\pb}darauf, Euch hier zu ſehen, ehe
               das Haus und die Kinder\pwindex{Zimmer, Christiane 14.05.1902 – 05.01.1987@\textsc{Zimmer, Christiane} (14.05.1902 – 05.01.1987)|pwv}\pwindex{Hofmannsthal, Raimund von 26.5.1906 – 20.03.1974@\textsc{Hofmannsthal, Raimund von} (26.5.1906 – 20.03.1974)|pwv}\pwindex{Hofmannsthal, Franz von 20.10.1903 – 13.07.1929@\textsc{Hofmannsthal, Franz von} (20.10.1903 – 13.07.1929)|pwv}{ }ſich Euch ganz entfremden – kam und hörte, nun
               wäret wieder Ihr im Fortgehen, da war ich wirklich ganz traurig. Doch kommt Ihr
               wieder und ſo wird dieſer Brief Sie bald finden und man wird dann nicht mehr lang
               ſein, ohne ſich zu ſehen.\pend
           
\pstart
           Vieles Gute Liebe an Olga\pwindex{Schnitzler, Olga 17.01.1882 – 13.01.1970@\textsc{Schnitzler, Olga} (17.01.1882 – 13.01.1970), \emph{Schauspieler/Schauspielerin, Sänger/Sängerin}|pw}.{\\[\baselineskip]}Ihr{\\[\baselineskip]}\spacefill\mbox{Hugo}\pend
           \leftskip=0em{}\selectlanguage{ngerman}\endnumbering\briefempfaengerindex{Schnitzler, Arthur@\textsc{Schnitzler, Arthur}!zzzHofmannsthal, Hugo von@\emph{von Hugo von Hofmannsthal}!1911-06-021@{2. 6. {[}1911{]}}|)be}\mylabel{L02021h}  \normalsize

\doendnotes{C}
\bigskip
\vfill

\clearpage

\footnotesize

\lohead{\textsc{register}}

% Definiere theindex-Environment komplett neu ohne reledmac
\makeatletter
\renewenvironment{theindex}{%
  \section*{\indexname}%
  \setlength{\parindent}{0pt}%
  \setlength{\parskip}{0pt plus 0.3pt}%
  \let\item\@idxitem
}{%
  \clearpage
}
\makeatother

\IfFileExists{\jobname-pw.ind}{\input{\jobname-pw.ind}}{}

\end{document}

      