%% latex-leseansicht-vorspann.tex
%% Vorspann für die Leseansicht.
%% Lädt die gemeinsame Datei latex-vorspann.tex mit nicht gesetztem Schalter.

\newif\ifkorrekturansicht
\korrekturansichtfalse

\input{../tex-inputs/latex-vorspann}


\section[Stefan Zweig an Arthur Schnitzler, 3. 6. 1908]{L03622 Stefan Zweig an Arthur Schnitzler, 3. 6. 1908}
\nopagebreak\mylabel{L03622v}
\rehead{ }\normalsize\beginnumbering\briefempfaengerindex{Schnitzler, Arthur@\textsc{Schnitzler, Arthur}!zzzZweig, Stefan@\emph{von Stefan Zweig}!1908-06-032@{3. 6. 1908}|(be}
\toendnotes[C]{\smallbreak\pagebreak[2]}
\correspDesc{Versand  durch Stefan Zweig am 3. 6. 1908 in Wien
\newline{}Weiterleitung  im Zeitraum [3. 6. 1908
                  – 4. 6. 1908?] in Wien
\newline{}Erhalt  durch Arthur Schnitzler im Zeitraum [4. 6. 1908 – 5. 6. 1908] in Hinterbrühl}\toendnotes[C]{\smallbreak}
\Standort{CUL, Schnitzler, B 118.}
\physDesc{Brief, 1 Blatt, 3 Seiten, 1703 Zeichen
\newline{}Handschrift: lila Tinte, lateinische Kurrent
\newline{}Schnitzler: mit Bleistift »\textsc{Zweig}« }
\buchAbdrucke{\weitereDrucke{Stefan Zweig: \emph{Briefwechsel mit Hermann Bahr, Sigmund Freud, Rainer Maria
                        Rilke und Arthur Schnitzler}. Herausgegeben von Jeffrey B. Berlin, Hans-Ulrich Lindken und Donald A. Prater. Frankfurt am Main: \emph{S. Fischer} 1987, S. 354–355.} }\toendnotes[C]{\smallbreak}
\pstart
           {\pb}\textcolor{gray}{\textbf{SZ}}\hfill \textcolor{gray}{\textbf{VIII. KOCHGASSE 8\oindex{Wien@\textbf{Wien}!VIII., Josefstadt@\textbf{VIII., Josefstadt}!Kochgasse 8@\textbf{Kochgasse 8}, \emph{Wohngebäude}|pw}}}\pend
           
\pstart
           \raggedleft{}\textcolor{gray}{\textbf{WIEN\oindex{Wien@\textbf{Wien}, \emph{Verwaltungsgebiet}|pw},}}{ }3. Juni 08\pend
           
\pstart{}Sehr verehrter Herr Doktor,\pend\vspace{0.5em}
\pstart
           ich hatte mir schon Sorge gemacht, Sie würden es vielleicht übel vermerkt haben, dass
               ich gestern mit nur raschem Gruss an Ihnen vorbeigieng
               – ich fürchtete, Sie zu stören – da bringt mir heute Ihr Buch\pwindex{Schnitzler, Arthur 15.\,5.\,1862 Wien – 21.\,10.\,1931 ebd.@\textsc{Schnitzler, Arthur} (15.\,5.\,1862 Wien – 21.\,10.\,1931 ebd.), \emph{Schriftsteller, Mediziner}!Weg ins Freie. Roman@\strich\emph{Der Weg ins Freie. Roman}|pwv} ein liebes Geschenk und ein nur noch Wertvolleres: das
                  \label{K_L03622-1v}\edtext{Zeichen freundlicher
                  Gesinnung}{\lemma{\textnormal{\emph{Zeichen … Gesinnung}}}\Cendnote{\textnormal{Schnitzler hatte am Vortag Widmungsexemplare
                  seines neuen Romans \emph{Der Weg ins Freie}\pwindex{Auernheimer, Raoul 15.\,4.\,1876 Wien – 6.\,1.\,1948 Oakland@\textsc{Auernheimer, Raoul} (15.\,4.\,1876 Wien – 6.\,1.\,1948 Oakland), \emph{Schriftsteller, Journalist, Kritiker}!Weg ins Freie@\strich\emph{Der Weg ins Freie}|pwk} zum
                  Versand an Freunde vorbereitet, vgl. A. S.: \emph{Tagebuch}, 2. 6. 1908. Zweig\pwindex{Zweig, Stefan 28.\,11.\,1881 Wien – 23.\,2.\,1942 Petrópolis@\textsc{Zweig, Stefan} (28.\,11.\,1881 Wien – 23.\,2.\,1942 Petrópolis), \emph{Schriftsteller}|pwk} bedankt sich im vorliegenden Brief für die Zusendung.}}}\label{K_L03622-1}. Ich bin so
               sehr froh, das Buch\pwindex{Schnitzler, Arthur 15.\,5.\,1862 Wien – 21.\,10.\,1931 ebd.@\textsc{Schnitzler, Arthur} (15.\,5.\,1862 Wien – 21.\,10.\,1931 ebd.), \emph{Schriftsteller, Mediziner}!Weg ins Freie. Roman@\strich\emph{Der Weg ins Freie. Roman}|pwv} von Ihnen zu besitzen: es
               wird mir nun vielleicht noch mehr sein, als es mir durch seine innere Gewalt ohnehin
               schon bedeutet. Uns, {\pb}den Jüngern, durch
               Blut und Heimatliebe Verwandten, ist es ja wohl zu\substVorne{}\textsuperscript{f}\substDazwischen{}g\substHinten{}eschrieben, uns wird es
               vielleicht mehr gehören, als jeder anderen Generation, jeder andern Stadt, jedem
               andern Kreis: mögen die andern das Äussere lieben, den Blick, den Griff, die Melodie,
               so stehen wir doch seinem Herzen am nächsten, denn – unbewusst vielleicht – für uns
               ist es geschrieben, ist es als Wegzeiger hin gestellt. So empfangen Sie mit meinem
               Dank den vieler anderer, den\introOben{}en\introOben{} die Freude nicht gegeben
               war, es direct aus Ihren Händen nehmen zu dürfen, einen Dank, nicht für Einzelnes,
               nicht für das Geschaffene allein, sondern für das Ganze, für den grossen schönen
               Willen und für alle die viele Liebe, die Sie diesen Menschen – für uns – mitgegeben
               haben.\pend
           
\pstart
           {\pb}Schade, dass \label{K_L03622-2v}\edtext{Auernheimer\pwindex{Auernheimer, Raoul 15.\,4.\,1876 Wien – 6.\,1.\,1948 Oakland@\textsc{Auernheimer, Raoul} (15.\,4.\,1876 Wien – 6.\,1.\,1948 Oakland), \emph{Schriftsteller, Journalist, Kritiker}|pw}}{\lemma{\textnormal{\emph{Auernheimer}}}\Cendnote{\textnormal{Raoul Auernheimer\pwindex{Auernheimer, Raoul 15.\,4.\,1876 Wien – 6.\,1.\,1948 Oakland@\textsc{Auernheimer, Raoul} (15.\,4.\,1876 Wien – 6.\,1.\,1948 Oakland), \emph{Schriftsteller, Journalist, Kritiker}|pwk}: \emph{Der Weg ins Freie}\pwindex{Auernheimer, Raoul 15.\,4.\,1876 Wien – 6.\,1.\,1948 Oakland@\textsc{Auernheimer, Raoul} (15.\,4.\,1876 Wien – 6.\,1.\,1948 Oakland), \emph{Schriftsteller, Journalist, Kritiker}!Weg ins Freie@\strich\emph{Der Weg ins Freie}|pwk}. In: \emph{Neue Freie Presse}\pwindex{Neue Freie Presse@\emph{Neue Freie Presse}|pwk}, Nr. 15.728, 3. 6. 1908, Morgenblatt, S. 1–3.
               }}}\label{K_L03622-2} durch die ängstliche Tendenz der Neuen Freien
                  Presse\orgindex{Neue Freie Presse@Neue Freie Presse|pw} genötigt war, dem eigentlichen Problem auszubiegen. \strikeout{Alles} Gerade die Idee der Amalgamierung des Jüdischen
               und Wienerischen\oindex{Wien@\textbf{Wien}, \emph{Verwaltungsgebiet}|pw} darin scheint mir das Neue,
               Bedeutsame und noch nie Gewagte und sie würde – und wird vielleicht – in einer
                  \label{K_L03622-3v}\edtext{Studie über den Roman}{\lemma{\textnormal{\emph{Studie über den Roman}}}\Cendnote{\textnormal{Stefan Zweig\pwindex{Zweig, Stefan 28.\,11.\,1881 Wien – 23.\,2.\,1942 Petrópolis@\textsc{Zweig, Stefan} (28.\,11.\,1881 Wien – 23.\,2.\,1942 Petrópolis), \emph{Schriftsteller}|pwk} verfasste keine Studie über
                  den Roman.}}}\label{K_L03622-3} mich am meisten beschäftigen.\pend
           
\pstart
           Nochmals: vielen Dank für Ihre Güte. Und seien Sie meiner lebhaften Verehrung
               aufs innigste versichert. Ihr sehr ergebener{\\[\baselineskip]}\spacefill\mbox{StefanZweig}\pend
           \leftskip=0em{}\selectlanguage{ngerman}\endnumbering\briefempfaengerindex{Schnitzler, Arthur@\textsc{Schnitzler, Arthur}!zzzZweig, Stefan@\emph{von Stefan Zweig}!1908-06-032@{3. 6. 1908}|)be}\mylabel{L03622h}  \newcommand{\dateiname}{L03622}\newcommand{\titel}{Stefan Zweig an Arthur Schnitzler, 3. 6. 1908}\newcommand{\editorInnen}{Selma Jahnke und Martin Anton Müller}%% latex-leseansicht-abspann.tex
%% Abspann für die Leseansicht.
%% Der Schalter \ifkorrekturansicht ist bereits durch den Vorspann gesetzt.

%% latex-abspann.tex
%% Gemeinsamer Abspann für Korrekturansicht und Leseansicht.
%% Setzt den Schalter \ifkorrekturansicht voraus (gesetzt in den
%% einbindenden Dateien latex-korrekturansicht-abspann.tex bzw.
%% latex-leseansicht-abspann.tex).
%% ---------------------------------------------------------------

\normalsize

% Das esempio-Environment wird nur in der Leseansicht benötigt
\ifkorrekturansicht\else
\newenvironment{esempio}[3]%
{
    \vspace{1.5ex}
    \rlap{\underline{#1}}
    \par
    \setlength{\parindent}{0cm}
    \nopagebreak
    \leftskip=#2cm
    \rightskip=#3cm
}
{
    \par
}
\fi

\doendnotes{C}
\bigskip
\vfill

\clearpage

\footnotesize

\ifkorrekturansicht
  \lohead{\textsc{register}}
\fi

% theindex-Environment neu definieren ohne reledmac
\makeatletter
\renewenvironment{theindex}{%
  \ifkorrekturansicht
    \section*{\indexname}%
  \else
    \subsubsection*{Index der erwähnten Entitäten}%
  \fi
  \setlength{\parindent}{0pt}%
  \setlength{\parskip}{0pt plus 0.3pt}%
  \let\item\@idxitem
}{%
  \ifkorrekturansicht\clearpage\fi
}
\makeatother

\IfFileExists{\jobname-pw.ind}{\input{\jobname-pw.ind}}{}

% Quellenangabe nur in der Leseansicht
\ifkorrekturansicht\else
% Fallback-Definitionen, falls die .tex-Datei \titel etc. nicht gesetzt hat
\providecommand{\titel}{}
\providecommand{\editorInnen}{}
\providecommand{\dateiname}{\jobname}

\vspace{3cm}

\vfill

\footnotesize
\textsc{Quelle}: \titel. Herausgegeben von {\editorInnen}. In: \emph{Arthur Schnitzler: Briefwechsel mit Autorinnen und Autoren}.
 Digitale Edition, https://schnitzler-briefe.acdh.oeaw.ac.at/{\dateiname}.html (Stand \today)
\fi

\end{document}


