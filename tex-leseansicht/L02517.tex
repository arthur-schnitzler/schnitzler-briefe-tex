%% latex-leseansicht-vorspann.tex
%% Vorspann für die Leseansicht.
%% Lädt die gemeinsame Datei latex-vorspann.tex mit nicht gesetztem Schalter.

\newif\ifkorrekturansicht
\korrekturansichtfalse

\input{../tex-inputs/latex-vorspann}


\section[Christiane von Hofmannsthal an Arthur Schnitzler, 5. 8. 1929]{L02517 Christiane von Hofmannsthal an Arthur Schnitzler, 5. 8. 1929}
\nopagebreak\mylabel{L02517v}
\rehead{ }\normalsize\beginnumbering\briefempfaengerindex{Schnitzler, Arthur@\textsc{Schnitzler, Arthur}!zzzZimmer, Christiane@\emph{von Christiane Zimmer}!1929-08-051@{13. 8. 1929}|(be}
\toendnotes[C]{\smallbreak\pagebreak[2]}
\correspDesc{Versand  durch Christiane von Hofmannsthal am 13. 8. 1929 in Bad Aussee
\newline{}Erhalt  durch Arthur Schnitzler im Zeitraum [6. 8. 1929
                  – 10. 8. 1929?] in Wien}\toendnotes[C]{\smallbreak}
\Standort{CUL, Schnitzler, B 43.}
\physDesc{Brief, 1 Blatt, 1 Seite, 714 Zeichen (Briefpapier mit Trauerrand)
\newline{}Schreibmaschine
\newline{}Handschrift: schwarze Tinte, lateinische Kurrent
\newline{}Schnitzler: mit rotem Buntstift vier Unterstreichungen sowie die
                                 Beschriftung: »\textsc{Hofm}« und »\textsc{\uuline{Christiane}}« }
\pstart
           \raggedleft{}{\pb}Bad Aussee\oindex{Bad Aussee@\textbf{Bad Aussee}, \emph{Hauptstadt}|pw}, am 5. August 1929\pend
           
\pstart
           \raggedleft{}{[}hs.:{]} Obertressen 6\oindex{Obertressen@\textbf{Obertressen}|pw}\pend
           
\pstart{}{[}ms.:{]} Lieber Arthur,\pend\vspace{0.5em}
\pstart
           Mama\pwindex{Hofmannsthal, Gertrude von 16.\,3.\,1880 Wien – 9.\,11.\,1959 Paddington@\textsc{Hofmannsthal, Gertrude von} (16.\,3.\,1880 Wien – 9.\,11.\,1959 Paddington)|pw} veranlasst mich, Ihnen für Ihren lieben
               Brief zu danken, da ihr schreiben noch schwer fällt.\pend
           
\pstart
           Wir wären Ihnen für baldigste Uebersendung der Briefe sowohl an Sie als an Gustav Schwarzkopf\pwindex{Schwarzkopf, Gustav 7.\,11.\,1853 Wien – 13.\,11.\,1939 ebd.@\textsc{Schwarzkopf, Gustav} (7.\,11.\,1853 Wien – 13.\,11.\,1939 ebd.), \emph{Schriftsteller}|pw} und wenn Sie können auch der
               unveröffentlichten Gedichte an diesen, sehr dankbar, wir würden möglichst schnell
               2 Abschriften davon herstellen und Sie bekommen die Originale und eine Copie wieder
               zurück. Es ist uns doch sehr
               wichtig, das
               ganze vorhandene Material überschauen zu können, bezüglich einer Veröffentlichung
               würde natürlich nichts geschehen ohne Ihre ausdrückliche Einwilligung.\pend
           
\pstart
           Wir gehen sehr achtsam damit um.\pend
           
\pstart
           Mit herzlichstem Dank und vielen Grüssen{\\[\baselineskip]}\spacefill\mbox{{[}hs.:{]} Christiane}\pend
           \leftskip=0em{}\selectlanguage{ngerman}\endnumbering\briefempfaengerindex{Schnitzler, Arthur@\textsc{Schnitzler, Arthur}!zzzZimmer, Christiane@\emph{von Christiane Zimmer}!1929-08-051@{13. 8. 1929}|)be}\mylabel{L02517h}  \newcommand{\dateiname}{L02517}\newcommand{\titel}{Christiane von Hofmannsthal an Arthur Schnitzler, 5. 8. 1929}\newcommand{\editorInnen}{Martin Anton Müller und Gerd-Hermann Susen}%% latex-leseansicht-abspann.tex
%% Abspann für die Leseansicht.
%% Der Schalter \ifkorrekturansicht ist bereits durch den Vorspann gesetzt.

%% latex-abspann.tex
%% Gemeinsamer Abspann für Korrekturansicht und Leseansicht.
%% Setzt den Schalter \ifkorrekturansicht voraus (gesetzt in den
%% einbindenden Dateien latex-korrekturansicht-abspann.tex bzw.
%% latex-leseansicht-abspann.tex).
%% ---------------------------------------------------------------

\normalsize

% Das esempio-Environment wird nur in der Leseansicht benötigt
\ifkorrekturansicht\else
\newenvironment{esempio}[3]%
{
    \vspace{1.5ex}
    \rlap{\underline{#1}}
    \par
    \setlength{\parindent}{0cm}
    \nopagebreak
    \leftskip=#2cm
    \rightskip=#3cm
}
{
    \par
}
\fi

\doendnotes{C}
\bigskip
\vfill

\clearpage

\footnotesize

\ifkorrekturansicht
  \lohead{\textsc{register}}
\fi

% theindex-Environment neu definieren ohne reledmac
\makeatletter
\renewenvironment{theindex}{%
  \ifkorrekturansicht
    \section*{\indexname}%
  \else
    \subsubsection*{Index der erwähnten Entitäten}%
  \fi
  \setlength{\parindent}{0pt}%
  \setlength{\parskip}{0pt plus 0.3pt}%
  \let\item\@idxitem
}{%
  \ifkorrekturansicht\clearpage\fi
}
\makeatother

\IfFileExists{\jobname-pw.ind}{\input{\jobname-pw.ind}}{}

% Quellenangabe nur in der Leseansicht
\ifkorrekturansicht\else
% Fallback-Definitionen, falls die .tex-Datei \titel etc. nicht gesetzt hat
\providecommand{\titel}{}
\providecommand{\editorInnen}{}
\providecommand{\dateiname}{\jobname}

\vspace{3cm}

\vfill

\footnotesize
\textsc{Quelle}: \titel. Herausgegeben von {\editorInnen}. In: \emph{Arthur Schnitzler: Briefwechsel mit Autorinnen und Autoren}.
 Digitale Edition, https://schnitzler-briefe.acdh.oeaw.ac.at/{\dateiname}.html (Stand \today)
\fi

\end{document}


