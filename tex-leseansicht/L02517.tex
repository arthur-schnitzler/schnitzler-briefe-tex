%% latex-korrekturansicht-vorspann.tex
%% Vorspann für die Korrekturansicht.
%% Lädt die gemeinsame Datei latex-vorspann.tex mit gesetztem Schalter.

\newif\ifkorrekturansicht
\korrekturansichttrue

\input{../tex-inputs/latex-vorspann}


\section[Christiane von Hofmannsthal an Arthur Schnitzler, 5. 8. 1929]{L02517 Christiane von Hofmannsthal an Arthur Schnitzler, 5. 8. 1929}
\nopagebreak\mylabel{L02517v}
\rehead{ }\normalsize\beginnumbering\briefempfaengerindex{Schnitzler, Arthur@\textsc{Schnitzler, Arthur}!zzzZimmer, Christiane@\emph{von Christiane Zimmer}!1929-08-051@{13. 8. 1929}|(be}
\toendnotes[C]{\smallbreak\pagebreak[2]}\Standort{CUL, Schnitzler, B 43.}
\physDesc{Brief, 1 Blatt, 1 Seite, 714 Zeichen (Briefpapier mit Trauerrand)
\newline{}Schreibmaschine
\newline{}Handschrift: schwarze Tinte, lateinische Kurrent (\noindent{}Unterschrift, Adresse)
\newline{}Schnitzler: mit rotem Buntstift vier Unterstreichungen sowie die
                                 Beschriftung: »\textsc{Hofm}« und »\textsc{\uuline{Christiane}}« }
\pstart
           \raggedleft{}{\pb}Bad Aussee\oindex{Bad Aussee@\textbf{Bad Aussee}, \emph{P.PPLA3}|pw}, am 5. August 1929\pend
           
\pstart
           \raggedleft{}{[}hs.:{]} Obertressen 6\oindex{Obertressen@\textbf{Obertressen}, \emph{P.PPL}|pw}\pend
           
\pstart{}{[}ms.:{]} Lieber Arthur,\pend\vspace{0.5em}
\pstart
           Mama\pwindex{Hofmannsthal, Gertrude von 16.03.1880 – 09.11.1959@\textsc{Hofmannsthal, Gertrude von} (16.03.1880 – 09.11.1959)|pw} veranlasst mich, Ihnen für Ihren lieben
               Brief zu danken, da ihr schreiben noch schwer fällt.\pend
           
\pstart
           Wir wären Ihnen für baldigste Uebersendung der Briefe sowohl an Sie als an Gustav Schwarzkopf\pwindex{Schwarzkopf, Gustav 07.11.1853 – 13.11.1939@\textsc{Schwarzkopf, Gustav} (07.11.1853 – 13.11.1939), \emph{Schriftsteller/Schriftstellerin}|pw} und wenn Sie können auch der
               unveröffentlichten Gedichte an diesen, sehr dankbar, wir würden möglichst schnell
               2 Abschriften davon herstellen und Sie bekommen die Originale und eine Copie wieder
               zurück. Es ist uns doch sehr
               wichtig, das
               ganze vorhandene Material überschauen zu können, bezüglich einer Veröffentlichung
               würde natürlich nichts geschehen ohne Ihre ausdrückliche Einwilligung.\pend
           
\pstart
           Wir gehen sehr achtsam damit um.\pend
           
\pstart
           Mit herzlichstem Dank und vielen Grüssen{\\[\baselineskip]}\spacefill\mbox{{[}hs.:{]} Christiane}\pend
           \leftskip=0em{}\selectlanguage{ngerman}\endnumbering\briefempfaengerindex{Schnitzler, Arthur@\textsc{Schnitzler, Arthur}!zzzZimmer, Christiane@\emph{von Christiane Zimmer}!1929-08-051@{13. 8. 1929}|)be}\mylabel{L02517h}  \normalsize

\doendnotes{C}
\bigskip
\vfill

\clearpage

\footnotesize

\lohead{\textsc{register}}

% Definiere theindex-Environment komplett neu ohne reledmac
\makeatletter
\renewenvironment{theindex}{%
  \section*{\indexname}%
  \setlength{\parindent}{0pt}%
  \setlength{\parskip}{0pt plus 0.3pt}%
  \let\item\@idxitem
}{%
  \clearpage
}
\makeatother

\IfFileExists{\jobname-pw.ind}{\input{\jobname-pw.ind}}{}

\end{document}

      