%% latex-leseansicht-vorspann.tex
%% Vorspann für die Leseansicht.
%% Lädt die gemeinsame Datei latex-vorspann.tex mit nicht gesetztem Schalter.

\newif\ifkorrekturansicht
\korrekturansichtfalse

\input{../tex-inputs/latex-vorspann}


\section[Richard Beer-Hofmann an Arthur Schnitzler, 10. 9. 1895]{L00480 Richard Beer-Hofmann an Arthur Schnitzler, 10. 9. 1895}
\nopagebreak\mylabel{L00480v}
\rehead{ }\normalsize\beginnumbering\briefempfaengerindex{Schnitzler, Arthur@\textsc{Schnitzler, Arthur}!zzzBeer-Hofmann, Richard@\emph{von Richard Beer-Hofmann}!1895-09-101@{10. 9. 1895}|(be}
\toendnotes[C]{\smallbreak\pagebreak[2]}
\correspDesc{Versand  durch Richard Beer-Hofmann am 10. 9. 1895 in Schönberg im Stubaital
\newline{}Erhalt  durch Arthur Schnitzler im Zeitraum [11. 9. 1895
                  – 15. 9. 1895?] in Wien}\toendnotes[C]{\smallbreak}
\Standort{CUL, Schnitzler, B 8.}
\physDesc{Brief, 1 Blatt, 3 Seiten, 1197 Zeichen
\newline{}Handschrift: Bleistift, lateinische Kurrent
\newline{}Schnitzler: mit Bleistift nummeriert: »68« }
\buchAbdrucke{\weitereDrucke{Arthur Schnitzler, Richard Beer-Hofmann: \emph{Briefwechsel 1891–1931}. Herausgegeben von Konstanze Fliedl. Wien, Zürich: \emph{Europaverlag} 1992, S. 79.} }
\pstart
           \raggedleft{}{\pb}\uline{Schönberg im Stubaithal\oindex{Schönberg im Stubaital@\textbf{Schönberg im Stubaital}, \emph{Hauptstadt}|pw}}{\\}10 Sept 1895\pend
           \vspace{0.5em}
\pstart
           Lieber Arthur, ich bin nicht in Kopenhagen\oindex{Kopenhagen@\textbf{Kopenhagen}, \emph{Hauptstadt}|pw}; am Abend vor der Abreise entdeckte ich, daß ich gar nicht nach
                  Kopenhagen\oindex{Kopenhagen@\textbf{Kopenhagen}, \emph{Hauptstadt}|pw} wollte und sagte einfach ab. Ich
               hatte Sehnsucht, wirkliche Sehnsucht, allein zu sein. So einfach gieng es nicht. Ich
               mußte, oder, besser ließ mich bereden, in ein Compromiß zu willigen, \strikeout{nac} nach welchem ich nicht sofort aber doch in 3–4
               Tagen allein sein werde. Vorläufig ist {\pb}Frau Lou\pwindex{Andreas-Salomé, Lou 12.\,2.\,1861 Sankt Petersburg – 5.\,2.\,1937 Göttingen@\textsc{Andreas-Salomé, Lou} (12.\,2.\,1861 Sankt Petersburg – 5.\,2.\,1937 Göttingen), \emph{Schriftstellerin}|pw} mit mir gereist; sie reist aber Ende der Woche ab. \uline{Offiziell ist sie verhindert nach Kopenhagen\oindex{Kopenhagen@\textbf{Kopenhagen}, \emph{Hauptstadt}|pw} jetzt zu reisen und kann es erst im Oktober.}
               Ich bitte das festzuhalten.\pend
           
\pstart
           – Auch ihr gegenüber. –\pend
           
\pstart
           Für alle Fälle habe ich \introOben{}an\introOben{}{ }Gusti\pwindex{Glümer, Auguste 16.\,3.\,1862 Wien – 1956@\textsc{Glümer, Auguste} (16.\,3.\,1862 Wien – 1956), \emph{Lehrerin}|pwu} telegrafirt, ob sie nicht
               Ende der Woche ko{\geminationm}en kann und warte auf Antwort. So will
               ich allein sein. Aber – übrigens das lässt sich besser besprechen, als beschreiben.
               Hier ist {\pb}{[}es{]} einfach herrlich. Das Dorf liegt über der Brennerstrasse\oindex{Brenner@\textbf{Brenner}, \emph{Pass}|pw}{ }\strikeout{zirc} über 1000 Meter hoch zwei einviertel Stunden mit
               Wagen von Innsbruck\oindex{Innsbruck@\textbf{Innsbruck}, \emph{Verwaltungsgebiet}|pw}. Absolute Ruhe, ein kleines
               Gasthaus – »Jagerhof\oindex{Gasthaus Jagerhof@\textbf{Gasthaus Jagerhof}, \emph{Gastgewerbegebäude}|pw}« für Fremde eingerichtet,
               aber absolut nicht Hôtel. Heute übernachtete ich in einem Bauernhof, weil mein Zimmer
               erst heute frei wird. Aber Frau Lou\pwindex{Andreas-Salomé, Lou 12.\,2.\,1861 Sankt Petersburg – 5.\,2.\,1937 Göttingen@\textsc{Andreas-Salomé, Lou} (12.\,2.\,1861 Sankt Petersburg – 5.\,2.\,1937 Göttingen), \emph{Schriftstellerin}|pw} ko{\geminationm}t soeben an den Tisch. Adieu.\pend
           \pstart Herzlichst \spacefill\mbox{Richard}\pend{}\selectlanguage{ngerman}\endnumbering\briefempfaengerindex{Schnitzler, Arthur@\textsc{Schnitzler, Arthur}!zzzBeer-Hofmann, Richard@\emph{von Richard Beer-Hofmann}!1895-09-101@{10. 9. 1895}|)be}\mylabel{L00480h}  \newcommand{\dateiname}{L00480}\newcommand{\titel}{Richard Beer-Hofmann an Arthur Schnitzler, 10. 9. 1895}\newcommand{\editorInnen}{Martin Anton Müller und Gerd-Hermann Susen}%% latex-leseansicht-abspann.tex
%% Abspann für die Leseansicht.
%% Der Schalter \ifkorrekturansicht ist bereits durch den Vorspann gesetzt.

%% latex-abspann.tex
%% Gemeinsamer Abspann für Korrekturansicht und Leseansicht.
%% Setzt den Schalter \ifkorrekturansicht voraus (gesetzt in den
%% einbindenden Dateien latex-korrekturansicht-abspann.tex bzw.
%% latex-leseansicht-abspann.tex).
%% ---------------------------------------------------------------

\normalsize

% Das esempio-Environment wird nur in der Leseansicht benötigt
\ifkorrekturansicht\else
\newenvironment{esempio}[3]%
{
    \vspace{1.5ex}
    \rlap{\underline{#1}}
    \par
    \setlength{\parindent}{0cm}
    \nopagebreak
    \leftskip=#2cm
    \rightskip=#3cm
}
{
    \par
}
\fi

\doendnotes{C}
\bigskip
\vfill

\clearpage

\footnotesize

\ifkorrekturansicht
  \lohead{\textsc{register}}
\fi

% theindex-Environment neu definieren ohne reledmac
\makeatletter
\renewenvironment{theindex}{%
  \ifkorrekturansicht
    \section*{\indexname}%
  \else
    \subsubsection*{Index der erwähnten Entitäten}%
  \fi
  \setlength{\parindent}{0pt}%
  \setlength{\parskip}{0pt plus 0.3pt}%
  \let\item\@idxitem
}{%
  \ifkorrekturansicht\clearpage\fi
}
\makeatother

\IfFileExists{\jobname-pw.ind}{\input{\jobname-pw.ind}}{}

% Quellenangabe nur in der Leseansicht
\ifkorrekturansicht\else
% Fallback-Definitionen, falls die .tex-Datei \titel etc. nicht gesetzt hat
\providecommand{\titel}{}
\providecommand{\editorInnen}{}
\providecommand{\dateiname}{\jobname}

\vspace{3cm}

\vfill

\footnotesize
\textsc{Quelle}: \titel. Herausgegeben von {\editorInnen}. In: \emph{Arthur Schnitzler: Briefwechsel mit Autorinnen und Autoren}.
 Digitale Edition, https://schnitzler-briefe.acdh.oeaw.ac.at/{\dateiname}.html (Stand \today)
\fi

\end{document}


