%% latex-leseansicht-vorspann.tex
%% Vorspann für die Leseansicht.
%% Lädt die gemeinsame Datei latex-vorspann.tex mit nicht gesetztem Schalter.

\newif\ifkorrekturansicht
\korrekturansichtfalse

\input{../tex-inputs/latex-vorspann}


\section[Robert Adam an Arthur Schnitzler, 20. 11. 1916]{L02246 Robert Adam an Arthur Schnitzler, 20. 11. 1916}
\nopagebreak\mylabel{L02246v}
\rehead{ }\normalsize\beginnumbering\briefempfaengerindex{Schnitzler, Arthur@\textsc{Schnitzler, Arthur}!zzzAdam, Robert@\emph{von Robert Adam}!1916-11-201@{20. 11. 1916}|(be}
\toendnotes[C]{\smallbreak\pagebreak[2]}
\correspDesc{Versand  durch Robert Adam am 20. 11. 1916 in Wien
\newline{}Erhalt  durch Arthur Schnitzler im Zeitraum [20. 11. 1916 – 24. 11. 1916?] in Wien}\toendnotes[C]{\smallbreak}
\Standort{DLA, A:Schnitzler, HS.NZ85.1.4230,15.}
\physDesc{Brief, 2 Blätter, 7 Seiten, 5285 Zeichen
\newline{}Handschrift: schwarze Tinte, deutsche Kurrent
\newline{}Schnitzler: 1) auf der ersten Seite des ersten Blattes beschriftet: »\textsc{Adam}« und: »\textsc{Meidl Hptst 58}\oindex{XII., Meidling@\textbf{XII., Meidling}, \emph{Verwaltungsgebiet}|pw}.«  2) auf der ersten Seite des zweiten Blattes nummeriert:
                                    »5«}\Standort{Wien, Österreichische Nationalbibliothek, Cod.ser. 52.263, 180–181 recto.}
\physDesc{Brief, maschinenschriftliche Abschrift, 1 Blatt, 1 Seite, 5285 Zeichen
\newline{}Schreibmaschine}\toendnotes[C]{\smallbreak}
\pstart
           \raggedleft{}{\pb}Wien\oindex{Wien@\textbf{Wien}, \emph{Verwaltungsgebiet}|pw}, am 20. November 1916\pend
           
\pstart{}Hochverehrter Herr Doktor!\pend\vspace{0.5em}
\pstart
           Wäre mir Ihre Karte nicht zugekommen (für die ich Ihnen beſtens danke),{ }ſo hätte ich
               es mir kaum herausgenommen, vor Vollendung eines neuen \textsc{Opus}
               Ihnen zu{ }ſchreiben: und wie es mit meiner{ }ſchriftſtelleriſchen Tätigkeit jetzt
               beſchaffen iſt, \strikeout{ſo} hätten Sie vielleicht früher die
               zehn Memoirenbände\pwindex{\textcolor{red}{\textsuperscript{XXXX indx1}}!Meine Memoiren@\strich\emph{Meine Memoiren}|pwv} hinter{ }ſich
               gebracht als ich mich hätte melden dürfen. Ich bin nicht gewillt, unausgeſetzt zu
               lamentieren (wenigſtens nicht außerhalb des engſten Familienkreiſes), aber es koſtet
               mich{ }ſchwere Mühe, mit Klagen hauszuhalten: Amt, Kriegs{\pb}not, Mangel an Zeit und Ruhe, Klavierſpiel zu Häupten und unter mir,
               Kindergeſchrei, ungeheure Zerſplitterung und Bewußtſein unheilbaren Dilettantismus,
               Huſten und Schnupfen, Verdruß und Überdruß – und endlich auch das Böſeſte: manchmal
               etwas Neid. Wieviel muß da jedesmal beiſeite gedrückt und zerſtampft werden, bevor
               eine ruhige Komödienſeite geſchrieben werden kann!\pend
           
\pstart
           Trotz alledem habe ich eben den erſten Akt einer neuen Komödie, oder eher einer
               »Phantaſie« im erſten Anlauf fast ganz umriſſen; nicht der Märchenkomödie\pwindex{Adam, Robert 20.\,4.\,1877 Wien – 16.\,10.\,1961 Baden bei Wien@\textsc{Adam, Robert} (20.\,4.\,1877 Wien – 16.\,10.\,1961 Baden bei Wien), \emph{Schriftsteller, Richter}!Märchenkomödie@\strich\emph{Märchenkomödie}|pwv}, von der ich Ihnen das
               letztemal erzählte (da ich fühlte,{ }ſie würde viel zu bitter, zu gallig, zu triſt
               ausfallen,{ }ſchob ich{ }ſie entſchloſſen in die Lade){ }ſondern einer{ }ſonderbaren Ehſtandstragödie\pwindex{Adam, Robert 20.\,4.\,1877 Wien – 16.\,10.\,1961 Baden bei Wien@\textsc{Adam, Robert} (20.\,4.\,1877 Wien – 16.\,10.\,1961 Baden bei Wien), \emph{Schriftsteller, Richter}!Wundervogel@\strich\emph{Wundervogel}|pwv}, deren Stoff{ }ſich plötzlich bildete, als ich Kemmerichs\pwindex{Kemmerich, Max 6.\,5.\,1876 Koblenz – 6.\,4.\,1932 München@\textsc{Kemmerich, Max} (6.\,5.\,1876 Koblenz – 6.\,4.\,1932 München)|pw}
                  »Profezeiungen\pwindex{Kemmerich, Max 6.\,5.\,1876 Koblenz – 6.\,4.\,1932 München@\textsc{Kemmerich, Max} (6.\,5.\,1876 Koblenz – 6.\,4.\,1932 München)!Prophezeiungen – Wahn oder Wirklichkeit?@\strich\emph{Prophezeiungen – Wahn oder Wirklichkeit?}|pw}« las. Ob{ }ſie andren als mir
               genießbar{ }ſein wird, weiß ich nicht; mir liegt{ }ſie – trotz des ba{\pb}rocken Stoffs – am Herzen, weil{ }ſie viel aufzunehmen
               vermag, was in den letzten Jahren um mich und in mir Peinliches vorging.\pend
           
\pstart
           Ich habe den Verſuch unternommen, dieſes Stück in Alexandrinern zu{ }ſchreiben, nicht
               in den jambiſchen Trimetern mit Mittelzäſur, die in der deutſchen Literatur als
               Alexandriner gelten, sondern in einer dem franzöſiſchen\oindex{Frankreich@\textbf{Frankreich}|pw} Alexandriner nachgeahmten Versform. Das Stück{ }ſpielt im alten
                  Frankreich\oindex{Frankreich@\textbf{Frankreich}|pw}, und{ }ſo war mir etwas daran
               gelegen, auch die franzöſiſche\oindex{Frankreich@\textbf{Frankreich}|pw} Versart zu
               verwenden. Aber ach! Zwei Szenen waren fertig, mit Mühe fertiggeſtellt, und ich
               begann, zu zweifeln und zu zagen. Es iſt nämlich nicht leicht, im deutſchen,{ }ſofern
               es{ }ſich um längere Arbeiten handelt, unjambiſch zu{ }ſchreiben, der Rythmus{ }ſchlägt
               immer wieder in den Jambentakt um. Die Zäſur macht – mir wenigſtens – ungeheure
               Schwierigkeiten: es gibt{ }ſo wenig \strikeout{deutſche}
               mehrſilbige deutſche Worte, die auf der letzten Silbe betont {\pb}ſind und die Abtötung unnötiger Vokalauslaute, die in
               den romaniſchen Sprachen der Wortbildung{ }ſo ungemein entgegenkommt\strikeout{en}, iſt uns Sünde und Greuel. So kam es, daß ich nach
               den erſten zwei Szenen, mutlos geworden, den Alexandriner verabſchiedete und im
               Knittelvers oder gar in Blankverſen weiterſchrieb. Nunmehr aber tut es mir wieder
               leid: wäre ich{ }ſicher, daß{ }ſich die auf den Alexandriner verwandte Mühe lohnte (ich{ }ſchätze{ }ſie auf das zehnfache jener, die mich der Knittelvers koſten würde), das
               heißt: daß der deutſche Alexandriner nicht nur mir »klänge« und daß er nicht etwa gar
               als abwechslungslos = leiermäßig empfunden würde, dann möchte ich neuerdings, ohne
               die Arbeit zu{ }ſcheuen, Alexandriner zu{ }ſchmieden beginnen (es iſt{ }ſchon harte
               Schmiedearbeit).\pend
           
\pstart
           Und{ }ſo rücke ich mit der Frage und Bitte heraus, ob Sie, hochverehrter Herr Doktor,
               wenn anders Sie demnächſt einmal überflüſſige Zeit haben, mir {\pb}in dieser proſodiſchen Zweifelsfrage einen Ratſchlag
               erteilen möchten. Ich würde, wenn Sie hiezu bereit wären, Ihnen eine Probe der
               Alexandrinerſzenen entweder zuſenden oder vorlegen, wie es Ihnen lieber wäre. (Es
               handelt{ }ſich um jetzt noch ganz unfertige Konzepte, an die Sie, was den Inhalt
               anbetrifft, am beſten gar keinen Maßſtab anlegen dürften:{ }ſonſt müßte ich mich
               genieren). –\pend
           
\pstart
           Ihre freundliche Erkundigung nach meinem körperlichen Befinden kann ich – von den
               vorhin erwähnten Verkühlungserſcheinungen abgeſehen – damit beantworten, daß ich die
               tiefere Gegenden berührende\strikeout{re} Katarrhperiode für
               abgeſchloſſen halten darf; dicker bin ich allerdings noch nicht geworden und ich
               glaube auch nicht, daß mein Gewicht,{ }ſolang das \label{K_L02246-1v}\edtext{Fettkartenregime}{\lemma{\textnormal{\emph{Fettkartenregime}}}\Cendnote{\textnormal{Seit dem 17. 9. 1916 war der Erwerb von Rohfetten, Speiseöl und
                  Fettprodukten nur mit amtlichen Ausweisen erlaubt.}}}\label{K_L02246-1} andauert, {\pb}ſich{ }ſteigern wird.\pend
           
\pstart
           Ich habe in den letzten Tagen den \textsc{Jean Christophe}\pwindex{Rolland, Romain 29.\,1.\,1866 Clamecy – 30.\,12.\,1944 Vézelay@\textsc{Rolland, Romain} (29.\,1.\,1866 Clamecy – 30.\,12.\,1944 Vézelay), \emph{Schriftsteller}!Jean-Christophe@\strich\emph{Jean-Christophe}|pw} beendet und freue mich, daß \textsc{Romain Rolland}\pwindex{Rolland, Romain 29.\,1.\,1866 Clamecy – 30.\,12.\,1944 Vézelay@\textsc{Rolland, Romain} (29.\,1.\,1866 Clamecy – 30.\,12.\,1944 Vézelay), \emph{Schriftsteller}|pw} den Nobelpreis\orgindex{Nobelpreis@Nobelpreis|pw} erhalten hat. Welch
               ungeheures Unternehmen, die Kulturentwicklung der letzten dreißig Jahre und alle
               künſtleriſchen und{ }ſozialen Hauptprobleme, die während dieſer Zeit aufgerollt und
                  über\textcolor{gray}{taucht} wurden, im Rahmen eines Wilhelm Meiſter\pwindex{\textcolor{red}{\textsuperscript{XXXX indx1}}!Wilhelm Meister@\strich\emph{Wilhelm Meister}|pw}-Romans darzuſtellen und zugleich das innerſte
               Weſen der hauptbeteiligten Kulturvölker, ihre Haupttypen, Männer und Weiber, ohne je
               zu dozieren und ennuyant zu werden, mit Gründlichkeit und und pſychologiſcher
               Feinheit \strikeout{her} zu{ }ſchildern. Wunderbar, daß es kein
               Deutſcher war, der{ }ſolchen Plan faßte und ausführte; denn der Plan hat deutſchen
               Charakter, mag auch die Durchführung – was ich zu bedauern {\pb}der Letzte wäre – nicht deutſch = gründlich \substVorne{}\textsuperscript{iſt}\substDazwischen{}ſein\substHinten{}. Intereſſant iſt das Werk auch als erſte große Frucht der Einwirkung Nietzſche\pwindex{Nietzsche, Friedrich 15.\,10.\,1844 Röcken – 25.\,8.\,1900 Weimar@\textsc{Nietzsche, Friedrich} (15.\,10.\,1844 Röcken – 25.\,8.\,1900 Weimar), \emph{Schriftsteller, Philosoph}|pw}’ſcher Ideen auf ein nichtdeutſches
               Genie; und ich bin gewiß, daß den Verächter alles Nurdeutſchen über dieſe Erfüllung{ }ſeiner Peter Gaſt\pwindex{Gast, Peter 10.\,1.\,1854 Annaberg-Buchholz – 15.\,8.\,1918 ebd.@\textsc{Gast, Peter} (10.\,1.\,1854 Annaberg-Buchholz – 15.\,8.\,1918 ebd.), \emph{Schriftsteller, Komponist}|pw}-Träume, hätte er den \textsc{Jean Christophe}\pwindex{Rolland, Romain 29.\,1.\,1866 Clamecy – 30.\,12.\,1944 Vézelay@\textsc{Rolland, Romain} (29.\,1.\,1866 Clamecy – 30.\,12.\,1944 Vézelay), \emph{Schriftsteller}!Jean-Christophe@\strich\emph{Jean-Christophe}|pw} erlebt, in helle Begeiſterung geraten wäre. –\pend
           
\pstart
           Aber ich{ }ſchließe, um Sie nicht zu ermüden (obwohl ich über den \textsc{Jean Christophe}\pwindex{Rolland, Romain 29.\,1.\,1866 Clamecy – 30.\,12.\,1944 Vézelay@\textsc{Rolland, Romain} (29.\,1.\,1866 Clamecy – 30.\,12.\,1944 Vézelay), \emph{Schriftsteller}!Jean-Christophe@\strich\emph{Jean-Christophe}|pw} noch lange fortſchwärmen könnte).\pend
           
\pstart
           Mit den herzlichſten Grüßen Ihr ergebener\pend
           \pstart \spacefill\mbox{Robert Adam}\pend{}\selectlanguage{ngerman}\endnumbering\briefempfaengerindex{Schnitzler, Arthur@\textsc{Schnitzler, Arthur}!zzzAdam, Robert@\emph{von Robert Adam}!1916-11-201@{20. 11. 1916}|)be}\mylabel{L02246h}  \newcommand{\dateiname}{L02246}\newcommand{\titel}{Robert Adam an Arthur Schnitzler, 20. 11. 1916}\newcommand{\editorInnen}{Martin Anton Müller und Gerd-Hermann Susen}%% latex-leseansicht-abspann.tex
%% Abspann für die Leseansicht.
%% Der Schalter \ifkorrekturansicht ist bereits durch den Vorspann gesetzt.

%% latex-abspann.tex
%% Gemeinsamer Abspann für Korrekturansicht und Leseansicht.
%% Setzt den Schalter \ifkorrekturansicht voraus (gesetzt in den
%% einbindenden Dateien latex-korrekturansicht-abspann.tex bzw.
%% latex-leseansicht-abspann.tex).
%% ---------------------------------------------------------------

\normalsize

% Das esempio-Environment wird nur in der Leseansicht benötigt
\ifkorrekturansicht\else
\newenvironment{esempio}[3]%
{
    \vspace{1.5ex}
    \rlap{\underline{#1}}
    \par
    \setlength{\parindent}{0cm}
    \nopagebreak
    \leftskip=#2cm
    \rightskip=#3cm
}
{
    \par
}
\fi

\doendnotes{C}
\bigskip
\vfill

\clearpage

\footnotesize

\ifkorrekturansicht
  \lohead{\textsc{register}}
\fi

% theindex-Environment neu definieren ohne reledmac
\makeatletter
\renewenvironment{theindex}{%
  \ifkorrekturansicht
    \section*{\indexname}%
  \else
    \subsubsection*{Index der erwähnten Entitäten}%
  \fi
  \setlength{\parindent}{0pt}%
  \setlength{\parskip}{0pt plus 0.3pt}%
  \let\item\@idxitem
}{%
  \ifkorrekturansicht\clearpage\fi
}
\makeatother

\IfFileExists{\jobname-pw.ind}{\input{\jobname-pw.ind}}{}

% Quellenangabe nur in der Leseansicht
\ifkorrekturansicht\else
% Fallback-Definitionen, falls die .tex-Datei \titel etc. nicht gesetzt hat
\providecommand{\titel}{}
\providecommand{\editorInnen}{}
\providecommand{\dateiname}{\jobname}

\vspace{3cm}

\vfill

\footnotesize
\textsc{Quelle}: \titel. Herausgegeben von {\editorInnen}. In: \emph{Arthur Schnitzler: Briefwechsel mit Autorinnen und Autoren}.
 Digitale Edition, https://schnitzler-briefe.acdh.oeaw.ac.at/{\dateiname}.html (Stand \today)
\fi

\end{document}


