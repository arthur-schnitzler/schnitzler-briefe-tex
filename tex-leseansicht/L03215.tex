%% latex-leseansicht-vorspann.tex
%% Vorspann für die Leseansicht.
%% Lädt die gemeinsame Datei latex-vorspann.tex mit nicht gesetztem Schalter.

\newif\ifkorrekturansicht
\korrekturansichtfalse

\input{../tex-inputs/latex-vorspann}

\begin{center}
            \textcolor{red}{ENTWURF, NICHT FERTIG KORRIGIERT}
                      \end{center}
            
         
         \renewcommand{\erwaehntePersonen}{Personen: Olga Schnitzler}
         \renewcommand{\erwaehnteOrte}{Orte: Frankfurt am Main, Wien}
         \renewcommand{\erwaehnteWerke}{Werke: Der Schleier der Beatrice. Schauspiel in fünf Akten}
               \section[ Paul Goldmann an Arthur Schnitzler, 30. 7. {[}1902{]}]{ Paul Goldmann an Arthur Schnitzler, 30. 7. {[}1902{]}}\nopagebreak\mylabel{v}\rehead{ }\begin{ledgroupsized}[t]{13cm}\normalsize\beginnumbering \toendnotes[C]{\smallbreak\pagebreak[2]} \Standort{DLA, A:Schnitzler, HS.NZ85.1.3172.}
\physDesc{Brief, 1 Blatt, 4 Seiten
\newline{}Handschrift: blaue Tinte, deutsche Kurrent
\newline{}Schnitzler: mit Bleistift das Jahr »{[}1{]}90\textcolor{gray}{2}«
                                            vermerkt }\pstart
           \centering{}{\pb}Frankfurt\oindex{Frankfurt am Main@\textbf{Frankfurt am Main}|pw}{ }30. Juli.\pend
           \pstart\center{}Mein lieber Freund,\pend\pstart
           Ich bin hier auf der Durchreiſe nach der Schweiz\textcolor{red}{\textsuperscript{\textbf{KEY}}}.
                    Bitte, ſchreib’ mir ein Wort über Dein und \textsc{Olga\pwindex{Schnitzler, Olga 17.01.1882 – 13.01.1970@\textsc{Schnitzler, Olga} (17.01.1882 – 13.01.1970), \emph{Schauspielerin, Sängerin}|pw}s} Ergehen \textsc{Poste}\textsc{restante} nach \textsc{Mürren\textcolor{red}{\textsuperscript{\textbf{KEY}}}}\textsc{(Schweiz)\textcolor{red}{\textsuperscript{\textbf{KEY}}}}, wo ich etwa \strikeout{\textcolor{gray}{den}} am 5. Auguſt eintreffe. Läßt ſich ſchon der Tag des großen
                        {\pb}\strikeout{Ereigniſſ\textcolor{gray}{te}} Ereigniſſes ungeführ präciſiren? Ich wäre für eine Depeſche \strikeout{übe} über das Ereigniß ſelbſt ſehr dankbar und
                    möchte namentlich wiſſen, ob Du den Sohn\textcolor{red}{\textsuperscript{\textbf{KEY}}} haſt,
                        \strikeout{den} den ich Dir wünſche. \pend
           \pstart
           Hier habe ich Deine Geſchichten\textcolor{red}{\textsuperscript{\textbf{KEY}}} en der »Zeit\textcolor{red}{\textsuperscript{\textbf{KEY}}}« und in der »Jugend\textcolor{red}{\textsuperscript{\textbf{KEY}}}« geleſen. Die erſte\textcolor{red}{\textsuperscript{\textbf{KEY}}} hat mir gar nicht
                    gefallen, die zweite\textcolor{red}{\textsuperscript{\textbf{KEY}}} finde ich {\pb} köſtlich. Oh Gott, wenn Du doch der Humoriſt,
                    der glänzende Humoriſt \strikeout{i\textcolor{gray}{mm}} immer ſein wollteſt, \strikeout{d\textcolor{gray}{e}} der Du biſt! Einen Stoff\textcolor{gray}{e} humoriſtiſch behandeln\textcolor{red}{\textsuperscript{\textbf{KEY}}} heißt ſich über ihn erheben. Ich glaube, das ſollte in
                    den Jahren der Reife das höchſte Ziel ſein. \pend
           \pstart
           Bitte, grüße mir \textsc{Richard\textcolor{red}{\textsuperscript{\textbf{KEY}}}}. Es thut mir unendlich leid, daß ich {\pb}
                    Dich und ihn\textcolor{red}{\textsuperscript{\textbf{KEY}}} jetzt nicht ſehen werde. {\\[\baselineskip]}Viel
                    treue Grüße!\pend
           \leftskip=0em{}\pstart
           {\\[\baselineskip]}Dein\pend
           \leftskip=0em{}\pstart
           {\\[\baselineskip]}\spacefill\mbox{Paul Goldmann}\pend
           \leftskip=0em{}\pstart
           Grüße an die Hinterbrühl\textcolor{red}{\textsuperscript{\textbf{KEY}}}.Was iſt mit der»\textsc{Beatrice\pwindex{Schnitzler, Arthur 15.05.1862 – 21.10.1931@\textsc{Schnitzler, Arthur} (15.05.1862 – 21.10.1931), \emph{Schriftsteller, Mediziner}!Schleier der Beatrice. Schauspiel in fuenf Akten1900-12-01@\strich\emph{Der Schleier der Beatrice. Schauspiel in fünf Akten} {[}1900-12-01{]}|pw}}«? \pend
           
         
         \endnumbering\mylabel{h}\end{ledgroupsized}\begin{anhang}\end{anhang}\newcommand{\dateiname}{L03215}\newcommand{\titel}{Paul Goldmann an Arthur Schnitzler, 30. 7. [1902]}\newcommand{\editorInnen}{Martin Anton Müller und Laura Untner}%% latex-leseansicht-abspann.tex
%% Abspann für die Leseansicht.
%% Der Schalter \ifkorrekturansicht ist bereits durch den Vorspann gesetzt.

%% latex-abspann.tex
%% Gemeinsamer Abspann für Korrekturansicht und Leseansicht.
%% Setzt den Schalter \ifkorrekturansicht voraus (gesetzt in den
%% einbindenden Dateien latex-korrekturansicht-abspann.tex bzw.
%% latex-leseansicht-abspann.tex).
%% ---------------------------------------------------------------

\normalsize

% Das esempio-Environment wird nur in der Leseansicht benötigt
\ifkorrekturansicht\else
\newenvironment{esempio}[3]%
{
    \vspace{1.5ex}
    \rlap{\underline{#1}}
    \par
    \setlength{\parindent}{0cm}
    \nopagebreak
    \leftskip=#2cm
    \rightskip=#3cm
}
{
    \par
}
\fi

\doendnotes{C}
\bigskip
\vfill

\clearpage

\footnotesize

\ifkorrekturansicht
  \lohead{\textsc{register}}
\fi

% theindex-Environment neu definieren ohne reledmac
\makeatletter
\renewenvironment{theindex}{%
  \ifkorrekturansicht
    \section*{\indexname}%
  \else
    \subsubsection*{Index der erwähnten Entitäten}%
  \fi
  \setlength{\parindent}{0pt}%
  \setlength{\parskip}{0pt plus 0.3pt}%
  \let\item\@idxitem
}{%
  \ifkorrekturansicht\clearpage\fi
}
\makeatother

\IfFileExists{\jobname-pw.ind}{\input{\jobname-pw.ind}}{}

% Quellenangabe nur in der Leseansicht
\ifkorrekturansicht\else
% Fallback-Definitionen, falls die .tex-Datei \titel etc. nicht gesetzt hat
\providecommand{\titel}{}
\providecommand{\editorInnen}{}
\providecommand{\dateiname}{\jobname}

\vspace{3cm}

\vfill

\footnotesize
\textsc{Quelle}: \titel. Herausgegeben von {\editorInnen}. In: \emph{Arthur Schnitzler: Briefwechsel mit Autorinnen und Autoren}.
 Digitale Edition, https://schnitzler-briefe.acdh.oeaw.ac.at/{\dateiname}.html (Stand \today)
\fi

\end{document}


      