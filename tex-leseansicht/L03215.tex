%% latex-korrekturansicht-vorspann.tex
%% Vorspann für die Korrekturansicht.
%% Lädt die gemeinsame Datei latex-vorspann.tex mit gesetztem Schalter.

\newif\ifkorrekturansicht
\korrekturansichttrue

\input{../tex-inputs/latex-vorspann}


\section[ Paul Goldmann an Arthur Schnitzler, 30. 7. {[}1902{]}]{L03215 Paul Goldmann an Arthur Schnitzler, 30. 7. {[}1902{]}}
\nopagebreak\mylabel{L03215v}
\rehead{ }\normalsize\beginnumbering\briefempfaengerindex{Schnitzler, Arthur@\textsc{Schnitzler, Arthur}!zzzGoldmann, Paul@\emph{von Paul Goldmann}!1902-07-301@{30. 7. {[}1902{]}}|(be}
\toendnotes[C]{\smallbreak\pagebreak[2]}\Standort{DLA, A:Schnitzler, HS.NZ85.1.3172.}
\physDesc{Brief, 1 Blatt, 4 Seiten, 974 Zeichen
\newline{}Handschrift: blaue Tinte, deutsche Kurrent
\newline{}Schnitzler: mit Bleistift das Jahr »90\textcolor{gray}{2}« vermerkt }\toendnotes[C]{\smallbreak}
\pstart
           \centering{}{\pb}Frankfurt\oindex{Frankfurt am Main@\textbf{Frankfurt am Main}, \emph{P.PPLA3}|pw}{ }30. Juli.\pend
           
\pstart\center{}Mein lieber Freund,\pend\vspace{0.5em}
\pstart
           Ich bin hier auf der Durchreiſe nach der Schweiz\oindex{Schweiz@\textbf{Schweiz}, \emph{A.PCLI}|pw}. Bitte, ſchreib’ mir ein Wort über Dein und \textsc{Olgas\pwindex{Schnitzler, Olga 17.01.1882 – 13.01.1970@\textsc{Schnitzler, Olga} (17.01.1882 – 13.01.1970), \emph{Schauspieler/Schauspielerin, Sänger/Sängerin}|pw}} Ergehen \textsc{\begin{otherlanguage}{french}Poste restante\end{otherlanguage}} nach \textsc{Mürren\oindex{Muerren@\textbf{Mürren}, \emph{P.PPL}|pw} (Schweiz\oindex{Schweiz@\textbf{Schweiz}, \emph{A.PCLI}|pw})}, wo ich etwa \strikeout{\textcolor{gray}{den}} am 5. Auguſt eintreffe. Läßt ſich ſchon der Tag
               des großen {\pb}\strikeout{Ereigniſſte}{ }\label{K_L03215-1v}\edtext{Ereigniſſes}{\lemma{\textnormal{\emph{Ereigniſſes}}}\Cendnote{\textnormal{Heinrich Schnitzlers\pwindex{Schnitzler, Heinrich 09.08.1902 – 12.07.1982@\textsc{Schnitzler, Heinrich} (09.08.1902 – 12.07.1982), \emph{Regisseur/Regisseurin, Schauspieler/Schauspielerin}|pwk} Geburt am 9. 8. 1902}}}\label{K_L03215-1} ungefähr präciſiren? Ich wäre für eine Depeſche \strikeout{übe} über das Ereigniß ſelbſt ſehr dankbar und möchte namentlich wiſſen, ob
               Du den Sohn\pwindex{Schnitzler, Heinrich 09.08.1902 – 12.07.1982@\textsc{Schnitzler, Heinrich} (09.08.1902 – 12.07.1982), \emph{Regisseur/Regisseurin, Schauspieler/Schauspielerin}|pwv} haſt, \strikeout{den} den ich Dir wünſche.\pend
           
\pstart
           Hier habe ich Deine \label{K_L03215-2v}\edtext{Geſchichten\pwindex{Andreas Thameyers letzter Brief@\emph{Andreas Thameyers letzter Brief}|pwv}\pwindex{Excentric@\emph{Excentric}|pwv}}{\lemma{\textnormal{\emph{Geſchichten}}}\Cendnote{\textnormal{Arthur Schnitzler: \emph{Andreas Thameyers letzter Brief}\pwindex{Andreas Thameyers letzter Brief@\emph{Andreas Thameyers letzter Brief}|pwk}. In: \emph{Die Zeit. Wiener Wochenschrift}\pwindex{Zeit. Wiener Wochenschrift@\emph{Die Zeit. Wiener Wochenschrift}|pwk}, Jg. 32, Nr. 408, 26. 7. 1902, S. 63–64; Arthur Schnitzler: \emph{Excentric}\pwindex{Excentric@\emph{Excentric}|pwk}. In: \emph{Jugend}\pwindex{Jugend@\emph{Jugend}|pwk}, Jg. 7, Nr. 30, {[}16.{]} 7. 1902,
                     S. 492–496.}}}\label{K_L03215-2} in der »Zeit\pwindex{Zeit. Wiener Wochenschrift@\emph{Die Zeit. Wiener Wochenschrift}|pw}«
               und in der »Jugend\pwindex{Jugend@\emph{Jugend}|pw}« geleſen. Die erſte\pwindex{Andreas Thameyers letzter Brief@\emph{Andreas Thameyers letzter Brief}|pwv} hat mir gar nicht gefallen, die zweite\pwindex{Excentric@\emph{Excentric}|pwv} finde ich {\pb}köſtlich. Oh Gott, wenn Du doch der Humoriſt, der
               glänzende Humoriſt \strikeout{i\textcolor{gray}{mm}} immer ſein wollteſt, \strikeout{d\textcolor{gray}{e}} der Du biſt! Einen Stoff humoriſtiſch behandeln heißt ſich über ihn erheben.
               Ich glaube, das ſollte in den Jahren der Reife das höchſte Ziel ſein.\pend
           
\pstart
           Bitte, grüße mir \textsc{Richard\pwindex{Beer-Hofmann, Richard 1866-07-11 – 1945-09-26@\textsc{Beer-Hofmann, Richard} (1866-07-11 – 1945-09-26), \emph{Schriftsteller/Schriftstellerin}|pw}}. Es thut mir unendlich leid, daß ich {\pb}Dich und
                  ihn\pwindex{Beer-Hofmann, Richard 1866-07-11 – 1945-09-26@\textsc{Beer-Hofmann, Richard} (1866-07-11 – 1945-09-26), \emph{Schriftsteller/Schriftstellerin}|pwv} jetzt nicht ſehen
               werde.\pend
           
\pstart
           Viele treue Grüße! {\\[\baselineskip]}Dein {\\[\baselineskip]}\spacefill\mbox{Paul Goldmann}\pend
           \leftskip=0em{}
\pstart
           \noindent{}Grüße an die \label{K_L03215-3v}\edtext{Hinterbrühl\oindex{Hinterbruehl@\textbf{Hinterbrühl}, \emph{P.PPLA3}|pw}}{\lemma{\textnormal{\emph{Hinterbrühl}}}\Cendnote{\textnormal{Siehe Paul Goldmann an Arthur Schnitzler, 14. 1. [1902].
                  }}}\label{K_L03215-3}.\pend
           
\pstart
           Was iſt mit der \label{K_L03215-4v}\edtext{»\textsc{Beatrice\pwindex{Schleier der Beatrice. Schauspiel in fuenf Akten@\emph{Der Schleier der Beatrice. Schauspiel in fünf Akten}|pw}}«}{\lemma{\textnormal{\emph{»Beatrice«}}}\Cendnote{\textnormal{Siehe Paul Goldmann an Arthur Schnitzler, 14. 7. [1902].
                  }}}\label{K_L03215-4}?\pend
           \selectlanguage{ngerman}\endnumbering\briefempfaengerindex{Schnitzler, Arthur@\textsc{Schnitzler, Arthur}!zzzGoldmann, Paul@\emph{von Paul Goldmann}!1902-07-301@{30. 7. {[}1902{]}}|)be}\mylabel{L03215h}  \normalsize

\doendnotes{C}
\bigskip
\vfill

\clearpage

\footnotesize

\lohead{\textsc{register}}

% Definiere theindex-Environment komplett neu ohne reledmac
\makeatletter
\renewenvironment{theindex}{%
  \section*{\indexname}%
  \setlength{\parindent}{0pt}%
  \setlength{\parskip}{0pt plus 0.3pt}%
  \let\item\@idxitem
}{%
  \clearpage
}
\makeatother

\IfFileExists{\jobname-pw.ind}{\input{\jobname-pw.ind}}{}

\end{document}

      