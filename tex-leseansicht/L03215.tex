%% latex-leseansicht-vorspann.tex
%% Vorspann für die Leseansicht.
%% Lädt die gemeinsame Datei latex-vorspann.tex mit nicht gesetztem Schalter.

\newif\ifkorrekturansicht
\korrekturansichtfalse

\input{../tex-inputs/latex-vorspann}


\section[ Paul Goldmann an Arthur Schnitzler, 30. 7. {[}1902{]}]{L03215 Paul Goldmann an Arthur Schnitzler,  30. 7. [1902]}
\nopagebreak\mylabel{L03215v}
\rehead{ }\normalsize\beginnumbering\briefempfaengerindex{Schnitzler, Arthur@\textsc{Schnitzler, Arthur}!zzzGoldmann, Paul@\emph{von Paul Goldmann}!1902-07-301@{30. 7. [1902]}|(be}
\toendnotes[C]{\smallbreak\pagebreak[2]}
\correspDesc{Versand  durch Paul Goldmann am 30. 7. [1902] in Frankfurt am Main
\newline{}Erhalt  durch Arthur Schnitzler im Zeitraum [31. 7. 1902
                  – 4. 8. 1902?] in Wien?}\toendnotes[C]{\smallbreak}
\Standort{DLA, A:Schnitzler, HS.NZ85.1.3172.}
\physDesc{Brief, 1 Blatt, 4 Seiten, 974 Zeichen
\newline{}Handschrift: blaue Tinte, deutsche Kurrent
\newline{}Schnitzler: mit Bleistift das Jahr »90\textcolor{gray}{2}« vermerkt }\toendnotes[C]{\smallbreak}
\pstart
           \centering{}{\pb}Frankfurt\oindex{Frankfurt am Main@\textbf{Frankfurt am Main}, \emph{Hauptstadt}|pw}{ }30. Juli.\pend
           
\pstart\center{}Mein lieber Freund,\pend\vspace{0.5em}
\pstart
           Ich bin hier auf der Durchreiſe nach der Schweiz\oindex{Schweiz@\textbf{Schweiz}|pw}. Bitte,{ }ſchreib’ mir ein Wort über Dein und \textsc{Olgas\pwindex{Schnitzler, Olga 17.\,1.\,1882 Wien – 13.\,1.\,1970 Lugano@\textsc{Schnitzler, Olga} (17.\,1.\,1882 Wien – 13.\,1.\,1970 Lugano), \emph{Schauspielerin, Sängerin}|pw}} Ergehen \textsc{\begin{otherlanguage}{french}Poste restante\end{otherlanguage}} nach \textsc{Mürren\oindex{Mürren@\textbf{Mürren}|pw} (Schweiz\oindex{Schweiz@\textbf{Schweiz}|pw})}, wo ich etwa \strikeout{\textcolor{gray}{den}} am 5. Auguſt eintreffe. Läßt{ }ſich{ }ſchon der Tag
               des großen {\pb}\strikeout{Ereigniſſte}{ }\label{K_L03215-1v}\edtext{Ereigniſſes}{\lemma{\textnormal{\emph{Ereignisses}}}\Cendnote{\textnormal{Heinrich Schnitzlers\pwindex{Schnitzler, Heinrich 9.\,8.\,1902 Hinterbrühl – 12.\,7.\,1982 Wien@\textsc{Schnitzler, Heinrich} (9.\,8.\,1902 Hinterbrühl – 12.\,7.\,1982 Wien), \emph{Regisseur, Schauspieler}|pwk} Geburt am 9. 8. 1902}}}\label{K_L03215-1} ungefähr präciſiren? Ich wäre für eine Depeſche \strikeout{übe} über das Ereigniß{ }ſelbſt{ }ſehr dankbar und möchte namentlich wiſſen, ob
               Du den Sohn\pwindex{Schnitzler, Heinrich 9.\,8.\,1902 Hinterbrühl – 12.\,7.\,1982 Wien@\textsc{Schnitzler, Heinrich} (9.\,8.\,1902 Hinterbrühl – 12.\,7.\,1982 Wien), \emph{Regisseur, Schauspieler}|pwv} haſt, \strikeout{den} den ich Dir wünſche.\pend
           
\pstart
           Hier habe ich Deine \label{K_L03215-2v}\edtext{Geſchichten\pwindex{Schnitzler, Arthur 15.\,5.\,1862 Wien – 21.\,10.\,1931 ebd.@\textsc{Schnitzler, Arthur} (15.\,5.\,1862 Wien – 21.\,10.\,1931 ebd.), \emph{Schriftsteller, Mediziner}!Andreas Thameyers letzter Brief@\strich\emph{Andreas Thameyers letzter Brief}|pwv}\pwindex{Schnitzler, Arthur 15.\,5.\,1862 Wien – 21.\,10.\,1931 ebd.@\textsc{Schnitzler, Arthur} (15.\,5.\,1862 Wien – 21.\,10.\,1931 ebd.), \emph{Schriftsteller, Mediziner}!Excentric@\strich\emph{Excentric}|pwv}}{\lemma{\textnormal{\emph{Geschichten}}}\Cendnote{\textnormal{Arthur Schnitzler: \emph{Andreas Thameyers letzter Brief}\pwindex{Schnitzler, Arthur 15.\,5.\,1862 Wien – 21.\,10.\,1931 ebd.@\textsc{Schnitzler, Arthur} (15.\,5.\,1862 Wien – 21.\,10.\,1931 ebd.), \emph{Schriftsteller, Mediziner}!Andreas Thameyers letzter Brief@\strich\emph{Andreas Thameyers letzter Brief}|pwk}. In: \emph{Die Zeit. Wiener Wochenschrift}\pwindex{Zeit. Wiener Wochenschrift@\emph{Die Zeit. Wiener Wochenschrift}|pwk}, Jg. 32, Nr. 408, 26. 7. 1902, S. 63–64; Arthur Schnitzler: \emph{Excentric}\pwindex{Schnitzler, Arthur 15.\,5.\,1862 Wien – 21.\,10.\,1931 ebd.@\textsc{Schnitzler, Arthur} (15.\,5.\,1862 Wien – 21.\,10.\,1931 ebd.), \emph{Schriftsteller, Mediziner}!Excentric@\strich\emph{Excentric}|pwk}. In: \emph{Jugend}\pwindex{Jugend@\emph{Jugend}|pwk}, Jg. 7, Nr. 30, [16.] 7. 1902,
                     S. 492–496.}}}\label{K_L03215-2} in der »Zeit\pwindex{Zeit. Wiener Wochenschrift@\emph{Die Zeit. Wiener Wochenschrift}|pw}«
               und in der »Jugend\pwindex{Jugend@\emph{Jugend}|pw}« geleſen. Die erſte\pwindex{Schnitzler, Arthur 15.\,5.\,1862 Wien – 21.\,10.\,1931 ebd.@\textsc{Schnitzler, Arthur} (15.\,5.\,1862 Wien – 21.\,10.\,1931 ebd.), \emph{Schriftsteller, Mediziner}!Andreas Thameyers letzter Brief@\strich\emph{Andreas Thameyers letzter Brief}|pwv} hat mir gar nicht gefallen, die zweite\pwindex{Schnitzler, Arthur 15.\,5.\,1862 Wien – 21.\,10.\,1931 ebd.@\textsc{Schnitzler, Arthur} (15.\,5.\,1862 Wien – 21.\,10.\,1931 ebd.), \emph{Schriftsteller, Mediziner}!Excentric@\strich\emph{Excentric}|pwv} finde ich {\pb}köſtlich. Oh Gott, wenn Du doch der Humoriſt, der
               glänzende Humoriſt \strikeout{i\textcolor{gray}{mm}} immer{ }ſein wollteſt, \strikeout{d\textcolor{gray}{e}} der Du biſt! Einen Stoff humoriſtiſch behandeln heißt{ }ſich über ihn erheben.
               Ich glaube, das{ }ſollte in den Jahren der Reife das höchſte Ziel{ }ſein.\pend
           
\pstart
           Bitte, grüße mir \textsc{Richard\pwindex{Beer-Hofmann, Richard 11.\,7.\,1866 Wien – 26.\,9.\,1945 New York City@\textsc{Beer-Hofmann, Richard} (11.\,7.\,1866 Wien – 26.\,9.\,1945 New York City), \emph{Schriftsteller}|pw}}. Es thut mir unendlich leid, daß ich {\pb}Dich und
                  ihn\pwindex{Beer-Hofmann, Richard 11.\,7.\,1866 Wien – 26.\,9.\,1945 New York City@\textsc{Beer-Hofmann, Richard} (11.\,7.\,1866 Wien – 26.\,9.\,1945 New York City), \emph{Schriftsteller}|pwv} jetzt nicht{ }ſehen
               werde.\pend
           
\pstart
           Viele treue Grüße! {\\[\baselineskip]}Dein {\\[\baselineskip]}\spacefill\mbox{Paul Goldmann}\pend
           \leftskip=0em{}
\pstart
           \noindent{}Grüße an die \label{K_L03215-3v}\edtext{Hinterbrühl\oindex{Hinterbrühl@\textbf{Hinterbrühl}, \emph{Hauptstadt}|pw}}{\lemma{\textnormal{\emph{Hinterbrühl}}}\Cendnote{\textnormal{Siehe XXXX Auszeichnungsfehler: Dokument L03192 nicht gefunden.
                  }}}\label{K_L03215-3}.\pend
           
\pstart
           Was iſt mit der \label{K_L03215-4v}\edtext{»\textsc{Beatrice\pwindex{Schnitzler, Arthur 15.\,5.\,1862 Wien – 21.\,10.\,1931 ebd.@\textsc{Schnitzler, Arthur} (15.\,5.\,1862 Wien – 21.\,10.\,1931 ebd.), \emph{Schriftsteller, Mediziner}!Schleier der Beatrice. Schauspiel in fünf Akten@\strich\emph{Der Schleier der Beatrice. Schauspiel in fünf Akten}|pw}}«}{\lemma{\textnormal{\emph{»Beatrice«}}}\Cendnote{\textnormal{Siehe XXXX Auszeichnungsfehler: Dokument L03213 nicht gefunden.
                  }}}\label{K_L03215-4}?\pend
           \selectlanguage{ngerman}\endnumbering\briefempfaengerindex{Schnitzler, Arthur@\textsc{Schnitzler, Arthur}!zzzGoldmann, Paul@\emph{von Paul Goldmann}!1902-07-301@{30. 7. [1902]}|)be}\mylabel{L03215h}  \newcommand{\dateiname}{L03215}\newcommand{\titel}{Paul Goldmann an Arthur Schnitzler, 30. 7. [1902]}\newcommand{\editorInnen}{Martin Anton Müller und Laura Untner}%% latex-leseansicht-abspann.tex
%% Abspann für die Leseansicht.
%% Der Schalter \ifkorrekturansicht ist bereits durch den Vorspann gesetzt.

%% latex-abspann.tex
%% Gemeinsamer Abspann für Korrekturansicht und Leseansicht.
%% Setzt den Schalter \ifkorrekturansicht voraus (gesetzt in den
%% einbindenden Dateien latex-korrekturansicht-abspann.tex bzw.
%% latex-leseansicht-abspann.tex).
%% ---------------------------------------------------------------

\normalsize

% Das esempio-Environment wird nur in der Leseansicht benötigt
\ifkorrekturansicht\else
\newenvironment{esempio}[3]%
{
    \vspace{1.5ex}
    \rlap{\underline{#1}}
    \par
    \setlength{\parindent}{0cm}
    \nopagebreak
    \leftskip=#2cm
    \rightskip=#3cm
}
{
    \par
}
\fi

\doendnotes{C}
\bigskip
\vfill

\clearpage

\footnotesize

\ifkorrekturansicht
  \lohead{\textsc{register}}
\fi

% theindex-Environment neu definieren ohne reledmac
\makeatletter
\renewenvironment{theindex}{%
  \ifkorrekturansicht
    \section*{\indexname}%
  \else
    \subsubsection*{Index der erwähnten Entitäten}%
  \fi
  \setlength{\parindent}{0pt}%
  \setlength{\parskip}{0pt plus 0.3pt}%
  \let\item\@idxitem
}{%
  \ifkorrekturansicht\clearpage\fi
}
\makeatother

\IfFileExists{\jobname-pw.ind}{\input{\jobname-pw.ind}}{}

% Quellenangabe nur in der Leseansicht
\ifkorrekturansicht\else
% Fallback-Definitionen, falls die .tex-Datei \titel etc. nicht gesetzt hat
\providecommand{\titel}{}
\providecommand{\editorInnen}{}
\providecommand{\dateiname}{\jobname}

\vspace{3cm}

\vfill

\footnotesize
\textsc{Quelle}: \titel. Herausgegeben von {\editorInnen}. In: \emph{Arthur Schnitzler: Briefwechsel mit Autorinnen und Autoren}.
 Digitale Edition, https://schnitzler-briefe.acdh.oeaw.ac.at/{\dateiname}.html (Stand \today)
\fi

\end{document}


