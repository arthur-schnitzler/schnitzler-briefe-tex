%% latex-leseansicht-vorspann.tex
%% Vorspann für die Leseansicht.
%% Lädt die gemeinsame Datei latex-vorspann.tex mit nicht gesetztem Schalter.

\newif\ifkorrekturansicht
\korrekturansichtfalse

\input{../tex-inputs/latex-vorspann}


         
         \renewcommand{\erwaehntePersonen}{Personen: Richard Beer-Hofmann, Olga Schnitzler, Heinrich Schnitzler}
         \renewcommand{\erwaehnteOrte}{Orte: Frankfurt am Main, Hinterbrühl, Mürren, Schweiz, Wien}
         \renewcommand{\erwaehnteWerke}{Werke: Andreas Thameyers letzter Brief, Der Schleier der Beatrice. Schauspiel in fünf Akten, Die Zeit. Wiener Wochenschrift, Excentric, Jugend}
               \section[ Paul Goldmann an Arthur Schnitzler, 30. 7. {[}1902{]}]{ Paul Goldmann an Arthur Schnitzler, 30. 7. {[}1902{]}}\nopagebreak\mylabel{v}\rehead{ }\begin{ledgroupsized}[t]{13cm}\normalsize\beginnumbering \toendnotes[C]{\smallbreak\pagebreak[2]} \Standort{DLA, A:Schnitzler, HS.NZ85.1.3172.}
\physDesc{Brief, 1 Blatt, 4 Seiten, 975 Zeichen
\newline{}Handschrift: blaue Tinte, deutsche Kurrent
\newline{}Schnitzler: mit Bleistift das Jahr »{[}1{]}90\textcolor{gray}{2}« vermerkt }\toendnotes[C]{\smallbreak}\pstart
           \centering{}{\pb}Frankfurt\oindex{Frankfurt am Main@\textbf{Frankfurt am Main}|pw}{ }30. Juli.\pend
           \pstart\center{}Mein lieber Freund,\pend\pstart
           Ich bin hier auf der Durchreiſe nach der Schweiz\oindex{Schweiz@\textbf{Schweiz}|pw}. Bitte, ſchreib’ mir ein Wort über Dein und \textsc{Olga\pwindex{Schnitzler, Olga 17.01.1882 – 13.01.1970@\textsc{Schnitzler, Olga} (17.01.1882 – 13.01.1970), \emph{Schauspielerin, Sängerin}|pw}s} Ergehen \textsc{\begin{otherlanguage}{french}Poste restante\end{otherlanguage}} nach \textsc{Mürren\oindex{Muerren@\textbf{Mürren}|pw} (Schweiz\oindex{Schweiz@\textbf{Schweiz}|pw})}, wo ich etwa \strikeout{\textcolor{gray}{den}} am 5. Auguſt
               eintreffe. Läßt ſich ſchon der Tag des großen {\pb}\strikeout{Ereigniſſte}{ }\label{K_L03215-1v}\edtext{Ereigniſſes}{\lemma{\textnormal{\emph{Ereigniſſes}}}\Cendnote{\textnormal{Heinrich Schnitzler\pwindex{Schnitzler, Heinrich 09.08.1902 – 12.07.1982@\textsc{Schnitzler, Heinrich} (09.08.1902 – 12.07.1982), \emph{Regisseur, Schauspieler}|pwk}s Geburt am 9. 8. 1902}}}\label{K_L03215-1h} ungefähr präciſiren? Ich wäre für eine Depeſche \strikeout{übe} über das Ereigniß ſelbſt ſehr dankbar und möchte namentlich wiſſen, ob
               Du den Sohn\pwindex{Schnitzler, Heinrich 09.08.1902 – 12.07.1982@\textsc{Schnitzler, Heinrich} (09.08.1902 – 12.07.1982), \emph{Regisseur, Schauspieler}|pwv} haſt, \strikeout{den} den ich Dir wünſche.\pend
           \pstart
           Hier habe ich Deine \label{K_L03215-2v}\edtext{Geſchichten\pwindex{Schnitzler, Arthur 15.05.1862 – 21.10.1931@\textsc{Schnitzler, Arthur} (15.05.1862 – 21.10.1931), \emph{Schriftsteller, Mediziner}!Andreas Thameyers letzter Brief1902-07-26@\strich\emph{Andreas Thameyers letzter Brief} {[}1902-07-26{]}|pwv}\pwindex{Schnitzler, Arthur 15.05.1862 – 21.10.1931@\textsc{Schnitzler, Arthur} (15.05.1862 – 21.10.1931), \emph{Schriftsteller, Mediziner}!Excentric16. 07. 1902@\strich\emph{Excentric} {[}16. 07. 1902{]}|pwv}}{\lemma{\textnormal{\emph{Geſchichten}}}\Cendnote{\textnormal{Arthur Schnitzler\pwindex{Schnitzler, Arthur 15.05.1862 – 21.10.1931@\textsc{Schnitzler, Arthur} (15.05.1862 – 21.10.1931), \emph{Schriftsteller, Mediziner}|pwk}: \emph{Andreas Thameyers letzter Brief}\pwindex{Schnitzler, Arthur 15.05.1862 – 21.10.1931@\textsc{Schnitzler, Arthur} (15.05.1862 – 21.10.1931), \emph{Schriftsteller, Mediziner}!Andreas Thameyers letzter Brief1902-07-26@\strich\emph{Andreas Thameyers letzter Brief} {[}1902-07-26{]}|pwk}. In: \emph{Die Zeit. Wiener Wochenschrift}\pwindex{Zeit. Wiener Wochenschrift1894 – 1904@\emph{Die Zeit. Wiener Wochenschrift} {[}1894 – 1904{]}|pwk}, Jg. 32, Nr. 408, 26. 7. 1902, S. 63–64; Arthur Schnitzler\pwindex{Schnitzler, Arthur 15.05.1862 – 21.10.1931@\textsc{Schnitzler, Arthur} (15.05.1862 – 21.10.1931), \emph{Schriftsteller, Mediziner}|pwk}: \emph{Excentric}\pwindex{Schnitzler, Arthur 15.05.1862 – 21.10.1931@\textsc{Schnitzler, Arthur} (15.05.1862 – 21.10.1931), \emph{Schriftsteller, Mediziner}!Excentric16. 07. 1902@\strich\emph{Excentric} {[}16. 07. 1902{]}|pwk}. In: \emph{Jugend}\pwindex{?? Werk@Nicht ermittelte Verfasserinnen und Verfasser!Jugend1896 – 1940@\emph{Jugend} {[}1896 – 1940{]}|pwk}, Jg. 7, Nr. 30, {[}16.{]} 7. 1902,
                     S. 492–496.}}}\label{K_L03215-2h} in der »Zeit\pwindex{Zeit. Wiener Wochenschrift1894 – 1904@\emph{Die Zeit. Wiener Wochenschrift} {[}1894 – 1904{]}|pw}«
               und in der »Jugend\pwindex{?? Werk@Nicht ermittelte Verfasserinnen und Verfasser!Jugend1896 – 1940@\emph{Jugend} {[}1896 – 1940{]}|pw}« geleſen. Die erſte\pwindex{Schnitzler, Arthur 15.05.1862 – 21.10.1931@\textsc{Schnitzler, Arthur} (15.05.1862 – 21.10.1931), \emph{Schriftsteller, Mediziner}!Andreas Thameyers letzter Brief1902-07-26@\strich\emph{Andreas Thameyers letzter Brief} {[}1902-07-26{]}|pwv} hat mir gar nicht gefallen, die zweite\pwindex{Schnitzler, Arthur 15.05.1862 – 21.10.1931@\textsc{Schnitzler, Arthur} (15.05.1862 – 21.10.1931), \emph{Schriftsteller, Mediziner}!Excentric16. 07. 1902@\strich\emph{Excentric} {[}16. 07. 1902{]}|pwv} finde ich {\pb}köſtlich. Oh Gott, wenn Du doch der Humoriſt, der
               glänzende Humoriſt \strikeout{i\textcolor{gray}{mm}} immer ſein wollteſt, \strikeout{d\textcolor{gray}{e}} der Du biſt! Einen Stoff humoriſtiſch behandeln heißt ſich über ihn erheben.
               Ich glaube, das ſollte in den Jahren der Reife das höchſte Ziel ſein.\pend
           \pstart
           Bitte, grüße mir \textsc{Richard\pwindex{Beer-Hofmann, Richard 1866-07-11 – 1945-09-26@\textsc{Beer-Hofmann, Richard} (1866-07-11 – 1945-09-26), \emph{Schriftsteller}|pw}}. Es thut mir unendlich leid, daß ich {\pb}Dich und
                  ihn\pwindex{Beer-Hofmann, Richard 1866-07-11 – 1945-09-26@\textsc{Beer-Hofmann, Richard} (1866-07-11 – 1945-09-26), \emph{Schriftsteller}|pwv} jetzt nicht ſehen
               werde.\pend
           \pstart
           Viele treue Grüße! {\\[\baselineskip]}Dein {\\[\baselineskip]}\spacefill\mbox{Paul Goldmann}\pend
           \leftskip=0em{}\pstart
           \noindent{}Grüße an die \label{K_L03215-3v}\edtext{Hinterbrühl\oindex{Hinterbruehl@\textbf{Hinterbrühl}|pw}}{\lemma{\textnormal{\emph{Hinterbrühl}}}\Cendnote{\textnormal{siehe Paul Goldmann an Arthur Schnitzler, 14. 1. [1902]}}}\label{K_L03215-3h}.\pend
           \pstart
           Was iſt mit der \label{K_L03215-6v}\edtext{»\textsc{Beatrice\pwindex{Schnitzler, Arthur 15.05.1862 – 21.10.1931@\textsc{Schnitzler, Arthur} (15.05.1862 – 21.10.1931), \emph{Schriftsteller, Mediziner}!Schleier der Beatrice. Schauspiel in fuenf Akten1900-12-01@\strich\emph{Der Schleier der Beatrice. Schauspiel in fünf Akten} {[}1900-12-01{]}|pw}}«}{\lemma{\textnormal{\emph{»Beatrice«}}}\Cendnote{\textnormal{siehe Paul Goldmann an Arthur Schnitzler, 14. 7. [1902]}}}\label{K_L03215-6h}?\pend
           
         
         \endnumbering\mylabel{h}\end{ledgroupsized}  \newcommand{\dateiname}{L03215}\newcommand{\titel}{Paul Goldmann an Arthur Schnitzler, 30. 7. [1902]}\newcommand{\editorInnen}{Martin Anton Müller und Laura Untner}%% latex-leseansicht-abspann.tex
%% Abspann für die Leseansicht.
%% Der Schalter \ifkorrekturansicht ist bereits durch den Vorspann gesetzt.

%% latex-abspann.tex
%% Gemeinsamer Abspann für Korrekturansicht und Leseansicht.
%% Setzt den Schalter \ifkorrekturansicht voraus (gesetzt in den
%% einbindenden Dateien latex-korrekturansicht-abspann.tex bzw.
%% latex-leseansicht-abspann.tex).
%% ---------------------------------------------------------------

\normalsize

% Das esempio-Environment wird nur in der Leseansicht benötigt
\ifkorrekturansicht\else
\newenvironment{esempio}[3]%
{
    \vspace{1.5ex}
    \rlap{\underline{#1}}
    \par
    \setlength{\parindent}{0cm}
    \nopagebreak
    \leftskip=#2cm
    \rightskip=#3cm
}
{
    \par
}
\fi

\doendnotes{C}
\bigskip
\vfill

\clearpage

\footnotesize

\ifkorrekturansicht
  \lohead{\textsc{register}}
\fi

% theindex-Environment neu definieren ohne reledmac
\makeatletter
\renewenvironment{theindex}{%
  \ifkorrekturansicht
    \section*{\indexname}%
  \else
    \subsubsection*{Index der erwähnten Entitäten}%
  \fi
  \setlength{\parindent}{0pt}%
  \setlength{\parskip}{0pt plus 0.3pt}%
  \let\item\@idxitem
}{%
  \ifkorrekturansicht\clearpage\fi
}
\makeatother

\IfFileExists{\jobname-pw.ind}{\input{\jobname-pw.ind}}{}

% Quellenangabe nur in der Leseansicht
\ifkorrekturansicht\else
% Fallback-Definitionen, falls die .tex-Datei \titel etc. nicht gesetzt hat
\providecommand{\titel}{}
\providecommand{\editorInnen}{}
\providecommand{\dateiname}{\jobname}

\vspace{3cm}

\vfill

\footnotesize
\textsc{Quelle}: \titel. Herausgegeben von {\editorInnen}. In: \emph{Arthur Schnitzler: Briefwechsel mit Autorinnen und Autoren}.
 Digitale Edition, https://schnitzler-briefe.acdh.oeaw.ac.at/{\dateiname}.html (Stand \today)
\fi

\end{document}


      