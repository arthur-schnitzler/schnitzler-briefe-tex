%% latex-korrekturansicht-vorspann.tex
%% Vorspann für die Korrekturansicht.
%% Lädt die gemeinsame Datei latex-vorspann.tex mit gesetztem Schalter.

\newif\ifkorrekturansicht
\korrekturansichttrue

\input{../tex-inputs/latex-vorspann}


\section[Arthur Schnitzler an Richard Beer-Hofmann, {[}22.? 8. 1891{]}]{L00036 Arthur Schnitzler an Richard Beer-Hofmann, {[}22.? 8. 1891{]}}
\nopagebreak\mylabel{L00036v}
\rehead{ }\normalsize\beginnumbering\briefempfaengerindex{Beer-Hofmann, Richard@\textsc{Beer-Hofmann, Richard}!zzzSchnitzler, Arthur@\emph{von Arthur Schnitzler}!1891-08-222@{{[}22.? 8. 1891{]}}|(be}
\toendnotes[C]{\smallbreak\pagebreak[2]}\Standort{YCGL, MSS 31.}
\physDesc{Briefkarte, 393 Zeichen
\newline{}Handschrift: schwarze Tinte, deutsche Kurrent}\toendnotes[C]{\smallbreak}
\pstart
           \noindent{}{\pb}Lieber Richard! Prof. \textsc{Kraft-Ebing}\pwindex{Krafft-Ebing, Richard von 14.08.1840 – 22.12.1902@\textsc{Krafft-Ebing, Richard von} (14.08.1840 – 22.12.1902), \emph{Sexualforscher/Sexualforscherin, Psychiater/Psychiaterin}|pw} iſt noch nicht in Wien\oindex{Wien@\textbf{Wien}, \emph{A.ADM2}|pw}; er ſoll etwa am
                  20. September wieder eintreffen\pend
           
\pstart
           – Loris\pwindex{Hofmannsthal, Hugo von 1874-02-01 – 1929-07-15@\textsc{Hofmannsthal, Hugo von} (1874-02-01 – 1929-07-15), \emph{Schriftsteller/Schriftstellerin}|pw} hat mir noch nach Iſchl\oindex{Bad Ischl@\textbf{Bad Ischl}, \emph{P.PPL}|pw} geſchrieben, die \label{K_L00036-1v}\edtext{Karte}{\lemma{\textnormal{\emph{Karte}}}\Cendnote{\textnormal{Vgl. Hugo von Hofmannsthal an Arthur Schnitzler, [16. 8. 1891].
               }}}\label{K_L00036-1} wurde mir nachgeſchickt, er war an jenen Tagen nicht wohl.\pend
           
\pstart
           – Ueber meine Pläne weiß ich ſelber {\pb}noch nichts,
               halten Sie mich jedenfalls am Laufenden, was Sie zu thun gedenken. – Laune
               miſerabel.\pend
           
\pstart
           Herzlichen Gruſs{\\[\baselineskip]}Ihr{\\[\baselineskip]}\spacefill\mbox{Arthur.}\pend
           \leftskip=0em{}
\pstart
           \noindent{}\textsc{Salten\pwindex{Salten, Felix 06.09.1869 – 08.10.1945@\textsc{Salten, Felix} (06.09.1869 – 08.10.1945), \emph{Schriftsteller/Schriftstellerin, Journalist/Journalistin, Chefredakteur/Chefredakteurin}|pw}} grüßt Sie herzlich. –\pend
           \selectlanguage{ngerman}\endnumbering\briefempfaengerindex{Beer-Hofmann, Richard@\textsc{Beer-Hofmann, Richard}!zzzSchnitzler, Arthur@\emph{von Arthur Schnitzler}!1891-08-222@{{[}22.? 8. 1891{]}}|)be}\mylabel{L00036h}  \normalsize

\doendnotes{C}
\bigskip
\vfill

\clearpage

\footnotesize

\lohead{\textsc{register}}

% Definiere theindex-Environment komplett neu ohne reledmac
\makeatletter
\renewenvironment{theindex}{%
  \section*{\indexname}%
  \setlength{\parindent}{0pt}%
  \setlength{\parskip}{0pt plus 0.3pt}%
  \let\item\@idxitem
}{%
  \clearpage
}
\makeatother

\IfFileExists{\jobname-pw.ind}{\input{\jobname-pw.ind}}{}

\end{document}

      