%% latex-leseansicht-vorspann.tex
%% Vorspann für die Leseansicht.
%% Lädt die gemeinsame Datei latex-vorspann.tex mit nicht gesetztem Schalter.

\newif\ifkorrekturansicht
\korrekturansichtfalse

\input{../tex-inputs/latex-vorspann}


\section[Gertrud Rung an Arthur Schnitzler, 17. 2. 1925]{L02433 Gertrud Rung an Arthur Schnitzler, 17. 2. 1925}
\nopagebreak\mylabel{L02433v}
\rehead{ }\normalsize\beginnumbering\briefempfaengerindex{Schnitzler, Arthur@\textsc{Schnitzler, Arthur}!zzzRung, Gertrud@\emph{von Gertrud Rung}!1925-02-171@{17. 2. 1925}|(be}
\toendnotes[C]{\smallbreak\pagebreak[2]}
\correspDesc{Versand  durch Gertrud Rung am 17. 2. 1925 in Kopenhagen
\newline{}Erhalt  durch Arthur Schnitzler im Zeitraum [18. 2. 1925
                  – 22. 2. 1925?] in Wien}\toendnotes[C]{\smallbreak}
\Standort{CUL, Schnitzler, B 17.}
\physDesc{Brief, 1 Blatt, 2 Seiten, 1042 Zeichen
\newline{}Handschrift: schwarze Tinte, lateinische Kurrent
\newline{}Schnitzler: 1) mit Bleistift beschriftet: »\noindent{}(\textsc{Brandes}{ / }\textsc{Rung})«  2) mit rotem Buntstift vereinzelte Unterstreichungen
\newline{}Ordnung: mit Bleistift von unbekannter Hand nummeriert:
                                    »56« }
\buchAbdrucke{\weitereDrucke{Georg Brandes, Arthur Schnitzler: \emph{Ein Briefwechsel}. Herausgegeben von Kurt Bergel. Bern: \emph{Francke} 1956, S. 144.} }\toendnotes[C]{\smallbreak}
\pstart
           \raggedleft{}{\pb}Kopenhagen\oindex{Kopenhagen@\textbf{Kopenhagen}, \emph{Hauptstadt}|pw}{ }17-2-25\pend
           
\pstart{}Hochverehrter Herr.\pend\vspace{0.5em}
\pstart
           Dr Georg Brandes\pwindex{Brandes, Georg 4.\,2.\,1842 Kopenhagen – 19.\,2.\,1927 ebd.@\textsc{Brandes, Georg} (4.\,2.\,1842 Kopenhagen – 19.\,2.\,1927 ebd.)|pw} bittet Sie dringend ihm nicht
               zu verübeln, daß er Ihnen diesmal nicht persönlich schreibt. Die bevorstehende
               Vortragsreise nimmt die Zeit des Doktors derartig in Anspruch, daß er zu müde ist
               sein Correspondenz selber zu führen.\pend
           
\pstart
           Dr Brandes\pwindex{Brandes, Georg 4.\,2.\,1842 Kopenhagen – 19.\,2.\,1927 ebd.@\textsc{Brandes, Georg} (4.\,2.\,1842 Kopenhagen – 19.\,2.\,1927 ebd.)|pw} beauftragt mich deshalb, Ihnen,
               hochverehrter Herr, zu sagen, daß es ihm eine ganz besondere Freude sein wird sich
               mit Ihnen irgendwo zusammen zu treffen.\hspace*{1.5em}Der erste
                  {\pb}\label{K_L02433-1v}\edtext{Vortrag}{\lemma{\textnormal{\emph{Vortrag}}}\Cendnote{\textnormal{Brandes\pwindex{Brandes, Georg 4.\,2.\,1842 Kopenhagen – 19.\,2.\,1927 ebd.@\textsc{Brandes, Georg} (4.\,2.\,1842 Kopenhagen – 19.\,2.\,1927 ebd.)|pwk} hielt in Berlin\oindex{Berlin@\textbf{Berlin}, \emph{Hauptstadt}|pwk} nur einen Vortrag: am 31. 3. 1925 im Blüthner-Saal\oindex{Blüthner-Saal@\textbf{Blüthner-Saal}, \emph{Veranstaltungsgebäude}|pwk} zum Thema: \emph{Das heutige Europa}\pwindex{Brandes, Georg 4.\,2.\,1842 Kopenhagen – 19.\,2.\,1927 ebd.@\textsc{Brandes, Georg} (4.\,2.\,1842 Kopenhagen – 19.\,2.\,1927 ebd.)!heutige Europa@\strich\emph{Das heutige Europa}|pwk}. Brandes\pwindex{Brandes, Georg 4.\,2.\,1842 Kopenhagen – 19.\,2.\,1927 ebd.@\textsc{Brandes, Georg} (4.\,2.\,1842 Kopenhagen – 19.\,2.\,1927 ebd.)|pwk}’ Aufenthalt und sein Vortrag fanden in der Berlin\oindex{Berlin@\textbf{Berlin}, \emph{Hauptstadt}|pwk}er Presse große Resonanz (vgl. A. F. Cohn\pwindex{Cohn, Alfons Fedor 12.\,4.\,1878 – 24.\,1.\,1933@\textsc{Cohn, Alfons Fedor} (12.\,4.\,1878 – 24.\,1.\,1933), \emph{Schriftsteller, Journalist, Beamter}|pwk}: \emph{Georg Brandes in Berlin}\pwindex{Cohn, Alfons Fedor 12.\,4.\,1878 – 24.\,1.\,1933@\textsc{Cohn, Alfons Fedor} (12.\,4.\,1878 – 24.\,1.\,1933), \emph{Schriftsteller, Journalist, Beamter}!Georg Brandes in Berlin@\strich\emph{Georg Brandes in Berlin}|pwk}. In: \emph{Berliner Tageblatt}\pwindex{Berliner Tageblatt@\emph{Berliner Tageblatt}|pwk}, Jg. 54, Nr. 153,
                        31. 3. 1925, Abend-Ausgabe, S. 4). Anschließend fuhr Brandes\pwindex{Brandes, Georg 4.\,2.\,1842 Kopenhagen – 19.\,2.\,1927 ebd.@\textsc{Brandes, Georg} (4.\,2.\,1842 Kopenhagen – 19.\,2.\,1927 ebd.)|pwk} nach Wien\oindex{Wien@\textbf{Wien}, \emph{Verwaltungsgebiet}|pwk}, um dort am 8. 4. 1931 den Vortrag zu
                  wiederholen.}}}\label{K_L02433-1} soll in Berlin\oindex{Berlin@\textbf{Berlin}, \emph{Hauptstadt}|pw} am 25 März stattfinden, der zweite folgt innerhalb einer Woche. Dr Brandes\pwindex{Brandes, Georg 4.\,2.\,1842 Kopenhagen – 19.\,2.\,1927 ebd.@\textsc{Brandes, Georg} (4.\,2.\,1842 Kopenhagen – 19.\,2.\,1927 ebd.)|pw} weißt noch nicht in welchem Hotel er
               wohnen wird, weil sein Impresario\pwindex{Span, J. @\textsc{Span, J.}, \emph{Veranstalter/Veranstalterin}|pwv} dies für ihn arrangieren wird.\hspace*{1.5em}Dr Brandes bittet\pwindex{Brandes, Georg 4.\,2.\,1842 Kopenhagen – 19.\,2.\,1927 ebd.@\textsc{Brandes, Georg} (4.\,2.\,1842 Kopenhagen – 19.\,2.\,1927 ebd.)|pw} Sie deshalb die Güte haben
               zu wollen bei diesem Herrn, J. Span\pwindex{Span, J. @\textsc{Span, J.}, \emph{Veranstalter/Veranstalterin}|pw}, Berlinerstraße 149 Charlottenburg\oindex{Straße des 17. Juni@\textbf{Straße des 17. Juni}, \emph{Straße}|pw}, Ihre Adresse
               abzugeben, so daß er sich gleich nach seiner Ankunft in Verbindung mit Ihnen setzen
               kann.\pend
           
\pstart
           Dr Brandes\pwindex{Brandes, Georg 4.\,2.\,1842 Kopenhagen – 19.\,2.\,1927 ebd.@\textsc{Brandes, Georg} (4.\,2.\,1842 Kopenhagen – 19.\,2.\,1927 ebd.)|pw} bittet Sie um seine Ergebenheit und
               warme Freundschaft versichert zu sein und grüßt Sie auf das herzlichste.\pend
           
\pstart
           Mit vorzüglicher Hochachtung{\\[\baselineskip]}für Dr. Georg
                  Brandes\pwindex{Brandes, Georg 4.\,2.\,1842 Kopenhagen – 19.\,2.\,1927 ebd.@\textsc{Brandes, Georg} (4.\,2.\,1842 Kopenhagen – 19.\,2.\,1927 ebd.)|pw}{\\[\baselineskip]}\spacefill\mbox{G. Rung / Sekretär.}\pend
           \leftskip=0em{}\selectlanguage{ngerman}\endnumbering\briefempfaengerindex{Schnitzler, Arthur@\textsc{Schnitzler, Arthur}!zzzRung, Gertrud@\emph{von Gertrud Rung}!1925-02-171@{17. 2. 1925}|)be}\mylabel{L02433h}  \newcommand{\dateiname}{L02433}\newcommand{\titel}{Gertrud Rung an Arthur Schnitzler, 17. 2. 1925}\newcommand{\editorInnen}{Martin Anton Müller und Gerd-Hermann Susen}%% latex-leseansicht-abspann.tex
%% Abspann für die Leseansicht.
%% Der Schalter \ifkorrekturansicht ist bereits durch den Vorspann gesetzt.

%% latex-abspann.tex
%% Gemeinsamer Abspann für Korrekturansicht und Leseansicht.
%% Setzt den Schalter \ifkorrekturansicht voraus (gesetzt in den
%% einbindenden Dateien latex-korrekturansicht-abspann.tex bzw.
%% latex-leseansicht-abspann.tex).
%% ---------------------------------------------------------------

\normalsize

% Das esempio-Environment wird nur in der Leseansicht benötigt
\ifkorrekturansicht\else
\newenvironment{esempio}[3]%
{
    \vspace{1.5ex}
    \rlap{\underline{#1}}
    \par
    \setlength{\parindent}{0cm}
    \nopagebreak
    \leftskip=#2cm
    \rightskip=#3cm
}
{
    \par
}
\fi

\doendnotes{C}
\bigskip
\vfill

\clearpage

\footnotesize

\ifkorrekturansicht
  \lohead{\textsc{register}}
\fi

% theindex-Environment neu definieren ohne reledmac
\makeatletter
\renewenvironment{theindex}{%
  \ifkorrekturansicht
    \section*{\indexname}%
  \else
    \subsubsection*{Index der erwähnten Entitäten}%
  \fi
  \setlength{\parindent}{0pt}%
  \setlength{\parskip}{0pt plus 0.3pt}%
  \let\item\@idxitem
}{%
  \ifkorrekturansicht\clearpage\fi
}
\makeatother

\IfFileExists{\jobname-pw.ind}{\input{\jobname-pw.ind}}{}

% Quellenangabe nur in der Leseansicht
\ifkorrekturansicht\else
% Fallback-Definitionen, falls die .tex-Datei \titel etc. nicht gesetzt hat
\providecommand{\titel}{}
\providecommand{\editorInnen}{}
\providecommand{\dateiname}{\jobname}

\vspace{3cm}

\vfill

\footnotesize
\textsc{Quelle}: \titel. Herausgegeben von {\editorInnen}. In: \emph{Arthur Schnitzler: Briefwechsel mit Autorinnen und Autoren}.
 Digitale Edition, https://schnitzler-briefe.acdh.oeaw.ac.at/{\dateiname}.html (Stand \today)
\fi

\end{document}


