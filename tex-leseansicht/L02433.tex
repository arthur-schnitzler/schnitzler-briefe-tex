%% latex-korrekturansicht-vorspann.tex
%% Vorspann für die Korrekturansicht.
%% Lädt die gemeinsame Datei latex-vorspann.tex mit gesetztem Schalter.

\newif\ifkorrekturansicht
\korrekturansichttrue

\input{../tex-inputs/latex-vorspann}


\section[Gertrud Rung an Arthur Schnitzler, 17. 2. 1925]{L02433 Gertrud Rung an Arthur Schnitzler, 17. 2. 1925}
\nopagebreak\mylabel{L02433v}
\rehead{ }\normalsize\beginnumbering\briefempfaengerindex{Schnitzler, Arthur@\textsc{Schnitzler, Arthur}!zzzRung, Gertrud@\emph{von Gertrud Rung}!1925-02-171@{17. 2. 1925}|(be}
\toendnotes[C]{\smallbreak\pagebreak[2]}\Standort{CUL, Schnitzler, B 17.}
\physDesc{Brief, 1 Blatt, 2 Seiten, 1042 Zeichen
\newline{}Handschrift: schwarze Tinte, lateinische Kurrent
\newline{}Schnitzler: 1) mit Bleistift beschriftet: »\noindent{}(\textsc{Brandes}{ / }\textsc{Rung})«  2) mit rotem Buntstift vereinzelte Unterstreichungen
\newline{}Ordnung: mit Bleistift von unbekannter Hand nummeriert:
                                    »56« }
\buchAbdrucke{\weitereDrucke{Georg Brandes, Arthur Schnitzler: \emph{Ein Briefwechsel}. Bern: \emph{Francke} 1956, S. 144.} }\toendnotes[C]{\smallbreak}
\pstart
           \raggedleft{}{\pb}Kopenhagen\oindex{Kopenhagen@\textbf{Kopenhagen}, \emph{P.PPLC}|pw}{ }17-2-25\pend
           
\pstart{}Hochverehrter Herr.\pend\vspace{0.5em}
\pstart
           Dr Georg Brandes\pwindex{Brandes, Georg 04.02.1842 – 19.02.1927@\textsc{Brandes, Georg} (04.02.1842 – 19.02.1927)|pw} bittet Sie dringend ihm nicht
               zu verübeln, daß er Ihnen diesmal nicht persönlich schreibt. Die bevorstehende
               Vortragsreise nimmt die Zeit des Doktors derartig in Anspruch, daß er zu müde ist
               sein Correspondenz selber zu führen.\pend
           
\pstart
           Dr Brandes\pwindex{Brandes, Georg 04.02.1842 – 19.02.1927@\textsc{Brandes, Georg} (04.02.1842 – 19.02.1927)|pw} beauftragt mich deshalb, Ihnen,
               hochverehrter Herr, zu sagen, daß es ihm eine ganz besondere Freude sein wird sich
               mit Ihnen irgendwo zusammen zu treffen.\hspace*{1.5em}Der erste
                  {\pb}\label{K_L02433-1v}\edtext{Vortrag}{\lemma{\textnormal{\emph{Vortrag}}}\Cendnote{\textnormal{Brandes\pwindex{Brandes, Georg 04.02.1842 – 19.02.1927@\textsc{Brandes, Georg} (04.02.1842 – 19.02.1927)|pwk} hielt in Berlin\oindex{Berlin@\textbf{Berlin}, \emph{P.PPLC}|pwk} nur einen Vortrag: am 31. 3. 1925 im Blüthner-Saal\oindex{Bluethner-Saal@\textbf{Blüthner-Saal}, \emph{Veranstaltungsgebäude (K.VSB)}|pwk} zum Thema: \emph{Das heutige Europa}\pwindex{heutige Europa@\emph{Das heutige Europa}|pwk}. Brandes\pwindex{Brandes, Georg 04.02.1842 – 19.02.1927@\textsc{Brandes, Georg} (04.02.1842 – 19.02.1927)|pwk}’ Aufenthalt und sein Vortrag fanden in der Berlin\oindex{Berlin@\textbf{Berlin}, \emph{P.PPLC}|pwk}er Presse große Resonanz (vgl. A. F. Cohn\pwindex{Cohn, Alfons Fedor 12.04.1878 – 24.01.1933@\textsc{Cohn, Alfons Fedor} (12.04.1878 – 24.01.1933), \emph{Schriftsteller/Schriftstellerin, Journalist/Journalistin, Beamter/Beamte}|pwk}: \emph{Georg Brandes in Berlin}\pwindex{Georg Brandes in Berlin@\emph{Georg Brandes in Berlin}|pwk}. In: \emph{Berliner Tageblatt}\pwindex{Berliner Tageblatt@\emph{Berliner Tageblatt}|pwk}, Jg. 54, Nr. 153,
                        31. 3. 1925, Abend-Ausgabe, S. 4). Anschließend fuhr Brandes\pwindex{Brandes, Georg 04.02.1842 – 19.02.1927@\textsc{Brandes, Georg} (04.02.1842 – 19.02.1927)|pwk} nach Wien\oindex{Wien@\textbf{Wien}, \emph{A.ADM2}|pwk}, um dort am 8. 4. 1931 den Vortrag zu
                  wiederholen.}}}\label{K_L02433-1} soll in Berlin\oindex{Berlin@\textbf{Berlin}, \emph{P.PPLC}|pw} am 25
                  März stattfinden, der zweite folgt innerhalb einer Woche. Dr Brandes\pwindex{Brandes, Georg 04.02.1842 – 19.02.1927@\textsc{Brandes, Georg} (04.02.1842 – 19.02.1927)|pw} weißt noch nicht in welchem Hotel er
               wohnen wird, weil sein Impresario\pwindex{Span, J. @\textsc{Span, J.}, \emph{Veranstalter/Veranstalterin}|pwv} dies für ihn arrangieren wird.\hspace*{1.5em}Dr Brandes bittet\pwindex{Brandes, Georg 04.02.1842 – 19.02.1927@\textsc{Brandes, Georg} (04.02.1842 – 19.02.1927)|pw} Sie deshalb die Güte haben
               zu wollen bei diesem Herrn, J. Span\pwindex{Span, J. @\textsc{Span, J.}, \emph{Veranstalter/Veranstalterin}|pw}, Berlinerstraße 149 Charlottenburg\oindex{Strasse des 17. Juni@\textbf{Straße des 17. Juni}, \emph{Straße (K.STR)}|pw}, Ihre Adresse
               abzugeben, so daß er sich gleich nach seiner Ankunft in Verbindung mit Ihnen setzen
               kann.\pend
           
\pstart
           Dr Brandes\pwindex{Brandes, Georg 04.02.1842 – 19.02.1927@\textsc{Brandes, Georg} (04.02.1842 – 19.02.1927)|pw} bittet Sie um seine Ergebenheit und
               warme Freundschaft versichert zu sein und grüßt Sie auf das herzlichste.\pend
           
\pstart
           Mit vorzüglicher Hochachtung{\\[\baselineskip]}für Dr. Georg
                  Brandes\pwindex{Brandes, Georg 04.02.1842 – 19.02.1927@\textsc{Brandes, Georg} (04.02.1842 – 19.02.1927)|pw}{\\[\baselineskip]}\spacefill\mbox{G. Rung / Sekretär.}\pend
           \leftskip=0em{}\selectlanguage{ngerman}\endnumbering\briefempfaengerindex{Schnitzler, Arthur@\textsc{Schnitzler, Arthur}!zzzRung, Gertrud@\emph{von Gertrud Rung}!1925-02-171@{17. 2. 1925}|)be}\mylabel{L02433h}  \normalsize

\doendnotes{C}
\bigskip
\vfill

\clearpage

\footnotesize

\lohead{\textsc{register}}

% Definiere theindex-Environment komplett neu ohne reledmac
\makeatletter
\renewenvironment{theindex}{%
  \section*{\indexname}%
  \setlength{\parindent}{0pt}%
  \setlength{\parskip}{0pt plus 0.3pt}%
  \let\item\@idxitem
}{%
  \clearpage
}
\makeatother

\IfFileExists{\jobname-pw.ind}{\input{\jobname-pw.ind}}{}

\end{document}

      