%% latex-leseansicht-vorspann.tex
%% Vorspann für die Leseansicht.
%% Lädt die gemeinsame Datei latex-vorspann.tex mit nicht gesetztem Schalter.

\newif\ifkorrekturansicht
\korrekturansichtfalse

\input{../tex-inputs/latex-vorspann}


         
         \newcommand{\erwaehntePersonen}{Personen: Georg Brandes, Alfons Fedor Cohn, J.  Span}
         \newcommand{\erwaehnteInstitutionen}{}
         \newcommand{\erwaehnteOrte}{Orte: Berlin, Blüthner-Saal, Kopenhagen, Straße des 17. Juni, Wien}
         \newcommand{\erwaehnteWerke}{Werke: Berliner Tageblatt, Das heutige Europa, Georg Brandes in Berlin}
               \section[Gertrud Rung an Arthur Schnitzler, 17. 2. 1925]{ Gertrud Rung an Arthur Schnitzler, 17. 2. 1925}\nopagebreak\mylabel{v}\rehead{ }\begin{ledgroupsized}[t]{13cm}\normalsize\beginnumbering \toendnotes[C]{\smallbreak\pagebreak[2]} \Standort{CUL, Schnitzler, B 17.}
\physDesc{Brief, 1 Blatt, 2 Seiten
\newline{}Handschrift: schwarze Tinte, lateinische Kurrent
\newline{}Schnitzler: 1) mit Bleistift beschriftet: »\noindent{}(\textsc{Brandes}{ / }\textsc{Rung})«  2) mit rotem Buntstift vereinzelte Unterstreichungen\newline{}Ordnung: mit Bleistift von unbekannter Hand nummeriert:
                                    »56« }\buchAbdrucke{\weitereDrucke{Georg Brandes, Arthur Schnitzler: \emph{Ein Briefwechsel}. Hg. Kurt Bergel. Bern: \emph{Francke} 1956, S. 144.} }\toendnotes[C]{\smallbreak}\pstart
           \raggedleft{}{\pb}Kopenhagen\oindex{Kopenhagen@\textbf{Kopenhagen}|pw}{ }17-2-25\pend
           \pstart{}Hochverehrter Herr.\pend\pstart
           Dr Georg Brandes\pwindex{Brandes, Georg 04.02.1842 – 19.02.1927@\textsc{Brandes, Georg} (04.02.1842 – 19.02.1927)|pw} bittet Sie dringend ihm nicht zu
               verübeln, daß er Ihnen diesmal nicht persönlich schreibt. Die bevorstehende
               Vortragsreise nimmt die Zeit des Doktors derartig in Anspruch, daß er zu müde ist
               sein Correspondenz selber zu führen.\pend
           \pstart
           Dr Brandes\pwindex{Brandes, Georg 04.02.1842 – 19.02.1927@\textsc{Brandes, Georg} (04.02.1842 – 19.02.1927)|pw} beauftragt mich deshalb, Ihnen,
               hochverehrter Herr, zu sagen, daß es ihm eine ganz besondere Freude sein wird sich
               mit Ihnen irgendwo zusammen zu treffen.\hspace*{1.5em}Der erste
                  {\pb}\label{K_L02433_1v}\edtext{Vortrag}{\lemma{\textnormal{\emph{Vortrag}}}\Cendnote{\textnormal{Brandes\pwindex{Brandes, Georg 04.02.1842 – 19.02.1927@\textsc{Brandes, Georg} (04.02.1842 – 19.02.1927)|pwk} hielt in Berlin\oindex{Berlin@\textbf{Berlin}|pwk} nur einen Vortrag: am 31. 3. 1925 im Blüthner-Saal\oindex{Bluethner-Saal@\textbf{Blüthner-Saal}|pwk} zum Thema: \emph{Das
                     heutige Europa}\pwindex{Brandes, Georg 04.02.1842 – 19.02.1927@\textsc{Brandes, Georg} (04.02.1842 – 19.02.1927)!heutige Europa1925@\strich\emph{Das heutige Europa} {[}1925{]}|pwk}. Brandes\pwindex{Brandes, Georg 04.02.1842 – 19.02.1927@\textsc{Brandes, Georg} (04.02.1842 – 19.02.1927)|pwk}’ Aufenthalt
                  und sein Vortrag fanden in der Berlin\oindex{Berlin@\textbf{Berlin}|pwk}er Presse
                  große Resonanz (vgl. A. F. Cohn\pwindex{Cohn, Alfons Fedor 12.04.1878 – 24.01.1933@\textsc{Cohn, Alfons Fedor} (12.04.1878 – 24.01.1933), \emph{Schriftsteller, Journalist, Beamter}|pwk}:
                        \emph{Georg Brandes in Berlin}\pwindex{Georg Brandes in Berlin31. 3. 1925@\emph{Georg Brandes in Berlin} {[}31. 3. 1925{]}|pwk}. In: \emph{Berliner Tageblatt}\pwindex{?? Werk@Nicht ermittelte Verfasserinnen und Verfasser!Berliner Tageblatt1872 – 1939@\emph{Berliner Tageblatt} {[}1872 – 1939{]}|pwk}, Jg. 54, Nr. 153,
                        31. 3. 1925, Abend-Ausgabe, S. 4. Anschließend fuhr Brandes\pwindex{Brandes, Georg 04.02.1842 – 19.02.1927@\textsc{Brandes, Georg} (04.02.1842 – 19.02.1927)|pwk} nach Wien\oindex{Wien@\textbf{Wien}|pwk}, um dort am 8. 4. 1931 den Vortrag zu
                  wiederholen.}}}\label{K_L02433_1h} soll in Berlin\oindex{Berlin@\textbf{Berlin}|pw} am 25
                  März stattfinden, der zweite folgt innerhalb einer Woche. Dr Brandes\pwindex{Brandes, Georg 04.02.1842 – 19.02.1927@\textsc{Brandes, Georg} (04.02.1842 – 19.02.1927)|pw} weißt noch nicht in welchem Hotel er
               wohnen wird, weil sein Impresario\pwindex{Span, J. @\textsc{Span, J.}, \emph{Veranstalter/Veranstalterin}|pwv} dies für ihn arrangieren wird.\hspace*{1.5em}Dr Brandes bittet\pwindex{Brandes, Georg 04.02.1842 – 19.02.1927@\textsc{Brandes, Georg} (04.02.1842 – 19.02.1927)|pw} Sie deshalb die Güte haben zu
               wollen bei diesem Herrn, J. Span\pwindex{Span, J. @\textsc{Span, J.}, \emph{Veranstalter/Veranstalterin}|pw}, Berlinerstraße 149 Charlottenburg\oindex{Strasse des 17. Juni@\textbf{Straße des 17. Juni}|pw}, Ihre Adresse abzugeben, so
               daß er sich gleich nach seiner Ankunft in Verbindung mit Ihnen setzen kann.\pend
           \pstart
           Dr Brandes\pwindex{Brandes, Georg 04.02.1842 – 19.02.1927@\textsc{Brandes, Georg} (04.02.1842 – 19.02.1927)|pw} bittet Sie um seine Ergebenheit und
               warme Freundschaft versichert zu sein und grüßt Sie auf das herzlichste.\pend
           \pstart
           Mit vorzüglicher Hochachtung{\\[\baselineskip]}für Dr. Georg
                  Brandes\pwindex{Brandes, Georg 04.02.1842 – 19.02.1927@\textsc{Brandes, Georg} (04.02.1842 – 19.02.1927)|pw}{\\[\baselineskip]}\spacefill\mbox{G. Rung / Sekretär.}\pend
           \leftskip=0em{}
         
         \endnumbering\mylabel{h}\end{ledgroupsized}  \newcommand{\dateiname}{L02433}\newcommand{\titel}{Gertrud Rung an Arthur Schnitzler, 17. 2. 1925}\newcommand{\editorInnen}{Martin Anton Müller und Gerd-Hermann Susen}%% latex-leseansicht-abspann.tex
%% Abspann für die Leseansicht.
%% Der Schalter \ifkorrekturansicht ist bereits durch den Vorspann gesetzt.

%% latex-abspann.tex
%% Gemeinsamer Abspann für Korrekturansicht und Leseansicht.
%% Setzt den Schalter \ifkorrekturansicht voraus (gesetzt in den
%% einbindenden Dateien latex-korrekturansicht-abspann.tex bzw.
%% latex-leseansicht-abspann.tex).
%% ---------------------------------------------------------------

\normalsize

% Das esempio-Environment wird nur in der Leseansicht benötigt
\ifkorrekturansicht\else
\newenvironment{esempio}[3]%
{
    \vspace{1.5ex}
    \rlap{\underline{#1}}
    \par
    \setlength{\parindent}{0cm}
    \nopagebreak
    \leftskip=#2cm
    \rightskip=#3cm
}
{
    \par
}
\fi

\doendnotes{C}
\bigskip
\vfill

\clearpage

\footnotesize

\ifkorrekturansicht
  \lohead{\textsc{register}}
\fi

% theindex-Environment neu definieren ohne reledmac
\makeatletter
\renewenvironment{theindex}{%
  \ifkorrekturansicht
    \section*{\indexname}%
  \else
    \subsubsection*{Index der erwähnten Entitäten}%
  \fi
  \setlength{\parindent}{0pt}%
  \setlength{\parskip}{0pt plus 0.3pt}%
  \let\item\@idxitem
}{%
  \ifkorrekturansicht\clearpage\fi
}
\makeatother

\IfFileExists{\jobname-pw.ind}{\input{\jobname-pw.ind}}{}

% Quellenangabe nur in der Leseansicht
\ifkorrekturansicht\else
% Fallback-Definitionen, falls die .tex-Datei \titel etc. nicht gesetzt hat
\providecommand{\titel}{}
\providecommand{\editorInnen}{}
\providecommand{\dateiname}{\jobname}

\vspace{3cm}

\vfill

\footnotesize
\textsc{Quelle}: \titel. Herausgegeben von {\editorInnen}. In: \emph{Arthur Schnitzler: Briefwechsel mit Autorinnen und Autoren}.
 Digitale Edition, https://schnitzler-briefe.acdh.oeaw.ac.at/{\dateiname}.html (Stand \today)
\fi

\end{document}


      