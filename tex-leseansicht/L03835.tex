%% latex-leseansicht-vorspann.tex
%% Vorspann für die Leseansicht.
%% Lädt die gemeinsame Datei latex-vorspann.tex mit nicht gesetztem Schalter.

\newif\ifkorrekturansicht
\korrekturansichtfalse

\input{../tex-inputs/latex-vorspann}


\section[Theodor Herzl an Arthur Schnitzler, 8. 11. 1894]{L03835 Theodor Herzl an Arthur Schnitzler, 8. 11. 1894}
\nopagebreak\mylabel{L03835v}
\rehead{ }\normalsize\beginnumbering\briefempfaengerindex{Schnitzler, Arthur@\textsc{Schnitzler, Arthur}!zzzHerzl, Theodor@\emph{von Theodor Herzl}!1894-11-083@{8. 11. 1894}|(be}
\toendnotes[C]{\smallbreak\pagebreak[2]}
\correspDesc{Versand  durch Theodor Herzl am 8. 11. 1894 in Paris
\newline{}Erhalt  durch Arthur Schnitzler im Zeitraum [9. 11. 1894
                  – 13. 11. 1894?] in Wien}\toendnotes[C]{\smallbreak}
\Standort{CUL, Schnitzler, B 39.}
\physDesc{Brief, 1 Blatt, 4 Seiten, 4400 Zeichen
\newline{}Handschrift: schwarze Tinte, lateinische Kurrent
\newline{}Ordnung: 1) mit Bleistift von unbekannter Hand nummeriert: »14«  2) mit blauem Buntstift mutmaßlich von Leon
                                    Kellner\pwindex{Kellner, Leon 17.\,4.\,1859 Tarnów – 5.\,12.\,1928 Wien@\textsc{Kellner, Leon} (17.\,4.\,1859 Tarnów – 5.\,12.\,1928 Wien), \emph{Zionist, Literaturhistoriker, Anglist}|pw} Markierung interessanter Stellen}
\buchAbdrucke{\weitereDrucke{Theodor Herzl: \emph{Briefe und
                        autobiographische Notizen 1866–1895}. Bearbeitet von Johannes Wachten in Zusammenarbeit mit Chaya Harel, Daisy Tycho und Manfred Winkler. Berlin, Frankfurt am Main, Wien: \emph{Propyläen} 1983, S. 553–555 (Briefe und Tagebücher. Herausgegeben von Alex Bein, Hermann Greive, Moshe Schaerf, Julius H. Schoeps und Johannes Wachten, 1).} }\toendnotes[C]{\smallbreak}
\pstart
           {\pb}\textcolor{gray}{\textbf{NOUVELLE PRESSE LIBRE }}\orgindex{Neue Freie Presse@Neue Freie Presse|pw}\hfill \textcolor{gray}{\textbf{8, Rue de Monceau }}\oindex{8, rue de Monceau@\textbf{8, rue de Monceau}, \emph{Wohngebäude}|pw}\pend
           
\pstart
           \textcolor{gray}{\textbf{D\textsuperscript{r}{ }TH. HERZL}}\hfill 8 November 894\pend
           
\pstart{}Mein lieber Schnitzer!\pend\vspace{0.5em}
\pstart
           Zu dieser Unternehmung brauche ich einen \begin{otherlanguage}{english}gentleman\end{otherlanguage}
               und Künstler. Ich habe an Sie gedacht. Die Sache ist folgende.\pend
           
\pstart
           Ich habe ein neues Stück\pwindex{Herzl, Theodor 2.\,5.\,1860 Budapest – 3.\,7.\,1904 Edlach@\textsc{Herzl, Theodor} (2.\,5.\,1860 Budapest – 3.\,7.\,1904 Edlach), \emph{Schriftsteller, Journalist}!neue Ghetto. Schauspiel in vier Acten@\strich\emph{Das neue Ghetto. Schauspiel in vier Acten}|pwv}
               geschrieben – in einem Rausch, den Ihnen die Herstellungszeit sagen wird. Der erste
               Act wurde begonnen am 21 October der vierte, letzte beendigt am 8 November. Also siebzehn Tage. Ist’s gut oder schlecht geworden – ich weiss
               es nicht. Wer weiss das?\pend
           
\pstart
           Aber die Stimmung, die ich sonst während des Schreibens u. nachher immer hatte, ist
               stärker als jemals. Sie besteht darin, dass ich neben dem leidenschaftlichen Wunsch,
               mein Werk\pwindex{Herzl, Theodor 2.\,5.\,1860 Budapest – 3.\,7.\,1904 Edlach@\textsc{Herzl, Theodor} (2.\,5.\,1860 Budapest – 3.\,7.\,1904 Edlach), \emph{Schriftsteller, Journalist}!neue Ghetto. Schauspiel in vier Acten@\strich\emph{Das neue Ghetto. Schauspiel in vier Acten}|pwv} der Welt
               mitzutheilen den noch viel, viel leidenschaftlicheren habe, mich zu verbergen u. zu
               vergraben. Es ist Hochmuth, Feigheit oder Scham, oder was Sie wollen. Es ist so.\pend
           
\pstart
           Im besonderen Fall dieses Stückes\pwindex{Herzl, Theodor 2.\,5.\,1860 Budapest – 3.\,7.\,1904 Edlach@\textsc{Herzl, Theodor} (2.\,5.\,1860 Budapest – 3.\,7.\,1904 Edlach), \emph{Schriftsteller, Journalist}!neue Ghetto. Schauspiel in vier Acten@\strich\emph{Das neue Ghetto. Schauspiel in vier Acten}|pwv} will ich meine Geschlechtstheile noch mehr verbergen, als
               irgendwann. Das Stück\pwindex{Herzl, Theodor 2.\,5.\,1860 Budapest – 3.\,7.\,1904 Edlach@\textsc{Herzl, Theodor} (2.\,5.\,1860 Budapest – 3.\,7.\,1904 Edlach), \emph{Schriftsteller, Journalist}!neue Ghetto. Schauspiel in vier Acten@\strich\emph{Das neue Ghetto. Schauspiel in vier Acten}|pwv} hat
               einen ganz besonderen Charakter, Sie werden es sehen, wenn Sie es lesen.\pend
           
\pstart
           Ich will also nicht als Autor bekannt werden, wenigstens vorläufig u. durch einige
               Monate oder Jahre nicht. Und dazu brauche ich die Mithilfe {\pb}eines feuerfesten, wasserdichten
               Freundes, der mir sein förmliches Ehrenwort gibt, zu schweigen u. mit keiner Miene zu
               verrathen, was er weiss, bevor ich ihn ebenso förmlich des Ehrenwortes entlasse.\pend
           
\pstart
           Wollen Sie das thun?\pend
           
\pstart
           Ich muss Ihnen vorhersagen, dass es mit einiger Mühe für Sie verbunden sein wird.
               Wenn das Pseudonymat undurchdringlich bleiben soll, so muss ein ganzer Roman erfunden
               u. durchgeführt werden. Ich will einen sehr gewöhnlichen Namen als Pseudonym wählen,
               zum Beispiel Albert Schnabel. Dieser Albert Schnabel hat bisher in Wien\oindex{Wien@\textbf{Wien}, \emph{Verwaltungsgebiet}|pw} gelebt und \strikeout{zwar} reist jetzt
               nach Italien\oindex{Italien@\textbf{Italien}|pw}, um Kunststudien zu treiben. Er
               reicht das Stück\pwindex{Herzl, Theodor 2.\,5.\,1860 Budapest – 3.\,7.\,1904 Edlach@\textsc{Herzl, Theodor} (2.\,5.\,1860 Budapest – 3.\,7.\,1904 Edlach), \emph{Schriftsteller, Journalist}!neue Ghetto. Schauspiel in vier Acten@\strich\emph{Das neue Ghetto. Schauspiel in vier Acten}|pwv} dem Deutschen Theater\orgindex{Deutsches Theater Berlin@Deutsches Theater Berlin|pw} in Berlin\oindex{Berlin@\textbf{Berlin}, \emph{Hauptstadt}|pw} ein – Postpacket in Wien\oindex{Wien@\textbf{Wien}, \emph{Verwaltungsgebiet}|pw} aufgegeben – mit folgendem Begleitbrief: Die Direction wird ersucht,
               sich binnen vier Wochen über die Annahme zu entscheiden. Nimmt sie das Stück\pwindex{Herzl, Theodor 2.\,5.\,1860 Budapest – 3.\,7.\,1904 Edlach@\textsc{Herzl, Theodor} (2.\,5.\,1860 Budapest – 3.\,7.\,1904 Edlach), \emph{Schriftsteller, Journalist}!neue Ghetto. Schauspiel in vier Acten@\strich\emph{Das neue Ghetto. Schauspiel in vier Acten}|pwv} an, so hat die Aufführung
               innerhalb zweier Monate zu erfolgen. Der Vertrag\strikeout{sentwurf} ist dem bevollmächtigten Wiener\oindex{Wien@\textbf{Wien}, \emph{Verwaltungsgebiet}|pw}
               Notar oder Advocaten X. Y. zuzusenden. Lehnt sie es ab, so wird sie gebeten es mit
               diesem Begleitbrief an das Lessingtheater\orgindex{Lessing-Theater@Lessing-Theater|pw}{ }\strikeout{unter den gleichen B} abzugeben. Dieses hat dieselben
               Bedingungen. Lehnt es auch ab, so geht das Stück\pwindex{Herzl, Theodor 2.\,5.\,1860 Budapest – 3.\,7.\,1904 Edlach@\textsc{Herzl, Theodor} (2.\,5.\,1860 Budapest – 3.\,7.\,1904 Edlach), \emph{Schriftsteller, Journalist}!neue Ghetto. Schauspiel in vier Acten@\strich\emph{Das neue Ghetto. Schauspiel in vier Acten}|pwv} ans Berliner\orgindex{Berliner Theater@Berliner Theater|pw} dann
               aus Neue Theater\orgindex{Neues Theater@Neues Theater|pw}. Will keins der {\pb}regulären Theater es spielen, so ists
               der Freien Bühne\orgindex{Freie Bühne@Freie Bühne|pw} zu geben. Lehnt auch diese ab,
               so ist das Stück\pwindex{Herzl, Theodor 2.\,5.\,1860 Budapest – 3.\,7.\,1904 Edlach@\textsc{Herzl, Theodor} (2.\,5.\,1860 Budapest – 3.\,7.\,1904 Edlach), \emph{Schriftsteller, Journalist}!neue Ghetto. Schauspiel in vier Acten@\strich\emph{Das neue Ghetto. Schauspiel in vier Acten}|pwv} an den
               Advocaten oder Notar zurückzusenden. Dann wird es im Druck erscheinen.\pend
           
\pstart
           In Wien\oindex{Wien@\textbf{Wien}, \emph{Verwaltungsgebiet}|pw} wird es nicht eingereicht.\pend
           
\pstart
           Was sagen Sie dazu?\pend
           
\pstart
           Wenn Sie sich bereit erklären, mich zu unterstützen, bitte ich Sie auch mir einen
               Notar oder Advocaten zu nennen, zu dem Sie volles Vertrauen haben. Mit diesem werden
               Sie allein verkehren. Er wird nur Sie kennen und die Verrechnungen, die an ihn kommen
               an Sie abführen.\pend
           
\pstart
           Auf diese Art erfährt Niemand, wer der Verfasser ist.\pend
           
\pstart
           Das liebe ich sehr. Es ist auch nicht unpraktisch. Denn da ich den Einfluss, den ich
               aus der Zeitung\orgindex{Neue Freie Presse@Neue Freie Presse|pwv} ziehen könnte,
               nie für mich verwende \introOben{}(Beispiel die Glosse\pwindex{Herzl, Theodor 2.\,5.\,1860 Budapest – 3.\,7.\,1904 Edlach@\textsc{Herzl, Theodor} (2.\,5.\,1860 Budapest – 3.\,7.\,1904 Edlach), \emph{Schriftsteller, Journalist}!Glosse. Lustspiel in einem Act@\strich\emph{Die Glosse. Lustspiel in einem Act}|pw})\introOben{}, ist es gleichgiltig ob mein Name auf dem
                  Stück\pwindex{Herzl, Theodor 2.\,5.\,1860 Budapest – 3.\,7.\,1904 Edlach@\textsc{Herzl, Theodor} (2.\,5.\,1860 Budapest – 3.\,7.\,1904 Edlach), \emph{Schriftsteller, Journalist}!neue Ghetto. Schauspiel in vier Acten@\strich\emph{Das neue Ghetto. Schauspiel in vier Acten}|pwv} steht. Ja, wenn ein
               von mir gezeichnetes Stück irgendwo aufgeführt wird, glauben dennoch Viele, dass ich
               mir es »gerichtet« habe. Andererseits habe ich nicht mit Vorurtheilen, die sich {\pb}an meine frühere Production heften, zu
               kämpfen.\pend
           
\pstart
           So ist dieser Entschluss nach vielen Richtungen hin überlegt. Es war mein Vergnügen
               während dieser glücklichen siebenzehn Tage, mir neben, unter u. über dem Stück\pwindex{Herzl, Theodor 2.\,5.\,1860 Budapest – 3.\,7.\,1904 Edlach@\textsc{Herzl, Theodor} (2.\,5.\,1860 Budapest – 3.\,7.\,1904 Edlach), \emph{Schriftsteller, Journalist}!neue Ghetto. Schauspiel in vier Acten@\strich\emph{Das neue Ghetto. Schauspiel in vier Acten}|pwv} diesen Verfasserroman
               auszuspinnen.\pend
           
\pstart
           Vielleicht habe ich mich wieder geirrt? Die Glosse\pwindex{Herzl, Theodor 2.\,5.\,1860 Budapest – 3.\,7.\,1904 Edlach@\textsc{Herzl, Theodor} (2.\,5.\,1860 Budapest – 3.\,7.\,1904 Edlach), \emph{Schriftsteller, Journalist}!Glosse. Lustspiel in einem Act@\strich\emph{Die Glosse. Lustspiel in einem Act}|pw} scheint ein solcher Irrthum gewesen zu sein, denn \label{K_L03835-1v}\edtext{alle Theater\orgindex{Burgtheater@Burgtheater|pwv}}{\lemma{\textnormal{\emph{alle Theater}}}\Cendnote{\textnormal{Aus Herzls\pwindex{Herzl, Theodor 2.\,5.\,1860 Budapest – 3.\,7.\,1904 Edlach@\textsc{Herzl, Theodor} (2.\,5.\,1860 Budapest – 3.\,7.\,1904 Edlach), \emph{Schriftsteller, Journalist}|pwk} bisher publizierten Korrespondenzen läßt sich nur die erfolglose
                  Einreichung am \emph{Burgtheater}\orgindex{Burgtheater@Burgtheater|pwk} und die im
                  Anschluss daran darauf angestrebte Drucklegung nachweisen, vgl. Theodor Herzl\pwindex{Herzl, Theodor 2.\,5.\,1860 Budapest – 3.\,7.\,1904 Edlach@\textsc{Herzl, Theodor} (2.\,5.\,1860 Budapest – 3.\,7.\,1904 Edlach), \emph{Schriftsteller, Journalist}|pwk}: \emph{Briefe
                        und Tagebücher}. Herausgegeben von Alex Bein, Hermann Greive, Moshe
                     Schaerf und Julius H. Schoeps, Bd. 1: \emph{Briefe und
                        autobiographische Notizen 1866–1895}, bearbeitet von Johannes
                     Wachten. Berlin, Frankfurt am Main,
                     Wien: \emph{Propyläen}{ }1983, S. 542–545.}}}\label{K_L03835-1} wo sie bisher eingereicht worden,
               haben sie abgelehnt.\pend
           
\pstart
           Fragezeichen.\pend
           
\pstart
           Es ist möglicherweise dumm, dass ich die Canaillen der Theaterdirection nicht auf die
               gemeine Art der Anderen zwinge, mich zu spielen. Wäre ich überzeugt, dass meine Werke
               es werth sind, so würde ich aus einer höheren Künstlermoral heraus auch zu Mitteln
               greifen, die mich anwidern. Aber diese Ueberzeugung habe ich nicht – der
               Productionsrausch ist was Anderes – und ohne solche Ueberzeugung wär’s blos gemein.\pend
           
\pstart
           Antworten Sie nur in recommandirtem Brief – und schweigen Sie mir über diesen, wenn
               Sie nicht mitthun wollen. Thun Sie aber mit, so habe ich ein Recht nicht nur auf Ihr
               Stillschweigen, sondern auch auf alle Ihre List und Vorsicht bis in die kleinsten
               Züge, damit das was \uline{nur} Sie und ich wissen ein volles
               Geheimnis bleibe.\pend
           
\pstart
           Herzlich Ihr{\\[\baselineskip]}\spacefill\mbox{Th. H.}\pend
           \leftskip=0em{}\selectlanguage{ngerman}\endnumbering\briefempfaengerindex{Schnitzler, Arthur@\textsc{Schnitzler, Arthur}!zzzHerzl, Theodor@\emph{von Theodor Herzl}!1894-11-083@{8. 11. 1894}|)be}\mylabel{L03835h}
\begin{anhang}
\end{anhang}\newcommand{\dateiname}{L03835}\newcommand{\titel}{Theodor Herzl an Arthur Schnitzler, 8. 11. 1894}\newcommand{\editorInnen}{Selma Jahnke und Martin Anton Müller}%% latex-leseansicht-abspann.tex
%% Abspann für die Leseansicht.
%% Der Schalter \ifkorrekturansicht ist bereits durch den Vorspann gesetzt.

%% latex-abspann.tex
%% Gemeinsamer Abspann für Korrekturansicht und Leseansicht.
%% Setzt den Schalter \ifkorrekturansicht voraus (gesetzt in den
%% einbindenden Dateien latex-korrekturansicht-abspann.tex bzw.
%% latex-leseansicht-abspann.tex).
%% ---------------------------------------------------------------

\normalsize

% Das esempio-Environment wird nur in der Leseansicht benötigt
\ifkorrekturansicht\else
\newenvironment{esempio}[3]%
{
    \vspace{1.5ex}
    \rlap{\underline{#1}}
    \par
    \setlength{\parindent}{0cm}
    \nopagebreak
    \leftskip=#2cm
    \rightskip=#3cm
}
{
    \par
}
\fi

\doendnotes{C}
\bigskip
\vfill

\clearpage

\footnotesize

\ifkorrekturansicht
  \lohead{\textsc{register}}
\fi

% theindex-Environment neu definieren ohne reledmac
\makeatletter
\renewenvironment{theindex}{%
  \ifkorrekturansicht
    \section*{\indexname}%
  \else
    \subsubsection*{Index der erwähnten Entitäten}%
  \fi
  \setlength{\parindent}{0pt}%
  \setlength{\parskip}{0pt plus 0.3pt}%
  \let\item\@idxitem
}{%
  \ifkorrekturansicht\clearpage\fi
}
\makeatother

\IfFileExists{\jobname-pw.ind}{\input{\jobname-pw.ind}}{}

% Quellenangabe nur in der Leseansicht
\ifkorrekturansicht\else
% Fallback-Definitionen, falls die .tex-Datei \titel etc. nicht gesetzt hat
\providecommand{\titel}{}
\providecommand{\editorInnen}{}
\providecommand{\dateiname}{\jobname}

\vspace{3cm}

\vfill

\footnotesize
\textsc{Quelle}: \titel. Herausgegeben von {\editorInnen}. In: \emph{Arthur Schnitzler: Briefwechsel mit Autorinnen und Autoren}.
 Digitale Edition, https://schnitzler-briefe.acdh.oeaw.ac.at/{\dateiname}.html (Stand \today)
\fi

\end{document}


