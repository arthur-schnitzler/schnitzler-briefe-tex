\input{../tex-inputs/latex-pdf-vorspann}
\begin{center}
            \textcolor{red}{ENTWURF. ENTZIFFERUNG NOCH NICHT KORREKTURGELESEN}
                      \end{center}
            
               \section[Richard Beer-Hofmann an Arthur Schnitzler, 15. 6. 1911]{ Richard Beer-Hofmann an Arthur Schnitzler, 15. 6. 1911}\nopagebreak\mylabel{v}\rehead{ }\begin{ledgroupsized}[t]{13cm}\normalsize\beginnumbering\briefempfaengerindex{Schnitzler, Arthur@\textsc{Schnitzler, Arthur}!zzzBeer-Hofmann, Richard@\emph{von Richard Beer-Hofmann}!1911-06-152@{15. 6. 1911}|(be} \toendnotes[C]{\smallbreak\pagebreak[2]} \Standort{CUL, Schnitzler, B 8.}
\physDesc{Kartenbrief
\newline{}Handschrift: Bleistift, lateinische Kurrent\newline{}Versand: ohne postalischen Übermittlungsvermerk \newline{}Ordnung: mit Bleistift von unbekannter Hand nummeriert:
                              »241« }\buchAbdrucke{\weitereDrucke{Arthur Schnitzler, Richard Beer-Hofmann: \emph{Briefwechsel 1891–1931}. Hg. Konstanze Fliedl. Wien, Zürich: \emph{Europaverlag} 1992, S. 214–215.} }\pstart{}{\pb}Herrn\pend{}\pstart{}D\textsuperscript{r} Arthur Schnitzler\pend{}{\bigskip}\pstart
           \raggedleft{}{\pb}15/VI 11\pend
           \pstart
           Lieber Arthur! Ich ko{\geminationm}e gerne – (ich
               wollte morgen ko{\geminationm}en – mich verabschieden – ich reise –
               (will reisen) Montag) und Paula\pwindex{Beer-Hofmann, Paula 25.02.1879 – 30.10.1939@\textsc{Beer-Hofmann, Paula} (25.02.1879 – 30.10.1939)|pw} ko{\geminationm}t auch wenn sie nicht zu müde ist (Sie raunzt seit drei
               Tagen ununterbrochen).\pend
           \pstart
           Herzlichst{\\[\baselineskip]}\spacefill\mbox{Richard}\pend
           \leftskip=0em{}\endnumbering\briefempfaengerindex{Schnitzler, Arthur@\textsc{Schnitzler, Arthur}!zzzBeer-Hofmann, Richard@\emph{von Richard Beer-Hofmann}!1911-06-152@{15. 6. 1911}|)be}\mylabel{h}\end{ledgroupsized}  \newcommand{\dateiname}{L02023}\newcommand{\titel}{Richard Beer-Hofmann an Arthur Schnitzler, 15. 6. 1911}\newcommand{\editorInnen}{Martin Anton Müller und Gerd-Hermann Susen}\input{../tex-inputs/latex-pdf-abspann}
      