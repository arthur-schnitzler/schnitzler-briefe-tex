%% latex-korrekturansicht-vorspann.tex
%% Vorspann für die Korrekturansicht.
%% Lädt die gemeinsame Datei latex-vorspann.tex mit gesetztem Schalter.

\newif\ifkorrekturansicht
\korrekturansichttrue

\input{../tex-inputs/latex-vorspann}


\section[Arthur Schnitzler an Hermann Bahr, 8. 6. 1903]{L01296 Arthur Schnitzler an Hermann Bahr, 8. 6. 1903}
\nopagebreak\mylabel{L01296v}
\rehead{ }\normalsize\beginnumbering\briefempfaengerindex{Bahr, Hermann@\textsc{Bahr, Hermann}!zzzSchnitzler, Arthur@\emph{von Arthur Schnitzler}!1903-06-081@{8. 6. 1903}|(be}
\toendnotes[C]{\smallbreak\pagebreak[2]}\Standort{TMW, HS AM 60014 Ba.}
\physDesc{Bildpostkarte, 98 Zeichen
\newline{}Handschrift: 1) Bleistift, deutsche Kurrent\hspace{1em}2) Bleistift, lateinische Kurrent (\noindent{}Adresse)\hspace{1em}
\newline{}Versand: Stempel: »\nobreak{}\oindex{Bahnhof@\textbf{Bahnhof}, \emph{Bahnhofsgebäude (K.BHF)}|pwk}Lugano Stazione, 8. VI. 03\nobreak{}«.  
\newline{}Ordnung: Lochung }
\buchAbdrucke{\weitereDrucke{Hermann Bahr, Arthur Schnitzler: \emph{Briefwechsel, Aufzeichnungen, Dokumente (1891–1931)}. Göttingen: \emph{Wallstein} 2018, S. 266.} }\pstart{}{\pb}Herrn Hermann
                  Bahr\pend{}\pstart{}Wien-Ob-St Veit\oindex{Ober Sankt Veit@\textbf{Ober Sankt Veit}, \emph{P.PPLX}|pw}\pend{}\pstart{}Veitlissengasse 3\oindex{Veitlissengasse@\textbf{Veitlissengasse}, \emph{Straße (K.STR)}|pw}\pend{}\pstart{}Austria\oindex{Oesterreich@\textbf{Österreich}, \emph{A.PCLI}|pw}\pend{}{\bigskip}
\pstart
           \noindent{}\centering{}{\pb}\textcolor{gray}{\textbf{Lugano. Monte Bré\oindex{Monte Bre@\textbf{Monte Bré}, \emph{T.MT}|pw}}}\pend
           \vspace{1em}
\pstart
           {\pb}8. 6. 903\pend
           \vspace{0.5em}
\pstart
           Herzlichſte Grüße! Dein \spacefill\mbox{Arth Schn}\pend
           \selectlanguage{ngerman}\endnumbering\briefempfaengerindex{Bahr, Hermann@\textsc{Bahr, Hermann}!zzzSchnitzler, Arthur@\emph{von Arthur Schnitzler}!1903-06-081@{8. 6. 1903}|)be}\mylabel{L01296h}  \normalsize

\doendnotes{C}
\bigskip
\vfill

\clearpage

\footnotesize

\lohead{\textsc{register}}

% Definiere theindex-Environment komplett neu ohne reledmac
\makeatletter
\renewenvironment{theindex}{%
  \section*{\indexname}%
  \setlength{\parindent}{0pt}%
  \setlength{\parskip}{0pt plus 0.3pt}%
  \let\item\@idxitem
}{%
  \clearpage
}
\makeatother

\IfFileExists{\jobname-pw.ind}{\input{\jobname-pw.ind}}{}

\end{document}

      