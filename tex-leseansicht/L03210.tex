%% latex-korrekturansicht-vorspann.tex
%% Vorspann für die Korrekturansicht.
%% Lädt die gemeinsame Datei latex-vorspann.tex mit gesetztem Schalter.

\newif\ifkorrekturansicht
\korrekturansichttrue

\input{../tex-inputs/latex-vorspann}


\section[ Paul Goldmann an Arthur Schnitzler, 9. 6. {[}1902{]}]{L03210 Paul Goldmann an Arthur Schnitzler, 9. 6. {[}1902{]}}
\nopagebreak\mylabel{L03210v}
\rehead{ }\normalsize\beginnumbering\briefempfaengerindex{Schnitzler, Arthur@\textsc{Schnitzler, Arthur}!zzzGoldmann, Paul@\emph{von Paul Goldmann}!1902-06-091@{9. 6. {[}1902{]}}|(be}
\toendnotes[C]{\smallbreak\pagebreak[2]}\Standort{DLA, A:Schnitzler, HS.NZ85.1.3172.}
\physDesc{Brief, 1 Blatt, 3 Seiten, 769 Zeichen
\newline{}Handschrift: blaue Tinte, deutsche Kurrent
\newline{}Schnitzler: mit Bleistift das Jahr »902« vermerkt }\toendnotes[C]{\smallbreak}
\pstart
           \raggedleft{}{\pb}\textcolor{gray}{\textbf{DESSAUERSTRASSE 19}}\oindex{Dessauer Strasse@\textbf{Dessauer Straße}, \emph{Straße (K.STR)}|pw}\pend
           
\pstart
           Berlin\oindex{Berlin@\textbf{Berlin}, \emph{P.PPLC}|pw}, 9. Juni.\pend
           
\pstart\center{}Mein lieber Freund,\pend\vspace{0.5em}
\pstart
           Seit ich \label{K_L03210-1v}\edtext{aus Wien\oindex{Wien@\textbf{Wien}, \emph{A.ADM2}|pw} zurück}{\lemma{\textnormal{\emph{aus Wien zurück}}}\Cendnote{\textnormal{Goldmanns\pwindex{Goldmann, Paul 31.01.1865 – 25.09.1935@\textsc{Goldmann, Paul} (31.01.1865 – 25.09.1935), \emph{Schriftsteller/Schriftstellerin, Journalist/Journalistin}|pwk} letzter nachgewiesener Tag in Wien\oindex{Wien@\textbf{Wien}, \emph{A.ADM2}|pwk} war der 25. 5. 1902.}}}\label{K_L03210-1} bin, will ich Dir ſchreiben.
               Es hätte ein großer Brief werden ſollen, aber aus Mangel an Zeit iſt es nicht einmal
               ein kleiner geworden, und da mir Deine lieben Nachrichten mangeln, ſo ſchreibe ich
               Dir heut nur, um Dich zu fragen, wie es Dir geht, was
                  \textsc{Olga\pwindex{Schnitzler, Olga 17.01.1882 – 13.01.1970@\textsc{Schnitzler, Olga} (17.01.1882 – 13.01.1970), \emph{Schauspieler/Schauspielerin, Sänger/Sängerin}|pw}} macht, was {\pb}es ſonſt Neues gibt, wie es mit
               Deinen Sommerplänen ſteht, \textsc{etc}\substVorne{}\textsuperscript{?}\substDazwischen{}.\substHinten{}\pend
           
\pstart
           Von mir kann ich nichts mittheilen, als daß ich viel und ſchwer zu arbeiten habe, und
               daß ich mich danach ſehne, ein paar Monate in Ruhe, mehr körperlich als geiſtig
               beſchäftigt, zu leben, was natürlich unmöglich ſein wird. Inzwiſchen habe ich {\pb}nach wie vor die Abſicht, \label{K_L03210-2v}\edtext{zwiſchen 20. und 25. Juli nach Wien\oindex{Wien@\textbf{Wien}, \emph{A.ADM2}|pw}}{\lemma{\textnormal{\emph{zwiſchen … Wien}}}\Cendnote{\textnormal{Dazu kam es nicht.}}}\label{K_L03210-2} zu kommen, wo ich
               Dich zu ſehen hoffe.\pend
           
\pstart
           Viele treue Grüße! {\\[\baselineskip]}Dein {\\[\baselineskip]}\spacefill\mbox{Paul Goldmnn}\pend
           \leftskip=0em{}
\pstart
           \noindent{}Viele Grüße an \textsc{Olga\pwindex{Schnitzler, Olga 17.01.1882 – 13.01.1970@\textsc{Schnitzler, Olga} (17.01.1882 – 13.01.1970), \emph{Schauspieler/Schauspielerin, Sänger/Sängerin}|pw}}!\pend
           \selectlanguage{ngerman}\endnumbering\briefempfaengerindex{Schnitzler, Arthur@\textsc{Schnitzler, Arthur}!zzzGoldmann, Paul@\emph{von Paul Goldmann}!1902-06-091@{9. 6. {[}1902{]}}|)be}\mylabel{L03210h}  \normalsize

\doendnotes{C}
\bigskip
\vfill

\clearpage

\footnotesize

\lohead{\textsc{register}}

% Definiere theindex-Environment komplett neu ohne reledmac
\makeatletter
\renewenvironment{theindex}{%
  \section*{\indexname}%
  \setlength{\parindent}{0pt}%
  \setlength{\parskip}{0pt plus 0.3pt}%
  \let\item\@idxitem
}{%
  \clearpage
}
\makeatother

\IfFileExists{\jobname-pw.ind}{\input{\jobname-pw.ind}}{}

\end{document}

      