%% latex-leseansicht-vorspann.tex
%% Vorspann für die Leseansicht.
%% Lädt die gemeinsame Datei latex-vorspann.tex mit nicht gesetztem Schalter.

\newif\ifkorrekturansicht
\korrekturansichtfalse

\input{../tex-inputs/latex-vorspann}


\section[Hermann Bahr an Arthur Schnitzler, 11. 12. 1909]{L01897 Hermann Bahr an Arthur Schnitzler, 11. 12. 1909}
\nopagebreak\mylabel{L01897v}
\rehead{ }\normalsize\beginnumbering\briefempfaengerindex{Schnitzler, Arthur@\textsc{Schnitzler, Arthur}!zzzBahr, Hermann@\emph{von Hermann Bahr}!1909-12-111@{11. 12. 1909}|(be}
\toendnotes[C]{\smallbreak\pagebreak[2]}
\correspDesc{Versand  durch Hermann Bahr am 11. 12. 1909 in Wien
\newline{}Erhalt  durch Arthur Schnitzler im Zeitraum [11. 12. 1909 – 15. 12. 1909?] in Wien}\toendnotes[C]{\smallbreak}
\Standort{CUL, Schnitzler, B 5b.}
\physDesc{Brief, 1 Blatt, 2 Seiten, 932 Zeichen
\newline{}Handschrift Lisa Clarus: blaue Tinte, lateinische Kurrent
\newline{}Handschrift Hermann Bahr: blaue Tinte (\noindent{}Unterschrift)
\newline{}Schnitzler: mit Bleistift ergänzt »Bahr« 
\newline{}Ordnung: mit Bleistift von unbekannter Hand nummeriert:
                                    »163« }
\buchAbdrucke{\weitereDrucke{Hermann Bahr, Arthur Schnitzler: \emph{Briefwechsel, Aufzeichnungen, Dokumente (1891–1931)}. Herausgegeben von Kurt Ifkovits und Martin Anton Müller. Göttingen: \emph{Wallstein} 2018, S. 428.} }\toendnotes[C]{\smallbreak}
\pstart
           \raggedleft{}{\pb}11. 12. 09\pend
           
\pstart
           \centering{}Wien XIII/\textsubscript{7}\oindex{Wien@\textbf{Wien}!XIII., Hietzing@\textbf{XIII., Hietzing}!Ober Sankt Veit@\textbf{Ober Sankt Veit}, \emph{Ehemaliger Ort}|pw}\pend
           
\pstart\center{}Lieber Arthur!\pend\vspace{0.5em}
\pstart
           In Halle \textsuperscript{a}/Saale\oindex{Halle (Saale)@\textbf{Halle (Saale)}|pw}, wo
               ich auch wieder einmal die Toten schweigen\pwindex{Schnitzler, Arthur 15.\,5.\,1862 Wien – 21.\,10.\,1931 ebd.@\textsc{Schnitzler, Arthur} (15.\,5.\,1862 Wien – 21.\,10.\,1931 ebd.), \emph{Schriftsteller, Mediziner}!Toten schweigen@\strich\emph{Die Toten schweigen}|pw}
               liess, hat man mich angefleht Dir doch zuzureden, dass Du selbst einmal hinkommen
               sollst. Ein Oberingenieur Bacher\pwindex{Bacher, Oskar @\textsc{Bacher, Oskar}, \emph{Oberingenieur}|pw}, der schon
               einmal mit Dir correspondiert haben will, beschwört Dich, wenn Du zum \label{T_L01897-1v}\edtext{Anathol\pwindex{Schnitzler, Arthur 15.\,5.\,1862 Wien – 21.\,10.\,1931 ebd.@\textsc{Schnitzler, Arthur} (15.\,5.\,1862 Wien – 21.\,10.\,1931 ebd.), \emph{Schriftsteller, Mediziner}!Anatol@\strich\emph{Anatol}|pw}}{\lemma{\textnormal{\emph{Anathol}}}\Cendnote{\textnormal{Das »h« vermutlich von
                  Schnitzler mit rotem Buntstift gestrichen.}}}\label{T_L01897-1} nach Berlin\oindex{Berlin@\textbf{Berlin}, \emph{Hauptstadt}|pw} fährst, doch den Weg über Halle\oindex{Halle (Saale)@\textbf{Halle (Saale)}|pw} zu nehmen. Ich bitte Dich, schreib ihm (Halle, Waidenplan 13\oindex{Weidenplan@\textbf{Weidenplan}, \emph{Straße}|pw}) ein Wort, und zwar baldigst. Denn der gute Mann {\pb}hat mir ein unfehlbares Mittel gegen die Gicht
               versprochen, das ich dringend brauche und er mir sicher nicht schickt, so lang ich
               mich nicht besonders um ihn verdient gemacht habe. Und: hast Du vielleicht eine neue
               kurze, womöglich lustige Novelle? Ich soll hier \label{K_L01897-1v}\edtext{für die freie Schule\orgindex{Verein »Freie Schule«@Verein »Freie Schule«|pw}}{\lemma{\textnormal{\emph{für die freie Schule}}}\Cendnote{\textnormal{Am 9. 1. 1910. Bahr\pwindex{Bahr, Hermann 19.\,7.\,1863 Linz – 15.\,1.\,1934 München@\textsc{Bahr, Hermann} (19.\,7.\,1863 Linz – 15.\,1.\,1934 München), \emph{Schriftsteller, Kritiker}|pwk} las nichts von Schnitzler.}}}\label{K_L01897-1} vorlesen und möchte was von Dir. Entschuldige, dass
               ich diktiere: ich bin totmüd, in grosser Hast und eben auf den Semmering\oindex{Semmering@\textbf{Semmering}, \emph{Verwaltungsgebiet}|pw} abreisend.\pend
           
\pstart
           Herzlichst mit den schönsten Grüssen an Frau\pwindex{Schnitzler, Olga 17.\,1.\,1882 Wien – 13.\,1.\,1970 Lugano@\textsc{Schnitzler, Olga} (17.\,1.\,1882 Wien – 13.\,1.\,1970 Lugano), \emph{Schauspielerin, Sängerin}|pwv} und Kinder\pwindex{Schnitzler, Heinrich 9.\,8.\,1902 Hinterbrühl – 12.\,7.\,1982 Wien@\textsc{Schnitzler, Heinrich} (9.\,8.\,1902 Hinterbrühl – 12.\,7.\,1982 Wien), \emph{Regisseur, Schauspieler}|pwv}\pwindex{Cappellini, Lili 13.\,9.\,1909 Wien – 26.\,7.\,1928 Venedig@\textsc{Cappellini, Lili} (13.\,9.\,1909 Wien – 26.\,7.\,1928 Venedig)|pwv}{\\[\baselineskip]}Dein alter{\\[\baselineskip]}\spacefill\mbox{{[}hs. Bahr:{]} HermannBahr}\pend
           \leftskip=0em{}\selectlanguage{ngerman}\endnumbering\briefempfaengerindex{Schnitzler, Arthur@\textsc{Schnitzler, Arthur}!zzzBahr, Hermann@\emph{von Hermann Bahr}!1909-12-111@{11. 12. 1909}|)be}\mylabel{L01897h}  \newcommand{\dateiname}{L01897}\newcommand{\titel}{Hermann Bahr an Arthur Schnitzler, 11. 12. 1909}\newcommand{\editorInnen}{Herausgegeben von Martin Anton Müller}%% latex-leseansicht-abspann.tex
%% Abspann für die Leseansicht.
%% Der Schalter \ifkorrekturansicht ist bereits durch den Vorspann gesetzt.

%% latex-abspann.tex
%% Gemeinsamer Abspann für Korrekturansicht und Leseansicht.
%% Setzt den Schalter \ifkorrekturansicht voraus (gesetzt in den
%% einbindenden Dateien latex-korrekturansicht-abspann.tex bzw.
%% latex-leseansicht-abspann.tex).
%% ---------------------------------------------------------------

\normalsize

% Das esempio-Environment wird nur in der Leseansicht benötigt
\ifkorrekturansicht\else
\newenvironment{esempio}[3]%
{
    \vspace{1.5ex}
    \rlap{\underline{#1}}
    \par
    \setlength{\parindent}{0cm}
    \nopagebreak
    \leftskip=#2cm
    \rightskip=#3cm
}
{
    \par
}
\fi

\doendnotes{C}
\bigskip
\vfill

\clearpage

\footnotesize

\ifkorrekturansicht
  \lohead{\textsc{register}}
\fi

% theindex-Environment neu definieren ohne reledmac
\makeatletter
\renewenvironment{theindex}{%
  \ifkorrekturansicht
    \section*{\indexname}%
  \else
    \subsubsection*{Index der erwähnten Entitäten}%
  \fi
  \setlength{\parindent}{0pt}%
  \setlength{\parskip}{0pt plus 0.3pt}%
  \let\item\@idxitem
}{%
  \ifkorrekturansicht\clearpage\fi
}
\makeatother

\IfFileExists{\jobname-pw.ind}{\input{\jobname-pw.ind}}{}

% Quellenangabe nur in der Leseansicht
\ifkorrekturansicht\else
% Fallback-Definitionen, falls die .tex-Datei \titel etc. nicht gesetzt hat
\providecommand{\titel}{}
\providecommand{\editorInnen}{}
\providecommand{\dateiname}{\jobname}

\vspace{3cm}

\vfill

\footnotesize
\textsc{Quelle}: \titel. Herausgegeben von {\editorInnen}. In: \emph{Arthur Schnitzler: Briefwechsel mit Autorinnen und Autoren}.
 Digitale Edition, https://schnitzler-briefe.acdh.oeaw.ac.at/{\dateiname}.html (Stand \today)
\fi

\end{document}


