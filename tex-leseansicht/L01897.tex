%% latex-korrekturansicht-vorspann.tex
%% Vorspann für die Korrekturansicht.
%% Lädt die gemeinsame Datei latex-vorspann.tex mit gesetztem Schalter.

\newif\ifkorrekturansicht
\korrekturansichttrue

\input{../tex-inputs/latex-vorspann}


\section[Hermann Bahr an Arthur Schnitzler, 11. 12. 1909]{L01897 Hermann Bahr an Arthur Schnitzler, 11. 12. 1909}
\nopagebreak\mylabel{L01897v}
\rehead{ }\normalsize\beginnumbering\briefempfaengerindex{Schnitzler, Arthur@\textsc{Schnitzler, Arthur}!zzzBahr, Hermann@\emph{von Hermann Bahr}!1909-12-111@{11. 12. 1909}|(be}
\toendnotes[C]{\smallbreak\pagebreak[2]}\Standort{CUL, Schnitzler, B 5b.}
\physDesc{Brief, 1 Blatt, 2 Seiten, 932 Zeichen
\newline{}Handschrift Lisa Clarus: blaue Tinte, lateinische Kurrent
\newline{}Handschrift Hermann Bahr: blaue Tinte (\noindent{}Unterschrift)
\newline{}Schnitzler: mit Bleistift ergänzt »Bahr« 
\newline{}Ordnung: mit Bleistift von unbekannter Hand nummeriert:
                                    »163« }
\buchAbdrucke{\weitereDrucke{Hermann Bahr, Arthur Schnitzler: \emph{Briefwechsel, Aufzeichnungen, Dokumente (1891–1931)}. Göttingen: \emph{Wallstein} 2018, S. 428.} }\toendnotes[C]{\smallbreak}
\pstart
           \raggedleft{}{\pb}11. 12. 09\pend
           
\pstart
           \centering{}Wien XIII/\textsubscript{7}\oindex{Ober Sankt Veit@\textbf{Ober Sankt Veit}, \emph{P.PPLX}|pw}\pend
           
\pstart\center{}Lieber Arthur!\pend\vspace{0.5em}
\pstart
           In Halle \textsuperscript{a}/Saale\oindex{Halle (Saale)@\textbf{Halle (Saale)}, \emph{P.PPL}|pw}, wo
               ich auch wieder einmal die Toten schweigen\pwindex{Toten schweigen@\emph{Die Toten schweigen}|pw}
               liess, hat man mich angefleht Dir doch zuzureden, dass Du selbst einmal hinkommen
               sollst. Ein Oberingenieur Bacher\pwindex{Bacher, Oskar @\textsc{Bacher, Oskar}, \emph{Oberingenieur/Oberingenieurin}|pw}, der schon
               einmal mit Dir correspondiert haben will, beschwört Dich, wenn Du zum \label{T_L01897-1v}\edtext{Anathol\pwindex{Anatol@\emph{Anatol}|pw}}{\lemma{\textnormal{\emph{Anathol}}}\Cendnote{\textnormal{Das »h« vermutlich von
                  Schnitzler mit rotem Buntstift gestrichen.}}}\label{T_L01897-1} nach Berlin\oindex{Berlin@\textbf{Berlin}, \emph{P.PPLC}|pw} fährst, doch den Weg über Halle\oindex{Halle (Saale)@\textbf{Halle (Saale)}, \emph{P.PPL}|pw} zu nehmen. Ich bitte Dich, schreib ihm (Halle, Waidenplan 13\oindex{Weidenplan@\textbf{Weidenplan}, \emph{Straße (K.STR)}|pw}) ein Wort, und zwar baldigst. Denn der gute Mann {\pb}hat mir ein unfehlbares Mittel gegen die Gicht
               versprochen, das ich dringend brauche und er mir sicher nicht schickt, so lang ich
               mich nicht besonders um ihn verdient gemacht habe. Und: hast Du vielleicht eine neue
               kurze, womöglich lustige Novelle? Ich soll hier \label{K_L01897-1v}\edtext{für die freie Schule\orgindex{Verein »Freie Schule«@Verein »Freie Schule«|pw}}{\lemma{\textnormal{\emph{für die freie Schule}}}\Cendnote{\textnormal{Am 9. 1. 1910. Bahr\pwindex{Bahr, Hermann 19.07.1863 – 15.01.1934@\textsc{Bahr, Hermann} (19.07.1863 – 15.01.1934), \emph{Schriftsteller/Schriftstellerin, Kritiker/Kritikerin}|pwk} las nichts von Schnitzler.}}}\label{K_L01897-1} vorlesen und möchte was von Dir. Entschuldige, dass
               ich diktiere: ich bin totmüd, in grosser Hast und eben auf den Semmering\oindex{Semmering@\textbf{Semmering}, \emph{A.ADM3}|pw} abreisend.\pend
           
\pstart
           Herzlichst mit den schönsten Grüssen an Frau\pwindex{Schnitzler, Olga 17.01.1882 – 13.01.1970@\textsc{Schnitzler, Olga} (17.01.1882 – 13.01.1970), \emph{Schauspieler/Schauspielerin, Sänger/Sängerin}|pwv} und Kinder\pwindex{Schnitzler, Heinrich 09.08.1902 – 12.07.1982@\textsc{Schnitzler, Heinrich} (09.08.1902 – 12.07.1982), \emph{Regisseur/Regisseurin, Schauspieler/Schauspielerin}|pwv}\pwindex{Cappellini, Lili 13.09.1909 – 26.07.1928@\textsc{Cappellini, Lili} (13.09.1909 – 26.07.1928)|pwv}{\\[\baselineskip]}Dein alter{\\[\baselineskip]}\spacefill\mbox{{[}hs. :{]} HermannBahr}\pend
           \leftskip=0em{}\selectlanguage{ngerman}\endnumbering\briefempfaengerindex{Schnitzler, Arthur@\textsc{Schnitzler, Arthur}!zzzBahr, Hermann@\emph{von Hermann Bahr}!1909-12-111@{11. 12. 1909}|)be}\mylabel{L01897h}  \normalsize

\doendnotes{C}
\bigskip
\vfill

\clearpage

\footnotesize

\lohead{\textsc{register}}

% Definiere theindex-Environment komplett neu ohne reledmac
\makeatletter
\renewenvironment{theindex}{%
  \section*{\indexname}%
  \setlength{\parindent}{0pt}%
  \setlength{\parskip}{0pt plus 0.3pt}%
  \let\item\@idxitem
}{%
  \clearpage
}
\makeatother

\IfFileExists{\jobname-pw.ind}{\input{\jobname-pw.ind}}{}

\end{document}

      