%% latex-leseansicht-vorspann.tex
%% Vorspann für die Leseansicht.
%% Lädt die gemeinsame Datei latex-vorspann.tex mit nicht gesetztem Schalter.

\newif\ifkorrekturansicht
\korrekturansichtfalse

\input{../tex-inputs/latex-vorspann}


\section[Arthur Schnitzler an Stefan Zweig, 12. 10. 1927]{L03745 Arthur Schnitzler an Stefan Zweig, 12. 10. 1927}
\nopagebreak\mylabel{L03745v}
\rehead{ }\normalsize\beginnumbering\briefempfaengerindex{Zweig, Stefan@\textsc{Zweig, Stefan}!zzzSchnitzler, Arthur@\emph{von Arthur Schnitzler}!1927-10-121@{12. 10. 1927}|(be}
\toendnotes[C]{\smallbreak\pagebreak[2]}
\correspDesc{Versand  durch Arthur Schnitzler am 12. 10. 1927 in Wien
\newline{}Erhalt  durch Stefan Zweig im Zeitraum [13. 10. 1927 – 17. 10. 1927?] in Salzburg}\toendnotes[C]{\smallbreak}
\Standort{Jerusalem, National Library of Israel, ARC. Ms. Var. 305 1 58 Stefan Zweig Collection.}
\physDesc{Brief, 1 Blatt, 2 Seiten, 1082 Zeichen
\newline{}Handschrift: schwarze Tinte, lateinische Kurrent}\toendnotes[C]{\smallbreak}
\pstart
           \raggedleft{}{\pb}Wien\oindex{Wien@\textbf{Wien}, \emph{Verwaltungsgebiet}|pw}, 12. Oct. 927\pend
           
\pstart{}lieber Doctor Zweig,\pend\vspace{0.5em}
\pstart
           es besteht eine Möglichkeit für mich, meine nächsten Sachen, vor Erscheinen in Deutschland\oindex{Deutschland@\textbf{Deutschland}|pw} an eine russ.\oindex{Russland@\textbf{Russland}|pw}{ }Verlagsanstalt zu verkaufen. Wie ich höre, haben
               Sie Ihr letztes \label{K_L03745-1v}\edtext{Novellenbuch\pwindex{Zweig, Stefan 28.\,11.\,1881 Wien – 23.\,2.\,1942 Petrópolis@\textsc{Zweig, Stefan} (28.\,11.\,1881 Wien – 23.\,2.\,1942 Petrópolis), \emph{Schriftsteller}!Smjatenie Chusto@\strich\emph{Smjatenie Chusto}|pwv}}{\lemma{\textnormal{\emph{Novellenbuch}}}\Cendnote{\textnormal{Der Verlag \emph{Wremla}\orgindex{Wremja@Wremja|pwk} hatte ohne Zweigs\pwindex{Zweig, Stefan 28.\,11.\,1881 Wien – 23.\,2.\,1942 Petrópolis@\textsc{Zweig, Stefan} (28.\,11.\,1881 Wien – 23.\,2.\,1942 Petrópolis), \emph{Schriftsteller}|pwk} Zustimmung 1925{ }\emph{Erstes Erlebnis}\pwindex{Zweig, Stefan 28.\,11.\,1881 Wien – 23.\,2.\,1942 Petrópolis@\textsc{Zweig, Stefan} (28.\,11.\,1881 Wien – 23.\,2.\,1942 Petrópolis), \emph{Schriftsteller}!Žgučaja tajna. Pervye pereživanija@\strich\emph{Žgučaja tajna. Pervye pereživanija}|pwk} und 1926{ }\emph{Amok}\pwindex{Zweig, Stefan 28.\,11.\,1881 Wien – 23.\,2.\,1942 Petrópolis@\textsc{Zweig, Stefan} (28.\,11.\,1881 Wien – 23.\,2.\,1942 Petrópolis), \emph{Schriftsteller}!Amok. Novelly@\strich\emph{Amok. Novelly}|pwk} auf russisch\oindex{Russland@\textbf{Russland}|pwk} publiziert. Nach der Kontaktaufnahme erschienen mit Zweigs\pwindex{Zweig, Stefan 28.\,11.\,1881 Wien – 23.\,2.\,1942 Petrópolis@\textsc{Zweig, Stefan} (28.\,11.\,1881 Wien – 23.\,2.\,1942 Petrópolis), \emph{Schriftsteller}|pwk} Zustimmung die zwei Novellen \emph{Verwirrung der Gefühle}\pwindex{Zweig, Stefan 28.\,11.\,1881 Wien – 23.\,2.\,1942 Petrópolis@\textsc{Zweig, Stefan} (28.\,11.\,1881 Wien – 23.\,2.\,1942 Petrópolis), \emph{Schriftsteller}!Verwirrung der Gefühle@\strich\emph{Verwirrung der Gefühle}|pwk} und \emph{Brief einer Unbekannten}\pwindex{Zweig, Stefan 28.\,11.\,1881 Wien – 23.\,2.\,1942 Petrópolis@\textsc{Zweig, Stefan} (28.\,11.\,1881 Wien – 23.\,2.\,1942 Petrópolis), \emph{Schriftsteller}!Brief einer Unbekannten@\strich\emph{Brief einer Unbekannten}|pwk} unter dem Titel \emph{Smjatenie Chusto}\pwindex{Zweig, Stefan 28.\,11.\,1881 Wien – 23.\,2.\,1942 Petrópolis@\textsc{Zweig, Stefan} (28.\,11.\,1881 Wien – 23.\,2.\,1942 Petrópolis), \emph{Schriftsteller}!Smjatenie Chusto@\strich\emph{Smjatenie Chusto}|pwk}.}}}\label{K_L03745-1} auch nach Russland\oindex{Russland@\textbf{Russland}|pw} verkauft, und es wäre mir sehr erwünscht zu wissen
                  (we{\geminationn} Sie über solche Dinge nicht principiell
               schweigen) welche Summe Ihnen bezahlt worden ist resp. unter welchen Bedingungen Sie
               abgeschlossen haben. Pauschalsummen\substVorne{}\textsuperscript{?}\substDazwischen{}\textcolor{gray}{,}\substHinten{} Perzente? Vorschuſs u. Perzente? U. s. w. Sie erweisen mir einen rechten
               Gefallen, we{\geminationn} Sie mich {\pb}aufklärten. Es handelt sich um einen Roman, der eben fertig geworden ist. »\label{K_L03745-2v}\edtext{Therese, Chronik eines Frauenlebens\pwindex{Schnitzler, Arthur 15.\,5.\,1862 Wien – 21.\,10.\,1931 ebd.@\textsc{Schnitzler, Arthur} (15.\,5.\,1862 Wien – 21.\,10.\,1931 ebd.), \emph{Schriftsteller, Mediziner}!Therese. Chronik eines Frauenlebens@\strich\emph{Therese. Chronik eines Frauenlebens}|pw}}{\lemma{\textnormal{\emph{Therese, … Frauenlebens}}}\Cendnote{\textnormal{Zu Lebzeiten Schnitzlers kam es zu keiner russischen\oindex{Russland@\textbf{Russland}|pwk} Übersetzung des Romans.}}}\label{K_L03745-2}.«\pend
           
\pstart
           Sie haben hoffentlich einen schönen So{\geminationm}er gehabt. Was
               mich anbelangt so war ich \label{K_L03745-3v}\edtext{in den Dolomiten\oindex{Dolomiten@\textbf{Dolomiten}, \emph{Gebirge}|pw}}{\lemma{\textnormal{\emph{in den Dolomiten}}}\Cendnote{\textnormal{Schnitzler war zwischen 11. 8. 1927 und 5. 9. 1927 an
                  verschiedenen Orten in Südtirol\oindex{Südtirol@\textbf{Südtirol}, \emph{Verwaltungsgebiet}|pwk} und Norditalien\oindex{Italien@\textbf{Italien}|pwk}. Am letztgenannten Tag langte er
                  in Venedig\oindex{Venedig@\textbf{Venedig}|pwk} an, wo er bis zum 15. 9. 1927 blieb.}}}\label{K_L03745-3}
               und zum Schluſs am Lido\oindex{Lido@\textbf{Lido}|pw}, resp in Venedig\oindex{Venedig@\textbf{Venedig}|pw} wo meine Tochter\pwindex{Cappellini, Lili 13.\,9.\,1909 Wien – 26.\,7.\,1928 Venedig@\textsc{Cappellini, Lili} (13.\,9.\,1909 Wien – 26.\,7.\,1928 Venedig)|pwv}, verheiratet mit dem Capitano Arnoldo Cappellini\pwindex{Cappellini, Arnoldo 10.\,8.\,1889 Venedig – 8.\,5.\,1954 Rom@\textsc{Cappellini, Arnoldo} (10.\,8.\,1889 Venedig – 8.\,5.\,1954 Rom)|pw}, in der Nähe der Frari Kirche\oindex{Santa Maria Gloriosa dei Frari@\textbf{Santa Maria Gloriosa dei Frari}, \emph{Kirche}|pw} lebt. Zurück bin ich \label{K_L03745-4v}\edtext{geflogen}{\lemma{\textnormal{\emph{geflogen}}}\Cendnote{\textnormal{Vgl. A. S.: \emph{Tagebuch}, 15. 9. 1927. }}}\label{K_L03745-4}. Das
               ist ein Erlebnis, das über alle Begriffe und sogar über alle
                  Feu{[}i{]}lletons geht.\pend
           
\pstart
           Ich hoffe wir sehen uns bald wieder.{\\[\baselineskip]}Sehr herzlich{\\[\baselineskip]} Ihr freundschaftlich ergebner{\\[\baselineskip]}\spacefill\mbox{ArthSchnitzler}\pend
           \leftskip=0em{}\selectlanguage{ngerman}\endnumbering\briefempfaengerindex{Zweig, Stefan@\textsc{Zweig, Stefan}!zzzSchnitzler, Arthur@\emph{von Arthur Schnitzler}!1927-10-121@{12. 10. 1927}|)be}\mylabel{L03745h}  \newcommand{\dateiname}{L03745}\newcommand{\titel}{Arthur Schnitzler an Stefan Zweig, 12. 10. 1927}\newcommand{\editorInnen}{Selma Jahnke und Martin Anton Müller}%% latex-leseansicht-abspann.tex
%% Abspann für die Leseansicht.
%% Der Schalter \ifkorrekturansicht ist bereits durch den Vorspann gesetzt.

%% latex-abspann.tex
%% Gemeinsamer Abspann für Korrekturansicht und Leseansicht.
%% Setzt den Schalter \ifkorrekturansicht voraus (gesetzt in den
%% einbindenden Dateien latex-korrekturansicht-abspann.tex bzw.
%% latex-leseansicht-abspann.tex).
%% ---------------------------------------------------------------

\normalsize

% Das esempio-Environment wird nur in der Leseansicht benötigt
\ifkorrekturansicht\else
\newenvironment{esempio}[3]%
{
    \vspace{1.5ex}
    \rlap{\underline{#1}}
    \par
    \setlength{\parindent}{0cm}
    \nopagebreak
    \leftskip=#2cm
    \rightskip=#3cm
}
{
    \par
}
\fi

\doendnotes{C}
\bigskip
\vfill

\clearpage

\footnotesize

\ifkorrekturansicht
  \lohead{\textsc{register}}
\fi

% theindex-Environment neu definieren ohne reledmac
\makeatletter
\renewenvironment{theindex}{%
  \ifkorrekturansicht
    \section*{\indexname}%
  \else
    \subsubsection*{Index der erwähnten Entitäten}%
  \fi
  \setlength{\parindent}{0pt}%
  \setlength{\parskip}{0pt plus 0.3pt}%
  \let\item\@idxitem
}{%
  \ifkorrekturansicht\clearpage\fi
}
\makeatother

\IfFileExists{\jobname-pw.ind}{\input{\jobname-pw.ind}}{}

% Quellenangabe nur in der Leseansicht
\ifkorrekturansicht\else
% Fallback-Definitionen, falls die .tex-Datei \titel etc. nicht gesetzt hat
\providecommand{\titel}{}
\providecommand{\editorInnen}{}
\providecommand{\dateiname}{\jobname}

\vspace{3cm}

\vfill

\footnotesize
\textsc{Quelle}: \titel. Herausgegeben von {\editorInnen}. In: \emph{Arthur Schnitzler: Briefwechsel mit Autorinnen und Autoren}.
 Digitale Edition, https://schnitzler-briefe.acdh.oeaw.ac.at/{\dateiname}.html (Stand \today)
\fi

\end{document}


