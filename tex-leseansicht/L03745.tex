%% latex-korrekturansicht-vorspann.tex
%% Vorspann für die Korrekturansicht.
%% Lädt die gemeinsame Datei latex-vorspann.tex mit gesetztem Schalter.

\newif\ifkorrekturansicht
\korrekturansichttrue

\input{../tex-inputs/latex-vorspann}


\section[Arthur Schnitzler an Stefan Zweig, 12. 10. 1927]{L03745 Arthur Schnitzler an Stefan Zweig, 12. 10. 1927}
\nopagebreak\mylabel{L03745v}
\rehead{ }\normalsize\beginnumbering\briefempfaengerindex{Zweig, Stefan@\textsc{Zweig, Stefan}!zzzSchnitzler, Arthur@\emph{von Arthur Schnitzler}!1927-10-121@{12. 10. 1927}|(be}
\toendnotes[C]{\smallbreak\pagebreak[2]}\Standort{Jerusalem, National Library of Israel, ARC. Ms. Var. 305 1 58 Stefan Zweig Collection.}
\physDesc{Brief, 1 Blatt, 2 Seiten, 1083 Zeichen
\newline{}Handschrift: schwarze Tinte, lateinische Kurrent}\toendnotes[C]{\smallbreak}
\pstart
           {\pb}Wien\oindex{Wien@\textbf{Wien}, \emph{A.ADM2}|pw},
                     12. Oct. 927\pend
           
\pstart{}lieber Doctor Zweig,\pend\vspace{0.5em}
\pstart
           es besteht eine Möglichkeit für mich, meine nächsten Sachen, vor Erscheinen in Deutschland\oindex{Deutschland@\textbf{Deutschland}, \emph{A.PCLI}|pw} an eine russ.\oindex{Russland@\textbf{Russland}, \emph{A.PCLI}|pw}{ }Verlagsanstalt zu
               verkaufen. Wie ist höre, haben Sie Ihr letztes \label{K_L03745-1v}\edtext{Novellenbuch\pwindex{Smjatenie Chusto@\emph{Smjatenie Chusto}|pwv}}{\lemma{\textnormal{\emph{Novellenbuch}}}\Cendnote{\textnormal{Der Verlag \emph{Wremla}\orgindex{Wremja@Wremja|pwk} hatte ohne Zweigs\pwindex{Zweig, Stefan 28.11.1881 – 23.02.1942@\textsc{Zweig, Stefan} (28.11.1881 – 23.02.1942), \emph{Schriftsteller/Schriftstellerin}|pwk} Zustimmung 1925{ }\emph{Erstes Erlebnis}\pwindex{Žgucaja tajna. Pervye pereživanija@\emph{Žgučaja tajna. Pervye pereživanija}|pwk} und 1926{ }\emph{Amok}\pwindex{Amok. Novelly@\emph{Amok. Novelly}|pwk} auf russisch\oindex{Russland@\textbf{Russland}, \emph{A.PCLI}|pwk} publiziert. Nach der Kontaktaufnahme erschienen mit Zweigs\pwindex{Zweig, Stefan 28.11.1881 – 23.02.1942@\textsc{Zweig, Stefan} (28.11.1881 – 23.02.1942), \emph{Schriftsteller/Schriftstellerin}|pwk} Zustimmung die zwei Novellen \emph{Verwirrung der Gefühle}\pwindex{Verwirrung der Gefuehle@\emph{Verwirrung der Gefühle}|pwk} und \emph{Brief einer Unbekannten}\pwindex{Brief einer Unbekannten@\emph{Brief einer Unbekannten}|pwk} unter dem Titel \emph{Smjatenie Chusto}\pwindex{Smjatenie Chusto@\emph{Smjatenie Chusto}|pwk}.}}}\label{K_L03745-1} auch nach Russland\oindex{Russland@\textbf{Russland}, \emph{A.PCLI}|pw} verkauft, und es wäre mir sehr erwünscht zu meinem
               (we{\geminationn} Sie über solche Dinge nicht principiell schweigen) welche Summe Ihnen bezahlt
               worden ist resp. unter welchen Bedingungen Sie abgeschlossen haben. Pauschalsummen?
               Perzente? Vorschuſs u. Perzente? U. s. w. Sie erweisen mir einen rechten
               Gefallen, we{\geminationn} Sie mich {\pb}aufklärten. Es handelt sich um
               einen Roman, der eben fertig geworden ist, »\label{K_L03745-2v}\edtext{Therese, Chronik eines
                  Frauenlebens\pwindex{Therese. Chronik eines Frauenlebens@\emph{Therese. Chronik eines Frauenlebens}|pw}}{\lemma{\textnormal{\emph{Therese, … Frauenlebens}}}\Cendnote{\textnormal{Zu Lebzeiten Schnitzlers kam es zu keiner russischen\oindex{Russland@\textbf{Russland}, \emph{A.PCLI}|pwk} Übersetzung des Romans.}}}\label{K_L03745-2}«. Sie haben
               hoffentlich einen schönen So{\geminationm}er gehabt. Was mich anbelangt so war ich \label{K_L03745-3v}\edtext{in den Dolomiten\oindex{Dolomiten@\textbf{Dolomiten}, \emph{Gebirge (N.GBR)}|pw}}{\lemma{\textnormal{\emph{in den Dolomiten}}}\Cendnote{\textnormal{Schnitzler war zwischen 11. 8. 1927 und 5. 9. 1927 an
                  verschiedenen Orten in Südtirol\oindex{Suedtirol@\textbf{Südtirol}, \emph{A.ADM2}|pwk} und Norditalien\oindex{Italien@\textbf{Italien}, \emph{A.PCLI}|pwk}. Am letztgenannten Tag langte er
                  in Venedig\oindex{Venedig@\textbf{Venedig}, \emph{P.PPLA}|pwk} an, wo er bis zum 15. 9. 1927 blieb.}}}\label{K_L03745-3}
               und zum Schluſs am Lido\oindex{Lido@\textbf{Lido}, \emph{P.PPL}|pw}, resp in Venedig\oindex{Venedig@\textbf{Venedig}, \emph{P.PPLA}|pw} wo meine Tochter\pwindex{Cappellini, Lili 13.09.1909 – 26.07.1928@\textsc{Cappellini, Lili} (13.09.1909 – 26.07.1928)|pwv} verheiratet mit dem Capitano Arnoldo Cappellini\pwindex{Cappellini, Arnoldo 10.08.1889 – 08.05.1954@\textsc{Cappellini, Arnoldo} (10.08.1889 – 08.05.1954)|pw}, in der Nähe der Frari Kirche\oindex{Santa Maria Gloriosa dei Frari@\textbf{Santa Maria Gloriosa dei Frari}, \emph{S.CH}|pw} lebt. Zurück bin ich \label{K_L03745-4v}\edtext{geflogen}{\lemma{\textnormal{\emph{geflogen}}}\Cendnote{\textnormal{vgl. A. S.: \emph{Tagebuch}, 15. 9. 1927.
               }}}\label{K_L03745-4}. Das ist
               ein Erlebnis, das über alle Begriffe und sogar über alle
               Feu{[}i{]}lletons geht.\pend
           
\pstart
           Ich hoffe wir sehen uns bald wieder.{\\[\baselineskip]}Sehr herzlich Ihr freundschaftlich ergebner{\\[\baselineskip]}\spacefill\mbox{ArthSchnitzler}\pend
           \leftskip=0em{}\selectlanguage{ngerman}\endnumbering\briefempfaengerindex{Zweig, Stefan@\textsc{Zweig, Stefan}!zzzSchnitzler, Arthur@\emph{von Arthur Schnitzler}!1927-10-121@{12. 10. 1927}|)be}\mylabel{L03745h}
\begin{anhang}
\end{anhang}\normalsize

\doendnotes{C}
\bigskip
\vfill

\clearpage

\footnotesize

\lohead{\textsc{register}}

% Definiere theindex-Environment komplett neu ohne reledmac
\makeatletter
\renewenvironment{theindex}{%
  \section*{\indexname}%
  \setlength{\parindent}{0pt}%
  \setlength{\parskip}{0pt plus 0.3pt}%
  \let\item\@idxitem
}{%
  \clearpage
}
\makeatother

\IfFileExists{\jobname-pw.ind}{\input{\jobname-pw.ind}}{}

\end{document}

      