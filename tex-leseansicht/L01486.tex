%% latex-korrekturansicht-vorspann.tex
%% Vorspann für die Korrekturansicht.
%% Lädt die gemeinsame Datei latex-vorspann.tex mit gesetztem Schalter.

\newif\ifkorrekturansicht
\korrekturansichttrue

\input{../tex-inputs/latex-vorspann}


\section[Hugo von Hofmannsthal an Arthur Schnitzler, 29. 12. 1904]{L01486 Hugo von Hofmannsthal an Arthur Schnitzler, 29. 12. 1904}
\nopagebreak\mylabel{L01486v}
\rehead{ }\normalsize\beginnumbering\briefempfaengerindex{Schnitzler, Arthur@\textsc{Schnitzler, Arthur}!zzzHofmannsthal, Hugo von@\emph{von Hugo von Hofmannsthal}!1904-12-291@{29. 12. 1904}|(be}
\toendnotes[C]{\smallbreak\pagebreak[2]}\Standort{CUL, Schnitzler, B 43.}
\physDesc{Postkarte, 380 Zeichen
\newline{}Handschrift: 1) schwarze Tinte, deutsche Kurrent\hspace{1em}2) schwarze Tinte, lateinische Kurrent (\noindent{}Adresse)\hspace{1em}
\newline{}Versand: 1) Stempel: »\nobreak{}\oindex{Rodaun@\textbf{Rodaun}, \emph{A.ADM4}|pwk}Rodaun, 29. 12. 04, \textcolor{gray}{7}–9N\nobreak{}«.   2) Stempel: »\nobreak{}\oindex{XVIII., Waehring@\textbf{XVIII., Währing}, \emph{A.ADM3}|pwk}18/1 Wien 110, 30. 12. 04, 12.V, Bestellt\nobreak{}«. 
\newline{}Schnitzler: mit Bleistift die Jahreszahl ergänzt: »04« 
\newline{}Ordnung: 1) mit Bleistift von unbekannter Hand nummeriert: »\strikeout{220}«  2) mit Bleistift von unbekannter Hand nummeriert: »245«}
\buchAbdrucke{\weitereDrucke{Hugo von Hofmannsthal, Arthur Schnitzler: \emph{Briefwechsel}. Frankfurt am Main: \emph{S. Fischer} 1964, S. 208.} }\toendnotes[C]{\smallbreak}\pstart{}{\pb}Herrn D\textsuperscript{r} Arthur Schnitzler\pend{}\pstart{}Wien\oindex{Wien@\textbf{Wien}, \emph{A.ADM2}|pw}\pend{}\pstart{}XVIII. Spöttelgasse 7\oindex{Edmund-Weiss-Gasse 7@\textbf{Edmund-Weiß-Gasse 7}, \emph{Wohngebäude (K.WHS)}|pw}\pend{}{\bigskip}\vspace{1em}
\pstart
           \raggedleft{}{\pb}29 XII.\pend
           \vspace{0.5em}
\pstart
           lieber, bitte doch gleich um ein Wort wann Sie \label{K_L01486-1v}\edtext{zurück}{\lemma{\textnormal{\emph{zurück}}}\Cendnote{\textnormal{Er war seit 26. 12. 1904 und noch bis zum 30. 12. 1904 in Lueg am Wolfgangsee\oindex{Lueg@\textbf{Lueg}, \emph{Teil eines besiedelten Ortes (A.BSOX)}|pwk}.}}}\label{K_L01486-1} ſind, damit man ſich noch einmal
                  \label{K_L01486-2v}\edtext{ſieht}{\lemma{\textnormal{\emph{ſieht}}}\Cendnote{\textnormal{Er reiste am 8. 1. 1905 nach Berlin\oindex{Berlin@\textbf{Berlin}, \emph{P.PPLC}|pwk}.}}}\label{K_L01486-2}. Richard\pwindex{Beer-Hofmann, Richard 1866-07-11 – 1945-09-26@\textsc{Beer-Hofmann, Richard} (1866-07-11 – 1945-09-26), \emph{Schriftsteller/Schriftstellerin}|pw} noch nicht zurück. – \textsc{Bassermann}\pwindex{Bassermann, Albert 07.09.1867 – 15.05.1952@\textsc{Bassermann, Albert} (07.09.1867 – 15.05.1952), \emph{Schauspieler/Schauspielerin}|pw} widerſtrebt der \textsc{Jaffier}\pwindex{gerettete Venedig. Trauerspiel in fuenf Aufzuegen@\emph{Das gerettete Venedig. Trauerspiel in fünf Aufzügen}|pwv}{ }ſo ſehr, daſs man ihm die Rolle abnehmen muſs. Brahm\pwindex{Brahm, Otto 05.02.1856 – 28.11.1912@\textsc{Brahm, Otto} (05.02.1856 – 28.11.1912), \emph{Theaterleiter/Theaterleiterin, Regisseur/Regisseurin}|pw} wünſcht ſie \uline{Grunwald}\pwindex{Grunwald, Willy 14.02.1870 – 1945-05-08@\textsc{Grunwald, Willy} (14.02.1870 – 1945-05-08), \emph{Theaterleiter/Theaterleiterin, Schauspieler/Schauspielerin}|pw} zu geben, der ſich heftig darum bewirbt. Brahm\pwindex{Brahm, Otto 05.02.1856 – 28.11.1912@\textsc{Brahm, Otto} (05.02.1856 – 28.11.1912), \emph{Theaterleiter/Theaterleiterin, Regisseur/Regisseurin}|pw} depeſchierte mir, ich ſollte mit Ihnen über G.\pwindex{Grunwald, Willy 14.02.1870 – 1945-05-08@\textsc{Grunwald, Willy} (14.02.1870 – 1945-05-08), \emph{Theaterleiter/Theaterleiterin, Schauspieler/Schauspielerin}|pw} reden.\pend
           
\pstart
           Ihr{\\[\baselineskip]}\spacefill\mbox{Hugo.}\pend
           \leftskip=0em{}\selectlanguage{ngerman}\endnumbering\briefempfaengerindex{Schnitzler, Arthur@\textsc{Schnitzler, Arthur}!zzzHofmannsthal, Hugo von@\emph{von Hugo von Hofmannsthal}!1904-12-291@{29. 12. 1904}|)be}\mylabel{L01486h}  \normalsize

\doendnotes{C}
\bigskip
\vfill

\clearpage

\footnotesize

\lohead{\textsc{register}}

% Definiere theindex-Environment komplett neu ohne reledmac
\makeatletter
\renewenvironment{theindex}{%
  \section*{\indexname}%
  \setlength{\parindent}{0pt}%
  \setlength{\parskip}{0pt plus 0.3pt}%
  \let\item\@idxitem
}{%
  \clearpage
}
\makeatother

\IfFileExists{\jobname-pw.ind}{\input{\jobname-pw.ind}}{}

\end{document}

      