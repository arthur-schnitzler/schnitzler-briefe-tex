%% latex-leseansicht-vorspann.tex
%% Vorspann für die Leseansicht.
%% Lädt die gemeinsame Datei latex-vorspann.tex mit nicht gesetztem Schalter.

\newif\ifkorrekturansicht
\korrekturansichtfalse

\input{../tex-inputs/latex-vorspann}

\begin{center}
            \textcolor{red}{ENTWURF. ENTZIFFERUNG NOCH NICHT KORREKTURGELESEN}
                      \end{center}
            
               \section[Hermann Bahr an Arthur Schnitzler, {[}23. 3. 1906{]}]{ Hermann Bahr an Arthur Schnitzler, {[}23. 3. 1906{]}}\nopagebreak\mylabel{v}\rehead{ }\begin{ledgroupsized}[t]{13cm}\normalsize\beginnumbering\briefempfaengerindex{Schnitzler, Arthur@\textsc{Schnitzler, Arthur}!zzzBahr, Hermann@\emph{von Hermann Bahr}!1906-03-233@{{[}23. 3. 1906{]}}|(be} \toendnotes[C]{\smallbreak\pagebreak[2]} \Standort{CUL, Schnitzler, B 5b.}
\physDesc{Telegramm
\newline{}Handschrift einer Schreibkraft: blaue Tinte, lateinische Kurrent\newline{}Versand: 1) »Wien\oindex{Wien@\textbf{Wien}|pw} 93 \textcolor{gray}{\textbf{Nr.}} 132 \textcolor{gray}{\textbf{Taxw.}} 11 \textcolor{gray}{\textbf{(W.{\dots} Ch.{\dots}) aufgegeben am {\dots}/{\dots}190{\dots} um}} 6 \textcolor{gray}{\textbf{Uhr}} n\textcolor{gray}{\textbf{Mittag.}}« 2) beschnitten
\newline{}Schnitzler: mit Bleistift datiert: »23/3/906« \newline{}Ordnung: mit Bleistift von unbekannter Hand nummeriert:
                                    »138« }\buchAbdrucke{\weitereDrucke{Hermann Bahr, Arthur Schnitzler: \emph{Briefwechsel, Aufzeichnungen, Dokumente (1891–1931)}. Hg. Kurt Ifkovits und Martin Anton Müller. Göttingen: \emph{Wallstein} 2018, S. 376.} }\toendnotes[C]{\smallbreak}\pstart
           \noindent{}{\pb}Mildenburg\pwindex{Bahr-Mildenburg, Anna 29.11.1872 – 27.01.1947@\textsc{Bahr-Mildenburg, Anna} (29.11.1872 – 27.01.1947), \emph{Sängerin}|pw}{ }\label{K_L01594_1v}\edtext{singt morgen}{\lemma{\textnormal{\emph{singt morgen}}}\Cendnote{\textnormal{Am 24. 3. 1906 brachte die Hofoper\oindex{Oper@\textbf{Oper}|pwk}{ }\emph{Don Giovanni}\pwindex{\textcolor{red}{\textsuperscript{XXXX1 indx}}!Don Giovanni1787@\strich\emph{Don Giovanni} {[}1787{]}|pwk}\pwindex{\textcolor{red}{\textsuperscript{XXXX1 indx}}!Don Giovanni1787@\strich\emph{Don Giovanni} {[}1787{]}|pwk}; Mildenburg\pwindex{Bahr-Mildenburg, Anna 29.11.1872 – 27.01.1947@\textsc{Bahr-Mildenburg, Anna} (29.11.1872 – 27.01.1947), \emph{Sängerin}|pwk} sang Donna
                     Anna\pwindex{\textcolor{red}{\textsuperscript{XXXX1 indx}}!Don Giovanni1787@\strich\emph{Don Giovanni} {[}1787{]}|pwkv}\pwindex{\textcolor{red}{\textsuperscript{XXXX1 indx}}!Don Giovanni1787@\strich\emph{Don Giovanni} {[}1787{]}|pwkv}.}}}\label{K_L01594_1h} herzlichſt \spacefill\mbox{herrmann}\pend
           \endnumbering\briefempfaengerindex{Schnitzler, Arthur@\textsc{Schnitzler, Arthur}!zzzBahr, Hermann@\emph{von Hermann Bahr}!1906-03-233@{{[}23. 3. 1906{]}}|)be}\mylabel{h}\end{ledgroupsized}  \newcommand{\dateiname}{L01594}\newcommand{\titel}{Hermann Bahr an Arthur Schnitzler, [23. 3. 1906]}\newcommand{\editorInnen}{ Kurt Ifkovits,  Martin Anton Müller}%% latex-leseansicht-abspann.tex
%% Abspann für die Leseansicht.
%% Der Schalter \ifkorrekturansicht ist bereits durch den Vorspann gesetzt.

%% latex-abspann.tex
%% Gemeinsamer Abspann für Korrekturansicht und Leseansicht.
%% Setzt den Schalter \ifkorrekturansicht voraus (gesetzt in den
%% einbindenden Dateien latex-korrekturansicht-abspann.tex bzw.
%% latex-leseansicht-abspann.tex).
%% ---------------------------------------------------------------

\normalsize

% Das esempio-Environment wird nur in der Leseansicht benötigt
\ifkorrekturansicht\else
\newenvironment{esempio}[3]%
{
    \vspace{1.5ex}
    \rlap{\underline{#1}}
    \par
    \setlength{\parindent}{0cm}
    \nopagebreak
    \leftskip=#2cm
    \rightskip=#3cm
}
{
    \par
}
\fi

\doendnotes{C}
\bigskip
\vfill

\clearpage

\footnotesize

\ifkorrekturansicht
  \lohead{\textsc{register}}
\fi

% theindex-Environment neu definieren ohne reledmac
\makeatletter
\renewenvironment{theindex}{%
  \ifkorrekturansicht
    \section*{\indexname}%
  \else
    \subsubsection*{Index der erwähnten Entitäten}%
  \fi
  \setlength{\parindent}{0pt}%
  \setlength{\parskip}{0pt plus 0.3pt}%
  \let\item\@idxitem
}{%
  \ifkorrekturansicht\clearpage\fi
}
\makeatother

\IfFileExists{\jobname-pw.ind}{\input{\jobname-pw.ind}}{}

% Quellenangabe nur in der Leseansicht
\ifkorrekturansicht\else
% Fallback-Definitionen, falls die .tex-Datei \titel etc. nicht gesetzt hat
\providecommand{\titel}{}
\providecommand{\editorInnen}{}
\providecommand{\dateiname}{\jobname}

\vspace{3cm}

\vfill

\footnotesize
\textsc{Quelle}: \titel. Herausgegeben von {\editorInnen}. In: \emph{Arthur Schnitzler: Briefwechsel mit Autorinnen und Autoren}.
 Digitale Edition, https://schnitzler-briefe.acdh.oeaw.ac.at/{\dateiname}.html (Stand \today)
\fi

\end{document}


      