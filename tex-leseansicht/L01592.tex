%% latex-korrekturansicht-vorspann.tex
%% Vorspann für die Korrekturansicht.
%% Lädt die gemeinsame Datei latex-vorspann.tex mit gesetztem Schalter.

\newif\ifkorrekturansicht
\korrekturansichttrue

\input{../tex-inputs/latex-vorspann}


\section[Arthur Schnitzler: Widmungsexemplar Marionetten für Hermann Bahr, 23. 3. 1906]{L01592 Arthur Schnitzler: Widmungsexemplar Marionetten für Hermann Bahr,
               23. 3. 1906}
\nopagebreak\mylabel{L01592v}
\rehead{ }\normalsize\beginnumbering\briefempfaengerindex{Bahr, Hermann@\textsc{Bahr, Hermann}!zzzSchnitzler, Arthur@\emph{von Arthur Schnitzler}!1906-03-231@{23. 3. 1906}|(be}
\toendnotes[C]{\smallbreak\pagebreak[2]}\Standort{Salzburg, Universitätsbibliothek, 32323-I.}
\physDesc{Widmung am Vorsatzblatt, 51 Zeichen
\newline{}Handschrift: schwarze Tinte, deutsche Kurrent}
\buchAbdrucke{\weitereDrucke{Hermann Bahr, Arthur Schnitzler: \emph{Briefwechsel, Aufzeichnungen, Dokumente (1891–1931)}. Göttingen: \emph{Wallstein} 2018, S. 376.} }
\pstart
           \noindent{}{\pb}Meinem lieben Hermann Bahr\pend
           \pstart \spacefill\mbox{ArthSch}\pend{}
\pstart
           \noindent{}Wien\oindex{Wien@\textbf{Wien}, \emph{A.ADM2}|pw}{ }23. 3. 906.\pend
           \selectlanguage{ngerman}\vspace{1em}{\vspace{1\baselineskip}}
\pstart
           \centering{}{\pb}\textcolor{gray}{\textbf{\so{MARIONETTEN}\pwindex{Marionetten. Drei Einakter@\emph{Marionetten. Drei Einakter}|pw}}}\pend
           
\pstart
           \centering{}\textcolor{gray}{\textbf{Drei Einakter von}}\pend
           
\pstart
           \centering{}\textcolor{gray}{\textbf{\so{Arthur Schnitzler}}}\pend
           {\vspace{1\baselineskip}}
\pstart
           \centering{}\textcolor{gray}{\textbf{\so{S. Fischer, Verlag}\orgindex{S. Fischer Verlag@S. Fischer Verlag|pw}\so{,{ }}\so{Berlin}\oindex{Berlin@\textbf{Berlin}, \emph{P.PPLC}|pw}}}\pend
           
\pstart
           \centering{}\textcolor{gray}{\textbf{1906}}\pend
           \selectlanguage{ngerman}\endnumbering\briefempfaengerindex{Bahr, Hermann@\textsc{Bahr, Hermann}!zzzSchnitzler, Arthur@\emph{von Arthur Schnitzler}!1906-03-231@{23. 3. 1906}|)be}\mylabel{L01592h}  \normalsize

\doendnotes{C}
\bigskip
\vfill

\clearpage

\footnotesize

\lohead{\textsc{register}}

% Definiere theindex-Environment komplett neu ohne reledmac
\makeatletter
\renewenvironment{theindex}{%
  \section*{\indexname}%
  \setlength{\parindent}{0pt}%
  \setlength{\parskip}{0pt plus 0.3pt}%
  \let\item\@idxitem
}{%
  \clearpage
}
\makeatother

\IfFileExists{\jobname-pw.ind}{\input{\jobname-pw.ind}}{}

\end{document}

      