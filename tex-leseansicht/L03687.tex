%% latex-korrekturansicht-vorspann.tex
%% Vorspann für die Korrekturansicht.
%% Lädt die gemeinsame Datei latex-vorspann.tex mit gesetztem Schalter.

\newif\ifkorrekturansicht
\korrekturansichttrue

\input{../tex-inputs/latex-vorspann}


\section[Stefan Zweig an Arthur Schnitzler, 14. 10. 1927]{L03687 Stefan Zweig an Arthur Schnitzler, 14. 10. 1927}
\nopagebreak\mylabel{L03687v}
\rehead{ }\normalsize\beginnumbering\briefempfaengerindex{Schnitzler, Arthur@\textsc{Schnitzler, Arthur}!zzzZweig, Stefan@\emph{von Stefan Zweig}!1927-10-141@{14. 10. 1927}|(be}
\toendnotes[C]{\smallbreak\pagebreak[2]}\Standort{CUL, Schnitzler, B 118.}
\physDesc{Brief, 1 Blatt, 2 Seiten, 1758 Zeichen
\newline{}Schreibmaschine
\newline{}Handschrift: blaue Tinte, lateinische Kurrent (\noindent{}Unterschrift)
\newline{}Schnitzler: 1) mit Bleistift beschriftet: »\textsc{Zweig}«  2) mit rotem Buntstift sechs Unterstreichungen}
\buchAbdrucke{\weitereDrucke{Stefan Zweig: \emph{Briefwechsel mit Hermann Bahr, Sigmund Freud, Rainer Maria Rilke und
                Arthur Schnitzler}. Frankfurt am Main: \emph{S. Fischer} 1987, S. 431–432.} }\toendnotes[C]{\smallbreak}
\pstart
           {\pb}\textcolor{gray}{\textbf{SZ}}\hfill \textcolor{gray}{\textbf{SALZBURG\oindex{Salzburg@\textbf{Salzburg}, \emph{A.ADM2}|pw},}}\pend
           
\pstart
           \raggedleft{}\textcolor{gray}{\textbf{KAPUZINERBERG 5\oindex{Paschinger Schloessl@\textbf{Paschinger Schlössl}, \emph{Wohngebäude (K.WHS)}|pw}}}\pend
           
\pstart
           \raggedleft{}14. Oktober 1927.\pend
           
\pstart{}Lieber, verehrter Herr Doktor!\pend\vspace{0.5em}
\pstart
           Ihre Handschrift erweckt immer freudiges Gefühl in mir und ich eile mich, Ihnen zu
          antworten, freilich nicht unbeschämt, denn meine Auskunft ist unverantwortlich ungenau.
          Ich bin in allen Honorardingen geradezu tölpisch leichtsinnig, kümmere mich um gar nichts
          und die Honorare, die ich bislang für \label{K_L03687-1v}\edtext{Verfilmungen meiner Novellen\pwindex{Brennendes Geheimnis@\emph{Brennendes Geheimnis}|pwv}\pwindex{Angst@\emph{Angst}|pwv}\pwindex{Amoklaeufer@\emph{Der Amokläufer}|pwv}}{\lemma{\textnormal{\emph{Verfilmungen … Novellen}}}\Cendnote{\textnormal{1923 wurde Zweigs\pwindex{Zweig, Stefan 28.11.1881 – 23.02.1942@\textsc{Zweig, Stefan} (28.11.1881 – 23.02.1942), \emph{Schriftsteller/Schriftstellerin}|pwk} Novelle \emph{Brennendes Geheimnis}\pwindex{Brennendes Geheimnis@\emph{Brennendes Geheimnis}|pwk} unter dem Titel \emph{Mutter, Dein Kind ruft!}\pwindex{Mutter, Dein Kind ruft@\emph{Mutter, Dein Kind ruft{\rufezeichen}}|pwk} verfilmt, 1924 sein
            Schauspiel \emph{Das Haus am Meer}\pwindex{Haus am Meer. Ein Schauspiel in zwei Teilen (drei Aufzuegen)@\emph{Das Haus am Meer. Ein Schauspiel in zwei Teilen (drei Aufzügen)}|pwk}, 1927 die
            Novelle \emph{Der Amokläufer}\pwindex{Amoklaeufer@\emph{Der Amokläufer}|pwk} unter der Regie von Kote Marjanishvili\pwindex{Marjanishvili, Kote †~1933-04-17@\textsc{Marjanishvili, Kote} (†~1933-04-17), \emph{Filmregisseur/Filmregisseurin}|pwk} in Russland\oindex{Russland@\textbf{Russland}, \emph{A.PCLI}|pwk} und im Folgejahr die Novelle \emph{Angst}\pwindex{Angst@\emph{Angst}|pwk} als Stummfilm\pwindex{Angst@\emph{Angst}|pwkv} unter der Regie von Hans
              Steinhoff\pwindex{Steinhoff, Hans 10.03.1882 – 20.04.1945@\textsc{Steinhoff, Hans} (10.03.1882 – 20.04.1945)|pwk}.}}}\label{K_L03687-1} erhielt, haben die Heiterkeit der Fachleute herausgefordert. So
          habe ich auch in Russland\oindex{Russland@\textbf{Russland}, \emph{A.PCLI}|pw} glattweg die Vorschläge
          angenommen, die mir die »Wremja\orgindex{Wremja@Wremja|pw}« stellte und die ich
          gar nicht mehr auswendig weiss. Ich kann nur feststellen, dass der Ertrag sich \label{K_L03687-2v}\edtext{bei dem letzten Buche\pwindex{Smjatenie Chusto@\emph{Smjatenie Chusto}|pwv}}{\lemma{\textnormal{\emph{bei dem letzten Buche}}}\Cendnote{\textnormal{Vgl. Arthur Schnitzler an Stefan Zweig, 12. 10. 1927.}}}\label{K_L03687-2} etwa auf 150 Dollar belief, bin aber gewiss, dass Sie das Vierfache
          erzielen können. Die Buchpreise sind ja drüben nicht sehr wesentlich, aber nach den neuen
          Vereinbarungen, deren Text ich noch nicht kenne, hat Lunatscharski\pwindex{Lunacarski, Anatolij W. 1875 – 1933@\textsc{Lunačarski, Anatolij W.} (1875 – 1933), \emph{Schriftsteller/Schriftstellerin, Politiker/Politikerin, Drehbuchautor/Drehbuchautorin}|pw} auch von den unerlaubten Nachdrucken jetzt eine gewisse Quote für
          den ausländischen Autor festgesetzt. Ob sie gezahlt wird, ist eine andere Sache. Ich
          persönlich würde Ihnen raten, sich Russland\oindex{Russland@\textbf{Russland}, \emph{A.PCLI}|pw}
          gegenüber nicht auf Perzente einzulassen, weil man ja jeder Kontrollmöglichkeit entzogen
          ist, und eine einmalige Dollarsumme zu fordern: es ist ja ohnehin ein Wunder, wenn man
          etwas aus Russland\oindex{Russland@\textbf{Russland}, \emph{A.PCLI}|pw} herausbekommt. Ich hoffe, Sie
          allerdings in sechs Monaten viel besser informieren zu können, denn ich möchte sehr gerne
          im März mir für vier Wochen die Sache \label{K_L03687-3v}\edtext{persönlich anschauen}{\lemma{\textnormal{\emph{persönlich anschauen}}}\Cendnote{\textnormal{Stefan Zweig\pwindex{Zweig, Stefan 28.11.1881 – 23.02.1942@\textsc{Zweig, Stefan} (28.11.1881 – 23.02.1942), \emph{Schriftsteller/Schriftstellerin}|pwk} reiste erst vom 7. bis
            zum 20. 9. 1928 nach Russland\oindex{Russland@\textbf{Russland}, \emph{A.PCLI}|pwk}, um an
            der Gedenkfeier zum 100. Geburtstag Tolstois\pwindex{Tolstoi, Leo N. von 09.09.1828 – 20.11.1910@\textsc{Tolstoi, Leo N. von} (09.09.1828 – 20.11.1910), \emph{Schriftsteller/Schriftstellerin, Schriftsteller/Schriftstellerin, Krimiautor/Krimiautorin}|pwk}
            teilzunehmen.}}}\label{K_L03687-3}.\pend
           
\pstart
           {\pb}Ich beglückwünsche Sie sehr dazu, so rasch und
          fleissig ein schöpferisches Buch dem anderen nachzusenden, was mir leider nicht gelingen
          will. Ich habe nur Kleineres zu bieten und dies mögen Sie heute mit der Biographie der Desbordes-Valmore\pwindex{Desbordes-Valmore, Marceline 1786-06-20 – 1859-07-23@\textsc{Desbordes-Valmore, Marceline} (1786-06-20 – 1859-07-23), \emph{Schauspieler/Schauspielerin, Sänger/Sängerin, Schriftsteller/Schriftstellerin}|pw}\pwindex{Marceline Desbordes-Valmore. Das Lebensbild einer Dichterin@\emph{Marceline Desbordes-Valmore. Das Lebensbild einer Dichterin}|pw} und den essayistischen Miniaturen\pwindex{Sternstunden der Menschheit@\emph{Sternstunden der Menschheit}|pwv} freundlich empfangen.\pend
           
\pstart
           In getreuer Liebe und Verehrung Ihr{\\[\baselineskip]}\spacefill\mbox{{[}hs.:{]} Stefan Zweig}\pend
           \leftskip=0em{}\selectlanguage{ngerman}\endnumbering\briefempfaengerindex{Schnitzler, Arthur@\textsc{Schnitzler, Arthur}!zzzZweig, Stefan@\emph{von Stefan Zweig}!1927-10-141@{14. 10. 1927}|)be}\mylabel{L03687h}
\begin{anhang}
\end{anhang}\normalsize

\doendnotes{C}
\bigskip
\vfill

\clearpage

\footnotesize

\lohead{\textsc{register}}

% Definiere theindex-Environment komplett neu ohne reledmac
\makeatletter
\renewenvironment{theindex}{%
  \section*{\indexname}%
  \setlength{\parindent}{0pt}%
  \setlength{\parskip}{0pt plus 0.3pt}%
  \let\item\@idxitem
}{%
  \clearpage
}
\makeatother

\IfFileExists{\jobname-pw.ind}{\input{\jobname-pw.ind}}{}

\end{document}

      