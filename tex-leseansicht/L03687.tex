%% latex-leseansicht-vorspann.tex
%% Vorspann für die Leseansicht.
%% Lädt die gemeinsame Datei latex-vorspann.tex mit nicht gesetztem Schalter.

\newif\ifkorrekturansicht
\korrekturansichtfalse

\input{../tex-inputs/latex-vorspann}


\section[Stefan Zweig an Arthur Schnitzler, 14. 10. 1927]{L03687 Stefan Zweig an Arthur Schnitzler, 14. 10. 1927}
\nopagebreak\mylabel{L03687v}
\rehead{ }\normalsize\beginnumbering\briefempfaengerindex{Schnitzler, Arthur@\textsc{Schnitzler, Arthur}!zzzZweig, Stefan@\emph{von Stefan Zweig}!1927-10-141@{14. 10. 1927}|(be}
\toendnotes[C]{\smallbreak\pagebreak[2]}
\correspDesc{Versand  durch Stefan Zweig am 14. 10. 1927 in Salzburg
\newline{}Erhalt  durch Arthur Schnitzler am [15. 10. 1927?] in Wien}\toendnotes[C]{\smallbreak}
\Standort{CUL, Schnitzler, B 118.}
\physDesc{Brief, 1 Blatt, 2 Seiten, 1756 Zeichen
\newline{}Schreibmaschine
\newline{}Handschrift: blaue Tinte (\noindent{}Unterschrift)
\newline{}Schnitzler: 1) mit Bleistift beschriftet: »\textsc{Zweig}«  2) mit rotem Buntstift sechs Unterstreichungen}
\buchAbdrucke{\weitereDrucke{Stefan Zweig: \emph{Briefwechsel mit Hermann Bahr, Sigmund Freud, Rainer Maria
                        Rilke und Arthur Schnitzler}. Herausgegeben von Jeffrey B. Berlin, Hans-Ulrich Lindken und Donald A. Prater. Frankfurt am Main: \emph{S. Fischer} 1987, S. 431–432.} }\toendnotes[C]{\smallbreak}
\pstart
           {\pb}\textcolor{gray}{\textbf{SZ}}\hfill \textcolor{gray}{\textbf{SALZBURG\oindex{Salzburg@\textbf{Salzburg}, \emph{Verwaltungsgebiet}|pw}}}\pend
           
\pstart
           \raggedleft{}\textcolor{gray}{\textbf{KAPUZINERBERG 5\oindex{Paschinger Schlössl@\textbf{Paschinger Schlössl}, \emph{Wohngebäude}|pw}}}\pend
           
\pstart
           \raggedleft{}14. Oktober 1927.\pend
           
\pstart{}Lieber, verehrter Herr Doktor!\pend\vspace{0.5em}
\pstart
           Ihre Handschrift erweckt immer freudiges Gefühl in mir und ich eile mich, Ihnen zu
               antworten, freilich nicht unbeschämt, denn meine Auskunft ist unverantwortlich
               ungenau. Ich bin in allen Honorardingen geradezu tölpisch leichtsinnig, kümmere mich
               um gar nichts und die Honorare, die ich bislang für \label{K_L03687-1v}\edtext{Verfilmungen meiner Novellen\pwindex{Zweig, Stefan 28.\,11.\,1881 Wien – 23.\,2.\,1942 Petrópolis@\textsc{Zweig, Stefan} (28.\,11.\,1881 Wien – 23.\,2.\,1942 Petrópolis), \emph{Schriftsteller}!Brennendes Geheimnis@\strich\emph{Brennendes Geheimnis}|pwv}\pwindex{Zweig, Stefan 28.\,11.\,1881 Wien – 23.\,2.\,1942 Petrópolis@\textsc{Zweig, Stefan} (28.\,11.\,1881 Wien – 23.\,2.\,1942 Petrópolis), \emph{Schriftsteller}!Angst@\strich\emph{Angst}|pwv}\pwindex{Zweig, Stefan 28.\,11.\,1881 Wien – 23.\,2.\,1942 Petrópolis@\textsc{Zweig, Stefan} (28.\,11.\,1881 Wien – 23.\,2.\,1942 Petrópolis), \emph{Schriftsteller}!Amokläufer@\strich\emph{Der Amokläufer}|pwv}}{\lemma{\textnormal{\emph{Verfilmungen … Novellen}}}\Cendnote{\textnormal{1923 wurde Zweigs\pwindex{Zweig, Stefan 28.\,11.\,1881 Wien – 23.\,2.\,1942 Petrópolis@\textsc{Zweig, Stefan} (28.\,11.\,1881 Wien – 23.\,2.\,1942 Petrópolis), \emph{Schriftsteller}|pwk} Novelle \emph{Brennendes Geheimnis}\pwindex{Zweig, Stefan 28.\,11.\,1881 Wien – 23.\,2.\,1942 Petrópolis@\textsc{Zweig, Stefan} (28.\,11.\,1881 Wien – 23.\,2.\,1942 Petrópolis), \emph{Schriftsteller}!Brennendes Geheimnis@\strich\emph{Brennendes Geheimnis}|pwk} unter dem Titel \emph{Mutter, Dein Kind ruft!}\pwindex{Zweig, Stefan 28.\,11.\,1881 Wien – 23.\,2.\,1942 Petrópolis@\textsc{Zweig, Stefan} (28.\,11.\,1881 Wien – 23.\,2.\,1942 Petrópolis), \emph{Schriftsteller}!Mutter, Dein Kind ruft@\strich\emph{Mutter, Dein Kind ruft{\rufezeichen}}|pwk}\pwindex{\textcolor{red}{\textsuperscript{XXXX indx1}}!Mutter, Dein Kind ruft@\strich\emph{Mutter, Dein Kind ruft{\rufezeichen}}|pwk}\pwindex{\textcolor{red}{\textsuperscript{XXXX indx1}}!Mutter, Dein Kind ruft@\strich\emph{Mutter, Dein Kind ruft{\rufezeichen}}|pwk}\pwindex{\textcolor{red}{\textsuperscript{XXXX indx1}}!Mutter, Dein Kind ruft@\strich\emph{Mutter, Dein Kind ruft{\rufezeichen}}|pwk}\pwindex{\textcolor{red}{\textsuperscript{XXXX indx1}}!Mutter, Dein Kind ruft@\strich\emph{Mutter, Dein Kind ruft{\rufezeichen}}|pwk} verfilmt,
                     1924 sein Schauspiel \emph{Das Haus am
                     Meer}\pwindex{Zweig, Stefan 28.\,11.\,1881 Wien – 23.\,2.\,1942 Petrópolis@\textsc{Zweig, Stefan} (28.\,11.\,1881 Wien – 23.\,2.\,1942 Petrópolis), \emph{Schriftsteller}!Haus am Meer. Ein Schauspiel in zwei Teilen (drei Aufzügen)@\strich\emph{Das Haus am Meer. Ein Schauspiel in zwei Teilen (drei Aufzügen)}|pwk}, 1927 die Novelle \emph{Der
                     Amokläufer}\pwindex{Zweig, Stefan 28.\,11.\,1881 Wien – 23.\,2.\,1942 Petrópolis@\textsc{Zweig, Stefan} (28.\,11.\,1881 Wien – 23.\,2.\,1942 Petrópolis), \emph{Schriftsteller}!Amokläufer@\strich\emph{Der Amokläufer}|pwk} unter der Regie von Kote
                     Marjanishvili\pwindex{Marjanishvili, Kote †~17.\,4.\,1933 Moskau@\textsc{Marjanishvili, Kote} (†~17.\,4.\,1933 Moskau), \emph{Filmregisseur}|pwk} in Russland\oindex{Russland@\textbf{Russland}|pwk} und im
                  Folgejahr die Novelle \emph{Angst}\pwindex{Zweig, Stefan 28.\,11.\,1881 Wien – 23.\,2.\,1942 Petrópolis@\textsc{Zweig, Stefan} (28.\,11.\,1881 Wien – 23.\,2.\,1942 Petrópolis), \emph{Schriftsteller}!Angst@\strich\emph{Angst}|pwk} als Stummfilm\pwindex{Steinhoff, Hans 10.\,3.\,1882 Marienberg – 20.\,4.\,1945 Glienig@\textsc{Steinhoff, Hans} (10.\,3.\,1882 Marienberg – 20.\,4.\,1945 Glienig)!Angst@\strich\emph{Angst}|pwkv} unter der Regie
                  von Hans Steinhoff\pwindex{Steinhoff, Hans 10.\,3.\,1882 Marienberg – 20.\,4.\,1945 Glienig@\textsc{Steinhoff, Hans} (10.\,3.\,1882 Marienberg – 20.\,4.\,1945 Glienig)|pwk}.}}}\label{K_L03687-1} erhielt, haben
               die Heiterkeit der Fachleute herausgefordert. So habe ich auch in Russland\oindex{Russland@\textbf{Russland}|pw} glattweg die Vorschläge angenommen, die mir die »Wremja\orgindex{Wremja@Wremja|pw}« stellte und die ich gar nicht mehr
               auswendig weiss. Ich kann nur feststellen, dass der Ertrag sich \label{K_L03687-2v}\edtext{bei dem letzten Buche\pwindex{Zweig, Stefan 28.\,11.\,1881 Wien – 23.\,2.\,1942 Petrópolis@\textsc{Zweig, Stefan} (28.\,11.\,1881 Wien – 23.\,2.\,1942 Petrópolis), \emph{Schriftsteller}!Smjatenie Chusto@\strich\emph{Smjatenie Chusto}|pwv}}{\lemma{\textnormal{\emph{bei dem letzten Buche}}}\Cendnote{\textnormal{Vgl. XXXX Auszeichnungsfehler: Dokument L03745 nicht gefunden.}}}\label{K_L03687-2} etwa auf 150
               Dollar belief, bin aber gewiss, dass Sie das Vierfache erzielen können. Die
               Buchpreise sind ja drüben nicht sehr wesentlich, aber nach den neuen Vereinbarungen,
               deren Text ich noch nicht kenne, hat Lunatscharski\pwindex{Lunačarski, Anatolij W. 1875 Poltawa – 1933 Menton@\textsc{Lunačarski, Anatolij W.} (1875 Poltawa – 1933 Menton), \emph{Schriftsteller, Politiker, Drehbuchautor}|pw} auch von den unerlaubten Nachdrucken jetzt eine gewisse Quote
               für den ausländischen Autor festgesetzt. Ob sie gezahlt wird, ist eine andere Sache.
               Ich persönlich würde Ihnen raten, sich Russland\oindex{Russland@\textbf{Russland}|pw}
               gegenüber nicht auf Perzente einzulassen, weil man ja jeder Kontrollmöglichkeit
               entzogen ist, und eine einmalige Dollarsumme zu fordern: es ist ja ohnehin ein
               Wunder, wenn man etwas aus Russland\oindex{Russland@\textbf{Russland}|pw}
               herausbekommt. Ich hoffe, Sie allerdings in sechs Monaten viel besser informieren zu
               können, denn ich möchte sehr gerne im März mir für vier Wochen die Sache
                  \label{K_L03687-3v}\edtext{persönlich anschauen}{\lemma{\textnormal{\emph{persönlich anschauen}}}\Cendnote{\textnormal{Stefan Zweig\pwindex{Zweig, Stefan 28.\,11.\,1881 Wien – 23.\,2.\,1942 Petrópolis@\textsc{Zweig, Stefan} (28.\,11.\,1881 Wien – 23.\,2.\,1942 Petrópolis), \emph{Schriftsteller}|pwk} reiste erst vom
                     7. 9. 1928 bis zum 20. 9. 1928 nach Russland\oindex{Russland@\textbf{Russland}|pwk}, um an der Gedenkfeier zum 100.
                  Geburtstag Tolstois\pwindex{Tolstoi, Lew Nikolajewitsch 9.\,9.\,1828 Yasnaya Polyana – 20.\,11.\,1910 Lev Tolstoy@\textsc{Tolstoi, Lew Nikolajewitsch} (9.\,9.\,1828 Yasnaya Polyana – 20.\,11.\,1910 Lev Tolstoy), \emph{Schriftsteller}|pwk}
               teilzunehmen.}}}\label{K_L03687-3}.\pend
           
\pstart
           {\pb}Ich beglückwünsche Sie sehr dazu, so
               rasch und fleissig ein schöpferisches Buch dem anderen\pwindex{Schnitzler, Arthur 15.\,5.\,1862 Wien – 21.\,10.\,1931 ebd.@\textsc{Schnitzler, Arthur} (15.\,5.\,1862 Wien – 21.\,10.\,1931 ebd.), \emph{Schriftsteller, Mediziner}!Geist im Wort und der Geist in der Tat@\strich\emph{Der Geist im Wort und der Geist in der Tat}|pwv} nachzusenden, was mir leider nicht gelingen will.
               Ich habe nur Kleineres zu bieten und dies mögen Sie heute mit der Biographie der Desbordes-Valmore\pwindex{Desbordes-Valmore, Marceline 20.\,6.\,1786 Douai – 23.\,7.\,1859 Paris@\textsc{Desbordes-Valmore, Marceline} (20.\,6.\,1786 Douai – 23.\,7.\,1859 Paris), \emph{Schauspielerin, Sängerin, Schriftstellerin}|pw}\pwindex{Zweig, Stefan 28.\,11.\,1881 Wien – 23.\,2.\,1942 Petrópolis@\textsc{Zweig, Stefan} (28.\,11.\,1881 Wien – 23.\,2.\,1942 Petrópolis), \emph{Schriftsteller}!Marceline Desbordes-Valmore. Das Lebensbild einer Dichterin@\strich\emph{Marceline Desbordes-Valmore. Das Lebensbild einer Dichterin}|pw} und den essayistischen Miniaturen\pwindex{Zweig, Stefan 28.\,11.\,1881 Wien – 23.\,2.\,1942 Petrópolis@\textsc{Zweig, Stefan} (28.\,11.\,1881 Wien – 23.\,2.\,1942 Petrópolis), \emph{Schriftsteller}!Sternstunden der Menschheit@\strich\emph{Sternstunden der Menschheit}|pwv} freundlich empfangen.\pend
           
\pstart
           In getreuer Liebe und Verehrung Ihr{\\[\baselineskip]}\spacefill\mbox{{[}hs.:{]} Stefan Zweig}\pend
           \leftskip=0em{}\selectlanguage{ngerman}\endnumbering\briefempfaengerindex{Schnitzler, Arthur@\textsc{Schnitzler, Arthur}!zzzZweig, Stefan@\emph{von Stefan Zweig}!1927-10-141@{14. 10. 1927}|)be}\mylabel{L03687h}  \newcommand{\dateiname}{L03687}\newcommand{\titel}{Stefan Zweig an Arthur Schnitzler, 14. 10. 1927}\newcommand{\editorInnen}{Selma Jahnke und Martin Anton Müller}%% latex-leseansicht-abspann.tex
%% Abspann für die Leseansicht.
%% Der Schalter \ifkorrekturansicht ist bereits durch den Vorspann gesetzt.

%% latex-abspann.tex
%% Gemeinsamer Abspann für Korrekturansicht und Leseansicht.
%% Setzt den Schalter \ifkorrekturansicht voraus (gesetzt in den
%% einbindenden Dateien latex-korrekturansicht-abspann.tex bzw.
%% latex-leseansicht-abspann.tex).
%% ---------------------------------------------------------------

\normalsize

% Das esempio-Environment wird nur in der Leseansicht benötigt
\ifkorrekturansicht\else
\newenvironment{esempio}[3]%
{
    \vspace{1.5ex}
    \rlap{\underline{#1}}
    \par
    \setlength{\parindent}{0cm}
    \nopagebreak
    \leftskip=#2cm
    \rightskip=#3cm
}
{
    \par
}
\fi

\doendnotes{C}
\bigskip
\vfill

\clearpage

\footnotesize

\ifkorrekturansicht
  \lohead{\textsc{register}}
\fi

% theindex-Environment neu definieren ohne reledmac
\makeatletter
\renewenvironment{theindex}{%
  \ifkorrekturansicht
    \section*{\indexname}%
  \else
    \subsubsection*{Index der erwähnten Entitäten}%
  \fi
  \setlength{\parindent}{0pt}%
  \setlength{\parskip}{0pt plus 0.3pt}%
  \let\item\@idxitem
}{%
  \ifkorrekturansicht\clearpage\fi
}
\makeatother

\IfFileExists{\jobname-pw.ind}{\input{\jobname-pw.ind}}{}

% Quellenangabe nur in der Leseansicht
\ifkorrekturansicht\else
% Fallback-Definitionen, falls die .tex-Datei \titel etc. nicht gesetzt hat
\providecommand{\titel}{}
\providecommand{\editorInnen}{}
\providecommand{\dateiname}{\jobname}

\vspace{3cm}

\vfill

\footnotesize
\textsc{Quelle}: \titel. Herausgegeben von {\editorInnen}. In: \emph{Arthur Schnitzler: Briefwechsel mit Autorinnen und Autoren}.
 Digitale Edition, https://schnitzler-briefe.acdh.oeaw.ac.at/{\dateiname}.html (Stand \today)
\fi

\end{document}


