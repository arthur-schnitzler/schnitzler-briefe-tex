%% latex-leseansicht-vorspann.tex
%% Vorspann für die Leseansicht.
%% Lädt die gemeinsame Datei latex-vorspann.tex mit nicht gesetztem Schalter.

\newif\ifkorrekturansicht
\korrekturansichtfalse

\input{../tex-inputs/latex-vorspann}

\begin{center}
            \textcolor{red}{ENTWURF, NICHT FERTIG KORRIGIERT}
                      \end{center}
            
         
         \renewcommand{\erwaehntePersonen}{Personen: Richard Beer-Hofmann, Ludwig van Beethoven, Max Eugen Burckhard, Heinrich Friedjung, Paul Goldmann, Clementine Goldmann, Heinrich Graetz, Gerhart Hauptmann, Victor Hehn, Georg Hirschfeld, Heinrich Kanner, Max Klinger, Margarethe Manassewitsch, Else Markbreiter, Louise Schnitzler, Olga Schnitzler, Franz Servaes}
         \renewcommand{\erwaehnteInstitutionen}{Institutionen: Buchhandlung Gebrüder Bornträger, Buchhandlung L. Rosner, Burgtheater, Die Zeit, J.G. Cotta’sche Buchhandlung Nachfolger, Oskar Leiner GmbH {\kaufmannsund}  Co. KG, Volkstheater, Wiener Secession, Wiener Tierschutz-Verein}
         \renewcommand{\erwaehnteOrte}{Orte: Berlin, Dessauer Straße, Leipzig, Prag, Pörtschach, Stuttgart, Tierschutzhaus, Wien}
         \renewcommand{\erwaehnteWerke}{Werke: Beethoven, Der Kampf um die Vorherrschaft in Deutschland 1859 bis 1866. 2 Bde., Der Weg zum Licht. Ein Salzburger Märchendrama in vier Akten, Der einsame Weg. Schauspiel in fünf Akten, Der tapfere Cassian. Puppenspiel in einem Akt, Die »neue Richtung«. Polemische Aufsätze über Berliner Theater-Aufführungen, Francesca da Rimini, Gedanken über Goethe, Goethe und das Publikum. Eine Literaturgeschichte im Kleinen, Klinger’s »Beethoven«, Neue Freie Presse, Tagebuch, Volkstümliche Geschichte der Juden in drei Bänden}
               \section[ Paul Goldmann an Arthur Schnitzler, 17. 4. {[}1902{]}]{ Paul Goldmann an Arthur Schnitzler, 17. 4. {[}1902{]}}\nopagebreak\mylabel{v}\rehead{ }\begin{ledgroupsized}[t]{13cm}\normalsize\beginnumbering \toendnotes[C]{\smallbreak\pagebreak[2]} \Standort{DLA, A:Schnitzler, HS.NZ85.1.3172.}
\physDesc{Brief, 3 Blätter, 10 Seiten, 4352 Zeichen
\newline{}Handschrift: blaue Tinte, deutsche Kurrent
\newline{}Schnitzler: 1) mit Bleistift das Jahr »902« vermerkt  2) mit rotem Buntstift elf Unterstreichungen}\toendnotes[C]{\smallbreak}\pstart
           \noindent{}\raggedleft{}{\pb}\textcolor{gray}{\textbf{DESSAUERSTRASSE 19}}\oindex{Dessauer Strasse@\textbf{Dessauer Straße}|pw}\pend
           \pstart
           Berlin\oindex{Berlin@\textbf{Berlin}|pw}, 17. April.\pend
           \pstart\center{}Mein lieber Freund,\pend\pstart
           Seit dem Empfang Deines letzten lieben Briefes, der nach meiner \label{K_L03204-1v}\edtext{Rückkehr aus \textsc{Prag\oindex{Prag@\textbf{Prag}|pw}}}{\lemma{\textnormal{\emph{Rückkehr aus Prag}}}\Cendnote{\textnormal{siehe Paul Goldmann an Arthur Schnitzler, 1. 4. [1902]}}}\label{K_L03204-1h} eintraf, will ich Dir \uline{täglich} ſchreiben, und
               täglich muß ich darauf verzichten. Es iſt unbeſchreiblich, was jetzt wieder Alles an
               Arbeit, Beſuchen \textsc{etc.} auf mich einſtrömt. Ich bin Dir ſehr
               dankbar, daß Du meine Antwort nicht abgewartet und mich abermals heut durch Deine lieben Nachrichten erfreut haſt. Dieſer
                  \label{K_L03204-2v}\edtext{Bernhardiner}{\lemma{\textnormal{\emph{Bernhardiner}}}\Cendnote{\textnormal{Schnitzler\pwindex{Schnitzler, Arthur 15.05.1862 – 21.10.1931@\textsc{Schnitzler, Arthur} (15.05.1862 – 21.10.1931), \emph{Schriftsteller, Mediziner}|pwk} besaß für kurze Zeit, vermutlich
                  ab dem 23. 3. 1902,
                  einen Bernhardiner namens »Bern«. Im Oktober wurde er in dem im
                  gleichen Monat eröffneten Tierschutzhaus\oindex{Tierschutzhaus@\textbf{Tierschutzhaus}|pwk} des
                     \emph{Wiener Tierschutz-Vereins}\orgindex{Wiener Tierschutz-Verein@Wiener Tierschutz-Verein|pwk} behandelt;
                     Mitte Dezember erneut. Ab Januar 1903 versucht er ihn
                  zu vermitteln, da wohnt er aber bereits nicht mehr bei ihnen (siehe Arthur Schnitzler an Richard Beer-Hofmann, 14. 1. 1903 und Hermann Bahr an Arthur Schnitzler, 4. 4. [1903]). In diesem Jahr finden sich noch drei Erwähnungen
                  im \emph{Tagebuch}\pwindex{\textcolor{red}{\textsuperscript{XXXX1 indx}}!Tagebuch1981 – 2000@\strich\emph{Tagebuch} {[}Hrsg., 1981 – 2000{]}|pwk} (23. 5. 1903, 18. 6. 1903 und 6. 8. 1903). Vgl. \emph{Briefe} II,118. }}}\label{K_L03204-2h} muß herrlich ſein. Ich freue mich
               ſchon ſehr darauf, ihn kennen zu lernen. {\pb}Was Du
               über \textsc{Hirschfeld\pwindex{Hirschfeld, Georg 11.02.1873 – 17.01.1942@\textsc{Hirschfeld, Georg} (11.02.1873 – 17.01.1942), \emph{Schriftsteller}|pw}} ſchreibſt, iſt ſehr ſchön geſagt. Die Freunde und »literariſchen Kritiker«, die
               den unentwickelten Burſchen\pwindex{Hirschfeld, Georg 11.02.1873 – 17.01.1942@\textsc{Hirschfeld, Georg} (11.02.1873 – 17.01.1942), \emph{Schriftsteller}|pwv},
               deſſen Sentimentalität ſie für Poeſie nehmen, zum Dichter ausgeſchrieen haben, haben
               allerdings viel Schuld an dem jämmerlichen Ende, – aber doch nicht die einzige. Wer
               im Stande iſt, ein flaches Machwerk, wie den »Weg zum
                  Licht\pwindex{Hirschfeld, Georg 11.02.1873 – 17.01.1942@\textsc{Hirschfeld, Georg} (11.02.1873 – 17.01.1942), \emph{Schriftsteller}!Weg zum Licht. Ein Salzburger Maerchendrama in vier Akten1902-04-05@\strich\emph{Der Weg zum Licht. Ein Salzburger Märchendrama in vier Akten} {[}1902-04-05{]}|pw}« zu ſchreiben, in dem auch nicht die leiſeſte Spur von Perſönlichkeit
               ſteckt, der hat eben niemals eine Perſönlichkeit gehabt. Denn das iſt vollkommen
               ausgeſchloſſen, daß man aus einem Dichter {\pb}plötzlich
               ein Flachkopf wird. Der »Weg zum Licht\pwindex{Hirschfeld, Georg 11.02.1873 – 17.01.1942@\textsc{Hirschfeld, Georg} (11.02.1873 – 17.01.1942), \emph{Schriftsteller}!Weg zum Licht. Ein Salzburger Maerchendrama in vier Akten1902-04-05@\strich\emph{Der Weg zum Licht. Ein Salzburger Märchendrama in vier Akten} {[}1902-04-05{]}|pw}« iſt
               nicht verfehlt, ſondern complet talentlos. Das iſt ein Unterſchied.\pend
           \pstart
           \textsc{Servaes\pwindex{Servaes, Franz 17.06.1862 – 14.07.1947@\textsc{Servaes, Franz} (17.06.1862 – 14.07.1947), \emph{Journalist, Kritiker}|pw}}{ }\label{K_L03204-3v}\edtext{Feuilleton\pwindex{Servaes, Franz 17.06.1862 – 14.07.1947@\textsc{Servaes, Franz} (17.06.1862 – 14.07.1947), \emph{Journalist, Kritiker}!Klinger s »Beethoven«1902-04-16@\strich\emph{Klinger’s »Beethoven«} {[}1902-04-16{]}|pwv}}{\lemma{\textnormal{\emph{Feuilleton}}}\Cendnote{\textnormal{Franz Servaes\pwindex{Servaes, Franz 17.06.1862 – 14.07.1947@\textsc{Servaes, Franz} (17.06.1862 – 14.07.1947), \emph{Journalist, Kritiker}|pwk}: \emph{Klinger’s »Beethoven«}\pwindex{Servaes, Franz 17.06.1862 – 14.07.1947@\textsc{Servaes, Franz} (17.06.1862 – 14.07.1947), \emph{Journalist, Kritiker}!Klinger s »Beethoven«1902-04-16@\strich\emph{Klinger’s »Beethoven«} {[}1902-04-16{]}|pwk}. In: \emph{Neue Freie Presse}\pwindex{Neue Freie Presse1864 – 1939@\emph{Neue Freie Presse} {[}1864 – 1939{]}|pwk}, Nr. 13.521, 16. 4. 1902, Morgenblatt, S. 1–3. Servaes\pwindex{Servaes, Franz 17.06.1862 – 14.07.1947@\textsc{Servaes, Franz} (17.06.1862 – 14.07.1947), \emph{Journalist, Kritiker}|pwk}’ Urteil fiel sehr gut aus.}}}\label{K_L03204-3h} über \textsc{Klinger\pwindex{Klinger, Max 18.02.1857 – 04.07.1920@\textsc{Klinger, Max} (18.02.1857 – 04.07.1920), \emph{Maler, Radierer, Bildhauer}!Beethoven1902@\strich\emph{Beethoven} {[}1902{]}|pwv}\pwindex{Klinger, Max 18.02.1857 – 04.07.1920@\textsc{Klinger, Max} (18.02.1857 – 04.07.1920), \emph{Maler, Radierer, Bildhauer}|pw}}, \strikeout{hat} das ich eben geleſen, hat mir ſehr gut
               gefallen. Aber iſt auch das Urtheil richtig? Oder iſt wieder ein \label{K_L03204-4v}\edtext{Seceſſion\orgindex{Wiener Secession@Wiener Secession|pw}s-Schwindel}{\lemma{\textnormal{\emph{Seceſſions-Schwindel}}}\Cendnote{\textnormal{Max Klinger\pwindex{Klinger, Max 18.02.1857 – 04.07.1920@\textsc{Klinger, Max} (18.02.1857 – 04.07.1920), \emph{Maler, Radierer, Bildhauer}|pwk}s \emph{Beethovenstatue}\pwindex{Klinger, Max 18.02.1857 – 04.07.1920@\textsc{Klinger, Max} (18.02.1857 – 04.07.1920), \emph{Maler, Radierer, Bildhauer}!Beethoven1902@\strich\emph{Beethoven} {[}1902{]}|pwk} stand im Mittelpunkt der 14. Ausstellung
                  der \emph{Wiener Secession}\orgindex{Wiener Secession@Wiener Secession|pwk}, die Beethoven\pwindex{Beethoven, Ludwig van 17.12.1770 – 26.03.1827@\textsc{Beethoven, Ludwig van} (17.12.1770 – 26.03.1827), \emph{Komponist}|pwk} gewidmet war und von 15. 4. 1902 bis 15. 6. 1902
                  stattfand.}}}\label{K_L03204-4h} dabei? Ich kann es mir allerdings kaum denken; ich ahne etwas
               Großes, wenn \textsc{Klinger\pwindex{Klinger, Max 18.02.1857 – 04.07.1920@\textsc{Klinger, Max} (18.02.1857 – 04.07.1920), \emph{Maler, Radierer, Bildhauer}|pw}} einen \textsc{Beethoven\pwindex{Beethoven, Ludwig van 17.12.1770 – 26.03.1827@\textsc{Beethoven, Ludwig van} (17.12.1770 – 26.03.1827), \emph{Komponist}|pw}\pwindex{Klinger, Max 18.02.1857 – 04.07.1920@\textsc{Klinger, Max} (18.02.1857 – 04.07.1920), \emph{Maler, Radierer, Bildhauer}!Beethoven1902@\strich\emph{Beethoven} {[}1902{]}|pwv}} gemacht hat.\pend
           \pstart
           Ich habe die Idee, etwa zehn meiner Theater-Feuilletons, die ſich mit \textsc{Hauptmann\pwindex{Hauptmann, Gerhart 15.11.1862 – 06.06.1946@\textsc{Hauptmann, Gerhart} (15.11.1862 – 06.06.1946), \emph{Schriftsteller}|pw}} und ſeinen Anhängern beſchäftigen, {\pb}zu ſammeln
               und als Kampf-Buch unter dem ironiſchen Titel \label{K_L03204-5v}\edtext{»Die neue
                  Dichtung\pwindex{Goldmann, Paul 31.01.1865 – 25.09.1935@\textsc{Goldmann, Paul} (31.01.1865 – 25.09.1935), \emph{Schriftsteller, Journalist}!»neue Richtung«. Polemische Aufsaetze ueber Berliner Theater-Auffuehrungen1902-10-17@\strich\emph{Die »neue Richtung«. Polemische Aufsätze über Berliner Theater-Aufführungen} {[}1902-10-17{]}|pwv}«}{\lemma{\textnormal{\emph{»Die neue
                  Dichtung«}}}\Cendnote{\textnormal{Paul Goldmann\pwindex{Goldmann, Paul 31.01.1865 – 25.09.1935@\textsc{Goldmann, Paul} (31.01.1865 – 25.09.1935), \emph{Schriftsteller, Journalist}|pwk}: \emph{Die »neue Richtung«. Polemische Aufsätze über Berliner
                        Theater-Aufführungen}\pwindex{Goldmann, Paul 31.01.1865 – 25.09.1935@\textsc{Goldmann, Paul} (31.01.1865 – 25.09.1935), \emph{Schriftsteller, Journalist}!»neue Richtung«. Polemische Aufsaetze ueber Berliner Theater-Auffuehrungen1902-10-17@\strich\emph{Die »neue Richtung«. Polemische Aufsätze über Berliner Theater-Aufführungen} {[}1902-10-17{]}|pwk}. Wien\oindex{Wien@\textbf{Wien}|pwk}: \emph{C. W. Stern (Buchhandlung L. Rosner)}\orgindex{Buchhandlung L. Rosner@Buchhandlung L. Rosner|pwk},
                     erschienen im Oktober 1902, vordatiert auf 1903. Der Umfang ist mit 19 Texten größer als hier noch angedacht, wobei vier
                  Feuilletons zu Stücken Hauptmann\pwindex{Hauptmann, Gerhart 15.11.1862 – 06.06.1946@\textsc{Hauptmann, Gerhart} (15.11.1862 – 06.06.1946), \emph{Schriftsteller}|pwk}s das Buch
                  eröffnen und dominieren.}}}\label{K_L03204-5h} herauszugeben. Glaubſt Du, daß ein ſolches Buch
               Leſer finden würde? Oder hängen Theater-Feuilletons nicht doch zu ſehr mit dem Tage
               zuſammen, als daß ſie in ein Buch hineingehörten? Die Idee kam mir, da ich neulich
               wieder hörte, wie ſehr die \textsc{Hauptmann\pwindex{Hauptmann, Gerhart 15.11.1862 – 06.06.1946@\textsc{Hauptmann, Gerhart} (15.11.1862 – 06.06.1946), \emph{Schriftsteller}|pw}-Clique} hier mich haßt. Man
               hat einer Dame Vorwürfe gemacht, daß ſie im Theater freundlich mit mir geſprochen
               hat! Wenn ich ſehe, daß man mit ſolchen Mitteln eine künſtleriſche Überzeugung {\pb}bekämpfen will, ſo habe ich den Drang, meine
               Überzeugung nur umſo ſtärker zu betonen.\pend
           \pstart
           Was Du mir vom \label{K_L03204-6v}\edtext{Tode der armen \textsc{Elsa Marktbreiter\pwindex{Markbreiter, Else 14.09.1870 – 30.03.1902@\textsc{Markbreiter, Else} (14.09.1870 – 30.03.1902)|pw}}}{\lemma{\textnormal{\emph{Tode … Marktbreiter}}}\Cendnote{\textnormal{Schnitzler\pwindex{Schnitzler, Arthur 15.05.1862 – 21.10.1931@\textsc{Schnitzler, Arthur} (15.05.1862 – 21.10.1931), \emph{Schriftsteller, Mediziner}|pwk}s Cousine Else Markbreiter\pwindex{Markbreiter, Else 14.09.1870 – 30.03.1902@\textsc{Markbreiter, Else} (14.09.1870 – 30.03.1902)|pwk} war am 30. 3. 1902 an Tuberkulose verstorben. Siehe A. S.: \emph{Tagebuch}, 31. 3. 1902.}}}\label{K_L03204-6h} ſchreibſt, iſt ergreifend. Aber
                  \strikeout{was} war es nicht eine Erlöſung? Freilich, das iſt
               auch eine dumme Phraſe. Erlöſt iſt man doch nur, wenn man \uline{weiß}, daß man erlöſt iſt.\pend
           \pstart
           Ich habe Deiner Frau Mutter\pwindex{Schnitzler, Louise 1840-07-08 – 1911-09-09@\textsc{Schnitzler, Louise} (1840-07-08 – 1911-09-09)|pwv}
               nicht kondolirt, weil ich nicht weiß, ob die Verwandtſchaft nahe genug war, um eine
               Condolenz zu rechtfertigen. Wenn ja, ſo {\pb}kondolire,
               bitte, in meinem Namen.\pend
           \pstart
           Und dieſe arme hübſche \label{K_L03204-7v}\edtext{\textsc{Grethl Mandl\pwindex{Manassewitsch, Margarethe 6.11.1880 – 21.09.1940@\textsc{Manassewitsch, Margarethe} (6.11.1880 – 21.09.1940)|pw}}}{\lemma{\textnormal{\emph{Grethl Mandl}}}\Cendnote{\textnormal{Margarethe Mandl\pwindex{Manassewitsch, Margarethe 6.11.1880 – 21.09.1940@\textsc{Manassewitsch, Margarethe} (6.11.1880 – 21.09.1940)|pwk}, ebenso eine Cousine Schnitzler\pwindex{Schnitzler, Arthur 15.05.1862 – 21.10.1931@\textsc{Schnitzler, Arthur} (15.05.1862 – 21.10.1931), \emph{Schriftsteller, Mediziner}|pwk}s, war, wie er vermutete, an
                  Neuritis erkrankt (vgl. A. S.: \emph{Tagebuch}, 13. 3. 1902), einer Nervenentzündung mit Lähmungserscheinungen. Gestorben
                  ist sie daran nicht.}}}\label{K_L03204-7h}! Wie, um Himmels Willen, iſt das ſo plötzlich
               gekommen? Sie hat mir in \label{K_L03204-8v}\edtext{\textsc{Pörtschach\oindex{Poertschach@\textbf{Pörtschach}|pw}}}{\lemma{\textnormal{\emph{Pörtschach}}}\Cendnote{\textnormal{vermutlich im Sommer 1901}}}\label{K_L03204-8h} noch ſo gut gefallen. Iſt Ausſicht auf Heilung vorhanden?\pend
           \pstart
           Haſt Du zu \label{K_L03204-9v}\edtext{arbeiten}{\lemma{\textnormal{\emph{arbeiten}}}\Cendnote{\textnormal{Schnitzler\pwindex{Schnitzler, Arthur 15.05.1862 – 21.10.1931@\textsc{Schnitzler, Arthur} (15.05.1862 – 21.10.1931), \emph{Schriftsteller, Mediziner}|pwk} hatte am 6. 4. 1902 das
                  einaktige Puppenspiel \emph{Der tapfere Cassian}\pwindex{Schnitzler, Arthur 15.05.1862 – 21.10.1931@\textsc{Schnitzler, Arthur} (15.05.1862 – 21.10.1931), \emph{Schriftsteller, Mediziner}!tapfere Cassian. Puppenspiel in einem Akt01. 02. 1904@\strich\emph{Der tapfere Cassian. Puppenspiel in einem Akt} {[}01. 02. 1904{]}|pwk}
                  begonnen. Ebenso hatte er Überlegungen zu seinem Schauspiel \emph{Der einsame Weg}\pwindex{Schnitzler, Arthur 15.05.1862 – 21.10.1931@\textsc{Schnitzler, Arthur} (15.05.1862 – 21.10.1931), \emph{Schriftsteller, Mediziner}!einsame Weg. Schauspiel in fuenf Akten1904@\strich\emph{Der einsame Weg. Schauspiel in fünf Akten} {[}1904{]}|pwk} angestellt (vgl. A. S.: \emph{Tagebuch}, 8. 4. 1902). Hinsichtlich
                     Goldmann\pwindex{Goldmann, Paul 31.01.1865 – 25.09.1935@\textsc{Goldmann, Paul} (31.01.1865 – 25.09.1935), \emph{Schriftsteller, Journalist}|pwk}s wiederholter Forderung, Schnitzler\pwindex{Schnitzler, Arthur 15.05.1862 – 21.10.1931@\textsc{Schnitzler, Arthur} (15.05.1862 – 21.10.1931), \emph{Schriftsteller, Mediziner}|pwk} solle eine Lustspiel schreiben,
                     siehe Paul Goldmann an Arthur Schnitzler, 2. 5. [1900].}}}\label{K_L03204-9h}
               angefangen? Denkſt Du an das Luſtſpiel? Ich weiß, Du wirſt über dieſe meine Frage
               wieder ſehr aufgebracht ſein, aber Du mußt mich ſchon entſchuldigen, wenn ich unſeren
               einzigen \strikeout{D} Dramatiker, der \strikeout{\textcolor{gray}{h}\textcolor{gray}{×}\-\textcolor{gray}{×}\-\textcolor{gray}{×}\-\textcolor{gray}{×}\-\textcolor{gray}{×}} Humor hat, hier und da danach frage, {\pb}ob er
               nicht ein Luſtſpiel ſchreiben möchte? Du wirſt wieder ſagen: »Es fällt \substVorne{}\textsuperscript{Dir}\substDazwischen{}mir\substHinten{} nichts ein.« \strikeout{Aben} Aber das \strikeout{Schreiben} Schreiben wäre ſehr einfach, wenn wir nur das
               zu ſchreiben brauchten, was uns \strikeout{einfiele}
               einfällt.\pend
           \pstart
           Wie geht es \textsc{Olga\pwindex{Schnitzler, Olga 17.01.1882 – 13.01.1970@\textsc{Schnitzler, Olga} (17.01.1882 – 13.01.1970), \emph{Schauspielerin, Sängerin}|pw}}? Grüße ſie herzlichſt von mir. Ich ſchreibe ihr nächſtens – jawohl, ganz gewiß,
               nächſtens!\pend
           \pstart
           Lies’ \textsc{Hehn\pwindex{Hehn, Victor 1813-10-08 – 1890-03-21@\textsc{Hehn, Victor} (1813-10-08 – 1890-03-21), \emph{Schriftsteller, Historiker, Bibliothekar}|pw}}: Gedanken über \textsc{Goethe}\pwindex{Hehn, Victor 1813-10-08 – 1890-03-21@\textsc{Hehn, Victor} (1813-10-08 – 1890-03-21), \emph{Schriftsteller, Historiker, Bibliothekar}!Gedanken ueber Goethe1887@\strich\emph{Gedanken über Goethe} {[}1887{]}|pw}, namentlich den Aufſatz \label{K_L03204-10v}\edtext{\textsc{Goethe} und das Publikum\pwindex{Hehn, Victor 1813-10-08 – 1890-03-21@\textsc{Hehn, Victor} (1813-10-08 – 1890-03-21), \emph{Schriftsteller, Historiker, Bibliothekar}!Goethe und das Publikum. Eine Literaturgeschichte im Kleinen1887@\strich\emph{Goethe und das Publikum. Eine Literaturgeschichte im Kleinen} {[}1887{]}|pw}}{\lemma{\textnormal{\emph{Goethe und das Publikum}}}\Cendnote{\textnormal{Viktor Hehn\pwindex{Hehn, Victor 1813-10-08 – 1890-03-21@\textsc{Hehn, Victor} (1813-10-08 – 1890-03-21), \emph{Schriftsteller, Historiker, Bibliothekar}|pwk}: \emph{Goethe und das Publikum. Eine Literaturgeschichte im
                        Kleinen}\pwindex{Hehn, Victor 1813-10-08 – 1890-03-21@\textsc{Hehn, Victor} (1813-10-08 – 1890-03-21), \emph{Schriftsteller, Historiker, Bibliothekar}!Goethe und das Publikum. Eine Literaturgeschichte im Kleinen1887@\strich\emph{Goethe und das Publikum. Eine Literaturgeschichte im Kleinen} {[}1887{]}|pwk}. In: \emph{Gedanken über
                     Goethe}\pwindex{Hehn, Victor 1813-10-08 – 1890-03-21@\textsc{Hehn, Victor} (1813-10-08 – 1890-03-21), \emph{Schriftsteller, Historiker, Bibliothekar}!Gedanken ueber Goethe1887@\strich\emph{Gedanken über Goethe} {[}1887{]}|pwk}. Berlin\oindex{Berlin@\textbf{Berlin}|pwk}: \emph{Gebrüder Borntraeger}\orgindex{Buchhandlung Gebrueder Borntraeger@Buchhandlung Gebrüder Bornträger|pwk}{ }1887, S. 49–185.}}}\label{K_L03204-10h}. Eine Fülle
               intereſſanten Materials in einem wundervoll klaren {\pb}Styl mitgetheilt. Der einzige Fehler iſt\strikeout{\textcolor{gray}{,}} ein irrſinniger Antiſemitismus.\pend
           \pstart
           \textsc{Kanner\pwindex{Kanner, Heinrich 09.11.1864 – 15.02.1930@\textsc{Kanner, Heinrich} (09.11.1864 – 15.02.1930), \emph{Herausgeber, Publizist}|pw}} war hier\oindex{Berlin@\textbf{Berlin}|pwv}. Ich ſoll
                  \label{K_L03204-11v}\edtext{zur »Zeit\orgindex{Zeit@Die Zeit|pw}« als Feuilleton-Redakteur}{\lemma{\textnormal{\emph{zur … Feuilleton-Redakteur}}}\Cendnote{\textnormal{Heinrich Kanner\pwindex{Kanner, Heinrich 09.11.1864 – 15.02.1930@\textsc{Kanner, Heinrich} (09.11.1864 – 15.02.1930), \emph{Herausgeber, Publizist}|pwk} dürfte seine Meinung also
                  geändert haben, siehe Paul Goldmann an Arthur Schnitzler, 25. 1. [1902].}}}\label{K_L03204-11h} kommen\substVorne{}\textsuperscript{,}\substDazwischen{}.\substHinten{}{ }Burgtheater\orgindex{Burgtheater@Burgtheater|pw} und Volkstheater\orgindex{Volkstheater@Volkstheater|pw} ſind allerdings ſchon an \textsc{Burckhardt\pwindex{Burckhard, Max Eugen 14.07.1854 – 16.03.1912@\textsc{Burckhard, Max Eugen} (14.07.1854 – 16.03.1912), \emph{Schriftsteller, Rechtswissenschaftler, Theaterleiter}|pw}} vergeben. Ich ſollte alſo nur Redaktions\substVorne{}\textsuperscript{-\textcolor{gray}{Kuli}}\substDazwischen{}-\textsc{Kuli}\substHinten{} ſein und eine rieſige Büreauarbeit leiſten: Kleines und großes Feuilleton,
               eine Sonntagsbeilage \textsc{etc}. Ich glaube nicht, daß ich unter
               dieſen Umſtänden annehmen werde, – umſomehr als meine Mutter\pwindex{Goldmann, Clementine 1842-05-15 – 1924-02-24@\textsc{Goldmann, Clementine} (1842-05-15 – 1924-02-24)|pwv} nicht nach Wien\oindex{Wien@\textbf{Wien}|pw}{ }{\pb}mitkommen würde\substVorne{}\textsuperscript{.}\substDazwischen{}{ }und ich meinen Hausſtand auflöſen
                     müßte.\substHinten{} Ja, wenn ich verheirathet wäre, ſo wäre das Alles anders. Haſt Du noch immer
               keine Parthie für mich?\pend
           \pstart
           \label{K_L03204-12v}\edtext{\textsc{Friedjung\pwindex{Friedjung, Heinrich 18.01.1851 – 14.07.1920@\textsc{Friedjung, Heinrich} (18.01.1851 – 14.07.1920), \emph{Historiker}|pw}s}{ }Buch\pwindex{Friedjung, Heinrich 18.01.1851 – 14.07.1920@\textsc{Friedjung, Heinrich} (18.01.1851 – 14.07.1920), \emph{Historiker}!Kampf um die Vorherrschaft in Deutschland 1859 bis 1866. 2 Bde.1897 – 1898@\strich\emph{Der Kampf um die Vorherrschaft in Deutschland 1859 bis 1866. 2 Bde.} {[}1897 – 1898{]}|pwv}}{\lemma{\textnormal{\emph{Friedjungs Buch}}}\Cendnote{\textnormal{Heinrich Friedjung\pwindex{Friedjung, Heinrich 18.01.1851 – 14.07.1920@\textsc{Friedjung, Heinrich} (18.01.1851 – 14.07.1920), \emph{Historiker}|pwk}: \emph{Der Kampf um die Vorherrschaft in Deutschland 1859 bis 1866. 2
                     Bde}\pwindex{Friedjung, Heinrich 18.01.1851 – 14.07.1920@\textsc{Friedjung, Heinrich} (18.01.1851 – 14.07.1920), \emph{Historiker}!Kampf um die Vorherrschaft in Deutschland 1859 bis 1866. 2 Bde.1897 – 1898@\strich\emph{Der Kampf um die Vorherrschaft in Deutschland 1859 bis 1866. 2 Bde.} {[}1897 – 1898{]}|pwk}. Stuttgart\oindex{Stuttgart@\textbf{Stuttgart}|pwk}: \emph{Cotta}\orgindex{J.G. Cotta sche Buchhandlung Nachfolger@J.G. Cotta’sche Buchhandlung Nachfolger|pwk}{ }1897–1898. Schnitzler\pwindex{Schnitzler, Arthur 15.05.1862 – 21.10.1931@\textsc{Schnitzler, Arthur} (15.05.1862 – 21.10.1931), \emph{Schriftsteller, Mediziner}|pwk} las das Buch\pwindex{Friedjung, Heinrich 18.01.1851 – 14.07.1920@\textsc{Friedjung, Heinrich} (18.01.1851 – 14.07.1920), \emph{Historiker}!Kampf um die Vorherrschaft in Deutschland 1859 bis 1866. 2 Bde.1897 – 1898@\strich\emph{Der Kampf um die Vorherrschaft in Deutschland 1859 bis 1866. 2 Bde.} {[}1897 – 1898{]}|pwkv} am 22. 3. 1902.}}}\label{K_L03204-12h} werde ich leſen. Jetzt ſtecke
               ich in \textsc{Grätz\pwindex{Graetz, Heinrich 1817-10-31 – 1891-09-07@\textsc{Graetz, Heinrich} (1817-10-31 – 1891-09-07), \emph{Historiker}|pw}}{ }\label{K_L03204-13v}\edtext{»Geſchichte der Juden\pwindex{Graetz, Heinrich 1817-10-31 – 1891-09-07@\textsc{Graetz, Heinrich} (1817-10-31 – 1891-09-07), \emph{Historiker}!Volkstuemliche Geschichte der Juden in drei Baenden1888@\strich\emph{Volkstümliche Geschichte der Juden in drei Bänden} {[}1888{]}|pw}« (Volksausgabe in drei Bänden)}{\lemma{\textnormal{\emph{»Geſchichte … Bänden)}}}\Cendnote{\textnormal{H. [ = Heinrich] Graetz\pwindex{Graetz, Heinrich 1817-10-31 – 1891-09-07@\textsc{Graetz, Heinrich} (1817-10-31 – 1891-09-07), \emph{Historiker}|pwk}: \emph{Volkstümliche Geschichte der Juden in drei
                        Bänden}\pwindex{Graetz, Heinrich 1817-10-31 – 1891-09-07@\textsc{Graetz, Heinrich} (1817-10-31 – 1891-09-07), \emph{Historiker}!Volkstuemliche Geschichte der Juden in drei Baenden1888@\strich\emph{Volkstümliche Geschichte der Juden in drei Bänden} {[}1888{]}|pwk}. Leipzig\oindex{Leipzig@\textbf{Leipzig}|pwk}: \emph{Oskar Leiner}\orgindex{Oskar Leiner GmbH und Co. KG@Oskar Leiner GmbH {\kaufmannsund}  Co. KG|pwk}{ }1888. Eine Lektüre durch Schnitzler\pwindex{Schnitzler, Arthur 15.05.1862 – 21.10.1931@\textsc{Schnitzler, Arthur} (15.05.1862 – 21.10.1931), \emph{Schriftsteller, Mediziner}|pwk} ist
                  nicht nachweisbar.}}}\label{K_L03204-13h}. Ein tauſendfach anregendes Buch\pwindex{Graetz, Heinrich 1817-10-31 – 1891-09-07@\textsc{Graetz, Heinrich} (1817-10-31 – 1891-09-07), \emph{Historiker}!Volkstuemliche Geschichte der Juden in drei Baenden1888@\strich\emph{Volkstümliche Geschichte der Juden in drei Bänden} {[}1888{]}|pwv}. Mußt Du leſen. \label{K_L03204-14v}\edtext{»\textsc{Francesca da Rimini\pwindex{\textcolor{red}{\textsuperscript{XXXX1 indx}}!Francesca da Rimini1901@\strich\emph{Francesca da Rimini} {[}1901{]}|pw}}«}{\lemma{\textnormal{\emph{»Francesca da Rimini«}}}\Cendnote{\textnormal{siehe A. S.: \emph{Tagebuch}, 2. 4. 1902}}}\label{K_L03204-14h} hat mich bodenlos gelangweilt.\pend
           \pstart
           Schreib’ mir bald wieder, {\pb}mein lieber Freund, und
               ſei vielmals und von Herzen gegrüßt von {\\}Deinem {\\}\spacefill\mbox{Paul Goldmn}\pend
           \pstart
           \noindent{}Was macht \textsc{Richard\pwindex{Beer-Hofmann, Richard 1866-07-11 – 1945-09-26@\textsc{Beer-Hofmann, Richard} (1866-07-11 – 1945-09-26), \emph{Schriftsteller}|pw}}?\pend
           
         
         \endnumbering\mylabel{h}\end{ledgroupsized}\begin{anhang}\end{anhang}\newcommand{\dateiname}{L03204}\newcommand{\titel}{Paul Goldmann an Arthur Schnitzler, 17. 4. [1902]}\newcommand{\editorInnen}{Martin Anton Müller und Laura Untner}%% latex-leseansicht-abspann.tex
%% Abspann für die Leseansicht.
%% Der Schalter \ifkorrekturansicht ist bereits durch den Vorspann gesetzt.

%% latex-abspann.tex
%% Gemeinsamer Abspann für Korrekturansicht und Leseansicht.
%% Setzt den Schalter \ifkorrekturansicht voraus (gesetzt in den
%% einbindenden Dateien latex-korrekturansicht-abspann.tex bzw.
%% latex-leseansicht-abspann.tex).
%% ---------------------------------------------------------------

\normalsize

% Das esempio-Environment wird nur in der Leseansicht benötigt
\ifkorrekturansicht\else
\newenvironment{esempio}[3]%
{
    \vspace{1.5ex}
    \rlap{\underline{#1}}
    \par
    \setlength{\parindent}{0cm}
    \nopagebreak
    \leftskip=#2cm
    \rightskip=#3cm
}
{
    \par
}
\fi

\doendnotes{C}
\bigskip
\vfill

\clearpage

\footnotesize

\ifkorrekturansicht
  \lohead{\textsc{register}}
\fi

% theindex-Environment neu definieren ohne reledmac
\makeatletter
\renewenvironment{theindex}{%
  \ifkorrekturansicht
    \section*{\indexname}%
  \else
    \subsubsection*{Index der erwähnten Entitäten}%
  \fi
  \setlength{\parindent}{0pt}%
  \setlength{\parskip}{0pt plus 0.3pt}%
  \let\item\@idxitem
}{%
  \ifkorrekturansicht\clearpage\fi
}
\makeatother

\IfFileExists{\jobname-pw.ind}{\input{\jobname-pw.ind}}{}

% Quellenangabe nur in der Leseansicht
\ifkorrekturansicht\else
% Fallback-Definitionen, falls die .tex-Datei \titel etc. nicht gesetzt hat
\providecommand{\titel}{}
\providecommand{\editorInnen}{}
\providecommand{\dateiname}{\jobname}

\vspace{3cm}

\vfill

\footnotesize
\textsc{Quelle}: \titel. Herausgegeben von {\editorInnen}. In: \emph{Arthur Schnitzler: Briefwechsel mit Autorinnen und Autoren}.
 Digitale Edition, https://schnitzler-briefe.acdh.oeaw.ac.at/{\dateiname}.html (Stand \today)
\fi

\end{document}


      