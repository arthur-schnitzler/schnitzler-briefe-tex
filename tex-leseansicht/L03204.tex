%% latex-leseansicht-vorspann.tex
%% Vorspann für die Leseansicht.
%% Lädt die gemeinsame Datei latex-vorspann.tex mit nicht gesetztem Schalter.

\newif\ifkorrekturansicht
\korrekturansichtfalse

\input{../tex-inputs/latex-vorspann}


\section[ Paul Goldmann an Arthur Schnitzler, 17. 4. [1902]]{L03204 Paul Goldmann an Arthur Schnitzler,  17. 4. [1902]}
\nopagebreak\mylabel{L03204v}
\rehead{ }\normalsize\beginnumbering\briefempfaengerindex{Schnitzler, Arthur@\textsc{Schnitzler, Arthur}!zzzGoldmann, Paul@\emph{von Paul Goldmann}!1902-04-171@{17. 4. [1902]}|(be}
\toendnotes[C]{\smallbreak\pagebreak[2]}
\correspDesc{Versand  durch Paul Goldmann am 17. 4. [1902] in Berlin
\newline{}Erhalt  durch Arthur Schnitzler im Zeitraum [18. 4. 1902
                  – 22. 4. 1902?] in Wien}\toendnotes[C]{\smallbreak}
\Standort{DLA, A:Schnitzler, HS.NZ85.1.3172.}
\physDesc{Brief, 3 Blätter, 10 Seiten, 4355 Zeichen
\newline{}Handschrift: blaue Tinte, deutsche Kurrent
\newline{}Schnitzler: 1) mit Bleistift das Jahr »902« vermerkt  2) mit rotem Buntstift elf Unterstreichungen}\toendnotes[C]{\smallbreak}
\pstart
           \raggedleft{}{\pb}\textcolor{gray}{\textbf{DESSAUERSTRASSE 19}}\oindex{Dessauer Straße@\textbf{Dessauer Straße}, \emph{Straße}|pw}\pend
           
\pstart
           Berlin\oindex{Berlin@\textbf{Berlin}, \emph{Hauptstadt}|pw}, 17. April.\pend
           
\pstart{}Mein lieber Freund,\pend\vspace{0.5em}
\pstart
           Seit dem Empfang Deines letzten lieben Briefes, de\substVorne{}\textsuperscript{\textcolor{gray}{n}}\substDazwischen{}r\substHinten{} nach meiner \label{K_L03204-1v}\edtext{Rückkehr aus \textsc{Prag\oindex{Prag@\textbf{Prag}, \emph{Land}|pw}}}{\lemma{\textnormal{\emph{Rückkehr aus Prag}}}\Cendnote{\textnormal{Siehe XXXX Auszeichnungsfehler: Dokument L03203 nicht gefunden. }}}\label{K_L03204-1} eintraf,
               will ich Dir \uline{täglich}{ }ſchreiben, und täglich muß ich
               darauf verzichten. Es iſt unbeſchreiblich, was jetzt wieder Alles an Arbeit, Beſuchen
                  \textsc{etc.} auf mich einſtürmt. Ich bin Dir{ }ſehr dankbar, daß
               Du meine Antwort nicht abgewartet und mich abermals heut durch Deine lieben Nachrichten erfreut haſt. Dieſer \label{K_L03204-2v}\edtext{Bernhardiner}{\lemma{\textnormal{\emph{Bernhardiner}}}\Cendnote{\textnormal{Schnitzler besaß für kurze Zeit, vermutlich
                  ab dem 23. 3. 1902,
                  einen Bernhardiner namens Bern. Im Oktober wurde er in
                  dem im gleichen Monat eröffneten Tierschutzhaus\oindex{Wien@\textbf{Wien}!XVI., Ottakring@\textbf{XVI., Ottakring}!Tierschutzhaus@\textbf{Tierschutzhaus}, \emph{Gebäude}|pwk} des \emph{Wiener
                     Tierschutz-Vereins}\orgindex{Wiener Tierschutz-Verein@Wiener Tierschutz-Verein|pwk} behandelt, Mitte Dezember erneut. Ab Januar 1903 versuchte
                     Schnitzler ihn zu vermitteln. Zu diesem
                  Zeitpunkt lebte er aber bereits nicht mehr bei ihnen (vgl. XXXX Auszeichnungsfehler: Dokument L01265 nicht gefunden und XXXX Auszeichnungsfehler: Dokument L01286 nicht gefunden). In diesem Jahr finden sich noch drei Erwähnungen
                  im \emph{Tagebuch}\pwindex{Schnitzler, Arthur 15. 5. 1862 Wien – 21. 10. 1931 ebd.@\textsc{Schnitzler, Arthur} (15. 5. 1862 Wien – 21. 10. 1931 ebd.), \emph{Schriftsteller, Mediziner}!Tagebuch@\strich\emph{Tagebuch}|pwk}: 23. 5. 1903, 18. 6. 1903 und 6. 8. 1903. Siehe auch \emph{Briefe 1913–1931}, S. 118.}}}\label{K_L03204-2} muß herrlich{ }ſein. Ich
               freue mich{ }ſchon{ }ſehr darauf, ihn kennen zu lernen. {\pb}Was Du über \textsc{Hirschfeld\pwindex{Hirschfeld, Georg 11.\,2.\,1873 Berlin – 17.\,1.\,1942 München@\textsc{Hirschfeld, Georg} (11.\,2.\,1873 Berlin – 17.\,1.\,1942 München), \emph{Schriftsteller}|pw}}{ }ſchreibſt, iſt{ }ſehr{ }ſchön geſagt. Die Freunde und »literariſchen Kritiker«, die
               den unentwickelten Burſchen\pwindex{Hirschfeld, Georg 11.\,2.\,1873 Berlin – 17.\,1.\,1942 München@\textsc{Hirschfeld, Georg} (11.\,2.\,1873 Berlin – 17.\,1.\,1942 München), \emph{Schriftsteller}|pwv},
               deſſen Sentimentalität{ }ſie für Poeſie nehmen, zum Dichter ausgeſchrieen haben, haben
               allerdings viel Schuld an dem jämmerlichen Ende, – aber doch nicht die einzige. Wer
               im Stande iſt, ein flaches Machwerk, wie den »Weg zum
                  Licht\pwindex{Hirschfeld, Georg 11.\,2.\,1873 Berlin – 17.\,1.\,1942 München@\textsc{Hirschfeld, Georg} (11.\,2.\,1873 Berlin – 17.\,1.\,1942 München), \emph{Schriftsteller}!Weg zum Licht. Ein Salzburger Märchendrama in vier Akten@\strich\emph{Der Weg zum Licht. Ein Salzburger Märchendrama in vier Akten}|pw}« zu{ }ſchreiben, in dem auch nicht die leiſeſte Spur von Perſönlichkeit{ }ſteckt, der hat eben niemals eine Perſönlichkeit gehabt. Denn das iſt vollkommen
               ausgeſchloſſen, daß man aus einem Dichter {\pb}plötzlich
               ein Flachkopf wird. Der »Weg zum Licht\pwindex{Hirschfeld, Georg 11.\,2.\,1873 Berlin – 17.\,1.\,1942 München@\textsc{Hirschfeld, Georg} (11.\,2.\,1873 Berlin – 17.\,1.\,1942 München), \emph{Schriftsteller}!Weg zum Licht. Ein Salzburger Märchendrama in vier Akten@\strich\emph{Der Weg zum Licht. Ein Salzburger Märchendrama in vier Akten}|pw}« iſt
               nicht verfehlt,{ }ſondern complet talentlos. Das iſt ein Unterſchied.\pend
           
\pstart
           \textsc{Servaes\pwindex{Servaes, Franz 17.\,6.\,1862 Köln – 14.\,7.\,1947 Wien@\textsc{Servaes, Franz} (17.\,6.\,1862 Köln – 14.\,7.\,1947 Wien), \emph{Journalist, Kritiker}|pw}}{ }\label{K_L03204-3v}\edtext{Feuilleton\pwindex{Servaes, Franz 17.\,6.\,1862 Köln – 14.\,7.\,1947 Wien@\textsc{Servaes, Franz} (17.\,6.\,1862 Köln – 14.\,7.\,1947 Wien), \emph{Journalist, Kritiker}!Klinger’s »Beethoven«@\strich\emph{Klinger’s »Beethoven«}|pwv}}{\lemma{\textnormal{\emph{Feuilleton}}}\Cendnote{\textnormal{Franz Servaes\pwindex{Servaes, Franz 17.\,6.\,1862 Köln – 14.\,7.\,1947 Wien@\textsc{Servaes, Franz} (17.\,6.\,1862 Köln – 14.\,7.\,1947 Wien), \emph{Journalist, Kritiker}|pwk}: \emph{Klinger’s »Beethoven«}\pwindex{Servaes, Franz 17.\,6.\,1862 Köln – 14.\,7.\,1947 Wien@\textsc{Servaes, Franz} (17.\,6.\,1862 Köln – 14.\,7.\,1947 Wien), \emph{Journalist, Kritiker}!Klinger’s »Beethoven«@\strich\emph{Klinger’s »Beethoven«}|pwk}. In: \emph{Neue Freie Presse}\pwindex{Neue Freie Presse@\emph{Neue Freie Presse}|pwk}, Nr. 13.521, 16. 4. 1902, Morgenblatt, S. 1–3. Servaes\pwindex{Servaes, Franz 17.\,6.\,1862 Köln – 14.\,7.\,1947 Wien@\textsc{Servaes, Franz} (17.\,6.\,1862 Köln – 14.\,7.\,1947 Wien), \emph{Journalist, Kritiker}|pwk}’ Urteil fiel sehr gut aus.}}}\label{K_L03204-3} über \textsc{Klinger\pwindex{Klinger, Max 18.\,2.\,1857 Leipzig – 4.\,7.\,1920 Großjena@\textsc{Klinger, Max} (18.\,2.\,1857 Leipzig – 4.\,7.\,1920 Großjena), \emph{Maler, Radierer, Bildhauer}!Beethoven@\strich\emph{Beethoven}|pwv}\pwindex{Klinger, Max 18.\,2.\,1857 Leipzig – 4.\,7.\,1920 Großjena@\textsc{Klinger, Max} (18.\,2.\,1857 Leipzig – 4.\,7.\,1920 Großjena), \emph{Maler, Radierer, Bildhauer}|pw}}, \strikeout{hat} das ich eben geleſen, hat mir{ }ſehr gut
               gefallen. Aber iſt auch das Urtheil richtig? Oder iſt wieder ein \label{K_L03204-4v}\edtext{Seceſſions\orgindex{Wiener Secession@Wiener Secession|pw}-Schwindel}{\lemma{\textnormal{\emph{Secessions-Schwindel}}}\Cendnote{\textnormal{Max Klingers\pwindex{Klinger, Max 18.\,2.\,1857 Leipzig – 4.\,7.\,1920 Großjena@\textsc{Klinger, Max} (18.\,2.\,1857 Leipzig – 4.\,7.\,1920 Großjena), \emph{Maler, Radierer, Bildhauer}|pwk}{ }\emph{Beethovenstatue}\pwindex{Klinger, Max 18.\,2.\,1857 Leipzig – 4.\,7.\,1920 Großjena@\textsc{Klinger, Max} (18.\,2.\,1857 Leipzig – 4.\,7.\,1920 Großjena), \emph{Maler, Radierer, Bildhauer}!Beethoven@\strich\emph{Beethoven}|pwk} stand im
                  Mittelpunkt der 14. Ausstellung der \emph{Wiener
                     Secession}\orgindex{Wiener Secession@Wiener Secession|pwk}, die Beethoven\pwindex{Beethoven, Ludwig van 17.\,12.\,1770 Bonn – 26.\,3.\,1827 Wien@\textsc{Beethoven, Ludwig van} (17.\,12.\,1770 Bonn – 26.\,3.\,1827 Wien), \emph{Komponist}|pwk} gewidmet
                  war und von 15. 4. 1902 bis 15. 6. 1902 stattfand.}}}\label{K_L03204-4} dabei? Ich kann es mir allerdings kaum
               denken; ich ahne etwas Großes, wenn \textsc{Klinger\pwindex{Klinger, Max 18.\,2.\,1857 Leipzig – 4.\,7.\,1920 Großjena@\textsc{Klinger, Max} (18.\,2.\,1857 Leipzig – 4.\,7.\,1920 Großjena), \emph{Maler, Radierer, Bildhauer}|pw}} einen \textsc{Beethoven\pwindex{Beethoven, Ludwig van 17.\,12.\,1770 Bonn – 26.\,3.\,1827 Wien@\textsc{Beethoven, Ludwig van} (17.\,12.\,1770 Bonn – 26.\,3.\,1827 Wien), \emph{Komponist}|pw}\pwindex{Klinger, Max 18.\,2.\,1857 Leipzig – 4.\,7.\,1920 Großjena@\textsc{Klinger, Max} (18.\,2.\,1857 Leipzig – 4.\,7.\,1920 Großjena), \emph{Maler, Radierer, Bildhauer}!Beethoven@\strich\emph{Beethoven}|pwv}} gemacht hat.\pend
           
\pstart
           Ich habe die Idee, etwa zehn meiner Theater-Feuilletons, die{ }ſich mit \textsc{Hauptmann\pwindex{Hauptmann, Gerhart 15.\,11.\,1862 Szczawno-Zdrój – 6.\,6.\,1946 Jagniątków@\textsc{Hauptmann, Gerhart} (15.\,11.\,1862 Szczawno-Zdrój – 6.\,6.\,1946 Jagniątków), \emph{Schriftsteller}|pw}} und{ }ſeinen Anhängern beſchäftigen, {\pb}zu{ }ſammeln
               und als Kampf-Buch unter dem ironiſchen Titel »\label{K_L03204-5v}\edtext{Die neue Richtung\pwindex{Goldmann, Paul 31.\,1.\,1865 Breslau – 25.\,9.\,1935 Wien@\textsc{Goldmann, Paul} (31.\,1.\,1865 Breslau – 25.\,9.\,1935 Wien), \emph{Schriftsteller, Journalist}!»neue Richtung«. Polemische Aufsätze über Berliner Theater-Aufführungen@\strich\emph{Die »neue Richtung«. Polemische Aufsätze über Berliner Theater-Aufführungen}|pwv}}{\lemma{\textnormal{\emph{Die neue Richtung}}}\Cendnote{\textnormal{Paul Goldmann\pwindex{Goldmann, Paul 31.\,1.\,1865 Breslau – 25.\,9.\,1935 Wien@\textsc{Goldmann, Paul} (31.\,1.\,1865 Breslau – 25.\,9.\,1935 Wien), \emph{Schriftsteller, Journalist}|pwk}: \emph{Die »neue Richtung«. Polemische Aufsätze über Berliner
                        Theater-Aufführungen}\pwindex{Goldmann, Paul 31.\,1.\,1865 Breslau – 25.\,9.\,1935 Wien@\textsc{Goldmann, Paul} (31.\,1.\,1865 Breslau – 25.\,9.\,1935 Wien), \emph{Schriftsteller, Journalist}!»neue Richtung«. Polemische Aufsätze über Berliner Theater-Aufführungen@\strich\emph{Die »neue Richtung«. Polemische Aufsätze über Berliner Theater-Aufführungen}|pwk}. Wien\oindex{Wien@\textbf{Wien}, \emph{Verwaltungsgebiet}|pwk}: \emph{C. W. Stern (Buchhandlung L. Rosner)}\orgindex{Buchhandlung L. Rosner@Buchhandlung L. Rosner|pwk},
                     erschienen im Oktober 1902, vordatiert auf 1903. Der Umfang ist mit 19 Texten größer als hier noch angedacht, wobei vier
                  Feuilletons zu Stücken Hauptmanns\pwindex{Hauptmann, Gerhart 15.\,11.\,1862 Szczawno-Zdrój – 6.\,6.\,1946 Jagniątków@\textsc{Hauptmann, Gerhart} (15.\,11.\,1862 Szczawno-Zdrój – 6.\,6.\,1946 Jagniątków), \emph{Schriftsteller}|pwk} das Buch
                  eröffnen und dominieren.}}}\label{K_L03204-5}« herauszugeben. Glaubſt Du, daß ein{ }ſolches Buch
               Leſer finden würde? Oder hängen Theater-Feuilletons nicht doch zu{ }ſehr mit dem Tage
               zuſammen, als daß{ }ſie in ein Buch hineingehörten? Die Idee kam mir, da ich neulich
               wieder hörte, wie{ }ſehr die \textsc{Hauptmann\pwindex{Hauptmann, Gerhart 15.\,11.\,1862 Szczawno-Zdrój – 6.\,6.\,1946 Jagniątków@\textsc{Hauptmann, Gerhart} (15.\,11.\,1862 Szczawno-Zdrój – 6.\,6.\,1946 Jagniątków), \emph{Schriftsteller}|pw}-Clique}{ }hier\oindex{Berlin@\textbf{Berlin}, \emph{Hauptstadt}|pwv} mich haßt. Man hat einer
               Dame Vorwürfe gemacht, daß{ }ſie im Theater freundlich mit mir geſprochen hat! Wenn ich{ }ſehe, daß man mit{ }ſolchen Mitteln eine künſtleriſche Überzeugung {\pb}bekämpfen will,{ }ſo habe ich den Drang, meine
               Überzeugung nur umſo{ }ſtärker zu betonen.\pend
           
\pstart
           Was Du mir vom \label{K_L03204-6v}\edtext{Tode der armen \textsc{Elsa Marktbreiter\pwindex{Markbreiter, Else 14.\,9.\,1870 Wien – 30.\,3.\,1902 ebd.@\textsc{Markbreiter, Else} (14.\,9.\,1870 Wien – 30.\,3.\,1902 ebd.)|pw}}}{\lemma{\textnormal{\emph{Tode … Marktbreiter}}}\Cendnote{\textnormal{Schnitzlers Cousine Else Markbreiter\pwindex{Markbreiter, Else 14.\,9.\,1870 Wien – 30.\,3.\,1902 ebd.@\textsc{Markbreiter, Else} (14.\,9.\,1870 Wien – 30.\,3.\,1902 ebd.)|pwk} war am 30. 3. 1902 an Tuberkulose verstorben, siehe A. S.: \emph{Tagebuch}, 31. 3. 1902.}}}\label{K_L03204-6}{ }ſchreibſt, iſt ergreifend. Aber
                  \strikeout{was} war es nicht eine Erlöſung? Freilich, das iſt
               auch eine dumme Phraſe. Erlöſt iſt man doch nur, wenn man \uline{weiß}, daß man erlöſt iſt.\pend
           
\pstart
           Ich habe Deiner Frau Mutter\pwindex{Schnitzler, Louise 8.\,7.\,1840 Kőszeg – 9.\,9.\,1911 Wien@\textsc{Schnitzler, Louise} (8.\,7.\,1840 Kőszeg – 9.\,9.\,1911 Wien)|pwv}
               nicht kondolirt, weil ich nicht weiß, ob die Verwandtſchaft nahe genug war, um eine
               Condolenz zu rechtfertigen. Wenn ja,{ }ſo {\pb}kondolire,
               bitte, in meinem Namen.\pend
           
\pstart
           Und dieſe arme hübſche \label{K_L03204-7v}\edtext{\textsc{Grethl Mandl\pwindex{Manasséwitsch, Margarethe 6.\,11.\,1880 – 21.\,9.\,1940 Genf@\textsc{Manasséwitsch, Margarethe} (6.\,11.\,1880 – 21.\,9.\,1940 Genf)|pw}}}{\lemma{\textnormal{\emph{Grethl Mandl}}}\Cendnote{\textnormal{Margarethe Mandl\pwindex{Manasséwitsch, Margarethe 6.\,11.\,1880 – 21.\,9.\,1940 Genf@\textsc{Manasséwitsch, Margarethe} (6.\,11.\,1880 – 21.\,9.\,1940 Genf)|pwk}, ebenso eine Cousine Schnitzlers, war, wie er vermutete, an
                  Neuritis erkrankt (vgl. A. S.: \emph{Tagebuch}, 13. 3. 1902), einer Nervenentzündung mit Lähmungserscheinungen. Gestorben
                  ist sie daran nicht.}}}\label{K_L03204-7}! Wie, um Himmels Willen, iſt das{ }ſo plötzlich
               gekommen? Sie hat mir in \label{K_L03204-8v}\edtext{\textsc{Pörtschach\oindex{Pörtschach am Wörthersee@\textbf{Pörtschach am Wörthersee}|pw}}}{\lemma{\textnormal{\emph{Pörtschach}}}\Cendnote{\textnormal{vermutlich im Sommer 1901}}}\label{K_L03204-8} noch{ }ſo gut gefallen. Iſt Ausſicht auf Heilung vorhanden?\pend
           
\pstart
           Haſt Du zu \label{K_L03204-9v}\edtext{arbeiten}{\lemma{\textnormal{\emph{arbeiten}}}\Cendnote{\textnormal{Schnitzler hatte am 6. 4. 1902 das
                  einaktige Puppenspiel \emph{Der tapfere Cassian}\pwindex{Schnitzler, Arthur 15. 5. 1862 Wien – 21. 10. 1931 ebd.@\textsc{Schnitzler, Arthur} (15. 5. 1862 Wien – 21. 10. 1931 ebd.), \emph{Schriftsteller, Mediziner}!tapfere Cassian. Puppenspiel in einem Akt@\strich\emph{Der tapfere Cassian. Puppenspiel in einem Akt}|pwk}
                  begonnen. Ebenso hatte er Überlegungen zu seinem Schauspiel \emph{Der einsame Weg}\pwindex{Schnitzler, Arthur 15. 5. 1862 Wien – 21. 10. 1931 ebd.@\textsc{Schnitzler, Arthur} (15. 5. 1862 Wien – 21. 10. 1931 ebd.), \emph{Schriftsteller, Mediziner}!einsame Weg. Schauspiel in fünf Akten@\strich\emph{Der einsame Weg. Schauspiel in fünf Akten}|pwk} angestellt (vgl. A. S.: \emph{Tagebuch}, 8. 4. 1902). Hinsichtlich
                     Goldmanns\pwindex{Goldmann, Paul 31.\,1.\,1865 Breslau – 25.\,9.\,1935 Wien@\textsc{Goldmann, Paul} (31.\,1.\,1865 Breslau – 25.\,9.\,1935 Wien), \emph{Schriftsteller, Journalist}|pwk} wiederholter Forderung, Schnitzler solle ein Lustspiel schreiben,
                     siehe XXXX Auszeichnungsfehler: Dokument L02914 nicht gefunden.}}}\label{K_L03204-9}
               angefangen? Denkſt Du an das Luſtſpiel? Ich weiß, Du wirſt über dieſe meine Frage
               wieder{ }ſehr aufgebracht{ }ſein, aber Du mußt mich{ }ſchon entſchuldigen, wenn ich unſeren
               einzigen \strikeout{D} Dramatiker, der \strikeout{\textcolor{gray}{h}\textcolor{gray}{×}\-\textcolor{gray}{×}\-\textcolor{gray}{×}\-\textcolor{gray}{×}\-\textcolor{gray}{×}} Humor hat, hier und da danach frage, {\pb}ob er
               nicht ein Luſtſpiel{ }ſchreiben möchte? Du wirſt wieder{ }ſagen: »Es fällt \substVorne{}\textsuperscript{Dir}\substDazwischen{}mir\substHinten{} nichts ein.« \strikeout{Aben} Aber das \strikeout{Schreiben} Schreiben wäre{ }ſehr einfach, wenn wir nur das
               zu{ }ſchreiben brauchten, was uns \strikeout{einfiele}
               einfällt.\pend
           
\pstart
           Wie geht es \textsc{Olga\pwindex{Schnitzler, Olga 17.\,1.\,1882 Wien – 13.\,1.\,1970 Lugano@\textsc{Schnitzler, Olga} (17.\,1.\,1882 Wien – 13.\,1.\,1970 Lugano), \emph{Schauspielerin, Sängerin}|pw}}? Grüße{ }ſie herzlichſt von mir. Ich{ }ſchreibe ihr nächſtens – jawohl, ganz gewiß,
               nächſtens!\pend
           
\pstart
           Lies’ \textsc{Hehn\pwindex{Hehn, Victor 8.\,10.\,1813 Tartu – 21.\,3.\,1890 Berlin@\textsc{Hehn, Victor} (8.\,10.\,1813 Tartu – 21.\,3.\,1890 Berlin), \emph{Schriftsteller, Historiker, Bibliothekar}|pw}}: Gedanken über \textsc{Goethe}\pwindex{Hehn, Victor 8.\,10.\,1813 Tartu – 21.\,3.\,1890 Berlin@\textsc{Hehn, Victor} (8.\,10.\,1813 Tartu – 21.\,3.\,1890 Berlin), \emph{Schriftsteller, Historiker, Bibliothekar}!Gedanken über Goethe@\strich\emph{Gedanken über Goethe}|pw}, namentlich den Aufſatz \label{K_L03204-10v}\edtext{\textsc{Goethe} und das Publikum\pwindex{Hehn, Victor 8.\,10.\,1813 Tartu – 21.\,3.\,1890 Berlin@\textsc{Hehn, Victor} (8.\,10.\,1813 Tartu – 21.\,3.\,1890 Berlin), \emph{Schriftsteller, Historiker, Bibliothekar}!Goethe und das Publikum. Eine Literaturgeschichte im Kleinen@\strich\emph{Goethe und das Publikum. Eine Literaturgeschichte im Kleinen}|pw}}{\lemma{\textnormal{\emph{Goethe und das Publikum}}}\Cendnote{\textnormal{Viktor Hehn\pwindex{Hehn, Victor 8.\,10.\,1813 Tartu – 21.\,3.\,1890 Berlin@\textsc{Hehn, Victor} (8.\,10.\,1813 Tartu – 21.\,3.\,1890 Berlin), \emph{Schriftsteller, Historiker, Bibliothekar}|pwk}: \emph{Goethe und das Publikum. Eine Literaturgeschichte im
                        Kleinen}\pwindex{Hehn, Victor 8.\,10.\,1813 Tartu – 21.\,3.\,1890 Berlin@\textsc{Hehn, Victor} (8.\,10.\,1813 Tartu – 21.\,3.\,1890 Berlin), \emph{Schriftsteller, Historiker, Bibliothekar}!Goethe und das Publikum. Eine Literaturgeschichte im Kleinen@\strich\emph{Goethe und das Publikum. Eine Literaturgeschichte im Kleinen}|pwk}. In: \emph{Gedanken über
                     Goethe}\pwindex{Hehn, Victor 8.\,10.\,1813 Tartu – 21.\,3.\,1890 Berlin@\textsc{Hehn, Victor} (8.\,10.\,1813 Tartu – 21.\,3.\,1890 Berlin), \emph{Schriftsteller, Historiker, Bibliothekar}!Gedanken über Goethe@\strich\emph{Gedanken über Goethe}|pwk}. Berlin\oindex{Berlin@\textbf{Berlin}, \emph{Hauptstadt}|pwk}: \emph{Gebrüder Borntraeger}\orgindex{Buchhandlung Gebrüder Bornträger@Buchhandlung Gebrüder Bornträger|pwk}{ }1887, S. 49–185.}}}\label{K_L03204-10}. Eine Fülle
               intereſſanten Materials in einem wundervoll klaren {\pb}Styl mitgetheilt. Der einzige Fehler iſt\strikeout{,} ein
               irrſinniger Antiſemitismus.\pend
           
\pstart
           \textsc{Kanner\pwindex{Kanner, Heinrich 9.\,11.\,1864 Galați – 15.\,2.\,1930 Wien@\textsc{Kanner, Heinrich} (9.\,11.\,1864 Galați – 15.\,2.\,1930 Wien), \emph{Herausgeber, Publizist}|pw}} war hier\oindex{Berlin@\textbf{Berlin}, \emph{Hauptstadt}|pwv}. Ich{ }ſoll
                  \label{K_L03204-11v}\edtext{zur »Zeit\orgindex{Zeit@Die Zeit|pw}« als Feuilleton-Redakteur}{\lemma{\textnormal{\emph{zur … Feuilleton-Redakteur}}}\Cendnote{\textnormal{Heinrich Kanner\pwindex{Kanner, Heinrich 9.\,11.\,1864 Galați – 15.\,2.\,1930 Wien@\textsc{Kanner, Heinrich} (9.\,11.\,1864 Galați – 15.\,2.\,1930 Wien), \emph{Herausgeber, Publizist}|pwk} dürfte seine Meinung also
                  geändert haben, siehe XXXX Auszeichnungsfehler: Dokument L03195 nicht gefunden.}}}\label{K_L03204-11} kommen\substVorne{}\textsuperscript{,}\substDazwischen{}.\substHinten{}{ }Burgtheater\orgindex{Burgtheater@Burgtheater|pw} und Volkstheater\orgindex{Volkstheater@Volkstheater|pw}{ }ſind allerdings{ }ſchon an \textsc{Burckhardt\pwindex{Burckhard, Max Eugen 14.\,7.\,1854 Korneuburg – 16.\,3.\,1912 Wien@\textsc{Burckhard, Max Eugen} (14.\,7.\,1854 Korneuburg – 16.\,3.\,1912 Wien), \emph{Schriftsteller, Rechtswissenschaftler, Theaterleiter}|pw}} vergeben. Ich{ }ſollte alſo nur Redaktions\substVorne{}\textsuperscript{-\textcolor{gray}{Kuli}}\substDazwischen{}-\textsc{Kuli}\substHinten{}{ }ſein und eine rieſige Büreauarbeit leiſten: Kleines und großes Feuilleton,
               eine Sonntagsbeilage \textsc{etc}. Ich glaube nicht, daß ich unter
               dieſen Umſtänden annehmen werde, – umſomehr als meine Mutter\pwindex{Goldmann, Clementine 15.\,5.\,1842 Breslau – 24.\,2.\,1924 Frankfurt am Main@\textsc{Goldmann, Clementine} (15.\,5.\,1842 Breslau – 24.\,2.\,1924 Frankfurt am Main)|pwv} nicht nach Wien\oindex{Wien@\textbf{Wien}, \emph{Verwaltungsgebiet}|pw}{ }{\pb}mitkommen würde\substVorne{}\textsuperscript{.}\substDazwischen{}{ }und ich meinen Hausſtand auflöſen
                     müßte.\substHinten{} Ja, wenn ich verheirathet wäre,{ }ſo wäre das Alles anders. Haſt Du noch immer
               keine Parthie für mich?\pend
           
\pstart
           \label{K_L03204-12v}\edtext{\textsc{Friedjungs\pwindex{Friedjung, Heinrich 18.\,1.\,1851 Roštín – 14.\,7.\,1920 Wien@\textsc{Friedjung, Heinrich} (18.\,1.\,1851 Roštín – 14.\,7.\,1920 Wien), \emph{Historiker}|pw}}{ }Buch\pwindex{Friedjung, Heinrich 18.\,1.\,1851 Roštín – 14.\,7.\,1920 Wien@\textsc{Friedjung, Heinrich} (18.\,1.\,1851 Roštín – 14.\,7.\,1920 Wien), \emph{Historiker}!Kampf um die Vorherrschaft in Deutschland 1859 bis 1866. 2 Bde.@\strich\emph{Der Kampf um die Vorherrschaft in Deutschland 1859 bis 1866. 2 Bde.}|pwv}}{\lemma{\textnormal{\emph{Friedjungs Buch}}}\Cendnote{\textnormal{Heinrich Friedjung\pwindex{Friedjung, Heinrich 18.\,1.\,1851 Roštín – 14.\,7.\,1920 Wien@\textsc{Friedjung, Heinrich} (18.\,1.\,1851 Roštín – 14.\,7.\,1920 Wien), \emph{Historiker}|pwk}: \emph{Der Kampf um die Vorherrschaft in Deutschland 1859 bis 1866. 2
                     Bde}\pwindex{Friedjung, Heinrich 18.\,1.\,1851 Roštín – 14.\,7.\,1920 Wien@\textsc{Friedjung, Heinrich} (18.\,1.\,1851 Roštín – 14.\,7.\,1920 Wien), \emph{Historiker}!Kampf um die Vorherrschaft in Deutschland 1859 bis 1866. 2 Bde.@\strich\emph{Der Kampf um die Vorherrschaft in Deutschland 1859 bis 1866. 2 Bde.}|pwk}. Stuttgart\oindex{Stuttgart@\textbf{Stuttgart}|pwk}: \emph{Cotta}\orgindex{J.G. Cotta’sche Buchhandlung Nachfolger@J.G. Cotta’sche Buchhandlung Nachfolger|pwk}{ }1897–1898. Schnitzler las das Buch\pwindex{Friedjung, Heinrich 18.\,1.\,1851 Roštín – 14.\,7.\,1920 Wien@\textsc{Friedjung, Heinrich} (18.\,1.\,1851 Roštín – 14.\,7.\,1920 Wien), \emph{Historiker}!Kampf um die Vorherrschaft in Deutschland 1859 bis 1866. 2 Bde.@\strich\emph{Der Kampf um die Vorherrschaft in Deutschland 1859 bis 1866. 2 Bde.}|pwkv} am 22. 3. 1902.}}}\label{K_L03204-12} werde ich leſen. Jetzt{ }ſtecke
               ich in \textsc{Grätz\pwindex{Graetz, Heinrich 31.\,10.\,1817 Gmina Książ Wielkopolski – 7.\,9.\,1891 München@\textsc{Graetz, Heinrich} (31.\,10.\,1817 Gmina Książ Wielkopolski – 7.\,9.\,1891 München), \emph{Historiker}|pw}}{ }\label{K_L03204-13v}\edtext{»Geſchichte der Juden\pwindex{Graetz, Heinrich 31.\,10.\,1817 Gmina Książ Wielkopolski – 7.\,9.\,1891 München@\textsc{Graetz, Heinrich} (31.\,10.\,1817 Gmina Książ Wielkopolski – 7.\,9.\,1891 München), \emph{Historiker}!Volkstümliche Geschichte der Juden in drei Bänden@\strich\emph{Volkstümliche Geschichte der Juden in drei Bänden}|pw}« (Volksausgabe in drei Bänden)}{\lemma{\textnormal{\emph{»Geschichte … Bänden)}}}\Cendnote{\textnormal{H. [ = Heinrich] Graetz\pwindex{Graetz, Heinrich 31.\,10.\,1817 Gmina Książ Wielkopolski – 7.\,9.\,1891 München@\textsc{Graetz, Heinrich} (31.\,10.\,1817 Gmina Książ Wielkopolski – 7.\,9.\,1891 München), \emph{Historiker}|pwk}: \emph{Volkstümliche Geschichte der Juden in drei
                        Bänden}\pwindex{Graetz, Heinrich 31.\,10.\,1817 Gmina Książ Wielkopolski – 7.\,9.\,1891 München@\textsc{Graetz, Heinrich} (31.\,10.\,1817 Gmina Książ Wielkopolski – 7.\,9.\,1891 München), \emph{Historiker}!Volkstümliche Geschichte der Juden in drei Bänden@\strich\emph{Volkstümliche Geschichte der Juden in drei Bänden}|pwk}. Leipzig\oindex{Leipzig@\textbf{Leipzig}, \emph{Hauptstadt}|pwk}: \emph{Oskar Leiner}\orgindex{Oskar Leiner GmbH und Co. KG@Oskar Leiner GmbH {\kaufmannsund}  Co. KG|pwk}{ }1888. Eine Lektüre durch Schnitzler ist
                  nicht nachweisbar.}}}\label{K_L03204-13}. Ein tauſendfach anregendes Buch\pwindex{Graetz, Heinrich 31.\,10.\,1817 Gmina Książ Wielkopolski – 7.\,9.\,1891 München@\textsc{Graetz, Heinrich} (31.\,10.\,1817 Gmina Książ Wielkopolski – 7.\,9.\,1891 München), \emph{Historiker}!Volkstümliche Geschichte der Juden in drei Bänden@\strich\emph{Volkstümliche Geschichte der Juden in drei Bänden}|pwv}. Mußt Du leſen. »\label{K_L03204-14v}\edtext{\textsc{Francesca da Rimini\pwindex{D’Annunzio, Gabriele 12.\,3.\,1863 Pescara – 1.\,3.\,1938 Cargnacco@\textsc{D’Annunzio, Gabriele} (12.\,3.\,1863 Pescara – 1.\,3.\,1938 Cargnacco), \emph{Schriftsteller}!Francesca da Rimini@\strich\emph{Francesca da Rimini}|pw}}}{\lemma{\textnormal{\emph{Francesca da Rimini}}}\Cendnote{\textnormal{Siehe A. S.: \emph{Tagebuch}, 2. 4. 1902. }}}\label{K_L03204-14}« hat
               mich bodenlos gelangweilt.\pend
           
\pstart
           Schreib’ mir bald wieder, {\pb}mein lieber Freund, und{ }ſei vielmals und von Herzen gegrüßt von {\\}Deinem {\\}\spacefill\mbox{Paul Goldm}\pend
           
\pstart
           \noindent{}Was macht \textsc{Richard\pwindex{Beer-Hofmann, Richard 11.\,7.\,1866 Wien – 26.\,9.\,1945 New York City@\textsc{Beer-Hofmann, Richard} (11.\,7.\,1866 Wien – 26.\,9.\,1945 New York City), \emph{Schriftsteller}|pw}}?\pend
           \selectlanguage{ngerman}\endnumbering\briefempfaengerindex{Schnitzler, Arthur@\textsc{Schnitzler, Arthur}!zzzGoldmann, Paul@\emph{von Paul Goldmann}!1902-04-171@{17. 4. [1902]}|)be}\mylabel{L03204h}  \newcommand{\dateiname}{L03204}\newcommand{\titel}{Paul Goldmann an Arthur Schnitzler, 17. 4. [1902]}\newcommand{\editorInnen}{Martin Anton Müller und Laura Untner}%% latex-leseansicht-abspann.tex
%% Abspann für die Leseansicht.
%% Der Schalter \ifkorrekturansicht ist bereits durch den Vorspann gesetzt.

%% latex-abspann.tex
%% Gemeinsamer Abspann für Korrekturansicht und Leseansicht.
%% Setzt den Schalter \ifkorrekturansicht voraus (gesetzt in den
%% einbindenden Dateien latex-korrekturansicht-abspann.tex bzw.
%% latex-leseansicht-abspann.tex).
%% ---------------------------------------------------------------

\normalsize

% Das esempio-Environment wird nur in der Leseansicht benötigt
\ifkorrekturansicht\else
\newenvironment{esempio}[3]%
{
    \vspace{1.5ex}
    \rlap{\underline{#1}}
    \par
    \setlength{\parindent}{0cm}
    \nopagebreak
    \leftskip=#2cm
    \rightskip=#3cm
}
{
    \par
}
\fi

\doendnotes{C}
\bigskip
\vfill

\clearpage

\footnotesize

\ifkorrekturansicht
  \lohead{\textsc{register}}
\fi

% theindex-Environment neu definieren ohne reledmac
\makeatletter
\renewenvironment{theindex}{%
  \ifkorrekturansicht
    \section*{\indexname}%
  \else
    \subsubsection*{Index der erwähnten Entitäten}%
  \fi
  \setlength{\parindent}{0pt}%
  \setlength{\parskip}{0pt plus 0.3pt}%
  \let\item\@idxitem
}{%
  \ifkorrekturansicht\clearpage\fi
}
\makeatother

\IfFileExists{\jobname-pw.ind}{\input{\jobname-pw.ind}}{}

% Quellenangabe nur in der Leseansicht
\ifkorrekturansicht\else
% Fallback-Definitionen, falls die .tex-Datei \titel etc. nicht gesetzt hat
\providecommand{\titel}{}
\providecommand{\editorInnen}{}
\providecommand{\dateiname}{\jobname}

\vspace{3cm}

\vfill

\footnotesize
\textsc{Quelle}: \titel. Herausgegeben von {\editorInnen}. In: \emph{Arthur Schnitzler: Briefwechsel mit Autorinnen und Autoren}.
 Digitale Edition, https://schnitzler-briefe.acdh.oeaw.ac.at/{\dateiname}.html (Stand \today)
\fi

\end{document}


