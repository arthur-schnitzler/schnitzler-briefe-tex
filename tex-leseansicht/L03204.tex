%% latex-korrekturansicht-vorspann.tex
%% Vorspann für die Korrekturansicht.
%% Lädt die gemeinsame Datei latex-vorspann.tex mit gesetztem Schalter.

\newif\ifkorrekturansicht
\korrekturansichttrue

\input{../tex-inputs/latex-vorspann}


\section[ Paul Goldmann an Arthur Schnitzler, 17. 4. {[}1902{]}]{L03204 Paul Goldmann an Arthur Schnitzler, 17. 4. {[}1902{]}}
\nopagebreak\mylabel{L03204v}
\rehead{ }\normalsize\beginnumbering\briefempfaengerindex{Schnitzler, Arthur@\textsc{Schnitzler, Arthur}!zzzGoldmann, Paul@\emph{von Paul Goldmann}!1902-04-171@{17. 4. {[}1902{]}}|(be}
\toendnotes[C]{\smallbreak\pagebreak[2]}\Standort{DLA, A:Schnitzler, HS.NZ85.1.3172.}
\physDesc{Brief, 3 Blätter, 10 Seiten, 4355 Zeichen
\newline{}Handschrift: blaue Tinte, deutsche Kurrent
\newline{}Schnitzler: 1) mit Bleistift das Jahr »902« vermerkt  2) mit rotem Buntstift elf Unterstreichungen}\toendnotes[C]{\smallbreak}
\pstart
           \raggedleft{}{\pb}\textcolor{gray}{\textbf{DESSAUERSTRASSE 19}}\oindex{Dessauer Strasse@\textbf{Dessauer Straße}, \emph{Straße (K.STR)}|pw}\pend
           
\pstart
           Berlin\oindex{Berlin@\textbf{Berlin}, \emph{P.PPLC}|pw}, 17. April.\pend
           
\pstart{}Mein lieber Freund,\pend\vspace{0.5em}
\pstart
           Seit dem Empfang Deines letzten lieben Briefes, de\substVorne{}\textsuperscript{\textcolor{gray}{n}}\substDazwischen{}r\substHinten{} nach meiner \label{K_L03204-1v}\edtext{Rückkehr aus \textsc{Prag\oindex{Prag@\textbf{Prag}, \emph{A.ADM1}|pw}}}{\lemma{\textnormal{\emph{Rückkehr aus Prag}}}\Cendnote{\textnormal{Siehe Paul Goldmann an Arthur Schnitzler, 1. 4. [1902]. }}}\label{K_L03204-1} eintraf,
               will ich Dir \uline{täglich} ſchreiben, und täglich muß ich
               darauf verzichten. Es iſt unbeſchreiblich, was jetzt wieder Alles an Arbeit, Beſuchen
                  \textsc{etc.} auf mich einſtürmt. Ich bin Dir ſehr dankbar, daß
               Du meine Antwort nicht abgewartet und mich abermals heut durch Deine lieben Nachrichten erfreut haſt. Dieſer \label{K_L03204-2v}\edtext{Bernhardiner}{\lemma{\textnormal{\emph{Bernhardiner}}}\Cendnote{\textnormal{Schnitzler besaß für kurze Zeit, vermutlich
                  ab dem 23. 3. 1902,
                  einen Bernhardiner namens Bern. Im Oktober wurde er in
                  dem im gleichen Monat eröffneten Tierschutzhaus\oindex{Tierschutzhaus@\textbf{Tierschutzhaus}, \emph{Gebäude (K.GBD)}|pwk} des \emph{Wiener
                     Tierschutz-Vereins}\orgindex{Wiener Tierschutz-Verein@Wiener Tierschutz-Verein|pwk} behandelt, Mitte
                     Dezember erneut. Ab Januar 1903 versuchte
                     Schnitzler ihn zu vermitteln. Zu diesem
                  Zeitpunkt lebte er aber bereits nicht mehr bei ihnen (vgl. Arthur Schnitzler an Richard Beer-Hofmann, 14. 1. 1903 und Hermann Bahr an Arthur Schnitzler, 4. 4. [1903]). In diesem Jahr finden sich noch drei Erwähnungen
                  im \emph{Tagebuch}\pwindex{Tagebuch@\emph{Tagebuch}|pwk}: 23. 5. 1903, 18. 6. 1903 und 6. 8. 1903. Siehe auch \emph{Briefe 1913–1931}, S. 118.}}}\label{K_L03204-2} muß herrlich ſein. Ich
               freue mich ſchon ſehr darauf, ihn kennen zu lernen. {\pb}Was Du über \textsc{Hirschfeld\pwindex{Hirschfeld, Georg 11.02.1873 – 17.01.1942@\textsc{Hirschfeld, Georg} (11.02.1873 – 17.01.1942), \emph{Schriftsteller/Schriftstellerin}|pw}} ſchreibſt, iſt ſehr ſchön geſagt. Die Freunde und »literariſchen Kritiker«, die
               den unentwickelten Burſchen\pwindex{Hirschfeld, Georg 11.02.1873 – 17.01.1942@\textsc{Hirschfeld, Georg} (11.02.1873 – 17.01.1942), \emph{Schriftsteller/Schriftstellerin}|pwv},
               deſſen Sentimentalität ſie für Poeſie nehmen, zum Dichter ausgeſchrieen haben, haben
               allerdings viel Schuld an dem jämmerlichen Ende, – aber doch nicht die einzige. Wer
               im Stande iſt, ein flaches Machwerk, wie den »Weg zum
                  Licht\pwindex{Weg zum Licht. Ein Salzburger Maerchendrama in vier Akten@\emph{Der Weg zum Licht. Ein Salzburger Märchendrama in vier Akten}|pw}« zu ſchreiben, in dem auch nicht die leiſeſte Spur von Perſönlichkeit
               ſteckt, der hat eben niemals eine Perſönlichkeit gehabt. Denn das iſt vollkommen
               ausgeſchloſſen, daß man aus einem Dichter {\pb}plötzlich
               ein Flachkopf wird. Der »Weg zum Licht\pwindex{Weg zum Licht. Ein Salzburger Maerchendrama in vier Akten@\emph{Der Weg zum Licht. Ein Salzburger Märchendrama in vier Akten}|pw}« iſt
               nicht verfehlt, ſondern complet talentlos. Das iſt ein Unterſchied.\pend
           
\pstart
           \textsc{Servaes\pwindex{Servaes, Franz 17.06.1862 – 14.07.1947@\textsc{Servaes, Franz} (17.06.1862 – 14.07.1947), \emph{Journalist/Journalistin, Kritiker/Kritikerin}|pw}}{ }\label{K_L03204-3v}\edtext{Feuilleton\pwindex{Klinger s »Beethoven«@\emph{Klinger’s »Beethoven«}|pwv}}{\lemma{\textnormal{\emph{Feuilleton}}}\Cendnote{\textnormal{Franz Servaes\pwindex{Servaes, Franz 17.06.1862 – 14.07.1947@\textsc{Servaes, Franz} (17.06.1862 – 14.07.1947), \emph{Journalist/Journalistin, Kritiker/Kritikerin}|pwk}: \emph{Klinger’s »Beethoven«}\pwindex{Klinger s »Beethoven«@\emph{Klinger’s »Beethoven«}|pwk}. In: \emph{Neue Freie Presse}\pwindex{Neue Freie Presse@\emph{Neue Freie Presse}|pwk}, Nr. 13.521, 16. 4. 1902, Morgenblatt, S. 1–3. Servaes\pwindex{Servaes, Franz 17.06.1862 – 14.07.1947@\textsc{Servaes, Franz} (17.06.1862 – 14.07.1947), \emph{Journalist/Journalistin, Kritiker/Kritikerin}|pwk}’ Urteil fiel sehr gut aus.}}}\label{K_L03204-3} über \textsc{Klinger\pwindex{Beethoven@\emph{Beethoven}|pwv}\pwindex{Klinger, Max 18.02.1857 – 04.07.1920@\textsc{Klinger, Max} (18.02.1857 – 04.07.1920), \emph{Maler/Malerin, Radierer/Radiererin, Bildhauer/Bildhauerin}|pw}}, \strikeout{hat} das ich eben geleſen, hat mir ſehr gut
               gefallen. Aber iſt auch das Urtheil richtig? Oder iſt wieder ein \label{K_L03204-4v}\edtext{SeceſſionsXXXX ORGangabe fehlt-Schwindel}{\lemma{\textnormal{\emph{Seceſſions-Schwindel}}}\Cendnote{\textnormal{Max Klingers\pwindex{Klinger, Max 18.02.1857 – 04.07.1920@\textsc{Klinger, Max} (18.02.1857 – 04.07.1920), \emph{Maler/Malerin, Radierer/Radiererin, Bildhauer/Bildhauerin}|pwk}{ }\emph{Beethovenstatue}\pwindex{Beethoven@\emph{Beethoven}|pwk} stand im
                  Mittelpunkt der 14. Ausstellung der \emph{Wiener
                     Secession}XXXX ORGangabe fehlt, die Beethoven\pwindex{Beethoven, Ludwig van 17.12.1770 – 26.03.1827@\textsc{Beethoven, Ludwig van} (17.12.1770 – 26.03.1827), \emph{Komponist/Komponistin}|pwk} gewidmet
                  war und von 15. 4. 1902 bis 15. 6. 1902 stattfand.}}}\label{K_L03204-4} dabei? Ich kann es mir allerdings kaum
               denken; ich ahne etwas Großes, wenn \textsc{Klinger\pwindex{Klinger, Max 18.02.1857 – 04.07.1920@\textsc{Klinger, Max} (18.02.1857 – 04.07.1920), \emph{Maler/Malerin, Radierer/Radiererin, Bildhauer/Bildhauerin}|pw}} einen \textsc{Beethoven\pwindex{Beethoven, Ludwig van 17.12.1770 – 26.03.1827@\textsc{Beethoven, Ludwig van} (17.12.1770 – 26.03.1827), \emph{Komponist/Komponistin}|pw}\pwindex{Beethoven@\emph{Beethoven}|pwv}} gemacht hat.\pend
           
\pstart
           Ich habe die Idee, etwa zehn meiner Theater-Feuilletons, die ſich mit \textsc{Hauptmann\pwindex{Hauptmann, Gerhart 15.11.1862 – 06.06.1946@\textsc{Hauptmann, Gerhart} (15.11.1862 – 06.06.1946), \emph{Schriftsteller/Schriftstellerin}|pw}} und ſeinen Anhängern beſchäftigen, {\pb}zu ſammeln
               und als Kampf-Buch unter dem ironiſchen Titel »\label{K_L03204-5v}\edtext{Die neue Richtung\pwindex{»neue Richtung«. Polemische Aufsaetze ueber Berliner Theater-Auffuehrungen@\emph{Die »neue Richtung«. Polemische Aufsätze über Berliner Theater-Aufführungen}|pwv}}{\lemma{\textnormal{\emph{Die neue Richtung}}}\Cendnote{\textnormal{Paul Goldmann\pwindex{Goldmann, Paul 31.01.1865 – 25.09.1935@\textsc{Goldmann, Paul} (31.01.1865 – 25.09.1935), \emph{Schriftsteller/Schriftstellerin, Journalist/Journalistin}|pwk}: \emph{Die »neue Richtung«. Polemische Aufsätze über Berliner
                        Theater-Aufführungen}\pwindex{»neue Richtung«. Polemische Aufsaetze ueber Berliner Theater-Auffuehrungen@\emph{Die »neue Richtung«. Polemische Aufsätze über Berliner Theater-Aufführungen}|pwk}. Wien\oindex{Wien@\textbf{Wien}, \emph{A.ADM2}|pwk}: \emph{C. W. Stern (Buchhandlung L. Rosner)}\orgindex{Buchhandlung L. Rosner@Buchhandlung L. Rosner|pwk},
                     erschienen im Oktober 1902, vordatiert auf 1903. Der Umfang ist mit 19 Texten größer als hier noch angedacht, wobei vier
                  Feuilletons zu Stücken Hauptmanns\pwindex{Hauptmann, Gerhart 15.11.1862 – 06.06.1946@\textsc{Hauptmann, Gerhart} (15.11.1862 – 06.06.1946), \emph{Schriftsteller/Schriftstellerin}|pwk} das Buch
                  eröffnen und dominieren.}}}\label{K_L03204-5}« herauszugeben. Glaubſt Du, daß ein ſolches Buch
               Leſer finden würde? Oder hängen Theater-Feuilletons nicht doch zu ſehr mit dem Tage
               zuſammen, als daß ſie in ein Buch hineingehörten? Die Idee kam mir, da ich neulich
               wieder hörte, wie ſehr die \textsc{Hauptmann\pwindex{Hauptmann, Gerhart 15.11.1862 – 06.06.1946@\textsc{Hauptmann, Gerhart} (15.11.1862 – 06.06.1946), \emph{Schriftsteller/Schriftstellerin}|pw}-Clique}{ }hier\oindex{Berlin@\textbf{Berlin}, \emph{P.PPLC}|pwv} mich haßt. Man hat einer
               Dame Vorwürfe gemacht, daß ſie im Theater freundlich mit mir geſprochen hat! Wenn ich
               ſehe, daß man mit ſolchen Mitteln eine künſtleriſche Überzeugung {\pb}bekämpfen will, ſo habe ich den Drang, meine
               Überzeugung nur umſo ſtärker zu betonen.\pend
           
\pstart
           Was Du mir vom \label{K_L03204-6v}\edtext{Tode der armen \textsc{Elsa Marktbreiter\pwindex{Markbreiter, Else 14.09.1870 – 30.03.1902@\textsc{Markbreiter, Else} (14.09.1870 – 30.03.1902)|pw}}}{\lemma{\textnormal{\emph{Tode … Marktbreiter}}}\Cendnote{\textnormal{Schnitzlers Cousine Else Markbreiter\pwindex{Markbreiter, Else 14.09.1870 – 30.03.1902@\textsc{Markbreiter, Else} (14.09.1870 – 30.03.1902)|pwk} war am 30. 3. 1902 an Tuberkulose verstorben, siehe A. S.: \emph{Tagebuch}, 31. 3. 1902.}}}\label{K_L03204-6} ſchreibſt, iſt ergreifend. Aber
                  \strikeout{was} war es nicht eine Erlöſung? Freilich, das iſt
               auch eine dumme Phraſe. Erlöſt iſt man doch nur, wenn man \uline{weiß}, daß man erlöſt iſt.\pend
           
\pstart
           Ich habe Deiner Frau Mutter\pwindex{Schnitzler, Louise 1840-07-08 – 1911-09-09@\textsc{Schnitzler, Louise} (1840-07-08 – 1911-09-09)|pwv}
               nicht kondolirt, weil ich nicht weiß, ob die Verwandtſchaft nahe genug war, um eine
               Condolenz zu rechtfertigen. Wenn ja, ſo {\pb}kondolire,
               bitte, in meinem Namen.\pend
           
\pstart
           Und dieſe arme hübſche \label{K_L03204-7v}\edtext{\textsc{Grethl Mandl\pwindex{Manassewitsch, Margarethe 6.11.1880 – 21.09.1940@\textsc{Manasséwitsch, Margarethe} (6.11.1880 – 21.09.1940)|pw}}}{\lemma{\textnormal{\emph{Grethl Mandl}}}\Cendnote{\textnormal{Margarethe Mandl\pwindex{Manassewitsch, Margarethe 6.11.1880 – 21.09.1940@\textsc{Manasséwitsch, Margarethe} (6.11.1880 – 21.09.1940)|pwk}, ebenso eine Cousine Schnitzlers, war, wie er vermutete, an
                  Neuritis erkrankt (vgl. A. S.: \emph{Tagebuch}, 13. 3. 1902), einer Nervenentzündung mit Lähmungserscheinungen. Gestorben
                  ist sie daran nicht.}}}\label{K_L03204-7}! Wie, um Himmels Willen, iſt das ſo plötzlich
               gekommen? Sie hat mir in \label{K_L03204-8v}\edtext{\textsc{Pörtschach\oindex{Poertschach am Woerthersee@\textbf{Pörtschach am Wörthersee}, \emph{P.PPL}|pw}}}{\lemma{\textnormal{\emph{Pörtschach}}}\Cendnote{\textnormal{vermutlich im Sommer 1901}}}\label{K_L03204-8} noch ſo gut gefallen. Iſt Ausſicht auf Heilung vorhanden?\pend
           
\pstart
           Haſt Du zu \label{K_L03204-9v}\edtext{arbeiten}{\lemma{\textnormal{\emph{arbeiten}}}\Cendnote{\textnormal{Schnitzler hatte am 6. 4. 1902 das
                  einaktige Puppenspiel \emph{Der tapfere Cassian}\pwindex{tapfere Cassian. Puppenspiel in einem Akt@\emph{Der tapfere Cassian. Puppenspiel in einem Akt}|pwk}
                  begonnen. Ebenso hatte er Überlegungen zu seinem Schauspiel \emph{Der einsame Weg}\pwindex{einsame Weg. Schauspiel in fuenf Akten@\emph{Der einsame Weg. Schauspiel in fünf Akten}|pwk} angestellt (vgl. A. S.: \emph{Tagebuch}, 8. 4. 1902). Hinsichtlich
                     Goldmanns\pwindex{Goldmann, Paul 31.01.1865 – 25.09.1935@\textsc{Goldmann, Paul} (31.01.1865 – 25.09.1935), \emph{Schriftsteller/Schriftstellerin, Journalist/Journalistin}|pwk} wiederholter Forderung, Schnitzler solle ein Lustspiel schreiben,
                     siehe Paul Goldmann an Arthur Schnitzler, 2. 5. [1900].}}}\label{K_L03204-9}
               angefangen? Denkſt Du an das Luſtſpiel? Ich weiß, Du wirſt über dieſe meine Frage
               wieder ſehr aufgebracht ſein, aber Du mußt mich ſchon entſchuldigen, wenn ich unſeren
               einzigen \strikeout{D} Dramatiker, der \strikeout{\textcolor{gray}{h}\textcolor{gray}{×}\-\textcolor{gray}{×}\-\textcolor{gray}{×}\-\textcolor{gray}{×}\-\textcolor{gray}{×}} Humor hat, hier und da danach frage, {\pb}ob er
               nicht ein Luſtſpiel ſchreiben möchte? Du wirſt wieder ſagen: »Es fällt \substVorne{}\textsuperscript{Dir}\substDazwischen{}mir\substHinten{} nichts ein.« \strikeout{Aben} Aber das \strikeout{Schreiben} Schreiben wäre ſehr einfach, wenn wir nur das
               zu ſchreiben brauchten, was uns \strikeout{einfiele}
               einfällt.\pend
           
\pstart
           Wie geht es \textsc{Olga\pwindex{Schnitzler, Olga 17.01.1882 – 13.01.1970@\textsc{Schnitzler, Olga} (17.01.1882 – 13.01.1970), \emph{Schauspieler/Schauspielerin, Sänger/Sängerin}|pw}}? Grüße ſie herzlichſt von mir. Ich ſchreibe ihr nächſtens – jawohl, ganz gewiß,
               nächſtens!\pend
           
\pstart
           Lies’ \textsc{Hehn\pwindex{Hehn, Victor 1813-10-08 – 1890-03-21@\textsc{Hehn, Victor} (1813-10-08 – 1890-03-21), \emph{Schriftsteller/Schriftstellerin, Historiker/Historikerin, Bibliothekar/Bibliothekarin}|pw}}: Gedanken über \textsc{Goethe}\pwindex{Gedanken ueber Goethe@\emph{Gedanken über Goethe}|pw}, namentlich den Aufſatz \label{K_L03204-10v}\edtext{\textsc{Goethe} und das Publikum\pwindex{Goethe und das Publikum. Eine Literaturgeschichte im Kleinen@\emph{Goethe und das Publikum. Eine Literaturgeschichte im Kleinen}|pw}}{\lemma{\textnormal{\emph{Goethe und das Publikum}}}\Cendnote{\textnormal{Viktor Hehn\pwindex{Hehn, Victor 1813-10-08 – 1890-03-21@\textsc{Hehn, Victor} (1813-10-08 – 1890-03-21), \emph{Schriftsteller/Schriftstellerin, Historiker/Historikerin, Bibliothekar/Bibliothekarin}|pwk}: \emph{Goethe und das Publikum. Eine Literaturgeschichte im
                        Kleinen}\pwindex{Goethe und das Publikum. Eine Literaturgeschichte im Kleinen@\emph{Goethe und das Publikum. Eine Literaturgeschichte im Kleinen}|pwk}. In: \emph{Gedanken über
                     Goethe}\pwindex{Gedanken ueber Goethe@\emph{Gedanken über Goethe}|pwk}. Berlin\oindex{Berlin@\textbf{Berlin}, \emph{P.PPLC}|pwk}: \emph{Gebrüder Borntraeger}\orgindex{Buchhandlung Gebrueder Borntraeger@Buchhandlung Gebrüder Bornträger|pwk}{ }1887, S. 49–185.}}}\label{K_L03204-10}. Eine Fülle
               intereſſanten Materials in einem wundervoll klaren {\pb}Styl mitgetheilt. Der einzige Fehler iſt\strikeout{,} ein
               irrſinniger Antiſemitismus.\pend
           
\pstart
           \textsc{Kanner\pwindex{Kanner, Heinrich 09.11.1864 – 15.02.1930@\textsc{Kanner, Heinrich} (09.11.1864 – 15.02.1930), \emph{Herausgeber/Herausgeberin, Publizist/Publizistin}|pw}} war hier\oindex{Berlin@\textbf{Berlin}, \emph{P.PPLC}|pwv}. Ich ſoll
                  \label{K_L03204-11v}\edtext{zur »Zeit\orgindex{Zeit@Die Zeit|pw}« als Feuilleton-Redakteur}{\lemma{\textnormal{\emph{zur … Feuilleton-Redakteur}}}\Cendnote{\textnormal{Heinrich Kanner\pwindex{Kanner, Heinrich 09.11.1864 – 15.02.1930@\textsc{Kanner, Heinrich} (09.11.1864 – 15.02.1930), \emph{Herausgeber/Herausgeberin, Publizist/Publizistin}|pwk} dürfte seine Meinung also
                  geändert haben, siehe Paul Goldmann an Arthur Schnitzler, 25. 1. [1902].}}}\label{K_L03204-11} kommen\substVorne{}\textsuperscript{,}\substDazwischen{}.\substHinten{}{ }Burgtheater\orgindex{Burgtheater@Burgtheater|pw} und Volkstheater\orgindex{Volkstheater@Volkstheater|pw} ſind allerdings ſchon an \textsc{Burckhardt\pwindex{Burckhard, Max Eugen 14.07.1854 – 16.03.1912@\textsc{Burckhard, Max Eugen} (14.07.1854 – 16.03.1912), \emph{Schriftsteller/Schriftstellerin, Rechtswissenschaftler/Rechtswissenschaftlerin, Theaterleiter/Theaterleiterin}|pw}} vergeben. Ich ſollte alſo nur Redaktions\substVorne{}\textsuperscript{-\textcolor{gray}{Kuli}}\substDazwischen{}-\textsc{Kuli}\substHinten{} ſein und eine rieſige Büreauarbeit leiſten: Kleines und großes Feuilleton,
               eine Sonntagsbeilage \textsc{etc}. Ich glaube nicht, daß ich unter
               dieſen Umſtänden annehmen werde, – umſomehr als meine Mutter\pwindex{Goldmann, Clementine 1842-05-15 – 1924-02-24@\textsc{Goldmann, Clementine} (1842-05-15 – 1924-02-24)|pwv} nicht nach Wien\oindex{Wien@\textbf{Wien}, \emph{A.ADM2}|pw}{ }{\pb}mitkommen würde\substVorne{}\textsuperscript{.}\substDazwischen{}{ }und ich meinen Hausſtand auflöſen
                     müßte.\substHinten{} Ja, wenn ich verheirathet wäre, ſo wäre das Alles anders. Haſt Du noch immer
               keine Parthie für mich?\pend
           
\pstart
           \label{K_L03204-12v}\edtext{\textsc{Friedjungs\pwindex{Friedjung, Heinrich 18.01.1851 – 14.07.1920@\textsc{Friedjung, Heinrich} (18.01.1851 – 14.07.1920), \emph{Historiker/Historikerin}|pw}}{ }Buch\pwindex{Kampf um die Vorherrschaft in Deutschland 1859 bis 1866. 2 Bde.@\emph{Der Kampf um die Vorherrschaft in Deutschland 1859 bis 1866. 2 Bde.}|pwv}}{\lemma{\textnormal{\emph{Friedjungs Buch}}}\Cendnote{\textnormal{Heinrich Friedjung\pwindex{Friedjung, Heinrich 18.01.1851 – 14.07.1920@\textsc{Friedjung, Heinrich} (18.01.1851 – 14.07.1920), \emph{Historiker/Historikerin}|pwk}: \emph{Der Kampf um die Vorherrschaft in Deutschland 1859 bis 1866. 2
                     Bde}\pwindex{Kampf um die Vorherrschaft in Deutschland 1859 bis 1866. 2 Bde.@\emph{Der Kampf um die Vorherrschaft in Deutschland 1859 bis 1866. 2 Bde.}|pwk}. Stuttgart\oindex{Stuttgart@\textbf{Stuttgart}, \emph{P.PPLA}|pwk}: \emph{Cotta}\orgindex{J.G. Cotta sche Buchhandlung Nachfolger@J.G. Cotta’sche Buchhandlung Nachfolger|pwk}{ }1897–1898. Schnitzler las das Buch\pwindex{Kampf um die Vorherrschaft in Deutschland 1859 bis 1866. 2 Bde.@\emph{Der Kampf um die Vorherrschaft in Deutschland 1859 bis 1866. 2 Bde.}|pwkv} am 22. 3. 1902.}}}\label{K_L03204-12} werde ich leſen. Jetzt ſtecke
               ich in \textsc{Grätz\pwindex{Graetz, Heinrich 1817-10-31 – 1891-09-07@\textsc{Graetz, Heinrich} (1817-10-31 – 1891-09-07), \emph{Historiker/Historikerin}|pw}}{ }\label{K_L03204-13v}\edtext{»Geſchichte der Juden\pwindex{Volkstuemliche Geschichte der Juden in drei Baenden@\emph{Volkstümliche Geschichte der Juden in drei Bänden}|pw}« (Volksausgabe in drei Bänden)}{\lemma{\textnormal{\emph{»Geſchichte … Bänden)}}}\Cendnote{\textnormal{H. [ = Heinrich] Graetz\pwindex{Graetz, Heinrich 1817-10-31 – 1891-09-07@\textsc{Graetz, Heinrich} (1817-10-31 – 1891-09-07), \emph{Historiker/Historikerin}|pwk}: \emph{Volkstümliche Geschichte der Juden in drei
                        Bänden}\pwindex{Volkstuemliche Geschichte der Juden in drei Baenden@\emph{Volkstümliche Geschichte der Juden in drei Bänden}|pwk}. Leipzig\oindex{Leipzig@\textbf{Leipzig}, \emph{P.PPLA3}|pwk}: \emph{Oskar Leiner}\orgindex{Oskar Leiner GmbH und Co. KG@Oskar Leiner GmbH {\kaufmannsund}  Co. KG|pwk}{ }1888. Eine Lektüre durch Schnitzler ist
                  nicht nachweisbar.}}}\label{K_L03204-13}. Ein tauſendfach anregendes Buch\pwindex{Volkstuemliche Geschichte der Juden in drei Baenden@\emph{Volkstümliche Geschichte der Juden in drei Bänden}|pwv}. Mußt Du leſen. »\label{K_L03204-14v}\edtext{\textsc{Francesca da Rimini\pwindex{Francesca da Rimini@\emph{Francesca da Rimini}|pw}}}{\lemma{\textnormal{\emph{Francesca da Rimini}}}\Cendnote{\textnormal{Siehe A. S.: \emph{Tagebuch}, 2. 4. 1902. }}}\label{K_L03204-14}« hat
               mich bodenlos gelangweilt.\pend
           
\pstart
           Schreib’ mir bald wieder, {\pb}mein lieber Freund, und
               ſei vielmals und von Herzen gegrüßt von {\\}Deinem {\\}\spacefill\mbox{Paul Goldm}\pend
           
\pstart
           \noindent{}Was macht \textsc{Richard\pwindex{Beer-Hofmann, Richard 1866-07-11 – 1945-09-26@\textsc{Beer-Hofmann, Richard} (1866-07-11 – 1945-09-26), \emph{Schriftsteller/Schriftstellerin}|pw}}?\pend
           \selectlanguage{ngerman}\endnumbering\briefempfaengerindex{Schnitzler, Arthur@\textsc{Schnitzler, Arthur}!zzzGoldmann, Paul@\emph{von Paul Goldmann}!1902-04-171@{17. 4. {[}1902{]}}|)be}\mylabel{L03204h}  \normalsize

\doendnotes{C}
\bigskip
\vfill

\clearpage

\footnotesize

\lohead{\textsc{register}}

% Definiere theindex-Environment komplett neu ohne reledmac
\makeatletter
\renewenvironment{theindex}{%
  \section*{\indexname}%
  \setlength{\parindent}{0pt}%
  \setlength{\parskip}{0pt plus 0.3pt}%
  \let\item\@idxitem
}{%
  \clearpage
}
\makeatother

\IfFileExists{\jobname-pw.ind}{\input{\jobname-pw.ind}}{}

\end{document}

      