%% latex-leseansicht-vorspann.tex
%% Vorspann für die Leseansicht.
%% Lädt die gemeinsame Datei latex-vorspann.tex mit nicht gesetztem Schalter.

\newif\ifkorrekturansicht
\korrekturansichtfalse

\input{../tex-inputs/latex-vorspann}


\section[Arthur Schnitzler: Widmungsexemplar Der Schleier der Beatrice für Richard Dehmel, {[}1. 1. 1902?{]}]{L02550 Arthur Schnitzler: Widmungsexemplar Der Schleier der Beatrice für
               Richard Dehmel, {[}1. 1. 1902?{]}}
\nopagebreak\mylabel{L02550v}
\rehead{ }\normalsize\beginnumbering\briefempfaengerindex{Dehmel, Richard@\textsc{Dehmel, Richard}!zzzSchnitzler, Arthur@\emph{von Arthur Schnitzler}!1902-01-012@{{[}1. 1. 1902?{]}}|(be}
\toendnotes[C]{\smallbreak\pagebreak[2]}
\correspDesc{Versand  durch Arthur Schnitzler am [1. 1. 1902?] in Berlin
\newline{}Erhalt  durch Richard Dehmel am 1. 1. 1902 in Blankenese}\toendnotes[C]{\smallbreak}
\Standort{Hamburg, Staats- und Universitätsbibliothek, Dehmel, NL DA Bib:478.}
\physDesc{Widmung am Vorsatzblatt, 70 Zeichen
\newline{}Handschrift: schwarze Tinte, deutsche Kurrent
\newline{}Dehmel: an der inneren vorderen Umschlagseite Exlibris }\toendnotes[C]{\smallbreak}
\pstart
           \noindent{}{\pb}Richard Dehmel{\\}mit verehrungsvollem \label{K_L02550-1v}\edtext{Neujahrsgruſs}{\lemma{\textnormal{\emph{Neujahrsgruss}}}\Cendnote{\textnormal{Da die Antwort auf die Zusendung ebenfalls am XXXX Auszeichnungsfehler: Dokument L01194 nicht gefunden verfasst wurde,
                  ist auch eine frühere Übermittlung des Buches denkbar.}}}\label{K_L02550-1}{ }1902\pend
           \pstart \spacefill\mbox{ArthurSchnitzler}\pend{}\selectlanguage{ngerman}\vspace{1em}{\vspace{1\baselineskip}}
\pstart
           \centering{}{\pb}\textcolor{gray}{\textbf{Der Schleier der Beatrice\pwindex{Schnitzler, Arthur 15.\,5.\,1862 Wien – 21.\,10.\,1931 ebd.@\textsc{Schnitzler, Arthur} (15.\,5.\,1862 Wien – 21.\,10.\,1931 ebd.), \emph{Schriftsteller, Mediziner}!Schleier der Beatrice. Schauspiel in fünf Akten@\strich\emph{Der Schleier der Beatrice. Schauspiel in fünf Akten}|pw}}}\pend
           
\pstart
           \centering{}\textcolor{gray}{\textbf{Schauſpiel in fünf Akten}}\pend
           
\pstart
           \centering{}\textcolor{gray}{\textbf{von}}\pend
           
\pstart
           \centering{}\textcolor{gray}{\textbf{\textbf{Arthur Schnitzler}}}\pend
           
\pstart
           \centering{}\textcolor{gray}{\textbf{Zweite Auflage}}\pend
           {\vspace{1\baselineskip}}
\pstart
           \centering{}\textcolor{gray}{\textbf{\textbf{Berlin}\oindex{Berlin@\textbf{Berlin}, \emph{Hauptstadt}|pw}}}\pend
           
\pstart
           \centering{}\textcolor{gray}{\textbf{S. Fiſcher, Verlag\orgindex{S. Fischer Verlag@S. Fischer Verlag|pw}}}\pend
           
\pstart
           \centering{}\textcolor{gray}{\textbf{1901}}\pend
           \selectlanguage{ngerman}\endnumbering\briefempfaengerindex{Dehmel, Richard@\textsc{Dehmel, Richard}!zzzSchnitzler, Arthur@\emph{von Arthur Schnitzler}!1902-01-012@{{[}1. 1. 1902?{]}}|)be}\mylabel{L02550h}  \newcommand{\dateiname}{L02550}\newcommand{\titel}{Arthur Schnitzler: Widmungsexemplar Der Schleier der Beatrice für Richard Dehmel, [1. 1. 1902?]}\newcommand{\editorInnen}{Herausgegeben von Martin Anton Müller}%% latex-leseansicht-abspann.tex
%% Abspann für die Leseansicht.
%% Der Schalter \ifkorrekturansicht ist bereits durch den Vorspann gesetzt.

%% latex-abspann.tex
%% Gemeinsamer Abspann für Korrekturansicht und Leseansicht.
%% Setzt den Schalter \ifkorrekturansicht voraus (gesetzt in den
%% einbindenden Dateien latex-korrekturansicht-abspann.tex bzw.
%% latex-leseansicht-abspann.tex).
%% ---------------------------------------------------------------

\normalsize

% Das esempio-Environment wird nur in der Leseansicht benötigt
\ifkorrekturansicht\else
\newenvironment{esempio}[3]%
{
    \vspace{1.5ex}
    \rlap{\underline{#1}}
    \par
    \setlength{\parindent}{0cm}
    \nopagebreak
    \leftskip=#2cm
    \rightskip=#3cm
}
{
    \par
}
\fi

\doendnotes{C}
\bigskip
\vfill

\clearpage

\footnotesize

\ifkorrekturansicht
  \lohead{\textsc{register}}
\fi

% theindex-Environment neu definieren ohne reledmac
\makeatletter
\renewenvironment{theindex}{%
  \ifkorrekturansicht
    \section*{\indexname}%
  \else
    \subsubsection*{Index der erwähnten Entitäten}%
  \fi
  \setlength{\parindent}{0pt}%
  \setlength{\parskip}{0pt plus 0.3pt}%
  \let\item\@idxitem
}{%
  \ifkorrekturansicht\clearpage\fi
}
\makeatother

\IfFileExists{\jobname-pw.ind}{\input{\jobname-pw.ind}}{}

% Quellenangabe nur in der Leseansicht
\ifkorrekturansicht\else
% Fallback-Definitionen, falls die .tex-Datei \titel etc. nicht gesetzt hat
\providecommand{\titel}{}
\providecommand{\editorInnen}{}
\providecommand{\dateiname}{\jobname}

\vspace{3cm}

\vfill

\footnotesize
\textsc{Quelle}: \titel. Herausgegeben von {\editorInnen}. In: \emph{Arthur Schnitzler: Briefwechsel mit Autorinnen und Autoren}.
 Digitale Edition, https://schnitzler-briefe.acdh.oeaw.ac.at/{\dateiname}.html (Stand \today)
\fi

\end{document}


