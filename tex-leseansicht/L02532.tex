%% latex-leseansicht-vorspann.tex
%% Vorspann für die Leseansicht.
%% Lädt die gemeinsame Datei latex-vorspann.tex mit nicht gesetztem Schalter.

\newif\ifkorrekturansicht
\korrekturansichtfalse

\input{../tex-inputs/latex-vorspann}


               \section[Hermann Bahr an Arthur Schnitzler, 18. 2. 1930]{ Hermann Bahr an Arthur Schnitzler, 18. 2. 1930}\nopagebreak\mylabel{v}\rehead{ }\begin{ledgroupsized}[t]{13cm}\normalsize\beginnumbering\briefempfaengerindex{Schnitzler, Arthur@\textsc{Schnitzler, Arthur}!zzzBahr, Hermann@\emph{von Hermann Bahr}!1930-02-181@{18. 2. 1930}|(be} \toendnotes[C]{\smallbreak\pagebreak[2]} \Standort{CUL, Schnitzler, B 5b.}
\physDesc{Brief, 1 Blatt, 2 Seiten
\newline{}Handschrift: schwarze Tinte, deutsche Kurrent
\newline{}Schnitzler: mit rotem Buntstift beschriftet: »Bahr« \newline{}Ordnung: mit Bleistift von unbekannter Hand nummeriert: »186« }\buchAbdrucke{\weitereDrucke{Hermann Bahr, Arthur Schnitzler: \emph{Briefwechsel, Aufzeichnungen, Dokumente (1891–1931)}. Hg. Kurt Ifkovits und Martin Anton Müller. Göttingen: \emph{Wallstein} 2018, S. 595.} }\toendnotes[C]{\smallbreak}\pstart
           \raggedleft{}{\pb}München Barerſtr. 50\oindex{Barerstrasse@\textbf{Barerstraße}|pw}{\\}18. 2. 30\pend
           \pstart{}Lieber Arthur!\pend\pstart
           Wenn ich Dir für die große Freude, die mir Dein lieber Brief bereitet, nur ganz kurz
                  \strikeout{antw} danke, ſo mußt Du das mit meinem elenden
               Zuſtand entſchuldigen: ich bin ſeit Jahren ſchon immer wenn der Februar beginnt und
               ſo lange bis der April kommt, krank, ſozusagen von oben bis unten und durch und durch
               krank; eben jetzt lag ich wieder eine Woche zu Bett, und das Schlimmste daran iſt,
               daß meine Sehkraft schwindet, ich bin auf dem rechten Auge ſchon erblindend und das
               linke will ſchon auch nicht mehr recht ſeinen Pflichten genügen. »\label{K_L02532_1v}\edtext{In Bereitſchaft ſein ist alles!\pwindex{\textcolor{red}{\textsuperscript{XXXX1 indx}}!Hamlet1600@\strich\emph{Hamlet} {[}1600{]}|pwv}}{\lemma{\textnormal{\emph{In … alles!}}}\Cendnote{\textnormal{\emph{Hamlet}\pwindex{\textcolor{red}{\textsuperscript{XXXX1 indx}}!Hamlet1600@\strich\emph{Hamlet} {[}1600{]}|pwk}, V, 2: »The readiness is
                     all«.}}}\label{K_L02532_1h}«, nun ich bin bereit, aber es iſt nicht angenehm.\pend
           \pstart
           Deine Bücher habe ich alle, beſonders die Sprüche und
                  Bedenken\pwindex{Schnitzler, Arthur 15.05.1862 – 21.10.1931@\textsc{Schnitzler, Arthur} (15.05.1862 – 21.10.1931), \emph{Schriftsteller, Mediziner}!Buch der Sprueche und Bedenken1927@\strich\emph{Buch der Sprüche und Bedenken} {[}1927{]}|pw}{ }ſind mir vertraut und wenn ich nicht mit {\pb}allem »einverſtanden« bin, ſo weiß ich mich in
               alles »einzufühlen«.\pend
           \pstart
           Sag’s nicht weiter, wenn ich Dir geſtehe, daß von Jahr zu Jahr mein Heimweh nach Wien\oindex{Wien@\textbf{Wien}|pw} wächſt, faſt ſo ſtark wie das meiner Frau\pwindex{Bahr-Mildenburg, Anna 29.11.1872 – 27.01.1947@\textsc{Bahr-Mildenburg, Anna} (29.11.1872 – 27.01.1947), \emph{Sängerin}|pwv}, die vor Sehnſucht, in Wien\oindex{Wien@\textbf{Wien}|pw} zu wirken, faſt vergeht. Aber Wien\oindex{Wien@\textbf{Wien}|pw} ist vergeßlich und ſo werden wir wohl in der Verbannung
               ſterben.\pend
           \pstart
           Herzlichſt Dein gedenkend, auch die paar Freunde, die noch meiner gedenken, beſtens
               grüßend\pend
           \pstart
           Dein alter, allzu alter{\\[\baselineskip]}\spacefill\mbox{Hermann}\pend
           \leftskip=0em{}\endnumbering\briefempfaengerindex{Schnitzler, Arthur@\textsc{Schnitzler, Arthur}!zzzBahr, Hermann@\emph{von Hermann Bahr}!1930-02-181@{18. 2. 1930}|)be}\mylabel{h}\end{ledgroupsized}  \newcommand{\dateiname}{L02532}\newcommand{\titel}{Hermann Bahr an Arthur Schnitzler, 18. 2. 1930}\newcommand{\editorInnen}{ Kurt Ifkovits,  Martin Anton Müller}%% latex-leseansicht-abspann.tex
%% Abspann für die Leseansicht.
%% Der Schalter \ifkorrekturansicht ist bereits durch den Vorspann gesetzt.

%% latex-abspann.tex
%% Gemeinsamer Abspann für Korrekturansicht und Leseansicht.
%% Setzt den Schalter \ifkorrekturansicht voraus (gesetzt in den
%% einbindenden Dateien latex-korrekturansicht-abspann.tex bzw.
%% latex-leseansicht-abspann.tex).
%% ---------------------------------------------------------------

\normalsize

% Das esempio-Environment wird nur in der Leseansicht benötigt
\ifkorrekturansicht\else
\newenvironment{esempio}[3]%
{
    \vspace{1.5ex}
    \rlap{\underline{#1}}
    \par
    \setlength{\parindent}{0cm}
    \nopagebreak
    \leftskip=#2cm
    \rightskip=#3cm
}
{
    \par
}
\fi

\doendnotes{C}
\bigskip
\vfill

\clearpage

\footnotesize

\ifkorrekturansicht
  \lohead{\textsc{register}}
\fi

% theindex-Environment neu definieren ohne reledmac
\makeatletter
\renewenvironment{theindex}{%
  \ifkorrekturansicht
    \section*{\indexname}%
  \else
    \subsubsection*{Index der erwähnten Entitäten}%
  \fi
  \setlength{\parindent}{0pt}%
  \setlength{\parskip}{0pt plus 0.3pt}%
  \let\item\@idxitem
}{%
  \ifkorrekturansicht\clearpage\fi
}
\makeatother

\IfFileExists{\jobname-pw.ind}{\input{\jobname-pw.ind}}{}

% Quellenangabe nur in der Leseansicht
\ifkorrekturansicht\else
% Fallback-Definitionen, falls die .tex-Datei \titel etc. nicht gesetzt hat
\providecommand{\titel}{}
\providecommand{\editorInnen}{}
\providecommand{\dateiname}{\jobname}

\vspace{3cm}

\vfill

\footnotesize
\textsc{Quelle}: \titel. Herausgegeben von {\editorInnen}. In: \emph{Arthur Schnitzler: Briefwechsel mit Autorinnen und Autoren}.
 Digitale Edition, https://schnitzler-briefe.acdh.oeaw.ac.at/{\dateiname}.html (Stand \today)
\fi

\end{document}


      