%% latex-korrekturansicht-vorspann.tex
%% Vorspann für die Korrekturansicht.
%% Lädt die gemeinsame Datei latex-vorspann.tex mit gesetztem Schalter.

\newif\ifkorrekturansicht
\korrekturansichttrue

\input{../tex-inputs/latex-vorspann}


\section[Arthur Schnitzler an Hugo von Hofmannsthal, {[}4. 6. 1898{]}]{L00802 Arthur Schnitzler an Hugo von Hofmannsthal, {[}4. 6. 1898{]}}
\nopagebreak\mylabel{L00802v}
\rehead{ }\normalsize\beginnumbering\briefempfaengerindex{Hofmannsthal, Hugo von@\textsc{Hofmannsthal, Hugo von}!zzzSchnitzler, Arthur@\emph{von Arthur Schnitzler}!1898-06-042@{{[}4. 6. 1898{]}}|(be}
\toendnotes[C]{\smallbreak\pagebreak[2]}\Standort{FDH, Hs-30885,66.}
\physDesc{Brief, 1 Blatt, 4 Seiten, 894 Zeichen
\newline{}Handschrift: Bleistift, deutsche Kurrent
\newline{}Ordnung: mit Bleistift von Schnitzler mutmaßlich bei der Durchsicht der Korrespondenz
                                    1929 datiert: »Anf? 98« }
\buchAbdrucke{\weitereDrucke{Hugo von Hofmannsthal, Arthur Schnitzler: \emph{Briefwechsel}. Frankfurt am Main: \emph{S. Fischer} 1964, S. 102.} }
\pstart
           \raggedleft{}{\pb}Samſtag.\pend
           \vspace{0.5em}
\pstart
           Lieber Hugo, morgen früh will ich auf den Semmering\oindex{Semmering@\textbf{Semmering}, \emph{A.ADM3}|pw} fahren, dann \textsc{per} Rad zum Richard\pwindex{Beer-Hofmann, Richard 1866-07-11 – 1945-09-26@\textsc{Beer-Hofmann, Richard} (1866-07-11 – 1945-09-26), \emph{Schriftsteller/Schriftstellerin}|pw}, wo ich wohl
                  Dinſtag{ }ſein werde. Wahrſcheinlich fahr ich allein; \textsc{Kramer}\pwindex{Kramer, Leopold 29.09.1869 – 29.10.1942@\textsc{Kramer, Leopold} (29.09.1869 – 29.10.1942), \emph{Theaterleiter/Theaterleiterin, Schauspieler/Schauspielerin}|pw}{ }ſcheint {\pb}unverläßlich.
               Daſs Sie \textsc{Kerr}\pwindex{Kerr, Alfred 25.12.1867 – 12.10.1948@\textsc{Kerr, Alfred} (25.12.1867 – 12.10.1948), \emph{Schriftsteller/Schriftstellerin, Kritiker/Kritikerin}|pw} nicht kennen gelernt haben, iſt ſchade; im Anfang befangen und etwas unſicher
               findet er ſich bald bei einigem Entgegenko{\geminationm}en und wirkt
               durch ſeinen Verſtand, ſeine Sympathie und mannigfache {\pb}günſtige Intentionen höchſt erfreulich. –\pend
           
\pstart
           Es geht mir mit der Sti{\geminationm}ung nun etwas beſſer; es iſt
               doch ſehr ſonderbar, wie auch \strikeout{\textcolor{gray}{ganz feſtſtehende}} ihrem Weſen nach unveränderliche ſeeliſche Laſten an Schwere gewinnen und
               verlieren können. – Ich möchte auch in Kärnthen\oindex{Kaernten@\textbf{Kärnten}, \emph{A.ADM1}|pw}{ }{\pb}ein bischen arbeiten. Sie können mir jedenfalls nach \textsc{Steindorf}\oindex{Steindorf am Ossiacher See@\textbf{Steindorf am Ossiacher See}, \emph{A.ADM3}|pw} zu R.\pwindex{Beer-Hofmann, Richard 1866-07-11 – 1945-09-26@\textsc{Beer-Hofmann, Richard} (1866-07-11 – 1945-09-26), \emph{Schriftsteller/Schriftstellerin}|pw}{ }ſchreiben; obzwar ich nicht glaube, dſs ich dort
               bleibe.\pend
           
\pstart
           Brahm\pwindex{Brahm, Otto 05.02.1856 – 28.11.1912@\textsc{Brahm, Otto} (05.02.1856 – 28.11.1912), \emph{Theaterleiter/Theaterleiterin, Regisseur/Regisseurin}|pw} läßt Sie vielmals grüßen; er hofft Sie
               werden noch oft Gelegenhei\textcolor{gray}{t} haben ſich am Dtſch Theater\orgindex{Deutsches Theater Berlin@Deutsches Theater Berlin|pw} wohl zu fühlen.\pend
           \pstart Herzlichſte Grüße Ihr \spacefill\mbox{A.}\pend{}\selectlanguage{ngerman}\endnumbering\briefempfaengerindex{Hofmannsthal, Hugo von@\textsc{Hofmannsthal, Hugo von}!zzzSchnitzler, Arthur@\emph{von Arthur Schnitzler}!1898-06-042@{{[}4. 6. 1898{]}}|)be}\mylabel{L00802h}  \normalsize

\doendnotes{C}
\bigskip
\vfill

\clearpage

\footnotesize

\lohead{\textsc{register}}

% Definiere theindex-Environment komplett neu ohne reledmac
\makeatletter
\renewenvironment{theindex}{%
  \section*{\indexname}%
  \setlength{\parindent}{0pt}%
  \setlength{\parskip}{0pt plus 0.3pt}%
  \let\item\@idxitem
}{%
  \clearpage
}
\makeatother

\IfFileExists{\jobname-pw.ind}{\input{\jobname-pw.ind}}{}

\end{document}

      