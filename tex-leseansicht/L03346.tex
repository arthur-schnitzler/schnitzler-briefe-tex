%% latex-korrekturansicht-vorspann.tex
%% Vorspann für die Korrekturansicht.
%% Lädt die gemeinsame Datei latex-vorspann.tex mit gesetztem Schalter.

\newif\ifkorrekturansicht
\korrekturansichttrue

\input{../tex-inputs/latex-vorspann}


\section[ Felix Salten an Arthur Schnitzler, 29. 9. 1903]{L03346 Felix Salten an Arthur Schnitzler, 29. 9. 1903}
\nopagebreak\mylabel{L03346v}
\rehead{ }\normalsize\beginnumbering\briefempfaengerindex{Schnitzler, Arthur@\textsc{Schnitzler, Arthur}!zzzSalten, Felix@\emph{von Felix Salten}!1903-09-292@{29. 9. 1903}|(be}
\toendnotes[C]{\smallbreak\pagebreak[2]}\Standort{CUL, Schnitzler, B 89, A 2.}
\physDesc{Karte, 334 Zeichen
\newline{}Handschrift: Bleistift, lateinische Kurrent
\newline{}Ordnung: mit Bleistift von unbekannter Hand nummeriert: »{\pb}17\substVorne{}\textsuperscript{0}\substDazwischen{}1\substHinten{}« }\toendnotes[C]{\smallbreak}
\pstart{}{\pb}Lieber,\pend\vspace{0.5em}
\pstart
           vielleicht ist es Ihnen oder Ihrer \label{K_L03346-1v}\edtext{Frau\pwindex{Schnitzler, Olga 17.01.1882 – 13.01.1970@\textsc{Schnitzler, Olga} (17.01.1882 – 13.01.1970), \emph{Schauspieler/Schauspielerin, Sänger/Sängerin}|pwv}}{\lemma{\textnormal{\emph{Frau}}}\Cendnote{\textnormal{Olga Schnitzler\pwindex{Schnitzler, Olga 17.01.1882 – 13.01.1970@\textsc{Schnitzler, Olga} (17.01.1882 – 13.01.1970), \emph{Schauspieler/Schauspielerin, Sänger/Sängerin}|pwk} hatte einen Brief von Mirjam Horwitz’ Vater\pwindex{Horwitz, Theodor 1845/1846 – 1913-02-08@\textsc{Horwitz, Theodor} (1845/1846 – 1913-02-08)|pwkv}
                  erhalten (vgl. A. S.: \emph{Tagebuch}, 20. 9. 1903).
                     Vgl. Felix Salten an Arthur Schnitzler, 3. 3. 1903.}}}\label{K_L03346-1} von
               Interesse, dass Mirjam\pwindex{Horwitz, Mirjam 1882-06-15 – 1967-09-26@\textsc{Horwitz, Mirjam} (1882-06-15 – 1967-09-26), \emph{Theaterleiter/Theaterleiterin, Schauspieler/Schauspielerin}|pw} bei ihren Eltern\pwindex{Horwitz, Theodor 1845/1846 – 1913-02-08@\textsc{Horwitz, Theodor} (1845/1846 – 1913-02-08)|pwv}\pwindex{Horwitz, Emma @\textsc{Horwitz, Emma}|pwv} bleibt.
               Dazu dürften neben dem Brief Ihrer Frau\pwindex{Schnitzler, Olga 17.01.1882 – 13.01.1970@\textsc{Schnitzler, Olga} (17.01.1882 – 13.01.1970), \emph{Schauspieler/Schauspielerin, Sänger/Sängerin}|pwv} an Mirjams
                  Vater\pwindex{Horwitz, Theodor 1845/1846 – 1913-02-08@\textsc{Horwitz, Theodor} (1845/1846 – 1913-02-08)|pwv}, wiederholte dringende Briefe von mir an Mirjam\pwindex{Horwitz, Mirjam 1882-06-15 – 1967-09-26@\textsc{Horwitz, Mirjam} (1882-06-15 – 1967-09-26), \emph{Theaterleiter/Theaterleiterin, Schauspieler/Schauspielerin}|pw} beigetragen haben. Für den Fall, dass M.\pwindex{Horwitz, Mirjam 1882-06-15 – 1967-09-26@\textsc{Horwitz, Mirjam} (1882-06-15 – 1967-09-26), \emph{Theaterleiter/Theaterleiterin, Schauspieler/Schauspielerin}|pw} Sie davon noch nicht in Kenntnis gesetzt hat, sende ich
               Ihnen diese Mittheilung,\pend
           
\pstart
           herzl. {\\[\baselineskip]}\spacefill\mbox{Salten.}\pend
           \leftskip=0em{}
\pstart
           29/IX. 03\pend
           \selectlanguage{ngerman}\endnumbering\briefempfaengerindex{Schnitzler, Arthur@\textsc{Schnitzler, Arthur}!zzzSalten, Felix@\emph{von Felix Salten}!1903-09-292@{29. 9. 1903}|)be}\mylabel{L03346h}  \normalsize

\doendnotes{C}
\bigskip
\vfill

\clearpage

\footnotesize

\lohead{\textsc{register}}

% Definiere theindex-Environment komplett neu ohne reledmac
\makeatletter
\renewenvironment{theindex}{%
  \section*{\indexname}%
  \setlength{\parindent}{0pt}%
  \setlength{\parskip}{0pt plus 0.3pt}%
  \let\item\@idxitem
}{%
  \clearpage
}
\makeatother

\IfFileExists{\jobname-pw.ind}{\input{\jobname-pw.ind}}{}

\end{document}

      