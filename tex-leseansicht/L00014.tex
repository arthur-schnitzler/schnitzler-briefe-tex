\input{../tex-inputs/latex-pdf-vorspann}
\begin{center}
            \textcolor{red}{ENTWURF. ENTZIFFERUNG NOCH NICHT KORREKTURGELESEN}
                      \end{center}
            
               \section[Arthur Schnitzler an Max Burckhard, {[}20.{]} 5. 1891]{ Arthur Schnitzler an Max Burckhard, {[}20.{]} 5. 1891}\nopagebreak\mylabel{v}\rehead{ }\begin{ledgroupsized}[t]{13cm}\normalsize\beginnumbering\briefempfaengerindex{Burckhard, Max Eugen@\textsc{Burckhard, Max Eugen}!zzzSchnitzler, Arthur@\emph{von Arthur Schnitzler}!1891-05-201@{{[}20.{]} 5. 1891}|(be} \toendnotes[C]{\smallbreak\pagebreak[2]} \buchAlsQuelle{\pwindex{Glossy, Karl 07.03.1848 – 09.09.1937@\textsc{Glossy, Karl} (07.03.1848 – 09.09.1937), \emph{Schriftsteller, Museumsleiter, Zensurbeirat}!Schnitzlers Einzug ins Burgtheater19. 12. 1931@\strich\emph{Schnitzlers Einzug ins Burgtheater} {[}19. 12. 1931{]}|pwk}\pwindex{Neue Freie Presse1864 – 1939@\emph{Neue Freie Presse}|pwk}Karl Glossy: \emph{Schnitzlers Einzug ins Burgtheater. Unbekannte Briefe des Dichters.} In: \emph{Neue Freie Presse}, Nr. 24162, 19. 12. 1931, S. 14.}\buchAbdrucke{\weitereDrucke{1) \pwindex{Glossy, Karl 07.03.1848 – 09.09.1937@\textsc{Glossy, Karl} (07.03.1848 – 09.09.1937), \emph{Schriftsteller, Museumsleiter, Zensurbeirat}!Schnitzlers Einzug ins Burgtheater19. 12. 1931@\strich\emph{Schnitzlers Einzug ins Burgtheater} {[}19. 12. 1931{]}|pwk}Karl Glossy: \emph{Schnitzlers Einzug ins Burgtheater. Unbekannte Briefe des Dichters.} In: \emph{Wiener Studien und Dokumente}. Zum 85. Geburtstag des Verfassers hg. von seinen Freunden. Wien: \emph{Steyrermühl} 1933, S. 166–168.} \weitereDrucke{2) Hans-Ulrich Lindken: \emph{Arthur Schnitzler. Aspekte und Akzente. Materialien zu Leben
                        und Werk}. Frankfurt am Main, Bern, Göttingen: \emph{Peter Lang} 1984, S. 243–246 (Europäische Hochschulschriften, Reihe 1, Deutsche Sprache und
                        Literatur, 754).} }\toendnotes[C]{\smallbreak}\pstart
           \noindent{}{\pb}\so{Schnitzler an Burckhard}, \label{K_L00014_1v}\edtext{Mai 1891}{\lemma{\textnormal{\emph{Mai 1891}}}\Cendnote{\textnormal{vgl. A. S.: \emph{Tagebuch}, 20. 5. 1891}}}\label{K_L00014_1h}: »Sehr
               geehrter Herr Direktor! Erlauben Sie mir, Ihnen beifolgend ein einaktiges
               dramatiſches Gedicht, »Alkandis Lied\pwindex{Schnitzler, Arthur 15.05.1862 – 21.10.1931@\textsc{Schnitzler, Arthur} (15.05.1862 – 21.10.1931), \emph{Schriftsteller, Mediziner}!Alkandi s Lied15.8.1890 – 1.9.1890@\strich\emph{Alkandi’s Lied} {[}15.8.1890 – 1.9.1890{]}|pw}«, vorzulegen.
               Vielleicht halten Sie es einer Aufführung für würdig; möglicherweiſe gibt Ihnen, ſehr
               geehrter Herr Direktor, die Lektüre des Stückes Anlaß zu der einen oder anderen
               Bemerkung, auf die ich Gewicht zu legen hätte. Jedenfalls, verehrter Herr, würden Sie
               mich unendlich verpflichten, wenn Sie dem Stücke\pwindex{Schnitzler, Arthur 15.05.1862 – 21.10.1931@\textsc{Schnitzler, Arthur} (15.05.1862 – 21.10.1931), \emph{Schriftsteller, Mediziner}!Alkandi s Lied15.8.1890 – 1.9.1890@\strich\emph{Alkandi’s Lied} {[}15.8.1890 – 1.9.1890{]}|pwv}, welches in Breslau\oindex{Breslau@\textbf{Breslau}|pw}
               zur Aufführung kommen \label{T_L00014_1v}\edtext{dürfte}{\lemma{\textnormal{\emph{dürfte}}}\Cendnote{\textnormal{korrigiert aus »durfte«. Da es zu der
                  Inszenierung nicht gekommen ist, ist anzunehmen, dass im nicht überlieferten
                  Original der Konjunktiv »dürfte« steht.}}}\label{T_L00014_1h}, gelegentlich eine
               Viertelſtunde Ihrer koſtbaren Zeit widmeten und mir gütigſt mitteilen wollten, ob und
               wann ich mir bei Ihnen Beſcheid holen dürfte. Hochachtungsvoll Dr. Arthur
               Schnitzler.«\pend
           \endnumbering\briefempfaengerindex{Burckhard, Max Eugen@\textsc{Burckhard, Max Eugen}!zzzSchnitzler, Arthur@\emph{von Arthur Schnitzler}!1891-05-201@{{[}20.{]} 5. 1891}|)be}\mylabel{h}\end{ledgroupsized}  \newcommand{\dateiname}{L00014}\newcommand{\titel}{Arthur Schnitzler an Max Burckhard, [20.] 5. 1891}\newcommand{\editorInnen}{Martin Anton Müller und Gerd-Hermann Susen}\input{../tex-inputs/latex-pdf-abspann}
      