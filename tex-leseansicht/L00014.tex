%% latex-leseansicht-vorspann.tex
%% Vorspann für die Leseansicht.
%% Lädt die gemeinsame Datei latex-vorspann.tex mit nicht gesetztem Schalter.

\newif\ifkorrekturansicht
\korrekturansichtfalse

\input{../tex-inputs/latex-vorspann}


         
         \renewcommand{\erwaehntePersonen}{Personen: Max Eugen Burckhard, Karl Glossy}
         \renewcommand{\erwaehnteOrte}{Orte: Breslau, Wien}
         \renewcommand{\erwaehnteWerke}{Werke: Alkandi’s Lied, Neue Freie Presse, Schnitzlers Einzug ins Burgtheater}
               \section[Arthur Schnitzler an Max Burckhard, {[}20.{]} 5. 1891]{ Arthur Schnitzler an Max Burckhard, {[}20.{]} 5. 1891}\nopagebreak\mylabel{v}\rehead{ }\begin{ledgroupsized}[t]{13cm}\normalsize\beginnumbering \toendnotes[C]{\smallbreak\pagebreak[2]} \buchAlsQuelle{\pwindex{Glossy, Karl 07.03.1848 – 09.09.1937@\textsc{Glossy, Karl} (07.03.1848 – 09.09.1937), \emph{Schriftsteller, Museumsleiter, Zensurbeirat}!Schnitzlers Einzug ins Burgtheater19. 12. 1931@\strich\emph{Schnitzlers Einzug ins Burgtheater} {[}19. 12. 1931{]}|pwk}\pwindex{Neue Freie Presse1864 – 1939@\emph{Neue Freie Presse} {[}1864 – 1939{]}|pwk}Karl Glossy: \emph{Schnitzlers Einzug ins Burgtheater. Unbekannte Briefe des Dichters.} In: \emph{Neue Freie Presse}, Nr. 24162, 19. 12. 1931, S. 14.}\buchAbdrucke{\weitereDrucke{1) \pwindex{Glossy, Karl 07.03.1848 – 09.09.1937@\textsc{Glossy, Karl} (07.03.1848 – 09.09.1937), \emph{Schriftsteller, Museumsleiter, Zensurbeirat}!Schnitzlers Einzug ins Burgtheater19. 12. 1931@\strich\emph{Schnitzlers Einzug ins Burgtheater} {[}19. 12. 1931{]}|pwk}Karl Glossy: \emph{Schnitzlers Einzug ins Burgtheater. Unbekannte Briefe des Dichters.} In: \emph{Wiener Studien und Dokumente}. Zum 85. Geburtstag des Verfassers hg. von seinen Freunden. Wien: \emph{Steyrermühl} 1933, S. 166–168.} \weitereDrucke{2) Hans-Ulrich Lindken: \emph{Arthur Schnitzler. Aspekte und Akzente. Materialien zu Leben
                        und Werk}. Frankfurt am Main, Bern, Göttingen: \emph{Peter Lang} 1984, S. 243–246 (Europäische Hochschulschriften, Reihe 1, Deutsche Sprache und
                        Literatur, 754).} }\toendnotes[C]{\smallbreak}\pstart
           \noindent{}{\pb}\so{Schnitzler an Burckhard}, \label{K_L00014-1v}\edtext{Mai 1891}{\lemma{\textnormal{\emph{Mai 1891}}}\Cendnote{\textnormal{vgl. A. S.: \emph{Tagebuch}, 20. 5. 1891}}}\label{K_L00014-1h}: »Sehr geehrter Herr Direktor! Erlauben Sie mir, Ihnen beifolgend ein
               einaktiges dramatiſches Gedicht, »Alkandis
               Lied\pwindex{Schnitzler, Arthur 15.05.1862 – 21.10.1931@\textsc{Schnitzler, Arthur} (15.05.1862 – 21.10.1931), \emph{Schriftsteller, Mediziner}!Alkandi s Lied15.8.1890 – 1.9.1890@\strich\emph{Alkandi’s Lied} {[}15.8.1890 – 1.9.1890{]}|pw}«, vorzulegen. Vielleicht halten Sie es einer Aufführung für würdig;
               möglicherweiſe gibt Ihnen, ſehr geehrter Herr Direktor, die Lektüre des Stückes Anlaß
               zu der einen oder anderen Bemerkung, auf die ich Gewicht zu legen hätte. Jedenfalls,
               verehrter Herr, würden Sie mich unendlich verpflichten, wenn Sie dem Stücke\pwindex{Schnitzler, Arthur 15.05.1862 – 21.10.1931@\textsc{Schnitzler, Arthur} (15.05.1862 – 21.10.1931), \emph{Schriftsteller, Mediziner}!Alkandi s Lied15.8.1890 – 1.9.1890@\strich\emph{Alkandi’s Lied} {[}15.8.1890 – 1.9.1890{]}|pwv}, welches in Breslau\oindex{Breslau@\textbf{Breslau}|pw} zur Aufführung kommen \label{T_L00014-1v}\edtext{dürfte}{\lemma{\textnormal{\emph{dürfte}}}\Cendnote{\textnormal{korrigiert aus »durfte«. Da es zu der Inszenierung nicht
                  gekommen ist, ist anzunehmen, dass im nicht überlieferten Original der Konjunktiv
                     »dürfte« steht.}}}\label{T_L00014-1h}, gelegentlich eine Viertelſtunde Ihrer
               koſtbaren Zeit widmeten und mir gütigſt mitteilen wollten, ob und wann ich mir bei
               Ihnen Beſcheid holen dürfte. Hochachtungsvoll Dr. Arthur Schnitzler.«\pend
           
         
         \endnumbering\mylabel{h}\end{ledgroupsized}  \newcommand{\dateiname}{L00014}\newcommand{\titel}{Arthur Schnitzler an Max Burckhard, [20.] 5. 1891}\newcommand{\editorInnen}{Martin Anton Müller und Gerd-Hermann Susen}%% latex-leseansicht-abspann.tex
%% Abspann für die Leseansicht.
%% Der Schalter \ifkorrekturansicht ist bereits durch den Vorspann gesetzt.

%% latex-abspann.tex
%% Gemeinsamer Abspann für Korrekturansicht und Leseansicht.
%% Setzt den Schalter \ifkorrekturansicht voraus (gesetzt in den
%% einbindenden Dateien latex-korrekturansicht-abspann.tex bzw.
%% latex-leseansicht-abspann.tex).
%% ---------------------------------------------------------------

\normalsize

% Das esempio-Environment wird nur in der Leseansicht benötigt
\ifkorrekturansicht\else
\newenvironment{esempio}[3]%
{
    \vspace{1.5ex}
    \rlap{\underline{#1}}
    \par
    \setlength{\parindent}{0cm}
    \nopagebreak
    \leftskip=#2cm
    \rightskip=#3cm
}
{
    \par
}
\fi

\doendnotes{C}
\bigskip
\vfill

\clearpage

\footnotesize

\ifkorrekturansicht
  \lohead{\textsc{register}}
\fi

% theindex-Environment neu definieren ohne reledmac
\makeatletter
\renewenvironment{theindex}{%
  \ifkorrekturansicht
    \section*{\indexname}%
  \else
    \subsubsection*{Index der erwähnten Entitäten}%
  \fi
  \setlength{\parindent}{0pt}%
  \setlength{\parskip}{0pt plus 0.3pt}%
  \let\item\@idxitem
}{%
  \ifkorrekturansicht\clearpage\fi
}
\makeatother

\IfFileExists{\jobname-pw.ind}{\input{\jobname-pw.ind}}{}

% Quellenangabe nur in der Leseansicht
\ifkorrekturansicht\else
% Fallback-Definitionen, falls die .tex-Datei \titel etc. nicht gesetzt hat
\providecommand{\titel}{}
\providecommand{\editorInnen}{}
\providecommand{\dateiname}{\jobname}

\vspace{3cm}

\vfill

\footnotesize
\textsc{Quelle}: \titel. Herausgegeben von {\editorInnen}. In: \emph{Arthur Schnitzler: Briefwechsel mit Autorinnen und Autoren}.
 Digitale Edition, https://schnitzler-briefe.acdh.oeaw.ac.at/{\dateiname}.html (Stand \today)
\fi

\end{document}


      