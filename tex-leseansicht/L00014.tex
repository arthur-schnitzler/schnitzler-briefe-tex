%% latex-korrekturansicht-vorspann.tex
%% Vorspann für die Korrekturansicht.
%% Lädt die gemeinsame Datei latex-vorspann.tex mit gesetztem Schalter.

\newif\ifkorrekturansicht
\korrekturansichttrue

\input{../tex-inputs/latex-vorspann}


\section[Arthur Schnitzler an Max Burckhard, {[}20.{]} 5. 1891]{L00014 Arthur Schnitzler an Max Burckhard, {[}20.{]} 5. 1891}
\nopagebreak\mylabel{L00014v}
\rehead{ }\normalsize\beginnumbering\briefempfaengerindex{Burckhard, Max Eugen@\textsc{Burckhard, Max Eugen}!zzzSchnitzler, Arthur@\emph{von Arthur Schnitzler}!1891-05-201@{{[}20.{]} 5. 1891}|(be}
\toendnotes[C]{\smallbreak\pagebreak[2]}\buchAlsQuelle{\pwindex{Schnitzlers Einzug ins Burgtheater@\emph{Schnitzlers Einzug ins Burgtheater}|pwk}\pwindex{Neue Freie Presse@\emph{Neue Freie Presse}|pwk}\emph{Neue Freie Presse}, Nr. 24162, 19. 12. 1931, S. 14.}
\buchAbdrucke{\weitereDrucke{1) \pwindex{Schnitzlers Einzug ins Burgtheater@\emph{Schnitzlers Einzug ins Burgtheater}|pwk}\emph{Wiener Studien und Dokumente}. Wien: \emph{Steyrermühl} 1933, S. 166–168.} \weitereDrucke{2) Hans-Ulrich Lindken: \emph{Arthur Schnitzler. Aspekte und Akzente. Materialien zu Leben
                        und Werk}. Frankfurt am Main, Bern, Göttingen: \emph{Peter Lang} 1984, S. 243–246.} }\toendnotes[C]{\smallbreak}
\pstart
           \noindent{}{\pb}\so{Schnitzler an Burckhard}, \label{K_L00014-1v}\edtext{Mai 1891}{\lemma{\textnormal{\emph{Mai 1891}}}\Cendnote{\textnormal{Vgl. A. S.: \emph{Tagebuch}, 20. 5. 1891.
               }}}\label{K_L00014-1}: »Sehr geehrter Herr Direktor! Erlauben Sie mir, Ihnen beifolgend ein
               einaktiges dramatiſches Gedicht, »Alkandis
               Lied\pwindex{Alkandi s Lied@\emph{Alkandi’s Lied}|pw}«, vorzulegen. Vielleicht halten Sie es einer Aufführung für würdig;
               möglicherweiſe gibt Ihnen, ſehr geehrter Herr Direktor, die Lektüre des Stückes Anlaß
               zu der einen oder anderen Bemerkung, auf die ich Gewicht zu legen hätte. Jedenfalls,
               verehrter Herr, würden Sie mich unendlich verpflichten, wenn Sie dem Stücke\pwindex{Alkandi s Lied@\emph{Alkandi’s Lied}|pwv}, welches in Breslau\oindex{Breslau@\textbf{Breslau}, \emph{P.PPLA}|pw} zur Aufführung kommen \label{T_L00014-1v}\edtext{dürfte}{\lemma{\textnormal{\emph{dürfte}}}\Cendnote{\textnormal{korrigiert aus »durfte«. Da es zu der Inszenierung nicht
                  gekommen ist, ist anzunehmen, dass im nicht überlieferten Original der Konjunktiv
                     »dürfte« steht.}}}\label{T_L00014-1}, gelegentlich eine Viertelſtunde Ihrer
               koſtbaren Zeit widmeten und mir gütigſt mitteilen wollten, ob und wann ich mir bei
               Ihnen Beſcheid holen dürfte. Hochachtungsvoll Dr. Arthur Schnitzler.«\pend
           \selectlanguage{ngerman}\endnumbering\briefempfaengerindex{Burckhard, Max Eugen@\textsc{Burckhard, Max Eugen}!zzzSchnitzler, Arthur@\emph{von Arthur Schnitzler}!1891-05-201@{{[}20.{]} 5. 1891}|)be}\mylabel{L00014h}  \normalsize

\doendnotes{C}
\bigskip
\vfill

\clearpage

\footnotesize

\lohead{\textsc{register}}

% Definiere theindex-Environment komplett neu ohne reledmac
\makeatletter
\renewenvironment{theindex}{%
  \section*{\indexname}%
  \setlength{\parindent}{0pt}%
  \setlength{\parskip}{0pt plus 0.3pt}%
  \let\item\@idxitem
}{%
  \clearpage
}
\makeatother

\IfFileExists{\jobname-pw.ind}{\input{\jobname-pw.ind}}{}

\end{document}

      