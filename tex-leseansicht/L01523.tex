%% latex-korrekturansicht-vorspann.tex
%% Vorspann für die Korrekturansicht.
%% Lädt die gemeinsame Datei latex-vorspann.tex mit gesetztem Schalter.

\newif\ifkorrekturansicht
\korrekturansichttrue

\input{../tex-inputs/latex-vorspann}


\section[Arthur Schnitzler an Hermann Bahr, 6. 6. 1905]{L01523 Arthur Schnitzler an Hermann Bahr, 6. 6. 1905}
\nopagebreak\mylabel{L01523v}
\rehead{ }\normalsize\beginnumbering\briefempfaengerindex{Bahr, Hermann@\textsc{Bahr, Hermann}!zzzSchnitzler, Arthur@\emph{von Arthur Schnitzler}!1905-06-061@{6. 6. 1905}|(be}
\toendnotes[C]{\smallbreak\pagebreak[2]}\Standort{TMW, HS AM 23374 Ba.}
\physDesc{Brief, 1 Blatt, 2 Seiten, 625 Zeichen
\newline{}Handschrift: schwarze Tinte, deutsche Kurrent
\newline{}Ordnung: Lochung }
\buchAbdrucke{\weitereDrucke{1) Arthur Schnitzler: \emph{The Letters of Arthur Schnitzler to Hermann Bahr}. Chapel Hill: \emph{The University of North Carolina Press} 1978, S. 89.} \weitereDrucke{2) Hermann Bahr, Arthur Schnitzler: \emph{Briefwechsel, Aufzeichnungen, Dokumente (1891–1931)}. Göttingen: \emph{Wallstein} 2018, S. 345.} }\toendnotes[C]{\smallbreak}
\pstart
           \raggedleft{}{\pb}\textsc{Wien\oindex{Wien@\textbf{Wien}, \emph{A.ADM2}|pw}}{ }6. Juni 905\pend
           
\pstart{}lieber Hermann\pend\vspace{0.5em}
\pstart
           ich gratulire dir herzlich zum geſtrigen Erfolg von \textsc{Sanna\pwindex{Sanna. Schauspiel in fuenf Aufzuegen@\emph{Sanna. Schauspiel in fünf Aufzügen}|pw}}. Einiges was mir nach der erſten Lectüre des Stücks nicht ganz eingeleuchtet,
               iſt mir geſtern, ſchön und ergreifend aufgegangen. Die Aufführung war etwas ganz
               einziges, und die Höflich\pwindex{Hoeflich, Lucie 20.02.1883 – 08.10.1956@\textsc{Höflich, Lucie} (20.02.1883 – 08.10.1956), \emph{Schauspieler/Schauspielerin}|pw}{ }{\pb}iſt – vielleicht nicht
               das echte Genie, aber, nach ihren Entwicklungsmöglichkeiten in alles tragiſche und
               heitre Gebiet, der größte Glücksfall, den die Deutſche Bühne ſeit der Sorma\pwindex{Sorma, Agnes 17.05.1862 – 10.02.1927@\textsc{Sorma, Agnes} (17.05.1862 – 10.02.1927), \emph{Schauspieler/Schauspielerin}|pw} erlebt hat.\pend
           
\pstart
           Ich habe mich ſehr gefreut, auch meine Frau\pwindex{Schnitzler, Olga 17.01.1882 – 13.01.1970@\textsc{Schnitzler, Olga} (17.01.1882 – 13.01.1970), \emph{Schauspieler/Schauspielerin, Sänger/Sängerin}|pwv} läßt dir von Herzen glückwünſchen.\pend
           
\pstart
           Hoffentlich ſeh ich dich bald; ich habe ein rechtes Bedürfnis, dir zu danken.\pend
           
\pstart
           Dein{\\[\baselineskip]}\spacefill\mbox{Arthur}\pend
           \leftskip=0em{}\selectlanguage{ngerman}\endnumbering\briefempfaengerindex{Bahr, Hermann@\textsc{Bahr, Hermann}!zzzSchnitzler, Arthur@\emph{von Arthur Schnitzler}!1905-06-061@{6. 6. 1905}|)be}\mylabel{L01523h}  \normalsize

\doendnotes{C}
\bigskip
\vfill

\clearpage

\footnotesize

\lohead{\textsc{register}}

% Definiere theindex-Environment komplett neu ohne reledmac
\makeatletter
\renewenvironment{theindex}{%
  \section*{\indexname}%
  \setlength{\parindent}{0pt}%
  \setlength{\parskip}{0pt plus 0.3pt}%
  \let\item\@idxitem
}{%
  \clearpage
}
\makeatother

\IfFileExists{\jobname-pw.ind}{\input{\jobname-pw.ind}}{}

\end{document}

      