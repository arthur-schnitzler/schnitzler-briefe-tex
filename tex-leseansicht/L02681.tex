%% latex-korrekturansicht-vorspann.tex
%% Vorspann für die Korrekturansicht.
%% Lädt die gemeinsame Datei latex-vorspann.tex mit gesetztem Schalter.

\newif\ifkorrekturansicht
\korrekturansichttrue

\input{../tex-inputs/latex-vorspann}


\section[Paul Goldmann an Arthur Schnitzler, 9. 6. {[}1899{]}]{L02681 Paul Goldmann an Arthur Schnitzler, 9. 6. {[}1899{]}}
\nopagebreak\mylabel{L02681v}
\rehead{ }\normalsize\beginnumbering\briefempfaengerindex{Schnitzler, Arthur@\textsc{Schnitzler, Arthur}!zzzGoldmann, Paul@\emph{von Paul Goldmann}!1899-06-091@{9. 6. {[}1899{]}}|(be}
\toendnotes[C]{\smallbreak\pagebreak[2]}\Standort{DLA, A:Schnitzler, HS.NZ85.1.3169.}
\physDesc{Telegramm, 139 Zeichen
\newline{}maschinell
\newline{}Schnitzler: mit Bleistift datiert: »9/6 99« 
\newline{}Ordnung: beschnitten }
\pstart
           \noindent{}\centering{}{\pb}fr frankfurtm\oindex{Frankfurt am Main@\textbf{Frankfurt am Main}, \emph{P.PPLA3}|pw} 898
               23 9/6{ }9 50 m =\pend
           
\pstart
           jch war verreist. kenne antoines\pwindex{Antoine, Andre 1858-01-31 – 1943-10-23@\textsc{Antoine, André} (1858-01-31 – 1943-10-23), \emph{Theaterleiter/Theaterleiterin, Schauspieler/Schauspielerin}|pw} adresse nicht.
               du bittest am besten thorel\pwindex{Thorel, Jean 1859-09-11 – 1916-08-20@\textsc{Thorel, Jean} (1859-09-11 – 1916-08-20), \emph{Übersetzer/Übersetzerin, Dramatiker/Dramatikerin}|pw} um uebermittlung
               briefes = \spacefill\mbox{goldmann. +}\pend
           \selectlanguage{ngerman}\endnumbering\briefempfaengerindex{Schnitzler, Arthur@\textsc{Schnitzler, Arthur}!zzzGoldmann, Paul@\emph{von Paul Goldmann}!1899-06-091@{9. 6. {[}1899{]}}|)be}\mylabel{L02681h}  \normalsize

\doendnotes{C}
\bigskip
\vfill

\clearpage

\footnotesize

\lohead{\textsc{register}}

% Definiere theindex-Environment komplett neu ohne reledmac
\makeatletter
\renewenvironment{theindex}{%
  \section*{\indexname}%
  \setlength{\parindent}{0pt}%
  \setlength{\parskip}{0pt plus 0.3pt}%
  \let\item\@idxitem
}{%
  \clearpage
}
\makeatother

\IfFileExists{\jobname-pw.ind}{\input{\jobname-pw.ind}}{}

\end{document}

      