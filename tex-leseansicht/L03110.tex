%% latex-leseansicht-vorspann.tex
%% Vorspann für die Leseansicht.
%% Lädt die gemeinsame Datei latex-vorspann.tex mit nicht gesetztem Schalter.

\newif\ifkorrekturansicht
\korrekturansichtfalse

\input{../tex-inputs/latex-vorspann}


\section[Felix Salten an Arthur Schnitzler, {[}4. 6. 1892{]}]{L03110 Felix Salten an Arthur Schnitzler, {[}4. 6. 1892{]}}
\nopagebreak\mylabel{L03110v}
\rehead{ }\normalsize\beginnumbering\briefempfaengerindex{Schnitzler, Arthur@\textsc{Schnitzler, Arthur}!zzzSalten, Felix@\emph{von Felix Salten}!1892-06-041@{{[}4. 6. 1892{]}}|(be}
\toendnotes[C]{\smallbreak\pagebreak[2]}
\correspDesc{Versand  durch Felix Salten am [4. 6. 1892] in Wien
\newline{}Erhalt  durch Arthur Schnitzler im Zeitraum [4. 6. 1892
                  – 8. 6. 1892?] in Wien}\toendnotes[C]{\smallbreak}
\Standort{CUL, Schnitzler, B 89, A 1.}
\physDesc{Brief, 1 Blatt, 4 Seiten, 803 Zeichen
\newline{}Handschrift: Bleistift, lateinische Kurrent
\newline{}Schnitzler: mit Bleistift datiert: »\substVorne{}\textsuperscript{6}\substDazwischen{}4\substHinten{}/6 92« 
\newline{}Ordnung: mit Bleistift von unbekannter Hand nummeriert: »12« }\toendnotes[C]{\smallbreak}
\pstart
           \noindent{}{\pb}lieber Freund! Vom Beer-Hofmann\pwindex{Beer-Hofmann, Richard 11.\,7.\,1866 Wien – 26.\,9.\,1945 New York City@\textsc{Beer-Hofmann, Richard} (11.\,7.\,1866 Wien – 26.\,9.\,1945 New York City), \emph{Schriftsteller}|pw} keine Nachricht. Er hat mich auch gestern, als er mich zur Laska\pwindex{Laska, Julie 13.\,6.\,1860 Bratislava – 17.\,5.\,1917 Wien@\textsc{Laska, Julie} (13.\,6.\,1860 Bratislava – 17.\,5.\,1917 Wien), \emph{Schauspielerin}|pw}
               abholen sollte, – ohne abzuschreiben – sitzen laßen. Auch von Loris\pwindex{Hofmannsthal, Hugo von 1.\,2.\,1874 Wien – 15.\,7.\,1929 Rodaun@\textsc{Hofmannsthal, Hugo von} (1.\,2.\,1874 Wien – 15.\,7.\,1929 Rodaun), \emph{Schriftsteller}|pw} keine Zeile. Ich verstehe das nicht.\pend
           
\pstart
           Heute{ }Abend, wenn’s nicht {\pb}fortfährt zu
                  regnen{[},{]} beim Schneider\oindex{Café Schneider@\textbf{Café Schneider}, \emph{Kaffeehaus}|pw}
               in der Ausstellung\orgindex{Internationale Ausstellung für Musik und Theaterwesen@Internationale Ausstellung für Musik und Theaterwesen|pw}.\pend
           
\pstart
           In Anbetracht Ihres gestrigen \label{K_L03110-1v}\edtext{Spielverlustes}{\lemma{\textnormal{\emph{Spielverlustes}}}\Cendnote{\textnormal{Vermutlich beim
                  Pokerspiel, vgl. A. S.: \emph{Tagebuch}, 5. 6. 1892.
               }}}\label{K_L03110-1} fällt es mir schwer, Sie anzupumpen, doch kann ich Ihnen, da {\pb}ich von Papa\pwindex{Salzmann, Philipp 24.\,12.\,1831 Miskolc – 2.\,4.\,1905 Wien@\textsc{Salzmann, Philipp} (24.\,12.\,1831 Miskolc – 2.\,4.\,1905 Wien), \emph{Bergbauunternehmer, Projektemacher}|pwv} vor seiner Abreise am Montag Geld bekomme, vielleicht auch morgen schon das selbe zurückgeben. Wenn es Ihnen also
               möglich ist, würde ich Sie sehr um 3 f. bitten.\pend
           
\pstart
           Was soll ich {\pb}mit Beer-Hofmann\pwindex{Beer-Hofmann, Richard 11.\,7.\,1866 Wien – 26.\,9.\,1945 New York City@\textsc{Beer-Hofmann, Richard} (11.\,7.\,1866 Wien – 26.\,9.\,1945 New York City), \emph{Schriftsteller}|pw} anfangen und mit Loris\pwindex{Hofmannsthal, Hugo von 1.\,2.\,1874 Wien – 15.\,7.\,1929 Rodaun@\textsc{Hofmannsthal, Hugo von} (1.\,2.\,1874 Wien – 15.\,7.\,1929 Rodaun), \emph{Schriftsteller}|pw}? Eigentlich ist’s mir ja lieber, wenn nicht gelesen
               wird, da ich jetzt wieder verbu{\geminationm}elt bin, u. Mutza\pwindex{Salten, Felix 6.\,9.\,1869 Budapest – 8.\,10.\,1945 Zürich@\textsc{Salten, Felix} (6.\,9.\,1869 Budapest – 8.\,10.\,1945 Zürich), \emph{Schriftsteller, Journalist, Chefredakteur}!Muza@\strich\emph{Muza}|pw} nicht fertig. Also entweder Schneider\oindex{Café Schneider@\textbf{Café Schneider}, \emph{Kaffeehaus}|pw}, oder im \uline{Regen}{ }\uuline{Kremser\oindex{Wien@\textbf{Wien}!I., Innere Stadt@\textbf{I., Innere Stadt}!Café Kremser@\textbf{Café Kremser}, \emph{Kaffeehaus}|pw}}, \label{K_L03110-2v}\edtext{heute noch}{\lemma{\textnormal{\emph{heute noch}}}\Cendnote{\textnormal{Siehe A. S.: \emph{Tagebuch}, 4. 6. 1892.
               }}}\label{K_L03110-2}, weil Richard\pwindex{Beer-Hofmann, Richard 11.\,7.\,1866 Wien – 26.\,9.\,1945 New York City@\textsc{Beer-Hofmann, Richard} (11.\,7.\,1866 Wien – 26.\,9.\,1945 New York City), \emph{Schriftsteller}|pw} nun kommen könnte.\pend
           \pstart Herzlich \spacefill\mbox{FelixS.}\pend{}\selectlanguage{ngerman}\endnumbering\briefempfaengerindex{Schnitzler, Arthur@\textsc{Schnitzler, Arthur}!zzzSalten, Felix@\emph{von Felix Salten}!1892-06-041@{{[}4. 6. 1892{]}}|)be}\mylabel{L03110h}  \newcommand{\dateiname}{L03110}\newcommand{\titel}{Felix Salten an Arthur Schnitzler, [4. 6. 1892]}\newcommand{\editorInnen}{Martin Anton Müller und Laura Untner}%% latex-leseansicht-abspann.tex
%% Abspann für die Leseansicht.
%% Der Schalter \ifkorrekturansicht ist bereits durch den Vorspann gesetzt.

%% latex-abspann.tex
%% Gemeinsamer Abspann für Korrekturansicht und Leseansicht.
%% Setzt den Schalter \ifkorrekturansicht voraus (gesetzt in den
%% einbindenden Dateien latex-korrekturansicht-abspann.tex bzw.
%% latex-leseansicht-abspann.tex).
%% ---------------------------------------------------------------

\normalsize

% Das esempio-Environment wird nur in der Leseansicht benötigt
\ifkorrekturansicht\else
\newenvironment{esempio}[3]%
{
    \vspace{1.5ex}
    \rlap{\underline{#1}}
    \par
    \setlength{\parindent}{0cm}
    \nopagebreak
    \leftskip=#2cm
    \rightskip=#3cm
}
{
    \par
}
\fi

\doendnotes{C}
\bigskip
\vfill

\clearpage

\footnotesize

\ifkorrekturansicht
  \lohead{\textsc{register}}
\fi

% theindex-Environment neu definieren ohne reledmac
\makeatletter
\renewenvironment{theindex}{%
  \ifkorrekturansicht
    \section*{\indexname}%
  \else
    \subsubsection*{Index der erwähnten Entitäten}%
  \fi
  \setlength{\parindent}{0pt}%
  \setlength{\parskip}{0pt plus 0.3pt}%
  \let\item\@idxitem
}{%
  \ifkorrekturansicht\clearpage\fi
}
\makeatother

\IfFileExists{\jobname-pw.ind}{\input{\jobname-pw.ind}}{}

% Quellenangabe nur in der Leseansicht
\ifkorrekturansicht\else
% Fallback-Definitionen, falls die .tex-Datei \titel etc. nicht gesetzt hat
\providecommand{\titel}{}
\providecommand{\editorInnen}{}
\providecommand{\dateiname}{\jobname}

\vspace{3cm}

\vfill

\footnotesize
\textsc{Quelle}: \titel. Herausgegeben von {\editorInnen}. In: \emph{Arthur Schnitzler: Briefwechsel mit Autorinnen und Autoren}.
 Digitale Edition, https://schnitzler-briefe.acdh.oeaw.ac.at/{\dateiname}.html (Stand \today)
\fi

\end{document}


