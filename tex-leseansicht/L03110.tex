%% latex-korrekturansicht-vorspann.tex
%% Vorspann für die Korrekturansicht.
%% Lädt die gemeinsame Datei latex-vorspann.tex mit gesetztem Schalter.

\newif\ifkorrekturansicht
\korrekturansichttrue

\input{../tex-inputs/latex-vorspann}


\section[Felix Salten an Arthur Schnitzler, {[}4. 6. 1892{]}]{L03110 Felix Salten an Arthur Schnitzler, {[}4. 6. 1892{]}}
\nopagebreak\mylabel{L03110v}
\rehead{ }\normalsize\beginnumbering\briefempfaengerindex{Schnitzler, Arthur@\textsc{Schnitzler, Arthur}!zzzSalten, Felix@\emph{von Felix Salten}!1892-06-041@{{[}4. 6. 1892{]}}|(be}
\toendnotes[C]{\smallbreak\pagebreak[2]}\Standort{CUL, Schnitzler, B 89, A 1.}
\physDesc{Brief, 1 Blatt, 4 Seiten, 803 Zeichen
\newline{}Handschrift: Bleistift, lateinische Kurrent
\newline{}Schnitzler: mit Bleistift datiert: »\substVorne{}\textsuperscript{6}\substDazwischen{}4\substHinten{}/6 92« 
\newline{}Ordnung: mit Bleistift von unbekannter Hand nummeriert: »12« }\toendnotes[C]{\smallbreak}
\pstart
           \noindent{}{\pb}lieber Freund! Vom Beer-Hofmann\pwindex{Beer-Hofmann, Richard 1866-07-11 – 1945-09-26@\textsc{Beer-Hofmann, Richard} (1866-07-11 – 1945-09-26), \emph{Schriftsteller/Schriftstellerin}|pw} keine Nachricht. Er hat mich auch gestern, als er mich zur Laska\pwindex{Laska, Julie 13.06.1860 – 17.05.1917@\textsc{Laska, Julie} (13.06.1860 – 17.05.1917), \emph{Schauspieler/Schauspielerin}|pw}
               abholen sollte, – ohne abzuschreiben – sitzen laßen. Auch von Loris\pwindex{Hofmannsthal, Hugo von 1874-02-01 – 1929-07-15@\textsc{Hofmannsthal, Hugo von} (1874-02-01 – 1929-07-15), \emph{Schriftsteller/Schriftstellerin}|pw} keine Zeile. Ich verstehe das nicht.\pend
           
\pstart
           Heute{ }Abend, wenn’s nicht {\pb}fortfährt zu
                  regnen{[},{]} beim Schneider\oindex{Cafe Schneider@\textbf{Café Schneider}, \emph{Kaffeehaus (K.KAF)}|pw}
               in der Ausstellung\orgindex{Internationale Ausstellung fuer Musik und Theaterwesen@Internationale Ausstellung für Musik und Theaterwesen|pw}.\pend
           
\pstart
           In Anbetracht Ihres gestrigen \label{K_L03110-1v}\edtext{Spielverlustes}{\lemma{\textnormal{\emph{Spielverlustes}}}\Cendnote{\textnormal{Vermutlich beim
                  Pokerspiel, vgl. A. S.: \emph{Tagebuch}, 5. 6. 1892.
               }}}\label{K_L03110-1} fällt es mir schwer, Sie anzupumpen, doch kann ich Ihnen, da {\pb}ich von Papa\pwindex{Salzmann, Philipp 1831-12-24 – 1905-04-02@\textsc{Salzmann, Philipp} (1831-12-24 – 1905-04-02), \emph{Bergbauunternehmer/Bergbauunternehmerin, Projektemacher/Projektemacherin}|pwv} vor seiner Abreise am Montag Geld bekomme, vielleicht auch morgen schon das selbe zurückgeben. Wenn es Ihnen also
               möglich ist, würde ich Sie sehr um 3 f. bitten.\pend
           
\pstart
           Was soll ich {\pb}mit Beer-Hofmann\pwindex{Beer-Hofmann, Richard 1866-07-11 – 1945-09-26@\textsc{Beer-Hofmann, Richard} (1866-07-11 – 1945-09-26), \emph{Schriftsteller/Schriftstellerin}|pw} anfangen und mit Loris\pwindex{Hofmannsthal, Hugo von 1874-02-01 – 1929-07-15@\textsc{Hofmannsthal, Hugo von} (1874-02-01 – 1929-07-15), \emph{Schriftsteller/Schriftstellerin}|pw}? Eigentlich ist’s mir ja lieber, wenn nicht gelesen
               wird, da ich jetzt wieder verbu{\geminationm}elt bin, u. Mutza\pwindex{Muza@\emph{Muza}|pw} nicht fertig. Also entweder Schneider\oindex{Cafe Schneider@\textbf{Café Schneider}, \emph{Kaffeehaus (K.KAF)}|pw}, oder im \uline{Regen}{ }\uuline{Kremser\oindex{Cafe Kremser@\textbf{Café Kremser}, \emph{Kaffeehaus (K.KAF)}|pw}}, \label{K_L03110-2v}\edtext{heute noch}{\lemma{\textnormal{\emph{heute noch}}}\Cendnote{\textnormal{Siehe A. S.: \emph{Tagebuch}, 4. 6. 1892.
               }}}\label{K_L03110-2}, weil Richard\pwindex{Beer-Hofmann, Richard 1866-07-11 – 1945-09-26@\textsc{Beer-Hofmann, Richard} (1866-07-11 – 1945-09-26), \emph{Schriftsteller/Schriftstellerin}|pw} nun kommen könnte.\pend
           \pstart Herzlich \spacefill\mbox{FelixS.}\pend{}\selectlanguage{ngerman}\endnumbering\briefempfaengerindex{Schnitzler, Arthur@\textsc{Schnitzler, Arthur}!zzzSalten, Felix@\emph{von Felix Salten}!1892-06-041@{{[}4. 6. 1892{]}}|)be}\mylabel{L03110h}  \normalsize

\doendnotes{C}
\bigskip
\vfill

\clearpage

\footnotesize

\lohead{\textsc{register}}

% Definiere theindex-Environment komplett neu ohne reledmac
\makeatletter
\renewenvironment{theindex}{%
  \section*{\indexname}%
  \setlength{\parindent}{0pt}%
  \setlength{\parskip}{0pt plus 0.3pt}%
  \let\item\@idxitem
}{%
  \clearpage
}
\makeatother

\IfFileExists{\jobname-pw.ind}{\input{\jobname-pw.ind}}{}

\end{document}

      