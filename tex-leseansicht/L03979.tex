%% latex-leseansicht-vorspann.tex
%% Vorspann für die Leseansicht.
%% Lädt die gemeinsame Datei latex-vorspann.tex mit nicht gesetztem Schalter.

\newif\ifkorrekturansicht
\korrekturansichtfalse

\input{../tex-inputs/latex-vorspann}


\section[Arthur Schnitzler an Berta Zuckerkandl, 2. 12. 1929]{L03979 Arthur Schnitzler an Berta Zuckerkandl, 2. 12. 1929}
\nopagebreak\mylabel{L03979v}
\rehead{ }\normalsize\beginnumbering\briefempfaengerindex{Zuckerkandl, Berta@\textsc{Zuckerkandl, Berta}!zzzSchnitzler, Arthur@\emph{von Arthur Schnitzler}!1929-12-021@{2. 12. 1929}|(be}
\toendnotes[C]{\smallbreak\pagebreak[2]}
\correspDesc{Versand  durch Arthur Schnitzler am 2. 12. 1929 in Wien
\newline{}Erhalt  durch Berta Zuckerkandl im Zeitraum [3. 12. 1929 – 7. 12. 1929?] in Paris}\toendnotes[C]{\smallbreak}
\Standort{DLA, HS.1985.1.2282.}
\physDesc{Brief, Durchschlag, 1 Blatt, 1 Seite, 271 Zeichen
\newline{}Schreibmaschine
\newline{}Handschrift: roter Buntstift, lateinische Kurrent (\noindent{}beschriftet: »\uline{Zuckerkandl}«, »Reigen«, »Remon« und »\uline{Paris}«, vier
                                 Unterstreichungen)}\toendnotes[C]{\smallbreak}
\pstart
           \raggedleft{}{\pb}2. 12. 1929.\pend
           \vspace{0.5em}
\pstart
           Hofrätin Bertha \label{T_L03979-1v}\edtext{Zuckerkandl,}{\lemma{\textnormal{\emph{Zuckerkandl,}}}\Cendnote{\textnormal{In der Vorlage steht:
                     »Zuckerkandl,,«.}}}\label{T_L03979-1}{ }Paris\oindex{Paris@\textbf{Paris}, \emph{Hauptstadt}|pw}.\pend
           
\pstart
           Schreibe in Ihrem Sinn \label{K_L03979-1v}\edtext{Brief an Rémon\pwindex{Rémon, Maurice 27.\,11.\,1861 Paris – 20.\,6.\,1945 Mérignac@\textsc{Rémon, Maurice} (27.\,11.\,1861 Paris – 20.\,6.\,1945 Mérignac), \emph{Übersetzer}|pw}}{\lemma{\textnormal{\emph{Brief an Rémon}}}\Cendnote{\textnormal{Arthur Schnitzler an Maurice Rémon\pwindex{Rémon, Maurice 27.\,11.\,1861 Paris – 20.\,6.\,1945 Mérignac@\textsc{Rémon, Maurice} (27.\,11.\,1861 Paris – 20.\,6.\,1945 Mérignac), \emph{Übersetzer}|pwk}, 2. 12. 1929, \emph{Deutsches Literaturarchiv Marbach},  HS.1985.1.1686.}}}\label{K_L03979-1}. Einverstanden Besnard\pwindex{Besnard, Lucien 19.\,1.\,1872 Nonancourt – 1955 Paris@\textsc{Besnard, Lucien} (19.\,1.\,1872 Nonancourt – 1955 Paris), \emph{Schriftsteller}|pw} 5 {\%}. Sie, liebe Freundin, jedesfalls
                  ein Viertel meines Anteils. Darf ich Sie Géraldy\pwindex{Géraldy, Paul 6.\,3.\,1885 Paris – 9.\,3.\,1983 Neuilly-sur-Seine@\textsc{Géraldy, Paul} (6.\,3.\,1885 Paris – 9.\,3.\,1983 Neuilly-sur-Seine), \emph{Schriftsteller}|pw} »Schwestern\pwindex{Schnitzler, Arthur 15. 5. 1862 Wien – 21. 10. 1931 ebd.@\textsc{Schnitzler, Arthur} (15. 5. 1862 Wien – 21. 10. 1931 ebd.), \emph{Schriftsteller, Mediziner}!Schwestern oder Casanova in Spa. Lustspiel in Versen@\strich\emph{Die Schwestern oder Casanova in Spa. Lustspiel in Versen}|pw}« erinnern. Clauser\pwindex{Clauser, Suzanne 16.\,5.\,1898 Wien – 11.\,9.\,1981 Paris@\textsc{Clauser, Suzanne} (16.\,5.\,1898 Wien – 11.\,9.\,1981 Paris), \emph{Schriftstellerin, Übersetzerin}|pw} hat indess Uebersetzung\pwindex{Schnitzler, Arthur 15. 5. 1862 Wien – 21. 10. 1931 ebd.@\textsc{Schnitzler, Arthur} (15. 5. 1862 Wien – 21. 10. 1931 ebd.), \emph{Schriftsteller, Mediziner}!?? [französische Übersetzung von Die Schwestern oder Casanova in Spa. Lustspiel in Versen]@\strich\emph{?? [französische Übersetzung von Die Schwestern oder Casanova in Spa. Lustspiel in Versen]}|pwv} meines Stücks\pwindex{Schnitzler, Arthur 15. 5. 1862 Wien – 21. 10. 1931 ebd.@\textsc{Schnitzler, Arthur} (15. 5. 1862 Wien – 21. 10. 1931 ebd.), \emph{Schriftsteller, Mediziner}!Schwestern oder Casanova in Spa. Lustspiel in Versen@\strich\emph{Die Schwestern oder Casanova in Spa. Lustspiel in Versen}|pwv} sehr schön vollendet.\pend
           \selectlanguage{ngerman}\endnumbering\briefempfaengerindex{Zuckerkandl, Berta@\textsc{Zuckerkandl, Berta}!zzzSchnitzler, Arthur@\emph{von Arthur Schnitzler}!1929-12-021@{2. 12. 1929}|)be}\mylabel{L03979h}
\begin{anhang}
\end{anhang}\newcommand{\dateiname}{L03979}\newcommand{\titel}{Arthur Schnitzler an Berta Zuckerkandl, 2. 12. 1929}\newcommand{\editorInnen}{Herausgegeben von Jahnke, SelmaMüller, Martin Anton}%% latex-leseansicht-abspann.tex
%% Abspann für die Leseansicht.
%% Der Schalter \ifkorrekturansicht ist bereits durch den Vorspann gesetzt.

%% latex-abspann.tex
%% Gemeinsamer Abspann für Korrekturansicht und Leseansicht.
%% Setzt den Schalter \ifkorrekturansicht voraus (gesetzt in den
%% einbindenden Dateien latex-korrekturansicht-abspann.tex bzw.
%% latex-leseansicht-abspann.tex).
%% ---------------------------------------------------------------

\normalsize

% Das esempio-Environment wird nur in der Leseansicht benötigt
\ifkorrekturansicht\else
\newenvironment{esempio}[3]%
{
    \vspace{1.5ex}
    \rlap{\underline{#1}}
    \par
    \setlength{\parindent}{0cm}
    \nopagebreak
    \leftskip=#2cm
    \rightskip=#3cm
}
{
    \par
}
\fi

\doendnotes{C}
\bigskip
\vfill

\clearpage

\footnotesize

\ifkorrekturansicht
  \lohead{\textsc{register}}
\fi

% theindex-Environment neu definieren ohne reledmac
\makeatletter
\renewenvironment{theindex}{%
  \ifkorrekturansicht
    \section*{\indexname}%
  \else
    \subsubsection*{Index der erwähnten Entitäten}%
  \fi
  \setlength{\parindent}{0pt}%
  \setlength{\parskip}{0pt plus 0.3pt}%
  \let\item\@idxitem
}{%
  \ifkorrekturansicht\clearpage\fi
}
\makeatother

\IfFileExists{\jobname-pw.ind}{\input{\jobname-pw.ind}}{}

% Quellenangabe nur in der Leseansicht
\ifkorrekturansicht\else
% Fallback-Definitionen, falls die .tex-Datei \titel etc. nicht gesetzt hat
\providecommand{\titel}{}
\providecommand{\editorInnen}{}
\providecommand{\dateiname}{\jobname}

\vspace{3cm}

\vfill

\footnotesize
\textsc{Quelle}: \titel. Herausgegeben von {\editorInnen}. In: \emph{Arthur Schnitzler: Briefwechsel mit Autorinnen und Autoren}.
 Digitale Edition, https://schnitzler-briefe.acdh.oeaw.ac.at/{\dateiname}.html (Stand \today)
\fi

\end{document}


