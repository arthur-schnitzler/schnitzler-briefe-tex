%% latex-korrekturansicht-vorspann.tex
%% Vorspann für die Korrekturansicht.
%% Lädt die gemeinsame Datei latex-vorspann.tex mit gesetztem Schalter.

\newif\ifkorrekturansicht
\korrekturansichttrue

\input{../tex-inputs/latex-vorspann}


\section[Hermann Bahr an Arthur Schnitzler, {[}20. 2. 1892{]}]{L00073 Hermann Bahr an Arthur Schnitzler, {[}20. 2. 1892{]}}
\nopagebreak\mylabel{L00073v}
\rehead{ }\normalsize\beginnumbering\briefempfaengerindex{Schnitzler, Arthur@\textsc{Schnitzler, Arthur}!zzzBahr, Hermann@\emph{von Hermann Bahr}!1892-02-201@{{[}20. 2. 1892{]}}|(be}
\toendnotes[C]{\smallbreak\pagebreak[2]}\Standort{CUL, Schnitzler, B 5b.}
\physDesc{Brief, 1 Blatt, 1 Seite, 146 Zeichen
\newline{}Handschrift: roter Buntstift, deutsche Kurrent
\newline{}Schnitzler: mit Bleistift ergänztes Datum »20/2 92«. An der Stelle, an der üblicherweise die Unterschrift
                                 steht, Vermerk: »Bahr« 
\newline{}Ordnung: 1) Blattränder oben und unten beschnitten  2) mit Bleistift von unbekannter Hand nummeriert: »\strikeout{5}« 3) mit Bleistift von unbekannter Hand nummeriert:
                                 »4«}
\buchAbdrucke{\weitereDrucke{Hermann Bahr, Arthur Schnitzler: \emph{Briefwechsel, Aufzeichnungen, Dokumente (1891–1931)}. Göttingen: \emph{Wallstein} 2018, S. 22.} }\toendnotes[C]{\smallbreak}
\pstart
           \noindent{}{\pb}Lieber Freund! Das
               hunniſch-tartariſche Engerl\pwindex{Pálmay, Ilka 1859-09-21 – 1944-02-17@\textsc{Pálmay, Ilka} (1859-09-21 – 1944-02-17), \emph{Schriftsteller/Schriftstellerin, Schauspieler/Schauspielerin, Sänger/Sängerin}|pwv}
               hat mich für morgen zwiſchen 12–1 beſtellt. Die Auskunft ſende ich Ihnen unmittelbar
               nachher\pend
           \pstart Herzlichſt\pend{}\selectlanguage{ngerman}\endnumbering\briefempfaengerindex{Schnitzler, Arthur@\textsc{Schnitzler, Arthur}!zzzBahr, Hermann@\emph{von Hermann Bahr}!1892-02-201@{{[}20. 2. 1892{]}}|)be}\mylabel{L00073h}  \normalsize

\doendnotes{C}
\bigskip
\vfill

\clearpage

\footnotesize

\lohead{\textsc{register}}

% Definiere theindex-Environment komplett neu ohne reledmac
\makeatletter
\renewenvironment{theindex}{%
  \section*{\indexname}%
  \setlength{\parindent}{0pt}%
  \setlength{\parskip}{0pt plus 0.3pt}%
  \let\item\@idxitem
}{%
  \clearpage
}
\makeatother

\IfFileExists{\jobname-pw.ind}{\input{\jobname-pw.ind}}{}

\end{document}

      