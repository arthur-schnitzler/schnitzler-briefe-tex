%% latex-leseansicht-vorspann.tex
%% Vorspann für die Leseansicht.
%% Lädt die gemeinsame Datei latex-vorspann.tex mit nicht gesetztem Schalter.

\newif\ifkorrekturansicht
\korrekturansichtfalse

\input{../tex-inputs/latex-vorspann}


\section[Arthur Schnitzler an Theodor Herzl, 22. 12. 1900]{L03938 Arthur Schnitzler an Theodor Herzl, 22. 12. 1900}
\nopagebreak\mylabel{L03938v}
\rehead{ }\normalsize\beginnumbering\briefempfaengerindex{Herzl, Theodor@\textsc{Herzl, Theodor}!zzzSchnitzler, Arthur@\emph{von Arthur Schnitzler}!1900-12-221@{22. 12. 1900}|(be}
\toendnotes[C]{\smallbreak\pagebreak[2]}
\correspDesc{Versand  durch Arthur Schnitzler am 22. 12. 1900 in Wien
\newline{}Erhalt  durch Theodor Herzl in Wien}\toendnotes[C]{\smallbreak}
\Standort{Jerusalem, Central Zionist Archives, H1:1926-4.}
\physDesc{,  Blätter,  Seiten
\newline{}Handschrift: , deutsche Kurrent}
\buchAbdrucke{\weitereDrucke{Arthur Schnitzler: \emph{Briefe 1875–1912}. Herausgegeben von Therese Nickl und Heinrich Schnitzler. Frankfurt am Main: \emph{S. Fischer} 1981, S. 398–399.} }\toendnotes[C]{\smallbreak}
\pstart
           {\pb}Wien\oindex{Wien@\textbf{Wien}, \emph{Verwaltungsgebiet}|pw}, 22. 12. 900\pend
           
\pstart{}lieber Freund,\pend\vspace{0.5em}
\pstart
           ich habe Ihnen den Lieutenant Guſtl\pwindex{Schnitzler, Arthur 15.\,5.\,1862 Wien – 21.\,10.\,1931 ebd.@\textsc{Schnitzler, Arthur} (15.\,5.\,1862 Wien – 21.\,10.\,1931 ebd.), \emph{Schriftsteller, Mediziner}!Lieutenant Gustl. Novelle@\strich\emph{Lieutenant Gustl. Novelle}|pw} auf Ihren Wunſch vor eben 6 Wochen für die
               Weihnachtsbeilage\pwindex{Neue Freie Presse@\emph{Neue Freie Presse}|pwv} geſchickt; habe damals u noch ſpäter
               ausdrücklich betont, daſs eine \uline{Theilung der Novelle\pwindex{Schnitzler, Arthur 15.\,5.\,1862 Wien – 21.\,10.\,1931 ebd.@\textsc{Schnitzler, Arthur} (15.\,5.\,1862 Wien – 21.\,10.\,1931 ebd.), \emph{Schriftsteller, Mediziner}!Lieutenant Gustl. Novelle@\strich\emph{Lieutenant Gustl. Novelle}|pwv} aus künſtle¬}\uline{riſchen Gründen unthunlich}, daſs ich aber \uline{gern zu Kürzungen bereit}{ }ſei: Ich ſelbſt machte Sie auf
               die Länge der Novelle\pwindex{Schnitzler, Arthur 15.\,5.\,1862 Wien – 21.\,10.\,1931 ebd.@\textsc{Schnitzler, Arthur} (15.\,5.\,1862 Wien – 21.\,10.\,1931 ebd.), \emph{Schriftsteller, Mediziner}!Lieutenant Gustl. Novelle@\strich\emph{Lieutenant Gustl. Novelle}|pwv}
               aufmerkſam und Sie, lieber Doctor sollen Gele{\pb}genheit, 6
               Wochen lang Gelegenheit, über die Möglichkeit der Unterbringung in der Weihnachtsbeilage\pwindex{Neue Freie Presse@\emph{Neue Freie Presse}|pwv} klar u ſchlüſſig zu
               werden. Und heute, am 22. Dez., zwei Tage vor Weihnachten,
               nachdem ich genöthigt war, Anträge andrer Zeitungen zurückzuweiſen, ko{\geminationm}en Sie mit
               der Mittheilung, daſs die Novelle\pwindex{Schnitzler, Arthur 15.\,5.\,1862 Wien – 21.\,10.\,1931 ebd.@\textsc{Schnitzler, Arthur} (15.\,5.\,1862 Wien – 21.\,10.\,1931 ebd.), \emph{Schriftsteller, Mediziner}!Lieutenant Gustl. Novelle@\strich\emph{Lieutenant Gustl. Novelle}|pwv} für die Weihnachtsbeilage\pwindex{Neue Freie Presse@\emph{Neue Freie Presse}|pwv} zu lang ſei, ſtellen in Ausſicht mir \strikeout{den Antrag},  meine Novelle\pwindex{Schnitzler, Arthur 15.\,5.\,1862 Wien – 21.\,10.\,1931 ebd.@\textsc{Schnitzler, Arthur} (15.\,5.\,1862 Wien – 21.\,10.\,1931 ebd.), \emph{Schriftsteller, Mediziner}!Lieutenant Gustl. Novelle@\strich\emph{Lieutenant Gustl. Novelle}|pwv} in Fortſetzungen erſcheinen zu laſſen und \uline{Sie} ſprechen nun von Raumrückſichten, auf die ich Sie
               längſt aufmerkſam gemacht, was Sie mit den Wor{\pb}ten
               zurückwieſen, das ſei Ihre Sorge! Mein lieber Freund, das ka{\geminationn} nicht Ihr Ernſt ſein.
               Ich glaube{ }ſogar annehmen zu dürfen, dſs nicht Sie es ſind, der ſich mir gegenüber
               dieſen verblüffenden Mangel an Rückſicht zu Schulden kommen läßt. Denn es iſt ganz{ }ſelbſtverſtändlich, daſs in dem vorliegenden Fall die Neue Freie Preſſe\orgindex{Neue Freie Presse@Neue Freie Presse|pw} verpflichtet \substVorne{}\textsuperscript{iſt}\substDazwischen{}wäre\substHinten{}, um die Vereinbarung gegen mich zu erfüllen, die ſie auf eigne Initiative,
               auf eignen Wunſch, trotz des von mir {\pb}ſelbſt vorgebrachten
               Bedenken, ausgegangen iſt, die Raumſchwierigkeiten durch Einfügen eines oder mehrerer
               Blätter mehr zu beſiegen. Soweit ich in Betracht komme, geſtatte ich aus den von
               Ihnen gekaſten und ſtillſchweigend gewürdigten künſtleriſchen Gründen den Abdruck der
               Novelle »Lieutenant Guſtl\pwindex{Schnitzler, Arthur 15.\,5.\,1862 Wien – 21.\,10.\,1931 ebd.@\textsc{Schnitzler, Arthur} (15.\,5.\,1862 Wien – 21.\,10.\,1931 ebd.), \emph{Schriftsteller, Mediziner}!Lieutenant Gustl. Novelle@\strich\emph{Lieutenant Gustl. Novelle}|pw}« in Fortſetzung, \uline{unter keiner Bedingung}, und müßte, we{\geminationn} die
               Raumſchwierigkeiten ſich nicht beheben laſſen, höflichſt um Rückſtellung meiner
               Arbeit erſuchen.\pend
           \pstart Herzlich grüßend Ihr \spacefill\mbox{Arthur Schnitzler}\pend{}\selectlanguage{ngerman}\endnumbering\briefempfaengerindex{Herzl, Theodor@\textsc{Herzl, Theodor}!zzzSchnitzler, Arthur@\emph{von Arthur Schnitzler}!1900-12-221@{22. 12. 1900}|)be}\mylabel{L03938h}
\begin{anhang}
\end{anhang}\newcommand{\dateiname}{L03938}\newcommand{\titel}{Arthur Schnitzler an Theodor Herzl, 22. 12. 1900}\newcommand{\editorInnen}{Herausgegeben von Jahnke, SelmaMüller, Martin Anton}%% latex-leseansicht-abspann.tex
%% Abspann für die Leseansicht.
%% Der Schalter \ifkorrekturansicht ist bereits durch den Vorspann gesetzt.

%% latex-abspann.tex
%% Gemeinsamer Abspann für Korrekturansicht und Leseansicht.
%% Setzt den Schalter \ifkorrekturansicht voraus (gesetzt in den
%% einbindenden Dateien latex-korrekturansicht-abspann.tex bzw.
%% latex-leseansicht-abspann.tex).
%% ---------------------------------------------------------------

\normalsize

% Das esempio-Environment wird nur in der Leseansicht benötigt
\ifkorrekturansicht\else
\newenvironment{esempio}[3]%
{
    \vspace{1.5ex}
    \rlap{\underline{#1}}
    \par
    \setlength{\parindent}{0cm}
    \nopagebreak
    \leftskip=#2cm
    \rightskip=#3cm
}
{
    \par
}
\fi

\doendnotes{C}
\bigskip
\vfill

\clearpage

\footnotesize

\ifkorrekturansicht
  \lohead{\textsc{register}}
\fi

% theindex-Environment neu definieren ohne reledmac
\makeatletter
\renewenvironment{theindex}{%
  \ifkorrekturansicht
    \section*{\indexname}%
  \else
    \subsubsection*{Index der erwähnten Entitäten}%
  \fi
  \setlength{\parindent}{0pt}%
  \setlength{\parskip}{0pt plus 0.3pt}%
  \let\item\@idxitem
}{%
  \ifkorrekturansicht\clearpage\fi
}
\makeatother

\IfFileExists{\jobname-pw.ind}{\input{\jobname-pw.ind}}{}

% Quellenangabe nur in der Leseansicht
\ifkorrekturansicht\else
% Fallback-Definitionen, falls die .tex-Datei \titel etc. nicht gesetzt hat
\providecommand{\titel}{}
\providecommand{\editorInnen}{}
\providecommand{\dateiname}{\jobname}

\vspace{3cm}

\vfill

\footnotesize
\textsc{Quelle}: \titel. Herausgegeben von {\editorInnen}. In: \emph{Arthur Schnitzler: Briefwechsel mit Autorinnen und Autoren}.
 Digitale Edition, https://schnitzler-briefe.acdh.oeaw.ac.at/{\dateiname}.html (Stand \today)
\fi

\end{document}


