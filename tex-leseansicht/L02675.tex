%% latex-leseansicht-vorspann.tex
%% Vorspann für die Leseansicht.
%% Lädt die gemeinsame Datei latex-vorspann.tex mit nicht gesetztem Schalter.

\newif\ifkorrekturansicht
\korrekturansichtfalse

\input{../tex-inputs/latex-vorspann}


\section[Paul Goldmann an Arthur Schnitzler, 18. 12. {[}1891{]}]{L02675 Paul Goldmann an Arthur Schnitzler, 18. 12. [1891]}
\nopagebreak\mylabel{L02675v}
\rehead{ }\normalsize\beginnumbering\briefempfaengerindex{Schnitzler, Arthur@\textsc{Schnitzler, Arthur}!zzzGoldmann, Paul@\emph{von Paul Goldmann}!1891-12-181@{18. 12. [1891]}|(be}
\toendnotes[C]{\smallbreak\pagebreak[2]}
\correspDesc{Versand  durch Paul Goldmann am 18. 12. [1891] in Paris
\newline{}Erhalt  durch Arthur Schnitzler im Zeitraum [19. 12. 1891 – 23. 12. 1891?] in Wien}\toendnotes[C]{\smallbreak}
\Standort{DLA, A:Schnitzler, HS.NZ85.1.3162.}
\physDesc{Brief, 1 Blatt, 4 Seiten, 1757 Zeichen
\newline{}Handschrift: schwarze Tinte, deutsche Kurrent
\newline{}Schnitzler: mit Bleistift das Jahr »91« vermerkt }\toendnotes[C]{\smallbreak}
\pstart
           \centering{}{\pb}Paris\oindex{Paris@\textbf{Paris}, \emph{Hauptstadt}|pw}, 18. December.\pend
           
\pstart\center{}Mein lieber Arthur!\pend\vspace{0.5em}
\pstart
           Ich habe gerade deinen Brief erhalten u. laufe raſch in das \label{K_L02675-1v}\edtext{nächſtliegende \textsc{Café de la Paix\oindex{Café de la Paix@\textbf{Café de la Paix}, \emph{Kaffeehaus}|pw}}}{\lemma{\textnormal{\emph{nächstliegende … Paix}}}\Cendnote{\textnormal{nächstliegend hier im Sinne von: in
                  der Nähe liegend; es gab nur das Café de la Paix\oindex{Café de la Paix@\textbf{Café de la Paix}, \emph{Kaffeehaus}|pwk}}}}\label{K_L02675-1} hinein, um mir meine Freude von der Seele zu{ }ſchreiben. Wie froh ich bin,
               Unrecht gehabt zu haben! Ich \label{K_L02675-2v}\edtext{beglückwünſche}{\lemma{\textnormal{\emph{beglückwünsche}}}\Cendnote{\textnormal{Goldmann\pwindex{Goldmann, Paul 31.\,1.\,1865 Breslau – 25.\,9.\,1935 Wien@\textsc{Goldmann, Paul} (31.\,1.\,1865 Breslau – 25.\,9.\,1935 Wien), \emph{Schriftsteller, Journalist}|pwk} gratulierte Schnitzler zur Annahme des \emph{Märchens}\pwindex{Schnitzler, Arthur 15.\,5.\,1862 Wien – 21.\,10.\,1931 ebd.@\textsc{Schnitzler, Arthur} (15.\,5.\,1862 Wien – 21.\,10.\,1931 ebd.), \emph{Schriftsteller, Mediziner}!Märchen. Schauspiel in drei Aufzügen@\strich\emph{Das Märchen. Schauspiel in drei Aufzügen}|pwk} am Berlin\oindex{Berlin@\textbf{Berlin}, \emph{Hauptstadt}|pwk}er \emph{Lessing-Theater}\orgindex{Lessing-Theater@Lessing-Theater|pwk} (siehe XXXX Auszeichnungsfehler: Dokument L00052 nicht gefunden). Zu dieser Inszenierung kam es
                  nicht.}}}\label{K_L02675-2} Dich innig und von ganzem Herzen, und ich rufe aller guten Engel
               Beiſtand auf Dich herab, auf daß das große Werk\pwindex{Schnitzler, Arthur 15.\,5.\,1862 Wien – 21.\,10.\,1931 ebd.@\textsc{Schnitzler, Arthur} (15.\,5.\,1862 Wien – 21.\,10.\,1931 ebd.), \emph{Schriftsteller, Mediziner}!Märchen. Schauspiel in drei Aufzügen@\strich\emph{Das Märchen. Schauspiel in drei Aufzügen}|pwv} gelinge. Iſt der Wind Dir nur ein wenig günſtig,{ }ſo biſt
               Du von heut auf morgen ein in ganz Deutſchland\oindex{Deutschland@\textbf{Deutschland}|pw}
               bekannter Mann. Wie eitel ich darauf bin, daß ich{ }ſo feſt an Dich geglaubt. Nun aber
               folge mir ein wenig, mein lieber Junge (entſchuldige, es iſt nicht wegen der Jugend,{ }ſondern {\pb}wegen der Herzlichkeit) und{ }ſei nicht
               bockbeinig und mache die Änderungen, die erfahrene Theaterpraktiker\pwindex{Blumenthal, Oskar 13.\,3.\,1852 Berlin – 24.\,4.\,1917 ebd.@\textsc{Blumenthal, Oskar} (13.\,3.\,1852 Berlin – 24.\,4.\,1917 ebd.), \emph{Schriftsteller, Journalist, Theaterleiter}|pwv} von Dir verlangen,{ }ſo roh{ }ſie Dir auch erſcheinen mögen. Das Geheimniß des Erfolges liegt nicht am Wenigſten in
               der Kunſt, Conceſſionen zu machen. Vor allem muß der dritte Akt\pwindex{Schnitzler, Arthur 15.\,5.\,1862 Wien – 21.\,10.\,1931 ebd.@\textsc{Schnitzler, Arthur} (15.\,5.\,1862 Wien – 21.\,10.\,1931 ebd.), \emph{Schriftsteller, Mediziner}!Märchen. Schauspiel in drei Aufzügen@\strich\emph{Das Märchen. Schauspiel in drei Aufzügen}|pwv} umgearbeitet werden – muß, glaube mir!
               Wenn Du die lauten Exploſionen verabſcheuſt – gut! Aber conciſer\substVorne{}\textsuperscript{{ }und}\substDazwischen{},\substHinten{} compacter, kräftiger anſteigend und einheitlicher muß die Sache werden. Eine
               Kleinigkeit: mach’ \textsc{Moritzki\pwindex{Schnitzler, Arthur 15.\,5.\,1862 Wien – 21.\,10.\,1931 ebd.@\textsc{Schnitzler, Arthur} (15.\,5.\,1862 Wien – 21.\,10.\,1931 ebd.), \emph{Schriftsteller, Mediziner}!Märchen. Schauspiel in drei Aufzügen@\strich\emph{Das Märchen. Schauspiel in drei Aufzügen}|pwv}} etwas komiſcher! {\pb}So iſt er zu trocken und
               ledern. Der poln\oindex{Polen@\textbf{Polen}|pw}iſche Accent allein genügt nicht;
               es muß auch in den Worten etwas{ }ſein. Ich bitte Dich, mich über die Änderungen
                  \label{K_L02675-3v}\edtext{\textsc{\begin{otherlanguage}{french}au courant\end{otherlanguage}}}{\lemma{\textnormal{\emph{au courant}}}\Cendnote{\textnormal{französisch: auf dem Laufenden}}}\label{K_L02675-3} zu
               erhalten. Vielleicht daß ich doch etwas noch dazu bemerken kann! Und nochmals: von
               ganzem Herzen Glückauf! Das Leben iſt doch manchmal auch gut, und das war eine
               freudige Überraſchung heut{ }Abend{\dotsfour}\pend
           
\pstart
           Vielen Dank für die lieben Empfehlungen!\pend
           
\pstart
           Grüß’ Dich Gott! {\\[\baselineskip]}Dein {\\[\baselineskip]}\spacefill\mbox{Paul Goldmann}\pend
           \leftskip=0em{}
\pstart
           \noindent{}\label{K_L02675-4v}\edtext{\uline{verte}}{\lemma{\textnormal{\emph{verte}}}\Cendnote{\textnormal{lateinisch: umblättern,
                  wenden}}}\label{K_L02675-4}!\pend
           
\pstart
           {\pb}Darf ich \textsc{Herzl\pwindex{Herzl, Theodor 2.\,5.\,1860 Budapest – 3.\,7.\,1904 Edlach@\textsc{Herzl, Theodor} (2.\,5.\,1860 Budapest – 3.\,7.\,1904 Edlach), \emph{Schriftsteller, Journalist}|pw}} dein Stück\pwindex{Schnitzler, Arthur 15.\,5.\,1862 Wien – 21.\,10.\,1931 ebd.@\textsc{Schnitzler, Arthur} (15.\,5.\,1862 Wien – 21.\,10.\,1931 ebd.), \emph{Schriftsteller, Mediziner}!Märchen. Schauspiel in drei Aufzügen@\strich\emph{Das Märchen. Schauspiel in drei Aufzügen}|pw} geben?\pend
           
\pstart
           Dabei fällt mir ein, daß dieſer Erfolg in nächſter Saiſon mich einen Freund koſten
                  wird. \strikeout{T} Du wirſt wohlwollend gegen mich werden.
                     \label{K_L02675-5v}\edtext{\textsc{\begin{otherlanguage}{french}Enfin, c’est la vie ça\end{otherlanguage}}}{\lemma{\textnormal{\emph{Enfin, c’est la vie ça}}}\Cendnote{\textnormal{französisch: nun, so ist das
                     Leben}}}\label{K_L02675-5}!\pend
           \selectlanguage{ngerman}\endnumbering\briefempfaengerindex{Schnitzler, Arthur@\textsc{Schnitzler, Arthur}!zzzGoldmann, Paul@\emph{von Paul Goldmann}!1891-12-181@{18. 12. [1891]}|)be}\mylabel{L02675h}  \newcommand{\dateiname}{L02675}\newcommand{\titel}{Paul Goldmann an Arthur Schnitzler, 18. 12. [1891]}\newcommand{\editorInnen}{Martin Anton Müller und Laura Untner}%% latex-leseansicht-abspann.tex
%% Abspann für die Leseansicht.
%% Der Schalter \ifkorrekturansicht ist bereits durch den Vorspann gesetzt.

%% latex-abspann.tex
%% Gemeinsamer Abspann für Korrekturansicht und Leseansicht.
%% Setzt den Schalter \ifkorrekturansicht voraus (gesetzt in den
%% einbindenden Dateien latex-korrekturansicht-abspann.tex bzw.
%% latex-leseansicht-abspann.tex).
%% ---------------------------------------------------------------

\normalsize

% Das esempio-Environment wird nur in der Leseansicht benötigt
\ifkorrekturansicht\else
\newenvironment{esempio}[3]%
{
    \vspace{1.5ex}
    \rlap{\underline{#1}}
    \par
    \setlength{\parindent}{0cm}
    \nopagebreak
    \leftskip=#2cm
    \rightskip=#3cm
}
{
    \par
}
\fi

\doendnotes{C}
\bigskip
\vfill

\clearpage

\footnotesize

\ifkorrekturansicht
  \lohead{\textsc{register}}
\fi

% theindex-Environment neu definieren ohne reledmac
\makeatletter
\renewenvironment{theindex}{%
  \ifkorrekturansicht
    \section*{\indexname}%
  \else
    \subsubsection*{Index der erwähnten Entitäten}%
  \fi
  \setlength{\parindent}{0pt}%
  \setlength{\parskip}{0pt plus 0.3pt}%
  \let\item\@idxitem
}{%
  \ifkorrekturansicht\clearpage\fi
}
\makeatother

\IfFileExists{\jobname-pw.ind}{\input{\jobname-pw.ind}}{}

% Quellenangabe nur in der Leseansicht
\ifkorrekturansicht\else
% Fallback-Definitionen, falls die .tex-Datei \titel etc. nicht gesetzt hat
\providecommand{\titel}{}
\providecommand{\editorInnen}{}
\providecommand{\dateiname}{\jobname}

\vspace{3cm}

\vfill

\footnotesize
\textsc{Quelle}: \titel. Herausgegeben von {\editorInnen}. In: \emph{Arthur Schnitzler: Briefwechsel mit Autorinnen und Autoren}.
 Digitale Edition, https://schnitzler-briefe.acdh.oeaw.ac.at/{\dateiname}.html (Stand \today)
\fi

\end{document}


