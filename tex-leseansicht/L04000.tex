%% latex-leseansicht-vorspann.tex
%% Vorspann für die Leseansicht.
%% Lädt die gemeinsame Datei latex-vorspann.tex mit nicht gesetztem Schalter.

\newif\ifkorrekturansicht
\korrekturansichtfalse

\input{../tex-inputs/latex-vorspann}


\section[Berta Zuckerkandl an Arthur Schnitzler, 30. 7. 1920]{L04000 Berta Zuckerkandl an Arthur Schnitzler, 30. 7. 1920}
\nopagebreak\mylabel{L04000v}
\rehead{ }\normalsize\beginnumbering\briefempfaengerindex{Schnitzler, Arthur@\textsc{Schnitzler, Arthur}!zzzZuckerkandl, Berta@\emph{von Berta Zuckerkandl}!1920-07-302@{30. 7. 1920}|(be}
\toendnotes[C]{\smallbreak\pagebreak[2]}
\correspDesc{Versand  durch Berta Zuckerkandl am 30. 7. 1920 in Breitenstein am
                  Semmering
\newline{}Erhalt  durch Arthur Schnitzler im Zeitraum [30. 7. 1920
                  – 31. 7. 1920?] in Wien}\toendnotes[C]{\smallbreak}
\Standort{CUL, Schnitzler, B 200.}
\physDesc{Karte, 612 Zeichen
\newline{}Handschrift: blaue Tinte, lateinische Kurrent}\toendnotes[C]{\smallbreak}
\pstart
           \raggedleft{}{\pb}30. \substVorne{}\textsuperscript{A.}\substDazwischen{}J.\substHinten{} 1920.\pend
           
\pstart{}Verehrter Freund!\pend\vspace{0.5em}
\pstart
           Darf ich Sie nur um ein Wort bitten, \label{K_L04000-1v}\edtext{wie es Ihnen geht}{\lemma{\textnormal{\emph{wie es Ihnen geht}}}\Cendnote{\textnormal{Berta Zuckerkandl\pwindex{Zuckerkandl, Berta 13.\,4.\,1864 Wien – 16.\,10.\,1945 Paris@\textsc{Zuckerkandl, Berta} (13.\,4.\,1864 Wien – 16.\,10.\,1945 Paris), \emph{Schriftstellerin, Journalistin, Übersetzerin}|pwk} vermittelte zwischen Artur und Olga Schnitzler\pwindex{Schnitzler, Olga 17.\,1.\,1882 Wien – 13.\,1.\,1970 Lugano@\textsc{Schnitzler, Olga} (17.\,1.\,1882 Wien – 13.\,1.\,1970 Lugano), \emph{Schauspielerin, Sängerin}|pwk} in der Krise, die schließlich die Trennung nach sich zog,
                  auch in den Tagen vor ihrer Abreise nach Breitenstein am Semmering\oindex{Breitenstein am Semmering@\textbf{Breitenstein am Semmering}|pwk}, vgl. A. S.: \emph{Tagebuch}, 21. 7. 1920, 22. 7. 1920, 23. 7. 1920.}}}\label{K_L04000-1}, und welchen Verlauf die
               letzte Reise genommen hat? Da das Wetter vorläufig trostlos ist, so habe ich wenig
               Hoffnung Sie eines Tages hier zu sehen. Allerdings wäre dieser Möglichkeit ein
               weiterer Spielraum gegeben, da ich \label{K_L04000-2v}\edtext{auf
                  {\pb}Bitten Alma’s\pwindex{Mahler-Werfel, Alma Maria 31.\,8.\,1879 Wien – 11.\,12.\,1964 New York City@\textsc{Mahler-Werfel, Alma Maria} (31.\,8.\,1879 Wien – 11.\,12.\,1964 New York City)|pw}}{\lemma{\textnormal{\emph{auf
                  Bitten Alma’s}}}\Cendnote{\textnormal{Alma Mahler\pwindex{Mahler-Werfel, Alma Maria 31.\,8.\,1879 Wien – 11.\,12.\,1964 New York City@\textsc{Mahler-Werfel, Alma Maria} (31.\,8.\,1879 Wien – 11.\,12.\,1964 New York City)|pwk} besaß ein Haus\oindex{Haus Mahler@\textbf{Haus Mahler}, \emph{Gebäude}|pwkv} bei Breitenstein am Semmering\oindex{Breitenstein am Semmering@\textbf{Breitenstein am Semmering}|pwk}.}}}\label{K_L04000-2} mein Bleiben bis
                     9\textsuperscript{ten} A. ausdehne. Also – falls die
               Himmelslaune sich austoben sollte – bliebe noch für die ganze nächste Woche ein Wiedersehen – für mich u. Alma\pwindex{Mahler-Werfel, Alma Maria 31.\,8.\,1879 Wien – 11.\,12.\,1964 New York City@\textsc{Mahler-Werfel, Alma Maria} (31.\,8.\,1879 Wien – 11.\,12.\,1964 New York City)|pw} die angenehmste Erwartung.\pend
           
\pstart
           Von diesen Zeilen sagen Sie bitte lieber nichts und geben Sie ein Lebenszeichen
               Ihrer getreuesten Freundin{\\[\baselineskip]}\spacefill\mbox{B. Z.}\pend
           \leftskip=0em{}\selectlanguage{ngerman}\endnumbering\briefempfaengerindex{Schnitzler, Arthur@\textsc{Schnitzler, Arthur}!zzzZuckerkandl, Berta@\emph{von Berta Zuckerkandl}!1920-07-302@{30. 7. 1920}|)be}\mylabel{L04000h}
\begin{anhang}
\end{anhang}\newcommand{\dateiname}{L04000}\newcommand{\titel}{Berta Zuckerkandl an Arthur Schnitzler, 30. 7. 1920}\newcommand{\editorInnen}{Herausgegeben von Jahnke, SelmaMüller, Martin Anton}%% latex-leseansicht-abspann.tex
%% Abspann für die Leseansicht.
%% Der Schalter \ifkorrekturansicht ist bereits durch den Vorspann gesetzt.

%% latex-abspann.tex
%% Gemeinsamer Abspann für Korrekturansicht und Leseansicht.
%% Setzt den Schalter \ifkorrekturansicht voraus (gesetzt in den
%% einbindenden Dateien latex-korrekturansicht-abspann.tex bzw.
%% latex-leseansicht-abspann.tex).
%% ---------------------------------------------------------------

\normalsize

% Das esempio-Environment wird nur in der Leseansicht benötigt
\ifkorrekturansicht\else
\newenvironment{esempio}[3]%
{
    \vspace{1.5ex}
    \rlap{\underline{#1}}
    \par
    \setlength{\parindent}{0cm}
    \nopagebreak
    \leftskip=#2cm
    \rightskip=#3cm
}
{
    \par
}
\fi

\doendnotes{C}
\bigskip
\vfill

\clearpage

\footnotesize

\ifkorrekturansicht
  \lohead{\textsc{register}}
\fi

% theindex-Environment neu definieren ohne reledmac
\makeatletter
\renewenvironment{theindex}{%
  \ifkorrekturansicht
    \section*{\indexname}%
  \else
    \subsubsection*{Index der erwähnten Entitäten}%
  \fi
  \setlength{\parindent}{0pt}%
  \setlength{\parskip}{0pt plus 0.3pt}%
  \let\item\@idxitem
}{%
  \ifkorrekturansicht\clearpage\fi
}
\makeatother

\IfFileExists{\jobname-pw.ind}{\input{\jobname-pw.ind}}{}

% Quellenangabe nur in der Leseansicht
\ifkorrekturansicht\else
% Fallback-Definitionen, falls die .tex-Datei \titel etc. nicht gesetzt hat
\providecommand{\titel}{}
\providecommand{\editorInnen}{}
\providecommand{\dateiname}{\jobname}

\vspace{3cm}

\vfill

\footnotesize
\textsc{Quelle}: \titel. Herausgegeben von {\editorInnen}. In: \emph{Arthur Schnitzler: Briefwechsel mit Autorinnen und Autoren}.
 Digitale Edition, https://schnitzler-briefe.acdh.oeaw.ac.at/{\dateiname}.html (Stand \today)
\fi

\end{document}


