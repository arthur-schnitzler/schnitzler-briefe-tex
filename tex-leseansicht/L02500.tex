%% latex-leseansicht-vorspann.tex
%% Vorspann für die Leseansicht.
%% Lädt die gemeinsame Datei latex-vorspann.tex mit nicht gesetztem Schalter.

\newif\ifkorrekturansicht
\korrekturansichtfalse

\input{../tex-inputs/latex-vorspann}


         
         \renewcommand{\erwaehntePersonen}{Personen: Robert Adam, Henrik Ibsen, Arthur Schopenhauer}
         \renewcommand{\erwaehnteOrte}{Orte: Jardin du Luxembourg, Prater, Schloss Schönbrunn, Volksgarten, Wien}
         \renewcommand{\erwaehnteWerke}{Werke: Ein Leben, Märchenkomödie, Therese. Chronik eines Frauenlebens, Upanishaden}
               \section[Robert Adam an Arthur Schnitzler, 11. 5. 1928]{ Robert Adam an Arthur Schnitzler, 11. 5. 1928}\nopagebreak\mylabel{v}\rehead{ }\begin{ledgroupsized}[t]{13cm}\normalsize\beginnumbering\briefempfaengerindex{Schnitzler, Arthur@\textsc{Schnitzler, Arthur}!zzzAdam, Robert@\emph{von Robert Adam}!1928-05-111@{11. 5. 1928}|(be} \toendnotes[C]{\smallbreak\pagebreak[2]} \Standort{CUL, Schnitzler, B 1.}
\physDesc{Brief, 1 Blatt, 4 Seiten, 2730 Zeichen (Briefpapier mit Trauerrand)
\newline{}Handschrift: schwarze Tinte, deutsche Kurrent
\newline{}Schnitzler: 1) mit Bleistift beschriftet: »\textsc{Adam}«  2) mit rotem Buntstift Vermerk: »\textsc{Therese}\pwindex{Schnitzler, Arthur 15.05.1862 – 21.10.1931@\textsc{Schnitzler, Arthur} (15.05.1862 – 21.10.1931), \emph{Schriftsteller, Mediziner}!Therese. Chronik eines Frauenlebens1928-03-27@\strich\emph{Therese. Chronik eines Frauenlebens} {[}1928-03-27{]}|pw}« und vereinzelte Unterstreichungen
\newline{}Ordnung: mit Bleistift von unbekannter Hand nummeriert:
                                    »20« }\Standort{Wien, Österreichische Nationalbibliothek, Cod.ser. 52.268, 355 verso, 356.}
\physDesc{handschriftliche Abschrift, 2 Blätter, 2 Seiten, 2730 Zeichen
\newline{}Handschrift: schwarze Tinte, Gabelsberger Kurzschrift}\Standort{Wien, Österreichische Nationalbibliothek, Cod.ser. 52.268, 355 verso, 356.}
\physDesc{maschinenschriftliche Abschrift, 2 Blätter, 2 Seiten, 2730 Zeichen
\newline{}Schreibmaschine}\toendnotes[C]{\smallbreak}\pstart
           \raggedleft{}{\pb}Wien\oindex{Wien@\textbf{Wien}|pw}, am 11. Mai 1928\pend
           \pstart{}Hochverehrter Herr Doktor!\pend\pstart
           Ich vermute, daß Sie nunmehr von Ihrer Reiſe in Gegenden, zu denen auch mich ſeit
               Jahren eine in meine ſtändigen Lektüre wurzelnde, noch unerfüllbare Sehnſucht oder
               Neugier lockt, von den Erdbeben unbetroffen zurückgekehrt ſind, und will Ihnen für
               zwei Dinge danken.\pend
           \pstart
           Vorerſt für Ihren Roman\pwindex{Schnitzler, Arthur 15.05.1862 – 21.10.1931@\textsc{Schnitzler, Arthur} (15.05.1862 – 21.10.1931), \emph{Schriftsteller, Mediziner}!Therese. Chronik eines Frauenlebens1928-03-27@\strich\emph{Therese. Chronik eines Frauenlebens} {[}1928-03-27{]}|pwv}, den
               ich in der freien Zeit, die mir meine jetzt grauſam-anstrengende Amtstätigkeit ließ,
               mit herzhafter Freude und bewunderndem Schauer geleſen habe. {\pb}Ich habe natürlich Ihre Thereſe gekannt,
               wenn auch nicht unter dieſem Namen; ich kannte ſie unter mancherlei Geſtalten, von
               Kindheit auf, als ſie um mich bemüht war – damals hieß ſie vor allem Fräulein
               Joſefine –, und ſpäterhin, als ich, ein junger Menſch, um ſie bemüht war, im Volksgarten\oindex{Volksgarten@\textbf{Volksgarten}|pw}, im Prater\oindex{Prater@\textbf{Prater}|pw}, in Schönbrunn\oindex{Schloss Schoenbrunn@\textbf{Schloss Schönbrunn}|pw} und auch im Luxembourg\oindex{Jardin du Luxembourg@\textbf{Jardin du Luxembourg}|pw}, und ſchließlich iſt ſie mir oft bei
               Gericht entgegengetreten. Aber in welch wunderbar-exakte einfache Chronik haben Sie
               den furchtbar-troſtloſen Lebenslauf dieser ſympathiſchen Alltagskreatur
               zuſammengefaßt! Ich kenne nur noch ein Buch, das, wie Ihr Schopenhauer\pwindex{Schopenhauer, Arthur 22.02.1788 – 21.09.1860@\textsc{Schopenhauer, Arthur} (22.02.1788 – 21.09.1860), \emph{Philosoph}|pw}iſches,die unendliche Troſt- und Fruchtloſigkeit
               des Menſchendaſeins (\textsc{\label{K_L02500-1v}\edtext{tat twam asi}{\lemma{\textnormal{\emph{tat twam asi}}}\Cendnote{\textnormal{»Das bist Du!«, wie Schopenhauer\pwindex{Schopenhauer, Arthur 22.02.1788 – 21.09.1860@\textsc{Schopenhauer, Arthur} (22.02.1788 – 21.09.1860), \emph{Philosoph}|pwk} den Satz aus den \emph{Upanishaden}\pwindex{?? Werk@Nicht ermittelte Verfasserinnen und Verfasser!Upanishaden@\emph{Upanishaden}|pwk} übersetzte.}}}\label{K_L02500-1h}}) im Aufrollen der Qual eines endloſen Einzelſchickſals aufzeigt: \textsc{Une Vie\pwindex{\textcolor{red}{\textsuperscript{XXXX1 indx}}!Leben1883@\strich\emph{Ein Leben} {[}1883{]}|pw}}.\pend
           \pstart
           {\pb}Nur der Juriſt in mir, dem alles
               Menſchliche nur Tatbestand iſt, fühlt ſich nicht gleich befriedigt: denn er ſchüttelt
               darüber den Kopf, daß Thereſens\pwindex{Schnitzler, Arthur 15.05.1862 – 21.10.1931@\textsc{Schnitzler, Arthur} (15.05.1862 – 21.10.1931), \emph{Schriftsteller, Mediziner}!Therese. Chronik eines Frauenlebens1928-03-27@\strich\emph{Therese. Chronik eines Frauenlebens} {[}1928-03-27{]}|pwv} böſer Bub ganz ohne Vormund auskommen muß – trotz der gut
               funktionierenden Wien\oindex{Wien@\textbf{Wien}|pw}er Vormundſchaftsgerichte –,
               und auch die Altersgrenze von ſechzehn Jahren (auf S. 277) will ihm nicht gefallen.
               Aber dieſe kleinlichen Bedenken der Juriſten haben einem großen Kunstwerk gegenüber,
               wie Ihr Roman\pwindex{Schnitzler, Arthur 15.05.1862 – 21.10.1931@\textsc{Schnitzler, Arthur} (15.05.1862 – 21.10.1931), \emph{Schriftsteller, Mediziner}!Therese. Chronik eines Frauenlebens1928-03-27@\strich\emph{Therese. Chronik eines Frauenlebens} {[}1928-03-27{]}|pwv} es iſt, wirklich
               nichts zu beſagen.\pend
           \pstart
           Und dann danke ich Ihnen herzlich für die Mühe, die Sie ſich mit der Lektüre meiner
               korpulenten Komödie\pwindex{Adam, Robert 20.04.1877 – 16.10.1961@\textsc{Adam, Robert} (20.04.1877 – 16.10.1961), \emph{Schriftsteller, Richter}!Maerchenkomoedie@\strich\emph{Märchenkomödie}|pwv} gemacht
               haben, und für Ihren liebenswürdigen kritiſierenden Brief. Ich bin für die Mängel
               meiner Arbeit keineswegs blind. Als einen ihrer Hauptfehler ſehe ich es an, daß der
               gedankliche Aufbau in einer theaterwidrigen und abſtruſen Szene – der Wanderung durch
               das Gehirn {\pb}und Unterbewußtſein in’s
               Tranſzendente – gipfelt, während der Höhepunkt des äußeren Geſchehens, der Sieg der
                  Revolu\damage{tio}n, ganz gegen den Schluß verſchoben iſt, ſodaß Inkongruenz und Unſymmetrie
               beſtehen. Auch die unwillkürliche Annäherung an den von mir zwar geehrten, aber tief
               perhorreſzierten Ibſen\pwindex{Ibsen, Henrik 20.03.1828 – 23.05.1906@\textsc{Ibsen, Henrik} (20.03.1828 – 23.05.1906), \emph{Schriftsteller}|pw} iſt mir ſehr unangenehm
               und für die Erſchaffung dieſer unverzeihlichen Liga möchte ich mich am liebſten,
               wenn’s nicht ohnedies zu ſpät wäre, ſelbſt prügeln.\pend
           \pstart
           Hoffentlich flicht ſich meine nächſte Arbeit um einen weniger abſurden Stoff. Es iſt
               ſchrecklich, daß man Stoffe nicht wählen kann.\pend
           \pstart
           Mit den beſten Grüßen und Empfehlungen\hspace*{1.5em}Ihr\pend
           \pstart
           tief ergebener{\\[\baselineskip]}\spacefill\mbox{D\textsuperscript{r}RAdam}\pend
           \leftskip=0em{}
         
         \endnumbering\mylabel{h}\end{ledgroupsized}  \newcommand{\dateiname}{L02500}\newcommand{\titel}{Robert Adam an Arthur Schnitzler, 11. 5. 1928}\newcommand{\editorInnen}{Martin Anton Müller und Gerd-Hermann Susen}%% latex-leseansicht-abspann.tex
%% Abspann für die Leseansicht.
%% Der Schalter \ifkorrekturansicht ist bereits durch den Vorspann gesetzt.

%% latex-abspann.tex
%% Gemeinsamer Abspann für Korrekturansicht und Leseansicht.
%% Setzt den Schalter \ifkorrekturansicht voraus (gesetzt in den
%% einbindenden Dateien latex-korrekturansicht-abspann.tex bzw.
%% latex-leseansicht-abspann.tex).
%% ---------------------------------------------------------------

\normalsize

% Das esempio-Environment wird nur in der Leseansicht benötigt
\ifkorrekturansicht\else
\newenvironment{esempio}[3]%
{
    \vspace{1.5ex}
    \rlap{\underline{#1}}
    \par
    \setlength{\parindent}{0cm}
    \nopagebreak
    \leftskip=#2cm
    \rightskip=#3cm
}
{
    \par
}
\fi

\doendnotes{C}
\bigskip
\vfill

\clearpage

\footnotesize

\ifkorrekturansicht
  \lohead{\textsc{register}}
\fi

% theindex-Environment neu definieren ohne reledmac
\makeatletter
\renewenvironment{theindex}{%
  \ifkorrekturansicht
    \section*{\indexname}%
  \else
    \subsubsection*{Index der erwähnten Entitäten}%
  \fi
  \setlength{\parindent}{0pt}%
  \setlength{\parskip}{0pt plus 0.3pt}%
  \let\item\@idxitem
}{%
  \ifkorrekturansicht\clearpage\fi
}
\makeatother

\IfFileExists{\jobname-pw.ind}{\input{\jobname-pw.ind}}{}

% Quellenangabe nur in der Leseansicht
\ifkorrekturansicht\else
% Fallback-Definitionen, falls die .tex-Datei \titel etc. nicht gesetzt hat
\providecommand{\titel}{}
\providecommand{\editorInnen}{}
\providecommand{\dateiname}{\jobname}

\vspace{3cm}

\vfill

\footnotesize
\textsc{Quelle}: \titel. Herausgegeben von {\editorInnen}. In: \emph{Arthur Schnitzler: Briefwechsel mit Autorinnen und Autoren}.
 Digitale Edition, https://schnitzler-briefe.acdh.oeaw.ac.at/{\dateiname}.html (Stand \today)
\fi

\end{document}


      