%% latex-leseansicht-vorspann.tex
%% Vorspann für die Leseansicht.
%% Lädt die gemeinsame Datei latex-vorspann.tex mit nicht gesetztem Schalter.

\newif\ifkorrekturansicht
\korrekturansichtfalse

\input{../tex-inputs/latex-vorspann}


\section[ Paul Goldmann an Arthur Schnitzler, 14. 2. {[}1901{]}]{L03058 Paul Goldmann an Arthur Schnitzler,  14. 2. [1901]}
\nopagebreak\mylabel{L03058v}
\rehead{ }\normalsize\beginnumbering\briefempfaengerindex{Schnitzler, Arthur@\textsc{Schnitzler, Arthur}!zzzGoldmann, Paul@\emph{von Paul Goldmann}!1901-02-141@{14. 2. [1901]}|(be}
\toendnotes[C]{\smallbreak\pagebreak[2]}
\correspDesc{Versand  durch Paul Goldmann am 14. 2. [1901] in Berlin
\newline{}Erhalt  durch Arthur Schnitzler im Zeitraum [15. 2. 1901
                  – 19. 2. 1901?] in Wien}\toendnotes[C]{\smallbreak}
\Standort{DLA, A:Schnitzler, HS.NZ85.1.3171.}
\physDesc{Brief, 1 Blatt, 1 Seite, 770 Zeichen
\newline{}Handschrift: blaue Tinte, deutsche Kurrent
\newline{}Beilage: handschriftlicher Brief von Meyer, 1 Blatt, 2 Seiten, schwarze
                                 Tinte, deutsche Kurrent 
\newline{}Schnitzler: mit Bleistift das Jahr »901« vermerkt }\toendnotes[C]{\smallbreak}
\pstart
           \raggedleft{}{\pb}\textcolor{gray}{\textbf{DESSAUERSTRASSE 19}}\oindex{Dessauer Straße@\textbf{Dessauer Straße}, \emph{Straße}|pw}\pend
           
\pstart
           Berlin\oindex{Berlin@\textbf{Berlin}, \emph{Hauptstadt}|pw}, 14. Februar.\pend
           
\pstart\center{}Mein lieber Freund,\pend\vspace{0.5em}
\pstart
           Ein \label{K_L03058-1v}\edtext{\textsc{Dr. Meyer\pwindex{Meyer @\textsc{Meyer}, \emph{Arzt}|pw}}}{\lemma{\textnormal{\emph{Dr. Meyer}}}\Cendnote{\textnormal{nicht ermittelt}}}\label{K_L03058-1}, der mit den \textsc{Glümers\pwindex{Glümer, Marie 3.\,7.\,1867 Wien – 16.\,11.\,1925 München@\textsc{Glümer, Marie} (3.\,7.\,1867 Wien – 16.\,11.\,1925 München), \emph{Schauspielerin}|pwv}\pwindex{Glümer, Auguste 16.\,3.\,1862 Wien – 1956@\textsc{Glümer, Auguste} (16.\,3.\,1862 Wien – 1956), \emph{Lehrerin}|pwv}} bekannt iſt, hat \textsc{Mizzi\pwindex{Glümer, Marie 3.\,7.\,1867 Wien – 16.\,11.\,1925 München@\textsc{Glümer, Marie} (3.\,7.\,1867 Wien – 16.\,11.\,1925 München), \emph{Schauspielerin}|pw}} zu \textsc{Prof. Renvers\pwindex{Renvers, Rudolf 18.\,2.\,1854 Aachen – 22.\,3.\,1909 Berlin@\textsc{Renvers, Rudolf} (18.\,2.\,1854 Aachen – 22.\,3.\,1909 Berlin), \emph{Mediziner, Universitätslehrer}|pw}} begleitet. Ich bat \textsc{Gusti\pwindex{Glümer, Auguste 16.\,3.\,1862 Wien – 1956@\textsc{Glümer, Auguste} (16.\,3.\,1862 Wien – 1956), \emph{Lehrerin}|pw}}, mich mit dieſem \textsc{Dr. Meyer\pwindex{Meyer @\textsc{Meyer}, \emph{Arzt}|pw}} in Verbindung zu{ }ſetzen. Die Folge\strikeout{n} war
               beiliegender Brief, aus dem ich auch nicht{ }ſehr klug werde. Vielleicht{ }ſagt er Dir
               mehr als mir.\pend
           
\pstart
           Viele Grüße! {\\[\baselineskip]}Dein {\\[\baselineskip]}\spacefill\mbox{Paul Goldmn}\pend
           \leftskip=0em{}\selectlanguage{ngerman}\vspace{1em}{\vspace{1\baselineskip}}
\pstart
           \raggedleft{}{\pb}{[}hs. Meyer:{]} B.\oindex{Berlin@\textbf{Berlin}, \emph{Hauptstadt}|pw}{ }Montag.\pend
           
\pstart\center{}Sehr geehrter Herr Doctor!\pend\vspace{0.5em}
\pstart
           Auf Wunſch von \label{K_L03058-2v}\edtext{Fräulein \textsc{Glümer\pwindex{Glümer, Marie 3.\,7.\,1867 Wien – 16.\,11.\,1925 München@\textsc{Glümer, Marie} (3.\,7.\,1867 Wien – 16.\,11.\,1925 München), \emph{Schauspielerin}|pw}}}{\lemma{\textnormal{\emph{Fräulein Glümer}}}\Cendnote{\textnormal{Siehe XXXX Auszeichnungsfehler: Dokument L03055 nicht gefunden.
               }}}\label{K_L03058-2} erlaube ich mir die ergebene Mitteilung, daß ihre Erkrankung auf einer{ }ſchlechten Zuſammenſetzung des Blutes + der übrigen Körperſäfte beruht, deren Schwere
               durch die lange Vernachläſſigung bedingt iſt. –\pend
           
\pstart
           Das Weſentliche für {\pb}ihre Freunde iſt ja die
               Thatſache, daß{ }ſie in 4 Wochen ca mit Sicherheit völlig geſund{ }ſein wird.\pend
           
\pstart
           Mit vorzüglichſter Hochſchätzung empfiehlt{ }ſich Ihnen {\\[\baselineskip]}ganz ergebſt { }{\\[\baselineskip]}\spacefill\mbox{Meyer}\pend
           \leftskip=0em{}\selectlanguage{ngerman}\endnumbering\briefempfaengerindex{Schnitzler, Arthur@\textsc{Schnitzler, Arthur}!zzzGoldmann, Paul@\emph{von Paul Goldmann}!1901-02-141@{14. 2. [1901]}|)be}\mylabel{L03058h}  \newcommand{\dateiname}{L03058}\newcommand{\titel}{Paul Goldmann an Arthur Schnitzler, 14. 2. [1901]}\newcommand{\editorInnen}{Martin Anton Müller und Laura Untner}%% latex-leseansicht-abspann.tex
%% Abspann für die Leseansicht.
%% Der Schalter \ifkorrekturansicht ist bereits durch den Vorspann gesetzt.

%% latex-abspann.tex
%% Gemeinsamer Abspann für Korrekturansicht und Leseansicht.
%% Setzt den Schalter \ifkorrekturansicht voraus (gesetzt in den
%% einbindenden Dateien latex-korrekturansicht-abspann.tex bzw.
%% latex-leseansicht-abspann.tex).
%% ---------------------------------------------------------------

\normalsize

% Das esempio-Environment wird nur in der Leseansicht benötigt
\ifkorrekturansicht\else
\newenvironment{esempio}[3]%
{
    \vspace{1.5ex}
    \rlap{\underline{#1}}
    \par
    \setlength{\parindent}{0cm}
    \nopagebreak
    \leftskip=#2cm
    \rightskip=#3cm
}
{
    \par
}
\fi

\doendnotes{C}
\bigskip
\vfill

\clearpage

\footnotesize

\ifkorrekturansicht
  \lohead{\textsc{register}}
\fi

% theindex-Environment neu definieren ohne reledmac
\makeatletter
\renewenvironment{theindex}{%
  \ifkorrekturansicht
    \section*{\indexname}%
  \else
    \subsubsection*{Index der erwähnten Entitäten}%
  \fi
  \setlength{\parindent}{0pt}%
  \setlength{\parskip}{0pt plus 0.3pt}%
  \let\item\@idxitem
}{%
  \ifkorrekturansicht\clearpage\fi
}
\makeatother

\IfFileExists{\jobname-pw.ind}{\input{\jobname-pw.ind}}{}

% Quellenangabe nur in der Leseansicht
\ifkorrekturansicht\else
% Fallback-Definitionen, falls die .tex-Datei \titel etc. nicht gesetzt hat
\providecommand{\titel}{}
\providecommand{\editorInnen}{}
\providecommand{\dateiname}{\jobname}

\vspace{3cm}

\vfill

\footnotesize
\textsc{Quelle}: \titel. Herausgegeben von {\editorInnen}. In: \emph{Arthur Schnitzler: Briefwechsel mit Autorinnen und Autoren}.
 Digitale Edition, https://schnitzler-briefe.acdh.oeaw.ac.at/{\dateiname}.html (Stand \today)
\fi

\end{document}


