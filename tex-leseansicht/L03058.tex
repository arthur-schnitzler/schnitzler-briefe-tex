%% latex-leseansicht-vorspann.tex
%% Vorspann für die Leseansicht.
%% Lädt die gemeinsame Datei latex-vorspann.tex mit nicht gesetztem Schalter.

\newif\ifkorrekturansicht
\korrekturansichtfalse

\input{../tex-inputs/latex-vorspann}


         
         \renewcommand{\erwaehntePersonen}{Personen: Marie Glümer, Auguste Glümer,  Meyer, Rudolf Renvers}
         \renewcommand{\erwaehnteOrte}{Orte: Berlin, Dessauer Straße, Wien}
         \renewcommand{\erwaehnteWerke}{}
               \section[ Paul Goldmann an Arthur Schnitzler, 14. 2. {[}1901{]}]{ Paul Goldmann an Arthur Schnitzler, 14. 2. {[}1901{]}}\nopagebreak\mylabel{v}\rehead{ }\begin{ledgroupsized}[t]{13cm}\normalsize\beginnumbering \toendnotes[C]{\smallbreak\pagebreak[2]} \Standort{DLA, A:Schnitzler, HS.NZ85.1.3171.}
\physDesc{Brief, 1 Blatt, 1 Seite, 770 Zeichen
\newline{}Handschrift: blaue Tinte, deutsche Kurrent
\newline{}Beilage: handschriftlicher Brief von Meyer, 1 Blatt, 2 Seiten, schwarze
                                 Tinte, deutsche Kurrent 
\newline{}Schnitzler: mit Bleistift das Jahr »{[}1{]}901« vermerkt }\toendnotes[C]{\smallbreak}\pstart
           \noindent{}\raggedleft{}{\pb}\textcolor{gray}{\textbf{DESSAUERSTRASSE 19}}\oindex{Dessauer Strasse@\textbf{Dessauer Straße}|pw}\pend
           \pstart
           Berlin\oindex{Berlin@\textbf{Berlin}|pw}, 14. Februar.\pend
           \pstart\center{}Mein lieber Freund,\pend\pstart
           Ein \label{K_L03058-1v}\edtext{\textsc{Dr. Meyer\pwindex{Meyer @\textsc{Meyer}, \emph{Arzt}|pw}}}{\lemma{\textnormal{\emph{Dr. Meyer}}}\Cendnote{\textnormal{nicht ermittelt}}}\label{K_L03058-1h}, der mit den \textsc{Glümers\pwindex{Gluemer, Marie 03.07.1867 – 16.11.1925@\textsc{Glümer, Marie} (03.07.1867 – 16.11.1925), \emph{Schauspielerin}|pwv}\pwindex{Gluemer, Auguste 16.03.1862 – 1956@\textsc{Glümer, Auguste} (16.03.1862 – 1956)|pwv}} bekannt iſt, hat \textsc{Mizzi\pwindex{Gluemer, Marie 03.07.1867 – 16.11.1925@\textsc{Glümer, Marie} (03.07.1867 – 16.11.1925), \emph{Schauspielerin}|pw}} zu \textsc{Prof. Renvers\pwindex{Renvers, Rudolf 1854-02-18 – 1909-03-22@\textsc{Renvers, Rudolf} (1854-02-18 – 1909-03-22), \emph{Mediziner, Universitätslehrer}|pw}} begleitet. Ich bat \textsc{Gusti\pwindex{Gluemer, Auguste 16.03.1862 – 1956@\textsc{Glümer, Auguste} (16.03.1862 – 1956)|pw}}, mich mit dieſem \textsc{Dr. Meyer\pwindex{Meyer @\textsc{Meyer}, \emph{Arzt}|pw}} in Verbindung zu ſetzen. Die Folge\strikeout{n} war
               beiliegender Brief, aus dem ich auch nicht ſehr klug werde. Vielleicht ſagt er Dir
               mehr als mir.\pend
           \pstart
           Viele Grüße! {\\[\baselineskip]}Dein {\\[\baselineskip]}\spacefill\mbox{Paul Goldmn}\pend
           \leftskip=0em{}{\bigskip}\pstart
           \raggedleft{}{\pb}{[}hs. Meyer:{]} B.\oindex{Berlin@\textbf{Berlin}|pw}{ }Montag.\pend
           \pstart\center{}Sehr geehrter Herr Doctor!\pend\pstart
           Auf Wunſch von \label{K_L03058-3v}\edtext{Fräulein \textsc{Glümer\pwindex{Gluemer, Marie 03.07.1867 – 16.11.1925@\textsc{Glümer, Marie} (03.07.1867 – 16.11.1925), \emph{Schauspielerin}|pw}}}{\lemma{\textnormal{\emph{Fräulein Glümer}}}\Cendnote{\textnormal{siehe Paul Goldmann an Arthur Schnitzler, 22. 1. [1901]}}}\label{K_L03058-3h} erlaube ich mir die ergebene Mitteilung, daß ihre Erkrankung auf einer
               ſchlechten Zuſammenſetzung des Blutes + der übrigen Körperſäfte beruht, deren Schwere
               durch die lange Vernachläſſigung bedingt iſt. –\pend
           \pstart
           Das Weſentliche für {\pb}ihre Freunde iſt ja die
               Thatſache, daß ſie in 4 Wochen ca mit Sicherheit völlig geſund ſein wird.\pend
           \pstart
           Mit vorzüglichſter Hochſchätzung empfiehlt ſich Ihnen {\\[\baselineskip]}ganz ergebſt { }{\\[\baselineskip]}\spacefill\mbox{Meyer}\pend
           \leftskip=0em{}
         
         \endnumbering\mylabel{h}\end{ledgroupsized}  \newcommand{\dateiname}{L03058}\newcommand{\titel}{Paul Goldmann an Arthur Schnitzler, 14. 2. [1901]}\newcommand{\editorInnen}{Martin Anton Müller und Laura Untner}%% latex-leseansicht-abspann.tex
%% Abspann für die Leseansicht.
%% Der Schalter \ifkorrekturansicht ist bereits durch den Vorspann gesetzt.

%% latex-abspann.tex
%% Gemeinsamer Abspann für Korrekturansicht und Leseansicht.
%% Setzt den Schalter \ifkorrekturansicht voraus (gesetzt in den
%% einbindenden Dateien latex-korrekturansicht-abspann.tex bzw.
%% latex-leseansicht-abspann.tex).
%% ---------------------------------------------------------------

\normalsize

% Das esempio-Environment wird nur in der Leseansicht benötigt
\ifkorrekturansicht\else
\newenvironment{esempio}[3]%
{
    \vspace{1.5ex}
    \rlap{\underline{#1}}
    \par
    \setlength{\parindent}{0cm}
    \nopagebreak
    \leftskip=#2cm
    \rightskip=#3cm
}
{
    \par
}
\fi

\doendnotes{C}
\bigskip
\vfill

\clearpage

\footnotesize

\ifkorrekturansicht
  \lohead{\textsc{register}}
\fi

% theindex-Environment neu definieren ohne reledmac
\makeatletter
\renewenvironment{theindex}{%
  \ifkorrekturansicht
    \section*{\indexname}%
  \else
    \subsubsection*{Index der erwähnten Entitäten}%
  \fi
  \setlength{\parindent}{0pt}%
  \setlength{\parskip}{0pt plus 0.3pt}%
  \let\item\@idxitem
}{%
  \ifkorrekturansicht\clearpage\fi
}
\makeatother

\IfFileExists{\jobname-pw.ind}{\input{\jobname-pw.ind}}{}

% Quellenangabe nur in der Leseansicht
\ifkorrekturansicht\else
% Fallback-Definitionen, falls die .tex-Datei \titel etc. nicht gesetzt hat
\providecommand{\titel}{}
\providecommand{\editorInnen}{}
\providecommand{\dateiname}{\jobname}

\vspace{3cm}

\vfill

\footnotesize
\textsc{Quelle}: \titel. Herausgegeben von {\editorInnen}. In: \emph{Arthur Schnitzler: Briefwechsel mit Autorinnen und Autoren}.
 Digitale Edition, https://schnitzler-briefe.acdh.oeaw.ac.at/{\dateiname}.html (Stand \today)
\fi

\end{document}


      