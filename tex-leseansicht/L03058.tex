%% latex-korrekturansicht-vorspann.tex
%% Vorspann für die Korrekturansicht.
%% Lädt die gemeinsame Datei latex-vorspann.tex mit gesetztem Schalter.

\newif\ifkorrekturansicht
\korrekturansichttrue

\input{../tex-inputs/latex-vorspann}


\section[ Paul Goldmann an Arthur Schnitzler, 14. 2. {[}1901{]}]{L03058 Paul Goldmann an Arthur Schnitzler, 14. 2. {[}1901{]}}
\nopagebreak\mylabel{L03058v}
\rehead{ }\normalsize\beginnumbering\briefempfaengerindex{Schnitzler, Arthur@\textsc{Schnitzler, Arthur}!zzzGoldmann, Paul@\emph{von Paul Goldmann}!1901-02-141@{14. 2. {[}1901{]}}|(be}
\toendnotes[C]{\smallbreak\pagebreak[2]}\Standort{DLA, A:Schnitzler, HS.NZ85.1.3171.}
\physDesc{Brief, 1 Blatt, 1 Seite, 770 Zeichen
\newline{}Handschrift: blaue Tinte, deutsche Kurrent
\newline{}Beilage: handschriftlicher Brief von Meyer, 1 Blatt, 2 Seiten, schwarze
                                 Tinte, deutsche Kurrent 
\newline{}Schnitzler: mit Bleistift das Jahr »901« vermerkt }\toendnotes[C]{\smallbreak}
\pstart
           \raggedleft{}{\pb}\textcolor{gray}{\textbf{DESSAUERSTRASSE 19}}\oindex{Dessauer Strasse@\textbf{Dessauer Straße}, \emph{Straße (K.STR)}|pw}\pend
           
\pstart
           Berlin\oindex{Berlin@\textbf{Berlin}, \emph{P.PPLC}|pw}, 14. Februar.\pend
           
\pstart\center{}Mein lieber Freund,\pend\vspace{0.5em}
\pstart
           Ein \label{K_L03058-1v}\edtext{\textsc{Dr. Meyer\pwindex{Meyer @\textsc{Meyer}, \emph{Arzt/Ärztin}|pw}}}{\lemma{\textnormal{\emph{Dr. Meyer}}}\Cendnote{\textnormal{nicht ermittelt}}}\label{K_L03058-1}, der mit den \textsc{Glümers\pwindex{Gluemer, Marie 03.07.1867 – 16.11.1925@\textsc{Glümer, Marie} (03.07.1867 – 16.11.1925), \emph{Schauspieler/Schauspielerin}|pwv}\pwindex{Gluemer, Auguste 1862-03-16 – 1956@\textsc{Glümer, Auguste} (1862-03-16 – 1956), \emph{Lehrer/Lehrerin}|pwv}} bekannt iſt, hat \textsc{Mizzi\pwindex{Gluemer, Marie 03.07.1867 – 16.11.1925@\textsc{Glümer, Marie} (03.07.1867 – 16.11.1925), \emph{Schauspieler/Schauspielerin}|pw}} zu \textsc{Prof. Renvers\pwindex{Renvers, Rudolf 1854-02-18 – 1909-03-22@\textsc{Renvers, Rudolf} (1854-02-18 – 1909-03-22), \emph{Mediziner/Medizinerin, Universitätslehrer/Universitätslehrerin}|pw}} begleitet. Ich bat \textsc{Gusti\pwindex{Gluemer, Auguste 1862-03-16 – 1956@\textsc{Glümer, Auguste} (1862-03-16 – 1956), \emph{Lehrer/Lehrerin}|pw}}, mich mit dieſem \textsc{Dr. Meyer\pwindex{Meyer @\textsc{Meyer}, \emph{Arzt/Ärztin}|pw}} in Verbindung zu ſetzen. Die Folge\strikeout{n} war
               beiliegender Brief, aus dem ich auch nicht ſehr klug werde. Vielleicht ſagt er Dir
               mehr als mir.\pend
           
\pstart
           Viele Grüße! {\\[\baselineskip]}Dein {\\[\baselineskip]}\spacefill\mbox{Paul Goldmn}\pend
           \leftskip=0em{}\selectlanguage{ngerman}\vspace{1em}{\vspace{1\baselineskip}}
\pstart
           \raggedleft{}{\pb}{[}hs. :{]} B.\oindex{Berlin@\textbf{Berlin}, \emph{P.PPLC}|pw}{ }Montag.\pend
           
\pstart\center{}Sehr geehrter Herr Doctor!\pend\vspace{0.5em}
\pstart
           Auf Wunſch von \label{K_L03058-2v}\edtext{Fräulein \textsc{Glümer\pwindex{Gluemer, Marie 03.07.1867 – 16.11.1925@\textsc{Glümer, Marie} (03.07.1867 – 16.11.1925), \emph{Schauspieler/Schauspielerin}|pw}}}{\lemma{\textnormal{\emph{Fräulein Glümer}}}\Cendnote{\textnormal{Siehe Paul Goldmann an Arthur Schnitzler, 22. 1. [1901].
               }}}\label{K_L03058-2} erlaube ich mir die ergebene Mitteilung, daß ihre Erkrankung auf einer
               ſchlechten Zuſammenſetzung des Blutes + der übrigen Körperſäfte beruht, deren Schwere
               durch die lange Vernachläſſigung bedingt iſt. –\pend
           
\pstart
           Das Weſentliche für {\pb}ihre Freunde iſt ja die
               Thatſache, daß ſie in 4 Wochen ca mit Sicherheit völlig geſund ſein wird.\pend
           
\pstart
           Mit vorzüglichſter Hochſchätzung empfiehlt ſich Ihnen {\\[\baselineskip]}ganz ergebſt { }{\\[\baselineskip]}\spacefill\mbox{Meyer}\pend
           \leftskip=0em{}\selectlanguage{ngerman}\endnumbering\briefempfaengerindex{Schnitzler, Arthur@\textsc{Schnitzler, Arthur}!zzzGoldmann, Paul@\emph{von Paul Goldmann}!1901-02-141@{14. 2. {[}1901{]}}|)be}\mylabel{L03058h}  \normalsize

\doendnotes{C}
\bigskip
\vfill

\clearpage

\footnotesize

\lohead{\textsc{register}}

% Definiere theindex-Environment komplett neu ohne reledmac
\makeatletter
\renewenvironment{theindex}{%
  \section*{\indexname}%
  \setlength{\parindent}{0pt}%
  \setlength{\parskip}{0pt plus 0.3pt}%
  \let\item\@idxitem
}{%
  \clearpage
}
\makeatother

\IfFileExists{\jobname-pw.ind}{\input{\jobname-pw.ind}}{}

\end{document}

      