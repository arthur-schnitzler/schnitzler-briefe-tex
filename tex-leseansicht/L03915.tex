%% latex-leseansicht-vorspann.tex
%% Vorspann für die Leseansicht.
%% Lädt die gemeinsame Datei latex-vorspann.tex mit nicht gesetztem Schalter.

\newif\ifkorrekturansicht
\korrekturansichtfalse

\input{../tex-inputs/latex-vorspann}


\section[Arthur Schnitzler an Theodor Herzl, 24. 4. 1896]{L03915 Arthur Schnitzler an Theodor Herzl, 24. 4. 1896}
\nopagebreak\mylabel{L03915v}
\rehead{ }\normalsize\beginnumbering\briefempfaengerindex{Herzl, Theodor@\textsc{Herzl, Theodor}!zzzSchnitzler, Arthur@\emph{von Arthur Schnitzler}!1896-04-241@{24. 4. 1896}|(be}
\toendnotes[C]{\smallbreak\pagebreak[2]}
\correspDesc{Versand  durch Arthur Schnitzler am 24. 4. 1896 in Wien
\newline{}Erhalt  durch Theodor Herzl in Wien}\toendnotes[C]{\smallbreak}
\Standort{Jerusalem, Central Zionist Archives, H1:1924-20.}
\physDesc{,  Blätter,  Seiten
\newline{}Handschrift: , deutsche Kurrent}\toendnotes[C]{\smallbreak}
\pstart
           {\pb}Wien\oindex{Wien@\textbf{Wien}, \emph{Verwaltungsgebiet}|pw}{ }24. 4. 9\textcolor{gray}{6}\textcolor{gray}{}.\pend
           
\pstart{}Lieber Freund!\pend\vspace{0.5em}
\pstart
           nehmen Sie das beifolgende Buch\pwindex{Schnitzler, Arthur 15.\,5.\,1862 Wien – 21.\,10.\,1931 ebd.@\textsc{Schnitzler, Arthur} (15.\,5.\,1862 Wien – 21.\,10.\,1931 ebd.), \emph{Schriftsteller, Mediziner}!Mourir. Roman@\strich\emph{Mourir. Roman}|pwv} freundlichſt entgegen. Ich kann diesmal nicht ohne eine Bitte. Es wäre
               mir begreiflicherweise ſehr erwünſcht, wenn die N.
                  Fr. Pr.\pwindex{Neue Freie Presse@\emph{Neue Freie Presse}|pw} von dem Erſcheinen dieſer Ueberſetzung\pwindex{Schnitzler, Arthur 15.\,5.\,1862 Wien – 21.\,10.\,1931 ebd.@\textsc{Schnitzler, Arthur} (15.\,5.\,1862 Wien – 21.\,10.\,1931 ebd.), \emph{Schriftsteller, Mediziner}!Mourir. Roman@\strich\emph{Mourir. Roman}|pwv}{ }\label{K_L03915-1v}\edtext{Notiz\pwindex{Frankreich beginnt von den jüngeren deutschen Schriftstellern Notiz zu nehmen]@\emph{[Frankreich beginnt von den jüngeren deutschen Schriftstellern Notiz zu nehmen]}|pwv}}{\lemma{\textnormal{\emph{Notiz}}}\Cendnote{\textnormal{Schnitzlers Wunsch
                     wurde am 5. 5. 1896 Folge geleistet: »– Frankreich\oindex{Frankreich@\textbf{Frankreich}|pw}
                     beginnt von den jüngeren deutſchen Schriftſtellern Notiz zu nehmen. Soeben iſt
                     im Pariſer Verlage Perrin {\kaufmannsund} Cie.\orgindex{Éditions Perrin@Éditions Perrin|pw} eine Ueberſetzung
                        der intereſſanten Erzählung ›\so{Sterben}\pwindex{Schnitzler, Arthur 15.\,5.\,1862 Wien – 21.\,10.\,1931 ebd.@\textsc{Schnitzler, Arthur} (15.\,5.\,1862 Wien – 21.\,10.\,1931 ebd.), \emph{Schriftsteller, Mediziner}!Mourir. Roman@\strich\emph{Mourir. Roman}|pw}‹ von Arthur \so{Schnitzler} erſchienen. Der Ueberſetzer Gaſpard \so{Vallette}\pwindex{Vallette, Gaspard 13.\,5.\,1865 Jussy – 6.\,8.\,1911 La Tène@\textsc{Vallette, Gaspard} (13.\,5.\,1865 Jussy – 6.\,8.\,1911 La Tène), \emph{Journalist, Übersetzer}|pw} hat{ }ſeine Sache{ }ſehr gut
                     gemacht.« (In: \emph{Neue Freie Presse}\pwindex{Neue Freie Presse@\emph{Neue Freie Presse}|pwk}, Nr. 11.386, 5. 5. 1896, Morgenblatt, S. 7.)}}}\label{K_L03915-1} nähme und brächte. Einfach die Thatſache (die ja nicht
               wegzuläugnen iſt) im Kunſt- und Theatertheil, {\pb}und, we{\geminationn} mir noch ein Wunſch geſtattet iſt, nicht mit dem
               verdächtigen Druck, in welchem E.\textcolor{red}{\textsuperscript{\textbf{KEY}}} die beiſpielloſen
               Erfolge des Fräulein X. in Leitomiſchl\oindex{Litomyšl@\textbf{Litomyšl}|pw} verewigt
               zu werden pflegen.\pend
           
\pstart
           Vielleicht werfen Sie einen Blick in das Buch\pwindex{Schnitzler, Arthur 15.\,5.\,1862 Wien – 21.\,10.\,1931 ebd.@\textsc{Schnitzler, Arthur} (15.\,5.\,1862 Wien – 21.\,10.\,1931 ebd.), \emph{Schriftsteller, Mediziner}!Mourir. Roman@\strich\emph{Mourir. Roman}|pwv} und finden, daſs die Uebertragung\pwindex{Schnitzler, Arthur 15.\,5.\,1862 Wien – 21.\,10.\,1931 ebd.@\textsc{Schnitzler, Arthur} (15.\,5.\,1862 Wien – 21.\,10.\,1931 ebd.), \emph{Schriftsteller, Mediziner}!Mourir. Roman@\strich\emph{Mourir. Roman}|pwv} gelungen iſt. Ob Sie das finden werden und ob
               Sie es in der Notiz\pwindex{Frankreich beginnt von den jüngeren deutschen Schriftstellern Notiz zu nehmen]@\emph{[Frankreich beginnt von den jüngeren deutschen Schriftstellern Notiz zu nehmen]}|pwv} auch bemerken
               wollen, iſt Sache Ihrer Geſchmacks und Ihrer Liebenswürdigkeit.\pend
           \pstart {\pb}Ich grüße Sie herzlich und bin Ihr dankbar ergebener
                  \spacefill\mbox{Arthur Schnitzler}\pend{}\selectlanguage{ngerman}\endnumbering\briefempfaengerindex{Herzl, Theodor@\textsc{Herzl, Theodor}!zzzSchnitzler, Arthur@\emph{von Arthur Schnitzler}!1896-04-241@{24. 4. 1896}|)be}\mylabel{L03915h}
\begin{anhang}
\end{anhang}\newcommand{\dateiname}{L03915}\newcommand{\titel}{Arthur Schnitzler an Theodor Herzl, 24. 4. 1896}\newcommand{\editorInnen}{Herausgegeben von Jahnke, SelmaMüller, Martin Anton}%% latex-leseansicht-abspann.tex
%% Abspann für die Leseansicht.
%% Der Schalter \ifkorrekturansicht ist bereits durch den Vorspann gesetzt.

%% latex-abspann.tex
%% Gemeinsamer Abspann für Korrekturansicht und Leseansicht.
%% Setzt den Schalter \ifkorrekturansicht voraus (gesetzt in den
%% einbindenden Dateien latex-korrekturansicht-abspann.tex bzw.
%% latex-leseansicht-abspann.tex).
%% ---------------------------------------------------------------

\normalsize

% Das esempio-Environment wird nur in der Leseansicht benötigt
\ifkorrekturansicht\else
\newenvironment{esempio}[3]%
{
    \vspace{1.5ex}
    \rlap{\underline{#1}}
    \par
    \setlength{\parindent}{0cm}
    \nopagebreak
    \leftskip=#2cm
    \rightskip=#3cm
}
{
    \par
}
\fi

\doendnotes{C}
\bigskip
\vfill

\clearpage

\footnotesize

\ifkorrekturansicht
  \lohead{\textsc{register}}
\fi

% theindex-Environment neu definieren ohne reledmac
\makeatletter
\renewenvironment{theindex}{%
  \ifkorrekturansicht
    \section*{\indexname}%
  \else
    \subsubsection*{Index der erwähnten Entitäten}%
  \fi
  \setlength{\parindent}{0pt}%
  \setlength{\parskip}{0pt plus 0.3pt}%
  \let\item\@idxitem
}{%
  \ifkorrekturansicht\clearpage\fi
}
\makeatother

\IfFileExists{\jobname-pw.ind}{\input{\jobname-pw.ind}}{}

% Quellenangabe nur in der Leseansicht
\ifkorrekturansicht\else
% Fallback-Definitionen, falls die .tex-Datei \titel etc. nicht gesetzt hat
\providecommand{\titel}{}
\providecommand{\editorInnen}{}
\providecommand{\dateiname}{\jobname}

\vspace{3cm}

\vfill

\footnotesize
\textsc{Quelle}: \titel. Herausgegeben von {\editorInnen}. In: \emph{Arthur Schnitzler: Briefwechsel mit Autorinnen und Autoren}.
 Digitale Edition, https://schnitzler-briefe.acdh.oeaw.ac.at/{\dateiname}.html (Stand \today)
\fi

\end{document}


