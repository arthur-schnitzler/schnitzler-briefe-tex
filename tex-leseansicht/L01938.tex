%% latex-leseansicht-vorspann.tex
%% Vorspann für die Leseansicht.
%% Lädt die gemeinsame Datei latex-vorspann.tex mit nicht gesetztem Schalter.

\newif\ifkorrekturansicht
\korrekturansichtfalse

\input{../tex-inputs/latex-vorspann}


\section[Gerty und Hugo von Hofmannsthal an Olga Schnitzler, 25. 6. 1910]{L01938 Gerty und Hugo von Hofmannsthal an Olga Schnitzler, 25. 6. 1910}
\nopagebreak\mylabel{L01938v}
\rehead{ }\normalsize\beginnumbering\briefempfaengerindex{Schnitzler, Olga@\textsc{Schnitzler, Olga}!zzzHofmannsthal, Hugo von@\emph{von Hugo von Hofmannsthal}!1910-06-253@{25. 6. 1910}|(be}\briefempfaengerindex{Schnitzler, Olga@\textsc{Schnitzler, Olga}!zzzHofmannsthal, Gertrude von@\emph{von Gertrude von Hofmannsthal}!1910-06-253@{25. 6. 1910}|(be}
\toendnotes[C]{\smallbreak\pagebreak[2]}
\correspDesc{Versand  durch Gerty von Hofmannsthal, Hugo von Hofmannsthal am 25. 6. 1910 in Lido
\newline{}Erhalt  durch Olga Schnitzler im Zeitraum [26. 6. 1910
                  – 30. 6. 1910?] in Wien}\toendnotes[C]{\smallbreak}
\Standort{CUL, Schnitzler, B 43.}
\physDesc{Bildpostkarte, 269 Zeichen
\newline{}Handschrift Gertrude von Hofmannsthal: Bleistift, lateinische Kurrent
\newline{}Handschrift Hugo von Hofmannsthal: Bleistift, lateinische Kurrent
\newline{}Versand: Stempel: »\nobreak{}\oindex{Santa Maria Elisabetta@\textbf{Santa Maria Elisabetta}, \emph{Bezirk}|pwk}S. Elisabetta d/Lido
                                       (Venezia), 25. 6. 1910\nobreak{}«.  
\newline{}Schnitzler: mit Bleistift beschriftet: »\textsc{Hugo}« 
\newline{}Ordnung: 1) mit Bleistift von unbekannter Hand nummeriert: »\strikeout{377}«  2) mit Bleistift von unbekannter Hand nummeriert: »319«}
\buchAbdrucke{\weitereDrucke{Hugo von Hofmannsthal, Arthur Schnitzler: \emph{Briefwechsel}. Herausgegeben von Therese Nickl und Heinrich Schnitzler. Frankfurt am Main: \emph{S. Fischer} 1964, S. 381.} }\toendnotes[C]{\smallbreak}\pstart{}{\pb}Frau Olga Schnitzler\pend{}\pstart{}Wien\oindex{Wien@\textbf{Wien}, \emph{Verwaltungsgebiet}|pw}\pend{}\pstart{}XVIII Spöttlgasse 7\oindex{Wien@\textbf{Wien}!XVIII., Währing@\textbf{XVIII., Währing}!Edmund-Weiß-Gasse 7@\textbf{Edmund-Weiß-Gasse 7}, \emph{Wohngebäude}|pw}\pend{}{\bigskip}
\pstart
           \noindent{}\centering{}{\pb}\textcolor{gray}{\textbf{Venezia – Lido\oindex{Lido@\textbf{Lido}|pw}}}\pend
           \vspace{1em}
\pstart
           \noindent{}{\pb}Herzliche Grüsse vom schönen Strand wo
               man riesig faulenzen kann sich sonnen und baden. Wir sind erste Tage
                  Juli wieder zuhaus und hoffen sehr Euch in Wien\oindex{Wien@\textbf{Wien}, \emph{Verwaltungsgebiet}|pw} anzutreffen.\pend
           \pstart Herzlichst \spacefill\mbox{Gerty}\pend{}\selectlanguage{ngerman}\vspace{1em}
\pstart
           \noindent{}{[}hs. Hofmannsthal:{]} Freuen uns Euch bald im neuen Haus\oindex{Wien@\textbf{Wien}!XVIII., Währing@\textbf{XVIII., Währing}!Sternwartestraße 71@\textbf{Sternwartestraße 71}, \emph{Wohngebäude}|pwv} zu finden.\pend
           \pstart \spacefill\mbox{Hugo}\pend{}\selectlanguage{ngerman}\endnumbering\briefempfaengerindex{Schnitzler, Olga@\textsc{Schnitzler, Olga}!zzzHofmannsthal, Hugo von@\emph{von Hugo von Hofmannsthal}!1910-06-253@{25. 6. 1910}|)be}\briefempfaengerindex{Schnitzler, Olga@\textsc{Schnitzler, Olga}!zzzHofmannsthal, Gertrude von@\emph{von Gertrude von Hofmannsthal}!1910-06-253@{25. 6. 1910}|)be}\mylabel{L01938h}  \newcommand{\dateiname}{L01938}\newcommand{\titel}{Gerty und Hugo von Hofmannsthal an Olga Schnitzler, 25. 6. 1910}\newcommand{\editorInnen}{Martin Anton Müller und Gerd-Hermann Susen}%% latex-leseansicht-abspann.tex
%% Abspann für die Leseansicht.
%% Der Schalter \ifkorrekturansicht ist bereits durch den Vorspann gesetzt.

%% latex-abspann.tex
%% Gemeinsamer Abspann für Korrekturansicht und Leseansicht.
%% Setzt den Schalter \ifkorrekturansicht voraus (gesetzt in den
%% einbindenden Dateien latex-korrekturansicht-abspann.tex bzw.
%% latex-leseansicht-abspann.tex).
%% ---------------------------------------------------------------

\normalsize

% Das esempio-Environment wird nur in der Leseansicht benötigt
\ifkorrekturansicht\else
\newenvironment{esempio}[3]%
{
    \vspace{1.5ex}
    \rlap{\underline{#1}}
    \par
    \setlength{\parindent}{0cm}
    \nopagebreak
    \leftskip=#2cm
    \rightskip=#3cm
}
{
    \par
}
\fi

\doendnotes{C}
\bigskip
\vfill

\clearpage

\footnotesize

\ifkorrekturansicht
  \lohead{\textsc{register}}
\fi

% theindex-Environment neu definieren ohne reledmac
\makeatletter
\renewenvironment{theindex}{%
  \ifkorrekturansicht
    \section*{\indexname}%
  \else
    \subsubsection*{Index der erwähnten Entitäten}%
  \fi
  \setlength{\parindent}{0pt}%
  \setlength{\parskip}{0pt plus 0.3pt}%
  \let\item\@idxitem
}{%
  \ifkorrekturansicht\clearpage\fi
}
\makeatother

\IfFileExists{\jobname-pw.ind}{\input{\jobname-pw.ind}}{}

% Quellenangabe nur in der Leseansicht
\ifkorrekturansicht\else
% Fallback-Definitionen, falls die .tex-Datei \titel etc. nicht gesetzt hat
\providecommand{\titel}{}
\providecommand{\editorInnen}{}
\providecommand{\dateiname}{\jobname}

\vspace{3cm}

\vfill

\footnotesize
\textsc{Quelle}: \titel. Herausgegeben von {\editorInnen}. In: \emph{Arthur Schnitzler: Briefwechsel mit Autorinnen und Autoren}.
 Digitale Edition, https://schnitzler-briefe.acdh.oeaw.ac.at/{\dateiname}.html (Stand \today)
\fi

\end{document}


