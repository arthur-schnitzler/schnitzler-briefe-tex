%% latex-leseansicht-vorspann.tex
%% Vorspann für die Leseansicht.
%% Lädt die gemeinsame Datei latex-vorspann.tex mit nicht gesetztem Schalter.

\newif\ifkorrekturansicht
\korrekturansichtfalse

\input{../tex-inputs/latex-vorspann}


\section[Hugo August von Hofmannsthal an Arthur Schnitzler, 2. 6. [1895]]{L00448 Hugo August von Hofmannsthal an Arthur Schnitzler, 2. 6. [1895]}
\nopagebreak\mylabel{L00448v}
\rehead{ }\normalsize\beginnumbering\briefempfaengerindex{Schnitzler, Arthur@\textsc{Schnitzler, Arthur}!zzzHofmannsthal, Hugo August von@\emph{von Hugo August von Hofmannsthal}!1895-06-021@{2. 6. 1895}|(be}
\toendnotes[C]{\smallbreak\pagebreak[2]}
\correspDesc{Versand  durch Hugo August von Hofmannsthal am 2. 6. 1895 in Wien
\newline{}Erhalt  durch Arthur Schnitzler im Zeitraum [2. 6. 1895
                  – 6. 6. 1895?] in Wien}\toendnotes[C]{\smallbreak}
\Standort{DLA, A:Schnitzler, HS.NZ85.1.3483.}
\physDesc{Briefkarte, 315 Zeichen
\newline{}Handschrift: schwarze Tinte, deutsche Kurrent
\newline{}Schnitzler: mit Bleistift die Jahreszahl »95.« ergänzt }\toendnotes[C]{\smallbreak}
\pstart
           \label{T_L00448-1v}\edtext{{\pb}\textcolor{gray}{\textbf{Oesterreichische Central-Boden-Credit-Bank
                        Wien\orgindex{Allgemeine Bodencreditanstalt@Allgemeine Bodencreditanstalt|pw}.}}}{\lemma{\textnormal{\emph{Oesterreichische … Wien.}}}\Cendnote{\textnormal{quer am linken Rand}}}\label{T_L00448-1}\pend
           
\pstart
           \raggedleft{}\textsc{Pfingstsonntag}\pend
           
\pstart{}Lieber Freund!\pend\vspace{0.5em}
\pstart
           \textsc{Hugo}\pwindex{Hofmannsthal, Hugo von 1.\,2.\,1874 Wien – 15.\,7.\,1929 Rodaun@\textsc{Hofmannsthal, Hugo von} (1.\,2.\,1874 Wien – 15.\,7.\,1929 Rodaun), \emph{Schriftsteller}|pw} iſt geſtern, leider wieder recht unwohl, hier eingetroffen. Der \label{T_L00448-2v}\edtext{Hustenkatarrh}{\lemma{\textnormal{\emph{Hustenkatarrh}}}\Cendnote{\textnormal{Doppelung des »r« durch Geminationsstrich}}}\label{T_L00448-2} iſt durch
               Erhitzung u Staub natürlich ärger und{ }ſind aus dieſer Situation gegenüber ganz
               rathlos. Er{ }ſelbſt deprimirt u übellaunig \textsc{au poſsible}!\pend
           
\pstart
           Mit den besten Grüßen Ihr ergebenster{\\[\baselineskip]}\spacefill\mbox{D\textsuperscript{r}Hofmannsthal}\pend
           \leftskip=0em{}
\pstart
           \textsc{Pfingstsonntag}\pend
           \selectlanguage{ngerman}\endnumbering\briefempfaengerindex{Schnitzler, Arthur@\textsc{Schnitzler, Arthur}!zzzHofmannsthal, Hugo August von@\emph{von Hugo August von Hofmannsthal}!1895-06-021@{2. 6. 1895}|)be}\mylabel{L00448h}  \newcommand{\dateiname}{L00448}\newcommand{\titel}{Hugo August von Hofmannsthal an Arthur Schnitzler, 2. 6. [1895]}\newcommand{\editorInnen}{Martin Anton Müller und Gerd-Hermann Susen}%% latex-leseansicht-abspann.tex
%% Abspann für die Leseansicht.
%% Der Schalter \ifkorrekturansicht ist bereits durch den Vorspann gesetzt.

%% latex-abspann.tex
%% Gemeinsamer Abspann für Korrekturansicht und Leseansicht.
%% Setzt den Schalter \ifkorrekturansicht voraus (gesetzt in den
%% einbindenden Dateien latex-korrekturansicht-abspann.tex bzw.
%% latex-leseansicht-abspann.tex).
%% ---------------------------------------------------------------

\normalsize

% Das esempio-Environment wird nur in der Leseansicht benötigt
\ifkorrekturansicht\else
\newenvironment{esempio}[3]%
{
    \vspace{1.5ex}
    \rlap{\underline{#1}}
    \par
    \setlength{\parindent}{0cm}
    \nopagebreak
    \leftskip=#2cm
    \rightskip=#3cm
}
{
    \par
}
\fi

\doendnotes{C}
\bigskip
\vfill

\clearpage

\footnotesize

\ifkorrekturansicht
  \lohead{\textsc{register}}
\fi

% theindex-Environment neu definieren ohne reledmac
\makeatletter
\renewenvironment{theindex}{%
  \ifkorrekturansicht
    \section*{\indexname}%
  \else
    \subsubsection*{Index der erwähnten Entitäten}%
  \fi
  \setlength{\parindent}{0pt}%
  \setlength{\parskip}{0pt plus 0.3pt}%
  \let\item\@idxitem
}{%
  \ifkorrekturansicht\clearpage\fi
}
\makeatother

\IfFileExists{\jobname-pw.ind}{\input{\jobname-pw.ind}}{}

% Quellenangabe nur in der Leseansicht
\ifkorrekturansicht\else
% Fallback-Definitionen, falls die .tex-Datei \titel etc. nicht gesetzt hat
\providecommand{\titel}{}
\providecommand{\editorInnen}{}
\providecommand{\dateiname}{\jobname}

\vspace{3cm}

\vfill

\footnotesize
\textsc{Quelle}: \titel. Herausgegeben von {\editorInnen}. In: \emph{Arthur Schnitzler: Briefwechsel mit Autorinnen und Autoren}.
 Digitale Edition, https://schnitzler-briefe.acdh.oeaw.ac.at/{\dateiname}.html (Stand \today)
\fi

\end{document}


