%% latex-leseansicht-vorspann.tex
%% Vorspann für die Leseansicht.
%% Lädt die gemeinsame Datei latex-vorspann.tex mit nicht gesetztem Schalter.

\newif\ifkorrekturansicht
\korrekturansichtfalse

\input{../tex-inputs/latex-vorspann}


\section[Oscar Blumenthal an Arthur Schnitzler, 1. 8. 1892]{L00110 Oscar Blumenthal an Arthur Schnitzler, 1. 8. 1892}
\nopagebreak\mylabel{L00110v}
\rehead{ }\normalsize\beginnumbering\briefempfaengerindex{Schnitzler, Arthur@\textsc{Schnitzler, Arthur}!zzzBlumenthal, Oskar@\emph{von Oskar Blumenthal}!1892-08-011@{1. 8. 1892}|(be}
\toendnotes[C]{\smallbreak\pagebreak[2]}
\correspDesc{Versand  durch Oscar Blumenthal am 1. 8. 1892 in Berlin
\newline{}Erhalt  durch Arthur Schnitzler im Zeitraum [2. 8. 1892
                  – 6. 8. 1892?] in Wien}\toendnotes[C]{\smallbreak}
\Standort{CUL, Schnitzler, B 15.}
\physDesc{Brief, 1 Blatt, 1 Seite, 536 Zeichen
\newline{}Handschrift Schreibkraft: schwarze Tinte, deutsche Kurrent
\newline{}Handschrift Oskar Blumenthal: schwarze Tinte, deutsche Kurrent (\noindent{}Unterschrift)
\newline{}Schnitzler: mit rotem Buntstift eine Unterstreichung und nummeriert:
                                    »3« }
\pstart
           \centering{}{\pb}\textcolor{gray}{\textbf{LESSING-THEATER\orgindex{Lessing-Theater@Lessing-Theater|pw}}}\pend
           
\pstart
           \centering{}\textcolor{gray}{\textbf{Director:}}{\\}\textcolor{gray}{\textbf{Dr. Oscar Blumenthal.}}\pend
           
\pstart
           \raggedleft{}\textcolor{gray}{\textbf{Berlin N.W.\oindex{Berlin@\textbf{Berlin}, \emph{Hauptstadt}|pw}, den}}{ }1. August \textcolor{gray}{\textbf{189}}2.{\\}\textcolor{gray}{\textbf{Friedrich-Carl-Ufer\oindex{Kapelle-Ufer@\textbf{Kapelle-Ufer}, \emph{Straße}|pw}}}.\pend
           
\pstart\center{}Werther Herr Doktor!\pend\vspace{0.5em}
\pstart
           Ueber den Aufführungstermin von »Das Märchen\pwindex{Schnitzler, Arthur 15.\,5.\,1862 Wien – 21.\,10.\,1931 ebd.@\textsc{Schnitzler, Arthur} (15.\,5.\,1862 Wien – 21.\,10.\,1931 ebd.), \emph{Schriftsteller, Mediziner}!Märchen. Schauspiel in drei Aufzügen@\strich\emph{Das Märchen. Schauspiel in drei Aufzügen}|pw}«
               kann ich Ihnen im Augenblick eine beſtimmte Zuſage nicht machen, da{ }ſich die
               Dispoſitionen für die neue Saiſon noch nicht klar genug überblicken laſſen. Doch wird
               jedenfalls erſt im zweiten Quartal die Aufführung{ }ſtattfinden können, da
               ich für die Monate Oktober, November, Dezember
               theils durch die abgeſchloſſenen Verträge, theils durch das Gaſtſpiel der \textsc{Duse}\pwindex{Duse, Eleonora 3.\,10.\,1858 Vigevano – 21.\,4.\,1924 Pittsburgh@\textsc{Duse, Eleonora} (3.\,10.\,1858 Vigevano – 21.\,4.\,1924 Pittsburgh), \emph{Schauspielerin}|pw}{ }ſehr eingeengt bin.\pend
           
\pstart
           Mit freundlichen Grüßen\hspace*{2.5em}Ihr{\\[\baselineskip]}\spacefill\mbox{{[}hs. Blumenthal:{]} Dr. Osc. Blumenthal}\pend
           \leftskip=0em{}
\pstart
           \noindent{}{[}hs.:{]} Herrn{\\}\textsc{Dr Arthur Schnitzler}{\\}\textsc{Wien I.\oindex{I., Innere Stadt@\textbf{I., Innere Stadt}, \emph{Verwaltungsgebiet}|pw}}\pend
           
\pstart
           cop.\pend
           \selectlanguage{ngerman}\endnumbering\briefempfaengerindex{Schnitzler, Arthur@\textsc{Schnitzler, Arthur}!zzzBlumenthal, Oskar@\emph{von Oskar Blumenthal}!1892-08-011@{1. 8. 1892}|)be}\mylabel{L00110h}  \newcommand{\dateiname}{L00110}\newcommand{\titel}{Oscar Blumenthal an Arthur Schnitzler, 1. 8. 1892}\newcommand{\editorInnen}{Martin Anton Müller und Gerd-Hermann Susen}%% latex-leseansicht-abspann.tex
%% Abspann für die Leseansicht.
%% Der Schalter \ifkorrekturansicht ist bereits durch den Vorspann gesetzt.

%% latex-abspann.tex
%% Gemeinsamer Abspann für Korrekturansicht und Leseansicht.
%% Setzt den Schalter \ifkorrekturansicht voraus (gesetzt in den
%% einbindenden Dateien latex-korrekturansicht-abspann.tex bzw.
%% latex-leseansicht-abspann.tex).
%% ---------------------------------------------------------------

\normalsize

% Das esempio-Environment wird nur in der Leseansicht benötigt
\ifkorrekturansicht\else
\newenvironment{esempio}[3]%
{
    \vspace{1.5ex}
    \rlap{\underline{#1}}
    \par
    \setlength{\parindent}{0cm}
    \nopagebreak
    \leftskip=#2cm
    \rightskip=#3cm
}
{
    \par
}
\fi

\doendnotes{C}
\bigskip
\vfill

\clearpage

\footnotesize

\ifkorrekturansicht
  \lohead{\textsc{register}}
\fi

% theindex-Environment neu definieren ohne reledmac
\makeatletter
\renewenvironment{theindex}{%
  \ifkorrekturansicht
    \section*{\indexname}%
  \else
    \subsubsection*{Index der erwähnten Entitäten}%
  \fi
  \setlength{\parindent}{0pt}%
  \setlength{\parskip}{0pt plus 0.3pt}%
  \let\item\@idxitem
}{%
  \ifkorrekturansicht\clearpage\fi
}
\makeatother

\IfFileExists{\jobname-pw.ind}{\input{\jobname-pw.ind}}{}

% Quellenangabe nur in der Leseansicht
\ifkorrekturansicht\else
% Fallback-Definitionen, falls die .tex-Datei \titel etc. nicht gesetzt hat
\providecommand{\titel}{}
\providecommand{\editorInnen}{}
\providecommand{\dateiname}{\jobname}

\vspace{3cm}

\vfill

\footnotesize
\textsc{Quelle}: \titel. Herausgegeben von {\editorInnen}. In: \emph{Arthur Schnitzler: Briefwechsel mit Autorinnen und Autoren}.
 Digitale Edition, https://schnitzler-briefe.acdh.oeaw.ac.at/{\dateiname}.html (Stand \today)
\fi

\end{document}


