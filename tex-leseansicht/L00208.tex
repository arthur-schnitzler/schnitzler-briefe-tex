%% latex-leseansicht-vorspann.tex
%% Vorspann für die Leseansicht.
%% Lädt die gemeinsame Datei latex-vorspann.tex mit nicht gesetztem Schalter.

\newif\ifkorrekturansicht
\korrekturansichtfalse

\input{../tex-inputs/latex-vorspann}


               \section[Arthur Schnitzler an Richard Beer-Hofmann, 3. 5. 1893]{ Arthur Schnitzler an Richard Beer-Hofmann, 3. 5. 1893}\nopagebreak\mylabel{v}\rehead{ }\begin{ledgroupsized}[t]{13cm}\normalsize\beginnumbering\briefempfaengerindex{Beer-Hofmann, Richard@\textsc{Beer-Hofmann, Richard}!zzzSchnitzler, Arthur@\emph{von Arthur Schnitzler}!1893-05-031@{3. 5. 1893}|(be} \toendnotes[C]{\smallbreak\pagebreak[2]} \Standort{YCGL, MSS 31.}
\physDesc{gedruckte Todesanzeige, Umschlag mit Trauerrand
\newline{}Druck: »M. ENGEL{ }{\kaufmannsund}{ }SÖHNE\orgindex{M. Engel und Soehne@M. Engel und Söhne|pw}{ }WIEN, 1.,
                                       LICHTENFELSGASSE 9\oindex{Lichtenfelsgasse@\textbf{Lichtenfelsgasse}|pw}«
\newline{}Handschrift: schwarze Tinte, deutsche Kurrent (\noindent{}Umschlag)\newline{}Versand: Stempel: »\nobreak{}Wien 1/1, 3. 5. 93, 3–4 N\nobreak{}«.  }\pstart{}{\pb}Herrn \textsc{Dr. Rich.
                     Beer-Hofmann}\pend{}\pstart{}\textsc{Wien\oindex{Wien@\textbf{Wien}|pw}}\pend{}\pstart{}\textsc{I Wollzeile 15\oindex{Wollzeile@\textbf{Wollzeile}|pw}}.\pend{}{\bigskip}\pstart
           \noindent{}{\pb}Tieferschüttert geben die Unterzeichneten hiemit im
               eigenen und im Namen der Familie Nachricht von dem Hinscheiden ihres innigstgeliebten
               Gatten, resp. Vaters, Bruders und Schwiegervaters, des Herrn\pend
           \pstart
           \centering{}Dr. Johann Schnitzler\pwindex{Schnitzler, Johann 10.04.1835 – 02.05.1893@\textsc{Schnitzler, Johann} (10.04.1835 – 02.05.1893), \emph{Laryngologe}|pw}\pend
           \pstart
           \noindent{}\centering{}k. k. Regierungsrath, k. k. a. o. Universitäts-Professor, Direktor der
                  allgemeinen Poliklinik\orgindex{Allgemeine Poliklinik@Allgemeine Poliklinik|pw}, Commandeur des dän.\oindex{Daenemark@\textbf{Dänemark}|pw}{ }Danebrog-Orden\orgindex{Dannebrogorden@Dannebrogorden|pw}s etc. etc.\pend
           \pstart
           \noindent{}welcher nach kurzem Leiden am 2. Mai 1893, Nachmittags ½ 2
                  Uhr, im 59. Lebensjahre verschieden ist.\pend
           \pstart
           Die irdische Hülle des theuren Verblichenen wird Donnerstag, den 4. Mai,
                  ½ 10 Uhr Vormittags vom Trauerhause I.,
                  Burgring 1\oindex{Burgring@\textbf{Burgring}|pw}, auf den Central-Friedhof (israel.
                  Abtheilung)\oindex{Wiener Zentralfriedhof@\textbf{Wiener Zentralfriedhof}|pw} überführt und dort zur ewigen Ruhe bestattet.\pend
           \leftskip=3em{}\pstart
           \noindent{}\so{Wien}\oindex{Wien@\textbf{Wien}|pw}, 3. Mai 1893.\pend
           \leftskip=0em{}\pstart
           \noindent{}\centering{}Louise Schnitzler\pwindex{Schnitzler, Louise 08.07.1840 – 09.09.1911@\textsc{Schnitzler, Louise} (08.07.1840 – 09.09.1911)|pw}{\\}geb. Markbreiter\pwindex{Schnitzler, Louise 08.07.1840 – 09.09.1911@\textsc{Schnitzler, Louise} (08.07.1840 – 09.09.1911)|pw}{\\}als Gattin.\pend
           \pstart
           \noindent{}Dr. Arthur Schnitzler\hfill Johanna Willheim\pwindex{Willheim, Johanna 1840/1841 – 1925@\textsc{Willheim, Johanna} (1840/1841 – 1925)|pw}\pend
           \pstart
           Dr. Julius Schnitzler\pwindex{Schnitzler, Julius 13.07.1865 – 29.06.1939@\textsc{Schnitzler, Julius} (13.07.1865 – 29.06.1939), \emph{Chirurg}|pw}\hfill geb. Schnitzler\pend
           \pstart
           Gisela Hajek\pwindex{Hajek, Gisela 20.12.1867 – 03.02.1953@\textsc{Hajek, Gisela} (20.12.1867 – 03.02.1953)|pw}\hfill als Schwester.\pend
           \pstart
           als Kinder.\hfill Dr. Marcus Hajek\pwindex{Hajek, Markus 25.11.1861 – 04.04.1941@\textsc{Hajek, Markus} (25.11.1861 – 04.04.1941), \emph{Mediziner, Laryngologe}|pw}\pend
           \pstart
           \raggedleft{}als Schwiegersohn\pend
           \endnumbering\briefempfaengerindex{Beer-Hofmann, Richard@\textsc{Beer-Hofmann, Richard}!zzzSchnitzler, Arthur@\emph{von Arthur Schnitzler}!1893-05-031@{3. 5. 1893}|)be}\mylabel{h}\end{ledgroupsized}  \newcommand{\dateiname}{L00208}\newcommand{\titel}{Arthur Schnitzler an Richard Beer-Hofmann, 3. 5. 1893}\newcommand{\editorInnen}{Martin Anton Müller und Gerd-Hermann Susen}%% latex-leseansicht-abspann.tex
%% Abspann für die Leseansicht.
%% Der Schalter \ifkorrekturansicht ist bereits durch den Vorspann gesetzt.

%% latex-abspann.tex
%% Gemeinsamer Abspann für Korrekturansicht und Leseansicht.
%% Setzt den Schalter \ifkorrekturansicht voraus (gesetzt in den
%% einbindenden Dateien latex-korrekturansicht-abspann.tex bzw.
%% latex-leseansicht-abspann.tex).
%% ---------------------------------------------------------------

\normalsize

% Das esempio-Environment wird nur in der Leseansicht benötigt
\ifkorrekturansicht\else
\newenvironment{esempio}[3]%
{
    \vspace{1.5ex}
    \rlap{\underline{#1}}
    \par
    \setlength{\parindent}{0cm}
    \nopagebreak
    \leftskip=#2cm
    \rightskip=#3cm
}
{
    \par
}
\fi

\doendnotes{C}
\bigskip
\vfill

\clearpage

\footnotesize

\ifkorrekturansicht
  \lohead{\textsc{register}}
\fi

% theindex-Environment neu definieren ohne reledmac
\makeatletter
\renewenvironment{theindex}{%
  \ifkorrekturansicht
    \section*{\indexname}%
  \else
    \subsubsection*{Index der erwähnten Entitäten}%
  \fi
  \setlength{\parindent}{0pt}%
  \setlength{\parskip}{0pt plus 0.3pt}%
  \let\item\@idxitem
}{%
  \ifkorrekturansicht\clearpage\fi
}
\makeatother

\IfFileExists{\jobname-pw.ind}{\input{\jobname-pw.ind}}{}

% Quellenangabe nur in der Leseansicht
\ifkorrekturansicht\else
% Fallback-Definitionen, falls die .tex-Datei \titel etc. nicht gesetzt hat
\providecommand{\titel}{}
\providecommand{\editorInnen}{}
\providecommand{\dateiname}{\jobname}

\vspace{3cm}

\vfill

\footnotesize
\textsc{Quelle}: \titel. Herausgegeben von {\editorInnen}. In: \emph{Arthur Schnitzler: Briefwechsel mit Autorinnen und Autoren}.
 Digitale Edition, https://schnitzler-briefe.acdh.oeaw.ac.at/{\dateiname}.html (Stand \today)
\fi

\end{document}


      