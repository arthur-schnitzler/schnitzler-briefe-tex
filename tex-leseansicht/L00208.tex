%% latex-leseansicht-vorspann.tex
%% Vorspann für die Leseansicht.
%% Lädt die gemeinsame Datei latex-vorspann.tex mit nicht gesetztem Schalter.

\newif\ifkorrekturansicht
\korrekturansichtfalse

\input{../tex-inputs/latex-vorspann}


\section[Arthur Schnitzler an Richard Beer-Hofmann, 3. 5. 1893]{L00208 Arthur Schnitzler an Richard Beer-Hofmann, 3. 5. 1893}
\nopagebreak\mylabel{L00208v}
\rehead{ }\normalsize\beginnumbering\briefempfaengerindex{Beer-Hofmann, Richard@\textsc{Beer-Hofmann, Richard}!zzzSchnitzler, Arthur@\emph{von Arthur Schnitzler}!1893-05-031@{3. 5. 1893}|(be}
\toendnotes[C]{\smallbreak\pagebreak[2]}
\correspDesc{Versand  durch Arthur Schnitzler am 3. 5. 1893 in Wien
\newline{}Erhalt  durch Richard Beer-Hofmann im Zeitraum [3. 5. 1893
                  – 7. 5. 1893?] \textbf{Ort fehlend} }\toendnotes[C]{\smallbreak}
\Standort{YCGL, MSS 31.}
\physDesc{Sonderfall, Kuvert, 912 Zeichen (gedruckte Todesanzeige, Umschlag mit Trauerrand )
\newline{}\noindent{}Druck: »M. ENGEL{ }{\kaufmannsund}{ }SÖHNE\orgindex{M. Engel und Söhne@M. Engel und Söhne|pw}{ }WIEN, 1., LICHTENFELSGASSE 9\oindex{Wien@\textbf{Wien}!I., Innere Stadt@\textbf{I., Innere Stadt}!Lichtenfelsgasse@\textbf{Lichtenfelsgasse}, \emph{Straße}|pw}« (\noindent{}Druck: »M. ENGEL{ }{\kaufmannsund}{ }SÖHNE\orgindex{M. Engel und Söhne@M. Engel und Söhne|pw}{ }WIEN, 1., LICHTENFELSGASSE 9\oindex{Wien@\textbf{Wien}!I., Innere Stadt@\textbf{I., Innere Stadt}!Lichtenfelsgasse@\textbf{Lichtenfelsgasse}, \emph{Straße}|pw}«)
\newline{}Handschrift: schwarze Tinte, deutsche Kurrent (\noindent{}Umschlag)
\newline{}Versand: Stempel: »\nobreak{}\oindex{Wien@\textbf{Wien}, \emph{Verwaltungsgebiet}|pwk}Wien 1/1, 3. 5. 93, 3–4 N\nobreak{}«.  }\pstart{}{\pb}Herrn \textsc{Dr. Rich.
                     Beer-Hofmann}\pend{}\pstart{}\textsc{Wien\oindex{Wien@\textbf{Wien}, \emph{Verwaltungsgebiet}|pw}}\pend{}\pstart{}\textsc{I Wollzeile 15\oindex{Wien@\textbf{Wien}!I., Innere Stadt@\textbf{I., Innere Stadt}!Wollzeile 15 (»Berthahof«)@\textbf{Wollzeile 15 (»Berthahof«)}, \emph{Wohngebäude}|pw}}.\pend{}{\bigskip}\vspace{1em}
\pstart
           \noindent{}{\pb}Tieferschüttert geben die Unterzeichneten hiemit im
               eigenen und im Namen der Familie Nachricht von dem Hinscheiden ihres innigstgeliebten
               Gatten, resp. Vaters, Bruders und Schwiegervaters, des Herrn\pend
           
\pstart
           \centering{}Dr. Johann Schnitzler\pwindex{Schnitzler, Johann 10.\,4.\,1835 Nagykanizsa – 2.\,5.\,1893 Wien@\textsc{Schnitzler, Johann} (10.\,4.\,1835 Nagykanizsa – 2.\,5.\,1893 Wien), \emph{Laryngologe}|pw}\pend
           
\pstart
           \centering{}k. k. Regierungsrath, k. k. a. o. Universitäts-Professor, Direktor der
                  allgemeinen Poliklinik\orgindex{Allgemeine Poliklinik@Allgemeine Poliklinik|pw}, Commandeur des dän.\oindex{Dänemark@\textbf{Dänemark}|pw}{ }Danebrog-Ordens\orgindex{Dannebrogorden@Dannebrogorden|pw} etc. etc.\pend
           
\pstart
           welcher nach kurzem Leiden am 2. Mai 1893, Nachmittags ½ 2
                  Uhr, im 59. Lebensjahre verschieden ist.\pend
           
\pstart
           Die irdische Hülle des theuren Verblichenen wird Donnerstag, den 4. Mai,
               ½ 10 Uhr Vormittags vom Trauerhause I., Burgring 1\oindex{Wien@\textbf{Wien}!I., Innere Stadt@\textbf{I., Innere Stadt}!Wohnung und Ordination Johann Schnitzler Burgring 1@\textbf{Wohnung und Ordination Johann Schnitzler Burgring 1}, \emph{Ordination}|pw}, auf den Central-Friedhof
                  (israel. Abtheilung)\oindex{Wien@\textbf{Wien}!XI., Simmering@\textbf{XI., Simmering}!Wiener Zentralfriedhof@\textbf{Wiener Zentralfriedhof}, \emph{Friedhof}|pw} überführt und dort zur ewigen Ruhe bestattet.\pend
           \leftskip=3em{}
\pstart
           \noindent{}\so{Wien}\oindex{Wien@\textbf{Wien}, \emph{Verwaltungsgebiet}|pw}, 3. Mai 1893.\pend
           \leftskip=0em{}
\pstart
           \centering{}Louise Schnitzler\pwindex{Schnitzler, Louise 8.\,7.\,1840 Kőszeg – 9.\,9.\,1911 Wien@\textsc{Schnitzler, Louise} (8.\,7.\,1840 Kőszeg – 9.\,9.\,1911 Wien)|pw}{\\}geb. Markbreiter\pwindex{Schnitzler, Louise 8.\,7.\,1840 Kőszeg – 9.\,9.\,1911 Wien@\textsc{Schnitzler, Louise} (8.\,7.\,1840 Kőszeg – 9.\,9.\,1911 Wien)|pw}{\\}als Gattin.\pend
           
\pstart
           
\pstart
           Dr. Arthur Schnitzler\pend
           
\pstart
           \raggedleft{}Johanna Willheim\pwindex{Wilheim, Johanna 1839? Nagykanizsa – Mitte September 1925 Budapest@\textsc{Wilheim, Johanna} (1839? Nagykanizsa – Mitte September 1925 Budapest)|pw}\pend
           \pend
           
\pstart
           
\pstart
           Dr. Julius Schnitzler\pwindex{Schnitzler, Julius 13.\,7.\,1865 Wien – 29.\,6.\,1939 ebd.@\textsc{Schnitzler, Julius} (13.\,7.\,1865 Wien – 29.\,6.\,1939 ebd.), \emph{Chirurg}|pw}\pend
           
\pstart
           \raggedleft{}geb. Schnitzler\pend
           \pend
           
\pstart
           
\pstart
           Gisela Hajek\pwindex{Hajek, Gisela 20.\,12.\,1867 Wien – 3.\,2.\,1953 Cambridge@\textsc{Hajek, Gisela} (20.\,12.\,1867 Wien – 3.\,2.\,1953 Cambridge)|pw}\pend
           
\pstart
           \raggedleft{}als Schwester.\pend
           \pend
           
\pstart
           
\pstart
           als Kinder.\pend
           
\pstart
           \raggedleft{}Dr. Marcus Hajek\pwindex{Hajek, Markus 25.\,11.\,1861 Vršac – 4.\,4.\,1941 London@\textsc{Hajek, Markus} (25.\,11.\,1861 Vršac – 4.\,4.\,1941 London), \emph{Mediziner, Laryngologe}|pw}\pend
           \pend
           
\pstart
           \raggedleft{}als Schwiegersohn\pend
           \selectlanguage{ngerman}\endnumbering\briefempfaengerindex{Beer-Hofmann, Richard@\textsc{Beer-Hofmann, Richard}!zzzSchnitzler, Arthur@\emph{von Arthur Schnitzler}!1893-05-031@{3. 5. 1893}|)be}\mylabel{L00208h}  \newcommand{\dateiname}{L00208}\newcommand{\titel}{Arthur Schnitzler an Richard Beer-Hofmann, 3. 5. 1893}\newcommand{\editorInnen}{Martin Anton Müller und Gerd-Hermann Susen}%% latex-leseansicht-abspann.tex
%% Abspann für die Leseansicht.
%% Der Schalter \ifkorrekturansicht ist bereits durch den Vorspann gesetzt.

%% latex-abspann.tex
%% Gemeinsamer Abspann für Korrekturansicht und Leseansicht.
%% Setzt den Schalter \ifkorrekturansicht voraus (gesetzt in den
%% einbindenden Dateien latex-korrekturansicht-abspann.tex bzw.
%% latex-leseansicht-abspann.tex).
%% ---------------------------------------------------------------

\normalsize

% Das esempio-Environment wird nur in der Leseansicht benötigt
\ifkorrekturansicht\else
\newenvironment{esempio}[3]%
{
    \vspace{1.5ex}
    \rlap{\underline{#1}}
    \par
    \setlength{\parindent}{0cm}
    \nopagebreak
    \leftskip=#2cm
    \rightskip=#3cm
}
{
    \par
}
\fi

\doendnotes{C}
\bigskip
\vfill

\clearpage

\footnotesize

\ifkorrekturansicht
  \lohead{\textsc{register}}
\fi

% theindex-Environment neu definieren ohne reledmac
\makeatletter
\renewenvironment{theindex}{%
  \ifkorrekturansicht
    \section*{\indexname}%
  \else
    \subsubsection*{Index der erwähnten Entitäten}%
  \fi
  \setlength{\parindent}{0pt}%
  \setlength{\parskip}{0pt plus 0.3pt}%
  \let\item\@idxitem
}{%
  \ifkorrekturansicht\clearpage\fi
}
\makeatother

\IfFileExists{\jobname-pw.ind}{\input{\jobname-pw.ind}}{}

% Quellenangabe nur in der Leseansicht
\ifkorrekturansicht\else
% Fallback-Definitionen, falls die .tex-Datei \titel etc. nicht gesetzt hat
\providecommand{\titel}{}
\providecommand{\editorInnen}{}
\providecommand{\dateiname}{\jobname}

\vspace{3cm}

\vfill

\footnotesize
\textsc{Quelle}: \titel. Herausgegeben von {\editorInnen}. In: \emph{Arthur Schnitzler: Briefwechsel mit Autorinnen und Autoren}.
 Digitale Edition, https://schnitzler-briefe.acdh.oeaw.ac.at/{\dateiname}.html (Stand \today)
\fi

\end{document}


