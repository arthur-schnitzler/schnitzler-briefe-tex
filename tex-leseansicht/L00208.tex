%% latex-korrekturansicht-vorspann.tex
%% Vorspann für die Korrekturansicht.
%% Lädt die gemeinsame Datei latex-vorspann.tex mit gesetztem Schalter.

\newif\ifkorrekturansicht
\korrekturansichttrue

\input{../tex-inputs/latex-vorspann}


\section[Arthur Schnitzler an Richard Beer-Hofmann, 3. 5. 1893]{L00208 Arthur Schnitzler an Richard Beer-Hofmann, 3. 5. 1893}
\nopagebreak\mylabel{L00208v}
\rehead{ }\normalsize\beginnumbering\briefempfaengerindex{Beer-Hofmann, Richard@\textsc{Beer-Hofmann, Richard}!zzzSchnitzler, Arthur@\emph{von Arthur Schnitzler}!1893-05-031@{3. 5. 1893}|(be}
\toendnotes[C]{\smallbreak\pagebreak[2]}\Standort{YCGL, MSS 31.}
\physDesc{Sonderfall, Umschlag, 912 Zeichen (gedruckte Todesanzeige, Umschlag mit Trauerrand )
\newline{}\noindent{}Druck: »M. ENGEL{ }{\kaufmannsund}{ }SÖHNE\orgindex{M. Engel und Soehne@M. Engel und Söhne|pw}{ }WIEN, 1., LICHTENFELSGASSE 9\oindex{Lichtenfelsgasse@\textbf{Lichtenfelsgasse}, \emph{Straße (K.STR)}|pw}«\noindent{}Druck: »M. ENGEL{ }{\kaufmannsund}{ }SÖHNE\orgindex{M. Engel und Soehne@M. Engel und Söhne|pw}{ }WIEN, 1., LICHTENFELSGASSE 9\oindex{Lichtenfelsgasse@\textbf{Lichtenfelsgasse}, \emph{Straße (K.STR)}|pw}«
\newline{}Handschrift: schwarze Tinte, deutsche Kurrent (\noindent{}Umschlag)
\newline{}Versand: Stempel: »\nobreak{}Wien 1/1, 3. 5. 93, 3–4 N\nobreak{}«.  }\pstart{}{\pb}Herrn \textsc{Dr. Rich.
                     Beer-Hofmann}\pend{}\pstart{}\textsc{Wien\oindex{Wien@\textbf{Wien}, \emph{A.ADM2}|pw}}\pend{}\pstart{}\textsc{I Wollzeile 15\oindex{Wollzeile@\textbf{Wollzeile}, \emph{Straße (K.STR)}|pw}}.\pend{}{\bigskip}\vspace{1em}
\pstart
           \noindent{}{\pb}Tieferschüttert geben die Unterzeichneten hiemit im
               eigenen und im Namen der Familie Nachricht von dem Hinscheiden ihres innigstgeliebten
               Gatten, resp. Vaters, Bruders und Schwiegervaters, des Herrn\pend
           
\pstart
           \centering{}Dr. Johann Schnitzler\pwindex{Schnitzler, Johann 10.04.1835 – 02.05.1893@\textsc{Schnitzler, Johann} (10.04.1835 – 02.05.1893), \emph{Laryngologe/Laryngologin}|pw}\pend
           
\pstart
           \centering{}k. k. Regierungsrath, k. k. a. o. Universitäts-Professor, Direktor der
                  allgemeinen Poliklinik\orgindex{Allgemeine Poliklinik@Allgemeine Poliklinik|pw}, Commandeur des dän.\oindex{Daenemark@\textbf{Dänemark}, \emph{A.PCLI}|pw}{ }Danebrog-Ordens\orgindex{Dannebrogorden@Dannebrogorden|pw} etc. etc.\pend
           
\pstart
           welcher nach kurzem Leiden am 2. Mai 1893, Nachmittags ½ 2
                  Uhr, im 59. Lebensjahre verschieden ist.\pend
           
\pstart
           Die irdische Hülle des theuren Verblichenen wird Donnerstag, den 4. Mai,
               ½ 10 Uhr Vormittags vom Trauerhause I., Burgring 1\oindex{Wohnung und Ordination Johann Schnitzler Burgring 1@\textbf{Wohnung und Ordination Johann Schnitzler Burgring 1}, \emph{Ordination}|pw}, auf den Central-Friedhof
                  (israel. Abtheilung)\oindex{Wiener Zentralfriedhof@\textbf{Wiener Zentralfriedhof}, \emph{Friedhof (K.FRH)}|pw} überführt und dort zur ewigen Ruhe bestattet.\pend
           \leftskip=3em{}
\pstart
           \noindent{}\so{Wien}\oindex{Wien@\textbf{Wien}, \emph{A.ADM2}|pw}, 3. Mai 1893.\pend
           \leftskip=0em{}
\pstart
           \centering{}Louise Schnitzler\pwindex{Schnitzler, Louise 1840-07-08 – 1911-09-09@\textsc{Schnitzler, Louise} (1840-07-08 – 1911-09-09)|pw}{\\}geb. Markbreiter\pwindex{Schnitzler, Louise 1840-07-08 – 1911-09-09@\textsc{Schnitzler, Louise} (1840-07-08 – 1911-09-09)|pw}{\\}als Gattin.\pend
           
\pstart
           
\pstart
           Dr. Arthur Schnitzler\pend
           
\pstart
           \raggedleft{}Johanna Willheim\pwindex{Wilheim, Johanna 1839? – September 1925@\textsc{Wilheim, Johanna} (1839? – September 1925)|pw}\pend
           \pend
           
\pstart
           
\pstart
           Dr. Julius Schnitzler\pwindex{Schnitzler, Julius 13.07.1865 – 29.06.1939@\textsc{Schnitzler, Julius} (13.07.1865 – 29.06.1939), \emph{Chirurg/Chirurgin}|pw}\pend
           
\pstart
           \raggedleft{}geb. Schnitzler\pend
           \pend
           
\pstart
           
\pstart
           Gisela Hajek\pwindex{Hajek, Gisela 20.12.1867 – 03.02.1953@\textsc{Hajek, Gisela} (20.12.1867 – 03.02.1953)|pw}\pend
           
\pstart
           \raggedleft{}als Schwester.\pend
           \pend
           
\pstart
           
\pstart
           als Kinder.\pend
           
\pstart
           \raggedleft{}Dr. Marcus Hajek\pwindex{Hajek, Markus 25.11.1861 – 04.04.1941@\textsc{Hajek, Markus} (25.11.1861 – 04.04.1941), \emph{Mediziner/Medizinerin, Laryngologe/Laryngologin}|pw}\pend
           \pend
           
\pstart
           \raggedleft{}als Schwiegersohn\pend
           \selectlanguage{ngerman}\endnumbering\briefempfaengerindex{Beer-Hofmann, Richard@\textsc{Beer-Hofmann, Richard}!zzzSchnitzler, Arthur@\emph{von Arthur Schnitzler}!1893-05-031@{3. 5. 1893}|)be}\mylabel{L00208h}  \normalsize

\doendnotes{C}
\bigskip
\vfill

\clearpage

\footnotesize

\lohead{\textsc{register}}

% Definiere theindex-Environment komplett neu ohne reledmac
\makeatletter
\renewenvironment{theindex}{%
  \section*{\indexname}%
  \setlength{\parindent}{0pt}%
  \setlength{\parskip}{0pt plus 0.3pt}%
  \let\item\@idxitem
}{%
  \clearpage
}
\makeatother

\IfFileExists{\jobname-pw.ind}{\input{\jobname-pw.ind}}{}

\end{document}

      