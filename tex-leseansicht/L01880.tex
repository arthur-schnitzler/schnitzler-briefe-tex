%% latex-korrekturansicht-vorspann.tex
%% Vorspann für die Korrekturansicht.
%% Lädt die gemeinsame Datei latex-vorspann.tex mit gesetztem Schalter.

\newif\ifkorrekturansicht
\korrekturansichttrue

\input{../tex-inputs/latex-vorspann}


\section[Albert Ehrenstein an Arthur Schnitzler, 17. 10. 1909]{L01880 Albert Ehrenstein an Arthur Schnitzler, 17. 10. 1909}
\nopagebreak\mylabel{L01880v}
\rehead{ }\normalsize\beginnumbering\briefempfaengerindex{Schnitzler, Arthur@\textsc{Schnitzler, Arthur}!zzzEhrenstein, Albert@\emph{von Albert Ehrenstein}!1909-10-171@{17. 10. 1909}|(be}
\toendnotes[C]{\smallbreak\pagebreak[2]}\Standort{CUL, Schnitzler, B 30.}
\physDesc{Brief, 1 Blatt, 4 Seiten, 3120 Zeichen
\newline{}Handschrift: schwarze Tinte, deutsche Kurrent
\newline{}Schnitzler: mit Bleistift beschriftet: »\textsc{Ehrenst\textcolor{gray}{ein}}« }
\buchAbdrucke{\weitereDrucke{Albert Ehrenstein: \emph{Briefe}. München: \emph{Boer} 1989, S. 32–33.} }\toendnotes[C]{\smallbreak}
\pstart
           
\pstart
           {\pb}\textsc{Albert Ehrenstein}\pend
           
\pstart
           \raggedleft{}17. Okt. 09. \pend
           \pend
           
\pstart
           \textsc{XVI. Ottakringerstr} 114\oindex{Ottakringer Strasse@\textbf{Ottakringer Straße}, \emph{Straße (K.STR)}|pw}.\pend
           
\pstart{}\textsc{Sehr geehrter Herr Doktor,}\pend\vspace{0.5em}
\pstart
           nachdem meine nach Venedig\oindex{Venedig@\textbf{Venedig}, \emph{P.PPLA}|pw} geſandten Manuſkripte
               einen Monat lang verſchollen und ich, da ſie auf 1000 K verſichert waren, bereits
               geträumt hatte, in den Beſitz dieſer Unſumme zu gelangen, geſchah es mir, daß ſie
               ſich doch noch vorfanden und einige Zeit nachher feierte ich denn auch ein halb
               gerührtes, halb ärgerliches Wiederſehen mit meinen Arbeiten. Um auch andere an meinen
               Gefühlen teilnehmen zu laſſen, transportierte ich einiges zu Herrn Auernheimer\pwindex{Auernheimer, Raoul 15.04.1876 – 06.01.1948@\textsc{Auernheimer, Raoul} (15.04.1876 – 06.01.1948), \emph{Schriftsteller/Schriftstellerin, Journalist/Journalistin, Kritiker/Kritikerin}|pw}, den ich nicht antraf. Weil mir die
               Angelegenheit {\pb}damals noch dringend
               ſchien, machte ich mich 14 Tage darauf wieder auf den Weg in die Neulinggaſſe\oindex{Neulinggasse@\textbf{Neulinggasse}, \emph{Straße (K.STR)}|pw}. Da nun ergab es ſich, daß A.\pwindex{Auernheimer, Raoul 15.04.1876 – 06.01.1948@\textsc{Auernheimer, Raoul} (15.04.1876 – 06.01.1948), \emph{Schriftsteller/Schriftstellerin, Journalist/Journalistin, Kritiker/Kritikerin}|pw} bis dahin jede Berührung mit meinen Operaten ängſtlich
               vermieden hatte und auch bis Mittwoch, als ich beſtelltermaßen zu ihm
               kam, hatte er noch nicht jenen Heroismus aufgebracht, der zur reſtloſen Bewältigung
               mir entſtammender ſchriftſtelleriſcher Gebilde leider unbedingt nötig ſein dürfte.
               Nichtsdeſtoweniger und obwohl er nur in kleineren und keineswegs in den für die Preſſe\orgindex{Neue Freie Presse@Neue Freie Presse|pw} beſtimmten Erzählungen geblättert hatte,
               kam er ſpielend zu einem erſchöpfenden Urteil über mich. Er nannte mich ein unreifes
               Talent, phantaſtiſch nach Meyrinks\pwindex{Meyrink, Gustav 19.01.1868 – 04.12.1932@\textsc{Meyrink, Gustav} (19.01.1868 – 04.12.1932), \emph{Schriftsteller/Schriftstellerin}|pw} Art, meine
               Sachen ungeeignet zur Publikation, möglich höchſtens für den »Hyperion\orgindex{Hyperion@Hyperion|pw}« oder »Spiegel\orgindex{Spiegel. Muenchner Halbmonatsschrift@Der Spiegel. Münchner Halbmonatsschrift|pw}« –
               Zeitſchriften {\pb}übrigens, die mein profanes
               Auge niemals ſchaute und von denen ich bloß weiß, daß ſie im Lande Blei\pwindex{Blei, Franz 18.01.1871 – 10.07.1942@\textsc{Blei, Franz} (18.01.1871 – 10.07.1942), \emph{Schriftsteller/Schriftstellerin}|pw} liegen. Seine Rede krönte er mit einem
               anſcheinend unſchuldigen Satz, dem vortrefflich gewählten \textsc{ceterum censeo}: »Was wollen ſie eigentlich? Falls bei ihnen einmal mehr als
               Anſätze, nämlich Erfüllungen vorhanden ſein ſollten, wird ſie Schnitzler an die Neue Rundſchau\orgindex{Neue Rundschau, Neue Deutsche Rundschau, Freie Buehne@Neue Rundschau, Neue Deutsche Rundschau, Freie Bühne|pw} empfehlen und das wird viel mehr
               ſein als wenn ſie in ſo einem Literatenblättchen gedruckt würden.« Schließlich
               verſtand er ſich dazu, mir die Zuſendung von Recenſionsexemplaren zu verſprechen,
               womit die ganze Affaire für mich abgetan ſein wird. Mehr brauche ich nämlich
               glücklicherweiſe von der Preſſe\orgindex{Neue Freie Presse@Neue Freie Presse|pw} nicht und wenn
               ich früher erfahren hätte, was ich leider erſt Donnerstag erfuhr, daß
               nämlich an der Verzögerung der Approbation {\pb}meiner Diſſertation\pwindex{Lage in Ungarn (Siebenbuergen und Serbien ausgenommen) im Jahre 1790@\emph{Die Lage in Ungarn (Siebenbürgen und Serbien ausgenommen) im Jahre 1790}|pwv} nicht ſo
               ſehr Übelwollen als Schlamperei die Schuld trug, dann hätte ich Ihnen, ſehr geehrter
               Herr Doktor, und mir allerhand erſparen können{\dotsfour} Allerdings
               ſehne ich mich noch immer danach, nicht etwa einer Zelle in jener papierenen Welt,
               ſondern eines Platzes an der Sonne teilhaftig zu werden, um endlich zu einigem Genuße
               meines Lebens zu gelangen. Meine Perſonalkenntniſſe der Wien\oindex{Wien@\textbf{Wien}, \emph{A.ADM2}|pw}er Journaliſtik wünſche ich dennoch nicht zu bereichern, ich möchte
               vielmehr äußerſt gern aus Ihrem Munde vernehmen, ob der in »Baber\pwindex{Tod des Zehir eddin Muhammed Baber@\emph{Tod des Zehir eddin Muhammed Baber}|pw}« und »Apaturien\pwindex{Apaturien@\emph{Apaturien}|pw}«
               gezeigte Stil für mich und andere von Wert iſt und ob eine Veröffentlichung oder
               Edition der beſſeren meiner Skizzen und Erzählungen einen materiellen Effekt haben
               könnte? Soll ich ſchon jetzt daran gehen, meine Sammlung redaktioneller Kundgebungen
               durch Angliederung ähnlich negativer Beſcheide von Verlegern gebührend auszubauen?
               Vielleicht können Sie, ſehr geehrter Herr Doktor, raten\pend
           
\pstart
           Ihrem ergebenſten{\\[\baselineskip]}\spacefill\mbox{Albert Ehrenstein.}\pend
           \leftskip=0em{}\selectlanguage{ngerman}\endnumbering\briefempfaengerindex{Schnitzler, Arthur@\textsc{Schnitzler, Arthur}!zzzEhrenstein, Albert@\emph{von Albert Ehrenstein}!1909-10-171@{17. 10. 1909}|)be}\mylabel{L01880h}  \normalsize

\doendnotes{C}
\bigskip
\vfill

\clearpage

\footnotesize

\lohead{\textsc{register}}

% Definiere theindex-Environment komplett neu ohne reledmac
\makeatletter
\renewenvironment{theindex}{%
  \section*{\indexname}%
  \setlength{\parindent}{0pt}%
  \setlength{\parskip}{0pt plus 0.3pt}%
  \let\item\@idxitem
}{%
  \clearpage
}
\makeatother

\IfFileExists{\jobname-pw.ind}{\input{\jobname-pw.ind}}{}

\end{document}

      