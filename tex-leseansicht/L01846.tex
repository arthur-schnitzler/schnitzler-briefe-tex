%% latex-leseansicht-vorspann.tex
%% Vorspann für die Leseansicht.
%% Lädt die gemeinsame Datei latex-vorspann.tex mit nicht gesetztem Schalter.

\newif\ifkorrekturansicht
\korrekturansichtfalse

\input{../tex-inputs/latex-vorspann}


         
         \renewcommand{\erwaehntePersonen}{Personen: Hermann Bahr, Olga Schnitzler}
         \renewcommand{\erwaehnteInstitutionen}{Institutionen: Paul Cassirer Verlag}
         \renewcommand{\erwaehnteOrte}{Orte: Bayreuth, Edmund-Weiß-Gasse, Santa Maria Elisabetta, Venedig, Wien, Österreich}
         \renewcommand{\erwaehnteWerke}{Werke: Drut. Roman, Tagebuch [Berlin: Paul Cassirer]}
               \section[Hermann Bahr an Arthur Schnitzler, 18. 6. 1909]{ Hermann Bahr an Arthur Schnitzler, 18. 6. 1909}\nopagebreak\mylabel{v}\rehead{ }\begin{ledgroupsized}[t]{13cm}\normalsize\beginnumbering \toendnotes[C]{\smallbreak\pagebreak[2]} \Standort{CUL, Schnitzler, B 5b.}
\physDesc{Bildpostkarte, 307 Zeichen
\newline{}Handschrift: 1) schwarze Tinte, deutsche Kurrent\hspace{1em}2) schwarze Tinte, lateinische Kurrent (\noindent{}Adresse)\hspace{1em}
\newline{}Versand: 1) Stempel: »\nobreak{}\oindex{Santa Maria Elisabetta@\textbf{Santa Maria Elisabetta}|pwk}S. Elisabetta\nobreak{}«.   2) Stempel: »\nobreak{}\oindex{Santa Maria Elisabetta@\textbf{Santa Maria Elisabetta}|pwk}S. Elisabetta Lido, 18 6 08\nobreak{}«. 
\newline{}Schnitzler: mit Bleistift ergänzt »\textsc{Bahr}« 
\newline{}Ordnung: mit Bleistift von unbekannter Hand nummeriert:
                                    »157« }\buchAbdrucke{\weitereDrucke{Hermann Bahr, Arthur Schnitzler: \emph{Briefwechsel, Aufzeichnungen, Dokumente (1891–1931)}. Hg. Kurt Ifkovits und Martin Anton Müller. Göttingen: \emph{Wallstein} 2018, S. 417.} }\toendnotes[C]{\smallbreak}\pstart{}{\pb}D\textsuperscript{r} Artur
                  Schnitzler\pend{}\pstart{}XVIII Spöttelgasse 7\oindex{XXXX Ortsangabe fehlt|pw}\pend{}\pstart{}Vienna\oindex{Wien@\textbf{Wien}|pw}\pend{}\pstart{}Austria\oindex{Oesterreich@\textbf{Österreich}|pw}\pend{}{\bigskip}\pstart
           \noindent{}\centering{}\textcolor{gray}{\textbf{{\pb}Venezia. Rio dei Mendicanti e fondamenta\oindex{Venedig@\textbf{Venedig}|pw}}}\pend
           \pstart
           \raggedleft{}{\pb}18. 6. 09\pend
           \pstart
           Lieber Artur! Hoffentlich iſt Dir »Drut\pwindex{Bahr, Hermann 19.07.1863 – 15.01.1934@\textsc{Bahr, Hermann} (19.07.1863 – 15.01.1934), \emph{Schriftsteller, Kritiker}!Drut. Roman1909@\strich\emph{Drut. Roman} {[}1909{]}|pw}« sowie mein »\label{K_L01846-1v}\edtext{Tagebuch\pwindex{Bahr, Hermann 19.07.1863 – 15.01.1934@\textsc{Bahr, Hermann} (19.07.1863 – 15.01.1934), \emph{Schriftsteller, Kritiker}!Tagebuch [Berlin: Paul Cassirer]1909@\strich\emph{Tagebuch [Berlin: Paul Cassirer]} {[}1909{]}|pw}}{\lemma{\textnormal{\emph{Tagebuch}}}\Cendnote{\textnormal{Das heißt die erste gesammelte
                  Buchausgabe der Kolumnen: Hermann Bahr\pwindex{Bahr, Hermann 19.07.1863 – 15.01.1934@\textsc{Bahr, Hermann} (19.07.1863 – 15.01.1934), \emph{Schriftsteller, Kritiker}|pwk}: \emph{Tagebuch}\pwindex{Bahr, Hermann 19.07.1863 – 15.01.1934@\textsc{Bahr, Hermann} (19.07.1863 – 15.01.1934), \emph{Schriftsteller, Kritiker}!Tagebuch [Berlin: Paul Cassirer]1909@\strich\emph{Tagebuch [Berlin: Paul Cassirer]} {[}1909{]}|pwk}. Berlin: \emph{Paul
                        Cassirer}\orgindex{Paul Cassirer Verlag@Paul Cassirer Verlag|pwk}{ }1909.}}}\label{K_L01846-1h}« richtig zugekommen. – Wir ſind ſeit drei Wochen hier und gehen
                  \textcolor{gray}{nun} nächſte Woche nach Bayreuth\oindex{Bayreuth@\textbf{Bayreuth}|pw}. – Grüß Deine verehrte liebe Frau\pwindex{Schnitzler, Olga 17.01.1882 – 13.01.1970@\textsc{Schnitzler, Olga} (17.01.1882 – 13.01.1970), \emph{Schauspielerin, Sängerin}|pwv} und habt einen ſchönen Sommer!\pend
           \pstart
           Herzlichſt{\\[\baselineskip]}Dein alter{\\[\baselineskip]}\spacefill\mbox{HermannBahr}\pend
           \leftskip=0em{}
         
         \endnumbering\mylabel{h}\end{ledgroupsized}  \newcommand{\dateiname}{L01846}\newcommand{\titel}{Hermann Bahr an Arthur Schnitzler, 18. 6. 1909}\newcommand{\editorInnen}{ Kurt Ifkovits,  Martin Anton Müller}%% latex-leseansicht-abspann.tex
%% Abspann für die Leseansicht.
%% Der Schalter \ifkorrekturansicht ist bereits durch den Vorspann gesetzt.

%% latex-abspann.tex
%% Gemeinsamer Abspann für Korrekturansicht und Leseansicht.
%% Setzt den Schalter \ifkorrekturansicht voraus (gesetzt in den
%% einbindenden Dateien latex-korrekturansicht-abspann.tex bzw.
%% latex-leseansicht-abspann.tex).
%% ---------------------------------------------------------------

\normalsize

% Das esempio-Environment wird nur in der Leseansicht benötigt
\ifkorrekturansicht\else
\newenvironment{esempio}[3]%
{
    \vspace{1.5ex}
    \rlap{\underline{#1}}
    \par
    \setlength{\parindent}{0cm}
    \nopagebreak
    \leftskip=#2cm
    \rightskip=#3cm
}
{
    \par
}
\fi

\doendnotes{C}
\bigskip
\vfill

\clearpage

\footnotesize

\ifkorrekturansicht
  \lohead{\textsc{register}}
\fi

% theindex-Environment neu definieren ohne reledmac
\makeatletter
\renewenvironment{theindex}{%
  \ifkorrekturansicht
    \section*{\indexname}%
  \else
    \subsubsection*{Index der erwähnten Entitäten}%
  \fi
  \setlength{\parindent}{0pt}%
  \setlength{\parskip}{0pt plus 0.3pt}%
  \let\item\@idxitem
}{%
  \ifkorrekturansicht\clearpage\fi
}
\makeatother

\IfFileExists{\jobname-pw.ind}{\input{\jobname-pw.ind}}{}

% Quellenangabe nur in der Leseansicht
\ifkorrekturansicht\else
% Fallback-Definitionen, falls die .tex-Datei \titel etc. nicht gesetzt hat
\providecommand{\titel}{}
\providecommand{\editorInnen}{}
\providecommand{\dateiname}{\jobname}

\vspace{3cm}

\vfill

\footnotesize
\textsc{Quelle}: \titel. Herausgegeben von {\editorInnen}. In: \emph{Arthur Schnitzler: Briefwechsel mit Autorinnen und Autoren}.
 Digitale Edition, https://schnitzler-briefe.acdh.oeaw.ac.at/{\dateiname}.html (Stand \today)
\fi

\end{document}


      