%% latex-korrekturansicht-vorspann.tex
%% Vorspann für die Korrekturansicht.
%% Lädt die gemeinsame Datei latex-vorspann.tex mit gesetztem Schalter.

\newif\ifkorrekturansicht
\korrekturansichttrue

\input{../tex-inputs/latex-vorspann}


\section[Hermann Bahr an Arthur Schnitzler, 18. 6. 1909]{L01846 Hermann Bahr an Arthur Schnitzler, 18. 6. 1909}
\nopagebreak\mylabel{L01846v}
\rehead{ }\normalsize\beginnumbering\briefempfaengerindex{Schnitzler, Arthur@\textsc{Schnitzler, Arthur}!zzzBahr, Hermann@\emph{von Hermann Bahr}!1909-06-182@{18. 6. 1909}|(be}
\toendnotes[C]{\smallbreak\pagebreak[2]}\Standort{CUL, Schnitzler, B 5b.}
\physDesc{Bildpostkarte, 307 Zeichen
\newline{}Handschrift: 1) schwarze Tinte, deutsche Kurrent\hspace{1em}2) schwarze Tinte, lateinische Kurrent (\noindent{}Adresse)\hspace{1em}
\newline{}Versand: 1) Stempel: »\nobreak{}\oindex{Santa Maria Elisabetta@\textbf{Santa Maria Elisabetta}, \emph{Bezirk (A.BZK)}|pwk}S. Elisabetta\nobreak{}«.   2) Stempel: »\nobreak{}\oindex{Santa Maria Elisabetta@\textbf{Santa Maria Elisabetta}, \emph{Bezirk (A.BZK)}|pwk}S. Elisabetta Lido, 18 6 08\nobreak{}«. 
\newline{}Schnitzler: mit Bleistift ergänzt »\textsc{Bahr}« 
\newline{}Ordnung: mit Bleistift von unbekannter Hand nummeriert:
                                    »157« }
\buchAbdrucke{\weitereDrucke{Hermann Bahr, Arthur Schnitzler: \emph{Briefwechsel, Aufzeichnungen, Dokumente (1891–1931)}. Göttingen: \emph{Wallstein} 2018, S. 417.} }\toendnotes[C]{\smallbreak}\pstart{}{\pb}D\textsuperscript{r} Artur
                  Schnitzler\pend{}\pstart{}XVIII Spöttelgasse 7\oindex{Edmund-Weiss-Gasse 7@\textbf{Edmund-Weiß-Gasse 7}, \emph{Wohngebäude (K.WHS)}|pw}\pend{}\pstart{}Vienna\oindex{Wien@\textbf{Wien}, \emph{A.ADM2}|pw}\pend{}\pstart{}Austria\oindex{Oesterreich@\textbf{Österreich}, \emph{A.PCLI}|pw}\pend{}{\bigskip}
\pstart
           \noindent{}\centering{}{\pb}\textcolor{gray}{\textbf{Venezia. Rio dei Mendicanti e fondamenta\oindex{Venedig@\textbf{Venedig}, \emph{P.PPLA}|pw}}}\pend
           \vspace{1em}
\pstart
           \raggedleft{}{\pb}18. 6. 09\pend
           \vspace{0.5em}
\pstart
           Lieber Artur! Hoffentlich iſt Dir »Drut\pwindex{Drut. Roman@\emph{Drut. Roman}|pw}« sowie mein »\label{K_L01846-1v}\edtext{Tagebuch\pwindex{Tagebuch [Berlin: Paul Cassirer]@\emph{Tagebuch [Berlin: Paul Cassirer]}|pw}}{\lemma{\textnormal{\emph{Tagebuch}}}\Cendnote{\textnormal{Das heißt die erste gesammelte
                  Buchausgabe der Kolumnen: Hermann Bahr\pwindex{Bahr, Hermann 19.07.1863 – 15.01.1934@\textsc{Bahr, Hermann} (19.07.1863 – 15.01.1934), \emph{Schriftsteller/Schriftstellerin, Kritiker/Kritikerin}|pwk}: \emph{Tagebuch}\pwindex{Tagebuch [Berlin: Paul Cassirer]@\emph{Tagebuch [Berlin: Paul Cassirer]}|pwk}. Berlin: \emph{Paul
                        Cassirer}\orgindex{Paul Cassirer Verlag@Paul Cassirer Verlag|pwk}{ }1909.}}}\label{K_L01846-1}« richtig zugekommen. – Wir ſind ſeit drei Wochen hier und gehen
                  \textcolor{gray}{nun} nächſte Woche nach Bayreuth\oindex{Bayreuth@\textbf{Bayreuth}, \emph{P.PPLA2}|pw}. – Grüß Deine verehrte liebe Frau\pwindex{Schnitzler, Olga 17.01.1882 – 13.01.1970@\textsc{Schnitzler, Olga} (17.01.1882 – 13.01.1970), \emph{Schauspieler/Schauspielerin, Sänger/Sängerin}|pwv} und habt einen ſchönen Sommer!\pend
           
\pstart
           Herzlichſt{\\[\baselineskip]}Dein alter{\\[\baselineskip]}\spacefill\mbox{HermannBahr}\pend
           \leftskip=0em{}\selectlanguage{ngerman}\endnumbering\briefempfaengerindex{Schnitzler, Arthur@\textsc{Schnitzler, Arthur}!zzzBahr, Hermann@\emph{von Hermann Bahr}!1909-06-182@{18. 6. 1909}|)be}\mylabel{L01846h}  \normalsize

\doendnotes{C}
\bigskip
\vfill

\clearpage

\footnotesize

\lohead{\textsc{register}}

% Definiere theindex-Environment komplett neu ohne reledmac
\makeatletter
\renewenvironment{theindex}{%
  \section*{\indexname}%
  \setlength{\parindent}{0pt}%
  \setlength{\parskip}{0pt plus 0.3pt}%
  \let\item\@idxitem
}{%
  \clearpage
}
\makeatother

\IfFileExists{\jobname-pw.ind}{\input{\jobname-pw.ind}}{}

\end{document}

      