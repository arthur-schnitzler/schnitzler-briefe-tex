%% latex-korrekturansicht-vorspann.tex
%% Vorspann für die Korrekturansicht.
%% Lädt die gemeinsame Datei latex-vorspann.tex mit gesetztem Schalter.

\newif\ifkorrekturansicht
\korrekturansichttrue

\input{../tex-inputs/latex-vorspann}


\section[Elsa Plessner an Arthur Schnitzler, 13. 1. 1916]{L03731 Elsa Plessner an Arthur Schnitzler, 13. 1. 1916}
\nopagebreak\mylabel{L03731v}
\rehead{ }\normalsize\beginnumbering\briefempfaengerindex{Schnitzler, Arthur@\textsc{Schnitzler, Arthur}!zzzPlessner, Elsa@\emph{von Elsa Plessner}!1916-01-131@{13. 1. 1916}|(be}
\toendnotes[C]{\smallbreak\pagebreak[2]}\Standort{DLA, A:Schnitzler, HS.1985.1.419.}
\physDesc{Brief,  Blätter, 2 Seiten, 1169 Zeichen
\newline{}Handschrift: , lateinische Kurrent
\newline{}Schnitzler: zwei Unterstreichungen }\toendnotes[C]{\smallbreak}
\pstart
           {\pb}München\oindex{Muenchen@\textbf{München}, \emph{P.PPLA}|pw}, den 13. Januar
                     1916Theresienstr. 78\oindex{Theresienstrasse 78@\textbf{Theresienstraße 78}, \emph{Wohngebäude (K.WHS)}|pw}\hfill \pend
           
\pstart{}Verehrter Herr Doctor!\pend\vspace{0.5em}
\pstart
           Soeben erhalte ich Ihr freundliches \label{K_L03731-1v}\edtext{Schreiben vom 12. d.}{\lemma{\textnormal{\emph{Schreiben vom 12. d.}}}\Cendnote{\textnormal{nicht überliefert}}}\label{K_L03731-1} und unterlasse es, Ihnen meine schmerzliche \label{K_L03731-2v}\edtext{Enttäuschung}{\lemma{\textnormal{\emph{Enttäuschung}}}\Cendnote{\textnormal{Auf Plessners\pwindex{Plessner, Elsa 22.08.1875 – 01.05.1932@\textsc{Plessner, Elsa} (22.08.1875 – 01.05.1932), \emph{Schriftsteller/Schriftstellerin}|pwk} Bitte um Lektüre ihres neuen Stückes\pwindex{Musik@\emph{Musik}|pwkv} und auf ihren vertraulichen Ton scheint Schnitzler ablehnend reagiert zu haben. Er notierte: »Las Nm. ein schlechtes Buch von Fr. Plessner, Mscrpt. aus München geschickt, mit eingebildetem Brief.«, A. S.: \emph{Tagebuch}, 16. 1. 1916.}}}\label{K_L03731-2}
               zu schildern: Indessen – gegen Principien ist nichts zu machen und man muss sie
               respectieren. In jedem Falle bitte ich Sie aber, aus dem mit I. \label{K_L03731-3v}\edtext{bezeichneten Umschlag}{\lemma{\textnormal{\emph{bezeichneten Umschlag}}}\Cendnote{\textnormal{Die Beilagen des vorangegangenen Briefes
                  vom 9. 1. 1916 sind nicht
                  überliefert. Es handelte sich um das Werkmanuskript von Plessners\pwindex{Plessner, Elsa 22.08.1875 – 01.05.1932@\textsc{Plessner, Elsa} (22.08.1875 – 01.05.1932), \emph{Schriftsteller/Schriftstellerin}|pwk} unveröffentlicht gebliebenen Schauspiels \emph{Musik}\pwindex{Musik@\emph{Musik}|pwk} und einen nicht identifizierten
                  Brief.}}}\label{K_L03731-3}
               den Brief herauszunehmen, der dem Manuscript\pwindex{Musik@\emph{Musik}|pwv} beigelegt ist. Es wäre möglich, dass dieser Brief Sie bestimmen würde, eine
               Ausnahme zu machen mir gegenüber, die seit zwanzig Jahren eine Art Gewohnheitsrecht
               zu besitzen glaubt – wenn es auch in den letzten Jahren nicht zur Anwendung kam.\pend
           
\pstart
           Inliegend erlaube ich mir, eine ausgefüllte \label{K_L03731-4v}\edtext{Postkarte}{\lemma{\textnormal{\emph{Postkarte}}}\Cendnote{\textnormal{nicht identifiziert}}}\label{K_L03731-4} beizulegen, die Sie der
               Beförderung übergeben mögen, wie es Ihnen, verehrter Herr Doctor! angemessen
               erscheint. Meine Mama\pwindex{Plessner, Clementine 1855-12-07 – 1943-02-27@\textsc{Plessner, Clementine} (1855-12-07 – 1943-02-27), \emph{Schauspieler/Schauspielerin, Filmschauspieler/Filmschauspielerin}|pwv} ist
               gleichzeitig benachrichtigt, so dass das Manuscript\pwindex{Musik@\emph{Musik}|pwv} sofort aus Ihrem Hause abgeholt werden kann.\pend
           
\pstart
           An eine Aufführung des »ersten Capitel\pwindex{erste Kapitel. Schauspiel in drei Akten@\emph{Das erste Kapitel. Schauspiel in drei Akten}|pw}« denke
                  {\pb}ich vorläufig überhaupt nicht mehr. Man trägt ja nicht alte Kleider,
               wenn man neue hat. Daher ist die \label{K_L03731-5v}\edtext{causa Ziegel\pwindex{Ziegel, Erich 1876-08-26 – 1950-11-30@\textsc{Ziegel, Erich} (1876-08-26 – 1950-11-30), \emph{Theaterleiter/Theaterleiterin, Regisseur/Regisseurin, Schauspieler/Schauspielerin}|pw}}{\lemma{\textnormal{\emph{causa Ziegel}}}\Cendnote{\textnormal{Vgl. Elsa Plessner an Arthur Schnitzler, 15. 11. 1915.
               }}}\label{K_L03731-5}, obwohl noch in Schwebe – für mich bedeutungslos geworden.\pend
           
\pstart
           Mit verbindlichsten Empfehlungen{\\[\baselineskip]}hochachtungsvoll{\\[\baselineskip]}\spacefill\mbox{Elsa Ginsberg}\pend
           \leftskip=0em{}\selectlanguage{ngerman}\endnumbering\briefempfaengerindex{Schnitzler, Arthur@\textsc{Schnitzler, Arthur}!zzzPlessner, Elsa@\emph{von Elsa Plessner}!1916-01-131@{13. 1. 1916}|)be}\mylabel{L03731h}
\begin{anhang}
\end{anhang}\normalsize

\doendnotes{C}
\bigskip
\vfill

\clearpage

\footnotesize

\lohead{\textsc{register}}

% Definiere theindex-Environment komplett neu ohne reledmac
\makeatletter
\renewenvironment{theindex}{%
  \section*{\indexname}%
  \setlength{\parindent}{0pt}%
  \setlength{\parskip}{0pt plus 0.3pt}%
  \let\item\@idxitem
}{%
  \clearpage
}
\makeatother

\IfFileExists{\jobname-pw.ind}{\input{\jobname-pw.ind}}{}

\end{document}

      