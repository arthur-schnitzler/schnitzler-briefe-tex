%% latex-leseansicht-vorspann.tex
%% Vorspann für die Leseansicht.
%% Lädt die gemeinsame Datei latex-vorspann.tex mit nicht gesetztem Schalter.

\newif\ifkorrekturansicht
\korrekturansichtfalse

\input{../tex-inputs/latex-vorspann}


\section[Elsa Ginsberg-Plessner an Arthur Schnitzler, 13. 1. 1916]{L03731 Elsa Ginsberg-Plessner an Arthur Schnitzler, 13. 1. 1916}
\nopagebreak\mylabel{L03731v}
\rehead{ }\normalsize\beginnumbering\briefempfaengerindex{Schnitzler, Arthur@\textsc{Schnitzler, Arthur}!zzzPlessner, Elsa@\emph{von Elsa Plessner}!1916-01-131@{13. 1. 1916}|(be}
\toendnotes[C]{\smallbreak\pagebreak[2]}
\correspDesc{Versand  durch Elsa Plessner am 13. 1. 1916 in München
\newline{}Erhalt  durch Arthur Schnitzler im Zeitraum [14. 1. 1916
                  – 18. 1. 1916?] in Wien}\toendnotes[C]{\smallbreak}
\Standort{DLA, A:Schnitzler, HS.1985.1.419.}
\physDesc{Brief, 1 Blatt, 2 Seiten, 1170 Zeichen
\newline{}Handschrift: schwarze Tinte, lateinische Kurrent
\newline{}Schnitzler: 1) mit rotem Buntstift zwei Unterstreichungen  2) mit Bleistift »\textsc{Plessner}«}\toendnotes[C]{\smallbreak}
\pstart
           {\pb}München\oindex{München@\textbf{München}|pw}, den 13. Januar 1916\hfill Theresienstr. 78\oindex{Theresienstraße 78@\textbf{Theresienstraße 78}, \emph{Wohngebäude}|pw}\pend
           
\pstart{}Verehrter Herr Doctor!\pend\vspace{0.5em}
\pstart
           Soeben erhalte ich Ihr freundliches \label{K_L03731-1v}\edtext{Schreiben vom 12. d.}{\lemma{\textnormal{\emph{Schreiben vom 12. d.}}}\Cendnote{\textnormal{nicht überliefert}}}\label{K_L03731-1} und unterlasse
               es, Ihnen meine schmerzliche \label{K_L03731-2v}\edtext{Enttäuschung}{\lemma{\textnormal{\emph{Enttäuschung}}}\Cendnote{\textnormal{Schnitzler
                  scheint ihr geantwortet zu haben, dass er aus Prinzip keine unverlangt zugesandten Stücke im Manuskript lese (eventuell noch mit einer zusätzlichen Einschränkung). Plessner\pwindex{Plessner, Elsa 22.\,8.\,1875 Wien – 7.\,5.\,1932 Alicante@\textsc{Plessner, Elsa} (22.\,8.\,1875 Wien – 7.\,5.\,1932 Alicante), \emph{Schriftstellerin}|pwk}
                  kündigt im vorliegenden Brief an, ihr Stück\pwindex{Plessner, Elsa 22.\,8.\,1875 Wien – 7.\,5.\,1932 Alicante@\textsc{Plessner, Elsa} (22.\,8.\,1875 Wien – 7.\,5.\,1932 Alicante), \emph{Schriftstellerin}!Musik@\strich\emph{Musik}|pwkv} abholen zu lassen. Trotzdem 
                  las es Schnitzler{ }16. 1. 1916
                  und nannte es im \emph{Tagebuch}\pwindex{Schnitzler, Arthur 15. 5. 1862 Wien – 21. 10. 1931 ebd.@\textsc{Schnitzler, Arthur} (15. 5. 1862 Wien – 21. 10. 1931 ebd.), \emph{Schriftsteller, Mediziner}!Tagebuch@\strich\emph{Tagebuch}|pwk} ein »schlechtes Buch«. Diesen Befund
                  dürfte er Plessner\pwindex{Plessner, Elsa 22.\,8.\,1875 Wien – 7.\,5.\,1932 Alicante@\textsc{Plessner, Elsa} (22.\,8.\,1875 Wien – 7.\,5.\,1932 Alicante), \emph{Schriftstellerin}|pwk} nicht mitgeteilt haben.}}}\label{K_L03731-2} zu schildern: Indessen
               – gegen Principien ist nichts zu machen und man muss sie respectieren. In jedem Falle
               bitte ich Sie aber, aus dem mit I. bezeichneten \label{K_L03731-3v}\edtext{Umschlage den Brief
               herauszunehmen, der dem Manuscript\pwindex{Plessner, Elsa 22.\,8.\,1875 Wien – 7.\,5.\,1932 Alicante@\textsc{Plessner, Elsa} (22.\,8.\,1875 Wien – 7.\,5.\,1932 Alicante), \emph{Schriftstellerin}!Musik@\strich\emph{Musik}|pwv} beigelegt}{\lemma{\textnormal{\emph{Umschlage … beigelegt}}}\Cendnote{\textnormal{Es würde überraschen, wenn 
                  Plessner\pwindex{Plessner, Elsa 22.\,8.\,1875 Wien – 7.\,5.\,1932 Alicante@\textsc{Plessner, Elsa} (22.\,8.\,1875 Wien – 7.\,5.\,1932 Alicante), \emph{Schriftstellerin}|pwk} in den Brief vom XXXX Auszeichnungsfehler: Dokument L03730 nicht gefunden noch einen weiteren
                  Brief eingelegt hätte. Wenn es nur einen Brief gab, war sie mit ihrem Ansinnen einer separaten Retournierung erfolglos. Dieser befindet sich heute im Nachlass von Schnitzler.}}}\label{K_L03731-3} ist. Es wäre möglich, dass dieser Brief Sie bestimmen
               würde, eine Ausnahme zu machen mir gegenüber, die seit zwanzig Jahren eine Art
               Gewohnheitsrecht zu besitzen glaubt – wenn es auch in den letzten Jahren nicht zur
               Anwendung kam.\pend
           
\pstart
           Inliegend erlaube ich mir, eine ausgefüllte \label{K_L03731-4v}\edtext{Postkarte}{\lemma{\textnormal{\emph{Postkarte}}}\Cendnote{\textnormal{Diese dürfte mit der Adresse ihrer
                  Mutter Clementine Plessner\pwindex{Plessner, Clementine 7.\,12.\,1855 Wien – 27.\,2.\,1943 Konzentrationslager Theresienstadt@\textsc{Plessner, Clementine} (7.\,12.\,1855 Wien – 27.\,2.\,1943 Konzentrationslager Theresienstadt), \emph{Schauspielerin, Filmschauspielerin}|pwk} vorausgefüllt gewesen sein. Schnitzler musste also nur ausfüllen, wann diese das Manuskript von \emph{Musik}\pwindex{Plessner, Elsa 22.\,8.\,1875 Wien – 7.\,5.\,1932 Alicante@\textsc{Plessner, Elsa} (22.\,8.\,1875 Wien – 7.\,5.\,1932 Alicante), \emph{Schriftstellerin}!Musik@\strich\emph{Musik}|pwk} bei 
                  ihm abholen konnte.}}}\label{K_L03731-4} beizulegen, die Sie der Beförderung übergeben mögen, wie es
               Ihnen, verehrter Herr Doctor, angemessen erscheint. Meine Mama\pwindex{Plessner, Clementine 7.\,12.\,1855 Wien – 27.\,2.\,1943 Konzentrationslager Theresienstadt@\textsc{Plessner, Clementine} (7.\,12.\,1855 Wien – 27.\,2.\,1943 Konzentrationslager Theresienstadt), \emph{Schauspielerin, Filmschauspielerin}|pwv} ist gleichzeitig benachrichtigt, so
               dass das Manuscript\pwindex{Plessner, Elsa 22.\,8.\,1875 Wien – 7.\,5.\,1932 Alicante@\textsc{Plessner, Elsa} (22.\,8.\,1875 Wien – 7.\,5.\,1932 Alicante), \emph{Schriftstellerin}!Musik@\strich\emph{Musik}|pwv} sofort
               aus Ihrem Hause abgeholt werden kann. – –\pend
           
\pstart
           An eine Aufführung des »ersten Capitel\pwindex{Plessner, Elsa 22.\,8.\,1875 Wien – 7.\,5.\,1932 Alicante@\textsc{Plessner, Elsa} (22.\,8.\,1875 Wien – 7.\,5.\,1932 Alicante), \emph{Schriftstellerin}!erste Kapitel. Schauspiel in drei Akten@\strich\emph{Das erste Kapitel. Schauspiel in drei Akten}|pw}« denke
                  {\pb}ich vorläufig überhaupt nicht mehr. Man trägt ja
               nicht alte Kleider, wenn man neue hat. Daher ist die \label{K_L03731-5v}\edtext{causa Ziegel\pwindex{Ziegel, Erich 26.\,8.\,1876 Schwerin – 30.\,11.\,1950 München@\textsc{Ziegel, Erich} (26.\,8.\,1876 Schwerin – 30.\,11.\,1950 München), \emph{Theaterleiter, Regisseur, Schauspieler}|pw}}{\lemma{\textnormal{\emph{causa Ziegel}}}\Cendnote{\textnormal{Vgl. XXXX Auszeichnungsfehler: Dokument L03729 nicht gefunden. }}}\label{K_L03731-5}, obwohl noch
               in Schwebe – für mich bedeutungslos geworden.\pend
           
\pstart
           Mit verbindlichsten Empfehlungen{\\[\baselineskip]}hochachtungsvoll{\\[\baselineskip]}\spacefill\mbox{Elsa Ginsberg}\pend
           \leftskip=0em{}\selectlanguage{ngerman}\endnumbering\briefempfaengerindex{Schnitzler, Arthur@\textsc{Schnitzler, Arthur}!zzzPlessner, Elsa@\emph{von Elsa Plessner}!1916-01-131@{13. 1. 1916}|)be}\mylabel{L03731h}  \newcommand{\dateiname}{L03731}\newcommand{\titel}{Elsa Ginsberg-Plessner an Arthur Schnitzler, 13. 1. 1916}\newcommand{\editorInnen}{Selma Jahnke und Martin Anton Müller}%% latex-leseansicht-abspann.tex
%% Abspann für die Leseansicht.
%% Der Schalter \ifkorrekturansicht ist bereits durch den Vorspann gesetzt.

%% latex-abspann.tex
%% Gemeinsamer Abspann für Korrekturansicht und Leseansicht.
%% Setzt den Schalter \ifkorrekturansicht voraus (gesetzt in den
%% einbindenden Dateien latex-korrekturansicht-abspann.tex bzw.
%% latex-leseansicht-abspann.tex).
%% ---------------------------------------------------------------

\normalsize

% Das esempio-Environment wird nur in der Leseansicht benötigt
\ifkorrekturansicht\else
\newenvironment{esempio}[3]%
{
    \vspace{1.5ex}
    \rlap{\underline{#1}}
    \par
    \setlength{\parindent}{0cm}
    \nopagebreak
    \leftskip=#2cm
    \rightskip=#3cm
}
{
    \par
}
\fi

\doendnotes{C}
\bigskip
\vfill

\clearpage

\footnotesize

\ifkorrekturansicht
  \lohead{\textsc{register}}
\fi

% theindex-Environment neu definieren ohne reledmac
\makeatletter
\renewenvironment{theindex}{%
  \ifkorrekturansicht
    \section*{\indexname}%
  \else
    \subsubsection*{Index der erwähnten Entitäten}%
  \fi
  \setlength{\parindent}{0pt}%
  \setlength{\parskip}{0pt plus 0.3pt}%
  \let\item\@idxitem
}{%
  \ifkorrekturansicht\clearpage\fi
}
\makeatother

\IfFileExists{\jobname-pw.ind}{\input{\jobname-pw.ind}}{}

% Quellenangabe nur in der Leseansicht
\ifkorrekturansicht\else
% Fallback-Definitionen, falls die .tex-Datei \titel etc. nicht gesetzt hat
\providecommand{\titel}{}
\providecommand{\editorInnen}{}
\providecommand{\dateiname}{\jobname}

\vspace{3cm}

\vfill

\footnotesize
\textsc{Quelle}: \titel. Herausgegeben von {\editorInnen}. In: \emph{Arthur Schnitzler: Briefwechsel mit Autorinnen und Autoren}.
 Digitale Edition, https://schnitzler-briefe.acdh.oeaw.ac.at/{\dateiname}.html (Stand \today)
\fi

\end{document}


