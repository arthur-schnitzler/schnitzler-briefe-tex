%% latex-korrekturansicht-vorspann.tex
%% Vorspann für die Korrekturansicht.
%% Lädt die gemeinsame Datei latex-vorspann.tex mit gesetztem Schalter.

\newif\ifkorrekturansicht
\korrekturansichttrue

\input{../tex-inputs/latex-vorspann}


\section[Friedrich M. Fels an Arthur Schnitzler, 28. 9. 1895]{L00495 Friedrich M. Fels an Arthur Schnitzler, 28. 9. 1895}
\nopagebreak\mylabel{L00495v}
\rehead{ }\normalsize\beginnumbering\briefempfaengerindex{Schnitzler, Arthur@\textsc{Schnitzler, Arthur}!zzzFels, Friedrich Michael@\emph{von Friedrich Michael Fels}!1895-09-281@{28. 9. 1895}|(be}
\toendnotes[C]{\smallbreak\pagebreak[2]}\Standort{DLA, A:Schnitzler, HS.NZ85.1.2956.}
\physDesc{Brief, 1 Blatt, 1 Seite, 716 Zeichen
\newline{}Handschrift: schwarze Tinte, lateinische Kurrent
\newline{}Schnitzler: mit Bleistift nummeriert: »25« }\toendnotes[C]{\smallbreak}
\pstart
           \raggedleft{}{\pb}Zürich\oindex{Zuerich@\textbf{Zürich}, \emph{P.PPLA}|pw}, am 28. Sept. 1895\pend
           
\pstart\center{}Lieber Doktor Schnitzler!\pend\vspace{0.5em}
\pstart
           Brief und Karte habe ich erhalten; meinen besten Dank für die Einlage, ich ko{\geminationn}te das Geld wirklich nötig brauchen. Aber nicht wahr?
               Sie sind so freundlich, sich in der Angelegenheit noch einmal an die anderen zu
               wenden; de{\geminationn} we{\geminationn} ich nicht
                  \introOben{}schleunigst\introOben{} noch etwas beko{\geminationm}e, ka{\geminationn} ich die Kiste nicht ordnen. Adreſse\oindex{Raemistrasse@\textbf{Rämistrasse}, \emph{Straße (K.STR)}|pwv} i{\geminationm}er
               noch: Bettauer\pwindex{Bettauer, Hugo 18.08.1872 – 26.03.1925@\textsc{Bettauer, Hugo} (18.08.1872 – 26.03.1925), \emph{Schriftsteller/Schriftstellerin, Journalist/Journalistin}|pw}.\pend
           
\pstart
           Verzeihen Sie, lieber Doktor, daſs ich Ihnen so viele Mühe mache; ich rechne in
               wirklich unverantwortlicher Weise mit Ihrer Gutmütigkeit und Freundlichkeit. Aber Sie
               wiſsen, we{\geminationn} man keinen andern Ausweg hat{\dots}\pend
           
\pstart
           Bei mit steht noch alles beim Alten. Ihnen gehts hoffentlich gut. Sie werden ja an
               der Burg\orgindex{Burgtheater@Burgtheater|pw} bald dranko{\geminationm}en\pwindex{Liebelei. Schauspiel in drei Akten@\emph{Liebelei. Schauspiel in drei Akten}|pwv}.\pend
           
\pstart
           Herzlichst{\\[\baselineskip]}Ihr{\\[\baselineskip]}dankbar ergebener{\\[\baselineskip]}\spacefill\mbox{Fels}\pend
           \leftskip=0em{}\selectlanguage{ngerman}\endnumbering\briefempfaengerindex{Schnitzler, Arthur@\textsc{Schnitzler, Arthur}!zzzFels, Friedrich Michael@\emph{von Friedrich Michael Fels}!1895-09-281@{28. 9. 1895}|)be}\mylabel{L00495h}  \normalsize

\doendnotes{C}
\bigskip
\vfill

\clearpage

\footnotesize

\lohead{\textsc{register}}

% Definiere theindex-Environment komplett neu ohne reledmac
\makeatletter
\renewenvironment{theindex}{%
  \section*{\indexname}%
  \setlength{\parindent}{0pt}%
  \setlength{\parskip}{0pt plus 0.3pt}%
  \let\item\@idxitem
}{%
  \clearpage
}
\makeatother

\IfFileExists{\jobname-pw.ind}{\input{\jobname-pw.ind}}{}

\end{document}

      