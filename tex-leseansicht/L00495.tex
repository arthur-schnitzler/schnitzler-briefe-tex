%% latex-leseansicht-vorspann.tex
%% Vorspann für die Leseansicht.
%% Lädt die gemeinsame Datei latex-vorspann.tex mit nicht gesetztem Schalter.

\newif\ifkorrekturansicht
\korrekturansichtfalse

\input{../tex-inputs/latex-vorspann}


\section[Friedrich M. Fels an Arthur Schnitzler, 28. 9. 1895]{L00495 Friedrich M. Fels an Arthur Schnitzler, 28. 9. 1895}
\nopagebreak\mylabel{L00495v}
\rehead{ }\normalsize\beginnumbering\briefempfaengerindex{Schnitzler, Arthur@\textsc{Schnitzler, Arthur}!zzzFels, Friedrich Michael@\emph{von Friedrich Michael Fels}!1895-09-281@{28. 9. 1895}|(be}
\toendnotes[C]{\smallbreak\pagebreak[2]}
\correspDesc{Versand  durch Friedrich M. Fels am 28. 9. 1895 in Zürich
\newline{}Erhalt  durch Arthur Schnitzler im Zeitraum [29. 9. 1895
                  – 3. 10. 1895?] in Wien}\toendnotes[C]{\smallbreak}
\Standort{DLA, A:Schnitzler, HS.NZ85.1.2956.}
\physDesc{Brief, 1 Blatt, 1 Seite, 716 Zeichen
\newline{}Handschrift: schwarze Tinte, lateinische Kurrent
\newline{}Schnitzler: mit Bleistift nummeriert: »25« }\toendnotes[C]{\smallbreak}
\pstart
           \raggedleft{}{\pb}Zürich\oindex{Zürich@\textbf{Zürich}|pw}, am 28. Sept. 1895\pend
           
\pstart\center{}Lieber Doktor Schnitzler!\pend\vspace{0.5em}
\pstart
           Brief und Karte habe ich erhalten; meinen besten Dank für die Einlage, ich ko{\geminationn}te das Geld wirklich nötig brauchen. Aber nicht wahr?
               Sie sind so freundlich, sich in der Angelegenheit noch einmal an die anderen zu
               wenden; de{\geminationn} we{\geminationn} ich nicht
                  \introOben{}schleunigst\introOben{} noch etwas beko{\geminationm}e, ka{\geminationn} ich die Kiste nicht ordnen. Adreſse\oindex{Rämistrasse@\textbf{Rämistrasse}, \emph{Straße}|pwv} i{\geminationm}er
               noch: Bettauer\pwindex{Bettauer, Hugo 18.\,8.\,1872 Baden bei Wien – 26.\,3.\,1925 Wien@\textsc{Bettauer, Hugo} (18.\,8.\,1872 Baden bei Wien – 26.\,3.\,1925 Wien), \emph{Schriftsteller, Journalist}|pw}.\pend
           
\pstart
           Verzeihen Sie, lieber Doktor, daſs ich Ihnen so viele Mühe mache; ich rechne in
               wirklich unverantwortlicher Weise mit Ihrer Gutmütigkeit und Freundlichkeit. Aber Sie
               wiſsen, we{\geminationn} man keinen andern Ausweg hat{\dots}\pend
           
\pstart
           Bei mit steht noch alles beim Alten. Ihnen gehts hoffentlich gut. Sie werden ja an
               der Burg\orgindex{Burgtheater@Burgtheater|pw} bald dranko{\geminationm}en\pwindex{Schnitzler, Arthur 15.\,5.\,1862 Wien – 21.\,10.\,1931 ebd.@\textsc{Schnitzler, Arthur} (15.\,5.\,1862 Wien – 21.\,10.\,1931 ebd.), \emph{Schriftsteller, Mediziner}!Liebelei. Schauspiel in drei Akten@\strich\emph{Liebelei. Schauspiel in drei Akten}|pwv}.\pend
           
\pstart
           Herzlichst{\\[\baselineskip]}Ihr{\\[\baselineskip]}dankbar ergebener{\\[\baselineskip]}\spacefill\mbox{Fels}\pend
           \leftskip=0em{}\selectlanguage{ngerman}\endnumbering\briefempfaengerindex{Schnitzler, Arthur@\textsc{Schnitzler, Arthur}!zzzFels, Friedrich Michael@\emph{von Friedrich Michael Fels}!1895-09-281@{28. 9. 1895}|)be}\mylabel{L00495h}  \newcommand{\dateiname}{L00495}\newcommand{\titel}{Friedrich M. Fels an Arthur Schnitzler, 28. 9. 1895}\newcommand{\editorInnen}{Martin Anton Müller und Gerd-Hermann Susen}%% latex-leseansicht-abspann.tex
%% Abspann für die Leseansicht.
%% Der Schalter \ifkorrekturansicht ist bereits durch den Vorspann gesetzt.

%% latex-abspann.tex
%% Gemeinsamer Abspann für Korrekturansicht und Leseansicht.
%% Setzt den Schalter \ifkorrekturansicht voraus (gesetzt in den
%% einbindenden Dateien latex-korrekturansicht-abspann.tex bzw.
%% latex-leseansicht-abspann.tex).
%% ---------------------------------------------------------------

\normalsize

% Das esempio-Environment wird nur in der Leseansicht benötigt
\ifkorrekturansicht\else
\newenvironment{esempio}[3]%
{
    \vspace{1.5ex}
    \rlap{\underline{#1}}
    \par
    \setlength{\parindent}{0cm}
    \nopagebreak
    \leftskip=#2cm
    \rightskip=#3cm
}
{
    \par
}
\fi

\doendnotes{C}
\bigskip
\vfill

\clearpage

\footnotesize

\ifkorrekturansicht
  \lohead{\textsc{register}}
\fi

% theindex-Environment neu definieren ohne reledmac
\makeatletter
\renewenvironment{theindex}{%
  \ifkorrekturansicht
    \section*{\indexname}%
  \else
    \subsubsection*{Index der erwähnten Entitäten}%
  \fi
  \setlength{\parindent}{0pt}%
  \setlength{\parskip}{0pt plus 0.3pt}%
  \let\item\@idxitem
}{%
  \ifkorrekturansicht\clearpage\fi
}
\makeatother

\IfFileExists{\jobname-pw.ind}{\input{\jobname-pw.ind}}{}

% Quellenangabe nur in der Leseansicht
\ifkorrekturansicht\else
% Fallback-Definitionen, falls die .tex-Datei \titel etc. nicht gesetzt hat
\providecommand{\titel}{}
\providecommand{\editorInnen}{}
\providecommand{\dateiname}{\jobname}

\vspace{3cm}

\vfill

\footnotesize
\textsc{Quelle}: \titel. Herausgegeben von {\editorInnen}. In: \emph{Arthur Schnitzler: Briefwechsel mit Autorinnen und Autoren}.
 Digitale Edition, https://schnitzler-briefe.acdh.oeaw.ac.at/{\dateiname}.html (Stand \today)
\fi

\end{document}


