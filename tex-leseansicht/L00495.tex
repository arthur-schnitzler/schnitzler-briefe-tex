\input{../tex-inputs/latex-pdf-vorspann}
\begin{center}
            \textcolor{red}{ENTWURF. ENTZIFFERUNG NOCH NICHT KORREKTURGELESEN}
                      \end{center}
            
               \section[Friedrich M. Fels an Arthur Schnitzler, 28. 9. 1895]{ Friedrich M. Fels an Arthur Schnitzler, 28. 9. 1895}\nopagebreak\mylabel{v}\rehead{ }\begin{ledgroupsized}[t]{13cm}\normalsize\beginnumbering\briefempfaengerindex{Schnitzler, Arthur@\textsc{Schnitzler, Arthur}!zzzFels, Friedrich Michael@\emph{von Friedrich Michael Fels}!1895-09-281@{28. 9. 1895}|(be} \toendnotes[C]{\smallbreak\pagebreak[2]} \Standort{DLA, A:Schnitzler, HS.NZ85.1.2956.}
\physDesc{Brief, 1 Blatt, 1 Seite
\newline{}Handschrift: schwarze Tinte, lateinische Kurrent
\newline{}Schnitzler: mit Bleistift nummeriert: »25« }\toendnotes[C]{\smallbreak}\pstart
           \raggedleft{}{\pb}Zürich\oindex{Zuerich@\textbf{Zürich}|pw}, am 28. Sept. 1895\pend
           \pstart\center{}Lieber Doktor Schnitzler!\pend\pstart
           Brief und Karte habe ich erhalten; meinen besten Dank für die Einlage, ich ko{\geminationn}te das Geld wirklich nötig brauchen. Aber nicht
                    wahr? Sie sind so freundlich, sich in der Angelegenheit noch einmal an die
                    anderen zu wenden; de{\geminationn} we{\geminationn} ich nicht \introOben{}schleunigst\introOben{}
                    noch etwas beko{\geminationm}e, ka{\geminationn}
                    ich die Kiste nicht ordnen. Adreſse\oindex{Raemistrasse@\textbf{Rämistrasse}|pwv} i{\geminationm}er noch: Bettauer\pwindex{Bettauer, Hugo 18.08.1872 – 26.03.1925@\textsc{Bettauer, Hugo} (18.08.1872 – 26.03.1925), \emph{Schriftsteller, Journalist}|pw}.\pend
           \pstart
           Verzeihen Sie, lieber Doktor, daſs ich Ihnen so viele Mühe mache; ich rechne in
                    wirklich unverantwortlicher Weise mit Ihrer Gutmütigkeit und Freundlichkeit.
                    Aber Sie wiſsen, we{\geminationn} man keinen andern Ausweg hat{\dots}\pend
           \pstart
           Bei mit steht noch alles beim Alten. Ihnen gehts hoffentlich gut. Sie werden ja
                    an der Burg\orgindex{Burgtheater@Burgtheater|pw} bald dranko{\geminationm}en\pwindex{Schnitzler, Arthur 15.05.1862 – 21.10.1931@\textsc{Schnitzler, Arthur} (15.05.1862 – 21.10.1931), \emph{Schriftsteller, Mediziner}!Liebelei. Schauspiel in drei Akten9. 10. 1895@\strich\emph{Liebelei. Schauspiel in drei Akten} {[}9. 10. 1895{]}|pwv}.\pend
           \pstart
           Herzlichst{\\[\baselineskip]}Ihr{\\[\baselineskip]}dankbar ergebener{\\[\baselineskip]}\spacefill\mbox{Fels}\pend
           \leftskip=0em{}\endnumbering\briefempfaengerindex{Schnitzler, Arthur@\textsc{Schnitzler, Arthur}!zzzFels, Friedrich Michael@\emph{von Friedrich Michael Fels}!1895-09-281@{28. 9. 1895}|)be}\mylabel{h}\end{ledgroupsized}  \newcommand{\dateiname}{L00495}\newcommand{\titel}{Friedrich M. Fels an Arthur Schnitzler, 28. 9. 1895}\newcommand{\editorInnen}{Martin Anton Müller und Gerd-Hermann Susen}\input{../tex-inputs/latex-pdf-abspann}
      