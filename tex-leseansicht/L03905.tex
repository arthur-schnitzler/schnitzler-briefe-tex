%% latex-leseansicht-vorspann.tex
%% Vorspann für die Leseansicht.
%% Lädt die gemeinsame Datei latex-vorspann.tex mit nicht gesetztem Schalter.

\newif\ifkorrekturansicht
\korrekturansichtfalse

\input{../tex-inputs/latex-vorspann}


\section[Arthur Schnitzler an Theodor Herzl, 19. 9. 1893]{L03905 Arthur Schnitzler an Theodor Herzl, 19. 9. 1893}
\nopagebreak\mylabel{L03905v}
\rehead{ }\normalsize\beginnumbering\briefempfaengerindex{Herzl, Theodor@\textsc{Herzl, Theodor}!zzzGoldmann, Paul@\emph{von Paul Goldmann}!1893-09-191@{19. 9. 1893}|(be}\briefempfaengerindex{Herzl, Theodor@\textsc{Herzl, Theodor}!zzzSchnitzler, Arthur@\emph{von Arthur Schnitzler}!1893-09-191@{19. 9. 1893}|(be}
\toendnotes[C]{\smallbreak\pagebreak[2]}
\correspDesc{Versand  durch Arthur Schnitzler, Paul Goldmann am 19. 9. 1893 in Salzburg
\newline{}Erhalt  durch Theodor Herzl in Baden bei Wien}\toendnotes[C]{\smallbreak}
\Standort{Jerusalem, Central Zionist Archives, H1:1926-9.}
\physDesc{,  Blätter,  Seiten
\newline{}Handschrift: , deutsche Kurrent}\toendnotes[C]{\smallbreak}\pstart{}{\pb}\textsc{Herrn Doctor Theodor Herzl}\pend{}\pstart{}\textsc{Baden bei} Wien\oindex{Baden bei Wien@\textbf{Baden bei Wien}, \emph{Hauptstadt}|pw}\pend{}\pstart{}\textsc{Helenenstraße 71\oindex{Helenenstraße 71@\textbf{Helenenstraße 71}, \emph{Wohngebäude}|pw}.}\pend{}{\bigskip}\vspace{1em}
\pstart
           {\pb}Salzburg\oindex{Salzburg@\textbf{Salzburg}, \emph{Verwaltungsgebiet}|pw} den 19. September\pend
           
\pstart{}Mein verehrtes Freund,\pend\vspace{0.5em}
\pstart
           ich will Sie noch, bevor ich die Freude habe, Sie in Baden\oindex{Baden bei Wien@\textbf{Baden bei Wien}, \emph{Hauptstadt}|pw} zu{ }ſprechen, von hier aus aufs allerherzlichſte grüßen. Sie errathen
               bereits, daſs ich hier mit Paul Goldmann\pwindex{Goldmann, Paul 31.\,1.\,1865 Breslau – 25.\,9.\,1935 Wien@\textsc{Goldmann, Paul} (31.\,1.\,1865 Breslau – 25.\,9.\,1935 Wien), \emph{Schriftsteller, Journalist}|pw} zusa{\geminationm}en
               getroffen bin, und zweifeln auch gewiſs nicht daran, daſs es vom Augen{\pb}blick meiner Ankunft an bis heute morgen, – Tag der
               Abreiſe – ununterbrochen geregnet hat. So konnten wir kaum aus der Stadt heraus, und
               haben in Kaffehäuſern und in den Straßen die letzten 2 oder 3 Jahre durchgeplaudert.
               Es war{ }ſehr{ }ſchön. Geſtern hab ich auch, die Intimität benützend, über Pauls Schulter
               weg, Ihren \label{K_L03905-1v}\edtext{Brief an ihn}{\lemma{\textnormal{\emph{Brief an ihn}}}\Cendnote{\textnormal{nicht überliefert}}}\label{K_L03905-1} gelesen – {\pb}und ka{\geminationn} nun nicht weiterſchreiben, weil man\footnote{\noindent{}Man = Paul Goldmann} neben mir ununterbrochen
               das \textsc{Intermezzo\pwindex{\textcolor{red}{\textsuperscript{XXXX indx1}}!Cavalleria rusticana. Oper in einem Aufzuge@\strich\emph{Cavalleria rusticana. Oper in einem Aufzuge}|pw}\pwindex{\textcolor{red}{\textsuperscript{XXXX indx1}}!Cavalleria rusticana. Oper in einem Aufzuge@\strich\emph{Cavalleria rusticana. Oper in einem Aufzuge}|pw}\pwindex{\textcolor{red}{\textsuperscript{XXXX indx1}}!Cavalleria rusticana. Oper in einem Aufzuge@\strich\emph{Cavalleria rusticana. Oper in einem Aufzuge}|pw}} pfeift und dreinredet.\pend
           
\pstart
           Auf baldiges Wiederſehen{\\[\baselineskip]}herzlich Ihr \spacefill\mbox{ArthSchnitzler}\pend
           \leftskip=0em{}\selectlanguage{ngerman}\vspace{1em}
\pstart{}{\pb}{[}hs. Goldmann:{]} Lieber
                  Freund!\pend\vspace{0.5em}
\pstart
           Ich danke Ihnen für Ihren{ }ſo{ }ſehr{ }ſchönen Brief, auf den ich heute nicht antworte,
               weil ich heut zu dumm bin. Dies{ }ſoll nur eine Empfangsbeſtätigung und ein Gruß{ }ſein.
               Arthur Schnitzler hat das Verſprechen, Ihnen keinen geiſtreichen Brief zu{ }ſchreiben,
               in niedrigſter Weiſe gebrochen. Ich kann alſo erſt recht nicht{ }ſchreiben, weil ich
               nicht abſtechen mag. Glückliche Heimreiſe alſo! Geben Sie mir einen \textsc{\textcolor{gray}{conſt} de}{ }\textsc{téléphone} bei der Ankunft.\pend
           
\pstart
           Herzlichſt Ihr{\\[\baselineskip]}\spacefill\mbox{Paul Goldmann}\pend
           \leftskip=0em{}\selectlanguage{ngerman}\endnumbering\briefempfaengerindex{Herzl, Theodor@\textsc{Herzl, Theodor}!zzzGoldmann, Paul@\emph{von Paul Goldmann}!1893-09-191@{19. 9. 1893}|)be}\briefempfaengerindex{Herzl, Theodor@\textsc{Herzl, Theodor}!zzzSchnitzler, Arthur@\emph{von Arthur Schnitzler}!1893-09-191@{19. 9. 1893}|)be}\mylabel{L03905h}
\begin{anhang}
\end{anhang}\newcommand{\dateiname}{L03905}\newcommand{\titel}{Arthur Schnitzler an Theodor Herzl, 19. 9. 1893}\newcommand{\editorInnen}{Herausgegeben von Jahnke, SelmaMüller, Martin Anton}%% latex-leseansicht-abspann.tex
%% Abspann für die Leseansicht.
%% Der Schalter \ifkorrekturansicht ist bereits durch den Vorspann gesetzt.

%% latex-abspann.tex
%% Gemeinsamer Abspann für Korrekturansicht und Leseansicht.
%% Setzt den Schalter \ifkorrekturansicht voraus (gesetzt in den
%% einbindenden Dateien latex-korrekturansicht-abspann.tex bzw.
%% latex-leseansicht-abspann.tex).
%% ---------------------------------------------------------------

\normalsize

% Das esempio-Environment wird nur in der Leseansicht benötigt
\ifkorrekturansicht\else
\newenvironment{esempio}[3]%
{
    \vspace{1.5ex}
    \rlap{\underline{#1}}
    \par
    \setlength{\parindent}{0cm}
    \nopagebreak
    \leftskip=#2cm
    \rightskip=#3cm
}
{
    \par
}
\fi

\doendnotes{C}
\bigskip
\vfill

\clearpage

\footnotesize

\ifkorrekturansicht
  \lohead{\textsc{register}}
\fi

% theindex-Environment neu definieren ohne reledmac
\makeatletter
\renewenvironment{theindex}{%
  \ifkorrekturansicht
    \section*{\indexname}%
  \else
    \subsubsection*{Index der erwähnten Entitäten}%
  \fi
  \setlength{\parindent}{0pt}%
  \setlength{\parskip}{0pt plus 0.3pt}%
  \let\item\@idxitem
}{%
  \ifkorrekturansicht\clearpage\fi
}
\makeatother

\IfFileExists{\jobname-pw.ind}{\input{\jobname-pw.ind}}{}

% Quellenangabe nur in der Leseansicht
\ifkorrekturansicht\else
% Fallback-Definitionen, falls die .tex-Datei \titel etc. nicht gesetzt hat
\providecommand{\titel}{}
\providecommand{\editorInnen}{}
\providecommand{\dateiname}{\jobname}

\vspace{3cm}

\vfill

\footnotesize
\textsc{Quelle}: \titel. Herausgegeben von {\editorInnen}. In: \emph{Arthur Schnitzler: Briefwechsel mit Autorinnen und Autoren}.
 Digitale Edition, https://schnitzler-briefe.acdh.oeaw.ac.at/{\dateiname}.html (Stand \today)
\fi

\end{document}


