%% latex-korrekturansicht-vorspann.tex
%% Vorspann für die Korrekturansicht.
%% Lädt die gemeinsame Datei latex-vorspann.tex mit gesetztem Schalter.

\newif\ifkorrekturansicht
\korrekturansichttrue

\input{../tex-inputs/latex-vorspann}


\section[Arthur Schnitzler an Hugo Hofmannsthal, 23. 12. 1924]{L02424 Arthur Schnitzler an Hugo Hofmannsthal, 23. 12. 1924}
\nopagebreak\mylabel{L02424v}
\rehead{ }\normalsize\beginnumbering\briefempfaengerindex{Hofmannsthal, Hugo von@\textsc{Hofmannsthal, Hugo von}!zzzSchnitzler, Arthur@\emph{von Arthur Schnitzler}!1924-12-231@{23. 12. 1924}|(be}
\toendnotes[C]{\smallbreak\pagebreak[2]}\Standort{FDH, Hs-30885,152.}
\physDesc{Postkarte, 389 Zeichen
\newline{}Handschrift: Bleistift, lateinische Kurrent
\newline{}Versand: Stempel: »\nobreak{}\textcolor{gray}{Wien}, 23 XII 24, 20\nobreak{}«.  }
\buchAbdrucke{\weitereDrucke{Hugo von Hofmannsthal, Arthur Schnitzler: \emph{Briefwechsel}. Frankfurt am Main: \emph{S. Fischer} 1964, S. 300.} }\toendnotes[C]{\smallbreak}\pstart{}{\pb}\label{T_L02424-1v}\edtext{\textcolor{gray}{\textbf{A. S.}}}{\lemma{\textnormal{\emph{A. S.}}}\Cendnote{\textnormal{ovaler Absenderkleber}}}\label{T_L02424-1}\pend{}\pstart{}\textcolor{gray}{\textbf{WIEN, XVIII.}}\oindex{XVIII., Waehring@\textbf{XVIII., Währing}, \emph{A.ADM3}|pw}\pend{}\pstart{}\textcolor{gray}{\textbf{STERNWARTESTR. 71}}\oindex{Sternwartestrasse 71@\textbf{Sternwartestraße 71}, \emph{Wohngebäude (K.WHS)}|pw}\pend{}{\bigskip}\pstart{}Hrn Dr.\pend{}\pstart{}Hugo v. Hofma{\geminationn}sthal\pend{}\pstart{}Rodaun\oindex{Rodaun@\textbf{Rodaun}, \emph{A.ADM4}|pw}\pend{}\pstart{}Badgasse\oindex{Badgasse@\textbf{Badgasse}, \emph{Straße (K.STR)}|pw}.\pend{}{\bigskip}\vspace{1em}
\pstart
           \raggedleft{}{\pb}Wien\oindex{Wien@\textbf{Wien}, \emph{A.ADM2}|pw}{ }23. 12. 24\pend
           \vspace{0.5em}
\pstart
           mein lieber Hugo, wollen Sie am \label{K_L02424-1v}\edtext{31., Mittwoch}{\lemma{\textnormal{\emph{31., Mittwoch}}}\Cendnote{\textnormal{Vgl. A. S.: \emph{Tagebuch}, 31. 12. 1924.
               }}}\label{K_L02424-1}, letztem Tag d\damage{es} Monats u Jahres bei mir zu Mittag essen, und vorher, je nach Laune u
               Wetter, im freien oder geschlossenen Raume »plaudern«, \substVorne{}\textsuperscript{– }\substDazwischen{}(\substHinten{}da ich doch beka{\geminationn}tlich ein solcher bin –?) Ich
               würd {\pb}Sie circa ½ 12 erwarten. Schreiben Sie
               mir eine Zeile, ob ja. Herzlichst wie allezeit der Ihre\pend
           \pstart \spacefill\mbox{Arthur}\pend{}\selectlanguage{ngerman}\endnumbering\briefempfaengerindex{Hofmannsthal, Hugo von@\textsc{Hofmannsthal, Hugo von}!zzzSchnitzler, Arthur@\emph{von Arthur Schnitzler}!1924-12-231@{23. 12. 1924}|)be}\mylabel{L02424h}  \normalsize

\doendnotes{C}
\bigskip
\vfill

\clearpage

\footnotesize

\lohead{\textsc{register}}

% Definiere theindex-Environment komplett neu ohne reledmac
\makeatletter
\renewenvironment{theindex}{%
  \section*{\indexname}%
  \setlength{\parindent}{0pt}%
  \setlength{\parskip}{0pt plus 0.3pt}%
  \let\item\@idxitem
}{%
  \clearpage
}
\makeatother

\IfFileExists{\jobname-pw.ind}{\input{\jobname-pw.ind}}{}

\end{document}

      