%% latex-leseansicht-vorspann.tex
%% Vorspann für die Leseansicht.
%% Lädt die gemeinsame Datei latex-vorspann.tex mit nicht gesetztem Schalter.

\newif\ifkorrekturansicht
\korrekturansichtfalse

\input{../tex-inputs/latex-vorspann}


\section[ Paul Goldmann an Arthur Schnitzler, 4. 6. {[}1896{]}]{L02776 Paul Goldmann an Arthur Schnitzler,  4. 6. [1896]}
\nopagebreak\mylabel{L02776v}
\rehead{ }\normalsize\beginnumbering\briefempfaengerindex{Schnitzler, Arthur@\textsc{Schnitzler, Arthur}!zzzGoldmann, Paul@\emph{von Paul Goldmann}!1896-06-042@{4. 6. [1896]}|(be}
\toendnotes[C]{\smallbreak\pagebreak[2]}
\correspDesc{Versand  durch Paul Goldmann am 4. 6. [1896] in Paris
\newline{}Erhalt  durch Arthur Schnitzler im Zeitraum [5. 6. 1896
                  – 9. 6. 1896?] in Wien}\toendnotes[C]{\smallbreak}
\Standort{DLA, A:Schnitzler, HS.NZ85.1.3166.}
\physDesc{Brief, 2 Blätter, 8 Seiten, 2377 Zeichen
\newline{}Handschrift: blaue Tinte, deutsche Kurrent
\newline{}Schnitzler: 1) mit Bleistift das Jahr »96« vermerkt  2) mit rotem Buntstift vier Unterstreichungen}\toendnotes[C]{\smallbreak}
\pstart
           {\pb}\textcolor{gray}{\textbf{\textbf{Frankfurter Zeitung\orgindex{Frankfurter Zeitung@Frankfurter Zeitung|pw}}}}\pend
           
\pstart
           \textcolor{gray}{\textbf{(\begin{otherlanguage}{french}Gazette de Francfort\end{otherlanguage}\orgindex{Frankfurter Zeitung@Frankfurter Zeitung|pw}).}}\pend
           
\pstart
           \textcolor{gray}{\textbf{\textbf{\begin{otherlanguage}{french}Fondateur M.\end{otherlanguage}{ }L. Sonnemann\pwindex{Sonnemann, Leopold 29.\,10.\,1831 Höchberg – 30.\,10.\,1909 Frankfurt am Main@\textsc{Sonnemann, Leopold} (29.\,10.\,1831 Höchberg – 30.\,10.\,1909 Frankfurt am Main), \emph{Journalist, Herausgeber}|pw}.}}}\pend
           
\pstart
           \begin{otherlanguage}{french}\textcolor{gray}{\textbf{Journal\pwindex{Frankfurter Zeitung@\emph{Frankfurter Zeitung}|pwv} politique,
                        financier,}}\end{otherlanguage}\pend
           
\pstart
           \begin{otherlanguage}{french}\textcolor{gray}{\textbf{commercial et littéraire.}}\end{otherlanguage}\pend
           
\pstart
           \begin{otherlanguage}{french}\textcolor{gray}{\textbf{\textbf{Paraissant trois fois par jour.}}}\end{otherlanguage}\pend
           
\pstart
           \begin{otherlanguage}{french}\textcolor{gray}{\textbf{\textbf{Bureau à Paris\oindex{Paris@\textbf{Paris}, \emph{Hauptstadt}|pw}}}}\end{otherlanguage}\hfill \textsc{Paris\oindex{Paris@\textbf{Paris}, \emph{Hauptstadt}|pw}}, 4. Juni.\pend
           
\pstart
           \begin{otherlanguage}{french}\textcolor{gray}{\textbf{\textbf{24. Rue Feydeau\oindex{rue Feydeau@\textbf{rue Feydeau}, \emph{Straße}|pw}.}}}\end{otherlanguage}\pend
           
\pstart\center{}Mein lieber Freund,\pend\vspace{0.5em}
\pstart
           In Eile nur ein Wort des Dankes für Deinen lieben Brief!\pend
           
\pstart
           So iſt es alſo abgemacht: Ich komme nach \label{K_L02776-1v}\edtext{Dänemark\oindex{Dänemark@\textbf{Dänemark}|pw}}{\lemma{\textnormal{\emph{Dänemark}}}\Cendnote{\textnormal{Siehe XXXX Auszeichnungsfehler: Dokument L02772 nicht gefunden.
               }}}\label{K_L02776-1}, – immer unter \strikeout{Vor} der Vorausſetzung, daß die
               weite Reiſe nicht über meine Mittel geht. Kannſt Du mir mittheilen, was man ungefähr
               pro Tag in \textsc{Scottsborg\oindex{Skodsborg@\textbf{Skodsborg}|pwv}} braucht? {\pb}Ich freue mich unendlich darauf,
               Dich wiederzusehen. Du wirſt mir wohl noch weitere Details angeben. Wann reiſt \textsc{Richard\pwindex{Beer-Hofmann, Richard 11.\,7.\,1866 Wien – 26.\,9.\,1945 New York City@\textsc{Beer-Hofmann, Richard} (11.\,7.\,1866 Wien – 26.\,9.\,1945 New York City), \emph{Schriftsteller}|pw}}? Zurück will ich dann über \label{K_L02776-2v}\edtext{Berlin\oindex{Berlin@\textbf{Berlin}, \emph{Hauptstadt}|pw}}{\lemma{\textnormal{\emph{Berlin}}}\Cendnote{\textnormal{Siehe A. S.: \emph{Tagebuch}, 26. 8. 1896.
               }}}\label{K_L02776-2} gehen.\pend
           
\pstart
           Die \label{K_L02776-3v}\edtext{Ernennung von \textsc{Antoine\pwindex{Antoine, André 31.\,1.\,1858 Limoges – 23.\,10.\,1943 Le Pouliguen@\textsc{Antoine, André} (31.\,1.\,1858 Limoges – 23.\,10.\,1943 Le Pouliguen), \emph{Theaterleiter, Schauspieler}|pw}} zum Director des \textsc{Odéon\orgindex{Odéon@Odéon|pw}}}{\lemma{\textnormal{\emph{Ernennung … Odéon}}}\Cendnote{\textnormal{André Antoine\pwindex{Antoine, André 31.\,1.\,1858 Limoges – 23.\,10.\,1943 Le Pouliguen@\textsc{Antoine, André} (31.\,1.\,1858 Limoges – 23.\,10.\,1943 Le Pouliguen), \emph{Theaterleiter, Schauspieler}|pwk} wurde 1896 neben Paul Ginisty\pwindex{Ginisty, Paul 4.\,4.\,1855 Paris – 5.\,3.\,1932 ebd.@\textsc{Ginisty, Paul} (4.\,4.\,1855 Paris – 5.\,3.\,1932 ebd.), \emph{Schriftsteller, Theaterleiter}|pwk} zum Ko-Direktor\pwindex{Antoine, André 31.\,1.\,1858 Limoges – 23.\,10.\,1943 Le Pouliguen@\textsc{Antoine, André} (31.\,1.\,1858 Limoges – 23.\,10.\,1943 Le Pouliguen), \emph{Theaterleiter, Schauspieler}|pwkv} des \emph{Odéon}\orgindex{Odéon@Odéon|pwk} ernannt.}}}\label{K_L02776-3} eröffnet uns eine
               unverhoffte Ausſicht, Dein Stück\pwindex{Schnitzler, Arthur 15.\,5.\,1862 Wien – 21.\,10.\,1931 ebd.@\textsc{Schnitzler, Arthur} (15.\,5.\,1862 Wien – 21.\,10.\,1931 ebd.), \emph{Schriftsteller, Mediziner}!Liebelei. Schauspiel in drei Akten@\strich\emph{Liebelei. Schauspiel in drei Akten}|pwv} doch noch hier auf ein großes Theater zu bringen. Nächſtens mehr
               darüber.\pend
           
\pstart
           {\pb}\textsc{M. Christian Schefer\pwindex{Schefer, Christian 14.\,7.\,1866 Paris – Februar 1944 Marokko@\textsc{Schefer, Christian} (14.\,7.\,1866 Paris – Februar 1944 Marokko), \emph{Journalist, Lehrer}|pw}} beſuchte mich dieſer Tage u.{ }ſagte mir, er habe einen \label{K_L02776-4v}\edtext{Artikel\pwindex{Schefer, Christian 14.\,7.\,1866 Paris – Februar 1944 Marokko@\textsc{Schefer, Christian} (14.\,7.\,1866 Paris – Februar 1944 Marokko), \emph{Journalist, Lehrer}!Un jeune écrivain viennois: M. Arthur Schnitzler@\strich\emph{Un jeune écrivain viennois: M. Arthur Schnitzler}|pwv}}{\lemma{\textnormal{\emph{Artikel}}}\Cendnote{\textnormal{Christian Schefer\pwindex{Schefer, Christian 14.\,7.\,1866 Paris – Februar 1944 Marokko@\textsc{Schefer, Christian} (14.\,7.\,1866 Paris – Februar 1944 Marokko), \emph{Journalist, Lehrer}|pwk}: \emph{Un jeune écrivain viennois: M. Arthur Schnitzler}\pwindex{Schefer, Christian 14.\,7.\,1866 Paris – Februar 1944 Marokko@\textsc{Schefer, Christian} (14.\,7.\,1866 Paris – Februar 1944 Marokko), \emph{Journalist, Lehrer}!Un jeune écrivain viennois: M. Arthur Schnitzler@\strich\emph{Un jeune écrivain viennois: M. Arthur Schnitzler}|pwk}. In:
                        \emph{La Nouvelle Revue}\pwindex{Nouvelle Revue@\emph{La Nouvelle Revue}|pwk}, Jg. 18, Nr. 100,
                        Mai–Juni 1896,
                     S. 855–859.}}}\label{K_L02776-4} über Dich geſchrieben, und derſelbe werde bereits in
               den nächſten Wochen erſcheinen. Er hat natürlich auch einige Ausſtellungen gemacht,
               und ich habe mich wohl gehütet, \strikeout{zu} ihn daran zu
               verhindern (ſo dumm ich auch{ }ſeine Einwände finde). Die »\textsc{Nouvelle Revue\pwindex{Nouvelle Revue@\emph{La Nouvelle Revue}|pw}}« iſt, wie Du {\pb}weißt, von der Deutſchen-Feindin
                  \textsc{Madame Adam\pwindex{Adam, Juliette 4.\,10.\,1836 Verberie – 23.\,8.\,1936 Callian@\textsc{Adam, Juliette} (4.\,10.\,1836 Verberie – 23.\,8.\,1936 Callian), \emph{Schriftstellerin, Frauenrechtlerin}|pw}} redigirt. Noch nie iſt darin ein ausführlicher Artikel über einen deutſchen
               Schriftſteller erſchienen; die Beſprechung\pwindex{Schefer, Christian 14.\,7.\,1866 Paris – Februar 1944 Marokko@\textsc{Schefer, Christian} (14.\,7.\,1866 Paris – Februar 1944 Marokko), \emph{Journalist, Lehrer}!Un jeune écrivain viennois: M. Arthur Schnitzler@\strich\emph{Un jeune écrivain viennois: M. Arthur Schnitzler}|pwv}, die Dir \textsc{M. Schefer\pwindex{Schefer, Christian 14.\,7.\,1866 Paris – Februar 1944 Marokko@\textsc{Schefer, Christian} (14.\,7.\,1866 Paris – Februar 1944 Marokko), \emph{Journalist, Lehrer}|pw}} widmet, iſt darum noch aus dieſem beſonderen Grunde ehrenvoll für Dich.\pend
           
\pstart
           Von mir{ }ſoll ich Dir{ }ſchreiben? Was denn, bitte? Ich weiß {\pb}nichts, was Dich intereſſiren könnte. Mein Leben{ }ſteht überdies faſt jeden Tag in der Frankfurter
                  Zeitung\pwindex{Frankfurter Zeitung@\emph{Frankfurter Zeitung}|pw}.\pend
           
\pstart
           Die »\textsc{Illustration\pwindex{Illustration@\emph{L’Illustration}|pw}}«{ }ſchicke ich Dir dieſer Tage.\pend
           
\pstart
           Gewiß, \textsc{Dehmel\pwindex{Dehmel, Richard 18.\,11.\,1863 Hermsdorf – 8.\,2.\,1920 Blankenese@\textsc{Dehmel, Richard} (18.\,11.\,1863 Hermsdorf – 8.\,2.\,1920 Blankenese), \emph{Schriftsteller, Schriftsteller, Krimiautor}|pw}} iſt mir widerwärtig – oh, und wie!\pend
           
\pstart
           Gewiß, der kleine \textsc{Loris\pwindex{Hofmannsthal, Hugo von 1.\,2.\,1874 Wien – 15.\,7.\,1929 Rodaun@\textsc{Hofmannsthal, Hugo von} (1.\,2.\,1874 Wien – 15.\,7.\,1929 Rodaun), \emph{Schriftsteller}|pw}} iſt nicht manierirt, {\pb}ſondern ehrlich – oder
               vielmehr{ }ſeine Manier iſt Ehrlichkeit. Aber das iſt eben das Schlimme, das eine{ }ſo
               ungünſtige \strikeout{Prog} Prognoſe rechtfertigt. \strikeout{\textcolor{gray}{W}} Wenns nur in der Haut{ }ſäße! Aber es{ }ſitzt tiefer, im Kern. Man hat dem kleinen
                  Burſchen\pwindex{Hofmannsthal, Hugo von 1.\,2.\,1874 Wien – 15.\,7.\,1929 Rodaun@\textsc{Hofmannsthal, Hugo von} (1.\,2.\,1874 Wien – 15.\,7.\,1929 Rodaun), \emph{Schriftsteller}|pwv}{ }ſolange
               eingeredet, daß er ein Genie iſt, bis er dahin gekommen iſt, jeden Sprung{ }ſeiner
               Gedanken für genial zu nehmen. {\pb}Er hat nicht eine
               der nothwendigſten Eigenſchaften des Talents: Selbſtzucht. Er empfindet drauf los und{ }ſchreibt \label{K_L02776-5v}\edtext{\textsc{idem}}{\lemma{\textnormal{\emph{idem}}}\Cendnote{\textnormal{lateinisch: entsprechend}}}\label{K_L02776-5}. Auch
               liegt Verbildung vor, – Überſtopfung mit Wiſſenskram. Man hat dieſen jungen Mann\pwindex{Hofmannsthal, Hugo von 1.\,2.\,1874 Wien – 15.\,7.\,1929 Rodaun@\textsc{Hofmannsthal, Hugo von} (1.\,2.\,1874 Wien – 15.\,7.\,1929 Rodaun), \emph{Schriftsteller}|pwv}{ }ſyſtematiſch zum Dichter
               ausbilden wollen, und das geht nicht. Die \textsc{Goethes\pwindex{Goethe, Johann Wolfgang von 28.\,8.\,1749 Frankfurt am Main – 22.\,3.\,1832 Weimar@\textsc{Goethe, Johann Wolfgang von} (28.\,8.\,1749 Frankfurt am Main – 22.\,3.\,1832 Weimar), \emph{Schriftsteller}|pw}} laſſen{ }ſich nicht züchten. Das
               Beſte in der Entwickelung {\pb}thut der Zufall (oder das
               Leben, wenn man demſelben Ding einen anderen Namen geben will, oder die Natur, was
               auch dasſelbe iſt).\pend
           
\pstart
           Grüß’ Dich Gott, mein lieber Freund!\pend
           
\pstart
           Dein treuer {\\[\baselineskip]}\spacefill\mbox{Paul Goldmann.}\pend
           \leftskip=0em{}\selectlanguage{ngerman}\endnumbering\briefempfaengerindex{Schnitzler, Arthur@\textsc{Schnitzler, Arthur}!zzzGoldmann, Paul@\emph{von Paul Goldmann}!1896-06-042@{4. 6. [1896]}|)be}\mylabel{L02776h}  \newcommand{\dateiname}{L02776}\newcommand{\titel}{Paul Goldmann an Arthur Schnitzler, 4. 6. [1896]}\newcommand{\editorInnen}{Martin Anton Müller und Laura Untner}%% latex-leseansicht-abspann.tex
%% Abspann für die Leseansicht.
%% Der Schalter \ifkorrekturansicht ist bereits durch den Vorspann gesetzt.

%% latex-abspann.tex
%% Gemeinsamer Abspann für Korrekturansicht und Leseansicht.
%% Setzt den Schalter \ifkorrekturansicht voraus (gesetzt in den
%% einbindenden Dateien latex-korrekturansicht-abspann.tex bzw.
%% latex-leseansicht-abspann.tex).
%% ---------------------------------------------------------------

\normalsize

% Das esempio-Environment wird nur in der Leseansicht benötigt
\ifkorrekturansicht\else
\newenvironment{esempio}[3]%
{
    \vspace{1.5ex}
    \rlap{\underline{#1}}
    \par
    \setlength{\parindent}{0cm}
    \nopagebreak
    \leftskip=#2cm
    \rightskip=#3cm
}
{
    \par
}
\fi

\doendnotes{C}
\bigskip
\vfill

\clearpage

\footnotesize

\ifkorrekturansicht
  \lohead{\textsc{register}}
\fi

% theindex-Environment neu definieren ohne reledmac
\makeatletter
\renewenvironment{theindex}{%
  \ifkorrekturansicht
    \section*{\indexname}%
  \else
    \subsubsection*{Index der erwähnten Entitäten}%
  \fi
  \setlength{\parindent}{0pt}%
  \setlength{\parskip}{0pt plus 0.3pt}%
  \let\item\@idxitem
}{%
  \ifkorrekturansicht\clearpage\fi
}
\makeatother

\IfFileExists{\jobname-pw.ind}{\input{\jobname-pw.ind}}{}

% Quellenangabe nur in der Leseansicht
\ifkorrekturansicht\else
% Fallback-Definitionen, falls die .tex-Datei \titel etc. nicht gesetzt hat
\providecommand{\titel}{}
\providecommand{\editorInnen}{}
\providecommand{\dateiname}{\jobname}

\vspace{3cm}

\vfill

\footnotesize
\textsc{Quelle}: \titel. Herausgegeben von {\editorInnen}. In: \emph{Arthur Schnitzler: Briefwechsel mit Autorinnen und Autoren}.
 Digitale Edition, https://schnitzler-briefe.acdh.oeaw.ac.at/{\dateiname}.html (Stand \today)
\fi

\end{document}


