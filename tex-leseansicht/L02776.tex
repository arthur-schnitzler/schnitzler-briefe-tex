%% latex-korrekturansicht-vorspann.tex
%% Vorspann für die Korrekturansicht.
%% Lädt die gemeinsame Datei latex-vorspann.tex mit gesetztem Schalter.

\newif\ifkorrekturansicht
\korrekturansichttrue

\input{../tex-inputs/latex-vorspann}


\section[ Paul Goldmann an Arthur Schnitzler, 4. 6. {[}1896{]}]{L02776 Paul Goldmann an Arthur Schnitzler, 4. 6. {[}1896{]}}
\nopagebreak\mylabel{L02776v}
\rehead{ }\normalsize\beginnumbering\briefempfaengerindex{Schnitzler, Arthur@\textsc{Schnitzler, Arthur}!zzzGoldmann, Paul@\emph{von Paul Goldmann}!1896-06-042@{4. 6. {[}1896{]}}|(be}
\toendnotes[C]{\smallbreak\pagebreak[2]}\Standort{DLA, A:Schnitzler, HS.NZ85.1.3166.}
\physDesc{Brief, 2 Blätter, 8 Seiten, 2377 Zeichen
\newline{}Handschrift: blaue Tinte, deutsche Kurrent
\newline{}Schnitzler: 1) mit Bleistift das Jahr »96« vermerkt  2) mit rotem Buntstift vier Unterstreichungen}\toendnotes[C]{\smallbreak}
\pstart
           {\pb}\textcolor{gray}{\textbf{\textbf{Frankfurter Zeitung\orgindex{Frankfurter Zeitung@Frankfurter Zeitung|pw}}}}\pend
           
\pstart
           \textcolor{gray}{\textbf{(\begin{otherlanguage}{french}Gazette de Francfort\end{otherlanguage}\orgindex{Frankfurter Zeitung@Frankfurter Zeitung|pw}).}}\pend
           
\pstart
           \textcolor{gray}{\textbf{\textbf{\begin{otherlanguage}{french}Fondateur M.\end{otherlanguage}{ }L. Sonnemann\pwindex{Sonnemann, Leopold 1831-10-29 – 1909-10-30@\textsc{Sonnemann, Leopold} (1831-10-29 – 1909-10-30), \emph{Journalist/Journalistin, Herausgeber/Herausgeberin}|pw}.}}}\pend
           
\pstart
           \begin{otherlanguage}{french}\textcolor{gray}{\textbf{Journal\pwindex{Frankfurter Zeitung@\emph{Frankfurter Zeitung}|pwv} politique,
                        financier,}}\end{otherlanguage}\pend
           
\pstart
           \begin{otherlanguage}{french}\textcolor{gray}{\textbf{commercial et littéraire.}}\end{otherlanguage}\pend
           
\pstart
           \begin{otherlanguage}{french}\textcolor{gray}{\textbf{\textbf{Paraissant trois fois par jour.}}}\end{otherlanguage}\pend
           
\pstart
           \begin{otherlanguage}{french}\textcolor{gray}{\textbf{\textbf{Bureau à Paris\oindex{Paris@\textbf{Paris}, \emph{P.PPLC}|pw}}}}\end{otherlanguage}\hfill \textsc{Paris\oindex{Paris@\textbf{Paris}, \emph{P.PPLC}|pw}}, 4. Juni.\pend
           
\pstart
           \begin{otherlanguage}{french}\textcolor{gray}{\textbf{\textbf{24. Rue Feydeau\oindex{rue Feydeau@\textbf{rue Feydeau}, \emph{Straße (K.STR)}|pw}.}}}\end{otherlanguage}\pend
           
\pstart\center{}Mein lieber Freund,\pend\vspace{0.5em}
\pstart
           In Eile nur ein Wort des Dankes für Deinen lieben Brief!\pend
           
\pstart
           So iſt es alſo abgemacht: Ich komme nach \label{K_L02776-1v}\edtext{Dänemark\oindex{Daenemark@\textbf{Dänemark}, \emph{A.PCLI}|pw}}{\lemma{\textnormal{\emph{Dänemark}}}\Cendnote{\textnormal{Siehe Paul Goldmann an Arthur Schnitzler, 29. 4. [1896].
               }}}\label{K_L02776-1}, – immer unter \strikeout{Vor} der Vorausſetzung, daß die
               weite Reiſe nicht über meine Mittel geht. Kannſt Du mir mittheilen, was man ungefähr
               pro Tag in \textsc{Scottsborg\oindex{Skodsborg@\textbf{Skodsborg}, \emph{P.PPL}|pwv}} braucht? {\pb}Ich freue mich unendlich darauf,
               Dich wiederzusehen. Du wirſt mir wohl noch weitere Details angeben. Wann reiſt \textsc{Richard\pwindex{Beer-Hofmann, Richard 1866-07-11 – 1945-09-26@\textsc{Beer-Hofmann, Richard} (1866-07-11 – 1945-09-26), \emph{Schriftsteller/Schriftstellerin}|pw}}? Zurück will ich dann über \label{K_L02776-2v}\edtext{Berlin\oindex{Berlin@\textbf{Berlin}, \emph{P.PPLC}|pw}}{\lemma{\textnormal{\emph{Berlin}}}\Cendnote{\textnormal{Siehe A. S.: \emph{Tagebuch}, 26. 8. 1896.
               }}}\label{K_L02776-2} gehen.\pend
           
\pstart
           Die \label{K_L02776-3v}\edtext{Ernennung von \textsc{Antoine\pwindex{Antoine, Andre 1858-01-31 – 1943-10-23@\textsc{Antoine, André} (1858-01-31 – 1943-10-23), \emph{Theaterleiter/Theaterleiterin, Schauspieler/Schauspielerin}|pw}} zum Director des \textsc{Odéon\orgindex{Odeon@Odéon|pw}}}{\lemma{\textnormal{\emph{Ernennung … Odéon}}}\Cendnote{\textnormal{André Antoine\pwindex{Antoine, Andre 1858-01-31 – 1943-10-23@\textsc{Antoine, André} (1858-01-31 – 1943-10-23), \emph{Theaterleiter/Theaterleiterin, Schauspieler/Schauspielerin}|pwk} wurde 1896 neben Paul Ginisty\pwindex{Ginisty, Paul 1855-04-04 – 1932-03-05@\textsc{Ginisty, Paul} (1855-04-04 – 1932-03-05), \emph{Schriftsteller/Schriftstellerin, Theaterleiter/Theaterleiterin}|pwk} zum Ko-Direktor\pwindex{Antoine, Andre 1858-01-31 – 1943-10-23@\textsc{Antoine, André} (1858-01-31 – 1943-10-23), \emph{Theaterleiter/Theaterleiterin, Schauspieler/Schauspielerin}|pwkv} des \emph{Odéon}\orgindex{Odeon@Odéon|pwk} ernannt.}}}\label{K_L02776-3} eröffnet uns eine
               unverhoffte Ausſicht, Dein Stück\pwindex{Liebelei. Schauspiel in drei Akten@\emph{Liebelei. Schauspiel in drei Akten}|pwv} doch noch hier auf ein großes Theater zu bringen. Nächſtens mehr
               darüber.\pend
           
\pstart
           {\pb}\textsc{M. Christian Schefer\pwindex{Schefer, Christian 1866-07-14 – Februar 1944@\textsc{Schefer, Christian} (1866-07-14 – Februar 1944), \emph{Journalist/Journalistin, Lehrer/Lehrerin}|pw}} beſuchte mich dieſer Tage u. ſagte mir, er habe einen \label{K_L02776-4v}\edtext{Artikel\pwindex{Un jeune ecrivain viennois: M. Arthur Schnitzler@\emph{Un jeune écrivain viennois: M. Arthur Schnitzler}|pwv}}{\lemma{\textnormal{\emph{Artikel}}}\Cendnote{\textnormal{Christian Schefer\pwindex{Schefer, Christian 1866-07-14 – Februar 1944@\textsc{Schefer, Christian} (1866-07-14 – Februar 1944), \emph{Journalist/Journalistin, Lehrer/Lehrerin}|pwk}: \emph{Un jeune écrivain viennois: M. Arthur Schnitzler}\pwindex{Un jeune ecrivain viennois: M. Arthur Schnitzler@\emph{Un jeune écrivain viennois: M. Arthur Schnitzler}|pwk}. In:
                        \emph{La Nouvelle Revue}\pwindex{Nouvelle Revue@\emph{La Nouvelle Revue}|pwk}, Jg. 18, Nr. 100,
                        Mai–Juni 1896,
                     S. 855–859.}}}\label{K_L02776-4} über Dich geſchrieben, und derſelbe werde bereits in
               den nächſten Wochen erſcheinen. Er hat natürlich auch einige Ausſtellungen gemacht,
               und ich habe mich wohl gehütet, \strikeout{zu} ihn daran zu
               verhindern (ſo dumm ich auch ſeine Einwände finde). Die »\textsc{Nouvelle Revue\pwindex{Nouvelle Revue@\emph{La Nouvelle Revue}|pw}}« iſt, wie Du {\pb}weißt, von der Deutſchen-Feindin
                  \textsc{Madame Adam\pwindex{Adam, Juliette 1836-10-04 – 1936-08-23@\textsc{Adam, Juliette} (1836-10-04 – 1936-08-23), \emph{Schriftsteller/Schriftstellerin, Frauenrechtler/Frauenrechtlerin}|pw}} redigirt. Noch nie iſt darin ein ausführlicher Artikel über einen deutſchen
               Schriftſteller erſchienen; die Beſprechung\pwindex{Un jeune ecrivain viennois: M. Arthur Schnitzler@\emph{Un jeune écrivain viennois: M. Arthur Schnitzler}|pwv}, die Dir \textsc{M. Schefer\pwindex{Schefer, Christian 1866-07-14 – Februar 1944@\textsc{Schefer, Christian} (1866-07-14 – Februar 1944), \emph{Journalist/Journalistin, Lehrer/Lehrerin}|pw}} widmet, iſt darum noch aus dieſem beſonderen Grunde ehrenvoll für Dich.\pend
           
\pstart
           Von mir ſoll ich Dir ſchreiben? Was denn, bitte? Ich weiß {\pb}nichts, was Dich intereſſiren könnte. Mein Leben
               ſteht überdies faſt jeden Tag in der Frankfurter
                  Zeitung\pwindex{Frankfurter Zeitung@\emph{Frankfurter Zeitung}|pw}.\pend
           
\pstart
           Die »\textsc{Illustration\pwindex{Illustration@\emph{L’Illustration}|pw}}« ſchicke ich Dir dieſer Tage.\pend
           
\pstart
           Gewiß, \textsc{Dehmel\pwindex{Dehmel, Richard 18.11.1863 – 08.02.1920@\textsc{Dehmel, Richard} (18.11.1863 – 08.02.1920), \emph{Schriftsteller/Schriftstellerin, Schriftsteller/Schriftstellerin, Krimiautor/Krimiautorin}|pw}} iſt mir widerwärtig – oh, und wie!\pend
           
\pstart
           Gewiß, der kleine \textsc{Loris\pwindex{Hofmannsthal, Hugo von 1874-02-01 – 1929-07-15@\textsc{Hofmannsthal, Hugo von} (1874-02-01 – 1929-07-15), \emph{Schriftsteller/Schriftstellerin}|pw}} iſt nicht manierirt, {\pb}ſondern ehrlich – oder
               vielmehr ſeine Manier iſt Ehrlichkeit. Aber das iſt eben das Schlimme, das eine ſo
               ungünſtige \strikeout{Prog} Prognoſe rechtfertigt. \strikeout{\textcolor{gray}{W}} Wenns nur in der Haut ſäße! Aber es ſitzt tiefer, im Kern. Man hat dem kleinen
                  Burſchen\pwindex{Hofmannsthal, Hugo von 1874-02-01 – 1929-07-15@\textsc{Hofmannsthal, Hugo von} (1874-02-01 – 1929-07-15), \emph{Schriftsteller/Schriftstellerin}|pwv} ſolange
               eingeredet, daß er ein Genie iſt, bis er dahin gekommen iſt, jeden Sprung ſeiner
               Gedanken für genial zu nehmen. {\pb}Er hat nicht eine
               der nothwendigſten Eigenſchaften des Talents: Selbſtzucht. Er empfindet drauf los und
               ſchreibt \label{K_L02776-5v}\edtext{\textsc{idem}}{\lemma{\textnormal{\emph{idem}}}\Cendnote{\textnormal{lateinisch: entsprechend}}}\label{K_L02776-5}. Auch
               liegt Verbildung vor, – Überſtopfung mit Wiſſenskram. Man hat dieſen jungen Mann\pwindex{Hofmannsthal, Hugo von 1874-02-01 – 1929-07-15@\textsc{Hofmannsthal, Hugo von} (1874-02-01 – 1929-07-15), \emph{Schriftsteller/Schriftstellerin}|pwv} ſyſtematiſch zum Dichter
               ausbilden wollen, und das geht nicht. Die \textsc{Goethes\pwindex{Goethe, Johann Wolfgang von 1749-08-28 – 1832-03-22@\textsc{Goethe, Johann Wolfgang von} (1749-08-28 – 1832-03-22), \emph{Schriftsteller/Schriftstellerin}|pw}} laſſen ſich nicht züchten. Das
               Beſte in der Entwickelung {\pb}thut der Zufall (oder das
               Leben, wenn man demſelben Ding einen anderen Namen geben will, oder die Natur, was
               auch dasſelbe iſt).\pend
           
\pstart
           Grüß’ Dich Gott, mein lieber Freund!\pend
           
\pstart
           Dein treuer {\\[\baselineskip]}\spacefill\mbox{Paul Goldmann.}\pend
           \leftskip=0em{}\selectlanguage{ngerman}\endnumbering\briefempfaengerindex{Schnitzler, Arthur@\textsc{Schnitzler, Arthur}!zzzGoldmann, Paul@\emph{von Paul Goldmann}!1896-06-042@{4. 6. {[}1896{]}}|)be}\mylabel{L02776h}  \normalsize

\doendnotes{C}
\bigskip
\vfill

\clearpage

\footnotesize

\lohead{\textsc{register}}

% Definiere theindex-Environment komplett neu ohne reledmac
\makeatletter
\renewenvironment{theindex}{%
  \section*{\indexname}%
  \setlength{\parindent}{0pt}%
  \setlength{\parskip}{0pt plus 0.3pt}%
  \let\item\@idxitem
}{%
  \clearpage
}
\makeatother

\IfFileExists{\jobname-pw.ind}{\input{\jobname-pw.ind}}{}

\end{document}

      