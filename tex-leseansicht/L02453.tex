%% latex-korrekturansicht-vorspann.tex
%% Vorspann für die Korrekturansicht.
%% Lädt die gemeinsame Datei latex-vorspann.tex mit gesetztem Schalter.

\newif\ifkorrekturansicht
\korrekturansichttrue

\input{../tex-inputs/latex-vorspann}


\section[Arthur Schnitzler an Hugo Hofmannsthal, 9. 10. 1925]{L02453 Arthur Schnitzler an Hugo Hofmannsthal, 9. 10. 1925}
\nopagebreak\mylabel{L02453v}
\rehead{ }\normalsize\beginnumbering\briefempfaengerindex{Hofmannsthal, Hugo von@\textsc{Hofmannsthal, Hugo von}!zzzSchnitzler, Arthur@\emph{von Arthur Schnitzler}!1925-10-091@{9. 10. 1925}|(be}
\toendnotes[C]{\smallbreak\pagebreak[2]}\Standort{FDH, Hs-30885,153.}
\physDesc{Postkarte, 1014 Zeichen
\newline{}Handschrift: Bleistift, lateinische Kurrent
\newline{}Versand: Stempel: »\nobreak{}\oindex{XVIII., Waehring@\textbf{XVIII., Währing}, \emph{A.ADM3}|pwk}18/1 Wien, 10. X. 25, 1\textcolor{gray}{8}\nobreak{}«.  }
\buchAbdrucke{\weitereDrucke{Hugo von Hofmannsthal, Arthur Schnitzler: \emph{Briefwechsel}. Frankfurt am Main: \emph{S. Fischer} 1964, S. 302.} }\toendnotes[C]{\smallbreak}\pstart{}{\pb}\label{T_L02453-1v}\edtext{\textcolor{gray}{\textbf{A. S.}}}{\lemma{\textnormal{\emph{A. S.}}}\Cendnote{\textnormal{ovaler Absenderkleber}}}\label{T_L02453-1}\pend{}\pstart{}\textcolor{gray}{\textbf{WIEN, XVIII.}}\oindex{XVIII., Waehring@\textbf{XVIII., Währing}, \emph{A.ADM3}|pw}\pend{}\pstart{}\textcolor{gray}{\textbf{STERNWARTESTR. 71}}\oindex{Sternwartestrasse 71@\textbf{Sternwartestraße 71}, \emph{Wohngebäude (K.WHS)}|pw}\pend{}{\bigskip}\pstart{}Hrn Hugo v Hofmannsthal\pend{}\pstart{}Bad Aussee\oindex{Bad Aussee@\textbf{Bad Aussee}, \emph{P.PPLA3}|pw}\pend{}\pstart{}Ramgut\oindex{Ramgut@\textbf{Ramgut}, \emph{Schloss (K.SLS)}|pw}.\pend{}{\bigskip}\vspace{1em}
\pstart
           \raggedleft{}{\pb}Wien\oindex{Wien@\textbf{Wien}, \emph{A.ADM2}|pw}, 9. X. 1925\pend
           \vspace{0.5em}
\pstart
           mein lieber Hugo,{ }So{\geminationn}tag fahre ich nach Berlin\oindex{Berlin@\textbf{Berlin}, \emph{P.PPLC}|pw}, (Hotel Esplanade\oindex{Hotel Esplanade [Berlin]@\textbf{Hotel Esplanade [Berlin]}, \emph{Hotel (K.HTL)}|pw}) –
               schicken Sie den Thurm\pwindex{Turm. Ein Trauerspiel@\emph{Der Turm. Ein Trauerspiel}|pw} gleich ab, so findet er
               mich dort, da ich wohl mindestens 8 Tage dort bleibe. Unter anderm werd ich dort
                  \label{K_L02453-1v}\edtext{Heini\pwindex{Schnitzler, Heinrich 09.08.1902 – 12.07.1982@\textsc{Schnitzler, Heinrich} (09.08.1902 – 12.07.1982), \emph{Regisseur/Regisseurin, Schauspieler/Schauspielerin}|pw} als Theodor\pwindex{Liebelei. Schauspiel in drei Akten@\emph{Liebelei. Schauspiel in drei Akten}|pwv}}{\lemma{\textnormal{\emph{Heini als Theodor}}}\Cendnote{\textnormal{Siehe A. S.: \emph{Tagebuch}, 13. 10. 1925.
               }}}\label{K_L02453-1} in der Liebelei\pwindex{Liebelei. Schauspiel in drei Akten@\emph{Liebelei. Schauspiel in drei Akten}|pw}{ }sehen (die \uline{heute}
               vor 30 Jahren in Wien\oindex{Wien@\textbf{Wien}, \emph{A.ADM2}|pw} zum »überhaupt« ersten Mal
               aufgeführt wurde.) Auch ein neues Stück nehm ich nach Berlin\oindex{Berlin@\textbf{Berlin}, \emph{P.PPLC}|pw} mit, in Versen, und heißt: {[}»{]}Der Gang zum Weiher\pwindex{Gang zum Weiher. Dramatische Dichtung@\emph{Der Gang zum Weiher. Dramatische Dichtung}|pw}«{[}.{]} Gegen
               die Aufführg von Kom. d. Verf.\pwindex{Komoedie der Verfuehrung. In drei Akten@\emph{Komödie der Verführung. In drei Akten}|pw} bei Barnowsky\pwindex{Barnowsky, Victor 10.09.1875 – 09.08.1952@\textsc{Barnowsky, Victor} (10.09.1875 – 09.08.1952), \emph{Theaterleiter/Theaterleiterin, Regisseur/Regisseurin, Schauspieler/Schauspielerin}|pw}{ }setze ich mich zur Wehre – (die Hauptrollen
               scheinen nemlich noch nicht besetzt zu sein.) Auch eine »Traumnovelle\pwindex{Traumnovelle@\emph{Traumnovelle}|pw}« (so heißt sie) erscheint nächstens. – Von Forte dei Marmi\oindex{Forte dei Marmi@\textbf{Forte dei Marmi}, \emph{P.PPLA3}|pw} bin ich nach Florenz\oindex{Florenz@\textbf{Florenz}, \emph{P.PPLA}|pw}, nach Venedig\oindex{Venedig@\textbf{Venedig}, \emph{P.PPLA}|pw}; und
               vor 3 Wochen nach Wien\oindex{Wien@\textbf{Wien}, \emph{A.ADM2}|pw}. Hoffentlich sieht man {\pb}sich einmal wieder – und bald. (Es wird immer später.)
                  Christiane\pwindex{Zimmer, Christiane 14.05.1902 – 05.01.1987@\textsc{Zimmer, Christiane} (14.05.1902 – 05.01.1987)|pw}{ }sah ich in Venedig\oindex{Venedig@\textbf{Venedig}, \emph{P.PPLA}|pw}; ich glaube, Lili\pwindex{Cappellini, Lili 13.09.1909 – 26.07.1928@\textsc{Cappellini, Lili} (13.09.1909 – 26.07.1928)|pw} u Olga\pwindex{Schnitzler, Olga 17.01.1882 – 13.01.1970@\textsc{Schnitzler, Olga} (17.01.1882 – 13.01.1970), \emph{Schauspieler/Schauspielerin, Sänger/Sängerin}|pw} haben sie nach meiner Abreise auch
               gesprochen. –\pend
           
\pstart
           Nichts von alldem ahnten wir heute vor 30 Jahren. Und eigentlich war es gestern.\pend
           
\pstart
           Leben Sie wohl.\pend
           
\pstart
           In Herzlichkeit Ihr{\\[\baselineskip]}\spacefill\mbox{A.}\pend
           \leftskip=0em{}\selectlanguage{ngerman}\endnumbering\briefempfaengerindex{Hofmannsthal, Hugo von@\textsc{Hofmannsthal, Hugo von}!zzzSchnitzler, Arthur@\emph{von Arthur Schnitzler}!1925-10-091@{9. 10. 1925}|)be}\mylabel{L02453h}  \normalsize

\doendnotes{C}
\bigskip
\vfill

\clearpage

\footnotesize

\lohead{\textsc{register}}

% Definiere theindex-Environment komplett neu ohne reledmac
\makeatletter
\renewenvironment{theindex}{%
  \section*{\indexname}%
  \setlength{\parindent}{0pt}%
  \setlength{\parskip}{0pt plus 0.3pt}%
  \let\item\@idxitem
}{%
  \clearpage
}
\makeatother

\IfFileExists{\jobname-pw.ind}{\input{\jobname-pw.ind}}{}

\end{document}

      