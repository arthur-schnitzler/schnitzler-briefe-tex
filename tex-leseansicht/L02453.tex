%% latex-leseansicht-vorspann.tex
%% Vorspann für die Leseansicht.
%% Lädt die gemeinsame Datei latex-vorspann.tex mit nicht gesetztem Schalter.

\newif\ifkorrekturansicht
\korrekturansichtfalse

\input{../tex-inputs/latex-vorspann}


         
         \newcommand{\erwaehntePersonen}{Personen: Victor Barnowsky, Hugo von Hofmannsthal, Christiane von Hofmannsthal, Heinrich Schnitzler, Lili Schnitzler, Olga Schnitzler}
         \newcommand{\erwaehnteOrte}{Orte: Bad Aussee, Berlin, Florenz, Forte dei Marmi, Hotel Esplanade, Ramgut, Sternwartestraße, Venedig, Wien, XVIII., Währing}
         \newcommand{\erwaehnteWerke}{Werke: Der Gang zum Weiher. Dramatische Dichtung, Der Turm. Ein Trauerspiel, Komödie der Verführung. In drei Akten, Liebelei. Schauspiel in drei Akten, Traumnovelle}
               \section[Arthur Schnitzler an Hugo Hofmannsthal, 9. 10. 1925]{ Arthur Schnitzler an Hugo Hofmannsthal, 9. 10. 1925}\nopagebreak\mylabel{v}\rehead{ }\begin{ledgroupsized}[t]{13cm}\normalsize\beginnumbering \toendnotes[C]{\smallbreak\pagebreak[2]} \Standort{FDH, Hs-30885,153.}
\physDesc{Postkarte
\newline{}Handschrift: Bleistift, lateinische Kurrent\newline{}Versand: Stempel: »\nobreak{}\oindex{XVIII., Waehring@\textbf{XVIII., Währing}|pwk}18/1 Wien, 10. X. 25, 1\textcolor{gray}{8}\nobreak{}«.  }\buchAbdrucke{\weitereDrucke{Hugo von Hofmannsthal, Arthur Schnitzler: \emph{Briefwechsel}. Hg. Therese Nickl und Heinrich Schnitzler. Frankfurt am Main: \emph{S. Fischer} 1964, S. 302.} }\toendnotes[C]{\smallbreak}\pstart{}{\pb}\label{T_L02453-1v}\edtext{\textcolor{gray}{\textbf{A. S.}}}{\lemma{\textnormal{\emph{A. S.}}}\Cendnote{\textnormal{ovaler Absenderkleber}}}\label{T_L02453-1h}\pend{}\pstart{}\textcolor{gray}{\textbf{WIEN, XVIII.}}\oindex{XVIII., Waehring@\textbf{XVIII., Währing}|pw}\pend{}\pstart{}\textcolor{gray}{\textbf{STERNWARTESTR. 71}}\oindex{Sternwartestrasse@\textbf{Sternwartestraße}|pw}\pend{}{\bigskip}\pstart{}Hrn Hugo v Hofmannsthal\pend{}\pstart{}Bad Aussee\oindex{Bad Aussee@\textbf{Bad Aussee}|pw}\pend{}\pstart{}Ramgut\oindex{Ramgut@\textbf{Ramgut}|pw}.\pend{}{\bigskip}\pstart
           \raggedleft{}{\pb}Wien\oindex{Wien@\textbf{Wien}|pw}, 9. X. 1925\pend
           \pstart
           mein lieber Hugo, So{\geminationn}tag fahre ich nach Berlin\oindex{Berlin@\textbf{Berlin}|pw}, (Hotel Esplanade\oindex{Hotel Esplanade@\textbf{Hotel Esplanade}|pw}) –
               schicken Sie den Thurm\pwindex{Hofmannsthal, Hugo von 1874-02-01 – 1929-07-15@\textsc{Hofmannsthal, Hugo von} (1874-02-01 – 1929-07-15), \emph{Schriftsteller}!Turm. Ein Trauerspiel1925@\strich\emph{Der Turm. Ein Trauerspiel} {[}1925{]}|pw} gleich ab, so findet er mich
               dort, da ich wohl mindestens 8 Tage dort bleibe. Unter anderm werd ich dort
                  \label{K_L02453_1v}\edtext{Heini\pwindex{Schnitzler, Heinrich 09.08.1902 – 12.07.1982@\textsc{Schnitzler, Heinrich} (09.08.1902 – 12.07.1982), \emph{Regisseur, Schauspieler}|pw} als Theodor\pwindex{Schnitzler, Arthur 15.05.1862 – 21.10.1931@\textsc{Schnitzler, Arthur} (15.05.1862 – 21.10.1931), \emph{Schriftsteller, Mediziner}!Liebelei. Schauspiel in drei Akten1895-10-09@\strich\emph{Liebelei. Schauspiel in drei Akten} {[}1895-10-09{]}|pwv}}{\lemma{\textnormal{\emph{Heini als Theodor}}}\Cendnote{\textnormal{siehe A. S.: \emph{Tagebuch}, 13. 10. 1925}}}\label{K_L02453_1h} in der Liebelei\pwindex{Schnitzler, Arthur 15.05.1862 – 21.10.1931@\textsc{Schnitzler, Arthur} (15.05.1862 – 21.10.1931), \emph{Schriftsteller, Mediziner}!Liebelei. Schauspiel in drei Akten1895-10-09@\strich\emph{Liebelei. Schauspiel in drei Akten} {[}1895-10-09{]}|pw}{ }sehen (die \uline{heute}
               vor 30 Jahren in Wien\oindex{Wien@\textbf{Wien}|pw} zum »überhaupt« ersten Mal
               aufgeführt wurde.) Auch ein neues Stück nehm ich nach Berlin\oindex{Berlin@\textbf{Berlin}|pw} mit, in Versen, und heißt: {[}»{]}Der Gang zum Weiher\pwindex{Schnitzler, Arthur 15.05.1862 – 21.10.1931@\textsc{Schnitzler, Arthur} (15.05.1862 – 21.10.1931), \emph{Schriftsteller, Mediziner}!Gang zum Weiher. Dramatische Dichtung1926@\strich\emph{Der Gang zum Weiher. Dramatische Dichtung} {[}1926{]}|pw}«{[}.{]} Gegen
               die Aufführg von Kom. d. Verf.\pwindex{Schnitzler, Arthur 15.05.1862 – 21.10.1931@\textsc{Schnitzler, Arthur} (15.05.1862 – 21.10.1931), \emph{Schriftsteller, Mediziner}!Komoedie der Verfuehrung. In drei Akten1924@\strich\emph{Komödie der Verführung. In drei Akten} {[}1924{]}|pw} bei Barnowsky\pwindex{Barnowsky, Victor 10.09.1875 – 09.08.1952@\textsc{Barnowsky, Victor} (10.09.1875 – 09.08.1952), \emph{Theaterleiter, Regisseur, Schauspieler}|pw}{ }setze ich mich zur Wehre – (die Hauptrollen
               scheinen nemlich noch nicht besetzt zu sein.) Auch eine »Traumnovelle\pwindex{Schnitzler, Arthur 15.05.1862 – 21.10.1931@\textsc{Schnitzler, Arthur} (15.05.1862 – 21.10.1931), \emph{Schriftsteller, Mediziner}!Traumnovelle1.12.1925 – 1.3.1926@\strich\emph{Traumnovelle} {[}1.12.1925 – 1.3.1926{]}|pw}« (so heißt sie) erscheint nächstens. – Von Forte dei Marmi\oindex{Forte dei Marmi@\textbf{Forte dei Marmi}|pw} bin ich nach Florenz\oindex{Florenz@\textbf{Florenz}|pw}, nach Venedig\oindex{Venedig@\textbf{Venedig}|pw}; und vor 3 Wochen
               nach Wien\oindex{Wien@\textbf{Wien}|pw}. Hoffentlich sieht man {\pb}sich einmal wieder – und bald. (Es wird immer später.)
                  Christiane\pwindex{Hofmannsthal, Christiane von 14.05.1902 – 05.01.1987@\textsc{Hofmannsthal, Christiane von} (14.05.1902 – 05.01.1987)|pw}{ }sah ich in Venedig\oindex{Venedig@\textbf{Venedig}|pw}; ich glaube, Lili\pwindex{Schnitzler, Lili 13.09.1909 – 26.07.1928@\textsc{Schnitzler, Lili} (13.09.1909 – 26.07.1928)|pw} u Olga\pwindex{Schnitzler, Olga 17.01.1882 – 13.01.1970@\textsc{Schnitzler, Olga} (17.01.1882 – 13.01.1970), \emph{Schauspielerin, Sängerin}|pw} haben sie nach meiner Abreise auch gesprochen. –\pend
           \pstart
           Nichts von alldem ahnten wir heute vor 30 Jahren. Und eigentlich war es gestern.\pend
           \pstart
           Leben Sie wohl.\pend
           \pstart
           In Herzlichkeit Ihr{\\[\baselineskip]}\spacefill\mbox{A.}\pend
           \leftskip=0em{}
         
         \endnumbering\mylabel{h}\end{ledgroupsized}  \newcommand{\dateiname}{L02453}\newcommand{\titel}{Arthur Schnitzler an Hugo Hofmannsthal, 9. 10. 1925}\newcommand{\editorInnen}{Martin Anton Müller und Gerd-Hermann Susen}%% latex-leseansicht-abspann.tex
%% Abspann für die Leseansicht.
%% Der Schalter \ifkorrekturansicht ist bereits durch den Vorspann gesetzt.

%% latex-abspann.tex
%% Gemeinsamer Abspann für Korrekturansicht und Leseansicht.
%% Setzt den Schalter \ifkorrekturansicht voraus (gesetzt in den
%% einbindenden Dateien latex-korrekturansicht-abspann.tex bzw.
%% latex-leseansicht-abspann.tex).
%% ---------------------------------------------------------------

\normalsize

% Das esempio-Environment wird nur in der Leseansicht benötigt
\ifkorrekturansicht\else
\newenvironment{esempio}[3]%
{
    \vspace{1.5ex}
    \rlap{\underline{#1}}
    \par
    \setlength{\parindent}{0cm}
    \nopagebreak
    \leftskip=#2cm
    \rightskip=#3cm
}
{
    \par
}
\fi

\doendnotes{C}
\bigskip
\vfill

\clearpage

\footnotesize

\ifkorrekturansicht
  \lohead{\textsc{register}}
\fi

% theindex-Environment neu definieren ohne reledmac
\makeatletter
\renewenvironment{theindex}{%
  \ifkorrekturansicht
    \section*{\indexname}%
  \else
    \subsubsection*{Index der erwähnten Entitäten}%
  \fi
  \setlength{\parindent}{0pt}%
  \setlength{\parskip}{0pt plus 0.3pt}%
  \let\item\@idxitem
}{%
  \ifkorrekturansicht\clearpage\fi
}
\makeatother

\IfFileExists{\jobname-pw.ind}{\input{\jobname-pw.ind}}{}

% Quellenangabe nur in der Leseansicht
\ifkorrekturansicht\else
% Fallback-Definitionen, falls die .tex-Datei \titel etc. nicht gesetzt hat
\providecommand{\titel}{}
\providecommand{\editorInnen}{}
\providecommand{\dateiname}{\jobname}

\vspace{3cm}

\vfill

\footnotesize
\textsc{Quelle}: \titel. Herausgegeben von {\editorInnen}. In: \emph{Arthur Schnitzler: Briefwechsel mit Autorinnen und Autoren}.
 Digitale Edition, https://schnitzler-briefe.acdh.oeaw.ac.at/{\dateiname}.html (Stand \today)
\fi

\end{document}


      