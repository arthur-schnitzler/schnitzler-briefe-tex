%% latex-leseansicht-vorspann.tex
%% Vorspann für die Leseansicht.
%% Lädt die gemeinsame Datei latex-vorspann.tex mit nicht gesetztem Schalter.

\newif\ifkorrekturansicht
\korrekturansichtfalse

\input{../tex-inputs/latex-vorspann}


\section[Arthur Schnitzler an Hugo Hofmannsthal, 9. 10. 1925]{L02453 Arthur Schnitzler an Hugo Hofmannsthal, 9. 10. 1925}
\nopagebreak\mylabel{L02453v}
\rehead{ }\normalsize\beginnumbering\briefempfaengerindex{Hofmannsthal, Hugo von@\textsc{Hofmannsthal, Hugo von}!zzzSchnitzler, Arthur@\emph{von Arthur Schnitzler}!1925-10-091@{9. 10. 1925}|(be}
\toendnotes[C]{\smallbreak\pagebreak[2]}
\correspDesc{Versand  durch Arthur Schnitzler am 9. 10. 1925 in Wien
\newline{}Erhalt  durch Hugo von Hofmannsthal im Zeitraum [10. 10. 1925 – 14. 10. 1925?] in Bad Aussee}\toendnotes[C]{\smallbreak}
\Standort{FDH, Hs-30885,153.}
\physDesc{Postkarte, 1014 Zeichen
\newline{}Handschrift: Bleistift, lateinische Kurrent
\newline{}Versand: Stempel: »\nobreak{}\oindex{XVIII., Währing@\textbf{XVIII., Währing}, \emph{Verwaltungsgebiet}|pwk}18/1 Wien, 10. X. 25, 1\textcolor{gray}{8}\nobreak{}«.  }
\buchAbdrucke{\weitereDrucke{Hugo von Hofmannsthal, Arthur Schnitzler: \emph{Briefwechsel}. Herausgegeben von Therese Nickl und Heinrich Schnitzler. Frankfurt am Main: \emph{S. Fischer} 1964, S. 302.} }\toendnotes[C]{\smallbreak}\pstart{}{\pb}\label{T_L02453-1v}\edtext{\textcolor{gray}{\textbf{A. S.}}}{\lemma{\textnormal{\emph{A. S.}}}\Cendnote{\textnormal{ovaler Absenderkleber}}}\label{T_L02453-1}\pend{}\pstart{}\textcolor{gray}{\textbf{WIEN, XVIII.}}\oindex{XVIII., Währing@\textbf{XVIII., Währing}, \emph{Verwaltungsgebiet}|pw}\pend{}\pstart{}\textcolor{gray}{\textbf{STERNWARTESTR. 71}}\oindex{Wien@\textbf{Wien}!XVIII., Währing@\textbf{XVIII., Währing}!Sternwartestraße 71@\textbf{Sternwartestraße 71}, \emph{Wohngebäude}|pw}\pend{}{\bigskip}\pstart{}Hrn Hugo v Hofmannsthal\pend{}\pstart{}Bad Aussee\oindex{Bad Aussee@\textbf{Bad Aussee}, \emph{Hauptstadt}|pw}\pend{}\pstart{}Ramgut\oindex{Ramgut@\textbf{Ramgut}, \emph{Schloss}|pw}.\pend{}{\bigskip}\vspace{1em}
\pstart
           \raggedleft{}{\pb}Wien\oindex{Wien@\textbf{Wien}, \emph{Verwaltungsgebiet}|pw}, 9. X. 1925\pend
           \vspace{0.5em}
\pstart
           mein lieber Hugo,{ }So{\geminationn}tag fahre ich nach Berlin\oindex{Berlin@\textbf{Berlin}, \emph{Hauptstadt}|pw}, (Hotel Esplanade\oindex{Hotel Esplanade [Berlin]@\textbf{Hotel Esplanade [Berlin]}, \emph{Hotel}|pw}) –
               schicken Sie den Thurm\pwindex{Hofmannsthal, Hugo von 1.\,2.\,1874 Wien – 15.\,7.\,1929 Rodaun@\textsc{Hofmannsthal, Hugo von} (1.\,2.\,1874 Wien – 15.\,7.\,1929 Rodaun), \emph{Schriftsteller}!Turm. Ein Trauerspiel@\strich\emph{Der Turm. Ein Trauerspiel}|pw} gleich ab, so findet er
               mich dort, da ich wohl mindestens 8 Tage dort bleibe. Unter anderm werd ich dort
                  \label{K_L02453-1v}\edtext{Heini\pwindex{Schnitzler, Heinrich 9.\,8.\,1902 Hinterbrühl – 12.\,7.\,1982 Wien@\textsc{Schnitzler, Heinrich} (9.\,8.\,1902 Hinterbrühl – 12.\,7.\,1982 Wien), \emph{Regisseur, Schauspieler}|pw} als Theodor\pwindex{Schnitzler, Arthur 15.\,5.\,1862 Wien – 21.\,10.\,1931 ebd.@\textsc{Schnitzler, Arthur} (15.\,5.\,1862 Wien – 21.\,10.\,1931 ebd.), \emph{Schriftsteller, Mediziner}!Liebelei. Schauspiel in drei Akten@\strich\emph{Liebelei. Schauspiel in drei Akten}|pwv}}{\lemma{\textnormal{\emph{Heini als Theodor}}}\Cendnote{\textnormal{Siehe A. S.: \emph{Tagebuch}, 13. 10. 1925.
               }}}\label{K_L02453-1} in der Liebelei\pwindex{Schnitzler, Arthur 15.\,5.\,1862 Wien – 21.\,10.\,1931 ebd.@\textsc{Schnitzler, Arthur} (15.\,5.\,1862 Wien – 21.\,10.\,1931 ebd.), \emph{Schriftsteller, Mediziner}!Liebelei. Schauspiel in drei Akten@\strich\emph{Liebelei. Schauspiel in drei Akten}|pw}{ }sehen (die \uline{heute}
               vor 30 Jahren in Wien\oindex{Wien@\textbf{Wien}, \emph{Verwaltungsgebiet}|pw} zum »überhaupt« ersten Mal
               aufgeführt wurde.) Auch ein neues Stück nehm ich nach Berlin\oindex{Berlin@\textbf{Berlin}, \emph{Hauptstadt}|pw} mit, in Versen, und heißt: {[}»{]}Der Gang zum Weiher\pwindex{Schnitzler, Arthur 15.\,5.\,1862 Wien – 21.\,10.\,1931 ebd.@\textsc{Schnitzler, Arthur} (15.\,5.\,1862 Wien – 21.\,10.\,1931 ebd.), \emph{Schriftsteller, Mediziner}!Gang zum Weiher. Dramatische Dichtung@\strich\emph{Der Gang zum Weiher. Dramatische Dichtung}|pw}«{[}.{]} Gegen
               die Aufführg von Kom. d. Verf.\pwindex{Schnitzler, Arthur 15.\,5.\,1862 Wien – 21.\,10.\,1931 ebd.@\textsc{Schnitzler, Arthur} (15.\,5.\,1862 Wien – 21.\,10.\,1931 ebd.), \emph{Schriftsteller, Mediziner}!Komödie der Verführung. In drei Akten@\strich\emph{Komödie der Verführung. In drei Akten}|pw} bei Barnowsky\pwindex{Barnowsky, Victor 10.\,9.\,1875 Berlin – 9.\,8.\,1952 New York City@\textsc{Barnowsky, Victor} (10.\,9.\,1875 Berlin – 9.\,8.\,1952 New York City), \emph{Theaterleiter, Regisseur, Schauspieler}|pw}{ }setze ich mich zur Wehre – (die Hauptrollen
               scheinen nemlich noch nicht besetzt zu sein.) Auch eine »Traumnovelle\pwindex{Schnitzler, Arthur 15.\,5.\,1862 Wien – 21.\,10.\,1931 ebd.@\textsc{Schnitzler, Arthur} (15.\,5.\,1862 Wien – 21.\,10.\,1931 ebd.), \emph{Schriftsteller, Mediziner}!Traumnovelle@\strich\emph{Traumnovelle}|pw}« (so heißt sie) erscheint nächstens. – Von Forte dei Marmi\oindex{Forte dei Marmi@\textbf{Forte dei Marmi}, \emph{Hauptstadt}|pw} bin ich nach Florenz\oindex{Florenz@\textbf{Florenz}|pw}, nach Venedig\oindex{Venedig@\textbf{Venedig}|pw}; und
               vor 3 Wochen nach Wien\oindex{Wien@\textbf{Wien}, \emph{Verwaltungsgebiet}|pw}. Hoffentlich sieht man {\pb}sich einmal wieder – und bald. (Es wird immer später.)
                  Christiane\pwindex{Zimmer, Christiane 14.\,5.\,1902 Rodaun – 5.\,1.\,1987 New York City@\textsc{Zimmer, Christiane} (14.\,5.\,1902 Rodaun – 5.\,1.\,1987 New York City)|pw}{ }sah ich in Venedig\oindex{Venedig@\textbf{Venedig}|pw}; ich glaube, Lili\pwindex{Cappellini, Lili 13.\,9.\,1909 Wien – 26.\,7.\,1928 Venedig@\textsc{Cappellini, Lili} (13.\,9.\,1909 Wien – 26.\,7.\,1928 Venedig)|pw} u Olga\pwindex{Schnitzler, Olga 17.\,1.\,1882 Wien – 13.\,1.\,1970 Lugano@\textsc{Schnitzler, Olga} (17.\,1.\,1882 Wien – 13.\,1.\,1970 Lugano), \emph{Schauspielerin, Sängerin}|pw} haben sie nach meiner Abreise auch
               gesprochen. –\pend
           
\pstart
           Nichts von alldem ahnten wir heute vor 30 Jahren. Und eigentlich war es gestern.\pend
           
\pstart
           Leben Sie wohl.\pend
           
\pstart
           In Herzlichkeit Ihr{\\[\baselineskip]}\spacefill\mbox{A.}\pend
           \leftskip=0em{}\selectlanguage{ngerman}\endnumbering\briefempfaengerindex{Hofmannsthal, Hugo von@\textsc{Hofmannsthal, Hugo von}!zzzSchnitzler, Arthur@\emph{von Arthur Schnitzler}!1925-10-091@{9. 10. 1925}|)be}\mylabel{L02453h}  \newcommand{\dateiname}{L02453}\newcommand{\titel}{Arthur Schnitzler an Hugo Hofmannsthal, 9. 10. 1925}\newcommand{\editorInnen}{Martin Anton Müller und Gerd-Hermann Susen}%% latex-leseansicht-abspann.tex
%% Abspann für die Leseansicht.
%% Der Schalter \ifkorrekturansicht ist bereits durch den Vorspann gesetzt.

%% latex-abspann.tex
%% Gemeinsamer Abspann für Korrekturansicht und Leseansicht.
%% Setzt den Schalter \ifkorrekturansicht voraus (gesetzt in den
%% einbindenden Dateien latex-korrekturansicht-abspann.tex bzw.
%% latex-leseansicht-abspann.tex).
%% ---------------------------------------------------------------

\normalsize

% Das esempio-Environment wird nur in der Leseansicht benötigt
\ifkorrekturansicht\else
\newenvironment{esempio}[3]%
{
    \vspace{1.5ex}
    \rlap{\underline{#1}}
    \par
    \setlength{\parindent}{0cm}
    \nopagebreak
    \leftskip=#2cm
    \rightskip=#3cm
}
{
    \par
}
\fi

\doendnotes{C}
\bigskip
\vfill

\clearpage

\footnotesize

\ifkorrekturansicht
  \lohead{\textsc{register}}
\fi

% theindex-Environment neu definieren ohne reledmac
\makeatletter
\renewenvironment{theindex}{%
  \ifkorrekturansicht
    \section*{\indexname}%
  \else
    \subsubsection*{Index der erwähnten Entitäten}%
  \fi
  \setlength{\parindent}{0pt}%
  \setlength{\parskip}{0pt plus 0.3pt}%
  \let\item\@idxitem
}{%
  \ifkorrekturansicht\clearpage\fi
}
\makeatother

\IfFileExists{\jobname-pw.ind}{\input{\jobname-pw.ind}}{}

% Quellenangabe nur in der Leseansicht
\ifkorrekturansicht\else
% Fallback-Definitionen, falls die .tex-Datei \titel etc. nicht gesetzt hat
\providecommand{\titel}{}
\providecommand{\editorInnen}{}
\providecommand{\dateiname}{\jobname}

\vspace{3cm}

\vfill

\footnotesize
\textsc{Quelle}: \titel. Herausgegeben von {\editorInnen}. In: \emph{Arthur Schnitzler: Briefwechsel mit Autorinnen und Autoren}.
 Digitale Edition, https://schnitzler-briefe.acdh.oeaw.ac.at/{\dateiname}.html (Stand \today)
\fi

\end{document}


