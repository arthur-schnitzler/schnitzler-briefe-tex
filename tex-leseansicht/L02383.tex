%% latex-leseansicht-vorspann.tex
%% Vorspann für die Leseansicht.
%% Lädt die gemeinsame Datei latex-vorspann.tex mit nicht gesetztem Schalter.

\newif\ifkorrekturansicht
\korrekturansichtfalse

\input{../tex-inputs/latex-vorspann}


         
         \renewcommand{\erwaehntePersonen}{Personen:  ?? [Köchin von Georg Brandes], Louise Schnitzler,  Sokrates}
         \renewcommand{\erwaehnteOrte}{Orte: Athen, Frankreich, Griechenland, Griechisches Parlament, Izmir, Kopenhagen, Nationale und Kapodistrias-Universität Athen, Wien}
         \renewcommand{\erwaehnteWerke}{}
               \section[Georg Brandes an Arthur Schnitzler, 15. 5. 1922]{ Georg Brandes an Arthur Schnitzler, 15. 5. 1922}\nopagebreak\mylabel{v}\rehead{ }\begin{ledgroupsized}[t]{13cm}\normalsize\beginnumbering \toendnotes[C]{\smallbreak\pagebreak[2]} \buchAlsQuelle{Georg Brandes, Arthur Schnitzler: \emph{Ein Briefwechsel}. Hg. Kurt Bergel. Bern: \emph{Francke} 1956, S. 136.}\toendnotes[C]{\smallbreak}\pstart
           \raggedleft{}{\pb}Kopenhagen\oindex{Kopenhagen@\textbf{Kopenhagen}|pw}, 15. Mai 1922\pend
           \pstart{}Mein lieber Freund\pend\pstart
           Im Jahre 1898 saß ich an diesem Tag an Ihrem Tisch in einem kleinen
               Kreis; auch Ihre Frau Mutter\pwindex{Schnitzler, Louise 1840-07-08 – 1911-09-09@\textsc{Schnitzler, Louise} (1840-07-08 – 1911-09-09)|pwv}
               war damals anwesend. Ich sagte das wenig geistvolle Wort: »Sie sind also gerade
               20 Jahre jünger als ich« und Sie antworteten lächelnd: »Und ich denke, wir werden
               auch in der Zukunft denselben Abstand von einander innehalten.«\pend
           \pstart
           Wir haben es also noch 20 Jahren gethan. Daß ich Ihnen Glück wünsche, versteht sich
               von selbst, wenn dieser mythologische Begriff sonst einen Sinn hat; ich wünsche Ihnen
               jedenfalls alles Gute, und ich danke Ihnen von Herzen für das, was Sie 30 Jahre
               hindurch mir gewesen sind, eine stets rinnende Quelle geistiger Genüsse, ja mehr als
               das: Sie haben mir das so seltene Gefühl gegeben, \emph{\label{K_L02383_1v}\edtext{in der Ferne einen congenialen
                     Freund}{\lemma{\textnormal{\emph{in … Freund}}}\Cendnote{\textnormal{Das Original dieses
                     Korrespondenzstücks ist verschollen. Es fehlt auch in den Abschriften, die vom
                     Briefwechsel gemacht wurden. Der Text folgt der Buchausgabe, die diese
                     Unterstreichung Schnitzler\pwindex{Schnitzler, Arthur 15.05.1862 – 21.10.1931@\textsc{Schnitzler, Arthur} (15.05.1862 – 21.10.1931), \emph{Schriftsteller, Mediziner}|pwk} zuschreibt,
                     zugleich aber durch Kursivsetzung in den edierten Text übernimmt.}}}\label{K_L02383_1h}} zu haben.\pend
           \pstart
           Als ich von meiner vierteljährigen Abwesenheit hier ankam, wurde mir allgemein
               gesagt, Sie würden am 11. Mai hier sein und hier einen Vortrag halten.
               Ich hatte schon gründlich überlegt, ob meine Köchin\pwindex{?? [Koechin von Georg Brandes] @\textsc{?? [Köchin von Georg Brandes]}|pwv} gut genug sei und welches Restaurant wir für Sie und
               mich und einige Freunde die beste vorkäme, und nun sind Sie nicht da und ich weiß
               nicht den Grund. Es ist eine arge Enttäuschung. Weshalb sind Sie nicht gekommen?
               Unsere Zeitungen sagen es nicht.\pend
           \pstart
           Ich war in Griechenland\oindex{Griechenland@\textbf{Griechenland}|pw}. Mir wurde in Athen\oindex{Athen@\textbf{Athen}|pw} viel Freundlichkeit erwiesen. Was nicht Sokrates\pwindex{Sokrates 469 v. u. Z. – 399 v. u. Z.@\textsc{Sokrates} (469 v. u. Z. – 399 v. u. Z.), \emph{Philosoph}|pw} gelang, geschah mir; ich wurde im
                  \label{K_L02383_2v}\edtext{Prytaneion\oindex{Griechisches Parlament@\textbf{Griechisches Parlament}|pwv}}{\lemma{\textnormal{\emph{Prytaneion}}}\Cendnote{\textnormal{im antiken Griechenland\oindex{Griechenland@\textbf{Griechenland}|pwk}: Regierungssitz.}}}\label{K_L02383_2h} versorgt. Da die
               Regierung erfuhr, ich sei in Athen\oindex{Athen@\textbf{Athen}|pw} – in den
               ersten Tagen kannte ich keinen Menschen – ließ sie mich wissen, sie räume mir drei
               schöne Zimmer mit Badezimmer ein; ich darf weder für Essen noch für Wein das
               geringste zahlen. Sogar meine Wäsche werden bezahlt, meine Wagen etc. Und in großer
               öffentlicher Sitzung wo schöne Reden gehalten wurden, machte die Universität\oindex{Nationale und Kapodistrias-Universitaet Athen@\textbf{Nationale und Kapodistrias-Universität Athen}|pw} in Gegenwart der Minister, der Prinzen, der
               Professoren und Studenten mich zum Ehrendoctor. Die jungen Studentinnen (meistens aus
                  Smyrna\oindex{Izmir@\textbf{Izmir}|pw}) warfen Rosenblätter über mich. Das
               war ein südländischer Feier. Glücklicherweise redete ich ganz gut – die
               anderen sprechen neugriechisch\oindex{Griechenland@\textbf{Griechenland}|pw} und altgriechisch\oindex{Griechenland@\textbf{Griechenland}|pw}, ich französisch\oindex{Frankreich@\textbf{Frankreich}|pw}.\pend
           \pstart
           Nun bin ich einsam hier, erwartete Sie, und Sie kommen nicht.\pend
           \pstart Ihr \spacefill\mbox{Georg Brandes}\pend{}
         
         \endnumbering\mylabel{h}\end{ledgroupsized}  \newcommand{\dateiname}{L02383}\newcommand{\titel}{Georg Brandes an Arthur Schnitzler, 15. 5. 1922}\newcommand{\editorInnen}{Martin Anton Müller und Gerd-Hermann Susen}%% latex-leseansicht-abspann.tex
%% Abspann für die Leseansicht.
%% Der Schalter \ifkorrekturansicht ist bereits durch den Vorspann gesetzt.

%% latex-abspann.tex
%% Gemeinsamer Abspann für Korrekturansicht und Leseansicht.
%% Setzt den Schalter \ifkorrekturansicht voraus (gesetzt in den
%% einbindenden Dateien latex-korrekturansicht-abspann.tex bzw.
%% latex-leseansicht-abspann.tex).
%% ---------------------------------------------------------------

\normalsize

% Das esempio-Environment wird nur in der Leseansicht benötigt
\ifkorrekturansicht\else
\newenvironment{esempio}[3]%
{
    \vspace{1.5ex}
    \rlap{\underline{#1}}
    \par
    \setlength{\parindent}{0cm}
    \nopagebreak
    \leftskip=#2cm
    \rightskip=#3cm
}
{
    \par
}
\fi

\doendnotes{C}
\bigskip
\vfill

\clearpage

\footnotesize

\ifkorrekturansicht
  \lohead{\textsc{register}}
\fi

% theindex-Environment neu definieren ohne reledmac
\makeatletter
\renewenvironment{theindex}{%
  \ifkorrekturansicht
    \section*{\indexname}%
  \else
    \subsubsection*{Index der erwähnten Entitäten}%
  \fi
  \setlength{\parindent}{0pt}%
  \setlength{\parskip}{0pt plus 0.3pt}%
  \let\item\@idxitem
}{%
  \ifkorrekturansicht\clearpage\fi
}
\makeatother

\IfFileExists{\jobname-pw.ind}{\input{\jobname-pw.ind}}{}

% Quellenangabe nur in der Leseansicht
\ifkorrekturansicht\else
% Fallback-Definitionen, falls die .tex-Datei \titel etc. nicht gesetzt hat
\providecommand{\titel}{}
\providecommand{\editorInnen}{}
\providecommand{\dateiname}{\jobname}

\vspace{3cm}

\vfill

\footnotesize
\textsc{Quelle}: \titel. Herausgegeben von {\editorInnen}. In: \emph{Arthur Schnitzler: Briefwechsel mit Autorinnen und Autoren}.
 Digitale Edition, https://schnitzler-briefe.acdh.oeaw.ac.at/{\dateiname}.html (Stand \today)
\fi

\end{document}


      