%% latex-leseansicht-vorspann.tex
%% Vorspann für die Leseansicht.
%% Lädt die gemeinsame Datei latex-vorspann.tex mit nicht gesetztem Schalter.

\newif\ifkorrekturansicht
\korrekturansichtfalse

\input{../tex-inputs/latex-vorspann}


\section[Paula Dehmel an Arthur Schnitzler, 4. 1. 1903]{L01262 Paula Dehmel an Arthur Schnitzler, 4. 1. 1903}
\nopagebreak\mylabel{L01262v}
\rehead{ }\normalsize\beginnumbering\briefempfaengerindex{Schnitzler, Arthur@\textsc{Schnitzler, Arthur}!zzzDehmel, Paula@\emph{von Paula Dehmel}!1903-01-042@{4. 1. 1903}|(be}
\toendnotes[C]{\smallbreak\pagebreak[2]}
\correspDesc{Versand  durch Paula Dehmel am 4. 1. 1903 in Wilmersdorf
\newline{}Erhalt  durch Arthur Schnitzler am 7. 1. 1903 in Wien}\toendnotes[C]{\smallbreak}
\Standort{CUL, Schnitzler, B 26.}
\physDesc{Postkarte, 401 Zeichen
\newline{}Handschrift: schwarze Tinte, lateinische Kurrent
\newline{}Versand: 1) Stempel: »\nobreak{}\oindex{Wilmersdorf@\textbf{Wilmersdorf}, \emph{Ehemaliger Ort}|pwk}Wilmersdorf bei Berlin, 6. 1. 03, 11–12V\nobreak{}«.   2) Stempel: »\nobreak{}\oindex{IX., Alsergrund@\textbf{IX., Alsergrund}, \emph{Verwaltungsgebiet}|pwk}9/3 Wien 72, 7. 1. 03, 1.N, Bestellt\nobreak{}«. }\toendnotes[C]{\smallbreak}\pstart{}{\pb}Herrn Arthur Schnitzler.\pend{}\pstart{}Schriftsteller\pend{}\pstart{}Wien\oindex{Wien@\textbf{Wien}, \emph{Verwaltungsgebiet}|pw}.\pend{}{\bigskip}\vspace{1em}
\pstart
           
\pstart
           {\pb}\textcolor{gray}{\textbf{PAULA DEHMEL}}\pend
           
\pstart
           \raggedleft{}\textcolor{gray}{\textbf{Wilhelmsaue 113\oindex{Wilhelmsaue@\textbf{Wilhelmsaue}, \emph{Straße}|pw} d}}en
                        4. 1. 03.\pend
           \pend
           
\pstart
           Wilmersdorf\oindex{Wilmersdorf@\textbf{Wilmersdorf}, \emph{Ehemaliger Ort}|pw}.\pend
           
\pstart{}Sehr verehrter Herr.\pend\vspace{0.5em}
\pstart
           Mahnerin spielen ist ein böses Amt, eher der Zweck heiligt die Mittel. Also: Bitte,
               bitte, denken Sie an meinen unglücklichen Freund\pwindex{Schlaf, Johannes 21.\,6.\,1862 Querfurt – 2.\,2.\,1941 ebd.@\textsc{Schlaf, Johannes} (21.\,6.\,1862 Querfurt – 2.\,2.\,1941 ebd.), \emph{Schriftsteller}|pwv}. Ein paar Monate hat er so ziemlich sorglos, wenn
               auch leider nicht gesund, verlebt, aber nun ist er wieder am Ende! Und ich kann nicht
               allein helfen.\pend
           \pstart \label{T_L01262-1v}\edtext{Mit ergebenstem Gruß. \spacefill\mbox{Paula
                  Dehmel}}{\lemma{\textnormal{\emph{Mit … Dehmel}}}\Cendnote{\textnormal{am oberen Rand auf dem Kopf}}}\label{T_L01262-1}\pend{}\selectlanguage{ngerman}\endnumbering\briefempfaengerindex{Schnitzler, Arthur@\textsc{Schnitzler, Arthur}!zzzDehmel, Paula@\emph{von Paula Dehmel}!1903-01-042@{4. 1. 1903}|)be}\mylabel{L01262h}  \newcommand{\dateiname}{L01262}\newcommand{\titel}{Paula Dehmel an Arthur Schnitzler, 4. 1. 1903}\newcommand{\editorInnen}{Martin Anton Müller und Gerd-Hermann Susen}%% latex-leseansicht-abspann.tex
%% Abspann für die Leseansicht.
%% Der Schalter \ifkorrekturansicht ist bereits durch den Vorspann gesetzt.

%% latex-abspann.tex
%% Gemeinsamer Abspann für Korrekturansicht und Leseansicht.
%% Setzt den Schalter \ifkorrekturansicht voraus (gesetzt in den
%% einbindenden Dateien latex-korrekturansicht-abspann.tex bzw.
%% latex-leseansicht-abspann.tex).
%% ---------------------------------------------------------------

\normalsize

% Das esempio-Environment wird nur in der Leseansicht benötigt
\ifkorrekturansicht\else
\newenvironment{esempio}[3]%
{
    \vspace{1.5ex}
    \rlap{\underline{#1}}
    \par
    \setlength{\parindent}{0cm}
    \nopagebreak
    \leftskip=#2cm
    \rightskip=#3cm
}
{
    \par
}
\fi

\doendnotes{C}
\bigskip
\vfill

\clearpage

\footnotesize

\ifkorrekturansicht
  \lohead{\textsc{register}}
\fi

% theindex-Environment neu definieren ohne reledmac
\makeatletter
\renewenvironment{theindex}{%
  \ifkorrekturansicht
    \section*{\indexname}%
  \else
    \subsubsection*{Index der erwähnten Entitäten}%
  \fi
  \setlength{\parindent}{0pt}%
  \setlength{\parskip}{0pt plus 0.3pt}%
  \let\item\@idxitem
}{%
  \ifkorrekturansicht\clearpage\fi
}
\makeatother

\IfFileExists{\jobname-pw.ind}{\input{\jobname-pw.ind}}{}

% Quellenangabe nur in der Leseansicht
\ifkorrekturansicht\else
% Fallback-Definitionen, falls die .tex-Datei \titel etc. nicht gesetzt hat
\providecommand{\titel}{}
\providecommand{\editorInnen}{}
\providecommand{\dateiname}{\jobname}

\vspace{3cm}

\vfill

\footnotesize
\textsc{Quelle}: \titel. Herausgegeben von {\editorInnen}. In: \emph{Arthur Schnitzler: Briefwechsel mit Autorinnen und Autoren}.
 Digitale Edition, https://schnitzler-briefe.acdh.oeaw.ac.at/{\dateiname}.html (Stand \today)
\fi

\end{document}


