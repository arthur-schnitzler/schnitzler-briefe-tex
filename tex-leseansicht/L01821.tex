%% latex-leseansicht-vorspann.tex
%% Vorspann für die Leseansicht.
%% Lädt die gemeinsame Datei latex-vorspann.tex mit nicht gesetztem Schalter.

\newif\ifkorrekturansicht
\korrekturansichtfalse

\input{../tex-inputs/latex-vorspann}


         
         \renewcommand{\erwaehntePersonen}{Personen: Karl Glossy, Hugo von Hofmannsthal, Felix von Oppenheimer}
         \renewcommand{\erwaehnteInstitutionen}{Institutionen: Österreichische Rundschau}
         \renewcommand{\erwaehnteOrte}{Orte: Wien}
         \renewcommand{\erwaehnteWerke}{Werke: Cristinas Heimreise. Komödie, Komödie in Prosa, Österreichische Rundschau}
               \section[Arthur Schnitzler an Hugo von Hofmannsthal, {[}15. 1. 1909?{]}]{ Arthur Schnitzler an Hugo von Hofmannsthal, {[}15. 1. 1909?{]}}\nopagebreak\mylabel{v}\rehead{ }\begin{ledgroupsized}[t]{13cm}\normalsize\beginnumbering \toendnotes[C]{\smallbreak\pagebreak[2]} \Standort{FDH, Hs-30885,142.}
\physDesc{Brief, 1 Blatt, 1 Seite, 200 Zeichen
\newline{}Handschrift: schwarze Tinte, deutsche Kurrent
\newline{}Ordnung: von Schnitzler – mutmaßlich bei der Durchsicht der Briefe
                                    1929 – datiert: »1910?« }\buchAbdrucke{\weitereDrucke{Hugo von Hofmannsthal, Arthur Schnitzler: \emph{Briefwechsel}. Hg. Therese Nickl und Heinrich Schnitzler. Frankfurt am Main: \emph{S. Fischer} 1964, S. 259.} }\toendnotes[C]{\smallbreak}\pstart
           \noindent{}{\pb}Ja richtig, eine Frage – we{\geminationn} Sie glauben ſie beantworten zu dürfen: wieviel haben Sie von der Oest. Rundſchau\orgindex{Oesterreichische Rundschau@Österreichische Rundschau|pw} für den \label{K_L01821_1v}\edtext{\textsc{Cristina-Act}\pwindex{Hofmannsthal, Hugo von 1874-02-01 – 1929-07-15@\textsc{Hofmannsthal, Hugo von} (1874-02-01 – 1929-07-15), \emph{Schriftsteller}!Cristinas Heimreise. Komoedie11. 2. 1910@\strich\emph{Cristinas Heimreise. Komödie} {[}11. 2. 1910{]}|pw}}{\lemma{\textnormal{\emph{Cristina-Act}}}\Cendnote{\textnormal{Hugo von Hofmannsthal\pwindex{Hofmannsthal, Hugo von 1874-02-01 – 1929-07-15@\textsc{Hofmannsthal, Hugo von} (1874-02-01 – 1929-07-15), \emph{Schriftsteller}|pwk}: \emph{Komödie in Prosa}\pwindex{?? Werk@Nicht ermittelte Verfasserinnen und Verfasser!Komoedie in Prosa1. 1. 1909@\emph{Komödie in Prosa} {[}1. 1. 1909{]}|pwk}. In: \emph{Österreichische Rundschau}\pwindex{Oesterreichische Rundschau1904 – 1924@\emph{Österreichische Rundschau} {[}1904 – 1924{]}|pwk}, Bd. 18, H. 1, 1. 1. 1909,
                     S. 11–23.}}}\label{K_L01821_1h} Honorar gekriegt? (Weil ich ihnen nemlich auch einen
               erſten \label{K_L01821_2v}\edtext{Act geben}{\lemma{\textnormal{\emph{Act geben}}}\Cendnote{\textnormal{Schnitzler\pwindex{Schnitzler, Arthur 15.05.1862 – 21.10.1931@\textsc{Schnitzler, Arthur} (15.05.1862 – 21.10.1931), \emph{Schriftsteller, Mediziner}|pwk}s Kontaktpersonen zur \emph{Österreichischen Rundschau}\orgindex{Oesterreichische Rundschau@Österreichische Rundschau|pwk} waren die beiden
                  Herausgeber Karl Glossy\pwindex{Glossy, Karl 07.03.1848 – 09.09.1937@\textsc{Glossy, Karl} (07.03.1848 – 09.09.1937), \emph{Schriftsteller, Museumsleiter, Zensurbeirat}|pwk} und Felix Oppenheimer\pwindex{Oppenheimer, Felix von 20.02.1874 – 15.11.1938@\textsc{Oppenheimer, Felix von} (20.02.1874 – 15.11.1938), \emph{Schriftsteller, Soziologe, Mäzen}|pwk}. Die nachweisbaren Kontakte
                     1910 sind zu Zeiten, an denen Hofmannsthal\pwindex{Hofmannsthal, Hugo von 1874-02-01 – 1929-07-15@\textsc{Hofmannsthal, Hugo von} (1874-02-01 – 1929-07-15), \emph{Schriftsteller}|pwk} sich gerade auf Reisen befindet. Eine solche formlose
                  Anfrage scheint damit eher unwahrscheinlich. Zwei Wochen nach Erscheinen des
                  teilweisen Vorabdrucks von \emph{Cristinas
                     Heimreise}\pwindex{Hofmannsthal, Hugo von 1874-02-01 – 1929-07-15@\textsc{Hofmannsthal, Hugo von} (1874-02-01 – 1929-07-15), \emph{Schriftsteller}!Cristinas Heimreise. Komoedie11. 2. 1910@\strich\emph{Cristinas Heimreise. Komödie} {[}11. 2. 1910{]}|pwk} (\emph{Komödie in Prosa}\pwindex{?? Werk@Nicht ermittelte Verfasserinnen und Verfasser!Komoedie in Prosa1. 1. 1909@\emph{Komödie in Prosa} {[}1. 1. 1909{]}|pwk}) – am
                     15. 1. 1909 –
                  vermerkt sich Schnitzler\pwindex{Schnitzler, Arthur 15.05.1862 – 21.10.1931@\textsc{Schnitzler, Arthur} (15.05.1862 – 21.10.1931), \emph{Schriftsteller, Mediziner}|pwk} den Besuch Oppenheimer\pwindex{Oppenheimer, Felix von 20.02.1874 – 15.11.1938@\textsc{Oppenheimer, Felix von} (20.02.1874 – 15.11.1938), \emph{Schriftsteller, Soziologe, Mäzen}|pwk}s, was mutmaßlich auch der
                  Ausgangspunkt für diese Überlegung darstellt. In der \emph{Österreichischen Rundschau}\pwindex{Oesterreichische Rundschau1904 – 1924@\emph{Österreichische Rundschau} {[}1904 – 1924{]}|pwk} erschien in Folge nichts von
                  Schnitzler.}}}\label{K_L01821_2h} will.)\pend
           
         
         \endnumbering\mylabel{h}\end{ledgroupsized}  \newcommand{\dateiname}{L01821}\newcommand{\titel}{Arthur Schnitzler an Hugo von Hofmannsthal, [15. 1. 1909?]}\newcommand{\editorInnen}{Martin Anton Müller und Gerd-Hermann Susen}%% latex-leseansicht-abspann.tex
%% Abspann für die Leseansicht.
%% Der Schalter \ifkorrekturansicht ist bereits durch den Vorspann gesetzt.

%% latex-abspann.tex
%% Gemeinsamer Abspann für Korrekturansicht und Leseansicht.
%% Setzt den Schalter \ifkorrekturansicht voraus (gesetzt in den
%% einbindenden Dateien latex-korrekturansicht-abspann.tex bzw.
%% latex-leseansicht-abspann.tex).
%% ---------------------------------------------------------------

\normalsize

% Das esempio-Environment wird nur in der Leseansicht benötigt
\ifkorrekturansicht\else
\newenvironment{esempio}[3]%
{
    \vspace{1.5ex}
    \rlap{\underline{#1}}
    \par
    \setlength{\parindent}{0cm}
    \nopagebreak
    \leftskip=#2cm
    \rightskip=#3cm
}
{
    \par
}
\fi

\doendnotes{C}
\bigskip
\vfill

\clearpage

\footnotesize

\ifkorrekturansicht
  \lohead{\textsc{register}}
\fi

% theindex-Environment neu definieren ohne reledmac
\makeatletter
\renewenvironment{theindex}{%
  \ifkorrekturansicht
    \section*{\indexname}%
  \else
    \subsubsection*{Index der erwähnten Entitäten}%
  \fi
  \setlength{\parindent}{0pt}%
  \setlength{\parskip}{0pt plus 0.3pt}%
  \let\item\@idxitem
}{%
  \ifkorrekturansicht\clearpage\fi
}
\makeatother

\IfFileExists{\jobname-pw.ind}{\input{\jobname-pw.ind}}{}

% Quellenangabe nur in der Leseansicht
\ifkorrekturansicht\else
% Fallback-Definitionen, falls die .tex-Datei \titel etc. nicht gesetzt hat
\providecommand{\titel}{}
\providecommand{\editorInnen}{}
\providecommand{\dateiname}{\jobname}

\vspace{3cm}

\vfill

\footnotesize
\textsc{Quelle}: \titel. Herausgegeben von {\editorInnen}. In: \emph{Arthur Schnitzler: Briefwechsel mit Autorinnen und Autoren}.
 Digitale Edition, https://schnitzler-briefe.acdh.oeaw.ac.at/{\dateiname}.html (Stand \today)
\fi

\end{document}


      