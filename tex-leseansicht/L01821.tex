%% latex-leseansicht-vorspann.tex
%% Vorspann für die Leseansicht.
%% Lädt die gemeinsame Datei latex-vorspann.tex mit nicht gesetztem Schalter.

\newif\ifkorrekturansicht
\korrekturansichtfalse

\input{../tex-inputs/latex-vorspann}


\section[Arthur Schnitzler an Hugo von Hofmannsthal, {{[}}15. 1. 1909?{{]}}]{L01821 Arthur Schnitzler an Hugo von Hofmannsthal, {[}15. 1. 1909?{]}}
\nopagebreak\mylabel{L01821v}
\rehead{ }\normalsize\beginnumbering\briefempfaengerindex{Hofmannsthal, Hugo von@\textsc{Hofmannsthal, Hugo von}!zzzSchnitzler, Arthur@\emph{von Arthur Schnitzler}!1909-01-151@{{[}15. 1. 1909?{]}}|(be}
\toendnotes[C]{\smallbreak\pagebreak[2]}
\correspDesc{Versand  durch Arthur Schnitzler am [15. 1. 1909?] in Wien
\newline{}Erhalt  durch Hugo von Hofmannsthal im Zeitraum [15. 1. 1909
                  – 19. 1. 1909?] \textbf{Ort fehlend} }\toendnotes[C]{\smallbreak}
\Standort{FDH, Hs-30885,142.}
\physDesc{Brief, 1 Blatt, 1 Seite, 200 Zeichen
\newline{}Handschrift: schwarze Tinte, deutsche Kurrent
\newline{}Ordnung: von Schnitzler – mutmaßlich bei der Durchsicht der Briefe
                                    1929 – datiert: »1910?« }
\buchAbdrucke{\weitereDrucke{Hugo von Hofmannsthal, Arthur Schnitzler: \emph{Briefwechsel}. Herausgegeben von Therese Nickl und Heinrich Schnitzler. Frankfurt am Main: \emph{S. Fischer} 1964, S. 259.} }\toendnotes[C]{\smallbreak}
\pstart
           \noindent{}{\pb}Ja richtig, eine Frage – we{\geminationn} Sie glauben{ }ſie beantworten zu dürfen: wieviel haben Sie von der Oest. Rundſchau\orgindex{Österreichische Rundschau@Österreichische Rundschau|pw} für den \label{K_L01821-1v}\edtext{\textsc{Cristina-Act}\pwindex{Hofmannsthal, Hugo von 1.\,2.\,1874 Wien – 15.\,7.\,1929 Rodaun@\textsc{Hofmannsthal, Hugo von} (1.\,2.\,1874 Wien – 15.\,7.\,1929 Rodaun), \emph{Schriftsteller}!Cristinas Heimreise. Komödie@\strich\emph{Cristinas Heimreise. Komödie}|pw}}{\lemma{\textnormal{\emph{Cristina-Act}}}\Cendnote{\textnormal{Hugo von Hofmannsthal\pwindex{Hofmannsthal, Hugo von 1.\,2.\,1874 Wien – 15.\,7.\,1929 Rodaun@\textsc{Hofmannsthal, Hugo von} (1.\,2.\,1874 Wien – 15.\,7.\,1929 Rodaun), \emph{Schriftsteller}|pwk}: \emph{Komödie in Prosa}\pwindex{Komödie in Prosa@\emph{Komödie in Prosa}|pwk}. In: \emph{Österreichische Rundschau}\pwindex{Österreichische Rundschau@\emph{Österreichische Rundschau}|pwk}, Bd. 18, H. 1, 1. 1. 1909,
                     S. 11–23.}}}\label{K_L01821-1} Honorar gekriegt? (Weil ich ihnen nemlich auch einen
               erſten \label{K_L01821-2v}\edtext{Act geben}{\lemma{\textnormal{\emph{Act geben}}}\Cendnote{\textnormal{Schnitzlers Kontaktpersonen zur \emph{Österreichischen Rundschau}\orgindex{Österreichische Rundschau@Österreichische Rundschau|pwk} waren die beiden
                  Herausgeber Karl Glossy\pwindex{Glossy, Karl 7.\,3.\,1848 Wien – 9.\,9.\,1937 ebd.@\textsc{Glossy, Karl} (7.\,3.\,1848 Wien – 9.\,9.\,1937 ebd.), \emph{Schriftsteller, Museumsleiter, Zensurbeirat}|pwk} und Felix Oppenheimer\pwindex{Oppenheimer, Felix von 20.\,2.\,1874 Wien – 15.\,11.\,1938 ebd.@\textsc{Oppenheimer, Felix von} (20.\,2.\,1874 Wien – 15.\,11.\,1938 ebd.), \emph{Schriftsteller, Soziologe, Mäzen}|pwk}. Die nachweisbaren Kontakte
                     1910 sind zu Zeiten, an denen Hofmannsthal\pwindex{Hofmannsthal, Hugo von 1.\,2.\,1874 Wien – 15.\,7.\,1929 Rodaun@\textsc{Hofmannsthal, Hugo von} (1.\,2.\,1874 Wien – 15.\,7.\,1929 Rodaun), \emph{Schriftsteller}|pwk} sich gerade auf Reisen befindet. Eine solche formlose
                  Anfrage scheint damit eher unwahrscheinlich. Zwei Wochen nach Erscheinen des
                  teilweisen Vorabdrucks von \emph{Cristinas
                     Heimreise}\pwindex{Hofmannsthal, Hugo von 1.\,2.\,1874 Wien – 15.\,7.\,1929 Rodaun@\textsc{Hofmannsthal, Hugo von} (1.\,2.\,1874 Wien – 15.\,7.\,1929 Rodaun), \emph{Schriftsteller}!Cristinas Heimreise. Komödie@\strich\emph{Cristinas Heimreise. Komödie}|pwk} (\emph{Komödie in Prosa}\pwindex{Komödie in Prosa@\emph{Komödie in Prosa}|pwk}) – am
                     15. 1. 1909 –
                  vermerkt sich Schnitzler den Besuch Oppenheimers\pwindex{Oppenheimer, Felix von 20.\,2.\,1874 Wien – 15.\,11.\,1938 ebd.@\textsc{Oppenheimer, Felix von} (20.\,2.\,1874 Wien – 15.\,11.\,1938 ebd.), \emph{Schriftsteller, Soziologe, Mäzen}|pwk}, was mutmaßlich auch der
                  Ausgangspunkt für diese Überlegung darstellt. In der \emph{Österreichischen Rundschau}\pwindex{Österreichische Rundschau@\emph{Österreichische Rundschau}|pwk} erschien in Folge nichts von
                  Schnitzler.}}}\label{K_L01821-2} will.)\pend
           \selectlanguage{ngerman}\endnumbering\briefempfaengerindex{Hofmannsthal, Hugo von@\textsc{Hofmannsthal, Hugo von}!zzzSchnitzler, Arthur@\emph{von Arthur Schnitzler}!1909-01-151@{{[}15. 1. 1909?{]}}|)be}\mylabel{L01821h}  \newcommand{\dateiname}{L01821}\newcommand{\titel}{Arthur Schnitzler an Hugo von Hofmannsthal, [15. 1. 1909?]}\newcommand{\editorInnen}{Martin Anton Müller und Gerd-Hermann Susen}%% latex-leseansicht-abspann.tex
%% Abspann für die Leseansicht.
%% Der Schalter \ifkorrekturansicht ist bereits durch den Vorspann gesetzt.

%% latex-abspann.tex
%% Gemeinsamer Abspann für Korrekturansicht und Leseansicht.
%% Setzt den Schalter \ifkorrekturansicht voraus (gesetzt in den
%% einbindenden Dateien latex-korrekturansicht-abspann.tex bzw.
%% latex-leseansicht-abspann.tex).
%% ---------------------------------------------------------------

\normalsize

% Das esempio-Environment wird nur in der Leseansicht benötigt
\ifkorrekturansicht\else
\newenvironment{esempio}[3]%
{
    \vspace{1.5ex}
    \rlap{\underline{#1}}
    \par
    \setlength{\parindent}{0cm}
    \nopagebreak
    \leftskip=#2cm
    \rightskip=#3cm
}
{
    \par
}
\fi

\doendnotes{C}
\bigskip
\vfill

\clearpage

\footnotesize

\ifkorrekturansicht
  \lohead{\textsc{register}}
\fi

% theindex-Environment neu definieren ohne reledmac
\makeatletter
\renewenvironment{theindex}{%
  \ifkorrekturansicht
    \section*{\indexname}%
  \else
    \subsubsection*{Index der erwähnten Entitäten}%
  \fi
  \setlength{\parindent}{0pt}%
  \setlength{\parskip}{0pt plus 0.3pt}%
  \let\item\@idxitem
}{%
  \ifkorrekturansicht\clearpage\fi
}
\makeatother

\IfFileExists{\jobname-pw.ind}{\input{\jobname-pw.ind}}{}

% Quellenangabe nur in der Leseansicht
\ifkorrekturansicht\else
% Fallback-Definitionen, falls die .tex-Datei \titel etc. nicht gesetzt hat
\providecommand{\titel}{}
\providecommand{\editorInnen}{}
\providecommand{\dateiname}{\jobname}

\vspace{3cm}

\vfill

\footnotesize
\textsc{Quelle}: \titel. Herausgegeben von {\editorInnen}. In: \emph{Arthur Schnitzler: Briefwechsel mit Autorinnen und Autoren}.
 Digitale Edition, https://schnitzler-briefe.acdh.oeaw.ac.at/{\dateiname}.html (Stand \today)
\fi

\end{document}


