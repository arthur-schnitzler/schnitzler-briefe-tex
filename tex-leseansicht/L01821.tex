%% latex-korrekturansicht-vorspann.tex
%% Vorspann für die Korrekturansicht.
%% Lädt die gemeinsame Datei latex-vorspann.tex mit gesetztem Schalter.

\newif\ifkorrekturansicht
\korrekturansichttrue

\input{../tex-inputs/latex-vorspann}


\section[Arthur Schnitzler an Hugo von Hofmannsthal, {[}15. 1. 1909?{]}]{L01821 Arthur Schnitzler an Hugo von Hofmannsthal, {[}15. 1. 1909?{]}}
\nopagebreak\mylabel{L01821v}
\rehead{ }\normalsize\beginnumbering\briefempfaengerindex{Hofmannsthal, Hugo von@\textsc{Hofmannsthal, Hugo von}!zzzSchnitzler, Arthur@\emph{von Arthur Schnitzler}!1909-01-151@{{[}15. 1. 1909?{]}}|(be}
\toendnotes[C]{\smallbreak\pagebreak[2]}\Standort{FDH, Hs-30885,142.}
\physDesc{Brief, 1 Blatt, 1 Seite, 200 Zeichen
\newline{}Handschrift: schwarze Tinte, deutsche Kurrent
\newline{}Ordnung: von Schnitzler – mutmaßlich bei der Durchsicht der Briefe
                                    1929 – datiert: »1910?« }
\buchAbdrucke{\weitereDrucke{Hugo von Hofmannsthal, Arthur Schnitzler: \emph{Briefwechsel}. Frankfurt am Main: \emph{S. Fischer} 1964, S. 259.} }\toendnotes[C]{\smallbreak}
\pstart
           \noindent{}{\pb}Ja richtig, eine Frage – we{\geminationn} Sie glauben ſie beantworten zu dürfen: wieviel haben Sie von der Oest. Rundſchau\orgindex{Oesterreichische Rundschau@Österreichische Rundschau|pw} für den \label{K_L01821-1v}\edtext{\textsc{Cristina-Act}\pwindex{Cristinas Heimreise. Komoedie@\emph{Cristinas Heimreise. Komödie}|pw}}{\lemma{\textnormal{\emph{Cristina-Act}}}\Cendnote{\textnormal{Hugo von Hofmannsthal\pwindex{Hofmannsthal, Hugo von 1874-02-01 – 1929-07-15@\textsc{Hofmannsthal, Hugo von} (1874-02-01 – 1929-07-15), \emph{Schriftsteller/Schriftstellerin}|pwk}: \emph{Komödie in Prosa}\pwindex{Komoedie in Prosa@\emph{Komödie in Prosa}|pwk}. In: \emph{Österreichische Rundschau}\pwindex{Oesterreichische Rundschau@\emph{Österreichische Rundschau}|pwk}, Bd. 18, H. 1, 1. 1. 1909,
                     S. 11–23.}}}\label{K_L01821-1} Honorar gekriegt? (Weil ich ihnen nemlich auch einen
               erſten \label{K_L01821-2v}\edtext{Act geben}{\lemma{\textnormal{\emph{Act geben}}}\Cendnote{\textnormal{Schnitzlers Kontaktpersonen zur \emph{Österreichischen Rundschau}\orgindex{Oesterreichische Rundschau@Österreichische Rundschau|pwk} waren die beiden
                  Herausgeber Karl Glossy\pwindex{Glossy, Karl 07.03.1848 – 09.09.1937@\textsc{Glossy, Karl} (07.03.1848 – 09.09.1937), \emph{Schriftsteller/Schriftstellerin, Museumsleiter/Museumsleiterin, Zensurbeirat/Zensurbeirätin}|pwk} und Felix Oppenheimer\pwindex{Oppenheimer, Felix von 20.02.1874 – 15.11.1938@\textsc{Oppenheimer, Felix von} (20.02.1874 – 15.11.1938), \emph{Schriftsteller/Schriftstellerin, Soziologe/Soziologin, Mäzen/Mäzenin}|pwk}. Die nachweisbaren Kontakte
                     1910 sind zu Zeiten, an denen Hofmannsthal\pwindex{Hofmannsthal, Hugo von 1874-02-01 – 1929-07-15@\textsc{Hofmannsthal, Hugo von} (1874-02-01 – 1929-07-15), \emph{Schriftsteller/Schriftstellerin}|pwk} sich gerade auf Reisen befindet. Eine solche formlose
                  Anfrage scheint damit eher unwahrscheinlich. Zwei Wochen nach Erscheinen des
                  teilweisen Vorabdrucks von \emph{Cristinas
                     Heimreise}\pwindex{Cristinas Heimreise. Komoedie@\emph{Cristinas Heimreise. Komödie}|pwk} (\emph{Komödie in Prosa}\pwindex{Komoedie in Prosa@\emph{Komödie in Prosa}|pwk}) – am
                     15. 1. 1909 –
                  vermerkt sich Schnitzler den Besuch Oppenheimers\pwindex{Oppenheimer, Felix von 20.02.1874 – 15.11.1938@\textsc{Oppenheimer, Felix von} (20.02.1874 – 15.11.1938), \emph{Schriftsteller/Schriftstellerin, Soziologe/Soziologin, Mäzen/Mäzenin}|pwk}, was mutmaßlich auch der
                  Ausgangspunkt für diese Überlegung darstellt. In der \emph{Österreichischen Rundschau}\pwindex{Oesterreichische Rundschau@\emph{Österreichische Rundschau}|pwk} erschien in Folge nichts von
                  Schnitzler.}}}\label{K_L01821-2} will.)\pend
           \selectlanguage{ngerman}\endnumbering\briefempfaengerindex{Hofmannsthal, Hugo von@\textsc{Hofmannsthal, Hugo von}!zzzSchnitzler, Arthur@\emph{von Arthur Schnitzler}!1909-01-151@{{[}15. 1. 1909?{]}}|)be}\mylabel{L01821h}  \normalsize

\doendnotes{C}
\bigskip
\vfill

\clearpage

\footnotesize

\lohead{\textsc{register}}

% Definiere theindex-Environment komplett neu ohne reledmac
\makeatletter
\renewenvironment{theindex}{%
  \section*{\indexname}%
  \setlength{\parindent}{0pt}%
  \setlength{\parskip}{0pt plus 0.3pt}%
  \let\item\@idxitem
}{%
  \clearpage
}
\makeatother

\IfFileExists{\jobname-pw.ind}{\input{\jobname-pw.ind}}{}

\end{document}

      