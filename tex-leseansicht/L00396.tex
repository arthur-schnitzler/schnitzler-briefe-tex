\input{../tex-inputs/latex-pdf-vorspann}
\begin{center}
            \textcolor{red}{ENTWURF. ENTZIFFERUNG NOCH NICHT KORREKTURGELESEN}
                      \end{center}
            
               \section[Max Burckhard an Arthur Schnitzler, {[}4. 11. 1894{]}]{ Max Burckhard an Arthur Schnitzler, {[}4. 11. 1894{]}}\nopagebreak\mylabel{v}\rehead{ }\begin{ledgroupsized}[t]{13cm}\normalsize\beginnumbering\briefempfaengerindex{Schnitzler, Arthur@\textsc{Schnitzler, Arthur}!zzzBurckhard, Max Eugen@\emph{von Max Eugen Burckhard}!1894-11-041@{{[}4. 11. 1894{]}}|(be} \toendnotes[C]{\smallbreak\pagebreak[2]} \Standort{CUL, Schnitzler, B 20.}
\physDesc{Briefkarte
\newline{}Handschrift: schwarze Tinte, deutsche Kurrent
\newline{}Schnitzler: mit Bleistift datiert: »4/11 94« \newline{}Ordnung: mit Bleistift von unbekannter Hand nummeriert:
                                 »3« }\toendnotes[C]{\smallbreak}\pstart{}{\pb}Sehr geehrter Herr Doctor!\pend\pstart
           Könnten Sie mir heute 1 Uhr im Bureau oder morgen ſo circa
                  3 Uhr in der \label{K_L00396_1v}\edtext{Wohnung}{\lemma{\textnormal{\emph{Wohnung}}}\Cendnote{\textnormal{Burckhard\pwindex{Burckhard, Max Eugen 14.07.1854 – 16.03.1912@\textsc{Burckhard, Max Eugen} (14.07.1854 – 16.03.1912), \emph{Schriftsteller, Rechtswissenschaftler, Theaterleiter}|pwk} ist im Adressverzeichnis \emph{Lehmann}\pwindex{?? Werk@Nicht ermittelte Verfasserinnen und Verfasser!Lehmann s Allgemeiner Wohnungs-Anzeiger1859 – 1942@\emph{Lehmann’s Allgemeiner Wohnungs-Anzeiger} {[}1859 – 1942{]}|pwk} von 1890 bis
                     1905 in der Frankgasse 1\oindex{Frankgasse@\textbf{Frankgasse}|pwk}
                     gelistet. Schnitzler\pwindex{Schnitzler, Arthur 15.05.1862 – 21.10.1931@\textsc{Schnitzler, Arthur} (15.05.1862 – 21.10.1931), \emph{Schriftsteller, Mediziner}|pwk} wohnt vom 15. 11. 1893 bis zum 11. 9. 1903 im selben Haus,
                  einen Stock tiefer.}}}\label{K_L00396_1h} das Vergnügen Ihres Beſuches machen?\pend
           \pstart
           Mit beſten Empfehlungen{\\[\baselineskip]}\spacefill\mbox{D\textsuperscript{r}Burckhard}\pend
           \leftskip=0em{}\endnumbering\briefempfaengerindex{Schnitzler, Arthur@\textsc{Schnitzler, Arthur}!zzzBurckhard, Max Eugen@\emph{von Max Eugen Burckhard}!1894-11-041@{{[}4. 11. 1894{]}}|)be}\mylabel{h}\end{ledgroupsized}  \newcommand{\dateiname}{L00396}\newcommand{\titel}{Max Burckhard an Arthur Schnitzler, [4. 11. 1894]}\newcommand{\editorInnen}{Martin Anton Müller und Gerd-Hermann Susen}\input{../tex-inputs/latex-pdf-abspann}
      