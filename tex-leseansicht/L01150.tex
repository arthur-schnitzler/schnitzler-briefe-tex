%% latex-leseansicht-vorspann.tex
%% Vorspann für die Leseansicht.
%% Lädt die gemeinsame Datei latex-vorspann.tex mit nicht gesetztem Schalter.

\newif\ifkorrekturansicht
\korrekturansichtfalse

\input{../tex-inputs/latex-vorspann}


         
         \renewcommand{\erwaehntePersonen}{Personen: Rosa Beer, Hermann Beer, Richard Beer-Hofmann, Paul Goldmann, Julius Schnitzler, Helene Schnitzler, Gustav Schwarzkopf}
         \renewcommand{\erwaehnteOrte}{Orte: Karersee, Lago di Garda, Pörtschach, Seiser Alm, Tiers, Torbole sul Garda, Vahrn, Villa Arnstein, Wien, Wörthersee}
         \renewcommand{\erwaehnteWerke}{Werke: Der Graf von Charolais. Ein Trauerspiel, Der einsame Weg. Schauspiel in fünf Akten, Professor Bernhardi. Komödie in fünf Akten}
               \section[Arthur Schnitzler an Richard Beer-Hofmann, 22. 7. 1901]{ Arthur Schnitzler an Richard Beer-Hofmann, 22. 7. 1901}\nopagebreak\mylabel{v}\rehead{ }\begin{ledgroupsized}[t]{13cm}\normalsize\beginnumbering \toendnotes[C]{\smallbreak\pagebreak[2]} \Standort{YCGL, MSS 31.}
\physDesc{Brief, 1 Blatt, 3 Seiten, Umschlag
\newline{}Handschrift: 1) Bleistift, deutsche Kurrent\hspace{1em}2) schwarze Tinte, lateinische Kurrent (\noindent{}Umschlag)\hspace{1em}\newline{}Versand: 1) Stempel: »\nobreak{}\oindex{Vahrn@\textbf{Vahrn}|pwk}Vahrn, 22. 7. 01\nobreak{}«.   2) Stempel: »\nobreak{}\oindex{Poertschach@\textbf{Pörtschach}|pwk}{\pb}Pörtschach am See, 23 7 01\nobreak{}«. }\buchAbdrucke{\weitereDrucke{Arthur Schnitzler, Richard Beer-Hofmann: \emph{Briefwechsel 1891–1931}. Hg. Konstanze Fliedl. Wien, Zürich: \emph{Europaverlag} 1992, S. 153–154.} }\toendnotes[C]{\smallbreak}\pstart{}{\pb}Herrn Dr. Rich. Beer-Hofmann\pend{}\pstart{}Pörtschach\oindex{Poertschach@\textbf{Pörtschach}|pw}\pend{}\pstart{}Villa Arnstein\oindex{Villa Arnstein@\textbf{Villa Arnstein}|pw}.\pend{}{\bigskip}\pstart
           \raggedleft{}{\pb}\textsc{Vahrn}\oindex{Vahrn@\textbf{Vahrn}|pw}, 22/7 901\pend
           \pstart
           lieber Richard, von dem Tod Ihrer Stiefmama\pwindex{Beer, Rosa 20.7.1847 – 1.7.1901@\textsc{Beer, Rosa} (20.7.1847 – 1.7.1901)|pwv} hab ich durch Schw.\pwindex{Schwarzkopf, Gustav 07.11.1853 – 13.11.1939@\textsc{Schwarzkopf, Gustav} (07.11.1853 – 13.11.1939), \emph{Schriftsteller}|pw} erfahren, noch eh Sie mirs ſchrieben, zu formeller Condolenz wars zu
               ſpät, bitte ſagen Sie Ihrem Papa\pwindex{Beer, Hermann 10.8.1835 – 03.10.1902@\textsc{Beer, Hermann} (10.8.1835 – 03.10.1902), \emph{Rechtsanwalt}|pwv} nachträglich, daſs ich ihm meine herzliche Theilnahme alſo lieber durch
               Sie ausdrücken laſſe. – Paul\pwindex{Goldmann, Paul 31.01.1865 – 25.09.1935@\textsc{Goldmann, Paul} (31.01.1865 – 25.09.1935), \emph{Schriftsteller, Journalist}|pw} dürfte ſchon in
               den nächſten Tagen an den Wörtherſee\oindex{Woerthersee@\textbf{Wörthersee}|pw} kommen, iſt
               erbittert über Sie, will Sie gar nicht besuchen u. ſ. w. Schreiben Sie ihm doch noch
               eheſtens ein Wort. Vom Wörtherſee\oindex{Woerthersee@\textbf{Wörthersee}|pw} ko{\geminationm}t G.\pwindex{Goldmann, Paul 31.01.1865 – 25.09.1935@\textsc{Goldmann, Paul} (31.01.1865 – 25.09.1935), \emph{Schriftsteller, Journalist}|pw} herunter,
               ich muſs mir noch irgend was höheres ſuchen {\pb}werde
               mich auf der Seiſer Alpe\oindex{Seiser Alm@\textbf{Seiser Alm}|pw} u im Tierſer Thal\oindex{Tiers@\textbf{Tiers}|pw} umſehn. Machen Sie’s doch möglich
               auch zu kommen. Die letzten Sommertage denk’ ich Gardaſee\oindex{Lago di Garda@\textbf{Lago di Garda}|pw}, \textsc{ev}. Torbole\oindex{Torbole sul Garda@\textbf{Torbole sul Garda}|pw}? –\pend
           \pstart
           Ich find es hier ſehr angenehm, die Zimmer offenbar neu hergerichtet ſehr hübſch, das
               Eſſen gut, wenig Leut, und warm. Ich \substVorne{}\textsuperscript{\textcolor{gray}{×}\-\textcolor{gray}{×}\-\textcolor{gray}{×}\-\textcolor{gray}{×}\-\textcolor{gray}{×}\-\textcolor{gray}{×}}\substDazwischen{}ſchreibe\substHinten{} (3a.
               Stück\pwindex{Schnitzler, Arthur 15.05.1862 – 21.10.1931@\textsc{Schnitzler, Arthur} (15.05.1862 – 21.10.1931), \emph{Schriftsteller, Mediziner}!einsame Weg. Schauspiel in fuenf Akten1904@\strich\emph{Der einsame Weg. Schauspiel in fünf Akten} {[}1904{]}|pwv}\pwindex{Schnitzler, Arthur 15.05.1862 – 21.10.1931@\textsc{Schnitzler, Arthur} (15.05.1862 – 21.10.1931), \emph{Schriftsteller, Mediziner}!Professor Bernhardi. Komoedie in fuenf Akten1912@\strich\emph{Professor Bernhardi. Komödie in fünf Akten} {[}1912{]}|pwv}). An der Zerſtörung der »Grämlichkeit« wird von berufener Seite mit Talent
               gearbeitet. We{\geminationn} mich etwas ſtört, iſt es nur der Um{\pb}ſtand, daſs man in der betreffenden Familie Sie für
               den weitaus hervorragendſten von {\dots} hm {\dots} Alt-Wien\oindex{Wien@\textbf{Wien}|pw} hält, eine Meinung, die Sie
               hoffentlich durch Ihr {\dots} wieder hm {\dots} nächſtes Stück\pwindex{Beer-Hofmann, Richard 1866-07-11 – 1945-09-26@\textsc{Beer-Hofmann, Richard} (1866-07-11 – 1945-09-26), \emph{Schriftsteller}!Graf von Charolais. Ein Trauerspiel1904-12-23@\strich\emph{Der Graf von Charolais. Ein Trauerspiel} {[}1904-12-23{]}|pwv}
               endgiltig begraben werden.\pend
           \pstart
           – Schreiben Sie \introOben{}– we{\geminationn}\introOben{} bald, da{\geminationn} noch hieher, ſonſt Wien\oindex{Wien@\textbf{Wien}|pw}.\pend
           \pstart
           Heute Ausflug Karerſee\oindex{Karersee@\textbf{Karersee}|pw}, wo Julius\pwindex{Schnitzler, Julius 13.07.1865 – 29.06.1939@\textsc{Schnitzler, Julius} (13.07.1865 – 29.06.1939), \emph{Chirurg}|pw} u Frau\pwindex{Schnitzler, Helene 16.07.1871 – September 1941@\textsc{Schnitzler, Helene} (16.07.1871 – September 1941)|pwv}.\pend
           \pstart
           Gehts den Ihren gut? Baden Sie viel? Sehn Sie die übrigen Rundwohner?\pend
           \pstart Von Herzen Ihr \spacefill\mbox{Arthur}\pend{}
         
         \endnumbering\mylabel{h}\end{ledgroupsized}  \newcommand{\dateiname}{L01150}\newcommand{\titel}{Arthur Schnitzler an Richard Beer-Hofmann, 22. 7. 1901}\newcommand{\editorInnen}{Martin Anton Müller und Gerd-Hermann Susen}%% latex-leseansicht-abspann.tex
%% Abspann für die Leseansicht.
%% Der Schalter \ifkorrekturansicht ist bereits durch den Vorspann gesetzt.

%% latex-abspann.tex
%% Gemeinsamer Abspann für Korrekturansicht und Leseansicht.
%% Setzt den Schalter \ifkorrekturansicht voraus (gesetzt in den
%% einbindenden Dateien latex-korrekturansicht-abspann.tex bzw.
%% latex-leseansicht-abspann.tex).
%% ---------------------------------------------------------------

\normalsize

% Das esempio-Environment wird nur in der Leseansicht benötigt
\ifkorrekturansicht\else
\newenvironment{esempio}[3]%
{
    \vspace{1.5ex}
    \rlap{\underline{#1}}
    \par
    \setlength{\parindent}{0cm}
    \nopagebreak
    \leftskip=#2cm
    \rightskip=#3cm
}
{
    \par
}
\fi

\doendnotes{C}
\bigskip
\vfill

\clearpage

\footnotesize

\ifkorrekturansicht
  \lohead{\textsc{register}}
\fi

% theindex-Environment neu definieren ohne reledmac
\makeatletter
\renewenvironment{theindex}{%
  \ifkorrekturansicht
    \section*{\indexname}%
  \else
    \subsubsection*{Index der erwähnten Entitäten}%
  \fi
  \setlength{\parindent}{0pt}%
  \setlength{\parskip}{0pt plus 0.3pt}%
  \let\item\@idxitem
}{%
  \ifkorrekturansicht\clearpage\fi
}
\makeatother

\IfFileExists{\jobname-pw.ind}{\input{\jobname-pw.ind}}{}

% Quellenangabe nur in der Leseansicht
\ifkorrekturansicht\else
% Fallback-Definitionen, falls die .tex-Datei \titel etc. nicht gesetzt hat
\providecommand{\titel}{}
\providecommand{\editorInnen}{}
\providecommand{\dateiname}{\jobname}

\vspace{3cm}

\vfill

\footnotesize
\textsc{Quelle}: \titel. Herausgegeben von {\editorInnen}. In: \emph{Arthur Schnitzler: Briefwechsel mit Autorinnen und Autoren}.
 Digitale Edition, https://schnitzler-briefe.acdh.oeaw.ac.at/{\dateiname}.html (Stand \today)
\fi

\end{document}


      