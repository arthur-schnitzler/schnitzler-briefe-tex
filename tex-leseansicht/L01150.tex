%% latex-korrekturansicht-vorspann.tex
%% Vorspann für die Korrekturansicht.
%% Lädt die gemeinsame Datei latex-vorspann.tex mit gesetztem Schalter.

\newif\ifkorrekturansicht
\korrekturansichttrue

\input{../tex-inputs/latex-vorspann}


\section[Arthur Schnitzler an Richard Beer-Hofmann, 22. 7. 1901]{L01150 Arthur Schnitzler an Richard Beer-Hofmann, 22. 7. 1901}
\nopagebreak\mylabel{L01150v}
\rehead{ }\normalsize\beginnumbering\briefempfaengerindex{Beer-Hofmann, Richard@\textsc{Beer-Hofmann, Richard}!zzzSchnitzler, Arthur@\emph{von Arthur Schnitzler}!1901-07-221@{22. 7. 1901}|(be}
\toendnotes[C]{\smallbreak\pagebreak[2]}\Standort{YCGL, MSS 31.}
\physDesc{Brief, 1 Blatt, 3 Seiten, Umschlag, 1360 Zeichen
\newline{}Handschrift: 1) Bleistift, deutsche Kurrent\hspace{1em}2) schwarze Tinte, lateinische Kurrent (\noindent{}Umschlag)\hspace{1em}
\newline{}Versand: 1) Stempel: »\nobreak{}\oindex{Vahrn@\textbf{Vahrn}, \emph{P.PPLA3}|pwk}Vahrn, 22. 7. 01\nobreak{}«.   2) Stempel: »\nobreak{}\oindex{Poertschach am Woerthersee@\textbf{Pörtschach am Wörthersee}, \emph{P.PPL}|pwk}{\pb}Pörtschach am See, 23 7 01\nobreak{}«. }
\buchAbdrucke{\weitereDrucke{Arthur Schnitzler, Richard Beer-Hofmann: \emph{Briefwechsel 1891–1931}. Wien, Zürich: \emph{Europaverlag} 1992, S. 153–154.} }\toendnotes[C]{\smallbreak}\pstart{}{\pb}Herrn Dr. Rich. Beer-Hofmann\pend{}\pstart{}Pörtschach\oindex{Poertschach am Woerthersee@\textbf{Pörtschach am Wörthersee}, \emph{P.PPL}|pw}\pend{}\pstart{}Villa Arnstein\oindex{Villa Arnstein@\textbf{Villa Arnstein}, \emph{Wohngebäude (K.WHS)}|pw}.\pend{}{\bigskip}\vspace{1em}
\pstart
           \raggedleft{}{\pb}\textsc{Vahrn}\oindex{Vahrn@\textbf{Vahrn}, \emph{P.PPLA3}|pw}, 22/7 901\pend
           \vspace{0.5em}
\pstart
           lieber Richard, von dem Tod Ihrer Stiefmama\pwindex{Beer, Rosa 20.7.1847 – 1.7.1901@\textsc{Beer, Rosa} (20.7.1847 – 1.7.1901)|pwv} hab ich durch Schw.\pwindex{Schwarzkopf, Gustav 07.11.1853 – 13.11.1939@\textsc{Schwarzkopf, Gustav} (07.11.1853 – 13.11.1939), \emph{Schriftsteller/Schriftstellerin}|pw} erfahren, noch eh Sie mirs ſchrieben, zu formeller Condolenz wars zu
               ſpät, bitte ſagen Sie Ihrem Papa\pwindex{Beer, Hermann 10.8.1835 – 03.10.1902@\textsc{Beer, Hermann} (10.8.1835 – 03.10.1902), \emph{Rechtsanwalt/Rechtsanwältin}|pwv} nachträglich, daſs ich ihm meine herzliche Theilnahme alſo lieber durch
               Sie ausdrücken laſſe. – Paul\pwindex{Goldmann, Paul 31.01.1865 – 25.09.1935@\textsc{Goldmann, Paul} (31.01.1865 – 25.09.1935), \emph{Schriftsteller/Schriftstellerin, Journalist/Journalistin}|pw} dürfte ſchon in
               den nächſten Tagen an den Wörtherſee\oindex{Woerthersee@\textbf{Wörthersee}, \emph{H.LK}|pw} kommen, iſt
               erbittert über Sie, will Sie gar nicht besuchen u. ſ. w. Schreiben Sie ihm doch noch
               eheſtens ein Wort. Vom Wörtherſee\oindex{Woerthersee@\textbf{Wörthersee}, \emph{H.LK}|pw} ko{\geminationm}t G.\pwindex{Goldmann, Paul 31.01.1865 – 25.09.1935@\textsc{Goldmann, Paul} (31.01.1865 – 25.09.1935), \emph{Schriftsteller/Schriftstellerin, Journalist/Journalistin}|pw} herunter,
               ich muſs mir noch irgend was höheres ſuchen {\pb}werde
               mich auf der Seiſer Alpe\oindex{Seiser Alm@\textbf{Seiser Alm}, \emph{Berg (N.BRG)}|pw} u im Tierſer Thal\oindex{Tiers@\textbf{Tiers}, \emph{A.ADM3}|pw} umſehn. Machen Sie’s doch möglich
               auch zu kommen. Die letzten Sommertage denk’ ich Gardaſee\oindex{Lago di Garda@\textbf{Lago di Garda}, \emph{See (N.SEE)}|pw}, \textsc{ev}. Torbole\oindex{Torbole sul Garda@\textbf{Torbole sul Garda}, \emph{P.PPLA3}|pw}? –\pend
           
\pstart
           Ich find es hier ſehr angenehm, die Zimmer offenbar neu hergerichtet ſehr hübſch, das
               Eſſen gut, wenig Leut, und warm. Ich \substVorne{}\textsuperscript{\textcolor{gray}{×}\-\textcolor{gray}{×}\-\textcolor{gray}{×}\-\textcolor{gray}{×}\-\textcolor{gray}{×}\-\textcolor{gray}{×}}\substDazwischen{}ſchreibe\substHinten{} (3a.
               Stück\pwindex{einsame Weg. Schauspiel in fuenf Akten@\emph{Der einsame Weg. Schauspiel in fünf Akten}|pwv}\pwindex{Professor Bernhardi. Komoedie in fuenf Akten@\emph{Professor Bernhardi. Komödie in fünf Akten}|pwv}). An der Zerſtörung der »Grämlichkeit« wird von berufener Seite mit Talent
               gearbeitet. We{\geminationn} mich etwas ſtört, iſt es nur der Um{\pb}ſtand, daſs man in der betreffenden Familie Sie für
               den weitaus hervorragendſten von {\dots} hm {\dots} Alt-Wien\oindex{Wien@\textbf{Wien}, \emph{A.ADM2}|pw} hält, eine Meinung, die Sie
               hoffentlich durch Ihr {\dots} wieder hm {\dots} nächſtes Stück\pwindex{Graf von Charolais. Ein Trauerspiel@\emph{Der Graf von Charolais. Ein Trauerspiel}|pwv}
               endgiltig begraben werden.\pend
           
\pstart
           – Schreiben Sie \introOben{}– we{\geminationn}\introOben{} bald, da{\geminationn} noch hieher, ſonſt Wien\oindex{Wien@\textbf{Wien}, \emph{A.ADM2}|pw}.\pend
           
\pstart
           Heute Ausflug Karerſee\oindex{Karersee@\textbf{Karersee}, \emph{See (N.SEE)}|pw}, wo Julius\pwindex{Schnitzler, Julius 13.07.1865 – 29.06.1939@\textsc{Schnitzler, Julius} (13.07.1865 – 29.06.1939), \emph{Chirurg/Chirurgin}|pw} u Frau\pwindex{Schnitzler, Helene 16.07.1871 – September 1941@\textsc{Schnitzler, Helene} (16.07.1871 – September 1941)|pwv}.\pend
           
\pstart
           Gehts den Ihren gut? Baden Sie viel? Sehn Sie die übrigen Rundwohner?\pend
           \pstart Von Herzen Ihr \spacefill\mbox{Arthur}\pend{}\selectlanguage{ngerman}\endnumbering\briefempfaengerindex{Beer-Hofmann, Richard@\textsc{Beer-Hofmann, Richard}!zzzSchnitzler, Arthur@\emph{von Arthur Schnitzler}!1901-07-221@{22. 7. 1901}|)be}\mylabel{L01150h}  \normalsize

\doendnotes{C}
\bigskip
\vfill

\clearpage

\footnotesize

\lohead{\textsc{register}}

% Definiere theindex-Environment komplett neu ohne reledmac
\makeatletter
\renewenvironment{theindex}{%
  \section*{\indexname}%
  \setlength{\parindent}{0pt}%
  \setlength{\parskip}{0pt plus 0.3pt}%
  \let\item\@idxitem
}{%
  \clearpage
}
\makeatother

\IfFileExists{\jobname-pw.ind}{\input{\jobname-pw.ind}}{}

\end{document}

      