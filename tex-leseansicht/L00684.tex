\input{../tex-inputs/latex-pdf-vorspann}
\begin{center}
            \textcolor{red}{ENTWURF. ENTZIFFERUNG NOCH NICHT KORREKTURGELESEN}
                      \end{center}
            
               \section[Richard Beer-Hofmann an Arthur Schnitzler, 11. 6. 1897]{ Richard Beer-Hofmann an Arthur Schnitzler, 11. 6. 1897}\nopagebreak\mylabel{v}\rehead{ }\begin{ledgroupsized}[t]{13cm}\normalsize\beginnumbering\briefempfaengerindex{Schnitzler, Arthur@\textsc{Schnitzler, Arthur}!zzzBeer-Hofmann, Richard@\emph{von Richard Beer-Hofmann}!1897-06-111@{11. 6. 1897}|(be} \toendnotes[C]{\smallbreak\pagebreak[2]} \Standort{CUL, Schnitzler, B 8.}
\physDesc{Brief, 1 Blatt, 2 Seiten
\newline{}Handschrift: Bleistift, lateinische Kurrent\newline{}Ordnung: mit Bleistift von unbekannter Hand nummeriert: »98« }\buchAbdrucke{\weitereDrucke{Arthur Schnitzler, Richard Beer-Hofmann: \emph{Briefwechsel 1891–1931}. Hg. Konstanze Fliedl. Wien, Zürich: \emph{Europaverlag} 1992, S. 108.} }\pstart
           \raggedleft{}{\pb}Ischl\oindex{Bad Ischl@\textbf{Bad Ischl}|pw}{ }11/VI 97\pend
           \pstart
           Mein lieber Arthur!\pend
           \pstart
           Ich war vor einigen Tagen bei Leopold\pwindex{Petter, Leopold 17.11.1850 – 03.07.1917@\textsc{Petter, Leopold} (17.11.1850 – 03.07.1917), \emph{Hotelier}|pw}; Sie werden
               die Zi{\geminationm}er nach Wunsch erhalten. Bicycle hab ich in Wien\oindex{Wien@\textbf{Wien}|pw} zu lernen angefangen, habe aber hier erst zwei
               Lectionen nehmen können wegen schlechten Wetters.\pend
           \pstart
           {\pb}Zu arbeiten habe ich begonnen –
               mit Unbehagen natürlich.\pend
           \pstart
           Warum ko{\geminationm}en Sie nicht früher? Schreiben Sie mir recht
               viel und grüßen Sie Hugo\pwindex{Hofmannsthal, Hugo von 01.02.1874 – 15.07.1929@\textsc{Hofmannsthal, Hugo von} (01.02.1874 – 15.07.1929), \emph{Schriftsteller}|pw} und Schwarzkopf\pwindex{Schwarzkopf, Gustav 07.11.1853 – 13.11.1939@\textsc{Schwarzkopf, Gustav} (07.11.1853 – 13.11.1939), \emph{Schriftsteller}|pw}.\pend
           \pstart
           Herzlichst{\\[\baselineskip]}Ihr\spacefill\mbox{Richard}\pend
           \leftskip=0em{}\endnumbering\briefempfaengerindex{Schnitzler, Arthur@\textsc{Schnitzler, Arthur}!zzzBeer-Hofmann, Richard@\emph{von Richard Beer-Hofmann}!1897-06-111@{11. 6. 1897}|)be}\mylabel{h}\end{ledgroupsized}  \newcommand{\dateiname}{L00684}\newcommand{\titel}{Richard Beer-Hofmann an Arthur Schnitzler, 11. 6. 1897}\newcommand{\editorInnen}{Martin Anton Müller und Gerd-Hermann Susen}\input{../tex-inputs/latex-pdf-abspann}
      