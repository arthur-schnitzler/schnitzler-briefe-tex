%% latex-korrekturansicht-vorspann.tex
%% Vorspann für die Korrekturansicht.
%% Lädt die gemeinsame Datei latex-vorspann.tex mit gesetztem Schalter.

\newif\ifkorrekturansicht
\korrekturansichttrue

\input{../tex-inputs/latex-vorspann}


\section[ Paul Goldmann an Arthur Schnitzler, 26. 7. {[}1896{]}]{L02783 Paul Goldmann an Arthur Schnitzler, 26. 7. {[}1896{]}}
\nopagebreak\mylabel{L02783v}
\rehead{ }\normalsize\beginnumbering\briefempfaengerindex{Schnitzler, Arthur@\textsc{Schnitzler, Arthur}!zzzGoldmann, Paul@\emph{von Paul Goldmann}!1896-07-263@{26. 7. {[}1896{]}}|(be}
\toendnotes[C]{\smallbreak\pagebreak[2]}\Standort{DLA, A:Schnitzler, HS.NZ85.1.3166.}
\physDesc{Brief, 1 Blatt, 2 Seiten, 671 Zeichen
\newline{}Handschrift: blaue Tinte, deutsche Kurrent
\newline{}Schnitzler: mit Bleistift das Jahr »96« und »nach Hamb{[}urg{]}\oindex{Hamburg@\textbf{Hamburg}, \emph{P.PPLA}|pwv} ant\textcolor{gray}{w}{[}orten{]}« vermerkt }\toendnotes[C]{\smallbreak}
\pstart
           {\pb}\textcolor{gray}{\textbf{\textbf{Frankfurter Zeitung\orgindex{Frankfurter Zeitung@Frankfurter Zeitung|pw}}}}\pend
           
\pstart
           \textcolor{gray}{\textbf{(\begin{otherlanguage}{french}Gazette de Francfort\end{otherlanguage}\orgindex{Frankfurter Zeitung@Frankfurter Zeitung|pw}).}}\pend
           
\pstart
           \textcolor{gray}{\textbf{\textbf{\begin{otherlanguage}{french}Fondateur M.\end{otherlanguage}{ }L. Sonnemann\pwindex{Sonnemann, Leopold 1831-10-29 – 1909-10-30@\textsc{Sonnemann, Leopold} (1831-10-29 – 1909-10-30), \emph{Journalist/Journalistin, Herausgeber/Herausgeberin}|pw}.}}}\pend
           
\pstart
           \begin{otherlanguage}{french}\textcolor{gray}{\textbf{Journal\pwindex{Frankfurter Zeitung@\emph{Frankfurter Zeitung}|pwv} politique,
                        financier,}}\end{otherlanguage}\pend
           
\pstart
           \begin{otherlanguage}{french}\textcolor{gray}{\textbf{commercial et littéraire.}}\end{otherlanguage}\pend
           
\pstart
           \begin{otherlanguage}{french}\textcolor{gray}{\textbf{\textbf{Paraissant trois fois par jour.}}}\end{otherlanguage}\hfill \textsc{Paris\oindex{Paris@\textbf{Paris}, \emph{P.PPLC}|pw}}, 26. Juli.\pend
           
\pstart
           \begin{otherlanguage}{french}\textcolor{gray}{\textbf{\textbf{Bureau à Paris\oindex{Paris@\textbf{Paris}, \emph{P.PPLC}|pw}}}}\end{otherlanguage}\pend
           
\pstart
           \begin{otherlanguage}{french}\textcolor{gray}{\textbf{\textbf{24. Rue Feydeau\oindex{rue Feydeau@\textbf{rue Feydeau}, \emph{Straße (K.STR)}|pw}.}}}\end{otherlanguage}\pend
           
\pstart\center{}Mein lieber Freund,\pend\vspace{0.5em}
\pstart
           Ich wollte eigentlich geſtern abreiſen, wurde aber
               durch die \label{K_L02783-1v}\edtext{Ereigniſſe von \textsc{Lille\oindex{Lille@\textbf{Lille}, \emph{P.PPLA}|pw}}}{\lemma{\textnormal{\emph{Ereigniſſe von Lille}}}\Cendnote{\textnormal{Rund um den Kongress der \emph{französischen Arbeiterpartei}\orgindex{Parti ouvrier français@Parti ouvrier français|pwk} (21. – 25. 7. 1896) spitzte sich Ende Juli 1896 in Lille\oindex{Lille@\textbf{Lille}, \emph{P.PPLA}|pwk}
                  die Situation zwischen sozialistisch und antisozialistisch gestimmten Bürgerinnen
                  und Bürgern zu. Dabei kam es auch zu gewalttätigen Ausschreitungen und
                  Verhaftungen.}}}\label{K_L02783-1} zurückgehalten. Auch wünſchte meine Redaction\orgindex{Frankfurter Zeitung@Frankfurter Zeitung|pwv}, ich ſolle bis Ende Monats hierbleiben. So komme ich kaum vor Freitag 31. Juli fort, vielleicht erſt Samſtag. Ich bleibe einen Tag in \textsc{Köln\oindex{Koeln@\textbf{Köln}, \emph{P.PPLA2}|pw}}, drei oder vier in \textsc{Hamburg\oindex{Hamburg@\textbf{Hamburg}, \emph{P.PPLA}|pw}}. Dann komme ich nach \textsc{Kopenhagen\oindex{Kopenhagen@\textbf{Kopenhagen}, \emph{P.PPLC}|pw}}. Noch habe ich keine Ahnung, wo {\pb}ich Dich
               treffe. Schreib’ mir Deine Adreſſe nach \textsc{\uline{Hamburg}\oindex{Hamburg@\textbf{Hamburg}, \emph{P.PPLA}|pw}}, \textsc{Poste restante}. Vielen Dank für Deine lieben
               Nachrichten von unterwegs! Ich bin in großer Sorge. Es will diesmal gar nicht gehen
               mit dem Fortkommen.\pend
           
\pstart
           Viele treue Grüße!\pend
           
\pstart
           Wie ſchön das iſt, daß ich Dich bald ſehen ſoll!\pend
           
\pstart
           In Treue {\\[\baselineskip]}Dein {\\[\baselineskip]}\spacefill\mbox{Paul Goldmn}\pend
           \leftskip=0em{}\selectlanguage{ngerman}\endnumbering\briefempfaengerindex{Schnitzler, Arthur@\textsc{Schnitzler, Arthur}!zzzGoldmann, Paul@\emph{von Paul Goldmann}!1896-07-263@{26. 7. {[}1896{]}}|)be}\mylabel{L02783h}  \normalsize

\doendnotes{C}
\bigskip
\vfill

\clearpage

\footnotesize

\lohead{\textsc{register}}

% Definiere theindex-Environment komplett neu ohne reledmac
\makeatletter
\renewenvironment{theindex}{%
  \section*{\indexname}%
  \setlength{\parindent}{0pt}%
  \setlength{\parskip}{0pt plus 0.3pt}%
  \let\item\@idxitem
}{%
  \clearpage
}
\makeatother

\IfFileExists{\jobname-pw.ind}{\input{\jobname-pw.ind}}{}

\end{document}

      