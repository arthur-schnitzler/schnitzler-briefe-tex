%% latex-leseansicht-vorspann.tex
%% Vorspann für die Leseansicht.
%% Lädt die gemeinsame Datei latex-vorspann.tex mit nicht gesetztem Schalter.

\newif\ifkorrekturansicht
\korrekturansichtfalse

\input{../tex-inputs/latex-vorspann}


         
         \renewcommand{\erwaehntePersonen}{Personen: Hugo von Hofmannsthal}
         \renewcommand{\erwaehnteInstitutionen}{Institutionen: Allgemeine Kunst-Chronik}
         \renewcommand{\erwaehnteOrte}{Orte: Reisnerstraße, Strohgasse, Wien}
         \renewcommand{\erwaehnteWerke}{}
               \section[Friedrich M. Fels an Arthur Schnitzler, {[}1. 1. 1893?{]}]{ Friedrich M. Fels an Arthur Schnitzler, {[}1. 1. 1893?{]}}\nopagebreak\mylabel{v}\rehead{ }\begin{ledgroupsized}[t]{13cm}\normalsize\beginnumbering \toendnotes[C]{\smallbreak\pagebreak[2]} \Standort{DLA, A:Schnitzler, HS.NZ85.1.2956.}
\physDesc{Brief, 1 Blatt, 1 Seite
\newline{}Handschrift: schwarze Tinte, lateinische Kurrent
\newline{}Schnitzler: mit Bleistift nummeriert: »3« }\toendnotes[C]{\smallbreak}\pstart
           \noindent{}{\pb}Lieber Dr Arthur Schnitzler! Gestern bald als Sie gingen, brachte
               mir der Diener zwei Wohnungen: 1. Reisnerstraſse\oindex{Reisnerstrasse@\textbf{Reisnerstraße}|pw}
               wenig vom \label{K_L00153_1v}\edtext{Bureau}{\lemma{\textnormal{\emph{Bureau}}}\Cendnote{\textnormal{Fels\pwindex{Fels, Friedrich Michael *~1864@\textsc{Fels, Friedrich Michael} (*~1864), \emph{Journalist}|pwk} dürfte bei der \emph{Allgemeinen Kunst-Chronik}\orgindex{Allgemeine Kunst-Chronik@Allgemeine Kunst-Chronik|pwk} in der Reisnerstrasse 3\oindex{Reisnerstrasse@\textbf{Reisnerstraße}|pwk} angestellt gewesen sein.}}}\label{K_L00153_1h} c. 16 fl und
                  \label{K_L00153_2v}\edtext{Strohgaſse\oindex{Strohgasse@\textbf{Strohgasse}|pw}}{\lemma{\textnormal{\emph{Strohgaſse}}}\Cendnote{\textnormal{Im Brief Hofmannsthal\pwindex{Hofmannsthal, Hugo von 1874-02-01 – 1929-07-15@\textsc{Hofmannsthal, Hugo von} (1874-02-01 – 1929-07-15), \emph{Schriftsteller}|pwk}s an Schnitzler\pwindex{Schnitzler, Arthur 15.05.1862 – 21.10.1931@\textsc{Schnitzler, Arthur} (15.05.1862 – 21.10.1931), \emph{Schriftsteller, Mediziner}|pwk} vom [9. 9. 1893] wird diese Wohnung
                  erwähnt. Damit kann dieses Korrespondenzstück zeitlich zumindest nach hinten
                  eingegrenzt werden.}}}\label{K_L00153_2h}{ }\uline{12 fl} – letztere angesehen, geno{\geminationm}en. Das Kabinet gut ausgestattet, die Verhältniſse
               scheinen ganz ordentlich zu sein; nur eines: auſserordentlich pünktlich im
               Bezahlen!\pend
           \pstart
           Lieber Doktor! Sie thäten mir wirklich einen Gefallen, \uline{nein}, Sie \uline{müſsen} mich heute noch aufsuchen,
               im Bureau, da{\geminationn} Wohnung. Ich habe Ihnen manches zu sagen,
               was gegen meine Beſserung spricht. Also Sie \uline{müſsen}
               heute ko{\geminationm}en.\pend
           \pstart
           Herzl.{\\[\baselineskip]}\spacefill\mbox{Fels}\pend
           \leftskip=0em{}
         
         \endnumbering\mylabel{h}\end{ledgroupsized}  \newcommand{\dateiname}{L00153}\newcommand{\titel}{Friedrich M. Fels an Arthur Schnitzler, [1. 1. 1893?]}\newcommand{\editorInnen}{Martin Anton Müller und Gerd-Hermann Susen}%% latex-leseansicht-abspann.tex
%% Abspann für die Leseansicht.
%% Der Schalter \ifkorrekturansicht ist bereits durch den Vorspann gesetzt.

%% latex-abspann.tex
%% Gemeinsamer Abspann für Korrekturansicht und Leseansicht.
%% Setzt den Schalter \ifkorrekturansicht voraus (gesetzt in den
%% einbindenden Dateien latex-korrekturansicht-abspann.tex bzw.
%% latex-leseansicht-abspann.tex).
%% ---------------------------------------------------------------

\normalsize

% Das esempio-Environment wird nur in der Leseansicht benötigt
\ifkorrekturansicht\else
\newenvironment{esempio}[3]%
{
    \vspace{1.5ex}
    \rlap{\underline{#1}}
    \par
    \setlength{\parindent}{0cm}
    \nopagebreak
    \leftskip=#2cm
    \rightskip=#3cm
}
{
    \par
}
\fi

\doendnotes{C}
\bigskip
\vfill

\clearpage

\footnotesize

\ifkorrekturansicht
  \lohead{\textsc{register}}
\fi

% theindex-Environment neu definieren ohne reledmac
\makeatletter
\renewenvironment{theindex}{%
  \ifkorrekturansicht
    \section*{\indexname}%
  \else
    \subsubsection*{Index der erwähnten Entitäten}%
  \fi
  \setlength{\parindent}{0pt}%
  \setlength{\parskip}{0pt plus 0.3pt}%
  \let\item\@idxitem
}{%
  \ifkorrekturansicht\clearpage\fi
}
\makeatother

\IfFileExists{\jobname-pw.ind}{\input{\jobname-pw.ind}}{}

% Quellenangabe nur in der Leseansicht
\ifkorrekturansicht\else
% Fallback-Definitionen, falls die .tex-Datei \titel etc. nicht gesetzt hat
\providecommand{\titel}{}
\providecommand{\editorInnen}{}
\providecommand{\dateiname}{\jobname}

\vspace{3cm}

\vfill

\footnotesize
\textsc{Quelle}: \titel. Herausgegeben von {\editorInnen}. In: \emph{Arthur Schnitzler: Briefwechsel mit Autorinnen und Autoren}.
 Digitale Edition, https://schnitzler-briefe.acdh.oeaw.ac.at/{\dateiname}.html (Stand \today)
\fi

\end{document}


      