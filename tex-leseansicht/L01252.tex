%% latex-leseansicht-vorspann.tex
%% Vorspann für die Leseansicht.
%% Lädt die gemeinsame Datei latex-vorspann.tex mit nicht gesetztem Schalter.

\newif\ifkorrekturansicht
\korrekturansichtfalse

\input{../tex-inputs/latex-vorspann}


\section[Arthur Schnitzler an Hugo von Hofmannsthal, {[}25.? 11. 1902{]}]{L01252 Arthur Schnitzler an Hugo von Hofmannsthal, {[}25.? 11. 1902{]}}
\nopagebreak\mylabel{L01252v}
\rehead{ }\normalsize\beginnumbering\briefempfaengerindex{Hofmannsthal, Hugo von@\textsc{Hofmannsthal, Hugo von}!zzzSchnitzler, Arthur@\emph{von Arthur Schnitzler}!1902-11-251@{{[}25.? 11. 1902{]}}|(be}
\toendnotes[C]{\smallbreak\pagebreak[2]}
\correspDesc{Versand  durch Arthur Schnitzler am [25.? 11. 1902] in Wien
\newline{}Erhalt  durch Hugo von Hofmannsthal im Zeitraum [25. 11. 1902 – 29. 11. 1902?] in Wien}\toendnotes[C]{\smallbreak}
\Standort{FDH, Hs-30885,100.}
\physDesc{Brief, 1 Blatt, 3 Seiten, 971 Zeichen
\newline{}Handschrift: Bleistift, deutsche Kurrent
\newline{}Ordnung: mit Bleistift von unbekannter Hand datiert: »1906??« }
\buchAbdrucke{\weitereDrucke{Hugo von Hofmannsthal, Arthur Schnitzler: \emph{Briefwechsel}. Herausgegeben von Therese Nickl und Heinrich Schnitzler. Frankfurt am Main: \emph{S. Fischer} 1964, S. 164.} }\toendnotes[C]{\smallbreak}
\pstart
           \noindent{}{\pb}lieber Hugo, ich habe, da auch ich keine andre Adreſſe weiſs, den
               Brief in die Direktion des Burg. Th.\oindex{Wien@\textbf{Wien}!I., Innere Stadt@\textbf{I., Innere Stadt}!Burgtheater@\textbf{Burgtheater}, \emph{Theater}|pw}
               geſchickt.\pend
           
\pstart
           – Es iſt jetzt mit dem Landfahren, beſonders abends \strikeout{übrigens} keine{ }ſehr begeiſternde Sache; es wäre mir{ }ſchon lieber, we{\geminationn} ich Sie, gelegentlich einer Wien\oindex{Wien@\textbf{Wien}, \emph{Verwaltungsgebiet}|pw}fahrt, vorerſt einmal hier zu{ }ſehen u zu{ }ſprechen bekäme. –
               Natürlich fahr ich, we{\geminationn}{ }\substVorne{}\textsuperscript{ich}\substDazwischen{}die\substHinten{}{ }Hauptma{\geminationn}\pwindex{Hauptmann, Gerhart 15.\,11.\,1862 Szczawno-Zdrój – 6.\,6.\,1946 Jagniątków@\textsc{Hauptmann, Gerhart} (15.\,11.\,1862 Szczawno-Zdrój – 6.\,6.\,1946 Jagniątków), \emph{Schriftsteller}|pw}geſchichte zu Stande ko{\geminationm}t, mit ihm zu Ihnen {\pb}hinaus. –\pend
           
\pstart
           Ich freue mich auf Ihr Stück\pwindex{Hofmannsthal, Hugo von 1.\,2.\,1874 Wien – 15.\,7.\,1929 Rodaun@\textsc{Hofmannsthal, Hugo von} (1.\,2.\,1874 Wien – 15.\,7.\,1929 Rodaun), \emph{Schriftsteller}!gerettete Venedig. Trauerspiel in fünf Aufzügen@\strich\emph{Das gerettete Venedig. Trauerspiel in fünf Aufzügen}|pwv}. – Ich habe geſtern die Skizze des meinen\pwindex{Schnitzler, Arthur 15.\,5.\,1862 Wien – 21.\,10.\,1931 ebd.@\textsc{Schnitzler, Arthur} (15.\,5.\,1862 Wien – 21.\,10.\,1931 ebd.), \emph{Schriftsteller, Mediziner}!einsame Weg. Schauspiel in fünf Akten@\strich\emph{Der einsame Weg. Schauspiel in fünf Akten}|pwv} – de{\geminationn} ich ka{\geminationn} es in keiner Weiſe ausgeführt nennen, – zu Ende
               dictirt, und ein{ }ſchwerer \label{K_L01252-1v}\edtext{Grundfehler}{\lemma{\textnormal{\emph{Grundfehler}}}\Cendnote{\textnormal{Siehe A. S.: \emph{Tagebuch}, 25. 11. 1902.
               }}}\label{K_L01252-1} des ganzen, der nun mit Evidenz zu Tage trat, hat mich auffallend tief
               verſtimmt; – mich in die Nacht und in meine Träume wie ein wirkliches Unglück ver{\pb}folgt. Solche Dinge haben natürlich i{\geminationm}er ihren Sinn: Mängel eines Werks, die man \uline{ſo}{ }ſchmerzlich empfindet,{ }ſind i{\geminationm}er Mängel des eigenen Weſens, auf die man in dieſer
               geheimnisvollen Weiſe geleitet wird.\pend
           
\pstart
           – Leben Sie wohl. Auf bald.\pend
           
\pstart
           Herzlichſt Ihr{\\[\baselineskip]}\spacefill\mbox{A.}\pend
           \leftskip=0em{}\selectlanguage{ngerman}\endnumbering\briefempfaengerindex{Hofmannsthal, Hugo von@\textsc{Hofmannsthal, Hugo von}!zzzSchnitzler, Arthur@\emph{von Arthur Schnitzler}!1902-11-251@{{[}25.? 11. 1902{]}}|)be}\mylabel{L01252h}  \newcommand{\dateiname}{L01252}\newcommand{\titel}{Arthur Schnitzler an Hugo von Hofmannsthal, [25.? 11. 1902]}\newcommand{\editorInnen}{Martin Anton Müller und Gerd-Hermann Susen}%% latex-leseansicht-abspann.tex
%% Abspann für die Leseansicht.
%% Der Schalter \ifkorrekturansicht ist bereits durch den Vorspann gesetzt.

%% latex-abspann.tex
%% Gemeinsamer Abspann für Korrekturansicht und Leseansicht.
%% Setzt den Schalter \ifkorrekturansicht voraus (gesetzt in den
%% einbindenden Dateien latex-korrekturansicht-abspann.tex bzw.
%% latex-leseansicht-abspann.tex).
%% ---------------------------------------------------------------

\normalsize

% Das esempio-Environment wird nur in der Leseansicht benötigt
\ifkorrekturansicht\else
\newenvironment{esempio}[3]%
{
    \vspace{1.5ex}
    \rlap{\underline{#1}}
    \par
    \setlength{\parindent}{0cm}
    \nopagebreak
    \leftskip=#2cm
    \rightskip=#3cm
}
{
    \par
}
\fi

\doendnotes{C}
\bigskip
\vfill

\clearpage

\footnotesize

\ifkorrekturansicht
  \lohead{\textsc{register}}
\fi

% theindex-Environment neu definieren ohne reledmac
\makeatletter
\renewenvironment{theindex}{%
  \ifkorrekturansicht
    \section*{\indexname}%
  \else
    \subsubsection*{Index der erwähnten Entitäten}%
  \fi
  \setlength{\parindent}{0pt}%
  \setlength{\parskip}{0pt plus 0.3pt}%
  \let\item\@idxitem
}{%
  \ifkorrekturansicht\clearpage\fi
}
\makeatother

\IfFileExists{\jobname-pw.ind}{\input{\jobname-pw.ind}}{}

% Quellenangabe nur in der Leseansicht
\ifkorrekturansicht\else
% Fallback-Definitionen, falls die .tex-Datei \titel etc. nicht gesetzt hat
\providecommand{\titel}{}
\providecommand{\editorInnen}{}
\providecommand{\dateiname}{\jobname}

\vspace{3cm}

\vfill

\footnotesize
\textsc{Quelle}: \titel. Herausgegeben von {\editorInnen}. In: \emph{Arthur Schnitzler: Briefwechsel mit Autorinnen und Autoren}.
 Digitale Edition, https://schnitzler-briefe.acdh.oeaw.ac.at/{\dateiname}.html (Stand \today)
\fi

\end{document}


