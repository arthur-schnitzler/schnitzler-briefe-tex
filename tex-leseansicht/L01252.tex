%% latex-korrekturansicht-vorspann.tex
%% Vorspann für die Korrekturansicht.
%% Lädt die gemeinsame Datei latex-vorspann.tex mit gesetztem Schalter.

\newif\ifkorrekturansicht
\korrekturansichttrue

\input{../tex-inputs/latex-vorspann}


\section[Arthur Schnitzler an Hugo von Hofmannsthal, {[}25.? 11. 1902{]}]{L01252 Arthur Schnitzler an Hugo von Hofmannsthal, {[}25.? 11. 1902{]}}
\nopagebreak\mylabel{L01252v}
\rehead{ }\normalsize\beginnumbering\briefempfaengerindex{Hofmannsthal, Hugo von@\textsc{Hofmannsthal, Hugo von}!zzzSchnitzler, Arthur@\emph{von Arthur Schnitzler}!1902-11-251@{{[}25.? 11. 1902{]}}|(be}
\toendnotes[C]{\smallbreak\pagebreak[2]}\Standort{FDH, Hs-30885,100.}
\physDesc{Brief, 1 Blatt, 3 Seiten, 971 Zeichen
\newline{}Handschrift: Bleistift, deutsche Kurrent
\newline{}Ordnung: mit Bleistift von unbekannter Hand datiert: »1906??« }
\buchAbdrucke{\weitereDrucke{Hugo von Hofmannsthal, Arthur Schnitzler: \emph{Briefwechsel}. Frankfurt am Main: \emph{S. Fischer} 1964, S. 164.} }\toendnotes[C]{\smallbreak}
\pstart
           \noindent{}{\pb}lieber Hugo, ich habe, da auch ich keine andre Adreſſe weiſs, den
               Brief in die Direktion des Burg. Th.\oindex{Burgtheater@\textbf{Burgtheater}, \emph{S.THTR}|pw}
               geſchickt.\pend
           
\pstart
           – Es iſt jetzt mit dem Landfahren, beſonders abends \strikeout{übrigens} keine ſehr begeiſternde Sache; es wäre mir ſchon lieber, we{\geminationn} ich Sie, gelegentlich einer Wien\oindex{Wien@\textbf{Wien}, \emph{A.ADM2}|pw}fahrt, vorerſt einmal hier zu ſehen u zu ſprechen bekäme. –
               Natürlich fahr ich, we{\geminationn}{ }\substVorne{}\textsuperscript{ich}\substDazwischen{}die\substHinten{}{ }Hauptma{\geminationn}\pwindex{Hauptmann, Gerhart 15.11.1862 – 06.06.1946@\textsc{Hauptmann, Gerhart} (15.11.1862 – 06.06.1946), \emph{Schriftsteller/Schriftstellerin}|pw}geſchichte zu Stande ko{\geminationm}t, mit ihm zu Ihnen {\pb}hinaus. –\pend
           
\pstart
           Ich freue mich auf Ihr Stück\pwindex{gerettete Venedig. Trauerspiel in fuenf Aufzuegen@\emph{Das gerettete Venedig. Trauerspiel in fünf Aufzügen}|pwv}. – Ich habe geſtern die Skizze des meinen\pwindex{einsame Weg. Schauspiel in fuenf Akten@\emph{Der einsame Weg. Schauspiel in fünf Akten}|pwv} – de{\geminationn} ich ka{\geminationn} es in keiner Weiſe ausgeführt nennen, – zu Ende
               dictirt, und ein ſchwerer \label{K_L01252-1v}\edtext{Grundfehler}{\lemma{\textnormal{\emph{Grundfehler}}}\Cendnote{\textnormal{Siehe A. S.: \emph{Tagebuch}, 25. 11. 1902.
               }}}\label{K_L01252-1} des ganzen, der nun mit Evidenz zu Tage trat, hat mich auffallend tief
               verſtimmt; – mich in die Nacht und in meine Träume wie ein wirkliches Unglück ver{\pb}folgt. Solche Dinge haben natürlich i{\geminationm}er ihren Sinn: Mängel eines Werks, die man \uline{ſo}{ }ſchmerzlich empfindet, ſind i{\geminationm}er Mängel des eigenen Weſens, auf die man in dieſer
               geheimnisvollen Weiſe geleitet wird.\pend
           
\pstart
           – Leben Sie wohl. Auf bald.\pend
           
\pstart
           Herzlichſt Ihr{\\[\baselineskip]}\spacefill\mbox{A.}\pend
           \leftskip=0em{}\selectlanguage{ngerman}\endnumbering\briefempfaengerindex{Hofmannsthal, Hugo von@\textsc{Hofmannsthal, Hugo von}!zzzSchnitzler, Arthur@\emph{von Arthur Schnitzler}!1902-11-251@{{[}25.? 11. 1902{]}}|)be}\mylabel{L01252h}  \normalsize

\doendnotes{C}
\bigskip
\vfill

\clearpage

\footnotesize

\lohead{\textsc{register}}

% Definiere theindex-Environment komplett neu ohne reledmac
\makeatletter
\renewenvironment{theindex}{%
  \section*{\indexname}%
  \setlength{\parindent}{0pt}%
  \setlength{\parskip}{0pt plus 0.3pt}%
  \let\item\@idxitem
}{%
  \clearpage
}
\makeatother

\IfFileExists{\jobname-pw.ind}{\input{\jobname-pw.ind}}{}

\end{document}

      