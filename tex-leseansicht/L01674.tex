%% latex-korrekturansicht-vorspann.tex
%% Vorspann für die Korrekturansicht.
%% Lädt die gemeinsame Datei latex-vorspann.tex mit gesetztem Schalter.

\newif\ifkorrekturansicht
\korrekturansichttrue

\input{../tex-inputs/latex-vorspann}


\section[Hermann Bahr an Arthur Schnitzler, 19. 5. {[}1907{]}]{L01674 Hermann Bahr an Arthur Schnitzler, 19. 5. {[}1907{]}}
\nopagebreak\mylabel{L01674v}
\rehead{ }\normalsize\beginnumbering\briefempfaengerindex{Schnitzler, Arthur@\textsc{Schnitzler, Arthur}!zzzBahr, Hermann@\emph{von Hermann Bahr}!1907-05-191@{19. 5. {[}1907{]}}|(be}
\toendnotes[C]{\smallbreak\pagebreak[2]}\Standort{CUL, Schnitzler, B 5b.}
\physDesc{Brief, 1 Blatt, 2 Seiten, 722 Zeichen
\newline{}Handschrift: blaue Tinte, deutsche Kurrent
\newline{}Ordnung: mit Bleistift von unbekannter Hand nummeriert:
                                    »150« }
\buchAbdrucke{\weitereDrucke{Hermann Bahr, Arthur Schnitzler: \emph{Briefwechsel, Aufzeichnungen, Dokumente (1891–1931)}. Göttingen: \emph{Wallstein} 2018, S. 393.} }\toendnotes[C]{\smallbreak}
\pstart
           \raggedleft{}{\pb}19. 5.\pend
           
\pstart{}Lieber Arthur!\pend\vspace{0.5em}
\pstart
           Danke ſchön für den zweiten Brehm\pwindex{Brehms Tierleben@\emph{Brehms Tierleben}|pw}, den ich noch
               einige Zeit behalten möchte, er macht mir ein unſinniges Vergnügen.\pend
           
\pstart
           Du biſt hoffentlich nicht bös und misverſtehſt es nicht, wenn ich Dir sage, daß ich
               gerade in den Anfängen einer neuen Arbeit\pwindex{gelbe Nachtigall@\emph{Die gelbe Nachtigall}|pwv}{ }ſtecke und daher, bei der lächerlichen nervöſen
               Angſt, die ich dann immer habe, ich könnte über Nacht meinen Gegenſtand wieder
               vergeſſen oder er könnte mir entweichen, ſogar Deinen mir immer ſo lieben Beſuch
               etwas hinausgeſchoben {\pb}wünſchen würde, es wäre
               denn, daß Du irgend was Dringendes mit mir zu beſprechen hätteſt, in welchem Falle
               ich natürlich zu jeder Stunde an jedem Tage für Dich bereit bin.\pend
           
\pstart
           Mit den herzlichſten Grüßen, auch an Frau Olga\pwindex{Schnitzler, Olga 17.01.1882 – 13.01.1970@\textsc{Schnitzler, Olga} (17.01.1882 – 13.01.1970), \emph{Schauspieler/Schauspielerin, Sänger/Sängerin}|pw},{\\[\baselineskip]}Dein alter{\\[\baselineskip]}\spacefill\mbox{Hermann}\pend
           \leftskip=0em{}\selectlanguage{ngerman}\endnumbering\briefempfaengerindex{Schnitzler, Arthur@\textsc{Schnitzler, Arthur}!zzzBahr, Hermann@\emph{von Hermann Bahr}!1907-05-191@{19. 5. {[}1907{]}}|)be}\mylabel{L01674h}  \normalsize

\doendnotes{C}
\bigskip
\vfill

\clearpage

\footnotesize

\lohead{\textsc{register}}

% Definiere theindex-Environment komplett neu ohne reledmac
\makeatletter
\renewenvironment{theindex}{%
  \section*{\indexname}%
  \setlength{\parindent}{0pt}%
  \setlength{\parskip}{0pt plus 0.3pt}%
  \let\item\@idxitem
}{%
  \clearpage
}
\makeatother

\IfFileExists{\jobname-pw.ind}{\input{\jobname-pw.ind}}{}

\end{document}

      