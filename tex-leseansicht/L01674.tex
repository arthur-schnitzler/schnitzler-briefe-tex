\input{../tex-inputs/latex-pdf-vorspann}
\begin{center}
            \textcolor{red}{ENTWURF. ENTZIFFERUNG NOCH NICHT KORREKTURGELESEN}
                      \end{center}
            
               \section[Hermann Bahr an Arthur Schnitzler, 19. 5. {[}1907{]}]{ Hermann Bahr an Arthur Schnitzler, 19. 5. {[}1907{]}}\nopagebreak\mylabel{v}\rehead{ }\begin{ledgroupsized}[t]{13cm}\normalsize\beginnumbering\briefempfaengerindex{Schnitzler, Arthur@\textsc{Schnitzler, Arthur}!zzzBahr, Hermann@\emph{von Hermann Bahr}!1907-05-191@{19. 5. {[}1907{]}}|(be} \toendnotes[C]{\smallbreak\pagebreak[2]} \Standort{CUL, Schnitzler, B 5b.}
\physDesc{Brief, 1 Blatt, 2 Seiten
\newline{}Handschrift: blaue Tinte, deutsche Kurrent\newline{}Ordnung: mit Bleistift von unbekannter Hand nummeriert: »150« }\buchAbdrucke{\weitereDrucke{Hermann Bahr, Arthur Schnitzler: \emph{Briefwechsel, Aufzeichnungen, Dokumente (1891–1931)}. Hg. Kurt Ifkovits und Martin Anton Müller. Göttingen: \emph{Wallstein} 2018, S. 393.} }\toendnotes[C]{\smallbreak}\pstart
           \raggedleft{}{\pb}19. 5.\pend
           \pstart{}Lieber Arthur!\pend\pstart
           Danke ſchön für den zweiten Brehm\pwindex{\textcolor{red}{\textsuperscript{XXXX1 indx}}!Brehms Tierleben1863 – 1869@\strich\emph{Brehms Tierleben} {[}1863 – 1869{]}|pw}, den ich noch
               einige Zeit behalten möchte, er macht mir ein unſinniges Vergnügen.\pend
           \pstart
           Du biſt hoffentlich nicht bös und misverſtehſt es nicht, wenn ich Dir sage, daß ich
               gerade in den Anfängen einer neuen Arbeit\pwindex{Bahr, Hermann 19.07.1863 – 15.01.1934@\textsc{Bahr, Hermann} (19.07.1863 – 15.01.1934), \emph{Schriftsteller, Kritiker}!gelbe Nachtigall1907@\strich\emph{Die gelbe Nachtigall} {[}1907{]}|pwv}{ }ſtecke und daher, bei der lächerlichen nervöſen Angſt, die ich dann
               immer habe, ich könnte über Nacht meinen Gegenſtand wieder vergeſſen oder er könnte
               mir entweichen, ſogar Deinen mir immer ſo lieben Beſuch etwas hinausgeſchoben {\pb}wünſchen würde, es wäre denn, daß Du irgend was
               Dringendes mit mir zu beſprechen hätteſt, in welchem Falle ich natürlich zu jeder
               Stunde an jedem Tage für Dich bereit bin.\pend
           \pstart
           Mit den herzlichſten Grüßen, auch an Frau Olga\pwindex{Schnitzler, Olga 17.01.1882 – 13.01.1970@\textsc{Schnitzler, Olga} (17.01.1882 – 13.01.1970), \emph{Schauspielerin, Sängerin}|pw},{\\[\baselineskip]}Dein alter{\\[\baselineskip]}\spacefill\mbox{Hermann}\pend
           \leftskip=0em{}\endnumbering\briefempfaengerindex{Schnitzler, Arthur@\textsc{Schnitzler, Arthur}!zzzBahr, Hermann@\emph{von Hermann Bahr}!1907-05-191@{19. 5. {[}1907{]}}|)be}\mylabel{h}\end{ledgroupsized}  \newcommand{\dateiname}{L01674}\newcommand{\titel}{Hermann Bahr an Arthur Schnitzler, 19. 5. [1907]}\newcommand{\editorInnen}{ Kurt Ifkovits,  Martin Anton Müller}\input{../tex-inputs/latex-pdf-abspann}
      