%% latex-korrekturansicht-vorspann.tex
%% Vorspann für die Korrekturansicht.
%% Lädt die gemeinsame Datei latex-vorspann.tex mit gesetztem Schalter.

\newif\ifkorrekturansicht
\korrekturansichttrue

\input{../tex-inputs/latex-vorspann}


\section[Arthur Schnitzler an Gerhart Hauptmann, 18. 1. 1905]{L01491 Arthur Schnitzler an Gerhart Hauptmann, 18. 1. 1905}
\nopagebreak\mylabel{L01491v}
\rehead{ }\normalsize\beginnumbering\briefempfaengerindex{Hauptmann, Gerhart@\textsc{Hauptmann, Gerhart}!zzzSchnitzler, Arthur@\emph{von Arthur Schnitzler}!1905-01-181@{18. 1. 1905}|(be}
\toendnotes[C]{\smallbreak\pagebreak[2]}\Standort{Staatsbibliothek Berlin – Preußischer Kulturbesitz, GHBrBl A:Schnitzler (10).}
\physDesc{Brief, 1 Blatt, 1 Seite, 279 Zeichen
\newline{}Handschrift: schwarze Tinte, deutsche Kurrent}\toendnotes[C]{\smallbreak}
\pstart
           \raggedleft{}{\pb}XVIII Spoettelgasse \label{T_L01491-1v}\edtext{7}{\lemma{\textnormal{\emph{7}}}\Cendnote{\textnormal{eine
                        undeutliche »7« durchgestrichen und daneben neuerlich
                        hingeschrieben}}}\label{T_L01491-1}\oindex{Edmund-Weiss-Gasse 7@\textbf{Edmund-Weiß-Gasse 7}, \emph{Wohngebäude (K.WHS)}|pw}{\\}Wien\oindex{Wien@\textbf{Wien}, \emph{A.ADM2}|pw}{ }18. 1. 905\pend
           
\pstart{}Lieber Herr Hauptmann,\pend\vspace{0.5em}
\pstart
           ich gratulire Ihnen herzlich zum nächſten Grillparzer-Preis\orgindex{Franz-Grillparzer-Preis@Franz-Grillparzer-Preis|pw}. Und zur Elga\pwindex{Elga@\emph{Elga}|pw} nicht
               minder, von der ich hoffe, daſs ſie ſo wie ſie iſt, auf die Bühne kommen möge.\pend
           
\pstart
           Mit den ſchönſten Grüßen an Sie und Frau Grethe\pwindex{Hauptmann, Margarete 07.01.1875 – 17.01.1957@\textsc{Hauptmann, Margarete} (07.01.1875 – 17.01.1957)|pw}{\\[\baselineskip]}Ihr{\\[\baselineskip]}\spacefill\mbox{Arthur Schnitzler}\pend
           \leftskip=0em{}\selectlanguage{ngerman}\endnumbering\briefempfaengerindex{Hauptmann, Gerhart@\textsc{Hauptmann, Gerhart}!zzzSchnitzler, Arthur@\emph{von Arthur Schnitzler}!1905-01-181@{18. 1. 1905}|)be}\mylabel{L01491h}  \normalsize

\doendnotes{C}
\bigskip
\vfill

\clearpage

\footnotesize

\lohead{\textsc{register}}

% Definiere theindex-Environment komplett neu ohne reledmac
\makeatletter
\renewenvironment{theindex}{%
  \section*{\indexname}%
  \setlength{\parindent}{0pt}%
  \setlength{\parskip}{0pt plus 0.3pt}%
  \let\item\@idxitem
}{%
  \clearpage
}
\makeatother

\IfFileExists{\jobname-pw.ind}{\input{\jobname-pw.ind}}{}

\end{document}

      