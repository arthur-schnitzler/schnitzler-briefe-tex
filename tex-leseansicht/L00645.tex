%% latex-leseansicht-vorspann.tex
%% Vorspann für die Leseansicht.
%% Lädt die gemeinsame Datei latex-vorspann.tex mit nicht gesetztem Schalter.

\newif\ifkorrekturansicht
\korrekturansichtfalse

\input{../tex-inputs/latex-vorspann}


\section[Arthur Schnitzler an Hugo von Hofmannsthal, {{[}}9. 2. 1897?{{]}}]{L00645 Arthur Schnitzler an Hugo von Hofmannsthal, {[}9. 2. 1897?{]}}
\nopagebreak\mylabel{L00645v}
\rehead{ }\normalsize\beginnumbering\briefempfaengerindex{Hofmannsthal, Hugo von@\textsc{Hofmannsthal, Hugo von}!zzzSchnitzler, Arthur@\emph{von Arthur Schnitzler}!1897-02-092@{{[}9. 2. 1897?{]}}|(be}
\toendnotes[C]{\smallbreak\pagebreak[2]}
\correspDesc{Versand  durch Arthur Schnitzler am [9. 2. 1897?] in Wien
\newline{}Erhalt  durch Hugo von Hofmannsthal im Zeitraum [9. 2. 1897
                  – 13. 2. 1897?] in Wien}\toendnotes[C]{\smallbreak}
\Standort{FDH, Hs-30885,54.}
\physDesc{Brief, 1 Blatt, 4 Seiten, 599 Zeichen
\newline{}Handschrift: Bleistift, deutsche Kurrent
\newline{}Ordnung: mit Bleistift von Schnitzler mutmaßlich bei der Durchsicht der Korrespondenz
                                    1929  datiert: »Anf 97« }
\buchAbdrucke{\weitereDrucke{Hugo von Hofmannsthal, Arthur Schnitzler: \emph{Briefwechsel}. Herausgegeben von Therese Nickl und Heinrich Schnitzler. Frankfurt am Main: \emph{S. Fischer} 1964, S. 78.} }
\pstart
           \noindent{}{\pb}Lieber Hugo, ich habe der \textsc{Minnie}\pwindex{Schaffgotsch, Hermine von 25.\,11.\,1871 Wien – 25.\,11.\,1928 Purgstall@\textsc{Schaffgotsch, Hermine von} (25.\,11.\,1871 Wien – 25.\,11.\,1928 Purgstall)|pw}{ }\textsc{teleph.} wa{\geminationn} morgen Probe{ }ſei,{ }ſie antwortete noch nicht besti{\geminationm}t, wahrſcheinlich
                  ½ 6; da{\geminationn} fragte ich, ob{ }ſie heute zu W.s\pwindex{Wärndorfer, August 30.\,3.\,1865 Wien – 17.\,2.\,1940 Schottland@\textsc{Wärndorfer, August} (30.\,3.\,1865 Wien – 17.\,2.\,1940 Schottland), \emph{Industrieller}|pw}\pwindex{Wärndorfer, Adrienne 10.\,1.\,1876 Wien – 17.\,1.\,1960 New York City@\textsc{Wärndorfer, Adrienne} (10.\,1.\,1876 Wien – 17.\,1.\,1960 New York City)|pw} komme, {\pb}worauf{ }ſie{ }ſagte,{ }ſie glaube nicht.\pend
           
\pstart
           Damit war das Geſpräch (»Alſo auf Wiederſehen« (ich)) beendet.\pend
           
\pstart
           Ich gehe alſo nicht zu W.s\pwindex{Wärndorfer, August 30.\,3.\,1865 Wien – 17.\,2.\,1940 Schottland@\textsc{Wärndorfer, August} (30.\,3.\,1865 Wien – 17.\,2.\,1940 Schottland), \emph{Industrieller}|pw}\pwindex{Wärndorfer, Adrienne 10.\,1.\,1876 Wien – 17.\,1.\,1960 New York City@\textsc{Wärndorfer, Adrienne} (10.\,1.\,1876 Wien – 17.\,1.\,1960 New York City)|pw}. Die
               Möglichkeit iſt zu bedenken, daſs{ }ſie nur nicht will, dſs \uline{ich} heut hinaus komme. {\pb}Vielleicht haben Sie \substVorne{}\textsuperscript{ke}\substDazwischen{}ir\substHinten{}gend eine Nachricht.\pend
           
\pstart
           Wollen Sie noch was wiſſen,{ }ſo können Sie mir wohl zu \textsc{Loebs}\pwindex{Loeb, Louis 29.\,6.\,1842 Mattersdorf – 6.\,6.\,1921 Wien@\textsc{Loeb, Louis} (29.\,6.\,1842 Mattersdorf – 6.\,6.\,1921 Wien), \emph{Bankier}|pw}\pwindex{Loeb, Regina 1850 – 5.\,2.\,1918 Wien@\textsc{Loeb, Regina} (1850 – 5.\,2.\,1918 Wien)|pw}{ }\textsc{teleph}. Ich bleibe dort wohl bis ½ 5 oder
                  5, da{\geminationn} geh ich zu mir nach Haus. Spät
               Abds (½ 11 denk ich) {\pb}bin ich im \textsc{Pucher}\oindex{Wien@\textbf{Wien}!I., Innere Stadt@\textbf{I., Innere Stadt}!Café Pucher@\textbf{Café Pucher}, \emph{Kaffeehaus}|pw}. –\pend
           
\pstart
           Herzlich der Ihre{\\[\baselineskip]}\spacefill\mbox{Arthur}\pend
           \leftskip=0em{}\selectlanguage{ngerman}\endnumbering\briefempfaengerindex{Hofmannsthal, Hugo von@\textsc{Hofmannsthal, Hugo von}!zzzSchnitzler, Arthur@\emph{von Arthur Schnitzler}!1897-02-092@{{[}9. 2. 1897?{]}}|)be}\mylabel{L00645h}  \newcommand{\dateiname}{L00645}\newcommand{\titel}{Arthur Schnitzler an Hugo von Hofmannsthal, [9. 2. 1897?]}\newcommand{\editorInnen}{Martin Anton Müller und Gerd-Hermann Susen}%% latex-leseansicht-abspann.tex
%% Abspann für die Leseansicht.
%% Der Schalter \ifkorrekturansicht ist bereits durch den Vorspann gesetzt.

%% latex-abspann.tex
%% Gemeinsamer Abspann für Korrekturansicht und Leseansicht.
%% Setzt den Schalter \ifkorrekturansicht voraus (gesetzt in den
%% einbindenden Dateien latex-korrekturansicht-abspann.tex bzw.
%% latex-leseansicht-abspann.tex).
%% ---------------------------------------------------------------

\normalsize

% Das esempio-Environment wird nur in der Leseansicht benötigt
\ifkorrekturansicht\else
\newenvironment{esempio}[3]%
{
    \vspace{1.5ex}
    \rlap{\underline{#1}}
    \par
    \setlength{\parindent}{0cm}
    \nopagebreak
    \leftskip=#2cm
    \rightskip=#3cm
}
{
    \par
}
\fi

\doendnotes{C}
\bigskip
\vfill

\clearpage

\footnotesize

\ifkorrekturansicht
  \lohead{\textsc{register}}
\fi

% theindex-Environment neu definieren ohne reledmac
\makeatletter
\renewenvironment{theindex}{%
  \ifkorrekturansicht
    \section*{\indexname}%
  \else
    \subsubsection*{Index der erwähnten Entitäten}%
  \fi
  \setlength{\parindent}{0pt}%
  \setlength{\parskip}{0pt plus 0.3pt}%
  \let\item\@idxitem
}{%
  \ifkorrekturansicht\clearpage\fi
}
\makeatother

\IfFileExists{\jobname-pw.ind}{\input{\jobname-pw.ind}}{}

% Quellenangabe nur in der Leseansicht
\ifkorrekturansicht\else
% Fallback-Definitionen, falls die .tex-Datei \titel etc. nicht gesetzt hat
\providecommand{\titel}{}
\providecommand{\editorInnen}{}
\providecommand{\dateiname}{\jobname}

\vspace{3cm}

\vfill

\footnotesize
\textsc{Quelle}: \titel. Herausgegeben von {\editorInnen}. In: \emph{Arthur Schnitzler: Briefwechsel mit Autorinnen und Autoren}.
 Digitale Edition, https://schnitzler-briefe.acdh.oeaw.ac.at/{\dateiname}.html (Stand \today)
\fi

\end{document}


