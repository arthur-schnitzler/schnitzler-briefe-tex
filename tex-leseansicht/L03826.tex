%% latex-leseansicht-vorspann.tex
%% Vorspann für die Leseansicht.
%% Lädt die gemeinsame Datei latex-vorspann.tex mit nicht gesetztem Schalter.

\newif\ifkorrekturansicht
\korrekturansichtfalse

\input{../tex-inputs/latex-vorspann}


\section[Theodor Herzl an Arthur Schnitzler, 2. 1. 1893]{L03826 Theodor Herzl an Arthur Schnitzler, 2. 1. 1893}
\nopagebreak\mylabel{L03826v}
\rehead{ }\normalsize\beginnumbering\briefempfaengerindex{Schnitzler, Arthur@\textsc{Schnitzler, Arthur}!zzzHerzl, Theodor@\emph{von Theodor Herzl}!1893-01-021@{2. 1. 1893}|(be}
\toendnotes[C]{\smallbreak\pagebreak[2]}
\correspDesc{Versand  durch Theodor Herzl am 2. 1. 1893 in Paris
\newline{}Erhalt  durch Arthur Schnitzler im Zeitraum [3. 1. 1893
                  – 7. 1. 1893?] in Wien}\toendnotes[C]{\smallbreak}
\Standort{CUL, Schnitzler, B 39.}
\physDesc{Brief, 1 Blatt, 3 Seiten, 1539 Zeichen
\newline{}Handschrift: schwarze Tinte, lateinische Kurrent
\newline{}Ordnung: mit Bleistift von unbekannter Hand nummeriert: »7« 
\newline{}Zusatz: mit blauem Buntstift Markierungen für den Druck von Leon Kellner }
\buchAbdrucke{\weitereDrucke{1) \pwindex{Kellner, Leon 17.\,4.\,1859 Tarnów – 5.\,12.\,1928 Wien@\textsc{Kellner, Leon} (17.\,4.\,1859 Tarnów – 5.\,12.\,1928 Wien), \emph{Zionist, Literaturhistoriker, Anglist}!Theodor Herzls Lehrjahre (1860–1895). Nach den handschriftlichen Quellen@\strich\emph{Theodor Herzls Lehrjahre (1860–1895). Nach den handschriftlichen Quellen}|pwk}\emph{[Auszug].} In: Leon Kellner: \emph{Theodor Herzls Lehrjahre (1860–1895). Nach den handschriftlichen Quellen}. Wien, Berlin: \emph{R. Löwit-Verlag} 1920, S. 113.} \weitereDrucke{2) M. [=Hermann Menkes]: \emph{Theodor Herzl’s Abkehr vom Theater. Interessante Briefe an Artur Schnitzler.} In: \emph{Neues Wiener Journal}, Jg. 28, Nr. 9543, 1. 6. 1920, S. 4.} \weitereDrucke{3) \emph{Herzl-Briefe}. Herausgegeben und eingeleitet Manfred Georg. Berlin: \emph{Brandusche Verlagsbuchhandlung} [1935], S. 31.} \weitereDrucke{4) Theodor Herzl: \emph{Briefe und
                        autobiographische Notizen 1866–1895}. Bearbeitet von Johannes Wachten in Zusammenarbeit mit Chaya Harel, Daisy Tycho und Manfred Winkler. Berlin, Frankfurt am Main, Wien: \emph{Propyläen} 1983, S. 508–509 (Briefe und Tagebücher. Herausgegeben von Alex Bein, Hermann Greive, Moshe Schaerf, Julius H. Schoeps und Johannes Wachten, 1).} }\toendnotes[C]{\smallbreak}
\pstart
           {\pb}\textcolor{gray}{\textbf{NOUVELLE PRESSE LIBRE }}\orgindex{Neue Freie Presse@Neue Freie Presse|pw}\hfill \textcolor{gray}{\textbf{8, Rue de Monceau }}\oindex{8, rue de Monceau@\textbf{8, rue de Monceau}, \emph{Wohngebäude}|pw}\pend
           
\pstart
           \textcolor{gray}{\textbf{D\textsuperscript{r}{ }TH. HERZL}}\hfill 2. I. 93\pend
           
\pstart{}Lieber Freund!\pend\vspace{0.5em}
\pstart
           Wenn Sie nur keinen Neujahrsbrief geschrieben hätten – noch dazu wars ein sehr lieber
               – würde ich mit der Antwort gewartet haben, bis ich Zeit gehabt hätte.\pend
           
\pstart
           So will ich Ihnen heute nur in Eile danken und Ihre freundlichen Grüsse herzlich
               erwiedern.\pend
           
\pstart
           Meine Manuscripte! Ich habe sie vergessen. Von der Kunstübung ist mir nur etwas Liebe
               zur Kunst geblieben u. an manchen Tagen oder in verlorenen Stunden ein Heimweh nach
               der Dichtung. Nicht ungestraft ist {\pb}man
               Journalist. Ich bemühe mich, dieses Métier, das der reizende kleine \label{K_L03826-1v}\edtext{Hoffmannsthal\pwindex{Hofmannsthal, Hugo von 1.\,2.\,1874 Wien – 15.\,7.\,1929 Rodaun@\textsc{Hofmannsthal, Hugo von} (1.\,2.\,1874 Wien – 15.\,7.\,1929 Rodaun), \emph{Schriftsteller}|pw} verachtet}{\lemma{\textnormal{\emph{Hoffmannsthal verachtet}}}\Cendnote{\textnormal{Hofmannsthals\pwindex{Hofmannsthal, Hugo von 1.\,2.\,1874 Wien – 15.\,7.\,1929 Rodaun@\textsc{Hofmannsthal, Hugo von} (1.\,2.\,1874 Wien – 15.\,7.\,1929 Rodaun), \emph{Schriftsteller}|pwk} Brief, mit dem er Herzl\pwindex{Herzl, Theodor 2.\,5.\,1860 Budapest – 3.\,7.\,1904 Edlach@\textsc{Herzl, Theodor} (2.\,5.\,1860 Budapest – 3.\,7.\,1904 Edlach), \emph{Schriftsteller, Journalist}|pwk} sein Stück \emph{Gestern}\pwindex{Hofmannsthal, Hugo von 1.\,2.\,1874 Wien – 15.\,7.\,1929 Rodaun@\textsc{Hofmannsthal, Hugo von} (1.\,2.\,1874 Wien – 15.\,7.\,1929 Rodaun), \emph{Schriftsteller}!Gestern. Dramatische Studie in einem Akt in Versen@\strich\emph{Gestern. Dramatische Studie in einem Akt in Versen}|pwk} übersandte, ist nicht überliefert, aber aus Herzls\pwindex{Herzl, Theodor 2.\,5.\,1860 Budapest – 3.\,7.\,1904 Edlach@\textsc{Herzl, Theodor} (2.\,5.\,1860 Budapest – 3.\,7.\,1904 Edlach), \emph{Schriftsteller, Journalist}|pwk} Antwort vom 2. 11. 1892
                  geht hervor, dass Hofmannsthal\pwindex{Hofmannsthal, Hugo von 1.\,2.\,1874 Wien – 15.\,7.\,1929 Rodaun@\textsc{Hofmannsthal, Hugo von} (1.\,2.\,1874 Wien – 15.\,7.\,1929 Rodaun), \emph{Schriftsteller}|pwk} eine Kritik
                  am Journalismus geäußert hat: »Ihre Antipathie gegen mein Métier theile ich
                  vollkommen und bin Ihnen sehr dankbar, dass Sie sie soweit überwunden haben, um an
                  mich zu schreiben.« (Theodor Herzl\pwindex{Herzl, Theodor 2.\,5.\,1860 Budapest – 3.\,7.\,1904 Edlach@\textsc{Herzl, Theodor} (2.\,5.\,1860 Budapest – 3.\,7.\,1904 Edlach), \emph{Schriftsteller, Journalist}|pwk}: \emph{Briefe
                        und Tagebücher}. Herausgegeben von Alex Bein, Hermann Greive, Moshe
                     Schaerf und Julius H. Schoeps, Bd. 1: \emph{Briefe und
                        autobiographische Notizen 1866–1895}, bearbeitet von Johannes
                     Wachten. Berlin, Frankfurt am Main, Wien: \emph{Propyläen}{ }1983, S. 504–505).}}}\label{K_L03826-1}, so \label{K_L03826-2v}\edtext{unpanamistisch}{\lemma{\textnormal{\emph{unpanamistisch}}}\Cendnote{\textnormal{Das Bekanntwerden von Bestechungen von Abgeordenten und Journalisten im großen
                        Stil beim Bauprojekt des Panamakanals\oindex{Panamakanal@\textbf{Panamakanal}|pwk} erschütterte im Herbst 1892{ }Frankreich\oindex{Frankreich@\textbf{Frankreich}|pwk}. Die Ereignisse hinter dem Skandal wurden zum Inbegriff
                  von Korruption.}}}\label{K_L03826-2} als möglich zu betreiben, und schaue der Politik zu.
               Manchmal komme ich mir vor, wie \label{K_L03826-3v}\edtext{David Copperfield\pwindex{Dickens, Charles 7.\,2.\,1812 Landport – 9.\,6.\,1870 Rochester [England]@\textsc{Dickens, Charles} (7.\,2.\,1812 Landport – 9.\,6.\,1870 Rochester [England]), \emph{Schriftsteller, Schriftsteller}!David Copperfield@\strich\emph{David Copperfield}|pwv} der
                  Stenograph}{\lemma{\textnormal{\emph{David … Stenograph}}}\Cendnote{\textnormal{Der Protagonist von Charles Dickens\pwindex{Dickens, Charles 7.\,2.\,1812 Landport – 9.\,6.\,1870 Rochester [England]@\textsc{Dickens, Charles} (7.\,2.\,1812 Landport – 9.\,6.\,1870 Rochester [England]), \emph{Schriftsteller, Schriftsteller}|pwk}{ }Roman\pwindex{Dickens, Charles 7.\,2.\,1812 Landport – 9.\,6.\,1870 Rochester [England]@\textsc{Dickens, Charles} (7.\,2.\,1812 Landport – 9.\,6.\,1870 Rochester [England]), \emph{Schriftsteller, Schriftsteller}!David Copperfield@\strich\emph{David Copperfield}|pwkv}
                  erlernt im 38. Kapitel unter Mühen Kurzschrift, um sich als
                  Parlamentsprotokollant zu verdingen.}}}\label{K_L03826-3} – erinnern Sie sich der wonnevollen
               Stelle? — u. manchmal halte ich mich für einen Staatsjuristen. Wirklich ist es in
               dieser Zeit interessant, der Politik zuzuschauen. Ich glaube, es wird hier heuer eine
               Revolution geben, u. wenn ich nicht rechtzeitig nach Brüssel\oindex{Brüssel@\textbf{Brüssel}, \emph{Hauptstadt}|pw} entkomme, werden sie mich vielleicht füsiliren, als Bourgeois oder
               deutschen Spion oder \strikeout{weiteren} Juden, oder Financier –
               während ich doch nur ein ausgedienter Seiltänzer bin.\pend
           
\pstart
           Wenn ich Zeit hätte, glaub’ ich, {\pb}könnte
               ich ein merkwürdiges Buch schreiben über das was ich in Paris\oindex{Paris@\textbf{Paris}, \emph{Hauptstadt}|pw} gesehen habe. Die politische Conclusion wäre: das Beste
               für das Volk ist ein »\begin{otherlanguage}{french}bon tyran\end{otherlanguage}«, was ja Renan\pwindex{Renan, Ernest 27.\,2.\,1823 Tréguier – 2.\,10.\,1892 Paris@\textsc{Renan, Ernest} (27.\,2.\,1823 Tréguier – 2.\,10.\,1892 Paris), \emph{Schriftsteller, Orientalist}|pw} gefunden hat. Ich erzähle das nicht
                  \label{K_L03826-4v}\edtext{\begin{otherlanguage}{french}pour rompre les chiens\end{otherlanguage}}{\lemma{\textnormal{\emph{pour rompre les chiens}}}\Cendnote{\textnormal{französisch: um die Hunde von der Fährte
                  abzulenken, das Thema zu wechseln (übertr.)}}}\label{K_L03826-4}{ }– wenn ich die alte Kiste mit
               den alten Manuscripten irgendwo finde, will ich Ihnen ein altes Stück schicken.\pend
           
\pstart
           Ich grüsse Sie herzlich{\\[\baselineskip]} Ihr Freund{\\[\baselineskip]}\spacefill\mbox{Herzl}\pend
           \leftskip=0em{}\selectlanguage{ngerman}\endnumbering\briefempfaengerindex{Schnitzler, Arthur@\textsc{Schnitzler, Arthur}!zzzHerzl, Theodor@\emph{von Theodor Herzl}!1893-01-021@{2. 1. 1893}|)be}\mylabel{L03826h}
\begin{anhang}
\end{anhang}\newcommand{\dateiname}{L03826}\newcommand{\titel}{Theodor Herzl an Arthur Schnitzler, 2. 1. 1893}\newcommand{\editorInnen}{Selma Jahnke und Martin Anton Müller}%% latex-leseansicht-abspann.tex
%% Abspann für die Leseansicht.
%% Der Schalter \ifkorrekturansicht ist bereits durch den Vorspann gesetzt.

%% latex-abspann.tex
%% Gemeinsamer Abspann für Korrekturansicht und Leseansicht.
%% Setzt den Schalter \ifkorrekturansicht voraus (gesetzt in den
%% einbindenden Dateien latex-korrekturansicht-abspann.tex bzw.
%% latex-leseansicht-abspann.tex).
%% ---------------------------------------------------------------

\normalsize

% Das esempio-Environment wird nur in der Leseansicht benötigt
\ifkorrekturansicht\else
\newenvironment{esempio}[3]%
{
    \vspace{1.5ex}
    \rlap{\underline{#1}}
    \par
    \setlength{\parindent}{0cm}
    \nopagebreak
    \leftskip=#2cm
    \rightskip=#3cm
}
{
    \par
}
\fi

\doendnotes{C}
\bigskip
\vfill

\clearpage

\footnotesize

\ifkorrekturansicht
  \lohead{\textsc{register}}
\fi

% theindex-Environment neu definieren ohne reledmac
\makeatletter
\renewenvironment{theindex}{%
  \ifkorrekturansicht
    \section*{\indexname}%
  \else
    \subsubsection*{Index der erwähnten Entitäten}%
  \fi
  \setlength{\parindent}{0pt}%
  \setlength{\parskip}{0pt plus 0.3pt}%
  \let\item\@idxitem
}{%
  \ifkorrekturansicht\clearpage\fi
}
\makeatother

\IfFileExists{\jobname-pw.ind}{\input{\jobname-pw.ind}}{}

% Quellenangabe nur in der Leseansicht
\ifkorrekturansicht\else
% Fallback-Definitionen, falls die .tex-Datei \titel etc. nicht gesetzt hat
\providecommand{\titel}{}
\providecommand{\editorInnen}{}
\providecommand{\dateiname}{\jobname}

\vspace{3cm}

\vfill

\footnotesize
\textsc{Quelle}: \titel. Herausgegeben von {\editorInnen}. In: \emph{Arthur Schnitzler: Briefwechsel mit Autorinnen und Autoren}.
 Digitale Edition, https://schnitzler-briefe.acdh.oeaw.ac.at/{\dateiname}.html (Stand \today)
\fi

\end{document}


