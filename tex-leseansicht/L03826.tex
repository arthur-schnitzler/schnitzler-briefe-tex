%% latex-korrekturansicht-vorspann.tex
%% Vorspann für die Korrekturansicht.
%% Lädt die gemeinsame Datei latex-vorspann.tex mit gesetztem Schalter.

\newif\ifkorrekturansicht
\korrekturansichttrue

\input{../tex-inputs/latex-vorspann}


\section[Theodor Herzl an Arthur Schnitzler, 2. 1. 1893]{L03826 Theodor Herzl an Arthur Schnitzler, 2. 1. 1893}
\nopagebreak\mylabel{L03826v}
\rehead{ }\normalsize\beginnumbering\briefempfaengerindex{Schnitzler, Arthur@\textsc{Schnitzler, Arthur}!zzzHerzl, Theodor@\emph{von Theodor Herzl}!1893-01-021@{2. 1. 1893}|(be}
\toendnotes[C]{\smallbreak\pagebreak[2]}\Standort{CUL, Schnitzler, B 39.}
\physDesc{Brief, 1 Blatt, 3 Seiten, 1539 Zeichen
\newline{}Handschrift: schwarze Tinte, lateinische Kurrent
\newline{}Ordnung: mit Bleistift von unbekannter Hand nummeriert: »7« 
\newline{}Zusatz: mit blauem Buntstift Markierungen für den Druck von Leon Kellner }\toendnotes[C]{\smallbreak}
\pstart
           {\pb}\textcolor{gray}{\textbf{NOUVELLE PRESSE LIBRE }}\orgindex{Neue Freie Presse@Neue Freie Presse|pw}\hfill \textcolor{gray}{\textbf{8, Rue de Monceau }}\oindex{8, Rue de Monceau@\textbf{8, Rue de Monceau}, \emph{Wohngebäude (K.WHS)}|pw}\pend
           
\pstart
           \textcolor{gray}{\textbf{D\textsuperscript{R} TH. HERZL}}\hfill 2. I. 93\pend
           
\pstart{}Lieber Freund!\pend\vspace{0.5em}
\pstart
           Wenn Sie nur keinen Neujahrsbrief geschrieben hätten – noch dazu wars ein sehr lieber
               – würde ich mit der Antwort gewartet haben, bis ich Zeit gehabt hätte. \pend
           
\pstart
           So will ich Ihnen heute nur in Eile danken und Ihre freundlichen Grüsse herzlich
               erwiedern. \pend
           
\pstart
           Meine Manuscripte! Ich habe sie vergessen. Von der Kunstübung ist mir nur etwas Liebe
               zur Kunst geblieben u. an manchen Tagen oder in verlorenen Stunden ein Heimweh nach
               der Dichtung. Nicht ungestraft ist {\pb}man
               Journalist. Ich bemühe mich, dieses Métier, das der reizende kleine \label{K_L03826-1v}\edtext{Hoffmannsthal\pwindex{Hofmannsthal, Hugo von 1874-02-01 – 1929-07-15@\textsc{Hofmannsthal, Hugo von} (1874-02-01 – 1929-07-15), \emph{Schriftsteller/Schriftstellerin}|pw} verachtet}{\lemma{\textnormal{\emph{Hoffmannsthal verachtet}}}\Cendnote{\textnormal{Hofmannsthals\pwindex{Hofmannsthal, Hugo von 1874-02-01 – 1929-07-15@\textsc{Hofmannsthal, Hugo von} (1874-02-01 – 1929-07-15), \emph{Schriftsteller/Schriftstellerin}|pwk} Brief, mit dem er Herzl\pwindex{Herzl, Theodor 1860-05-02 – 1904-07-03@\textsc{Herzl, Theodor} (1860-05-02 – 1904-07-03), \emph{Schriftsteller/Schriftstellerin, Journalist/Journalistin}|pwk} sein Stück \emph{Gestern}\pwindex{Gestern. Dramatische Studie in einem Akt in Versen@\emph{Gestern. Dramatische Studie in einem Akt in Versen}|pwk} übersandte, ist nicht überliefert, aber aus Herzls\pwindex{Herzl, Theodor 1860-05-02 – 1904-07-03@\textsc{Herzl, Theodor} (1860-05-02 – 1904-07-03), \emph{Schriftsteller/Schriftstellerin, Journalist/Journalistin}|pwk} Antwort vom 2. 11. 1892
                  geht hervor, dass Hofmannsthal\pwindex{\textcolor{red}{\textsuperscript{XXXX indx}}|pwk} eine Kritik
                  am Journalismus geäußert hat: »Ihre Antipathie gegen mein Métier theile ich
                  vollkommen und bin Ihnen sehr dankbar, dass Sie sie soweit überwunden haben, um an
                  mich zu schreiben.« (Theodor Herzl\pwindex{Herzl, Theodor 1860-05-02 – 1904-07-03@\textsc{Herzl, Theodor} (1860-05-02 – 1904-07-03), \emph{Schriftsteller/Schriftstellerin, Journalist/Journalistin}|pwk}: \emph{Briefe
                        und Tagebücher}. Herausgegeben von Alex Bein, Hermann Greive, Moshe
                     Schaerf und Julius H. Schoeps, Bd. 1: \emph{Briefe und
                        autobiographische Notizen 1866–1895}, bearbeitet von Johannes
                     Wachten. Berlin, Frankfurt am Main, Wien>: \emph{Propyläen}{ }1983, S. 504–505).}}}\label{K_L03826-1}, so \label{K_L03826-2v}\edtext{unpanamistisch}{\lemma{\textnormal{\emph{unpanamistisch}}}\Cendnote{\textnormal{Das Bekanntwerden von Bestechungen von Abgeordenten und Journalisten im großen
                        Stil beim Bauprojekt des Panamakanals\oindex{Panamakanal@\textbf{Panamakanal}, \emph{Kanal}|pwk} erschütterte im Herbst 1892{ }Frankreich\oindex{Frankreich@\textbf{Frankreich}, \emph{A.PCLI}|pwk}. Die Ereignisse hinter dem Skandal wurden zum Inbegriff
                  von Korruption.}}}\label{K_L03826-2} als möglich zu betreiben, und schaue der Politik zu.
               Manchmal komme ich mir vor, wie \label{K_L03826-3v}\edtext{David Copperfield\pwindex{David Copperfield@\emph{David Copperfield}|pwv} der
                  Stenograph}{\lemma{\textnormal{\emph{David … Stenograph}}}\Cendnote{\textnormal{Der Protagonist von Charles Dickens\pwindex{Dickens, Charles 07.02.1812 – 09.06.1870@\textsc{Dickens, Charles} (07.02.1812 – 09.06.1870), \emph{Schriftsteller/Schriftstellerin, Schriftsteller/Schriftstellerin}|pwk}{ }Roman\pwindex{David Copperfield@\emph{David Copperfield}|pwkv}
                  erlernt im 38. Kapitel unter Mühen Kurzschrift, um sich als
                  Parlamentsprotokollant zu verdingen.}}}\label{K_L03826-3} – erinnern Sie sich der wonnevollen
               Stelle? — u. manchmal halte ich mich für einen Staatsjuristen. Wirklich ist es in
               dieser Zeit interessant, der Politik zuzuschauen. Ich glaube, es wird hier heuer eine
               Revolution geben, u. wenn ich nicht rechtzeitig nach Brüssel\oindex{Bruessel@\textbf{Brüssel}, \emph{P.PPLC}|pw} entkomme, werden sie mich vielleicht füsiliren, als Bourgeois oder
               deutschen Spion oder \strikeout{weiteren} Juden, oder Financier –
               während ich doch nur ein ausgedienter Seiltänzer bin.\pend
           
\pstart
           Wenn ich Zeit hätte, glaub' ich, {\pb}könnte
               ich ein merkwürdiges Buch schreiben über das was ich in Paris\oindex{Paris@\textbf{Paris}, \emph{P.PPLC}|pw} gesehen habe. Die politische Conclusion wäre: das Beste
               für das Volk ist ein »\begin{otherlanguage}{french}bon tyran\end{otherlanguage}«, was ja Renan\pwindex{Renan, Ernest 27.02.1823 – 02.10.1892@\textsc{Renan, Ernest} (27.02.1823 – 02.10.1892), \emph{Schriftsteller/Schriftstellerin, Orientalist/Orientalistin}|pw} gefunden hat. Ich erzähle das nicht
                  \label{K_L03826-4v}\edtext{\begin{otherlanguage}{french}pour rompre les chiens\end{otherlanguage}}{\lemma{\textnormal{\emph{pour rompre les chiens}}}\Cendnote{\textnormal{französisch: um die Hunde von der Fährte
                  abzulenken, das Thema zu wechseln (übertr.)}}}\label{K_L03826-4} – wenn ich die alte Kiste mit
               den alten Manuscripten irgendwo finde, will ich Ihnen ein altes Stück schicken.\pend
           
\pstart
           Ich grüsse Sie herzlich{\\[\baselineskip]} Ihr Freund{\\[\baselineskip]}\spacefill\mbox{Herzl}\pend
           \leftskip=0em{}\selectlanguage{ngerman}\endnumbering\briefempfaengerindex{Schnitzler, Arthur@\textsc{Schnitzler, Arthur}!zzzHerzl, Theodor@\emph{von Theodor Herzl}!1893-01-021@{2. 1. 1893}|)be}\mylabel{L03826h}
\begin{anhang}
\end{anhang}\normalsize

\doendnotes{C}
\bigskip
\vfill

\clearpage

\footnotesize

\lohead{\textsc{register}}

% Definiere theindex-Environment komplett neu ohne reledmac
\makeatletter
\renewenvironment{theindex}{%
  \section*{\indexname}%
  \setlength{\parindent}{0pt}%
  \setlength{\parskip}{0pt plus 0.3pt}%
  \let\item\@idxitem
}{%
  \clearpage
}
\makeatother

\IfFileExists{\jobname-pw.ind}{\input{\jobname-pw.ind}}{}

\end{document}

      