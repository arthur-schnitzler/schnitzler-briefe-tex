%% latex-leseansicht-vorspann.tex
%% Vorspann für die Leseansicht.
%% Lädt die gemeinsame Datei latex-vorspann.tex mit nicht gesetztem Schalter.

\newif\ifkorrekturansicht
\korrekturansichtfalse

\input{../tex-inputs/latex-vorspann}


         
         \newcommand{\erwaehntePersonen}{Personen: }
         \newcommand{\erwaehnteInstitutionen}{}
         \newcommand{\erwaehnteOrte}{}
         \newcommand{\erwaehnteWerke}{
               \section[Hugo von Hofmannsthal an Arthur Schnitzler, {[}19. 4. 1898{]}]{ Hugo von Hofmannsthal an Arthur Schnitzler, {[}19. 4. 1898{]}}\nopagebreak\mylabel{v}\rehead{ }\begin{ledgroupsized}[t]{13cm}\normalsize\beginnumbering \toendnotes[C]{\smallbreak\pagebreak[2]} \Standort{CUL, Schnitzler, B 43b/1.}
\physDesc{Brief, 1 Blatt (Briefkopf mit Möwen und einem Segelschiff), 3 Seiten
\newline{}Handschrift: schwarze Tinte, deutsche Kurrent
\newline{}Schnitzler: mit Bleistift datiert: »19/4/98« \newline{}Ordnung: 1) mit Bleistift von unbekannter Hand nummeriert: »\strikeout{113}«  2) mit Bleistift von unbekannter Hand nummeriert: »111«}\buchAbdrucke{\weitereDrucke{Hugo von Hofmannsthal, Arthur Schnitzler: \emph{Briefwechsel}. Hg. Therese Nickl und Heinrich Schnitzler. Frankfurt am Main: \emph{S. Fischer} 1964, S. 100–101.} }\toendnotes[C]{\smallbreak}\pstart{}{\pb}lieber Arthur\pend\pstart
           möchten Sie am \label{K_L00792_1v}\edtext{Donnerstag}{\lemma{\textnormal{\emph{Donnerstag}}}\Cendnote{\textnormal{Die angesprochene Radpartie fand am
                            21. 4. 1898 – dem besagten Donnerstag – unter Teilnahme
                            Schnitzler\pwindex{\textcolor{red}{\textsuperscript{XXXX1 indx}}|pwk}s statt.}}}\label{K_L00792_1h} eine
                        Rad-Tages-partie{ }\strikeout{nach} machen nämlich mit mir, Mutter\pwindex{\textcolor{red}{\textsuperscript{XXXX1 indx}}|pwv} und Tochter\pwindex{\textcolor{red}{\textsuperscript{XXXX1 indx}}|pwv}{ }Schleſinger\pwindex{\textcolor{red}{\textsuperscript{XXXX1 indx}}|pw}\pwindex{\textcolor{red}{\textsuperscript{XXXX1 indx}}|pw} und den beiden Franckenſteins\pwindex{\textcolor{red}{\textsuperscript{XXXX1 indx}}|pw}\pwindex{\textcolor{red}{\textsuperscript{XXXX1 indx}}|pw}. Natürlich eine \uline{kleine} Partie {\pb}z. B. \textsc{Pressbaum}\oindex{XXXX Ortsangabe fehlt|pw}–Baden\oindex{XXXX Ortsangabe fehlt|pw}.\pend
           \pstart
           Den Weg müſſten Sie wiſſen, wir wiſſen alle nichts aber man hat ja Karten. Bitte
                    antworten Sie mir umgehend aber ſehr ungeniert natürlich, wenn Sie keine Luſt
                    haben braucht es ja keinen anderen Grund. – Ich danke vielmals {\pb}für Ihr Geſpräch mit
                        Schlenther\pwindex{\textcolor{red}{\textsuperscript{XXXX1 indx}}|pw}. Ich wär natürlich rieſig
                    froh, wenn etwas daraus würde, beſonders in \uline{der}
                    Beſetzung.\pend
           \pstart
           Geſtern abend war ich mit Richard\pwindex{\textcolor{red}{\textsuperscript{XXXX1 indx}}|pw} 1 Stunde im
                        \textsc{Europe}\oindex{XXXX Ortsangabe fehlt|pw}.\pend
           \pstart
           Morgen nach 11\textsuperscript{h} werd ich ins Kaiſerhof\oindex{XXXX Ortsangabe fehlt|pw}{ }ſchauen,
                        \uline{ohne} gegenſeitige Bindung. Adieu.\pend
           \pstart \spacefill\mbox{Hugo.}\pend{}
         
         \endnumbering\mylabel{h}\end{ledgroupsized}  \newcommand{\dateiname}{L00792}\newcommand{\titel}{Hugo von Hofmannsthal an Arthur Schnitzler, [19. 4. 1898]}\newcommand{\editorInnen}{Martin Anton Müller und Gerd-Hermann Susen}%% latex-leseansicht-abspann.tex
%% Abspann für die Leseansicht.
%% Der Schalter \ifkorrekturansicht ist bereits durch den Vorspann gesetzt.

%% latex-abspann.tex
%% Gemeinsamer Abspann für Korrekturansicht und Leseansicht.
%% Setzt den Schalter \ifkorrekturansicht voraus (gesetzt in den
%% einbindenden Dateien latex-korrekturansicht-abspann.tex bzw.
%% latex-leseansicht-abspann.tex).
%% ---------------------------------------------------------------

\normalsize

% Das esempio-Environment wird nur in der Leseansicht benötigt
\ifkorrekturansicht\else
\newenvironment{esempio}[3]%
{
    \vspace{1.5ex}
    \rlap{\underline{#1}}
    \par
    \setlength{\parindent}{0cm}
    \nopagebreak
    \leftskip=#2cm
    \rightskip=#3cm
}
{
    \par
}
\fi

\doendnotes{C}
\bigskip
\vfill

\clearpage

\footnotesize

\ifkorrekturansicht
  \lohead{\textsc{register}}
\fi

% theindex-Environment neu definieren ohne reledmac
\makeatletter
\renewenvironment{theindex}{%
  \ifkorrekturansicht
    \section*{\indexname}%
  \else
    \subsubsection*{Index der erwähnten Entitäten}%
  \fi
  \setlength{\parindent}{0pt}%
  \setlength{\parskip}{0pt plus 0.3pt}%
  \let\item\@idxitem
}{%
  \ifkorrekturansicht\clearpage\fi
}
\makeatother

\IfFileExists{\jobname-pw.ind}{\input{\jobname-pw.ind}}{}

% Quellenangabe nur in der Leseansicht
\ifkorrekturansicht\else
% Fallback-Definitionen, falls die .tex-Datei \titel etc. nicht gesetzt hat
\providecommand{\titel}{}
\providecommand{\editorInnen}{}
\providecommand{\dateiname}{\jobname}

\vspace{3cm}

\vfill

\footnotesize
\textsc{Quelle}: \titel. Herausgegeben von {\editorInnen}. In: \emph{Arthur Schnitzler: Briefwechsel mit Autorinnen und Autoren}.
 Digitale Edition, https://schnitzler-briefe.acdh.oeaw.ac.at/{\dateiname}.html (Stand \today)
\fi

\end{document}


      