%% latex-korrekturansicht-vorspann.tex
%% Vorspann für die Korrekturansicht.
%% Lädt die gemeinsame Datei latex-vorspann.tex mit gesetztem Schalter.

\newif\ifkorrekturansicht
\korrekturansichttrue

\input{../tex-inputs/latex-vorspann}


\section[Hugo von Hofmannsthal an Arthur Schnitzler, {[}19. 4. 1898{]}]{L00792 Hugo von Hofmannsthal an Arthur Schnitzler, {[}19. 4. 1898{]}}
\nopagebreak\mylabel{L00792v}
\rehead{ }\normalsize\beginnumbering\briefempfaengerindex{Schnitzler, Arthur@\textsc{Schnitzler, Arthur}!zzzHofmannsthal, Hugo von@\emph{von Hugo von Hofmannsthal}!1898-04-192@{{[}19. 4. 1898{]}}|(be}
\toendnotes[C]{\smallbreak\pagebreak[2]}\Standort{CUL, Schnitzler, B 43b/1.}
\physDesc{Brief, 1 Blatt, 3 Seiten, 655 Zeichen (Briefkopf mit Möwen und einem Segelschiff)
\newline{}Handschrift: schwarze Tinte, deutsche Kurrent
\newline{}Schnitzler: mit Bleistift datiert: »19/4/98« 
\newline{}Ordnung: 1) mit Bleistift von unbekannter Hand nummeriert: »\strikeout{113}«  2) mit Bleistift von unbekannter Hand nummeriert:
                                    »111«}
\buchAbdrucke{\weitereDrucke{Hugo von Hofmannsthal, Arthur Schnitzler: \emph{Briefwechsel}. Frankfurt am Main: \emph{S. Fischer} 1964, S. 100–101.} }\toendnotes[C]{\smallbreak}
\pstart{}{\pb}lieber Arthur\pend\vspace{0.5em}
\pstart
           möchten Sie am \label{K_L00792-1v}\edtext{Donnerstag}{\lemma{\textnormal{\emph{Donnerstag}}}\Cendnote{\textnormal{Die angesprochene Radpartie fand am
                     21. 4. 1898 – dem besagten Donnerstag – unter Teilnahme Schnitzlers statt.}}}\label{K_L00792-1} eine
                  Rad-Tages-partie{ }\strikeout{nach} machen nämlich mit mir, Mutter\pwindex{Schlesinger, Franziska 17.08.1851 – 11.08.1932@\textsc{Schlesinger, Franziska} (17.08.1851 – 11.08.1932)|pwv} und Tochter\pwindex{Hofmannsthal, Gertrude von 16.03.1880 – 09.11.1959@\textsc{Hofmannsthal, Gertrude von} (16.03.1880 – 09.11.1959)|pwv}{ }Schleſinger\pwindex{Schlesinger, Franziska 17.08.1851 – 11.08.1932@\textsc{Schlesinger, Franziska} (17.08.1851 – 11.08.1932)|pw}\pwindex{Hofmannsthal, Gertrude von 16.03.1880 – 09.11.1959@\textsc{Hofmannsthal, Gertrude von} (16.03.1880 – 09.11.1959)|pw} und den beiden Franckenſteins\pwindex{Franckenstein, Clemens von 14.07.1875 – 19.08.1942@\textsc{Franckenstein, Clemens von} (14.07.1875 – 19.08.1942), \emph{Theaterleiter/Theaterleiterin, Komponist/Komponistin, Dirigent/Dirigentin}|pw}\pwindex{Franckenstein, Georg von 18.03.1878 – 14.10.1953@\textsc{Franckenstein, Georg von} (18.03.1878 – 14.10.1953), \emph{Diplomat/Diplomatin}|pw}. Natürlich eine \uline{kleine} Partie {\pb}z. B. \textsc{Pressbaum}\oindex{Pressbaum@\textbf{Pressbaum}, \emph{P.PPLA3}|pw}–Baden\oindex{Baden bei Wien@\textbf{Baden bei Wien}, \emph{P.PPLA3}|pw}.\pend
           
\pstart
           Den Weg müſſten Sie wiſſen, wir wiſſen alle nichts aber man hat ja Karten. Bitte
               antworten Sie mir umgehend aber ſehr ungeniert natürlich, wenn Sie keine Luſt haben
               braucht es ja keinen anderen Grund. – Ich danke vielmals {\pb}für Ihr Geſpräch mit Schlenther\pwindex{Schlenther, Paul 20.08.1854 – 30.04.1916@\textsc{Schlenther, Paul} (20.08.1854 – 30.04.1916), \emph{Schriftsteller/Schriftstellerin, Kritiker/Kritikerin, Theaterleiter/Theaterleiterin}|pw}. Ich wär natürlich rieſig froh, wenn
               etwas daraus würde, beſonders in \uline{der} Beſetzung.\pend
           
\pstart
           Geſtern abend war ich mit Richard\pwindex{Beer-Hofmann, Richard 1866-07-11 – 1945-09-26@\textsc{Beer-Hofmann, Richard} (1866-07-11 – 1945-09-26), \emph{Schriftsteller/Schriftstellerin}|pw} 1 Stunde im
                  \textsc{Europe}\oindex{Cafe de l Europe@\textbf{Café de l’Europe}, \emph{Kaffeehaus (K.KAF)}|pw}.\pend
           
\pstart
           Morgen nach 11\textsuperscript{h} werd ich ins Kaiſerhof\oindex{Cafe Kaiserhof (Inh. Johann Wortner) [Wien]@\textbf{Café Kaiserhof (Inh. Johann Wortner) [Wien]}, \emph{Kaffeehaus (K.KAF)}|pw}{ }ſchauen, \uline{ohne}
               gegenſeitige Bindung. Adieu.\pend
           \pstart \spacefill\mbox{Hugo.}\pend{}\selectlanguage{ngerman}\endnumbering\briefempfaengerindex{Schnitzler, Arthur@\textsc{Schnitzler, Arthur}!zzzHofmannsthal, Hugo von@\emph{von Hugo von Hofmannsthal}!1898-04-192@{{[}19. 4. 1898{]}}|)be}\mylabel{L00792h}  \normalsize

\doendnotes{C}
\bigskip
\vfill

\clearpage

\footnotesize

\lohead{\textsc{register}}

% Definiere theindex-Environment komplett neu ohne reledmac
\makeatletter
\renewenvironment{theindex}{%
  \section*{\indexname}%
  \setlength{\parindent}{0pt}%
  \setlength{\parskip}{0pt plus 0.3pt}%
  \let\item\@idxitem
}{%
  \clearpage
}
\makeatother

\IfFileExists{\jobname-pw.ind}{\input{\jobname-pw.ind}}{}

\end{document}

      