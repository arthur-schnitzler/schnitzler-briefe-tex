%% latex-leseansicht-vorspann.tex
%% Vorspann für die Leseansicht.
%% Lädt die gemeinsame Datei latex-vorspann.tex mit nicht gesetztem Schalter.

\newif\ifkorrekturansicht
\korrekturansichtfalse

\input{../tex-inputs/latex-vorspann}


\section[Paul Goldmann an Arthur Schnitzler, 21. 3. [1894]]{L02613 Paul Goldmann an Arthur Schnitzler, 21. 3. [1894]}
\nopagebreak\mylabel{L02613v}
\rehead{ }\normalsize\beginnumbering\briefempfaengerindex{Schnitzler, Arthur@\textsc{Schnitzler, Arthur}!zzzGoldmann, Paul@\emph{von Paul Goldmann}!1894-03-213@{21. 3. [1894]}|(be}
\toendnotes[C]{\smallbreak\pagebreak[2]}
\correspDesc{Versand  durch Paul Goldmann am 21. 3. [1894] in Paris
\newline{}Erhalt  durch Arthur Schnitzler im Zeitraum [22. 3. 1894
                  – 26. 3. 1894?] in Wien}\toendnotes[C]{\smallbreak}
\Standort{DLA, A:Schnitzler, HS.NZ85.1.3164.}
\physDesc{Brief, 1 Blatt, 4 Seiten, 1686 Zeichen
\newline{}Handschrift: schwarze Tinte, deutsche Kurrent
\newline{}Schnitzler: 1) mit Bleistift auf dem ersten Blatt die Jahreszahl »94« vermerkt  2) mit rotem Buntstift eine Unterstreichung}\toendnotes[C]{\smallbreak}
\pstart
           \raggedleft{}{\pb}\textsc{Paris\oindex{Paris@\textbf{Paris}, \emph{Hauptstadt}|pw}}, 21. März.\pend
           
\pstart\center{}Mein lieber Freund,\pend\vspace{0.5em}
\pstart
           Es iſt wirklich wahr: Seit dem Empfang Deines lieben Briefes iſt kein Tag vergangen,
               wo ich Dir nicht{ }ſchreiben wollte. Heut habe ich
               endlich einmal ein wenig Zeit.\pend
           
\pstart
           Die Überſetzung Deiner \label{K_L02613-1v}\edtext{Artikel ins
                  Franzöſiſche}{\lemma{\textnormal{\emph{Artikel ins
                  Französische}}}\Cendnote{\textnormal{Es kam in Folge nur zur
                     Übersetzung\pwindex{Schnitzler, Arthur 15.\,5.\,1862 Wien – 21.\,10.\,1931 ebd.@\textsc{Schnitzler, Arthur} (15.\,5.\,1862 Wien – 21.\,10.\,1931 ebd.), \emph{Schriftsteller, Mediziner}!Emplettes de Noël@\strich\emph{Les Emplettes de Noël}|pwkv} des
                  Einakters \emph{Weihnachts-Einkäufe}\pwindex{Schnitzler, Arthur 15.\,5.\,1862 Wien – 21.\,10.\,1931 ebd.@\textsc{Schnitzler, Arthur} (15.\,5.\,1862 Wien – 21.\,10.\,1931 ebd.), \emph{Schriftsteller, Mediziner}!Weihnachts-Einkäufe@\strich\emph{Weihnachts-Einkäufe}|pwk}. Im Brief Alberts\pwindex{Albert, Henri 16.\,11.\,1869 Niederbronn-les-Bains – 3.\,8.\,1921 Straßburg@\textsc{Albert, Henri} (16.\,11.\,1869 Niederbronn-les-Bains – 3.\,8.\,1921 Straßburg), \emph{Journalist, Kritiker, Übersetzer}|pwk} an Schnitzler vom 9. 4. 1894 schrieb er, dass
                  er bereits an der Übersetzung\pwindex{Schnitzler, Arthur 15.\,5.\,1862 Wien – 21.\,10.\,1931 ebd.@\textsc{Schnitzler, Arthur} (15.\,5.\,1862 Wien – 21.\,10.\,1931 ebd.), \emph{Schriftsteller, Mediziner}!Emplettes de Noël@\strich\emph{Les Emplettes de Noël}|pwkv} sitze (\emph{DLA}, HS.1985.1.2331,1). Arthur Schnitzler: \emph{Les Emplettes de Noël}\pwindex{Schnitzler, Arthur 15.\,5.\,1862 Wien – 21.\,10.\,1931 ebd.@\textsc{Schnitzler, Arthur} (15.\,5.\,1862 Wien – 21.\,10.\,1931 ebd.), \emph{Schriftsteller, Mediziner}!Emplettes de Noël@\strich\emph{Les Emplettes de Noël}|pwk}. Traduit de l’allemand par Henri Albert\pwindex{Albert, Henri 16.\,11.\,1869 Niederbronn-les-Bains – 3.\,8.\,1921 Straßburg@\textsc{Albert, Henri} (16.\,11.\,1869 Niederbronn-les-Bains – 3.\,8.\,1921 Straßburg), \emph{Journalist, Kritiker, Übersetzer}|pwk}. In: \emph{L’Idée libre. Revue mensuelle de Littérature et d’Art}\pwindex{Idée libre. Revue mensuelle de Littérature et d'Art@\emph{L’Idée libre. Revue mensuelle de Littérature et d'Art}|pwk},
                     Jg. 3, Nr. 5–6, Mai–Juni 1894, S. 215–225. Schnitzler beurteilte die Qualität der Übersetzung\pwindex{Schnitzler, Arthur 15.\,5.\,1862 Wien – 21.\,10.\,1931 ebd.@\textsc{Schnitzler, Arthur} (15.\,5.\,1862 Wien – 21.\,10.\,1931 ebd.), \emph{Schriftsteller, Mediziner}!Emplettes de Noël@\strich\emph{Les Emplettes de Noël}|pwkv} negativ, vgl. A. S.: \emph{Tagebuch}, 21. 7. 1894.}}}\label{K_L02613-1} habe ich{ }ſofort nach meiner Bekanntwerdung mit \textsc{Albert\pwindex{Albert, Henri 16.\,11.\,1869 Niederbronn-les-Bains – 3.\,8.\,1921 Straßburg@\textsc{Albert, Henri} (16.\,11.\,1869 Niederbronn-les-Bains – 3.\,8.\,1921 Straßburg), \emph{Journalist, Kritiker, Übersetzer}|pw}} beſprochen. Er iſt gleich bereit, wird gewiß auch etwas in einer der Jungen
               Revüen anbringen können. Aber ein {\pb}Haken iſt da: die
               Revüen zahlen nicht, \textsc{Albert\pwindex{Albert, Henri 16.\,11.\,1869 Niederbronn-les-Bains – 3.\,8.\,1921 Straßburg@\textsc{Albert, Henri} (16.\,11.\,1869 Niederbronn-les-Bains – 3.\,8.\,1921 Straßburg), \emph{Journalist, Kritiker, Übersetzer}|pw}} muß von{ }ſeiner Feder leben. Du kannſt \strikeout{ihm} daher
               die Frage am Beſten löſen, indem Du ihm ein Honorar anbieteſt. Natürlich macht er{ }ſehr geringe Anſprüche. \label{K_L02613-2v}\edtext{Schicke ihm
               alſo Deine Schriften, mache ihm unumwunden den Honorar-Vorſchlag}{\lemma{\textnormal{\emph{Schicke … Honorar-Vorschlag}}}\Cendnote{\textnormal{Aus dem Brief, den Albert\pwindex{Albert, Henri 16.\,11.\,1869 Niederbronn-les-Bains – 3.\,8.\,1921 Straßburg@\textsc{Albert, Henri} (16.\,11.\,1869 Niederbronn-les-Bains – 3.\,8.\,1921 Straßburg), \emph{Journalist, Kritiker, Übersetzer}|pwk} am 23. 5. 1894 an Schnitzler schrieb, geht hervor, dass eine nicht näher
                  bezeichnete Summe bezahlt wurde (\emph{DLA}, HS.1985.1.2331,2). Er bedankt sich zudem für
                  die Zusendung des \emph{Märchens}\pwindex{Schnitzler, Arthur 15.\,5.\,1862 Wien – 21.\,10.\,1931 ebd.@\textsc{Schnitzler, Arthur} (15.\,5.\,1862 Wien – 21.\,10.\,1931 ebd.), \emph{Schriftsteller, Mediziner}!Märchen. Schauspiel in drei Aufzügen@\strich\emph{Das Märchen. Schauspiel in drei Aufzügen}|pwk}.}}}\label{K_L02613-2}, indem Du
               Dich auf meinen Brief beziehſt, und überlaß mir das übrige. Die Fixirung der Summe
               mache ich dann{ }ſchon aus, um zwiſchen {\pb}Euch Beiden
               keine \label{K_L02613-3v}\edtext{\textsc{\begin{otherlanguage}{french}Gêne\end{otherlanguage}}}{\lemma{\textnormal{\emph{Gêne}}}\Cendnote{\textnormal{französisch: Befangenheit, Verlegenheit.
                  »Être dans la gêne« bedeutet »in Geldverlegenheit sein«.}}}\label{K_L02613-3} aufkommen zu
               laſſen. Schreibe ihm{ }ſofort. Denn er hat gerade jetzt etwas Zeit, die er mit einer
               Überſetzung ausfüllen könnte.\pend
           
\pstart
           Sonſt erfahre ich aus Deinem Briefe mit Freuden, daß du rüſtig weiter{ }ſchaffſt. Mehr
               brauche ich nicht zu wiſſen. Über den Erfolg bin ich beruhigt. Aber ich habe{ }ſchon
               gar{ }ſo lange nichts von Dir geleſen. Könnteſt Du mir nicht einmal eine Kleinigkeit{ }ſchicken? Ich gebe{ }ſie eventuell wieder zurück.\pend
           
\pstart
           {\pb}Vielen Dank für die intereſſanten poſitiven
               Mittheilungen. \textsc{Hermann Bahr\pwindex{Bahr, Hermann 19.\,7.\,1863 Linz – 15.\,1.\,1934 München@\textsc{Bahr, Hermann} (19.\,7.\,1863 Linz – 15.\,1.\,1934 München), \emph{Schriftsteller, Kritiker}|pw}}{ }\label{K_L02613-4v}\edtext{gründet ein Blatt\orgindex{Zeit. Wiener Wochenschrift@Die Zeit. Wiener Wochenschrift|pwv}}{\lemma{\textnormal{\emph{gründet ein Blatt}}}\Cendnote{\textnormal{Es handelt sich um die seit
                     Frühjahr 1894 laufende Entwicklung der »Wiener Wochenschrift« \emph{Die Zeit}\orgindex{Zeit. Wiener Wochenschrift@Die Zeit. Wiener Wochenschrift|pwk}, die ab Oktober des Jahres erschien. Als Herausgeber fungierte Hermann Bahr\pwindex{Bahr, Hermann 19.\,7.\,1863 Linz – 15.\,1.\,1934 München@\textsc{Bahr, Hermann} (19.\,7.\,1863 Linz – 15.\,1.\,1934 München), \emph{Schriftsteller, Kritiker}|pwk} gemeinsam mit Heinrich Kanner\pwindex{Kanner, Heinrich 9.\,11.\,1864 Galați – 15.\,2.\,1930 Wien@\textsc{Kanner, Heinrich} (9.\,11.\,1864 Galați – 15.\,2.\,1930 Wien), \emph{Herausgeber, Publizist}|pwk} und Isidor Singer\pwindex{Singer, Isidor 16.\,1.\,1857 Budapest – 8.\,12.\,1927 Wien@\textsc{Singer, Isidor} (16.\,1.\,1857 Budapest – 8.\,12.\,1927 Wien), \emph{Journalist, Herausgeber, Soziologe}|pwk}. Bahr\pwindex{Bahr, Hermann 19.\,7.\,1863 Linz – 15.\,1.\,1934 München@\textsc{Bahr, Hermann} (19.\,7.\,1863 Linz – 15.\,1.\,1934 München), \emph{Schriftsteller, Kritiker}|pwk} verantwortete
                  den Kulturteil.}}}\label{K_L02613-4}? Der Burſch\pwindex{Bahr, Hermann 19.\,7.\,1863 Linz – 15.\,1.\,1934 München@\textsc{Bahr, Hermann} (19.\,7.\,1863 Linz – 15.\,1.\,1934 München), \emph{Schriftsteller, Kritiker}|pwv} weiß wirklich aus Steinen Brot zu machen. Iſt das aber auch{ }ſeriös?\pend
           
\pstart
           Von mir? Hoffnungslosigkeit und Verzweiflung.\pend
           
\pstart
           Grüß die Freunde vielmals und vergiß nicht, daß wir Zwei uns im Sommer treffen
               wollen. Sei von Herzen gegrüßt und bedankt für Deine Treue (Du biſt der \uline{Einzige}, der meine Artikel lobt!). Schreibe recht
               bald.\pend
           
\pstart
           In Treue {\\[\baselineskip]}Dein \spacefill\mbox{Paul Goldm}\pend
           \leftskip=0em{}\selectlanguage{ngerman}\endnumbering\briefempfaengerindex{Schnitzler, Arthur@\textsc{Schnitzler, Arthur}!zzzGoldmann, Paul@\emph{von Paul Goldmann}!1894-03-213@{21. 3. [1894]}|)be}\mylabel{L02613h}  \newcommand{\dateiname}{L02613}\newcommand{\titel}{Paul Goldmann an Arthur Schnitzler, 21. 3. [1894]}\newcommand{\editorInnen}{Martin Anton Müller und Laura Untner}%% latex-leseansicht-abspann.tex
%% Abspann für die Leseansicht.
%% Der Schalter \ifkorrekturansicht ist bereits durch den Vorspann gesetzt.

%% latex-abspann.tex
%% Gemeinsamer Abspann für Korrekturansicht und Leseansicht.
%% Setzt den Schalter \ifkorrekturansicht voraus (gesetzt in den
%% einbindenden Dateien latex-korrekturansicht-abspann.tex bzw.
%% latex-leseansicht-abspann.tex).
%% ---------------------------------------------------------------

\normalsize

% Das esempio-Environment wird nur in der Leseansicht benötigt
\ifkorrekturansicht\else
\newenvironment{esempio}[3]%
{
    \vspace{1.5ex}
    \rlap{\underline{#1}}
    \par
    \setlength{\parindent}{0cm}
    \nopagebreak
    \leftskip=#2cm
    \rightskip=#3cm
}
{
    \par
}
\fi

\doendnotes{C}
\bigskip
\vfill

\clearpage

\footnotesize

\ifkorrekturansicht
  \lohead{\textsc{register}}
\fi

% theindex-Environment neu definieren ohne reledmac
\makeatletter
\renewenvironment{theindex}{%
  \ifkorrekturansicht
    \section*{\indexname}%
  \else
    \subsubsection*{Index der erwähnten Entitäten}%
  \fi
  \setlength{\parindent}{0pt}%
  \setlength{\parskip}{0pt plus 0.3pt}%
  \let\item\@idxitem
}{%
  \ifkorrekturansicht\clearpage\fi
}
\makeatother

\IfFileExists{\jobname-pw.ind}{\input{\jobname-pw.ind}}{}

% Quellenangabe nur in der Leseansicht
\ifkorrekturansicht\else
% Fallback-Definitionen, falls die .tex-Datei \titel etc. nicht gesetzt hat
\providecommand{\titel}{}
\providecommand{\editorInnen}{}
\providecommand{\dateiname}{\jobname}

\vspace{3cm}

\vfill

\footnotesize
\textsc{Quelle}: \titel. Herausgegeben von {\editorInnen}. In: \emph{Arthur Schnitzler: Briefwechsel mit Autorinnen und Autoren}.
 Digitale Edition, https://schnitzler-briefe.acdh.oeaw.ac.at/{\dateiname}.html (Stand \today)
\fi

\end{document}


