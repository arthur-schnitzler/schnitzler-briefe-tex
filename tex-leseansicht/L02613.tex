%% latex-korrekturansicht-vorspann.tex
%% Vorspann für die Korrekturansicht.
%% Lädt die gemeinsame Datei latex-vorspann.tex mit gesetztem Schalter.

\newif\ifkorrekturansicht
\korrekturansichttrue

\input{../tex-inputs/latex-vorspann}


\section[Paul Goldmann an Arthur Schnitzler, 21. 3. {[}1894{]}]{L02613 Paul Goldmann an Arthur Schnitzler, 21. 3. {[}1894{]}}
\nopagebreak\mylabel{L02613v}
\rehead{ }\normalsize\beginnumbering\briefempfaengerindex{Schnitzler, Arthur@\textsc{Schnitzler, Arthur}!zzzGoldmann, Paul@\emph{von Paul Goldmann}!1894-03-213@{21. 3. {[}1894{]}}|(be}
\toendnotes[C]{\smallbreak\pagebreak[2]}\Standort{DLA, A:Schnitzler, HS.NZ85.1.3164.}
\physDesc{Brief, 1 Blatt, 4 Seiten, 1686 Zeichen
\newline{}Handschrift: schwarze Tinte, deutsche Kurrent
\newline{}Schnitzler: 1) mit Bleistift auf dem ersten Blatt die Jahreszahl »94« vermerkt  2) mit rotem Buntstift eine Unterstreichung}\toendnotes[C]{\smallbreak}
\pstart
           \raggedleft{}{\pb}\textsc{Paris\oindex{Paris@\textbf{Paris}, \emph{P.PPLC}|pw}}, 21. März.\pend
           
\pstart\center{}Mein lieber Freund,\pend\vspace{0.5em}
\pstart
           Es iſt wirklich wahr: Seit dem Empfang Deines lieben Briefes iſt kein Tag vergangen,
               wo ich Dir nicht ſchreiben wollte. Heut habe ich
               endlich einmal ein wenig Zeit.\pend
           
\pstart
           Die Überſetzung Deiner \label{K_L02613-1v}\edtext{Artikel ins
                  Franzöſiſche}{\lemma{\textnormal{\emph{Artikel ins
                  Franzöſiſche}}}\Cendnote{\textnormal{Es kam in Folge nur zur
                     Übersetzung\pwindex{Emplettes de Noel@\emph{Les Emplettes de Noël}|pwkv} des
                  Einakters \emph{Weihnachts-Einkäufe}\pwindex{Weihnachts-Einkaeufe@\emph{Weihnachts-Einkäufe}|pwk}. Im Brief Alberts\pwindex{Albert, Henri 1869-11-16 – 1921-08-03@\textsc{Albert, Henri} (1869-11-16 – 1921-08-03), \emph{Journalist/Journalistin, Kritiker/Kritikerin, Übersetzer/Übersetzerin}|pwk} an Schnitzler vom 9. 4. 1894 schrieb er, dass
                  er bereits an der Übersetzung\pwindex{Emplettes de Noel@\emph{Les Emplettes de Noël}|pwkv} sitze (\emph{DLA}, HS.1985.1.2331,1). Arthur Schnitzler: \emph{Les Emplettes de Noël}\pwindex{Emplettes de Noel@\emph{Les Emplettes de Noël}|pwk}. Traduit de l’allemand par Henri Albert\pwindex{Albert, Henri 1869-11-16 – 1921-08-03@\textsc{Albert, Henri} (1869-11-16 – 1921-08-03), \emph{Journalist/Journalistin, Kritiker/Kritikerin, Übersetzer/Übersetzerin}|pwk}. In: \emph{L’Idée libre. Revue mensuelle de Littérature et d’Art}\pwindex{L'Idee libre. Revue mensuelle de Litterature et d'Art@\emph{L'Idée libre. Revue mensuelle de Littérature et d'Art}|pwk},
                     Jg. 3, Nr. 5–6, Mai–Juni 1894, S. 215–225. Schnitzler beurteilte die Qualität der Übersetzung\pwindex{Emplettes de Noel@\emph{Les Emplettes de Noël}|pwkv} negativ, vgl. A. S.: \emph{Tagebuch}, 21. 7. 1894.}}}\label{K_L02613-1} habe ich
               ſofort nach meiner Bekanntwerdung mit \textsc{Albert\pwindex{Albert, Henri 1869-11-16 – 1921-08-03@\textsc{Albert, Henri} (1869-11-16 – 1921-08-03), \emph{Journalist/Journalistin, Kritiker/Kritikerin, Übersetzer/Übersetzerin}|pw}} beſprochen. Er iſt gleich bereit, wird gewiß auch etwas in einer der Jungen
               Revüen anbringen können. Aber ein {\pb}Haken iſt da: die
               Revüen zahlen nicht, \textsc{Albert\pwindex{Albert, Henri 1869-11-16 – 1921-08-03@\textsc{Albert, Henri} (1869-11-16 – 1921-08-03), \emph{Journalist/Journalistin, Kritiker/Kritikerin, Übersetzer/Übersetzerin}|pw}} muß von ſeiner Feder leben. Du kannſt \strikeout{ihm} daher
               die Frage am Beſten löſen, indem Du ihm ein Honorar anbieteſt. Natürlich macht er
               ſehr geringe Anſprüche. \label{K_L02613-2v}\edtext{Schicke ihm
               alſo Deine Schriften, mache ihm unumwunden den Honorar-Vorſchlag}{\lemma{\textnormal{\emph{Schicke … Honorar-Vorſchlag}}}\Cendnote{\textnormal{Aus dem Brief, den Albert\pwindex{Albert, Henri 1869-11-16 – 1921-08-03@\textsc{Albert, Henri} (1869-11-16 – 1921-08-03), \emph{Journalist/Journalistin, Kritiker/Kritikerin, Übersetzer/Übersetzerin}|pwk} am 23. 5. 1894 an Schnitzler schrieb, geht hervor, dass eine nicht näher
                  bezeichnete Summe bezahlt wurde (\emph{DLA}, HS.1985.1.2331,2). Er bedankt sich zudem für
                  die Zusendung des \emph{Märchens}\pwindex{Maerchen. Schauspiel in drei Aufzuegen@\emph{Das Märchen. Schauspiel in drei Aufzügen}|pwk}.}}}\label{K_L02613-2}, indem Du
               Dich auf meinen Brief beziehſt, und überlaß mir das übrige. Die Fixirung der Summe
               mache ich dann ſchon aus, um zwiſchen {\pb}Euch Beiden
               keine \label{K_L02613-3v}\edtext{\textsc{\begin{otherlanguage}{french}Gêne\end{otherlanguage}}}{\lemma{\textnormal{\emph{Gêne}}}\Cendnote{\textnormal{französisch: Befangenheit, Verlegenheit.
                  »Être dans la gêne« bedeutet »in Geldverlegenheit sein«.}}}\label{K_L02613-3} aufkommen zu
               laſſen. Schreibe ihm ſofort. Denn er hat gerade jetzt etwas Zeit, die er mit einer
               Überſetzung ausfüllen könnte.\pend
           
\pstart
           Sonſt erfahre ich aus Deinem Briefe mit Freuden, daß du rüſtig weiter ſchaffſt. Mehr
               brauche ich nicht zu wiſſen. Über den Erfolg bin ich beruhigt. Aber ich habe ſchon
               gar ſo lange nichts von Dir geleſen. Könnteſt Du mir nicht einmal eine Kleinigkeit
               ſchicken? Ich gebe ſie eventuell wieder zurück.\pend
           
\pstart
           {\pb}Vielen Dank für die intereſſanten poſitiven
               Mittheilungen. \textsc{Hermann Bahr\pwindex{Bahr, Hermann 19.07.1863 – 15.01.1934@\textsc{Bahr, Hermann} (19.07.1863 – 15.01.1934), \emph{Schriftsteller/Schriftstellerin, Kritiker/Kritikerin}|pw}}{ }\label{K_L02613-4v}\edtext{gründet ein Blatt\orgindex{Zeit. Wiener Wochenschrift@Die Zeit. Wiener Wochenschrift|pwv}}{\lemma{\textnormal{\emph{gründet ein Blatt}}}\Cendnote{\textnormal{Es handelt sich um die seit
                     Frühjahr 1894 laufende Entwicklung der »Wiener Wochenschrift« \emph{Die Zeit}\orgindex{Zeit. Wiener Wochenschrift@Die Zeit. Wiener Wochenschrift|pwk}, die ab Oktober des
                     Jahres erschien. Als Herausgeber fungierte Hermann Bahr\pwindex{Bahr, Hermann 19.07.1863 – 15.01.1934@\textsc{Bahr, Hermann} (19.07.1863 – 15.01.1934), \emph{Schriftsteller/Schriftstellerin, Kritiker/Kritikerin}|pwk} gemeinsam mit Heinrich Kanner\pwindex{Kanner, Heinrich 09.11.1864 – 15.02.1930@\textsc{Kanner, Heinrich} (09.11.1864 – 15.02.1930), \emph{Herausgeber/Herausgeberin, Publizist/Publizistin}|pwk} und Isidor Singer\pwindex{Singer, Isidor 16.01.1857 – 08.12.1927@\textsc{Singer, Isidor} (16.01.1857 – 08.12.1927), \emph{Journalist/Journalistin, Herausgeber/Herausgeberin, Soziologe/Soziologin}|pwk}. Bahr\pwindex{Bahr, Hermann 19.07.1863 – 15.01.1934@\textsc{Bahr, Hermann} (19.07.1863 – 15.01.1934), \emph{Schriftsteller/Schriftstellerin, Kritiker/Kritikerin}|pwk} verantwortete
                  den Kulturteil.}}}\label{K_L02613-4}? Der Burſch\pwindex{Bahr, Hermann 19.07.1863 – 15.01.1934@\textsc{Bahr, Hermann} (19.07.1863 – 15.01.1934), \emph{Schriftsteller/Schriftstellerin, Kritiker/Kritikerin}|pwv} weiß wirklich aus Steinen Brot zu machen. Iſt das aber auch
               ſeriös?\pend
           
\pstart
           Von mir? Hoffnungslosigkeit und Verzweiflung.\pend
           
\pstart
           Grüß die Freunde vielmals und vergiß nicht, daß wir Zwei uns im Sommer treffen
               wollen. Sei von Herzen gegrüßt und bedankt für Deine Treue (Du biſt der \uline{Einzige}, der meine Artikel lobt!). Schreibe recht
               bald.\pend
           
\pstart
           In Treue {\\[\baselineskip]}Dein \spacefill\mbox{Paul Goldm}\pend
           \leftskip=0em{}\selectlanguage{ngerman}\endnumbering\briefempfaengerindex{Schnitzler, Arthur@\textsc{Schnitzler, Arthur}!zzzGoldmann, Paul@\emph{von Paul Goldmann}!1894-03-213@{21. 3. {[}1894{]}}|)be}\mylabel{L02613h}  \normalsize

\doendnotes{C}
\bigskip
\vfill

\clearpage

\footnotesize

\lohead{\textsc{register}}

% Definiere theindex-Environment komplett neu ohne reledmac
\makeatletter
\renewenvironment{theindex}{%
  \section*{\indexname}%
  \setlength{\parindent}{0pt}%
  \setlength{\parskip}{0pt plus 0.3pt}%
  \let\item\@idxitem
}{%
  \clearpage
}
\makeatother

\IfFileExists{\jobname-pw.ind}{\input{\jobname-pw.ind}}{}

\end{document}

      