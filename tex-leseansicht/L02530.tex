%% latex-leseansicht-vorspann.tex
%% Vorspann für die Leseansicht.
%% Lädt die gemeinsame Datei latex-vorspann.tex mit nicht gesetztem Schalter.

\newif\ifkorrekturansicht
\korrekturansichtfalse

\input{../tex-inputs/latex-vorspann}


         
         \renewcommand{\erwaehntePersonen}{Personen: Robert Adam, Alexander Moissi}
         \renewcommand{\erwaehnteOrte}{Orte: Wien}
         \renewcommand{\erwaehnteWerke}{Werke: Im Spiel der Sommerlüfte. In drei Aufzügen}
               \section[Robert Adam an Arthur Schnitzler, 5. 1. 1930]{ Robert Adam an Arthur Schnitzler, 5. 1. 1930}\nopagebreak\mylabel{v}\rehead{ }\begin{ledgroupsized}[t]{13cm}\normalsize\beginnumbering\briefempfaengerindex{Schnitzler, Arthur@\textsc{Schnitzler, Arthur}!zzzAdam, Robert@\emph{von Robert Adam}!1930-01-051@{5. 1. 1930}|(be} \toendnotes[C]{\smallbreak\pagebreak[2]} \Standort{CUL, Schnitzler, B 1.}
\physDesc{Brief, 1 Blatt, 2 Seiten, 844 Zeichen
\newline{}Handschrift: schwarze Tinte, deutsche Kurrent
\newline{}Schnitzler: 1) mit rotem Buntstift beschriftet: »\textsc{Spiel\pwindex{Schnitzler, Arthur 15.05.1862 – 21.10.1931@\textsc{Schnitzler, Arthur} (15.05.1862 – 21.10.1931), \emph{Schriftsteller, Mediziner}!Im Spiel der Sommerluefte. In drei Aufzuegen1929-12-21@\strich\emph{Im Spiel der Sommerlüfte. In drei Aufzügen} {[}1929-12-21{]}|pw}}«  2) mit rotem Buntstift vereinzelte Unterstreichungen
\newline{}Ordnung: mit Bleistift von unbekannter Hand nummeriert:
                                    »24« }\Standort{Wien, Österreichische Nationalbibliothek, Cod.ser. 52.269, 153 recto, 155 recto.}
\physDesc{handschriftliche Abschrift, 2 Blätter, 2 Seiten
\newline{}Handschrift: schwarze Tinte, Gabelsberger Kurzschrift}\Standort{Wien, Österreichische Nationalbibliothek, Cod.ser. 52.269, 153 recto, 155 recto.}
\physDesc{maschinenschriftliche Abschrift, 2 Blätter, 2 Seiten
\newline{}Schreibmaschine}\toendnotes[C]{\smallbreak}\pstart
           \raggedleft{}{\pb}Wien\oindex{Wien@\textbf{Wien}|pw}, am 5. Januar 1930\pend
           \pstart{}Hochverehrter Herr Doktor!\pend\pstart
           Nehmen Sie vor allem meinen beſten Dank für Ihren Brief, der mich über Verdienſt
               erfreute, und zugleich für die liebenswürdige Anweiſung der Sitze zum »Spiel der Sommerlüfte\pwindex{Schnitzler, Arthur 15.05.1862 – 21.10.1931@\textsc{Schnitzler, Arthur} (15.05.1862 – 21.10.1931), \emph{Schriftsteller, Mediziner}!Im Spiel der Sommerluefte. In drei Aufzuegen1929-12-21@\strich\emph{Im Spiel der Sommerlüfte. In drei Aufzügen} {[}1929-12-21{]}|pw}«. Ich komme jetzt ſo ſelten
               in’s Theater, daß ich nicht weiß, ob ich ein Urteil äußern darf; ich möchte aber doch
               ſagen, daß mir die Aufführung vortrefflich zu ſein ſchien. Selbſt mit dem Darſteller\pwindex{Moissi, Alexander 02.04.1879 – 22.03.1935@\textsc{Moissi, Alexander} (02.04.1879 – 22.03.1935), \emph{Schauspieler}|pwv} des Kaplans\pwindex{Schnitzler, Arthur 15.05.1862 – 21.10.1931@\textsc{Schnitzler, Arthur} (15.05.1862 – 21.10.1931), \emph{Schriftsteller, Mediziner}!Im Spiel der Sommerluefte. In drei Aufzuegen1929-12-21@\strich\emph{Im Spiel der Sommerlüfte. In drei Aufzügen} {[}1929-12-21{]}|pwv}, deſſen Sprache, Stimme
               und Gehaben mir nie recht behagten, konnte ich mich diesmal befreunden, ſodaß ich in
               den allgemeinen Beifall auch inſoweit er den Schauspielern galt mit gutem Gewiſſen
               einſtimmen durfte. Manches Zarte Ihrer Komödie\pwindex{Schnitzler, Arthur 15.05.1862 – 21.10.1931@\textsc{Schnitzler, Arthur} (15.05.1862 – 21.10.1931), \emph{Schriftsteller, Mediziner}!Im Spiel der Sommerluefte. In drei Aufzuegen1929-12-21@\strich\emph{Im Spiel der Sommerlüfte. In drei Aufzügen} {[}1929-12-21{]}|pwv} iſt allerdings vergröbert, aber ich möchte meinen,
               daß dieſes Übel mit jeder Bühnendarſtellung unweigerlich ver{\pb}bunden iſt.\pend
           \pstart
           Mit vielen Grüßen und Empfehlungen Ihr ergebener\pend
           \pstart \spacefill\mbox{D\textsuperscript{r}RAdam}\pend{}
         
         \endnumbering\mylabel{h}\end{ledgroupsized}  \newcommand{\dateiname}{L02530}\newcommand{\titel}{Robert Adam an Arthur Schnitzler, 5. 1. 1930}\newcommand{\editorInnen}{Martin Anton Müller und Gerd-Hermann Susen}%% latex-leseansicht-abspann.tex
%% Abspann für die Leseansicht.
%% Der Schalter \ifkorrekturansicht ist bereits durch den Vorspann gesetzt.

%% latex-abspann.tex
%% Gemeinsamer Abspann für Korrekturansicht und Leseansicht.
%% Setzt den Schalter \ifkorrekturansicht voraus (gesetzt in den
%% einbindenden Dateien latex-korrekturansicht-abspann.tex bzw.
%% latex-leseansicht-abspann.tex).
%% ---------------------------------------------------------------

\normalsize

% Das esempio-Environment wird nur in der Leseansicht benötigt
\ifkorrekturansicht\else
\newenvironment{esempio}[3]%
{
    \vspace{1.5ex}
    \rlap{\underline{#1}}
    \par
    \setlength{\parindent}{0cm}
    \nopagebreak
    \leftskip=#2cm
    \rightskip=#3cm
}
{
    \par
}
\fi

\doendnotes{C}
\bigskip
\vfill

\clearpage

\footnotesize

\ifkorrekturansicht
  \lohead{\textsc{register}}
\fi

% theindex-Environment neu definieren ohne reledmac
\makeatletter
\renewenvironment{theindex}{%
  \ifkorrekturansicht
    \section*{\indexname}%
  \else
    \subsubsection*{Index der erwähnten Entitäten}%
  \fi
  \setlength{\parindent}{0pt}%
  \setlength{\parskip}{0pt plus 0.3pt}%
  \let\item\@idxitem
}{%
  \ifkorrekturansicht\clearpage\fi
}
\makeatother

\IfFileExists{\jobname-pw.ind}{\input{\jobname-pw.ind}}{}

% Quellenangabe nur in der Leseansicht
\ifkorrekturansicht\else
% Fallback-Definitionen, falls die .tex-Datei \titel etc. nicht gesetzt hat
\providecommand{\titel}{}
\providecommand{\editorInnen}{}
\providecommand{\dateiname}{\jobname}

\vspace{3cm}

\vfill

\footnotesize
\textsc{Quelle}: \titel. Herausgegeben von {\editorInnen}. In: \emph{Arthur Schnitzler: Briefwechsel mit Autorinnen und Autoren}.
 Digitale Edition, https://schnitzler-briefe.acdh.oeaw.ac.at/{\dateiname}.html (Stand \today)
\fi

\end{document}


      