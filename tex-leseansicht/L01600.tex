%% latex-leseansicht-vorspann.tex
%% Vorspann für die Leseansicht.
%% Lädt die gemeinsame Datei latex-vorspann.tex mit nicht gesetztem Schalter.

\newif\ifkorrekturansicht
\korrekturansichtfalse

\input{../tex-inputs/latex-vorspann}


\section[Mirjam Beer-Hofmann an Olga Schnitzler, 12. 6. 1906]{L01600 Mirjam Beer-Hofmann an Olga Schnitzler, 12. 6. 1906}
\nopagebreak\mylabel{L01600v}
\rehead{ }\normalsize\beginnumbering\briefempfaengerindex{Schnitzler, Olga@\textsc{Schnitzler, Olga}!zzzBeer-Hofmann, Mirjam@\emph{von Mirjam Beer-Hofmann}!1906-06-121@{12. 6. 1906}|(be}
\toendnotes[C]{\smallbreak\pagebreak[2]}
\correspDesc{Versand  durch Mirjam Beer-Hofmann am 12. 6. 1906 in Dahmsdorf
\newline{}Erhalt  durch Olga Schnitzler im Zeitraum [13. 6. 1906
                  – 17. 6. 1906?] in Wien}\toendnotes[C]{\smallbreak}
\Standort{DLA, A:Schnitzler, HS.NZ.85.1.5204.}
\physDesc{Bildpostkarte, 123 Zeichen
\newline{}Handschrift: schwarze Tinte, lateinische Kurrent
\newline{}Versand: Stempel: »\nobreak{}\oindex{Dahmsdorf@\textbf{Dahmsdorf}|pwk}Dahmsdorf Müncheberg, 12 6 06, 8–12N\nobreak{}«.  }\toendnotes[C]{\smallbreak}\pstart{}{\pb}Frau\pend{}\pstart{}Olga Schnitzler\pend{}\pstart{}Wien XVIII\oindex{XVIII., Währing@\textbf{XVIII., Währing}, \emph{Verwaltungsgebiet}|pw}.\pend{}\pstart{}Spöttelgasse 7\oindex{Wien@\textbf{Wien}!XVIII., Währing@\textbf{XVIII., Währing}!Edmund-Weiß-Gasse 7@\textbf{Edmund-Weiß-Gasse 7}, \emph{Wohngebäude}|pw}\pend{}{\bigskip}
\pstart
           \noindent{}\centering{}{\pb}\textcolor{gray}{\textbf{Buckow Märk.-Schweiz\oindex{Buckow@\textbf{Buckow}|pw} – Neue Promenade\oindex{Neue Promenade@\textbf{Neue Promenade}, \emph{Straße}|pw}}}\pend
           \vspace{1em}
\pstart
           \noindent{}{\pb}\label{T_L01600-1v}\edtext{Das ist}{\lemma{\textnormal{\emph{Das ist}}}\Cendnote{\textnormal{durch einen Strich auf der Abbildung kenntlich gemacht}}}\label{T_L01600-1}
               unsere Wohnung. Im 1\textsuperscript{ten} Stock.\pend
           
\pstart
           {\pb}Schreib, Elende!\pend
           
\pstart
           Elende{ }ſchreib!\pend
           
\pstart
           Deine {\\[\baselineskip]}\spacefill\mbox{Mirjam.}\pend
           \leftskip=0em{}\selectlanguage{ngerman}\endnumbering\briefempfaengerindex{Schnitzler, Olga@\textsc{Schnitzler, Olga}!zzzBeer-Hofmann, Mirjam@\emph{von Mirjam Beer-Hofmann}!1906-06-121@{12. 6. 1906}|)be}\mylabel{L01600h}  \newcommand{\dateiname}{L01600}\newcommand{\titel}{Mirjam Beer-Hofmann an Olga Schnitzler, 12. 6. 1906}\newcommand{\editorInnen}{Martin Anton Müller und Gerd-Hermann Susen}%% latex-leseansicht-abspann.tex
%% Abspann für die Leseansicht.
%% Der Schalter \ifkorrekturansicht ist bereits durch den Vorspann gesetzt.

%% latex-abspann.tex
%% Gemeinsamer Abspann für Korrekturansicht und Leseansicht.
%% Setzt den Schalter \ifkorrekturansicht voraus (gesetzt in den
%% einbindenden Dateien latex-korrekturansicht-abspann.tex bzw.
%% latex-leseansicht-abspann.tex).
%% ---------------------------------------------------------------

\normalsize

% Das esempio-Environment wird nur in der Leseansicht benötigt
\ifkorrekturansicht\else
\newenvironment{esempio}[3]%
{
    \vspace{1.5ex}
    \rlap{\underline{#1}}
    \par
    \setlength{\parindent}{0cm}
    \nopagebreak
    \leftskip=#2cm
    \rightskip=#3cm
}
{
    \par
}
\fi

\doendnotes{C}
\bigskip
\vfill

\clearpage

\footnotesize

\ifkorrekturansicht
  \lohead{\textsc{register}}
\fi

% theindex-Environment neu definieren ohne reledmac
\makeatletter
\renewenvironment{theindex}{%
  \ifkorrekturansicht
    \section*{\indexname}%
  \else
    \subsubsection*{Index der erwähnten Entitäten}%
  \fi
  \setlength{\parindent}{0pt}%
  \setlength{\parskip}{0pt plus 0.3pt}%
  \let\item\@idxitem
}{%
  \ifkorrekturansicht\clearpage\fi
}
\makeatother

\IfFileExists{\jobname-pw.ind}{\input{\jobname-pw.ind}}{}

% Quellenangabe nur in der Leseansicht
\ifkorrekturansicht\else
% Fallback-Definitionen, falls die .tex-Datei \titel etc. nicht gesetzt hat
\providecommand{\titel}{}
\providecommand{\editorInnen}{}
\providecommand{\dateiname}{\jobname}

\vspace{3cm}

\vfill

\footnotesize
\textsc{Quelle}: \titel. Herausgegeben von {\editorInnen}. In: \emph{Arthur Schnitzler: Briefwechsel mit Autorinnen und Autoren}.
 Digitale Edition, https://schnitzler-briefe.acdh.oeaw.ac.at/{\dateiname}.html (Stand \today)
\fi

\end{document}


