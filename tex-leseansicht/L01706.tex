%% latex-korrekturansicht-vorspann.tex
%% Vorspann für die Korrekturansicht.
%% Lädt die gemeinsame Datei latex-vorspann.tex mit gesetztem Schalter.

\newif\ifkorrekturansicht
\korrekturansichttrue

\input{../tex-inputs/latex-vorspann}


\section[Arthur Schnitzler an Richard Beer-Hofmann, 9. 9. 1907]{L01706 Arthur Schnitzler an Richard Beer-Hofmann, 9. 9. 1907}
\nopagebreak\mylabel{L01706v}
\rehead{ }\normalsize\beginnumbering\briefempfaengerindex{Beer-Hofmann, Richard@\textsc{Beer-Hofmann, Richard}!zzzSchnitzler, Arthur@\emph{von Arthur Schnitzler}!1907-09-091@{9. 9. 1907}|(be}
\toendnotes[C]{\smallbreak\pagebreak[2]}\Standort{YCGL, MSS 31.}
\physDesc{Bildpostkarte, 308 Zeichen
\newline{}Handschrift: 1) blaue Tinte, deutsche Kurrent\hspace{1em}2) blaue Tinte, lateinische Kurrent (\noindent{}Adresse)\hspace{1em}
\newline{}Versand: 1) Stempel: »\nobreak{}\oindex{Meran@\textbf{Meran}, \emph{P.PPLA3}|pwk}Mer{[}an{]}, 9. IX. 07, 7\nobreak{}«.   2) Stempel: »\nobreak{}\oindex{Venedig@\textbf{Venedig}, \emph{P.PPLA}|pwk}\textcolor{gray}{S}ez\textcolor{gray}{io}ni Riunite
                                       Venezia, 9 9 {[}1907{]}, 1S\nobreak{}«.  3) Stempel: »\nobreak{}\oindex{Santa Maria Elisabetta@\textbf{Santa Maria Elisabetta}, \emph{Bezirk (A.BZK)}|pwk}S. Elisabetta di Lido
                                       (Venezia), 9. 9. 07\nobreak{}«. 
\newline{}Ordnung: mit Bleistift von unbekannter Hand datiert: »9. 9.« }\pstart{}{\pb}Dr. Richard Beer-Hofmann\pend{}\pstart{}Venedig Lido\oindex{Lido@\textbf{Lido}, \emph{P.PPL}|pw}\pend{}{\bigskip}
\pstart
           \noindent{}\centering{}{\pb}\textcolor{gray}{\textbf{PALAST-HOTEL\oindex{Palasthotel Meran@\textbf{Palasthotel Meran}, \emph{Hotel (K.HTL)}|pw}}}\pend
           
\pstart
           \centering{}\textcolor{gray}{\textbf{MERAN\oindex{Meran@\textbf{Meran}, \emph{P.PPLA3}|pw}}}\pend
           
\pstart
           \raggedleft{}\textcolor{gray}{\textbf{Palast-Hotel, Meran\oindex{Palasthotel Meran@\textbf{Palasthotel Meran}, \emph{Hotel (K.HTL)}|pw}}}{\\}\textcolor{gray}{\textbf{Franz Leibl\pwindex{Leibl, Franz 1854 – 1921-08-27@\textsc{Leibl, Franz} (1854 – 1921-08-27), \emph{Hotelbesitzer/Hotelbesitzerin}|pw}, Hotelier}}\pend
           \vspace{1em}
\pstart
           \noindent{}{\pb}lieber Richard, ich habe keine Nachricht von Ihnen. In Bozen\oindex{Bozen@\textbf{Bozen}, \emph{P.PPLA2}|pw} war nichts (3 Briefe u ein Telegra{\geminationm} ſcheint ein Unberufener abgeholt zu haben) laſſen Sie
               mich ein Wort hören. Nach Wien\oindex{Wien@\textbf{Wien}, \emph{A.ADM2}|pw}, wo wir (über Bozen\oindex{Bozen@\textbf{Bozen}, \emph{P.PPLA2}|pw}, Mendel\oindex{Mendelpass@\textbf{Mendelpass}, \emph{Pass (N.PAS)}|pw}, Sa\textcolor{gray}{lzg}, I{\geminationn}sbruck\oindex{Innsbruck@\textbf{Innsbruck}, \emph{A.ADM2}|pw}, ev. Salzburg\oindex{Salzburg@\textbf{Salzburg}, \emph{A.ADM2}|pw}) Freitag zu ſein hoffen. \pend
           
\pstart
           Herzlichſt Ihr{\\[\baselineskip]}\spacefill\mbox{A.}\pend
           \leftskip=0em{}\selectlanguage{ngerman}\endnumbering\briefempfaengerindex{Beer-Hofmann, Richard@\textsc{Beer-Hofmann, Richard}!zzzSchnitzler, Arthur@\emph{von Arthur Schnitzler}!1907-09-091@{9. 9. 1907}|)be}\mylabel{L01706h}  \normalsize

\doendnotes{C}
\bigskip
\vfill

\clearpage

\footnotesize

\lohead{\textsc{register}}

% Definiere theindex-Environment komplett neu ohne reledmac
\makeatletter
\renewenvironment{theindex}{%
  \section*{\indexname}%
  \setlength{\parindent}{0pt}%
  \setlength{\parskip}{0pt plus 0.3pt}%
  \let\item\@idxitem
}{%
  \clearpage
}
\makeatother

\IfFileExists{\jobname-pw.ind}{\input{\jobname-pw.ind}}{}

\end{document}

      