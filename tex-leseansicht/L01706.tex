%% latex-leseansicht-vorspann.tex
%% Vorspann für die Leseansicht.
%% Lädt die gemeinsame Datei latex-vorspann.tex mit nicht gesetztem Schalter.

\newif\ifkorrekturansicht
\korrekturansichtfalse

\input{../tex-inputs/latex-vorspann}


\section[Arthur Schnitzler an Richard Beer-Hofmann, 9. 9. 1907]{L01706 Arthur Schnitzler an Richard Beer-Hofmann, 9. 9. 1907}
\nopagebreak\mylabel{L01706v}
\rehead{ }\normalsize\beginnumbering\briefempfaengerindex{Beer-Hofmann, Richard@\textsc{Beer-Hofmann, Richard}!zzzSchnitzler, Arthur@\emph{von Arthur Schnitzler}!1907-09-091@{9. 9. 1907}|(be}
\toendnotes[C]{\smallbreak\pagebreak[2]}
\correspDesc{Versand  durch Arthur Schnitzler am 9. 9. 1907 in Meran
\newline{}Erhalt  durch Richard Beer-Hofmann am 9. 9. 1907 in Lido}\toendnotes[C]{\smallbreak}
\Standort{YCGL, MSS 31.}
\physDesc{Bildpostkarte, 308 Zeichen
\newline{}Handschrift: blaue Tinte, deutsche Kurrent
\newline{}Versand: 1) Stempel: »\nobreak{}\oindex{Meran@\textbf{Meran}, \emph{Hauptstadt}|pwk}Mer{[}an{]}, 9. IX. 07, 7\nobreak{}«.   2) Stempel: »\nobreak{}\oindex{Venedig@\textbf{Venedig}|pwk}\textcolor{gray}{S}ez\textcolor{gray}{io}ni Riunite
                                       Venezia, 9 9 {[}1907{]}, 1S\nobreak{}«.  3) Stempel: »\nobreak{}\oindex{Santa Maria Elisabetta@\textbf{Santa Maria Elisabetta}, \emph{Bezirk}|pwk}S. Elisabetta di Lido
                                       (Venezia), 9. 9. 07\nobreak{}«. 
\newline{}Ordnung: mit Bleistift von unbekannter Hand datiert: »9. 9.« }\pstart{}\textsc{{\pb}Dr. Richard Beer-Hofmann}\pend{}\pstart{}\textsc{Venedig Lido\oindex{Lido@\textbf{Lido}|pw}}\pend{}{\bigskip}
\pstart
           \noindent{}\centering{}{\pb}\textcolor{gray}{\textbf{PALAST-HOTEL\oindex{Palasthotel Meran@\textbf{Palasthotel Meran}, \emph{Hotel}|pw}}}\pend
           
\pstart
           \centering{}\textcolor{gray}{\textbf{MERAN\oindex{Meran@\textbf{Meran}, \emph{Hauptstadt}|pw}}}\pend
           
\pstart
           \raggedleft{}\textcolor{gray}{\textbf{Palast-Hotel, Meran\oindex{Palasthotel Meran@\textbf{Palasthotel Meran}, \emph{Hotel}|pw}}}{\\}\textcolor{gray}{\textbf{Franz Leibl\pwindex{Leibl, Franz 1854 – 27.\,8.\,1921@\textsc{Leibl, Franz} (1854 – 27.\,8.\,1921), \emph{Hotelbesitzer}|pw}, Hotelier}}\pend
           \vspace{1em}
\pstart
           \noindent{}{\pb}lieber Richard, ich habe keine Nachricht von Ihnen. In Bozen\oindex{Bozen@\textbf{Bozen}, \emph{Hauptstadt}|pw} war nichts (3 Briefe u ein Telegra{\geminationm}{ }ſcheint ein Unberufener abgeholt zu haben) laſſen Sie
               mich ein Wort hören. Nach Wien\oindex{Wien@\textbf{Wien}, \emph{Verwaltungsgebiet}|pw}, wo wir (über Bozen\oindex{Bozen@\textbf{Bozen}, \emph{Hauptstadt}|pw}, Mendel\oindex{Mendelpass@\textbf{Mendelpass}, \emph{Pass}|pw}, Sa\textcolor{gray}{lzg}, I{\geminationn}sbruck\oindex{Innsbruck@\textbf{Innsbruck}, \emph{Verwaltungsgebiet}|pw}, ev. Salzburg\oindex{Salzburg@\textbf{Salzburg}, \emph{Verwaltungsgebiet}|pw}) Freitag zu{ }ſein hoffen.\pend
           
\pstart
           Herzlichſt Ihr{\\[\baselineskip]}\spacefill\mbox{A.}\pend
           \leftskip=0em{}\selectlanguage{ngerman}\endnumbering\briefempfaengerindex{Beer-Hofmann, Richard@\textsc{Beer-Hofmann, Richard}!zzzSchnitzler, Arthur@\emph{von Arthur Schnitzler}!1907-09-091@{9. 9. 1907}|)be}\mylabel{L01706h}  \newcommand{\dateiname}{L01706}\newcommand{\titel}{Arthur Schnitzler an Richard Beer-Hofmann, 9. 9. 1907}\newcommand{\editorInnen}{Martin Anton Müller und Gerd-Hermann Susen}%% latex-leseansicht-abspann.tex
%% Abspann für die Leseansicht.
%% Der Schalter \ifkorrekturansicht ist bereits durch den Vorspann gesetzt.

%% latex-abspann.tex
%% Gemeinsamer Abspann für Korrekturansicht und Leseansicht.
%% Setzt den Schalter \ifkorrekturansicht voraus (gesetzt in den
%% einbindenden Dateien latex-korrekturansicht-abspann.tex bzw.
%% latex-leseansicht-abspann.tex).
%% ---------------------------------------------------------------

\normalsize

% Das esempio-Environment wird nur in der Leseansicht benötigt
\ifkorrekturansicht\else
\newenvironment{esempio}[3]%
{
    \vspace{1.5ex}
    \rlap{\underline{#1}}
    \par
    \setlength{\parindent}{0cm}
    \nopagebreak
    \leftskip=#2cm
    \rightskip=#3cm
}
{
    \par
}
\fi

\doendnotes{C}
\bigskip
\vfill

\clearpage

\footnotesize

\ifkorrekturansicht
  \lohead{\textsc{register}}
\fi

% theindex-Environment neu definieren ohne reledmac
\makeatletter
\renewenvironment{theindex}{%
  \ifkorrekturansicht
    \section*{\indexname}%
  \else
    \subsubsection*{Index der erwähnten Entitäten}%
  \fi
  \setlength{\parindent}{0pt}%
  \setlength{\parskip}{0pt plus 0.3pt}%
  \let\item\@idxitem
}{%
  \ifkorrekturansicht\clearpage\fi
}
\makeatother

\IfFileExists{\jobname-pw.ind}{\input{\jobname-pw.ind}}{}

% Quellenangabe nur in der Leseansicht
\ifkorrekturansicht\else
% Fallback-Definitionen, falls die .tex-Datei \titel etc. nicht gesetzt hat
\providecommand{\titel}{}
\providecommand{\editorInnen}{}
\providecommand{\dateiname}{\jobname}

\vspace{3cm}

\vfill

\footnotesize
\textsc{Quelle}: \titel. Herausgegeben von {\editorInnen}. In: \emph{Arthur Schnitzler: Briefwechsel mit Autorinnen und Autoren}.
 Digitale Edition, https://schnitzler-briefe.acdh.oeaw.ac.at/{\dateiname}.html (Stand \today)
\fi

\end{document}


