%% latex-leseansicht-vorspann.tex
%% Vorspann für die Leseansicht.
%% Lädt die gemeinsame Datei latex-vorspann.tex mit nicht gesetztem Schalter.

\newif\ifkorrekturansicht
\korrekturansichtfalse

\input{../tex-inputs/latex-vorspann}


         
         \renewcommand{\erwaehntePersonen}{Personen: Maximilian Harden}
         \renewcommand{\erwaehnteOrte}{Orte: Wien}
         \renewcommand{\erwaehnteWerke}{Werke: Der Ruf des Lebens. Schauspiel in drei Akten, Die Zukunft, Leidenschaft. Trauerspiel in fünf Aufzügen, Oedipus und die Sphinx. Tragödie in drei Aufzügen, Theater, Und Pippa tanzt{\rufezeichen}}
               \section[Hugo von Hofmannsthal an Arthur Schnitzler, {[}4. 3. 1906{]}]{ Hugo von Hofmannsthal an Arthur Schnitzler, {[}4. 3. 1906{]}}\nopagebreak\mylabel{v}\rehead{ }\begin{ledgroupsized}[t]{13cm}\normalsize\beginnumbering \toendnotes[C]{\smallbreak\pagebreak[2]} \Standort{CUL, Schnitzler, B 43.}
\physDesc{Brief, 1 Blatt, 4 Seiten, 1044 Zeichen
\newline{}Handschrift: schwarze Tinte, deutsche Kurrent
\newline{}Schnitzler: mit Bleistift datiert: »4/3 906« 
\newline{}Ordnung: 1) mit Bleistift von unbekannter Hand nummeriert: »\strikeout{264}«  2) mit Bleistift von unbekannter Hand nummeriert:
                                    »261«}\buchAbdrucke{\weitereDrucke{Hugo von Hofmannsthal, Arthur Schnitzler: \emph{Briefwechsel}. Hg. Therese Nickl und Heinrich Schnitzler. Frankfurt am Main: \emph{S. Fischer} 1964, S. 217.} }\toendnotes[C]{\smallbreak}\pstart
           \raggedleft{}{\pb}Sonntag.\pend
           \pstart{}mein lieber Arthur \pend\pstart
           ich wünſche mir ſo ſehr, ein paar Stunden mit Ihnen ruhig zu verbringen, von Ihrem
                  Stück\pwindex{Schnitzler, Arthur 15.05.1862 – 21.10.1931@\textsc{Schnitzler, Arthur} (15.05.1862 – 21.10.1931), \emph{Schriftsteller, Mediziner}!Ruf des Lebens. Schauspiel in drei Akten1906-02-20@\strich\emph{Der Ruf des Lebens. Schauspiel in drei Akten} {[}1906-02-20{]}|pwv} zu reden, das ich ſo
               ſehr ſchön finde (habs wieder geleſen) und von anderen Dingen.\pend
           \pstart
           Bitte ſchlagen Sie uns einen Abend der Woche vor, uns iſt jeder recht. Soll man denn
               alt werden und einander ſo wenig gehabt haben? – Völlig {\pb}bestürzt, direct getroffen wie von
               etwas ganz Schlechtem, die Nerven aufregenden bin ich von dieſem unſinnigen brutalen
                  \label{K_L01586-1v}\edtext{Aufſatz\pwindex{Theater03. 03. 1906@\emph{Theater} {[}03. 03. 1906{]}|pwv}}{\lemma{\textnormal{\emph{Aufſatz}}}\Cendnote{\textnormal{Harden\pwindex{Harden, Maximilian 20.10.1861 – 30.10.1927@\textsc{Harden, Maximilian} (20.10.1861 – 30.10.1927), \emph{Schriftsteller, Publizist}|pwk} hatte einer längeren, ausführlichen
                  Besprechung von \emph{Ödipus und die Sphinx}\pwindex{Hofmannsthal, Hugo von 1874-02-01 – 1929-07-15@\textsc{Hofmannsthal, Hugo von} (1874-02-01 – 1929-07-15), \emph{Schriftsteller}!Oedipus und die Sphinx. Tragoedie in drei Aufzuegen1906@\strich\emph{Oedipus und die Sphinx. Tragödie in drei Aufzügen} {[}1906{]}|pwk} einen
                  einseitigen Verriss von \emph{Der Ruf des Lebens}\pwindex{Schnitzler, Arthur 15.05.1862 – 21.10.1931@\textsc{Schnitzler, Arthur} (15.05.1862 – 21.10.1931), \emph{Schriftsteller, Mediziner}!Ruf des Lebens. Schauspiel in drei Akten1906-02-20@\strich\emph{Der Ruf des Lebens. Schauspiel in drei Akten} {[}1906-02-20{]}|pwk}
                  angehängt (M. H.\pwindex{Harden, Maximilian 20.10.1861 – 30.10.1927@\textsc{Harden, Maximilian} (20.10.1861 – 30.10.1927), \emph{Schriftsteller, Publizist}|pwk}: \emph{Theater}\pwindex{Theater03. 03. 1906@\emph{Theater} {[}03. 03. 1906{]}|pwk}. In: \emph{Die Zukunft}\pwindex{Zukunft1892 – 1922@\emph{Die Zukunft} {[}1892 – 1922{]}|pwk}, Bd. 54,
                     H. 9, 3. 3. 1906, S. 346–356).}}}\label{K_L01586-1h} von \textsc{Harden}\pwindex{Harden, Maximilian 20.10.1861 – 30.10.1927@\textsc{Harden, Maximilian} (20.10.1861 – 30.10.1927), \emph{Schriftsteller, Publizist}|pw}. So muſs man ſich denn entſchließen, dieſen bedeutenden Menſchen zu den
               pathologiſchen Existenzen, deren Gefährlichkeit mit ihrer {\pb}Unberechenbarkeit wächſt, zu
               werfen! Wie traurig. Ich mühe mich, es zu begreifen, die Wurzel dieſer wilden, um
               ſich freſſenden Parteilichkeit, dieſer fieberhaften Zerrüttung zu faſſen –\hspace*{1.5em}Ich habe an ihn \label{K_L01586-2v}\edtext{geſchrieben}{\lemma{\textnormal{\emph{geſchrieben}}}\Cendnote{\textnormal{der
                  Brief vom 4. 3. 1906 (Hans Georg Schede, Hg.: \emph{Hugo von Hofmannsthal – Maximilian Harden}. In: \emph{Hofmannsthal-Jahrbuch}, Jg. 6, 1998, S. 93–97).
                  Die noch harschere Antwort Harden\pwindex{Harden, Maximilian 20.10.1861 – 30.10.1927@\textsc{Harden, Maximilian} (20.10.1861 – 30.10.1927), \emph{Schriftsteller, Publizist}|pwk}s ist
                  nicht überliefert, Hofmannsthal\pwindex{Hofmannsthal, Hugo von 1874-02-01 – 1929-07-15@\textsc{Hofmannsthal, Hugo von} (1874-02-01 – 1929-07-15), \emph{Schriftsteller}|pwk} zog dann
                  aber – wohl in Abstimmung mit Schnitzler\pwindex{Schnitzler, Arthur 15.05.1862 – 21.10.1931@\textsc{Schnitzler, Arthur} (15.05.1862 – 21.10.1931), \emph{Schriftsteller, Mediziner}|pwk} –
                  seinen Vorschlag einer Replik zurück.}}}\label{K_L01586-2h}, mit den bitterſten Vorwürfen und ihn
               gefragt, ob er mir {\pb}erlauben will,
               in der Zukunft\pwindex{Zukunft1892 – 1922@\emph{Die Zukunft} {[}1892 – 1922{]}|pw} ein »Geſpräch über einige neue
               Theaterſtücke« (ich denke an Ruf des Lebens\pwindex{Schnitzler, Arthur 15.05.1862 – 21.10.1931@\textsc{Schnitzler, Arthur} (15.05.1862 – 21.10.1931), \emph{Schriftsteller, Mediziner}!Ruf des Lebens. Schauspiel in drei Akten1906-02-20@\strich\emph{Der Ruf des Lebens. Schauspiel in drei Akten} {[}1906-02-20{]}|pw} –
                  Pippa\pwindex{\textcolor{red}{\textsuperscript{XXXX1 indx}}!Und Pippa tanzt1906@\strich\emph{Und Pippa tanzt{\rufezeichen}} {[}1906{]}|pw} – Leidenſchaft\pwindex{\textcolor{red}{\textsuperscript{XXXX1 indx}}!Leidenschaft. Trauerspiel in fuenf Aufzuegen1901@\strich\emph{Leidenschaft. Trauerspiel in fünf Aufzügen} {[}1901{]}|pw}) zu bringen. Bin neugierig, was er antwortet.\pend
           \pstart
           Ihr{\\[\baselineskip]}\spacefill\mbox{Hugo.}\pend
           \leftskip=0em{}
         
         \endnumbering\mylabel{h}\end{ledgroupsized}  \newcommand{\dateiname}{L01586}\newcommand{\titel}{Hugo von Hofmannsthal an Arthur Schnitzler, [4. 3. 1906]}\newcommand{\editorInnen}{Martin Anton Müller und Gerd-Hermann Susen}%% latex-leseansicht-abspann.tex
%% Abspann für die Leseansicht.
%% Der Schalter \ifkorrekturansicht ist bereits durch den Vorspann gesetzt.

%% latex-abspann.tex
%% Gemeinsamer Abspann für Korrekturansicht und Leseansicht.
%% Setzt den Schalter \ifkorrekturansicht voraus (gesetzt in den
%% einbindenden Dateien latex-korrekturansicht-abspann.tex bzw.
%% latex-leseansicht-abspann.tex).
%% ---------------------------------------------------------------

\normalsize

% Das esempio-Environment wird nur in der Leseansicht benötigt
\ifkorrekturansicht\else
\newenvironment{esempio}[3]%
{
    \vspace{1.5ex}
    \rlap{\underline{#1}}
    \par
    \setlength{\parindent}{0cm}
    \nopagebreak
    \leftskip=#2cm
    \rightskip=#3cm
}
{
    \par
}
\fi

\doendnotes{C}
\bigskip
\vfill

\clearpage

\footnotesize

\ifkorrekturansicht
  \lohead{\textsc{register}}
\fi

% theindex-Environment neu definieren ohne reledmac
\makeatletter
\renewenvironment{theindex}{%
  \ifkorrekturansicht
    \section*{\indexname}%
  \else
    \subsubsection*{Index der erwähnten Entitäten}%
  \fi
  \setlength{\parindent}{0pt}%
  \setlength{\parskip}{0pt plus 0.3pt}%
  \let\item\@idxitem
}{%
  \ifkorrekturansicht\clearpage\fi
}
\makeatother

\IfFileExists{\jobname-pw.ind}{\input{\jobname-pw.ind}}{}

% Quellenangabe nur in der Leseansicht
\ifkorrekturansicht\else
% Fallback-Definitionen, falls die .tex-Datei \titel etc. nicht gesetzt hat
\providecommand{\titel}{}
\providecommand{\editorInnen}{}
\providecommand{\dateiname}{\jobname}

\vspace{3cm}

\vfill

\footnotesize
\textsc{Quelle}: \titel. Herausgegeben von {\editorInnen}. In: \emph{Arthur Schnitzler: Briefwechsel mit Autorinnen und Autoren}.
 Digitale Edition, https://schnitzler-briefe.acdh.oeaw.ac.at/{\dateiname}.html (Stand \today)
\fi

\end{document}


      