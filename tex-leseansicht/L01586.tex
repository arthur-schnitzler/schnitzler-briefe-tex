%% latex-leseansicht-vorspann.tex
%% Vorspann für die Leseansicht.
%% Lädt die gemeinsame Datei latex-vorspann.tex mit nicht gesetztem Schalter.

\newif\ifkorrekturansicht
\korrekturansichtfalse

\input{../tex-inputs/latex-vorspann}


\section[Hugo von Hofmannsthal an Arthur Schnitzler, {{[}}4. 3. 1906{{]}}]{L01586 Hugo von Hofmannsthal an Arthur Schnitzler, {[}4. 3. 1906{]}}
\nopagebreak\mylabel{L01586v}
\rehead{ }\normalsize\beginnumbering\briefempfaengerindex{Schnitzler, Arthur@\textsc{Schnitzler, Arthur}!zzzHofmannsthal, Hugo von@\emph{von Hugo von Hofmannsthal}!1906-03-041@{{[}4. 3. 1906{]}}|(be}
\toendnotes[C]{\smallbreak\pagebreak[2]}
\correspDesc{Versand  durch Hugo von Hofmannsthal am [4. 3. 1906] in Wien
\newline{}Erhalt  durch Arthur Schnitzler im Zeitraum [4. 3. 1906
                  – 8. 3. 1906?] in Wien}\toendnotes[C]{\smallbreak}
\Standort{CUL, Schnitzler, B 43.}
\physDesc{Brief, 1 Blatt, 4 Seiten, 1044 Zeichen
\newline{}Handschrift: schwarze Tinte, deutsche Kurrent
\newline{}Schnitzler: mit Bleistift datiert: »4/3 906« 
\newline{}Ordnung: 1) mit Bleistift von unbekannter Hand nummeriert: »\strikeout{264}«  2) mit Bleistift von unbekannter Hand nummeriert:
                                    »261«}
\buchAbdrucke{\weitereDrucke{Hugo von Hofmannsthal, Arthur Schnitzler: \emph{Briefwechsel}. Herausgegeben von Therese Nickl und Heinrich Schnitzler. Frankfurt am Main: \emph{S. Fischer} 1964, S. 217.} }\toendnotes[C]{\smallbreak}
\pstart
           \raggedleft{}{\pb}Sonntag.\pend
           
\pstart{}mein lieber Arthur\pend\vspace{0.5em}
\pstart
           ich wünſche mir{ }ſo{ }ſehr, ein paar Stunden mit Ihnen ruhig zu verbringen, von Ihrem
                  Stück\pwindex{Schnitzler, Arthur 15.\,5.\,1862 Wien – 21.\,10.\,1931 ebd.@\textsc{Schnitzler, Arthur} (15.\,5.\,1862 Wien – 21.\,10.\,1931 ebd.), \emph{Schriftsteller, Mediziner}!Ruf des Lebens. Schauspiel in drei Akten@\strich\emph{Der Ruf des Lebens. Schauspiel in drei Akten}|pwv} zu reden, das ich{ }ſo{ }ſehr{ }ſchön finde (habs wieder geleſen) und von anderen Dingen.\pend
           
\pstart
           Bitte{ }ſchlagen Sie uns einen Abend der Woche vor, uns iſt jeder recht. Soll man denn
               alt werden und einander{ }ſo wenig gehabt haben? – Völlig {\pb}bestürzt, direct getroffen wie von
               etwas ganz Schlechtem, die Nerven aufregenden bin ich von dieſem unſinnigen brutalen
                  \label{K_L01586-1v}\edtext{Aufſatz\pwindex{Harden, Maximilian 20.\,10.\,1861 Berlin – 30.\,10.\,1927 Montana@\textsc{Harden, Maximilian} (20.\,10.\,1861 Berlin – 30.\,10.\,1927 Montana), \emph{Schriftsteller, Publizist}!Theater@\strich\emph{Theater}|pwv}}{\lemma{\textnormal{\emph{Aufsatz}}}\Cendnote{\textnormal{Harden\pwindex{Harden, Maximilian 20.\,10.\,1861 Berlin – 30.\,10.\,1927 Montana@\textsc{Harden, Maximilian} (20.\,10.\,1861 Berlin – 30.\,10.\,1927 Montana), \emph{Schriftsteller, Publizist}|pwk} hatte einer längeren, ausführlichen
                  Besprechung von \emph{Ödipus und die Sphinx}\pwindex{Hofmannsthal, Hugo von 1.\,2.\,1874 Wien – 15.\,7.\,1929 Rodaun@\textsc{Hofmannsthal, Hugo von} (1.\,2.\,1874 Wien – 15.\,7.\,1929 Rodaun), \emph{Schriftsteller}!Oedipus und die Sphinx. Tragödie in drei Aufzügen@\strich\emph{Oedipus und die Sphinx. Tragödie in drei Aufzügen}|pwk} einen
                  einseitigen Verriss von \emph{Der Ruf des Lebens}\pwindex{Schnitzler, Arthur 15.\,5.\,1862 Wien – 21.\,10.\,1931 ebd.@\textsc{Schnitzler, Arthur} (15.\,5.\,1862 Wien – 21.\,10.\,1931 ebd.), \emph{Schriftsteller, Mediziner}!Ruf des Lebens. Schauspiel in drei Akten@\strich\emph{Der Ruf des Lebens. Schauspiel in drei Akten}|pwk}
                  angehängt (M. H.\pwindex{Harden, Maximilian 20.\,10.\,1861 Berlin – 30.\,10.\,1927 Montana@\textsc{Harden, Maximilian} (20.\,10.\,1861 Berlin – 30.\,10.\,1927 Montana), \emph{Schriftsteller, Publizist}|pwk}: \emph{Theater}\pwindex{Harden, Maximilian 20.\,10.\,1861 Berlin – 30.\,10.\,1927 Montana@\textsc{Harden, Maximilian} (20.\,10.\,1861 Berlin – 30.\,10.\,1927 Montana), \emph{Schriftsteller, Publizist}!Theater@\strich\emph{Theater}|pwk}. In: \emph{Die Zukunft}\pwindex{Zukunft@\emph{Die Zukunft}|pwk}, Bd. 54,
                     H. 9, 3. 3. 1906, S. 346–356).}}}\label{K_L01586-1} von \textsc{Harden}\pwindex{Harden, Maximilian 20.\,10.\,1861 Berlin – 30.\,10.\,1927 Montana@\textsc{Harden, Maximilian} (20.\,10.\,1861 Berlin – 30.\,10.\,1927 Montana), \emph{Schriftsteller, Publizist}|pw}. So muſs man{ }ſich denn entſchließen, dieſen bedeutenden Menſchen zu den
               pathologiſchen Existenzen, deren Gefährlichkeit mit ihrer {\pb}Unberechenbarkeit wächſt, zu
               werfen! Wie traurig. Ich mühe mich, es zu begreifen, die Wurzel dieſer wilden, um{ }ſich freſſenden Parteilichkeit, dieſer fieberhaften Zerrüttung zu faſſen –\hspace*{1.5em}Ich habe an ihn \label{K_L01586-2v}\edtext{geſchrieben}{\lemma{\textnormal{\emph{geschrieben}}}\Cendnote{\textnormal{Der
                  Brief vom 4. 3. 1906 ist abgedruckt in: Hans Georg Schede, Herausgeber: \emph{Hugo von Hofmannsthal – Maximilian Harden}. In: \emph{Hofmannsthal-Jahrbuch}, Jg. 6, 1998, S. 93–97.
                  Die noch harschere Antwort Hardens\pwindex{Harden, Maximilian 20.\,10.\,1861 Berlin – 30.\,10.\,1927 Montana@\textsc{Harden, Maximilian} (20.\,10.\,1861 Berlin – 30.\,10.\,1927 Montana), \emph{Schriftsteller, Publizist}|pwk} ist
                  nicht überliefert, Hofmannsthal\pwindex{Hofmannsthal, Hugo von 1.\,2.\,1874 Wien – 15.\,7.\,1929 Rodaun@\textsc{Hofmannsthal, Hugo von} (1.\,2.\,1874 Wien – 15.\,7.\,1929 Rodaun), \emph{Schriftsteller}|pwk} zog dann
                  aber – wohl in Abstimmung mit Schnitzler –
                  seinen Vorschlag einer Replik zurück.}}}\label{K_L01586-2}, mit den bitterſten Vorwürfen und ihn
               gefragt, ob er mir {\pb}erlauben will,
               in der Zukunft\pwindex{Zukunft@\emph{Die Zukunft}|pw} ein »Geſpräch über einige neue
               Theaterſtücke« (ich denke an Ruf des Lebens\pwindex{Schnitzler, Arthur 15.\,5.\,1862 Wien – 21.\,10.\,1931 ebd.@\textsc{Schnitzler, Arthur} (15.\,5.\,1862 Wien – 21.\,10.\,1931 ebd.), \emph{Schriftsteller, Mediziner}!Ruf des Lebens. Schauspiel in drei Akten@\strich\emph{Der Ruf des Lebens. Schauspiel in drei Akten}|pw} –
                  Pippa\pwindex{\textcolor{red}{\textsuperscript{XXXX indx1}}!Und Pippa tanzt@\strich\emph{Und Pippa tanzt{\rufezeichen}}|pw} – Leidenſchaft\pwindex{\textcolor{red}{\textsuperscript{XXXX indx1}}!Leidenschaft. Trauerspiel in fünf Aufzügen@\strich\emph{Leidenschaft. Trauerspiel in fünf Aufzügen}|pw}) zu bringen. Bin neugierig, was er antwortet.\pend
           
\pstart
           Ihr{\\[\baselineskip]}\spacefill\mbox{Hugo.}\pend
           \leftskip=0em{}\selectlanguage{ngerman}\endnumbering\briefempfaengerindex{Schnitzler, Arthur@\textsc{Schnitzler, Arthur}!zzzHofmannsthal, Hugo von@\emph{von Hugo von Hofmannsthal}!1906-03-041@{{[}4. 3. 1906{]}}|)be}\mylabel{L01586h}  \newcommand{\dateiname}{L01586}\newcommand{\titel}{Hugo von Hofmannsthal an Arthur Schnitzler, [4. 3. 1906]}\newcommand{\editorInnen}{Martin Anton Müller und Gerd-Hermann Susen}%% latex-leseansicht-abspann.tex
%% Abspann für die Leseansicht.
%% Der Schalter \ifkorrekturansicht ist bereits durch den Vorspann gesetzt.

%% latex-abspann.tex
%% Gemeinsamer Abspann für Korrekturansicht und Leseansicht.
%% Setzt den Schalter \ifkorrekturansicht voraus (gesetzt in den
%% einbindenden Dateien latex-korrekturansicht-abspann.tex bzw.
%% latex-leseansicht-abspann.tex).
%% ---------------------------------------------------------------

\normalsize

% Das esempio-Environment wird nur in der Leseansicht benötigt
\ifkorrekturansicht\else
\newenvironment{esempio}[3]%
{
    \vspace{1.5ex}
    \rlap{\underline{#1}}
    \par
    \setlength{\parindent}{0cm}
    \nopagebreak
    \leftskip=#2cm
    \rightskip=#3cm
}
{
    \par
}
\fi

\doendnotes{C}
\bigskip
\vfill

\clearpage

\footnotesize

\ifkorrekturansicht
  \lohead{\textsc{register}}
\fi

% theindex-Environment neu definieren ohne reledmac
\makeatletter
\renewenvironment{theindex}{%
  \ifkorrekturansicht
    \section*{\indexname}%
  \else
    \subsubsection*{Index der erwähnten Entitäten}%
  \fi
  \setlength{\parindent}{0pt}%
  \setlength{\parskip}{0pt plus 0.3pt}%
  \let\item\@idxitem
}{%
  \ifkorrekturansicht\clearpage\fi
}
\makeatother

\IfFileExists{\jobname-pw.ind}{\input{\jobname-pw.ind}}{}

% Quellenangabe nur in der Leseansicht
\ifkorrekturansicht\else
% Fallback-Definitionen, falls die .tex-Datei \titel etc. nicht gesetzt hat
\providecommand{\titel}{}
\providecommand{\editorInnen}{}
\providecommand{\dateiname}{\jobname}

\vspace{3cm}

\vfill

\footnotesize
\textsc{Quelle}: \titel. Herausgegeben von {\editorInnen}. In: \emph{Arthur Schnitzler: Briefwechsel mit Autorinnen und Autoren}.
 Digitale Edition, https://schnitzler-briefe.acdh.oeaw.ac.at/{\dateiname}.html (Stand \today)
\fi

\end{document}


