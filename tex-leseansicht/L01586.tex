%% latex-korrekturansicht-vorspann.tex
%% Vorspann für die Korrekturansicht.
%% Lädt die gemeinsame Datei latex-vorspann.tex mit gesetztem Schalter.

\newif\ifkorrekturansicht
\korrekturansichttrue

\input{../tex-inputs/latex-vorspann}


\section[Hugo von Hofmannsthal an Arthur Schnitzler, {[}4. 3. 1906{]}]{L01586 Hugo von Hofmannsthal an Arthur Schnitzler, {[}4. 3. 1906{]}}
\nopagebreak\mylabel{L01586v}
\rehead{ }\normalsize\beginnumbering\briefempfaengerindex{Schnitzler, Arthur@\textsc{Schnitzler, Arthur}!zzzHofmannsthal, Hugo von@\emph{von Hugo von Hofmannsthal}!1906-03-041@{{[}4. 3. 1906{]}}|(be}
\toendnotes[C]{\smallbreak\pagebreak[2]}\Standort{CUL, Schnitzler, B 43.}
\physDesc{Brief, 1 Blatt, 4 Seiten, 1044 Zeichen
\newline{}Handschrift: schwarze Tinte, deutsche Kurrent
\newline{}Schnitzler: mit Bleistift datiert: »4/3 906« 
\newline{}Ordnung: 1) mit Bleistift von unbekannter Hand nummeriert: »\strikeout{264}«  2) mit Bleistift von unbekannter Hand nummeriert:
                                    »261«}
\buchAbdrucke{\weitereDrucke{Hugo von Hofmannsthal, Arthur Schnitzler: \emph{Briefwechsel}. Frankfurt am Main: \emph{S. Fischer} 1964, S. 217.} }\toendnotes[C]{\smallbreak}
\pstart
           \raggedleft{}{\pb}Sonntag.\pend
           
\pstart{}mein lieber Arthur \pend\vspace{0.5em}
\pstart
           ich wünſche mir ſo ſehr, ein paar Stunden mit Ihnen ruhig zu verbringen, von Ihrem
                  Stück\pwindex{Ruf des Lebens. Schauspiel in drei Akten@\emph{Der Ruf des Lebens. Schauspiel in drei Akten}|pwv} zu reden, das ich ſo
               ſehr ſchön finde (habs wieder geleſen) und von anderen Dingen.\pend
           
\pstart
           Bitte ſchlagen Sie uns einen Abend der Woche vor, uns iſt jeder recht. Soll man denn
               alt werden und einander ſo wenig gehabt haben? – Völlig {\pb}bestürzt, direct getroffen wie von
               etwas ganz Schlechtem, die Nerven aufregenden bin ich von dieſem unſinnigen brutalen
                  \label{K_L01586-1v}\edtext{Aufſatz\pwindex{Theater@\emph{Theater}|pwv}}{\lemma{\textnormal{\emph{Aufſatz}}}\Cendnote{\textnormal{Harden\pwindex{Harden, Maximilian 20.10.1861 – 30.10.1927@\textsc{Harden, Maximilian} (20.10.1861 – 30.10.1927), \emph{Schriftsteller/Schriftstellerin, Publizist/Publizistin}|pwk} hatte einer längeren, ausführlichen
                  Besprechung von \emph{Ödipus und die Sphinx}\pwindex{Oedipus und die Sphinx. Tragoedie in drei Aufzuegen@\emph{Oedipus und die Sphinx. Tragödie in drei Aufzügen}|pwk} einen
                  einseitigen Verriss von \emph{Der Ruf des Lebens}\pwindex{Ruf des Lebens. Schauspiel in drei Akten@\emph{Der Ruf des Lebens. Schauspiel in drei Akten}|pwk}
                  angehängt (M. H.\pwindex{Harden, Maximilian 20.10.1861 – 30.10.1927@\textsc{Harden, Maximilian} (20.10.1861 – 30.10.1927), \emph{Schriftsteller/Schriftstellerin, Publizist/Publizistin}|pwk}: \emph{Theater}\pwindex{Theater@\emph{Theater}|pwk}. In: \emph{Die Zukunft}\pwindex{Zukunft@\emph{Die Zukunft}|pwk}, Bd. 54,
                     H. 9, 3. 3. 1906, S. 346–356).}}}\label{K_L01586-1} von \textsc{Harden}\pwindex{Harden, Maximilian 20.10.1861 – 30.10.1927@\textsc{Harden, Maximilian} (20.10.1861 – 30.10.1927), \emph{Schriftsteller/Schriftstellerin, Publizist/Publizistin}|pw}. So muſs man ſich denn entſchließen, dieſen bedeutenden Menſchen zu den
               pathologiſchen Existenzen, deren Gefährlichkeit mit ihrer {\pb}Unberechenbarkeit wächſt, zu
               werfen! Wie traurig. Ich mühe mich, es zu begreifen, die Wurzel dieſer wilden, um
               ſich freſſenden Parteilichkeit, dieſer fieberhaften Zerrüttung zu faſſen –\hspace*{1.5em}Ich habe an ihn \label{K_L01586-2v}\edtext{geſchrieben}{\lemma{\textnormal{\emph{geſchrieben}}}\Cendnote{\textnormal{Der
                  Brief vom 4. 3. 1906 ist abgedruckt in: Hans Georg Schede, Herausgeber: \emph{Hugo von Hofmannsthal – Maximilian Harden}. In: \emph{Hofmannsthal-Jahrbuch}, Jg. 6, 1998, S. 93–97.
                  Die noch harschere Antwort Hardens\pwindex{Harden, Maximilian 20.10.1861 – 30.10.1927@\textsc{Harden, Maximilian} (20.10.1861 – 30.10.1927), \emph{Schriftsteller/Schriftstellerin, Publizist/Publizistin}|pwk} ist
                  nicht überliefert, Hofmannsthal\pwindex{Hofmannsthal, Hugo von 1874-02-01 – 1929-07-15@\textsc{Hofmannsthal, Hugo von} (1874-02-01 – 1929-07-15), \emph{Schriftsteller/Schriftstellerin}|pwk} zog dann
                  aber – wohl in Abstimmung mit Schnitzler –
                  seinen Vorschlag einer Replik zurück.}}}\label{K_L01586-2}, mit den bitterſten Vorwürfen und ihn
               gefragt, ob er mir {\pb}erlauben will,
               in der Zukunft\pwindex{Zukunft@\emph{Die Zukunft}|pw} ein »Geſpräch über einige neue
               Theaterſtücke« (ich denke an Ruf des Lebens\pwindex{Ruf des Lebens. Schauspiel in drei Akten@\emph{Der Ruf des Lebens. Schauspiel in drei Akten}|pw} –
                  Pippa\pwindex{Und Pippa tanzt@\emph{Und Pippa tanzt{\rufezeichen}}|pw} – Leidenſchaft\pwindex{Leidenschaft. Trauerspiel in fuenf Aufzuegen@\emph{Leidenschaft. Trauerspiel in fünf Aufzügen}|pw}) zu bringen. Bin neugierig, was er antwortet.\pend
           
\pstart
           Ihr{\\[\baselineskip]}\spacefill\mbox{Hugo.}\pend
           \leftskip=0em{}\selectlanguage{ngerman}\endnumbering\briefempfaengerindex{Schnitzler, Arthur@\textsc{Schnitzler, Arthur}!zzzHofmannsthal, Hugo von@\emph{von Hugo von Hofmannsthal}!1906-03-041@{{[}4. 3. 1906{]}}|)be}\mylabel{L01586h}  \normalsize

\doendnotes{C}
\bigskip
\vfill

\clearpage

\footnotesize

\lohead{\textsc{register}}

% Definiere theindex-Environment komplett neu ohne reledmac
\makeatletter
\renewenvironment{theindex}{%
  \section*{\indexname}%
  \setlength{\parindent}{0pt}%
  \setlength{\parskip}{0pt plus 0.3pt}%
  \let\item\@idxitem
}{%
  \clearpage
}
\makeatother

\IfFileExists{\jobname-pw.ind}{\input{\jobname-pw.ind}}{}

\end{document}

      