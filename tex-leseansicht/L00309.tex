\input{../tex-inputs/latex-pdf-vorspann}
\begin{center}
            \textcolor{red}{ENTWURF. ENTZIFFERUNG NOCH NICHT KORREKTURGELESEN}
                      \end{center}
            
               \section[Arthur Schnitzler an Richard Beer-Hofmann, {[}2. 4. 1894?{]}]{ Arthur Schnitzler an Richard Beer-Hofmann, {[}2. 4. 1894?{]}}\nopagebreak\mylabel{v}\rehead{ }\begin{ledgroupsized}[t]{13cm}\normalsize\beginnumbering\briefempfaengerindex{Beer-Hofmann, Richard@\textsc{Beer-Hofmann, Richard}!zzzSchnitzler, Arthur@\emph{von Arthur Schnitzler}!1894-04-021@{{[}2. 4. 1894?{]}}|(be} \toendnotes[C]{\smallbreak\pagebreak[2]} \Standort{YCGL, MSS 31.}
\physDesc{Brief, 1 Blatt (Briefpapier mit Trauerrand), 4 Seiten
\newline{}Handschrift: Bleistift, deutsche Kurrent}\buchAbdrucke{\weitereDrucke{1) Arthur Schnitzler, Richard Beer-Hofmann: \emph{Briefwechsel 1891–1931}. Hg. Konstanze Fliedl. Wien, Zürich: \emph{Europaverlag} 1992, S. 54.} \weitereDrucke{2) Hermann Bahr, Arthur Schnitzler: \emph{Briefwechsel, Aufzeichnungen, Dokumente
                                (1891–1931)}. Hg. Kurt Ifkovits und Martin Anton Müller. Göttingen: \emph{Wallstein} 2018.} }\toendnotes[C]{\smallbreak}\pstart{}{\pb}Lieber Richard,\pend\pstart
           Donnerſtag 11 Uhr hol ich Sie ab, wenn’s Ihnen recht iſt. Sie können das Fahren
                    ein paar Mal probiren, ohne ſich im geringſten zu verpflichten, und schli{\geminationm}ſten Falls zahlen Sie einen Mit{\pb}gliedsbeitrag auf ¼ Jahr, wodurch Sie zu
                        \strikeout{zu} gar nichts genötigt werden, weder zum
                    Kaufen eines Rades, noch zum Weiterverbleiben im Club\orgindex{Vorwaerts@Vorwärts|pwv}. –\pend
           \pstart
           Bitte ſehr, ſenden Sie dieſen Brief gleich {\pb}an Hermann Bahr\pwindex{Bahr, Hermann 19.07.1863 – 15.01.1934@\textsc{Bahr, Hermann} (19.07.1863 – 15.01.1934), \emph{Schriftsteller, Kritiker}|pw}, welcher hiedurch unter
                    einem gebeten wird, ſich um 11 am Donnerstag bei Ihnen einzufinden, we{\geminationn} er es nicht vorzieht, um 11 Uhr 30
                    vor dem Hauſe \label{K_L00309_1v}\edtext{\textsc{Untere
                            Augartenstraße} 28}{\lemma{\textnormal{\emph{Untere
                            Augartenstraße 28}}}\Cendnote{\textnormal{Sitz der
                            Radfahrunion \emph{Vorwärts}\orgindex{Vorwaerts@Vorwärts|pwk}.}}}\label{K_L00309_1h}\oindex{Untere Augartenstrasse@\textbf{Untere Augartenstraße}|pw} auf mich \textsc{resp}. uns zu warten.\pend
           \pstart
           {\pb}Beifolgend Statuten, von denen 1 Exemplar an
                        \textsc{Bahr}\pwindex{Bahr, Hermann 19.07.1863 – 15.01.1934@\textsc{Bahr, Hermann} (19.07.1863 – 15.01.1934), \emph{Schriftsteller, Kritiker}|pw}; in dieſem hab ich den § 15 unterſtrichen. Für Sie den § 5. –\pend
           \pstart
           Herzliche Grüße.{\\[\baselineskip]}\spacefill\mbox{ArthurSch}\pend
           \leftskip=0em{}\endnumbering\briefempfaengerindex{Beer-Hofmann, Richard@\textsc{Beer-Hofmann, Richard}!zzzSchnitzler, Arthur@\emph{von Arthur Schnitzler}!1894-04-021@{{[}2. 4. 1894?{]}}|)be}\mylabel{h}\end{ledgroupsized}  \newcommand{\dateiname}{L00309}\newcommand{\titel}{Arthur Schnitzler an Richard Beer-Hofmann, [2. 4. 1894?]}\newcommand{\editorInnen}{ Martin Anton Müller und Gerd-Hermann Susen}\input{../tex-inputs/latex-pdf-abspann}
      