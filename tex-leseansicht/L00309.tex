%% latex-korrekturansicht-vorspann.tex
%% Vorspann für die Korrekturansicht.
%% Lädt die gemeinsame Datei latex-vorspann.tex mit gesetztem Schalter.

\newif\ifkorrekturansicht
\korrekturansichttrue

\input{../tex-inputs/latex-vorspann}


\section[Arthur Schnitzler an Richard Beer-Hofmann, {[}2. 4. 1894?{]}]{L00309 Arthur Schnitzler an Richard Beer-Hofmann, {[}2. 4. 1894?{]}}
\nopagebreak\mylabel{L00309v}
\rehead{ }\normalsize\beginnumbering\briefempfaengerindex{Beer-Hofmann, Richard@\textsc{Beer-Hofmann, Richard}!zzzSchnitzler, Arthur@\emph{von Arthur Schnitzler}!1894-04-021@{{[}2. 4. 1894?{]}}|(be}
\toendnotes[C]{\smallbreak\pagebreak[2]}\Standort{YCGL, MSS 31.}
\physDesc{Brief, 1 Blatt, 4 Seiten, 719 Zeichen (Briefpapier mit Trauerrand)
\newline{}Handschrift: Bleistift, deutsche Kurrent}
\buchAbdrucke{\weitereDrucke{1) Arthur Schnitzler, Richard Beer-Hofmann: \emph{Briefwechsel 1891–1931}. Wien, Zürich: \emph{Europaverlag} 1992, S. 54.} \weitereDrucke{2) Hermann Bahr, Arthur Schnitzler: \emph{Briefwechsel, Aufzeichnungen, Dokumente (1891–1931)}. Göttingen: \emph{Wallstein} 2018.} }\toendnotes[C]{\smallbreak}
\pstart{}{\pb}Lieber Richard,\pend\vspace{0.5em}
\pstart
           Donnerſtag 11 Uhr hol ich Sie ab, wenn’s Ihnen recht iſt. Sie können das Fahren ein
               paar Mal probiren, ohne ſich im geringſten zu verpflichten, und schli{\geminationm}ſten Falls zahlen Sie einen Mit{\pb}gliedsbeitrag auf ¼ Jahr, wodurch Sie zu \strikeout{zu} gar nichts genötigt werden, weder zum Kaufen eines
               Rades, noch zum Weiterverbleiben im Club\orgindex{Vorwaerts@Vorwärts|pwv}. –\pend
           
\pstart
           Bitte ſehr, ſenden Sie dieſen Brief gleich {\pb}an Hermann Bahr\pwindex{Bahr, Hermann 19.07.1863 – 15.01.1934@\textsc{Bahr, Hermann} (19.07.1863 – 15.01.1934), \emph{Schriftsteller/Schriftstellerin, Kritiker/Kritikerin}|pw}, welcher hiedurch unter einem
               gebeten wird, ſich um 11 am Donnerstag bei Ihnen einzufinden, we{\geminationn} er es nicht vorzieht, um 11 Uhr 30 vor
               dem Hauſe \label{K_L00309-1v}\edtext{\textsc{Untere Augartenstraße} 28}{\lemma{\textnormal{\emph{Untere Augartenstraße 28}}}\Cendnote{\textnormal{Sitz der Radfahrunion \emph{Vorwärts}\orgindex{Vorwaerts@Vorwärts|pwk}.}}}\label{K_L00309-1}\oindex{Untere Augartenstrasse@\textbf{Untere Augartenstraße}, \emph{Straße (K.STR)}|pw} auf mich \textsc{resp}. uns zu warten.\pend
           
\pstart
           {\pb}Beifolgend Statuten, von denen 1 Exemplar an \textsc{Bahr}\pwindex{Bahr, Hermann 19.07.1863 – 15.01.1934@\textsc{Bahr, Hermann} (19.07.1863 – 15.01.1934), \emph{Schriftsteller/Schriftstellerin, Kritiker/Kritikerin}|pw}; in dieſem hab ich den § 15 unterſtrichen. Für Sie den § 5. –\pend
           
\pstart
           Herzliche Grüße.{\\[\baselineskip]}\spacefill\mbox{ArthurSch}\pend
           \leftskip=0em{}\selectlanguage{ngerman}\endnumbering\briefempfaengerindex{Beer-Hofmann, Richard@\textsc{Beer-Hofmann, Richard}!zzzSchnitzler, Arthur@\emph{von Arthur Schnitzler}!1894-04-021@{{[}2. 4. 1894?{]}}|)be}\mylabel{L00309h}  \normalsize

\doendnotes{C}
\bigskip
\vfill

\clearpage

\footnotesize

\lohead{\textsc{register}}

% Definiere theindex-Environment komplett neu ohne reledmac
\makeatletter
\renewenvironment{theindex}{%
  \section*{\indexname}%
  \setlength{\parindent}{0pt}%
  \setlength{\parskip}{0pt plus 0.3pt}%
  \let\item\@idxitem
}{%
  \clearpage
}
\makeatother

\IfFileExists{\jobname-pw.ind}{\input{\jobname-pw.ind}}{}

\end{document}

      