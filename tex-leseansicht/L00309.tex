%% latex-leseansicht-vorspann.tex
%% Vorspann für die Leseansicht.
%% Lädt die gemeinsame Datei latex-vorspann.tex mit nicht gesetztem Schalter.

\newif\ifkorrekturansicht
\korrekturansichtfalse

\input{../tex-inputs/latex-vorspann}


\section[Arthur Schnitzler an Richard Beer-Hofmann, {[}2. 4. 1894?{]}]{L00309 Arthur Schnitzler an Richard Beer-Hofmann, {[}2. 4. 1894?{]}}
\nopagebreak\mylabel{L00309v}
\rehead{ }\normalsize\beginnumbering\briefempfaengerindex{Beer-Hofmann, Richard@\textsc{Beer-Hofmann, Richard}!zzzSchnitzler, Arthur@\emph{von Arthur Schnitzler}!1894-04-021@{{[}2. 4. 1894?{]}}|(be}
\toendnotes[C]{\smallbreak\pagebreak[2]}
\correspDesc{Versand  durch Arthur Schnitzler am [2. 4. 1894?] in Wien
\newline{}Erhalt  durch Richard Beer-Hofmann im Zeitraum [2. 4. 1894
                  – 6. 4. 1894?] in Wien}\toendnotes[C]{\smallbreak}
\Standort{YCGL, MSS 31.}
\physDesc{Brief, 1 Blatt, 4 Seiten, 719 Zeichen (Briefpapier mit Trauerrand)
\newline{}Handschrift: Bleistift, deutsche Kurrent}
\buchAbdrucke{\weitereDrucke{1) Arthur Schnitzler, Richard Beer-Hofmann: \emph{Briefwechsel 1891–1931}. Herausgegeben von Konstanze Fliedl. Wien, Zürich: \emph{Europaverlag} 1992, S. 54.} \weitereDrucke{2) Hermann Bahr, Arthur Schnitzler: \emph{Briefwechsel, Aufzeichnungen, Dokumente (1891–1931)}. Herausgegeben von Kurt Ifkovits und Martin Anton Müller. Göttingen: \emph{Wallstein} 2018.} }\toendnotes[C]{\smallbreak}
\pstart{}{\pb}Lieber Richard,\pend\vspace{0.5em}
\pstart
           Donnerſtag 11 Uhr hol ich Sie ab, wenn’s Ihnen recht iſt. Sie können das Fahren ein
               paar Mal probiren, ohne{ }ſich im geringſten zu verpflichten, und schli{\geminationm}ſten Falls zahlen Sie einen Mit{\pb}gliedsbeitrag auf ¼ Jahr, wodurch Sie zu \strikeout{zu} gar nichts genötigt werden, weder zum Kaufen eines
               Rades, noch zum Weiterverbleiben im Club\orgindex{Vorwärts@Vorwärts|pwv}. –\pend
           
\pstart
           Bitte{ }ſehr,{ }ſenden Sie dieſen Brief gleich {\pb}an Hermann Bahr\pwindex{Bahr, Hermann 19.\,7.\,1863 Linz – 15.\,1.\,1934 München@\textsc{Bahr, Hermann} (19.\,7.\,1863 Linz – 15.\,1.\,1934 München), \emph{Schriftsteller, Kritiker}|pw}, welcher hiedurch unter einem
               gebeten wird,{ }ſich um 11 am Donnerstag bei Ihnen einzufinden, we{\geminationn} er es nicht vorzieht, um 11 Uhr 30 vor
               dem Hauſe \label{K_L00309-1v}\edtext{\textsc{Untere Augartenstraße} 28}{\lemma{\textnormal{\emph{Untere Augartenstraße 28}}}\Cendnote{\textnormal{Sitz der Radfahrunion \emph{Vorwärts}\orgindex{Vorwärts@Vorwärts|pwk}.}}}\label{K_L00309-1}\oindex{Wien@\textbf{Wien}!II., Leopoldstadt@\textbf{II., Leopoldstadt}!Untere Augartenstraße@\textbf{Untere Augartenstraße}, \emph{Straße}|pw} auf mich \textsc{resp}. uns zu warten.\pend
           
\pstart
           {\pb}Beifolgend Statuten, von denen 1 Exemplar an \textsc{Bahr}\pwindex{Bahr, Hermann 19.\,7.\,1863 Linz – 15.\,1.\,1934 München@\textsc{Bahr, Hermann} (19.\,7.\,1863 Linz – 15.\,1.\,1934 München), \emph{Schriftsteller, Kritiker}|pw}; in dieſem hab ich den § 15 unterſtrichen. Für Sie den § 5. –\pend
           
\pstart
           Herzliche Grüße.{\\[\baselineskip]}\spacefill\mbox{ArthurSch}\pend
           \leftskip=0em{}\selectlanguage{ngerman}\endnumbering\briefempfaengerindex{Beer-Hofmann, Richard@\textsc{Beer-Hofmann, Richard}!zzzSchnitzler, Arthur@\emph{von Arthur Schnitzler}!1894-04-021@{{[}2. 4. 1894?{]}}|)be}\mylabel{L00309h}  \newcommand{\dateiname}{L00309}\newcommand{\titel}{Arthur Schnitzler an Richard Beer-Hofmann, [2. 4. 1894?]}\newcommand{\editorInnen}{Herausgegeben von Martin Anton Müller}%% latex-leseansicht-abspann.tex
%% Abspann für die Leseansicht.
%% Der Schalter \ifkorrekturansicht ist bereits durch den Vorspann gesetzt.

%% latex-abspann.tex
%% Gemeinsamer Abspann für Korrekturansicht und Leseansicht.
%% Setzt den Schalter \ifkorrekturansicht voraus (gesetzt in den
%% einbindenden Dateien latex-korrekturansicht-abspann.tex bzw.
%% latex-leseansicht-abspann.tex).
%% ---------------------------------------------------------------

\normalsize

% Das esempio-Environment wird nur in der Leseansicht benötigt
\ifkorrekturansicht\else
\newenvironment{esempio}[3]%
{
    \vspace{1.5ex}
    \rlap{\underline{#1}}
    \par
    \setlength{\parindent}{0cm}
    \nopagebreak
    \leftskip=#2cm
    \rightskip=#3cm
}
{
    \par
}
\fi

\doendnotes{C}
\bigskip
\vfill

\clearpage

\footnotesize

\ifkorrekturansicht
  \lohead{\textsc{register}}
\fi

% theindex-Environment neu definieren ohne reledmac
\makeatletter
\renewenvironment{theindex}{%
  \ifkorrekturansicht
    \section*{\indexname}%
  \else
    \subsubsection*{Index der erwähnten Entitäten}%
  \fi
  \setlength{\parindent}{0pt}%
  \setlength{\parskip}{0pt plus 0.3pt}%
  \let\item\@idxitem
}{%
  \ifkorrekturansicht\clearpage\fi
}
\makeatother

\IfFileExists{\jobname-pw.ind}{\input{\jobname-pw.ind}}{}

% Quellenangabe nur in der Leseansicht
\ifkorrekturansicht\else
% Fallback-Definitionen, falls die .tex-Datei \titel etc. nicht gesetzt hat
\providecommand{\titel}{}
\providecommand{\editorInnen}{}
\providecommand{\dateiname}{\jobname}

\vspace{3cm}

\vfill

\footnotesize
\textsc{Quelle}: \titel. Herausgegeben von {\editorInnen}. In: \emph{Arthur Schnitzler: Briefwechsel mit Autorinnen und Autoren}.
 Digitale Edition, https://schnitzler-briefe.acdh.oeaw.ac.at/{\dateiname}.html (Stand \today)
\fi

\end{document}


