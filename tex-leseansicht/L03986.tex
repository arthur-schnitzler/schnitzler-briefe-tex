%% latex-leseansicht-vorspann.tex
%% Vorspann für die Leseansicht.
%% Lädt die gemeinsame Datei latex-vorspann.tex mit nicht gesetztem Schalter.

\newif\ifkorrekturansicht
\korrekturansichtfalse

\input{../tex-inputs/latex-vorspann}


\section[Arthur Schnitzler an Berta Zuckerkandl, 23. 4. 1928]{L03986 Arthur Schnitzler an Berta Zuckerkandl, 23. 4. 1928}
\nopagebreak\mylabel{L03986v}
\rehead{ }\normalsize\beginnumbering\briefempfaengerindex{Zuckerkandl, Berta@\textsc{Zuckerkandl, Berta}!zzzSchnitzler, Arthur@\emph{von Arthur Schnitzler}!1928-04-232@{23. 4. 1928}|(be}
\toendnotes[C]{\smallbreak\pagebreak[2]}
\correspDesc{Versand  durch Arthur Schnitzler am 23. 4. 1928 in Istanbul
\newline{}Erhalt  durch Berta Zuckerkandl im Zeitraum [24. 4. 1928 – 28. 4. 1928?] in Wien}\toendnotes[C]{\smallbreak}
\Standort{Wien, Österreichische Nationalbibliothek, 405/B78/6 LIT MAG.}
\physDesc{Bildpostkarte, 198 Zeichen
\newline{}Handschrift: Bleistift, lateinische Kurrent
\newline{}Versand: Stempel: »\nobreak{}23. 4. 28, \oindex{Istanbul@\textbf{Istanbul}, \emph{Land}|pwk}STAMBOUL\nobreak{}«.  }\toendnotes[C]{\smallbreak}\pstart{}{\pb}Autrich{[}e{]}\oindex{Österreich@\textbf{Österreich}|pw}\pend{}\pstart{}Hofrätin\pend{}\pstart{}Berta Zuckerkandel\pend{}\pstart{}Wien I\oindex{I., Innere Stadt@\textbf{I., Innere Stadt}, \emph{Verwaltungsgebiet}|pw}\pend{}\pstart{}Oppolzerstr 6\oindex{Wien@\textbf{Wien}!I., Innere Stadt@\textbf{I., Innere Stadt}!Oppolzergasse 6@\textbf{Oppolzergasse 6}, \emph{Wohngebäude}|pw}\pend{}{\bigskip}
\pstart
           \centering{}{\pb}\textcolor{gray}{\textbf{\begin{otherlanguage}{french}Constantinople\oindex{Istanbul@\textbf{Istanbul}, \emph{Land}|pw}. Vue panoramique du port et du
                     Bosphore\oindex{Bosporus@\textbf{Bosporus}, \emph{Meerenge}|pw}\end{otherlanguage}}}\pend
           \vspace{1em}
\pstart
           {\pb}Constantinopel\oindex{Istanbul@\textbf{Istanbul}, \emph{Land}|pw}{ }23/4 28\pend
           \vspace{0.5em}
\pstart
           Viele herzliche Grüße von dieser schönen \label{K_L03986-1v}\edtext{Reise}{\lemma{\textnormal{\emph{Reise}}}\Cendnote{\textnormal{Schnitzler, seine Tochter Lili\pwindex{Cappellini, Lili 13.\,9.\,1909 Wien – 26.\,7.\,1928 Venedig@\textsc{Cappellini, Lili} (13.\,9.\,1909 Wien – 26.\,7.\,1928 Venedig)|pwk} und deren Mann Arnoldo Cappellini\pwindex{Cappellini, Arnoldo 10.\,8.\,1889 Venedig – 8.\,5.\,1954 Rom@\textsc{Cappellini, Arnoldo} (10.\,8.\,1889 Venedig – 8.\,5.\,1954 Rom)|pwk} schifften sich am 14. 4. 1928 auf der \emph{Stella d'Italia}\orgindex{Stella d’Italia@Stella d’Italia|pwk} in Triest\oindex{Triest@\textbf{Triest}, \emph{Verwaltungsgebiet}|pwk} ein, um über Korfu\oindex{Korfu@\textbf{Korfu}, \emph{Insel}|pwk} und Athen\oindex{Athen@\textbf{Athen}, \emph{Hauptstadt}|pwk} nach Istanbul\oindex{Istanbul@\textbf{Istanbul}, \emph{Land}|pwk} und über Rhodos\oindex{Rhodos@\textbf{Rhodos}, \emph{Hauptstadt}|pwk} zurück nach Vedenig\oindex{Venedig@\textbf{Venedig}|pwk} zu reisen. Am 3. 5. 1928 kehrte Schnitzler mit dem Flugzeug nach Wien\oindex{Wien@\textbf{Wien}, \emph{Verwaltungsgebiet}|pwk} zurück.}}}\label{K_L03986-1} und auf ein gutes Wiedersehen
               hoffentlich \label{K_L03986-2v}\edtext{Anfang Mai}{\lemma{\textnormal{\emph{Anfang Mai}}}\Cendnote{\textnormal{Zum ersten Wiedersehen nach der Reise kam es laut \emph{Tagebuch}\pwindex{Schnitzler, Arthur 15. 5. 1862 Wien – 21. 10. 1931 ebd.@\textsc{Schnitzler, Arthur} (15. 5. 1862 Wien – 21. 10. 1931 ebd.), \emph{Schriftsteller, Mediziner}!Tagebuch@\strich\emph{Tagebuch}|pwk} am 7. 5. 1928.}}}\label{K_L03986-2}.\pend
           
\pstart
           Ihr getreuer{\\[\baselineskip]}\spacefill\mbox{ArthSchnitzler}\pend
           \leftskip=0em{}\selectlanguage{ngerman}\endnumbering\briefempfaengerindex{Zuckerkandl, Berta@\textsc{Zuckerkandl, Berta}!zzzSchnitzler, Arthur@\emph{von Arthur Schnitzler}!1928-04-232@{23. 4. 1928}|)be}\mylabel{L03986h}
\begin{anhang}
\end{anhang}\newcommand{\dateiname}{L03986}\newcommand{\titel}{Arthur Schnitzler an Berta Zuckerkandl, 23. 4. 1928}\newcommand{\editorInnen}{Herausgegeben von Jahnke, SelmaMüller, Martin Anton}%% latex-leseansicht-abspann.tex
%% Abspann für die Leseansicht.
%% Der Schalter \ifkorrekturansicht ist bereits durch den Vorspann gesetzt.

%% latex-abspann.tex
%% Gemeinsamer Abspann für Korrekturansicht und Leseansicht.
%% Setzt den Schalter \ifkorrekturansicht voraus (gesetzt in den
%% einbindenden Dateien latex-korrekturansicht-abspann.tex bzw.
%% latex-leseansicht-abspann.tex).
%% ---------------------------------------------------------------

\normalsize

% Das esempio-Environment wird nur in der Leseansicht benötigt
\ifkorrekturansicht\else
\newenvironment{esempio}[3]%
{
    \vspace{1.5ex}
    \rlap{\underline{#1}}
    \par
    \setlength{\parindent}{0cm}
    \nopagebreak
    \leftskip=#2cm
    \rightskip=#3cm
}
{
    \par
}
\fi

\doendnotes{C}
\bigskip
\vfill

\clearpage

\footnotesize

\ifkorrekturansicht
  \lohead{\textsc{register}}
\fi

% theindex-Environment neu definieren ohne reledmac
\makeatletter
\renewenvironment{theindex}{%
  \ifkorrekturansicht
    \section*{\indexname}%
  \else
    \subsubsection*{Index der erwähnten Entitäten}%
  \fi
  \setlength{\parindent}{0pt}%
  \setlength{\parskip}{0pt plus 0.3pt}%
  \let\item\@idxitem
}{%
  \ifkorrekturansicht\clearpage\fi
}
\makeatother

\IfFileExists{\jobname-pw.ind}{\input{\jobname-pw.ind}}{}

% Quellenangabe nur in der Leseansicht
\ifkorrekturansicht\else
% Fallback-Definitionen, falls die .tex-Datei \titel etc. nicht gesetzt hat
\providecommand{\titel}{}
\providecommand{\editorInnen}{}
\providecommand{\dateiname}{\jobname}

\vspace{3cm}

\vfill

\footnotesize
\textsc{Quelle}: \titel. Herausgegeben von {\editorInnen}. In: \emph{Arthur Schnitzler: Briefwechsel mit Autorinnen und Autoren}.
 Digitale Edition, https://schnitzler-briefe.acdh.oeaw.ac.at/{\dateiname}.html (Stand \today)
\fi

\end{document}


