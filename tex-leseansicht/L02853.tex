%% latex-leseansicht-vorspann.tex
%% Vorspann für die Leseansicht.
%% Lädt die gemeinsame Datei latex-vorspann.tex mit nicht gesetztem Schalter.

\newif\ifkorrekturansicht
\korrekturansichtfalse

\input{../tex-inputs/latex-vorspann}


\section[ Paul Goldmann an Arthur Schnitzler, 21. 7. {[}1898{]}]{L02853 Paul Goldmann an Arthur Schnitzler,  21. 7. [1898]}
\nopagebreak\mylabel{L02853v}
\rehead{ }\normalsize\beginnumbering\briefempfaengerindex{Schnitzler, Arthur@\textsc{Schnitzler, Arthur}!zzzGoldmann, Paul@\emph{von Paul Goldmann}!1898-07-212@{21. 7. [1898]}|(be}
\toendnotes[C]{\smallbreak\pagebreak[2]}
\correspDesc{Versand  durch Paul Goldmann am 21. 7. [1898] in Shanghai
\newline{}Erhalt  durch Arthur Schnitzler im Zeitraum [22. 7. 1898
                  – 26. 7. 1898?] in Salzburg}\toendnotes[C]{\smallbreak}
\Standort{DLA, A:Schnitzler, HS.NZ85.1.3168.}
\physDesc{Brief, 1 Blatt, 3 Seiten, 1313 Zeichen
\newline{}Handschrift: blaue Tinte, deutsche Kurrent
\newline{}Schnitzler: mit rotem Buntstift eine Unterstreichung }\toendnotes[C]{\smallbreak}
\pstart
           \raggedleft{}{\pb}\textsc{Shanghai\oindex{Shanghai@\textbf{Shanghai}|pw}}, 21. Juli.\pend
           
\pstart\center{}Mein lieber Freund,\pend\vspace{0.5em}
\pstart
           Dieſer Tage empfing ich Deine lieben \label{K_L02853-1v}\edtext{Karten aus Steiermark\oindex{Steiermark@\textbf{Steiermark}, \emph{Land}|pw}}{\lemma{\textnormal{\emph{Karten aus Steiermark}}}\Cendnote{\textnormal{Vom 5. 6. 1898 bis zum 10. 6. 1898 machten Schnitzler und Leopold
                     Kramer\pwindex{Kramer, Leopold 29.\,9.\,1869 Prag – 29.\,10.\,1942 Wien@\textsc{Kramer, Leopold} (29.\,9.\,1869 Prag – 29.\,10.\,1942 Wien), \emph{Theaterleiter, Schauspieler}|pwk} eine gemeinsame Radpartie durch die Steiermark\oindex{Steiermark@\textbf{Steiermark}, \emph{Land}|pwk} bis Kärnten\oindex{Kärnten@\textbf{Kärnten}, \emph{Land}|pwk}. Am 7. 6. 1898 stiegen
                  sie für eine Nacht in Steindorf am
                     Ossiachersee\oindex{Steindorf am Ossiacher See@\textbf{Steindorf am Ossiacher See}, \emph{Verwaltungsgebiet}|pwk} ab, wo Richard\pwindex{Beer-Hofmann, Richard 11.\,7.\,1866 Wien – 26.\,9.\,1945 New York City@\textsc{Beer-Hofmann, Richard} (11.\,7.\,1866 Wien – 26.\,9.\,1945 New York City), \emph{Schriftsteller}|pwk} und Paula Beer-Hofmann\pwindex{Beer-Hofmann, Paula 25.\,2.\,1879 Wien – 30.\,10.\,1939 Zürich@\textsc{Beer-Hofmann, Paula} (25.\,2.\,1879 Wien – 30.\,10.\,1939 Zürich)|pwk} für den Sommer
                  wohnten.}}}\label{K_L02853-1}. Ich{ }ſage Dir, \textsc{Richard\pwindex{Beer-Hofmann, Richard 11.\,7.\,1866 Wien – 26.\,9.\,1945 New York City@\textsc{Beer-Hofmann, Richard} (11.\,7.\,1866 Wien – 26.\,9.\,1945 New York City), \emph{Schriftsteller}|pw}} u.{ }ſeiner Frau\pwindex{Beer-Hofmann, Paula 25.\,2.\,1879 Wien – 30.\,10.\,1939 Zürich@\textsc{Beer-Hofmann, Paula} (25.\,2.\,1879 Wien – 30.\,10.\,1939 Zürich)|pwv} vielen
               Dank, daß Ihr an mich gedacht habt. Auch dem Herrn \textsc{Kramer\pwindex{Kramer, Leopold 29.\,9.\,1869 Prag – 29.\,10.\,1942 Wien@\textsc{Kramer, Leopold} (29.\,9.\,1869 Prag – 29.\,10.\,1942 Wien), \emph{Theaterleiter, Schauspieler}|pw}} bitte ich, zu danken; wenn ich wieder einmal ein Familienblatt\pwindex{der schönen blauen Donau@\emph{An der schönen blauen Donau}|pwv} herausgebe,{ }ſo werde ich alle
               Gedichte von ihm nehmen.\pend
           
\pstart
           Ich leide hier ganz namenlos unter der fürchterlichen Hitze des tropiſchen chin\oindex{China@\textbf{China}|pwv}eſiſchen Sommers. Seit
               Wochen{ }ſchlafe {\pb}ich keine Nacht mehr als zwei bis
               drei Stunden. Es iſt einfach zum Verrücktwerden; und da es im Norden dieſes
               verfluchten Land\oindex{China@\textbf{China}|pwv}es genau{ }ſo
               heiß iſt, wie im Süden, gibt es keine Flucht vor der Hitze. Auch habe ich China\oindex{China@\textbf{China}|pw}{ }ſatt bis oben hinauf. Letzte Woche kam ich
               in einen Chin\oindex{China@\textbf{China}|pwv}eſen-Aufruhr
               hinein und wäre beinahe todt geſchlagen worden. Den{ }ſchlimmſten Theil der Reiſe habe
               ich leider noch vor mir: \textsc{Kiau-tschou\oindex{Kiautschou@\textbf{Kiautschou}|pw}}, wo es noch kein europ\oindex{Europa@\textbf{Europa}|pwv}äiſches Haus gibt, und \textsc{Peking\oindex{Peking@\textbf{Peking}, \emph{Hauptstadt}|pw}}, das gräßlichſte Schmutzneſt der {\pb}Welt, wo man
               die Pocken kriegen kann, wie nichts. Nächſten Montag
               fahre ich nach \textsc{Kiautschou\oindex{Kiautschou@\textbf{Kiautschou}|pw}} (Meine Adreſſe bleibt \textsc{Shanghai\oindex{Shanghai@\textbf{Shanghai}|pw}}). Ich{ }ſage Dir: vierzehn Tage in Florenz\oindex{Florenz@\textbf{Florenz}|pw}{ }ſind beſſer, als{ }ſechs Monate in China\oindex{China@\textbf{China}|pw}. Das
               Heimweh plagt mich unabläſſig, und ich wünſchte, ich wäre{ }ſchon wieder in Europa\oindex{Europa@\textbf{Europa}|pw}.\pend
           
\pstart
           Hoffentlich höre ich bald wieder von Dir. Grüß’ mir Deine Freundin\pwindex{Reinhard, Marie 13.\,3.\,1871 Wien – 18.\,3.\,1899 ebd.@\textsc{Reinhard, Marie} (13.\,3.\,1871 Wien – 18.\,3.\,1899 ebd.), \emph{Gesangspädagogin}|pwv} u.{ }ſei Du{ }ſelbſt von Herzen
               gegrüßt!\pend
           
\pstart
           Dein treuer {\\[\baselineskip]}\spacefill\mbox{Paul Goldmann.}\pend
           \leftskip=0em{}\selectlanguage{ngerman}\endnumbering\briefempfaengerindex{Schnitzler, Arthur@\textsc{Schnitzler, Arthur}!zzzGoldmann, Paul@\emph{von Paul Goldmann}!1898-07-212@{21. 7. [1898]}|)be}\mylabel{L02853h}  \newcommand{\dateiname}{L02853}\newcommand{\titel}{Paul Goldmann an Arthur Schnitzler, 21. 7. [1898]}\newcommand{\editorInnen}{Martin Anton Müller und Laura Untner}%% latex-leseansicht-abspann.tex
%% Abspann für die Leseansicht.
%% Der Schalter \ifkorrekturansicht ist bereits durch den Vorspann gesetzt.

%% latex-abspann.tex
%% Gemeinsamer Abspann für Korrekturansicht und Leseansicht.
%% Setzt den Schalter \ifkorrekturansicht voraus (gesetzt in den
%% einbindenden Dateien latex-korrekturansicht-abspann.tex bzw.
%% latex-leseansicht-abspann.tex).
%% ---------------------------------------------------------------

\normalsize

% Das esempio-Environment wird nur in der Leseansicht benötigt
\ifkorrekturansicht\else
\newenvironment{esempio}[3]%
{
    \vspace{1.5ex}
    \rlap{\underline{#1}}
    \par
    \setlength{\parindent}{0cm}
    \nopagebreak
    \leftskip=#2cm
    \rightskip=#3cm
}
{
    \par
}
\fi

\doendnotes{C}
\bigskip
\vfill

\clearpage

\footnotesize

\ifkorrekturansicht
  \lohead{\textsc{register}}
\fi

% theindex-Environment neu definieren ohne reledmac
\makeatletter
\renewenvironment{theindex}{%
  \ifkorrekturansicht
    \section*{\indexname}%
  \else
    \subsubsection*{Index der erwähnten Entitäten}%
  \fi
  \setlength{\parindent}{0pt}%
  \setlength{\parskip}{0pt plus 0.3pt}%
  \let\item\@idxitem
}{%
  \ifkorrekturansicht\clearpage\fi
}
\makeatother

\IfFileExists{\jobname-pw.ind}{\input{\jobname-pw.ind}}{}

% Quellenangabe nur in der Leseansicht
\ifkorrekturansicht\else
% Fallback-Definitionen, falls die .tex-Datei \titel etc. nicht gesetzt hat
\providecommand{\titel}{}
\providecommand{\editorInnen}{}
\providecommand{\dateiname}{\jobname}

\vspace{3cm}

\vfill

\footnotesize
\textsc{Quelle}: \titel. Herausgegeben von {\editorInnen}. In: \emph{Arthur Schnitzler: Briefwechsel mit Autorinnen und Autoren}.
 Digitale Edition, https://schnitzler-briefe.acdh.oeaw.ac.at/{\dateiname}.html (Stand \today)
\fi

\end{document}


