%% latex-korrekturansicht-vorspann.tex
%% Vorspann für die Korrekturansicht.
%% Lädt die gemeinsame Datei latex-vorspann.tex mit gesetztem Schalter.

\newif\ifkorrekturansicht
\korrekturansichttrue

\input{../tex-inputs/latex-vorspann}


\section[ Paul Goldmann an Arthur Schnitzler, 21. 7. {[}1898{]}]{L02853 Paul Goldmann an Arthur Schnitzler, 21. 7. {[}1898{]}}
\nopagebreak\mylabel{L02853v}
\rehead{ }\normalsize\beginnumbering\briefempfaengerindex{Schnitzler, Arthur@\textsc{Schnitzler, Arthur}!zzzGoldmann, Paul@\emph{von Paul Goldmann}!1898-07-211@{21. 7. {[}1898{]}}|(be}
\toendnotes[C]{\smallbreak\pagebreak[2]}\Standort{DLA, A:Schnitzler, HS.NZ85.1.3168.}
\physDesc{Brief, 1 Blatt, 3 Seiten, 1313 Zeichen
\newline{}Handschrift: blaue Tinte, deutsche Kurrent
\newline{}Schnitzler: mit rotem Buntstift eine Unterstreichung }\toendnotes[C]{\smallbreak}
\pstart
           \raggedleft{}{\pb}\textsc{Shanghai\oindex{Shanghai@\textbf{Shanghai}, \emph{P.PPLA}|pw}}, 21. Juli.\pend
           
\pstart\center{}Mein lieber Freund,\pend\vspace{0.5em}
\pstart
           Dieſer Tage empfing ich Deine lieben \label{K_L02853-1v}\edtext{Karten aus Steiermark\oindex{Steiermark@\textbf{Steiermark}, \emph{A.ADM1}|pw}}{\lemma{\textnormal{\emph{Karten aus Steiermark}}}\Cendnote{\textnormal{Vom 5. 6. 1898 bis zum 10. 6. 1898 machten Schnitzler und Leopold
                     Kramer\pwindex{Kramer, Leopold 29.09.1869 – 29.10.1942@\textsc{Kramer, Leopold} (29.09.1869 – 29.10.1942), \emph{Theaterleiter/Theaterleiterin, Schauspieler/Schauspielerin}|pwk} eine gemeinsame Radpartie durch die Steiermark\oindex{Steiermark@\textbf{Steiermark}, \emph{A.ADM1}|pwk} bis Kärnten\oindex{Kaernten@\textbf{Kärnten}, \emph{A.ADM1}|pwk}. Am 7. 6. 1898 stiegen
                  sie für eine Nacht in Steindorf am
                     Ossiachersee\oindex{Steindorf am Ossiacher See@\textbf{Steindorf am Ossiacher See}, \emph{A.ADM3}|pwk} ab, wo Richard\pwindex{Beer-Hofmann, Richard 1866-07-11 – 1945-09-26@\textsc{Beer-Hofmann, Richard} (1866-07-11 – 1945-09-26), \emph{Schriftsteller/Schriftstellerin}|pwk} und Paula Beer-Hofmann\pwindex{Beer-Hofmann, Paula 25.02.1879 – 30.10.1939@\textsc{Beer-Hofmann, Paula} (25.02.1879 – 30.10.1939)|pwk} für den Sommer
                  wohnten.}}}\label{K_L02853-1}. Ich ſage Dir, \textsc{Richard\pwindex{Beer-Hofmann, Richard 1866-07-11 – 1945-09-26@\textsc{Beer-Hofmann, Richard} (1866-07-11 – 1945-09-26), \emph{Schriftsteller/Schriftstellerin}|pw}} u. ſeiner Frau\pwindex{Beer-Hofmann, Paula 25.02.1879 – 30.10.1939@\textsc{Beer-Hofmann, Paula} (25.02.1879 – 30.10.1939)|pwv} vielen
               Dank, daß Ihr an mich gedacht habt. Auch dem Herrn \textsc{Kramer\pwindex{Kramer, Leopold 29.09.1869 – 29.10.1942@\textsc{Kramer, Leopold} (29.09.1869 – 29.10.1942), \emph{Theaterleiter/Theaterleiterin, Schauspieler/Schauspielerin}|pw}} bitte ich, zu danken; wenn ich wieder einmal ein Familienblatt\pwindex{der schoenen blauen Donau@\emph{An der schönen blauen Donau}|pwv} herausgebe, ſo werde ich alle
               Gedichte von ihm nehmen.\pend
           
\pstart
           Ich leide hier ganz namenlos unter der fürchterlichen Hitze des tropiſchen chin\oindex{China@\textbf{China}, \emph{A.PCLI}|pwv}eſiſchen Sommers. Seit
               Wochen ſchlafe {\pb}ich keine Nacht mehr als zwei bis
               drei Stunden. Es iſt einfach zum Verrücktwerden; und da es im Norden dieſes
               verfluchten Land\oindex{China@\textbf{China}, \emph{A.PCLI}|pwv}es genau ſo
               heiß iſt, wie im Süden, gibt es keine Flucht vor der Hitze. Auch habe ich China\oindex{China@\textbf{China}, \emph{A.PCLI}|pw} ſatt bis oben hinauf. Letzte Woche kam ich
               in einen Chin\oindex{China@\textbf{China}, \emph{A.PCLI}|pwv}eſen-Aufruhr
               hinein und wäre beinahe todt geſchlagen worden. Den ſchlimmſten Theil der Reiſe habe
               ich leider noch vor mir: \textsc{Kiau-tschou\oindex{Kiautschou@\textbf{Kiautschou}, \emph{Region}|pw}}, wo es noch kein europ\oindex{Europa@\textbf{Europa}, \emph{Kontinent (A.KNT)}|pwv}äiſches Haus gibt, und \textsc{Peking\oindex{Peking@\textbf{Peking}, \emph{P.PPLC}|pw}}, das gräßlichſte Schmutzneſt der {\pb}Welt, wo man
               die Pocken kriegen kann, wie nichts. Nächſten Montag
               fahre ich nach \textsc{Kiautschou\oindex{Kiautschou@\textbf{Kiautschou}, \emph{Region}|pw}} (Meine Adreſſe bleibt \textsc{Shanghai\oindex{Shanghai@\textbf{Shanghai}, \emph{P.PPLA}|pw}}). Ich ſage Dir: vierzehn Tage in Florenz\oindex{Florenz@\textbf{Florenz}, \emph{P.PPLA}|pw}
               ſind beſſer, als ſechs Monate in China\oindex{China@\textbf{China}, \emph{A.PCLI}|pw}. Das
               Heimweh plagt mich unabläſſig, und ich wünſchte, ich wäre ſchon wieder in Europa\oindex{Europa@\textbf{Europa}, \emph{Kontinent (A.KNT)}|pw}.\pend
           
\pstart
           Hoffentlich höre ich bald wieder von Dir. Grüß’ mir Deine Freundin\pwindex{Reinhard, Marie 1871-03-13 – 1899-03-18@\textsc{Reinhard, Marie} (1871-03-13 – 1899-03-18), \emph{Gesangspädagoge/Gesangspädagogin}|pwv} u. ſei Du ſelbſt von Herzen
               gegrüßt!\pend
           
\pstart
           Dein treuer {\\[\baselineskip]}\spacefill\mbox{Paul Goldmann.}\pend
           \leftskip=0em{}\selectlanguage{ngerman}\endnumbering\briefempfaengerindex{Schnitzler, Arthur@\textsc{Schnitzler, Arthur}!zzzGoldmann, Paul@\emph{von Paul Goldmann}!1898-07-211@{21. 7. {[}1898{]}}|)be}\mylabel{L02853h}  \normalsize

\doendnotes{C}
\bigskip
\vfill

\clearpage

\footnotesize

\lohead{\textsc{register}}

% Definiere theindex-Environment komplett neu ohne reledmac
\makeatletter
\renewenvironment{theindex}{%
  \section*{\indexname}%
  \setlength{\parindent}{0pt}%
  \setlength{\parskip}{0pt plus 0.3pt}%
  \let\item\@idxitem
}{%
  \clearpage
}
\makeatother

\IfFileExists{\jobname-pw.ind}{\input{\jobname-pw.ind}}{}

\end{document}

      