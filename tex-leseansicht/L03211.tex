%% latex-korrekturansicht-vorspann.tex
%% Vorspann für die Korrekturansicht.
%% Lädt die gemeinsame Datei latex-vorspann.tex mit gesetztem Schalter.

\newif\ifkorrekturansicht
\korrekturansichttrue

\input{../tex-inputs/latex-vorspann}


\section[ Paul Goldmann an Arthur Schnitzler, 16. 6. {[}1902{]}]{L03211 Paul Goldmann an Arthur Schnitzler, 16. 6. {[}1902{]}}
\nopagebreak\mylabel{L03211v}
\rehead{ }\normalsize\beginnumbering\briefempfaengerindex{Schnitzler, Arthur@\textsc{Schnitzler, Arthur}!zzzGoldmann, Paul@\emph{von Paul Goldmann}!1902-06-162@{16. 6. {[}1902{]}}|(be}
\toendnotes[C]{\smallbreak\pagebreak[2]}\Standort{DLA, A:Schnitzler, HS.NZ85.1.3172.}
\physDesc{Brief, 1 Blatt, 4 Seiten, 1909 Zeichen
\newline{}Handschrift: blaue Tinte, deutsche Kurrent
\newline{}Schnitzler: 1) mit Bleistift das Jahr »902« vermerkt  2) mit rotem Buntstift eine Unterstreichung}\toendnotes[C]{\smallbreak}
\pstart
           \raggedleft{}{\pb}\textcolor{gray}{\textbf{DESSAUERSTRASSE 19}}\oindex{Dessauer Strasse@\textbf{Dessauer Straße}, \emph{Straße (K.STR)}|pw}\pend
           
\pstart
           Berlin\oindex{Berlin@\textbf{Berlin}, \emph{P.PPLC}|pw}, 16. Juni.\pend
           
\pstart\center{}Mein lieber Freund,\pend\vspace{0.5em}
\pstart
           Ich habe mich ſehr gefreut, wieder von Dir zu hören. Die \label{K_L03211-1v}\edtext{Budapeſt\oindex{Budapest@\textbf{Budapest}, \emph{P.PPLC}|pw}er Reiſe}{\lemma{\textnormal{\emph{Budapeſter Reiſe}}}\Cendnote{\textnormal{Auf Otto Brahms\pwindex{Brahm, Otto 05.02.1856 – 28.11.1912@\textsc{Brahm, Otto} (05.02.1856 – 28.11.1912), \emph{Theaterleiter/Theaterleiterin, Regisseur/Regisseurin}|pwk}
                  Einladung hin war Schnitzler am 7. 6. 1902 und 8. 6. 1902 in Budapest\oindex{Budapest@\textbf{Budapest}, \emph{P.PPLC}|pwk} gewesen, wo im Lustspielhaus\oindex{Lustspieltheater [Budapest]@\textbf{Lustspieltheater [Budapest]}, \emph{Theater (K.THE)}|pwk} die \emph{Lebendigen Stunden}\pwindex{Lebendige Stunden. Vier Einakter@\emph{Lebendige Stunden. Vier Einakter}|pwk} gegeben wurden. Vgl. \emph{Der
                        Briefwechsel Arthur Schnitzler – Otto Brahm}. Vollständige Ausgabe.
                     Herausgegeben, eingeleitet und erläutert von Oskar Seidlin.
                     Tübingen: \emph{Niemeyer}{ }1975, S. 123.}}}\label{K_L03211-1} muß recht intereſſant
               geweſen ſein. Hat ſich \label{K_L03211-2v}\edtext{\textsc{Brahm\pwindex{Brahm, Otto 05.02.1856 – 28.11.1912@\textsc{Brahm, Otto} (05.02.1856 – 28.11.1912), \emph{Theaterleiter/Theaterleiterin, Regisseur/Regisseurin}|pw}} über die »\textsc{Beatrice\pwindex{Schleier der Beatrice. Schauspiel in fuenf Akten@\emph{Der Schleier der Beatrice. Schauspiel in fünf Akten}|pw}}« entſchieden}{\lemma{\textnormal{\emph{Brahm … entſchieden}}}\Cendnote{\textnormal{\emph{Der Schleier der Beatrice}\pwindex{Schleier der Beatrice. Schauspiel in fuenf Akten@\emph{Der Schleier der Beatrice. Schauspiel in fünf Akten}|pwk} wurde von Otto Brahm\pwindex{Brahm, Otto 05.02.1856 – 28.11.1912@\textsc{Brahm, Otto} (05.02.1856 – 28.11.1912), \emph{Theaterleiter/Theaterleiterin, Regisseur/Regisseurin}|pwk} für das Deutsche Theater Berlin\oindex{Deutsches Theater Berlin@\textbf{Deutsches Theater Berlin}, \emph{Theater (K.THE)}|pwk} angenommen und feierte dort am 7. 3. 1903
                  Premiere.}}}\label{K_L03211-2}? Wenn er die \label{K_L03211-3v}\edtext{»\textsc{Monna Vanna\pwindex{Monna Vanna. Schauspiel in drei Akten@\emph{Monna Vanna. Schauspiel in drei Akten}|pw}}« von \textsc{Maeterlinck\pwindex{Maeterlinck, Maurice 29.08.1862 – 06.05.1949@\textsc{Maeterlinck, Maurice} (29.08.1862 – 06.05.1949), \emph{Schriftsteller/Schriftstellerin}|pw}} gibt}{\lemma{\textnormal{\emph{»Monna … gibt}}}\Cendnote{\textnormal{Maurice Maeterlincks\pwindex{Maeterlinck, Maurice 29.08.1862 – 06.05.1949@\textsc{Maeterlinck, Maurice} (29.08.1862 – 06.05.1949), \emph{Schriftsteller/Schriftstellerin}|pwk}{ }\emph{Monna Vanna. Pièce en trois actes}\pwindex{Monna Vanna. Piece en trois actes@\emph{Monna Vanna. Pièce en trois actes}|pwk} in der Übersetzung\pwindex{Monna Vanna. Schauspiel in drei Akten@\emph{Monna Vanna. Schauspiel in drei Akten}|pwkv} von 
                  Friedrich von Oppeln-Bronikowski\pwindex{Oppeln-Bronikowski, Friedrich von 07.04.1873 – 09.10.1936@\textsc{Oppeln-Bronikowski, Friedrich von} (07.04.1873 – 09.10.1936), \emph{Schriftsteller/Schriftstellerin, Übersetzer/Übersetzerin}|pwk}
                  wurde ab dem 12. 10. 1902 über hundertmal im Deutschen
                        Theater Berlin\oindex{Deutsches Theater Berlin@\textbf{Deutsches Theater Berlin}, \emph{Theater (K.THE)}|pwk} aufgeführt.  Das Stück und \emph{Der Schleier der Beatrice}\pwindex{Schleier der Beatrice. Schauspiel in fuenf Akten@\emph{Der Schleier der Beatrice. Schauspiel in fünf Akten}|pwk} haben offensichtliche Parallelen, vor allem
                  im Ort der Handlung und in der zentralen Figur einer Frau zwischen zwei Männern.
                  Obzwar Schnitzlers{ }Stück\pwindex{Schleier der Beatrice. Schauspiel in fuenf Akten@\emph{Der Schleier der Beatrice. Schauspiel in fünf Akten}|pwkv} früher erschienen war, dauerte es
                  Monate, bis man sich einig war, ob es auch am \emph{Deutschen Theater}\orgindex{Deutsches Theater Berlin@Deutsches Theater Berlin|pwk} gegeben werden sollte, nachdem dort Maurice Maeterlincks\pwindex{Maeterlinck, Maurice 29.08.1862 – 06.05.1949@\textsc{Maeterlinck, Maurice} (29.08.1862 – 06.05.1949), \emph{Schriftsteller/Schriftstellerin}|pwk}{ }Stück\pwindex{Monna Vanna. Schauspiel in drei Akten@\emph{Monna Vanna. Schauspiel in drei Akten}|pwkv} am Spielplan
                  gestanden hatte. Schnitzler besuchte die
                  Inszenierung am 14. 10. 1902. 
               }}}\label{K_L03211-3}, muß er auch die »\textsc{Beatrice\pwindex{Schleier der Beatrice. Schauspiel in fuenf Akten@\emph{Der Schleier der Beatrice. Schauspiel in fünf Akten}|pw}}« geben können. Dein \label{K_L03211-4v}\edtext{Stück\pwindex{einsame Weg. Schauspiel in fuenf Akten@\emph{Der einsame Weg. Schauspiel in fünf Akten}|pwv}}{\lemma{\textnormal{\emph{Stück}}}\Cendnote{\textnormal{Schnitzler hatte die Konzeption für \emph{Der einsame Weg}\pwindex{einsame Weg. Schauspiel in fuenf Akten@\emph{Der einsame Weg. Schauspiel in fünf Akten}|pwk} am 2. 6. 1902
                  abgeschlossen und begann es am 9. 8. 1902 zu schreiben.}}}\label{K_L03211-4} laß’ nur ruhig noch warten, bis Du
               ordentlich Luſt bekommſt, es zu ſchreiben. Daß Du kurze \label{K_L03211-5v}\edtext{Geſchichten\pwindex{griechische Taenzerin. Novellette@\emph{Die griechische Tänzerin. Novellette}|pwv}\pwindex{Weissagung@\emph{Die Weissagung}|pwv}}{\lemma{\textnormal{\emph{Geſchichten}}}\Cendnote{\textnormal{Bezug auf \emph{Die griechische Tänzerin}\pwindex{griechische Taenzerin. Novellette@\emph{Die griechische Tänzerin. Novellette}|pwk} und \emph{Die Weissagung}\pwindex{Weissagung@\emph{Die Weissagung}|pwk}, die Schnitzler am 7. 6. 1902 neu begonnen hatte}}}\label{K_L03211-5} ſchreibſt, gefällt mir ſehr.
               Ich glaube, auf dieſem Gebiete iſt viel für Dich zu holen.\pend
           
\pstart
           Daß ſich der Vater\pwindex{Gussmann, Rudolf 05.03.1842 – 24.01.1921@\textsc{Gussmann, Rudolf} (05.03.1842 – 24.01.1921), \emph{Handelsagent/Handelsagentin}|pwv} der Mädels\pwindex{Schnitzler, Olga 17.01.1882 – 13.01.1970@\textsc{Schnitzler, Olga} (17.01.1882 – 13.01.1970), \emph{Schauspieler/Schauspielerin, Sänger/Sängerin}|pwv}\pwindex{Steinrueck, Elisabeth 19.11.1885 – 07.04.1920@\textsc{Steinrück, Elisabeth} (19.11.1885 – 07.04.1920)|pwv}{ }{\pb}\label{K_L03211-6v}\edtext{verheirathet}{\lemma{\textnormal{\emph{verheirathet}}}\Cendnote{\textnormal{Amalia Gussmann\pwindex{Gussmann, Amalia 1855-04-15 – 1899-11-14@\textsc{Gussmann, Amalia} (1855-04-15 – 1899-11-14)|pwk}, die Mutter von Olga\pwindex{Schnitzler, Olga 17.01.1882 – 13.01.1970@\textsc{Schnitzler, Olga} (17.01.1882 – 13.01.1970), \emph{Schauspieler/Schauspielerin, Sänger/Sängerin}|pwk} und Elisabeth\pwindex{Steinrueck, Elisabeth 19.11.1885 – 07.04.1920@\textsc{Steinrück, Elisabeth} (19.11.1885 – 07.04.1920)|pwk}, war am 14. 11. 1899 nach
                  achtzehnjähriger Ehe verstorben. Rudolf
                     Gussmanns\pwindex{Gussmann, Rudolf 05.03.1842 – 24.01.1921@\textsc{Gussmann, Rudolf} (05.03.1842 – 24.01.1921), \emph{Handelsagent/Handelsagentin}|pwk} zweite Frau war Johanna
                     Steiner\pwindex{Gussmann, Johanna 1870-05-25 – 1905-06-30@\textsc{Gussmann, Johanna} (1870-05-25 – 1905-06-30)|pwk}. Die beiden heirateten am 8. 4. 1902 in Wien\oindex{Wien@\textbf{Wien}, \emph{A.ADM2}|pwk}. Die Ehe endete bald durch ihren Tod am
                     30. 6. 1905.}}}\label{K_L03211-6} hat, iſt zugleich komiſch und gemein. Dieſer
               Hundsfott! Wie hat ſich die \label{K_L03211-7v}\edtext{Geſchichte
               mit dem Advokaten}{\lemma{\textnormal{\emph{Geſchichte … Advokaten}}}\Cendnote{\textnormal{Womöglich handelte es
                  sich um eine Erbschaftsangelegenheit, die durch die Eheschließung des Vaters\pwindex{Gussmann, Rudolf 05.03.1842 – 24.01.1921@\textsc{Gussmann, Rudolf} (05.03.1842 – 24.01.1921), \emph{Handelsagent/Handelsagentin}|pwkv} dringlich wurde.
                  Jedenfalls wurden die beiden Schwestern\pwindex{Schnitzler, Olga 17.01.1882 – 13.01.1970@\textsc{Schnitzler, Olga} (17.01.1882 – 13.01.1970), \emph{Schauspieler/Schauspielerin, Sänger/Sängerin}|pwk}\pwindex{Steinrueck, Elisabeth 19.11.1885 – 07.04.1920@\textsc{Steinrück, Elisabeth} (19.11.1885 – 07.04.1920)|pwk} am 14. 5. 1902 kurzfristig verhaftet.}}}\label{K_L03211-7} abgewickelt?\pend
           
\pstart
           Was \textsc{Liesl\pwindex{Steinrueck, Elisabeth 19.11.1885 – 07.04.1920@\textsc{Steinrück, Elisabeth} (19.11.1885 – 07.04.1920)|pw}} anlangt, ſo bitte ich Dich, einmal mit einem Donnerwetter dazwiſchenzufahren.
               Den an mich gerichteten \label{K_L03211-8v}\edtext{Brief von \textsc{Löwenfeld\pwindex{Loewenfeld, Raphael 11.02.1854 – 28.12.1910@\textsc{Löwenfeld, Raphael} (11.02.1854 – 28.12.1910), \emph{Theaterleiter/Theaterleiterin}|pw}}}{\lemma{\textnormal{\emph{Brief von Löwenfeld}}}\Cendnote{\textnormal{Heute findet sich dieser in der Korrespondenz zwischen Goldmann\pwindex{Goldmann, Paul 31.01.1865 – 25.09.1935@\textsc{Goldmann, Paul} (31.01.1865 – 25.09.1935), \emph{Schriftsteller/Schriftstellerin, Journalist/Journalistin}|pwk} und Elisabeth
                        Gussmann\pwindex{Steinrueck, Elisabeth 19.11.1885 – 07.04.1920@\textsc{Steinrück, Elisabeth} (19.11.1885 – 07.04.1920)|pwk} verwahrt: \emph{DLA}, HS.1985.1.5246. Elisabeth Gussmann\pwindex{Steinrueck, Elisabeth 19.11.1885 – 07.04.1920@\textsc{Steinrück, Elisabeth} (19.11.1885 – 07.04.1920)|pwk} schloss am 2. 8. 1902 einen Vertrag mit dem \emph{Schiller-Theater}\orgindex{Schiller-Theater@Schiller-Theater|pwk} ab. Das Beschäftigungsverhältnis dauerte
                  von 1. 9. 1902 bis 30. 6. 1907.}}}\label{K_L03211-8} haſt Du wohl geleſen? Ich ſchließe daraus, daß eine
               Möglichkeit des Engagements am Schillertheater\orgindex{Schiller-Theater@Schiller-Theater|pw}
               beſteht, wenn man nur ein wenig nachhilft. Ich bin gern bereit, nachzuhelfen\strikeout{,} und den perſönlichen Beſuch zu machen, zu dem er mich
               auffordert. Aber vorher muß ich wiſſen, ob \textsc{Liesl\pwindex{Steinrueck, Elisabeth 19.11.1885 – 07.04.1920@\textsc{Steinrück, Elisabeth} (19.11.1885 – 07.04.1920)|pw}} ihm geſchrieben hat, nachdem ſie mir bereits {\pb}einmal \strikeout{geſ} vorgeſchwindelt hat, ſie habe ihm
               geſchrieben, ohne es gethan zu haben. Ich warte alſo auf Antwort und bekomme keine.
               Veranlaſſe doch, \strikeout{\textcolor{gray}{×}\-\textcolor{gray}{×}\-\textcolor{gray}{×}\-\textcolor{gray}{×}} daß die junge \strikeout{Dam\textcolor{gray}{e}\pwindex{Steinrueck, Elisabeth 19.11.1885 – 07.04.1920@\textsc{Steinrück, Elisabeth} (19.11.1885 – 07.04.1920)|pw}}{ }Dame\pwindex{Steinrueck, Elisabeth 19.11.1885 – 07.04.1920@\textsc{Steinrück, Elisabeth} (19.11.1885 – 07.04.1920)|pw} ſich aufrafft und zur Feder greift, und
               ſage ihr, bitte, in meinem Namen, daß ich wüthend bin und daß man mit ſolch’ einer
               verfluchten Schlamperei keine Engagements bekommt!\pend
           
\pstart
           Grüße \textsc{Olga\pwindex{Schnitzler, Olga 17.01.1882 – 13.01.1970@\textsc{Schnitzler, Olga} (17.01.1882 – 13.01.1970), \emph{Schauspieler/Schauspielerin, Sänger/Sängerin}|pw}} recht herzlich. Ich hoffe, ſie übt die \textsc{Löwe\pwindex{Loewe, Carl 30.11.1796 – 20.04.1869@\textsc{Loewe, Carl} (30.11.1796 – 20.04.1869), \emph{Sänger/Sängerin, Komponist/Komponistin}|pw}}’ſchen Balladen\pwindex{Tom der Reimer@\emph{Tom der Reimer}|pwv}\pwindex{Heinrich der Vogler@\emph{Heinrich der Vogler}|pwv} (Tom der Reimer\pwindex{Tom der Reimer@\emph{Tom der Reimer}|pw}, Heinrich der Vogler\pwindex{Heinrich der Vogler@\emph{Heinrich der Vogler}|pw}). Wenn ich nach Wien\oindex{Wien@\textbf{Wien}, \emph{A.ADM2}|pw} komme, will ich ſie vorgeſungen haben.\pend
           
\pstart
           {\pb}Meine Pläne bleiben einſtweilen die alten: \label{K_L03211-9v}\edtext{Zwiſchen 20.
               u. 25. Juli{ }Wien\oindex{Wien@\textbf{Wien}, \emph{A.ADM2}|pw}, dann \textsc{Trafoi\oindex{Trafoi@\textbf{Trafoi}, \emph{P.PPL}|pw}}}{\lemma{\textnormal{\emph{Zwiſchen … Trafoi}}}\Cendnote{\textnormal{Dazu kam es nicht, siehe Paul Goldmann an Arthur Schnitzler, 25. 7. [1902].
               }}}\label{K_L03211-9}. Von \label{K_L03211-10v}\edtext{Fräulein \textsc{F.\pwindex{Goldmann, Eva Marie 27.10.1877 – 02.11.1937@\textsc{Goldmann, Eva Marie} (27.10.1877 – 02.11.1937)|pwv}}}{\lemma{\textnormal{\emph{Fräulein F.}}}\Cendnote{\textnormal{womöglich Goldmanns\pwindex{Goldmann, Paul 31.01.1865 – 25.09.1935@\textsc{Goldmann, Paul} (31.01.1865 – 25.09.1935), \emph{Schriftsteller/Schriftstellerin, Journalist/Journalistin}|pwk} nachmalige Ehefrau Eva Marie Fränkel\pwindex{Goldmann, Eva Marie 27.10.1877 – 02.11.1937@\textsc{Goldmann, Eva Marie} (27.10.1877 – 02.11.1937)|pwk}, die aber in der Zeit bis zur
                  Eheschließung 1908 noch mit einem Herrn Kobler\pwindex{Kobler @\textsc{Kobler}|pwk} verheiratet war.}}}\label{K_L03211-10} erhalte ich hier und da
               einen Brief. Aber das Schreiben iſt eine dumme Sache. Die Fäden ſind abgeriſſen. Sie
               ſchreibt mir übrigens, daß ſie öfter mit \textsc{Salten\pwindex{Salten, Felix 06.09.1869 – 08.10.1945@\textsc{Salten, Felix} (06.09.1869 – 08.10.1945), \emph{Schriftsteller/Schriftstellerin, Journalist/Journalistin, Chefredakteur/Chefredakteurin}|pw}} zuſammen iſt.\pend
           
\pstart
           Schreib’ mir bald wieder und ſei vielmals und von Herzen gegrüßt! {\\[\baselineskip]}Dein {\\[\baselineskip]}\spacefill\mbox{Paul Goldmnn}\pend
           \leftskip=0em{}\selectlanguage{ngerman}\endnumbering\briefempfaengerindex{Schnitzler, Arthur@\textsc{Schnitzler, Arthur}!zzzGoldmann, Paul@\emph{von Paul Goldmann}!1902-06-162@{16. 6. {[}1902{]}}|)be}\mylabel{L03211h}  \normalsize

\doendnotes{C}
\bigskip
\vfill

\clearpage

\footnotesize

\lohead{\textsc{register}}

% Definiere theindex-Environment komplett neu ohne reledmac
\makeatletter
\renewenvironment{theindex}{%
  \section*{\indexname}%
  \setlength{\parindent}{0pt}%
  \setlength{\parskip}{0pt plus 0.3pt}%
  \let\item\@idxitem
}{%
  \clearpage
}
\makeatother

\IfFileExists{\jobname-pw.ind}{\input{\jobname-pw.ind}}{}

\end{document}

      