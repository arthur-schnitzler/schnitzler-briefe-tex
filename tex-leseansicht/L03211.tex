%% latex-leseansicht-vorspann.tex
%% Vorspann für die Leseansicht.
%% Lädt die gemeinsame Datei latex-vorspann.tex mit nicht gesetztem Schalter.

\newif\ifkorrekturansicht
\korrekturansichtfalse

\input{../tex-inputs/latex-vorspann}


         
         \renewcommand{\erwaehntePersonen}{Personen: Otto Brahm, Paul Goldmann, Eva Marie Goldmann, Rudolf Gussmann, Amalia Gussmann, Johanna Gussmann,  Kobler, Carl Loewe, Raphael Löwenfeld, Maurice Maeterlinck, Felix Salten, Olga Schnitzler, Elisabeth Steinrück}
         \renewcommand{\erwaehnteInstitutionen}{Institutionen: Schiller-Theater}
         \renewcommand{\erwaehnteOrte}{Orte: Berlin, Budapest, Dessauer Straße, Deutsches Theater Berlin, Lustspieltheater (Budapest), Trafoi, Wien}
         \renewcommand{\erwaehnteWerke}{Werke: Der Schleier der Beatrice. Schauspiel in fünf Akten, Der einsame Weg. Schauspiel in fünf Akten, Die Weissagung, Die griechische Tänzerin. Novellette, Heinrich der Vogler, Lebendige Stunden. Vier Einakter, Monna Vanna. Schauspiel in drei Akten, Tom der Reimer}
               \section[ Paul Goldmann an Arthur Schnitzler, 16. 6. {[}1902{]}]{ Paul Goldmann an Arthur Schnitzler, 16. 6. {[}1902{]}}\nopagebreak\mylabel{v}\rehead{ }\begin{ledgroupsized}[t]{13cm}\normalsize\beginnumbering \toendnotes[C]{\smallbreak\pagebreak[2]} \Standort{DLA, A:Schnitzler, HS.NZ85.1.3172.}
\physDesc{Brief, 1 Blatt, 4 Seiten, 1909 Zeichen
\newline{}Handschrift: blaue Tinte, deutsche Kurrent
\newline{}Schnitzler: 1) mit Bleistift das Jahr »902« vermerkt  2) mit rotem Buntstift eine Unterstreichung}\toendnotes[C]{\smallbreak}\pstart
           \noindent{}\raggedleft{}{\pb}\textcolor{gray}{\textbf{DESSAUERSTRASSE 19}}\oindex{Dessauer Strasse@\textbf{Dessauer Straße}|pw}\pend
           \pstart
           Berlin\oindex{Berlin@\textbf{Berlin}|pw}, 16. Juni.\pend
           \pstart\center{}Mein lieber Freund,\pend\pstart
           Ich habe mich ſehr gefreut, wieder von Dir zu hören. Die \label{K_L03211-1v}\edtext{Budapeſt\oindex{Budapest@\textbf{Budapest}|pw}er Reiſe}{\lemma{\textnormal{\emph{Budapeſter Reiſe}}}\Cendnote{\textnormal{Auf Otto Brahm\pwindex{Brahm, Otto 05.02.1856 – 28.11.1912@\textsc{Brahm, Otto} (05.02.1856 – 28.11.1912), \emph{Theaterleiter, Regisseur}|pwk}s
                  Einladung hin war Schnitzler\pwindex{Schnitzler, Arthur 15.05.1862 – 21.10.1931@\textsc{Schnitzler, Arthur} (15.05.1862 – 21.10.1931), \emph{Schriftsteller, Mediziner}|pwk} am 7. 6. 1902 und 8. 6. 1902 in Budapest\oindex{Budapest@\textbf{Budapest}|pwk} gewesen, wo im Lustspielhaus\oindex{Lustspieltheater (Budapest)@\textbf{Lustspieltheater (Budapest)}|pwk} die \emph{Lebendigen Stunden}\pwindex{Schnitzler, Arthur 15.05.1862 – 21.10.1931@\textsc{Schnitzler, Arthur} (15.05.1862 – 21.10.1931), \emph{Schriftsteller, Mediziner}!Lebendige Stunden. Vier Einakter1901-12-23@\strich\emph{Lebendige Stunden. Vier Einakter} {[}1901-12-23{]}|pwk} gegeben wurden. Vgl. \emph{Der
                        Briefwechsel Arthur Schnitzler — Otto Brahm}. Vollständige Ausgabe.
                     Herausgegeben, eingeleitet und erläutert von Oskar Seidlin.
                     Tübingen: \emph{Niemeyer}{ }1975, S. 123.}}}\label{K_L03211-1h} muß recht intereſſant
               geweſen ſein. Hat ſich \label{K_L03211-2v}\edtext{\textsc{Brahm\pwindex{Brahm, Otto 05.02.1856 – 28.11.1912@\textsc{Brahm, Otto} (05.02.1856 – 28.11.1912), \emph{Theaterleiter, Regisseur}|pw}} über die »\textsc{Beatrice\pwindex{Schnitzler, Arthur 15.05.1862 – 21.10.1931@\textsc{Schnitzler, Arthur} (15.05.1862 – 21.10.1931), \emph{Schriftsteller, Mediziner}!Schleier der Beatrice. Schauspiel in fuenf Akten1900-12-01@\strich\emph{Der Schleier der Beatrice. Schauspiel in fünf Akten} {[}1900-12-01{]}|pw}}« entſchieden}{\lemma{\textnormal{\emph{Brahm … entſchieden}}}\Cendnote{\textnormal{\emph{Der Schleier der Beatrice}\pwindex{Schnitzler, Arthur 15.05.1862 – 21.10.1931@\textsc{Schnitzler, Arthur} (15.05.1862 – 21.10.1931), \emph{Schriftsteller, Mediziner}!Schleier der Beatrice. Schauspiel in fuenf Akten1900-12-01@\strich\emph{Der Schleier der Beatrice. Schauspiel in fünf Akten} {[}1900-12-01{]}|pwk} wurde von Otto Brahm\pwindex{Brahm, Otto 05.02.1856 – 28.11.1912@\textsc{Brahm, Otto} (05.02.1856 – 28.11.1912), \emph{Theaterleiter, Regisseur}|pwk} für das Deutsche Theater Berlin\oindex{Deutsches Theater Berlin@\textbf{Deutsches Theater Berlin}|pwk} angenommen und feierte dort am 7. 3. 1903
                  Premiere.}}}\label{K_L03211-2h}? Wenn er die \label{K_L03211-3v}\edtext{»\textsc{Monna Vanna\pwindex{Maeterlinck, Maurice 29.08.1862 – 06.05.1949@\textsc{Maeterlinck, Maurice} (29.08.1862 – 06.05.1949), \emph{Schriftsteller}!Monna Vanna. Schauspiel in drei Akten1903@\strich\emph{Monna Vanna. Schauspiel in drei Akten} {[}1903{]}|pw}}« von \textsc{Maeterlinck\pwindex{Maeterlinck, Maurice 29.08.1862 – 06.05.1949@\textsc{Maeterlinck, Maurice} (29.08.1862 – 06.05.1949), \emph{Schriftsteller}|pw}} gibt}{\lemma{\textnormal{\emph{»Monna … gibt}}}\Cendnote{\textnormal{Maurice Maeterlinck\pwindex{Maeterlinck, Maurice 29.08.1862 – 06.05.1949@\textsc{Maeterlinck, Maurice} (29.08.1862 – 06.05.1949), \emph{Schriftsteller}|pwk}s \emph{Monna Vanna. Pièce en trois actes}\textcolor{red}{\textsuperscript{XXXX indx}} in der Übersetzung\pwindex{Maeterlinck, Maurice 29.08.1862 – 06.05.1949@\textsc{Maeterlinck, Maurice} (29.08.1862 – 06.05.1949), \emph{Schriftsteller}!Monna Vanna. Schauspiel in drei Akten1903@\strich\emph{Monna Vanna. Schauspiel in drei Akten} {[}1903{]}|pwkv} von 
                  Friedrich von Oppeln-Bronikowski\pwindex{\textcolor{red}{\textsuperscript{XXXX1 indx}}|pwk}
                  wurde ab dem 12. 10. 1902 über 100 Mal im Deutschen
                        Theater Berlin\oindex{Deutsches Theater Berlin@\textbf{Deutsches Theater Berlin}|pwk} aufgeführt.  Das Stück und \emph{Der Schleier der Beatrice}\pwindex{Schnitzler, Arthur 15.05.1862 – 21.10.1931@\textsc{Schnitzler, Arthur} (15.05.1862 – 21.10.1931), \emph{Schriftsteller, Mediziner}!Schleier der Beatrice. Schauspiel in fuenf Akten1900-12-01@\strich\emph{Der Schleier der Beatrice. Schauspiel in fünf Akten} {[}1900-12-01{]}|pwk} haben offensichtliche Parallelen, vor allem
                  im Ort der Handlung und in der zentralen Figur einer Frau zwischen zwei Männern.
                  Obzwar Schnitzler\pwindex{Schnitzler, Arthur 15.05.1862 – 21.10.1931@\textsc{Schnitzler, Arthur} (15.05.1862 – 21.10.1931), \emph{Schriftsteller, Mediziner}|pwk}s Stück\pwindex{Schnitzler, Arthur 15.05.1862 – 21.10.1931@\textsc{Schnitzler, Arthur} (15.05.1862 – 21.10.1931), \emph{Schriftsteller, Mediziner}!Schleier der Beatrice. Schauspiel in fuenf Akten1900-12-01@\strich\emph{Der Schleier der Beatrice. Schauspiel in fünf Akten} {[}1900-12-01{]}|pwkv} früher erschienen war, dauerte es
                  Monate, bis man sich einig war, ob es auch am \emph{Deutschen Theater}XXXX ORGangabe fehlt gegeben werden sollte, nachdem dort Maeterlinck\pwindex{Maeterlinck, Maurice 29.08.1862 – 06.05.1949@\textsc{Maeterlinck, Maurice} (29.08.1862 – 06.05.1949), \emph{Schriftsteller}|pwk}s{ }Stück\pwindex{Maeterlinck, Maurice 29.08.1862 – 06.05.1949@\textsc{Maeterlinck, Maurice} (29.08.1862 – 06.05.1949), \emph{Schriftsteller}!Monna Vanna. Schauspiel in drei Akten1903@\strich\emph{Monna Vanna. Schauspiel in drei Akten} {[}1903{]}|pwkv} am Spielplan
                  gestanden hatte. Schnitzler\pwindex{Schnitzler, Arthur 15.05.1862 – 21.10.1931@\textsc{Schnitzler, Arthur} (15.05.1862 – 21.10.1931), \emph{Schriftsteller, Mediziner}|pwk} besuchte die
                  Inszenierung am 14. 10. 1902. 
               }}}\label{K_L03211-3h}, muß er auch die »\textsc{Beatrice\pwindex{Schnitzler, Arthur 15.05.1862 – 21.10.1931@\textsc{Schnitzler, Arthur} (15.05.1862 – 21.10.1931), \emph{Schriftsteller, Mediziner}!Schleier der Beatrice. Schauspiel in fuenf Akten1900-12-01@\strich\emph{Der Schleier der Beatrice. Schauspiel in fünf Akten} {[}1900-12-01{]}|pw}}« geben können. Dein \label{K_L03211-4v}\edtext{Stück\pwindex{Schnitzler, Arthur 15.05.1862 – 21.10.1931@\textsc{Schnitzler, Arthur} (15.05.1862 – 21.10.1931), \emph{Schriftsteller, Mediziner}!einsame Weg. Schauspiel in fuenf Akten1904@\strich\emph{Der einsame Weg. Schauspiel in fünf Akten} {[}1904{]}|pwv}}{\lemma{\textnormal{\emph{Stück}}}\Cendnote{\textnormal{Schnitzler\pwindex{Schnitzler, Arthur 15.05.1862 – 21.10.1931@\textsc{Schnitzler, Arthur} (15.05.1862 – 21.10.1931), \emph{Schriftsteller, Mediziner}|pwk} hatte die Konzeption für \emph{Der einsame Weg}\pwindex{Schnitzler, Arthur 15.05.1862 – 21.10.1931@\textsc{Schnitzler, Arthur} (15.05.1862 – 21.10.1931), \emph{Schriftsteller, Mediziner}!einsame Weg. Schauspiel in fuenf Akten1904@\strich\emph{Der einsame Weg. Schauspiel in fünf Akten} {[}1904{]}|pwk} am 2. 6. 1902
                  abgeschlossen und begann es am 9. 8. 1902 zu schreiben.}}}\label{K_L03211-4h} laß’ nur ruhig noch warten, bis Du
               ordentlich Luſt bekommſt, es zu ſchreiben. Daß Du kurze \label{K_L03211-5v}\edtext{Geſchichten\pwindex{Schnitzler, Arthur 15.05.1862 – 21.10.1931@\textsc{Schnitzler, Arthur} (15.05.1862 – 21.10.1931), \emph{Schriftsteller, Mediziner}!griechische Taenzerin. Novellette28. 09. 1902@\strich\emph{Die griechische Tänzerin. Novellette} {[}28. 09. 1902{]}|pwv}\pwindex{Schnitzler, Arthur 15.05.1862 – 21.10.1931@\textsc{Schnitzler, Arthur} (15.05.1862 – 21.10.1931), \emph{Schriftsteller, Mediziner}!Weissagung24. 12. 1905@\strich\emph{Die Weissagung} {[}24. 12. 1905{]}|pwv}}{\lemma{\textnormal{\emph{Geſchichten}}}\Cendnote{\textnormal{Bezug auf \emph{Die griechische Tänzerin}\pwindex{Schnitzler, Arthur 15.05.1862 – 21.10.1931@\textsc{Schnitzler, Arthur} (15.05.1862 – 21.10.1931), \emph{Schriftsteller, Mediziner}!griechische Taenzerin. Novellette28. 09. 1902@\strich\emph{Die griechische Tänzerin. Novellette} {[}28. 09. 1902{]}|pwk} und \emph{Die Weissagung}\pwindex{Schnitzler, Arthur 15.05.1862 – 21.10.1931@\textsc{Schnitzler, Arthur} (15.05.1862 – 21.10.1931), \emph{Schriftsteller, Mediziner}!Weissagung24. 12. 1905@\strich\emph{Die Weissagung} {[}24. 12. 1905{]}|pwk}, die Schnitzler\pwindex{Schnitzler, Arthur 15.05.1862 – 21.10.1931@\textsc{Schnitzler, Arthur} (15.05.1862 – 21.10.1931), \emph{Schriftsteller, Mediziner}|pwk} am 7. 6. 1902 neu begonnen hatte}}}\label{K_L03211-5h} ſchreibſt, gefällt mir ſehr.
               Ich glaube, auf dieſem Gebiete iſt viel für Dich zu holen.\pend
           \pstart
           Daß ſich der Vater\pwindex{Gussmann, Rudolf 05.03.1842 – 24.01.1921@\textsc{Gussmann, Rudolf} (05.03.1842 – 24.01.1921), \emph{Handelsagent}|pwv} der Mädels\pwindex{Schnitzler, Olga 17.01.1882 – 13.01.1970@\textsc{Schnitzler, Olga} (17.01.1882 – 13.01.1970), \emph{Schauspielerin, Sängerin}|pwv}\pwindex{Steinrueck, Elisabeth 19.11.1885 – 07.04.1920@\textsc{Steinrück, Elisabeth} (19.11.1885 – 07.04.1920)|pwv}{ }{\pb}\label{K_L03211-6v}\edtext{verheirathet}{\lemma{\textnormal{\emph{verheirathet}}}\Cendnote{\textnormal{Amalia Gussmann\pwindex{Gussmann, Amalia 1855-04-15 – 1899-11-14@\textsc{Gussmann, Amalia} (1855-04-15 – 1899-11-14)|pwk}, die Mutter von Olga\pwindex{Schnitzler, Olga 17.01.1882 – 13.01.1970@\textsc{Schnitzler, Olga} (17.01.1882 – 13.01.1970), \emph{Schauspielerin, Sängerin}|pwk} und Elisabeth\pwindex{Steinrueck, Elisabeth 19.11.1885 – 07.04.1920@\textsc{Steinrück, Elisabeth} (19.11.1885 – 07.04.1920)|pwk}, war am 14. 11. 1899 nach
                  achtzehnjähriger Ehe verstorben. Rudolf
                     Gussmann\pwindex{Gussmann, Rudolf 05.03.1842 – 24.01.1921@\textsc{Gussmann, Rudolf} (05.03.1842 – 24.01.1921), \emph{Handelsagent}|pwk}s zweite Frau war Johanna
                     Steiner\pwindex{Gussmann, Johanna 1870-05-25 – 1905-06-30@\textsc{Gussmann, Johanna} (1870-05-25 – 1905-06-30)|pwk}. Die beiden heirateten am 8. 4. 1902 in Wien\oindex{Wien@\textbf{Wien}|pwk}. Die Ehe endete bald durch ihren Tod am
                     30. 6. 1905.}}}\label{K_L03211-6h} hat, iſt zugleich komiſch und gemein. Dieſer
               Hundsfott! Wie hat ſich die \label{K_L03211-7v}\edtext{Geſchichte
               mit dem Advokaten }{\lemma{\textnormal{\emph{Geſchichte … Advokaten}}}\Cendnote{\textnormal{Womöglich handelte es
                  sich um eine Erbschaftsangelegenheit, die durch die Eheschließung des Vaters\pwindex{Gussmann, Rudolf 05.03.1842 – 24.01.1921@\textsc{Gussmann, Rudolf} (05.03.1842 – 24.01.1921), \emph{Handelsagent}|pwkv} dringlich wurde.
                  Jedenfalls wurden die beiden Schwestern\pwindex{Schnitzler, Olga 17.01.1882 – 13.01.1970@\textsc{Schnitzler, Olga} (17.01.1882 – 13.01.1970), \emph{Schauspielerin, Sängerin}|pwk}\pwindex{Steinrueck, Elisabeth 19.11.1885 – 07.04.1920@\textsc{Steinrück, Elisabeth} (19.11.1885 – 07.04.1920)|pwk} am 14. 5. 1902 kurzfristig verhaftet.}}}\label{K_L03211-7h} abgewickelt?\pend
           \pstart
           Was \textsc{Liesl\pwindex{Steinrueck, Elisabeth 19.11.1885 – 07.04.1920@\textsc{Steinrück, Elisabeth} (19.11.1885 – 07.04.1920)|pw}} anlangt, ſo bitte ich Dich, einmal mit einem Donnerwetter dazwiſchenzufahren.
               Den an mich gerichteten \label{K_L03211-8v}\edtext{Brief von \textsc{Löwenfeld\pwindex{Loewenfeld, Raphael 11.02.1854 – 28.12.1910@\textsc{Löwenfeld, Raphael} (11.02.1854 – 28.12.1910), \emph{Theaterleiter}|pw}}}{\lemma{\textnormal{\emph{Brief von Löwenfeld}}}\Cendnote{\textnormal{Heute verwahrt in der Korrespondenz zwischen Goldmann\pwindex{Goldmann, Paul 31.01.1865 – 25.09.1935@\textsc{Goldmann, Paul} (31.01.1865 – 25.09.1935), \emph{Schriftsteller, Journalist}|pwk} und Elisabeth
                        Gussmann\pwindex{Steinrueck, Elisabeth 19.11.1885 – 07.04.1920@\textsc{Steinrück, Elisabeth} (19.11.1885 – 07.04.1920)|pwk}: \emph{DLA}, HS.1985.1.5246. Elisabeth Gussmann\pwindex{Steinrueck, Elisabeth 19.11.1885 – 07.04.1920@\textsc{Steinrück, Elisabeth} (19.11.1885 – 07.04.1920)|pwk} schloss am 2. 8. 1902 einen Vertrag mit dem \emph{Schiller-Theater}\orgindex{Schiller-Theater@Schiller-Theater|pwk} ab. Das Beschäftigungsverhältnis dauerte
                  von 1. 9. 1902 bis 30. 6. 1907.}}}\label{K_L03211-8h} haſt Du wohl geleſen? Ich ſchließe daraus, daß eine
               Möglichkeit des Engagements am Schillertheater\orgindex{Schiller-Theater@Schiller-Theater|pw}
               beſteht, wenn man nur ein wenig nachhilft. Ich bin gern bereit, nachzuhelfen\strikeout{,} und den perſönlichen Beſuch zu machen, zu dem er mich
               auffordert. Aber vorher muß ich wiſſen, ob \textsc{Liesl\pwindex{Steinrueck, Elisabeth 19.11.1885 – 07.04.1920@\textsc{Steinrück, Elisabeth} (19.11.1885 – 07.04.1920)|pw}} ihm geſchrieben hat, nachdem ſie mir bereits {\pb}einmal \strikeout{geſ} vorgeſchwindelt hat, ſie habe ihm
               geſchrieben, ohne es gethan zu haben. Ich warte alſo auf Antwort und bekomme keine.
               Veranlaſſe doch, \strikeout{\textcolor{gray}{×}\-\textcolor{gray}{×}\-\textcolor{gray}{×}\-\textcolor{gray}{×}} daß die junge \strikeout{Dam\textcolor{gray}{e}\pwindex{Steinrueck, Elisabeth 19.11.1885 – 07.04.1920@\textsc{Steinrück, Elisabeth} (19.11.1885 – 07.04.1920)|pw}}{ }Dame\pwindex{Steinrueck, Elisabeth 19.11.1885 – 07.04.1920@\textsc{Steinrück, Elisabeth} (19.11.1885 – 07.04.1920)|pw} ſich aufrafft und zur Feder greift, und
               ſage ihr, bitte, in meinem Namen, daß ich wüthend bin und daß man mit ſolch’ einer
               verfluchten Schlamperei keine Engagements bekommt!\pend
           \pstart
           Grüße \textsc{Olga\pwindex{Schnitzler, Olga 17.01.1882 – 13.01.1970@\textsc{Schnitzler, Olga} (17.01.1882 – 13.01.1970), \emph{Schauspielerin, Sängerin}|pw}} recht herzlich. Ich hoffe, ſie übt die \textsc{Löwe\pwindex{Loewe, Carl 30.11.1796 – 20.04.1869@\textsc{Loewe, Carl} (30.11.1796 – 20.04.1869), \emph{Sänger, Komponist}|pw}}’ſchen Balladen\pwindex{Loewe, Carl 30.11.1796 – 20.04.1869@\textsc{Loewe, Carl} (30.11.1796 – 20.04.1869), \emph{Sänger, Komponist}!Tom der Reimer– 1867@\strich\emph{Tom der Reimer} {[}– 1867{]}|pwv}\pwindex{\textcolor{red}{\textsuperscript{XXXX1 indx}}!Tom der Reimer– 1867@\strich\emph{Tom der Reimer} {[}– 1867{]}|pwv}\pwindex{Loewe, Carl 30.11.1796 – 20.04.1869@\textsc{Loewe, Carl} (30.11.1796 – 20.04.1869), \emph{Sänger, Komponist}!Heinrich der Vogler@\strich\emph{Heinrich der Vogler}|pwv}\pwindex{\textcolor{red}{\textsuperscript{XXXX1 indx}}!Heinrich der Vogler@\strich\emph{Heinrich der Vogler}|pwv} (Tom der Reimer\pwindex{Loewe, Carl 30.11.1796 – 20.04.1869@\textsc{Loewe, Carl} (30.11.1796 – 20.04.1869), \emph{Sänger, Komponist}!Tom der Reimer– 1867@\strich\emph{Tom der Reimer} {[}– 1867{]}|pw}\pwindex{\textcolor{red}{\textsuperscript{XXXX1 indx}}!Tom der Reimer– 1867@\strich\emph{Tom der Reimer} {[}– 1867{]}|pw}, Heinrich der Vogler\pwindex{Loewe, Carl 30.11.1796 – 20.04.1869@\textsc{Loewe, Carl} (30.11.1796 – 20.04.1869), \emph{Sänger, Komponist}!Heinrich der Vogler@\strich\emph{Heinrich der Vogler}|pw}\pwindex{\textcolor{red}{\textsuperscript{XXXX1 indx}}!Heinrich der Vogler@\strich\emph{Heinrich der Vogler}|pw}). Wenn ich nach Wien\oindex{Wien@\textbf{Wien}|pw} komme, will ich ſie vorgeſungen haben.\pend
           \pstart
           {\pb}Meine Pläne bleiben einſtweilen die alten: \label{K_L03211-9v}\edtext{Zwiſchen 20.
               u. 25. Juli{ }Wien\oindex{Wien@\textbf{Wien}|pw}, dann \textsc{Trafoi\oindex{Trafoi@\textbf{Trafoi}|pw}}}{\lemma{\textnormal{\emph{Zwiſchen … Trafoi}}}\Cendnote{\textnormal{nicht geschehen, siehe Paul Goldmann an Arthur Schnitzler, 25. 7. [1902]}}}\label{K_L03211-9h}. Von \label{K_L03211-10v}\edtext{Fräulein \textsc{F.\pwindex{Goldmann, Eva Marie 27.10.1877 – 02.11.1937@\textsc{Goldmann, Eva Marie} (27.10.1877 – 02.11.1937)|pwv}}}{\lemma{\textnormal{\emph{Fräulein F.}}}\Cendnote{\textnormal{womöglich Goldmann\pwindex{Goldmann, Paul 31.01.1865 – 25.09.1935@\textsc{Goldmann, Paul} (31.01.1865 – 25.09.1935), \emph{Schriftsteller, Journalist}|pwk}s nachmalige Ehefrau Eva Marie Fränkel\pwindex{Goldmann, Eva Marie 27.10.1877 – 02.11.1937@\textsc{Goldmann, Eva Marie} (27.10.1877 – 02.11.1937)|pwk}, die aber in der Zeit bis zur
                  Eheschließung 1908 noch mit einem Herrn Kobler\pwindex{Kobler @\textsc{Kobler}|pwk} verheiratet war.}}}\label{K_L03211-10h} erhalte ich hier und da
               einen Brief. Aber das Schreiben iſt eine dumme Sache. Die Fäden ſind abgeriſſen. Sie
               ſchreibt mir übrigens, daß ſie öfter mit \textsc{Salten\pwindex{Salten, Felix 06.09.1869 – 08.10.1945@\textsc{Salten, Felix} (06.09.1869 – 08.10.1945), \emph{Schriftsteller, Journalist}|pw}} zuſammen iſt.\pend
           \pstart
           Schreib’ mir bald wieder und ſei vielmals und von Herzen gegrüßt! {\\[\baselineskip]}Dein {\\[\baselineskip]}\spacefill\mbox{Paul Goldmnn}\pend
           \leftskip=0em{}
         
         \endnumbering\mylabel{h}\end{ledgroupsized}  \newcommand{\dateiname}{L03211}\newcommand{\titel}{Paul Goldmann an Arthur Schnitzler, 16. 6. [1902]}\newcommand{\editorInnen}{Martin Anton Müller und Laura Untner}%% latex-leseansicht-abspann.tex
%% Abspann für die Leseansicht.
%% Der Schalter \ifkorrekturansicht ist bereits durch den Vorspann gesetzt.

%% latex-abspann.tex
%% Gemeinsamer Abspann für Korrekturansicht und Leseansicht.
%% Setzt den Schalter \ifkorrekturansicht voraus (gesetzt in den
%% einbindenden Dateien latex-korrekturansicht-abspann.tex bzw.
%% latex-leseansicht-abspann.tex).
%% ---------------------------------------------------------------

\normalsize

% Das esempio-Environment wird nur in der Leseansicht benötigt
\ifkorrekturansicht\else
\newenvironment{esempio}[3]%
{
    \vspace{1.5ex}
    \rlap{\underline{#1}}
    \par
    \setlength{\parindent}{0cm}
    \nopagebreak
    \leftskip=#2cm
    \rightskip=#3cm
}
{
    \par
}
\fi

\doendnotes{C}
\bigskip
\vfill

\clearpage

\footnotesize

\ifkorrekturansicht
  \lohead{\textsc{register}}
\fi

% theindex-Environment neu definieren ohne reledmac
\makeatletter
\renewenvironment{theindex}{%
  \ifkorrekturansicht
    \section*{\indexname}%
  \else
    \subsubsection*{Index der erwähnten Entitäten}%
  \fi
  \setlength{\parindent}{0pt}%
  \setlength{\parskip}{0pt plus 0.3pt}%
  \let\item\@idxitem
}{%
  \ifkorrekturansicht\clearpage\fi
}
\makeatother

\IfFileExists{\jobname-pw.ind}{\input{\jobname-pw.ind}}{}

% Quellenangabe nur in der Leseansicht
\ifkorrekturansicht\else
% Fallback-Definitionen, falls die .tex-Datei \titel etc. nicht gesetzt hat
\providecommand{\titel}{}
\providecommand{\editorInnen}{}
\providecommand{\dateiname}{\jobname}

\vspace{3cm}

\vfill

\footnotesize
\textsc{Quelle}: \titel. Herausgegeben von {\editorInnen}. In: \emph{Arthur Schnitzler: Briefwechsel mit Autorinnen und Autoren}.
 Digitale Edition, https://schnitzler-briefe.acdh.oeaw.ac.at/{\dateiname}.html (Stand \today)
\fi

\end{document}


      