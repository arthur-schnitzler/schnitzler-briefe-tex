%% latex-leseansicht-vorspann.tex
%% Vorspann für die Leseansicht.
%% Lädt die gemeinsame Datei latex-vorspann.tex mit nicht gesetztem Schalter.

\newif\ifkorrekturansicht
\korrekturansichtfalse

\input{../tex-inputs/latex-vorspann}


\section[ Paul Goldmann an Arthur Schnitzler, 16. 6. {[}1902{]}]{L03211 Paul Goldmann an Arthur Schnitzler,  16. 6. [1902]}
\nopagebreak\mylabel{L03211v}
\rehead{ }\normalsize\beginnumbering\briefempfaengerindex{Schnitzler, Arthur@\textsc{Schnitzler, Arthur}!zzzGoldmann, Paul@\emph{von Paul Goldmann}!1902-06-162@{16. 6. [1902]}|(be}
\toendnotes[C]{\smallbreak\pagebreak[2]}
\correspDesc{Versand  durch Paul Goldmann am 16. 6. [1902] in Berlin
\newline{}Erhalt  durch Arthur Schnitzler im Zeitraum [17. 6. 1902
                  – 21. 6. 1902?] in Wien}\toendnotes[C]{\smallbreak}
\Standort{DLA, A:Schnitzler, HS.NZ85.1.3172.}
\physDesc{Brief, 1 Blatt, 4 Seiten, 1909 Zeichen
\newline{}Handschrift: blaue Tinte, deutsche Kurrent
\newline{}Schnitzler: 1) mit Bleistift das Jahr »902« vermerkt  2) mit rotem Buntstift eine Unterstreichung}\toendnotes[C]{\smallbreak}
\pstart
           \raggedleft{}{\pb}\textcolor{gray}{\textbf{DESSAUERSTRASSE 19}}\oindex{Dessauer Straße@\textbf{Dessauer Straße}, \emph{Straße}|pw}\pend
           
\pstart
           Berlin\oindex{Berlin@\textbf{Berlin}, \emph{Hauptstadt}|pw}, 16. Juni.\pend
           
\pstart\center{}Mein lieber Freund,\pend\vspace{0.5em}
\pstart
           Ich habe mich{ }ſehr gefreut, wieder von Dir zu hören. Die \label{K_L03211-1v}\edtext{Budapeſt\oindex{Budapest@\textbf{Budapest}, \emph{Hauptstadt}|pw}er Reiſe}{\lemma{\textnormal{\emph{Budapester Reise}}}\Cendnote{\textnormal{Auf Otto Brahms\pwindex{Brahm, Otto 5.\,2.\,1856 Hamburg – 28.\,11.\,1912 Berlin@\textsc{Brahm, Otto} (5.\,2.\,1856 Hamburg – 28.\,11.\,1912 Berlin), \emph{Theaterleiter, Regisseur}|pwk}
                  Einladung hin war Schnitzler am 7. 6. 1902 und 8. 6. 1902 in Budapest\oindex{Budapest@\textbf{Budapest}, \emph{Hauptstadt}|pwk} gewesen, wo im Lustspielhaus\oindex{Lustspieltheater [Budapest]@\textbf{Lustspieltheater [Budapest]}, \emph{Theater}|pwk} die \emph{Lebendigen Stunden}\pwindex{Schnitzler, Arthur 15.\,5.\,1862 Wien – 21.\,10.\,1931 ebd.@\textsc{Schnitzler, Arthur} (15.\,5.\,1862 Wien – 21.\,10.\,1931 ebd.), \emph{Schriftsteller, Mediziner}!Lebendige Stunden. Vier Einakter@\strich\emph{Lebendige Stunden. Vier Einakter}|pwk} gegeben wurden. Vgl. \emph{Der
                        Briefwechsel Arthur Schnitzler – Otto Brahm}. Vollständige Ausgabe.
                     Herausgegeben, eingeleitet und erläutert von Oskar Seidlin.
                     Tübingen: \emph{Niemeyer}{ }1975, S. 123.}}}\label{K_L03211-1} muß recht intereſſant
               geweſen{ }ſein. Hat{ }ſich \label{K_L03211-2v}\edtext{\textsc{Brahm\pwindex{Brahm, Otto 5.\,2.\,1856 Hamburg – 28.\,11.\,1912 Berlin@\textsc{Brahm, Otto} (5.\,2.\,1856 Hamburg – 28.\,11.\,1912 Berlin), \emph{Theaterleiter, Regisseur}|pw}} über die »\textsc{Beatrice\pwindex{Schnitzler, Arthur 15.\,5.\,1862 Wien – 21.\,10.\,1931 ebd.@\textsc{Schnitzler, Arthur} (15.\,5.\,1862 Wien – 21.\,10.\,1931 ebd.), \emph{Schriftsteller, Mediziner}!Schleier der Beatrice. Schauspiel in fünf Akten@\strich\emph{Der Schleier der Beatrice. Schauspiel in fünf Akten}|pw}}« entſchieden}{\lemma{\textnormal{\emph{Brahm … entschieden}}}\Cendnote{\textnormal{\emph{Der Schleier der Beatrice}\pwindex{Schnitzler, Arthur 15.\,5.\,1862 Wien – 21.\,10.\,1931 ebd.@\textsc{Schnitzler, Arthur} (15.\,5.\,1862 Wien – 21.\,10.\,1931 ebd.), \emph{Schriftsteller, Mediziner}!Schleier der Beatrice. Schauspiel in fünf Akten@\strich\emph{Der Schleier der Beatrice. Schauspiel in fünf Akten}|pwk} wurde von Otto Brahm\pwindex{Brahm, Otto 5.\,2.\,1856 Hamburg – 28.\,11.\,1912 Berlin@\textsc{Brahm, Otto} (5.\,2.\,1856 Hamburg – 28.\,11.\,1912 Berlin), \emph{Theaterleiter, Regisseur}|pwk} für das Deutsche Theater Berlin\oindex{Deutsches Theater Berlin@\textbf{Deutsches Theater Berlin}, \emph{Theater}|pwk} angenommen und feierte dort am 7. 3. 1903
                  Premiere.}}}\label{K_L03211-2}? Wenn er die \label{K_L03211-3v}\edtext{»\textsc{Monna Vanna\pwindex{Maeterlinck, Maurice 29.\,8.\,1862 Gent – 6.\,5.\,1949 Nizza@\textsc{Maeterlinck, Maurice} (29.\,8.\,1862 Gent – 6.\,5.\,1949 Nizza), \emph{Schriftsteller}!Monna Vanna. Schauspiel in drei Akten@\strich\emph{Monna Vanna. Schauspiel in drei Akten}|pw}}« von \textsc{Maeterlinck\pwindex{Maeterlinck, Maurice 29.\,8.\,1862 Gent – 6.\,5.\,1949 Nizza@\textsc{Maeterlinck, Maurice} (29.\,8.\,1862 Gent – 6.\,5.\,1949 Nizza), \emph{Schriftsteller}|pw}} gibt}{\lemma{\textnormal{\emph{»Monna … gibt}}}\Cendnote{\textnormal{Maurice Maeterlincks\pwindex{Maeterlinck, Maurice 29.\,8.\,1862 Gent – 6.\,5.\,1949 Nizza@\textsc{Maeterlinck, Maurice} (29.\,8.\,1862 Gent – 6.\,5.\,1949 Nizza), \emph{Schriftsteller}|pwk}{ }\emph{Monna Vanna. Pièce en trois actes}\pwindex{Maeterlinck, Maurice 29.\,8.\,1862 Gent – 6.\,5.\,1949 Nizza@\textsc{Maeterlinck, Maurice} (29.\,8.\,1862 Gent – 6.\,5.\,1949 Nizza), \emph{Schriftsteller}!Monna Vanna. Pièce en trois actes@\strich\emph{Monna Vanna. Pièce en trois actes}|pwk} in der Übersetzung\pwindex{Maeterlinck, Maurice 29.\,8.\,1862 Gent – 6.\,5.\,1949 Nizza@\textsc{Maeterlinck, Maurice} (29.\,8.\,1862 Gent – 6.\,5.\,1949 Nizza), \emph{Schriftsteller}!Monna Vanna. Schauspiel in drei Akten@\strich\emph{Monna Vanna. Schauspiel in drei Akten}|pwkv} von 
                  Friedrich von Oppeln-Bronikowski\pwindex{Oppeln-Bronikowski, Friedrich von 7.\,4.\,1873 Kassel – 9.\,10.\,1936 Berlin@\textsc{Oppeln-Bronikowski, Friedrich von} (7.\,4.\,1873 Kassel – 9.\,10.\,1936 Berlin), \emph{Schriftsteller, Übersetzer}|pwk}
                  wurde ab dem 12. 10. 1902 über hundertmal im Deutschen
                        Theater Berlin\oindex{Deutsches Theater Berlin@\textbf{Deutsches Theater Berlin}, \emph{Theater}|pwk} aufgeführt.  Das Stück und \emph{Der Schleier der Beatrice}\pwindex{Schnitzler, Arthur 15.\,5.\,1862 Wien – 21.\,10.\,1931 ebd.@\textsc{Schnitzler, Arthur} (15.\,5.\,1862 Wien – 21.\,10.\,1931 ebd.), \emph{Schriftsteller, Mediziner}!Schleier der Beatrice. Schauspiel in fünf Akten@\strich\emph{Der Schleier der Beatrice. Schauspiel in fünf Akten}|pwk} haben offensichtliche Parallelen, vor allem
                  im Ort der Handlung und in der zentralen Figur einer Frau zwischen zwei Männern.
                  Obzwar Schnitzlers{ }Stück\pwindex{Schnitzler, Arthur 15.\,5.\,1862 Wien – 21.\,10.\,1931 ebd.@\textsc{Schnitzler, Arthur} (15.\,5.\,1862 Wien – 21.\,10.\,1931 ebd.), \emph{Schriftsteller, Mediziner}!Schleier der Beatrice. Schauspiel in fünf Akten@\strich\emph{Der Schleier der Beatrice. Schauspiel in fünf Akten}|pwkv} früher erschienen war, dauerte es
                  Monate, bis man sich einig war, ob es auch am \emph{Deutschen Theater}\orgindex{Deutsches Theater Berlin@Deutsches Theater Berlin|pwk} gegeben werden sollte, nachdem dort Maurice Maeterlincks\pwindex{Maeterlinck, Maurice 29.\,8.\,1862 Gent – 6.\,5.\,1949 Nizza@\textsc{Maeterlinck, Maurice} (29.\,8.\,1862 Gent – 6.\,5.\,1949 Nizza), \emph{Schriftsteller}|pwk}{ }Stück\pwindex{Maeterlinck, Maurice 29.\,8.\,1862 Gent – 6.\,5.\,1949 Nizza@\textsc{Maeterlinck, Maurice} (29.\,8.\,1862 Gent – 6.\,5.\,1949 Nizza), \emph{Schriftsteller}!Monna Vanna. Schauspiel in drei Akten@\strich\emph{Monna Vanna. Schauspiel in drei Akten}|pwkv} am Spielplan
                  gestanden hatte. Schnitzler besuchte die
                  Inszenierung am 14. 10. 1902. 
               }}}\label{K_L03211-3}, muß er auch die »\textsc{Beatrice\pwindex{Schnitzler, Arthur 15.\,5.\,1862 Wien – 21.\,10.\,1931 ebd.@\textsc{Schnitzler, Arthur} (15.\,5.\,1862 Wien – 21.\,10.\,1931 ebd.), \emph{Schriftsteller, Mediziner}!Schleier der Beatrice. Schauspiel in fünf Akten@\strich\emph{Der Schleier der Beatrice. Schauspiel in fünf Akten}|pw}}« geben können. Dein \label{K_L03211-4v}\edtext{Stück\pwindex{Schnitzler, Arthur 15.\,5.\,1862 Wien – 21.\,10.\,1931 ebd.@\textsc{Schnitzler, Arthur} (15.\,5.\,1862 Wien – 21.\,10.\,1931 ebd.), \emph{Schriftsteller, Mediziner}!einsame Weg. Schauspiel in fünf Akten@\strich\emph{Der einsame Weg. Schauspiel in fünf Akten}|pwv}}{\lemma{\textnormal{\emph{Stück}}}\Cendnote{\textnormal{Schnitzler hatte die Konzeption für \emph{Der einsame Weg}\pwindex{Schnitzler, Arthur 15.\,5.\,1862 Wien – 21.\,10.\,1931 ebd.@\textsc{Schnitzler, Arthur} (15.\,5.\,1862 Wien – 21.\,10.\,1931 ebd.), \emph{Schriftsteller, Mediziner}!einsame Weg. Schauspiel in fünf Akten@\strich\emph{Der einsame Weg. Schauspiel in fünf Akten}|pwk} am 2. 6. 1902
                  abgeschlossen und begann es am 9. 8. 1902 zu schreiben.}}}\label{K_L03211-4} laß’ nur ruhig noch warten, bis Du
               ordentlich Luſt bekommſt, es zu{ }ſchreiben. Daß Du kurze \label{K_L03211-5v}\edtext{Geſchichten\pwindex{Schnitzler, Arthur 15.\,5.\,1862 Wien – 21.\,10.\,1931 ebd.@\textsc{Schnitzler, Arthur} (15.\,5.\,1862 Wien – 21.\,10.\,1931 ebd.), \emph{Schriftsteller, Mediziner}!griechische Tänzerin. Novellette@\strich\emph{Die griechische Tänzerin. Novellette}|pwv}\pwindex{Schnitzler, Arthur 15.\,5.\,1862 Wien – 21.\,10.\,1931 ebd.@\textsc{Schnitzler, Arthur} (15.\,5.\,1862 Wien – 21.\,10.\,1931 ebd.), \emph{Schriftsteller, Mediziner}!Weissagung@\strich\emph{Die Weissagung}|pwv}}{\lemma{\textnormal{\emph{Geschichten}}}\Cendnote{\textnormal{Bezug auf \emph{Die griechische Tänzerin}\pwindex{Schnitzler, Arthur 15.\,5.\,1862 Wien – 21.\,10.\,1931 ebd.@\textsc{Schnitzler, Arthur} (15.\,5.\,1862 Wien – 21.\,10.\,1931 ebd.), \emph{Schriftsteller, Mediziner}!griechische Tänzerin. Novellette@\strich\emph{Die griechische Tänzerin. Novellette}|pwk} und \emph{Die Weissagung}\pwindex{Schnitzler, Arthur 15.\,5.\,1862 Wien – 21.\,10.\,1931 ebd.@\textsc{Schnitzler, Arthur} (15.\,5.\,1862 Wien – 21.\,10.\,1931 ebd.), \emph{Schriftsteller, Mediziner}!Weissagung@\strich\emph{Die Weissagung}|pwk}, die Schnitzler am 7. 6. 1902 neu begonnen hatte}}}\label{K_L03211-5}{ }ſchreibſt, gefällt mir{ }ſehr.
               Ich glaube, auf dieſem Gebiete iſt viel für Dich zu holen.\pend
           
\pstart
           Daß{ }ſich der Vater\pwindex{Gussmann, Rudolf 5.\,3.\,1842 Veprovac – 24.\,1.\,1921 Wien@\textsc{Gussmann, Rudolf} (5.\,3.\,1842 Veprovac – 24.\,1.\,1921 Wien), \emph{Handelsagent}|pwv} der Mädels\pwindex{Schnitzler, Olga 17.\,1.\,1882 Wien – 13.\,1.\,1970 Lugano@\textsc{Schnitzler, Olga} (17.\,1.\,1882 Wien – 13.\,1.\,1970 Lugano), \emph{Schauspielerin, Sängerin}|pwv}\pwindex{Steinrück, Elisabeth 19.\,11.\,1885 – 7.\,4.\,1920 Partenkirchen@\textsc{Steinrück, Elisabeth} (19.\,11.\,1885 – 7.\,4.\,1920 Partenkirchen)|pwv}{ }{\pb}\label{K_L03211-6v}\edtext{verheirathet}{\lemma{\textnormal{\emph{verheirathet}}}\Cendnote{\textnormal{Amalia Gussmann\pwindex{Gussmann, Amalia 15.\,4.\,1855 – 14.\,11.\,1899 Baden bei Wien@\textsc{Gussmann, Amalia} (15.\,4.\,1855 – 14.\,11.\,1899 Baden bei Wien)|pwk}, die Mutter von Olga\pwindex{Schnitzler, Olga 17.\,1.\,1882 Wien – 13.\,1.\,1970 Lugano@\textsc{Schnitzler, Olga} (17.\,1.\,1882 Wien – 13.\,1.\,1970 Lugano), \emph{Schauspielerin, Sängerin}|pwk} und Elisabeth\pwindex{Steinrück, Elisabeth 19.\,11.\,1885 – 7.\,4.\,1920 Partenkirchen@\textsc{Steinrück, Elisabeth} (19.\,11.\,1885 – 7.\,4.\,1920 Partenkirchen)|pwk}, war am 14. 11. 1899 nach
                  achtzehnjähriger Ehe verstorben. Rudolf
                     Gussmanns\pwindex{Gussmann, Rudolf 5.\,3.\,1842 Veprovac – 24.\,1.\,1921 Wien@\textsc{Gussmann, Rudolf} (5.\,3.\,1842 Veprovac – 24.\,1.\,1921 Wien), \emph{Handelsagent}|pwk} zweite Frau war Johanna
                     Steiner\pwindex{Gussmann, Johanna 25.\,5.\,1870 Innsbruck – 30.\,6.\,1905 Wien@\textsc{Gussmann, Johanna} (25.\,5.\,1870 Innsbruck – 30.\,6.\,1905 Wien)|pwk}. Die beiden heirateten am 8. 4. 1902 in Wien\oindex{Wien@\textbf{Wien}, \emph{Verwaltungsgebiet}|pwk}. Die Ehe endete bald durch ihren Tod am
                     30. 6. 1905.}}}\label{K_L03211-6} hat, iſt zugleich komiſch und gemein. Dieſer
               Hundsfott! Wie hat{ }ſich die \label{K_L03211-7v}\edtext{Geſchichte
               mit dem Advokaten}{\lemma{\textnormal{\emph{Geschichte … Advokaten}}}\Cendnote{\textnormal{Womöglich handelte es
                  sich um eine Erbschaftsangelegenheit, die durch die Eheschließung des Vaters\pwindex{Gussmann, Rudolf 5.\,3.\,1842 Veprovac – 24.\,1.\,1921 Wien@\textsc{Gussmann, Rudolf} (5.\,3.\,1842 Veprovac – 24.\,1.\,1921 Wien), \emph{Handelsagent}|pwkv} dringlich wurde.
                  Jedenfalls wurden die beiden Schwestern\pwindex{Schnitzler, Olga 17.\,1.\,1882 Wien – 13.\,1.\,1970 Lugano@\textsc{Schnitzler, Olga} (17.\,1.\,1882 Wien – 13.\,1.\,1970 Lugano), \emph{Schauspielerin, Sängerin}|pwk}\pwindex{Steinrück, Elisabeth 19.\,11.\,1885 – 7.\,4.\,1920 Partenkirchen@\textsc{Steinrück, Elisabeth} (19.\,11.\,1885 – 7.\,4.\,1920 Partenkirchen)|pwk} am 14. 5. 1902 kurzfristig verhaftet.}}}\label{K_L03211-7} abgewickelt?\pend
           
\pstart
           Was \textsc{Liesl\pwindex{Steinrück, Elisabeth 19.\,11.\,1885 – 7.\,4.\,1920 Partenkirchen@\textsc{Steinrück, Elisabeth} (19.\,11.\,1885 – 7.\,4.\,1920 Partenkirchen)|pw}} anlangt,{ }ſo bitte ich Dich, einmal mit einem Donnerwetter dazwiſchenzufahren.
               Den an mich gerichteten \label{K_L03211-8v}\edtext{Brief von \textsc{Löwenfeld\pwindex{Löwenfeld, Raphael 11.\,2.\,1854 Poznan – 28.\,12.\,1910 Berlin@\textsc{Löwenfeld, Raphael} (11.\,2.\,1854 Poznan – 28.\,12.\,1910 Berlin), \emph{Theaterleiter}|pw}}}{\lemma{\textnormal{\emph{Brief von Löwenfeld}}}\Cendnote{\textnormal{Heute findet sich dieser in der Korrespondenz zwischen Goldmann\pwindex{Goldmann, Paul 31.\,1.\,1865 Breslau – 25.\,9.\,1935 Wien@\textsc{Goldmann, Paul} (31.\,1.\,1865 Breslau – 25.\,9.\,1935 Wien), \emph{Schriftsteller, Journalist}|pwk} und Elisabeth
                        Gussmann\pwindex{Steinrück, Elisabeth 19.\,11.\,1885 – 7.\,4.\,1920 Partenkirchen@\textsc{Steinrück, Elisabeth} (19.\,11.\,1885 – 7.\,4.\,1920 Partenkirchen)|pwk} verwahrt: \emph{DLA}, HS.1985.1.5246. Elisabeth Gussmann\pwindex{Steinrück, Elisabeth 19.\,11.\,1885 – 7.\,4.\,1920 Partenkirchen@\textsc{Steinrück, Elisabeth} (19.\,11.\,1885 – 7.\,4.\,1920 Partenkirchen)|pwk} schloss am 2. 8. 1902 einen Vertrag mit dem \emph{Schiller-Theater}\orgindex{Schiller-Theater@Schiller-Theater|pwk} ab. Das Beschäftigungsverhältnis dauerte
                  von 1. 9. 1902 bis 30. 6. 1907.}}}\label{K_L03211-8} haſt Du wohl geleſen? Ich{ }ſchließe daraus, daß eine
               Möglichkeit des Engagements am Schillertheater\orgindex{Schiller-Theater@Schiller-Theater|pw}
               beſteht, wenn man nur ein wenig nachhilft. Ich bin gern bereit, nachzuhelfen\strikeout{,} und den perſönlichen Beſuch zu machen, zu dem er mich
               auffordert. Aber vorher muß ich wiſſen, ob \textsc{Liesl\pwindex{Steinrück, Elisabeth 19.\,11.\,1885 – 7.\,4.\,1920 Partenkirchen@\textsc{Steinrück, Elisabeth} (19.\,11.\,1885 – 7.\,4.\,1920 Partenkirchen)|pw}} ihm geſchrieben hat, nachdem{ }ſie mir bereits {\pb}einmal \strikeout{geſ} vorgeſchwindelt hat,{ }ſie habe ihm
               geſchrieben, ohne es gethan zu haben. Ich warte alſo auf Antwort und bekomme keine.
               Veranlaſſe doch, \strikeout{\textcolor{gray}{×}\-\textcolor{gray}{×}\-\textcolor{gray}{×}\-\textcolor{gray}{×}} daß die junge \strikeout{Dam\textcolor{gray}{e}\pwindex{Steinrück, Elisabeth 19.\,11.\,1885 – 7.\,4.\,1920 Partenkirchen@\textsc{Steinrück, Elisabeth} (19.\,11.\,1885 – 7.\,4.\,1920 Partenkirchen)|pw}}{ }Dame\pwindex{Steinrück, Elisabeth 19.\,11.\,1885 – 7.\,4.\,1920 Partenkirchen@\textsc{Steinrück, Elisabeth} (19.\,11.\,1885 – 7.\,4.\,1920 Partenkirchen)|pw}{ }ſich aufrafft und zur Feder greift, und{ }ſage ihr, bitte, in meinem Namen, daß ich wüthend bin und daß man mit{ }ſolch’ einer
               verfluchten Schlamperei keine Engagements bekommt!\pend
           
\pstart
           Grüße \textsc{Olga\pwindex{Schnitzler, Olga 17.\,1.\,1882 Wien – 13.\,1.\,1970 Lugano@\textsc{Schnitzler, Olga} (17.\,1.\,1882 Wien – 13.\,1.\,1970 Lugano), \emph{Schauspielerin, Sängerin}|pw}} recht herzlich. Ich hoffe,{ }ſie übt die \textsc{Löwe\pwindex{Loewe, Carl 30.\,11.\,1796 Löbejün – 20.\,4.\,1869 Kiel@\textsc{Loewe, Carl} (30.\,11.\,1796 Löbejün – 20.\,4.\,1869 Kiel), \emph{Sänger, Komponist}|pw}}’ſchen Balladen\pwindex{Loewe, Carl 30.\,11.\,1796 Löbejün – 20.\,4.\,1869 Kiel@\textsc{Loewe, Carl} (30.\,11.\,1796 Löbejün – 20.\,4.\,1869 Kiel), \emph{Sänger, Komponist}!Tom der Reimer@\strich\emph{Tom der Reimer}|pwv}\pwindex{\textcolor{red}{\textsuperscript{XXXX indx1}}!Tom der Reimer@\strich\emph{Tom der Reimer}|pwv}\pwindex{Loewe, Carl 30.\,11.\,1796 Löbejün – 20.\,4.\,1869 Kiel@\textsc{Loewe, Carl} (30.\,11.\,1796 Löbejün – 20.\,4.\,1869 Kiel), \emph{Sänger, Komponist}!Heinrich der Vogler@\strich\emph{Heinrich der Vogler}|pwv}\pwindex{\textcolor{red}{\textsuperscript{XXXX indx1}}!Heinrich der Vogler@\strich\emph{Heinrich der Vogler}|pwv} (Tom der Reimer\pwindex{Loewe, Carl 30.\,11.\,1796 Löbejün – 20.\,4.\,1869 Kiel@\textsc{Loewe, Carl} (30.\,11.\,1796 Löbejün – 20.\,4.\,1869 Kiel), \emph{Sänger, Komponist}!Tom der Reimer@\strich\emph{Tom der Reimer}|pw}\pwindex{\textcolor{red}{\textsuperscript{XXXX indx1}}!Tom der Reimer@\strich\emph{Tom der Reimer}|pw}, Heinrich der Vogler\pwindex{Loewe, Carl 30.\,11.\,1796 Löbejün – 20.\,4.\,1869 Kiel@\textsc{Loewe, Carl} (30.\,11.\,1796 Löbejün – 20.\,4.\,1869 Kiel), \emph{Sänger, Komponist}!Heinrich der Vogler@\strich\emph{Heinrich der Vogler}|pw}\pwindex{\textcolor{red}{\textsuperscript{XXXX indx1}}!Heinrich der Vogler@\strich\emph{Heinrich der Vogler}|pw}). Wenn ich nach Wien\oindex{Wien@\textbf{Wien}, \emph{Verwaltungsgebiet}|pw} komme, will ich{ }ſie vorgeſungen haben.\pend
           
\pstart
           {\pb}Meine Pläne bleiben einſtweilen die alten: \label{K_L03211-9v}\edtext{Zwiſchen 20.
               u. 25. Juli{ }Wien\oindex{Wien@\textbf{Wien}, \emph{Verwaltungsgebiet}|pw}, dann \textsc{Trafoi\oindex{Trafoi@\textbf{Trafoi}|pw}}}{\lemma{\textnormal{\emph{Zwischen … Trafoi}}}\Cendnote{\textnormal{Dazu kam es nicht, siehe XXXX Auszeichnungsfehler: Dokument L03214 nicht gefunden.
               }}}\label{K_L03211-9}. Von \label{K_L03211-10v}\edtext{Fräulein \textsc{F.\pwindex{Goldmann, Eva Marie 27.\,10.\,1877 Wien – 2.\,11.\,1937 ebd.@\textsc{Goldmann, Eva Marie} (27.\,10.\,1877 Wien – 2.\,11.\,1937 ebd.)|pwv}}}{\lemma{\textnormal{\emph{Fräulein F.}}}\Cendnote{\textnormal{womöglich Goldmanns\pwindex{Goldmann, Paul 31.\,1.\,1865 Breslau – 25.\,9.\,1935 Wien@\textsc{Goldmann, Paul} (31.\,1.\,1865 Breslau – 25.\,9.\,1935 Wien), \emph{Schriftsteller, Journalist}|pwk} nachmalige Ehefrau Eva Marie Fränkel\pwindex{Goldmann, Eva Marie 27.\,10.\,1877 Wien – 2.\,11.\,1937 ebd.@\textsc{Goldmann, Eva Marie} (27.\,10.\,1877 Wien – 2.\,11.\,1937 ebd.)|pwk}, die aber in der Zeit bis zur
                  Eheschließung 1908 noch mit einem Herrn Kobler\pwindex{Kobler @\textsc{Kobler}|pwk} verheiratet war.}}}\label{K_L03211-10} erhalte ich hier und da
               einen Brief. Aber das Schreiben iſt eine dumme Sache. Die Fäden{ }ſind abgeriſſen. Sie{ }ſchreibt mir übrigens, daß{ }ſie öfter mit \textsc{Salten\pwindex{Salten, Felix 6.\,9.\,1869 Budapest – 8.\,10.\,1945 Zürich@\textsc{Salten, Felix} (6.\,9.\,1869 Budapest – 8.\,10.\,1945 Zürich), \emph{Schriftsteller, Journalist, Chefredakteur}|pw}} zuſammen iſt.\pend
           
\pstart
           Schreib’ mir bald wieder und{ }ſei vielmals und von Herzen gegrüßt! {\\[\baselineskip]}Dein {\\[\baselineskip]}\spacefill\mbox{Paul Goldmnn}\pend
           \leftskip=0em{}\selectlanguage{ngerman}\endnumbering\briefempfaengerindex{Schnitzler, Arthur@\textsc{Schnitzler, Arthur}!zzzGoldmann, Paul@\emph{von Paul Goldmann}!1902-06-162@{16. 6. [1902]}|)be}\mylabel{L03211h}  \newcommand{\dateiname}{L03211}\newcommand{\titel}{Paul Goldmann an Arthur Schnitzler, 16. 6. [1902]}\newcommand{\editorInnen}{Martin Anton Müller und Laura Untner}%% latex-leseansicht-abspann.tex
%% Abspann für die Leseansicht.
%% Der Schalter \ifkorrekturansicht ist bereits durch den Vorspann gesetzt.

%% latex-abspann.tex
%% Gemeinsamer Abspann für Korrekturansicht und Leseansicht.
%% Setzt den Schalter \ifkorrekturansicht voraus (gesetzt in den
%% einbindenden Dateien latex-korrekturansicht-abspann.tex bzw.
%% latex-leseansicht-abspann.tex).
%% ---------------------------------------------------------------

\normalsize

% Das esempio-Environment wird nur in der Leseansicht benötigt
\ifkorrekturansicht\else
\newenvironment{esempio}[3]%
{
    \vspace{1.5ex}
    \rlap{\underline{#1}}
    \par
    \setlength{\parindent}{0cm}
    \nopagebreak
    \leftskip=#2cm
    \rightskip=#3cm
}
{
    \par
}
\fi

\doendnotes{C}
\bigskip
\vfill

\clearpage

\footnotesize

\ifkorrekturansicht
  \lohead{\textsc{register}}
\fi

% theindex-Environment neu definieren ohne reledmac
\makeatletter
\renewenvironment{theindex}{%
  \ifkorrekturansicht
    \section*{\indexname}%
  \else
    \subsubsection*{Index der erwähnten Entitäten}%
  \fi
  \setlength{\parindent}{0pt}%
  \setlength{\parskip}{0pt plus 0.3pt}%
  \let\item\@idxitem
}{%
  \ifkorrekturansicht\clearpage\fi
}
\makeatother

\IfFileExists{\jobname-pw.ind}{\input{\jobname-pw.ind}}{}

% Quellenangabe nur in der Leseansicht
\ifkorrekturansicht\else
% Fallback-Definitionen, falls die .tex-Datei \titel etc. nicht gesetzt hat
\providecommand{\titel}{}
\providecommand{\editorInnen}{}
\providecommand{\dateiname}{\jobname}

\vspace{3cm}

\vfill

\footnotesize
\textsc{Quelle}: \titel. Herausgegeben von {\editorInnen}. In: \emph{Arthur Schnitzler: Briefwechsel mit Autorinnen und Autoren}.
 Digitale Edition, https://schnitzler-briefe.acdh.oeaw.ac.at/{\dateiname}.html (Stand \today)
\fi

\end{document}


