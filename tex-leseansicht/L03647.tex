%% latex-leseansicht-vorspann.tex
%% Vorspann für die Leseansicht.
%% Lädt die gemeinsame Datei latex-vorspann.tex mit nicht gesetztem Schalter.

\newif\ifkorrekturansicht
\korrekturansichtfalse

\input{../tex-inputs/latex-vorspann}


\section[Stefan Zweig an Arthur Schnitzler, {[}26. 11. 1914?{]}]{L03647 Stefan Zweig an Arthur Schnitzler, {[}26. 11. 1914?{]}}
\nopagebreak\mylabel{L03647v}
\rehead{ }\normalsize\beginnumbering\briefempfaengerindex{Schnitzler, Arthur@\textsc{Schnitzler, Arthur}!zzzZweig, Stefan@\emph{von Stefan Zweig}!1914-11-261@{{[}26. 11. 1914?{]}}|(be}
\toendnotes[C]{\smallbreak\pagebreak[2]}
\correspDesc{Versand  durch Stefan Zweig am [26. 11. 1914?] in Wien
\newline{}Erhalt  durch Arthur Schnitzler im Zeitraum [26. oder
                  27. 11. 1914?] in Wien}\toendnotes[C]{\smallbreak}
\Standort{CUL, Schnitzler, B 118.}
\physDesc{Briefkarte, 899 Zeichen
\newline{}Handschrift: blaue Tinte, lateinische Kurrent
\newline{}Schnitzler: mit rotem Buntstift eine Unterstreichung }
\buchAbdrucke{\weitereDrucke{Stefan Zweig: \emph{Briefwechsel mit Hermann Bahr, Sigmund Freud, Rainer Maria
                        Rilke und Arthur Schnitzler}. Herausgegeben von Jeffrey B. Berlin, Hans-Ulrich Lindken und Donald A. Prater. Frankfurt am Main: \emph{S. Fischer} 1987, S. 381–382.} }\toendnotes[C]{\smallbreak}
\pstart
           {\pb}\textcolor{gray}{\textbf{SZ}}\hfill \textcolor{gray}{\textbf{VIII. KOCHGASSE 8.\oindex{Wien@\textbf{Wien}!VIII., Josefstadt@\textbf{VIII., Josefstadt}!Kochgasse 8@\textbf{Kochgasse 8}, \emph{Wohngebäude}|pw}}}\pend
           \vspace{0.5em}
\pstart
           Verehrter Herr Doktor, ich bin sehr unglücklich: Sie haben mich
               vergebens angerufen. Aber ich unterschätzte das Militär und meinte, dass wenn man um
               6 Uhr früh ausrückte, das \label{K_L03647-1v}\edtext{Salutieren zu
               erlernen}{\lemma{\textnormal{\emph{Salutieren zu
               erlernen}}}\Cendnote{\textnormal{Am 12. 11. 1914 wurde Zweig\pwindex{Zweig, Stefan 28.\,11.\,1881 Wien – 23.\,2.\,1942 Petrópolis@\textsc{Zweig, Stefan} (28.\,11.\,1881 Wien – 23.\,2.\,1942 Petrópolis), \emph{Schriftsteller}|pwk} in den Militärdienst aufgenommen, am
                  14. 11. 1914 war er erstmals bei seiner vorläufigen Einsatzstelle in
                  Klosterneuburg\oindex{Klosterneuburg@\textbf{Klosterneuburg}, \emph{Hauptstadt}|pwk}, vom 23. bis 30. 11. 1914 vermerkte er im Tagebuch eine Woche zeitraubender
                  Exerzierübungen ebendort, vgl. Stefan Zweig\pwindex{Zweig, Stefan 28.\,11.\,1881 Wien – 23.\,2.\,1942 Petrópolis@\textsc{Zweig, Stefan} (28.\,11.\,1881 Wien – 23.\,2.\,1942 Petrópolis), \emph{Schriftsteller}|pwk}: \emph{Tagebuch im Kriegsjahr 1914}\pwindex{Zweig, Stefan 28.\,11.\,1881 Wien – 23.\,2.\,1942 Petrópolis@\textsc{Zweig, Stefan} (28.\,11.\,1881 Wien – 23.\,2.\,1942 Petrópolis), \emph{Schriftsteller}!Tagebuch im Kriegsjahr 1914 vom Tage der deutschen Kriegserklärung an Rußland@\strich\emph{Tagebuch im Kriegsjahr 1914 vom Tage der deutschen Kriegserklärung an Rußland}|pwk}. In:
                     https://stefanzweig.digital, SZ-AAP/L2.}}}\label{K_L03647-1}, um 12 Uhr schon zu Hause sein könnte. In Wirklichkeit wurde es 4
               Uhr und ich weiss noch nicht bestimmt, ob ich die Materie beherrsche. All das sind
               Vorbereitungen für meinen Dienst: am 1. Dez.{ }{\pb}bin ich ins Kriegsarchiv\orgindex{Kriegsarchiv@Kriegsarchiv|pw} einberufen und werde dort (unter Aufsicht von
                  Bartsch\pwindex{Bartsch, Rudolf Hans 11.\,2.\,1873 Graz – 7.\,2.\,1952 ebd.@\textsc{Bartsch, Rudolf Hans} (11.\,2.\,1873 Graz – 7.\,2.\,1952 ebd.), \emph{Schriftsteller}|pw} und Ginzkey\pwindex{Ginzkey, Franz Karl 8.\,9.\,1871 Pula – 11.\,4.\,1963 Wien@\textsc{Ginzkey, Franz Karl} (8.\,9.\,1871 Pula – 11.\,4.\,1963 Wien), \emph{Schriftsteller}|pw}) die vielfach geheimen Documente des Krieges zu ordnen
               und zu gestalten haben, eine Arbeit auf die ich mich so sehr freue wie nur möglich,
               obzwar sie viel fordert. So versäumte ich die Freude, Sie sprechen zu können, auch
               die nächsten Tage exerciere ich in Klosterneuburg\oindex{Klosterneuburg@\textbf{Klosterneuburg}, \emph{Hauptstadt}|pw}
               und bitte Sie darum, mir die \label{K_L03647-2v}\edtext{Berichtigung\pwindex{Schnitzler, Arthur 15.\,5.\,1862 Wien – 21.\,10.\,1931 ebd.@\textsc{Schnitzler, Arthur} (15.\,5.\,1862 Wien – 21.\,10.\,1931 ebd.), \emph{Schriftsteller, Mediziner}!Brief Artur Schnitzlers@\strich\emph{Ein Brief Artur Schnitzlers}|pwv}}{\lemma{\textnormal{\emph{Berichtigung}}}\Cendnote{\textnormal{Der Brief ist undatiert.
                  Schnitzler kontaktierte Zweig\pwindex{Zweig, Stefan 28.\,11.\,1881 Wien – 23.\,2.\,1942 Petrópolis@\textsc{Zweig, Stefan} (28.\,11.\,1881 Wien – 23.\,2.\,1942 Petrópolis), \emph{Schriftsteller}|pwk}, um mit ihm den Text der
                  Berichtigung\pwindex{Schnitzler, Arthur 15.\,5.\,1862 Wien – 21.\,10.\,1931 ebd.@\textsc{Schnitzler, Arthur} (15.\,5.\,1862 Wien – 21.\,10.\,1931 ebd.), \emph{Schriftsteller, Mediziner}!Brief Artur Schnitzlers@\strich\emph{Ein Brief Artur Schnitzlers}|pwkv} eines ihn diffamierenden Interviews\pwindex{\textcolor{red}{\textsuperscript{XXXX indx1}}!?? [Fiktives Interview aus St. Petersburg, 1914]@\strich\emph{?? [Fiktives Interview aus St. Petersburg, 1914]}|pwkv} durchzugehen, 
                  das in St. Petersburg\oindex{Sankt Petersburg@\textbf{Sankt Petersburg}|pwk} erschienen war. Bis zum 24. 11. 1914 feilte er nachweislich
                  am Text, was als der früheste Zeitpunkt, an dem dieses Schreiben verfasst sein kann, zu gelten hat. Da aber Zweig\pwindex{Zweig, Stefan 28.\,11.\,1881 Wien – 23.\,2.\,1942 Petrópolis@\textsc{Zweig, Stefan} (28.\,11.\,1881 Wien – 23.\,2.\,1942 Petrópolis), \emph{Schriftsteller}|pwk}
                  informiert zu sein scheint, dass der Text fertiggestellt war und um postalische Übermittelung bittet, dürfte
                  das Schreiben Schnitzlers vom XXXX Auszeichnungsfehler: Dokument L03774 nicht gefunden die unmittelbare Antwort
                  auf die vorliegende Karte darstellen. Die Dringlichkeit, die aus der Verwendung des Telefons durch Schnitzler
                  ablesbar ist, spricht dafür, dass dieser umgehend auf die vorliegende
                  Karte reagierte und nicht ein, zwei Tage zuwartete. Zweig\pwindex{Zweig, Stefan 28.\,11.\,1881 Wien – 23.\,2.\,1942 Petrópolis@\textsc{Zweig, Stefan} (28.\,11.\,1881 Wien – 23.\,2.\,1942 Petrópolis), \emph{Schriftsteller}|pwk} verfasste seine Karte
                  nach vier Uhr am Abend, so dass Schnitzler am nächsten Tag der Sekretärin
                  seine Antwort diktiert haben dürfte.}}}\label{K_L03647-2} brieflich zu senden – ich bin nicht mehr
               Herr meiner Zeit.\pend
           
\pstart
           Viele viele Grüsse Ihres aufrichtig getreuen{\\[\baselineskip]}\spacefill\mbox{Stefan Zweig}\pend
           \leftskip=0em{}\selectlanguage{ngerman}\endnumbering\briefempfaengerindex{Schnitzler, Arthur@\textsc{Schnitzler, Arthur}!zzzZweig, Stefan@\emph{von Stefan Zweig}!1914-11-261@{{[}26. 11. 1914?{]}}|)be}\mylabel{L03647h}  \newcommand{\dateiname}{L03647}\newcommand{\titel}{Stefan Zweig an Arthur Schnitzler, [26. 11. 1914?]}\newcommand{\editorInnen}{Selma Jahnke und Martin Anton Müller}%% latex-leseansicht-abspann.tex
%% Abspann für die Leseansicht.
%% Der Schalter \ifkorrekturansicht ist bereits durch den Vorspann gesetzt.

%% latex-abspann.tex
%% Gemeinsamer Abspann für Korrekturansicht und Leseansicht.
%% Setzt den Schalter \ifkorrekturansicht voraus (gesetzt in den
%% einbindenden Dateien latex-korrekturansicht-abspann.tex bzw.
%% latex-leseansicht-abspann.tex).
%% ---------------------------------------------------------------

\normalsize

% Das esempio-Environment wird nur in der Leseansicht benötigt
\ifkorrekturansicht\else
\newenvironment{esempio}[3]%
{
    \vspace{1.5ex}
    \rlap{\underline{#1}}
    \par
    \setlength{\parindent}{0cm}
    \nopagebreak
    \leftskip=#2cm
    \rightskip=#3cm
}
{
    \par
}
\fi

\doendnotes{C}
\bigskip
\vfill

\clearpage

\footnotesize

\ifkorrekturansicht
  \lohead{\textsc{register}}
\fi

% theindex-Environment neu definieren ohne reledmac
\makeatletter
\renewenvironment{theindex}{%
  \ifkorrekturansicht
    \section*{\indexname}%
  \else
    \subsubsection*{Index der erwähnten Entitäten}%
  \fi
  \setlength{\parindent}{0pt}%
  \setlength{\parskip}{0pt plus 0.3pt}%
  \let\item\@idxitem
}{%
  \ifkorrekturansicht\clearpage\fi
}
\makeatother

\IfFileExists{\jobname-pw.ind}{\input{\jobname-pw.ind}}{}

% Quellenangabe nur in der Leseansicht
\ifkorrekturansicht\else
% Fallback-Definitionen, falls die .tex-Datei \titel etc. nicht gesetzt hat
\providecommand{\titel}{}
\providecommand{\editorInnen}{}
\providecommand{\dateiname}{\jobname}

\vspace{3cm}

\vfill

\footnotesize
\textsc{Quelle}: \titel. Herausgegeben von {\editorInnen}. In: \emph{Arthur Schnitzler: Briefwechsel mit Autorinnen und Autoren}.
 Digitale Edition, https://schnitzler-briefe.acdh.oeaw.ac.at/{\dateiname}.html (Stand \today)
\fi

\end{document}


