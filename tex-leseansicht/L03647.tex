%% latex-korrekturansicht-vorspann.tex
%% Vorspann für die Korrekturansicht.
%% Lädt die gemeinsame Datei latex-vorspann.tex mit gesetztem Schalter.

\newif\ifkorrekturansicht
\korrekturansichttrue

\input{../tex-inputs/latex-vorspann}


\section[Stefan Zweig an Arthur Schnitzler, {[}zwischen 14. und 27. 11. 1914?{]}]{L03647 Stefan Zweig an Arthur Schnitzler, {[}zwischen 14. und
               27. 11. 1914?{]}}
\nopagebreak\mylabel{L03647v}
\rehead{ }\normalsize\beginnumbering\briefempfaengerindex{Schnitzler, Arthur@\textsc{Schnitzler, Arthur}!zzzZweig, Stefan@\emph{von Stefan Zweig}!1914-11-271@{{[}zwischen 14. und
                  27. 11. 1914?{]}}|(be}
\toendnotes[C]{\smallbreak\pagebreak[2]}\Standort{CUL, Schnitzler, B 118.}
\physDesc{Briefkarte, 900 Zeichen
\newline{}Handschrift: lila Tinte, lateinische Kurrent
\newline{}Schnitzler: mit rotem Buntstift Eine Unterstreichung }
\buchAbdrucke{\weitereDrucke{Stefan Zweig: \emph{Briefwechsel mit Hermann Bahr, Sigmund Freud, Rainer Maria
                        Rilke und Arthur Schnitzler}. Frankfurt am Main: \emph{S. Fischer} 1987, S. 381–382.} }\toendnotes[C]{\smallbreak}
\pstart
           {\pb}\textcolor{gray}{\textbf{SZ}}\hfill \textcolor{gray}{\textbf{VIII. KOCHGASSE 8.\oindex{Kochgasse 8@\textbf{Kochgasse 8}, \emph{Wohngebäude (K.WHS)}|pw}}}\pend
           \vspace{0.5em}
\pstart
           Verehrter Herr Doktor, ich bin sehr unglücklich: Sie haben mich
               vergebens angerufen. Aber ich unterschätzte das Militär und meinte, dass wenn man um
               6 Uhr früh ausrückte das \label{K_L03647-1v}\edtext{Salutieren zu
                  erlernen}{\lemma{\textnormal{\emph{Salutieren zu
                  erlernen}}}\Cendnote{\textnormal{Der undatierte Brief lässt
                  sich mit Hinweis auf die militärische Grundausbildung in der zweiten
                  Novemberhälfte verorten: am 12. 11. 1914 wurde Zweig\pwindex{Zweig, Stefan 28.11.1881 – 23.02.1942@\textsc{Zweig, Stefan} (28.11.1881 – 23.02.1942), \emph{Schriftsteller/Schriftstellerin}|pwk} in den Militärdienst aufgenommen, am
                     14. 11. 1914 war er erstmals bei seiner vorläufigen Einsatzstelle in
                     Klosterneuburg\oindex{Klosterneuburg@\textbf{Klosterneuburg}, \emph{P.PPLA3}|pwk}, vom 23. bis
                     30. 11. 1914 vermerkte er im Tagebuch eine Woche zeitraubender
                  Exerzierübungen ebendort, vgl. Stefan Zweig\pwindex{Zweig, Stefan 28.11.1881 – 23.02.1942@\textsc{Zweig, Stefan} (28.11.1881 – 23.02.1942), \emph{Schriftsteller/Schriftstellerin}|pwk}: \emph{Tagebuch im Kriegsjahr 1914}\pwindex{Tagebuch im Kriegsjahr 1914 vom Tage der deutschen Kriegserklaerung an Russland@\emph{Tagebuch im Kriegsjahr 1914 vom Tage der deutschen Kriegserklärung an Rußland}|pwk}. In:
                     https://stefanzweig.digital, SZ-AAP/L2. Der Ausblick, dass er auch die
                  nächsten Tage exerzieren müsse, lässt vermuten, dass der Brief einige Tage
                  vor Ablauf des Monats verfasst wurde, zumal Schnitzler in seinem Brief vom 27. 11. 1914 zu Zweigs\pwindex{Zweig, Stefan 28.11.1881 – 23.02.1942@\textsc{Zweig, Stefan} (28.11.1881 – 23.02.1942), \emph{Schriftsteller/Schriftstellerin}|pwk} neuem Arbeitsort
                  im \emph{Kriegsarchiv}\orgindex{Kriegsarchiv@Kriegsarchiv|pwk} gratulierte, von der Zweig\pwindex{Zweig, Stefan 28.11.1881 – 23.02.1942@\textsc{Zweig, Stefan} (28.11.1881 – 23.02.1942), \emph{Schriftsteller/Schriftstellerin}|pwk} ihn in diesem Brief
               berichtete.}}}\label{K_L03647-1}, um 12 Uhr schon zu Hause sein könnte. In Wirklichkeit wurde es 4
               Uhr und ich weiss noch nicht bestimmt, ob ich die Materie beherrsche. All das sind
               Vorbereitungen für meinen Dienst: am 1. Dez.{ }{\pb}bin ich ins Kriegsarchiv\orgindex{Kriegsarchiv@Kriegsarchiv|pw} einberufen und werde dort (unter Aufsicht von
                  Bartsch\pwindex{Bartsch, Rudolf Hans 11.02.1873 – 07.02.1952@\textsc{Bartsch, Rudolf Hans} (11.02.1873 – 07.02.1952), \emph{Schriftsteller/Schriftstellerin}|pw} und Ginzkey\pwindex{Ginzkey, Franz Karl 08.09.1871 – 11.04.1963@\textsc{Ginzkey, Franz Karl} (08.09.1871 – 11.04.1963), \emph{Schriftsteller/Schriftstellerin}|pw}) die vielfach geheimen Documente des Krieges zu ordnen
               und zu gestalten haben, eine Arbeit auf die ich mich so sehr freue wie nur möglich,
               obzwar sie viel fordert. So versäumte ich die Freude, Sie sprechen zu können, auch
               die nächsten Tage exerciere ich in Klosterneuburg\oindex{Klosterneuburg@\textbf{Klosterneuburg}, \emph{P.PPLA3}|pw}
               und bitte Sie darum, mir die Berichtigung brieflich zu senden – ich bin nicht mehr
               Herr meiner Zeit.\pend
           
\pstart
           Viele viele Grüsse Ihres aufrichtig getreuen{\\[\baselineskip]}\spacefill\mbox{Stefan Zweig}\pend
           \leftskip=0em{}\selectlanguage{ngerman}\endnumbering\briefempfaengerindex{Schnitzler, Arthur@\textsc{Schnitzler, Arthur}!zzzZweig, Stefan@\emph{von Stefan Zweig}!1914-11-141@{{[}zwischen 14. und
                  27. 11. 1914?{]}}|)be}\mylabel{L03647h}
\begin{anhang}
\end{anhang}\normalsize

\doendnotes{C}
\bigskip
\vfill

\clearpage

\footnotesize

\lohead{\textsc{register}}

% Definiere theindex-Environment komplett neu ohne reledmac
\makeatletter
\renewenvironment{theindex}{%
  \section*{\indexname}%
  \setlength{\parindent}{0pt}%
  \setlength{\parskip}{0pt plus 0.3pt}%
  \let\item\@idxitem
}{%
  \clearpage
}
\makeatother

\IfFileExists{\jobname-pw.ind}{\input{\jobname-pw.ind}}{}

\end{document}

      