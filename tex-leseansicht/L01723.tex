%% latex-leseansicht-vorspann.tex
%% Vorspann für die Leseansicht.
%% Lädt die gemeinsame Datei latex-vorspann.tex mit nicht gesetztem Schalter.

\newif\ifkorrekturansicht
\korrekturansichtfalse

\input{../tex-inputs/latex-vorspann}


\section[Olga Schnitzler an Richard Beer-Hofmann, {[}18. 10. 1907{]}]{L01723 Olga Schnitzler an Richard Beer-Hofmann, {[}18. 10. 1907{]}}
\nopagebreak\mylabel{L01723v}
\rehead{ }\normalsize\beginnumbering\briefempfaengerindex{Beer-Hofmann, Richard@\textsc{Beer-Hofmann, Richard}!zzzSchnitzler, Olga@\emph{von Olga Schnitzler}!1907-10-181@{{[}18. 10. 1907{]}}|(be}
\toendnotes[C]{\smallbreak\pagebreak[2]}
\correspDesc{Versand  durch Olga Schnitzler am [18. 10. 1907] in Wien
\newline{}Erhalt  durch Richard Beer-Hofmann am [18. 10. 1907] in Wien}\toendnotes[C]{\smallbreak}
\Standort{YCGL, MSS 31.}
\physDesc{Briefkarte, , Kuvert, 472 Zeichen
\newline{}Handschrift: schwarze Tinte, lateinische Kurrent
\newline{}Versand: ohne postalischen Übermittlungsvermerk }\toendnotes[C]{\smallbreak}\pstart{}{\pb}\textcolor{gray}{\textbf{O. S.}}\pend{}{\bigskip}\pstart{}{\pb}Herrn D\textsuperscript{r} Richard
                  Beer-Hofmann \pend{}{\bigskip}\vspace{1em}
\pstart
           {\pb}\textcolor{gray}{\textbf{O. S.}}\pend
           \vspace{0.5em}
\pstart
           Lieber Herr Doctor, ich habe gestern im \label{K_L01723-1v}\edtext{Antiquitäten-Geschäft}{\lemma{\textnormal{\emph{Antiquitäten-Geschäft}}}\Cendnote{\textnormal{Es dürfte sich um ein temporäres Geschäft aus dem Nachlass des
                     1904 verstorbenen Sammlers und Schätzmeisters Heinrich Cubasch\pwindex{Cubasch, Heinrich 1857/1858 – 13.\,8.\,1904 Wien@\textsc{Cubasch, Heinrich} (1857/1858 – 13.\,8.\,1904 Wien), \emph{Antiquitätenhändler, Schätzmeister}|pwk} gehandelt haben.}}}\label{K_L01723-1} im Gebäude des
                  Central-Bades\orgindex{Zentralbad@Zentralbad|pw}, Weihburggasse\oindex{Wien@\textbf{Wien}!I., Innere Stadt@\textbf{I., Innere Stadt}!Weihburggasse@\textbf{Weihburggasse}, \emph{Straße}|pw}, eine herrliche Spitze gesehen; sie hängt in
               der Auslage, hat ungefähr diese Form: {[}Umriss einer
                  Zigarrenspitze{]}\pend
           
\pstart
           {\pb}Es ist noch ein zweites, ebensolches Stück da, die
               beiden kosten 60 fl. Vielleicht interessieren Sie sich dafür. – Auf Wiedersehen
                  \label{K_L01723-2v}\edtext{morgen}{\lemma{\textnormal{\emph{morgen}}}\Cendnote{\textnormal{Das ermöglicht die Datierung. Vgl. A. S.: \emph{Tagebuch}, 19. 10. 1907.}}}\label{K_L01723-2} in der General-Probe der »Fledermaus\orgindex{Cabaret Fledermaus@Cabaret Fledermaus|pw}«.\pend
           
\pstart
           Vo\substVorne{}\textsuperscript{m}\substDazwischen{}n\substHinten{} uns zu Ihnen Beiden\pwindex{Beer-Hofmann, Paula 25.\,2.\,1879 Wien – 30.\,10.\,1939 Zürich@\textsc{Beer-Hofmann, Paula} (25.\,2.\,1879 Wien – 30.\,10.\,1939 Zürich)|pwv}
               die herzlichsten Grüsse!{\\[\baselineskip]}\spacefill\mbox{OlgaS.}\pend
           \leftskip=0em{}
\pstart
           Freitag.\pend
           \selectlanguage{ngerman}\endnumbering\briefempfaengerindex{Beer-Hofmann, Richard@\textsc{Beer-Hofmann, Richard}!zzzSchnitzler, Olga@\emph{von Olga Schnitzler}!1907-10-181@{{[}18. 10. 1907{]}}|)be}\mylabel{L01723h}  \newcommand{\dateiname}{L01723}\newcommand{\titel}{Olga Schnitzler an Richard Beer-Hofmann, [18. 10. 1907]}\newcommand{\editorInnen}{Martin Anton Müller und Gerd-Hermann Susen}%% latex-leseansicht-abspann.tex
%% Abspann für die Leseansicht.
%% Der Schalter \ifkorrekturansicht ist bereits durch den Vorspann gesetzt.

%% latex-abspann.tex
%% Gemeinsamer Abspann für Korrekturansicht und Leseansicht.
%% Setzt den Schalter \ifkorrekturansicht voraus (gesetzt in den
%% einbindenden Dateien latex-korrekturansicht-abspann.tex bzw.
%% latex-leseansicht-abspann.tex).
%% ---------------------------------------------------------------

\normalsize

% Das esempio-Environment wird nur in der Leseansicht benötigt
\ifkorrekturansicht\else
\newenvironment{esempio}[3]%
{
    \vspace{1.5ex}
    \rlap{\underline{#1}}
    \par
    \setlength{\parindent}{0cm}
    \nopagebreak
    \leftskip=#2cm
    \rightskip=#3cm
}
{
    \par
}
\fi

\doendnotes{C}
\bigskip
\vfill

\clearpage

\footnotesize

\ifkorrekturansicht
  \lohead{\textsc{register}}
\fi

% theindex-Environment neu definieren ohne reledmac
\makeatletter
\renewenvironment{theindex}{%
  \ifkorrekturansicht
    \section*{\indexname}%
  \else
    \subsubsection*{Index der erwähnten Entitäten}%
  \fi
  \setlength{\parindent}{0pt}%
  \setlength{\parskip}{0pt plus 0.3pt}%
  \let\item\@idxitem
}{%
  \ifkorrekturansicht\clearpage\fi
}
\makeatother

\IfFileExists{\jobname-pw.ind}{\input{\jobname-pw.ind}}{}

% Quellenangabe nur in der Leseansicht
\ifkorrekturansicht\else
% Fallback-Definitionen, falls die .tex-Datei \titel etc. nicht gesetzt hat
\providecommand{\titel}{}
\providecommand{\editorInnen}{}
\providecommand{\dateiname}{\jobname}

\vspace{3cm}

\vfill

\footnotesize
\textsc{Quelle}: \titel. Herausgegeben von {\editorInnen}. In: \emph{Arthur Schnitzler: Briefwechsel mit Autorinnen und Autoren}.
 Digitale Edition, https://schnitzler-briefe.acdh.oeaw.ac.at/{\dateiname}.html (Stand \today)
\fi

\end{document}


