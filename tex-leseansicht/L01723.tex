%% latex-korrekturansicht-vorspann.tex
%% Vorspann für die Korrekturansicht.
%% Lädt die gemeinsame Datei latex-vorspann.tex mit gesetztem Schalter.

\newif\ifkorrekturansicht
\korrekturansichttrue

\input{../tex-inputs/latex-vorspann}


\section[Olga Schnitzler an Richard Beer-Hofmann, {[}18. 10. 1907{]}]{L01723 Olga Schnitzler an Richard Beer-Hofmann, {[}18. 10. 1907{]}}
\nopagebreak\mylabel{L01723v}
\rehead{ }\normalsize\beginnumbering\briefempfaengerindex{Beer-Hofmann, Richard@\textsc{Beer-Hofmann, Richard}!zzzSchnitzler, Olga@\emph{von Olga Schnitzler}!1907-10-181@{{[}18. 10. 1907{]}}|(be}
\toendnotes[C]{\smallbreak\pagebreak[2]}\Standort{YCGL, MSS 31.}
\physDesc{Briefkarte, , Umschlag, 472 Zeichen
\newline{}Handschrift: schwarze Tinte, lateinische Kurrent
\newline{}Versand: ohne postalischen Übermittlungsvermerk }\toendnotes[C]{\smallbreak}\pstart{}{\pb}\textcolor{gray}{\textbf{O. S.}}\pend{}{\bigskip}\pstart{}{\pb}Herrn D\textsuperscript{r} Richard
                  Beer-Hofmann \pend{}{\bigskip}\vspace{1em}
\pstart
           {\pb}\textcolor{gray}{\textbf{O. S.}}\pend
           \vspace{0.5em}
\pstart
           Lieber Herr Doctor, ich habe gestern im \label{K_L01723-1v}\edtext{Antiquitäten-Geschäft}{\lemma{\textnormal{\emph{Antiquitäten-Geschäft}}}\Cendnote{\textnormal{Es dürfte sich um ein temporäres Geschäft aus dem Nachlass des
                     1904 verstorbenen Sammlers und Schätzmeisters Heinrich Cubasch\pwindex{Cubasch, Heinrich 1857/1858 – 1904-08-13@\textsc{Cubasch, Heinrich} (1857/1858 – 1904-08-13), \emph{Antiquitätenhändler/Antiquitätenhändlerin, Schätzmeister/Schätzmeisterin}|pwk} gehandelt haben.}}}\label{K_L01723-1} im Gebäude des
                  Central-Bades\orgindex{Zentralbad@Zentralbad|pw}, Weihburggasse\oindex{Weihburggasse@\textbf{Weihburggasse}, \emph{Straße (K.STR)}|pw}, eine herrliche Spitze gesehen; sie hängt in
               der Auslage, hat ungefähr diese Form: {[}Umriss einer
                  Zigarrenspitze{]}\pend
           
\pstart
           {\pb}Es ist noch ein zweites, ebensolches Stück da, die
               beiden kosten 60 fl. Vielleicht interessieren Sie sich dafür. – Auf Wiedersehen
                  \label{K_L01723-2v}\edtext{morgen}{\lemma{\textnormal{\emph{morgen}}}\Cendnote{\textnormal{Das ermöglicht die Datierung. Vgl. A. S.: \emph{Tagebuch}, 19. 10. 1907.}}}\label{K_L01723-2} in der General-Probe der »Fledermaus\orgindex{Cabaret Fledermaus@Cabaret Fledermaus|pw}«.\pend
           
\pstart
           Vo\substVorne{}\textsuperscript{m}\substDazwischen{}n\substHinten{} uns zu Ihnen Beiden\pwindex{Beer-Hofmann, Paula 25.02.1879 – 30.10.1939@\textsc{Beer-Hofmann, Paula} (25.02.1879 – 30.10.1939)|pwv}
               die herzlichsten Grüsse!{\\[\baselineskip]}\spacefill\mbox{OlgaS.}\pend
           \leftskip=0em{}
\pstart
           Freitag.\pend
           \selectlanguage{ngerman}\endnumbering\briefempfaengerindex{Beer-Hofmann, Richard@\textsc{Beer-Hofmann, Richard}!zzzSchnitzler, Olga@\emph{von Olga Schnitzler}!1907-10-181@{{[}18. 10. 1907{]}}|)be}\mylabel{L01723h}  \normalsize

\doendnotes{C}
\bigskip
\vfill

\clearpage

\footnotesize

\lohead{\textsc{register}}

% Definiere theindex-Environment komplett neu ohne reledmac
\makeatletter
\renewenvironment{theindex}{%
  \section*{\indexname}%
  \setlength{\parindent}{0pt}%
  \setlength{\parskip}{0pt plus 0.3pt}%
  \let\item\@idxitem
}{%
  \clearpage
}
\makeatother

\IfFileExists{\jobname-pw.ind}{\input{\jobname-pw.ind}}{}

\end{document}

      