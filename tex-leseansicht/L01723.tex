%% latex-leseansicht-vorspann.tex
%% Vorspann für die Leseansicht.
%% Lädt die gemeinsame Datei latex-vorspann.tex mit nicht gesetztem Schalter.

\newif\ifkorrekturansicht
\korrekturansichtfalse

\input{../tex-inputs/latex-vorspann}


         
         \newcommand{\erwaehntePersonen}{Personen: Richard Beer-Hofmann, Paula Beer-Hofmann, Heinrich Cubasch}
         \newcommand{\erwaehnteInstitutionen}{Institutionen: Cabaret Fledermaus, Zentralbad}
         \newcommand{\erwaehnteOrte}{Orte: Weihburggasse, Wien}
         \newcommand{\erwaehnteWerke}{
               \section[Olga Schnitzler an Richard Beer-Hofmann, {[}18. 10. 1907{]}]{ Olga Schnitzler an Richard Beer-Hofmann, {[}18. 10. 1907{]}}\nopagebreak\mylabel{v}\rehead{ }\begin{ledgroupsized}[t]{13cm}\normalsize\beginnumbering \toendnotes[C]{\smallbreak\pagebreak[2]} \Standort{YCGL, MSS 31.}
\physDesc{Briefkarte, Umschlag
\newline{}Handschrift: schwarze Tinte, lateinische Kurrent\newline{}Versand: ohne postalischen Übermittlungsvermerk }\toendnotes[C]{\smallbreak}\pstart{}{\pb}\textcolor{gray}{\textbf{O. S.}}\pend{}{\bigskip}\pstart{}{\pb}Herrn D\textsuperscript{r} Richard
                  Beer-Hofmann \pend{}{\bigskip}\pstart
           \noindent{}{\pb}\textcolor{gray}{\textbf{O. S.}}\pend
           \pstart
           Lieber Herr Doctor, ich habe gestern im \label{K_L01723_1v}\edtext{Antiquitäten-Geschäft}{\lemma{\textnormal{\emph{Antiquitäten-Geschäft}}}\Cendnote{\textnormal{Es dürfte sich um ein temporäres Geschäft aus dem Nachlass des
                     1904 verstorbenen Sammlers und Schätzmeisters Heinrich Cubasch\pwindex{Cubasch, Heinrich 1857/1858 – 1904-08-13@\textsc{Cubasch, Heinrich} (1857/1858 – 1904-08-13), \emph{Antiquitätenhändler, Schätzmeister}|pwk} gehandelt haben.}}}\label{K_L01723_1h} im Gebäude des Central-Bades\orgindex{Zentralbad@Zentralbad|pw}, Weihburggasse\oindex{Weihburggasse@\textbf{Weihburggasse}|pw}, eine herrliche Spitze gesehen; sie hängt in der Auslage, hat
               ungefähr diese Form: {[}Umriss einer Zigarrenspitze{]}\pend
           \pstart
           {\pb}Es ist noch ein zweites, ebensolches Stück da, die
               beiden kosten 60 fl. Vielleicht interessieren Sie sich dafür. – Auf Wiedersehen
                  \label{K_L01723_2v}\edtext{morgen}{\lemma{\textnormal{\emph{morgen}}}\Cendnote{\textnormal{Das ermöglicht die Datierung. Vgl. A. S.: \emph{Tagebuch}, 19. 10. 1907}}}\label{K_L01723_2h} in der General-Probe der »Fledermaus\orgindex{Cabaret Fledermaus@Cabaret Fledermaus|pw}«.\pend
           \pstart
           Vo\substVorne{}\textsuperscript{m}\substDazwischen{}n\substHinten{} uns zu Ihnen Beiden\pwindex{Beer-Hofmann, Paula 25.02.1879 – 30.10.1939@\textsc{Beer-Hofmann, Paula} (25.02.1879 – 30.10.1939)|pwv}
               die herzlichsten Grüsse!{\\[\baselineskip]}\spacefill\mbox{OlgaS.}\pend
           \leftskip=0em{}\pstart
           Freitag.\pend
           
         
         \endnumbering\mylabel{h}\end{ledgroupsized}  \newcommand{\dateiname}{L01723}\newcommand{\titel}{Olga Schnitzler an Richard Beer-Hofmann, [18. 10. 1907]}\newcommand{\editorInnen}{Martin Anton Müller und Gerd-Hermann Susen}%% latex-leseansicht-abspann.tex
%% Abspann für die Leseansicht.
%% Der Schalter \ifkorrekturansicht ist bereits durch den Vorspann gesetzt.

%% latex-abspann.tex
%% Gemeinsamer Abspann für Korrekturansicht und Leseansicht.
%% Setzt den Schalter \ifkorrekturansicht voraus (gesetzt in den
%% einbindenden Dateien latex-korrekturansicht-abspann.tex bzw.
%% latex-leseansicht-abspann.tex).
%% ---------------------------------------------------------------

\normalsize

% Das esempio-Environment wird nur in der Leseansicht benötigt
\ifkorrekturansicht\else
\newenvironment{esempio}[3]%
{
    \vspace{1.5ex}
    \rlap{\underline{#1}}
    \par
    \setlength{\parindent}{0cm}
    \nopagebreak
    \leftskip=#2cm
    \rightskip=#3cm
}
{
    \par
}
\fi

\doendnotes{C}
\bigskip
\vfill

\clearpage

\footnotesize

\ifkorrekturansicht
  \lohead{\textsc{register}}
\fi

% theindex-Environment neu definieren ohne reledmac
\makeatletter
\renewenvironment{theindex}{%
  \ifkorrekturansicht
    \section*{\indexname}%
  \else
    \subsubsection*{Index der erwähnten Entitäten}%
  \fi
  \setlength{\parindent}{0pt}%
  \setlength{\parskip}{0pt plus 0.3pt}%
  \let\item\@idxitem
}{%
  \ifkorrekturansicht\clearpage\fi
}
\makeatother

\IfFileExists{\jobname-pw.ind}{\input{\jobname-pw.ind}}{}

% Quellenangabe nur in der Leseansicht
\ifkorrekturansicht\else
% Fallback-Definitionen, falls die .tex-Datei \titel etc. nicht gesetzt hat
\providecommand{\titel}{}
\providecommand{\editorInnen}{}
\providecommand{\dateiname}{\jobname}

\vspace{3cm}

\vfill

\footnotesize
\textsc{Quelle}: \titel. Herausgegeben von {\editorInnen}. In: \emph{Arthur Schnitzler: Briefwechsel mit Autorinnen und Autoren}.
 Digitale Edition, https://schnitzler-briefe.acdh.oeaw.ac.at/{\dateiname}.html (Stand \today)
\fi

\end{document}


      