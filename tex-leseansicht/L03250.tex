%% latex-korrekturansicht-vorspann.tex
%% Vorspann für die Korrekturansicht.
%% Lädt die gemeinsame Datei latex-vorspann.tex mit gesetztem Schalter.

\newif\ifkorrekturansicht
\korrekturansichttrue

\input{../tex-inputs/latex-vorspann}


\section[ Paul Goldmann an Arthur Schnitzler, 14. 9. 1906]{L03250 Paul Goldmann an Arthur Schnitzler, 14. 9. 1906}
\nopagebreak\mylabel{L03250v}
\rehead{ }\normalsize\beginnumbering\briefempfaengerindex{Schnitzler, Arthur@\textsc{Schnitzler, Arthur}!zzzGoldmann, Paul@\emph{von Paul Goldmann}!1906-09-141@{14. 9. 1906}|(be}
\toendnotes[C]{\smallbreak\pagebreak[2]}\Standort{DLA, A:Schnitzler, HS.NZ85.1.3175.}
\physDesc{Postkarte, 179 Zeichen
\newline{}Handschrift: 1) schwarze Tinte, deutsche Kurrent\hspace{1em}2) schwarze Tinte, lateinische Kurrent (\noindent{}Adresse)\hspace{1em}
\newline{}Versand: 1) Stempel: »\nobreak{}1/\textsubscript{1} Wien 16, 14. IX. 06, 7\nobreak{}«.   2) Stempel: »\nobreak{}\oindex{XVIII., Waehring@\textbf{XVIII., Währing}, \emph{A.ADM3}|pwk}18/\textsubscript{1} Wien, 15. IX. 06, Be{[}stellt{]}\nobreak{}«. 
\newline{}Schnitzler: mit Bleistift das Jahr »906« vermerkt }\toendnotes[C]{\smallbreak}\pstart{}{\pb}Herrn\pend{}\pstart{}Dr. Arthur Schnitzler\pend{}\pstart{}Wien\oindex{Wien@\textbf{Wien}, \emph{A.ADM2}|pw}\pend{}\pstart{}XVIII. Spöttelgaſse 7\oindex{Edmund-Weiss-Gasse 7@\textbf{Edmund-Weiß-Gasse 7}, \emph{Wohngebäude (K.WHS)}|pw}.\pend{}{\bigskip}\vspace{1em}
\pstart
           \noindent{}{\pb}\textsc{Wien\oindex{Wien@\textbf{Wien}, \emph{A.ADM2}|pw}}, \textsc{14. Sept}. Lieber Freund, Biſt Du in Wien\oindex{Wien@\textbf{Wien}, \emph{A.ADM2}|pw}? Und wenn ja, – wann darf ich Dich \label{K_L03250-1v}\edtext{beſuchen}{\lemma{\textnormal{\emph{beſuchen}}}\Cendnote{\textnormal{Ein Treffen kam nicht zustande, da sich Schnitzler zwischen 10. 9. 1906 und 20. 9. 1906 auf dem Semmering\oindex{Semmering@\textbf{Semmering}, \emph{A.ADM3}|pwk} aufhielt. Siehe auch Paul Goldmann an Arthur Schnitzler, 18. 9. [1906].}}}\label{K_L03250-1}?\pend
           
\pstart
           Herzl. Gruß! {\\[\baselineskip]}Dein \spacefill\mbox{Paul Goldmann}\pend
           \leftskip=0em{}
\pstart
           \noindent{}\raggedleft{}\textsc{Grand Hotel\oindex{Grand Hotel Wien@\textbf{Grand Hotel Wien}, \emph{Hotel (K.HTL)}|pw}}.\pend
           \selectlanguage{ngerman}\endnumbering\briefempfaengerindex{Schnitzler, Arthur@\textsc{Schnitzler, Arthur}!zzzGoldmann, Paul@\emph{von Paul Goldmann}!1906-09-141@{14. 9. 1906}|)be}\mylabel{L03250h}  \normalsize

\doendnotes{C}
\bigskip
\vfill

\clearpage

\footnotesize

\lohead{\textsc{register}}

% Definiere theindex-Environment komplett neu ohne reledmac
\makeatletter
\renewenvironment{theindex}{%
  \section*{\indexname}%
  \setlength{\parindent}{0pt}%
  \setlength{\parskip}{0pt plus 0.3pt}%
  \let\item\@idxitem
}{%
  \clearpage
}
\makeatother

\IfFileExists{\jobname-pw.ind}{\input{\jobname-pw.ind}}{}

\end{document}

      