%% latex-korrekturansicht-vorspann.tex
%% Vorspann für die Korrekturansicht.
%% Lädt die gemeinsame Datei latex-vorspann.tex mit gesetztem Schalter.

\newif\ifkorrekturansicht
\korrekturansichttrue

\input{../tex-inputs/latex-vorspann}


\section[Hugo von Hofmannsthal an Arthur Schnitzler, {[}20. 3. 1899{]}]{L00907 Hugo von Hofmannsthal an Arthur Schnitzler, {[}20. 3. 1899{]}}
\nopagebreak\mylabel{L00907v}
\rehead{ }\normalsize\beginnumbering\briefempfaengerindex{Schnitzler, Arthur@\textsc{Schnitzler, Arthur}!zzzHofmannsthal, Hugo von@\emph{von Hugo von Hofmannsthal}!1899-03-203@{{[}20. 3. 1899{]}}|(be}
\toendnotes[C]{\smallbreak\pagebreak[2]}\Standort{CUL, Schnitzler, B 43.}
\physDesc{Brief, 1 Blatt, 1 Seite, 310 Zeichen
\newline{}Handschrift: Bleistift, deutsche Kurrent
\newline{}Schnitzler: mit Bleistift datiert: »am 20 März
                                    99.« 
\newline{}Ordnung: 1) mit Bleistift von unbekannter Hand nummeriert: »\strikeout{142}«  2) mit Bleistift von unbekannter Hand nummeriert:
                                    »139«}
\buchAbdrucke{\weitereDrucke{Hugo von Hofmannsthal, Arthur Schnitzler: \emph{Briefwechsel}. Frankfurt am Main: \emph{S. Fischer} 1964, S. 119.} }\toendnotes[C]{\smallbreak}
\pstart{}{\pb}mein guter lieber
                  Arthur\pend\vspace{0.5em}
\pstart
           es thut mir ſo unausſprechlich \label{K_L00907-1v}\edtext{leid um
                  Sie}{\lemma{\textnormal{\emph{leid um
                  Sie}}}\Cendnote{\textnormal{Schnitzler trauerte um seine langjährige
                  Partnerin Marie Reinhard\pwindex{Reinhard, Marie 1871-03-13 – 1899-03-18@\textsc{Reinhard, Marie} (1871-03-13 – 1899-03-18), \emph{Gesangspädagoge/Gesangspädagogin}|pwk}, die am
                     18. 3. 1899 an Sepsis gestorben war.}}}\label{K_L00907-1}, und ich kann nicht
               einmal ein biſſl um Sie ſein, ich denk faſt den ganzen Tag an Sie. Heut war meine
                  \label{K_L00907-2v}\edtext{\textsc{Promotion}}{\lemma{\textnormal{\emph{Promotion}}}\Cendnote{\textnormal{Die Arbeit war betitelt: \emph{Über den Sprachgebrauch bei den Dichtern der
                     Pléjade}\pwindex{Ueber den Sprachgebrauch bei den Dichtern der Plejade@\emph{Über den Sprachgebrauch bei den Dichtern der Pléjade}|pwk}.}}}\label{K_L00907-2}, von morgen bin ich in \textsc{Berlin\oindex{Berlin@\textbf{Berlin}, \emph{P.PPLC}|pw}}\pend
           
\pstart
           \centering{}\textsc{Hotel Windsor\oindex{Hotel Windsor@\textbf{Hotel Windsor}, \emph{Hotel (K.HTL)}|pw}\hspace*{1em}Behrenstrasse\oindex{Behrenstrasse@\textbf{Behrenstraße}, \emph{Straße (K.STR)}|pw}}.\pend
           
\pstart
           Bitte \uuline{bitte}{ }ſchreiben Sie mir und arbeiten Sie, zwingen Sie
               ſich.\pend
           
\pstart
           Ihr alter{\\[\baselineskip]}\spacefill\mbox{Hugo}\pend
           \leftskip=0em{}\selectlanguage{ngerman}\endnumbering\briefempfaengerindex{Schnitzler, Arthur@\textsc{Schnitzler, Arthur}!zzzHofmannsthal, Hugo von@\emph{von Hugo von Hofmannsthal}!1899-03-203@{{[}20. 3. 1899{]}}|)be}\mylabel{L00907h}  \normalsize

\doendnotes{C}
\bigskip
\vfill

\clearpage

\footnotesize

\lohead{\textsc{register}}

% Definiere theindex-Environment komplett neu ohne reledmac
\makeatletter
\renewenvironment{theindex}{%
  \section*{\indexname}%
  \setlength{\parindent}{0pt}%
  \setlength{\parskip}{0pt plus 0.3pt}%
  \let\item\@idxitem
}{%
  \clearpage
}
\makeatother

\IfFileExists{\jobname-pw.ind}{\input{\jobname-pw.ind}}{}

\end{document}

      