%% latex-korrekturansicht-vorspann.tex
%% Vorspann für die Korrekturansicht.
%% Lädt die gemeinsame Datei latex-vorspann.tex mit gesetztem Schalter.

\newif\ifkorrekturansicht
\korrekturansichttrue

\input{../tex-inputs/latex-vorspann}


\section[Olga Schnitzler an Paula Beer-Hofmann, {[}1912?{]}]{L02566 Olga Schnitzler an Paula Beer-Hofmann, {[}1912?{]}}
\nopagebreak\mylabel{L02566v}
\rehead{ }\normalsize\beginnumbering\briefempfaengerindex{Beer-Hofmann, Paula@\textsc{Beer-Hofmann, Paula}!zzzSchnitzler, Olga@\emph{von Olga Schnitzler}!1912-12-311@{{[}1912?{]}}|(be}
\toendnotes[C]{\smallbreak\pagebreak[2]}\Standort{YCGL, MSS 31.}
\physDesc{Briefkarte, , Umschlag, 227 Zeichen
\newline{}Handschrift: schwarze Tinte, lateinische Kurrent
\newline{}Versand: ohne postalischen Übermittlungsvermerk 
\newline{}Ordnung: mit Bleistift von unbekannter Hand datiert: »1912« }\toendnotes[C]{\smallbreak}\pstart{}{\pb}Frau Paula Beer-Hofmann\pend{}{\bigskip}\vspace{1em}
\pstart
           \noindent{}{\pb}Liebe Paula, Ihr Fräulein\pwindex{Cerino, Maria Anna *~17.9.1888@\textsc{Cerino, Maria Anna} (*~17.9.1888), \emph{Kinderbetreuer/Kinderbetreuerin}|pwuv} hat nur den Mantel
               mitgeschickt, – ich bitte Sie, dem Frl. Reiter\pwindex{Reiter, Anna @\textsc{Reiter, Anna}, \emph{Hausschneider/Hausschneiderin}|pw}
               auch das Unterkleid zu zeigen wegen des Verschlusses, und damit ich weiss, wieviel
               Seide {\pb}ich brauche.\pend
           
\pstart
           Vielen Dank!\pend
           \pstart \spacefill\mbox{Olga.}\pend{}\selectlanguage{ngerman}\endnumbering\briefempfaengerindex{Beer-Hofmann, Paula@\textsc{Beer-Hofmann, Paula}!zzzSchnitzler, Olga@\emph{von Olga Schnitzler}!1912-01-011@{{[}1912?{]}}|)be}\mylabel{L02566h}  \normalsize

\doendnotes{C}
\bigskip
\vfill

\clearpage

\footnotesize

\lohead{\textsc{register}}

% Definiere theindex-Environment komplett neu ohne reledmac
\makeatletter
\renewenvironment{theindex}{%
  \section*{\indexname}%
  \setlength{\parindent}{0pt}%
  \setlength{\parskip}{0pt plus 0.3pt}%
  \let\item\@idxitem
}{%
  \clearpage
}
\makeatother

\IfFileExists{\jobname-pw.ind}{\input{\jobname-pw.ind}}{}

\end{document}

      