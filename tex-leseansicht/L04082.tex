%% latex-leseansicht-vorspann.tex
%% Vorspann für die Leseansicht.
%% Lädt die gemeinsame Datei latex-vorspann.tex mit nicht gesetztem Schalter.

\newif\ifkorrekturansicht
\korrekturansichtfalse

\input{../tex-inputs/latex-vorspann}


\section[Arthur Schnitzler an Gustav Schwarzkopf, 8. 10. 1899]{L04082 Arthur Schnitzler an Gustav Schwarzkopf, 8. 10. 1899}
\nopagebreak\mylabel{L04082v}
\rehead{ }\normalsize\beginnumbering\briefempfaengerindex{Schwarzkopf, Gustav@\textsc{Schwarzkopf, Gustav}!zzzSchnitzler, Arthur@\emph{von Arthur Schnitzler}!1899-10-083@{8. 10. 1899}|(be}
\toendnotes[C]{\smallbreak\pagebreak[2]}
\correspDesc{Versand  durch Arthur Schnitzler am 8. 10. 1899 in Berlin
\newline{}Erhalt  durch Gustav Schwarzkopf im Zeitraum [9. 10. 1899 – 13. 10. 1899?] in Wien}\toendnotes[C]{\smallbreak}
\Standort{CUL, Schnitzler, B 96.}
\physDesc{Briefkarte, 721 Zeichen
\newline{}Handschrift: schwarze Tinte, deutsche Kurrent}\toendnotes[C]{\smallbreak}
\pstart
           \raggedleft{}{\pb}\textsc{Berlin\oindex{Berlin@\textbf{Berlin}, \emph{Hauptstadt}|pw}}{ }8. 10. 99.\pend
           \vspace{0.5em}
\pstart
           lieber Guſtav, meine Abſicht iſts, \label{K_L04082-1v}\edtext{Donnerſtg oder
               Freitag}{\lemma{\textnormal{\emph{Donnerstg oder
               Freitag}}}\Cendnote{\textnormal{Er kam bereits am Donnerstag, dem 12. 10. 1899 in Wien\oindex{Wien@\textbf{Wien}, \emph{Verwaltungsgebiet}|pwk} an und traf
               noch am selben Tag mit Schwarzkopf\pwindex{Schwarzkopf, Gustav 7.\,11.\,1853 Wien – 13.\,11.\,1939 ebd.@\textsc{Schwarzkopf, Gustav} (7.\,11.\,1853 Wien – 13.\,11.\,1939 ebd.), \emph{Schriftsteller}|pwk} zusammen.}}}\label{K_L04082-1} in Wien\oindex{Wien@\textbf{Wien}, \emph{Verwaltungsgebiet}|pw} zu ſein. \label{K_L04082-2v}\edtext{Paul Goldma{\geminationn}\pwindex{Goldmann, Paul 31.\,1.\,1865 Breslau – 25.\,9.\,1935 Wien@\textsc{Goldmann, Paul} (31.\,1.\,1865 Breslau – 25.\,9.\,1935 Wien), \emph{Schriftsteller, Journalist}|pw} ko{\geminationm}t ungefähr am gleichen Tag}{\lemma{\textnormal{\emph{Paul … Tag}}}\Cendnote{\textnormal{Vgl. XXXX Auszeichnungsfehler: Dokument L02683 nicht gefunden.}}}\label{K_L04082-2} aus Florenz\oindex{Luzern@\textbf{Luzern}|pw} an, und wird \label{K_L04082-3v}\edtext{acht Tage}{\lemma{\textnormal{\emph{acht Tage}}}\Cendnote{\textnormal{Er blieb bis zum 21. 10. 1899.}}}\label{K_L04082-3} in Wien\oindex{Wien@\textbf{Wien}, \emph{Verwaltungsgebiet}|pw},
               bei mir, wohnen. So ſpar ich mir die Berlin\oindex{Berlin@\textbf{Berlin}, \emph{Hauptstadt}|pw}er
               Berichte auf unſre nächſte mündliche Unterhaltg auf. Die Damen G.\pwindex{Schnitzler, Olga 17.\,1.\,1882 Wien – 13.\,1.\,1970 Lugano@\textsc{Schnitzler, Olga} (17.\,1.\,1882 Wien – 13.\,1.\,1970 Lugano), \emph{Schauspielerin, Sängerin}|pw}\pwindex{Steinrück, Elisabeth 19.\,11.\,1885 Wien – 7.\,4.\,1920 Partenkirchen@\textsc{Steinrück, Elisabeth} (19.\,11.\,1885 Wien – 7.\,4.\,1920 Partenkirchen)|pw} grüßen beſtens zurück. – Heut vor einem Jahr war {\pb}hier die Vermächtnis\pwindex{Schnitzler, Arthur 15. 5. 1862 Wien – 21. 10. 1931 ebd.@\textsc{Schnitzler, Arthur} (15. 5. 1862 Wien – 21. 10. 1931 ebd.), \emph{Schriftsteller, Mediziner}!Vermächtnis. Schauspiel in drei Akten@\strich\emph{Das Vermächtnis. Schauspiel in drei Akten}|pw}{ }\textsc{Première}\eventindex{Deutsches Theater Berlin@\textbf{Deutsches Theater Berlin}!Uraufführung von Das Vermächtnis, 8.10.1898@Uraufführung von Das Vermächtnis, 8.10.1898|pw}! Lang lang iſts her!– Dafür hab ich geſtern dem Brahm\pwindex{Brahm, Otto 5.\,2.\,1856 Hamburg – 28.\,11.\,1912 Berlin@\textsc{Brahm, Otto} (5.\,2.\,1856 Hamburg – 28.\,11.\,1912 Berlin), \emph{Theaterleiter, Regisseur}|pw} die \textsc{Beatrice\pwindex{Schnitzler, Arthur 15. 5. 1862 Wien – 21. 10. 1931 ebd.@\textsc{Schnitzler, Arthur} (15. 5. 1862 Wien – 21. 10. 1931 ebd.), \emph{Schriftsteller, Mediziner}!Schleier der Beatrice. Schauspiel in fünf Akten@\strich\emph{Der Schleier der Beatrice. Schauspiel in fünf Akten}|pw}}{ }vorgeleſen\eventindex{Luisenplatz 2@\textbf{Luisenplatz 2}!Private Lesung von Der Schleier der Beatrice, 7.10.1899@Private Lesung von Der Schleier der Beatrice, 7.10.1899|pwv}; mir ko{\geminationm}t vor, ſie hat einen gewiſſen
               Eindruck auf ihn gemacht. Aber ich gebe ſie noch nicht her; habe noch manches daran
               zu thun. Überdies iſt ſie hier nahezu unſpielbar. Ich las (mit einer Unterbrechung,
               Souper,) Von 7 – Mitternacht. –\pend
           
\pstart
           leben Sie wohl und auf baldg Wiederſehn!{\\[\baselineskip]}Herzlichſt Ihr \spacefill\mbox{Arthur}\pend
           \leftskip=0em{}\selectlanguage{ngerman}\endnumbering\briefempfaengerindex{Schwarzkopf, Gustav@\textsc{Schwarzkopf, Gustav}!zzzSchnitzler, Arthur@\emph{von Arthur Schnitzler}!1899-10-083@{8. 10. 1899}|)be}\mylabel{L04082h}
\begin{anhang}
\end{anhang}\newcommand{\dateiname}{L04082}\newcommand{\titel}{Arthur Schnitzler an Gustav Schwarzkopf, 8. 10. 1899}\newcommand{\editorInnen}{Herausgegeben von Jahnke, SelmaMüller, Martin Anton}%% latex-leseansicht-abspann.tex
%% Abspann für die Leseansicht.
%% Der Schalter \ifkorrekturansicht ist bereits durch den Vorspann gesetzt.

%% latex-abspann.tex
%% Gemeinsamer Abspann für Korrekturansicht und Leseansicht.
%% Setzt den Schalter \ifkorrekturansicht voraus (gesetzt in den
%% einbindenden Dateien latex-korrekturansicht-abspann.tex bzw.
%% latex-leseansicht-abspann.tex).
%% ---------------------------------------------------------------

\normalsize

% Das esempio-Environment wird nur in der Leseansicht benötigt
\ifkorrekturansicht\else
\newenvironment{esempio}[3]%
{
    \vspace{1.5ex}
    \rlap{\underline{#1}}
    \par
    \setlength{\parindent}{0cm}
    \nopagebreak
    \leftskip=#2cm
    \rightskip=#3cm
}
{
    \par
}
\fi

\doendnotes{C}
\bigskip
\vfill

\clearpage

\footnotesize

\ifkorrekturansicht
  \lohead{\textsc{register}}
\fi

% theindex-Environment neu definieren ohne reledmac
\makeatletter
\renewenvironment{theindex}{%
  \ifkorrekturansicht
    \section*{\indexname}%
  \else
    \subsubsection*{Index der erwähnten Entitäten}%
  \fi
  \setlength{\parindent}{0pt}%
  \setlength{\parskip}{0pt plus 0.3pt}%
  \let\item\@idxitem
}{%
  \ifkorrekturansicht\clearpage\fi
}
\makeatother

\IfFileExists{\jobname-pw.ind}{\input{\jobname-pw.ind}}{}

% Quellenangabe nur in der Leseansicht
\ifkorrekturansicht\else
% Fallback-Definitionen, falls die .tex-Datei \titel etc. nicht gesetzt hat
\providecommand{\titel}{}
\providecommand{\editorInnen}{}
\providecommand{\dateiname}{\jobname}

\vspace{3cm}

\vfill

\footnotesize
\textsc{Quelle}: \titel. Herausgegeben von {\editorInnen}. In: \emph{Arthur Schnitzler: Briefwechsel mit Autorinnen und Autoren}.
 Digitale Edition, https://schnitzler-briefe.acdh.oeaw.ac.at/{\dateiname}.html (Stand \today)
\fi

\end{document}


