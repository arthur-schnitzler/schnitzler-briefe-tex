%% latex-korrekturansicht-vorspann.tex
%% Vorspann für die Korrekturansicht.
%% Lädt die gemeinsame Datei latex-vorspann.tex mit gesetztem Schalter.

\newif\ifkorrekturansicht
\korrekturansichttrue

\input{../tex-inputs/latex-vorspann}


\section[Friedrich M. Fels an Arthur Schnitzler, 25. 5. 1894]{L00329 Friedrich M. Fels an Arthur Schnitzler, 25. 5. 1894}
\nopagebreak\mylabel{L00329v}
\rehead{ }\normalsize\beginnumbering\briefempfaengerindex{Schnitzler, Arthur@\textsc{Schnitzler, Arthur}!zzzFels, Friedrich Michael@\emph{von Friedrich Michael Fels}!1894-05-251@{25. 5. 1894}|(be}
\toendnotes[C]{\smallbreak\pagebreak[2]}\Standort{DLA, A:Schnitzler, HS.NZ85.1.2956.}
\physDesc{Kartenbrief, 509 Zeichen
\newline{}Handschrift: schwarze Tinte, lateinische Kurrent
\newline{}Versand: 1) Stempel: »\nobreak{}W{[}ien{]}
                                       110, 25. 5. 1894, 8–9V\nobreak{}«.   2) Stempel: »\nobreak{}\oindex{IX., Alsergrund@\textbf{IX., Alsergrund}, \emph{A.ADM3}|pwk}Wien 9/\textcolor{gray}{3}, 25. 5. 94, 10.V, Bestellt\nobreak{}«. 
\newline{}Schnitzler: mit Bleistift datiert: »25/5 94« und nummeriert: »14« }\toendnotes[C]{\smallbreak}\pstart{}{\pb}Herrn Dr. Arthur Schnitzler\pend{}\pstart{}Wien\oindex{Wien@\textbf{Wien}, \emph{A.ADM2}|pw}\pend{}\pstart{}IX, Frankgaſse 1\oindex{Frankgasse 1@\textbf{Frankgasse 1}, \emph{Wohngebäude (K.WHS)}|pw}\pend{}{\bigskip}\vspace{1em}
\pstart
           \raggedleft{}{\pb}Wien XVIII, Exnergasse 3\oindex{Kruetznergasse@\textbf{Krütznergasse}, \emph{Straße (K.STR)}|pw}\textsuperscript{III. St. Th. 22}\pend
           \vspace{0.5em}
\pstart
           Lieber Dr Schnitzler! Habe von Dr Beer-Hofma{\geminationn}\pwindex{Beer-Hofmann, Richard 1866-07-11 – 1945-09-26@\textsc{Beer-Hofmann, Richard} (1866-07-11 – 1945-09-26), \emph{Schriftsteller/Schriftstellerin}|pw} noch nichts empfangen und muss zum Überfluss noch wohl ein paar Tage zu Hause
               bleiben, da ich schreckliche Zahnschmerzen habe und wieder ein Geschwür zu beko{\geminationm}en scheine. Wären Sie vielleicht so freundlich, mir
               eine Kleinigkeit zu senden, da es ganz unbesti{\geminationm}t ist, ob
               und wa{\geminationn}{ }Beer-Hofma{\geminationn}\pwindex{Beer-Hofmann, Richard 1866-07-11 – 1945-09-26@\textsc{Beer-Hofmann, Richard} (1866-07-11 – 1945-09-26), \emph{Schriftsteller/Schriftstellerin}|pw} es thun wird. Seien Sie mir nicht böse und bestens gegrüsst von Ihrem\pend
           \pstart \spacefill\mbox{Fels}\pend{}
\pstart
           \noindent{}\label{K_L00329-1v}\edtext{scripsit in tormentis}{\lemma{\textnormal{\emph{scripsit in tormentis}}}\Cendnote{\textnormal{lateinisch: geschrieben unter Qualen}}}\label{K_L00329-1}\pend
           \selectlanguage{ngerman}\endnumbering\briefempfaengerindex{Schnitzler, Arthur@\textsc{Schnitzler, Arthur}!zzzFels, Friedrich Michael@\emph{von Friedrich Michael Fels}!1894-05-251@{25. 5. 1894}|)be}\mylabel{L00329h}  \normalsize

\doendnotes{C}
\bigskip
\vfill

\clearpage

\footnotesize

\lohead{\textsc{register}}

% Definiere theindex-Environment komplett neu ohne reledmac
\makeatletter
\renewenvironment{theindex}{%
  \section*{\indexname}%
  \setlength{\parindent}{0pt}%
  \setlength{\parskip}{0pt plus 0.3pt}%
  \let\item\@idxitem
}{%
  \clearpage
}
\makeatother

\IfFileExists{\jobname-pw.ind}{\input{\jobname-pw.ind}}{}

\end{document}

      