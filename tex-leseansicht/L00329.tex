\input{../tex-inputs/latex-pdf-vorspann}
\begin{center}
            \textcolor{red}{ENTWURF. ENTZIFFERUNG NOCH NICHT KORREKTURGELESEN}
                      \end{center}
            
               \section[Friedrich M. Fels an Arthur Schnitzler, 25. 5. 1894]{ Friedrich M. Fels an Arthur Schnitzler, 25. 5. 1894}\nopagebreak\mylabel{v}\rehead{ }\begin{ledgroupsized}[t]{13cm}\normalsize\beginnumbering\briefempfaengerindex{Schnitzler, Arthur@\textsc{Schnitzler, Arthur}!zzzFels, Friedrich Michael@\emph{von Friedrich Michael Fels}!1894-05-251@{25. 5. 1894}|(be} \toendnotes[C]{\smallbreak\pagebreak[2]} \Standort{DLA, A:Schnitzler, HS.NZ85.1.2956.}
\physDesc{Kartenbrief
\newline{}Handschrift: schwarze Tinte, lateinische Kurrent\newline{}Versand: 1) Stempel: »\nobreak{}W{[}ien{]}
                                                  110, 25. 5. 1894, 8–9V\nobreak{}«.  2) Stempel: »\nobreak{}\oindex{IX., Alsergrund@\textbf{IX., Alsergrund}|pwk}Wien 9/\textcolor{gray}{3}, 25. 5. 94, 10.V, Bestellt\nobreak{}«. 
\newline{}Schnitzler: mit Bleistift datiert: »25/5 94« und nummeriert: »14« }\toendnotes[C]{\smallbreak}\pstart{}{\pb}Herrn Dr. Arthur Schnitzler\pend{}\pstart{}Wien\oindex{Wien@\textbf{Wien}|pw}\pend{}\pstart{}IX, Frankgaſse 1\oindex{Frankgasse@\textbf{Frankgasse}|pw}\pend{}{\bigskip}\pstart
           \noindent{}\raggedleft{}{\pb}Wien XVIII, Exnergasse 3\oindex{Kruetznergasse@\textbf{Krütznergasse}|pw}\textsuperscript{III. St. Th. 22}\pend
           \pstart
           Lieber Dr Schnitzler! Habe von Dr Beer-Hofma{\geminationn}\pwindex{Beer-Hofmann, Richard 11.07.1866 – 26.09.1945@\textsc{Beer-Hofmann, Richard} (11.07.1866 – 26.09.1945), \emph{Schriftsteller}|pw} noch nichts empfangen und muss zum Überfluss noch wohl ein paar Tage zu
                    Hause bleiben, da ich schreckliche Zahnschmerzen habe und wieder ein Geschwür zu
                        beko{\geminationm}en scheine. Wären Sie vielleicht so
                    freundlich, mir eine Kleinigkeit zu senden, da es ganz unbesti{\geminationm}t ist, ob und wa{\geminationn}{ }Beer-Hofma{\geminationn}\pwindex{Beer-Hofmann, Richard 11.07.1866 – 26.09.1945@\textsc{Beer-Hofmann, Richard} (11.07.1866 – 26.09.1945), \emph{Schriftsteller}|pw} es thun wird. Seien Sie mir nicht böse und bestens gegrüsst von Ihrem\pend
           \pstart \spacefill\mbox{Fels}\pend{}\pstart
           \noindent{}\label{K_L00329_1v}\edtext{scripsit in tormentis}{\lemma{\textnormal{\emph{scripsit in tormentis}}}\Cendnote{\textnormal{lat. geschrieben unter
                            Qualen.}}}\label{K_L00329_1h}\pend
           \endnumbering\briefempfaengerindex{Schnitzler, Arthur@\textsc{Schnitzler, Arthur}!zzzFels, Friedrich Michael@\emph{von Friedrich Michael Fels}!1894-05-251@{25. 5. 1894}|)be}\mylabel{h}\end{ledgroupsized}  \newcommand{\dateiname}{L00329}\newcommand{\titel}{Friedrich M. Fels an Arthur Schnitzler, 25. 5. 1894}\newcommand{\editorInnen}{Martin Anton Müller und Gerd-Hermann Susen}\input{../tex-inputs/latex-pdf-abspann}
      