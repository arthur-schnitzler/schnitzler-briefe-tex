%% latex-leseansicht-vorspann.tex
%% Vorspann für die Leseansicht.
%% Lädt die gemeinsame Datei latex-vorspann.tex mit nicht gesetztem Schalter.

\newif\ifkorrekturansicht
\korrekturansichtfalse

\input{../tex-inputs/latex-vorspann}


         
         \renewcommand{\erwaehntePersonen}{Personen: Felix Salten, Boris Sapiro}
         \renewcommand{\erwaehnteOrte}{Orte: Wien}
         \renewcommand{\erwaehnteWerke}{}
               \section[ Felix Salten an Arthur Schnitzler, {[}16. 11. 1929{]}]{ Felix Salten an Arthur Schnitzler, {[}16. 11. 1929{]}}\nopagebreak\mylabel{v}\rehead{ }\begin{ledgroupsized}[t]{13cm}\normalsize\beginnumbering\briefempfaengerindex{Schnitzler, Arthur@\textsc{Schnitzler, Arthur}!zzzSalten, Felix@\emph{von Felix Salten}!1929-11-161@{{[}16. 11. 1929{]}}|(be} \toendnotes[C]{\smallbreak\pagebreak[2]} \Standort{CUL, Schnitzler, B 89, B 2.}
\physDesc{Briefkarte, 186 Zeichen
\newline{}Handschrift: Bleistift, lateinische Kurrent
\newline{}Schnitzler: 1) mit Bleistift datiert: »16. 11. 1929«  2) mit rotem Buntstift Vermerk »\textsc{Salten\pwindex{Salten, Felix 06.09.1869 – 08.10.1945@\textsc{Salten, Felix} (06.09.1869 – 08.10.1945), \emph{Schriftsteller, Journalist, Chefredakteur}|pw}}«
\newline{}Ordnung: mit Bleistift von unbekannter Hand nummeriert: »302« }\toendnotes[C]{\smallbreak}\pstart
           \raggedleft{}{\pb}Samstag.\pend
           \pstart
           Lieber, Sie wissen, ich \label{K_L03589-11v}\edtext{schicke Niemanden}{\lemma{\textnormal{\emph{schicke Niemanden}}}\Cendnote{\textnormal{vgl. Felix Salten an Arthur Schnitzler, [22. 2. 1922?].}}}\label{K_L03589-11h} zu Ihnen, aber diesen jungen, vielerfahrenen, merkwürdigen \label{K_L03589-1v}\edtext{Menschen\pwindex{Sapiro, Boris 24.01.1910 – 1960-12-16@\textsc{Sapiro, Boris} (24.01.1910 – 1960-12-16), \emph{Regisseur, Schauspieler, Rezitator}|pwv}}{\lemma{\textnormal{\emph{Menschen}}}\Cendnote{\textnormal{Boris Sapiro\pwindex{Sapiro, Boris 24.01.1910 – 1960-12-16@\textsc{Sapiro, Boris} (24.01.1910 – 1960-12-16), \emph{Regisseur, Schauspieler, Rezitator}|pwk}, siehe A. S.: \emph{Tagebuch}, 18. 11. 1929}}}\label{K_L03589-1h} werden Sie mit
               Interesse und amüsiert anhören.\pend
           \pstart
           Herzlichst Ihr {\\[\baselineskip]}\spacefill\mbox{Felix Salten}\pend
           \leftskip=0em{}
         
         \endnumbering\mylabel{h}\end{ledgroupsized}  \newcommand{\dateiname}{L03589}\newcommand{\titel}{Felix Salten an Arthur Schnitzler, [16. 11. 1929]}\newcommand{\editorInnen}{Martin Anton Müller und Laura Untner}%% latex-leseansicht-abspann.tex
%% Abspann für die Leseansicht.
%% Der Schalter \ifkorrekturansicht ist bereits durch den Vorspann gesetzt.

%% latex-abspann.tex
%% Gemeinsamer Abspann für Korrekturansicht und Leseansicht.
%% Setzt den Schalter \ifkorrekturansicht voraus (gesetzt in den
%% einbindenden Dateien latex-korrekturansicht-abspann.tex bzw.
%% latex-leseansicht-abspann.tex).
%% ---------------------------------------------------------------

\normalsize

% Das esempio-Environment wird nur in der Leseansicht benötigt
\ifkorrekturansicht\else
\newenvironment{esempio}[3]%
{
    \vspace{1.5ex}
    \rlap{\underline{#1}}
    \par
    \setlength{\parindent}{0cm}
    \nopagebreak
    \leftskip=#2cm
    \rightskip=#3cm
}
{
    \par
}
\fi

\doendnotes{C}
\bigskip
\vfill

\clearpage

\footnotesize

\ifkorrekturansicht
  \lohead{\textsc{register}}
\fi

% theindex-Environment neu definieren ohne reledmac
\makeatletter
\renewenvironment{theindex}{%
  \ifkorrekturansicht
    \section*{\indexname}%
  \else
    \subsubsection*{Index der erwähnten Entitäten}%
  \fi
  \setlength{\parindent}{0pt}%
  \setlength{\parskip}{0pt plus 0.3pt}%
  \let\item\@idxitem
}{%
  \ifkorrekturansicht\clearpage\fi
}
\makeatother

\IfFileExists{\jobname-pw.ind}{\input{\jobname-pw.ind}}{}

% Quellenangabe nur in der Leseansicht
\ifkorrekturansicht\else
% Fallback-Definitionen, falls die .tex-Datei \titel etc. nicht gesetzt hat
\providecommand{\titel}{}
\providecommand{\editorInnen}{}
\providecommand{\dateiname}{\jobname}

\vspace{3cm}

\vfill

\footnotesize
\textsc{Quelle}: \titel. Herausgegeben von {\editorInnen}. In: \emph{Arthur Schnitzler: Briefwechsel mit Autorinnen und Autoren}.
 Digitale Edition, https://schnitzler-briefe.acdh.oeaw.ac.at/{\dateiname}.html (Stand \today)
\fi

\end{document}


      