%% latex-korrekturansicht-vorspann.tex
%% Vorspann für die Korrekturansicht.
%% Lädt die gemeinsame Datei latex-vorspann.tex mit gesetztem Schalter.

\newif\ifkorrekturansicht
\korrekturansichttrue

\input{../tex-inputs/latex-vorspann}


\section[ Felix Salten an Arthur Schnitzler, {[}16. 11. 1929{]}]{L03589 Felix Salten an Arthur Schnitzler, {[}16. 11. 1929{]}}
\nopagebreak\mylabel{L03589v}
\rehead{ }\normalsize\beginnumbering\briefempfaengerindex{Schnitzler, Arthur@\textsc{Schnitzler, Arthur}!zzzSalten, Felix@\emph{von Felix Salten}!1929-11-161@{{[}16. 11. 1929{]}}|(be}
\toendnotes[C]{\smallbreak\pagebreak[2]}\Standort{CUL, Schnitzler, B 89, B 2.}
\physDesc{Briefkarte, 185 Zeichen
\newline{}Handschrift: Bleistift, lateinische Kurrent
\newline{}Schnitzler: 1) mit Bleistift datiert: »16. 11. 1929«  2) mit rotem Buntstift Vermerk »\textsc{Salten\pwindex{Salten, Felix 06.09.1869 – 08.10.1945@\textsc{Salten, Felix} (06.09.1869 – 08.10.1945), \emph{Schriftsteller/Schriftstellerin, Journalist/Journalistin, Chefredakteur/Chefredakteurin}|pw}}«
\newline{}Ordnung: mit Bleistift von unbekannter Hand nummeriert: »302« }\toendnotes[C]{\smallbreak}
\pstart
           \raggedleft{}{\pb}Samstag.\pend
           \vspace{0.5em}
\pstart
           Lieber, Sie wissen, ich \label{K_L03589-1v}\edtext{schicke Niemanden}{\lemma{\textnormal{\emph{schicke Niemanden}}}\Cendnote{\textnormal{Vgl. Felix Salten an Arthur Schnitzler, [22. 2. 1922?].}}}\label{K_L03589-1} zu Ihnen, aber diesen jungen, vielerfahrenen, merkwürdigen \label{K_L03589-2v}\edtext{Menschen\pwindex{Sapiro, Boris 24.01.1910 – 1960-12-16@\textsc{Sapiro, Boris} (24.01.1910 – 1960-12-16), \emph{Regisseur/Regisseurin, Schauspieler/Schauspielerin, Rezitator/Rezitatorin}|pwv}}{\lemma{\textnormal{\emph{Menschen}}}\Cendnote{\textnormal{Boris Sapiro\pwindex{Sapiro, Boris 24.01.1910 – 1960-12-16@\textsc{Sapiro, Boris} (24.01.1910 – 1960-12-16), \emph{Regisseur/Regisseurin, Schauspieler/Schauspielerin, Rezitator/Rezitatorin}|pwk}, siehe A. S.: \emph{Tagebuch}, 18. 11. 1929.}}}\label{K_L03589-2} werden Sie mit
               Interesse und amüsiert anhören.\pend
           
\pstart
           Herzlichst Ihr {\\[\baselineskip]}\spacefill\mbox{Felix Salten}\pend
           \leftskip=0em{}\selectlanguage{ngerman}\endnumbering\briefempfaengerindex{Schnitzler, Arthur@\textsc{Schnitzler, Arthur}!zzzSalten, Felix@\emph{von Felix Salten}!1929-11-161@{{[}16. 11. 1929{]}}|)be}\mylabel{L03589h}  \normalsize

\doendnotes{C}
\bigskip
\vfill

\clearpage

\footnotesize

\lohead{\textsc{register}}

% Definiere theindex-Environment komplett neu ohne reledmac
\makeatletter
\renewenvironment{theindex}{%
  \section*{\indexname}%
  \setlength{\parindent}{0pt}%
  \setlength{\parskip}{0pt plus 0.3pt}%
  \let\item\@idxitem
}{%
  \clearpage
}
\makeatother

\IfFileExists{\jobname-pw.ind}{\input{\jobname-pw.ind}}{}

\end{document}

      