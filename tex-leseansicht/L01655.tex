%% latex-leseansicht-vorspann.tex
%% Vorspann für die Leseansicht.
%% Lädt die gemeinsame Datei latex-vorspann.tex mit nicht gesetztem Schalter.

\newif\ifkorrekturansicht
\korrekturansichtfalse

\input{../tex-inputs/latex-vorspann}


\section[Hugo von Hofmannsthal an Arthur Schnitzler, {{[}}zwischen 3. – 7. 2. 1907{{]}}]{L01655 Hugo von Hofmannsthal an Arthur Schnitzler, {[}zwischen 3. – 7. 2. 1907{]}}
\nopagebreak\mylabel{L01655v}
\rehead{ }\normalsize\beginnumbering\briefempfaengerindex{Schnitzler, Arthur@\textsc{Schnitzler, Arthur}!zzzHofmannsthal, Hugo von@\emph{von Hugo von Hofmannsthal}!1907-02-071@{{[}zwischen 3. – 7. 2. 1907{]}}|(be}
\toendnotes[C]{\smallbreak\pagebreak[2]}
\correspDesc{Versand  durch Hugo von Hofmannsthal im Zeitraum [zwischen
                  3. – 7. 2. 1907] in Gerngroß
\newline{}Erhalt  durch Arthur Schnitzler im Zeitraum [zwischen
                  4.–10. 2. 1907] in Wien}\toendnotes[C]{\smallbreak}
\Standort{CUL, Schnitzler, B 43.}
\physDesc{Brief, 1 Blatt, 1 Seite, 587 Zeichen, Fragment
\newline{}Handschrift: schwarze Tinte, deutsche Kurrent
\newline{}Ordnung: 1) mit Bleistift von unbekannter Hand nummeriert: »\strikeout{269}«  2) mit Bleistift von unbekannter Hand nummeriert:
                                    »269« und beschriftet: »lacking
                                    Sheet 1?«}
\buchAbdrucke{\weitereDrucke{Hugo von Hofmannsthal, Arthur Schnitzler: \emph{Briefwechsel}. Herausgegeben von Therese Nickl und Heinrich Schnitzler. Frankfurt am Main: \emph{S. Fischer} 1964, S. 227.} }\toendnotes[C]{\smallbreak}
\pstart{}{\pb}lieber,\pend\vspace{0.5em}
\pstart
           man{ }ſieht{ }ſich \uline{nie}. Momentan{ }ſind wieder Gerty\pwindex{Hofmannsthal, Gertrude von 16.\,3.\,1880 Wien – 9.\,11.\,1959 Paddington@\textsc{Hofmannsthal, Gertrude von} (16.\,3.\,1880 Wien – 9.\,11.\,1959 Paddington)|pw} und ich nicht recht wohl, können nicht in
               die Stadt. Ich habe \label{K_L01655-1v}\edtext{böſes Aug,{ }ſchlechten Hals, wehen Fuß}{\lemma{\textnormal{\emph{böses … Fuß}}}\Cendnote{\textnormal{Offensichtlich
                  um diese auszuheilen, reiste Hofmannsthal\pwindex{Hofmannsthal, Hugo von 1.\,2.\,1874 Wien – 15.\,7.\,1929 Rodaun@\textsc{Hofmannsthal, Hugo von} (1.\,2.\,1874 Wien – 15.\,7.\,1929 Rodaun), \emph{Schriftsteller}|pwk} am
                     12. 2. 1907 ins Südbahnhotel\oindex{Südbahnhotel [Semmering]@\textbf{Südbahnhotel [Semmering]}, \emph{Hotel}|pwk}
                  am Semmering\oindex{Semmering@\textbf{Semmering}, \emph{Verwaltungsgebiet}|pwk}, während seine Frau\pwindex{Hofmannsthal, Gertrude von 16.\,3.\,1880 Wien – 9.\,11.\,1959 Paddington@\textsc{Hofmannsthal, Gertrude von} (16.\,3.\,1880 Wien – 9.\,11.\,1959 Paddington)|pwk} zu Hause blieb. Das Schreiben kann demnach nur mit
                  nötigem Abstand zum einzig verbleibenden Wochenende im Februar 1907
                  davor entstanden sein.}}}\label{K_L01655-1}. Kann nicht{ }ſingen, nicht{ }ſtehen, nicht{ }ſchauen.
               Wünſche mir{ }ſehr Geſellſchaft. Seid doch einmal im Leben nett (zum Unterſchied von
               dem † † † Bärenviehzeug\pwindex{Beer-Hofmann, Richard 11.\,7.\,1866 Wien – 26.\,9.\,1945 New York City@\textsc{Beer-Hofmann, Richard} (11.\,7.\,1866 Wien – 26.\,9.\,1945 New York City), \emph{Schriftsteller}|pwv}\pwindex{Beer-Hofmann, Paula 25.\,2.\,1879 Wien – 30.\,10.\,1939 Zürich@\textsc{Beer-Hofmann, Paula} (25.\,2.\,1879 Wien – 30.\,10.\,1939 Zürich)|pwv}). Es iſt jetzt{ }ſo hübſch hier, Schnee und hübſch und dabei
               mild, alſo {\pb}kommt einmal her, oder Samstag oder Sonntag; oder zum Eſſen,
               oder zum Nachmittag oder zum Nachtmahl oder alles zugleich.\pend
           
\pstart
           Depeſchiert{ }ſchön gleich Eure werte Antwort.\pend
           
\pstart
           Euer unvergleichlicher und ergebenſter Diener{\\[\baselineskip]}\spacefill\mbox{Hugo}\pend
           \leftskip=0em{}\selectlanguage{ngerman}\endnumbering\briefempfaengerindex{Schnitzler, Arthur@\textsc{Schnitzler, Arthur}!zzzHofmannsthal, Hugo von@\emph{von Hugo von Hofmannsthal}!1907-02-031@{{[}zwischen 3. – 7. 2. 1907{]}}|)be}\mylabel{L01655h}  \newcommand{\dateiname}{L01655}\newcommand{\titel}{Hugo von Hofmannsthal an Arthur Schnitzler, [zwischen 3. – 7. 2. 1907]}\newcommand{\editorInnen}{Martin Anton Müller und Gerd-Hermann Susen}%% latex-leseansicht-abspann.tex
%% Abspann für die Leseansicht.
%% Der Schalter \ifkorrekturansicht ist bereits durch den Vorspann gesetzt.

%% latex-abspann.tex
%% Gemeinsamer Abspann für Korrekturansicht und Leseansicht.
%% Setzt den Schalter \ifkorrekturansicht voraus (gesetzt in den
%% einbindenden Dateien latex-korrekturansicht-abspann.tex bzw.
%% latex-leseansicht-abspann.tex).
%% ---------------------------------------------------------------

\normalsize

% Das esempio-Environment wird nur in der Leseansicht benötigt
\ifkorrekturansicht\else
\newenvironment{esempio}[3]%
{
    \vspace{1.5ex}
    \rlap{\underline{#1}}
    \par
    \setlength{\parindent}{0cm}
    \nopagebreak
    \leftskip=#2cm
    \rightskip=#3cm
}
{
    \par
}
\fi

\doendnotes{C}
\bigskip
\vfill

\clearpage

\footnotesize

\ifkorrekturansicht
  \lohead{\textsc{register}}
\fi

% theindex-Environment neu definieren ohne reledmac
\makeatletter
\renewenvironment{theindex}{%
  \ifkorrekturansicht
    \section*{\indexname}%
  \else
    \subsubsection*{Index der erwähnten Entitäten}%
  \fi
  \setlength{\parindent}{0pt}%
  \setlength{\parskip}{0pt plus 0.3pt}%
  \let\item\@idxitem
}{%
  \ifkorrekturansicht\clearpage\fi
}
\makeatother

\IfFileExists{\jobname-pw.ind}{\input{\jobname-pw.ind}}{}

% Quellenangabe nur in der Leseansicht
\ifkorrekturansicht\else
% Fallback-Definitionen, falls die .tex-Datei \titel etc. nicht gesetzt hat
\providecommand{\titel}{}
\providecommand{\editorInnen}{}
\providecommand{\dateiname}{\jobname}

\vspace{3cm}

\vfill

\footnotesize
\textsc{Quelle}: \titel. Herausgegeben von {\editorInnen}. In: \emph{Arthur Schnitzler: Briefwechsel mit Autorinnen und Autoren}.
 Digitale Edition, https://schnitzler-briefe.acdh.oeaw.ac.at/{\dateiname}.html (Stand \today)
\fi

\end{document}


