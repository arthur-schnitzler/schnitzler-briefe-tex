%% latex-korrekturansicht-vorspann.tex
%% Vorspann für die Korrekturansicht.
%% Lädt die gemeinsame Datei latex-vorspann.tex mit gesetztem Schalter.

\newif\ifkorrekturansicht
\korrekturansichttrue

\input{../tex-inputs/latex-vorspann}


\section[Hugo von Hofmannsthal an Arthur Schnitzler, 26. {[}3. 1911{]}]{L02014 Hugo von Hofmannsthal an Arthur Schnitzler, 26. {[}3. 1911{]}}
\nopagebreak\mylabel{L02014v}
\rehead{ }\normalsize\beginnumbering\briefempfaengerindex{Schnitzler, Arthur@\textsc{Schnitzler, Arthur}!zzzHofmannsthal, Hugo von@\emph{von Hugo von Hofmannsthal}!1911-03-261@{26. {[}3. 1911{]}}|(be}
\toendnotes[C]{\smallbreak\pagebreak[2]}\Standort{CUL, Schnitzler, B 43.}
\physDesc{Briefkarte, 438 Zeichen
\newline{}Handschrift: schwarze Tinte, deutsche Kurrent
\newline{}Schnitzler: mit Bleistift datiert: »3. 910« und beschriftet: »Hugo« 
\newline{}Ordnung: 1) mit Bleistift von unbekannter Hand nummeriert: »\strikeout{320}«  2) mit Bleistift von unbekannter Hand nummeriert:
                                    »315«}
\buchAbdrucke{\weitereDrucke{Hugo von Hofmannsthal, Arthur Schnitzler: \emph{Briefwechsel}. Frankfurt am Main: \emph{S. Fischer} 1964, S. 261.} }\toendnotes[C]{\smallbreak}
\pstart
           \raggedleft{}{\pb}Sonntag 26\textsuperscript{ten}\pend
           
\pstart{}mein lieber Arthur, \pend\vspace{0.5em}
\pstart
           ich habe alſo eine \label{K_L02014-1v}\edtext{Loge}{\lemma{\textnormal{\emph{Loge}}}\Cendnote{\textnormal{für die Wien\oindex{Wien@\textbf{Wien}, \emph{A.ADM2}|pwk}er Erstaufführung von \emph{Der
                     Rosenkavalier}\pwindex{Rosenkavalier@\emph{Der Rosenkavalier}|pwk}.}}}\label{K_L02014-1}{ }\textsc{parterre} oder I\textsuperscript{ter} Rang für Sie
               und Richard\pwindex{Beer-Hofmann, Richard 1866-07-11 – 1945-09-26@\textsc{Beer-Hofmann, Richard} (1866-07-11 – 1945-09-26), \emph{Schriftsteller/Schriftstellerin}|pw} beſtellt für den 8\textsuperscript{ten} April\pwindex{Rosenkavalier@\emph{Der Rosenkavalier}|pwv}. Nun höre ich auf einmal daſs die Bären\pwindex{Beer-Hofmann, Richard 1866-07-11 – 1945-09-26@\textsc{Beer-Hofmann, Richard} (1866-07-11 – 1945-09-26), \emph{Schriftsteller/Schriftstellerin}|pw}\pwindex{Beer-Hofmann, Paula 25.02.1879 – 30.10.1939@\textsc{Beer-Hofmann, Paula} (25.02.1879 – 30.10.1939)|pw} abreiſen. Ich hoffe aber trotzdem, {\pb}daſs es Ihnen vielleicht recht
               iſt, eine Loge zu haben und ſie mit \textcolor{gray}{Ih}rem Bruder\pwindex{Schnitzler, Julius 13.07.1865 – 29.06.1939@\textsc{Schnitzler, Julius} (13.07.1865 – 29.06.1939), \emph{Chirurg/Chirurgin}|pwv} oder ſonſt jemand zu theilen; denn
               ſonſt müſste ich verſuchen, bei \textsc{Horsetzky}\pwindex{Horsetzky-Hornthal, Viktor von 05.02.1853 – 31.03.1932@\textsc{Horsetzky-Hornthal, Viktor von} (05.02.1853 – 31.03.1932), \emph{Kanzleidirektor/Kanzleidirektorin}|pw} die für mich \strikeout{vorgeſetz} vorgemerkte Liſte zu
               ändern. Bitte um ein kleines Wort!\pend
           \pstart Herzlich Ihr\spacefill\mbox{Hugo.}\pend{}\selectlanguage{ngerman}\endnumbering\briefempfaengerindex{Schnitzler, Arthur@\textsc{Schnitzler, Arthur}!zzzHofmannsthal, Hugo von@\emph{von Hugo von Hofmannsthal}!1911-03-261@{26. {[}3. 1911{]}}|)be}\mylabel{L02014h}  \normalsize

\doendnotes{C}
\bigskip
\vfill

\clearpage

\footnotesize

\lohead{\textsc{register}}

% Definiere theindex-Environment komplett neu ohne reledmac
\makeatletter
\renewenvironment{theindex}{%
  \section*{\indexname}%
  \setlength{\parindent}{0pt}%
  \setlength{\parskip}{0pt plus 0.3pt}%
  \let\item\@idxitem
}{%
  \clearpage
}
\makeatother

\IfFileExists{\jobname-pw.ind}{\input{\jobname-pw.ind}}{}

\end{document}

      