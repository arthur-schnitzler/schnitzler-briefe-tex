\input{../tex-inputs/latex-pdf-vorspann}
\begin{center}
            \textcolor{red}{ENTWURF. ENTZIFFERUNG NOCH NICHT KORREKTURGELESEN}
                      \end{center}
            
               \section[Hugo von Hofmannsthal an Arthur Schnitzler, 26. {[}3. 1911{]}]{ Hugo von Hofmannsthal an Arthur Schnitzler, 26. {[}3. 1911{]}}\nopagebreak\mylabel{v}\rehead{ }\begin{ledgroupsized}[t]{13cm}\normalsize\beginnumbering\briefempfaengerindex{Schnitzler, Arthur@\textsc{Schnitzler, Arthur}!zzzHofmannsthal, Hugo von@\emph{von Hugo von Hofmannsthal}!1911-03-261@{26. {[}3. 1911{]}}|(be} \toendnotes[C]{\smallbreak\pagebreak[2]} \Standort{CUL, Schnitzler, B 43.}
\physDesc{Briefkarte
\newline{}Handschrift: schwarze Tinte, deutsche Kurrent
\newline{}Schnitzler: mit Bleistift datiert: »3. 910« und beschriftet: »Hugo« \newline{}Ordnung: 1) mit Bleistift von unbekannter Hand nummeriert: »\strikeout{320}« 2) mit Bleistift von unbekannter Hand nummeriert:
                                    »315«}\buchAbdrucke{\weitereDrucke{Hugo von Hofmannsthal, Arthur Schnitzler: \emph{Briefwechsel}. Hg. Therese Nickl und Heinrich Schnitzler. Frankfurt am Main: \emph{S. Fischer} 1964, S. 261.} }\toendnotes[C]{\smallbreak}\pstart
           \raggedleft{}{\pb}Sonntag 26\textsuperscript{ten}\pend
           \pstart{}mein lieber Arthur, \pend\pstart
           ich habe alſo eine \label{K_L02014_1v}\edtext{Loge}{\lemma{\textnormal{\emph{Loge}}}\Cendnote{\textnormal{für die Wien\oindex{Wien@\textbf{Wien}|pwk}er Erstaufführung von \emph{Der
                     Rosenkavalier}\pwindex{Hofmannsthal, Hugo von 01.02.1874 – 15.07.1929@\textsc{Hofmannsthal, Hugo von} (01.02.1874 – 15.07.1929), \emph{Schriftsteller}!Rosenkavalier1911@\strich\emph{Der Rosenkavalier} {[}1911{]}|pwk}.}}}\label{K_L02014_1h}{ }\textsc{parterre} oder I\textsuperscript{ter} Rang für Sie
               und Richard\pwindex{Beer-Hofmann, Richard 11.07.1866 – 26.09.1945@\textsc{Beer-Hofmann, Richard} (11.07.1866 – 26.09.1945), \emph{Schriftsteller}|pw} beſtellt für den 8\textsuperscript{ten} April\pwindex{Hofmannsthal, Hugo von 01.02.1874 – 15.07.1929@\textsc{Hofmannsthal, Hugo von} (01.02.1874 – 15.07.1929), \emph{Schriftsteller}!Rosenkavalier1911@\strich\emph{Der Rosenkavalier} {[}1911{]}|pwv}. Nun höre ich auf einmal daſs die Bären\pwindex{Beer-Hofmann, Richard 11.07.1866 – 26.09.1945@\textsc{Beer-Hofmann, Richard} (11.07.1866 – 26.09.1945), \emph{Schriftsteller}|pw}\pwindex{Beer-Hofmann, Paula 25.02.1879 – 30.10.1939@\textsc{Beer-Hofmann, Paula} (25.02.1879 – 30.10.1939)|pw} abreiſen. Ich hoffe aber trotzdem, {\pb}daſs es Ihnen vielleicht recht
               iſt, eine Loge zu haben und ſie mit \textcolor{gray}{Ih}rem Bruder\pwindex{Schnitzler, Julius 13.07.1865 – 29.06.1939@\textsc{Schnitzler, Julius} (13.07.1865 – 29.06.1939), \emph{Chirurg}|pwv} oder ſonſt jemand zu theilen; denn
               ſonſt müſste ich verſuchen, bei \textsc{Horsetzky}\pwindex{Horsetzky-Hornthal, Viktor von 05.02.1853 – 31.03.1932@\textsc{Horsetzky-Hornthal, Viktor von} (05.02.1853 – 31.03.1932), \emph{Kanzleidirektor}|pw} die für mich \strikeout{vorgeſetz} vorgemerkte Liſte zu
               ändern. Bitte um ein kleines Wort!\pend
           \pstart Herzlich Ihr\spacefill\mbox{Hugo.}\pend{}\endnumbering\briefempfaengerindex{Schnitzler, Arthur@\textsc{Schnitzler, Arthur}!zzzHofmannsthal, Hugo von@\emph{von Hugo von Hofmannsthal}!1911-03-261@{26. {[}3. 1911{]}}|)be}\mylabel{h}\end{ledgroupsized}  \newcommand{\dateiname}{L02014}\newcommand{\titel}{Hugo von Hofmannsthal an Arthur Schnitzler, 26. [3. 1911]}\newcommand{\editorInnen}{Martin Anton Müller und Gerd-Hermann Susen}\input{../tex-inputs/latex-pdf-abspann}
      