%% latex-korrekturansicht-vorspann.tex
%% Vorspann für die Korrekturansicht.
%% Lädt die gemeinsame Datei latex-vorspann.tex mit gesetztem Schalter.

\newif\ifkorrekturansicht
\korrekturansichttrue

\input{../tex-inputs/latex-vorspann}


\section[ Felix Salten an Arthur Schnitzler, 21. 9. 1901]{L03319 Felix Salten an Arthur Schnitzler, 21. 9. 1901}
\nopagebreak\mylabel{L03319v}
\rehead{ }\normalsize\beginnumbering\briefempfaengerindex{Schnitzler, Arthur@\textsc{Schnitzler, Arthur}!zzzSalten, Felix@\emph{von Felix Salten}!1901-09-212@{21. 9. 1901}|(be}
\toendnotes[C]{\smallbreak\pagebreak[2]}\Standort{CUL, Schnitzler, B 89, A 2.}
\physDesc{Briefkarte, 444 Zeichen
\newline{}Handschrift: schwarze Tinte, lateinische Kurrent
\newline{}Ordnung: mit Bleistift von unbekannter Hand nummeriert: »143« }\toendnotes[C]{\smallbreak}
\pstart
           {\pb}\textcolor{gray}{\textbf{Jung-Wiener Theater\orgindex{Jung-Wiener Theater zum Lieben Augustin@Jung-Wiener Theater zum Lieben Augustin|pw}}}\hfill \substVorne{}\textsuperscript{\textcolor{gray}{\textbf{Wien\oindex{Wien@\textbf{Wien}, \emph{A.ADM2}|pw}}}}\substDazwischen{}Berlin\oindex{Berlin@\textbf{Berlin}, \emph{P.PPLC}|pw}\substHinten{}\textcolor{gray}{\textbf{,}}{ }21. Septemb. \textcolor{gray}{\textbf{190}}1\pend
           
\pstart
           \textcolor{gray}{\textbf{Zum lieben Augustin\orgindex{Jung-Wiener Theater zum Lieben Augustin@Jung-Wiener Theater zum Lieben Augustin|pw}.}}\hfill \textcolor{gray}{\textbf{(Theater a. d.
                        Wien\oindex{Theater an der Wien@\textbf{Theater an der Wien}, \emph{Theater (K.THE)}|pw})}}\pend
           
\pstart
           \textcolor{gray}{\textbf{Direction.}}\pend
           \vspace{0.5em}
\pstart
           Lieber Freund, bin seit einigen Tagen hier, und werde nach meiner
               Rückkehr das \label{K_L03319-1v}\edtext{verl. Manuscript}{\lemma{\textnormal{\emph{verl. Manuscript}}}\Cendnote{\textnormal{Siehe Arthur Schnitzler an Felix Salten, 16. 9. 1901.
               }}}\label{K_L03319-1} zum Hausbesorger\pwindex{?? [Hausmeister von Felix Salten in der Kochgasse 1901] @\textsc{?? [Hausmeister von Felix Salten in der Kochgasse 1901]}|pwv}
               legen. Da ich bis jetzt krank und ziemlich unmöglich war habe ich weder Goldmann\pwindex{Goldmann, Paul 31.01.1865 – 25.09.1935@\textsc{Goldmann, Paul} (31.01.1865 – 25.09.1935), \emph{Schriftsteller/Schriftstellerin, Journalist/Journalistin}|pw} noch {\pb}Kerr\pwindex{Kerr, Alfred 25.12.1867 – 12.10.1948@\textsc{Kerr, Alfred} (25.12.1867 – 12.10.1948), \emph{Schriftsteller/Schriftstellerin, Kritiker/Kritikerin}|pw} bisher aufgesucht. Wedekind\pwindex{Wedekind, Frank 24.07.1864 – 09.03.1918@\textsc{Wedekind, Frank} (24.07.1864 – 09.03.1918), \emph{Schriftsteller/Schriftstellerin, Schauspieler/Schauspielerin, Schriftsteller/Schriftstellerin}|pw} hat mir eben \label{K_L03319-2v}\edtext{für Wien\oindex{Wien@\textbf{Wien}, \emph{A.ADM2}|pw}\orgindex{Jung-Wiener Theater zum Lieben Augustin@Jung-Wiener Theater zum Lieben Augustin|pwv}}{\lemma{\textnormal{\emph{für Wien}}}\Cendnote{\textnormal{für das \emph{Jung-Wiener Theater zum Lieben Augustin}\orgindex{Jung-Wiener Theater zum Lieben Augustin@Jung-Wiener Theater zum Lieben Augustin|pwk}}}}\label{K_L03319-2} zugesagt. Hier\oindex{Berlin@\textbf{Berlin}, \emph{P.PPLC}|pwv} werde
               ich wol kaum etwas finden. Das ist ein Niveau hier – ganz unwahrscheinlich. Und Salzer\pwindex{Salzer, Marcell 27.03.1873 – 17.03.1930@\textsc{Salzer, Marcell} (27.03.1873 – 17.03.1930), \emph{Schauspieler/Schauspielerin, Rezitator/Rezitatorin}|pw} nicht das Schlimmste dabei!! Donnerstag bin ich wieder in Wien\oindex{Wien@\textbf{Wien}, \emph{A.ADM2}|pw}.\pend
           
\pstart
           Herzlichst Ihr {\\[\baselineskip]}\spacefill\mbox{Salten}\pend
           \leftskip=0em{}\selectlanguage{ngerman}\endnumbering\briefempfaengerindex{Schnitzler, Arthur@\textsc{Schnitzler, Arthur}!zzzSalten, Felix@\emph{von Felix Salten}!1901-09-212@{21. 9. 1901}|)be}\mylabel{L03319h}  \normalsize

\doendnotes{C}
\bigskip
\vfill

\clearpage

\footnotesize

\lohead{\textsc{register}}

% Definiere theindex-Environment komplett neu ohne reledmac
\makeatletter
\renewenvironment{theindex}{%
  \section*{\indexname}%
  \setlength{\parindent}{0pt}%
  \setlength{\parskip}{0pt plus 0.3pt}%
  \let\item\@idxitem
}{%
  \clearpage
}
\makeatother

\IfFileExists{\jobname-pw.ind}{\input{\jobname-pw.ind}}{}

\end{document}

      