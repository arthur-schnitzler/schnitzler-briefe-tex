%% latex-leseansicht-vorspann.tex
%% Vorspann für die Leseansicht.
%% Lädt die gemeinsame Datei latex-vorspann.tex mit nicht gesetztem Schalter.

\newif\ifkorrekturansicht
\korrekturansichtfalse

\input{../tex-inputs/latex-vorspann}

\begin{center}
            \textcolor{red}{ENTWURF, NICHT FERTIG KORRIGIERT}
                      \end{center}
            
         
         \renewcommand{\erwaehntePersonen}{Personen:  ?? [Hausmeister von Felix Salten in der Kochgasse 1901], Paul Goldmann, Alfred Kerr, Adolf Lantz, Marcell Salzer, Frank Wedekind}
         \renewcommand{\erwaehnteInstitutionen}{Institutionen: Jung-Wiener Theater zum Lieben Augustin}
         \renewcommand{\erwaehnteOrte}{Orte: Berlin, Theater an der Wien, Wien}
         \renewcommand{\erwaehnteWerke}{}
               \section[Felix Salten an Arthur Schnitzler, 21. 9. 1901]{ Felix Salten an Arthur Schnitzler, 21. 9. 1901}\nopagebreak\mylabel{v}\rehead{ }\begin{ledgroupsized}[t]{13cm}\normalsize\beginnumbering \toendnotes[C]{\smallbreak\pagebreak[2]} \Standort{CUL, Schnitzler, B 89, A 2.}
\physDesc{Briefkarte, 445 Zeichen
\newline{}Handschrift: schwarze Tinte, lateinische Kurrent
\newline{}Ordnung: mit Bleistift von unbekannter Hand nummeriert:
                                    »143« }\toendnotes[C]{\smallbreak}\pstart
           \noindent{}{\pb}\textcolor{gray}{\textbf{Jung-Wiener Theater\orgindex{Jung-Wiener Theater zum Lieben Augustin@Jung-Wiener Theater zum Lieben Augustin|pw}}}\hfill \substVorne{}\textsuperscript{\textcolor{gray}{\textbf{Wien\oindex{Wien@\textbf{Wien}|pw}}}}\substDazwischen{}Berlin\oindex{Berlin@\textbf{Berlin}|pw}\substHinten{}\textcolor{gray}{\textbf{, }}{ }21. Septemb. \textcolor{gray}{\textbf{190}}1\pend
           \pstart
           \textcolor{gray}{\textbf{Zum lieben Augustin\orgindex{Jung-Wiener Theater zum Lieben Augustin@Jung-Wiener Theater zum Lieben Augustin|pw}.}}\hfill \strikeout{\textcolor{gray}{\textbf{(Theater a. d.
                           Wien\oindex{Theater an der Wien@\textbf{Theater an der Wien}|pw})}}}\pend
           \pstart
           \textcolor{gray}{\textbf{Direction.}}\pend
           \pstart
           Lieber Freund, bin seit einigen Tagen hier, und werde nach meiner
               Rückkehr das \label{K_L03319-1v}\edtext{verl. Manuscript}{\lemma{\textnormal{\emph{verl. Manuscript}}}\Cendnote{\textnormal{von Lanz\pwindex{Lantz, Adolf 10.11.1882 – 19.08.1949@\textsc{Lantz, Adolf} (10.11.1882 – 19.08.1949), \emph{Schriftsteller, Theaterleiter, Dramaturg}|pwuk}}}}\label{K_L03319-1h} zum Hausbesorger\pwindex{?? [Hausmeister von Felix Salten in der Kochgasse 1901] @\textsc{?? [Hausmeister von Felix Salten in der Kochgasse 1901]}|pwv}
               legen. Da ich bis jetzt krank und ziemlich unmöglich war habe ich weder Goldmann\pwindex{Goldmann, Paul 31.01.1865 – 25.09.1935@\textsc{Goldmann, Paul} (31.01.1865 – 25.09.1935), \emph{Schriftsteller, Journalist}|pw} noch {\pb}Kerr\pwindex{Kerr, Alfred 25.12.1867 – 12.10.1948@\textsc{Kerr, Alfred} (25.12.1867 – 12.10.1948), \emph{Schriftsteller, Kritiker}|pw} bisher aufgesucht. Wedekind\pwindex{Wedekind, Frank 24.07.1864 – 09.03.1918@\textsc{Wedekind, Frank} (24.07.1864 – 09.03.1918), \emph{Schriftsteller, Schauspieler}|pw} hat mir eben für Wien\oindex{Wien@\textbf{Wien}|pw} zugesagt. \textcolor{gray}{Usw.} werde ich wol kaum etwas finden. Das
               ist ein Niveau hier – ganz unwahrscheinlich. Und Salzer\pwindex{Salzer, Marcell 27.03.1873 – 17.03.1930@\textsc{Salzer, Marcell} (27.03.1873 – 17.03.1930), \emph{Schauspieler, Rezitator}|pw} nicht das Schlimmste dabei!! Donnerstag bin ich wieder
               in Wien\oindex{Wien@\textbf{Wien}|pw}. \pend
           \pstart
           Herzlichst Ihr {\\[\baselineskip]}\spacefill\mbox{Salten}\pend
           \leftskip=0em{}
         
         \endnumbering\mylabel{h}\end{ledgroupsized}\begin{anhang}\end{anhang}\newcommand{\dateiname}{L03319}\newcommand{\titel}{Felix Salten an Arthur Schnitzler, 21. 9. 1901}\newcommand{\editorInnen}{Martin Anton Müller und Laura Untner}%% latex-leseansicht-abspann.tex
%% Abspann für die Leseansicht.
%% Der Schalter \ifkorrekturansicht ist bereits durch den Vorspann gesetzt.

%% latex-abspann.tex
%% Gemeinsamer Abspann für Korrekturansicht und Leseansicht.
%% Setzt den Schalter \ifkorrekturansicht voraus (gesetzt in den
%% einbindenden Dateien latex-korrekturansicht-abspann.tex bzw.
%% latex-leseansicht-abspann.tex).
%% ---------------------------------------------------------------

\normalsize

% Das esempio-Environment wird nur in der Leseansicht benötigt
\ifkorrekturansicht\else
\newenvironment{esempio}[3]%
{
    \vspace{1.5ex}
    \rlap{\underline{#1}}
    \par
    \setlength{\parindent}{0cm}
    \nopagebreak
    \leftskip=#2cm
    \rightskip=#3cm
}
{
    \par
}
\fi

\doendnotes{C}
\bigskip
\vfill

\clearpage

\footnotesize

\ifkorrekturansicht
  \lohead{\textsc{register}}
\fi

% theindex-Environment neu definieren ohne reledmac
\makeatletter
\renewenvironment{theindex}{%
  \ifkorrekturansicht
    \section*{\indexname}%
  \else
    \subsubsection*{Index der erwähnten Entitäten}%
  \fi
  \setlength{\parindent}{0pt}%
  \setlength{\parskip}{0pt plus 0.3pt}%
  \let\item\@idxitem
}{%
  \ifkorrekturansicht\clearpage\fi
}
\makeatother

\IfFileExists{\jobname-pw.ind}{\input{\jobname-pw.ind}}{}

% Quellenangabe nur in der Leseansicht
\ifkorrekturansicht\else
% Fallback-Definitionen, falls die .tex-Datei \titel etc. nicht gesetzt hat
\providecommand{\titel}{}
\providecommand{\editorInnen}{}
\providecommand{\dateiname}{\jobname}

\vspace{3cm}

\vfill

\footnotesize
\textsc{Quelle}: \titel. Herausgegeben von {\editorInnen}. In: \emph{Arthur Schnitzler: Briefwechsel mit Autorinnen und Autoren}.
 Digitale Edition, https://schnitzler-briefe.acdh.oeaw.ac.at/{\dateiname}.html (Stand \today)
\fi

\end{document}


      