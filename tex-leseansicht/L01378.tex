%% latex-korrekturansicht-vorspann.tex
%% Vorspann für die Korrekturansicht.
%% Lädt die gemeinsame Datei latex-vorspann.tex mit gesetztem Schalter.

\newif\ifkorrekturansicht
\korrekturansichttrue

\input{../tex-inputs/latex-vorspann}


\section[Richard Dehmel an Arthur Schnitzler, {[}6. 3. 1904?{]}]{L01378 Richard Dehmel an Arthur Schnitzler, {[}6. 3. 1904?{]}}
\nopagebreak\mylabel{L01378v}
\rehead{ }\normalsize\beginnumbering\briefempfaengerindex{Schnitzler, Arthur@\textsc{Schnitzler, Arthur}!zzzDehmel, Richard@\emph{von Richard Dehmel}!1904-03-061@{{[}6. 3. 1904?{]}}|(be}
\toendnotes[C]{\smallbreak\pagebreak[2]}\Standort{CUL, Schnitzler, B 26.}
\physDesc{Visitenkarte, 20 Zeichen
\newline{}Handschrift: schwarze Tinte, lateinische Kurrent}\toendnotes[C]{\smallbreak}
\pstart
           \noindent{}\centering{}{\pb}\textcolor{gray}{\textbf{\label{T_L01378-1v}\edtext{R⋅Dehmel}{\lemma{\textnormal{\emph{R⋅Dehmel}}}\Cendnote{\textnormal{Die Verwendung der »Behrensschrift« von Peter Behrens\pwindex{Behrens, Peter 14.04.1868 – 27.02.1940@\textsc{Behrens, Peter} (14.04.1868 – 27.02.1940), \emph{Architekt/Architektin}|pwk} stellt sicher, dass die Visitenkarte
                     frühstens 1901 produziert wurde. Danach gibt es drei mögliche
                     Begegnungen, zu denen diese Karte im Vorfeld abgegeben worden sein kann: 7. 3. 1904, 15. 11. 1908 und
                        5. 11. 1911.
                     Gegen 1908 scheint die briefliche Kommunikation zu sprechen, die
                     das Treffen vorbereitet. 1911 ist ein geladenes Diner bei Dritten.
                     Deshalb ist der Vortag zum stattgefundenen Treffen 1904 ein
                     wahrscheinlicher Termin für die Abgabe dieser Karte. Es kann aber nicht
                     ausgeschlossen werden, dass sie zu einem nicht durch schriftliche Dokumente
                     feststellbaren Zeitpunkt übermittelt wurde.}}}\label{T_L01378-1}}}\pend
           
\pstart
           \centering{}mit ergebenstem Gruß\pend
           \selectlanguage{ngerman}\endnumbering\briefempfaengerindex{Schnitzler, Arthur@\textsc{Schnitzler, Arthur}!zzzDehmel, Richard@\emph{von Richard Dehmel}!1904-03-061@{{[}6. 3. 1904?{]}}|)be}\mylabel{L01378h}  \normalsize

\doendnotes{C}
\bigskip
\vfill

\clearpage

\footnotesize

\lohead{\textsc{register}}

% Definiere theindex-Environment komplett neu ohne reledmac
\makeatletter
\renewenvironment{theindex}{%
  \section*{\indexname}%
  \setlength{\parindent}{0pt}%
  \setlength{\parskip}{0pt plus 0.3pt}%
  \let\item\@idxitem
}{%
  \clearpage
}
\makeatother

\IfFileExists{\jobname-pw.ind}{\input{\jobname-pw.ind}}{}

\end{document}

      