%% latex-leseansicht-vorspann.tex
%% Vorspann für die Leseansicht.
%% Lädt die gemeinsame Datei latex-vorspann.tex mit nicht gesetztem Schalter.

\newif\ifkorrekturansicht
\korrekturansichtfalse

\input{../tex-inputs/latex-vorspann}


         
         \renewcommand{\erwaehntePersonen}{Personen: Peter Behrens}
         \renewcommand{\erwaehnteOrte}{Orte: Wien}
         \renewcommand{\erwaehnteWerke}{}
               \section[Richard Dehmel an Arthur Schnitzler, {[}6. 3. 1904?{]}]{ Richard Dehmel an Arthur Schnitzler, {[}6. 3. 1904?{]}}\nopagebreak\mylabel{v}\rehead{ }\begin{ledgroupsized}[t]{13cm}\normalsize\beginnumbering \toendnotes[C]{\smallbreak\pagebreak[2]} \Standort{CUL, Schnitzler, B 26.}
\physDesc{Visitenkarte, 20 Zeichen
\newline{}Handschrift: schwarze Tinte, lateinische Kurrent}\toendnotes[C]{\smallbreak}\pstart
           \noindent{}\centering{}{\pb}\textcolor{gray}{\textbf{\label{T_L01378_1v}\edtext{R⋅Dehmel}{\lemma{\textnormal{\emph{R⋅Dehmel}}}\Cendnote{\textnormal{Die Verwendung der »Behrensschrift« von Peter Behrens\pwindex{Behrens, Peter 14.04.1868 – 27.02.1940@\textsc{Behrens, Peter} (14.04.1868 – 27.02.1940), \emph{Architekt}|pwk} stellt sicher, dass die Visitenkarte
                     frühstens 1901 produziert wurde. Danach gibt es drei mögliche
                     Begegnungen, im Vorfeld dessen diese Karte abgegeben worden sein kann: 7. 3. 1904, 15. 11. 1908 und
                        5. 11. 1911.
                     Gegen 1908 scheint die briefliche Kommunikation zu sprechen, die
                     das Treffen vorbereitet. 1911 ist ein geladenes Diner bei Dritten.
                     Deshalb ist der Vortag zum stattgefundenen Treffen 1904 ein
                     wahrscheinlicher Termin für die Abgabe dieser Karte. Es kann aber nicht
                     ausgeschlossen werden, dass sie zu einem nicht durch schriftliche Dokumente
                     feststellbaren Zeitpunkt übermittelt wurde.}}}\label{T_L01378_1h}}}\pend
           \pstart
           \noindent{}\centering{}mit ergebenstem Gruß\pend
           
         
         \endnumbering\mylabel{h}\end{ledgroupsized}  \newcommand{\dateiname}{L01378}\newcommand{\titel}{Richard Dehmel an Arthur Schnitzler, [6. 3. 1904?]}\newcommand{\editorInnen}{Martin Anton Müller und Gerd-Hermann Susen}%% latex-leseansicht-abspann.tex
%% Abspann für die Leseansicht.
%% Der Schalter \ifkorrekturansicht ist bereits durch den Vorspann gesetzt.

%% latex-abspann.tex
%% Gemeinsamer Abspann für Korrekturansicht und Leseansicht.
%% Setzt den Schalter \ifkorrekturansicht voraus (gesetzt in den
%% einbindenden Dateien latex-korrekturansicht-abspann.tex bzw.
%% latex-leseansicht-abspann.tex).
%% ---------------------------------------------------------------

\normalsize

% Das esempio-Environment wird nur in der Leseansicht benötigt
\ifkorrekturansicht\else
\newenvironment{esempio}[3]%
{
    \vspace{1.5ex}
    \rlap{\underline{#1}}
    \par
    \setlength{\parindent}{0cm}
    \nopagebreak
    \leftskip=#2cm
    \rightskip=#3cm
}
{
    \par
}
\fi

\doendnotes{C}
\bigskip
\vfill

\clearpage

\footnotesize

\ifkorrekturansicht
  \lohead{\textsc{register}}
\fi

% theindex-Environment neu definieren ohne reledmac
\makeatletter
\renewenvironment{theindex}{%
  \ifkorrekturansicht
    \section*{\indexname}%
  \else
    \subsubsection*{Index der erwähnten Entitäten}%
  \fi
  \setlength{\parindent}{0pt}%
  \setlength{\parskip}{0pt plus 0.3pt}%
  \let\item\@idxitem
}{%
  \ifkorrekturansicht\clearpage\fi
}
\makeatother

\IfFileExists{\jobname-pw.ind}{\input{\jobname-pw.ind}}{}

% Quellenangabe nur in der Leseansicht
\ifkorrekturansicht\else
% Fallback-Definitionen, falls die .tex-Datei \titel etc. nicht gesetzt hat
\providecommand{\titel}{}
\providecommand{\editorInnen}{}
\providecommand{\dateiname}{\jobname}

\vspace{3cm}

\vfill

\footnotesize
\textsc{Quelle}: \titel. Herausgegeben von {\editorInnen}. In: \emph{Arthur Schnitzler: Briefwechsel mit Autorinnen und Autoren}.
 Digitale Edition, https://schnitzler-briefe.acdh.oeaw.ac.at/{\dateiname}.html (Stand \today)
\fi

\end{document}


      