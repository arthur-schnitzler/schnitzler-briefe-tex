%% latex-leseansicht-vorspann.tex
%% Vorspann für die Leseansicht.
%% Lädt die gemeinsame Datei latex-vorspann.tex mit nicht gesetztem Schalter.

\newif\ifkorrekturansicht
\korrekturansichtfalse

\input{../tex-inputs/latex-vorspann}


\section[Arthur Schnitzler an Richard Beer-Hofmann, 21. 6. 1899]{L00928 Arthur Schnitzler an Richard Beer-Hofmann, 21. 6. 1899}
\nopagebreak\mylabel{L00928v}
\rehead{ }\normalsize\beginnumbering\briefempfaengerindex{Beer-Hofmann, Richard@\textsc{Beer-Hofmann, Richard}!zzzSchnitzler, Arthur@\emph{von Arthur Schnitzler}!1899-06-211@{21. 6. 1899}|(be}
\toendnotes[C]{\smallbreak\pagebreak[2]}
\correspDesc{Versand  durch Arthur Schnitzler am 21. 6. 1899 in Wien
\newline{}Erhalt  durch Richard Beer-Hofmann im Zeitraum [22. 6. 1899
                  – 26. 6. 1899?] in Seeboden}\toendnotes[C]{\smallbreak}
\Standort{CUL, Schnitzler, B 8.1, S. 79.}
\physDesc{Brief, maschinenschriftliche Abschrift, 1 Blatt, 1 Seite, 1772 Zeichen
\newline{}Schreibmaschine
\newline{}Ordnung: mit Bleistift von unbekannter Hand nummeriert:
                                    »136« }
\buchAbdrucke{\weitereDrucke{Arthur Schnitzler, Richard Beer-Hofmann: \emph{Briefwechsel 1891–1931}. Herausgegeben von Konstanze Fliedl. Wien, Zürich: \emph{Europaverlag} 1992, S. 130.} }\toendnotes[C]{\smallbreak}
\pstart
           \raggedleft{}{\pb}Wien\oindex{Wien@\textbf{Wien}, \emph{Verwaltungsgebiet}|pw}, 21. 6. 99.\pend
           \vspace{0.5em}
\pstart
           Lieber Richard, ich habe gestern Abend mit Mayer\pwindex{Mayer, Oskar 1876 – 15.\,5.\,1915 München@\textsc{Mayer, Oskar} (1876 – 15.\,5.\,1915 München), \emph{Schriftsteller, Beamter}|pw} gesprochen. Wir schlagen folgendes vor: dass wir etwa
                  Mitte Juli zu Ihnen kommen und Sie von dort mitnehmen (etwa 5 Tage
               später). Wohin? Mir wäre ebenso wie Mayer\pwindex{Mayer, Oskar 1876 – 15.\,5.\,1915 München@\textsc{Mayer, Oskar} (1876 – 15.\,5.\,1915 München), \emph{Schriftsteller, Beamter}|pw} eine
               Tour im südtirol\oindex{Südtirol@\textbf{Südtirol}, \emph{Verwaltungsgebiet}|pw}ischen am sympathischesten (eine
               Zusammenstellung hab ich); ich will nämlich dann, vielleicht mit Mayer\pwindex{Mayer, Oskar 1876 – 15.\,5.\,1915 München@\textsc{Mayer, Oskar} (1876 – 15.\,5.\,1915 München), \emph{Schriftsteller, Beamter}|pw}, an irgend einen hohen Punkt (San Martino\oindex{San Martino di Castrozza@\textbf{San Martino di Castrozza}|pw}) 2–3 Wochen bleiben, auch länger und dort zu
               arbeiten versuchen. Denn ich fühle, dass mein Organismus nach Höhenluft verlangt.
               Ihrer wahrscheinlich auch. Man hat ja offenbar nie recht, einem Menschen zu sagen, er
               habe keinen Grund verstimmt zu sein; – aber dass es \uline{mir} heuer sehr nahe liegt, Ihnen irgendwas in der Art zu sagen, werden Sie
               verzeihlich finden. Ich hoffe, Sie erholen sich – von was? – Mir kommt vor, ich wär
               an Ihrer Stelle so glücklich, dass mich schauern müsste, aber offenbar irr ich mich.
               Aber im Ernst, was haben Sie? – Mir scheint nun einmal, dass Sie selbst einfach durch
               Willen einiges dazu thun könnten, um wohl zu sein. Sie lassen sich gehn. Freilich,
               auch dagegen scheinen Sie keine Gewalt zu haben.\pend
           
\pstart
           Was mich anbelangt, so fühle ich jenes \label{K_L00928-1v}\edtext{Unglück}{\lemma{\textnormal{\emph{Unglück}}}\Cendnote{\textnormal{der Tod Marie Reinhards\pwindex{Reinhard, Marie 13.\,3.\,1871 Wien – 18.\,3.\,1899 ebd.@\textsc{Reinhard, Marie} (13.\,3.\,1871 Wien – 18.\,3.\,1899 ebd.), \emph{Gesangspädagogin}|pwk} am 18. 3. 1899}}}\label{K_L00928-1} mit {\pb}jedem Tag tiefer; der Sommer
               hat so seine eigenen Qualen. – Zu arbeiten hab ich versucht. – Mit Hugo\pwindex{Hofmannsthal, Hugo von 1.\,2.\,1874 Wien – 15.\,7.\,1929 Rodaun@\textsc{Hofmannsthal, Hugo von} (1.\,2.\,1874 Wien – 15.\,7.\,1929 Rodaun), \emph{Schriftsteller}|pw} hab ich gestern eine schöne Radpartie
               gemacht: Edlacher Hof\oindex{Hotel Edlacherhof@\textbf{Hotel Edlacherhof}, \emph{Hotel}|pw} – Singerin\oindex{Gasthof zur Singerin@\textbf{Gasthof zur Singerin}, \emph{Gastgewerbegebäude}|pw} – Gutenstein\oindex{Gutenstein@\textbf{Gutenstein}, \emph{Verwaltungsgebiet}|pw} – Pottenstein\oindex{Pottenstein@\textbf{Pottenstein}, \emph{Hauptstadt}|pw} – Vöslau\oindex{Bad Vöslau@\textbf{Bad Vöslau}, \emph{Hauptstadt}|pw}.\pend
           
\pstart
           Morgen Abend fahr ich nach \label{K_L00928-2v}\edtext{Slavonien\oindex{Slawonien@\textbf{Slawonien}, \emph{Region}|pw}}{\lemma{\textnormal{\emph{Slavonien}}}\Cendnote{\textnormal{Am 21. 6. 1899 reiste Schnitzler nach Belišće\oindex{Belišće@\textbf{Belišće}, \emph{Verwaltungsgebiet}|pwk}, blieb 2 Tage, fuhr dann weiter nach Orahovica\oindex{Orahovica@\textbf{Orahovica}, \emph{Hauptstadt}|pwk}, wo er ebenfalls für zwei Tage blieb. Über Budapest\oindex{Budapest@\textbf{Budapest}, \emph{Hauptstadt}|pwk} reiste er am 21. 6. 1899 retour.}}}\label{K_L00928-2} und wünsche in den
               letzten Junitagen wieder hier zu sein. Dann bleib ich etwa 10–12 Tage
               hier.\pend
           
\pstart
           Paul Goldmanns\pwindex{Goldmann, Paul 31.\,1.\,1865 Breslau – 25.\,9.\,1935 Wien@\textsc{Goldmann, Paul} (31.\,1.\,1865 Breslau – 25.\,9.\,1935 Wien), \emph{Schriftsteller, Journalist}|pw} Adresse einfach Frankfurter Zeitung\orgindex{Frankfurter Zeitung@Frankfurter Zeitung|pw}.\pend
           
\pstart
           Die tirolische\oindex{Tirol@\textbf{Tirol}, \emph{Land}|pw}\oindex{Südtirol@\textbf{Südtirol}, \emph{Verwaltungsgebiet}|pw} Tour ist ungefähr; oder
               wäre: Niederdorf\oindex{Niederdorf@\textbf{Niederdorf}, \emph{Verwaltungsgebiet}|pw} – Schluderbach\oindex{Carbonin@\textbf{Carbonin}|pw} – Tre \label{T_L00928-1v}\edtext{Croci}{\lemma{\textnormal{\emph{Croci}}}\Cendnote{\textnormal{In der Abschrift steht: »Croce«.}}}\label{T_L00928-1}\oindex{Tre Croci@\textbf{Tre Croci}, \emph{Berg}|pw} – Cortina\oindex{Cortina d’Ampezzo@\textbf{Cortina d’Ampezzo}, \emph{Hauptstadt}|pw} – Caprile\oindex{Alleghe@\textbf{Alleghe}, \emph{Hauptstadt}|pw} – \label{T_L00928-2v}\edtext{Fedaja\oindex{Passo Fedaia@\textbf{Passo Fedaia}, \emph{Pass}|pw}}{\lemma{\textnormal{\emph{Fedaja}}}\Cendnote{\textnormal{In der Abschrift steht:
                     »Tevaja«.}}}\label{T_L00928-2} – \label{T_L00928-3v}\edtext{Karersee\oindex{Karersee@\textbf{Karersee}, \emph{See}|pw}}{\lemma{\textnormal{\emph{Karersee}}}\Cendnote{\textnormal{In der Abschrift steht
                     »Karrersee«.}}}\label{T_L00928-3} – Rollepass\oindex{Passo Rolle@\textbf{Passo Rolle}, \emph{Pass}|pw} – Martino\oindex{San Martino di Castrozza@\textbf{San Martino di Castrozza}|pw} – Trient\oindex{Trient@\textbf{Trient}|pw}.\pend
           
\pstart
           Einfacher: Bozen\oindex{Bozen@\textbf{Bozen}, \emph{Hauptstadt}|pw} – \label{T_L00928-4v}\edtext{Karersee\oindex{Karersee@\textbf{Karersee}, \emph{See}|pw}}{\lemma{\textnormal{\emph{Karersee}}}\Cendnote{\textnormal{In der Abschrift steht
                     »Karrersee«.}}}\label{T_L00928-4} u. s. w.\pend
           
\pstart
           Leben Sie wohl, grüssen Sie Weib\pwindex{Beer-Hofmann, Paula 25.\,2.\,1879 Wien – 30.\,10.\,1939 Zürich@\textsc{Beer-Hofmann, Paula} (25.\,2.\,1879 Wien – 30.\,10.\,1939 Zürich)|pwv} und Kind\pwindex{Beer-Hofmann, Mirjam 4.\,9.\,1897 Wien – 24.\,12.\,1984 New York City@\textsc{Beer-Hofmann, Mirjam} (4.\,9.\,1897 Wien – 24.\,12.\,1984 New York City)|pwv}\pwindex{Beer-Hofmann, Naëmah 20.\,12.\,1898 Wien – 10.\,11.\,1971 New York City@\textsc{Beer-Hofmann, Naëmah} (20.\,12.\,1898 Wien – 10.\,11.\,1971 New York City)|pwv}.\pend
           \pstart Herzlich der Ihre \spacefill\mbox{Arthur.}\pend{}
\pstart
           \noindent{}(nach Seeboden\oindex{Seeboden am Millstättersee@\textbf{Seeboden am Millstättersee}|pw})\pend
           \selectlanguage{ngerman}\endnumbering\briefempfaengerindex{Beer-Hofmann, Richard@\textsc{Beer-Hofmann, Richard}!zzzSchnitzler, Arthur@\emph{von Arthur Schnitzler}!1899-06-211@{21. 6. 1899}|)be}\mylabel{L00928h}  \newcommand{\dateiname}{L00928}\newcommand{\titel}{Arthur Schnitzler an Richard Beer-Hofmann, 21. 6. 1899}\newcommand{\editorInnen}{Martin Anton Müller und Gerd-Hermann Susen}%% latex-leseansicht-abspann.tex
%% Abspann für die Leseansicht.
%% Der Schalter \ifkorrekturansicht ist bereits durch den Vorspann gesetzt.

%% latex-abspann.tex
%% Gemeinsamer Abspann für Korrekturansicht und Leseansicht.
%% Setzt den Schalter \ifkorrekturansicht voraus (gesetzt in den
%% einbindenden Dateien latex-korrekturansicht-abspann.tex bzw.
%% latex-leseansicht-abspann.tex).
%% ---------------------------------------------------------------

\normalsize

% Das esempio-Environment wird nur in der Leseansicht benötigt
\ifkorrekturansicht\else
\newenvironment{esempio}[3]%
{
    \vspace{1.5ex}
    \rlap{\underline{#1}}
    \par
    \setlength{\parindent}{0cm}
    \nopagebreak
    \leftskip=#2cm
    \rightskip=#3cm
}
{
    \par
}
\fi

\doendnotes{C}
\bigskip
\vfill

\clearpage

\footnotesize

\ifkorrekturansicht
  \lohead{\textsc{register}}
\fi

% theindex-Environment neu definieren ohne reledmac
\makeatletter
\renewenvironment{theindex}{%
  \ifkorrekturansicht
    \section*{\indexname}%
  \else
    \subsubsection*{Index der erwähnten Entitäten}%
  \fi
  \setlength{\parindent}{0pt}%
  \setlength{\parskip}{0pt plus 0.3pt}%
  \let\item\@idxitem
}{%
  \ifkorrekturansicht\clearpage\fi
}
\makeatother

\IfFileExists{\jobname-pw.ind}{\input{\jobname-pw.ind}}{}

% Quellenangabe nur in der Leseansicht
\ifkorrekturansicht\else
% Fallback-Definitionen, falls die .tex-Datei \titel etc. nicht gesetzt hat
\providecommand{\titel}{}
\providecommand{\editorInnen}{}
\providecommand{\dateiname}{\jobname}

\vspace{3cm}

\vfill

\footnotesize
\textsc{Quelle}: \titel. Herausgegeben von {\editorInnen}. In: \emph{Arthur Schnitzler: Briefwechsel mit Autorinnen und Autoren}.
 Digitale Edition, https://schnitzler-briefe.acdh.oeaw.ac.at/{\dateiname}.html (Stand \today)
\fi

\end{document}


