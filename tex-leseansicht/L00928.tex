%% latex-korrekturansicht-vorspann.tex
%% Vorspann für die Korrekturansicht.
%% Lädt die gemeinsame Datei latex-vorspann.tex mit gesetztem Schalter.

\newif\ifkorrekturansicht
\korrekturansichttrue

\input{../tex-inputs/latex-vorspann}


\section[Arthur Schnitzler an Richard Beer-Hofmann, 21. 6. 1899]{L00928 Arthur Schnitzler an Richard Beer-Hofmann, 21. 6. 1899}
\nopagebreak\mylabel{L00928v}
\rehead{ }\normalsize\beginnumbering\briefempfaengerindex{Beer-Hofmann, Richard@\textsc{Beer-Hofmann, Richard}!zzzSchnitzler, Arthur@\emph{von Arthur Schnitzler}!1899-06-211@{21. 6. 1899}|(be}
\toendnotes[C]{\smallbreak\pagebreak[2]}\Standort{CUL, Schnitzler, B 8.1, S. 79.}
\physDesc{Brief, maschinenschriftliche Abschrift1 Blatt, 1 Seite, 1772 Zeichen
\newline{}Schreibmaschine
\newline{}Ordnung: mit Bleistift von unbekannter Hand nummeriert:
                                    »136« }
\buchAbdrucke{\weitereDrucke{Arthur Schnitzler, Richard Beer-Hofmann: \emph{Briefwechsel 1891–1931}. Wien, Zürich: \emph{Europaverlag} 1992, S. 130.} }\toendnotes[C]{\smallbreak}
\pstart
           \raggedleft{}{\pb}Wien\oindex{Wien@\textbf{Wien}, \emph{A.ADM2}|pw}, 21. 6. 99.\pend
           \vspace{0.5em}
\pstart
           Lieber Richard, ich habe gestern Abend mit Mayer\pwindex{Mayer, Oskar 1876 – 15.05.1915@\textsc{Mayer, Oskar} (1876 – 15.05.1915), \emph{Schriftsteller/Schriftstellerin, Beamter/Beamte}|pw} gesprochen. Wir schlagen folgendes vor: dass wir etwa
                  Mitte Juli zu Ihnen kommen und Sie von dort mitnehmen (etwa 5 Tage
               später). Wohin? Mir wäre ebenso wie Mayer\pwindex{Mayer, Oskar 1876 – 15.05.1915@\textsc{Mayer, Oskar} (1876 – 15.05.1915), \emph{Schriftsteller/Schriftstellerin, Beamter/Beamte}|pw} eine
               Tour im südtirol\oindex{Suedtirol@\textbf{Südtirol}, \emph{A.ADM2}|pw}ischen am sympathischesten (eine
               Zusammenstellung hab ich); ich will nämlich dann, vielleicht mit Mayer\pwindex{Mayer, Oskar 1876 – 15.05.1915@\textsc{Mayer, Oskar} (1876 – 15.05.1915), \emph{Schriftsteller/Schriftstellerin, Beamter/Beamte}|pw}, an irgend einen hohen Punkt (San Martino\oindex{San Martino di Castrozza@\textbf{San Martino di Castrozza}, \emph{P.PPL}|pw}) 2–3 Wochen bleiben, auch länger und dort zu
               arbeiten versuchen. Denn ich fühle, dass mein Organismus nach Höhenluft verlangt.
               Ihrer wahrscheinlich auch. Man hat ja offenbar nie recht, einem Menschen zu sagen, er
               habe keinen Grund verstimmt zu sein; – aber dass es \uline{mir} heuer sehr nahe liegt, Ihnen irgendwas in der Art zu sagen, werden Sie
               verzeihlich finden. Ich hoffe, Sie erholen sich – von was? – Mir kommt vor, ich wär
               an Ihrer Stelle so glücklich, dass mich schauern müsste, aber offenbar irr ich mich.
               Aber im Ernst, was haben Sie? – Mir scheint nun einmal, dass Sie selbst einfach durch
               Willen einiges dazu thun könnten, um wohl zu sein. Sie lassen sich gehn. Freilich,
               auch dagegen scheinen Sie keine Gewalt zu haben.\pend
           
\pstart
           Was mich anbelangt, so fühle ich jenes \label{K_L00928-1v}\edtext{Unglück}{\lemma{\textnormal{\emph{Unglück}}}\Cendnote{\textnormal{der Tod Marie Reinhards\pwindex{Reinhard, Marie 1871-03-13 – 1899-03-18@\textsc{Reinhard, Marie} (1871-03-13 – 1899-03-18), \emph{Gesangspädagoge/Gesangspädagogin}|pwk} am 18. 3. 1899}}}\label{K_L00928-1} mit {\pb}jedem Tag tiefer; der Sommer
               hat so seine eigenen Qualen. – Zu arbeiten hab ich versucht. – Mit Hugo\pwindex{Hofmannsthal, Hugo von 1874-02-01 – 1929-07-15@\textsc{Hofmannsthal, Hugo von} (1874-02-01 – 1929-07-15), \emph{Schriftsteller/Schriftstellerin}|pw} hab ich gestern eine schöne Radpartie
               gemacht: Edlacher Hof\oindex{Hotel Edlacherhof@\textbf{Hotel Edlacherhof}, \emph{Hotel (K.HTL)}|pw} – Singerin\oindex{Gasthof zur Singerin@\textbf{Gasthof zur Singerin}, \emph{Gastgewerbegebäude (K.GGW)}|pw} – Gutenstein\oindex{Gutenstein@\textbf{Gutenstein}, \emph{A.ADM3}|pw} – Pottenstein\oindex{Pottenstein@\textbf{Pottenstein}, \emph{P.PPLA3}|pw} – Vöslau\oindex{Bad Voeslau@\textbf{Bad Vöslau}, \emph{P.PPLA3}|pw}.\pend
           
\pstart
           Morgen Abend fahr ich nach \label{K_L00928-2v}\edtext{Slavonien\oindex{Slawonien@\textbf{Slawonien}, \emph{L.RGN}|pw}}{\lemma{\textnormal{\emph{Slavonien}}}\Cendnote{\textnormal{Am 21. 6. 1899 reiste Schnitzler nach Belišće\oindex{Belišće@\textbf{Belišće}, \emph{A.ADM2}|pwk}, blieb 2 Tage, fuhr dann weiter nach Orahovica\oindex{Orahovica@\textbf{Orahovica}, \emph{P.PPLA2}|pwk}, wo er ebenfalls für zwei Tage blieb. Über Budapest\oindex{Budapest@\textbf{Budapest}, \emph{P.PPLC}|pwk} reiste er am 21. 6. 1899 retour.}}}\label{K_L00928-2} und wünsche in den
               letzten Junitagen wieder hier zu sein. Dann bleib ich etwa 10–12 Tage
               hier.\pend
           
\pstart
           Paul Goldmanns\pwindex{Goldmann, Paul 31.01.1865 – 25.09.1935@\textsc{Goldmann, Paul} (31.01.1865 – 25.09.1935), \emph{Schriftsteller/Schriftstellerin, Journalist/Journalistin}|pw} Adresse einfach Frankfurter Zeitung\orgindex{Frankfurter Zeitung@Frankfurter Zeitung|pw}.\pend
           
\pstart
           Die tirolische\oindex{Tirol@\textbf{Tirol}, \emph{A.ADM1}|pw}\oindex{Suedtirol@\textbf{Südtirol}, \emph{A.ADM2}|pw} Tour ist ungefähr; oder
               wäre: Niederdorf\oindex{Niederdorf@\textbf{Niederdorf}, \emph{A.ADM3}|pw} – Schluderbach\oindex{Carbonin@\textbf{Carbonin}, \emph{P.PPL}|pw} – Tre \label{T_L00928-1v}\edtext{Croci}{\lemma{\textnormal{\emph{Croci}}}\Cendnote{\textnormal{In der Abschrift steht: »Croce«.}}}\label{T_L00928-1}\oindex{Tre Croci@\textbf{Tre Croci}, \emph{Berg (N.BRG)}|pw} – Cortina\oindex{Cortina DAmpezzo@\textbf{Cortina d’Ampezzo}, \emph{P.PPLA3}|pw} – Caprile\oindex{Alleghe@\textbf{Alleghe}, \emph{P.PPLA3}|pw} – \label{T_L00928-2v}\edtext{Fedaja\oindex{Passo Fedaia@\textbf{Passo Fedaia}, \emph{Pass (N.PAS)}|pw}}{\lemma{\textnormal{\emph{Fedaja}}}\Cendnote{\textnormal{In der Abschrift steht:
                     »Tevaja«.}}}\label{T_L00928-2} – \label{T_L00928-3v}\edtext{Karersee\oindex{Karersee@\textbf{Karersee}, \emph{See (N.SEE)}|pw}}{\lemma{\textnormal{\emph{Karersee}}}\Cendnote{\textnormal{In der Abschrift steht
                     »Karrersee«.}}}\label{T_L00928-3} – Rollepass\oindex{Passo Rolle@\textbf{Passo Rolle}, \emph{Pass (N.PAS)}|pw} – Martino\oindex{San Martino di Castrozza@\textbf{San Martino di Castrozza}, \emph{P.PPL}|pw} – Trient\oindex{Trient@\textbf{Trient}, \emph{P.PPLA}|pw}.\pend
           
\pstart
           Einfacher: Bozen\oindex{Bozen@\textbf{Bozen}, \emph{P.PPLA2}|pw} – \label{T_L00928-4v}\edtext{Karersee\oindex{Karersee@\textbf{Karersee}, \emph{See (N.SEE)}|pw}}{\lemma{\textnormal{\emph{Karersee}}}\Cendnote{\textnormal{In der Abschrift steht
                     »Karrersee«.}}}\label{T_L00928-4} u. s. w.\pend
           
\pstart
           Leben Sie wohl, grüssen Sie Weib\pwindex{Beer-Hofmann, Paula 25.02.1879 – 30.10.1939@\textsc{Beer-Hofmann, Paula} (25.02.1879 – 30.10.1939)|pwv} und Kind\pwindex{Beer-Hofmann, Mirjam 04.09.1897 – 24.12.1984@\textsc{Beer-Hofmann, Mirjam} (04.09.1897 – 24.12.1984)|pwv}\pwindex{Beer-Hofmann, Naemah 20.12.1898 – 10.11.1971@\textsc{Beer-Hofmann, Naëmah} (20.12.1898 – 10.11.1971)|pwv}.\pend
           \pstart Herzlich der Ihre \spacefill\mbox{Arthur.}\pend{}
\pstart
           \noindent{}(nach Seeboden\oindex{Seeboden@\textbf{Seeboden}, \emph{A.ADM3}|pw})\pend
           \selectlanguage{ngerman}\endnumbering\briefempfaengerindex{Beer-Hofmann, Richard@\textsc{Beer-Hofmann, Richard}!zzzSchnitzler, Arthur@\emph{von Arthur Schnitzler}!1899-06-211@{21. 6. 1899}|)be}\mylabel{L00928h}  \normalsize

\doendnotes{C}
\bigskip
\vfill

\clearpage

\footnotesize

\lohead{\textsc{register}}

% Definiere theindex-Environment komplett neu ohne reledmac
\makeatletter
\renewenvironment{theindex}{%
  \section*{\indexname}%
  \setlength{\parindent}{0pt}%
  \setlength{\parskip}{0pt plus 0.3pt}%
  \let\item\@idxitem
}{%
  \clearpage
}
\makeatother

\IfFileExists{\jobname-pw.ind}{\input{\jobname-pw.ind}}{}

\end{document}

      