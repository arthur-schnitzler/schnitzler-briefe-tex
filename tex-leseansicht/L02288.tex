%% latex-korrekturansicht-vorspann.tex
%% Vorspann für die Korrekturansicht.
%% Lädt die gemeinsame Datei latex-vorspann.tex mit gesetztem Schalter.

\newif\ifkorrekturansicht
\korrekturansichttrue

\input{../tex-inputs/latex-vorspann}


\section[Arthur Schnitzler an Robert Adam, 15. 6. 1918]{L02288 Arthur Schnitzler an Robert Adam, 15. 6. 1918}
\nopagebreak\mylabel{L02288v}
\rehead{ }\normalsize\beginnumbering\briefempfaengerindex{Adam, Robert@\textsc{Adam, Robert}!zzzSchnitzler, Arthur@\emph{von Arthur Schnitzler}!1918-06-151@{15. 6. 1918}|(be}
\toendnotes[C]{\smallbreak\pagebreak[2]}\Standort{DLA, 96.34.2/9.}
\physDesc{Postkarte, 365 Zeichen
\newline{}Handschrift: schwarze Tinte, deutsche Kurrent
\newline{}Versand: Stempel: »\nobreak{}15. {[}6. 1918{]}\nobreak{}«.  }\pstart{}{\pb}Wien XVIII. \textsc{Sternwartestr}. 71\oindex{VIII., Josefstadt@\textbf{VIII., Josefstadt}, \emph{A.ADM3}|pw}.\pend{}{\bigskip}\pstart{}Herrn \textsc{Dr. Robert Adam}\pend{}\pstart{}\textsc{Pollak}.\pend{}\pstart{}\textsc{Wien} XII\oindex{XII., Meidling@\textbf{XII., Meidling}, \emph{A.ADM3}|pw}\pend{}\pstart{}\textsc{Meidlinger Hauptstr} 58\oindex{Meidlinger Hauptstrasse@\textbf{Meidlinger Hauptstraße}, \emph{Straße (K.STR)}|pw}.\pend{}{\bigskip}\vspace{1em}
\pstart
           {\pb}\textcolor{gray}{\textbf{A. S.}}\hfill 15. 6. 18\pend
           
\pstart{}Verehrter Herr Doktor, \pend\vspace{0.5em}
\pstart
           vielen Dank für Ihren lieben Brief. Paſſt es Ihnen, ſo erwarte ich Sie gern am
                  Mitwwoch (19.) gegen 7 Uhr Abends Oder geben Sie ſelbſt
               einen anderen Tag an. Jedenfalls ſreue ich mich Sie bald wiederzuſehen.\pend
           
\pstart
           Herzlich grüßt Sie ſehr ergeb{\\[\baselineskip]}\spacefill\mbox{Arthur Schnitzler.}\pend
           \leftskip=0em{}\selectlanguage{ngerman}\endnumbering\briefempfaengerindex{Adam, Robert@\textsc{Adam, Robert}!zzzSchnitzler, Arthur@\emph{von Arthur Schnitzler}!1918-06-151@{15. 6. 1918}|)be}\mylabel{L02288h}  \normalsize

\doendnotes{C}
\bigskip
\vfill

\clearpage

\footnotesize

\lohead{\textsc{register}}

% Definiere theindex-Environment komplett neu ohne reledmac
\makeatletter
\renewenvironment{theindex}{%
  \section*{\indexname}%
  \setlength{\parindent}{0pt}%
  \setlength{\parskip}{0pt plus 0.3pt}%
  \let\item\@idxitem
}{%
  \clearpage
}
\makeatother

\IfFileExists{\jobname-pw.ind}{\input{\jobname-pw.ind}}{}

\end{document}

      