%% latex-leseansicht-vorspann.tex
%% Vorspann für die Leseansicht.
%% Lädt die gemeinsame Datei latex-vorspann.tex mit nicht gesetztem Schalter.

\newif\ifkorrekturansicht
\korrekturansichtfalse

\input{../tex-inputs/latex-vorspann}


\section[Arthur Schnitzler: Widmungsexemplar Episode für Hugo von Hofmannsthal, {[}2. 2. 1892?{]}]{L00070 Arthur Schnitzler: Widmungsexemplar Episode für Hugo von Hofmannsthal, {[}2. 2. 1892?{]}}
\nopagebreak\mylabel{L00070v}
\rehead{ }\normalsize\beginnumbering\briefempfaengerindex{Hofmannsthal, Hugo von@\textsc{Hofmannsthal, Hugo von}!zzzSchnitzler, Arthur@\emph{von Arthur Schnitzler}!1892-02-021@{{[}2. 2. 1892?{]}}|(be}
\toendnotes[C]{\smallbreak\pagebreak[2]}
\correspDesc{Versand  durch Arthur Schnitzler am [2. 2. 1892?] in Wien
\newline{}Erhalt  durch Hugo von Hofmannsthal im Zeitraum [2. 2. 1892
                  – 6. 2. 1892?] \textbf{Ort fehlend} }\toendnotes[C]{\smallbreak}
\Standort{FDH, FDH 3223.}
\physDesc{Widmung am Titelblatt, 50 Zeichen
\newline{}Handschrift: schwarze Tinte, deutsche Kurrent}
\buchAbdrucke{\weitereDrucke{Hugo von Hofmannsthal: \emph{Bibliothek}. Herausgegeben von Ellen Ritter † in Zusammenarbeit mit Dalia Bukauskaité und Konrad Heumann. Frankfurt am Main: \emph{S. Fischer} 2011, S. 603 (Sämtliche Werke. Kritische Ausgabe, XL).} }\toendnotes[C]{\smallbreak}
\pstart
           \noindent{}{\pb}Meinem{ }ſehr verehrten Freunde \textsc{Loris}\pend
           
\pstart
           herzlichſt{\\[\baselineskip]}\spacefill\mbox{Arth}\pend
           \leftskip=0em{}{\vspace{1\baselineskip}}
\pstart
           \centering{}\textcolor{gray}{\textbf{\textbf{Epiſode\pwindex{Schnitzler, Arthur 15.\,5.\,1862 Wien – 21.\,10.\,1931 ebd.@\textsc{Schnitzler, Arthur} (15.\,5.\,1862 Wien – 21.\,10.\,1931 ebd.), \emph{Schriftsteller, Mediziner}!Episode@\strich\emph{Episode}|pw}.}}}\pend
           
\pstart
           \centering{}\textcolor{gray}{\textbf{Von}}{\\}\textcolor{gray}{\textbf{Arthur Schnitzler.}}\pend
           {\vspace{1\baselineskip}}
\pstart
           \centering{}\textcolor{gray}{\textbf{Den Bühnen gegenüber als Manuſcript.}}\pend
           
\pstart
           \centering{}\textcolor{gray}{\textbf{Wien\oindex{Wien@\textbf{Wien}, \emph{Verwaltungsgebiet}|pw}, \label{K_L00070-1v}\edtext{1889}{\lemma{\textnormal{\emph{1889}}}\Cendnote{\textnormal{Die Widmung ist nicht datiert. Der
                     Separatdruck aus Heft 18 der Zeitschrift \emph{An der
                        schönen blauen Donau}\pwindex{der schönen blauen Donau@\emph{An der schönen blauen Donau}|pwk} erschien am 8. 9. 1889, als sich die beiden noch
                     nicht kannten. Da auch handschriftliche Änderungen enthalten sind, die in der
                     Druckfassung berücksichtigt wurden, ist eine Übergabe kurz vor Ablieferung des
                     Verlagsmanuskripts der Buchausgabe\pwindex{Schnitzler, Arthur 15.\,5.\,1862 Wien – 21.\,10.\,1931 ebd.@\textsc{Schnitzler, Arthur} (15.\,5.\,1862 Wien – 21.\,10.\,1931 ebd.), \emph{Schriftsteller, Mediziner}!Anatol@\strich\emph{Anatol}|pwkv} am 5. 2. 1892 ein wahrscheinlicher Termin. Und da bietet sich
                     wiederum der 2. 2. 1892 an, da an diesem Tag Hofmannsthal\pwindex{Hofmannsthal, Hugo von 1.\,2.\,1874 Wien – 15.\,7.\,1929 Rodaun@\textsc{Hofmannsthal, Hugo von} (1.\,2.\,1874 Wien – 15.\,7.\,1929 Rodaun), \emph{Schriftsteller}|pwk} seinen \emph{Prolog}\pwindex{Hofmannsthal, Hugo von 1.\,2.\,1874 Wien – 15.\,7.\,1929 Rodaun@\textsc{Hofmannsthal, Hugo von} (1.\,2.\,1874 Wien – 15.\,7.\,1929 Rodaun), \emph{Schriftsteller}!Prolog [zum Anatol]@\strich\emph{Prolog [zum Anatol]}|pwk} verfasst hat.}}}\label{K_L00070-1}.}}\pend
           \selectlanguage{ngerman}\endnumbering\briefempfaengerindex{Hofmannsthal, Hugo von@\textsc{Hofmannsthal, Hugo von}!zzzSchnitzler, Arthur@\emph{von Arthur Schnitzler}!1892-02-021@{{[}2. 2. 1892?{]}}|)be}\mylabel{L00070h}  \newcommand{\dateiname}{L00070}\newcommand{\titel}{Arthur Schnitzler: Widmungsexemplar Episode für Hugo von Hofmannsthal, [2. 2. 1892?]}\newcommand{\editorInnen}{Martin Anton Müller und Gerd-Hermann Susen}%% latex-leseansicht-abspann.tex
%% Abspann für die Leseansicht.
%% Der Schalter \ifkorrekturansicht ist bereits durch den Vorspann gesetzt.

%% latex-abspann.tex
%% Gemeinsamer Abspann für Korrekturansicht und Leseansicht.
%% Setzt den Schalter \ifkorrekturansicht voraus (gesetzt in den
%% einbindenden Dateien latex-korrekturansicht-abspann.tex bzw.
%% latex-leseansicht-abspann.tex).
%% ---------------------------------------------------------------

\normalsize

% Das esempio-Environment wird nur in der Leseansicht benötigt
\ifkorrekturansicht\else
\newenvironment{esempio}[3]%
{
    \vspace{1.5ex}
    \rlap{\underline{#1}}
    \par
    \setlength{\parindent}{0cm}
    \nopagebreak
    \leftskip=#2cm
    \rightskip=#3cm
}
{
    \par
}
\fi

\doendnotes{C}
\bigskip
\vfill

\clearpage

\footnotesize

\ifkorrekturansicht
  \lohead{\textsc{register}}
\fi

% theindex-Environment neu definieren ohne reledmac
\makeatletter
\renewenvironment{theindex}{%
  \ifkorrekturansicht
    \section*{\indexname}%
  \else
    \subsubsection*{Index der erwähnten Entitäten}%
  \fi
  \setlength{\parindent}{0pt}%
  \setlength{\parskip}{0pt plus 0.3pt}%
  \let\item\@idxitem
}{%
  \ifkorrekturansicht\clearpage\fi
}
\makeatother

\IfFileExists{\jobname-pw.ind}{\input{\jobname-pw.ind}}{}

% Quellenangabe nur in der Leseansicht
\ifkorrekturansicht\else
% Fallback-Definitionen, falls die .tex-Datei \titel etc. nicht gesetzt hat
\providecommand{\titel}{}
\providecommand{\editorInnen}{}
\providecommand{\dateiname}{\jobname}

\vspace{3cm}

\vfill

\footnotesize
\textsc{Quelle}: \titel. Herausgegeben von {\editorInnen}. In: \emph{Arthur Schnitzler: Briefwechsel mit Autorinnen und Autoren}.
 Digitale Edition, https://schnitzler-briefe.acdh.oeaw.ac.at/{\dateiname}.html (Stand \today)
\fi

\end{document}


