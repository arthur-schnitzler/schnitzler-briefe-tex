%% latex-korrekturansicht-vorspann.tex
%% Vorspann für die Korrekturansicht.
%% Lädt die gemeinsame Datei latex-vorspann.tex mit gesetztem Schalter.

\newif\ifkorrekturansicht
\korrekturansichttrue

\input{../tex-inputs/latex-vorspann}


\section[Arthur Schnitzler: Widmungsexemplar Episode für Hugo von Hofmannsthal, {[}2. 2. 1892?{]}]{L00070 Arthur Schnitzler: Widmungsexemplar Episode für Hugo von Hofmannsthal,
               {[}2. 2. 1892?{]}}
\nopagebreak\mylabel{L00070v}
\rehead{ }\normalsize\beginnumbering\briefempfaengerindex{Hofmannsthal, Hugo von@\textsc{Hofmannsthal, Hugo von}!zzzSchnitzler, Arthur@\emph{von Arthur Schnitzler}!1892-02-021@{{[}2. 2. 1892?{]}}|(be}
\toendnotes[C]{\smallbreak\pagebreak[2]}\Standort{FDH, FDH 3223.}
\physDesc{, 50 Zeichen
\newline{}Handschrift: schwarze Tinte, deutsche Kurrent}
\buchAbdrucke{\weitereDrucke{Hugo von Hofmannsthal: \emph{Bibliothek}. Frankfurt am Main: \emph{S. Fischer} 2011, S. 603.} }\toendnotes[C]{\smallbreak}
\pstart
           \noindent{}{\pb}Meinem ſehr verehrten Freunde \textsc{Loris}\pend
           
\pstart
           herzlichſt{\\[\baselineskip]}\spacefill\mbox{Arth}\pend
           \leftskip=0em{}{\vspace{1\baselineskip}}
\pstart
           \centering{}\textcolor{gray}{\textbf{\textbf{Epiſode\pwindex{Episode@\emph{Episode}|pw}.}}}\pend
           
\pstart
           \centering{}\textcolor{gray}{\textbf{Von}}{\\}\textcolor{gray}{\textbf{Arthur Schnitzler.}}\pend
           {\vspace{1\baselineskip}}
\pstart
           \centering{}\textcolor{gray}{\textbf{Den Bühnen gegenüber als Manuſcript.}}\pend
           
\pstart
           \centering{}\textcolor{gray}{\textbf{Wien\oindex{Wien@\textbf{Wien}, \emph{A.ADM2}|pw}, \label{K_L00070-1v}\edtext{1889}{\lemma{\textnormal{\emph{1889}}}\Cendnote{\textnormal{Die Widmung ist nicht datiert. Der
                     Separatdruck aus Heft 18 der Zeitschrift \emph{An der
                        schönen blauen Donau}\pwindex{der schoenen blauen Donau@\emph{An der schönen blauen Donau}|pwk} erschien am 8. 9. 1889, als sich die beiden noch
                     nicht kannten. Da auch handschriftliche Änderungen enthalten sind, die in der
                     Druckfassung berücksichtigt wurden, ist eine Übergabe kurz vor Ablieferung des
                     Verlagsmanuskripts der Buchausgabe\pwindex{Anatol@\emph{Anatol}|pwkv} am 5. 2. 1892 ein wahrscheinlicher Termin. Und da bietet sich
                     wiederum der 2. 2. 1892 an, da an diesem Tag Hofmannsthal\pwindex{Hofmannsthal, Hugo von 1874-02-01 – 1929-07-15@\textsc{Hofmannsthal, Hugo von} (1874-02-01 – 1929-07-15), \emph{Schriftsteller/Schriftstellerin}|pwk} seinen \emph{Prolog}\pwindex{Prolog [zum Anatol]@\emph{Prolog [zum Anatol]}|pwk} verfasst hat.}}}\label{K_L00070-1}.}}\pend
           \selectlanguage{ngerman}\endnumbering\briefempfaengerindex{Hofmannsthal, Hugo von@\textsc{Hofmannsthal, Hugo von}!zzzSchnitzler, Arthur@\emph{von Arthur Schnitzler}!1892-02-021@{{[}2. 2. 1892?{]}}|)be}\mylabel{L00070h}  \normalsize

\doendnotes{C}
\bigskip
\vfill

\clearpage

\footnotesize

\lohead{\textsc{register}}

% Definiere theindex-Environment komplett neu ohne reledmac
\makeatletter
\renewenvironment{theindex}{%
  \section*{\indexname}%
  \setlength{\parindent}{0pt}%
  \setlength{\parskip}{0pt plus 0.3pt}%
  \let\item\@idxitem
}{%
  \clearpage
}
\makeatother

\IfFileExists{\jobname-pw.ind}{\input{\jobname-pw.ind}}{}

\end{document}

      