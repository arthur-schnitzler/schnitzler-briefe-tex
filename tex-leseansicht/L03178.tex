%% latex-leseansicht-vorspann.tex
%% Vorspann für die Leseansicht.
%% Lädt die gemeinsame Datei latex-vorspann.tex mit nicht gesetztem Schalter.

\newif\ifkorrekturansicht
\korrekturansichtfalse

\input{../tex-inputs/latex-vorspann}

\begin{center}
            \textcolor{red}{ENTWURF, NICHT FERTIG KORRIGIERT}
                      \end{center}
            
         
         \renewcommand{\erwaehntePersonen}{Personen: Richard Beer-Hofmann, Paula Beer-Hofmann, Paul Goldmann, Moriz Metzl, Friedrich Mitterwurzer, Wilhelmine Mitterwurzer, Ottilie Salten, Franziska Schlesinger, Emil Schlesinger}
         \renewcommand{\erwaehnteOrte}{Orte: Bad Aussee, Bad Ischl, Skodsborg, Wien}
         \renewcommand{\erwaehnteWerke}{Werke: ?? [Einakter], Anatol, Das Märchen. Schauspiel in drei Aufzügen, Herodes und Mariamne. Eine Tragödie in fünf Aufzügen, Judith. Eine Tragödie in fünf Aufzügen}
               \section[Felix Salten an Arthur Schnitzler, 8. 8. 1896]{ Felix Salten an Arthur Schnitzler, 8. 8. 1896}\nopagebreak\mylabel{v}\rehead{ }\begin{ledgroupsized}[t]{13cm}\normalsize\beginnumbering \toendnotes[C]{\smallbreak\pagebreak[2]} \Standort{CUL, Schnitzler, B 89, A 1.}
\physDesc{Brief, 1 Blatt, 1 Seite
\newline{}Handschrift: blaue Tinte, lateinische Kurrent\newline{}Ordnung: mit Bleistift von unbekannter Hand nummeriert: »77« }\toendnotes[C]{\smallbreak}\pstart
           \raggedleft{}{\pb}Ischl\oindex{Bad Ischl@\textbf{Bad Ischl}|pw}, 8. Aug. 96.\pend
           \pstart
           Lieber Arthur, die \label{K_L03178-112v}\edtext{Tischkarte}{\lemma{\textnormal{\emph{Tischkarte}}}\Cendnote{\textnormal{vgl. Felix Salten u. a. an Arthur Schnitzler, 6. 8. 1896}}}\label{K_L03178-112h}, welche Ihnen von Schlesingers\pwindex{Schlesinger, Franziska 17.08.1851 – 11.08.1932@\textsc{Schlesinger, Franziska} (17.08.1851 – 11.08.1932)|pw}\pwindex{Schlesinger, Emil 10.05.1844 – 31.05.1899@\textsc{Schlesinger, Emil} (10.05.1844 – 31.05.1899), \emph{Bankdirektor}|pw} aus zukam, kann auch als Dokument für die Langeweile gelten,
               mit der man hier seine Zeit hin bringt. Ich wohne mit den Mädeln auf einem Gang, was
               einige Annäherung unvermeidlich mit sich gebracht hat. Frl. M.\pwindex{Salten, Ottilie 07.03.1868 – 22.06.1942@\textsc{Salten, Ottilie} (07.03.1868 – 22.06.1942), \emph{Schauspielerin}|pw} und ich stehen gerade zu einander, wie in Wien\oindex{Wien@\textbf{Wien}|pw}. Die Radtour konnte noch nicht unternommen werden, weil ihr
               83jähriger Vater\pwindex{Metzl, Moriz 1814-09-16 – 1896-12-21@\textsc{Metzl, Moriz} (1814-09-16 – 1896-12-21)|pwv} krank ist,
               und außerdem noch, weil es beständig schüttet. \pend
           \pstart
           Neulich war ich bei \label{K_L03178-345v}\edtext{Mitterwurzer\pwindex{Mitterwurzer, Friedrich 16.10.1844 – 13.02.1897@\textsc{Mitterwurzer, Friedrich} (16.10.1844 – 13.02.1897), \emph{Schauspieler}|pw}\pwindex{Mitterwurzer, Wilhelmine 27.03.1848 – 03.08.1909@\textsc{Mitterwurzer, Wilhelmine} (27.03.1848 – 03.08.1909), \emph{Schauspielerin}|pw} zu
               Tisch in Aussee\oindex{Bad Aussee@\textbf{Bad Aussee}|pw}. Er\pwindex{Mitterwurzer, Friedrich 16.10.1844 – 13.02.1897@\textsc{Mitterwurzer, Friedrich} (16.10.1844 – 13.02.1897), \emph{Schauspieler}|pwv} war auch da, und fand Ihren Anatol\pwindex{Schnitzler, Arthur 15.05.1862 – 21.10.1931@\textsc{Schnitzler, Arthur} (15.05.1862 – 21.10.1931), \emph{Schriftsteller, Mediziner}!Anatol1892-10-29@\strich\emph{Anatol} {[}1892-10-29{]}|pw}, wie auch das Märchen\pwindex{Schnitzler, Arthur 15.05.1862 – 21.10.1931@\textsc{Schnitzler, Arthur} (15.05.1862 – 21.10.1931), \emph{Schriftsteller, Mediziner}!Maerchen. Schauspiel in drei Aufzuegen1893-12-01@\strich\emph{Das Märchen. Schauspiel in drei Aufzügen} {[}1893-12-01{]}|pw} »frivol«}{\lemma{\textnormal{\emph{Mitterwurzer … »frivol«}}}\Cendnote{\textnormal{vgl. A. S.: \emph{Tagebuch}, 5. 9. 1896}}}\label{K_L03178-345h}. Er studiert den Holofernes\pwindex{\textcolor{red}{\textsuperscript{XXXX1 indx}}!Judith. Eine Tragoedie in fuenf Aufzuegen1840-07-06@\strich\emph{Judith. Eine Tragödie in fünf Aufzügen} {[}1840-07-06{]}|pwv} und wird auf meine Veranlaßung
               auch den Herodes\pwindex{\textcolor{red}{\textsuperscript{XXXX1 indx}}!Herodes und Mariamne. Eine Tragoedie in fuenf Aufzuegen1848@\strich\emph{Herodes und Mariamne. Eine Tragödie in fünf Aufzügen} {[}1848{]}|pwv} ansehen. Mein
                  Stück\pwindex{Salten, Felix 06.09.1869 – 08.10.1945@\textsc{Salten, Felix} (06.09.1869 – 08.10.1945), \emph{Schriftsteller, Journalist}!?? [Einakter]None@\strich\emph{?? [Einakter]} {[}None{]}|pwv}, (den Einacter) hab
               ich ihm erzählt, und es gefiel ihm ganz besonders. Man braucht Einakter dieses Jahr
               und so hab’ ich vielleicht einige Chance, wenn ich nur damit zustande komme. Grüßen
               Sie Richard\pwindex{Beer-Hofmann, Richard 1866-07-11 – 1945-09-26@\textsc{Beer-Hofmann, Richard} (1866-07-11 – 1945-09-26), \emph{Schriftsteller}|pw} und Paula\pwindex{Beer-Hofmann, Paula 25.02.1879 – 30.10.1939@\textsc{Beer-Hofmann, Paula} (25.02.1879 – 30.10.1939)|pw}, und wenn er schon da ist D\textsuperscript{r.}{ }Goldmann\pwindex{Goldmann, Paul 31.01.1865 – 25.09.1935@\textsc{Goldmann, Paul} (31.01.1865 – 25.09.1935), \emph{Schriftsteller, Journalist}|pw}.\pend
           \pstart
           Herzlichst Ihr {\\[\baselineskip]}\spacefill\mbox{Salten}\pend
           \leftskip=0em{}
         
         \endnumbering\mylabel{h}\end{ledgroupsized}\begin{anhang}\end{anhang}\newcommand{\dateiname}{L03178}\newcommand{\titel}{Felix Salten an Arthur Schnitzler, 8. 8. 1896}\newcommand{\editorInnen}{Martin Anton Müller und Laura Untner}%% latex-leseansicht-abspann.tex
%% Abspann für die Leseansicht.
%% Der Schalter \ifkorrekturansicht ist bereits durch den Vorspann gesetzt.

%% latex-abspann.tex
%% Gemeinsamer Abspann für Korrekturansicht und Leseansicht.
%% Setzt den Schalter \ifkorrekturansicht voraus (gesetzt in den
%% einbindenden Dateien latex-korrekturansicht-abspann.tex bzw.
%% latex-leseansicht-abspann.tex).
%% ---------------------------------------------------------------

\normalsize

% Das esempio-Environment wird nur in der Leseansicht benötigt
\ifkorrekturansicht\else
\newenvironment{esempio}[3]%
{
    \vspace{1.5ex}
    \rlap{\underline{#1}}
    \par
    \setlength{\parindent}{0cm}
    \nopagebreak
    \leftskip=#2cm
    \rightskip=#3cm
}
{
    \par
}
\fi

\doendnotes{C}
\bigskip
\vfill

\clearpage

\footnotesize

\ifkorrekturansicht
  \lohead{\textsc{register}}
\fi

% theindex-Environment neu definieren ohne reledmac
\makeatletter
\renewenvironment{theindex}{%
  \ifkorrekturansicht
    \section*{\indexname}%
  \else
    \subsubsection*{Index der erwähnten Entitäten}%
  \fi
  \setlength{\parindent}{0pt}%
  \setlength{\parskip}{0pt plus 0.3pt}%
  \let\item\@idxitem
}{%
  \ifkorrekturansicht\clearpage\fi
}
\makeatother

\IfFileExists{\jobname-pw.ind}{\input{\jobname-pw.ind}}{}

% Quellenangabe nur in der Leseansicht
\ifkorrekturansicht\else
% Fallback-Definitionen, falls die .tex-Datei \titel etc. nicht gesetzt hat
\providecommand{\titel}{}
\providecommand{\editorInnen}{}
\providecommand{\dateiname}{\jobname}

\vspace{3cm}

\vfill

\footnotesize
\textsc{Quelle}: \titel. Herausgegeben von {\editorInnen}. In: \emph{Arthur Schnitzler: Briefwechsel mit Autorinnen und Autoren}.
 Digitale Edition, https://schnitzler-briefe.acdh.oeaw.ac.at/{\dateiname}.html (Stand \today)
\fi

\end{document}


      