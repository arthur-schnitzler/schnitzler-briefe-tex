%% latex-leseansicht-vorspann.tex
%% Vorspann für die Leseansicht.
%% Lädt die gemeinsame Datei latex-vorspann.tex mit nicht gesetztem Schalter.

\newif\ifkorrekturansicht
\korrekturansichtfalse

\input{../tex-inputs/latex-vorspann}


\section[ Felix Salten an Arthur Schnitzler, 8. 8. 1896]{L03178 Felix Salten an Arthur Schnitzler,  8. 8. 1896}
\nopagebreak\mylabel{L03178v}
\rehead{ }\normalsize\beginnumbering\briefempfaengerindex{Schnitzler, Arthur@\textsc{Schnitzler, Arthur}!zzzSalten, Felix@\emph{von Felix Salten}!1896-08-081@{8. 8. 1896}|(be}
\toendnotes[C]{\smallbreak\pagebreak[2]}
\correspDesc{Versand  durch Felix Salten am 8. 8. 1896 in Bad Ischl
\newline{}Erhalt  durch Arthur Schnitzler im Zeitraum [10. 8. 1896
                  – 14. 8. 1896?] in Skodsborg}\toendnotes[C]{\smallbreak}
\Standort{CUL, Schnitzler, B 89, A 1.}
\physDesc{Brief, 1 Blatt, 1 Seite, 927 Zeichen
\newline{}Handschrift: blaue Tinte, lateinische Kurrent
\newline{}Ordnung: mit Bleistift von unbekannter Hand nummeriert: »77« }\toendnotes[C]{\smallbreak}
\pstart
           \raggedleft{}{\pb}Ischl\oindex{Bad Ischl@\textbf{Bad Ischl}|pw}, 8. Aug. 96.\pend
           \vspace{0.5em}
\pstart
           Lieber Arthur, die \label{K_L03178-1v}\edtext{Tischkarte}{\lemma{\textnormal{\emph{Tischkarte}}}\Cendnote{\textnormal{XXXX Auszeichnungsfehler: Dokument L03177 nicht gefunden.
               }}}\label{K_L03178-1}, welche Ihnen von Schlesingers\pwindex{Schlesinger, Franziska 17.\,8.\,1851 Wien – 11.\,8.\,1932 ebd.@\textsc{Schlesinger, Franziska} (17.\,8.\,1851 Wien – 11.\,8.\,1932 ebd.)|pw}\pwindex{Schlesinger, Emil 10.\,5.\,1844 Wien – 31.\,5.\,1899 ebd.@\textsc{Schlesinger, Emil} (10.\,5.\,1844 Wien – 31.\,5.\,1899 ebd.), \emph{Bankdirektor}|pw} aus zukam, kann auch als Document für die Langeweile gelten,
               mit der man hier seine Zeit hinbringt. Ich wohne mit den Mädeln\pwindex{Berger, Else 20.\,10.\,1874 Wien – 24.\,11.\,1956 ebd.@\textsc{Berger, Else} (20.\,10.\,1874 Wien – 24.\,11.\,1956 ebd.)|pwuv}\pwindex{Schlesinger, Margherita *~25.\,12.\,1881 Wien@\textsc{Schlesinger, Margherita} (*~25.\,12.\,1881 Wien)|pwuv} auf einem
               Gang, was einige Annäherung unvermeidlich mit sich gebracht hat. Frl. M.\pwindex{Salten, Ottilie 7.\,3.\,1868 Prag – 22.\,6.\,1942 Zürich@\textsc{Salten, Ottilie} (7.\,3.\,1868 Prag – 22.\,6.\,1942 Zürich), \emph{Schauspielerin}|pw} und ich stehen geradeso zu einander, wie in
                  Wien\oindex{Wien@\textbf{Wien}, \emph{Verwaltungsgebiet}|pw}. Die \label{K_L03178-2v}\edtext{Radtour}{\lemma{\textnormal{\emph{Radtour}}}\Cendnote{\textnormal{Vgl. XXXX Auszeichnungsfehler: Dokument L03175 nicht gefunden.
               }}}\label{K_L03178-2} konnte noch nicht unternommen werden, weil ihr 83 jähriger \label{K_L03178-3v}\edtext{Vater\pwindex{Metzl, Moriz 16.\,9.\,1814 Prag – 21.\,12.\,1896 Wien@\textsc{Metzl, Moriz} (16.\,9.\,1814 Prag – 21.\,12.\,1896 Wien)|pwv} krank}{\lemma{\textnormal{\emph{Vater krank}}}\Cendnote{\textnormal{Moriz Metzl\pwindex{Metzl, Moriz 16.\,9.\,1814 Prag – 21.\,12.\,1896 Wien@\textsc{Metzl, Moriz} (16.\,9.\,1814 Prag – 21.\,12.\,1896 Wien)|pwk} verstarb noch im selben Jahr,
                  am 21. 12. 1896.}}}\label{K_L03178-3} ist, und außerdem noch,
               weil es beständig schüttet.\pend
           
\pstart
           Neulich war ich bei \label{K_L03178-4v}\edtext{Mitterwurzer\pwindex{Mitterwurzer, Friedrich 16.\,10.\,1844 Dresden – 13.\,2.\,1897 Wien@\textsc{Mitterwurzer, Friedrich} (16.\,10.\,1844 Dresden – 13.\,2.\,1897 Wien), \emph{Schauspieler}|pw}\pwindex{Mitterwurzer, Wilhelmine 27.\,3.\,1848 Freiburg im Breisgau – 3.\,8.\,1909 Wien@\textsc{Mitterwurzer, Wilhelmine} (27.\,3.\,1848 Freiburg im Breisgau – 3.\,8.\,1909 Wien), \emph{Schauspielerin}|pw} zu Tisch in Aussee\oindex{Bad Aussee@\textbf{Bad Aussee}, \emph{Hauptstadt}|pw}. Er\pwindex{Mitterwurzer, Friedrich 16.\,10.\,1844 Dresden – 13.\,2.\,1897 Wien@\textsc{Mitterwurzer, Friedrich} (16.\,10.\,1844 Dresden – 13.\,2.\,1897 Wien), \emph{Schauspieler}|pwv} war auch da, und fand Ihren Anatol\pwindex{Schnitzler, Arthur 15.\,5.\,1862 Wien – 21.\,10.\,1931 ebd.@\textsc{Schnitzler, Arthur} (15.\,5.\,1862 Wien – 21.\,10.\,1931 ebd.), \emph{Schriftsteller, Mediziner}!Anatol@\strich\emph{Anatol}|pw}, wie auch das Märchen\pwindex{Schnitzler, Arthur 15.\,5.\,1862 Wien – 21.\,10.\,1931 ebd.@\textsc{Schnitzler, Arthur} (15.\,5.\,1862 Wien – 21.\,10.\,1931 ebd.), \emph{Schriftsteller, Mediziner}!Märchen. Schauspiel in drei Aufzügen@\strich\emph{Das Märchen. Schauspiel in drei Aufzügen}|pw} »frivol«}{\lemma{\textnormal{\emph{Mitterwurzer … »frivol«}}}\Cendnote{\textnormal{Siehe A. S.: \emph{Tagebuch}, 5. 9. 1896.
               }}}\label{K_L03178-4}. Er studirt den Holofernes\pwindex{\textcolor{red}{\textsuperscript{XXXX indx1}}!Judith. Eine Tragödie in fünf Aufzügen@\strich\emph{Judith. Eine Tragödie in fünf Aufzügen}|pwv} und wird auf meine Veranlaßung auch den Herodes\pwindex{\textcolor{red}{\textsuperscript{XXXX indx1}}!Herodes und Mariamne. Eine Tragödie in fünf Aufzügen@\strich\emph{Herodes und Mariamne. Eine Tragödie in fünf Aufzügen}|pwv} ansehen. Mein \label{K_L03178-5v}\edtext{Stück\pwindex{Salten, Felix 6.\,9.\,1869 Budapest – 8.\,10.\,1945 Zürich@\textsc{Salten, Felix} (6.\,9.\,1869 Budapest – 8.\,10.\,1945 Zürich), \emph{Schriftsteller, Journalist, Chefredakteur}!Engagement@\strich\emph{Ein Engagement}|pwuv}}{\lemma{\textnormal{\emph{Stück}}}\Cendnote{\textnormal{Es könnte sich um das kurze Stück \emph{Ein Engagement}\pwindex{Salten, Felix 6.\,9.\,1869 Budapest – 8.\,10.\,1945 Zürich@\textsc{Salten, Felix} (6.\,9.\,1869 Budapest – 8.\,10.\,1945 Zürich), \emph{Schriftsteller, Journalist, Chefredakteur}!Engagement@\strich\emph{Ein Engagement}|pwk} handeln, das Salten\pwindex{Salten, Felix 6.\,9.\,1869 Budapest – 8.\,10.\,1945 Zürich@\textsc{Salten, Felix} (6.\,9.\,1869 Budapest – 8.\,10.\,1945 Zürich), \emph{Schriftsteller, Journalist, Chefredakteur}|pwk} drei Jahre später, am 11. 12. 1899, in der \emph{Wiener Allgemeinen Montags-Zeitung}\pwindex{Salten, Felix 6.\,9.\,1869 Budapest – 8.\,10.\,1945 Zürich@\textsc{Salten, Felix} (6.\,9.\,1869 Budapest – 8.\,10.\,1945 Zürich), \emph{Schriftsteller, Journalist, Chefredakteur}!Engagement@\strich\emph{Ein Engagement}|pwk} (S. 5–6) veröffentlichte.}}}\label{K_L03178-5} (den Einacter) hab ich ihm erzählt, und es gefiel ihm ganz
               besonders. Man braucht Einacter dieses Jahr und so hab’ ich vielleicht einige Chance,
               wenn ich nur damit zustande komme. Grüßen Sie Richard\pwindex{Beer-Hofmann, Richard 11.\,7.\,1866 Wien – 26.\,9.\,1945 New York City@\textsc{Beer-Hofmann, Richard} (11.\,7.\,1866 Wien – 26.\,9.\,1945 New York City), \emph{Schriftsteller}|pw} und Paula\pwindex{Beer-Hofmann, Paula 25.\,2.\,1879 Wien – 30.\,10.\,1939 Zürich@\textsc{Beer-Hofmann, Paula} (25.\,2.\,1879 Wien – 30.\,10.\,1939 Zürich)|pw}, und – \label{K_L03178-6v}\edtext{wenn er schon da ist}{\lemma{\textnormal{\emph{wenn er schon da ist}}}\Cendnote{\textnormal{Paul Goldmann\pwindex{Goldmann, Paul 31.\,1.\,1865 Breslau – 25.\,9.\,1935 Wien@\textsc{Goldmann, Paul} (31.\,1.\,1865 Breslau – 25.\,9.\,1935 Wien), \emph{Schriftsteller, Journalist}|pwk} kam am 5. 8. 1896 in Kopenhagen\oindex{Kopenhagen@\textbf{Kopenhagen}, \emph{Hauptstadt}|pwk} an und war seither mit den anderen in Skodsborg\oindex{Skodsborg@\textbf{Skodsborg}|pwk}.}}}\label{K_L03178-6} – D\textsuperscript{r}{ }Goldmann\pwindex{Goldmann, Paul 31.\,1.\,1865 Breslau – 25.\,9.\,1935 Wien@\textsc{Goldmann, Paul} (31.\,1.\,1865 Breslau – 25.\,9.\,1935 Wien), \emph{Schriftsteller, Journalist}|pw}.\pend
           
\pstart
           Herzlichst Ihr {\\[\baselineskip]}\spacefill\mbox{Salten}\pend
           \leftskip=0em{}\selectlanguage{ngerman}\endnumbering\briefempfaengerindex{Schnitzler, Arthur@\textsc{Schnitzler, Arthur}!zzzSalten, Felix@\emph{von Felix Salten}!1896-08-081@{8. 8. 1896}|)be}\mylabel{L03178h}  \newcommand{\dateiname}{L03178}\newcommand{\titel}{Felix Salten an Arthur Schnitzler, 8. 8. 1896}\newcommand{\editorInnen}{Martin Anton Müller und Laura Untner}%% latex-leseansicht-abspann.tex
%% Abspann für die Leseansicht.
%% Der Schalter \ifkorrekturansicht ist bereits durch den Vorspann gesetzt.

%% latex-abspann.tex
%% Gemeinsamer Abspann für Korrekturansicht und Leseansicht.
%% Setzt den Schalter \ifkorrekturansicht voraus (gesetzt in den
%% einbindenden Dateien latex-korrekturansicht-abspann.tex bzw.
%% latex-leseansicht-abspann.tex).
%% ---------------------------------------------------------------

\normalsize

% Das esempio-Environment wird nur in der Leseansicht benötigt
\ifkorrekturansicht\else
\newenvironment{esempio}[3]%
{
    \vspace{1.5ex}
    \rlap{\underline{#1}}
    \par
    \setlength{\parindent}{0cm}
    \nopagebreak
    \leftskip=#2cm
    \rightskip=#3cm
}
{
    \par
}
\fi

\doendnotes{C}
\bigskip
\vfill

\clearpage

\footnotesize

\ifkorrekturansicht
  \lohead{\textsc{register}}
\fi

% theindex-Environment neu definieren ohne reledmac
\makeatletter
\renewenvironment{theindex}{%
  \ifkorrekturansicht
    \section*{\indexname}%
  \else
    \subsubsection*{Index der erwähnten Entitäten}%
  \fi
  \setlength{\parindent}{0pt}%
  \setlength{\parskip}{0pt plus 0.3pt}%
  \let\item\@idxitem
}{%
  \ifkorrekturansicht\clearpage\fi
}
\makeatother

\IfFileExists{\jobname-pw.ind}{\input{\jobname-pw.ind}}{}

% Quellenangabe nur in der Leseansicht
\ifkorrekturansicht\else
% Fallback-Definitionen, falls die .tex-Datei \titel etc. nicht gesetzt hat
\providecommand{\titel}{}
\providecommand{\editorInnen}{}
\providecommand{\dateiname}{\jobname}

\vspace{3cm}

\vfill

\footnotesize
\textsc{Quelle}: \titel. Herausgegeben von {\editorInnen}. In: \emph{Arthur Schnitzler: Briefwechsel mit Autorinnen und Autoren}.
 Digitale Edition, https://schnitzler-briefe.acdh.oeaw.ac.at/{\dateiname}.html (Stand \today)
\fi

\end{document}


