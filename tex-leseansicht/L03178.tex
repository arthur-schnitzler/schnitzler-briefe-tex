%% latex-leseansicht-vorspann.tex
%% Vorspann für die Leseansicht.
%% Lädt die gemeinsame Datei latex-vorspann.tex mit nicht gesetztem Schalter.

\newif\ifkorrekturansicht
\korrekturansichtfalse

\input{../tex-inputs/latex-vorspann}

\begin{center}
            \textcolor{red}{ENTWURF, NICHT FERTIG KORRIGIERT}
                      \end{center}
            
         
         \renewcommand{\erwaehntePersonen}{Personen: Richard Beer-Hofmann, Paula Beer-Hofmann, Else Berger, Paul Goldmann, Moriz Metzl, Friedrich Mitterwurzer, Wilhelmine Mitterwurzer, Felix Salten, Ottilie Salten, Franziska Schlesinger, Emil Schlesinger, Margherita Schlesinger}
         \renewcommand{\erwaehnteOrte}{Orte: Bad Aussee, Bad Ischl, Kopenhagen, Skodsborg, Wien}
         \renewcommand{\erwaehnteWerke}{Werke: Anatol, Das Märchen. Schauspiel in drei Aufzügen, Ein Engagement, Herodes und Mariamne. Eine Tragödie in fünf Aufzügen, Judith. Eine Tragödie in fünf Aufzügen}
               \section[ Felix Salten an Arthur Schnitzler, 8. 8. 1896]{ Felix Salten an Arthur Schnitzler, 8. 8. 1896}\nopagebreak\mylabel{v}\rehead{ }\begin{ledgroupsized}[t]{13cm}\normalsize\beginnumbering\briefempfaengerindex{Schnitzler, Arthur@\textsc{Schnitzler, Arthur}!zzzSalten, Felix@\emph{von Felix Salten}!1896-08-081@{8. 8. 1896}|(be} \toendnotes[C]{\smallbreak\pagebreak[2]} \Standort{CUL, Schnitzler, B 89, A 1.}
\physDesc{Brief, 1 Blatt, 1 Seite, 927 Zeichen
\newline{}Handschrift: blaue Tinte, lateinische Kurrent
\newline{}Ordnung: mit Bleistift von unbekannter Hand nummeriert: »77« }\toendnotes[C]{\smallbreak}\pstart
           \raggedleft{}{\pb}Ischl\oindex{Bad Ischl@\textbf{Bad Ischl}|pw}, 8. Aug. 96.\pend
           \pstart
           Lieber Arthur, die \label{K_L03178-1v}\edtext{Tischkarte}{\lemma{\textnormal{\emph{Tischkarte}}}\Cendnote{\textnormal{Felix Salten u. a. an Arthur Schnitzler, 6. 8. 1896}}}\label{K_L03178-1h}, welche Ihnen von Schlesinger\pwindex{Schlesinger, Franziska 17.08.1851 – 11.08.1932@\textsc{Schlesinger, Franziska} (17.08.1851 – 11.08.1932)|pw}\pwindex{Schlesinger, Emil 10.05.1844 – 31.05.1899@\textsc{Schlesinger, Emil} (10.05.1844 – 31.05.1899), \emph{Bankdirektor}|pw}s aus zukam, kann auch als Document für die Langeweile gelten,
               mit der man hier seine Zeit hinbringt. Ich wohne mit den Mädeln\pwindex{Berger, Else 20.10.1874 – 24.11.1956@\textsc{Berger, Else} (20.10.1874 – 24.11.1956)|pwuv}\pwindex{Schlesinger, Margherita *~1881-12-25@\textsc{Schlesinger, Margherita} (*~1881-12-25)|pwuv} auf einem
               Gang, was einige Annäherung unvermeidlich mit sich gebracht hat. Frl. M.\pwindex{Salten, Ottilie 07.03.1868 – 22.06.1942@\textsc{Salten, Ottilie} (07.03.1868 – 22.06.1942), \emph{Schauspielerin}|pw} und ich stehen geradeso zu einander, wie in
                  Wien\oindex{Wien@\textbf{Wien}|pw}. Die \label{K_L03178-2v}\edtext{Radtour}{\lemma{\textnormal{\emph{Radtour}}}\Cendnote{\textnormal{vgl. Felix Salten an Arthur Schnitzler, 21. 7. 1896}}}\label{K_L03178-2h} konnte noch nicht unternommen werden, weil ihr 83 jähriger \label{K_L03178-3v}\edtext{Vater\pwindex{Metzl, Moriz 1814-09-16 – 1896-12-21@\textsc{Metzl, Moriz} (1814-09-16 – 1896-12-21)|pwv} krank}{\lemma{\textnormal{\emph{Vater krank}}}\Cendnote{\textnormal{Moriz Metzl\pwindex{Metzl, Moriz 1814-09-16 – 1896-12-21@\textsc{Metzl, Moriz} (1814-09-16 – 1896-12-21)|pwk} verstarb noch im selben Jahr,
                  am 21. 12. 1896.}}}\label{K_L03178-3h} ist, und außerdem noch,
               weil es beständig schüttet.\pend
           \pstart
           Neulich war ich bei \label{K_L03178-4v}\edtext{Mitterwurzer\pwindex{Mitterwurzer, Friedrich 16.10.1844 – 13.02.1897@\textsc{Mitterwurzer, Friedrich} (16.10.1844 – 13.02.1897), \emph{Schauspieler}|pw}\pwindex{Mitterwurzer, Wilhelmine 27.03.1848 – 03.08.1909@\textsc{Mitterwurzer, Wilhelmine} (27.03.1848 – 03.08.1909), \emph{Schauspielerin}|pw} zu Tisch in Aussee\oindex{Bad Aussee@\textbf{Bad Aussee}|pw}. Er\pwindex{Mitterwurzer, Friedrich 16.10.1844 – 13.02.1897@\textsc{Mitterwurzer, Friedrich} (16.10.1844 – 13.02.1897), \emph{Schauspieler}|pwv} war auch da, und fand Ihren Anatol\pwindex{Schnitzler, Arthur 15.05.1862 – 21.10.1931@\textsc{Schnitzler, Arthur} (15.05.1862 – 21.10.1931), \emph{Schriftsteller, Mediziner}!Anatol1892-10-29@\strich\emph{Anatol} {[}1892-10-29{]}|pw}, wie auch das Märchen\pwindex{Schnitzler, Arthur 15.05.1862 – 21.10.1931@\textsc{Schnitzler, Arthur} (15.05.1862 – 21.10.1931), \emph{Schriftsteller, Mediziner}!Maerchen. Schauspiel in drei Aufzuegen1893-12-01@\strich\emph{Das Märchen. Schauspiel in drei Aufzügen} {[}1893-12-01{]}|pw} »frivol«}{\lemma{\textnormal{\emph{Mitterwurzer … »frivol«}}}\Cendnote{\textnormal{siehe A. S.: \emph{Tagebuch}, 5. 9. 1896}}}\label{K_L03178-4h}. Er studirt den Holofernes\pwindex{\textcolor{red}{\textsuperscript{XXXX1 indx}}!Judith. Eine Tragoedie in fuenf Aufzuegen1840-07-06@\strich\emph{Judith. Eine Tragödie in fünf Aufzügen} {[}1840-07-06{]}|pwv} und wird auf meine Veranlaßung auch den Herodes\pwindex{\textcolor{red}{\textsuperscript{XXXX1 indx}}!Herodes und Mariamne. Eine Tragoedie in fuenf Aufzuegen1848@\strich\emph{Herodes und Mariamne. Eine Tragödie in fünf Aufzügen} {[}1848{]}|pwv} ansehen. Mein \label{K_L03178-5v}\edtext{Stück\pwindex{Salten, Felix 06.09.1869 – 08.10.1945@\textsc{Salten, Felix} (06.09.1869 – 08.10.1945), \emph{Schriftsteller, Journalist}!Engagement1899-12-11@\strich\emph{Ein Engagement} {[}1899-12-11{]}|pwuv}}{\lemma{\textnormal{\emph{Stück}}}\Cendnote{\textnormal{Es könnte sich um das kurze Stück \emph{Ein Engagement}\pwindex{Salten, Felix 06.09.1869 – 08.10.1945@\textsc{Salten, Felix} (06.09.1869 – 08.10.1945), \emph{Schriftsteller, Journalist}!Engagement1899-12-11@\strich\emph{Ein Engagement} {[}1899-12-11{]}|pwk} handeln, das Salten\pwindex{Salten, Felix 06.09.1869 – 08.10.1945@\textsc{Salten, Felix} (06.09.1869 – 08.10.1945), \emph{Schriftsteller, Journalist}|pwk} drei Jahre später, am 11. 12. 1899, in der \emph{Wiener Allgemeinen Montags-Zeitung}\pwindex{Salten, Felix 06.09.1869 – 08.10.1945@\textsc{Salten, Felix} (06.09.1869 – 08.10.1945), \emph{Schriftsteller, Journalist}!Engagement1899-12-11@\strich\emph{Ein Engagement} {[}1899-12-11{]}|pwk} (S. 5–6) veröffentlichte.}}}\label{K_L03178-5h} (den Einacter) hab ich ihm erzählt, und es gefiel ihm ganz
               besonders. Man braucht Einacter dieses Jahr und so hab’ ich vielleicht einige Chance,
               wenn ich nur damit zustande komme. Grüßen Sie Richard\pwindex{Beer-Hofmann, Richard 1866-07-11 – 1945-09-26@\textsc{Beer-Hofmann, Richard} (1866-07-11 – 1945-09-26), \emph{Schriftsteller}|pw} und Paula\pwindex{Beer-Hofmann, Paula 25.02.1879 – 30.10.1939@\textsc{Beer-Hofmann, Paula} (25.02.1879 – 30.10.1939)|pw}, und – \label{K_L03178-6v}\edtext{wenn er schon da ist}{\lemma{\textnormal{\emph{wenn er schon da ist}}}\Cendnote{\textnormal{Paul Goldmann\pwindex{Goldmann, Paul 31.01.1865 – 25.09.1935@\textsc{Goldmann, Paul} (31.01.1865 – 25.09.1935), \emph{Schriftsteller, Journalist}|pwk} kam am 5. 8. 1896 in Kopenhagen\oindex{Kopenhagen@\textbf{Kopenhagen}|pwk} an und war seither mit den anderen in Skodsborg\oindex{Skodsborg@\textbf{Skodsborg}|pwk}.}}}\label{K_L03178-6h} – D\textsuperscript{r}{ }Goldmann\pwindex{Goldmann, Paul 31.01.1865 – 25.09.1935@\textsc{Goldmann, Paul} (31.01.1865 – 25.09.1935), \emph{Schriftsteller, Journalist}|pw}.\pend
           \pstart
           Herzlichst Ihr {\\[\baselineskip]}\spacefill\mbox{Salten}\pend
           \leftskip=0em{}
         
         \endnumbering\mylabel{h}\end{ledgroupsized}  \newcommand{\dateiname}{L03178}\newcommand{\titel}{Felix Salten an Arthur Schnitzler, 8. 8. 1896}\newcommand{\editorInnen}{Martin Anton Müller und Laura Untner}%% latex-leseansicht-abspann.tex
%% Abspann für die Leseansicht.
%% Der Schalter \ifkorrekturansicht ist bereits durch den Vorspann gesetzt.

%% latex-abspann.tex
%% Gemeinsamer Abspann für Korrekturansicht und Leseansicht.
%% Setzt den Schalter \ifkorrekturansicht voraus (gesetzt in den
%% einbindenden Dateien latex-korrekturansicht-abspann.tex bzw.
%% latex-leseansicht-abspann.tex).
%% ---------------------------------------------------------------

\normalsize

% Das esempio-Environment wird nur in der Leseansicht benötigt
\ifkorrekturansicht\else
\newenvironment{esempio}[3]%
{
    \vspace{1.5ex}
    \rlap{\underline{#1}}
    \par
    \setlength{\parindent}{0cm}
    \nopagebreak
    \leftskip=#2cm
    \rightskip=#3cm
}
{
    \par
}
\fi

\doendnotes{C}
\bigskip
\vfill

\clearpage

\footnotesize

\ifkorrekturansicht
  \lohead{\textsc{register}}
\fi

% theindex-Environment neu definieren ohne reledmac
\makeatletter
\renewenvironment{theindex}{%
  \ifkorrekturansicht
    \section*{\indexname}%
  \else
    \subsubsection*{Index der erwähnten Entitäten}%
  \fi
  \setlength{\parindent}{0pt}%
  \setlength{\parskip}{0pt plus 0.3pt}%
  \let\item\@idxitem
}{%
  \ifkorrekturansicht\clearpage\fi
}
\makeatother

\IfFileExists{\jobname-pw.ind}{\input{\jobname-pw.ind}}{}

% Quellenangabe nur in der Leseansicht
\ifkorrekturansicht\else
% Fallback-Definitionen, falls die .tex-Datei \titel etc. nicht gesetzt hat
\providecommand{\titel}{}
\providecommand{\editorInnen}{}
\providecommand{\dateiname}{\jobname}

\vspace{3cm}

\vfill

\footnotesize
\textsc{Quelle}: \titel. Herausgegeben von {\editorInnen}. In: \emph{Arthur Schnitzler: Briefwechsel mit Autorinnen und Autoren}.
 Digitale Edition, https://schnitzler-briefe.acdh.oeaw.ac.at/{\dateiname}.html (Stand \today)
\fi

\end{document}


      