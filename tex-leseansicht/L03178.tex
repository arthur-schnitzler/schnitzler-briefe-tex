%% latex-korrekturansicht-vorspann.tex
%% Vorspann für die Korrekturansicht.
%% Lädt die gemeinsame Datei latex-vorspann.tex mit gesetztem Schalter.

\newif\ifkorrekturansicht
\korrekturansichttrue

\input{../tex-inputs/latex-vorspann}


\section[ Felix Salten an Arthur Schnitzler, 8. 8. 1896]{L03178 Felix Salten an Arthur Schnitzler, 8. 8. 1896}
\nopagebreak\mylabel{L03178v}
\rehead{ }\normalsize\beginnumbering\briefempfaengerindex{Schnitzler, Arthur@\textsc{Schnitzler, Arthur}!zzzSalten, Felix@\emph{von Felix Salten}!1896-08-081@{8. 8. 1896}|(be}
\toendnotes[C]{\smallbreak\pagebreak[2]}\Standort{CUL, Schnitzler, B 89, A 1.}
\physDesc{Brief, 1 Blatt, 1 Seite, 927 Zeichen
\newline{}Handschrift: blaue Tinte, lateinische Kurrent
\newline{}Ordnung: mit Bleistift von unbekannter Hand nummeriert: »77« }\toendnotes[C]{\smallbreak}
\pstart
           \raggedleft{}{\pb}Ischl\oindex{Bad Ischl@\textbf{Bad Ischl}, \emph{P.PPL}|pw}, 8. Aug. 96.\pend
           \vspace{0.5em}
\pstart
           Lieber Arthur, die \label{K_L03178-1v}\edtext{Tischkarte}{\lemma{\textnormal{\emph{Tischkarte}}}\Cendnote{\textnormal{Felix Salten u. a. an Arthur Schnitzler, 6. 8. 1896.
               }}}\label{K_L03178-1}, welche Ihnen von Schlesingers\pwindex{Schlesinger, Franziska 17.08.1851 – 11.08.1932@\textsc{Schlesinger, Franziska} (17.08.1851 – 11.08.1932)|pw}\pwindex{Schlesinger, Emil 10.05.1844 – 31.05.1899@\textsc{Schlesinger, Emil} (10.05.1844 – 31.05.1899), \emph{Bankdirektor/Bankdirektorin}|pw} aus zukam, kann auch als Document für die Langeweile gelten,
               mit der man hier seine Zeit hinbringt. Ich wohne mit den Mädeln\pwindex{Berger, Else 20.10.1874 – 24.11.1956@\textsc{Berger, Else} (20.10.1874 – 24.11.1956)|pwuv}\pwindex{Schlesinger, Margherita *~1881-12-25@\textsc{Schlesinger, Margherita} (*~1881-12-25)|pwuv} auf einem
               Gang, was einige Annäherung unvermeidlich mit sich gebracht hat. Frl. M.\pwindex{Salten, Ottilie 07.03.1868 – 22.06.1942@\textsc{Salten, Ottilie} (07.03.1868 – 22.06.1942), \emph{Schauspieler/Schauspielerin}|pw} und ich stehen geradeso zu einander, wie in
                  Wien\oindex{Wien@\textbf{Wien}, \emph{A.ADM2}|pw}. Die \label{K_L03178-2v}\edtext{Radtour}{\lemma{\textnormal{\emph{Radtour}}}\Cendnote{\textnormal{Vgl. Felix Salten an Arthur Schnitzler, 21. 7. 1896.
               }}}\label{K_L03178-2} konnte noch nicht unternommen werden, weil ihr 83 jähriger \label{K_L03178-3v}\edtext{Vater\pwindex{Metzl, Moriz 1814-09-16 – 1896-12-21@\textsc{Metzl, Moriz} (1814-09-16 – 1896-12-21)|pwv} krank}{\lemma{\textnormal{\emph{Vater krank}}}\Cendnote{\textnormal{Moriz Metzl\pwindex{Metzl, Moriz 1814-09-16 – 1896-12-21@\textsc{Metzl, Moriz} (1814-09-16 – 1896-12-21)|pwk} verstarb noch im selben Jahr,
                  am 21. 12. 1896.}}}\label{K_L03178-3} ist, und außerdem noch,
               weil es beständig schüttet.\pend
           
\pstart
           Neulich war ich bei \label{K_L03178-4v}\edtext{Mitterwurzer\pwindex{Mitterwurzer, Friedrich 16.10.1844 – 13.02.1897@\textsc{Mitterwurzer, Friedrich} (16.10.1844 – 13.02.1897), \emph{Schauspieler/Schauspielerin}|pw}\pwindex{Mitterwurzer, Wilhelmine 27.03.1848 – 03.08.1909@\textsc{Mitterwurzer, Wilhelmine} (27.03.1848 – 03.08.1909), \emph{Schauspieler/Schauspielerin}|pw} zu Tisch in Aussee\oindex{Bad Aussee@\textbf{Bad Aussee}, \emph{P.PPLA3}|pw}. Er\pwindex{Mitterwurzer, Friedrich 16.10.1844 – 13.02.1897@\textsc{Mitterwurzer, Friedrich} (16.10.1844 – 13.02.1897), \emph{Schauspieler/Schauspielerin}|pwv} war auch da, und fand Ihren Anatol\pwindex{Anatol@\emph{Anatol}|pw}, wie auch das Märchen\pwindex{Maerchen. Schauspiel in drei Aufzuegen@\emph{Das Märchen. Schauspiel in drei Aufzügen}|pw} »frivol«}{\lemma{\textnormal{\emph{Mitterwurzer … »frivol«}}}\Cendnote{\textnormal{Siehe A. S.: \emph{Tagebuch}, 5. 9. 1896.
               }}}\label{K_L03178-4}. Er studirt den Holofernes\pwindex{Judith. Eine Tragoedie in fuenf Aufzuegen@\emph{Judith. Eine Tragödie in fünf Aufzügen}|pwv} und wird auf meine Veranlaßung auch den Herodes\pwindex{Herodes und Mariamne. Eine Tragoedie in fuenf Aufzuegen@\emph{Herodes und Mariamne. Eine Tragödie in fünf Aufzügen}|pwv} ansehen. Mein \label{K_L03178-5v}\edtext{Stück\pwindex{Engagement@\emph{Ein Engagement}|pwuv}}{\lemma{\textnormal{\emph{Stück}}}\Cendnote{\textnormal{Es könnte sich um das kurze Stück \emph{Ein Engagement}\pwindex{Engagement@\emph{Ein Engagement}|pwk} handeln, das Salten\pwindex{Salten, Felix 06.09.1869 – 08.10.1945@\textsc{Salten, Felix} (06.09.1869 – 08.10.1945), \emph{Schriftsteller/Schriftstellerin, Journalist/Journalistin, Chefredakteur/Chefredakteurin}|pwk} drei Jahre später, am 11. 12. 1899, in der \emph{Wiener Allgemeinen Montags-Zeitung}\pwindex{Engagement@\emph{Ein Engagement}|pwk} (S. 5–6) veröffentlichte.}}}\label{K_L03178-5} (den Einacter) hab ich ihm erzählt, und es gefiel ihm ganz
               besonders. Man braucht Einacter dieses Jahr und so hab’ ich vielleicht einige Chance,
               wenn ich nur damit zustande komme. Grüßen Sie Richard\pwindex{Beer-Hofmann, Richard 1866-07-11 – 1945-09-26@\textsc{Beer-Hofmann, Richard} (1866-07-11 – 1945-09-26), \emph{Schriftsteller/Schriftstellerin}|pw} und Paula\pwindex{Beer-Hofmann, Paula 25.02.1879 – 30.10.1939@\textsc{Beer-Hofmann, Paula} (25.02.1879 – 30.10.1939)|pw}, und – \label{K_L03178-6v}\edtext{wenn er schon da ist}{\lemma{\textnormal{\emph{wenn er schon da ist}}}\Cendnote{\textnormal{Paul Goldmann\pwindex{Goldmann, Paul 31.01.1865 – 25.09.1935@\textsc{Goldmann, Paul} (31.01.1865 – 25.09.1935), \emph{Schriftsteller/Schriftstellerin, Journalist/Journalistin}|pwk} kam am 5. 8. 1896 in Kopenhagen\oindex{Kopenhagen@\textbf{Kopenhagen}, \emph{P.PPLC}|pwk} an und war seither mit den anderen in Skodsborg\oindex{Skodsborg@\textbf{Skodsborg}, \emph{P.PPL}|pwk}.}}}\label{K_L03178-6} – D\textsuperscript{r}{ }Goldmann\pwindex{Goldmann, Paul 31.01.1865 – 25.09.1935@\textsc{Goldmann, Paul} (31.01.1865 – 25.09.1935), \emph{Schriftsteller/Schriftstellerin, Journalist/Journalistin}|pw}.\pend
           
\pstart
           Herzlichst Ihr {\\[\baselineskip]}\spacefill\mbox{Salten}\pend
           \leftskip=0em{}\selectlanguage{ngerman}\endnumbering\briefempfaengerindex{Schnitzler, Arthur@\textsc{Schnitzler, Arthur}!zzzSalten, Felix@\emph{von Felix Salten}!1896-08-081@{8. 8. 1896}|)be}\mylabel{L03178h}  \normalsize

\doendnotes{C}
\bigskip
\vfill

\clearpage

\footnotesize

\lohead{\textsc{register}}

% Definiere theindex-Environment komplett neu ohne reledmac
\makeatletter
\renewenvironment{theindex}{%
  \section*{\indexname}%
  \setlength{\parindent}{0pt}%
  \setlength{\parskip}{0pt plus 0.3pt}%
  \let\item\@idxitem
}{%
  \clearpage
}
\makeatother

\IfFileExists{\jobname-pw.ind}{\input{\jobname-pw.ind}}{}

\end{document}

      