%% latex-leseansicht-vorspann.tex
%% Vorspann für die Leseansicht.
%% Lädt die gemeinsame Datei latex-vorspann.tex mit nicht gesetztem Schalter.

\newif\ifkorrekturansicht
\korrekturansichtfalse

\input{../tex-inputs/latex-vorspann}


         
         \newcommand{\erwaehntePersonen}{Personen: }
         \newcommand{\erwaehnteInstitutionen}{}
         \newcommand{\erwaehnteOrte}{}
         \newcommand{\erwaehnteWerke}{
               \section[Hermann Bahr an Arthur Schnitzler, 10. 11. 1897]{ Hermann Bahr an Arthur Schnitzler, 10. 11. 1897}\nopagebreak\mylabel{v}\rehead{ }\begin{ledgroupsized}[t]{13cm}\normalsize\beginnumbering \toendnotes[C]{\smallbreak\pagebreak[2]} \Standort{CUL, Schnitzler, B 5b.}
\physDesc{Brief, 1 Blatt, 2 Seiten
\newline{}Handschrift: schwarze Tinte, deutsche Kurrent\newline{}Ordnung: mit Bleistift von unbekannter Hand nummeriert: »55« }\buchAbdrucke{\weitereDrucke{Hermann Bahr, Arthur Schnitzler: \emph{Briefwechsel, Aufzeichnungen, Dokumente (1891–1931)}. Hg. Kurt Ifkovits und Martin Anton Müller. Göttingen: \emph{Wallstein} 2018, S. 154–155.} }\toendnotes[C]{\smallbreak}\pstart
           \noindent{}{\pb}\textcolor{gray}{\textbf{»Die
                        ZeitXXXX ORGangabe fehlt«}}\hfill \textcolor{gray}{\textbf{\textbf{Wien\oindex{XXXX Ortsangabe fehlt|pw}}, den }}10. November \textcolor{gray}{\textbf{189}}7\pend
           \pstart
           \textcolor{gray}{\textbf{Wiener Wochenſchrift}}\hfill \textcolor{gray}{\textbf{IX/3, Günthergaſſe 1\oindex{XXXX Ortsangabe fehlt|pw}.}}\pend
           \pstart
           \textcolor{gray}{\textbf{\textbf{Herausgeber}:}}{\\}\textcolor{gray}{\textbf{Profeſſor Dr. I. Singer\pwindex{\textcolor{red}{\textsuperscript{XXXX1 indx}}|pw}, Hermann Bahr\pwindex{\textcolor{red}{\textsuperscript{XXXX1 indx}}|pw},
                        Dr. Heinrich Kanner\pwindex{\textcolor{red}{\textsuperscript{XXXX1 indx}}|pw}.}}\pend
           \pstart
           \textcolor{gray}{\textbf{Telephon Nr. 6415.}}\pend
           \pstart\center{}Lieber Arthur!\pend\pstart
           Möchteſt Du mir erlauben, bei meiner nächſten Conference (am 28. d. M.)
               Deine \label{K_L00737_1v}\edtext{Geſchichte\textcolor{red}{\textsuperscript{XXXX indx}}}{\lemma{\textnormal{\emph{Geſchichte}}}\Cendnote{\textnormal{Arthur Schnitzler\pwindex{\textcolor{red}{\textsuperscript{XXXX1 indx}}|pwk}: \emph{Die Toten schweigen}\textcolor{red}{\textsuperscript{XXXX indx}}. In: \emph{Cosmopolis}\textcolor{red}{\textsuperscript{XXXX indx}}, Jg. 2, Bd. 8, Nr. 22,
                        1. 10. 1897, S. 193–211.}}}\label{K_L00737_1h} aus dem
               letzten Heft der »\textsc{Cosmopolis}\textcolor{red}{\textsuperscript{XXXX indx}}« vorzuleſen? Ich bilde mir ein, daß ich den Ton treffen
               werde, und irre ich mich darin nicht, ſo iſt die Wirkung glaub ich ſicher. Alſo, wenn
               es Dir {\pb}recht iſt, ſo ſchreib oder telephonir mir
               bitte ein Wort.\pend
           \pstart
           Und vergiß doch nicht ganz auf die »ZeitXXXX ORGangabe fehlt«. Haſt Du nicht wieder was Kleines? Ich würde das neue Jahr ſehr gern
               wieder mit etwas von Dir beginnen. Laß mich wiſſen, ob ich darauf rechnen kann.\pend
           \pstart
           Herzlichſt{\\[\baselineskip]}Dein alter{\\[\baselineskip]}\spacefill\mbox{HermannBahr}\pend
           \leftskip=0em{}\pstart
           \textcolor{gray}{\textbf{\label{T_L00737_1v}\edtext{Alle für »Die ZeitXXXX ORGangabe fehlt« beſtimmten Zuſchriften und Sendungen ſind an
                  die Redaction der »ZeitXXXX ORGangabe fehlt« und \textbf{nicht} an die Perſon eines der Herausgeber zu richten.}{\lemma{\textnormal{\emph{Alle … richten.}}}\Cendnote{\textnormal{am unteren Rand der ersten Seite}}}\label{T_L00737_1h}}}\pend
           
         
         \endnumbering\mylabel{h}\end{ledgroupsized}  \newcommand{\dateiname}{L00737}\newcommand{\titel}{Hermann Bahr an Arthur Schnitzler, 10. 11. 1897}\newcommand{\editorInnen}{ Kurt Ifkovits,  Martin Anton Müller}%% latex-leseansicht-abspann.tex
%% Abspann für die Leseansicht.
%% Der Schalter \ifkorrekturansicht ist bereits durch den Vorspann gesetzt.

%% latex-abspann.tex
%% Gemeinsamer Abspann für Korrekturansicht und Leseansicht.
%% Setzt den Schalter \ifkorrekturansicht voraus (gesetzt in den
%% einbindenden Dateien latex-korrekturansicht-abspann.tex bzw.
%% latex-leseansicht-abspann.tex).
%% ---------------------------------------------------------------

\normalsize

% Das esempio-Environment wird nur in der Leseansicht benötigt
\ifkorrekturansicht\else
\newenvironment{esempio}[3]%
{
    \vspace{1.5ex}
    \rlap{\underline{#1}}
    \par
    \setlength{\parindent}{0cm}
    \nopagebreak
    \leftskip=#2cm
    \rightskip=#3cm
}
{
    \par
}
\fi

\doendnotes{C}
\bigskip
\vfill

\clearpage

\footnotesize

\ifkorrekturansicht
  \lohead{\textsc{register}}
\fi

% theindex-Environment neu definieren ohne reledmac
\makeatletter
\renewenvironment{theindex}{%
  \ifkorrekturansicht
    \section*{\indexname}%
  \else
    \subsubsection*{Index der erwähnten Entitäten}%
  \fi
  \setlength{\parindent}{0pt}%
  \setlength{\parskip}{0pt plus 0.3pt}%
  \let\item\@idxitem
}{%
  \ifkorrekturansicht\clearpage\fi
}
\makeatother

\IfFileExists{\jobname-pw.ind}{\input{\jobname-pw.ind}}{}

% Quellenangabe nur in der Leseansicht
\ifkorrekturansicht\else
% Fallback-Definitionen, falls die .tex-Datei \titel etc. nicht gesetzt hat
\providecommand{\titel}{}
\providecommand{\editorInnen}{}
\providecommand{\dateiname}{\jobname}

\vspace{3cm}

\vfill

\footnotesize
\textsc{Quelle}: \titel. Herausgegeben von {\editorInnen}. In: \emph{Arthur Schnitzler: Briefwechsel mit Autorinnen und Autoren}.
 Digitale Edition, https://schnitzler-briefe.acdh.oeaw.ac.at/{\dateiname}.html (Stand \today)
\fi

\end{document}


      