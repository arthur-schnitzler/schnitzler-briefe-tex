%% latex-leseansicht-vorspann.tex
%% Vorspann für die Leseansicht.
%% Lädt die gemeinsame Datei latex-vorspann.tex mit nicht gesetztem Schalter.

\newif\ifkorrekturansicht
\korrekturansichtfalse

\input{../tex-inputs/latex-vorspann}


         
         \renewcommand{\erwaehntePersonen}{Personen: Ottilie Salten}
         \renewcommand{\erwaehnteOrte}{Orte: Pension Kirsch, Wien}
         \renewcommand{\erwaehnteWerke}{}
               \section[ Ottilie Salten an Arthur Schnitzler, {[}24. 5. 1902?{]}]{ Ottilie Salten an Arthur Schnitzler, {[}24. 5. 1902?{]}}\nopagebreak\mylabel{v}\rehead{ }\begin{ledgroupsized}[t]{13cm}\normalsize\beginnumbering\briefempfaengerindex{Schnitzler, Arthur@\textsc{Schnitzler, Arthur}!zzzSalten, Ottilie@\emph{von Ottilie Salten}!1902-05-241@{{[}24. 5. 1902{]}}|(be} \toendnotes[C]{\smallbreak\pagebreak[2]} \Standort{CUL, Schnitzler, B 89, A 1.}
\physDesc{Brief, 1 Blatt, 1 Seite, 301 Zeichen
\newline{}Handschrift: schwarze Tinte, deutsche Kurrent
\newline{}Schnitzler: mit Bleistift datiert »24/5 90\textcolor{gray}{×}« 
\newline{}Ordnung: mit Bleistift von unbekannter Hand nummeriert: »1« }\toendnotes[C]{\smallbreak}\pstart
           \noindent{}{\pb}\textsc{Sehr geehrter Herr Doctor}, ich danke Ihnen herzlich für die große \label{K_L03100-1v}\edtext{Liebenswürdigkeit}{\lemma{\textnormal{\emph{Liebenswürdigkeit}}}\Cendnote{\textnormal{Um welche Art von Geschenk es sich handelt, ließ sich nicht ermitteln.}}}\label{K_L03100-1h}. Ich habe \textsc{Felix} ſofort geſchrieben
               und \substVorne{}\textsuperscript{\textcolor{gray}{d}}\substDazwischen{}I\substHinten{}hr freun{[}d{]}liches Schreiben beigeſchloſſen.\pend
           \pstart
           Er hatte ſehr ſchlechtes Wetter. Jetzt iſt er in \label{K_L03100-2v}\edtext{Florenz \textsc{Casa Kirsch}{ }\textsc{\begin{otherlanguage}{italian}Lungarno\end{otherlanguage}.}\oindex{Pension Kirsch@\textbf{Pension Kirsch}|pw}}{\lemma{\textnormal{\emph{Florenz … Lungarno.}}}\Cendnote{\textnormal{Dadurch ist der Brief, trotz Schnitzler\pwindex{Schnitzler, Arthur 15.05.1862 – 21.10.1931@\textsc{Schnitzler, Arthur} (15.05.1862 – 21.10.1931), \emph{Schriftsteller, Mediziner}|pwk}s unleserlicher Jahresangabe, auf
                  das Jahr 1902 datierbar, vgl. Felix Salten an Arthur Schnitzler, 2[3]. 5. 1902.}}}\label{K_L03100-2h}\pend
           \pstart
           Nochmals herzlichen Dank und Gruß. {\\[\baselineskip]}Ihre ergebene {\\[\baselineskip]}\spacefill\mbox{Ottilie S.}\pend
           \leftskip=0em{}\pstart
           \textsc{Samstag}\pend
           
         
         \endnumbering\mylabel{h}\end{ledgroupsized}  \newcommand{\dateiname}{L03100}\newcommand{\titel}{Ottilie Salten an Arthur Schnitzler, [24. 5. 1902?]}\newcommand{\editorInnen}{Martin Anton Müller und Laura Untner}%% latex-leseansicht-abspann.tex
%% Abspann für die Leseansicht.
%% Der Schalter \ifkorrekturansicht ist bereits durch den Vorspann gesetzt.

%% latex-abspann.tex
%% Gemeinsamer Abspann für Korrekturansicht und Leseansicht.
%% Setzt den Schalter \ifkorrekturansicht voraus (gesetzt in den
%% einbindenden Dateien latex-korrekturansicht-abspann.tex bzw.
%% latex-leseansicht-abspann.tex).
%% ---------------------------------------------------------------

\normalsize

% Das esempio-Environment wird nur in der Leseansicht benötigt
\ifkorrekturansicht\else
\newenvironment{esempio}[3]%
{
    \vspace{1.5ex}
    \rlap{\underline{#1}}
    \par
    \setlength{\parindent}{0cm}
    \nopagebreak
    \leftskip=#2cm
    \rightskip=#3cm
}
{
    \par
}
\fi

\doendnotes{C}
\bigskip
\vfill

\clearpage

\footnotesize

\ifkorrekturansicht
  \lohead{\textsc{register}}
\fi

% theindex-Environment neu definieren ohne reledmac
\makeatletter
\renewenvironment{theindex}{%
  \ifkorrekturansicht
    \section*{\indexname}%
  \else
    \subsubsection*{Index der erwähnten Entitäten}%
  \fi
  \setlength{\parindent}{0pt}%
  \setlength{\parskip}{0pt plus 0.3pt}%
  \let\item\@idxitem
}{%
  \ifkorrekturansicht\clearpage\fi
}
\makeatother

\IfFileExists{\jobname-pw.ind}{\input{\jobname-pw.ind}}{}

% Quellenangabe nur in der Leseansicht
\ifkorrekturansicht\else
% Fallback-Definitionen, falls die .tex-Datei \titel etc. nicht gesetzt hat
\providecommand{\titel}{}
\providecommand{\editorInnen}{}
\providecommand{\dateiname}{\jobname}

\vspace{3cm}

\vfill

\footnotesize
\textsc{Quelle}: \titel. Herausgegeben von {\editorInnen}. In: \emph{Arthur Schnitzler: Briefwechsel mit Autorinnen und Autoren}.
 Digitale Edition, https://schnitzler-briefe.acdh.oeaw.ac.at/{\dateiname}.html (Stand \today)
\fi

\end{document}


      