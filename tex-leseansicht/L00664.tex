%% latex-korrekturansicht-vorspann.tex
%% Vorspann für die Korrekturansicht.
%% Lädt die gemeinsame Datei latex-vorspann.tex mit gesetztem Schalter.

\newif\ifkorrekturansicht
\korrekturansichttrue

\input{../tex-inputs/latex-vorspann}


\section[Hugo von Hofmannsthal an Arthur Schnitzler, {[}6. 4. 1897{]}]{L00664 Hugo von Hofmannsthal an Arthur Schnitzler, {[}6. 4. 1897{]}}
\nopagebreak\mylabel{L00664v}
\rehead{ }\normalsize\beginnumbering\briefempfaengerindex{Schnitzler, Arthur@\textsc{Schnitzler, Arthur}!zzzHofmannsthal, Hugo von@\emph{von Hugo von Hofmannsthal}!1897-04-061@{{[}6. 4. 1897{]}}|(be}
\toendnotes[C]{\smallbreak\pagebreak[2]}\Standort{CUL, Schnitzler, B 43.}
\physDesc{Brief, 1 Blatt, 1 Seite, 217 Zeichen
\newline{}Handschrift: Bleistift, deutsche Kurrent
\newline{}Schnitzler: mit Bleistift datiert: »6/4 97« 
\newline{}Ordnung: von unbekannter Hand nummeriert: »87a« }
\buchAbdrucke{\weitereDrucke{Hugo von Hofmannsthal, Arthur Schnitzler: \emph{Briefwechsel}. Frankfurt am Main: \emph{S. Fischer} 1964, S. 79.} }
\pstart
           \raggedleft{}{\pb}½ 11\textsuperscript{h} Früh\pend
           
\pstart{}lieber Arthur\pend\vspace{0.5em}
\pstart
           ich bin geko{\geminationm}en um Sie noch einmal ein biſſel zu ſehen,
               hab Sie aber leider verſäumt.\pend
           
\pstart
           Soll ich morgen Früh ko{\geminationm}en? Ein \textsc{rendez vous} im Café möcht ich nicht, das iſt ſo zuwider.\pend
           
\pstart
           Herzlich Ihr Freund{\\[\baselineskip]}\spacefill\mbox{Hugo.}\pend
           \leftskip=0em{}\selectlanguage{ngerman}\endnumbering\briefempfaengerindex{Schnitzler, Arthur@\textsc{Schnitzler, Arthur}!zzzHofmannsthal, Hugo von@\emph{von Hugo von Hofmannsthal}!1897-04-061@{{[}6. 4. 1897{]}}|)be}\mylabel{L00664h}  \normalsize

\doendnotes{C}
\bigskip
\vfill

\clearpage

\footnotesize

\lohead{\textsc{register}}

% Definiere theindex-Environment komplett neu ohne reledmac
\makeatletter
\renewenvironment{theindex}{%
  \section*{\indexname}%
  \setlength{\parindent}{0pt}%
  \setlength{\parskip}{0pt plus 0.3pt}%
  \let\item\@idxitem
}{%
  \clearpage
}
\makeatother

\IfFileExists{\jobname-pw.ind}{\input{\jobname-pw.ind}}{}

\end{document}

      