%% latex-korrekturansicht-vorspann.tex
%% Vorspann für die Korrekturansicht.
%% Lädt die gemeinsame Datei latex-vorspann.tex mit gesetztem Schalter.

\newif\ifkorrekturansicht
\korrekturansichttrue

\input{../tex-inputs/latex-vorspann}


\section[Arthur Schnitzler an Richard Beer-Hofmann, 6. 9. 1904]{L01437 Arthur Schnitzler an Richard Beer-Hofmann, 6. 9. 1904}
\nopagebreak\mylabel{L01437v}
\rehead{ }\normalsize\beginnumbering\briefempfaengerindex{Beer-Hofmann, Richard@\textsc{Beer-Hofmann, Richard}!zzzSchnitzler, Arthur@\emph{von Arthur Schnitzler}!1904-09-061@{6. 9. 1904}|(be}
\toendnotes[C]{\smallbreak\pagebreak[2]}\Standort{YCGL, MSS 31.}
\physDesc{Telegramm, 271 Zeichen
\newline{}Handschrift einer Schreibkraft: Bleistift, lateinische Kurrent
\newline{}Versand: »\noindent{}\textcolor{gray}{\textbf{Eingangs-Nr.}} 22b{ / }\textcolor{gray}{\textbf{Aufgegeben am {\dots} 190{\dots} um}}{ }9 \textcolor{gray}{\textbf{Uhr}} 30 \textcolor{gray}{\textbf{Min.}}{ }\textcolor{gray}{\textbf{{\dots}Mi{[}ttag{]}}}{ / }\textcolor{gray}{\textbf{Eingelangt von}} I. \textcolor{gray}{\textbf{auf Leitung Nr.}}{ }15876 \textcolor{gray}{\textbf{am {\dots} 190{\dots} um}}{ }9 \textcolor{gray}{\textbf{Uhr}} 30 \textcolor{gray}{\textbf{Min.}}{ }\textcolor{gray}{vor}{ }\textcolor{gray}{\textbf{Mittag}}{ / }\textcolor{gray}{\textbf{Von}}{ }\textsc{St Gilgen}\oindex{St. Gilgen@\textbf{St. Gilgen}, \emph{A.ADM3}|pw}{ }\textcolor{gray}{\textbf{Aufgabe-Nr.}} 148{ }\textcolor{gray}{\textbf{mit}}{ }{\dots}{ }\textcolor{gray}{\textbf{Taxworten (}}\substVorne{}\textsuperscript{3}\substDazwischen{}4\substHinten{}1 \textcolor{gray}{\textbf{Worten {\dots} Chiffern)}}« 
\newline{}Ordnung: mit Bleistift von unbekannter Hand datiert: »{[}6. 9. 1904{]}« }\pstart{}{\pb}Richard Beerhofmann\pend{}\pstart{}Villa Frühling\oindex{Villa Fruehling@\textbf{Villa Frühling}, \emph{Gebäude (K.GBD)}|pw}\pend{}\pstart{}Markt \textcolor{gray}{\textbf{\textit{AUSSEE IN ST.}}}\oindex{Bad Aussee@\textbf{Bad Aussee}, \emph{P.PPLA3}|pw}\pend{}{\bigskip}\vspace{1em}
\pstart
           \noindent{}{\pb}Hier ists wunderschön wir bleiben daher vorläufig und
               fahren nicht Auſſee\oindex{Bad Aussee@\textbf{Bad Aussee}, \emph{P.PPLA3}|pw} bitte ko{\geminationm}en Sie doch auf einige Tage her jedenfalls aber morgen
               da Hugos\pwindex{Hofmannsthal, Hugo von 1874-02-01 – 1929-07-15@\textsc{Hofmannsthal, Hugo von} (1874-02-01 – 1929-07-15), \emph{Schriftsteller/Schriftstellerin}|pw}\pwindex{Hofmannsthal, Gertrude von 16.03.1880 – 09.11.1959@\textsc{Hofmannsthal, Gertrude von} (16.03.1880 – 09.11.1959)|pw} morgen abends abreisen und
               Sie noch ſehen möchten =\pend
           
\pstart
           Herzliche Grüße{\\[\baselineskip]}\spacefill\mbox{= Arthur}\pend
           \leftskip=0em{}\selectlanguage{ngerman}\endnumbering\briefempfaengerindex{Beer-Hofmann, Richard@\textsc{Beer-Hofmann, Richard}!zzzSchnitzler, Arthur@\emph{von Arthur Schnitzler}!1904-09-061@{6. 9. 1904}|)be}\mylabel{L01437h}  \normalsize

\doendnotes{C}
\bigskip
\vfill

\clearpage

\footnotesize

\lohead{\textsc{register}}

% Definiere theindex-Environment komplett neu ohne reledmac
\makeatletter
\renewenvironment{theindex}{%
  \section*{\indexname}%
  \setlength{\parindent}{0pt}%
  \setlength{\parskip}{0pt plus 0.3pt}%
  \let\item\@idxitem
}{%
  \clearpage
}
\makeatother

\IfFileExists{\jobname-pw.ind}{\input{\jobname-pw.ind}}{}

\end{document}

      