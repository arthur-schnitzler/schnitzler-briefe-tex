%% latex-leseansicht-vorspann.tex
%% Vorspann für die Leseansicht.
%% Lädt die gemeinsame Datei latex-vorspann.tex mit nicht gesetztem Schalter.

\newif\ifkorrekturansicht
\korrekturansichtfalse

\input{../tex-inputs/latex-vorspann}


\section[Arthur Schnitzler an Richard Beer-Hofmann, 6. 9. 1904]{L01437 Arthur Schnitzler an Richard Beer-Hofmann, 6. 9. 1904}
\nopagebreak\mylabel{L01437v}
\rehead{ }\normalsize\beginnumbering\briefempfaengerindex{Beer-Hofmann, Richard@\textsc{Beer-Hofmann, Richard}!zzzSchnitzler, Arthur@\emph{von Arthur Schnitzler}!1904-09-061@{6. 9. 1904}|(be}
\toendnotes[C]{\smallbreak\pagebreak[2]}
\correspDesc{Versand  durch Arthur Schnitzler am 6. 9. 1904 in St. Gilgen
\newline{}Erhalt  durch Richard Beer-Hofmann am 6. 9. 1904 in Bad Aussee}\toendnotes[C]{\smallbreak}
\Standort{YCGL, MSS 31.}
\physDesc{Telegramm, 271 Zeichen
\newline{}HandschriftX2 einer Schreibkraft: Bleistift, lateinische Kurrent
\newline{}Versand: »\noindent{}\textcolor{gray}{\textbf{Eingangs-Nr.}} 22b{ / }\textcolor{gray}{\textbf{Aufgegeben am {\dots} 190{\dots} um}}{ }9 \textcolor{gray}{\textbf{Uhr}} 30 \textcolor{gray}{\textbf{Min.}}{ }\textcolor{gray}{\textbf{{\dots}Mi{[}ttag{]}}}{ / }\textcolor{gray}{\textbf{Eingelangt von}} I. \textcolor{gray}{\textbf{auf Leitung Nr.}}{ }15876 \textcolor{gray}{\textbf{am {\dots} 190{\dots} um}}{ }9 \textcolor{gray}{\textbf{Uhr}} 30 \textcolor{gray}{\textbf{Min.}}{ }\textcolor{gray}{vor}{ }\textcolor{gray}{\textbf{Mittag}}{ / }\textcolor{gray}{\textbf{Von}}{ }\textsc{St Gilgen}\oindex{St. Gilgen@\textbf{St. Gilgen}, \emph{Verwaltungsgebiet}|pw}{ }\textcolor{gray}{\textbf{Aufgabe-Nr.}} 148{ }\textcolor{gray}{\textbf{mit}}{ }{\dots}{ }\textcolor{gray}{\textbf{Taxworten (}}\substVorne{}\textsuperscript{3}\substDazwischen{}4\substHinten{}1 \textcolor{gray}{\textbf{Worten {\dots} Chiffern)}}« 
\newline{}Ordnung: mit Bleistift von unbekannter Hand datiert: »{[}6. 9. 1904{]}« }\pstart{}{\pb}Richard Beerhofmann\pend{}\pstart{}Villa Frühling\oindex{Villa Frühling@\textbf{Villa Frühling}, \emph{Gebäude}|pw}\pend{}\pstart{}Markt \textcolor{gray}{\textbf{\textit{AUSSEE IN ST.}}}\oindex{Bad Aussee@\textbf{Bad Aussee}, \emph{Hauptstadt}|pw}\pend{}{\bigskip}\vspace{1em}
\pstart
           \noindent{}{\pb}Hier ists wunderschön wir bleiben daher vorläufig und
               fahren nicht Auſſee\oindex{Bad Aussee@\textbf{Bad Aussee}, \emph{Hauptstadt}|pw} bitte ko{\geminationm}en Sie doch auf einige Tage her jedenfalls aber morgen
               da Hugos\pwindex{Hofmannsthal, Hugo von 1.\,2.\,1874 Wien – 15.\,7.\,1929 Rodaun@\textsc{Hofmannsthal, Hugo von} (1.\,2.\,1874 Wien – 15.\,7.\,1929 Rodaun), \emph{Schriftsteller}|pw}\pwindex{Hofmannsthal, Gertrude von 16.\,3.\,1880 Wien – 9.\,11.\,1959 Paddington@\textsc{Hofmannsthal, Gertrude von} (16.\,3.\,1880 Wien – 9.\,11.\,1959 Paddington)|pw} morgen abends abreisen und
               Sie noch{ }ſehen möchten =\pend
           
\pstart
           Herzliche Grüße{\\[\baselineskip]}\spacefill\mbox{= Arthur}\pend
           \leftskip=0em{}\selectlanguage{ngerman}\endnumbering\briefempfaengerindex{Beer-Hofmann, Richard@\textsc{Beer-Hofmann, Richard}!zzzSchnitzler, Arthur@\emph{von Arthur Schnitzler}!1904-09-061@{6. 9. 1904}|)be}\mylabel{L01437h}  \newcommand{\dateiname}{L01437}\newcommand{\titel}{Arthur Schnitzler an Richard Beer-Hofmann, 6. 9. 1904}\newcommand{\editorInnen}{Martin Anton Müller und Gerd-Hermann Susen}%% latex-leseansicht-abspann.tex
%% Abspann für die Leseansicht.
%% Der Schalter \ifkorrekturansicht ist bereits durch den Vorspann gesetzt.

%% latex-abspann.tex
%% Gemeinsamer Abspann für Korrekturansicht und Leseansicht.
%% Setzt den Schalter \ifkorrekturansicht voraus (gesetzt in den
%% einbindenden Dateien latex-korrekturansicht-abspann.tex bzw.
%% latex-leseansicht-abspann.tex).
%% ---------------------------------------------------------------

\normalsize

% Das esempio-Environment wird nur in der Leseansicht benötigt
\ifkorrekturansicht\else
\newenvironment{esempio}[3]%
{
    \vspace{1.5ex}
    \rlap{\underline{#1}}
    \par
    \setlength{\parindent}{0cm}
    \nopagebreak
    \leftskip=#2cm
    \rightskip=#3cm
}
{
    \par
}
\fi

\doendnotes{C}
\bigskip
\vfill

\clearpage

\footnotesize

\ifkorrekturansicht
  \lohead{\textsc{register}}
\fi

% theindex-Environment neu definieren ohne reledmac
\makeatletter
\renewenvironment{theindex}{%
  \ifkorrekturansicht
    \section*{\indexname}%
  \else
    \subsubsection*{Index der erwähnten Entitäten}%
  \fi
  \setlength{\parindent}{0pt}%
  \setlength{\parskip}{0pt plus 0.3pt}%
  \let\item\@idxitem
}{%
  \ifkorrekturansicht\clearpage\fi
}
\makeatother

\IfFileExists{\jobname-pw.ind}{\input{\jobname-pw.ind}}{}

% Quellenangabe nur in der Leseansicht
\ifkorrekturansicht\else
% Fallback-Definitionen, falls die .tex-Datei \titel etc. nicht gesetzt hat
\providecommand{\titel}{}
\providecommand{\editorInnen}{}
\providecommand{\dateiname}{\jobname}

\vspace{3cm}

\vfill

\footnotesize
\textsc{Quelle}: \titel. Herausgegeben von {\editorInnen}. In: \emph{Arthur Schnitzler: Briefwechsel mit Autorinnen und Autoren}.
 Digitale Edition, https://schnitzler-briefe.acdh.oeaw.ac.at/{\dateiname}.html (Stand \today)
\fi

\end{document}


