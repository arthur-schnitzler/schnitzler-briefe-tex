%% latex-leseansicht-vorspann.tex
%% Vorspann für die Leseansicht.
%% Lädt die gemeinsame Datei latex-vorspann.tex mit nicht gesetztem Schalter.

\newif\ifkorrekturansicht
\korrekturansichtfalse

\input{../tex-inputs/latex-vorspann}


\section[Arthur Schnitzler an Hugo von Hofmannsthal, 29. 7. 1892]{L00109 Arthur Schnitzler an Hugo von Hofmannsthal, 29. 7. 1892}
\nopagebreak\mylabel{L00109v}
\rehead{ }\normalsize\beginnumbering\briefempfaengerindex{Hofmannsthal, Hugo von@\textsc{Hofmannsthal, Hugo von}!zzzSchnitzler, Arthur@\emph{von Arthur Schnitzler}!1892-07-291@{29. 7. 1892}|(be}
\toendnotes[C]{\smallbreak\pagebreak[2]}
\correspDesc{Versand  durch Arthur Schnitzler am 29. 7. 1892 in Wien
\newline{}Erhalt  durch Hugo von Hofmannsthal im Zeitraum [29. 7. 1892
                  – 2. 8. 1892?] in Wien}\toendnotes[C]{\smallbreak}
\Standort{FDH, Hs-30885,22.}
\physDesc{Brief, 1 Blatt, 4 Seiten, 1260 Zeichen
\newline{}Handschrift: schwarze Tinte, deutsche Kurrent}
\buchAbdrucke{\weitereDrucke{Hugo von Hofmannsthal, Arthur Schnitzler: \emph{Briefwechsel}. Herausgegeben von Therese Nickl und Heinrich Schnitzler. Frankfurt am Main: \emph{S. Fischer} 1964, S. 25.} }\toendnotes[C]{\smallbreak}
\pstart
           
\pstart
           {\pb}Wien\oindex{Wien@\textbf{Wien}, \emph{Verwaltungsgebiet}|pw}\pend
           
\pstart
           \raggedleft{}29/7 92\pend
           \pend
           
\pstart{}Lieber Freund,\pend\vspace{0.5em}
\pstart
           nachdem Sie Ihr Gedicht\pwindex{Hofmannsthal, Hugo von 1.\,2.\,1874 Wien – 15.\,7.\,1929 Rodaun@\textsc{Hofmannsthal, Hugo von} (1.\,2.\,1874 Wien – 15.\,7.\,1929 Rodaun), \emph{Schriftsteller}!Prolog [zum Anatol]@\strich\emph{Prolog [zum Anatol]}|pwv} nicht
               im Inhalt haben wollen, möchte ich auch jeden Titel weglaſſen, und es nur im{ }ſelben
               Druck wie alles übrige \introOben{}haben\introOben{}, jedoch mit oben weit
               freigelaſſenen Rändern \strikeout{haben}. – Einverſtanden? –\pend
           
\pstart
           Vorgeſtern habe ich meine Novelle\pwindex{Schnitzler, Arthur 15.\,5.\,1862 Wien – 21.\,10.\,1931 ebd.@\textsc{Schnitzler, Arthur} (15.\,5.\,1862 Wien – 21.\,10.\,1931 ebd.), \emph{Schriftsteller, Mediziner}!Sterben. Novelle@\strich\emph{Sterben. Novelle}|pwv} beendet. – Ich hoffe, {\pb}ſie wird, we{\geminationn}{ }ſie erſt durchgefeilt iſt, als ehrenwerte Studie
               gelten können. Ich habe{ }ſie plötzlich zu Ende{ }ſchreiben müſſen, Nachts im Cafè,
               während{ }ſchläfrige Kellner bereits die Seſſel aufeinander thürmten. Ich habe{ }ſie{ }ſehr
               lieb gehabt – ich fühle mich ordentlich einſam,{ }ſeit ich nicht mehr drüber denken
               muſs. {\pb}(Siehe Freund \textsc{Y}\pwindex{Schnitzler, Arthur 15.\,5.\,1862 Wien – 21.\,10.\,1931 ebd.@\textsc{Schnitzler, Arthur} (15.\,5.\,1862 Wien – 21.\,10.\,1931 ebd.), \emph{Schriftsteller, Mediziner}!Mein Freund Ypsilon. Aus den Papieren eines Arztes@\strich\emph{Mein Freund Ypsilon. Aus den Papieren eines Arztes}|pw}). – Nun will ich wieder ans Stück\pwindex{Schnitzler, Arthur 15.\,5.\,1862 Wien – 21.\,10.\,1931 ebd.@\textsc{Schnitzler, Arthur} (15.\,5.\,1862 Wien – 21.\,10.\,1931 ebd.), \emph{Schriftsteller, Mediziner}!Liebelei. Schauspiel in drei Akten@\strich\emph{Liebelei. Schauspiel in drei Akten}|pwv}. – Eben hab ich Blumenthal\pwindex{Blumenthal, Oskar 13.\,3.\,1852 Berlin – 24.\,4.\,1917 ebd.@\textsc{Blumenthal, Oskar} (13.\,3.\,1852 Berlin – 24.\,4.\,1917 ebd.), \emph{Schriftsteller, Journalist, Theaterleiter}|pw} u
                  Reicher\pwindex{Reicher, Emanuel 18.\,6.\,1849 Bochnia – 15.\,5.\,1924 Berlin@\textsc{Reicher, Emanuel} (18.\,6.\,1849 Bochnia – 15.\,5.\,1924 Berlin), \emph{Schauspieler}|pw} geſchrieben! – wie verdreht
               eigentlich die Welt iſt! –\pend
           
\pstart
           Was macht Ihr Stück\pwindex{Hofmannsthal, Hugo von 1.\,2.\,1874 Wien – 15.\,7.\,1929 Rodaun@\textsc{Hofmannsthal, Hugo von} (1.\,2.\,1874 Wien – 15.\,7.\,1929 Rodaun), \emph{Schriftsteller}!Ascanio und Gioconda@\strich\emph{Ascanio und Gioconda}|pwv}? – Ich
               wundre mich, daſs Sie zugleich zweiten und fünften Akt{ }ſchreiben können. So{ }ſicher
               bin ich meiner Geſtalten nie! Es kann ihnen doch im dritten Akt {\pb}was einfallen oder gar paſſiren, wovon ich im zweiten noch
               nichts rechtes weiſs. Selbſt we{\geminationn} eine genaue Skizze
               vorliegt, wage ich es nicht und habe gewiſs keine Luſt dazu! Ich will mit ihnen
               weiter leben, und erleben, Gedanke für Gedanke und That für That, wie{ }ſie{ }ſelber. Ich
               darf manches vorausahnen, aber wiſſen darf ichs nicht.\pend
           \pstart Herzlichſt Ihr \spacefill\mbox{Arthur}\pend{}\selectlanguage{ngerman}\endnumbering\briefempfaengerindex{Hofmannsthal, Hugo von@\textsc{Hofmannsthal, Hugo von}!zzzSchnitzler, Arthur@\emph{von Arthur Schnitzler}!1892-07-291@{29. 7. 1892}|)be}\mylabel{L00109h}  \newcommand{\dateiname}{L00109}\newcommand{\titel}{Arthur Schnitzler an Hugo von Hofmannsthal, 29. 7. 1892}\newcommand{\editorInnen}{Martin Anton Müller und Gerd-Hermann Susen}%% latex-leseansicht-abspann.tex
%% Abspann für die Leseansicht.
%% Der Schalter \ifkorrekturansicht ist bereits durch den Vorspann gesetzt.

%% latex-abspann.tex
%% Gemeinsamer Abspann für Korrekturansicht und Leseansicht.
%% Setzt den Schalter \ifkorrekturansicht voraus (gesetzt in den
%% einbindenden Dateien latex-korrekturansicht-abspann.tex bzw.
%% latex-leseansicht-abspann.tex).
%% ---------------------------------------------------------------

\normalsize

% Das esempio-Environment wird nur in der Leseansicht benötigt
\ifkorrekturansicht\else
\newenvironment{esempio}[3]%
{
    \vspace{1.5ex}
    \rlap{\underline{#1}}
    \par
    \setlength{\parindent}{0cm}
    \nopagebreak
    \leftskip=#2cm
    \rightskip=#3cm
}
{
    \par
}
\fi

\doendnotes{C}
\bigskip
\vfill

\clearpage

\footnotesize

\ifkorrekturansicht
  \lohead{\textsc{register}}
\fi

% theindex-Environment neu definieren ohne reledmac
\makeatletter
\renewenvironment{theindex}{%
  \ifkorrekturansicht
    \section*{\indexname}%
  \else
    \subsubsection*{Index der erwähnten Entitäten}%
  \fi
  \setlength{\parindent}{0pt}%
  \setlength{\parskip}{0pt plus 0.3pt}%
  \let\item\@idxitem
}{%
  \ifkorrekturansicht\clearpage\fi
}
\makeatother

\IfFileExists{\jobname-pw.ind}{\input{\jobname-pw.ind}}{}

% Quellenangabe nur in der Leseansicht
\ifkorrekturansicht\else
% Fallback-Definitionen, falls die .tex-Datei \titel etc. nicht gesetzt hat
\providecommand{\titel}{}
\providecommand{\editorInnen}{}
\providecommand{\dateiname}{\jobname}

\vspace{3cm}

\vfill

\footnotesize
\textsc{Quelle}: \titel. Herausgegeben von {\editorInnen}. In: \emph{Arthur Schnitzler: Briefwechsel mit Autorinnen und Autoren}.
 Digitale Edition, https://schnitzler-briefe.acdh.oeaw.ac.at/{\dateiname}.html (Stand \today)
\fi

\end{document}


