%% latex-korrekturansicht-vorspann.tex
%% Vorspann für die Korrekturansicht.
%% Lädt die gemeinsame Datei latex-vorspann.tex mit gesetztem Schalter.

\newif\ifkorrekturansicht
\korrekturansichttrue

\input{../tex-inputs/latex-vorspann}


\section[Arthur Schnitzler an Hugo von Hofmannsthal, 29. 7. 1892]{L00109 Arthur Schnitzler an Hugo von Hofmannsthal, 29. 7. 1892}
\nopagebreak\mylabel{L00109v}
\rehead{ }\normalsize\beginnumbering\briefempfaengerindex{Hofmannsthal, Hugo von@\textsc{Hofmannsthal, Hugo von}!zzzSchnitzler, Arthur@\emph{von Arthur Schnitzler}!1892-07-291@{29. 7. 1892}|(be}
\toendnotes[C]{\smallbreak\pagebreak[2]}\Standort{FDH, Hs-30885,22.}
\physDesc{Brief, 1 Blatt, 4 Seiten, 1260 Zeichen
\newline{}Handschrift: schwarze Tinte, deutsche Kurrent}
\buchAbdrucke{\weitereDrucke{Hugo von Hofmannsthal, Arthur Schnitzler: \emph{Briefwechsel}. Frankfurt am Main: \emph{S. Fischer} 1964, S. 25.} }\toendnotes[C]{\smallbreak}
\pstart
           
\pstart
           {\pb}Wien\oindex{Wien@\textbf{Wien}, \emph{A.ADM2}|pw}\pend
           
\pstart
           \raggedleft{}29/7 92\pend
           \pend
           
\pstart{}Lieber Freund,\pend\vspace{0.5em}
\pstart
           nachdem Sie Ihr Gedicht\pwindex{Einleitung@\emph{Einleitung}|pwv} nicht
               im Inhalt haben wollen, möchte ich auch jeden Titel weglaſſen, und es nur im ſelben
               Druck wie alles übrige \introOben{}haben\introOben{}, jedoch mit oben weit
               freigelaſſenen Rändern \strikeout{haben}. – Einverſtanden? –\pend
           
\pstart
           Vorgeſtern habe ich meine Novelle\pwindex{Sterben. Novelle@\emph{Sterben. Novelle}|pwv} beendet. – Ich hoffe, {\pb}ſie wird, we{\geminationn}{ }ſie erſt durchgefeilt iſt, als ehrenwerte Studie
               gelten können. Ich habe ſie plötzlich zu Ende ſchreiben müſſen, Nachts im Cafè,
               während ſchläfrige Kellner bereits die Seſſel aufeinander thürmten. Ich habe ſie ſehr
               lieb gehabt – ich fühle mich ordentlich einſam, ſeit ich nicht mehr drüber denken
               muſs. {\pb}(Siehe Freund \textsc{Y}\pwindex{Mein Freund Ypsilon. Aus den Papieren eines Arztes@\emph{Mein Freund Ypsilon. Aus den Papieren eines Arztes}|pw}). – Nun will ich wieder ans Stück\pwindex{Liebelei. Schauspiel in drei Akten@\emph{Liebelei. Schauspiel in drei Akten}|pwv}. – Eben hab ich Blumenthal\pwindex{Blumenthal, Oskar 13.03.1852 – 24.04.1917@\textsc{Blumenthal, Oskar} (13.03.1852 – 24.04.1917), \emph{Schriftsteller/Schriftstellerin, Journalist/Journalistin, Theaterleiter/Theaterleiterin}|pw} u
                  Reicher\pwindex{Reicher, Emanuel 18.06.1849 – 15.05.1924@\textsc{Reicher, Emanuel} (18.06.1849 – 15.05.1924), \emph{Schauspieler/Schauspielerin}|pw} geſchrieben! – wie verdreht
               eigentlich die Welt iſt! –\pend
           
\pstart
           Was macht Ihr Stück\pwindex{Ascanio und Gioconda@\emph{Ascanio und Gioconda}|pwv}? – Ich
               wundre mich, daſs Sie zugleich zweiten und fünften Akt ſchreiben können. So ſicher
               bin ich meiner Geſtalten nie! Es kann ihnen doch im dritten Akt {\pb}was einfallen oder gar paſſiren, wovon ich im zweiten noch
               nichts rechtes weiſs. Selbſt we{\geminationn} eine genaue Skizze
               vorliegt, wage ich es nicht und habe gewiſs keine Luſt dazu! Ich will mit ihnen
               weiter leben, und erleben, Gedanke für Gedanke und That für That, wie ſie ſelber. Ich
               darf manches vorausahnen, aber wiſſen darf ichs nicht.\pend
           \pstart Herzlichſt Ihr \spacefill\mbox{Arthur}\pend{}\selectlanguage{ngerman}\endnumbering\briefempfaengerindex{Hofmannsthal, Hugo von@\textsc{Hofmannsthal, Hugo von}!zzzSchnitzler, Arthur@\emph{von Arthur Schnitzler}!1892-07-291@{29. 7. 1892}|)be}\mylabel{L00109h}  \normalsize

\doendnotes{C}
\bigskip
\vfill

\clearpage

\footnotesize

\lohead{\textsc{register}}

% Definiere theindex-Environment komplett neu ohne reledmac
\makeatletter
\renewenvironment{theindex}{%
  \section*{\indexname}%
  \setlength{\parindent}{0pt}%
  \setlength{\parskip}{0pt plus 0.3pt}%
  \let\item\@idxitem
}{%
  \clearpage
}
\makeatother

\IfFileExists{\jobname-pw.ind}{\input{\jobname-pw.ind}}{}

\end{document}

      