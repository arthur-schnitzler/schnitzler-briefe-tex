%% latex-leseansicht-vorspann.tex
%% Vorspann für die Leseansicht.
%% Lädt die gemeinsame Datei latex-vorspann.tex mit nicht gesetztem Schalter.

\newif\ifkorrekturansicht
\korrekturansichtfalse

\input{../tex-inputs/latex-vorspann}


         
         \renewcommand{\erwaehntePersonen}{Personen: Richard Beer-Hofmann, Paul Goldmann, Robert Hirschfeld, Alfred Kerr, Olga Schnitzler, Elisabeth Steinrück, Oscar Tietz}
         \renewcommand{\erwaehnteInstitutionen}{Institutionen: Hermann Tietz und Co., Neue Freie Presse}
         \renewcommand{\erwaehnteOrte}{Orte: Berlin, Breslau, Dessauer Straße, Dolomiten, Lago di Garda, Sekirn, Trient, Vahrn, Val Gardena, Wien, Wörthersee}
         \renewcommand{\erwaehnteWerke}{Werke: Baedeker-Reiseführer, Insolvenz des großen Berliner Waarenhauses Tietz, Neue Freie Presse}
               \section[ Paul Goldmann an Arthur Schnitzler, 19. 7. {[}1901{]}]{ Paul Goldmann an Arthur Schnitzler, 19. 7. {[}1901{]}}\nopagebreak\mylabel{v}\rehead{ }\begin{ledgroupsized}[t]{13cm}\normalsize\beginnumbering\briefempfaengerindex{Schnitzler, Arthur@\textsc{Schnitzler, Arthur}!zzzGoldmann, Paul@\emph{von Paul Goldmann}!1901-07-191@{19. 7. {[}1901{]}}|(be} \toendnotes[C]{\smallbreak\pagebreak[2]} \Standort{DLA, A:Schnitzler, HS.NZ85.1.3171.}
\physDesc{Brief, 2 Blätter, 6 Seiten, 2524 Zeichen
\newline{}Handschrift: blaue Tinte, deutsche Kurrent
\newline{}Schnitzler: 1) mit Bleistift das Jahr »1901« vermerkt  2) mit rotem Buntstift fünf Unterstreichungen}\toendnotes[C]{\smallbreak}\pstart
           \noindent{}\raggedleft{}{\pb}\textcolor{gray}{\textbf{DESSAUERSTRASSE 19}}\oindex{Dessauer Strasse@\textbf{Dessauer Straße}|pw}\pend
           \pstart
           Berlin\oindex{Berlin@\textbf{Berlin}|pw}, 19. Juli.\pend
           \pstart{}Mein lieber Freund,\pend\pstart
           Ich wollte morgen fahren, aber dieſe verfluchte Bande
               (die Redaktion\orgindex{Neue Freie Presse@Neue Freie Presse|pwv}) läßt mich nicht
               fort. Ich führe hier Ausgleichs-Verhandlungen mit dem Beſitzer\pwindex{Tietz, Oscar 1858-04-18 – 1923-01-17@\textsc{Tietz, Oscar} (1858-04-18 – 1923-01-17), \emph{Kaufmann}|pwv} des großen Waarenhauſes \strikeout{T\textcolor{gray}{ie}}{ }\textsc{Tietz\orgindex{Hermann Tietz und Co.@Hermann Tietz und Co.|pw}}, deſſen Inſolvenz die N. Fr. Pr.\orgindex{Neue Freie Presse@Neue Freie Presse|pw} fälſchlich
                  \label{K_L03073-1v}\edtext{gemeldet\pwindex{?? Werk@Nicht ermittelte Verfasserinnen und Verfasser!Insolvenz des grossen Berliner Waarenhauses Tietz1901-07-05@\emph{Insolvenz des großen Berliner Waarenhauses Tietz} {[}1901-07-05{]}|pwv}}{\lemma{\textnormal{\emph{gemeldet}}}\Cendnote{\textnormal{[O. V.]: \emph{Insolvenz des großen Berliner
                        Waarenhauses Tietz}\pwindex{?? Werk@Nicht ermittelte Verfasserinnen und Verfasser!Insolvenz des grossen Berliner Waarenhauses Tietz1901-07-05@\emph{Insolvenz des großen Berliner Waarenhauses Tietz} {[}1901-07-05{]}|pwk}. In: \emph{Neue Freie
                        Presse}\pwindex{Neue Freie Presse1864 – 1939@\emph{Neue Freie Presse} {[}1864 – 1939{]}|pwk}, Nr. 13.239, 5. 7. 1901,
                     Abendblatt, S. 3.}}}\label{K_L03073-1h} und der das Blatt\orgindex{Neue Freie Presse@Neue Freie Presse|pwv} klagen will. (Das ſage ich Dir im Vertrauen).
               Nach 14 tägigen Verhandlungen habe ich den Ausgleich hier endlich zuſtande gebracht.
               Da macht auf einmal die N. Fr. Pr.\orgindex{Neue Freie Presse@Neue Freie Presse|pw} neue
               Schwierigkeiten, und Alles iſt {\pb}wieder in Frage
               geſtellt.\pend
           \pstart
           Vielleicht kann ich doch wenigſtens Montag (22. Juli)
               fahren. Dann bleibe ich einen Tag in Breslau\oindex{Breslau@\textbf{Breslau}|pw},
               zwei oder drei Tage in \textsc{Wien\oindex{Wien@\textbf{Wien}|pw}}, gehe hierauf an den Wörtherſee\oindex{Woerthersee@\textbf{Wörthersee}|pw} zu \textsc{Hirschfeld\pwindex{Hirschfeld, Robert 17.09.1857 – 02.04.1914@\textsc{Hirschfeld, Robert} (17.09.1857 – 02.04.1914), \emph{Journalist, Musikkritiker}|pw}} und werde irgendwo dort wohnen. Das Beſte alſo iſt, Du ſendeſt mir weitere
               Nachricht an die Adreſſe von \textsc{Hirschfeld\pwindex{Hirschfeld, Robert 17.09.1857 – 02.04.1914@\textsc{Hirschfeld, Robert} (17.09.1857 – 02.04.1914), \emph{Journalist, Musikkritiker}|pw}} in \textsc{Seekirn\oindex{Sekirn@\textbf{Sekirn}|pw}}. Ich möchte am Wörtherſee\oindex{Woerthersee@\textbf{Wörthersee}|pw} nicht allzulange
               bleiben. \textsc{Richard\pwindex{Beer-Hofmann, Richard 1866-07-11 – 1945-09-26@\textsc{Beer-Hofmann, Richard} (1866-07-11 – 1945-09-26), \emph{Schriftsteller}|pw}}, der mir während des ganzen Jahres kein Wort geſchrieben und auch jetzt ſich
               nicht einmal zu einer Zeile {\pb}aufgeſchwungen hat, in
               der er den Wunſch ausſpricht, mich zu ſehen, werde ich wahrſcheinlich überhaupt nicht
                  \label{K_L03073-2v}\edtext{aufſuchen}{\lemma{\textnormal{\emph{aufſuchen}}}\Cendnote{\textnormal{Siehe Paul Goldmann an Arthur Schnitzler, 29. 7. [1901].
               }}}\label{K_L03073-2h}.\pend
           \pstart
           Mir liegt nun daran, in Ruhe irgendwo \uline{möglichſt hoch}
               ein paar Wochen zu verbringen, am Liebſten in den Dolomiten\oindex{Dolomiten@\textbf{Dolomiten}|pw}, wenn das Grödner Thal\oindex{Val Gardena@\textbf{Val Gardena}|pw} zu
               ſonnig iſt. Die Idee, den Schluß am Gardaſee\oindex{Lago di Garda@\textbf{Lago di Garda}|pw} zu
               machen, finde ich entzückend. Den Ort, wo wir bis dahin bleiben wollen, magſt Du
                  \label{K_L03073-3v}\edtext{beſtimmen}{\lemma{\textnormal{\emph{beſtimmen}}}\Cendnote{\textnormal{Siehe Paul Goldmann an Arthur Schnitzler, 26. 4. [1901].
               }}}\label{K_L03073-3h}. Nur bitte ich Dich, dabei auch ein klein wenig meine Wünſche zu
               berückſichtigen. So ſehr {\pb}es mir auch zur Befriedigung
               gereichen würde, an einem Orte mich aufzuhalten, wo Du Dich wohl befindeſt, ſo wäre
               es mir doch nicht \strikeout{\textcolor{gray}{×}} unangenehm, wenn an dieſem Orte auch ich mich wohlbefinden könnte. Ich
               brauche, was ein Menſch mit völlig zerrütteten Nerven braucht: Ruhe, Höhenluft,
               Kühle. Und in landſchaftlicher Beziehung habe ich, wie geſagt, ein großes \strikeout{Verlan} Verlangen nach einer Dolomiten\oindex{Dolomiten@\textbf{Dolomiten}|pw}-Gegend\substVorne{}\textsuperscript{.}\substDazwischen{} (\substHinten{}vielleicht bei Trient\oindex{Trient@\textbf{Trient}|pw}). Aber ich möchte,
               daß dies Alles ſchon vor meiner Ankunft {\pb}feſtgeſetzt
               wäre. Denn ich möchte nicht wieder, wie im vorigen Jahre,
               dreiviertel meines Urlaubs mit dem Studium von \textsc{Bädekers\pwindex{?? Werk@Nicht ermittelte Verfasserinnen und Verfasser!Baedeker-Reisefuehrer1832@\emph{Baedeker-Reiseführer} {[}1832{]}|pw}} und Eiſenbahn-Fahrplänen verbringen.\pend
           \pstart
           \textsc{Kerr\pwindex{Kerr, Alfred 25.12.1867 – 12.10.1948@\textsc{Kerr, Alfred} (25.12.1867 – 12.10.1948), \emph{Schriftsteller, Kritiker}|pw}} kann hier erſt gegen Mitte Auguſt fort. Er will
               dann \label{K_L03073-4v}\edtext{zu uns ſtoßen}{\lemma{\textnormal{\emph{zu uns ſtoßen}}}\Cendnote{\textnormal{Dazu kam es nicht.}}}\label{K_L03073-4h} und möchte gern, daß
               wir womöglich eine mehrtägige gemeinſame Fußwanderung im Gebirge machten. Auch \textsc{Hirschfeld\pwindex{Hirschfeld, Robert 17.09.1857 – 02.04.1914@\textsc{Hirschfeld, Robert} (17.09.1857 – 02.04.1914), \emph{Journalist, Musikkritiker}|pw}}{ }{\pb}werde ich dazu animiren, bei einer ſolchen Parthie
                  \label{K_L03073-5v}\edtext{mitzuhalten}{\lemma{\textnormal{\emph{mitzuhalten}}}\Cendnote{\textnormal{Dazu kam es nicht.}}}\label{K_L03073-5h}.\pend
           \pstart
           Schreib’ mir alſo nach \textsc{Seekirn\oindex{Sekirn@\textbf{Sekirn}|pw}} an \textsc{Hirschfelds\pwindex{Hirschfeld, Robert 17.09.1857 – 02.04.1914@\textsc{Hirschfeld, Robert} (17.09.1857 – 02.04.1914), \emph{Journalist, Musikkritiker}|pw}} Adreſſe. Viele treue Grüße
               Dir und den beiden lieblichen Schweſtern\pwindex{Schnitzler, Olga 17.01.1882 – 13.01.1970@\textsc{Schnitzler, Olga} (17.01.1882 – 13.01.1970), \emph{Schauspielerin, Sängerin}|pwv}\pwindex{Steinrueck, Elisabeth 19.11.1885 – 07.04.1920@\textsc{Steinrück, Elisabeth} (19.11.1885 – 07.04.1920)|pwv}! {\\[\baselineskip]}Dein {\\[\baselineskip]}\spacefill\mbox{Paul Goldmann.}\pend
           \leftskip=0em{}\pstart
           \noindent{}\label{T_L03073-1v}\edtext{Wie lange ich bei Euch bleibe? Je
                  nachdem Ihr Euch zu mir benehmt: ſehr lange oder ſehr kurz.}{\lemma{\textnormal{\emph{Wie … kurz.}}}\Cendnote{\textnormal{entlang des rechten Blattrandes, normal zum Text}}}\label{T_L03073-1h}\pend
           
         
         \endnumbering\mylabel{h}\end{ledgroupsized}  \newcommand{\dateiname}{L03073}\newcommand{\titel}{Paul Goldmann an Arthur Schnitzler, 19. 7. [1901]}\newcommand{\editorInnen}{Martin Anton Müller und Laura Untner}%% latex-leseansicht-abspann.tex
%% Abspann für die Leseansicht.
%% Der Schalter \ifkorrekturansicht ist bereits durch den Vorspann gesetzt.

%% latex-abspann.tex
%% Gemeinsamer Abspann für Korrekturansicht und Leseansicht.
%% Setzt den Schalter \ifkorrekturansicht voraus (gesetzt in den
%% einbindenden Dateien latex-korrekturansicht-abspann.tex bzw.
%% latex-leseansicht-abspann.tex).
%% ---------------------------------------------------------------

\normalsize

% Das esempio-Environment wird nur in der Leseansicht benötigt
\ifkorrekturansicht\else
\newenvironment{esempio}[3]%
{
    \vspace{1.5ex}
    \rlap{\underline{#1}}
    \par
    \setlength{\parindent}{0cm}
    \nopagebreak
    \leftskip=#2cm
    \rightskip=#3cm
}
{
    \par
}
\fi

\doendnotes{C}
\bigskip
\vfill

\clearpage

\footnotesize

\ifkorrekturansicht
  \lohead{\textsc{register}}
\fi

% theindex-Environment neu definieren ohne reledmac
\makeatletter
\renewenvironment{theindex}{%
  \ifkorrekturansicht
    \section*{\indexname}%
  \else
    \subsubsection*{Index der erwähnten Entitäten}%
  \fi
  \setlength{\parindent}{0pt}%
  \setlength{\parskip}{0pt plus 0.3pt}%
  \let\item\@idxitem
}{%
  \ifkorrekturansicht\clearpage\fi
}
\makeatother

\IfFileExists{\jobname-pw.ind}{\input{\jobname-pw.ind}}{}

% Quellenangabe nur in der Leseansicht
\ifkorrekturansicht\else
% Fallback-Definitionen, falls die .tex-Datei \titel etc. nicht gesetzt hat
\providecommand{\titel}{}
\providecommand{\editorInnen}{}
\providecommand{\dateiname}{\jobname}

\vspace{3cm}

\vfill

\footnotesize
\textsc{Quelle}: \titel. Herausgegeben von {\editorInnen}. In: \emph{Arthur Schnitzler: Briefwechsel mit Autorinnen und Autoren}.
 Digitale Edition, https://schnitzler-briefe.acdh.oeaw.ac.at/{\dateiname}.html (Stand \today)
\fi

\end{document}


      