%% latex-leseansicht-vorspann.tex
%% Vorspann für die Leseansicht.
%% Lädt die gemeinsame Datei latex-vorspann.tex mit nicht gesetztem Schalter.

\newif\ifkorrekturansicht
\korrekturansichtfalse

\input{../tex-inputs/latex-vorspann}


\section[Gerhart Hauptmann an Arthur Schnitzler, 15. 5. 1922]{L02384 Gerhart Hauptmann an Arthur Schnitzler, 15. 5. 1922}
\nopagebreak\mylabel{L02384v}
\rehead{ }\normalsize\beginnumbering\briefempfaengerindex{Schnitzler, Arthur@\textsc{Schnitzler, Arthur}!zzzHauptmann, Gerhart@\emph{von Gerhart Hauptmann}!1922-05-153@{15. 5. 1922}|(be}
\toendnotes[C]{\smallbreak\pagebreak[2]}
\correspDesc{Versand  durch Gerhart Hauptmann am 15. 5. 1922 in Agnetendorf
\newline{}Erhalt  durch Arthur Schnitzler am 15. 5. 1922 in Wien}\toendnotes[C]{\smallbreak}
\Standort{DLA, A:Schnitzler, 85.1.557.}
\physDesc{Telegramm, 319 Zeichen
\newline{}maschinell
\newline{}Versand: »\textcolor{gray}{\textbf{Aufgenommen von}}{ }\textcolor{gray}{\textbf{\textit{BL}}}{ }\textcolor{gray}{\textbf{auf Ltg.Nr.}}{ }\textcolor{gray}{\textbf{\textit{SI,25}}}{ }\textcolor{gray}{\textbf{am}}{ }\textcolor{gray}{\textbf{\textit{15. MAI}}}\textcolor{gray}{\textbf{192{\dots}}}{ }\textcolor{gray}{\textbf{um {\dots} Uhr {\dots} M. {\dots}
                                       Mitt. durch:}}{ }\textcolor{gray}{\textbf{\textit{OTTO\pwindex{Otto, Anna @\textsc{Otto, Anna}, \emph{Telegrafenbeamtin}|pw}}}}« 
\newline{}Schnitzler: mit rotem Buntstift eine Unterstreichung 
\newline{}Zusatz: umseitig eine Werbung für einen
                                    Opalograph-Vervielfältiger }
\buchAbdrucke{\weitereDrucke{Hans-Ulrich Lindken: \emph{Arthur Schnitzler. Aspekte und Akzente. Materialien zu Leben
                        und Werk}. Frankfurt am Main, Bern, Göttingen: \emph{Peter Lang} 1984, S. 416 (Europäische Hochschulschriften, Reihe 1, Deutsche Sprache und
                        Literatur, 754).} }\toendnotes[C]{\smallbreak}\pstart{}{\pb}arthur schmitzler\pend{}\pstart{}sertnwartestr 71 wien\oindex{Wien@\textbf{Wien}!XVIII., Währing@\textbf{XVIII., Währing}!Sternwartestraße 71@\textbf{Sternwartestraße 71}, \emph{Wohngebäude}|pw}\pend{}{\bigskip}\vspace{1em}
\pstart
           {\pb}178/15 agnetendorf\oindex{Jagniątków@\textbf{Jagniątków}|pw} sp 97 41 15/5{ }11,5m\pend
           \vspace{0.5em}
\pstart
           seien sie herzlichst begruesst und nehmen sie meene und meiner frau\pwindex{Hauptmann, Margarete 7.\,1.\,1875 – 17.\,1.\,1957 Ebenhausen@\textsc{Hauptmann, Margarete} (7.\,1.\,1875 – 17.\,1.\,1957 Ebenhausen)|pwv} innige wuensche zum heutigen tage und
               besonders fuer kommende schoene arbeitreiche jahre dankbar fuer genossene gaben
               wollen wir mehr mehr in alter ergebenheit\pend
           \pstart \spacefill\mbox{= gerhart hauptmann +}\pend{}\selectlanguage{ngerman}\endnumbering\briefempfaengerindex{Schnitzler, Arthur@\textsc{Schnitzler, Arthur}!zzzHauptmann, Gerhart@\emph{von Gerhart Hauptmann}!1922-05-153@{15. 5. 1922}|)be}\mylabel{L02384h}  \newcommand{\dateiname}{L02384}\newcommand{\titel}{Gerhart Hauptmann an Arthur Schnitzler, 15. 5. 1922}\newcommand{\editorInnen}{Herausgegeben von Martin Anton Müller}%% latex-leseansicht-abspann.tex
%% Abspann für die Leseansicht.
%% Der Schalter \ifkorrekturansicht ist bereits durch den Vorspann gesetzt.

%% latex-abspann.tex
%% Gemeinsamer Abspann für Korrekturansicht und Leseansicht.
%% Setzt den Schalter \ifkorrekturansicht voraus (gesetzt in den
%% einbindenden Dateien latex-korrekturansicht-abspann.tex bzw.
%% latex-leseansicht-abspann.tex).
%% ---------------------------------------------------------------

\normalsize

% Das esempio-Environment wird nur in der Leseansicht benötigt
\ifkorrekturansicht\else
\newenvironment{esempio}[3]%
{
    \vspace{1.5ex}
    \rlap{\underline{#1}}
    \par
    \setlength{\parindent}{0cm}
    \nopagebreak
    \leftskip=#2cm
    \rightskip=#3cm
}
{
    \par
}
\fi

\doendnotes{C}
\bigskip
\vfill

\clearpage

\footnotesize

\ifkorrekturansicht
  \lohead{\textsc{register}}
\fi

% theindex-Environment neu definieren ohne reledmac
\makeatletter
\renewenvironment{theindex}{%
  \ifkorrekturansicht
    \section*{\indexname}%
  \else
    \subsubsection*{Index der erwähnten Entitäten}%
  \fi
  \setlength{\parindent}{0pt}%
  \setlength{\parskip}{0pt plus 0.3pt}%
  \let\item\@idxitem
}{%
  \ifkorrekturansicht\clearpage\fi
}
\makeatother

\IfFileExists{\jobname-pw.ind}{\input{\jobname-pw.ind}}{}

% Quellenangabe nur in der Leseansicht
\ifkorrekturansicht\else
% Fallback-Definitionen, falls die .tex-Datei \titel etc. nicht gesetzt hat
\providecommand{\titel}{}
\providecommand{\editorInnen}{}
\providecommand{\dateiname}{\jobname}

\vspace{3cm}

\vfill

\footnotesize
\textsc{Quelle}: \titel. Herausgegeben von {\editorInnen}. In: \emph{Arthur Schnitzler: Briefwechsel mit Autorinnen und Autoren}.
 Digitale Edition, https://schnitzler-briefe.acdh.oeaw.ac.at/{\dateiname}.html (Stand \today)
\fi

\end{document}


