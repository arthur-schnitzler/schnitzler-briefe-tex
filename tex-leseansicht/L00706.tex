%% latex-korrekturansicht-vorspann.tex
%% Vorspann für die Korrekturansicht.
%% Lädt die gemeinsame Datei latex-vorspann.tex mit gesetztem Schalter.

\newif\ifkorrekturansicht
\korrekturansichttrue

\input{../tex-inputs/latex-vorspann}


\section[Hugo von Hofmannsthal an Arthur Schnitzler, {[}19. 7. 1897{]}]{L00706 Hugo von Hofmannsthal an Arthur Schnitzler, {[}19. 7. 1897{]}}
\nopagebreak\mylabel{L00706v}
\rehead{ }\normalsize\beginnumbering\briefempfaengerindex{Schnitzler, Arthur@\textsc{Schnitzler, Arthur}!zzzHofmannsthal, Hugo von@\emph{von Hugo von Hofmannsthal}!1897-07-191@{{[}19. 7. 1897{]}}|(be}
\toendnotes[C]{\smallbreak\pagebreak[2]}\Standort{CUL, Schnitzler, B 43.}
\physDesc{Brief, 1 Blatt, 3 Seiten, 770 Zeichen
\newline{}Handschrift: Bleistift, deutsche Kurrent
\newline{}Schnitzler: mit Bleistift falsch datiert: »1\substVorne{}\textsuperscript{8}\substDazwischen{}9\substHinten{}/7 96« 
\newline{}Ordnung: 1) mit Bleistift von unbekannter Hand nummeriert: »\strikeout{95}«  2) mit Bleistift von unbekannter Hand nummeriert:
                                    »78a«}
\buchAbdrucke{\weitereDrucke{Hugo von Hofmannsthal, Arthur Schnitzler: \emph{Briefwechsel}. Frankfurt am Main: \emph{S. Fischer} 1964, S. 93.} }\toendnotes[C]{\smallbreak}
\pstart
           \raggedleft{}{\pb}Montag.\pend
           
\pstart{}\strikeout{Herr}{ }mein lieber Arthur!\pend\vspace{0.5em}
\pstart
           ich habe erſt heute erfahren, daſs Papa\pwindex{Hofmannsthal, Hugo August von 21.12.1841 – 08.12.1915@\textsc{Hofmannsthal, Hugo August von} (21.12.1841 – 08.12.1915), \emph{Bankdirektor/Bankdirektorin}|pwv} nächſten Montag von hier\oindex{Bad Fusch@\textbf{Bad Fusch}, \emph{A.ADM3}|pwv} abreiſt; ſo möchte ich nicht gern den
               letzten Tag von hier fort und wir laſſen alſo lieber das \textsc{rendez
                  vous}. Es thut mir ſehr leid, aber wenn wir beide etwas gearbeitet haben
               werden, wird es eine große Freude ſein, uns im Spätherbſt wieder{\pb}zuſehen. Sie ſchreiben mir wohl
               hie und da eine Zeile nach Italien\oindex{Italien@\textbf{Italien}, \emph{A.PCLI}|pw}, ich werde
               Ihnen immer meine Adreſſe zuko{\geminationm}en laſſen.\pend
           
\pstart
           Die Mozart\pwindex{Mozart, Wolfgang Amadeus 27.01.1756 – 05.12.1791@\textsc{Mozart, Wolfgang Amadeus} (27.01.1756 – 05.12.1791), \emph{Komponist/Komponistin}|pw}-biographie\pwindex{W. A. Mozart@\emph{W. A. Mozart}|pwv}
                iſt ein entzückendes Buch von einer unglaublichen Ausführlichkeit und
               Intimität. Man gewinnt ihn\pwindex{Mozart, Wolfgang Amadeus 27.01.1756 – 05.12.1791@\textsc{Mozart, Wolfgang Amadeus} (27.01.1756 – 05.12.1791), \emph{Komponist/Komponistin}|pwv}{ }ſehr lieb. Ich ſchicke Ihnen die beiden Bände im
                  Auguſt nach Wien\oindex{Wien@\textbf{Wien}, \emph{A.ADM2}|pw}.\pend
           
\pstart
           {\pb}Werd ich von Richard\pwindex{Beer-Hofmann, Richard 1866-07-11 – 1945-09-26@\textsc{Beer-Hofmann, Richard} (1866-07-11 – 1945-09-26), \emph{Schriftsteller/Schriftstellerin}|pw} nie auch nur eine Zeile bekommen?\pend
           
\pstart
           Es ärgert mich ſehr.\pend
           
\pstart
           Ich wünſche Ihnen für die nächſten 2 Monate alles Gute.\pend
           
\pstart
           Von Herzen Ihr{\\[\baselineskip]}\spacefill\mbox{Hugo.}\pend
           \leftskip=0em{}\selectlanguage{ngerman}\endnumbering\briefempfaengerindex{Schnitzler, Arthur@\textsc{Schnitzler, Arthur}!zzzHofmannsthal, Hugo von@\emph{von Hugo von Hofmannsthal}!1897-07-191@{{[}19. 7. 1897{]}}|)be}\mylabel{L00706h}  \normalsize

\doendnotes{C}
\bigskip
\vfill

\clearpage

\footnotesize

\lohead{\textsc{register}}

% Definiere theindex-Environment komplett neu ohne reledmac
\makeatletter
\renewenvironment{theindex}{%
  \section*{\indexname}%
  \setlength{\parindent}{0pt}%
  \setlength{\parskip}{0pt plus 0.3pt}%
  \let\item\@idxitem
}{%
  \clearpage
}
\makeatother

\IfFileExists{\jobname-pw.ind}{\input{\jobname-pw.ind}}{}

\end{document}

      