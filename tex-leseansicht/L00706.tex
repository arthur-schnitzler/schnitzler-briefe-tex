%% latex-leseansicht-vorspann.tex
%% Vorspann für die Leseansicht.
%% Lädt die gemeinsame Datei latex-vorspann.tex mit nicht gesetztem Schalter.

\newif\ifkorrekturansicht
\korrekturansichtfalse

\input{../tex-inputs/latex-vorspann}


\section[Hugo von Hofmannsthal an Arthur Schnitzler, {[}19. 7. 1897{]}]{L00706 Hugo von Hofmannsthal an Arthur Schnitzler, {[}19. 7. 1897{]}}
\nopagebreak\mylabel{L00706v}
\rehead{ }\normalsize\beginnumbering\briefempfaengerindex{Schnitzler, Arthur@\textsc{Schnitzler, Arthur}!zzzHofmannsthal, Hugo von@\emph{von Hugo von Hofmannsthal}!1897-07-191@{{[}19. 7. 1897{]}}|(be}
\toendnotes[C]{\smallbreak\pagebreak[2]}
\correspDesc{Versand  durch Hugo von Hofmannsthal am [19. 7. 1897] in Bad Fusch
\newline{}Erhalt  durch Arthur Schnitzler im Zeitraum [20. 7. 1897
                  – 24. 7. 1897?] in Wien}\toendnotes[C]{\smallbreak}
\Standort{CUL, Schnitzler, B 43.}
\physDesc{Brief, 1 Blatt, 3 Seiten, 770 Zeichen
\newline{}Handschrift: Bleistift, deutsche Kurrent
\newline{}Schnitzler: mit Bleistift falsch datiert: »1\substVorne{}\textsuperscript{8}\substDazwischen{}9\substHinten{}/7 96« 
\newline{}Ordnung: 1) mit Bleistift von unbekannter Hand nummeriert: »\strikeout{95}«  2) mit Bleistift von unbekannter Hand nummeriert:
                                    »78a«}
\buchAbdrucke{\weitereDrucke{Hugo von Hofmannsthal, Arthur Schnitzler: \emph{Briefwechsel}. Herausgegeben von Therese Nickl und Heinrich Schnitzler. Frankfurt am Main: \emph{S. Fischer} 1964, S. 93.} }\toendnotes[C]{\smallbreak}
\pstart
           \raggedleft{}{\pb}Montag.\pend
           
\pstart{}\strikeout{Herr}{ }mein lieber Arthur!\pend\vspace{0.5em}
\pstart
           ich habe erſt heute erfahren, daſs Papa\pwindex{Hofmannsthal, Hugo August von 21.\,12.\,1841 Wien – 8.\,12.\,1915 ebd.@\textsc{Hofmannsthal, Hugo August von} (21.\,12.\,1841 Wien – 8.\,12.\,1915 ebd.), \emph{Bankdirektor}|pwv} nächſten Montag von hier\oindex{Bad Fusch@\textbf{Bad Fusch}|pwv} abreiſt;{ }ſo möchte ich nicht gern den
               letzten Tag von hier fort und wir laſſen alſo lieber das \textsc{rendez
                  vous}. Es thut mir{ }ſehr leid, aber wenn wir beide etwas gearbeitet haben
               werden, wird es eine große Freude{ }ſein, uns im Spätherbſt wieder{\pb}zuſehen. Sie{ }ſchreiben mir wohl
               hie und da eine Zeile nach Italien\oindex{Italien@\textbf{Italien}|pw}, ich werde
               Ihnen immer meine Adreſſe zuko{\geminationm}en laſſen.\pend
           
\pstart
           Die Mozart\pwindex{Mozart, Wolfgang Amadeus 27.\,1.\,1756 Salzburg – 5.\,12.\,1791 Wien@\textsc{Mozart, Wolfgang Amadeus} (27.\,1.\,1756 Salzburg – 5.\,12.\,1791 Wien), \emph{Komponist}|pw}-biographie\pwindex{\textcolor{red}{\textsuperscript{XXXX indx1}}!W. A. Mozart@\strich\emph{W. A. Mozart}|pwv}
                iſt ein entzückendes Buch von einer unglaublichen Ausführlichkeit und
               Intimität. Man gewinnt ihn\pwindex{Mozart, Wolfgang Amadeus 27.\,1.\,1756 Salzburg – 5.\,12.\,1791 Wien@\textsc{Mozart, Wolfgang Amadeus} (27.\,1.\,1756 Salzburg – 5.\,12.\,1791 Wien), \emph{Komponist}|pwv}{ }ſehr lieb. Ich{ }ſchicke Ihnen die beiden Bände im
                  Auguſt nach Wien\oindex{Wien@\textbf{Wien}, \emph{Verwaltungsgebiet}|pw}.\pend
           
\pstart
           {\pb}Werd ich von Richard\pwindex{Beer-Hofmann, Richard 11.\,7.\,1866 Wien – 26.\,9.\,1945 New York City@\textsc{Beer-Hofmann, Richard} (11.\,7.\,1866 Wien – 26.\,9.\,1945 New York City), \emph{Schriftsteller}|pw} nie auch nur eine Zeile bekommen?\pend
           
\pstart
           Es ärgert mich{ }ſehr.\pend
           
\pstart
           Ich wünſche Ihnen für die nächſten 2 Monate alles Gute.\pend
           
\pstart
           Von Herzen Ihr{\\[\baselineskip]}\spacefill\mbox{Hugo.}\pend
           \leftskip=0em{}\selectlanguage{ngerman}\endnumbering\briefempfaengerindex{Schnitzler, Arthur@\textsc{Schnitzler, Arthur}!zzzHofmannsthal, Hugo von@\emph{von Hugo von Hofmannsthal}!1897-07-191@{{[}19. 7. 1897{]}}|)be}\mylabel{L00706h}  \newcommand{\dateiname}{L00706}\newcommand{\titel}{Hugo von Hofmannsthal an Arthur Schnitzler, [19. 7. 1897]}\newcommand{\editorInnen}{Martin Anton Müller und Gerd-Hermann Susen}%% latex-leseansicht-abspann.tex
%% Abspann für die Leseansicht.
%% Der Schalter \ifkorrekturansicht ist bereits durch den Vorspann gesetzt.

%% latex-abspann.tex
%% Gemeinsamer Abspann für Korrekturansicht und Leseansicht.
%% Setzt den Schalter \ifkorrekturansicht voraus (gesetzt in den
%% einbindenden Dateien latex-korrekturansicht-abspann.tex bzw.
%% latex-leseansicht-abspann.tex).
%% ---------------------------------------------------------------

\normalsize

% Das esempio-Environment wird nur in der Leseansicht benötigt
\ifkorrekturansicht\else
\newenvironment{esempio}[3]%
{
    \vspace{1.5ex}
    \rlap{\underline{#1}}
    \par
    \setlength{\parindent}{0cm}
    \nopagebreak
    \leftskip=#2cm
    \rightskip=#3cm
}
{
    \par
}
\fi

\doendnotes{C}
\bigskip
\vfill

\clearpage

\footnotesize

\ifkorrekturansicht
  \lohead{\textsc{register}}
\fi

% theindex-Environment neu definieren ohne reledmac
\makeatletter
\renewenvironment{theindex}{%
  \ifkorrekturansicht
    \section*{\indexname}%
  \else
    \subsubsection*{Index der erwähnten Entitäten}%
  \fi
  \setlength{\parindent}{0pt}%
  \setlength{\parskip}{0pt plus 0.3pt}%
  \let\item\@idxitem
}{%
  \ifkorrekturansicht\clearpage\fi
}
\makeatother

\IfFileExists{\jobname-pw.ind}{\input{\jobname-pw.ind}}{}

% Quellenangabe nur in der Leseansicht
\ifkorrekturansicht\else
% Fallback-Definitionen, falls die .tex-Datei \titel etc. nicht gesetzt hat
\providecommand{\titel}{}
\providecommand{\editorInnen}{}
\providecommand{\dateiname}{\jobname}

\vspace{3cm}

\vfill

\footnotesize
\textsc{Quelle}: \titel. Herausgegeben von {\editorInnen}. In: \emph{Arthur Schnitzler: Briefwechsel mit Autorinnen und Autoren}.
 Digitale Edition, https://schnitzler-briefe.acdh.oeaw.ac.at/{\dateiname}.html (Stand \today)
\fi

\end{document}


