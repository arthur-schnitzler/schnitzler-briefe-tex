%% latex-leseansicht-vorspann.tex
%% Vorspann für die Leseansicht.
%% Lädt die gemeinsame Datei latex-vorspann.tex mit nicht gesetztem Schalter.

\newif\ifkorrekturansicht
\korrekturansichtfalse

\input{../tex-inputs/latex-vorspann}


               \section[Felix Braun an Arthur Schnitzler, 26. 12. 1924]{ Felix Braun an Arthur Schnitzler, 26. 12. 1924}\nopagebreak\mylabel{v}\rehead{ }\begin{ledgroupsized}[t]{13cm}\normalsize\beginnumbering\briefempfaengerindex{Schnitzler, Arthur@\textsc{Schnitzler, Arthur}!zzzBraun, Felix@\emph{von Felix Braun}!1924-12-261@{26. 12. 1924}|(be} \toendnotes[C]{\smallbreak\pagebreak[2]} \Standort{DLA, A:Schnitzler, HS.NZ85.1.2604,8.}
\physDesc{Brief, 1 Blatt, 2 Seiten
\newline{}Handschrift: schwarze Tinte, deutsche Kurrent
\newline{}Schnitzler: 1) mit Bleistift beschriftet: »\textsc{Felix Braun}« 2) mit rotem Buntstift zwei Unterstreichungen}\toendnotes[C]{\smallbreak}\pstart
           \centering{}{\pb}Wien\oindex{Wien@\textbf{Wien}|pw}, den 26. XII. 1924\pend
           \pstart{}Verehrter Herr Doktor!\pend\pstart
           Herzlich danke ich Ihnen, daß Sie meinen Wunſch ſo lieb erfüllt haben. Gerade am
                    heiligen Abend kam das ſchöne Geſchenk, zu meiner großen Freude. In einem Zug
                    habe ich das Buch\pwindex{Schnitzler, Arthur 15.05.1862 – 21.10.1931@\textsc{Schnitzler, Arthur} (15.05.1862 – 21.10.1931), \emph{Schriftsteller, Mediziner}!Fraeulein Else01. 10. 1924@\strich\emph{Fräulein Else} {[}01. 10. 1924{]}|pwv} geleſen,
                    das jeder Zoll ein Werk eines Meiſters iſt. Ein ſpätes Gegenſtück zu »Sterben\pwindex{Schnitzler, Arthur 15.05.1862 – 21.10.1931@\textsc{Schnitzler, Arthur} (15.05.1862 – 21.10.1931), \emph{Schriftsteller, Mediziner}!Sterben. Novelle1.10.1894 – 1.12.1894@\strich\emph{Sterben. Novelle} {[}1.10.1894 – 1.12.1894{]}|pw}«: ein epiſches Monodrama, ſicherlich
                    eine Kunſtform, die Ihr einziges Eigentum iſt. Man hört erfreulicher Weiſe nur
                    Lob über dies Buch\pwindex{Schnitzler, Arthur 15.05.1862 – 21.10.1931@\textsc{Schnitzler, Arthur} (15.05.1862 – 21.10.1931), \emph{Schriftsteller, Mediziner}!Fraeulein Else01. 10. 1924@\strich\emph{Fräulein Else} {[}01. 10. 1924{]}|pwv}, das
                    hoffentlich auch Ihnen weiter Freude macht.\pend
           \pstart
           Herzlich dankbar und verehrungsvoll {\pb}und mit
                    allen guten Wünſchen für das neue Jahr bleibe ich, werter Herr Doktor,\hspace*{1.5em}Ihr ergebener{\\[\baselineskip]}\spacefill\mbox{Felix Braun.}\pend
           \leftskip=0em{}\endnumbering\briefempfaengerindex{Schnitzler, Arthur@\textsc{Schnitzler, Arthur}!zzzBraun, Felix@\emph{von Felix Braun}!1924-12-261@{26. 12. 1924}|)be}\mylabel{h}\end{ledgroupsized}  \newcommand{\dateiname}{L02425}\newcommand{\titel}{Felix Braun an Arthur Schnitzler, 26. 12. 1924}\newcommand{\editorInnen}{Martin Anton Müller und Gerd-Hermann Susen}%% latex-leseansicht-abspann.tex
%% Abspann für die Leseansicht.
%% Der Schalter \ifkorrekturansicht ist bereits durch den Vorspann gesetzt.

%% latex-abspann.tex
%% Gemeinsamer Abspann für Korrekturansicht und Leseansicht.
%% Setzt den Schalter \ifkorrekturansicht voraus (gesetzt in den
%% einbindenden Dateien latex-korrekturansicht-abspann.tex bzw.
%% latex-leseansicht-abspann.tex).
%% ---------------------------------------------------------------

\normalsize

% Das esempio-Environment wird nur in der Leseansicht benötigt
\ifkorrekturansicht\else
\newenvironment{esempio}[3]%
{
    \vspace{1.5ex}
    \rlap{\underline{#1}}
    \par
    \setlength{\parindent}{0cm}
    \nopagebreak
    \leftskip=#2cm
    \rightskip=#3cm
}
{
    \par
}
\fi

\doendnotes{C}
\bigskip
\vfill

\clearpage

\footnotesize

\ifkorrekturansicht
  \lohead{\textsc{register}}
\fi

% theindex-Environment neu definieren ohne reledmac
\makeatletter
\renewenvironment{theindex}{%
  \ifkorrekturansicht
    \section*{\indexname}%
  \else
    \subsubsection*{Index der erwähnten Entitäten}%
  \fi
  \setlength{\parindent}{0pt}%
  \setlength{\parskip}{0pt plus 0.3pt}%
  \let\item\@idxitem
}{%
  \ifkorrekturansicht\clearpage\fi
}
\makeatother

\IfFileExists{\jobname-pw.ind}{\input{\jobname-pw.ind}}{}

% Quellenangabe nur in der Leseansicht
\ifkorrekturansicht\else
% Fallback-Definitionen, falls die .tex-Datei \titel etc. nicht gesetzt hat
\providecommand{\titel}{}
\providecommand{\editorInnen}{}
\providecommand{\dateiname}{\jobname}

\vspace{3cm}

\vfill

\footnotesize
\textsc{Quelle}: \titel. Herausgegeben von {\editorInnen}. In: \emph{Arthur Schnitzler: Briefwechsel mit Autorinnen und Autoren}.
 Digitale Edition, https://schnitzler-briefe.acdh.oeaw.ac.at/{\dateiname}.html (Stand \today)
\fi

\end{document}


      