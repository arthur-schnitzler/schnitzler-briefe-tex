%% latex-korrekturansicht-vorspann.tex
%% Vorspann für die Korrekturansicht.
%% Lädt die gemeinsame Datei latex-vorspann.tex mit gesetztem Schalter.

\newif\ifkorrekturansicht
\korrekturansichttrue

\input{../tex-inputs/latex-vorspann}


\section[Felix Braun an Arthur Schnitzler, 26. 12. 1924]{L02425 Felix Braun an Arthur Schnitzler, 26. 12. 1924}
\nopagebreak\mylabel{L02425v}
\rehead{ }\normalsize\beginnumbering\briefempfaengerindex{Schnitzler, Arthur@\textsc{Schnitzler, Arthur}!zzzBraun, Felix@\emph{von Felix Braun}!1924-12-261@{26. 12. 1924}|(be}
\toendnotes[C]{\smallbreak\pagebreak[2]}\Standort{DLA, A:Schnitzler, HS.NZ85.1.2604,8.}
\physDesc{Brief, 1 Blatt, 2 Seiten, 627 Zeichen
\newline{}Handschrift: schwarze Tinte, deutsche Kurrent
\newline{}Schnitzler: 1) mit Bleistift beschriftet: »\textsc{Felix Braun}«  2) mit rotem Buntstift zwei Unterstreichungen}\toendnotes[C]{\smallbreak}
\pstart
           \centering{}{\pb}Wien\oindex{Wien@\textbf{Wien}, \emph{A.ADM2}|pw}, den 26. XII. 1924\pend
           
\pstart{}Verehrter Herr Doktor!\pend\vspace{0.5em}
\pstart
           Herzlich danke ich Ihnen, daß Sie meinen Wunſch ſo lieb erfüllt haben. Gerade am
               heiligen Abend kam das ſchöne Geſchenk, zu meiner großen Freude. In einem Zug habe
               ich das Buch\pwindex{Fraeulein Else@\emph{Fräulein Else}|pwv} geleſen, das
               jeder Zoll ein Werk eines Meiſters iſt. Ein ſpätes Gegenſtück zu »Sterben\pwindex{Sterben. Novelle@\emph{Sterben. Novelle}|pw}«: ein epiſches Monodrama, ſicherlich eine Kunſtform, die
               Ihr einziges Eigentum iſt. Man hört erfreulicher Weiſe nur Lob über dies Buch\pwindex{Fraeulein Else@\emph{Fräulein Else}|pwv}, das hoffentlich auch
               Ihnen weiter Freude macht.\pend
           
\pstart
           Herzlich dankbar und verehrungsvoll {\pb}und mit
               allen guten Wünſchen für das neue Jahr bleibe ich, werter Herr Doktor,\hspace*{1.5em}Ihr ergebener{\\[\baselineskip]}\spacefill\mbox{Felix Braun.}\pend
           \leftskip=0em{}\selectlanguage{ngerman}\endnumbering\briefempfaengerindex{Schnitzler, Arthur@\textsc{Schnitzler, Arthur}!zzzBraun, Felix@\emph{von Felix Braun}!1924-12-261@{26. 12. 1924}|)be}\mylabel{L02425h}  \normalsize

\doendnotes{C}
\bigskip
\vfill

\clearpage

\footnotesize

\lohead{\textsc{register}}

% Definiere theindex-Environment komplett neu ohne reledmac
\makeatletter
\renewenvironment{theindex}{%
  \section*{\indexname}%
  \setlength{\parindent}{0pt}%
  \setlength{\parskip}{0pt plus 0.3pt}%
  \let\item\@idxitem
}{%
  \clearpage
}
\makeatother

\IfFileExists{\jobname-pw.ind}{\input{\jobname-pw.ind}}{}

\end{document}

      