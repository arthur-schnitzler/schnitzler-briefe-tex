%% latex-korrekturansicht-vorspann.tex
%% Vorspann für die Korrekturansicht.
%% Lädt die gemeinsame Datei latex-vorspann.tex mit gesetztem Schalter.

\newif\ifkorrekturansicht
\korrekturansichttrue

\input{../tex-inputs/latex-vorspann}


\section[ Paul Goldmann an Arthur Schnitzler, 7. 8. {[}1900{]}]{L02927 Paul Goldmann an Arthur Schnitzler, 7. 8. {[}1900{]}}
\nopagebreak\mylabel{L02927v}
\rehead{ }\normalsize\beginnumbering\briefempfaengerindex{Schnitzler, Arthur@\textsc{Schnitzler, Arthur}!zzzGoldmann, Paul@\emph{von Paul Goldmann}!1900-08-071@{7. 8. {[}1900{]}}|(be}
\toendnotes[C]{\smallbreak\pagebreak[2]}\Standort{DLA, A:Schnitzler, HS.NZ85.1.3170.}
\physDesc{Brief, 1 Blatt, 2 Seiten, 562 Zeichen
\newline{}Handschrift: blaue Tinte, deutsche Kurrent
\newline{}Schnitzler: 1) mit schwarzer Tinte das Jahr »90\textcolor{gray}{0}.« vermerkt  2) mit rotem Buntstift eine Unterstreichung}\toendnotes[C]{\smallbreak}
\pstart
           \noindent{}
\pstart
           {\pb}Berlin\oindex{Berlin@\textbf{Berlin}, \emph{P.PPLC}|pw}, 7. Auguſt.\pend
           
\pstart
           \raggedleft{}\textcolor{gray}{\textbf{DESSAUERSTRASSE 19}}\oindex{Dessauer Strasse@\textbf{Dessauer Straße}, \emph{Straße (K.STR)}|pw}\pend
           \pend
           
\pstart
           \centering{}Mein lieber Freund,\pend
           
\pstart
           Ich muß meine Abreiſe wieder verſchieben. Die »Neue
                  Freie Preſſe\orgindex{Neue Freie Presse@Neue Freie Presse|pw}« will einen \label{K_L02927-1v}\edtext{Vertreter\pwindex{?? [Urlaubsvertretung von Paul Goldmann, 2. Augusthaelfte 1900] @\textsc{?? [Urlaubsvertretung von Paul Goldmann, 2. Augusthälfte 1900]}|pwv}}{\lemma{\textnormal{\emph{Vertreter}}}\Cendnote{\textnormal{nicht ermittelt}}}\label{K_L02927-1}{ }hierher\oindex{Berlin@\textbf{Berlin}, \emph{P.PPLC}|pwv} ſenden, und dieſer
               ſchreibt mir eben, er könne am 10. Auguſt nicht kommen
               und werde erſt »einige Tage ſpäter« eintreffen. \strikeout{Ich}
               Es iſt die \strikeout{ge\textcolor{gray}{w}} übliche Rückſichtsloſigkeit und Schweinewirthſchaft. Aber da iſt nichts zu
               machen. {\pb}Bitte \textsc{Richard\pwindex{Beer-Hofmann, Richard 1866-07-11 – 1945-09-26@\textsc{Beer-Hofmann, Richard} (1866-07-11 – 1945-09-26), \emph{Schriftsteller/Schriftstellerin}|pw}} und \textsc{Kerr\pwindex{Kerr, Alfred 25.12.1867 – 12.10.1948@\textsc{Kerr, Alfred} (25.12.1867 – 12.10.1948), \emph{Schriftsteller/Schriftstellerin, Kritiker/Kritikerin}|pw}} (\textsc{Toblach\oindex{Toblach@\textbf{Toblach}, \emph{A.ADM3}|pw}}, \textsc{Schwarzer Adler\oindex{Schwarzer Adler [Toblach]@\textbf{Schwarzer Adler [Toblach]}, \emph{Gastgewerbegebäude (K.GGW)}|pw}}) zu benachrichtigen. Ich habe in dieſen Tagen keine Zeit.\pend
           
\pstart
           Viele treue Grüße! {\\[\baselineskip]}Dein {\\[\baselineskip]}\spacefill\mbox{Paul Goldmnn}\pend
           \leftskip=0em{}
\pstart
           \noindent{}\textsc{Brandes\pwindex{Brandes, Georg 04.02.1842 – 19.02.1927@\textsc{Brandes, Georg} (04.02.1842 – 19.02.1927)|pw}} iſt hier\oindex{Berlin@\textbf{Berlin}, \emph{P.PPLC}|pwv}. Wir waren
                     geſtern{ }Abend zuſammen und haben viel von Dir geſprochen.\pend
           \selectlanguage{ngerman}\endnumbering\briefempfaengerindex{Schnitzler, Arthur@\textsc{Schnitzler, Arthur}!zzzGoldmann, Paul@\emph{von Paul Goldmann}!1900-08-071@{7. 8. {[}1900{]}}|)be}\mylabel{L02927h}  \normalsize

\doendnotes{C}
\bigskip
\vfill

\clearpage

\footnotesize

\lohead{\textsc{register}}

% Definiere theindex-Environment komplett neu ohne reledmac
\makeatletter
\renewenvironment{theindex}{%
  \section*{\indexname}%
  \setlength{\parindent}{0pt}%
  \setlength{\parskip}{0pt plus 0.3pt}%
  \let\item\@idxitem
}{%
  \clearpage
}
\makeatother

\IfFileExists{\jobname-pw.ind}{\input{\jobname-pw.ind}}{}

\end{document}

      