%% latex-leseansicht-vorspann.tex
%% Vorspann für die Leseansicht.
%% Lädt die gemeinsame Datei latex-vorspann.tex mit nicht gesetztem Schalter.

\newif\ifkorrekturansicht
\korrekturansichtfalse

\input{../tex-inputs/latex-vorspann}


\section[ Paul Goldmann an Arthur Schnitzler, 7. 8. [1900]]{L02927 Paul Goldmann an Arthur Schnitzler,  7. 8. [1900]}
\nopagebreak\mylabel{L02927v}
\rehead{ }\normalsize\beginnumbering\briefempfaengerindex{Schnitzler, Arthur@\textsc{Schnitzler, Arthur}!zzzGoldmann, Paul@\emph{von Paul Goldmann}!1900-08-071@{7. 8. [1900]}|(be}
\toendnotes[C]{\smallbreak\pagebreak[2]}
\correspDesc{Versand  durch Paul Goldmann am 7. 8. [1900] in Berlin
\newline{}Erhalt  durch Arthur Schnitzler im Zeitraum [8. 8. 1900
                  – 10. 8. 1900?] in Bad Ischl?}\toendnotes[C]{\smallbreak}
\Standort{DLA, A:Schnitzler, HS.NZ85.1.3170.}
\physDesc{Brief, 1 Blatt, 2 Seiten, 562 Zeichen
\newline{}Handschrift: blaue Tinte, deutsche Kurrent
\newline{}Schnitzler: 1) mit schwarzer Tinte das Jahr »90\textcolor{gray}{0}.« vermerkt  2) mit rotem Buntstift eine Unterstreichung}\toendnotes[C]{\smallbreak}
\pstart
           \noindent{}
\pstart
           {\pb}Berlin\oindex{Berlin@\textbf{Berlin}, \emph{Hauptstadt}|pw}, 7. Auguſt.\pend
           
\pstart
           \raggedleft{}\textcolor{gray}{\textbf{DESSAUERSTRASSE 19}}\oindex{Dessauer Straße@\textbf{Dessauer Straße}, \emph{Straße}|pw}\pend
           \pend
           
\pstart
           \centering{}Mein lieber Freund,\pend
           
\pstart
           Ich muß meine Abreiſe wieder verſchieben. Die »Neue
                  Freie Preſſe\orgindex{Neue Freie Presse@Neue Freie Presse|pw}« will einen \label{K_L02927-1v}\edtext{Vertreter\pwindex{?? [Urlaubsvertretung von Paul Goldmann, 2. Augusthälfte 1900] @\textsc{?? [Urlaubsvertretung von Paul Goldmann, 2. Augusthälfte 1900]}|pwv}}{\lemma{\textnormal{\emph{Vertreter}}}\Cendnote{\textnormal{nicht ermittelt}}}\label{K_L02927-1}{ }hierher\oindex{Berlin@\textbf{Berlin}, \emph{Hauptstadt}|pwv}{ }ſenden, und dieſer{ }ſchreibt mir eben, er könne am 10. Auguſt nicht kommen
               und werde erſt »einige Tage{ }ſpäter« eintreffen. \strikeout{Ich}
               Es iſt die \strikeout{ge\textcolor{gray}{w}} übliche Rückſichtsloſigkeit und Schweinewirthſchaft. Aber da iſt nichts zu
               machen. {\pb}Bitte \textsc{Richard\pwindex{Beer-Hofmann, Richard 11.\,7.\,1866 Wien – 26.\,9.\,1945 New York City@\textsc{Beer-Hofmann, Richard} (11.\,7.\,1866 Wien – 26.\,9.\,1945 New York City), \emph{Schriftsteller}|pw}} und \textsc{Kerr\pwindex{Kerr, Alfred 25.\,12.\,1867 Breslau – 12.\,10.\,1948 Hamburg@\textsc{Kerr, Alfred} (25.\,12.\,1867 Breslau – 12.\,10.\,1948 Hamburg), \emph{Schriftsteller, Kritiker}|pw}} (\textsc{Toblach\oindex{Toblach@\textbf{Toblach}, \emph{Verwaltungsgebiet}|pw}}, \textsc{Schwarzer Adler\oindex{Schwarzer Adler [Toblach]@\textbf{Schwarzer Adler [Toblach]}, \emph{Gastgewerbegebäude}|pw}}) zu benachrichtigen. Ich habe in dieſen Tagen keine Zeit.\pend
           
\pstart
           Viele treue Grüße! {\\[\baselineskip]}Dein {\\[\baselineskip]}\spacefill\mbox{Paul Goldmnn}\pend
           \leftskip=0em{}
\pstart
           \noindent{}\textsc{Brandes\pwindex{Brandes, Georg 4.\,2.\,1842 Kopenhagen – 19.\,2.\,1927 ebd.@\textsc{Brandes, Georg} (4.\,2.\,1842 Kopenhagen – 19.\,2.\,1927 ebd.)|pw}} iſt hier\oindex{Berlin@\textbf{Berlin}, \emph{Hauptstadt}|pwv}. Wir waren
                     geſtern{ }Abend zuſammen und haben viel von Dir geſprochen.\pend
           \selectlanguage{ngerman}\endnumbering\briefempfaengerindex{Schnitzler, Arthur@\textsc{Schnitzler, Arthur}!zzzGoldmann, Paul@\emph{von Paul Goldmann}!1900-08-071@{7. 8. [1900]}|)be}\mylabel{L02927h}  \newcommand{\dateiname}{L02927}\newcommand{\titel}{Paul Goldmann an Arthur Schnitzler, 7. 8. [1900]}\newcommand{\editorInnen}{Martin Anton Müller und Laura Untner}%% latex-leseansicht-abspann.tex
%% Abspann für die Leseansicht.
%% Der Schalter \ifkorrekturansicht ist bereits durch den Vorspann gesetzt.

%% latex-abspann.tex
%% Gemeinsamer Abspann für Korrekturansicht und Leseansicht.
%% Setzt den Schalter \ifkorrekturansicht voraus (gesetzt in den
%% einbindenden Dateien latex-korrekturansicht-abspann.tex bzw.
%% latex-leseansicht-abspann.tex).
%% ---------------------------------------------------------------

\normalsize

% Das esempio-Environment wird nur in der Leseansicht benötigt
\ifkorrekturansicht\else
\newenvironment{esempio}[3]%
{
    \vspace{1.5ex}
    \rlap{\underline{#1}}
    \par
    \setlength{\parindent}{0cm}
    \nopagebreak
    \leftskip=#2cm
    \rightskip=#3cm
}
{
    \par
}
\fi

\doendnotes{C}
\bigskip
\vfill

\clearpage

\footnotesize

\ifkorrekturansicht
  \lohead{\textsc{register}}
\fi

% theindex-Environment neu definieren ohne reledmac
\makeatletter
\renewenvironment{theindex}{%
  \ifkorrekturansicht
    \section*{\indexname}%
  \else
    \subsubsection*{Index der erwähnten Entitäten}%
  \fi
  \setlength{\parindent}{0pt}%
  \setlength{\parskip}{0pt plus 0.3pt}%
  \let\item\@idxitem
}{%
  \ifkorrekturansicht\clearpage\fi
}
\makeatother

\IfFileExists{\jobname-pw.ind}{\input{\jobname-pw.ind}}{}

% Quellenangabe nur in der Leseansicht
\ifkorrekturansicht\else
% Fallback-Definitionen, falls die .tex-Datei \titel etc. nicht gesetzt hat
\providecommand{\titel}{}
\providecommand{\editorInnen}{}
\providecommand{\dateiname}{\jobname}

\vspace{3cm}

\vfill

\footnotesize
\textsc{Quelle}: \titel. Herausgegeben von {\editorInnen}. In: \emph{Arthur Schnitzler: Briefwechsel mit Autorinnen und Autoren}.
 Digitale Edition, https://schnitzler-briefe.acdh.oeaw.ac.at/{\dateiname}.html (Stand \today)
\fi

\end{document}


