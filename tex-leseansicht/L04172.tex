%% latex-leseansicht-vorspann.tex
%% Vorspann für die Leseansicht.
%% Lädt die gemeinsame Datei latex-vorspann.tex mit nicht gesetztem Schalter.

\newif\ifkorrekturansicht
\korrekturansichtfalse

\input{../tex-inputs/latex-vorspann}


\section[Arthur Schnitzler an Gustav Schwarzkopf, {[}24.? 9. 1897{]}]{L04172 Arthur Schnitzler an Gustav Schwarzkopf, {[}24.? 9. 1897{]}}
\nopagebreak\mylabel{L04172v}
\rehead{ }\normalsize\beginnumbering\briefempfaengerindex{Schwarzkopf, Gustav@\textsc{Schwarzkopf, Gustav}!zzzSchnitzler, Arthur@\emph{von Arthur Schnitzler}!1897-09-241@{{[}24.? 9. 1897{]}}|(be}
\toendnotes[C]{\smallbreak\pagebreak[2]}
\correspDesc{Versand  durch Arthur Schnitzler am [24.? 9. 1897] in Wien
\newline{}Erhalt  durch Gustav Schwarzkopf am [24. 9. 1897?] in Wien}\toendnotes[C]{\smallbreak}
\Standort{CUL, Schnitzler, B 96.}
\physDesc{Brief, 1 Blatt, 2 Seiten, 399 Zeichen
\newline{}Handschrift: Bleistift, deutsche Kurrent}\toendnotes[C]{\smallbreak}
\pstart
           \noindent{}{\pb}Lieber Guſtav, Sie waren geſtern bei mir u haben mich natürlich
               nicht getroffen. Es waren \label{K_L04172-1v}\edtext{entſetzliche Tage, nutzlos entsetzliche – das Kind\pwindex{?? [Totgeborener Sohn von Arthur Schnitzler und Marie Reinhard] 24.\,9.\,1897 Endresstraße 68 – 24.\,9.\,1897 ebd.@\textsc{?? [Totgeborener Sohn von Arthur Schnitzler und Marie Reinhard]} (24.\,9.\,1897 Endresstraße 68 – 24.\,9.\,1897 ebd.)|pwv} iſt während der Geburt
               geſtorben}{\lemma{\textnormal{\emph{entsetzliche … gestorben}}}\Cendnote{\textnormal{Das Korrespondenzstück ist undatiert, kann aber durch den Inhalt 
                  zeitlich verortet werden. Am 24. 9. 1897 kam im Vorort Mauer\oindex{Wien@\textbf{Wien}!XXIII., Liesing@\textbf{XXIII., Liesing}!Mauer@\textbf{Mauer}|pwk} der
                  gemeinsame Sohn\pwindex{?? [Totgeborener Sohn von Arthur Schnitzler und Marie Reinhard] 24.\,9.\,1897 Endresstraße 68 – 24.\,9.\,1897 ebd.@\textsc{?? [Totgeborener Sohn von Arthur Schnitzler und Marie Reinhard]} (24.\,9.\,1897 Endresstraße 68 – 24.\,9.\,1897 ebd.)|pwkv} von Marie Reinhard\pwindex{Reinhard, Marie 13.\,3.\,1871 Wien – 18.\,3.\,1899 ebd.@\textsc{Reinhard, Marie} (13.\,3.\,1871 Wien – 18.\,3.\,1899 ebd.), \emph{Gesangspädagogin}|pwk} und Schnitzler
                  tot  auf die Welt. Die Mutter hatte bis dahin fünf Tage starke Schmerzen
                  durchgemacht. Am Abend des 24. 9. 1897 ging Schnitzler
                  ins Kaffeehaus. Auch wenn es möglich wäre, das Korrespondenzstück später anzusetzen, sprechen mehrere Indizien
                  dagegen. Am Folgetag informierte Schnitzler seine Freunde (XXXX Auszeichnungsfehler: Dokument L00723 nicht gefunden, XXXX Auszeichnungsfehler: Dokument L02965 nicht gefunden) mittels Briefen, so
                  dass spätestens dann ein Korrespondenzstück für Schwarzkopf\pwindex{Schwarzkopf, Gustav 7.\,11.\,1853 Wien – 13.\,11.\,1939 ebd.@\textsc{Schwarzkopf, Gustav} (7.\,11.\,1853 Wien – 13.\,11.\,1939 ebd.), \emph{Schriftsteller}|pwk} zu erwarten wäre. In
                  diesen Schreiben an die Freunde trifft er keine Verabredung für’s Kaffeehaus, was dagegen spricht, dass das vorliegende Schreiben zeitgleich 
                  verfasst wurde.}}}\label{K_L04172-1}. Man {\pb}ahnt gar nicht, wie
               traurig das iſt; beſonders we{\geminationn} ſo gar keine innere
               Notwendigkeit vorliegt, ſondern nur ein unglückſeliges Zuſa{\geminationm}entreffen von Umſtänden.\pend
           \pstart Ihr \spacefill\mbox{Arthur}\pend{}
\pstart
           \noindent{}Ich bin heut nach zehn im \textsc{Arkadencafé\oindex{Wien@\textbf{Wien}!I., Innere Stadt@\textbf{I., Innere Stadt}!Café Arkaden@\textbf{Café Arkaden}, \emph{Kaffeehaus}|pw}}, vielleicht ſeh ich Sie?\pend
           \selectlanguage{ngerman}\endnumbering\briefempfaengerindex{Schwarzkopf, Gustav@\textsc{Schwarzkopf, Gustav}!zzzSchnitzler, Arthur@\emph{von Arthur Schnitzler}!1897-09-241@{{[}24.? 9. 1897{]}}|)be}\mylabel{L04172h}
\begin{anhang}
\end{anhang}\newcommand{\dateiname}{L04172}\newcommand{\titel}{Arthur Schnitzler an Gustav Schwarzkopf, [24.? 9. 1897]}\newcommand{\editorInnen}{Herausgegeben von Jahnke, SelmaMüller, Martin Anton}%% latex-leseansicht-abspann.tex
%% Abspann für die Leseansicht.
%% Der Schalter \ifkorrekturansicht ist bereits durch den Vorspann gesetzt.

%% latex-abspann.tex
%% Gemeinsamer Abspann für Korrekturansicht und Leseansicht.
%% Setzt den Schalter \ifkorrekturansicht voraus (gesetzt in den
%% einbindenden Dateien latex-korrekturansicht-abspann.tex bzw.
%% latex-leseansicht-abspann.tex).
%% ---------------------------------------------------------------

\normalsize

% Das esempio-Environment wird nur in der Leseansicht benötigt
\ifkorrekturansicht\else
\newenvironment{esempio}[3]%
{
    \vspace{1.5ex}
    \rlap{\underline{#1}}
    \par
    \setlength{\parindent}{0cm}
    \nopagebreak
    \leftskip=#2cm
    \rightskip=#3cm
}
{
    \par
}
\fi

\doendnotes{C}
\bigskip
\vfill

\clearpage

\footnotesize

\ifkorrekturansicht
  \lohead{\textsc{register}}
\fi

% theindex-Environment neu definieren ohne reledmac
\makeatletter
\renewenvironment{theindex}{%
  \ifkorrekturansicht
    \section*{\indexname}%
  \else
    \subsubsection*{Index der erwähnten Entitäten}%
  \fi
  \setlength{\parindent}{0pt}%
  \setlength{\parskip}{0pt plus 0.3pt}%
  \let\item\@idxitem
}{%
  \ifkorrekturansicht\clearpage\fi
}
\makeatother

\IfFileExists{\jobname-pw.ind}{\input{\jobname-pw.ind}}{}

% Quellenangabe nur in der Leseansicht
\ifkorrekturansicht\else
% Fallback-Definitionen, falls die .tex-Datei \titel etc. nicht gesetzt hat
\providecommand{\titel}{}
\providecommand{\editorInnen}{}
\providecommand{\dateiname}{\jobname}

\vspace{3cm}

\vfill

\footnotesize
\textsc{Quelle}: \titel. Herausgegeben von {\editorInnen}. In: \emph{Arthur Schnitzler: Briefwechsel mit Autorinnen und Autoren}.
 Digitale Edition, https://schnitzler-briefe.acdh.oeaw.ac.at/{\dateiname}.html (Stand \today)
\fi

\end{document}


