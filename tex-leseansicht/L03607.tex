%% latex-leseansicht-vorspann.tex
%% Vorspann für die Leseansicht.
%% Lädt die gemeinsame Datei latex-vorspann.tex mit nicht gesetztem Schalter.

\newif\ifkorrekturansicht
\korrekturansichtfalse

\input{../tex-inputs/latex-vorspann}

\begin{center}
            \textcolor{red}{ENTWURF, NICHT FERTIG KORRIGIERT}
                      \end{center}
            
         
         \renewcommand{\erwaehntePersonen}{Personen: Felix Salten}
         \renewcommand{\erwaehnteInstitutionen}{Institutionen: S. Fischer Verlag}
         \renewcommand{\erwaehnteOrte}{Orte: Berlin}
         \renewcommand{\erwaehnteWerke}{Werke: Börsenblatt für den Deutschen Buchhandel, Der Ruf des Lebens. Schauspiel in drei Akten}
               \section[Arthur Schnitzler: Widmungsexemplar Der Ruf des Lebens für Felix Salten, {[}zwischen 4.–7.?{]} 2. 1906]{ Arthur Schnitzler: Widmungsexemplar Der Ruf des Lebens für Felix
               Salten, {[}zwischen 4.–7.?{]} 2. 1906}\nopagebreak\mylabel{v}\rehead{ }\begin{ledgroupsized}[t]{13cm}\normalsize\beginnumbering \toendnotes[C]{\smallbreak\pagebreak[2]} \Standort{Wienbibliothek im Rathaus, A-124219/2.Ex., DS-2019-711.}
\physDesc{Widmung am Vorsatzblatt, 55 Zeichen
\newline{}Handschrift: schwarze Tinte, deutsche Kurrent
\newline{}Ordnung: mit schwarzer Tinte ausgefüllter Stempel: »\noindent{}\textcolor{gray}{\textbf{\textit{Felix Salten}}}{ / }\textcolor{gray}{\textbf{\textit{Inv. Nr.}}}{ }4478{ / }\textcolor{gray}{\textbf{\textit{Werk Nr.}}}{ }2206{ / }\textcolor{gray}{\textbf{\textit{Schrank}}}{ }XIV A. Z. \textcolor{gray}{\textbf{\textit{Fach}}} b« }\toendnotes[C]{\smallbreak}\pstart
           \noindent{}{\pb}Meinem lieben Felix Salten\pend
           \pstart \spacefill\mbox{Arthur Sch}\pend{}\pstart
           Berlin\oindex{Berlin@\textbf{Berlin}|pw}{ }\label{K_L03607-1v}\edtext{Feber 906}{\lemma{\textnormal{\emph{Feber 906}}}\Cendnote{\textnormal{\emph{Der Ruf des Lebens}\pwindex{Schnitzler, Arthur 15.05.1862 – 21.10.1931@\textsc{Schnitzler, Arthur} (15.05.1862 – 21.10.1931), \emph{Schriftsteller, Mediziner}!Ruf des Lebens. Schauspiel in drei Akten1906-02-20@\strich\emph{Der Ruf des Lebens. Schauspiel in drei Akten} {[}1906-02-20{]}|pwk} wurde am 13. 2. 1906 vom \emph{Börsenblatt für den deutschen Buchhandel}\pwindex{?? Werk@Nicht ermittelte Verfasserinnen und Verfasser!Boersenblatt fuer den Deutschen Buchhandel1843-01-03@\emph{Börsenblatt für den Deutschen Buchhandel} {[}1843-01-03{]}|pwk} als Neuerscheinung gemeldet.
                        Schnitzler\pwindex{Schnitzler, Arthur 15.05.1862 – 21.10.1931@\textsc{Schnitzler, Arthur} (15.05.1862 – 21.10.1931), \emph{Schriftsteller, Mediziner}|pwk} war von 4. 2. 1906 bis
                        7. 2. 1906
                     und von 18. 2. 1906 bis 27. 2. 1906 in Berlin\oindex{Berlin@\textbf{Berlin}|pwk}. Es ist anzunehmen, dass er
                     die Widmungen während der ersten Reise verfasste.}}}\label{K_L03607-1h}.\pend
           {\bigskip}\pstart
           \noindent{}\centering{}{\pb}\textcolor{gray}{\textbf{\textbf{\so{Der Ruf des Lebens}\pwindex{Schnitzler, Arthur 15.05.1862 – 21.10.1931@\textsc{Schnitzler, Arthur} (15.05.1862 – 21.10.1931), \emph{Schriftsteller, Mediziner}!Ruf des Lebens. Schauspiel in drei Akten1906-02-20@\strich\emph{Der Ruf des Lebens. Schauspiel in drei Akten} {[}1906-02-20{]}|pw}}}}\pend
           \pstart
           \noindent{}\centering{}\textcolor{gray}{\textbf{\so{Schauſpiel in drei Akten von}}}{\\}\textcolor{gray}{\textbf{\textbf{\so{Arthur Schnitzler}}}}\pend
           {\bigskip}\pstart
           \noindent{}\centering{}\textcolor{gray}{\textbf{\so{S. Fiſcher, Verlag}\orgindex{S. Fischer Verlag@S. Fischer Verlag|pw}\so{,{ }}\so{Berlin}\oindex{Berlin@\textbf{Berlin}|pw}}}\pend
           \pstart
           \noindent{}\centering{}\textcolor{gray}{\textbf{\so{1906}}}\pend
           
         
         \endnumbering\mylabel{h}\end{ledgroupsized}  \newcommand{\dateiname}{L03607}\newcommand{\titel}{Arthur Schnitzler: Widmungsexemplar Der Ruf des Lebens für Felix Salten, [zwischen 4.–7.?] 2. 1906}\newcommand{\editorInnen}{Martin Anton Müller und Laura Untner}%% latex-leseansicht-abspann.tex
%% Abspann für die Leseansicht.
%% Der Schalter \ifkorrekturansicht ist bereits durch den Vorspann gesetzt.

%% latex-abspann.tex
%% Gemeinsamer Abspann für Korrekturansicht und Leseansicht.
%% Setzt den Schalter \ifkorrekturansicht voraus (gesetzt in den
%% einbindenden Dateien latex-korrekturansicht-abspann.tex bzw.
%% latex-leseansicht-abspann.tex).
%% ---------------------------------------------------------------

\normalsize

% Das esempio-Environment wird nur in der Leseansicht benötigt
\ifkorrekturansicht\else
\newenvironment{esempio}[3]%
{
    \vspace{1.5ex}
    \rlap{\underline{#1}}
    \par
    \setlength{\parindent}{0cm}
    \nopagebreak
    \leftskip=#2cm
    \rightskip=#3cm
}
{
    \par
}
\fi

\doendnotes{C}
\bigskip
\vfill

\clearpage

\footnotesize

\ifkorrekturansicht
  \lohead{\textsc{register}}
\fi

% theindex-Environment neu definieren ohne reledmac
\makeatletter
\renewenvironment{theindex}{%
  \ifkorrekturansicht
    \section*{\indexname}%
  \else
    \subsubsection*{Index der erwähnten Entitäten}%
  \fi
  \setlength{\parindent}{0pt}%
  \setlength{\parskip}{0pt plus 0.3pt}%
  \let\item\@idxitem
}{%
  \ifkorrekturansicht\clearpage\fi
}
\makeatother

\IfFileExists{\jobname-pw.ind}{\input{\jobname-pw.ind}}{}

% Quellenangabe nur in der Leseansicht
\ifkorrekturansicht\else
% Fallback-Definitionen, falls die .tex-Datei \titel etc. nicht gesetzt hat
\providecommand{\titel}{}
\providecommand{\editorInnen}{}
\providecommand{\dateiname}{\jobname}

\vspace{3cm}

\vfill

\footnotesize
\textsc{Quelle}: \titel. Herausgegeben von {\editorInnen}. In: \emph{Arthur Schnitzler: Briefwechsel mit Autorinnen und Autoren}.
 Digitale Edition, https://schnitzler-briefe.acdh.oeaw.ac.at/{\dateiname}.html (Stand \today)
\fi

\end{document}


      