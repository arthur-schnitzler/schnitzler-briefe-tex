%% latex-korrekturansicht-vorspann.tex
%% Vorspann für die Korrekturansicht.
%% Lädt die gemeinsame Datei latex-vorspann.tex mit gesetztem Schalter.

\newif\ifkorrekturansicht
\korrekturansichttrue

\input{../tex-inputs/latex-vorspann}


\section[Arthur Schnitzler: Widmungsexemplar Der Ruf des Lebens für Felix Salten, {[}21.?{]} 2. 1906]{L03607 Arthur Schnitzler: Widmungsexemplar Der Ruf des Lebens für Felix
               Salten, {[}21.?{]} 2. 1906}
\nopagebreak\mylabel{L03607v}
\rehead{ }\normalsize\beginnumbering\briefempfaengerindex{Salten, Felix@\textsc{Salten, Felix}!zzzSchnitzler, Arthur@\emph{von Arthur Schnitzler}!1906-02-212@{{[}21.?{]} 2. 1906}|(be}
\toendnotes[C]{\smallbreak\pagebreak[2]}\Standort{Wienbibliothek im Rathaus, A-124219/2.Ex., DS-2019-711.}
\physDesc{Widmung am Vorsatzblatt, 55 Zeichen
\newline{}Handschrift: schwarze Tinte, deutsche Kurrent
\newline{}Ordnung: mit schwarzer Tinte ausgefüllter Stempel: »\noindent{}\textcolor{gray}{\textbf{\textit{Felix Salten}}}{ / }\textcolor{gray}{\textbf{\textit{Inv. Nr.}}}{ }4478{ / }\textcolor{gray}{\textbf{\textit{Werk Nr.}}}{ }2206{ / }\textcolor{gray}{\textbf{\textit{Schrank}}}{ }XIV A. Z. \textcolor{gray}{\textbf{\textit{Fach}}} b« }\toendnotes[C]{\smallbreak}
\pstart
           \noindent{}{\pb}Meinem lieben Felix Salten\pend
           \pstart \spacefill\mbox{Arthur Sch}\pend{}
\pstart
           Berlin\oindex{Berlin@\textbf{Berlin}, \emph{P.PPLC}|pw}{ }\label{K_L03607-1v}\edtext{Feber 906}{\lemma{\textnormal{\emph{Feber 906}}}\Cendnote{\textnormal{\emph{Der Ruf des Lebens}\pwindex{Ruf des Lebens. Schauspiel in drei Akten@\emph{Der Ruf des Lebens. Schauspiel in drei Akten}|pwk} wurde am 13. 2. 1906 vom \emph{Börsenblatt für den deutschen Buchhandel}\pwindex{Boersenblatt fuer den Deutschen Buchhandel@\emph{Börsenblatt für den Deutschen Buchhandel}|pwk} als Neuerscheinung gemeldet.
                     Am 21. 2. 1906 war Schnitzler in Berlin\oindex{Berlin@\textbf{Berlin}, \emph{P.PPLC}|pwk} und vermerkte
                     sich im  \emph{Tagebuch}\pwindex{Tagebuch@\emph{Tagebuch}|pwk}: »Bei S. Fischer\pwindex{Fischer, Samuel 24.12.1859 – 15.10.1934@\textsc{Fischer, Samuel} (24.12.1859 – 15.10.1934), \emph{Verleger/Verlegerin}|pw} (Bücher verschickt).«}}}\label{K_L03607-1}.\pend
           \selectlanguage{ngerman}\vspace{1em}{\vspace{1\baselineskip}}
\pstart
           \centering{}{\pb}\textcolor{gray}{\textbf{\textbf{\so{Der Ruf des Lebens}\pwindex{Ruf des Lebens. Schauspiel in drei Akten@\emph{Der Ruf des Lebens. Schauspiel in drei Akten}|pw}}}}\pend
           
\pstart
           \centering{}\textcolor{gray}{\textbf{\so{Schauſpiel in drei Akten von}}}{\\}\textcolor{gray}{\textbf{\textbf{\so{Arthur Schnitzler}}}}\pend
           {\vspace{1\baselineskip}}
\pstart
           \centering{}\textcolor{gray}{\textbf{\so{S. Fiſcher, Verlag}\orgindex{S. Fischer Verlag@S. Fischer Verlag|pw}\so{,{ }}\so{Berlin}\oindex{Berlin@\textbf{Berlin}, \emph{P.PPLC}|pw}}}\pend
           
\pstart
           \centering{}\textcolor{gray}{\textbf{\so{1906}}}\pend
           \selectlanguage{ngerman}\endnumbering\briefempfaengerindex{Salten, Felix@\textsc{Salten, Felix}!zzzSchnitzler, Arthur@\emph{von Arthur Schnitzler}!1906-02-212@{{[}21.?{]} 2. 1906}|)be}\mylabel{L03607h}  \normalsize

\doendnotes{C}
\bigskip
\vfill

\clearpage

\footnotesize

\lohead{\textsc{register}}

% Definiere theindex-Environment komplett neu ohne reledmac
\makeatletter
\renewenvironment{theindex}{%
  \section*{\indexname}%
  \setlength{\parindent}{0pt}%
  \setlength{\parskip}{0pt plus 0.3pt}%
  \let\item\@idxitem
}{%
  \clearpage
}
\makeatother

\IfFileExists{\jobname-pw.ind}{\input{\jobname-pw.ind}}{}

\end{document}

      