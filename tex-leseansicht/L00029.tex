%% latex-korrekturansicht-vorspann.tex
%% Vorspann für die Korrekturansicht.
%% Lädt die gemeinsame Datei latex-vorspann.tex mit gesetztem Schalter.

\newif\ifkorrekturansicht
\korrekturansichttrue

\input{../tex-inputs/latex-vorspann}


\section[Arthur Schnitzler an Richard Beer-Hofmann, 11. 8. 1891]{L00029 Arthur Schnitzler an Richard Beer-Hofmann,11. 8. 1891}
\nopagebreak\mylabel{L00029v}
\rehead{ }\normalsize\beginnumbering\briefempfaengerindex{Beer-Hofmann, Richard@\textsc{Beer-Hofmann, Richard}!zzzSchnitzler, Arthur@\emph{von Arthur Schnitzler}!1891-08-112@{11. 8. 1891}|(be}
\toendnotes[C]{\smallbreak\pagebreak[2]}\Standort{YCGL, MSS 31.}
\physDesc{Brief, 1 Blatt, 2 Seiten, 467 Zeichen
\newline{}Handschrift: 1) schwarze Tinte, deutsche Kurrent\hspace{1em}2) schwarze Tinte, lateinische Kurrent (\noindent{}Adresse)\hspace{1em}
\newline{}Versand: 1) Stempel: »\nobreak{}Wien, 11 \textcolor{gray}{8}{[}1891{]}, 4.N\nobreak{}«.   2) Stempel: »\nobreak{}\oindex{Bad Aussee@\textbf{Bad Aussee}|pwk}{\pb}Aussee in
                                          Stei\textcolor{gray}{ermark}, 12. {[}8.{]} 9\textcolor{gray}{1}\nobreak{}«. }
\buchAbdrucke{\weitereDrucke{Arthur Schnitzler, Richard Beer-Hofmann: \emph{Briefwechsel 1891–1931}. Wien, Zürich: \emph{Europaverlag} 1992, S. 31.} }\toendnotes[C]{\smallbreak}\pstart{}{\pb}Herrn Dr. Rich. Beer-Hofmann\pend{}\pstart{}Aussee\oindex{Bad Aussee@\textbf{Bad Aussee}|pw}\pend{}\pstart{}Steiermark\oindex{Steiermark@\textbf{Steiermark}|pw}\pend{}{\bigskip}\vspace{1em}
\pstart
           \raggedleft{}{\pb}11. Aug 91.\pend
           \vspace{0.5em}
\pstart
           Daß Sie mir noch nicht eine Zeile geſchrieben haben – na reden wir nicht drüber!
               Alſo, mein lieber, ich bin \uline{wahrscheinlich} die \label{K_L00029-1v}\edtext{2 Feiertage}{\lemma{\textnormal{\emph{2 Feiertage}}}\Cendnote{\textnormal{Der
                     15. 8. 1891 – Mariä Himmelfahrt –, war ein Samstag.
                     Dienstag, der 18. 8. war Geburtstag des Kaisers Franz Joseph\pwindex{Franz Joseph I. von Oesterreich-Ungarn 18.\,8.\,1830 Wien – 21.\,11.\,1916 ebd.@\textsc{Franz Joseph I. von Österreich-Ungarn} (18.\,8.\,1830 Wien – 21.\,11.\,1916 ebd.), \emph{Kaiser}|pwk}.}}}\label{K_L00029-1} in Iſchl\oindex{Bad Ischl@\textbf{Bad Ischl}|pw}. Es wäre wunderſchön, we{\geminationn} wir uns da
               begegneten. Ich habe auch an \textsc{Loris}\pwindex{Hofmannsthal, Hugo von 1.\,2.\,1874 Wien – 15.\,7.\,1929 Rodaun@\textsc{Hofmannsthal, Hugo von} (1.\,2.\,1874 Wien – 15.\,7.\,1929 Rodaun), \emph{Schriftsteller}|pw} nach {\pb}\textsc{Strobl}\oindex{Strobl@\textbf{Strobl}|pw} geſchrieben. Theilen Sie mir nur mit, ob Sie überhaupt zu erreichen{ }ſind, ob
               Sie nach Iſchl\oindex{Bad Ischl@\textbf{Bad Ischl}|pw} kommen wollen \textsc{etc. etc.} –\pend
           
\pstart
           Es geht Ihnen doch{ }ſo gut wie ichs Ihnen wünſche?\pend
           
\pstart
           Herzlichen Gruſs.{\\[\baselineskip]}Ihr\spacefill\mbox{Arthur.}\pend
           \leftskip=0em{}\selectlanguage{ngerman}\endnumbering\briefempfaengerindex{Beer-Hofmann, Richard@\textsc{Beer-Hofmann, Richard}!zzzSchnitzler, Arthur@\emph{von Arthur Schnitzler}!1891-08-112@{11. 8. 1891}|)be}\mylabel{L00029h}  \normalsize

\doendnotes{C}
\bigskip
\vfill

\clearpage

\footnotesize

\lohead{\textsc{register}}

% Definiere theindex-Environment komplett neu ohne reledmac
\makeatletter
\renewenvironment{theindex}{%
  \section*{\indexname}%
  \setlength{\parindent}{0pt}%
  \setlength{\parskip}{0pt plus 0.3pt}%
  \let\item\@idxitem
}{%
  \clearpage
}
\makeatother

\IfFileExists{\jobname-pw.ind}{\input{\jobname-pw.ind}}{}

\end{document}

      