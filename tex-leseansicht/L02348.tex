\input{../tex-inputs/latex-pdf-vorspann}
\begin{center}
            \textcolor{red}{ENTWURF. ENTZIFFERUNG NOCH NICHT KORREKTURGELESEN}
                      \end{center}
            
               \section[Richard Beer-Hofmann an Arthur Schnitzler, 12. 7. 1920]{ Richard Beer-Hofmann an Arthur Schnitzler,
               12. 7. 1920}\nopagebreak\mylabel{v}\rehead{ }\begin{ledgroupsized}[t]{13cm}\normalsize\beginnumbering\briefempfaengerindex{Schnitzler, Arthur@\textsc{Schnitzler, Arthur}!zzzBeer-Hofmann, Richard@\emph{von Richard Beer-Hofmann}!1920-07-121@{12. 7. 1920}|(be} \toendnotes[C]{\smallbreak\pagebreak[2]} \Standort{CUL, Schnitzler, B 8.}
\physDesc{Brief, 1 Blatt, 2 Seiten
\newline{}Handschrift: Bleistift, lateinische Kurrent\newline{}Ordnung: mit Bleistift von unbekannter Hand nummeriert: »270« }\buchAbdrucke{\weitereDrucke{Arthur Schnitzler, Richard Beer-Hofmann: \emph{Briefwechsel 1891–1931}. Hg. Konstanze Fliedl. Wien, Zürich: \emph{Europaverlag} 1992, S. 227.} }\pstart
           {\pb}Bad Aussee\oindex{Bad Aussee@\textbf{Bad Aussee}|pw}{ }12. VII.  20\pend
           \pstart
           Lieber Arthur! Eben erhalte ich von S. Fischer\pwindex{Fischer, Samuel 24.12.1859 – 15.10.1934@\textsc{Fischer, Samuel} (24.12.1859 – 15.10.1934), \emph{Verleger}|pw} die Mitteilung von einem 25 {\%}
               Teuerungszuschlag – der »\uline{tantièmenfrei}« sein soll.
               Wie stellen Sie sich dazu? Wie Hugo\pwindex{Hofmannsthal, Hugo von 01.02.1874 – 15.07.1929@\textsc{Hofmannsthal, Hugo von} (01.02.1874 – 15.07.1929), \emph{Schriftsteller}|pw}, der ja noch
               in Wien\oindex{Wien@\textbf{Wien}|pw} ist. Bitte schreiben Sie mir zwei Zeilen was
               Sie tun. Ich finde es unerhört! Tatsächlich \strikeout{tra} beko{\geminationm}t der Autor 15
               od. 16 {\%} des Ladenpreises der Sortimenter mindestens 50 wozu
               noch sein privater {\pb}25 {\%} Teuerungszuschlag ko{\geminationm}t. Muss
               man sich das gefallen lassen?\pend
           \pstart
           Herzlichst Ihr{\\[\baselineskip]}\spacefill\mbox{Richard}\pend
           \leftskip=0em{}\endnumbering\briefempfaengerindex{Schnitzler, Arthur@\textsc{Schnitzler, Arthur}!zzzBeer-Hofmann, Richard@\emph{von Richard Beer-Hofmann}!1920-07-121@{12. 7. 1920}|)be}\mylabel{h}\end{ledgroupsized}  \newcommand{\dateiname}{L02348}\newcommand{\titel}{Richard Beer-Hofmann an Arthur Schnitzler, 12. 7. 1920}\newcommand{\editorInnen}{Martin Anton Müller und Gerd-Hermann Susen}\input{../tex-inputs/latex-pdf-abspann}
      