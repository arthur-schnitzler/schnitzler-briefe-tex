%% latex-leseansicht-vorspann.tex
%% Vorspann für die Leseansicht.
%% Lädt die gemeinsame Datei latex-vorspann.tex mit nicht gesetztem Schalter.

\newif\ifkorrekturansicht
\korrekturansichtfalse

\input{../tex-inputs/latex-vorspann}


\section[Olga Schnitzler an Richard und Paula Beer-Hofmann, {[}9. 6. 1909?{]}]{L02549 Olga Schnitzler an Richard und Paula Beer-Hofmann, {[}9. 6. 1909?{]}}
\nopagebreak\mylabel{L02549v}
\rehead{ }\normalsize\beginnumbering\briefempfaengerindex{Beer-Hofmann, Paula@\textsc{Beer-Hofmann, Paula}!zzzSchnitzler, Olga@\emph{von Olga Schnitzler}!1909-06-091@{{[}9. 6. 1909?{]}}|(be}\briefempfaengerindex{Beer-Hofmann, Richard@\textsc{Beer-Hofmann, Richard}!zzzSchnitzler, Olga@\emph{von Olga Schnitzler}!1909-06-091@{{[}9. 6. 1909?{]}}|(be}
\toendnotes[C]{\smallbreak\pagebreak[2]}
\correspDesc{Versand  durch Olga Schnitzler am [9. 6. 1909?] in Wien
\newline{}Erhalt  durch Richard Beer-Hofmann, Paula Beer-Hofmann am [9. 6. 1909?] in Wien}\toendnotes[C]{\smallbreak}
\Standort{YCGL, MSS 31.}
\physDesc{Brief, 1 Blatt, 1 Seite, Kuvert, 344 Zeichen
\newline{}Handschrift: schwarze Tinte, lateinische Kurrent
\newline{}Versand: ohne postalischen Übermittlungsvermerk }\toendnotes[C]{\smallbreak}\pstart{}{\pb}\textcolor{gray}{\textbf{O. S.}}\pend{}{\bigskip}\pstart{}{\pb}Herrn u. Frau D\textsuperscript{r}
                  Richard Beer-Hofmann\pend{}{\bigskip}\vspace{1em}
\pstart
           {\pb}\textcolor{gray}{\textbf{O. S.}}\pend
           \vspace{0.5em}
\pstart
           Meine Lieben, der Kapellmeister Walter\pwindex{Walter, Bruno 15.\,9.\,1876 Berlin – 17.\,2.\,1962 Beverly Hills@\textsc{Walter, Bruno} (15.\,9.\,1876 Berlin – 17.\,2.\,1962 Beverly Hills), \emph{Theaterleiter, Komponist, Dirigent}|pw} hat sich für \label{K_L02549-1v}\edtext{morgen
                  Donnerstag}{\lemma{\textnormal{\emph{morgen
                  Donnerstag}}}\Cendnote{\textnormal{Die Datierung ergibt sich
                  aus dem Tagebucheintrag vom 10. 6. 1909.}}}\label{K_L02549-1}{ }½ 8 Uhr Abend bei uns angesagt, wir bitten Euch, ebenfalls zu
               kommen.\pend
           
\pstart
           Ihr werdet einen wirklich aussergewöhnlichen und sehr lieben Menschen kennen
               lernen\pend
           
\pstart
           Seine Frau\pwindex{Walter, Elsa 1871 – 26.\,3.\,1945@\textsc{Walter, Elsa} (1871 – 26.\,3.\,1945), \emph{Sängerin}|pwv} ist schon
               verreist, \label{K_L02549-2v}\edtext{tan mieux}{\lemma{\textnormal{\emph{tan mieux}}}\Cendnote{\textnormal{französisch \begin{otherlanguage}{french}tant mieux\end{otherlanguage}:
                   so viel besser}}}\label{K_L02549-2}. Wir rechnen bestimmt auf Euch.\pend
           \pstart Herzliche Grüsse \spacefill\mbox{OlgaS.}\pend{}\selectlanguage{ngerman}\endnumbering\briefempfaengerindex{Beer-Hofmann, Paula@\textsc{Beer-Hofmann, Paula}!zzzSchnitzler, Olga@\emph{von Olga Schnitzler}!1909-06-091@{{[}9. 6. 1909?{]}}|)be}\briefempfaengerindex{Beer-Hofmann, Richard@\textsc{Beer-Hofmann, Richard}!zzzSchnitzler, Olga@\emph{von Olga Schnitzler}!1909-06-091@{{[}9. 6. 1909?{]}}|)be}\mylabel{L02549h}  \newcommand{\dateiname}{L02549}\newcommand{\titel}{Olga Schnitzler an Richard und Paula Beer-Hofmann, [9. 6. 1909?]}\newcommand{\editorInnen}{Martin Anton Müller und Gerd-Hermann Susen}%% latex-leseansicht-abspann.tex
%% Abspann für die Leseansicht.
%% Der Schalter \ifkorrekturansicht ist bereits durch den Vorspann gesetzt.

%% latex-abspann.tex
%% Gemeinsamer Abspann für Korrekturansicht und Leseansicht.
%% Setzt den Schalter \ifkorrekturansicht voraus (gesetzt in den
%% einbindenden Dateien latex-korrekturansicht-abspann.tex bzw.
%% latex-leseansicht-abspann.tex).
%% ---------------------------------------------------------------

\normalsize

% Das esempio-Environment wird nur in der Leseansicht benötigt
\ifkorrekturansicht\else
\newenvironment{esempio}[3]%
{
    \vspace{1.5ex}
    \rlap{\underline{#1}}
    \par
    \setlength{\parindent}{0cm}
    \nopagebreak
    \leftskip=#2cm
    \rightskip=#3cm
}
{
    \par
}
\fi

\doendnotes{C}
\bigskip
\vfill

\clearpage

\footnotesize

\ifkorrekturansicht
  \lohead{\textsc{register}}
\fi

% theindex-Environment neu definieren ohne reledmac
\makeatletter
\renewenvironment{theindex}{%
  \ifkorrekturansicht
    \section*{\indexname}%
  \else
    \subsubsection*{Index der erwähnten Entitäten}%
  \fi
  \setlength{\parindent}{0pt}%
  \setlength{\parskip}{0pt plus 0.3pt}%
  \let\item\@idxitem
}{%
  \ifkorrekturansicht\clearpage\fi
}
\makeatother

\IfFileExists{\jobname-pw.ind}{\input{\jobname-pw.ind}}{}

% Quellenangabe nur in der Leseansicht
\ifkorrekturansicht\else
% Fallback-Definitionen, falls die .tex-Datei \titel etc. nicht gesetzt hat
\providecommand{\titel}{}
\providecommand{\editorInnen}{}
\providecommand{\dateiname}{\jobname}

\vspace{3cm}

\vfill

\footnotesize
\textsc{Quelle}: \titel. Herausgegeben von {\editorInnen}. In: \emph{Arthur Schnitzler: Briefwechsel mit Autorinnen und Autoren}.
 Digitale Edition, https://schnitzler-briefe.acdh.oeaw.ac.at/{\dateiname}.html (Stand \today)
\fi

\end{document}


