%% latex-leseansicht-vorspann.tex
%% Vorspann für die Leseansicht.
%% Lädt die gemeinsame Datei latex-vorspann.tex mit nicht gesetztem Schalter.

\newif\ifkorrekturansicht
\korrekturansichtfalse

\input{../tex-inputs/latex-vorspann}


\section[Arthur Schnitzler an Robert Adam, 31. 10. 1910]{L01974 Arthur Schnitzler an Robert Adam, 31. 10. 1910}
\nopagebreak\mylabel{L01974v}
\rehead{ }\normalsize\beginnumbering\briefempfaengerindex{Adam, Robert@\textsc{Adam, Robert}!zzzSchnitzler, Arthur@\emph{von Arthur Schnitzler}!1910-10-311@{31. 10. 1910}|(be}
\toendnotes[C]{\smallbreak\pagebreak[2]}
\correspDesc{Versand  durch Arthur Schnitzler am 31. 10. 1910 in Wien
\newline{}Erhalt  durch Robert Adam im Zeitraum [31. 10. 1910 – 4. 11. 1910?] in Wien}\toendnotes[C]{\smallbreak}
\Standort{DLA, 96.34.1/2.}
\physDesc{Brief, 1 Blatt, 1 Seite, Kuvert, 434 Zeichen
\newline{}Schreibmaschine
\newline{}Handschrift: schwarze Tinte (\noindent{}Unterschrift)
\newline{}Versand: Stempel: »\nobreak{}\oindex{IX., Alsergrund@\textbf{IX., Alsergrund}, \emph{Verwaltungsgebiet}|pwk}9/1 Wien 66, 31. X. 10, 3\nobreak{}«.  }\Standort{DLA, A:Schnitzler, 85.1.1621.}
\physDesc{Brief, Durchschlag, 1 Blatt, 1 Seite, Kuvert, 434 Zeichen
\newline{}Schreibmaschine
\newline{}Handschrift: Bleistift, lateinische Kurrent (\noindent{}»Adam«)}\toendnotes[C]{\smallbreak}\pstart{}{\pb}\textcolor{gray}{\textbf{Dr. Arthur Schnitzler}}\pend{}\pstart{}\textcolor{gray}{\textbf{Wien, XVIII. Sternwartestrasse 71}}\oindex{Wien@\textbf{Wien}!XVIII., Währing@\textbf{XVIII., Währing}!Sternwartestraße 71@\textbf{Sternwartestraße 71}, \emph{Wohngebäude}|pw}\pend{}{\bigskip}\pstart{}{\pb}Herrn Robert Adam\pend{}\pstart{}\so{Wien XII}\oindex{XII., Meidling@\textbf{XII., Meidling}, \emph{Verwaltungsgebiet}|pw}.\pend{}\pstart{}Meidlinger Hauptstraße 56\oindex{Wien@\textbf{Wien}!XII., Meidling@\textbf{XII., Meidling}!Meidlinger Hauptstraße@\textbf{Meidlinger Hauptstraße}, \emph{Straße}|pw}.\pend{}{\bigskip}\vspace{1em}
\pstart
           
\pstart
           {\pb}\textcolor{gray}{\textbf{Dr. Arthur Schnitzler}}\pend
           
\pstart
           \raggedleft{}31. 10. 1910.\pend
           \pend
           
\pstart
           \textcolor{gray}{\textbf{Wien XVIII. Sternwartestrasse 71\oindex{Wien@\textbf{Wien}!XVIII., Währing@\textbf{XVIII., Währing}!Sternwartestraße 71@\textbf{Sternwartestraße 71}, \emph{Wohngebäude}|pw}}}\pend
           
\pstart\center{}Sehr geehrter Herr Adam.\pend\vspace{0.5em}
\pstart
           Ihrer \introOben{}»\introOben{}Ali Bekkar\introOben{}«\introOben{}-Komödie\pwindex{Adam, Robert 20.\,4.\,1877 Wien – 16.\,10.\,1961 Baden bei Wien@\textsc{Adam, Robert} (20.\,4.\,1877 Wien – 16.\,10.\,1961 Baden bei Wien), \emph{Schriftsteller, Richter}!Geschichte des Alî ibn Bekkâr mit Schams an-Nahâr@\strich\emph{Die Geschichte des Alî ibn Bekkâr mit Schams an-Nahâr}|pw}
               erinnere ich gern und so interessiert es mich natürlich auch Ihr neues Stück\pwindex{Adam, Robert 20.\,4.\,1877 Wien – 16.\,10.\,1961 Baden bei Wien@\textsc{Adam, Robert} (20.\,4.\,1877 Wien – 16.\,10.\,1961 Baden bei Wien), \emph{Schriftsteller, Richter}!Neidhard@\strich\emph{Neidhard}|pwv} kennen zu lernen, nur
               möchte bitten 1. es mir in Schreibmaschinenschrift zu\introOben{}{ }\introOben{}senden zu wollen und 2. sich fre{[}u{]}ndlichst bis nach meiner
                  \label{K_L01974-1v}\edtext{Burgtheater\oindex{Wien@\textbf{Wien}!I., Innere Stadt@\textbf{I., Innere Stadt}!Burgtheater@\textbf{Burgtheater}, \emph{Theater}|pw}première\eventindex{Lessing-Theater@\textbf{Lessing-Theater}!Premiere von Der junge Medardus, 24.10.1914@Premiere von Der junge Medardus, 24.10.1914|pwv}}{\lemma{\textnormal{\emph{Burgtheaterpremière}}}\Cendnote{\textnormal{Am 24. 11. 1910 fand die Uraufführung\eventindex{Lessing-Theater@\textbf{Lessing-Theater}!Premiere von Der junge Medardus, 24.10.1914@Premiere von Der junge Medardus, 24.10.1914|pwkv} von \emph{Der junge Medardus}\pwindex{Schnitzler, Arthur 15.\,5.\,1862 Wien – 21.\,10.\,1931 ebd.@\textsc{Schnitzler, Arthur} (15.\,5.\,1862 Wien – 21.\,10.\,1931 ebd.), \emph{Schriftsteller, Mediziner}!junge Medardus. Dramatische Historie in einem Vorspiel und fünf Aufzügen@\strich\emph{Der junge Medardus. Dramatische Historie in einem Vorspiel und fünf Aufzügen}|pwk} am \emph{Burgtheater}\orgindex{Burgtheater@Burgtheater|pwk} statt.
               }}}\label{K_L01974-1} zu gedulden.\pend
           
\pstart
           Mit verbindlichem Gruss{\\[\baselineskip]}Ihr sehr ergebener\spacefill\mbox{{[}hs.:{]} Arthur Schnitzler}\pend
           \leftskip=0em{}
\pstart
           \noindent{}{[}ms.:{]} Herrn Robert Adam, Wien\oindex{Wien@\textbf{Wien}, \emph{Verwaltungsgebiet}|pw}.\pend
           \selectlanguage{ngerman}\endnumbering\briefempfaengerindex{Adam, Robert@\textsc{Adam, Robert}!zzzSchnitzler, Arthur@\emph{von Arthur Schnitzler}!1910-10-311@{31. 10. 1910}|)be}\mylabel{L01974h}  \newcommand{\dateiname}{L01974}\newcommand{\titel}{Arthur Schnitzler an Robert Adam, 31. 10. 1910}\newcommand{\editorInnen}{Martin Anton Müller und Gerd-Hermann Susen}%% latex-leseansicht-abspann.tex
%% Abspann für die Leseansicht.
%% Der Schalter \ifkorrekturansicht ist bereits durch den Vorspann gesetzt.

%% latex-abspann.tex
%% Gemeinsamer Abspann für Korrekturansicht und Leseansicht.
%% Setzt den Schalter \ifkorrekturansicht voraus (gesetzt in den
%% einbindenden Dateien latex-korrekturansicht-abspann.tex bzw.
%% latex-leseansicht-abspann.tex).
%% ---------------------------------------------------------------

\normalsize

% Das esempio-Environment wird nur in der Leseansicht benötigt
\ifkorrekturansicht\else
\newenvironment{esempio}[3]%
{
    \vspace{1.5ex}
    \rlap{\underline{#1}}
    \par
    \setlength{\parindent}{0cm}
    \nopagebreak
    \leftskip=#2cm
    \rightskip=#3cm
}
{
    \par
}
\fi

\doendnotes{C}
\bigskip
\vfill

\clearpage

\footnotesize

\ifkorrekturansicht
  \lohead{\textsc{register}}
\fi

% theindex-Environment neu definieren ohne reledmac
\makeatletter
\renewenvironment{theindex}{%
  \ifkorrekturansicht
    \section*{\indexname}%
  \else
    \subsubsection*{Index der erwähnten Entitäten}%
  \fi
  \setlength{\parindent}{0pt}%
  \setlength{\parskip}{0pt plus 0.3pt}%
  \let\item\@idxitem
}{%
  \ifkorrekturansicht\clearpage\fi
}
\makeatother

\IfFileExists{\jobname-pw.ind}{\input{\jobname-pw.ind}}{}

% Quellenangabe nur in der Leseansicht
\ifkorrekturansicht\else
% Fallback-Definitionen, falls die .tex-Datei \titel etc. nicht gesetzt hat
\providecommand{\titel}{}
\providecommand{\editorInnen}{}
\providecommand{\dateiname}{\jobname}

\vspace{3cm}

\vfill

\footnotesize
\textsc{Quelle}: \titel. Herausgegeben von {\editorInnen}. In: \emph{Arthur Schnitzler: Briefwechsel mit Autorinnen und Autoren}.
 Digitale Edition, https://schnitzler-briefe.acdh.oeaw.ac.at/{\dateiname}.html (Stand \today)
\fi

\end{document}


