%% latex-korrekturansicht-vorspann.tex
%% Vorspann für die Korrekturansicht.
%% Lädt die gemeinsame Datei latex-vorspann.tex mit gesetztem Schalter.

\newif\ifkorrekturansicht
\korrekturansichttrue

\input{../tex-inputs/latex-vorspann}


\section[Arthur Schnitzler an Richard Beer-Hofmann, {[}30.? 6. 1915{]}]{L02212 Arthur Schnitzler an Richard Beer-Hofmann, {[}30.? 6. 1915{]}}
\nopagebreak\mylabel{L02212v}
\rehead{ }\normalsize\beginnumbering\briefempfaengerindex{Beer-Hofmann, Richard@\textsc{Beer-Hofmann, Richard}!zzzSchnitzler, Arthur@\emph{von Arthur Schnitzler}!1915-06-301@{{[}30.? 6. 1915{]}}|(be}
\toendnotes[C]{\smallbreak\pagebreak[2]}\Standort{CUL, Schnitzler, B 8.1, S. 148.}
\physDesc{Karte, maschinenschriftliche Abschrift1 Blatt, 1 Seite, 411 Zeichen
\newline{}Schreibmaschine
\newline{}Zusatz: von unbekannter Hand als Briefnummer »335«
                                 gekennzeichnet }
\buchAbdrucke{\weitereDrucke{Arthur Schnitzler, Richard Beer-Hofmann: \emph{Briefwechsel 1891–1931}. Wien, Zürich: \emph{Europaverlag} 1992, S. 221.} }\toendnotes[C]{\smallbreak}
\pstart
           \raggedleft{}{\pb}Wien\oindex{Wien@\textbf{Wien}, \emph{A.ADM2}|pw}, ? 1915.\pend
           \vspace{0.5em}
\pstart
           Lieber Richard, Dr. Reik\pwindex{Reik, Theodor 12.05.1888 – 31.12.1969@\textsc{Reik, Theodor} (12.05.1888 – 31.12.1969), \emph{Psychoanalytiker/Psychoanalytikerin}|pw}
               wohnt in Wien XVIII. Lazaristengasse 2\oindex{Lazaristengasse@\textbf{Lazaristengasse}, \emph{Straße (K.STR)}|pw}. – Auf
               dem Semmering\oindex{Semmering@\textbf{Semmering}, \emph{A.ADM3}|pw} wars sehr schön (das Essen leider
               unmöglich, und lächerlich theuer), – seit \label{K_L02212-1v}\edtext{Montag}{\lemma{\textnormal{\emph{Montag}}}\Cendnote{\textnormal{28. 6. 1915, was eine unmittelbare Antwort an Beer-Hofmann\pwindex{Beer-Hofmann, Richard 1866-07-11 – 1945-09-26@\textsc{Beer-Hofmann, Richard} (1866-07-11 – 1945-09-26), \emph{Schriftsteller/Schriftstellerin}|pwk} wahrscheinlich macht.}}}\label{K_L02212-1} sind wir zurück
               und seither regnet’s. Dass wir Ende Juli nach Ischl\oindex{Bad Ischl@\textbf{Bad Ischl}, \emph{P.PPL}|pw} kommen (auf circa 10–14 Tage) ist nicht unwahrscheinlich.
               Lassen Sie sichs wohl ergehn, und all den Ihren. Wir grüssen herzlichst. Ihr
                  \spacefill\mbox{Arthur.}\pend
           
\pstart
           \noindent{}(nach Ischl, Doctor-Sterz-Weg 14\oindex{Doktor-Sterz-Weg@\textbf{Doktor-Sterz-Weg}, \emph{Straße (K.STR)}|pw})\pend
           \selectlanguage{ngerman}\endnumbering\briefempfaengerindex{Beer-Hofmann, Richard@\textsc{Beer-Hofmann, Richard}!zzzSchnitzler, Arthur@\emph{von Arthur Schnitzler}!1915-06-301@{{[}30.? 6. 1915{]}}|)be}\mylabel{L02212h}  \normalsize

\doendnotes{C}
\bigskip
\vfill

\clearpage

\footnotesize

\lohead{\textsc{register}}

% Definiere theindex-Environment komplett neu ohne reledmac
\makeatletter
\renewenvironment{theindex}{%
  \section*{\indexname}%
  \setlength{\parindent}{0pt}%
  \setlength{\parskip}{0pt plus 0.3pt}%
  \let\item\@idxitem
}{%
  \clearpage
}
\makeatother

\IfFileExists{\jobname-pw.ind}{\input{\jobname-pw.ind}}{}

\end{document}

      