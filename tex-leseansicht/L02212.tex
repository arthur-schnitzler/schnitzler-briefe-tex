%% latex-leseansicht-vorspann.tex
%% Vorspann für die Leseansicht.
%% Lädt die gemeinsame Datei latex-vorspann.tex mit nicht gesetztem Schalter.

\newif\ifkorrekturansicht
\korrekturansichtfalse

\input{../tex-inputs/latex-vorspann}


         
         \renewcommand{\erwaehntePersonen}{Personen: Richard Beer-Hofmann, Theodor Reik}
         \renewcommand{\erwaehnteOrte}{Orte: Bad Ischl, Doktor-Sterz-Weg, Lazaristengasse, Semmering, Wien}
         \renewcommand{\erwaehnteWerke}{}
               \section[Arthur Schnitzler an Richard Beer-Hofmann, {[}30.? 6. 1915{]}]{ Arthur Schnitzler an Richard Beer-Hofmann, {[}30.? 6. 1915{]}}\nopagebreak\mylabel{v}\rehead{ }\begin{ledgroupsized}[t]{13cm}\normalsize\beginnumbering \toendnotes[C]{\smallbreak\pagebreak[2]} \Standort{CUL, Schnitzler, B 8.1, S. 148.}
\physDesc{maschinelle Abschrift
\newline{}Schreibmaschine\newline{}Zusatz: von unbekannter Hand als Briefnummer »335«
                                 gekennzeichnet }\buchAbdrucke{\weitereDrucke{Arthur Schnitzler, Richard Beer-Hofmann: \emph{Briefwechsel 1891–1931}. Hg. Konstanze Fliedl. Wien, Zürich: \emph{Europaverlag} 1992, S. 221.} }\toendnotes[C]{\smallbreak}\pstart
           \raggedleft{}{\pb}Wien\oindex{Wien@\textbf{Wien}|pw}, ? 1915.\pend
           \pstart
           Lieber Richard, Dr. Reik\pwindex{Reik, Theodor 12.05.1888 – 31.12.1969@\textsc{Reik, Theodor} (12.05.1888 – 31.12.1969), \emph{Psychoanalytiker}|pw} wohnt
               in Wien XVIII. Lazaristengasse 2\oindex{Lazaristengasse@\textbf{Lazaristengasse}|pw}. – Auf dem Semmering\oindex{Semmering@\textbf{Semmering}|pw} wars sehr schön (das Essen leider
               unmöglich, und lächerlich theuer), – seit \label{K_L02212_1v}\edtext{Montag}{\lemma{\textnormal{\emph{Montag}}}\Cendnote{\textnormal{28. 6. 1915, was eine
                  unmittelbare Antwort an Beer-Hofmann\pwindex{Beer-Hofmann, Richard 1866-07-11 – 1945-09-26@\textsc{Beer-Hofmann, Richard} (1866-07-11 – 1945-09-26), \emph{Schriftsteller}|pwk}
                  wahrscheinlich macht.}}}\label{K_L02212_1h} sind wir zurück und seither regnet’s. Dass wir
                  Ende Juli nach Ischl\oindex{Bad Ischl@\textbf{Bad Ischl}|pw} kommen (auf
               circa 10–14 Tage) ist nicht unwahrscheinlich. Lassen Sie sichs wohl ergehn, und all
               den Ihren. Wir grüssen herzlichst. Ihr \spacefill\mbox{Arthur.}\pend
           \pstart
           \noindent{}(nach Ischl, Doctor-Sterz-Weg 14\oindex{Doktor-Sterz-Weg@\textbf{Doktor-Sterz-Weg}|pw})\pend
           
         
         \endnumbering\mylabel{h}\end{ledgroupsized}  \newcommand{\dateiname}{L02212}\newcommand{\titel}{Arthur Schnitzler an Richard Beer-Hofmann, [30.? 6. 1915]}\newcommand{\editorInnen}{Martin Anton Müller und Gerd-Hermann Susen}%% latex-leseansicht-abspann.tex
%% Abspann für die Leseansicht.
%% Der Schalter \ifkorrekturansicht ist bereits durch den Vorspann gesetzt.

%% latex-abspann.tex
%% Gemeinsamer Abspann für Korrekturansicht und Leseansicht.
%% Setzt den Schalter \ifkorrekturansicht voraus (gesetzt in den
%% einbindenden Dateien latex-korrekturansicht-abspann.tex bzw.
%% latex-leseansicht-abspann.tex).
%% ---------------------------------------------------------------

\normalsize

% Das esempio-Environment wird nur in der Leseansicht benötigt
\ifkorrekturansicht\else
\newenvironment{esempio}[3]%
{
    \vspace{1.5ex}
    \rlap{\underline{#1}}
    \par
    \setlength{\parindent}{0cm}
    \nopagebreak
    \leftskip=#2cm
    \rightskip=#3cm
}
{
    \par
}
\fi

\doendnotes{C}
\bigskip
\vfill

\clearpage

\footnotesize

\ifkorrekturansicht
  \lohead{\textsc{register}}
\fi

% theindex-Environment neu definieren ohne reledmac
\makeatletter
\renewenvironment{theindex}{%
  \ifkorrekturansicht
    \section*{\indexname}%
  \else
    \subsubsection*{Index der erwähnten Entitäten}%
  \fi
  \setlength{\parindent}{0pt}%
  \setlength{\parskip}{0pt plus 0.3pt}%
  \let\item\@idxitem
}{%
  \ifkorrekturansicht\clearpage\fi
}
\makeatother

\IfFileExists{\jobname-pw.ind}{\input{\jobname-pw.ind}}{}

% Quellenangabe nur in der Leseansicht
\ifkorrekturansicht\else
% Fallback-Definitionen, falls die .tex-Datei \titel etc. nicht gesetzt hat
\providecommand{\titel}{}
\providecommand{\editorInnen}{}
\providecommand{\dateiname}{\jobname}

\vspace{3cm}

\vfill

\footnotesize
\textsc{Quelle}: \titel. Herausgegeben von {\editorInnen}. In: \emph{Arthur Schnitzler: Briefwechsel mit Autorinnen und Autoren}.
 Digitale Edition, https://schnitzler-briefe.acdh.oeaw.ac.at/{\dateiname}.html (Stand \today)
\fi

\end{document}


      