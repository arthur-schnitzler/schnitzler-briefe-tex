%% latex-korrekturansicht-vorspann.tex
%% Vorspann für die Korrekturansicht.
%% Lädt die gemeinsame Datei latex-vorspann.tex mit gesetztem Schalter.

\newif\ifkorrekturansicht
\korrekturansichttrue

\input{../tex-inputs/latex-vorspann}


\section[Hugo von Hofmannsthal an Arthur Schnitzler, {[}30. 1. 1898{]}]{L00771 Hugo von Hofmannsthal an Arthur Schnitzler, {[}30. 1. 1898{]}}
\nopagebreak\mylabel{L00771v}
\rehead{ }\normalsize\beginnumbering\briefempfaengerindex{Schnitzler, Arthur@\textsc{Schnitzler, Arthur}!zzzHofmannsthal, Hugo von@\emph{von Hugo von Hofmannsthal}!1898-01-301@{{[}30. 1. 1898{]}}|(be}
\toendnotes[C]{\smallbreak\pagebreak[2]}\Standort{CUL, Schnitzler, B 43b/1.}
\physDesc{Briefkarte, 290 Zeichen
\newline{}Handschrift: Bleistift, deutsche Kurrent
\newline{}Schnitzler: mit Bleistift datiert: »30/1 98« 
\newline{}Ordnung: 1) mit Bleistift von unbekannter Hand nummeriert: »\strikeout{108}«  2) mit Bleistift von unbekannter Hand nummeriert:
                                    »107«}
\buchAbdrucke{\weitereDrucke{Hugo von Hofmannsthal, Arthur Schnitzler: \emph{Briefwechsel}. Frankfurt am Main: \emph{S. Fischer} 1964, S. 99.} }\toendnotes[C]{\smallbreak}
\pstart
           \noindent{}{\pb}lieber, ſeien {[}Sie{]} nicht bös. Sie müſſen
               miſsverſtanden haben, ich hab meinen Sitz zur \label{K_L00771-1v}\edtext{Landi\pwindex{Landi, Camilla 1863-06-20 – 1944-01-05@\textsc{Landi, Camilla} (1863-06-20 – 1944-01-05), \emph{Sänger/Sängerin}|pw}}{\lemma{\textnormal{\emph{Landi}}}\Cendnote{\textnormal{Camilla Landi\pwindex{Landi, Camilla 1863-06-20 – 1944-01-05@\textsc{Landi, Camilla} (1863-06-20 – 1944-01-05), \emph{Sänger/Sängerin}|pwk} trat am
                     11. 2. 1898 im Bösendorfersaal\oindex{Boesendorfer-Saal@\textbf{Bösendorfer-Saal}, \emph{Veranstaltungsgebäude (K.VSB)}|pwk}
                  auf. Schnitzler war zu dem Zeitpunkt nicht
                  in Wien\oindex{Wien@\textbf{Wien}, \emph{A.ADM2}|pwk} und besuchte die Vorstellung
                  nicht.}}}\label{K_L00771-1}{ }ſchon ſeit 10 Tagen. Ich glaube Richard\pwindex{Beer-Hofmann, Richard 1866-07-11 – 1945-09-26@\textsc{Beer-Hofmann, Richard} (1866-07-11 – 1945-09-26), \emph{Schriftsteller/Schriftstellerin}|pw} hat Sie gebeten, ich nur um 3 Sitze zur \label{K_L00771-2v}\edtext{\textsc{première}\pwindex{Freiwild. Schauspiel in 3 Akten@\emph{Freiwild. Schauspiel in 3 Akten}|pwv}}{\lemma{\textnormal{\emph{première}}}\Cendnote{\textnormal{von \emph{Freiwild}\pwindex{Freiwild. Schauspiel in 3 Akten@\emph{Freiwild. Schauspiel in 3 Akten}|pwk} am 4. 2. 1898 im Carl-Theater\oindex{Carl-Theater@\textbf{Carl-Theater}, \emph{Theater (K.THE)}|pwk}}}}\label{K_L00771-2}.\pend
           
\pstart
           {\pb}Die Brandes\pwindex{Brandes, Georg 04.02.1842 – 19.02.1927@\textsc{Brandes, Georg} (04.02.1842 – 19.02.1927)|pw}abende waren ſehr hübſch und haben mir ſehr viel Freude
               gemacht. Ich hoff, ich ſeh Sie bald wieder.\pend
           \pstart Ihr \spacefill\mbox{Hugo}\pend{}\selectlanguage{ngerman}\endnumbering\briefempfaengerindex{Schnitzler, Arthur@\textsc{Schnitzler, Arthur}!zzzHofmannsthal, Hugo von@\emph{von Hugo von Hofmannsthal}!1898-01-301@{{[}30. 1. 1898{]}}|)be}\mylabel{L00771h}  \normalsize

\doendnotes{C}
\bigskip
\vfill

\clearpage

\footnotesize

\lohead{\textsc{register}}

% Definiere theindex-Environment komplett neu ohne reledmac
\makeatletter
\renewenvironment{theindex}{%
  \section*{\indexname}%
  \setlength{\parindent}{0pt}%
  \setlength{\parskip}{0pt plus 0.3pt}%
  \let\item\@idxitem
}{%
  \clearpage
}
\makeatother

\IfFileExists{\jobname-pw.ind}{\input{\jobname-pw.ind}}{}

\end{document}

      