%% latex-korrekturansicht-vorspann.tex
%% Vorspann für die Korrekturansicht.
%% Lädt die gemeinsame Datei latex-vorspann.tex mit gesetztem Schalter.

\newif\ifkorrekturansicht
\korrekturansichttrue

\input{../tex-inputs/latex-vorspann}


\section[Arthur und Olga Schnitzler und Otto Brahm an Richard Beer-Hofmann, 14. 11. 1906]{L01637 Arthur und Olga Schnitzler und Otto Brahm an Richard Beer-Hofmann,
               14. 11. 1906}
\nopagebreak\mylabel{L01637v}
\rehead{ }\normalsize\beginnumbering\briefempfaengerindex{Beer-Hofmann, Richard@\textsc{Beer-Hofmann, Richard}!zzzBrahm, Otto@\emph{von Otto Brahm}!1906-11-141@{14. 11. 1906}|(be}\briefempfaengerindex{Beer-Hofmann, Richard@\textsc{Beer-Hofmann, Richard}!zzzSchnitzler, Olga@\emph{von Olga Schnitzler}!1906-11-141@{14. 11. 1906}|(be}\briefempfaengerindex{Beer-Hofmann, Richard@\textsc{Beer-Hofmann, Richard}!zzzSchnitzler, Arthur@\emph{von Arthur Schnitzler}!1906-11-141@{14. 11. 1906}|(be}
\toendnotes[C]{\smallbreak\pagebreak[2]}\Standort{YCGL, MSS 31.}
\physDesc{Bildpostkarte, 257 Zeichen
\newline{}Handschrift Arthur Schnitzler: Bleistift, lateinische Kurrent
\newline{}Handschrift Olga Schnitzler: Bleistift
\newline{}Handschrift Otto Brahm: Bleistift, lateinische Kurrent
\newline{}Versand: Stempel: »\nobreak{}\oindex{Semmering@\textbf{Semmering}, \emph{A.ADM3}|pwk}Semmering, 14. XI. 1906, 6\nobreak{}«.  
\newline{}Ordnung: mit Bleistift von unbekannter Hand beschriftet: »=Arthur
                                    u. Olga Schnitzler« und datiert: »14. 11.« }
\buchAbdrucke{\weitereDrucke{Arthur Schnitzler, Richard Beer-Hofmann: \emph{Briefwechsel 1891–1931}. Wien, Zürich: \emph{Europaverlag} 1992, S. 180.} }\toendnotes[C]{\smallbreak}\pstart{}{\pb}Dr. Richard\pend{}\pstart{}Beer-Hofmann\pend{}\pstart{}Wien XVIII.\oindex{XVIII., Waehring@\textbf{XVIII., Währing}, \emph{A.ADM3}|pw}\pend{}\pstart{} Hasenauerstraße \substVorne{}\textsuperscript{56}\substDazwischen{}59\substHinten{}\oindex{Hasenauerstrasse 59@\textbf{Hasenauerstraße 59}, \emph{Wohngebäude (K.WHS)}|pw}.\pend{}{\bigskip}
\pstart
           \noindent{}\centering{}{\pb}\textcolor{gray}{\textbf{Semmering. Greiß\oindex{Greis@\textbf{Greis}, \emph{P.PPL}|pw}}}\pend
           \vspace{1em}
\pstart
           \noindent{}{\pb}Grüaß Gott!\pend
           
\pstart
           Ihr liabn \label{K_L01637-1v}\edtext{Nachbarsleutln}{\lemma{\textnormal{\emph{Nachbarsleutln}}}\Cendnote{\textnormal{Richard\pwindex{Beer-Hofmann, Richard 1866-07-11 – 1945-09-26@\textsc{Beer-Hofmann, Richard} (1866-07-11 – 1945-09-26), \emph{Schriftsteller/Schriftstellerin}|pwk} und 
               Paula Beer-Hofmann\pwindex{Beer-Hofmann, Paula 25.02.1879 – 30.10.1939@\textsc{Beer-Hofmann, Paula} (25.02.1879 – 30.10.1939)|pwk}
               wohnten seit wenigen Tagen in dem für
               sie neu gebauten Haus in der Hasenauerstraße\oindex{Hasenauerstrasse 59@\textbf{Hasenauerstraße 59}, \emph{Wohngebäude (K.WHS)}|pwk}, waren also nunmehr Nachbarn.}}}\label{K_L01637-1}!\pend
           
\pstart
           \spacefill\mbox{»D’r Turl«}{\\[\baselineskip]}\spacefill\mbox{{[}hs. :{]} »D’Olga}«\pend
           \leftskip=0em{}
\pstart
           {[}hs. :{]} 14. 11. 906. \pend
           \selectlanguage{ngerman}\vspace{1em}
\pstart
           \noindent{}{\pb}Is Ihnen \substVorne{}\textsuperscript{schon}\substDazwischen{}scho\substHinten{} besser? –\pend
           \pstart \spacefill\mbox{A.}\pend{}\selectlanguage{ngerman}\vspace{1em}
\pstart
           \noindent{}{[}hs. :{]} Ich brauche wohl nicht zu bemerken, dass dieser A. mir
               alle Hendln wegisst. Herzlichen Gruss jedennoch.\pend
           \pstart \spacefill\mbox{Brahm.}\pend{}\selectlanguage{ngerman}\endnumbering\briefempfaengerindex{Beer-Hofmann, Richard@\textsc{Beer-Hofmann, Richard}!zzzBrahm, Otto@\emph{von Otto Brahm}!1906-11-141@{14. 11. 1906}|)be}\briefempfaengerindex{Beer-Hofmann, Richard@\textsc{Beer-Hofmann, Richard}!zzzSchnitzler, Olga@\emph{von Olga Schnitzler}!1906-11-141@{14. 11. 1906}|)be}\briefempfaengerindex{Beer-Hofmann, Richard@\textsc{Beer-Hofmann, Richard}!zzzSchnitzler, Arthur@\emph{von Arthur Schnitzler}!1906-11-141@{14. 11. 1906}|)be}\mylabel{L01637h}  \normalsize

\doendnotes{C}
\bigskip
\vfill

\clearpage

\footnotesize

\lohead{\textsc{register}}

% Definiere theindex-Environment komplett neu ohne reledmac
\makeatletter
\renewenvironment{theindex}{%
  \section*{\indexname}%
  \setlength{\parindent}{0pt}%
  \setlength{\parskip}{0pt plus 0.3pt}%
  \let\item\@idxitem
}{%
  \clearpage
}
\makeatother

\IfFileExists{\jobname-pw.ind}{\input{\jobname-pw.ind}}{}

\end{document}

      