%% latex-leseansicht-vorspann.tex
%% Vorspann für die Leseansicht.
%% Lädt die gemeinsame Datei latex-vorspann.tex mit nicht gesetztem Schalter.

\newif\ifkorrekturansicht
\korrekturansichtfalse

\input{../tex-inputs/latex-vorspann}


         
         \newcommand{\erwaehntePersonen}{Personen: }
         \newcommand{\erwaehnteInstitutionen}{}
         \newcommand{\erwaehnteOrte}{}
         \newcommand{\erwaehnteWerke}{
               \section[Arthur und Olga Schnitzler und Otto Brahm an Richard Beer-Hofmann, 14. 11. 1906]{ Arthur und Olga Schnitzler und Otto Brahm an Richard Beer-Hofmann,
               14. 11. 1906}\nopagebreak\mylabel{v}\rehead{ }\begin{ledgroupsized}[t]{13cm}\normalsize\beginnumbering \toendnotes[C]{\smallbreak\pagebreak[2]} \Standort{YCGL, MSS 31.}
\physDesc{Bildpostkarte
\newline{}Handschrift  : Bleistift, lateinische Kurrent\newline{}Handschrift  : Bleistift\newline{}Handschrift  : Bleistift, lateinische Kurrent\newline{}Versand: Stempel: »\nobreak{}\oindex{XXXX Ortsangabe fehlt|pwk}Semmering, 14. XI. 1906, 6\nobreak{}«.  \newline{}Ordnung: mit Bleistift von unbekannter Hand beschriftet: »=Arthur
                                    u. Olga Schnitzler« und datiert: »14. 11.« }\buchAbdrucke{\weitereDrucke{Arthur Schnitzler, Richard Beer-Hofmann: \emph{Briefwechsel 1891–1931}. Hg. Konstanze Fliedl. Wien, Zürich: \emph{Europaverlag} 1992, S. 180.} }\toendnotes[C]{\smallbreak}\pstart{}{\pb}Dr. Richard\pend{}\pstart{}Beer-Hofmann\pend{}\pstart{}Wien XVIII.\oindex{XXXX Ortsangabe fehlt|pw}\pend{}\pstart{} Hasenauerstraße \substVorne{}\textsuperscript{56}\substDazwischen{}59\substHinten{}\oindex{XXXX Ortsangabe fehlt|pw}.\pend{}{\bigskip}\pstart
           \noindent{}\centering{}{\pb}\textcolor{gray}{\textbf{Semmering. Greiß\oindex{XXXX Ortsangabe fehlt|pw}}}\pend
           \pstart
           {\pb}Grüaß Gott!\pend
           \pstart
           Ihr liabn \label{K_L01637_1v}\edtext{Nachbarsleutln}{\lemma{\textnormal{\emph{Nachbarsleutln}}}\Cendnote{\textnormal{Diese wohnten seit wenigen Tagen in dem für
                  sie neu gebauten Haus in der Hasenauerstraße\oindex{XXXX Ortsangabe fehlt|pwk}.}}}\label{K_L01637_1h}!\pend
           \pstart
           \spacefill\mbox{»D’r Turl«}{\\[\baselineskip]}\spacefill\mbox{{[}hs. :{]} »D’Olga}«\pend
           \leftskip=0em{}\pstart
           {[}hs. :{]} 14. 11. 906. \pend
           \pstart
           \noindent{}{\pb}Is Ihnen \substVorne{}\textsuperscript{schon}\substDazwischen{}scho\substHinten{} besser? –\pend
           \pstart \spacefill\mbox{A.}\pend{}\pstart
           \noindent{}{[}hs. :{]} Ich brauche wohl nicht zu bemerken, dass dieser A. mir
               alle Hendln wegisst. Herzlichen Gruss jedennoch.\pend
           \pstart \spacefill\mbox{Brahm.}\pend{}
         
         \endnumbering\mylabel{h}\end{ledgroupsized}  \newcommand{\dateiname}{L01637}\newcommand{\titel}{Arthur und Olga Schnitzler und Otto Brahm an Richard Beer-Hofmann, 14. 11. 1906}\newcommand{\editorInnen}{Martin Anton Müller und Gerd-Hermann Susen}%% latex-leseansicht-abspann.tex
%% Abspann für die Leseansicht.
%% Der Schalter \ifkorrekturansicht ist bereits durch den Vorspann gesetzt.

%% latex-abspann.tex
%% Gemeinsamer Abspann für Korrekturansicht und Leseansicht.
%% Setzt den Schalter \ifkorrekturansicht voraus (gesetzt in den
%% einbindenden Dateien latex-korrekturansicht-abspann.tex bzw.
%% latex-leseansicht-abspann.tex).
%% ---------------------------------------------------------------

\normalsize

% Das esempio-Environment wird nur in der Leseansicht benötigt
\ifkorrekturansicht\else
\newenvironment{esempio}[3]%
{
    \vspace{1.5ex}
    \rlap{\underline{#1}}
    \par
    \setlength{\parindent}{0cm}
    \nopagebreak
    \leftskip=#2cm
    \rightskip=#3cm
}
{
    \par
}
\fi

\doendnotes{C}
\bigskip
\vfill

\clearpage

\footnotesize

\ifkorrekturansicht
  \lohead{\textsc{register}}
\fi

% theindex-Environment neu definieren ohne reledmac
\makeatletter
\renewenvironment{theindex}{%
  \ifkorrekturansicht
    \section*{\indexname}%
  \else
    \subsubsection*{Index der erwähnten Entitäten}%
  \fi
  \setlength{\parindent}{0pt}%
  \setlength{\parskip}{0pt plus 0.3pt}%
  \let\item\@idxitem
}{%
  \ifkorrekturansicht\clearpage\fi
}
\makeatother

\IfFileExists{\jobname-pw.ind}{\input{\jobname-pw.ind}}{}

% Quellenangabe nur in der Leseansicht
\ifkorrekturansicht\else
% Fallback-Definitionen, falls die .tex-Datei \titel etc. nicht gesetzt hat
\providecommand{\titel}{}
\providecommand{\editorInnen}{}
\providecommand{\dateiname}{\jobname}

\vspace{3cm}

\vfill

\footnotesize
\textsc{Quelle}: \titel. Herausgegeben von {\editorInnen}. In: \emph{Arthur Schnitzler: Briefwechsel mit Autorinnen und Autoren}.
 Digitale Edition, https://schnitzler-briefe.acdh.oeaw.ac.at/{\dateiname}.html (Stand \today)
\fi

\end{document}


      