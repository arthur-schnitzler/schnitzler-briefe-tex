%% latex-korrekturansicht-vorspann.tex
%% Vorspann für die Korrekturansicht.
%% Lädt die gemeinsame Datei latex-vorspann.tex mit gesetztem Schalter.

\newif\ifkorrekturansicht
\korrekturansichttrue

\input{../tex-inputs/latex-vorspann}


\section[Arthur Schnitzler: Widmungsexemplar Anatol für Hermann Bahr, {[}29. 10.?{]} 1892]{L00129 Arthur Schnitzler: Widmungsexemplar Anatol für Hermann Bahr,
               {[}29. 10.?{]} 1892}
\nopagebreak\mylabel{L00129v}
\rehead{ }\normalsize\beginnumbering\briefempfaengerindex{Bahr, Hermann@\textsc{Bahr, Hermann}!zzzSchnitzler, Arthur@\emph{von Arthur Schnitzler}!1892-10-291@{{[}29. 10.?{]} 1892}|(be}
\toendnotes[C]{\smallbreak\pagebreak[2]}\Standort{Salzburg, Universitätsbibliothek, 32342-I.}
\physDesc{, 59 Zeichen
\newline{}Handschrift: schwarze Tinte, deutsche Kurrent}
\buchAbdrucke{\weitereDrucke{Hermann Bahr, Arthur Schnitzler: \emph{Briefwechsel, Aufzeichnungen, Dokumente (1891–1931)}. Göttingen: \emph{Wallstein} 2018, S. 28.} }\toendnotes[C]{\smallbreak}
\pstart
           \noindent{}{\pb}Herrn \textsc{Hermann Bahr}{\\}freundſchaftlich u verehrungsvoll\pend
           \pstart \spacefill\mbox{ArthSch}\pend{}{\vspace{1\baselineskip}}
\pstart
           \centering{}\textcolor{gray}{\textbf{\textbf{Arthur Schnitzler.}}}\pend
           
\pstart
           \centering{}\textcolor{gray}{\textbf{Anatol\pwindex{Anatol@\emph{Anatol}|pw}.}}\pend
           {\vspace{1\baselineskip}}
\pstart
           \centering{}\textcolor{gray}{\textbf{\textbf{Berlin\oindex{Berlin@\textbf{Berlin}, \emph{P.PPLC}|pw},{ }\label{K_L00129-1v}\edtext{1893}{\lemma{\textnormal{\emph{1893}}}\Cendnote{\textnormal{Vgl. A. S.: \emph{Tagebuch}, 29. 10. 1892. Am
                              3. 11. 1892 wurde das Buch vom \emph{Börsenblatt für den deutschen Buchhandel}\pwindex{Boersenblatt fuer den Deutschen Buchhandel@\emph{Börsenblatt für den Deutschen Buchhandel}|pwk} als
                           Neuerscheinung gemeldet.}}}\label{K_L00129-1}.}}}\pend
           
\pstart
           \centering{}\textcolor{gray}{\textbf{Verlag des Bibliographiſchen Bureaus\orgindex{Bibliographisches Bureau@Bibliographisches Bureau|pw}.}}\pend
           
\pstart
           \centering{}\textcolor{gray}{\textbf{Alexanderſtraße 2\oindex{Alexanderstrasse@\textbf{Alexanderstraße}, \emph{Straße (K.STR)}|pw}.}}\pend
           \selectlanguage{ngerman}\endnumbering\briefempfaengerindex{Bahr, Hermann@\textsc{Bahr, Hermann}!zzzSchnitzler, Arthur@\emph{von Arthur Schnitzler}!1892-10-291@{{[}29. 10.?{]} 1892}|)be}\mylabel{L00129h}  \normalsize

\doendnotes{C}
\bigskip
\vfill

\clearpage

\footnotesize

\lohead{\textsc{register}}

% Definiere theindex-Environment komplett neu ohne reledmac
\makeatletter
\renewenvironment{theindex}{%
  \section*{\indexname}%
  \setlength{\parindent}{0pt}%
  \setlength{\parskip}{0pt plus 0.3pt}%
  \let\item\@idxitem
}{%
  \clearpage
}
\makeatother

\IfFileExists{\jobname-pw.ind}{\input{\jobname-pw.ind}}{}

\end{document}

      