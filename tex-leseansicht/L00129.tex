%% latex-leseansicht-vorspann.tex
%% Vorspann für die Leseansicht.
%% Lädt die gemeinsame Datei latex-vorspann.tex mit nicht gesetztem Schalter.

\newif\ifkorrekturansicht
\korrekturansichtfalse

\input{../tex-inputs/latex-vorspann}


               \section[Arthur Schnitzler: Widmungsexemplar Anatol für Hermann Bahr, {[}29. 10.?{]} 1892]{ Arthur Schnitzler: Widmungsexemplar Anatol für Hermann Bahr,
               {[}29. 10.?{]} 1892}\nopagebreak\mylabel{v}\rehead{ }\begin{ledgroupsized}[t]{13cm}\normalsize\beginnumbering\briefempfaengerindex{Bahr, Hermann@\textsc{Bahr, Hermann}!zzzSchnitzler, Arthur@\emph{von Arthur Schnitzler}!1892-10-291@{{[}29. 10.?{]} 1892}|(be} \toendnotes[C]{\smallbreak\pagebreak[2]} \Standort{Salzburg, Universitätsbibliothek, 32342-I.}
\physDesc{Widmung am Titelblatt
\newline{}Handschrift: schwarze Tinte, deutsche Kurrent}\buchAbdrucke{\weitereDrucke{Hermann Bahr, Arthur Schnitzler: \emph{Briefwechsel, Aufzeichnungen, Dokumente (1891–1931)}. Hg. Kurt Ifkovits und Martin Anton Müller. Göttingen: \emph{Wallstein} 2018, S. 28.} }\toendnotes[C]{\smallbreak}\pstart
           \noindent{}{\pb}Herrn \textsc{Hermann Bahr}{\\}freundſchaftlich u verehrungsvoll\pend
           \pstart \spacefill\mbox{ArthSch}\pend{}{\bigskip}\pstart
           \noindent{}\centering{}\textcolor{gray}{\textbf{\textbf{Arthur Schnitzler.}}}\pend
           \pstart
           \noindent{}\centering{}\textcolor{gray}{\textbf{Anatol\pwindex{Schnitzler, Arthur 15.05.1862 – 21.10.1931@\textsc{Schnitzler, Arthur} (15.05.1862 – 21.10.1931), \emph{Schriftsteller, Mediziner}!Anatol1892-10-29 – 1892-10-29@\strich\emph{Anatol} {[}1892-10-29 – 1892-10-29{]}|pw}.}}\pend
           {\bigskip}\pstart
           \noindent{}\centering{}\textcolor{gray}{\textbf{\textbf{Berlin\oindex{Berlin@\textbf{Berlin}|pw},{ }\label{K_L00129_1v}\edtext{1893}{\lemma{\textnormal{\emph{1893}}}\Cendnote{\textnormal{Vgl. A. S.: \emph{Tagebuch}, 29. 10. 1892. Am
                           3. 11. 1892 vom \emph{Börsenblatt für den deutschen
                              Buchhandel}\pwindex{Boersenblatt fuer den deutschen Buchhandel1843-01-03@\emph{Börsenblatt für den deutschen Buchhandel}|pwk} als Neuerscheinung gemeldet}}}\label{K_L00129_1h}.}}}\pend
           \pstart
           \noindent{}\centering{}\textcolor{gray}{\textbf{Verlag des Bibliographiſchen Bureaus\orgindex{Bibliographisches Bureau@Bibliographisches Bureau|pw}.}}\pend
           \pstart
           \noindent{}\centering{}\textcolor{gray}{\textbf{Alexanderſtraße 2\oindex{Alexanderstrasse@\textbf{Alexanderstraße}|pw}.}}\pend
                     \endnumbering\briefempfaengerindex{Bahr, Hermann@\textsc{Bahr, Hermann}!zzzSchnitzler, Arthur@\emph{von Arthur Schnitzler}!1892-10-291@{{[}29. 10.?{]} 1892}|)be}\mylabel{h}\end{ledgroupsized}  \newcommand{\dateiname}{L00129}\newcommand{\titel}{Arthur Schnitzler: Widmungsexemplar Anatol für Hermann Bahr, [29. 10.?] 1892}\newcommand{\editorInnen}{ Kurt Ifkovits,  Martin Anton Müller}
            \footnotesize
\begin{ledgroupsized}[t]{11.5cm}
\doendnotes{C}
\end{ledgroupsized}
         %% latex-leseansicht-abspann.tex
%% Abspann für die Leseansicht.
%% Der Schalter \ifkorrekturansicht ist bereits durch den Vorspann gesetzt.

%% latex-abspann.tex
%% Gemeinsamer Abspann für Korrekturansicht und Leseansicht.
%% Setzt den Schalter \ifkorrekturansicht voraus (gesetzt in den
%% einbindenden Dateien latex-korrekturansicht-abspann.tex bzw.
%% latex-leseansicht-abspann.tex).
%% ---------------------------------------------------------------

\normalsize

% Das esempio-Environment wird nur in der Leseansicht benötigt
\ifkorrekturansicht\else
\newenvironment{esempio}[3]%
{
    \vspace{1.5ex}
    \rlap{\underline{#1}}
    \par
    \setlength{\parindent}{0cm}
    \nopagebreak
    \leftskip=#2cm
    \rightskip=#3cm
}
{
    \par
}
\fi

\doendnotes{C}
\bigskip
\vfill

\clearpage

\footnotesize

\ifkorrekturansicht
  \lohead{\textsc{register}}
\fi

% theindex-Environment neu definieren ohne reledmac
\makeatletter
\renewenvironment{theindex}{%
  \ifkorrekturansicht
    \section*{\indexname}%
  \else
    \subsubsection*{Index der erwähnten Entitäten}%
  \fi
  \setlength{\parindent}{0pt}%
  \setlength{\parskip}{0pt plus 0.3pt}%
  \let\item\@idxitem
}{%
  \ifkorrekturansicht\clearpage\fi
}
\makeatother

\IfFileExists{\jobname-pw.ind}{\input{\jobname-pw.ind}}{}

% Quellenangabe nur in der Leseansicht
\ifkorrekturansicht\else
% Fallback-Definitionen, falls die .tex-Datei \titel etc. nicht gesetzt hat
\providecommand{\titel}{}
\providecommand{\editorInnen}{}
\providecommand{\dateiname}{\jobname}

\vspace{3cm}

\vfill

\footnotesize
\textsc{Quelle}: \titel. Herausgegeben von {\editorInnen}. In: \emph{Arthur Schnitzler: Briefwechsel mit Autorinnen und Autoren}.
 Digitale Edition, https://schnitzler-briefe.acdh.oeaw.ac.at/{\dateiname}.html (Stand \today)
\fi

\end{document}


      