%% latex-leseansicht-vorspann.tex
%% Vorspann für die Leseansicht.
%% Lädt die gemeinsame Datei latex-vorspann.tex mit nicht gesetztem Schalter.

\newif\ifkorrekturansicht
\korrekturansichtfalse

\input{../tex-inputs/latex-vorspann}


               \section[Arthur Schnitzler an Hugo von Hofmannsthal, 26. 6. 1903]{ Arthur Schnitzler an Hugo von Hofmannsthal, 26. 6. 1903}\nopagebreak\mylabel{v}\rehead{ }\begin{ledgroupsized}[t]{13cm}\normalsize\beginnumbering\briefempfaengerindex{Hofmannsthal, Hugo von@\textsc{Hofmannsthal, Hugo von}!zzzSchnitzler, Arthur@\emph{von Arthur Schnitzler}!1903-06-261@{26. 6. 1903}|(be} \toendnotes[C]{\smallbreak\pagebreak[2]} \Standort{FDH, Hs-30885,103.}
\physDesc{Brief, 2 Blätter, 8 Seiten
\newline{}Handschrift: schwarze Tinte, deutsche Kurrent\newline{}Ordnung: mit Bleistift von unbekannter Hand das zweite Blatt datiert: »26/6 903« }\buchAbdrucke{\weitereDrucke{1) Hugo von Hofmannsthal, Arthur Schnitzler: \emph{Briefwechsel}. Hg. Therese Nickl und Heinrich Schnitzler. Frankfurt am Main: \emph{S. Fischer} 1964, S. 170–172.} \weitereDrucke{2) Arthur Schnitzler: \emph{Briefe 1875–1912}. Hg. Therese Nickl und Heinrich Schnitzler. Frankfurt am Main: \emph{S. Fischer} 1981, S. 463–464.} \weitereDrucke{3) Hermann Bahr, Arthur Schnitzler: \emph{Briefwechsel, Aufzeichnungen, Dokumente
                                (1891–1931)}. Hg. Kurt Ifkovits und Martin Anton Müller. Göttingen: \emph{Wallstein} 2018, S. 267.} }\toendnotes[C]{\smallbreak}\pstart
           \raggedleft{}{\pb}Wien\oindex{Wien@\textbf{Wien}|pw}, 26. 6. 903\pend
           \pstart
           mein lieber Hugo, aus Ihrem Brief muſs ich entnehmen, daſs
                    unſre Karten von der Reiſe gar nicht zu Ihnen gelangt sind. Ich habe Ihnen aus
                        Venedig\oindex{Venedig@\textbf{Venedig}|pw} (auch Hans\pwindex{Schlesinger, Hans Bernhard 20.07.1875 – 13.3.1932@\textsc{Schlesinger, Hans Bernhard} (20.07.1875 – 13.3.1932), \emph{Maler}|pw} war auf dieſer Karte unterſchrieben) und aus Lugano\oindex{Lugano@\textbf{Lugano}|pw} eine (ſogar \textsc{versificirte}) Nachricht geſandt. In Lugano\oindex{Lugano@\textbf{Lugano}|pw} haben wir im \textsc{H. d. parc\oindex{Hôtel du Parc@\textbf{Hôtel du Parc}|pw}} gewohnt, und die liebenswürdige verheiratete Tochter\pwindex{Ober, Bertha @\textsc{Ober, Bertha}, \emph{Malerin, Hotelière}|pwv} der Madame \textsc{Bèha\pwindex{Beha, Elisa @\textsc{Béha, Elisa}, \emph{Hoteldirektorin}|pw}} zeigte uns die »Stätte«, wo {\pb}Sie zu ſchreiben
                    pflegten. Was war es nur, das Sie damals arbeiteten? Vom Wetter waren wir nicht
                    ſehr begünſtigt; auf dem \textsc{Generoso\oindex{Monte Generoso@\textbf{Monte Generoso}|pw}} Nebel, Gewitter; in \textsc{Varese\oindex{Varese@\textbf{Varese}|pw}} ein Platzregen, daſs wir nicht \introOben{}bis\introOben{} zum \textsc{Grd Hotel\oindex{Grand Hotel Varese@\textbf{Grand Hotel Varese}|pw}} gelangten u lieber gleich zurück fuhren. Die andern Seen fielen ſozuſagen
                    ins Waſſer, was ſie doch gar nicht mehr notwendig haben. Vor Lugano\oindex{Lugano@\textbf{Lugano}|pw}: Venedig\oindex{Venedig@\textbf{Venedig}|pw} (Hans\pwindex{Schlesinger, Hans Bernhard 20.07.1875 – 13.3.1932@\textsc{Schlesinger, Hans Bernhard} (20.07.1875 – 13.3.1932), \emph{Maler}|pw} zeigte uns einige {\pb}palazzi, die wir ſonſt gewiſs nicht geſehen
                    hätten), Segelfahrt nach \textsc{Torcello\oindex{Torcello@\textbf{Torcello}|pw}} (wenn Sie es nicht kennen, verſäumen Sie’s nicht bei nächſter Venezianer\oindex{Ponte di Rialto@\textbf{Ponte di Rialto}|pw} Gelegenheit) – \textsc{Padua\oindex{Padua@\textbf{Padua}|pw}}, \textsc{Vicenza\oindex{Vicenza@\textbf{Vicenza}|pw}}, \textsc{Verona\oindex{Verona@\textbf{Verona}|pw}}, \textsc{Mailand\oindex{Mailand@\textbf{Mailand}|pw}}. Luini\pwindex{Luini, Bernardino um 1481 – 1532@\textsc{Luini, Bernardino} (um 1481 – 1532), \emph{Maler}|pw}, an dem ich (rein körperlich
                    gemeint) vor Jahren vorbeigegangen war, ging mir wundervoll auf. –\pend
           \pstart
           Von »geordneter« Arbeit wäre nichts mitzutheilen. Zumeiſt beſchäftigte mich das
                    ſonderbare, {\pb}oft begonnene, einige Mal beendete,
                    jedes Mal hingeworfene Junggeſellen\textcolor{gray}{-}Egoiſtenſtück\pwindex{Schnitzler, Arthur 15.05.1862 – 21.10.1931@\textsc{Schnitzler, Arthur} (15.05.1862 – 21.10.1931), \emph{Schriftsteller, Mediziner}!einsame Weg. Schauspiel in fuenf Akten1904@\strich\emph{Der einsame Weg. Schauspiel in fünf Akten} {[}1904{]}|pwv}\pwindex{Schnitzler, Arthur 15.05.1862 – 21.10.1931@\textsc{Schnitzler, Arthur} (15.05.1862 – 21.10.1931), \emph{Schriftsteller, Mediziner}!Professor Bernhardi. Komoedie in fuenf Akten1912@\strich\emph{Professor Bernhardi. Komödie in fünf Akten} {[}1912{]}|pwv}; Sie wiſſen, daſs es
                    zuletzt als Misgeburt zur Welt kam, ſiameſiſch gezwillingt. Nun ſcheint der
                    operative Eingriff, der mit Vorſicht unternommen werden mußte, gelungen – d. h.
                    beide Geſchöpfe leben, das eine ſchwächlich, das andre mit höherer Vitalkraft
                    begnadet, {\pb}aber ob ſie endgiltig gedeihen werden,
                    iſt noch nicht zu ſagen. Das eine Kind wird eben aufgepäppelt.\pend
           \pstart
           – Am Roman\pwindex{Schnitzler, Arthur 15.05.1862 – 21.10.1931@\textsc{Schnitzler, Arthur} (15.05.1862 – 21.10.1931), \emph{Schriftsteller, Mediziner}!Weg ins Freie. Roman1.1.1908 – 1.6.1908@\strich\emph{Der Weg ins Freie. Roman} {[}1.1.1908 – 1.6.1908{]}|pwv} geſchah nichts
                    weiteres; über eine luſtſpielartige, moderne Komödie wurde meditirt. Im ganzen
                    mehr Kunſt- und Gedankenſpiel als Schaffensintenſität. –\pend
           \pstart
           Mit großem Vergnügen las ich die \textsc{mousquetaires}\pwindex{Dumas, Alexandre pere 24.07.1802 – 05.12.1870@\textsc{Dumas, Alexandre père} (24.07.1802 – 05.12.1870), \emph{Schriftsteller}!drei Musketiere1844@\strich\emph{Die drei Musketiere} {[}1844{]}|pw} v. \textsc{Dumas}\pwindex{Dumas, Alexandre pere 24.07.1802 – 05.12.1870@\textsc{Dumas, Alexandre père} (24.07.1802 – 05.12.1870), \emph{Schriftsteller}|pw} auf der Reiſe. Welche Leichtigkeit,
                    welcher Reichtum! Einiger Leichtſinn verzeiht ſich von ſelbſt; {\pb}und die paar falſchen Münzen wirken, als machte
                    sich ein Kind \strikeout{damit} einen Spaſs ſie ſtatt
                    echten, die doch da ſind, auszuſtreuen. –\pend
           \pstart
           – \textsc{Bahr\pwindex{Bahr, Hermann 19.07.1863 – 15.01.1934@\textsc{Bahr, Hermann} (19.07.1863 – 15.01.1934), \emph{Schriftsteller, Kritiker}|pw}} hat mir von Ihren letzten \label{K_L01300_1v}\edtext{Plänen}{\lemma{\textnormal{\emph{Plänen}}}\Cendnote{\textnormal{\emph{Elektra}\pwindex{Hofmannsthal, Hugo von 01.02.1874 – 15.07.1929@\textsc{Hofmannsthal, Hugo von} (01.02.1874 – 15.07.1929), \emph{Schriftsteller}!Elektra. Tragoedie in einem Aufzug1903@\strich\emph{Elektra. Tragödie in einem Aufzug} {[}1903{]}|pwk} und \emph{Das gerettete Venedig}\pwindex{Hofmannsthal, Hugo von 01.02.1874 – 15.07.1929@\textsc{Hofmannsthal, Hugo von} (01.02.1874 – 15.07.1929), \emph{Schriftsteller}!gerettete Venedig. Trauerspiel in fuenf Aufzuegen1905@\strich\emph{Das gerettete Venedig. Trauerspiel in fünf Aufzügen} {[}1905{]}|pwk}.}}}\label{K_L01300_1h} erzählt, Richard\pwindex{Beer-Hofmann, Richard 11.07.1866 – 26.09.1945@\textsc{Beer-Hofmann, Richard} (11.07.1866 – 26.09.1945), \emph{Schriftsteller}|pw}, der geſtern mit Paula\pwindex{Beer-Hofmann, Paula 25.02.1879 – 30.10.1939@\textsc{Beer-Hofmann, Paula} (25.02.1879 – 30.10.1939)|pw} u Mirjam\pwindex{Beer-Hofmann, Mirjam 04.09.1897 – 24.12.1984@\textsc{Beer-Hofmann, Mirjam} (04.09.1897 – 24.12.1984)|pw} bei mir war,
                    desgleichen. Ich wünſchte bald zu hören wie weit Sie gediehen ſind.\pend
           \pstart
           Die deutſchen\oindex{Deutschland@\textbf{Deutschland}|pw}{ }Schall u Raucher\oindex{Schall und Rauch@\textbf{Schall und Rauch}|pw}{ }ſah ich \introOben{}vor\introOben{}geſtern, Erdgeiſt\pwindex{Wedekind, Frank 24.07.1864 – 09.03.1918@\textsc{Wedekind, Frank} (24.07.1864 – 09.03.1918), \emph{Schriftsteller, Schauspieler}!Erdgeist. Tragoedie in vier Aufzuegen1895@\strich\emph{Erdgeist. Tragödie in vier Aufzügen} {[}1895{]}|pw}, das Talent, das große Wedekind\pwindex{Wedekind, Frank 24.07.1864 – 09.03.1918@\textsc{Wedekind, Frank} (24.07.1864 – 09.03.1918), \emph{Schriftsteller, Schauspieler}|pw}eſche {\pb}blitzt meines
                    Erachtens nur ſelten auf. Vielleicht ernſthaft nur in der Figur des Dr Schön\pwindex{Wedekind, Frank 24.07.1864 – 09.03.1918@\textsc{Wedekind, Frank} (24.07.1864 – 09.03.1918), \emph{Schriftsteller, Schauspieler}!Erdgeist. Tragoedie in vier Aufzuegen1895@\strich\emph{Erdgeist. Tragödie in vier Aufzügen} {[}1895{]}|pwv} (der einzigen, die
                    wirklich vollendet geſpielt wurde \introOben{}(\textsc{Reicher\pwindex{Reicher, Emanuel 18.06.1849 – 15.05.1924@\textsc{Reicher, Emanuel} (18.06.1849 – 15.05.1924), \emph{Schauspieler}|pw}})\introOben{}.) Das unerträgliche aber an dem Stück iſt mir, daſs der Humor
                    darin der ſich ſo sataniſch geberdet, nicht viel teufliſcher iſt als ein
                    weitgereiſter Commis \introOben{}als \textsc{Mephisto\pwindex{\textcolor{red}{\textsuperscript{XXXX1 indx}}!Faust1790 – 1832@\strich\emph{Faust} {[}1790 – 1832{]}|pwv}}\introOben{} auf einem Maskenball, – der mit dämoniſchen Weibern Champagner zu trinken
                    vermeint – während es {\pb}ſich um Köchinnen und \textsc{Kleinoscheg} handelt. – Im ganzen lieb ich Dichter
                    nicht, die ihren Nachlaſs bei Lebzeiten herausgeben. –\pend
           \pstart
           Wie steht es mit Ihren ferneren Sommerplänen? Ich denke etwa um den 10.
                        Auguſt nach Südtirol\oindex{Suedtirol@\textbf{Südtirol}|pw} zu gehen. \textsc{Mendel}, Campiglio\oindex{Madonna di Campiglio@\textbf{Madonna di Campiglio}|pw}{[}.{]}{ }Richard\pwindex{Beer-Hofmann, Richard 11.07.1866 – 26.09.1945@\textsc{Beer-Hofmann, Richard} (11.07.1866 – 26.09.1945), \emph{Schriftsteller}|pw} will mit – radeln.\pend
           \pstart
           Laſſen Sie baldigſt von ſich hören. Wir grüßen Sie und Gerty\pwindex{Hofmannsthal, Gertrude von 16.03.1880 – 09.11.1959@\textsc{Hofmannsthal, Gertrude von} (16.03.1880 – 09.11.1959)|pw} herzlichſt.\pend
           \pstart
           Ihr{\\[\baselineskip]}\spacefill\mbox{A.}\pend
           \leftskip=0em{}\endnumbering\briefempfaengerindex{Hofmannsthal, Hugo von@\textsc{Hofmannsthal, Hugo von}!zzzSchnitzler, Arthur@\emph{von Arthur Schnitzler}!1903-06-261@{26. 6. 1903}|)be}\mylabel{h}\end{ledgroupsized}  \newcommand{\dateiname}{L01300}\newcommand{\titel}{Arthur Schnitzler an Hugo von Hofmannsthal, 26. 6. 1903}\newcommand{\editorInnen}{ Martin Anton Müller und Gerd-Hermann Susen}%% latex-leseansicht-abspann.tex
%% Abspann für die Leseansicht.
%% Der Schalter \ifkorrekturansicht ist bereits durch den Vorspann gesetzt.

%% latex-abspann.tex
%% Gemeinsamer Abspann für Korrekturansicht und Leseansicht.
%% Setzt den Schalter \ifkorrekturansicht voraus (gesetzt in den
%% einbindenden Dateien latex-korrekturansicht-abspann.tex bzw.
%% latex-leseansicht-abspann.tex).
%% ---------------------------------------------------------------

\normalsize

% Das esempio-Environment wird nur in der Leseansicht benötigt
\ifkorrekturansicht\else
\newenvironment{esempio}[3]%
{
    \vspace{1.5ex}
    \rlap{\underline{#1}}
    \par
    \setlength{\parindent}{0cm}
    \nopagebreak
    \leftskip=#2cm
    \rightskip=#3cm
}
{
    \par
}
\fi

\doendnotes{C}
\bigskip
\vfill

\clearpage

\footnotesize

\ifkorrekturansicht
  \lohead{\textsc{register}}
\fi

% theindex-Environment neu definieren ohne reledmac
\makeatletter
\renewenvironment{theindex}{%
  \ifkorrekturansicht
    \section*{\indexname}%
  \else
    \subsubsection*{Index der erwähnten Entitäten}%
  \fi
  \setlength{\parindent}{0pt}%
  \setlength{\parskip}{0pt plus 0.3pt}%
  \let\item\@idxitem
}{%
  \ifkorrekturansicht\clearpage\fi
}
\makeatother

\IfFileExists{\jobname-pw.ind}{\input{\jobname-pw.ind}}{}

% Quellenangabe nur in der Leseansicht
\ifkorrekturansicht\else
% Fallback-Definitionen, falls die .tex-Datei \titel etc. nicht gesetzt hat
\providecommand{\titel}{}
\providecommand{\editorInnen}{}
\providecommand{\dateiname}{\jobname}

\vspace{3cm}

\vfill

\footnotesize
\textsc{Quelle}: \titel. Herausgegeben von {\editorInnen}. In: \emph{Arthur Schnitzler: Briefwechsel mit Autorinnen und Autoren}.
 Digitale Edition, https://schnitzler-briefe.acdh.oeaw.ac.at/{\dateiname}.html (Stand \today)
\fi

\end{document}


      