%% latex-leseansicht-vorspann.tex
%% Vorspann für die Leseansicht.
%% Lädt die gemeinsame Datei latex-vorspann.tex mit nicht gesetztem Schalter.

\newif\ifkorrekturansicht
\korrekturansichtfalse

\input{../tex-inputs/latex-vorspann}


\section[Arthur Schnitzler an Hugo von Hofmannsthal, 26. 6. 1903]{L01300 Arthur Schnitzler an Hugo von Hofmannsthal, 26. 6. 1903}
\nopagebreak\mylabel{L01300v}
\rehead{ }\normalsize\beginnumbering\briefempfaengerindex{Hofmannsthal, Hugo von@\textsc{Hofmannsthal, Hugo von}!zzzSchnitzler, Arthur@\emph{von Arthur Schnitzler}!1903-06-261@{26. 6. 1903}|(be}
\toendnotes[C]{\smallbreak\pagebreak[2]}
\correspDesc{Versand  durch Arthur Schnitzler am 26. 6. 1903 in Wien
\newline{}Erhalt  durch Hugo von Hofmannsthal im Zeitraum [26. 6. 1903
                  – 30. 6. 1903?] \textbf{Ort fehlend} }\toendnotes[C]{\smallbreak}
\Standort{FDH, Hs-30885,103.}
\physDesc{Brief, 2 Blätter, 8 Seiten, 3024 Zeichen
\newline{}Handschrift: schwarze Tinte, deutsche Kurrent
\newline{}Ordnung: mit Bleistift von unbekannter Hand das zweite Blatt datiert: »26/6 903« }
\buchAbdrucke{\weitereDrucke{1) Hugo von Hofmannsthal, Arthur Schnitzler: \emph{Briefwechsel}. Herausgegeben von Therese Nickl und Heinrich Schnitzler. Frankfurt am Main: \emph{S. Fischer} 1964, S. 170–172.} \weitereDrucke{2) Arthur Schnitzler: \emph{Briefe 1875–1912}. Herausgegeben von Therese Nickl und Heinrich Schnitzler. Frankfurt am Main: \emph{S. Fischer} 1981, S. 463–464.} \weitereDrucke{3) Hermann Bahr, Arthur Schnitzler: \emph{Briefwechsel, Aufzeichnungen, Dokumente (1891–1931)}. Herausgegeben von Kurt Ifkovits und Martin Anton Müller. Göttingen: \emph{Wallstein} 2018, S. 267.} }\toendnotes[C]{\smallbreak}
\pstart
           \raggedleft{}{\pb}Wien\oindex{Wien@\textbf{Wien}, \emph{Verwaltungsgebiet}|pw}, 26. 6. 903\pend
           \vspace{0.5em}
\pstart
           mein lieber Hugo, aus Ihrem Brief muſs ich entnehmen, daſs unſre
               Karten von der Reiſe gar nicht zu Ihnen gelangt sind. Ich habe Ihnen aus Venedig\oindex{Venedig@\textbf{Venedig}|pw} (auch Hans\pwindex{Schlesinger, Hans Bernhard 20.\,7.\,1875 Wien – 13.\,3.\,1932 Salzburg@\textsc{Schlesinger, Hans Bernhard} (20.\,7.\,1875 Wien – 13.\,3.\,1932 Salzburg), \emph{Maler}|pw} war auf dieſer Karte unterſchrieben) und aus Lugano\oindex{Lugano@\textbf{Lugano}, \emph{Hauptstadt}|pw} eine (ſogar \textsc{versificirte})
               Nachricht geſandt. In Lugano\oindex{Lugano@\textbf{Lugano}, \emph{Hauptstadt}|pw} haben wir im \textsc{H. d. parc\oindex{Hôtel du Parc@\textbf{Hôtel du Parc}, \emph{Hotel}|pw}} gewohnt, und die liebenswürdige verheiratete Tochter\pwindex{Ober, Bertha @\textsc{Ober, Bertha}, \emph{Malerin, Hotelière}|pwv} der Madame \textsc{Bèha\pwindex{Béha, Elisa @\textsc{Béha, Elisa}, \emph{Schriftstellerin, Übersetzerin, Hoteldirektorin}|pw}} zeigte uns die »Stätte«, wo {\pb}Sie zu{ }ſchreiben
               pflegten. Was war es nur, das Sie damals arbeiteten? Vom Wetter waren wir nicht{ }ſehr
               begünſtigt; auf dem \textsc{Generoso\oindex{Monte Generoso@\textbf{Monte Generoso}, \emph{Berg}|pw}} Nebel, Gewitter; in \textsc{Varese\oindex{Varese@\textbf{Varese}, \emph{Verwaltungsgebiet}|pw}} ein Platzregen, daſs wir nicht \introOben{}bis\introOben{} zum \textsc{Grd Hotel\oindex{Grand Hotel Varese@\textbf{Grand Hotel Varese}, \emph{Hotel}|pw}} gelangten u lieber gleich zurück fuhren. Die andern Seen fielen{ }ſozuſagen ins
               Waſſer, was{ }ſie doch gar nicht mehr notwendig haben. Vor Lugano\oindex{Lugano@\textbf{Lugano}, \emph{Hauptstadt}|pw}: Venedig\oindex{Venedig@\textbf{Venedig}|pw} (Hans\pwindex{Schlesinger, Hans Bernhard 20.\,7.\,1875 Wien – 13.\,3.\,1932 Salzburg@\textsc{Schlesinger, Hans Bernhard} (20.\,7.\,1875 Wien – 13.\,3.\,1932 Salzburg), \emph{Maler}|pw} zeigte uns einige {\pb}palazzi, die wir{ }ſonſt gewiſs nicht geſehen hätten),
               Segelfahrt nach \textsc{Torcello\oindex{Torcello@\textbf{Torcello}|pw}} (wenn Sie es nicht kennen, verſäumen Sie’s nicht bei nächſter Venezianer\oindex{Ponte di Rialto@\textbf{Ponte di Rialto}, \emph{Brücke}|pw} Gelegenheit) – \textsc{Padua\oindex{Padua@\textbf{Padua}, \emph{Hauptstadt}|pw}}, \textsc{Vicenza\oindex{Vicenza@\textbf{Vicenza}, \emph{Hauptstadt}|pw}}, \textsc{Verona\oindex{Verona@\textbf{Verona}, \emph{Hauptstadt}|pw}}, \textsc{Mailand\oindex{Mailand@\textbf{Mailand}|pw}}. Luini\pwindex{Luini, Bernardino um 1481 Luino – 1532 Mailand@\textsc{Luini, Bernardino} (um 1481 Luino – 1532 Mailand), \emph{Maler}|pw}, an dem ich (rein körperlich
               gemeint) vor Jahren vorbeigegangen war, ging mir wundervoll auf. –\pend
           
\pstart
           Von »geordneter« Arbeit wäre nichts mitzutheilen. Zumeiſt beſchäftigte mich das{ }ſonderbare, {\pb}oft begonnene, einige Mal beendete, jedes
               Mal hingeworfene Junggeſellen\textcolor{gray}{-}Egoiſtenſtück\pwindex{Schnitzler, Arthur 15.\,5.\,1862 Wien – 21.\,10.\,1931 ebd.@\textsc{Schnitzler, Arthur} (15.\,5.\,1862 Wien – 21.\,10.\,1931 ebd.), \emph{Schriftsteller, Mediziner}!einsame Weg. Schauspiel in fünf Akten@\strich\emph{Der einsame Weg. Schauspiel in fünf Akten}|pwv}\pwindex{Schnitzler, Arthur 15.\,5.\,1862 Wien – 21.\,10.\,1931 ebd.@\textsc{Schnitzler, Arthur} (15.\,5.\,1862 Wien – 21.\,10.\,1931 ebd.), \emph{Schriftsteller, Mediziner}!Professor Bernhardi. Komödie in fünf Akten@\strich\emph{Professor Bernhardi. Komödie in fünf Akten}|pwv}; Sie wiſſen, daſs es
               zuletzt als Misgeburt zur Welt kam,{ }ſiameſiſch gezwillingt. Nun{ }ſcheint der operative
               Eingriff, der mit Vorſicht unternommen werden mußte, gelungen – d. h. beide Geſchöpfe
               leben, das eine{ }ſchwächlich, das andre mit höherer Vitalkraft begnadet, {\pb}aber ob{ }ſie endgiltig gedeihen werden, iſt noch nicht zu{ }ſagen. Das eine Kind wird eben aufgepäppelt.\pend
           
\pstart
           – Am Roman\pwindex{Schnitzler, Arthur 15.\,5.\,1862 Wien – 21.\,10.\,1931 ebd.@\textsc{Schnitzler, Arthur} (15.\,5.\,1862 Wien – 21.\,10.\,1931 ebd.), \emph{Schriftsteller, Mediziner}!Weg ins Freie. Roman@\strich\emph{Der Weg ins Freie. Roman}|pwv} geſchah nichts
               weiteres; über eine \label{K_L01300-1v}\edtext{luſtſpielartige,
               moderne Komödie\pwindex{Schnitzler, Arthur 15.\,5.\,1862 Wien – 21.\,10.\,1931 ebd.@\textsc{Schnitzler, Arthur} (15.\,5.\,1862 Wien – 21.\,10.\,1931 ebd.), \emph{Schriftsteller, Mediziner}!Fink und Fliederbusch. Komödie in drei Akten@\strich\emph{Fink und Fliederbusch. Komödie in drei Akten}|pwv}}{\lemma{\textnormal{\emph{lustspielartige, … Komödie}}}\Cendnote{\textnormal{Vgl. XXXX Auszeichnungsfehler: Dokument L03373 nicht gefunden.
               }}}\label{K_L01300-1} wurde meditirt. Im ganzen mehr Kunſt- und
               Gedankenſpiel als Schaffensintenſität. –\pend
           
\pstart
           Mit großem Vergnügen las ich die \textsc{mousquetaires}\pwindex{Dumas, Alexandre père 24.\,7.\,1802 Villers-Cotterêts – 5.\,12.\,1870 Puys@\textsc{Dumas, Alexandre père} (24.\,7.\,1802 Villers-Cotterêts – 5.\,12.\,1870 Puys), \emph{Schriftsteller}!drei Musketiere@\strich\emph{Die drei Musketiere}|pw} v. \textsc{Dumas}\pwindex{Dumas, Alexandre père 24.\,7.\,1802 Villers-Cotterêts – 5.\,12.\,1870 Puys@\textsc{Dumas, Alexandre père} (24.\,7.\,1802 Villers-Cotterêts – 5.\,12.\,1870 Puys), \emph{Schriftsteller}|pw} auf der Reiſe. Welche Leichtigkeit, welcher Reichtum! Einiger Leichtſinn
               verzeiht{ }ſich von{ }ſelbſt; {\pb}und die paar falſchen Münzen
               wirken, als machte sich ein Kind \strikeout{damit} einen Spaſs{ }ſie{ }ſtatt echten, die doch da{ }ſind, auszuſtreuen. –\pend
           
\pstart
           – \textsc{Bahr\pwindex{Bahr, Hermann 19.\,7.\,1863 Linz – 15.\,1.\,1934 München@\textsc{Bahr, Hermann} (19.\,7.\,1863 Linz – 15.\,1.\,1934 München), \emph{Schriftsteller, Kritiker}|pw}} hat mir von Ihren letzten \label{K_L01300-2v}\edtext{Plänen}{\lemma{\textnormal{\emph{Plänen}}}\Cendnote{\textnormal{\emph{Elektra}\pwindex{Hofmannsthal, Hugo von 1.\,2.\,1874 Wien – 15.\,7.\,1929 Rodaun@\textsc{Hofmannsthal, Hugo von} (1.\,2.\,1874 Wien – 15.\,7.\,1929 Rodaun), \emph{Schriftsteller}!Elektra. Tragödie in einem Aufzug@\strich\emph{Elektra. Tragödie in einem Aufzug}|pwk} und \emph{Das gerettete Venedig}\pwindex{Hofmannsthal, Hugo von 1.\,2.\,1874 Wien – 15.\,7.\,1929 Rodaun@\textsc{Hofmannsthal, Hugo von} (1.\,2.\,1874 Wien – 15.\,7.\,1929 Rodaun), \emph{Schriftsteller}!gerettete Venedig. Trauerspiel in fünf Aufzügen@\strich\emph{Das gerettete Venedig. Trauerspiel in fünf Aufzügen}|pwk}.}}}\label{K_L01300-2} erzählt, Richard\pwindex{Beer-Hofmann, Richard 11.\,7.\,1866 Wien – 26.\,9.\,1945 New York City@\textsc{Beer-Hofmann, Richard} (11.\,7.\,1866 Wien – 26.\,9.\,1945 New York City), \emph{Schriftsteller}|pw}, der geſtern mit Paula\pwindex{Beer-Hofmann, Paula 25.\,2.\,1879 Wien – 30.\,10.\,1939 Zürich@\textsc{Beer-Hofmann, Paula} (25.\,2.\,1879 Wien – 30.\,10.\,1939 Zürich)|pw} u Mirjam\pwindex{Beer-Hofmann, Mirjam 4.\,9.\,1897 Wien – 24.\,12.\,1984 New York City@\textsc{Beer-Hofmann, Mirjam} (4.\,9.\,1897 Wien – 24.\,12.\,1984 New York City)|pw} bei mir war, desgleichen. Ich wünſchte
               bald zu hören wie weit Sie gediehen{ }ſind.\pend
           
\pstart
           Die deutſchen\oindex{Deutschland@\textbf{Deutschland}|pw}{ }Schall u Raucher\oindex{Schall und Rauch@\textbf{Schall und Rauch}, \emph{Kabarett}|pw}{ }ſah ich \introOben{}vor\introOben{}geſtern, Erdgeiſt\pwindex{Wedekind, Frank 24.\,7.\,1864 Hannover – 9.\,3.\,1918 München@\textsc{Wedekind, Frank} (24.\,7.\,1864 Hannover – 9.\,3.\,1918 München), \emph{Schriftsteller, Schauspieler, Schriftsteller}!Erdgeist. Tragödie in vier Aufzügen@\strich\emph{Erdgeist. Tragödie in vier Aufzügen}|pw}, das Talent, das große Wedekind\pwindex{Wedekind, Frank 24.\,7.\,1864 Hannover – 9.\,3.\,1918 München@\textsc{Wedekind, Frank} (24.\,7.\,1864 Hannover – 9.\,3.\,1918 München), \emph{Schriftsteller, Schauspieler, Schriftsteller}|pw}eſche {\pb}blitzt meines Erachtens nur{ }ſelten auf. Vielleicht ernſthaft nur in der Figur des
               Dr Schön\pwindex{Wedekind, Frank 24.\,7.\,1864 Hannover – 9.\,3.\,1918 München@\textsc{Wedekind, Frank} (24.\,7.\,1864 Hannover – 9.\,3.\,1918 München), \emph{Schriftsteller, Schauspieler, Schriftsteller}!Erdgeist. Tragödie in vier Aufzügen@\strich\emph{Erdgeist. Tragödie in vier Aufzügen}|pwv} (der einzigen, die
               wirklich vollendet geſpielt wurde \introOben{}(\textsc{Reicher\pwindex{Reicher, Emanuel 18.\,6.\,1849 Bochnia – 15.\,5.\,1924 Berlin@\textsc{Reicher, Emanuel} (18.\,6.\,1849 Bochnia – 15.\,5.\,1924 Berlin), \emph{Schauspieler}|pw}})\introOben{}.) Das unerträgliche aber an dem Stück iſt mir, daſs der Humor darin
               der{ }ſich{ }ſo sataniſch geberdet, nicht viel teufliſcher iſt als ein weitgereiſter
               Commis \introOben{}als \textsc{Mephisto\pwindex{\textcolor{red}{\textsuperscript{XXXX indx1}}!Faust. Eine Tragödie@\strich\emph{Faust. Eine Tragödie}|pwv}}\introOben{} auf einem Maskenball, – der mit dämoniſchen Weibern Champagner zu trinken
               vermeint – während es {\pb}ſich um Köchinnen und \textsc{Kleinoscheg} handelt. – Im ganzen lieb ich Dichter nicht, die
               ihren Nachlaſs bei Lebzeiten herausgeben. –\pend
           
\pstart
           Wie steht es mit Ihren ferneren Sommerplänen? Ich denke etwa um den 10.
               Auguſt nach Südtirol\oindex{Südtirol@\textbf{Südtirol}, \emph{Verwaltungsgebiet}|pw} zu gehen. \textsc{Mendel}\oindex{Mendelgebirge@\textbf{Mendelgebirge}, \emph{Gebirge}|pw}, Campiglio\oindex{Madonna di Campiglio@\textbf{Madonna di Campiglio}|pw}{[}.{]}{ }Richard\pwindex{Beer-Hofmann, Richard 11.\,7.\,1866 Wien – 26.\,9.\,1945 New York City@\textsc{Beer-Hofmann, Richard} (11.\,7.\,1866 Wien – 26.\,9.\,1945 New York City), \emph{Schriftsteller}|pw} will mit – radeln.\pend
           
\pstart
           Laſſen Sie baldigſt von{ }ſich hören. Wir grüßen Sie und Gerty\pwindex{Hofmannsthal, Gertrude von 16.\,3.\,1880 Wien – 9.\,11.\,1959 Paddington@\textsc{Hofmannsthal, Gertrude von} (16.\,3.\,1880 Wien – 9.\,11.\,1959 Paddington)|pw} herzlichſt.\pend
           
\pstart
           Ihr{\\[\baselineskip]}\spacefill\mbox{A.}\pend
           \leftskip=0em{}\selectlanguage{ngerman}\endnumbering\briefempfaengerindex{Hofmannsthal, Hugo von@\textsc{Hofmannsthal, Hugo von}!zzzSchnitzler, Arthur@\emph{von Arthur Schnitzler}!1903-06-261@{26. 6. 1903}|)be}\mylabel{L01300h}  \newcommand{\dateiname}{L01300}\newcommand{\titel}{Arthur Schnitzler an Hugo von Hofmannsthal, 26. 6. 1903}\newcommand{\editorInnen}{Herausgegeben von Martin Anton Müller}%% latex-leseansicht-abspann.tex
%% Abspann für die Leseansicht.
%% Der Schalter \ifkorrekturansicht ist bereits durch den Vorspann gesetzt.

%% latex-abspann.tex
%% Gemeinsamer Abspann für Korrekturansicht und Leseansicht.
%% Setzt den Schalter \ifkorrekturansicht voraus (gesetzt in den
%% einbindenden Dateien latex-korrekturansicht-abspann.tex bzw.
%% latex-leseansicht-abspann.tex).
%% ---------------------------------------------------------------

\normalsize

% Das esempio-Environment wird nur in der Leseansicht benötigt
\ifkorrekturansicht\else
\newenvironment{esempio}[3]%
{
    \vspace{1.5ex}
    \rlap{\underline{#1}}
    \par
    \setlength{\parindent}{0cm}
    \nopagebreak
    \leftskip=#2cm
    \rightskip=#3cm
}
{
    \par
}
\fi

\doendnotes{C}
\bigskip
\vfill

\clearpage

\footnotesize

\ifkorrekturansicht
  \lohead{\textsc{register}}
\fi

% theindex-Environment neu definieren ohne reledmac
\makeatletter
\renewenvironment{theindex}{%
  \ifkorrekturansicht
    \section*{\indexname}%
  \else
    \subsubsection*{Index der erwähnten Entitäten}%
  \fi
  \setlength{\parindent}{0pt}%
  \setlength{\parskip}{0pt plus 0.3pt}%
  \let\item\@idxitem
}{%
  \ifkorrekturansicht\clearpage\fi
}
\makeatother

\IfFileExists{\jobname-pw.ind}{\input{\jobname-pw.ind}}{}

% Quellenangabe nur in der Leseansicht
\ifkorrekturansicht\else
% Fallback-Definitionen, falls die .tex-Datei \titel etc. nicht gesetzt hat
\providecommand{\titel}{}
\providecommand{\editorInnen}{}
\providecommand{\dateiname}{\jobname}

\vspace{3cm}

\vfill

\footnotesize
\textsc{Quelle}: \titel. Herausgegeben von {\editorInnen}. In: \emph{Arthur Schnitzler: Briefwechsel mit Autorinnen und Autoren}.
 Digitale Edition, https://schnitzler-briefe.acdh.oeaw.ac.at/{\dateiname}.html (Stand \today)
\fi

\end{document}


