%% latex-korrekturansicht-vorspann.tex
%% Vorspann für die Korrekturansicht.
%% Lädt die gemeinsame Datei latex-vorspann.tex mit gesetztem Schalter.

\newif\ifkorrekturansicht
\korrekturansichttrue

\input{../tex-inputs/latex-vorspann}


\section[Arthur Schnitzler an Hugo von Hofmannsthal, 26. 6. 1903]{L01300 Arthur Schnitzler an Hugo von Hofmannsthal, 26. 6. 1903}
\nopagebreak\mylabel{L01300v}
\rehead{ }\normalsize\beginnumbering\briefempfaengerindex{Hofmannsthal, Hugo von@\textsc{Hofmannsthal, Hugo von}!zzzSchnitzler, Arthur@\emph{von Arthur Schnitzler}!1903-06-261@{26. 6. 1903}|(be}
\toendnotes[C]{\smallbreak\pagebreak[2]}\Standort{FDH, Hs-30885,103.}
\physDesc{Brief, 2 Blätter, 8 Seiten, 3024 Zeichen
\newline{}Handschrift: schwarze Tinte, deutsche Kurrent
\newline{}Ordnung: mit Bleistift von unbekannter Hand das zweite Blatt datiert: »26/6 903« }
\buchAbdrucke{\weitereDrucke{1) Hugo von Hofmannsthal, Arthur Schnitzler: \emph{Briefwechsel}. Frankfurt am Main: \emph{S. Fischer} 1964, S. 170–172.} \weitereDrucke{2) Arthur Schnitzler: \emph{Briefe 1875–1912}. Frankfurt am Main: \emph{S. Fischer} 1981, S. 463–464.} \weitereDrucke{3) Hermann Bahr, Arthur Schnitzler: \emph{Briefwechsel, Aufzeichnungen, Dokumente (1891–1931)}. Göttingen: \emph{Wallstein} 2018, S. 267.} }\toendnotes[C]{\smallbreak}
\pstart
           \raggedleft{}{\pb}Wien\oindex{Wien@\textbf{Wien}, \emph{A.ADM2}|pw}, 26. 6. 903\pend
           \vspace{0.5em}
\pstart
           mein lieber Hugo, aus Ihrem Brief muſs ich entnehmen, daſs unſre
               Karten von der Reiſe gar nicht zu Ihnen gelangt sind. Ich habe Ihnen aus Venedig\oindex{Venedig@\textbf{Venedig}, \emph{P.PPLA}|pw} (auch Hans\pwindex{Schlesinger, Hans Bernhard 20.07.1875 – 13.3.1932@\textsc{Schlesinger, Hans Bernhard} (20.07.1875 – 13.3.1932), \emph{Maler/Malerin}|pw} war auf dieſer Karte unterſchrieben) und aus Lugano\oindex{Lugano@\textbf{Lugano}, \emph{P.PPLA2}|pw} eine (ſogar \textsc{versificirte})
               Nachricht geſandt. In Lugano\oindex{Lugano@\textbf{Lugano}, \emph{P.PPLA2}|pw} haben wir im \textsc{H. d. parc\oindex{Hôtel du Parc@\textbf{Hôtel du Parc}, \emph{Hotel (K.HTL)}|pw}} gewohnt, und die liebenswürdige verheiratete Tochter\pwindex{Ober, Bertha @\textsc{Ober, Bertha}, \emph{Maler/Malerin, Hotelier/Hotelière}|pwv} der Madame \textsc{Bèha\pwindex{Beha, Elisa @\textsc{Béha, Elisa}, \emph{Schriftsteller/Schriftstellerin, Übersetzer/Übersetzerin, Hoteldirektor/Hoteldirektorin}|pw}} zeigte uns die »Stätte«, wo {\pb}Sie zu ſchreiben
               pflegten. Was war es nur, das Sie damals arbeiteten? Vom Wetter waren wir nicht ſehr
               begünſtigt; auf dem \textsc{Generoso\oindex{Monte Generoso@\textbf{Monte Generoso}, \emph{T.MT}|pw}} Nebel, Gewitter; in \textsc{Varese\oindex{Varese@\textbf{Varese}, \emph{A.ADM3}|pw}} ein Platzregen, daſs wir nicht \introOben{}bis\introOben{} zum \textsc{Grd Hotel\oindex{Grand Hotel Varese@\textbf{Grand Hotel Varese}, \emph{Hotel (K.HTL)}|pw}} gelangten u lieber gleich zurück fuhren. Die andern Seen fielen ſozuſagen ins
               Waſſer, was ſie doch gar nicht mehr notwendig haben. Vor Lugano\oindex{Lugano@\textbf{Lugano}, \emph{P.PPLA2}|pw}: Venedig\oindex{Venedig@\textbf{Venedig}, \emph{P.PPLA}|pw} (Hans\pwindex{Schlesinger, Hans Bernhard 20.07.1875 – 13.3.1932@\textsc{Schlesinger, Hans Bernhard} (20.07.1875 – 13.3.1932), \emph{Maler/Malerin}|pw} zeigte uns einige {\pb}palazzi, die wir ſonſt gewiſs nicht geſehen hätten),
               Segelfahrt nach \textsc{Torcello\oindex{Torcello@\textbf{Torcello}, \emph{P.PPL}|pw}} (wenn Sie es nicht kennen, verſäumen Sie’s nicht bei nächſter Venezianer\oindex{Ponte di Rialto@\textbf{Ponte di Rialto}, \emph{Brücke (K.BRK)}|pw} Gelegenheit) – \textsc{Padua\oindex{Padua@\textbf{Padua}, \emph{P.PPLA2}|pw}}, \textsc{Vicenza\oindex{Vicenza@\textbf{Vicenza}, \emph{P.PPLA2}|pw}}, \textsc{Verona\oindex{Verona@\textbf{Verona}, \emph{P.PPLA2}|pw}}, \textsc{Mailand\oindex{Mailand@\textbf{Mailand}, \emph{P.PPLA}|pw}}. Luini\pwindex{Luini, Bernardino um 1481 – 1532@\textsc{Luini, Bernardino} (um 1481 – 1532), \emph{Maler/Malerin}|pw}, an dem ich (rein körperlich
               gemeint) vor Jahren vorbeigegangen war, ging mir wundervoll auf. –\pend
           
\pstart
           Von »geordneter« Arbeit wäre nichts mitzutheilen. Zumeiſt beſchäftigte mich das
               ſonderbare, {\pb}oft begonnene, einige Mal beendete, jedes
               Mal hingeworfene Junggeſellen\textcolor{gray}{-}Egoiſtenſtück\pwindex{einsame Weg. Schauspiel in fuenf Akten@\emph{Der einsame Weg. Schauspiel in fünf Akten}|pwv}\pwindex{Professor Bernhardi. Komoedie in fuenf Akten@\emph{Professor Bernhardi. Komödie in fünf Akten}|pwv}; Sie wiſſen, daſs es
               zuletzt als Misgeburt zur Welt kam, ſiameſiſch gezwillingt. Nun ſcheint der operative
               Eingriff, der mit Vorſicht unternommen werden mußte, gelungen – d. h. beide Geſchöpfe
               leben, das eine ſchwächlich, das andre mit höherer Vitalkraft begnadet, {\pb}aber ob ſie endgiltig gedeihen werden, iſt noch nicht zu
               ſagen. Das eine Kind wird eben aufgepäppelt.\pend
           
\pstart
           – Am Roman\pwindex{Weg ins Freie. Roman@\emph{Der Weg ins Freie. Roman}|pwv} geſchah nichts
               weiteres; über eine \label{K_L01300-1v}\edtext{luſtſpielartige,
               moderne Komödie\pwindex{Fink und Fliederbusch. Komoedie in drei Akten@\emph{Fink und Fliederbusch. Komödie in drei Akten}|pwv}}{\lemma{\textnormal{\emph{luſtſpielartige, … Komödie}}}\Cendnote{\textnormal{Vgl. Paul Goldmann an Arthur Schnitzler, 2[2?]. 5. [1903].
               }}}\label{K_L01300-1} wurde meditirt. Im ganzen mehr Kunſt- und
               Gedankenſpiel als Schaffensintenſität. –\pend
           
\pstart
           Mit großem Vergnügen las ich die \textsc{mousquetaires}\pwindex{drei Musketiere@\emph{Die drei Musketiere}|pw} v. \textsc{Dumas}\pwindex{Dumas, Alexandre pere 24.07.1802 – 05.12.1870@\textsc{Dumas, Alexandre père} (24.07.1802 – 05.12.1870), \emph{Schriftsteller/Schriftstellerin}|pw} auf der Reiſe. Welche Leichtigkeit, welcher Reichtum! Einiger Leichtſinn
               verzeiht ſich von ſelbſt; {\pb}und die paar falſchen Münzen
               wirken, als machte sich ein Kind \strikeout{damit} einen Spaſs
               ſie ſtatt echten, die doch da ſind, auszuſtreuen. –\pend
           
\pstart
           – \textsc{Bahr\pwindex{Bahr, Hermann 19.07.1863 – 15.01.1934@\textsc{Bahr, Hermann} (19.07.1863 – 15.01.1934), \emph{Schriftsteller/Schriftstellerin, Kritiker/Kritikerin}|pw}} hat mir von Ihren letzten \label{K_L01300-2v}\edtext{Plänen}{\lemma{\textnormal{\emph{Plänen}}}\Cendnote{\textnormal{\emph{Elektra}\pwindex{Elektra. Tragoedie in einem Aufzug@\emph{Elektra. Tragödie in einem Aufzug}|pwk} und \emph{Das gerettete Venedig}\pwindex{gerettete Venedig. Trauerspiel in fuenf Aufzuegen@\emph{Das gerettete Venedig. Trauerspiel in fünf Aufzügen}|pwk}.}}}\label{K_L01300-2} erzählt, Richard\pwindex{Beer-Hofmann, Richard 1866-07-11 – 1945-09-26@\textsc{Beer-Hofmann, Richard} (1866-07-11 – 1945-09-26), \emph{Schriftsteller/Schriftstellerin}|pw}, der geſtern mit Paula\pwindex{Beer-Hofmann, Paula 25.02.1879 – 30.10.1939@\textsc{Beer-Hofmann, Paula} (25.02.1879 – 30.10.1939)|pw} u Mirjam\pwindex{Beer-Hofmann, Mirjam 04.09.1897 – 24.12.1984@\textsc{Beer-Hofmann, Mirjam} (04.09.1897 – 24.12.1984)|pw} bei mir war, desgleichen. Ich wünſchte
               bald zu hören wie weit Sie gediehen ſind.\pend
           
\pstart
           Die deutſchen\oindex{Deutschland@\textbf{Deutschland}, \emph{A.PCLI}|pw}{ }Schall u Raucher\oindex{Schall und Rauch@\textbf{Schall und Rauch}, \emph{Kabarett (K.KBR)}|pw}{ }ſah ich \introOben{}vor\introOben{}geſtern, Erdgeiſt\pwindex{Erdgeist. Tragoedie in vier Aufzuegen@\emph{Erdgeist. Tragödie in vier Aufzügen}|pw}, das Talent, das große Wedekind\pwindex{Wedekind, Frank 24.07.1864 – 09.03.1918@\textsc{Wedekind, Frank} (24.07.1864 – 09.03.1918), \emph{Schriftsteller/Schriftstellerin, Schauspieler/Schauspielerin, Schriftsteller/Schriftstellerin}|pw}eſche {\pb}blitzt meines Erachtens nur ſelten auf. Vielleicht ernſthaft nur in der Figur des
               Dr Schön\pwindex{Erdgeist. Tragoedie in vier Aufzuegen@\emph{Erdgeist. Tragödie in vier Aufzügen}|pwv} (der einzigen, die
               wirklich vollendet geſpielt wurde \introOben{}(\textsc{Reicher\pwindex{Reicher, Emanuel 18.06.1849 – 15.05.1924@\textsc{Reicher, Emanuel} (18.06.1849 – 15.05.1924), \emph{Schauspieler/Schauspielerin}|pw}})\introOben{}.) Das unerträgliche aber an dem Stück iſt mir, daſs der Humor darin
               der ſich ſo sataniſch geberdet, nicht viel teufliſcher iſt als ein weitgereiſter
               Commis \introOben{}als \textsc{Mephisto\pwindex{Faust. Eine Tragoedie@\emph{Faust. Eine Tragödie}|pwv}}\introOben{} auf einem Maskenball, – der mit dämoniſchen Weibern Champagner zu trinken
               vermeint – während es {\pb}ſich um Köchinnen und \textsc{Kleinoscheg} handelt. – Im ganzen lieb ich Dichter nicht, die
               ihren Nachlaſs bei Lebzeiten herausgeben. –\pend
           
\pstart
           Wie steht es mit Ihren ferneren Sommerplänen? Ich denke etwa um den 10.
               Auguſt nach Südtirol\oindex{Suedtirol@\textbf{Südtirol}, \emph{A.ADM2}|pw} zu gehen. \textsc{Mendel}\oindex{Mendelgebirge@\textbf{Mendelgebirge}, \emph{Gebirge (N.GBR)}|pw}, Campiglio\oindex{Madonna di Campiglio@\textbf{Madonna di Campiglio}, \emph{P.PPL}|pw}{[}.{]}{ }Richard\pwindex{Beer-Hofmann, Richard 1866-07-11 – 1945-09-26@\textsc{Beer-Hofmann, Richard} (1866-07-11 – 1945-09-26), \emph{Schriftsteller/Schriftstellerin}|pw} will mit – radeln.\pend
           
\pstart
           Laſſen Sie baldigſt von ſich hören. Wir grüßen Sie und Gerty\pwindex{Hofmannsthal, Gertrude von 16.03.1880 – 09.11.1959@\textsc{Hofmannsthal, Gertrude von} (16.03.1880 – 09.11.1959)|pw} herzlichſt.\pend
           
\pstart
           Ihr{\\[\baselineskip]}\spacefill\mbox{A.}\pend
           \leftskip=0em{}\selectlanguage{ngerman}\endnumbering\briefempfaengerindex{Hofmannsthal, Hugo von@\textsc{Hofmannsthal, Hugo von}!zzzSchnitzler, Arthur@\emph{von Arthur Schnitzler}!1903-06-261@{26. 6. 1903}|)be}\mylabel{L01300h}  \normalsize

\doendnotes{C}
\bigskip
\vfill

\clearpage

\footnotesize

\lohead{\textsc{register}}

% Definiere theindex-Environment komplett neu ohne reledmac
\makeatletter
\renewenvironment{theindex}{%
  \section*{\indexname}%
  \setlength{\parindent}{0pt}%
  \setlength{\parskip}{0pt plus 0.3pt}%
  \let\item\@idxitem
}{%
  \clearpage
}
\makeatother

\IfFileExists{\jobname-pw.ind}{\input{\jobname-pw.ind}}{}

\end{document}

      