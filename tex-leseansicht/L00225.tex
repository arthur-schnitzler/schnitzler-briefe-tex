%% latex-korrekturansicht-vorspann.tex
%% Vorspann für die Korrekturansicht.
%% Lädt die gemeinsame Datei latex-vorspann.tex mit gesetztem Schalter.

\newif\ifkorrekturansicht
\korrekturansichttrue

\input{../tex-inputs/latex-vorspann}


\section[Hugo von Hofmannsthal an Arthur Schnitzler, {[}21.? 6. 1893{]}]{L00225 Hugo von Hofmannsthal an Arthur Schnitzler, {[}21.? 6. 1893{]}}
\nopagebreak\mylabel{L00225v}
\rehead{ }\normalsize\beginnumbering\briefempfaengerindex{Schnitzler, Arthur@\textsc{Schnitzler, Arthur}!zzzHofmannsthal, Hugo von@\emph{von Hugo von Hofmannsthal}!1893-06-212@{{[}21.? 6. 1893{]}}|(be}
\toendnotes[C]{\smallbreak\pagebreak[2]}\Standort{CUL, Schnitzler, B 43.}
\physDesc{Brief, 1 Blatt, 1 Seite, 209 Zeichen (mit aufgeprägtem Wappen)
\newline{}Handschrift: schwarze Tinte, deutsche Kurrent (\noindent{}Tinte stark verwischt)
\newline{}Schnitzler: mit Bleistift datiert: »Juni 93« und nummeriert »49« }
\buchAbdrucke{\weitereDrucke{Hugo von Hofmannsthal, Arthur Schnitzler: \emph{Briefwechsel}. Frankfurt am Main: \emph{S. Fischer} 1964, S. 39.} }\toendnotes[C]{\smallbreak}
\pstart{}{\pb}Lieber Arthur.\pend\vspace{0.5em}
\pstart
           Heute geht nicht. Möchte morgen auf ganzen Tag, außer Regen. Schreiben Sie
               pneumatiſch, ob recht iſt. Wenn Sie nicht auf \uline{viele}
               Zeit nach \label{K_L00225-1v}\edtext{Baden\oindex{Baden bei Wien@\textbf{Baden bei Wien}, \emph{P.PPLA3}|pw}}{\lemma{\textnormal{\emph{Baden}}}\Cendnote{\textnormal{Traut man der Datierung Schnitzlers mit »Juni 93«, so dürfte dieses Korrespondenzstück am Vortag von Schnitzlers einzigem Besuch in Baden\oindex{Baden bei Wien@\textbf{Baden bei Wien}, \emph{P.PPLA3}|pwk} verfasst sein. Zu einem Treffen kam es laut Schnitzlers{ }\emph{Tagebuch}\pwindex{Tagebuch@\emph{Tagebuch}|pwk} dann aber nicht.}}}\label{K_L00225-1} müſſen, ſtehts ja doch
               dafür. Vielleicht \uline{Salten}\pwindex{Salten, Felix 06.09.1869 – 08.10.1945@\textsc{Salten, Felix} (06.09.1869 – 08.10.1945), \emph{Schriftsteller/Schriftstellerin, Journalist/Journalistin, Chefredakteur/Chefredakteurin}|pw} auch.\pend
           \pstart \spacefill\mbox{Hugo.}\pend{}\selectlanguage{ngerman}\endnumbering\briefempfaengerindex{Schnitzler, Arthur@\textsc{Schnitzler, Arthur}!zzzHofmannsthal, Hugo von@\emph{von Hugo von Hofmannsthal}!1893-06-212@{{[}21.? 6. 1893{]}}|)be}\mylabel{L00225h}  \normalsize

\doendnotes{C}
\bigskip
\vfill

\clearpage

\footnotesize

\lohead{\textsc{register}}

% Definiere theindex-Environment komplett neu ohne reledmac
\makeatletter
\renewenvironment{theindex}{%
  \section*{\indexname}%
  \setlength{\parindent}{0pt}%
  \setlength{\parskip}{0pt plus 0.3pt}%
  \let\item\@idxitem
}{%
  \clearpage
}
\makeatother

\IfFileExists{\jobname-pw.ind}{\input{\jobname-pw.ind}}{}

\end{document}

      