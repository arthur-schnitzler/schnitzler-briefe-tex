%% latex-korrekturansicht-vorspann.tex
%% Vorspann für die Korrekturansicht.
%% Lädt die gemeinsame Datei latex-vorspann.tex mit gesetztem Schalter.

\newif\ifkorrekturansicht
\korrekturansichttrue

\input{../tex-inputs/latex-vorspann}


\section[ Paul Goldmann an Arthur Schnitzler, 26. 4. 1899]{L02872 Paul Goldmann an Arthur Schnitzler, 26. 4. 1899}
\nopagebreak\mylabel{L02872v}
\rehead{ }\normalsize\beginnumbering\briefempfaengerindex{Schnitzler, Arthur@\textsc{Schnitzler, Arthur}!zzzGoldmann, Paul@\emph{von Paul Goldmann}!1899-04-261@{26. 4. 1899}|(be}
\toendnotes[C]{\smallbreak\pagebreak[2]}\Standort{DLA, A:Schnitzler, HS.NZ85.1.3169.}
\physDesc{Brief, 1 Blatt, 2 Seiten, 1217 Zeichen
\newline{}Handschrift: schwarze Tinte, deutsche Kurrent}\toendnotes[C]{\smallbreak}
\pstart
           {\pb}\textcolor{gray}{\textbf{\textbf{Frankfurter Zeitung}}}\orgindex{Frankfurter Zeitung@Frankfurter Zeitung|pw}\hfill \textcolor{gray}{\textbf{\textbf{Frankfurt a. M.\oindex{Frankfurt am Main@\textbf{Frankfurt am Main}, \emph{P.PPLA3}|pw},}}}{ }26. April \textcolor{gray}{\textbf{189}}9.\pend
           
\pstart
           \textcolor{gray}{\textbf{und}}\pend
           
\pstart
           \textcolor{gray}{\textbf{Handelsblatt.}}\pend
           
\pstart
           \textcolor{gray}{\textbf{\textbf{Redaktion\orgindex{Frankfurter Zeitung@Frankfurter Zeitung|pwv}.}\noindent{}\textcolor{gray}{\textbf{Für die Redaktion\orgindex{Frankfurter Zeitung@Frankfurter Zeitung|pwv} beſtimmte Briefe und Sendungen wolle man
                                 \so{nicht} an die Perſon eines Redakteurs,
                              ſondern ſtets \textbf{an die Redaktion der Frankfurter Zeitung\orgindex{Frankfurter Zeitung@Frankfurter Zeitung|pw}} adreſſiren.}}}}\pend
           
\pstart
           \textcolor{gray}{\textbf{Telegramm-Adreſſe:}}\pend
           
\pstart
           \textcolor{gray}{\textbf{\textbf{Zeitung\orgindex{Frankfurter Zeitung@Frankfurter Zeitung|pwv}{ }Frankfurt Main\oindex{Frankfurt am Main@\textbf{Frankfurt am Main}, \emph{P.PPLA3}|pw}.}}}\pend
           
\pstart{}Mein lieber Freund,\pend\vspace{0.5em}
\pstart
           Seit drei Wochen muß ich hier die \label{K_L02872-1v}\edtext{\textsc{Dreyfus\pwindex{Dreyfus, Alfred 1859-10-09 – 1935-07-12@\textsc{Dreyfus, Alfred} (1859-10-09 – 1935-07-12), \emph{Militär/Militärin}|pw}-}\begin{otherlanguage}{french}\textsc{Enquête}\end{otherlanguage}}{\lemma{\textnormal{\emph{Dreyfus-Enquête}}}\Cendnote{\textnormal{»\begin{otherlanguage}{french}Enquête\end{otherlanguage}« meint hier die laufenden Untersuchungen zur Affäre Dreyfus\pwindex{Dreyfus, Alfred 1859-10-09 – 1935-07-12@\textsc{Dreyfus, Alfred} (1859-10-09 – 1935-07-12), \emph{Militär/Militärin}|pwk}, die am 3. 6. 1899
                  zu einer Aufhebung des Urteils vom 22. 12. 1894
                  führten. Am 8. 8. 1899 begann für Dreyfus\pwindex{Dreyfus, Alfred 1859-10-09 – 1935-07-12@\textsc{Dreyfus, Alfred} (1859-10-09 – 1935-07-12), \emph{Militär/Militärin}|pwk} ein neuer Kriegsgerichtsprozess.}}}\label{K_L02872-1} bearbeiten.
               Das bedeutet: täglich um 7 Uhr aufſtehen (um den ungeheuren Stoff zu
               bewältigen) und bis Nachmittags durcharbeiten. Wenn ich mit dieſem Tagespenſum fertig
               bin, bin ich ſo todtmüde, daß ich zu nichts mehr Kraft habe, nicht einmal zu einem
               Briefe an Dich. Die Folge iſt, daß ich nun ſchon Wochen lang ohne Nachricht von Dir
               bin. Gerade in dieſer Zeit iſt mir das beſonders ſchmerzlich. Ich ſende Dir alſo
                  heut (in Erwartung des Tages, wo ich Zeit haben
               werde, Dir ausführlicher zu ſchreiben) dieſe wenigen Zeilen, um Dich zu bitten, mir
               ein Wort über Dein Ergehen zu ſchreiben, {\pb}ſei es
               auch nur eine Poſtkarte. Und wenn Du zu Deiner \label{K_L02872-2v}\edtext{\begin{otherlanguage}{french}\textsc{Première\pwindex{gruene Kakadu – Paracelsus – Die Gefaehrtin. Drei Einakter@\emph{Der grüne Kakadu – Paracelsus – Die Gefährtin. Drei Einakter}|pwv}}\end{otherlanguage}}{\lemma{\textnormal{\emph{Première}}}\Cendnote{\textnormal{Schnitzler war für den Zeitraum vom 25. 4. 1899 bis zum 2. 5. 1899 aus Anlass der
                  Premiere von \emph{Der grüne Kakadu –
                     Paracelsus – Die Gefährtin. Drei Einakter}\pwindex{gruene Kakadu – Paracelsus – Die Gefaehrtin. Drei Einakter@\emph{Der grüne Kakadu – Paracelsus – Die Gefährtin. Drei Einakter}|pwk}  nach Berlin\oindex{Berlin@\textbf{Berlin}, \emph{P.PPLC}|pwk} gereist. Diese fand
                  am 29. 4. 1899 am \emph{Deutschen Theater}\orgindex{Deutsches Theater Berlin@Deutsches Theater Berlin|pwk} statt.}}}\label{K_L02872-2} am Samſtag nach Berlin\oindex{Berlin@\textbf{Berlin}, \emph{P.PPLC}|pw} gehſt, ſo bitte
               ich Dich recht, recht herzlich, auf dem Hinwege oder Rückwege \strikeout{d\textcolor{gray}{en}}{ }\label{K_L02872-3v}\edtext{über Frankfurt\oindex{Frankfurt am Main@\textbf{Frankfurt am Main}, \emph{P.PPLA3}|pw}}{\lemma{\textnormal{\emph{über Frankfurt}}}\Cendnote{\textnormal{Dazu kam es nicht.}}}\label{K_L02872-3} zu kommen. Laß’
               Dich die Eiſenbahnfahrt nicht verdrießen! Du wirſt Dich hier ausruhen und erholen.
               Wohnen kannſt Du nicht bei mir, aber alle Mahlzeiten nimmſt Du ſelbſtverſtändlich mit
                  \strikeout{\textcolor{gray}{mi}t} mir ein. Auch die Meinigen würden ſich Alle ſehr mit
               Dir freuen. Bitte, komm!\pend
           
\pstart
           Viele treue Grüße! {\\[\baselineskip]}Dein {\\[\baselineskip]}\spacefill\mbox{Paul Goldmann.}\pend
           \leftskip=0em{}
\pstart
           \noindent{}Wir leſen hier die »Fackel\pwindex{Fackel@\emph{Die Fackel}|pw}«. Ein ſchönes
                  Saublatt. Aber \label{K_L02872-4v}\edtext{mit \textsc{Julius Bauer\pwindex{Bauer, Julius 15.10.1853 – 11.06.1941@\textsc{Bauer, Julius} (15.10.1853 – 11.06.1941), \emph{Schriftsteller/Schriftstellerin, Journalist/Journalistin, Kritiker/Kritikerin}|pw}} hat er\pwindex{Kraus, Karl 28.04.1874 – 12.06.1936@\textsc{Kraus, Karl} (28.04.1874 – 12.06.1936), \emph{Schriftsteller/Schriftstellerin, Publizist/Publizistin, Schriftsteller/Schriftstellerin}|pwv} Recht}{\lemma{\textnormal{\emph{mit … Recht}}}\Cendnote{\textnormal{Bereits in der ersten Ausgabe der \emph{Fackel}\pwindex{Fackel@\emph{Die Fackel}|pwk}, die Anfang April 1899 erschienen war, polemisierte Karl Kraus\pwindex{Kraus, Karl 28.04.1874 – 12.06.1936@\textsc{Kraus, Karl} (28.04.1874 – 12.06.1936), \emph{Schriftsteller/Schriftstellerin, Publizist/Publizistin, Schriftsteller/Schriftstellerin}|pwk} gegen Julius Bauer\pwindex{Bauer, Julius 15.10.1853 – 11.06.1941@\textsc{Bauer, Julius} (15.10.1853 – 11.06.1941), \emph{Schriftsteller/Schriftstellerin, Journalist/Journalistin, Kritiker/Kritikerin}|pwk}.
                     Vgl. Karl Kraus\pwindex{Kraus, Karl 28.04.1874 – 12.06.1936@\textsc{Kraus, Karl} (28.04.1874 – 12.06.1936), \emph{Schriftsteller/Schriftstellerin, Publizist/Publizistin, Schriftsteller/Schriftstellerin}|pwk}: \emph{Die Vertreibung aus dem Paradiese}\pwindex{Vertreibung aus dem Paradiese@\emph{Die Vertreibung aus dem Paradiese}|pwk}. In: \emph{Die Fackel}\pwindex{Fackel@\emph{Die Fackel}|pwk}, Jg. 1, Nr. 1, Anfang April 1899, S. 12–23.}}}\label{K_L02872-4}.\pend
           \selectlanguage{ngerman}\endnumbering\briefempfaengerindex{Schnitzler, Arthur@\textsc{Schnitzler, Arthur}!zzzGoldmann, Paul@\emph{von Paul Goldmann}!1899-04-261@{26. 4. 1899}|)be}\mylabel{L02872h}  \normalsize

\doendnotes{C}
\bigskip
\vfill

\clearpage

\footnotesize

\lohead{\textsc{register}}

% Definiere theindex-Environment komplett neu ohne reledmac
\makeatletter
\renewenvironment{theindex}{%
  \section*{\indexname}%
  \setlength{\parindent}{0pt}%
  \setlength{\parskip}{0pt plus 0.3pt}%
  \let\item\@idxitem
}{%
  \clearpage
}
\makeatother

\IfFileExists{\jobname-pw.ind}{\input{\jobname-pw.ind}}{}

\end{document}

      