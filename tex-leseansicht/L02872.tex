%% latex-leseansicht-vorspann.tex
%% Vorspann für die Leseansicht.
%% Lädt die gemeinsame Datei latex-vorspann.tex mit nicht gesetztem Schalter.

\newif\ifkorrekturansicht
\korrekturansichtfalse

\input{../tex-inputs/latex-vorspann}


         
         \renewcommand{\erwaehntePersonen}{Personen: Julius Bauer, Alfred Dreyfus, Paul Goldmann, Karl Kraus}
         \renewcommand{\erwaehnteInstitutionen}{Institutionen: Frankfurter Zeitung}
         \renewcommand{\erwaehnteOrte}{Orte: Berlin, Deutsches Theater Berlin, Frankfurt am Main}
         \renewcommand{\erwaehnteWerke}{Werke: Der grüne Kakadu – Paracelsus – Die Gefährtin. Drei Einakter, Die Fackel, Die Vertreibung aus dem Paradiese}
               \section[ Paul Goldmann an Arthur Schnitzler, 26. 4. 1899]{ Paul Goldmann an Arthur Schnitzler, 26. 4. 1899}\nopagebreak\mylabel{v}\rehead{ }\begin{ledgroupsized}[t]{13cm}\normalsize\beginnumbering \toendnotes[C]{\smallbreak\pagebreak[2]} \Standort{DLA, A:Schnitzler, HS.NZ85.1.3169.}
\physDesc{Brief, 1 Blatt, 2 Seiten, 1217 Zeichen
\newline{}Handschrift: schwarze Tinte, deutsche Kurrent}\toendnotes[C]{\smallbreak}\pstart
           \noindent{}{\pb}\textcolor{gray}{\textbf{\textbf{Frankfurter Zeitung}}}\orgindex{Frankfurter Zeitung@Frankfurter Zeitung|pw}\hfill \textcolor{gray}{\textbf{\textbf{Frankfurt a. M.\oindex{Frankfurt am Main@\textbf{Frankfurt am Main}|pw},}}}{ }26. April \textcolor{gray}{\textbf{189}}9.\pend
           \pstart
           \textcolor{gray}{\textbf{und}}\pend
           \pstart
           \textcolor{gray}{\textbf{Handelsblatt.}}\pend
           \pstart
           \textcolor{gray}{\textbf{\textbf{Redaktion\orgindex{Frankfurter Zeitung@Frankfurter Zeitung|pwv}.}\footnote{\noindent{}\textcolor{gray}{\textbf{Für die Redaktion\orgindex{Frankfurter Zeitung@Frankfurter Zeitung|pwv} beſtimmte Briefe und Sendungen wolle man
                                 \so{nicht} an die Perſon eines Redakteurs,
                              ſondern ſtets \textbf{an die Redaktion der Frankfurter Zeitung\orgindex{Frankfurter Zeitung@Frankfurter Zeitung|pw}} adreſſiren.}}}}}\pend
           \pstart
           \textcolor{gray}{\textbf{Telegramm-Adreſſe:}}\pend
           \pstart
           \textcolor{gray}{\textbf{\textbf{Zeitung\orgindex{Frankfurter Zeitung@Frankfurter Zeitung|pwv}{ }Frankfurt Main\oindex{Frankfurt am Main@\textbf{Frankfurt am Main}|pw}.}}}\pend
           \pstart{}Mein lieber Freund,\pend\pstart
           Seit drei Wochen muß ich hier die \label{K_L02872-1v}\edtext{\textsc{Dreyfus\pwindex{Dreyfus, Alfred 1859-10-09 – 1935-07-12@\textsc{Dreyfus, Alfred} (1859-10-09 – 1935-07-12), \emph{Militär}|pw}-}\begin{otherlanguage}{french}\textsc{Enquête}\end{otherlanguage}}{\lemma{\textnormal{\emph{Dreyfus-Enquête}}}\Cendnote{\textnormal{»\begin{otherlanguage}{french}Enquête\end{otherlanguage}« meint hier die laufenden Untersuchungen zur Affäre Dreyfus\pwindex{Dreyfus, Alfred 1859-10-09 – 1935-07-12@\textsc{Dreyfus, Alfred} (1859-10-09 – 1935-07-12), \emph{Militär}|pwk}, die am 3. 6. 1899
                  zu einer Aufhebung des Urteils vom 22. 12. 1894
                  führten. Am 8. 8. 1899 begann für Dreyfus\pwindex{Dreyfus, Alfred 1859-10-09 – 1935-07-12@\textsc{Dreyfus, Alfred} (1859-10-09 – 1935-07-12), \emph{Militär}|pwk} ein neuer Kriegsgerichtsprozess.}}}\label{K_L02872-1h} bearbeiten.
               Das bedeutet: täglich um 7 Uhr aufſtehen (um den ungeheuren Stoff zu
               bewältigen) und bis Nachmittags durcharbeiten. Wenn ich mit dieſem Tagespenſum fertig
               bin, bin ich ſo todtmüde, daß ich zu nichts mehr Kraft habe, nicht einmal zu einem
               Briefe an Dich. Die Folge iſt, daß ich nun ſchon Wochen lang ohne Nachricht von Dir
               bin. Gerade in dieſer Zeit iſt mir das beſonders ſchmerzlich. Ich ſende Dir alſo
                  heut (in Erwartung des Tages, wo ich Zeit haben
               werde, Dir ausführlicher zu ſchreiben) dieſe wenigen Zeilen, um Dich zu bitten, mir
               ein Wort über Dein Ergehen zu ſchreiben, {\pb}ſei es
               auch nur eine Poſtkarte. Und wenn Du zu Deiner \label{K_L02872-2v}\edtext{\begin{otherlanguage}{french}\textsc{Première\pwindex{Schnitzler, Arthur 15.05.1862 – 21.10.1931@\textsc{Schnitzler, Arthur} (15.05.1862 – 21.10.1931), \emph{Schriftsteller, Mediziner}!gruene Kakadu – Paracelsus – Die Gefaehrtin. Drei Einakter1898 – 1899@\strich\emph{Der grüne Kakadu – Paracelsus – Die Gefährtin. Drei Einakter} {[}1898 – 1899{]}|pwv}}\end{otherlanguage}}{\lemma{\textnormal{\emph{Première}}}\Cendnote{\textnormal{Schnitzler\pwindex{Schnitzler, Arthur 15.05.1862 – 21.10.1931@\textsc{Schnitzler, Arthur} (15.05.1862 – 21.10.1931), \emph{Schriftsteller, Mediziner}|pwk} war von 25. 4. 1899 bis 2. 5. 1899 für die
                  Premiere von Der grüne Kakadu –
                     Paracelsus – Die Gefährtin. Drei Einakter\pwindex{Schnitzler, Arthur 15.05.1862 – 21.10.1931@\textsc{Schnitzler, Arthur} (15.05.1862 – 21.10.1931), \emph{Schriftsteller, Mediziner}!gruene Kakadu – Paracelsus – Die Gefaehrtin. Drei Einakter1898 – 1899@\strich\emph{Der grüne Kakadu – Paracelsus – Die Gefährtin. Drei Einakter} {[}1898 – 1899{]}|pwkv} (am 29. 4. 1899) am Deutschen Theater\oindex{Deutsches Theater Berlin@\textbf{Deutsches Theater Berlin}|pwk} nach Berlin\oindex{Berlin@\textbf{Berlin}|pwk} gereist.}}}\label{K_L02872-2h} am Samſtag nach Berlin\oindex{Berlin@\textbf{Berlin}|pw} gehſt, ſo bitte
               ich Dich recht, recht herzlich, auf dem Hinwege oder Rückwege \strikeout{d\textcolor{gray}{en}}{ }\label{K_L02872-3v}\edtext{über Frankfurt\oindex{Frankfurt am Main@\textbf{Frankfurt am Main}|pw}}{\lemma{\textnormal{\emph{über Frankfurt}}}\Cendnote{\textnormal{nicht geschehen}}}\label{K_L02872-3h} zu kommen. Laß’
               Dich die Eiſenbahnfahrt nicht verdrießen! Du wirſt Dich hier ausruhen und erholen.
               Wohnen kannſt Du nicht bei mir, aber alle Mahlzeiten nimmſt Du ſelbſtverſtändlich mit
                  \strikeout{\textcolor{gray}{mi}t} mir ein. Auch die Meinigen würden ſich Alle ſehr mit
               Dir freuen. Bitte, komm!\pend
           \pstart
           Viele treue Grüße! {\\[\baselineskip]}Dein {\\[\baselineskip]}\spacefill\mbox{Paul Goldmann.}\pend
           \leftskip=0em{}\pstart
           \noindent{}Wir leſen hier die »Fackel\pwindex{Fackel1899-04 – 1936@\emph{Die Fackel} {[}1899-04 – 1936{]}|pw}«. Ein ſchönes
                  Saublatt. Aber \label{K_L02872-4v}\edtext{mit \textsc{Julius Bauer\pwindex{Bauer, Julius 15.10.1853 – 11.06.1941@\textsc{Bauer, Julius} (15.10.1853 – 11.06.1941), \emph{Schriftsteller, Journalist, Kritiker}|pw}} hat er\pwindex{Kraus, Karl 28.04.1874 – 12.06.1936@\textsc{Kraus, Karl} (28.04.1874 – 12.06.1936), \emph{Schriftsteller, Publizist}|pwv} Recht}{\lemma{\textnormal{\emph{mit … Recht}}}\Cendnote{\textnormal{Bereits in der ersten Ausgabe der \emph{Fackel}\pwindex{Fackel1899-04 – 1936@\emph{Die Fackel} {[}1899-04 – 1936{]}|pwk}, die Anfang April 1899 erschien, polemisierte Karl Kraus\pwindex{Kraus, Karl 28.04.1874 – 12.06.1936@\textsc{Kraus, Karl} (28.04.1874 – 12.06.1936), \emph{Schriftsteller, Publizist}|pwk} gegen Julius Bauer\pwindex{Bauer, Julius 15.10.1853 – 11.06.1941@\textsc{Bauer, Julius} (15.10.1853 – 11.06.1941), \emph{Schriftsteller, Journalist, Kritiker}|pwk}.
                     Vgl. Karl Kraus\pwindex{Kraus, Karl 28.04.1874 – 12.06.1936@\textsc{Kraus, Karl} (28.04.1874 – 12.06.1936), \emph{Schriftsteller, Publizist}|pwk}: \emph{Die Vertreibung aus dem Paradiese}\pwindex{Kraus, Karl 28.04.1874 – 12.06.1936@\textsc{Kraus, Karl} (28.04.1874 – 12.06.1936), \emph{Schriftsteller, Publizist}!Vertreibung aus dem Paradiese1899-04@\strich\emph{Die Vertreibung aus dem Paradiese} {[}1899-04{]}|pwk}. In: \emph{Die Fackel}\pwindex{Fackel1899-04 – 1936@\emph{Die Fackel} {[}1899-04 – 1936{]}|pwk}, Jg. 1, Nr. 1, Anfang April 1899, S. 12–23.}}}\label{K_L02872-4h}.\pend
           
         
         \endnumbering\mylabel{h}\end{ledgroupsized}  \newcommand{\dateiname}{L02872}\newcommand{\titel}{Paul Goldmann an Arthur Schnitzler, 26. 4. 1899}\newcommand{\editorInnen}{Martin Anton Müller und Laura Untner}%% latex-leseansicht-abspann.tex
%% Abspann für die Leseansicht.
%% Der Schalter \ifkorrekturansicht ist bereits durch den Vorspann gesetzt.

%% latex-abspann.tex
%% Gemeinsamer Abspann für Korrekturansicht und Leseansicht.
%% Setzt den Schalter \ifkorrekturansicht voraus (gesetzt in den
%% einbindenden Dateien latex-korrekturansicht-abspann.tex bzw.
%% latex-leseansicht-abspann.tex).
%% ---------------------------------------------------------------

\normalsize

% Das esempio-Environment wird nur in der Leseansicht benötigt
\ifkorrekturansicht\else
\newenvironment{esempio}[3]%
{
    \vspace{1.5ex}
    \rlap{\underline{#1}}
    \par
    \setlength{\parindent}{0cm}
    \nopagebreak
    \leftskip=#2cm
    \rightskip=#3cm
}
{
    \par
}
\fi

\doendnotes{C}
\bigskip
\vfill

\clearpage

\footnotesize

\ifkorrekturansicht
  \lohead{\textsc{register}}
\fi

% theindex-Environment neu definieren ohne reledmac
\makeatletter
\renewenvironment{theindex}{%
  \ifkorrekturansicht
    \section*{\indexname}%
  \else
    \subsubsection*{Index der erwähnten Entitäten}%
  \fi
  \setlength{\parindent}{0pt}%
  \setlength{\parskip}{0pt plus 0.3pt}%
  \let\item\@idxitem
}{%
  \ifkorrekturansicht\clearpage\fi
}
\makeatother

\IfFileExists{\jobname-pw.ind}{\input{\jobname-pw.ind}}{}

% Quellenangabe nur in der Leseansicht
\ifkorrekturansicht\else
% Fallback-Definitionen, falls die .tex-Datei \titel etc. nicht gesetzt hat
\providecommand{\titel}{}
\providecommand{\editorInnen}{}
\providecommand{\dateiname}{\jobname}

\vspace{3cm}

\vfill

\footnotesize
\textsc{Quelle}: \titel. Herausgegeben von {\editorInnen}. In: \emph{Arthur Schnitzler: Briefwechsel mit Autorinnen und Autoren}.
 Digitale Edition, https://schnitzler-briefe.acdh.oeaw.ac.at/{\dateiname}.html (Stand \today)
\fi

\end{document}


      