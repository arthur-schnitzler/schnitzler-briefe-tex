%% latex-leseansicht-vorspann.tex
%% Vorspann für die Leseansicht.
%% Lädt die gemeinsame Datei latex-vorspann.tex mit nicht gesetztem Schalter.

\newif\ifkorrekturansicht
\korrekturansichtfalse

\input{../tex-inputs/latex-vorspann}


\section[ Paul Goldmann an Arthur Schnitzler, 17. 9. {[}1903?{]}]{L03387 Paul Goldmann an Arthur Schnitzler,  17. 9. [1903?]}
\nopagebreak\mylabel{L03387v}
\rehead{ }\normalsize\beginnumbering\briefempfaengerindex{Schnitzler, Arthur@\textsc{Schnitzler, Arthur}!zzzGoldmann, Paul@\emph{von Paul Goldmann}!1903-09-171@{17. [9. 1903?]}|(be}
\toendnotes[C]{\smallbreak\pagebreak[2]}
\correspDesc{Versand  durch Paul Goldmann am 17. [9. 1903?] in Wien
\newline{}Erhalt  durch Arthur Schnitzler am 17. [9. 1903?] in Wien}\toendnotes[C]{\smallbreak}
\Standort{DLA, A:Schnitzler, HS.NZ85.1.3173.}
\physDesc{Postkarte, 348 Zeichen
\newline{}Handschrift: schwarze Tinte, deutsche Kurrent
\newline{}Versand: Stempel: »\nobreak{}\oindex{Wien@\textbf{Wien}, \emph{Verwaltungsgebiet}|pwk}Wien 15, 17. \textcolor{gray}{I}{ }{[}X. 03{]}, 2 10 N\nobreak{}«. Stempel: »\nobreak{}\oindex{XVIII., Währing@\textbf{XVIII., Währing}, \emph{Verwaltungsgebiet}|pwk}Wien 18 111, 17. \textcolor{gray}{IX}{ }{[}03{]}, 3 10 N\nobreak{}«.  
\newline{}Schnitzler: mit Bleistift durch fehlerhafte Entzifferung des Stempels falsch datiert:
                                       »3/10 903.« }\toendnotes[C]{\smallbreak}\pstart{}\textsc{{\pb}Herrn}\pend{}\pstart{}\textsc{Dr. Arthur Schnitzler}\pend{}\pstart{}\textsc{XVIII. Spöttelgaſse 7\oindex{Wien@\textbf{Wien}!XVIII., Währing@\textbf{XVIII., Währing}!Edmund-Weiß-Gasse 7@\textbf{Edmund-Weiß-Gasse 7}, \emph{Wohngebäude}|pw}}\pend{}{\bigskip}\vspace{1em}
\pstart
           \raggedleft{}{\pb}Donnerſtag\pend
           
\pstart{}Mein lieber Freund,\pend\vspace{0.5em}
\pstart
           Ich habe für heut{ }Abend eine \strikeout{Loge}{ }Loge im »\label{K_L03387-1v}\edtext{Theater an der Wien\oindex{Wien@\textbf{Wien}!VI., Mariahilf@\textbf{VI., Mariahilf}!Theater an der Wien@\textbf{Theater an der Wien}, \emph{Theater}|pw}}{\lemma{\textnormal{\emph{Theater an der Wien}}}\Cendnote{\textnormal{Am 17. 9. 1903 wurde im Theater an der Wien\oindex{Wien@\textbf{Wien}!VI., Mariahilf@\textbf{VI., Mariahilf}!Theater an der Wien@\textbf{Theater an der Wien}, \emph{Theater}|pwk} die Operette \emph{Venedig in Paris}\pwindex{Siraudin, Paul 18.\,12.\,1812 Paris – 8.\,9.\,1883 Enghien-les-Bains@\textsc{Siraudin, Paul} (18.\,12.\,1812 Paris – 8.\,9.\,1883 Enghien-les-Bains), \emph{Schriftsteller, Librettist}!Venedig in Paris. Operette in drei Akten@\strich\emph{Venedig in Paris. Operette in drei Akten}|pwk}\pwindex{Moinaux, Jules 25.\,10.\,1825 Tours – 6.\,12.\,1895 Paris@\textsc{Moinaux, Jules} (25.\,10.\,1825 Tours – 6.\,12.\,1895 Paris), \emph{Schriftsteller, Librettist}!Venedig in Paris. Operette in drei Akten@\strich\emph{Venedig in Paris. Operette in drei Akten}|pwk} (Musik: Jacques Offenbach\pwindex{Offenbach, Jacques 20.\,6.\,1819 Köln – 5.\,10.\,1880 Paris@\textsc{Offenbach, Jacques} (20.\,6.\,1819 Köln – 5.\,10.\,1880 Paris), \emph{Komponist}|pwk}, Libretto: Paul
                     Siraudin\pwindex{Siraudin, Paul 18.\,12.\,1812 Paris – 8.\,9.\,1883 Enghien-les-Bains@\textsc{Siraudin, Paul} (18.\,12.\,1812 Paris – 8.\,9.\,1883 Enghien-les-Bains), \emph{Schriftsteller, Librettist}|pwk} und Jules Moinaux\pwindex{Moinaux, Jules 25.\,10.\,1825 Tours – 6.\,12.\,1895 Paris@\textsc{Moinaux, Jules} (25.\,10.\,1825 Tours – 6.\,12.\,1895 Paris), \emph{Schriftsteller, Librettist}|pwk})
                  gegeben. Schnitzlers{ }\emph{Tagebuch}\pwindex{Schnitzler, Arthur 15.\,5.\,1862 Wien – 21.\,10.\,1931 ebd.@\textsc{Schnitzler, Arthur} (15.\,5.\,1862 Wien – 21.\,10.\,1931 ebd.), \emph{Schriftsteller, Mediziner}!Tagebuch@\strich\emph{Tagebuch}|pwk} enthält für diesen Tag keinen Eintrag, auch die
                  Aufstellung der Theaterbesuche (\emph{CUL}, A 179a) erwähnt die Aufführung nicht, sodass
                  es eher unwahrscheinlich ist, dass Schnitzler der Einladung Folge leistete.}}}\label{K_L03387-1}« (Balkonloge I. \textsc{Gallerie} links \textsc{N\textsuperscript{o}} 2). Ich bitte Dich und Deine Frau\pwindex{Schnitzler, Olga 17.\,1.\,1882 Wien – 13.\,1.\,1970 Lugano@\textsc{Schnitzler, Olga} (17.\,1.\,1882 Wien – 13.\,1.\,1970 Lugano), \emph{Schauspielerin, Sängerin}|pwv}, auch hinzukommen, – umſomehr, als dies für mich vielleicht die einzige
               Möglichkeit iſt, Dich jetzt \label{K_L03387-2v}\edtext{noch
                  einmal}{\lemma{\textnormal{\emph{noch
                  einmal}}}\Cendnote{\textnormal{Siehe XXXX Auszeichnungsfehler: Dokument L03386 nicht gefunden.
               }}}\label{K_L03387-2} zu{ }ſehen.\pend
           
\pstart
           Herzlichſt Dein {\\[\baselineskip]}\spacefill\mbox{Paul Goldmann.}\pend
           \leftskip=0em{}\selectlanguage{ngerman}\endnumbering\briefempfaengerindex{Schnitzler, Arthur@\textsc{Schnitzler, Arthur}!zzzGoldmann, Paul@\emph{von Paul Goldmann}!1903-09-171@{17. [9. 1903?]}|)be}\mylabel{L03387h}  \newcommand{\dateiname}{L03387}\newcommand{\titel}{Paul Goldmann an Arthur Schnitzler, 17. 9. [1903?]}\newcommand{\editorInnen}{Martin Anton Müller und Laura Untner}%% latex-leseansicht-abspann.tex
%% Abspann für die Leseansicht.
%% Der Schalter \ifkorrekturansicht ist bereits durch den Vorspann gesetzt.

%% latex-abspann.tex
%% Gemeinsamer Abspann für Korrekturansicht und Leseansicht.
%% Setzt den Schalter \ifkorrekturansicht voraus (gesetzt in den
%% einbindenden Dateien latex-korrekturansicht-abspann.tex bzw.
%% latex-leseansicht-abspann.tex).
%% ---------------------------------------------------------------

\normalsize

% Das esempio-Environment wird nur in der Leseansicht benötigt
\ifkorrekturansicht\else
\newenvironment{esempio}[3]%
{
    \vspace{1.5ex}
    \rlap{\underline{#1}}
    \par
    \setlength{\parindent}{0cm}
    \nopagebreak
    \leftskip=#2cm
    \rightskip=#3cm
}
{
    \par
}
\fi

\doendnotes{C}
\bigskip
\vfill

\clearpage

\footnotesize

\ifkorrekturansicht
  \lohead{\textsc{register}}
\fi

% theindex-Environment neu definieren ohne reledmac
\makeatletter
\renewenvironment{theindex}{%
  \ifkorrekturansicht
    \section*{\indexname}%
  \else
    \subsubsection*{Index der erwähnten Entitäten}%
  \fi
  \setlength{\parindent}{0pt}%
  \setlength{\parskip}{0pt plus 0.3pt}%
  \let\item\@idxitem
}{%
  \ifkorrekturansicht\clearpage\fi
}
\makeatother

\IfFileExists{\jobname-pw.ind}{\input{\jobname-pw.ind}}{}

% Quellenangabe nur in der Leseansicht
\ifkorrekturansicht\else
% Fallback-Definitionen, falls die .tex-Datei \titel etc. nicht gesetzt hat
\providecommand{\titel}{}
\providecommand{\editorInnen}{}
\providecommand{\dateiname}{\jobname}

\vspace{3cm}

\vfill

\footnotesize
\textsc{Quelle}: \titel. Herausgegeben von {\editorInnen}. In: \emph{Arthur Schnitzler: Briefwechsel mit Autorinnen und Autoren}.
 Digitale Edition, https://schnitzler-briefe.acdh.oeaw.ac.at/{\dateiname}.html (Stand \today)
\fi

\end{document}


