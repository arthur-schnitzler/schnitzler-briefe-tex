%% latex-leseansicht-vorspann.tex
%% Vorspann für die Leseansicht.
%% Lädt die gemeinsame Datei latex-vorspann.tex mit nicht gesetztem Schalter.

\newif\ifkorrekturansicht
\korrekturansichtfalse

\input{../tex-inputs/latex-vorspann}

\begin{center}
            \textcolor{red}{ENTWURF, NICHT FERTIG KORRIGIERT}
                      \end{center}
            
         
         \renewcommand{\erwaehntePersonen}{Personen: Harry Nicholls, Olga Schnitzler, James Tolman Tanner}
         \renewcommand{\erwaehnteOrte}{Orte: Edmund-Weiß-Gasse, Theater an der Wien, Wien}
         \renewcommand{\erwaehnteWerke}{Werke: The Toreador}
               \section[ Paul Goldmann an Arthur Schnitzler, 2. 10. {[}1903{]}]{ Paul Goldmann an Arthur Schnitzler, 2. 10. {[}1903{]}}\nopagebreak\mylabel{v}\rehead{ }\begin{ledgroupsized}[t]{13cm}\normalsize\beginnumbering \toendnotes[C]{\smallbreak\pagebreak[2]} \Standort{DLA, A:Schnitzler, HS.NZ85.1.3173.}
\physDesc{Postkarte
\newline{}Handschrift: 1) schwarze Tinte, deutsche Kurrent\hspace{1em}2) schwarze Tinte, lateinische Kurrent (\noindent{}Adresse)\hspace{1em}\newline{}Versand: Stempel: »\nobreak{}Wien 15 {[}r{]}, 2 10, N 17 \textcolor{gray}{1}{[}5{]}\nobreak{}«. Stempel: »\nobreak{}Wien 1\textcolor{gray}{8} 111 r, 3 10, V 17\nobreak{}«.  
\newline{}Schnitzler: mit Bleistift das Empfangsdatum »3/10 {[}1{]}903.« vermerkt }\toendnotes[C]{\smallbreak}\pstart{}{\pb}Herrn\pend{}\pstart{}Dr. Arthur Schnitzler\pend{}\pstart{}XVIII. Spöttelgaſse 7\oindex{Edmund-Weiss-Gasse@\textbf{Edmund-Weiß-Gasse}|pw}\pend{}{\bigskip}\pstart
           \raggedleft{}{\pb}Donnerſtag\pend
           \pstart{}Mein lieber Freund,\pend\pstart
           Ich habe für heut{ }Abend eine \strikeout{\textsc{Log\textcolor{gray}{e}}}{ }\textsc{Loge} im »\label{K_L03387-1v}\edtext{Theater an der Wien\oindex{Theater an der Wien@\textbf{Theater an der Wien}|pw}}{\lemma{\textnormal{\emph{Theater an der Wien}}}\Cendnote{\textnormal{Am Abend des 2. 10. 1903 wurde im Theater an der Wien\oindex{Theater an der Wien@\textbf{Theater an der Wien}|pwk} das Musical \emph{Der
                     Toreador}\pwindex{Tanner, James Tolman 1858-10-17 – 1915-06-18@\textsc{Tanner, James Tolman} (1858-10-17 – 1915-06-18), \emph{Regisseur, Dramatiker}!Toreador1901-06-17@\strich\emph{The Toreador} {[}1901-06-17{]}|pwk}\pwindex{Nicholls, Harry 1852-03-01 – 1926-11-29@\textsc{Nicholls, Harry} (1852-03-01 – 1926-11-29), \emph{Schauspieler, Dramatiker, Komiker}!Toreador1901-06-17@\strich\emph{The Toreador} {[}1901-06-17{]}|pwk} (engl. \emph{The Toreador}\pwindex{Tanner, James Tolman 1858-10-17 – 1915-06-18@\textsc{Tanner, James Tolman} (1858-10-17 – 1915-06-18), \emph{Regisseur, Dramatiker}!Toreador1901-06-17@\strich\emph{The Toreador} {[}1901-06-17{]}|pwk}\pwindex{Nicholls, Harry 1852-03-01 – 1926-11-29@\textsc{Nicholls, Harry} (1852-03-01 – 1926-11-29), \emph{Schauspieler, Dramatiker, Komiker}!Toreador1901-06-17@\strich\emph{The Toreador} {[}1901-06-17{]}|pwk}, James T. Tanner\pwindex{Tanner, James Tolman 1858-10-17 – 1915-06-18@\textsc{Tanner, James Tolman} (1858-10-17 – 1915-06-18), \emph{Regisseur, Dramatiker}|pwk} und Harry Nicholls\pwindex{Nicholls, Harry 1852-03-01 – 1926-11-29@\textsc{Nicholls, Harry} (1852-03-01 – 1926-11-29), \emph{Schauspieler, Dramatiker, Komiker}|pwk}, 1901) gegeben.
                     Schnitzler\pwindex{Schnitzler, Arthur 15.05.1862 – 21.10.1931@\textsc{Schnitzler, Arthur} (15.05.1862 – 21.10.1931), \emph{Schriftsteller, Mediziner}|pwk} kam nicht mit (vgl. A. S.: \emph{Tagebuch}, 2. 10. 1903).}}}\label{K_L03387-1h}«
               (Balkonloge I. \textsc{Gallerie} links \textsc{N\textsuperscript{o}} 2). Ich bitte Dich und Deine Frau\pwindex{Schnitzler, Olga 17.01.1882 – 13.01.1970@\textsc{Schnitzler, Olga} (17.01.1882 – 13.01.1970), \emph{Schauspielerin, Sängerin}|pwv}, auch hinzukommen, – umſomehr, als dies für mich vielleicht die einzige
               Möglichkeit iſt, Dich jetzt \label{K_L03387-2v}\edtext{noch
                  einmal}{\lemma{\textnormal{\emph{noch
                  einmal}}}\Cendnote{\textnormal{siehe Paul Goldmann an Arthur Schnitzler, 7. 9. 1903}}}\label{K_L03387-2h} zu ſehen. Herzlichſt Dein \spacefill\mbox{Paul
                  Goldmann}\pend
           
         
         \endnumbering\mylabel{h}\end{ledgroupsized}\begin{anhang}\end{anhang}\newcommand{\dateiname}{L03387}\newcommand{\titel}{Paul Goldmann an Arthur Schnitzler, 2. 10. [1903]}\newcommand{\editorInnen}{Martin Anton Müller und Laura Untner}%% latex-leseansicht-abspann.tex
%% Abspann für die Leseansicht.
%% Der Schalter \ifkorrekturansicht ist bereits durch den Vorspann gesetzt.

%% latex-abspann.tex
%% Gemeinsamer Abspann für Korrekturansicht und Leseansicht.
%% Setzt den Schalter \ifkorrekturansicht voraus (gesetzt in den
%% einbindenden Dateien latex-korrekturansicht-abspann.tex bzw.
%% latex-leseansicht-abspann.tex).
%% ---------------------------------------------------------------

\normalsize

% Das esempio-Environment wird nur in der Leseansicht benötigt
\ifkorrekturansicht\else
\newenvironment{esempio}[3]%
{
    \vspace{1.5ex}
    \rlap{\underline{#1}}
    \par
    \setlength{\parindent}{0cm}
    \nopagebreak
    \leftskip=#2cm
    \rightskip=#3cm
}
{
    \par
}
\fi

\doendnotes{C}
\bigskip
\vfill

\clearpage

\footnotesize

\ifkorrekturansicht
  \lohead{\textsc{register}}
\fi

% theindex-Environment neu definieren ohne reledmac
\makeatletter
\renewenvironment{theindex}{%
  \ifkorrekturansicht
    \section*{\indexname}%
  \else
    \subsubsection*{Index der erwähnten Entitäten}%
  \fi
  \setlength{\parindent}{0pt}%
  \setlength{\parskip}{0pt plus 0.3pt}%
  \let\item\@idxitem
}{%
  \ifkorrekturansicht\clearpage\fi
}
\makeatother

\IfFileExists{\jobname-pw.ind}{\input{\jobname-pw.ind}}{}

% Quellenangabe nur in der Leseansicht
\ifkorrekturansicht\else
% Fallback-Definitionen, falls die .tex-Datei \titel etc. nicht gesetzt hat
\providecommand{\titel}{}
\providecommand{\editorInnen}{}
\providecommand{\dateiname}{\jobname}

\vspace{3cm}

\vfill

\footnotesize
\textsc{Quelle}: \titel. Herausgegeben von {\editorInnen}. In: \emph{Arthur Schnitzler: Briefwechsel mit Autorinnen und Autoren}.
 Digitale Edition, https://schnitzler-briefe.acdh.oeaw.ac.at/{\dateiname}.html (Stand \today)
\fi

\end{document}


      