%% latex-leseansicht-vorspann.tex
%% Vorspann für die Leseansicht.
%% Lädt die gemeinsame Datei latex-vorspann.tex mit nicht gesetztem Schalter.

\newif\ifkorrekturansicht
\korrekturansichtfalse

\input{../tex-inputs/latex-vorspann}

\begin{center}
            \textcolor{red}{ENTWURF, NICHT FERTIG KORRIGIERT}
                      \end{center}
            
         
         \renewcommand{\erwaehntePersonen}{Personen: Paul Goldmann, Jules Moinaux, Jacques Offenbach, Olga Schnitzler, Paul Siraudin}
         \renewcommand{\erwaehnteOrte}{Orte: Edmund-Weiß-Gasse 7, Theater an der Wien, Wien, XVIII., Währing}
         \renewcommand{\erwaehnteWerke}{Werke: Tagebuch, Venedig in Paris. Operette in drei Akten}
               \section[ Paul Goldmann an Arthur Schnitzler, 17. 9. {[}1903?{]}]{ Paul Goldmann an Arthur Schnitzler, 17. 9. {[}1903?{]}}\nopagebreak\mylabel{v}\rehead{ }\begin{ledgroupsized}[t]{13cm}\normalsize\beginnumbering\briefempfaengerindex{Schnitzler, Arthur@\textsc{Schnitzler, Arthur}!zzzGoldmann, Paul@\emph{von Paul Goldmann}!1903-09-171@{17. {[}9. 1903?{]}}|(be} \toendnotes[C]{\smallbreak\pagebreak[2]} \Standort{DLA, A:Schnitzler, HS.NZ85.1.3173.}
\physDesc{Postkarte, 349 Zeichen
\newline{}Handschrift: 1) schwarze Tinte, deutsche Kurrent\hspace{1em}2) schwarze Tinte, lateinische Kurrent (\noindent{}Adresse)\hspace{1em}
\newline{}Versand: Stempel: »\nobreak{}Wien 15, 17. {[}9. 1903{]}, 2\textsuperscript{10}N\nobreak{}«. Stempel: »\nobreak{}\oindex{XVIII., Waehring@\textbf{XVIII., Währing}|pwk}Wien 18 111, 17. {[}9. 1903{]}, 3\textsuperscript{10}N\nobreak{}«.  
\newline{}Schnitzler: mit Bleistift durch fehlerhafte Entzifferung des Stempels falsch
                                 datiert: »3/10 903.« }\toendnotes[C]{\smallbreak}\pstart{}{\pb}Herrn\pend{}\pstart{}Dr. Arthur Schnitzler\pend{}\pstart{}XVIII. Spöttelgaſse 7\oindex{Edmund-Weiss-Gasse 7@\textbf{Edmund-Weiß-Gasse 7}|pw}\pend{}{\bigskip}\pstart
           \raggedleft{}{\pb}Donnerſtag\pend
           \pstart{}Mein lieber Freund,\pend\pstart
           Ich habe für heut{ }Abend eine \strikeout{Log\textcolor{gray}{e}}{ }Loge im »\label{K_L03387-1v}\edtext{Theater an der Wien\oindex{Theater an der Wien@\textbf{Theater an der Wien}|pw}}{\lemma{\textnormal{\emph{Theater an der Wien}}}\Cendnote{\textnormal{Am 17. 9. 1903 wurde im Theater an der Wien\oindex{Theater an der Wien@\textbf{Theater an der Wien}|pwk} die Operette \emph{Venedig in Paris}\pwindex{Offenbach, Jacques 20.06.1819 – 05.10.1880@\textsc{Offenbach, Jacques} (20.06.1819 – 05.10.1880), \emph{Komponist}!Venedig in Paris. Operette in drei Akten@\strich\emph{Venedig in Paris. Operette in drei Akten} {[}Vertonung{]}|pwk}\pwindex{Siraudin, Paul 1812-12-18 – 1883-09-08@\textsc{Siraudin, Paul} (1812-12-18 – 1883-09-08), \emph{Schriftsteller, Librettist}!Venedig in Paris. Operette in drei Akten@\strich\emph{Venedig in Paris. Operette in drei Akten}|pwk} (Musik Jacques Offenbach\pwindex{Offenbach, Jacques 20.06.1819 – 05.10.1880@\textsc{Offenbach, Jacques} (20.06.1819 – 05.10.1880), \emph{Komponist}|pwk}, Libretto Paul
                     Siraudin\pwindex{Siraudin, Paul 1812-12-18 – 1883-09-08@\textsc{Siraudin, Paul} (1812-12-18 – 1883-09-08), \emph{Schriftsteller, Librettist}|pwk} und Jules Moinaux\pwindex{Moinaux, Jules 25.10.1825 – 06.12.1895@\textsc{Moinaux, Jules} (25.10.1825 – 06.12.1895), \emph{Schriftsteller, Librettist}|pwk})
                  gegeben. Schnitzler\pwindex{Schnitzler, Arthur 15.05.1862 – 21.10.1931@\textsc{Schnitzler, Arthur} (15.05.1862 – 21.10.1931), \emph{Schriftsteller, Mediziner}|pwk}s \emph{Tagebuch}\pwindex{\textcolor{red}{\textsuperscript{XXXX1 indx}}!Tagebuch1981 – 2000@\strich\emph{Tagebuch} {[}Hrsg., 1981 – 2000{]}|pwk} enthält für diesen Tag keinen Eintrag, auch die
                  Aufstellung der Theaterbesuche (\emph{CUL}, A 179a) erwähnt die Aufführung nicht, so
                  dass es eher unwahrscheinlich ist, dass Schnitzler\pwindex{Schnitzler, Arthur 15.05.1862 – 21.10.1931@\textsc{Schnitzler, Arthur} (15.05.1862 – 21.10.1931), \emph{Schriftsteller, Mediziner}|pwk} der Einladung Folge leistete.}}}\label{K_L03387-1h}« (Balkonloge I. \textsc{Gallerie} links \textsc{N\textsuperscript{o}} 2). Ich bitte Dich und Deine Frau\pwindex{Schnitzler, Olga 17.01.1882 – 13.01.1970@\textsc{Schnitzler, Olga} (17.01.1882 – 13.01.1970), \emph{Schauspielerin, Sängerin}|pwv}, auch hinzukommen, – umſomehr, als dies für mich vielleicht die einzige
               Möglichkeit iſt, Dich jetzt \label{K_L03387-2v}\edtext{noch
                  einmal}{\lemma{\textnormal{\emph{noch
                  einmal}}}\Cendnote{\textnormal{siehe Paul Goldmann an Arthur Schnitzler, 7. 9. 1903}}}\label{K_L03387-2h} zu ſehen. Herzlichſt Dein \spacefill\mbox{Paul Goldmann.}\pend
           
         
         \endnumbering\mylabel{h}\end{ledgroupsized}\begin{anhang}\end{anhang}\newcommand{\dateiname}{L03387}\newcommand{\titel}{Paul Goldmann an Arthur Schnitzler, 17. 9. [1903?]}\newcommand{\editorInnen}{Martin Anton Müller und Laura Untner}%% latex-leseansicht-abspann.tex
%% Abspann für die Leseansicht.
%% Der Schalter \ifkorrekturansicht ist bereits durch den Vorspann gesetzt.

%% latex-abspann.tex
%% Gemeinsamer Abspann für Korrekturansicht und Leseansicht.
%% Setzt den Schalter \ifkorrekturansicht voraus (gesetzt in den
%% einbindenden Dateien latex-korrekturansicht-abspann.tex bzw.
%% latex-leseansicht-abspann.tex).
%% ---------------------------------------------------------------

\normalsize

% Das esempio-Environment wird nur in der Leseansicht benötigt
\ifkorrekturansicht\else
\newenvironment{esempio}[3]%
{
    \vspace{1.5ex}
    \rlap{\underline{#1}}
    \par
    \setlength{\parindent}{0cm}
    \nopagebreak
    \leftskip=#2cm
    \rightskip=#3cm
}
{
    \par
}
\fi

\doendnotes{C}
\bigskip
\vfill

\clearpage

\footnotesize

\ifkorrekturansicht
  \lohead{\textsc{register}}
\fi

% theindex-Environment neu definieren ohne reledmac
\makeatletter
\renewenvironment{theindex}{%
  \ifkorrekturansicht
    \section*{\indexname}%
  \else
    \subsubsection*{Index der erwähnten Entitäten}%
  \fi
  \setlength{\parindent}{0pt}%
  \setlength{\parskip}{0pt plus 0.3pt}%
  \let\item\@idxitem
}{%
  \ifkorrekturansicht\clearpage\fi
}
\makeatother

\IfFileExists{\jobname-pw.ind}{\input{\jobname-pw.ind}}{}

% Quellenangabe nur in der Leseansicht
\ifkorrekturansicht\else
% Fallback-Definitionen, falls die .tex-Datei \titel etc. nicht gesetzt hat
\providecommand{\titel}{}
\providecommand{\editorInnen}{}
\providecommand{\dateiname}{\jobname}

\vspace{3cm}

\vfill

\footnotesize
\textsc{Quelle}: \titel. Herausgegeben von {\editorInnen}. In: \emph{Arthur Schnitzler: Briefwechsel mit Autorinnen und Autoren}.
 Digitale Edition, https://schnitzler-briefe.acdh.oeaw.ac.at/{\dateiname}.html (Stand \today)
\fi

\end{document}


      