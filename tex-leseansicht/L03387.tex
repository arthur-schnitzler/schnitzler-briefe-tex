%% latex-korrekturansicht-vorspann.tex
%% Vorspann für die Korrekturansicht.
%% Lädt die gemeinsame Datei latex-vorspann.tex mit gesetztem Schalter.

\newif\ifkorrekturansicht
\korrekturansichttrue

\input{../tex-inputs/latex-vorspann}


\section[ Paul Goldmann an Arthur Schnitzler, 17. 9. {[}1903?{]}]{L03387 Paul Goldmann an Arthur Schnitzler, 17. 9. {[}1903?{]}}
\nopagebreak\mylabel{L03387v}
\rehead{ }\normalsize\beginnumbering\briefempfaengerindex{Schnitzler, Arthur@\textsc{Schnitzler, Arthur}!zzzGoldmann, Paul@\emph{von Paul Goldmann}!1903-09-171@{17. {[}9. 1903?{]}}|(be}
\toendnotes[C]{\smallbreak\pagebreak[2]}\Standort{DLA, A:Schnitzler, HS.NZ85.1.3173.}
\physDesc{Postkarte, 348 Zeichen
\newline{}Handschrift: 1) schwarze Tinte, deutsche Kurrent\hspace{1em}2) schwarze Tinte, lateinische Kurrent (\noindent{}Adresse)\hspace{1em}
\newline{}Versand: Stempel: »\nobreak{}Wien 15, 17. \textcolor{gray}{I}{ }{[}X. 03{]}, 2 10 N\nobreak{}«. Stempel: »\nobreak{}\oindex{XVIII., Waehring@\textbf{XVIII., Währing}, \emph{A.ADM3}|pwk}Wien 18 111, 17. \textcolor{gray}{IX}{ }{[}03{]}, 3 10 N\nobreak{}«.  
\newline{}Schnitzler: mit Bleistift durch fehlerhafte Entzifferung des Stempels falsch datiert:
                                       »3/10 903.« }\toendnotes[C]{\smallbreak}\pstart{}{\pb}Herrn\pend{}\pstart{}Dr. Arthur Schnitzler\pend{}\pstart{}XVIII. Spöttelgaſse 7\oindex{Edmund-Weiss-Gasse 7@\textbf{Edmund-Weiß-Gasse 7}, \emph{Wohngebäude (K.WHS)}|pw}\pend{}{\bigskip}\vspace{1em}
\pstart
           \raggedleft{}{\pb}Donnerſtag\pend
           
\pstart{}Mein lieber Freund,\pend\vspace{0.5em}
\pstart
           Ich habe für heut{ }Abend eine \strikeout{Loge}{ }Loge im »\label{K_L03387-1v}\edtext{Theater an der Wien\oindex{Theater an der Wien@\textbf{Theater an der Wien}, \emph{Theater (K.THE)}|pw}}{\lemma{\textnormal{\emph{Theater an der Wien}}}\Cendnote{\textnormal{Am 17. 9. 1903 wurde im Theater an der Wien\oindex{Theater an der Wien@\textbf{Theater an der Wien}, \emph{Theater (K.THE)}|pwk} die Operette \emph{Venedig in Paris}\pwindex{Venedig in Paris. Operette in drei Akten@\emph{Venedig in Paris. Operette in drei Akten}|pwk} (Musik: Jacques Offenbach\pwindex{Offenbach, Jacques 20.06.1819 – 05.10.1880@\textsc{Offenbach, Jacques} (20.06.1819 – 05.10.1880), \emph{Komponist/Komponistin}|pwk}, Libretto: Paul
                     Siraudin\pwindex{Siraudin, Paul 1812-12-18 – 1883-09-08@\textsc{Siraudin, Paul} (1812-12-18 – 1883-09-08), \emph{Schriftsteller/Schriftstellerin, Librettist/Librettistin}|pwk} und Jules Moinaux\pwindex{Moinaux, Jules 25.10.1825 – 06.12.1895@\textsc{Moinaux, Jules} (25.10.1825 – 06.12.1895), \emph{Schriftsteller/Schriftstellerin, Librettist/Librettistin}|pwk})
                  gegeben. Schnitzlers{ }\emph{Tagebuch}\pwindex{Tagebuch@\emph{Tagebuch}|pwk} enthält für diesen Tag keinen Eintrag, auch die
                  Aufstellung der Theaterbesuche (\emph{CUL}, A 179a) erwähnt die Aufführung nicht, sodass
                  es eher unwahrscheinlich ist, dass Schnitzler der Einladung Folge leistete.}}}\label{K_L03387-1}« (Balkonloge I. \textsc{Gallerie} links \textsc{N\textsuperscript{o}} 2). Ich bitte Dich und Deine Frau\pwindex{Schnitzler, Olga 17.01.1882 – 13.01.1970@\textsc{Schnitzler, Olga} (17.01.1882 – 13.01.1970), \emph{Schauspieler/Schauspielerin, Sänger/Sängerin}|pwv}, auch hinzukommen, – umſomehr, als dies für mich vielleicht die einzige
               Möglichkeit iſt, Dich jetzt \label{K_L03387-2v}\edtext{noch
                  einmal}{\lemma{\textnormal{\emph{noch
                  einmal}}}\Cendnote{\textnormal{Siehe Paul Goldmann an Arthur Schnitzler, 7. 9. 1903.
               }}}\label{K_L03387-2} zu ſehen.\pend
           
\pstart
           Herzlichſt Dein {\\[\baselineskip]}\spacefill\mbox{Paul Goldmann.}\pend
           \leftskip=0em{}\selectlanguage{ngerman}\endnumbering\briefempfaengerindex{Schnitzler, Arthur@\textsc{Schnitzler, Arthur}!zzzGoldmann, Paul@\emph{von Paul Goldmann}!1903-09-171@{17. {[}9. 1903?{]}}|)be}\mylabel{L03387h}  \normalsize

\doendnotes{C}
\bigskip
\vfill

\clearpage

\footnotesize

\lohead{\textsc{register}}

% Definiere theindex-Environment komplett neu ohne reledmac
\makeatletter
\renewenvironment{theindex}{%
  \section*{\indexname}%
  \setlength{\parindent}{0pt}%
  \setlength{\parskip}{0pt plus 0.3pt}%
  \let\item\@idxitem
}{%
  \clearpage
}
\makeatother

\IfFileExists{\jobname-pw.ind}{\input{\jobname-pw.ind}}{}

\end{document}

      