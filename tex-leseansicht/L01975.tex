%% latex-korrekturansicht-vorspann.tex
%% Vorspann für die Korrekturansicht.
%% Lädt die gemeinsame Datei latex-vorspann.tex mit gesetztem Schalter.

\newif\ifkorrekturansicht
\korrekturansichttrue

\input{../tex-inputs/latex-vorspann}


\section[Arthur Schnitzler an Hugo von Hofmannsthal, 2. 11. 1910]{L01975 Arthur Schnitzler an Hugo von Hofmannsthal, 2. 11. 1910}
\nopagebreak\mylabel{L01975v}
\rehead{ }\normalsize\beginnumbering\briefempfaengerindex{Hofmannsthal, Hugo von@\textsc{Hofmannsthal, Hugo von}!zzzSchnitzler, Arthur@\emph{von Arthur Schnitzler}!1910-11-021@{2. 11. 1910}|(be}
\toendnotes[C]{\smallbreak\pagebreak[2]}\Standort{FDH, Hs-30885,140.}
\physDesc{Brief, Durchschlag1 Blatt, 2 Seiten, 2106 Zeichen
\newline{}Schreibmaschine
\newline{}Handschrift: 1) Bleistift, deutsche Kurrent (\noindent{}Beschriftung: »\textsc{Hofmannsth}«)\hspace{1em}2) roter Buntstift, deutsche Kurrent (\noindent{}zwei Unterstreichungen)\hspace{1em}
\newline{}Ordnung: 1) Schnitzler dürfte dieses Korrespondenzstück bei der Durchsicht
                                 der Briefe 1929 aus seiner Ablage zu den Briefen aus
                                 Hofmannsthals Nachlass hinzugefügt haben  2) Lochung}
\buchAbdrucke{\weitereDrucke{1) Hugo von Hofmannsthal, Arthur Schnitzler: \emph{Briefwechsel}. Frankfurt am Main: \emph{S. Fischer} 1964, S. 256.} \weitereDrucke{2) Arthur Schnitzler: \emph{Briefe 1875–1912}. Frankfurt am Main: \emph{S. Fischer} 1981, S. 631–632.} }\toendnotes[C]{\smallbreak}
\pstart
           \centering{}{\pb}Schluss des Briefes vom 2. 11. 1910.\pend
           \vspace{0.5em}
\pstart
           Zu Ihrer Nachschrift habe ich einiges zu bemerken. Dass Ihnen ein Buch\pwindex{Weg ins Freie. Roman@\emph{Der Weg ins Freie. Roman}|pwv} von mir künstlerisch oder menschlich
               zuwider ist, das ist Ihr gutes Recht. Dass Sie es mir sagen Ihre Pflicht. Wie Sie
               sich zu Andern darüber äussern, Sache Ihres Temperaments und Ihres Geschmacks. Aber
               dass Sie irgend ein Buch von mir, Ihnen persönlich zugeeignet, lieber Hugo, »halb
               zufällig halb absichtlich in der Bahn liegen lassen« und dass Sie es notwendig finden
               mir zwei Jahre nachher \strikeout{mir} davon Mitteilung zu
               machen, das scheint mir für einen Spass nicht lustig genug und ernst genommen völlig
               unvereinbar mit unseren künstlerischen und menschlichen Beziehungen, wie ich sie
               bisher gesehen habe. Drängte es Sie so sehr den Eindruck von damals nachzuprüfen, so
               hatten Sie es leicht genug diese Absicht durchzuführen, ohne gerade von dem Autor
               selbst ein zweites Exemplar zu erbitten, dem Autor, gegen dessen erste
               freundschaftliche Widmung Sie sich in einer so wenig üblichen Weise betragen haben,
               wie es mir dem verwerflichsten Produkte eines Unbekannten gegenüber niemals einge{\pb}fallen wäre, der mir die Höflichkeit einer Dedikation
               erwiesen. Aber wenn Ihr auf Neuerwerbung dieses Buches\pwindex{Weg ins Freie. Roman@\emph{Der Weg ins Freie. Roman}|pwv} abzielender Wunsch, der ja gewiss liebenswürdig und
               taktvoll gemeint war, Ihrer feinen Feder wie unter einem dämonischen Zwang so ganz
               ins Gegenteil geraten musste, so ist mir das ein Beweis, dass die gewiss nichts
               weniger als oberflächlichen Gründe für Ihr unglückliches Verhältnis zu meinem Roman\pwindex{Weg ins Freie. Roman@\emph{Der Weg ins Freie. Roman}|pwv} auch heute noch
               fortbestehen und ein Versuch von Ihrer Seite sich zu dieser persönlichsten meiner
               Schöpfungen in ein neues Verhältnis zu setzen vorläufig nur wenig Aussicht auf Erfolg
               haben dürften. Und ehe ich mein Kind\pwindex{Weg ins Freie. Roman@\emph{Der Weg ins Freie. Roman}|pwv}, wie Sie den Roman\pwindex{Weg ins Freie. Roman@\emph{Der Weg ins Freie. Roman}|pwv} mit einer fast über das Bild hinausgehenden Richtigkeit bezeichnen,
               zum zweiten Mal der Gefahr eines \label{K_L01975-1v}\edtext{meskinen}{\lemma{\textnormal{\emph{meskinen}}}\Cendnote{\textnormal{französisch \begin{otherlanguage}{french}mesquin\end{otherlanguage}:
                  dürftig, knauserig}}}\label{K_L01975-1} Eisenbahnunfalls aussetzen möchte, ziehe ich es doch vor
               es weiter im Unfrieden mit Ihnen leben zu lassen, ein Zustand, bei dem Sie sich
               meines Wissens geradeso wohl befunden haben wie das liebe Kind\pwindex{Weg ins Freie. Roman@\emph{Der Weg ins Freie. Roman}|pwv} und dessen getreuer Vater, der Sie wie
               immer herzlichst grüsst als\pend
           \pstart Ihr\pend{}\selectlanguage{ngerman}\endnumbering\briefempfaengerindex{Hofmannsthal, Hugo von@\textsc{Hofmannsthal, Hugo von}!zzzSchnitzler, Arthur@\emph{von Arthur Schnitzler}!1910-11-021@{2. 11. 1910}|)be}\mylabel{L01975h}  \normalsize

\doendnotes{C}
\bigskip
\vfill

\clearpage

\footnotesize

\lohead{\textsc{register}}

% Definiere theindex-Environment komplett neu ohne reledmac
\makeatletter
\renewenvironment{theindex}{%
  \section*{\indexname}%
  \setlength{\parindent}{0pt}%
  \setlength{\parskip}{0pt plus 0.3pt}%
  \let\item\@idxitem
}{%
  \clearpage
}
\makeatother

\IfFileExists{\jobname-pw.ind}{\input{\jobname-pw.ind}}{}

\end{document}

      