%% latex-leseansicht-vorspann.tex
%% Vorspann für die Leseansicht.
%% Lädt die gemeinsame Datei latex-vorspann.tex mit nicht gesetztem Schalter.

\newif\ifkorrekturansicht
\korrekturansichtfalse

\input{../tex-inputs/latex-vorspann}


         
         \newcommand{\erwaehntePersonen}{Personen: Hugo von Hofmannsthal}
         \newcommand{\erwaehnteOrte}{Orte: Wien}
         \newcommand{\erwaehnteWerke}{Werke: Der Weg ins Freie. Roman}
               \section[Arthur Schnitzler an Hugo von Hofmannsthal, 2. 11. 1910]{ Arthur Schnitzler an Hugo von Hofmannsthal, 2. 11. 1910}\nopagebreak\mylabel{v}\rehead{ }\begin{ledgroupsized}[t]{13cm}\normalsize\beginnumbering \toendnotes[C]{\smallbreak\pagebreak[2]} \Standort{FDH, Hs-30885,140.}
\physDesc{Brief, 1 Blatt, 2 Seiten, maschineller Durchschlag
\newline{}Schreibmaschine
\newline{}Handschrift: 1) Bleistift, deutsche Kurrent (\noindent{}Beschriftung: »\textsc{Hofmannsth}«)\hspace{1em}2) roter Buntstift, deutsche Kurrent (\noindent{}zwei Unterstreichungen)\hspace{1em}\newline{}Ordnung: 1) Schnitzler dürfte dieses Korrespondenzstück bei der Durchsicht
                                 der Briefe 1929 aus seiner Ablage zu den Briefen aus
                                 Hofmannsthals Nachlass hinzugefügt haben  2) Lochung}\buchAbdrucke{\weitereDrucke{1) Hugo von Hofmannsthal, Arthur Schnitzler: \emph{Briefwechsel}. Hg. Therese Nickl und Heinrich Schnitzler. Frankfurt am Main: \emph{S. Fischer} 1964, S. 256.} \weitereDrucke{2) Arthur Schnitzler: \emph{Briefe 1875–1912}. Hg. Therese Nickl und Heinrich Schnitzler. Frankfurt am Main: \emph{S. Fischer} 1981, S. 631–632.} }\toendnotes[C]{\smallbreak}\pstart
           \noindent{}\centering{}{\pb}Schluss des Briefes vom 2. 11. 1910.\pend
           \pstart
           Zu Ihrer Nachschrift habe ich einiges zu bemerken. Dass Ihnen ein Buch\pwindex{Schnitzler, Arthur 15.05.1862 – 21.10.1931@\textsc{Schnitzler, Arthur} (15.05.1862 – 21.10.1931), \emph{Schriftsteller, Mediziner}!Weg ins Freie. Roman1.1.1908 – 1.6.1908@\strich\emph{Der Weg ins Freie. Roman} {[}1.1.1908 – 1.6.1908{]}|pwv} von mir künstlerisch oder menschlich
               zuwider ist, das ist Ihr gutes Recht. Dass Sie es mir sagen Ihre Pflicht. Wie Sie
               sich zu Andern darüber äussern, Sache Ihres Temperaments und Ihres Geschmacks. Aber
               dass Sie irgend ein Buch von mir, Ihnen persönlich zugeeignet, lieber Hugo, »halb
               zufällig halb absichtlich in der Bahn liegen lassen« und dass Sie es notwendig finden
               mir zwei Jahre nachher \strikeout{mir} davon Mitteilung zu
               machen, das scheint mir für einen Spass nicht lustig genug und ernst genommen völlig
               unvereinbar mit unseren künstlerischen und menschlichen Beziehungen, wie ich sie
               bisher gesehen habe. Drängte es Sie so sehr den Eindruck von damals nachzuprüfen, so
               hatten Sie es leicht genug diese Absicht durchzuführen, ohne gerade von dem Autor
               selbst ein zweites Exemplar zu erbitten, dem Autor, gegen dessen erste
               freundschaftliche Widmung Sie sich in einer so wenig üblichen Weise betragen haben,
               wie es mir dem verwerflichsten Produkte eines Unbekannten gegenüber niemals einge{\pb}fallen wäre, der mir die Höflichkeit einer Dedikation
               erwiesen. Aber wenn Ihr auf Neuerwerbung dieses Buches\pwindex{Schnitzler, Arthur 15.05.1862 – 21.10.1931@\textsc{Schnitzler, Arthur} (15.05.1862 – 21.10.1931), \emph{Schriftsteller, Mediziner}!Weg ins Freie. Roman1.1.1908 – 1.6.1908@\strich\emph{Der Weg ins Freie. Roman} {[}1.1.1908 – 1.6.1908{]}|pwv} abzielender Wunsch, der ja gewiss liebenswürdig und
               taktvoll gemeint war, Ihrer feinen Feder wie unter einem dämonischen Zwang so ganz
               ins Gegenteil geraten musste, so ist mir das ein Beweis, dass die gewiss nichts
               weniger als oberflächlichen Gründe für Ihr unglückliches Verhältnis zu meinem Roman\pwindex{Schnitzler, Arthur 15.05.1862 – 21.10.1931@\textsc{Schnitzler, Arthur} (15.05.1862 – 21.10.1931), \emph{Schriftsteller, Mediziner}!Weg ins Freie. Roman1.1.1908 – 1.6.1908@\strich\emph{Der Weg ins Freie. Roman} {[}1.1.1908 – 1.6.1908{]}|pwv} auch heute noch fortbestehen
               und ein Versuch von Ihrer Seite sich zu dieser persönlichsten meiner Schöpfungen in
               ein neues Verhältnis zu setzen vorläufig nur wenig Aussicht auf Erfolg haben dürften.
               Und ehe ich mein Kind\pwindex{Schnitzler, Arthur 15.05.1862 – 21.10.1931@\textsc{Schnitzler, Arthur} (15.05.1862 – 21.10.1931), \emph{Schriftsteller, Mediziner}!Weg ins Freie. Roman1.1.1908 – 1.6.1908@\strich\emph{Der Weg ins Freie. Roman} {[}1.1.1908 – 1.6.1908{]}|pwv}, wie Sie
               den Roman\pwindex{Schnitzler, Arthur 15.05.1862 – 21.10.1931@\textsc{Schnitzler, Arthur} (15.05.1862 – 21.10.1931), \emph{Schriftsteller, Mediziner}!Weg ins Freie. Roman1.1.1908 – 1.6.1908@\strich\emph{Der Weg ins Freie. Roman} {[}1.1.1908 – 1.6.1908{]}|pwv} mit einer fast über das
               Bild hinausgehenden Richtigkeit bezeichnen, zum zweiten Mal der Gefahr eines \label{K_L01975_1v}\edtext{meskinen}{\lemma{\textnormal{\emph{meskinen}}}\Cendnote{\textnormal{französisch mesquin: dürftig, knauserig}}}\label{K_L01975_1h}
               Eisenbahnunfalls aussetzen möchte, ziehe ich es doch vor es weiter im Unfrieden mit
               Ihnen leben zu lassen, ein Zustand, bei dem Sie sich meines Wissens geradeso wohl
               befunden haben wie das liebe Kind\pwindex{Schnitzler, Arthur 15.05.1862 – 21.10.1931@\textsc{Schnitzler, Arthur} (15.05.1862 – 21.10.1931), \emph{Schriftsteller, Mediziner}!Weg ins Freie. Roman1.1.1908 – 1.6.1908@\strich\emph{Der Weg ins Freie. Roman} {[}1.1.1908 – 1.6.1908{]}|pwv}
               und dessen getreuer Vater, der Sie wie immer herzlichst grüsst als\pend
           \pstart Ihr\pend{}
         
         \endnumbering\mylabel{h}\end{ledgroupsized}  \newcommand{\dateiname}{L01975}\newcommand{\titel}{Arthur Schnitzler an Hugo von Hofmannsthal, 2. 11. 1910}\newcommand{\editorInnen}{Martin Anton Müller und Gerd-Hermann Susen}%% latex-leseansicht-abspann.tex
%% Abspann für die Leseansicht.
%% Der Schalter \ifkorrekturansicht ist bereits durch den Vorspann gesetzt.

%% latex-abspann.tex
%% Gemeinsamer Abspann für Korrekturansicht und Leseansicht.
%% Setzt den Schalter \ifkorrekturansicht voraus (gesetzt in den
%% einbindenden Dateien latex-korrekturansicht-abspann.tex bzw.
%% latex-leseansicht-abspann.tex).
%% ---------------------------------------------------------------

\normalsize

% Das esempio-Environment wird nur in der Leseansicht benötigt
\ifkorrekturansicht\else
\newenvironment{esempio}[3]%
{
    \vspace{1.5ex}
    \rlap{\underline{#1}}
    \par
    \setlength{\parindent}{0cm}
    \nopagebreak
    \leftskip=#2cm
    \rightskip=#3cm
}
{
    \par
}
\fi

\doendnotes{C}
\bigskip
\vfill

\clearpage

\footnotesize

\ifkorrekturansicht
  \lohead{\textsc{register}}
\fi

% theindex-Environment neu definieren ohne reledmac
\makeatletter
\renewenvironment{theindex}{%
  \ifkorrekturansicht
    \section*{\indexname}%
  \else
    \subsubsection*{Index der erwähnten Entitäten}%
  \fi
  \setlength{\parindent}{0pt}%
  \setlength{\parskip}{0pt plus 0.3pt}%
  \let\item\@idxitem
}{%
  \ifkorrekturansicht\clearpage\fi
}
\makeatother

\IfFileExists{\jobname-pw.ind}{\input{\jobname-pw.ind}}{}

% Quellenangabe nur in der Leseansicht
\ifkorrekturansicht\else
% Fallback-Definitionen, falls die .tex-Datei \titel etc. nicht gesetzt hat
\providecommand{\titel}{}
\providecommand{\editorInnen}{}
\providecommand{\dateiname}{\jobname}

\vspace{3cm}

\vfill

\footnotesize
\textsc{Quelle}: \titel. Herausgegeben von {\editorInnen}. In: \emph{Arthur Schnitzler: Briefwechsel mit Autorinnen und Autoren}.
 Digitale Edition, https://schnitzler-briefe.acdh.oeaw.ac.at/{\dateiname}.html (Stand \today)
\fi

\end{document}


      