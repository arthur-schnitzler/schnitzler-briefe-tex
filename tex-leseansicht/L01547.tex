%% latex-korrekturansicht-vorspann.tex
%% Vorspann für die Korrekturansicht.
%% Lädt die gemeinsame Datei latex-vorspann.tex mit gesetztem Schalter.

\newif\ifkorrekturansicht
\korrekturansichttrue

\input{../tex-inputs/latex-vorspann}


\section[Arthur Schnitzler an Hermann Bahr, 17. 9. 1905]{L01547 Arthur Schnitzler an Hermann Bahr, 17. 9. 1905}
\nopagebreak\mylabel{L01547v}
\rehead{ }\normalsize\beginnumbering\briefempfaengerindex{Bahr, Hermann@\textsc{Bahr, Hermann}!zzzSchnitzler, Arthur@\emph{von Arthur Schnitzler}!1905-09-171@{17. 9. 1905}|(be}
\toendnotes[C]{\smallbreak\pagebreak[2]}\Standort{TMW, HS AM 23376 Ba.}
\physDesc{Brief, 1 Blatt, 3 Seiten, 849 Zeichen
\newline{}Handschrift: schwarze Tinte, deutsche Kurrent
\newline{}Ordnung: Lochung }
\buchAbdrucke{\weitereDrucke{1) Arthur Schnitzler: \emph{Briefe 1875–1912}. Frankfurt am Main: \emph{S. Fischer} 1981, S. 516–517.} \weitereDrucke{2) Arthur Schnitzler: \emph{The Letters of Arthur Schnitzler to Hermann Bahr}. Chapel Hill: \emph{The University of North Carolina Press} 1978, S. 90–91.} \weitereDrucke{3) Hermann Bahr, Arthur Schnitzler: \emph{Briefwechsel, Aufzeichnungen, Dokumente (1891–1931)}. Göttingen: \emph{Wallstein} 2018, S. 351.} }\toendnotes[C]{\smallbreak}
\pstart
           \raggedleft{}{\pb}17. 9. 905\pend
           \vspace{0.5em}
\pstart
           lieber Hermann, für den Fall, dſs ich dich nicht zu Hauſe treffe,
               ſchreibe ich \damage{d}ir gleich.\pend
           
\pstart
           Das gedruckte Stück »\label{K_L01547-1v}\edtext{Zwiſchenſpiel\pwindex{Zwischenspiel. Komoedie in drei Akten@\emph{Zwischenspiel. Komödie in drei Akten}|pw}}{\lemma{\textnormal{\emph{Zwiſchenſpiel}}}\Cendnote{\textnormal{Entsprechend dürfte die erste
                  Buchausgabe auf 1906 vordatiert sein: Arthur Schnitzler: \emph{Das Zwischenspiel. Komödie in drei Akten}\pwindex{Zwischenspiel. Komoedie in drei Akten@\emph{Zwischenspiel. Komödie in drei Akten}|pwk}. Berlin: \emph{S. Fischer}\orgindex{S. Fischer Verlag@S. Fischer Verlag|pwk}{ }1906.}}}\label{K_L01547-1}« und »Der Ruf des Lebens\pwindex{Zwischenspiel. Komoedie in drei Akten@\emph{Zwischenspiel. Komödie in drei Akten}|pw}«
               liegen hier bei.\pend
           
\pstart
           Über das erſtere iſt weiter nichts zu ſagen; lies es bitte und betrachte es im
               übrigen vorläufig ſorgfältg als \textsc{\uline{Mscrpt}}.\pend
           
\pstart
           Am »Ruf des Lebens\pwindex{Ruf des Lebens. Schauspiel in drei Akten@\emph{Der Ruf des Lebens. Schauspiel in drei Akten}|pw}« ist noch einiges weniges zu
               machen. Ich bring es {\pb}dir ſchon heute, weil ich die Frage an dich richten möchte, ob du die \uline{Widmung} des Buches annehmen willſt? Es iſt vielleicht
               in dem Stück eine Ahnung von dem \label{K_L01547-2v}\edtext{Wunsch erfüllſt, den du anläßlich des Puppenſpielers\pwindex{Puppenspieler. Studie in einem Aufzuge@\emph{Der Puppenspieler. Studie in einem Aufzuge}|pw} oeffentlich ausſprachſt}{\lemma{\textnormal{\emph{Wunsch … ausſprachſt}}}\Cendnote{\textnormal{Vgl. Arthur Schnitzler an Hermann Bahr, 14. 12. 1904 und Hermann Bahr, Arthur Schnitzler: \emph{Briefwechsel, Aufzeichnungen, Dokumente (1891–1931)}, Hermann Bahr: Der Puppenspieler, 13. 12. 1904.
               }}}\label{K_L01547-2}. –\pend
           
\pstart
           Schreib mir bitte ein Wort, wa{\geminationn} wir zuſa{\geminationm}en ſein könnten. Möchteſ\damage{t} du nicht einmal bei uns nachtmahlen? Auch meine Frau\pwindex{Schnitzler, Olga 17.01.1882 – 13.01.1970@\textsc{Schnitzler, Olga} (17.01.1882 – 13.01.1970), \emph{Schauspieler/Schauspielerin, Sänger/Sängerin}|pwv} würde ſich ſoſehr freuen. Oder wenn dir
               die Spöttelgaſſe\oindex{Edmund-Weiss-Gasse 7@\textbf{Edmund-Weiß-Gasse 7}, \emph{Wohngebäude (K.WHS)}|pw} unbe{\pb}quem, Hietzing\oindex{XIII., Hietzing@\textbf{XIII., Hietzing}, \emph{A.ADM3}|pw}? Man ſieht einander doch gar zu wenig! Ich grüße dich
               herzlich.\pend
           
\pstart
           Dein{\\[\baselineskip]}\spacefill\mbox{A.}\pend
           \leftskip=0em{}\selectlanguage{ngerman}\endnumbering\briefempfaengerindex{Bahr, Hermann@\textsc{Bahr, Hermann}!zzzSchnitzler, Arthur@\emph{von Arthur Schnitzler}!1905-09-171@{17. 9. 1905}|)be}\mylabel{L01547h}  \normalsize

\doendnotes{C}
\bigskip
\vfill

\clearpage

\footnotesize

\lohead{\textsc{register}}

% Definiere theindex-Environment komplett neu ohne reledmac
\makeatletter
\renewenvironment{theindex}{%
  \section*{\indexname}%
  \setlength{\parindent}{0pt}%
  \setlength{\parskip}{0pt plus 0.3pt}%
  \let\item\@idxitem
}{%
  \clearpage
}
\makeatother

\IfFileExists{\jobname-pw.ind}{\input{\jobname-pw.ind}}{}

\end{document}

      