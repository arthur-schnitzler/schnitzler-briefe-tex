%% latex-leseansicht-vorspann.tex
%% Vorspann für die Leseansicht.
%% Lädt die gemeinsame Datei latex-vorspann.tex mit nicht gesetztem Schalter.

\newif\ifkorrekturansicht
\korrekturansichtfalse

\input{../tex-inputs/latex-vorspann}


\section[Arthur Schnitzler an Hermann Bahr, 17. 9. 1905]{L01547 Arthur Schnitzler an Hermann Bahr, 17. 9. 1905}
\nopagebreak\mylabel{L01547v}
\rehead{ }\normalsize\beginnumbering\briefempfaengerindex{Bahr, Hermann@\textsc{Bahr, Hermann}!zzzSchnitzler, Arthur@\emph{von Arthur Schnitzler}!1905-09-171@{17. 9. 1905}|(be}
\toendnotes[C]{\smallbreak\pagebreak[2]}
\correspDesc{Versand  durch Arthur Schnitzler am 17. 9. 1905 in Wien
\newline{}Erhalt  durch Hermann Bahr im Zeitraum [17. 9. 1905
                  – 21. 9. 1905?] in Wien}\toendnotes[C]{\smallbreak}
\Standort{TMW, HS AM 23376 Ba.}
\physDesc{Brief, 1 Blatt, 3 Seiten, 849 Zeichen
\newline{}Handschrift: schwarze Tinte, deutsche Kurrent
\newline{}Ordnung: Lochung }
\buchAbdrucke{\weitereDrucke{1) Arthur Schnitzler: \emph{Briefe 1875–1912}. Herausgegeben von Therese Nickl und Heinrich Schnitzler. Frankfurt am Main: \emph{S. Fischer} 1981, S. 516–517.} \weitereDrucke{2) \emph{17. 9. 1905.} In: Arthur Schnitzler: \emph{The Letters of Arthur Schnitzler to Hermann Bahr}. Edited, annotated, and with an introduction, by Donald G. Daviau. Chapel Hill: \emph{The University of North Carolina Press} 1978, S. 90–91 (University of North Carolina studies in the Germanic languages
                        and literatures, 89).} \weitereDrucke{3) Hermann Bahr, Arthur Schnitzler: \emph{Briefwechsel, Aufzeichnungen, Dokumente (1891–1931)}. Herausgegeben von Kurt Ifkovits und Martin Anton Müller. Göttingen: \emph{Wallstein} 2018, S. 351.} }\toendnotes[C]{\smallbreak}
\pstart
           \raggedleft{}{\pb}17. 9. 905\pend
           \vspace{0.5em}
\pstart
           lieber Hermann, für den Fall, dſs ich dich nicht zu Hauſe treffe,{ }ſchreibe ich \damage{d}ir gleich.\pend
           
\pstart
           Das gedruckte Stück »\label{K_L01547-1v}\edtext{Zwiſchenſpiel\pwindex{Schnitzler, Arthur 15.\,5.\,1862 Wien – 21.\,10.\,1931 ebd.@\textsc{Schnitzler, Arthur} (15.\,5.\,1862 Wien – 21.\,10.\,1931 ebd.), \emph{Schriftsteller, Mediziner}!Zwischenspiel. Komödie in drei Akten@\strich\emph{Zwischenspiel. Komödie in drei Akten}|pw}}{\lemma{\textnormal{\emph{Zwischenspiel}}}\Cendnote{\textnormal{Entsprechend dürfte die erste
                  Buchausgabe auf 1906 vordatiert sein: Arthur Schnitzler: \emph{Das Zwischenspiel. Komödie in drei Akten}\pwindex{Schnitzler, Arthur 15.\,5.\,1862 Wien – 21.\,10.\,1931 ebd.@\textsc{Schnitzler, Arthur} (15.\,5.\,1862 Wien – 21.\,10.\,1931 ebd.), \emph{Schriftsteller, Mediziner}!Zwischenspiel. Komödie in drei Akten@\strich\emph{Zwischenspiel. Komödie in drei Akten}|pwk}. Berlin: \emph{S. Fischer}\orgindex{S. Fischer Verlag@S. Fischer Verlag|pwk}{ }1906.}}}\label{K_L01547-1}« und »Der Ruf des Lebens\pwindex{Schnitzler, Arthur 15.\,5.\,1862 Wien – 21.\,10.\,1931 ebd.@\textsc{Schnitzler, Arthur} (15.\,5.\,1862 Wien – 21.\,10.\,1931 ebd.), \emph{Schriftsteller, Mediziner}!Zwischenspiel. Komödie in drei Akten@\strich\emph{Zwischenspiel. Komödie in drei Akten}|pw}«
               liegen hier bei.\pend
           
\pstart
           Über das erſtere iſt weiter nichts zu{ }ſagen; lies es bitte und betrachte es im
               übrigen vorläufig{ }ſorgfältg als \textsc{\uline{Mscrpt}}.\pend
           
\pstart
           Am »Ruf des Lebens\pwindex{Schnitzler, Arthur 15.\,5.\,1862 Wien – 21.\,10.\,1931 ebd.@\textsc{Schnitzler, Arthur} (15.\,5.\,1862 Wien – 21.\,10.\,1931 ebd.), \emph{Schriftsteller, Mediziner}!Ruf des Lebens. Schauspiel in drei Akten@\strich\emph{Der Ruf des Lebens. Schauspiel in drei Akten}|pw}« ist noch einiges weniges zu
               machen. Ich bring es {\pb}dir{ }ſchon heute, weil ich die Frage an dich richten möchte, ob du die \uline{Widmung} des Buches annehmen willſt? Es iſt vielleicht
               in dem Stück eine Ahnung von dem \label{K_L01547-2v}\edtext{Wunsch erfüllſt, den du anläßlich des Puppenſpielers\pwindex{Schnitzler, Arthur 15.\,5.\,1862 Wien – 21.\,10.\,1931 ebd.@\textsc{Schnitzler, Arthur} (15.\,5.\,1862 Wien – 21.\,10.\,1931 ebd.), \emph{Schriftsteller, Mediziner}!Puppenspieler. Studie in einem Aufzuge@\strich\emph{Der Puppenspieler. Studie in einem Aufzuge}|pw} oeffentlich ausſprachſt}{\lemma{\textnormal{\emph{Wunsch … aussprachst}}}\Cendnote{\textnormal{Vgl. XXXX Auszeichnungsfehler: Dokument L01477 nicht gefunden und Hermann Bahr, Arthur Schnitzler: \emph{Briefwechsel, Aufzeichnungen, Dokumente (1891–1931)}, Hermann Bahr: Der Puppenspieler, 13. 12. 1904.
               }}}\label{K_L01547-2}. –\pend
           
\pstart
           Schreib mir bitte ein Wort, wa{\geminationn} wir zuſa{\geminationm}en{ }ſein könnten. Möchteſ\damage{t} du nicht einmal bei uns nachtmahlen? Auch meine Frau\pwindex{Schnitzler, Olga 17.\,1.\,1882 Wien – 13.\,1.\,1970 Lugano@\textsc{Schnitzler, Olga} (17.\,1.\,1882 Wien – 13.\,1.\,1970 Lugano), \emph{Schauspielerin, Sängerin}|pwv} würde{ }ſich{ }ſoſehr freuen. Oder wenn dir
               die Spöttelgaſſe\oindex{Wien@\textbf{Wien}!XVIII., Währing@\textbf{XVIII., Währing}!Edmund-Weiß-Gasse 7@\textbf{Edmund-Weiß-Gasse 7}, \emph{Wohngebäude}|pw} unbe{\pb}quem, Hietzing\oindex{XIII., Hietzing@\textbf{XIII., Hietzing}, \emph{Verwaltungsgebiet}|pw}? Man{ }ſieht einander doch gar zu wenig! Ich grüße dich
               herzlich.\pend
           
\pstart
           Dein{\\[\baselineskip]}\spacefill\mbox{A.}\pend
           \leftskip=0em{}\selectlanguage{ngerman}\endnumbering\briefempfaengerindex{Bahr, Hermann@\textsc{Bahr, Hermann}!zzzSchnitzler, Arthur@\emph{von Arthur Schnitzler}!1905-09-171@{17. 9. 1905}|)be}\mylabel{L01547h}  \newcommand{\dateiname}{L01547}\newcommand{\titel}{Arthur Schnitzler an Hermann Bahr, 17. 9. 1905}\newcommand{\editorInnen}{Herausgegeben von Martin Anton Müller}%% latex-leseansicht-abspann.tex
%% Abspann für die Leseansicht.
%% Der Schalter \ifkorrekturansicht ist bereits durch den Vorspann gesetzt.

%% latex-abspann.tex
%% Gemeinsamer Abspann für Korrekturansicht und Leseansicht.
%% Setzt den Schalter \ifkorrekturansicht voraus (gesetzt in den
%% einbindenden Dateien latex-korrekturansicht-abspann.tex bzw.
%% latex-leseansicht-abspann.tex).
%% ---------------------------------------------------------------

\normalsize

% Das esempio-Environment wird nur in der Leseansicht benötigt
\ifkorrekturansicht\else
\newenvironment{esempio}[3]%
{
    \vspace{1.5ex}
    \rlap{\underline{#1}}
    \par
    \setlength{\parindent}{0cm}
    \nopagebreak
    \leftskip=#2cm
    \rightskip=#3cm
}
{
    \par
}
\fi

\doendnotes{C}
\bigskip
\vfill

\clearpage

\footnotesize

\ifkorrekturansicht
  \lohead{\textsc{register}}
\fi

% theindex-Environment neu definieren ohne reledmac
\makeatletter
\renewenvironment{theindex}{%
  \ifkorrekturansicht
    \section*{\indexname}%
  \else
    \subsubsection*{Index der erwähnten Entitäten}%
  \fi
  \setlength{\parindent}{0pt}%
  \setlength{\parskip}{0pt plus 0.3pt}%
  \let\item\@idxitem
}{%
  \ifkorrekturansicht\clearpage\fi
}
\makeatother

\IfFileExists{\jobname-pw.ind}{\input{\jobname-pw.ind}}{}

% Quellenangabe nur in der Leseansicht
\ifkorrekturansicht\else
% Fallback-Definitionen, falls die .tex-Datei \titel etc. nicht gesetzt hat
\providecommand{\titel}{}
\providecommand{\editorInnen}{}
\providecommand{\dateiname}{\jobname}

\vspace{3cm}

\vfill

\footnotesize
\textsc{Quelle}: \titel. Herausgegeben von {\editorInnen}. In: \emph{Arthur Schnitzler: Briefwechsel mit Autorinnen und Autoren}.
 Digitale Edition, https://schnitzler-briefe.acdh.oeaw.ac.at/{\dateiname}.html (Stand \today)
\fi

\end{document}


