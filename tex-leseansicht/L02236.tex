%% latex-leseansicht-vorspann.tex
%% Vorspann für die Leseansicht.
%% Lädt die gemeinsame Datei latex-vorspann.tex mit nicht gesetztem Schalter.

\newif\ifkorrekturansicht
\korrekturansichtfalse

\input{../tex-inputs/latex-vorspann}


         
         \renewcommand{\erwaehntePersonen}{Personen: Richard Beer-Hofmann, Olga Schnitzler}
         \renewcommand{\erwaehnteOrte}{Orte: Altaussee, Aschau, Bad Ischl, Grazer Straße, Hotel Kaiserkrone, Loser, Restaurant Sonnenschein, Salzkammergut}
         \renewcommand{\erwaehnteWerke}{}
               \section[Arthur Schnitzler an Richard Beer-Hofmann, 1. 8. 1916]{ Arthur Schnitzler an Richard Beer-Hofmann, 1. 8. 1916}\nopagebreak\mylabel{v}\rehead{ }\begin{ledgroupsized}[t]{13cm}\normalsize\beginnumbering \toendnotes[C]{\smallbreak\pagebreak[2]} \Standort{YCGL, MSS 31.}
\physDesc{Bildpostkarte, 528 Zeichen
\newline{}Handschrift: Bleistift, deutsche Kurrent
\newline{}Versand: Stempel: »\nobreak{}\oindex{Altaussee@\textbf{Altaussee}|pwk}Alt Aussee, 1. VIII. 16\nobreak{}«.  
\newline{}Beer-Hofmann: mit blauem Buntstift Erhalt
                                 festgehalten: »E.« }\pstart{}{\pb}\textsc{Herrn Dr. Richard}\pend{}\pstart{}\textsc{Beer}\substVorne{}\textsuperscript{h}\substDazwischen{}\textsc{H}\substHinten{}\textsc{ofmann}\pend{}\pstart{}\textsc{Bad Ischl}\oindex{Bad Ischl@\textbf{Bad Ischl}|pw}\pend{}\pstart{}\textsc{Grazerstr. 52}\oindex{Grazer Strasse@\textbf{Grazer Straße}|pw}\pend{}{\bigskip}\pstart
           \noindent{}\centering{}{\pb}\textcolor{gray}{\textbf{Salzkammergut\oindex{Salzkammergut@\textbf{Salzkammergut}|pw}. Alt Aussee\oindex{Altaussee@\textbf{Altaussee}|pw} mit dem Loser\oindex{Loser@\textbf{Loser}|pw}.}}\pend
           \pstart
           \raggedleft{}{\pb}1. 8. 1916\pend
           \pstart
           lieber Richard, wir haben in der Kaiſerkr.\oindex{Hotel Kaiserkrone@\textbf{Hotel Kaiserkrone}|pw} kein Zi{\geminationm}er beko{\geminationm}en, übernachten daher nicht in Iſchl\oindex{Bad Ischl@\textbf{Bad Ischl}|pw}. Statt Mittwoch gedenke ich Do{\geminationn}erſtag hinüber zu wandern und um
                  ½ 2 (oder etwas früher) beim \uline{So{\geminationn}enſchein}\oindex{Restaurant Sonnenschein@\textbf{Restaurant Sonnenschein}|pw} zu eſſen.\strikeout{)}{ }\uline{Ebenſo Abends} ſchon circa ½, ¾ 8, da wir
               um 9 nach Auſſee\oindex{Altaussee@\textbf{Altaussee}|pw} zurückfahren
               müſſen. O.\pwindex{Schnitzler, Olga 17.01.1882 – 13.01.1970@\textsc{Schnitzler, Olga} (17.01.1882 – 13.01.1970), \emph{Schauspielerin, Sängerin}|pw} iſſt zu Mittag in der Aſchau\oindex{Aschau@\textbf{Aschau}|pw}. In jedem Fall hoffen wir alſo zum
               Nachtm. mit Ihnen zuſa{\geminationm}en zu ſein; ich melde mich
               natürlich früher {\pb}bei Ihnen. Ganz unverbindlich, auch
               für Sie!\pend
           \pstart
           Herzlichſt Ihr{\\[\baselineskip]}\spacefill\mbox{Arthur}\pend
           \leftskip=0em{}
         
         \endnumbering\mylabel{h}\end{ledgroupsized}  \newcommand{\dateiname}{L02236}\newcommand{\titel}{Arthur Schnitzler an Richard Beer-Hofmann, 1. 8. 1916}\newcommand{\editorInnen}{Martin Anton Müller und Gerd-Hermann Susen}%% latex-leseansicht-abspann.tex
%% Abspann für die Leseansicht.
%% Der Schalter \ifkorrekturansicht ist bereits durch den Vorspann gesetzt.

%% latex-abspann.tex
%% Gemeinsamer Abspann für Korrekturansicht und Leseansicht.
%% Setzt den Schalter \ifkorrekturansicht voraus (gesetzt in den
%% einbindenden Dateien latex-korrekturansicht-abspann.tex bzw.
%% latex-leseansicht-abspann.tex).
%% ---------------------------------------------------------------

\normalsize

% Das esempio-Environment wird nur in der Leseansicht benötigt
\ifkorrekturansicht\else
\newenvironment{esempio}[3]%
{
    \vspace{1.5ex}
    \rlap{\underline{#1}}
    \par
    \setlength{\parindent}{0cm}
    \nopagebreak
    \leftskip=#2cm
    \rightskip=#3cm
}
{
    \par
}
\fi

\doendnotes{C}
\bigskip
\vfill

\clearpage

\footnotesize

\ifkorrekturansicht
  \lohead{\textsc{register}}
\fi

% theindex-Environment neu definieren ohne reledmac
\makeatletter
\renewenvironment{theindex}{%
  \ifkorrekturansicht
    \section*{\indexname}%
  \else
    \subsubsection*{Index der erwähnten Entitäten}%
  \fi
  \setlength{\parindent}{0pt}%
  \setlength{\parskip}{0pt plus 0.3pt}%
  \let\item\@idxitem
}{%
  \ifkorrekturansicht\clearpage\fi
}
\makeatother

\IfFileExists{\jobname-pw.ind}{\input{\jobname-pw.ind}}{}

% Quellenangabe nur in der Leseansicht
\ifkorrekturansicht\else
% Fallback-Definitionen, falls die .tex-Datei \titel etc. nicht gesetzt hat
\providecommand{\titel}{}
\providecommand{\editorInnen}{}
\providecommand{\dateiname}{\jobname}

\vspace{3cm}

\vfill

\footnotesize
\textsc{Quelle}: \titel. Herausgegeben von {\editorInnen}. In: \emph{Arthur Schnitzler: Briefwechsel mit Autorinnen und Autoren}.
 Digitale Edition, https://schnitzler-briefe.acdh.oeaw.ac.at/{\dateiname}.html (Stand \today)
\fi

\end{document}


      