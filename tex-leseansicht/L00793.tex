%% latex-leseansicht-vorspann.tex
%% Vorspann für die Leseansicht.
%% Lädt die gemeinsame Datei latex-vorspann.tex mit nicht gesetztem Schalter.

\newif\ifkorrekturansicht
\korrekturansichtfalse

\input{../tex-inputs/latex-vorspann}


\section[Fritz Schlesinger u. a. an Hermann Bahr, 21. 4. 1898]{L00793 Fritz Schlesinger u. a. an Hermann Bahr, 21. 4. 1898}
\nopagebreak\mylabel{L00793v}
\rehead{ }\normalsize\beginnumbering\briefempfaengerindex{Bahr, Hermann@\textsc{Bahr, Hermann}!zzzFranckenstein, Georg von@\emph{von Georg von Franckenstein}!1898-04-211@{21. 4. 1898}|(be}\briefempfaengerindex{Bahr, Hermann@\textsc{Bahr, Hermann}!zzzSchnitzler, Arthur@\emph{von Arthur Schnitzler}!1898-04-211@{21. 4. 1898}|(be}\briefempfaengerindex{Bahr, Hermann@\textsc{Bahr, Hermann}!zzzHofmannsthal, Gertrude von@\emph{von Gertrude von Hofmannsthal}!1898-04-211@{21. 4. 1898}|(be}\briefempfaengerindex{Bahr, Hermann@\textsc{Bahr, Hermann}!zzzSchlesinger, Friedrich@\emph{von Friedrich Schlesinger}!1898-04-211@{21. 4. 1898}|(be}
\toendnotes[C]{\smallbreak\pagebreak[2]}
\correspDesc{Versand  durch Fritz Schlesinger, Gerty von Schlesinger, Arthur Schnitzler, Georg von Franckenstein am 21. 4. 1898 in Breitenfurt bei Wien
\newline{}Erhalt  durch Hermann Bahr am 22. 4. 1898 in Wien}\toendnotes[C]{\smallbreak}
\Standort{TMW, HS AM 57775 Ba.}
\physDesc{Postkarte, 287 Zeichen
\newline{}Handschrift Friedrich Schlesinger: Bleistift, lateinische Kurrent
\newline{}Handschrift Gertrude von Hofmannsthal: Bleistift, lateinische Kurrent
\newline{}Handschrift Arthur Schnitzler: Bleistift, deutsche Kurrent
\newline{}Handschrift Georg von Franckenstein: Bleistift, lateinische Kurrent
\newline{}Versand: 1) Stempel: »\nobreak{}\oindex{Breitenfurt bei Wien@\textbf{Breitenfurt bei Wien}, \emph{Hauptstadt}|pwk}\textcolor{gray}{Brei}\damage{ten}furt, 21 4 98\nobreak{}«.   2) Stempel: »\nobreak{}Bestellt, \oindex{IX., Alsergrund@\textbf{IX., Alsergrund}, \emph{Verwaltungsgebiet}|pwk}Wien 9/2, 22 4. 98, 2 \textcolor{gray}{½} N\nobreak{}«. }
\buchAbdrucke{\weitereDrucke{Hermann Bahr, Arthur Schnitzler: \emph{Briefwechsel, Aufzeichnungen, Dokumente (1891–1931)}. Herausgegeben von Kurt Ifkovits und Martin Anton Müller. Göttingen: \emph{Wallstein} 2018, S. 162.} }\toendnotes[C]{\smallbreak}\pstart{}{\pb}Herrn Hermann
                  Bahr\pend{}\pstart{}IX. Porzellangasse 37\oindex{Wien@\textbf{Wien}!IX., Alsergrund@\textbf{IX., Alsergrund}!Porzellangasse@\textbf{Porzellangasse}, \emph{Straße}|pw}\pend{}\pstart{}Wien\oindex{Wien@\textbf{Wien}, \emph{Verwaltungsgebiet}|pw}\pend{}{\bigskip}\begin{figure}[H]\centering\includegraphics[width=10cm]{../tex-inputs/img/hansschlesinger.jpg}\end{figure}\vspace{1em}
\pstart
           \noindent{}{\pb}\uline{Breitenfurth\oindex{Breitenfurt bei Wien@\textbf{Breitenfurt bei Wien}, \emph{Hauptstadt}|pw}}.\pend
           
\pstart
           Der Dichter ist oft sehr zerstreut\pend
           
\pstart
           Was sein Bicycle nicht erfreut\pend
           
\pstart
           Die Bremse wohl sehr wichtig ist\pend
           
\pstart
           Weil sonst man in den Graben schießt. \introOben{}\label{K_L00793-1v}\edtext{Hugo\pwindex{Hofmannsthal, Hugo von 1.\,2.\,1874 Wien – 15.\,7.\,1929 Rodaun@\textsc{Hofmannsthal, Hugo von} (1.\,2.\,1874 Wien – 15.\,7.\,1929 Rodaun), \emph{Schriftsteller}|pw}}{\lemma{\textnormal{\emph{Hugo}}}\Cendnote{\textnormal{Als Beschriftung der stürzenden
                     Person auf der Bleistiftzeichnung gewertet. Es ließe sich auch als Unterschrift
                        Hofmannsthals\pwindex{Hofmannsthal, Hugo von 1.\,2.\,1874 Wien – 15.\,7.\,1929 Rodaun@\textsc{Hofmannsthal, Hugo von} (1.\,2.\,1874 Wien – 15.\,7.\,1929 Rodaun), \emph{Schriftsteller}|pwk} deuten. Im \emph{Tagebuch}\pwindex{Schnitzler, Arthur 15.\,5.\,1862 Wien – 21.\,10.\,1931 ebd.@\textsc{Schnitzler, Arthur} (15.\,5.\,1862 Wien – 21.\,10.\,1931 ebd.), \emph{Schriftsteller, Mediziner}!Tagebuch@\strich\emph{Tagebuch}|pwk} nennt Schnitzler diesen und zusätzlich die Mutter Franziska Schlesinger\pwindex{Schlesinger, Franziska 17.\,8.\,1851 Wien – 11.\,8.\,1932 ebd.@\textsc{Schlesinger, Franziska} (17.\,8.\,1851 Wien – 11.\,8.\,1932 ebd.)|pwk} als weitere
                     Teilnehmer der Radtour, übergeht jedoch Fritz
                        Schlesinger\pwindex{Schlesinger, Friedrich 1.\,11.\,1883 Wien – 30.\,12.\,1938 Konzentrationslager Buchenwald@\textsc{Schlesinger, Friedrich} (1.\,11.\,1883 Wien – 30.\,12.\,1938 Konzentrationslager Buchenwald), \emph{Industrieller}|pwk}.}}}\label{K_L00793-1}\introOben{}\pend
           
\pstart
           \spacefill\mbox{Fritz Schlesinger}{\\[\baselineskip]}\spacefill\mbox{{[}hs. Franckenstein:{]} G Franckenste\damage{in}}{\\[\baselineskip]}{[}hs. Hofmannsthal:{]} Beneiden Sie uns ein bisserl, ja?\hspace*{2.5em}\spacefill\mbox{Gerty}{\\[\baselineskip]}{[}hs. Schnitzler:{]} HerzGruß\hspace*{1.5em}\spacefill\mbox{ArthSchnitzler}\pend
           \leftskip=0em{}\selectlanguage{ngerman}\endnumbering\briefempfaengerindex{Bahr, Hermann@\textsc{Bahr, Hermann}!zzzFranckenstein, Georg von@\emph{von Georg von Franckenstein}!1898-04-211@{21. 4. 1898}|)be}\briefempfaengerindex{Bahr, Hermann@\textsc{Bahr, Hermann}!zzzSchnitzler, Arthur@\emph{von Arthur Schnitzler}!1898-04-211@{21. 4. 1898}|)be}\briefempfaengerindex{Bahr, Hermann@\textsc{Bahr, Hermann}!zzzHofmannsthal, Gertrude von@\emph{von Gertrude von Hofmannsthal}!1898-04-211@{21. 4. 1898}|)be}\briefempfaengerindex{Bahr, Hermann@\textsc{Bahr, Hermann}!zzzSchlesinger, Friedrich@\emph{von Friedrich Schlesinger}!1898-04-211@{21. 4. 1898}|)be}\mylabel{L00793h}  \newcommand{\dateiname}{L00793}\newcommand{\titel}{Fritz Schlesinger u. a. an Hermann Bahr, 21. 4. 1898}\newcommand{\editorInnen}{Kurt Ifkovits und Martin Anton Müller}%% latex-leseansicht-abspann.tex
%% Abspann für die Leseansicht.
%% Der Schalter \ifkorrekturansicht ist bereits durch den Vorspann gesetzt.

%% latex-abspann.tex
%% Gemeinsamer Abspann für Korrekturansicht und Leseansicht.
%% Setzt den Schalter \ifkorrekturansicht voraus (gesetzt in den
%% einbindenden Dateien latex-korrekturansicht-abspann.tex bzw.
%% latex-leseansicht-abspann.tex).
%% ---------------------------------------------------------------

\normalsize

% Das esempio-Environment wird nur in der Leseansicht benötigt
\ifkorrekturansicht\else
\newenvironment{esempio}[3]%
{
    \vspace{1.5ex}
    \rlap{\underline{#1}}
    \par
    \setlength{\parindent}{0cm}
    \nopagebreak
    \leftskip=#2cm
    \rightskip=#3cm
}
{
    \par
}
\fi

\doendnotes{C}
\bigskip
\vfill

\clearpage

\footnotesize

\ifkorrekturansicht
  \lohead{\textsc{register}}
\fi

% theindex-Environment neu definieren ohne reledmac
\makeatletter
\renewenvironment{theindex}{%
  \ifkorrekturansicht
    \section*{\indexname}%
  \else
    \subsubsection*{Index der erwähnten Entitäten}%
  \fi
  \setlength{\parindent}{0pt}%
  \setlength{\parskip}{0pt plus 0.3pt}%
  \let\item\@idxitem
}{%
  \ifkorrekturansicht\clearpage\fi
}
\makeatother

\IfFileExists{\jobname-pw.ind}{\input{\jobname-pw.ind}}{}

% Quellenangabe nur in der Leseansicht
\ifkorrekturansicht\else
% Fallback-Definitionen, falls die .tex-Datei \titel etc. nicht gesetzt hat
\providecommand{\titel}{}
\providecommand{\editorInnen}{}
\providecommand{\dateiname}{\jobname}

\vspace{3cm}

\vfill

\footnotesize
\textsc{Quelle}: \titel. Herausgegeben von {\editorInnen}. In: \emph{Arthur Schnitzler: Briefwechsel mit Autorinnen und Autoren}.
 Digitale Edition, https://schnitzler-briefe.acdh.oeaw.ac.at/{\dateiname}.html (Stand \today)
\fi

\end{document}


