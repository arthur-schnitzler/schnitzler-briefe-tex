%% latex-korrekturansicht-vorspann.tex
%% Vorspann für die Korrekturansicht.
%% Lädt die gemeinsame Datei latex-vorspann.tex mit gesetztem Schalter.

\newif\ifkorrekturansicht
\korrekturansichttrue

\input{../tex-inputs/latex-vorspann}


\section[Fritz Schlesinger u. a. an Hermann Bahr, 21. 4. 1898]{L00793 Fritz Schlesinger u. a. an Hermann Bahr, 21. 4. 1898}
\nopagebreak\mylabel{L00793v}
\rehead{ }\normalsize\beginnumbering\briefempfaengerindex{Bahr, Hermann@\textsc{Bahr, Hermann}!zzzFranckenstein, Georg von@\emph{von Georg von Franckenstein}!1898-04-211@{21. 4. 1898}|(be}\briefempfaengerindex{Bahr, Hermann@\textsc{Bahr, Hermann}!zzzSchnitzler, Arthur@\emph{von Arthur Schnitzler}!1898-04-211@{21. 4. 1898}|(be}\briefempfaengerindex{Bahr, Hermann@\textsc{Bahr, Hermann}!zzzHofmannsthal, Gertrude von@\emph{von Gertrude von Hofmannsthal}!1898-04-211@{21. 4. 1898}|(be}\briefempfaengerindex{Bahr, Hermann@\textsc{Bahr, Hermann}!zzzSchlesinger, Friedrich@\emph{von Friedrich Schlesinger}!1898-04-211@{21. 4. 1898}|(be}
\toendnotes[C]{\smallbreak\pagebreak[2]}\Standort{TMW, HS AM 57775 Ba.}
\physDesc{Postkarte, 287 Zeichen
\newline{}Handschrift Friedrich Schlesinger: Bleistift, lateinische Kurrent
\newline{}Handschrift Gertrude von Hofmannsthal: Bleistift, lateinische Kurrent
\newline{}Handschrift Arthur Schnitzler: Bleistift, deutsche Kurrent
\newline{}Handschrift Georg von Franckenstein: Bleistift, lateinische Kurrent
\newline{}Versand: 1) Stempel: »\nobreak{}\oindex{Breitenfurt bei Wien@\textbf{Breitenfurt bei Wien}, \emph{P.PPLA3}|pwk}\textcolor{gray}{Brei}\damage{ten}furt, 21 4 98\nobreak{}«.   2) Stempel: »\nobreak{}Bestellt, \oindex{IX., Alsergrund@\textbf{IX., Alsergrund}, \emph{A.ADM3}|pwk}Wien 9/2, 22 4. 98, 2 \textcolor{gray}{½} N\nobreak{}«. }
\buchAbdrucke{\weitereDrucke{Hermann Bahr, Arthur Schnitzler: \emph{Briefwechsel, Aufzeichnungen, Dokumente (1891–1931)}. Göttingen: \emph{Wallstein} 2018, S. 162.} }\toendnotes[C]{\smallbreak}\pstart{}{\pb}Herrn Hermann
                  Bahr\pend{}\pstart{}IX. Porzellangasse 37\oindex{Porzellangasse@\textbf{Porzellangasse}, \emph{Straße (K.STR)}|pw}\pend{}\pstart{}Wien\oindex{Wien@\textbf{Wien}, \emph{A.ADM2}|pw}\pend{}{\bigskip}\begin{figure}[H]\centering\includegraphics[width=10cm]{../tex-inputs/img/hansschlesinger.jpg}\end{figure}\vspace{1em}
\pstart
           \noindent{}{\pb}\uline{Breitenfurth\oindex{Breitenfurt bei Wien@\textbf{Breitenfurt bei Wien}, \emph{P.PPLA3}|pw}}.\pend
           
\pstart
           Der Dichter ist oft sehr zerstreut\pend
           
\pstart
           Was sein Bicycle nicht erfreut\pend
           
\pstart
           Die Bremse wohl sehr wichtig ist\pend
           
\pstart
           Weil sonst man in den Graben schießt. \introOben{}\label{K_L00793-1v}\edtext{Hugo\pwindex{Hofmannsthal, Hugo von 1874-02-01 – 1929-07-15@\textsc{Hofmannsthal, Hugo von} (1874-02-01 – 1929-07-15), \emph{Schriftsteller/Schriftstellerin}|pw}}{\lemma{\textnormal{\emph{Hugo}}}\Cendnote{\textnormal{Als Beschriftung der stürzenden
                     Person auf der Bleistiftzeichnung gewertet. Es ließe sich auch als Unterschrift
                        Hofmannsthals\pwindex{Hofmannsthal, Hugo von 1874-02-01 – 1929-07-15@\textsc{Hofmannsthal, Hugo von} (1874-02-01 – 1929-07-15), \emph{Schriftsteller/Schriftstellerin}|pwk} deuten. Im \emph{Tagebuch}\pwindex{Tagebuch@\emph{Tagebuch}|pwk} nennt Schnitzler diesen und zusätzlich die Mutter Franziska Schlesinger\pwindex{Schlesinger, Franziska 17.08.1851 – 11.08.1932@\textsc{Schlesinger, Franziska} (17.08.1851 – 11.08.1932)|pwk} als weitere
                     Teilnehmer der Radtour, übergeht jedoch Fritz
                        Schlesinger\pwindex{Schlesinger, Friedrich 01.11.1883 – 30.12.1938@\textsc{Schlesinger, Friedrich} (01.11.1883 – 30.12.1938), \emph{Industrieller/Industrielle}|pwk}.}}}\label{K_L00793-1}\introOben{}\pend
           
\pstart
           \spacefill\mbox{Fritz Schlesinger}{\\[\baselineskip]}\spacefill\mbox{{[}hs. :{]} G Franckenste\damage{in}}{\\[\baselineskip]}{[}hs. :{]} Beneiden Sie uns ein bisserl, ja?\hspace*{2.5em}\spacefill\mbox{Gerty}{\\[\baselineskip]}{[}hs. :{]} HerzGruß\hspace*{1.5em}\spacefill\mbox{ArthSchnitzler}\pend
           \leftskip=0em{}\selectlanguage{ngerman}\endnumbering\briefempfaengerindex{Bahr, Hermann@\textsc{Bahr, Hermann}!zzzFranckenstein, Georg von@\emph{von Georg von Franckenstein}!1898-04-211@{21. 4. 1898}|)be}\briefempfaengerindex{Bahr, Hermann@\textsc{Bahr, Hermann}!zzzSchnitzler, Arthur@\emph{von Arthur Schnitzler}!1898-04-211@{21. 4. 1898}|)be}\briefempfaengerindex{Bahr, Hermann@\textsc{Bahr, Hermann}!zzzHofmannsthal, Gertrude von@\emph{von Gertrude von Hofmannsthal}!1898-04-211@{21. 4. 1898}|)be}\briefempfaengerindex{Bahr, Hermann@\textsc{Bahr, Hermann}!zzzSchlesinger, Friedrich@\emph{von Friedrich Schlesinger}!1898-04-211@{21. 4. 1898}|)be}\mylabel{L00793h}  \normalsize

\doendnotes{C}
\bigskip
\vfill

\clearpage

\footnotesize

\lohead{\textsc{register}}

% Definiere theindex-Environment komplett neu ohne reledmac
\makeatletter
\renewenvironment{theindex}{%
  \section*{\indexname}%
  \setlength{\parindent}{0pt}%
  \setlength{\parskip}{0pt plus 0.3pt}%
  \let\item\@idxitem
}{%
  \clearpage
}
\makeatother

\IfFileExists{\jobname-pw.ind}{\input{\jobname-pw.ind}}{}

\end{document}

      