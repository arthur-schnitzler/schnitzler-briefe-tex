%% latex-leseansicht-vorspann.tex
%% Vorspann für die Leseansicht.
%% Lädt die gemeinsame Datei latex-vorspann.tex mit nicht gesetztem Schalter.

\newif\ifkorrekturansicht
\korrekturansichtfalse

\input{../tex-inputs/latex-vorspann}


\section[Arthur und Olga Schnitzler an Richard Beer-Hofmann, 22. 5. 1910]{L01933 Arthur und Olga Schnitzler an Richard Beer-Hofmann, 22. 5. 1910}
\nopagebreak\mylabel{L01933v}
\rehead{ }\normalsize\beginnumbering\briefempfaengerindex{Beer-Hofmann, Richard@\textsc{Beer-Hofmann, Richard}!zzzSchnitzler, Olga@\emph{von Olga Schnitzler}!1910-05-221@{22. 5. 1910}|(be}\briefempfaengerindex{Beer-Hofmann, Richard@\textsc{Beer-Hofmann, Richard}!zzzSchnitzler, Arthur@\emph{von Arthur Schnitzler}!1910-05-221@{22. 5. 1910}|(be}
\toendnotes[C]{\smallbreak\pagebreak[2]}
\correspDesc{Versand  durch Arthur Schnitzler, Olga Schnitzler am 22. 5. 1910 in Luzern
\newline{}Erhalt  durch Richard Beer-Hofmann im Zeitraum [23. 5. 1910
                  – 27. 5. 1910?] in Wien}\toendnotes[C]{\smallbreak}
\Standort{YCGL, MSS 31.}
\physDesc{Bildpostkarte, 274 Zeichen
\newline{}Handschrift Arthur Schnitzler: Bleistift, deutsche Kurrent
\newline{}Handschrift Olga Schnitzler: Bleistift, lateinische Kurrent
\newline{}Versand: 1) Stempel: »\nobreak{}\oindex{Bürgenstock@\textbf{Bürgenstock}, \emph{Berg}|pwk}Bürgenstock b. Luzern
                                       Hammetschwand-Lift, 1132 m. {[}ü.{]} M. Höhe des Lifts
                                       170 m.\nobreak{}«.   2) Stempel: »\nobreak{}\oindex{Luzern@\textbf{Luzern}|pwk}Luzern, 22 V 10, 9\nobreak{}«. 
\newline{}Ordnung: mit Bleistift von unbekannter Hand datiert: »22. 5.« }\toendnotes[C]{\smallbreak}\pstart{}{\pb}\textsc{Dr. Richard Beer-Hofma{\geminationn}}\pend{}\pstart{}Wien XVIII\oindex{XVIII., Währing@\textbf{XVIII., Währing}, \emph{Verwaltungsgebiet}|pw}\pend{}\pstart{}\textsc{Hasenauerstr 59\oindex{Wien@\textbf{Wien}!XVIII., Währing@\textbf{XVIII., Währing}!Hasenauerstraße 59@\textbf{Hasenauerstraße 59}, \emph{Wohngebäude}|pw}}\pend{}{\bigskip}
\pstart
           \noindent{}\centering{}{\pb}\textcolor{gray}{\textbf{\textbf{\label{T_L01933-1v}\edtext{Bürgenstock\oindex{Bürgenstock@\textbf{Bürgenstock}, \emph{Berg}|pw}}{\lemma{\textnormal{\emph{Bürgenstock}}}\Cendnote{\textnormal{die Bilderläuterung zweimal auf
                        der Karte; bei jener auf der Adressseite von Schnitzler die Ortsangabe
                        unterstrichen}}}\label{T_L01933-1}.} Ausblick vom Känzeli\oindex{Känzeli@\textbf{Känzeli}, \emph{Ausflugsziel}|pw} am Felsenweg\oindex{Felsenweg@\textbf{Felsenweg}, \emph{Straße}|pw} auf Rigi\oindex{Rigi@\textbf{Rigi}, \emph{Gebirge}|pw} u. Vierwaldstättersee\oindex{Vierwaldstättersee@\textbf{Vierwaldstättersee}, \emph{See}|pw}.}}\pend
           \vspace{1em}
\pstart
           \noindent{}{\pb}Wär’ wieder was für Sie! Wundervoll! Reiſen Sie mit
                  Paula\pwindex{Beer-Hofmann, Paula 25.\,2.\,1879 Wien – 30.\,10.\,1939 Zürich@\textsc{Beer-Hofmann, Paula} (25.\,2.\,1879 Wien – 30.\,10.\,1939 Zürich)|pw}{ }\substVorne{}\textsuperscript{\textcolor{gray}{auch}}\substDazwischen{}nach\substHinten{}{ }Luzern\oindex{Luzern@\textbf{Luzern}|pw} und beſichtigen die Ufer. (Wien\oindex{Wien@\textbf{Wien}, \emph{Verwaltungsgebiet}|pw}–Zürich\oindex{Zürich@\textbf{Zürich}|pw},
               Schlafwagen 2. \textsc{Classe}, (wie wir) ab Wien\oindex{Wien@\textbf{Wien}, \emph{Verwaltungsgebiet}|pw} 8 Abend, Zürich\oindex{Zürich@\textbf{Zürich}|pw} 2 Uhr
                  \introOben{}Mg\introOben{}, Luzern\oindex{Luzern@\textbf{Luzern}|pw} 4.30.)\pend
           
\pstart
           Das wünſcht Ihnen{\\[\baselineskip]}herzlichſt Ihr \spacefill\mbox{A.}{\\[\baselineskip]}{[}hs. Schnitzler:{]} und \spacefill\mbox{Olga.}\pend
           \leftskip=0em{}\selectlanguage{ngerman}\endnumbering\briefempfaengerindex{Beer-Hofmann, Richard@\textsc{Beer-Hofmann, Richard}!zzzSchnitzler, Olga@\emph{von Olga Schnitzler}!1910-05-221@{22. 5. 1910}|)be}\briefempfaengerindex{Beer-Hofmann, Richard@\textsc{Beer-Hofmann, Richard}!zzzSchnitzler, Arthur@\emph{von Arthur Schnitzler}!1910-05-221@{22. 5. 1910}|)be}\mylabel{L01933h}  \newcommand{\dateiname}{L01933}\newcommand{\titel}{Arthur und Olga Schnitzler an Richard Beer-Hofmann, 22. 5. 1910}\newcommand{\editorInnen}{Martin Anton Müller und Gerd-Hermann Susen}%% latex-leseansicht-abspann.tex
%% Abspann für die Leseansicht.
%% Der Schalter \ifkorrekturansicht ist bereits durch den Vorspann gesetzt.

%% latex-abspann.tex
%% Gemeinsamer Abspann für Korrekturansicht und Leseansicht.
%% Setzt den Schalter \ifkorrekturansicht voraus (gesetzt in den
%% einbindenden Dateien latex-korrekturansicht-abspann.tex bzw.
%% latex-leseansicht-abspann.tex).
%% ---------------------------------------------------------------

\normalsize

% Das esempio-Environment wird nur in der Leseansicht benötigt
\ifkorrekturansicht\else
\newenvironment{esempio}[3]%
{
    \vspace{1.5ex}
    \rlap{\underline{#1}}
    \par
    \setlength{\parindent}{0cm}
    \nopagebreak
    \leftskip=#2cm
    \rightskip=#3cm
}
{
    \par
}
\fi

\doendnotes{C}
\bigskip
\vfill

\clearpage

\footnotesize

\ifkorrekturansicht
  \lohead{\textsc{register}}
\fi

% theindex-Environment neu definieren ohne reledmac
\makeatletter
\renewenvironment{theindex}{%
  \ifkorrekturansicht
    \section*{\indexname}%
  \else
    \subsubsection*{Index der erwähnten Entitäten}%
  \fi
  \setlength{\parindent}{0pt}%
  \setlength{\parskip}{0pt plus 0.3pt}%
  \let\item\@idxitem
}{%
  \ifkorrekturansicht\clearpage\fi
}
\makeatother

\IfFileExists{\jobname-pw.ind}{\input{\jobname-pw.ind}}{}

% Quellenangabe nur in der Leseansicht
\ifkorrekturansicht\else
% Fallback-Definitionen, falls die .tex-Datei \titel etc. nicht gesetzt hat
\providecommand{\titel}{}
\providecommand{\editorInnen}{}
\providecommand{\dateiname}{\jobname}

\vspace{3cm}

\vfill

\footnotesize
\textsc{Quelle}: \titel. Herausgegeben von {\editorInnen}. In: \emph{Arthur Schnitzler: Briefwechsel mit Autorinnen und Autoren}.
 Digitale Edition, https://schnitzler-briefe.acdh.oeaw.ac.at/{\dateiname}.html (Stand \today)
\fi

\end{document}


