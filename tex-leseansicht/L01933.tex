%% latex-korrekturansicht-vorspann.tex
%% Vorspann für die Korrekturansicht.
%% Lädt die gemeinsame Datei latex-vorspann.tex mit gesetztem Schalter.

\newif\ifkorrekturansicht
\korrekturansichttrue

\input{../tex-inputs/latex-vorspann}


\section[Arthur und Olga Schnitzler an Richard Beer-Hofmann, 22. 5. 1910]{L01933 Arthur und Olga Schnitzler an Richard Beer-Hofmann, 22. 5. 1910}
\nopagebreak\mylabel{L01933v}
\rehead{ }\normalsize\beginnumbering\briefempfaengerindex{Beer-Hofmann, Richard@\textsc{Beer-Hofmann, Richard}!zzzSchnitzler, Olga@\emph{von Olga Schnitzler}!1910-05-221@{22. 5. 1910}|(be}\briefempfaengerindex{Beer-Hofmann, Richard@\textsc{Beer-Hofmann, Richard}!zzzSchnitzler, Arthur@\emph{von Arthur Schnitzler}!1910-05-221@{22. 5. 1910}|(be}
\toendnotes[C]{\smallbreak\pagebreak[2]}\Standort{YCGL, MSS 31.}
\physDesc{Bildpostkarte, 274 Zeichen
\newline{}Handschrift Arthur Schnitzler: Bleistift, deutsche Kurrent
\newline{}Handschrift Olga Schnitzler: Bleistift, lateinische Kurrent
\newline{}Versand: 1) Stempel: »\nobreak{}\oindex{Buergenstock@\textbf{Bürgenstock}, \emph{Berg (N.BRG)}|pwk}Bürgenstock b. Luzern
                                       Hammetschwand-Lift, 1132 m. {[}ü.{]} M. Höhe des Lifts
                                       170 m.\nobreak{}«.   2) Stempel: »\nobreak{}\oindex{Luzern@\textbf{Luzern}, \emph{P.PPLA}|pwk}Luzern, 22 V 10, 9\nobreak{}«. 
\newline{}Ordnung: mit Bleistift von unbekannter Hand datiert: »22. 5.« }\toendnotes[C]{\smallbreak}\pstart{}{\pb}\textsc{Dr. Richard Beer-Hofma{\geminationn}}\pend{}\pstart{}Wien XVIII\oindex{XVIII., Waehring@\textbf{XVIII., Währing}, \emph{A.ADM3}|pw}\pend{}\pstart{}\textsc{Hasenauerstr 59\oindex{Hasenauerstrasse 59@\textbf{Hasenauerstraße 59}, \emph{Wohngebäude (K.WHS)}|pw}}\pend{}{\bigskip}
\pstart
           \noindent{}\centering{}{\pb}\textcolor{gray}{\textbf{\textbf{\label{T_L01933-1v}\edtext{Bürgenstock\oindex{Buergenstock@\textbf{Bürgenstock}, \emph{Berg (N.BRG)}|pw}}{\lemma{\textnormal{\emph{Bürgenstock}}}\Cendnote{\textnormal{die Bilderläuterung zweimal auf
                        der Karte; bei jener auf der Adressseite von Schnitzler die Ortsangabe
                        unterstrichen}}}\label{T_L01933-1}.} Ausblick vom Känzeli\oindex{Kaenzeli@\textbf{Känzeli}, \emph{Ausflugsziel}|pw} am Felsenweg\oindex{Felsenweg@\textbf{Felsenweg}, \emph{Straße (K.STR)}|pw} auf Rigi\oindex{Rigi@\textbf{Rigi}, \emph{T.MTS}|pw} u. Vierwaldstättersee\oindex{Vierwaldstaettersee@\textbf{Vierwaldstättersee}, \emph{See (N.SEE)}|pw}.}}\pend
           \vspace{1em}
\pstart
           \noindent{}{\pb}Wär’ wieder was für Sie! Wundervoll! Reiſen Sie mit
                  Paula\pwindex{Beer-Hofmann, Paula 25.02.1879 – 30.10.1939@\textsc{Beer-Hofmann, Paula} (25.02.1879 – 30.10.1939)|pw}{ }\substVorne{}\textsuperscript{\textcolor{gray}{auch}}\substDazwischen{}nach\substHinten{}{ }Luzern\oindex{Luzern@\textbf{Luzern}, \emph{P.PPLA}|pw} und beſichtigen die Ufer. (Wien\oindex{Wien@\textbf{Wien}, \emph{A.ADM2}|pw}–Zürich\oindex{Zuerich@\textbf{Zürich}, \emph{P.PPLA}|pw},
               Schlafwagen 2. \textsc{Classe}, (wie wir) ab Wien\oindex{Wien@\textbf{Wien}, \emph{A.ADM2}|pw} 8 Abend, Zürich\oindex{Zuerich@\textbf{Zürich}, \emph{P.PPLA}|pw} 2 Uhr
                  \introOben{}Mg\introOben{}, Luzern\oindex{Luzern@\textbf{Luzern}, \emph{P.PPLA}|pw} 4.30.)\pend
           
\pstart
           Das wünſcht Ihnen{\\[\baselineskip]}herzlichſt Ihr \spacefill\mbox{A.}{\\[\baselineskip]}{[}hs. :{]} und \spacefill\mbox{Olga.}\pend
           \leftskip=0em{}\selectlanguage{ngerman}\endnumbering\briefempfaengerindex{Beer-Hofmann, Richard@\textsc{Beer-Hofmann, Richard}!zzzSchnitzler, Olga@\emph{von Olga Schnitzler}!1910-05-221@{22. 5. 1910}|)be}\briefempfaengerindex{Beer-Hofmann, Richard@\textsc{Beer-Hofmann, Richard}!zzzSchnitzler, Arthur@\emph{von Arthur Schnitzler}!1910-05-221@{22. 5. 1910}|)be}\mylabel{L01933h}  \normalsize

\doendnotes{C}
\bigskip
\vfill

\clearpage

\footnotesize

\lohead{\textsc{register}}

% Definiere theindex-Environment komplett neu ohne reledmac
\makeatletter
\renewenvironment{theindex}{%
  \section*{\indexname}%
  \setlength{\parindent}{0pt}%
  \setlength{\parskip}{0pt plus 0.3pt}%
  \let\item\@idxitem
}{%
  \clearpage
}
\makeatother

\IfFileExists{\jobname-pw.ind}{\input{\jobname-pw.ind}}{}

\end{document}

      