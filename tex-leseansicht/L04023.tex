%% latex-leseansicht-vorspann.tex
%% Vorspann für die Leseansicht.
%% Lädt die gemeinsame Datei latex-vorspann.tex mit nicht gesetztem Schalter.

\newif\ifkorrekturansicht
\korrekturansichtfalse

\input{../tex-inputs/latex-vorspann}


\section[Arthur Schnitzler an Gustav Schwarzkopf, 18. 7. 1905]{L04023 Arthur Schnitzler an Gustav Schwarzkopf, 18. 7. 1905}
\nopagebreak\mylabel{L04023v}
\rehead{ }\normalsize\beginnumbering\briefempfaengerindex{Schwarzkopf, Gustav@\textsc{Schwarzkopf, Gustav}!zzzSchnitzler, Arthur@\emph{von Arthur Schnitzler}!1905-07-181@{18. 7. 1905}|(be}
\toendnotes[C]{\smallbreak\pagebreak[2]}
\correspDesc{Versand  durch Arthur Schnitzler am 18. 7. 1905 in Reichenau an der Rax
\newline{}Erhalt  durch Gustav Schwarzkopf im Zeitraum [19. 7. 1905 – 23. 7. 1905?] in Wien}\toendnotes[C]{\smallbreak}
\Standort{CUL, Schnitzler, B 96.}
\physDesc{Brief, 3 Blätter, 1 Seite, 899 Zeichen
\newline{}Handschrift: schwarze Tinte, deutsche Kurrent}\toendnotes[C]{\smallbreak}
\pstart
           \raggedleft{}{\pb}Reichenau b/Payerbach\oindex{Reichenau an der Rax@\textbf{Reichenau an der Rax}, \emph{Verwaltungsgebiet}|pw}\pend
           
\pstart
           \raggedleft{}Kurhaus\oindex{Kurhaus Rudolfsbad@\textbf{Kurhaus Rudolfsbad}, \emph{Sanatorium}|pwv},
                     18. 7. 905\pend
           
\pstart{}lieber Guſtav,\pend\vspace{0.5em}
\pstart
           Sie wiſſen dſs Paul Marx\pwindex{Marx, Paul 21.\,7.\,1879 Wien – 30.\,10.\,1956 ebd.@\textsc{Marx, Paul} (21.\,7.\,1879 Wien – 30.\,10.\,1956 ebd.), \emph{Regisseur, Schauspieler}|pw} jetzt hier iſt, und,
               wenn Sie überhaupt kommen wollen, was wirklich hübſch von Ihnen wäre, ſo werden
               Sie ſichs wohl gern ſo einrichten, dſs Sie{ }ſowohl ihn als auch mich hier vorfinden.
               Nur wollen wir Mitte der Woche auf eine kleine Wanderſchaft {\pb}gehen, Hochschwabgebiet\oindex{Hochschwab@\textbf{Hochschwab}, \emph{Gebirge}|pw}, und bſind, denken wir, So{\geminationn}tag{ }\introOben{}(23.)\introOben{} (ſpäteſtens) zurück. Wie wärs nun, wenn Sie
               Anfangs nächſter Woche hier erſchienen \introOben{}(So{\geminationn}tag Abd. od Montag?)\introOben{}, aber nicht auf 1 Tag,{ }ſondern auf eine Woche etwa, im Anfang wäre da{\geminationn} auch Paul M.\pwindex{Marx, Paul 21.\,7.\,1879 Wien – 30.\,10.\,1956 ebd.@\textsc{Marx, Paul} (21.\,7.\,1879 Wien – 30.\,10.\,1956 ebd.), \emph{Regisseur, Schauspieler}|pw} noch da, und für den Reſt der Zeit
               könnten Sie \textcolor{gray}{ſich} vielleicht doch notdürftig mit uns und der Natur
               behelfen? Es iſt wirklich ſehr {\pb}ſchön
               hier. Theilen Sie uns näheres mit, auch ev. Wünſche, Zimmer betreffend \textsc{etc.} Herzlich grüßend, in der Hoffng Sie wirklich bald
               wiederzuſehn\pend
           
\pstart
           Ihr{\\[\baselineskip]}\spacefill\mbox{Arthur}\pend
           \leftskip=0em{}
\pstart
           \noindent{}Indeß hab ich hier das andere Stück\pwindex{Schnitzler, Arthur 15. 5. 1862 Wien – 21. 10. 1931 ebd.@\textsc{Schnitzler, Arthur} (15. 5. 1862 Wien – 21. 10. 1931 ebd.), \emph{Schriftsteller, Mediziner}!Ruf des Lebens. Schauspiel in drei Akten@\strich\emph{Der Ruf des Lebens. Schauspiel in drei Akten}|pwv}
                  ziemlich beendigt.\pend
           \selectlanguage{ngerman}\endnumbering\briefempfaengerindex{Schwarzkopf, Gustav@\textsc{Schwarzkopf, Gustav}!zzzSchnitzler, Arthur@\emph{von Arthur Schnitzler}!1905-07-181@{18. 7. 1905}|)be}\mylabel{L04023h}
\begin{anhang}
\end{anhang}\newcommand{\dateiname}{L04023}\newcommand{\titel}{Arthur Schnitzler an Gustav Schwarzkopf, 18. 7. 1905}\newcommand{\editorInnen}{Herausgegeben von Jahnke, SelmaMüller, Martin Anton}%% latex-leseansicht-abspann.tex
%% Abspann für die Leseansicht.
%% Der Schalter \ifkorrekturansicht ist bereits durch den Vorspann gesetzt.

%% latex-abspann.tex
%% Gemeinsamer Abspann für Korrekturansicht und Leseansicht.
%% Setzt den Schalter \ifkorrekturansicht voraus (gesetzt in den
%% einbindenden Dateien latex-korrekturansicht-abspann.tex bzw.
%% latex-leseansicht-abspann.tex).
%% ---------------------------------------------------------------

\normalsize

% Das esempio-Environment wird nur in der Leseansicht benötigt
\ifkorrekturansicht\else
\newenvironment{esempio}[3]%
{
    \vspace{1.5ex}
    \rlap{\underline{#1}}
    \par
    \setlength{\parindent}{0cm}
    \nopagebreak
    \leftskip=#2cm
    \rightskip=#3cm
}
{
    \par
}
\fi

\doendnotes{C}
\bigskip
\vfill

\clearpage

\footnotesize

\ifkorrekturansicht
  \lohead{\textsc{register}}
\fi

% theindex-Environment neu definieren ohne reledmac
\makeatletter
\renewenvironment{theindex}{%
  \ifkorrekturansicht
    \section*{\indexname}%
  \else
    \subsubsection*{Index der erwähnten Entitäten}%
  \fi
  \setlength{\parindent}{0pt}%
  \setlength{\parskip}{0pt plus 0.3pt}%
  \let\item\@idxitem
}{%
  \ifkorrekturansicht\clearpage\fi
}
\makeatother

\IfFileExists{\jobname-pw.ind}{\input{\jobname-pw.ind}}{}

% Quellenangabe nur in der Leseansicht
\ifkorrekturansicht\else
% Fallback-Definitionen, falls die .tex-Datei \titel etc. nicht gesetzt hat
\providecommand{\titel}{}
\providecommand{\editorInnen}{}
\providecommand{\dateiname}{\jobname}

\vspace{3cm}

\vfill

\footnotesize
\textsc{Quelle}: \titel. Herausgegeben von {\editorInnen}. In: \emph{Arthur Schnitzler: Briefwechsel mit Autorinnen und Autoren}.
 Digitale Edition, https://schnitzler-briefe.acdh.oeaw.ac.at/{\dateiname}.html (Stand \today)
\fi

\end{document}


