%% latex-korrekturansicht-vorspann.tex
%% Vorspann für die Korrekturansicht.
%% Lädt die gemeinsame Datei latex-vorspann.tex mit gesetztem Schalter.

\newif\ifkorrekturansicht
\korrekturansichttrue

\input{../tex-inputs/latex-vorspann}


\section[Richard Beer-Hofmann an Arthur Schnitzler, 5. 3. 1900]{L01018 Richard Beer-Hofmann an Arthur Schnitzler, 5. 3. 1900}
\nopagebreak\mylabel{L01018v}
\rehead{ }\normalsize\beginnumbering\briefempfaengerindex{Schnitzler, Arthur@\textsc{Schnitzler, Arthur}!zzzBeer-Hofmann, Richard@\emph{von Richard Beer-Hofmann}!1900-03-051@{5. 3. 1900}|(be}
\toendnotes[C]{\smallbreak\pagebreak[2]}\Standort{CUL, Schnitzler, B 8.}
\physDesc{Bildpostkarte, 145 Zeichen
\newline{}Handschrift: schwarze Tinte, lateinische Kurrent
\newline{}Versand: 1) Stempel: »\nobreak{}\oindex{Bahnhof Firenze Santa Maria Novella@\textbf{Bahnhof Firenze Santa Maria Novella}, \emph{Bahnhofsgebäude (K.BHF)}|pwk}Firenze Ferrovia, 5 3 {[}00{]}, 8 S\nobreak{}«.   2) Stempel: »\nobreak{}\oindex{IX., Alsergrund@\textbf{IX., Alsergrund}, \emph{A.ADM3}|pwk}{[}Wien 9/{]}3, 7. 3. 00, 8.V\nobreak{}«. 
\newline{}Ordnung: mit Bleistift von unbekannter Hand nummeriert:
                                    »152« }\pstart{}{\pb}D\textsuperscript{r}
                  Arthur Schnitzler\pend{}\pstart{}Wien\oindex{Wien@\textbf{Wien}, \emph{A.ADM2}|pw}\pend{}\pstart{}IX Frankgasse 1\oindex{Frankgasse 1@\textbf{Frankgasse 1}, \emph{Wohngebäude (K.WHS)}|pw}\pend{}\pstart{}Austria\oindex{Oesterreich@\textbf{Österreich}, \emph{A.PCLI}|pw}\pend{}{\bigskip}
\pstart
           \noindent{}\centering{}{\pb}\textcolor{gray}{\textbf{R. Galleria Uffizi\oindex{Uffizien@\textbf{Uffizien}, \emph{Museum (K.MUS)}|pw}. – L’incoronazione della Vergine\pwindex{Kroenung der Jungfrau@\emph{Die Krönung der Jungfrau}|pw} (Botticelli\pwindex{Botticelli, Sandro 1445-03-01 – 1510-05-17@\textsc{Botticelli, Sandro} (1445-03-01 – 1510-05-17), \emph{Maler/Malerin}|pw}) Firenze\oindex{Florenz@\textbf{Florenz}, \emph{P.PPLA}|pw}}}\pend
           \vspace{1em}
\pstart
           \raggedleft{}{\pb}5/III 900\pend
           \vspace{0.5em}
\pstart
           Lieber Arthur! Ich denke am 11, vielleicht schon
                  10. Nachts wieder in Wien\oindex{Wien@\textbf{Wien}, \emph{A.ADM2}|pw} zu sein.
               Herzlich\pend
           \pstart Ihr \spacefill\mbox{R.}\pend{}\selectlanguage{ngerman}\endnumbering\briefempfaengerindex{Schnitzler, Arthur@\textsc{Schnitzler, Arthur}!zzzBeer-Hofmann, Richard@\emph{von Richard Beer-Hofmann}!1900-03-051@{5. 3. 1900}|)be}\mylabel{L01018h}  \normalsize

\doendnotes{C}
\bigskip
\vfill

\clearpage

\footnotesize

\lohead{\textsc{register}}

% Definiere theindex-Environment komplett neu ohne reledmac
\makeatletter
\renewenvironment{theindex}{%
  \section*{\indexname}%
  \setlength{\parindent}{0pt}%
  \setlength{\parskip}{0pt plus 0.3pt}%
  \let\item\@idxitem
}{%
  \clearpage
}
\makeatother

\IfFileExists{\jobname-pw.ind}{\input{\jobname-pw.ind}}{}

\end{document}

      