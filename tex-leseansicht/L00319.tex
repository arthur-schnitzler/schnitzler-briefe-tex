%% latex-korrekturansicht-vorspann.tex
%% Vorspann für die Korrekturansicht.
%% Lädt die gemeinsame Datei latex-vorspann.tex mit gesetztem Schalter.

\newif\ifkorrekturansicht
\korrekturansichttrue

\input{../tex-inputs/latex-vorspann}


\section[Arthur Schnitzler: Widmungsexemplar Das Märchen für Hermann Bahr, {[}8. 5.?{]} 1894]{L00319 Arthur Schnitzler: Widmungsexemplar Das Märchen für Hermann Bahr,
               {[}8. 5.?{]} 1894}
\nopagebreak\mylabel{L00319v}
\rehead{ }\normalsize\beginnumbering\briefempfaengerindex{Bahr, Hermann@\textsc{Bahr, Hermann}!zzzSchnitzler, Arthur@\emph{von Arthur Schnitzler}!1894-05-081@{{[}8. 5.?{]} 1894}|(be}
\toendnotes[C]{\smallbreak\pagebreak[2]}\Standort{Salzburg, Universitätsbibliothek, 32340-I.}
\physDesc{Widmung am Vorsatzblatt, 45 Zeichen
\newline{}Handschrift: schwarze Tinte, deutsche Kurrent}
\buchAbdrucke{\weitereDrucke{Hermann Bahr, Arthur Schnitzler: \emph{Briefwechsel, Aufzeichnungen, Dokumente (1891–1931)}. Göttingen: \emph{Wallstein} 2018, S. 71.} }\toendnotes[C]{\smallbreak}
\pstart
           \noindent{}{\pb}Meinem lieben Hermann Bahr{\\}herzlichſt\pend
           \pstart \spacefill\mbox{ArthSch}\pend{}\selectlanguage{ngerman}\vspace{1em}{\vspace{1\baselineskip}}
\pstart
           \centering{}{\pb}\textcolor{gray}{\textbf{Das Märchen\pwindex{Maerchen. Schauspiel in drei Aufzuegen@\emph{Das Märchen. Schauspiel in drei Aufzügen}|pw}.}}\pend
           
\pstart
           \centering{}\textcolor{gray}{\textbf{\textbf{Schauſpiel in drei Aufzügen}}}{\\}\textcolor{gray}{\textbf{von}}{\\}\textcolor{gray}{\textbf{\textbf{Arthur Schnitzler.}}}\pend
           {\vspace{1\baselineskip}}
\pstart
           \centering{}\textcolor{gray}{\textbf{\textbf{Dresden}\oindex{Dresden@\textbf{Dresden}, \emph{P.PPLA}|pw} und \textbf{Leipzig}\oindex{Leipzig@\textbf{Leipzig}, \emph{P.PPLA3}|pw}}}\pend
           
\pstart
           \centering{}\textcolor{gray}{\textbf{\so{E. Pierſon’s Verlag}\orgindex{E. Pierson s Verlag@E. Pierson’s Verlag|pw}}}\pend
           
\pstart
           \centering{}\textcolor{gray}{\textbf{\label{K_L00319-1v}\edtext{1894}{\lemma{\textnormal{\emph{1894}}}\Cendnote{\textnormal{\emph{Das Märchen}\pwindex{Maerchen. Schauspiel in drei Aufzuegen@\emph{Das Märchen. Schauspiel in drei Aufzügen}|pwk} war am 5. 5. 1894 vom \emph{Börsenblatt für den deutschen Buchhandel}\pwindex{Boersenblatt fuer den Deutschen Buchhandel@\emph{Börsenblatt für den Deutschen Buchhandel}|pwk} als Neuerscheinung
                        gemeldet worden. Die Datierung folgt jener der Widmung an Felix Salten\pwindex{Salten, Felix 06.09.1869 – 08.10.1945@\textsc{Salten, Felix} (06.09.1869 – 08.10.1945), \emph{Schriftsteller/Schriftstellerin, Journalist/Journalistin, Chefredakteur/Chefredakteurin}|pwk} vom 8. 5. 94.}}}\label{K_L00319-1}.}}\pend
           \selectlanguage{ngerman}\endnumbering\briefempfaengerindex{Bahr, Hermann@\textsc{Bahr, Hermann}!zzzSchnitzler, Arthur@\emph{von Arthur Schnitzler}!1894-05-081@{{[}8. 5.?{]} 1894}|)be}\mylabel{L00319h}  \normalsize

\doendnotes{C}
\bigskip
\vfill

\clearpage

\footnotesize

\lohead{\textsc{register}}

% Definiere theindex-Environment komplett neu ohne reledmac
\makeatletter
\renewenvironment{theindex}{%
  \section*{\indexname}%
  \setlength{\parindent}{0pt}%
  \setlength{\parskip}{0pt plus 0.3pt}%
  \let\item\@idxitem
}{%
  \clearpage
}
\makeatother

\IfFileExists{\jobname-pw.ind}{\input{\jobname-pw.ind}}{}

\end{document}

      