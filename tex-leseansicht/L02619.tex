%% latex-korrekturansicht-vorspann.tex
%% Vorspann für die Korrekturansicht.
%% Lädt die gemeinsame Datei latex-vorspann.tex mit gesetztem Schalter.

\newif\ifkorrekturansicht
\korrekturansichttrue

\input{../tex-inputs/latex-vorspann}


\section[Paul Goldmann an Arthur Schnitzler, 1. 5. {[}1894{]}]{L02619 Paul Goldmann an Arthur Schnitzler, 1. 5. {[}1894{]}}
\nopagebreak\mylabel{L02619v}
\rehead{ }\normalsize\beginnumbering\briefempfaengerindex{Schnitzler, Arthur@\textsc{Schnitzler, Arthur}!zzzGoldmann, Paul@\emph{von Paul Goldmann}!1894-05-011@{1. 5. {[}1894{]}}|(be}
\toendnotes[C]{\smallbreak\pagebreak[2]}\Standort{DLA, A:Schnitzler, HS.NZ85.1.3164.}
\physDesc{Brief, 2 Blätter, 8 Seiten, 4005 Zeichen
\newline{}Handschrift: schwarze Tinte, deutsche Kurrent
\newline{}Schnitzler: 1) mit Bleistift auf dem ersten Blatt die Jahreszahl »94« vermerkt  2) mit rotem Buntstift sieben Unterstreichungen}\toendnotes[C]{\smallbreak}
\pstart
           {\pb}\textcolor{gray}{\textbf{Frankfurter Zeitung\orgindex{Frankfurter Zeitung@Frankfurter Zeitung|pw}}}\hfill \textsc{Paris\oindex{Paris@\textbf{Paris}, \emph{P.PPLC}|pw}}, 1. Mai.\pend
           
\pstart
           \textcolor{gray}{\textbf{(Gazette de
                     Francfort\orgindex{Frankfurter Zeitung@Frankfurter Zeitung|pw}).}}\pend
           
\pstart
           \textcolor{gray}{\textbf{\begin{otherlanguage}{french}Fondateur \textbf{M. L. Sonnemann\pwindex{Sonnemann, Leopold 1831-10-29 – 1909-10-30@\textsc{Sonnemann, Leopold} (1831-10-29 – 1909-10-30), \emph{Journalist/Journalistin, Herausgeber/Herausgeberin}|pw}}.\end{otherlanguage}}}\pend
           
\pstart
           \textcolor{gray}{\textbf{\begin{otherlanguage}{french}Journal politique, financier,\end{otherlanguage}}}\pend
           
\pstart
           \textcolor{gray}{\textbf{\begin{otherlanguage}{french}commercial et littéraire.\end{otherlanguage}}}\pend
           
\pstart
           \textcolor{gray}{\textbf{\begin{otherlanguage}{french}\textbf{Paraissant trois fois par jour}\end{otherlanguage}}}.\pend
           
\pstart
           \textcolor{gray}{\textbf{\begin{otherlanguage}{french}\textbf{Bureaux à Paris\oindex{Paris@\textbf{Paris}, \emph{P.PPLC}|pw}:}\end{otherlanguage}}}\pend
           
\pstart
           \textcolor{gray}{\textbf{\begin{otherlanguage}{french}\textbf{24. Rue Feydeau}\oindex{rue Feydeau@\textbf{rue Feydeau}, \emph{Straße (K.STR)}|pw}.\end{otherlanguage}}}\pend
           
\pstart\center{}Mein lieber Arthur,\pend\vspace{0.5em}
\pstart
           Anbei erhälſt Du den \label{K_L02619-1v}\edtext{»\textsc{Mercure de France\pwindex{Mercure de France@\emph{Mercure de France}|pw}}«, wo \textsc{Henri Albert\pwindex{Albert, Henri 1869-11-16 – 1921-08-03@\textsc{Albert, Henri} (1869-11-16 – 1921-08-03), \emph{Journalist/Journalistin, Kritiker/Kritikerin, Übersetzer/Übersetzerin}|pw}}{ }\strikeout{Dich} gelegentlich wieder von Deinem Talente ſpricht
                  (S.{ }92\pwindex{Journaux et Revues@\emph{Journaux et Revues}|pwv})}{\lemma{\textnormal{\emph{»Mercure … 92)}}}\Cendnote{\textnormal{Henri Albert\pwindex{Albert, Henri 1869-11-16 – 1921-08-03@\textsc{Albert, Henri} (1869-11-16 – 1921-08-03), \emph{Journalist/Journalistin, Kritiker/Kritikerin, Übersetzer/Übersetzerin}|pwk}: \emph{Journaux et Revues}\pwindex{Journaux et Revues@\emph{Journaux et Revues}|pwk}. In: \emph{Mercure de France}\pwindex{Mercure de France@\emph{Mercure de France}|pwk}, Jg. 11, Nr. 53, Mai 1894, S. 87–92, hier:
                     S. 92.}}}\label{K_L02619-1}. Was zahlſt Du uns eigentlich für die Reklame?\pend
           
\pstart
           Ich danke Dir herzlichſt für die Überſendung der beiden Skizzen\pwindex{ueberspannte Person@\emph{Die überspannte Person}|pwv}\pwindex{Halb Zwei@\emph{Halb Zwei}|pwv}\textcolor{gray}{,} komme erſt Ende der Woche dazu, ſie in Ruhe zu leſen, und
                  \label{K_L02619-2v}\edtext{ſchreibe Dir dann ſofort
                  darüber.}{\lemma{\textnormal{\emph{ſchreibe … darüber.}}}\Cendnote{\textnormal{Siehe Paul Goldmann an Arthur Schnitzler, 29. 5. [1894].
               }}}\label{K_L02619-2}{ }\textsc{Albert\pwindex{Albert, Henri 1869-11-16 – 1921-08-03@\textsc{Albert, Henri} (1869-11-16 – 1921-08-03), \emph{Journalist/Journalistin, Kritiker/Kritikerin, Übersetzer/Übersetzerin}|pw}} ſehe ich morgen und werde Dir dann berichten,
               wie es mit Deiner Überſetzung\pwindex{Emplettes de Noel@\emph{Les Emplettes de Noël}|pwv}
               ſteht. Schicke ihm das Honorar, wenn Du kannſt, gleich, an ſeine {\pb}Adreſſe\oindex{rue Jacob@\textbf{rue Jacob}, \emph{Straße (K.STR)}|pwv}, ohne weitere
               Bemerkung. \strikeout{\textcolor{gray}{E}} Ich beſorge ſchon den nöthigen Commentar. Ich denke 10 bis 12 Gulden, wenn Dir
               das nicht zu viel iſt. Kannſt Du jetzt nicht, ſo warte ruhig, bis Du von ihm etwas
               Poſitives über den Ausgang der Arbeit erfährſt. Ich veranlaſſe ihn jedenfalls,
               demnächſt an Dich zu ſchreiben{\dotsfour}\pend
           
\pstart
           Bitte, dementire auf das Energiſcheſte das Gerücht von meiner Candidatur auf \label{K_L02619-3v}\edtext{\textsc{Herzls\pwindex{Herzl, Theodor 1860-05-02 – 1904-07-03@\textsc{Herzl, Theodor} (1860-05-02 – 1904-07-03), \emph{Schriftsteller/Schriftstellerin, Journalist/Journalistin}|pw}} Nachfolge}{\lemma{\textnormal{\emph{Herzls Nachfolge}}}\Cendnote{\textnormal{Als Korrespondent der \emph{Neuen Freien Presse}\orgindex{Neue Freie Presse@Neue Freie Presse|pwk} in Paris\oindex{Paris@\textbf{Paris}, \emph{P.PPLC}|pwk}. Herzl\pwindex{Herzl, Theodor 1860-05-02 – 1904-07-03@\textsc{Herzl, Theodor} (1860-05-02 – 1904-07-03), \emph{Schriftsteller/Schriftstellerin, Journalist/Journalistin}|pwk} hatte die Stellung
                  von Oktober 1891 bis Juli 1895 inne.}}}\label{K_L02619-3}. Es iſt nicht ein wahres Wort daran, und wenn es
               meiner Redaction\orgindex{Frankfurter Zeitung@Frankfurter Zeitung|pwv} zu Ohren
               kommt, kann es nur meine jetzige Stellung gefährden. Daß \textsc{Herzl\pwindex{Herzl, Theodor 1860-05-02 – 1904-07-03@\textsc{Herzl, Theodor} (1860-05-02 – 1904-07-03), \emph{Schriftsteller/Schriftstellerin, Journalist/Journalistin}|pw}} weggeht {\pb}iſt möglich. Aber niemals wird man
               mich zur »Neuen Fr. Preſſe\orgindex{Neue Freie Presse@Neue Freie Presse|pw}« nehmen. Zwiſchen dem
                  Blatte\orgindex{Neue Freie Presse@Neue Freie Presse|pwv} und meinem Onkel\pwindex{Mamroth, Fedor 21.02.1851 – 25.06.1907@\textsc{Mamroth, Fedor} (21.02.1851 – 25.06.1907), \emph{Journalist/Journalistin, Kritiker/Kritikerin}|pwv} beſteht, wie Du wohl
               weißt, eine \label{K_L02619-4v}\edtext{tödtliche Feindſchaft}{\lemma{\textnormal{\emph{tödtliche Feindſchaft}}}\Cendnote{\textnormal{Mamroth\pwindex{Mamroth, Fedor 21.02.1851 – 25.06.1907@\textsc{Mamroth, Fedor} (21.02.1851 – 25.06.1907), \emph{Journalist/Journalistin, Kritiker/Kritikerin}|pwk} hatte seine Laufbahn
                     1873 als Korrektor bei der \emph{Neuen
                     Freien Presse}\orgindex{Neue Freie Presse@Neue Freie Presse|pwk} begonnen, wechselte dann in Folge aber zu anderen Wien\oindex{Wien@\textbf{Wien}, \emph{A.ADM2}|pwk}er Zeitschriften und Zeitungen, bevor er ab
                     1. 4. 1889 das Feuilleton der \emph{Frankfurter Zeitung}\orgindex{Frankfurter Zeitung@Frankfurter Zeitung|pwk} betreute.}}}\label{K_L02619-4}. Und dieſe Leute mit ihren
               Börſenjobber-Seelen haſſen bis ins ſiebente Glied. Als \textsc{Benedict\pwindex{Benedikt, Moriz 27.05.1849 – 18.03.1920@\textsc{Benedikt, Moriz} (27.05.1849 – 18.03.1920), \emph{Journalist/Journalistin, Herausgeber/Herausgeberin}|pw}} vor einigen Monaten hier war, hat er es abgelehnt, daß ich ihm vorgeſtellt
               werde! Dazu kommt, daß \textsc{Herzl\pwindex{Herzl, Theodor 1860-05-02 – 1904-07-03@\textsc{Herzl, Theodor} (1860-05-02 – 1904-07-03), \emph{Schriftsteller/Schriftstellerin, Journalist/Journalistin}|pw}} ſelbſt keinen Finger rühren wird, um meine Candidatur zu ſtützen, eher das
               Gegentheil. Ich habe ihn hier genau kennen gelernt. Er iſt {\pb}eine ſeltſame Miſchung von Künſtler und jüdiſchem
               Journaliſten. Auf der einen, der Künſtler-Seite, charmant, glänzend, ſympathiſch; auf
               der andern Seite: kleinlich, eiferſüchtig, \strikeout{b\textcolor{gray}{er}} geheimnißthueriſch, berechnend und größenwahnſinnig. Ich will ja nicht ſagen,
               daß er gegen meine Candidatur intriguiren würde – obwohl es mich nicht erſtaunen
               würde, wenn ers thäte – aber er wird ſicher nicht das Mindeſte thun, um mich, vor
               deſſen Nebenbuhlerſchaft er ſich fürchtet – der Dummkopf\pwindex{Herzl, Theodor 1860-05-02 – 1904-07-03@\textsc{Herzl, Theodor} (1860-05-02 – 1904-07-03), \emph{Schriftsteller/Schriftstellerin, Journalist/Journalistin}|pwv}! – an ſeine Stelle zu bringen. Das Alles hindert
               aber {\pb}nicht, daß er jetzt einen Einakter\pwindex{Glosse. Lustspiel in einem Act@\emph{Die Glosse. Lustspiel in einem Act}|pwu} in Verſen geſchrieben, der ein Stück
               köſtlicher und großer Kunſt iſt. Zu Niemandem ein Wort von alledem, nicht wahr? Noch
               eins: \textsc{Dr. Schwitzer\pwindex{Schwitzer, Ludwig 1850 – 1937@\textsc{Schwitzer, Ludwig} (1850 – 1937), \emph{Journalist/Journalistin}|pw}}, früheres Mitglied\pwindex{Schwitzer, Ludwig 1850 – 1937@\textsc{Schwitzer, Ludwig} (1850 – 1937), \emph{Journalist/Journalistin}|pwv} der
               volkswirthſchaftlichen Redaction der N. Fr. Pr.\orgindex{Neue Freie Presse@Neue Freie Presse|pw},
               iſt plötzlich hier\oindex{Paris@\textbf{Paris}, \emph{P.PPLC}|pwv} aufgetaucht
               und ich glaube, \label{K_L02619-5v}\edtext{\begin{otherlanguage}{french}\textsc{c’est pour recueillir la succession}\end{otherlanguage}}{\lemma{\textnormal{\emph{c’est … succession}}}\Cendnote{\textnormal{französisch: um die Nachfolge zu
                  besorgen}}}\label{K_L02619-5}.\pend
           
\pstart
           \textsc{Rudolf Lothar\pwindex{Lothar, Rudolf 23.2.1865 – 2.10.1943@\textsc{Lothar, Rudolf} (23.2.1865 – 2.10.1943), \emph{Schriftsteller/Schriftstellerin, Journalist/Journalistin, Theaterdirektor/Theaterdirektorin}|pw}} iſt auf einer ſeiner literariſchen Handlungsreiſen auch hier\oindex{Paris@\textbf{Paris}, \emph{P.PPLC}|pwv} eingetroffen. Er will alle {\pb}möglichen Leute interviewen, \textsc{Pailleron\pwindex{Pailleron, Edouard 1834-09-07 – 1890@\textsc{Pailleron, Édouard} (1834-09-07 – 1890), \emph{Schriftsteller/Schriftstellerin}|pw}} und \textsc{Verlaine\pwindex{Verlaine, Paul 30.03.1844 – 08.01.1896@\textsc{Verlaine, Paul} (30.03.1844 – 08.01.1896), \emph{Schriftsteller/Schriftstellerin}|pw}}, Kraut und Rüben durcheinander. Er hat ſich an \textsc{Henri Albert\pwindex{Albert, Henri 1869-11-16 – 1921-08-03@\textsc{Albert, Henri} (1869-11-16 – 1921-08-03), \emph{Journalist/Journalistin, Kritiker/Kritikerin, Übersetzer/Übersetzerin}|pw}} herangedrängt, um \label{K_L02619-6v}\edtext{im »\textsc{Mercure\pwindex{Mercure de France@\emph{Mercure de France}|pw}}« genannt}{\lemma{\textnormal{\emph{im »Mercure« genannt}}}\Cendnote{\textnormal{nicht ermittelt}}}\label{K_L02619-6} zu
               werden \textsc{etc}. Ich habe einen grämlichen Haß gegen dieſen Burſchen\pwindex{Lothar, Rudolf 23.2.1865 – 2.10.1943@\textsc{Lothar, Rudolf} (23.2.1865 – 2.10.1943), \emph{Schriftsteller/Schriftstellerin, Journalist/Journalistin, Theaterdirektor/Theaterdirektorin}|pwv}, der im führenden Blatte\pwindex{Neue Freie Presse@\emph{Neue Freie Presse}|pwv} Literaturmeinung macht
               und deſſen Stücke als die Blüthe des jungen Geiſtes \strikeout{\textcolor{gray}{a}\textcolor{gray}{×}} auf allen Jahrmärkten angeprieſen werden, während Du vorläufig nur von einer
               Elite gekannt und gewürdigt biſt. Ich finde, er hat Dir direct ſeine Celebrität
               geſtohlen. Und als ich dieſen geſchäftigen {\pb}Barbiergeſellen\pwindex{Lothar, Rudolf 23.2.1865 – 2.10.1943@\textsc{Lothar, Rudolf} (23.2.1865 – 2.10.1943), \emph{Schriftsteller/Schriftstellerin, Journalist/Journalistin, Theaterdirektor/Theaterdirektorin}|pwv} neulich im
               Theater traf, drehte ich ihm einfach den Rücken. Das war wohl exceſſiv, aber ich kann
               nichts gegen mein Temperament.\pend
           
\pstart
           Ein grünes einſames windſtilles Land! Wie, wenn Du auch nach \textsc{Hamburg\oindex{Hamburg@\textbf{Hamburg}, \emph{P.PPLA}|pw}} kämeſt, wo ich wahrscheinlich meinen Uraub werde verbringen müſſen. Und wann,
               wann endlich werde ich Dich \label{K_L02619-7v}\edtext{in \textsc{Paris\oindex{Paris@\textbf{Paris}, \emph{P.PPLC}|pw}} ſehen}{\lemma{\textnormal{\emph{in Paris ſehen}}}\Cendnote{\textnormal{Erst 1897 reiste
                  Schnitzler nach Paris\oindex{Paris@\textbf{Paris}, \emph{P.PPLC}|pwk}.}}}\label{K_L02619-7}? Komm doch
               wenigſtens auf 14 Tage! Wenn Du nicht ſo ein verwöhnter Prinz wäreſt, könnteſt Du
               ſogar bei mir wohnen\substVorne{}\textsuperscript{.}\substDazwischen{},\substHinten{} aber ohne jeden Comfort.\pend
           
\pstart
           {\pb}Tauſend Dank auch für alles Liebe, das Du mir ſonſt
               ſagſt. Es iſt immer Feſttag bei mir, wenn ein Brief von Dir ankommt. Wie kann ich Dir
               das Alles lohnen\substVorne{}\textsuperscript{?}\substDazwischen{}!\substHinten{}!\pend
           
\pstart
           Möchte gern etwas Näheres über die große \label{K_L02619-8v}\edtext{Erzählung\pwindex{Sterben. Novelle@\emph{Sterben. Novelle}|pwv}}{\lemma{\textnormal{\emph{Erzählung}}}\Cendnote{\textnormal{Die Novelle \emph{Sterben}\pwindex{Sterben. Novelle@\emph{Sterben. Novelle}|pwk} war im Frühjahr 1894 vom \emph{S. Fischer-Verlag}\orgindex{S. Fischer Verlag@S. Fischer Verlag|pwk} akzeptiert worden. Der
                  Erstdruck erschien zwischen Oktober und Dezember in drei
                  Teilen in der \emph{Neuen Deutschen
                  Rundschau}\pwindex{Neue Deutsche Rundschau@\emph{Neue Deutsche Rundschau}|pwk}.}}}\label{K_L02619-8} wiſſen.\pend
           
\pstart
           Weißt Du, daß deine Schrift immer ſchlechter wird? Ich kann ſie zur Noth noch
               entziffern, weil ich die hiſtorische Entwickelung mitgemacht habe. Aber die Andern?
               Dein zukünftiger Biograph? Der Sammler deiner nachgelaſſenen Schriften? {\dotsfour}\pend
           
\pstart
           Grüß’ Dich Gott, mein theurer Freund, und ſchreib’ mir bald. Auch von den Andern, \textsc{Loris\pwindex{Hofmannsthal, Hugo von 1874-02-01 – 1929-07-15@\textsc{Hofmannsthal, Hugo von} (1874-02-01 – 1929-07-15), \emph{Schriftsteller/Schriftstellerin}|pw}} u. \textsc{Richard\pwindex{Beer-Hofmann, Richard 1866-07-11 – 1945-09-26@\textsc{Beer-Hofmann, Richard} (1866-07-11 – 1945-09-26), \emph{Schriftsteller/Schriftstellerin}|pw}}.\pend
           
\pstart
           Dein treuer {\\[\baselineskip]}\spacefill\mbox{Paul Goldm.}\pend
           \leftskip=0em{}\selectlanguage{ngerman}\endnumbering\briefempfaengerindex{Schnitzler, Arthur@\textsc{Schnitzler, Arthur}!zzzGoldmann, Paul@\emph{von Paul Goldmann}!1894-05-011@{1. 5. {[}1894{]}}|)be}\mylabel{L02619h}  \normalsize

\doendnotes{C}
\bigskip
\vfill

\clearpage

\footnotesize

\lohead{\textsc{register}}

% Definiere theindex-Environment komplett neu ohne reledmac
\makeatletter
\renewenvironment{theindex}{%
  \section*{\indexname}%
  \setlength{\parindent}{0pt}%
  \setlength{\parskip}{0pt plus 0.3pt}%
  \let\item\@idxitem
}{%
  \clearpage
}
\makeatother

\IfFileExists{\jobname-pw.ind}{\input{\jobname-pw.ind}}{}

\end{document}

      