%% latex-leseansicht-vorspann.tex
%% Vorspann für die Leseansicht.
%% Lädt die gemeinsame Datei latex-vorspann.tex mit nicht gesetztem Schalter.

\newif\ifkorrekturansicht
\korrekturansichtfalse

\input{../tex-inputs/latex-vorspann}


\section[Arthur Schnitzler an Gustav Schwarzkopf, 18. 8. 1899]{L04203 Arthur Schnitzler an Gustav Schwarzkopf, 18. 8. 1899}
\nopagebreak\mylabel{L04203v}
\rehead{ }\normalsize\beginnumbering\briefempfaengerindex{Schwarzkopf, Gustav@\textsc{Schwarzkopf, Gustav}!zzzSchnitzler, Arthur@\emph{von Arthur Schnitzler}!1899-08-181@{18. 8. 1899}|(be}
\toendnotes[C]{\smallbreak\pagebreak[2]}
\correspDesc{Versand  durch Arthur Schnitzler am 18. 8. 1899 in Bad Ischl
\newline{}Erhalt  durch Gustav Schwarzkopf am 18. 8. 1899 in Wien}\toendnotes[C]{\smallbreak}
\Standort{CUL, Schnitzler, B 96.}
\physDesc{Telegramm, 143 Zeichen
\newline{}maschinell
\newline{}Versand: 1) »\noindent{}\textcolor{gray}{\textbf{\textit{18/8{ }\textcolor{gray}{×}{ }106}}}{ / }\textcolor{gray}{\textbf{\textit{Hans Schick\pwindex{\textcolor{red}{\textsuperscript{XXXX indx1}}|pw}}}}«  2) »\textcolor{gray}{\textbf{\textit{Ausgefertigt 18 Aug 9{ }2}}}«}\toendnotes[C]{\smallbreak}\pstart{}{\pb}gustav schwarzkopf\pend{}\pstart{}wien tiefergraben 23\oindex{Wien@\textbf{Wien}!I., Innere Stadt@\textbf{I., Innere Stadt}!Tiefer Graben 23@\textbf{Tiefer Graben 23}, \emph{Wohngebäude}|pw}=\pend{}{\bigskip}\vspace{1em}
\pstart
           \centering{}{\pb}win\oindex{Wien@\textbf{Wien}, \emph{Verwaltungsgebiet}|pw} fr ischl\oindex{Bad Ischl@\textbf{Bad Ischl}|pw}. 3803 17 18/8{ }7/45 +\pend
           \vspace{0.5em}
\pstart
           laszen sie dieses telegramm den \label{T_L04203-1v}\edtext{rueck
                  \hspace*{2.5em} sein}{\lemma{\textnormal{\emph{rueck
                   sein}}}\Cendnote{\textnormal{Ob der Leerraum andeutet, dass ein Teil der Botschaft nicht entziffert ist oder
                  ob nur beim Empfang ein Wort falsch transkribiert wurde, ist nicht zu
                  klären.}}}\label{T_L04203-1} und kommen sie herzlichst \spacefill\mbox{arthur ="}\pend
           \selectlanguage{ngerman}\endnumbering\briefempfaengerindex{Schwarzkopf, Gustav@\textsc{Schwarzkopf, Gustav}!zzzSchnitzler, Arthur@\emph{von Arthur Schnitzler}!1899-08-181@{18. 8. 1899}|)be}\mylabel{L04203h}
\begin{anhang}
\end{anhang}\newcommand{\dateiname}{L04203}\newcommand{\titel}{Arthur Schnitzler an Gustav Schwarzkopf, 18. 8. 1899}\newcommand{\editorInnen}{Herausgegeben von Jahnke, SelmaMüller, Martin Anton}%% latex-leseansicht-abspann.tex
%% Abspann für die Leseansicht.
%% Der Schalter \ifkorrekturansicht ist bereits durch den Vorspann gesetzt.

%% latex-abspann.tex
%% Gemeinsamer Abspann für Korrekturansicht und Leseansicht.
%% Setzt den Schalter \ifkorrekturansicht voraus (gesetzt in den
%% einbindenden Dateien latex-korrekturansicht-abspann.tex bzw.
%% latex-leseansicht-abspann.tex).
%% ---------------------------------------------------------------

\normalsize

% Das esempio-Environment wird nur in der Leseansicht benötigt
\ifkorrekturansicht\else
\newenvironment{esempio}[3]%
{
    \vspace{1.5ex}
    \rlap{\underline{#1}}
    \par
    \setlength{\parindent}{0cm}
    \nopagebreak
    \leftskip=#2cm
    \rightskip=#3cm
}
{
    \par
}
\fi

\doendnotes{C}
\bigskip
\vfill

\clearpage

\footnotesize

\ifkorrekturansicht
  \lohead{\textsc{register}}
\fi

% theindex-Environment neu definieren ohne reledmac
\makeatletter
\renewenvironment{theindex}{%
  \ifkorrekturansicht
    \section*{\indexname}%
  \else
    \subsubsection*{Index der erwähnten Entitäten}%
  \fi
  \setlength{\parindent}{0pt}%
  \setlength{\parskip}{0pt plus 0.3pt}%
  \let\item\@idxitem
}{%
  \ifkorrekturansicht\clearpage\fi
}
\makeatother

\IfFileExists{\jobname-pw.ind}{\input{\jobname-pw.ind}}{}

% Quellenangabe nur in der Leseansicht
\ifkorrekturansicht\else
% Fallback-Definitionen, falls die .tex-Datei \titel etc. nicht gesetzt hat
\providecommand{\titel}{}
\providecommand{\editorInnen}{}
\providecommand{\dateiname}{\jobname}

\vspace{3cm}

\vfill

\footnotesize
\textsc{Quelle}: \titel. Herausgegeben von {\editorInnen}. In: \emph{Arthur Schnitzler: Briefwechsel mit Autorinnen und Autoren}.
 Digitale Edition, https://schnitzler-briefe.acdh.oeaw.ac.at/{\dateiname}.html (Stand \today)
\fi

\end{document}


