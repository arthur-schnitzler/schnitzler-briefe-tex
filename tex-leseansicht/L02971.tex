%% latex-leseansicht-vorspann.tex
%% Vorspann für die Leseansicht.
%% Lädt die gemeinsame Datei latex-vorspann.tex mit nicht gesetztem Schalter.

\newif\ifkorrekturansicht
\korrekturansichtfalse

\input{../tex-inputs/latex-vorspann}


         
         \renewcommand{\erwaehntePersonen}{Personen: Otto Eisenschitz, Edmond Huot de Goncourt, Friedrich Hebbel, Gottfried Keller, Catulle Mendès, Felix Salten, Hugo Salus, Pierre Veber}
         \renewcommand{\erwaehnteInstitutionen}{Institutionen: Jung-Wiener Theater zum Lieben Augustin}
         \renewcommand{\erwaehnteOrte}{Orte: Berlin, Wien}
         \renewcommand{\erwaehnteWerke}{Werke: Altes Ghettoliedchen, Am Fenster, Das Pfeifchen, Die Insel. Monatsschrift mit Buchschmuck und Illustrationen, Illustrirtes Wiener Extrablatt, Klage der Magd, Schlange}
               \section[ Arthur Schnitzler an Felix Salten, 6. 10. 1901]{ Arthur Schnitzler an Felix Salten, 6. 10. 1901}\nopagebreak\mylabel{v}\rehead{ }\begin{ledgroupsized}[t]{13cm}\normalsize\beginnumbering\briefempfaengerindex{Salten, Felix@\textsc{Salten, Felix}!zzzSchnitzler, Arthur@\emph{von Arthur Schnitzler}!1901-10-061@{6. 10. 1901}|(be} \toendnotes[C]{\smallbreak\pagebreak[2]} \Standort{Wienbibliothek im Rathaus, ZPH 1681, 2.1.516.}
\physDesc{Brief, 1 Blatt, 2 Seiten, 411 Zeichen
\newline{}Handschrift: Bleistift, deutsche Kurrent
\newline{}Ordnung: mit Bleistift von unbekannter Hand nummeriert: »22« }\toendnotes[C]{\smallbreak}\pstart
           \raggedleft{}{\pb}6/10 901\pend
           \pstart
           lieber, hier iſt \label{K_L02971-1v}\edtext{Inſel\pwindex{Insel. Monatsschrift mit Buchschmuck und Illustrationen1899 – 1902@\emph{Die Insel. Monatsschrift mit Buchschmuck und Illustrationen} {[}1899 – 1902{]}|pw}}{\lemma{\textnormal{\emph{Inſel}}}\Cendnote{\textnormal{Vgl. Felix Salten an Arthur Schnitzler, 28. 7. 1901. }}}\label{K_L02971-1h} und \label{K_L02971-2v}\edtext{Schlange\pwindex{?? Werk@Nicht ermittelte Verfasserinnen und Verfasser!Schlange@\emph{Schlange}|pw}}{\lemma{\textnormal{\emph{Schlange}}}\Cendnote{\textnormal{nicht identifiziert; Schnitzlers\pwindex{Schnitzler, Arthur 15.05.1862 – 21.10.1931@\textsc{Schnitzler, Arthur} (15.05.1862 – 21.10.1931), \emph{Schriftsteller, Mediziner}|pwk} Lektüreliste erwähnt \emph{Die goldene Schlange}\textcolor{red}{\textsuperscript{XXXX indx}} von Hermann Heiberg\pwindex{\textcolor{red}{\textsuperscript{XXXX1 indx}}|pwk} aus dem Jahr 1884, siehe A. S.: \emph{Lektüren}, deutschsprachige Literatur. Alternativ und da im Folgenden vor allem mögliche Titel für das \emph{Jung-Wiener Theater zum Lieben Augustin}\orgindex{Jung-Wiener Theater zum Lieben Augustin@Jung-Wiener Theater zum Lieben Augustin|pwk} diskutiert
                  wurden, könnte es sich um ein Gedicht oder ein Lied gehandelt haben.}}}\label{K_L02971-2h}.\pend
           \pstart
           Könnte man nicht die Namen der \label{K_L02971-3v}\edtext{2 Einakter}{\lemma{\textnormal{\emph{2 Einakter}}}\Cendnote{\textnormal{Auch Mitte Oktober 1901 stand das Programm des
                  Eröffnungsabends des von Salten\pwindex{Salten, Felix 06.09.1869 – 08.10.1945@\textsc{Salten, Felix} (06.09.1869 – 08.10.1945), \emph{Schriftsteller, Journalist}|pwk} gegründeten
                  Kabaretts \emph{Jung-Wiener Theater zum Lieben
                     Augustin}\orgindex{Jung-Wiener Theater zum Lieben Augustin@Jung-Wiener Theater zum Lieben Augustin|pwk} nicht fest. Weder von Goncourt\pwindex{Goncourt, Edmond Huot de 26.05.1822 – 16.07.1896@\textsc{Goncourt, Edmond Huot de} (26.05.1822 – 16.07.1896), \emph{Schriftsteller}|pwkv} noch von Mendès\pwindex{Mendes, Catulle 20.05.1841 – 08.02.1909@\textsc{Mendès, Catulle} (20.05.1841 – 08.02.1909), \emph{Schriftsteller}|pwk} kam ein Stück zur Aufführung. Am 27. 10. 1901 meldete das \emph{Illustrirte
                     Wiener Extrablatt}\pwindex{Illustrirtes Wiener Extrablatt1872 – 1928@\emph{Illustrirtes Wiener Extrablatt} {[}1872 – 1928{]}|pwk}, das Theater\orgindex{Jung-Wiener Theater zum Lieben Augustin@Jung-Wiener Theater zum Lieben Augustin|pwkv} habe die zwei Einakter \emph{Am
                     Fenster}\pwindex{Veber, Pierre 1869-05-15 – 1942-08-20@\textsc{Veber, Pierre} (1869-05-15 – 1942-08-20), \emph{Schriftsteller}!Am Fenster1901@\strich\emph{Am Fenster} {[}1901{]}|pwk} und \emph{Das Pfeifchen}\pwindex{Veber, Pierre 1869-05-15 – 1942-08-20@\textsc{Veber, Pierre} (1869-05-15 – 1942-08-20), \emph{Schriftsteller}!Pfeifchen1901@\strich\emph{Das Pfeifchen} {[}1901{]}|pwk} von Pierre Veber\pwindex{Veber, Pierre 1869-05-15 – 1942-08-20@\textsc{Veber, Pierre} (1869-05-15 – 1942-08-20), \emph{Schriftsteller}|pwk} erworben (vgl. Jg. 30,
                     Nr. 295, S. 5). Mit dem in der Fußnote genannten Übersetzer wäre dann
                     Otto Eisenschütz\pwindex{Eisenschitz, Otto 22.02.1863 – 11.09.1942@\textsc{Eisenschitz, Otto} (22.02.1863 – 11.09.1942), \emph{Schriftsteller, Journalist, Dramaturg}|pwk} gemeint.}}}\label{K_L02971-3h}
               erfahren, um ſie früher franzöſiſch zu leſen, insbeſondre \textsc{Goncourt\pwindex{Goncourt, Edmond Huot de 26.05.1822 – 16.07.1896@\textsc{Goncourt, Edmond Huot de} (26.05.1822 – 16.07.1896), \emph{Schriftsteller}|pwv}}, womöglich auch \textsc{Mendès\pwindex{Mendes, Catulle 20.05.1841 – 08.02.1909@\textsc{Mendès, Catulle} (20.05.1841 – 08.02.1909), \emph{Schriftsteller}|pw}}\footnote{\noindent{}Bedenken Sie die Unverläßlichkeit ja Lügenhaftigkeit des vorausſichtlichen Überſetzers\pwindex{Eisenschitz, Otto 22.02.1863 – 11.09.1942@\textsc{Eisenschitz, Otto} (22.02.1863 – 11.09.1942), \emph{Schriftsteller, Journalist, Dramaturg}|pwuv}!}\pend
           \pstart
           – Ferner: an welches Hebbel\pwindex{Hebbel, Friedrich 18.03.1813 – 13.12.1863@\textsc{Hebbel, Friedrich} (18.03.1813 – 13.12.1863), \emph{Schriftsteller}|pw}{ }Gedicht denken Sie? –\pend
           \pstart
           {\pb}Haben Sie, endlich und vorletztens eine
               Abſchrift des \label{K_L02971-4v}\edtext{Eſtherl\pwindex{Salus, Hugo 03.08.1866 – 04.02.1929@\textsc{Salus, Hugo} (03.08.1866 – 04.02.1929), \emph{Schriftsteller, Mediziner}!Altes Ghettoliedchen1901@\strich\emph{Altes Ghettoliedchen} {[}1901{]}|pw}}{\lemma{\textnormal{\emph{Eſtherl}}}\Cendnote{\textnormal{Das \emph{Alte Ghettoliedchen}\pwindex{Salus, Hugo 03.08.1866 – 04.02.1929@\textsc{Salus, Hugo} (03.08.1866 – 04.02.1929), \emph{Schriftsteller, Mediziner}!Altes Ghettoliedchen1901@\strich\emph{Altes Ghettoliedchen} {[}1901{]}|pwk} von Hugo Salus\pwindex{Salus, Hugo 03.08.1866 – 04.02.1929@\textsc{Salus, Hugo} (03.08.1866 – 04.02.1929), \emph{Schriftsteller, Mediziner}|pwk}
                  beginnt mit »Estherl, mein Schwesterl«.}}}\label{K_L02971-4h} zur Verfügung? –\pend
           \pstart
           – Letztens hab ich den Titel des Keller\pwindex{Keller, Gottfried 19.07.1819 – 16.07.1890@\textsc{Keller, Gottfried} (19.07.1819 – 16.07.1890), \emph{Schriftsteller}|pw}ſchen
               Gedichtes ſchon wieder vergeſſen. »Die
               Magd\pwindex{Keller, Gottfried 19.07.1819 – 16.07.1890@\textsc{Keller, Gottfried} (19.07.1819 – 16.07.1890), \emph{Schriftsteller}!Klage der Magd1847@\strich\emph{Klage der Magd} {[}1847{]}|pw}«?\pend
           \pstart
           Gute \label{K_L02971-5v}\edtext{Reiſe!}{\lemma{\textnormal{\emph{Reiſe!}}}\Cendnote{\textnormal{nach Berlin\oindex{Berlin@\textbf{Berlin}|pwk}, vgl. Felix Salten an Arthur Schnitzler, 9. 10. 1901.}}}\label{K_L02971-5h}{ }{\\[\baselineskip]}Herzlichſt Ihr {\\[\baselineskip]}\spacefill\mbox{Arthur}\pend
           \leftskip=0em{}
         
         \endnumbering\mylabel{h}\end{ledgroupsized}  \newcommand{\dateiname}{L02971}\newcommand{\titel}{Arthur Schnitzler an Felix Salten, 6. 10. 1901}\newcommand{\editorInnen}{Martin Anton Müller und Laura Untner}%% latex-leseansicht-abspann.tex
%% Abspann für die Leseansicht.
%% Der Schalter \ifkorrekturansicht ist bereits durch den Vorspann gesetzt.

%% latex-abspann.tex
%% Gemeinsamer Abspann für Korrekturansicht und Leseansicht.
%% Setzt den Schalter \ifkorrekturansicht voraus (gesetzt in den
%% einbindenden Dateien latex-korrekturansicht-abspann.tex bzw.
%% latex-leseansicht-abspann.tex).
%% ---------------------------------------------------------------

\normalsize

% Das esempio-Environment wird nur in der Leseansicht benötigt
\ifkorrekturansicht\else
\newenvironment{esempio}[3]%
{
    \vspace{1.5ex}
    \rlap{\underline{#1}}
    \par
    \setlength{\parindent}{0cm}
    \nopagebreak
    \leftskip=#2cm
    \rightskip=#3cm
}
{
    \par
}
\fi

\doendnotes{C}
\bigskip
\vfill

\clearpage

\footnotesize

\ifkorrekturansicht
  \lohead{\textsc{register}}
\fi

% theindex-Environment neu definieren ohne reledmac
\makeatletter
\renewenvironment{theindex}{%
  \ifkorrekturansicht
    \section*{\indexname}%
  \else
    \subsubsection*{Index der erwähnten Entitäten}%
  \fi
  \setlength{\parindent}{0pt}%
  \setlength{\parskip}{0pt plus 0.3pt}%
  \let\item\@idxitem
}{%
  \ifkorrekturansicht\clearpage\fi
}
\makeatother

\IfFileExists{\jobname-pw.ind}{\input{\jobname-pw.ind}}{}

% Quellenangabe nur in der Leseansicht
\ifkorrekturansicht\else
% Fallback-Definitionen, falls die .tex-Datei \titel etc. nicht gesetzt hat
\providecommand{\titel}{}
\providecommand{\editorInnen}{}
\providecommand{\dateiname}{\jobname}

\vspace{3cm}

\vfill

\footnotesize
\textsc{Quelle}: \titel. Herausgegeben von {\editorInnen}. In: \emph{Arthur Schnitzler: Briefwechsel mit Autorinnen und Autoren}.
 Digitale Edition, https://schnitzler-briefe.acdh.oeaw.ac.at/{\dateiname}.html (Stand \today)
\fi

\end{document}


      