%% latex-korrekturansicht-vorspann.tex
%% Vorspann für die Korrekturansicht.
%% Lädt die gemeinsame Datei latex-vorspann.tex mit gesetztem Schalter.

\newif\ifkorrekturansicht
\korrekturansichttrue

\input{../tex-inputs/latex-vorspann}


\section[ Arthur Schnitzler an Felix Salten, 6. 10. 1901]{L02971 Arthur Schnitzler an Felix Salten, 6. 10. 1901}
\nopagebreak\mylabel{L02971v}
\rehead{ }\normalsize\beginnumbering\briefempfaengerindex{Salten, Felix@\textsc{Salten, Felix}!zzzSchnitzler, Arthur@\emph{von Arthur Schnitzler}!1901-10-061@{6. 10. 1901}|(be}
\toendnotes[C]{\smallbreak\pagebreak[2]}\Standort{Wienbibliothek im Rathaus, ZPH 1681, 2.1.516.}
\physDesc{Brief, 1 Blatt, 2 Seiten, 411 Zeichen
\newline{}Handschrift: Bleistift, deutsche Kurrent
\newline{}Ordnung: mit Bleistift von unbekannter Hand nummeriert: »22« }\toendnotes[C]{\smallbreak}
\pstart
           \raggedleft{}{\pb}6/10 901\pend
           \vspace{0.5em}
\pstart
           lieber, hier iſt \label{K_L02971-1v}\edtext{Inſel\pwindex{Insel. Monatsschrift mit Buchschmuck und Illustrationen@\emph{Die Insel. Monatsschrift mit Buchschmuck und Illustrationen}|pw}}{\lemma{\textnormal{\emph{Inſel}}}\Cendnote{\textnormal{Vgl. Felix Salten an Arthur Schnitzler, 28. 7. 1901. }}}\label{K_L02971-1} und \label{K_L02971-2v}\edtext{Schlange\pwindex{Schlange@\emph{Schlange}|pw}}{\lemma{\textnormal{\emph{Schlange}}}\Cendnote{\textnormal{nicht identifiziert; Schnitzlers Lektüreliste erwähnt \emph{Die goldene Schlange}\pwindex{goldene Schlange. Roman@\emph{Die goldene Schlange. Roman}|pwk} von Hermann Heiberg\pwindex{Heiberg, Hermann 1840-11-17 – 1910-02-16@\textsc{Heiberg, Hermann} (1840-11-17 – 1910-02-16), \emph{Schriftsteller/Schriftstellerin, Redakteur/Redakteurin, Buchhändler/Buchhändlerin}|pwk} aus dem Jahr 1884, siehe A. S.: \emph{Lektüren}, deutschsprachige Literatur. Alternativ und da im Folgenden vor allem mögliche Titel für das \emph{Jung-Wiener Theater zum Lieben Augustin}\orgindex{Jung-Wiener Theater zum Lieben Augustin@Jung-Wiener Theater zum Lieben Augustin|pwk} diskutiert
                  wurden, könnte es sich um ein Gedicht oder ein Lied gehandelt haben.}}}\label{K_L02971-2}.\pend
           
\pstart
           Könnte man nicht die Namen der \label{K_L02971-3v}\edtext{2 Einakter}{\lemma{\textnormal{\emph{2 Einakter}}}\Cendnote{\textnormal{Auch Mitte Oktober 1901 stand das Programm des
                  Eröffnungsabends des von Salten\pwindex{Salten, Felix 06.09.1869 – 08.10.1945@\textsc{Salten, Felix} (06.09.1869 – 08.10.1945), \emph{Schriftsteller/Schriftstellerin, Journalist/Journalistin, Chefredakteur/Chefredakteurin}|pwk} gegründeten
                  Kabaretts \emph{Jung-Wiener Theater zum Lieben
                     Augustin}\orgindex{Jung-Wiener Theater zum Lieben Augustin@Jung-Wiener Theater zum Lieben Augustin|pwk} nicht fest. Weder von Goncourt\pwindex{Goncourt, Edmond Huot de 26.05.1822 – 16.07.1896@\textsc{Goncourt, Edmond Huot de} (26.05.1822 – 16.07.1896), \emph{Schriftsteller/Schriftstellerin}|pwkv} noch von Mendès\pwindex{Mendes, Catulle 20.05.1841 – 08.02.1909@\textsc{Mendès, Catulle} (20.05.1841 – 08.02.1909), \emph{Schriftsteller/Schriftstellerin}|pwk} kam ein Stück zur Aufführung. Am 27. 10. 1901 meldete das \emph{Illustrirte
                     Wiener Extrablatt}\pwindex{Illustrirtes Wiener Extrablatt@\emph{Illustrirtes Wiener Extrablatt}|pwk}, das Theater\orgindex{Jung-Wiener Theater zum Lieben Augustin@Jung-Wiener Theater zum Lieben Augustin|pwkv} habe die zwei Einakter \emph{Am
                     Fenster}\pwindex{Am Fenster@\emph{Am Fenster}|pwk} und \emph{Das Pfeifchen}\pwindex{Pfeifchen@\emph{Das Pfeifchen}|pwk} von Pierre Veber\pwindex{Veber, Pierre 1869-05-15 – 1942-08-20@\textsc{Veber, Pierre} (1869-05-15 – 1942-08-20), \emph{Schriftsteller/Schriftstellerin}|pwk} erworben (vgl. Jg. 30,
                     Nr. 295, S. 5). Mit dem in der Fußnote genannten Übersetzer wäre dann
                     Otto Eisenschütz\pwindex{Eisenschitz, Otto 22.02.1863 – 11.09.1942@\textsc{Eisenschitz, Otto} (22.02.1863 – 11.09.1942), \emph{Schriftsteller/Schriftstellerin, Journalist/Journalistin, Dramaturg/Dramaturgin}|pwk} gemeint.}}}\label{K_L02971-3}
               erfahren, um ſie früher franzöſiſch zu leſen, insbeſondre \textsc{Goncourt\pwindex{Goncourt, Edmond Huot de 26.05.1822 – 16.07.1896@\textsc{Goncourt, Edmond Huot de} (26.05.1822 – 16.07.1896), \emph{Schriftsteller/Schriftstellerin}|pwv}}, womöglich auch \textsc{Mendès\pwindex{Mendes, Catulle 20.05.1841 – 08.02.1909@\textsc{Mendès, Catulle} (20.05.1841 – 08.02.1909), \emph{Schriftsteller/Schriftstellerin}|pw}}\noindent{}Bedenken Sie die Unverläßlichkeit ja Lügenhaftigkeit des vorausſichtlichen Überſetzers\pwindex{Eisenschitz, Otto 22.02.1863 – 11.09.1942@\textsc{Eisenschitz, Otto} (22.02.1863 – 11.09.1942), \emph{Schriftsteller/Schriftstellerin, Journalist/Journalistin, Dramaturg/Dramaturgin}|pwuv}!\pend
           
\pstart
           – Ferner: an welches Hebbel\pwindex{Hebbel, Friedrich 18.03.1813 – 13.12.1863@\textsc{Hebbel, Friedrich} (18.03.1813 – 13.12.1863), \emph{Schriftsteller/Schriftstellerin}|pw}{ }Gedicht denken Sie? –\pend
           
\pstart
           {\pb}Haben Sie, endlich und vorletztens eine
               Abſchrift des \label{K_L02971-4v}\edtext{Eſtherl\pwindex{Altes Ghettoliedchen@\emph{Altes Ghettoliedchen}|pw}}{\lemma{\textnormal{\emph{Eſtherl}}}\Cendnote{\textnormal{Das \emph{Alte Ghettoliedchen}\pwindex{Altes Ghettoliedchen@\emph{Altes Ghettoliedchen}|pwk} von Hugo Salus\pwindex{Salus, Hugo 03.08.1866 – 04.02.1929@\textsc{Salus, Hugo} (03.08.1866 – 04.02.1929), \emph{Schriftsteller/Schriftstellerin, Mediziner/Medizinerin}|pwk}
                  beginnt mit »Estherl, mein Schwesterl«.}}}\label{K_L02971-4} zur Verfügung? –\pend
           
\pstart
           – Letztens hab ich den Titel des Keller\pwindex{Keller, Gottfried 19.07.1819 – 16.07.1890@\textsc{Keller, Gottfried} (19.07.1819 – 16.07.1890), \emph{Schriftsteller/Schriftstellerin}|pw}ſchen
               Gedichtes ſchon wieder vergeſſen. »Die
               Magd\pwindex{Klage der Magd@\emph{Klage der Magd}|pw}«?\pend
           
\pstart
           Gute \label{K_L02971-5v}\edtext{Reiſe!}{\lemma{\textnormal{\emph{Reiſe!}}}\Cendnote{\textnormal{nach Berlin\oindex{Berlin@\textbf{Berlin}, \emph{P.PPLC}|pwk}, vgl. Felix Salten an Arthur Schnitzler, 9. 10. 1901.}}}\label{K_L02971-5}{ }{\\[\baselineskip]}Herzlichſt Ihr {\\[\baselineskip]}\spacefill\mbox{Arthur}\pend
           \leftskip=0em{}\selectlanguage{ngerman}\endnumbering\briefempfaengerindex{Salten, Felix@\textsc{Salten, Felix}!zzzSchnitzler, Arthur@\emph{von Arthur Schnitzler}!1901-10-061@{6. 10. 1901}|)be}\mylabel{L02971h}  \normalsize

\doendnotes{C}
\bigskip
\vfill

\clearpage

\footnotesize

\lohead{\textsc{register}}

% Definiere theindex-Environment komplett neu ohne reledmac
\makeatletter
\renewenvironment{theindex}{%
  \section*{\indexname}%
  \setlength{\parindent}{0pt}%
  \setlength{\parskip}{0pt plus 0.3pt}%
  \let\item\@idxitem
}{%
  \clearpage
}
\makeatother

\IfFileExists{\jobname-pw.ind}{\input{\jobname-pw.ind}}{}

\end{document}

      