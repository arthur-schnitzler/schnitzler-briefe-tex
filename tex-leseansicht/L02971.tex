%% latex-leseansicht-vorspann.tex
%% Vorspann für die Leseansicht.
%% Lädt die gemeinsame Datei latex-vorspann.tex mit nicht gesetztem Schalter.

\newif\ifkorrekturansicht
\korrekturansichtfalse

\input{../tex-inputs/latex-vorspann}


\section[ Arthur Schnitzler an Felix Salten, 6. 10. 1901]{L02971 Arthur Schnitzler an Felix Salten,  6. 10. 1901}
\nopagebreak\mylabel{L02971v}
\rehead{ }\normalsize\beginnumbering\briefempfaengerindex{Salten, Felix@\textsc{Salten, Felix}!zzzSchnitzler, Arthur@\emph{von Arthur Schnitzler}!1901-10-061@{6. 10. 1901}|(be}
\toendnotes[C]{\smallbreak\pagebreak[2]}
\correspDesc{Versand  durch Arthur Schnitzler am 6. 10. 1901 in Wien
\newline{}Erhalt  durch Felix Salten im Zeitraum [6. 10. 1901
                  – 8. 10. 1901?] in Wien}\toendnotes[C]{\smallbreak}
\Standort{Wienbibliothek im Rathaus, ZPH 1681, 2.1.516.}
\physDesc{Brief, 1 Blatt, 2 Seiten, 411 Zeichen
\newline{}Handschrift: Bleistift, deutsche Kurrent
\newline{}Ordnung: mit Bleistift von unbekannter Hand nummeriert: »22« }\toendnotes[C]{\smallbreak}
\pstart
           \raggedleft{}{\pb}6/10 901\pend
           \vspace{0.5em}
\pstart
           lieber, hier iſt \label{K_L02971-1v}\edtext{Inſel\pwindex{Insel. Monatsschrift mit Buchschmuck und Illustrationen@\emph{Die Insel. Monatsschrift mit Buchschmuck und Illustrationen}|pw}}{\lemma{\textnormal{\emph{Insel}}}\Cendnote{\textnormal{Vgl. XXXX Auszeichnungsfehler: Dokument L03316 nicht gefunden. }}}\label{K_L02971-1} und \label{K_L02971-2v}\edtext{Schlange\pwindex{Schlange@\emph{Schlange}|pw}}{\lemma{\textnormal{\emph{Schlange}}}\Cendnote{\textnormal{nicht identifiziert; Schnitzlers Lektüreliste erwähnt \emph{Die goldene Schlange}\pwindex{Heiberg, Hermann 17.\,11.\,1840 Südschleswig – 16.\,2.\,1910 ebd.@\textsc{Heiberg, Hermann} (17.\,11.\,1840 Südschleswig – 16.\,2.\,1910 ebd.), \emph{Schriftsteller, Redakteur, Buchhändler}!goldene Schlange. Roman@\strich\emph{Die goldene Schlange. Roman}|pwk} von Hermann Heiberg\pwindex{Heiberg, Hermann 17.\,11.\,1840 Südschleswig – 16.\,2.\,1910 ebd.@\textsc{Heiberg, Hermann} (17.\,11.\,1840 Südschleswig – 16.\,2.\,1910 ebd.), \emph{Schriftsteller, Redakteur, Buchhändler}|pwk} aus dem Jahr 1884, siehe A. S.: \emph{Lektüren}, deutschsprachige Literatur. Alternativ und da im Folgenden vor allem mögliche Titel für das \emph{Jung-Wiener Theater zum Lieben Augustin}\orgindex{Jung-Wiener Theater zum Lieben Augustin@Jung-Wiener Theater zum Lieben Augustin|pwk} diskutiert
                  wurden, könnte es sich um ein Gedicht oder ein Lied gehandelt haben.}}}\label{K_L02971-2}.\pend
           
\pstart
           Könnte man nicht die Namen der \label{K_L02971-3v}\edtext{2 Einakter}{\lemma{\textnormal{\emph{2 Einakter}}}\Cendnote{\textnormal{Auch Mitte Oktober 1901 stand das Programm des
                  Eröffnungsabends des von Salten\pwindex{Salten, Felix 6.\,9.\,1869 Budapest – 8.\,10.\,1945 Zürich@\textsc{Salten, Felix} (6.\,9.\,1869 Budapest – 8.\,10.\,1945 Zürich), \emph{Schriftsteller, Journalist, Chefredakteur}|pwk} gegründeten
                  Kabaretts \emph{Jung-Wiener Theater zum Lieben
                     Augustin}\orgindex{Jung-Wiener Theater zum Lieben Augustin@Jung-Wiener Theater zum Lieben Augustin|pwk} nicht fest. Weder von Goncourt\pwindex{Goncourt, Edmond Huot de 26.\,5.\,1822 Nancy – 16.\,7.\,1896 Draveil@\textsc{Goncourt, Edmond Huot de} (26.\,5.\,1822 Nancy – 16.\,7.\,1896 Draveil), \emph{Schriftsteller}|pwkv} noch von Mendès\pwindex{Mendès, Catulle 20.\,5.\,1841 Bordeaux – 8.\,2.\,1909 Saint-Germain-en-Laye@\textsc{Mendès, Catulle} (20.\,5.\,1841 Bordeaux – 8.\,2.\,1909 Saint-Germain-en-Laye), \emph{Schriftsteller}|pwk} kam ein Stück zur Aufführung. Am 27. 10. 1901 meldete das \emph{Illustrirte
                     Wiener Extrablatt}\pwindex{Illustrirtes Wiener Extrablatt@\emph{Illustrirtes Wiener Extrablatt}|pwk}, das Theater\orgindex{Jung-Wiener Theater zum Lieben Augustin@Jung-Wiener Theater zum Lieben Augustin|pwkv} habe die zwei Einakter \emph{Am
                     Fenster}\pwindex{Veber, Pierre 15.\,5.\,1869 Paris – 20.\,8.\,1942 ebd.@\textsc{Veber, Pierre} (15.\,5.\,1869 Paris – 20.\,8.\,1942 ebd.), \emph{Schriftsteller}!Am Fenster@\strich\emph{Am Fenster}|pwk} und \emph{Das Pfeifchen}\pwindex{Veber, Pierre 15.\,5.\,1869 Paris – 20.\,8.\,1942 ebd.@\textsc{Veber, Pierre} (15.\,5.\,1869 Paris – 20.\,8.\,1942 ebd.), \emph{Schriftsteller}!Pfeifchen@\strich\emph{Das Pfeifchen}|pwk} von Pierre Veber\pwindex{Veber, Pierre 15.\,5.\,1869 Paris – 20.\,8.\,1942 ebd.@\textsc{Veber, Pierre} (15.\,5.\,1869 Paris – 20.\,8.\,1942 ebd.), \emph{Schriftsteller}|pwk} erworben (vgl. Jg. 30,
                     Nr. 295, S. 5). Mit dem in der Fußnote genannten Übersetzer wäre dann
                     Otto Eisenschütz\pwindex{Eisenschitz, Otto 22.\,2.\,1863 Wien – 11.\,9.\,1942 Konzentrationslager Theresienstadt@\textsc{Eisenschitz, Otto} (22.\,2.\,1863 Wien – 11.\,9.\,1942 Konzentrationslager Theresienstadt), \emph{Schriftsteller, Journalist, Dramaturg}|pwk} gemeint.}}}\label{K_L02971-3}
               erfahren, um{ }ſie früher franzöſiſch zu leſen, insbeſondre \textsc{Goncourt\pwindex{Goncourt, Edmond Huot de 26.\,5.\,1822 Nancy – 16.\,7.\,1896 Draveil@\textsc{Goncourt, Edmond Huot de} (26.\,5.\,1822 Nancy – 16.\,7.\,1896 Draveil), \emph{Schriftsteller}|pwv}}, womöglich auch \textsc{Mendès\pwindex{Mendès, Catulle 20.\,5.\,1841 Bordeaux – 8.\,2.\,1909 Saint-Germain-en-Laye@\textsc{Mendès, Catulle} (20.\,5.\,1841 Bordeaux – 8.\,2.\,1909 Saint-Germain-en-Laye), \emph{Schriftsteller}|pw}}\footnote{\noindent{}Bedenken Sie die Unverläßlichkeit ja Lügenhaftigkeit des vorausſichtlichen Überſetzers\pwindex{Eisenschitz, Otto 22.\,2.\,1863 Wien – 11.\,9.\,1942 Konzentrationslager Theresienstadt@\textsc{Eisenschitz, Otto} (22.\,2.\,1863 Wien – 11.\,9.\,1942 Konzentrationslager Theresienstadt), \emph{Schriftsteller, Journalist, Dramaturg}|pwuv}!}\pend
           
\pstart
           – Ferner: an welches Hebbel\pwindex{Hebbel, Friedrich 18.\,3.\,1813 Wesselburen – 13.\,12.\,1863 Wien@\textsc{Hebbel, Friedrich} (18.\,3.\,1813 Wesselburen – 13.\,12.\,1863 Wien), \emph{Schriftsteller}|pw}{ }Gedicht denken Sie? –\pend
           
\pstart
           {\pb}Haben Sie, endlich und vorletztens eine
               Abſchrift des \label{K_L02971-4v}\edtext{Eſtherl\pwindex{Salus, Hugo 3.\,8.\,1866 Česká Lípa – 4.\,2.\,1929 Prag@\textsc{Salus, Hugo} (3.\,8.\,1866 Česká Lípa – 4.\,2.\,1929 Prag), \emph{Schriftsteller, Mediziner}!Altes Ghettoliedchen@\strich\emph{Altes Ghettoliedchen}|pw}}{\lemma{\textnormal{\emph{Estherl}}}\Cendnote{\textnormal{Das \emph{Alte Ghettoliedchen}\pwindex{Salus, Hugo 3.\,8.\,1866 Česká Lípa – 4.\,2.\,1929 Prag@\textsc{Salus, Hugo} (3.\,8.\,1866 Česká Lípa – 4.\,2.\,1929 Prag), \emph{Schriftsteller, Mediziner}!Altes Ghettoliedchen@\strich\emph{Altes Ghettoliedchen}|pwk} von Hugo Salus\pwindex{Salus, Hugo 3.\,8.\,1866 Česká Lípa – 4.\,2.\,1929 Prag@\textsc{Salus, Hugo} (3.\,8.\,1866 Česká Lípa – 4.\,2.\,1929 Prag), \emph{Schriftsteller, Mediziner}|pwk}
                  beginnt mit »Estherl, mein Schwesterl«.}}}\label{K_L02971-4} zur Verfügung? –\pend
           
\pstart
           – Letztens hab ich den Titel des Keller\pwindex{Keller, Gottfried 19.\,7.\,1819 Zürich – 16.\,7.\,1890 ebd.@\textsc{Keller, Gottfried} (19.\,7.\,1819 Zürich – 16.\,7.\,1890 ebd.), \emph{Schriftsteller}|pw}ſchen
               Gedichtes{ }ſchon wieder vergeſſen. »Die
               Magd\pwindex{Keller, Gottfried 19.\,7.\,1819 Zürich – 16.\,7.\,1890 ebd.@\textsc{Keller, Gottfried} (19.\,7.\,1819 Zürich – 16.\,7.\,1890 ebd.), \emph{Schriftsteller}!Klage der Magd@\strich\emph{Klage der Magd}|pw}«?\pend
           
\pstart
           Gute \label{K_L02971-5v}\edtext{Reiſe!}{\lemma{\textnormal{\emph{Reise!}}}\Cendnote{\textnormal{nach Berlin\oindex{Berlin@\textbf{Berlin}, \emph{Hauptstadt}|pwk}, vgl. XXXX Auszeichnungsfehler: Dokument L03320 nicht gefunden.}}}\label{K_L02971-5}{ }{\\[\baselineskip]}Herzlichſt Ihr {\\[\baselineskip]}\spacefill\mbox{Arthur}\pend
           \leftskip=0em{}\selectlanguage{ngerman}\endnumbering\briefempfaengerindex{Salten, Felix@\textsc{Salten, Felix}!zzzSchnitzler, Arthur@\emph{von Arthur Schnitzler}!1901-10-061@{6. 10. 1901}|)be}\mylabel{L02971h}  \newcommand{\dateiname}{L02971}\newcommand{\titel}{Arthur Schnitzler an Felix Salten, 6. 10. 1901}\newcommand{\editorInnen}{Martin Anton Müller und Laura Untner}%% latex-leseansicht-abspann.tex
%% Abspann für die Leseansicht.
%% Der Schalter \ifkorrekturansicht ist bereits durch den Vorspann gesetzt.

%% latex-abspann.tex
%% Gemeinsamer Abspann für Korrekturansicht und Leseansicht.
%% Setzt den Schalter \ifkorrekturansicht voraus (gesetzt in den
%% einbindenden Dateien latex-korrekturansicht-abspann.tex bzw.
%% latex-leseansicht-abspann.tex).
%% ---------------------------------------------------------------

\normalsize

% Das esempio-Environment wird nur in der Leseansicht benötigt
\ifkorrekturansicht\else
\newenvironment{esempio}[3]%
{
    \vspace{1.5ex}
    \rlap{\underline{#1}}
    \par
    \setlength{\parindent}{0cm}
    \nopagebreak
    \leftskip=#2cm
    \rightskip=#3cm
}
{
    \par
}
\fi

\doendnotes{C}
\bigskip
\vfill

\clearpage

\footnotesize

\ifkorrekturansicht
  \lohead{\textsc{register}}
\fi

% theindex-Environment neu definieren ohne reledmac
\makeatletter
\renewenvironment{theindex}{%
  \ifkorrekturansicht
    \section*{\indexname}%
  \else
    \subsubsection*{Index der erwähnten Entitäten}%
  \fi
  \setlength{\parindent}{0pt}%
  \setlength{\parskip}{0pt plus 0.3pt}%
  \let\item\@idxitem
}{%
  \ifkorrekturansicht\clearpage\fi
}
\makeatother

\IfFileExists{\jobname-pw.ind}{\input{\jobname-pw.ind}}{}

% Quellenangabe nur in der Leseansicht
\ifkorrekturansicht\else
% Fallback-Definitionen, falls die .tex-Datei \titel etc. nicht gesetzt hat
\providecommand{\titel}{}
\providecommand{\editorInnen}{}
\providecommand{\dateiname}{\jobname}

\vspace{3cm}

\vfill

\footnotesize
\textsc{Quelle}: \titel. Herausgegeben von {\editorInnen}. In: \emph{Arthur Schnitzler: Briefwechsel mit Autorinnen und Autoren}.
 Digitale Edition, https://schnitzler-briefe.acdh.oeaw.ac.at/{\dateiname}.html (Stand \today)
\fi

\end{document}


