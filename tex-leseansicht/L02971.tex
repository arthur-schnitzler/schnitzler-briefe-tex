%% latex-leseansicht-vorspann.tex
%% Vorspann für die Leseansicht.
%% Lädt die gemeinsame Datei latex-vorspann.tex mit nicht gesetztem Schalter.

\newif\ifkorrekturansicht
\korrekturansichtfalse

\input{../tex-inputs/latex-vorspann}

\begin{center}
            \textcolor{red}{ENTWURF, NICHT FERTIG KORRIGIERT}
                      \end{center}
            
         
         \renewcommand{\erwaehntePersonen}{Personen: Friedrich Hebbel, Catulle Mendès, Felix Salten}
         \renewcommand{\erwaehnteOrte}{Orte: Wien}
         \renewcommand{\erwaehnteWerke}{
               \section[Arthur Schnitzler an Felix Salten, 6. 10. 1901]{ Arthur Schnitzler an Felix Salten, 6. 10. 1901}\nopagebreak\mylabel{v}\rehead{ }\begin{ledgroupsized}[t]{13cm}\normalsize\beginnumbering \toendnotes[C]{\smallbreak\pagebreak[2]} \Standort{Wienbibliothek im Rathaus, ZPH 1681, 2.1.516.}
\physDesc{
\newline{}Handschrift: , deutsche Kurrent}\toendnotes[C]{\smallbreak}\pstart
           \raggedleft{}{\pb}6/10 901\pend
           \pstart
           lieber, hier iſt Inſel\textcolor{red}{\textsuperscript{\textbf{KEY}}} und Schlange\textcolor{red}{\textsuperscript{\textbf{KEY}}}. \pend
           \pstart
           Könnte man nicht die Namen der2
                  Einakter\textcolor{red}{\textsuperscript{\textbf{KEY}}} erfahren, um ſie früher franzöſiſch zu leſen, insbeſondre \textsc{Goncourt\textcolor{red}{\textsuperscript{\textbf{KEY}}}\textcolor{red}{\textsuperscript{\textbf{KEY}}}}, womöglich auch \textsc{Mendès\textcolor{red}{\textsuperscript{\textbf{KEY}}}\pwindex{Mendes, Catulle 20.05.1841 – 08.02.1909@\textsc{Mendès, Catulle} (20.05.1841 – 08.02.1909), \emph{Schriftsteller}|pw}}\footnote{\noindent{}Bedenken Sie die Unverläßlichkeit ja Lügenhaftigkeit des vorausſichtlichen Überſetzers\textcolor{red}{\textsuperscript{\textbf{KEY}}}!} – Ferner: an welches Hebbel\pwindex{Hebbel, Friedrich 18.03.1813 – 13.12.1863@\textsc{Hebbel, Friedrich} (18.03.1813 – 13.12.1863), \emph{Schriftsteller}|pw}{ }Gedicht\textcolor{red}{\textsuperscript{\textbf{KEY}}} denken Sie?– {\pb}Haben Sie, endlich und vorletztens
               eine Abſchrift des Eſtherl\textcolor{red}{\textsuperscript{\textbf{KEY}}} zur Verfügung? – \pend
           \pstart
           – Letztens hab ich den Titel des Keller\textcolor{red}{\textsuperscript{\textbf{KEY}}}ſchen Gedichtes\textcolor{red}{\textsuperscript{\textbf{KEY}}} ſchon wieder vergeſſen. »Die Magd\textcolor{red}{\textsuperscript{\textbf{KEY}}}?« \pend
           \pstart
           Gute Reiſe! {\\[\baselineskip]}Herzlichſt Ihr {\\[\baselineskip]}\spacefill\mbox{Arthur}\pend
           \leftskip=0em{}
         
         \endnumbering\mylabel{h}\end{ledgroupsized}\begin{anhang}\end{anhang}\newcommand{\dateiname}{L02971}\newcommand{\titel}{Arthur Schnitzler an Felix Salten, 6. 10. 1901}\newcommand{\editorInnen}{Martin Anton Müller und Laura Untner}%% latex-leseansicht-abspann.tex
%% Abspann für die Leseansicht.
%% Der Schalter \ifkorrekturansicht ist bereits durch den Vorspann gesetzt.

%% latex-abspann.tex
%% Gemeinsamer Abspann für Korrekturansicht und Leseansicht.
%% Setzt den Schalter \ifkorrekturansicht voraus (gesetzt in den
%% einbindenden Dateien latex-korrekturansicht-abspann.tex bzw.
%% latex-leseansicht-abspann.tex).
%% ---------------------------------------------------------------

\normalsize

% Das esempio-Environment wird nur in der Leseansicht benötigt
\ifkorrekturansicht\else
\newenvironment{esempio}[3]%
{
    \vspace{1.5ex}
    \rlap{\underline{#1}}
    \par
    \setlength{\parindent}{0cm}
    \nopagebreak
    \leftskip=#2cm
    \rightskip=#3cm
}
{
    \par
}
\fi

\doendnotes{C}
\bigskip
\vfill

\clearpage

\footnotesize

\ifkorrekturansicht
  \lohead{\textsc{register}}
\fi

% theindex-Environment neu definieren ohne reledmac
\makeatletter
\renewenvironment{theindex}{%
  \ifkorrekturansicht
    \section*{\indexname}%
  \else
    \subsubsection*{Index der erwähnten Entitäten}%
  \fi
  \setlength{\parindent}{0pt}%
  \setlength{\parskip}{0pt plus 0.3pt}%
  \let\item\@idxitem
}{%
  \ifkorrekturansicht\clearpage\fi
}
\makeatother

\IfFileExists{\jobname-pw.ind}{\input{\jobname-pw.ind}}{}

% Quellenangabe nur in der Leseansicht
\ifkorrekturansicht\else
% Fallback-Definitionen, falls die .tex-Datei \titel etc. nicht gesetzt hat
\providecommand{\titel}{}
\providecommand{\editorInnen}{}
\providecommand{\dateiname}{\jobname}

\vspace{3cm}

\vfill

\footnotesize
\textsc{Quelle}: \titel. Herausgegeben von {\editorInnen}. In: \emph{Arthur Schnitzler: Briefwechsel mit Autorinnen und Autoren}.
 Digitale Edition, https://schnitzler-briefe.acdh.oeaw.ac.at/{\dateiname}.html (Stand \today)
\fi

\end{document}


      