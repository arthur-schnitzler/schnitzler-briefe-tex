%% latex-leseansicht-vorspann.tex
%% Vorspann für die Leseansicht.
%% Lädt die gemeinsame Datei latex-vorspann.tex mit nicht gesetztem Schalter.

\newif\ifkorrekturansicht
\korrekturansichtfalse

\input{../tex-inputs/latex-vorspann}


         
         \newcommand{\erwaehntePersonen}{Personen: Richard Beer-Hofmann}
         \newcommand{\erwaehnteOrte}{Orte: Berchtesgaden, Mayrhofen, Tirol, Watzmann, Wien, Zillertal}
         \newcommand{\erwaehnteWerke}{
               \section[Arthur Schnitzler an Richard Beer-Hofmann, {[}zwischen 10. und 28. 8.{]} 1922]{ Arthur Schnitzler an Richard Beer-Hofmann, {[}zwischen 10. und
               28. 8.{]} 1922}\nopagebreak\mylabel{v}\rehead{ }\begin{ledgroupsized}[t]{13cm}\normalsize\beginnumbering \toendnotes[C]{\smallbreak\pagebreak[2]} \Standort{YCGL, MSS 31.}
\physDesc{Bildpostkarte
\newline{}Handschrift: Bleistift, lateinische Kurrent\newline{}Versand: Stempel: »\nobreak{}\oindex{Berchtesgaden@\textbf{Berchtesgaden}|pwk}\textcolor{gray}{Berchtesgaden}, \textcolor{gray}{×}\-\textcolor{gray}{×}.{ }Aug 22, 8–9N\nobreak{}«.  \newline{}Ordnung: mit Bleistift von unbekannter Hand datiert »Aug. 1922« }\toendnotes[C]{\smallbreak}\pstart{}{\pb}Hrn Dr. Rich Beer-Hofmann\pend{}\pstart{}aus Wien\oindex{Wien@\textbf{Wien}|pw}\pend{}\pstart{}Meyerhofen\oindex{Mayrhofen@\textbf{Mayrhofen}|pw}\pend{}\pstart{} im Zillerthal\oindex{Zillertal@\textbf{Zillertal}|pw}\pend{}\pstart{}Tirol\oindex{Tirol@\textbf{Tirol}|pw}\pend{}{\bigskip}\pstart
           \noindent{}\centering{}{\pb}\textcolor{gray}{\textbf{Berchtesgaden\oindex{Berchtesgaden@\textbf{Berchtesgaden}|pw} mit dem Watzmann\oindex{Watzmann@\textbf{Watzmann}|pw}}}\pend
           \pstart
           {\pb}Herzliche \label{KLL02391_Beer-Hofmann-1v}\edtext{Grüße}{\lemma{\textnormal{\emph{Grüße}}}\Cendnote{\textnormal{Der Stempel erlaubt die Einordnung in den August 1922, durch Schnitzler\pwindex{Schnitzler, Arthur 15.05.1862 – 21.10.1931@\textsc{Schnitzler, Arthur} (15.05.1862 – 21.10.1931), \emph{Schriftsteller, Mediziner}|pwk}s Anreise am 10. 8. 1922 und Beer-Hofmann\pwindex{Beer-Hofmann, Richard 1866-07-11 – 1945-09-26@\textsc{Beer-Hofmann, Richard} (1866-07-11 – 1945-09-26), \emph{Schriftsteller}|pwk}s Abreise aus seiner Urlaubsdestination um den
                     28. 8. 1922 ist das Zeitfenster weiter eingeschränkt.}}}\label{KLL02391_Beer-Hofmann-1h}!\pend
           \pstart
           Ihr{\\[\baselineskip]}\spacefill\mbox{ArthurS}\pend
           \leftskip=0em{}
         
         \endnumbering\mylabel{h}\end{ledgroupsized}  \newcommand{\dateiname}{L02391}\newcommand{\titel}{Arthur Schnitzler an Richard Beer-Hofmann, [zwischen 10. und 28. 8.] 1922}\newcommand{\editorInnen}{Martin Anton Müller und Gerd-Hermann Susen}%% latex-leseansicht-abspann.tex
%% Abspann für die Leseansicht.
%% Der Schalter \ifkorrekturansicht ist bereits durch den Vorspann gesetzt.

%% latex-abspann.tex
%% Gemeinsamer Abspann für Korrekturansicht und Leseansicht.
%% Setzt den Schalter \ifkorrekturansicht voraus (gesetzt in den
%% einbindenden Dateien latex-korrekturansicht-abspann.tex bzw.
%% latex-leseansicht-abspann.tex).
%% ---------------------------------------------------------------

\normalsize

% Das esempio-Environment wird nur in der Leseansicht benötigt
\ifkorrekturansicht\else
\newenvironment{esempio}[3]%
{
    \vspace{1.5ex}
    \rlap{\underline{#1}}
    \par
    \setlength{\parindent}{0cm}
    \nopagebreak
    \leftskip=#2cm
    \rightskip=#3cm
}
{
    \par
}
\fi

\doendnotes{C}
\bigskip
\vfill

\clearpage

\footnotesize

\ifkorrekturansicht
  \lohead{\textsc{register}}
\fi

% theindex-Environment neu definieren ohne reledmac
\makeatletter
\renewenvironment{theindex}{%
  \ifkorrekturansicht
    \section*{\indexname}%
  \else
    \subsubsection*{Index der erwähnten Entitäten}%
  \fi
  \setlength{\parindent}{0pt}%
  \setlength{\parskip}{0pt plus 0.3pt}%
  \let\item\@idxitem
}{%
  \ifkorrekturansicht\clearpage\fi
}
\makeatother

\IfFileExists{\jobname-pw.ind}{\input{\jobname-pw.ind}}{}

% Quellenangabe nur in der Leseansicht
\ifkorrekturansicht\else
% Fallback-Definitionen, falls die .tex-Datei \titel etc. nicht gesetzt hat
\providecommand{\titel}{}
\providecommand{\editorInnen}{}
\providecommand{\dateiname}{\jobname}

\vspace{3cm}

\vfill

\footnotesize
\textsc{Quelle}: \titel. Herausgegeben von {\editorInnen}. In: \emph{Arthur Schnitzler: Briefwechsel mit Autorinnen und Autoren}.
 Digitale Edition, https://schnitzler-briefe.acdh.oeaw.ac.at/{\dateiname}.html (Stand \today)
\fi

\end{document}


      