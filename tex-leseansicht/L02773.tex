%% latex-leseansicht-vorspann.tex
%% Vorspann für die Leseansicht.
%% Lädt die gemeinsame Datei latex-vorspann.tex mit nicht gesetztem Schalter.

\newif\ifkorrekturansicht
\korrekturansichtfalse

\input{../tex-inputs/latex-vorspann}


         
         \renewcommand{\erwaehntePersonen}{Personen: Christian Schefer, Leopold Sonnemann}
         \renewcommand{\erwaehnteInstitutionen}{Institutionen: Frankfurter Zeitung, Nouvelle Revue}
         \renewcommand{\erwaehnteOrte}{Orte: Paris, Rue Doré, Wien, rue Feydeau}
         \renewcommand{\erwaehnteWerke}{Werke: Anatol, Frankfurter Zeitung, La Nouvelle Revue, Liebelei. Schauspiel in drei Akten, Sterben. Novelle, Un jeune écrivain viennois: M. Arthur Schnitzler}
               \section[ Paul Goldmann an Arthur Schnitzler, 4. 5. {[}1896{]}]{ Paul Goldmann an Arthur Schnitzler, 4. 5. {[}1896{]}}\nopagebreak\mylabel{v}\rehead{ }\begin{ledgroupsized}[t]{13cm}\normalsize\beginnumbering \toendnotes[C]{\smallbreak\pagebreak[2]} \Standort{DLA, A:Schnitzler, HS.NZ85.1.3166.}
\physDesc{Brief, 1 Blatt, 1 Seite
\newline{}Handschrift: blaue Tinte, deutsche Kurrent\newline{}Beilage: maschinschriftlicher Brief mit handschriftlicher
                                 Unterschrift, 1 Blatt, 1 Seite 
\newline{}Schnitzler: mit Bleistift das Jahr »96« vermerkt }\toendnotes[C]{\smallbreak}\pstart
           \noindent{}{\pb}\textcolor{gray}{\textbf{\textbf{Frankfurter Zeitung\orgindex{Frankfurter Zeitung@Frankfurter Zeitung|pw}}}}\pend
           \pstart
           \textcolor{gray}{\textbf{(\begin{otherlanguage}{french}Gazette de Francfort\end{otherlanguage}\orgindex{Frankfurter Zeitung@Frankfurter Zeitung|pw}).}}\pend
           \pstart
           \textcolor{gray}{\textbf{\textbf{\begin{otherlanguage}{french}Fondateur M.\end{otherlanguage}{ }L. Sonnemann\pwindex{Sonnemann, Leopold 1831-10-29 – 1909-10-30@\textsc{Sonnemann, Leopold} (1831-10-29 – 1909-10-30), \emph{Journalist, Herausgeber}|pw}.}}}\pend
           \pstart
           \begin{otherlanguage}{french}\textcolor{gray}{\textbf{Journal\pwindex{?? Werk@Nicht ermittelte Verfasserinnen und Verfasser!Frankfurter Zeitung1856 – 1943@\emph{Frankfurter Zeitung} {[}1856 – 1943{]}|pwv} politique,
                        financier,}}\end{otherlanguage}\pend
           \pstart
           \begin{otherlanguage}{french}\textcolor{gray}{\textbf{commercial et littéraire.}}\end{otherlanguage}\pend
           \pstart
           \begin{otherlanguage}{french}\textcolor{gray}{\textbf{\textbf{Paraissant trois fois par jour.}}}\end{otherlanguage}\pend
           \pstart
           \begin{otherlanguage}{french}\textcolor{gray}{\textbf{\textbf{Bureau à Paris\oindex{Paris@\textbf{Paris}|pw}:}}}\end{otherlanguage}\pend
           \pstart
           \begin{otherlanguage}{french}\textcolor{gray}{\textbf{\textbf{24. Rue Feydeau\oindex{rue Feydeau@\textbf{rue Feydeau}|pw}.}}}\end{otherlanguage}\hfill \textsc{Paris\oindex{Paris@\textbf{Paris}|pw}}, 4. Mai.\pend
           \pstart
           {\pb}Entſchuldige nur, mein lieber
                  Freund. Ich habe einfach vergeſſen, den Brief mit den anderen ins Couvert
               zu legen, und den Irrthum \label{K_L02773-1v}\edtext{ſofort}{\lemma{\textnormal{\emph{ſofort}}}\Cendnote{\textnormal{Nachdem der vorige Brief bereits am 4. 5. [1896] verfasst ist,
                  dürfte sich das »sofort« auf eine zu diesem Zeitpunkt bereits
                  erfolgte Beschwerde Schnitzler\pwindex{Schnitzler, Arthur 15.05.1862 – 21.10.1931@\textsc{Schnitzler, Arthur} (15.05.1862 – 21.10.1931), \emph{Schriftsteller, Mediziner}|pwk}s
                  beziehen.}}}\label{K_L02773-1h} nach der Abſendung bemerkt.\pend
           \pstart
           Herzlichſt {\\[\baselineskip]}Dein {\\[\baselineskip]}\spacefill\mbox{P. Goldm}\pend
           \leftskip=0em{}{\bigskip}\pstart
           {\pb}{[}ms.:{]} \begin{otherlanguage}{french}MELUN, 12 rue Doré\oindex{Rue Dore@\textbf{Rue Doré}|pw}, ce jeudi 9 avril.\end{otherlanguage}\pend
           \pstart{}\begin{otherlanguage}{french}Cher Monsieur\end{otherlanguage},\pend\pstart
           \label{K_L02773-111v}\edtext{\begin{otherlanguage}{french}Je mets à la poste, en même temps que la présente lettre, le
                     volume\pwindex{Schnitzler, Arthur 15.05.1862 – 21.10.1931@\textsc{Schnitzler, Arthur} (15.05.1862 – 21.10.1931), \emph{Schriftsteller, Mediziner}!Liebelei. Schauspiel in drei Akten1895-10-09@\strich\emph{Liebelei. Schauspiel in drei Akten} {[}1895-10-09{]}|pwv} que vous avez
                  bien voulu me prèter et que je n’ai pu vous renvoyer plus tôt, n’étant pas certain
                  de votre adresse. Je vous suis très reconnaissant de m’avoir ainsi fait connaitre
                     »Liebelei\pwindex{Schnitzler, Arthur 15.05.1862 – 21.10.1931@\textsc{Schnitzler, Arthur} (15.05.1862 – 21.10.1931), \emph{Schriftsteller, Mediziner}!Liebelei. Schauspiel in drei Akten1895-10-09@\strich\emph{Liebelei. Schauspiel in drei Akten} {[}1895-10-09{]}|pw}«, que j’ai lu avec beaucoup
                  d’intérêt, et puisque vous m’avez dit que je recevrais à la Nouvelle Revue\orgindex{Nouvelle Revue@Nouvelle Revue|pw}, les autres écrits\pwindex{Schnitzler, Arthur 15.05.1862 – 21.10.1931@\textsc{Schnitzler, Arthur} (15.05.1862 – 21.10.1931), \emph{Schriftsteller, Mediziner}!Sterben. Novelle1894-10-01 – 1894-12-01@\strich\emph{Sterben. Novelle} {[}1894-10-01 – 1894-12-01{]}|pwv}\pwindex{Schnitzler, Arthur 15.05.1862 – 21.10.1931@\textsc{Schnitzler, Arthur} (15.05.1862 – 21.10.1931), \emph{Schriftsteller, Mediziner}!Anatol1892-10-29@\strich\emph{Anatol} {[}1892-10-29{]}|pwv} de M. Schnitzler, je
                  lui consacrerai certainement une \label{K_L02773-987v}\edtext{chronique\pwindex{Schefer, Christian 1866-07-14 – Februar 1944@\textsc{Schefer, Christian} (1866-07-14 – Februar 1944), \emph{Journalist, Lehrer}!Un jeune ecrivain viennois: M. Arthur Schnitzler1896-06-15@\strich\emph{Un jeune écrivain viennois: M. Arthur Schnitzler} {[}1896-06-15{]}|pwv}}{\lemma{\textnormal{\emph{chronique}}}\Cendnote{\textnormal{Christian Schefer\pwindex{Schefer, Christian 1866-07-14 – Februar 1944@\textsc{Schefer, Christian} (1866-07-14 – Februar 1944), \emph{Journalist, Lehrer}|pwk}: \emph{Un jeune écrivain viennois: M. Arthur Schnitzler}\pwindex{Schefer, Christian 1866-07-14 – Februar 1944@\textsc{Schefer, Christian} (1866-07-14 – Februar 1944), \emph{Journalist, Lehrer}!Un jeune ecrivain viennois: M. Arthur Schnitzler1896-06-15@\strich\emph{Un jeune écrivain viennois: M. Arthur Schnitzler} {[}1896-06-15{]}|pwk}.
                        In: \emph{La Nouvelle Revue}\pwindex{?? Werk@Nicht ermittelte Verfasserinnen und Verfasser!Nouvelle Revue1879 – 1940@\emph{La Nouvelle Revue} {[}1879 – 1940{]}|pwk}, Jg. 18, Nr. 100,
                           Mai–Juni 1896,
                        S. 855–859. Siehe Paul Goldmann an Arthur Schnitzler, 2. 4. [1896].}}}\label{K_L02773-987h}.\end{otherlanguage}}{\lemma{\textnormal{\emph{Je … chronique.}}}\Cendnote{\textnormal{französisch: Lieber Herr\pwindex{Goldmann, Paul 31.01.1865 – 25.09.1935@\textsc{Goldmann, Paul} (31.01.1865 – 25.09.1935), \emph{Schriftsteller, Journalist}|pwv}, ich retourniere
                     mit dem vorliegenden Brief das Buch\pwindex{Schnitzler, Arthur 15.05.1862 – 21.10.1931@\textsc{Schnitzler, Arthur} (15.05.1862 – 21.10.1931), \emph{Schriftsteller, Mediziner}!Liebelei. Schauspiel in drei Akten1895-10-09@\strich\emph{Liebelei. Schauspiel in drei Akten} {[}1895-10-09{]}|pwv}, das Sie mir liehen und das ich nicht früher
                     zurückschicken konnte, weil ich mir Ihrer Adresse nicht sicher war. Ich bin
                     Ihnen sehr dankbar, dass Sie mich mit »Liebelei\pwindex{Schnitzler, Arthur 15.05.1862 – 21.10.1931@\textsc{Schnitzler, Arthur} (15.05.1862 – 21.10.1931), \emph{Schriftsteller, Mediziner}!Liebelei. Schauspiel in drei Akten1895-10-09@\strich\emph{Liebelei. Schauspiel in drei Akten} {[}1895-10-09{]}|pw}« bekannt gemacht haben, das ich mit großem Interesse gelesen
                     habe; und da Sie mir gesagt haben, dass ich an die Nouvelle Revue\orgindex{Nouvelle Revue@Nouvelle Revue|pw} auch die anderen Schriften\pwindex{Schnitzler, Arthur 15.05.1862 – 21.10.1931@\textsc{Schnitzler, Arthur} (15.05.1862 – 21.10.1931), \emph{Schriftsteller, Mediziner}!Sterben. Novelle1894-10-01 – 1894-12-01@\strich\emph{Sterben. Novelle} {[}1894-10-01 – 1894-12-01{]}|pwv}\pwindex{Schnitzler, Arthur 15.05.1862 – 21.10.1931@\textsc{Schnitzler, Arthur} (15.05.1862 – 21.10.1931), \emph{Schriftsteller, Mediziner}!Liebelei. Schauspiel in drei Akten1895-10-09@\strich\emph{Liebelei. Schauspiel in drei Akten} {[}1895-10-09{]}|pwv}\pwindex{Schnitzler, Arthur 15.05.1862 – 21.10.1931@\textsc{Schnitzler, Arthur} (15.05.1862 – 21.10.1931), \emph{Schriftsteller, Mediziner}!Anatol1892-10-29@\strich\emph{Anatol} {[}1892-10-29{]}|pwv} von
                     Herrn Schnitzler\pwindex{Schnitzler, Arthur 15.05.1862 – 21.10.1931@\textsc{Schnitzler, Arthur} (15.05.1862 – 21.10.1931), \emph{Schriftsteller, Mediziner}|pw} gesandt bekomme, werde
                     ich ihm sicherliche eine Besprechung\pwindex{Schefer, Christian 1866-07-14 – Februar 1944@\textsc{Schefer, Christian} (1866-07-14 – Februar 1944), \emph{Journalist, Lehrer}!Un jeune ecrivain viennois: M. Arthur Schnitzler1896-06-15@\strich\emph{Un jeune écrivain viennois: M. Arthur Schnitzler} {[}1896-06-15{]}|pwv} widmen. Sehr geehrter Herr\pwindex{Goldmann, Paul 31.01.1865 – 25.09.1935@\textsc{Goldmann, Paul} (31.01.1865 – 25.09.1935), \emph{Schriftsteller, Journalist}|pwv}, in Verbindung mit erneutem Dank verbleibe ich mit
                     freundlichen Grüßen.}}}\label{K_L02773-111h}\pend
           \pstart
           \label{K_L02773-88v}\edtext{\begin{otherlanguage}{french}Agréez, Cher Monsieur, en même temps que mes nouveaux
                  remerciements, l’assurance de mes sentiments très distingués.\end{otherlanguage}}{\lemma{\textnormal{\emph{Agréez, … distingués.}}}\Cendnote{\textnormal{französisch: »Nehmen Sie,
                     verehrter Herr\pwindex{Goldmann, Paul 31.01.1865 – 25.09.1935@\textsc{Goldmann, Paul} (31.01.1865 – 25.09.1935), \emph{Schriftsteller, Journalist}|pwv},
                     zusammen mit meinem neuerlichen Dank, die Versicherung meiner vorzüglichsten
                     Gefühle entgegen.«}}}\label{K_L02773-88h}\pend
           \pstart \spacefill\mbox{{[}hs. Schefer:{]} Christian Schefer\pwindex{Schefer, Christian 1866-07-14 – Februar 1944@\textsc{Schefer, Christian} (1866-07-14 – Februar 1944), \emph{Journalist, Lehrer}|pw}.}\pend{}
         
         \endnumbering\mylabel{h}\end{ledgroupsized}  \newcommand{\dateiname}{L02773}\newcommand{\titel}{Paul Goldmann an Arthur Schnitzler, 4. 5. [1896]}\newcommand{\editorInnen}{Martin Anton Müller und Laura Untner}%% latex-leseansicht-abspann.tex
%% Abspann für die Leseansicht.
%% Der Schalter \ifkorrekturansicht ist bereits durch den Vorspann gesetzt.

%% latex-abspann.tex
%% Gemeinsamer Abspann für Korrekturansicht und Leseansicht.
%% Setzt den Schalter \ifkorrekturansicht voraus (gesetzt in den
%% einbindenden Dateien latex-korrekturansicht-abspann.tex bzw.
%% latex-leseansicht-abspann.tex).
%% ---------------------------------------------------------------

\normalsize

% Das esempio-Environment wird nur in der Leseansicht benötigt
\ifkorrekturansicht\else
\newenvironment{esempio}[3]%
{
    \vspace{1.5ex}
    \rlap{\underline{#1}}
    \par
    \setlength{\parindent}{0cm}
    \nopagebreak
    \leftskip=#2cm
    \rightskip=#3cm
}
{
    \par
}
\fi

\doendnotes{C}
\bigskip
\vfill

\clearpage

\footnotesize

\ifkorrekturansicht
  \lohead{\textsc{register}}
\fi

% theindex-Environment neu definieren ohne reledmac
\makeatletter
\renewenvironment{theindex}{%
  \ifkorrekturansicht
    \section*{\indexname}%
  \else
    \subsubsection*{Index der erwähnten Entitäten}%
  \fi
  \setlength{\parindent}{0pt}%
  \setlength{\parskip}{0pt plus 0.3pt}%
  \let\item\@idxitem
}{%
  \ifkorrekturansicht\clearpage\fi
}
\makeatother

\IfFileExists{\jobname-pw.ind}{\input{\jobname-pw.ind}}{}

% Quellenangabe nur in der Leseansicht
\ifkorrekturansicht\else
% Fallback-Definitionen, falls die .tex-Datei \titel etc. nicht gesetzt hat
\providecommand{\titel}{}
\providecommand{\editorInnen}{}
\providecommand{\dateiname}{\jobname}

\vspace{3cm}

\vfill

\footnotesize
\textsc{Quelle}: \titel. Herausgegeben von {\editorInnen}. In: \emph{Arthur Schnitzler: Briefwechsel mit Autorinnen und Autoren}.
 Digitale Edition, https://schnitzler-briefe.acdh.oeaw.ac.at/{\dateiname}.html (Stand \today)
\fi

\end{document}


      