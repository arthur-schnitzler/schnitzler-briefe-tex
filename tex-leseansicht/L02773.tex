%% latex-leseansicht-vorspann.tex
%% Vorspann für die Leseansicht.
%% Lädt die gemeinsame Datei latex-vorspann.tex mit nicht gesetztem Schalter.

\newif\ifkorrekturansicht
\korrekturansichtfalse

\input{../tex-inputs/latex-vorspann}


\section[ Paul Goldmann an Arthur Schnitzler, 4. 5. [1896]]{L02773 Paul Goldmann an Arthur Schnitzler,  4. 5. [1896]}
\nopagebreak\mylabel{L02773v}
\rehead{ }\normalsize\beginnumbering\briefempfaengerindex{Schnitzler, Arthur@\textsc{Schnitzler, Arthur}!zzzGoldmann, Paul@\emph{von Paul Goldmann}!1896-05-041@{4. 5. [1896]}|(be}
\toendnotes[C]{\smallbreak\pagebreak[2]}
\correspDesc{Versand  durch Paul Goldmann am 4. 5. [1896] in Paris
\newline{}Erhalt  durch Arthur Schnitzler im Zeitraum [5. 5. 1896
                  – 9. 5. 1896?] in Wien}\toendnotes[C]{\smallbreak}
\Standort{DLA, A:Schnitzler, HS.NZ85.1.3166.}
\physDesc{Brief, 1 Blatt, 1 Seite, 817 Zeichen
\newline{}Handschrift: blaue Tinte, deutsche Kurrent
\newline{}Beilage: maschinschriftlicher Brief mit handschriftlicher
                                 Unterschrift, 1 Blatt, 1 Seite 
\newline{}Schnitzler: mit Bleistift das Jahr »96« vermerkt }\toendnotes[C]{\smallbreak}
\pstart
           {\pb}\textcolor{gray}{\textbf{\textbf{Frankfurter Zeitung\orgindex{Frankfurter Zeitung@Frankfurter Zeitung|pw}}}}\pend
           
\pstart
           \textcolor{gray}{\textbf{(\begin{otherlanguage}{french}Gazette de Francfort\end{otherlanguage}\orgindex{Frankfurter Zeitung@Frankfurter Zeitung|pw}).}}\pend
           
\pstart
           \textcolor{gray}{\textbf{\textbf{\begin{otherlanguage}{french}Fondateur M.\end{otherlanguage}{ }L. Sonnemann\pwindex{Sonnemann, Leopold 29.\,10.\,1831 Höchberg – 30.\,10.\,1909 Frankfurt am Main@\textsc{Sonnemann, Leopold} (29.\,10.\,1831 Höchberg – 30.\,10.\,1909 Frankfurt am Main), \emph{Journalist, Herausgeber}|pw}.}}}\pend
           
\pstart
           \begin{otherlanguage}{french}\textcolor{gray}{\textbf{Journal\pwindex{Frankfurter Zeitung@\emph{Frankfurter Zeitung}|pwv} politique,
                        financier,}}\end{otherlanguage}\pend
           
\pstart
           \begin{otherlanguage}{french}\textcolor{gray}{\textbf{commercial et littéraire.}}\end{otherlanguage}\pend
           
\pstart
           \begin{otherlanguage}{french}\textcolor{gray}{\textbf{\textbf{Paraissant trois fois par jour.}}}\end{otherlanguage}\pend
           
\pstart
           \begin{otherlanguage}{french}\textcolor{gray}{\textbf{\textbf{Bureau à Paris\oindex{Paris@\textbf{Paris}, \emph{Hauptstadt}|pw}:}}}\end{otherlanguage}\pend
           
\pstart
           \begin{otherlanguage}{french}\textcolor{gray}{\textbf{\textbf{24. Rue Feydeau\oindex{rue Feydeau@\textbf{rue Feydeau}, \emph{Straße}|pw}.}}}\end{otherlanguage}\hfill \textsc{Paris\oindex{Paris@\textbf{Paris}, \emph{Hauptstadt}|pw}}, 4. Mai.\pend
           \vspace{0.5em}
\pstart
           {\pb}Entſchuldige nur, mein lieber
                  Freund. Ich habe einfach vergeſſen, den Brief mit den anderen ins Couvert
               zu legen, und den Irrthum \label{K_L02773-1v}\edtext{ſofort}{\lemma{\textnormal{\emph{sofort}}}\Cendnote{\textnormal{Nachdem der vorige Brief bereits am XXXX Auszeichnungsfehler: Dokument L02773 nicht gefunden verfasst worden ist,
                  dürfte sich das »sofort« auf eine zu diesem Zeitpunkt bereits
                  erfolgte Beschwerde Schnitzlers beziehen.}}}\label{K_L02773-1} nach der Abſendung bemerkt.\pend
           
\pstart
           Herzlichſt {\\[\baselineskip]}Dein {\\[\baselineskip]}\spacefill\mbox{P. Goldm}\pend
           \leftskip=0em{}\selectlanguage{ngerman}\vspace{1em}{\vspace{1\baselineskip}}
\pstart
           {\pb}{[}ms.:{]} \begin{otherlanguage}{french}MELUN, 12 rue Doré\oindex{Rue Doré@\textbf{Rue Doré}, \emph{Straße}|pw}, ce jeudi 9 avril.\end{otherlanguage}\pend
           
\pstart{}\begin{otherlanguage}{french}Cher Monsieur\end{otherlanguage},\pend\vspace{0.5em}
\pstart
           \label{K_L02773-2v}\edtext{\begin{otherlanguage}{french}Je mets à la poste, en même temps que la présente lettre, le
                     volume\pwindex{Schnitzler, Arthur 15.\,5.\,1862 Wien – 21.\,10.\,1931 ebd.@\textsc{Schnitzler, Arthur} (15.\,5.\,1862 Wien – 21.\,10.\,1931 ebd.), \emph{Schriftsteller, Mediziner}!Liebelei. Schauspiel in drei Akten@\strich\emph{Liebelei. Schauspiel in drei Akten}|pwv} que vous avez
                  bien voulu me prèter et que je n’ai pu vous renvoyer plus tôt, n’étant pas certain
                  de votre adresse. Je vous suis très reconnaissant de m’avoir ainsi fait connaitre
                     »Liebelei\pwindex{Schnitzler, Arthur 15.\,5.\,1862 Wien – 21.\,10.\,1931 ebd.@\textsc{Schnitzler, Arthur} (15.\,5.\,1862 Wien – 21.\,10.\,1931 ebd.), \emph{Schriftsteller, Mediziner}!Liebelei. Schauspiel in drei Akten@\strich\emph{Liebelei. Schauspiel in drei Akten}|pw}«, que j’ai lu avec beaucoup
                  d’intérêt, et puisque vous m’avez dit que je recevrais à la Nouvelle Revue\orgindex{Nouvelle Revue@Nouvelle Revue|pw}, les autres écrits\pwindex{Schnitzler, Arthur 15.\,5.\,1862 Wien – 21.\,10.\,1931 ebd.@\textsc{Schnitzler, Arthur} (15.\,5.\,1862 Wien – 21.\,10.\,1931 ebd.), \emph{Schriftsteller, Mediziner}!Sterben. Novelle@\strich\emph{Sterben. Novelle}|pwv}\pwindex{Schnitzler, Arthur 15.\,5.\,1862 Wien – 21.\,10.\,1931 ebd.@\textsc{Schnitzler, Arthur} (15.\,5.\,1862 Wien – 21.\,10.\,1931 ebd.), \emph{Schriftsteller, Mediziner}!Anatol@\strich\emph{Anatol}|pwv} de M. Schnitzler, je
                  lui consacrerai certainement une \label{K_L02773-3v}\edtext{chronique\pwindex{Schefer, Christian 14.\,7.\,1866 Paris – Februar 1944 Marokko@\textsc{Schefer, Christian} (14.\,7.\,1866 Paris – Februar 1944 Marokko), \emph{Journalist, Lehrer}!Un jeune écrivain viennois: M. Arthur Schnitzler@\strich\emph{Un jeune écrivain viennois: M. Arthur Schnitzler}|pwv}}{\lemma{\textnormal{\emph{chronique}}}\Cendnote{\textnormal{Christian Schefer\pwindex{Schefer, Christian 14.\,7.\,1866 Paris – Februar 1944 Marokko@\textsc{Schefer, Christian} (14.\,7.\,1866 Paris – Februar 1944 Marokko), \emph{Journalist, Lehrer}|pwk}: \emph{Un jeune écrivain viennois: M. Arthur Schnitzler}\pwindex{Schefer, Christian 14.\,7.\,1866 Paris – Februar 1944 Marokko@\textsc{Schefer, Christian} (14.\,7.\,1866 Paris – Februar 1944 Marokko), \emph{Journalist, Lehrer}!Un jeune écrivain viennois: M. Arthur Schnitzler@\strich\emph{Un jeune écrivain viennois: M. Arthur Schnitzler}|pwk}.
                        In: \emph{La Nouvelle Revue}\pwindex{Nouvelle Revue@\emph{La Nouvelle Revue}|pwk}, Jg. 18, Nr. 100,
                           Mai–Juni 1896,
                        S. 855–859. (Siehe XXXX Auszeichnungsfehler: Dokument L02770 nicht gefunden.)}}}\label{K_L02773-3}.\end{otherlanguage}}{\lemma{\textnormal{\emph{Je … chronique.}}}\Cendnote{\textnormal{französisch: Lieber Herr\pwindex{Goldmann, Paul 31.\,1.\,1865 Breslau – 25.\,9.\,1935 Wien@\textsc{Goldmann, Paul} (31.\,1.\,1865 Breslau – 25.\,9.\,1935 Wien), \emph{Schriftsteller, Journalist}|pwv}, ich retourniere
                     mit dem vorliegenden Brief das Buch\pwindex{Schnitzler, Arthur 15.\,5.\,1862 Wien – 21.\,10.\,1931 ebd.@\textsc{Schnitzler, Arthur} (15.\,5.\,1862 Wien – 21.\,10.\,1931 ebd.), \emph{Schriftsteller, Mediziner}!Liebelei. Schauspiel in drei Akten@\strich\emph{Liebelei. Schauspiel in drei Akten}|pwv}, das Sie mir liehen und das ich nicht früher
                     zurückschicken konnte, weil ich mir Ihrer Adresse nicht sicher war. Ich bin
                     Ihnen sehr dankbar, dass Sie mich mit ›Liebelei\pwindex{Schnitzler, Arthur 15.\,5.\,1862 Wien – 21.\,10.\,1931 ebd.@\textsc{Schnitzler, Arthur} (15.\,5.\,1862 Wien – 21.\,10.\,1931 ebd.), \emph{Schriftsteller, Mediziner}!Liebelei. Schauspiel in drei Akten@\strich\emph{Liebelei. Schauspiel in drei Akten}|pw}‹ bekannt gemacht haben, das ich mit großem Interesse gelesen
                     habe; und da Sie mir gesagt haben, dass ich an die Nouvelle Revue\orgindex{Nouvelle Revue@Nouvelle Revue|pw} auch die anderen Schriften\pwindex{Schnitzler, Arthur 15.\,5.\,1862 Wien – 21.\,10.\,1931 ebd.@\textsc{Schnitzler, Arthur} (15.\,5.\,1862 Wien – 21.\,10.\,1931 ebd.), \emph{Schriftsteller, Mediziner}!Sterben. Novelle@\strich\emph{Sterben. Novelle}|pwv}\pwindex{Schnitzler, Arthur 15.\,5.\,1862 Wien – 21.\,10.\,1931 ebd.@\textsc{Schnitzler, Arthur} (15.\,5.\,1862 Wien – 21.\,10.\,1931 ebd.), \emph{Schriftsteller, Mediziner}!Liebelei. Schauspiel in drei Akten@\strich\emph{Liebelei. Schauspiel in drei Akten}|pwv}\pwindex{Schnitzler, Arthur 15.\,5.\,1862 Wien – 21.\,10.\,1931 ebd.@\textsc{Schnitzler, Arthur} (15.\,5.\,1862 Wien – 21.\,10.\,1931 ebd.), \emph{Schriftsteller, Mediziner}!Anatol@\strich\emph{Anatol}|pwv} von
                     Herrn Schnitzler gesandt bekomme, werde
                     ich ihm sicherlich eine Besprechung\pwindex{Schefer, Christian 14.\,7.\,1866 Paris – Februar 1944 Marokko@\textsc{Schefer, Christian} (14.\,7.\,1866 Paris – Februar 1944 Marokko), \emph{Journalist, Lehrer}!Un jeune écrivain viennois: M. Arthur Schnitzler@\strich\emph{Un jeune écrivain viennois: M. Arthur Schnitzler}|pwv} widmen. Sehr geehrter Herr\pwindex{Goldmann, Paul 31.\,1.\,1865 Breslau – 25.\,9.\,1935 Wien@\textsc{Goldmann, Paul} (31.\,1.\,1865 Breslau – 25.\,9.\,1935 Wien), \emph{Schriftsteller, Journalist}|pwv}, in Verbindung mit erneutem Dank verbleibe ich mit
                     freundlichen Grüßen.}}}\label{K_L02773-2}\pend
           
\pstart
           \label{K_L02773-4v}\edtext{\begin{otherlanguage}{french}Agréez, Cher Monsieur, en même temps que mes nouveaux
                  remerciements, l’assurance de mes sentiments très distingués.\end{otherlanguage}}{\lemma{\textnormal{\emph{Agréez, … distingués.}}}\Cendnote{\textnormal{französisch: Nehmen Sie,
                     verehrter Herr\pwindex{Goldmann, Paul 31.\,1.\,1865 Breslau – 25.\,9.\,1935 Wien@\textsc{Goldmann, Paul} (31.\,1.\,1865 Breslau – 25.\,9.\,1935 Wien), \emph{Schriftsteller, Journalist}|pwv},
                     zusammen mit meinem neuerlichen Dank, die Versicherung meiner vorzüglichsten
                     Gefühle entgegen.}}}\label{K_L02773-4}\pend
           \pstart \spacefill\mbox{{[}hs. Schefer:{]} Christian Schefer\pwindex{Schefer, Christian 14.\,7.\,1866 Paris – Februar 1944 Marokko@\textsc{Schefer, Christian} (14.\,7.\,1866 Paris – Februar 1944 Marokko), \emph{Journalist, Lehrer}|pw}.}\pend{}\selectlanguage{ngerman}\endnumbering\briefempfaengerindex{Schnitzler, Arthur@\textsc{Schnitzler, Arthur}!zzzGoldmann, Paul@\emph{von Paul Goldmann}!1896-05-041@{4. 5. [1896]}|)be}\mylabel{L02773h}  \newcommand{\dateiname}{L02773}\newcommand{\titel}{Paul Goldmann an Arthur Schnitzler, 4. 5. [1896]}\newcommand{\editorInnen}{Martin Anton Müller und Laura Untner}%% latex-leseansicht-abspann.tex
%% Abspann für die Leseansicht.
%% Der Schalter \ifkorrekturansicht ist bereits durch den Vorspann gesetzt.

%% latex-abspann.tex
%% Gemeinsamer Abspann für Korrekturansicht und Leseansicht.
%% Setzt den Schalter \ifkorrekturansicht voraus (gesetzt in den
%% einbindenden Dateien latex-korrekturansicht-abspann.tex bzw.
%% latex-leseansicht-abspann.tex).
%% ---------------------------------------------------------------

\normalsize

% Das esempio-Environment wird nur in der Leseansicht benötigt
\ifkorrekturansicht\else
\newenvironment{esempio}[3]%
{
    \vspace{1.5ex}
    \rlap{\underline{#1}}
    \par
    \setlength{\parindent}{0cm}
    \nopagebreak
    \leftskip=#2cm
    \rightskip=#3cm
}
{
    \par
}
\fi

\doendnotes{C}
\bigskip
\vfill

\clearpage

\footnotesize

\ifkorrekturansicht
  \lohead{\textsc{register}}
\fi

% theindex-Environment neu definieren ohne reledmac
\makeatletter
\renewenvironment{theindex}{%
  \ifkorrekturansicht
    \section*{\indexname}%
  \else
    \subsubsection*{Index der erwähnten Entitäten}%
  \fi
  \setlength{\parindent}{0pt}%
  \setlength{\parskip}{0pt plus 0.3pt}%
  \let\item\@idxitem
}{%
  \ifkorrekturansicht\clearpage\fi
}
\makeatother

\IfFileExists{\jobname-pw.ind}{\input{\jobname-pw.ind}}{}

% Quellenangabe nur in der Leseansicht
\ifkorrekturansicht\else
% Fallback-Definitionen, falls die .tex-Datei \titel etc. nicht gesetzt hat
\providecommand{\titel}{}
\providecommand{\editorInnen}{}
\providecommand{\dateiname}{\jobname}

\vspace{3cm}

\vfill

\footnotesize
\textsc{Quelle}: \titel. Herausgegeben von {\editorInnen}. In: \emph{Arthur Schnitzler: Briefwechsel mit Autorinnen und Autoren}.
 Digitale Edition, https://schnitzler-briefe.acdh.oeaw.ac.at/{\dateiname}.html (Stand \today)
\fi

\end{document}


