%% latex-korrekturansicht-vorspann.tex
%% Vorspann für die Korrekturansicht.
%% Lädt die gemeinsame Datei latex-vorspann.tex mit gesetztem Schalter.

\newif\ifkorrekturansicht
\korrekturansichttrue

\input{../tex-inputs/latex-vorspann}


\section[ Paul Goldmann an Arthur Schnitzler, 4. 5. {[}1896{]}]{L02773 Paul Goldmann an Arthur Schnitzler, 4. 5. {[}1896{]}}
\nopagebreak\mylabel{L02773v}
\rehead{ }\normalsize\beginnumbering\briefempfaengerindex{Schnitzler, Arthur@\textsc{Schnitzler, Arthur}!zzzGoldmann, Paul@\emph{von Paul Goldmann}!1896-05-041@{4. 5. {[}1896{]}}|(be}
\toendnotes[C]{\smallbreak\pagebreak[2]}\Standort{DLA, A:Schnitzler, HS.NZ85.1.3166.}
\physDesc{Brief, 1 Blatt, 1 Seite, 817 Zeichen
\newline{}Handschrift: blaue Tinte, deutsche Kurrent
\newline{}Beilage: maschinschriftlicher Brief mit handschriftlicher
                                 Unterschrift, 1 Blatt, 1 Seite 
\newline{}Schnitzler: mit Bleistift das Jahr »96« vermerkt }\toendnotes[C]{\smallbreak}
\pstart
           {\pb}\textcolor{gray}{\textbf{\textbf{Frankfurter Zeitung\orgindex{Frankfurter Zeitung@Frankfurter Zeitung|pw}}}}\pend
           
\pstart
           \textcolor{gray}{\textbf{(\begin{otherlanguage}{french}Gazette de Francfort\end{otherlanguage}\orgindex{Frankfurter Zeitung@Frankfurter Zeitung|pw}).}}\pend
           
\pstart
           \textcolor{gray}{\textbf{\textbf{\begin{otherlanguage}{french}Fondateur M.\end{otherlanguage}{ }L. Sonnemann\pwindex{Sonnemann, Leopold 1831-10-29 – 1909-10-30@\textsc{Sonnemann, Leopold} (1831-10-29 – 1909-10-30), \emph{Journalist/Journalistin, Herausgeber/Herausgeberin}|pw}.}}}\pend
           
\pstart
           \begin{otherlanguage}{french}\textcolor{gray}{\textbf{Journal\pwindex{Frankfurter Zeitung@\emph{Frankfurter Zeitung}|pwv} politique,
                        financier,}}\end{otherlanguage}\pend
           
\pstart
           \begin{otherlanguage}{french}\textcolor{gray}{\textbf{commercial et littéraire.}}\end{otherlanguage}\pend
           
\pstart
           \begin{otherlanguage}{french}\textcolor{gray}{\textbf{\textbf{Paraissant trois fois par jour.}}}\end{otherlanguage}\pend
           
\pstart
           \begin{otherlanguage}{french}\textcolor{gray}{\textbf{\textbf{Bureau à Paris\oindex{Paris@\textbf{Paris}, \emph{P.PPLC}|pw}:}}}\end{otherlanguage}\pend
           
\pstart
           \begin{otherlanguage}{french}\textcolor{gray}{\textbf{\textbf{24. Rue Feydeau\oindex{rue Feydeau@\textbf{rue Feydeau}, \emph{Straße (K.STR)}|pw}.}}}\end{otherlanguage}\hfill \textsc{Paris\oindex{Paris@\textbf{Paris}, \emph{P.PPLC}|pw}}, 4. Mai.\pend
           \vspace{0.5em}
\pstart
           {\pb}Entſchuldige nur, mein lieber
                  Freund. Ich habe einfach vergeſſen, den Brief mit den anderen ins Couvert
               zu legen, und den Irrthum \label{K_L02773-1v}\edtext{ſofort}{\lemma{\textnormal{\emph{ſofort}}}\Cendnote{\textnormal{Nachdem der vorige Brief bereits am 4. 5. [1896] verfasst worden ist,
                  dürfte sich das »sofort« auf eine zu diesem Zeitpunkt bereits
                  erfolgte Beschwerde Schnitzlers beziehen.}}}\label{K_L02773-1} nach der Abſendung bemerkt.\pend
           
\pstart
           Herzlichſt {\\[\baselineskip]}Dein {\\[\baselineskip]}\spacefill\mbox{P. Goldm}\pend
           \leftskip=0em{}\selectlanguage{ngerman}\vspace{1em}{\vspace{1\baselineskip}}
\pstart
           {\pb}{[}ms.:{]} \begin{otherlanguage}{french}MELUN, 12 rue Doré\oindex{Rue Dore@\textbf{Rue Doré}, \emph{Straße (K.STR)}|pw}, ce jeudi 9 avril.\end{otherlanguage}\pend
           
\pstart{}\begin{otherlanguage}{french}Cher Monsieur\end{otherlanguage},\pend\vspace{0.5em}
\pstart
           \label{K_L02773-2v}\edtext{\begin{otherlanguage}{french}Je mets à la poste, en même temps que la présente lettre, le
                     volume\pwindex{Liebelei. Schauspiel in drei Akten@\emph{Liebelei. Schauspiel in drei Akten}|pwv} que vous avez
                  bien voulu me prèter et que je n’ai pu vous renvoyer plus tôt, n’étant pas certain
                  de votre adresse. Je vous suis très reconnaissant de m’avoir ainsi fait connaitre
                     »Liebelei\pwindex{Liebelei. Schauspiel in drei Akten@\emph{Liebelei. Schauspiel in drei Akten}|pw}«, que j’ai lu avec beaucoup
                  d’intérêt, et puisque vous m’avez dit que je recevrais à la Nouvelle Revue\orgindex{Nouvelle Revue@Nouvelle Revue|pw}, les autres écrits\pwindex{Sterben. Novelle@\emph{Sterben. Novelle}|pwv}\pwindex{Anatol@\emph{Anatol}|pwv} de M. Schnitzler, je
                  lui consacrerai certainement une \label{K_L02773-3v}\edtext{chronique\pwindex{Un jeune ecrivain viennois: M. Arthur Schnitzler@\emph{Un jeune écrivain viennois: M. Arthur Schnitzler}|pwv}}{\lemma{\textnormal{\emph{chronique}}}\Cendnote{\textnormal{Christian Schefer\pwindex{Schefer, Christian 1866-07-14 – Februar 1944@\textsc{Schefer, Christian} (1866-07-14 – Februar 1944), \emph{Journalist/Journalistin, Lehrer/Lehrerin}|pwk}: \emph{Un jeune écrivain viennois: M. Arthur Schnitzler}\pwindex{Un jeune ecrivain viennois: M. Arthur Schnitzler@\emph{Un jeune écrivain viennois: M. Arthur Schnitzler}|pwk}.
                        In: \emph{La Nouvelle Revue}\pwindex{Nouvelle Revue@\emph{La Nouvelle Revue}|pwk}, Jg. 18, Nr. 100,
                           Mai–Juni 1896,
                        S. 855–859. (Siehe Paul Goldmann an Arthur Schnitzler, 2. 4. [1896].)}}}\label{K_L02773-3}.\end{otherlanguage}}{\lemma{\textnormal{\emph{Je … chronique.}}}\Cendnote{\textnormal{französisch: Lieber Herr\pwindex{Goldmann, Paul 31.01.1865 – 25.09.1935@\textsc{Goldmann, Paul} (31.01.1865 – 25.09.1935), \emph{Schriftsteller/Schriftstellerin, Journalist/Journalistin}|pwv}, ich retourniere
                     mit dem vorliegenden Brief das Buch\pwindex{Liebelei. Schauspiel in drei Akten@\emph{Liebelei. Schauspiel in drei Akten}|pwv}, das Sie mir liehen und das ich nicht früher
                     zurückschicken konnte, weil ich mir Ihrer Adresse nicht sicher war. Ich bin
                     Ihnen sehr dankbar, dass Sie mich mit ›Liebelei\pwindex{Liebelei. Schauspiel in drei Akten@\emph{Liebelei. Schauspiel in drei Akten}|pw}‹ bekannt gemacht haben, das ich mit großem Interesse gelesen
                     habe; und da Sie mir gesagt haben, dass ich an die Nouvelle Revue\orgindex{Nouvelle Revue@Nouvelle Revue|pw} auch die anderen Schriften\pwindex{Sterben. Novelle@\emph{Sterben. Novelle}|pwv}\pwindex{Liebelei. Schauspiel in drei Akten@\emph{Liebelei. Schauspiel in drei Akten}|pwv}\pwindex{Anatol@\emph{Anatol}|pwv} von
                     Herrn Schnitzler gesandt bekomme, werde
                     ich ihm sicherlich eine Besprechung\pwindex{Un jeune ecrivain viennois: M. Arthur Schnitzler@\emph{Un jeune écrivain viennois: M. Arthur Schnitzler}|pwv} widmen. Sehr geehrter Herr\pwindex{Goldmann, Paul 31.01.1865 – 25.09.1935@\textsc{Goldmann, Paul} (31.01.1865 – 25.09.1935), \emph{Schriftsteller/Schriftstellerin, Journalist/Journalistin}|pwv}, in Verbindung mit erneutem Dank verbleibe ich mit
                     freundlichen Grüßen.}}}\label{K_L02773-2}\pend
           
\pstart
           \label{K_L02773-4v}\edtext{\begin{otherlanguage}{french}Agréez, Cher Monsieur, en même temps que mes nouveaux
                  remerciements, l’assurance de mes sentiments très distingués.\end{otherlanguage}}{\lemma{\textnormal{\emph{Agréez, … distingués.}}}\Cendnote{\textnormal{französisch: Nehmen Sie,
                     verehrter Herr\pwindex{Goldmann, Paul 31.01.1865 – 25.09.1935@\textsc{Goldmann, Paul} (31.01.1865 – 25.09.1935), \emph{Schriftsteller/Schriftstellerin, Journalist/Journalistin}|pwv},
                     zusammen mit meinem neuerlichen Dank, die Versicherung meiner vorzüglichsten
                     Gefühle entgegen.}}}\label{K_L02773-4}\pend
           \pstart \spacefill\mbox{{[}hs. :{]} Christian Schefer\pwindex{Schefer, Christian 1866-07-14 – Februar 1944@\textsc{Schefer, Christian} (1866-07-14 – Februar 1944), \emph{Journalist/Journalistin, Lehrer/Lehrerin}|pw}.}\pend{}\selectlanguage{ngerman}\endnumbering\briefempfaengerindex{Schnitzler, Arthur@\textsc{Schnitzler, Arthur}!zzzGoldmann, Paul@\emph{von Paul Goldmann}!1896-05-041@{4. 5. {[}1896{]}}|)be}\mylabel{L02773h}  \normalsize

\doendnotes{C}
\bigskip
\vfill

\clearpage

\footnotesize

\lohead{\textsc{register}}

% Definiere theindex-Environment komplett neu ohne reledmac
\makeatletter
\renewenvironment{theindex}{%
  \section*{\indexname}%
  \setlength{\parindent}{0pt}%
  \setlength{\parskip}{0pt plus 0.3pt}%
  \let\item\@idxitem
}{%
  \clearpage
}
\makeatother

\IfFileExists{\jobname-pw.ind}{\input{\jobname-pw.ind}}{}

\end{document}

      