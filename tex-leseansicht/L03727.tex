%% latex-korrekturansicht-vorspann.tex
%% Vorspann für die Korrekturansicht.
%% Lädt die gemeinsame Datei latex-vorspann.tex mit gesetztem Schalter.

\newif\ifkorrekturansicht
\korrekturansichttrue

\input{../tex-inputs/latex-vorspann}


\section[Elsa Plessner an Arthur Schnitzler, 8. 10. 1900]{L03727 Elsa Plessner an Arthur Schnitzler, 8. 10. 1900}
\nopagebreak\mylabel{L03727v}
\rehead{ }\normalsize\beginnumbering\briefempfaengerindex{Schnitzler, Arthur@\textsc{Schnitzler, Arthur}!zzzPlessner, Elsa@\emph{von Elsa Plessner}!1900-10-081@{8. 10. 1900}|(be}
\toendnotes[C]{\smallbreak\pagebreak[2]}\Standort{DLA, A:Schnitzler, HS.1985.1.419.}
\physDesc{Brief,  Blätter, 4 Seiten, 2330 Zeichen
\newline{}Handschrift: , lateinische Kurrent}\toendnotes[C]{\smallbreak}
\pstart
           {\pb}Wien I. Kärnthnerstraße N\textsuperscript{o} 10\oindex{Kaerntner Strasse 10@\textbf{Kärntner Straße 10}, \emph{Wohngebäude (K.WHS)}|pw}\pend
           
\pstart
           \raggedleft{}den 8. October 1900\pend
           
\pstart{}Verehrter, lieber Herr Doctor!\pend\vspace{0.5em}
\pstart
           Da ist also endlich \label{K_L03727-1v}\edtext{das Buch\pwindex{glaeserne Kaefig. Skizzen und Novellen@\emph{Der gläserne Käfig. Skizzen und Novellen}|pwv}}{\lemma{\textnormal{\emph{das Buch}}}\Cendnote{\textnormal{(Nicht überlieferte) Beilage des Briefes
                  war Plessners\pwindex{Plessner, Elsa 22.08.1875 – 01.05.1932@\textsc{Plessner, Elsa} (22.08.1875 – 01.05.1932), \emph{Schriftsteller/Schriftstellerin}|pwk} neu erschienene Textsammlung
                     \emph{Der gläserne Käfig}\pwindex{glaeserne Kaefig. Skizzen und Novellen@\emph{Der gläserne Käfig. Skizzen und Novellen}|pwk}.}}}\label{K_L03727-1}, das, wie Sie
               wissen, eigentlich Ihnen zugeeignet ist. Die Widmung drucken zu lassen, wäre aber
               geschmacklos gewesen und ich weiß zu genau, wie Sie \label{K_L03727-2v}\edtext{darüber denken}{\lemma{\textnormal{\emph{darüber denken}}}\Cendnote{\textnormal{Vgl. Elsa Plessner an Arthur Schnitzler, 10. 1. 1900.}}}\label{K_L03727-2}. – \pend
           
\pstart
           Sie werden natürlich lauter, alte Bekannte unter den Arbeiten finden, die schon
               früher \label{K_L03727-3v}\edtext{Ihrer Kritik überantwortet}{\lemma{\textnormal{\emph{Ihrer … überantwortet}}}\Cendnote{\textnormal{Bereits mit ihrem Brief vom 15. 9. 1896 sandte Plessner\pwindex{Plessner, Elsa 22.08.1875 – 01.05.1932@\textsc{Plessner, Elsa} (22.08.1875 – 01.05.1932), \emph{Schriftsteller/Schriftstellerin}|pwk}{ }Schnitzler zehn kurze Texte als erste
                  Zusammenstellung ihres Bandes\pwindex{glaeserne Kaefig. Skizzen und Novellen@\emph{Der gläserne Käfig. Skizzen und Novellen}|pwkv}. Anderen Briefen lagen einzelne Entwürfe bei z. B. \emph{Der gläserne Käfig}\pwindex{glaeserne Kaefig. Eine Parabel@\emph{Der gläserne Käfig. Eine Parabel}|pwk} am 18. 3. 1897 und \emph{Der neue Lehrer}\pwindex{neue Lehrer. Novelle@\emph{Der neue Lehrer. Novelle}|pwk} am 2. 1. 1899.}}}\label{K_L03727-3} waren, und die Sie meist zur
               ganzen oder theilweisen Umarbeitung verurtheilt haben.\pend
           
\pstart
           Seien Sie nicht böse, dass ich Ihnen darin nicht immer Folge geleistet habe. Nur zum
               kleinsten Theil geschah es aus principiellen Grün{\pb}den, dass ich die
               einmal vorliegende Fassung der Arbeit gegen Ihre Kritik aufrechterhielt. (Siehe »Warten\pwindex{Warten@\emph{Warten}|pw}, Warum\pwindex{Warum@\emph{Warum}|pw}«) Zum größten Theil war es die mir leider anhaftende Eigenschaft, mich
               mit einem Stoff, dessen Ausgestaltung – ob gut oder schlecht – fertig vor mir liegt,
               nicht nochmals befassen zu können. Es ist keine Leichtfertigkeit – glauben Sie das ja
               nicht – und auch nicht Mangel an Selbstkritik, denn meistens sagt mir mein
               künstlerisches Gewissen dasselbe, was Ihre Kritik – nur in verschärfter Tonart –
               bemängelt. Aber ich entwickele mich so rapid, (leider? oder \label{K_L03727-4v}\edtext{G. s. D.}{\lemma{\textnormal{\emph{G. s. D.}}}\Cendnote{\textnormal{Gott sei
                  Dank}}}\label{K_L03727-4}?) dass ich in Schnellzugsgeschwindigkeit die Stationen durcheile und
               wenn man von mir verlangt, nach einer überholten Haltestelle zurückzukehren, so finde
               ich weder Stimmung noch Gedanken der Arbeit {\pb}rein und unbeeinträchtigt
               wieder. Es käme einfach \uline{gar nichts} heraus!\pend
           
\pstart
           Ich weiß, Sie werden wieder schimpfen. Aber Sie glauben gar nicht, wie dankbar ich
               Ihnen dafür bin und dürfen nicht in die falsche Meinung verfallen, dass Ihre Kritik
               an meinen Arbeiten resultatlos sei. O \uline{nein}!!! Was Sie
               mir über eine Arbeit sagen, trägt an der nächst folgenden Früchte. So erziehen Sie
               mich seit fünf Jahren – wahrscheinlich ohne es selbst zu wissen. –\pend
           
\pstart
           Ich will damit nicht sagen dass ich mich nicht manchmal gegen Ihre Meinung auflehne –
               besonders auf dramatisch-technischem Gebiet –. Aber wäre mein Talent Ihrer Kritik
               wert, {\pb}wenn es sich so rückhaltlos einer anderen künstlerischen
               Individualität unterordnen könnte?\pend
           
\pstart
           Ich hoffe, Sie werden über diese literarische Liebeserklärung nicht lachen und nur
               freundlichst Nachricht geben, welchen Eindruck das Buch\pwindex{glaeserne Kaefig. Skizzen und Novellen@\emph{Der gläserne Käfig. Skizzen und Novellen}|pwv} in seiner Gesammtheit auf Sie gemacht hat. Ich bin
               sehr gespannt darauf.\pend
           
\pstart
           Verehrungsvollen \uline{u} herzliche Grüße von{\\[\baselineskip]}\spacefill\mbox{Elsa Plessner}. \pend
           \leftskip=0em{}\selectlanguage{ngerman}\endnumbering\briefempfaengerindex{Schnitzler, Arthur@\textsc{Schnitzler, Arthur}!zzzPlessner, Elsa@\emph{von Elsa Plessner}!1900-10-081@{8. 10. 1900}|)be}\mylabel{L03727h}
\begin{anhang}
\end{anhang}\normalsize

\doendnotes{C}
\bigskip
\vfill

\clearpage

\footnotesize

\lohead{\textsc{register}}

% Definiere theindex-Environment komplett neu ohne reledmac
\makeatletter
\renewenvironment{theindex}{%
  \section*{\indexname}%
  \setlength{\parindent}{0pt}%
  \setlength{\parskip}{0pt plus 0.3pt}%
  \let\item\@idxitem
}{%
  \clearpage
}
\makeatother

\IfFileExists{\jobname-pw.ind}{\input{\jobname-pw.ind}}{}

\end{document}

      