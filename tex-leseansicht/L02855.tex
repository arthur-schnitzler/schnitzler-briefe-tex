%% latex-leseansicht-vorspann.tex
%% Vorspann für die Leseansicht.
%% Lädt die gemeinsame Datei latex-vorspann.tex mit nicht gesetztem Schalter.

\newif\ifkorrekturansicht
\korrekturansichtfalse

\input{../tex-inputs/latex-vorspann}


         \renewcommand{\erwaehnteInstitutionen}{Institutionen: Graphische Gesellschaft (Berlin)}
         \renewcommand{\erwaehnteOrte}{Orte: Artillerielager Tsingtau, Berlin, China, Frankgasse, Kiautschou, Qingdao, Wien}
         \renewcommand{\erwaehnteWerke}{}
               \section[ Paul Goldmann an Arthur Schnitzler, 5. 8. 1898]{ Paul Goldmann an Arthur Schnitzler, 5. 8. 1898}\nopagebreak\mylabel{v}\rehead{ }\begin{ledgroupsized}[t]{13cm}\normalsize\beginnumbering \toendnotes[C]{\smallbreak\pagebreak[2]} \Standort{DLA, A:Schnitzler, HS.NZ85.1.3168.}
\physDesc{Bildpostkarte, 109 Zeichen
\newline{}Handschrift: 1) blaue Tinte, deutsche Kurrent\hspace{1em}2) blaue Tinte, lateinische Kurrent (\noindent{}Adresse)\hspace{1em}
\newline{}Versand: 1) Stempel: »\nobreak{}\oindex{Qingdao@\textbf{Qingdao}|pwk}Tsintau China, 5/8 98\nobreak{}«.   2) Stempel: »\nobreak{}Wien 9/3 72, 19. 9. 98, \textcolor{gray}{9}. V, Bestellt\nobreak{}«. 
\newline{}Schnitzler: mit Bleistift das Jahr »98« vermerkt }\pstart{}{\pb}Herrn\pend{}\pstart{}Dr. Arthur Schnitzler\pend{}\pstart{}Wien\oindex{Wien@\textbf{Wien}|pw}\pend{}\pstart{}IX. Frankgaſse 1\oindex{Frankgasse@\textbf{Frankgasse}|pw}.\pend{}{\bigskip}\pstart
           \noindent{}\centering{}{\pb}\textcolor{gray}{\textbf{Tsingtau: Hauptthor des
                     Artillerielagers\oindex{Artillerielager Tsingtau@\textbf{Artillerielager Tsingtau}|pw}.}}\pend
           \pstart
           Herzlichſten Gruß aus \textsc{Kiautschou\oindex{Kiautschou@\textbf{Kiautschou}|pw}}! \spacefill\mbox{Paul Goldmann}\pend
           \pstart
           \textsc{Tsintau\oindex{Qingdao@\textbf{Qingdao}|pw}}, 5. Auguſt\pend
           \pstart
           \centering{}\textcolor{gray}{\textbf{Aufnahme in China\oindex{China@\textbf{China}|pw} u.
                     Ausführung: Graph. Gesellschaft\orgindex{Graphische Gesellschaft (Berlin)@Graphische Gesellschaft (Berlin)|pw}, Berlin\oindex{Berlin@\textbf{Berlin}|pw}{ }1898.}}\pend
           
         
         \endnumbering\mylabel{h}\end{ledgroupsized}  \newcommand{\dateiname}{L02855}\newcommand{\titel}{Paul Goldmann an Arthur Schnitzler, 5. 8. 1898}\newcommand{\editorInnen}{Martin Anton Müller und Laura Untner}%% latex-leseansicht-abspann.tex
%% Abspann für die Leseansicht.
%% Der Schalter \ifkorrekturansicht ist bereits durch den Vorspann gesetzt.

%% latex-abspann.tex
%% Gemeinsamer Abspann für Korrekturansicht und Leseansicht.
%% Setzt den Schalter \ifkorrekturansicht voraus (gesetzt in den
%% einbindenden Dateien latex-korrekturansicht-abspann.tex bzw.
%% latex-leseansicht-abspann.tex).
%% ---------------------------------------------------------------

\normalsize

% Das esempio-Environment wird nur in der Leseansicht benötigt
\ifkorrekturansicht\else
\newenvironment{esempio}[3]%
{
    \vspace{1.5ex}
    \rlap{\underline{#1}}
    \par
    \setlength{\parindent}{0cm}
    \nopagebreak
    \leftskip=#2cm
    \rightskip=#3cm
}
{
    \par
}
\fi

\doendnotes{C}
\bigskip
\vfill

\clearpage

\footnotesize

\ifkorrekturansicht
  \lohead{\textsc{register}}
\fi

% theindex-Environment neu definieren ohne reledmac
\makeatletter
\renewenvironment{theindex}{%
  \ifkorrekturansicht
    \section*{\indexname}%
  \else
    \subsubsection*{Index der erwähnten Entitäten}%
  \fi
  \setlength{\parindent}{0pt}%
  \setlength{\parskip}{0pt plus 0.3pt}%
  \let\item\@idxitem
}{%
  \ifkorrekturansicht\clearpage\fi
}
\makeatother

\IfFileExists{\jobname-pw.ind}{\input{\jobname-pw.ind}}{}

% Quellenangabe nur in der Leseansicht
\ifkorrekturansicht\else
% Fallback-Definitionen, falls die .tex-Datei \titel etc. nicht gesetzt hat
\providecommand{\titel}{}
\providecommand{\editorInnen}{}
\providecommand{\dateiname}{\jobname}

\vspace{3cm}

\vfill

\footnotesize
\textsc{Quelle}: \titel. Herausgegeben von {\editorInnen}. In: \emph{Arthur Schnitzler: Briefwechsel mit Autorinnen und Autoren}.
 Digitale Edition, https://schnitzler-briefe.acdh.oeaw.ac.at/{\dateiname}.html (Stand \today)
\fi

\end{document}


      