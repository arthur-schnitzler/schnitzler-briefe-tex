%% latex-korrekturansicht-vorspann.tex
%% Vorspann für die Korrekturansicht.
%% Lädt die gemeinsame Datei latex-vorspann.tex mit gesetztem Schalter.

\newif\ifkorrekturansicht
\korrekturansichttrue

\input{../tex-inputs/latex-vorspann}


\section[ Paul Goldmann an Arthur Schnitzler, 5. 8. 1898]{L02855 Paul Goldmann an Arthur Schnitzler, 5. 8. 1898}
\nopagebreak\mylabel{L02855v}
\rehead{ }\normalsize\beginnumbering\briefempfaengerindex{Schnitzler, Arthur@\textsc{Schnitzler, Arthur}!zzzGoldmann, Paul@\emph{von Paul Goldmann}!1898-08-052@{5. 8. 1898}|(be}
\toendnotes[C]{\smallbreak\pagebreak[2]}\Standort{DLA, A:Schnitzler, HS.NZ85.1.3168.}
\physDesc{Bildpostkarte, 109 Zeichen
\newline{}Handschrift: 1) blaue Tinte, deutsche Kurrent\hspace{1em}2) blaue Tinte, lateinische Kurrent (\noindent{}Adresse)\hspace{1em}
\newline{}Versand: 1) Stempel: »\nobreak{}\oindex{Qingdao@\textbf{Qingdao}, \emph{Besiedelter Ort (A.BSO)}|pwk}Tsintau China, 5/8 98\nobreak{}«.   2) Stempel: »\nobreak{}\oindex{IX., Alsergrund@\textbf{IX., Alsergrund}, \emph{A.ADM3}|pwk}Wien 9/3 72, 19. 9. 98, \textcolor{gray}{9}. V, Bestellt\nobreak{}«. 
\newline{}Schnitzler: mit Bleistift das Jahr »98« vermerkt }\pstart{}{\pb}Herrn\pend{}\pstart{}Dr. Arthur Schnitzler\pend{}\pstart{}Wien\oindex{Wien@\textbf{Wien}, \emph{A.ADM2}|pw}\pend{}\pstart{}IX. Frankgaſse 1\oindex{Frankgasse 1@\textbf{Frankgasse 1}, \emph{Wohngebäude (K.WHS)}|pw}.\pend{}{\bigskip}
\pstart
           \noindent{}\centering{}{\pb}\textcolor{gray}{\textbf{Tsingtau: Hauptthor des
                        Artillerielagers\oindex{Artillerielager Tsingtau@\textbf{Artillerielager Tsingtau}, \emph{Gebäude (K.GBD)}|pw}.}}\pend
           \vspace{1em}
\pstart
           \noindent{}{\pb}Herzlichſten Gruß aus \textsc{Kiautschou\oindex{Kiautschou@\textbf{Kiautschou}, \emph{Region}|pw}}! \spacefill\mbox{Paul Goldmann}\pend
           
\pstart
           \textsc{Tsintau\oindex{Qingdao@\textbf{Qingdao}, \emph{Besiedelter Ort (A.BSO)}|pw}}, 5. Auguſt\pend
           
\pstart
           \centering{}\textcolor{gray}{\textbf{Aufnahme in China\oindex{China@\textbf{China}, \emph{A.PCLI}|pw} u.
                     Ausführung: Graph. Gesellschaft\orgindex{Graphische Gesellschaft (Berlin)@Graphische Gesellschaft (Berlin)|pw}, Berlin\oindex{Berlin@\textbf{Berlin}, \emph{P.PPLC}|pw}{ }1898.}}\pend
           \selectlanguage{ngerman}\endnumbering\briefempfaengerindex{Schnitzler, Arthur@\textsc{Schnitzler, Arthur}!zzzGoldmann, Paul@\emph{von Paul Goldmann}!1898-08-052@{5. 8. 1898}|)be}\mylabel{L02855h}  \normalsize

\doendnotes{C}
\bigskip
\vfill

\clearpage

\footnotesize

\lohead{\textsc{register}}

% Definiere theindex-Environment komplett neu ohne reledmac
\makeatletter
\renewenvironment{theindex}{%
  \section*{\indexname}%
  \setlength{\parindent}{0pt}%
  \setlength{\parskip}{0pt plus 0.3pt}%
  \let\item\@idxitem
}{%
  \clearpage
}
\makeatother

\IfFileExists{\jobname-pw.ind}{\input{\jobname-pw.ind}}{}

\end{document}

      