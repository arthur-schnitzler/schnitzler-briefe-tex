%% latex-leseansicht-vorspann.tex
%% Vorspann für die Leseansicht.
%% Lädt die gemeinsame Datei latex-vorspann.tex mit nicht gesetztem Schalter.

\newif\ifkorrekturansicht
\korrekturansichtfalse

\input{../tex-inputs/latex-vorspann}


\section[Arthur Schnitzler an Felix Salten, 8. 10. 1892]{L02957 Arthur Schnitzler an Felix Salten, 8. 10. 1892}
\nopagebreak\mylabel{L02957v}
\rehead{ }\normalsize\beginnumbering\briefempfaengerindex{Salten, Felix@\textsc{Salten, Felix}!zzzSchnitzler, Arthur@\emph{von Arthur Schnitzler}!1892-10-081@{8. 10. 1892}|(be}
\toendnotes[C]{\smallbreak\pagebreak[2]}
\correspDesc{Versand  durch Arthur Schnitzler am 8. 10. 1892 in Wien
\newline{}Erhalt  durch Felix Salten am 8. 10. 1892 in Wien}\toendnotes[C]{\smallbreak}
\Standort{Wienbibliothek im Rathaus, ZPH 1681, 2.1.516.}
\physDesc{Kartenbrief, 200 Zeichen
\newline{}Handschrift: schwarze Tinte, deutsche Kurrent
\newline{}Versand: 1) Stempel: »\nobreak{}\oindex{I., Innere Stadt@\textbf{I., Innere Stadt}, \emph{Verwaltungsgebiet}|pwk}Wien 1/1 1, 8. 10. \textcolor{gray}{9}2, 2–3N\nobreak{}«.   2) Stempel: »\nobreak{}\oindex{IX., Alsergrund@\textbf{IX., Alsergrund}, \emph{Verwaltungsgebiet}|pwk}Wien 9/1 66, 8. 10. 92, 5N, Bestellt\nobreak{}«. 
\newline{}Ordnung: mit Bleistift von unbekannter Hand nummeriert:
                                    »86« }\toendnotes[C]{\smallbreak}\pstart{}{\pb}Herrn \textsc{Felix
                     Salten}\pend{}\pstart{}\textsc{Wien\oindex{Wien@\textbf{Wien}, \emph{Verwaltungsgebiet}|pw}}\pend{}\pstart{}\textsc{IX Berggasse 13\oindex{Wien@\textbf{Wien}!IX., Alsergrund@\textbf{IX., Alsergrund}!Berggasse@\textbf{Berggasse}, \emph{Straße}|pw}}\pend{}{\bigskip}\vspace{1em}
\pstart{}{\pb}Lieber Salten,\pend\vspace{0.5em}
\pstart
           morgen So{\geminationn}tag{ }Nachmittag \uuline{4 Uhr}{ }ſind \textsc{BHofm\pwindex{Beer-Hofmann, Richard 11.\,7.\,1866 Wien – 26.\,9.\,1945 New York City@\textsc{Beer-Hofmann, Richard} (11.\,7.\,1866 Wien – 26.\,9.\,1945 New York City), \emph{Schriftsteller}|pw}} u ich \textsc{Stefansplatz\oindex{Wien@\textbf{Wien}!I., Innere Stadt@\textbf{I., Innere Stadt}!Stephansplatz@\textbf{Stephansplatz}, \emph{Platz}|pw}}, wollen in die \label{K_L02957-1v}\edtext{Ausſtellung\orgindex{Internationale Ausstellung für Musik und Theaterwesen@Internationale Ausstellung für Musik und Theaterwesen|pw}}{\lemma{\textnormal{\emph{Ausstellung}}}\Cendnote{\textnormal{die \emph{Internationale Ausstellung für Musik und Theaterwesen}\orgindex{Internationale Ausstellung für Musik und Theaterwesen@Internationale Ausstellung für Musik und Theaterwesen|pwk} im
                     Wien\oindex{Wien@\textbf{Wien}, \emph{Verwaltungsgebiet}|pwk}er Prater\oindex{Wien@\textbf{Wien}!II., Leopoldstadt@\textbf{II., Leopoldstadt}!Prater@\textbf{Prater}, \emph{Park}|pwk}}}}\label{K_L02957-1}. – Ich{ }ſchrieb auch an \textsc{Torresani\pwindex{Torresani-Lanzenfeld, Carl von 19.\,4.\,1846 Mailand – 16.\,4.\,1907 Nago-Torbole@\textsc{Torresani-Lanzenfeld, Carl von} (19.\,4.\,1846 Mailand – 16.\,4.\,1907 Nago-Torbole), \emph{Schriftsteller, Offizier}|pw}}. Ko{\geminationm}en Sie doch auch!\pend
           \pstart Herzlich \spacefill\mbox{ArthSchn}\pend{}\selectlanguage{ngerman}\endnumbering\briefempfaengerindex{Salten, Felix@\textsc{Salten, Felix}!zzzSchnitzler, Arthur@\emph{von Arthur Schnitzler}!1892-10-081@{8. 10. 1892}|)be}\mylabel{L02957h}  \newcommand{\dateiname}{L02957}\newcommand{\titel}{Arthur Schnitzler an Felix Salten, 8. 10. 1892}\newcommand{\editorInnen}{Martin Anton Müller und Laura Untner}%% latex-leseansicht-abspann.tex
%% Abspann für die Leseansicht.
%% Der Schalter \ifkorrekturansicht ist bereits durch den Vorspann gesetzt.

%% latex-abspann.tex
%% Gemeinsamer Abspann für Korrekturansicht und Leseansicht.
%% Setzt den Schalter \ifkorrekturansicht voraus (gesetzt in den
%% einbindenden Dateien latex-korrekturansicht-abspann.tex bzw.
%% latex-leseansicht-abspann.tex).
%% ---------------------------------------------------------------

\normalsize

% Das esempio-Environment wird nur in der Leseansicht benötigt
\ifkorrekturansicht\else
\newenvironment{esempio}[3]%
{
    \vspace{1.5ex}
    \rlap{\underline{#1}}
    \par
    \setlength{\parindent}{0cm}
    \nopagebreak
    \leftskip=#2cm
    \rightskip=#3cm
}
{
    \par
}
\fi

\doendnotes{C}
\bigskip
\vfill

\clearpage

\footnotesize

\ifkorrekturansicht
  \lohead{\textsc{register}}
\fi

% theindex-Environment neu definieren ohne reledmac
\makeatletter
\renewenvironment{theindex}{%
  \ifkorrekturansicht
    \section*{\indexname}%
  \else
    \subsubsection*{Index der erwähnten Entitäten}%
  \fi
  \setlength{\parindent}{0pt}%
  \setlength{\parskip}{0pt plus 0.3pt}%
  \let\item\@idxitem
}{%
  \ifkorrekturansicht\clearpage\fi
}
\makeatother

\IfFileExists{\jobname-pw.ind}{\input{\jobname-pw.ind}}{}

% Quellenangabe nur in der Leseansicht
\ifkorrekturansicht\else
% Fallback-Definitionen, falls die .tex-Datei \titel etc. nicht gesetzt hat
\providecommand{\titel}{}
\providecommand{\editorInnen}{}
\providecommand{\dateiname}{\jobname}

\vspace{3cm}

\vfill

\footnotesize
\textsc{Quelle}: \titel. Herausgegeben von {\editorInnen}. In: \emph{Arthur Schnitzler: Briefwechsel mit Autorinnen und Autoren}.
 Digitale Edition, https://schnitzler-briefe.acdh.oeaw.ac.at/{\dateiname}.html (Stand \today)
\fi

\end{document}


