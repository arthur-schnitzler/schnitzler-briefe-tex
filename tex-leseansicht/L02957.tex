%% latex-korrekturansicht-vorspann.tex
%% Vorspann für die Korrekturansicht.
%% Lädt die gemeinsame Datei latex-vorspann.tex mit gesetztem Schalter.

\newif\ifkorrekturansicht
\korrekturansichttrue

\input{../tex-inputs/latex-vorspann}


\section[Arthur Schnitzler an Felix Salten, 8. 10. 1892]{L02957 Arthur Schnitzler an Felix Salten, 8. 10. 1892}
\nopagebreak\mylabel{L02957v}
\rehead{ }\normalsize\beginnumbering\briefempfaengerindex{Salten, Felix@\textsc{Salten, Felix}!zzzSchnitzler, Arthur@\emph{von Arthur Schnitzler}!1892-10-081@{8. 10. 1892}|(be}
\toendnotes[C]{\smallbreak\pagebreak[2]}\Standort{Wienbibliothek im Rathaus, ZPH 1681, 2.1.516.}
\physDesc{Kartenbrief, 200 Zeichen
\newline{}Handschrift: schwarze Tinte, deutsche Kurrent
\newline{}Versand: 1) Stempel: »\nobreak{}\oindex{I., Innere Stadt@\textbf{I., Innere Stadt}, \emph{A.ADM3}|pwk}Wien 1/1 1, 8. 10. \textcolor{gray}{9}2, 2–3N\nobreak{}«.   2) Stempel: »\nobreak{}\oindex{IX., Alsergrund@\textbf{IX., Alsergrund}, \emph{A.ADM3}|pwk}Wien 9/1 66, 8. 10. 92, 5N, Bestellt\nobreak{}«. 
\newline{}Ordnung: mit Bleistift von unbekannter Hand nummeriert:
                                    »86« }\toendnotes[C]{\smallbreak}\pstart{}{\pb}Herrn \textsc{Felix
                     Salten}\pend{}\pstart{}\textsc{Wien\oindex{Wien@\textbf{Wien}, \emph{A.ADM2}|pw}}\pend{}\pstart{}\textsc{IX Berggasse 13\oindex{Berggasse@\textbf{Berggasse}, \emph{Straße (K.STR)}|pw}}\pend{}{\bigskip}\vspace{1em}
\pstart{}{\pb}Lieber Salten,\pend\vspace{0.5em}
\pstart
           morgen So{\geminationn}tag{ }Nachmittag \uuline{4 Uhr} ſind \textsc{BHofm\pwindex{Beer-Hofmann, Richard 1866-07-11 – 1945-09-26@\textsc{Beer-Hofmann, Richard} (1866-07-11 – 1945-09-26), \emph{Schriftsteller/Schriftstellerin}|pw}} u ich \textsc{Stefansplatz\oindex{Stephansplatz@\textbf{Stephansplatz}, \emph{S.SQR}|pw}}, wollen in die \label{K_L02957-1v}\edtext{Ausſtellung\orgindex{Internationale Ausstellung fuer Musik und Theaterwesen@Internationale Ausstellung für Musik und Theaterwesen|pw}}{\lemma{\textnormal{\emph{Ausſtellung}}}\Cendnote{\textnormal{die \emph{Internationale Ausstellung für Musik und Theaterwesen}\orgindex{Internationale Ausstellung fuer Musik und Theaterwesen@Internationale Ausstellung für Musik und Theaterwesen|pwk} im
                     Wien\oindex{Wien@\textbf{Wien}, \emph{A.ADM2}|pwk}er Prater\oindex{Prater@\textbf{Prater}, \emph{Park (K.PRK)}|pwk}}}}\label{K_L02957-1}. – Ich ſchrieb auch an \textsc{Torresani\pwindex{Torresani-Lanzenfeld, Carl von 19.04.1846 – 16.04.1907@\textsc{Torresani-Lanzenfeld, Carl von} (19.04.1846 – 16.04.1907), \emph{Schriftsteller/Schriftstellerin, Offizier/Offizierin}|pw}}. Ko{\geminationm}en Sie doch auch!
            \pend
           \pstart Herzlich \spacefill\mbox{ArthSchn}\pend{}\selectlanguage{ngerman}\endnumbering\briefempfaengerindex{Salten, Felix@\textsc{Salten, Felix}!zzzSchnitzler, Arthur@\emph{von Arthur Schnitzler}!1892-10-081@{8. 10. 1892}|)be}\mylabel{L02957h}  \normalsize

\doendnotes{C}
\bigskip
\vfill

\clearpage

\footnotesize

\lohead{\textsc{register}}

% Definiere theindex-Environment komplett neu ohne reledmac
\makeatletter
\renewenvironment{theindex}{%
  \section*{\indexname}%
  \setlength{\parindent}{0pt}%
  \setlength{\parskip}{0pt plus 0.3pt}%
  \let\item\@idxitem
}{%
  \clearpage
}
\makeatother

\IfFileExists{\jobname-pw.ind}{\input{\jobname-pw.ind}}{}

\end{document}

      