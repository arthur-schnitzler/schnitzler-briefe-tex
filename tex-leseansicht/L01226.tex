%% latex-leseansicht-vorspann.tex
%% Vorspann für die Leseansicht.
%% Lädt die gemeinsame Datei latex-vorspann.tex mit nicht gesetztem Schalter.

\newif\ifkorrekturansicht
\korrekturansichtfalse

\input{../tex-inputs/latex-vorspann}


\section[Hugo von Hofmannsthal und Arthur Schnitzler an Richard Beer-Hofmann, 30. 6. 1902]{L01226 Hugo von Hofmannsthal und Arthur Schnitzler an Richard Beer-Hofmann, 30. 6. 1902}
\nopagebreak\mylabel{L01226v}
\rehead{ }\normalsize\beginnumbering\briefempfaengerindex{Beer-Hofmann, Richard@\textsc{Beer-Hofmann, Richard}!zzzSchnitzler, Arthur@\emph{von Arthur Schnitzler}!1902-06-301@{30. 6. 1902}|(be}\briefempfaengerindex{Beer-Hofmann, Richard@\textsc{Beer-Hofmann, Richard}!zzzHofmannsthal, Hugo von@\emph{von Hugo von Hofmannsthal}!1902-06-301@{30. 6. 1902}|(be}
\toendnotes[C]{\smallbreak\pagebreak[2]}
\correspDesc{Versand  durch Hugo von Hofmannsthal, Arthur Schnitzler am 30. 6. 1902 in Salzburg
\newline{}Erhalt  durch Richard Beer-Hofmann am 1. 7. 1902 in Rodaun}\toendnotes[C]{\smallbreak}
\Standort{YCGL, MSS 31.}
\physDesc{Bildpostkarte, 83 Zeichen
\newline{}Handschrift Hugo von Hofmannsthal: blauer Buntstift, lateinische Kurrent
\newline{}Handschrift Arthur Schnitzler: Bleistift, deutsche Kurrent
\newline{}Versand: 1) Stempel: »\nobreak{}\oindex{Hauptbahnhof Salzburg@\textbf{Hauptbahnhof Salzburg}, \emph{Bahnhofsgebäude}|pwk}{[}Sal{]}zburg Bahn\textcolor{gray}{hof}\nobreak{}«.   2) Stempel: »\nobreak{}\oindex{Wien@\textbf{Wien}!XXIII., Liesing@\textbf{XXIII., Liesing}!Rodaun@\textbf{Rodaun}, \emph{Region}|pwk}Rodaun, {[}1.{]} 7. 02, 9–\textcolor{gray}{12}V\nobreak{}«. 
\newline{}Ordnung: mit Bleistift von unbekannter Hand datiert: »30. 6.« }\pstart{}{\pb}Herrn D\textsuperscript{r} Richard Beer
                  Hofmann \pend{}\pstart{}Rodaun beim Liesing\oindex{Wien@\textbf{Wien}!XXIII., Liesing@\textbf{XXIII., Liesing}!Rodaun@\textbf{Rodaun}, \emph{Region}|pw}\pend{}\pstart{}N. Ö.\oindex{Niederösterreich@\textbf{Niederösterreich}, \emph{Land}|pw}\pend{}{\bigskip}
\pstart
           \noindent{}\centering{}{\pb}\textcolor{gray}{\textbf{Salzburg\oindex{Salzburg@\textbf{Salzburg}, \emph{Verwaltungsgebiet}|pw}.}}\pend
           \vspace{1em}\pstart \spacefill\mbox{{\pb}Hugo}\pend{}\selectlanguage{ngerman}\vspace{1em}
\pstart
           {[}hs. Schnitzler:{]} Herzl Gruſs{\\[\baselineskip]}\spacefill\mbox{Arthur}\pend
           \leftskip=0em{}
\pstart
           30. 6. 902\pend
           \selectlanguage{ngerman}\endnumbering\briefempfaengerindex{Beer-Hofmann, Richard@\textsc{Beer-Hofmann, Richard}!zzzSchnitzler, Arthur@\emph{von Arthur Schnitzler}!1902-06-301@{30. 6. 1902}|)be}\briefempfaengerindex{Beer-Hofmann, Richard@\textsc{Beer-Hofmann, Richard}!zzzHofmannsthal, Hugo von@\emph{von Hugo von Hofmannsthal}!1902-06-301@{30. 6. 1902}|)be}\mylabel{L01226h}  \newcommand{\dateiname}{L01226}\newcommand{\titel}{Hugo von Hofmannsthal und Arthur Schnitzler an Richard Beer-Hofmann, 30. 6. 1902}\newcommand{\editorInnen}{Martin Anton Müller und Gerd-Hermann Susen}%% latex-leseansicht-abspann.tex
%% Abspann für die Leseansicht.
%% Der Schalter \ifkorrekturansicht ist bereits durch den Vorspann gesetzt.

%% latex-abspann.tex
%% Gemeinsamer Abspann für Korrekturansicht und Leseansicht.
%% Setzt den Schalter \ifkorrekturansicht voraus (gesetzt in den
%% einbindenden Dateien latex-korrekturansicht-abspann.tex bzw.
%% latex-leseansicht-abspann.tex).
%% ---------------------------------------------------------------

\normalsize

% Das esempio-Environment wird nur in der Leseansicht benötigt
\ifkorrekturansicht\else
\newenvironment{esempio}[3]%
{
    \vspace{1.5ex}
    \rlap{\underline{#1}}
    \par
    \setlength{\parindent}{0cm}
    \nopagebreak
    \leftskip=#2cm
    \rightskip=#3cm
}
{
    \par
}
\fi

\doendnotes{C}
\bigskip
\vfill

\clearpage

\footnotesize

\ifkorrekturansicht
  \lohead{\textsc{register}}
\fi

% theindex-Environment neu definieren ohne reledmac
\makeatletter
\renewenvironment{theindex}{%
  \ifkorrekturansicht
    \section*{\indexname}%
  \else
    \subsubsection*{Index der erwähnten Entitäten}%
  \fi
  \setlength{\parindent}{0pt}%
  \setlength{\parskip}{0pt plus 0.3pt}%
  \let\item\@idxitem
}{%
  \ifkorrekturansicht\clearpage\fi
}
\makeatother

\IfFileExists{\jobname-pw.ind}{\input{\jobname-pw.ind}}{}

% Quellenangabe nur in der Leseansicht
\ifkorrekturansicht\else
% Fallback-Definitionen, falls die .tex-Datei \titel etc. nicht gesetzt hat
\providecommand{\titel}{}
\providecommand{\editorInnen}{}
\providecommand{\dateiname}{\jobname}

\vspace{3cm}

\vfill

\footnotesize
\textsc{Quelle}: \titel. Herausgegeben von {\editorInnen}. In: \emph{Arthur Schnitzler: Briefwechsel mit Autorinnen und Autoren}.
 Digitale Edition, https://schnitzler-briefe.acdh.oeaw.ac.at/{\dateiname}.html (Stand \today)
\fi

\end{document}


