%% latex-leseansicht-vorspann.tex
%% Vorspann für die Leseansicht.
%% Lädt die gemeinsame Datei latex-vorspann.tex mit nicht gesetztem Schalter.

\newif\ifkorrekturansicht
\korrekturansichtfalse

\input{../tex-inputs/latex-vorspann}


\section[ Felix Salten an Arthur Schnitzler, {[}6. 9. 1894{]}]{L03144 Felix Salten an Arthur Schnitzler,  [6. 9. 1894]}
\nopagebreak\mylabel{L03144v}
\rehead{ }\normalsize\beginnumbering\briefempfaengerindex{Schnitzler, Arthur@\textsc{Schnitzler, Arthur}!zzzSalten, Felix@\emph{von Felix Salten}!1894-09-061@{{[}6. 9. 1894{]}}|(be}
\toendnotes[C]{\smallbreak\pagebreak[2]}
\correspDesc{Versand  durch Felix Salten am [6. 9. 1894] in Wien
\newline{}Erhalt  durch Arthur Schnitzler am [6. 9. 1894] in Wien}\toendnotes[C]{\smallbreak}
\Standort{CUL, Schnitzler, B 89, A 1.}
\physDesc{Visitenkarte, 147 Zeichen
\newline{}Handschrift: schwarze Tinte, lateinische Kurrent
\newline{}Schnitzler: mit Bleistift datiert: »6/9 94« 
\newline{}Ordnung: mit Bleistift von unbekannter Hand nummeriert: »45« }\toendnotes[C]{\smallbreak}
\pstart
           \centering{}{\pb}\textcolor{gray}{\textbf{FELIX SALTEN}}\pend
           
\pstart
           \textcolor{gray}{\textbf{WIEN\oindex{Wien@\textbf{Wien}, \emph{Verwaltungsgebiet}|pw},}}\hfill \textcolor{gray}{\textbf{»Berliner Neueste
                        Nachrichten\orgindex{Berliner Neueste Nachrichten@Berliner Neueste Nachrichten|pw}.«}}\pend
           
\pstart
           \textcolor{gray}{\textbf{IX., Hörlgasse 16\oindex{Wien@\textbf{Wien}!IX., Alsergrund@\textbf{IX., Alsergrund}!Hörlgasse 16@\textbf{Hörlgasse 16}, \emph{Wohngebäude}|pw}.}}\hfill \textcolor{gray}{\textbf{»Münchener
                        General-Anzeiger\orgindex{Münchener General-Anzeiger@Münchener General-Anzeiger|pw}.«}}\pend
           \vspace{0.5em}
\pstart
           {\pb}Bitte sehr, wenn Sie \label{K_L03144-1v}\edtext{heute noch hieher\oindex{Wien@\textbf{Wien}, \emph{Verwaltungsgebiet}|pwv} kommen}{\lemma{\textnormal{\emph{heute noch hieher kommen}}}\Cendnote{\textnormal{Vermutlich wusste Salten\pwindex{Salten, Felix 6.\,9.\,1869 Budapest – 8.\,10.\,1945 Zürich@\textsc{Salten, Felix} (6.\,9.\,1869 Budapest – 8.\,10.\,1945 Zürich), \emph{Schriftsteller, Journalist, Chefredakteur}|pwk} nicht, dass Schnitzler bereits am 4. 9. 1894 aus Ischl\oindex{Bad Ischl@\textbf{Bad Ischl}|pwk} zurückgekehrt war.}}}\label{K_L03144-1}, so kommen Sie
                  \uline{besti{\geminationm}t} auf einen
               Sprung in’s 
               Café Wortner\oindex{Wien@\textbf{Wien}!I., Innere Stadt@\textbf{I., Innere Stadt}!Café Kaiserhof (Inh. Johann Wortner) [Wien]@\textbf{Café Kaiserhof (Inh. Johann Wortner) [Wien]}, \emph{Kaffeehaus}|pw} (Kaiserhof\oindex{Wien@\textbf{Wien}!I., Innere Stadt@\textbf{I., Innere Stadt}!Café Kaiserhof (Inh. Johann Wortner) [Wien]@\textbf{Café Kaiserhof (Inh. Johann Wortner) [Wien]}, \emph{Kaffeehaus}|pw})\pend
           
\pstart
           Ich muss Sie notwendig sprechen\pend
           
\pstart
           Ihr {\\[\baselineskip]}\spacefill\mbox{Salten}\pend
           \leftskip=0em{}\selectlanguage{ngerman}\endnumbering\briefempfaengerindex{Schnitzler, Arthur@\textsc{Schnitzler, Arthur}!zzzSalten, Felix@\emph{von Felix Salten}!1894-09-061@{{[}6. 9. 1894{]}}|)be}\mylabel{L03144h}  \newcommand{\dateiname}{L03144}\newcommand{\titel}{Felix Salten an Arthur Schnitzler, [6. 9. 1894]}\newcommand{\editorInnen}{Martin Anton Müller und Laura Untner}%% latex-leseansicht-abspann.tex
%% Abspann für die Leseansicht.
%% Der Schalter \ifkorrekturansicht ist bereits durch den Vorspann gesetzt.

%% latex-abspann.tex
%% Gemeinsamer Abspann für Korrekturansicht und Leseansicht.
%% Setzt den Schalter \ifkorrekturansicht voraus (gesetzt in den
%% einbindenden Dateien latex-korrekturansicht-abspann.tex bzw.
%% latex-leseansicht-abspann.tex).
%% ---------------------------------------------------------------

\normalsize

% Das esempio-Environment wird nur in der Leseansicht benötigt
\ifkorrekturansicht\else
\newenvironment{esempio}[3]%
{
    \vspace{1.5ex}
    \rlap{\underline{#1}}
    \par
    \setlength{\parindent}{0cm}
    \nopagebreak
    \leftskip=#2cm
    \rightskip=#3cm
}
{
    \par
}
\fi

\doendnotes{C}
\bigskip
\vfill

\clearpage

\footnotesize

\ifkorrekturansicht
  \lohead{\textsc{register}}
\fi

% theindex-Environment neu definieren ohne reledmac
\makeatletter
\renewenvironment{theindex}{%
  \ifkorrekturansicht
    \section*{\indexname}%
  \else
    \subsubsection*{Index der erwähnten Entitäten}%
  \fi
  \setlength{\parindent}{0pt}%
  \setlength{\parskip}{0pt plus 0.3pt}%
  \let\item\@idxitem
}{%
  \ifkorrekturansicht\clearpage\fi
}
\makeatother

\IfFileExists{\jobname-pw.ind}{\input{\jobname-pw.ind}}{}

% Quellenangabe nur in der Leseansicht
\ifkorrekturansicht\else
% Fallback-Definitionen, falls die .tex-Datei \titel etc. nicht gesetzt hat
\providecommand{\titel}{}
\providecommand{\editorInnen}{}
\providecommand{\dateiname}{\jobname}

\vspace{3cm}

\vfill

\footnotesize
\textsc{Quelle}: \titel. Herausgegeben von {\editorInnen}. In: \emph{Arthur Schnitzler: Briefwechsel mit Autorinnen und Autoren}.
 Digitale Edition, https://schnitzler-briefe.acdh.oeaw.ac.at/{\dateiname}.html (Stand \today)
\fi

\end{document}


