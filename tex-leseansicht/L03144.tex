%% latex-korrekturansicht-vorspann.tex
%% Vorspann für die Korrekturansicht.
%% Lädt die gemeinsame Datei latex-vorspann.tex mit gesetztem Schalter.

\newif\ifkorrekturansicht
\korrekturansichttrue

\input{../tex-inputs/latex-vorspann}


\section[ Felix Salten an Arthur Schnitzler, {[}6. 9. 1894{]}]{L03144 Felix Salten an Arthur Schnitzler, {[}6. 9. 1894{]}}
\nopagebreak\mylabel{L03144v}
\rehead{ }\normalsize\beginnumbering\briefempfaengerindex{Schnitzler, Arthur@\textsc{Schnitzler, Arthur}!zzzSalten, Felix@\emph{von Felix Salten}!1894-09-061@{{[}6. 9. 1894{]}}|(be}
\toendnotes[C]{\smallbreak\pagebreak[2]}\Standort{CUL, Schnitzler, B 89, A 1.}
\physDesc{Visitenkarte, 147 Zeichen
\newline{}Handschrift: schwarze Tinte, lateinische Kurrent
\newline{}Schnitzler: mit Bleistift datiert: »6/9 94« 
\newline{}Ordnung: mit Bleistift von unbekannter Hand nummeriert: »45« }\toendnotes[C]{\smallbreak}
\pstart
           \centering{}{\pb}\textcolor{gray}{\textbf{FELIX SALTEN}}\pend
           
\pstart
           \textcolor{gray}{\textbf{WIEN\oindex{Wien@\textbf{Wien}, \emph{A.ADM2}|pw},}}\hfill \textcolor{gray}{\textbf{»Berliner Neueste
                        Nachrichten\orgindex{Berliner Neueste Nachrichten@Berliner Neueste Nachrichten|pw}.«}}\pend
           
\pstart
           \textcolor{gray}{\textbf{IX., Hörlgasse 16\oindex{Hoerlgasse 16@\textbf{Hörlgasse 16}, \emph{Wohngebäude (K.WHS)}|pw}.}}\hfill \textcolor{gray}{\textbf{»Münchener
                        General-Anzeiger\orgindex{Muenchener General-Anzeiger@Münchener General-Anzeiger|pw}.«}}\pend
           \vspace{0.5em}
\pstart
           {\pb}Bitte sehr, wenn Sie \label{K_L03144-1v}\edtext{heute noch hieher\oindex{Wien@\textbf{Wien}, \emph{A.ADM2}|pwv} kommen}{\lemma{\textnormal{\emph{heute noch hieher kommen}}}\Cendnote{\textnormal{Vermutlich wusste Salten\pwindex{Salten, Felix 06.09.1869 – 08.10.1945@\textsc{Salten, Felix} (06.09.1869 – 08.10.1945), \emph{Schriftsteller/Schriftstellerin, Journalist/Journalistin, Chefredakteur/Chefredakteurin}|pwk} nicht, dass Schnitzler bereits am 4. 9. 1894 aus Ischl\oindex{Bad Ischl@\textbf{Bad Ischl}, \emph{P.PPL}|pwk} zurückgekehrt war.}}}\label{K_L03144-1}, so kommen Sie
                  \uline{besti{\geminationm}t} auf einen
               Sprung in’s 
               Café Wortner\oindex{Cafe Kaiserhof (Inh. Johann Wortner) [Wien]@\textbf{Café Kaiserhof (Inh. Johann Wortner) [Wien]}, \emph{Kaffeehaus (K.KAF)}|pw} (Kaiserhof\oindex{Cafe Kaiserhof (Inh. Johann Wortner) [Wien]@\textbf{Café Kaiserhof (Inh. Johann Wortner) [Wien]}, \emph{Kaffeehaus (K.KAF)}|pw})\pend
           
\pstart
           Ich muss Sie notwendig sprechen\pend
           
\pstart
           Ihr {\\[\baselineskip]}\spacefill\mbox{Salten}\pend
           \leftskip=0em{}\selectlanguage{ngerman}\endnumbering\briefempfaengerindex{Schnitzler, Arthur@\textsc{Schnitzler, Arthur}!zzzSalten, Felix@\emph{von Felix Salten}!1894-09-061@{{[}6. 9. 1894{]}}|)be}\mylabel{L03144h}  \normalsize

\doendnotes{C}
\bigskip
\vfill

\clearpage

\footnotesize

\lohead{\textsc{register}}

% Definiere theindex-Environment komplett neu ohne reledmac
\makeatletter
\renewenvironment{theindex}{%
  \section*{\indexname}%
  \setlength{\parindent}{0pt}%
  \setlength{\parskip}{0pt plus 0.3pt}%
  \let\item\@idxitem
}{%
  \clearpage
}
\makeatother

\IfFileExists{\jobname-pw.ind}{\input{\jobname-pw.ind}}{}

\end{document}

      