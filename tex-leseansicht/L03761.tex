%% latex-korrekturansicht-vorspann.tex
%% Vorspann für die Korrekturansicht.
%% Lädt die gemeinsame Datei latex-vorspann.tex mit gesetztem Schalter.

\newif\ifkorrekturansicht
\korrekturansichttrue

\input{../tex-inputs/latex-vorspann}


\section[Arthur Schnitzler an Stefan Zweig, 25. 4. 1915]{L03761 Arthur Schnitzler an Stefan Zweig, 25. 4. 1915}
\nopagebreak\mylabel{L03761v}
\rehead{ }\normalsize\beginnumbering\briefempfaengerindex{Zweig, Stefan@\textsc{Zweig, Stefan}!zzzSchnitzler, Arthur@\emph{von Arthur Schnitzler}!1915-04-251@{25. 4. 1915}|(be}
\toendnotes[C]{\smallbreak\pagebreak[2]}\Standort{Jerusalem, National Library of Israel, ARC. Ms. Var. 305 1 58 Stefan Zweig Collection.}
\physDesc{Briefkarte, 1 Blatt, 2 Seiten, 306 Zeichen
\newline{}Handschrift: schwarze Tinte, lateinische Kurrent}\toendnotes[C]{\smallbreak}
\pstart
           {\pb}\textcolor{gray}{\textbf{Dr. Arthur
                        Schnitzler}}\hfill 21. 4. 915\pend
           
\pstart
           \textcolor{gray}{\textbf{Wien XVIII.
                        Sternwartestrasse 71\oindex{Sternwartestrasse 71@\textbf{Sternwartestraße 71}, \emph{Wohngebäude (K.WHS)}|pw}}}\pend
           
\pstart{}lieber Herr Doktor,\pend\vspace{0.5em}
\pstart
           für Ihre wunderbaren Worte zu \label{K_L03768-1v}\edtext{Gustav Mahlers\pwindex{Mahler, Gustav 07.07.1860 – 18.05.1911@\textsc{Mahler, Gustav} (07.07.1860 – 18.05.1911), \emph{Theaterleiter/Theaterleiterin, Komponist/Komponistin, Dirigent/Dirigentin}|pw}
                  Wiederkehr\pwindex{Gustav Mahlers Wiederkehr@\emph{Gustav Mahlers Wiederkehr}|pw}}{\lemma{\textnormal{\emph{Gustav … Wiederkehr}}}\Cendnote{\textnormal{Stefan Zweig\pwindex{Zweig, Stefan 28.11.1881 – 23.02.1942@\textsc{Zweig, Stefan} (28.11.1881 – 23.02.1942), \emph{Schriftsteller/Schriftstellerin}|pwk}: \emph{Gustav Mahlers Wiederkehr}\pwindex{Gustav Mahlers Wiederkehr@\emph{Gustav Mahlers Wiederkehr}|pwk}. In: \emph{Neue Freie Presse}\pwindex{Neue Freie Presse@\emph{Neue Freie Presse}|pwk}, Nr. 18.201,
                        25. 4. 1915, Morgenblatt, S. 1–4.}}}\label{K_L03768-1}
               laſſen Sie mich Ihnen von ganzem Herzen die Hand drücken. Sie haben mich im innerſten
               bewegt, und meiner Frau\pwindex{Schnitzler, Olga 17.01.1882 – 13.01.1970@\textsc{Schnitzler, Olga} (17.01.1882 – 13.01.1970), \emph{Schauspieler/Schauspielerin, Sänger/Sängerin}|pwv} iſt
               es gerade ſo ergangen. Wir danken ſchön und hoffen Sie bald wiederzuſehen.\pend
           \pstart Wie immer der Ihrige, \spacefill\mbox{Arthur Schnitzer}\pend{}\selectlanguage{ngerman}\endnumbering\briefempfaengerindex{Zweig, Stefan@\textsc{Zweig, Stefan}!zzzSchnitzler, Arthur@\emph{von Arthur Schnitzler}!1915-04-251@{25. 4. 1915}|)be}\mylabel{L03761h}
\begin{anhang}
\end{anhang}\normalsize

\doendnotes{C}
\bigskip
\vfill

\clearpage

\footnotesize

\lohead{\textsc{register}}

% Definiere theindex-Environment komplett neu ohne reledmac
\makeatletter
\renewenvironment{theindex}{%
  \section*{\indexname}%
  \setlength{\parindent}{0pt}%
  \setlength{\parskip}{0pt plus 0.3pt}%
  \let\item\@idxitem
}{%
  \clearpage
}
\makeatother

\IfFileExists{\jobname-pw.ind}{\input{\jobname-pw.ind}}{}

\end{document}

      