%% latex-leseansicht-vorspann.tex
%% Vorspann für die Leseansicht.
%% Lädt die gemeinsame Datei latex-vorspann.tex mit nicht gesetztem Schalter.

\newif\ifkorrekturansicht
\korrekturansichtfalse

\input{../tex-inputs/latex-vorspann}


\section[Max Burckhard: Widmungsexemplar Simon Thums für Arthur Schnitzler, {[}2. 10. 1897?{]}]{L00727 Max Burckhard: Widmungsexemplar Simon Thums für Arthur Schnitzler, {[}2. 10. 1897?{]}}
\nopagebreak\mylabel{L00727v}
\rehead{ }\normalsize\beginnumbering\briefempfaengerindex{Schnitzler, Arthur@\textsc{Schnitzler, Arthur}!zzzBurckhard, Max Eugen@\emph{von Max Eugen Burckhard}!1897-10-022@{{[}2. 10. 1897?{]}}|(be}
\toendnotes[C]{\smallbreak\pagebreak[2]}
\correspDesc{Versand  durch Max Burckhard am [2. 10. 1897?] \textbf{Ort fehlend} 
\newline{}Erhalt  durch Arthur Schnitzler im Zeitraum [2. 10. 1897
                  – 6. 10. 1897?] in Wien}\toendnotes[C]{\smallbreak}
\Standort{DLA, G:Schnitzler, Arthur (Sammlung Heinrich Schnitzler).}
\physDesc{Widmung am Schmutztitel, 54 Zeichen
\newline{}Handschrift: schwarze Tinte, deutsche Kurrent
\newline{}Ordnung: bei der Enteignung des Exemplars 1938 von
                                 unbekannter Hand mit Bleistift ergänzte Information:
                                    »= 93216-B« }\toendnotes[C]{\smallbreak}
\pstart
           \noindent{}{\pb}H D\textsuperscript{r}
               Schnitzler\pend
           
\pstart
           in aufrichtiger Verehrung{\\[\baselineskip]}\spacefill\mbox{D\textsuperscript{r} Burckhard}\pend
           \leftskip=0em{}{\vspace{1\baselineskip}}
\pstart
           \centering{}\textcolor{gray}{\textbf{Simon Thums\pwindex{Burckhard, Max Eugen 14.\,7.\,1854 Korneuburg – 16.\,3.\,1912 Wien@\textsc{Burckhard, Max Eugen} (14.\,7.\,1854 Korneuburg – 16.\,3.\,1912 Wien), \emph{Schriftsteller, Rechtswissenschaftler, Theaterleiter}!Simon Thums. Einige Tage aus seinem Leben@\strich\emph{Simon Thums. Einige Tage aus seinem Leben}|pw}.}}\pend
           \selectlanguage{ngerman}\vspace{1em}{\vspace{1\baselineskip}}
\pstart
           \centering{}{\pb}\textcolor{gray}{\textbf{Simon Thums\pwindex{Burckhard, Max Eugen 14.\,7.\,1854 Korneuburg – 16.\,3.\,1912 Wien@\textsc{Burckhard, Max Eugen} (14.\,7.\,1854 Korneuburg – 16.\,3.\,1912 Wien), \emph{Schriftsteller, Rechtswissenschaftler, Theaterleiter}!Simon Thums. Einige Tage aus seinem Leben@\strich\emph{Simon Thums. Einige Tage aus seinem Leben}|pw}.}}\pend
           
\pstart
           \centering{}\textcolor{gray}{\textbf{\so{Einige Tage aus{ }ſeinem Leben.}}}\pend
           
\pstart
           \centering{}\textcolor{gray}{\textbf{Von}}{\\}\textcolor{gray}{\textbf{Max Burckhard}}.\pend
           {\vspace{1\baselineskip}}
\pstart
           \centering{}\textcolor{gray}{\textbf{Stuttgart\oindex{Stuttgart@\textbf{Stuttgart}|pw}{ }\label{K_L00727-1v}\edtext{1897}{\lemma{\textnormal{\emph{1897}}}\Cendnote{\textnormal{am 2. 10. 1897 vom \emph{Börsenblatt für den deutschen Buchhandel}\pwindex{Börsenblatt für den Deutschen Buchhandel@\emph{Börsenblatt für den Deutschen Buchhandel}|pwk}
                     als Neuerscheinung gemeldet}}}\label{K_L00727-1}.}}\pend
           
\pstart
           \centering{}\textcolor{gray}{\textbf{\so{Verlag der J. G. Cotta’ſchen Buchhandlung}\orgindex{J.G. Cotta’sche Buchhandlung Nachfolger@J.G. Cotta’sche Buchhandlung Nachfolger|pw}}}{\\}\textcolor{gray}{\textbf{Nachfolger.}}\pend
           \selectlanguage{ngerman}\endnumbering\briefempfaengerindex{Schnitzler, Arthur@\textsc{Schnitzler, Arthur}!zzzBurckhard, Max Eugen@\emph{von Max Eugen Burckhard}!1897-10-022@{{[}2. 10. 1897?{]}}|)be}\mylabel{L00727h}  \newcommand{\dateiname}{L00727}\newcommand{\titel}{Max Burckhard: Widmungsexemplar Simon Thums für Arthur Schnitzler, [2. 10. 1897?]}\newcommand{\editorInnen}{Martin Anton Müller und Gerd-Hermann Susen}%% latex-leseansicht-abspann.tex
%% Abspann für die Leseansicht.
%% Der Schalter \ifkorrekturansicht ist bereits durch den Vorspann gesetzt.

%% latex-abspann.tex
%% Gemeinsamer Abspann für Korrekturansicht und Leseansicht.
%% Setzt den Schalter \ifkorrekturansicht voraus (gesetzt in den
%% einbindenden Dateien latex-korrekturansicht-abspann.tex bzw.
%% latex-leseansicht-abspann.tex).
%% ---------------------------------------------------------------

\normalsize

% Das esempio-Environment wird nur in der Leseansicht benötigt
\ifkorrekturansicht\else
\newenvironment{esempio}[3]%
{
    \vspace{1.5ex}
    \rlap{\underline{#1}}
    \par
    \setlength{\parindent}{0cm}
    \nopagebreak
    \leftskip=#2cm
    \rightskip=#3cm
}
{
    \par
}
\fi

\doendnotes{C}
\bigskip
\vfill

\clearpage

\footnotesize

\ifkorrekturansicht
  \lohead{\textsc{register}}
\fi

% theindex-Environment neu definieren ohne reledmac
\makeatletter
\renewenvironment{theindex}{%
  \ifkorrekturansicht
    \section*{\indexname}%
  \else
    \subsubsection*{Index der erwähnten Entitäten}%
  \fi
  \setlength{\parindent}{0pt}%
  \setlength{\parskip}{0pt plus 0.3pt}%
  \let\item\@idxitem
}{%
  \ifkorrekturansicht\clearpage\fi
}
\makeatother

\IfFileExists{\jobname-pw.ind}{\input{\jobname-pw.ind}}{}

% Quellenangabe nur in der Leseansicht
\ifkorrekturansicht\else
% Fallback-Definitionen, falls die .tex-Datei \titel etc. nicht gesetzt hat
\providecommand{\titel}{}
\providecommand{\editorInnen}{}
\providecommand{\dateiname}{\jobname}

\vspace{3cm}

\vfill

\footnotesize
\textsc{Quelle}: \titel. Herausgegeben von {\editorInnen}. In: \emph{Arthur Schnitzler: Briefwechsel mit Autorinnen und Autoren}.
 Digitale Edition, https://schnitzler-briefe.acdh.oeaw.ac.at/{\dateiname}.html (Stand \today)
\fi

\end{document}


