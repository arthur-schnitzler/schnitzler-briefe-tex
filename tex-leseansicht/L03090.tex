%% latex-korrekturansicht-vorspann.tex
%% Vorspann für die Korrekturansicht.
%% Lädt die gemeinsame Datei latex-vorspann.tex mit gesetztem Schalter.

\newif\ifkorrekturansicht
\korrekturansichttrue

\input{../tex-inputs/latex-vorspann}


\section[ Paul Goldmann an Arthur Schnitzler, 9. 11. {[}1901{]}]{L03090 Paul Goldmann an Arthur Schnitzler, 9. 11. {[}1901{]}}
\nopagebreak\mylabel{L03090v}
\rehead{ }\normalsize\beginnumbering\briefempfaengerindex{Schnitzler, Arthur@\textsc{Schnitzler, Arthur}!zzzGoldmann, Paul@\emph{von Paul Goldmann}!1901-11-092@{9. 11. {[}1901{]}}|(be}
\toendnotes[C]{\smallbreak\pagebreak[2]}\Standort{DLA, A:Schnitzler, HS.NZ85.1.3171.}
\physDesc{Brief, 1 Blatt, 4 Seiten, 1663 Zeichen
\newline{}Handschrift: blaue Tinte, deutsche Kurrent
\newline{}Schnitzler: 1) mit Bleistift das Jahr »1901« vermerkt  2) mit rotem Buntstift drei Unterstreichungen}\toendnotes[C]{\smallbreak}
\pstart
           \raggedleft{}{\pb}\textcolor{gray}{\textbf{DESSAUERSTRASSE 19}}\oindex{Dessauer Strasse@\textbf{Dessauer Straße}, \emph{Straße (K.STR)}|pw}\pend
           
\pstart
           Berlin\oindex{Berlin@\textbf{Berlin}, \emph{P.PPLC}|pw}, 9. November.\pend
           
\pstart\center{}Mein lieber Freund,\pend\vspace{0.5em}
\pstart
           Ich habe \introOben{}mich\introOben{} ſehr gefreut, endlich wieder einmal etwas von
               Dir zu hören. Daß die Aufführung Deiner Stücke\pwindex{Lebendige Stunden. Vier Einakter@\emph{Lebendige Stunden. Vier Einakter}|pwv} bis \label{K_L03090-1v}\edtext{Februar}{\lemma{\textnormal{\emph{Februar}}}\Cendnote{\textnormal{Grund war ein geplantes Gastspiel Irene Trieschs\pwindex{Triesch, Irene 13.04.1877 – 24.11.1964@\textsc{Triesch, Irene} (13.04.1877 – 24.11.1964), \emph{Schauspieler/Schauspielerin}|pwk}, die in den weiblichen
                  Hauptrollen von \emph{Lebendige Stunden}\pwindex{Lebendige Stunden. Vier Einakter@\emph{Lebendige Stunden. Vier Einakter}|pwk} auftrat
                     (vgl. \emph{Der Briefwechsel Arthur Schnitzler – Otto
                        Brahm}. Vollständige Ausgabe. Herausgegeben, eingeleitet und
                     erläutert von Oskar Seidlin. Tübingen:
                        \emph{Niemeyer}{ }1975, S. 102). Die Uraufführung konnte
                  schließlich noch vor Trieschs\pwindex{Triesch, Irene 13.04.1877 – 24.11.1964@\textsc{Triesch, Irene} (13.04.1877 – 24.11.1964), \emph{Schauspieler/Schauspielerin}|pwk} geplanter
                  Abwesenheit (Mitte Januar bis Mitte Februar 1902), am 4. 1. 1902,
                  stattfinden.}}}\label{K_L03090-1} verſchoben werden ſoll, iſt bedauerlich. Könnteſt Du nicht
               wenigſtens anderswo, in Hamburg\oindex{Hamburg@\textbf{Hamburg}, \emph{P.PPLA}|pw}, München\oindex{Muenchen@\textbf{München}, \emph{P.PPLA}|pw}, vielleicht gar in Wien\oindex{Wien@\textbf{Wien}, \emph{A.ADM2}|pw}, eine frühere Aufführung veranlaſſen \introOben{}damit Dir nicht
                  der Winter verloren geht\introOben{}? Die \textsc{Triesch\pwindex{Triesch, Irene 13.04.1877 – 24.11.1964@\textsc{Triesch, Irene} (13.04.1877 – 24.11.1964), \emph{Schauspieler/Schauspielerin}|pw}} wird hier von der kunſtunverſtändigen Kritik ſo {\pb}\label{K_L03090-2v}\edtext{wenig begriffen}{\lemma{\textnormal{\emph{wenig begriffen}}}\Cendnote{\textnormal{Siehe etwa F. M.\pwindex{Mauthner, Fritz 1849-11-20 – 1923-06-29@\textsc{Mauthner, Fritz} (1849-11-20 – 1923-06-29), \emph{Schriftsteller/Schriftstellerin, Journalist/Journalistin, Philosoph/Philosophin}|pwkv} [ = Fritz Mauthner\pwindex{Mauthner, Fritz 1849-11-20 – 1923-06-29@\textsc{Mauthner, Fritz} (1849-11-20 – 1923-06-29), \emph{Schriftsteller/Schriftstellerin, Journalist/Journalistin, Philosoph/Philosophin}|pwk}]: \emph{Hebbels »Maria Magdalena«. (Deutsches Theater)}\pwindex{Hebbels »Maria Magdalena«. (Deutsches Theater.)@\emph{Hebbels »Maria Magdalena«. (Deutsches Theater.)}|pwk}. In:
                        \emph{Berliner Tageblatt}\pwindex{Berliner Tageblatt@\emph{Berliner Tageblatt}|pwk}, Jg. 30, Nr. 565,
                        6. 11. 1901, S. [3].}}}\label{K_L03090-2}, daß es
               beinahe eine Gefahr für Deine Stücke\pwindex{Lebendige Stunden. Vier Einakter@\emph{Lebendige Stunden. Vier Einakter}|pwv} iſt, wenn ſie die \label{K_L03090-3v}\edtext{Hauptrolle}{\lemma{\textnormal{\emph{Hauptrolle}}}\Cendnote{\textnormal{Siehe Paul Goldmann an Arthur Schnitzler, 23. 9. [1901].
               }}}\label{K_L03090-3} ſpielt, die ſie natürlich herrlich ſpielen wird. Ich habe mit dieſer
               hyſteriſchen Jüdin, die mir unerträglich geworden iſt, alle Beziehungen
               abgebrochen.\pend
           
\pstart
           Daß \label{K_L03090-4v}\edtext{\textsc{Olga\pwindex{Schnitzler, Olga 17.01.1882 – 13.01.1970@\textsc{Schnitzler, Olga} (17.01.1882 – 13.01.1970), \emph{Schauspieler/Schauspielerin, Sänger/Sängerin}|pw}} krank}{\lemma{\textnormal{\emph{Olga krank}}}\Cendnote{\textnormal{Sie\pwindex{Schnitzler, Olga 17.01.1882 – 13.01.1970@\textsc{Schnitzler, Olga} (17.01.1882 – 13.01.1970), \emph{Schauspieler/Schauspielerin, Sänger/Sängerin}|pwkv} hatte Angina (vgl. A. S.: \emph{Tagebuch}, 25. 10. 1901).}}}\label{K_L03090-4} war,
               habe ich mit Bedauern vernommen. Was ihr gefehlt hat, habe ich, trotz langjähriger
               Kenntniß Deiner Handſchrift, nicht entziffern können. Immerhin freue ich mich, daß
               ſie wieder geſund iſt, und bitte Dich, ſie ſammt {\pb}der Schweſter\pwindex{Steinrueck, Elisabeth 19.11.1885 – 07.04.1920@\textsc{Steinrück, Elisabeth} (19.11.1885 – 07.04.1920)|pwv} zu
               grüßen.\pend
           
\pstart
           Was meine \label{K_L03090-5v}\edtext{Feuilletons\pwindex{Einsame Menschen. Drama@\emph{Einsame Menschen. Drama}|pwv}\pwindex{Berliner Brief. [»Schluck und Jau« von Gerhart Hauptmann am Deutschen Theater]@\emph{Berliner Brief. [»Schluck und Jau« von Gerhart Hauptmann am Deutschen Theater]}|pwv}\pwindex{Michael Kramer.«@\emph{»Michael Kramer.«}|pwv}}{\lemma{\textnormal{\emph{Feuilletons}}}\Cendnote{\textnormal{Paul Goldmann\pwindex{Goldmann, Paul 31.01.1865 – 25.09.1935@\textsc{Goldmann, Paul} (31.01.1865 – 25.09.1935), \emph{Schriftsteller/Schriftstellerin, Journalist/Journalistin}|pwk}: \emph{Berliner Brief}\pwindex{Berliner Brief. [»Schluck und Jau« von Gerhart Hauptmann am Deutschen Theater]@\emph{Berliner Brief. [»Schluck und Jau« von Gerhart Hauptmann am Deutschen Theater]}|pwk}. In: \emph{Neue Freie Presse}\pwindex{Neue Freie Presse@\emph{Neue Freie Presse}|pwk}, Nr. 12.735, 6. 2. 1900, Morgenblatt, S. 1–3; Paul Goldmann\pwindex{Goldmann, Paul 31.01.1865 – 25.09.1935@\textsc{Goldmann, Paul} (31.01.1865 – 25.09.1935), \emph{Schriftsteller/Schriftstellerin, Journalist/Journalistin}|pwk}: \emph{»Michael Kramer«}\pwindex{Michael Kramer.«@\emph{»Michael Kramer.«}|pwk}. In: \emph{Neue Freie Presse}\pwindex{Neue Freie Presse@\emph{Neue Freie Presse}|pwk}, Nr. 13.055, 28. 12. 1900, Morgenblatt, S. 1–3; Paul Goldmann\pwindex{Goldmann, Paul 31.01.1865 – 25.09.1935@\textsc{Goldmann, Paul} (31.01.1865 – 25.09.1935), \emph{Schriftsteller/Schriftstellerin, Journalist/Journalistin}|pwk}: \emph{Berliner Theater. »Einsame Menschen« im Deutschen Theater}\pwindex{Berliner Theater. »Einsame Menschen« im Deutschen Theater@\emph{Berliner Theater. »Einsame Menschen« im Deutschen Theater}|pwk}. In: \emph{Neue Freie Presse}\pwindex{Neue Freie Presse@\emph{Neue Freie Presse}|pwk}, Nr. 13.345, 19. 10. 1901, Morgenblatt, S. 1–3.}}}\label{K_L03090-5}
               über \textsc{Gerhart Hauptmann\pwindex{Hauptmann, Gerhart 15.11.1862 – 06.06.1946@\textsc{Hauptmann, Gerhart} (15.11.1862 – 06.06.1946), \emph{Schriftsteller/Schriftstellerin}|pw}} anlangt, ſo ſtimmen mir noch andere Leute zu, als Herr \textsc{Ebermann\pwindex{Ebermann, Leo 16.07.1863 – 09.10.1914@\textsc{Ebermann, Leo} (16.07.1863 – 09.10.1914), \emph{Schriftsteller/Schriftstellerin, Journalist/Journalistin, Rechtswissenschaftler/Rechtswissenschaftlerin}|pw}}. Im Übrigen wäre es mir ſehr gleichgiltig, auch wenn Niemand mir zuſtimmte, da
               ich weiß, daß ich Recht habe. Was Du über den \label{K_L03090-6v}\edtext{»Ton«}{\lemma{\textnormal{\emph{»Ton«}}}\Cendnote{\textnormal{Siehe A. S.: \emph{Tagebuch}, 27. 11. 1901 und Paul Goldmann an Arthur Schnitzler, 6. 12. [1901].
               }}}\label{K_L03090-6} ſchreibſt, verſtehe ich nicht. Das heißt, ich begreife nicht, wie Einer, der
               ſelbſt ſchreibt, dieſen Einwand erheben kann. Mein Ton bin nämlich ich ſelbſt. Aus
               dieſem Grunde wird es nicht leicht ſein, ihn zu ändern.\pend
           
\pstart
           Es thut mir unendlich leid, daß {\pb}durch den Aufſchub
               der Aufführung Deiner Stücke\pwindex{Lebendige Stunden. Vier Einakter@\emph{Lebendige Stunden. Vier Einakter}|pwv}{ }\strikeout{D\textcolor{gray}{ei}} auch Deine \label{K_L03090-7v}\edtext{Reiſe nach Berlin\oindex{Berlin@\textbf{Berlin}, \emph{P.PPLC}|pw} verſchoben}{\lemma{\textnormal{\emph{Reiſe … verſchoben}}}\Cendnote{\textnormal{Schnitzler war letztendlich vom 28. 12. 1901 bis zum 6. 1. 1902 in Berlin\oindex{Berlin@\textbf{Berlin}, \emph{P.PPLC}|pwk}.}}}\label{K_L03090-7} iſt.\pend
           
\pstart
           Haſt Du den \label{K_L03090-8v}\edtext{\textsc{Chamfort\pwindex{Œuvres choisies de N. Chamfort, publiees avec preface, notes et tables@\emph{Œuvres choisies de N. Chamfort, publiées avec préface, notes et tables}|pwv}\pwindex{Chamfort, Sebastien Roch Nicolas 06.04.1741 – 13.04.1794@\textsc{Chamfort, Sébastien Roch Nicolas} (06.04.1741 – 13.04.1794), \emph{Schriftsteller/Schriftstellerin}|pw}}}{\lemma{\textnormal{\emph{Chamfort}}}\Cendnote{\textnormal{Siehe Paul Goldmann an Arthur Schnitzler, 23. 9. [1901].
               }}}\label{K_L03090-8} nun endlich erhalten? Und haſt Du ihn geleſen? Lies’ auch die eben von \textsc{Griesebach\pwindex{Grisebach, Eduard 1845-10-09 – 1906-03-22@\textsc{Grisebach, Eduard} (1845-10-09 – 1906-03-22), \emph{Schriftsteller/Schriftstellerin, Diplomat/Diplomatin, Jurist/Juristin}|pw}} herausgegebenen \label{K_L03090-9v}\edtext{Geſpräche mit \textsc{Schopenhauer\pwindex{Schopenhauer, Arthur 22.02.1788 – 21.09.1860@\textsc{Schopenhauer, Arthur} (22.02.1788 – 21.09.1860), \emph{Philosoph/Philosophin}|pw}}\pwindex{Schopenhauer s Gespraeche und Selbstgespraeche: Nach der Handschrift eis heauton@\emph{Schopenhauer’s Gespräche und Selbstgespräche: Nach der Handschrift eis heauton}|pw}}{\lemma{\textnormal{\emph{Geſpräche mit Schopenhauer}}}\Cendnote{\textnormal{\emph{Schopenhauer’s Gespräche und Selbstgespräche:
                        Nach der Handschrift eis heauton}\pwindex{Schopenhauer s Gespraeche und Selbstgespraeche: Nach der Handschrift eis heauton@\emph{Schopenhauer’s Gespräche und Selbstgespräche: Nach der Handschrift eis heauton}|pwk}. Herausgegeben von Eduard Grisebach\pwindex{Grisebach, Eduard 1845-10-09 – 1906-03-22@\textsc{Grisebach, Eduard} (1845-10-09 – 1906-03-22), \emph{Schriftsteller/Schriftstellerin, Diplomat/Diplomatin, Jurist/Juristin}|pwk}. Berlin\oindex{Berlin@\textbf{Berlin}, \emph{P.PPLC}|pwk}: \emph{Ernst Hofmann {\kaufmannsund} Co.}\orgindex{Ernst Hofmann und Co.@Ernst Hofmann {\kaufmannsund}  Co.|pwk}{ }1898. Eine Lektüre durch Schnitzler ist
                  nicht belegt.}}}\label{K_L03090-9}.\pend
           
\pstart
           Leb’ wohl für heut! Viele treue Grüße! {\\[\baselineskip]}Dein {\\[\baselineskip]}\spacefill\mbox{Paul Goldmann.}\pend
           \leftskip=0em{}\selectlanguage{ngerman}\endnumbering\briefempfaengerindex{Schnitzler, Arthur@\textsc{Schnitzler, Arthur}!zzzGoldmann, Paul@\emph{von Paul Goldmann}!1901-11-092@{9. 11. {[}1901{]}}|)be}\mylabel{L03090h}  \normalsize

\doendnotes{C}
\bigskip
\vfill

\clearpage

\footnotesize

\lohead{\textsc{register}}

% Definiere theindex-Environment komplett neu ohne reledmac
\makeatletter
\renewenvironment{theindex}{%
  \section*{\indexname}%
  \setlength{\parindent}{0pt}%
  \setlength{\parskip}{0pt plus 0.3pt}%
  \let\item\@idxitem
}{%
  \clearpage
}
\makeatother

\IfFileExists{\jobname-pw.ind}{\input{\jobname-pw.ind}}{}

\end{document}

      