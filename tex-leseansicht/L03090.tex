%% latex-leseansicht-vorspann.tex
%% Vorspann für die Leseansicht.
%% Lädt die gemeinsame Datei latex-vorspann.tex mit nicht gesetztem Schalter.

\newif\ifkorrekturansicht
\korrekturansichtfalse

\input{../tex-inputs/latex-vorspann}


         
         \renewcommand{\erwaehntePersonen}{Personen: Sébastien Roch Nicolas Chamfort, Leo Ebermann, Eduard Grisebach, Gerhart Hauptmann, Fritz Mauthner, Olga Schnitzler, Arthur Schopenhauer, Elisabeth Steinrück, Irene Triesch}
         \renewcommand{\erwaehnteInstitutionen}{Institutionen: Ernst Hofmann {\kaufmannsund}  Co.}
         \renewcommand{\erwaehnteOrte}{Orte: Berlin, Dessauer Straße, Hamburg, München, Payerbach, Wien}
         \renewcommand{\erwaehnteWerke}{Werke: Berliner Brief. [»Schluck und Jau« von Gerhart Hauptmann am Deutschen Theater], Berliner Tageblatt, Berliner Theater. »Einsame Menschen« im Deutschen Theater, Einsame Menschen. Drama, Hebbels »Maria Magdalena«. (Deutsches Theater.), Lebendige Stunden. Vier Einakter, Neue Freie Presse, Schopenhauer’s Gespräche und Selbstgespräche: Nach der Handschrift eis heauton, »Michael Kramer.«, Œuvres choisies de N. Chamfort, publiées avec préface, notes et tables}
               \section[ Paul Goldmann an Arthur Schnitzler, 9. 11. {[}1901{]}]{ Paul Goldmann an Arthur Schnitzler, 9. 11. {[}1901{]}}\nopagebreak\mylabel{v}\rehead{ }\begin{ledgroupsized}[t]{13cm}\normalsize\beginnumbering \toendnotes[C]{\smallbreak\pagebreak[2]} \Standort{DLA, A:Schnitzler, HS.NZ85.1.3171.}
\physDesc{Brief, 1 Blatt, 4 Seiten, 1663 Zeichen
\newline{}Handschrift: blaue Tinte, deutsche Kurrent
\newline{}Schnitzler: 1) mit Bleistift das Jahr »1901« vermerkt  2) mit rotem Buntstift drei Unterstreichungen}\toendnotes[C]{\smallbreak}\pstart
           \noindent{}\raggedleft{}{\pb}\textcolor{gray}{\textbf{DESSAUERSTRASSE 19}}\oindex{Dessauer Strasse@\textbf{Dessauer Straße}|pw}\pend
           \pstart
           Berlin\oindex{Berlin@\textbf{Berlin}|pw}, 9. November.\pend
           \pstart\center{}Mein lieber Freund,\pend\pstart
           Ich habe \introOben{}mich\introOben{} ſehr gefreut, endlich wieder einmal etwas von
               Dir zu hören. Daß die Aufführung Deiner Stücke\pwindex{Schnitzler, Arthur 15.05.1862 – 21.10.1931@\textsc{Schnitzler, Arthur} (15.05.1862 – 21.10.1931), \emph{Schriftsteller, Mediziner}!Lebendige Stunden. Vier Einakter1901-12-23@\strich\emph{Lebendige Stunden. Vier Einakter} {[}1901-12-23{]}|pwv} bis \label{K_L03090-1v}\edtext{Februar}{\lemma{\textnormal{\emph{Februar}}}\Cendnote{\textnormal{Grund war ein geplantes Gastspiel Irene Triesch\pwindex{Triesch, Irene 13.04.1877 – 24.11.1964@\textsc{Triesch, Irene} (13.04.1877 – 24.11.1964), \emph{Schauspielerin}|pwk}s, die in den weiblichen
                  Hauptrollen von \emph{Lebendige Stunden}\pwindex{Schnitzler, Arthur 15.05.1862 – 21.10.1931@\textsc{Schnitzler, Arthur} (15.05.1862 – 21.10.1931), \emph{Schriftsteller, Mediziner}!Lebendige Stunden. Vier Einakter1901-12-23@\strich\emph{Lebendige Stunden. Vier Einakter} {[}1901-12-23{]}|pwk} auftrat
                     (vgl. \emph{Der Briefwechsel Arthur Schnitzler — Otto
                        Brahm}. Vollständige Ausgabe. Herausgegeben, eingeleitet und
                     erläutert von Oskar Seidlin. Tübingen:
                        \emph{Niemeyer}{ }1975, S. 102). Die Uraufführung konnte
                  schließlich noch vor Triesch\pwindex{Triesch, Irene 13.04.1877 – 24.11.1964@\textsc{Triesch, Irene} (13.04.1877 – 24.11.1964), \emph{Schauspielerin}|pwk}s geplanter
                  Abwesenheit (Mitte Januar bis Mitte Februar 1902), am 4. 1. 1902,
                  stattfinden.}}}\label{K_L03090-1h} verſchoben werden ſoll, iſt bedauerlich. Könnteſt Du nicht
               wenigſtens anderswo, in Hamburg\oindex{Hamburg@\textbf{Hamburg}|pw}, München\oindex{Muenchen@\textbf{München}|pw}, vielleicht gar in Wien\oindex{Wien@\textbf{Wien}|pw}, eine frühere Aufführung veranlaſſen \introOben{}damit Dir nicht
                  der Winter verloren geht\introOben{}? Die \textsc{Triesch\pwindex{Triesch, Irene 13.04.1877 – 24.11.1964@\textsc{Triesch, Irene} (13.04.1877 – 24.11.1964), \emph{Schauspielerin}|pw}} wird hier von der kunſtunverſtändigen Kritik ſo {\pb}\label{K_L03090-2v}\edtext{wenig begriffen}{\lemma{\textnormal{\emph{wenig begriffen}}}\Cendnote{\textnormal{Siehe etwa F. M.\pwindex{Mauthner, Fritz 1849-11-20 – 1923-06-29@\textsc{Mauthner, Fritz} (1849-11-20 – 1923-06-29), \emph{Schriftsteller, Journalist, Philosoph}|pwkv} [ = Fritz Mauthner\pwindex{Mauthner, Fritz 1849-11-20 – 1923-06-29@\textsc{Mauthner, Fritz} (1849-11-20 – 1923-06-29), \emph{Schriftsteller, Journalist, Philosoph}|pwk}]: \emph{Hebbels »Maria Magdalena«. (Deutsches Theater.)}\pwindex{Hebbels »Maria Magdalena«. (Deutsches Theater.)1901-11-06@\emph{Hebbels »Maria Magdalena«. (Deutsches Theater.)} {[}1901-11-06{]}|pwk}. In:
                        \emph{Berliner Tageblatt}\pwindex{?? Werk@Nicht ermittelte Verfasserinnen und Verfasser!Berliner Tageblatt1872 – 1939@\emph{Berliner Tageblatt} {[}1872 – 1939{]}|pwk}, Jg. 30, Nr. 565,
                        6. 11. 1901, S. [3].}}}\label{K_L03090-2h}, daß es
               beinahe eine Gefahr für Deine Stücke\pwindex{Schnitzler, Arthur 15.05.1862 – 21.10.1931@\textsc{Schnitzler, Arthur} (15.05.1862 – 21.10.1931), \emph{Schriftsteller, Mediziner}!Lebendige Stunden. Vier Einakter1901-12-23@\strich\emph{Lebendige Stunden. Vier Einakter} {[}1901-12-23{]}|pwv} iſt, wenn ſie die \label{K_L03090-3v}\edtext{Hauptrolle}{\lemma{\textnormal{\emph{Hauptrolle}}}\Cendnote{\textnormal{siehe Paul Goldmann an Arthur Schnitzler, 23. 9. [1901]}}}\label{K_L03090-3h} ſpielt, die ſie natürlich herrlich ſpielen wird. Ich habe mit dieſer
               hyſteriſchen Jüdin, die mir unerträglich geworden iſt, alle Beziehungen
               abgebrochen.\pend
           \pstart
           Daß \label{K_L03090-4v}\edtext{\textsc{Olga\pwindex{Schnitzler, Olga 17.01.1882 – 13.01.1970@\textsc{Schnitzler, Olga} (17.01.1882 – 13.01.1970), \emph{Schauspielerin, Sängerin}|pw}} krank}{\lemma{\textnormal{\emph{Olga krank}}}\Cendnote{\textnormal{Sie\pwindex{Schnitzler, Olga 17.01.1882 – 13.01.1970@\textsc{Schnitzler, Olga} (17.01.1882 – 13.01.1970), \emph{Schauspielerin, Sängerin}|pwkv} hatte Angina (vgl. A. S.: \emph{Tagebuch}, 25. 10. 1901).}}}\label{K_L03090-4h} war,
               habe ich mit Bedauern vernommen. Was ihr gefehlt hat, habe ich, trotz langjähriger
               Kenntniß Deiner Handſchrift, nicht entziffern können. Immerhin freue ich mich, daß
               ſie wieder geſund iſt, und bitte Dich, ſie ſammt {\pb}der Schweſter\pwindex{Steinrueck, Elisabeth 19.11.1885 – 07.04.1920@\textsc{Steinrück, Elisabeth} (19.11.1885 – 07.04.1920)|pwv} zu
               grüßen.\pend
           \pstart
           Was meine \label{K_L03090-5v}\edtext{Feuilletons\pwindex{Hauptmann, Gerhart 15.11.1862 – 06.06.1946@\textsc{Hauptmann, Gerhart} (15.11.1862 – 06.06.1946), \emph{Schriftsteller}!Einsame Menschen. Drama1891-01-11@\strich\emph{Einsame Menschen. Drama} {[}1891-01-11{]}|pwv}\pwindex{Goldmann, Paul 31.01.1865 – 25.09.1935@\textsc{Goldmann, Paul} (31.01.1865 – 25.09.1935), \emph{Schriftsteller, Journalist}!Berliner Brief. [»Schluck und Jau« von Gerhart Hauptmann am Deutschen Theater]1900-02-06@\strich\emph{Berliner Brief. [»Schluck und Jau« von Gerhart Hauptmann am Deutschen Theater]} {[}1900-02-06{]}|pwv}\pwindex{Goldmann, Paul 31.01.1865 – 25.09.1935@\textsc{Goldmann, Paul} (31.01.1865 – 25.09.1935), \emph{Schriftsteller, Journalist}!Michael Kramer.«1900-12-28@\strich\emph{»Michael Kramer.«} {[}1900-12-28{]}|pwv}}{\lemma{\textnormal{\emph{Feuilletons}}}\Cendnote{\textnormal{Paul Goldmann\pwindex{Goldmann, Paul 31.01.1865 – 25.09.1935@\textsc{Goldmann, Paul} (31.01.1865 – 25.09.1935), \emph{Schriftsteller, Journalist}|pwk}: \emph{Berliner Brief}\pwindex{Goldmann, Paul 31.01.1865 – 25.09.1935@\textsc{Goldmann, Paul} (31.01.1865 – 25.09.1935), \emph{Schriftsteller, Journalist}!Berliner Brief. [»Schluck und Jau« von Gerhart Hauptmann am Deutschen Theater]1900-02-06@\strich\emph{Berliner Brief. [»Schluck und Jau« von Gerhart Hauptmann am Deutschen Theater]} {[}1900-02-06{]}|pwk}. In: \emph{Neue Freie Presse}\pwindex{Neue Freie Presse1864 – 1939@\emph{Neue Freie Presse} {[}1864 – 1939{]}|pwk}, Nr. 12.735, 6. 2. 1900, Morgenblatt, S. 1–3; ders.\pwindex{Goldmann, Paul 31.01.1865 – 25.09.1935@\textsc{Goldmann, Paul} (31.01.1865 – 25.09.1935), \emph{Schriftsteller, Journalist}|pwkv}: \emph{»Michael Kramer.«}\pwindex{Goldmann, Paul 31.01.1865 – 25.09.1935@\textsc{Goldmann, Paul} (31.01.1865 – 25.09.1935), \emph{Schriftsteller, Journalist}!Michael Kramer.«1900-12-28@\strich\emph{»Michael Kramer.«} {[}1900-12-28{]}|pwk}. In: ebd.\pwindex{Neue Freie Presse1864 – 1939@\emph{Neue Freie Presse} {[}1864 – 1939{]}|pwkv}, Nr. 13.055, 28. 12. 1900, Morgenblatt, S. 1–3; ders.\pwindex{Goldmann, Paul 31.01.1865 – 25.09.1935@\textsc{Goldmann, Paul} (31.01.1865 – 25.09.1935), \emph{Schriftsteller, Journalist}|pwk}: \emph{Berliner Theater. »Einsame Menschen« im Deutschen Theater}\pwindex{Goldmann, Paul 31.01.1865 – 25.09.1935@\textsc{Goldmann, Paul} (31.01.1865 – 25.09.1935), \emph{Schriftsteller, Journalist}!Berliner Theater. »Einsame Menschen« im Deutschen Theater19. 10. 1901@\strich\emph{Berliner Theater. »Einsame Menschen« im Deutschen Theater} {[}19. 10. 1901{]}|pwk}. In: ebd.\pwindex{Neue Freie Presse1864 – 1939@\emph{Neue Freie Presse} {[}1864 – 1939{]}|pwkv}, Nr. 13.345, 19. 10. 1901, Morgenblatt, S. 1–3.}}}\label{K_L03090-5h}
               über \textsc{Gerhart Hauptmann\pwindex{Hauptmann, Gerhart 15.11.1862 – 06.06.1946@\textsc{Hauptmann, Gerhart} (15.11.1862 – 06.06.1946), \emph{Schriftsteller}|pw}} anlangt, ſo ſtimmen mir noch andere Leute zu, als Herr \textsc{Ebermann\pwindex{Ebermann, Leo 16.07.1863 – 09.10.1914@\textsc{Ebermann, Leo} (16.07.1863 – 09.10.1914), \emph{Schriftsteller, Journalist, Rechtswissenschaftler}|pw}}. Im Übrigen wäre es mir ſehr gleichgiltig, auch wenn Niemand mir zuſtimmte, da
               ich weiß, daß ich Recht habe. Was Du über den \label{K_L03090-6v}\edtext{»Ton«}{\lemma{\textnormal{\emph{»Ton«}}}\Cendnote{\textnormal{siehe A. S.: \emph{Tagebuch}, 27. 11. 1901 und Paul Goldmann an Arthur Schnitzler, 6. 12. [1901]}}}\label{K_L03090-6h} ſchreibſt, verſtehe ich nicht. Das heißt, ich begreife nicht, wie Einer, der
               ſelbſt ſchreibt, dieſen Einwand erheben kann. Mein Ton bin nämlich ich ſelbſt. Aus
               dieſem Grunde wird es nicht leicht ſein, ihn zu ändern.\pend
           \pstart
           Es thut mir unendlich leid, daß {\pb}durch den Aufſchub
               der Aufführung Deiner Stücke\pwindex{Schnitzler, Arthur 15.05.1862 – 21.10.1931@\textsc{Schnitzler, Arthur} (15.05.1862 – 21.10.1931), \emph{Schriftsteller, Mediziner}!Lebendige Stunden. Vier Einakter1901-12-23@\strich\emph{Lebendige Stunden. Vier Einakter} {[}1901-12-23{]}|pwv}{ }\strikeout{D\textcolor{gray}{ei}} auch Deine \label{K_L03090-7v}\edtext{Reiſe nach Berlin\oindex{Berlin@\textbf{Berlin}|pw} verſchoben}{\lemma{\textnormal{\emph{Reiſe … verſchoben}}}\Cendnote{\textnormal{Schnitzler\pwindex{Schnitzler, Arthur 15.05.1862 – 21.10.1931@\textsc{Schnitzler, Arthur} (15.05.1862 – 21.10.1931), \emph{Schriftsteller, Mediziner}|pwk} war letztendlich von 28. 12. 1901 bis 6. 1. 1902 in Berlin\oindex{Berlin@\textbf{Berlin}|pwk}.}}}\label{K_L03090-7h} iſt.\pend
           \pstart
           Haſt Du den \label{K_L03090-8v}\edtext{\textsc{Chamfort\pwindex{\textcolor{red}{\textsuperscript{XXXX1 indx}}!Œuvres choisies de N. Chamfort, publiees avec preface, notes et tables1892@\strich\emph{Œuvres choisies de N. Chamfort, publiées avec préface, notes et tables} {[}Hrsg., 1892{]}|pwv}\pwindex{Chamfort, Sebastien Roch Nicolas 06.04.1741 – 13.04.1794@\textsc{Chamfort, Sébastien Roch Nicolas} (06.04.1741 – 13.04.1794), \emph{Schriftsteller}|pw}}}{\lemma{\textnormal{\emph{Chamfort}}}\Cendnote{\textnormal{siehe Paul Goldmann an Arthur Schnitzler, 23. 9. [1901]}}}\label{K_L03090-8h} nun endlich erhalten? Und haſt Du ihn geleſen? Lies’ auch die eben von \textsc{Griesebach\pwindex{Grisebach, Eduard 1845-10-09 – 1906-03-22@\textsc{Grisebach, Eduard} (1845-10-09 – 1906-03-22), \emph{Schriftsteller, Diplomat, Jurist}|pw}} herausgegebenen \label{K_L03090-9v}\edtext{Geſpräche mit \textsc{Schopenhauer\pwindex{Schopenhauer, Arthur 22.02.1788 – 21.09.1860@\textsc{Schopenhauer, Arthur} (22.02.1788 – 21.09.1860), \emph{Philosoph}|pw}}\pwindex{Grisebach, Eduard 1845-10-09 – 1906-03-22@\textsc{Grisebach, Eduard} (1845-10-09 – 1906-03-22), \emph{Schriftsteller, Diplomat, Jurist}!Schopenhauer s Gespraeche und Selbstgespraeche: Nach der Handschrift eis heauton1898@\strich\emph{Schopenhauer’s Gespräche und Selbstgespräche: Nach der Handschrift eis heauton} {[}Hrsg., 1898{]}|pw}}{\lemma{\textnormal{\emph{Geſpräche mit Schopenhauer}}}\Cendnote{\textnormal{\emph{Schopenhauer’s Gespräche und Selbstgespräche:
                        Nach der Handschrift eis heauton}\pwindex{Grisebach, Eduard 1845-10-09 – 1906-03-22@\textsc{Grisebach, Eduard} (1845-10-09 – 1906-03-22), \emph{Schriftsteller, Diplomat, Jurist}!Schopenhauer s Gespraeche und Selbstgespraeche: Nach der Handschrift eis heauton1898@\strich\emph{Schopenhauer’s Gespräche und Selbstgespräche: Nach der Handschrift eis heauton} {[}Hrsg., 1898{]}|pwk}. Herausgegeben von Eduard Grisebach\pwindex{Grisebach, Eduard 1845-10-09 – 1906-03-22@\textsc{Grisebach, Eduard} (1845-10-09 – 1906-03-22), \emph{Schriftsteller, Diplomat, Jurist}|pwk}. Berlin\oindex{Berlin@\textbf{Berlin}|pwk}: \emph{Ernst Hofmann {\kaufmannsund} Co.}\orgindex{Ernst Hofmann und Co.@Ernst Hofmann {\kaufmannsund}  Co.|pwk}{ }1898. Eine Lektüre durch Schnitzler\pwindex{Schnitzler, Arthur 15.05.1862 – 21.10.1931@\textsc{Schnitzler, Arthur} (15.05.1862 – 21.10.1931), \emph{Schriftsteller, Mediziner}|pwk} ist
                  nicht belegt.}}}\label{K_L03090-9h}.\pend
           \pstart
           Leb’ wohl für heut! Viele treue Grüße! {\\[\baselineskip]}Dein {\\[\baselineskip]}\spacefill\mbox{Paul Goldmann.}\pend
           \leftskip=0em{}
         
         \endnumbering\mylabel{h}\end{ledgroupsized}  \newcommand{\dateiname}{L03090}\newcommand{\titel}{Paul Goldmann an Arthur Schnitzler, 9. 11. [1901]}\newcommand{\editorInnen}{Martin Anton Müller und Laura Untner}%% latex-leseansicht-abspann.tex
%% Abspann für die Leseansicht.
%% Der Schalter \ifkorrekturansicht ist bereits durch den Vorspann gesetzt.

%% latex-abspann.tex
%% Gemeinsamer Abspann für Korrekturansicht und Leseansicht.
%% Setzt den Schalter \ifkorrekturansicht voraus (gesetzt in den
%% einbindenden Dateien latex-korrekturansicht-abspann.tex bzw.
%% latex-leseansicht-abspann.tex).
%% ---------------------------------------------------------------

\normalsize

% Das esempio-Environment wird nur in der Leseansicht benötigt
\ifkorrekturansicht\else
\newenvironment{esempio}[3]%
{
    \vspace{1.5ex}
    \rlap{\underline{#1}}
    \par
    \setlength{\parindent}{0cm}
    \nopagebreak
    \leftskip=#2cm
    \rightskip=#3cm
}
{
    \par
}
\fi

\doendnotes{C}
\bigskip
\vfill

\clearpage

\footnotesize

\ifkorrekturansicht
  \lohead{\textsc{register}}
\fi

% theindex-Environment neu definieren ohne reledmac
\makeatletter
\renewenvironment{theindex}{%
  \ifkorrekturansicht
    \section*{\indexname}%
  \else
    \subsubsection*{Index der erwähnten Entitäten}%
  \fi
  \setlength{\parindent}{0pt}%
  \setlength{\parskip}{0pt plus 0.3pt}%
  \let\item\@idxitem
}{%
  \ifkorrekturansicht\clearpage\fi
}
\makeatother

\IfFileExists{\jobname-pw.ind}{\input{\jobname-pw.ind}}{}

% Quellenangabe nur in der Leseansicht
\ifkorrekturansicht\else
% Fallback-Definitionen, falls die .tex-Datei \titel etc. nicht gesetzt hat
\providecommand{\titel}{}
\providecommand{\editorInnen}{}
\providecommand{\dateiname}{\jobname}

\vspace{3cm}

\vfill

\footnotesize
\textsc{Quelle}: \titel. Herausgegeben von {\editorInnen}. In: \emph{Arthur Schnitzler: Briefwechsel mit Autorinnen und Autoren}.
 Digitale Edition, https://schnitzler-briefe.acdh.oeaw.ac.at/{\dateiname}.html (Stand \today)
\fi

\end{document}


      