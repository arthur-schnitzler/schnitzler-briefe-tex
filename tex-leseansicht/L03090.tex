%% latex-leseansicht-vorspann.tex
%% Vorspann für die Leseansicht.
%% Lädt die gemeinsame Datei latex-vorspann.tex mit nicht gesetztem Schalter.

\newif\ifkorrekturansicht
\korrekturansichtfalse

\input{../tex-inputs/latex-vorspann}


\section[ Paul Goldmann an Arthur Schnitzler, 9. 11. [1901]]{L03090 Paul Goldmann an Arthur Schnitzler,  9. 11. [1901]}
\nopagebreak\mylabel{L03090v}
\rehead{ }\normalsize\beginnumbering\briefempfaengerindex{Schnitzler, Arthur@\textsc{Schnitzler, Arthur}!zzzGoldmann, Paul@\emph{von Paul Goldmann}!1901-11-092@{9. 11. [1901]}|(be}
\toendnotes[C]{\smallbreak\pagebreak[2]}
\correspDesc{Versand  durch Paul Goldmann am 9. 11. [1901] in Berlin
\newline{}Erhalt  durch Arthur Schnitzler im Zeitraum [10. 11. 1901 – 14. 11. 1901?] in Payerbach}\toendnotes[C]{\smallbreak}
\Standort{DLA, A:Schnitzler, HS.NZ85.1.3171.}
\physDesc{Brief, 1 Blatt, 4 Seiten, 1663 Zeichen
\newline{}Handschrift: blaue Tinte, deutsche Kurrent
\newline{}Schnitzler: 1) mit Bleistift das Jahr »1901« vermerkt  2) mit rotem Buntstift drei Unterstreichungen}\toendnotes[C]{\smallbreak}
\pstart
           \raggedleft{}{\pb}\textcolor{gray}{\textbf{DESSAUERSTRASSE 19}}\oindex{Dessauer Straße@\textbf{Dessauer Straße}, \emph{Straße}|pw}\pend
           
\pstart
           Berlin\oindex{Berlin@\textbf{Berlin}, \emph{Hauptstadt}|pw}, 9. November.\pend
           
\pstart\center{}Mein lieber Freund,\pend\vspace{0.5em}
\pstart
           Ich habe \introOben{}mich\introOben{}{ }ſehr gefreut, endlich wieder einmal etwas von
               Dir zu hören. Daß die Aufführung Deiner Stücke\pwindex{Schnitzler, Arthur 15.\,5.\,1862 Wien – 21.\,10.\,1931 ebd.@\textsc{Schnitzler, Arthur} (15.\,5.\,1862 Wien – 21.\,10.\,1931 ebd.), \emph{Schriftsteller, Mediziner}!Lebendige Stunden. Vier Einakter@\strich\emph{Lebendige Stunden. Vier Einakter}|pwv} bis \label{K_L03090-1v}\edtext{Februar}{\lemma{\textnormal{\emph{Februar}}}\Cendnote{\textnormal{Grund war ein geplantes Gastspiel Irene Trieschs\pwindex{Triesch, Irene 13.\,4.\,1877 Wien – 24.\,11.\,1964 Basel@\textsc{Triesch, Irene} (13.\,4.\,1877 Wien – 24.\,11.\,1964 Basel), \emph{Schauspielerin}|pwk}, die in den weiblichen
                  Hauptrollen von \emph{Lebendige Stunden}\pwindex{Schnitzler, Arthur 15.\,5.\,1862 Wien – 21.\,10.\,1931 ebd.@\textsc{Schnitzler, Arthur} (15.\,5.\,1862 Wien – 21.\,10.\,1931 ebd.), \emph{Schriftsteller, Mediziner}!Lebendige Stunden. Vier Einakter@\strich\emph{Lebendige Stunden. Vier Einakter}|pwk} auftrat
                     (vgl. \emph{Der Briefwechsel Arthur Schnitzler – Otto
                        Brahm}. Vollständige Ausgabe. Herausgegeben, eingeleitet und
                     erläutert von Oskar Seidlin. Tübingen:
                        \emph{Niemeyer}{ }1975, S. 102). Die Uraufführung\eventindex{Deutsches Theater Berlin@\textbf{Deutsches Theater Berlin}!Uraufführung von Lebendige Stunden, 4.1.1902@Uraufführung von Lebendige Stunden, 4.1.1902|pwkv} konnte
                  schließlich noch vor Trieschs\pwindex{Triesch, Irene 13.\,4.\,1877 Wien – 24.\,11.\,1964 Basel@\textsc{Triesch, Irene} (13.\,4.\,1877 Wien – 24.\,11.\,1964 Basel), \emph{Schauspielerin}|pwk} geplanter
                  Abwesenheit (Mitte Januar bis Mitte Februar 1902), am 4. 1. 1902,
                  stattfinden.}}}\label{K_L03090-1} verſchoben werden{ }ſoll, iſt bedauerlich. Könnteſt Du nicht
               wenigſtens anderswo, in Hamburg\oindex{Hamburg@\textbf{Hamburg}|pw}, München\oindex{München@\textbf{München}|pw}, vielleicht gar in Wien\oindex{Wien@\textbf{Wien}, \emph{Verwaltungsgebiet}|pw}, eine frühere Aufführung veranlaſſen \introOben{}damit Dir nicht
                  der Winter verloren geht\introOben{}? Die \textsc{Triesch\pwindex{Triesch, Irene 13.\,4.\,1877 Wien – 24.\,11.\,1964 Basel@\textsc{Triesch, Irene} (13.\,4.\,1877 Wien – 24.\,11.\,1964 Basel), \emph{Schauspielerin}|pw}} wird hier von der kunſtunverſtändigen Kritik{ }ſo {\pb}\label{K_L03090-2v}\edtext{wenig begriffen}{\lemma{\textnormal{\emph{wenig begriffen}}}\Cendnote{\textnormal{Siehe etwa F. M.\pwindex{Mauthner, Fritz 20.\,11.\,1849 Hořice – 29.\,6.\,1923 Meersburg@\textsc{Mauthner, Fritz} (20.\,11.\,1849 Hořice – 29.\,6.\,1923 Meersburg), \emph{Schriftsteller, Journalist, Philosoph}|pwkv} [ = Fritz Mauthner\pwindex{Mauthner, Fritz 20.\,11.\,1849 Hořice – 29.\,6.\,1923 Meersburg@\textsc{Mauthner, Fritz} (20.\,11.\,1849 Hořice – 29.\,6.\,1923 Meersburg), \emph{Schriftsteller, Journalist, Philosoph}|pwk}]: \emph{Hebbels »Maria Magdalena«. (Deutsches Theater)}\pwindex{Mauthner, Fritz 20.\,11.\,1849 Hořice – 29.\,6.\,1923 Meersburg@\textsc{Mauthner, Fritz} (20.\,11.\,1849 Hořice – 29.\,6.\,1923 Meersburg), \emph{Schriftsteller, Journalist, Philosoph}!Hebbels »Maria Magdalena«. (Deutsches Theater.)@\strich\emph{Hebbels »Maria Magdalena«. (Deutsches Theater.)}|pwk}. In:
                        \emph{Berliner Tageblatt}\pwindex{Berliner Tageblatt@\emph{Berliner Tageblatt}|pwk}, Jg. 30, Nr. 565,
                        6. 11. 1901, S. [3].}}}\label{K_L03090-2}, daß es
               beinahe eine Gefahr für Deine Stücke\pwindex{Schnitzler, Arthur 15.\,5.\,1862 Wien – 21.\,10.\,1931 ebd.@\textsc{Schnitzler, Arthur} (15.\,5.\,1862 Wien – 21.\,10.\,1931 ebd.), \emph{Schriftsteller, Mediziner}!Lebendige Stunden. Vier Einakter@\strich\emph{Lebendige Stunden. Vier Einakter}|pwv} iſt, wenn{ }ſie die \label{K_L03090-3v}\edtext{Hauptrolle}{\lemma{\textnormal{\emph{Hauptrolle}}}\Cendnote{\textnormal{Siehe XXXX Auszeichnungsfehler: Dokument L03085 nicht gefunden.
               }}}\label{K_L03090-3}{ }ſpielt, die{ }ſie natürlich herrlich{ }ſpielen wird. Ich habe mit dieſer
               hyſteriſchen Jüdin, die mir unerträglich geworden iſt, alle Beziehungen
               abgebrochen.\pend
           
\pstart
           Daß \label{K_L03090-4v}\edtext{\textsc{Olga\pwindex{Schnitzler, Olga 17.\,1.\,1882 Wien – 13.\,1.\,1970 Lugano@\textsc{Schnitzler, Olga} (17.\,1.\,1882 Wien – 13.\,1.\,1970 Lugano), \emph{Schauspielerin, Sängerin}|pw}} krank}{\lemma{\textnormal{\emph{Olga krank}}}\Cendnote{\textnormal{Sie\pwindex{Schnitzler, Olga 17.\,1.\,1882 Wien – 13.\,1.\,1970 Lugano@\textsc{Schnitzler, Olga} (17.\,1.\,1882 Wien – 13.\,1.\,1970 Lugano), \emph{Schauspielerin, Sängerin}|pwkv} hatte Angina (vgl. A. S.: \emph{Tagebuch}, 25. 10. 1901).}}}\label{K_L03090-4} war,
               habe ich mit Bedauern vernommen. Was ihr gefehlt hat, habe ich, trotz langjähriger
               Kenntniß Deiner Handſchrift, nicht entziffern können. Immerhin freue ich mich, daß{ }ſie wieder geſund iſt, und bitte Dich,{ }ſie{ }ſammt {\pb}der Schweſter\pwindex{Steinrück, Elisabeth 19.\,11.\,1885 – 7.\,4.\,1920 Partenkirchen@\textsc{Steinrück, Elisabeth} (19.\,11.\,1885 – 7.\,4.\,1920 Partenkirchen)|pwv} zu
               grüßen.\pend
           
\pstart
           Was meine \label{K_L03090-5v}\edtext{Feuilletons\pwindex{Hauptmann, Gerhart 15.\,11.\,1862 Szczawno-Zdrój – 6.\,6.\,1946 Jagniątków@\textsc{Hauptmann, Gerhart} (15.\,11.\,1862 Szczawno-Zdrój – 6.\,6.\,1946 Jagniątków), \emph{Schriftsteller}!Einsame Menschen. Drama@\strich\emph{Einsame Menschen. Drama}|pwv}\pwindex{Goldmann, Paul 31.\,1.\,1865 Breslau – 25.\,9.\,1935 Wien@\textsc{Goldmann, Paul} (31.\,1.\,1865 Breslau – 25.\,9.\,1935 Wien), \emph{Schriftsteller, Journalist}!Berliner Brief. [»Schluck und Jau« von Gerhart Hauptmann am Deutschen Theater]@\strich\emph{Berliner Brief. [»Schluck und Jau« von Gerhart Hauptmann am Deutschen Theater]}|pwv}\pwindex{Goldmann, Paul 31.\,1.\,1865 Breslau – 25.\,9.\,1935 Wien@\textsc{Goldmann, Paul} (31.\,1.\,1865 Breslau – 25.\,9.\,1935 Wien), \emph{Schriftsteller, Journalist}!Michael Kramer.«@\strich\emph{»Michael Kramer.«}|pwv}}{\lemma{\textnormal{\emph{Feuilletons}}}\Cendnote{\textnormal{Paul Goldmann\pwindex{Goldmann, Paul 31.\,1.\,1865 Breslau – 25.\,9.\,1935 Wien@\textsc{Goldmann, Paul} (31.\,1.\,1865 Breslau – 25.\,9.\,1935 Wien), \emph{Schriftsteller, Journalist}|pwk}: \emph{Berliner Brief}\pwindex{Goldmann, Paul 31.\,1.\,1865 Breslau – 25.\,9.\,1935 Wien@\textsc{Goldmann, Paul} (31.\,1.\,1865 Breslau – 25.\,9.\,1935 Wien), \emph{Schriftsteller, Journalist}!Berliner Brief. [»Schluck und Jau« von Gerhart Hauptmann am Deutschen Theater]@\strich\emph{Berliner Brief. [»Schluck und Jau« von Gerhart Hauptmann am Deutschen Theater]}|pwk}. In: \emph{Neue Freie Presse}\pwindex{Neue Freie Presse@\emph{Neue Freie Presse}|pwk}, Nr. 12.735, 6. 2. 1900, Morgenblatt, S. 1–3; Paul Goldmann\pwindex{Goldmann, Paul 31.\,1.\,1865 Breslau – 25.\,9.\,1935 Wien@\textsc{Goldmann, Paul} (31.\,1.\,1865 Breslau – 25.\,9.\,1935 Wien), \emph{Schriftsteller, Journalist}|pwk}: \emph{»Michael Kramer«}\pwindex{Goldmann, Paul 31.\,1.\,1865 Breslau – 25.\,9.\,1935 Wien@\textsc{Goldmann, Paul} (31.\,1.\,1865 Breslau – 25.\,9.\,1935 Wien), \emph{Schriftsteller, Journalist}!Michael Kramer.«@\strich\emph{»Michael Kramer.«}|pwk}. In: \emph{Neue Freie Presse}\pwindex{Neue Freie Presse@\emph{Neue Freie Presse}|pwk}, Nr. 13.055, 28. 12. 1900, Morgenblatt, S. 1–3; Paul Goldmann\pwindex{Goldmann, Paul 31.\,1.\,1865 Breslau – 25.\,9.\,1935 Wien@\textsc{Goldmann, Paul} (31.\,1.\,1865 Breslau – 25.\,9.\,1935 Wien), \emph{Schriftsteller, Journalist}|pwk}: \emph{Berliner Theater. »Einsame Menschen« im Deutschen Theater}\pwindex{Goldmann, Paul 31.\,1.\,1865 Breslau – 25.\,9.\,1935 Wien@\textsc{Goldmann, Paul} (31.\,1.\,1865 Breslau – 25.\,9.\,1935 Wien), \emph{Schriftsteller, Journalist}!Berliner Theater. »Einsame Menschen« im Deutschen Theater@\strich\emph{Berliner Theater. »Einsame Menschen« im Deutschen Theater}|pwk}. In: \emph{Neue Freie Presse}\pwindex{Neue Freie Presse@\emph{Neue Freie Presse}|pwk}, Nr. 13.345, 19. 10. 1901, Morgenblatt, S. 1–3.}}}\label{K_L03090-5}
               über \textsc{Gerhart Hauptmann\pwindex{Hauptmann, Gerhart 15.\,11.\,1862 Szczawno-Zdrój – 6.\,6.\,1946 Jagniątków@\textsc{Hauptmann, Gerhart} (15.\,11.\,1862 Szczawno-Zdrój – 6.\,6.\,1946 Jagniątków), \emph{Schriftsteller}|pw}} anlangt,{ }ſo{ }ſtimmen mir noch andere Leute zu, als Herr \textsc{Ebermann\pwindex{Ebermann, Leo 16.\,7.\,1863 Draganovka – 9.\,10.\,1914 Wien@\textsc{Ebermann, Leo} (16.\,7.\,1863 Draganovka – 9.\,10.\,1914 Wien), \emph{Schriftsteller, Journalist, Rechtswissenschaftler}|pw}}. Im Übrigen wäre es mir{ }ſehr gleichgiltig, auch wenn Niemand mir zuſtimmte, da
               ich weiß, daß ich Recht habe. Was Du über den \label{K_L03090-6v}\edtext{»Ton«}{\lemma{\textnormal{\emph{»Ton«}}}\Cendnote{\textnormal{Siehe A. S.: \emph{Tagebuch}, 27. 11. 1901 und XXXX Auszeichnungsfehler: Dokument L03094 nicht gefunden.
               }}}\label{K_L03090-6}{ }ſchreibſt, verſtehe ich nicht. Das heißt, ich begreife nicht, wie Einer, der{ }ſelbſt{ }ſchreibt, dieſen Einwand erheben kann. Mein Ton bin nämlich ich{ }ſelbſt. Aus
               dieſem Grunde wird es nicht leicht{ }ſein, ihn zu ändern.\pend
           
\pstart
           Es thut mir unendlich leid, daß {\pb}durch den Aufſchub
               der Aufführung Deiner Stücke\pwindex{Schnitzler, Arthur 15.\,5.\,1862 Wien – 21.\,10.\,1931 ebd.@\textsc{Schnitzler, Arthur} (15.\,5.\,1862 Wien – 21.\,10.\,1931 ebd.), \emph{Schriftsteller, Mediziner}!Lebendige Stunden. Vier Einakter@\strich\emph{Lebendige Stunden. Vier Einakter}|pwv}{ }\strikeout{D\textcolor{gray}{ei}} auch Deine \label{K_L03090-7v}\edtext{Reiſe nach Berlin\oindex{Berlin@\textbf{Berlin}, \emph{Hauptstadt}|pw} verſchoben}{\lemma{\textnormal{\emph{Reise … verschoben}}}\Cendnote{\textnormal{Schnitzler war letztendlich vom 28. 12. 1901 bis zum 6. 1. 1902 in Berlin\oindex{Berlin@\textbf{Berlin}, \emph{Hauptstadt}|pwk}.}}}\label{K_L03090-7} iſt.\pend
           
\pstart
           Haſt Du den \label{K_L03090-8v}\edtext{\textsc{Chamfort\pwindex{Chamfort, Sébastien Roch Nicolas 6.\,4.\,1741 Clermont – 13.\,4.\,1794 Paris@\textsc{Chamfort, Sébastien Roch Nicolas} (6.\,4.\,1741 Clermont – 13.\,4.\,1794 Paris), \emph{Schriftsteller}!Œuvres choisies de N. Chamfort, publiées avec préface, notes et tables@\strich\emph{Œuvres choisies de N. Chamfort, publiées avec préface, notes et tables}|pwv}\pwindex{Chamfort, Sébastien Roch Nicolas 6.\,4.\,1741 Clermont – 13.\,4.\,1794 Paris@\textsc{Chamfort, Sébastien Roch Nicolas} (6.\,4.\,1741 Clermont – 13.\,4.\,1794 Paris), \emph{Schriftsteller}|pw}}}{\lemma{\textnormal{\emph{Chamfort}}}\Cendnote{\textnormal{Siehe XXXX Auszeichnungsfehler: Dokument L03085 nicht gefunden.
               }}}\label{K_L03090-8} nun endlich erhalten? Und haſt Du ihn geleſen? Lies’ auch die eben von \textsc{Griesebach\pwindex{Grisebach, Eduard 9.\,10.\,1845 Göttingen – 22.\,3.\,1906 Charlottenburg@\textsc{Grisebach, Eduard} (9.\,10.\,1845 Göttingen – 22.\,3.\,1906 Charlottenburg), \emph{Schriftsteller, Diplomat, Jurist}|pw}} herausgegebenen \label{K_L03090-9v}\edtext{Geſpräche mit \textsc{Schopenhauer\pwindex{Schopenhauer, Arthur 22.\,2.\,1788 Danzig – 21.\,9.\,1860 Frankfurt am Main@\textsc{Schopenhauer, Arthur} (22.\,2.\,1788 Danzig – 21.\,9.\,1860 Frankfurt am Main), \emph{Philosoph}|pw}}\pwindex{Schopenhauer, Arthur 22.\,2.\,1788 Danzig – 21.\,9.\,1860 Frankfurt am Main@\textsc{Schopenhauer, Arthur} (22.\,2.\,1788 Danzig – 21.\,9.\,1860 Frankfurt am Main), \emph{Philosoph}!Schopenhauer’s Gespräche und Selbstgespräche: Nach der Handschrift eis heauton@\strich\emph{Schopenhauer’s Gespräche und Selbstgespräche: Nach der Handschrift eis heauton}|pw}}{\lemma{\textnormal{\emph{Gespräche mit Schopenhauer}}}\Cendnote{\textnormal{\emph{Schopenhauer’s Gespräche und Selbstgespräche:
                        Nach der Handschrift eis heauton}\pwindex{Schopenhauer, Arthur 22.\,2.\,1788 Danzig – 21.\,9.\,1860 Frankfurt am Main@\textsc{Schopenhauer, Arthur} (22.\,2.\,1788 Danzig – 21.\,9.\,1860 Frankfurt am Main), \emph{Philosoph}!Schopenhauer’s Gespräche und Selbstgespräche: Nach der Handschrift eis heauton@\strich\emph{Schopenhauer’s Gespräche und Selbstgespräche: Nach der Handschrift eis heauton}|pwk}. Herausgegeben von Eduard Grisebach\pwindex{Grisebach, Eduard 9.\,10.\,1845 Göttingen – 22.\,3.\,1906 Charlottenburg@\textsc{Grisebach, Eduard} (9.\,10.\,1845 Göttingen – 22.\,3.\,1906 Charlottenburg), \emph{Schriftsteller, Diplomat, Jurist}|pwk}. Berlin\oindex{Berlin@\textbf{Berlin}, \emph{Hauptstadt}|pwk}: \emph{Ernst Hofmann {\kaufmannsund} Co.}\orgindex{Ernst Hofmann und Co.@Ernst Hofmann {\kaufmannsund}  Co.|pwk}{ }1898. Eine Lektüre durch Schnitzler ist
                  nicht belegt.}}}\label{K_L03090-9}.\pend
           
\pstart
           Leb’ wohl für heut! Viele treue Grüße! {\\[\baselineskip]}Dein {\\[\baselineskip]}\spacefill\mbox{Paul Goldmann.}\pend
           \leftskip=0em{}\selectlanguage{ngerman}\endnumbering\briefempfaengerindex{Schnitzler, Arthur@\textsc{Schnitzler, Arthur}!zzzGoldmann, Paul@\emph{von Paul Goldmann}!1901-11-092@{9. 11. [1901]}|)be}\mylabel{L03090h}  \newcommand{\dateiname}{L03090}\newcommand{\titel}{Paul Goldmann an Arthur Schnitzler, 9. 11. [1901]}\newcommand{\editorInnen}{Martin Anton Müller und Laura Untner}%% latex-leseansicht-abspann.tex
%% Abspann für die Leseansicht.
%% Der Schalter \ifkorrekturansicht ist bereits durch den Vorspann gesetzt.

%% latex-abspann.tex
%% Gemeinsamer Abspann für Korrekturansicht und Leseansicht.
%% Setzt den Schalter \ifkorrekturansicht voraus (gesetzt in den
%% einbindenden Dateien latex-korrekturansicht-abspann.tex bzw.
%% latex-leseansicht-abspann.tex).
%% ---------------------------------------------------------------

\normalsize

% Das esempio-Environment wird nur in der Leseansicht benötigt
\ifkorrekturansicht\else
\newenvironment{esempio}[3]%
{
    \vspace{1.5ex}
    \rlap{\underline{#1}}
    \par
    \setlength{\parindent}{0cm}
    \nopagebreak
    \leftskip=#2cm
    \rightskip=#3cm
}
{
    \par
}
\fi

\doendnotes{C}
\bigskip
\vfill

\clearpage

\footnotesize

\ifkorrekturansicht
  \lohead{\textsc{register}}
\fi

% theindex-Environment neu definieren ohne reledmac
\makeatletter
\renewenvironment{theindex}{%
  \ifkorrekturansicht
    \section*{\indexname}%
  \else
    \subsubsection*{Index der erwähnten Entitäten}%
  \fi
  \setlength{\parindent}{0pt}%
  \setlength{\parskip}{0pt plus 0.3pt}%
  \let\item\@idxitem
}{%
  \ifkorrekturansicht\clearpage\fi
}
\makeatother

\IfFileExists{\jobname-pw.ind}{\input{\jobname-pw.ind}}{}

% Quellenangabe nur in der Leseansicht
\ifkorrekturansicht\else
% Fallback-Definitionen, falls die .tex-Datei \titel etc. nicht gesetzt hat
\providecommand{\titel}{}
\providecommand{\editorInnen}{}
\providecommand{\dateiname}{\jobname}

\vspace{3cm}

\vfill

\footnotesize
\textsc{Quelle}: \titel. Herausgegeben von {\editorInnen}. In: \emph{Arthur Schnitzler: Briefwechsel mit Autorinnen und Autoren}.
 Digitale Edition, https://schnitzler-briefe.acdh.oeaw.ac.at/{\dateiname}.html (Stand \today)
\fi

\end{document}


