%% latex-korrekturansicht-vorspann.tex
%% Vorspann für die Korrekturansicht.
%% Lädt die gemeinsame Datei latex-vorspann.tex mit gesetztem Schalter.

\newif\ifkorrekturansicht
\korrekturansichttrue

\input{../tex-inputs/latex-vorspann}


\section[ Felix Salten an Arthur Schnitzler, 12. 7. 1909]{L03502 Felix Salten an Arthur Schnitzler, 12. 7. 1909}
\nopagebreak\mylabel{L03502v}
\rehead{ }\normalsize\beginnumbering\briefempfaengerindex{Schnitzler, Arthur@\textsc{Schnitzler, Arthur}!zzzSalten, Felix@\emph{von Felix Salten}!1909-07-123@{12. 7. 1909}|(be}
\toendnotes[C]{\smallbreak\pagebreak[2]}\Standort{CUL, Schnitzler, B 89, B 1.}
\physDesc{Postkarte, 561 Zeichen
\newline{}Handschrift: schwarze Tinte, lateinische Kurrent
\newline{}Versand: Stempel: »\nobreak{}\oindex{Grado@\textbf{Grado}, \emph{P.PPLA3}|pwk}{[}Gra{]}\textcolor{gray}{d}o\nobreak{}«.  
\newline{}Schnitzler: mit Bleistift Vermerk: »\textsc{Salten}« 
\newline{}Ordnung: mit Bleistift von unbekannter Hand nummeriert: »252« }\toendnotes[C]{\smallbreak}\pstart{}{\pb}Salten, Grado\oindex{Grado@\textbf{Grado}, \emph{P.PPLA3}|pw}\pend{}\pstart{}Villa Bauer\oindex{Villa Bauer@\textbf{Villa Bauer}, \emph{Wohngebäude (K.WHS)}|pw}.\pend{}{\bigskip}\pstart{}Herrn\pend{}\pstart{}D\textsuperscript{r} Arthur Schnitzler\pend{}\pstart{}Wien\oindex{Wien@\textbf{Wien}, \emph{A.ADM2}|pw}\pend{}\pstart{}XVIII. Spöttelgaße 7\oindex{Edmund-Weiss-Gasse 7@\textbf{Edmund-Weiß-Gasse 7}, \emph{Wohngebäude (K.WHS)}|pw}\pend{}{\bigskip}\vspace{1em}
\pstart{}{\pb}Lieber,\pend\vspace{0.5em}
\pstart
           es tut uns herzlich leid, dass der arme \label{K_L03502-1v}\edtext{Heini\pwindex{Schnitzler, Heinrich 09.08.1902 – 12.07.1982@\textsc{Schnitzler, Heinrich} (09.08.1902 – 12.07.1982), \emph{Regisseur/Regisseurin, Schauspieler/Schauspielerin}|pw} von diesem bösen Husten geplagt}{\lemma{\textnormal{\emph{Heini … geplagt}}}\Cendnote{\textnormal{Vgl. A. S.: \emph{Tagebuch}, 1. 7. 1909.
               }}}\label{K_L03502-1} ist, und dass Sie wie Frau Olga\pwindex{Schnitzler, Olga 17.01.1882 – 13.01.1970@\textsc{Schnitzler, Olga} (17.01.1882 – 13.01.1970), \emph{Schauspieler/Schauspielerin, Sänger/Sängerin}|pw} nun
               auch diese Sorge haben. Wir wüßten sehr gerne, wie es Heini\pwindex{Schnitzler, Heinrich 09.08.1902 – 12.07.1982@\textsc{Schnitzler, Heinrich} (09.08.1902 – 12.07.1982), \emph{Regisseur/Regisseurin, Schauspieler/Schauspielerin}|pw} geht, und wären für eine Nachricht dankbar!\pend
           
\pstart
           Annerle\pwindex{Rehmann, Anna Katharina 18.08.1904 – 27.03.1977@\textsc{Rehmann, Anna Katharina} (18.08.1904 – 27.03.1977), \emph{Schauspieler/Schauspielerin, Übersetzer/Übersetzerin}|pw} hat uns vor ein paar Tagen einen
               großen Schreck bereitet, indem sie über 40° Fieber bekam. Zweimal. Der Arzt\pwindex{?? [Mediziner in Grado] @\textsc{?? [Mediziner in Grado]}|pwv} glaubt, an Malaria, was
               sich heute entscheiden müßte.\pend
           
\pstart
           Wir reisen Donnerstag{ }früh und sind Freitag in Landro\oindex{Hoehlenstein@\textbf{Höhlenstein}, \emph{P.PPLQ}|pw}!\pend
           
\pstart
           Alles herzliche von uns zu Ihnen {\\[\baselineskip]}Ihr {\\[\baselineskip]}\spacefill\mbox{Salten}\pend
           \leftskip=0em{}
\pstart
           Grado\oindex{Grado@\textbf{Grado}, \emph{P.PPLA3}|pw}, 12. Juli 09\pend
           \selectlanguage{ngerman}\endnumbering\briefempfaengerindex{Schnitzler, Arthur@\textsc{Schnitzler, Arthur}!zzzSalten, Felix@\emph{von Felix Salten}!1909-07-123@{12. 7. 1909}|)be}\mylabel{L03502h}  \normalsize

\doendnotes{C}
\bigskip
\vfill

\clearpage

\footnotesize

\lohead{\textsc{register}}

% Definiere theindex-Environment komplett neu ohne reledmac
\makeatletter
\renewenvironment{theindex}{%
  \section*{\indexname}%
  \setlength{\parindent}{0pt}%
  \setlength{\parskip}{0pt plus 0.3pt}%
  \let\item\@idxitem
}{%
  \clearpage
}
\makeatother

\IfFileExists{\jobname-pw.ind}{\input{\jobname-pw.ind}}{}

\end{document}

      