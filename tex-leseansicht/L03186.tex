%% latex-leseansicht-vorspann.tex
%% Vorspann für die Leseansicht.
%% Lädt die gemeinsame Datei latex-vorspann.tex mit nicht gesetztem Schalter.

\newif\ifkorrekturansicht
\korrekturansichtfalse

\input{../tex-inputs/latex-vorspann}

\begin{center}
            \textcolor{red}{ENTWURF, NICHT FERTIG KORRIGIERT}
                      \end{center}
            
         
         \renewcommand{\erwaehntePersonen}{Personen: Cesare Lombroso}
         \renewcommand{\erwaehnteOrte}{Orte: Kärntnerring, Unterach am Attersee, Wien}
         \renewcommand{\erwaehnteWerke}{Werke: Felix Salten}
               \section[Felix Salten an Arthur Schnitzler, 10. 8. 1892]{ Felix Salten an Arthur Schnitzler, 10. 8. 1892}\nopagebreak\mylabel{v}\rehead{ }\begin{ledgroupsized}[t]{13cm}\normalsize\beginnumbering \toendnotes[C]{\smallbreak\pagebreak[2]} \Standort{CUL, Schnitzler, B 89, A 1.}
\physDesc{Kartenbrief
\newline{}Handschrift: schwarze Tinte, lateinische Kurrent\newline{}Versand: Stempel: »\nobreak{}\oindex{Unterach am Attersee@\textbf{Unterach am Attersee}|pwk}Unterach am
                                       \textcolor{gray}{Att}ersee, 10/8 92\nobreak{}«.  \newline{}Ordnung: mit Bleistift von unbekannter Hand nummeriert: »15« }\toendnotes[C]{\smallbreak}\pstart{}{\pb}Herrn D\textsuperscript{r} Arthur Schnitzler\pend{}\pstart{}\strikeout{Unterach}\oindex{Unterach am Attersee@\textbf{Unterach am Attersee}|pw}{ }Wien\oindex{Wien@\textbf{Wien}|pw}\pend{}\pstart{}I. Kärntnerring 12\oindex{Kaerntnerring@\textbf{Kärntnerring}|pw}\pend{}{\bigskip}\pstart
           {\pb}Unterach\oindex{Unterach am Attersee@\textbf{Unterach am Attersee}|pw},
                  10. VIII. 92.\pend
           \pstart
           Ich habe viele Menschen, die mir werth sind, die ich schätze und die mir sympathisch
               sind, ich habe aber nur \uline{einen}, den ich \uline{wirklich liebe} und nur einen, dem \uline{ich wirklich} Freund bin, und das sind Sie! Bitte Sie \label{K_L03186-77v}\edtext{\uline{aufrichtigst} schreiben Sie mir umgehend \uline{Alles}}{\lemma{\textnormal{\emph{aufrichtigst … Alles}}}\Cendnote{\textnormal{Salten\pwindex{Salten, Felix 06.09.1869 – 08.10.1945@\textsc{Salten, Felix} (06.09.1869 – 08.10.1945), \emph{Schriftsteller, Journalist}|pwk} hatte zu dieser Zeit die Erlaubnis,
                  ohne Rücksprache in Schnitzler\pwindex{Schnitzler, Arthur 15.05.1862 – 21.10.1931@\textsc{Schnitzler, Arthur} (15.05.1862 – 21.10.1931), \emph{Schriftsteller, Mediziner}|pwk}s Wohnung zu
                  übernachten. Schnitzler\pwindex{Schnitzler, Arthur 15.05.1862 – 21.10.1931@\textsc{Schnitzler, Arthur} (15.05.1862 – 21.10.1931), \emph{Schriftsteller, Mediziner}|pwk} bemerkte, dass
                  Schmuckstücke und vor allem Bücher verschwanden. Der letzte Beweis gegen Salten\pwindex{Salten, Felix 06.09.1869 – 08.10.1945@\textsc{Salten, Felix} (06.09.1869 – 08.10.1945), \emph{Schriftsteller, Journalist}|pwk} bildete das Exemplar eines Buches von
                     Cesare Lombroso\pwindex{Lombroso, Cesare 18.11.1836 – 19.10.1909@\textsc{Lombroso, Cesare} (18.11.1836 – 19.10.1909), \emph{Mediziner, Psychologe, Anthropologe}|pwk} mit Seitennotizen von
                     Schnitzler\pwindex{Schnitzler, Arthur 15.05.1862 – 21.10.1931@\textsc{Schnitzler, Arthur} (15.05.1862 – 21.10.1931), \emph{Schriftsteller, Mediziner}|pwk}, das er Salten\pwindex{Salten, Felix 06.09.1869 – 08.10.1945@\textsc{Salten, Felix} (06.09.1869 – 08.10.1945), \emph{Schriftsteller, Journalist}|pwk} geliehen hatte, und in einem Antiquariat wiederfand.
                        (Arthur Schnitzler\pwindex{Schnitzler, Arthur 15.05.1862 – 21.10.1931@\textsc{Schnitzler, Arthur} (15.05.1862 – 21.10.1931), \emph{Schriftsteller, Mediziner}|pwk}: \emph{Felix Salten}\pwindex{Schnitzler, Arthur 15.05.1862 – 21.10.1931@\textsc{Schnitzler, Arthur} (15.05.1862 – 21.10.1931), \emph{Schriftsteller, Mediziner}!Felix SaltenNone@\strich\emph{Felix Salten} {[}None{]}|pwk}, unveröffentlichtes
                     Typoskript, \emph{DLA}, HS.NZ85.1.116)}}}\label{K_L03186-77h}, was Sie
               mir gegenüber auf der Seele haben, schreiben Sie es mir bitte gleich, denn ich werde
               hier nicht ruhig sein, bis ich nicht Alles von Ihnen gehört. Dass ich meine Abreise
               nicht dennoch um einen Tag hinausgeschoben thut mir jetzt \uuline{sehr}{ }\uuline{leid}. Ich hoffe Sie nehmen sich die halbe Stunde
               Zeit, damit wir wieder in klare Luft kommen. Das ist nun mein ungeduldiger Wun\damage{sch}\pend
           \pstart Ihr aufrichtig ergeb\damage{ener}\pend{}
         
         \endnumbering\mylabel{h}\end{ledgroupsized}\begin{anhang}\end{anhang}\newcommand{\dateiname}{L03186}\newcommand{\titel}{Felix Salten an Arthur Schnitzler, 10. 8. 1892}\newcommand{\editorInnen}{Martin Anton Müller und Laura Untner}%% latex-leseansicht-abspann.tex
%% Abspann für die Leseansicht.
%% Der Schalter \ifkorrekturansicht ist bereits durch den Vorspann gesetzt.

%% latex-abspann.tex
%% Gemeinsamer Abspann für Korrekturansicht und Leseansicht.
%% Setzt den Schalter \ifkorrekturansicht voraus (gesetzt in den
%% einbindenden Dateien latex-korrekturansicht-abspann.tex bzw.
%% latex-leseansicht-abspann.tex).
%% ---------------------------------------------------------------

\normalsize

% Das esempio-Environment wird nur in der Leseansicht benötigt
\ifkorrekturansicht\else
\newenvironment{esempio}[3]%
{
    \vspace{1.5ex}
    \rlap{\underline{#1}}
    \par
    \setlength{\parindent}{0cm}
    \nopagebreak
    \leftskip=#2cm
    \rightskip=#3cm
}
{
    \par
}
\fi

\doendnotes{C}
\bigskip
\vfill

\clearpage

\footnotesize

\ifkorrekturansicht
  \lohead{\textsc{register}}
\fi

% theindex-Environment neu definieren ohne reledmac
\makeatletter
\renewenvironment{theindex}{%
  \ifkorrekturansicht
    \section*{\indexname}%
  \else
    \subsubsection*{Index der erwähnten Entitäten}%
  \fi
  \setlength{\parindent}{0pt}%
  \setlength{\parskip}{0pt plus 0.3pt}%
  \let\item\@idxitem
}{%
  \ifkorrekturansicht\clearpage\fi
}
\makeatother

\IfFileExists{\jobname-pw.ind}{\input{\jobname-pw.ind}}{}

% Quellenangabe nur in der Leseansicht
\ifkorrekturansicht\else
% Fallback-Definitionen, falls die .tex-Datei \titel etc. nicht gesetzt hat
\providecommand{\titel}{}
\providecommand{\editorInnen}{}
\providecommand{\dateiname}{\jobname}

\vspace{3cm}

\vfill

\footnotesize
\textsc{Quelle}: \titel. Herausgegeben von {\editorInnen}. In: \emph{Arthur Schnitzler: Briefwechsel mit Autorinnen und Autoren}.
 Digitale Edition, https://schnitzler-briefe.acdh.oeaw.ac.at/{\dateiname}.html (Stand \today)
\fi

\end{document}


      