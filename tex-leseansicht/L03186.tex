%% latex-leseansicht-vorspann.tex
%% Vorspann für die Leseansicht.
%% Lädt die gemeinsame Datei latex-vorspann.tex mit nicht gesetztem Schalter.

\newif\ifkorrekturansicht
\korrekturansichtfalse

\input{../tex-inputs/latex-vorspann}


         
         \renewcommand{\erwaehntePersonen}{Personen:  ?? [Hausmeister von Kärntnerring 12/Bösendorferstraße 11],  ?? [Hausmeisterin von Kärntnerring 12/Bösendorferstraße 11], Marie Glümer, Cesare Lombroso, Felix Salten}
         \renewcommand{\erwaehnteInstitutionen}{Institutionen: J. Deibler, Buchhandel und Antiquariat}
         \renewcommand{\erwaehnteOrte}{Orte: Allgemeine Poliklinik, Herrengasse, Kärntnerring 12/Bösendorferstraße 11, Unterach am Attersee, Wien}
         \renewcommand{\erwaehnteWerke}{Werke: Genie und Irrsinn in ihren Beziehungen zum Gesetz, zur Kritik und zur Geschichte}
               \section[Felix Salten an Arthur Schnitzler, 10. 8. 1892]{ Felix Salten an Arthur Schnitzler, 10. 8. 1892}\nopagebreak\mylabel{v}\rehead{ }\begin{ledgroupsized}[t]{13cm}\normalsize\beginnumbering\briefempfaengerindex{Schnitzler, Arthur@\textsc{Schnitzler, Arthur}!zzzSalten, Felix@\emph{von Felix Salten}!1892-08-102@{10. 8. 1892}|(be} \toendnotes[C]{\smallbreak\pagebreak[2]} \Standort{CUL, Schnitzler, B 89, A 1.}
\physDesc{Kartenbrief, 720 Zeichen
\newline{}Handschrift: schwarze Tinte, lateinische Kurrent
\newline{}Versand: Stempel: »\nobreak{}\oindex{Unterach am Attersee@\textbf{Unterach am Attersee}|pwk}Unterach am
                                       {[}Att{]}ersee, 10 8 92\nobreak{}«.  
\newline{}Ordnung: mit Bleistift von unbekannter Hand nummeriert: »15« }\toendnotes[C]{\smallbreak}\pstart{}{\pb}Herrn D\textsuperscript{r} Arthur Schnitzler\pend{}\pstart{}\strikeout{Unterach}\oindex{Unterach am Attersee@\textbf{Unterach am Attersee}|pw}{ }Wien\oindex{Wien@\textbf{Wien}|pw}\pend{}\pstart{}I. Kärntnerring 12\oindex{Kaerntnerring 12/Boesendorferstrasse 11@\textbf{Kärntnerring 12/Bösendorferstraße 11}|pw}\pend{}{\bigskip}\pstart
           \raggedleft{}{\pb}Unterach\oindex{Unterach am Attersee@\textbf{Unterach am Attersee}|pw}, 10. VIII. 92.\pend
           \pstart
           Ich habe viele Menschen, die mir werth sind, die ich schätze und die mir sympathisch
               sind, ich habe aber nur \uline{einen}{[},{]}
               den ich \uline{wirklich liebe} und nur
                  einen{[},{]} dem \uline{ich wirklich}
               Freund bin, und das sind Sie! Bitte Sie \label{K_L03186-1v}\edtext{\uline{aufrichtigst} schreiben Sie mir umgehend \uline{Alles}}{\lemma{\textnormal{\emph{aufrichtigst … Alles}}}\Cendnote{\textnormal{In undatierten Erinnerungen Schnitzlers\pwindex{Schnitzler, Arthur 15.05.1862 – 21.10.1931@\textsc{Schnitzler, Arthur} (15.05.1862 – 21.10.1931), \emph{Schriftsteller, Mediziner}|pwk} wird der Hintergrund beleuchtet:
                     Salten\pwindex{Salten, Felix 06.09.1869 – 08.10.1945@\textsc{Salten, Felix} (06.09.1869 – 08.10.1945), \emph{Schriftsteller, Journalist, Chefredakteur}|pwk} hatte zu dieser Zeit die
                  Erlaubnis, ohne Rücksprache in Schnitzlers\pwindex{Schnitzler, Arthur 15.05.1862 – 21.10.1931@\textsc{Schnitzler, Arthur} (15.05.1862 – 21.10.1931), \emph{Schriftsteller, Mediziner}|pwk}{ }Wohnung\oindex{Kaerntnerring 12/Boesendorferstrasse 11@\textbf{Kärntnerring 12/Bösendorferstraße 11}|pwkv} zu übernachten,
                  wenn er die letzte Straßenbahn versäumt hatte. »Ich verlasse das Haus
                        meist früher als er, da ich auf die Poliklinik\oindex{Allgemeine Poliklinik@\textbf{Allgemeine Poliklinik}|pw} muss. Er schläft weiter. Bald merke ich, dass mir
                        allerlei wegkommt, ein Ring, eine Nadel, auffallend viel Bücher.{ / }M. G.\pwindex{Gluemer, Marie 03.07.1867 – 16.11.1925@\textsc{Glümer, Marie} (03.07.1867 – 16.11.1925), \emph{Schauspielerin}|pw} hat sofort einen Verdacht,
                        den ich ohne rechte Ueberzeugung bekämpfe.{ / }Er leiht sich Bücher aus, ohne sie mir zurückzugeben.{ / }Indess geht meinem Hausmeister\pwindex{?? [Hausmeister von Kaerntnerring 12/Boesendorferstrasse 11] @\textsc{?? [Hausmeister von Kärntnerring 12/Bösendorferstraße 11]}|pwv} sein junges Weibchen\pwindex{?? [Hausmeisterin von Kaerntnerring 12/Boesendorferstrasse 11] @\textsc{?? [Hausmeisterin von Kärntnerring 12/Bösendorferstraße 11]}|pwv} mit einem Liebhaber durch und
                           jener{[},{]} von den Diebstählen bei mir in Kenntnis
                        gesetzt, spricht den Verdacht aus, dass seine Ungetreue\pwindex{?? [Hausmeisterin von Kaerntnerring 12/Boesendorferstrasse 11] @\textsc{?? [Hausmeisterin von Kärntnerring 12/Bösendorferstraße 11]}|pwv} wie allerlei aus des
                           Ehegatten\pwindex{?? [Hausmeister von Kaerntnerring 12/Boesendorferstrasse 11] @\textsc{?? [Hausmeister von Kärntnerring 12/Bösendorferstraße 11]}|pwv}
                        Wohnung auch manches aus der meinen entwendet haben könnte, die sie
                        aufzuräumen pflegte.{ / }Einmal beim Antiquariat\orgindex{J. Deibler, Buchhandel und Antiquariat@J. Deibler, Buchhandel und Antiquariat|pwv} in der Herrengasse\oindex{Herrengasse@\textbf{Herrengasse}|pw}{[},{]}{ }Deibler\orgindex{J. Deibler, Buchhandel und Antiquariat@J. Deibler, Buchhandel und Antiquariat|pw}{[},{]}
                        entdecke ich ein Buch\pwindex{\textcolor{red}{\textsuperscript{XXXX1 indx}}!Genie und Irrsinn in ihren Beziehungen zum Gesetz, zur Kritik und zur Geschichte1887@\strich\emph{Genie und Irrsinn in ihren Beziehungen zum Gesetz, zur Kritik und zur Geschichte} {[}Übersetzung, 1887{]}|pwuv} von Lombroso\pwindex{Lombroso, Cesare 18.11.1836 – 19.10.1909@\textsc{Lombroso, Cesare} (18.11.1836 – 19.10.1909), \emph{Mediziner, Psychologe, Anthropologe}|pw}, das ich F. S.\pwindex{Salten, Felix 06.09.1869 – 08.10.1945@\textsc{Salten, Felix} (06.09.1869 – 08.10.1945), \emph{Schriftsteller, Journalist, Chefredakteur}|pw}
                        geliehen. Um mich zu vergewissern, lasse ich mir das Buch\pwindex{\textcolor{red}{\textsuperscript{XXXX1 indx}}!Genie und Irrsinn in ihren Beziehungen zum Gesetz, zur Kritik und zur Geschichte1887@\strich\emph{Genie und Irrsinn in ihren Beziehungen zum Gesetz, zur Kritik und zur Geschichte} {[}Übersetzung, 1887{]}|pwuv} zeigen und
                        entdecke gewisse Schriftzeichen, die ich bei Gelegenheit kritischer
                        Besprechung eingetragen, so dass ein Irrtum ausgeschlossen ist.{ / }Ich begebe mich zu F. S.\pwindex{Salten, Felix 06.09.1869 – 08.10.1945@\textsc{Salten, Felix} (06.09.1869 – 08.10.1945), \emph{Schriftsteller, Journalist, Chefredakteur}|pw}, er
                        liegt noch zu Bett, ich ersuche ihn um Rückgabe meiner Bücher, insbesondere
                        des Lo{[}m{]}broso\pwindex{\textcolor{red}{\textsuperscript{XXXX1 indx}}!Genie und Irrsinn in ihren Beziehungen zum Gesetz, zur Kritik und zur Geschichte1887@\strich\emph{Genie und Irrsinn in ihren Beziehungen zum Gesetz, zur Kritik und zur Geschichte} {[}Übersetzung, 1887{]}|pwu}\pwindex{Lombroso, Cesare 18.11.1836 – 19.10.1909@\textsc{Lombroso, Cesare} (18.11.1836 – 19.10.1909), \emph{Mediziner, Psychologe, Anthropologe}|pw}, er erklärt,
                        dass er leider seinen Schlüssel verloren habe. Es kommt wohl nicht zu einer
                        Aussprache, doch zu einer Unterredung, in der er meinen begründeten Verdacht
                        zu verkennen nicht mehr in der Lage ist. Er schreibt mir einen halb
                        aufrichtigen, halb reuigen Brief, den er Jahre später von mir
                        zurückerbittet, von dem ich mir aber eine Abschrift behalte.«
                        (\emph{DLA}, A:Schnitzler, Verschiedenes
                           Autobiographisches, »Felix Salten«, HS.NZ851.116) Die erwähnte Abschrift findet sich hier: Felix Salten an Arthur Schnitzler, 17. 8. 1892.}}}\label{K_L03186-1h}, was Sie mir
               gegenüber auf der Seele haben, schreiben Sie es mir bitte gleich, denn ich werde hier
               nicht ruhig sein, bis ich nicht Alles von Ihnen gehört. Dass ich meine Abreise nicht
               dennoch um einen Tag hinausgeschoben{[},{]} thut mir jetzt \uuline{sehr}{ }\uuline{leid}. Ich hoffe Sie nehmen sich die halbe Stunde
               Zeit, damit wir wieder in klare Luft kommen. Das ist nun mein ungeduldiger Wun\damage{sch.}\pend
           \pstart Ihr aufrichtig er\textcolor{gray}{g}ebe\damage{ner}\pend{}
         
         \endnumbering\mylabel{h}\end{ledgroupsized}  \newcommand{\dateiname}{L03186}\newcommand{\titel}{Felix Salten an Arthur Schnitzler, 10. 8. 1892}\newcommand{\editorInnen}{Martin Anton Müller und Laura Untner}%% latex-leseansicht-abspann.tex
%% Abspann für die Leseansicht.
%% Der Schalter \ifkorrekturansicht ist bereits durch den Vorspann gesetzt.

%% latex-abspann.tex
%% Gemeinsamer Abspann für Korrekturansicht und Leseansicht.
%% Setzt den Schalter \ifkorrekturansicht voraus (gesetzt in den
%% einbindenden Dateien latex-korrekturansicht-abspann.tex bzw.
%% latex-leseansicht-abspann.tex).
%% ---------------------------------------------------------------

\normalsize

% Das esempio-Environment wird nur in der Leseansicht benötigt
\ifkorrekturansicht\else
\newenvironment{esempio}[3]%
{
    \vspace{1.5ex}
    \rlap{\underline{#1}}
    \par
    \setlength{\parindent}{0cm}
    \nopagebreak
    \leftskip=#2cm
    \rightskip=#3cm
}
{
    \par
}
\fi

\doendnotes{C}
\bigskip
\vfill

\clearpage

\footnotesize

\ifkorrekturansicht
  \lohead{\textsc{register}}
\fi

% theindex-Environment neu definieren ohne reledmac
\makeatletter
\renewenvironment{theindex}{%
  \ifkorrekturansicht
    \section*{\indexname}%
  \else
    \subsubsection*{Index der erwähnten Entitäten}%
  \fi
  \setlength{\parindent}{0pt}%
  \setlength{\parskip}{0pt plus 0.3pt}%
  \let\item\@idxitem
}{%
  \ifkorrekturansicht\clearpage\fi
}
\makeatother

\IfFileExists{\jobname-pw.ind}{\input{\jobname-pw.ind}}{}

% Quellenangabe nur in der Leseansicht
\ifkorrekturansicht\else
% Fallback-Definitionen, falls die .tex-Datei \titel etc. nicht gesetzt hat
\providecommand{\titel}{}
\providecommand{\editorInnen}{}
\providecommand{\dateiname}{\jobname}

\vspace{3cm}

\vfill

\footnotesize
\textsc{Quelle}: \titel. Herausgegeben von {\editorInnen}. In: \emph{Arthur Schnitzler: Briefwechsel mit Autorinnen und Autoren}.
 Digitale Edition, https://schnitzler-briefe.acdh.oeaw.ac.at/{\dateiname}.html (Stand \today)
\fi

\end{document}


      