%% latex-korrekturansicht-vorspann.tex
%% Vorspann für die Korrekturansicht.
%% Lädt die gemeinsame Datei latex-vorspann.tex mit gesetztem Schalter.

\newif\ifkorrekturansicht
\korrekturansichttrue

\input{../tex-inputs/latex-vorspann}


\section[ Arthur Schnitzler an Felix Salten, 22. 7. 1923]{L03020 Arthur Schnitzler an Felix Salten, 22. 7. 1923}
\nopagebreak\mylabel{L03020v}
\rehead{ }\normalsize\beginnumbering\briefempfaengerindex{Salten, Felix@\textsc{Salten, Felix}!zzzSchnitzler, Arthur@\emph{von Arthur Schnitzler}!1923-07-221@{22. 7. 1923}|(be}
\toendnotes[C]{\smallbreak\pagebreak[2]}\Standort{Wienbibliothek im Rathaus, ZPH 1681, 2.1.516.}
\physDesc{Postkarte, 472 Zeichen
\newline{}Handschrift: Bleistift, lateinische Kurrent
\newline{}Versand: Stempel: »\nobreak{}18/\textsubscript{1} Wien 11\textcolor{gray}{0}, 24. VII. 23, 9\nobreak{}«.  
\newline{}Ordnung: mit Bleistift von unbekannter Hand nummeriert: »5« }
\buchAbdrucke{\weitereDrucke{Arthur Schnitzler: \emph{Briefe 1913–1931}. Frankfurt am Main: \emph{S. Fischer} 1984, S. 322–323.} }\toendnotes[C]{\smallbreak}\pstart{}{\pb}\label{T_L03020-1v}\edtext{\textcolor{gray}{\textbf{A. S.}}}{\lemma{\textnormal{\emph{A. S.}}}\Cendnote{\textnormal{ovaler Absenderkleber}}}\label{T_L03020-1}\pend{}\pstart{}\textcolor{gray}{\textbf{WIEN, XVIII.}}\oindex{XVIII., Waehring@\textbf{XVIII., Währing}, \emph{A.ADM3}|pw}\pend{}\pstart{}\textcolor{gray}{\textbf{STERNWARTESTR. 71}}\oindex{Sternwartestrasse 71@\textbf{Sternwartestraße 71}, \emph{Wohngebäude (K.WHS)}|pw}\pend{}{\bigskip}\pstart{}Ob. Oe.\oindex{Oberoesterreich@\textbf{Oberösterreich}, \emph{A.ADM1}|pw}\pend{}\pstart{}Herrn\pend{}\pstart{}Felix Salten\pend{}\pstart{}Unterach\oindex{Unterach am Attersee@\textbf{Unterach am Attersee}, \emph{P.PPL}|pw} am Attersee\oindex{Attersee@\textbf{Attersee}, \emph{H.LK}|pw}\pend{}\pstart{}Berghof\oindex{Berghof@\textbf{Berghof}, \emph{Wohngebäude (K.WHS)}|pw}\pend{}{\bigskip}\vspace{1em}
\pstart
           \raggedleft{}{\pb}Wien\oindex{Wien@\textbf{Wien}, \emph{A.ADM2}|pw}, 22. 7. 23\pend
           \vspace{0.5em}
\pstart
           lieber, lassen Sie sich die Hand drücken für Ihr \textcolor{gray}{schönes}{ }\label{K_L03020-1v}\edtext{Voltaire\pwindex{Voltaire 21.11.1694 – 30.05.1778@\textsc{Voltaire} (21.11.1694 – 30.05.1778), \emph{Schriftsteller/Schriftstellerin, Philosoph/Philosophin}|pw}
                  Feu{[}i{]}lleton\pwindex{Voltaire@\emph{Voltaire}|pwv}}{\lemma{\textnormal{\emph{Voltaire
                  Feuilleton}}}\Cendnote{\textnormal{Felix Salten\pwindex{Salten, Felix 06.09.1869 – 08.10.1945@\textsc{Salten, Felix} (06.09.1869 – 08.10.1945), \emph{Schriftsteller/Schriftstellerin, Journalist/Journalistin, Chefredakteur/Chefredakteurin}|pwk}: \emph{Voltaire}\pwindex{Voltaire@\emph{Voltaire}|pwk}. In: \emph{Neue
                        Freie Presse}\pwindex{Neue Freie Presse@\emph{Neue Freie Presse}|pwk}, Nr. 21.144, 22. 7. 1923,
                     Morgenblatt, S. 1–3.}}}\label{K_L03020-1} – u rechnen Sie nicht nach, wie viele
               ähnliche Händedrucke ich Ihnen schuldig bin!\pend
           
\pstart
           Ich lebe ziemlich stille Tage in Wien\oindex{Wien@\textbf{Wien}, \emph{A.ADM2}|pw}, und werde
                  Anfang August, vermutlich \label{K_L03020-2v}\edtext{über Baden Baden\oindex{Baden-Baden@\textbf{Baden-Baden}, \emph{P.PPL}|pw}, wo die
                  Kinder\pwindex{Cappellini, Lili 13.09.1909 – 26.07.1928@\textsc{Cappellini, Lili} (13.09.1909 – 26.07.1928)|pwv}\pwindex{Schnitzler, Heinrich 09.08.1902 – 12.07.1982@\textsc{Schnitzler, Heinrich} (09.08.1902 – 12.07.1982), \emph{Regisseur/Regisseurin, Schauspieler/Schauspielerin}|pwv} bei Olga\pwindex{Schnitzler, Olga 17.01.1882 – 13.01.1970@\textsc{Schnitzler, Olga} (17.01.1882 – 13.01.1970), \emph{Schauspieler/Schauspielerin, Sänger/Sängerin}|pw}{ }{\pb}sommerweilen, in die Schweiz\oindex{Schweiz@\textbf{Schweiz}, \emph{A.PCLI}|pw} – oder sonstwohin fahren}{\lemma{\textnormal{\emph{über … fahren}}}\Cendnote{\textnormal{Schnitzler reiste am 3. 8. 1923 nach Salzburg\oindex{Salzburg@\textbf{Salzburg}, \emph{A.ADM2}|pwk} ab und kam am 6. 8. 1923 in Baden-Baden\oindex{Baden-Baden@\textbf{Baden-Baden}, \emph{P.PPL}|pwk} an. Am 15. 8. 1923 reiste er
                  weiter in die Schweiz\oindex{Schweiz@\textbf{Schweiz}, \emph{A.PCLI}|pwk}.}}}\label{K_L03020-2}.\pend
           
\pstart
           Lassen Sie mich wissen, wies Ihnen und den Ihren geht u ob Sie arbeiten.\pend
           \pstart Herzlichst Ihr \spacefill\mbox{Arthur}\pend{}\selectlanguage{ngerman}\endnumbering\briefempfaengerindex{Salten, Felix@\textsc{Salten, Felix}!zzzSchnitzler, Arthur@\emph{von Arthur Schnitzler}!1923-07-221@{22. 7. 1923}|)be}\mylabel{L03020h}  \normalsize

\doendnotes{C}
\bigskip
\vfill

\clearpage

\footnotesize

\lohead{\textsc{register}}

% Definiere theindex-Environment komplett neu ohne reledmac
\makeatletter
\renewenvironment{theindex}{%
  \section*{\indexname}%
  \setlength{\parindent}{0pt}%
  \setlength{\parskip}{0pt plus 0.3pt}%
  \let\item\@idxitem
}{%
  \clearpage
}
\makeatother

\IfFileExists{\jobname-pw.ind}{\input{\jobname-pw.ind}}{}

\end{document}

      