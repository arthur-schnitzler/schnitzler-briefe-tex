%% latex-leseansicht-vorspann.tex
%% Vorspann für die Leseansicht.
%% Lädt die gemeinsame Datei latex-vorspann.tex mit nicht gesetztem Schalter.

\newif\ifkorrekturansicht
\korrekturansichtfalse

\input{../tex-inputs/latex-vorspann}

\begin{center}
            \textcolor{red}{ENTWURF, NICHT FERTIG KORRIGIERT}
                      \end{center}
            
         
         \newcommand{\erwaehntePersonen}{Personen: Felix Salten, Lili Schnitzler, Heinrich Schnitzler, Olga Schnitzler,  Voltaire}
         \newcommand{\erwaehnteOrte}{Orte: Attersee, Baden-Baden, Berghof, Oberösterreich, Schweiz, Unterach am Attersee, Wien}
         \newcommand{\erwaehnteWerke}{Werke: Neue Freie Presse, Voltaire}
               \section[Arthur Schnitzler an Felix Salten, 22. 7. 1923]{ Arthur Schnitzler an Felix Salten, 22. 7. 1923}\nopagebreak\mylabel{v}\rehead{ }\begin{ledgroupsized}[t]{13cm}\normalsize\beginnumbering \toendnotes[C]{\smallbreak\pagebreak[2]} \Standort{Wienbibliothek im Rathaus, ZPH 1681, 2.1.516.}
\physDesc{
\newline{}Handschrift: , deutsche Kurrent}\toendnotes[C]{\smallbreak}\pstart{}{\pb}Ob.Oe.\oindex{Oberoesterreich@\textbf{Oberösterreich}|pw}\pend{}\pstart{}Herrn\pend{}\pstart{}Felix Salten\pend{}\pstart{}Unterach\oindex{Unterach am Attersee@\textbf{Unterach am Attersee}|pw}\pend{}\pstart{}am Attersee\oindex{Attersee@\textbf{Attersee}|pw}\pend{}\pstart{}Berghof \oindex{Berghof@\textbf{Berghof}|pw}\pend{}{\bigskip}\pstart
           \raggedleft{}{\pb}Wien\oindex{Wien@\textbf{Wien}|pw}, 22. 7. 23\pend
           \pstart
           lieber, lassen Sie sich die Hand drücken für Ihr \textcolor{gray}{schönes}\label{K_L03020-1v}\edtext{Voltaire\pwindex{Voltaire 21.11.1694 – 30.05.1778@\textsc{Voltaire} (21.11.1694 – 30.05.1778), \emph{Schriftsteller, Philosoph}|pw}-Feu{[}i{]}lleton\pwindex{Salten, Felix 06.09.1869 – 08.10.1945@\textsc{Salten, Felix} (06.09.1869 – 08.10.1945), \emph{Schriftsteller, Journalist}!Voltaire22. 07. 1923@\strich\emph{Voltaire} {[}22. 07. 1923{]}|pw}}{\lemma{\textnormal{\emph{Voltaire-Feuilleton}}}\Cendnote{\textnormal{Felix Salten\pwindex{Salten, Felix 06.09.1869 – 08.10.1945@\textsc{Salten, Felix} (06.09.1869 – 08.10.1945), \emph{Schriftsteller, Journalist}|pwk}: \emph{Voltaire}\pwindex{Salten, Felix 06.09.1869 – 08.10.1945@\textsc{Salten, Felix} (06.09.1869 – 08.10.1945), \emph{Schriftsteller, Journalist}!Voltaire22. 07. 1923@\strich\emph{Voltaire} {[}22. 07. 1923{]}|pwk}. In: \emph{Neue Freie Presse}\pwindex{Neue Freie Presse1864 – 1939@\emph{Neue Freie Presse} {[}1864 – 1939{]}|pwk}, Nr. 21.144,
                        22. 7. 1923, Morgenblatt, S. 1–3.}}}\label{K_L03020-1h}
               – u rechnen Sie nicht nach, wie viele ähnliche Händedrucke ich Ihnen schuldig bin! \pend
           \pstart
           Ich lebe ziemlich stille Tage in Wien\oindex{Wien@\textbf{Wien}|pw}, und werde
               Anfang August, vermutlich über Baden Baden\oindex{Baden-Baden@\textbf{Baden-Baden}|pw}, wo
               die Kinder\pwindex{Schnitzler, Lili 13.09.1909 – 26.07.1928@\textsc{Schnitzler, Lili} (13.09.1909 – 26.07.1928)|pwv}\pwindex{Schnitzler, Heinrich 09.08.1902 – 12.07.1982@\textsc{Schnitzler, Heinrich} (09.08.1902 – 12.07.1982), \emph{Regisseur, Schauspieler}|pwv} bei Olga\pwindex{Schnitzler, Olga 17.01.1882 – 13.01.1970@\textsc{Schnitzler, Olga} (17.01.1882 – 13.01.1970), \emph{Schauspielerin, Sängerin}|pw}{\pb}sommerweilen, in die Schweiz\oindex{Schweiz@\textbf{Schweiz}|pw} – oder
               sonstwohin fahren. \pend
           \pstart
           Lassen Sie mich wissen, wies Ihnen und den Ihren geht u ob Sie arbeiten. \pend
           \pstart Herzlichst Ihr \spacefill\mbox{Arthur}\pend{}
         
         \endnumbering\mylabel{h}\end{ledgroupsized}\begin{anhang}\end{anhang}\newcommand{\dateiname}{L03020}\newcommand{\titel}{Arthur Schnitzler an Felix Salten, 22. 7. 1923}\newcommand{\editorInnen}{Martin Anton Müller und Laura Untner}%% latex-leseansicht-abspann.tex
%% Abspann für die Leseansicht.
%% Der Schalter \ifkorrekturansicht ist bereits durch den Vorspann gesetzt.

%% latex-abspann.tex
%% Gemeinsamer Abspann für Korrekturansicht und Leseansicht.
%% Setzt den Schalter \ifkorrekturansicht voraus (gesetzt in den
%% einbindenden Dateien latex-korrekturansicht-abspann.tex bzw.
%% latex-leseansicht-abspann.tex).
%% ---------------------------------------------------------------

\normalsize

% Das esempio-Environment wird nur in der Leseansicht benötigt
\ifkorrekturansicht\else
\newenvironment{esempio}[3]%
{
    \vspace{1.5ex}
    \rlap{\underline{#1}}
    \par
    \setlength{\parindent}{0cm}
    \nopagebreak
    \leftskip=#2cm
    \rightskip=#3cm
}
{
    \par
}
\fi

\doendnotes{C}
\bigskip
\vfill

\clearpage

\footnotesize

\ifkorrekturansicht
  \lohead{\textsc{register}}
\fi

% theindex-Environment neu definieren ohne reledmac
\makeatletter
\renewenvironment{theindex}{%
  \ifkorrekturansicht
    \section*{\indexname}%
  \else
    \subsubsection*{Index der erwähnten Entitäten}%
  \fi
  \setlength{\parindent}{0pt}%
  \setlength{\parskip}{0pt plus 0.3pt}%
  \let\item\@idxitem
}{%
  \ifkorrekturansicht\clearpage\fi
}
\makeatother

\IfFileExists{\jobname-pw.ind}{\input{\jobname-pw.ind}}{}

% Quellenangabe nur in der Leseansicht
\ifkorrekturansicht\else
% Fallback-Definitionen, falls die .tex-Datei \titel etc. nicht gesetzt hat
\providecommand{\titel}{}
\providecommand{\editorInnen}{}
\providecommand{\dateiname}{\jobname}

\vspace{3cm}

\vfill

\footnotesize
\textsc{Quelle}: \titel. Herausgegeben von {\editorInnen}. In: \emph{Arthur Schnitzler: Briefwechsel mit Autorinnen und Autoren}.
 Digitale Edition, https://schnitzler-briefe.acdh.oeaw.ac.at/{\dateiname}.html (Stand \today)
\fi

\end{document}


      