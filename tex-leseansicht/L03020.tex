%% latex-leseansicht-vorspann.tex
%% Vorspann für die Leseansicht.
%% Lädt die gemeinsame Datei latex-vorspann.tex mit nicht gesetztem Schalter.

\newif\ifkorrekturansicht
\korrekturansichtfalse

\input{../tex-inputs/latex-vorspann}


\section[ Arthur Schnitzler an Felix Salten, 22. 7. 1923]{L03020 Arthur Schnitzler an Felix Salten,  22. 7. 1923}
\nopagebreak\mylabel{L03020v}
\rehead{ }\normalsize\beginnumbering\briefempfaengerindex{Salten, Felix@\textsc{Salten, Felix}!zzzSchnitzler, Arthur@\emph{von Arthur Schnitzler}!1923-07-221@{22. 7. 1923}|(be}
\toendnotes[C]{\smallbreak\pagebreak[2]}
\correspDesc{Versand  durch Arthur Schnitzler am 22. 7. 1923 in Wien
\newline{}Übermittlung  am 24. 7. 1923 in Wien
\newline{}Erhalt  durch Felix Salten im Zeitraum [25. 7. 1923
                  – 29. 7. 1923?] in Unterach am Attersee}\toendnotes[C]{\smallbreak}
\Standort{Wienbibliothek im Rathaus, ZPH 1681, 2.1.516.}
\physDesc{Postkarte, 472 Zeichen
\newline{}Handschrift: Bleistift, lateinische Kurrent
\newline{}Versand: Stempel: »\nobreak{}\oindex{Wien@\textbf{Wien}, \emph{Verwaltungsgebiet}|pwk}18/\textsubscript{1} Wien 11\textcolor{gray}{0}, 24. VII. 23, 9\nobreak{}«.  
\newline{}Ordnung: mit Bleistift von unbekannter Hand nummeriert: »5« }
\buchAbdrucke{\weitereDrucke{Arthur Schnitzler: \emph{Briefe 1913–1931}. Herausgegeben von Peter Michael Braunwarth, Richard Miklin, Susanne Pertlik und Heinrich Schnitzler. Frankfurt am Main: \emph{S. Fischer} 1984, S. 322–323.} }\toendnotes[C]{\smallbreak}\pstart{}{\pb}\label{T_L03020-1v}\edtext{\textcolor{gray}{\textbf{A. S.}}}{\lemma{\textnormal{\emph{A. S.}}}\Cendnote{\textnormal{ovaler Absenderkleber}}}\label{T_L03020-1}\pend{}\pstart{}\textcolor{gray}{\textbf{WIEN, XVIII.}}\oindex{XVIII., Währing@\textbf{XVIII., Währing}, \emph{Verwaltungsgebiet}|pw}\pend{}\pstart{}\textcolor{gray}{\textbf{STERNWARTESTR. 71}}\oindex{Wien@\textbf{Wien}!XVIII., Währing@\textbf{XVIII., Währing}!Sternwartestraße 71@\textbf{Sternwartestraße 71}, \emph{Wohngebäude}|pw}\pend{}{\bigskip}\pstart{}Ob. Oe.\oindex{Oberösterreich@\textbf{Oberösterreich}, \emph{Land}|pw}\pend{}\pstart{}Herrn\pend{}\pstart{}Felix Salten\pend{}\pstart{}Unterach\oindex{Unterach am Attersee@\textbf{Unterach am Attersee}|pw} am Attersee\oindex{Attersee@\textbf{Attersee}, \emph{See}|pw}\pend{}\pstart{}Berghof\oindex{Berghof@\textbf{Berghof}, \emph{Wohngebäude}|pw}\pend{}{\bigskip}\vspace{1em}
\pstart
           \raggedleft{}{\pb}Wien\oindex{Wien@\textbf{Wien}, \emph{Verwaltungsgebiet}|pw}, 22. 7. 23\pend
           \vspace{0.5em}
\pstart
           lieber, lassen Sie sich die Hand drücken für Ihr \textcolor{gray}{schönes}{ }\label{K_L03020-1v}\edtext{Voltaire\pwindex{Voltaire 21.\,11.\,1694 Paris – 30.\,5.\,1778 ebd.@\textsc{Voltaire} (21.\,11.\,1694 Paris – 30.\,5.\,1778 ebd.), \emph{Schriftsteller, Philosoph}|pw}
                  Feu{[}i{]}lleton\pwindex{Salten, Felix 6.\,9.\,1869 Budapest – 8.\,10.\,1945 Zürich@\textsc{Salten, Felix} (6.\,9.\,1869 Budapest – 8.\,10.\,1945 Zürich), \emph{Schriftsteller, Journalist, Chefredakteur}!Voltaire@\strich\emph{Voltaire}|pwv}}{\lemma{\textnormal{\emph{Voltaire
                  Feuilleton}}}\Cendnote{\textnormal{Felix Salten\pwindex{Salten, Felix 6.\,9.\,1869 Budapest – 8.\,10.\,1945 Zürich@\textsc{Salten, Felix} (6.\,9.\,1869 Budapest – 8.\,10.\,1945 Zürich), \emph{Schriftsteller, Journalist, Chefredakteur}|pwk}: \emph{Voltaire}\pwindex{Salten, Felix 6.\,9.\,1869 Budapest – 8.\,10.\,1945 Zürich@\textsc{Salten, Felix} (6.\,9.\,1869 Budapest – 8.\,10.\,1945 Zürich), \emph{Schriftsteller, Journalist, Chefredakteur}!Voltaire@\strich\emph{Voltaire}|pwk}. In: \emph{Neue
                        Freie Presse}\pwindex{Neue Freie Presse@\emph{Neue Freie Presse}|pwk}, Nr. 21.144, 22. 7. 1923,
                     Morgenblatt, S. 1–3.}}}\label{K_L03020-1} – u rechnen Sie nicht nach, wie viele
               ähnliche Händedrucke ich Ihnen schuldig bin!\pend
           
\pstart
           Ich lebe ziemlich stille Tage in Wien\oindex{Wien@\textbf{Wien}, \emph{Verwaltungsgebiet}|pw}, und werde
                  Anfang August, vermutlich \label{K_L03020-2v}\edtext{über Baden Baden\oindex{Baden-Baden@\textbf{Baden-Baden}|pw}, wo die
                  Kinder\pwindex{Cappellini, Lili 13.\,9.\,1909 Wien – 26.\,7.\,1928 Venedig@\textsc{Cappellini, Lili} (13.\,9.\,1909 Wien – 26.\,7.\,1928 Venedig)|pwv}\pwindex{Schnitzler, Heinrich 9.\,8.\,1902 Hinterbrühl – 12.\,7.\,1982 Wien@\textsc{Schnitzler, Heinrich} (9.\,8.\,1902 Hinterbrühl – 12.\,7.\,1982 Wien), \emph{Regisseur, Schauspieler}|pwv} bei Olga\pwindex{Schnitzler, Olga 17.\,1.\,1882 Wien – 13.\,1.\,1970 Lugano@\textsc{Schnitzler, Olga} (17.\,1.\,1882 Wien – 13.\,1.\,1970 Lugano), \emph{Schauspielerin, Sängerin}|pw}{ }{\pb}sommerweilen, in die Schweiz\oindex{Schweiz@\textbf{Schweiz}|pw} – oder sonstwohin fahren}{\lemma{\textnormal{\emph{über … fahren}}}\Cendnote{\textnormal{Schnitzler reiste am 3. 8. 1923 nach Salzburg\oindex{Salzburg@\textbf{Salzburg}, \emph{Verwaltungsgebiet}|pwk} ab und kam am 6. 8. 1923 in Baden-Baden\oindex{Baden-Baden@\textbf{Baden-Baden}|pwk} an. Am 15. 8. 1923 reiste er
                  weiter in die Schweiz\oindex{Schweiz@\textbf{Schweiz}|pwk}.}}}\label{K_L03020-2}.\pend
           
\pstart
           Lassen Sie mich wissen, wies Ihnen und den Ihren geht u ob Sie arbeiten.\pend
           \pstart Herzlichst Ihr \spacefill\mbox{Arthur}\pend{}\selectlanguage{ngerman}\endnumbering\briefempfaengerindex{Salten, Felix@\textsc{Salten, Felix}!zzzSchnitzler, Arthur@\emph{von Arthur Schnitzler}!1923-07-221@{22. 7. 1923}|)be}\mylabel{L03020h}  \newcommand{\dateiname}{L03020}\newcommand{\titel}{Arthur Schnitzler an Felix Salten, 22. 7. 1923}\newcommand{\editorInnen}{Martin Anton Müller und Laura Untner}%% latex-leseansicht-abspann.tex
%% Abspann für die Leseansicht.
%% Der Schalter \ifkorrekturansicht ist bereits durch den Vorspann gesetzt.

%% latex-abspann.tex
%% Gemeinsamer Abspann für Korrekturansicht und Leseansicht.
%% Setzt den Schalter \ifkorrekturansicht voraus (gesetzt in den
%% einbindenden Dateien latex-korrekturansicht-abspann.tex bzw.
%% latex-leseansicht-abspann.tex).
%% ---------------------------------------------------------------

\normalsize

% Das esempio-Environment wird nur in der Leseansicht benötigt
\ifkorrekturansicht\else
\newenvironment{esempio}[3]%
{
    \vspace{1.5ex}
    \rlap{\underline{#1}}
    \par
    \setlength{\parindent}{0cm}
    \nopagebreak
    \leftskip=#2cm
    \rightskip=#3cm
}
{
    \par
}
\fi

\doendnotes{C}
\bigskip
\vfill

\clearpage

\footnotesize

\ifkorrekturansicht
  \lohead{\textsc{register}}
\fi

% theindex-Environment neu definieren ohne reledmac
\makeatletter
\renewenvironment{theindex}{%
  \ifkorrekturansicht
    \section*{\indexname}%
  \else
    \subsubsection*{Index der erwähnten Entitäten}%
  \fi
  \setlength{\parindent}{0pt}%
  \setlength{\parskip}{0pt plus 0.3pt}%
  \let\item\@idxitem
}{%
  \ifkorrekturansicht\clearpage\fi
}
\makeatother

\IfFileExists{\jobname-pw.ind}{\input{\jobname-pw.ind}}{}

% Quellenangabe nur in der Leseansicht
\ifkorrekturansicht\else
% Fallback-Definitionen, falls die .tex-Datei \titel etc. nicht gesetzt hat
\providecommand{\titel}{}
\providecommand{\editorInnen}{}
\providecommand{\dateiname}{\jobname}

\vspace{3cm}

\vfill

\footnotesize
\textsc{Quelle}: \titel. Herausgegeben von {\editorInnen}. In: \emph{Arthur Schnitzler: Briefwechsel mit Autorinnen und Autoren}.
 Digitale Edition, https://schnitzler-briefe.acdh.oeaw.ac.at/{\dateiname}.html (Stand \today)
\fi

\end{document}


