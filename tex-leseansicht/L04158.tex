%% latex-leseansicht-vorspann.tex
%% Vorspann für die Leseansicht.
%% Lädt die gemeinsame Datei latex-vorspann.tex mit nicht gesetztem Schalter.

\newif\ifkorrekturansicht
\korrekturansichtfalse

\input{../tex-inputs/latex-vorspann}


\section[Arthur Schnitzler an Gustav Schwarzkopf, 24. 2. 1920]{L04158 Arthur Schnitzler an Gustav Schwarzkopf, 24. 2. 1920}
\nopagebreak\mylabel{L04158v}
\rehead{ }\normalsize\beginnumbering\briefempfaengerindex{Schwarzkopf, Gustav@\textsc{Schwarzkopf, Gustav}!zzzSchnitzler, Arthur@\emph{von Arthur Schnitzler}!1920-02-241@{24. 2. 1920}|(be}
\toendnotes[C]{\smallbreak\pagebreak[2]}
\correspDesc{Versand  durch Arthur Schnitzler am 24. 2. 1920 in Wien
\newline{}Erhalt  durch Gustav Schwarzkopf im Zeitraum [25. 2. 1920 – 29. 2. 1920?] in Baden bei Wien}\toendnotes[C]{\smallbreak}
\Standort{CUL, Schnitzler, B 96.}
\physDesc{Brief, 1 Blatt, 2 Seiten, 1277 Zeichen
\newline{}Handschrift: Bleistift, deutsche Kurrent}
\buchAbdrucke{\weitereDrucke{Arthur Schnitzler: \emph{Briefe 1913–1931}. Herausgegeben von Peter Michael Braunwarth, Richard Miklin, Susanne Pertlik und Heinrich Schnitzler. Frankfurt am Main: \emph{S. Fischer} 1984, S. 200.} }\toendnotes[C]{\smallbreak}
\pstart
           \raggedleft{}{\pb}24. 2. 920\pend
           \vspace{0.5em}
\pstart
           lieber Guſtav, vielen Dank für die \label{K_L04158-1v}\edtext{Karte}{\lemma{\textnormal{\emph{Karte}}}\Cendnote{\textnormal{{XXXX ref}. }}}\label{K_L04158-1}. Hoffentlich
      gefällt es Ihnen weiter{ }ſo gut und das Wetter
      hält ſich. Ich habe dienſtliche Abſicht Sie
      zu \label{K_L04158-2v}\edtext{beſuchen}{\lemma{\textnormal{\emph{besuchen}}}\Cendnote{\textnormal{Der Plan wurde nicht verwirkklicht}}}\label{K_L04158-2} we{\geminationn}s bahntechniſch möglich
      iſt. Von dem unerwarteten Hinſcheiden der
      armen Lili Stroſs\pwindex{Stross, Elisabeth 12.\,12.\,1895 Wien – 21.\,2.\,1920 ebd.@\textsc{Stross, Elisabeth} (12.\,12.\,1895 Wien – 21.\,2.\,1920 ebd.)|pw} haben Sie geleſen und
      gehört nehm ich an. Grippe, in ein paar Tagen.
      Auch für Olga\pwindex{Schnitzler, Olga 17.\,1.\,1882 Wien – 13.\,1.\,1970 Lugano@\textsc{Schnitzler, Olga} (17.\,1.\,1882 Wien – 13.\,1.\,1970 Lugano), \emph{Schauspielerin, Sängerin}|pw} iſt es ein beſonders ſchwerer
      Schlag. Wir haben ſie alle ſehr gern gehabt\textcolor{gray}{.}
      Sie war ein wahrhaft liebliches Weſen –
      man hätte das Wort erfinden müſſen! Ich
      erzähl Ihnen mündlich mehr. –\pend
           
\pstart
           Bei uns  geht es im übrigen ganz leidlich,–
         Heini\pwindex{Schnitzler, Heinrich 9.\,8.\,1902 Hinterbrühl – 12.\,7.\,1982 Wien@\textsc{Schnitzler, Heinrich} (9.\,8.\,1902 Hinterbrühl – 12.\,7.\,1982 Wien), \emph{Regisseur, Schauspieler}|pw} hat noch immer ſeine Sehnenſcheidenentzündg u. darf nicht Klavier ſpielen. Von \textsc{Rein{\pb}hard\pwindex{Reinhardt, Max 9.\,9.\,1873 Baden bei Wien – 30.\,10.\,1943 New York City@\textsc{Reinhardt, Max} (9.\,9.\,1873 Baden bei Wien – 30.\,10.\,1943 New York City), \emph{Theaterleiter, Regisseur, Schauspieler}|pw}} das übliche Telegramm mit Ausflüchten \textsc{etc.}{ }\label{K_L04158-3v}\edtext{\textsc{Première}\pwindex{Schnitzler, Arthur 15. 5. 1862 Wien – 21. 10. 1931 ebd.@\textsc{Schnitzler, Arthur} (15. 5. 1862 Wien – 21. 10. 1931 ebd.), \emph{Schriftsteller, Mediziner}!Reigen. Zehn Dialoge@\strich\emph{Reigen. Zehn Dialoge}|pwv}}{\lemma{\textnormal{\emph{Première}}}\Cendnote{\textnormal{
                        Von Max Reinhardts\pwindex{Reinhardt, Max 9.\,9.\,1873 Baden bei Wien – 30.\,10.\,1943 New York City@\textsc{Reinhardt, Max} (9.\,9.\,1873 Baden bei Wien – 30.\,10.\,1943 New York City), \emph{Theaterleiter, Regisseur, Schauspieler}|pwk} Vorhaben, \emph{Reigen}\pwindex{Schnitzler, Arthur 15. 5. 1862 Wien – 21. 10. 1931 ebd.@\textsc{Schnitzler, Arthur} (15. 5. 1862 Wien – 21. 10. 1931 ebd.), \emph{Schriftsteller, Mediziner}!Reigen. Zehn Dialoge@\strich\emph{Reigen. Zehn Dialoge}|pwk} zu inszenieren,
                        ist nur ein Regiebuch überliefert, zur Inszenierung kam es nicht.}}}\label{K_L04158-3}
         angeblich Mitte März. Reigen\pwindex{Schnitzler, Arthur 15. 5. 1862 Wien – 21. 10. 1931 ebd.@\textsc{Schnitzler, Arthur} (15. 5. 1862 Wien – 21. 10. 1931 ebd.), \emph{Schriftsteller, Mediziner}!Reigen. Zehn Dialoge@\strich\emph{Reigen. Zehn Dialoge}|pw} mein ich.
         Schweſtern\pwindex{Schnitzler, Arthur 15. 5. 1862 Wien – 21. 10. 1931 ebd.@\textsc{Schnitzler, Arthur} (15. 5. 1862 Wien – 21. 10. 1931 ebd.), \emph{Schriftsteller, Mediziner}!Schwestern oder Casanova in Spa. Lustspiel in Versen@\strich\emph{Die Schwestern oder Casanova in Spa. Lustspiel in Versen}|pw} noch nicht bestimmt; nach 31. März
         laß ichs nicht mehr zu. \label{K_L04158-4v}\edtext{\uline{Hier} am 25 März\eventindex{Burgtheater@\textbf{Burgtheater}!Uraufführung von Die Schwestern oder Casanova in Spa, 26.3.1920@Uraufführung von Die Schwestern oder Casanova in Spa, 26.3.1920|pwv}}{\lemma{\textnormal{\emph{Hier am 25 März}}}\Cendnote{\textnormal{Die Uraufführung von \emph{Die Schwestern oder Casanova in Spa}\pwindex{Schnitzler, Arthur 15. 5. 1862 Wien – 21. 10. 1931 ebd.@\textsc{Schnitzler, Arthur} (15. 5. 1862 Wien – 21. 10. 1931 ebd.), \emph{Schriftsteller, Mediziner}!Schwestern oder Casanova in Spa. Lustspiel in Versen@\strich\emph{Die Schwestern oder Casanova in Spa. Lustspiel in Versen}|pwk}\eventindex{Burgtheater@\textbf{Burgtheater}!Uraufführung von Die Schwestern oder Casanova in Spa, 26.3.1920@Uraufführung von Die Schwestern oder Casanova in Spa, 26.3.1920|pwk} verschob sich noch um einen Tag und fand am 26. 3. 1920 im Burgtheater\oindex{Wien@\textbf{Wien}!I., Innere Stadt@\textbf{I., Innere Stadt}!Burgtheater@\textbf{Burgtheater}, \emph{Theater}|pwk} statt.
               }}}\label{K_L04158-4}. –\pend
           
\pstart
           – Am 
                  12.\eventindex{Volkstheater@\textbf{Volkstheater}!Premiere von Der Puppenspieler, Der grüne Kakadu, Komtesse Mizzi, 12.3.1920@Premiere von Der Puppenspieler, Der grüne Kakadu, Komtesse Mizzi, 12.3.1920|pwv}
               am Volksth.\orgindex{Volkstheater@Volkstheater|pw}{ }Puppenſpieler\pwindex{Schnitzler, Arthur 15. 5. 1862 Wien – 21. 10. 1931 ebd.@\textsc{Schnitzler, Arthur} (15. 5. 1862 Wien – 21. 10. 1931 ebd.), \emph{Schriftsteller, Mediziner}!Puppenspieler. Studie in einem Aufzuge@\strich\emph{Der Puppenspieler. Studie in einem Aufzuge}|pw}, Kakadu\pwindex{Schnitzler, Arthur 15. 5. 1862 Wien – 21. 10. 1931 ebd.@\textsc{Schnitzler, Arthur} (15. 5. 1862 Wien – 21. 10. 1931 ebd.), \emph{Schriftsteller, Mediziner}!grüne Kakadu. Groteske in einem Akt@\strich\emph{Der grüne Kakadu. Groteske in einem Akt}|pw},
               Comt Mizzi\pwindex{Schnitzler, Arthur 15. 5. 1862 Wien – 21. 10. 1931 ebd.@\textsc{Schnitzler, Arthur} (15. 5. 1862 Wien – 21. 10. 1931 ebd.), \emph{Schriftsteller, Mediziner}!Komtesse Mizzi oder: Der Familientag@\strich\emph{Komtesse Mizzi oder: Der Familientag}|pw}; – \textsc{Rosenthal\pwindex{Rosenthal, Friedrich 20.\,7.\,1885 Wien – 31.\,8.\,1942 Konzentrationslager Auschwitz-Birkenau@\textsc{Rosenthal, Friedrich} (20.\,7.\,1885 Wien – 31.\,8.\,1942 Konzentrationslager Auschwitz-Birkenau), \emph{Regisseur, Dramaturg}|pw}} hat die Regie. –\pend
           
\pstart
           – Frühlingstage. Wir ſaßen heute im Garten\oindex{Wien@\textbf{Wien}!XVIII., Währing@\textbf{XVIII., Währing}!Sternwartestraße 71@\textbf{Sternwartestraße 71}, \emph{Wohngebäude}|pwv}
      und ließen uns besonnen. Heiter waren
      wir nicht. – Doch »\label{K_L04158-5v}\edtext{Du, der da weiter lebt{\dotsfour}\pwindex{Schnitzler, Arthur 15. 5. 1862 Wien – 21. 10. 1931 ebd.@\textsc{Schnitzler, Arthur} (15. 5. 1862 Wien – 21. 10. 1931 ebd.), \emph{Schriftsteller, Mediziner}!einsame Weg. Schauspiel in fünf Akten@\strich\emph{Der einsame Weg. Schauspiel in fünf Akten}|pwv}}{\lemma{\textnormal{\emph{Du, der da weiter lebt}}}\Cendnote{\textnormal{In
         \emph{Der einsame Weg}\pwindex{Schnitzler, Arthur 15. 5. 1862 Wien – 21. 10. 1931 ebd.@\textsc{Schnitzler, Arthur} (15. 5. 1862 Wien – 21. 10. 1931 ebd.), \emph{Schriftsteller, Mediziner}!einsame Weg. Schauspiel in fünf Akten@\strich\emph{Der einsame Weg. Schauspiel in fünf Akten}|pwk} sagt Stephan von Sala: »Du, der da weiterlebt, laß’ ab zu weinen,{ }ſagt Omar Nameh, geboren zu Bagdad\oindex{Baghdad@\textbf{Baghdad}, \emph{Hauptstadt}|pw} im Jahre 412 der mohammedaniſchen
            Zeitrechnung als Sohn eines Keſſelflickers.« Arthur Schnitzler: \emph{Der einsame Weg. Schauspiel in fünf Akten}\pwindex{Schnitzler, Arthur 15. 5. 1862 Wien – 21. 10. 1931 ebd.@\textsc{Schnitzler, Arthur} (15. 5. 1862 Wien – 21. 10. 1931 ebd.), \emph{Schriftsteller, Mediziner}!einsame Weg. Schauspiel in fünf Akten@\strich\emph{Der einsame Weg. Schauspiel in fünf Akten}|pwk}. Berlin\oindex{Berlin@\textbf{Berlin}, \emph{Hauptstadt}|pwk}: \emph{S. Fischer}XXXX ORGangabe fehlt{ }1904,
            S. 20. (1. Akt, 2. Szene)
      }}}\label{K_L04158-5}«
               wie jener Keſſelflicker\pwindex{Schnitzler, Arthur 15. 5. 1862 Wien – 21. 10. 1931 ebd.@\textsc{Schnitzler, Arthur} (15. 5. 1862 Wien – 21. 10. 1931 ebd.), \emph{Schriftsteller, Mediziner}!einsame Weg. Schauspiel in fünf Akten@\strich\emph{Der einsame Weg. Schauspiel in fünf Akten}|pwv} ſagt.
               Rührend war der kleine Raimund Hofmannsthal\pwindex{Hofmannsthal, Raimund von 26.\,5.\,1906 Rodaun – 20.\,3.\,1974 London@\textsc{Hofmannsthal, Raimund von} (26.\,5.\,1906 Rodaun – 20.\,3.\,1974 London)|pw}, der \label{K_L04158-6v}\edtext{geſtern im Matroſenanzug hinter
               dem Sarge}{\lemma{\textnormal{\emph{gestern … Sarge}}}\Cendnote{\textnormal{Vgl. A. S.: \emph{Tagebuch}, 21. 2. 1920.}}}\label{K_L04158-6} einherging; – Lili L.\pwindex{Stross, Elisabeth 12.\,12.\,1895 Wien – 21.\,2.\,1920 ebd.@\textsc{Stross, Elisabeth} (12.\,12.\,1895 Wien – 21.\,2.\,1920 ebd.)|pw} war ſeine
      erſte Liebe geweſen. –\pend
           
\pstart
           Laſſen Sie ſichs wohl ergehen, lieber Guſtav,
      wir grüßen Sie aufs allerherzlichſte.\pend
           
\pstart
           Ihr{\\[\baselineskip]}\spacefill\mbox{Arthur}\pend
           \leftskip=0em{}\selectlanguage{ngerman}\endnumbering\briefempfaengerindex{Schwarzkopf, Gustav@\textsc{Schwarzkopf, Gustav}!zzzSchnitzler, Arthur@\emph{von Arthur Schnitzler}!1920-02-241@{24. 2. 1920}|)be}\mylabel{L04158h}
\begin{anhang}
\end{anhang}\newcommand{\dateiname}{L04158}\newcommand{\titel}{Arthur Schnitzler an Gustav Schwarzkopf, 24. 2. 1920}\newcommand{\editorInnen}{Herausgegeben von Jahnke, SelmaMüller, Martin Anton}%% latex-leseansicht-abspann.tex
%% Abspann für die Leseansicht.
%% Der Schalter \ifkorrekturansicht ist bereits durch den Vorspann gesetzt.

%% latex-abspann.tex
%% Gemeinsamer Abspann für Korrekturansicht und Leseansicht.
%% Setzt den Schalter \ifkorrekturansicht voraus (gesetzt in den
%% einbindenden Dateien latex-korrekturansicht-abspann.tex bzw.
%% latex-leseansicht-abspann.tex).
%% ---------------------------------------------------------------

\normalsize

% Das esempio-Environment wird nur in der Leseansicht benötigt
\ifkorrekturansicht\else
\newenvironment{esempio}[3]%
{
    \vspace{1.5ex}
    \rlap{\underline{#1}}
    \par
    \setlength{\parindent}{0cm}
    \nopagebreak
    \leftskip=#2cm
    \rightskip=#3cm
}
{
    \par
}
\fi

\doendnotes{C}
\bigskip
\vfill

\clearpage

\footnotesize

\ifkorrekturansicht
  \lohead{\textsc{register}}
\fi

% theindex-Environment neu definieren ohne reledmac
\makeatletter
\renewenvironment{theindex}{%
  \ifkorrekturansicht
    \section*{\indexname}%
  \else
    \subsubsection*{Index der erwähnten Entitäten}%
  \fi
  \setlength{\parindent}{0pt}%
  \setlength{\parskip}{0pt plus 0.3pt}%
  \let\item\@idxitem
}{%
  \ifkorrekturansicht\clearpage\fi
}
\makeatother

\IfFileExists{\jobname-pw.ind}{\input{\jobname-pw.ind}}{}

% Quellenangabe nur in der Leseansicht
\ifkorrekturansicht\else
% Fallback-Definitionen, falls die .tex-Datei \titel etc. nicht gesetzt hat
\providecommand{\titel}{}
\providecommand{\editorInnen}{}
\providecommand{\dateiname}{\jobname}

\vspace{3cm}

\vfill

\footnotesize
\textsc{Quelle}: \titel. Herausgegeben von {\editorInnen}. In: \emph{Arthur Schnitzler: Briefwechsel mit Autorinnen und Autoren}.
 Digitale Edition, https://schnitzler-briefe.acdh.oeaw.ac.at/{\dateiname}.html (Stand \today)
\fi

\end{document}


