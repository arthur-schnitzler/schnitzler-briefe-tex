%% latex-leseansicht-vorspann.tex
%% Vorspann für die Leseansicht.
%% Lädt die gemeinsame Datei latex-vorspann.tex mit nicht gesetztem Schalter.

\newif\ifkorrekturansicht
\korrekturansichtfalse

\input{../tex-inputs/latex-vorspann}


         
         \renewcommand{\erwaehntePersonen}{Personen: Richard Beer-Hofmann, Paula Beer-Hofmann, Otto Brahm}
         \renewcommand{\erwaehnteOrte}{Orte: Bergen (Norwegen), Hamburg, I., Innere Stadt, Oslo, Salzburg, Trondheim, Wien}
         \renewcommand{\erwaehnteWerke}{}
               \section[Arthur Schnitzler an Richard Beer-Hofmann, 27. 6. 1896]{ Arthur Schnitzler an Richard Beer-Hofmann,
               27. 6. 1896}\nopagebreak\mylabel{v}\rehead{ }\begin{ledgroupsized}[t]{13cm}\normalsize\beginnumbering \toendnotes[C]{\smallbreak\pagebreak[2]} \Standort{YCGL, MSS 31.}
\physDesc{Postkarte
\newline{}Handschrift: schwarze Tinte, deutsche Kurrent\newline{}Versand: 1) Stempel: »\nobreak{}\oindex{I., Innere Stadt@\textbf{I., Innere Stadt}|pwk}Wien 1/1, 27. 6. 96, 8–9N\nobreak{}«.   2) Stempel: »\nobreak{}\oindex{Salzburg@\textbf{Salzburg}|pwk}Salzburg Stadt, 28 6 96, 10F\nobreak{}«. }\buchAbdrucke{\weitereDrucke{Arthur Schnitzler, Richard Beer-Hofmann: \emph{Briefwechsel 1891–1931}. Hg. Konstanze Fliedl. Wien, Zürich: \emph{Europaverlag} 1992, S. 91.} }\toendnotes[C]{\smallbreak}\pstart{}{\pb}Herrn \textsc{Dr. Richard
                     Beer-Hofmann}\pend{}\pstart{}\textsc{Salzburg\oindex{Salzburg@\textbf{Salzburg}|pw}}\pend{}\pstart{}\textsc{post restante.}\pend{}{\bigskip}\pstart
           \raggedleft{}{\pb}27. 6. 96. \pend
           \pstart
           Lieber Richard, ich bin hier noch bis zum 2. Juli
               für Briefe anzutreffen. Ich notire Ihnen hier gleich die Daten, wann u. wohin Sie
               event. \uuline{Telegramm} abzuſenden haben:\pend
           \pstart
           am 6. Juli nach Hamburg\oindex{Hamburg@\textbf{Hamburg}|pw}\pend
           \pstart
           am 9. Juli nach Bergen (Norwegen)\oindex{XXXX Ortsangabe fehlt|pw}\pend
           \pstart
           am 14. Juli nach Trondjhem\oindex{Trondheim@\textbf{Trondheim}|pw}\pend
           \pstart
           am 23. Juli nach Trondjhem\oindex{Trondheim@\textbf{Trondheim}|pw}\pend
           \pstart
           am 25. Juli nach Kriſtiania\oindex{Oslo@\textbf{Oslo}|pw}.\pend
           \pstart
           Briefe, wiſſen Sie ja. –\pend
           \pstart
           Wünſch Ihnen gute Sti{\geminationm}ung und hoffe häufige Nachrichten.
               Grüßen Sie Paula\pwindex{Beer-Hofmann, Paula 25.02.1879 – 30.10.1939@\textsc{Beer-Hofmann, Paula} (25.02.1879 – 30.10.1939)|pw}. Herzlich der Ihre
                  \spacefill\mbox{Arthur}\pend
           \pstart
           \noindent{}\label{T_L00555_1v}\edtext{Brahm\pwindex{Brahm, Otto 05.02.1856 – 28.11.1912@\textsc{Brahm, Otto} (05.02.1856 – 28.11.1912), \emph{Theaterleiter, Regisseur}|pw} läßt Sie grüßen.}{\lemma{\textnormal{\emph{Brahm läßt Sie grüßen.}}}\Cendnote{\textnormal{quer am rechten Rand}}}\label{T_L00555_1h}\pend
           
         
         \endnumbering\mylabel{h}\end{ledgroupsized}  \newcommand{\dateiname}{L00555}\newcommand{\titel}{Arthur Schnitzler an Richard Beer-Hofmann, 27. 6. 1896}\newcommand{\editorInnen}{Martin Anton Müller und Gerd-Hermann Susen}%% latex-leseansicht-abspann.tex
%% Abspann für die Leseansicht.
%% Der Schalter \ifkorrekturansicht ist bereits durch den Vorspann gesetzt.

%% latex-abspann.tex
%% Gemeinsamer Abspann für Korrekturansicht und Leseansicht.
%% Setzt den Schalter \ifkorrekturansicht voraus (gesetzt in den
%% einbindenden Dateien latex-korrekturansicht-abspann.tex bzw.
%% latex-leseansicht-abspann.tex).
%% ---------------------------------------------------------------

\normalsize

% Das esempio-Environment wird nur in der Leseansicht benötigt
\ifkorrekturansicht\else
\newenvironment{esempio}[3]%
{
    \vspace{1.5ex}
    \rlap{\underline{#1}}
    \par
    \setlength{\parindent}{0cm}
    \nopagebreak
    \leftskip=#2cm
    \rightskip=#3cm
}
{
    \par
}
\fi

\doendnotes{C}
\bigskip
\vfill

\clearpage

\footnotesize

\ifkorrekturansicht
  \lohead{\textsc{register}}
\fi

% theindex-Environment neu definieren ohne reledmac
\makeatletter
\renewenvironment{theindex}{%
  \ifkorrekturansicht
    \section*{\indexname}%
  \else
    \subsubsection*{Index der erwähnten Entitäten}%
  \fi
  \setlength{\parindent}{0pt}%
  \setlength{\parskip}{0pt plus 0.3pt}%
  \let\item\@idxitem
}{%
  \ifkorrekturansicht\clearpage\fi
}
\makeatother

\IfFileExists{\jobname-pw.ind}{\input{\jobname-pw.ind}}{}

% Quellenangabe nur in der Leseansicht
\ifkorrekturansicht\else
% Fallback-Definitionen, falls die .tex-Datei \titel etc. nicht gesetzt hat
\providecommand{\titel}{}
\providecommand{\editorInnen}{}
\providecommand{\dateiname}{\jobname}

\vspace{3cm}

\vfill

\footnotesize
\textsc{Quelle}: \titel. Herausgegeben von {\editorInnen}. In: \emph{Arthur Schnitzler: Briefwechsel mit Autorinnen und Autoren}.
 Digitale Edition, https://schnitzler-briefe.acdh.oeaw.ac.at/{\dateiname}.html (Stand \today)
\fi

\end{document}


      