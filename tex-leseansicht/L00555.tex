%% latex-korrekturansicht-vorspann.tex
%% Vorspann für die Korrekturansicht.
%% Lädt die gemeinsame Datei latex-vorspann.tex mit gesetztem Schalter.

\newif\ifkorrekturansicht
\korrekturansichttrue

\input{../tex-inputs/latex-vorspann}


\section[Arthur Schnitzler an Richard Beer-Hofmann, 27. 6. 1896]{L00555 Arthur Schnitzler an Richard Beer-Hofmann, 27. 6. 1896}
\nopagebreak\mylabel{L00555v}
\rehead{ }\normalsize\beginnumbering\briefempfaengerindex{Beer-Hofmann, Richard@\textsc{Beer-Hofmann, Richard}!zzzSchnitzler, Arthur@\emph{von Arthur Schnitzler}!1896-06-271@{27. 6. 1896}|(be}
\toendnotes[C]{\smallbreak\pagebreak[2]}\Standort{YCGL, MSS 31.}
\physDesc{Postkarte, 485 Zeichen
\newline{}Handschrift: schwarze Tinte, deutsche Kurrent
\newline{}Versand: 1) Stempel: »\nobreak{}\oindex{I., Innere Stadt@\textbf{I., Innere Stadt}, \emph{A.ADM3}|pwk}Wien 1/1, 27. 6. 96, 8–9N\nobreak{}«.   2) Stempel: »\nobreak{}\oindex{Salzburg@\textbf{Salzburg}, \emph{A.ADM2}|pwk}Salzburg Stadt, 28 6 96, 10F\nobreak{}«. }
\buchAbdrucke{\weitereDrucke{Arthur Schnitzler, Richard Beer-Hofmann: \emph{Briefwechsel 1891–1931}. Wien, Zürich: \emph{Europaverlag} 1992, S. 91.} }\toendnotes[C]{\smallbreak}\pstart{}{\pb}Herrn \textsc{Dr. Richard
                     Beer-Hofmann}\pend{}\pstart{}\textsc{Salzburg\oindex{Salzburg@\textbf{Salzburg}, \emph{A.ADM2}|pw}}\pend{}\pstart{}\textsc{post restante.}\pend{}{\bigskip}\vspace{1em}
\pstart
           \raggedleft{}{\pb}27. 6. 96. \pend
           \vspace{0.5em}
\pstart
           Lieber Richard, ich bin hier noch bis zum 2. Juli für
               Briefe anzutreffen. Ich notire Ihnen hier gleich die Daten, wann u. wohin Sie event.
                  \uuline{Telegramm} abzuſenden haben:\pend
           
\pstart
           am 6. Juli nach Hamburg\oindex{Hamburg@\textbf{Hamburg}, \emph{P.PPLA}|pw}\pend
           
\pstart
           am 9. Juli nach Bergen (Norwegen)\oindex{Bergen [Norwegen]@\textbf{Bergen [Norwegen]}, \emph{P.PPLA}|pw}\pend
           
\pstart
           am 14. Juli nach Trondjhem\oindex{Trondheim@\textbf{Trondheim}, \emph{P.PPLA2}|pw}\pend
           
\pstart
           am 23. Juli nach Trondjhem\oindex{Trondheim@\textbf{Trondheim}, \emph{P.PPLA2}|pw}\pend
           
\pstart
           am 25. Juli nach Kriſtiania\oindex{Oslo@\textbf{Oslo}, \emph{P.PPLC}|pw}.\pend
           
\pstart
           Briefe, wiſſen Sie ja. –\pend
           
\pstart
           Wünſch Ihnen gute Sti{\geminationm}ung und hoffe häufige Nachrichten.
               Grüßen Sie Paula\pwindex{Beer-Hofmann, Paula 25.02.1879 – 30.10.1939@\textsc{Beer-Hofmann, Paula} (25.02.1879 – 30.10.1939)|pw}. Herzlich der Ihre
                  \spacefill\mbox{Arthur}\pend
           
\pstart
           \noindent{}\label{T_L00555-1v}\edtext{Brahm\pwindex{Brahm, Otto 05.02.1856 – 28.11.1912@\textsc{Brahm, Otto} (05.02.1856 – 28.11.1912), \emph{Theaterleiter/Theaterleiterin, Regisseur/Regisseurin}|pw} läßt Sie grüßen.}{\lemma{\textnormal{\emph{Brahm läßt Sie grüßen.}}}\Cendnote{\textnormal{quer am rechten Rand}}}\label{T_L00555-1}\pend
           \selectlanguage{ngerman}\endnumbering\briefempfaengerindex{Beer-Hofmann, Richard@\textsc{Beer-Hofmann, Richard}!zzzSchnitzler, Arthur@\emph{von Arthur Schnitzler}!1896-06-271@{27. 6. 1896}|)be}\mylabel{L00555h}  \normalsize

\doendnotes{C}
\bigskip
\vfill

\clearpage

\footnotesize

\lohead{\textsc{register}}

% Definiere theindex-Environment komplett neu ohne reledmac
\makeatletter
\renewenvironment{theindex}{%
  \section*{\indexname}%
  \setlength{\parindent}{0pt}%
  \setlength{\parskip}{0pt plus 0.3pt}%
  \let\item\@idxitem
}{%
  \clearpage
}
\makeatother

\IfFileExists{\jobname-pw.ind}{\input{\jobname-pw.ind}}{}

\end{document}

      