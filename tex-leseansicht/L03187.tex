%% latex-leseansicht-vorspann.tex
%% Vorspann für die Leseansicht.
%% Lädt die gemeinsame Datei latex-vorspann.tex mit nicht gesetztem Schalter.

\newif\ifkorrekturansicht
\korrekturansichtfalse

\input{../tex-inputs/latex-vorspann}

\begin{center}
            \textcolor{red}{ENTWURF, NICHT FERTIG KORRIGIERT}
                      \end{center}
            
         
         \renewcommand{\erwaehntePersonen}{Personen: Georg Hirschfeld}
         \renewcommand{\erwaehnteInstitutionen}{Institutionen: Wiener Allgemeine Zeitung}
         \renewcommand{\erwaehnteOrte}{Orte: Berlin, Schulerstraße, Universitätsstraße, Wien}
         \renewcommand{\erwaehnteWerke}{Werke: Die Mütter. Schauspiel in vier Acten}
               \section[Felix Salten an Arthur Schnitzler, 6. 11. 1896]{ Felix Salten an Arthur Schnitzler, 6. 11. 1896}\nopagebreak\mylabel{v}\rehead{ }\begin{ledgroupsized}[t]{13cm}\normalsize\beginnumbering \toendnotes[C]{\smallbreak\pagebreak[2]} \Standort{CUL, Schnitzler, B 89, A 1.}
\physDesc{Brief, 1 Blatt, 1 Seite
\newline{}Handschrift: schwarze Tinte, lateinische Kurrent\newline{}Ordnung: mit Bleistift von unbekannter Hand nummeriert: »81« }\toendnotes[C]{\smallbreak}\pstart
           \noindent{}{\pb}\textcolor{gray}{\textbf{\textbf{»Wiener Allgemeine
                        Zeitung\orgindex{Wiener Allgemeine Zeitung@Wiener Allgemeine Zeitung|pw}«}}}\pend
           \pstart
           \textcolor{gray}{\textbf{Redaction:}}\pend
           \pstart
           \textcolor{gray}{\textbf{\textbf{IX/3, Univerſitätsstraße Nr. 6\oindex{Universitaetsstrasse@\textbf{Universitätsstraße}|pw}.}}}\pend
           \pstart
           \textcolor{gray}{\textbf{Administration:}}\hfill \textcolor{gray}{\textbf{Wien\oindex{Wien@\textbf{Wien}|pw}, am}}{ }6. Nov. \textcolor{gray}{\textbf{189}}6.\pend
           \pstart
           \textcolor{gray}{\textbf{\textbf{I. Schulerſtraße Nr. 20\oindex{Schulerstrasse@\textbf{Schulerstraße}|pw}. }}}\pend
           \pstart
           \textcolor{gray}{\textbf{Telegramm-Adreſſe: »Allgemeine, Wien\oindex{Wien@\textbf{Wien}|pw}«.}}\pend
           \pstart
           \textcolor{gray}{\textbf{Telephon der Redaction: Nr. 805 u. 2180.}}\pend
           \pstart
           \textcolor{gray}{\textbf{\hspace*{2.5em}„\hspace*{2.5em}„\hspace*{2.5em} Adminiſtration: Nr. 1024.}}\pend
           \pstart
           Lieber Freund, ich hab die neue Adreße Hirschfelds\pwindex{Hirschfeld, Georg 11.02.1873 – 17.01.1942@\textsc{Hirschfeld, Georg} (11.02.1873 – 17.01.1942), \emph{Schriftsteller}|pw} verlegt. Sie sind wol so freundl. und \label{K_L03187-1v}\edtext{laßen ihm die Zeitungen, die ich eben
               absandte, zugehen}{\lemma{\textnormal{\emph{laßen … zugehen}}}\Cendnote{\textnormal{Diese separat versandte
                  Beilage nicht erhalten. Es dürfte sich um Wien\oindex{Wien@\textbf{Wien}|pwk}er Besprechungen von Georg
                     Hirschfelds\pwindex{Hirschfeld, Georg 11.02.1873 – 17.01.1942@\textsc{Hirschfeld, Georg} (11.02.1873 – 17.01.1942), \emph{Schriftsteller}|pwk} Stück \emph{Die Mütter}\pwindex{Hirschfeld, Georg 11.02.1873 – 17.01.1942@\textsc{Hirschfeld, Georg} (11.02.1873 – 17.01.1942), \emph{Schriftsteller}!Muetter. Schauspiel in vier Acten1896@\strich\emph{Die Mütter. Schauspiel in vier Acten} {[}1896{]}|pwk} handeln,
                  das am 17. 10. 1896
                  in Wien\oindex{Wien@\textbf{Wien}|pwk} Premiere gehabt hatte.}}}\label{K_L03187-1h}. Die
                  \label{K_L03187-2v}\edtext{Wien\oindex{Wien@\textbf{Wien}|pw}er Blätter}{\lemma{\textnormal{\emph{Wiener Blätter}}}\Cendnote{\textnormal{Zur Uraufführung
                  von \emph{Freiwild}\textcolor{red}{\textsuperscript{XXXX indx}} am 3. 11. 1896.}}}\label{K_L03187-2h} werd ich Ihnen
               aufheben. Hier haben die Leute sehr stark den Eindruck eines grossen Erfolges. \pend
           \pstart
           Herzlich, Ihr {\\[\baselineskip]}\spacefill\mbox{Salten}\pend
           \leftskip=0em{}
         
         \endnumbering\mylabel{h}\end{ledgroupsized}\begin{anhang}\end{anhang}\newcommand{\dateiname}{L03187}\newcommand{\titel}{Felix Salten an Arthur Schnitzler, 6. 11. 1896}\newcommand{\editorInnen}{Martin Anton Müller und Laura Untner}%% latex-leseansicht-abspann.tex
%% Abspann für die Leseansicht.
%% Der Schalter \ifkorrekturansicht ist bereits durch den Vorspann gesetzt.

%% latex-abspann.tex
%% Gemeinsamer Abspann für Korrekturansicht und Leseansicht.
%% Setzt den Schalter \ifkorrekturansicht voraus (gesetzt in den
%% einbindenden Dateien latex-korrekturansicht-abspann.tex bzw.
%% latex-leseansicht-abspann.tex).
%% ---------------------------------------------------------------

\normalsize

% Das esempio-Environment wird nur in der Leseansicht benötigt
\ifkorrekturansicht\else
\newenvironment{esempio}[3]%
{
    \vspace{1.5ex}
    \rlap{\underline{#1}}
    \par
    \setlength{\parindent}{0cm}
    \nopagebreak
    \leftskip=#2cm
    \rightskip=#3cm
}
{
    \par
}
\fi

\doendnotes{C}
\bigskip
\vfill

\clearpage

\footnotesize

\ifkorrekturansicht
  \lohead{\textsc{register}}
\fi

% theindex-Environment neu definieren ohne reledmac
\makeatletter
\renewenvironment{theindex}{%
  \ifkorrekturansicht
    \section*{\indexname}%
  \else
    \subsubsection*{Index der erwähnten Entitäten}%
  \fi
  \setlength{\parindent}{0pt}%
  \setlength{\parskip}{0pt plus 0.3pt}%
  \let\item\@idxitem
}{%
  \ifkorrekturansicht\clearpage\fi
}
\makeatother

\IfFileExists{\jobname-pw.ind}{\input{\jobname-pw.ind}}{}

% Quellenangabe nur in der Leseansicht
\ifkorrekturansicht\else
% Fallback-Definitionen, falls die .tex-Datei \titel etc. nicht gesetzt hat
\providecommand{\titel}{}
\providecommand{\editorInnen}{}
\providecommand{\dateiname}{\jobname}

\vspace{3cm}

\vfill

\footnotesize
\textsc{Quelle}: \titel. Herausgegeben von {\editorInnen}. In: \emph{Arthur Schnitzler: Briefwechsel mit Autorinnen und Autoren}.
 Digitale Edition, https://schnitzler-briefe.acdh.oeaw.ac.at/{\dateiname}.html (Stand \today)
\fi

\end{document}


      