%% latex-korrekturansicht-vorspann.tex
%% Vorspann für die Korrekturansicht.
%% Lädt die gemeinsame Datei latex-vorspann.tex mit gesetztem Schalter.

\newif\ifkorrekturansicht
\korrekturansichttrue

\input{../tex-inputs/latex-vorspann}


\section[ Felix Salten an Arthur Schnitzler, 6. 11. 1896]{L03187 Felix Salten an Arthur Schnitzler, 6. 11. 1896}
\nopagebreak\mylabel{L03187v}
\rehead{ }\normalsize\beginnumbering\briefempfaengerindex{Schnitzler, Arthur@\textsc{Schnitzler, Arthur}!zzzSalten, Felix@\emph{von Felix Salten}!1896-11-061@{6. 11. 1896}|(be}
\toendnotes[C]{\smallbreak\pagebreak[2]}\Standort{CUL, Schnitzler, B 89, A 1.}
\physDesc{Brief, 1 Blatt, 1 Seite, 279 Zeichen
\newline{}Handschrift: schwarze Tinte, lateinische Kurrent
\newline{}Ordnung: mit Bleistift von unbekannter Hand nummeriert: »81« }\toendnotes[C]{\smallbreak}
\pstart
           {\pb}\textcolor{gray}{\textbf{\textbf{»Wiener Allgemeine
                        Zeitung\orgindex{Wiener Allgemeine Zeitung@Wiener Allgemeine Zeitung|pw}«}}}\pend
           
\pstart
           \textcolor{gray}{\textbf{Redaction:}}\pend
           
\pstart
           \textcolor{gray}{\textbf{\textbf{IX/3, Univerſitätsſtraße Nr. 6\oindex{Universitaetsstrasse@\textbf{Universitätsstraße}, \emph{Straße (K.STR)}|pw}}}}\pend
           
\pstart
           \textcolor{gray}{\textbf{Adminiſtration:}}\hfill \textcolor{gray}{\textbf{Wien\oindex{Wien@\textbf{Wien}, \emph{A.ADM2}|pw}, am}}{ }6. Nov. \textcolor{gray}{\textbf{189}}6.\pend
           
\pstart
           \textcolor{gray}{\textbf{\textbf{I. Schulerſtraße Nr. 20\oindex{Schulerstrasse@\textbf{Schulerstraße}, \emph{Straße (K.STR)}|pw}. }}}\pend
           
\pstart
           \textcolor{gray}{\textbf{Telegramm-Adreſſe: »Allgemeine, Wien\oindex{Wien@\textbf{Wien}, \emph{A.ADM2}|pw}«.}}\pend
           
\pstart
           \textcolor{gray}{\textbf{Telephon der Redaction: Nr. 805 u. 2180.}}\pend
           
\pstart
           \textcolor{gray}{\textbf{\hspace*{1.5em}„\hspace*{1.5em}„\hspace*{1.5em} Adminiſtration: Nr. 1024.}}\pend
           \vspace{0.5em}
\pstart
           Lieber Freund, ich hab die neue \label{K_L03187-1v}\edtext{Adreße Hirschfelds\pwindex{Hirschfeld, Georg 11.02.1873 – 17.01.1942@\textsc{Hirschfeld, Georg} (11.02.1873 – 17.01.1942), \emph{Schriftsteller/Schriftstellerin}|pw}}{\lemma{\textnormal{\emph{Adreße Hirschfelds}}}\Cendnote{\textnormal{Schnitzler hielt sich in Berlin\oindex{Berlin@\textbf{Berlin}, \emph{P.PPLC}|pwk} auf. Er begegnete bereits am Folgetag, mutmaßlich am Tag des
                  Empfangs dieses Korrespondenzstücks, Hirschfeld\pwindex{Hirschfeld, Georg 11.02.1873 – 17.01.1942@\textsc{Hirschfeld, Georg} (11.02.1873 – 17.01.1942), \emph{Schriftsteller/Schriftstellerin}|pwk}.}}}\label{K_L03187-1} verlegt. Sie sind wol so freundl. und \label{K_L03187-2v}\edtext{laßen ihm die Zeitungen, die ich eben
               absandte, zugehen}{\lemma{\textnormal{\emph{laßen … zugehen}}}\Cendnote{\textnormal{Diese als Drucksache
                  separat versandte Beilage ist nicht erhalten. Sie dürfte Besprechungen von Georg Hirschfelds\pwindex{Hirschfeld, Georg 11.02.1873 – 17.01.1942@\textsc{Hirschfeld, Georg} (11.02.1873 – 17.01.1942), \emph{Schriftsteller/Schriftstellerin}|pwk} Stück \emph{Die Mütter}\pwindex{Muetter. Schauspiel in vier Acten@\emph{Die Mütter. Schauspiel in vier Acten}|pwk} enthalten haben, das am 17. 10. 1896 in Wien\oindex{Wien@\textbf{Wien}, \emph{A.ADM2}|pwk} Premiere gehabt hatte.}}}\label{K_L03187-2}.\hspace*{2em}Die \label{K_L03187-3v}\edtext{Wien\oindex{Wien@\textbf{Wien}, \emph{A.ADM2}|pw}er Blätter}{\lemma{\textnormal{\emph{Wiener Blätter}}}\Cendnote{\textnormal{Mit Wien\oindex{Wien@\textbf{Wien}, \emph{A.ADM2}|pwk}er Besprechungen
                  der Uraufführung von \emph{Freiwild}\pwindex{Freiwild. Schauspiel in 3 Akten@\emph{Freiwild. Schauspiel in 3 Akten}|pwk} am 3. 11. 1896 am Deutschen Theater\oindex{Deutsches Theater Berlin@\textbf{Deutsches Theater Berlin}, \emph{Theater (K.THE)}|pwk} in Berlin\oindex{Berlin@\textbf{Berlin}, \emph{P.PPLC}|pwk}. }}}\label{K_L03187-3} werd ich Ihnen aufheben.\hspace*{2em}Hier haben die Leute sehr stark den Eindruck eines grossen
               Erfolges.\pend
           
\pstart
           Herzlich {\\[\baselineskip]}Ihr {\\[\baselineskip]}\spacefill\mbox{Salten}\pend
           \leftskip=0em{}\selectlanguage{ngerman}\endnumbering\briefempfaengerindex{Schnitzler, Arthur@\textsc{Schnitzler, Arthur}!zzzSalten, Felix@\emph{von Felix Salten}!1896-11-061@{6. 11. 1896}|)be}\mylabel{L03187h}  \normalsize

\doendnotes{C}
\bigskip
\vfill

\clearpage

\footnotesize

\lohead{\textsc{register}}

% Definiere theindex-Environment komplett neu ohne reledmac
\makeatletter
\renewenvironment{theindex}{%
  \section*{\indexname}%
  \setlength{\parindent}{0pt}%
  \setlength{\parskip}{0pt plus 0.3pt}%
  \let\item\@idxitem
}{%
  \clearpage
}
\makeatother

\IfFileExists{\jobname-pw.ind}{\input{\jobname-pw.ind}}{}

\end{document}

      