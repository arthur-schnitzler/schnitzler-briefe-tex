%% latex-leseansicht-vorspann.tex
%% Vorspann für die Leseansicht.
%% Lädt die gemeinsame Datei latex-vorspann.tex mit nicht gesetztem Schalter.

\newif\ifkorrekturansicht
\korrekturansichtfalse

\input{../tex-inputs/latex-vorspann}


\section[Georg Engländer an Arthur Schnitzler, 27. 2. 1919]{L02321 Georg Engländer an Arthur Schnitzler, 27. 2. 1919}
\nopagebreak\mylabel{L02321v}
\rehead{ }\normalsize\beginnumbering\briefempfaengerindex{Schnitzler, Arthur@\textsc{Schnitzler, Arthur}!zzzEngländer, Georg@\emph{von Georg Engländer}!1919-02-271@{27. 2. 1919}|(be}
\toendnotes[C]{\smallbreak\pagebreak[2]}
\correspDesc{Versand  durch Georg Engländer am 27. 2. 1919 in Wien
\newline{}Erhalt  durch Arthur Schnitzler im Zeitraum [27. 2. 1919
                  – 3. 3. 1919?] in Wien}\toendnotes[C]{\smallbreak}
\Standort{DLA, A:Schnitzler, HS.NZ85.1.2889.}
\physDesc{Brief, 1 Blatt, 2 Seiten, 1405 Zeichen
\newline{}Handschrift: schwarze Tinte, lateinische Kurrent
\newline{}Schnitzler: mit rotem Buntstift zwei Unterstreichungen }\toendnotes[C]{\smallbreak}
\pstart
           {\pb}\textcolor{gray}{\textbf{Georg Engländer}}\hfill \textcolor{gray}{\textbf{Wien\oindex{Wien@\textbf{Wien}, \emph{Verwaltungsgebiet}|pw},}} den 27/2 19\pend
           
\pstart
           \textcolor{gray}{\textbf{IX. Nußdorferſtraße Nr. 10\oindex{Wien@\textbf{Wien}!IX., Alsergrund@\textbf{IX., Alsergrund}!Nussdorfer Straße@\textbf{Nussdorfer Straße}, \emph{Straße}|pw}.}}\pend
           
\pstart
           \textcolor{gray}{\textbf{Betrifft: Nachlaß \textbf{Peter Altenberg\pwindex{Altenberg, Peter 9.\,3.\,1859 Wien – 8.\,1.\,1919 ebd.@\textsc{Altenberg, Peter} (9.\,3.\,1859 Wien – 8.\,1.\,1919 ebd.), \emph{Schriftsteller}|pw}}.}}\pend
           
\pstart{}Geehrter Meister!\pend\vspace{0.5em}
\pstart
           Erst heute kann ich meinen tiefinnigsten Dank für die so schönen {\kaufmannsund} ehrenvollen Worte abstatten, die Sie werther Meister
               anlässlich Ihrer Condolenz meinem Bruder\pwindex{Altenberg, Peter 9.\,3.\,1859 Wien – 8.\,1.\,1919 ebd.@\textsc{Altenberg, Peter} (9.\,3.\,1859 Wien – 8.\,1.\,1919 ebd.), \emph{Schriftsteller}|pwv} geſpendet; lt. innliegendem Kouvert dessen letzter sichtbarer Stempel
               d. 22/II trägt, hat der Brief eine beinahe 8wöchentliche Wanderung
               durchgemacht bevor er gestern an mich gelangte; so kann ich den Scheine löschen, als
               hätte ich, so werthvolle Freunde {\kaufmannsund} Gönner Peter\pwindex{Altenberg, Peter 9.\,3.\,1859 Wien – 8.\,1.\,1919 ebd.@\textsc{Altenberg, Peter} (9.\,3.\,1859 Wien – 8.\,1.\,1919 ebd.), \emph{Schriftsteller}|pw}\textsuperscript{s} nicht, sofort u. zu allererst berücksichtigend, \substVorne{}\textsuperscript{mit}\substDazwischen{}in\substHinten{} ergebenster {\kaufmannsund} dankbarster Art, mit Erdwiederung
               bedacht.\pend
           
\pstart
           Ich wünschte Meister, Ihre prognostische Werthung, möge in Erfüllung gehen, ich will
               selbst Alles, als Nachlasserbe, auch dazu thun {\kaufmannsund} denke
               noch in den folgenden Jahren noch 2 oder 3 {\pb}Bände mit
               Hinterlassenem, ausführlicher Biographie, Briefen an Freunde {\kaufmannsund} Freundinnen in seinem Sinne erscheinen zu lassen; auch
               will ich durch Vorträge den Kreis der ihn Verstehenden erweitern.\pend
           
\pstart
           Mittwoch, d. 5 März{ }\uline{½} 6\footnote{\noindent{}\uline{Kl. Konzerthaus-Saal\oindex{Wien@\textbf{Wien}!III., Landstraße@\textbf{III., Landstraße}!Wiener Konzerthaus@\textbf{Wiener Konzerthaus}, \emph{Konzertsaal}|pw}.}{\\}½ 6 Uhr\hspace*{1em}5/III 19.} findet der erste Abend statt, dem ich ein selbst gewähltes Programm mehr
               lyrischen Charakters {\kaufmannsund} doch sehr abwechslungsreich
                  besti{\geminationm}t habe; ich habe mir erlaubt Ihnen werther
               Meister 2 Sitze zugehen zu lassen, wäre besonders geehrt wenn Sie davon Gebrauch
               machen, um Ihr mir besonders maassgebendes Urtheil für diese Form der beabsichtigten
               litterarischen Popularisirung des Verewigten\pwindex{Altenberg, Peter 9.\,3.\,1859 Wien – 8.\,1.\,1919 ebd.@\textsc{Altenberg, Peter} (9.\,3.\,1859 Wien – 8.\,1.\,1919 ebd.), \emph{Schriftsteller}|pwv}, erfahren zu können.\pend
           
\pstart
           In grösster Hochachtung{\\[\baselineskip]}Ihr ganz ergebenster{\\[\baselineskip]}\spacefill\mbox{G. Engländer}\pend
           \leftskip=0em{}\selectlanguage{ngerman}\endnumbering\briefempfaengerindex{Schnitzler, Arthur@\textsc{Schnitzler, Arthur}!zzzEngländer, Georg@\emph{von Georg Engländer}!1919-02-271@{27. 2. 1919}|)be}\mylabel{L02321h}  \newcommand{\dateiname}{L02321}\newcommand{\titel}{Georg Engländer an Arthur Schnitzler, 27. 2. 1919}\newcommand{\editorInnen}{Martin Anton Müller und Gerd-Hermann Susen}%% latex-leseansicht-abspann.tex
%% Abspann für die Leseansicht.
%% Der Schalter \ifkorrekturansicht ist bereits durch den Vorspann gesetzt.

%% latex-abspann.tex
%% Gemeinsamer Abspann für Korrekturansicht und Leseansicht.
%% Setzt den Schalter \ifkorrekturansicht voraus (gesetzt in den
%% einbindenden Dateien latex-korrekturansicht-abspann.tex bzw.
%% latex-leseansicht-abspann.tex).
%% ---------------------------------------------------------------

\normalsize

% Das esempio-Environment wird nur in der Leseansicht benötigt
\ifkorrekturansicht\else
\newenvironment{esempio}[3]%
{
    \vspace{1.5ex}
    \rlap{\underline{#1}}
    \par
    \setlength{\parindent}{0cm}
    \nopagebreak
    \leftskip=#2cm
    \rightskip=#3cm
}
{
    \par
}
\fi

\doendnotes{C}
\bigskip
\vfill

\clearpage

\footnotesize

\ifkorrekturansicht
  \lohead{\textsc{register}}
\fi

% theindex-Environment neu definieren ohne reledmac
\makeatletter
\renewenvironment{theindex}{%
  \ifkorrekturansicht
    \section*{\indexname}%
  \else
    \subsubsection*{Index der erwähnten Entitäten}%
  \fi
  \setlength{\parindent}{0pt}%
  \setlength{\parskip}{0pt plus 0.3pt}%
  \let\item\@idxitem
}{%
  \ifkorrekturansicht\clearpage\fi
}
\makeatother

\IfFileExists{\jobname-pw.ind}{\input{\jobname-pw.ind}}{}

% Quellenangabe nur in der Leseansicht
\ifkorrekturansicht\else
% Fallback-Definitionen, falls die .tex-Datei \titel etc. nicht gesetzt hat
\providecommand{\titel}{}
\providecommand{\editorInnen}{}
\providecommand{\dateiname}{\jobname}

\vspace{3cm}

\vfill

\footnotesize
\textsc{Quelle}: \titel. Herausgegeben von {\editorInnen}. In: \emph{Arthur Schnitzler: Briefwechsel mit Autorinnen und Autoren}.
 Digitale Edition, https://schnitzler-briefe.acdh.oeaw.ac.at/{\dateiname}.html (Stand \today)
\fi

\end{document}


