\input{../tex-inputs/latex-pdf-vorspann}
\begin{center}
            \textcolor{red}{ENTWURF. ENTZIFFERUNG NOCH NICHT KORREKTURGELESEN}
                      \end{center}
            
               \section[Arthur Schnitzler an Hermann Bahr, 23. 3. 1897]{ Arthur Schnitzler an Hermann Bahr, 23. 3. 1897}\nopagebreak\mylabel{v}\rehead{ }\begin{ledgroupsized}[t]{13cm}\normalsize\beginnumbering\briefempfaengerindex{Bahr, Hermann@\textsc{Bahr, Hermann}!zzzSchnitzler, Arthur@\emph{von Arthur Schnitzler}!1897-03-231@{23. 3. 1897}|(be} \toendnotes[C]{\smallbreak\pagebreak[2]} \Standort{TMW, HS AM 23329 Ba.}
\physDesc{Brief, 1 Blatt, 4 Seiten
\newline{}Handschrift: Bleistift, deutsche Kurrent\newline{}Ordnung: Lochung }\buchAbdrucke{\weitereDrucke{1) \emph{23. 3. 1897.} In: Arthur Schnitzler: \emph{The Letters of Arthur Schnitzler to Hermann Bahr}. Edited, annotated, and with an introduction, by Donald G.
                        Daviau. Chapel Hill: \emph{The University of North Carolina Press} 1978, S. 60–61 (University of North Carolina studies in the Germanic languages
                        and literatures, 89).} \weitereDrucke{2) Hermann Bahr, Arthur Schnitzler: \emph{Briefwechsel, Aufzeichnungen, Dokumente (1891–1931)}. Hg. Kurt Ifkovits und Martin Anton Müller. Göttingen: \emph{Wallstein} 2018, S. 139–140.} }\toendnotes[C]{\smallbreak}\pstart
           \noindent{}{\pb}Lieber Hermann, wie ka{\geminationn} ich dir den
               Titel ſagen, wenn ich noch nicht weiſs was ich leſe? Das zu entſcheiden ko{\geminationm}en wir ja morgen zuſa{\geminationm}en.
               Wahrſcheinlich eine \label{K_L00658_1v}\edtext{Novellette\pwindex{Schnitzler, Arthur 15.05.1862 – 21.10.1931@\textsc{Schnitzler, Arthur} (15.05.1862 – 21.10.1931), \emph{Schriftsteller, Mediziner}!Ehrentag15. 01. 1898@\strich\emph{Der Ehrentag} {[}15. 01. 1898{]}|pwv}}{\lemma{\textnormal{\emph{Novellette}}}\Cendnote{\textnormal{\emph{Der Ehrentag}\pwindex{Schnitzler, Arthur 15.05.1862 – 21.10.1931@\textsc{Schnitzler, Arthur} (15.05.1862 – 21.10.1931), \emph{Schriftsteller, Mediziner}!Ehrentag15. 01. 1898@\strich\emph{Der Ehrentag} {[}15. 01. 1898{]}|pwk} (Erstdruck in: \emph{Die Romanwelt}\orgindex{Romanwelt@Romanwelt|pwk}, Jg. 5 (1897/1898),
                     H. 16, {[}15.{]} 1. 1898, S. 507–516).}}}\label{K_L00658_1h}, die ich
               vorgeſtern zu Ende geſchrieben, {\pb}vielleicht \label{K_L00658_2v}\edtext{eine, die morgen fertig wird\pwindex{Schnitzler, Arthur 15.05.1862 – 21.10.1931@\textsc{Schnitzler, Arthur} (15.05.1862 – 21.10.1931), \emph{Schriftsteller, Mediziner}!Toten schweigen01. 10. 1897@\strich\emph{Die Toten schweigen} {[}01. 10. 1897{]}|pwv}}{\lemma{\textnormal{\emph{eine, … wird}}}\Cendnote{\textnormal{\emph{Die Toten schweigen}\pwindex{Schnitzler, Arthur 15.05.1862 – 21.10.1931@\textsc{Schnitzler, Arthur} (15.05.1862 – 21.10.1931), \emph{Schriftsteller, Mediziner}!Toten schweigen01. 10. 1897@\strich\emph{Die Toten schweigen} {[}01. 10. 1897{]}|pwk} (Erstdruck in: \emph{Cosmopolis}\pwindex{Cosmopolis1896 – 1898@\emph{Cosmopolis}|pwk}, Jg. 2, Bd. 8, Nr. 22,
                        1. 10. 1897, S. 193–211).}}}\label{K_L00658_2h} – am Ende was ganz
               anderes. Es iſt nemlich zu bedenken dſs du, Hirſchfeld\pwindex{Hirschfeld, Georg 11.02.1873 – 17.01.1942@\textsc{Hirschfeld, Georg} (11.02.1873 – 17.01.1942), \emph{Schriftsteller}|pw} und ich Novelletten leſen, (Hugo\pwindex{Hofmannsthal, Hugo von 01.02.1874 – 15.07.1929@\textsc{Hofmannsthal, Hugo von} (01.02.1874 – 15.07.1929), \emph{Schriftsteller}|pw} wirkt nicht mit) – daſs alſo das Progra{\geminationm}
               von einer beiſpielloſen Langwei{\pb}ligkeit ſein wird.
               Meine Hoffnung iſt, dſs uns morgen Abend doch noch was geſcheidtes einfällt. – Hirſchfelds\pwindex{Hirschfeld, Georg 11.02.1873 – 17.01.1942@\textsc{Hirschfeld, Georg} (11.02.1873 – 17.01.1942), \emph{Schriftsteller}|pw} Geſchichte heißt: »\label{K_L00658_3v}\edtext{Bei beiden\pwindex{Hirschfeld, Georg 11.02.1873 – 17.01.1942@\textsc{Hirschfeld, Georg} (11.02.1873 – 17.01.1942), \emph{Schriftsteller}!Bei Beiden1894@\strich\emph{Bei Beiden} {[}1894{]}|pw}}{\lemma{\textnormal{\emph{Bei beiden}}}\Cendnote{\textnormal{Erstdruck in: \emph{Neue deutsche Rundschau}\pwindex{Neue Deutsche Rundschau1894-01-01 – 1903-12-31@\emph{Neue Deutsche Rundschau}|pwk},
                     Jg. 5, H. 10, 1. 10. 1894, S. 919–927, Erstausgabe in
                        \emph{Dämon Kleist. Novellen}\pwindex{Hirschfeld, Georg 11.02.1873 – 17.01.1942@\textsc{Hirschfeld, Georg} (11.02.1873 – 17.01.1942), \emph{Schriftsteller}!Daemon Kleist1895@\strich\emph{Dämon Kleist} {[}1895{]}|pwk}. Berlin: \emph{S. Fischer}\orgindex{S. Fischer Verlag@S. Fischer Verlag|pwk}{ }1895, S. 152–179.}}}\label{K_L00658_3h}.« Von mir ka{\geminationn}ſt du ſagen, daſs ich eine ungedruckte Novellette
               vorleſen werde. We{\geminationn} das Programm Freitag gedruckt wird,
               iſt Zeit genug, meiner Ansicht nach. Sterben {\pb}ſterb’ \damage{ich}, aber hetzen l\damage{a}ſs ich mich nicht.\pend
           \pstart Herzlich dein \spacefill\mbox{Arthur}\pend{}\pstart
           23. 3. 97.\pend
           \pstart
           Der \label{K_L00658_4v}\edtext{Donnerſtag Notiz}{\lemma{\textnormal{\emph{Donnerſtag Notiz}}}\Cendnote{\textnormal{nicht nachgewiesen}}}\label{K_L00658_4h} wäre
                  jedenfalls mehr Geſchmack zu wünſchen als \label{K_L00658_5v}\edtext{die von Sonntag\pwindex{?? Werk@Nicht ermittelte Verfasserinnen und Verfasser!Ankuendigung der Vorlesung]21. 03. 1897@\emph{[Ankündigung der Vorlesung]} {[}21. 03. 1897{]}|pwv}}{\lemma{\textnormal{\emph{die von Sonntag}}}\Cendnote{\textnormal{\emph{}\pwindex{?? Werk@Nicht ermittelte Verfasserinnen und Verfasser!Ankuendigung der Vorlesung]21. 03. 1897@\emph{[Ankündigung der Vorlesung]} {[}21. 03. 1897{]}|pwk}Etwa in: \emph{Neue
                           Freie Presse}\orgindex{Neue Freie Presse@Neue Freie Presse|pwk}, 21. 3. 1897, S. 9: »– Am
                        Sonntag den 28. d., Abends, findet im Bösendorfer-Saale\oindex{Boesendorfer-Saal@\textbf{Bösendorfer-Saal}|pw} eine Vorlesung statt, die von vier der
                        bekanntesten Vertreter jungdeutscher Literatur zu wohlthätigem Zwecke
                        veranstaltet wird. Am Vorlesertische werden erscheinen als Interpreten ihrer
                        eigenen Werke: Hermann \so{Bahr}\pwindex{Bahr, Hermann 19.07.1863 – 15.01.1934@\textsc{Bahr, Hermann} (19.07.1863 – 15.01.1934), \emph{Schriftsteller, Kritiker}|pw}, der erst jüngst anläßlich der Aufführung seines ›Tschaperl\pwindex{Bahr, Hermann 19.07.1863 – 15.01.1934@\textsc{Bahr, Hermann} (19.07.1863 – 15.01.1934), \emph{Schriftsteller, Kritiker}!Tschaperl1896@\strich\emph{Das Tschaperl} {[}1896{]}|pw}‹ so vielbesprochene Führer Jung-Wien\oindex{Wien@\textbf{Wien}|pw}s; Arthur \so{Schnitzler}\pwindex{Schnitzler, Arthur 15.05.1862 – 21.10.1931@\textsc{Schnitzler, Arthur} (15.05.1862 – 21.10.1931), \emph{Schriftsteller, Mediziner}|pw}, der Verfasser der ›Liebelei\pwindex{Schnitzler, Arthur 15.05.1862 – 21.10.1931@\textsc{Schnitzler, Arthur} (15.05.1862 – 21.10.1931), \emph{Schriftsteller, Mediziner}!Liebelei. Schauspiel in drei Akten9. 10. 1895@\strich\emph{Liebelei. Schauspiel in drei Akten} {[}9. 10. 1895{]}|pw}‹; Hugo \so{v. Hoffmannsthal}\pwindex{Hofmannsthal, Hugo von 01.02.1874 – 15.07.1929@\textsc{Hofmannsthal, Hugo von} (01.02.1874 – 15.07.1929), \emph{Schriftsteller}|pw} (Loris\pwindex{Hofmannsthal, Hugo von 01.02.1874 – 15.07.1929@\textsc{Hofmannsthal, Hugo von} (01.02.1874 – 15.07.1929), \emph{Schriftsteller}|pw}), ein interessantes Talent
                        des modernen Oesterreich\oindex{Oesterreich@\textbf{Österreich}|pw}, und Georg \so{Hirschfeld}\pwindex{Hirschfeld, Georg 11.02.1873 – 17.01.1942@\textsc{Hirschfeld, Georg} (11.02.1873 – 17.01.1942), \emph{Schriftsteller}|pw}, dessen ›Mütter\pwindex{Hirschfeld, Georg 11.02.1873 – 17.01.1942@\textsc{Hirschfeld, Georg} (11.02.1873 – 17.01.1942), \emph{Schriftsteller}!Muetter. Schauspiel in vier Acten1896@\strich\emph{Die Mütter. Schauspiel in vier Acten} {[}1896{]}|pw}‹ vor Kurzem am Deutschen Volkstheater\oindex{Volkstheater@\textbf{Volkstheater}|pw} einen
                        Sensations-Erfolg errangen. Bürgen schon die Namen der Vorleser für den
                        interessanten Verlauf des Abends, so noch mehr der Umstand, daß die vier
                        Herren fast durchwegs neue oder mindestens für Wien\oindex{Wien@\textbf{Wien}|pw} neue Dichtungen zum Vortrage bringen werden. Der Kartenverkauf
                        für diesen originellen literarischen Abend findet bei Bösendorfer\oindex{Boesendorfer-Saal@\textbf{Bösendorfer-Saal}|pw}{ }statt.«}}}\label{K_L00658_5h} verrieth. Wir ſind
                  ja nicht Mitglieder des Vereins »Gemütliche
                     Harmonie\orgindex{Gemuetliche Harmonie@Gemütliche Harmonie|pw}«, daſs man uns durch \label{K_L00658_6v}\edtext{\textsc{Epitheta}}{\lemma{\textnormal{\emph{Epitheta}}}\Cendnote{\textnormal{schmückende Beiworte}}}\label{K_L00658_6h} erklären
                  muſs.\pend
           \endnumbering\briefempfaengerindex{Bahr, Hermann@\textsc{Bahr, Hermann}!zzzSchnitzler, Arthur@\emph{von Arthur Schnitzler}!1897-03-231@{23. 3. 1897}|)be}\mylabel{h}\end{ledgroupsized}  \newcommand{\dateiname}{L00658}\newcommand{\titel}{Arthur Schnitzler an Hermann Bahr, 23. 3. 1897}\newcommand{\editorInnen}{ Kurt Ifkovits,  Martin Anton Müller}\input{../tex-inputs/latex-pdf-abspann}
      