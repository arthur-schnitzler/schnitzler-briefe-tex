%% latex-korrekturansicht-vorspann.tex
%% Vorspann für die Korrekturansicht.
%% Lädt die gemeinsame Datei latex-vorspann.tex mit gesetztem Schalter.

\newif\ifkorrekturansicht
\korrekturansichttrue

\input{../tex-inputs/latex-vorspann}


\section[Arthur Schnitzler an Hermann Bahr, 23. 3. 1897]{L00658 Arthur Schnitzler an Hermann Bahr, 23. 3. 1897}
\nopagebreak\mylabel{L00658v}
\rehead{ }\normalsize\beginnumbering\briefempfaengerindex{Bahr, Hermann@\textsc{Bahr, Hermann}!zzzSchnitzler, Arthur@\emph{von Arthur Schnitzler}!1897-03-231@{23. 3. 1897}|(be}
\toendnotes[C]{\smallbreak\pagebreak[2]}\Standort{TMW, HS AM 23329 Ba.}
\physDesc{Brief, 1 Blatt, 4 Seiten, 984 Zeichen
\newline{}Handschrift: Bleistift, deutsche Kurrent
\newline{}Ordnung: Lochung }
\buchAbdrucke{\weitereDrucke{1) Arthur Schnitzler: \emph{The Letters of Arthur Schnitzler to Hermann Bahr}. Chapel Hill: \emph{The University of North Carolina Press} 1978, S. 60–61.} \weitereDrucke{2) Hermann Bahr, Arthur Schnitzler: \emph{Briefwechsel, Aufzeichnungen, Dokumente (1891–1931)}. Göttingen: \emph{Wallstein} 2018, S. 139–140.} }\toendnotes[C]{\smallbreak}
\pstart
           \noindent{}{\pb}Lieber Hermann, wie ka{\geminationn} ich dir den
               Titel ſagen, wenn ich noch nicht weiſs was ich leſe? Das zu entſcheiden ko{\geminationm}en wir ja morgen zuſa{\geminationm}en.
               Wahrſcheinlich eine \label{K_L00658-1v}\edtext{Novellette\pwindex{Ehrentag@\emph{Der Ehrentag}|pwv}}{\lemma{\textnormal{\emph{Novellette}}}\Cendnote{\textnormal{\emph{Der Ehrentag}\pwindex{Ehrentag@\emph{Der Ehrentag}|pwk} (Erstdruck in: \emph{Die Romanwelt}\orgindex{Romanwelt@Romanwelt|pwk}, Jg. 5
                        (1897/1898), H. 16, {[}15.{]} 1. 1898,
                     S. 507–516).}}}\label{K_L00658-1}, die ich vorgeſtern zu Ende geſchrieben, {\pb}vielleicht \label{K_L00658-2v}\edtext{eine, die morgen fertig wird\pwindex{Toten schweigen@\emph{Die Toten schweigen}|pwv}}{\lemma{\textnormal{\emph{eine, … wird}}}\Cendnote{\textnormal{\emph{Die Toten schweigen}\pwindex{Toten schweigen@\emph{Die Toten schweigen}|pwk} (Erstdruck in: \emph{Cosmopolis}\pwindex{Cosmopolis@\emph{Cosmopolis}|pwk}, Jg. 2, Bd. 8, Nr. 22,
                        1. 10. 1897, S. 193–211).}}}\label{K_L00658-2} – am Ende was ganz
               anderes. Es iſt nemlich zu bedenken dſs du, Hirſchfeld\pwindex{Hirschfeld, Georg 11.02.1873 – 17.01.1942@\textsc{Hirschfeld, Georg} (11.02.1873 – 17.01.1942), \emph{Schriftsteller/Schriftstellerin}|pw} und ich Novelletten leſen, (Hugo\pwindex{Hofmannsthal, Hugo von 1874-02-01 – 1929-07-15@\textsc{Hofmannsthal, Hugo von} (1874-02-01 – 1929-07-15), \emph{Schriftsteller/Schriftstellerin}|pw} wirkt nicht mit) – daſs alſo das Progra{\geminationm}
               von einer beiſpielloſen Langwei{\pb}ligkeit ſein wird.
               Meine Hoffnung iſt, dſs uns morgen Abend doch noch was geſcheidtes einfällt. – Hirſchfelds\pwindex{Hirschfeld, Georg 11.02.1873 – 17.01.1942@\textsc{Hirschfeld, Georg} (11.02.1873 – 17.01.1942), \emph{Schriftsteller/Schriftstellerin}|pw} Geſchichte heißt: »\label{K_L00658-3v}\edtext{Bei beiden\pwindex{Bei Beiden@\emph{Bei Beiden}|pw}}{\lemma{\textnormal{\emph{Bei beiden}}}\Cendnote{\textnormal{Erstdruck in: \emph{Neue deutsche Rundschau}\pwindex{Neue Deutsche Rundschau@\emph{Neue Deutsche Rundschau}|pwk},
                     Jg. 5, H. 10, 1. 10. 1894, S. 919–927, Erstausgabe in
                        \emph{Dämon Kleist. Novellen}\pwindex{Daemon Kleist@\emph{Dämon Kleist}|pwk}. Berlin: \emph{S. Fischer}\orgindex{S. Fischer Verlag@S. Fischer Verlag|pwk}{ }1895, S. 152–179.}}}\label{K_L00658-3}.« Von mir ka{\geminationn}ſt du ſagen, daſs ich eine ungedruckte Novellette
               vorleſen werde. We{\geminationn} das Programm Freitag gedruckt wird,
               iſt Zeit genug, meiner Ansicht nach. Sterben {\pb}ſterb’ \damage{ich}, aber hetzen l\damage{a}ſs ich mich nicht.\pend
           \pstart Herzlich dein \spacefill\mbox{Arthur}\pend{}
\pstart
           23. 3. 97.\pend
           
\pstart
           Der \label{K_L00658-4v}\edtext{Donnerſtag Notiz}{\lemma{\textnormal{\emph{Donnerſtag Notiz}}}\Cendnote{\textnormal{nicht nachgewiesen}}}\label{K_L00658-4} wäre
                  jedenfalls mehr Geſchmack zu wünſchen als \label{K_L00658-5v}\edtext{die von Sonntag\pwindex{Ankuendigung der Vorlesung]@\emph{[Ankündigung der Vorlesung]}|pwv}}{\lemma{\textnormal{\emph{die von Sonntag}}}\Cendnote{\textnormal{Etwa in: \emph{Neue Freie Presse}\orgindex{Neue Freie Presse@Neue Freie Presse|pwk}, 21. 3. 1897, S. 9: »–
                        Am Sonntag den 28. d., Abends, findet im Bösendorfer-Saale\oindex{Boesendorfer-Saal@\textbf{Bösendorfer-Saal}, \emph{Veranstaltungsgebäude (K.VSB)}|pw} eine Vorlesung statt, die von vier
                        der bekanntesten Vertreter jungdeutscher Literatur zu wohlthätigem Zwecke
                        veranstaltet wird. Am Vorlesertische werden erscheinen als Interpreten ihrer
                        eigenen Werke: Hermann \so{Bahr}\pwindex{Bahr, Hermann 19.07.1863 – 15.01.1934@\textsc{Bahr, Hermann} (19.07.1863 – 15.01.1934), \emph{Schriftsteller/Schriftstellerin, Kritiker/Kritikerin}|pw}, der erst jüngst anläßlich der Aufführung seines ›Tschaperl\pwindex{Tschaperl. Ein Wiener Stueck in vier Aufzuegen@\emph{Das Tschaperl. Ein Wiener Stück in vier Aufzügen}|pw}‹ so vielbesprochene Führer Jung-Wiens\oindex{Wien@\textbf{Wien}, \emph{A.ADM2}|pw}; Arthur \so{Schnitzler}, der Verfasser der ›Liebelei\pwindex{Liebelei. Schauspiel in drei Akten@\emph{Liebelei. Schauspiel in drei Akten}|pw}‹;
                           Hugo \so{v. Hoffmannsthal}\pwindex{Hofmannsthal, Hugo von 1874-02-01 – 1929-07-15@\textsc{Hofmannsthal, Hugo von} (1874-02-01 – 1929-07-15), \emph{Schriftsteller/Schriftstellerin}|pw} (Loris\pwindex{Hofmannsthal, Hugo von 1874-02-01 – 1929-07-15@\textsc{Hofmannsthal, Hugo von} (1874-02-01 – 1929-07-15), \emph{Schriftsteller/Schriftstellerin}|pw}), ein interessantes
                        Talent des modernen Oesterreich\oindex{Oesterreich@\textbf{Österreich}, \emph{A.PCLI}|pw}, und
                           Georg \so{Hirschfeld}\pwindex{Hirschfeld, Georg 11.02.1873 – 17.01.1942@\textsc{Hirschfeld, Georg} (11.02.1873 – 17.01.1942), \emph{Schriftsteller/Schriftstellerin}|pw}, dessen ›Mütter\pwindex{Muetter. Schauspiel in vier Acten@\emph{Die Mütter. Schauspiel in vier Acten}|pw}‹ vor Kurzem am
                        Deutschen Volkstheater\orgindex{Volkstheater@Volkstheater|pw} einen
                        Sensations-Erfolg errangen. Bürgen schon die Namen der Vorleser für den
                        interessanten Verlauf des Abends, so noch mehr der Umstand, daß die vier
                        Herren fast durchwegs neue oder mindestens für Wien\oindex{Wien@\textbf{Wien}, \emph{A.ADM2}|pw} neue Dichtungen zum Vortrage bringen werden. Der Kartenverkauf
                        für diesen originellen literarischen Abend findet bei Bösendorfer\oindex{Boesendorfer-Saal@\textbf{Bösendorfer-Saal}, \emph{Veranstaltungsgebäude (K.VSB)}|pw}{ }statt.\pwindex{Ankuendigung der Vorlesung]@\emph{[Ankündigung der Vorlesung]}|pwv}«}}}\label{K_L00658-5} verrieth. Wir ſind
                  ja nicht Mitglieder des Vereins »Gemütliche
                     Harmonie\orgindex{Gemuetliche Harmonie@Gemütliche Harmonie|pw}«, daſs man uns durch \label{K_L00658-6v}\edtext{\textsc{Epitheta}}{\lemma{\textnormal{\emph{Epitheta}}}\Cendnote{\textnormal{schmückende Beiworte}}}\label{K_L00658-6} erklären
                  muſs.\pend
           \selectlanguage{ngerman}\endnumbering\briefempfaengerindex{Bahr, Hermann@\textsc{Bahr, Hermann}!zzzSchnitzler, Arthur@\emph{von Arthur Schnitzler}!1897-03-231@{23. 3. 1897}|)be}\mylabel{L00658h}  \normalsize

\doendnotes{C}
\bigskip
\vfill

\clearpage

\footnotesize

\lohead{\textsc{register}}

% Definiere theindex-Environment komplett neu ohne reledmac
\makeatletter
\renewenvironment{theindex}{%
  \section*{\indexname}%
  \setlength{\parindent}{0pt}%
  \setlength{\parskip}{0pt plus 0.3pt}%
  \let\item\@idxitem
}{%
  \clearpage
}
\makeatother

\IfFileExists{\jobname-pw.ind}{\input{\jobname-pw.ind}}{}

\end{document}

      