%% latex-korrekturansicht-vorspann.tex
%% Vorspann für die Korrekturansicht.
%% Lädt die gemeinsame Datei latex-vorspann.tex mit gesetztem Schalter.

\newif\ifkorrekturansicht
\korrekturansichttrue

\input{../tex-inputs/latex-vorspann}


\section[Hermann Bahr an Arthur Schnitzler, 20. 9. 1893]{L00266 Hermann Bahr an Arthur Schnitzler, 20. 9. 1893}
\nopagebreak\mylabel{L00266v}
\rehead{ }\normalsize\beginnumbering\briefempfaengerindex{Schnitzler, Arthur@\textsc{Schnitzler, Arthur}!zzzBahr, Hermann@\emph{von Hermann Bahr}!1893-09-201@{20. 9. 1893}|(be}
\toendnotes[C]{\smallbreak\pagebreak[2]}\Standort{CUL, Schnitzler, B 5b.}
\physDesc{Visitenkarte, 130 Zeichen
\newline{}Handschrift: Bleistift, deutsche Kurrent
\newline{}Schnitzler: 1) mit Bleistift datiert: »20/1 \textcolor{gray}{93}«  2) mit rotem Buntstift die Monatsangabe der Bleistiftdatierung mit
                                    »9« überschrieben und nummeriert:
                                    »14« sowie ein Strich seitlich der Anrede
\newline{}Ordnung: mit Bleistift von unbekannter Hand die Nummerierung mit Rotstift
                                 verdeutlicht und neuerlich nummeriert: »14« }
\buchAbdrucke{\weitereDrucke{Hermann Bahr, Arthur Schnitzler: \emph{Briefwechsel, Aufzeichnungen, Dokumente (1891–1931)}. Göttingen: \emph{Wallstein} 2018, S. 37.} }\toendnotes[C]{\smallbreak}
\pstart
           \centering{}{\pb}\textcolor{gray}{\textbf{Hermann Bahr}}\pend
           
\pstart
           \centering{}\textcolor{gray}{\textbf{Redacteur der »Deutschen
                        Zeitung\orgindex{Deutsche Zeitung@Deutsche Zeitung|pw}«}}\pend
           
\pstart
           \raggedleft{}\textcolor{gray}{\textbf{Wien, III., Salesianergasse 12\oindex{Salesianergasse 12@\textbf{Salesianergasse 12}, \emph{Wohngebäude (K.WHS)}|pw}.}}\pend
           
\pstart{}{\pb}Lieber Freund!\pend\vspace{0.5em}
\pstart
           Ich konnte leider heute vor 4 nicht frei werden, doch hoffe ich Sie
               beſti{\geminationm}t morgen um 3 am \label{K_L00266-1v}\edtext{Burgring\oindex{Wohnung und Ordination Johann Schnitzler Burgring 1@\textbf{Wohnung und Ordination Johann Schnitzler Burgring 1}, \emph{Ordination}|pw}\oindex{Wohnung und Ordination Johann Schnitzler Burgring 1@\textbf{Wohnung und Ordination Johann Schnitzler Burgring 1}, \emph{Ordination}|pwv}}{\lemma{\textnormal{\emph{Burgring}}}\Cendnote{\textnormal{Schnitzler
                  dürfte nach dem Tod seines Vaters\pwindex{Schnitzler, Johann 10.04.1835 – 02.05.1893@\textsc{Schnitzler, Johann} (10.04.1835 – 02.05.1893), \emph{Laryngologe/Laryngologin}|pwkv} dessen Ordination\oindex{Wohnung und Ordination Johann Schnitzler Burgring 1@\textbf{Wohnung und Ordination Johann Schnitzler Burgring 1}, \emph{Ordination}|pwkv} weiter betreut haben.}}}\label{K_L00266-1} zu ſehen.\pend
           
\pstart
           Herzlichſt\pend
           
\pstart
           \raggedleft{}Ihr\pend
           \selectlanguage{ngerman}\endnumbering\briefempfaengerindex{Schnitzler, Arthur@\textsc{Schnitzler, Arthur}!zzzBahr, Hermann@\emph{von Hermann Bahr}!1893-09-201@{20. 9. 1893}|)be}\mylabel{L00266h}  \normalsize

\doendnotes{C}
\bigskip
\vfill

\clearpage

\footnotesize

\lohead{\textsc{register}}

% Definiere theindex-Environment komplett neu ohne reledmac
\makeatletter
\renewenvironment{theindex}{%
  \section*{\indexname}%
  \setlength{\parindent}{0pt}%
  \setlength{\parskip}{0pt plus 0.3pt}%
  \let\item\@idxitem
}{%
  \clearpage
}
\makeatother

\IfFileExists{\jobname-pw.ind}{\input{\jobname-pw.ind}}{}

\end{document}

      