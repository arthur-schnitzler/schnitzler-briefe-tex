\input{../tex-inputs/latex-pdf-vorspann}
\begin{center}
            \textcolor{red}{ENTWURF. ENTZIFFERUNG NOCH NICHT KORREKTURGELESEN}
                      \end{center}
            
               \section[Hermann Bahr an Arthur Schnitzler, 20. 9. 1893]{ Hermann Bahr an Arthur Schnitzler, 20. 9. 1893}\nopagebreak\mylabel{v}\rehead{ }\begin{ledgroupsized}[t]{13cm}\normalsize\beginnumbering\briefempfaengerindex{Schnitzler, Arthur@\textsc{Schnitzler, Arthur}!zzzBahr, Hermann@\emph{von Hermann Bahr}!1893-09-201@{20. 9. 1893}|(be} \toendnotes[C]{\smallbreak\pagebreak[2]} \Standort{CUL, Schnitzler, B 5b.}
\physDesc{Visitenkarte
\newline{}Handschrift: Bleistift, deutsche Kurrent
\newline{}Schnitzler: 1) mit Bleistift datiert: »20/1 \textcolor{gray}{93}« 2) mit rotem Buntstift die Monatsangabe der Bleistiftdatierung mit
                                    »9« überschrieben und nummeriert:
                                    »14« sowie ein Strich seitlich der
                                 Anrede\newline{}Ordnung: mit Bleistift von unbekannter Hand die Nummerierung mit Rotstift
                                 verdeutlicht und neuerlich nummeriert: »14« }\buchAbdrucke{\weitereDrucke{Hermann Bahr, Arthur Schnitzler: \emph{Briefwechsel, Aufzeichnungen, Dokumente (1891–1931)}. Hg. Kurt Ifkovits und Martin Anton Müller. Göttingen: \emph{Wallstein} 2018, S. 37.} }\pstart
           \noindent{}\centering{}{\pb}\textcolor{gray}{\textbf{Hermann Bahr}}\pend
           \pstart
           \noindent{}\centering{}\textcolor{gray}{\textbf{Redacteur der »Deutschen
                        Zeitung\orgindex{Deutsche Zeitung@Deutsche Zeitung|pw}«}}\pend
           \pstart
           \noindent{}\raggedleft{}\textcolor{gray}{\textbf{Wien, III., Salesianergasse 12\oindex{Salesianergasse@\textbf{Salesianergasse}|pw}.}}\pend
           \pstart{}{\pb}Lieber Freund!\pend\pstart
           Ich konnte leider heute vor 4 nicht frei werden, doch hoffe ich Sie
                  beſti{\geminationm}t morgen um 3 am Burgring\oindex{Burgring@\textbf{Burgring}|pw} zu ſehen.\pend
           \pstart
           Herzlichſt\pend
           \pstart
           \raggedleft{}Ihr\pend
           \endnumbering\briefempfaengerindex{Schnitzler, Arthur@\textsc{Schnitzler, Arthur}!zzzBahr, Hermann@\emph{von Hermann Bahr}!1893-09-201@{20. 9. 1893}|)be}\mylabel{h}\end{ledgroupsized}  \newcommand{\dateiname}{L00266}\newcommand{\titel}{Hermann Bahr an Arthur Schnitzler, 20. 9. 1893}\newcommand{\editorInnen}{ Kurt Ifkovits,  Martin Anton Müller}\input{../tex-inputs/latex-pdf-abspann}
      