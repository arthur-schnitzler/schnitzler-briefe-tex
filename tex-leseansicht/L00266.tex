%% latex-leseansicht-vorspann.tex
%% Vorspann für die Leseansicht.
%% Lädt die gemeinsame Datei latex-vorspann.tex mit nicht gesetztem Schalter.

\newif\ifkorrekturansicht
\korrekturansichtfalse

\input{../tex-inputs/latex-vorspann}


\section[Hermann Bahr an Arthur Schnitzler, 20. 9. 1893]{L00266 Hermann Bahr an Arthur Schnitzler, 20. 9. 1893}
\nopagebreak\mylabel{L00266v}
\rehead{ }\normalsize\beginnumbering\briefempfaengerindex{Schnitzler, Arthur@\textsc{Schnitzler, Arthur}!zzzBahr, Hermann@\emph{von Hermann Bahr}!1893-09-201@{20. 9. 1893}|(be}
\toendnotes[C]{\smallbreak\pagebreak[2]}
\correspDesc{Versand  durch Hermann Bahr am 20. 9. 1893 \textbf{Ort fehlend} 
\newline{}Erhalt  durch Arthur Schnitzler im Zeitraum [20. 9. 1893
                  – 24. 9. 1893?] \textbf{Ort fehlend} }\toendnotes[C]{\smallbreak}
\Standort{CUL, Schnitzler, B 5b.}
\physDesc{Visitenkarte, 130 Zeichen
\newline{}Handschrift: Bleistift, deutsche Kurrent
\newline{}Schnitzler: 1) mit Bleistift datiert: »20/1 \textcolor{gray}{93}«  2) mit rotem Buntstift die Monatsangabe der Bleistiftdatierung mit
                                    »9« überschrieben und nummeriert:
                                    »14« sowie ein Strich seitlich der Anrede
\newline{}Ordnung: mit Bleistift von unbekannter Hand die Nummerierung mit Rotstift
                                 verdeutlicht und neuerlich nummeriert: »14« }
\buchAbdrucke{\weitereDrucke{Hermann Bahr, Arthur Schnitzler: \emph{Briefwechsel, Aufzeichnungen, Dokumente (1891–1931)}. Herausgegeben von Kurt Ifkovits und Martin Anton Müller. Göttingen: \emph{Wallstein} 2018, S. 37.} }\toendnotes[C]{\smallbreak}
\pstart
           \centering{}{\pb}\textcolor{gray}{\textbf{Hermann Bahr}}\pend
           
\pstart
           \centering{}\textcolor{gray}{\textbf{Redacteur der »Deutschen
                        Zeitung\orgindex{Deutsche Zeitung@Deutsche Zeitung|pw}«}}\pend
           
\pstart
           \raggedleft{}\textcolor{gray}{\textbf{Wien, III., Salesianergasse 12\oindex{Wien@\textbf{Wien}!III., Landstraße@\textbf{III., Landstraße}!Salesianergasse 12@\textbf{Salesianergasse 12}, \emph{Wohngebäude}|pw}.}}\pend
           
\pstart{}{\pb}Lieber Freund!\pend\vspace{0.5em}
\pstart
           Ich konnte leider heute vor 4 nicht frei werden, doch hoffe ich Sie
               beſti{\geminationm}t morgen um 3 am \label{K_L00266-1v}\edtext{Burgring\oindex{Wien@\textbf{Wien}!I., Innere Stadt@\textbf{I., Innere Stadt}!Wohnung und Ordination Johann Schnitzler Burgring 1@\textbf{Wohnung und Ordination Johann Schnitzler Burgring 1}, \emph{Ordination}|pw}\oindex{Wien@\textbf{Wien}!I., Innere Stadt@\textbf{I., Innere Stadt}!Wohnung und Ordination Johann Schnitzler Burgring 1@\textbf{Wohnung und Ordination Johann Schnitzler Burgring 1}, \emph{Ordination}|pwv}}{\lemma{\textnormal{\emph{Burgring}}}\Cendnote{\textnormal{Schnitzler
                  dürfte nach dem Tod seines Vaters\pwindex{Schnitzler, Johann 10.\,4.\,1835 Nagykanizsa – 2.\,5.\,1893 Wien@\textsc{Schnitzler, Johann} (10.\,4.\,1835 Nagykanizsa – 2.\,5.\,1893 Wien), \emph{Laryngologe}|pwkv} dessen Ordination\oindex{Wien@\textbf{Wien}!I., Innere Stadt@\textbf{I., Innere Stadt}!Wohnung und Ordination Johann Schnitzler Burgring 1@\textbf{Wohnung und Ordination Johann Schnitzler Burgring 1}, \emph{Ordination}|pwkv} weiter betreut haben.}}}\label{K_L00266-1} zu{ }ſehen.\pend
           
\pstart
           Herzlichſt\pend
           
\pstart
           \raggedleft{}Ihr\pend
           \selectlanguage{ngerman}\endnumbering\briefempfaengerindex{Schnitzler, Arthur@\textsc{Schnitzler, Arthur}!zzzBahr, Hermann@\emph{von Hermann Bahr}!1893-09-201@{20. 9. 1893}|)be}\mylabel{L00266h}  \newcommand{\dateiname}{L00266}\newcommand{\titel}{Hermann Bahr an Arthur Schnitzler, 20. 9. 1893}\newcommand{\editorInnen}{Herausgegeben von Martin Anton Müller}%% latex-leseansicht-abspann.tex
%% Abspann für die Leseansicht.
%% Der Schalter \ifkorrekturansicht ist bereits durch den Vorspann gesetzt.

%% latex-abspann.tex
%% Gemeinsamer Abspann für Korrekturansicht und Leseansicht.
%% Setzt den Schalter \ifkorrekturansicht voraus (gesetzt in den
%% einbindenden Dateien latex-korrekturansicht-abspann.tex bzw.
%% latex-leseansicht-abspann.tex).
%% ---------------------------------------------------------------

\normalsize

% Das esempio-Environment wird nur in der Leseansicht benötigt
\ifkorrekturansicht\else
\newenvironment{esempio}[3]%
{
    \vspace{1.5ex}
    \rlap{\underline{#1}}
    \par
    \setlength{\parindent}{0cm}
    \nopagebreak
    \leftskip=#2cm
    \rightskip=#3cm
}
{
    \par
}
\fi

\doendnotes{C}
\bigskip
\vfill

\clearpage

\footnotesize

\ifkorrekturansicht
  \lohead{\textsc{register}}
\fi

% theindex-Environment neu definieren ohne reledmac
\makeatletter
\renewenvironment{theindex}{%
  \ifkorrekturansicht
    \section*{\indexname}%
  \else
    \subsubsection*{Index der erwähnten Entitäten}%
  \fi
  \setlength{\parindent}{0pt}%
  \setlength{\parskip}{0pt plus 0.3pt}%
  \let\item\@idxitem
}{%
  \ifkorrekturansicht\clearpage\fi
}
\makeatother

\IfFileExists{\jobname-pw.ind}{\input{\jobname-pw.ind}}{}

% Quellenangabe nur in der Leseansicht
\ifkorrekturansicht\else
% Fallback-Definitionen, falls die .tex-Datei \titel etc. nicht gesetzt hat
\providecommand{\titel}{}
\providecommand{\editorInnen}{}
\providecommand{\dateiname}{\jobname}

\vspace{3cm}

\vfill

\footnotesize
\textsc{Quelle}: \titel. Herausgegeben von {\editorInnen}. In: \emph{Arthur Schnitzler: Briefwechsel mit Autorinnen und Autoren}.
 Digitale Edition, https://schnitzler-briefe.acdh.oeaw.ac.at/{\dateiname}.html (Stand \today)
\fi

\end{document}


