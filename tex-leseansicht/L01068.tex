%% latex-korrekturansicht-vorspann.tex
%% Vorspann für die Korrekturansicht.
%% Lädt die gemeinsame Datei latex-vorspann.tex mit gesetztem Schalter.

\newif\ifkorrekturansicht
\korrekturansichttrue

\input{../tex-inputs/latex-vorspann}


\section[Arthur Schnitzler und Paul Goldmann an Richard Beer-Hofmann, 26. 8. 1900]{L01068 Arthur Schnitzler und Paul Goldmann an Richard Beer-Hofmann,
               26. 8. 1900}
\nopagebreak\mylabel{L01068v}
\rehead{ }\normalsize\beginnumbering\briefempfaengerindex{Beer-Hofmann, Richard@\textsc{Beer-Hofmann, Richard}!zzzGoldmann, Paul@\emph{von Paul Goldmann}!1900-08-261@{26. 8. 1900}|(be}\briefempfaengerindex{Beer-Hofmann, Richard@\textsc{Beer-Hofmann, Richard}!zzzSchnitzler, Arthur@\emph{von Arthur Schnitzler}!1900-08-261@{26. 8. 1900}|(be}
\toendnotes[C]{\smallbreak\pagebreak[2]}\Standort{YCGL, MSS 31.}
\physDesc{Postkarte, 181 Zeichen
\newline{}Handschrift Paul Goldmann: Bleistift, deutsche Kurrent
\newline{}Handschrift Arthur Schnitzler: Bleistift, deutsche Kurrent
\newline{}Versand: 1) Stempel: »\nobreak{}\oindex{Bormio@\textbf{Bormio}, \emph{P.PPLA3}|pwk}Bagni nuovi di Bormio, 27 \textcolor{gray}{8} 00, 6–7V\nobreak{}«.   2) Stempel: »\nobreak{}\oindex{Altaussee@\textbf{Altaussee}, \emph{A.ADM3}|pwk}\textcolor{gray}{Alt-Aussee}, 30 8 00\nobreak{}«. 
\newline{}Ordnung: mit Bleistift von unbekannter Hand datiert: »26. 8.« }\toendnotes[C]{\smallbreak}\pstart{}\textsc{{\pb}Steiermark\oindex{Steiermark@\textbf{Steiermark}, \emph{A.ADM1}|pw}.}\pend{}\pstart{}{\pb}Dr. \textsc{Richard Beer-Hofmann}\pend{}\pstart{}\textsc{Altaussee}\oindex{Altaussee@\textbf{Altaussee}, \emph{A.ADM3}|pw}\pend{}{\bigskip}\vspace{1em}
\pstart
           \centering{}{\pb}\textcolor{gray}{\textbf{Bormio\oindex{Bormio@\textbf{Bormio}, \emph{P.PPLA3}|pw}}}\pend
           
\pstart
           \centering{}\textcolor{gray}{\textbf{I VECCHI BAGNI\oindex{Alte Therme@\textbf{Alte Therme}, \emph{Gebäude (K.GBD)}|pw}.}}\pend
           
\pstart
           26. 8. 900.\pend
           \vspace{0.5em}
\pstart
           Schönes Wetter, zerriſſene Straßen, \label{K_L01068-1v}\edtext{Tirol\oindex{Tirol@\textbf{Tirol}, \emph{A.ADM1}|pw}\oindex{Suedtirol@\textbf{Südtirol}, \emph{A.ADM2}|pw}er Sänger}{\lemma{\textnormal{\emph{Tiroler Sänger}}}\Cendnote{\textnormal{Das Erlebnis des Sängers/der Sänger dürfte für die Entwicklung
                  der Novelle \emph{Der blinde Geronimo und sein
                     Bruder}\pwindex{blinde Geronimo und sein Bruder@\emph{Der blinde Geronimo und sein Bruder}|pwk} bedeutsam geworden sein, vgl. Paul Goldmann an Arthur Schnitzler, 28. 8. [1900], 
                  Paul Goldmann an Arthur Schnitzler, 18. 2. [1901] und Paul Goldmann\pwindex{Goldmann, Paul 31.01.1865 – 25.09.1935@\textsc{Goldmann, Paul} (31.01.1865 – 25.09.1935), \emph{Schriftsteller/Schriftstellerin, Journalist/Journalistin}|pwk}: \emph{Erinnerungen an Arthur Schnitzler}\pwindex{Erinnerungen an Arthur Schnitzler@\emph{Erinnerungen an Arthur Schnitzler}|pwk}. In: \emph{Neue Freie Presse}\pwindex{Neue Freie Presse@\emph{Neue Freie Presse}|pwk}, Nr. 24.121, 8. 11. 1931, Morgenblatt, S. 25–26, hier: S. 26.}}}\label{K_L01068-1},
               herzliche Grüße.\pend
           \pstart \spacefill\mbox{Arthur}\pend{}\selectlanguage{ngerman}\vspace{1em}
\pstart
           \noindent{}{[}hs. :{]} Ich wünſche Du wäreſt auch da.\pend
           
\pstart
           Herzlichſt{\\[\baselineskip]}\spacefill\mbox{Paul Goldmann.}\pend
           \leftskip=0em{}\selectlanguage{ngerman}\endnumbering\briefempfaengerindex{Beer-Hofmann, Richard@\textsc{Beer-Hofmann, Richard}!zzzGoldmann, Paul@\emph{von Paul Goldmann}!1900-08-261@{26. 8. 1900}|)be}\briefempfaengerindex{Beer-Hofmann, Richard@\textsc{Beer-Hofmann, Richard}!zzzSchnitzler, Arthur@\emph{von Arthur Schnitzler}!1900-08-261@{26. 8. 1900}|)be}\mylabel{L01068h}  \normalsize

\doendnotes{C}
\bigskip
\vfill

\clearpage

\footnotesize

\lohead{\textsc{register}}

% Definiere theindex-Environment komplett neu ohne reledmac
\makeatletter
\renewenvironment{theindex}{%
  \section*{\indexname}%
  \setlength{\parindent}{0pt}%
  \setlength{\parskip}{0pt plus 0.3pt}%
  \let\item\@idxitem
}{%
  \clearpage
}
\makeatother

\IfFileExists{\jobname-pw.ind}{\input{\jobname-pw.ind}}{}

\end{document}

      