%% latex-leseansicht-vorspann.tex
%% Vorspann für die Leseansicht.
%% Lädt die gemeinsame Datei latex-vorspann.tex mit nicht gesetztem Schalter.

\newif\ifkorrekturansicht
\korrekturansichtfalse

\input{../tex-inputs/latex-vorspann}


         
         \renewcommand{\erwaehntePersonen}{Personen: Richard Beer-Hofmann, Paul Goldmann}
         \renewcommand{\erwaehnteOrte}{Orte: Altaussee, Alte Therme, Bormio, Steiermark, Südtirol, Tirol}
         \renewcommand{\erwaehnteWerke}{Werke: Der blinde Geronimo und sein Bruder, Erinnerungen an Arthur Schnitzler, Neue Freie Presse}
               \section[Arthur Schnitzler und Paul Goldmann an Richard Beer-Hofmann, 26. 8. 1900]{ Arthur Schnitzler und Paul Goldmann an Richard Beer-Hofmann,
               26. 8. 1900}\nopagebreak\mylabel{v}\rehead{ }\begin{ledgroupsized}[t]{13cm}\normalsize\beginnumbering\briefempfaengerindex{Beer-Hofmann, Richard@\textsc{Beer-Hofmann, Richard}!zzzGoldmann, Paul@\emph{von Paul Goldmann}!1900-08-261@{26. 8. 1900}|(be}\briefempfaengerindex{Beer-Hofmann, Richard@\textsc{Beer-Hofmann, Richard}!zzzSchnitzler, Arthur@\emph{von Arthur Schnitzler}!1900-08-261@{26. 8. 1900}|(be} \toendnotes[C]{\smallbreak\pagebreak[2]} \Standort{YCGL, MSS 31.}
\physDesc{Postkarte, 181 Zeichen
\newline{}Handschrift Paul Goldmann: Bleistift, deutsche Kurrent\newline{}Handschrift Arthur Schnitzler: Bleistift, deutsche Kurrent
\newline{}Versand: 1) Stempel: »\nobreak{}\oindex{Bormio@\textbf{Bormio}|pwk}Bagni nuovi di Bormio, 27 \textcolor{gray}{8} 00, 6–7V\nobreak{}«.   2) Stempel: »\nobreak{}\oindex{Altaussee@\textbf{Altaussee}|pwk}\textcolor{gray}{Alt-Aussee}, 30 8 00\nobreak{}«. 
\newline{}Ordnung: mit Bleistift von unbekannter Hand datiert: »26. 8.« }\toendnotes[C]{\smallbreak}\pstart{}\textsc{{\pb}Steiermark\oindex{Steiermark@\textbf{Steiermark}|pw}.}\pend{}\pstart{}{\pb}Dr. \textsc{Richard Beer-Hofmann}\pend{}\pstart{}\textsc{Altaussee}\oindex{Altaussee@\textbf{Altaussee}|pw}\pend{}{\bigskip}\pstart
           \noindent{}\centering{}{\pb}\textcolor{gray}{\textbf{Bormio\oindex{Bormio@\textbf{Bormio}|pw}}}\pend
           \pstart
           \noindent{}\centering{}\textcolor{gray}{\textbf{I VECCHI BAGNI\oindex{Alte Therme@\textbf{Alte Therme}|pw}.}}\pend
           \pstart
           26. 8. 900.\pend
           \pstart
           Schönes Wetter, zerriſſene Straßen, \label{K_L01068-1v}\edtext{Tirol\oindex{Tirol@\textbf{Tirol}|pw}\oindex{Suedtirol@\textbf{Südtirol}|pw}er Sänger}{\lemma{\textnormal{\emph{Tiroler Sänger}}}\Cendnote{\textnormal{Das Erlebnis des Sängers/der Sänger dürfte für die Entwicklung
                  der Novelle \emph{Der blinde Geronimo und sein
                     Bruder}\pwindex{Schnitzler, Arthur 15.05.1862 – 21.10.1931@\textsc{Schnitzler, Arthur} (15.05.1862 – 21.10.1931), \emph{Schriftsteller, Mediziner}!blinde Geronimo und sein Bruder22.12.1900 – 12.1.1901@\strich\emph{Der blinde Geronimo und sein Bruder} {[}22.12.1900 – 12.1.1901{]}|pwk} bedeutsam geworden sein, vgl. Paul Goldmann an Arthur Schnitzler, 28. 8. [1900], 
                  Paul Goldmann an Arthur Schnitzler, 18. 2. [1901] und Paul Goldmann\pwindex{Goldmann, Paul 31.01.1865 – 25.09.1935@\textsc{Goldmann, Paul} (31.01.1865 – 25.09.1935), \emph{Schriftsteller, Journalist}|pwk}: \emph{Erinnerungen an Arthur Schnitzler}\pwindex{Goldmann, Paul 31.01.1865 – 25.09.1935@\textsc{Goldmann, Paul} (31.01.1865 – 25.09.1935), \emph{Schriftsteller, Journalist}!Erinnerungen an Arthur Schnitzler1931-11-08@\strich\emph{Erinnerungen an Arthur Schnitzler} {[}1931-11-08{]}|pwk}. In: \emph{Neue Freie Presse}\pwindex{Neue Freie Presse1864 – 1939@\emph{Neue Freie Presse} {[}1864 – 1939{]}|pwk}, Nr. 24.121, 8. 11. 1931, Morgenblatt, S. 25–26, hier: S. 26.}}}\label{K_L01068-1h},
               herzliche Grüße.\pend
           \pstart \spacefill\mbox{Arthur}\pend{}\pstart
           \noindent{}{[}hs. Goldmann:{]} Ich wünſche Du wäreſt auch da.\pend
           \pstart
           Herzlichſt{\\[\baselineskip]}\spacefill\mbox{Paul Goldmann.}\pend
           \leftskip=0em{}
         
         \endnumbering\mylabel{h}\end{ledgroupsized}  \newcommand{\dateiname}{L01068}\newcommand{\titel}{Arthur Schnitzler und Paul Goldmann an Richard Beer-Hofmann, 26. 8. 1900}\newcommand{\editorInnen}{Martin Anton Müller und Gerd-Hermann Susen}%% latex-leseansicht-abspann.tex
%% Abspann für die Leseansicht.
%% Der Schalter \ifkorrekturansicht ist bereits durch den Vorspann gesetzt.

%% latex-abspann.tex
%% Gemeinsamer Abspann für Korrekturansicht und Leseansicht.
%% Setzt den Schalter \ifkorrekturansicht voraus (gesetzt in den
%% einbindenden Dateien latex-korrekturansicht-abspann.tex bzw.
%% latex-leseansicht-abspann.tex).
%% ---------------------------------------------------------------

\normalsize

% Das esempio-Environment wird nur in der Leseansicht benötigt
\ifkorrekturansicht\else
\newenvironment{esempio}[3]%
{
    \vspace{1.5ex}
    \rlap{\underline{#1}}
    \par
    \setlength{\parindent}{0cm}
    \nopagebreak
    \leftskip=#2cm
    \rightskip=#3cm
}
{
    \par
}
\fi

\doendnotes{C}
\bigskip
\vfill

\clearpage

\footnotesize

\ifkorrekturansicht
  \lohead{\textsc{register}}
\fi

% theindex-Environment neu definieren ohne reledmac
\makeatletter
\renewenvironment{theindex}{%
  \ifkorrekturansicht
    \section*{\indexname}%
  \else
    \subsubsection*{Index der erwähnten Entitäten}%
  \fi
  \setlength{\parindent}{0pt}%
  \setlength{\parskip}{0pt plus 0.3pt}%
  \let\item\@idxitem
}{%
  \ifkorrekturansicht\clearpage\fi
}
\makeatother

\IfFileExists{\jobname-pw.ind}{\input{\jobname-pw.ind}}{}

% Quellenangabe nur in der Leseansicht
\ifkorrekturansicht\else
% Fallback-Definitionen, falls die .tex-Datei \titel etc. nicht gesetzt hat
\providecommand{\titel}{}
\providecommand{\editorInnen}{}
\providecommand{\dateiname}{\jobname}

\vspace{3cm}

\vfill

\footnotesize
\textsc{Quelle}: \titel. Herausgegeben von {\editorInnen}. In: \emph{Arthur Schnitzler: Briefwechsel mit Autorinnen und Autoren}.
 Digitale Edition, https://schnitzler-briefe.acdh.oeaw.ac.at/{\dateiname}.html (Stand \today)
\fi

\end{document}


      