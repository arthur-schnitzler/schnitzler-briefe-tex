%% latex-leseansicht-vorspann.tex
%% Vorspann für die Leseansicht.
%% Lädt die gemeinsame Datei latex-vorspann.tex mit nicht gesetztem Schalter.

\newif\ifkorrekturansicht
\korrekturansichtfalse

\input{../tex-inputs/latex-vorspann}


\section[Arthur Schnitzler und Paul Goldmann an Richard Beer-Hofmann, 26. 8. 1900]{L01068 Arthur Schnitzler und Paul Goldmann an Richard Beer-Hofmann, 26. 8. 1900}
\nopagebreak\mylabel{L01068v}
\rehead{ }\normalsize\beginnumbering\briefempfaengerindex{Beer-Hofmann, Richard@\textsc{Beer-Hofmann, Richard}!zzzGoldmann, Paul@\emph{von Paul Goldmann}!1900-08-261@{26. 8. 1900}|(be}\briefempfaengerindex{Beer-Hofmann, Richard@\textsc{Beer-Hofmann, Richard}!zzzSchnitzler, Arthur@\emph{von Arthur Schnitzler}!1900-08-261@{26. 8. 1900}|(be}
\toendnotes[C]{\smallbreak\pagebreak[2]}
\correspDesc{Versand  durch Arthur Schnitzler, Paul Goldmann am 26. 8. 1900 in Bormio
\newline{}Übermittlung  am 27. 8. 1900 in Bormio
\newline{}Erhalt  durch Richard Beer-Hofmann am 30. 8. 1900 in Altaussee}\toendnotes[C]{\smallbreak}
\Standort{YCGL, MSS 31.}
\physDesc{Postkarte, 181 Zeichen
\newline{}Handschrift Paul Goldmann: Bleistift, deutsche Kurrent
\newline{}Handschrift Arthur Schnitzler: Bleistift, deutsche Kurrent
\newline{}Versand: 1) Stempel: »\nobreak{}\oindex{Bormio@\textbf{Bormio}, \emph{Hauptstadt}|pwk}Bagni nuovi di Bormio, 27 \textcolor{gray}{8} 00, 6–7V\nobreak{}«.   2) Stempel: »\nobreak{}\oindex{Altaussee@\textbf{Altaussee}, \emph{Verwaltungsgebiet}|pwk}\textcolor{gray}{Alt-Aussee}, 30 8 00\nobreak{}«. 
\newline{}Ordnung: mit Bleistift von unbekannter Hand datiert: »26. 8.« }\toendnotes[C]{\smallbreak}\pstart{}\textsc{{\pb}Steiermark\oindex{Steiermark@\textbf{Steiermark}, \emph{Land}|pw}.}\pend{}\pstart{}{\pb}Dr. \textsc{Richard Beer-Hofmann}\pend{}\pstart{}\textsc{Altaussee}\oindex{Altaussee@\textbf{Altaussee}, \emph{Verwaltungsgebiet}|pw}\pend{}{\bigskip}\vspace{1em}
\pstart
           \centering{}{\pb}\textcolor{gray}{\textbf{Bormio\oindex{Bormio@\textbf{Bormio}, \emph{Hauptstadt}|pw}}}\pend
           
\pstart
           \centering{}\textcolor{gray}{\textbf{I VECCHI BAGNI\oindex{Alte Therme@\textbf{Alte Therme}, \emph{Gebäude}|pw}.}}\pend
           
\pstart
           26. 8. 900.\pend
           \vspace{0.5em}
\pstart
           Schönes Wetter, zerriſſene Straßen, \label{K_L01068-1v}\edtext{Tirol\oindex{Tirol@\textbf{Tirol}, \emph{Land}|pw}\oindex{Südtirol@\textbf{Südtirol}, \emph{Verwaltungsgebiet}|pw}er Sänger}{\lemma{\textnormal{\emph{Tiroler Sänger}}}\Cendnote{\textnormal{Das Erlebnis des Sängers/der Sänger dürfte für die Entwicklung
                  der Novelle \emph{Der blinde Geronimo und sein
                     Bruder}\pwindex{Schnitzler, Arthur 15.\,5.\,1862 Wien – 21.\,10.\,1931 ebd.@\textsc{Schnitzler, Arthur} (15.\,5.\,1862 Wien – 21.\,10.\,1931 ebd.), \emph{Schriftsteller, Mediziner}!blinde Geronimo und sein Bruder@\strich\emph{Der blinde Geronimo und sein Bruder}|pwk} bedeutsam geworden sein, vgl. XXXX Auszeichnungsfehler: Dokument L02928 nicht gefunden, 
                  XXXX Auszeichnungsfehler: Dokument L03059 nicht gefunden und Paul Goldmann\pwindex{Goldmann, Paul 31.\,1.\,1865 Breslau – 25.\,9.\,1935 Wien@\textsc{Goldmann, Paul} (31.\,1.\,1865 Breslau – 25.\,9.\,1935 Wien), \emph{Schriftsteller, Journalist}|pwk}: \emph{Erinnerungen an Arthur Schnitzler}\pwindex{Goldmann, Paul 31.\,1.\,1865 Breslau – 25.\,9.\,1935 Wien@\textsc{Goldmann, Paul} (31.\,1.\,1865 Breslau – 25.\,9.\,1935 Wien), \emph{Schriftsteller, Journalist}!Erinnerungen an Arthur Schnitzler@\strich\emph{Erinnerungen an Arthur Schnitzler}|pwk}. In: \emph{Neue Freie Presse}\pwindex{Neue Freie Presse@\emph{Neue Freie Presse}|pwk}, Nr. 24.121, 8. 11. 1931, Morgenblatt, S. 25–26, hier: S. 26.}}}\label{K_L01068-1},
               herzliche Grüße.\pend
           \pstart \spacefill\mbox{Arthur}\pend{}\selectlanguage{ngerman}\vspace{1em}
\pstart
           \noindent{}{[}hs. Goldmann:{]} Ich wünſche Du wäreſt auch da.\pend
           
\pstart
           Herzlichſt{\\[\baselineskip]}\spacefill\mbox{Paul Goldmann.}\pend
           \leftskip=0em{}\selectlanguage{ngerman}\endnumbering\briefempfaengerindex{Beer-Hofmann, Richard@\textsc{Beer-Hofmann, Richard}!zzzGoldmann, Paul@\emph{von Paul Goldmann}!1900-08-261@{26. 8. 1900}|)be}\briefempfaengerindex{Beer-Hofmann, Richard@\textsc{Beer-Hofmann, Richard}!zzzSchnitzler, Arthur@\emph{von Arthur Schnitzler}!1900-08-261@{26. 8. 1900}|)be}\mylabel{L01068h}  \newcommand{\dateiname}{L01068}\newcommand{\titel}{Arthur Schnitzler und Paul Goldmann an Richard Beer-Hofmann, 26. 8. 1900}\newcommand{\editorInnen}{Martin Anton Müller und Gerd-Hermann Susen}%% latex-leseansicht-abspann.tex
%% Abspann für die Leseansicht.
%% Der Schalter \ifkorrekturansicht ist bereits durch den Vorspann gesetzt.

%% latex-abspann.tex
%% Gemeinsamer Abspann für Korrekturansicht und Leseansicht.
%% Setzt den Schalter \ifkorrekturansicht voraus (gesetzt in den
%% einbindenden Dateien latex-korrekturansicht-abspann.tex bzw.
%% latex-leseansicht-abspann.tex).
%% ---------------------------------------------------------------

\normalsize

% Das esempio-Environment wird nur in der Leseansicht benötigt
\ifkorrekturansicht\else
\newenvironment{esempio}[3]%
{
    \vspace{1.5ex}
    \rlap{\underline{#1}}
    \par
    \setlength{\parindent}{0cm}
    \nopagebreak
    \leftskip=#2cm
    \rightskip=#3cm
}
{
    \par
}
\fi

\doendnotes{C}
\bigskip
\vfill

\clearpage

\footnotesize

\ifkorrekturansicht
  \lohead{\textsc{register}}
\fi

% theindex-Environment neu definieren ohne reledmac
\makeatletter
\renewenvironment{theindex}{%
  \ifkorrekturansicht
    \section*{\indexname}%
  \else
    \subsubsection*{Index der erwähnten Entitäten}%
  \fi
  \setlength{\parindent}{0pt}%
  \setlength{\parskip}{0pt plus 0.3pt}%
  \let\item\@idxitem
}{%
  \ifkorrekturansicht\clearpage\fi
}
\makeatother

\IfFileExists{\jobname-pw.ind}{\input{\jobname-pw.ind}}{}

% Quellenangabe nur in der Leseansicht
\ifkorrekturansicht\else
% Fallback-Definitionen, falls die .tex-Datei \titel etc. nicht gesetzt hat
\providecommand{\titel}{}
\providecommand{\editorInnen}{}
\providecommand{\dateiname}{\jobname}

\vspace{3cm}

\vfill

\footnotesize
\textsc{Quelle}: \titel. Herausgegeben von {\editorInnen}. In: \emph{Arthur Schnitzler: Briefwechsel mit Autorinnen und Autoren}.
 Digitale Edition, https://schnitzler-briefe.acdh.oeaw.ac.at/{\dateiname}.html (Stand \today)
\fi

\end{document}


