%% latex-leseansicht-vorspann.tex
%% Vorspann für die Leseansicht.
%% Lädt die gemeinsame Datei latex-vorspann.tex mit nicht gesetztem Schalter.

\newif\ifkorrekturansicht
\korrekturansichtfalse

\input{../tex-inputs/latex-vorspann}


         
         \renewcommand{\erwaehntePersonen}{Personen: Richard Beer-Hofmann, Olga Schnitzler}
         \renewcommand{\erwaehnteInstitutionen}{Institutionen: Synagoge Währing}
         \renewcommand{\erwaehnteOrte}{Orte: Liesingerstraße, Rodaun, Schopenhauerstraße, Wien}
         \renewcommand{\erwaehnteWerke}{}
               \section[Arthur Schnitzler an Richard Beer-Hofmann, 23. 8. 1903]{ Arthur Schnitzler an Richard Beer-Hofmann, 23. 8. 1903}\nopagebreak\mylabel{v}\rehead{ }\begin{ledgroupsized}[t]{13cm}\normalsize\beginnumbering\briefempfaengerindex{Beer-Hofmann, Richard@\textsc{Beer-Hofmann, Richard}!zzzSchnitzler, Arthur@\emph{von Arthur Schnitzler}!1903-08-231@{23. 8. 1903}|(be} \toendnotes[C]{\smallbreak\pagebreak[2]} \Standort{YCGL, MSS 31.}
\physDesc{Telegramm, 198 Zeichen
\newline{}Handschrift einer Schreibkraft: Bleistift, deutsche Kurrent
\newline{}Versand: »\noindent{}\textcolor{gray}{\textbf{Dienstliche Angaben.}} RP{ / }\textcolor{gray}{\textbf{Aufgegeben am}}{ }2\textcolor{gray}{0}/8 \textcolor{gray}{\textbf{190}}2{ }\textcolor{gray}{\textbf{um {\dots} Uhr {\dots} Min {\dots} Mittag}}{ / }\textcolor{gray}{\textbf{Eingelangt von}} 1584 Wien\oindex{Wien@\textbf{Wien}|pw}{ }\textcolor{gray}{\textbf{auf Leitung Nr. {\dots} am}}{ }22/8\textcolor{gray}{\textbf{190.}}{ }\textcolor{gray}{\textbf{um}}{ }9 \textcolor{gray}{\textbf{Uhr}} 9 \textcolor{gray}{\textbf{Min. {\dots} Mittag}}{ / }\textcolor{gray}{\textbf{Aufgenommen durch}}{ }Ba{ / }\textcolor{gray}{\textbf{Von}}{ }Wien\oindex{Wien@\textbf{Wien}|pw}{ }\textcolor{gray}{\textbf{Aufgabe-Nr.}} 9999{ }\textcolor{gray}{\textbf{mit}} 31 \textcolor{gray}{\textbf{Taxworten ({\dots} Worten {\dots} Chiffern)}}« }\buchAbdrucke{\weitereDrucke{Arthur Schnitzler, Richard Beer-Hofmann: \emph{Briefwechsel 1891–1931}. Hg. Konstanze Fliedl. Wien, Zürich: \emph{Europaverlag} 1992, S. 163.} }\toendnotes[C]{\smallbreak}\pstart{}{\pb}Richard Beer Hofma{\geminationn}\pend{}\pstart{}Lieſinger Straſſe 2\oindex{Liesingerstrasse@\textbf{Liesingerstraße}|pw}\pend{}\pstart{}\textcolor{gray}{\textbf{\textit{Rodaun\oindex{Rodaun@\textbf{Rodaun}|pw}}}}\pend{}{\bigskip}\pstart
           \noindent{}{\pb}Ich bitte Sie allſo Mittwoch{ }halb eins{ }\label{K_L01312-1v}\edtext{Schopenhauergaſſe 37\oindex{Schopenhauerstrasse@\textbf{Schopenhauerstraße}|pw}}{\lemma{\textnormal{\emph{Schopenhauergaſſe 37}}}\Cendnote{\textnormal{Schopenhauerstraße 39\oindex{Schopenhauerstrasse@\textbf{Schopenhauerstraße}|pwk} war die Adresse der
                     Synagoge\orgindex{Synagoge Waehring@Synagoge Währing|pwkv}, in der am 26. 8. 1903 die
                  Trauung von Schnitzler\pwindex{Schnitzler, Arthur 15.05.1862 – 21.10.1931@\textsc{Schnitzler, Arthur} (15.05.1862 – 21.10.1931), \emph{Schriftsteller, Mediziner}|pwk} und Olga Gussmann\pwindex{Schnitzler, Olga 17.01.1882 – 13.01.1970@\textsc{Schnitzler, Olga} (17.01.1882 – 13.01.1970), \emph{Schauspielerin, Sängerin}|pwk} stattfand.}}}\label{K_L01312-1h} als Beistand zu fungiren
               und die Sache durchaus als vertraulich zu behandeln\pend
           \pstart Herzlichſt Ihr \spacefill\mbox{Arthur}\pend{}
         
         \endnumbering\mylabel{h}\end{ledgroupsized}  \newcommand{\dateiname}{L01312}\newcommand{\titel}{Arthur Schnitzler an Richard Beer-Hofmann, 23. 8. 1903}\newcommand{\editorInnen}{Martin Anton Müller und Gerd-Hermann Susen}%% latex-leseansicht-abspann.tex
%% Abspann für die Leseansicht.
%% Der Schalter \ifkorrekturansicht ist bereits durch den Vorspann gesetzt.

%% latex-abspann.tex
%% Gemeinsamer Abspann für Korrekturansicht und Leseansicht.
%% Setzt den Schalter \ifkorrekturansicht voraus (gesetzt in den
%% einbindenden Dateien latex-korrekturansicht-abspann.tex bzw.
%% latex-leseansicht-abspann.tex).
%% ---------------------------------------------------------------

\normalsize

% Das esempio-Environment wird nur in der Leseansicht benötigt
\ifkorrekturansicht\else
\newenvironment{esempio}[3]%
{
    \vspace{1.5ex}
    \rlap{\underline{#1}}
    \par
    \setlength{\parindent}{0cm}
    \nopagebreak
    \leftskip=#2cm
    \rightskip=#3cm
}
{
    \par
}
\fi

\doendnotes{C}
\bigskip
\vfill

\clearpage

\footnotesize

\ifkorrekturansicht
  \lohead{\textsc{register}}
\fi

% theindex-Environment neu definieren ohne reledmac
\makeatletter
\renewenvironment{theindex}{%
  \ifkorrekturansicht
    \section*{\indexname}%
  \else
    \subsubsection*{Index der erwähnten Entitäten}%
  \fi
  \setlength{\parindent}{0pt}%
  \setlength{\parskip}{0pt plus 0.3pt}%
  \let\item\@idxitem
}{%
  \ifkorrekturansicht\clearpage\fi
}
\makeatother

\IfFileExists{\jobname-pw.ind}{\input{\jobname-pw.ind}}{}

% Quellenangabe nur in der Leseansicht
\ifkorrekturansicht\else
% Fallback-Definitionen, falls die .tex-Datei \titel etc. nicht gesetzt hat
\providecommand{\titel}{}
\providecommand{\editorInnen}{}
\providecommand{\dateiname}{\jobname}

\vspace{3cm}

\vfill

\footnotesize
\textsc{Quelle}: \titel. Herausgegeben von {\editorInnen}. In: \emph{Arthur Schnitzler: Briefwechsel mit Autorinnen und Autoren}.
 Digitale Edition, https://schnitzler-briefe.acdh.oeaw.ac.at/{\dateiname}.html (Stand \today)
\fi

\end{document}


      