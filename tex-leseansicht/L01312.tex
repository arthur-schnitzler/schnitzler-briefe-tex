%% latex-korrekturansicht-vorspann.tex
%% Vorspann für die Korrekturansicht.
%% Lädt die gemeinsame Datei latex-vorspann.tex mit gesetztem Schalter.

\newif\ifkorrekturansicht
\korrekturansichttrue

\input{../tex-inputs/latex-vorspann}


\section[Arthur Schnitzler an Richard Beer-Hofmann, 23. 8. 1903]{L01312 Arthur Schnitzler an Richard Beer-Hofmann, 23. 8. 1903}
\nopagebreak\mylabel{L01312v}
\rehead{ }\normalsize\beginnumbering\briefempfaengerindex{Beer-Hofmann, Richard@\textsc{Beer-Hofmann, Richard}!zzzSchnitzler, Arthur@\emph{von Arthur Schnitzler}!1903-08-231@{23. 8. 1903}|(be}
\toendnotes[C]{\smallbreak\pagebreak[2]}\Standort{YCGL, MSS 31.}
\physDesc{Telegramm, 198 Zeichen
\newline{}Handschrift einer Schreibkraft: Bleistift, deutsche Kurrent
\newline{}Versand: »\noindent{}\textcolor{gray}{\textbf{Dienstliche Angaben.}} RP{ / }\textcolor{gray}{\textbf{Aufgegeben am}}{ }2\textcolor{gray}{0}/8 \textcolor{gray}{\textbf{190}}2{ }\textcolor{gray}{\textbf{um {\dots} Uhr {\dots} Min {\dots} Mittag}}{ / }\textcolor{gray}{\textbf{Eingelangt von}} 1584 Wien\oindex{Wien@\textbf{Wien}, \emph{A.ADM2}|pw}{ }\textcolor{gray}{\textbf{auf Leitung Nr. {\dots} am}}{ }22/8\textcolor{gray}{\textbf{190.}}{ }\textcolor{gray}{\textbf{um}}{ }9 \textcolor{gray}{\textbf{Uhr}} 9 \textcolor{gray}{\textbf{Min. {\dots} Mittag}}{ / }\textcolor{gray}{\textbf{Aufgenommen durch}}{ }Ba{ / }\textcolor{gray}{\textbf{Von}}{ }Wien\oindex{Wien@\textbf{Wien}, \emph{A.ADM2}|pw}{ }\textcolor{gray}{\textbf{Aufgabe-Nr.}} 9999{ }\textcolor{gray}{\textbf{mit}} 31 \textcolor{gray}{\textbf{Taxworten ({\dots} Worten {\dots} Chiffern)}}« }
\buchAbdrucke{\weitereDrucke{Arthur Schnitzler, Richard Beer-Hofmann: \emph{Briefwechsel 1891–1931}. Wien, Zürich: \emph{Europaverlag} 1992, S. 163.} }\toendnotes[C]{\smallbreak}\pstart{}{\pb}Richard Beer Hofma{\geminationn}\pend{}\pstart{}Lieſinger Straſſe 2\oindex{Liesingerstrasse@\textbf{Liesingerstraße}, \emph{Straße (K.STR)}|pw}\pend{}\pstart{}\textcolor{gray}{\textbf{\textit{Rodaun\oindex{Rodaun@\textbf{Rodaun}, \emph{A.ADM4}|pw}}}}\pend{}{\bigskip}\vspace{1em}
\pstart
           \noindent{}{\pb}Ich bitte Sie allſo Mittwoch{ }halb eins{ }\label{K_L01312-1v}\edtext{Schopenhauergaſſe 37\oindex{Schopenhauerstrasse@\textbf{Schopenhauerstraße}, \emph{Straße (K.STR)}|pw}}{\lemma{\textnormal{\emph{Schopenhauergaſſe 37}}}\Cendnote{\textnormal{Schopenhauerstraße 39\oindex{Schopenhauerstrasse@\textbf{Schopenhauerstraße}, \emph{Straße (K.STR)}|pwk} war die Adresse der
                     Synagoge\orgindex{Synagoge Waehring@Synagoge Währing|pwkv}, in der am 26. 8. 1903 die
                  Trauung von Schnitzler und Olga Gussmann\pwindex{Schnitzler, Olga 17.01.1882 – 13.01.1970@\textsc{Schnitzler, Olga} (17.01.1882 – 13.01.1970), \emph{Schauspieler/Schauspielerin, Sänger/Sängerin}|pwk} stattfand.}}}\label{K_L01312-1} als Beistand zu fungiren
               und die Sache durchaus als vertraulich zu behandeln\pend
           \pstart Herzlichſt Ihr \spacefill\mbox{Arthur}\pend{}\selectlanguage{ngerman}\endnumbering\briefempfaengerindex{Beer-Hofmann, Richard@\textsc{Beer-Hofmann, Richard}!zzzSchnitzler, Arthur@\emph{von Arthur Schnitzler}!1903-08-231@{23. 8. 1903}|)be}\mylabel{L01312h}  \normalsize

\doendnotes{C}
\bigskip
\vfill

\clearpage

\footnotesize

\lohead{\textsc{register}}

% Definiere theindex-Environment komplett neu ohne reledmac
\makeatletter
\renewenvironment{theindex}{%
  \section*{\indexname}%
  \setlength{\parindent}{0pt}%
  \setlength{\parskip}{0pt plus 0.3pt}%
  \let\item\@idxitem
}{%
  \clearpage
}
\makeatother

\IfFileExists{\jobname-pw.ind}{\input{\jobname-pw.ind}}{}

\end{document}

      