%% latex-leseansicht-vorspann.tex
%% Vorspann für die Leseansicht.
%% Lädt die gemeinsame Datei latex-vorspann.tex mit nicht gesetztem Schalter.

\newif\ifkorrekturansicht
\korrekturansichtfalse

\input{../tex-inputs/latex-vorspann}


\section[Theodor Herzl an Arthur Schnitzler, 12. 9. 1893]{L03895 Theodor Herzl an Arthur Schnitzler, 12. 9. 1893}
\nopagebreak\mylabel{L03895v}
\rehead{ }\normalsize\beginnumbering\briefempfaengerindex{Schnitzler, Arthur@\textsc{Schnitzler, Arthur}!zzzHerzl, Theodor@\emph{von Theodor Herzl}!1893-09-122@{12. 9. 1893}|(be}
\toendnotes[C]{\smallbreak\pagebreak[2]}
\correspDesc{Versand  durch Theodor Herzl am 12. 9. 1893 in Baden bei Wien
\newline{}Erhalt  durch Arthur Schnitzler im Zeitraum [13. 9. 1893
                  – 17. 9. 1893?] in Wien}\toendnotes[C]{\smallbreak}
\Standort{Wien, Österreichische Gesellschaft für Literatur, Abschrift Herzl.}
\physDesc{Brief, maschinenschriftliche Abschrift, 1 Blatt, 1 Seite, 397 Zeichen
\newline{}maschinell
\newline{}Zusatz: In der Nachlassmappe B 39 hat Heinrich Schnitzler\pwindex{Schnitzler, Heinrich 9.\,8.\,1902 Hinterbrühl – 12.\,7.\,1982 Wien@\textsc{Schnitzler, Heinrich} (9.\,8.\,1902 Hinterbrühl – 12.\,7.\,1982 Wien), \emph{Regisseur, Schauspieler}|pw} vermerkt: »\noindent{}2 Briefe
                                       geschenkt ans Wolf-Museum Eisenstadt\orgindex{Landesmuseum Burgenland@Landesmuseum Burgenland|pw}{ }22. VIII. 1937.{ / }1 Brief entnommen{ / }1 Brief geschenkt an Paul Marx\pwindex{Marx, Paul 21.\,7.\,1879 Wien – 30.\,10.\,1956 ebd.@\textsc{Marx, Paul} (21.\,7.\,1879 Wien – 30.\,10.\,1956 ebd.), \emph{Regisseur, Schauspieler}|pw}{ }15. VIII. 1936.{ / }1 Brief gegeben an Mutter\pwindex{Schnitzler, Olga 17.\,1.\,1882 Wien – 13.\,1.\,1970 Lugano@\textsc{Schnitzler, Olga} (17.\,1.\,1882 Wien – 13.\,1.\,1970 Lugano), \emph{Schauspielerin, Sängerin}|pwv}, 15. VIII. 36.« Das entspricht
                                 der Anzahl von fünf Korrespondenzstücken von Herzl, die nicht im Original überliefert sind. Alle finden sich in einer Abschrift, die nach
                                 Arthur Schnitzlers Tod im Zeitraum 1932 bis 1936 entstanden sein dürfte. }
\buchAbdrucke{\weitereDrucke{Theodor Herzl: \emph{Briefe und autobiographische Notizen 1866–1895}. Bearbeitet von Johannes Wachten in Zusammenarbeit mit Chaya Harel, Daisy Tycho und Manfred Winkler. Berlin, Frankfurt am Main, Wien: \emph{Propyläen} 1983, S. 538–539 (Briefe und Tagebücher. Herausgegeben von Alex Bein, Hermann Greive, Moshe Schaerf, Julius H. Schoeps und Johannes Wachten, 1).} }\toendnotes[C]{\smallbreak}
\pstart
           {\pb}H 13\pend
           
\pstart
           \raggedleft{}12. 9. 1893.\pend
           
\pstart{}Lieber Freund!\pend\vspace{0.5em}
\pstart
           Bleibe \label{K_L03895-1v}\edtext{hier\oindex{Baden bei Wien@\textbf{Baden bei Wien}, \emph{Hauptstadt}|pwv}}{\lemma{\textnormal{\emph{hier}}}\Cendnote{\textnormal{Baden bei Wien\oindex{Baden bei Wien@\textbf{Baden bei Wien}, \emph{Hauptstadt}|pwk}, vgl. A. S.: \emph{Tagebuch}, 22. 9. 1893 und 24. 9. 1893.}}}\label{K_L03895-1} ungefähr drei Wochen, werde
               mich sehr freuen Sie hier zu sehen und länger mit Ihnen zu dischkurieren. Ich bin
               meistens hier, \label{K_L03895-55v}\edtext{selten in Wien\oindex{Wien@\textbf{Wien}, \emph{Verwaltungsgebiet}|pw}}{\lemma{\textnormal{\emph{selten in Wien}}}\Cendnote{\textnormal{Zufällig traf man sich
                  am 18. 9. 1895 in einem Wien\oindex{Wien@\textbf{Wien}, \emph{Verwaltungsgebiet}|pwk}er Kaffeehaus.}}}\label{K_L03895-55}. Vorsichtsweise
               zeigen Sie doch Ihren Besuch telegraphisch an. Nächsten \label{K_L03895-2v}\edtext{Samstag}{\lemma{\textnormal{\emph{Samstag}}}\Cendnote{\textnormal{Am 16. 9. 1893
               reiste Schnitzler Abends nach Salzburg\oindex{Salzburg@\textbf{Salzburg}, \emph{Verwaltungsgebiet}|pwk}. Ein Treffen mit
                  Herzl\pwindex{Herzl, Theodor 2.\,5.\,1860 Budapest – 3.\,7.\,1904 Edlach@\textsc{Herzl, Theodor} (2.\,5.\,1860 Budapest – 3.\,7.\,1904 Edlach), \emph{Schriftsteller, Journalist}|pwk} ist für diesen Tag nicht belegt.}}}\label{K_L03895-2} fahre ich nach Wien\oindex{Wien@\textbf{Wien}, \emph{Verwaltungsgebiet}|pw}; wenn ich kann, springe ich einen Augenblick
               zu Ihnen. Nicht sicher.\pend
           
\pstart
           Aber sicher meine herzliche Ergebenheit.{\\[\baselineskip]}Ihr{\\[\baselineskip]}\spacefill\mbox{Th. Herzl.}\pend
           \leftskip=0em{}\selectlanguage{ngerman}\endnumbering\briefempfaengerindex{Schnitzler, Arthur@\textsc{Schnitzler, Arthur}!zzzHerzl, Theodor@\emph{von Theodor Herzl}!1893-09-122@{12. 9. 1893}|)be}\mylabel{L03895h}
\begin{anhang}
\end{anhang}\newcommand{\dateiname}{L03895}\newcommand{\titel}{Theodor Herzl an Arthur Schnitzler, 12. 9. 1893}\newcommand{\editorInnen}{Selma Jahnke und Martin Anton Müller}%% latex-leseansicht-abspann.tex
%% Abspann für die Leseansicht.
%% Der Schalter \ifkorrekturansicht ist bereits durch den Vorspann gesetzt.

%% latex-abspann.tex
%% Gemeinsamer Abspann für Korrekturansicht und Leseansicht.
%% Setzt den Schalter \ifkorrekturansicht voraus (gesetzt in den
%% einbindenden Dateien latex-korrekturansicht-abspann.tex bzw.
%% latex-leseansicht-abspann.tex).
%% ---------------------------------------------------------------

\normalsize

% Das esempio-Environment wird nur in der Leseansicht benötigt
\ifkorrekturansicht\else
\newenvironment{esempio}[3]%
{
    \vspace{1.5ex}
    \rlap{\underline{#1}}
    \par
    \setlength{\parindent}{0cm}
    \nopagebreak
    \leftskip=#2cm
    \rightskip=#3cm
}
{
    \par
}
\fi

\doendnotes{C}
\bigskip
\vfill

\clearpage

\footnotesize

\ifkorrekturansicht
  \lohead{\textsc{register}}
\fi

% theindex-Environment neu definieren ohne reledmac
\makeatletter
\renewenvironment{theindex}{%
  \ifkorrekturansicht
    \section*{\indexname}%
  \else
    \subsubsection*{Index der erwähnten Entitäten}%
  \fi
  \setlength{\parindent}{0pt}%
  \setlength{\parskip}{0pt plus 0.3pt}%
  \let\item\@idxitem
}{%
  \ifkorrekturansicht\clearpage\fi
}
\makeatother

\IfFileExists{\jobname-pw.ind}{\input{\jobname-pw.ind}}{}

% Quellenangabe nur in der Leseansicht
\ifkorrekturansicht\else
% Fallback-Definitionen, falls die .tex-Datei \titel etc. nicht gesetzt hat
\providecommand{\titel}{}
\providecommand{\editorInnen}{}
\providecommand{\dateiname}{\jobname}

\vspace{3cm}

\vfill

\footnotesize
\textsc{Quelle}: \titel. Herausgegeben von {\editorInnen}. In: \emph{Arthur Schnitzler: Briefwechsel mit Autorinnen und Autoren}.
 Digitale Edition, https://schnitzler-briefe.acdh.oeaw.ac.at/{\dateiname}.html (Stand \today)
\fi

\end{document}


