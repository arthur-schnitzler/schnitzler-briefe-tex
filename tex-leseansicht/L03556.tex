%% latex-korrekturansicht-vorspann.tex
%% Vorspann für die Korrekturansicht.
%% Lädt die gemeinsame Datei latex-vorspann.tex mit gesetztem Schalter.

\newif\ifkorrekturansicht
\korrekturansichttrue

\input{../tex-inputs/latex-vorspann}


\section[ Felix Salten an Arthur Schnitzler, {[}18. 2. 1912{]}]{L03556 Felix Salten an Arthur Schnitzler, {[}18. 2. 1912{]}}
\nopagebreak\mylabel{L03556v}
\rehead{ }\normalsize\beginnumbering\briefempfaengerindex{Schnitzler, Arthur@\textsc{Schnitzler, Arthur}!zzzSalten, Felix@\emph{von Felix Salten}!1912-02-181@{{[}18. 2. 1912{]}}|(be}
\toendnotes[C]{\smallbreak\pagebreak[2]}\Standort{CUL, Schnitzler, B 89, B 2.}
\physDesc{Briefkarte, 289 Zeichen
\newline{}Handschrift: Bleistift, lateinische Kurrent
\newline{}Schnitzler: mit Bleistift datiert: »18/2 912« }\toendnotes[C]{\smallbreak}
\pstart
           \centering{}{\pb}\textcolor{gray}{\textbf{\textcolor{gray}{\textbf{\textsc{Felix Salten}}}}}\pend
           \vspace{0.5em}
\pstart
           Lieber, ich hätte gerne eine halbe Stunde mit Ihnen gesprochen, wenn
               ich Sie heute nicht allzusehr störe. (Allerlei \label{K_L03556-1v}\edtext{Dramaturgisches}{\lemma{\textnormal{\emph{Dramaturgisches}}}\Cendnote{\textnormal{Siehe A. S.: \emph{Tagebuch}, 18. 2. 1912.
               }}}\label{K_L03556-1}, das mich sehr beschäfigt) Wollen Sie mir, bitte, sagen laßen, ob ich kommen
               kann?\pend
           
\pstart
           Herzlichst{\\[\baselineskip]} Ihr {\\[\baselineskip]}\spacefill\mbox{Salten}\pend
           \leftskip=0em{}
\pstart
           \noindent{}Dank für den \label{K_L03556-2v}\edtext{Gratulations\textcolor{gray}{-Strauß}}{\lemma{\textnormal{\emph{Gratulations-Strauß}}}\Cendnote{\textnormal{Salten\pwindex{Salten, Felix 06.09.1869 – 08.10.1945@\textsc{Salten, Felix} (06.09.1869 – 08.10.1945), \emph{Schriftsteller/Schriftstellerin, Journalist/Journalistin, Chefredakteur/Chefredakteurin}|pwk} hatte gemeinsam mit vier anderen – darunter Jakob Wassermann\pwindex{Wassermann, Jakob 10.03.1873 – 01.01.1934@\textsc{Wassermann, Jakob} (10.03.1873 – 01.01.1934), \emph{Schriftsteller/Schriftstellerin}|pwk} – den \emph{Bauernfeld-Preis}\orgindex{Bauernfeld-Preis@Bauernfeld-Preis|pwk} zuerkannt bekommen.}}}\label{K_L03556-2}.
                  Aber \textcolor{gray}{Jackl Wasser} – – – –\pwindex{Wassermann, Jakob 10.03.1873 – 01.01.1934@\textsc{Wassermann, Jakob} (10.03.1873 – 01.01.1934), \emph{Schriftsteller/Schriftstellerin}|pwu}{ }\textcolor{gray}{na}!\pend
           \selectlanguage{ngerman}\endnumbering\briefempfaengerindex{Schnitzler, Arthur@\textsc{Schnitzler, Arthur}!zzzSalten, Felix@\emph{von Felix Salten}!1912-02-181@{{[}18. 2. 1912{]}}|)be}\mylabel{L03556h}  \normalsize

\doendnotes{C}
\bigskip
\vfill

\clearpage

\footnotesize

\lohead{\textsc{register}}

% Definiere theindex-Environment komplett neu ohne reledmac
\makeatletter
\renewenvironment{theindex}{%
  \section*{\indexname}%
  \setlength{\parindent}{0pt}%
  \setlength{\parskip}{0pt plus 0.3pt}%
  \let\item\@idxitem
}{%
  \clearpage
}
\makeatother

\IfFileExists{\jobname-pw.ind}{\input{\jobname-pw.ind}}{}

\end{document}

      