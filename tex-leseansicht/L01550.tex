\input{../tex-inputs/latex-pdf-vorspann}
\begin{center}
            \textcolor{red}{ENTWURF. ENTZIFFERUNG NOCH NICHT KORREKTURGELESEN}
                      \end{center}
            
               \section[Hermann Bahr an Arthur Schnitzler, 20. 9. 1905]{ Hermann Bahr an Arthur Schnitzler, 20. 9. 1905}\nopagebreak\mylabel{v}\rehead{ }\begin{ledgroupsized}[t]{13cm}\normalsize\beginnumbering\briefempfaengerindex{Schnitzler, Arthur@\textsc{Schnitzler, Arthur}!zzzBahr, Hermann@\emph{von Hermann Bahr}!1905-09-201@{20. 9. 1905}|(be} \toendnotes[C]{\smallbreak\pagebreak[2]} \Standort{CUL, Schnitzler, B 5b.}
\physDesc{Kartenbrief
\newline{}Handschrift: schwarze Tinte, deutsche Kurrent\newline{}Versand: 1) Stempel: »\nobreak{}\oindex{XIII., Hietzing@\textbf{XIII., Hietzing}|pwk}Wien 13/5, 20. IX. 05\nobreak{}«.  2) Stempel: »\nobreak{}Bestellt, \oindex{XVIII., Waehring@\textbf{XVIII., Währing}|pwk}18/1 Wien, 20 IX 05\nobreak{}«. 
\newline{}Schnitzler: mit Bleistift die Jahreszahl ergänzt: »905« \newline{}Ordnung: mit Bleistift von unbekannter Hand nummeriert: »133« }\buchAbdrucke{\weitereDrucke{Hermann Bahr, Arthur Schnitzler: \emph{Briefwechsel, Aufzeichnungen, Dokumente (1891–1931)}. Hg. Kurt Ifkovits und Martin Anton Müller. Göttingen: \emph{Wallstein} 2018, S. 354.} }\toendnotes[C]{\smallbreak}\pstart{}{\pb}Herrn \textsc{Dr Arthur
                     Schnitzler}\pend{}\pstart{}\textsc{Wien XIX}\oindex{XIX., Doebling@\textbf{XIX., Döbling}|pw}\pend{}\pstart{}Spöttelgaſſe 7\oindex{Edmund-Weiss-Gasse@\textbf{Edmund-Weiß-Gasse}|pw}\pend{}{\bigskip}\pstart
           \raggedleft{}{\pb}20. 9.\pend
           \pstart{}Lieber Arthur!\pend\pstart
           Ich hab nun auch das Zwischenſpiel\pwindex{Schnitzler, Arthur 15.05.1862 – 21.10.1931@\textsc{Schnitzler, Arthur} (15.05.1862 – 21.10.1931), \emph{Schriftsteller, Mediziner}!Zwischenspiel. Komoedie in drei Akten1905-10-12 – 1905-10-12@\strich\emph{Zwischenspiel. Komödie in drei Akten} {[}1905-10-12 – 1905-10-12{]}|pw} geleſen, mit
               einem ſehr großen artiſtiſchen Vergnügen. Es iſt eine reizende Comödie und ich finde
               es wunderbar, wie Du in die Form des alten Burgtheater\oindex{Burgtheater@\textbf{Burgtheater}|pw}ſtücks die feinſte \textsc{Psychologi}e und
               unſere neueſten Probleme gebracht haſt. Mich ſtört nur manchmal der (gewiß
               beabſichtigte) Cafehauston zwiſchen den beiden Freunden, eine Art von \textsc{philosoph}isch wieneriſch\oindex{Wien@\textbf{Wien}|pw}
               jüdiſcher Schnoddrigkeit, die in früheren Jahren mir vielleicht noch geläufiger als
               Dir war, aber ſeien wir froh, daß es vorbei iſt! Mehr noch ſtört mich Dein Fürſt\pwindex{Schnitzler, Arthur 15.05.1862 – 21.10.1931@\textsc{Schnitzler, Arthur} (15.05.1862 – 21.10.1931), \emph{Schriftsteller, Mediziner}!Zwischenspiel. Komoedie in drei Akten1905-10-12 – 1905-10-12@\strich\emph{Zwischenspiel. Komödie in drei Akten} {[}1905-10-12 – 1905-10-12{]}|pwv}. Warum mußt Du einen ſich in
               einer heiklen Situation ſehr nett benehmenden Menſchen in eine Kaſte verſetzen, in
               welcher Roheit die Regel, ſittlicher Takt unbekannt ist? Und wie unangenehm wird
               einem die Frau, die ſich von ſo einem hofieren läßt! Aber dies alles mündlich. Könnte
               ich nicht nächſte Woche einmal Vormittag auf ein paar Stunden zu Dir kommen? An
               Abenden macht ſichs zu ſchwer. Grüß Deine Frau\pwindex{Schnitzler, Olga 17.01.1882 – 13.01.1970@\textsc{Schnitzler, Olga} (17.01.1882 – 13.01.1970), \emph{Schauspielerin, Sängerin}|pwv} herzlichſt! \spacefill\mbox{H.}\pend
           \endnumbering\briefempfaengerindex{Schnitzler, Arthur@\textsc{Schnitzler, Arthur}!zzzBahr, Hermann@\emph{von Hermann Bahr}!1905-09-201@{20. 9. 1905}|)be}\mylabel{h}\end{ledgroupsized}  \newcommand{\dateiname}{L01550}\newcommand{\titel}{Hermann Bahr an Arthur Schnitzler, 20. 9. 1905}\newcommand{\editorInnen}{ Kurt Ifkovits,  Martin Anton Müller}\input{../tex-inputs/latex-pdf-abspann}
      