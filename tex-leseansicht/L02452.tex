%% latex-korrekturansicht-vorspann.tex
%% Vorspann für die Korrekturansicht.
%% Lädt die gemeinsame Datei latex-vorspann.tex mit gesetztem Schalter.

\newif\ifkorrekturansicht
\korrekturansichttrue

\input{../tex-inputs/latex-vorspann}


\section[Hugo Hofmannsthal an Arthur Schnitzler, 8. 10. 1925]{L02452 Hugo Hofmannsthal an Arthur Schnitzler, 8. 10. 1925}
\nopagebreak\mylabel{L02452v}
\rehead{ }\normalsize\beginnumbering\briefempfaengerindex{Schnitzler, Arthur@\textsc{Schnitzler, Arthur}!zzzHofmannsthal, Hugo von@\emph{von Hugo von Hofmannsthal}!1925-10-081@{8. 10. 1925}|(be}
\toendnotes[C]{\smallbreak\pagebreak[2]}\Standort{CUL, Schnitzler, B 43.}
\physDesc{Postkarte, 404 Zeichen
\newline{}Handschrift: schwarze Tinte, lateinische Kurrent
\newline{}Versand: Stempel: »\nobreak{}\oindex{Bad Aussee@\textbf{Bad Aussee}, \emph{P.PPLA3}|pwk}Bad Aussee, 4\nobreak{}«.  
\newline{}Ordnung: 1) mit Bleistift von unbekannter Hand nummeriert: »\strikeout{289}\strikeout{366}\strikeout{393}«  2) mit Bleistift von unbekannter Hand nummeriert:
                                    »390«}
\buchAbdrucke{\weitereDrucke{Hugo von Hofmannsthal, Arthur Schnitzler: \emph{Briefwechsel}. Frankfurt am Main: \emph{S. Fischer} 1964, S. 301.} }\toendnotes[C]{\smallbreak}\pstart{}{\pb}Herrn D\textsuperscript{r} Arthur Schnitzler\pend{}\pstart{}Wien\oindex{Wien@\textbf{Wien}, \emph{A.ADM2}|pw}\pend{}\pstart{}XVIII Sternwartestrasse 71\oindex{Sternwartestrasse 71@\textbf{Sternwartestraße 71}, \emph{Wohngebäude (K.WHS)}|pw}.\pend{}{\bigskip}\vspace{1em}
\pstart
           \raggedleft{}{\pb}Bad Aussee\oindex{Bad Aussee@\textbf{Bad Aussee}, \emph{P.PPLA3}|pw}\pend
           
\pstart
           \raggedleft{}8 X. 25\pend
           
\pstart{}Lieber Arthur \pend\vspace{0.5em}
\pstart
           eine \label{K_L02452-1v}\edtext{Karte aus Forte dei marmi\oindex{Forte dei Marmi@\textbf{Forte dei Marmi}, \emph{P.PPLA3}|pw}}{\lemma{\textnormal{\emph{Karte … marmi}}}\Cendnote{\textnormal{nicht erhalten}}}\label{K_L02452-1} war mir sehr lieb
               als Zeichen Ihres Wohlseins {\pb}u.
               Gedenkens. – Ich bin jetzt hier u. arbeite. (Seit Anfang September.)\pend
           
\pstart
           Geben Sie mir bitte nur durch Ihre Unterschrift auf einer Karte ein Zeichen, dass Sie
               dort sind, wo diese Karte Sie aufsucht, so geht der »Turm\pwindex{Turm. Ein Trauerspiel@\emph{Der Turm. Ein Trauerspiel}|pw}« ebendahin ab.\pend
           
\pstart
           Alles Liebe! {\\[\baselineskip]}\spacefill\mbox{Hugo.}\pend
           \leftskip=0em{}\selectlanguage{ngerman}\endnumbering\briefempfaengerindex{Schnitzler, Arthur@\textsc{Schnitzler, Arthur}!zzzHofmannsthal, Hugo von@\emph{von Hugo von Hofmannsthal}!1925-10-081@{8. 10. 1925}|)be}\mylabel{L02452h}  \normalsize

\doendnotes{C}
\bigskip
\vfill

\clearpage

\footnotesize

\lohead{\textsc{register}}

% Definiere theindex-Environment komplett neu ohne reledmac
\makeatletter
\renewenvironment{theindex}{%
  \section*{\indexname}%
  \setlength{\parindent}{0pt}%
  \setlength{\parskip}{0pt plus 0.3pt}%
  \let\item\@idxitem
}{%
  \clearpage
}
\makeatother

\IfFileExists{\jobname-pw.ind}{\input{\jobname-pw.ind}}{}

\end{document}

      