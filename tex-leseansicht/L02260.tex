%% latex-korrekturansicht-vorspann.tex
%% Vorspann für die Korrekturansicht.
%% Lädt die gemeinsame Datei latex-vorspann.tex mit gesetztem Schalter.

\newif\ifkorrekturansicht
\korrekturansichttrue

\input{../tex-inputs/latex-vorspann}


\section[Robert Adam an Arthur Schnitzler, 16. 5. 1917]{L02260 Robert Adam an Arthur Schnitzler, 16. 5. 1917}
\nopagebreak\mylabel{L02260v}
\rehead{ }\normalsize\beginnumbering\briefempfaengerindex{Schnitzler, Arthur@\textsc{Schnitzler, Arthur}!zzzAdam, Robert@\emph{von Robert Adam}!1917-05-161@{16. 5. 1917}|(be}
\toendnotes[C]{\smallbreak\pagebreak[2]}\Standort{DLA, A:Schnitzler, HS.NZ85.1.4230,18.}
\physDesc{Brief, 1 Blatt, 3 Seiten, 2039 Zeichen
\newline{}Handschrift: schwarze Tinte, deutsche Kurrent
\newline{}Schnitzler: 1) mit Bleistift beschriftet: »\textsc{Adam}«  2) mit rotem Buntstift eine Unterstreichung}\Standort{Wien, Österreichische Nationalbibliothek, Cod.ser. 52.263, 192.}
\physDesc{Brief, maschinenschriftliche Abschrift1 Blatt, 1 Seite, 2039 Zeichen
\newline{}Schreibmaschine}\toendnotes[C]{\smallbreak}
\pstart
           \raggedleft{}{\pb}Wien\oindex{Wien@\textbf{Wien}, \emph{A.ADM2}|pw}, am 16. Mai 1917\pend
           
\pstart\center{}Hochverehrter Herr Doktor!\pend\vspace{0.5em}
\pstart
           Ich empfinde nachgerade ein gewiſſes Schamgefühl, da jede Mitteilung, die ich Ihnen
               über meine literariſchen Geſchicke zu machen habe, die von einem Mißerfolg iſt. Alſo
               ſeit Jahren und nun alſo auch heute.\pend
           
\pstart
           Das Münchner Hoftheater\orgindex{Nationaltheater Muenchen@Nationaltheater München|pw} hat den »Neidhard\pwindex{Neidhard@\emph{Neidhard}|pw}« abgelehnt »wegen verſchiedener Mängel
               im dramatiſchen Aufbau« – \strikeout{gegen die} für die ich
               ſelbſt, bei Gott, nicht blind bin – »und wegen allzugroßer Längen« – deren
               Beteiligung im Wege von Strichen ich allerdings vorgeſchlagen hatte. Den {\pb}Dramaturgen\pwindex{Gutherz, Gerhard 07.09.1877 – 21.03.1942@\textsc{Gutherz, Gerhard} (07.09.1877 – 21.03.1942), \emph{Schriftsteller/Schriftstellerin, Dramaturg/Dramaturgin}|pwv} hat indeß »die an
               vielen Stellen aufleuchtende Poeſie und Lyrik (ein \label{K_L02260-1v}\edtext{\griechisch{ἓν διὰ δυοῖν}}{\lemma{\textnormal{\emph{ἓν διὰ δυοῖν}}}\Cendnote{\textnormal{altgriechisch: eins mit zwei; Ausdruck
                  der Rhetorik, bei dem ein neuer Begriff aus zwei Wörten gebildet wird, wie hier
                  »Poesie und Lyrik«}}}\label{K_L02260-1}) »ebenſo wie der witzige, fein pointierte Dialog in den
               Zwiſchenſpielen« »ſtark gefeſſelt«. Schade, daß die Zwiſchenſpiele nicht abendfüllend
               sind!\pend
           
\pstart
           Da ſteh ich nun, ich armer
               Tor\pwindex{Faust. Eine Tragoedie@\emph{Faust. Eine Tragödie}|pwv}, und bin entſchloſſen, das Ende des Krieges abzuwarten und damit das
               Herankommen einer Zeit, die der ſcheußlichen deutſchfeindlichen Geſinnung, deren
               meiner Anſicht nach der »Neidhard\pwindex{Neidhard@\emph{Neidhard}|pw}« voll iſt,
               verſtändnisvoller gegenüberſtehen dürfte als die Hindenburg\pwindex{Hindenburg, Paul von 02.10.1847 – 02.08.1934@\textsc{Hindenburg, Paul von} (02.10.1847 – 02.08.1934), \emph{Politiker/Politikerin}|pw}iſche. Oder ſoll ich das kühne Experiment wagen, den »Neidhard\pwindex{Neidhard@\emph{Neidhard}|pw}«, ſobald er wieder in meinen Händen iſt,
               neuerlich zuſammenzupacken und dem Burgtheater\oindex{Burgtheater@\textbf{Burgtheater}, \emph{S.THTR}|pw} mit
               der Verſicherung einzureichen, daß er dem chriſtlich-germaniſchen Schönheitsideal
               entſpricht? Da dieſes angefeindet {\pb}durch
               Nichtverwendung babylon\oindex{Babylon@\textbf{Babylon}, \emph{A.ADMD}|pw}iſcher Motive negativ
               determiniert iſt, iſt’s ſehr wohl möglich, daß der antichriſtlich-antigermaniſche
                  »Neidhard\pwindex{Neidhard@\emph{Neidhard}|pw}« ſeine volle Erfüllung bedeutet.
               Der Spaß wäre nicht ſo übel, und hätte ich nicht zu befürchten, daß in Folge des zu
               erwartenden Anſturms aller germaniſchen Chriſten und der dadurch bewirkten
               Ueberlaſtung des Lektors der arme »Neidhard\pwindex{Neidhard@\emph{Neidhard}|pw}«
                  \strikeout{nie} weit über die bevorſtehende Wiedergeburt Öſterreichs\oindex{Oesterreich@\textbf{Österreich}, \emph{A.PCLI}|pw} hinaus im Archive lagern bliebe, ich
               wagte wirklich gerne den Verſuch. –\pend
           
\pstart
           Nehmen Sie, hochverehrter Herr Doktor, neuerlich meinen Dank für Ihre liebenswürdige
               Bemühung entgegen (wie geſagt, ich ſchäme mich meines unumbringbaren Pechs) und
               empfangen Sie die ergebenſten Grüße von Ihrem\pend
           \pstart \spacefill\mbox{Robert Adam}\pend{}\selectlanguage{ngerman}\endnumbering\briefempfaengerindex{Schnitzler, Arthur@\textsc{Schnitzler, Arthur}!zzzAdam, Robert@\emph{von Robert Adam}!1917-05-161@{16. 5. 1917}|)be}\mylabel{L02260h}  \normalsize

\doendnotes{C}
\bigskip
\vfill

\clearpage

\footnotesize

\lohead{\textsc{register}}

% Definiere theindex-Environment komplett neu ohne reledmac
\makeatletter
\renewenvironment{theindex}{%
  \section*{\indexname}%
  \setlength{\parindent}{0pt}%
  \setlength{\parskip}{0pt plus 0.3pt}%
  \let\item\@idxitem
}{%
  \clearpage
}
\makeatother

\IfFileExists{\jobname-pw.ind}{\input{\jobname-pw.ind}}{}

\end{document}

      