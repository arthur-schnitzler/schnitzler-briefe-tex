%% latex-leseansicht-vorspann.tex
%% Vorspann für die Leseansicht.
%% Lädt die gemeinsame Datei latex-vorspann.tex mit nicht gesetztem Schalter.

\newif\ifkorrekturansicht
\korrekturansichtfalse

\input{../tex-inputs/latex-vorspann}


         
         \renewcommand{\erwaehntePersonen}{Personen: Gerhard Gutherz, Paul von Hindenburg}
         \renewcommand{\erwaehnteInstitutionen}{Institutionen: Nationaltheater München}
         \renewcommand{\erwaehnteOrte}{Orte: Babylon, Burgtheater, Wien, Österreich}
         \renewcommand{\erwaehnteWerke}{Werke: Faust. Eine Tragödie, Neidhard}
               \section[Robert Adam an Arthur Schnitzler, 16. 5. 1917]{ Robert Adam an Arthur Schnitzler, 16. 5. 1917}\nopagebreak\mylabel{v}\rehead{ }\begin{ledgroupsized}[t]{13cm}\normalsize\beginnumbering \toendnotes[C]{\smallbreak\pagebreak[2]} \Standort{DLA, A:Schnitzler, HS.NZ85.1.4230,18.}
\physDesc{Brief, 1 Blatt, 3 Seiten
\newline{}Handschrift: schwarze Tinte, deutsche Kurrent
\newline{}Schnitzler: 1) mit Bleistift beschriftet: »\textsc{Adam}«  2) mit rotem Buntstift eine Unterstreichung}\Standort{Wien, Österreichische Nationalbibliothek, Cod.ser. 52.263, 192.}
\physDesc{Brief, Maschinenschriftliche Abschrift, 1 Blatt, 1 Seite
\newline{}Schreibmaschine}\toendnotes[C]{\smallbreak}\pstart
           \raggedleft{}{\pb}Wien\oindex{Wien@\textbf{Wien}|pw}, am 16. Mai 1917\pend
           \pstart\center{}Hochverehrter Herr Doktor!\pend\pstart
           Ich empfinde nachgerade ein gewiſſes Schamgefühl, da jede Mitteilung, die ich Ihnen
               über meine literariſchen Geſchicke zu machen habe, die von einem Mißerfolg iſt. Alſo
               ſeit Jahren und nun alſo auch heute.\pend
           \pstart
           Das Münchner Hoftheater\orgindex{Nationaltheater Muenchen@Nationaltheater München|pw} hat den »Neidhard\pwindex{Adam, Robert 20.04.1877 – 16.10.1961@\textsc{Adam, Robert} (20.04.1877 – 16.10.1961), \emph{Schriftsteller, Richter}!NeidhardNone@\strich\emph{Neidhard} {[}None{]}|pw}« abgelehnt »wegen verſchiedener Mängel im dramatiſchen
               Aufbau« – \strikeout{gegen die} für die ich ſelbſt, bei Gott,
               nicht blind bin – »und wegen allzugroßer Längen« – deren Beteiligung im Wege von
               Strichen ich allerdings vorgeſchlagen hatte. Den {\pb}Dramaturgen\pwindex{Gutherz, Gerhard 07.09.1877 – 21.03.1942@\textsc{Gutherz, Gerhard} (07.09.1877 – 21.03.1942), \emph{Schriftsteller, Dramaturg}|pwv} hat indeß »die an
               vielen Stellen aufleuchtende Poeſie und Lyrik (ein \label{K_L02260_1v}\edtext{\griechisch{ἓν διὰ δυοῖν}}{\lemma{\textnormal{\emph{ἓν διὰ δυοῖν}}}\Cendnote{\textnormal{altgriechisch: eins mit zwei; Ausdruck der
                  Rhetorik, bei dem ein neuer Begriff aus zwei Wörten gebildet wird, wie hier
                  »Poesie und Lyrik«}}}\label{K_L02260_1h}) »ebenſo wie der witzige, fein pointierte Dialog in den
               Zwiſchenſpielen« »ſtark gefeſſelt«. Schade, daß die Zwiſchenſpiele nicht abendfüllend
               sind!\pend
           \pstart
           Da ſteh ich nun, ich armer Tor\pwindex{\textcolor{red}{\textsuperscript{XXXX1 indx}}!Faust. Eine Tragoedie1808@\strich\emph{Faust. Eine Tragödie} {[}1808{]}|pwv},
               und bin entſchloſſen, das Ende des Krieges abzuwarten und damit das Herankommen einer
               Zeit, die der ſcheußlichen deutſchfeindlichen Geſinnung, deren meiner Anſicht nach
               der »Neidhard\pwindex{Adam, Robert 20.04.1877 – 16.10.1961@\textsc{Adam, Robert} (20.04.1877 – 16.10.1961), \emph{Schriftsteller, Richter}!NeidhardNone@\strich\emph{Neidhard} {[}None{]}|pw}« voll iſt, verſtändnisvoller
               gegenüberſtehen dürfte als die Hindenburg\pwindex{Hindenburg, Paul von 02.10.1847 – 02.08.1934@\textsc{Hindenburg, Paul von} (02.10.1847 – 02.08.1934), \emph{Politiker}|pw}iſche.
               Oder ſoll ich das kühne Experiment wagen, den »Neidhard\pwindex{Adam, Robert 20.04.1877 – 16.10.1961@\textsc{Adam, Robert} (20.04.1877 – 16.10.1961), \emph{Schriftsteller, Richter}!NeidhardNone@\strich\emph{Neidhard} {[}None{]}|pw}«, ſobald er wieder in meinen Händen iſt, neuerlich zuſammenzupacken
               und dem Burgtheater\oindex{Burgtheater@\textbf{Burgtheater}|pw} mit der Verſicherung
               einzureichen, daß er dem chriſtlich-germaniſchen Schönheitsideal entſpricht? Da
               dieſes angefeindet {\pb}durch Nichtverwendung babylon\oindex{Babylon@\textbf{Babylon}|pw}iſcher Motive negativ determiniert iſt, iſt’s
               ſehr wohl möglich, daß der antichriſtlich-antigermaniſche »Neidhard\pwindex{Adam, Robert 20.04.1877 – 16.10.1961@\textsc{Adam, Robert} (20.04.1877 – 16.10.1961), \emph{Schriftsteller, Richter}!NeidhardNone@\strich\emph{Neidhard} {[}None{]}|pw}« ſeine volle Erfüllung bedeutet. Der Spaß wäre nicht ſo
               übel, und hätte ich nicht zu befürchten, daß in Folge des zu erwartenden Anſturms
               aller germaniſchen Chriſten und der dadurch bewirkten Ueberlaſtung des Lektors der
               arme »Neidhard\pwindex{Adam, Robert 20.04.1877 – 16.10.1961@\textsc{Adam, Robert} (20.04.1877 – 16.10.1961), \emph{Schriftsteller, Richter}!NeidhardNone@\strich\emph{Neidhard} {[}None{]}|pw}« \strikeout{nie}
               weit über die bevorſtehende Wiedergeburt Öſterreichs\oindex{Oesterreich@\textbf{Österreich}|pw} hinaus im Archive lagern bliebe, ich wagte wirklich gerne den
               Verſuch. –\pend
           \pstart
           Nehmen Sie, hochverehrter Herr Doktor, neuerlich meinen Dank für Ihre liebenswürdige
               Bemühung entgegen (wie geſagt, ich ſchäme mich meines unumbringbaren Pechs) und
               empfangen Sie die ergebenſten Grüße von Ihrem\pend
           \pstart \spacefill\mbox{Robert Adam}\pend{}
         
         \endnumbering\mylabel{h}\end{ledgroupsized}  \newcommand{\dateiname}{L02260}\newcommand{\titel}{Robert Adam an Arthur Schnitzler, 16. 5. 1917}\newcommand{\editorInnen}{Martin Anton Müller und Gerd-Hermann Susen}%% latex-leseansicht-abspann.tex
%% Abspann für die Leseansicht.
%% Der Schalter \ifkorrekturansicht ist bereits durch den Vorspann gesetzt.

%% latex-abspann.tex
%% Gemeinsamer Abspann für Korrekturansicht und Leseansicht.
%% Setzt den Schalter \ifkorrekturansicht voraus (gesetzt in den
%% einbindenden Dateien latex-korrekturansicht-abspann.tex bzw.
%% latex-leseansicht-abspann.tex).
%% ---------------------------------------------------------------

\normalsize

% Das esempio-Environment wird nur in der Leseansicht benötigt
\ifkorrekturansicht\else
\newenvironment{esempio}[3]%
{
    \vspace{1.5ex}
    \rlap{\underline{#1}}
    \par
    \setlength{\parindent}{0cm}
    \nopagebreak
    \leftskip=#2cm
    \rightskip=#3cm
}
{
    \par
}
\fi

\doendnotes{C}
\bigskip
\vfill

\clearpage

\footnotesize

\ifkorrekturansicht
  \lohead{\textsc{register}}
\fi

% theindex-Environment neu definieren ohne reledmac
\makeatletter
\renewenvironment{theindex}{%
  \ifkorrekturansicht
    \section*{\indexname}%
  \else
    \subsubsection*{Index der erwähnten Entitäten}%
  \fi
  \setlength{\parindent}{0pt}%
  \setlength{\parskip}{0pt plus 0.3pt}%
  \let\item\@idxitem
}{%
  \ifkorrekturansicht\clearpage\fi
}
\makeatother

\IfFileExists{\jobname-pw.ind}{\input{\jobname-pw.ind}}{}

% Quellenangabe nur in der Leseansicht
\ifkorrekturansicht\else
% Fallback-Definitionen, falls die .tex-Datei \titel etc. nicht gesetzt hat
\providecommand{\titel}{}
\providecommand{\editorInnen}{}
\providecommand{\dateiname}{\jobname}

\vspace{3cm}

\vfill

\footnotesize
\textsc{Quelle}: \titel. Herausgegeben von {\editorInnen}. In: \emph{Arthur Schnitzler: Briefwechsel mit Autorinnen und Autoren}.
 Digitale Edition, https://schnitzler-briefe.acdh.oeaw.ac.at/{\dateiname}.html (Stand \today)
\fi

\end{document}


      