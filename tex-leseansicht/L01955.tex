%% latex-korrekturansicht-vorspann.tex
%% Vorspann für die Korrekturansicht.
%% Lädt die gemeinsame Datei latex-vorspann.tex mit gesetztem Schalter.

\newif\ifkorrekturansicht
\korrekturansichttrue

\input{../tex-inputs/latex-vorspann}


\section[Arthur und Olga Schnitzler an Richard und Paula Beer-Hofmann, 17. 9. 1910]{L01955 Arthur und Olga Schnitzler an Richard und Paula Beer-Hofmann,
               17. 9. 1910}
\nopagebreak\mylabel{L01955v}
\rehead{ }\normalsize\beginnumbering\briefempfaengerindex{Beer-Hofmann, Paula@\textsc{Beer-Hofmann, Paula}!zzzSchnitzler, Olga@\emph{von Olga Schnitzler}!1910-09-171@{17. 9. 1910}|(be}\briefempfaengerindex{Beer-Hofmann, Paula@\textsc{Beer-Hofmann, Paula}!zzzSchnitzler, Arthur@\emph{von Arthur Schnitzler}!1910-09-171@{17. 9. 1910}|(be}\briefempfaengerindex{Beer-Hofmann, Richard@\textsc{Beer-Hofmann, Richard}!zzzSchnitzler, Olga@\emph{von Olga Schnitzler}!1910-09-171@{17. 9. 1910}|(be}\briefempfaengerindex{Beer-Hofmann, Richard@\textsc{Beer-Hofmann, Richard}!zzzSchnitzler, Arthur@\emph{von Arthur Schnitzler}!1910-09-171@{17. 9. 1910}|(be}
\toendnotes[C]{\smallbreak\pagebreak[2]}\Standort{YCGL, MSS 31.}
\physDesc{Bildpostkarte, 203 Zeichen
\newline{}Handschrift Arthur Schnitzler: Bleistift, deutsche Kurrent
\newline{}Handschrift Olga Schnitzler: Bleistift, lateinische Kurrent
\newline{}Versand: Stempel: »\nobreak{}\oindex{Ruedesheim@\textbf{Rüdesheim}, \emph{P.PPL}|pwk}Rüdesheim (Rhein), 17. 9. 10, 8–9N\nobreak{}«.  }\toendnotes[C]{\smallbreak}\pstart{}{\pb}Hrn Dr. \textsc{Rich. Beer
                     Hofmann}\pend{}\pstart{}u Gemahlin\pend{}\pstart{}Wien XVIII\oindex{XVIII., Waehring@\textbf{XVIII., Währing}, \emph{A.ADM3}|pw}\pend{}\pstart{}\textsc{Hasenauerstr. 59\oindex{Hasenauerstrasse 59@\textbf{Hasenauerstraße 59}, \emph{Wohngebäude (K.WHS)}|pw}}\pend{}{\bigskip}
\pstart
           \noindent{}\centering{}{\pb}\textcolor{gray}{\textbf{Rüdesheim\oindex{Ruedesheim@\textbf{Rüdesheim}, \emph{P.PPL}|pw}.\hspace*{1.5em}Der Rhein\oindex{Rhein@\textbf{Rhein}, \emph{Fluss (N.FLS)}|pw}.}}\pend
           \vspace{1em}
\pstart
           \noindent{}{\pb}Herzliche Grüße. Eigentlich ſind wir in Frankfurt \textsuperscript{a}/M \oindex{Frankfurt am Main@\textbf{Frankfurt am Main}, \emph{P.PPLA3}|pw}; dies
               iſt ein Ausflug.\pend
           \pstart \spacefill\mbox{A.}\pend{}
\pstart
           17. 9. 1910\pend
           \selectlanguage{ngerman}\vspace{1em}
\pstart
           \noindent{}{[}hs. :{]} Herzliche Grüsse, \label{K_L01955-1v}\edtext{Dienstag}{\lemma{\textnormal{\emph{Dienstag}}}\Cendnote{\textnormal{Siehe A. S.: \emph{Tagebuch}, 20. 9. 1910.
               }}}\label{K_L01955-1} hoffen wir in Wien\oindex{Wien@\textbf{Wien}, \emph{A.ADM2}|pw} zu sein.\pend
           \pstart \spacefill\mbox{Olga.}\pend{}\selectlanguage{ngerman}\endnumbering\briefempfaengerindex{Beer-Hofmann, Paula@\textsc{Beer-Hofmann, Paula}!zzzSchnitzler, Olga@\emph{von Olga Schnitzler}!1910-09-171@{17. 9. 1910}|)be}\briefempfaengerindex{Beer-Hofmann, Paula@\textsc{Beer-Hofmann, Paula}!zzzSchnitzler, Arthur@\emph{von Arthur Schnitzler}!1910-09-171@{17. 9. 1910}|)be}\briefempfaengerindex{Beer-Hofmann, Richard@\textsc{Beer-Hofmann, Richard}!zzzSchnitzler, Olga@\emph{von Olga Schnitzler}!1910-09-171@{17. 9. 1910}|)be}\briefempfaengerindex{Beer-Hofmann, Richard@\textsc{Beer-Hofmann, Richard}!zzzSchnitzler, Arthur@\emph{von Arthur Schnitzler}!1910-09-171@{17. 9. 1910}|)be}\mylabel{L01955h}  \normalsize

\doendnotes{C}
\bigskip
\vfill

\clearpage

\footnotesize

\lohead{\textsc{register}}

% Definiere theindex-Environment komplett neu ohne reledmac
\makeatletter
\renewenvironment{theindex}{%
  \section*{\indexname}%
  \setlength{\parindent}{0pt}%
  \setlength{\parskip}{0pt plus 0.3pt}%
  \let\item\@idxitem
}{%
  \clearpage
}
\makeatother

\IfFileExists{\jobname-pw.ind}{\input{\jobname-pw.ind}}{}

\end{document}

      