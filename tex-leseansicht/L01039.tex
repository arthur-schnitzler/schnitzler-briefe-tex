%% latex-korrekturansicht-vorspann.tex
%% Vorspann für die Korrekturansicht.
%% Lädt die gemeinsame Datei latex-vorspann.tex mit gesetztem Schalter.

\newif\ifkorrekturansicht
\korrekturansichttrue

\input{../tex-inputs/latex-vorspann}


\section[Adalbert Seligmann an Arthur Schnitzler, {[}20. 5. 1900{]}]{L01039 Adalbert Seligmann an Arthur Schnitzler, {[}20. 5. 1900{]}}
\nopagebreak\mylabel{L01039v}
\rehead{ }\normalsize\beginnumbering\briefempfaengerindex{Schnitzler, Arthur@\textsc{Schnitzler, Arthur}!zzzSeligmann, Adalbert Franz@\emph{von Adalbert Franz Seligmann}!1900-05-201@{{[}20. 5. 1900{]}}|(be}
\toendnotes[C]{\smallbreak\pagebreak[2]}\Standort{CUL, Schnitzler, B 97.}
\physDesc{Brief, 1 Blatt, 1 Seite, 393 Zeichen
\newline{}Handschrift: schwarze Tinte, deutsche Kurrent
\newline{}Schnitzler: mit Bleistift datiert: »20. Mai 900« und nummeriert: »1« }
\pstart{}{\pb}Sehr verehrter Freund!\pend\vspace{0.5em}
\pstart
           Sie \uline{müſſen} mir ein Exemplar von Ihrem »Reigen\pwindex{Reigen. Zehn Dialoge@\emph{Reigen. Zehn Dialoge}|pw}« ſchicken, und wenn Sie noch ein Exemplar
               für mich drucken laſſen müßten. Sie ſollen nicht umſonſt einen Blick »hinter das Leben\pwindex{Hinter dem Leben@\emph{Hinter dem Leben}|pw}« gethan haben. Aug’ um Auge, Zahn um Zahn
               Literatur um Literatur. In Erwartung einer freundlichen Sendung (meinetwegen in
               Begleitung eines groben Briefes)\pend
           
\pstart
           Ihr ergebener{\\[\baselineskip]}\spacefill\mbox{AF. Seligmann}\pend
           \leftskip=0em{}
\pstart
           \noindent{}\textsc{I. Bäc\damage{\textcolor{gray}{k}}erstraße 1}\oindex{Baeckerstrasse@\textbf{Bäckerstraße}, \emph{Straße (K.STR)}|pw}.\pend
           \selectlanguage{ngerman}\endnumbering\briefempfaengerindex{Schnitzler, Arthur@\textsc{Schnitzler, Arthur}!zzzSeligmann, Adalbert Franz@\emph{von Adalbert Franz Seligmann}!1900-05-201@{{[}20. 5. 1900{]}}|)be}\mylabel{L01039h}  \normalsize

\doendnotes{C}
\bigskip
\vfill

\clearpage

\footnotesize

\lohead{\textsc{register}}

% Definiere theindex-Environment komplett neu ohne reledmac
\makeatletter
\renewenvironment{theindex}{%
  \section*{\indexname}%
  \setlength{\parindent}{0pt}%
  \setlength{\parskip}{0pt plus 0.3pt}%
  \let\item\@idxitem
}{%
  \clearpage
}
\makeatother

\IfFileExists{\jobname-pw.ind}{\input{\jobname-pw.ind}}{}

\end{document}

      