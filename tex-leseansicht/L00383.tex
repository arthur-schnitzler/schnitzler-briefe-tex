%% latex-leseansicht-vorspann.tex
%% Vorspann für die Leseansicht.
%% Lädt die gemeinsame Datei latex-vorspann.tex mit nicht gesetztem Schalter.

\newif\ifkorrekturansicht
\korrekturansichtfalse

\input{../tex-inputs/latex-vorspann}


\section[Richard Beer-Hofmann an Arthur Schnitzler, 15. 10. 1894]{L00383 Richard Beer-Hofmann an Arthur Schnitzler, 15. 10. 1894}
\nopagebreak\mylabel{L00383v}
\rehead{ }\normalsize\beginnumbering\briefempfaengerindex{Schnitzler, Arthur@\textsc{Schnitzler, Arthur}!zzzBeer-Hofmann, Richard@\emph{von Richard Beer-Hofmann}!1894-10-152@{15. 10. 1894}|(be}
\toendnotes[C]{\smallbreak\pagebreak[2]}
\correspDesc{Versand  durch Richard Beer-Hofmann am 15. 10. 1894 in Frascati
\newline{}Erhalt  durch Arthur Schnitzler im Zeitraum [16. 10. 1894 – 20. 10. 1894?] in Wien}\toendnotes[C]{\smallbreak}
\Standort{CUL, Schnitzler, B 8.}
\physDesc{Brief, 1 Blatt, 4 Seiten, 701 Zeichen
\newline{}Handschrift: Bleistift, deutsche Kurrent
\newline{}Schnitzler: mit Bleistift datiert: »15/10 94« und nummeriert: »40« 
\newline{}Ordnung: mit Bleistift von unbekannter Hand nummeriert:
                                    »40« }
\buchAbdrucke{\weitereDrucke{Arthur Schnitzler, Richard Beer-Hofmann: \emph{Briefwechsel 1891–1931}. Herausgegeben von Konstanze Fliedl. Wien, Zürich: \emph{Europaverlag} 1992, S. 63.} }\toendnotes[C]{\smallbreak}
\pstart
           \raggedleft{}{\pb}Fraskati\oindex{Frascati@\textbf{Frascati}, \emph{Hauptstadt}|pw}{ }Sonntag{ }½ 8\pend
           \vspace{0.5em}
\pstart
           {\pb}Lieber Arthur, diesen Brief schreibe ich au\substVorne{}\textsuperscript{s}\substDazwischen{}f\substHinten{}{ }\substVorne{}\textsuperscript{a}\substDazwischen{}e\substHinten{}iner Terrasse \strikeout{b} in Fraskati\oindex{Frascati@\textbf{Frascati}, \emph{Hauptstadt}|pw}, stehend, im Mondlicht; ich habe nämlich noch eine
               halbe Stunde Zeit bis zum Abgang des Zuges nach Rom\oindex{Rom@\textbf{Rom}, \emph{Hauptstadt}|pw}. {\pb}Ich bin sehr »des Gottes voll\pwindex{\textcolor{red}{\textsuperscript{XXXX indx1}}!Kraniche des Ibykus@\strich\emph{Die Kraniche des Ibykus}|pwv}« aber \uline{arbeite} gar nichts, und notire mittelmäßig viel. Ich
               sehe vieles anders und verstehe Einiges was mir fremd war. Arroganter werd ich {\pb}sein als je, wenn ich zurückko{\geminationm}e. Wenn man tagsüber mit schönen Bildern, einer Natur
               die hier Künstlerin ist, und mit – seinen Gedanken – verkehrt {\pb}findet man die Gesellschaft die um
               uns (– wie heißt das analoge Wort zu\pend
           \settowidth{\longeste}{crepiren!}\settowidth{\longestz}{–}\settowidth{\longestd}{sterben}\settowidth{\longestv}{}\settowidth{\longestf}{}\addtolength\longeste{1em}
        \addtolength\longestz{1em}
        \addtolength\longestd{1em}
      \pstart\noindent\makebox[\the\longeste][l]{crepiren\strikeout{!}}\makebox[\the\longestz][l]{–}
                  \makebox[\the\longestd][l]{sterben}\pend\pstart\noindent\makebox[\the\longeste][l]{×}\makebox[\the\longestz][l]{–}
                  \makebox[\the\longestd][l]{leben)}\pend
\pstart
           unmöglich; ich bin am 4. od. 5. voraussichtlich in Wien\oindex{Wien@\textbf{Wien}, \emph{Verwaltungsgebiet}|pw}; von morgen an \uline{Neapel\oindex{Neapel@\textbf{Neapel}|pw}} a posta ferma.\pend
           \pstart Herzlichst Ihr \spacefill\mbox{R}\pend{}\selectlanguage{ngerman}\endnumbering\briefempfaengerindex{Schnitzler, Arthur@\textsc{Schnitzler, Arthur}!zzzBeer-Hofmann, Richard@\emph{von Richard Beer-Hofmann}!1894-10-152@{15. 10. 1894}|)be}\mylabel{L00383h}  \newcommand{\dateiname}{L00383}\newcommand{\titel}{Richard Beer-Hofmann an Arthur Schnitzler, 15. 10. 1894}\newcommand{\editorInnen}{Martin Anton Müller und Gerd-Hermann Susen}%% latex-leseansicht-abspann.tex
%% Abspann für die Leseansicht.
%% Der Schalter \ifkorrekturansicht ist bereits durch den Vorspann gesetzt.

%% latex-abspann.tex
%% Gemeinsamer Abspann für Korrekturansicht und Leseansicht.
%% Setzt den Schalter \ifkorrekturansicht voraus (gesetzt in den
%% einbindenden Dateien latex-korrekturansicht-abspann.tex bzw.
%% latex-leseansicht-abspann.tex).
%% ---------------------------------------------------------------

\normalsize

% Das esempio-Environment wird nur in der Leseansicht benötigt
\ifkorrekturansicht\else
\newenvironment{esempio}[3]%
{
    \vspace{1.5ex}
    \rlap{\underline{#1}}
    \par
    \setlength{\parindent}{0cm}
    \nopagebreak
    \leftskip=#2cm
    \rightskip=#3cm
}
{
    \par
}
\fi

\doendnotes{C}
\bigskip
\vfill

\clearpage

\footnotesize

\ifkorrekturansicht
  \lohead{\textsc{register}}
\fi

% theindex-Environment neu definieren ohne reledmac
\makeatletter
\renewenvironment{theindex}{%
  \ifkorrekturansicht
    \section*{\indexname}%
  \else
    \subsubsection*{Index der erwähnten Entitäten}%
  \fi
  \setlength{\parindent}{0pt}%
  \setlength{\parskip}{0pt plus 0.3pt}%
  \let\item\@idxitem
}{%
  \ifkorrekturansicht\clearpage\fi
}
\makeatother

\IfFileExists{\jobname-pw.ind}{\input{\jobname-pw.ind}}{}

% Quellenangabe nur in der Leseansicht
\ifkorrekturansicht\else
% Fallback-Definitionen, falls die .tex-Datei \titel etc. nicht gesetzt hat
\providecommand{\titel}{}
\providecommand{\editorInnen}{}
\providecommand{\dateiname}{\jobname}

\vspace{3cm}

\vfill

\footnotesize
\textsc{Quelle}: \titel. Herausgegeben von {\editorInnen}. In: \emph{Arthur Schnitzler: Briefwechsel mit Autorinnen und Autoren}.
 Digitale Edition, https://schnitzler-briefe.acdh.oeaw.ac.at/{\dateiname}.html (Stand \today)
\fi

\end{document}


