%% latex-korrekturansicht-vorspann.tex
%% Vorspann für die Korrekturansicht.
%% Lädt die gemeinsame Datei latex-vorspann.tex mit gesetztem Schalter.

\newif\ifkorrekturansicht
\korrekturansichttrue

\input{../tex-inputs/latex-vorspann}


\section[Richard Beer-Hofmann an Arthur Schnitzler, 15. 10. 1894]{L00383 Richard Beer-Hofmann an Arthur Schnitzler, 15. 10. 1894}
\nopagebreak\mylabel{L00383v}
\rehead{ }\normalsize\beginnumbering\briefempfaengerindex{Schnitzler, Arthur@\textsc{Schnitzler, Arthur}!zzzBeer-Hofmann, Richard@\emph{von Richard Beer-Hofmann}!1894-10-152@{15. 10. 1894}|(be}
\toendnotes[C]{\smallbreak\pagebreak[2]}\Standort{CUL, Schnitzler, B 8.}
\physDesc{Brief, 1 Blatt, 4 Seiten, 701 Zeichen
\newline{}Handschrift: Bleistift, deutsche Kurrent
\newline{}Schnitzler: mit Bleistift datiert: »15/10 94« und nummeriert: »40« 
\newline{}Ordnung: mit Bleistift von unbekannter Hand nummeriert:
                                    »40« }
\buchAbdrucke{\weitereDrucke{Arthur Schnitzler, Richard Beer-Hofmann: \emph{Briefwechsel 1891–1931}. Wien, Zürich: \emph{Europaverlag} 1992, S. 63.} }\toendnotes[C]{\smallbreak}
\pstart
           \raggedleft{}{\pb}Fraskati\oindex{Frascati@\textbf{Frascati}, \emph{P.PPLA3}|pw}{ }Sonntag{ }½ 8\pend
           \vspace{0.5em}
\pstart
           {\pb}Lieber Arthur, diesen Brief schreibe ich au\substVorne{}\textsuperscript{s}\substDazwischen{}f\substHinten{}{ }\substVorne{}\textsuperscript{a}\substDazwischen{}e\substHinten{}iner Terrasse \strikeout{b} in Fraskati\oindex{Frascati@\textbf{Frascati}, \emph{P.PPLA3}|pw}, stehend, im Mondlicht; ich habe nämlich noch eine
               halbe Stunde Zeit bis zum Abgang des Zuges nach Rom\oindex{Rom@\textbf{Rom}, \emph{P.PPLC}|pw}. {\pb}Ich bin sehr »des Gottes voll\pwindex{Kraniche des Ibykus@\emph{Die Kraniche des Ibykus}|pwv}« aber \uline{arbeite} gar nichts, und notire mittelmäßig viel. Ich
               sehe vieles anders und verstehe Einiges was mir fremd war. Arroganter werd ich {\pb}sein als je, wenn ich zurückko{\geminationm}e. Wenn man tagsüber mit schönen Bildern, einer Natur
               die hier Künstlerin ist, und mit – seinen Gedanken – verkehrt {\pb}findet man die Gesellschaft die um
               uns (– wie heißt das analoge Wort zu\pend
           \settowidth{\longeste}{}\settowidth{\longestz}{}\settowidth{\longestd}{}\settowidth{\longestv}{}\settowidth{\longestf}{}\addtolength\longeste{1em}
        \addtolength\longestz{1em}
      
\pstart
           unmöglich; ich bin am 4. od. 5. voraussichtlich in Wien\oindex{Wien@\textbf{Wien}, \emph{A.ADM2}|pw}; von morgen an \uline{Neapel\oindex{Neapel@\textbf{Neapel}, \emph{P.PPLA}|pw}} a posta ferma.\pend
           \pstart Herzlichst Ihr \spacefill\mbox{R}\pend{}\selectlanguage{ngerman}\endnumbering\briefempfaengerindex{Schnitzler, Arthur@\textsc{Schnitzler, Arthur}!zzzBeer-Hofmann, Richard@\emph{von Richard Beer-Hofmann}!1894-10-152@{15. 10. 1894}|)be}\mylabel{L00383h}  \normalsize

\doendnotes{C}
\bigskip
\vfill

\clearpage

\footnotesize

\lohead{\textsc{register}}

% Definiere theindex-Environment komplett neu ohne reledmac
\makeatletter
\renewenvironment{theindex}{%
  \section*{\indexname}%
  \setlength{\parindent}{0pt}%
  \setlength{\parskip}{0pt plus 0.3pt}%
  \let\item\@idxitem
}{%
  \clearpage
}
\makeatother

\IfFileExists{\jobname-pw.ind}{\input{\jobname-pw.ind}}{}

\end{document}

      