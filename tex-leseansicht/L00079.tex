%% latex-korrekturansicht-vorspann.tex
%% Vorspann für die Korrekturansicht.
%% Lädt die gemeinsame Datei latex-vorspann.tex mit gesetztem Schalter.

\newif\ifkorrekturansicht
\korrekturansichttrue

\input{../tex-inputs/latex-vorspann}


\section[Arthur Schnitzler an Richard Beer-Hofmann, 11. 3. 1892]{L00079 Arthur Schnitzler an Richard Beer-Hofmann, 11. 3. 1892}
\nopagebreak\mylabel{L00079v}
\rehead{ }\normalsize\beginnumbering\briefempfaengerindex{Beer-Hofmann, Richard@\textsc{Beer-Hofmann, Richard}!zzzSchnitzler, Arthur@\emph{von Arthur Schnitzler}!1892-03-111@{11. 3. 1892}|(be}
\toendnotes[C]{\smallbreak\pagebreak[2]}\Standort{YCGL, MSS 31.}
\physDesc{Brief, 2 Blätter, 8 Seiten, Umschlag, 2976 Zeichen
\newline{}Handschrift: schwarze Tinte, deutsche Kurrent
\newline{}Versand: 1) Stempel: »\nobreak{}\oindex{I., Innere Stadt@\textbf{I., Innere Stadt}, \emph{A.ADM3}|pwk}Wien 1/1, 11 3 92, 7–8 N\nobreak{}«.   2) Stempel: »\nobreak{}\oindex{Opatija@\textbf{Opatija}, \emph{P.PPLA2}|pwk}Abbazia, 13{[}. 3.{]} 92\nobreak{}«. }
\buchAbdrucke{\weitereDrucke{1) Arthur Schnitzler: \emph{Briefe 1875–1912}. Frankfurt am Main: \emph{S. Fischer} 1981, S. 121–122.} \weitereDrucke{2) Arthur Schnitzler: \emph{Briefe 1875–1912}. Frankfurt am Main: \emph{S. Fischer} 1981, S. 120–121.} \weitereDrucke{3) Arthur Schnitzler, Richard Beer-Hofmann: \emph{Briefwechsel 1891–1931}. Wien, Zürich: \emph{Europaverlag} 1992, S. 34–35.} \weitereDrucke{4) Hermann Bahr, Arthur Schnitzler: \emph{Briefwechsel, Aufzeichnungen, Dokumente (1891–1931)}. Göttingen: \emph{Wallstein} 2018, S. 22–23.} }\toendnotes[C]{\smallbreak}\pstart{}{\pb}\textcolor{gray}{\textbf{\textit{\label{T_L00079-1v}\edtext{AS}{\lemma{\textnormal{\emph{AS}}}\Cendnote{\textnormal{rotes Wachssiegel}}}\label{T_L00079-1}}}}\pend{}{\bigskip}\pstart{}{\pb}Herrn \textsc{Dr. Rich.
                        Beer-Hofm\damage{a}nn}\pend{}\pstart{}\textsc{Abbazia\oindex{Opatija@\textbf{Opatija}, \emph{P.PPLA2}|pw}}\pend{}\pstart{}\textsc{Pension Quisisana\oindex{Pension Quisisana@\textbf{Pension Quisisana}, \emph{Hotel (K.HTL)}|pw}}\pend{}{\bigskip}\vspace{1em}
\pstart
           \raggedleft{}{\pb}Wien\oindex{Wien@\textbf{Wien}, \emph{A.ADM2}|pw}, 11. März 92.\pend
           
\pstart{}Lieber Richard,\pend\vspace{0.5em}
\pstart
           Kafka\pwindex{Kafka, Eduard Michael 11.03.1869 – 06.08.1893@\textsc{Kafka, Eduard Michael} (11.03.1869 – 06.08.1893), \emph{Redakteur/Redakteurin}|pw} habe ich die letzten Tage nicht geſehn.
               Das letzte Mal an unſerem Vereinsabend\orgindex{»Freie Buehne« Verein fuer moderne Literatur@»Freie Bühne« Verein für moderne Literatur|pwv}, der nur einen Lichtpunkt hatte: Bahr’s\pwindex{Bahr, Hermann 19.07.1863 – 15.01.1934@\textsc{Bahr, Hermann} (19.07.1863 – 15.01.1934), \emph{Schriftsteller/Schriftstellerin, Kritiker/Kritikerin}|pw} »\label{K_L00079-1v}\edtext{treue Adele\pwindex{treue Adele. Eine vergessliche Geschichte@\emph{Die treue Adele. Eine vergeßliche Geschichte}|pw}}{\lemma{\textnormal{\emph{treue Adele}}}\Cendnote{\textnormal{Hermann Bahr\pwindex{Bahr, Hermann 19.07.1863 – 15.01.1934@\textsc{Bahr, Hermann} (19.07.1863 – 15.01.1934), \emph{Schriftsteller/Schriftstellerin, Kritiker/Kritikerin}|pwk}: \emph{Die treue Adele. Eine vergeßliche Geschichte}\pwindex{treue Adele. Eine vergessliche Geschichte@\emph{Die treue Adele. Eine vergeßliche Geschichte}|pwk}. In: \emph{Die Gesellschaft}\pwindex{Gesellschaft. Monatsschrift fuer Litteratur, Kunst und Sozialpolitik@\emph{Die Gesellschaft. Monatsschrift für Litteratur, Kunst und Sozialpolitik}|pwk}, Jg. 5, Nr. 11,
                        November 1889, S. 1556–1564 (Erstausgabe in \emph{Fin de Siècle}\pwindex{Fin de Siecle@\emph{Fin de Siècle}|pwk}, S. 71–88).}}}\label{K_L00079-1}« von Bahr\pwindex{Bahr, Hermann 19.07.1863 – 15.01.1934@\textsc{Bahr, Hermann} (19.07.1863 – 15.01.1934), \emph{Schriftsteller/Schriftstellerin, Kritiker/Kritikerin}|pw} vorgeleſen. Er las entzückend. \textsc{Meixner}\pwindex{Meixner, Julius 15.06.1850 – 03.01.1913@\textsc{Meixner, Julius} (15.06.1850 – 03.01.1913), \emph{Schriftsteller/Schriftstellerin, Schauspieler/Schauspielerin, Pädagoge/Pädagogin}|pw} las Parabeln von Kafka\pwindex{Kafka, Eduard Michael 11.03.1869 – 06.08.1893@\textsc{Kafka, Eduard Michael} (11.03.1869 – 06.08.1893), \emph{Redakteur/Redakteurin}|pw} und ein Gedicht
                  \textsc{Liliencron}\pwindex{Liliencron, Detlev von 03.06.1844 – 22.07.1909@\textsc{Liliencron, Detlev von} (03.06.1844 – 22.07.1909), \emph{Schriftsteller/Schriftstellerin, Dichter/Dichterin, Dramatiker/Dramatikerin}|pw}{ }ſehr ſchlecht vor. \textsc{Polland}\pwindex{Pollandt, Max 26.10.1861 – 18.07.1905@\textsc{Pollandt, Max} (26.10.1861 – 18.07.1905), \emph{Schauspieler/Schauspielerin}|pw} das Kaffehaus\pwindex{Kaffeehaus]@\emph{[Das Kaffeehaus]}|pw} von \textsc{Salten}\pwindex{Salten, Felix 06.09.1869 – 08.10.1945@\textsc{Salten, Felix} (06.09.1869 – 08.10.1945), \emph{Schriftsteller/Schriftstellerin, Journalist/Journalistin, Chefredakteur/Chefredakteurin}|pw}, Gedichte von \textsc{Loris}\pwindex{Hofmannsthal, Hugo von 1874-02-01 – 1929-07-15@\textsc{Hofmannsthal, Hugo von} (1874-02-01 – 1929-07-15), \emph{Schriftsteller/Schriftstellerin}|pw}, Korff\pwindex{Korff, Heinrich von 05.06.1868 – 18.08.1938@\textsc{Korff, Heinrich von} (05.06.1868 – 18.08.1938), \emph{Journalist/Journalistin}|pw} u mir unbeſchreiblich
               entſetzlich. Es iſt unmöglich, ſich von dieſer talentloſen Brüllerei einen Begriff zu
               machen, we{\geminationn} man nicht dabei {\pb}war. – Zum Schluſs wurde getanzt. Von mir nicht,
               bitte. – \pend
           
\pstart
           \textsc{Blumenthal}\pwindex{Blumenthal, Oskar 13.03.1852 – 24.04.1917@\textsc{Blumenthal, Oskar} (13.03.1852 – 24.04.1917), \emph{Schriftsteller/Schriftstellerin, Journalist/Journalistin, Theaterleiter/Theaterleiterin}|pw} war hier, ich ſprach ihn. Er will Kürzungen und einige Aenderungen am Mährchen\pwindex{Maerchen. Schauspiel in drei Aufzuegen@\emph{Das Märchen. Schauspiel in drei Aufzügen}|pw}. Einiges wird ſich wohl thun laſſen; ich
               habe mich ſchon daran gemacht, und die ſchöne Fremdheit, die mich vom Märchen\pwindex{Maerchen. Schauspiel in drei Aufzuegen@\emph{Das Märchen. Schauspiel in drei Aufzügen}|pw} bereits tre{\geminationn}t, läßt mich die Dinge leichter vollbringen. Daß \textsc{Blumenthal}\pwindex{Blumenthal, Oskar 13.03.1852 – 24.04.1917@\textsc{Blumenthal, Oskar} (13.03.1852 – 24.04.1917), \emph{Schriftsteller/Schriftstellerin, Journalist/Journalistin, Theaterleiter/Theaterleiterin}|pw} auch den Titel des Stückes geändert haben möchte, iſt Caeſarenwahnſinn. Es iſt
               ihm auch ſchon ſelbſt ein neuer eingefallen – er{\pb}ſchrecken Sie nicht – »Die Vergangenheit.« Erke{\geminationn}en Sie
               ihn!? Und noch i{\geminationm}er läßt man die erſt- und zweitgradigen
               frei herum laufen, die doch nur dazu da ſind, um den dritt und viertgradigen das
               Leben zu vermießen. –\pend
           
\pstart
           Geſtern hab ich mein neues Stück\pwindex{Familie@\emph{Familie}|pwv} begonnen. Außerdem schreibe ich \textsc{slowly},
               langſam an meiner Novelle\pwindex{Sterben. Novelle@\emph{Sterben. Novelle}|pwv}. –\pend
           
\pstart
           \textsc{Fontane} (Verlag)\orgindex{F. Fontane@F. Fontane|pw} hat mich freundlichſt erſucht, den
                  \textsc{Anatol-Cyclus}\pwindex{Anatol@\emph{Anatol}|pw} – \uline{nicht} einzuſenden, {\pb}»da ſie kaum die Zeit finden dürften, meiner Sa{\geminationm}lung einen ſorgfältigen u energiſchen Vertrieb
               angedeihen zu laſſen \textsc{etc etc}«\pend
           
\pstart
           – Aus den »\textsc{Aveugles\pwindex{Blinden@\emph{Die Blinden}|pw}}« ſcheint wirklich was zu werden. Doch ſoll dazu weder Pantomime noch Abschiedsſouper\pwindex{Abschiedssouper@\emph{Abschiedssouper}|pw} gegeben werden, ſondern »\textsc{l’Intrus\pwindex{Intruse@\emph{L’Intruse}|pw}}«. – Zu den beiden ein Vortrag von \textsc{Bahr}\pwindex{Bahr, Hermann 19.07.1863 – 15.01.1934@\textsc{Bahr, Hermann} (19.07.1863 – 15.01.1934), \emph{Schriftsteller/Schriftstellerin, Kritiker/Kritikerin}|pw}. Später ſoll ein Pantomimen u Luſtſpielabend arrangirt werden. Man kam mit dem
                  \label{K_L00079-2v}\edtext{\textsc{fait accompli}}{\lemma{\textnormal{\emph{fait accompli}}}\Cendnote{\textnormal{französisch: beschlossene Sache}}}\label{K_L00079-2}
               zu uns, das {\pb}freilich meinen Beifall nicht hat. –\pend
           
\pstart
           \textsc{Loris}\pwindex{Hofmannsthal, Hugo von 1874-02-01 – 1929-07-15@\textsc{Hofmannsthal, Hugo von} (1874-02-01 – 1929-07-15), \emph{Schriftsteller/Schriftstellerin}|pw}{ }ſchreibt viel, \textsc{Salten}\pwindex{Salten, Felix 06.09.1869 – 08.10.1945@\textsc{Salten, Felix} (06.09.1869 – 08.10.1945), \emph{Schriftsteller/Schriftstellerin, Journalist/Journalistin, Chefredakteur/Chefredakteurin}|pw}{ }ſchreibt wenig. Die andern ſeh ich gar nicht; das
                  \textsc{Café Griensteidl}\oindex{Cafe Griensteidl@\textbf{Café Griensteidl}, \emph{Kaffeehaus (K.KAF)}|pw} exiſtirt für mich nicht mehr. – \pend
           
\pstart
           Ich leſe \textsc{Taine}\pwindex{Taine, Hippolyte 21.04.1828 – 05.03.1893@\textsc{Taine, Hippolyte} (21.04.1828 – 05.03.1893), \emph{Philosoph/Philosophin, Geschichtsschreiber/Geschichtsschreiberin}|pw}, \textsc{ancien régime}\pwindex{Ancien regime@\emph{L’Ancien régime}|pw}, \textsc{Du Prel}\pwindex{Du Prel, Carl 1839-04-03 – 1899-08-05@\textsc{Du Prel, Carl} (1839-04-03 – 1899-08-05), \emph{Schriftsteller/Schriftstellerin, Philosoph/Philosophin}|pw}, Philoſophie der Myſtik\pwindex{Philosophie der Mystik@\emph{Die Philosophie der Mystik}|pw}, \textsc{Restif de la Bretonne}\pwindex{Retif de la Bretonne, Nicolas 1734-10-23 – 1806-02-03@\textsc{Rétif de la Bretonne, Nicolas} (1734-10-23 – 1806-02-03), \emph{Schriftsteller/Schriftstellerin}|pw}, \textsc{l’amour à 45 ans}\pwindex{Sara, ou L amour à quarante-cinq ans@\emph{Sara, ou L’amour à quarante-cinq ans}|pw}, \textsc{Kretzer}\pwindex{Kretzer, Max 1854-06-07 – 1941@\textsc{Kretzer, Max} (1854-06-07 – 1941), \emph{Schriftsteller/Schriftstellerin}|pw}, die Betrogenen\pwindex{Betrogenen@\emph{Die Betrogenen}|pw} u. a. – \pend
           
\pstart
           Die Menſchen \textsc{enerviren} mich. Manche miſchen ſich in meine
               Privatangelegenheiten, und nie{\pb}manden gehen ſie an.
               Das Geſindel hat tauſend Augen für Vorfälle, dafür taube Ohren für Einfälle. Aber mit
               der Zeit wird ſich die Menſchheit wohl »ausſchalten« laſſen, wie? Einen Harfeniſten
                  ka{\geminationn} man aus dem Hofe weiſen laſſen, we{\geminationn} er einen mit ſeinem Geklimper quält; wer aber befreit
               mich von den – andern? \pend
           
\pstart
           Ich will verſuchen, ein Virtuoſe der Einſamkeit zu werden. Eines ſchönen Tages werden
               alle Leute, die mich geniren, {\pb}nicht mehr daſein – und
               werden es nicht einmal bemerken. So wollen wir die Unbequemen zu relativem Tod
               verurtheilen: wir vom »großen Orden«! – Oder hätte Sie \textsc{Salten}\pwindex{Salten, Felix 06.09.1869 – 08.10.1945@\textsc{Salten, Felix} (06.09.1869 – 08.10.1945), \emph{Schriftsteller/Schriftstellerin, Journalist/Journalistin, Chefredakteur/Chefredakteurin}|pw} abreiſen laſſen, ohne Ihnen den großen Orden zu erläutern? –\pend
           
\pstart
           Schreiben Sie mir bald, und möglichſt viel, es muſs doch ganz ſchön ſein, we{\geminationn} man einmal wo anders iſt. Und dann, ſchreiben Sie –
               wir erwarten es, wir – vom großen Orden. –\pend
           
\pstart
           {\pb}Herzlichſt Ihr{\\[\baselineskip]}\spacefill\mbox{Arthur Sch}\pend
           \leftskip=0em{}\selectlanguage{ngerman}\endnumbering\briefempfaengerindex{Beer-Hofmann, Richard@\textsc{Beer-Hofmann, Richard}!zzzSchnitzler, Arthur@\emph{von Arthur Schnitzler}!1892-03-111@{11. 3. 1892}|)be}\mylabel{L00079h}  \normalsize

\doendnotes{C}
\bigskip
\vfill

\clearpage

\footnotesize

\lohead{\textsc{register}}

% Definiere theindex-Environment komplett neu ohne reledmac
\makeatletter
\renewenvironment{theindex}{%
  \section*{\indexname}%
  \setlength{\parindent}{0pt}%
  \setlength{\parskip}{0pt plus 0.3pt}%
  \let\item\@idxitem
}{%
  \clearpage
}
\makeatother

\IfFileExists{\jobname-pw.ind}{\input{\jobname-pw.ind}}{}

\end{document}

      