%% latex-leseansicht-vorspann.tex
%% Vorspann für die Leseansicht.
%% Lädt die gemeinsame Datei latex-vorspann.tex mit nicht gesetztem Schalter.

\newif\ifkorrekturansicht
\korrekturansichtfalse

\input{../tex-inputs/latex-vorspann}


\section[Arthur Schnitzler an Richard Beer-Hofmann, 11. 3. 1892]{L00079 Arthur Schnitzler an Richard Beer-Hofmann, 11. 3. 1892}
\nopagebreak\mylabel{L00079v}
\rehead{ }\normalsize\beginnumbering\briefempfaengerindex{Beer-Hofmann, Richard@\textsc{Beer-Hofmann, Richard}!zzzSchnitzler, Arthur@\emph{von Arthur Schnitzler}!1892-03-111@{11. 3. 1892}|(be}
\toendnotes[C]{\smallbreak\pagebreak[2]}
\correspDesc{Versand  durch Arthur Schnitzler am 11. 3. 1892 in Wien
\newline{}Erhalt  durch Richard Beer-Hofmann am 13. 3. 92 in Opatija}\toendnotes[C]{\smallbreak}
\Standort{YCGL, MSS 31.}
\physDesc{Brief, 2 Blätter, 8 Seiten, Kuvert, 2976 Zeichen
\newline{}Handschrift: schwarze Tinte, deutsche Kurrent
\newline{}Versand: 1) Stempel: »\nobreak{}\oindex{I., Innere Stadt@\textbf{I., Innere Stadt}, \emph{Verwaltungsgebiet}|pwk}Wien 1/1, 11 3 92, 7–8 N\nobreak{}«.   2) Stempel: »\nobreak{}\oindex{Opatija@\textbf{Opatija}, \emph{Hauptstadt}|pwk}Abbazia, 13{[}. 3.{]} 92\nobreak{}«. }
\buchAbdrucke{\weitereDrucke{1) Arthur Schnitzler: \emph{Briefe 1875–1912}. Herausgegeben von Therese Nickl und Heinrich Schnitzler. Frankfurt am Main: \emph{S. Fischer} 1981, S. 121–122.} \weitereDrucke{2) Arthur Schnitzler: \emph{Briefe 1875–1912}. Herausgegeben von Therese Nickl und Heinrich Schnitzler. Frankfurt am Main: \emph{S. Fischer} 1981, S. 120–121.} \weitereDrucke{3) Arthur Schnitzler, Richard Beer-Hofmann: \emph{Briefwechsel 1891–1931}. Herausgegeben von Konstanze Fliedl. Wien, Zürich: \emph{Europaverlag} 1992, S. 34–35.} \weitereDrucke{4) Hermann Bahr, Arthur Schnitzler: \emph{Briefwechsel, Aufzeichnungen, Dokumente (1891–1931)}. Herausgegeben von Kurt Ifkovits und Martin Anton Müller. Göttingen: \emph{Wallstein} 2018, S. 22–23.} }\toendnotes[C]{\smallbreak}\pstart{}{\pb}\textcolor{gray}{\textbf{\textit{\label{T_L00079-1v}\edtext{AS}{\lemma{\textnormal{\emph{AS}}}\Cendnote{\textnormal{rotes Wachssiegel}}}\label{T_L00079-1}}}}\pend{}{\bigskip}\pstart{}{\pb}Herrn \textsc{Dr. Rich.
                        Beer-Hofm\damage{a}nn}\pend{}\pstart{}\textsc{Abbazia\oindex{Opatija@\textbf{Opatija}, \emph{Hauptstadt}|pw}}\pend{}\pstart{}\textsc{Pension Quisisana\oindex{Pension Quisisana@\textbf{Pension Quisisana}, \emph{Hotel}|pw}}\pend{}{\bigskip}\vspace{1em}
\pstart
           \raggedleft{}{\pb}Wien\oindex{Wien@\textbf{Wien}, \emph{Verwaltungsgebiet}|pw}, 11. März 92.\pend
           
\pstart{}Lieber Richard,\pend\vspace{0.5em}
\pstart
           Kafka\pwindex{Kafka, Eduard Michael 11.\,3.\,1869 Wien – 6.\,8.\,1893 Brünn@\textsc{Kafka, Eduard Michael} (11.\,3.\,1869 Wien – 6.\,8.\,1893 Brünn), \emph{Redakteur}|pw} habe ich die letzten Tage nicht geſehn.
               Das letzte Mal an unſerem Vereinsabend\orgindex{»Freie Bühne« Verein für moderne Literatur@»Freie Bühne« Verein für moderne Literatur|pwv}, der nur einen Lichtpunkt hatte: Bahr’s\pwindex{Bahr, Hermann 19.\,7.\,1863 Linz – 15.\,1.\,1934 München@\textsc{Bahr, Hermann} (19.\,7.\,1863 Linz – 15.\,1.\,1934 München), \emph{Schriftsteller, Kritiker}|pw} »\label{K_L00079-1v}\edtext{treue Adele\pwindex{Bahr, Hermann 19.\,7.\,1863 Linz – 15.\,1.\,1934 München@\textsc{Bahr, Hermann} (19.\,7.\,1863 Linz – 15.\,1.\,1934 München), \emph{Schriftsteller, Kritiker}!treue Adele. Eine vergeßliche Geschichte@\strich\emph{Die treue Adele. Eine vergeßliche Geschichte}|pw}}{\lemma{\textnormal{\emph{treue Adele}}}\Cendnote{\textnormal{Hermann Bahr\pwindex{Bahr, Hermann 19.\,7.\,1863 Linz – 15.\,1.\,1934 München@\textsc{Bahr, Hermann} (19.\,7.\,1863 Linz – 15.\,1.\,1934 München), \emph{Schriftsteller, Kritiker}|pwk}: \emph{Die treue Adele. Eine vergeßliche Geschichte}\pwindex{Bahr, Hermann 19.\,7.\,1863 Linz – 15.\,1.\,1934 München@\textsc{Bahr, Hermann} (19.\,7.\,1863 Linz – 15.\,1.\,1934 München), \emph{Schriftsteller, Kritiker}!treue Adele. Eine vergeßliche Geschichte@\strich\emph{Die treue Adele. Eine vergeßliche Geschichte}|pwk}. In: \emph{Die Gesellschaft}\pwindex{Gesellschaft. Monatsschrift für Litteratur, Kunst und Sozialpolitik@\emph{Die Gesellschaft. Monatsschrift für Litteratur, Kunst und Sozialpolitik}|pwk}, Jg. 5, Nr. 11,
                        November 1889, S. 1556–1564 (Erstausgabe in \emph{Fin de Siècle}\pwindex{Bahr, Hermann 19.\,7.\,1863 Linz – 15.\,1.\,1934 München@\textsc{Bahr, Hermann} (19.\,7.\,1863 Linz – 15.\,1.\,1934 München), \emph{Schriftsteller, Kritiker}!Fin de Siècle@\strich\emph{Fin de Siècle}|pwk}, S. 71–88).}}}\label{K_L00079-1}« von Bahr\pwindex{Bahr, Hermann 19.\,7.\,1863 Linz – 15.\,1.\,1934 München@\textsc{Bahr, Hermann} (19.\,7.\,1863 Linz – 15.\,1.\,1934 München), \emph{Schriftsteller, Kritiker}|pw} vorgeleſen. Er las entzückend. \textsc{Meixner}\pwindex{Meixner, Julius 15.\,6.\,1850 Tarnów – 3.\,1.\,1913 Bad Vöslau@\textsc{Meixner, Julius} (15.\,6.\,1850 Tarnów – 3.\,1.\,1913 Bad Vöslau), \emph{Schriftsteller, Schauspieler, Pädagoge}|pw} las Parabeln von Kafka\pwindex{Kafka, Eduard Michael 11.\,3.\,1869 Wien – 6.\,8.\,1893 Brünn@\textsc{Kafka, Eduard Michael} (11.\,3.\,1869 Wien – 6.\,8.\,1893 Brünn), \emph{Redakteur}|pw} und ein Gedicht
                  \textsc{Liliencron}\pwindex{Liliencron, Detlev von 3.\,6.\,1844 Kiel – 22.\,7.\,1909 Rahlstedt@\textsc{Liliencron, Detlev von} (3.\,6.\,1844 Kiel – 22.\,7.\,1909 Rahlstedt), \emph{Schriftsteller, Dichter, Dramatiker}|pw}{ }ſehr{ }ſchlecht vor. \textsc{Polland}\pwindex{Pollandt, Max 26.\,10.\,1861 Wien – 18.\,7.\,1905 Pernitz@\textsc{Pollandt, Max} (26.\,10.\,1861 Wien – 18.\,7.\,1905 Pernitz), \emph{Schauspieler}|pw} das Kaffehaus\pwindex{Salten, Felix 6.\,9.\,1869 Budapest – 8.\,10.\,1945 Zürich@\textsc{Salten, Felix} (6.\,9.\,1869 Budapest – 8.\,10.\,1945 Zürich), \emph{Schriftsteller, Journalist, Chefredakteur}!Kaffeehaus]@\strich\emph{[Das Kaffeehaus]}|pw} von \textsc{Salten}\pwindex{Salten, Felix 6.\,9.\,1869 Budapest – 8.\,10.\,1945 Zürich@\textsc{Salten, Felix} (6.\,9.\,1869 Budapest – 8.\,10.\,1945 Zürich), \emph{Schriftsteller, Journalist, Chefredakteur}|pw}, Gedichte von \textsc{Loris}\pwindex{Hofmannsthal, Hugo von 1.\,2.\,1874 Wien – 15.\,7.\,1929 Rodaun@\textsc{Hofmannsthal, Hugo von} (1.\,2.\,1874 Wien – 15.\,7.\,1929 Rodaun), \emph{Schriftsteller}|pw}, Korff\pwindex{Korff, Heinrich von 5.\,6.\,1868 Wien – 18.\,8.\,1938 Berlin@\textsc{Korff, Heinrich von} (5.\,6.\,1868 Wien – 18.\,8.\,1938 Berlin), \emph{Journalist}|pw} u mir unbeſchreiblich
               entſetzlich. Es iſt unmöglich,{ }ſich von dieſer talentloſen Brüllerei einen Begriff zu
               machen, we{\geminationn} man nicht dabei {\pb}war. – Zum Schluſs wurde getanzt. Von mir nicht,
               bitte. –\pend
           
\pstart
           \textsc{Blumenthal}\pwindex{Blumenthal, Oskar 13.\,3.\,1852 Berlin – 24.\,4.\,1917 ebd.@\textsc{Blumenthal, Oskar} (13.\,3.\,1852 Berlin – 24.\,4.\,1917 ebd.), \emph{Schriftsteller, Journalist, Theaterleiter}|pw} war hier, ich{ }ſprach ihn. Er will Kürzungen und einige Aenderungen am Mährchen\pwindex{Schnitzler, Arthur 15.\,5.\,1862 Wien – 21.\,10.\,1931 ebd.@\textsc{Schnitzler, Arthur} (15.\,5.\,1862 Wien – 21.\,10.\,1931 ebd.), \emph{Schriftsteller, Mediziner}!Märchen. Schauspiel in drei Aufzügen@\strich\emph{Das Märchen. Schauspiel in drei Aufzügen}|pw}. Einiges wird{ }ſich wohl thun laſſen; ich
               habe mich{ }ſchon daran gemacht, und die{ }ſchöne Fremdheit, die mich vom Märchen\pwindex{Schnitzler, Arthur 15.\,5.\,1862 Wien – 21.\,10.\,1931 ebd.@\textsc{Schnitzler, Arthur} (15.\,5.\,1862 Wien – 21.\,10.\,1931 ebd.), \emph{Schriftsteller, Mediziner}!Märchen. Schauspiel in drei Aufzügen@\strich\emph{Das Märchen. Schauspiel in drei Aufzügen}|pw} bereits tre{\geminationn}t, läßt mich die Dinge leichter vollbringen. Daß \textsc{Blumenthal}\pwindex{Blumenthal, Oskar 13.\,3.\,1852 Berlin – 24.\,4.\,1917 ebd.@\textsc{Blumenthal, Oskar} (13.\,3.\,1852 Berlin – 24.\,4.\,1917 ebd.), \emph{Schriftsteller, Journalist, Theaterleiter}|pw} auch den Titel des Stückes geändert haben möchte, iſt Caeſarenwahnſinn. Es iſt
               ihm auch{ }ſchon{ }ſelbſt ein neuer eingefallen – er{\pb}ſchrecken Sie nicht – »Die Vergangenheit.« Erke{\geminationn}en Sie
               ihn!? Und noch i{\geminationm}er läßt man die erſt- und zweitgradigen
               frei herum laufen, die doch nur dazu da{ }ſind, um den dritt und viertgradigen das
               Leben zu vermießen. –\pend
           
\pstart
           Geſtern hab ich mein neues Stück\pwindex{Schnitzler, Arthur 15.\,5.\,1862 Wien – 21.\,10.\,1931 ebd.@\textsc{Schnitzler, Arthur} (15.\,5.\,1862 Wien – 21.\,10.\,1931 ebd.), \emph{Schriftsteller, Mediziner}!Familie@\strich\emph{Familie}|pwv} begonnen. Außerdem schreibe ich \textsc{slowly},
               langſam an meiner Novelle\pwindex{Schnitzler, Arthur 15.\,5.\,1862 Wien – 21.\,10.\,1931 ebd.@\textsc{Schnitzler, Arthur} (15.\,5.\,1862 Wien – 21.\,10.\,1931 ebd.), \emph{Schriftsteller, Mediziner}!Sterben. Novelle@\strich\emph{Sterben. Novelle}|pwv}. –\pend
           
\pstart
           \textsc{Fontane} (Verlag)\orgindex{F. Fontane@F. Fontane|pw} hat mich freundlichſt erſucht, den
                  \textsc{Anatol-Cyclus}\pwindex{Schnitzler, Arthur 15.\,5.\,1862 Wien – 21.\,10.\,1931 ebd.@\textsc{Schnitzler, Arthur} (15.\,5.\,1862 Wien – 21.\,10.\,1931 ebd.), \emph{Schriftsteller, Mediziner}!Anatol@\strich\emph{Anatol}|pw} – \uline{nicht} einzuſenden, {\pb}»da{ }ſie kaum die Zeit finden dürften, meiner Sa{\geminationm}lung einen{ }ſorgfältigen u energiſchen Vertrieb
               angedeihen zu laſſen \textsc{etc etc}«\pend
           
\pstart
           – Aus den »\textsc{Aveugles\pwindex{\textcolor{red}{\textsuperscript{XXXX indx1}}!Blinden@\strich\emph{Die Blinden}|pw}}«{ }ſcheint wirklich was zu werden. Doch{ }ſoll dazu weder Pantomime noch Abschiedsſouper\pwindex{Schnitzler, Arthur 15.\,5.\,1862 Wien – 21.\,10.\,1931 ebd.@\textsc{Schnitzler, Arthur} (15.\,5.\,1862 Wien – 21.\,10.\,1931 ebd.), \emph{Schriftsteller, Mediziner}!Abschiedssouper@\strich\emph{Abschiedssouper}|pw} gegeben werden,{ }ſondern »\textsc{l’Intrus\pwindex{\textcolor{red}{\textsuperscript{XXXX indx1}}!Intruse@\strich\emph{L’Intruse}|pw}}«. – Zu den beiden ein Vortrag von \textsc{Bahr}\pwindex{Bahr, Hermann 19.\,7.\,1863 Linz – 15.\,1.\,1934 München@\textsc{Bahr, Hermann} (19.\,7.\,1863 Linz – 15.\,1.\,1934 München), \emph{Schriftsteller, Kritiker}|pw}. Später{ }ſoll ein Pantomimen u Luſtſpielabend arrangirt werden. Man kam mit dem
                  \label{K_L00079-2v}\edtext{\textsc{fait accompli}}{\lemma{\textnormal{\emph{fait accompli}}}\Cendnote{\textnormal{französisch: beschlossene Sache}}}\label{K_L00079-2}
               zu uns, das {\pb}freilich meinen Beifall nicht hat. –\pend
           
\pstart
           \textsc{Loris}\pwindex{Hofmannsthal, Hugo von 1.\,2.\,1874 Wien – 15.\,7.\,1929 Rodaun@\textsc{Hofmannsthal, Hugo von} (1.\,2.\,1874 Wien – 15.\,7.\,1929 Rodaun), \emph{Schriftsteller}|pw}{ }ſchreibt viel, \textsc{Salten}\pwindex{Salten, Felix 6.\,9.\,1869 Budapest – 8.\,10.\,1945 Zürich@\textsc{Salten, Felix} (6.\,9.\,1869 Budapest – 8.\,10.\,1945 Zürich), \emph{Schriftsteller, Journalist, Chefredakteur}|pw}{ }ſchreibt wenig. Die andern{ }ſeh ich gar nicht; das
                  \textsc{Café Griensteidl}\oindex{Wien@\textbf{Wien}!I., Innere Stadt@\textbf{I., Innere Stadt}!Café Griensteidl@\textbf{Café Griensteidl}, \emph{Kaffeehaus}|pw} exiſtirt für mich nicht mehr. –\pend
           
\pstart
           Ich leſe \textsc{Taine}\pwindex{Taine, Hippolyte 21.\,4.\,1828 Vouziers – 5.\,3.\,1893 Paris@\textsc{Taine, Hippolyte} (21.\,4.\,1828 Vouziers – 5.\,3.\,1893 Paris), \emph{Philosoph, Geschichtsschreiber}|pw}, \textsc{ancien régime}\pwindex{Taine, Hippolyte 21.\,4.\,1828 Vouziers – 5.\,3.\,1893 Paris@\textsc{Taine, Hippolyte} (21.\,4.\,1828 Vouziers – 5.\,3.\,1893 Paris), \emph{Philosoph, Geschichtsschreiber}!Ancien régime@\strich\emph{L’Ancien régime}|pw}, \textsc{Du Prel}\pwindex{Du Prel, Carl 3.\,4.\,1839 Landshut – 5.\,8.\,1899 Hall in Tirol@\textsc{Du Prel, Carl} (3.\,4.\,1839 Landshut – 5.\,8.\,1899 Hall in Tirol), \emph{Schriftsteller, Philosoph}|pw}, Philoſophie der Myſtik\pwindex{Du Prel, Carl 3.\,4.\,1839 Landshut – 5.\,8.\,1899 Hall in Tirol@\textsc{Du Prel, Carl} (3.\,4.\,1839 Landshut – 5.\,8.\,1899 Hall in Tirol), \emph{Schriftsteller, Philosoph}!Philosophie der Mystik@\strich\emph{Die Philosophie der Mystik}|pw}, \textsc{Restif de la Bretonne}\pwindex{Rétif de la Bretonne, Nicolas 23.\,10.\,1734 Sacy – 3.\,2.\,1806 Paris@\textsc{Rétif de la Bretonne, Nicolas} (23.\,10.\,1734 Sacy – 3.\,2.\,1806 Paris), \emph{Schriftsteller}|pw}, \textsc{l’amour à 45 ans}\pwindex{Rétif de la Bretonne, Nicolas 23.\,10.\,1734 Sacy – 3.\,2.\,1806 Paris@\textsc{Rétif de la Bretonne, Nicolas} (23.\,10.\,1734 Sacy – 3.\,2.\,1806 Paris), \emph{Schriftsteller}!Sara, ou L’amour à quarante-cinq ans@\strich\emph{Sara, ou L’amour à quarante-cinq ans}|pw}, \textsc{Kretzer}\pwindex{Kretzer, Max 7.\,6.\,1854 Poznan – 1941 Berlin@\textsc{Kretzer, Max} (7.\,6.\,1854 Poznan – 1941 Berlin), \emph{Schriftsteller}|pw}, die Betrogenen\pwindex{Kretzer, Max 7.\,6.\,1854 Poznan – 1941 Berlin@\textsc{Kretzer, Max} (7.\,6.\,1854 Poznan – 1941 Berlin), \emph{Schriftsteller}!Betrogenen@\strich\emph{Die Betrogenen}|pw} u. a. –\pend
           
\pstart
           Die Menſchen \textsc{enerviren} mich. Manche miſchen{ }ſich in meine
               Privatangelegenheiten, und nie{\pb}manden gehen{ }ſie an.
               Das Geſindel hat tauſend Augen für Vorfälle, dafür taube Ohren für Einfälle. Aber mit
               der Zeit wird{ }ſich die Menſchheit wohl »ausſchalten« laſſen, wie? Einen Harfeniſten
                  ka{\geminationn} man aus dem Hofe weiſen laſſen, we{\geminationn} er einen mit{ }ſeinem Geklimper quält; wer aber befreit
               mich von den – andern?\pend
           
\pstart
           Ich will verſuchen, ein Virtuoſe der Einſamkeit zu werden. Eines{ }ſchönen Tages werden
               alle Leute, die mich geniren, {\pb}nicht mehr daſein – und
               werden es nicht einmal bemerken. So wollen wir die Unbequemen zu relativem Tod
               verurtheilen: wir vom »großen Orden«! – Oder hätte Sie \textsc{Salten}\pwindex{Salten, Felix 6.\,9.\,1869 Budapest – 8.\,10.\,1945 Zürich@\textsc{Salten, Felix} (6.\,9.\,1869 Budapest – 8.\,10.\,1945 Zürich), \emph{Schriftsteller, Journalist, Chefredakteur}|pw} abreiſen laſſen, ohne Ihnen den großen Orden zu erläutern? –\pend
           
\pstart
           Schreiben Sie mir bald, und möglichſt viel, es muſs doch ganz{ }ſchön{ }ſein, we{\geminationn} man einmal wo anders iſt. Und dann,{ }ſchreiben Sie –
               wir erwarten es, wir – vom großen Orden. –\pend
           
\pstart
           {\pb}Herzlichſt Ihr{\\[\baselineskip]}\spacefill\mbox{Arthur Sch}\pend
           \leftskip=0em{}\selectlanguage{ngerman}\endnumbering\briefempfaengerindex{Beer-Hofmann, Richard@\textsc{Beer-Hofmann, Richard}!zzzSchnitzler, Arthur@\emph{von Arthur Schnitzler}!1892-03-111@{11. 3. 1892}|)be}\mylabel{L00079h}  \newcommand{\dateiname}{L00079}\newcommand{\titel}{Arthur Schnitzler an Richard Beer-Hofmann, 11. 3. 1892}\newcommand{\editorInnen}{Herausgegeben von Martin Anton Müller}%% latex-leseansicht-abspann.tex
%% Abspann für die Leseansicht.
%% Der Schalter \ifkorrekturansicht ist bereits durch den Vorspann gesetzt.

%% latex-abspann.tex
%% Gemeinsamer Abspann für Korrekturansicht und Leseansicht.
%% Setzt den Schalter \ifkorrekturansicht voraus (gesetzt in den
%% einbindenden Dateien latex-korrekturansicht-abspann.tex bzw.
%% latex-leseansicht-abspann.tex).
%% ---------------------------------------------------------------

\normalsize

% Das esempio-Environment wird nur in der Leseansicht benötigt
\ifkorrekturansicht\else
\newenvironment{esempio}[3]%
{
    \vspace{1.5ex}
    \rlap{\underline{#1}}
    \par
    \setlength{\parindent}{0cm}
    \nopagebreak
    \leftskip=#2cm
    \rightskip=#3cm
}
{
    \par
}
\fi

\doendnotes{C}
\bigskip
\vfill

\clearpage

\footnotesize

\ifkorrekturansicht
  \lohead{\textsc{register}}
\fi

% theindex-Environment neu definieren ohne reledmac
\makeatletter
\renewenvironment{theindex}{%
  \ifkorrekturansicht
    \section*{\indexname}%
  \else
    \subsubsection*{Index der erwähnten Entitäten}%
  \fi
  \setlength{\parindent}{0pt}%
  \setlength{\parskip}{0pt plus 0.3pt}%
  \let\item\@idxitem
}{%
  \ifkorrekturansicht\clearpage\fi
}
\makeatother

\IfFileExists{\jobname-pw.ind}{\input{\jobname-pw.ind}}{}

% Quellenangabe nur in der Leseansicht
\ifkorrekturansicht\else
% Fallback-Definitionen, falls die .tex-Datei \titel etc. nicht gesetzt hat
\providecommand{\titel}{}
\providecommand{\editorInnen}{}
\providecommand{\dateiname}{\jobname}

\vspace{3cm}

\vfill

\footnotesize
\textsc{Quelle}: \titel. Herausgegeben von {\editorInnen}. In: \emph{Arthur Schnitzler: Briefwechsel mit Autorinnen und Autoren}.
 Digitale Edition, https://schnitzler-briefe.acdh.oeaw.ac.at/{\dateiname}.html (Stand \today)
\fi

\end{document}


