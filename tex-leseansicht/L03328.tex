%% latex-leseansicht-vorspann.tex
%% Vorspann für die Leseansicht.
%% Lädt die gemeinsame Datei latex-vorspann.tex mit nicht gesetztem Schalter.

\newif\ifkorrekturansicht
\korrekturansichtfalse

\input{../tex-inputs/latex-vorspann}


         
         \renewcommand{\erwaehntePersonen}{Personen: Felix Salten, Ottilie Salten, Siegfried Trebitsch}
         \renewcommand{\erwaehnteOrte}{Orte: Wien}
         \renewcommand{\erwaehnteWerke}{}
               \section[ Felix Salten an Arthur Schnitzler, {[}11?. 4. 1902{]}]{ Felix Salten an Arthur Schnitzler, {[}11?. 4. 1902{]}}\nopagebreak\mylabel{v}\rehead{ }\begin{ledgroupsized}[t]{13cm}\normalsize\beginnumbering\briefempfaengerindex{Schnitzler, Arthur@\textsc{Schnitzler, Arthur}!zzzSalten, Felix@\emph{von Felix Salten}!1902-04-111@{{[}11?. 4. 1902{]}}|(be} \toendnotes[C]{\smallbreak\pagebreak[2]} \Standort{CUL, Schnitzler, B 89, A 2.}
\physDesc{Brief, 1 Blatt, 2 Seiten, 387 Zeichen
\newline{}Handschrift: Bleistift, lateinische Kurrent
\newline{}Schnitzler: mit Bleistift datiert: »10. 4. 902« 
\newline{}Ordnung: mit Bleistift von unbekannter Hand nummeriert: »152« }\toendnotes[C]{\smallbreak}\pstart
           \noindent{}{\pb}Lieber Freund, also \uline{doch}{ }\label{K_L03328-1v}\edtext{Sonntag}{\lemma{\textnormal{\emph{Sonntag}}}\Cendnote{\textnormal{Am Sonntag, dem 13. 4. 1902, fand die Hochzeit von Ottilie Metzl\pwindex{Salten, Ottilie 07.03.1868 – 22.06.1942@\textsc{Salten, Ottilie} (07.03.1868 – 22.06.1942), \emph{Schauspielerin}|pwk} und Felix
                        Salten\pwindex{Salten, Felix 06.09.1869 – 08.10.1945@\textsc{Salten, Felix} (06.09.1869 – 08.10.1945), \emph{Schriftsteller, Journalist}|pwk} statt. Die Trauzeugen waren Schnitzler\pwindex{Schnitzler, Arthur 15.05.1862 – 21.10.1931@\textsc{Schnitzler, Arthur} (15.05.1862 – 21.10.1931), \emph{Schriftsteller, Mediziner}|pwk} und Siegfried
                     Trebitsch\pwindex{Trebitsch, Siegfried 22.12.1868 – 03.06.1956@\textsc{Trebitsch, Siegfried} (22.12.1868 – 03.06.1956), \emph{Schriftsteller, Übersetzer}|pwk}.}}}\label{K_L03328-1h}. Könnten Sie dabei sein, wäre es
               mir \uline{sehr} lieb, u. a. auch deswegen, weil ich es sonst
               Niemandem anzeigen will, nicht einmal in meiner Familie. Wäre also sehr dankbar, wenn
               Sie Sonntag um 5\textsuperscript{h} zu mir kämen. herzlichst {\\}\spacefill\mbox{Salten}\pend
           \pstart
           \noindent{}Holen Sie mich bitte \label{K_L03328-2v}\edtext{morgen{ }N. M.}{\lemma{\textnormal{\emph{morgen N. M.}}}\Cendnote{\textnormal{Das deutet darauf hin, dass sich Schnitzler\pwindex{Schnitzler, Arthur 15.05.1862 – 21.10.1931@\textsc{Schnitzler, Arthur} (15.05.1862 – 21.10.1931), \emph{Schriftsteller, Mediziner}|pwk} bei seiner Datierung um einen
                     Tag vertan hat. Salten\pwindex{Salten, Felix 06.09.1869 – 08.10.1945@\textsc{Salten, Felix} (06.09.1869 – 08.10.1945), \emph{Schriftsteller, Journalist}|pwk} wusste, dass die
                     Impfung am Samstag, dem 12. 4. 1902, stattfinden sollte (vgl. Arthur Schnitzler an Felix Salten, [10. 4. 1902]).}}}\label{K_L03328-2h} zum {\pb}\label{K_L03328-3v}\edtext{Impfen}{\lemma{\textnormal{\emph{Impfen}}}\Cendnote{\textnormal{siehe A. S.: \emph{Tagebuch}, 12. 4. 1902}}}\label{K_L03328-3h} ab? Und sind Sie \label{K_L03328-4v}\edtext{heut{ }Abend im Caféhaus}{\lemma{\textnormal{\emph{heut Abend im Caféhaus}}}\Cendnote{\textnormal{nicht
                     nachweisbar}}}\label{K_L03328-4h}? Wenn ja, senden Sie mir ein Wort, sonst geh ich garnicht
                  hin.\pend
           
         
         \endnumbering\mylabel{h}\end{ledgroupsized}  \newcommand{\dateiname}{L03328}\newcommand{\titel}{Felix Salten an Arthur Schnitzler, [11?. 4. 1902]}\newcommand{\editorInnen}{Martin Anton Müller und Laura Untner}%% latex-leseansicht-abspann.tex
%% Abspann für die Leseansicht.
%% Der Schalter \ifkorrekturansicht ist bereits durch den Vorspann gesetzt.

%% latex-abspann.tex
%% Gemeinsamer Abspann für Korrekturansicht und Leseansicht.
%% Setzt den Schalter \ifkorrekturansicht voraus (gesetzt in den
%% einbindenden Dateien latex-korrekturansicht-abspann.tex bzw.
%% latex-leseansicht-abspann.tex).
%% ---------------------------------------------------------------

\normalsize

% Das esempio-Environment wird nur in der Leseansicht benötigt
\ifkorrekturansicht\else
\newenvironment{esempio}[3]%
{
    \vspace{1.5ex}
    \rlap{\underline{#1}}
    \par
    \setlength{\parindent}{0cm}
    \nopagebreak
    \leftskip=#2cm
    \rightskip=#3cm
}
{
    \par
}
\fi

\doendnotes{C}
\bigskip
\vfill

\clearpage

\footnotesize

\ifkorrekturansicht
  \lohead{\textsc{register}}
\fi

% theindex-Environment neu definieren ohne reledmac
\makeatletter
\renewenvironment{theindex}{%
  \ifkorrekturansicht
    \section*{\indexname}%
  \else
    \subsubsection*{Index der erwähnten Entitäten}%
  \fi
  \setlength{\parindent}{0pt}%
  \setlength{\parskip}{0pt plus 0.3pt}%
  \let\item\@idxitem
}{%
  \ifkorrekturansicht\clearpage\fi
}
\makeatother

\IfFileExists{\jobname-pw.ind}{\input{\jobname-pw.ind}}{}

% Quellenangabe nur in der Leseansicht
\ifkorrekturansicht\else
% Fallback-Definitionen, falls die .tex-Datei \titel etc. nicht gesetzt hat
\providecommand{\titel}{}
\providecommand{\editorInnen}{}
\providecommand{\dateiname}{\jobname}

\vspace{3cm}

\vfill

\footnotesize
\textsc{Quelle}: \titel. Herausgegeben von {\editorInnen}. In: \emph{Arthur Schnitzler: Briefwechsel mit Autorinnen und Autoren}.
 Digitale Edition, https://schnitzler-briefe.acdh.oeaw.ac.at/{\dateiname}.html (Stand \today)
\fi

\end{document}


      