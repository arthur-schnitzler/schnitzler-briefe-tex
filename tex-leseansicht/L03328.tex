%% latex-korrekturansicht-vorspann.tex
%% Vorspann für die Korrekturansicht.
%% Lädt die gemeinsame Datei latex-vorspann.tex mit gesetztem Schalter.

\newif\ifkorrekturansicht
\korrekturansichttrue

\input{../tex-inputs/latex-vorspann}


\section[ Felix Salten an Arthur Schnitzler, {[}11?. 4. 1902{]}]{L03328 Felix Salten an Arthur Schnitzler, {[}11?. 4. 1902{]}}
\nopagebreak\mylabel{L03328v}
\rehead{ }\normalsize\beginnumbering\briefempfaengerindex{Schnitzler, Arthur@\textsc{Schnitzler, Arthur}!zzzSalten, Felix@\emph{von Felix Salten}!1902-04-111@{{[}11?. 4. 1902{]}}|(be}
\toendnotes[C]{\smallbreak\pagebreak[2]}\Standort{CUL, Schnitzler, B 89, A 2.}
\physDesc{Brief, 1 Blatt, 2 Seiten, 387 Zeichen
\newline{}Handschrift: Bleistift, lateinische Kurrent
\newline{}Schnitzler: mit Bleistift datiert: »10. 4. 902« 
\newline{}Ordnung: mit Bleistift von unbekannter Hand nummeriert: »152« }\toendnotes[C]{\smallbreak}
\pstart
           \noindent{}{\pb}Lieber Freund, also \uline{doch}{ }\label{K_L03328-1v}\edtext{Sonntag}{\lemma{\textnormal{\emph{Sonntag}}}\Cendnote{\textnormal{Am Sonntag, dem 13. 4. 1902 fand die Hochzeit von Ottilie Metzl\pwindex{Salten, Ottilie 07.03.1868 – 22.06.1942@\textsc{Salten, Ottilie} (07.03.1868 – 22.06.1942), \emph{Schauspieler/Schauspielerin}|pwk} und Felix
                        Salten\pwindex{Salten, Felix 06.09.1869 – 08.10.1945@\textsc{Salten, Felix} (06.09.1869 – 08.10.1945), \emph{Schriftsteller/Schriftstellerin, Journalist/Journalistin, Chefredakteur/Chefredakteurin}|pwk} statt. Die Trauzeugen waren Schnitzler und Siegfried
                     Trebitsch\pwindex{Trebitsch, Siegfried 22.12.1868 – 03.06.1956@\textsc{Trebitsch, Siegfried} (22.12.1868 – 03.06.1956), \emph{Schriftsteller/Schriftstellerin, Übersetzer/Übersetzerin}|pwk}.}}}\label{K_L03328-1}. Könnten Sie dabei sein, wäre es
               mir \uline{sehr} lieb, u. a. auch deswegen, weil ich es sonst
               Niemandem anzeigen will, nicht einmal in meiner Familie. Wäre also sehr dankbar, wenn
               Sie Sonntag um 5\textsuperscript{h} zu mir kämen. herzlichst {\\}\spacefill\mbox{Salten}\pend
           
\pstart
           \noindent{}Holen Sie mich bitte \label{K_L03328-2v}\edtext{morgen{ }N. M.}{\lemma{\textnormal{\emph{morgen N. M.}}}\Cendnote{\textnormal{Das deutet darauf hin, dass sich Schnitzler bei seiner Datierung um einen
                     Tag vertan hat. Salten\pwindex{Salten, Felix 06.09.1869 – 08.10.1945@\textsc{Salten, Felix} (06.09.1869 – 08.10.1945), \emph{Schriftsteller/Schriftstellerin, Journalist/Journalistin, Chefredakteur/Chefredakteurin}|pwk} wusste, dass die
                     Impfung am Samstag, dem 12. 4. 1902 stattfinden sollte (vgl. Arthur Schnitzler an Felix Salten, [10. 4. 1902]).}}}\label{K_L03328-2} zum {\pb}\label{K_L03328-3v}\edtext{Impfen}{\lemma{\textnormal{\emph{Impfen}}}\Cendnote{\textnormal{Siehe A. S.: \emph{Tagebuch}, 12. 4. 1902.
                  }}}\label{K_L03328-3} ab? Und sind Sie \label{K_L03328-4v}\edtext{heut{ }Abend im Caféhaus}{\lemma{\textnormal{\emph{heut Abend im Caféhaus}}}\Cendnote{\textnormal{nicht
                     nachweisbar}}}\label{K_L03328-4}? Wenn ja, senden Sie mir ein Wort, sonst geh ich garnicht
                  hin.\pend
           \selectlanguage{ngerman}\endnumbering\briefempfaengerindex{Schnitzler, Arthur@\textsc{Schnitzler, Arthur}!zzzSalten, Felix@\emph{von Felix Salten}!1902-04-111@{{[}11?. 4. 1902{]}}|)be}\mylabel{L03328h}  \normalsize

\doendnotes{C}
\bigskip
\vfill

\clearpage

\footnotesize

\lohead{\textsc{register}}

% Definiere theindex-Environment komplett neu ohne reledmac
\makeatletter
\renewenvironment{theindex}{%
  \section*{\indexname}%
  \setlength{\parindent}{0pt}%
  \setlength{\parskip}{0pt plus 0.3pt}%
  \let\item\@idxitem
}{%
  \clearpage
}
\makeatother

\IfFileExists{\jobname-pw.ind}{\input{\jobname-pw.ind}}{}

\end{document}

      