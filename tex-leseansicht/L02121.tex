%% latex-korrekturansicht-vorspann.tex
%% Vorspann für die Korrekturansicht.
%% Lädt die gemeinsame Datei latex-vorspann.tex mit gesetztem Schalter.

\newif\ifkorrekturansicht
\korrekturansichttrue

\input{../tex-inputs/latex-vorspann}


\section[Hermann Bahr an Arthur Schnitzler, 16. 4. 1913]{L02121 Hermann Bahr an Arthur Schnitzler, 16. 4. 1913}
\nopagebreak\mylabel{L02121v}
\rehead{ }\normalsize\beginnumbering\briefempfaengerindex{Schnitzler, Arthur@\textsc{Schnitzler, Arthur}!zzzBahr, Hermann@\emph{von Hermann Bahr}!1913-04-161@{16. 4. 1913}|(be}
\toendnotes[C]{\smallbreak\pagebreak[2]}\Standort{CUL, Schnitzler, B 5b.}
\physDesc{Kartenbrief, 914 Zeichen
\newline{}Handschrift: schwarze Tinte, deutsche Kurrent
\newline{}Versand: Stempel: »\nobreak{}\oindex{Salzburg@\textbf{Salzburg}, \emph{A.ADM2}|pwk}Sa{[}lzburg{]}, 16. IV. 13, 10\nobreak{}«.  
\newline{}Schnitzler: mit Bleistift ergänzt »Bahr« 
\newline{}Ordnung: mit Bleistift von unbekannter Hand nummeriert:
                                    »176« }
\buchAbdrucke{\weitereDrucke{Hermann Bahr, Arthur Schnitzler: \emph{Briefwechsel, Aufzeichnungen, Dokumente (1891–1931)}. Göttingen: \emph{Wallstein} 2018, S. 482.} }\toendnotes[C]{\smallbreak}\pstart{}{\pb}Abſ. \textsc{Hermann Bahr}\pend{}\pstart{}\textsc{Salzburg\oindex{Salzburg@\textbf{Salzburg}, \emph{A.ADM2}|pw}}\pend{}{\bigskip}\pstart{} Herrn \textsc{D\textsuperscript{r} Arthur
                     Schnitzler}\pend{}\pstart{}\textsc{Wien XVIII\oindex{XVIII., Waehring@\textbf{XVIII., Währing}, \emph{A.ADM3}|pw}}\pend{}\pstart{}Sternwarteſtraße 71\oindex{Sternwartestrasse 71@\textbf{Sternwartestraße 71}, \emph{Wohngebäude (K.WHS)}|pw}\pend{}{\bigskip}\vspace{1em}
\pstart
           \raggedleft{}{\pb}Salzburg\oindex{Salzburg@\textbf{Salzburg}, \emph{A.ADM2}|pw}{ }16. 4. 13\pend
           \vspace{0.5em}
\pstart
           Lieber Arthur! Ich erhielt eben einen etwas verworrenen Brief Peter Altenbergs\pwindex{Altenberg, Peter 09.03.1859 – 08.01.1919@\textsc{Altenberg, Peter} (09.03.1859 – 08.01.1919), \emph{Schriftsteller/Schriftstellerin}|pw}, worin er mich anfleht, ihn zu
               retten, der im Steinhof\oindex{Otto-Wagner-Spital@\textbf{Otto-Wagner-Spital}, \emph{Krankenhaus (K.KKH)}|pw} »wie ein giftiges
               irrſinniges Tier« behandelt und zu Tod gequält werde. Es iſt möglich, daß das
               »Einbildungen« ſind. Es iſt ebenſo möglich, daß es wahr iſt. Ich weiß gar nicht, was
               ich von hier aus tun soll, und weiß auch nicht, wie ich mir, in Wien\oindex{Wien@\textbf{Wien}, \emph{A.ADM2}|pw} angekommen, den Eintritt im Steinhof\oindex{Otto-Wagner-Spital@\textbf{Otto-Wagner-Spital}, \emph{Krankenhaus (K.KKH)}|pw} erzwingen könnte. Du biſt »Arzt«, Du wirſt eher wiſſen, ob und wie
               man helfen könnte. Willſt Du Dich der Sache annehmen? Und mir dann ſagen, ob Du
               glaubſt, daß ich was tun kann? Ich bin natürlich gern zu allem bereit – Mordsſkandal
               in der Öffentlichkeit oder auch gewaltſame Entführung, die ja mit Geld dort leicht zu
               bewerkſtelligen ſein wird. Bitte ſchreib bald\pend
           
\pstart
           Deinem alten{\\[\baselineskip]}\spacefill\mbox{Hermann}\pend
           \leftskip=0em{}
\pstart
           \noindent{}Grüße an Olga\pwindex{Schnitzler, Olga 17.01.1882 – 13.01.1970@\textsc{Schnitzler, Olga} (17.01.1882 – 13.01.1970), \emph{Schauspieler/Schauspielerin, Sänger/Sängerin}|pw} u die Kinder\pwindex{Schnitzler, Heinrich 09.08.1902 – 12.07.1982@\textsc{Schnitzler, Heinrich} (09.08.1902 – 12.07.1982), \emph{Regisseur/Regisseurin, Schauspieler/Schauspielerin}|pwv}\pwindex{Cappellini, Lili 13.09.1909 – 26.07.1928@\textsc{Cappellini, Lili} (13.09.1909 – 26.07.1928)|pwv}!\pend
           \selectlanguage{ngerman}\endnumbering\briefempfaengerindex{Schnitzler, Arthur@\textsc{Schnitzler, Arthur}!zzzBahr, Hermann@\emph{von Hermann Bahr}!1913-04-161@{16. 4. 1913}|)be}\mylabel{L02121h}  \normalsize

\doendnotes{C}
\bigskip
\vfill

\clearpage

\footnotesize

\lohead{\textsc{register}}

% Definiere theindex-Environment komplett neu ohne reledmac
\makeatletter
\renewenvironment{theindex}{%
  \section*{\indexname}%
  \setlength{\parindent}{0pt}%
  \setlength{\parskip}{0pt plus 0.3pt}%
  \let\item\@idxitem
}{%
  \clearpage
}
\makeatother

\IfFileExists{\jobname-pw.ind}{\input{\jobname-pw.ind}}{}

\end{document}

      