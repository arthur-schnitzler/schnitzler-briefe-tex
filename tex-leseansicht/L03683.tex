%% latex-korrekturansicht-vorspann.tex
%% Vorspann für die Korrekturansicht.
%% Lädt die gemeinsame Datei latex-vorspann.tex mit gesetztem Schalter.

\newif\ifkorrekturansicht
\korrekturansichttrue

\input{../tex-inputs/latex-vorspann}


\section[Stefan Zweig an Arthur Schnitzler, 3. 12. 1914]{L03683 Stefan Zweig an Arthur Schnitzler, 3. 12. 1914}
\nopagebreak\mylabel{L03683v}
\rehead{ }\normalsize\beginnumbering\briefempfaengerindex{Schnitzler, Arthur@\textsc{Schnitzler, Arthur}!zzzZweig, Stefan@\emph{von Stefan Zweig}!1914-12-031@{3. 12. 1914}|(be}
\toendnotes[C]{\smallbreak\pagebreak[2]}\Standort{CUL, Schnitzler, B 118.}
\physDesc{Brief, 1 Blatt, 2 Seiten, 2236 Zeichen
\newline{}Schreibmaschine
\newline{}Handschrift: blaue Tinte, lateinische Kurrent (\noindent{}Korrekturen, Unterschrift und Postskriptum)
\newline{}Schnitzler: 1) mit rotem Buntstift drei Unterstreichungen  2) mit Bleistift beschriftet: »\textsc{Zweig}«}
\buchAbdrucke{\weitereDrucke{Stefan Zweig: \emph{Briefwechsel mit Hermann Bahr, Sigmund Freud, Rainer Maria
                        Rilke und Arthur Schnitzler}. Frankfurt am Main: \emph{S. Fischer} 1987, S. 386–388.} }\toendnotes[C]{\smallbreak}
\pstart
           {\pb}\textcolor{gray}{\textbf{SZ}}\hfill \textcolor{gray}{\textbf{VIII. KOCHGASSE 8\oindex{Kochgasse 8@\textbf{Kochgasse 8}, \emph{Wohngebäude (K.WHS)}|pw}}}\pend
           
\pstart
           \raggedleft{}\textcolor{gray}{\textbf{WIEN\oindex{Wien@\textbf{Wien}, \emph{A.ADM2}|pw},}}\pend
           
\pstart
           \raggedleft{}Wien\oindex{Wien@\textbf{Wien}, \emph{A.ADM2}|pw}, 3. Dezember 14\pend
           
\pstart{}Sehr verehrter lieber Herr Doktor\pend\vspace{0.5em}
\pstart
           Ich danke Ihnen viele Male für Ihren lieben Brief und das schöne \label{K_L03683-1v}\edtext{Dokument\pwindex{Une protestation DArthur Schnitzler@\emph{Une protestation d’Arthur Schnitzler}|pwv} Ihrer gerechten
                  Gesinnung}{\lemma{\textnormal{\emph{Dokument … Gesinnung}}}\Cendnote{\textnormal{Nachdem Schnitzler zugetragen worden war, dass unter seinem Namen
                  in einer russischen\oindex{Russland@\textbf{Russland}, \emph{A.PCLI}|pwk} Zeitung Leo Tolstoi\pwindex{Tolstoi, Leo N. von 09.09.1828 – 20.11.1910@\textsc{Tolstoi, Leo N. von} (09.09.1828 – 20.11.1910), \emph{Schriftsteller/Schriftstellerin, Schriftsteller/Schriftstellerin, Krimiautor/Krimiautorin}|pwk}, Maurice Maeterlinck\pwindex{Maeterlinck, Maurice 29.08.1862 – 06.05.1949@\textsc{Maeterlinck, Maurice} (29.08.1862 – 06.05.1949), \emph{Schriftsteller/Schriftstellerin}|pwk}, Anatole France\pwindex{France, Anatole 16.04.1844 – 12.10.1924@\textsc{France, Anatole} (16.04.1844 – 12.10.1924), \emph{Schriftsteller/Schriftstellerin}|pwk} und William
                     Shakespeare\pwindex{Shakespeare, William 23.4.1564? – 03.05.1616@\textsc{Shakespeare, William} (23.4.1564? – 03.05.1616), \emph{Schauspieler/Schauspielerin, Dramatiker/Dramatikerin}|pwk} verunglimpft worden waren (vgl. A. S.: \emph{Tagebuch}, 23. 11. 1914), verfasste er ein Dementi, das mit einem
                  Vorwort Romain Rollands\pwindex{Rolland, Romain 29.01.1866 – 30.12.1944@\textsc{Rolland, Romain} (29.01.1866 – 30.12.1944), \emph{Schriftsteller/Schriftstellerin}|pwk} und in dessen
                  Übersetzung ins Französische\oindex{Frankreich@\textbf{Frankreich}, \emph{A.PCLI}|pwk} zunächst in der
                     französischen Schweiz\oindex{Romandy@\textbf{Romandy}|pwk} publiziert wurde (Romain Rolland\pwindex{Rolland, Romain 29.01.1866 – 30.12.1944@\textsc{Rolland, Romain} (29.01.1866 – 30.12.1944), \emph{Schriftsteller/Schriftstellerin}|pwk}, Schnitzler: \emph{Une
                        protestation d’Arthur Schnitzler}\pwindex{Une protestation DArthur Schnitzler@\emph{Une protestation d’Arthur Schnitzler}|pwk}. In: \emph{Journal de Genève}\pwindex{Journal de Geneve@\emph{Journal de Genève}|pwk}, Jg. 85, 3. Ausgabe, 21. 12. 1914,
                     S. [1]) und später ohne die Übersetzung in weiteren Zeitungen Abdruck
                  fand.}}}\label{K_L03683-1}. Ich glaube, dass auch ein so geleg{[}e{]}ntliches
               Wort nur durch den Geist und die Güte, die es bezeugt, in diesen Tagen zum Manifest
               wird und zweifle nicht, dass es überall (ausser bei jenen Menschen, mit denen eine
               innere Verständigung über alles für uns unmöglich ist) die vorteilhafteste Wirkung im
               Gefolge haben \substVorne{}\textsuperscript{wird}\substDazwischen{}muss\substHinten{}. Ich habe es Romain Rolland\pwindex{Rolland, Romain 29.01.1866 – 30.12.1944@\textsc{Rolland, Romain} (29.01.1866 – 30.12.1944), \emph{Schriftsteller/Schriftstellerin}|pw} gesandt
               und ihn gebeten, die Uebersetzung ins Französische\oindex{Frankreich@\textbf{Frankreich}, \emph{A.PCLI}|pw} womöglich selbst vorzunehmen, damit auch nicht ein Wort in
               seiner Bedeutung oder bloss \introOben{}in\introOben{} seinem Tonfall durch
               schlechte Nachbildung verändert werde. Ich bin sicher, dass er sich eine Freude
               daraus machen wird\introOben{},\introOben{} Ihnen und vor allem der uns gemeinsamen
               Sache der gegenseitigen Aufklärung dienlich zu sein. In wenigen Tagen werde ich mehr
               darüber wissen.\pend
           
\pstart
           Eine Veröffentlichung in Wien\oindex{Wien@\textbf{Wien}, \emph{A.ADM2}|pw} wäre vielleicht
               vorteilhafter, sobald der Abdruck\pwindex{Une protestation DArthur Schnitzler@\emph{Une protestation d’Arthur Schnitzler}|pwv} in der Schweiz\oindex{Schweiz@\textbf{Schweiz}, \emph{A.PCLI}|pw} erfolgt ist und
               der Regierungsrat v. Winternitz\pwindex{Winternitz, Jakob von 03.03.1843 – 26.01.1921@\textsc{Winternitz, Jakob von} (03.03.1843 – 26.01.1921), \emph{Ministerialbeamter/Ministerialbeamte}|pw} würde
               sicherlich gerne die offizielle Verlautbarung übernehmen. Seine Privatadresse ist VIII. Kochgasse 29\oindex{Kochgasse 29@\textbf{Kochgasse 29}, \emph{Wohngebäude (K.WHS)}|pw}. Ich hoffe aber, ihn schon
               in diesen Tagen sprechen und mich seiner zweifellosen Zustimmung versichern zu
               können. \pend
           
\pstart
           Ich wäre sehr glücklich, wenn ich Sie, verehrter Herr \label{K_L03683-2v}\edtext{Doktor}{\lemma{\textnormal{\emph{Doktor}}}\Cendnote{\textnormal{Im
                  Manuskript steht Doktir.}}}\label{K_L03683-2} bald sehen oder wenigstens Ihre Stimme durch das
               Telephon hören dürfte. Ich bin jetzt {\pb}immer zwischen 4 und 5 Uhr zuhause, vorher hält mich der kriegerische Dienst,
               nachher verlockt mich jetzt oft und öfter die Musik. Aber ich will gern jede Stunde
               des Nachmittags von 4 Uhr, die Sie mir erlauben wollen, dazu wahrnehmen, um in das
                  Cottage\oindex{Waehringer Cottage@\textbf{Währinger Cottage}, \emph{Teil eines besiedelten Ortes (A.BSOX)}|pw} hinauszukommen oder wohin immer es
               Ihnen gutdünkt und Sie dann nicht\strikeout{s} nur Nachts im
               Traum, ohne Ihre Erlaubnis, sondern am lichten Tag, mit Ihrer freundlichen
               Verstattung heim \introOben{}zu\introOben{}suchen{[}.{]}\pend
           
\pstart
           Ich beschäftige mich auch damit, für Ihre Frau Gemahlin\pwindex{Schnitzler, Olga 17.01.1882 – 13.01.1970@\textsc{Schnitzler, Olga} (17.01.1882 – 13.01.1970), \emph{Schauspieler/Schauspielerin, Sänger/Sängerin}|pwv} ein paar schöne \label{K_L03683-3v}\edtext{Lieder}{\lemma{\textnormal{\emph{Lieder}}}\Cendnote{\textnormal{Aus Schnitzlers{ }\emph{Tagebuch}\pwindex{Tagebuch@\emph{Tagebuch}|pwk}eintrag geht hervor, dass Olga Schnitzler\pwindex{Schnitzler, Olga 17.01.1882 – 13.01.1970@\textsc{Schnitzler, Olga} (17.01.1882 – 13.01.1970), \emph{Schauspieler/Schauspielerin, Sänger/Sängerin}|pwk} Lieder von Schumann\pwindex{Schumann, Robert 08.06.1810 – 29.07.1856@\textsc{Schumann, Robert} (08.06.1810 – 29.07.1856), \emph{Komponist/Komponistin}|pwk} und Schubert\pwindex{Schubert, Franz Peter 31.01.1797 – 19.11.1828@\textsc{Schubert, Franz Peter} (31.01.1797 – 19.11.1828), \emph{Komponist/Komponistin}|pwk} sang, darunter das Lied \emph{Wegweiser}\pwindex{Winterreise. Der Wegweiser@\emph{Winterreise. Der Wegweiser}|pwk} aus der \emph{Winterreise}\pwindex{Winterreise [op. 89 D 911]@\emph{Winterreise [op. 89 D 911]}|pwk}, vgl. A. S.: \emph{Tagebuch}, 3. 1. 1915.}}}\label{K_L03683-3} für jenen
                  \label{K_L03683-4v}\edtext{Liliencron\pwindex{Liliencron, Detlev von 03.06.1844 – 22.07.1909@\textsc{Liliencron, Detlev von} (03.06.1844 – 22.07.1909), \emph{Schriftsteller/Schriftstellerin, Dichter/Dichterin, Dramatiker/Dramatikerin}|pw}-Abend}{\lemma{\textnormal{\emph{Liliencron-Abend}}}\Cendnote{\textnormal{Die Veranstaltung fand am 3. 1. 1915 im Volksheim\oindex{Volkshochschule Ottakring@\textbf{Volkshochschule Ottakring}, \emph{Gebäude (K.GBD)}|pwk} statt zum Andenken an den Dichter Detlev von Liliencron\pwindex{Liliencron, Detlev von 03.06.1844 – 22.07.1909@\textsc{Liliencron, Detlev von} (03.06.1844 – 22.07.1909), \emph{Schriftsteller/Schriftstellerin, Dichter/Dichterin, Dramatiker/Dramatikerin}|pwk}, der im Vorjahr
                  siebzig Jahre alt geworden wäre. Der Vortrag wurde publiziert als Stefan Zweig\pwindex{Zweig, Stefan 28.11.1881 – 23.02.1942@\textsc{Zweig, Stefan} (28.11.1881 – 23.02.1942), \emph{Schriftsteller/Schriftstellerin}|pwk}: \emph{Liliencron}\pwindex{Liliencron@\emph{Liliencron}|pwk}. In: \emph{Die
                        Schaubühne}\pwindex{Schaubuehne@\emph{Die Schaubühne}|pwk}, Jg. 11/I, Nr. 8, 25. 2. 1915,
                     S. 176–181.}}}\label{K_L03683-4} zusammenzustellen, dessen Gelingen mich schon um des
               denkbaren Arbeiterpublikums willen so sehr freuen würde. Bishin vielen Dank und die
               herzlichsten Grüsse von\pend
           
\pstart
           Ihrem immer getreuen{\\[\baselineskip]}\spacefill\mbox{{[}hs.:{]} Stefan Zweig}\pend
           \leftskip=0em{}
\pstart
           \noindent{}Verzeihen Sie die Schreibmaschine! Ich schreibe den ganzen Vormittag im Amt und
                  gebe dann meinern Fingern Rast!\pend
           \selectlanguage{ngerman}\endnumbering\briefempfaengerindex{Schnitzler, Arthur@\textsc{Schnitzler, Arthur}!zzzZweig, Stefan@\emph{von Stefan Zweig}!1914-12-031@{3. 12. 1914}|)be}\mylabel{L03683h}
\begin{anhang}
\end{anhang}\normalsize

\doendnotes{C}
\bigskip
\vfill

\clearpage

\footnotesize

\lohead{\textsc{register}}

% Definiere theindex-Environment komplett neu ohne reledmac
\makeatletter
\renewenvironment{theindex}{%
  \section*{\indexname}%
  \setlength{\parindent}{0pt}%
  \setlength{\parskip}{0pt plus 0.3pt}%
  \let\item\@idxitem
}{%
  \clearpage
}
\makeatother

\IfFileExists{\jobname-pw.ind}{\input{\jobname-pw.ind}}{}

\end{document}

      