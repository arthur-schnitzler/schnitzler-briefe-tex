%% latex-leseansicht-vorspann.tex
%% Vorspann für die Leseansicht.
%% Lädt die gemeinsame Datei latex-vorspann.tex mit nicht gesetztem Schalter.

\newif\ifkorrekturansicht
\korrekturansichtfalse

\input{../tex-inputs/latex-vorspann}


\section[Stefan Zweig an Arthur Schnitzler, 3. 12. 1914]{L03683 Stefan Zweig an Arthur Schnitzler, 3. 12. 1914}
\nopagebreak\mylabel{L03683v}
\rehead{ }\normalsize\beginnumbering\briefempfaengerindex{Schnitzler, Arthur@\textsc{Schnitzler, Arthur}!zzzZweig, Stefan@\emph{von Stefan Zweig}!1914-12-031@{3. 12. 1914}|(be}
\toendnotes[C]{\smallbreak\pagebreak[2]}
\correspDesc{Versand  durch Stefan Zweig am 3. 12. 1914 in Wien
\newline{}Erhalt  durch Arthur Schnitzler im Zeitraum [3. 12. 1914
                  – 5. 12. 1914?] in Wien}\toendnotes[C]{\smallbreak}
\Standort{CUL, Schnitzler, B 118.}
\physDesc{Brief, 1 Blatt, 2 Seiten, 2232 Zeichen
\newline{}Schreibmaschine
\newline{}Handschrift: blaue Tinte, lateinische Kurrent (\noindent{}Korrekturen, Unterschrift und Postskriptum)
\newline{}Schnitzler: 1) mit rotem Buntstift drei Unterstreichungen  2) mit Bleistift beschriftet: »\textsc{Zweig}«}
\buchAbdrucke{\weitereDrucke{Stefan Zweig: \emph{Briefwechsel mit Hermann Bahr, Sigmund Freud, Rainer Maria
                        Rilke und Arthur Schnitzler}. Herausgegeben von Jeffrey B. Berlin, Hans-Ulrich Lindken und Donald A. Prater. Frankfurt am Main: \emph{S. Fischer} 1987, S. 386–388.} }\toendnotes[C]{\smallbreak}
\pstart
           {\pb}\textcolor{gray}{\textbf{SZ}}\hfill \textcolor{gray}{\textbf{VIII. KOCHGASSE 8\oindex{Wien@\textbf{Wien}!VIII., Josefstadt@\textbf{VIII., Josefstadt}!Kochgasse 8@\textbf{Kochgasse 8}, \emph{Wohngebäude}|pw}}}\pend
           
\pstart
           \raggedleft{}\textcolor{gray}{\textbf{WIEN\oindex{Wien@\textbf{Wien}, \emph{Verwaltungsgebiet}|pw},}}\pend
           
\pstart
           \raggedleft{}Wien\oindex{Wien@\textbf{Wien}, \emph{Verwaltungsgebiet}|pw}, 3. Dezember 14\pend
           
\pstart\center{}Sehr verehrter lieber Herr Doktor1\pend\vspace{0.5em}
\pstart
           Ich danke Ihnen viele Mal\introOben{}e\introOben{} für Ihren lieben \label{K_L03683-1v}\edtext{Brief}{\lemma{\textnormal{\emph{Brief}}}\Cendnote{\textnormal{XXXX Auszeichnungsfehler: Dokument L03779 nicht gefunden. }}}\label{K_L03683-1} und das schöne Dokument\pwindex{Schnitzler, Arthur 15.\,5.\,1862 Wien – 21.\,10.\,1931 ebd.@\textsc{Schnitzler, Arthur} (15.\,5.\,1862 Wien – 21.\,10.\,1931 ebd.), \emph{Schriftsteller, Mediziner}!Une protestation d’Arthur Schnitzler@\strich\emph{Une protestation d’Arthur Schnitzler}|pwv} Ihrer gerechten
               Gesinnung. Ich glaube, dass auch ein so geleg{[}e{]}ntliches Wort nur
               durch den Geist und die Güte, die es bezeugt, in diesen Tagen zum Manifest wird und
               zweifle nicht, dass es überall (ausser bei jenen Menschen, mit denen eine innere
               Verständigung über alles für uns unmöglich ist) die vorteilhafteste Wirkung im
               Gefolge haben \substVorne{}\textsuperscript{wird}\substDazwischen{}muss\substHinten{}. Ich habe es Romain Rolland\pwindex{Rolland, Romain 29.\,1.\,1866 Clamecy – 30.\,12.\,1944 Vézelay@\textsc{Rolland, Romain} (29.\,1.\,1866 Clamecy – 30.\,12.\,1944 Vézelay), \emph{Schriftsteller}|pw} gesandt
               und ihn gebeten, die Uebersetzung ins Französische\oindex{Frankreich@\textbf{Frankreich}|pw} womöglich selbst vorzunehmen, damit auch nicht ein Wort in
               seiner Bedeutung oder bloss \introOben{}in\introOben{} seinem Tonfall durch
               schlechte Nachbildung verändert werde. Ich bin sicher, dass er sich eine Freude
               daraus machen wird\introOben{},\introOben{} Ihnen und vor allem der uns gemeinsamen
               Sache der gegenseitigen Aufklärung dienlich zu sein. In wenigen Tagen werde ich mehr
               darüber wissen.\pend
           
\pstart
           Eine Veröffentlichung in Wien\oindex{Wien@\textbf{Wien}, \emph{Verwaltungsgebiet}|pw} wäre vielleicht
               vorteilhafter, sobald der Abdruck\pwindex{Schnitzler, Arthur 15.\,5.\,1862 Wien – 21.\,10.\,1931 ebd.@\textsc{Schnitzler, Arthur} (15.\,5.\,1862 Wien – 21.\,10.\,1931 ebd.), \emph{Schriftsteller, Mediziner}!Une protestation d’Arthur Schnitzler@\strich\emph{Une protestation d’Arthur Schnitzler}|pwv} in der Schweiz\oindex{Schweiz@\textbf{Schweiz}|pw} erfolgt ist und
               der Regierungsrat v. Winternitz\pwindex{Winternitz, Jakob von 3.\,3.\,1843 Horažďovice – 26.\,1.\,1921 Wien@\textsc{Winternitz, Jakob von} (3.\,3.\,1843 Horažďovice – 26.\,1.\,1921 Wien), \emph{Ministerialbeamter}|pw} würde
               sicherlich gerne die offizielle Verlautbarung übernehmen. Seine Privatadresse ist VIII. Kochgasse 29\oindex{Wien@\textbf{Wien}!VIII., Josefstadt@\textbf{VIII., Josefstadt}!Kochgasse 29@\textbf{Kochgasse 29}, \emph{Wohngebäude}|pw}. Ich hoffe aber, ihn schon
               in diesen Tagen sprechen und mich seiner zweifellosen Zustimmung versichern zu
               können.\pend
           
\pstart
           Ich wäre sehr glücklich, wenn ich Sie, verehrter Herr \label{T_L03683-1v}\edtext{Doktor}{\lemma{\textnormal{\emph{Doktor}}}\Cendnote{\textnormal{Im
                  Manuskript steht »Doktir«.}}}\label{T_L03683-1} bald sehen oder wenigstens Ihre
               Stimme durch das Telephon hören dürfte. Ich bin jetzt {\pb}immer zwischen 4 und 5 Uhr zuhause, vorher
               hält mich der kriegerische Dienst, nachher verlockt mich jetzt oft und öfter die
               Musik. Aber ich will gern jede Stunde des Nachmittags von 4 Uhr, die Sie mir erlauben
               wollen, dazu wahrnehmen, um in das Cottage\oindex{Wien@\textbf{Wien}!XVIII., Währing@\textbf{XVIII., Währing}!Währinger Cottage@\textbf{Währinger Cottage}, \emph{Teil eines besiedelten Ortes}|pw}
               hinauszukommen oder wohin immer es Ihnen gutdünkt und Sie dann nicht\strikeout{s} nur Nachts im Traum, ohne Ihre Erlaubnis, sondern am
               lichten Tag, mit Ihrer freundlichen Verstattung heim\introOben{}zu\introOben{}suchen\pend
           
\pstart
           Ich beschäftige mich auch damit, für Ihre Frau Gemahlin\pwindex{Schnitzler, Olga 17.\,1.\,1882 Wien – 13.\,1.\,1970 Lugano@\textsc{Schnitzler, Olga} (17.\,1.\,1882 Wien – 13.\,1.\,1970 Lugano), \emph{Schauspielerin, Sängerin}|pwv} ein paar schöne \label{K_L03683-2v}\edtext{Lieder für jenen Liliencron\pwindex{Liliencron, Detlev von 3.\,6.\,1844 Kiel – 22.\,7.\,1909 Rahlstedt@\textsc{Liliencron, Detlev von} (3.\,6.\,1844 Kiel – 22.\,7.\,1909 Rahlstedt), \emph{Schriftsteller, Dichter, Dramatiker}|pw}-Abend\eventindex{Volkshochschule Ottakring@\textbf{Volkshochschule Ottakring}!Dichterabend Detlev von Liliencron, 3.1.1915@Dichterabend Detlev von Liliencron, 3.1.1915|pw}}{\lemma{\textnormal{\emph{Lieder … Liliencron-Abend}}}\Cendnote{\textnormal{Es handelt sich
                        um die Planung des Detlev von
                           Liliencron\pwindex{Liliencron, Detlev von 3.\,6.\,1844 Kiel – 22.\,7.\,1909 Rahlstedt@\textsc{Liliencron, Detlev von} (3.\,6.\,1844 Kiel – 22.\,7.\,1909 Rahlstedt), \emph{Schriftsteller, Dichter, Dramatiker}|pwk} gewidmeten »Dichterabends« am 3. 1. 1915 im Volksheim\oindex{Wien@\textbf{Wien}!XVI., Ottakring@\textbf{XVI., Ottakring}!Volkshochschule Ottakring@\textbf{Volkshochschule Ottakring}, \emph{Gebäude}|pwk}. (Detlev von
                                    Liliencron\pwindex{Liliencron, Detlev von 3.\,6.\,1844 Kiel – 22.\,7.\,1909 Rahlstedt@\textsc{Liliencron, Detlev von} (3.\,6.\,1844 Kiel – 22.\,7.\,1909 Rahlstedt), \emph{Schriftsteller, Dichter, Dramatiker}|pwk} wäre im Vorjahr siebzig Jahre alt geworden.) Obwohl nicht groß
                        angekündigt, war Olga Schnitzler\pwindex{Schnitzler, Olga 17.\,1.\,1882 Wien – 13.\,1.\,1970 Lugano@\textsc{Schnitzler, Olga} (17.\,1.\,1882 Wien – 13.\,1.\,1970 Lugano), \emph{Schauspielerin, Sängerin}|pwk} beteiligt
                        und sang Lieder von Schumann\pwindex{Schumann, Robert 8.\,6.\,1810 Zwickau – 29.\,7.\,1856 Endenich@\textsc{Schumann, Robert} (8.\,6.\,1810 Zwickau – 29.\,7.\,1856 Endenich), \emph{Komponist}|pwk} und Schubert\pwindex{Schubert, Franz Peter 31.\,1.\,1797 Lichtental [Wien] – 19.\,11.\,1828 Wien@\textsc{Schubert, Franz Peter} (31.\,1.\,1797 Lichtental [Wien] – 19.\,11.\,1828 Wien), \emph{Komponist}|pwk}, darunter das Lied \emph{Wegweiser}\pwindex{\textcolor{red}{\textsuperscript{XXXX indx1}}!Wegweiser [g-Moll D 911]@\strich\emph{Der Wegweiser [g-Moll D 911]}|pwk}\pwindex{Schubert, Franz Peter 31.\,1.\,1797 Lichtental [Wien] – 19.\,11.\,1828 Wien@\textsc{Schubert, Franz Peter} (31.\,1.\,1797 Lichtental [Wien] – 19.\,11.\,1828 Wien), \emph{Komponist}!Wegweiser [g-Moll D 911]@\strich\emph{Der Wegweiser [g-Moll D 911]}|pwk} aus der \emph{Winterreise}\pwindex{Schubert, Franz Peter 31.\,1.\,1797 Lichtental [Wien] – 19.\,11.\,1828 Wien@\textsc{Schubert, Franz Peter} (31.\,1.\,1797 Lichtental [Wien] – 19.\,11.\,1828 Wien), \emph{Komponist}!Winterreise [op. 89 D 911]@\strich\emph{Winterreise [op. 89 D 911]}|pwk}\pwindex{\textcolor{red}{\textsuperscript{XXXX indx1}}!Winterreise [op. 89 D 911]@\strich\emph{Winterreise [op. 89 D 911]}|pwk}, vgl. A. S.: \emph{Tagebuch}, 3. 1. 1915. Zweig\pwindex{Zweig, Stefan 28.\,11.\,1881 Wien – 23.\,2.\,1942 Petrópolis@\textsc{Zweig, Stefan} (28.\,11.\,1881 Wien – 23.\,2.\,1942 Petrópolis), \emph{Schriftsteller}|pwk} hielt einleitende
                        Worte.}}}\label{K_L03683-2}
               zusammenzustellen, dessen Gelingen mich schon um des \label{K_L03683-3v}\edtext{denkbaren}{\lemma{\textnormal{\emph{denkbaren}}}\Cendnote{\textnormal{gemeint: dankbaren?}}}\label{K_L03683-3} Arbeiterpublikums willen so sehr freuen würde. Bishin
               vielen Dank und die herzlichsten Grüsse von\pend
           
\pstart
           Ihrem immer getreuen{\\[\baselineskip]}\spacefill\mbox{{[}hs.:{]} Stefan Zweig}\pend
           \leftskip=0em{}
\pstart
           \noindent{}\label{K_L03683-11v}\edtext{Verzeihen Sie die Schreibmaschine}{\lemma{\textnormal{\emph{Verzeihen … Schreibmaschine}}}\Cendnote{\textnormal{Er diktierte also. Die Praxis, dass Schriftsteller
               selbst die Maschine bedienten, wurde erst nach dem 1. Weltkrieg gängig.}}}\label{K_L03683-11}! Ich schreibe den ganzen Vormittag im Amt und
                  gebe dann meinern Fingern Rast!\pend
           \selectlanguage{ngerman}\endnumbering\briefempfaengerindex{Schnitzler, Arthur@\textsc{Schnitzler, Arthur}!zzzZweig, Stefan@\emph{von Stefan Zweig}!1914-12-031@{3. 12. 1914}|)be}\mylabel{L03683h}  \newcommand{\dateiname}{L03683}\newcommand{\titel}{Stefan Zweig an Arthur Schnitzler, 3. 12. 1914}\newcommand{\editorInnen}{Selma Jahnke und Martin Anton Müller}%% latex-leseansicht-abspann.tex
%% Abspann für die Leseansicht.
%% Der Schalter \ifkorrekturansicht ist bereits durch den Vorspann gesetzt.

%% latex-abspann.tex
%% Gemeinsamer Abspann für Korrekturansicht und Leseansicht.
%% Setzt den Schalter \ifkorrekturansicht voraus (gesetzt in den
%% einbindenden Dateien latex-korrekturansicht-abspann.tex bzw.
%% latex-leseansicht-abspann.tex).
%% ---------------------------------------------------------------

\normalsize

% Das esempio-Environment wird nur in der Leseansicht benötigt
\ifkorrekturansicht\else
\newenvironment{esempio}[3]%
{
    \vspace{1.5ex}
    \rlap{\underline{#1}}
    \par
    \setlength{\parindent}{0cm}
    \nopagebreak
    \leftskip=#2cm
    \rightskip=#3cm
}
{
    \par
}
\fi

\doendnotes{C}
\bigskip
\vfill

\clearpage

\footnotesize

\ifkorrekturansicht
  \lohead{\textsc{register}}
\fi

% theindex-Environment neu definieren ohne reledmac
\makeatletter
\renewenvironment{theindex}{%
  \ifkorrekturansicht
    \section*{\indexname}%
  \else
    \subsubsection*{Index der erwähnten Entitäten}%
  \fi
  \setlength{\parindent}{0pt}%
  \setlength{\parskip}{0pt plus 0.3pt}%
  \let\item\@idxitem
}{%
  \ifkorrekturansicht\clearpage\fi
}
\makeatother

\IfFileExists{\jobname-pw.ind}{\input{\jobname-pw.ind}}{}

% Quellenangabe nur in der Leseansicht
\ifkorrekturansicht\else
% Fallback-Definitionen, falls die .tex-Datei \titel etc. nicht gesetzt hat
\providecommand{\titel}{}
\providecommand{\editorInnen}{}
\providecommand{\dateiname}{\jobname}

\vspace{3cm}

\vfill

\footnotesize
\textsc{Quelle}: \titel. Herausgegeben von {\editorInnen}. In: \emph{Arthur Schnitzler: Briefwechsel mit Autorinnen und Autoren}.
 Digitale Edition, https://schnitzler-briefe.acdh.oeaw.ac.at/{\dateiname}.html (Stand \today)
\fi

\end{document}


