%% latex-leseansicht-vorspann.tex
%% Vorspann für die Leseansicht.
%% Lädt die gemeinsame Datei latex-vorspann.tex mit nicht gesetztem Schalter.

\newif\ifkorrekturansicht
\korrekturansichtfalse

\input{../tex-inputs/latex-vorspann}


\section[Arthur Schnitzler an Richard Beer-Hofmann, 14. 9. 1904]{L01444 Arthur Schnitzler an Richard Beer-Hofmann, 14. 9. 1904}
\nopagebreak\mylabel{L01444v}
\rehead{ }\normalsize\beginnumbering\briefempfaengerindex{Beer-Hofmann, Richard@\textsc{Beer-Hofmann, Richard}!zzzSchnitzler, Arthur@\emph{von Arthur Schnitzler}!1904-09-141@{14. 9. 1904}|(be}
\toendnotes[C]{\smallbreak\pagebreak[2]}
\correspDesc{Versand  durch Arthur Schnitzler am 14. 9. 1904 in St. Gilgen
\newline{}Erhalt  durch Richard Beer-Hofmann am 15. 9. 1904 in Bad Aussee}\toendnotes[C]{\smallbreak}
\Standort{YCGL, MSS 31.}
\physDesc{Brief, 1 Blatt, 3 Seiten, Kuvert, 954 Zeichen
\newline{}Handschrift: Bleistift, deutsche Kurrent
\newline{}Versand: 1) Stempel: »\nobreak{}\oindex{St. Gilgen@\textbf{St. Gilgen}, \emph{Verwaltungsgebiet}|pwk}St. Gilgen, 14. 9. 04, 3–4N\nobreak{}«.   2) Stempel: »\nobreak{}\oindex{Bad Aussee@\textbf{Bad Aussee}, \emph{Hauptstadt}|pwk}{\pb}\textcolor{gray}{A}ussee in Steiermark, 15 9 04\nobreak{}«. }
\buchAbdrucke{\weitereDrucke{Arthur Schnitzler, Richard Beer-Hofmann: \emph{Briefwechsel 1891–1931}. Herausgegeben von Konstanze Fliedl. Wien, Zürich: \emph{Europaverlag} 1992, S. 166–167.} }\toendnotes[C]{\smallbreak}\pstart{}{\pb}\textsc{Herrn Dr Rich. Beer-Hofmann}\pend{}\pstart{}\textsc{Markt Aussee\oindex{Bad Aussee@\textbf{Bad Aussee}, \emph{Hauptstadt}|pw}}\pend{}\pstart{}\textsc{Villa Frühling}\oindex{Villa Frühling@\textbf{Villa Frühling}, \emph{Gebäude}|pw}. \pend{}{\bigskip}\vspace{1em}
\pstart
           \raggedleft{}{\pb}\textsc{Lueg}\oindex{Lueg@\textbf{Lueg}, \emph{Teil eines besiedelten Ortes}|pw}, 14. 9. 904\pend
           \vspace{0.5em}
\pstart
           lieber Richard, eben ko{\geminationm}t, wie ich im
               Begriff bin Ihnen zu telegrafiren, \substVorne{}\textsuperscript{ein}\substDazwischen{}Ihr\substHinten{} Brief. Wir möchten Samſtag den 17. von hier nach Salzburg\oindex{Salzburg@\textbf{Salzburg}, \emph{Verwaltungsgebiet}|pw} reiſen und dort einige Tage bleiben.
               (Möchten diesmal verſuchsweiſe Nelböck\oindex{Hotel und Pension Nelböck@\textbf{Hotel und Pension Nelböck}, \emph{Hotel}|pw}
               wohnen.) Ich{ }ſchlage Ihnen nun vor, Freitag nach \textsc{Lueg}\oindex{Lueg@\textbf{Lueg}, \emph{Teil eines besiedelten Ortes}|pw} zu ko{\geminationm}en und Samſtag mit uns zu
               fahren, oder uns \strikeout{vielleicht}{ }{\pb}zu{ }ſchreiben, wann Sie in \textsc{Lueg}\oindex{Lueg@\textbf{Lueg}, \emph{Teil eines besiedelten Ortes}|pw} durchkommen,{ }ſo daſs wir hier zu Ihnen einſteigen. (Der Zug, der Iſchl\oindex{Bad Ischl@\textbf{Bad Ischl}|pw}{ }8.55 früh verläßt u 9.59{ }\textsc{Lueg}\oindex{Lueg@\textbf{Lueg}, \emph{Teil eines besiedelten Ortes}|pw} paſſirt, wäre mir der weitaus{ }ſympathiſcheſte.) In Salzburg\oindex{Salzburg@\textbf{Salzburg}, \emph{Verwaltungsgebiet}|pw} möcht ich bis mindeſtens 21.,
                  22. bleiben; von dort fahren wir aller Wahrſcheinlichkeit direct nach
                  Wien\oindex{Wien@\textbf{Wien}, \emph{Verwaltungsgebiet}|pw}.\pend
           
\pstart
           Telegrafiren Sie bitte Ihre Entſcheidg, ev. auch wo Sie in Salzb.\oindex{Salzburg@\textbf{Salzburg}, \emph{Verwaltungsgebiet}|pw} zu {\pb}wohnen gedenken, und
               ob Sie nicht vielleicht von Freitag bis So{\geminationn}tag in \textsc{Lueg}\oindex{Lueg@\textbf{Lueg}, \emph{Teil eines besiedelten Ortes}|pw} bleiben und mir hier den Grafen \textsc{Ch}.\pwindex{Beer-Hofmann, Richard 11.\,7.\,1866 Wien – 26.\,9.\,1945 New York City@\textsc{Beer-Hofmann, Richard} (11.\,7.\,1866 Wien – 26.\,9.\,1945 New York City), \emph{Schriftsteller}!Graf von Charolais. Ein Trauerspiel@\strich\emph{Der Graf von Charolais. Ein Trauerspiel}|pw} vorleſen möchten.\pend
           
\pstart
           Für alle Fälle hoff ich{ }ſind wir noch ein paar Tage beiſammen.\pend
           
\pstart
           Herzlichſt Ihr{\\[\baselineskip]}\spacefill\mbox{A.}\pend
           \leftskip=0em{}
\pstart
           \noindent{}Grüße von Gaſthof\oindex{Hotel und Pension Lueg@\textbf{Hotel und Pension Lueg}, \emph{Hotel}|pwv} zu Villa\oindex{Villa Frühling@\textbf{Villa Frühling}, \emph{Gebäude}|pwv}.\pend
           \selectlanguage{ngerman}\endnumbering\briefempfaengerindex{Beer-Hofmann, Richard@\textsc{Beer-Hofmann, Richard}!zzzSchnitzler, Arthur@\emph{von Arthur Schnitzler}!1904-09-141@{14. 9. 1904}|)be}\mylabel{L01444h}  \newcommand{\dateiname}{L01444}\newcommand{\titel}{Arthur Schnitzler an Richard Beer-Hofmann, 14. 9. 1904}\newcommand{\editorInnen}{Martin Anton Müller und Gerd-Hermann Susen}%% latex-leseansicht-abspann.tex
%% Abspann für die Leseansicht.
%% Der Schalter \ifkorrekturansicht ist bereits durch den Vorspann gesetzt.

%% latex-abspann.tex
%% Gemeinsamer Abspann für Korrekturansicht und Leseansicht.
%% Setzt den Schalter \ifkorrekturansicht voraus (gesetzt in den
%% einbindenden Dateien latex-korrekturansicht-abspann.tex bzw.
%% latex-leseansicht-abspann.tex).
%% ---------------------------------------------------------------

\normalsize

% Das esempio-Environment wird nur in der Leseansicht benötigt
\ifkorrekturansicht\else
\newenvironment{esempio}[3]%
{
    \vspace{1.5ex}
    \rlap{\underline{#1}}
    \par
    \setlength{\parindent}{0cm}
    \nopagebreak
    \leftskip=#2cm
    \rightskip=#3cm
}
{
    \par
}
\fi

\doendnotes{C}
\bigskip
\vfill

\clearpage

\footnotesize

\ifkorrekturansicht
  \lohead{\textsc{register}}
\fi

% theindex-Environment neu definieren ohne reledmac
\makeatletter
\renewenvironment{theindex}{%
  \ifkorrekturansicht
    \section*{\indexname}%
  \else
    \subsubsection*{Index der erwähnten Entitäten}%
  \fi
  \setlength{\parindent}{0pt}%
  \setlength{\parskip}{0pt plus 0.3pt}%
  \let\item\@idxitem
}{%
  \ifkorrekturansicht\clearpage\fi
}
\makeatother

\IfFileExists{\jobname-pw.ind}{\input{\jobname-pw.ind}}{}

% Quellenangabe nur in der Leseansicht
\ifkorrekturansicht\else
% Fallback-Definitionen, falls die .tex-Datei \titel etc. nicht gesetzt hat
\providecommand{\titel}{}
\providecommand{\editorInnen}{}
\providecommand{\dateiname}{\jobname}

\vspace{3cm}

\vfill

\footnotesize
\textsc{Quelle}: \titel. Herausgegeben von {\editorInnen}. In: \emph{Arthur Schnitzler: Briefwechsel mit Autorinnen und Autoren}.
 Digitale Edition, https://schnitzler-briefe.acdh.oeaw.ac.at/{\dateiname}.html (Stand \today)
\fi

\end{document}


