%% latex-korrekturansicht-vorspann.tex
%% Vorspann für die Korrekturansicht.
%% Lädt die gemeinsame Datei latex-vorspann.tex mit gesetztem Schalter.

\newif\ifkorrekturansicht
\korrekturansichttrue

\input{../tex-inputs/latex-vorspann}


\section[Paul Goldmann an Arthur Schnitzler, {[}22.? 8. 1894{]}]{L02606 Paul Goldmann an Arthur Schnitzler, {[}22.? 8. 1894{]}}
\nopagebreak\mylabel{L02606v}
\rehead{ }\normalsize\beginnumbering\briefempfaengerindex{Schnitzler, Arthur@\textsc{Schnitzler, Arthur}!zzzGoldmann, Paul@\emph{von Paul Goldmann}!1894-08-221@{{[}22.? 8. 1894{]}}|(be}
\toendnotes[C]{\smallbreak\pagebreak[2]}\Standort{DLA, A:Schnitzler, HS.NZ85.1.3164.}
\physDesc{Telegramm, 68 Zeichen
\newline{}Handschrift: blaue Tinte, lateinische Kurrent
\newline{}Schnitzler: mit Bleistift Vermerk »\textsc{August 94}« 
\newline{}Ordnung: beschnitten }\toendnotes[C]{\smallbreak}
\pstart
           \noindent{}{\pb}Ankomme \label{K_L02606-1v}\edtext{morgen nachmittag}{\lemma{\textnormal{\emph{morgen nachmittag}}}\Cendnote{\textnormal{Am Nachmittag des 23. 8. 1894 holte Schnitzler{ }Goldmann\pwindex{Goldmann, Paul 31.01.1865 – 25.09.1935@\textsc{Goldmann, Paul} (31.01.1865 – 25.09.1935), \emph{Schriftsteller/Schriftstellerin, Journalist/Journalistin}|pwk} in Ischl\oindex{Bad Ischl@\textbf{Bad Ischl}, \emph{P.PPL}|pwk} vom Zug ab, womit die Datierung des Telegramms möglich wird.}}}\label{K_L02606-1}
               suche mir bitte Preismäßiges Quartier\pend
           \pstart \spacefill\mbox{\textcolor{gray}{G}{[}oldmann{]}}\pend{}\selectlanguage{ngerman}\endnumbering\briefempfaengerindex{Schnitzler, Arthur@\textsc{Schnitzler, Arthur}!zzzGoldmann, Paul@\emph{von Paul Goldmann}!1894-08-221@{{[}22.? 8. 1894{]}}|)be}\mylabel{L02606h}  \normalsize

\doendnotes{C}
\bigskip
\vfill

\clearpage

\footnotesize

\lohead{\textsc{register}}

% Definiere theindex-Environment komplett neu ohne reledmac
\makeatletter
\renewenvironment{theindex}{%
  \section*{\indexname}%
  \setlength{\parindent}{0pt}%
  \setlength{\parskip}{0pt plus 0.3pt}%
  \let\item\@idxitem
}{%
  \clearpage
}
\makeatother

\IfFileExists{\jobname-pw.ind}{\input{\jobname-pw.ind}}{}

\end{document}

      