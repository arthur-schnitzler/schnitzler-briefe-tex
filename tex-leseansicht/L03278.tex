%% latex-leseansicht-vorspann.tex
%% Vorspann für die Leseansicht.
%% Lädt die gemeinsame Datei latex-vorspann.tex mit nicht gesetztem Schalter.

\newif\ifkorrekturansicht
\korrekturansichtfalse

\input{../tex-inputs/latex-vorspann}


         \renewcommand{\erwaehnteOrte}{Orte: Burgtheater, Wien}
         \renewcommand{\erwaehnteWerke}{Werke: Hanneles Himmelfahrt. Traumdichtung in zwei Teilen, König Ödipus. Tragödie in einem Aufzuge, Tagebuch}
               \section[ Felix Salten an Arthur Schnitzler, {[}4. 1.? 1898{]}]{ Felix Salten an Arthur Schnitzler, {[}4. 1.? 1898{]}}\nopagebreak\mylabel{v}\rehead{ }\begin{ledgroupsized}[t]{13cm}\normalsize\beginnumbering \toendnotes[C]{\smallbreak\pagebreak[2]} \Standort{CUL, Schnitzler, B 89, A 2.}
\physDesc{Brief, 1 Blatt, 1 Seite, 161 Zeichen
\newline{}Handschrift: Bleistift, lateinische Kurrent
\newline{}Schnitzler: mit Bleistift auf das Jahr »98« datiert 
\newline{}Ordnung: mit Bleistift von unbekannter Hand nummeriert: »101« }\toendnotes[C]{\smallbreak}\pstart
           \noindent{}{\pb}Lieber Arthur, ich kann Ihnen den \label{K_L03278-1v}\edtext{Sitz}{\lemma{\textnormal{\emph{Sitz}}}\Cendnote{\textnormal{Das
                  Korrespondenzstück wird von Schnitzler\pwindex{Schnitzler, Arthur 15.05.1862 – 21.10.1931@\textsc{Schnitzler, Arthur} (15.05.1862 – 21.10.1931), \emph{Schriftsteller, Mediziner}|pwk} nur innerhalb des
                  Jahres 1898 verortet. Gleicht man die 22 in diesem Jahr nachweisbaren
                  Besuche Schnitzler\pwindex{Schnitzler, Arthur 15.05.1862 – 21.10.1931@\textsc{Schnitzler, Arthur} (15.05.1862 – 21.10.1931), \emph{Schriftsteller, Mediziner}|pwk}s im Burgtheater\oindex{Burgtheater@\textbf{Burgtheater}|pwk} (»Vestibül\oindex{Burgtheater@\textbf{Burgtheater}|pwv}«) mit den Erwähnungen Salten\pwindex{Salten, Felix 06.09.1869 – 08.10.1945@\textsc{Salten, Felix} (06.09.1869 – 08.10.1945), \emph{Schriftsteller, Journalist}|pwk}s im
                     \emph{Tagebuch}\pwindex{\textcolor{red}{\textsuperscript{XXXX1 indx}}!Tagebuch1981 – 2000@\strich\emph{Tagebuch} {[}Hrsg., 1981 – 2000{]}|pwk} in dieser Zeit ab, so ergibt sich
                  nur ein gemeinsamer Besuch, für den Salten\pwindex{Salten, Felix 06.09.1869 – 08.10.1945@\textsc{Salten, Felix} (06.09.1869 – 08.10.1945), \emph{Schriftsteller, Journalist}|pwk}
                  die Karten besorgt haben könnte. Demnach ist hier von der Aufführung von \emph{König Oidipus}\pwindex{\textcolor{red}{\textsuperscript{XXXX1 indx}}!Koenig Oedipus. Tragoedie in einem Aufzuge@\strich\emph{König Ödipus. Tragödie in einem Aufzuge}|pwk} und \emph{Hanneles Himmelfahrt}\pwindex{\textcolor{red}{\textsuperscript{XXXX1 indx}}!Hanneles Himmelfahrt. Traumdichtung in zwei Teilen1893-11-14@\strich\emph{Hanneles Himmelfahrt. Traumdichtung in zwei Teilen} {[}1893-11-14{]}|pwk} am 4. 1. 1898 die Rede. }}}\label{K_L03278-1h} jetzt nicht
               schicken, weil der Diener eine Dummheit gemacht hat. Treffen wir uns also
                  Abends um \label{K_L03278-2v}\edtext{¼ 8}{\lemma{\textnormal{\emph{¼ 8}}}\Cendnote{\textnormal{7 Uhr 15}}}\label{K_L03278-2h} im Vestibül\oindex{Burgtheater@\textbf{Burgtheater}|pwv}.\pend
           \pstart
           Herzlich Ihr {\\[\baselineskip]}\spacefill\mbox{Salten}\pend
           \leftskip=0em{}
         
         \endnumbering\mylabel{h}\end{ledgroupsized}  \newcommand{\dateiname}{L03278}\newcommand{\titel}{Felix Salten an Arthur Schnitzler, [4. 1.? 1898]}\newcommand{\editorInnen}{Martin Anton Müller und Laura Untner}%% latex-leseansicht-abspann.tex
%% Abspann für die Leseansicht.
%% Der Schalter \ifkorrekturansicht ist bereits durch den Vorspann gesetzt.

%% latex-abspann.tex
%% Gemeinsamer Abspann für Korrekturansicht und Leseansicht.
%% Setzt den Schalter \ifkorrekturansicht voraus (gesetzt in den
%% einbindenden Dateien latex-korrekturansicht-abspann.tex bzw.
%% latex-leseansicht-abspann.tex).
%% ---------------------------------------------------------------

\normalsize

% Das esempio-Environment wird nur in der Leseansicht benötigt
\ifkorrekturansicht\else
\newenvironment{esempio}[3]%
{
    \vspace{1.5ex}
    \rlap{\underline{#1}}
    \par
    \setlength{\parindent}{0cm}
    \nopagebreak
    \leftskip=#2cm
    \rightskip=#3cm
}
{
    \par
}
\fi

\doendnotes{C}
\bigskip
\vfill

\clearpage

\footnotesize

\ifkorrekturansicht
  \lohead{\textsc{register}}
\fi

% theindex-Environment neu definieren ohne reledmac
\makeatletter
\renewenvironment{theindex}{%
  \ifkorrekturansicht
    \section*{\indexname}%
  \else
    \subsubsection*{Index der erwähnten Entitäten}%
  \fi
  \setlength{\parindent}{0pt}%
  \setlength{\parskip}{0pt plus 0.3pt}%
  \let\item\@idxitem
}{%
  \ifkorrekturansicht\clearpage\fi
}
\makeatother

\IfFileExists{\jobname-pw.ind}{\input{\jobname-pw.ind}}{}

% Quellenangabe nur in der Leseansicht
\ifkorrekturansicht\else
% Fallback-Definitionen, falls die .tex-Datei \titel etc. nicht gesetzt hat
\providecommand{\titel}{}
\providecommand{\editorInnen}{}
\providecommand{\dateiname}{\jobname}

\vspace{3cm}

\vfill

\footnotesize
\textsc{Quelle}: \titel. Herausgegeben von {\editorInnen}. In: \emph{Arthur Schnitzler: Briefwechsel mit Autorinnen und Autoren}.
 Digitale Edition, https://schnitzler-briefe.acdh.oeaw.ac.at/{\dateiname}.html (Stand \today)
\fi

\end{document}


      