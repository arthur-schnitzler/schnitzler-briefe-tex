%% latex-korrekturansicht-vorspann.tex
%% Vorspann für die Korrekturansicht.
%% Lädt die gemeinsame Datei latex-vorspann.tex mit gesetztem Schalter.

\newif\ifkorrekturansicht
\korrekturansichttrue

\input{../tex-inputs/latex-vorspann}


\section[ Felix Salten an Arthur Schnitzler, {[}4. 1.? 1898{]}]{L03278 Felix Salten an Arthur Schnitzler, {[}4. 1.? 1898{]}}
\nopagebreak\mylabel{L03278v}
\rehead{ }\normalsize\beginnumbering\briefempfaengerindex{Schnitzler, Arthur@\textsc{Schnitzler, Arthur}!zzzSalten, Felix@\emph{von Felix Salten}!1898-01-042@{{[}4. 1.? 1898{]}}|(be}
\toendnotes[C]{\smallbreak\pagebreak[2]}\Standort{CUL, Schnitzler, B 89, A 2.}
\physDesc{Brief, 1 Blatt, 1 Seite, 161 Zeichen
\newline{}Handschrift: Bleistift, lateinische Kurrent
\newline{}Schnitzler: mit Bleistift auf das Jahr »98« datiert 
\newline{}Ordnung: mit Bleistift von unbekannter Hand nummeriert: »101« }\toendnotes[C]{\smallbreak}
\pstart
           \noindent{}{\pb}Lieber Arthur, ich kann Ihnen den \label{K_L03278-1v}\edtext{Sitz}{\lemma{\textnormal{\emph{Sitz}}}\Cendnote{\textnormal{Das
                  Korrespondenzstück wurde von Schnitzler nur innerhalb des
                  Jahres 1898 verortet. Gleicht man die 22 in diesem Jahr nachweisbaren
                  Besuche Schnitzlers im Burgtheater\oindex{Burgtheater@\textbf{Burgtheater}, \emph{S.THTR}|pwk} (»Vestibül\oindex{Burgtheater@\textbf{Burgtheater}, \emph{S.THTR}|pwv}«) mit den Erwähnungen Saltens\pwindex{Salten, Felix 06.09.1869 – 08.10.1945@\textsc{Salten, Felix} (06.09.1869 – 08.10.1945), \emph{Schriftsteller/Schriftstellerin, Journalist/Journalistin, Chefredakteur/Chefredakteurin}|pwk} im
                     \emph{Tagebuch}\pwindex{Tagebuch@\emph{Tagebuch}|pwk} in dieser Zeit ab, so ergibt sich
                  nur ein gemeinsamer Besuch, für den Salten\pwindex{Salten, Felix 06.09.1869 – 08.10.1945@\textsc{Salten, Felix} (06.09.1869 – 08.10.1945), \emph{Schriftsteller/Schriftstellerin, Journalist/Journalistin, Chefredakteur/Chefredakteurin}|pwk}
                  die Karten besorgt haben könnte. Demnach ist hier von der Aufführung von \emph{König Oidipus}\pwindex{Koenig Oedipus. Tragoedie in einem Aufzuge@\emph{König Ödipus. Tragödie in einem Aufzuge}|pwk} und \emph{Hanneles Himmelfahrt}\pwindex{Hanneles Himmelfahrt. Traumdichtung in zwei Teilen@\emph{Hanneles Himmelfahrt. Traumdichtung in zwei Teilen}|pwk} am 4. 1. 1898 die Rede. }}}\label{K_L03278-1} jetzt nicht
               schicken, weil der Diener eine Dummheit gemacht hat. Treffen wir uns also
                  Abends um \label{K_L03278-2v}\edtext{¼ 8}{\lemma{\textnormal{\emph{¼ 8}}}\Cendnote{\textnormal{7 Uhr 15}}}\label{K_L03278-2} im Vestibül\oindex{Burgtheater@\textbf{Burgtheater}, \emph{S.THTR}|pwv}.\pend
           
\pstart
           Herzlich Ihr {\\[\baselineskip]}\spacefill\mbox{Salten}\pend
           \leftskip=0em{}\selectlanguage{ngerman}\endnumbering\briefempfaengerindex{Schnitzler, Arthur@\textsc{Schnitzler, Arthur}!zzzSalten, Felix@\emph{von Felix Salten}!1898-01-042@{{[}4. 1.? 1898{]}}|)be}\mylabel{L03278h}  \normalsize

\doendnotes{C}
\bigskip
\vfill

\clearpage

\footnotesize

\lohead{\textsc{register}}

% Definiere theindex-Environment komplett neu ohne reledmac
\makeatletter
\renewenvironment{theindex}{%
  \section*{\indexname}%
  \setlength{\parindent}{0pt}%
  \setlength{\parskip}{0pt plus 0.3pt}%
  \let\item\@idxitem
}{%
  \clearpage
}
\makeatother

\IfFileExists{\jobname-pw.ind}{\input{\jobname-pw.ind}}{}

\end{document}

      