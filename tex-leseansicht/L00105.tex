%% latex-korrekturansicht-vorspann.tex
%% Vorspann für die Korrekturansicht.
%% Lädt die gemeinsame Datei latex-vorspann.tex mit gesetztem Schalter.

\newif\ifkorrekturansicht
\korrekturansichttrue

\input{../tex-inputs/latex-vorspann}


\section[Hugo von Hofmannsthal an Arthur Schnitzler, 19. 7. {[}1892{]}]{L00105 Hugo von Hofmannsthal an Arthur Schnitzler, 19. 7. {[}1892{]}}
\nopagebreak\mylabel{L00105v}
\rehead{ }\normalsize\beginnumbering\briefempfaengerindex{Schnitzler, Arthur@\textsc{Schnitzler, Arthur}!zzzHofmannsthal, Hugo von@\emph{von Hugo von Hofmannsthal}!1892-07-191@{19. 7. {[}1892{]}}|(be}
\toendnotes[C]{\smallbreak\pagebreak[2]}\Standort{CUL, Schnitzler, B 43.}
\physDesc{Brief, 1 Blatt, 4 Seiten, 2647 Zeichen (aufgeprägtes Wappen)
\newline{}Handschrift: schwarze Tinte, deutsche Kurrent
\newline{}Schnitzler: mit Bleistift die Jahreszahl ergänzt: »92« 
\newline{}Ordnung: mit Bleistift von unbekannter Hand nummeriert:
                                    »26« }
\buchAbdrucke{\weitereDrucke{1) \emph{Die neue Rundschau}, Jg. 41, Nr. 4, April 1930, S. 512–513.} \weitereDrucke{2) Hugo von Hofmannsthal: \emph{Briefe. 1890–1901}. Berlin: \emph{S. Fischer} 1935, S. 48–50.} \weitereDrucke{3) Hugo von Hofmannsthal, Arthur Schnitzler: \emph{Briefwechsel}. Frankfurt am Main: \emph{S. Fischer} 1964, S. 23–24.} \weitereDrucke{4) Hermann Bahr, Arthur Schnitzler: \emph{Briefwechsel, Aufzeichnungen, Dokumente (1891–1931)}. Göttingen: \emph{Wallstein} 2018, S. 25.} }\toendnotes[C]{\smallbreak}
\pstart
           \raggedleft{}{\pb}Fuſch\oindex{Bad Fusch@\textbf{Bad Fusch}, \emph{A.ADM3}|pw}. 19. Juli.\pend
           
\pstart{}lieber Arthur,\pend\vspace{0.5em}
\pstart
           an Ihrem guten und lieben Brief ſtört mich nur die Nachricht, wie viel Arbeit Sie
               sich jetzt zumuthen wollen. Deshalb wünſche ich für Sie ſoſehr den äußeren Erfolg,
               den Sie als Künſtler vor ſich ſelbſt und vor uns gewiſs nicht nothwendig haben, damit
               ſich die Perſpectiven, in denen Sie ſelbſt und Ihr Vater\pwindex{Schnitzler, Johann 10.04.1835 – 02.05.1893@\textsc{Schnitzler, Johann} (10.04.1835 – 02.05.1893), \emph{Laryngologe/Laryngologin}|pwv} Ihr äußeres Leben, Ziele,
                  Pflichten\strikeout{,} und Stil der Lebensführung, anſchauen,
               endlich ändern. Vorläufig iſt es ja ſehr gut, daſs Sie nachts ſchaffen und ſo reich
               und lebhaft aufnehmen können, wie Ihre Hebbel\pwindex{Hebbel, Friedrich 18.03.1813 – 13.12.1863@\textsc{Hebbel, Friedrich} (18.03.1813 – 13.12.1863), \emph{Schriftsteller/Schriftstellerin}|pw}eindrücke dies zeigen. Gewiſs iſt Hebbel\pwindex{Hebbel, Friedrich 18.03.1813 – 13.12.1863@\textsc{Hebbel, Friedrich} (18.03.1813 – 13.12.1863), \emph{Schriftsteller/Schriftstellerin}|pw} ein ſehr großer, tiefer und reicher Geiſt, mit den innerlichſten und
               eindringendſten {\pb}Anſchauungen vom
               Weſen der Naturdinge und des Menſchen, aufwühlend und anregend wie keiner ſonſt,
               sodaſs ſich einem die geheimſten, ſonſt erſtarrten inneren Tiefen regen und das
               eigentlich Dämoniſche in uns, das naturverwandte, dumpf und berauſchend mittönt. Eine
               Überſchrift bei Goethe\pwindex{Goethe, Johann Wolfgang von 1749-08-28 – 1832-03-22@\textsc{Goethe, Johann Wolfgang von} (1749-08-28 – 1832-03-22), \emph{Schriftsteller/Schriftstellerin}|pw} irgendwo: »Urworte; orphiſch\pwindex{Urworte. Orphisch@\emph{Urworte. Orphisch}|pw}« ſuggeriert mir immer den Duft
               der Poeſie Hebbels\pwindex{Hebbel, Friedrich 18.03.1813 – 13.12.1863@\textsc{Hebbel, Friedrich} (18.03.1813 – 13.12.1863), \emph{Schriftsteller/Schriftstellerin}|pw}.\pend
           
\pstart
           Papa\pwindex{Hofmannsthal, Hugo August von 21.12.1841 – 08.12.1915@\textsc{Hofmannsthal, Hugo August von} (21.12.1841 – 08.12.1915), \emph{Bankdirektor/Bankdirektorin}|pwv} iſt befriedigend wohl
               und grüßt Sie, Bahr\pwindex{Bahr, Hermann 19.07.1863 – 15.01.1934@\textsc{Bahr, Hermann} (19.07.1863 – 15.01.1934), \emph{Schriftsteller/Schriftstellerin, Kritiker/Kritikerin}|pw} und Salten\pwindex{Salten, Felix 06.09.1869 – 08.10.1945@\textsc{Salten, Felix} (06.09.1869 – 08.10.1945), \emph{Schriftsteller/Schriftstellerin, Journalist/Journalistin, Chefredakteur/Chefredakteurin}|pw}.\pend
           
\pstart
           Ich habe mich vor einer gewiſſen inneren Öde und Abſpannung in die Tragödie\pwindex{Ascanio und Gioconda@\emph{Ascanio und Gioconda}|pwv} gerettet; eine 5 actige Renaiſſancetragödie\pwindex{Ascanio und Gioconda@\emph{Ascanio und Gioconda}|pwv},
               dramatiſierte Novelle, äußerlich im Stil von Romeo u.
                  Julie\pwindex{Romeo and Juliet@\emph{Romeo and Juliet}|pw}, für die wirkliche brutale Bühne gearbeitet, mit {\pb}großem, ſchlankem Aufbau und
               grellen Farbenflecken, Freskotechnik; ich hoffe vorläufig noch genug lebendige
               Pſychologie in mir zu haben, um das große Gerippe mit lebendigem Fleiſch zu
               umkleiden; ich arbeite ohne Scenarium, mit einzelnen, ſuggeſtiven Notizen;
               geſchrieben habe ich bis jetzt ein paar Scenen aus dem 2\textsuperscript{ten} und eine aus dem 5\textsuperscript{ten} Act; das iſt zwar
               nicht viel aber ich ſehe alles andere recht deutlich und arbeite leicht. Was mich
               lockt und worauf ich eigentlich innerlich hinarbeite, iſt die eigenthümlich
               dunkelglühende, dionyſiſche Luſt im Erfinden und Ausführen tragiſcher Menſchen in
               tragiſchen Situationen; dieſe Luſt, deren ſymboliſches Aequivalent etwa das Anhören
                  {\pb}feierlicher,
               prunkvoll-trauriger Muſik iſt oder das Anſchauen mancher Bilder der \textsc{Renaissance}, mit dunkelgoldnen Panzern und blaſſen ſchönen
               Profilen auf ſehr finſterem Grund. Es wäre ſehr schön, wenn Octobernachmittage
               würden, mit dieſen zwei Leſepremièren. Wie weit iſt die Familie\pwindex{Familie@\emph{Familie}|pw}? \hspace*{2em}\textsc{Richard\pwindex{Beer-Hofmann, Richard 1866-07-11 – 1945-09-26@\textsc{Beer-Hofmann, Richard} (1866-07-11 – 1945-09-26), \emph{Schriftsteller/Schriftstellerin}|pw}}{ }ſchreibt mir, ungern und nur weil er von Papas\pwindex{Hofmannsthal, Hugo August von 21.12.1841 – 08.12.1915@\textsc{Hofmannsthal, Hugo August von} (21.12.1841 – 08.12.1915), \emph{Bankdirektor/Bankdirektorin}|pwv} Krankheit gehört hat;
               er ist verſtimmt, arbeitet aber doch an einer ſeiner Novellen\pwindex{Kind@\emph{Das Kind}|pwv}. Wann iſt Ihre Waffenübung? was
               ist es mit der Verlagsanſtalt für Anatol\pwindex{Anatol@\emph{Anatol}|pw}? laſſen
               Sie ſich doch ja nicht durch ganz gleichgiltige Miſserfolge vom Weiterſuchen
               abſchrecken. Bitte, ſchreiben Sie mir bald, Briefe bekommen iſt hier das
               luſtigſte.\pend
           \pstart \spacefill\mbox{Loris.}\pend{}\selectlanguage{ngerman}\endnumbering\briefempfaengerindex{Schnitzler, Arthur@\textsc{Schnitzler, Arthur}!zzzHofmannsthal, Hugo von@\emph{von Hugo von Hofmannsthal}!1892-07-191@{19. 7. {[}1892{]}}|)be}\mylabel{L00105h}  \normalsize

\doendnotes{C}
\bigskip
\vfill

\clearpage

\footnotesize

\lohead{\textsc{register}}

% Definiere theindex-Environment komplett neu ohne reledmac
\makeatletter
\renewenvironment{theindex}{%
  \section*{\indexname}%
  \setlength{\parindent}{0pt}%
  \setlength{\parskip}{0pt plus 0.3pt}%
  \let\item\@idxitem
}{%
  \clearpage
}
\makeatother

\IfFileExists{\jobname-pw.ind}{\input{\jobname-pw.ind}}{}

\end{document}

      