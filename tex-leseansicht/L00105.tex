%% latex-leseansicht-vorspann.tex
%% Vorspann für die Leseansicht.
%% Lädt die gemeinsame Datei latex-vorspann.tex mit nicht gesetztem Schalter.

\newif\ifkorrekturansicht
\korrekturansichtfalse

\input{../tex-inputs/latex-vorspann}


\section[Hugo von Hofmannsthal an Arthur Schnitzler, 19. 7. {[}1892{]}]{L00105 Hugo von Hofmannsthal an Arthur Schnitzler, 19. 7. [1892]}
\nopagebreak\mylabel{L00105v}
\rehead{ }\normalsize\beginnumbering\briefempfaengerindex{Schnitzler, Arthur@\textsc{Schnitzler, Arthur}!zzzHofmannsthal, Hugo von@\emph{von Hugo von Hofmannsthal}!1892-07-191@{19. 7. [1892]}|(be}
\toendnotes[C]{\smallbreak\pagebreak[2]}
\correspDesc{Versand  durch Hugo von Hofmannsthal am 19. 7. [1892] in Bad Fusch
\newline{}Erhalt  durch Arthur Schnitzler im Zeitraum [20. 7. 1892
                  – 24. 7. 1892?] in Wien}\toendnotes[C]{\smallbreak}
\Standort{CUL, Schnitzler, B 43.}
\physDesc{Brief, 1 Blatt, 4 Seiten, 2647 Zeichen (aufgeprägtes Wappen)
\newline{}Handschrift: schwarze Tinte, deutsche Kurrent
\newline{}Schnitzler: mit Bleistift die Jahreszahl ergänzt: »92« 
\newline{}Ordnung: mit Bleistift von unbekannter Hand nummeriert:
                                    »26« }
\buchAbdrucke{\weitereDrucke{1) Hugo von Hofmannsthal: \emph{Briefe an Freunde.} In: \emph{Die neue Rundschau}, Jg. 41, Nr. 4, April 1930, S. 512–513.} \weitereDrucke{2) Hugo von Hofmannsthal: \emph{Briefe. 1890–1901}. Berlin: \emph{S. Fischer} 1935, S. 48–50.} \weitereDrucke{3) Hugo von Hofmannsthal, Arthur Schnitzler: \emph{Briefwechsel}. Herausgegeben von Therese Nickl und Heinrich Schnitzler. Frankfurt am Main: \emph{S. Fischer} 1964, S. 23–24.} \weitereDrucke{4) Hermann Bahr, Arthur Schnitzler: \emph{Briefwechsel, Aufzeichnungen, Dokumente (1891–1931)}. Herausgegeben von Kurt Ifkovits und Martin Anton Müller. Göttingen: \emph{Wallstein} 2018, S. 25.} }\toendnotes[C]{\smallbreak}
\pstart
           \raggedleft{}{\pb}Fuſch\oindex{Bad Fusch@\textbf{Bad Fusch}|pw}. 19. Juli.\pend
           
\pstart{}lieber Arthur,\pend\vspace{0.5em}
\pstart
           an Ihrem guten und lieben Brief{ }ſtört mich nur die Nachricht, wie viel Arbeit Sie
               sich jetzt zumuthen wollen. Deshalb wünſche ich für Sie{ }ſoſehr den äußeren Erfolg,
               den Sie als Künſtler vor{ }ſich{ }ſelbſt und vor uns gewiſs nicht nothwendig haben, damit{ }ſich die Perſpectiven, in denen Sie{ }ſelbſt und Ihr Vater\pwindex{Schnitzler, Johann 10.\,4.\,1835 Nagykanizsa – 2.\,5.\,1893 Wien@\textsc{Schnitzler, Johann} (10.\,4.\,1835 Nagykanizsa – 2.\,5.\,1893 Wien), \emph{Laryngologe}|pwv} Ihr äußeres Leben, Ziele,
                  Pflichten\strikeout{,} und Stil der Lebensführung, anſchauen,
               endlich ändern. Vorläufig iſt es ja{ }ſehr gut, daſs Sie nachts{ }ſchaffen und{ }ſo reich
               und lebhaft aufnehmen können, wie Ihre Hebbel\pwindex{Hebbel, Friedrich 18.\,3.\,1813 Wesselburen – 13.\,12.\,1863 Wien@\textsc{Hebbel, Friedrich} (18.\,3.\,1813 Wesselburen – 13.\,12.\,1863 Wien), \emph{Schriftsteller}|pw}eindrücke dies zeigen. Gewiſs iſt Hebbel\pwindex{Hebbel, Friedrich 18.\,3.\,1813 Wesselburen – 13.\,12.\,1863 Wien@\textsc{Hebbel, Friedrich} (18.\,3.\,1813 Wesselburen – 13.\,12.\,1863 Wien), \emph{Schriftsteller}|pw} ein{ }ſehr großer, tiefer und reicher Geiſt, mit den innerlichſten und
               eindringendſten {\pb}Anſchauungen vom
               Weſen der Naturdinge und des Menſchen, aufwühlend und anregend wie keiner{ }ſonſt,
               sodaſs{ }ſich einem die geheimſten,{ }ſonſt erſtarrten inneren Tiefen regen und das
               eigentlich Dämoniſche in uns, das naturverwandte, dumpf und berauſchend mittönt. Eine
               Überſchrift bei Goethe\pwindex{Goethe, Johann Wolfgang von 28.\,8.\,1749 Frankfurt am Main – 22.\,3.\,1832 Weimar@\textsc{Goethe, Johann Wolfgang von} (28.\,8.\,1749 Frankfurt am Main – 22.\,3.\,1832 Weimar), \emph{Schriftsteller}|pw} irgendwo: »Urworte; orphiſch\pwindex{Goethe, Johann Wolfgang von 28.\,8.\,1749 Frankfurt am Main – 22.\,3.\,1832 Weimar@\textsc{Goethe, Johann Wolfgang von} (28.\,8.\,1749 Frankfurt am Main – 22.\,3.\,1832 Weimar), \emph{Schriftsteller}!Urworte. Orphisch@\strich\emph{Urworte. Orphisch}|pw}«{ }ſuggeriert mir immer den Duft
               der Poeſie Hebbels\pwindex{Hebbel, Friedrich 18.\,3.\,1813 Wesselburen – 13.\,12.\,1863 Wien@\textsc{Hebbel, Friedrich} (18.\,3.\,1813 Wesselburen – 13.\,12.\,1863 Wien), \emph{Schriftsteller}|pw}.\pend
           
\pstart
           Papa\pwindex{Hofmannsthal, Hugo August von 21.\,12.\,1841 Wien – 8.\,12.\,1915 ebd.@\textsc{Hofmannsthal, Hugo August von} (21.\,12.\,1841 Wien – 8.\,12.\,1915 ebd.), \emph{Bankdirektor}|pwv} iſt befriedigend wohl
               und grüßt Sie, Bahr\pwindex{Bahr, Hermann 19.\,7.\,1863 Linz – 15.\,1.\,1934 München@\textsc{Bahr, Hermann} (19.\,7.\,1863 Linz – 15.\,1.\,1934 München), \emph{Schriftsteller, Kritiker}|pw} und Salten\pwindex{Salten, Felix 6.\,9.\,1869 Budapest – 8.\,10.\,1945 Zürich@\textsc{Salten, Felix} (6.\,9.\,1869 Budapest – 8.\,10.\,1945 Zürich), \emph{Schriftsteller, Journalist, Chefredakteur}|pw}.\pend
           
\pstart
           Ich habe mich vor einer gewiſſen inneren Öde und Abſpannung in die Tragödie\pwindex{Hofmannsthal, Hugo von 1.\,2.\,1874 Wien – 15.\,7.\,1929 Rodaun@\textsc{Hofmannsthal, Hugo von} (1.\,2.\,1874 Wien – 15.\,7.\,1929 Rodaun), \emph{Schriftsteller}!Ascanio und Gioconda@\strich\emph{Ascanio und Gioconda}|pwv} gerettet; eine 5 actige Renaiſſancetragödie\pwindex{Hofmannsthal, Hugo von 1.\,2.\,1874 Wien – 15.\,7.\,1929 Rodaun@\textsc{Hofmannsthal, Hugo von} (1.\,2.\,1874 Wien – 15.\,7.\,1929 Rodaun), \emph{Schriftsteller}!Ascanio und Gioconda@\strich\emph{Ascanio und Gioconda}|pwv},
               dramatiſierte Novelle, äußerlich im Stil von Romeo u.
                  Julie\pwindex{\textcolor{red}{\textsuperscript{XXXX indx1}}!Romeo and Juliet@\strich\emph{Romeo and Juliet}|pw}\pwindex{\textcolor{red}{\textsuperscript{XXXX indx1}}!Romeo and Juliet@\strich\emph{Romeo and Juliet}|pw}, für die wirkliche brutale Bühne gearbeitet, mit {\pb}großem,{ }ſchlankem Aufbau und
               grellen Farbenflecken, Freskotechnik; ich hoffe vorläufig noch genug lebendige
               Pſychologie in mir zu haben, um das große Gerippe mit lebendigem Fleiſch zu
               umkleiden; ich arbeite ohne Scenarium, mit einzelnen,{ }ſuggeſtiven Notizen;
               geſchrieben habe ich bis jetzt ein paar Scenen aus dem 2\textsuperscript{ten} und eine aus dem 5\textsuperscript{ten} Act; das iſt zwar
               nicht viel aber ich{ }ſehe alles andere recht deutlich und arbeite leicht. Was mich
               lockt und worauf ich eigentlich innerlich hinarbeite, iſt die eigenthümlich
               dunkelglühende, dionyſiſche Luſt im Erfinden und Ausführen tragiſcher Menſchen in
               tragiſchen Situationen; dieſe Luſt, deren{ }ſymboliſches Aequivalent etwa das Anhören
                  {\pb}feierlicher,
               prunkvoll-trauriger Muſik iſt oder das Anſchauen mancher Bilder der \textsc{Renaissance}, mit dunkelgoldnen Panzern und blaſſen{ }ſchönen
               Profilen auf{ }ſehr finſterem Grund. Es wäre{ }ſehr schön, wenn Octobernachmittage
               würden, mit dieſen zwei Leſepremièren. Wie weit iſt die Familie\pwindex{Schnitzler, Arthur 15.\,5.\,1862 Wien – 21.\,10.\,1931 ebd.@\textsc{Schnitzler, Arthur} (15.\,5.\,1862 Wien – 21.\,10.\,1931 ebd.), \emph{Schriftsteller, Mediziner}!Familie@\strich\emph{Familie}|pw}? \hspace*{2em}\textsc{Richard\pwindex{Beer-Hofmann, Richard 11.\,7.\,1866 Wien – 26.\,9.\,1945 New York City@\textsc{Beer-Hofmann, Richard} (11.\,7.\,1866 Wien – 26.\,9.\,1945 New York City), \emph{Schriftsteller}|pw}}{ }ſchreibt mir, ungern und nur weil er von Papas\pwindex{Hofmannsthal, Hugo August von 21.\,12.\,1841 Wien – 8.\,12.\,1915 ebd.@\textsc{Hofmannsthal, Hugo August von} (21.\,12.\,1841 Wien – 8.\,12.\,1915 ebd.), \emph{Bankdirektor}|pwv} Krankheit gehört hat;
               er ist verſtimmt, arbeitet aber doch an einer{ }ſeiner Novellen\pwindex{Beer-Hofmann, Richard 11.\,7.\,1866 Wien – 26.\,9.\,1945 New York City@\textsc{Beer-Hofmann, Richard} (11.\,7.\,1866 Wien – 26.\,9.\,1945 New York City), \emph{Schriftsteller}!Kind@\strich\emph{Das Kind}|pwv}. Wann iſt Ihre Waffenübung? was
               ist es mit der Verlagsanſtalt für Anatol\pwindex{Schnitzler, Arthur 15.\,5.\,1862 Wien – 21.\,10.\,1931 ebd.@\textsc{Schnitzler, Arthur} (15.\,5.\,1862 Wien – 21.\,10.\,1931 ebd.), \emph{Schriftsteller, Mediziner}!Anatol@\strich\emph{Anatol}|pw}? laſſen
               Sie{ }ſich doch ja nicht durch ganz gleichgiltige Miſserfolge vom Weiterſuchen
               abſchrecken. Bitte,{ }ſchreiben Sie mir bald, Briefe bekommen iſt hier das
               luſtigſte.\pend
           \pstart \spacefill\mbox{Loris.}\pend{}\selectlanguage{ngerman}\endnumbering\briefempfaengerindex{Schnitzler, Arthur@\textsc{Schnitzler, Arthur}!zzzHofmannsthal, Hugo von@\emph{von Hugo von Hofmannsthal}!1892-07-191@{19. 7. [1892]}|)be}\mylabel{L00105h}  \newcommand{\dateiname}{L00105}\newcommand{\titel}{Hugo von Hofmannsthal an Arthur Schnitzler, 19. 7. [1892]}\newcommand{\editorInnen}{Herausgegeben von Martin Anton Müller}%% latex-leseansicht-abspann.tex
%% Abspann für die Leseansicht.
%% Der Schalter \ifkorrekturansicht ist bereits durch den Vorspann gesetzt.

%% latex-abspann.tex
%% Gemeinsamer Abspann für Korrekturansicht und Leseansicht.
%% Setzt den Schalter \ifkorrekturansicht voraus (gesetzt in den
%% einbindenden Dateien latex-korrekturansicht-abspann.tex bzw.
%% latex-leseansicht-abspann.tex).
%% ---------------------------------------------------------------

\normalsize

% Das esempio-Environment wird nur in der Leseansicht benötigt
\ifkorrekturansicht\else
\newenvironment{esempio}[3]%
{
    \vspace{1.5ex}
    \rlap{\underline{#1}}
    \par
    \setlength{\parindent}{0cm}
    \nopagebreak
    \leftskip=#2cm
    \rightskip=#3cm
}
{
    \par
}
\fi

\doendnotes{C}
\bigskip
\vfill

\clearpage

\footnotesize

\ifkorrekturansicht
  \lohead{\textsc{register}}
\fi

% theindex-Environment neu definieren ohne reledmac
\makeatletter
\renewenvironment{theindex}{%
  \ifkorrekturansicht
    \section*{\indexname}%
  \else
    \subsubsection*{Index der erwähnten Entitäten}%
  \fi
  \setlength{\parindent}{0pt}%
  \setlength{\parskip}{0pt plus 0.3pt}%
  \let\item\@idxitem
}{%
  \ifkorrekturansicht\clearpage\fi
}
\makeatother

\IfFileExists{\jobname-pw.ind}{\input{\jobname-pw.ind}}{}

% Quellenangabe nur in der Leseansicht
\ifkorrekturansicht\else
% Fallback-Definitionen, falls die .tex-Datei \titel etc. nicht gesetzt hat
\providecommand{\titel}{}
\providecommand{\editorInnen}{}
\providecommand{\dateiname}{\jobname}

\vspace{3cm}

\vfill

\footnotesize
\textsc{Quelle}: \titel. Herausgegeben von {\editorInnen}. In: \emph{Arthur Schnitzler: Briefwechsel mit Autorinnen und Autoren}.
 Digitale Edition, https://schnitzler-briefe.acdh.oeaw.ac.at/{\dateiname}.html (Stand \today)
\fi

\end{document}


