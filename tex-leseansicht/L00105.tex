%% latex-leseansicht-vorspann.tex
%% Vorspann für die Leseansicht.
%% Lädt die gemeinsame Datei latex-vorspann.tex mit nicht gesetztem Schalter.

\newif\ifkorrekturansicht
\korrekturansichtfalse

\input{../tex-inputs/latex-vorspann}

\begin{center}
            \textcolor{red}{ENTWURF. ENTZIFFERUNG NOCH NICHT KORREKTURGELESEN}
                      \end{center}
            
               \section[Hugo von Hofmannsthal an Arthur Schnitzler, 19. 7. {[}1892{]}]{ Hugo von Hofmannsthal an Arthur Schnitzler, 19. 7. {[}1892{]}}\nopagebreak\mylabel{v}\rehead{ }\begin{ledgroupsized}[t]{13cm}\normalsize\beginnumbering\briefempfaengerindex{Schnitzler, Arthur@\textsc{Schnitzler, Arthur}!zzzHofmannsthal, Hugo von@\emph{von Hugo von Hofmannsthal}!1892-07-191@{19. 7. {[}1892{]}}|(be} \toendnotes[C]{\smallbreak\pagebreak[2]} \Standort{CUL, Schnitzler, B 43.}
\physDesc{Brief, 1 Blatt (Briefpapier mit aufgeprägtem Wappen), 4 Seiten
\newline{}Handschrift: schwarze Tinte, deutsche Kurrent
\newline{}Schnitzler: mit Bleistift die Jahreszahl ergänzt: »92« \newline{}Ordnung: mit Bleistift von unbekannter Hand nummeriert:
                                    »26« }\buchAbdrucke{\weitereDrucke{1) Hugo von Hofmannsthal: \emph{Briefe an Freunde.} In: \emph{Die neue Rundschau}, Jg. 41, Nr. 4, April 1930, S. 512–513.} \weitereDrucke{2) Hugo von Hofmannsthal: \emph{Briefe. 1890–1901}. Berlin: \emph{S. Fischer} 1935, S. 48–50.} \weitereDrucke{3) Hugo von Hofmannsthal, Arthur Schnitzler: \emph{Briefwechsel}. Hg. Therese Nickl und Heinrich Schnitzler. Frankfurt am Main: \emph{S. Fischer} 1964, S. 23–24.} \weitereDrucke{4) Hermann Bahr, Arthur Schnitzler: \emph{Briefwechsel, Aufzeichnungen, Dokumente (1891–1931)}. Hg. Kurt Ifkovits und Martin Anton Müller. Göttingen: \emph{Wallstein} 2018, S. 25.} }\toendnotes[C]{\smallbreak}\pstart
           \raggedleft{}{\pb}Fuſch\oindex{Bad Fusch@\textbf{Bad Fusch}|pw}. 19. Juli.\pend
           \pstart{}lieber Arthur,\pend\pstart
           an Ihrem guten und lieben Brief ſtört mich nur die Nachricht, wie viel Arbeit Sie
               sich jetzt zumuthen wollen. Deshalb wünſche ich für Sie ſoſehr den äußeren Erfolg,
               den Sie als Künſtler vor ſich ſelbſt und vor uns gewiſs nicht nothwendig haben, damit
               ſich die Perſpectiven, in denen Sie ſelbſt und Ihr Vater\pwindex{Schnitzler, Johann 10.04.1835 – 02.05.1893@\textsc{Schnitzler, Johann} (10.04.1835 – 02.05.1893), \emph{Laryngologe}|pwv} Ihr äußeres Leben, Ziele, Pflichten\strikeout{,} und Stil der Lebensführung, anſchauen, endlich
               ändern. Vorläufig iſt es ja ſehr gut, daſs Sie nachts ſchaffen und ſo reich und
               lebhaft aufnehmen können, wie Ihre Hebbel\pwindex{Hebbel, Friedrich 18.03.1813 – 13.12.1863@\textsc{Hebbel, Friedrich} (18.03.1813 – 13.12.1863), \emph{Schriftsteller}|pw}eindrücke
               dies zeigen. Gewiſs iſt Hebbel\pwindex{Hebbel, Friedrich 18.03.1813 – 13.12.1863@\textsc{Hebbel, Friedrich} (18.03.1813 – 13.12.1863), \emph{Schriftsteller}|pw} ein ſehr großer,
               tiefer und reicher Geiſt, mit den innerlichſten und eindringendſten {\pb}Anſchauungen vom Weſen der
               Naturdinge und des Menſchen, aufwühlend und anregend wie keiner ſonſt, sodaſs ſich
               einem die geheimſten, ſonſt erſtarrten inneren Tiefen regen und das eigentlich
               Dämoniſche in uns, das naturverwandte, dumpf und berauſchend mittönt. Eine
               Überſchrift bei Goethe\pwindex{Goethe, Johann Wolfgang von 28.08.1749 – 22.03.1832@\textsc{Goethe, Johann Wolfgang von} (28.08.1749 – 22.03.1832), \emph{Schriftsteller}|pw} irgendwo: »Urworte; orphiſch\pwindex{Goethe, Johann Wolfgang von 28.08.1749 – 22.03.1832@\textsc{Goethe, Johann Wolfgang von} (28.08.1749 – 22.03.1832), \emph{Schriftsteller}!Urworte. Orphisch1820@\strich\emph{Urworte. Orphisch} {[}1820{]}|pw}« ſuggeriert mir immer den Duft der Poeſie Hebbel\pwindex{Hebbel, Friedrich 18.03.1813 – 13.12.1863@\textsc{Hebbel, Friedrich} (18.03.1813 – 13.12.1863), \emph{Schriftsteller}|pw}s.\pend
           \pstart
           Papa\pwindex{Hofmannsthal, Hugo August von 21.12.1841 – 08.12.1915@\textsc{Hofmannsthal, Hugo August von} (21.12.1841 – 08.12.1915), \emph{Bankdirektor}|pwv} iſt befriedigend wohl und
               grüßt Sie, Bahr\pwindex{Bahr, Hermann 19.07.1863 – 15.01.1934@\textsc{Bahr, Hermann} (19.07.1863 – 15.01.1934), \emph{Schriftsteller, Kritiker}|pw} und Salten\pwindex{Salten, Felix 06.09.1869 – 08.10.1945@\textsc{Salten, Felix} (06.09.1869 – 08.10.1945), \emph{Schriftsteller, Journalist}|pw}.\pend
           \pstart
           Ich habe mich vor einer gewiſſen inneren Öde und Abſpannung in die Tragödie\pwindex{Hofmannsthal, Hugo von 01.02.1874 – 15.07.1929@\textsc{Hofmannsthal, Hugo von} (01.02.1874 – 15.07.1929), \emph{Schriftsteller}!Ascanio und Gioconda1979@\strich\emph{Ascanio und Gioconda} {[}1979{]}|pwv} gerettet; eine 5 actige Renaiſſancetragödie\pwindex{Hofmannsthal, Hugo von 01.02.1874 – 15.07.1929@\textsc{Hofmannsthal, Hugo von} (01.02.1874 – 15.07.1929), \emph{Schriftsteller}!Ascanio und Gioconda1979@\strich\emph{Ascanio und Gioconda} {[}1979{]}|pwv}, dramatiſierte
               Novelle, äußerlich im Stil von Romeo u. Julie\pwindex{\textcolor{red}{\textsuperscript{XXXX1 indx}}!Romeo und Julia1594@\strich\emph{Romeo und Julia} {[}1594{]}|pw}, für
               die wirkliche brutale Bühne gearbeitet, mit {\pb}großem, ſchlankem Aufbau und
               grellen Farbenflecken, Freskotechnik; ich hoffe vorläufig noch genug lebendige
               Pſychologie in mir zu haben, um das große Gerippe mit lebendigem Fleiſch zu
               umkleiden; ich arbeite ohne Scenarium, mit einzelnen, ſuggeſtiven Notizen;
               geſchrieben habe ich bis jetzt ein paar Scenen aus dem 2\textsuperscript{ten} und eine aus dem 5\textsuperscript{ten} Act; das iſt zwar
               nicht viel aber ich ſehe alles andere recht deutlich und arbeite leicht. Was mich
               lockt und worauf ich eigentlich innerlich hinarbeite, iſt die eigenthümlich
               dunkelglühende, dionyſiſche Luſt im Erfinden und Ausführen tragiſcher Menſchen in
               tragiſchen Situationen; dieſe Luſt, deren ſymboliſches Aequivalent etwa das Anhören
                  {\pb}feierlicher,
               prunkvoll-trauriger Muſik iſt oder das Anſchauen mancher Bilder der \textsc{Renaissance}, mit dunkelgoldnen Panzern und blaſſen ſchönen
               Profilen auf ſehr finſterem Grund. Es wäre ſehr schön, wenn Octobernachmittage
               würden, mit dieſen zwei Leſepremièren. Wie weit iſt die Familie\pwindex{Schnitzler, Arthur 15.05.1862 – 21.10.1931@\textsc{Schnitzler, Arthur} (15.05.1862 – 21.10.1931), \emph{Schriftsteller, Mediziner}!Familie1977@\strich\emph{Familie} {[}1977{]}|pw}? \hspace*{2em}\textsc{Richard\pwindex{Beer-Hofmann, Richard 11.07.1866 – 26.09.1945@\textsc{Beer-Hofmann, Richard} (11.07.1866 – 26.09.1945), \emph{Schriftsteller}|pw}}{ }ſchreibt mir, ungern und nur weil er von Papas\pwindex{Hofmannsthal, Hugo August von 21.12.1841 – 08.12.1915@\textsc{Hofmannsthal, Hugo August von} (21.12.1841 – 08.12.1915), \emph{Bankdirektor}|pwv} Krankheit gehört hat; er
               ist verſtimmt, arbeitet aber doch an einer ſeiner Novellen\pwindex{Beer-Hofmann, Richard 11.07.1866 – 26.09.1945@\textsc{Beer-Hofmann, Richard} (11.07.1866 – 26.09.1945), \emph{Schriftsteller}!Kind1893@\strich\emph{Das Kind} {[}1893{]}|pwv}. Wann iſt Ihre Waffenübung? was ist es mit der
               Verlagsanſtalt für Anatol\pwindex{Schnitzler, Arthur 15.05.1862 – 21.10.1931@\textsc{Schnitzler, Arthur} (15.05.1862 – 21.10.1931), \emph{Schriftsteller, Mediziner}!Anatol1892-10-29 – 1892-10-29@\strich\emph{Anatol} {[}1892-10-29 – 1892-10-29{]}|pw}? laſſen Sie ſich doch ja
               nicht durch ganz gleichgiltige Miſserfolge vom Weiterſuchen abſchrecken. Bitte,
               ſchreiben Sie mir bald, Briefe bekommen iſt hier das luſtigſte.\pend
           \pstart \spacefill\mbox{Loris.}\pend{}\endnumbering\briefempfaengerindex{Schnitzler, Arthur@\textsc{Schnitzler, Arthur}!zzzHofmannsthal, Hugo von@\emph{von Hugo von Hofmannsthal}!1892-07-191@{19. 7. {[}1892{]}}|)be}\mylabel{h}\end{ledgroupsized}  \newcommand{\dateiname}{L00105}\newcommand{\titel}{Hugo von Hofmannsthal an Arthur Schnitzler, 19. 7. [1892]}\newcommand{\editorInnen}{ Martin Anton Müller und Gerd-Hermann Susen}%% latex-leseansicht-abspann.tex
%% Abspann für die Leseansicht.
%% Der Schalter \ifkorrekturansicht ist bereits durch den Vorspann gesetzt.

%% latex-abspann.tex
%% Gemeinsamer Abspann für Korrekturansicht und Leseansicht.
%% Setzt den Schalter \ifkorrekturansicht voraus (gesetzt in den
%% einbindenden Dateien latex-korrekturansicht-abspann.tex bzw.
%% latex-leseansicht-abspann.tex).
%% ---------------------------------------------------------------

\normalsize

% Das esempio-Environment wird nur in der Leseansicht benötigt
\ifkorrekturansicht\else
\newenvironment{esempio}[3]%
{
    \vspace{1.5ex}
    \rlap{\underline{#1}}
    \par
    \setlength{\parindent}{0cm}
    \nopagebreak
    \leftskip=#2cm
    \rightskip=#3cm
}
{
    \par
}
\fi

\doendnotes{C}
\bigskip
\vfill

\clearpage

\footnotesize

\ifkorrekturansicht
  \lohead{\textsc{register}}
\fi

% theindex-Environment neu definieren ohne reledmac
\makeatletter
\renewenvironment{theindex}{%
  \ifkorrekturansicht
    \section*{\indexname}%
  \else
    \subsubsection*{Index der erwähnten Entitäten}%
  \fi
  \setlength{\parindent}{0pt}%
  \setlength{\parskip}{0pt plus 0.3pt}%
  \let\item\@idxitem
}{%
  \ifkorrekturansicht\clearpage\fi
}
\makeatother

\IfFileExists{\jobname-pw.ind}{\input{\jobname-pw.ind}}{}

% Quellenangabe nur in der Leseansicht
\ifkorrekturansicht\else
% Fallback-Definitionen, falls die .tex-Datei \titel etc. nicht gesetzt hat
\providecommand{\titel}{}
\providecommand{\editorInnen}{}
\providecommand{\dateiname}{\jobname}

\vspace{3cm}

\vfill

\footnotesize
\textsc{Quelle}: \titel. Herausgegeben von {\editorInnen}. In: \emph{Arthur Schnitzler: Briefwechsel mit Autorinnen und Autoren}.
 Digitale Edition, https://schnitzler-briefe.acdh.oeaw.ac.at/{\dateiname}.html (Stand \today)
\fi

\end{document}


      