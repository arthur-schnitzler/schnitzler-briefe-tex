%% latex-leseansicht-vorspann.tex
%% Vorspann für die Leseansicht.
%% Lädt die gemeinsame Datei latex-vorspann.tex mit nicht gesetztem Schalter.

\newif\ifkorrekturansicht
\korrekturansichtfalse

\input{../tex-inputs/latex-vorspann}


               \section[Georg Brandes an Arthur Schnitzler, 13. 12. 1915]{ Georg Brandes an Arthur Schnitzler, 13. 12. 1915}\nopagebreak\mylabel{v}\rehead{ }\begin{ledgroupsized}[t]{13cm}\normalsize\beginnumbering\briefempfaengerindex{Schnitzler, Arthur@\textsc{Schnitzler, Arthur}!zzzBrandes, Georg@\emph{von Georg Brandes}!1915-12-131@{13. 12. 1915}|(be} \toendnotes[C]{\smallbreak\pagebreak[2]} \Standort{CUL, Schnitzler, B 17.}
\physDesc{Brief, 1 Blatt, 4 Seiten
\newline{}Handschrift: schwarze Tinte, lateinische Kurrent
\newline{}Schnitzler: 1) mit Bleistift beschriftet: »\textsc{Brandes}« 2) mit rotem Buntstift vereinzelte Unterstreichungen\newline{}Ordnung: mit Bleistift von unbekannter Hand nummeriert:
                                    »46« }\buchAbdrucke{\weitereDrucke{Georg Brandes, Arthur Schnitzler: \emph{Ein Briefwechsel}. Hg. Kurt Bergel. Bern: \emph{Francke} 1956, S. 119–120.} }\toendnotes[C]{\smallbreak}\pstart
           \raggedleft{}{\pb}Kopenhagen\oindex{Kopenhagen@\textbf{Kopenhagen}|pw} d. 13 December 15\pend
           \pstart
           Verehrter Freund \hspace*{1.5em}Es war mir eine angenehme Ueberraschung, so bald
               von Ihnen zu hören. Ich erwartete das nicht. Ich bin leider bettlägerig. Sie wissen
               als Artzt, wie langwierig dies verdammte Uebel ist, gegen welches das Serum erst zehn
               Jahre nach meinem Tode gefunden wird. Warme Umschläge imponiren den Bacillen nicht,
               und ich kann ihnen das nicht verdenken.\pend
           \pstart
           Peter Nansen\pwindex{Nansen, Peter 20.01.1861 – 31.07.1918@\textsc{Nansen, Peter} (20.01.1861 – 31.07.1918), \emph{Schriftsteller, Journalist, Verleger}|pw} soll besser sein. Es ist nur eine
               Bronchitis, die ein schwaches Fieber verursacht.\pend
           \pstart
           Sie haben ja völlig und unbestrittenes Recht, wenn Sie behaupten, als Dramatiker
               nicht mit irgend einer Ihrer Persönlichkeiten identisch zu sein. Aber die Schlussscene\pwindex{Schnitzler, Arthur 15.05.1862 – 21.10.1931@\textsc{Schnitzler, Arthur} (15.05.1862 – 21.10.1931), \emph{Schriftsteller, Mediziner}!Professor Bernhardi. Komoedie in fuenf Akten1912@\strich\emph{Professor Bernhardi. Komödie in fünf Akten} {[}1912{]}|pwv} scheint mir jedoch
               den Totaleindruck zusammenfassen zu sollen. {\pb}Ueber die Feierlichen denke ich
               natürlich wie Sie. Da ich die 20 Jahre älter bin gewiss mit noch grösserem Widerwille
               als Sie.\pend
           \pstart
           Was Sie über die Kritiker sagen erstaunt mich nicht; ich kenne nichts widerlicheres
               und dümmeres.\pend
           \pstart
           Lassalle\pwindex{Lassalle, Ferdinand 11.04.1825 – 31.08.1864@\textsc{Lassalle, Ferdinand} (11.04.1825 – 31.08.1864), \emph{Schriftsteller, Politiker, Publizist}|pw} sagte »\label{K_L02223_1v}\edtext{Zwei Arten von Menschen sind mir vor Allem verhasst
               Journalisten und Juden – und ich bin beides.}{\lemma{\textnormal{\emph{Zwei … beides.}}}\Cendnote{\textnormal{Nicht genauer nachgewiesenes Zitat in: \emph{Ferdinand Lassalle\pwindex{Lassalle, Ferdinand 11.04.1825 – 31.08.1864@\textsc{Lassalle, Ferdinand} (11.04.1825 – 31.08.1864), \emph{Schriftsteller, Politiker, Publizist}|pwk}’s Briefe an Georg Herwegh\pwindex{Herwegh, Georg 31.05.1817 – 07.04.1875@\textsc{Herwegh, Georg} (31.05.1817 – 07.04.1875), \emph{Schriftsteller/Schriftstellerin, Übersetzer/Übersetzerin}|pwk}. Nebst Briefen der Gräfin
                           Sophie Hatzfeldt\pwindex{Hatzfeld, Sophie 10.08.1805 – 25.01.1881@\textsc{Hatzfeld, Sophie} (10.08.1805 – 25.01.1881), \emph{Sozialistin}|pwk} an Frau Emma Herwegh\pwindex{Herwegh, Emma 10.05.1817 – 24.03.1904@\textsc{Herwegh, Emma} (10.05.1817 – 24.03.1904), \emph{Revolutionärin}|pwk}}\pwindex{Lassalle, Ferdinand 11.04.1825 – 31.08.1864@\textsc{Lassalle, Ferdinand} (11.04.1825 – 31.08.1864), \emph{Schriftsteller, Politiker, Publizist}!Ferdinand Lassalle s Briefe an Georg Herwegh1896@\strich\emph{Ferdinand Lassalle’s Briefe an Georg Herwegh} {[}1896{]}|pwk}\pwindex{Herwegh, Georg 31.05.1817 – 07.04.1875@\textsc{Herwegh, Georg} (31.05.1817 – 07.04.1875), \emph{Schriftsteller/Schriftstellerin, Übersetzer/Übersetzerin}!Ferdinand Lassalle s Briefe an Georg Herwegh1896@\strich\emph{Ferdinand Lassalle’s Briefe an Georg Herwegh} {[}1896{]}|pwk}. Herausgegeben von Marcel Herwegh\pwindex{Herwegh, Marcel 14.05.1858 – 1937@\textsc{Herwegh, Marcel} (14.05.1858 – 1937), \emph{Schriftsteller, Musiker}|pwk}. Mit einem Bild und Brief Lassalle’s\pwindex{Lassalle, Ferdinand 11.04.1825 – 31.08.1864@\textsc{Lassalle, Ferdinand} (11.04.1825 – 31.08.1864), \emph{Schriftsteller, Politiker, Publizist}|pwk}. Zürich: \emph{Albert
                        Müller’s Verlag}{ }1896, S. 4–5: »›Zwei Dinge in der Welt‹ – pflegte
                        Lassalle\pwindex{Lassalle, Ferdinand 11.04.1825 – 31.08.1864@\textsc{Lassalle, Ferdinand} (11.04.1825 – 31.08.1864), \emph{Schriftsteller, Politiker, Publizist}|pw} zu sagen – ›sind mir vor Allem
                     verhaßt: Journalisten und Juden; und Beides bin ich!‹«.}}}\label{K_L02223_1h}«– Ich
               hasse die Kritiker und verachte sie, besonders die moralisierenden.\pend
           \pstart
           Einen Punkt muss ich beantworten, eine schwache \substVorne{}\textsuperscript{Andeutung}{\allowbreak}\substDazwischen{}Anspielung\substHinten{}. Sie sagen, ich wisse wohl jetzt mehr über den Krieg als im Anfang. Einst
               schrieben Sie mir ebenfalls, ich solle doch nicht glauben, in Wien\oindex{Wien@\textbf{Wien}|pw} herrsche \label{K_L02223_2v}\edtext{Hungersnoth}{\lemma{\textnormal{\emph{Hungersnoth}}}\Cendnote{\textnormal{Arthur Schnitzler an Georg Brandes, 20. 10. 1914}}}\label{K_L02223_2h}. {\pb}Ich vergass damals zu
               antworten.\pend
           \pstart
           Irgend ein erbärmlicher Wicht von Journalist\pwindex{?? [Daenischer Journalist] *~1915@\textsc{?? [Dänischer Journalist]} (*~1915)|pwv}, der in einem dänischen\oindex{Daenemark@\textbf{Dänemark}|pw}{ }Blatt irgend einen der gewöhnlichen idiotischen Artikel\pwindex{?? [Korrespondent aus Russland] *~1914@\textsc{?? [Korrespondent aus Russland]} (*~1914)!?? [Russischer Korrespondentenbericht]1914@\strich\emph{?? [Russischer Korrespondentenbericht]} {[}1914{]}|pwv} von einem sogenannt russischen\oindex{Russland@\textbf{Russland}|pw}{ }Correspondenten\pwindex{?? [Korrespondent aus Russland] *~1914@\textsc{?? [Korrespondent aus Russland]} (*~1914)|pwv} gelesen hatte,
               bekam den Einfall, im Anfang des Krieges, \uline{mich}
               deshalb anzugreifen, \uline{mich} dafür verantwortlich zu
               machen. Darin soll gestanden haben, in Wien\oindex{Wien@\textbf{Wien}|pw} hungere
               man.\pend
           \pstart
           Ich hatte den Artikel\pwindex{?? [Korrespondent aus Russland] *~1914@\textsc{?? [Korrespondent aus Russland]} (*~1914)!?? [Russischer Korrespondentenbericht]1914@\strich\emph{?? [Russischer Korrespondentenbericht]} {[}1914{]}|pwv}{ }\uline{nie gelesen}, nie gesehen, viel weniger geschrieben
               oder aufgenommen. Nun ging diese \label{K_L02223_3v}\edtext{Idiotie}{\lemma{\textnormal{\emph{Idiotie}}}\Cendnote{\textnormal{nicht nachgewiesen}}}\label{K_L02223_3h}
               wie ein Lauffeuer durch die deutsche\oindex{Deutschland@\textbf{Deutschland}|pw} und österreichische\oindex{Oesterreich@\textbf{Österreich}|pw} Presse, mit imbecilen Schimpfworten
               gegen mich.\pend
           \pstart
           Sie scheinen daran geglaubt zu haben. So sind wir alle. Wie viele tausend Mal wir
               erfahren haben, dass das Gedruckte {\pb}nur Lüge war, immer wieder glauben wir an etwas.\pend
           \pstart
           Mein junger Schwiegersohn\pwindex{Philipp, Reinhold 15.08.1883 – 1968@\textsc{Philipp, Reinhold} (15.08.1883 – 1968), \emph{Fabrikant}|pwv} ist
               noch nicht verwundet, aber leidet grässlich an \uline{Leere},
               fühlt es, als verliere er den Verstand, werde alt und grau. Es ist immer besser als
               Wunden und Tod.\pend
           \pstart
           Ich bin von ganzem Herzen Ihr Freund{\\[\baselineskip]}\spacefill\mbox{G B}\pend
           \leftskip=0em{}\endnumbering\briefempfaengerindex{Schnitzler, Arthur@\textsc{Schnitzler, Arthur}!zzzBrandes, Georg@\emph{von Georg Brandes}!1915-12-131@{13. 12. 1915}|)be}\mylabel{h}\end{ledgroupsized}  \newcommand{\dateiname}{L02223}\newcommand{\titel}{Georg Brandes an Arthur Schnitzler, 13. 12. 1915}\newcommand{\editorInnen}{Martin Anton Müller und Gerd-Hermann Susen}%% latex-leseansicht-abspann.tex
%% Abspann für die Leseansicht.
%% Der Schalter \ifkorrekturansicht ist bereits durch den Vorspann gesetzt.

%% latex-abspann.tex
%% Gemeinsamer Abspann für Korrekturansicht und Leseansicht.
%% Setzt den Schalter \ifkorrekturansicht voraus (gesetzt in den
%% einbindenden Dateien latex-korrekturansicht-abspann.tex bzw.
%% latex-leseansicht-abspann.tex).
%% ---------------------------------------------------------------

\normalsize

% Das esempio-Environment wird nur in der Leseansicht benötigt
\ifkorrekturansicht\else
\newenvironment{esempio}[3]%
{
    \vspace{1.5ex}
    \rlap{\underline{#1}}
    \par
    \setlength{\parindent}{0cm}
    \nopagebreak
    \leftskip=#2cm
    \rightskip=#3cm
}
{
    \par
}
\fi

\doendnotes{C}
\bigskip
\vfill

\clearpage

\footnotesize

\ifkorrekturansicht
  \lohead{\textsc{register}}
\fi

% theindex-Environment neu definieren ohne reledmac
\makeatletter
\renewenvironment{theindex}{%
  \ifkorrekturansicht
    \section*{\indexname}%
  \else
    \subsubsection*{Index der erwähnten Entitäten}%
  \fi
  \setlength{\parindent}{0pt}%
  \setlength{\parskip}{0pt plus 0.3pt}%
  \let\item\@idxitem
}{%
  \ifkorrekturansicht\clearpage\fi
}
\makeatother

\IfFileExists{\jobname-pw.ind}{\input{\jobname-pw.ind}}{}

% Quellenangabe nur in der Leseansicht
\ifkorrekturansicht\else
% Fallback-Definitionen, falls die .tex-Datei \titel etc. nicht gesetzt hat
\providecommand{\titel}{}
\providecommand{\editorInnen}{}
\providecommand{\dateiname}{\jobname}

\vspace{3cm}

\vfill

\footnotesize
\textsc{Quelle}: \titel. Herausgegeben von {\editorInnen}. In: \emph{Arthur Schnitzler: Briefwechsel mit Autorinnen und Autoren}.
 Digitale Edition, https://schnitzler-briefe.acdh.oeaw.ac.at/{\dateiname}.html (Stand \today)
\fi

\end{document}


      