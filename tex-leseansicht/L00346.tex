%% latex-leseansicht-vorspann.tex
%% Vorspann für die Leseansicht.
%% Lädt die gemeinsame Datei latex-vorspann.tex mit nicht gesetztem Schalter.

\newif\ifkorrekturansicht
\korrekturansichtfalse

\input{../tex-inputs/latex-vorspann}


\section[Peter Altenberg an Arthur Schnitzler, {[}5.? 7. 1894{]}]{L00346 Peter Altenberg an Arthur Schnitzler, {[}5.? 7. 1894{]}}
\nopagebreak\mylabel{L00346v}
\rehead{ }\normalsize\beginnumbering\briefempfaengerindex{Schnitzler, Arthur@\textsc{Schnitzler, Arthur}!zzzAltenberg, Peter@\emph{von Peter Altenberg}!1894-07-052@{{[}5.? 7. 1894{]}}|(be}
\toendnotes[C]{\smallbreak\pagebreak[2]}
\correspDesc{Versand  durch Peter Altenberg am [5.? 7. 1894] \textbf{Ort fehlend} 
\newline{}Erhalt  durch Arthur Schnitzler im Zeitraum [5. 7. 1894
                  – 9. 7. 1894?] in Wien}\toendnotes[C]{\smallbreak}
\Standort{CUL, Schnitzler, B 2.}
\physDesc{Brief, 1 Blatt, 1 Seite, 163 Zeichen
\newline{}Handschrift: schwarze Tinte, deutsche Kurrent
\newline{}Schnitzler: 1) mit rotem Buntstift eine Unterstreichung  2) mit Bleistift datiert: »Anf Juli 94.« und nummeriert: »2«
\newline{}Ordnung: mit Bleistift von unbekannter Hand nummeriert:
                                 »2« }
\buchAbdrucke{\weitereDrucke{Peter Altenberg: \emph{Die Selbsterfindung eines Dichters. Briefe und Dokumente
                        1892–1896}. Herausgegeben und mit einem Nachwort von Leo A. Lensing. Göttingen: \emph{Wallstein} 2009, S. 23.} }\toendnotes[C]{\smallbreak}
\pstart{}{\pb}Lieber \textsc{D\textsuperscript{r.}} Arthur Schnitzler.\pend\vspace{0.5em}
\pstart
           Auf ihren Wunſch{ }ſende ich Ihnen eine Skizze »\label{K_L00346-1v}\edtext{See-Ufer\pwindex{Altenberg, Peter 9.\,3.\,1859 Wien – 8.\,1.\,1919 ebd.@\textsc{Altenberg, Peter} (9.\,3.\,1859 Wien – 8.\,1.\,1919 ebd.), \emph{Schriftsteller}!See-Ufer@\strich\emph{See-Ufer}|pw}}{\lemma{\textnormal{\emph{See-Ufer}}}\Cendnote{\textnormal{Die Skizze ist nicht überliefert, sehr
                  wohl aber verwendete Altenberg\pwindex{Altenberg, Peter 9.\,3.\,1859 Wien – 8.\,1.\,1919 ebd.@\textsc{Altenberg, Peter} (9.\,3.\,1859 Wien – 8.\,1.\,1919 ebd.), \emph{Schriftsteller}|pwk} den »See-Ufer« als
                  Sammeltitel für eine Skizzenreihe in seiner ersten Buchveröffentlichung \emph{Wie ich es sehe}\pwindex{Altenberg, Peter 9.\,3.\,1859 Wien – 8.\,1.\,1919 ebd.@\textsc{Altenberg, Peter} (9.\,3.\,1859 Wien – 8.\,1.\,1919 ebd.), \emph{Schriftsteller}!Wie ich es sehe@\strich\emph{Wie ich es sehe}|pwk} (Berlin: 
                     \emph{S. Fischer}\orgindex{S. Fischer Verlag@S. Fischer Verlag|pwk}{ }1896).}}}\label{K_L00346-1}« u. hoffe, daß dieſelbe Ihnen nicht zu{ }ſehr miſſfallen wird.\pend
           
\pstart
           Ihr{\\[\baselineskip]}\spacefill\mbox{Richard Engländer.}\pend
           \leftskip=0em{}\selectlanguage{ngerman}\endnumbering\briefempfaengerindex{Schnitzler, Arthur@\textsc{Schnitzler, Arthur}!zzzAltenberg, Peter@\emph{von Peter Altenberg}!1894-07-052@{{[}5.? 7. 1894{]}}|)be}\mylabel{L00346h}  \newcommand{\dateiname}{L00346}\newcommand{\titel}{Peter Altenberg an Arthur Schnitzler, [5.? 7. 1894]}\newcommand{\editorInnen}{Martin Anton Müller und Gerd-Hermann Susen}%% latex-leseansicht-abspann.tex
%% Abspann für die Leseansicht.
%% Der Schalter \ifkorrekturansicht ist bereits durch den Vorspann gesetzt.

%% latex-abspann.tex
%% Gemeinsamer Abspann für Korrekturansicht und Leseansicht.
%% Setzt den Schalter \ifkorrekturansicht voraus (gesetzt in den
%% einbindenden Dateien latex-korrekturansicht-abspann.tex bzw.
%% latex-leseansicht-abspann.tex).
%% ---------------------------------------------------------------

\normalsize

% Das esempio-Environment wird nur in der Leseansicht benötigt
\ifkorrekturansicht\else
\newenvironment{esempio}[3]%
{
    \vspace{1.5ex}
    \rlap{\underline{#1}}
    \par
    \setlength{\parindent}{0cm}
    \nopagebreak
    \leftskip=#2cm
    \rightskip=#3cm
}
{
    \par
}
\fi

\doendnotes{C}
\bigskip
\vfill

\clearpage

\footnotesize

\ifkorrekturansicht
  \lohead{\textsc{register}}
\fi

% theindex-Environment neu definieren ohne reledmac
\makeatletter
\renewenvironment{theindex}{%
  \ifkorrekturansicht
    \section*{\indexname}%
  \else
    \subsubsection*{Index der erwähnten Entitäten}%
  \fi
  \setlength{\parindent}{0pt}%
  \setlength{\parskip}{0pt plus 0.3pt}%
  \let\item\@idxitem
}{%
  \ifkorrekturansicht\clearpage\fi
}
\makeatother

\IfFileExists{\jobname-pw.ind}{\input{\jobname-pw.ind}}{}

% Quellenangabe nur in der Leseansicht
\ifkorrekturansicht\else
% Fallback-Definitionen, falls die .tex-Datei \titel etc. nicht gesetzt hat
\providecommand{\titel}{}
\providecommand{\editorInnen}{}
\providecommand{\dateiname}{\jobname}

\vspace{3cm}

\vfill

\footnotesize
\textsc{Quelle}: \titel. Herausgegeben von {\editorInnen}. In: \emph{Arthur Schnitzler: Briefwechsel mit Autorinnen und Autoren}.
 Digitale Edition, https://schnitzler-briefe.acdh.oeaw.ac.at/{\dateiname}.html (Stand \today)
\fi

\end{document}


