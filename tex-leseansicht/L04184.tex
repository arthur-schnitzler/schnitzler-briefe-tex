%% latex-leseansicht-vorspann.tex
%% Vorspann für die Leseansicht.
%% Lädt die gemeinsame Datei latex-vorspann.tex mit nicht gesetztem Schalter.

\newif\ifkorrekturansicht
\korrekturansichtfalse

\input{../tex-inputs/latex-vorspann}


\section[Arthur Schnitzler an Gustav Schwarzkopf, 31. 8. 1918]{L04184 Arthur Schnitzler an Gustav Schwarzkopf, 31. 8. 1918}
\nopagebreak\mylabel{L04184v}
\rehead{ }\normalsize\beginnumbering\briefempfaengerindex{Schwarzkopf, Gustav@\textsc{Schwarzkopf, Gustav}!zzzSchnitzler, Arthur@\emph{von Arthur Schnitzler}!1918-08-311@{31. 8. 1918}|(be}
\toendnotes[C]{\smallbreak\pagebreak[2]}
\correspDesc{Versand  durch Arthur Schnitzler am 31. 8. 1918 in Partenkirchen
\newline{}Erhalt  durch Gustav Schwarzkopf im Zeitraum [1. 9. 1918 – 5. 9. 1918?] in Wien}\toendnotes[C]{\smallbreak}
\Standort{CUL, Schnitzler, B 96.}
\physDesc{Postkarte, 1041 Zeichen
\newline{}Handschrift: Bleistift, lateinische Kurrent
\newline{}Versand: Stempel: »\nobreak{}\oindex{Garmisch-Partenkirchen@\textbf{Garmisch-Partenkirchen}, \emph{Hauptstadt}|pwk}Garmisch Partenkirchen, 31. 8. 18, 8–\textcolor{gray}{11}\nobreak{}«.  }\toendnotes[C]{\smallbreak}\pstart{}{\pb}Dr Arthur
                  Schnitzler\pend{}\pstart{}Partenkirchen\oindex{Partenkirchen@\textbf{Partenkirchen}, \emph{Teil eines besiedelten Ortes}|pw}\pend{}\pstart{}Haus Tannenberg\oindex{Haus Tannenberg@\textbf{Haus Tannenberg}, \emph{Beherbergungsgebäude}|pw}\pend{}{\bigskip}\pstart{}Herrn Gustav Schwarzkopf\pend{}\pstart{}Wien I\oindex{I., Innere Stadt@\textbf{I., Innere Stadt}, \emph{Verwaltungsgebiet}|pw}\pend{}\pstart{}Tiefer Graben 17\oindex{Wien@\textbf{Wien}!I., Innere Stadt@\textbf{I., Innere Stadt}!Tiefer Graben 17@\textbf{Tiefer Graben 17}, \emph{Wohngebäude}|pw}.\pend{}{\bigskip}\vspace{1em}
\pstart
           \raggedleft{}{\pb}Partenkirchen\oindex{Partenkirchen@\textbf{Partenkirchen}, \emph{Teil eines besiedelten Ortes}|pw}{ }31. 8. 18\pend
           \vspace{0.5em}
\pstart
           lieber Gustav, es ist heut ein Spätso{\geminationm}ertag in einer Klarheit und Schönheit, daß einem nicht nur der Krieg sondern beinah
               auch Wien\oindex{Wien@\textbf{Wien}, \emph{Verwaltungsgebiet}|pw} wie etwas ganz unwahrscheinliches
                  vorko{\geminationm}t. Ich freue mich daß ich hergeko{\geminationm}en bin. Die Reise war vollko{\geminationm}en bequem. Die Pension\oindex{Haus Tannenberg@\textbf{Haus Tannenberg}, \emph{Beherbergungsgebäude}|pwv} (außer uns nur mehr eine Dame\pwindex{?? [Pensionsgast Haus Tannenberg, August 1918] @\textsc{?? [Pensionsgast Haus Tannenberg, August 1918]}|pwv} da!) charmant. Kost gut, – auch ohne die gelegenlichen Aufbesserungen
               aus und in Liesls\pwindex{Steinrück, Elisabeth 19.\,11.\,1885 Wien – 7.\,4.\,1920 Partenkirchen@\textsc{Steinrück, Elisabeth} (19.\,11.\,1885 Wien – 7.\,4.\,1920 Partenkirchen)|pw}{ }Hause\oindex{Villa Zufriedenheit@\textbf{Villa Zufriedenheit}, \emph{Gebäude}|pw}. Überhaupt der Unterschied – in so manchem!
               Hierüber mündlich. Liesl\pwindex{Steinrück, Elisabeth 19.\,11.\,1885 Wien – 7.\,4.\,1920 Partenkirchen@\textsc{Steinrück, Elisabeth} (19.\,11.\,1885 Wien – 7.\,4.\,1920 Partenkirchen)|pw} gehts wieder, den
               Umständen entsprechend, ganz gut; ihre Sti{\geminationm}ung ist
               vorzüglich, und man unterhält sich mit ihr wie in den besten Zeiten. Albert\pwindex{Steinrück, Albert 20.\,5.\,1872 Wetterburg – 11.\,2.\,1929 Berlin@\textsc{Steinrück, Albert} (20.\,5.\,1872 Wetterburg – 11.\,2.\,1929 Berlin), \emph{Schauspieler}|pw} ist auf Gastspielen: Paris\oindex{Paris@\textbf{Paris}, \emph{Hauptstadt}|pw}. Heinrich Ma{\geminationn}\pwindex{Mann, Heinrich 27.\,3.\,1871 Lübeck – 11.\,3.\,1950 Santa Monica@\textsc{Mann, Heinrich} (27.\,3.\,1871 Lübeck – 11.\,3.\,1950 Santa Monica), \emph{Schriftsteller}|pw} u etliche andre werden erwartet. Olga\pwindex{Schnitzler, Olga 17.\,1.\,1882 Wien – 13.\,1.\,1970 Lugano@\textsc{Schnitzler, Olga} (17.\,1.\,1882 Wien – 13.\,1.\,1970 Lugano), \emph{Schauspielerin, Sängerin}|pw}
               fühlt sich, von Wirtschafts{\pb}sorgen
               befreit, sehr wohl und auch meine Nerven erholen sich leidlich. Die Landschaft
               entzückt mich wieder – es ist im Grunde eine Vereinigung von Tirol\oindex{Tirol@\textbf{Tirol}, \emph{Land}|pw} u Salzka{\geminationm}ergut\oindex{Salzkammergut@\textbf{Salzkammergut}, \emph{Region}|pw}. Ich schwelge in Wiesen- u Waldspaziergängen.\pend
           \pstart Wir grüßen Sie herzlich Ihr A.\pend{}\selectlanguage{ngerman}\endnumbering\briefempfaengerindex{Schwarzkopf, Gustav@\textsc{Schwarzkopf, Gustav}!zzzSchnitzler, Arthur@\emph{von Arthur Schnitzler}!1918-08-311@{31. 8. 1918}|)be}\mylabel{L04184h}
\begin{anhang}
\end{anhang}\newcommand{\dateiname}{L04184}\newcommand{\titel}{Arthur Schnitzler an Gustav Schwarzkopf, 31. 8. 1918}\newcommand{\editorInnen}{Herausgegeben von Jahnke, SelmaMüller, Martin Anton}%% latex-leseansicht-abspann.tex
%% Abspann für die Leseansicht.
%% Der Schalter \ifkorrekturansicht ist bereits durch den Vorspann gesetzt.

%% latex-abspann.tex
%% Gemeinsamer Abspann für Korrekturansicht und Leseansicht.
%% Setzt den Schalter \ifkorrekturansicht voraus (gesetzt in den
%% einbindenden Dateien latex-korrekturansicht-abspann.tex bzw.
%% latex-leseansicht-abspann.tex).
%% ---------------------------------------------------------------

\normalsize

% Das esempio-Environment wird nur in der Leseansicht benötigt
\ifkorrekturansicht\else
\newenvironment{esempio}[3]%
{
    \vspace{1.5ex}
    \rlap{\underline{#1}}
    \par
    \setlength{\parindent}{0cm}
    \nopagebreak
    \leftskip=#2cm
    \rightskip=#3cm
}
{
    \par
}
\fi

\doendnotes{C}
\bigskip
\vfill

\clearpage

\footnotesize

\ifkorrekturansicht
  \lohead{\textsc{register}}
\fi

% theindex-Environment neu definieren ohne reledmac
\makeatletter
\renewenvironment{theindex}{%
  \ifkorrekturansicht
    \section*{\indexname}%
  \else
    \subsubsection*{Index der erwähnten Entitäten}%
  \fi
  \setlength{\parindent}{0pt}%
  \setlength{\parskip}{0pt plus 0.3pt}%
  \let\item\@idxitem
}{%
  \ifkorrekturansicht\clearpage\fi
}
\makeatother

\IfFileExists{\jobname-pw.ind}{\input{\jobname-pw.ind}}{}

% Quellenangabe nur in der Leseansicht
\ifkorrekturansicht\else
% Fallback-Definitionen, falls die .tex-Datei \titel etc. nicht gesetzt hat
\providecommand{\titel}{}
\providecommand{\editorInnen}{}
\providecommand{\dateiname}{\jobname}

\vspace{3cm}

\vfill

\footnotesize
\textsc{Quelle}: \titel. Herausgegeben von {\editorInnen}. In: \emph{Arthur Schnitzler: Briefwechsel mit Autorinnen und Autoren}.
 Digitale Edition, https://schnitzler-briefe.acdh.oeaw.ac.at/{\dateiname}.html (Stand \today)
\fi

\end{document}


