%% latex-leseansicht-vorspann.tex
%% Vorspann für die Leseansicht.
%% Lädt die gemeinsame Datei latex-vorspann.tex mit nicht gesetztem Schalter.

\newif\ifkorrekturansicht
\korrekturansichtfalse

\input{../tex-inputs/latex-vorspann}


         
         \renewcommand{\erwaehntePersonen}{Personen: Caroline Kotter, Felix Salten, Ottilie Salten}
         \renewcommand{\erwaehnteInstitutionen}{Institutionen: Neue akademische Vereinigung}
         \renewcommand{\erwaehnteOrte}{Orte: Brünn, Deutsches Haus, Wien}
         \renewcommand{\erwaehnteWerke}{}
               \section[ Arthur Schnitzler an Felix Salten, 15. 10. 1903]{ Arthur Schnitzler an Felix Salten, 15. 10. 1903}\nopagebreak\mylabel{v}\rehead{ }\begin{ledgroupsized}[t]{13cm}\normalsize\beginnumbering\briefempfaengerindex{Salten, Felix@\textsc{Salten, Felix}!zzzSchnitzler, Arthur@\emph{von Arthur Schnitzler}!1903-10-151@{15. 10. 1903}|(be} \toendnotes[C]{\smallbreak\pagebreak[2]} \Standort{Wienbibliothek im Rathaus, ZPH 1681, 2.1.516.}
\physDesc{Brief, 1 Blatt, 4 Seiten, 659 Zeichen
\newline{}Handschrift: Bleistift, deutsche Kurrent
\newline{}Ordnung: mit Bleistift von unbekannter Hand Nummerierung der Doppelseiten des
                                 Konvoluts: »51«–»52« }\toendnotes[C]{\smallbreak}\pstart
           \raggedleft{}{\pb}15. 10. 903.\pend
           \pstart
           lieber, gegen \label{K_L02985-1v}\edtext{Mittwoch nächſter Woche}{\lemma{\textnormal{\emph{Mittwoch nächſter Woche}}}\Cendnote{\textnormal{Siehe A. S.: \emph{Tagebuch}, 21. 10. 1903.
               }}}\label{K_L02985-1h} hab ich nichts einzuwenden. \textcolor{gray}{×}\-\textcolor{gray}{×}{ }\textcolor{gray}{×}\-\textcolor{gray}{×}\pend
           \pstart
           Tagesausflug iſt mir kein verführeriſcher Gedanke. Hingegen ſchlag ich Ihnen vor, mit
                  Otti\pwindex{Salten, Ottilie 07.03.1868 – 22.06.1942@\textsc{Salten, Ottilie} (07.03.1868 – 22.06.1942), \emph{Schauspielerin}|pw} und dem kleinen Fräulein\pwindex{Kotter, Caroline 1893-07-07 – 1964-07-01@\textsc{Kotter, Caroline} (1893-07-07 – 1964-07-01)|pwv}{ }\label{K_L02985-2v}\edtext{So{\geminationn}tag}{\lemma{\textnormal{\emph{Sonntag}}}\Cendnote{\textnormal{Siehe A. S.: \emph{Tagebuch}, 18. 10. 1903.
               }}}\label{K_L02985-2h} (um 1, we{\geminationn}s {\pb}Ihnen recht iſt) bei uns zu ſpeiſen – We{\geminationn} das Wetter ſchön iſt, iſt bei uns auch Land. Und dann
               können Sie noch immer in fernere Fernen. –\pend
           \pstart
           Wenn nicht (was ſchade wäre) ſo wählen Sie bitte irgend einen Abend der {\pb}nächſten Woche, an dem wir das Vergnügen
               haben können, Sie bei uns zu ſehen – nur nicht Montag:
               da wartet mein der \label{K_L02985-3v}\edtext{Vorleſetiſch in dem
                  Tuchmacherſtädtchen\oindex{Bruenn@\textbf{Brünn}|pwv}}{\lemma{\textnormal{\emph{Vorleſetiſch … Tuchmacherſtädtchen}}}\Cendnote{\textnormal{Schnitzler\pwindex{Schnitzler, Arthur 15.05.1862 – 21.10.1931@\textsc{Schnitzler, Arthur} (15.05.1862 – 21.10.1931), \emph{Schriftsteller, Mediziner}|pwk}
                  las am
                  19. 10. 1903
                  für die
                  \emph{Neue akademische Vereinigung}\orgindex{Neue akademische Vereinigung@Neue akademische Vereinigung|pwk}
                  im kleinen Festsaal des
                  Deutschen Hauses\oindex{Deutsches Haus@\textbf{Deutsches Haus}|pwk}.}}}\label{K_L02985-3h}. –\pend
           \pstart
           Herzlichſt {\\[\baselineskip]}Ihr {\\[\baselineskip]}\spacefill\mbox{A.}\pend
           \leftskip=0em{}\pstart
           \noindent{}{\pb}Wollen Sie So{\geminationn}tag eine andere Stunde, ſo beſtimmen
                  Sie\pend
           \pstart
           \strikeout{\textcolor{gray}{[2 Zeilen unleserlich{]} }}\pend
           \pstart
           {[}Zeichnung einer Straßenbahn{]}\pend
           
         
         \endnumbering\mylabel{h}\end{ledgroupsized}  \newcommand{\dateiname}{L02985}\newcommand{\titel}{Arthur Schnitzler an Felix Salten, 15. 10. 1903}\newcommand{\editorInnen}{Martin Anton Müller und Laura Untner}%% latex-leseansicht-abspann.tex
%% Abspann für die Leseansicht.
%% Der Schalter \ifkorrekturansicht ist bereits durch den Vorspann gesetzt.

%% latex-abspann.tex
%% Gemeinsamer Abspann für Korrekturansicht und Leseansicht.
%% Setzt den Schalter \ifkorrekturansicht voraus (gesetzt in den
%% einbindenden Dateien latex-korrekturansicht-abspann.tex bzw.
%% latex-leseansicht-abspann.tex).
%% ---------------------------------------------------------------

\normalsize

% Das esempio-Environment wird nur in der Leseansicht benötigt
\ifkorrekturansicht\else
\newenvironment{esempio}[3]%
{
    \vspace{1.5ex}
    \rlap{\underline{#1}}
    \par
    \setlength{\parindent}{0cm}
    \nopagebreak
    \leftskip=#2cm
    \rightskip=#3cm
}
{
    \par
}
\fi

\doendnotes{C}
\bigskip
\vfill

\clearpage

\footnotesize

\ifkorrekturansicht
  \lohead{\textsc{register}}
\fi

% theindex-Environment neu definieren ohne reledmac
\makeatletter
\renewenvironment{theindex}{%
  \ifkorrekturansicht
    \section*{\indexname}%
  \else
    \subsubsection*{Index der erwähnten Entitäten}%
  \fi
  \setlength{\parindent}{0pt}%
  \setlength{\parskip}{0pt plus 0.3pt}%
  \let\item\@idxitem
}{%
  \ifkorrekturansicht\clearpage\fi
}
\makeatother

\IfFileExists{\jobname-pw.ind}{\input{\jobname-pw.ind}}{}

% Quellenangabe nur in der Leseansicht
\ifkorrekturansicht\else
% Fallback-Definitionen, falls die .tex-Datei \titel etc. nicht gesetzt hat
\providecommand{\titel}{}
\providecommand{\editorInnen}{}
\providecommand{\dateiname}{\jobname}

\vspace{3cm}

\vfill

\footnotesize
\textsc{Quelle}: \titel. Herausgegeben von {\editorInnen}. In: \emph{Arthur Schnitzler: Briefwechsel mit Autorinnen und Autoren}.
 Digitale Edition, https://schnitzler-briefe.acdh.oeaw.ac.at/{\dateiname}.html (Stand \today)
\fi

\end{document}


      