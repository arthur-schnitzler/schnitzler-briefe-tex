%% latex-leseansicht-vorspann.tex
%% Vorspann für die Leseansicht.
%% Lädt die gemeinsame Datei latex-vorspann.tex mit nicht gesetztem Schalter.

\newif\ifkorrekturansicht
\korrekturansichtfalse

\input{../tex-inputs/latex-vorspann}


\section[Arthur Schnitzler an Hermann Bahr, {{[}}20. 9. 1896?{{]}}]{L00595 Arthur Schnitzler an Hermann Bahr, {[}20. 9. 1896?{]}}
\nopagebreak\mylabel{L00595v}
\rehead{ }\normalsize\beginnumbering\briefempfaengerindex{Bahr, Hermann@\textsc{Bahr, Hermann}!zzzSchnitzler, Arthur@\emph{von Arthur Schnitzler}!1896-09-201@{{[}20. 9. 1896?{]}}|(be}
\toendnotes[C]{\smallbreak\pagebreak[2]}
\correspDesc{Versand  durch Arthur Schnitzler am [20. 9. 1896?] in Wien
\newline{}Erhalt  durch Hermann Bahr im Zeitraum [20. 9. 1896
                  – 24. 9. 1896?] in Wien}\toendnotes[C]{\smallbreak}
\Standort{TMW, HS AM 60153 Ba.}
\physDesc{Briefkarte, 368 Zeichen
\newline{}Handschrift: Bleistift, deutsche Kurrent
\newline{}Ordnung: Lochung }
\buchAbdrucke{\weitereDrucke{1) \emph{[20. 9. 1896?], Abschrift.} In: Arthur Schnitzler: \emph{The Letters of Arthur Schnitzler to Hermann Bahr}. Edited, annotated, and with an introduction, by Donald G. Daviau. Chapel Hill: \emph{The University of North Carolina Press} 1978, S. 59 (University of North Carolina studies in the Germanic languages
                        and literatures, 89).} \weitereDrucke{2) Hermann Bahr, Arthur Schnitzler: \emph{Briefwechsel, Aufzeichnungen, Dokumente (1891–1931)}. Herausgegeben von Kurt Ifkovits und Martin Anton Müller. Göttingen: \emph{Wallstein} 2018, S. 126.} }\toendnotes[C]{\smallbreak}
\pstart
           \raggedleft{}{\pb}\label{K_L00595-1v}\edtext{So{\geminationn}tag abd}{\lemma{\textnormal{\emph{Sonntag abd}}}\Cendnote{\textnormal{undatierte Briefkarte; am 14. 9. 1896 hatte
                        Schnitzler{ }Beer-Hofmann\pwindex{Beer-Hofmann, Richard 11.\,7.\,1866 Wien – 26.\,9.\,1945 New York City@\textsc{Beer-Hofmann, Richard} (11.\,7.\,1866 Wien – 26.\,9.\,1945 New York City), \emph{Schriftsteller}|pwk} nicht in Baden\oindex{Baden bei Wien@\textbf{Baden bei Wien}, \emph{Hauptstadt}|pwk} angetroffen, worauf ihm dieser mitteilte, er werde »am
                           24. in Wien\oindex{Wien@\textbf{Wien}, \emph{Verwaltungsgebiet}|pw}{ }sein« (XXXX Auszeichnungsfehler: Dokument L00591 nicht gefunden). Der 20. 9. 1896 war ein
                     Sonntag.}}}\label{K_L00595-1}\pend
           \vspace{0.5em}
\pstart
           Lieber Hermann, als ich geſtern Abend fragte, wußte man noch nichts
               von deiner Sendung, jetzt eben beim Nachhauſegehen übergab mir die Hausmeiſterin\pwindex{?? [Hausmeisterin von Arthur Schnitzler] @\textsc{?? [Hausmeisterin von Arthur Schnitzler]}|pwv} das Paket; da dein Brief mit
               der Adreſſe mit eingeschloſſen war, hatte sie nicht gewußt, daſs es für mich gehörte.
               – \substVorne{}\textsuperscript{h}\substDazwischen{}H\substHinten{}erzlichen Dank! {\pb}Richard\pwindex{Beer-Hofmann, Richard 11.\,7.\,1866 Wien – 26.\,9.\,1945 New York City@\textsc{Beer-Hofmann, Richard} (11.\,7.\,1866 Wien – 26.\,9.\,1945 New York City), \emph{Schriftsteller}|pw} wohnt \textsc{Baden, \label{K_L00595-2v}\edtext{Franzensgasse}{\lemma{\textnormal{\emph{Franzensgasse}}}\Cendnote{\textnormal{Ein Irrtum Schnitzlers, Beer-Hofmann\pwindex{Beer-Hofmann, Richard 11.\,7.\,1866 Wien – 26.\,9.\,1945 New York City@\textsc{Beer-Hofmann, Richard} (11.\,7.\,1866 Wien – 26.\,9.\,1945 New York City), \emph{Schriftsteller}|pwk} wohnte in der Franzensstraße\oindex{Kaiser-Franz-Ring@\textbf{Kaiser-Franz-Ring}, \emph{Straße}|pwk}.}}}\label{K_L00595-2} 54\oindex{Kaiser-Franz-Ring@\textbf{Kaiser-Franz-Ring}, \emph{Straße}|pw}}, ko{\geminationm}t am 24. herein. –\pend
           
\pstart
           Herzlichen Gruſs dein{\\[\baselineskip]}\spacefill\mbox{Arth}\pend
           \leftskip=0em{}\selectlanguage{ngerman}\endnumbering\briefempfaengerindex{Bahr, Hermann@\textsc{Bahr, Hermann}!zzzSchnitzler, Arthur@\emph{von Arthur Schnitzler}!1896-09-201@{{[}20. 9. 1896?{]}}|)be}\mylabel{L00595h}  \newcommand{\dateiname}{L00595}\newcommand{\titel}{Arthur Schnitzler an Hermann Bahr, [20. 9. 1896?]}\newcommand{\editorInnen}{Herausgegeben von Martin Anton Müller}%% latex-leseansicht-abspann.tex
%% Abspann für die Leseansicht.
%% Der Schalter \ifkorrekturansicht ist bereits durch den Vorspann gesetzt.

%% latex-abspann.tex
%% Gemeinsamer Abspann für Korrekturansicht und Leseansicht.
%% Setzt den Schalter \ifkorrekturansicht voraus (gesetzt in den
%% einbindenden Dateien latex-korrekturansicht-abspann.tex bzw.
%% latex-leseansicht-abspann.tex).
%% ---------------------------------------------------------------

\normalsize

% Das esempio-Environment wird nur in der Leseansicht benötigt
\ifkorrekturansicht\else
\newenvironment{esempio}[3]%
{
    \vspace{1.5ex}
    \rlap{\underline{#1}}
    \par
    \setlength{\parindent}{0cm}
    \nopagebreak
    \leftskip=#2cm
    \rightskip=#3cm
}
{
    \par
}
\fi

\doendnotes{C}
\bigskip
\vfill

\clearpage

\footnotesize

\ifkorrekturansicht
  \lohead{\textsc{register}}
\fi

% theindex-Environment neu definieren ohne reledmac
\makeatletter
\renewenvironment{theindex}{%
  \ifkorrekturansicht
    \section*{\indexname}%
  \else
    \subsubsection*{Index der erwähnten Entitäten}%
  \fi
  \setlength{\parindent}{0pt}%
  \setlength{\parskip}{0pt plus 0.3pt}%
  \let\item\@idxitem
}{%
  \ifkorrekturansicht\clearpage\fi
}
\makeatother

\IfFileExists{\jobname-pw.ind}{\input{\jobname-pw.ind}}{}

% Quellenangabe nur in der Leseansicht
\ifkorrekturansicht\else
% Fallback-Definitionen, falls die .tex-Datei \titel etc. nicht gesetzt hat
\providecommand{\titel}{}
\providecommand{\editorInnen}{}
\providecommand{\dateiname}{\jobname}

\vspace{3cm}

\vfill

\footnotesize
\textsc{Quelle}: \titel. Herausgegeben von {\editorInnen}. In: \emph{Arthur Schnitzler: Briefwechsel mit Autorinnen und Autoren}.
 Digitale Edition, https://schnitzler-briefe.acdh.oeaw.ac.at/{\dateiname}.html (Stand \today)
\fi

\end{document}


