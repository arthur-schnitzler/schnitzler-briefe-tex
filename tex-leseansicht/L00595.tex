%% latex-korrekturansicht-vorspann.tex
%% Vorspann für die Korrekturansicht.
%% Lädt die gemeinsame Datei latex-vorspann.tex mit gesetztem Schalter.

\newif\ifkorrekturansicht
\korrekturansichttrue

\input{../tex-inputs/latex-vorspann}


\section[Arthur Schnitzler an Hermann Bahr, {[}20. 9. 1896?{]}]{L00595 Arthur Schnitzler an Hermann Bahr, {[}20. 9. 1896?{]}}
\nopagebreak\mylabel{L00595v}
\rehead{ }\normalsize\beginnumbering\briefempfaengerindex{Bahr, Hermann@\textsc{Bahr, Hermann}!zzzSchnitzler, Arthur@\emph{von Arthur Schnitzler}!1896-09-201@{{[}20. 9. 1896?{]}}|(be}
\toendnotes[C]{\smallbreak\pagebreak[2]}\Standort{TMW, HS AM 60153 Ba.}
\physDesc{Briefkarte, 368 Zeichen
\newline{}Handschrift: Bleistift, deutsche Kurrent
\newline{}Ordnung: Lochung }
\buchAbdrucke{\weitereDrucke{1) Arthur Schnitzler: \emph{The Letters of Arthur Schnitzler to Hermann Bahr}. Chapel Hill: \emph{The University of North Carolina Press} 1978, S. 59.} \weitereDrucke{2) Hermann Bahr, Arthur Schnitzler: \emph{Briefwechsel, Aufzeichnungen, Dokumente (1891–1931)}. Göttingen: \emph{Wallstein} 2018, S. 126.} }\toendnotes[C]{\smallbreak}
\pstart
           \raggedleft{}{\pb}\label{K_L00595-1v}\edtext{So{\geminationn}tag abd}{\lemma{\textnormal{\emph{Sonntag abd}}}\Cendnote{\textnormal{undatierte Briefkarte; am 14. 9. 1896 hatte
                        Schnitzler{ }Beer-Hofmann\pwindex{Beer-Hofmann, Richard 1866-07-11 – 1945-09-26@\textsc{Beer-Hofmann, Richard} (1866-07-11 – 1945-09-26), \emph{Schriftsteller/Schriftstellerin}|pwk} nicht in Baden\oindex{Baden bei Wien@\textbf{Baden bei Wien}, \emph{P.PPLA3}|pwk} angetroffen, worauf ihm dieser mitteilte, er werde »am
                           24. in Wien\oindex{Wien@\textbf{Wien}, \emph{A.ADM2}|pw}{ }sein« (Richard Beer-Hofmann an Arthur Schnitzler, 15. 9. 1896). Der 20. 9. 1896 war ein
                     Sonntag.}}}\label{K_L00595-1}\pend
           \vspace{0.5em}
\pstart
           Lieber Hermann, als ich geſtern Abend fragte, wußte man noch nichts
               von deiner Sendung, jetzt eben beim Nachhauſegehen übergab mir die Hausmeiſterin\pwindex{?? [Hausmeisterin von Arthur Schnitzler] @\textsc{?? [Hausmeisterin von Arthur Schnitzler]}|pwv} das Paket; da dein Brief mit
               der Adreſſe mit eingeschloſſen war, hatte sie nicht gewußt, daſs es für mich gehörte.
               – \substVorne{}\textsuperscript{h}\substDazwischen{}H\substHinten{}erzlichen Dank! {\pb}Richard\pwindex{Beer-Hofmann, Richard 1866-07-11 – 1945-09-26@\textsc{Beer-Hofmann, Richard} (1866-07-11 – 1945-09-26), \emph{Schriftsteller/Schriftstellerin}|pw} wohnt \textsc{Baden, \label{K_L00595-2v}\edtext{Franzensgasse}{\lemma{\textnormal{\emph{Franzensgasse}}}\Cendnote{\textnormal{Ein Irrtum Schnitzlers, Beer-Hofmann\pwindex{Beer-Hofmann, Richard 1866-07-11 – 1945-09-26@\textsc{Beer-Hofmann, Richard} (1866-07-11 – 1945-09-26), \emph{Schriftsteller/Schriftstellerin}|pwk} wohnte in der Franzensstraße\oindex{Kaiser-Franz-Ring@\textbf{Kaiser-Franz-Ring}, \emph{Straße (K.STR)}|pwk}.}}}\label{K_L00595-2} 54\oindex{Kaiser-Franz-Ring@\textbf{Kaiser-Franz-Ring}, \emph{Straße (K.STR)}|pw}}, ko{\geminationm}t am 24. herein. – \pend
           
\pstart
           Herzlichen Gruſs dein{\\[\baselineskip]}\spacefill\mbox{Arth}\pend
           \leftskip=0em{}\selectlanguage{ngerman}\endnumbering\briefempfaengerindex{Bahr, Hermann@\textsc{Bahr, Hermann}!zzzSchnitzler, Arthur@\emph{von Arthur Schnitzler}!1896-09-201@{{[}20. 9. 1896?{]}}|)be}\mylabel{L00595h}  \normalsize

\doendnotes{C}
\bigskip
\vfill

\clearpage

\footnotesize

\lohead{\textsc{register}}

% Definiere theindex-Environment komplett neu ohne reledmac
\makeatletter
\renewenvironment{theindex}{%
  \section*{\indexname}%
  \setlength{\parindent}{0pt}%
  \setlength{\parskip}{0pt plus 0.3pt}%
  \let\item\@idxitem
}{%
  \clearpage
}
\makeatother

\IfFileExists{\jobname-pw.ind}{\input{\jobname-pw.ind}}{}

\end{document}

      