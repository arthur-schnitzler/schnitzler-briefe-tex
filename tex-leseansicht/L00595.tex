%% latex-leseansicht-vorspann.tex
%% Vorspann für die Leseansicht.
%% Lädt die gemeinsame Datei latex-vorspann.tex mit nicht gesetztem Schalter.

\newif\ifkorrekturansicht
\korrekturansichtfalse

\input{../tex-inputs/latex-vorspann}


               \section[Arthur Schnitzler an Hermann Bahr, {[}20. 9. 1896?{]}]{ Arthur Schnitzler an Hermann Bahr, {[}20. 9. 1896?{]}}\nopagebreak\mylabel{v}\rehead{ }\begin{ledgroupsized}[t]{13cm}\normalsize\beginnumbering\briefempfaengerindex{Bahr, Hermann@\textsc{Bahr, Hermann}!zzzSchnitzler, Arthur@\emph{von Arthur Schnitzler}!1896-09-201@{{[}20. 9. 1896?{]}}|(be} \toendnotes[C]{\smallbreak\pagebreak[2]} \Standort{TMW, HS AM 60153 Ba.}
\physDesc{Briefkarte
\newline{}Handschrift: Bleistift, deutsche Kurrent\newline{}Ordnung: Lochung }\buchAbdrucke{\weitereDrucke{1) \emph{[20. 9. 1896?], Abschrift.} In: Arthur Schnitzler: \emph{The Letters of Arthur Schnitzler to Hermann Bahr}. Edited, annotated, and with an introduction, by Donald G.
                        Daviau. Chapel Hill: \emph{The University of North Carolina Press} 1978, S. 59 (University of North Carolina studies in the Germanic languages
                        and literatures, 89).} \weitereDrucke{2) Hermann Bahr, Arthur Schnitzler: \emph{Briefwechsel, Aufzeichnungen, Dokumente (1891–1931)}. Hg. Kurt Ifkovits und Martin Anton Müller. Göttingen: \emph{Wallstein} 2018, S. 126.} }\toendnotes[C]{\smallbreak}\pstart
           \raggedleft{}{\pb}\label{K_L00595_1v}\edtext{So{\geminationn}tag abd}{\lemma{\textnormal{\emph{Sotag abd}}}\Cendnote{\textnormal{undatierte Briefkarte; am 14. 9. 1896 traf Schnitzler\pwindex{Schnitzler, Arthur 15.05.1862 – 21.10.1931@\textsc{Schnitzler, Arthur} (15.05.1862 – 21.10.1931), \emph{Schriftsteller, Mediziner}|pwk}{ }Beer-Hofmann\pwindex{Beer-Hofmann, Richard 11.07.1866 – 26.09.1945@\textsc{Beer-Hofmann, Richard} (11.07.1866 – 26.09.1945), \emph{Schriftsteller}|pwk} nicht in Baden\oindex{Baden bei Wien@\textbf{Baden bei Wien}|pwk} an, worauf ihm dieser mitteilte, er werde »am
                           24. in Wien\oindex{Wien@\textbf{Wien}|pw}{ }sein« (Richard Beer-Hofmann an Arthur Schnitzler,
               15. 9. 1896). Der 20. 9. 1896 ist ein
                     Sonntag.}}}\label{K_L00595_1h}\pend
           \pstart
           Lieber Hermann, als ich geſtern Abend fragte, wußte man noch nichts
               von deiner Sendung, jetzt eben beim Nachhauſegehen übergab mir die Hausmeiſterin\pwindex{?? [Hausmeisterin von Arthur Schnitzler] 1896 – 1896@\textsc{?? [Hausmeisterin von Arthur Schnitzler]} (1896 – 1896)|pwv} das Paket; da dein Brief mit der
               Adreſſe mit eingeschloſſen war, hatte sie nicht gewußt, daſs es für mich gehörte. – \substVorne{}\textsuperscript{h}\substDazwischen{}H\substHinten{}erzlichen Dank! {\pb}Richard\pwindex{Beer-Hofmann, Richard 11.07.1866 – 26.09.1945@\textsc{Beer-Hofmann, Richard} (11.07.1866 – 26.09.1945), \emph{Schriftsteller}|pw} wohnt \textsc{Baden, \label{K_L00595_2v}\edtext{Franzensgasse}{\lemma{\textnormal{\emph{Franzensgasse}}}\Cendnote{\textnormal{Ein Irrtum Schnitzler\pwindex{Schnitzler, Arthur 15.05.1862 – 21.10.1931@\textsc{Schnitzler, Arthur} (15.05.1862 – 21.10.1931), \emph{Schriftsteller, Mediziner}|pwk}s, Beer-Hofmann\pwindex{Beer-Hofmann, Richard 11.07.1866 – 26.09.1945@\textsc{Beer-Hofmann, Richard} (11.07.1866 – 26.09.1945), \emph{Schriftsteller}|pwk} wohnte in der Franzensstraße\oindex{Kaiser-Franz-Ring@\textbf{Kaiser-Franz-Ring}|pwk}.}}}\label{K_L00595_2h} 54\oindex{Kaiser-Franz-Ring@\textbf{Kaiser-Franz-Ring}|pw}}, ko{\geminationm}t am 24. herein. – \pend
           \pstart
           Herzlichen Gruſs dein{\\[\baselineskip]}\spacefill\mbox{Arth}\pend
           \leftskip=0em{}          \endnumbering\briefempfaengerindex{Bahr, Hermann@\textsc{Bahr, Hermann}!zzzSchnitzler, Arthur@\emph{von Arthur Schnitzler}!1896-09-201@{{[}20. 9. 1896?{]}}|)be}\mylabel{h}\end{ledgroupsized}  \newcommand{\dateiname}{L00595}\newcommand{\titel}{Arthur Schnitzler an Hermann Bahr, [20. 9. 1896?]}\newcommand{\editorInnen}{ Kurt Ifkovits,  Martin Anton Müller}
            \footnotesize
\begin{ledgroupsized}[t]{11.5cm}
\doendnotes{C}
\end{ledgroupsized}
         %% latex-leseansicht-abspann.tex
%% Abspann für die Leseansicht.
%% Der Schalter \ifkorrekturansicht ist bereits durch den Vorspann gesetzt.

%% latex-abspann.tex
%% Gemeinsamer Abspann für Korrekturansicht und Leseansicht.
%% Setzt den Schalter \ifkorrekturansicht voraus (gesetzt in den
%% einbindenden Dateien latex-korrekturansicht-abspann.tex bzw.
%% latex-leseansicht-abspann.tex).
%% ---------------------------------------------------------------

\normalsize

% Das esempio-Environment wird nur in der Leseansicht benötigt
\ifkorrekturansicht\else
\newenvironment{esempio}[3]%
{
    \vspace{1.5ex}
    \rlap{\underline{#1}}
    \par
    \setlength{\parindent}{0cm}
    \nopagebreak
    \leftskip=#2cm
    \rightskip=#3cm
}
{
    \par
}
\fi

\doendnotes{C}
\bigskip
\vfill

\clearpage

\footnotesize

\ifkorrekturansicht
  \lohead{\textsc{register}}
\fi

% theindex-Environment neu definieren ohne reledmac
\makeatletter
\renewenvironment{theindex}{%
  \ifkorrekturansicht
    \section*{\indexname}%
  \else
    \subsubsection*{Index der erwähnten Entitäten}%
  \fi
  \setlength{\parindent}{0pt}%
  \setlength{\parskip}{0pt plus 0.3pt}%
  \let\item\@idxitem
}{%
  \ifkorrekturansicht\clearpage\fi
}
\makeatother

\IfFileExists{\jobname-pw.ind}{\input{\jobname-pw.ind}}{}

% Quellenangabe nur in der Leseansicht
\ifkorrekturansicht\else
% Fallback-Definitionen, falls die .tex-Datei \titel etc. nicht gesetzt hat
\providecommand{\titel}{}
\providecommand{\editorInnen}{}
\providecommand{\dateiname}{\jobname}

\vspace{3cm}

\vfill

\footnotesize
\textsc{Quelle}: \titel. Herausgegeben von {\editorInnen}. In: \emph{Arthur Schnitzler: Briefwechsel mit Autorinnen und Autoren}.
 Digitale Edition, https://schnitzler-briefe.acdh.oeaw.ac.at/{\dateiname}.html (Stand \today)
\fi

\end{document}


      