%% latex-leseansicht-vorspann.tex
%% Vorspann für die Leseansicht.
%% Lädt die gemeinsame Datei latex-vorspann.tex mit nicht gesetztem Schalter.

\newif\ifkorrekturansicht
\korrekturansichtfalse

\input{../tex-inputs/latex-vorspann}


\section[Charlotte Ehrenstein an Arthur Schnitzler, {{[}}22. 1.? 1906{{]}}]{L01576 Charlotte Ehrenstein an Arthur Schnitzler, {[}22. 1.? 1906{]}}
\nopagebreak\mylabel{L01576v}
\rehead{ }\normalsize\beginnumbering\briefempfaengerindex{Schnitzler, Arthur@\textsc{Schnitzler, Arthur}!zzzEhrenstein, Charlotte@\emph{von Charlotte Ehrenstein}!1906-01-222@{{[}22. 1.? 1906{]}}|(be}
\toendnotes[C]{\smallbreak\pagebreak[2]}
\correspDesc{Versand  durch Charlotte Ehrenstein am [22. 1.? 1906] in Wien
\newline{}Erhalt  durch Arthur Schnitzler im Zeitraum [22. 1. 1906
                  – 26. 1. 1906?] in Wien}\toendnotes[C]{\smallbreak}
\Standort{DLA, A:Schnitzler, HS.NZ85.1.2837,3.}
\physDesc{Brief, 1 Blatt, 3 Seiten, 1020 Zeichen
\newline{}Handschrift: schwarze Tinte, deutsche Kurrent}\toendnotes[C]{\smallbreak}
\pstart
           {\pb}\textsc{Sr. Hochwohlgeb. Herrn Dr. Arthur Schnitzler}.\pend
           
\pstart\center{}Sehr geehrter Herr Doctor!\pend\vspace{0.5em}
\pstart
           Von Ihrer gütigen Erlaubnis Gebrauch machend, geſtatte ich mir über den Zuſtand
               meines l. Albert\pwindex{Ehrenstein, Albert 23.\,12.\,1886 Wien – 8.\,4.\,1950 New York City@\textsc{Ehrenstein, Albert} (23.\,12.\,1886 Wien – 8.\,4.\,1950 New York City), \emph{Schriftsteller}|pw} zu berichten. Am \label{K_L01576-1v}\edtext{Samstag}{\lemma{\textnormal{\emph{Samstag}}}\Cendnote{\textnormal{Obzwar undatiert, dürfte dieses
                  Korrespondenzstück wegen der inhaltlichen Übereinstimmung am selben Tag wie das
                  Schreiben von Adolf Treibl\pwindex{Treibl, Adolf 2.\,3.\,1865 Sobotište – 1935 Bělice@\textsc{Treibl, Adolf} (2.\,3.\,1865 Sobotište – 1935 Bělice), \emph{Fabrikant}|pwk} vom XXXX Auszeichnungsfehler: Dokument L01575 nicht gefunden verfasst sein.}}}\label{K_L01576-1} war Dr. Kornfeld\pwindex{Kornfeld, Sigmund 21.\,4.\,1859 Golčův Jeníkov – 15.\,4.\,1927 Wien@\textsc{Kornfeld, Sigmund} (21.\,4.\,1859 Golčův Jeníkov – 15.\,4.\,1927 Wien), \emph{Psychiater}|pw} nochmals hier und{ }ſah, daſs Albert\pwindex{Ehrenstein, Albert 23.\,12.\,1886 Wien – 8.\,4.\,1950 New York City@\textsc{Ehrenstein, Albert} (23.\,12.\,1886 Wien – 8.\,4.\,1950 New York City), \emph{Schriftsteller}|pw}{ }ſich ziemlich beruhigte, daher entſchloſs er{ }ſich
               ihn in häuſlicher Pflege zu laſſen, womit auch mein l. Patient ganz einverſtanden
               iſt. Die Beſſerung macht nun, wie H. Dr. Kornfeld\pwindex{Kornfeld, Sigmund 21.\,4.\,1859 Golčův Jeníkov – 15.\,4.\,1927 Wien@\textsc{Kornfeld, Sigmund} (21.\,4.\,1859 Golčův Jeníkov – 15.\,4.\,1927 Wien), \emph{Psychiater}|pw}{ }ſagt, und auch ich bemerken kann, befriedigende
               Fortſchritte {\pb}und{ }ſind nun mein l. Mann\pwindex{Ehrenstein, Alexander 29.\,3.\,1857 Skalice – 29.\,5.\,1925 Wien@\textsc{Ehrenstein, Alexander} (29.\,3.\,1857 Skalice – 29.\,5.\,1925 Wien), \emph{Kassier}|pwv} und ich auch beruhigter.\pend
           
\pstart
           Und nun geſtatten Sie{ }ſehr geehrter Herr Doctor mir für die vielen Beweiſe von
               Hochherzigkeit, Güte u. Liebenswürdigkeit, welche Sie meinem l. Albert\pwindex{Ehrenstein, Albert 23.\,12.\,1886 Wien – 8.\,4.\,1950 New York City@\textsc{Ehrenstein, Albert} (23.\,12.\,1886 Wien – 8.\,4.\,1950 New York City), \emph{Schriftsteller}|pw}, meinem l. Mann\pwindex{Ehrenstein, Alexander 29.\,3.\,1857 Skalice – 29.\,5.\,1925 Wien@\textsc{Ehrenstein, Alexander} (29.\,3.\,1857 Skalice – 29.\,5.\,1925 Wien), \emph{Kassier}|pwv} u. mir erwieſen recht herzlichſt zu danken, u. mir zu
               verzeihen, daſs ich durch dieſen traurigen Zwiſchenfall, dieſe{ }ſo{ }ſehr in Anſpruch
               nahm.\pend
           
\pstart
           {\pb}Nochmals Sie{ }ſehr geehrter Herr Doctor unſerer{ }ſteten
               Dankbarkeit verſichernd, Ihre verehrte Frau Gemahlin\pwindex{Schnitzler, Olga 17.\,1.\,1882 Wien – 13.\,1.\,1970 Lugano@\textsc{Schnitzler, Olga} (17.\,1.\,1882 Wien – 13.\,1.\,1970 Lugano), \emph{Schauspielerin, Sängerin}|pwv} um Verzeihung und Nachſicht bittend bin ich Ihre
               Sie verehrende\pend
           \pstart \spacefill\mbox{Charlotte Ehrenſtein}\pend{}\selectlanguage{ngerman}\endnumbering\briefempfaengerindex{Schnitzler, Arthur@\textsc{Schnitzler, Arthur}!zzzEhrenstein, Charlotte@\emph{von Charlotte Ehrenstein}!1906-01-222@{{[}22. 1.? 1906{]}}|)be}\mylabel{L01576h}  \newcommand{\dateiname}{L01576}\newcommand{\titel}{Charlotte Ehrenstein an Arthur Schnitzler, [22. 1.? 1906]}\newcommand{\editorInnen}{Martin Anton Müller und Gerd-Hermann Susen}%% latex-leseansicht-abspann.tex
%% Abspann für die Leseansicht.
%% Der Schalter \ifkorrekturansicht ist bereits durch den Vorspann gesetzt.

%% latex-abspann.tex
%% Gemeinsamer Abspann für Korrekturansicht und Leseansicht.
%% Setzt den Schalter \ifkorrekturansicht voraus (gesetzt in den
%% einbindenden Dateien latex-korrekturansicht-abspann.tex bzw.
%% latex-leseansicht-abspann.tex).
%% ---------------------------------------------------------------

\normalsize

% Das esempio-Environment wird nur in der Leseansicht benötigt
\ifkorrekturansicht\else
\newenvironment{esempio}[3]%
{
    \vspace{1.5ex}
    \rlap{\underline{#1}}
    \par
    \setlength{\parindent}{0cm}
    \nopagebreak
    \leftskip=#2cm
    \rightskip=#3cm
}
{
    \par
}
\fi

\doendnotes{C}
\bigskip
\vfill

\clearpage

\footnotesize

\ifkorrekturansicht
  \lohead{\textsc{register}}
\fi

% theindex-Environment neu definieren ohne reledmac
\makeatletter
\renewenvironment{theindex}{%
  \ifkorrekturansicht
    \section*{\indexname}%
  \else
    \subsubsection*{Index der erwähnten Entitäten}%
  \fi
  \setlength{\parindent}{0pt}%
  \setlength{\parskip}{0pt plus 0.3pt}%
  \let\item\@idxitem
}{%
  \ifkorrekturansicht\clearpage\fi
}
\makeatother

\IfFileExists{\jobname-pw.ind}{\input{\jobname-pw.ind}}{}

% Quellenangabe nur in der Leseansicht
\ifkorrekturansicht\else
% Fallback-Definitionen, falls die .tex-Datei \titel etc. nicht gesetzt hat
\providecommand{\titel}{}
\providecommand{\editorInnen}{}
\providecommand{\dateiname}{\jobname}

\vspace{3cm}

\vfill

\footnotesize
\textsc{Quelle}: \titel. Herausgegeben von {\editorInnen}. In: \emph{Arthur Schnitzler: Briefwechsel mit Autorinnen und Autoren}.
 Digitale Edition, https://schnitzler-briefe.acdh.oeaw.ac.at/{\dateiname}.html (Stand \today)
\fi

\end{document}


