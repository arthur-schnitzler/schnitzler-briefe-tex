%% latex-korrekturansicht-vorspann.tex
%% Vorspann für die Korrekturansicht.
%% Lädt die gemeinsame Datei latex-vorspann.tex mit gesetztem Schalter.

\newif\ifkorrekturansicht
\korrekturansichttrue

\input{../tex-inputs/latex-vorspann}


\section[ Arthur Schnitzler an Felix Salten, 26. 4. 1897]{L02963 Arthur Schnitzler an Felix Salten, 26. 4. 1897}
\nopagebreak\mylabel{L02963v}
\rehead{ }\normalsize\beginnumbering\briefempfaengerindex{Salten, Felix@\textsc{Salten, Felix}!zzzSchnitzler, Arthur@\emph{von Arthur Schnitzler}!1897-04-263@{26. 4. 1897}|(be}
\toendnotes[C]{\smallbreak\pagebreak[2]}\Standort{Wienbibliothek im Rathaus, ZPH 1681, 2.1.516.}
\physDesc{Brief, 1 Blatt, 3 Seiten, 632 Zeichen
\newline{}Handschrift: schwarze Tinte, deutsche Kurrent
\newline{}Ordnung: mit Bleistift von unbekannter Hand Nummerierung der Doppelseiten des
                                 Konvoluts: »76«–»77« }
\buchAbdrucke{\weitereDrucke{Arthur Schnitzler: \emph{Briefe 1875–1912}. Frankfurt am Main: \emph{S. Fischer} 1981, S. 317.} }\toendnotes[C]{\smallbreak}
\pstart
           \raggedleft{}{\pb}\textsc{5 rue de Maubeuge}\oindex{rue de Maubeuge@\textbf{rue de Maubeuge}, \emph{Straße (K.STR)}|pw}\pend
           
\pstart
           \raggedleft{}\textsc{Paris\oindex{Paris@\textbf{Paris}, \emph{P.PPLC}|pw}{ }26. 4. 97}. \pend
           
\pstart{}lieber Freund,\pend\vspace{0.5em}
\pstart
           \label{K_L02963-11v}\edtext{Richard\pwindex{Beer-Hofmann, Richard 1866-07-11 – 1945-09-26@\textsc{Beer-Hofmann, Richard} (1866-07-11 – 1945-09-26), \emph{Schriftsteller/Schriftstellerin}|pw} ſchreibt mir}{\lemma{\textnormal{\emph{Richard ſchreibt mir}}}\Cendnote{\textnormal{Siehe Richard Beer-Hofmann an Arthur Schnitzler, 21. 4. 1897.
               }}}\label{K_L02963-11}, Sie ſind
               wenige Tage verreiſt? Wie? wo? –\pend
           
\pstart
           Ich hab\textcolor{gray}{e} mir hier\oindex{Paris@\textbf{Paris}, \emph{P.PPLC}|pwv} mein Leben ſo
               gut als möglich eingerichtet und bin trotz \label{K_L02963-1v}\edtext{»Thür an Thür«}{\lemma{\textnormal{\emph{»Thür an Thür«}}}\Cendnote{\textnormal{Schnitzler war seit 12. 4. 1897 und noch
                  bis 23. 5. 1897
                  gemeinsam mit seiner schwangeren Partnerin Marie Reinhard\pwindex{Reinhard, Marie 1871-03-13 – 1899-03-18@\textsc{Reinhard, Marie} (1871-03-13 – 1899-03-18), \emph{Gesangspädagoge/Gesangspädagogin}|pwk} in Paris\oindex{Paris@\textbf{Paris}, \emph{P.PPLC}|pwk}.}}}\label{K_L02963-1} leidlich {\pb}ungeſtört. Auch hat es ſogar ſein angenehmes.
               Theater, jeden Abend – wie wird man fertig? – Muſeen – jeden Tag – wie wird man
               fertig? Wohne recht wohl, ſpeiſe nicht übel. – Arbeite nichts; bin aber ſehr
               aufnahmsfähig. – {\pb}Entbehre Pilſner u
               Virginier mit afrika\oindex{Afrika@\textbf{Afrika}, \emph{L.CONT}|pw}reiſender Leichtigkeit.
                  Ko{\geminationm}e mir vor wie einer, der Strapazen gewachſen iſt.
               –\pend
           
\pstart
           Einzelheiten in Wien\oindex{Wien@\textbf{Wien}, \emph{A.ADM2}|pw}.\pend
           
\pstart
           Sagen Sie mir, wie es Ihnen geht, in jeder Beziehung. Herzlich {\\}Ihr
                  \spacefill\mbox{Arthur Sch}\pend
           \selectlanguage{ngerman}\endnumbering\briefempfaengerindex{Salten, Felix@\textsc{Salten, Felix}!zzzSchnitzler, Arthur@\emph{von Arthur Schnitzler}!1897-04-263@{26. 4. 1897}|)be}\mylabel{L02963h}  \normalsize

\doendnotes{C}
\bigskip
\vfill

\clearpage

\footnotesize

\lohead{\textsc{register}}

% Definiere theindex-Environment komplett neu ohne reledmac
\makeatletter
\renewenvironment{theindex}{%
  \section*{\indexname}%
  \setlength{\parindent}{0pt}%
  \setlength{\parskip}{0pt plus 0.3pt}%
  \let\item\@idxitem
}{%
  \clearpage
}
\makeatother

\IfFileExists{\jobname-pw.ind}{\input{\jobname-pw.ind}}{}

\end{document}

      