%% latex-leseansicht-vorspann.tex
%% Vorspann für die Leseansicht.
%% Lädt die gemeinsame Datei latex-vorspann.tex mit nicht gesetztem Schalter.

\newif\ifkorrekturansicht
\korrekturansichtfalse

\input{../tex-inputs/latex-vorspann}

\begin{center}
            \textcolor{red}{ENTWURF, NICHT FERTIG KORRIGIERT}
                      \end{center}
            
         
         \newcommand{\erwaehntePersonen}{Personen: Richard Beer-Hofmann, Felix Salten}
         \newcommand{\erwaehnteOrte}{Orte: Paris, Wien, rue de Maubeuge}
         \newcommand{\erwaehnteWerke}{
               \section[Arthur Schnitzler an Felix Salten, 26. 4. 1897]{ Arthur Schnitzler an Felix Salten, 26. 4. 1897}\nopagebreak\mylabel{v}\rehead{ }\begin{ledgroupsized}[t]{13cm}\normalsize\beginnumbering \toendnotes[C]{\smallbreak\pagebreak[2]} \Standort{Wienbibliothek im Rathaus, ZPH 1681, 2.1.516.}
\physDesc{
\newline{}Handschrift: , deutsche Kurrent}\pstart
           \noindent{}\raggedleft{}{\pb}5 rue de Maubeuge\oindex{rue de Maubeuge@\textbf{rue de Maubeuge}|pw}\pend
           \pstart
           \raggedleft{}\textsc{Paris\oindex{Paris@\textbf{Paris}|pw}{ }26. 4. 97}\pend
           \pstart{}lieber Freund,\pend\pstart
           Richard\pwindex{Beer-Hofmann, Richard 1866-07-11 – 1945-09-26@\textsc{Beer-Hofmann, Richard} (1866-07-11 – 1945-09-26), \emph{Schriftsteller}|pw} ſchreibt mir Sie ſind wenige Tage
               verreiſt? Wie? wo? Ich habe mir hier mein Leben ſo gut als möglich eingerichtet und
               bin trotz »Thür an Thür« leidlich {\pb}ungeſtört. Auch hat es ſogar ſein angenehmes. Theater, jeden Abend – wie wird man
               fertig? – Muſeen – jeden Tag – wie wir man fertig? Wohne recht wohl, ſpeiſe nicht
               übel. Arbeite nichts; bin aber ſehr aufnahmsfähig.– {\pb}Entbehre Pilſner u Virginier mit
               afrikareiſender Leichtigkeit. Ko{\geminationm}e mir vor wie einer,
               der Strapazen gewachſsen iſt.– \pend
           \pstart
           Einzelheiten in Wien\oindex{Wien@\textbf{Wien}|pw}. \pend
           \pstart
           Sagen Sie mir, wie es Ihnen geht, in jeder Beziehung. Herzlich \pend
           
         
         \endnumbering\mylabel{h}\end{ledgroupsized}\begin{anhang}\end{anhang}\newcommand{\dateiname}{L02963}\newcommand{\titel}{Arthur Schnitzler an Felix Salten, 26. 4. 1897}\newcommand{\editorInnen}{Martin Anton Müller und Laura Untner}%% latex-leseansicht-abspann.tex
%% Abspann für die Leseansicht.
%% Der Schalter \ifkorrekturansicht ist bereits durch den Vorspann gesetzt.

%% latex-abspann.tex
%% Gemeinsamer Abspann für Korrekturansicht und Leseansicht.
%% Setzt den Schalter \ifkorrekturansicht voraus (gesetzt in den
%% einbindenden Dateien latex-korrekturansicht-abspann.tex bzw.
%% latex-leseansicht-abspann.tex).
%% ---------------------------------------------------------------

\normalsize

% Das esempio-Environment wird nur in der Leseansicht benötigt
\ifkorrekturansicht\else
\newenvironment{esempio}[3]%
{
    \vspace{1.5ex}
    \rlap{\underline{#1}}
    \par
    \setlength{\parindent}{0cm}
    \nopagebreak
    \leftskip=#2cm
    \rightskip=#3cm
}
{
    \par
}
\fi

\doendnotes{C}
\bigskip
\vfill

\clearpage

\footnotesize

\ifkorrekturansicht
  \lohead{\textsc{register}}
\fi

% theindex-Environment neu definieren ohne reledmac
\makeatletter
\renewenvironment{theindex}{%
  \ifkorrekturansicht
    \section*{\indexname}%
  \else
    \subsubsection*{Index der erwähnten Entitäten}%
  \fi
  \setlength{\parindent}{0pt}%
  \setlength{\parskip}{0pt plus 0.3pt}%
  \let\item\@idxitem
}{%
  \ifkorrekturansicht\clearpage\fi
}
\makeatother

\IfFileExists{\jobname-pw.ind}{\input{\jobname-pw.ind}}{}

% Quellenangabe nur in der Leseansicht
\ifkorrekturansicht\else
% Fallback-Definitionen, falls die .tex-Datei \titel etc. nicht gesetzt hat
\providecommand{\titel}{}
\providecommand{\editorInnen}{}
\providecommand{\dateiname}{\jobname}

\vspace{3cm}

\vfill

\footnotesize
\textsc{Quelle}: \titel. Herausgegeben von {\editorInnen}. In: \emph{Arthur Schnitzler: Briefwechsel mit Autorinnen und Autoren}.
 Digitale Edition, https://schnitzler-briefe.acdh.oeaw.ac.at/{\dateiname}.html (Stand \today)
\fi

\end{document}


      