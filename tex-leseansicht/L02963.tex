%% latex-leseansicht-vorspann.tex
%% Vorspann für die Leseansicht.
%% Lädt die gemeinsame Datei latex-vorspann.tex mit nicht gesetztem Schalter.

\newif\ifkorrekturansicht
\korrekturansichtfalse

\input{../tex-inputs/latex-vorspann}


\section[ Arthur Schnitzler an Felix Salten, 26. 4. 1897]{L02963 Arthur Schnitzler an Felix Salten,  26. 4. 1897}
\nopagebreak\mylabel{L02963v}
\rehead{ }\normalsize\beginnumbering\briefempfaengerindex{Salten, Felix@\textsc{Salten, Felix}!zzzSchnitzler, Arthur@\emph{von Arthur Schnitzler}!1897-04-263@{26. 4. 1897}|(be}
\toendnotes[C]{\smallbreak\pagebreak[2]}
\correspDesc{Versand  durch Arthur Schnitzler am 26. 4. 1897 in Paris
\newline{}Erhalt  durch Felix Salten im Zeitraum [27. 4. 1897
                  – 30. 4. 1897?] in Wien}\toendnotes[C]{\smallbreak}
\Standort{Wienbibliothek im Rathaus, ZPH 1681, 2.1.516.}
\physDesc{Brief, 1 Blatt, 3 Seiten, 632 Zeichen
\newline{}Handschrift: schwarze Tinte, deutsche Kurrent
\newline{}Ordnung: mit Bleistift von unbekannter Hand Nummerierung der Doppelseiten des
                                 Konvoluts: »76«–»77« }
\buchAbdrucke{\weitereDrucke{Arthur Schnitzler: \emph{Briefe 1875–1912}. Herausgegeben von Therese Nickl und Heinrich Schnitzler. Frankfurt am Main: \emph{S. Fischer} 1981, S. 317.} }\toendnotes[C]{\smallbreak}
\pstart
           \raggedleft{}{\pb}\textsc{5 rue de Maubeuge}\oindex{5, rue de Maubeuge@\textbf{5, rue de Maubeuge}, \emph{Wohngebäude}|pw}\pend
           
\pstart
           \raggedleft{}\textsc{Paris\oindex{Paris@\textbf{Paris}, \emph{Hauptstadt}|pw}{ }26. 4. 97}.\pend
           
\pstart{}lieber Freund,\pend\vspace{0.5em}
\pstart
           \label{K_L02963-11v}\edtext{Richard\pwindex{Beer-Hofmann, Richard 11.\,7.\,1866 Wien – 26.\,9.\,1945 New York City@\textsc{Beer-Hofmann, Richard} (11.\,7.\,1866 Wien – 26.\,9.\,1945 New York City), \emph{Schriftsteller}|pw}{ }ſchreibt mir}{\lemma{\textnormal{\emph{Richard schreibt mir}}}\Cendnote{\textnormal{Siehe XXXX Auszeichnungsfehler: Dokument L00667 nicht gefunden.
               }}}\label{K_L02963-11}, Sie{ }ſind
               wenige Tage verreiſt? Wie? wo? –\pend
           
\pstart
           Ich hab\textcolor{gray}{e} mir hier\oindex{Paris@\textbf{Paris}, \emph{Hauptstadt}|pwv} mein Leben{ }ſo
               gut als möglich eingerichtet und bin trotz \label{K_L02963-1v}\edtext{»Thür an Thür«}{\lemma{\textnormal{\emph{»Thür an Thür«}}}\Cendnote{\textnormal{Schnitzler war seit 12. 4. 1897 und noch
                  bis 23. 5. 1897
                  gemeinsam mit seiner schwangeren Partnerin Marie Reinhard\pwindex{Reinhard, Marie 13.\,3.\,1871 Wien – 18.\,3.\,1899 ebd.@\textsc{Reinhard, Marie} (13.\,3.\,1871 Wien – 18.\,3.\,1899 ebd.), \emph{Gesangspädagogin}|pwk} in Paris\oindex{Paris@\textbf{Paris}, \emph{Hauptstadt}|pwk}.}}}\label{K_L02963-1} leidlich {\pb}ungeſtört. Auch hat es{ }ſogar{ }ſein angenehmes.
               Theater, jeden Abend – wie wird man fertig? – Muſeen – jeden Tag – wie wird man
               fertig? Wohne recht wohl,{ }ſpeiſe nicht übel. – Arbeite nichts; bin aber{ }ſehr
               aufnahmsfähig. – {\pb}Entbehre Pilſner u
               Virginier mit afrika\oindex{Afrika@\textbf{Afrika}|pw}reiſender Leichtigkeit.
                  Ko{\geminationm}e mir vor wie einer, der Strapazen gewachſen iſt.
               –\pend
           
\pstart
           Einzelheiten in Wien\oindex{Wien@\textbf{Wien}, \emph{Verwaltungsgebiet}|pw}.\pend
           
\pstart
           Sagen Sie mir, wie es Ihnen geht, in jeder Beziehung. Herzlich {\\}Ihr
                  \spacefill\mbox{Arthur Sch}\pend
           \selectlanguage{ngerman}\endnumbering\briefempfaengerindex{Salten, Felix@\textsc{Salten, Felix}!zzzSchnitzler, Arthur@\emph{von Arthur Schnitzler}!1897-04-263@{26. 4. 1897}|)be}\mylabel{L02963h}  \newcommand{\dateiname}{L02963}\newcommand{\titel}{Arthur Schnitzler an Felix Salten, 26. 4. 1897}\newcommand{\editorInnen}{Martin Anton Müller und Laura Untner}%% latex-leseansicht-abspann.tex
%% Abspann für die Leseansicht.
%% Der Schalter \ifkorrekturansicht ist bereits durch den Vorspann gesetzt.

%% latex-abspann.tex
%% Gemeinsamer Abspann für Korrekturansicht und Leseansicht.
%% Setzt den Schalter \ifkorrekturansicht voraus (gesetzt in den
%% einbindenden Dateien latex-korrekturansicht-abspann.tex bzw.
%% latex-leseansicht-abspann.tex).
%% ---------------------------------------------------------------

\normalsize

% Das esempio-Environment wird nur in der Leseansicht benötigt
\ifkorrekturansicht\else
\newenvironment{esempio}[3]%
{
    \vspace{1.5ex}
    \rlap{\underline{#1}}
    \par
    \setlength{\parindent}{0cm}
    \nopagebreak
    \leftskip=#2cm
    \rightskip=#3cm
}
{
    \par
}
\fi

\doendnotes{C}
\bigskip
\vfill

\clearpage

\footnotesize

\ifkorrekturansicht
  \lohead{\textsc{register}}
\fi

% theindex-Environment neu definieren ohne reledmac
\makeatletter
\renewenvironment{theindex}{%
  \ifkorrekturansicht
    \section*{\indexname}%
  \else
    \subsubsection*{Index der erwähnten Entitäten}%
  \fi
  \setlength{\parindent}{0pt}%
  \setlength{\parskip}{0pt plus 0.3pt}%
  \let\item\@idxitem
}{%
  \ifkorrekturansicht\clearpage\fi
}
\makeatother

\IfFileExists{\jobname-pw.ind}{\input{\jobname-pw.ind}}{}

% Quellenangabe nur in der Leseansicht
\ifkorrekturansicht\else
% Fallback-Definitionen, falls die .tex-Datei \titel etc. nicht gesetzt hat
\providecommand{\titel}{}
\providecommand{\editorInnen}{}
\providecommand{\dateiname}{\jobname}

\vspace{3cm}

\vfill

\footnotesize
\textsc{Quelle}: \titel. Herausgegeben von {\editorInnen}. In: \emph{Arthur Schnitzler: Briefwechsel mit Autorinnen und Autoren}.
 Digitale Edition, https://schnitzler-briefe.acdh.oeaw.ac.at/{\dateiname}.html (Stand \today)
\fi

\end{document}


