%% latex-korrekturansicht-vorspann.tex
%% Vorspann für die Korrekturansicht.
%% Lädt die gemeinsame Datei latex-vorspann.tex mit gesetztem Schalter.

\newif\ifkorrekturansicht
\korrekturansichttrue

\input{../tex-inputs/latex-vorspann}


\section[ Felix Salten an Arthur Schnitzler, 17. 8. 1893]{L03125 Felix Salten an Arthur Schnitzler, 17. 8. 1893}
\nopagebreak\mylabel{L03125v}
\rehead{ }\normalsize\beginnumbering\briefempfaengerindex{Schnitzler, Arthur@\textsc{Schnitzler, Arthur}!zzzSalten, Felix@\emph{von Felix Salten}!1893-08-172@{17. 8. 1893}|(be}
\toendnotes[C]{\smallbreak\pagebreak[2]}\Standort{CUL, Schnitzler, B 89, A 1.}
\physDesc{Bildpostkarte, 512 Zeichen
\newline{}Handschrift: Bleistift, lateinische Kurrent
\newline{}Versand: 1) Stempel: »\nobreak{}\oindex{Heiligenblut am Grossglockner@\textbf{Heiligenblut am Großglockner}, \emph{P.PPLA3}|pwk}Heilig{[}enbl{]}ut, 18/8 93\nobreak{}«.   2) Stempel: »\nobreak{}\oindex{I., Innere Stadt@\textbf{I., Innere Stadt}, \emph{A.ADM3}|pwk}Wien 1/1 1, 19/8. 93, 11½V–1N, Bestellt\nobreak{}«. 
\newline{}Ordnung: mit Bleistift von unbekannter Hand nummeriert: »28« }\toendnotes[C]{\smallbreak}\pstart{}{\pb}Herrn D\textsuperscript{r} Arthur Schnitzler\pend{}\pstart{}Wien\oindex{Wien@\textbf{Wien}, \emph{A.ADM2}|pw}\pend{}\pstart{}I. Grillparzerstraße 7\oindex{Grillparzerstrasse@\textbf{Grillparzerstraße}, \emph{R.ST}|pw}.\pend{}{\bigskip}
\pstart
           \noindent{}\centering{}{\pb}\textcolor{gray}{\textbf{\textbf{»Gruss aus Heiligenblut\oindex{Heiligenblut am Grossglockner@\textbf{Heiligenblut am Großglockner}, \emph{P.PPLA3}|pw}«
                     (Kärnten\oindex{Kaernten@\textbf{Kärnten}, \emph{A.ADM1}|pw}).}}}\pend
           
\pstart
           \centering{}\textcolor{gray}{\textbf{Robert Bernard’s Gasthof\oindex{Robert BernarDs Gasthof@\textbf{Robert Bernard’s Gasthof}, \emph{Gastgewerbegebäude (K.GGW)}|pw}.}}\pend
           \vspace{1em}
\pstart
           \raggedleft{}{\pb}17. VIII. 93\pend
           \vspace{0.5em}
\pstart
           Lieber Freund! Von Cortina\oindex{Cortina DAmpezzo@\textbf{Cortina d’Ampezzo}, \emph{P.PPLA3}|pw} zurück, befinde ich mich auf einer 2tägigen Tour auf den Glockner\oindex{Grossglockner@\textbf{Großglockner}, \emph{T.MT}|pw}. Hier mit meinem Bruder\pwindex{Salzmann, Michael Emil 1858-01-19 – 1908-06-26@\textsc{Salzmann, Michael Emil} (1858-01-19 – 1908-06-26), \emph{Versicherungsbeamter/Versicherungsbeamtin}|pwuv}. Ich danke herzlich
               für Ihren \label{K_L03125-1v}\edtext{Brief}{\lemma{\textnormal{\emph{Brief}}}\Cendnote{\textnormal{Arthur Schnitzler an Felix Salten, [14. 8. 1893].
               }}}\label{K_L03125-1}, den ich nach Rückkehr ausführlich beantworte. Für heute nur die unangenehme Mittheilung, dass mein Rad zwischen Mittewald\oindex{Mittewald an der Drau@\textbf{Mittewald an der Drau}, \emph{P.PPL}|pw}{ }{\kaufmannsund}{ }Lienz\oindex{Lienz@\textbf{Lienz}, \emph{P.PPLA3}|pw} gebrochen ist, u. sich in Lienz\oindex{Lienz@\textbf{Lienz}, \emph{P.PPLA3}|pw} zur Reparatur befindet. Da das \label{K_L03125-2v}\edtext{Gouvernal}{\lemma{\textnormal{\emph{Gouvernal}}}\Cendnote{\textnormal{Fahrradlenker}}}\label{K_L03125-2} verletzt ist, dürfte die Sache länger dauern, ich \label{K_L03125-3v}\edtext{schreibe oder telegrafire noch am Samstag}{\lemma{\textnormal{\emph{schreibe … Samstag}}}\Cendnote{\textnormal{Er schrieb sogar einen Tag früher, siehe Felix Salten an Arthur Schnitzler, 18. 8. 1893.}}}\label{K_L03125-3}\pend
           
\pstart
           Herzl.{\\[\baselineskip]} Ihr{\\[\baselineskip]}\spacefill\mbox{Salten.}\pend
           \leftskip=0em{}\selectlanguage{ngerman}\endnumbering\briefempfaengerindex{Schnitzler, Arthur@\textsc{Schnitzler, Arthur}!zzzSalten, Felix@\emph{von Felix Salten}!1893-08-172@{17. 8. 1893}|)be}\mylabel{L03125h}  \normalsize

\doendnotes{C}
\bigskip
\vfill

\clearpage

\footnotesize

\lohead{\textsc{register}}

% Definiere theindex-Environment komplett neu ohne reledmac
\makeatletter
\renewenvironment{theindex}{%
  \section*{\indexname}%
  \setlength{\parindent}{0pt}%
  \setlength{\parskip}{0pt plus 0.3pt}%
  \let\item\@idxitem
}{%
  \clearpage
}
\makeatother

\IfFileExists{\jobname-pw.ind}{\input{\jobname-pw.ind}}{}

\end{document}

      