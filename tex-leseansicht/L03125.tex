%% latex-leseansicht-vorspann.tex
%% Vorspann für die Leseansicht.
%% Lädt die gemeinsame Datei latex-vorspann.tex mit nicht gesetztem Schalter.

\newif\ifkorrekturansicht
\korrekturansichtfalse

\input{../tex-inputs/latex-vorspann}

\begin{center}
            \textcolor{red}{ENTWURF, NICHT FERTIG KORRIGIERT}
                      \end{center}
            
         
         \renewcommand{\erwaehntePersonen}{Personen: Michael Emil Salzmann}
         \renewcommand{\erwaehnteOrte}{Orte: Cortina d'Ampezzo, Grillparzerstraße, Großglockner, Heiligenblut am Großglockner, I., Innere Stadt, Kärnten, Lienz, Mittewald an der Drau, Robert Bernard’s Gasthof, Wien}
         \renewcommand{\erwaehnteWerke}{}
               \section[Felix Salten an Arthur Schnitzler, 17. 8. 1893]{ Felix Salten an Arthur Schnitzler, 17. 8. 1893}\nopagebreak\mylabel{v}\rehead{ }\begin{ledgroupsized}[t]{13cm}\normalsize\beginnumbering \toendnotes[C]{\smallbreak\pagebreak[2]} \Standort{CUL, Schnitzler, B 89, A 1.}
\physDesc{Bildpostkarte
\newline{}Handschrift: Bleistift, lateinische Kurrent\newline{}Versand: 1) Stempel: »\nobreak{}\oindex{Heiligenblut am Grossglockner@\textbf{Heiligenblut am Großglockner}|pwk}Heilig{[}enbl{]}ut, 18/8 93\nobreak{}«.   2) Stempel: »\nobreak{}\oindex{I., Innere Stadt@\textbf{I., Innere Stadt}|pwk}Wien 1/1 1, 19. 8. 93, 11½V–1N, Bestellt\nobreak{}«. \newline{}Ordnung: mit Bleistift von unbekannter Hand nummeriert: »28« }\toendnotes[C]{\smallbreak}\pstart{}{\pb}Herrn D\textsuperscript{r} Arthur Schnitzler \pend{}\pstart{}\textsc{Wien\oindex{Wien@\textbf{Wien}|pw}}\pend{}\pstart{}I. Grillparzerstraße 7\oindex{Grillparzerstrasse@\textbf{Grillparzerstraße}|pw}. \pend{}{\bigskip}\pstart
           \noindent{}\centering{}{\pb}\textcolor{gray}{\textbf{»Gruss aus Heiligenblut\oindex{Heiligenblut am Grossglockner@\textbf{Heiligenblut am Großglockner}|pw}« (Kärnten\oindex{Kaernten@\textbf{Kärnten}|pw}).}}\pend
           \pstart
           \noindent{}\centering{}\textcolor{gray}{\textbf{Robert Bernard’s Gasthof\oindex{Robert BernarDs Gasthof@\textbf{Robert Bernard’s Gasthof}|pw}. }}\pend
           \pstart
           \raggedleft{}17. VIII. 93\pend
           \pstart
           Lieber Freund! Von Cortina\oindex{Cortina d'Ampezzo@\textbf{Cortina d'Ampezzo}|pw}
               zurück, befinde ich mich auf einer 2tägigen Tour auf den Glockner\oindex{Grossglockner@\textbf{Großglockner}|pw}. Hier mit meinem Bruder\pwindex{Salzmann, Michael Emil 1858-01-19 – 1908-06-26@\textsc{Salzmann, Michael Emil} (1858-01-19 – 1908-06-26), \emph{Versicherungsbeamter}|pwuv}. Ich danke herzlich für Ihren Brief, den ich nach
               Rückkehr ausführlich beantworte. Für heute nur die unangenehme Mittheilung, dass mein
               Rad zwischen Mittewald\oindex{Mittewald an der Drau@\textbf{Mittewald an der Drau}|pw}{ }{\kaufmannsund}{ }Lienz\oindex{Lienz@\textbf{Lienz}|pw} gebrochen ist u. sich in Lienz\oindex{Lienz@\textbf{Lienz}|pw} zur Reparatur befindet, Da das
                  \textcolor{gray}{Vorderrad} verletzt ist, dürfte die Sache länger dauern, ich
               schreibe oder telegrafire noch am Samstag.\pend
           \pstart Herzl. Ihr \spacefill\mbox{Salten}\pend{}
         
         \endnumbering\mylabel{h}\end{ledgroupsized}\begin{anhang}\end{anhang}\newcommand{\dateiname}{L03125}\newcommand{\titel}{Felix Salten an Arthur Schnitzler, 17. 8. 1893}\newcommand{\editorInnen}{Martin Anton Müller und Laura Untner}%% latex-leseansicht-abspann.tex
%% Abspann für die Leseansicht.
%% Der Schalter \ifkorrekturansicht ist bereits durch den Vorspann gesetzt.

%% latex-abspann.tex
%% Gemeinsamer Abspann für Korrekturansicht und Leseansicht.
%% Setzt den Schalter \ifkorrekturansicht voraus (gesetzt in den
%% einbindenden Dateien latex-korrekturansicht-abspann.tex bzw.
%% latex-leseansicht-abspann.tex).
%% ---------------------------------------------------------------

\normalsize

% Das esempio-Environment wird nur in der Leseansicht benötigt
\ifkorrekturansicht\else
\newenvironment{esempio}[3]%
{
    \vspace{1.5ex}
    \rlap{\underline{#1}}
    \par
    \setlength{\parindent}{0cm}
    \nopagebreak
    \leftskip=#2cm
    \rightskip=#3cm
}
{
    \par
}
\fi

\doendnotes{C}
\bigskip
\vfill

\clearpage

\footnotesize

\ifkorrekturansicht
  \lohead{\textsc{register}}
\fi

% theindex-Environment neu definieren ohne reledmac
\makeatletter
\renewenvironment{theindex}{%
  \ifkorrekturansicht
    \section*{\indexname}%
  \else
    \subsubsection*{Index der erwähnten Entitäten}%
  \fi
  \setlength{\parindent}{0pt}%
  \setlength{\parskip}{0pt plus 0.3pt}%
  \let\item\@idxitem
}{%
  \ifkorrekturansicht\clearpage\fi
}
\makeatother

\IfFileExists{\jobname-pw.ind}{\input{\jobname-pw.ind}}{}

% Quellenangabe nur in der Leseansicht
\ifkorrekturansicht\else
% Fallback-Definitionen, falls die .tex-Datei \titel etc. nicht gesetzt hat
\providecommand{\titel}{}
\providecommand{\editorInnen}{}
\providecommand{\dateiname}{\jobname}

\vspace{3cm}

\vfill

\footnotesize
\textsc{Quelle}: \titel. Herausgegeben von {\editorInnen}. In: \emph{Arthur Schnitzler: Briefwechsel mit Autorinnen und Autoren}.
 Digitale Edition, https://schnitzler-briefe.acdh.oeaw.ac.at/{\dateiname}.html (Stand \today)
\fi

\end{document}


      