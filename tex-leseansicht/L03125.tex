%% latex-leseansicht-vorspann.tex
%% Vorspann für die Leseansicht.
%% Lädt die gemeinsame Datei latex-vorspann.tex mit nicht gesetztem Schalter.

\newif\ifkorrekturansicht
\korrekturansichtfalse

\input{../tex-inputs/latex-vorspann}


\section[ Felix Salten an Arthur Schnitzler, 17. 8. 1893]{L03125 Felix Salten an Arthur Schnitzler,  17. 8. 1893}
\nopagebreak\mylabel{L03125v}
\rehead{ }\normalsize\beginnumbering\briefempfaengerindex{Schnitzler, Arthur@\textsc{Schnitzler, Arthur}!zzzSalten, Felix@\emph{von Felix Salten}!1893-08-172@{17. 8. 1893}|(be}
\toendnotes[C]{\smallbreak\pagebreak[2]}
\correspDesc{Versand  durch Felix Salten am 17. 8. 1893 in Heiligenblut am Großglockner
\newline{}Übermittlung  am 18. 8. 1893 in Heiligenblut am Großglockner
\newline{}Erhalt  durch Arthur Schnitzler am 19. 8. 1893 in Wien}\toendnotes[C]{\smallbreak}
\Standort{CUL, Schnitzler, B 89, A 1.}
\physDesc{Bildpostkarte, 512 Zeichen
\newline{}Handschrift: Bleistift, lateinische Kurrent
\newline{}Versand: 1) Stempel: »\nobreak{}\oindex{Heiligenblut am Großglockner@\textbf{Heiligenblut am Großglockner}, \emph{Hauptstadt}|pwk}Heilig{[}enbl{]}ut, 18/8 93\nobreak{}«.   2) Stempel: »\nobreak{}\oindex{I., Innere Stadt@\textbf{I., Innere Stadt}, \emph{Verwaltungsgebiet}|pwk}Wien 1/1 1, 19/8. 93, 11½V–1N, Bestellt\nobreak{}«. 
\newline{}Ordnung: mit Bleistift von unbekannter Hand nummeriert: »28« }\toendnotes[C]{\smallbreak}\pstart{}{\pb}Herrn D\textsuperscript{r} Arthur Schnitzler\pend{}\pstart{}Wien\oindex{Wien@\textbf{Wien}, \emph{Verwaltungsgebiet}|pw}\pend{}\pstart{}I. Grillparzerstraße 7\oindex{Wien@\textbf{Wien}!I., Innere Stadt@\textbf{I., Innere Stadt}!Grillparzerstraße@\textbf{Grillparzerstraße}, \emph{Straße}|pw}.\pend{}{\bigskip}
\pstart
           \noindent{}\centering{}{\pb}\textcolor{gray}{\textbf{\textbf{»Gruss aus Heiligenblut\oindex{Heiligenblut am Großglockner@\textbf{Heiligenblut am Großglockner}, \emph{Hauptstadt}|pw}«
                     (Kärnten\oindex{Kärnten@\textbf{Kärnten}, \emph{Land}|pw}).}}}\pend
           
\pstart
           \centering{}\textcolor{gray}{\textbf{Robert Bernard’s Gasthof\oindex{Robert Bernard’s Gasthof@\textbf{Robert Bernard’s Gasthof}, \emph{Gastgewerbegebäude}|pw}.}}\pend
           \vspace{1em}
\pstart
           \raggedleft{}{\pb}17. VIII. 93\pend
           \vspace{0.5em}
\pstart
           Lieber Freund! Von Cortina\oindex{Cortina d’Ampezzo@\textbf{Cortina d’Ampezzo}, \emph{Hauptstadt}|pw} zurück, befinde ich mich auf einer 2tägigen Tour auf den Glockner\oindex{Großglockner@\textbf{Großglockner}, \emph{Berg}|pw}. Hier mit meinem Bruder\pwindex{Salzmann, Michael Emil 19.\,1.\,1858 Szigetvár – 26.\,6.\,1908 Wien@\textsc{Salzmann, Michael Emil} (19.\,1.\,1858 Szigetvár – 26.\,6.\,1908 Wien), \emph{Versicherungsbeamter}|pwuv}. Ich danke herzlich
               für Ihren \label{K_L03125-1v}\edtext{Brief}{\lemma{\textnormal{\emph{Brief}}}\Cendnote{\textnormal{XXXX Auszeichnungsfehler: Dokument L02960 nicht gefunden.
               }}}\label{K_L03125-1}, den ich nach Rückkehr ausführlich beantworte. Für heute nur die unangenehme Mittheilung, dass mein Rad zwischen Mittewald\oindex{Mittewald an der Drau@\textbf{Mittewald an der Drau}|pw}{ }{\kaufmannsund}{ }Lienz\oindex{Lienz@\textbf{Lienz}, \emph{Hauptstadt}|pw} gebrochen ist, u. sich in Lienz\oindex{Lienz@\textbf{Lienz}, \emph{Hauptstadt}|pw} zur Reparatur befindet. Da das \label{K_L03125-2v}\edtext{Gouvernal}{\lemma{\textnormal{\emph{Gouvernal}}}\Cendnote{\textnormal{Fahrradlenker}}}\label{K_L03125-2} verletzt ist, dürfte die Sache länger dauern, ich \label{K_L03125-3v}\edtext{schreibe oder telegrafire noch am Samstag}{\lemma{\textnormal{\emph{schreibe … Samstag}}}\Cendnote{\textnormal{Er schrieb sogar einen Tag früher, siehe XXXX Auszeichnungsfehler: Dokument L03128 nicht gefunden.}}}\label{K_L03125-3}\pend
           
\pstart
           Herzl.{\\[\baselineskip]} Ihr{\\[\baselineskip]}\spacefill\mbox{Salten.}\pend
           \leftskip=0em{}\selectlanguage{ngerman}\endnumbering\briefempfaengerindex{Schnitzler, Arthur@\textsc{Schnitzler, Arthur}!zzzSalten, Felix@\emph{von Felix Salten}!1893-08-172@{17. 8. 1893}|)be}\mylabel{L03125h}  \newcommand{\dateiname}{L03125}\newcommand{\titel}{Felix Salten an Arthur Schnitzler, 17. 8. 1893}\newcommand{\editorInnen}{Martin Anton Müller und Laura Untner}%% latex-leseansicht-abspann.tex
%% Abspann für die Leseansicht.
%% Der Schalter \ifkorrekturansicht ist bereits durch den Vorspann gesetzt.

%% latex-abspann.tex
%% Gemeinsamer Abspann für Korrekturansicht und Leseansicht.
%% Setzt den Schalter \ifkorrekturansicht voraus (gesetzt in den
%% einbindenden Dateien latex-korrekturansicht-abspann.tex bzw.
%% latex-leseansicht-abspann.tex).
%% ---------------------------------------------------------------

\normalsize

% Das esempio-Environment wird nur in der Leseansicht benötigt
\ifkorrekturansicht\else
\newenvironment{esempio}[3]%
{
    \vspace{1.5ex}
    \rlap{\underline{#1}}
    \par
    \setlength{\parindent}{0cm}
    \nopagebreak
    \leftskip=#2cm
    \rightskip=#3cm
}
{
    \par
}
\fi

\doendnotes{C}
\bigskip
\vfill

\clearpage

\footnotesize

\ifkorrekturansicht
  \lohead{\textsc{register}}
\fi

% theindex-Environment neu definieren ohne reledmac
\makeatletter
\renewenvironment{theindex}{%
  \ifkorrekturansicht
    \section*{\indexname}%
  \else
    \subsubsection*{Index der erwähnten Entitäten}%
  \fi
  \setlength{\parindent}{0pt}%
  \setlength{\parskip}{0pt plus 0.3pt}%
  \let\item\@idxitem
}{%
  \ifkorrekturansicht\clearpage\fi
}
\makeatother

\IfFileExists{\jobname-pw.ind}{\input{\jobname-pw.ind}}{}

% Quellenangabe nur in der Leseansicht
\ifkorrekturansicht\else
% Fallback-Definitionen, falls die .tex-Datei \titel etc. nicht gesetzt hat
\providecommand{\titel}{}
\providecommand{\editorInnen}{}
\providecommand{\dateiname}{\jobname}

\vspace{3cm}

\vfill

\footnotesize
\textsc{Quelle}: \titel. Herausgegeben von {\editorInnen}. In: \emph{Arthur Schnitzler: Briefwechsel mit Autorinnen und Autoren}.
 Digitale Edition, https://schnitzler-briefe.acdh.oeaw.ac.at/{\dateiname}.html (Stand \today)
\fi

\end{document}


