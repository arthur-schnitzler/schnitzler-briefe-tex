%% latex-korrekturansicht-vorspann.tex
%% Vorspann für die Korrekturansicht.
%% Lädt die gemeinsame Datei latex-vorspann.tex mit gesetztem Schalter.

\newif\ifkorrekturansicht
\korrekturansichttrue

\input{../tex-inputs/latex-vorspann}


\section[Arthur Schnitzler an Thomas Mann, 28. 12. 1922]{L02394 Arthur Schnitzler an Thomas Mann, 28. 12. 1922}
\nopagebreak\mylabel{L02394v}
\rehead{ }\normalsize\beginnumbering\briefempfaengerindex{Mann, Thomas@\textsc{Mann, Thomas}!zzzSchnitzler, Arthur@\emph{von Arthur Schnitzler}!1922-12-281@{28. 12. 1922}|(be}
\toendnotes[C]{\smallbreak\pagebreak[2]}\Standort{DLA, A:Schnitzler, 85.1.1371,1.}
\physDesc{Brief, Durchschlag1 Blatt, 1 Seite, 2909 Zeichen
\newline{}Schreibmaschine
\newline{}Handschrift: roter Buntstift, lateinische Kurrent (\noindent{}eine Klammer, Unterstreichungen, Beschriftung: »Ma{\geminationn}« und »K{[}opie{]}«)
\newline{}Ordnung: Der grammatikalisch unvollständige Satz »Und es ist alles
                                    eher… wurde durch Ergänzung »wenn ich
                                    finde,« von unbekannter Hand (Heinrich Schnitzler\pwindex{Schnitzler, Heinrich 09.08.1902 – 12.07.1982@\textsc{Schnitzler, Heinrich} (09.08.1902 – 12.07.1982), \emph{Regisseur/Regisseurin, Schauspieler/Schauspielerin}|pw}?) richtiggestellt }
\buchAbdrucke{\weitereDrucke{1) Arthur Schnitzler: \emph{Briefe 1913–1931}. Frankfurt am Main: \emph{S. Fischer} 1984, S. 298–299.} \weitereDrucke{2) \emph{Modern Austrian Literature}, Jg. 7 (1974) Nr. 1/2, S. 19–20.} \weitereDrucke{3) Hans-Ulrich Lindken: \emph{Arthur Schnitzler. Aspekte und Akzente. Materialien zu Leben
                        und Werk}. Frankfurt am Main, Bern, Göttingen: \emph{Peter Lang} 1984, S. 401–402.} }\toendnotes[C]{\smallbreak}
\pstart
           \raggedleft{}{\pb}28. 12. 1922.\pend
           
\pstart{}Sehr verehrter Herr Thomas Mann. \pend\vspace{0.5em}
\pstart
           Das Jahr darf doch nicht zu Ende gehen, ohne dass ich Ihnen – recht sehr verspätet –
               für Ihren lieben Brief vom 4. September d. J. danke. Ihre freundlichen
               Worte über »Casanovas Heimfahrt\pwindex{Casanovas Heimfahrt@\emph{Casanovas Heimfahrt}|pw}« haben mich sehr
               gefreut. Indess hat auch dieses Werk sein Schicksal oder wenigstens seine kleine
               Affaire gehabt \introOben{}(\introOben{}\label{T_L02394-1v}\edtext{ich}{\lemma{\textnormal{\emph{ich}}}\Cendnote{\textnormal{Die Vorlage hat: »cih«.}}}\label{T_L02394-1} bin dergleichen
               ziemlich gewöhnt; –\introOben{})\introOben{} in Amerika\oindex{Vereinigte Staaten von Amerika [USA]@\textbf{Vereinigte Staaten von Amerika [USA]}, \emph{A.PCLI}|pw} hat die Gesellschaft zur Bekämpfung
                  des Lasters\orgindex{New York Society for the Suppression of Vice@The New York Society for the Suppression of Vice|pw} die Konfiskation der englischen Uebersetzung beantragt, der Verleger\pwindex{Seltzer, Thomas 22.2.1875 – 11.09.1943@\textsc{Seltzer, Thomas} (22.2.1875 – 11.09.1943), \emph{Übersetzer/Übersetzerin, Verleger/Verlegerin}|pwv} wurde in
               Anklagezustand versetzt, ich glaube sogar verhaftet, aber die Angelegenheit endete
               diesmal mit einer erheblichen Blamage der Tugendbolde und für mich
                  hatt{[}e{]} die Sache überdies den Vorteil, dass der Verleger\pwindex{Seltzer, Thomas 22.2.1875 – 11.09.1943@\textsc{Seltzer, Thomas} (22.2.1875 – 11.09.1943), \emph{Übersetzer/Übersetzerin, Verleger/Verlegerin}|pwv} in Erwartung
               künftiger Geschäfte mir einen Teil des Geldes zahlte, das er mir noch schuldig
               war.\pend
           
\pstart
           Ich höre – fällt mir in diesem Zusammenhang ein – dass Sie in Amerika\oindex{Vereinigte Staaten von Amerika [USA]@\textbf{Vereinigte Staaten von Amerika [USA]}, \emph{A.PCLI}|pw} von Kirpatrik {\kaufmannsund} Brandt\orgindex{Brandt und Kirkpatrick@Brandt {\kaufmannsund}  Kirkpatrick|pw}, den Agenten des Verlag Fischer\orgindex{S. Fischer Verlag@S. Fischer Verlag|pw}, vertreten werden. Wäre es sehr indiskret
               Sie zu fragen, ob Sie mit den Leuten gute Erfahrungen gemacht haben?\pend
           
\pstart
           Ihren Artikel\pwindex{Von deutscher Republik. Gerhart Hauptmann zum sechzigsten Geburtstag@\emph{Von deutscher Republik. Gerhart Hauptmann zum sechzigsten Geburtstag}|pwv} in der Neuen Rundschau\pwindex{neue Rundschau@\emph{Die neue Rundschau}|pw}, auf den Sie mich schon vor
               Erscheinen aufmerksam zu machen so gütig waren, habe ich natürlich mit dem grössten
               Interesse gelesen. Er ist, da Sie das Wort nun einmal lieben, im schönsten Sinne
               human. Aber ganz abgesehen von allem Inhaltlichen, selbst wenn ich nicht ganz
               einverstanden wäre, Ihrer wunderbaren Prosa würde ich mich immer erfreuen, wie mich
               eine {\pb}edle Stimme entzückte, auch wenn sie Vokalisen
               sänge. Und es ist alles eher als eine Einwendung gegen den tieferen Sinn Ihrer Worte,
               wenn mir persönlich für die innere und äussere Entwicklung eines Volkes die Frage der
               Staatsform von einer ziemlich nebensächlichen Bedeutung erscheint, und dass sich jede
               grosse politische Führernatur selbst die Form zu schaffen pflegt, innerhalb deren sie
               sich betätigt und wirkt, ob er nun Kaiser, König, Präsident oder Kanzler heissen mag.
               Zu einem Menschen kann ich mich zuweilen bekennen, kaum je ohne Vorbehalt, zu einer
               Staatsform als solcher nie. Das wäre vielleicht sehr republikanisch gedacht, wenn
               jede Republik – wenn jemals eine Republik – wenn überhaupt jemals irgend eine Form
               ihre eigene, ihre imanente Idee zu erfüllen fähig wäre. Aber ich gerate
               ins Allgemeine, in einen Essay, das ist meine Sache nicht, ich brächte doch keinen zu
               Ende, er müsste auf dem Wege sterben an der Menge von Parenthesen, die ich immer
               wieder für unerlässlich hielte.\pend
           
\pstart
           Sie kommen im Jänner nach Wien\oindex{Wien@\textbf{Wien}, \emph{A.ADM2}|pw}, da
               werde ich Sie ja hoffentlich sehen. Ich bin im Herbst in der Cechoslowakei\oindex{Tschechische Republik@\textbf{Tschechische Republik}, \emph{A.PCLI}|pw}\oindex{Slowakei@\textbf{Slowakei}, \emph{A.PCLI}|pw} gewesen (in Teplitz\oindex{Slowakei@\textbf{Slowakei}, \emph{A.PCLI}|pw} machten sich die Hakenkreuzler peinlich
               bemerkbar), im März{ }soll ich wieder hin, diesmal nach östlicheren
               Gegenden, im Frühjahr fahre ich vielleicht nach Dänemark\oindex{Daenemark@\textbf{Dänemark}, \emph{A.PCLI}|pw} und Schweden\oindex{Schweden@\textbf{Schweden}, \emph{A.PCLI}|pw}. Ihr Roman\pwindex{Zauberberg. Roman@\emph{Der Zauberberg. Roman}|pwv}{ }schreitet hoffentlich seiner Vollendung entgegen.
               Ich freue mich ihm und Ihnen entgegen.\pend
           
\pstart
           Seien Sie vielmals und herzlichst gegrüsst von{\\[\baselineskip]}Ihrem ergebenen\pend
           \leftskip=0em{}{\vspace{1\baselineskip}}
\pstart
           \noindent{}Herrn Thomas Mann{\\}\pend
           
\pstart
           München\oindex{Muenchen@\textbf{München}, \emph{P.PPLA}|pw}\pend
           \selectlanguage{ngerman}\endnumbering\briefempfaengerindex{Mann, Thomas@\textsc{Mann, Thomas}!zzzSchnitzler, Arthur@\emph{von Arthur Schnitzler}!1922-12-281@{28. 12. 1922}|)be}\mylabel{L02394h}  \normalsize

\doendnotes{C}
\bigskip
\vfill

\clearpage

\footnotesize

\lohead{\textsc{register}}

% Definiere theindex-Environment komplett neu ohne reledmac
\makeatletter
\renewenvironment{theindex}{%
  \section*{\indexname}%
  \setlength{\parindent}{0pt}%
  \setlength{\parskip}{0pt plus 0.3pt}%
  \let\item\@idxitem
}{%
  \clearpage
}
\makeatother

\IfFileExists{\jobname-pw.ind}{\input{\jobname-pw.ind}}{}

\end{document}

      