%% latex-korrekturansicht-vorspann.tex
%% Vorspann für die Korrekturansicht.
%% Lädt die gemeinsame Datei latex-vorspann.tex mit gesetztem Schalter.

\newif\ifkorrekturansicht
\korrekturansichttrue

\input{../tex-inputs/latex-vorspann}


\section[Arthur Schnitzler an Richard Beer-Hofmann, 29. 7. 1899]{L00952 Arthur Schnitzler an Richard Beer-Hofmann, 29. 7. 1899}
\nopagebreak\mylabel{L00952v}
\rehead{ }\normalsize\beginnumbering\briefempfaengerindex{Beer-Hofmann, Richard@\textsc{Beer-Hofmann, Richard}!zzzSchnitzler, Arthur@\emph{von Arthur Schnitzler}!1899-07-291@{29. 7. 1899}|(be}
\toendnotes[C]{\smallbreak\pagebreak[2]}\Standort{YCGL, MSS 31.}
\physDesc{Telegramm, 179 Zeichen
\newline{}Handschrift Schreibkraft: 1) schwarze Tinte, deutsche Kurrent\hspace{1em}2) schwarze Tinte, lateinische Kurrent (\noindent{}Adresse)\hspace{1em}
\newline{}Versand: »\noindent{}\textcolor{gray}{\textbf{Eingelangt von}}{ }Vi\oindex{Villach@\textbf{Villach}, \emph{A.ADM3}|pw}{ }\textcolor{gray}{\textbf{auf Leitung Nr.}} 381 \textcolor{gray}{\textbf{am}}{ }29/7 \textcolor{gray}{\textbf{189}}9{ }\textcolor{gray}{\textbf{um}}{ }8 \textcolor{gray}{\textbf{Uhr}} 50 \textcolor{gray}{\textbf{Min.}}{ }V \textcolor{gray}{\textbf{Mittag}}{ / }\textcolor{gray}{\textbf{Aufgenommen durch}}{ }\textcolor{gray}{T}\textcolor{gray}{×}\-\textcolor{gray}{×}{ / }\textcolor{gray}{\textbf{Von}}{ }Tarvis\oindex{Tarvisio@\textbf{Tarvisio}, \emph{P.PPLA3}|pw}{ / }\textcolor{gray}{\textbf{Aufgabe-Nr.}} 139 \textcolor{gray}{\textbf{mit}} 24 \textcolor{gray}{\textbf{Taxworten ({\dots} Worten {\dots} Chiffern)}}{ / }\textcolor{gray}{\textbf{Aufgegeben am}}{ }29/7{ }\textcolor{gray}{\textbf{um}}{ }8 \textcolor{gray}{\textbf{Uhr}} 15 \textcolor{gray}{\textbf{Min.}} V« }\pstart{}{\pb}Richard Beer\pend{}\pstart{}Hofman Villa Platzer\oindex{Villa Platzer@\textbf{Villa Platzer}, \emph{Gebäude (K.GBD)}|pw}\pend{}\pstart{}Seeboden\oindex{Seeboden@\textbf{Seeboden}, \emph{A.ADM3}|pw}\pend{}{\bigskip}\vspace{1em}
\pstart
           \noindent{}{\pb}Gedenken morgen Frühſtunde Villach\oindex{Villach@\textbf{Villach}, \emph{A.ADM3}|pw}{ }Hotel{ }\textsc{Mosser}\oindex{Hotel Mosser@\textbf{Hotel Mosser}, \emph{Hotel (K.HTL)}|pw} verlaſſen Vormittag bei Ihnen ſein Spital\oindex{Spittal an der Drau@\textbf{Spittal an der Drau}, \emph{P.PPLA3}|pw}
               übernachten übermorgen \textsc{Toblach\oindex{Toblach@\textbf{Toblach}, \emph{A.ADM3}|pw}} reisen – Herzlichſt \spacefill\mbox{Arthur}\pend
           \selectlanguage{ngerman}\endnumbering\briefempfaengerindex{Beer-Hofmann, Richard@\textsc{Beer-Hofmann, Richard}!zzzSchnitzler, Arthur@\emph{von Arthur Schnitzler}!1899-07-291@{29. 7. 1899}|)be}\mylabel{L00952h}  \normalsize

\doendnotes{C}
\bigskip
\vfill

\clearpage

\footnotesize

\lohead{\textsc{register}}

% Definiere theindex-Environment komplett neu ohne reledmac
\makeatletter
\renewenvironment{theindex}{%
  \section*{\indexname}%
  \setlength{\parindent}{0pt}%
  \setlength{\parskip}{0pt plus 0.3pt}%
  \let\item\@idxitem
}{%
  \clearpage
}
\makeatother

\IfFileExists{\jobname-pw.ind}{\input{\jobname-pw.ind}}{}

\end{document}

      