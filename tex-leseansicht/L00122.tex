%% latex-leseansicht-vorspann.tex
%% Vorspann für die Leseansicht.
%% Lädt die gemeinsame Datei latex-vorspann.tex mit nicht gesetztem Schalter.

\newif\ifkorrekturansicht
\korrekturansichtfalse

\input{../tex-inputs/latex-vorspann}


         
         \newcommand{\erwaehntePersonen}{Personen: Edmond Huot de Goncourt, Jules Huot de Goncourt, Hugo von Hofmannsthal}
         \newcommand{\erwaehnteOrte}{Orte: Bad Ischl, Brenner, Lago di Garda, Riva del Garda, Schafberg (Wien), Semmering, Wien}
         \newcommand{\erwaehnteWerke}{Werke: Journal des Goncourt. Mémoires de la vie littéraire}
               \section[Arthur Schnitzler an Hugo von Hofmannsthal, 11. 9. 1892]{ Arthur Schnitzler an Hugo von Hofmannsthal, 11. 9. 1892}\nopagebreak\mylabel{v}\rehead{ }\begin{ledgroupsized}[t]{13cm}\normalsize\beginnumbering \toendnotes[C]{\smallbreak\pagebreak[2]} \Standort{FDH, Hs-30885,25.}
\physDesc{Brief, 1 Blatt, 3 Seiten
\newline{}Handschrift: schwarze Tinte, deutsche Kurrent}\buchAbdrucke{\weitereDrucke{Hugo von Hofmannsthal, Arthur Schnitzler: \emph{Briefwechsel}. Hg. Therese Nickl und Heinrich Schnitzler. Frankfurt am Main: \emph{S. Fischer} 1964, S. 29.} }\pstart
           \raggedleft{}{\pb}11. 9. 92.\pend
           \pstart{}Lieber Loris. –\pend\pstart
           Heute verlaſſe ich Iſchl\oindex{Bad Ischl@\textbf{Bad Ischl}|pw}. Ueber den Brenner\oindex{Brenner@\textbf{Brenner}|pw} nach Riva\oindex{Riva del Garda@\textbf{Riva del Garda}|pw} am Gardaſee\oindex{Lago di Garda@\textbf{Lago di Garda}|pw}, wo ich wohl
                    einige Zeit, dh. 5–8 Tage verbleibe. Dann Semmering\oindex{Semmering@\textbf{Semmering}|pw}, denk’ ich, dann Wien\oindex{Wien@\textbf{Wien}|pw}.
                    Neulich auf dem Schafberg\oindex{Schafberg (Wien)@\textbf{Schafberg (Wien)}|pw} geweſen – tiefer
                    Schnee, Geſtöber. –\pend
           \pstart
           Hier auch weiterhin nichts gethan. Der Tag vergeht doch. Das Journal\pwindex{Goncourt, Edmond Huot de 26.05.1822 – 16.07.1896@\textsc{Goncourt, Edmond Huot de} (26.05.1822 – 16.07.1896), \emph{Schriftsteller}!Journal des Goncourt. Memoires de la vie litteraire1887 – 1896@\strich\emph{Journal des Goncourt. Mémoires de la vie littéraire} {[}1887 – 1896{]}|pw}\pwindex{Goncourt, Jules Huot de 17.12.1830 – 20.06.1870@\textsc{Goncourt, Jules Huot de} (17.12.1830 – 20.06.1870), \emph{Schriftsteller}!Journal des Goncourt. Memoires de la vie litteraire1887 – 1896@\strich\emph{Journal des Goncourt. Mémoires de la vie littéraire} {[}1887 – 1896{]}|pw} v d Goncourts\pwindex{Goncourt, Edmond Huot de 26.05.1822 – 16.07.1896@\textsc{Goncourt, Edmond Huot de} (26.05.1822 – 16.07.1896), \emph{Schriftsteller}|pw}\pwindex{Goncourt, Jules Huot de 17.12.1830 – 20.06.1870@\textsc{Goncourt, Jules Huot de} (17.12.1830 – 20.06.1870), \emph{Schriftsteller}|pw} geleſen, Karten geſpielt, in den Straßen herum,
                    faſt i{\geminationm}er Regen. {\pb}Jetzt
                    will ich packen, was ich nicht kann.\pend
           \pstart
           Wenn Sie mir nach Riva\oindex{Riva del Garda@\textbf{Riva del Garda}|pw}{ }ſchreiben wollen, ein paar Zeilen, was ſehr
                    hübſch wäre, \textsc{post rest}, bitte. –\pend
           \pstart
           Mich frieren die Fingerſpitzen. Im Zi{\geminationm}er iſt es
                    kalt. Im Hotel wird i{\geminationm}erfort geklingelt, kein
                    Menſch weiſs warum. Schritte im Corridor: i{\geminationm}er, als
                        we{\geminationn}{ }ſie gerad zu meiner Thür kämen. Alles in
                    Wolken. {\pb}Freue mich, noch nicht nach Wien\oindex{Wien@\textbf{Wien}|pw} zu reiſen.\pend
           \pstart
           Herzlichſt der Ihre{\\[\baselineskip]}\spacefill\mbox{Arthur.}\pend
           \leftskip=0em{}
         
         \endnumbering\mylabel{h}\end{ledgroupsized}  \newcommand{\dateiname}{L00122}\newcommand{\titel}{Arthur Schnitzler an Hugo von Hofmannsthal, 11. 9. 1892}\newcommand{\editorInnen}{Martin Anton Müller und Gerd-Hermann Susen}%% latex-leseansicht-abspann.tex
%% Abspann für die Leseansicht.
%% Der Schalter \ifkorrekturansicht ist bereits durch den Vorspann gesetzt.

%% latex-abspann.tex
%% Gemeinsamer Abspann für Korrekturansicht und Leseansicht.
%% Setzt den Schalter \ifkorrekturansicht voraus (gesetzt in den
%% einbindenden Dateien latex-korrekturansicht-abspann.tex bzw.
%% latex-leseansicht-abspann.tex).
%% ---------------------------------------------------------------

\normalsize

% Das esempio-Environment wird nur in der Leseansicht benötigt
\ifkorrekturansicht\else
\newenvironment{esempio}[3]%
{
    \vspace{1.5ex}
    \rlap{\underline{#1}}
    \par
    \setlength{\parindent}{0cm}
    \nopagebreak
    \leftskip=#2cm
    \rightskip=#3cm
}
{
    \par
}
\fi

\doendnotes{C}
\bigskip
\vfill

\clearpage

\footnotesize

\ifkorrekturansicht
  \lohead{\textsc{register}}
\fi

% theindex-Environment neu definieren ohne reledmac
\makeatletter
\renewenvironment{theindex}{%
  \ifkorrekturansicht
    \section*{\indexname}%
  \else
    \subsubsection*{Index der erwähnten Entitäten}%
  \fi
  \setlength{\parindent}{0pt}%
  \setlength{\parskip}{0pt plus 0.3pt}%
  \let\item\@idxitem
}{%
  \ifkorrekturansicht\clearpage\fi
}
\makeatother

\IfFileExists{\jobname-pw.ind}{\input{\jobname-pw.ind}}{}

% Quellenangabe nur in der Leseansicht
\ifkorrekturansicht\else
% Fallback-Definitionen, falls die .tex-Datei \titel etc. nicht gesetzt hat
\providecommand{\titel}{}
\providecommand{\editorInnen}{}
\providecommand{\dateiname}{\jobname}

\vspace{3cm}

\vfill

\footnotesize
\textsc{Quelle}: \titel. Herausgegeben von {\editorInnen}. In: \emph{Arthur Schnitzler: Briefwechsel mit Autorinnen und Autoren}.
 Digitale Edition, https://schnitzler-briefe.acdh.oeaw.ac.at/{\dateiname}.html (Stand \today)
\fi

\end{document}


      