%% latex-korrekturansicht-vorspann.tex
%% Vorspann für die Korrekturansicht.
%% Lädt die gemeinsame Datei latex-vorspann.tex mit gesetztem Schalter.

\newif\ifkorrekturansicht
\korrekturansichttrue

\input{../tex-inputs/latex-vorspann}


\section[Arthur Schnitzler an Hugo von Hofmannsthal, 11. 9. 1892]{L00122 Arthur Schnitzler an Hugo von Hofmannsthal, 11. 9. 1892}
\nopagebreak\mylabel{L00122v}
\rehead{ }\normalsize\beginnumbering\briefempfaengerindex{Hofmannsthal, Hugo von@\textsc{Hofmannsthal, Hugo von}!zzzSchnitzler, Arthur@\emph{von Arthur Schnitzler}!1892-09-111@{11. 9. 1892}|(be}
\toendnotes[C]{\smallbreak\pagebreak[2]}\Standort{FDH, Hs-30885,25.}
\physDesc{Brief, 1 Blatt, 3 Seiten, 778 Zeichen
\newline{}Handschrift: schwarze Tinte, deutsche Kurrent}
\buchAbdrucke{\weitereDrucke{Hugo von Hofmannsthal, Arthur Schnitzler: \emph{Briefwechsel}. Frankfurt am Main: \emph{S. Fischer} 1964, S. 29.} }
\pstart
           \raggedleft{}{\pb}11. 9. 92.\pend
           
\pstart{}Lieber Loris. –\pend\vspace{0.5em}
\pstart
           Heute verlaſſe ich Iſchl\oindex{Bad Ischl@\textbf{Bad Ischl}, \emph{P.PPL}|pw}. Ueber den Brenner\oindex{Brenner@\textbf{Brenner}, \emph{T.PASS}|pw} nach Riva\oindex{Riva del Garda@\textbf{Riva del Garda}, \emph{P.PPLA3}|pw} am Gardaſee\oindex{Lago di Garda@\textbf{Lago di Garda}, \emph{See (N.SEE)}|pw}, wo ich wohl einige
               Zeit, dh. 5–8 Tage verbleibe. Dann Semmering\oindex{Semmering@\textbf{Semmering}, \emph{A.ADM3}|pw},
               denk’ ich, dann Wien\oindex{Wien@\textbf{Wien}, \emph{A.ADM2}|pw}. Neulich auf dem Schafberg\oindex{Schafberg [Wien]@\textbf{Schafberg [Wien]}, \emph{T.HLL}|pw} geweſen – tiefer Schnee, Geſtöber. –\pend
           
\pstart
           Hier auch weiterhin nichts gethan. Der Tag vergeht doch. Das Journal\pwindex{Journal des Goncourt. Memoires de la vie litteraire@\emph{Journal des Goncourt. Mémoires de la vie littéraire}|pw} v d Goncourts\pwindex{Goncourt, Edmond Huot de 26.05.1822 – 16.07.1896@\textsc{Goncourt, Edmond Huot de} (26.05.1822 – 16.07.1896), \emph{Schriftsteller/Schriftstellerin}|pw}\pwindex{Goncourt, Jules Huot de 17.12.1830 – 20.06.1870@\textsc{Goncourt, Jules Huot de} (17.12.1830 – 20.06.1870), \emph{Schriftsteller/Schriftstellerin}|pw} geleſen, Karten geſpielt, in den Straßen herum, faſt i{\geminationm}er Regen. {\pb}Jetzt will ich
               packen, was ich nicht kann.\pend
           
\pstart
           Wenn Sie mir nach Riva\oindex{Riva del Garda@\textbf{Riva del Garda}, \emph{P.PPLA3}|pw}{ }ſchreiben wollen, ein paar Zeilen, was ſehr hübſch
               wäre, \textsc{post rest}, bitte. –\pend
           
\pstart
           Mich frieren die Fingerſpitzen. Im Zi{\geminationm}er iſt es kalt. Im
               Hotel wird i{\geminationm}erfort geklingelt, kein Menſch weiſs warum.
               Schritte im Corridor: i{\geminationm}er, als we{\geminationn}{ }ſie gerad zu meiner Thür kämen. Alles in Wolken.
                  {\pb}Freue mich, noch nicht nach Wien\oindex{Wien@\textbf{Wien}, \emph{A.ADM2}|pw} zu reiſen.\pend
           
\pstart
           Herzlichſt der Ihre{\\[\baselineskip]}\spacefill\mbox{Arthur.}\pend
           \leftskip=0em{}\selectlanguage{ngerman}\endnumbering\briefempfaengerindex{Hofmannsthal, Hugo von@\textsc{Hofmannsthal, Hugo von}!zzzSchnitzler, Arthur@\emph{von Arthur Schnitzler}!1892-09-111@{11. 9. 1892}|)be}\mylabel{L00122h}  \normalsize

\doendnotes{C}
\bigskip
\vfill

\clearpage

\footnotesize

\lohead{\textsc{register}}

% Definiere theindex-Environment komplett neu ohne reledmac
\makeatletter
\renewenvironment{theindex}{%
  \section*{\indexname}%
  \setlength{\parindent}{0pt}%
  \setlength{\parskip}{0pt plus 0.3pt}%
  \let\item\@idxitem
}{%
  \clearpage
}
\makeatother

\IfFileExists{\jobname-pw.ind}{\input{\jobname-pw.ind}}{}

\end{document}

      