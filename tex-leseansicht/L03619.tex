%% latex-korrekturansicht-vorspann.tex
%% Vorspann für die Korrekturansicht.
%% Lädt die gemeinsame Datei latex-vorspann.tex mit gesetztem Schalter.

\newif\ifkorrekturansicht
\korrekturansichttrue

\input{../tex-inputs/latex-vorspann}


\section[Karl Emil Franzos an Arthur Schnitzler, {[}3. 5. 1888 – 11. 5. 1888?{]}]{L03619 Karl Emil Franzos an Arthur Schnitzler, {[}3. 5. 1888 –
               11. 5. 1888?{]}}
\nopagebreak\mylabel{L03619v}
\rehead{ }\normalsize\beginnumbering\briefempfaengerindex{Schnitzler, Arthur@\textsc{Schnitzler, Arthur}!zzzFranzos, Karl Emil@\emph{von Karl Emil Franzos}!1888-05-112@{{[}3. 5. 1888 – 11. 5. 1888?{]}}|(be}
\toendnotes[C]{\smallbreak\pagebreak[2]}\Standort{DLA, A:Schnitzler, HS.1985.1.3025.}
\physDesc{Brief, 1 Blatt, 3 Seiten, 2073 Zeichen
\newline{}Handschrift Ottilie Franzos: schwarze Tinte, deutsche Kurrent
\newline{}Handschrift Karl Emil Franzos: schwarze Tinte, deutsche Kurrent (\noindent{}zwei Einfügungen, Unterschrift und Nachschrift)
\newline{}Schnitzler: 1) mit rotem Buntstift eine Unterstreichung  2) mit Bleistift »\textsc{Franzos}«}\toendnotes[C]{\smallbreak}
\pstart
           \centering{}{\pb}\textcolor{gray}{\textbf{Redaction der »Deutſchen
                        Dichtung\orgindex{Deutsche Dichtung@Deutsche Dichtung|pw}«.}}\pend
           
\pstart
           \textcolor{gray}{\textbf{Herausgeber:}}\hfill \textcolor{gray}{\textbf{Verlag:}}\pend
           
\pstart
           \textcolor{gray}{\textbf{Karl Emil Franzos}}\pwindex{Franzos, Karl Emil 25.10.1848 – 28.01.1904@\textsc{Franzos, Karl Emil} (25.10.1848 – 28.01.1904), \emph{Schriftsteller/Schriftstellerin, Journalist/Journalistin}|pw}\hfill \textcolor{gray}{\textbf{Adolf Bonz {\kaufmannsund} Comp.}}\orgindex{Adolf Bonz und Comp.@Adolf Bonz {\kaufmannsund}  Comp.|pw}\pend
           
\pstart
           \textcolor{gray}{\textbf{Berlin\oindex{Berlin@\textbf{Berlin}, \emph{P.PPLC}|pw}.}}\hfill \textcolor{gray}{\textbf{Stuttgart\oindex{Stuttgart@\textbf{Stuttgart}, \emph{P.PPLA}|pw}.}}\pend
           
\pstart
           \raggedleft{}\textcolor{gray}{\textbf{Berlin\oindex{Berlin@\textbf{Berlin}, \emph{P.PPLC}|pw},}} den 3. Mai \textcolor{gray}{\textbf{188}}8. \pend
           
\pstart
           \raggedleft{}\textcolor{gray}{\textbf{W. Kaiserin Auguſtaſtraße 71\oindex{Kaiserin-Augusta-Strasse 71@\textbf{Kaiserin-Augusta-Straße 71}, \emph{Wohngebäude (K.WHS)}|pw}.}}\pend
           
\pstart\center{}Geehrter Herr Doctor!\pend\vspace{0.5em}
\pstart
           Ein an ſich nicht gerade erfreulicher Umſtand, ein Unwohlſein nämlich, welches mich
               für einige Tage an’s Bett bannte und mir eine unfreiwillige Muße auferlegte, hat mir
               andrerſeits ermöglicht, Ihrem Wunſche, Ihnen meine Anſicht über Ihre beiden \label{K_L03619-1v}\edtext{Novellen\pwindex{Amerika@\emph{Amerika}|pwv}\pwindex{Mein Freund Ypsilon. Aus den Papieren eines Arztes@\emph{Mein Freund Ypsilon. Aus den Papieren eines Arztes}|pwv}\pwindex{Erbschaft@\emph{Erbschaft}|pwv}}{\lemma{\textnormal{\emph{Novellen}}}\Cendnote{\textnormal{Vgl. Arthur Schnitzler an Karl Emil Franzos, 29. 4. 1888.
               }}}\label{K_L03619-1} zu ſagen, ſchon jetzt entſprechen zu können, mehr aber als eben eine
               ſubjektive Anſchauung beanſpruche ich gewiß nicht zu bieten. Beide Arbeiten\pwindex{Amerika@\emph{Amerika}|pwv}\pwindex{Mein Freund Ypsilon. Aus den Papieren eines Arztes@\emph{Mein Freund Ypsilon. Aus den Papieren eines Arztes}|pwv}\pwindex{Erbschaft@\emph{Erbschaft}|pwv} waren mir
               insbeſondere ihrer Entſtehung \introOben{}{[}hs.:{]} nach\introOben{} pſychologiſch intereſſant, ſie ſind
               ſichtlich die Erzeugniſſe eines jungen Arztes, welcher den realen Thatſachen ſeines
               Berufs dadurch eine Art idealiſirenden Gegengewichts zu geben verſucht. {\pb}Daraus erklärt ſich das eigenthümliche
               Gegenüberſtehen der beiden Momente, welche ſich in den Novellen\pwindex{Amerika@\emph{Amerika}|pwv}\pwindex{Mein Freund Ypsilon. Aus den Papieren eines Arztes@\emph{Mein Freund Ypsilon. Aus den Papieren eines Arztes}|pwv}\pwindex{Erbschaft@\emph{Erbschaft}|pwv} gleich ſcharf
               vertreten finden, der romantiſchen Erfindung und der realiſtischen Wahl des
               Grundproblems, welches ja in beiden ein rein pathologisches iſt. Es iſt aber eben
               auch nur ein Nebeneinanderſtehen und keine harmoniſche Miſchung, was wohl darin ſeine
               Erklärung findet, daß beide Elemente in ihrer extremſten Ausprägung hier vertreten
               erſcheinen. Einerſeits wird die Romantik in beiden Novellen\pwindex{Amerika@\emph{Amerika}|pwv}\pwindex{Mein Freund Ypsilon. Aus den Papieren eines Arztes@\emph{Mein Freund Ypsilon. Aus den Papieren eines Arztes}|pwv}\pwindex{Erbschaft@\emph{Erbschaft}|pwv} zur
               Hyperromantik \introOben{}{[}hs.:{]} getrieben\introOben{}, andrerſeits wird das pathologiſche
               Problem ſehr hart und ſtreng betont. Dies iſt meines bescheidenen Ermeſſens jene
               Klippe, welche Sie künftig zu umſchiffen haben werden, denn obwohl beide Novellen\pwindex{Amerika@\emph{Amerika}|pwv}\pwindex{Mein Freund Ypsilon. Aus den Papieren eines Arztes@\emph{Mein Freund Ypsilon. Aus den Papieren eines Arztes}|pwv}\pwindex{Erbschaft@\emph{Erbschaft}|pwv}
               meines Erachtens nicht ſo druckreif ſind, als daß ich einem ernſthaft ſtrebenden
               Manne damit vor die Öffentlichkeit zu treten anrathen könnte, ſo wäre es doch
               zunächſt für Sie und {\pb}wenn Sie die Arbeit ernſthaft
               anfaſſen, wohl nicht für Sie allein Schade, wenn Sie es dabei bewenden laſſen
               wollten.\pend
           
\pstart
           Mit beſten Empfehlungen{\\[\baselineskip]} Ihr ergebenſter {\\[\baselineskip]}\spacefill\mbox{{[}hs. :{]} Franzos}\pend
           \leftskip=0em{}
\pstart
           \noindent{}{[}hs. :{]} Herrn \textsc{Dr. A.
                  Schnitzler}.\pend
           \selectlanguage{ngerman}\vspace{1em}{\vspace{1\baselineskip}}
\pstart
           {[}hs. :{]} Geehrter Herr Dr! Der vorſtehende Brief iſt leider
               durch ein \label{K_L03619-3v}\edtext{Überſehen meiner Gattin\pwindex{Franzos, Ottilie 24.09.1856 – 05.03.1932@\textsc{Franzos, Ottilie} (24.09.1856 – 05.03.1932), \emph{Schriftsteller/Schriftstellerin}|pwv}}{\lemma{\textnormal{\emph{Überſehen meiner Gattin}}}\Cendnote{\textnormal{Die Involvierung von Ottilie Franzos\pwindex{Franzos, Ottilie 24.09.1856 – 05.03.1932@\textsc{Franzos, Ottilie} (24.09.1856 – 05.03.1932), \emph{Schriftsteller/Schriftstellerin}|pwk} in der Begründung lässt sich als Hinweis
                  lesen, dass sie den vorliegenden Brief auch für ihren Mann geschrieben hat.}}}\label{K_L03619-3}
               bis \label{K_L03619-2v}\edtext{heute}{\lemma{\textnormal{\emph{heute}}}\Cendnote{\textnormal{Die Nachschrift ist undatiert und folglich lässt sich nicht
                  mit Sicherheit feststellen, ob Schnitzler
                  das Korrespondenzstück noch vor seiner (vorgezogenen) Abreise aus Berlin\oindex{Berlin@\textbf{Berlin}, \emph{P.PPLC}|pwk} am 12. 5. 1888 erhalten hat – oder es ihm nach Wien\oindex{Wien@\textbf{Wien}, \emph{A.ADM2}|pwk} nachgesandt wurde. Es ist vorstellbar, dass
                     Franzos\pwindex{Franzos, Karl Emil 25.10.1848 – 28.01.1904@\textsc{Franzos, Karl Emil} (25.10.1848 – 28.01.1904), \emph{Schriftsteller/Schriftstellerin, Journalist/Journalistin}|pwk} selbst bemerkte, dass seine
                  Antwort liegen geblieben war. Naheliegend ist aber, dass Schnitzlers Brief vom 11. 5. 1888{ }Franzos\pwindex{Franzos, Karl Emil 25.10.1848 – 28.01.1904@\textsc{Franzos, Karl Emil} (25.10.1848 – 28.01.1904), \emph{Schriftsteller/Schriftstellerin, Journalist/Journalistin}|pwk} an sein nicht abgesandtes Schreiben
                  erinnerte und er die Nachschrift verfasste und schnell noch spedierte, um sie Schnitzler noch vor der Abreise zukommen zu
                  lassen.}}}\label{K_L03619-2} unbeſtellt geblieben. Ich ſende Ihnen denſelben nun und unſere
               beſten Abſchiedsgrüße dazu. Vergeſſen Sie uns nicht, we{\geminationn}
               Sie Ihr Weg wieder hierher führt und ſagen Sie Ihrem Herrn Vater\pwindex{Schnitzler, Johann 10.04.1835 – 02.05.1893@\textsc{Schnitzler, Johann} (10.04.1835 – 02.05.1893), \emph{Laryngologe/Laryngologin}|pwv} unſere beſten Empfehlungen. Herzlich grüßend\pend
           \pstart  Ihr \spacefill\mbox{Fr.}\pend{}\selectlanguage{ngerman}\endnumbering\briefempfaengerindex{Schnitzler, Arthur@\textsc{Schnitzler, Arthur}!zzzFranzos, Karl Emil@\emph{von Karl Emil Franzos}!1888-05-112@{{[}3. 5. 1888 – 11. 5. 1888?{]}}|)be}\mylabel{L03619h}  \normalsize

\doendnotes{C}
\bigskip
\vfill

\clearpage

\footnotesize

\lohead{\textsc{register}}

% Definiere theindex-Environment komplett neu ohne reledmac
\makeatletter
\renewenvironment{theindex}{%
  \section*{\indexname}%
  \setlength{\parindent}{0pt}%
  \setlength{\parskip}{0pt plus 0.3pt}%
  \let\item\@idxitem
}{%
  \clearpage
}
\makeatother

\IfFileExists{\jobname-pw.ind}{\input{\jobname-pw.ind}}{}

\end{document}

      